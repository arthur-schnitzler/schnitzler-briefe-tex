%% latex-korrekturansicht-vorspann.tex
%% Vorspann für die Korrekturansicht.
%% Lädt die gemeinsame Datei latex-vorspann.tex mit gesetztem Schalter.

\newif\ifkorrekturansicht
\korrekturansichttrue

\input{../tex-inputs/latex-vorspann}


               \section[Hugo Hofmannsthal an Arthur Schnitzler, 10. 7. {[}1928{]}]{ Hugo Hofmannsthal an Arthur Schnitzler, 10. 7. {[}1928{]}}\nopagebreak\mylabel{v}\rehead{ }\normalsize\beginnumbering\briefempfaengerindex{Schnitzler, Arthur@\textsc{Schnitzler, Arthur}!zzzHofmannsthal, Hugo von@\emph{von Hugo von Hofmannsthal}!1928-07-101@{10. 7. {[}1928{]}}|(be} \toendnotes[C]{\smallbreak\pagebreak[2]} \Standort{CUL, Schnitzler, B 43.}
\physDesc{Brief, 2 Blätter, 4 Seiten
\newline{}Handschrift: schwarze Tinte, lateinische Kurrent
\newline{}Schnitzler: 1) mit Bleistift datiert: »10/7 28« und beschriftet: »HvH« 2) mit rotem Buntstift mehrere Unterstreichungen\newline{}Ordnung: 1) mit Bleistift von unbekannter Hand nummeriert: »\strikeout{371}« 2) mit Bleistift von unbekannter Hand nummeriert:
                                    »380«}\buchAbdrucke{\weitereDrucke{Hugo von Hofmannsthal, Arthur Schnitzler: \emph{Briefwechsel}. Hg. Therese Nickl und Heinrich Schnitzler. Frankfurt am Main: \emph{S. Fischer} 1964, S. 309.} }\toendnotes[C]{\smallbreak}\pstart
           {\pb}\textcolor{pink}{Haus Mahler}{}\ledrightnote{\textcolor{pink}{Haus Mahler}}\hspace*{1em}\textcolor{pink}{Breitenstein am Semmering.}{}\ledrightnote{\textcolor{pink}{Breitenstein am Semmering}}\pend
           \pstart
           \centering{}10\textsuperscript{ter} Juli.\pend
           \pstart
           mein lieber Arthur,\hspace*{1.5em}schon seit ich das Buch gelesen habe, wollte ich
               Ihnen ein paar Worte über den Roman »\textcolor{green}{Therese}{}\ledrightnote{\textcolor{green}{Therese. Chronik eines Frauenlebens}}«
               sagen. Aber der letzte Monat war bei mir sehr unruhig, durch die \label{K_L02503_1v}\edtext{beiden \textcolor{green}{Opernpremieren}{}\ledrightnote{→\textcolor{green}{Die ägyptische Helena}}}{\lemma{\textnormal{\emph{beiden Opernpremieren}}}\Cendnote{\textnormal{\emph{\textcolor{green}{Die ägyptische Helena}} wurde am
                     6. 6. 1928 in \textcolor{pink}{Dresden}, am
                     12. 6. 1928 in \textcolor{pink}{Wien}
                  aufgeführt.}}}\label{K_L02503_1h} und verschiedenes Andere. Auch war ich dazwischen \label{K_L02503_2v}\edtext{eine Woche in \textcolor{pink}{Salzburg}{}\ledrightnote{\textcolor{pink}{Salzburg}}}{\lemma{\textnormal{\emph{eine Woche in Salzburg}}}\Cendnote{\textnormal{von 19. 6. 1928 bis zum
                     25. 6. 1928}}}\label{K_L02503_2h}, um \textcolor{blue}{Reinhardt}{}\ledrightnote{\textcolor{blue}{Max Reinhardt}} bei einem \textcolor{green}{Film}{}\ledrightnote{→\textcolor{green}{Film für Lillian Gish}}
               zu helfen, dies nur aus dem Grund, weil es – im Fall des Gelingens – ein Stück Geld
               einträgt und ich alles daran setzen möchte für \textcolor{blue}{Christiane}{}\ledrightnote{\textcolor{blue}{Christiane von Hofmannsthal}} ein kleines Haus in \textcolor{pink}{Heidelberg}{}\ledrightnote{\textcolor{pink}{Heidelberg}}
               zu kaufen (natürlich in den bescheidensten Dimensionen) – denn die Wohnverhältnisse
               dort sind unerträglich.\pend
           \pstart
           Sie haben nicht auf mich gewartet, um zu hören, dass Sie in einer Epoche in der es
               sehr wenige Meister gibt, ein Meister der Erzählung sind. In allen Ihren kurzen und
               mittelgroßen Erzählungen ist ein wunderbar sicheres Maßgefühl wirksam – und dadurch,
               durch ihre schönen Maße, bleiben sie auch so schön und lebendig in der Erinnerung.
                  {\pb}Dabei ist in ihnen alles mit
               sparsamen aber sehr reinen Farben gemalt, die Abstufungen der Farbe mit dem
               sichersten Instinct hingesetzt, das Ganze ist nie grellbunt, nie aber stumpf – von
               den ungeheuren rhythmischen Vorzügen aber will ich gar nicht sprechen. Die große
               Lebenserzählung \textcolor{green}{Therese}{}\ledrightnote{\textcolor{green}{Therese. Chronik eines Frauenlebens}} aber hat mich besonders
               gefesselt und beschäftigt. Schon der Stoff gehört ganz nur Ihnen. Indem Sie diesen
               Stoff wählten: das Leben einer \textcolor{pink}{Wien}{}\ledrightnote{\textcolor{pink}{Wien}}er Gouvernante –
               war schon eine ganze Welt hingestellt, und ein großer Reichtum von Aspecten, Sti{\geminationm}ungen, Gefühlen und gedankenhaften Halbgefühlen im
               verstehenden Leser gesichert. Ganz besonders groß aber tritt Ihr Vorzug, einem Stoff
               den Rhythmus zu geben, wodurch er Dichtung wird, hier hervor. Eben was dem stumpfen
               Leser monoton scheinen kö{\geminationn}te, dass sich sozusagen die
               Figur des Erlebnisses bis zur beabsichtigten Unzählbarkeit wiederholt, das hat Ihnen
               ermöglicht, Ihre rhythmische Kraft bis zum Zauberhaften zu entfalten. Es sind diese
               Vorzüge, die ein Kunstwerk über viele andere scheinbar ähnliche, bis zur
               Unvergleichbarkeit erheben, und die {\pb}es auf lange lebendig erhalten
               werden.\pend
           \pstart
           Über \textcolor{blue}{Christiane}{}\ledrightnote{\textcolor{blue}{Christiane von Hofmannsthal}}s \label{K_L02503_3v}\edtext{Vermählung}{\lemma{\textnormal{\emph{Vermählung}}}\Cendnote{\textnormal{Diese
                  hatte Mitte Juni 1928 stattgefunden.}}}\label{K_L02503_3h} freuen wir uns sehr. Sie
               hat ein besonders liebenswertes Wesen, einen sehr schönen loyalen Character, viel
               Verstand, aber einen menschlichen keinen frauenhaften, und gerade die subtilen Waffen
               für den Lebenskampf, die nur der Frau, je mehr Frau sie ist, umso wirksamer gegeben
               sind, sind ihr versagt. Es war vielleicht zu fürchten dass gerade der Mann, der ihren
               Wert zu erkennen bestimmt war, sich unter den Besten dieser Generation, den
               Gefallenen, befunden hätte. Aber \textcolor{blue}{dieser}{}\ledrightnote{→\textcolor{blue}{Heinrich Zimmer}} gerade, den sie nun gefunden hat, ist aus vierjährigem
               Schützengrabendasein munter und unversehrt hervorgestiegen.\pend
           \pstart
           Ich lernte ihn diesen Winter in \textcolor{pink}{Heidelberg}{}\ledrightnote{\textcolor{pink}{Heidelberg}} kennen, und ich muss sagen, er gefiel mir sehr. Alles was er
               sagte, und wie er es sagte, war mir gleich sympathisch. Dabei streifte mich nicht
               einmal der Gedanke dass die {\pb}zwischen ihm und \textcolor{blue}{Christiane}{}\ledrightnote{\textcolor{blue}{Christiane von Hofmannsthal}} bestehende muntere
               gesprächige Freundschaft je zu etwas anderem führen könnte, als eben zu
               Freundschaft.\pend
           \pstart
           Dass Sie, wie ich von Freunden öfters gehört habe, an Ihrem \label{K_L02503_4v}\edtext{\textcolor{blue}{Schwiegersohn}{}\ledrightnote{→\textcolor{blue}{Arnoldo Cappellini}}}{\lemma{\textnormal{\emph{Schwiegersohn}}}\Cendnote{\textnormal{Die Hochzeit der noch nicht 18-jährigen
                     \textcolor{blue}{Lili} mit dem \textcolor{pink}{italienischen} Faschisten \textcolor{blue}{Arnoldo
                     Cappellini} hatte am 30. 6. 1927 stattgefunden.}}}\label{K_L02503_4h} wirklich einen Freund gewonnen
               haben, und eine Bereicherung Ihres Lebens, nehme ich als ein gutes Omen.\pend
           \pstart
           Ich drücke Ihnen herzlich die Hand, lieber guter Arthur.\pend
           \pstart Ihr\spacefill\mbox{Hugo.}\pend{}\endnumbering\briefempfaengerindex{Schnitzler, Arthur@\textsc{Schnitzler, Arthur}!zzzHofmannsthal, Hugo von@\emph{von Hugo von Hofmannsthal}!1928-07-101@{10. 7. {[}1928{]}}|)be}\mylabel{h}  \normalsize

\doendnotes{C}
\bigskip
\vfill

\clearpage

\footnotesize

\lohead{\textsc{register}}

% Definiere theindex-Environment komplett neu ohne reledmac
\makeatletter
\renewenvironment{theindex}{%
  \section*{\indexname}%
  \setlength{\parindent}{0pt}%
  \setlength{\parskip}{0pt plus 0.3pt}%
  \let\item\@idxitem
}{%
  \clearpage
}
\makeatother

\IfFileExists{\jobname-pw.ind}{\input{\jobname-pw.ind}}{}

\end{document}

      