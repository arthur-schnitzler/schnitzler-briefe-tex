%% latex-korrekturansicht-vorspann.tex
%% Vorspann für die Korrekturansicht.
%% Lädt die gemeinsame Datei latex-vorspann.tex mit gesetztem Schalter.

\newif\ifkorrekturansicht
\korrekturansichttrue

\input{../tex-inputs/latex-vorspann}


         
         \renewcommand{\erwaehntePersonen}{Personen: Hermann Bahr, Julius Bauer, Richard Beer-Hofmann, Mirjam Beer-Hofmann, Jakob Julius David, Robert Hirschfeld, Fedor Mamroth, Felix Salten, Paul Schlenther, Olga Schnitzler, Ludwig Speidel, Elisabeth Steinrück}
         \renewcommand{\erwaehnteInstitutionen}{Institutionen: Deutsches Theater Berlin, Neue Freie Presse}
         \renewcommand{\erwaehnteOrte}{Orte: Altaussee, Berlin, Rotensterngasse, Wien}
         \renewcommand{\erwaehnteWerke}{Werke: Berliner Tageblatt, Der Schleier der Beatrice. Schauspiel in fünf Akten, Erklärung [Schleier der Beatrice], Neues Wiener Tagblatt, Paul Schlenther und die Wiener Kritik, Schlaflied für Mirjam}
               \section[ Paul Goldmann an Arthur Schnitzler, 19. 9. {[}1900{]}]{Paul Goldmann an Arthur Schnitzler, 19. 9. {[}1900{]}}\nopagebreak\mylabel{v}\rehead{ }\normalsize\beginnumbering\briefempfaengerindex{Schnitzler, Arthur@\textsc{Schnitzler, Arthur}!zzzGoldmann, Paul@\emph{von Paul Goldmann}!1900-09-191@{19. 9. {[}1900{]}}|(be} \toendnotes[C]{\smallbreak\pagebreak[2]} \Standort{DLA, A:Schnitzler, HS.NZ85.1.3170.}
\physDesc{Brief, 1 Blatt, 4 Seiten
\newline{}Handschrift: blaue Tinte, deutsche Kurrent
\newline{}Schnitzler: 1) mit Bleistift das Jahr »{[}1{]}900« sowie »\textcolor{gray}{I)}« vermerkt  2) mit rotem Buntstift drei Unterstreichungen und eine seitliche
                                 Markierung}\toendnotes[C]{\smallbreak}\pstart
           \raggedleft{}{\pb}\textcolor{pink}{Berlin}{}\ledrightnote{\textcolor{pink}{Berlin}}, 19. September.\pend
           \pstart\center{}Mein lieber \label{K_L02931-1v}\edtext{\textcolor{blue}{Onkel}{}\ledrightnote{{$\rightarrow$}\textcolor{blue}{Fedor Mamroth}}}{\lemma{\textnormal{\emph{Onkel}}}\Cendnote{\textnormal{Es handelt sich hierbei nur um einen
                     Witz, der Brief erging nicht an \textcolor{blue}{Goldmann}s Onkel \textcolor{blue}{Fedor
                     Mamroth}.}}}\label{K_L02931-1h},\pend\pstart
           Den \label{K_L02931-11v}\edtext{\textcolor{green}{Artikel}{}\ledrightnote{{$\rightarrow$}\textcolor{green}{Paul Schlenther und die Wiener Kritik}} des »\textcolor{green}{Berliner Tageblatt}{}\ledrightnote{\textcolor{green}{Berliner Tageblatt}}«}{\lemma{\textnormal{\emph{Artikel … Tageblatt«}}}\Cendnote{\textnormal{[O. V.]: \emph{\textcolor{green}{Paul Schlenther und die Wiener
                        Kritik}}. In: \emph{\textcolor{green}{Berliner Tageblatt}},
                     Jg. 29, Nr. 470, 15. 9. 1900, Abend-Ausgabe,
                     S. 1–2.}}}\label{K_L02931-11h} hatte ich natürlich, unter Hervorhebung der Dir günſtigen
               Stellen, telegraphirt; die \textcolor{brown}{Redaktion}{}\ledrightnote{{$\rightarrow$}\textcolor{brown}{Neue Freie Presse}} hat mein Telegramm, wie ich heut
               ſehe, nicht veröffentlicht (was ich Dir im Vertrauen mittheile).\pend
           \pstart
           Im Übrigen iſt die \label{K_L02931-2v}\edtext{Affaire}{\lemma{\textnormal{\emph{Affaire}}}\Cendnote{\textnormal{siehe Richard Beer-Hofmann an Arthur Schnitzler, 14. 9. 1900}}}\label{K_L02931-2h} ſehr günſtig für Dich und ſehr ungünſtig für Herrn \textsc{\textcolor{blue}{Schlenther}{}\ledrightnote{\textcolor{blue}{Paul Schlenther}}}! Selbſt in \textcolor{pink}{Berlin}{}\ledrightnote{\textcolor{pink}{Berlin}} war man genöthigt, ihm
               harte Wahrheiten zu ſagen. Und was auch die Leute darüber ſagen, – und obwohl Du
               ſelbſt {\pb}ganz gewiß nicht \uline{dieſen} Zweck im Auge gehabt haſt, – die Wirkung iſt: \strikeout{alle} alle Welt iſt auf Dein \textcolor{green}{Stück}{}\ledrightnote{{$\rightarrow$}\textcolor{green}{Der Schleier der Beatrice. Schauspiel in fünf Akten}} aufmerkſam geworden, und die Bühnen
               haben einen Grund mehr, Dich aufzuführen. Daß die Fernſtehenden durch die Affaire ein
               falſches Bild von \strikeout{d} Dir gewinnen könnten, ſoll Dich
               nicht kümmern. Erſtens ſehe ich nicht ein, aus welchem Grunde. Und zweitens, ſelbſt
               wenn es ſo ſein ſollte: glaubſt Du, ſie haben vorher ein richtiges Bild von Dir
               gehabt? {\pb}Immerhin iſt zu conſtatiren, daß von den
                  \textcolor{pink}{Berlin}{}\ledrightnote{\textcolor{pink}{Berlin}}er Blättern, die Dir doch gewiß
               fernſtehen, keine ſich in einer Weiſe über Dich geäußert hat, die Dich hätte
               verletzen können. Und wenn das \textcolor{green}{Berliner
                  Tageblatt}{}\ledrightnote{\textcolor{green}{Berliner Tageblatt}} die \label{K_L02931-3v}\edtext{\textcolor{green}{Preisgebung des \textsc{\textcolor{blue}{Schlenther}{}\ledrightnote{\textcolor{blue}{Paul Schlenther}}}’ſchen Briefes}{}\ledrightnote{{$\rightarrow$}\textcolor{green}{Erklärung [Schleier der Beatrice]}}}{\lemma{\textnormal{\emph{Preisgebung … Briefes}}}\Cendnote{\textnormal{\textcolor{blue}{Hermann Bahr}, \textcolor{blue}{Julius Bauer}, \textcolor{blue}{J. J.
                        David}, \textcolor{blue}{Robert Hirschfeld}, \textcolor{blue}{Felix Salten} und \textcolor{blue}{Ludwig Speidel}: \emph{\textcolor{green}{Erklärung}}. In: \emph{\textcolor{green}{Neues Wiener
                        Tagblatt}} [u. a.], Jg. 34, Nr. 252, 14. 9. 1900, S. 9–10, hier: S. 9.}}}\label{K_L02931-3h} als inkorrekt \textcolor{green}{bezeichnet}{}\ledrightnote{{$\rightarrow$}\textcolor{green}{Paul Schlenther und die Wiener Kritik}} hat, ſo geſchieht
               dies wohl hauptſächlich darum, \strikeout{d\textcolor{gray}{a}} weil ſich die \textcolor{pink}{Berlin}{}\ledrightnote{\textcolor{pink}{Berlin}}er über \label{K_L02931-4v}\edtext{den das »\textcolor{brown}{Deutſche Theater}{}\ledrightnote{\textcolor{brown}{Deutsches Theater Berlin}}« betreffenden \textcolor{green}{Paſſus}{}\ledrightnote{{$\rightarrow$}\textcolor{green}{Paul Schlenther und die Wiener Kritik}}}{\lemma{\textnormal{\emph{den … Paſſus}}}\Cendnote{\textnormal{In dem erwähnten \textcolor{green}{Artikel} des \emph{\textcolor{green}{Berliner Tageblatt}} wird ein Brief \textcolor{blue}{Paul Schlenther}s zitiert, in dem er \textcolor{blue}{Schnitzler} vor dem \emph{\textcolor{brown}{Deutschen
                     Theater}} »warnen möchte«, da dieses der
                     »Riesenaufgabe« einer Aufführung von \emph{\textcolor{green}{Der Schleier der Beatrice}} »nicht gewachsen«
                  sei.}}}\label{K_L02931-4h} ärgern.\pend
           \pstart
           Daß ich \textsc{\textcolor{blue}{Richard}{}\ledrightnote{\textcolor{blue}{Richard Beer-Hofmann}}} verfehlt habe, thut mir unendlich leid. Anderſeits war ich ja über {\pb}eine Woche in \textcolor{pink}{Wien}{}\ledrightnote{\textcolor{pink}{Wien}}; und wenn er wirklich das Bedürfniß gehabt hätte, mit mir zuſammen zu
               ſein, ſo hätte er auch etwas früher \label{K_L02931-5v}\edtext{zurückkommen}{\lemma{\textnormal{\emph{zurückkommen}}}\Cendnote{\textnormal{aus \textcolor{pink}{Altaussee}, siehe Richard Beer-Hofmann an Arthur Schnitzler, 14. 9. 1900}}}\label{K_L02931-5h} können. Grüße ihn recht herzlich von mir und ſage ihm, daß ich ihm eine der
               wenigen freundlichen \strikeout{Erin} Erinnerungen an \strikeout{\textcolor{gray}{u}} meine diesjährige Urlaubsreiſe danke. Und er ſoll mir \textcolor{green}{\textsc{\textcolor{blue}{Mirjam}{}\ledrightnote{\textcolor{blue}{Mirjam Beer-Hofmann}}s} Wiegenlied}{}\ledrightnote{{$\rightarrow$}\textcolor{green}{Schlaflied für Mirjam}}
               ſchicken.\pend
           \pstart
           Ich leide, ſeit ich zurück bin, an einem Tag und Nacht andauernden, wühlenden
               Kopfſchmerz, bin vollkommen arbeitsunfähig und fürchte unheimliche Dinge in meinem
               Gehirn. Viele Grüße! Dein {\\}\spacefill\mbox{P. G.}\pend
           \pstart
           \noindent{}{\pb}\label{T_L02931-1v}\edtext{Viele Grüße an die beiden \label{K_L02931-7v}\edtext{\textcolor{blue}{Fräulein}{}\ledrightnote{{$\rightarrow$}\textcolor{blue}{Olga Schnitzler}{\newline}{$\rightarrow$}\textcolor{blue}{Elisabeth Steinrück}} aus der \textcolor{pink}{Rothe-Stern-Gaſſe}{}\ledrightnote{\textcolor{pink}{Rotensterngasse}}}{\lemma{\textnormal{\emph{Fräulein … Rothe-Stern-Gaſſe}}}\Cendnote{\textnormal{höchstwahrscheinlich Bezug auf \textcolor{blue}{Olga Gussmann} (später \textcolor{blue}{Schnitzler}) und ihre Schwester \textcolor{blue}{Elisabeth} (später \textcolor{blue}{Steinrück}) (vgl. A. S.: \emph{Tagebuch}, 21. 12. 1920)}}}\label{K_L02931-7h}!}{\lemma{\textnormal{\emph{Viele … Rothe-Stern-Gaſſe!}}}\Cendnote{\textnormal{kopfüber am oberen Rand der ersten
                     Seite}}}\label{T_L02931-1h}\pend
           \endnumbering\briefempfaengerindex{Schnitzler, Arthur@\textsc{Schnitzler, Arthur}!zzzGoldmann, Paul@\emph{von Paul Goldmann}!1900-09-191@{19. 9. {[}1900{]}}|)be}\mylabel{h}\begin{anhang}\end{anhang}\normalsize

\doendnotes{C}
\bigskip
\vfill

\clearpage

\footnotesize

\lohead{\textsc{register}}

% Definiere theindex-Environment komplett neu ohne reledmac
\makeatletter
\renewenvironment{theindex}{%
  \section*{\indexname}%
  \setlength{\parindent}{0pt}%
  \setlength{\parskip}{0pt plus 0.3pt}%
  \let\item\@idxitem
}{%
  \clearpage
}
\makeatother

\IfFileExists{\jobname-pw.ind}{\input{\jobname-pw.ind}}{}

\end{document}

      