%% latex-korrekturansicht-vorspann.tex
%% Vorspann für die Korrekturansicht.
%% Lädt die gemeinsame Datei latex-vorspann.tex mit gesetztem Schalter.

\newif\ifkorrekturansicht
\korrekturansichttrue

\input{../tex-inputs/latex-vorspann}


\renewcommand{\erwaehntePersonen}{Personen: Fedor Mamroth, Heinrich Schnitzler, Olga Schnitzler}
\renewcommand{\erwaehnteInstitutionen}{Institutionen: Schiller-Theater}
\renewcommand{\erwaehnteOrte}{Orte: Berlin, Deutsches Theater Berlin, Grand Hotel und Kurhaus Mürren, Hinterbrühl, Mürren, Schillertheater, Schweiz, Wien}
\renewcommand{\erwaehnteWerke}{Werke: Der Schleier der Beatrice. Schauspiel in fünf Akten, Der einsame Weg. Schauspiel in fünf Akten, Der junge Medardus. Dramatische Historie in einem Vorspiel und fünf Aufzügen}
\section[ Paul Goldmann an Arthur Schnitzler, 12. 8. {[}1902{]}]{Paul Goldmann an Arthur Schnitzler, 12. 8. {[}1902{]}}
\nopagebreak\mylabel{v}
\rehead{ }\normalsize\beginnumbering\briefempfaengerindex{Schnitzler, Arthur@\textsc{Schnitzler, Arthur}!zzzGoldmann, Paul@\emph{von Paul Goldmann}!1902-08-122@{12. 8. {[}1902{]}}|(be}
\toendnotes[C]{\smallbreak\pagebreak[2]}\Standort{DLA, A:Schnitzler, HS.NZ85.1.3172.}
\physDesc{Brief, 1 Blatt, 4 Seiten
\newline{}Handschrift: schwarze Tinte, deutsche Kurrent
\newline{}Schnitzler: mit Bleistift das Jahr »{[}1{]}902« vermerkt }\toendnotes[C]{\smallbreak}
\pstart
           \noindent{}\centering{}{\pb}\textcolor{gray}{\textbf{\textsc{\textcolor{pink}{Grand Hôtel {\kaufmannsund}
                           Kurhaus}{}\ledrightnote{\textcolor{pink}{Grand Hotel und Kurhaus Mürren}}, \textcolor{pink}{Mürren}{}\ledrightnote{\textcolor{pink}{Mürren}}}}}\pend
           
\pstart
           \noindent{}\centering{}\textcolor{gray}{\textbf{(\begin{otherlanguage}{french}\textcolor{pink}{SUISSE}{}\ledrightnote{\textcolor{pink}{Schweiz}}\end{otherlanguage})}}\pend
           
\pstart
           12. Auguſt\pend
           
\pstart{}Mein lieber Freund,\pend
\pstart
           Nochmals innigſte Glückwünſche. Nun haſt Du auch einen \label{K_L03219-1v}\edtext{\textcolor{blue}{Sohn}{}\ledrightnote{{$\rightarrow$}\textcolor{blue}{Heinrich Schnitzler}}}{\lemma{\textnormal{\emph{Sohn}}}\Cendnote{\textnormal{\textcolor{blue}{Heinrich Schnitzler}, geboren am 9. 8. 1902 in der \textcolor{pink}{Hinterbrühl}}}}\label{K_L03219-1h}. So kommt Alles. Ich wünſche Deinem \textcolor{blue}{Sohn}{}\ledrightnote{{$\rightarrow$}\textcolor{blue}{Heinrich Schnitzler}} all’ das Gute und Liebe, das ich Dir ſelbſt wünſche, –
               und das iſt ſehr viel. Wie wird er heißen? Sieht er ſchon Jemandem ähnlich? Was macht
               die \textcolor{blue}{Mutter}{}\ledrightnote{{$\rightarrow$}\textcolor{blue}{Olga Schnitzler}}? Sage ihr, bitte,
               in meinem Namen alles Herzliche.\pend
           
\pstart
           Über Deine literariſche {\pb}Produktivität freue ich
               mich nicht weniger. Von dem \textcolor{green}{Junggeſellenſtück}{}\ledrightnote{{$\rightarrow$}\textcolor{green}{Der einsame Weg. Schauspiel in fünf Akten}} verſpreche ich mir ſehr viel. Auf das \label{K_L03219-23v}\edtext{Alt-\textcolor{pink}{Wien}{}\ledrightnote{\textcolor{pink}{Wien}}er \textcolor{green}{Stück}{}\ledrightnote{{$\rightarrow$}\textcolor{green}{Der junge Medardus. Dramatische Historie in einem Vorspiel und fünf Aufzügen}}}{\lemma{\textnormal{\emph{Alt-Wiener Stück}}}\Cendnote{\textnormal{siehe dazu auch Paul Goldmann an Arthur Schnitzler, 24. 8. [1898]}}}\label{K_L03219-23h} bin ich beſonders neugierig; auch da erwarte ich mir etwas \strikeout{beſond\textcolor{gray}{e}} beſonders Feines! Wie haſt Du über die \label{K_L03219-3v}\edtext{»\textsc{\textcolor{green}{Beatrice}{}\ledrightnote{\textcolor{green}{Der Schleier der Beatrice. Schauspiel in fünf Akten}}}«}{\lemma{\textnormal{\emph{»Beatrice«}}}\Cendnote{\textnormal{siehe Paul Goldmann an Arthur Schnitzler, 14. 7. [1902]}}}\label{K_L03219-3h} entſchieden? Im »\textcolor{pink}{Schillertheater}{}\ledrightnote{\textcolor{pink}{Schillertheater}}« wird
               ſie aller Wahrſcheinlichkeit nach beſſer geſpielt werden, als im »\label{K_L03219-4v}\edtext{\textcolor{pink}{Deutſchen}{}\ledrightnote{{$\rightarrow$}\textcolor{pink}{Deutsches Theater Berlin}}}{\lemma{\textnormal{\emph{Deutſchen}}}\Cendnote{\textnormal{\textcolor{pink}{Deutschen Theater}}}}\label{K_L03219-4h}«, aber das \textcolor{brown}{Schillertheater}{}\ledrightnote{\textcolor{brown}{Schiller-Theater}} hat doch nicht
               das große literariſche Publikum und iſt ein Provinztheater in der \strikeout{\textcolor{gray}{H}}{ }\textcolor{pink}{Hauptſtadt}{}\ledrightnote{{$\rightarrow$}\textcolor{pink}{Berlin}}. Bitte, ſchreib’ mir
               bald {\pb}einige Einzelheiten über das Ereigniß in der
                  \textcolor{pink}{Hinterbrühl}{}\ledrightnote{\textcolor{pink}{Hinterbrühl}}, – an meine \textcolor{pink}{Berlin}{}\ledrightnote{\textcolor{pink}{Berlin}}er Adreſſe. Ich werde morgen{ }\textcolor{pink}{hier}{}\ledrightnote{{$\rightarrow$}\textcolor{pink}{Mürren}} von meinem \textcolor{blue}{Onkel}{}\ledrightnote{{$\rightarrow$}\textcolor{blue}{Fedor Mamroth}} abgeholt und weiß noch
               nicht, wohin wir wandern werden. Wir ſitzen \textcolor{pink}{hier}{}\ledrightnote{{$\rightarrow$}\textcolor{pink}{Mürren}} ſeit zwei Tagen im dichten Schneegeſtöber.
               Weihnachtswetter im Auguſt. Hände und Füße ſind mir ſtarr vor Kälte; das iſt der \introOben{}Grund\introOben{}{ }\substVorne{}\textsuperscript{Brief}\substDazwischen{}\strikeout{G\textcolor{gray}{run}}\substHinten{}, weshalb \strikeout{der} dieſer Brief nicht länger
               wird.\pend
           
\pstart
           {\pb}Tauſend Grüße! {\\[\baselineskip]}Dein {\\[\baselineskip]}\spacefill\mbox{Paul Goldmann.}\pend
           \leftskip=0em{}\endnumbering\briefempfaengerindex{Schnitzler, Arthur@\textsc{Schnitzler, Arthur}!zzzGoldmann, Paul@\emph{von Paul Goldmann}!1902-08-122@{12. 8. {[}1902{]}}|)be}\mylabel{h}
\begin{anhang}
\end{anhang}\normalsize

\doendnotes{C}
\bigskip
\vfill

\clearpage

\footnotesize

\lohead{\textsc{register}}

% Definiere theindex-Environment komplett neu ohne reledmac
\makeatletter
\renewenvironment{theindex}{%
  \section*{\indexname}%
  \setlength{\parindent}{0pt}%
  \setlength{\parskip}{0pt plus 0.3pt}%
  \let\item\@idxitem
}{%
  \clearpage
}
\makeatother

\IfFileExists{\jobname-pw.ind}{\input{\jobname-pw.ind}}{}

\end{document}

      