%% latex-korrekturansicht-vorspann.tex
%% Vorspann für die Korrekturansicht.
%% Lädt die gemeinsame Datei latex-vorspann.tex mit gesetztem Schalter.

\newif\ifkorrekturansicht
\korrekturansichttrue

\input{../tex-inputs/latex-vorspann}


               \section[Arthur Schnitzler an Richard Beer-Hofmann, 20. 5. 1897]{ Arthur Schnitzler an Richard Beer-Hofmann, 20. 5. 1897}\nopagebreak\mylabel{v}\rehead{ }\normalsize\beginnumbering\briefempfaengerindex{Beer-Hofmann, Richard@\textsc{Beer-Hofmann, Richard}!zzzSchnitzler, Arthur@\emph{von Arthur Schnitzler}!1897-05-201@{20. 5. 1897}|(be} \toendnotes[C]{\smallbreak\pagebreak[2]} \Standort{YCGL, MSS 31.}
\physDesc{Brief, 2 Blätter, 8 Seiten, Umschlag
\newline{}Handschrift: schwarze Tinte, deutsche Kurrent\newline{}Versand: 1) Stempel: »\nobreak{}\oindex{rue Milton@\textbf{rue Milton}, \emph{Straße (K.STR)}|pwk}Paris 2 B. Milton, 20 Mai 97, 7\textsuperscript{E}\nobreak{}«.  2) Stempel: »\nobreak{}\oindex{I., Innere Stadt@\textbf{I., Innere Stadt}, \emph{Bezirk (A.BZK)}|pwk}Wien 1/1, 22 5. 97, 9–10½V., Bestellt\nobreak{}«. }\buchAbdrucke{\weitereDrucke{1) Arthur Schnitzler: \emph{Briefe 1875–1912}. Hg. Therese Nickl und Heinrich Schnitzler. Frankfurt am Main: \emph{S. Fischer} 1981, S. 322–323.} \weitereDrucke{2) Arthur Schnitzler, Richard Beer-Hofmann: \emph{Briefwechsel 1891–1931}. Hg. Konstanze Fliedl. Wien, Zürich: \emph{Europaverlag} 1992, S. 104–105.} \weitereDrucke{3) Hermann Bahr, Arthur Schnitzler: \emph{Briefwechsel, Aufzeichnungen, Dokumente (1891–1931)}. Hg. Kurt Ifkovits und Martin Anton Müller. Göttingen: \emph{Wallstein} 2018.} }\toendnotes[C]{\smallbreak}\pstart{}{\pb}\textsc{Mr Dr Richard Beer-Hofmann}\pend{}\pstart{}\textcolor{pink}{\textsc{Wien}}{}\ledrightnote{\textcolor{pink}{Wien}}\pend{}\pstart{}\textcolor{pink}{\textsc{I. Wollzeile 15}}{}\ledrightnote{\textcolor{pink}{Wollzeile}}\pend{}\pstart{}\textcolor{pink}{\textsc{Autriche}}{}\ledrightnote{\textcolor{pink}{Österreich}}\pend{}{\bigskip}\pstart
           \raggedleft{}{\pb}20. 5. 97{\\}\textcolor{pink}{\textsc{Paris}}{}\ledrightnote{\textcolor{pink}{Paris}}.\pend
           \pstart
           Lieber Richard, die \textcolor{pink}{Pariſer}{}\ledrightnote{\textcolor{pink}{Paris}} Tage
               – ſie werden wahrſcheinlich bald »ſehr ſchön geweſen« ſein – nahen ihrem Ende; Montag
               fahre ich nach \textcolor{pink}{London}{}\ledrightnote{\textcolor{pink}{London}} und bin in den ersten
               Junitagen in \textcolor{pink}{Wien}{}\ledrightnote{\textcolor{pink}{Wien}}. Sie aber fahren bereits in den
               ſelben erſten Junitagen nach \textcolor{pink}{Iſchl}{}\ledrightnote{\textcolor{pink}{Bad Ischl}}?\pend
           \pstart
           Ich werde Sie doch hoffentlich noch in \textcolor{pink}{Wien}{}\ledrightnote{\textcolor{pink}{Wien}}
               finden? Beruhigen {\pb}Sie mich darüber, indem Sie mir
               eine Zeile nach \textcolor{pink}{London}{}\ledrightnote{\textcolor{pink}{London}}{ }ſchreiben. Meine Adreſſe ist sehr complicirt: bei
                  \textsc{\textcolor{blue}{Felix Markbreiter}{}\ledrightnote{\textcolor{blue}{Felix Markbreiter}}{ }\textcolor{pink}{London S E. Honor Oak, Woodville Hall}{}\ledrightnote{\textcolor{pink}{Honor Oak}}}. –\pend
           \pstart
           \textcolor{blue}{Paul}{}\ledrightnote{\textcolor{blue}{Paul Goldmann}} behauptet, ſo oft ich irgend ein
               Entzücken oder eine Befriedigung über irgend was hier äußere – und es wi{\geminationm}elt von ſolchen Gelegenheiten, dſs Sie einmal ge{\pb}äußert, \textcolor{pink}{Paris}{}\ledrightnote{\textcolor{pink}{Paris}}
               hätte Ihnen nichts zu ſagen. Sie werden das einmal beſchämt zurücknehmen. Sie ahnen
               nicht, was Ihnen \textcolor{pink}{Paris}{}\ledrightnote{\textcolor{pink}{Paris}} alles zu ſagen hätte und
               wie viel Sie gerne antworten möchten. Dieſe Stadt dampft von Cultur, und ich hab mich
               kaum über einen Menſchen ärgern kö{\geminationn}en, der mir zufällig
               heute grad ſagte, er ſei in \textcolor{pink}{Wien}{}\ledrightnote{\textcolor{pink}{Wien}} geweſen, {\pb}denke gern dran zurück: \textsc{c’est une
                  gentille petite ville}. Man ſpürt auch etwas wahres in dieſer Phraſe: dſs
               eigentlich die ganze Welt in \textcolor{pink}{Paris}{}\ledrightnote{\textcolor{pink}{Paris}} enthalten ſei;
               man hat eine Ahnung von Unendlichkeit, in der man beinah so einſam ſein könnte wie in
               der Wüſte. Wiſſen Sie, was mir eine große Freude ſein würde? einmal mit Ihnen hieher
               zu kommen – nicht {\pb}ohne Ihnen das Verſprechen abgeno{\geminationm}en zu haben, nicht bei jeder Auslage stehn zu bleiben.
               Ich würde Sie aber nie an die Seine führen, wo an den Quais auf den Steinbrüſtungen
               Millionen Bücher liegen – Sie würden dazu allein zwanzig Jahre brauchen. Dort findet
               man, wie Sie gleich ſehen werden, alle Bücher der Welt; {\pb}um mir eine Emotion zu verſchaffen, hab ich mit einer
               Verkäuferin um ein Exemplar von »\textcolor{green}{\textsc{Mourir}}{}\ledrightnote{\textcolor{green}{Sterben. Novelle}}« »gefeilſcht« – das Luder hat’s mir für 60 \textsc{centimes}
               gelaſſen – unaufgeſchnitten! (das Buch mein ich.)\pend
           \pstart
           – Mit \textcolor{blue}{Ihr}{}\ledrightnote{→\textcolor{blue}{Marie Reinhard}} bin ich ſehr
               zufrieden; ſanft, lieb, ein bischen rührend. Ich hab ſie wahrſcheinlich viel lieber,
               als wenn ich ſie lieb hätte. – Wir {\dots} na, wir reden ja in
                  \textcolor{pink}{Wien}{}\ledrightnote{\textcolor{pink}{Wien}} darüber. –\pend
           \pstart
           {\pb}Der \label{K_L00678_1v}\edtext{\textcolor{blue}{Graf}{}\ledrightnote{\textcolor{blue}{Max Graf}}}{\lemma{\textnormal{\emph{Graf}}}\Cendnote{\textnormal{\textcolor{blue}{Max Graf}}}}\label{K_L00678_1h}, dem Sie die Empfehlung an \strikeout{Richard}{ }\textcolor{blue}{Paul}{}\ledrightnote{\textcolor{blue}{Paul Goldmann}} mitgegeben, iſt, losgelöſt von den
               Leuten, unter denen er noch einer der anſtändigſten iſt, ein ganz widerliches
               Subjekt; verlogen und verlottert. Moralſchule \textcolor{blue}{Altenberg}{}\ledrightnote{\textcolor{blue}{Peter Altenberg}}, Beobachtungsſchule \textcolor{blue}{Bahr}{}\ledrightnote{\textcolor{blue}{Hermann Bahr}}.\pend
           \pstart
           \textcolor{blue}{Sie}{}\ledrightnote{→\textcolor{blue}{Marie Reinhard}}{ }ſitzt, während ich Ihnen ſchreibe, im Nebenzimmer
               und lieſt eben die \textcolor{green}{Scene}{}\ledrightnote{→\textcolor{green}{Reigen. Zehn Dialoge}}
               zwiſchen dem {\pb}Dichter (\textcolor{green}{Biebitz}{}\ledrightnote{→\textcolor{green}{Reigen. Zehn Dialoge}}) und der Schauſpielerin, die ich übrigens geändert
               habe, ſo dſs man ſagen kann: \textcolor{green}{Biebitz}{}\ledrightnote{→\textcolor{green}{Reigen. Zehn Dialoge}} bleibt \textcolor{green}{Biebitz}{}\ledrightnote{→\textcolor{green}{Reigen. Zehn Dialoge}}! – Aber ſonſt haben Sie hoffentlich mehr gearbeitet als ich. Nach
               dieſen zwei Dingen ſehn ich mich unbeſchreiblich: nach dem Schreiben und nach dem \textsc{Bicycle}! – Kö{\geminationn}en Sie’s endlich?
               (Bicycle natürlich. –)\pend
           \pstart
           Seien Sie herzlich gegrüßt. Ihr{\\[\baselineskip]}\spacefill\mbox{Arthur.}\pend
           \leftskip=0em{}\endnumbering\briefempfaengerindex{Beer-Hofmann, Richard@\textsc{Beer-Hofmann, Richard}!zzzSchnitzler, Arthur@\emph{von Arthur Schnitzler}!1897-05-201@{20. 5. 1897}|)be}\mylabel{h}  \normalsize

\doendnotes{C}
\bigskip
\vfill

\clearpage

\footnotesize

\lohead{\textsc{register}}

% Definiere theindex-Environment komplett neu ohne reledmac
\makeatletter
\renewenvironment{theindex}{%
  \section*{\indexname}%
  \setlength{\parindent}{0pt}%
  \setlength{\parskip}{0pt plus 0.3pt}%
  \let\item\@idxitem
}{%
  \clearpage
}
\makeatother

\IfFileExists{\jobname-pw.ind}{\input{\jobname-pw.ind}}{}

\end{document}

      