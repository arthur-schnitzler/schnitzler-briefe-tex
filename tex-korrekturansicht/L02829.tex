%% latex-korrekturansicht-vorspann.tex
%% Vorspann für die Korrekturansicht.
%% Lädt die gemeinsame Datei latex-vorspann.tex mit gesetztem Schalter.

\newif\ifkorrekturansicht
\korrekturansichttrue

\input{../tex-inputs/latex-vorspann}


               \section[ Paul Goldmann an Arthur Schnitzler, 15. 10. {[}1897{]}]{Paul Goldmann an Arthur Schnitzler, 15. 10. {[}1897{]}}\nopagebreak\mylabel{v}\rehead{ }\normalsize\beginnumbering\briefempfaengerindex{Schnitzler, Arthur@\textsc{Schnitzler, Arthur}!zzzGoldmann, Paul@\emph{von Paul Goldmann}!1897-10-151@{15. 10. {[}1897{]}}|(be} \toendnotes[C]{\smallbreak\pagebreak[2]} \Standort{DLA, A:Schnitzler, HS.NZ85.1.3167.}
\physDesc{Brief, 1 Blatt, 4 Seiten
\newline{}Handschrift: blaue Tinte, deutsche Kurrent
\newline{}Schnitzler: 1) mit Bleistift das Jahr »97« vermerkt 2) mit rotem Buntstift vier Unterstreichungen}\toendnotes[C]{\smallbreak}\pstart
           \noindent{}{\pb}\textcolor{gray}{\textbf{\textbf{\textcolor{brown}{Frankfurter Zeitung}{}\ledrightnote{\textcolor{brown}{Frankfurter Zeitung}}}}}\pend
           \pstart
           \textcolor{gray}{\textbf{(\textcolor{brown}{\begin{otherlanguage}{french}Gazette de Francfort\end{otherlanguage}}{}\ledrightnote{\textcolor{brown}{Frankfurter Zeitung}}).}}\pend
           \pstart
           \textcolor{gray}{\textbf{\textbf{\begin{otherlanguage}{french}Fondateur M.\end{otherlanguage}{ }\textcolor{blue}{L. Sonnemann}{}\ledrightnote{\textcolor{blue}{Leopold Sonnemann}}.}}}\pend
           \pstart
           \begin{otherlanguage}{french}\textcolor{gray}{\textbf{Journal politique, financier,}}\end{otherlanguage}\pend
           \pstart
           \begin{otherlanguage}{french}\textcolor{gray}{\textbf{commercial et littéraire.}}\end{otherlanguage}\pend
           \pstart
           \begin{otherlanguage}{french}\textcolor{gray}{\textbf{\textbf{Paraissant trois fois par jour.}}}\end{otherlanguage}\hfill \textsc{\textcolor{pink}{Paris}{}\ledrightnote{\textcolor{pink}{Paris}}}, 15. October.\pend
           \pstart
           \begin{otherlanguage}{french}\textcolor{gray}{\textbf{\textbf{Bureau à \textcolor{pink}{Paris}{}\ledrightnote{\textcolor{pink}{Paris}}}}}\end{otherlanguage}\pend
           \pstart
           \begin{otherlanguage}{french}\textcolor{gray}{\textbf{\textbf{\textcolor{pink}{10 Rue de la Bourse}{}\ledrightnote{\textcolor{pink}{rue de la Bourse}}.}}}\end{otherlanguage}\pend
           \pstart\center{}Mein lieber Freund,\pend\pstart
           Ich wollte Dir täglich ſchreiben, habe aber jetzt ausnahmsweiſe viel zu thun. Heut erſt kann ich Dir für Deinen lieben Brief danken,
               der mich wahrhaft beruhigt hat. Ich war wirklich ſchon in Sorge, weil ich ſo lange
               nichts \introOben{}von Dir\introOben{} hörte.\pend
           \pstart
           Wenn von dem Allen nur das Eine zurückbleibt, daß Du »\textcolor{blue}{Sie}{}\ledrightnote{→\textcolor{blue}{Marie Reinhard}}« lieber haſt als je, ſo weiß ich, wozu
               es gut war. Ich glaube immer mehr, daß »\textcolor{blue}{Sie}{}\ledrightnote{→\textcolor{blue}{Marie Reinhard}}« in Deinem Leben die Treue, die Ruhe, die Ordnung
               darſtellt. Je feſter Du mit ihr verbunden biſt, umſo beſſer iſts für Dich. Wie
               herrlich doch {\pb}das Leben waltet! Auch Noth und Tod
               ſind ihm nur ein Mittel, um neue Liebe hervorzurufen{\dotsfive}\pend
           \pstart
           Auch die ſonſtigen Mittheilungen Deines Briefes haben mich ſehr befriedigt. Wenn das
                  \textcolor{green}{Stück}{}\ledrightnote{→\textcolor{green}{Das Vermächtnis. Schauspiel in drei Akten}} ſo weit
               iſt, bekomme ichs wohl einmal auf einen Tag im Manuſkript zu ſehen? Zu düſter
               ſollteſt Du es freilich nicht machen. Kannſt Du nicht eine heitere oder wenigſtens
               verſöhnende Epiſoden-Figur einflicken?{\dotsfive}\pend
           \pstart
           Ich habe Dir noch nicht geſagt, wie ſehr ich mich in \textsc{\label{K_L02829-66v}\edtext{\textcolor{pink}{Salzburg}{}\ledrightnote{\textcolor{pink}{Salzburg}}}{\lemma{\textnormal{\emph{Salzburg}}}\Cendnote{\textnormal{Wohl ein Aufenthalt
                     nach der Abreise aus \textcolor{pink}{Ischl}, also Anfang
                        September 1897}}}\label{K_L02829-66h}} mit dem \textsc{\textcolor{blue}{Leo}{}\ledrightnote{\textcolor{blue}{Leo Van-Jung}}} gefreut habe. Was für ein lieber Menſch! Er kommt mir vor wie ein treuer Löwe.
                  {\pb}\textsc{\textcolor{blue}{Richard}{}\ledrightnote{\textcolor{blue}{Richard Beer-Hofmann}}} hatte ſein Möglichſtes gethan, um ihn davon abzureden, nach \textsc{\textcolor{pink}{Salzburg}{}\ledrightnote{\textcolor{pink}{Salzburg}}} zu kommen!\pend
           \pstart
           Von \textsc{\textcolor{blue}{Richard}{}\ledrightnote{\textcolor{blue}{Richard Beer-Hofmann}}} höre ich natürlich kein Wort. Vielleicht ſchreibſt Du mir einmal eine Zeile,
               wie es ihm, \textsc{\textcolor{blue}{Paula}{}\ledrightnote{\textcolor{blue}{Paula Beer-Hofmann}}} und »\textsc{\textcolor{blue}{Mirjam}{}\ledrightnote{\textcolor{blue}{Mirjam Beer-Hofmann}}}« geht? Auch \textsc{\textcolor{blue}{Salten}{}\ledrightnote{\textcolor{blue}{Felix Salten}}}, den ich in \textsc{\textcolor{pink}{Salzburg}{}\ledrightnote{\textcolor{pink}{Salzburg}}} ſah, hat mir ſehr gut gefallen. Iſt ein charmanter Menſch geworden. Daß Dir \textsc{\textcolor{blue}{Herzl}{}\ledrightnote{\textcolor{blue}{Theodor Herzl}}} zuwider iſt, glaub’ ich gern. So viel Prätention und nichts dahinter! So
               geiſtreich und ſo urtheilslos! Und ſo gar keinen Zuſammenhang mit dem wirklichen \strikeout{Leb} Leben. Aber ſchwarzer Bart und impoſantes
               Auftreten. Das ſind die Leute, die im Journalismus die großen Erfolge haben. {\pb}Bitte, ſchreib’ mir, ob Du \label{K_L02829-5v}\edtext{nach \textsc{\textcolor{pink}{Prag}{}\ledrightnote{\textcolor{pink}{Prag}}} vorleſen}{\lemma{\textnormal{\emph{nach Prag vorleſen}}}\Cendnote{\textnormal{\textcolor{blue}{Schnitzler} hielt sich von 24. 11. 1897 bis 28. 11. 1897 in \textcolor{pink}{Prag} auf. Am 25. 11. 1897 las er im \textcolor{pink}{Deutschen Casino} aus \emph{\textcolor{green}{Die Toten schweigen}} und \emph{\textcolor{green}{Weihnachts-Einkäufe}}. Am 27. 11. 1897 fand in \textcolor{blue}{Schnitzler}s Anwesenheit die Premiere von \emph{\textcolor{green}{Freiwild}} im \textcolor{pink}{Neuen Deutschen
                     Theater} statt.}}}\label{K_L02829-5h} gehſt? Und wann?\pend
           \pstart
           Von mir ſchreibe ich Dir lieber nichts. Es iſt die alte Geſchichte, ohne einen Zug
               von Änderung, \substVorne{}\textsuperscript{höchſtens}{\allowbreak}\substDazwischen{}eher\substHinten{} ſchlimmer als beſſer. Das iſt wirklich nicht intereſſant.\pend
           \pstart
           Grüße Deine \textcolor{blue}{Freundin}{}\ledrightnote{→\textcolor{blue}{Marie Reinhard}} und ſei
               Du ſelbſt von Herzen gegrüßt!\pend
           \pstart
           Dein {\\[\baselineskip]}\spacefill\mbox{Paul Goldmnn}\pend
           \leftskip=0em{}\endnumbering\briefempfaengerindex{Schnitzler, Arthur@\textsc{Schnitzler, Arthur}!zzzGoldmann, Paul@\emph{von Paul Goldmann}!1897-10-151@{15. 10. {[}1897{]}}|)be}\mylabel{h}\begin{anhang}\end{anhang}\normalsize

\doendnotes{C}
\bigskip
\vfill

\clearpage

\footnotesize

\lohead{\textsc{register}}

% Definiere theindex-Environment komplett neu ohne reledmac
\makeatletter
\renewenvironment{theindex}{%
  \section*{\indexname}%
  \setlength{\parindent}{0pt}%
  \setlength{\parskip}{0pt plus 0.3pt}%
  \let\item\@idxitem
}{%
  \clearpage
}
\makeatother

\IfFileExists{\jobname-pw.ind}{\input{\jobname-pw.ind}}{}

\end{document}

      