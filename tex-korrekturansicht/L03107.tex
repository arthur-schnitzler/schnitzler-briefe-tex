%% latex-korrekturansicht-vorspann.tex
%% Vorspann für die Korrekturansicht.
%% Lädt die gemeinsame Datei latex-vorspann.tex mit gesetztem Schalter.

\newif\ifkorrekturansicht
\korrekturansichttrue

\input{../tex-inputs/latex-vorspann}


\renewcommand{\erwaehntePersonen}{Personen: Julius Bauer, Eduard Michael Kafka, Bertha Karlsburg, Max L., Felix Salten}
\renewcommand{\erwaehnteOrte}{Orte: Berggasse, Café Kremser, Ordination Dr. Arthur Schnitzler Giselastraße 11, Wien, Wohnung und Ordination Johann Schnitzler Burgring 1}
\renewcommand{\erwaehnteWerke}{Werke: Tagebuch}
\section[Felix Salten an Arthur Schnitzler, {[}vor dem 24.? 1. 1892{]}]{Felix Salten an Arthur Schnitzler, {[}vor dem 24.? 1. 1892{]}}
\nopagebreak\mylabel{v}
\rehead{ }\normalsize\beginnumbering\briefempfaengerindex{Schnitzler, Arthur@\textsc{Schnitzler, Arthur}!zzzSalten, Felix@\emph{von Felix Salten}!1892-01-234@{{[}vor dem
                  24.? 1. 1892{]}}|(be}
\toendnotes[C]{\smallbreak\pagebreak[2]}\Standort{CUL, Schnitzler, B 89, A 1.}
\physDesc{Visitenkarte, 444 Zeichen
\newline{}Handschrift: Bleistift, lateinische Kurrent
\newline{}Schnitzler: mit Bleistift datiert: »Anfang 92« 
\newline{}Ordnung: mit Bleistift von unbekannter Hand nummeriert: »8« }\toendnotes[C]{\smallbreak}
\pstart
           \noindent{}{\pb}lieber Freund! Es wäre mir gerade gestern{ }\uline{sehr} lieb gewesen, wenn Sie in’s \textcolor{pink}{Kremser}{}\ledrightnote{\textcolor{pink}{Café Kremser}} geko{\geminationm}en wären. Ich hatte
               eine \label{K_L03107-1v}\edtext{Begegnung mit \textcolor{blue}{B}{}\ledrightnote{\textcolor{blue}{Bertha Karlsburg}}}{\lemma{\textnormal{\emph{Begegnung mit B}}}\Cendnote{\textnormal{Es dürfte sich bei »\textcolor{blue}{B.}« um \textcolor{blue}{Bertha Karlsburg} und damit
                  jene Person handeln, von der \textcolor{blue}{Schnitzler} am
                     24. 1. 1892 in
                  sein \emph{\textcolor{green}{Tagebuch}} schrieb: »\textcolor{blue}{Salten} hat von \textcolor{blue}{Kafka} erfahren, daß seine \textcolor{blue}{Gel.} seit Sommer ein Verh. mit \textcolor{blue}{Max L.} habe. Trotzdem verführt sie ihn
                     weiter.« – Der Eintrag dürfte zeitlich nach diesem Schreiben
                  anzusiedeln sein und nicht vom selben Tag stammen, da \textcolor{blue}{Schnitzler} an einem Sonntag kaum in seiner \textcolor{pink}{Ordination} anzutreffen gewesen sein
                  dürfte.}}}\label{K_L03107-1h}, hatte Gefühlsergüße anzuhören, und bin infolgedessen ganz hin.\pend
           
\pstart
           Ich muss jetzt zu \textcolor{blue}{Kafka}{}\ledrightnote{\textcolor{blue}{Eduard Michael Kafka}}, u. dann rasch zu \textcolor{blue}{Bauer}{}\ledrightnote{\textcolor{blue}{Julius Bauer}}, sonst wäre ich in Ihre \textcolor{pink}{Ordination}{}\ledrightnote{{$\rightarrow$}\textcolor{pink}{Ordination Dr. Arthur Schnitzler Giselastraße 11}} gekommen. Es ist möglich, dass
                  \textcolor{blue}{B.}{}\ledrightnote{\textcolor{blue}{Bertha Karlsburg}} mich noch \label{K_L03107-2v}\edtext{aufpaßt}{\lemma{\textnormal{\emph{aufpaßt}}}\Cendnote{\textnormal{im
                  Sinne von: auflauern}}}\label{K_L03107-2h}, ich habe heute schon wenigstens von ihr einen
                  überschweng{\pb}lichen Brief
               bekommen.\pend
           
\pstart
           Bitte, seien Sie im \label{K_L03107-3v}\edtext{\textcolor{pink}{Kremser}{}\ledrightnote{\textcolor{pink}{Café Kremser}}{ }heute{ }abend}{\lemma{\textnormal{\emph{Kremser heute abend}}}\Cendnote{\textnormal{Besuche im \textcolor{pink}{Café Kremser} sind in diesen Tagen keine im \emph{\textcolor{green}{Tagebuch}} vermerkt.}}}\label{K_L03107-3h}\textcolor{gray}{.}\pend
           \pstart Herzlich Ihr\pend{}
\pstart
           \centering{}\textcolor{gray}{\textbf{FELIX SALTEN}}\pend
           
\pstart
           \noindent{}\raggedleft{}\textcolor{gray}{\textbf{\textcolor{pink}{IX., BERGGASSE 13}{}\ledrightnote{\textcolor{pink}{Berggasse}}.}}\pend
           \endnumbering\briefempfaengerindex{Schnitzler, Arthur@\textsc{Schnitzler, Arthur}!zzzSalten, Felix@\emph{von Felix Salten}!1892-01-014@{{[}vor dem
                  24.? 1. 1892{]}}|)be}\mylabel{h}  \normalsize

\doendnotes{C}
\bigskip
\vfill

\clearpage

\footnotesize

\lohead{\textsc{register}}

% Definiere theindex-Environment komplett neu ohne reledmac
\makeatletter
\renewenvironment{theindex}{%
  \section*{\indexname}%
  \setlength{\parindent}{0pt}%
  \setlength{\parskip}{0pt plus 0.3pt}%
  \let\item\@idxitem
}{%
  \clearpage
}
\makeatother

\IfFileExists{\jobname-pw.ind}{\input{\jobname-pw.ind}}{}

\end{document}

      