%% latex-korrekturansicht-vorspann.tex
%% Vorspann für die Korrekturansicht.
%% Lädt die gemeinsame Datei latex-vorspann.tex mit gesetztem Schalter.

\newif\ifkorrekturansicht
\korrekturansichttrue

\input{../tex-inputs/latex-vorspann}


               \section[Arthur Schnitzler an Hermann Bahr, 12. 10. 1913]{ Arthur Schnitzler an Hermann Bahr, 12. 10. 1913}\nopagebreak\mylabel{v}\rehead{ }\normalsize\beginnumbering\briefempfaengerindex{Bahr, Hermann@\textsc{Bahr, Hermann}!zzzSchnitzler, Arthur@\emph{von Arthur Schnitzler}!1913-10-121@{12. 10. 1913}|(be} \toendnotes[C]{\smallbreak\pagebreak[2]} \Standort{TMW, HS AM 23394 Ba.}
\physDesc{Kartenbrief
\newline{}Handschrift: schwarze Tinte, deutsche Kurrent\newline{}Versand: Briefmarke nicht gestempelt \newline{}Ordnung: Lochung }\buchAbdrucke{\weitereDrucke{1) \emph{12. 10. 1913.} In: Arthur Schnitzler: \emph{The Letters of Arthur Schnitzler to Hermann Bahr}. Edited, annotated, and with an introduction, by Donald G.
                        Daviau. Chapel Hill: \emph{The University of North Carolina Press} 1978, S. 112 (University of North Carolina studies in the Germanic languages
                        and literatures, 89).} \weitereDrucke{2) Hermann Bahr, Arthur Schnitzler: \emph{Briefwechsel, Aufzeichnungen, Dokumente (1891–1931)}. Hg. Kurt Ifkovits und Martin Anton Müller. Göttingen: \emph{Wallstein} 2018, S. 491.} }\toendnotes[C]{\smallbreak}\pstart{}{\pb}Herrn Hermann Bahr, \pend{}\pstart{}\textcolor{pink}{Salzburg}{}\ledrightnote{\textcolor{pink}{Salzburg}}\pend{}\pstart{}\textsc{\textcolor{pink}{Schloss Arenberg}{}\ledrightnote{\textcolor{pink}{Schloss Arenberg}}}\pend{}{\bigskip}\pstart
           \raggedleft{}{\pb}\textcolor{pink}{Wien}{}\ledrightnote{\textcolor{pink}{Wien}}, 12. X. 913\pend
           \pstart{}Mein lieber Hermann,\pend\pstart
           dein ſchönes \textcolor{green}{Burkhardbuch}{}\ledrightnote{→\textcolor{green}{Erinnerung an Burckhard}}, von
               dem mir die \label{K_L02152_1v}\edtext{meiſten Kapitel ſchon
                  bekannt}{\lemma{\textnormal{\emph{meiſten … bekannt}}}\Cendnote{\textnormal{Vorabdrucke aus \emph{\textcolor{green}{Erinnerung an Burckhard}} waren in \emph{\textcolor{brown}{Der Merker}}, \emph{\textcolor{brown}{Neue Freie Presse}} und \emph{\textcolor{green}{Die neue Rundschau}} erschienen.}}}\label{K_L02152_1h} waren hab
               ich nun als ganzes, mit neuer Ergriffenheit geleſen, und danke dir von Herzen. Wenn
               es überhaupt möglich iſt \introOben{}einen\introOben{} Menſchen Leuten, die \substVorne{}\textsuperscript{\textcolor{gray}{\textcolor{blue}{Burckhar}{}\ledrightnote{\textcolor{blue}{Max Eugen Burckhard}}}}{\allowbreak}\substDazwischen{}ihn\substHinten{} nicht gekannt haben, näher zu bringen – ich glaube, mit deiner Geſtaltung
                  \textcolor{blue}{Burckhards}{}\ledrightnote{\textcolor{blue}{Max Eugen Burckhard}} m\substVorne{}\textsuperscript{uſs}\substDazwischen{}üßte\substHinten{} es gelungen sein. Dir und einigen wenigen andern bleibt ja in jedem Fall das
               Glück ihn gekannt und erkannt zu haben. Wie ſehr ſind die zu bedauern, die das eine
               verſäumt, das andre nicht vermocht haben! –\pend
           \pstart
           Viele Grüße von \textcolor{blue}{uns}{}\ledrightnote{→\textcolor{blue}{Olga Schnitzler}} zu \textcolor{blue}{Euch}{}\ledrightnote{→\textcolor{blue}{Anna Bahr-Mildenburg}}!{\\[\baselineskip]}Dein
                  \spacefill\mbox{Arthur}\pend
           \leftskip=0em{}\endnumbering\briefempfaengerindex{Bahr, Hermann@\textsc{Bahr, Hermann}!zzzSchnitzler, Arthur@\emph{von Arthur Schnitzler}!1913-10-121@{12. 10. 1913}|)be}\mylabel{h}  \normalsize

\doendnotes{C}
\bigskip
\vfill

\clearpage

\footnotesize

\lohead{\textsc{register}}

% Definiere theindex-Environment komplett neu ohne reledmac
\makeatletter
\renewenvironment{theindex}{%
  \section*{\indexname}%
  \setlength{\parindent}{0pt}%
  \setlength{\parskip}{0pt plus 0.3pt}%
  \let\item\@idxitem
}{%
  \clearpage
}
\makeatother

\IfFileExists{\jobname-pw.ind}{\input{\jobname-pw.ind}}{}

\end{document}

      