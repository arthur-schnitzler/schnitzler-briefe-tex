%% latex-korrekturansicht-vorspann.tex
%% Vorspann für die Korrekturansicht.
%% Lädt die gemeinsame Datei latex-vorspann.tex mit gesetztem Schalter.

\newif\ifkorrekturansicht
\korrekturansichttrue

\input{../tex-inputs/latex-vorspann}


\renewcommand{\erwaehntePersonen}{Personen: Theodore Rottenberg, Olga Schnitzler, Heinrich Schnitzler}
\renewcommand{\erwaehnteOrte}{Orte: Berlin, Dessauer Straße, Deutsches Theater Berlin, Italien, Neapel, Palermo, Pompei, Rom, Sizilien, Taormina, Wien}
\renewcommand{\erwaehnteWerke}{Werke: Berliner Theater. »Der einsame Weg«. Von Arthur Schnitzler, Der einsame Weg. Schauspiel in fünf Akten, Neue Freie Presse}
\section[ Paul Goldmann an Arthur Schnitzler, 14. 3. {[}1904{]}]{Paul Goldmann an Arthur Schnitzler, 14. 3. {[}1904{]}}
\nopagebreak\mylabel{v}
\rehead{ }\normalsize\beginnumbering\briefempfaengerindex{Schnitzler, Arthur@\textsc{Schnitzler, Arthur}!zzzGoldmann, Paul@\emph{von Paul Goldmann}!1904-03-142@{14. 3. {[}1904{]}}|(be}
\toendnotes[C]{\smallbreak\pagebreak[2]}\Standort{DLA, A:Schnitzler, HS.NZ85.1.3174.}
\physDesc{Brief, 2 Blätter, 5 Seiten, 3195 Zeichen
\newline{}Handschrift: blaue Tinte, deutsche Kurrent
\newline{}Schnitzler: 1) mit Bleistift das Jahr »{[}1{]}904« vermerkt  2) mit rotem Buntstift eine Unterstreichung}\toendnotes[C]{\smallbreak}
\pstart
           \noindent{}\raggedleft{}{\pb}\textcolor{gray}{\textbf{\textcolor{pink}{DESSAUERSTRASSE 19}{}\ledrightnote{\textcolor{pink}{Dessauer Straße}}}}\pend
           
\pstart
           \textcolor{pink}{Berlin}{}\ledrightnote{\textcolor{pink}{Berlin}}, 14. März.\pend
           
\pstart{}Mein lieber Freund,\pend
\pstart
           Dein lieber Brief, der mich, wenigſtens durch ſeinen Schlußabſatz, ſehr erfreut hat,
               traf mich inmitten einer ſtürmiſch \strikeout{be} bewegten Zeit.
               Meine \textcolor{blue}{Freundin}{}\ledrightnote{{$\rightarrow$}\textcolor{blue}{Theodore Rottenberg}} war – aus
               Gründen, die Du Dir denken kannſt – erkrankt, \strikeout{ſie} ſie
               hat längere Zeit hier auf einer Klinik gelegen, auch jetzt iſt ſie noch recht leidend
               und immer noch hier. Ich habe viel Aufregungen und Sorgen durchgemacht, und ſo kommt
               es, daß ich \introOben{}für\introOben{} Deinen Brief, den ich, wenn ich meinem
               Wunſche hätte \strikeout{\textcolor{gray}{erf}} folgen können, ſofort beantwortet hätte, Dir erſt heute danken kann.\pend
           
\pstart
           Ich unterlaſſe es, auf das Einzelne {\pb}einzugehen.
                  \label{K_L03440-3v}\edtext{Äußerungen}{\lemma{\textnormal{\emph{Äußerungen}}}\Cendnote{\textnormal{\textcolor{blue}{Schnitzler}s nicht überlieferter Brief dürfte
                  höchstwahrscheinlich eine Abrechnung mit \textcolor{blue}{Goldmann}s \textcolor{green}{Rezension} der Uraufführung von \emph{\textcolor{green}{Der
                     einsame Weg}} enthalten haben. Diese Aufführung fand am 13. 2. 1904 am \textcolor{pink}{Deutschen Theater} in \textcolor{pink}{Berlin} statt. \textcolor{blue}{Paul Goldmann}: \emph{\textcolor{green}{Berliner Theater. »Der einsame Weg«. Von Arthur
                        Schnitzler}}. In: \emph{\textcolor{green}{Neue Freie Presse}},
                     Nr. 14.187, 23. 2. 1904, Morgenblatt,
                     S. 1–3.}}}\label{K_L03440-3h} in Deinem Briefe wie »Dein kritiſches Gebahren«, – die
               Meinung, ich hätte Dir zugemuthet, das \textcolor{green}{Stück}{}\ledrightnote{{$\rightarrow$}\textcolor{green}{Der einsame Weg. Schauspiel in fünf Akten}} ſtatt als Trauerſpiel als Luſtſpiel zu ſchreiben – die
               Aufforderung »ich ſollte \strikeout{D} mir den Inhalt des Ganzen
               einmal überlegen«, – die Anſicht, ich wiſſe nicht immer »mit ſoviel Klugheit und
               Würde zu wägen« \textsc{etc.} – das alles zeigt mir nur von Neuem,
               wie unrichtig Du \strikeout{me} meine kritiſche Thätigkeit
               beurtheilſt und \strikeout{\textcolor{gray}{mit}} wie ſehr es Dir (wenn Du auch mir ein offenes Wort erlaubſt) an Verſtändniß
               für den Ernſt und die Höhe meines Strebens fehlt. Darüber läßt ſich, meiner Anſicht,
               nicht diskutiren, und Diskuſſionen ſchaffen nur \strikeout{eine}
               unnütze Verbitterung in einem Fall, wo, wie in dem {\pb}unſerigen, nicht eine Verſchiedenheit der Anſichten, ſondern eine Verſchiedenheit
               der Standpunkte vorliegt, die ihren Grund wohl darin haben, daß \substVorne{}\textsuperscript{ſich d\textcolor{gray}{×}\-\textcolor{gray}{×}}{\allowbreak}\substDazwischen{}unſere\substHinten{} Lebenswege ſich ſeit Langem getrennt und in verſchiedenen Richtungen bewegt
               haben.\pend
           
\pstart
           Eines nur bitte ich Dich, mir zu glauben: Es gehört zu den peinlichſten Aufgaben
               meiner Stellung, ein Stück von Dir \strikeout{\textcolor{gray}{×}} kritiſiren zu müſſen, wenn ich nicht ganz damit einverſtanden bin; und ich
               habe den ſehnlichen Wunſch, Dein nächſtes Stück möge ſo ſchön ſein, daß ich mit
               rückhaltsloſer Anerkennung darüber berichten kann, oder es \strikeout{\textcolor{gray}{×}} möge mir überhaupt erſpart bleiben, darüber zu berichten{\dotsfive}\pend
           
\pstart
           Von ganzem Herzen \strikeout{abe\textcolor{gray}{r}} aber ſtimme ich dem Schluß Deines Briefes zu, und ich danke Dir für dieſe
               lieben {\pb}und ſchönen Worte. Du haſt ganz recht, wenn
               Du ſagſt, daß das Beſte gelebt und nicht geſchrieben wird. Vielleicht wird es gut
               ſein, wenn wir fürs Erſte überhaupt vermeiden, über Literatur zu ſprechen. Aber im
               großen Leben bildet die Literatur ja nur ein ganz kleines Gebiet, und es bleibt noch
               Raum genug für eine Freundſchaft die auf dieſem literariſchen Gebiete nicht mehr
               zufammengehen kann. Was mich anlangt, ſo hoffe ich Dir dieſe Freundſchaft noch oft
               beweiſen zu können; und \strikeout{wenn} wenn Du mir Deine Hände
               reichſt, ſo wirſt Du die meinen immer bereit finden, ſie \strikeout{\textcolor{gray}{×}\-\textcolor{gray}{×}\-\textcolor{gray}{×}} in alter Treue und Herzlichkeit zu drücken.\pend
           
\pstart
           \strikeout{D\textcolor{gray}{×}\-\textcolor{gray}{×}} Ich merke aber, daß ich ein wenig in die großen Worte hineingerathen bin. Das
               iſt überflüſſig, und ich denke, wir Zwei verſtehen uns {\pb}auch ohne dieſe ſehr gut und werden uns – im Weſentlichen – immer verſtehen{\dotsfour}\pend
           
\pstart
           Ich hoffe, daß dieſer Brief Dich bereits inmitten der Vorbereitungen zur \label{K_L03440-4v}\edtext{\textcolor{pink}{ſicili}{}\ledrightnote{{$\rightarrow$}\textcolor{pink}{Sizilien}}aniſchen Reiſe}{\lemma{\textnormal{\emph{ſicilianiſchen Reiſe}}}\Cendnote{\textnormal{Zwischen 1. 5. 1904 und 29. 5. 1904 reisten \textcolor{blue}{Arthur und Olga Schnitzler} nach \textcolor{pink}{Italien}, u. a. nach \textcolor{pink}{Rom}, \textcolor{pink}{Neapel}, \textcolor{pink}{Pompei}, \textcolor{pink}{Palermo} und \textcolor{pink}{Taormina}.}}}\label{K_L03440-4h} trifft. Zu meiner Freude höre
               ich, daß der »\textcolor{green}{Einſame Weg}{}\ledrightnote{\textcolor{green}{Der einsame Weg. Schauspiel in fünf Akten}}« dem \textcolor{pink}{Berlin}{}\ledrightnote{\textcolor{pink}{Berlin}}er Publikum gefällt und daß das \textcolor{pink}{Theater}{}\ledrightnote{{$\rightarrow$}\textcolor{pink}{Deutsches Theater Berlin}} immer voll iſt. Laß’ mich wiſſen, wie
               es Dir und Deiner kleinen \textcolor{blue}{Familie}{}\ledrightnote{{$\rightarrow$}\textcolor{blue}{Olga Schnitzler}{\newline}{$\rightarrow$}\textcolor{blue}{Heinrich Schnitzler}} geht, und ſei herzlichſt gegrüßt von Deinem
               getreuen \spacefill\mbox{Paul Goldmann}\pend
           
\pstart
           \noindent{}Meine \textcolor{blue}{Freundin}{}\ledrightnote{{$\rightarrow$}\textcolor{blue}{Theodore Rottenberg}} bittet
                  mich, Dich zu grüßen.\pend
           \endnumbering\briefempfaengerindex{Schnitzler, Arthur@\textsc{Schnitzler, Arthur}!zzzGoldmann, Paul@\emph{von Paul Goldmann}!1904-03-142@{14. 3. {[}1904{]}}|)be}\mylabel{h}
\begin{anhang}
\end{anhang}\normalsize

\doendnotes{C}
\bigskip
\vfill

\clearpage

\footnotesize

\lohead{\textsc{register}}

% Definiere theindex-Environment komplett neu ohne reledmac
\makeatletter
\renewenvironment{theindex}{%
  \section*{\indexname}%
  \setlength{\parindent}{0pt}%
  \setlength{\parskip}{0pt plus 0.3pt}%
  \let\item\@idxitem
}{%
  \clearpage
}
\makeatother

\IfFileExists{\jobname-pw.ind}{\input{\jobname-pw.ind}}{}

\end{document}

      