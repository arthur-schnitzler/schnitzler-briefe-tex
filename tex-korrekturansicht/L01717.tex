%% latex-korrekturansicht-vorspann.tex
%% Vorspann für die Korrekturansicht.
%% Lädt die gemeinsame Datei latex-vorspann.tex mit gesetztem Schalter.

\newif\ifkorrekturansicht
\korrekturansichttrue

\input{../tex-inputs/latex-vorspann}


               \section[Stefan Großmann an Arthur Schnitzler, 7. 10. 1907]{ Stefan Großmann an Arthur Schnitzler, 7. 10. 1907}\nopagebreak\mylabel{v}\rehead{ }\normalsize\beginnumbering\briefempfaengerindex{Schnitzler, Arthur@\textsc{Schnitzler, Arthur}!zzzGrossmann, Stefan@\emph{von Stefan Großmann}!1907-10-073@{7. 10. 1907}|(be} \toendnotes[C]{\smallbreak\pagebreak[2]} \Standort{CUL, Schnitzler, B 34.}
\physDesc{Brief, 1 Blatt (Briefpapier mit Trauerrand), 3 Seiten
\newline{}Handschrift: schwarze Tinte, deutsche Kurrent
\newline{}Schnitzler: 1) mit Bleistift beschriftet: »Großma{\geminationn}« 2) auf der dritten Seite eine Antwortskizze mit Bleistift, die nur
                                 unsicher zu entziffern ist: »\noindent{}{\pb}Unter d Dichg – find
                                       ich nicht\textcolor{gray}{s} heiter –{ / }glaube, daſ\textcolor{gray}{s} 1 Nur \textsc{Excentric} für \strikeout{das}{ }\textcolor{brown}{N V} Publ
                                          paſſ\textcolor{gray}{e} (L Pb
                                          amuſi\textcolor{gray}{erte} ſehr.)–{ / }\uline{Nummer des Hauses}? –{ / }\strikeout{Bin froh}{ }\strikeout{Wo ist genau \textcolor{gray}{×}}.{ / }Bei\textcolor{gray}{de} Titel, d i. nicht ofter{ / }K\textcolor{gray}{ö}nnte: \textcolor{green}{N. L.} – \textcolor{green}{D. l.
                                             \textcolor{gray}{M}.}«\newline{}Ordnung: mit Bleistift von unbekannter Hand nummeriert:
                                 »6« }\toendnotes[C]{\smallbreak}\pstart
           \noindent{}{\pb}\textcolor{gray}{\textbf{\textcolor{brown}{Freie Volksbühne}{}\ledrightnote{\textcolor{brown}{Wiener Freie Volksbühne}}}}\pend
           \pstart
           \textcolor{gray}{\textbf{\textcolor{pink}{Wien}{}\ledrightnote{\textcolor{pink}{Wien}} VI/\textsubscript{1}.}}\pend
           \pstart
           \textcolor{gray}{\textbf{\textcolor{pink}{Mariahilferſtraße Nr. 89}{}\ledrightnote{\textcolor{pink}{Mariahilferstraße}}.}}\hfill \textcolor{gray}{\textbf{\textcolor{pink}{Wien}{}\ledrightnote{\textcolor{pink}{Wien}}, am}} 7. Okt. \textcolor{gray}{\textbf{190}}7\pend
           \pstart
           \textcolor{gray}{\textbf{Poſtſparkaſſen-Konto Nr. 87.544.}}\pend
           \pstart\center{}Sehr geehrter Herr.\pend\pstart
           Ich bitte um Entſchuldigung, daſs ich Ihr freundliches Schreiben 2 Tage unerledigt
               ließ.\pend
           \pstart
           Diese 2 Tage wurden jedoch zur Aufnehmung des Vortraglokales benöthigt. Wenn es Ihnen
               alſo recht iſt, findet die Vorleſung\pend
           \pstart
           \centering{}\uline{Mit{[}t{]}woch}, den \uline{16.} Oktober\pend
           \pstart
           \noindent{}\centering{}\uline{acht Uhr abends}\pend
           \pstart
           \noindent{}im Saale des \uline{\textcolor{pink}{Verbandsheim}{}\ledrightnote{\textcolor{pink}{Verbandsheim}}}{ }\textcolor{pink}{Wien VI}{}\ledrightnote{\textcolor{pink}{VI., Mariahilf}}. \uline{\textcolor{pink}{Königsegggaſſe}{}\ledrightnote{\textcolor{pink}{Königseggasse}}} (neben der \textcolor{pink}{Gumpendorferſtraße}{}\ledrightnote{\textcolor{pink}{Gumpendorfer Straße}}) statt. Der
               Saal faſst 500 Personen.\pend
           \pstart
           Auch ich würde es für ſehr gut halten, wenn außer {\pb}dem »\textcolor{green}{\textsc{\uline{\label{T_L01717_1v}\edtext{Lieutenant}{\lemma{\textnormal{\emph{Lieutenant}}}\Cendnote{\textnormal{Er schreibt:
                              »Leuitenant«.}}}\label{T_L01717_1h}}}\textsc{\uline{{ }Gustl}}}{}\ledrightnote{\textcolor{green}{Lieutenant Gustl. Novelle}}« eine \uline{dialogiſche} Arbeit vorgeleſen würde, weil
               dies als Contraſt zu jenem großen \strikeout{Monl} Monolog
               belebend wirken würde. Leider kann ich beim beſten Willen die \strikeout{Werk} Titel nicht entziffern, die Sie angeben.\pend
           \pstart
           Es verſteht ſich von ſelbſt, daſs jene Arbeiten die paſſendsten ſind, die mit dem
               Ideenkreis der Zuhörer \introOben{}durch\introOben{} die ſtärkſten \strikeout{Be} Berührungspunkte verbunden ſind.\pend
           \pstart
           Und im Übrigen würde ich den Leuten nach der ſcharfen Eindringlichkeit des »\textcolor{green}{\label{T_L01717_2v}\edtext{Leuitenant}{\lemma{\textnormal{\emph{Leuitenant}}}\Cendnote{\textnormal{Er schreibt: »Leuitenant«.}}}\label{T_L01717_2h} Guſtl}{}\ledrightnote{\textcolor{green}{Lieutenant Gustl. Novelle}}«
               eine \strikeout{Erl} Weile Lächeln u Lachen gönnen.\pend
           \pstart
           Ihre gütige Entſcheidungen erhoffend{\\[\baselineskip]}ſehr ergeben:\spacefill\mbox{Stefan
                  Großmann}\pend
           \leftskip=0em{}\endnumbering\briefempfaengerindex{Schnitzler, Arthur@\textsc{Schnitzler, Arthur}!zzzGrossmann, Stefan@\emph{von Stefan Großmann}!1907-10-073@{7. 10. 1907}|)be}\mylabel{h}  \normalsize

\doendnotes{C}
\bigskip
\vfill

\clearpage

\footnotesize

\lohead{\textsc{register}}

% Definiere theindex-Environment komplett neu ohne reledmac
\makeatletter
\renewenvironment{theindex}{%
  \section*{\indexname}%
  \setlength{\parindent}{0pt}%
  \setlength{\parskip}{0pt plus 0.3pt}%
  \let\item\@idxitem
}{%
  \clearpage
}
\makeatother

\IfFileExists{\jobname-pw.ind}{\input{\jobname-pw.ind}}{}

\end{document}

      