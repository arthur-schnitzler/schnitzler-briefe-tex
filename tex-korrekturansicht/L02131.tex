%% latex-korrekturansicht-vorspann.tex
%% Vorspann für die Korrekturansicht.
%% Lädt die gemeinsame Datei latex-vorspann.tex mit gesetztem Schalter.

\newif\ifkorrekturansicht
\korrekturansichttrue

\input{../tex-inputs/latex-vorspann}


               \section[Georg Engländer an Arthur Schnitzler, 25. 4. 1913]{ Georg Engländer an Arthur Schnitzler, 25. 4. 1913}\nopagebreak\mylabel{v}\rehead{ }\normalsize\beginnumbering\briefempfaengerindex{Schnitzler, Arthur@\textsc{Schnitzler, Arthur}!zzzEnglaender, Georg@\emph{von Georg Engländer}!1913-04-251@{25. 4. 1913}|(be} \toendnotes[C]{\smallbreak\pagebreak[2]} \Standort{DLA, A:Schnitzler, HS.NZ85.1.2889.}
\physDesc{Brief, 1 Blatt, 2 Seiten
\newline{}Schreibmaschine
\newline{}Handschrift: schwarze Tinte (\noindent{}Unterschrift)}\toendnotes[C]{\smallbreak}\pstart
           \noindent{}{\pb}\textcolor{gray}{\textbf{\textit{Georg Engländer}}}\pend
           \pstart
           \textcolor{gray}{\textbf{\textit{\textcolor{pink}{Wien, III. Seidlgasse
                            23}{}\ledrightnote{\textcolor{pink}{Seidlgasse}}.}}}\pend
           \pstart
           \raggedleft{}\textcolor{pink}{Wien}{}\ledrightnote{\textcolor{pink}{Wien}},
                            25. April 1913\pend
           \pstart{}Hochgeehrter Herr!\pend\pstart
           Ich freue mich Ihnen die Mitteilung machen zu könne{[}n{]}, dass
                    ich heute von Dr. \textcolor{blue}{Hansy}{}\ledrightnote{\textcolor{blue}{Franz Hansy}}{ }\textcolor{pink}{Semmering}{}\ledrightnote{\textcolor{pink}{Semmering}} die Antwort
                    erhielt, dass er nicht nur gerne bereit ist, meinem Bruder \textcolor{blue}{Peter}{}\ledrightnote{\textcolor{blue}{Peter Altenberg}} für einige Zeit, quasi als Nachkur, in seiner Anstalt
                    aufzunehmen, sondern ihm auch in entgegenkommendster Weise einen
                    ausserordentlich bescheidenen Preis per Tag notiert hat.\pend
           \pstart
           Ihre besonders freundschaftliche Teilnahme sowie Ihre besonders liebenswürdige
                    Mühe, die Sie hierauf verwendet, verpflichten mich selbstverständlich, Ihnen
                    sofort hievon Bericht zu geben, wie Ihnen auch gleichzeitig zu melden, dass ich
                        Sonntag{ }{\pb}Nachmittag mit dem \textcolor{blue}{Bruder}{}\ledrightnote{→\textcolor{blue}{Peter Altenberg}} die diesbezügliche Entscheidung treffen werde
                    und es seinem Belieben überlassen werde, ob er Montag vorerst für
                    ein od. zwei Tage unter meiner Aufsicht in \textcolor{pink}{Wien}{}\ledrightnote{\textcolor{pink}{Wien}}
                    verbringen will, oder sofort schon Montag mit mir od. meiner \textcolor{blue}{Schwester}{}\ledrightnote{→\textcolor{blue}{Margarethe Engländer}} auf den \textcolor{pink}{Semmering}{}\ledrightnote{\textcolor{pink}{Semmering}} fahren will.\pend
           \pstart
           Ich hoffe nunmehr, dass der peinliche Konflikt zwischen unserer Verantwortung und
                    dem natürlichen Drange meines \textcolor{blue}{Bruders}{}\ledrightnote{→\textcolor{blue}{Peter Altenberg}} zu seiner möglichsten Unabhängigkeit beigelegt sein dürfte und
                    verbleibe mit nochmaligem ausserordentlichen und herzlichstem Danke Ihr in\pend
           \pstart
           Hochachtung ergebenster{\\[\baselineskip]}\spacefill\mbox{{[}hs.:{]} Georg
                    Engländer}\pend
           \leftskip=0em{}\endnumbering\briefempfaengerindex{Schnitzler, Arthur@\textsc{Schnitzler, Arthur}!zzzEnglaender, Georg@\emph{von Georg Engländer}!1913-04-251@{25. 4. 1913}|)be}\mylabel{h}  \normalsize

\doendnotes{C}
\bigskip
\vfill

\clearpage

\footnotesize

\lohead{\textsc{register}}

% Definiere theindex-Environment komplett neu ohne reledmac
\makeatletter
\renewenvironment{theindex}{%
  \section*{\indexname}%
  \setlength{\parindent}{0pt}%
  \setlength{\parskip}{0pt plus 0.3pt}%
  \let\item\@idxitem
}{%
  \clearpage
}
\makeatother

\IfFileExists{\jobname-pw.ind}{\input{\jobname-pw.ind}}{}

\end{document}

      