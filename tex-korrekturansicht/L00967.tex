%% latex-korrekturansicht-vorspann.tex
%% Vorspann für die Korrekturansicht.
%% Lädt die gemeinsame Datei latex-vorspann.tex mit gesetztem Schalter.

\newif\ifkorrekturansicht
\korrekturansichttrue

\input{../tex-inputs/latex-vorspann}


               \section[Richard Beer-Hofmann an Arthur Schnitzler, 2. 9. 1899]{ Richard Beer-Hofmann an Arthur Schnitzler, 2. 9. 1899}\nopagebreak\mylabel{v}\rehead{ }\normalsize\beginnumbering\briefempfaengerindex{Schnitzler, Arthur@\textsc{Schnitzler, Arthur}!zzzBeer-Hofmann, Richard@\emph{von Richard Beer-Hofmann}!1899-09-021@{2. 9. 1899}|(be} \toendnotes[C]{\smallbreak\pagebreak[2]} \Standort{CUL, Schnitzler, B 8.}
\physDesc{Bildpostkarte
\newline{}Handschrift: schwarze Tinte, lateinische Kurrent\newline{}Versand: 1) Stempel: »\nobreak{}\oindex{Seeboden@\textbf{Seeboden}, \emph{http://www.geonames.org/ontologyA.ADM3}|pwk}Seeboden, 9 9 99\nobreak{}«.  2) Stempel: »\nobreak{}\oindex{Bad Ischl@\textbf{Bad Ischl}, \emph{Besiedelter Ort (A.BSO)}|pwk}Ischl, 4. 9. 99, 10–11V\nobreak{}«. \newline{}Ordnung: mit Bleistift von unbekannter Hand nummeriert:
                                    »139« }\buchAbdrucke{\weitereDrucke{Arthur Schnitzler, Richard Beer-Hofmann: \emph{Briefwechsel 1891–1931}. Hg. Konstanze Fliedl. Wien, Zürich: \emph{Europaverlag} 1992, S. 134.} }\toendnotes[C]{\smallbreak}\pstart{}{\pb}D\textsuperscript{r}
                  Arthur Schnitzler\pend{}\pstart{}\textcolor{pink}{Ischl}{}\ledrightnote{\textcolor{pink}{Bad Ischl}}\pend{}\pstart{}\textcolor{pink}{Pension Petter}{}\ledrightnote{\textcolor{pink}{Hotel und Pension Rudolfshöhe (Leopold Petter)}}. \pend{}{\bigskip}\pstart
           \noindent{}\centering{}\textcolor{gray}{\textbf{{\pb}\textcolor{pink}{Villa Platzer}{}\ledrightnote{\textcolor{pink}{Villa Platzer}}}}\pend
           \pstart
           \raggedleft{}\textcolor{pink}{\textcolor{gray}{\textbf{Seeboden}}}{}\ledrightnote{\textcolor{pink}{Seeboden}}, { }2/IX \textcolor{gray}{\textbf{18}}99\pend
           \pstart
           Erst hab ich »\label{K_L00967_1v}\edtext{Gigarl}{\lemma{\textnormal{\emph{Gigarl}}}\Cendnote{\textnormal{österreichisch: Geck}}}\label{K_L00967_1h}« gelesen, dann
               begriffen daß es »Gigant« heißen \uline{muß}. Beflaggen wäre
               zu wenig gewesen. Von \uline{6. Sept} an ist meine Adresse
                  \uline{\textcolor{pink}{Sachsenburg}{}\ledrightnote{\textcolor{pink}{Sachsenburg}}{ }\textcolor{pink}{Kärnten}{}\ledrightnote{\textcolor{pink}{Kärnten}}{ }\textcolor{pink}{Gasthof ›Fritz‹}{}\ledrightnote{\textcolor{pink}{Gasthof Fritz}}.}{ }\textcolor{blue}{Hugo}{}\ledrightnote{\textcolor{blue}{Hugo von Hofmannsthal}} u Sie grüßt herzl.\pend
           \pstart \spacefill\mbox{R.}\pend{}\endnumbering\briefempfaengerindex{Schnitzler, Arthur@\textsc{Schnitzler, Arthur}!zzzBeer-Hofmann, Richard@\emph{von Richard Beer-Hofmann}!1899-09-021@{2. 9. 1899}|)be}\mylabel{h}  \normalsize

\doendnotes{C}
\bigskip
\vfill

\clearpage

\footnotesize

\lohead{\textsc{register}}

% Definiere theindex-Environment komplett neu ohne reledmac
\makeatletter
\renewenvironment{theindex}{%
  \section*{\indexname}%
  \setlength{\parindent}{0pt}%
  \setlength{\parskip}{0pt plus 0.3pt}%
  \let\item\@idxitem
}{%
  \clearpage
}
\makeatother

\IfFileExists{\jobname-pw.ind}{\input{\jobname-pw.ind}}{}

\end{document}

      