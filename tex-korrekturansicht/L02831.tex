%% latex-korrekturansicht-vorspann.tex
%% Vorspann für die Korrekturansicht.
%% Lädt die gemeinsame Datei latex-vorspann.tex mit gesetztem Schalter.

\newif\ifkorrekturansicht
\korrekturansichttrue

\input{../tex-inputs/latex-vorspann}


               \section[ Paul Goldmann an Arthur Schnitzler, 19. 11. {[}1897{]}]{Paul Goldmann an Arthur Schnitzler, 19. 11. {[}1897{]}}\nopagebreak\mylabel{v}\rehead{ }\normalsize\beginnumbering\briefempfaengerindex{Schnitzler, Arthur@\textsc{Schnitzler, Arthur}!zzzGoldmann, Paul@\emph{von Paul Goldmann}!1897-11-192@{19. 11. {[}1897{]}}|(be} \toendnotes[C]{\smallbreak\pagebreak[2]} \Standort{DLA, A:Schnitzler, HS.NZ85.1.3167.}
\physDesc{Brief, 2 Blätter, 8 Seiten
\newline{}Handschrift: blaue Tinte, deutsche Kurrent
\newline{}Schnitzler: 1) mit Bleistift das Jahr »97« vermerkt 2) mit rotem Buntstift drei Unterstreichungen}\toendnotes[C]{\smallbreak}\pstart
           \noindent{}{\pb}\textcolor{gray}{\textbf{\textbf{\textcolor{brown}{Frankfurter Zeitung}{}\ledrightnote{\textcolor{brown}{Frankfurter Zeitung}}}}}\pend
           \pstart
           \textcolor{gray}{\textbf{(\textcolor{brown}{\begin{otherlanguage}{french}Gazette de Francfort\end{otherlanguage}}{}\ledrightnote{\textcolor{brown}{Frankfurter Zeitung}}).}}\pend
           \pstart
           \textcolor{gray}{\textbf{\textbf{\begin{otherlanguage}{french}Fondateur M.\end{otherlanguage}{ }\textcolor{blue}{L. Sonnemann}{}\ledrightnote{\textcolor{blue}{Leopold Sonnemann}}.}}}\pend
           \pstart
           \begin{otherlanguage}{french}\textcolor{gray}{\textbf{Journal politique, financier,}}\end{otherlanguage}\pend
           \pstart
           \begin{otherlanguage}{french}\textcolor{gray}{\textbf{commercial et littéraire.}}\end{otherlanguage}\pend
           \pstart
           \begin{otherlanguage}{french}\textcolor{gray}{\textbf{\textbf{Paraissant trois fois par jour.}}}\end{otherlanguage}\pend
           \pstart
           \begin{otherlanguage}{french}\textcolor{gray}{\textbf{\textbf{Bureau à \textcolor{pink}{Paris}{}\ledrightnote{\textcolor{pink}{Paris}}}}}\end{otherlanguage}\hfill \textsc{\textcolor{pink}{Paris}{}\ledrightnote{\textcolor{pink}{Paris}}}, 19. Nov.\pend
           \pstart
           \begin{otherlanguage}{french}\textcolor{gray}{\textbf{\textbf{\textcolor{pink}{10 Rue de la Bourse}{}\ledrightnote{\textcolor{pink}{rue de la Bourse}}.}}}\end{otherlanguage}\pend
           \pstart{}Mein lieber Freund,\pend\pstart
           Ich ſchreibe Dir heut nur in Kürze, um mich zu
                  entſchuldigen\strikeout{\textcolor{gray}{.}} und Dir für Deine Nachſicht zu danken. Seit Wochen warte ich vergebens auf
               eine freie Stunde, um \strikeout{\textcolor{gray}{×}\-\textcolor{gray}{×}\-\textcolor{gray}{×}} Dir zu \strikeout{ſ\textcolor{gray}{c}h\textcolor{gray}{×}} ſchreiben. Seit ich Deinen letzten, ſo ſchönen und ergreifenden Brief mit der
               traurigen Nachricht erhielt, vergeht kein Tag, wo ich nicht mit der Abſicht aufſtehe:
               Heut wird geſchrieben. Aber die Ereigniſſe ſind erbarmungslos und laſſen mich nicht
               zu Athem kommen. \strikeout{Du} Du kannſt Dir nicht vorſtellen,
               welche Zeit wir {\pb}hier durchmachen. Es geht zu wie im
               Tollhaus. Seit Wochen leiſte ich übermenſchliche Arbeits-Anſtrengungen. Du verfolgſt
               ja vielleicht auch von fern das Wiedererwachen der Affaire \textsc{\textcolor{blue}{Dreyfu\textcolor{gray}{s}}{}\ledrightnote{\textcolor{blue}{Alfred Dreyfus}}}. Seit ich Journaliſt bin, habe ich etwas ſo Aufregendes nicht miterlebt. Es
               wird allmälig eine Kriſis daraus, die das ganze \textcolor{pink}{Land}{}\ledrightnote{→\textcolor{pink}{Frankreich}} zu ergreifen beginnt. Es herrſcht eine Fieber-Athmoſphäre,
               und wenn man da mitten drin lebt und außerdem die Pflichten des Berufes erfüllen, das
               heißt ſich Meinungen bilden und das Publicum informiren muß, und wenn man außerdem
               eine \label{K_L02831-2v}\edtext{perſönliche {\pb}Stellung in der Angelegenheit eingenommen}{\lemma{\textnormal{\emph{perſönliche … eingenommen}}}\Cendnote{\textnormal{siehe Paul Goldmann an Arthur Schnitzler, 22. 9. [1896]}}}\label{K_L02831-2h} hat und keinen Tag die Zeitungen in die Hand nehmen kann, ohne fürchten zu
               müſſen, ſich als Spion oder Verräther entehrt zu ſehen, – wenn das Alles und noch
               mehr auf Einen einſtürmt, ſo kannſt Du Dir denken, in welcher Gemüths- und
               Nerven-Verfaſſung man ſich befindet. Die Ruhe, um auf Deine ſo lieben und ſchönen
               Briefe auch nur annähernd in einem \strikeout{ent} entſprechenden
               Ton zu antworten, iſt unmöglich zu finden. Nachdem {\pb}Du mir ſolange verziehen haſt, verzeihſt Du mir wohl noch ein wenig, bis endlich,
               endlich \strikeout{\textcolor{gray}{d}} die Stunde der Sammlung kommt, um Dir den ſeit Wochen geplanten langen Brief
               zu ſchreiben.\pend
           \pstart
           Und nun habe ich noch eine große Bitte. Mit der \label{K_L02831-45v}\edtext{\textcolor{blue}{Familie \textsc{B.}}{}\ledrightnote{→\textcolor{blue}{Charlotte Bondy}{\newline}→\textcolor{blue}{Vít Šalomoun Bondy}{\newline}→\textcolor{blue}{Alice Ziegler}}}{\lemma{\textnormal{\emph{Familie B.}}}\Cendnote{\textnormal{\textcolor{blue}{Vít Šalomoun und Charlotte Bondy},
                  bzw. die jüngere Tochter \textcolor{blue}{Alice} (verh. \textcolor{blue}{Ziegler})}}}\label{K_L02831-45h} in \textsc{\textcolor{pink}{Prag}{}\ledrightnote{\textcolor{pink}{Prag}}} unterhalte ich eine Correſpondenz. Die \textcolor{blue}{Mutter}{}\ledrightnote{→\textcolor{blue}{Charlotte Bondy}} ſcheint eine blöde Gans zu ſein, das \textcolor{blue}{Mädchen}{}\ledrightnote{→\textcolor{blue}{Alice Ziegler}} aber iſt wohl ein liebes Kind. Ich
               kann mir kaum \strikeout{\textcolor{gray}{de}} denken, daß alle Träume, welche ich ſeit dieſer kurzen \textsc{\textcolor{pink}{Ischl}{}\ledrightnote{\textcolor{pink}{Bad Ischl}}er} Bekanntſchaft in mir herumtrage,
               jemals {\pb}zu Wirklichkeiten werden ſollten. Aber es
               iſt mir eine Wohlthat, hier in der Heimatloſigkeit, in dieſer Hölle von Anſtrengungen
               und Aufregungen, an ein liebes \textcolor{blue}{Mädchen}{}\ledrightnote{→\textcolor{blue}{Alice Ziegler}}-Geſicht denken zu können, wie an eine Hoffnung. Darum bitte ich Dich
               recht ſehr: \label{K_L02831-4v}\edtext{Geh’ zu den \textcolor{blue}{Leuten}{}\ledrightnote{→\textcolor{blue}{Charlotte Bondy}{\newline}→\textcolor{blue}{Vít Šalomoun Bondy}{\newline}→\textcolor{blue}{Alice Ziegler}}
                  hin}{\lemma{\textnormal{\emph{Geh’ zu den Leuten
                  hin}}}\Cendnote{\textnormal{\textcolor{blue}{Schnitzler} traf \textcolor{blue}{Charlotte und Vít Šalomoun Bondy} am 24. 11. 1897, 25. 11. 1897, 27. 11. 1897 und 28. 11. 1897.}}}\label{K_L02831-4h}
                  (\textsc{\textcolor{pink}{Mariengaſse 45}{}\ledrightnote{\textcolor{pink}{Mariannengasse}}}), ſchau Dir an, wer ſie ſind, höre auch, was die Anderen über ſie ſagen, und,
               wenn Du es für gut findeſt, ſprich ein freundliches Wort über mich. Jedenfalls {\pb}aber ſende mir einen recht ausführlichen Bericht!
               Ja? Das iſt ein wahrer Freundſchaftsdienſt, den ich verlange.\pend
           \pstart
           Ich wünſche Dir von Herzen Glück zu Deiner \label{K_L02831-7v}\edtext{\textcolor{green}{Vorleſung}{}\ledrightnote{→\textcolor{green}{Die Toten schweigen}{\newline}→\textcolor{green}{Weihnachts-Einkäufe}} und Deiner
                  \textsc{\textcolor{green}{Premiere}{}\ledrightnote{→\textcolor{green}{Freiwild. Schauspiel in 3 Akten}}} in \textsc{\textcolor{pink}{Prag}{}\ledrightnote{\textcolor{pink}{Prag}}}}{\lemma{\textnormal{\emph{Vorleſung … Prag}}}\Cendnote{\textnormal{siehe Paul Goldmann an Arthur Schnitzler, 15. 10. [1897]}}}\label{K_L02831-7h} und grüße Dich Tauſend Mal in Treue\pend
           \pstart
           Dein {\\[\baselineskip]}\spacefill\mbox{Paul Goldm}\pend
           \leftskip=0em{}\pstart
           \noindent{}Ich ſchreibe in höchſter Eile und kann der nur mit einem {\pb}Wort ſagen, wie ſehr mich die Nachricht vom
                     \label{K_L02831-11v}\edtext{Tode der armen \textcolor{blue}{Frau}{}\ledrightnote{→\textcolor{blue}{Olga Waissnix}}}{\lemma{\textnormal{\emph{Tode der armen Frau}}}\Cendnote{\textnormal{\textcolor{blue}{Olga Waissnix} verstarb am 4. 11. 1897 in \textcolor{pink}{Wien}. \textcolor{blue}{Schnitzler} erfuhr davon am
                        6. 11. 1897.}}}\label{K_L02831-11h} ergriffen hat. Wieder ein Stück Jugend unwiederbringlich verloren!
                  Wie ſich um uns \strikeout{her} herum die Vergangenheit
                  auszudehnen beginnt, das Geweſene, – das nie mehr wieder ſein wird, – das bereits
                  verbrauchte Leben! Und dieſe \textcolor{blue}{Ärmſte}{}\ledrightnote{→\textcolor{blue}{Olga Waissnix}}, die fort mußte, ehe ſie ſich ausleben gekonnt, die wahrſcheinlich
                  erwartete, daß das Eigentliche noch kommen würde! Wie man ſich alſo darauf
                  vorbereiten muß, daß das Ende eines {\pb}ſchönen
                  Tages kommen kann, ohne daß man Zeit gehabt hat, auch nur mit irgend etwas fertig
                  zu werden! Und dann, ohne lange Worte: die arme, liebe, ſchöne \textcolor{blue}{Frau}{}\ledrightnote{→\textcolor{blue}{Olga Waissnix}}!!\pend
           \endnumbering\briefempfaengerindex{Schnitzler, Arthur@\textsc{Schnitzler, Arthur}!zzzGoldmann, Paul@\emph{von Paul Goldmann}!1897-11-192@{19. 11. {[}1897{]}}|)be}\mylabel{h}\begin{anhang}\end{anhang}\normalsize

\doendnotes{C}
\bigskip
\vfill

\clearpage

\footnotesize

\lohead{\textsc{register}}

% Definiere theindex-Environment komplett neu ohne reledmac
\makeatletter
\renewenvironment{theindex}{%
  \section*{\indexname}%
  \setlength{\parindent}{0pt}%
  \setlength{\parskip}{0pt plus 0.3pt}%
  \let\item\@idxitem
}{%
  \clearpage
}
\makeatother

\IfFileExists{\jobname-pw.ind}{\input{\jobname-pw.ind}}{}

\end{document}

      