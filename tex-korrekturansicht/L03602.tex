%% latex-korrekturansicht-vorspann.tex
%% Vorspann für die Korrekturansicht.
%% Lädt die gemeinsame Datei latex-vorspann.tex mit gesetztem Schalter.

\newif\ifkorrekturansicht
\korrekturansichttrue

\input{../tex-inputs/latex-vorspann}


\renewcommand{\erwaehntePersonen}{Personen: Felix Salten}
\renewcommand{\erwaehnteInstitutionen}{Institutionen: S. Fischer Verlag}
\renewcommand{\erwaehnteOrte}{Orte: Berlin, Wien}
\renewcommand{\erwaehnteWerke}{Werke: Börsenblatt für den Deutschen Buchhandel, Der grüne Kakadu – Paracelsus – Die Gefährtin. Drei Einakter}
\section[Arthur Schnitzler: Widmungsexemplar Der grüne Kakadu für Felix Salten, {[}zwischen 1. und 22.{]} 6. 1899]{Arthur Schnitzler: Widmungsexemplar Der grüne Kakadu für Felix Salten,
               {[}zwischen 1. und 22.{]} 6. 1899}
\nopagebreak\mylabel{v}
\rehead{ }\normalsize\beginnumbering\briefempfaengerindex{Salten, Felix@\textsc{Salten, Felix}!zzzSchnitzler, Arthur@\emph{von Arthur Schnitzler}!2@{{[}zwischen 1. und 22.{]} 6. 1899}|(be}
\toendnotes[C]{\smallbreak\pagebreak[2]}\Standort{Wienbibliothek im Rathaus, A-355798, DS-2019-31.}
\physDesc{Widmung am Vorsatzblatt, 55 Zeichen
\newline{}Handschrift: schwarze Tinte, deutsche Kurrent
\newline{}Ordnung: 1) mit schwarzer Tinte am Titelblatt gestrichene Regalerfassung: »\noindent{}IN\textsuperscript{o} 2462 WN\textsuperscript{o} 1531{ / }XI b«  2) mit schwarzer Tinte ausgefüllter Stempel: »\noindent{}\textcolor{gray}{\textbf{\textit{Felix Salten}}}{ / }\textcolor{gray}{\textbf{\textit{Inv. Nr.}}}{ }4469{ / }\textcolor{gray}{\textbf{\textit{Werk Nr.}}}{ }2197{ / }\textcolor{gray}{\textbf{\textit{Schrank}}}{ }XIV A. Z. \textcolor{gray}{\textbf{\textit{Fach}}} b«}\toendnotes[C]{\smallbreak}
\pstart
           \noindent{}{\pb}Meinem lieben Felix Salten\pend
           
\pstart
           herzlichſt{\\[\baselineskip]}\spacefill\mbox{Arth Sch}\pend
           \leftskip=0em{}
\pstart
           Juni 99.\pend
           {\bigskip}
\pstart
           \noindent{}\centering{}{\pb}\textcolor{gray}{\textbf{\textcolor{green}{\so{Der grüne Kakadu}}{}\ledrightnote{\textcolor{green}{Der grüne Kakadu – Paracelsus – Die Gefährtin. Drei Einakter}}}}\pend
           
\pstart
           \noindent{}\centering{}\textcolor{gray}{\textbf{\textcolor{green}{\textbf{Paracelsus – Die Gefährtin}}{}\ledrightnote{\textcolor{green}{Der grüne Kakadu – Paracelsus – Die Gefährtin. Drei Einakter}}}}\pend
           {\bigskip}
\pstart
           \noindent{}\centering{}\textcolor{gray}{\textbf{\so{Drei Einakter}}}\pend
           
\pstart
           \noindent{}\centering{}\textcolor{gray}{\textbf{von}}\pend
           
\pstart
           \noindent{}\centering{}\textcolor{gray}{\textbf{\textbf{Arthur Schnitzler}}}\pend
           
\pstart
           \noindent{}\centering{}\textcolor{gray}{\textbf{Zweite Auflage}}\pend
           {\bigskip}
\pstart
           \noindent{}\centering{}\textcolor{gray}{\textbf{\textcolor{pink}{\so{Berlin}}{}\ledrightnote{\textcolor{pink}{Berlin}}}}\pend
           
\pstart
           \noindent{}\centering{}\textcolor{gray}{\textbf{\textcolor{brown}{S. Fiſcher, Verlag}{}\ledrightnote{\textcolor{brown}{S. Fischer Verlag}}}}\pend
           
\pstart
           \noindent{}\centering{}\textcolor{gray}{\textbf{\label{K_L03602-1v}\edtext{1899}{\lemma{\textnormal{\emph{1899}}}\Cendnote{\textnormal{Der \textcolor{green}{Einakterzyklus} war am 29. 4. 1899
                        vom \emph{\textcolor{green}{Börsenblatt für den deutschen
                           Buchhandel}} als Neuerscheinung gemeldet worden.}}}\label{K_L03602-1h}}}\pend
           \endnumbering\briefempfaengerindex{Salten, Felix@\textsc{Salten, Felix}!zzzSchnitzler, Arthur@\emph{von Arthur Schnitzler}!1899-06-012@{{[}zwischen 1. und 22.{]} 6. 1899}|)be}\mylabel{h}  \normalsize

\doendnotes{C}
\bigskip
\vfill

\clearpage

\footnotesize

\lohead{\textsc{register}}

% Definiere theindex-Environment komplett neu ohne reledmac
\makeatletter
\renewenvironment{theindex}{%
  \section*{\indexname}%
  \setlength{\parindent}{0pt}%
  \setlength{\parskip}{0pt plus 0.3pt}%
  \let\item\@idxitem
}{%
  \clearpage
}
\makeatother

\IfFileExists{\jobname-pw.ind}{\input{\jobname-pw.ind}}{}

\end{document}

      