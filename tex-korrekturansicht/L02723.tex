%% latex-korrekturansicht-vorspann.tex
%% Vorspann für die Korrekturansicht.
%% Lädt die gemeinsame Datei latex-vorspann.tex mit gesetztem Schalter.

\newif\ifkorrekturansicht
\korrekturansichttrue

\input{../tex-inputs/latex-vorspann}


               \section[Paul Goldmann an Arthur Schnitzler, 8. 12. {[}1893{]}]{ Paul Goldmann an Arthur Schnitzler, 8. 12. {[}1893{]}}\nopagebreak\mylabel{v}\rehead{ }\normalsize\beginnumbering\briefempfaengerindex{Schnitzler, Arthur@\textsc{Schnitzler, Arthur}!zzzGoldmann, Paul@\emph{von Paul Goldmann}!1893-12-081@{8. 12. {[}1893{]}}|(be} \toendnotes[C]{\smallbreak\pagebreak[2]} \Standort{DLA, A:Schnitzler, HS.NZ85.1.3163.}
\physDesc{Brief, 1 Blatt, 4 Seiten
\newline{}Handschrift: schwarze Tinte, deutsche Kurrent
\newline{}Schnitzler: 1) mit Bleistift das Jahr »93« vermerkt 2) mit rotem Buntstift drei Unterstreichungen sowie ein Pfeil, der
                                 den ganzen Absatz zu \textcolor{blue}{Hofmannsthal} markieren soll}\toendnotes[C]{\smallbreak}\pstart
           \noindent{}{\pb}\textcolor{gray}{\textbf{\textbf{\textcolor{brown}{Frankfurter Zeitung}{}\ledrightnote{\textcolor{brown}{Frankfurter Zeitung}}.}}}\pend
           \pstart
           \textcolor{gray}{\textbf{\textbf{(\textcolor{brown}{\begin{otherlanguage}{french}Gazette de Francfort\end{otherlanguage}}{}\ledrightnote{\textcolor{brown}{Frankfurter Zeitung}}.)}}}\pend
           \pstart
           \textcolor{gray}{\textbf{\begin{otherlanguage}{french}\textcolor{blue}{Directeur}{}\ledrightnote{→\textcolor{blue}{Leopold Sonnemann}}\end{otherlanguage}{ }\textbf{M. \textcolor{blue}{L. Sonnemann}{}\ledrightnote{\textcolor{blue}{Leopold Sonnemann}}.}}}\hfill \textsc{\textcolor{pink}{Paris}{}\ledrightnote{\textcolor{pink}{Paris}}}, 8. December.\pend
           \pstart
           \begin{otherlanguage}{french}\textcolor{gray}{\textbf{\textcolor{green}{Journal}{}\ledrightnote{\textcolor{green}{Frankfurter Zeitung}} politique, financier,}}\end{otherlanguage}\pend
           \pstart
           \begin{otherlanguage}{french}\textcolor{gray}{\textbf{commercial et litteraire.}}\end{otherlanguage}\pend
           \pstart
           \begin{otherlanguage}{french}\textcolor{gray}{\textbf{\textbf{Paraissant trois fois par jour}}}\end{otherlanguage}\pend
           \pstart
           \begin{otherlanguage}{french}\textcolor{gray}{\textbf{\textbf{Bureaux à \textcolor{pink}{Paris}{}\ledrightnote{\textcolor{pink}{Paris}}:}}}\end{otherlanguage}\pend
           \pstart
           \begin{otherlanguage}{french}\textcolor{gray}{\textbf{\textbf{\textcolor{pink}{rue Richelieu 75}{}\ledrightnote{\textcolor{pink}{rue Richelieu}}.}}}\end{otherlanguage}\pend
           \pstart\center{}Mein lieber Freund!\pend\pstart
           Dank für die \label{K_L02723-1v}\edtext{Kritiken}{\lemma{\textnormal{\emph{Kritiken}}}\Cendnote{\textnormal{zu den ihm bekannten Kritiken vgl. Paul Goldmann an Arthur Schnitzler, 5. 12. [1893]}}}\label{K_L02723-1h}; ich kannte ſie größtentheils ſchon. Drei oder vier verſtehen Dich oder geben
               ſich wenigſtens ehrliche Mühe, Dich zu verſtehen. Der \label{K_L02723-23v}\edtext{kleine \textcolor{blue}{\textsc{\textcolor{brown}{Salonblatt}{}\ledrightnote{\textcolor{brown}{Wiener Salonblatt}}}-Mann}{}\ledrightnote{→\textcolor{blue}{Alfred Maria Willner}}, der \strikeout{Dich} Dir zum Luſtſpiel}{\lemma{\textnormal{\emph{kleine … Luſtſpiel}}}\Cendnote{\textnormal{\textcolor{blue}{A. M. W.} [=\textcolor{blue}{Alfred Maria Willner}]: \emph{\textcolor{green}{Notizen eines Theater-Habitués. (Raimund-Theater. – Das
                        Märchen.)}}. In: \emph{\textcolor{green}{Wiener Salonblatt}},
                     Jg. 24, Nr. 49, 3. 12. 1893, S. 8–9. Das
                  Thema »Lustspiel« blieb für \textcolor{blue}{Schnitzler}
                  zeitlebens eine Herausforderung, die er immer wieder erwog, an woran er aber auch
                  scheiterte.}}}\label{K_L02723-23h} räth, iſt auch auf der richtigen Fährte. Du brauchteſt
               unbedingt ein paar Monate \textcolor{pink}{Pariſ}{}\ledrightnote{\textcolor{pink}{Paris}}er Theater; Du
               würdeſt die unermüdliche Anſtrengung des \label{K_L02723-2v}\edtext{jungen Stücks}{\lemma{\textnormal{\emph{jungen Stücks}}}\Cendnote{\textnormal{Siehe dazu etwa Sally Debra Charnow: \emph{Theatre, Politics,
                        and Markets in Fin-de-Siècle Paris. Staging Modernity}.
                     Basingstoke: \emph{Palgrave Macmillan}{ }2005.}}}\label{K_L02723-2h} ſehen, objectiv, kurz, natürlich, luſtig zu werden. Das iſt der
               Weg, {\pb}der geradeaus in die Zukunft geht. Das iſt
               auch der Weg Deines Talents. Ein Luſtſpiel, theuerſter Freund, – oder ein Schauſpiel,
               aber ohne Herzensergüſſe! Könnteſt Du Dich nur mit meinen Augen ſehen – Du würdeſt
               keinen Augenblick mehr zögern, und in einem Jahre wäre die Vollendung da, in
               Production wie Erfolg. Bitte ſchreib’ mir ein Wort über Deine Pläne.\pend
           \pstart
           \textcolor{blue}{\textsc{Bahr}}{}\ledrightnote{\textcolor{blue}{Hermann Bahr}} – der kränkt Dich ſo? Er iſt frech, größenwahnſinnig, unausſtehlich doctrinär.
                  \strikeout{E} Der \label{K_L02723-45v}\edtext{\textcolor{green}{Verweis}{}\ledrightnote{→\textcolor{green}{Das Märchen (Schauspiel in drei Aufzügen von Arthur Schnitzler)}} auf ſeine »\textcolor{green}{Neuen Menſchen}{}\ledrightnote{\textcolor{green}{Die neuen Menschen. Ein Schauspiel}}« iſt eine glatte Gemeinheit}{\lemma{\textnormal{\emph{Verweis … Gemeinheit}}}\Cendnote{\textnormal{\textcolor{blue}{Hermann Bahr}: \emph{\textcolor{green}{Die neuen Menschen. Ein Schauspiel}} .
                     Zürich: \emph{\textcolor{brown}{Verlags-Magazin (J.
                        Schabelitz)}}{ }1887. In seiner Rezension kommt \textcolor{blue}{Bahr} auf
                  die vielen Stücke zu sprechen, die im \emph{\textcolor{green}{Märchen}}
                  anklingen, darunter sein eigenes: »das Stück jenes Zwistes von Verstand und
                     Gefühl, das auch ich einmal, im Sturme der ersten Jugend, mit meinen › \textcolor{green}{neuen Menschen}‹ versuchte.« (\textcolor{blue}{Hermann Bahr}: \emph{\textcolor{green}{Das Märchen (Schauspiel in drei Aufzügen von Arthur
                        Schnitzler. Zum ersten Male aufgeführt am Deutschen Volkstheater den 1.
                        December)}}. In: \emph{\textcolor{green}{Deutsche Zeitung}},
                     Jg. 23, Nr. 7.879, 2. 12. 1893, Morgen-Ausgabe,
                     S. 1–3, hier S. 2.)}}}\label{K_L02723-45h}. Und doch finde ich ihn nicht reſpectlos; und
               doch finde ich, daß {\pb}er manches Richtige ſagt.
               Vielleicht aber fehlt mir auch das richtige Urtheil; ich bin ſo außer Zuſammenhang
               mit den \textcolor{pink}{Wien}{}\ledrightnote{\textcolor{pink}{Wien}}er Verhältniſſen. Heiter iſt nur, wie
               der \textcolor{blue}{Burſch}{}\ledrightnote{→\textcolor{blue}{Hermann Bahr}} franzöſiſche
               Dinge citirt – \label{K_L02723-3v}\edtext{»\textcolor{green}{\begin{otherlanguage}{french}Le grappin\end{otherlanguage}}{}\ledrightnote{\textcolor{green}{Le Grappin. Comédie en trois actes}}«}{\lemma{\textnormal{\emph{»Le grappin«}}}\Cendnote{\textnormal{ Der entsprechende Absatz in \textcolor{blue}{Bahr}s \textcolor{green}{Kritik} lautet: »Er konnte die Eifersucht der
                     Vergangenheit am \textcolor{green}{Werke}
                     zeigen; wie etwa \textcolor{green}{Othello} die Eifersucht in
                     der Gegenwart zeigt: er nahm dann eine Liebe und ließ sie an der Vergangenheit
                     des Mädchens verderben, die allmälig sei es gestanden, sei es verrathen wird;
                     der Schmerz des Mannes zwischen Leidenschaft und Ehre und die Buße der
                     Gefallenen waren da die Kräfte, die die Handlung trieben. Oder er konnte einen
                     Spötter gegen diese Eifersucht zeigen, der sich über sie heben will, aber
                     leidend von ihrem Rechte gezwungen wird; er schrieb dann das \textcolor{green}{Stück}, das \textcolor{blue}{Gaston Salandri} als ›\textcolor{green}{Le
                        Grappin}‹ geschrieben und die \textcolor{pink}{Paris}er \textcolor{brown}{Freie Bühne}
                     gespielt hat, die Geschichte des Herrn \textcolor{green}{Jacques Privat}, der das Vorurtheil verachtet und sich
                     mit seiner Geliebten vermählt, obwohl er weiß, daß sie vor ihm Anderen gehörte
                     und liederlich lebte; da wird gezeigt, daß alle Liebe die Vergangenheit nicht
                     tilgen, nicht verwischen kann, ja, durch die tausend Stiche der Nerven, des
                     Gemüthes und die Kränkungen der Ehre sich in Zorn, Ekel, Haß verwandeln muß.
                     Mit dem ersten \textcolor{green}{Stücke}
                     geht der Hörer, auch wenn er diese Eifersucht nicht hat, weil er sich doch aus
                     Anderen in sie denken kann. Mit dem \textcolor{green}{zweiten} kann er gegen das Vorurtheil, das ja von dem
                     Helden bestritten, und er kann für das Vorurtheil mit ihm gehen, das doch
                     schließlich bestätigt wird. Er ist \textcolor{green}{Beiden} empfänglich.«
                     (S. 1)}}}\label{K_L02723-3h}, das \textcolor{green}{\textcolor{brown}{Theâtre-Libre}{}\ledrightnote{\textcolor{brown}{Théâtre Libre}}-Stück}{}\ledrightnote{→\textcolor{green}{Le Grappin. Comédie en trois actes}}, von dem er ſpricht,
               behandelt etwas abſolut Anderes als das, was er behauptet. Ein frecher Schwindel, um
               ſich in allen Sätteln moderner \introOben{}franzöſiſcher\introOben{} Literatur
               gerecht zu zeigen.\pend
           \pstart
           \label{K_L02723-12v}\edtext{\textsc{\textcolor{blue}{Granichstaedten}{}\ledrightnote{\textcolor{blue}{Emil Granichstaedten}}} hätte ich an Deiner Stelle geohrfeigt. Das iſt keine \textcolor{green}{Kritik}{}\ledrightnote{→\textcolor{green}{Theater- und Kunstnachrichten [Uraufführung Das Märchen]}{\newline}→\textcolor{green}{Feuilleton. Deutsches Volkstheater [Märchen]}}}{\lemma{\textnormal{\emph{Granichstaedten … Kritik}}}\Cendnote{\textnormal{\textcolor{blue}{Emil Granichstaedten} verfasste eine
                  Nachtkritik (\textcolor{blue}{g.}: \emph{\textcolor{green}{Theater- und Kunstnachrichten}}. In: \emph{\textcolor{green}{Die Presse}}, Jg. 46, Nr. 333, 2. 12. 1893, S. 11) und am Folgetag ein Feuilleton (\textcolor{blue}{Emil Granichstaedten}: \emph{\textcolor{green}{Feuilleton. Deutsches Volkstheater}}. In: \emph{\textcolor{green}{Die Presse}}, Jg. 46, Nr. 334, 3. 12. 1893, S. 1–2). Auch \textcolor{blue}{Schnitzler} war über die \textcolor{green}{Nachtkritik} verärgert und bezeichnete sie im \emph{\textcolor{green}{Tagebuch}} als »[p]erfid dumm« (2. 12. 1893). \textcolor{blue}{Granichstaedten} lobte die Schauspielkunst \textcolor{blue}{Adele Sandrock}s, spielte aber auf sexuelle
                  Aspekte im \emph{\textcolor{green}{Märchen}} recht abschätzig an.
                  Zwischen den Zeilen kritisierte er die \textcolor{green}{Handlung} an sich und die Figuren des \textcolor{green}{Fedor} und der \textcolor{green}{Fanny}. Am 3. 12. 1893 positionierte \textcolor{blue}{Granichstaedten} sich auf der Seite des Naturalismus und holte weiter aus.
                  Angefangen beim »Pessimismus unserer ›\textcolor{pink}{Wien}er Modernen‹« (S. 1) kritisierte er auf
                  abwertende Weise ganz grundsätzlich das junge Werk \textcolor{blue}{Schnitzler}s und bezog sich auch auf den \emph{\textcolor{green}{Anatol-Zyklus}}. Der \textcolor{blue}{Autor} orientiere sich zu stark an »modernen«, französischen
                  Strömungen, was ihm jedoch nicht gelinge: »Für dieſen \textcolor{green}{Fedor} und dieſe \textcolor{green}{Fanny} kann kein Publikum der Welt
                     sich interessieren.« (S. 2). \emph{\textcolor{green}{Das Märchen}} sei »nicht tugendhaft« und
                     »[u]m Reinlichkeit wird gebeten«. (S. 2)}}}\label{K_L02723-12h},
               ſondern ein Gaſſenbubenſtreich.\pend
           \pstart
           Freut mich, daß Du nicht {\pb}verbittert biſt. Das
               gehört ſich auch ſo. Ich meine, Du kannſt mit Deinem \textcolor{green}{Debüt}{}\ledrightnote{→\textcolor{green}{Das Märchen. Schauspiel in drei Aufzügen}} ſehr zufrieden ſein. Man gibt Dir
               Credit, und das iſt enorm für einen Jungen.\pend
           \pstart
           Haſt Du \label{K_L02723-4v}\edtext{\textcolor{green}{\textsc{\textcolor{blue}{Loris}{}\ledrightnote{\textcolor{blue}{Hugo von Hofmannsthal}}} über \textsc{\textcolor{blue}{Bauernfeld}{}\ledrightnote{\textcolor{blue}{Eduard von Bauernfeld}}}}{}\ledrightnote{→\textcolor{green}{Eduard von Bauernfeld’s dramatischer Nachlaß}}}{\lemma{\textnormal{\emph{Loris über Bauernfeld}}}\Cendnote{\textnormal{\textcolor{blue}{Loris}: \emph{\textcolor{green}{Eduard von Bauernfeld’s dramatischer Nachlaß}}. In: \emph{\textcolor{green}{Frankfurter Zeitung}}, Jg. 38, Nr. 338, 6. 12. 1893, erstes Morgenblatt,
                  S. 1.}}}\label{K_L02723-4h} geleſen? Wie aus dieſem gottbegnadeten Menſchen die
               entzückenden Dinge herausquellen, ſo leicht und ſprudelnd. Ein Dichter! Derjenige
               vielleicht, den man ſeit fünfzig Jahren erwartet!\pend
           \pstart
           Grüß’ ihn von mir, denn ich habe \label{K_L02723-67v}\edtext{keine directe Verbindung mehr}{\lemma{\textnormal{\emph{keine … mehr}}}\Cendnote{\textnormal{Im
                  Nachlass \textcolor{blue}{Hofmannsthal}s sind keine
                  Korrespondenzstücke \textcolor{blue}{Goldmann}s überliefert.
                  In den Briefen \textcolor{blue}{Beer-Hofmann}s in der
                        \emph{Houghton Library} dürften keine
                     Korrespondenzstücke aus dem Zeitraum Sommer 1893 –
                        1895 erhalten sein, wobei viele Briefe ohne Jahresangabe sind
                     und eine genauere Zuordnung notwendig wäre, um die Behauptung mit letzter
                     Sicherheit treffen zu können.}}}\label{K_L02723-67h} mit ihm; Grüße auch \textsc{\textcolor{blue}{Richard}{}\ledrightnote{\textcolor{blue}{Richard Beer-Hofmann}}} aus ſelbigem Grunde; ſei ſelbſt herzlichſt gegrüßt und ſchreibe bald!\pend
           \pstart
           Dein{\\[\baselineskip]}\spacefill\mbox{Paul Goldm}\pend
           \leftskip=0em{}\endnumbering\briefempfaengerindex{Schnitzler, Arthur@\textsc{Schnitzler, Arthur}!zzzGoldmann, Paul@\emph{von Paul Goldmann}!1893-12-081@{8. 12. {[}1893{]}}|)be}\mylabel{h}  \normalsize

\doendnotes{C}
\bigskip
\vfill

\clearpage

\footnotesize

\lohead{\textsc{register}}

% Definiere theindex-Environment komplett neu ohne reledmac
\makeatletter
\renewenvironment{theindex}{%
  \section*{\indexname}%
  \setlength{\parindent}{0pt}%
  \setlength{\parskip}{0pt plus 0.3pt}%
  \let\item\@idxitem
}{%
  \clearpage
}
\makeatother

\IfFileExists{\jobname-pw.ind}{\input{\jobname-pw.ind}}{}

\end{document}

      