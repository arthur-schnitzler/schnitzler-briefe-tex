%% latex-korrekturansicht-vorspann.tex
%% Vorspann für die Korrekturansicht.
%% Lädt die gemeinsame Datei latex-vorspann.tex mit gesetztem Schalter.

\newif\ifkorrekturansicht
\korrekturansichttrue

\input{../tex-inputs/latex-vorspann}


               \section[Arthur Schnitzler an Hermann Bahr, 11. 11. 1897]{ Arthur Schnitzler an Hermann Bahr, 11. 11. 1897}\nopagebreak\mylabel{v}\rehead{ }\normalsize\beginnumbering\briefempfaengerindex{Bahr, Hermann@\textsc{Bahr, Hermann}!zzzSchnitzler, Arthur@\emph{von Arthur Schnitzler}!1897-11-111@{11. 11. 1897}|(be} \toendnotes[C]{\smallbreak\pagebreak[2]} \Standort{TMW, HS AM 60135 Ba.}
\physDesc{Briefkarte
\newline{}Handschrift: Bleistift, deutsche Kurrent\newline{}Ordnung: Lochung }\buchAbdrucke{\weitereDrucke{1) \emph{11. 11. 1897, Abschrift.} In: Arthur Schnitzler: \emph{The Letters of Arthur Schnitzler to Hermann Bahr}. Edited, annotated, and with an introduction, by Donald G.
                        Daviau. Chapel Hill: \emph{The University of North Carolina Press} 1978, S. 62 (University of North Carolina studies in the Germanic languages
                        and literatures, 89).} \weitereDrucke{2) Hermann Bahr, Arthur Schnitzler: \emph{Briefwechsel, Aufzeichnungen, Dokumente (1891–1931)}. Hg. Kurt Ifkovits und Martin Anton Müller. Göttingen: \emph{Wallstein} 2018, S. 155.} }\pstart
           \noindent{}{\pb}lieber Hermann, we{\geminationn} du also »\textcolor{green}{Die Todten ſchweigen}{}\ledrightnote{\textcolor{green}{Die Toten schweigen}}« leſen wi\damage{ll}ſt, würds mich freuen. \uline{Nur bitte ich dich ſehr,
                  nichts zu ſtreichen.} Mir fällt das eben ein, wie ich die Geſchichte ſelbſt
               wieder durchleſe und z. B. die Schilderung der \textcolor{pink}{Reichsbrücke}{}\ledrightnote{\textcolor{pink}{Reichsbrücke}}{ }ſehe, die ja gewiſs zu\substVorne{}\textsuperscript{\textcolor{gray}{r}}\substDazwischen{}m\substHinten{} »Verſtändnis« des ganzen \introOben{}nicht\introOben{} nothwendig iſt, aber
               für die Sti{\geminationm}ung {\pb}ſo
               unerläſſlich, – wie ſchließlich alles, was der Au\damage{t}or zu rechter Zeit erwähnt. Hiemit will ich alſo deine eventuellen
               Kürzungsideen im Mutterleib erwürgen.\pend
           \pstart
           Herzlich grüßend Dein{\\[\baselineskip]}\spacefill\mbox{Arthur}\pend
           \leftskip=0em{}\pstart
           11. 11. 97\pend
           \endnumbering\briefempfaengerindex{Bahr, Hermann@\textsc{Bahr, Hermann}!zzzSchnitzler, Arthur@\emph{von Arthur Schnitzler}!1897-11-111@{11. 11. 1897}|)be}\mylabel{h}  \normalsize

\doendnotes{C}
\bigskip
\vfill

\clearpage

\footnotesize

\lohead{\textsc{register}}

% Definiere theindex-Environment komplett neu ohne reledmac
\makeatletter
\renewenvironment{theindex}{%
  \section*{\indexname}%
  \setlength{\parindent}{0pt}%
  \setlength{\parskip}{0pt plus 0.3pt}%
  \let\item\@idxitem
}{%
  \clearpage
}
\makeatother

\IfFileExists{\jobname-pw.ind}{\input{\jobname-pw.ind}}{}

\end{document}

      