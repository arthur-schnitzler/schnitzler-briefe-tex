%% latex-korrekturansicht-vorspann.tex
%% Vorspann für die Korrekturansicht.
%% Lädt die gemeinsame Datei latex-vorspann.tex mit gesetztem Schalter.

\newif\ifkorrekturansicht
\korrekturansichttrue

\input{../tex-inputs/latex-vorspann}


\renewcommand{\erwaehntePersonen}{Personen: Richard Beer-Hofmann, Paula Beer-Hofmann, Olga Schnitzler}
\renewcommand{\erwaehnteOrte}{Orte: Badehotel, Berlin, Dänemark, Helsingør, Marienlyst, Skodsborg}
\renewcommand{\erwaehnteWerke}{}
\section[ Paul Goldmann an Arthur Schnitzler, 3. 8. 1906]{Paul Goldmann an Arthur Schnitzler, 3. 8. 1906}
\nopagebreak\mylabel{v}
\rehead{ }\normalsize\beginnumbering\briefempfaengerindex{Schnitzler, Arthur@\textsc{Schnitzler, Arthur}!zzzGoldmann, Paul@\emph{von Paul Goldmann}!1906-08-033@{3. 8. 1906}|(be}
\toendnotes[C]{\smallbreak\pagebreak[2]}\Standort{DLA, A:Schnitzler, HS.NZ85.1.3175.}
\physDesc{Postkarte
\newline{}Handschrift: 1) blaue Tinte, deutsche Kurrent\hspace{1em}2) blaue Tinte, lateinische Kurrent (\noindent{}Adresse)\hspace{1em}
\newline{}Versand: 1) Stempel: »\nobreak{}\oindex{Berlin@\textbf{Berlin}, \emph{https://www.geonames.org/ontologyP.PPLC}|pwk}Berlin, S.W. 11, 3. 8. 06, 6–7N.\nobreak{}«.   2) Stempel: »\nobreak{}\oindex{Helsingør@\textbf{Helsingør}, \emph{Besiedelter Ort (A.BSO)}|pwk}Helsingør, 4. 8. 06, 11–12F\nobreak{}«. 
\newline{}Schnitzler: mit Bleistift das Jahr »{[}1{]}906« vermerkt }\toendnotes[C]{\smallbreak}\pstart{}{\pb}Welt-\textcolor{gray}{\textbf{Poſtkarte}}\pend{}\pstart{}Herrn\pend{}\pstart{}Dr. Arthur Schnitzler\pend{}\pstart{}\textcolor{pink}{Marienlyst}{}\ledrightnote{\textcolor{pink}{Marienlyst}} (\textcolor{pink}{Dänemark}{}\ledrightnote{\textcolor{pink}{Dänemark}})\pend{}
{\bigskip}
\pstart
           \noindent{}{\pb}3. Auguſt. Mein lieber
                  Freund, Auch mich hat der Anblick des \textsc{\textcolor{pink}{Skodsborger Hotels}{}\ledrightnote{{$\rightarrow$}\textcolor{pink}{Badehotel}}} auf Deiner lieben
                  \label{K-L03249-1v}\edtext{Karte}{\lemma{\textnormal{\emph{Karte}}}\Cendnote{\textnormal{Am 
                  1. 8. 1906
                     machte \textcolor{blue}{Schnitzler} einen Ausflug
                     nach \textcolor{pink}{Skodsborg}, wo er sich 
                     im
                     August 1896 gemeinsam mit \textcolor{blue}{Goldmann}, 
                     \textcolor{blue}{Richard Beer-Hofmann} und dessen Partnerin
                     \textcolor{blue}{Paula Lissy} im \textcolor{pink}{Badehotel} aufgehalten hatte. Die nicht erhaltene Karte an \textcolor{blue}{Goldmann}
                     könnte das gleiche Motiv gehabt haben wie Arthur Schnitzler an Richard Beer-Hofmann, 1. 8. 1906.
               }}}\label{K-L03249-1h} recht wehmütig geſtimmt. Ja, ja, die zehn Jahre ſind fort, unwiederbringlich! Was aber das Talent anlangt, –
               glaubſt Du nicht, daß Du noch ein wenig übrig haſt? Oder ſollte ich Dich
               überſchätzen?\pend
           
\pstart
           Herzliche Grüße Dir u. Deiner \textcolor{blue}{Frau}{}\ledrightnote{{$\rightarrow$}\textcolor{blue}{Olga Schnitzler}} u. herzlichen Dank für Dein freundliches Gedenken! {\\[\baselineskip]}Dein getreuer {\\[\baselineskip]}\spacefill\mbox{Paul Goldmann.}\pend
           \leftskip=0em{}\endnumbering\briefempfaengerindex{Schnitzler, Arthur@\textsc{Schnitzler, Arthur}!zzzGoldmann, Paul@\emph{von Paul Goldmann}!1906-08-033@{3. 8. 1906}|)be}\mylabel{h}  \normalsize

\doendnotes{C}
\bigskip
\vfill

\clearpage

\footnotesize

\lohead{\textsc{register}}

% Definiere theindex-Environment komplett neu ohne reledmac
\makeatletter
\renewenvironment{theindex}{%
  \section*{\indexname}%
  \setlength{\parindent}{0pt}%
  \setlength{\parskip}{0pt plus 0.3pt}%
  \let\item\@idxitem
}{%
  \clearpage
}
\makeatother

\IfFileExists{\jobname-pw.ind}{\input{\jobname-pw.ind}}{}

\end{document}

      