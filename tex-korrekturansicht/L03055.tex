%% latex-korrekturansicht-vorspann.tex
%% Vorspann für die Korrekturansicht.
%% Lädt die gemeinsame Datei latex-vorspann.tex mit gesetztem Schalter.

\newif\ifkorrekturansicht
\korrekturansichttrue

\input{../tex-inputs/latex-vorspann}


\renewcommand{\erwaehntePersonen}{Personen: Richard Beer-Hofmann, Otto Brahm, Marie Glümer, Alfred Kerr, Leopoldine Müller, Olga Schnitzler, Stefan Vacano}
\renewcommand{\erwaehnteInstitutionen}{Institutionen: Deutsches Theater Berlin}
\renewcommand{\erwaehnteOrte}{Orte: Berlin, Dessauer Straße, Wien}
\renewcommand{\erwaehnteWerke}{Werke: Der Tag, Marionetten, Michael Kramer. Drama, Neue Freie Presse, »Michael Kramer.«}
\section[ Paul Goldmann an Arthur Schnitzler, 22. 1. {[}1901{]}]{Paul Goldmann an Arthur Schnitzler, 22. 1. {[}1901{]}}
\nopagebreak\mylabel{v}
\rehead{ }\normalsize\beginnumbering\briefempfaengerindex{Schnitzler, Arthur@\textsc{Schnitzler, Arthur}!zzzGoldmann, Paul@\emph{von Paul Goldmann}!1901-01-221@{22. 1. {[}1901{]}}|(be}
\toendnotes[C]{\smallbreak\pagebreak[2]}\Standort{DLA, A:Schnitzler, HS.NZ85.1.3171.}
\physDesc{Brief, 1 Blatt, 4 Seiten
\newline{}Handschrift: blaue Tinte, deutsche Kurrent
\newline{}Schnitzler: mit rotem Buntstift sechs Unterstreichungen }\toendnotes[C]{\smallbreak}
\pstart
           \noindent{}\raggedleft{}{\pb}\textcolor{pink}{\textcolor{gray}{\textbf{DESSAUERSTRASSE 19}}}{}\ledrightnote{\textcolor{pink}{Dessauer Straße}}\pend
           
\pstart
           \textcolor{pink}{Berlin}{}\ledrightnote{\textcolor{pink}{Berlin}}, 22. Januar.\pend
           
\pstart\center{}Mein lieber Freund,\pend
\pstart
           \textsc{\textcolor{blue}{Mizzi Glümer}{}\ledrightnote{\textcolor{blue}{Marie Glümer}}} iſt krank, liegt zu Bett und ſieht ſo elend aus, daß ich erſchrocken bin (Unter
                  uns!){[}.{]} Du ſollteſt dem armen \textcolor{blue}{Mädel}{}\ledrightnote{{$\rightarrow$}\textcolor{blue}{Marie Glümer}} einen guten Brief ſchreiben.\pend
           
\pstart
           \textsc{\textcolor{blue}{Brahm}{}\ledrightnote{\textcolor{blue}{Otto Brahm}}} ſagte mir bei einer der letzten \begin{otherlanguage}{french}\textsc{Premièren}\end{otherlanguage}, er möchte von Dir einen oder zwei \label{K_L03055-1v}\edtext{Einakter}{\lemma{\textnormal{\emph{Einakter}}}\Cendnote{\textnormal{\emph{\textcolor{green}{Schnitzler}} übersandte \textcolor{blue}{Brahm} wenig später die \emph{\textcolor{green}{Marionetten}}, vgl. \emph{Der Briefwechsel Arthur
                        Schnitzler — Otto Brahm}. Vollständige Ausgabe. Herausgegeben,
                     eingeleitet und erläutert von Oskar Seidlin. Tübingen:
                        \emph{Niemeyer}{ }1975, S. 88.}}}\label{K_L03055-1h} haben. Wer iſt \textsc{\textcolor{blue}{Stefan Vacano}{}\ledrightnote{\textcolor{blue}{Stefan Vacano}}}? Ich kann mir die Aufführung ſeines \textcolor{green}{Stück}{}\ledrightnote{{$\rightarrow$}\textcolor{green}{Der Tag}}es nur durch \label{K_L03055-2v}\edtext{Beziehungen}{\lemma{\textnormal{\emph{Beziehungen}}}\Cendnote{\textnormal{Der
                     \textcolor{pink}{Wien}er \textcolor{blue}{Stefan Vacano} und \textcolor{blue}{Otto Brahm} waren
                  befreundet. \textcolor{blue}{Brahm} agierte auch als \textcolor{blue}{Vacano}s Förderer. So gelang etwa \textcolor{blue}{Vacano}s Vierakter \emph{\textcolor{green}{Der Tag}} am 19. 1. 1901 am
                     \emph{\textcolor{brown}{Deutschen Theater}} in \textcolor{pink}{Berlin} zur Uraufführung.}}}\label{K_L03055-2h} zwiſchen \textsc{\textcolor{blue}{Brahm}{}\ledrightnote{\textcolor{blue}{Otto Brahm}}} und ihm erklären, die nicht blos diejenigen des \textcolor{blue}{Theaterdirektor}{}\ledrightnote{{$\rightarrow$}\textcolor{blue}{Otto Brahm}}s zum \textcolor{blue}{Autor}{}\ledrightnote{{$\rightarrow$}\textcolor{blue}{Stefan Vacano}} ſind. Der {\pb}\textcolor{blue}{Dichter}{}\ledrightnote{{$\rightarrow$}\textcolor{blue}{Stefan Vacano}} des »\textcolor{green}{Tag}{}\ledrightnote{\textcolor{green}{Der Tag}}« ſieht auch danach aus. \textsc{\textcolor{blue}{Brahm}{}\ledrightnote{\textcolor{blue}{Otto Brahm}}} gleichfalls.\pend
           
\pstart
           Von \textsc{\textcolor{blue}{Olga G.}{}\ledrightnote{\textcolor{blue}{Olga Schnitzler}}} erhielt ich einen beinahe ſchwermüthigen Brief. Angenehmes Liebesglück! Warum
                  \label{K_L03055-3v}\edtext{quälſt}{\lemma{\textnormal{\emph{quälſt}}}\Cendnote{\textnormal{wohl Bezug auf \textcolor{blue}{Schnitzler}s gleichzeitige Liaison mit \textcolor{blue}{Leopoldine Müller}}}}\label{K_L03055-3h} Du das \textcolor{blue}{Mädel}{}\ledrightnote{{$\rightarrow$}\textcolor{blue}{Olga Schnitzler}} ſo?\pend
           
\pstart
           Es wäre ſchön, wenn Du in den \strikeout{\textcolor{gray}{B}} Anſichts- und Poſtkarten-Verkehr, den Du mit mir unterhältſt, auch einmal
               durch Abſendung eines Briefes eine erfriſchende Abwechſelung brächteſt. Ich wüßte
               beiſpielsweiſe gern, was \textsc{\textcolor{blue}{Richard}{}\ledrightnote{\textcolor{blue}{Richard Beer-Hofmann}}} macht. Selbſtverſtändlich ſchreibt er mir nicht. Er wid mir niemals ſo lange
               nicht ſchreiben können, als {\pb}ich im
               Stande ſein werde, mich darüber zu empören. In meiner \label{K_L03055-4v}\edtext{\textcolor{green}{Kritik}{}\ledrightnote{{$\rightarrow$}\textcolor{green}{»Michael Kramer.«}}}{\lemma{\textnormal{\emph{Kritik}}}\Cendnote{\textnormal{\textcolor{blue}{Paul Goldmann}: \emph{\textcolor{green}{Feuilleton. »Michael Kramer.«}}. In: \emph{\textcolor{green}{Neue Freie Presse}}, Nr. 13055, 28. 12. 1900, Morgenblatt, S. 1–3.}}}\label{K_L03055-4h} über »\textcolor{green}{Michael Kramer}{}\ledrightnote{\textcolor{green}{Michael Kramer. Drama}}« ſoll er, wie ich höre, –
               Schadenfreude gefunden haben. Es iſt intereſſant, daß dieſer feinſte \strikeout{\textcolor{gray}{×}\-\textcolor{gray}{×}\-\textcolor{gray}{×}\-\textcolor{gray}{×}\-\textcolor{gray}{×}}\textcolor{blue}{Menſchenkenner}{}\ledrightnote{{$\rightarrow$}\textcolor{blue}{Richard Beer-Hofmann}} gerade mich
               weniger kennt, als irgend Jemand, und daß gerade dieſer bewundernswürdig geſcheite
                  \textcolor{blue}{Menſch}{}\ledrightnote{{$\rightarrow$}\textcolor{blue}{Richard Beer-Hofmann}} ſo dumm über mich
               urtheilt. Ich werde für ihn einen Commentar über mich ſchreiben. Bitte ſag’ ihm das,
               – und daß ich ihn ſehr vermiſſe und daß ich viel darum gäbe, könnte ich ihn immer in
               meiner Nähe haben. {\pb}Ich bin vollſtändig
               ohne Verkehr, – vollſtändig einſam. \textsc{\textcolor{blue}{Kerr}{}\ledrightnote{\textcolor{blue}{Alfred Kerr}}} benimmt ſich blödſinnig. Seit Du aus \textcolor{pink}{Berlin}{}\ledrightnote{\textcolor{pink}{Berlin}}
               fort biſt, habe ich ihn nicht mehr geſprochen. Wenn er mich im Theater ſieht, drückt
               er mir raſch die Hand und läuft weg{\dotssix}\pend
           
\pstart
           Schreib’ mir bald!\pend
           
\pstart
           Viele treue Grüße! {\\[\baselineskip]}Dein {\\[\baselineskip]}\spacefill\mbox{Paul Goldmann.}\pend
           \leftskip=0em{}\endnumbering\briefempfaengerindex{Schnitzler, Arthur@\textsc{Schnitzler, Arthur}!zzzGoldmann, Paul@\emph{von Paul Goldmann}!1901-01-221@{22. 1. {[}1901{]}}|)be}\mylabel{h}
\begin{anhang}
\end{anhang}\normalsize

\doendnotes{C}
\bigskip
\vfill

\clearpage

\footnotesize

\lohead{\textsc{register}}

% Definiere theindex-Environment komplett neu ohne reledmac
\makeatletter
\renewenvironment{theindex}{%
  \section*{\indexname}%
  \setlength{\parindent}{0pt}%
  \setlength{\parskip}{0pt plus 0.3pt}%
  \let\item\@idxitem
}{%
  \clearpage
}
\makeatother

\IfFileExists{\jobname-pw.ind}{\input{\jobname-pw.ind}}{}

\end{document}

      