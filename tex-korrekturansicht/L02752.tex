%% latex-korrekturansicht-vorspann.tex
%% Vorspann für die Korrekturansicht.
%% Lädt die gemeinsame Datei latex-vorspann.tex mit gesetztem Schalter.

\newif\ifkorrekturansicht
\korrekturansichttrue

\input{../tex-inputs/latex-vorspann}


               \section[Paul Goldmann an Arthur Schnitzler, 14. 10. {[}1895{]}]{ Paul Goldmann an Arthur Schnitzler, 14. 10. {[}1895{]}}\nopagebreak\mylabel{v}\rehead{ }\normalsize\beginnumbering\briefempfaengerindex{Schnitzler, Arthur@\textsc{Schnitzler, Arthur}!zzzGoldmann, Paul@\emph{von Paul Goldmann}!1895-10-141@{14. 10. {[}1895{]}}|(be} \toendnotes[C]{\smallbreak\pagebreak[2]} \Standort{DLA, A:Schnitzler, HS.NZ85.1.3165.}
\physDesc{Brief, 1 Blatt, 2 Seiten
\newline{}Handschrift: blaue Tinte, deutsche Kurrent
\newline{}Schnitzler: 1) mit Bleistift das Jahr » 95« vermerkt 2) mit rotem Buntstift eine Unterstreichung}\toendnotes[C]{\smallbreak}\pstart
           \noindent{}{\pb}\textcolor{gray}{\textbf{\textbf{\textcolor{brown}{Frankfurter Zeitung}{}\ledrightnote{\textcolor{brown}{Frankfurter Zeitung}}}}}\pend
           \pstart
           \textcolor{gray}{\textbf{(\textcolor{brown}{\begin{otherlanguage}{french}Gazette de Francfort\end{otherlanguage}}{}\ledrightnote{\textcolor{brown}{Frankfurter Zeitung}}). }}\pend
           \pstart
           \textcolor{gray}{\textbf{\textbf{\begin{otherlanguage}{french}Fondateur M. \textcolor{blue}{L.
                                 Sonnemann}{}\ledrightnote{\textcolor{blue}{Leopold Sonnemann}}\end{otherlanguage}.}}}\hfill \textsc{\textcolor{pink}{Paris}{}\ledrightnote{\textcolor{pink}{Paris}}}, 14. October.\pend
           \pstart
           \begin{otherlanguage}{french}\textcolor{gray}{\textbf{\textcolor{green}{Journal}{}\ledrightnote{→\textcolor{green}{Frankfurter Zeitung}} politique,
                        financier,}}\end{otherlanguage}\pend
           \pstart
           \begin{otherlanguage}{french}\textcolor{gray}{\textbf{commercial et littéraire.}}\end{otherlanguage}\pend
           \pstart
           \begin{otherlanguage}{french}\textcolor{gray}{\textbf{\textbf{Paraissant trois fois par jour.}}}\end{otherlanguage}\pend
           \pstart
           \begin{otherlanguage}{french}\textcolor{gray}{\textbf{\textbf{Bureau à \textcolor{pink}{Paris}{}\ledrightnote{\textcolor{pink}{Paris}}:}}}\end{otherlanguage}\pend
           \pstart
           \begin{otherlanguage}{french}\textcolor{gray}{\textbf{\textbf{\textcolor{pink}{24. Rue Feydeau}{}\ledrightnote{\textcolor{pink}{rue Feydeau}}.}}}\end{otherlanguage}\pend
           \pstart\center{}Mein lieber Freund,\pend\pstart
           Dank für Deinen lieben Brief! Schreib’ mir ausführlicher, ſobald Du kannſt, aber
               nicht früher: ich warte gern.\pend
           \pstart
           Ich ſchreibe Dir heut nur, weil ich ſoeben \textsc{\textcolor{blue}{Bahr}{}\ledrightnote{\textcolor{blue}{Hermann Bahr}}s}{ }\label{K_L02752-1v}\edtext{\textcolor{green}{Referat}{}\ledrightnote{→\textcolor{green}{Burgtheater (Liebelei, Schauspiel in drei Acten von Arthur Schnitzler. Rechte der Seele, Schauspiel in einem Act von Guiseppe Giacosa. Zum ersten Mal aufgeführt am 9. October)}}}{\lemma{\textnormal{\emph{Referat}}}\Cendnote{\textnormal{\textcolor{blue}{Hermann Bahr}: \emph{\textcolor{green}{Burgtheater (Liebelei, Schauspiel in drei Acten von Arthur
                        Schnitzler. Rechte der Seele, Schauspiel in einem Act von Guiseppe Giacosa.
                        Zum ersten Mal aufgeführt am 9. October)}}. In: \emph{\textcolor{green}{Die Zeit}}, Bd. 5, Nr. 54, 12. 10. 1895, S. 27–28.}}}\label{K_L02752-1h} geleſen habe. Das iſt keine
               Kritik, das iſt ein Bubenſtreich. Ich ſehe von der Dummheit und Gemeinheit ab, mit
               der die literariſche Berurtheilung abgefaßt iſt. Aber dieſer \textcolor{green}{Artikel}{}\ledrightnote{→\textcolor{green}{Burgtheater (Liebelei, Schauspiel in drei Acten von Arthur Schnitzler. Rechte der Seele, Schauspiel in einem Act von Guiseppe Giacosa. Zum ersten Mal aufgeführt am 9. October)}} enthält \label{K_L02752-55v}\edtext{perſönliche Beleidigungen}{\lemma{\textnormal{\emph{perſönliche Beleidigungen}}}\Cendnote{\textnormal{Die Kritik lässt sich in diesem Satz zusammenfassen: \textcolor{blue}{Schnitzler} »weiß die neuen Elemente
                     unserer Stadt zu fühlen, auch zu schildern; ›dramatisieren‹ kann er sie noch
                     nicht.« Wo \textcolor{blue}{Goldmann} genau die
                  persönliche Beleidigung festmacht, ist nicht genau zu bestimmen, eventuell in der
                  behaupteten Nähe von \textcolor{blue}{Schnitzler} und den
                  Lebemännern, die er schildert, oder in dieser Aussage: »›Er ist für eine
                     andere gestorben! für eine Frau, die er geliebt hat – ihr Mann hat ihn
                     umgebracht! Und ich – was bin ich denn? Was war denn ich? Was bin denn ich ihm
                     gewesen?‹ Diese Klage hat einen so innigen und echten Ton, dass man merkt, sie
                     kommt dem Autor vom Herzen; das sehr wienerische Elend, an dem Leben so daneben
                     vorbeizuleben, hat er, das vernimmt man, wohl an sich selbst
                  gespürt.«}}}\label{K_L02752-55h} gegen Dich. {\pb}Ich habe
               vor Entrüſtung gezittert, als ich das las. Wäre ich in \textcolor{pink}{Wien}{}\ledrightnote{\textcolor{pink}{Wien}}, ſo würde \uline{ich} den \textcolor{blue}{Menſchen}{}\ledrightnote{→\textcolor{blue}{Hermann Bahr}} zur Rechenſchaft gezogen haben. Du
               ſelbſt kannſt kaum etwas machen, da die Welt Dir in jedem Falle Unrecht geben würde.
               Aber ich halte es für abſolut unumgänglich, daß Du Deine perſönlichen Beziehungen zu
               dem \textcolor{blue}{Burſchen}{}\ledrightnote{→\textcolor{blue}{Hermann Bahr}} abbrichſt.
                  \label{K_L02752-44v}\edtext{Das Gleiche erwarte ich von \textsc{\textcolor{blue}{Richard}{}\ledrightnote{\textcolor{blue}{Richard Beer-Hofmann}}}}{\lemma{\textnormal{\emph{Das … Richard}}}\Cendnote{\textnormal{Auch \textcolor{blue}{Schnitzler} hat sich vorgestellt, \textcolor{blue}{Beer-Hofmann} und \textcolor{blue}{Hofmannsthal} von \textcolor{blue}{Bahr} vorgestellt,
                     vgl. A. S.: \emph{Tagebuch}, 6. 11. 1895}}}\label{K_L02752-44h}.
               Ein \textcolor{blue}{Bube}{}\ledrightnote{→\textcolor{blue}{Hermann Bahr}}, der mit Schmutz
               wirft, gehört nicht in Eure Geſellſchaft.\pend
           \pstart
           Viele treue Grüße! Dein {\\[\baselineskip]}\spacefill\mbox{Paul Goldmann.}\pend
           \leftskip=0em{}\endnumbering\briefempfaengerindex{Schnitzler, Arthur@\textsc{Schnitzler, Arthur}!zzzGoldmann, Paul@\emph{von Paul Goldmann}!1895-10-141@{14. 10. {[}1895{]}}|)be}\mylabel{h}  \normalsize

\doendnotes{C}
\bigskip
\vfill

\clearpage

\footnotesize

\lohead{\textsc{register}}

% Definiere theindex-Environment komplett neu ohne reledmac
\makeatletter
\renewenvironment{theindex}{%
  \section*{\indexname}%
  \setlength{\parindent}{0pt}%
  \setlength{\parskip}{0pt plus 0.3pt}%
  \let\item\@idxitem
}{%
  \clearpage
}
\makeatother

\IfFileExists{\jobname-pw.ind}{\input{\jobname-pw.ind}}{}

\end{document}

      