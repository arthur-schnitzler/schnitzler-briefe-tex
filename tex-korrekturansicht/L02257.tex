%% latex-korrekturansicht-vorspann.tex
%% Vorspann für die Korrekturansicht.
%% Lädt die gemeinsame Datei latex-vorspann.tex mit gesetztem Schalter.

\newif\ifkorrekturansicht
\korrekturansichttrue

\input{../tex-inputs/latex-vorspann}


               \section[Gerty von Hofmannsthal an Arthur Schnitzler, 23. 2. 1917]{ Gerty von Hofmannsthal an Arthur Schnitzler, 23. 2. 1917}\nopagebreak\mylabel{v}\rehead{ }\normalsize\beginnumbering\briefempfaengerindex{Schnitzler, Arthur@\textsc{Schnitzler, Arthur}!zzzHofmannsthal, Gertrude von@\emph{von Gertrude von Hofmannsthal}!1917-02-231@{23. 2. 1917}|(be} \toendnotes[C]{\smallbreak\pagebreak[2]} \Standort{CUL, Schnitzler, B 43.}
\physDesc{Briefkarte
\newline{}Handschrift: schwarze Tinte, lateinische Kurrent
\newline{}Schnitzler: mit Bleistift beschriftet: »\textsc{Ger Hofmannsthal}« \newline{}Ordnung: 1) mit Bleistift von \textcolor{blue}{Frieda Pollak} (?) mit dem Buchstaben »A« (Abgeschrieben/Abschrift) gekennzeichnet 2) mit Bleistift von unbekannter Hand nummeriert: »\strikeout{345}«3) mit Bleistift von unbekannter Hand nummeriert:
                                    »357«}\buchAbdrucke{\weitereDrucke{Hugo von Hofmannsthal, Arthur Schnitzler: \emph{Briefwechsel}. Hg. Therese Nickl und Heinrich Schnitzler. Frankfurt am Main: \emph{S. Fischer} 1964, S. 389.} }\toendnotes[C]{\smallbreak}\pstart
           \raggedleft{}{\pb}23. II. 1917\pend
           \pstart
           Lieber Arthur, ich musste Ihren Brief an \textcolor{blue}{Hugo}{}\ledrightnote{\textcolor{blue}{Hugo von Hofmannsthal}} öffnen weil er inzwischen abgereist ist. Ich glaube, dass
                  \textcolor{blue}{Hugo}{}\ledrightnote{\textcolor{blue}{Hugo von Hofmannsthal}} ebensowenig wie ich weiss, ob Herr \textcolor{blue}{B}{}\ledrightnote{\textcolor{blue}{Jean Billiter}} Pläne mit seinen Stücken hat; ich glaube das
               Wichtigste war ihm ein »Urtheil« und wie ich {\pb}aus Ihrem Brief entnehme, kann es wohl
               nicht sehr günstig sein. Vielleicht wäre es eher gesund diesem sonst so begabten und
               interessanten \textcolor{blue}{Menschen}{}\ledrightnote{→\textcolor{blue}{Jean Billiter}} die
               Wahrheit zu sagen. So viel ich von meinem \textcolor{blue}{Schwager}{}\ledrightnote{\textcolor{blue}{Arnold Schereschewsky}} weiss, mit dem er sehr befreundet ist, hat er sich noch nie
               literarisch betätigt.\pend
           \pstart
           Viele Grüsse an \textcolor{blue}{Olga}{}\ledrightnote{\textcolor{blue}{Olga Schnitzler}}{\\[\baselineskip]}Ihre \spacefill\mbox{Gerty.}\pend
           \leftskip=0em{}\endnumbering\briefempfaengerindex{Schnitzler, Arthur@\textsc{Schnitzler, Arthur}!zzzHofmannsthal, Gertrude von@\emph{von Gertrude von Hofmannsthal}!1917-02-231@{23. 2. 1917}|)be}\mylabel{h}  \normalsize

\doendnotes{C}
\bigskip
\vfill

\clearpage

\footnotesize

\lohead{\textsc{register}}

% Definiere theindex-Environment komplett neu ohne reledmac
\makeatletter
\renewenvironment{theindex}{%
  \section*{\indexname}%
  \setlength{\parindent}{0pt}%
  \setlength{\parskip}{0pt plus 0.3pt}%
  \let\item\@idxitem
}{%
  \clearpage
}
\makeatother

\IfFileExists{\jobname-pw.ind}{\input{\jobname-pw.ind}}{}

\end{document}

      