%% latex-korrekturansicht-vorspann.tex
%% Vorspann für die Korrekturansicht.
%% Lädt die gemeinsame Datei latex-vorspann.tex mit gesetztem Schalter.

\newif\ifkorrekturansicht
\korrekturansichttrue

\input{../tex-inputs/latex-vorspann}


\renewcommand{\erwaehntePersonen}{Personen: Hermann Bahr, Richard Beer-Hofmann, Hugo von Hofmannsthal, Karl Kraus, Karl Peter Rosner, Julius Schaumberger}
\renewcommand{\erwaehnteOrte}{Orte: Café Luitpold, Grillparzerstraße, I., Innere Stadt, München, Wien}
\renewcommand{\erwaehnteWerke}{}
\section[Karl Kraus u. a. an Arthur Schnitzler, 30. 9. 1893]{Karl Kraus u. a. an Arthur Schnitzler, 30. 9. 1893}
\nopagebreak\mylabel{v}
\rehead{ }\normalsize\beginnumbering\briefempfaengerindex{Schnitzler, Arthur@\textsc{Schnitzler, Arthur}!zzzSchaumberger, Julius@\emph{von Julius Schaumberger}!1893-09-301@{30. 9. 1893}|(be}\briefempfaengerindex{Schnitzler, Arthur@\textsc{Schnitzler, Arthur}!zzzRosner, Karl Peter@\emph{von Karl Peter Rosner}!1893-09-301@{30. 9. 1893}|(be}\briefempfaengerindex{Schnitzler, Arthur@\textsc{Schnitzler, Arthur}!zzzKraus, Karl@\emph{von Karl Kraus}!1893-09-301@{30. 9. 1893}|(be}
\toendnotes[C]{\smallbreak\pagebreak[2]}\Standort{CUL, Schnitzler, B 55.}
\physDesc{Postkarte, 396 Zeichen
\newline{}Handschrift Karl Kraus: 1) Bleistift, deutsche Kurrent\hspace{1em}2) Bleistift, lateinische Kurrent (\noindent{}Adresse)\hspace{1em}
\newline{}Handschrift Karl Peter Rosner: Bleistift, lateinische Kurrent
\newline{}Handschrift Julius Schaumberger: Bleistift, deutsche Kurrent
\newline{}Versand: Stempel: »\nobreak{}\oindex{Muenchen@\textbf{München}, \emph{P.PPLA}|pwk}München II, 8–9\nobreak{}«.  
\newline{}Ordnung: mit Bleistift von unbekannter Hand nummeriert:
                                 »4« }
\buchAbdrucke{\weitereDrucke{1) \emph{Karl Kraus und Arthur Schnitzler. Eine Dokumentation.} Hg. Reinhard Urbach. In: \emph{Literatur und Kritik}, Bd. 49, Oktober 1970, S. 515–516.} \weitereDrucke{2) Hermann Bahr, Arthur Schnitzler: \emph{Briefwechsel, Aufzeichnungen, Dokumente (1891–1931)}. Hg. Kurt Ifkovits und Martin Anton Müller. Göttingen: \emph{Wallstein} 2018, S. 43.} }\pstart{}{\pb}Herrn D\textsuperscript{r.}
                  Arthur Schnitzler\pend{}\pstart{}\textcolor{pink}{Wien I}{}\ledrightnote{\textcolor{pink}{I., Innere Stadt}}\pend{}\pstart{}\textcolor{pink}{Grillparzerstraße 7. I}{}\ledrightnote{\textcolor{pink}{Grillparzerstraße}}\pend{}
{\bigskip}\vspace{1em}
\pstart
           {\pb}\textcolor{pink}{München, Café Luitpold}{}\ledrightnote{\textcolor{pink}{Café Luitpold}},
                  30/9 93.\pend
           \vspace{0.5em}
\pstart
           Liebster Doktor, herzlichſte Grüße.\pend
           
\pstart
           Grüßen Sie beſtens auch \textcolor{blue}{Beer-Hofmann}{}\ledrightnote{\textcolor{blue}{Richard Beer-Hofmann}}{ }\introOben{}\textcolor{blue}{Loris}{}\ledrightnote{\textcolor{blue}{Hugo von Hofmannsthal}}\introOben{}. Ich habe Ihnen \uuline{vieles}{ }Sie \substVorne{}\textsuperscript{i}\substDazwischen{}I\substHinten{}ntereſſierende zu ſagen.\pend
           \pstart Ihr \spacefill\mbox{Kraus}\hspace*{1.5em}poste restante\pend{}\vspace{1em}
\pstart
           \noindent{}{[}hs. Rosner:{]} \textsc{Viele innige Grüße an Sie, \textcolor{blue}{Hoffmann}{}\ledrightnote{\textcolor{blue}{Richard Beer-Hofmann}}, \textcolor{blue}{Loris}{}\ledrightnote{\textcolor{blue}{Hugo von Hofmannsthal}}, \textcolor{blue}{Bahr}{}\ledrightnote{\textcolor{blue}{Hermann Bahr}}}\pend
           
\pstart
           \textsc{Ihr treuer}{\\[\baselineskip]}\spacefill\mbox{Karl Rosner.}\pend
           \leftskip=0em{}\vspace{1em}
\pstart
           \noindent{}{[}hs. Kraus:{]} Dieſer Mensch hat ſich \uline{ſehr} gebeſſert, alle Poſen ſich abgewöhnt. \spacefill\mbox{Kraus}\pend
           \vspace{1em}
\pstart
           \noindent{}\centering{}–------\pend
           
\pstart
           \noindent{}{[}hs. Schaumberger:{]} \textsc{Prosit}\spacefill\mbox{JSchaumberger}\pend
           \endnumbering\briefempfaengerindex{Schnitzler, Arthur@\textsc{Schnitzler, Arthur}!zzzSchaumberger, Julius@\emph{von Julius Schaumberger}!1893-09-301@{30. 9. 1893}|)be}\briefempfaengerindex{Schnitzler, Arthur@\textsc{Schnitzler, Arthur}!zzzRosner, Karl Peter@\emph{von Karl Peter Rosner}!1893-09-301@{30. 9. 1893}|)be}\briefempfaengerindex{Schnitzler, Arthur@\textsc{Schnitzler, Arthur}!zzzKraus, Karl@\emph{von Karl Kraus}!1893-09-301@{30. 9. 1893}|)be}\mylabel{h}  \normalsize

\doendnotes{C}
\bigskip
\vfill

\clearpage

\footnotesize

\lohead{\textsc{register}}

% Definiere theindex-Environment komplett neu ohne reledmac
\makeatletter
\renewenvironment{theindex}{%
  \section*{\indexname}%
  \setlength{\parindent}{0pt}%
  \setlength{\parskip}{0pt plus 0.3pt}%
  \let\item\@idxitem
}{%
  \clearpage
}
\makeatother

\IfFileExists{\jobname-pw.ind}{\input{\jobname-pw.ind}}{}

\end{document}

      