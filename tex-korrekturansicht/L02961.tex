%% latex-korrekturansicht-vorspann.tex
%% Vorspann für die Korrekturansicht.
%% Lädt die gemeinsame Datei latex-vorspann.tex mit gesetztem Schalter.

\newif\ifkorrekturansicht
\korrekturansichttrue

\input{../tex-inputs/latex-vorspann}


\renewcommand{\erwaehntePersonen}{Personen: Felix Salten, Josefine Lydia von Weisswasser}
\renewcommand{\erwaehnteOrte}{Orte: Dölsach, Pressbaum, Wien}
\renewcommand{\erwaehnteWerke}{}
\section[ Arthur Schnitzler an Felix Salten, 17. 8. 1893]{Arthur Schnitzler an Felix Salten, 17. 8. 1893}
\nopagebreak\mylabel{v}
\rehead{ }\normalsize\beginnumbering\briefempfaengerindex{Salten, Felix@\textsc{Salten, Felix}!zzzSchnitzler, Arthur@\emph{von Arthur Schnitzler}!1893-08-171@{17. 8. 1893}|(be}
\toendnotes[C]{\smallbreak\pagebreak[2]}\Standort{Wienbibliothek im Rathaus, ZPH 1681, 2.1.516.}
\physDesc{Brief, 1 Blatt, 4 Seiten, 688 Zeichen (Briefpapier mit Trauerrand)
\newline{}Handschrift: Bleistift, deutsche Kurrent
\newline{}Ordnung: mit Bleistift von unbekannter Hand Nummerierung der Blätter des Konvoluts:
                                    »78«–»79« }
\buchAbdrucke{\weitereDrucke{Arthur Schnitzler: \emph{Briefe 1875–1912}. Hg. Therese Nickl und Heinrich Schnitzler. Frankfurt am Main: \emph{S. Fischer} 1981, S. 213.} }\toendnotes[C]{\smallbreak}
\pstart
           \raggedleft{}{\pb}\uline{17. 8. 93}\pend
           
\pstart{}Lieber Freund,\pend
\pstart
           ich ka{\geminationn}{ }\label{K_L02961-1v}\edtext{Montag oder Dinſtg bei
               Ihnen ſein}{\lemma{\textnormal{\emph{Montag … ſein}}}\Cendnote{\textnormal{siehe Felix Salten an Arthur Schnitzler, 14. 8. 1893}}}\label{K_L02961-1h}. Aber ſchreiben Sie mir gefälligſt, \uline{wohin} ich
               fahren ſoll, wo Sie mich erwarten wollen, {\pb}und, ſoweit dies möglich, wie unſre Partie ſich eigentlich geſtalten wird. –\pend
           
\pstart
           Sie müſſen mir \uline{gleich} ſchreiben. –\pend
           
\pstart
           Plötzlich iſt eine unterträgliche Hitze über \textcolor{pink}{Wien}{}\ledrightnote{\textcolor{pink}{Wien}}
               hereingebrochen. {\pb}Heute{ }früh kam ich \textsc{per}{ }\textsc{Bic.} aus \textcolor{pink}{Preßbaum}{}\ledrightnote{\textcolor{pink}{Pressbaum}}
               herein, wo ich eine \label{K_L02961-2v}\edtext{\textcolor{blue}{Nacht der »Liebe«}{}\ledrightnote{{$\rightarrow$}\textcolor{blue}{Josefine Lydia von Weisswasser}}}{\lemma{\textnormal{\emph{Nacht der »Liebe«}}}\Cendnote{\textnormal{siehe A. S.: \emph{Tagebuch}, 16. 8. 1893}}}\label{K_L02961-2h} verbracht hatte. Dumpfiges \label{K_L02961-3v}\edtext{Gaſthof}{\lemma{\textnormal{\emph{Gaſthof}}}\Cendnote{\textnormal{nicht ermittelt}}}\label{K_L02961-3h}zi{\geminationm}er mit ſchlechten Betten – der Abend
               vorher war ganz ſchön; – denn was lügt einem die Si{\geminationn}lichkeit nach dem {\pb}Nachtmahl \introOben{}nicht\introOben{} alles vor!\pend
           
\pstart
           – Wodurch \textcolor{blue}{ſie}{}\ledrightnote{{$\rightarrow$}\textcolor{blue}{Josefine Lydia von Weisswasser}} ſich von den
               Weibern unterſcheidet, die auch vor dem Nachtmahl lügen. –\pend
           
\pstart
           – Leben Sie wohl, {\\[\baselineskip]}ſeien Sie herzlich gegrüßt, {\\[\baselineskip]}\spacefill\mbox{Arthur}\pend
           \leftskip=0em{}\endnumbering\briefempfaengerindex{Salten, Felix@\textsc{Salten, Felix}!zzzSchnitzler, Arthur@\emph{von Arthur Schnitzler}!1893-08-171@{17. 8. 1893}|)be}\mylabel{h}  \normalsize

\doendnotes{C}
\bigskip
\vfill

\clearpage

\footnotesize

\lohead{\textsc{register}}

% Definiere theindex-Environment komplett neu ohne reledmac
\makeatletter
\renewenvironment{theindex}{%
  \section*{\indexname}%
  \setlength{\parindent}{0pt}%
  \setlength{\parskip}{0pt plus 0.3pt}%
  \let\item\@idxitem
}{%
  \clearpage
}
\makeatother

\IfFileExists{\jobname-pw.ind}{\input{\jobname-pw.ind}}{}

\end{document}

      