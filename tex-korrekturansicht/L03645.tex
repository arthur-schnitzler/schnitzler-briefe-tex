%% latex-korrekturansicht-vorspann.tex
%% Vorspann für die Korrekturansicht.
%% Lädt die gemeinsame Datei latex-vorspann.tex mit gesetztem Schalter.

\newif\ifkorrekturansicht
\korrekturansichttrue

\input{../tex-inputs/latex-vorspann}


\section[Stefan Zweig an Arthur Schnitzler, {[}7.?{]} 12. 1914]{L03645 Stefan Zweig an Arthur Schnitzler, {[}7.?{]} 12. 1914}
\nopagebreak\mylabel{L03645v}
\rehead{ }\normalsize\beginnumbering\briefempfaengerindex{, @\textsc{, }!zzz, @\emph{von  }!1914-12-071@{{[}7.?{]} 12. 1914}|(be}
\toendnotes[C]{\smallbreak\pagebreak[2]}\Standort{CUL, Schnitzler, B 118.}
\physDesc{Bildpostkarte, 438 Zeichen
\newline{}Handschrift: blaue Tinte, lateinische Kurrent
\newline{}Versand: Stempel: »\nobreak{}\oindex{Wien@\textbf{Wien}, \emph{Verwaltungsgebiet}|pwk}Wien, \textcolor{gray}{7.} XII. \textcolor{gray}{1}4, 2\nobreak{}«.  }
\buchAbdrucke{\weitereDrucke{Stefan Zweig: \emph{Briefwechsel mit Hermann Bahr, Sigmund Freud, Rainer Maria
                        Rilke und Arthur Schnitzler}. Frankfurt am Main: \emph{S. Fischer} 1987, S. 388.} }\toendnotes[C]{\smallbreak}\pstart{}{\pb}D\textsuperscript{r} Arthur
                  Schnitzler\pend{}\pstart{}\textcolor{pink}{Wien – Cottage}\oindex{Wien@\textbf{Wien}!XVIII., Währing@\textbf{XVIII., Währing}!Währinger Cottage@\textbf{Währinger Cottage}, \emph{Teil eines besiedelten Ortes}|pw}{}\ledrightnote{\textcolor{pink}{Währinger Cottage}}\pend{}\pstart{}\textcolor{pink}{Sternwartestrasse 71}\oindex{Wien@\textbf{Wien}!XVIII., Währing@\textbf{XVIII., Währing}!Sternwartestraße 71@\textbf{Sternwartestraße 71}, \emph{Wohngebäude}|pw}{}\ledrightnote{\textcolor{pink}{Sternwartestraße 71}}\pend{}{\bigskip}
\pstart
           \noindent{}\textcolor{gray}{\textbf{{\pb}\textcolor{blue}{GUSTINUS AMBROSI}\pwindex{Ambrosi, Gustinus 24.\,2.\,1893 Eisenstadt – 30.\,6.\,1975 Wien@\textsc{Ambrosi, Gustinus} (24.\,2.\,1893 Eisenstadt – 30.\,6.\,1975 Wien), \emph{Schriftsteller, Bildhauer, Philosoph}|pw}{}\ledrightnote{\textcolor{blue}{Gustinus Ambrosi}}}}\pend
           
\pstart
           \textcolor{gray}{\textbf{\textcolor{green}{BÜSTE STEFAN ZWEIG}\pwindex{Ambrosi, Gustinus 24.\,2.\,1893 Eisenstadt – 30.\,6.\,1975 Wien@\textsc{Ambrosi, Gustinus} (24.\,2.\,1893 Eisenstadt – 30.\,6.\,1975 Wien), \emph{Schriftsteller, Bildhauer, Philosoph}!Stefan Zweig@\strich\emph{Stefan Zweig}|pw}{}\ledrightnote{\textcolor{green}{Stefan Zweig}}}}\pend
           \vspace{1em}
\pstart{}{\pb}Verehrter Herr Doktor,\pend\vspace{0.5em}
\pstart
           ich komme \label{K_L03645-1v}\edtext{Donnerstag}{\lemma{\textnormal{\emph{Donnerstag}}}\Cendnote{\textnormal{Vgl. A. S.: \emph{Tagebuch}, 10. 12. 1914.}}}\label{K_L03645-1} freudigst und pünktlichst. Diese
               Karte stellt ein \textcolor{green}{Werk}\pwindex{Ambrosi, Gustinus 24.\,2.\,1893 Eisenstadt – 30.\,6.\,1975 Wien@\textsc{Ambrosi, Gustinus} (24.\,2.\,1893 Eisenstadt – 30.\,6.\,1975 Wien), \emph{Schriftsteller, Bildhauer, Philosoph}!Stefan Zweig@\strich\emph{Stefan Zweig}|pwv}{}\ledrightnote{{$\rightarrow$}\emph{\textcolor{green}{Stefan Zweig}}} des
               wirklich genialen taubstummen Bildhauers \textcolor{blue}{Ambrosi}\pwindex{Ambrosi, Gustinus 24.\,2.\,1893 Eisenstadt – 30.\,6.\,1975 Wien@\textsc{Ambrosi, Gustinus} (24.\,2.\,1893 Eisenstadt – 30.\,6.\,1975 Wien), \emph{Schriftsteller, Bildhauer, Philosoph}|pw}{}\ledrightnote{\textcolor{blue}{Gustinus Ambrosi}} dar, der in diesem Jahre bei \textcolor{blue}{Gerhardt Hauptmann}\pwindex{Hauptmann, Gerhart 15.\,11.\,1862 Szczawno-Zdrój – 6.\,6.\,1946 Jagniątków@\textsc{Hauptmann, Gerhart} (15.\,11.\,1862 Szczawno-Zdrój – 6.\,6.\,1946 Jagniątków), \emph{Schriftsteller}|pw}{}\ledrightnote{\textcolor{blue}{Gerhart Hauptmann}} ein wundervolles \textcolor{green}{Portrait}\pwindex{Ambrosi, Gustinus 24.\,2.\,1893 Eisenstadt – 30.\,6.\,1975 Wien@\textsc{Ambrosi, Gustinus} (24.\,2.\,1893 Eisenstadt – 30.\,6.\,1975 Wien), \emph{Schriftsteller, Bildhauer, Philosoph}!Gerhart Hauptmann@\strich\emph{Gerhart Hauptmann}|pwv}{}\ledrightnote{{$\rightarrow$}\emph{\textcolor{green}{Gerhart Hauptmann}}} machte und keinen sehnlichern \label{K_L03645-2v}\edtext{Wunsch}{\lemma{\textnormal{\emph{Wunsch}}}\Cendnote{\textnormal{\textcolor{blue}{Schnitzler}
               dürfte auf den Wunsch nicht reagiert haben. In seinem Nachlass ist nur der Durchschlag eines 
               Schreibens an \textcolor{blue}{Ambrosi}\pwindex{Ambrosi, Gustinus 24.\,2.\,1893 Eisenstadt – 30.\,6.\,1975 Wien@\textsc{Ambrosi, Gustinus} (24.\,2.\,1893 Eisenstadt – 30.\,6.\,1975 Wien), \emph{Schriftsteller, Bildhauer, Philosoph}|pwk} ein Jahrzehnt später überliefert (20. 11. 1924, \emph{Deutsches Literaturarchiv Marbach}, HS.1985.1.242).}}}\label{K_L03645-2} als den: Sie
               möchten ihm einmal 2 x 2 Stunden widmen, dass er auch die Ihre schaffen könnte.\pend
           \pstart Mit vielen Grüssen Ihr getreuer \spacefill\mbox{Stefan Zweig}\pend{}\selectlanguage{ngerman}\endnumbering\briefempfaengerindex{, @\textsc{, }!zzz, @\emph{von  }!1914-12-071@{{[}7.?{]} 12. 1914}|)be}\mylabel{L03645h}  \normalsize

\doendnotes{C}
\bigskip
\vfill

\clearpage

\footnotesize

\lohead{\textsc{register}}

% Definiere theindex-Environment komplett neu ohne reledmac
\makeatletter
\renewenvironment{theindex}{%
  \section*{\indexname}%
  \setlength{\parindent}{0pt}%
  \setlength{\parskip}{0pt plus 0.3pt}%
  \let\item\@idxitem
}{%
  \clearpage
}
\makeatother

\IfFileExists{\jobname-pw.ind}{\input{\jobname-pw.ind}}{}

\end{document}

      