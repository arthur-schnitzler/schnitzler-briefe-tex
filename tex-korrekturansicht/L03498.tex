%% latex-korrekturansicht-vorspann.tex
%% Vorspann für die Korrekturansicht.
%% Lädt die gemeinsame Datei latex-vorspann.tex mit gesetztem Schalter.

\newif\ifkorrekturansicht
\korrekturansichttrue

\input{../tex-inputs/latex-vorspann}


\renewcommand{\erwaehntePersonen}{Personen: Felix Salten}
\renewcommand{\erwaehnteOrte}{Orte: Noordwijk, Seis am Schlern, Tirol, Österreich}
\renewcommand{\erwaehnteWerke}{}
\section[ Felix Salten an Arthur Schnitzler, 14. 7. 1908]{Felix Salten an Arthur Schnitzler, 14. 7. 1908}
\nopagebreak\mylabel{v}
\rehead{ }\normalsize\beginnumbering\briefempfaengerindex{Schnitzler, Arthur@\textsc{Schnitzler, Arthur}!zzzSalten, Felix@\emph{von Felix Salten}!1908-07-142@{14. 7. 1908}|(be}
\toendnotes[C]{\smallbreak\pagebreak[2]}\Standort{CUL, Schnitzler, B 89, B 1.}
\physDesc{Bildpostkarte, 213 Zeichen
\newline{}Handschrift: schwarze Tinte, lateinische Kurrent
\newline{}Versand: Stempel: »\nobreak{}\oindex{Noordwijk@\textbf{Noordwijk}, \emph{A.ADM2}|pwk}Noordwijk \textsuperscript{a}/Zee 1, 15. 7\textcolor{gray}{.} 08, 1–2 N.\nobreak{}«.  
\newline{}Schnitzler: mit Bleistift Vermerk: »\textsc{Salten}« 
\newline{}Ordnung: mit Bleistift von unbekannter Hand nummeriert: »247« }\toendnotes[C]{\smallbreak}\pstart{}{\pb}\textcolor{pink}{Österreich}{}\ledrightnote{\textcolor{pink}{Österreich}}\pend{}\pstart{}\textcolor{pink}{Tirol}{}\ledrightnote{\textcolor{pink}{Tirol}}\pend{}\pstart{}Herrn D\textsuperscript{r} Arthur Schnitzler\pend{}\pstart{}\textcolor{pink}{Seis am Schlern}{}\ledrightnote{\textcolor{pink}{Seis am Schlern}}\pend{}\pstart{}\textcolor{pink}{Tirol}{}\ledrightnote{\textcolor{pink}{Tirol}}\pend{}
{\bigskip}
\pstart
           \noindent{}\centering{}{\pb}\textcolor{gray}{\textbf{\textcolor{pink}{Noordwyk}{}\ledrightnote{\textcolor{pink}{Noordwijk}} – Tulpen Velden}}\pend
           
\pstart
           Wir sind nun nach langen Fahrten endlich \textcolor{pink}{hier}{}\ledrightnote{{$\rightarrow$}\textcolor{pink}{Noordwijk}}, wo es bei gutem Wetter sehr angenehm sein kann. Schönste
               Grüße von Haus zu Haus\pend
           \pstart Ihr \spacefill\mbox{Salten}\pend{}
\pstart
           \textcolor{pink}{Noordwijk}{}\ledrightnote{\textcolor{pink}{Noordwijk}}, 14. 7. 08\pend
           \endnumbering\briefempfaengerindex{Schnitzler, Arthur@\textsc{Schnitzler, Arthur}!zzzSalten, Felix@\emph{von Felix Salten}!1908-07-142@{14. 7. 1908}|)be}\mylabel{h}  \normalsize

\doendnotes{C}
\bigskip
\vfill

\clearpage

\footnotesize

\lohead{\textsc{register}}

% Definiere theindex-Environment komplett neu ohne reledmac
\makeatletter
\renewenvironment{theindex}{%
  \section*{\indexname}%
  \setlength{\parindent}{0pt}%
  \setlength{\parskip}{0pt plus 0.3pt}%
  \let\item\@idxitem
}{%
  \clearpage
}
\makeatother

\IfFileExists{\jobname-pw.ind}{\input{\jobname-pw.ind}}{}

\end{document}

      