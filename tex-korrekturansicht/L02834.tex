%% latex-korrekturansicht-vorspann.tex
%% Vorspann für die Korrekturansicht.
%% Lädt die gemeinsame Datei latex-vorspann.tex mit gesetztem Schalter.

\newif\ifkorrekturansicht
\korrekturansichttrue

\input{../tex-inputs/latex-vorspann}


               \section[ Paul Goldmann an Arthur Schnitzler, 23. 12. {[}1897{]}]{Paul Goldmann an Arthur Schnitzler, 23. 12. {[}1897{]}}\nopagebreak\mylabel{v}\rehead{ }\normalsize\beginnumbering\briefempfaengerindex{Schnitzler, Arthur@\textsc{Schnitzler, Arthur}!zzzGoldmann, Paul@\emph{von Paul Goldmann}!1897-12-231@{23. 12. {[}1897{]}}|(be} \toendnotes[C]{\smallbreak\pagebreak[2]} \Standort{DLA, A:Schnitzler, HS.NZ85.1.3167.}
\physDesc{Brief, 3 Blätter, 9 Seiten
\newline{}Handschrift: blaue Tinte, deutsche Kurrent
\newline{}Schnitzler: 1) mit Bleistift das Jahr »97« vermerkt 2) mit rotem Buntstift fünf Unterstreichungen}\toendnotes[C]{\smallbreak}\pstart
           \noindent{}{\pb}\textcolor{gray}{\textbf{\textbf{\textcolor{brown}{Frankfurter Zeitung}{}\ledrightnote{\textcolor{brown}{Frankfurter Zeitung}}}}}\pend
           \pstart
           \textcolor{gray}{\textbf{(\textcolor{brown}{\begin{otherlanguage}{french}Gazette de Francfort\end{otherlanguage}}{}\ledrightnote{\textcolor{brown}{Frankfurter Zeitung}}).}}\pend
           \pstart
           \textcolor{gray}{\textbf{\textbf{\begin{otherlanguage}{french}Fondateur M.\end{otherlanguage}{ }\textcolor{blue}{L. Sonnemann}{}\ledrightnote{\textcolor{blue}{Leopold Sonnemann}}.}}}\pend
           \pstart
           \begin{otherlanguage}{french}\textcolor{gray}{\textbf{Journal politique, financier,}}\end{otherlanguage}\pend
           \pstart
           \begin{otherlanguage}{french}\textcolor{gray}{\textbf{commercial et littéraire.}}\end{otherlanguage}\pend
           \pstart
           \begin{otherlanguage}{french}\textcolor{gray}{\textbf{\textbf{Paraissant trois fois par jour.}}}\end{otherlanguage}\pend
           \pstart
           \begin{otherlanguage}{french}\textcolor{gray}{\textbf{\textbf{Bureau à \textcolor{pink}{Paris}{}\ledrightnote{\textcolor{pink}{Paris}}}}}\end{otherlanguage}\hfill \textsc{\textcolor{pink}{Paris}{}\ledrightnote{\textcolor{pink}{Paris}}}, 23. December.\pend
           \pstart
           \begin{otherlanguage}{french}\textcolor{gray}{\textbf{\textbf{\textcolor{pink}{10 Rue de la Bourse}{}\ledrightnote{\textcolor{pink}{rue de la Bourse}}.}}}\end{otherlanguage}\pend
           \pstart{}Frohe Weihnachten, liebſter Freund!\pend\pstart
           Mit Deinem \label{K_L02834-1v}\edtext{Auge}{\lemma{\textnormal{\emph{Auge}}}\Cendnote{\textnormal{Vermutlich spezifiziert das den Eintrag: »Neue
                     Hypochondrien« im \emph{\textcolor{green}{Tagebuch}} zum 21. 12. 1897.}}}\label{K_L02834-1h} geht es wohl beſſer? Dein letzter lieber Brief war recht verſtimmt.
               Freilich, mit einem Abſceß im Augenlid ſieht ſich das Leben nicht ſchön an.\pend
           \pstart
           Und doch hat mich Dein letzter Brief nachdenklich gemacht. Du darfſt mir nicht
               hypochondriſch werden! Und wenn es Dir ſchon im Ohre klingt! Muß man denn ganz geſund
               ſein?! Wer von uns iſt geſund? Man lebt und leidet eben. Iſt das nicht eine alte
               Geſchichte? Und lebt man deshalb weniger, weil man leidet? Eher mehr.\pend
           \pstart
           {\pb}Bei alledem glaube ich Dir Deine Krankheit gar
               nicht. Du haſt das, weil Dir, Gott ſei Dank, nichts Ernſtes fehlt. Du haſt viel Gutes
               und Herrliches ſchon genoſſen, Du biſt ein wenig abgeſtumpft geworden gegen all’ die
               ſchönen Dinge in Deinem Leben, das Errungene bildet darum kein rechtes Gegengewicht
               mehr gegen die Melancholie, die von Natur aus in dir wohnt, und ich glaube faſt, daß
               die Hypochondrie bei Dir eine Form der Blaſirtheit iſt.\pend
           \pstart
           Aufgeſchüttelt werden müßteſt Du, heraus müßteſt Du aus Deinem behaglichen \textcolor{pink}{Wien}{}\ledrightnote{\textcolor{pink}{Wien}}er Neſt, heraus in die Kälte, in die Fremde! Es
               iſt ganz natürlich, daß Du ſo, im gleichmäßigen {\pb}Weiterſchreiten, das Bewußthein der Kräfte verlierſt, die in Dir wohnen.\pend
           \pstart
           Wie darfſt Du ſagen, daß Du nicht an Deine Zukunft glaubſt?! Wer hat Zukunft, wenn
               nicht Du?! Nur muß die Zukunft von ſelbſt erwachſen, als natürliche Frucht einer
               kräftigen Gegenwart. Ruhig leben, ſeine Kraft ſtärken, ausreifen laſſen, was reifen
               ſoll, und keine Ungeduld! Wenn man natürlich ſich jeden Tag hinſetzt und ſeine
               Zukunft machen will, ſo geht es nicht. Auch hier gibt es \strikeout{e\textcolor{gray}{r}} eine pſychiſche Impotenz. Nein, ſei ruhig und Deiner ſelbſt ſicher (weiß Gott,
               Du kannſt es!), {\pb}wenn es mit \strikeout{d\textcolor{gray}{e}} dem Produciren nicht geht, ſo leg’ es ein wenig beiſeite, ſchaffe Dir ſchöne
               Tage, und laß’ aus Tagen und Tagen ganz unmerklich die Zukunft werden!{\dotsfour}\pend
           \pstart
           Übrigens, was rede ich? Wenn Du dieſen Brief bekommſt, biſt Du ſicherlich bereits in
               ganz anderer Stimmung, wie damals, wo Du mir \strikeout{d\textcolor{gray}{e}} den Brief ſchriebſt, der vor mir liegt.\pend
           \pstart
           Keiner von Deinen Briefen aus de\substVorne{}\textsuperscript{r}\substDazwischen{}n\substHinten{} letzten Monaten iſt mir \label{K_L02834-3v}\edtext{geſtohlen}{\lemma{\textnormal{\emph{geſtohlen}}}\Cendnote{\textnormal{siehe Paul Goldmann an Arthur Schnitzler, 10. 12. [1897]}}}\label{K_L02834-3h} worden. Sei ganz beruhigt! Es handelt ſich um einige wenige Briefe früheren
               Datums, in denen ſicher nichts Wichtiges oder beſonders Vertrauliches ſteht.\pend
           \pstart
           {\pb}Was iſt mit dem \label{K_L02834-4v}\edtext{\textcolor{brown}{Burgtheater}{}\ledrightnote{\textcolor{brown}{Burgtheater}}}{\lemma{\textnormal{\emph{Burgtheater}}}\Cendnote{\textnormal{\textcolor{blue}{Max Burckhard} trat als Direktor des
                     \emph{\textcolor{brown}{Burgtheater}}s zurück, weil seine Position
                  unhaltbar geworden war, nachdem er als Dramatiker an einem anderen Theater in Erscheinung trat.
                  Unter den potenziellen Nachfolgern fanden sich \textcolor{blue}{Heinrich Bulthaupt}, \textcolor{blue}{Emil Claar}, \textcolor{blue}{Jocza Savits} und \textcolor{blue}{Paul Schlenther}. Letztlich wurde \textcolor{blue}{Schlenther} am 25. 1. 1898
                  zum neuen Direktor bestimmt.}}}\label{K_L02834-4h}? Alſo hat es den \textsc{\textcolor{blue}{Burckhardt}{}\ledrightnote{\textcolor{blue}{Max Eugen Burckhard}}} doch \strikeout{er} ereilt? Ich wundere mich nur, daß ich
               nicht den \textsc{\textcolor{blue}{Bahr}{}\ledrightnote{\textcolor{blue}{Hermann Bahr}}} unter den \textcolor{blue}{\textcolor{brown}{Directions}{}\ledrightnote{→\textcolor{brown}{Burgtheater}}-Candidaten}{}\ledrightnote{→\textcolor{blue}{Alfred Heinrich Bulthaupt}{\newline}→\textcolor{blue}{Emil Claar}{\newline}→\textcolor{blue}{Jocza Savits}{\newline}→\textcolor{blue}{Paul Schlenther}}
               leſe. Der \textcolor{blue}{Kerl}{}\ledrightnote{→\textcolor{blue}{Hermann Bahr}} hat in \textcolor{pink}{Wien}{}\ledrightnote{\textcolor{pink}{Wien}}{ }\strikeout{d\textcolor{gray}{en}} den ſchlechten und faulen Boden gefunden, in dem allein er gedeihen konnte,
               und er gedeiht. Er wird großer \textsc{Pontifex} werden, und ich
               denke, \label{K_L02834-666v}\edtext{in ein paar Jahren}{\lemma{\textnormal{\emph{in ein paar Jahren}}}\Cendnote{\textnormal{Das war gewissermaßen prophetisch. \textcolor{blue}{Hermann Bahr}
                  wurde im September 1918 als Teil des Dreierkollegiums (gemeinsam mit \textcolor{blue}{Max Devrient} und \textcolor{blue}{Robert Michel}) erster Dramaturg des \emph{\textcolor{brown}{Burgtheater}}s. Vgl. A. S.: \emph{Tagebuch}, 20. 9. 1918: » Wer ihm’s prophezeit hätte – vor 25 Jahren – daß seine erste Amtshandlung im
                                 \textcolor{pink}{B. Th.}
                                 sein würde, des ›Kampfgenossen aus Jugendjahren‹ Stück – zu refusiren – weil dem
                                 \textcolor{blue}{Cardinal}
                                 die Aufführung peinlich sein könnte!–«}}}\label{K_L02834-666h} wird man ihm auch das \textcolor{brown}{Burgtheater}{}\ledrightnote{\textcolor{brown}{Burgtheater}} anbieten. Eines Tages werden dann vielleicht auch andere Leute
               entdecken, daß er ein unehrlicher und unverſtändiger Menſch iſt, aber dann wird es zu ſpät
               ſein.\pend
           \pstart
           {\pb}Dir ſollten ſie das \textcolor{brown}{Burgtheater}{}\ledrightnote{\textcolor{brown}{Burgtheater}} geben. Ich wüßte in der Welt keinen beſſeren Director. \textsc{\textcolor{blue}{Schlenther}{}\ledrightnote{\textcolor{blue}{Paul Schlenther}}}? Wäre das der \strikeout{\textcolor{gray}{×}} Richtige? Dieſer \textcolor{blue}{\textcolor{pink}{Berlin}{}\ledrightnote{\textcolor{pink}{Berlin}}er}{}\ledrightnote{→\textcolor{blue}{Paul Schlenther}} und \textcolor{blue}{Proteſtant}{}\ledrightnote{→\textcolor{blue}{Paul Schlenther}}, der wahrſcheinlich ein kluger
               Mann, aber ſicherlich ein kalter und \strikeout{\textcolor{gray}{un}k\textcolor{gray}{ü}nſ} unkünſtleriſcher Mann iſt?\pend
           \pstart
           Bitte, grüß’ mir Deine \textcolor{blue}{Freundin}{}\ledrightnote{→\textcolor{blue}{Marie Reinhard}} recht herzlich. Ich bringe es nicht fertig, ihr irgend etwas von
               meinen Arbeiten zu ſchicken. Ich weiß, daß das, was ich ſchreibe, der Vergeſſenheit
               verfallen iſt, und dieſes Bewußtſein lähmt mich ſo, daß ich nicht \strikeout{es} e\textcolor{gray}{i}nmal die Kraft habe, einen
               Artikel {\pb}herauszuſuchen und ihn auf die Poſt zu
               geben. Ich bin eben ein Journaliſt und nichts Anderes. Frage nur den Herrn 
               \textsc{\textcolor{blue}{Bahr}{}\ledrightnote{\textcolor{blue}{Hermann Bahr}}} und ſeine Bande,
               ſie werden es Dir ſchon ſagen.\pend
           \pstart
           Was macht \textsc{\textcolor{blue}{Richard}{}\ledrightnote{\textcolor{blue}{Richard Beer-Hofmann}}}? Iſt ſeine \label{K_L02834-32v}\edtext{\textcolor{green}{Novelle}{}\ledrightnote{→\textcolor{green}{Der Tod Georgs}} beendet}{\lemma{\textnormal{\emph{Novelle beendet}}}\Cendnote{\textnormal{\textcolor{blue}{Richard Beer-Hofmann} stellte \emph{\textcolor{green}{Der Tod Georgs}} erst Ende Juli 1899 fertig (vgl. Richard Beer-Hofmann an Arthur Schnitzler, 31. 7. 1899).}}}\label{K_L02834-32h}? Ich fürchte ſehr, daß es dem Helden einfallen könnte,
               zum Schluß noch von einem anderen Tempel zu träumen, und das würde dann wieder ein
               bis zwei Jahre dauern. Und \label{K_L02834-8v}\edtext{\textsc{\textcolor{blue}{Mirjam}{}\ledrightnote{\textcolor{blue}{Mirjam Beer-Hofmann}}}}{\lemma{\textnormal{\emph{Mirjam}}}\Cendnote{\textnormal{\textcolor{blue}{Beer-Hofmann}s dreieinhalb
                  Monate alte \textcolor{blue}{Tochter}}}}\label{K_L02834-8h}?{\dotsfour}\pend
           \pstart
           Ich habe arge Wochen durchgemacht und fürchterlich gelitten. Es iſt ſchlimm, \label{K_L02834-9v}\edtext{Beſchimpfungen}{\lemma{\textnormal{\emph{Beſchimpfungen}}}\Cendnote{\textnormal{siehe Paul Goldmann an Arthur Schnitzler, 10. 12. [1897]}}}\label{K_L02834-9h} ertragen zu müſſen, {\pb}ohne ſich wehren zu
               können, und zu fühlen, wie rings um Einen das Mißtrauen ſchleicht. Und dabei ganz
               allein, im fremden \strikeout{L\textcolor{gray}{an}}{ }\textcolor{pink}{Lande}{}\ledrightnote{→\textcolor{pink}{Frankreich}}, ohne Freund, ohne
               ermuthigenden Zuſpruch! Und nichts thun können, als einfach ruhig bei ſeiner
               Überzeugung bleiben. Man muß \strikeout{ſt\textcolor{gray}{i}llſ} ſtillſtehen und ſeine Pflicht thun, und in dieſer
               harten Pflichterfüllung iſt keinerlei Ruhe\strikeout{\textcolor{gray}{r}} zu holen. Nichts als Schläge, und bitterer Zweifel im Innern! Und doch, ich
               kann mich nicht entſchließen, jede Hoffnung aufzugeben. Auf der einen Seite die
               Wahrheit, auf der anderen Seite ein ganzes Volk. Es iſt nicht geſagt, {\pb}daß das Volk der ſtärkere Theil ſein
               muß.\pend
           \pstart
           Ich habe \textsc{\textcolor{pink}{Paris}{}\ledrightnote{\textcolor{pink}{Paris}}} ſatt über alle Maßen. Ich möchte ſo gerne fort, aber meine \textcolor{brown}{Zeitung}{}\ledrightnote{→\textcolor{brown}{Frankfurter Zeitung}} will \strikeout{\textcolor{gray}{m}} es bisher nicht zugeben. Es iſt ihnen ſo bequem, mich als \strikeout{\textcolor{gray}{Ar}} Arbeitsthier hier zu haben.\pend
           \pstart
           Nicht wahr, liebſter Freund, Du ſchreibſt mir bald?\pend
           \pstart
           Und nochmals von Herzen fröhliche Feiertage!\pend
           \pstart
           In Treue {\\[\baselineskip]}Dein {\\[\baselineskip]}\spacefill\mbox{Paul Goldmann}\pend
           \leftskip=0em{}\endnumbering\briefempfaengerindex{Schnitzler, Arthur@\textsc{Schnitzler, Arthur}!zzzGoldmann, Paul@\emph{von Paul Goldmann}!1897-12-231@{23. 12. {[}1897{]}}|)be}\mylabel{h}  \normalsize

\doendnotes{C}
\bigskip
\vfill

\clearpage

\footnotesize

\lohead{\textsc{register}}

% Definiere theindex-Environment komplett neu ohne reledmac
\makeatletter
\renewenvironment{theindex}{%
  \section*{\indexname}%
  \setlength{\parindent}{0pt}%
  \setlength{\parskip}{0pt plus 0.3pt}%
  \let\item\@idxitem
}{%
  \clearpage
}
\makeatother

\IfFileExists{\jobname-pw.ind}{\input{\jobname-pw.ind}}{}

\end{document}

      