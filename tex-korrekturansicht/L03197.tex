%% latex-korrekturansicht-vorspann.tex
%% Vorspann für die Korrekturansicht.
%% Lädt die gemeinsame Datei latex-vorspann.tex mit gesetztem Schalter.

\newif\ifkorrekturansicht
\korrekturansichttrue

\input{../tex-inputs/latex-vorspann}


\renewcommand{\erwaehntePersonen}{Personen: Ignacy Daszyński, Peter Dorner,  Leopold II. von Belgien,  Louise von Belgien, Fedor Mamroth, Géza von Mattachich, Olga Schnitzler, Elisabeth Steinrück, Irene Triesch, Émile Zola}
\renewcommand{\erwaehnteInstitutionen}{Institutionen: Die Zeit. Wiener Wochenschrift, Neue Freie Presse, Reichsrat}
\renewcommand{\erwaehnteOrte}{Orte: Berlin, Dessauer Straße, Deutsches Theater Berlin, Deutschland, Hoftheater Stuttgart, Kroatien, Mödling, Stuttgart, Wien, Österreich}
\renewcommand{\erwaehnteWerke}{Werke: Die Frau mit dem Dolche, Extrapost. Unparteiische Montags-Zeitung, Lebendige Stunden. Vier Einakter, Literatur, Politische Glossen. Der Ernst der Volksvertreter, Sehnsucht}
\section[ Paul Goldmann an Arthur Schnitzler, 10. 2. {[}1902{]}]{Paul Goldmann an Arthur Schnitzler, 10. 2. {[}1902{]}}
\nopagebreak\mylabel{v}
\rehead{ }\normalsize\beginnumbering\briefempfaengerindex{Schnitzler, Arthur@\textsc{Schnitzler, Arthur}!zzzGoldmann, Paul@\emph{von Paul Goldmann}!1902-02-101@{10. 2. {[}1902{]}}|(be}
\toendnotes[C]{\smallbreak\pagebreak[2]}\Standort{DLA, A:Schnitzler, HS.NZ85.1.3172.}
\physDesc{Brief, 1 Blatt, 3 Seiten
\newline{}Handschrift: blaue Tinte, deutsche Kurrent
\newline{}Beilage: ein Zeitungsausschnitt, beschnitten und eingeklebt 
\newline{}Schnitzler: 1) mit Bleistift das Jahr »{[}1{]}902« vermerkt  2) mit rotem Buntstift eine Unterstreichung}\toendnotes[C]{\smallbreak}
\pstart
           \noindent{}\raggedleft{}{\pb}\textcolor{pink}{\textcolor{gray}{\textbf{DESSAUERSTRASSE 19}}}{}\ledrightnote{\textcolor{pink}{Dessauer Straße}}\pend
           
\pstart
           \textcolor{pink}{Berlin}{}\ledrightnote{\textcolor{pink}{Berlin}}, 10. Februar.\pend
           
\pstart{}Mein lieber Freund,\pend
\pstart
           Wenn ich \textsc{Arthur Schnitzler} wäre, weißt Du, was ich thäte?
               Ich hätte den Ehrgeiz, nach all’ den ſchönen literariſchen Leiſtungen auch noch eine
               menſchlich große That zu vollbringen. Und würde mich darum an die Spitze einer
               Bewegung ſtellen, die zum Zweck hätte, den \label{K_L03197-1v}\edtext{Fall \textsc{\textcolor{blue}{Matassich-Keglevich}{}\ledrightnote{\textcolor{blue}{Géza von Mattachich}}}}{\lemma{\textnormal{\emph{Fall Matassich-Keglevich}}}\Cendnote{\textnormal{Oberstleutnant \textcolor{blue}{Géza von Mattachich} hatte ab 1895 eine intime
                  Beziehung mit \textcolor{blue}{Louise von Belgien}. Da sie
                  als älteste Tochter annehmen konnte, nach dem Tod ihres Vaters \textcolor{blue}{Leopold II. von Belgien} ein großes Vermögen zu erben,
                  lebte sie über ihre Verhältnisse und machte Schulden. Die beiden wurden im
                     Mai 1898 in \textcolor{pink}{Kroatien}
                  verhaftet und der Geldwechselfälschung beschuldigt. Während sie in eine
                  psychiatrische Verwahrung kam, wurde er zu sechs Jahren schwerem Kerker
                  verurteilt. Am 8. 2. 1902 hielt \textcolor{blue}{Ignacy Daszińsky} im \emph{\textcolor{brown}{Reichsrat}} eine Rede ([O. V.:] \emph{\textcolor{green}{Politische Glossen. Der Ernst der
                           Volksvertreter}}. In: \emph{\textcolor{green}{Extrapost.
                              Unparteiische Montags-Zeitung}}. Jg. 21, Nr. 1.045, 10. 2. 1902, S. 2.). Noch
                  im selben Monat wurde \textcolor{blue}{Mattachich} für unschuldig erklärt und
                  begnadigt.}}}\label{K_L03197-1h}, in dem ſicherlich ein gemeiner Juſtizmord verübt worden iſt,
               aufzuklären. \textsc{\textcolor{blue}{Zola}{}\ledrightnote{\textcolor{blue}{Émile Zola}}} gibt das große Vorbild. Ein \label{K_L03197-2v}\edtext{Artikel}{\lemma{\textnormal{\emph{Artikel}}}\Cendnote{\textnormal{\textcolor{blue}{Schnitzler} verzichtete nicht nur hier, sondern zeitlebens darauf, seinen Namen für
                  eine größere (kultur-)politische Kampagne zu verwenden.}}}\label{K_L03197-2h} in einem großen \textcolor{pink}{Wien}{}\ledrightnote{\textcolor{pink}{Wien}}er oder \textcolor{pink}{reichsdeutſch}{}\ledrightnote{{$\rightarrow$}\textcolor{pink}{Deutschland}}en Blatte mit Darlegung des ganzen Materials (das ſicherlich in
                  \textcolor{pink}{Wien}{}\ledrightnote{\textcolor{pink}{Wien}}{ }{\pb}zu bekommen iſt, wahrſcheinlich vom Abg. \textsc{\textcolor{blue}{Daszinsky}{}\ledrightnote{\textcolor{blue}{Ignacy Daszyński}}}), mit \textsc{Arthur Schnitzlers} klangvollem Namen
               unterzeichnet, würde die Bewegung einleiten und alle empfänglichen
                  He\textcolor{gray}{r}zen in \textcolor{pink}{Deutſchland}{}\ledrightnote{\textcolor{pink}{Deutschland}} und
                  \textcolor{pink}{Öſterreich}{}\ledrightnote{\textcolor{pink}{Österreich}} für den Fall intereſſiren.
               Vielleicht iſt die Sache in \textcolor{pink}{Wien}{}\ledrightnote{\textcolor{pink}{Wien}} mit der »\textcolor{brown}{Zeit}{}\ledrightnote{\textcolor{brown}{Die Zeit. Wiener Wochenschrift}}« zu machen. Vielleicht auch mit der \textcolor{brown}{N. Fr. Pr.}{}\ledrightnote{\textcolor{brown}{Neue Freie Presse}}\pend
           
\pstart
           Wie geht es \textsc{\textcolor{blue}{Olga}{}\ledrightnote{\textcolor{blue}{Olga Schnitzler}}}? Seid Ihr ſchon in \label{K_L03197-43v}\edtext{\textcolor{pink}{\textsc{Mödling}}{}\ledrightnote{\textcolor{pink}{Mödling}}}{\lemma{\textnormal{\emph{Mödling}}}\Cendnote{\textnormal{siehe Paul Goldmann an Arthur Schnitzler, 14. 1. [1902]}}}\label{K_L03197-43h}? Herzliche Grüße an die \textcolor{blue}{Mädels}{}\ledrightnote{{$\rightarrow$}\textcolor{blue}{Olga Schnitzler}{\newline}{$\rightarrow$}\textcolor{blue}{Elisabeth Steinrück}}!\pend
           
\pstart
           Ich habe unbeſchreiblich viel zu thun.\pend
           
\pstart
           Dank für Deinen letzten lieben Brief! {\\[\baselineskip]}Viele treue Grüße! {\\[\baselineskip]}Dein
                  \spacefill\mbox{Paul Goldm}\pend
           \leftskip=0em{}
\pstart
           \noindent{}{\pb}Das \textcolor{green}{Stück}{}\ledrightnote{{$\rightarrow$}\textcolor{green}{Sehnsucht}} meines \textcolor{blue}{Onkel}{}\ledrightnote{{$\rightarrow$}\textcolor{blue}{Fedor Mamroth}}s, das unter dem Namen »\textcolor{green}{Sehnſucht}{}\ledrightnote{\textcolor{green}{Sehnsucht}}« in \label{K_L03197-6v}\edtext{\textcolor{pink}{Stuttgart}{}\ledrightnote{\textcolor{pink}{Stuttgart}}}{\lemma{\textnormal{\emph{Stuttgart}}}\Cendnote{\textnormal{Am 4. 2. 1902 wurde \textcolor{blue}{Fedor
                     Mamroth}s vieraktige Komödie \emph{\textcolor{green}{Sehnsucht}} (unter dem Pseudonym \textcolor{blue}{F. Albert}) am \textcolor{pink}{Stuttgarter
                        Hoftheater} uraufgeführt.}}}\label{K_L03197-6h} aufgeführt wurde, hatte dort einen ſehr
                  ſchönen Erfolg.\pend
           
\pstart
           Wie hat ſich die \label{K_L03197-34v}\edtext{Angelegenheit \textsc{\textcolor{blue}{Peter Dorner}{}\ledrightnote{\textcolor{blue}{Peter Dorner}}}}{\lemma{\textnormal{\emph{Angelegenheit Peter Dorner}}}\Cendnote{\textnormal{siehe Paul Goldmann an Arthur Schnitzler, 23. 9. [1901]}}}\label{K_L03197-34h} noch entwickelt?\pend
           {\bigskip}
\pstart
           \noindent{}\textcolor{gray}{\textbf{\textbf{– }\label{K_L03197-32v}\edtext{\textbf{Arthur Schnitzler’s »}\textcolor{green}{\textbf{Lebendige Stunden}}{}\ledrightnote{\textcolor{green}{Lebendige Stunden. Vier Einakter}}\textbf{«,} die bisher in zwanzig Wiederholungen bei unverminderter
                     Zugkraft im \textcolor{pink}{\so{Deutſchen Theater}}{}\ledrightnote{\textcolor{pink}{Deutsches Theater Berlin}} in Szene gingen, können in den folgenden Wochen nur je einmal auf dem
                     Spielplan erſcheinen, da \textcolor{blue}{\so{Irene Trieſch}}{}\ledrightnote{\textcolor{blue}{Irene Triesch}} einen kontraktlichen Urlaub angetreten hat, jedoch allwöchentlich einmal,
                     zunächſt am Mittwoch, den 12., nach \textcolor{pink}{Berlin}{}\ledrightnote{\textcolor{pink}{Berlin}} zurückkehren wird, um die von ihr in
                     den »\textcolor{green}{Lebendigen Stunden}{}\ledrightnote{\textcolor{green}{Lebendige Stunden. Vier Einakter}}« geſpielten beiden
                     weiblichen Hauptrollen weiterhin darzuſtellen.}{\lemma{\textnormal{\emph{Arthur … darzuſtellen.}}}\Cendnote{\textnormal{Quelle nicht ermittelt; in \emph{\textcolor{green}{Die Frau mit dem Dolche}} spielte \textcolor{blue}{Irene Triesch} die Rolle der \textcolor{green}{Pauline} und in \emph{\textcolor{green}{Literatur}} jene der \textcolor{green}{Margarete}.}}}\label{K_L03197-32h}}}\pend
           \endnumbering\briefempfaengerindex{Schnitzler, Arthur@\textsc{Schnitzler, Arthur}!zzzGoldmann, Paul@\emph{von Paul Goldmann}!1902-02-101@{10. 2. {[}1902{]}}|)be}\mylabel{h}  \normalsize

\doendnotes{C}
\bigskip
\vfill

\clearpage

\footnotesize

\lohead{\textsc{register}}

% Definiere theindex-Environment komplett neu ohne reledmac
\makeatletter
\renewenvironment{theindex}{%
  \section*{\indexname}%
  \setlength{\parindent}{0pt}%
  \setlength{\parskip}{0pt plus 0.3pt}%
  \let\item\@idxitem
}{%
  \clearpage
}
\makeatother

\IfFileExists{\jobname-pw.ind}{\input{\jobname-pw.ind}}{}

\end{document}

      