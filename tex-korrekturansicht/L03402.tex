%% latex-korrekturansicht-vorspann.tex
%% Vorspann für die Korrekturansicht.
%% Lädt die gemeinsame Datei latex-vorspann.tex mit gesetztem Schalter.

\newif\ifkorrekturansicht
\korrekturansichttrue

\input{../tex-inputs/latex-vorspann}


\renewcommand{\erwaehntePersonen}{Personen:  Donatello, Gustav Mahler, Ottilie Salten}
\renewcommand{\erwaehnteOrte}{Orte: Musikverein, Riedhof, Wien}
\renewcommand{\erwaehnteWerke}{Werke: ?? [Gipsnachbildung einer Statue von Donatello], Symphonie Nr. 3 D-Moll}
\section[ Felix Salten an Arthur Schnitzler, {[}20. 12. 1904{]}]{Felix Salten an Arthur Schnitzler, {[}20. 12. 1904{]}}
\nopagebreak\mylabel{v}
\rehead{ }\normalsize\beginnumbering\briefempfaengerindex{Schnitzler, Arthur@\textsc{Schnitzler, Arthur}!zzzSalten, Felix@\emph{von Felix Salten}!1904-12-201@{{[}20. 12. 1904{]}}|(be}
\toendnotes[C]{\smallbreak\pagebreak[2]}\Standort{CUL, Schnitzler, B 89, B 1.}
\physDesc{Brief, 1 Blatt, 1 Seite, 236 Zeichen
\newline{}Handschrift: schwarze Tinte, lateinische Kurrent
\newline{}Schnitzler: mit Bleistift datiert: »20/12 904« 
\newline{}Ordnung: mit Bleistift von unbekannter Hand nummeriert: »195« }\toendnotes[C]{\smallbreak}
\pstart
           \raggedleft{}{\pb}Dienstag\pend
           
\pstart
           Lieber, für den überraschenden und prächtigen \textcolor{blue}{\textcolor{green}{\label{K_L03402-1v}\edtext{Donatello}{\lemma{\textnormal{\emph{Donatello}}}\Cendnote{\textnormal{siehe A. S.: \emph{Tagebuch}, 9. 12. 1904}}}\label{K_L03402-1h}}{}\ledrightnote{{$\rightarrow$}\textcolor{green}{?? [Gipsnachbildung einer Statue von Donatello]}}}{}\ledrightnote{\textcolor{blue}{Donatello}} bedanken \textcolor{blue}{wir}{}\ledrightnote{{$\rightarrow$}\textcolor{blue}{Ottilie Salten}} uns
               herzlich und erfreut.\pend
           
\pstart
           Wir sind auch beim \label{K_L03402-2v}\edtext{\textcolor{blue}{Mahler}{}\ledrightnote{\textcolor{blue}{Gustav Mahler}}-Conzert}{\lemma{\textnormal{\emph{Mahler-Conzert}}}\Cendnote{\textnormal{Am 22. 12. 1904 wurde die \emph{\textcolor{green}{3. Sinfonie in
                     d-Moll}} im \textcolor{pink}{Großen Musikvereinssaal}
                  gegeben. Wie aus den folgenden Briefen hervorgeht, verpassten sie sich im \textcolor{pink}{Riedhof}.}}}\label{K_L03402-2h}, und könnten dann ev. zusammen
               in den \textcolor{pink}{Riedhof}{}\ledrightnote{\textcolor{pink}{Riedhof}}, jedesfalls aber uns dort nachher
               treffen.\pend
           
\pstart
           Herzlichst {\\[\baselineskip]}Ihr {\\[\baselineskip]}\spacefill\mbox{Salten}\pend
           \leftskip=0em{}\endnumbering\briefempfaengerindex{Schnitzler, Arthur@\textsc{Schnitzler, Arthur}!zzzSalten, Felix@\emph{von Felix Salten}!1904-12-201@{{[}20. 12. 1904{]}}|)be}\mylabel{h}  \normalsize

\doendnotes{C}
\bigskip
\vfill

\clearpage

\footnotesize

\lohead{\textsc{register}}

% Definiere theindex-Environment komplett neu ohne reledmac
\makeatletter
\renewenvironment{theindex}{%
  \section*{\indexname}%
  \setlength{\parindent}{0pt}%
  \setlength{\parskip}{0pt plus 0.3pt}%
  \let\item\@idxitem
}{%
  \clearpage
}
\makeatother

\IfFileExists{\jobname-pw.ind}{\input{\jobname-pw.ind}}{}

\end{document}

      