%% latex-korrekturansicht-vorspann.tex
%% Vorspann für die Korrekturansicht.
%% Lädt die gemeinsame Datei latex-vorspann.tex mit gesetztem Schalter.

\newif\ifkorrekturansicht
\korrekturansichttrue

\input{../tex-inputs/latex-vorspann}


               \section[Arthur Schnitzler an Bertha von Suttner, 17. 12. 1913]{ Arthur Schnitzler an Bertha von Suttner,
                    17. 12. 1913}\nopagebreak\mylabel{v}\rehead{ }\normalsize\beginnumbering\briefempfaengerindex{Suttner, Bertha von@\textsc{Suttner, Bertha von}!zzzSchnitzler, Arthur@\emph{von Arthur Schnitzler}!1913-12-171@{17. 12. 1913}|(be} \toendnotes[C]{\smallbreak\pagebreak[2]} \Standort{Genf, United Nations Archives, BvS/27/352-1.}
\physDesc{Briefkarte
\newline{}Handschrift: schwarze Tinte, deutsche Kurrent\newline{}Ordnung: mit Bleistift von unbekannter Hand beschriftet: »X15« }\toendnotes[C]{\smallbreak}\pstart
           \noindent{}{\pb}\textcolor{gray}{\textbf{Dr. Arthur Schnitzler}}\hfill 1\substVorne{}\textsuperscript{\textcolor{gray}{9}}\substDazwischen{}7\substHinten{}. 12. 913.\pend
           \pstart
           \textcolor{gray}{\textbf{\textcolor{pink}{Wien XVIII. Sternwartestrasse 71}{}\ledrightnote{\textcolor{pink}{Sternwartestraße}}}}\pend
           \pstart{}verehrte Frau Baronin, \pend\pstart
           dürften wir am \uline{Montag} 22., \uline{gegen 6 Uhr} Nm unſern lange gehegten Wunſch verwirklichen und bei Ihnen
                    vorſprechen?\pend
           \pstart
           mit den ergebenſten Grüßen, auch von meiner \textcolor{blue}{Frau}{}\ledrightnote{→\textcolor{blue}{Olga Schnitzler}}{\\[\baselineskip]}Ihr Sie wahrhaft verehrender{\\[\baselineskip]}\spacefill\mbox{Arthur Schnitzler}\pend
           \leftskip=0em{}\endnumbering\briefempfaengerindex{Suttner, Bertha von@\textsc{Suttner, Bertha von}!zzzSchnitzler, Arthur@\emph{von Arthur Schnitzler}!1913-12-171@{17. 12. 1913}|)be}\mylabel{h}  \normalsize

\doendnotes{C}
\bigskip
\vfill

\clearpage

\footnotesize

\lohead{\textsc{register}}

% Definiere theindex-Environment komplett neu ohne reledmac
\makeatletter
\renewenvironment{theindex}{%
  \section*{\indexname}%
  \setlength{\parindent}{0pt}%
  \setlength{\parskip}{0pt plus 0.3pt}%
  \let\item\@idxitem
}{%
  \clearpage
}
\makeatother

\IfFileExists{\jobname-pw.ind}{\input{\jobname-pw.ind}}{}

\end{document}

      