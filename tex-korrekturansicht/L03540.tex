%% latex-korrekturansicht-vorspann.tex
%% Vorspann für die Korrekturansicht.
%% Lädt die gemeinsame Datei latex-vorspann.tex mit gesetztem Schalter.

\newif\ifkorrekturansicht
\korrekturansichttrue

\input{../tex-inputs/latex-vorspann}


\renewcommand{\erwaehntePersonen}{Personen: Paul Goldmann, Dolly Landsberger, Louise Schnitzler, Gertrud Wertheim, Wolf Waldemar Wertheim, Gisbert von Wolff-Metternich}
\renewcommand{\erwaehnteInstitutionen}{Institutionen: Vossische Zeitung}
\renewcommand{\erwaehnteOrte}{Orte: Berlin, Bozen, Schöneberger Ufer, Wien}
\renewcommand{\erwaehnteWerke}{Werke: Der Prozeß gegen den Grafen Wolff-Metternich. Eine Erklärung der Frau Wertheim, Neues Wiener Journal, [Leserbrief von Gertrud Wertheim mit Klage über Paul Goldmann]}
\section[ Eva Marie Goldmann an Arthur Schnitzler, 21. 9. 1911]{Eva Marie Goldmann an Arthur Schnitzler, 21. 9. 1911}
\nopagebreak\mylabel{v}
\rehead{ }\normalsize\beginnumbering\briefempfaengerindex{Schnitzler, Arthur@\textsc{Schnitzler, Arthur}!zzzGoldmann, Eva Marie@\emph{von Eva Marie Goldmann}!1911-09-211@{21. 9. 1911}|(be}
\toendnotes[C]{\smallbreak\pagebreak[2]}\Standort{DLA, A:Schnitzler, HS.NZ85.1.3160.}
\physDesc{Brief, 1 Blatt, 3 Seiten, 725 Zeichen
\newline{}Handschrift: , lateinische Kurrent
\newline{}Schnitzler: mit Bleistift Vermerk »\textcolor{blue}{Goldmann}.«
                               }\toendnotes[C]{\smallbreak}
\pstart
           \raggedleft{}{\pb}21. IX. 1911.\pend
           
\pstart
           \textcolor{gray}{\textbf{EG}}\hfill \textcolor{gray}{\textbf{\textcolor{pink}{W. SCHÖNEBERGER-UFER 34}{}\ledrightnote{\textcolor{pink}{Schöneberger Ufer}}.}}\pend
           
\pstart{}Verehrter Herr Doctor, \pend
\pstart
           ich danke Ihnen vielmals für Ihre \label{K_L03540-1v}\edtext{freundlichen Zeilen}{\lemma{\textnormal{\emph{freundlichen Zeilen}}}\Cendnote{\textnormal{Es ist anzunehmen,
                  dass auch sie \textcolor{blue}{Schnitzler} anlässlich des
                  Todes seiner \textcolor{blue}{Mutter}
                  kondoliert hatte.}}}\label{K_L03540-1h}.\pend
           
\pstart
           Und ich möchte Ihnen sagen, dass \textcolor{blue}{Paul}{}\ledrightnote{\textcolor{blue}{Paul Goldmann}} unter
               dem \label{K_L03540-2v}\edtext{Zerwürfnis}{\lemma{\textnormal{\emph{Zerwürfnis}}}\Cendnote{\textnormal{Ende 1910/Anfang 1911, siehe insbesondere Paul Goldmann an Arthur Schnitzler, 13. 1. 1911 und Arthur Schnitzler an Paul Goldmann, 1. 2. 1911}}}\label{K_L03540-2h} mit Ihnen sehr \label{K_L03540-3v}\edtext{gelitten}{\lemma{\textnormal{\emph{gelitten}}}\Cendnote{\textnormal{vgl. A. S.: \emph{Tagebuch}, 23. 9. 1911}}}\label{K_L03540-3h} hat.\pend
           
\pstart
           In seinem, und auch {\pb}schon in meinem Alter, kommt kein
               Ersatz mehr für das, was einem genommen wird, was einem theuer war und ein Stück
               Jugend bedeutet hat.\pend
           
\pstart
           Ich würde Sie, verehrter Herr Doctor, gerne einmal wieder sprechen, und ich bilde mir
               ein, dass alles anders gekommen wäre, wenn ich im vergangenen Winter mit in \textcolor{pink}{Wien}{}\ledrightnote{\textcolor{pink}{Wien}} gewesen wäre.\pend
           
\pstart
           \label{K_L03540-4v}\edtext{\textcolor{blue}{Paul}{}\ledrightnote{\textcolor{blue}{Paul Goldmann}}{ }\uline{ahnt nicht}}{\lemma{\textnormal{\emph{Paul ahnt nicht}}}\Cendnote{\textnormal{Ob \textcolor{blue}{Schnitzler} auf
                  diesen Brief antwortete, ist nicht überliefert. vgl. A. S.: \emph{Tagebuch}, 5. 10. 1911.}}}\label{K_L03540-4h}, dass ich Ihnen heute schreibe, u. wird es auch nicht erfahren.\pend
           
\pstart
           Er ist augenblicklich nicht \textcolor{pink}{hier}{}\ledrightnote{{$\rightarrow$}\textcolor{pink}{Berlin}}, sondern wegen eines widerwärtigen \label{K_L03540-5v}\edtext{Processes in \textcolor{pink}{Wien}{}\ledrightnote{\textcolor{pink}{Wien}}}{\lemma{\textnormal{\emph{Processes in Wien}}}\Cendnote{\textnormal{Wieso \textcolor{blue}{Goldmann} wegen dieses Prozesses in \textcolor{pink}{Wien} weilte, ist unklar. Es dürfte sich aber um eine Seitenaffäre einer
                  sehr beachteten Ankage gegen Graf \textcolor{blue}{Gisbert von
                     Wolff-Metternich} handeln. Dieser war bereits mehrfach des Betrugs
                  beschuldigt worden und hatte hohe Schulden. Einen Prozess wollte er abwehren,
                  indem er angab, die wohlhabende \textcolor{blue}{Dolly
                     Landsberger} zu heiraten. Deren Mutter \textcolor{blue}{Gertrud Wertheim} (bekannt als Schriftstellerin \textcolor{blue}{Truth}) trat als Belastungszeugin gegen ihn auf, wodurch
                  auch ihr »Vorleben« als Dichterin und das Erbe aus erster Ehe für die Verteidigung
                  und in Folge die Klatschpresse interessant wurde. \textcolor{blue}{Goldmann} dürfte sich \textcolor{pink}{Bozen} dafür stark gemacht haben, dass weder \textcolor{blue}{Landsberger} noch ihre \textcolor{blue}{Mutter} an dem Prozess teilnahmen: »\textcolor{green}{Frau \textcolor{blue}{\so{Wertheim}} hat der ›\emph{\textcolor{brown}{Voss. Ztg.}}‹ einen \textcolor{green}{Brief} zugeſchickt, in
                     dem ſie gegen den \textcolor{pink}{Berlin}er Schriftſteller
                        \textcolor{blue}{\so{Goldmann}} den Vorwurf erhebt, ſich in eine ihm ganz fremde Angelegenheit unbefugt
                     eingemengt zu haben. Er habe in \textcolor{pink}{Bozen}
                     ihren \textcolor{blue}{Mann}, ihre \textcolor{blue}{Tochter} und ſie ſelbſt
                     in der intenſivſten Art und Weiſe beſchworen, daß Frau \textcolor{blue}{Wertheim} nicht zum \textcolor{blue}{Metternich}-Prozeß fahre. ›Er malte jedem einzelnen‹, heißt es in dem
                        \textcolor{green}{Brief}, ›in den
                     düſterſten Farben mein bevorſtehendes Geſchick aus und ſagte weiter, aus der
                        \textcolor{blue}{Zeugin} würde eine
                     Angeklagte werden. Eine Kataſtrophe würde eintreten. Da wir alle Herrn \textcolor{blue}{Paul Goldmann} nur ganz flüchtig kennen,
                     erregte ſeine Art und Weiſe begreifliche Verwunderung. Er gab mir ſogar den
                     Rat, mich, die ich damals noch heiter und vergnügt war, durch ärztliche Atteſte
                     zu ſchützen. Dieſes gewiß befremdende Benehmen konnte ich mir nur dadurch
                     erklären, daß Herr \textcolor{blue}{Goldmann} von
                     irgendeiner Seite beauftragt war. Denn ein derartiges Eingreifen würde
                     höchſtens bei Freunden oder ſonſt Nächſtſtehenden zu erklären oder zu
                     entſchuldigen sein.{[}‹{]}}« ([O. V.]: \emph{\textcolor{green}{Der Prozeß gegen den
                        Grafen Wolff-Metternich. Eine Erklärung der Frau Wertheim}}. In: \emph{\textcolor{green}{Neues Wiener Journal}}, Jg. 19, Nr. 6.454,
                        10. 10. 1911, S. 9)}}}\label{K_L03540-5h}.\pend
           
\pstart
           Mit den besten Wünschen für {\pb}Sie u. die Ihren
               {\\[\baselineskip]}Ihre {\\[\baselineskip]}\spacefill\mbox{EvaMarieGoldmann.}\pend
           \leftskip=0em{}\endnumbering\briefempfaengerindex{Schnitzler, Arthur@\textsc{Schnitzler, Arthur}!zzzGoldmann, Eva Marie@\emph{von Eva Marie Goldmann}!1911-09-211@{21. 9. 1911}|)be}\mylabel{h}  \normalsize

\doendnotes{C}
\bigskip
\vfill

\clearpage

\footnotesize

\lohead{\textsc{register}}

% Definiere theindex-Environment komplett neu ohne reledmac
\makeatletter
\renewenvironment{theindex}{%
  \section*{\indexname}%
  \setlength{\parindent}{0pt}%
  \setlength{\parskip}{0pt plus 0.3pt}%
  \let\item\@idxitem
}{%
  \clearpage
}
\makeatother

\IfFileExists{\jobname-pw.ind}{\input{\jobname-pw.ind}}{}

\end{document}

      