%% latex-korrekturansicht-vorspann.tex
%% Vorspann für die Korrekturansicht.
%% Lädt die gemeinsame Datei latex-vorspann.tex mit gesetztem Schalter.

\newif\ifkorrekturansicht
\korrekturansichttrue

\input{../tex-inputs/latex-vorspann}


\renewcommand{\erwaehntePersonen}{Personen: Winston Churchill, Richard Haldane, David Lloyd George, Max Meyerfeld, Anna Katharina Rehmann, Felix Salten, Ottilie Salten, Paul Salten, Julius Ferdinand Wollf}
\renewcommand{\erwaehnteInstitutionen}{Institutionen: Deutsch-amerikanische Seepost, Kronprinz Wilhelm, Norddeutscher Lloyd}
\renewcommand{\erwaehnteOrte}{Orte: Bansin, Berlin, Bremerhaven, Cambridge, Dänemark, Edmund-Weiß-Gasse 7, England, Heringsdorf, Klopeiner See, London, Marienlyst, Nordsee, Southampton, Stratford-upon-Avon, Wien, Österreich}
\renewcommand{\erwaehnteWerke}{Werke: Erinnerungen}
\section[ Felix Salten an Arthur Schnitzler, 19. 6. 1906]{Felix Salten an Arthur Schnitzler, 19. 6. 1906}
\nopagebreak\mylabel{v}
\rehead{ }\normalsize\beginnumbering\briefempfaengerindex{Schnitzler, Arthur@\textsc{Schnitzler, Arthur}!zzzSalten, Felix@\emph{von Felix Salten}!1906-06-191@{19. 6. 1906}|(be}
\toendnotes[C]{\smallbreak\pagebreak[2]}\Standort{CUL, Schnitzler, B 89, B 1.}
\physDesc{Bildpostkarte, 292 Zeichen
\newline{}Handschrift: schwarze Tinte, lateinische Kurrent
\newline{}Versand: Stempel: »\nobreak{}2\textcolor{gray}{0}. 6. 06, \textcolor{brown}{Deutsch-amerikanische Seepost
                                          Bremen–New York}\nobreak{}«.  
\newline{}Ordnung: mit Bleistift von unbekannter Hand nummeriert: »218« }\toendnotes[C]{\smallbreak}\pstart{}{\pb}Herrn D\textsuperscript{r} Arthur Schnitzler\pend{}\pstart{}\textcolor{pink}{Wien}{}\ledrightnote{\textcolor{pink}{Wien}}\pend{}\pstart{}\textcolor{pink}{XVIII. Spöttelgasse 7}{}\ledrightnote{\textcolor{pink}{Edmund-Weiß-Gasse 7}}\pend{}\pstart{}\textcolor{pink}{Österreich}{}\ledrightnote{\textcolor{pink}{Österreich}}.\pend{}
{\bigskip}
\pstart
           \noindent{}{\pb}\textcolor{gray}{\textbf{\textcolor{brown}{Nordd. LLoyd}{}\ledrightnote{\textcolor{brown}{Norddeutscher Lloyd}}. »\textcolor{brown}{Kronprinz Wilhelm}{}\ledrightnote{\textcolor{brown}{Kronprinz Wilhelm}}«.}}\hfill \textcolor{gray}{\textbf{Rauchsalon I. Klasse.}}\pend
           
\pstart
           ebenda. 19. VI. 06.\pend
           
\pstart
           Lieber,{ }\uline{so} sieht nun die \label{K_L03427-1v}\edtext{Radpartie}{\lemma{\textnormal{\emph{Radpartie}}}\Cendnote{\textnormal{siehe Felix Salten an Arthur Schnitzler, 28. 3. 1906}}}\label{K_L03427-1h} und der \textcolor{pink}{Klopeiner See}{}\ledrightnote{\textcolor{pink}{Klopeiner See}} aus. Ich gehe
                  auf \label{K_L03427-2v}\edtext{14 Tage nach \textcolor{pink}{England}{}\ledrightnote{\textcolor{pink}{England}}}{\lemma{\textnormal{\emph{14 Tage nach England}}}\Cendnote{\textnormal{Er war
                  beruflich unterwegs. In seinen \emph{\textcolor{green}{Erinnerungen}} (\emph{Wienbibliothek im Rathaus}, Nachlass \textcolor{blue}{Salten}, ZPH 1681/1 1.1.1.9.1, [S. 19–20])
                  schildert \textcolor{blue}{Salten} eine offizielle Reise mit
                  anderen Journalisten (\textcolor{blue}{Julius Ferdinand
                     Wollf} und \textcolor{blue}{Max Meyerfeld}) über \textcolor{pink}{Bremerhaven} nach \textcolor{pink}{Southampton} und weiter nach \textcolor{pink}{London}. Dort will er mit \textcolor{blue}{Winston
                     Churchill}, \textcolor{blue}{David Lloyd George} und
                     \textcolor{blue}{Richard Haldane} gesprochen haben, die
                  damals alle amtierende Minister waren. Die weiteren Stationen (\textcolor{pink}{Stratford-upon-Avon} und \textcolor{pink}{Cambridge}) decken sich mit den Postkarten, die er am 23. 6. 1906 und am 27. 6. 1906 an \textcolor{blue}{Schnitzler} schreibt.}}}\label{K_L03427-2h}. \textcolor{blue}{Otti}{}\ledrightnote{\textcolor{blue}{Ottilie Salten}} ist mit den \textcolor{blue}{Kinder}{}\ledrightnote{{$\rightarrow$}\textcolor{blue}{Anna Katharina Rehmann}{\newline}{$\rightarrow$}\textcolor{blue}{Paul Salten}}n in \textcolor{pink}{Bansin}{}\ledrightnote{\textcolor{pink}{Bansin}}, bei \textcolor{pink}{Heringsdorf}{}\ledrightnote{\textcolor{pink}{Heringsdorf}}. Vielleicht \label{K_L03427-3v}\edtext{sehen
               wir uns, wenn Sie nach \textcolor{pink}{Dänemark}{}\ledrightnote{\textcolor{pink}{Dänemark}} fahren}{\lemma{\textnormal{\emph{sehen … fahren}}}\Cendnote{\textnormal{Am Weg nach \textcolor{pink}{Dänemark} (Ende Juni) sahen sie
                  sich nicht, da \textcolor{blue}{Salten} während \textcolor{blue}{Schnitzler}s \textcolor{pink}{Berlin}-Aufenthalt nicht vor Ort war (vgl. Felix Salten an Arthur Schnitzler, 6. 7. 1906). In \textcolor{pink}{Marienlyst} sahen sie sich am 2. 8. 1906.}}}\label{K_L03427-3h}. Herzlichst Ihr {\\}\spacefill\mbox{Salten}\pend
           \endnumbering\briefempfaengerindex{Schnitzler, Arthur@\textsc{Schnitzler, Arthur}!zzzSalten, Felix@\emph{von Felix Salten}!1906-06-191@{19. 6. 1906}|)be}\mylabel{h}  \normalsize

\doendnotes{C}
\bigskip
\vfill

\clearpage

\footnotesize

\lohead{\textsc{register}}

% Definiere theindex-Environment komplett neu ohne reledmac
\makeatletter
\renewenvironment{theindex}{%
  \section*{\indexname}%
  \setlength{\parindent}{0pt}%
  \setlength{\parskip}{0pt plus 0.3pt}%
  \let\item\@idxitem
}{%
  \clearpage
}
\makeatother

\IfFileExists{\jobname-pw.ind}{\input{\jobname-pw.ind}}{}

\end{document}

      