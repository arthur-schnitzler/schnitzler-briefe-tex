%% latex-korrekturansicht-vorspann.tex
%% Vorspann für die Korrekturansicht.
%% Lädt die gemeinsame Datei latex-vorspann.tex mit gesetztem Schalter.

\newif\ifkorrekturansicht
\korrekturansichttrue

\input{../tex-inputs/latex-vorspann}


               \section[Hugo von Hofmannsthal an Arthur Schnitzler, 15. 7. 1900]{ Hugo von Hofmannsthal an Arthur Schnitzler, 15. 7. 1900}\nopagebreak\mylabel{v}\rehead{ }\normalsize\beginnumbering\briefempfaengerindex{Schnitzler, Arthur@\textsc{Schnitzler, Arthur}!zzzHofmannsthal, Hugo von@\emph{von Hugo von Hofmannsthal}!1900-07-152@{15. 7. 1900}|(be} \toendnotes[C]{\smallbreak\pagebreak[2]} \Standort{CUL, Schnitzler, B 43.}
\physDesc{Brief, 1 Blatt, 4 Seiten
\newline{}Handschrift: schwarze Tinte, deutsche Kurrent
\newline{}Schnitzler: mit Bleistift das Datum vervollständigt: »/7 900« \newline{}Ordnung: 1) mit Bleistift von unbekannter Hand nummeriert: »\strikeout{164}« 2) mit Bleistift von unbekannter Hand nummeriert: »163«}\buchAbdrucke{\weitereDrucke{Hugo von Hofmannsthal, Arthur Schnitzler: \emph{Briefwechsel}. Hg. Therese Nickl und Heinrich Schnitzler. Frankfurt am Main: \emph{S. Fischer} 1964, S. 140.} }\toendnotes[C]{\smallbreak}\pstart
           \raggedleft{}{\pb}\textcolor{pink}{Bad Fuſch}{}\ledrightnote{\textcolor{pink}{Bad Fusch}}{ }15\textsuperscript{ten}\pend
           \pstart{}mein guter lieber Arthur\pend\pstart
           wie die Dinge einmal eigenſinnig und unbegreiflich ſind, finde ich hier, in
               vollkommener Ruhe, bei unverſtörten äußern Umſtänden ſeit 14 Tagen nicht nur nicht
               die leiſeſte Möglichkeit des Arbeitens, ſondern ich verſinke auch in eine ſolche
               Verdroſſenheit, ſolche Gelähmtheit aller inneren Sinne, {\pb}daſs ich ein Buch nach dem andern
               aus der Hand lege und weder am Morgen noch am Abend die geringſte Freude habe. Nun
               iſt mir vor 2 Stunden eingefallen, es mit einem Ausflug zu verſuchen. Wie ſchön, wenn
               man in ſolchen Momenten nicht ſo weit auseinander wäre! Auch mein Rad iſt in der {\pb}\textcolor{pink}{Brühl}{}\ledrightnote{\textcolor{pink}{Brühl}}, ich will nicht abwarten, bis es herkäme,
               ſondern fahre gleich nach \textcolor{pink}{\textsc{Saalfelden}}{}\ledrightnote{\textcolor{pink}{Saalfelden am Steinernen Meer}}, von dort mit der Post an den \textcolor{pink}{\textsc{Hintersee}}{}\ledrightnote{\textcolor{pink}{Hintersee}}, wo es ſehr ſchön ſein ſoll und von da entweder über \textcolor{pink}{\textsc{Salzburg}}{}\ledrightnote{\textcolor{pink}{Salzburg}} oder \textcolor{pink}{\textsc{Golling}}{}\ledrightnote{\textcolor{pink}{Golling}} oder ſonſt zurück. Ich ſehne mich unendlich nach Dörfern, die ich noch nicht
               geſehen habe, nach kleinen Häuſern am {\pb}Waldrand, Mühlen in einem tiefen
               Grund, Brücken, Alleen und andern ſolchen Dingen. Von \textcolor{blue}{Richard}{}\ledrightnote{\textcolor{blue}{Richard Beer-Hofmann}} bin ich ohne irgend eine Nachricht ſeit \textcolor{pink}{Wien}{}\ledrightnote{\textcolor{pink}{Wien}}.\pend
           \pstart
           \textcolor{blue}{Papa}{}\ledrightnote{→\textcolor{blue}{Hugo August von Hofmannsthal}} iſt gottlob wohl, meine
                  \textcolor{blue}{Eltern}{}\ledrightnote{→\textcolor{blue}{Anna von Hofmannsthal}{\newline}→\textcolor{blue}{Hugo August von Hofmannsthal}} grüßen Sie
               vielmals; bitte ſchreiben Sie mir bald, in 3 Tagen bin ich wieder hier.\pend
           \pstart
           Von Herzen Ihr{\\[\baselineskip]}\spacefill\mbox{Hugo.}\pend
           \leftskip=0em{}\endnumbering\briefempfaengerindex{Schnitzler, Arthur@\textsc{Schnitzler, Arthur}!zzzHofmannsthal, Hugo von@\emph{von Hugo von Hofmannsthal}!1900-07-152@{15. 7. 1900}|)be}\mylabel{h}  \normalsize

\doendnotes{C}
\bigskip
\vfill

\clearpage

\footnotesize

\lohead{\textsc{register}}

% Definiere theindex-Environment komplett neu ohne reledmac
\makeatletter
\renewenvironment{theindex}{%
  \section*{\indexname}%
  \setlength{\parindent}{0pt}%
  \setlength{\parskip}{0pt plus 0.3pt}%
  \let\item\@idxitem
}{%
  \clearpage
}
\makeatother

\IfFileExists{\jobname-pw.ind}{\input{\jobname-pw.ind}}{}

\end{document}

      