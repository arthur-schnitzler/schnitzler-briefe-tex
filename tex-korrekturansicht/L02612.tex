%% latex-korrekturansicht-vorspann.tex
%% Vorspann für die Korrekturansicht.
%% Lädt die gemeinsame Datei latex-vorspann.tex mit gesetztem Schalter.

\newif\ifkorrekturansicht
\korrekturansichttrue

\input{../tex-inputs/latex-vorspann}


               \section[Paul Goldmann an Arthur Schnitzler, 8. 9. {[}1894{]}]{ Paul Goldmann an Arthur Schnitzler, 8. 9. {[}1894{]}}\nopagebreak\mylabel{v}\rehead{ }\normalsize\beginnumbering\briefempfaengerindex{Schnitzler, Arthur@\textsc{Schnitzler, Arthur}!zzzGoldmann, Paul@\emph{von Paul Goldmann}!1894-09-081@{8. 9. {[}1894{]}}|(be} \toendnotes[C]{\smallbreak\pagebreak[2]} \Standort{DLA, A:Schnitzler, HS.NZ85.1.3164.}
\physDesc{Brief, 2 Blätter, 8 Seiten
\newline{}Handschrift: schwarze Tinte, deutsche Kurrent
\newline{}Schnitzler: 1) mit Bleistift auf dem ersten Blatt die Jahreszahl
                                       »94« vermerkt 2) mit rotem Buntstift eine Unterstreichung}\toendnotes[C]{\smallbreak}\pstart
           \raggedleft{}{\pb}\textcolor{pink}{Frankfurt}{}\ledrightnote{\textcolor{pink}{Frankfurt am Main}}{ }8. September.\pend
           \pstart\center{}Mein lieber Freund,\pend\pstart
           Ich danke Dir noch von Herzen für die köſtlichen Tage in \label{K_L02612-1v}\edtext{\textsc{\textcolor{pink}{Ischl}{}\ledrightnote{\textcolor{pink}{Bad Ischl}}}}{\lemma{\textnormal{\emph{Ischl}}}\Cendnote{\textnormal{Von 23. 8. 1894 bis 3. 9. 1894 verbrachten
                  Schnitzler und Goldmann einige Zeit gemeinsam in \textcolor{pink}{Bad
                     Ischl} und \textcolor{pink}{Bad Aussee}.}}}\label{K_L02612-1h}. Ich bin
               ruhig und froh geweſen, wie ſchon lange nicht. Ich danke \textcolor{blue}{Euch}{}\ledrightnote{→\textcolor{blue}{Richard Beer-Hofmann}}, daß Ihr mir meine
               Geſpenſter auf ein paar Stunden geſcheucht habt, daß Ihr mich Treue und Gute habt
               fühlen laſſen, {\pb}daß Ihr mir gar – oh
                  Wunder\textcolor{gray}{,} – ein wenig Glauben an mich ſelbſt gegeben habt. Ich
               bin heut muthig und beinahe heiter. Das iſt Euer Werk!
               Und ich bin \textcolor{blue}{Euch}{}\ledrightnote{→\textcolor{blue}{Richard Beer-Hofmann}}
               tief dafür \strikeout{\textcolor{gray}{v}} verpflichtet{\dotsfive}\pend
           \pstart
           Bei dem Regen wirſt Du kaum Deine \textsc{Bicycle}-Partie gemacht
               haben, und Du biſt gewiß ſchon in \textcolor{pink}{Wien}{}\ledrightnote{\textcolor{pink}{Wien}} für den Winter
               inſtalliert und ſitzeſt über der Arbeit. Der \label{K_mets_Goldmann_94-partII-999v}\edtext{\textcolor{green}{Artikel}{}\ledrightnote{→\textcolor{green}{Ein Märchen}}}{\lemma{\textnormal{\emph{Artikel}}}\Cendnote{\textnormal{\textcolor{blue}{Laura
                        Marholm}: \emph{\textcolor{green}{Ein Märchen}}. In: \emph{\textcolor{green}{Die Zukunft}}, Jg. 8,
                        25. 8. 1894, S. 368–371.}}}\label{K_mets_Goldmann_94-partII-999h}{ }{\pb}von der \textsc{\textcolor{blue}{Marholm}{}\ledrightnote{\textcolor{blue}{Laura Marholm}}}, den ich mit
               Hochgenuß gleich in \textsc{\textcolor{pink}{Nuernberg}{}\ledrightnote{\textcolor{pink}{Nürnberg}}} geleſen habe, iſt \strikeout{w} wie eine
               Antwort auf unſer letztes Geſpräch gekommen. Jetzt wirſt Du hoffentlich lange nicht
               mehr daran zweifeln, daß \textsc{Arthur Schnitzler} eine
               Individualität iſt. Ich beglückwünſche Dich zu dieſem schönen Erfolge.\pend
           \pstart
           Mit \strikeout{M} meinem \textcolor{blue}{Onkel}{}\ledrightnote{→\textcolor{blue}{Fedor Mamroth}}{ }{\pb}habe ich ſofort geſprochen. Ich habe ihn unerwartet
               liebevoll und warm vorgefunden, auch voll ſreundſchaftlichen Intereſſes für Dich. Er
               ging ſofort auf meinen Vorſchlag ein, Dir einen Theil des \label{K_L02612-2v}\edtext{Bücher-Reſerats}{\lemma{\textnormal{\emph{Bücher-Reſerats}}}\Cendnote{\textnormal{XXXX}}}\label{K_L02612-2h} zu übertragen. Das iſt nur ein Anſang. Wenn Du regelmäßig arbeiteſt,
               kann noch {\pb}allerlei Anderes daraus werden. Die
               Hauptſache iſt, wie geſagt, daß Du die Sachen regelmäßig erledigſt – nicht ſür
               beſtimmte Termine, aber doch in beſtimmten nicht allzu langen Friſten. Mach’ ruhig
               den Verſuch; ich bin überzeugt, daß es ſo gehen wird. Das Feuilleton bringt, {\pb}glaube ich, \textsc{40 Mark}.\pend
           \pstart
           Ich bleibe noch bis nächſten Samſtag hier. Haſt Du
               Zeit, ſo ſchreib’ mir ein Wort hierher (Adreſſe: \textsc{\textcolor{blue}{Frau Clementine Goldmann}{}\ledrightnote{\textcolor{blue}{Clementine Goldmann}}}, \textsc{\textcolor{pink}{Lindenstraße 1}{}\ledrightnote{\textcolor{pink}{Lindenstraße}}}). Vor
               Allem: Wie geht es mit Deiner Arbeit? Hat \textsc{\textcolor{blue}{Richard}{}\ledrightnote{\textcolor{blue}{Richard Beer-Hofmann}}}{ }{\pb} ſeine Reiſe angetreten? Was hört man von der neuen
                  \textsc{\textcolor{brown}{Revue}{}\ledrightnote{→\textcolor{brown}{Die Zeit. Wiener Wochenschrift}}}?\pend
           \pstart
           Die \textcolor{blue}{Meinigen}{}\ledrightnote{→\textcolor{blue}{Clementine Goldmann}}
               grüßen Dich herzlichſt. Bitte, empfiehl’ mich Deiner Frau \textcolor{blue}{Mutter}{}\ledrightnote{→\textcolor{blue}{Louise Schnitzler}} und \label{K_L02612-11v}\edtext{Danke auch ihr}{\lemma{\textnormal{\emph{Danke auch ihr}}}\Cendnote{\textnormal{\textcolor{blue}{Schnitzler} urlaubte mit seiner Familie in
                     \textcolor{pink}{Ischl}; die hier angesprochene Danksagung
                  dürfte auf eine Form der Gastfreundschaft bezogen sein, die \textcolor{blue}{Louise Schnitzler}{ }\textcolor{blue}{Paul Goldmann} bei seinem Besuch zukommen
                  ließ.}}}\label{K_L02612-11h} nochmals in meinem Namen. Grüß’ mir Deinen \textcolor{blue}{Bruder}{}\ledrightnote{→\textcolor{blue}{Julius Schnitzler}} u. Deine \textcolor{blue}{Schwägerin}{}\ledrightnote{→\textcolor{blue}{Helene Schnitzler}}.\pend
           \pstart
           {\pb}Und ſei Du ſelbſt von Herzen und in Treue
               gegrüßt von{\\[\baselineskip]} Deinem{\\[\baselineskip]}\spacefill\mbox{Paul Goldmann}\pend
           \leftskip=0em{}\endnumbering\briefempfaengerindex{Schnitzler, Arthur@\textsc{Schnitzler, Arthur}!zzzGoldmann, Paul@\emph{von Paul Goldmann}!1894-09-081@{8. 9. {[}1894{]}}|)be}\mylabel{h}  \normalsize

\doendnotes{C}
\bigskip
\vfill

\clearpage

\footnotesize

\lohead{\textsc{register}}

% Definiere theindex-Environment komplett neu ohne reledmac
\makeatletter
\renewenvironment{theindex}{%
  \section*{\indexname}%
  \setlength{\parindent}{0pt}%
  \setlength{\parskip}{0pt plus 0.3pt}%
  \let\item\@idxitem
}{%
  \clearpage
}
\makeatother

\IfFileExists{\jobname-pw.ind}{\input{\jobname-pw.ind}}{}

\end{document}

      