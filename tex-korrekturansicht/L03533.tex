%% latex-korrekturansicht-vorspann.tex
%% Vorspann für die Korrekturansicht.
%% Lädt die gemeinsame Datei latex-vorspann.tex mit gesetztem Schalter.

\newif\ifkorrekturansicht
\korrekturansichttrue

\input{../tex-inputs/latex-vorspann}


\renewcommand{\erwaehntePersonen}{Personen: Felix Salten}
\renewcommand{\erwaehnteInstitutionen}{Institutionen: Dampfer Berlin, Deutsch-amerikanische Seepost, Norddeutscher Lloyd}
\renewcommand{\erwaehnteOrte}{Orte: Atlantischer Ozean, Bremen, Europa, New York City, Sternwartestraße 71, Wien, Österreich}
\renewcommand{\erwaehnteWerke}{}
\section[ Felix Salten an Arthur Schnitzler, 17. 5. 1930]{Felix Salten an Arthur Schnitzler, 17. 5. 1930}
\nopagebreak\mylabel{v}
\rehead{ }\normalsize\beginnumbering\briefempfaengerindex{Schnitzler, Arthur@\textsc{Schnitzler, Arthur}!zzzSalten, Felix@\emph{von Felix Salten}!1930-05-171@{17. 5. 1930}|(be}
\toendnotes[C]{\smallbreak\pagebreak[2]}\Standort{CUL, Schnitzler, B 89, B 2.}
\physDesc{Bildpostkarte, 248 Zeichen
\newline{}Handschrift: schwarze Tinte, lateinische Kurrent
\newline{}Versand: Stempel: »\nobreak{}18 5 \textcolor{gray}{30}, \textcolor{brown}{Deutsche Seepost Linie Bremen
                                          – New York}\nobreak{}«.  
\newline{}Ordnung: mit Bleistift von unbekannter Hand nummeriert: »304« }\pstart{}{\pb}\textcolor{pink}{Europa}{}\ledrightnote{\textcolor{pink}{Europa}}\pend{}\pstart{}\textcolor{pink}{Austria}{}\ledrightnote{\textcolor{pink}{Österreich}}\pend{}\pstart{}Herrn\pend{}\pstart{}D\textsuperscript{r} Arthur Schnitzler\pend{}\pstart{}\textcolor{pink}{Wien}{}\ledrightnote{\textcolor{pink}{Wien}}\pend{}\pstart{}\textcolor{pink}{18. Sternwartestrasse 71}{}\ledrightnote{\textcolor{pink}{Sternwartestraße 71}}\pend{}
{\bigskip}
\pstart
           \noindent{}\centering{}{\pb}\textcolor{gray}{\textbf{\textcolor{brown}{NORDDEUTSCHER LLOYD}{}\ledrightnote{\textcolor{brown}{Norddeutscher Lloyd}}, \textcolor{pink}{BREMEN}{}\ledrightnote{\textcolor{pink}{Bremen}}}}\pend
           
\pstart
           \noindent{}\centering{}\textcolor{brown}{\textcolor{gray}{\textbf{D. »Berlin«}}}{}\ledrightnote{\textcolor{brown}{Dampfer Berlin}}\pend
           
\pstart
           \noindent{}\centering{}\textcolor{gray}{\textbf{Gesellsc{[}h{]}aftshalle 1. Kl.}}\pend
           
\pstart
           \raggedleft{}{\pb}vor \textcolor{pink}{Newyork}{}\ledrightnote{\textcolor{pink}{New York City}}{ }17. 5. 30\pend
           
\pstart
           Lieber, die Fahrt, die nach zehn unruhigen Tagen morgen{ }früh endigt, war trotz allem sehr schön. Ich denke viel und gut an Sie
               und grüße sie herzlichst\pend
           
\pstart
           Ihr {\\[\baselineskip]}\spacefill\mbox{Felix Salten}\pend
           \leftskip=0em{}\endnumbering\briefempfaengerindex{Schnitzler, Arthur@\textsc{Schnitzler, Arthur}!zzzSalten, Felix@\emph{von Felix Salten}!1930-05-171@{17. 5. 1930}|)be}\mylabel{h}  \normalsize

\doendnotes{C}
\bigskip
\vfill

\clearpage

\footnotesize

\lohead{\textsc{register}}

% Definiere theindex-Environment komplett neu ohne reledmac
\makeatletter
\renewenvironment{theindex}{%
  \section*{\indexname}%
  \setlength{\parindent}{0pt}%
  \setlength{\parskip}{0pt plus 0.3pt}%
  \let\item\@idxitem
}{%
  \clearpage
}
\makeatother

\IfFileExists{\jobname-pw.ind}{\input{\jobname-pw.ind}}{}

\end{document}

      