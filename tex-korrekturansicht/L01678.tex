%% latex-korrekturansicht-vorspann.tex
%% Vorspann für die Korrekturansicht.
%% Lädt die gemeinsame Datei latex-vorspann.tex mit gesetztem Schalter.

\newif\ifkorrekturansicht
\korrekturansichttrue

\input{../tex-inputs/latex-vorspann}


               \section[Hugo von Hofmannsthal an Arthur Schnitzler mit Beilage Christiane Thun an Hofmannsthal, {[}25. 5. 1907{]}]{ Hugo von Hofmannsthal an Arthur Schnitzler mit Beilage Christiane Thun
               an Hofmannsthal, {[}25. 5. 1907{]}}\nopagebreak\mylabel{v}\rehead{ }\normalsize\beginnumbering\briefempfaengerindex{Schnitzler, Arthur@\textsc{Schnitzler, Arthur}!zzzHofmannsthal, Hugo von@\emph{von Hugo von Hofmannsthal}!1907-05-251@{{[}25. 5. 1907{]}}|(be} \toendnotes[C]{\smallbreak\pagebreak[2]} \Standort{CUL, Schnitzler, B 43.}
\physDesc{Brief, 1 Blatt, 4 Seiten
\newline{}Handschrift: schwarze Tinte, deutsche Kurrent\newline{}Beilage: \textcolor{blue}{Christine Thun-Salm}: Briefkarte, schwarze Tinte, Lateinschrift 
\newline{}Schnitzler: mit Bleistift datiert: »25/5 907« \newline{}Ordnung: 1) mit Bleistift von unbekannter Hand nummeriert: »\strikeout{279}« 2) mit Bleistift von unbekannter Hand nummeriert: »277«}\buchAbdrucke{\weitereDrucke{Hugo von Hofmannsthal, Arthur Schnitzler: \emph{Briefwechsel}. Hg. Therese Nickl und Heinrich Schnitzler. Frankfurt am Main: \emph{S. Fischer} 1964, S. 228.} }\toendnotes[C]{\smallbreak}\pstart
           \raggedleft{}{\pb}Samstag\pend
           \pstart{}mein lieber Arthur\pend\pstart
           habe \textcolor{blue}{Brahm}{}\ledrightnote{\textcolor{blue}{Otto Brahm}} das Original vorgewieſen:
               2975 Mark. Er bezahlt. Reiſe heute Abend, zunächſt \textcolor{pink}{\textsc{Ravenna}}{}\ledrightnote{\textcolor{pink}{Ravenna}}, dann \textcolor{pink}{\textsc{Umbrien}}{}\ledrightnote{\textcolor{pink}{Umbrien}}. Hoffe ich finde Sie noch in \textcolor{pink}{Wien}{}\ledrightnote{\textcolor{pink}{Wien}} oder nahe \textcolor{pink}{Wien}{}\ledrightnote{\textcolor{pink}{Wien}} gegen
                     10\textsuperscript{ten} July. Ich empfinde es \uline{ſehr}{ }ſchmerzlich wie
               ſelten man ſich sieht. –\pend
           \pstart
           {\pb}Schicke Ihnen dieſen Brief der
               Gräfin \textcolor{blue}{Thun}{}\ledrightnote{\textcolor{blue}{Christiane von Thun-Hohenstein-Salm-Reifferscheidt}}, geſchrieben noch nachdem ſie mir
               damals Adieu (für immer) geſagt hatte, weil es Sie wahrſcheinlich freuen wird, wie
               herzlich ſie in einem ſolchen Moment des letzten Überblicks Ihrer gedenkt.\hspace*{1.5em}Wenn ſie davon kommt – es {\pb}ſcheint Hoffnung zu ſein –
               trotzdem die Operation \uline{ſehr}{ }\uline{ſchwer} war – ſo beſuchen Sie ſie
               vielleicht im \textcolor{pink}{Sanatorium}{}\ledrightnote{→\textcolor{pink}{Sanatorium Loew}}, oder
               ſchicken ihr vielleicht die \textcolor{green}{Dä{\geminationm}erſeelen}{}\ledrightnote{\textcolor{green}{Dämmerseelen. Novellen}}, die ſie noch nicht kennt.\pend
           \pstart
           Adieu. Ich freue mich von Herzen auf den \textcolor{green}{Roman}{}\ledrightnote{→\textcolor{green}{Der Weg ins Freie. Roman}}, das \textcolor{green}{Stück}{}\ledrightnote{→\textcolor{green}{Das Wort. Tragikomödie in fünf Akten}}, auf
               alles was Sie machen. {\pb}Denn ich
               habe noch nie eines Ihrer Bücher ohne tiefe Mitfreude wieder in die Hand geno{\geminationm}en.\pend
           \pstart
           Adieu.{\\[\baselineskip]}Ihr\spacefill\mbox{Hugo.}\pend
           \leftskip=0em{}{\bigskip}\pstart
           \raggedleft{}{\pb}{[}hs. Thun-Hohenstein-Salm-Reifferscheidt:{]} 21. 5. 1907\pend
           \pstart
           \raggedleft{}\textcolor{pink}{Wien, Sanatorium Löw}{}\ledrightnote{\textcolor{pink}{Sanatorium Loew}}.\pend
           \pstart
           Ich habe mich sehr gefreut, Sie heute noch zu sehen. Nachdem Sie bei mir waren, bin
               ich ins \textcolor{pink}{Sanatorium}{}\ledrightnote{→\textcolor{pink}{Sanatorium Loew}} gefahren. Es
               scheint hier sehr voll zu sein, {\kaufmannsund} ich habe ein
               Schandloch auf die Gasse hinaus. –\pend
           \pstart
           Im besten Fall 4 Wochen hier zu sitzen ist eine abscheuliche Aussicht!\pend
           \pstart
           Leben Sie wohl! Sagen Sie Ihrer \textcolor{blue}{Frau}{}\ledrightnote{→\textcolor{blue}{Gertrude von Hofmannsthal}} viel Liebes von mir {\kaufmannsund} seien Sie herzlich
               von mir gegrüsst!\pend
           \pstart
           {\pb}Danke noch für alle Ihre
               Freundschaft! Ich habe auch für Sie immer sehr viel Freundschaft gehabt.\pend
           \pstart
           Möge es Ihnen gut gehen! Das wünscht Ihnen von
                  Herzen{\\[\baselineskip]}\spacefill\mbox{ChristThunSalm}\pend
           \leftskip=0em{}\pstart
           \noindent{}Wenn Sie Dtr. Arthur Schnitzler sehen, dann bitte grüssen Sie ihn herzlich von
                  mir!\pend
           \endnumbering\briefempfaengerindex{Schnitzler, Arthur@\textsc{Schnitzler, Arthur}!zzzHofmannsthal, Hugo von@\emph{von Hugo von Hofmannsthal}!1907-05-251@{{[}25. 5. 1907{]}}|)be}\mylabel{h}  \normalsize

\doendnotes{C}
\bigskip
\vfill

\clearpage

\footnotesize

\lohead{\textsc{register}}

% Definiere theindex-Environment komplett neu ohne reledmac
\makeatletter
\renewenvironment{theindex}{%
  \section*{\indexname}%
  \setlength{\parindent}{0pt}%
  \setlength{\parskip}{0pt plus 0.3pt}%
  \let\item\@idxitem
}{%
  \clearpage
}
\makeatother

\IfFileExists{\jobname-pw.ind}{\input{\jobname-pw.ind}}{}

\end{document}

      