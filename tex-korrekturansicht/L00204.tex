%% latex-korrekturansicht-vorspann.tex
%% Vorspann für die Korrekturansicht.
%% Lädt die gemeinsame Datei latex-vorspann.tex mit gesetztem Schalter.

\newif\ifkorrekturansicht
\korrekturansichttrue

\input{../tex-inputs/latex-vorspann}


               \section[Arthur Schnitzler an Richard Beer-Hofmann, {[}29. 4. 1893?{]}]{ Arthur Schnitzler an Richard Beer-Hofmann, {[}29. 4. 1893?{]}}\nopagebreak\mylabel{v}\rehead{ }\normalsize\beginnumbering\briefempfaengerindex{Beer-Hofmann, Richard@\textsc{Beer-Hofmann, Richard}!zzzSchnitzler, Arthur@\emph{von Arthur Schnitzler}!1893-04-292@{{[}29. 4. 1893?{]}}|(be} \toendnotes[C]{\smallbreak\pagebreak[2]} \Standort{YCGL, MSS 31.}
\physDesc{Brief, 1 Blatt, 3 Seiten, Umschlag
\newline{}Handschrift: Bleistift, deutsche Kurrent\newline{}Versand: ohne postalischen Übermittlungsvermerk }\buchAbdrucke{\weitereDrucke{Arthur Schnitzler, Richard Beer-Hofmann: \emph{Briefwechsel 1891–1931}. Hg. Konstanze Fliedl. Wien, Zürich: \emph{Europaverlag} 1992, S. 44.} }\toendnotes[C]{\smallbreak}\pstart{}{\pb}\textsc{Herrn Dr. Rich Beer-Hofmann}\pend{}\pstart{}\textcolor{pink}{Wien}{}\ledrightnote{\textcolor{pink}{Wien}}.\pend{}\pstart{}\textsc{\textcolor{pink}{I Wollzeile 15}{}\ledrightnote{\textcolor{pink}{Wollzeile}}}.\pend{}{\bigskip}\pstart
           \noindent{}{\pb}Lieber Richard, hier iſt der Sitz,
               Sie bringen ihn ſicher noch leicht an \introOben{}(\substVorne{}\textsuperscript{\textcolor{gray}{womö}}\substDazwischen{}ſchli{\geminationm}\substHinten{}ſtenfalls an d\textcolor{gray}{er}{ }\textsc{Casse})\introOben{}. – Ich ka{\geminationn}
               nicht gehen, wegen \textcolor{blue}{Papa}{}\ledrightnote{→\textcolor{blue}{Johann Schnitzler}}, der
               ſtark fiebert und meinetwegen, der, Abends wenigſtens, ſchwach fiebert. Ich werde
               ſehen, ob ich heute um 10 ins Cafè {\pb}ko{\geminationm}en kann – ich hoffe! –\pend
           \pstart
           – Von \textcolor{blue}{\textsc{Fels}}{}\ledrightnote{\textcolor{blue}{Friedrich Michael Fels}} kam Telegra{\geminationm}: er bittet um 25 fl, um abreiſen zu
               können. Ich ſandte ihm die 15 von \textcolor{blue}{\textsc{Loris}}{}\ledrightnote{\textcolor{blue}{Hugo von Hofmannsthal}}{ }\textsc{resp}{ }\textcolor{blue}{Fiſcher}{}\ledrightnote{\textcolor{blue}{Robert Fischer}}, u. von mir zehn. – –\pend
           \pstart
           \textcolor{blue}{\textsc{Specht}}{}\ledrightnote{\textcolor{blue}{Richard Specht}} geht vielleicht zum \label{K_L00204_1v}\edtext{\textcolor{green}{ledigen Hof}{}\ledrightnote{\textcolor{green}{Der ledige Hof. Schauspiel in 4 Akten}}}{\lemma{\textnormal{\emph{ledigen Hof}}}\Cendnote{\textnormal{Mehrere Stellen des undatierten
                  Briefes erlauben gemeinsam eine zeitliche Einordnung. Am 29. 4. 1893
                  fand im Zuge eines Gastspiels die Aufführung von \textcolor{blue}{Ludwig Anzengruber}s \emph{\textcolor{green}{Der ledige Hof}} im
                     \textcolor{pink}{Carltheater}
                   statt. Am Vortag vermerkte sich
                     \textcolor{blue}{Schnitzler} im \emph{\textcolor{green}{Tagebuch}}, dass sein \textcolor{blue}{Vater} krank sei und er es werde. Die Verortung vor dem Sonntag spricht
                  gleichfalls für den Samstag.}}}\label{K_L00204_1h}? –\pend
           \pstart
           {\pb}Vielleicht theilen Sie mir irgendwie mit, was
               für So{\geminationn}tag morgen Nachmittag
               projektirt ift; ka{\geminationn} ich auf ein paar Stunden mit Euch
               ſein, möcht ichs gerne. –\pend
           \pstart
           Herzlich der Ihre{\\[\baselineskip]}\spacefill\mbox{Arthur}\pend
           \leftskip=0em{}\endnumbering\briefempfaengerindex{Beer-Hofmann, Richard@\textsc{Beer-Hofmann, Richard}!zzzSchnitzler, Arthur@\emph{von Arthur Schnitzler}!1893-04-292@{{[}29. 4. 1893?{]}}|)be}\mylabel{h}  \normalsize

\doendnotes{C}
\bigskip
\vfill

\clearpage

\footnotesize

\lohead{\textsc{register}}

% Definiere theindex-Environment komplett neu ohne reledmac
\makeatletter
\renewenvironment{theindex}{%
  \section*{\indexname}%
  \setlength{\parindent}{0pt}%
  \setlength{\parskip}{0pt plus 0.3pt}%
  \let\item\@idxitem
}{%
  \clearpage
}
\makeatother

\IfFileExists{\jobname-pw.ind}{\input{\jobname-pw.ind}}{}

\end{document}

      