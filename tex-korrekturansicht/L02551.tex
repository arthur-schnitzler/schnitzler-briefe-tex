%% latex-korrekturansicht-vorspann.tex
%% Vorspann für die Korrekturansicht.
%% Lädt die gemeinsame Datei latex-vorspann.tex mit gesetztem Schalter.

\newif\ifkorrekturansicht
\korrekturansichttrue

\input{../tex-inputs/latex-vorspann}


               \section[Fedor Mamroth und Paul Goldmann an Arthur Schnitzler, 9. 12. 1888]{ Fedor Mamroth und Paul Goldmann an Arthur Schnitzler,
               9. 12. 1888}\nopagebreak\mylabel{v}\rehead{ }\normalsize\beginnumbering\briefempfaengerindex{Schnitzler, Arthur@\textsc{Schnitzler, Arthur}!zzzGoldmann, Paul@\emph{von Paul Goldmann}!1888-12-091@{9. 12. 1888}|(be}\briefempfaengerindex{Schnitzler, Arthur@\textsc{Schnitzler, Arthur}!zzzMamroth, Fedor@\emph{von Fedor Mamroth}!1888-12-091@{9. 12. 1888}|(be} \toendnotes[C]{\smallbreak\pagebreak[2]} \Standort{DLA, A:Schnitzler, HS.NZ85.1.3162.}
\physDesc{Brief, 1 Blatt, 4 Seiten
\newline{}Handschrift Paul Goldmann: blaue Tinte, deutsche Kurrent}\toendnotes[C]{\smallbreak}\pstart
           \noindent{}\centering{}{\pb}\textcolor{gray}{\textbf{\textbf{Adminiſtration: \textcolor{pink}{VII.
                           Seidengaſſe 7}{}\ledrightnote{\textcolor{pink}{Seidengasse}}} (\textcolor{brown}{Jos. Eberle {\kaufmannsund} Co.}{}\ledrightnote{\textcolor{brown}{Josef Eberle  Stein-, Buch und Musikaliendruckerei}})}}\pend
           \pstart
           \noindent{}\centering{}\textcolor{gray}{\textbf{\textcolor{brown}{An der Schönen Blauen Donau}{}\ledrightnote{\textcolor{brown}{An der schönen blauen Donau}}}}\pend
           \pstart
           \noindent{}\centering{}\textcolor{gray}{\textbf{Chef-Redacteur: Dr. F. Mamroth – Redaction: \textcolor{pink}{IX., Berggaſſe 31}{}\ledrightnote{\textcolor{pink}{Berggasse}}.}}\pend
           \pstart
           \raggedleft{}\textcolor{gray}{\textbf{\textcolor{pink}{Wien}{}\ledrightnote{\textcolor{pink}{Wien}}, den}}{ }9. Dezember \textcolor{gray}{\textbf{18}}88.\pend
           \pstart\center{}Hochgeehrter Herr!\pend\pstart
           Wir haben die \label{K_L02551-2v}\edtext{\textcolor{green}{Erzählung}{}\ledrightnote{→\textcolor{green}{Mein Freund Ypsilon}}}{\lemma{\textnormal{\emph{Erzählung}}}\Cendnote{\textnormal{vgl. A. S.: \emph{Tagebuch}, 10. 12. 1888}}}\label{K_L02551-2h}, die Sie uns freundlichſt eingeſandt, mit dem lebhafteſten Intereſſe
               geleſen. Wir finden die Idee Ihrer \textcolor{green}{Arbeit}{}\ledrightnote{→\textcolor{green}{Mein Freund Ypsilon}} originell und feſſelnd, die Durchführung recht gewandt; überhaupt
               ſcheint ſie uns zu einem neuen Genre zu gehören, das verdient kultiviert zu
               werden.\pend
           \pstart
           Wir ſind freilich auch mit einigem in Ihrer \textcolor{green}{Arbeit}{}\ledrightnote{→\textcolor{green}{Mein Freund Ypsilon}} nicht {\pb}einverſtanden.
               Wir meinen, es dürfe nicht, wie das geſchieht, der Leſer bis zum Schluſſe im Unklaren
               gelaſſen werden, ob er einen Wahnſinnigen oder einen Phantaſten vor ſich hat. Wir
               glauben, es würde der \textcolor{green}{Erzählung}{}\ledrightnote{→\textcolor{green}{Mein Freund Ypsilon}} entſchieden zum Vortheil gereichen, wenn das erzählende »Ich« als
               Mediziner hingeſtellt würde, der ſich über das Benehmen ſeines Freundes im Verlaufe
               der Entwicklung ziemlich entſchieden vom mediziniſchen Standpunkt ausſpräche; er
               braucht ihn ja nicht geradezu als irrſinnig zu erklären, aber er kann doch hier und
               da auf die flüſſige Grenze zwiſchen Wahnſinn und dichteriſchem Talent hinweiſen und
               ausdrücken, daß {\pb}der Fall ſeines Freundes in dieſes
               Grenzgebiet gehöre. Mit einem Worte: die Erzählung ſoll einen Stich ins
                  Mediziniſ\textcolor{gray}{c}he bekommen.\pend
           \pstart
           Wenn Sie, hochgeehrter Herr, ſich freundlichſt bereit finden, eine Änderung Ihrer \textcolor{green}{Arbeit}{}\ledrightnote{→\textcolor{green}{Mein Freund Ypsilon}} in dieſem Sinne
               vorzunehmen, ſo ſind wir mit vielem Vergnügen bereit, dieſelbe in unſerem \textcolor{brown}{Blatte}{}\ledrightnote{→\textcolor{brown}{An der schönen blauen Donau}} zu veröffentlichen.\pend
           \pstart
           Wir bitten Sie, uns baldgefälligſt antworten zu wollen, und empfehlen {\pb}uns Ihnen\pend
           \pstart
           Hochachtungsvoll{\\[\baselineskip]}\textcolor{gray}{\textbf{\textit{Die Redaction}}}{\\[\baselineskip]}\textcolor{gray}{\textbf{\textit{der}}}{\\[\baselineskip]}\textcolor{gray}{\textbf{\textit{»\textcolor{brown}{Schönen blauen Donau}{}\ledrightnote{\textcolor{brown}{An der schönen blauen Donau}}«}}}{\\[\baselineskip]}\spacefill\mbox{\label{K_L02551-1v}\edtext{p.}{\lemma{\textnormal{\emph{p.}}}\Cendnote{\textnormal{für »per«, vgl. Fedor Mamroth an Arthur Schnitzler, 4. 4. 1894}}}\label{K_L02551-1h} Dr. F. Mamroth.}\pend
           \leftskip=0em{}\endnumbering\briefempfaengerindex{Schnitzler, Arthur@\textsc{Schnitzler, Arthur}!zzzGoldmann, Paul@\emph{von Paul Goldmann}!1888-12-091@{9. 12. 1888}|)be}\briefempfaengerindex{Schnitzler, Arthur@\textsc{Schnitzler, Arthur}!zzzMamroth, Fedor@\emph{von Fedor Mamroth}!1888-12-091@{9. 12. 1888}|)be}\mylabel{h}  \normalsize

\doendnotes{C}
\bigskip
\vfill

\clearpage

\footnotesize

\lohead{\textsc{register}}

% Definiere theindex-Environment komplett neu ohne reledmac
\makeatletter
\renewenvironment{theindex}{%
  \section*{\indexname}%
  \setlength{\parindent}{0pt}%
  \setlength{\parskip}{0pt plus 0.3pt}%
  \let\item\@idxitem
}{%
  \clearpage
}
\makeatother

\IfFileExists{\jobname-pw.ind}{\input{\jobname-pw.ind}}{}

\end{document}

      