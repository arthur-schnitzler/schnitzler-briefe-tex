%% latex-korrekturansicht-vorspann.tex
%% Vorspann für die Korrekturansicht.
%% Lädt die gemeinsame Datei latex-vorspann.tex mit gesetztem Schalter.

\newif\ifkorrekturansicht
\korrekturansichttrue

\input{../tex-inputs/latex-vorspann}


               \section[Arthur Schnitzler an Richard Beer-Hofmann, 26. 8. 1918]{ Arthur Schnitzler an Richard Beer-Hofmann, 26. 8. 1918}\nopagebreak\mylabel{v}\rehead{ }\normalsize\beginnumbering\briefempfaengerindex{Beer-Hofmann, Richard@\textsc{Beer-Hofmann, Richard}!zzzSchnitzler, Arthur@\emph{von Arthur Schnitzler}!1918-08-261@{26. 8. 1918}|(be} \toendnotes[C]{\smallbreak\pagebreak[2]} \Standort{YCGL, MSS 31.}
\physDesc{Bildpostkarte
\newline{}Handschrift: Bleistift, lateinische Kurrent\newline{}Versand: Stempel: »\nobreak{}\textcolor{gray}{Wien}, 26. VIII. 18\nobreak{}«.  
\newline{}Beer-Hofmann: mit blauem Buntstift Erhalt und Beantwortung
                                    vermerkt: »E. B. 28./VIII 18« \newline{}Zusatz: Postkartenmotiv mit \textcolor{blue}{Olga} und
                                    \textcolor{blue}{Heinrich} links vor dem Haus
                                 und Schnitzler und \textcolor{blue}{Lili} auf dem
                                 Söller }\buchAbdrucke{\weitereDrucke{Arthur Schnitzler, Richard Beer-Hofmann: \emph{Briefwechsel 1891–1931}. Hg. Konstanze Fliedl. Wien, Zürich: \emph{Europaverlag} 1992, S. 226.} }\toendnotes[C]{\smallbreak}\pstart{}{\pb}Herrn Dr. Richard Beer Hofmann\pend{}\pstart{}\textcolor{pink}{Bad Ischl}{}\ledrightnote{\textcolor{pink}{Bad Ischl}}\pend{}\pstart{}\textcolor{pink}{Grazerstr. 56}{}\ledrightnote{\textcolor{pink}{Grazer Straße}}\pend{}{\bigskip}\pstart
           \noindent{}\centering{}{\pb}\textcolor{gray}{\textbf{\textcolor{pink}{Wien, XVIII, Sternwartestr. 71}{}\ledrightnote{\textcolor{pink}{Sternwartestraße}}.}}\pend
           \pstart
           \noindent{}\raggedleft{}A. S.\pend
           \pstart
           {\pb}lieber Richard, aus \textcolor{pink}{Salzburg}{}\ledrightnote{\textcolor{pink}{Salzburg}} ist nun
               doch nichts geworden; ich fahre morgen, möglichst direct \textcolor{pink}{München}{}\ledrightnote{\textcolor{pink}{München}} – \textcolor{pink}{Partenkirchen}{}\ledrightnote{\textcolor{pink}{Partenkirchen}}; es
               scheint meiner \textcolor{blue}{Schwägerin}{}\ledrightnote{→\textcolor{blue}{Elisabeth Steinrück}} wieder
               schlechter zu gehn. Bitte um ein Wort nach \textcolor{pink}{P.}{}\ledrightnote{\textcolor{pink}{Partenkirchen}} (\textcolor{pink}{Haus Tannenberg}{}\ledrightnote{\textcolor{pink}{Haus Tannenberg}}.) Hat der \textcolor{blue}{Herzog von \textcolor{pink}{Leopoldskron}{}\ledrightnote{\textcolor{pink}{Salzburg-Leopoldskron}}}{}\ledrightnote{→\textcolor{blue}{Max Reinhardt}} Ihnen einen besti{\geminationm}ten \label{K_L02301-v}\edtext{\textcolor{green}{Termin}{}\ledrightnote{→\textcolor{green}{Jaákobs Traum. Ein Vorspiel}}}{\lemma{\textnormal{\emph{Termin}}}\Cendnote{\textnormal{Die \textcolor{pink}{Berlin}er Premiere verzögerte sich bis zum
                  7. 11. 1919.}}}\label{K_L02301-h} gegeben? Ihnen ev. auch etwas über den \label{K_L02301-2v}\edtext{Termin der »\textcolor{green}{Schwestern}{}\ledrightnote{\textcolor{green}{Die Schwestern oder Casanova in Spa. Lustspiel in Versen}}«}{\lemma{\textnormal{\emph{Termin der »Schwestern«}}}\Cendnote{\textnormal{Trotz eines
                  Vorvertrags vom 20. 12. 1917 kam keine Inszenierung am von \textcolor{blue}{Max Reinhardt} geleiteten \emph{\textcolor{brown}{Deutschen Theater}} zustande.}}}\label{K_L02301-2h} verrathen? Herzlichst\pend
           \pstart \spacefill\mbox{A.}\pend{}\endnumbering\briefempfaengerindex{Beer-Hofmann, Richard@\textsc{Beer-Hofmann, Richard}!zzzSchnitzler, Arthur@\emph{von Arthur Schnitzler}!1918-08-261@{26. 8. 1918}|)be}\mylabel{h}  \normalsize

\doendnotes{C}
\bigskip
\vfill

\clearpage

\footnotesize

\lohead{\textsc{register}}

% Definiere theindex-Environment komplett neu ohne reledmac
\makeatletter
\renewenvironment{theindex}{%
  \section*{\indexname}%
  \setlength{\parindent}{0pt}%
  \setlength{\parskip}{0pt plus 0.3pt}%
  \let\item\@idxitem
}{%
  \clearpage
}
\makeatother

\IfFileExists{\jobname-pw.ind}{\input{\jobname-pw.ind}}{}

\end{document}

      