%% latex-korrekturansicht-vorspann.tex
%% Vorspann für die Korrekturansicht.
%% Lädt die gemeinsame Datei latex-vorspann.tex mit gesetztem Schalter.

\newif\ifkorrekturansicht
\korrekturansichttrue

\input{../tex-inputs/latex-vorspann}


         
         \renewcommand{\erwaehntePersonen}{Personen: Gerhart Hauptmann, Alfred Kerr}
         \renewcommand{\erwaehnteOrte}{Orte: Berlin, Frankfurt am Main, Reuterweg, Wien}
         \renewcommand{\erwaehnteWerke}{Werke: Der Biberpelz, Die Weber, Die versunkene Glocke, Feuilleton. »Michael Kramer.«, Fuhrmann Henschel, Hanneles Himmelfahrt. Traumdichtung in zwei Teilen, Michael Kramer, Neue Freie Presse, Schluck und Jau}
               \section[ Paul Goldmann an Arthur Schnitzler, 31. 12. {[}1900{]}]{Paul Goldmann an Arthur Schnitzler, 31. 12. {[}1900{]}}\nopagebreak\mylabel{v}\rehead{ }\normalsize\beginnumbering\briefempfaengerindex{Schnitzler, Arthur@\textsc{Schnitzler, Arthur}!zzzGoldmann, Paul@\emph{von Paul Goldmann}!1900-12-311@{31. 12. {[}1900{]}}|(be} \toendnotes[C]{\smallbreak\pagebreak[2]} \Standort{DLA, A:Schnitzler, HS.NZ85.1.3170.}
\physDesc{Brief, 1 Blatt, 4 Seiten
\newline{}Handschrift: 1) blaue Tinte, deutsche Kurrent\hspace{1em}2) schwarze Tinte, deutsche Kurrent (\noindent{}sechs Zeilen auf der ersten Seite)\hspace{1em}
\newline{}Schnitzler: 1) mit Bleistift das Jahr »{[}1{]}900« vermerkt  2) mit rotem Buntstift zwei Unterstreichungen}\toendnotes[C]{\smallbreak}\pstart
           \noindent{}{\pb}\textcolor{pink}{Frankfurt}{}\ledrightnote{\textcolor{pink}{Frankfurt am Main}}, 31. December.\hfill \textcolor{gray}{\textbf{\textcolor{pink}{Reuterweg 59}{}\ledrightnote{\textcolor{pink}{Reuterweg}}.}}\pend
           \pstart
           \centering{}Mein lieber Freund,\pend
           \pstart
           \noindent{}Ich danke Dir für Deine eingehende Erörterung meines \label{K_L02947-1v}\edtext{\textcolor{green}{Feuilleton}{}\ledrightnote{{$\rightarrow$}\textcolor{green}{Feuilleton. »Michael Kramer.«}}s}{\lemma{\textnormal{\emph{Feuilletons}}}\Cendnote{\textnormal{\textcolor{blue}{Paul Goldmann}: \emph{\textcolor{green}{Feuilleton. »Michael Kramer.«}}. In: \emph{\textcolor{green}{Neue Freie Presse}}, Nr. 13055, 28. 12. 1900, Morgenblatt, S. 1–3.}}}\label{K_L02947-1h}, finde aber, daß
               ich abſolut Recht habe und würde ſelbſt jetzt, wo ich weiß, daß Dir gewiſſe \textcolor{green}{Bemerkungen}{}\ledrightnote{{$\rightarrow$}\textcolor{green}{Feuilleton. »Michael Kramer.«}} unangebracht
               erſcheinen, dieſe \textcolor{green}{Bemerkungen}{}\ledrightnote{{$\rightarrow$}\textcolor{green}{Feuilleton. »Michael Kramer.«}}
               nochmals mit ruhigem Gewiſſen niederſchreiben. Ich habe die \textcolor{green}{Kritik}{}\ledrightnote{{$\rightarrow$}\textcolor{green}{Feuilleton. »Michael Kramer.«}} im hellen Zorn verfaßt, im Zorn
               nicht nur gegen die Kritikloſigkeit der \textsc{\textcolor{blue}{Hauptmann}{}\ledrightnote{\textcolor{blue}{Gerhart Hauptmann}}}-Anhänger (unter denen ſich unſer Freund \textsc{\textcolor{blue}{Kerr}{}\ledrightnote{\textcolor{blue}{Alfred Kerr}}} beſonders hevorgethan hat) ſondern namentlich gegen den \textcolor{blue}{Autor}{}\ledrightnote{{$\rightarrow$}\textcolor{blue}{Gerhart Hauptmann}}, der durch ſeine theils
               urtheilsunfähige und unkünſtleriſche, theils auch verlogene Anhängerſchaft {\pb}zum größten der modernen deutſchen Dichter
               ausgerufen worden iſt, der dieſe Rolle als ihm gebührend widerſpruchslos acceptirt
               hat und der nun Stück auf Stück ſchreibt, \strikeout{in de} (\textcolor{green}{Verſunkene Glocke}{}\ledrightnote{\textcolor{green}{Die versunkene Glocke}}, \textcolor{green}{Fuhrmann Henſchel}{}\ledrightnote{\textcolor{green}{Fuhrmann Henschel}}, \textcolor{green}{Schluck
                  und Jau}{}\ledrightnote{\textcolor{green}{Schluck und Jau}}, \textcolor{green}{Michael Kramer}{}\ledrightnote{\textcolor{green}{Michael Kramer}}), in dem er
               ſeine Mittelmäßigkeit, ſeine Flachheit immer deutlicher enthüllt. Der Mangel an
               innerem Werth iſt nirgends noch ſo klar hevorgetreten, als im »\textcolor{green}{Michael Kramer}{}\ledrightnote{\textcolor{green}{Michael Kramer}}«. Ein Dichter darf ein Werk verfehlen, wenn er
               es als Dichter verfehlt. Es kann auch im verunglückten Werk \strikeout{et} etwas von Perſönlichkeit ſtecken, das zum Reſpekt zwingt. {\pb}Wenn aber ein Werk deutlich zeigt, daß jede
               Perſönlichkeit fehlt, – wenn es zeigt, daß keine Weltanſchauung vorhanden iſt und daß
               der Verſuch, eine ſolche auszudrücken, zu \strikeout{p\textcolor{gray}{rä}} prätentiöſem Geſchwätz führt, – wenn Alles hohl, albern und unfähig iſt, dann
               kann der Kritiker ſeine Ausdrücke nicht erbarmungslos genug \strikeout{feh} wählen. Das iſt nicht ein Irren eines Dichters, dem Großes gelungen
               iſt, das iſt das Zutagetreten einer Mediokrität, der Zeitſtimmung und allerlei andere
               Chancen die Möglichkeit gegeben haben, hier und da etwas Hübſches zu ſchreiben und
               ſich daraufhin als Dichter aufzuſpielen. Die »\textcolor{green}{Weber}{}\ledrightnote{\textcolor{green}{Die Weber}}« {\pb}ſind ein ſchönes \textcolor{green}{Stück}{}\ledrightnote{{$\rightarrow$}\textcolor{green}{Die Weber}} (aber vielmehr \strikeout{wä} waren es ſeinerzeit; \strikeout{ob
                  ſ} ob ſie es heut noch ſind, müßte man erſt \strikeout{\textcolor{gray}{rec}h} ſehen); »\textcolor{green}{Hannele}{}\ledrightnote{\textcolor{green}{Hanneles Himmelfahrt. Traumdichtung in zwei Teilen}}« \strikeout{\textcolor{gray}{i}ſ\textcolor{gray}{t}} kenne ich nicht auf der Bühne; der \label{K_L02947-4v}\edtext{»\textcolor{green}{Bibelpelz}{}\ledrightnote{{$\rightarrow$}\textcolor{green}{Der Biberpelz}}«}{\lemma{\textnormal{\emph{»Bibelpelz«}}}\Cendnote{\textnormal{eigentlich \emph{\textcolor{green}{Biberpelz}}}}}\label{K_L02947-4h} iſt ein hübſcher Entwurf zu einem Luſtſpiel, den auszuführen die Kunft
               gemangelt hat. \textsc{\textcolor{blue}{Hauptmann}{}\ledrightnote{\textcolor{blue}{Gerhart Hauptmann}}s} Stern iſt im Sinken. Ich
               freue mich deſſen, weil dadurch eine der literariſchen Lügen unſerer Zeit zu Grunde
               geht, und werde es bei nächſter Gelegenheit wieder ſchreiben.\pend
           \pstart
           Viele treue Grüße und nochmals {\\[\baselineskip]}von Herzen alles Glück zum {\\[\baselineskip]}neuen Jahr! {\\[\baselineskip]}Dein {\\[\baselineskip]}\spacefill\mbox{Paul Goldmann}\pend
           \leftskip=0em{}\pstart
           \noindent{}\strikeout{V\textcolor{gray}{on} übe\textcolor{gray}{r}}{ }Übermorgen fahre ich wieder nach \textcolor{pink}{Berlin}{}\ledrightnote{\textcolor{pink}{Berlin}}.\pend
           \endnumbering\briefempfaengerindex{Schnitzler, Arthur@\textsc{Schnitzler, Arthur}!zzzGoldmann, Paul@\emph{von Paul Goldmann}!1900-12-311@{31. 12. {[}1900{]}}|)be}\mylabel{h}\begin{anhang}\end{anhang}\normalsize

\doendnotes{C}
\bigskip
\vfill

\clearpage

\footnotesize

\lohead{\textsc{register}}

% Definiere theindex-Environment komplett neu ohne reledmac
\makeatletter
\renewenvironment{theindex}{%
  \section*{\indexname}%
  \setlength{\parindent}{0pt}%
  \setlength{\parskip}{0pt plus 0.3pt}%
  \let\item\@idxitem
}{%
  \clearpage
}
\makeatother

\IfFileExists{\jobname-pw.ind}{\input{\jobname-pw.ind}}{}

\end{document}

      