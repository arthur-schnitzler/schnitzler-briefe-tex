%% latex-korrekturansicht-vorspann.tex
%% Vorspann für die Korrekturansicht.
%% Lädt die gemeinsame Datei latex-vorspann.tex mit gesetztem Schalter.

\newif\ifkorrekturansicht
\korrekturansichttrue

\input{../tex-inputs/latex-vorspann}


\renewcommand{\erwaehntePersonen}{Personen: Emilie Mewes-Béha, Ottilie Salten, Paul Salten, Olga Schnitzler}
\renewcommand{\erwaehnteOrte}{Orte: Wien}
\renewcommand{\erwaehnteWerke}{Werke: Die Zeit, Studie}
\section[ Felix Salten an Arthur Schnitzler, 11. 8. 1903]{Felix Salten an Arthur Schnitzler, 11. 8. 1903}
\nopagebreak\mylabel{v}
\rehead{ }\normalsize\beginnumbering\briefempfaengerindex{Schnitzler, Arthur@\textsc{Schnitzler, Arthur}!zzzSalten, Felix@\emph{von Felix Salten}!1903-08-111@{11. 8. 1903}|(be}
\toendnotes[C]{\smallbreak\pagebreak[2]}\Standort{CUL, Schnitzler, B 89, A 2.}
\physDesc{Brief, 1 Blatt, 1 Seite, 604 Zeichen
\newline{}Handschrift: blaue Tinte, lateinische Kurrent
\newline{}Ordnung: mit Bleistift von unbekannter Hand nummeriert: »167« }\toendnotes[C]{\smallbreak}
\pstart
           \raggedleft{}{\pb}11. VIII. 03\pend
           
\pstart
           Lieber, Ihre \textcolor{green}{Sendung}{}\ledrightnote{{$\rightarrow$}\textcolor{green}{Studie}} hab ich heute bei meiner Rückkehr
               vorgefunden und gleich gelesen. Es ist nichts besonderes, aber doch so
                  \textcolor{gray}{–}, dass man es in der \textcolor{green}{Sonntags-Zeit}{}\ledrightnote{\textcolor{green}{Die Zeit}} einmal bringen kann, was ich denn auch mit Vergnügen thue, da
               es Ihnen offenbar sehr erwünscht ist. Hab’ ich Ihren Brief recht gelesen, so soll die
                  »\label{K_L03342-1v}\edtext{\textcolor{green}{Studie}{}\ledrightnote{\textcolor{green}{Studie}}}{\lemma{\textnormal{\emph{Studie}}}\Cendnote{\textnormal{\textcolor{blue}{E. Mewes-Béha}: \emph{\textcolor{green}{Studie}}. In: \emph{\textcolor{green}{Die
                        Zeit}}, Jg. 2, Nr. 364, 4. 10. 1903, Die
                     Sonntags-Zeit, S. 2–3.}}}\label{K_L03342-1h}« erst in der zweiten Hälfte September publizirt werden. Ich habe das auf dem Mscpt
               vorgemerkt.\pend
           
\pstart
           Heute{ }Nachmittag um ¾ 2 hat meine \textcolor{blue}{Frau}{}\ledrightnote{{$\rightarrow$}\textcolor{blue}{Ottilie Salten}} einen \label{K_L03342-2v}\edtext{\textcolor{blue}{Buben}{}\ledrightnote{{$\rightarrow$}\textcolor{blue}{Paul Salten}}}{\lemma{\textnormal{\emph{Buben}}}\Cendnote{\textnormal{\textcolor{blue}{Paul Salten}, siehe auch A. S.: \emph{Tagebuch}, 12. 8. 1903}}}\label{K_L03342-2h} bekommen und befindet sich sehr wol. Wir freuen uns sehr, wie Sie sich denken
               können. Wollen Sie es, bitte, an \textcolor{blue}{Olga}{}\ledrightnote{\textcolor{blue}{Olga Schnitzler}}
               mittheilen.\pend
           
\pstart
           Herzlichst {\\[\baselineskip]}Ihr {\\[\baselineskip]}\spacefill\mbox{Salten}\pend
           \leftskip=0em{}\endnumbering\briefempfaengerindex{Schnitzler, Arthur@\textsc{Schnitzler, Arthur}!zzzSalten, Felix@\emph{von Felix Salten}!1903-08-111@{11. 8. 1903}|)be}\mylabel{h}  \normalsize

\doendnotes{C}
\bigskip
\vfill

\clearpage

\footnotesize

\lohead{\textsc{register}}

% Definiere theindex-Environment komplett neu ohne reledmac
\makeatletter
\renewenvironment{theindex}{%
  \section*{\indexname}%
  \setlength{\parindent}{0pt}%
  \setlength{\parskip}{0pt plus 0.3pt}%
  \let\item\@idxitem
}{%
  \clearpage
}
\makeatother

\IfFileExists{\jobname-pw.ind}{\input{\jobname-pw.ind}}{}

\end{document}

      