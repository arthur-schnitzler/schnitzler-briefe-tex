%% latex-korrekturansicht-vorspann.tex
%% Vorspann für die Korrekturansicht.
%% Lädt die gemeinsame Datei latex-vorspann.tex mit gesetztem Schalter.

\newif\ifkorrekturansicht
\korrekturansichttrue

\input{../tex-inputs/latex-vorspann}


\section[Elsa Plessner an Arthur Schnitzler, 23. 3. 1897]{L03713 Elsa Plessner an Arthur Schnitzler, 23. 3. 1897}
\nopagebreak\mylabel{L03713v}
\rehead{ }\normalsize\beginnumbering\briefempfaengerindex{Schnitzler, Arthur@\textsc{Schnitzler, Arthur}!zzzPlessner, Elsa@\emph{von Elsa Plessner}!1897-03-234@{23. 3. 1897}|(be}
\toendnotes[C]{\smallbreak\pagebreak[2]}
\correspDesc{Versand  durch Elsa Plessner am 23. 3. 1897 in Wien
\newline{}Erhalt  durch Arthur Schnitzler im Zeitraum [24. 3. 1897
                  – 29. 3. 1897?] in Wien}\toendnotes[C]{\smallbreak}
\Standort{DLA, A:Schnitzler, HS.1985.1.419.}
\physDesc{Brief, 1 Blatt, 3 Seiten, 3250 Zeichen
\newline{}Handschrift: schwarze Tinte, lateinische Kurrent}\toendnotes[C]{\smallbreak}
\pstart
           {\pb}\textcolor{pink}{Wien VIII. Florianigasse N\textsuperscript{o} 44}\oindex{Wien@\textbf{Wien}!VIII., Josefstadt@\textbf{VIII., Josefstadt}!Florianigasse 44@\textbf{Florianigasse 44}, \emph{Wohngebäude}|pw}{}\ledrightnote{\textcolor{pink}{Florianigasse 44}}, den 23. III 97.\pend
           
\pstart{}Hochverehrter Herr Doctor!\pend\vspace{0.5em}
\pstart
           Vorgestern hier angelangt, erhielt ich fast als erstes »Willkommen« Ihre lieben, lieben \label{K_L03713-1v}\edtext{Zeilen}{\lemma{\textnormal{\emph{Zeilen}}}\Cendnote{\textnormal{nicht überliefert}}}\label{K_L03713-1}, die mich gefreut haben – na – wie –
               ! – Das werden Sie sich ja so ungefähr ausmalen können. Wieder ein paar lobende Worte
               von Ihnen zu erhalten, das wirkte wie frisches Wasser auf mich. Also jetzt beginne
               ich wieder zu schwimmen. Vielen, vielen herzlichsten Dank! Die \textcolor{pink}{Meraner}\oindex{Meran@\textbf{Meran}, \emph{Hauptstadt}|pw}{}\ledrightnote{\textcolor{pink}{Meran}} böse Zeit liegt hinter mir, ob eine bessere,
               »heitere« folgt, das ist unbestimmt – aber jedenfalls habe ich wieder Arbeitsfreude.
               Und das ist ja {\pb}das beste Theil, nicht wahr? – – Also, wenn Sie nicht
               seither sich anders besonnen haben, würde ich Sie bitten, mir die Aufgaben zu
               stellen, so, wie Sie es mir im Winter vorgeschlagen haben. Ich hätte ja
               damals gleich von Ihrer unerschöpflichen Liebenswürdigkeit Gebrauch gemacht aber ich
               war so auf dem Hund, körperlich und geistig noch mehr – dass es ein ganz verfehltes
               und zweckloses Verfahren gewesen wäre. Mit dem damaligen Zustand entschuldigen sich
               auch die »\textcolor{green}{Orchideen}\pwindex{Plessner, Elsa 22.\,8.\,1875 Wien – 7.\,5.\,1932 Alicante@\textsc{Plessner, Elsa} (22.\,8.\,1875 Wien – 7.\,5.\,1932 Alicante), \emph{Schriftstellerin}!Orchideen [Schauspiel in drei Akten]@\strich\emph{Orchideen [Schauspiel in drei Akten]}|pw}{}\ledrightnote{\textcolor{green}{Orchideen [Schauspiel in drei Akten]}}« ganz von selbst. – – Jetzt
               wo ich glaube, wieder ein bisschen beisammen zu sein, und alle Gefahr, das \label{K_L03713-2v}\edtext{Ende der \textcolor{blue}{Baskirtseff}\pwindex{Bashkirtseff, Marie 23.\,11.\,1860 Havrontsi – 31.\,10.\,1884 Paris@\textsc{Bashkirtseff, Marie} (23.\,11.\,1860 Havrontsi – 31.\,10.\,1884 Paris), \emph{Malerin}|pw}{}\ledrightnote{\textcolor{blue}{Marie Bashkirtseff}}}{\lemma{\textnormal{\emph{Ende der Baskirtseff}}}\Cendnote{\textnormal{Die Malerin \textcolor{blue}{Marie Bashkirtseff}\pwindex{Bashkirtseff, Marie 23.\,11.\,1860 Havrontsi – 31.\,10.\,1884 Paris@\textsc{Bashkirtseff, Marie} (23.\,11.\,1860 Havrontsi – 31.\,10.\,1884 Paris), \emph{Malerin}|pwk} starb im Alter von knapp 26 Jahren an
                  Tuberkulose. Ihr bis wenige Tage vor dem Tod geführtes \emph{\textcolor{green}{Tagebuch}\pwindex{Bashkirtseff, Marie 23.\,11.\,1860 Havrontsi – 31.\,10.\,1884 Paris@\textsc{Bashkirtseff, Marie} (23.\,11.\,1860 Havrontsi – 31.\,10.\,1884 Paris), \emph{Malerin}!Journal@\strich\emph{Journal}|pwk}} wurde postum publiziert ein vielbeachtetes
                  Buch. Ungewiss bleibt, ob \textcolor{blue}{Plessner}\pwindex{Plessner, Elsa 22.\,8.\,1875 Wien – 7.\,5.\,1932 Alicante@\textsc{Plessner, Elsa} (22.\,8.\,1875 Wien – 7.\,5.\,1932 Alicante), \emph{Schriftstellerin}|pwk} an dieser Stelle ausdrückt, Tuberkulose überstanden zu haben,
                  oder sich nur auf die gesundheitliche Gefahr bezieht, die zu ihrem vorzeitigen Ableben hätte führen können.}}}\label{K_L03713-2} zu kopieren – geschwunden ist, wende ich mich also an Sie und Ihre
               mir so oft bewiesene Güte – nehmen Sie mich in die Schule – stellen Sie mir
               Aufgaben!! – Außerordentlich hat es mich gefreut, dass {\pb}die letzte \label{K_L03713-3v}\edtext{\textcolor{green}{Arbeit}\pwindex{Brief@\emph{Ein Brief}|pwv}{}\ledrightnote{{$\rightarrow$}\emph{\textcolor{green}{Ein Brief}}} im \textcolor{green}{W\textsuperscript{r} Journal}\pwindex{Neues Wiener Journal@\emph{Neues Wiener Journal}|pw}{}\ledrightnote{\textcolor{green}{Neues Wiener Journal}}}{\lemma{\textnormal{\emph{Arbeit … Journal}}}\Cendnote{\textnormal{\textcolor{blue}{E. Plessner}\pwindex{Plessner, Elsa 22.\,8.\,1875 Wien – 7.\,5.\,1932 Alicante@\textsc{Plessner, Elsa} (22.\,8.\,1875 Wien – 7.\,5.\,1932 Alicante), \emph{Schriftstellerin}|pwk}: 
                     \emph{\textcolor{green}{Ein Brief}\pwindex{Brief@\emph{Ein Brief}|pwk}}. In: \emph{\textcolor{green}{Neues Wiener Journal}\pwindex{Neues Wiener Journal@\emph{Neues Wiener Journal}|pwk}}, Nr. 1206, 2. 3. 1897, S. 1–2.}}}\label{K_L03713-3} Ihnen nicht
               missfallen hat. – Sehen Sie, verehrter Herr Doctor, die habe ich wirklich in 1½ Stunden
               hingeschmiert ohne viel an anderes zu denken, als an das \textcolor{green}{W\textsuperscript{r} Journal}\pwindex{Neues Wiener Journal@\emph{Neues Wiener Journal}|pw}{}\ledrightnote{\textcolor{green}{Neues Wiener Journal}}. Von »Talentprobe«
               und so weiter gar keinen Schimmer im Kopf. Und das ist nicht ganz missglückt!!? Das
               freut mich. Muss doch noch ein bisschen künstlerische Grütze haben! D. h. Ideen
               drängen sich mir wirklich eine ganze Menge auf – es fällt mir leicht das
               künstlerische Bild ins Auge – aber die Technik, es herauszuarbeiten!! Da steckts! –
               Darum möchte ich von Ihnen lernen. – Mit dem »\textcolor{green}{Käfig}\pwindex{Plessner, Elsa 22.\,8.\,1875 Wien – 7.\,5.\,1932 Alicante@\textsc{Plessner, Elsa} (22.\,8.\,1875 Wien – 7.\,5.\,1932 Alicante), \emph{Schriftstellerin}!gläserne Käfig. Eine Parabel@\strich\emph{Der gläserne Käfig. Eine Parabel}|pw}{}\ledrightnote{\textcolor{green}{Der gläserne Käfig. Eine Parabel}}« habe ich mir viel und ehrliche Mühe gegeben, auch stylistisch und Sie
               habens gemerkt. Nicht wahr? – Dieselbe Grundidee wie »\textcolor{green}{Orchideen}\pwindex{Plessner, Elsa 22.\,8.\,1875 Wien – 7.\,5.\,1932 Alicante@\textsc{Plessner, Elsa} (22.\,8.\,1875 Wien – 7.\,5.\,1932 Alicante), \emph{Schriftstellerin}!Orchideen [Schauspiel in drei Akten]@\strich\emph{Orchideen [Schauspiel in drei Akten]}|pw}{}\ledrightnote{\textcolor{green}{Orchideen [Schauspiel in drei Akten]}}« – d. h. kaum ein bisschen verändert, denn die
               ließ mich nicht los; aber jetzt habe ich sie geprägt, wie ich glaube, und denke, mit
               ihr fertig zu sein. – Eine klare, schlichte Schreibweise, ohne Mätzchen und
               stylistische Eiertänze, so, wie Sie deren glücklicher Besitzer sind, möchte ich mir
               gerne aneignen. Die jüngstmodernen affectirten Posen sind mir immer zuwider gewesen.
               Daher die von Ihnen oft gerügten Trivialitäten und saloppen Plattheiten. – Leider habe
               ich in jüngster Zeit mit den Dialogen Unglück! Das geht mir stark gegen
               den Strich. Dialoge waren sonst meine Stärke – siehe »\textcolor{green}{Heimweh}\pwindex{Plessner, Elsa 22.\,8.\,1875 Wien – 7.\,5.\,1932 Alicante@\textsc{Plessner, Elsa} (22.\,8.\,1875 Wien – 7.\,5.\,1932 Alicante), \emph{Schriftstellerin}!Heimweh [dreiaktige Tragikomödie]@\strich\emph{Heimweh [dreiaktige Tragikomödie]}|pw}{}\ledrightnote{\textcolor{green}{Heimweh [dreiaktige Tragikomödie]}}«. – Also – hier die Inventur. – – – – – – \pend
           
\pstart
           Mein letzter – d.h. der \label{K_L03713-4v}\edtext{fragliche
                  Brief}{\lemma{\textnormal{\emph{fragliche
                  Brief}}}\Cendnote{\textnormal{Elsa Plessner an Arthur Schnitzler, 13. 1. 1897. }}}\label{K_L03713-4} ärgert mich, weil er mir
               noch immer als Zudringlichkeit – d. h. als Geschmacklosigkeit erscheint. Sie sind nur
               aus Höflichkeit darüber weggegangen. Diese private Ohrenbeichte und intimste \strikeout{Privat} Kundgebung an Sie zu richten – – na, das ist
               mindestens überspannt. – – Und lächerlich noch dazu. Bitte, ich bin so eitel – denken
               Sie besser von mir, als Sie nach dem \label{K_L03713-5v}\edtext{Brief}{\lemma{\textnormal{\emph{Brief}}}\Cendnote{\textnormal{Elsa Plessner an Arthur Schnitzler, 13. 1. 1897. }}}\label{K_L03713-5} Ursache hätten es zu thun
               und nehmen Sie meine aufrichtige Verehrung sowie herzliche Grüsse entgegen.\pend
           \pstart \spacefill\mbox{Elsa Plessner}\pend{}\selectlanguage{ngerman}\endnumbering\briefempfaengerindex{Schnitzler, Arthur@\textsc{Schnitzler, Arthur}!zzzPlessner, Elsa@\emph{von Elsa Plessner}!1897-03-234@{23. 3. 1897}|)be}\mylabel{L03713h}  \normalsize

\doendnotes{C}
\bigskip
\vfill

\clearpage

\footnotesize

\lohead{\textsc{register}}

% Definiere theindex-Environment komplett neu ohne reledmac
\makeatletter
\renewenvironment{theindex}{%
  \section*{\indexname}%
  \setlength{\parindent}{0pt}%
  \setlength{\parskip}{0pt plus 0.3pt}%
  \let\item\@idxitem
}{%
  \clearpage
}
\makeatother

\IfFileExists{\jobname-pw.ind}{\input{\jobname-pw.ind}}{}

\end{document}

      