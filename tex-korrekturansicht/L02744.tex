%% latex-korrekturansicht-vorspann.tex
%% Vorspann für die Korrekturansicht.
%% Lädt die gemeinsame Datei latex-vorspann.tex mit gesetztem Schalter.

\newif\ifkorrekturansicht
\korrekturansichttrue

\input{../tex-inputs/latex-vorspann}


               \section[Paul Goldmann an Arthur Schnitzler, 13. 8. {[}1895{]}]{ Paul Goldmann an Arthur Schnitzler, 13. 8. {[}1895{]}}\nopagebreak\mylabel{v}\rehead{ }\normalsize\beginnumbering\briefempfaengerindex{Schnitzler, Arthur@\textsc{Schnitzler, Arthur}!zzzGoldmann, Paul@\emph{von Paul Goldmann}!1895-08-131@{13. 8. {[}1895{]}}|(be} \toendnotes[C]{\smallbreak\pagebreak[2]} \Standort{DLA, A:Schnitzler, HS.NZ85.1.3165.}
\physDesc{Brief, 1 Blatt, 4 Seiten
\newline{}Handschrift: schwarze Tinte, deutsche Kurrent
\newline{}Schnitzler: mit Bleistift das Jahr »95« vermerkt }\toendnotes[C]{\smallbreak}\pstart
           \noindent{}{\pb}\textcolor{gray}{\textbf{\textbf{\textcolor{brown}{Frankfurter Zeitung}{}\ledrightnote{\textcolor{brown}{Frankfurter Zeitung}}}}}\pend
           \pstart
           \textcolor{gray}{\textbf{(\textcolor{brown}{\begin{otherlanguage}{french}Gazette de Francfort\end{otherlanguage}}{}\ledrightnote{\textcolor{brown}{Frankfurter Zeitung}}.) }}\hfill \textsc{\textcolor{pink}{Toelz}{}\ledrightnote{\textcolor{pink}{Bad Tölz}}}, 13. Auguſt.\pend
           \pstart
           \textcolor{gray}{\textbf{\textbf{\begin{otherlanguage}{french}Fondateur M. \textcolor{blue}{L.
                              Sonnemann}{}\ledrightnote{\textcolor{blue}{Leopold Sonnemann}}\end{otherlanguage}.}}}\pend
           \pstart
           \begin{otherlanguage}{french}\textcolor{gray}{\textbf{\textcolor{green}{Journal}{}\ledrightnote{→\textcolor{green}{Frankfurter Zeitung}} politique,
                        financier,}}\end{otherlanguage}\pend
           \pstart
           \begin{otherlanguage}{french}\textcolor{gray}{\textbf{commercial et littéraire.}}\end{otherlanguage}\pend
           \pstart
           \begin{otherlanguage}{french}\textcolor{gray}{\textbf{\textbf{Paraissant trois fois par jour.}}}\end{otherlanguage}\pend
           \pstart
           \begin{otherlanguage}{french}\textcolor{gray}{\textbf{\textbf{Bureau à \textcolor{pink}{Paris}{}\ledrightnote{\textcolor{pink}{Paris}}:}}}\end{otherlanguage}\pend
           \pstart
           \begin{otherlanguage}{french}\textcolor{gray}{\textbf{\textbf{\textcolor{pink}{24. Rue Feydeau}{}\ledrightnote{\textcolor{pink}{rue Feydeau}}.}}}\end{otherlanguage}\pend
           \pstart\center{}Mein lieber Freund,\pend\pstart
           das wäre ſchön, wenn Du ein wenig hieher kommen wollteſt! Freilich, es wäre ein
               wahres Opfer. Denn der \textcolor{pink}{Ort}{}\ledrightnote{→\textcolor{pink}{Bad Tölz}}
               bietet nichts. Die Berge ſind nur von fern zu ſehen, und ſelbſt dieſe Fernſichten
               ſind in den \textcolor{pink}{öſterreich}{}\ledrightnote{\textcolor{pink}{Österreich}}iſchen \textcolor{pink}{Alpen}{}\ledrightnote{\textcolor{pink}{Alpen}} ſchöner. Man ißt ſchlecht u. wohnt ohne Comfort. Das
               Bade-Pulblicum iſt einfach unmöglich. Ich verkehre nur mit Bauern. {\pb}Endlich ich ſelbſt tr\strikeout{\textcolor{gray}{eb}} treibe Selbſtpein und brüte Schwermuth. Wenn Du freilich trotz alledem kommen
               wollteſt, ſo wärs ſchön u. dankenswerth im höchſten Grade.\pend
           \pstart
           Nach \textsc{\textcolor{pink}{Salzburg}{}\ledrightnote{\textcolor{pink}{Salzburg}}} werde ich nicht kommen können, der Kur wegen.\pend
           \pstart
           Warum willſt Du auf einmal ſo mit aller Gewalt nach dem Norden?\pend
           \pstart
           ich gehe ſo ſtundenweit über Land u. leſe den
                  »\textcolor{green}{Fauſt}{}\ledrightnote{\textcolor{green}{Faust}}«. Wie man in das {\pb}\textcolor{green}{Buch}{}\ledrightnote{→\textcolor{green}{Faust}} hineingewachſen iſt!
               Jetzt iſt Alles ſo einfach und klar, und das Meiſte hat man ſelbſt erlebt. Aber
               gelungen iſt ihm – dem \textsc{\textcolor{blue}{Goethe}{}\ledrightnote{\textcolor{blue}{Johann Wolfgang von Goethe}}} – doch eigentlich nur das Menſchliche u. das Teufliſche (das iſt das ſelbe;
               denn das Teufliſche iſt nur Ironie über das Menſchliche). Aber wo er vom Himmel
               ſpricht, wird er conventionell oder rhetoriſch{\dotsfive}\pend
           \pstart
           \strikeout{\textcolor{gray}{×}\-\textcolor{gray}{×}} Ich hoffe, Du biſt wohlbehalten \label{K_L02744-1v}\edtext{von \textcolor{pink}{Wien}{}\ledrightnote{\textcolor{pink}{Wien}}{ }{\pb}zurückgekehrt}{\lemma{\textnormal{\emph{von Wien zurückgekehrt}}}\Cendnote{\textnormal{Zwischen 11. 8. 1895 und 14. 8. 1895 unterbrach \textcolor{blue}{Schnitzler} seinen Aufenthalt in \textcolor{pink}{Ischl} und kehrte nach \textcolor{pink}{Wien} zurück.}}}\label{K_L02744-1h}. Nun ſchreibſt Du mir wohl bald wieder, beſonders: ob
               u. wann Du kommſt?\pend
           \pstart
           Viele treue Grüße Dir u. \textsc{\textcolor{blue}{Richard}{}\ledrightnote{\textcolor{blue}{Richard Beer-Hofmann}}}\pend
           \pstart
           Dein {\\[\baselineskip]}\spacefill\mbox{Paul Goldmann}\pend
           \leftskip=0em{}\endnumbering\briefempfaengerindex{Schnitzler, Arthur@\textsc{Schnitzler, Arthur}!zzzGoldmann, Paul@\emph{von Paul Goldmann}!1895-08-131@{13. 8. {[}1895{]}}|)be}\mylabel{h}  \normalsize

\doendnotes{C}
\bigskip
\vfill

\clearpage

\footnotesize

\lohead{\textsc{register}}

% Definiere theindex-Environment komplett neu ohne reledmac
\makeatletter
\renewenvironment{theindex}{%
  \section*{\indexname}%
  \setlength{\parindent}{0pt}%
  \setlength{\parskip}{0pt plus 0.3pt}%
  \let\item\@idxitem
}{%
  \clearpage
}
\makeatother

\IfFileExists{\jobname-pw.ind}{\input{\jobname-pw.ind}}{}

\end{document}

      