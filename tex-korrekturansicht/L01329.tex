%% latex-korrekturansicht-vorspann.tex
%% Vorspann für die Korrekturansicht.
%% Lädt die gemeinsame Datei latex-vorspann.tex mit gesetztem Schalter.

\newif\ifkorrekturansicht
\korrekturansichttrue

\input{../tex-inputs/latex-vorspann}


               \section[Franz Blei an Arthur Schnitzler, 17. 10. 1903]{ Franz Blei an Arthur Schnitzler, 17. 10. 1903}\nopagebreak\mylabel{v}\rehead{ }\normalsize\beginnumbering\briefempfaengerindex{Schnitzler, Arthur@\textsc{Schnitzler, Arthur}!zzzBlei, Franz@\emph{von Franz Blei}!1903-10-171@{17. 10. 1903}|(be} \toendnotes[C]{\smallbreak\pagebreak[2]} \Standort{DLA, A:Schnitzler, 66.180.}
\physDesc{Brief, 1 Blatt, 3 Seiten
\newline{}Handschrift: schwarze Tinte, lateinische Kurrent
\newline{}Schnitzler: 1) mit Bleistift beschriftet: »\textsc{Blei}« 2) mit rotem Buntstift einige Unterstreichungen, neben »der
                                    Brief blieb liegen« ein Ausrufezeichen}\toendnotes[C]{\smallbreak}\pstart
           \noindent{}\centering{}{\pb}\textcolor{gray}{\textbf{\textcolor{brown}{DIE ELF SCHARFRICHTER}{}\ledrightnote{\textcolor{brown}{Die elf Scharfrichter}}{ }\textcolor{pink}{MÜNCHEN TÜRKENSTR. 28}{}\ledrightnote{\textcolor{pink}{Türkenstraße}}}}\pend
           \pstart
           \noindent{}\raggedleft{}\textcolor{pink}{Arcisstrasse 19}{}\ledrightnote{\textcolor{pink}{Arcisstraße}}. \pend
           \pstart{}Sehr geehrter Herr Schnitzler,\pend\pstart
           als ich vor acht Tagen den Dialog aus dem \textcolor{green}{Reigen}{}\ledrightnote{\textcolor{green}{Reigen. Zehn Dialoge}}
               auf den Spielplan setzte, geschah es auf die Versicherung der Direktion der \textcolor{brown}{11 S.}{}\ledrightnote{\textcolor{brown}{Die elf Scharfrichter}} hier, dass man das Aufführungsrecht schon
               erworben hätte, bevor ich mich um die Dramaturgie des \textcolor{brown}{Scharfrichtertheater}{}\ledrightnote{\textcolor{brown}{Die elf Scharfrichter}}s kümmerte. Dies stellte sich nun als ein Irrthum
               heraus; ich gab in der Kanzlei den Auftrag, Ihnen von der ersten Aufführung und dem
               Tantièmensatz Mittheilung zu machen – der Brief blieb liegen. Ich muss nun für diese
               Schlampereien um Entschuldigung bitten, obzwar mich keine Schuld an ihnen trifft.
               Natürlich setze ich die \textcolor{green}{Scene}{}\ledrightnote{→\textcolor{green}{Reigen. Zehn Dialoge}}{ }sofort vom Programm, wenn Sie es wünschen, und
               vermag ich die Ungehörigkeit {\pb}nicht anders gut zu machen als
               dass ich um Entschuldigung bitte und mich Ihren Wünschen füge. Der Tantièmenbetrag,
               der pro Vorstellung ungefähr 3–4 Mark ausmacht, wird Ihnen umgehend übersandt werden.
               Sollten Sie die Freundlichkeit \introOben{}haben\introOben{}, nichts gegen die
               weiteren Aufführungen einzuwenden, würde die \textcolor{green}{Scene}{}\ledrightnote{→\textcolor{green}{Reigen. Zehn Dialoge}} bis zum 1. November allabendlich
               gespielt werden und betrügen die Tantièmen dann mindestens 85 Mark, die Ihnen am
                  1. November zugehen.\pend
           \pstart
           Und nochmals: der Vorfall ist unentschuldbar, aber ich bitte Sie, verehrter Herr
               Schnitzler, den Willen, die Sache gut zu machen als Entschuldigung zu nehmen.\pend
           \pstart
           Ihr ganz ergebener{\\[\baselineskip]}\spacefill\mbox{Franz Blei}\pend
           \leftskip=0em{}\pstart
           17. 10. 1903.\pend
           \pstart
           P.S. Vor einigen Tagen schickte ich an die Adresse: \label{K_L01329_1v}\edtext{\textcolor{pink}{\textcolor{gray}{F}rankstraße}{}\ledrightnote{\textcolor{pink}{Frankgasse}}}{\lemma{\textnormal{\emph{Frankstraße}}}\Cendnote{\textnormal{Zu diesem Zeitpunkt wohnte \textcolor{blue}{Schnitzler} bereits mehrere Wochen in der \textcolor{pink}{Spöttelgasse}.}}}\label{K_L01329_1h} ein Briefersuchen einer
                  Miss \textcolor{blue}{Johnson}{}\ledrightnote{\textcolor{blue}{Fanny Johnson}} an Sie, welche für die \textcolor{pink}{englische}{}\ledrightnote{\textcolor{pink}{England}} Bühne arbeitet und der ich die
                  Übertragung des {\pb}\textcolor{green}{Grünen Kakadu}{}\ledrightnote{\textcolor{green}{Der grüne Kakadu. Groteske in einem Akt}} empfohlen habe. Die Dame ersucht
                  Sie um Autorisation und Bedingungen. \spacefill\mbox{B}\pend
           \endnumbering\briefempfaengerindex{Schnitzler, Arthur@\textsc{Schnitzler, Arthur}!zzzBlei, Franz@\emph{von Franz Blei}!1903-10-171@{17. 10. 1903}|)be}\mylabel{h}  \normalsize

\doendnotes{C}
\bigskip
\vfill

\clearpage

\footnotesize

\lohead{\textsc{register}}

% Definiere theindex-Environment komplett neu ohne reledmac
\makeatletter
\renewenvironment{theindex}{%
  \section*{\indexname}%
  \setlength{\parindent}{0pt}%
  \setlength{\parskip}{0pt plus 0.3pt}%
  \let\item\@idxitem
}{%
  \clearpage
}
\makeatother

\IfFileExists{\jobname-pw.ind}{\input{\jobname-pw.ind}}{}

\end{document}

      