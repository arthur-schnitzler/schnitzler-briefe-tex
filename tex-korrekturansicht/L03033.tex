%% latex-korrekturansicht-vorspann.tex
%% Vorspann für die Korrekturansicht.
%% Lädt die gemeinsame Datei latex-vorspann.tex mit gesetztem Schalter.

\newif\ifkorrekturansicht
\korrekturansichttrue

\input{../tex-inputs/latex-vorspann}


\renewcommand{\erwaehntePersonen}{Personen: Felix Salten}
\renewcommand{\erwaehnteOrte}{Orte: Café Pucher, Wien}
\renewcommand{\erwaehnteWerke}{Werke: Burgtheater. (»Cyrano von Bergerac«, romantische Komödie in fünf Aufzügen von Edmond Rostand, deutsch von Ludwig Fulda. – Zum erstenmale aufgeführt am 11. October 1898.), Cyrano von Bergerac. Romantische Komödie in fünf Aufzügen, Deutsches Volkstheater. (»Mutter Erde«, Drama in fünf Acten von Max Halbe. – Zum erstenmale am 8. October 1898.), Im weißen Rößl. Schwank in drei Acten, Mutter Erde. Drama in fünf Acten, Wiener Allgemeine Zeitung}
\section[ Arthur Schnitzler an Felix Salten, {[}14. 10. 1898?{]}]{Arthur Schnitzler an Felix Salten, {[}14. 10. 1898?{]}}
\nopagebreak\mylabel{v}
\rehead{ }\normalsize\beginnumbering\briefempfaengerindex{Salten, Felix@\textsc{Salten, Felix}!zzzSchnitzler, Arthur@\emph{von Arthur Schnitzler}!1898-10-143@{{[}14. 10. 1898?{]}}|(be}
\toendnotes[C]{\smallbreak\pagebreak[2]}\Standort{Wienbibliothek im Rathaus, ZPH 1681, 2.1.516.}
\physDesc{Karte, 233 Zeichen
\newline{}Handschrift: Bleistift, deutsche Kurrent
\newline{}Ordnung: mit Bleistift von unbekannter Hand Nummerierung der Blätter des Konvoluts:
                                    »35« }\toendnotes[C]{\smallbreak}
\pstart
           \noindent{}{\pb}Lieber Freund, vielleicht ſind Sie \label{K_L03033-1v}\edtext{morgen nach dem \textcolor{green}{weißen
                  Röſſl}{}\ledrightnote{\textcolor{green}{Im weißen Rößl. Schwank in drei Acten}}}{\lemma{\textnormal{\emph{morgen … Röſſl}}}\Cendnote{\textnormal{Das erlaubt die Datierung des
                  Korrespondenzstücks. Die Premiere von \emph{\textcolor{green}{Im weißen
                     Rößl}} fand am 15. 10. 1898 statt. 
                  \textcolor{blue}{Schnitzler}  besuchte die Aufführung nicht,
                  sondern erst jene am 11. 11. 1898. Trotzdem 
                  dürfte er sich auf die Premiere und nicht den eigenen Besuch beziehen, da er sich in Folge auf zwei wenige Tage
                  zuvor erschienene Artikel \textcolor{blue}{Salten}s bezieht,
                  über die er einen Monat später längst mit \textcolor{blue}{Salten}
                  gesprochen haben könnte.}}}\label{K_L03033-1h} im
                  \label{K_L03033-2v}\edtext{\textcolor{pink}{Pucher}{}\ledrightnote{\textcolor{pink}{Café Pucher}}}{\lemma{\textnormal{\emph{Pucher}}}\Cendnote{\textnormal{nicht belegt}}}\label{K_L03033-2h}? – Sehr ſchön
               haben Sie über die \label{K_L03033-3v}\edtext{\textcolor{green}{Mutter Erde}{}\ledrightnote{\textcolor{green}{Mutter Erde. Drama in fünf Acten}}}{\lemma{\textnormal{\emph{Mutter Erde}}}\Cendnote{\textnormal{\textcolor{blue}{Felix Salten}: \emph{\textcolor{green}{Deutsches Volkstheater. (»Mutter Erde«, Drama in fünf Acten
                     von Max Halbe. – Zum erstenmale am 8. October 1898.)}} In: \emph{\textcolor{green}{Wiener Allgemeine Zeitung}}, Nr. 6.183, 11. 10. 1898, S. 2}}}\label{K_L03033-3h} u den \label{K_L03033-4v}\edtext{\textcolor{green}{Cyrano}{}\ledrightnote{\textcolor{green}{Cyrano von Bergerac. Romantische Komödie in fünf Aufzügen}}}{\lemma{\textnormal{\emph{Cyrano}}}\Cendnote{\textnormal{\textcolor{blue}{Felix Salten}: \emph{\textcolor{green}{Burgtheater. (»Cyrano von Bergerac«, romantische Komödie in
                              fünf Aufzügen von Edmond Rostand, deutsch von Ludwig Fulda. – Zum erstenmale
                              aufgeführt am 11. October 1898.)}} In: \emph{\textcolor{green}{Wiener Allgemeine Zeitung}}, Nr. 6.185, 13. 10. 1898, S. 2–3.}}}\label{K_L03033-4h}{ }\textcolor{green}{geſchrieben}{}\ledrightnote{{$\rightarrow$}\textcolor{green}{Deutsches Volkstheater. (»Mutter Erde«, Drama in fünf Acten von Max Halbe. – Zum erstenmale am 8. October 1898.)}{\newline}{$\rightarrow$}\textcolor{green}{Burgtheater. (»Cyrano von Bergerac«, romantische Komödie in fünf Aufzügen von Edmond Rostand, deutsch von Ludwig Fulda. – Zum erstenmale aufgeführt am 11. October 1898.)}}
                – beide {\pb}Mal gleich dorthin gegriffen, wo die Dinge zu
               faſſen ſind.\pend
           
\pstart
           Auf Wiederſehen {\\[\baselineskip]}Herzlichſt Ihr {\\[\baselineskip]}\spacefill\mbox{A. S.}\pend
           \leftskip=0em{}\endnumbering\briefempfaengerindex{Salten, Felix@\textsc{Salten, Felix}!zzzSchnitzler, Arthur@\emph{von Arthur Schnitzler}!1898-10-143@{{[}14. 10. 1898?{]}}|)be}\mylabel{h}  \normalsize

\doendnotes{C}
\bigskip
\vfill

\clearpage

\footnotesize

\lohead{\textsc{register}}

% Definiere theindex-Environment komplett neu ohne reledmac
\makeatletter
\renewenvironment{theindex}{%
  \section*{\indexname}%
  \setlength{\parindent}{0pt}%
  \setlength{\parskip}{0pt plus 0.3pt}%
  \let\item\@idxitem
}{%
  \clearpage
}
\makeatother

\IfFileExists{\jobname-pw.ind}{\input{\jobname-pw.ind}}{}

\end{document}

      