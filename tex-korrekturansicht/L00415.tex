%% latex-korrekturansicht-vorspann.tex
%% Vorspann für die Korrekturansicht.
%% Lädt die gemeinsame Datei latex-vorspann.tex mit gesetztem Schalter.

\newif\ifkorrekturansicht
\korrekturansichttrue

\input{../tex-inputs/latex-vorspann}


               \section[Arthur Schnitzler an Richard Beer-Hofmann, {[}19. 1. 1895?{]}]{ Arthur Schnitzler an Richard Beer-Hofmann, {[}19. 1. 1895?{]}}\nopagebreak\mylabel{v}\rehead{ }\normalsize\beginnumbering\briefempfaengerindex{Beer-Hofmann, Richard@\textsc{Beer-Hofmann, Richard}!zzzSchnitzler, Arthur@\emph{von Arthur Schnitzler}!1895-01-191@{{[}19. 1. 1895?{]}}|(be} \toendnotes[C]{\smallbreak\pagebreak[2]} \Standort{YCGL, MSS 31.}
\physDesc{Brief, 1 Blatt, 3 Seiten
\newline{}Handschrift: Bleistift, deutsche Kurrent}\buchAbdrucke{\weitereDrucke{Arthur Schnitzler, Richard Beer-Hofmann: \emph{Briefwechsel 1891–1931}. Hg. Konstanze Fliedl. Wien, Zürich: \emph{Europaverlag} 1992, S. 71.} }\toendnotes[C]{\smallbreak}\pstart
           \noindent{}{\pb}Lieber Richard. Ko{\geminationm}en Sie in die
               Loge\pend
           \pstart
           \centering{}\textsc{Nr. eilf}, I. Stock links.\pend
           \pstart
           \noindent{}War nichts {\pb}andres zu beko{\geminationm}en. –\pend
           \pstart
           Hoffe, zur \label{K_L00415_1v}\edtext{Repartirung}{\lemma{\textnormal{\emph{Repartirung}}}\Cendnote{\textnormal{Aufteilung (der Kosten)}}}\label{K_L00415_1h}, daſs mein
                  \textcolor{blue}{Bruder}{}\ledrightnote{→\textcolor{blue}{Julius Schnitzler}} u \textcolor{blue}{Schwägerin}{}\ledrightnote{→\textcolor{blue}{Helene Schnitzler}} mitko{\geminationm}en.\pend
           \pstart
           Die Loge hab ich. –\pend
           \pstart
           Nachher sind wir, dh. Sie, \textcolor{blue}{Qualle}{}\ledrightnote{→\textcolor{blue}{Adele Sandrock}}, {\pb}\textcolor{blue}{Schweſter}{}\ledrightnote{→\textcolor{blue}{Wilhelmine Sandrock}} u \textcolor{blue}{Salten}{}\ledrightnote{\textcolor{blue}{Felix Salten}}{ }\introOben{}u ich\introOben{} zusa{\geminationm}en. Bitte \uuline{dringend}{ }\uuline{keine}{ }\label{K_L00415_2v}\edtext{Elegance}{\lemma{\textnormal{\emph{Elegance}}}\Cendnote{\textnormal{Das Korrespondenzstück ist undatiert, die Hinweise sind
                  spärlich. Der Umstand, dass \textcolor{blue}{Schnitzler} das
                  Reglement zur Kleidungswahl bestimmt, deutet auf eine von ihm organisierte
                  Festlichkeit. Naheliegend ist dafür der 19. 1. 1895, jener Tag, an
                  dem in der Zeitung steht, dass die \emph{\textcolor{green}{Liebelei}} zur
                  Aufführung am \emph{\textcolor{brown}{Burgtheater}} angenommen worden ist.
                  An diesem Abend trafen sich die Genannten – ohne \textcolor{blue}{Willy Sandrock}, dafür aber mit \textcolor{blue}{Robert
                     Nhil}. Grund für die Loge im Theater wäre dann wiederum, dass am selben
                  Abend \textcolor{blue}{Josef Giampietro} in der Premiere von \emph{\textcolor{green}{Die Kameraden}} seine Rolle offensichtlich \textcolor{blue}{Schnitzler} nachahmend anlegte.}}}\label{K_L00415_2h}.\pend
           \pstart
           Herzlich Ihr{\\[\baselineskip]}\spacefill\mbox{Arthur}\pend
           \leftskip=0em{}\pstart
           \noindent{}(Ich gehe ſchwarzes \textsc{Sacco}.)\pend
           \pstart
           Vielleicht doch \textsc{smoking}\pend
           \endnumbering\briefempfaengerindex{Beer-Hofmann, Richard@\textsc{Beer-Hofmann, Richard}!zzzSchnitzler, Arthur@\emph{von Arthur Schnitzler}!1895-01-191@{{[}19. 1. 1895?{]}}|)be}\mylabel{h}  \normalsize

\doendnotes{C}
\bigskip
\vfill

\clearpage

\footnotesize

\lohead{\textsc{register}}

% Definiere theindex-Environment komplett neu ohne reledmac
\makeatletter
\renewenvironment{theindex}{%
  \section*{\indexname}%
  \setlength{\parindent}{0pt}%
  \setlength{\parskip}{0pt plus 0.3pt}%
  \let\item\@idxitem
}{%
  \clearpage
}
\makeatother

\IfFileExists{\jobname-pw.ind}{\input{\jobname-pw.ind}}{}

\end{document}

      