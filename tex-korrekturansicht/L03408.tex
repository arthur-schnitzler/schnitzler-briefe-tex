%% latex-korrekturansicht-vorspann.tex
%% Vorspann für die Korrekturansicht.
%% Lädt die gemeinsame Datei latex-vorspann.tex mit gesetztem Schalter.

\newif\ifkorrekturansicht
\korrekturansichttrue

\input{../tex-inputs/latex-vorspann}


\renewcommand{\erwaehntePersonen}{Personen: Richard Metzl, Max Reinhardt, Ottilie Salten, Olga Schnitzler}
\renewcommand{\erwaehnteInstitutionen}{Institutionen: Kleines Theater, Neues Theater}
\renewcommand{\erwaehnteOrte}{Orte: Edmund-Weiß-Gasse 7, Florenz, I., Innere Stadt, Mariazell, Pötzleinsdorferstraße, Starkfriedgassse, Theater an der Wien, Wien, XVIII., Währing}
\renewcommand{\erwaehnteWerke}{Werke: Die Zeit, Ein Sommernachtstraum. Komödie in fünf Aufzügen, Schiller-Feier, Tagebuch, Zum großen Wurstel. Burleske in einem Akt}
\section[ Felix Salten an Arthur Schnitzler, 6. 5. 1905]{Felix Salten an Arthur Schnitzler, 6. 5. 1905}
\nopagebreak\mylabel{v}
\rehead{ }\normalsize\beginnumbering\briefempfaengerindex{Schnitzler, Arthur@\textsc{Schnitzler, Arthur}!zzzSalten, Felix@\emph{von Felix Salten}!1905-05-061@{6. 5. 1905}|(be}
\toendnotes[C]{\smallbreak\pagebreak[2]}\Standort{CUL, Schnitzler, B 89, B 1.}
\physDesc{Postkarte, 558 Zeichen
\newline{}Handschrift: Bleistift, lateinische Kurrent
\newline{}Versand: Stempel: »\nobreak{}\oindex{I., Innere Stadt@\textbf{I., Innere Stadt}, \emph{A.ADM3}|pwk}Wien 1/1 1, 6. 5. 05, 11–12 N.\nobreak{}«.  
\newline{}Ordnung: mit Bleistift von unbekannter Hand nummeriert: »200« }\toendnotes[C]{\smallbreak}\pstart{}{\pb}Herrn D\textsuperscript{r} Arthur Schnitzler\pend{}\pstart{}\textcolor{pink}{Wien XVIII.}{}\ledrightnote{\textcolor{pink}{XVIII., Währing}}\pend{}\pstart{}\textcolor{pink}{Spoettelgaße 7}{}\ledrightnote{\textcolor{pink}{Edmund-Weiß-Gasse 7}}\pend{}
{\bigskip}
\pstart
           \raggedleft{}{\pb}6/5 05\pend
           
\pstart
           Lieber – wir \label{K_L03408-1v}\edtext{wohnen
               schon \textcolor{pink}{Pötzleinsdorferstraße 88}{}\ledrightnote{\textcolor{pink}{Pötzleinsdorferstraße}}}{\lemma{\textnormal{\emph{wohnen schon Pötzleinsdorferstraße 88}}}\Cendnote{\textnormal{Bei dieser Adresse – ebenso wie bei der
                     \textcolor{pink}{Starkfriedgasse 12} im Vorjahr, die 650
                  Meter entfernt liegt – handelte es sich um Sommersitze, die nur für die warme
                  Jahreszeit angemietet wurden.}}}\label{K_L03408-1h}. Spaziergänge, Sommerpläne u. s. w. können
               jetzt besprochen werden. Nach dem \label{K_L03408-2v}\edtext{\textcolor{green}{Sommernachtstraum}{}\ledrightnote{\textcolor{green}{Ein Sommernachtstraum. Komödie in fünf Aufzügen}}}{\lemma{\textnormal{\emph{Sommernachtstraum}}}\Cendnote{\textnormal{Das \textcolor{green}{Stück} – in der Inszenierung von \textcolor{blue}{Max Reinhardt} – wurde in \textcolor{pink}{Wien} erstmals am 20. 5. 1905 beim Gastspiel des \emph{\textcolor{brown}{Kleinen
                     Theaters}} und des \emph{\textcolor{brown}{Neuen Theaters}} am \textcolor{pink}{Theater an der Wien} gegeben. \textcolor{blue}{Schnitzler} besuchte die Aufführung, vgl. A. S.: \emph{Tagebuch}, 20. 5. 1905.}}}\label{K_L03408-2h} wollen
               wir nach \textcolor{pink}{Maria Zell}{}\ledrightnote{\textcolor{pink}{Mariazell}}. (Ersatz für \textcolor{pink}{Florenz}{}\ledrightnote{\textcolor{pink}{Florenz}}, das aus Zeitmangel entfiel) Vielleicht machen wir die
                  \label{K_L03408-3v}\edtext{Parthie zu \textcolor{blue}{viert}{}\ledrightnote{{$\rightarrow$}\textcolor{blue}{Olga Schnitzler}{\newline}{$\rightarrow$}\textcolor{blue}{Ottilie Salten}}}{\lemma{\textnormal{\emph{Parthie zu viert}}}\Cendnote{\textnormal{Das Vorhaben verschob sich bis
                  Ende Juli 1905. Letztlich fuhr nur \textcolor{blue}{Salten} mit seinem Schwager \textcolor{blue}{Richard Metzl},
                  vgl. Felix Salten und Richard Metzl an Arthur
               Schnitzler, [30. 7. 1905?]; A. S.: \emph{Tagebuch}, 31. 7. 1905. Die Möglichkeit einer 
                  gemeinsamen Reise stand aber bis kurz vorher im Raum, vgl. Arthur Schnitzler an Felix Salten, 20. 7. 1905.}}}\label{K_L03408-3h}, wie’s ja besprochen war?\pend
           
\pstart
           Schreiben Sie, wenn man Sie am besten trifft, und wann Ihre \textcolor{blue}{Frau}{}\ledrightnote{{$\rightarrow$}\textcolor{blue}{Olga Schnitzler}} am wenigsten gestört wird. Wir wollen
                  \label{K_L03408-4v}\edtext{bald einmal Vormittag oder
               Nachmittag zu Ihnen}{\lemma{\textnormal{\emph{bald … Ihnen}}}\Cendnote{\textnormal{Ein solcher Besuch
                  ist nicht im \emph{\textcolor{green}{Tagebuch}}{ }\textcolor{blue}{Schnitzler}s belegt.}}}\label{K_L03408-4h}.
               – Die gewünschten \label{K_L03408-5v}\edtext{\textcolor{green}{12 Exemplare}{}\ledrightnote{{$\rightarrow$}\textcolor{green}{Die Zeit}{\newline}{$\rightarrow$}\textcolor{green}{Zum großen Wurstel. Burleske in einem Akt}{\newline}{$\rightarrow$}\textcolor{green}{Schiller-Feier}}}{\lemma{\textnormal{\emph{12 Exemplare}}}\Cendnote{\textnormal{siehe Arthur Schnitzler an Felix Salten, 29. 4. 1905}}}\label{K_L03408-5h} haben Sie wol schon erhalten?\pend
           
\pstart
           Herzlich Ihr {\\[\baselineskip]}\spacefill\mbox{S.}\pend
           \leftskip=0em{}\endnumbering\briefempfaengerindex{Schnitzler, Arthur@\textsc{Schnitzler, Arthur}!zzzSalten, Felix@\emph{von Felix Salten}!1905-05-061@{6. 5. 1905}|)be}\mylabel{h}  \normalsize

\doendnotes{C}
\bigskip
\vfill

\clearpage

\footnotesize

\lohead{\textsc{register}}

% Definiere theindex-Environment komplett neu ohne reledmac
\makeatletter
\renewenvironment{theindex}{%
  \section*{\indexname}%
  \setlength{\parindent}{0pt}%
  \setlength{\parskip}{0pt plus 0.3pt}%
  \let\item\@idxitem
}{%
  \clearpage
}
\makeatother

\IfFileExists{\jobname-pw.ind}{\input{\jobname-pw.ind}}{}

\end{document}

      