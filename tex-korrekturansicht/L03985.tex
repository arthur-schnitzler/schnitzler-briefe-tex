%% latex-korrekturansicht-vorspann.tex
%% Vorspann für die Korrekturansicht.
%% Lädt die gemeinsame Datei latex-vorspann.tex mit gesetztem Schalter.

\newif\ifkorrekturansicht
\korrekturansichttrue

\input{../tex-inputs/latex-vorspann}


\section[Arthur Schnitzler an Berta Zuckerkandl, {[}28. oder 29. 8. 1929?{]}]{L03985 Arthur Schnitzler an Berta Zuckerkandl, {[}28. oder 29. 8. 1929?{]}}
\nopagebreak\mylabel{L03985v}
\rehead{ }\normalsize\beginnumbering\briefempfaengerindex{Zuckerkandl, Berta@\textsc{Zuckerkandl, Berta}!zzzSchnitzler, Arthur@\emph{von Arthur Schnitzler}!1929-08-291@{{[}zwischen 27. und 29. 8. 1929?{]}}|(be}
\toendnotes[C]{\smallbreak\pagebreak[2]}
\correspDesc{Versand  durch Arthur Schnitzler im Zeitraum [zwischen 27. und
                  29. 8. 1929?] in Caux
\newline{}Erhalt  durch Berta Zuckerkandl \textbf{Ort fehlend} }\toendnotes[C]{\smallbreak}
\Standort{Wien, Österreichische Nationalbibliothek, 405/B78/7 LIT MAG.}
\physDesc{Brief, 1 Blatt, 2 Seiten, 2602 Zeichen
\newline{}Handschrift: Bleistift, lateinische Kurrent}\toendnotes[C]{\smallbreak}
\pstart
           \noindent{}{\pb}liebe verehrte Freundin, meiner \label{K_L03985-1v}\edtext{Depesche}{\lemma{\textnormal{\emph{Depesche}}}\Cendnote{\textnormal{nicht
                  überliefert}}}\label{K_L03985-1} muß ich ergänzend beifügen: \pend
           
\pstart
           Es ist klar, daſs sowohl \textcolor{blue}{Mauget}\pwindex{Mauget, Irénée 1881 Angoulême – 1976@\textsc{Mauget, Irénée} (1881 Angoulême – 1976), \emph{Herausgeber, Theaterdirektor, Schriftsteller}|pw}{}\ledrightnote{\textcolor{blue}{Irénée Mauget}} als \textcolor{blue}{Rémon}\pwindex{Rémon, Maurice 27.\,11.\,1861 Paris – 20.\,6.\,1945 Mérignac@\textsc{Rémon, Maurice} (27.\,11.\,1861 Paris – 20.\,6.\,1945 Mérignac), \emph{Übersetzer}|pw}{}\ledrightnote{\textcolor{blue}{Maurice Rémon}} die Aufführung des \textcolor{green}{Reigen}\pwindex{Schnitzler, Arthur 15. 5. 1862 Wien – 21. 10. 1931 ebd.@\textsc{Schnitzler, Arthur} (15. 5. 1862 Wien – 21. 10. 1931 ebd.), \emph{Schriftsteller, Mediziner}!Reigen. Zehn Dialoge@\strich\emph{Reigen. Zehn Dialoge}|pw}{}\ledrightnote{\textcolor{green}{Reigen. Zehn Dialoge}} haben möchten; – es ist auch möglich, daſs sie gut
               würde; – wir haben aber dafür keinerlei Garantien – als die Überzeugung – und den
               Wunsch der beiden \textcolor{blue}{Herren}\pwindex{Mauget, Irénée 1881 Angoulême – 1976@\textsc{Mauget, Irénée} (1881 Angoulême – 1976), \emph{Herausgeber, Theaterdirektor, Schriftsteller}|pwv}\pwindex{Rémon, Maurice 27.\,11.\,1861 Paris – 20.\,6.\,1945 Mérignac@\textsc{Rémon, Maurice} (27.\,11.\,1861 Paris – 20.\,6.\,1945 Mérignac), \emph{Übersetzer}|pwv}{}\ledrightnote{{$\rightarrow$}\emph{\textcolor{blue}{Irénée Mauget}}{\newline}{$\rightarrow$}\emph{\textcolor{blue}{Maurice Rémon}}} – was uns in diesem Fall nicht genügen kann. Ich habe die \textcolor{green}{Übersetzg}\pwindex{Schnitzler, Arthur 15. 5. 1862 Wien – 21. 10. 1931 ebd.@\textsc{Schnitzler, Arthur} (15. 5. 1862 Wien – 21. 10. 1931 ebd.), \emph{Schriftsteller, Mediziner}!ronde. Dix scènes dialoguées@\strich\emph{La ronde. Dix scènes dialoguées}|pwv}{}\ledrightnote{{$\rightarrow$}\emph{\textcolor{green}{La ronde. Dix scènes dialoguées}}}{ }\textcolor{blue}{Rémons}\pwindex{Rémon, Maurice 27.\,11.\,1861 Paris – 20.\,6.\,1945 Mérignac@\textsc{Rémon, Maurice} (27.\,11.\,1861 Paris – 20.\,6.\,1945 Mérignac), \emph{Übersetzer}|pw}{}\ledrightnote{\textcolor{blue}{Maurice Rémon}} wieder durchgesehen; – überdies hat ihn,
               außer Frau \textcolor{blue}{Pollaczek}\pwindex{Pollaczek, Clara Katharina 15.\,1.\,1875 Wien – 22.\,7.\,1951 ebd.@\textsc{Pollaczek, Clara Katharina} (15.\,1.\,1875 Wien – 22.\,7.\,1951 ebd.), \emph{Schriftstellerin}|pw}{}\ledrightnote{\textcolor{blue}{Clara Katharina Pollaczek}}, auch \uline{Mme \textcolor{blue}{Clauser}\pwindex{Clauser, Suzanne 16.\,5.\,1898 Wien – 11.\,9.\,1981 Paris@\textsc{Clauser, Suzanne} (16.\,5.\,1898 Wien – 11.\,9.\,1981 Paris), \emph{Schriftstellerin, Übersetzerin}|pw}{}\ledrightnote{\textcolor{blue}{Suzanne Clauser}}} aufs sorgfältigste durchgesehen – schade dſs ich Ihnen von hier aus kein mit
               Anmerkungen versehenes \textcolor{green}{Exemplar}\pwindex{Schnitzler, Arthur 15. 5. 1862 Wien – 21. 10. 1931 ebd.@\textsc{Schnitzler, Arthur} (15. 5. 1862 Wien – 21. 10. 1931 ebd.), \emph{Schriftsteller, Mediziner}!ronde. Dix scènes dialoguées@\strich\emph{La ronde. Dix scènes dialoguées}|pwv}{}\ledrightnote{{$\rightarrow$}\emph{\textcolor{green}{La ronde. Dix scènes dialoguées}}} zusenden ka{\geminationn}; – Sie würden sehen, daſs
               eine Revision (um mich milde auszudrücken) unumgänglich ist. Das \textcolor{green}{Buch}\pwindex{Schnitzler, Arthur 15. 5. 1862 Wien – 21. 10. 1931 ebd.@\textsc{Schnitzler, Arthur} (15. 5. 1862 Wien – 21. 10. 1931 ebd.), \emph{Schriftsteller, Mediziner}!ronde. Dix scènes dialoguées@\strich\emph{La ronde. Dix scènes dialoguées}|pwv}{}\ledrightnote{{$\rightarrow$}\emph{\textcolor{green}{La ronde. Dix scènes dialoguées}}} ist \uline{vergriffen}, es ist zu befürchten, daß \label{K_L03985-2v}\edtext{\textcolor{blue}{Stock}\pwindex{Stock, Pierre-Victor 22.\,7.\,1861 18. arrondissement [Paris] – 30.\,4.\,1943 Saint-Antoine Hospital@\textsc{Stock, Pierre-Victor} (22.\,7.\,1861 18. arrondissement [Paris] – 30.\,4.\,1943 Saint-Antoine Hospital), \emph{Verlagsinhaber}|pw}{}\ledrightnote{\textcolor{blue}{Pierre-Victor Stock}}}{\lemma{\textnormal{\emph{Stock}}}\Cendnote{\textnormal{Tatsächlich gehörte der Verlag \emph{\textcolor{brown}{Éditions Stock}\orgindex{Éditions Stock@Éditions Stock|pwk}} nach einem Konkurs
                     1921 nicht mehr \textcolor{blue}{Pierre-Victor
                     Stock}\pwindex{Stock, Pierre-Victor 22.\,7.\,1861 18. arrondissement [Paris] – 30.\,4.\,1943 Saint-Antoine Hospital@\textsc{Stock, Pierre-Victor} (22.\,7.\,1861 18. arrondissement [Paris] – 30.\,4.\,1943 Saint-Antoine Hospital), \emph{Verlagsinhaber}|pwk}, sondern \textcolor{blue}{Maurice Delamain}\pwindex{Delamain, Maurice 28.\,4.\,1883 Jarnac – 2.\,5.\,1974 Paris@\textsc{Delamain, Maurice} (28.\,4.\,1883 Jarnac – 2.\,5.\,1974 Paris), \emph{Kritiker, Rechtsanwalt, Verleger}|pwk} und
                     \textcolor{blue}{Jacques Chardonne}\pwindex{Chardonne, Jacques 2.\,1.\,1884 Barbezieux-Saint-Hilaire – 29.\,5.\,1968 La Frette-sur-Seine@\textsc{Chardonne, Jacques} (2.\,1.\,1884 Barbezieux-Saint-Hilaire – 29.\,5.\,1968 La Frette-sur-Seine), \emph{Schriftsteller, Verleger}|pwk}.}}}\label{K_L03985-2}, leichtfertig
               wie Verleger sind – im Fall einer Aufführung die schlechte \textcolor{green}{Übersetzg}\pwindex{Schnitzler, Arthur 15. 5. 1862 Wien – 21. 10. 1931 ebd.@\textsc{Schnitzler, Arthur} (15. 5. 1862 Wien – 21. 10. 1931 ebd.), \emph{Schriftsteller, Mediziner}!ronde. Dix scènes dialoguées@\strich\emph{La ronde. Dix scènes dialoguées}|pwv}{}\ledrightnote{{$\rightarrow$}\emph{\textcolor{green}{La ronde. Dix scènes dialoguées}}} so wie sie ist neu herausgeben
               wird, was ausschließlich ein Schaden für mich wäre. Wir haben so lange gewartet – und
               ich bin so gar nicht aufführungshungrig – ganz besonders hinsichtlich des \textcolor{green}{Reigen}\pwindex{Schnitzler, Arthur 15. 5. 1862 Wien – 21. 10. 1931 ebd.@\textsc{Schnitzler, Arthur} (15. 5. 1862 Wien – 21. 10. 1931 ebd.), \emph{Schriftsteller, Mediziner}!Reigen. Zehn Dialoge@\strich\emph{Reigen. Zehn Dialoge}|pw}{}\ledrightnote{\textcolor{green}{Reigen. Zehn Dialoge}}. Ich selbst bin fern davon den \textcolor{green}{Reigen}\pwindex{Schnitzler, Arthur 15. 5. 1862 Wien – 21. 10. 1931 ebd.@\textsc{Schnitzler, Arthur} (15. 5. 1862 Wien – 21. 10. 1931 ebd.), \emph{Schriftsteller, Mediziner}!Reigen. Zehn Dialoge@\strich\emph{Reigen. Zehn Dialoge}|pw}{}\ledrightnote{\textcolor{green}{Reigen. Zehn Dialoge}} zu unterschätzen – aber es ist zu klar,
               daſs man ihn nur aus geschäftlichen Gründen herausbringen will – sonst wäre ja schon
               manche Gelegenheit gewesen. – Und nicht einmal ein \label{K_L03985-3v}\edtext{\begin{otherlanguage}{french}A valoir\end{otherlanguage}}{\lemma{\textnormal{\emph{A valoir}}}\Cendnote{\textnormal{französisch: Vorschuss}}}\label{K_L03985-3}? We{\geminationn} Herr \textcolor{blue}{Mauget}\pwindex{Mauget, Irénée 1881 Angoulême – 1976@\textsc{Mauget, Irénée} (1881 Angoulême – 1976), \emph{Herausgeber, Theaterdirektor, Schriftsteller}|pw}{}\ledrightnote{\textcolor{blue}{Irénée Mauget}} so
               besondern Werth auf Aufführung des \textcolor{green}{Reigen}\pwindex{Schnitzler, Arthur 15. 5. 1862 Wien – 21. 10. 1931 ebd.@\textsc{Schnitzler, Arthur} (15. 5. 1862 Wien – 21. 10. 1931 ebd.), \emph{Schriftsteller, Mediziner}!Reigen. Zehn Dialoge@\strich\emph{Reigen. Zehn Dialoge}|pw}{}\ledrightnote{\textcolor{green}{Reigen. Zehn Dialoge}} legt,
               so müsst er sich allermindestens dazu entschließen – {\pb}und 12.000 \strikeout{Schill} Francs \introOben{}(für mich (resp.uns){[}){]}\introOben{}, kaum 3000 Sch. wäre \textcolor{gray}{voraus} wenig. Und dieses \begin{otherlanguage}{french}à
                  valoir\end{otherlanguage} natürlich \uline{bei} Abschluss, also \uline{vor} Aufführung. Principielle Abneigungen der Verleger
               u. Directoren lass ich nicht gelten; – dann eben nicht. Die Erfahrungen Ihres \textcolor{blue}{Gewährsmanns}\pwindex{?? [Person, die gute Erfahrungen mit Maurice Rémon gemacht hat] @\textsc{?? [Person, die gute Erfahrungen mit Maurice Rémon gemacht hat]}|pwv}{}\ledrightnote{{$\rightarrow$}\emph{\textcolor{blue}{?? [Person, die gute Erfahrungen mit Maurice Rémon gemacht hat]}}} (mit \textcolor{blue}{Remon}\pwindex{Rémon, Maurice 27.\,11.\,1861 Paris – 20.\,6.\,1945 Mérignac@\textsc{Rémon, Maurice} (27.\,11.\,1861 Paris – 20.\,6.\,1945 Mérignac), \emph{Übersetzer}|pw}{}\ledrightnote{\textcolor{blue}{Maurice Rémon}}) sind mir nicht maßgebend (auch halt ich kleine
               Erinnerungstäuschungen nicht für ausgeschlossen.) – Auf ein \begin{otherlanguage}{french}a
                  valoir\end{otherlanguage} würd ich gern verzichten – \uline{wo ich mit
                  völligem Vertrauen bei der Sache wäre}. –\pend
           
\pstart
           Also nochmals: – Bedingungen 1)
               Revision der \textcolor{blue}{Rémon}\pwindex{Rémon, Maurice 27.\,11.\,1861 Paris – 20.\,6.\,1945 Mérignac@\textsc{Rémon, Maurice} (27.\,11.\,1861 Paris – 20.\,6.\,1945 Mérignac), \emph{Übersetzer}|pw}{}\ledrightnote{\textcolor{blue}{Maurice Rémon}}ſchen \textcolor{green}{Übersetzung}\pwindex{Schnitzler, Arthur 15. 5. 1862 Wien – 21. 10. 1931 ebd.@\textsc{Schnitzler, Arthur} (15. 5. 1862 Wien – 21. 10. 1931 ebd.), \emph{Schriftsteller, Mediziner}!ronde. Dix scènes dialoguées@\strich\emph{La ronde. Dix scènes dialoguées}|pwv}{}\ledrightnote{{$\rightarrow$}\emph{\textcolor{green}{La ronde. Dix scènes dialoguées}}} (eventuell durch Madame \textcolor{blue}{Clauser}\pwindex{Clauser, Suzanne 16.\,5.\,1898 Wien – 11.\,9.\,1981 Paris@\textsc{Clauser, Suzanne} (16.\,5.\,1898 Wien – 11.\,9.\,1981 Paris), \emph{Schriftstellerin, Übersetzerin}|pw}{}\ledrightnote{\textcolor{blue}{Suzanne Clauser}} zu besorgen – oder \textcolor{blue}{Bourdet}\pwindex{Bourdet, Édouard 26.\,10.\,1887 Saint-Germain-en-Laye – 17.\,1.\,1945 Paris@\textsc{Bourdet, Édouard} (26.\,10.\,1887 Saint-Germain-en-Laye – 17.\,1.\,1945 Paris), \emph{Schriftsteller}|pw}{}\ledrightnote{\textcolor{blue}{Édouard Bourdet}}?) 2) \begin{otherlanguage}{french}a valoir\end{otherlanguage}. 3)
               die Tantiementheilung zwischen Autor und Übersetzer, wie in den Verträgen der \textcolor{brown}{Société}\orgindex{Société des Auteurs et Compositeurs Dramatiques@Société des Auteurs et Compositeurs Dramatiques|pwv}{}\ledrightnote{{$\rightarrow$}\emph{\textcolor{brown}{Société des Auteurs et Compositeurs Dramatiques}}} üblich. –\pend
           
\pstart
           – Darf ich in
               diesem Zusa{\geminationm}enhang Ihnen den Vorschlag machen, Ihre
               15perzentige Provision auf eine 25 {\%} zu erhöhen, \introOben{}so\introOben{} daſs Sie \introOben{}von meinen\introOben{} in allen Fällen, wo Sie
               liebe Freundin meine Agenden in \textcolor{pink}{Frankreich}\oindex{Frankreich@\textbf{Frankreich}|pw}{}\ledrightnote{\textcolor{pink}{Frankreich}} führen, ein \uline{Viertel} der auf mich entfallenden Einnahmen erhalten? Sie haben soviel Mühe
               mit mir – nun, hoffentlich rentirt sichs einmal für uns Beide. –\pend
           
\pstart
           Das mir ein
               geschäftlicher Brief – morgen schreib ich weitere – \uline{meine Adresse von \label{K_L03985-4v}\edtext{So{\geminationn}tag{ }\introOben{}31. d.\introOben{} an \textcolor{pink}{Territet}\oindex{Territet@\textbf{Territet}|pw}{}\ledrightnote{\textcolor{pink}{Territet}}, \textcolor{pink}{Hotel des Alpes}\oindex{Hôtel des Alpes-Grand Hôtel@\textbf{Hôtel des Alpes-Grand Hôtel}, \emph{Hotel}|pw}{}\ledrightnote{\textcolor{pink}{Hôtel des Alpes-Grand Hôtel}}}{\lemma{\textnormal{\emph{Sonntag … Alpes}}}\Cendnote{\textnormal{Das Korrespondenzstück ist undatiert. Am
                     27. 8. 1929
                  besichtigten \textcolor{blue}{Schnitzler} und \textcolor{blue}{Clara Katharina Pollaczek}\pwindex{Pollaczek, Clara Katharina 15.\,1.\,1875 Wien – 22.\,7.\,1951 ebd.@\textsc{Pollaczek, Clara Katharina} (15.\,1.\,1875 Wien – 22.\,7.\,1951 ebd.), \emph{Schriftstellerin}|pwk} das \textcolor{pink}{Hôtel des Alpes-Grand Hôtel}\oindex{Hôtel des Alpes-Grand Hôtel@\textbf{Hôtel des Alpes-Grand Hôtel}, \emph{Hotel}|pwk}, das sie aber
                  erst am Montag, dem 2. 9. 1929 besiedeln. Dadurch lässt sich die Zeitspanne, in der der
                  Brief verfasst sein kann, auf die dazwischen liegenden Tage eingrenzen.}}}\label{K_L03985-4}}. Es
               ist herrlich hier. Tausend Grüße, auch von Frau \textcolor{blue}{Pollaczek}\pwindex{Pollaczek, Clara Katharina 15.\,1.\,1875 Wien – 22.\,7.\,1951 ebd.@\textsc{Pollaczek, Clara Katharina} (15.\,1.\,1875 Wien – 22.\,7.\,1951 ebd.), \emph{Schriftstellerin}|pw}{}\ledrightnote{\textcolor{blue}{Clara Katharina Pollaczek}} alles herzliche.\pend
           
\pstart
           Ihr dankbarer{\\[\baselineskip]}\spacefill\mbox{ArtSch}\pend
           \leftskip=0em{}
\pstart
           \noindent{}{\pb}\label{T_L03985-1v}\edtext{Grüße Sie bitte meinen lieben \textcolor{blue}{Geraldy}\pwindex{Géraldy, Paul 6.\,3.\,1885 Paris – 9.\,3.\,1983 Neuilly-sur-Seine@\textsc{Géraldy, Paul} (6.\,3.\,1885 Paris – 9.\,3.\,1983 Neuilly-sur-Seine), \emph{Schriftsteller}|pw}{}\ledrightnote{\textcolor{blue}{Paul Géraldy}}.}{\lemma{\textnormal{\emph{Grüße … Geraldy.}}}\Cendnote{\textnormal{am rechten oberen Rand der ersten Seite um 180° gedreht}}}\label{T_L03985-1}\pend
           \selectlanguage{ngerman}\endnumbering\briefempfaengerindex{Zuckerkandl, Berta@\textsc{Zuckerkandl, Berta}!zzzSchnitzler, Arthur@\emph{von Arthur Schnitzler}!1929-08-271@{{[}zwischen 27. und 29. 8. 1929?{]}}|)be}\mylabel{L03985h}
\begin{anhang}
\end{anhang}\normalsize

\doendnotes{C}
\bigskip
\vfill

\clearpage

\footnotesize

\lohead{\textsc{register}}

% Definiere theindex-Environment komplett neu ohne reledmac
\makeatletter
\renewenvironment{theindex}{%
  \section*{\indexname}%
  \setlength{\parindent}{0pt}%
  \setlength{\parskip}{0pt plus 0.3pt}%
  \let\item\@idxitem
}{%
  \clearpage
}
\makeatother

\IfFileExists{\jobname-pw.ind}{\input{\jobname-pw.ind}}{}

\end{document}

      