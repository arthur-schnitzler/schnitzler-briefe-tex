%% latex-korrekturansicht-vorspann.tex
%% Vorspann für die Korrekturansicht.
%% Lädt die gemeinsame Datei latex-vorspann.tex mit gesetztem Schalter.

\newif\ifkorrekturansicht
\korrekturansichttrue

\input{../tex-inputs/latex-vorspann}


               \section[Richard Beer-Hofmann an Arthur Schnitzler, 18. 7. 1893]{ Richard Beer-Hofmann an Arthur Schnitzler, 18. 7. 1893}\nopagebreak\mylabel{v}\rehead{ }\normalsize\beginnumbering\briefempfaengerindex{Schnitzler, Arthur@\textsc{Schnitzler, Arthur}!zzzBeer-Hofmann, Richard@\emph{von Richard Beer-Hofmann}!1893-07-181@{18. 7. 1893}|(be} \toendnotes[C]{\smallbreak\pagebreak[2]} \Standort{CUL, Schnitzler, B 8.}
\physDesc{Brief, 1 Blatt, 2 Seiten
\newline{}Handschrift: Bleistift, lateinische Kurrent\newline{}Ordnung: mit Bleistift von unbekannter Hand nummeriert: »20« }\buchAbdrucke{\weitereDrucke{Arthur Schnitzler, Richard Beer-Hofmann: \emph{Briefwechsel 1891–1931}. Hg. Konstanze Fliedl. Wien, Zürich: \emph{Europaverlag} 1992, S. 46.} }\toendnotes[C]{\smallbreak}\pstart
           \noindent{}{\pb}Lieber Arthur! Hier
               die \textcolor{green}{Novelle}{}\ledrightnote{→\textcolor{green}{Das Kind}} – bis auf das letzte
               Capitel das ich noch ändere. Bitte tun Sie was Sie können um die Abschrift zu
               beschleunigen, \uline{und schreiben Sie mir \introOben{}für\introOben{} wann er es verspricht}; geben Sie ihm eventuell eine Prämie für
               Beschleunigung. Vielleicht schicke ich auch das letzte \textcolor{green}{Capitel}{}\ledrightnote{→\textcolor{green}{Das Kind}} ein, aber warten Sie keinesfalls darauf.\pend
           \pstart
           \textcolor{blue}{Devrient}{}\ledrightnote{\textcolor{blue}{Max Devrient}} wollte gestern \textcolor{green}{Gedichte}{}\ledrightnote{→\textcolor{green}{[Gedichte]}} von Ihnen als Zugabe lesen, man
               schickte zu mir, – ich hatte begreiflicherweise keine. Schade! \textcolor{blue}{Bauer}{}\ledrightnote{\textcolor{blue}{Ludwig Bauer}}s \textcolor{green}{Notiz}{}\ledrightnote{→\textcolor{green}{[Abschiedsouper in Ischl]}} – er sagte mir gestern den Wortlaut {[}–{]} ist gut.
               Mit \textcolor{blue}{Paul Horn}{}\ledrightnote{\textcolor{blue}{Paul Horn}} habe ich wegen »\textcolor{green}{\textcolor{green}{Börsencourir}{}\ledrightnote{\textcolor{green}{Berliner Börsen-Courier}}}{}\ledrightnote{→\textcolor{green}{[Man schreibt uns aus Ischl]}}« gesprochen. \textcolor{blue}{Lautenburg}{}\ledrightnote{\textcolor{blue}{Sigmund Lautenburg}} ist {\pb}\strikeout{heut} gestern geko{\geminationm}en.\pend
           \pstart
           Bitte also nochmals tun Sie was Sie können.\pend
           \pstart
           Herzlichst{\\[\baselineskip]}\spacefill\mbox{Richard}\pend
           \leftskip=0em{}\pstart
           \textcolor{blue}{Schwarzkopf}{}\ledrightnote{\textcolor{blue}{Gustav Schwarzkopf}}, \textcolor{blue}{Salten}{}\ledrightnote{\textcolor{blue}{Felix Salten}}, herzlichst gegrüßt.\pend
           \pstart
           Dienstag 18 Juli 93.\pend
           \endnumbering\briefempfaengerindex{Schnitzler, Arthur@\textsc{Schnitzler, Arthur}!zzzBeer-Hofmann, Richard@\emph{von Richard Beer-Hofmann}!1893-07-181@{18. 7. 1893}|)be}\mylabel{h}  \normalsize

\doendnotes{C}
\bigskip
\vfill

\clearpage

\footnotesize

\lohead{\textsc{register}}

% Definiere theindex-Environment komplett neu ohne reledmac
\makeatletter
\renewenvironment{theindex}{%
  \section*{\indexname}%
  \setlength{\parindent}{0pt}%
  \setlength{\parskip}{0pt plus 0.3pt}%
  \let\item\@idxitem
}{%
  \clearpage
}
\makeatother

\IfFileExists{\jobname-pw.ind}{\input{\jobname-pw.ind}}{}

\end{document}

      