%% latex-korrekturansicht-vorspann.tex
%% Vorspann für die Korrekturansicht.
%% Lädt die gemeinsame Datei latex-vorspann.tex mit gesetztem Schalter.

\newif\ifkorrekturansicht
\korrekturansichttrue

\input{../tex-inputs/latex-vorspann}


\section[Sigmund Freud an Arthur Schnitzler, 8. 3. 1926]{L03818 Sigmund Freud an Arthur Schnitzler, 8. 3. 1926}
\nopagebreak\mylabel{L03818v}
\rehead{ }\normalsize\beginnumbering\briefempfaengerindex{, @\textsc{, }!zzz, @\emph{von  }!1926-03-081@{8. 3. 1926}|(be}
\toendnotes[C]{\smallbreak\pagebreak[2]}\Standort{CUL, Schnitzler, B 31.}
\physDesc{Brief, 1 Blatt, 2 Seiten, 568 Zeichen
\newline{}Handschrift: blaue Tinte, deutsche Kurrent}
\buchAbdrucke{\weitereDrucke{1) Sigmund Freud: \emph{Briefe an Arthur Schnitzler.} Herausgegeben von Henry Schnitzler. In: \emph{Neue deutsche Rundschau}, Jg. 66 (Januar 1955) Nr. 1, S. 99.} \weitereDrucke{2) Sigmund Freud: \emph{Sigmund Freud Edition. Digitale historisch-kritische Gesamtausgabe}. Herausgegeben von Christine Diercks,  Arkadi Blatow und  Elisabeth Skale. (2014–2025) \url{https://www.freudedition.net/briefe/freud-sigmund/schnitzler-arthur/1926/03/08}.} }\toendnotes[C]{\smallbreak}
\pstart
           \raggedleft{}{\pb}8.\,3.\,26\pend
           
\pstart
           \textcolor{gray}{\textbf{PROF. D\textsuperscript{R.} FREUD }}\hfill \textcolor{gray}{\textbf{\textcolor{pink}{WIEN IX., BERGGASSE 19}\oindex{Wien@\textbf{Wien}!IX., Alsergrund@\textbf{IX., Alsergrund}!Berggasse 19@\textbf{Berggasse 19}, \emph{Wohngebäude}|pw}{}\ledrightnote{\textcolor{pink}{Berggasse 19}}}}\pend
           
\pstart{}Verehrteſter!\pend\vspace{0.5em}
\pstart
           Ich war Ihnen noch nie ſo nah. Ich hauſe \label{K_L03818-1v}\edtext{im \textcolor{pink}{Sanatorium}\oindex{Wien@\textbf{Wien}!XVIII., Währing@\textbf{XVIII., Währing}!Cottage-Sanatorium für Nerven- und Stoffwechselkranke@\textbf{Cottage-Sanatorium für Nerven- und Stoffwechselkranke}, \emph{Sanatorium}|pwv}{}\ledrightnote{{$\rightarrow$}\emph{\textcolor{pink}{Cottage-Sanatorium für Nerven- und Stoffwechselkranke}}}}{\lemma{\textnormal{\emph{im Sanatorium}}}\Cendnote{\textnormal{Vom 5. 3. bis zum
                     2. 4. 1926 hielt sich \textcolor{blue}{Sigmund
                     Freud}\pwindex{Freud, Sigmund 6.\,5.\,1856 Pribor – 23.\,9.\,1939 London@\textsc{Freud, Sigmund} (6.\,5.\,1856 Pribor – 23.\,9.\,1939 London), \emph{Psychoanalytiker}|pwk} im \textcolor{pink}{Cottage-Sanatorium}\oindex{Wien@\textbf{Wien}!XVIII., Währing@\textbf{XVIII., Währing}!Cottage-Sanatorium für Nerven- und Stoffwechselkranke@\textbf{Cottage-Sanatorium für Nerven- und Stoffwechselkranke}, \emph{Sanatorium}|pwk} in der
                     \textcolor{pink}{Sternwartestraße 74}\oindex{Wien@\textbf{Wien}!XVIII., Währing@\textbf{XVIII., Währing}!Sternwartestraße 74@\textbf{Sternwartestraße 74}, \emph{Gebäude}|pwk} auf. \textcolor{blue}{Schnitzler} besuchte ihn dort zwei Mal, vgl. A. S.: \emph{Tagebuch}, 12. 3. 1926, und A. S.: \emph{Tagebuch}, 26. 3. 1926.}}}\label{K_L03818-1} in Ihrer \textcolor{pink}{Straße}\oindex{Wien@\textbf{Wien}!XVIII., Währing@\textbf{XVIII., Währing}!Sternwartestraße@\textbf{Sternwartestraße}, \emph{Straße}|pw}{}\ledrightnote{\textcolor{pink}{Sternwartestraße}} u mache auf Wunſch der Interniſten
               Herztherapie, befinde mich aber ſubjektiv recht wol. \pend
           
\pstart
           Infolge eines früheren Verſäumnißes kann ich mich heute in Einem für
               zwei Ihrer \label{K_L03818-2v}\edtext{\textcolor{green}{Geſchenke}\pwindex{Schnitzler, Arthur 15. 5. 1862 Wien – 21. 10. 1931 ebd.@\textsc{Schnitzler, Arthur} (15. 5. 1862 Wien – 21. 10. 1931 ebd.), \emph{Schriftsteller, Mediziner}!Gang zum Weiher. Dramatische Dichtung@\strich\emph{Der Gang zum Weiher. Dramatische Dichtung}|pwv}\pwindex{Schnitzler, Arthur 15. 5. 1862 Wien – 21. 10. 1931 ebd.@\textsc{Schnitzler, Arthur} (15. 5. 1862 Wien – 21. 10. 1931 ebd.), \emph{Schriftsteller, Mediziner}!Frau des Richters. Novelle@\strich\emph{Die Frau des Richters. Novelle}|pwv}{}\ledrightnote{{$\rightarrow$}\emph{\textcolor{green}{Der Gang zum Weiher. Dramatische Dichtung}}{\newline}{$\rightarrow$}\emph{\textcolor{green}{Die Frau des Richters. Novelle}}}}{\lemma{\textnormal{\emph{Geſchenke}}}\Cendnote{\textnormal{Anfang März erschien
                  \emph{\textcolor{green}{Der Gang zum Weiher}\pwindex{Schnitzler, Arthur 15. 5. 1862 Wien – 21. 10. 1931 ebd.@\textsc{Schnitzler, Arthur} (15. 5. 1862 Wien – 21. 10. 1931 ebd.), \emph{Schriftsteller, Mediziner}!Gang zum Weiher. Dramatische Dichtung@\strich\emph{Der Gang zum Weiher. Dramatische Dichtung}|pwk}}. Beim zweiten Werk dürfte es sich um \emph{\textcolor{green}{Die Frau des Richters}\pwindex{Schnitzler, Arthur 15. 5. 1862 Wien – 21. 10. 1931 ebd.@\textsc{Schnitzler, Arthur} (15. 5. 1862 Wien – 21. 10. 1931 ebd.), \emph{Schriftsteller, Mediziner}!Frau des Richters. Novelle@\strich\emph{Die Frau des Richters. Novelle}|pwk}} handeln.}}}\label{K_L03818-2} bedanken. Die \label{K_L03818-3v}\edtext{begleitende \textcolor{green}{Brochüre}\pwindex{Freud, Sigmund 6.\,5.\,1856 Pribor – 23.\,9.\,1939 London@\textsc{Freud, Sigmund} (6.\,5.\,1856 Pribor – 23.\,9.\,1939 London), \emph{Psychoanalytiker}!Hemmung, Symptom und Angst@\strich\emph{Hemmung, Symptom und Angst}|pwv}{}\ledrightnote{{$\rightarrow$}\emph{\textcolor{green}{Hemmung, Symptom und Angst}}}}{\lemma{\textnormal{\emph{begleitende Brochüre}}}\Cendnote{\textnormal{\textcolor{blue}{Schnitzlers}{ }\emph{\textcolor{green}{Tagebuch}\pwindex{Schnitzler, Arthur 15. 5. 1862 Wien – 21. 10. 1931 ebd.@\textsc{Schnitzler, Arthur} (15. 5. 1862 Wien – 21. 10. 1931 ebd.), \emph{Schriftsteller, Mediziner}!Tagebuch@\strich\emph{Tagebuch}|pwk}}eintrag bestätigt den Erhalt von und
                  die Beschäftigung mit \textcolor{blue}{Freuds}\pwindex{Freud, Sigmund 6.\,5.\,1856 Pribor – 23.\,9.\,1939 London@\textsc{Freud, Sigmund} (6.\,5.\,1856 Pribor – 23.\,9.\,1939 London), \emph{Psychoanalytiker}|pwk}{ }\textcolor{green}{Text}\pwindex{Freud, Sigmund 6.\,5.\,1856 Pribor – 23.\,9.\,1939 London@\textsc{Freud, Sigmund} (6.\,5.\,1856 Pribor – 23.\,9.\,1939 London), \emph{Psychoanalytiker}!Hemmung, Symptom und Angst@\strich\emph{Hemmung, Symptom und Angst}|pwkv} (\emph{\textcolor{green}{Hemmung, Symptom und Angst}\pwindex{Freud, Sigmund 6.\,5.\,1856 Pribor – 23.\,9.\,1939 London@\textsc{Freud, Sigmund} (6.\,5.\,1856 Pribor – 23.\,9.\,1939 London), \emph{Psychoanalytiker}!Hemmung, Symptom und Angst@\strich\emph{Hemmung, Symptom und Angst}|pwk}}. \textcolor{pink}{Leipzig}\oindex{Leipzig@\textbf{Leipzig}, \emph{Hauptstadt}|pwk}, \textcolor{pink}{Wien}\oindex{Wien@\textbf{Wien}, \emph{Verwaltungsgebiet}|pwk}, \textcolor{pink}{Zürich}\oindex{Zürich@\textbf{Zürich}|pwk}: \emph{\textcolor{brown}{Internationaler Psychoanalytischer Verlag}\orgindex{Internationaler Psychoanalytischer Verlag@Internationaler Psychoanalytischer Verlag|pwk}}{ }1926), vgl. A. S.: \emph{Tagebuch}, 9. 3. 1926.}}}\label{K_L03818-3} ſoll in keiner Weiſe eine Revanche ſein, ſie iſt eben nur meine letzte
                  \textcolor{green}{Publikation}\pwindex{Freud, Sigmund 6.\,5.\,1856 Pribor – 23.\,9.\,1939 London@\textsc{Freud, Sigmund} (6.\,5.\,1856 Pribor – 23.\,9.\,1939 London), \emph{Psychoanalytiker}!Hemmung, Symptom und Angst@\strich\emph{Hemmung, Symptom und Angst}|pwv}{}\ledrightnote{{$\rightarrow$}\emph{\textcolor{green}{Hemmung, Symptom und Angst}}} – vielleicht in
               jedem Sinne – ſonst aber recht \strikeout{untereſ} unintereſſant
               und beſonders für Sie unwichtig. \pend
           
\pstart
           Troſt, daß Sie sie ja weder zu leſen noch ſich darüber {\pb}zu äußern brauchen.\pend
           
\pstart
           Mit herzl Gruß{\\[\baselineskip]}\spacefill\mbox{Ihr Freud}\pend
           \leftskip=0em{}\selectlanguage{ngerman}\endnumbering\briefempfaengerindex{, @\textsc{, }!zzz, @\emph{von  }!1926-03-081@{8. 3. 1926}|)be}\mylabel{L03818h}
\begin{anhang}
\end{anhang}\normalsize

\doendnotes{C}
\bigskip
\vfill

\clearpage

\footnotesize

\lohead{\textsc{register}}

% Definiere theindex-Environment komplett neu ohne reledmac
\makeatletter
\renewenvironment{theindex}{%
  \section*{\indexname}%
  \setlength{\parindent}{0pt}%
  \setlength{\parskip}{0pt plus 0.3pt}%
  \let\item\@idxitem
}{%
  \clearpage
}
\makeatother

\IfFileExists{\jobname-pw.ind}{\input{\jobname-pw.ind}}{}

\end{document}

      