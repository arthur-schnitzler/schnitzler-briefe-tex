%% latex-korrekturansicht-vorspann.tex
%% Vorspann für die Korrekturansicht.
%% Lädt die gemeinsame Datei latex-vorspann.tex mit gesetztem Schalter.

\newif\ifkorrekturansicht
\korrekturansichttrue

\input{../tex-inputs/latex-vorspann}


               \section[Hugo von Hofmannsthal an Arthur Schnitzler, 19. 3. {[}1909{]}]{ Hugo von Hofmannsthal an Arthur Schnitzler, 19. 3. {[}1909{]}}\nopagebreak\mylabel{v}\rehead{ }\normalsize\beginnumbering\briefempfaengerindex{Schnitzler, Arthur@\textsc{Schnitzler, Arthur}!zzzHofmannsthal, Hugo von@\emph{von Hugo von Hofmannsthal}!1909-03-191@{19. 3. {[}1909{]}}|(be} \toendnotes[C]{\smallbreak\pagebreak[2]} \Standort{CUL, Schnitzler, B 43.}
\physDesc{Brief, 1 Blatt, 3 Seiten
\newline{}Handschrift: schwarze Tinte, deutsche Kurrent
\newline{}Schnitzler: mit Bleistift die Jahreszahl ergänzt: »909« und beschriftet: »Hugo Hofmannsthal« \newline{}Ordnung: 1) mit Bleistift von unbekannter Hand nummeriert: »\strikeout{299}« 2) mit Bleistift von unbekannter Hand nummeriert: »295«}\buchAbdrucke{\weitereDrucke{Hugo von Hofmannsthal, Arthur Schnitzler: \emph{Briefwechsel}. Hg. Therese Nickl und Heinrich Schnitzler. Frankfurt am Main: \emph{S. Fischer} 1964, S. 243.} }\toendnotes[C]{\smallbreak}\pstart
           \raggedleft{}{\pb}\textcolor{pink}{R.}{}\ledrightnote{\textcolor{pink}{Rodaun}}{ }19 III.\pend
           \pstart
           lieber, bitte erwähnen Sie das Folgende gegen niemanden, am
               wenigſten gegen \textcolor{blue}{Waſſermanns}{}\ledrightnote{\textcolor{blue}{Jakob Wassermann}{\newline}\textcolor{blue}{Julie Wassermann}}, am wenigſten
               gegen \textcolor{blue}{\textsc{Trebitsch}}{}\ledrightnote{\textcolor{blue}{Siegfried Trebitsch}}, am wenigſten gegen \textcolor{blue}{\textsc{Sil Vara}}{}\ledrightnote{\textcolor{blue}{Geza Silberer}}, alſo gut. Nämlich: bitte ko{\geminationm}en Sie zur
               Generalprobe von unſerer wohltönenden herzigen \textcolor{green}{Elektra}{}\ledrightnote{\textcolor{green}{Elektra (op. 58)}} d. h. am Montag um ¾ 11{ }{\pb}pünktlich gehen
               Sie beim \textcolor{pink}{Directionseingang}{}\ledrightnote{→\textcolor{pink}{Oper}} hinein
                  (\textcolor{pink}{Kärtnerſtraße}{}\ledrightnote{\textcolor{pink}{Kärntner Straße}}) in den erſten Stock hinauf
               dort im Bureau des Oberrates \textcolor{blue}{Ribitſch}{}\ledrightnote{\textcolor{blue}{Gabriel Ribitsch}}{ }ſteht Ihr werter und angeſehener Name auf einer
               Liste, worauf man Sie in eine Loge führt. Parkett ist nicht.\pend
           \pstart
           Ihr lieber{\\[\baselineskip]}\spacefill\mbox{Hugo.}\pend
           \leftskip=0em{}\endnumbering\briefempfaengerindex{Schnitzler, Arthur@\textsc{Schnitzler, Arthur}!zzzHofmannsthal, Hugo von@\emph{von Hugo von Hofmannsthal}!1909-03-191@{19. 3. {[}1909{]}}|)be}\mylabel{h}  \normalsize

\doendnotes{C}
\bigskip
\vfill

\clearpage

\footnotesize

\lohead{\textsc{register}}

% Definiere theindex-Environment komplett neu ohne reledmac
\makeatletter
\renewenvironment{theindex}{%
  \section*{\indexname}%
  \setlength{\parindent}{0pt}%
  \setlength{\parskip}{0pt plus 0.3pt}%
  \let\item\@idxitem
}{%
  \clearpage
}
\makeatother

\IfFileExists{\jobname-pw.ind}{\input{\jobname-pw.ind}}{}

\end{document}

      