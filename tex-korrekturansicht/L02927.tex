%% latex-korrekturansicht-vorspann.tex
%% Vorspann für die Korrekturansicht.
%% Lädt die gemeinsame Datei latex-vorspann.tex mit gesetztem Schalter.

\newif\ifkorrekturansicht
\korrekturansichttrue

\input{../tex-inputs/latex-vorspann}


         
         \renewcommand{\erwaehntePersonen}{Personen:  ?? [Urlaubsvertretung von Paul Goldmann, 2. Augusthälfte 1900], Richard Beer-Hofmann, Georg Brandes, Alfred Kerr}
         \renewcommand{\erwaehnteInstitutionen}{Institutionen: Neue Freie Presse}
         \renewcommand{\erwaehnteOrte}{Orte: Bad Ischl, Berlin, Dessauer Straße, Schwarzer Adler, Toblach}
         \renewcommand{\erwaehnteWerke}{}
               \section[ Paul Goldmann an Arthur Schnitzler, 7. 8. {[}1900{]}]{Paul Goldmann an Arthur Schnitzler, 7. 8. {[}1900{]}}\nopagebreak\mylabel{v}\rehead{ }\normalsize\beginnumbering\briefempfaengerindex{Schnitzler, Arthur@\textsc{Schnitzler, Arthur}!zzzGoldmann, Paul@\emph{von Paul Goldmann}!1900-08-071@{7. 8. {[}1900{]}}|(be} \toendnotes[C]{\smallbreak\pagebreak[2]} \Standort{DLA, A:Schnitzler, HS.NZ85.1.3170.}
\physDesc{Brief, 1 Blatt, 2 Seiten
\newline{}Handschrift: blaue Tinte, deutsche Kurrent
\newline{}Schnitzler: 1) mit schwarzer Tinte das Jahr »{[}1{]}90\textcolor{gray}{0}.« vermerkt  2) mit rotem Buntstift eine Unterstreichung}\toendnotes[C]{\smallbreak}\pstart
           \noindent{}{\pb}\textcolor{pink}{Berlin}{}\ledrightnote{\textcolor{pink}{Berlin}}, 7. Auguſt.\hfill \textcolor{pink}{\textcolor{gray}{\textbf{DESSAUERSTRASSE 19}}}{}\ledrightnote{\textcolor{pink}{Dessauer Straße}}\pend
           \pstart
           \centering{}Mein lieber Freund,\pend
           \pstart
           \noindent{}Ich muß meine Abreiſe wieder verſchieben. Die »\textcolor{brown}{Neue
               Freie Preſſe}{}\ledrightnote{\textcolor{brown}{Neue Freie Presse}}« will einen \label{K_L02927-1v}\edtext{\textcolor{blue}{Vertreter}{}\ledrightnote{{$\rightarrow$}\textcolor{blue}{?? [Urlaubsvertretung von Paul Goldmann, 2. Augusthälfte 1900]}}}{\lemma{\textnormal{\emph{Vertreter}}}\Cendnote{\textnormal{nicht ermittelt}}}\label{K_L02927-1h}
               hierher ſenden, und dieſer ſchreibt mir eben, er könne am 10. Auguſt nicht kommen und werde erſt »einige Tage ſpäter« eintreffen.
                  \strikeout{Ich} Es iſt die \strikeout{ge\textcolor{gray}{w}} übliche Rückſichtsloſigkeit und Schweinewirthſchaft. Aber da iſt nichts zu
               machen. {\pb}Bitte \textsc{\textcolor{blue}{Richard}{}\ledrightnote{\textcolor{blue}{Richard Beer-Hofmann}}} und \textsc{\textcolor{blue}{Kerr}{}\ledrightnote{\textcolor{blue}{Alfred Kerr}}} (\textsc{\textcolor{pink}{Toblach}{}\ledrightnote{\textcolor{pink}{Toblach}}}, \textsc{\textcolor{pink}{Schwarzer Adler}{}\ledrightnote{\textcolor{pink}{Schwarzer Adler}}}) zu benachrichtigen. Ich habe in dieſen Tagen keine Zeit.\pend
           \pstart
           Viele treue Grüße! {\\[\baselineskip]}Dein {\\[\baselineskip]}\spacefill\mbox{Paul Goldmnn}\pend
           \leftskip=0em{}\pstart
           \noindent{}\textsc{\textcolor{blue}{Brandes}{}\ledrightnote{\textcolor{blue}{Georg Brandes}}} iſt \textcolor{pink}{hier}{}\ledrightnote{{$\rightarrow$}\textcolor{pink}{Berlin}}. Wir waren
                     geſtern{ }Abend zuſammen und haben viel von Dir geſprochen.\pend
           \endnumbering\briefempfaengerindex{Schnitzler, Arthur@\textsc{Schnitzler, Arthur}!zzzGoldmann, Paul@\emph{von Paul Goldmann}!1900-08-071@{7. 8. {[}1900{]}}|)be}\mylabel{h}\begin{anhang}\end{anhang}\normalsize

\doendnotes{C}
\bigskip
\vfill

\clearpage

\footnotesize

\lohead{\textsc{register}}

% Definiere theindex-Environment komplett neu ohne reledmac
\makeatletter
\renewenvironment{theindex}{%
  \section*{\indexname}%
  \setlength{\parindent}{0pt}%
  \setlength{\parskip}{0pt plus 0.3pt}%
  \let\item\@idxitem
}{%
  \clearpage
}
\makeatother

\IfFileExists{\jobname-pw.ind}{\input{\jobname-pw.ind}}{}

\end{document}

      