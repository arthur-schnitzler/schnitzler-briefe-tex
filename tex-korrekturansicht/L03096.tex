%% latex-korrekturansicht-vorspann.tex
%% Vorspann für die Korrekturansicht.
%% Lädt die gemeinsame Datei latex-vorspann.tex mit gesetztem Schalter.

\newif\ifkorrekturansicht
\korrekturansichttrue

\input{../tex-inputs/latex-vorspann}


\renewcommand{\erwaehntePersonen}{Personen: Richard Beer-Hofmann, Otto Brahm, Hugo von Hofmannsthal, Felix Salten, Gustav Schwarzkopf}
\renewcommand{\erwaehnteOrte}{Orte: Berlin, Dessauer Straße, Wien}
\renewcommand{\erwaehnteWerke}{Werke: Die Frau mit dem Dolche, Lebendige Stunden. Vier Einakter}
\section[ Paul Goldmann an Arthur Schnitzler, 19. 12. {[}1901{]}]{Paul Goldmann an Arthur Schnitzler, 19. 12. {[}1901{]}}
\nopagebreak\mylabel{v}
\rehead{ }\normalsize\beginnumbering\briefempfaengerindex{Schnitzler, Arthur@\textsc{Schnitzler, Arthur}!zzzGoldmann, Paul@\emph{von Paul Goldmann}!1901-12-191@{19. 12. {[}1901{]}}|(be}
\toendnotes[C]{\smallbreak\pagebreak[2]}\Standort{DLA, A:Schnitzler, HS.NZ85.1.3171.}
\physDesc{Brief, 1 Blatt, 1 Seite
\newline{}Handschrift: blaue Tinte, deutsche Kurrent
\newline{}Schnitzler: 1) mit Bleistift das Jahr »{[}1{]}901.« vermerkt  2) mit rotem Buntstift eine Unterstreichung}\toendnotes[C]{\smallbreak}
\pstart
           \noindent{}\raggedleft{}{\pb}\textcolor{pink}{\textcolor{gray}{\textbf{DESSAUERSTRASSE 19}}}{}\ledrightnote{\textcolor{pink}{Dessauer Straße}}\pend
           
\pstart
           \textcolor{pink}{Berlin}{}\ledrightnote{\textcolor{pink}{Berlin}}, 19. Dezember.\pend
           
\pstart\center{}Mein lieber Freund,\pend
\pstart
           Ich werde meine Reiſe verſchieben und Dich \label{K_L03096-1v}\edtext{Montag}{\lemma{\textnormal{\emph{Montag}}}\Cendnote{\textnormal{\textcolor{blue}{Schnitzler} kam erst am Samstag, dem 28. 12. 1901, in \textcolor{pink}{Berlin} an.}}}\label{K_L03096-1h} erwarten. \label{K_L03096-2v}\edtext{\textsc{\textcolor{blue}{Brahm}{}\ledrightnote{\textcolor{blue}{Otto Brahm}}}}{\lemma{\textnormal{\emph{Brahm}}}\Cendnote{\textnormal{\textcolor{blue}{Otto Brahm} wollte, dass \textcolor{blue}{Schnitzler}{ }\emph{\textcolor{green}{Die Frau mit dem Dolche}} zurückzieht.
                     Vgl. \emph{Der Briefwechsel Arthur Schnitzler — Otto
                        Brahm}. Vollständige Ausgabe. Herausgegeben, eingeleitet und
                     erläutert von Oskar Seidlin. Tübingen:
                        \emph{Niemeyer}{ }1975, S. 103 ff. und A. S.: \emph{Tagebuch}, 18. 2. 1901.}}}\label{K_L03096-2h} iſt blödſinnig. Du
               darfſt die »\textcolor{green}{Frau mit dem Dolch}{}\ledrightnote{\textcolor{green}{Die Frau mit dem Dolche}}« unter keinen
               Umſtänden zurückziehen. Ich war bereits über die \textcolor{pink}{Wien}{}\ledrightnote{\textcolor{pink}{Wien}}er \label{K_L03096-3v}\edtext{\textcolor{blue}{Freunde}{}\ledrightnote{{$\rightarrow$}\textcolor{blue}{Felix Salten}{\newline}{$\rightarrow$}\textcolor{blue}{Gustav Schwarzkopf}}}{\lemma{\textnormal{\emph{Freunde}}}\Cendnote{\textnormal{»bereits« impliziert
                  einen zeitlichen Abstand, daher höchstwahrscheinlich Bezug auf die private
                  Vorlesung der \emph{\textcolor{green}{Lebendigen Stunden}} vor \textcolor{blue}{Felix Salten} und \textcolor{blue}{Gustav Schwarzkopf} am 4. 9. 1901 und nicht auf jene am 14. 12. 1901 vor \textcolor{blue}{Hugo von Hofmannsthal} und \textcolor{blue}{Richard Beer-Hofmann}}}}\label{K_L03096-3h} erbittert, die mit kaum glaublicher Urtheilsloſigkeit Bedenken gegen dieſen
               beſten unter den vier \textcolor{green}{Einaktern}{}\ledrightnote{{$\rightarrow$}\textcolor{green}{Lebendige Stunden. Vier Einakter}} geäußert haben.\pend
           
\pstart
           Viele treue Grüße! {\\[\baselineskip]}Dein {\\[\baselineskip]}\spacefill\mbox{P. G.}\pend
           \leftskip=0em{}\endnumbering\briefempfaengerindex{Schnitzler, Arthur@\textsc{Schnitzler, Arthur}!zzzGoldmann, Paul@\emph{von Paul Goldmann}!1901-12-191@{19. 12. {[}1901{]}}|)be}\mylabel{h}
\begin{anhang}
\end{anhang}\normalsize

\doendnotes{C}
\bigskip
\vfill

\clearpage

\footnotesize

\lohead{\textsc{register}}

% Definiere theindex-Environment komplett neu ohne reledmac
\makeatletter
\renewenvironment{theindex}{%
  \section*{\indexname}%
  \setlength{\parindent}{0pt}%
  \setlength{\parskip}{0pt plus 0.3pt}%
  \let\item\@idxitem
}{%
  \clearpage
}
\makeatother

\IfFileExists{\jobname-pw.ind}{\input{\jobname-pw.ind}}{}

\end{document}

      