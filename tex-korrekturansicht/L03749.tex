%% latex-korrekturansicht-vorspann.tex
%% Vorspann für die Korrekturansicht.
%% Lädt die gemeinsame Datei latex-vorspann.tex mit gesetztem Schalter.

\newif\ifkorrekturansicht
\korrekturansichttrue

\input{../tex-inputs/latex-vorspann}


\section[Arthur Schnitzler an Stefan Zweig, 25. 6. 1923]{L03749 Arthur Schnitzler an Stefan Zweig, 25. 6. 1923}
\nopagebreak\mylabel{L03749v}
\rehead{ }\normalsize\beginnumbering\briefempfaengerindex{, @\textsc{, }!zzz, @\emph{von  }!1923-06-251@{25. 6. 1923}|(be}
\toendnotes[C]{\smallbreak\pagebreak[2]}\Standort{Jerusalem, National Library of Israel, ARC. Ms. Var. 305 1 58 Stefan Zweig Collection.}
\physDesc{Postkarte, 721 Zeichen
\newline{}Handschrift: Bleistift, lateinische Kurrent
\newline{}Versand: Stempel: »\nobreak{}\oindex{IX., Alsergrund@\textbf{IX., Alsergrund}, \emph{Verwaltungsgebiet}|pwk}9 Wien 72, 26. VI. 23, 16\nobreak{}«.  
\newline{}Zweig: mit schwarzer Tinte Vermerk: »\textsc{beantw.}« }\toendnotes[C]{\smallbreak}\pstart{}{\pb}\label{T_L03749-1v}\edtext{\textcolor{gray}{\textbf{A. S.}}}{\lemma{\textnormal{\emph{A. S.}}}\Cendnote{\textnormal{ovaler Absenderkleber}}}\label{T_L03749-1}\pend{}\pstart{}\textcolor{pink}{\textcolor{gray}{\textbf{WIEN, XVIII.}}}\oindex{XVIII., Währing@\textbf{XVIII., Währing}, \emph{Verwaltungsgebiet}|pw}{}\ledrightnote{\textcolor{pink}{XVIII., Währing}}\pend{}\pstart{}\textcolor{pink}{\textcolor{gray}{\textbf{STERNWARTESTR. 71}}}\oindex{Wien@\textbf{Wien}!XVIII., Währing@\textbf{XVIII., Währing}!Sternwartestraße 71@\textbf{Sternwartestraße 71}, \emph{Wohngebäude}|pw}{}\ledrightnote{\textcolor{pink}{Sternwartestraße 71}}\pend{}{\bigskip}\pstart{}Hrn\pend{}\pstart{}Dr Stefan Zweig\pend{}\pstart{}\textcolor{pink}{Salzburg}\oindex{Salzburg@\textbf{Salzburg}, \emph{Verwaltungsgebiet}|pw}{}\ledrightnote{\textcolor{pink}{Salzburg}}\pend{}\pstart{}\textcolor{pink}{Kapuzinerberg 5}\oindex{Paschinger Schlössl@\textbf{Paschinger Schlössl}, \emph{Wohngebäude}|pw}{}\ledrightnote{\textcolor{pink}{Paschinger Schlössl}}\pend{}{\bigskip}\vspace{1em}
\pstart
           \raggedleft{}{\pb}\textcolor{pink}{Wien}\oindex{Wien@\textbf{Wien}, \emph{Verwaltungsgebiet}|pw}{}\ledrightnote{\textcolor{pink}{Wien}}, 25. 6. 23\pend
           \vspace{0.5em}
\pstart
           lieber Herr Doctor Zweig, das »\textcolor{green}{Gänsemännchen}\pwindex{Wassermann, Jakob 10.\,3.\,1873 Fürth – 1.\,1.\,1934 Altaussee@\textsc{Wassermann, Jakob} (10.\,3.\,1873 Fürth – 1.\,1.\,1934 Altaussee), \emph{Schriftsteller}!Gänsemännchen. Roman@\strich\emph{Das Gänsemännchen. Roman}|pw}{}\ledrightnote{\textcolor{green}{Das Gänsemännchen. Roman}}« auf dessen Erscheinen im Antiqu. Catalog \textcolor{brown}{Hirsch}\orgindex{Antiquariat Emil Hirsch@Antiquariat Emil Hirsch|pw}{}\ledrightnote{\textcolor{brown}{Antiquariat Emil Hirsch}} Sie mich liebenswürdiger Weise aufmerksam gemacht
               haben, beko{\geminationm} ich zurück. Ein \textcolor{blue}{Bekannter}\pwindex{Krell, Max 24.\,9.\,1887 Hubertusburg – 11.\,6.\,1962 Florenz@\textsc{Krell, Max} (24.\,9.\,1887 Hubertusburg – 11.\,6.\,1962 Florenz), \emph{Schriftsteller, Verlagslektor}|pwv}{}\ledrightnote{{$\rightarrow$}\emph{\textcolor{blue}{Max Krell}}} meiner Schwägerin \textcolor{blue}{Steinrück}\pwindex{Steinrück, Elisabeth 19.\,11.\,1885 – 7.\,4.\,1920 Partenkirchen@\textsc{Steinrück, Elisabeth} (19.\,11.\,1885 – 7.\,4.\,1920 Partenkirchen)|pw}{}\ledrightnote{\textcolor{blue}{Elisabeth Steinrück}},
               dem sie das Exempl. angeblich vermacht hatte, hat es zur Versteigerung dem \textcolor{blue}{Hirsch}\pwindex{Hirsch, Emil 14.\,3.\,1866 Bad Mergentheim – 27.\,7.\,1954 New York City@\textsc{Hirsch, Emil} (14.\,3.\,1866 Bad Mergentheim – 27.\,7.\,1954 New York City), \emph{Verleger, Antiquar}|pw}{}\ledrightnote{\textcolor{blue}{Emil Hirsch}} überlassen.\pend
           
\pstart
           – Eigentlich aber
               schreib ich Ihnen um Ihnen zu sagen, wie sehr mich Ihr wunderschöner \label{K_L03749-1v}\edtext{\textcolor{green}{Artikel}\pwindex{Zweig, Stefan 28.\,11.\,1881 Wien – 23.\,2.\,1942 Petrópolis@\textsc{Zweig, Stefan} (28.\,11.\,1881 Wien – 23.\,2.\,1942 Petrópolis), \emph{Schriftsteller}!Zum Andenken Walter Rathenaus. Am Jahrestage seiner Ermordung, 24. Juni 1922@\strich\emph{Zum Andenken Walter Rathenaus. Am Jahrestage seiner Ermordung, 24. Juni 1922}|pwv}{}\ledrightnote{{$\rightarrow$}\emph{\textcolor{green}{Zum Andenken Walter Rathenaus. Am Jahrestage seiner Ermordung, 24. Juni 1922}}} über \textcolor{blue}{Rathenau}\pwindex{Rathenau, Walther 29.\,9.\,1867 Berlin – 24.\,6.\,1922 ebd.@\textsc{Rathenau, Walther} (29.\,9.\,1867 Berlin – 24.\,6.\,1922 ebd.), \emph{Politiker, Industrieller}|pw}{}\ledrightnote{\textcolor{blue}{Walther Rathenau}}}{\lemma{\textnormal{\emph{Artikel über Rathenau}}}\Cendnote{\textnormal{\textcolor{blue}{Stefan Zweig}\pwindex{Zweig, Stefan 28.\,11.\,1881 Wien – 23.\,2.\,1942 Petrópolis@\textsc{Zweig, Stefan} (28.\,11.\,1881 Wien – 23.\,2.\,1942 Petrópolis), \emph{Schriftsteller}|pwk}: \emph{\textcolor{green}{Zum
                        Andenken Walter Rathenaus. Am Jahrestage seiner Ermordung, 24. Juni
                        1922}\pwindex{Zweig, Stefan 28.\,11.\,1881 Wien – 23.\,2.\,1942 Petrópolis@\textsc{Zweig, Stefan} (28.\,11.\,1881 Wien – 23.\,2.\,1942 Petrópolis), \emph{Schriftsteller}!Zum Andenken Walter Rathenaus. Am Jahrestage seiner Ermordung, 24. Juni 1922@\strich\emph{Zum Andenken Walter Rathenaus. Am Jahrestage seiner Ermordung, 24. Juni 1922}|pwk}}. In: \emph{\textcolor{green}{Neue Freie Presse}\pwindex{Neue Freie Presse@\emph{Neue Freie Presse}|pwk}},
                     Nr. 21.116, 24. 6. 1923, Morgenblatt,
                     S. 1–3.}}}\label{K_L03749-1} ergriffen hat; als essayistisches Meisterstück und
               als menschliches Document. Ich habe \textcolor{blue}{R}\pwindex{Rathenau, Walther 29.\,9.\,1867 Berlin – 24.\,6.\,1922 ebd.@\textsc{Rathenau, Walther} (29.\,9.\,1867 Berlin – 24.\,6.\,1922 ebd.), \emph{Politiker, Industrieller}|pw}{}\ledrightnote{\textcolor{blue}{Walther Rathenau}} nicht
               gekannt, aber nie ist mir seine Persönlichkeit so einleuchtend geworden, als aus
               Ihrer  Gestaltung.\pend
           \pstart Seien Sie bedankt und gegrüßt! Herzlichst Ihr sehr ergebner \spacefill\mbox{ArthSchnitzler}\pend{}\selectlanguage{ngerman}\endnumbering\briefempfaengerindex{, @\textsc{, }!zzz, @\emph{von  }!1923-06-251@{25. 6. 1923}|)be}\mylabel{L03749h}  \normalsize

\doendnotes{C}
\bigskip
\vfill

\clearpage

\footnotesize

\lohead{\textsc{register}}

% Definiere theindex-Environment komplett neu ohne reledmac
\makeatletter
\renewenvironment{theindex}{%
  \section*{\indexname}%
  \setlength{\parindent}{0pt}%
  \setlength{\parskip}{0pt plus 0.3pt}%
  \let\item\@idxitem
}{%
  \clearpage
}
\makeatother

\IfFileExists{\jobname-pw.ind}{\input{\jobname-pw.ind}}{}

\end{document}

      