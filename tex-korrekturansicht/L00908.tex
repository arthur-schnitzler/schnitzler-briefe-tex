%% latex-korrekturansicht-vorspann.tex
%% Vorspann für die Korrekturansicht.
%% Lädt die gemeinsame Datei latex-vorspann.tex mit gesetztem Schalter.

\newif\ifkorrekturansicht
\korrekturansichttrue

\input{../tex-inputs/latex-vorspann}


               \section[Arthur Schnitzler an Hugo von Hofmannsthal, 22. 3. 1899]{ Arthur Schnitzler an Hugo von Hofmannsthal, 22. 3. 1899}\nopagebreak\mylabel{v}\rehead{ }\normalsize\beginnumbering\briefempfaengerindex{Hofmannsthal, Hugo von@\textsc{Hofmannsthal, Hugo von}!zzzSchnitzler, Arthur@\emph{von Arthur Schnitzler}!1899-03-221@{22. 3. 1899}|(be} \toendnotes[C]{\smallbreak\pagebreak[2]} \Standort{FDH, Hs-30885,80.}
\physDesc{Brief, 1 Blatt, 4 Seiten
\newline{}Handschrift: Bleistift, deutsche Kurrent}\buchAbdrucke{\weitereDrucke{1) Hugo von Hofmannsthal, Arthur Schnitzler: \emph{Briefwechsel}. Hg. Therese Nickl und Heinrich Schnitzler. Frankfurt am Main: \emph{S. Fischer} 1964, S. 119–120.} \weitereDrucke{2) Arthur Schnitzler: \emph{Briefe 1875–1912}. Hg. Therese Nickl und Heinrich Schnitzler. Frankfurt am Main: \emph{S. Fischer} 1981, S. 369.} }\toendnotes[C]{\smallbreak}\pstart
           \raggedleft{}{\pb}\uline{22. 3. 99}\pend
           \pstart
           Mein lieber Hugo! ich danke Ihnen ſehr dſs Sie noch einmal bei
                    mir waren. Was ſoll ich Ihnen heute weiter ſagen. Ein Tag ist ſchrecklicher als
                    der andre; es iſt viel grauenvoller und hoffnungsloſer als irgend ein Wort
                    darüber. Ich habe das Gefühl, fertig zu ſein; Zeichen genug werden mir geſandt!
                    Vom Morgen aus der Ausblick ins leere, {\pb}leere – die
                    Erinnerungen an \textcolor{blue}{ihr}{}\ledrightnote{→\textcolor{blue}{Marie Reinhard}} Leben
                    voll Pein, an ihren Tod von einer grenzenloſen Entſetzlichkeit{\dotstwo} die letzten Blicke, die letzten Worte unvergeßlich
                    – die letzte Angſt auf i{\geminationm}er alles zerſtörend, was
                    noch ko{\geminationm}en könnte. Eine ungeheure Gleichgiltigkeit
                    gegen alles, was mir auch Inhalt des Lebens ſchien – ſchauen ins leere, {\pb}greifen ins leere, ja{\geminationm}ern ins leere.\pend
           \pstart
           Vielleicht fahre ich auf einen Tag nach \textcolor{pink}{Graz}{}\ledrightnote{\textcolor{pink}{Graz}}, wo
                    ihre \textcolor{blue}{Schweſter}{}\ledrightnote{→\textcolor{blue}{Caroline Burger}} und jetzt
                    auch ihr \textcolor{blue}{Vater}{}\ledrightnote{→\textcolor{blue}{Carl Reinhard}} u von
                    morgen an ihre \textcolor{blue}{Mutter}{}\ledrightnote{→\textcolor{blue}{Therese Reinhard}} iſt.
                    Alle Menſchen ſind ſehr gut zu mir; – ich möchte danken können. Eine Einſamkeit
                    ohne gleichen – ich muß dran denken, wie ich doch i{\geminationm}er die Menſchen zu ſchildern verſucht habe, die ihr geliebteſtes verlieren –
                        {\pb}es gibt eben etwas, das nicht auszudrücken iſt –
                    ſo gut wie die Ewigkeit, die Unendlichkeit: – die Einſamkeit, das \uline{Ver}einſamtſein; \uline{ver}einſamt \uline{werden}.\pend
           \pstart
           Leben Sie wohl, liebſter Hugo. Ko{\geminationm}en Sie bald
                    zurück!? Bitte ſchreiben Sie mir nur äußere Vorkommniſſe, \uline{nichts}{ }\uuline{da}\uline{rüber}.\pend
           \pstart
           – Sagen Sie es \textcolor{blue}{Brahm}{}\ledrightnote{\textcolor{blue}{Otto Brahm}} u \textcolor{blue}{Hirſchfeld}{}\ledrightnote{\textcolor{blue}{Georg Hirschfeld}}, damit ſie’s wiſſen, we{\geminationn} ich komme.\pend
           \pstart Von Herzen Ihr \spacefill\mbox{Arthur}\pend{}\endnumbering\briefempfaengerindex{Hofmannsthal, Hugo von@\textsc{Hofmannsthal, Hugo von}!zzzSchnitzler, Arthur@\emph{von Arthur Schnitzler}!1899-03-221@{22. 3. 1899}|)be}\mylabel{h}  \normalsize

\doendnotes{C}
\bigskip
\vfill

\clearpage

\footnotesize

\lohead{\textsc{register}}

% Definiere theindex-Environment komplett neu ohne reledmac
\makeatletter
\renewenvironment{theindex}{%
  \section*{\indexname}%
  \setlength{\parindent}{0pt}%
  \setlength{\parskip}{0pt plus 0.3pt}%
  \let\item\@idxitem
}{%
  \clearpage
}
\makeatother

\IfFileExists{\jobname-pw.ind}{\input{\jobname-pw.ind}}{}

\end{document}

      