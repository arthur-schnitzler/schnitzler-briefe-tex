%% latex-korrekturansicht-vorspann.tex
%% Vorspann für die Korrekturansicht.
%% Lädt die gemeinsame Datei latex-vorspann.tex mit gesetztem Schalter.

\newif\ifkorrekturansicht
\korrekturansichttrue

\input{../tex-inputs/latex-vorspann}


\section[Stefan Zweig an Arthur Schnitzler, 23. 7. 1923]{L03667 Stefan Zweig an Arthur Schnitzler, 23. 7. 1923}
\nopagebreak\mylabel{L03667v}
\rehead{ }\normalsize\beginnumbering\briefempfaengerindex{, @\textsc{, }!zzz, @\emph{von  }!1923-07-232@{23. 7. 1923}|(be}
\toendnotes[C]{\smallbreak\pagebreak[2]}\Standort{CUL, Schnitzler, B 118.}
\physDesc{Brief, 1 Blatt, 1 Seite, 810 Zeichen
\newline{}Handschrift: blaue Tinte, lateinische Kurrent}
\buchAbdrucke{\weitereDrucke{Stefan Zweig: \emph{Briefwechsel mit Hermann Bahr, Sigmund Freud, Rainer Maria
                        Rilke und Arthur Schnitzler}. Herausgegeben von Jeffrey B. Berlin, Hans-Ulrich Lindken und Donald A. Prater. Frankfurt am Main: \emph{S. Fischer} 1987, S. 416.} }\toendnotes[C]{\smallbreak}
\pstart
           {\pb}\textcolor{gray}{\textbf{SZ}}\hfill \textcolor{gray}{\textbf{\textcolor{pink}{KAPUZINERBERG 5}\oindex{Paschinger Schlössl@\textbf{Paschinger Schlössl}, \emph{Wohngebäude}|pw}{}\ledrightnote{\textcolor{pink}{Paschinger Schlössl}}}}{ }23. Juli 1923\pend
           
\pstart
           \raggedleft{}\textcolor{gray}{\textbf{\textcolor{pink}{SALZBURG}\oindex{Salzburg@\textbf{Salzburg}, \emph{Verwaltungsgebiet}|pw}{}\ledrightnote{\textcolor{pink}{Salzburg}}}}\pend
           \vspace{0.5em}
\pstart
           Lieber verehrter Herr Doktor, ich weiss nicht, wo Sie den Sommer
               verbringen, vermute Sie aber im \textcolor{pink}{Salzkammergut}\oindex{Salzkammergut@\textbf{Salzkammergut}, \emph{Region}|pw}{}\ledrightnote{\textcolor{pink}{Salzkammergut}}
               oder im \textcolor{pink}{Bayrischen}\oindex{Bayern@\textbf{Bayern}, \emph{Land}|pw}{}\ledrightnote{\textcolor{pink}{Bayern}} und wollte nicht verabsäumen,
               Ihnen etwas – \uline{allerdings streng vertraulich!} – zu
               sagen. Unser verehrter Freund \textcolor{blue}{Romain Rolland}\pwindex{Rolland, Romain 29.\,1.\,1866 Clamecy – 30.\,12.\,1944 Vézelay@\textsc{Rolland, Romain} (29.\,1.\,1866 Clamecy – 30.\,12.\,1944 Vézelay), \emph{Schriftsteller}|pw}{}\ledrightnote{\textcolor{blue}{Romain Rolland}}
               wird vom 1–10 August bei uns in \textcolor{pink}{Salzburg}\oindex{Salzburg@\textbf{Salzburg}, \emph{Verwaltungsgebiet}|pw}{}\ledrightnote{\textcolor{pink}{Salzburg}} zu Gast sein und, wenn Sie nahe sind, wäre es doch
               wunderschon, Sie \label{K_L03667-1v}\edtext{kämen für einen Tag}{\lemma{\textnormal{\emph{kämen für einen Tag}}}\Cendnote{\textnormal{\textcolor{blue}{Schnitzler} folgte der Einladung
                  und verbrachte den 4. 8. 1923 mit \textcolor{blue}{Romain Rolland}\pwindex{Rolland, Romain 29.\,1.\,1866 Clamecy – 30.\,12.\,1944 Vézelay@\textsc{Rolland, Romain} (29.\,1.\,1866 Clamecy – 30.\,12.\,1944 Vézelay), \emph{Schriftsteller}|pwk} und \textcolor{blue}{Zweig}\pwindex{Zweig, Stefan 28.\,11.\,1881 Wien – 23.\,2.\,1942 Petrópolis@\textsc{Zweig, Stefan} (28.\,11.\,1881 Wien – 23.\,2.\,1942 Petrópolis), \emph{Schriftsteller}|pwk}
                  in \textcolor{pink}{Salzburg}\oindex{Salzburg@\textbf{Salzburg}, \emph{Verwaltungsgebiet}|pwk}.}}}\label{K_L03667-1} vorbei. Nichts scheint mir nötiger, als dass die
               paar wesentlichen Menschen unserer zersprengten Zeit einander persönlich kennen; und
               wenn Sie gerade in der Nähe sind, wäre es doch natürlich, dass Sie Sich und ihm (und
               uns) die Freude Ihrer Gegenwart machten. Ich mute Ihnen natürlich nicht eine Reise
               zu, aber einen Ausflug von nahe her ist dieser wunderbare Mensch wohl wert. \pend
           
\pstart
           Immer in Liebe und Verehrung der Ihre!{\\[\baselineskip]}\spacefill\mbox{Stefan Zweig}\pend
           \leftskip=0em{}\selectlanguage{ngerman}\endnumbering\briefempfaengerindex{, @\textsc{, }!zzz, @\emph{von  }!1923-07-232@{23. 7. 1923}|)be}\mylabel{L03667h}
\begin{anhang}
\end{anhang}\normalsize

\doendnotes{C}
\bigskip
\vfill

\clearpage

\footnotesize

\lohead{\textsc{register}}

% Definiere theindex-Environment komplett neu ohne reledmac
\makeatletter
\renewenvironment{theindex}{%
  \section*{\indexname}%
  \setlength{\parindent}{0pt}%
  \setlength{\parskip}{0pt plus 0.3pt}%
  \let\item\@idxitem
}{%
  \clearpage
}
\makeatother

\IfFileExists{\jobname-pw.ind}{\input{\jobname-pw.ind}}{}

\end{document}

      