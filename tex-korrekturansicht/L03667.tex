%% latex-korrekturansicht-vorspann.tex
%% Vorspann für die Korrekturansicht.
%% Lädt die gemeinsame Datei latex-vorspann.tex mit gesetztem Schalter.

\newif\ifkorrekturansicht
\korrekturansichttrue

\input{../tex-inputs/latex-vorspann}


\renewcommand{\erwaehntePersonen}{Personen: Romain Rolland, Stefan Zweig}
\renewcommand{\erwaehnteOrte}{Orte: Bayern, Paschinger Schlössl, Salzburg, Salzkammergut, Wien}
\renewcommand{\erwaehnteWerke}{}
\section[Stefan Zweig an Arthur Schnitzler, 23. 7. 1923]{Stefan Zweig an Arthur Schnitzler, 23. 7. 1923}
\nopagebreak\mylabel{v}
\rehead{ }\normalsize\beginnumbering\briefempfaengerindex{Schnitzler, Arthur@\textsc{Schnitzler, Arthur}!zzzZweig, Stefan@\emph{von Stefan Zweig}!1923-07-231@{23. 7. 1923}|(be}
\toendnotes[C]{\smallbreak\pagebreak[2]}\Standort{CUL, Schnitzler, B 118.}
\physDesc{Brief, 1 Blatt, 1 Seite, 810 Zeichen
\newline{}Handschrift: blaue Tinte, lateinische Kurrent}\toendnotes[C]{\smallbreak}
\pstart
           {\pb}\textcolor{gray}{\textbf{SZ}}\hfill \textcolor{gray}{\textbf{\textcolor{pink}{KAPUZINERBERG 5}{}\ledrightnote{\textcolor{pink}{Paschinger Schlössl}}}}{ }23. Juli 1923\pend
           
\pstart
           \raggedleft{}\textcolor{gray}{\textbf{\textcolor{pink}{SALZBURG}{}\ledrightnote{\textcolor{pink}{Salzburg}}}}\pend
           
\pstart
           Lieber verehrter Herr Doktor, ich weiss nicht, wo Sie den Sommer
               verbringen, vermute Sie aber im \textcolor{pink}{Salzkammergut}{}\ledrightnote{\textcolor{pink}{Salzkammergut}}
               oder im \textcolor{pink}{Bayrischen}{}\ledrightnote{\textcolor{pink}{Bayern}} und wollte nicht verabsäumen,
               Ihnen etwas – \uline{allerdings streng vertraulich!} – zu
               sagen. Unser verehrter Freund \textcolor{blue}{Romain Rolland}{}\ledrightnote{\textcolor{blue}{Romain Rolland}}
               wird vom 1–10 August bei uns in \textcolor{pink}{Salzburg}{}\ledrightnote{\textcolor{pink}{Salzburg}} zu Gast sein und, wenn Sie nahe sind, wäre es doch
               wunderschon, Sie \label{K_L03667-1v}\edtext{kämen für einen Tag}{\lemma{\textnormal{\emph{kämen für einen Tag}}}\Cendnote{\textnormal{\textcolor{blue}{Schnitzler} folgte der Einladung
                  und verbrachte den 4. 8. 1923 mit \textcolor{blue}{Romain Rolland} und \textcolor{blue}{Zweig}
                  in \textcolor{pink}{Salzburg}.}}}\label{K_L03667-1h} vorbei. Nichts scheint mir nötiger, als dass die
               paar wesentlichen Menschen unserer zersprengten Zeit einander persönlich kennen; und
               wenn Sie gerade in der Nähe sind, wäre es doch natürlich, dass Sie Sich und ihm (und
               uns) die Freude Ihrer Gegenwart machten. Ich mute Ihnen natürlich nicht eine Reise
               zu, aber einen Ausflug von nahe her ist dieser wunderbare Mensch wohl wert. \pend
           
\pstart
           Immer in Liebe und Verehrung der Ihre!{\\[\baselineskip]}\spacefill\mbox{Stefan Zweig}\pend
           \leftskip=0em{}\endnumbering\briefempfaengerindex{Schnitzler, Arthur@\textsc{Schnitzler, Arthur}!zzzZweig, Stefan@\emph{von Stefan Zweig}!1923-07-231@{23. 7. 1923}|)be}\mylabel{h}
\begin{anhang}
\end{anhang}\normalsize

\doendnotes{C}
\bigskip
\vfill

\clearpage

\footnotesize

\lohead{\textsc{register}}

% Definiere theindex-Environment komplett neu ohne reledmac
\makeatletter
\renewenvironment{theindex}{%
  \section*{\indexname}%
  \setlength{\parindent}{0pt}%
  \setlength{\parskip}{0pt plus 0.3pt}%
  \let\item\@idxitem
}{%
  \clearpage
}
\makeatother

\IfFileExists{\jobname-pw.ind}{\input{\jobname-pw.ind}}{}

\end{document}

      