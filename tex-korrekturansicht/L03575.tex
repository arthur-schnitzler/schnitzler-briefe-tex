%% latex-korrekturansicht-vorspann.tex
%% Vorspann für die Korrekturansicht.
%% Lädt die gemeinsame Datei latex-vorspann.tex mit gesetztem Schalter.

\newif\ifkorrekturansicht
\korrekturansichttrue

\input{../tex-inputs/latex-vorspann}


\renewcommand{\erwaehntePersonen}{Personen: Lili Cappellini, Helene Jarosy, Richard Metzl, Felix Salten, Ottilie Salten, Olga Schnitzler, Heinrich Schnitzler, Julius Ferdinand Wollf, Johanna Sophie Wollf}
\renewcommand{\erwaehnteOrte}{Orte: Brennerriesensteig, Brijuni, Salzkammergut, Schafberg (St. Gilgen), Steinbach am Attersee, Unterach am Attersee}
\renewcommand{\erwaehnteWerke}{Werke: Tagebuch}
\section[Felix Salten u. a. an Arthur und Olga Schnitzler, {[}Ende Juli – 24. 8. 1912?{]}]{Felix Salten u. a. an Arthur und Olga Schnitzler, {[}Ende Juli –
               24. 8. 1912?{]}}
\nopagebreak\mylabel{v}
\rehead{ }\normalsize\beginnumbering\briefempfaengerindex{Schnitzler, Olga@\textsc{Schnitzler, Olga}!zzzMetzl, Richard@\emph{von Richard Metzl}!1912-08-242@{{[}Ende Juli –
                  24. 8. 1912?{]}}|(be}\briefempfaengerindex{Schnitzler, Olga@\textsc{Schnitzler, Olga}!zzzJarosy, Helene@\emph{von Helene Jarosy}!1912-08-242@{{[}Ende Juli –
                  24. 8. 1912?{]}}|(be}\briefempfaengerindex{Schnitzler, Olga@\textsc{Schnitzler, Olga}!zzzWollf, Julius Ferdinand@\emph{von Julius Ferdinand Wollf}!1912-08-242@{{[}Ende Juli –
                  24. 8. 1912?{]}}|(be}\briefempfaengerindex{Schnitzler, Olga@\textsc{Schnitzler, Olga}!zzzSalten, Ottilie@\emph{von Ottilie Salten}!1912-08-242@{{[}Ende Juli –
                  24. 8. 1912?{]}}|(be}\briefempfaengerindex{Schnitzler, Olga@\textsc{Schnitzler, Olga}!zzzSalten, Felix@\emph{von Felix Salten}!1912-08-242@{{[}Ende Juli –
                  24. 8. 1912?{]}}|(be}\briefempfaengerindex{Schnitzler, Arthur@\textsc{Schnitzler, Arthur}!zzzMetzl, Richard@\emph{von Richard Metzl}!1912-08-242@{{[}Ende Juli –
                  24. 8. 1912?{]}}|(be}\briefempfaengerindex{Schnitzler, Arthur@\textsc{Schnitzler, Arthur}!zzzJarosy, Helene@\emph{von Helene Jarosy}!1912-08-242@{{[}Ende Juli –
                  24. 8. 1912?{]}}|(be}\briefempfaengerindex{Schnitzler, Arthur@\textsc{Schnitzler, Arthur}!zzzWollf, Julius Ferdinand@\emph{von Julius Ferdinand Wollf}!1912-08-242@{{[}Ende Juli –
                  24. 8. 1912?{]}}|(be}\briefempfaengerindex{Schnitzler, Arthur@\textsc{Schnitzler, Arthur}!zzzSalten, Ottilie@\emph{von Ottilie Salten}!1912-08-242@{{[}Ende Juli –
                  24. 8. 1912?{]}}|(be}\briefempfaengerindex{Schnitzler, Arthur@\textsc{Schnitzler, Arthur}!zzzSalten, Felix@\emph{von Felix Salten}!1912-08-242@{{[}Ende Juli –
                  24. 8. 1912?{]}}|(be}
\toendnotes[C]{\smallbreak\pagebreak[2]}\Standort{CUL, Schnitzler, B 89, B 2.}
\physDesc{Bildpostkarte, 403 Zeichen
\newline{}Handschrift Felix Salten: schwarze Tinte, lateinische Kurrent
\newline{}Handschrift Ottilie Salten: schwarze Tinte, lateinische Kurrent
\newline{}Handschrift Julius Ferdinand Wollf: schwarze Tinte, lateinische Kurrent
\newline{}Handschrift Helene Jarosy: schwarze Tinte, lateinische Kurrent
\newline{}Handschrift Richard Metzl: schwarze Tinte, deutsche Kurrent
\newline{}Versand: Stempel: »\nobreak{}\oindex{Unterach am Attersee@\textbf{Unterach am Attersee}, \emph{P.PPL}|pwk}Unter\textcolor{gray}{ach am
                                       Attersee}\nobreak{}«.  
\newline{}Ordnung: mit Bleistift von unbekannter Hand nummeriert: »288« }\toendnotes[C]{\smallbreak}\pstart{}{\pb}Herrn u. Frau\pend{}\pstart{}D\textsuperscript{r} Arthur Schnitzler\pend{}\pstart{}\textcolor{pink}{Brioni}{}\ledrightnote{\textcolor{pink}{Brijuni}}\pend{}
{\bigskip}
\pstart
           \noindent{}\centering{}{\pb}\textcolor{gray}{\textbf{\textcolor{pink}{Salzkammergut}{}\ledrightnote{\textcolor{pink}{Salzkammergut}}. Blick vom \textcolor{pink}{Brennerriesensteig}{}\ledrightnote{\textcolor{pink}{Brennerriesensteig}} bei \textcolor{pink}{Steinbach auf den Attersee}{}\ledrightnote{\textcolor{pink}{Steinbach am Attersee}} u. \textcolor{pink}{Schafberg}{}\ledrightnote{\textcolor{pink}{Schafberg (St. Gilgen)}}.}}\pend
           
\pstart
           {\pb}Lieber Arthur und liebe Olga, wir haben \label{K_L03575-1v}\edtext{heute}{\lemma{\textnormal{\emph{heute}}}\Cendnote{\textnormal{Die Bildpostkarte ist undatiert und der
                  Stempel nur teilweise gedruckt. In Frage kommen zwei längere Aufenthalte \textcolor{blue}{Schnitzler}s in \textcolor{pink}{Brijuni}: von 21. 7. 1912 bis 
                  zum 24. 8. 1912
                  und, im Folgejahr, von 24. 7. 1913
                  bis zum 22. 8. 1913. Für beide 
                  Jahre ist im \emph{\textcolor{green}{Tagebuch}} keine persönliche Interaktion zwischen \textcolor{blue}{Schnitzler} und \textcolor{blue}{Salten}
                  in und rund um die Zeiträume festgehalten. Nur für das Jahr 1912 liegen
                  Korrespondenzstücke vor (Felix Salten an Arthur Schnitzler, 2. 7. 1912 und 22. 7. 1912), die belegen, dass ein Austausch stattfand. Das wird als entscheidendes Indiz gewertet, dass 
                  diese Karte im Jahr zu verorten ist. Auch lässt sich für 1912 ein
                  dreiwöchiger Besuch des \textcolor{blue}{Ehepaar Wollf} belegen (siehe Felix Salten an Olga Schnitzler, 2. 9. 1912).
                 Damit ist die Karte aber nach \textcolor{blue}{Salten}s Brief vom 22. 7. 1912 einzuordnen,
                  da dieser mit \textcolor{blue}{Schnitzler}s Urlaubsbeginn
                  zusammenfällt und keine Anwesenheit weiterer Freunde thematisiert wird. 
                   Nach hinten ist die Datierung durch \textcolor{blue}{Schnitzler}s Abreise am 24. 8. 1912 eingrenzbar.}}}\label{K_L03575-1h} in Herzlichkeit
               Ihrer gedacht und senden Ihnen viele Grüße! Hoffentlich haben Sie mit den \textcolor{blue}{Kinder}{}\ledrightnote{{$\rightarrow$}\textcolor{blue}{Heinrich Schnitzler}{\newline}{$\rightarrow$}\textcolor{blue}{Lili Cappellini}}n schöne Tage.
               Herzlichst Ihr {\\}\spacefill\mbox{Salten}\pend
           
\pstart
           \noindent{}{[}hs. Ottilie Salten:{]} Viele herzliche Grüße
                  \spacefill\mbox{Ottilie}\pend
           
\pstart
           \noindent{}{[}hs. Wollf:{]} Viele Grüsse von Ihrem ergebenen\pend
           \pstart \spacefill\mbox{Julius Ferdinand Wollf} und seiner \textcolor{blue}{Frau}{}\ledrightnote{{$\rightarrow$}\textcolor{blue}{Johanna Sophie Wollf}}\pend{}
\pstart
           \noindent{}{[}hs. Jarosy:{]} Die schönsten Grüße Ihnen und der gnädigen Frau
                  \spacefill\mbox{Helene Jaroſy}\pend
           
\pstart
           \noindent{}{[}hs. Metzl:{]} Beſte Grüße {\\}Ihr ergebener {\\}\spacefill\mbox{RichardMetzl}\pend
           \endnumbering\briefempfaengerindex{Schnitzler, Olga@\textsc{Schnitzler, Olga}!zzzMetzl, Richard@\emph{von Richard Metzl}!1912-07-222@{{[}Ende Juli –
                  24. 8. 1912?{]}}|)be}\briefempfaengerindex{Schnitzler, Olga@\textsc{Schnitzler, Olga}!zzzJarosy, Helene@\emph{von Helene Jarosy}!1912-07-222@{{[}Ende Juli –
                  24. 8. 1912?{]}}|)be}\briefempfaengerindex{Schnitzler, Olga@\textsc{Schnitzler, Olga}!zzzWollf, Julius Ferdinand@\emph{von Julius Ferdinand Wollf}!1912-07-222@{{[}Ende Juli –
                  24. 8. 1912?{]}}|)be}\briefempfaengerindex{Schnitzler, Olga@\textsc{Schnitzler, Olga}!zzzSalten, Ottilie@\emph{von Ottilie Salten}!1912-07-222@{{[}Ende Juli –
                  24. 8. 1912?{]}}|)be}\briefempfaengerindex{Schnitzler, Olga@\textsc{Schnitzler, Olga}!zzzSalten, Felix@\emph{von Felix Salten}!1912-07-222@{{[}Ende Juli –
                  24. 8. 1912?{]}}|)be}\briefempfaengerindex{Schnitzler, Arthur@\textsc{Schnitzler, Arthur}!zzzMetzl, Richard@\emph{von Richard Metzl}!1912-07-222@{{[}Ende Juli –
                  24. 8. 1912?{]}}|)be}\briefempfaengerindex{Schnitzler, Arthur@\textsc{Schnitzler, Arthur}!zzzJarosy, Helene@\emph{von Helene Jarosy}!1912-07-222@{{[}Ende Juli –
                  24. 8. 1912?{]}}|)be}\briefempfaengerindex{Schnitzler, Arthur@\textsc{Schnitzler, Arthur}!zzzWollf, Julius Ferdinand@\emph{von Julius Ferdinand Wollf}!1912-07-222@{{[}Ende Juli –
                  24. 8. 1912?{]}}|)be}\briefempfaengerindex{Schnitzler, Arthur@\textsc{Schnitzler, Arthur}!zzzSalten, Ottilie@\emph{von Ottilie Salten}!1912-07-222@{{[}Ende Juli –
                  24. 8. 1912?{]}}|)be}\briefempfaengerindex{Schnitzler, Arthur@\textsc{Schnitzler, Arthur}!zzzSalten, Felix@\emph{von Felix Salten}!1912-07-222@{{[}Ende Juli –
                  24. 8. 1912?{]}}|)be}\mylabel{h}  \normalsize

\doendnotes{C}
\bigskip
\vfill

\clearpage

\footnotesize

\lohead{\textsc{register}}

% Definiere theindex-Environment komplett neu ohne reledmac
\makeatletter
\renewenvironment{theindex}{%
  \section*{\indexname}%
  \setlength{\parindent}{0pt}%
  \setlength{\parskip}{0pt plus 0.3pt}%
  \let\item\@idxitem
}{%
  \clearpage
}
\makeatother

\IfFileExists{\jobname-pw.ind}{\input{\jobname-pw.ind}}{}

\end{document}

      