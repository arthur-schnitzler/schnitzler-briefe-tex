%% latex-korrekturansicht-vorspann.tex
%% Vorspann für die Korrekturansicht.
%% Lädt die gemeinsame Datei latex-vorspann.tex mit gesetztem Schalter.

\newif\ifkorrekturansicht
\korrekturansichttrue

\input{../tex-inputs/latex-vorspann}


\section[Sigmund Freud an Arthur Schnitzler, 24. 3. 1926]{L03889 Sigmund Freud an Arthur Schnitzler, 24. 3. 1926}
\nopagebreak\mylabel{L03889v}
\rehead{ }\normalsize\beginnumbering\briefempfaengerindex{, @\textsc{, }!zzz, @\emph{von  }!1926-03-241@{24. 3. 1926}|(be}
\toendnotes[C]{\smallbreak\pagebreak[2]}\Standort{Washington, DC, Library of Congress, Freud Archives, C41F8.}
\physDesc{Brief, Fotokopie, 1 Blatt, 1 Seite, 478 Zeichen
\newline{}Schreibmaschine
\newline{}Handschrift: schwarze Tinte (\noindent{}Unterschrift)
\newline{}Schnitzler: mit rotem Buntstift eine Unterstreichung 
\newline{}Zusatz: Der Verbleib des Originals ist ungeklärt. Zum Zeitpunkt der
                                 ersten Edition 1955 befand es sich im Besitz von \textcolor{blue}{Heinrich Schnitzler}\pwindex{Schnitzler, Heinrich 9.\,8.\,1902 Hinterbrühl – 12.\,7.\,1982 Wien@\textsc{Schnitzler, Heinrich} (9.\,8.\,1902 Hinterbrühl – 12.\,7.\,1982 Wien), \emph{Regisseur, Schauspieler}|pw}. }
\buchAbdrucke{\weitereDrucke{1) Sigmund Freud: \emph{Briefe an Arthur Schnitzler.} Herausgegeben von Henry Schnitzler. In: \emph{Neue deutsche Rundschau}, Jg. 66 (Januar 1955) Nr. 1, S. 99.} \weitereDrucke{2) Sigmund Freud: \emph{Sigmund Freud Edition. Digitale historisch-kritische
                        Gesamtausgabe}. Herausgegeben von Christine Diercks,  Arkadi Blatow und  Elisabeth Skale. (2014–2025) \url{https://www.freudedition.net/briefe/freud-sigmund/schnitzler-arthur/1926/03/24}.} }\toendnotes[C]{\smallbreak}
\pstart
           {\pb}\textcolor{gray}{\textbf{PROF. D\textsuperscript{R.} FREUD}}\hfill \textcolor{gray}{\textbf{\textcolor{pink}{WIEN IX., BERGGASSE 19}\oindex{Wien@\textbf{Wien}!IX., Alsergrund@\textbf{IX., Alsergrund}!Berggasse 19@\textbf{Berggasse 19}, \emph{Wohngebäude}|pw}{}\ledrightnote{\textcolor{pink}{Berggasse 19}}. }}\pend
           
\pstart
           \raggedleft{}24. III. 26. \pend
           
\pstart{}Verehrtester\pend\vspace{0.5em}
\pstart
           Es hat mir ausserordentlich leid getan, dass Sie unlängst einen erfolglosen Besuch
               bei mir machten. Mein Tag ist in diesem \textcolor{green}{\textcolor{pink}{Zauberberg}\oindex{Wien@\textbf{Wien}!XVIII., Währing@\textbf{XVIII., Währing}!Cottage-Sanatorium für Nerven- und Stoffwechselkranke@\textbf{Cottage-Sanatorium für Nerven- und Stoffwechselkranke}, \emph{Sanatorium}|pwv}{}\ledrightnote{{$\rightarrow$}\emph{\textcolor{pink}{Cottage-Sanatorium für Nerven- und Stoffwechselkranke}}}}\pwindex{Mann, Thomas 6.\,6.\,1875 Lübeck – 12.\,8.\,1955 Zürich@\textsc{Mann, Thomas} (6.\,6.\,1875 Lübeck – 12.\,8.\,1955 Zürich), \emph{Schriftsteller}!Zauberberg. Roman@\strich\emph{Der Zauberberg. Roman}|pwv}{}\ledrightnote{{$\rightarrow$}\emph{\textcolor{green}{Der Zauberberg. Roman}}} oder dieser \textcolor{pink}{Zauberhöhle}\oindex{Wien@\textbf{Wien}!XVIII., Währing@\textbf{XVIII., Währing}!Cottage-Sanatorium für Nerven- und Stoffwechselkranke@\textbf{Cottage-Sanatorium für Nerven- und Stoffwechselkranke}, \emph{Sanatorium}|pwv}{}\ledrightnote{{$\rightarrow$}\emph{\textcolor{pink}{Cottage-Sanatorium für Nerven- und Stoffwechselkranke}}} so kunstvoll eingeteilt, dass mir für Genüsse
               nur der Abend bleibt. Darf ich Ihnen vorschlagen, mich \label{K_L03889-1v}\edtext{heute}{\lemma{\textnormal{\emph{heute}}}\Cendnote{\textnormal{\textcolor{blue}{Schnitzler} besuchte \textcolor{blue}{Freud}\pwindex{Freud, Sigmund 6.\,5.\,1856 Pribor – 23.\,9.\,1939 London@\textsc{Freud, Sigmund} (6.\,5.\,1856 Pribor – 23.\,9.\,1939 London), \emph{Psychoanalytiker}|pwk} erst am 26. 3. 1926 im \textcolor{pink}{Cottage-Sanatorium}\oindex{Wien@\textbf{Wien}!XVIII., Währing@\textbf{XVIII., Währing}!Cottage-Sanatorium für Nerven- und Stoffwechselkranke@\textbf{Cottage-Sanatorium für Nerven- und Stoffwechselkranke}, \emph{Sanatorium}|pwk}.}}}\label{K_L03889-1} nach 8 oder 8 ¼ Uhr,
               nachdem das Nachtmahl absolviert ist, auf Gedankenaustausch und Zigarre zu beehren?
               Oder Ueberbringer dieses eine andere Bestimmung mitzugeben?\pend
           
\pstart
           Mit nachbarlichem Gruss{\\[\baselineskip]} Ihr \spacefill\mbox{{[}hs.:{]} Freud}\pend
           \leftskip=0em{}\selectlanguage{ngerman}\endnumbering\briefempfaengerindex{, @\textsc{, }!zzz, @\emph{von  }!1926-03-241@{24. 3. 1926}|)be}\mylabel{L03889h}
\begin{anhang}
\end{anhang}\normalsize

\doendnotes{C}
\bigskip
\vfill

\clearpage

\footnotesize

\lohead{\textsc{register}}

% Definiere theindex-Environment komplett neu ohne reledmac
\makeatletter
\renewenvironment{theindex}{%
  \section*{\indexname}%
  \setlength{\parindent}{0pt}%
  \setlength{\parskip}{0pt plus 0.3pt}%
  \let\item\@idxitem
}{%
  \clearpage
}
\makeatother

\IfFileExists{\jobname-pw.ind}{\input{\jobname-pw.ind}}{}

\end{document}

      