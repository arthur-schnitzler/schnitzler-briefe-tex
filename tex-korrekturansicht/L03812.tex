%% latex-korrekturansicht-vorspann.tex
%% Vorspann für die Korrekturansicht.
%% Lädt die gemeinsame Datei latex-vorspann.tex mit gesetztem Schalter.

\newif\ifkorrekturansicht
\korrekturansichttrue

\input{../tex-inputs/latex-vorspann}


\section[Olga Schnitzler an Stefan Zweig, 3. 5. 1916]{L03812 Olga Schnitzler an Stefan Zweig, 3. 5. 1916}
\nopagebreak\mylabel{L03812v}
\rehead{ }\normalsize\beginnumbering\briefempfaengerindex{, @\textsc{, }!zzz, @\emph{von  }!1916-05-031@{3. 5. 1916}|(be}
\toendnotes[C]{\smallbreak\pagebreak[2]}\Standort{Jerusalem, National Library of Israel, ARC. Ms. Var. 305 1 58 Stefan Zweig Collection.}
\physDesc{Brief, 1 Blatt, 2 Seiten, 893 Zeichen
\newline{}Handschrift: schwarze Tinte, lateinische Kurrent}\toendnotes[C]{\smallbreak}
\pstart
           {\pb}\textcolor{gray}{\textbf{O. S.}}\pend
           \vspace{0.5em}
\pstart
           Lieber Herr Doctor, für Ihre so \label{K_L03812-1v}\edtext{freundlichen Worte}{\lemma{\textnormal{\emph{freundlichen Worte}}}\Cendnote{\textnormal{nicht erhalten}}}\label{K_L03812-1} haben Sie herzlichen Dank! ich freue
               mich, dass meine Stimme \label{K_L03812-2v}\edtext{\textcolor{violet}{neulich}\eventindex{Allgemeine Poliklinik [neues Gebäude]@\textbf{Allgemeine Poliklinik [neues Gebäude]}!Gesangskonzert von Olga Schnitzler, 29.4.1916@Gesangskonzert von Olga Schnitzler, 29.4.1916|pwv}{}\ledrightnote{{$\rightarrow$}\emph{\textcolor{violet}{Gesangskonzert von Olga Schnitzler, 29.4.1916}}}}{\lemma{\textnormal{\emph{neulich}}}\Cendnote{\textnormal{Am 29. 4. 1916{ }\textcolor{violet}{sang \textcolor{blue}{Olga Schnitzler}\pwindex{Schnitzler, Olga 17.\,1.\,1882 Wien – 13.\,1.\,1970 Lugano@\textsc{Schnitzler, Olga} (17.\,1.\,1882 Wien – 13.\,1.\,1970 Lugano), \emph{Schauspielerin, Sängerin}|pwk}}\eventindex{Allgemeine Poliklinik [neues Gebäude]@\textbf{Allgemeine Poliklinik [neues Gebäude]}!Gesangskonzert von Olga Schnitzler, 29.4.1916@Gesangskonzert von Olga Schnitzler, 29.4.1916|pwkv} im neuen Hörsaal der \textcolor{pink}{Allgemeinen
                     Poliklinik}\oindex{Wien@\textbf{Wien}!IX., Alsergrund@\textbf{IX., Alsergrund}!Allgemeine Poliklinik [neues Gebäude]@\textbf{Allgemeine Poliklinik [neues Gebäude]}, \emph{Krankenhaus}|pwk}. }}}\label{K_L03812-2}, trotz des unbehaglichen und ungünstigen Raumes, doch
               halbwegs gut geklungen hat.\pend
           
\pstart
           An \label{K_L03812-3v}\edtext{jenem
                  \textcolor{violet}{Abend im \textcolor{pink}{Volksheim}\oindex{Wien@\textbf{Wien}!XVI., Ottakring@\textbf{XVI., Ottakring}!Volkshochschule Ottakring@\textbf{Volkshochschule Ottakring}, \emph{Gebäude}|pw}{}\ledrightnote{\textcolor{pink}{Volkshochschule Ottakring}}}\eventindex{Volkshochschule Ottakring@\textbf{Volkshochschule Ottakring}!Gesangskonzert Olga Schnitzler, 5.2.1911@Gesangskonzert Olga Schnitzler, 5.2.1911|pwv}{}\ledrightnote{{$\rightarrow$}\emph{\textcolor{violet}{Gesangskonzert Olga Schnitzler, 5.2.1911}}}}{\lemma{\textnormal{\emph{jenem
                  Abend im Volksheim}}}\Cendnote{\textnormal{Der erste öffentliche Auftritt von \textcolor{blue}{Olga Schnitzler}\pwindex{Schnitzler, Olga 17.\,1.\,1882 Wien – 13.\,1.\,1970 Lugano@\textsc{Schnitzler, Olga} (17.\,1.\,1882 Wien – 13.\,1.\,1970 Lugano), \emph{Schauspielerin, Sängerin}|pwk} als Sängering fand am \textcolor{violet}{5. 2. 1911 am \emph{\textcolor{brown}{Verein Volksheim}\orgindex{Verein Volksheim@Verein Volksheim|pwk}}}\eventindex{Volkshochschule Ottakring@\textbf{Volkshochschule Ottakring}!Gesangskonzert Olga Schnitzler, 5.2.1911@Gesangskonzert Olga Schnitzler, 5.2.1911|pwkv} statt. }}}\label{K_L03812-3} habe ich zum überhaupt ersten Mal öffentlich gesungen, und
               habe damals weder meine Stimme noch meine Nerven beherrschen können. Hätt ich nur
               damals schon bei Herrn Kammersänger \textcolor{blue}{Steiner}\pwindex{Steiner, Franz 15.\,9.\,1873 Sopron – 4.\,11.\,1954 Mexico City@\textsc{Steiner, Franz} (15.\,9.\,1873 Sopron – 4.\,11.\,1954 Mexico City), \emph{Sänger}|pw}{}\ledrightnote{\textcolor{blue}{Franz Steiner}}
               studiert! mir wäre mancher Umweg erspart geblieben.\pend
           
\pstart
           {\pb}Es wird Sie wahrscheinlich interessieren, denke ich, dass
                  \textcolor{blue}{Arthur}{}\ledrightnote{} wieder glücklicherweise in’s Arbeiten
               gekommen ist, – er hat mir am \label{K_L03812-4v}\edtext{Ostermontag eine neue \textcolor{green}{Novelle}\pwindex{Schnitzler, Arthur 15. 5. 1862 Wien – 21. 10. 1931 ebd.@\textsc{Schnitzler, Arthur} (15. 5. 1862 Wien – 21. 10. 1931 ebd.), \emph{Schriftsteller, Mediziner}!Flucht in die Finsternis@\strich\emph{Flucht in die Finsternis}|pwv}{}\ledrightnote{{$\rightarrow$}\emph{\textcolor{green}{Flucht in die Finsternis}}} im Umfang von »\textcolor{green}{Frau Beate}\pwindex{Schnitzler, Arthur 15. 5. 1862 Wien – 21. 10. 1931 ebd.@\textsc{Schnitzler, Arthur} (15. 5. 1862 Wien – 21. 10. 1931 ebd.), \emph{Schriftsteller, Mediziner}!Frau Beate und ihr Sohn. Novelle@\strich\emph{Frau Beate und ihr Sohn. Novelle}|pw}{}\ledrightnote{\textcolor{green}{Frau Beate und ihr Sohn. Novelle}}«
               \textcolor{violet}{vorgelesen}\eventindex{Sternwartestraße 71@\textbf{Sternwartestraße 71}!Private Lesung von Wahnsinnsnovelle [Flucht in die Finsternis], 23.4.1916@Private Lesung von Wahnsinnsnovelle [Flucht in die Finsternis], 23.4.1916|pwv}{}\ledrightnote{{$\rightarrow$}\emph{\textcolor{violet}{Private Lesung von Wahnsinnsnovelle [Flucht in die Finsternis], 23.4.1916}}}}{\lemma{\textnormal{\emph{Ostermontag … vorgelesen}}}\Cendnote{\textnormal{Tatsächlich dürfte es der
                  Ostersonntag gewesen sein, vgl. A. S.: \emph{Kulturveranstaltungen}, 23. 4. 1916.}}}\label{K_L03812-4}, – eine ebenso \label{K_L03812-5v}\edtext{\textcolor{green}{grosse}\pwindex{Schnitzler, Arthur 15. 5. 1862 Wien – 21. 10. 1931 ebd.@\textsc{Schnitzler, Arthur} (15. 5. 1862 Wien – 21. 10. 1931 ebd.), \emph{Schriftsteller, Mediziner}!Doktor Gräsler, Badearzt@\strich\emph{Doktor Gräsler, Badearzt}|pwv}{}\ledrightnote{{$\rightarrow$}\emph{\textcolor{green}{Doktor Gräsler, Badearzt}}} ist, seit Monaten
                  fertig}{\lemma{\textnormal{\emph{grosse … fertig}}}\Cendnote{\textnormal{Vgl. A. S.: \emph{Tagebuch}, 8. 11. 1914.}}}\label{K_L03812-5}, – und
               nehmen Sie mir’s nicht übel, wenn ich \textcolor{green}{beide}\pwindex{Schnitzler, Arthur 15. 5. 1862 Wien – 21. 10. 1931 ebd.@\textsc{Schnitzler, Arthur} (15. 5. 1862 Wien – 21. 10. 1931 ebd.), \emph{Schriftsteller, Mediziner}!Flucht in die Finsternis@\strich\emph{Flucht in die Finsternis}|pwv}\pwindex{Schnitzler, Arthur 15. 5. 1862 Wien – 21. 10. 1931 ebd.@\textsc{Schnitzler, Arthur} (15. 5. 1862 Wien – 21. 10. 1931 ebd.), \emph{Schriftsteller, Mediziner}!Doktor Gräsler, Badearzt@\strich\emph{Doktor Gräsler, Badearzt}|pwv}{}\ledrightnote{{$\rightarrow$}\emph{\textcolor{green}{Flucht in die Finsternis}}{\newline}{$\rightarrow$}\emph{\textcolor{green}{Doktor Gräsler, Badearzt}}} – so verschieden sie sind, – sehr schön finde.\pend
           
\pstart
           Seien Sie herzlich gegrüsst – hoffentlich hören wir bald wieder von Ihnen.{\\[\baselineskip]}Ihre{\\[\baselineskip]}\spacefill\mbox{OlgaSchnitzler}\pend
           \leftskip=0em{}
\pstart
           3. Mai 1916.\pend
           \selectlanguage{ngerman}\endnumbering\briefempfaengerindex{, @\textsc{, }!zzz, @\emph{von  }!1916-05-031@{3. 5. 1916}|)be}\mylabel{L03812h}  \normalsize

\doendnotes{C}
\bigskip
\vfill

\clearpage

\footnotesize

\lohead{\textsc{register}}

% Definiere theindex-Environment komplett neu ohne reledmac
\makeatletter
\renewenvironment{theindex}{%
  \section*{\indexname}%
  \setlength{\parindent}{0pt}%
  \setlength{\parskip}{0pt plus 0.3pt}%
  \let\item\@idxitem
}{%
  \clearpage
}
\makeatother

\IfFileExists{\jobname-pw.ind}{\input{\jobname-pw.ind}}{}

\end{document}

      