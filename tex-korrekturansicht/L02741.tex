%% latex-korrekturansicht-vorspann.tex
%% Vorspann für die Korrekturansicht.
%% Lädt die gemeinsame Datei latex-vorspann.tex mit gesetztem Schalter.

\newif\ifkorrekturansicht
\korrekturansichttrue

\input{../tex-inputs/latex-vorspann}


               \section[Paul Goldmann an Arthur Schnitzler, 25. 7. {[}1895{]}]{ Paul Goldmann an Arthur Schnitzler, 25. 7. {[}1895{]}}\nopagebreak\mylabel{v}\rehead{ }\normalsize\beginnumbering\briefempfaengerindex{Schnitzler, Arthur@\textsc{Schnitzler, Arthur}!zzzGoldmann, Paul@\emph{von Paul Goldmann}!1895-07-252@{25. 7. {[}1895{]}}|(be} \toendnotes[C]{\smallbreak\pagebreak[2]} \Standort{DLA, A:Schnitzler, HS.NZ85.1.3165.}
\physDesc{Brief, 2 Blätter, 8 Seiten
\newline{}Handschrift: schwarze Tinte, deutsche Kurrent
\newline{}Schnitzler: 1) mit Bleistift das Jahr »95« vermerkt 2) mit rotem Buntstift zwei Unterstreichungen}\toendnotes[C]{\smallbreak}\pstart
           \noindent{}{\pb}\textcolor{gray}{\textbf{\textbf{\textcolor{brown}{Frankfurter Zeitung}{}\ledrightnote{\textcolor{brown}{Frankfurter Zeitung}}}}}\pend
           \pstart
           \textcolor{gray}{\textbf{(\textcolor{brown}{\begin{otherlanguage}{french}Gazette de Francfort\end{otherlanguage}}{}\ledrightnote{\textcolor{brown}{Frankfurter Zeitung}}). }}\pend
           \pstart
           \textcolor{gray}{\textbf{\textbf{\begin{otherlanguage}{french}Fondateur M. \textcolor{blue}{L.
                                 Sonnemann}{}\ledrightnote{\textcolor{blue}{Leopold Sonnemann}}\end{otherlanguage}.}}}\hfill \textsc{\textcolor{pink}{Paris}{}\ledrightnote{\textcolor{pink}{Paris}}}, 25. Juli.\pend
           \pstart
           \begin{otherlanguage}{french}\textcolor{gray}{\textbf{\textcolor{green}{Journal}{}\ledrightnote{→\textcolor{green}{Frankfurter Zeitung}} politique,
                        financier,}}\end{otherlanguage}\pend
           \pstart
           \begin{otherlanguage}{french}\textcolor{gray}{\textbf{commercial et littéraire.}}\end{otherlanguage}\pend
           \pstart
           \begin{otherlanguage}{french}\textcolor{gray}{\textbf{\textbf{Paraissant trois fois par jour.}}}\end{otherlanguage}\pend
           \pstart
           \begin{otherlanguage}{french}\textcolor{gray}{\textbf{\textbf{Bureau à \textcolor{pink}{Paris}{}\ledrightnote{\textcolor{pink}{Paris}}:}}}\end{otherlanguage}\pend
           \pstart
           \begin{otherlanguage}{french}\textcolor{gray}{\textbf{\textbf{\textcolor{pink}{24. Rue Feydeau}{}\ledrightnote{\textcolor{pink}{rue Feydeau}}.}}}\end{otherlanguage}\pend
           \pstart\center{}Mein lieber Freund,\pend\pstart
           Gern hätte ich Dir Deinen lieben Brief von neulich gleich beantwortet. Aber es gab
               gar ſoviel zu thun.\pend
           \pstart
           Alſo Ihr geht doch noch nach \textsc{\textcolor{pink}{Kopenhagen}{}\ledrightnote{\textcolor{pink}{Kopenhagen}}}? Habt Ihr Nachrichten von Frau \textsc{\textcolor{blue}{Andreas}{}\ledrightnote{\textcolor{blue}{Lou Andreas-Salomé}}}?\pend
           \pstart
           Was mich anlangt, ſo gedenke ich am 1. Auguſt hier
               abzureiſen. Ich gehe nach \textsc{\textcolor{pink}{Toelz}{}\ledrightnote{\textcolor{pink}{Bad Tölz}}} zum Kur-Gebrauche. Ich bin ſehr kank. Seit faſt einem Jahre leide ich an einer
               unerklärlichen Affection des rechten Auges: \textsc{Pupillen}-Ungleichheit. Schmerzen, {\pb}Sehſtörungen \textsc{etc}. Die \strikeout{Ä\textcolor{gray}{r}zte} Ärzte ſagen mir nichts u. drängen nur zur Kur.
               Ich fürchte \label{K_L02741-1v}\edtext{\textsc{tumor cerebri}}{\lemma{\textnormal{\emph{tumor cerebri}}}\Cendnote{\textnormal{lateinisch: Hirntumor}}}\label{K_L02741-1h}.\pend
           \pstart
           So bleibe ich alſo in \textsc{\textcolor{pink}{Toelz}{}\ledrightnote{\textcolor{pink}{Bad Tölz}}} vorausſichtlich vier Wochen. \textsc{\textcolor{pink}{Toelz}{}\ledrightnote{\textcolor{pink}{Bad Tölz}}} liegt etwa zwei Bahnſtunden von \textsc{\textcolor{pink}{Muenchen}{}\ledrightnote{\textcolor{pink}{München}}} entfernt. Zwiſchen dem 23. u. 30. Auguſt bin ich jedenfalls noch dort. Vielleicht
               treffen wir uns alſo in \textsc{\textcolor{pink}{Muenchen}{}\ledrightnote{\textcolor{pink}{München}}} (wenn ich die Kur unterbrechen darf). Oder auch ſonſtwo – ich erwarte Deine
               Dispoſitionen. Wenn Du mir ſofort antworteſt, ſo erreicht mich ein Brief von Dir noch
               hier. Jedenfalls theile ich Dir {\pb}ſofort meine \strikeout{U\textcolor{gray}{n}} Unterwegs-Adreſſe mit, und wir bleiben dann wohl in Verbindung. Wie innig ich
               mich darauf freue, Dich wiederzuſehen, brauche ich kaum zu ſagen. Und \label{K_L02741-255v}\edtext{\textsc{\textcolor{blue}{Richard}{}\ledrightnote{\textcolor{blue}{Richard Beer-Hofmann}}}}{\lemma{\textnormal{\emph{Richard}}}\Cendnote{\textnormal{\textcolor{blue}{Goldmann}, \textcolor{blue}{Schnitzler} und \textcolor{blue}{Richard
                     Beer-Hofmann} sahen sich zwischen 31. 8. 1895 und 6. 9. 1895 mehrfach in und um \textcolor{pink}{München}.}}}\label{K_L02741-255h}, werde ich den auch ſehen?\pend
           \pstart
           Ich habe oft in dieſen Wochen der ſchönen Tage im \label{K_L02741-656v}\edtext{vorigen Jahre}{\lemma{\textnormal{\emph{vorigen Jahre}}}\Cendnote{\textnormal{siehe A. S.: \emph{Tagebuch}, 23. 8. 1894}}}\label{K_L02741-656h} gedacht. Ich wünſchte, ich könnte wieder hin, nach \textsc{\textcolor{pink}{Ischl}{}\ledrightnote{\textcolor{pink}{Bad Ischl}}}{ }\strikeout{z} und zu Euch. Ich habe Heimweh nach dem Allen. Du
               ahnſt nicht, mein lieber Freund, wie verzweifelt und troſlos ich bin. Manchmal ſtaune
               ich über mich ſelber, daß ich {\pb}noch aufrechtſtehe{\dotssix}\pend
           \pstart
           Ich ſende Dir anbei die geſammelten \textcolor{green}{Artikel}{}\ledrightnote{→\textcolor{green}{Querelles littéraires}} von \textsc{\textcolor{blue}{Henry Becque}{}\ledrightnote{\textcolor{blue}{Henry Becque}}}, mit der Bitte, mir das \label{K_L02741-88v}\edtext{\textcolor{green}{Buch}{}\ledrightnote{→\textcolor{green}{Querelles littéraires}}}{\lemma{\textnormal{\emph{Buch}}}\Cendnote{\textnormal{\textcolor{blue}{Henry Becque}:
                        \emph{\textcolor{green}{Querelles Littéraires}}. Avec un portrait
                     hors texte. Paris: \emph{Les éditions G.
                        Crès}{ }1890.}}}\label{K_L02741-88h} gelegentlich zurückzuſchicken. Es iſt Alles
               perſönliche Polemik, recht dürr und wenig erfreulich. Aber ich denke mir, wenn Dich
               die Theater-Canaillen kränken, wirſt Du vielleicht ein wenig Troſt darin finden, daß
               es Anderen noch ſchlimmer geht \textcolor{gray}{–} auch iſt doch der Haß des \strikeout{\textcolor{blue}{Mann}{}\ledrightnote{→\textcolor{blue}{Henry Becque}}\textcolor{gray}{e}}{ }\textcolor{blue}{Mann}{}\ledrightnote{→\textcolor{blue}{Henry Becque}}es (\textsc{\textcolor{blue}{Becque}{}\ledrightnote{\textcolor{blue}{Henry Becque}}}) mit all’ dem Klatſch, den er aufrührt, manchmal recht amüſant. In den
               Druckſachen, die ich Dir dieſer Tage {\pb}ſandte, iſt
               diesmal wenig Beſonderes. Ich empfehle Dir nur in der »\textsc{\textcolor{green}{Revue Blanche}{}\ledrightnote{\textcolor{green}{La Revue blanche}}}« \strikeout{die \textcolor{green}{Geſchi}{}\ledrightnote{→\textcolor{green}{Pour le Cœur gros de la Poupée}}} die recht nette \label{K_L02741-7v}\edtext{\textcolor{green}{Geſchichte}{}\ledrightnote{→\textcolor{green}{Pour le Cœur gros de la Poupée}}}{\lemma{\textnormal{\emph{Geſchichte}}}\Cendnote{\textnormal{\textcolor{blue}{Lucien Muhlfeld}: \emph{\textcolor{green}{Pour le Cœur gros de la Poupée}}. In: \emph{\textcolor{green}{La revue blanche}}, Jg. 9,
                        Nr. 50, 1. 7. 1895, S. 14–18.}}}\label{K_L02741-7h} von \textsc{\textcolor{blue}{Muhlfeld}{}\ledrightnote{\textcolor{blue}{Lucien Muhlfeld}}}.\pend
           \pstart
           Ob ich durch \textsc{\textcolor{blue}{Becque}{}\ledrightnote{\textcolor{blue}{Henry Becque}}} etwas für Deinen Verlag durchſetzen werde, weiß ich nicht. Er iſt ſo ſehr mit
               ſich beſchäftigt, daß es ſchwer iſt, ihn für einen Anderen dauernd zu
               intereſſiren.\pend
           \pstart
           Daß dein \textcolor{blue}{Bruder}{}\ledrightnote{→\textcolor{blue}{Julius Schnitzler}} und Deine
                  \textcolor{blue}{Schwägerin}{}\ledrightnote{→\textcolor{blue}{Helene Schnitzler}} einen \label{K_L02741-4v}\edtext{\textcolor{blue}{Sohn}{}\ledrightnote{→\textcolor{blue}{Hans Schnitzler}}}{\lemma{\textnormal{\emph{Sohn}}}\Cendnote{\textnormal{Hans Schnitzler wurde am 11. 7. 1895 geboren.}}}\label{K_L02741-4h} haben, habe ich mit
               Freude {\pb}vernommen. Ich glaube, ſie konnten nichts
               Anderes haben als einen \textcolor{blue}{Sohn}{}\ledrightnote{→\textcolor{blue}{Hans Schnitzler}}. Der wird ein geſcheiter und lieber \textcolor{blue}{Burſch}{}\ledrightnote{→\textcolor{blue}{Hans Schnitzler}} werden. Ich möchte ihnen gern direct ſchreiben und
               gratuliren, aber ich wags nicht. Denn ich habe mich noch immer nicht für das reizende
                  \label{K_L02741-5v}\edtext{Bild}{\lemma{\textnormal{\emph{Bild}}}\Cendnote{\textnormal{siehe Paul Goldmann an Arthur Schnitzler, 5. 1. [1895]}}}\label{K_L02741-5h} bedankt, das ſie mir zu Neujahr geſchenkt. Ich
               wollte die Antwort bis zum Gegengeſchenk aufſchieben und habe bis heut nichts Paſſendes gefunden. Was müſſen die ſich von
               mir denken!\pend
           \pstart
           {\pb}Deine Frau \label{K_L02741-6v}\edtext{\textcolor{blue}{Mutter}{}\ledrightnote{→\textcolor{blue}{Louise Schnitzler}}}{\lemma{\textnormal{\emph{Mutter}}}\Cendnote{\textnormal{siehe A. S.: \emph{Tagebuch}, 18. 7. 1895}}}\label{K_L02741-6h} dürſte mit Dir ſein. Bitte empfiehl’ mich ihr recht angelegentlich.\pend
           \pstart
           Meine \textcolor{blue}{Mutter}{}\ledrightnote{→\textcolor{blue}{Clementine Goldmann}} iſt ſeit zwei
               Monaten zu Beſuch bei mir \strikeout{und}. Wir ſprechen oft von
               Dir, und ſie dankt Dir die Freundſchaft, die Du mir bezeigſt, nicht minder, wie ich
               ſelbſt. Sie iſt krank, die \textcolor{blue}{Ärmſte}{}\ledrightnote{→\textcolor{blue}{Clementine Goldmann}}, ohne es zu ahnen (\textsc{Diabetes}). Jetzt erſt,
               wo ich denken muß, ſie zu verlieren, ſehe ich, was ſie mir iſt. Die \textcolor{blue}{Einzige}{}\ledrightnote{→\textcolor{blue}{Clementine Goldmann}} auf der Welt, die mich noch \strikeout{f\textcolor{gray}{ü}r} mit den alten {\pb}Augen anſieht, für die ſich nichts geändert, für die
               ich noch der hoffnungsreiche und wohlgeſtalte Sohn bin! Und dieſe rührende,
               geräuſchloſe Liebe, die immer um Einen iſt, wie ein ſtiller Segen, und nie etwas für
               ſich verlangt! Manchmal gehen wir mitſammen über die Straße, und da denke ich, wie
                  \strikeout{trotz} ich ſie mir ſo nahe und ſo unentbehrlich
               fühle und wie trotzdem bereits in jedem von uns das Grauenhafte lebendig iſt, das uns
               auseinanderreißen wird.\pend
           \pstart
           Sie hat Dich ſchon oft grüßen laſſen, ich hab’s aber immer vergeſſen.\pend
           \pstart
           Leb' wohl, liebſter Freund! {\\[\baselineskip]}Dein \spacefill\mbox{Paul Goldmann}\pend
           \leftskip=0em{}\pstart
           \noindent{}Viele Grüße an \textsc{\textcolor{blue}{Richard}{}\ledrightnote{\textcolor{blue}{Richard Beer-Hofmann}}}!\pend
           \endnumbering\briefempfaengerindex{Schnitzler, Arthur@\textsc{Schnitzler, Arthur}!zzzGoldmann, Paul@\emph{von Paul Goldmann}!1895-07-252@{25. 7. {[}1895{]}}|)be}\mylabel{h}\begin{anhang}\end{anhang}\normalsize

\doendnotes{C}
\bigskip
\vfill

\clearpage

\footnotesize

\lohead{\textsc{register}}

% Definiere theindex-Environment komplett neu ohne reledmac
\makeatletter
\renewenvironment{theindex}{%
  \section*{\indexname}%
  \setlength{\parindent}{0pt}%
  \setlength{\parskip}{0pt plus 0.3pt}%
  \let\item\@idxitem
}{%
  \clearpage
}
\makeatother

\IfFileExists{\jobname-pw.ind}{\input{\jobname-pw.ind}}{}

\end{document}

      