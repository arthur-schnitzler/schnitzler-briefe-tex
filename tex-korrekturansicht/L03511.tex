%% latex-korrekturansicht-vorspann.tex
%% Vorspann für die Korrekturansicht.
%% Lädt die gemeinsame Datei latex-vorspann.tex mit gesetztem Schalter.

\newif\ifkorrekturansicht
\korrekturansichttrue

\input{../tex-inputs/latex-vorspann}


\renewcommand{\erwaehntePersonen}{Personen: Felix Salten}
\renewcommand{\erwaehnteOrte}{Orte: Berlin, Edmund-Weiß-Gasse 7, Marienbad, Wien, XIX., Döbling, XVIII., Währing}
\renewcommand{\erwaehnteWerke}{}
\section[ Felix Salten an Arthur Schnitzler, 5. 9. 1907]{Felix Salten an Arthur Schnitzler, 5. 9. 1907}
\nopagebreak\mylabel{v}
\rehead{ }\normalsize\beginnumbering\briefempfaengerindex{Schnitzler, Arthur@\textsc{Schnitzler, Arthur}!zzzSalten, Felix@\emph{von Felix Salten}!1907-09-052@{5. 9. 1907}|(be}
\toendnotes[C]{\smallbreak\pagebreak[2]}\Standort{CUL, Schnitzler, B 89, B 1.}
\physDesc{Postkarte, 335 Zeichen
\newline{}Handschrift: schwarze Tinte, lateinische Kurrent
\newline{}Versand: Stempel: »\nobreak{}\textcolor{gray}{1}/\textcolor{gray}{×} Wien \textcolor{gray}{8}, 5. IX. 07, 5\nobreak{}«. Stempel: »\nobreak{}\oindex{XVIII., Waehring@\textbf{XVIII., Währing}, \emph{A.ADM3}|pwk}18/1 Wien 110 \textcolor{gray}{4}, 6. IX. 07, VIII\nobreak{}«.  
\newline{}Ordnung: mit Bleistift von unbekannter Hand nummeriert: »234« }\toendnotes[C]{\smallbreak}\pstart{}{\pb}Salten\pend{}\pstart{}\textcolor{pink}{Wien}{}\ledrightnote{\textcolor{pink}{Wien}}\pend{}\pstart{}\textcolor{pink}{XIX.}{}\ledrightnote{\textcolor{pink}{XIX., Döbling}}\pend{}
{\bigskip}\pstart{}Herrn D\textsuperscript{r} Arthur Schnitzler\pend{}\pstart{}\textcolor{pink}{Wien XVIII.}{}\ledrightnote{\textcolor{pink}{XVIII., Währing}}\pend{}\pstart{}\textcolor{pink}{Spöttelgaße 7}{}\ledrightnote{\textcolor{pink}{Edmund-Weiß-Gasse 7}}\pend{}
{\bigskip}
\pstart
           \raggedleft{}{\pb}\textcolor{pink}{Wien}{}\ledrightnote{\textcolor{pink}{Wien}}, 5. IX. 07\pend
           
\pstart
           Lieber, warum hört man nichts von Ihnen? Ich fahre heute{ }Abend nach \textcolor{pink}{Marienbad}{}\ledrightnote{\textcolor{pink}{Marienbad}}, von dort nach
                  \textcolor{pink}{Berlin}{}\ledrightnote{\textcolor{pink}{Berlin}} und bin in etwa acht Tagen wieder da.
                  \label{K_L03511-1v}\edtext{Und Sie?}{\lemma{\textnormal{\emph{Und Sie?}}}\Cendnote{\textnormal{\textcolor{blue}{Schnitzler} kehrte am 13. 9. 1907 nach \textcolor{pink}{Wien} zurück. Tennis spielten sie nachweislich
                  kurz darauf, am 18. 9. 1907.}}}\label{K_L03511-1h} Man müßte doch noch einmal wieder Tennisspielen,
               ehe dieser lächerliche Sommer vollständig einwintert.\pend
           
\pstart
           herzlichst{\\[\baselineskip]}Ihr \spacefill\mbox{F.S.}\pend
           \leftskip=0em{}\endnumbering\briefempfaengerindex{Schnitzler, Arthur@\textsc{Schnitzler, Arthur}!zzzSalten, Felix@\emph{von Felix Salten}!1907-09-052@{5. 9. 1907}|)be}\mylabel{h}  \normalsize

\doendnotes{C}
\bigskip
\vfill

\clearpage

\footnotesize

\lohead{\textsc{register}}

% Definiere theindex-Environment komplett neu ohne reledmac
\makeatletter
\renewenvironment{theindex}{%
  \section*{\indexname}%
  \setlength{\parindent}{0pt}%
  \setlength{\parskip}{0pt plus 0.3pt}%
  \let\item\@idxitem
}{%
  \clearpage
}
\makeatother

\IfFileExists{\jobname-pw.ind}{\input{\jobname-pw.ind}}{}

\end{document}

      