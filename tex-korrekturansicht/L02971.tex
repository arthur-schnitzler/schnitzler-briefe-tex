%% latex-korrekturansicht-vorspann.tex
%% Vorspann für die Korrekturansicht.
%% Lädt die gemeinsame Datei latex-vorspann.tex mit gesetztem Schalter.

\newif\ifkorrekturansicht
\korrekturansichttrue

\input{../tex-inputs/latex-vorspann}


\renewcommand{\erwaehntePersonen}{Personen: Otto Eisenschitz, Edmond Huot de Goncourt, Friedrich Hebbel, Gottfried Keller, Catulle Mendès, Felix Salten, Hugo Salus, Pierre Veber}
\renewcommand{\erwaehnteInstitutionen}{Institutionen: Jung-Wiener Theater zum Lieben Augustin}
\renewcommand{\erwaehnteOrte}{Orte: Berlin, Wien}
\renewcommand{\erwaehnteWerke}{Werke: Altes Ghettoliedchen, Am Fenster, Das Pfeifchen, Die Insel. Monatsschrift mit Buchschmuck und Illustrationen, Illustrirtes Wiener Extrablatt, Klage der Magd, Schlange}
\section[ Arthur Schnitzler an Felix Salten, 6. 10. 1901]{Arthur Schnitzler an Felix Salten, 6. 10. 1901}
\nopagebreak\mylabel{v}
\rehead{ }\normalsize\beginnumbering\briefempfaengerindex{Salten, Felix@\textsc{Salten, Felix}!zzzSchnitzler, Arthur@\emph{von Arthur Schnitzler}!1901-10-061@{6. 10. 1901}|(be}
\toendnotes[C]{\smallbreak\pagebreak[2]}\Standort{Wienbibliothek im Rathaus, ZPH 1681, 2.1.516.}
\physDesc{Brief, 1 Blatt, 2 Seiten, 411 Zeichen
\newline{}Handschrift: Bleistift, deutsche Kurrent
\newline{}Ordnung: mit Bleistift von unbekannter Hand nummeriert: »22« }\toendnotes[C]{\smallbreak}
\pstart
           \raggedleft{}{\pb}6/10 901\pend
           
\pstart
           lieber, hier iſt \label{K_L02971-1v}\edtext{\textcolor{green}{Inſel}{}\ledrightnote{\textcolor{green}{Die Insel. Monatsschrift mit Buchschmuck und Illustrationen}}}{\lemma{\textnormal{\emph{Inſel}}}\Cendnote{\textnormal{vgl. Felix Salten an Arthur Schnitzler, 28. 7. 1901}}}\label{K_L02971-1h} und \label{K_L02971-2v}\edtext{\textcolor{green}{Schlange}{}\ledrightnote{\textcolor{green}{Schlange}}}{\lemma{\textnormal{\emph{Schlange}}}\Cendnote{\textnormal{Nicht identifiziert. Da im Folgenden vor
                  allem mögliche Titel für das \emph{\textcolor{brown}{Jung-Wiener Theater
                     zum Lieben Augustin}} diskutiert wurden, könnte es sich um ein Gedicht oder
                  ein Lied gehandelt haben.}}}\label{K_L02971-2h}.\pend
           
\pstart
           Könnte man nicht die Namen der \label{K_L02971-3v}\edtext{2 Einakter}{\lemma{\textnormal{\emph{2 Einakter}}}\Cendnote{\textnormal{Auch Mitte Oktober 1901 stand das Programm des
                  Eröffnungsabends des von \textcolor{blue}{Salten} gegründeten
                  Kabaretts \emph{\textcolor{brown}{Jung-Wiener Theater zum Lieben
                     Augustin}} nicht fest. Weder von \textcolor{blue}{Goncourt} noch von \textcolor{blue}{Mendès} kam ein Stück zur Aufführung. Am 27. 10. 1901 meldete das \emph{\textcolor{green}{Illustrirte
                     Wiener Extrablatt}}, das \textcolor{brown}{Theater} habe die zwei Einakter \emph{\textcolor{green}{Am
                     Fenster}} und \emph{\textcolor{green}{Das Pfeifchen}} von \textcolor{blue}{Pierre Veber} erworben (vgl. Jg. 30,
                     Nr. 295, S. 5). Mit dem in der Fußnote genannten Überſetzer wäre dann
                     \textcolor{blue}{Otto Eisenschütz} gemeint.}}}\label{K_L02971-3h}
               erfahren, um ſie früher franzöſiſch zu leſen, insbeſondre \textsc{\textcolor{blue}{Goncourt}{}\ledrightnote{{$\rightarrow$}\textcolor{blue}{Edmond Huot de Goncourt}}}, womöglich auch \textsc{\textcolor{blue}{Mendès}{}\ledrightnote{\textcolor{blue}{Catulle Mendès}}}\footnote{\noindent{}Bedenken Sie die Unverläßlichkeit ja Lügenhaftigkeit des vorausſichtlichen \textcolor{blue}{Überſetzer}s!}\pend
           
\pstart
           – Ferner: an welches \textcolor{blue}{Hebbel}{}\ledrightnote{\textcolor{blue}{Friedrich Hebbel}}{ }Gedicht denken Sie? –\pend
           
\pstart
           {\pb}Haben Sie, endlich und vorletztens eine
               Abſchrift des \label{K_L02971-4v}\edtext{\textcolor{green}{Eſtherl}{}\ledrightnote{\textcolor{green}{Altes Ghettoliedchen}}}{\lemma{\textnormal{\emph{Eſtherl}}}\Cendnote{\textnormal{Das \emph{\textcolor{green}{Alte Ghettoliedchen}} von \textcolor{blue}{Hugo Salus}
                  beginnt mit »Estherl, mein Schwesterl«.}}}\label{K_L02971-4h} zur Verfügung? –\pend
           
\pstart
           – Letztens hab ich den Titel des \textcolor{blue}{Keller}{}\ledrightnote{\textcolor{blue}{Gottfried Keller}}ſchen
               Gedichtes ſchon wieder vergeſſen. »\textcolor{green}{Die
               Magd}{}\ledrightnote{\textcolor{green}{Klage der Magd}}?«\pend
           
\pstart
           Gute \label{K_L02971-5v}\edtext{Reiſe!}{\lemma{\textnormal{\emph{Reiſe!}}}\Cendnote{\textnormal{nach \textcolor{pink}{Berlin}, vgl. Felix Salten an Arthur Schnitzler, 9. 10. 1901}}}\label{K_L02971-5h}{ }{\\[\baselineskip]}Herzlichſt Ihr {\\[\baselineskip]}\spacefill\mbox{Arthur}\pend
           \leftskip=0em{}\endnumbering\briefempfaengerindex{Salten, Felix@\textsc{Salten, Felix}!zzzSchnitzler, Arthur@\emph{von Arthur Schnitzler}!1901-10-061@{6. 10. 1901}|)be}\mylabel{h}  \normalsize

\doendnotes{C}
\bigskip
\vfill

\clearpage

\footnotesize

\lohead{\textsc{register}}

% Definiere theindex-Environment komplett neu ohne reledmac
\makeatletter
\renewenvironment{theindex}{%
  \section*{\indexname}%
  \setlength{\parindent}{0pt}%
  \setlength{\parskip}{0pt plus 0.3pt}%
  \let\item\@idxitem
}{%
  \clearpage
}
\makeatother

\IfFileExists{\jobname-pw.ind}{\input{\jobname-pw.ind}}{}

\end{document}

      