%% latex-korrekturansicht-vorspann.tex
%% Vorspann für die Korrekturansicht.
%% Lädt die gemeinsame Datei latex-vorspann.tex mit gesetztem Schalter.

\newif\ifkorrekturansicht
\korrekturansichttrue

\input{../tex-inputs/latex-vorspann}


               \section[Lou Andreas-Salomé an Arthur Schnitzler, 25. 5. 1895]{ Lou Andreas-Salomé an Arthur Schnitzler, 25. 5. 1895}\nopagebreak\mylabel{v}\rehead{ }\normalsize\beginnumbering\briefempfaengerindex{Schnitzler, Arthur@\textsc{Schnitzler, Arthur}!zzzAndreas-Salome, Lou@\emph{von Lou Andreas-Salomé}!1895-05-251@{25. 5. 1895}|(be} \toendnotes[C]{\smallbreak\pagebreak[2]} \Standort{DLA, A:Schnitzler, HS.NZ85.1.3165,14.}
\physDesc{Brief, 1 Blatt, 2 Seiten
\newline{}Handschrift: schwarze Tinte, deutsche Kurrent\newline{}Zusatz: Der Brief ist als Beilage des Briefes von \textcolor{blue}{Paul Goldmann} an Schnitzler vom 2\textcolor{gray}{1}. 6. 1895 überliefert. }\pstart
           {\pb}25 Mai 1895.\pend
           \pstart{}Lieber Herr D\textsuperscript{r},\pend\pstart
           ich \uline{danke} Ihnen für \textcolor{pink}{Wien}{}\ledrightnote{\textcolor{pink}{Wien}}. Ich denke mit Freude und Sehnſucht dorthin zurück und bin mir
                    bewußt daß Sie es ſind, der das Schönſte das dieſe ſchöne Zeit für mich beſaß,
                    geſchenkt hat. Wie gut begreife ich es jetzt, daß Sie ſich \uline{nur dort} heimiſch fühlen konnten, wie tief und deutlich empfand
                    ich es aber auch daß Sie im Grunde \textcolor{pink}{Wien}{}\ledrightnote{\textcolor{pink}{Wien}} niemals
                    verlaſſen haben noch auch verlaſſen werden, ſondern dort mitten {\pb}unter Ihren Freunden ſtehen,
                    die Ihnen immer und auf das Innigſte nah ſind.\pend
           \pstart
           Ihre Ihnen dankbare{\\[\baselineskip]}\spacefill\mbox{Lou Andreas-Salomé.}\pend
           \leftskip=0em{}\endnumbering\briefempfaengerindex{Schnitzler, Arthur@\textsc{Schnitzler, Arthur}!zzzAndreas-Salome, Lou@\emph{von Lou Andreas-Salomé}!1895-05-251@{25. 5. 1895}|)be}\mylabel{h}  \normalsize

\doendnotes{C}
\bigskip
\vfill

\clearpage

\footnotesize

\lohead{\textsc{register}}

% Definiere theindex-Environment komplett neu ohne reledmac
\makeatletter
\renewenvironment{theindex}{%
  \section*{\indexname}%
  \setlength{\parindent}{0pt}%
  \setlength{\parskip}{0pt plus 0.3pt}%
  \let\item\@idxitem
}{%
  \clearpage
}
\makeatother

\IfFileExists{\jobname-pw.ind}{\input{\jobname-pw.ind}}{}

\end{document}

      