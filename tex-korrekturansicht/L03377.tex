%% latex-korrekturansicht-vorspann.tex
%% Vorspann für die Korrekturansicht.
%% Lädt die gemeinsame Datei latex-vorspann.tex mit gesetztem Schalter.

\newif\ifkorrekturansicht
\korrekturansichttrue

\input{../tex-inputs/latex-vorspann}


\renewcommand{\erwaehntePersonen}{Personen: Richard Beer-Hofmann, Rudyard Kipling, Bertold Löffler, Friedrich Wilhelm Riemer, Theodore Rottenberg, Olga Schnitzler}
\renewcommand{\erwaehnteInstitutionen}{Institutionen: Wiener Verlag}
\renewcommand{\erwaehnteOrte}{Orte: Berlin, Dessauer Straße, Eppan an der Weinstraße, Frankfurt am Main, Leipzig, Südtirol, Welsberg-Taisten, Wien}
\renewcommand{\erwaehnteWerke}{Werke: Bibliothek berühmter Autoren, Das Mädchen aus Birma, Das Mädchen aus Birma und andere Geschichten}
\section[ Paul Goldmann an Arthur Schnitzler, 19. 7. {[}1903{]}]{Paul Goldmann an Arthur Schnitzler, 19. 7. {[}1903{]}}
\nopagebreak\mylabel{v}
\rehead{ }\normalsize\beginnumbering\briefempfaengerindex{Schnitzler, Arthur@\textsc{Schnitzler, Arthur}!zzzGoldmann, Paul@\emph{von Paul Goldmann}!1903-07-192@{19. 7. {[}1903{]}}|(be}
\toendnotes[C]{\smallbreak\pagebreak[2]}\Standort{DLA, A:Schnitzler, HS.NZ85.1.3173.}
\physDesc{Brief, 1 Blatt, 3 Seiten
\newline{}Handschrift: blaue Tinte, deutsche Kurrent
\newline{}Schnitzler: 1) mit Bleistift das Jahr »{[}1{]}903« vermerkt  2) mit rotem Buntstift eine Unterstreichung}\toendnotes[C]{\smallbreak}
\pstart
           \noindent{}\raggedleft{}{\pb}\textcolor{gray}{\textbf{\textcolor{pink}{DESSAUERSTRASSE 19}{}\ledrightnote{\textcolor{pink}{Dessauer Straße}}}}\pend
           
\pstart
           \textcolor{pink}{Berlin}{}\ledrightnote{\textcolor{pink}{Berlin}}, 19. Juli.\pend
           
\pstart\center{}Mein lieber Freund,\pend
\pstart
           Ich war in \textcolor{pink}{Frankfurt}{}\ledrightnote{\textcolor{pink}{Frankfurt am Main}}, ich habe \label{K_L03377-1v}\edtext{\textcolor{blue}{ſie}{}\ledrightnote{{$\rightarrow$}\textcolor{blue}{Theodore Rottenberg}}}{\lemma{\textnormal{\emph{ſie}}}\Cendnote{\textnormal{\textcolor{blue}{Theodore Rottenberg}, die das seit 1899 andauernde Verhältnis mit \textcolor{blue}{Goldmann} Anfang 1903 beendet
                  hatte (vgl. Paul Goldmann an Arthur Schnitzler, 3. 1. [1903])}}}\label{K_L03377-1h}
               wiedergeſehen, und ich weiß jetzt: daß dieſe \textcolor{blue}{Frau}{}\ledrightnote{{$\rightarrow$}\textcolor{blue}{Theodore Rottenberg}} (trotz Allem) rein und wahr und ein Engel von Güte iſt.
               Ich war Jahre lang ein blinder Thor und ich habe mein Glück mit Füßen von mir
               geſtoßen. Sie liebt mich nicht mehr, weil die Verachtung die Liebe in ihr ertödtet
               hat. Aber ſie hat den Wunſch, mich wieder lieben zu können. Wenn ich in \textcolor{pink}{Frankfurt}{}\ledrightnote{\textcolor{pink}{Frankfurt am Main}} lebte, könnte ich ſie vielleicht
               wiedergewinnen. Die Entfernung verurtheilt mich zur Ohnmacht. Aber ich habe \strikeout{ich} ihr geſagt, daß mein Leben jetzt ihr gehört; und
               ſie hat dieſe Gabe angenommen, ohne ſich einſtweilen jedoch ihrerſeits zu {\pb}binden. Das Alles kann ich Dir nur mündlich
               erklären. Zum Schreiben fehlt mir die Zeit und die Kunft.\pend
           
\pstart
           Meine Sommerpläne hängen von \textcolor{blue}{ihr}{}\ledrightnote{{$\rightarrow$}\textcolor{blue}{Theodore Rottenberg}} ab. Es iſt nämlich eine, allerdings ſehr ſchwache Möglichkeit, daß ſie
                  \label{K_L03377-2v}\edtext{mit mir auf 14 Tage nach \textcolor{pink}{Südtirol}{}\ledrightnote{\textcolor{pink}{Südtirol}}}{\lemma{\textnormal{\emph{mit … Südtirol}}}\Cendnote{\textnormal{\textcolor{blue}{Rottenberg} kam mit, vgl. Paul Goldmann an Arthur Schnitzler, 27. 6. [1903]}}}\label{K_L03377-2h} kommt. Weißt Du einen ſchönen, kühlen, \uline{billigen} Ort, abſeits von der Touriſten-Heerſtraße? \textsc{\textcolor{pink}{Welsberg}{}\ledrightnote{\textcolor{pink}{Welsberg-Taisten}}} iſt ausgeſchloſſen, weil dort \label{K_L03377-3v}\edtext{\textcolor{pink}{Belin}{}\ledrightnote{\textcolor{pink}{Berlin}}er Bekannte}{\lemma{\textnormal{\emph{Beliner Bekannte}}}\Cendnote{\textnormal{Vermutlich wollte \textcolor{blue}{Goldmann} keine Bekannten in Begleitung von \textcolor{blue}{Rottenberg} treffen, da sie verheiratet war.}}}\label{K_L03377-3h} von mir
               ſind. Wenn die Reiſe zuſtandekommt, wirſt Du, wie ich hoffe, es einrichten können,
               mit uns \label{K_L03377-4v}\edtext{zuſammenzutreffen}{\lemma{\textnormal{\emph{zuſammenzutreffen}}}\Cendnote{\textnormal{siehe Paul Goldmann an Arthur Schnitzler, 27. 6. [1903]}}}\label{K_L03377-4h}. Aber, wie geſagt, das liegt Alles noch ſehr im Nebel.\pend
           
\pstart
           {\pb}Jedenfalls gib’ mir einen Rath, wo man ſich
               wiedertreffen könnte. Iſt \textsc{\textcolor{pink}{Eppan}{}\ledrightnote{\textcolor{pink}{Eppan an der Weinstraße}}} ſchön, wo \label{K_L03377-6v}\edtext{\textsc{\textcolor{blue}{Richard}{}\ledrightnote{\textcolor{blue}{Richard Beer-Hofmann}}}}{\lemma{\textnormal{\emph{Richard}}}\Cendnote{\textnormal{vermutlich Bezug auf \textcolor{blue}{Beer-Hofmann}s \textcolor{pink}{Eppan}-Aufenthalt im Herbst 1899}}}\label{K_L03377-6h} war?\pend
           
\pstart
           Grüße mir \textsc{\textcolor{blue}{Olga}{}\ledrightnote{\textcolor{blue}{Olga Schnitzler}}} (seid \strikeout{\textcolor{gray}{×}\-\textcolor{gray}{×}} Ihr nun \label{K_L03377-12v}\edtext{verheirathet}{\lemma{\textnormal{\emph{verheirathet}}}\Cendnote{\textnormal{Sie heirateten am 26. 8. 1903.}}}\label{K_L03377-12h}
               oder nicht?) und ſei ſelbſt tauſendmal gegrüßt von {\\[\baselineskip]}Deinem getreuen {\\[\baselineskip]}\spacefill\mbox{Paul Goldmann}\pend
           \leftskip=0em{}
\pstart
           \noindent{}Dank für \label{K_L03377-32v}\edtext{\textsc{\textcolor{blue}{Riemer}{}\ledrightnote{\textcolor{blue}{Friedrich Wilhelm Riemer}}}}{\lemma{\textnormal{\emph{Riemer}}}\Cendnote{\textnormal{Werk nicht ermittelt}}}\label{K_L03377-32h}!\pend
           
\pstart
           Lies: \label{K_L03377-44v}\edtext{\textsc{\textcolor{blue}{Kipling}{}\ledrightnote{\textcolor{blue}{Rudyard Kipling}}}, \textcolor{green}{Das Mädchen von \textsc{Birma}}{}\ledrightnote{\textcolor{green}{Das Mädchen aus Birma}}}{\lemma{\textnormal{\emph{Kipling, … Birma}}}\Cendnote{\textnormal{\emph{\textcolor{green}{Das Mädchen aus Birma}} ist enthalten in: \textcolor{blue}{Rudyard Kipling}: \emph{\textcolor{green}{Das Mädchen aus Birma und andere Geschichten.
                           Autorisierte Übersetzung aus dem Englischen}}. Umschlag von \textcolor{blue}{Berthold Löffler}. \textcolor{pink}{Wien}/\textcolor{pink}{Leipzig}: \emph{\textcolor{brown}{Wiener Verlag}}{ }1903. (\emph{\textcolor{green}{Bibliothek
                           berühmter Autoren}} 8) Eine Lektüre durch \textcolor{blue}{Schnitzler} ist nicht bekannt.}}}\label{K_L03377-44h}.\pend
           \endnumbering\briefempfaengerindex{Schnitzler, Arthur@\textsc{Schnitzler, Arthur}!zzzGoldmann, Paul@\emph{von Paul Goldmann}!1903-07-192@{19. 7. {[}1903{]}}|)be}\mylabel{h}
\begin{anhang}
\end{anhang}\normalsize

\doendnotes{C}
\bigskip
\vfill

\clearpage

\footnotesize

\lohead{\textsc{register}}

% Definiere theindex-Environment komplett neu ohne reledmac
\makeatletter
\renewenvironment{theindex}{%
  \section*{\indexname}%
  \setlength{\parindent}{0pt}%
  \setlength{\parskip}{0pt plus 0.3pt}%
  \let\item\@idxitem
}{%
  \clearpage
}
\makeatother

\IfFileExists{\jobname-pw.ind}{\input{\jobname-pw.ind}}{}

\end{document}

      