%% latex-korrekturansicht-vorspann.tex
%% Vorspann für die Korrekturansicht.
%% Lädt die gemeinsame Datei latex-vorspann.tex mit gesetztem Schalter.

\newif\ifkorrekturansicht
\korrekturansichttrue

\input{../tex-inputs/latex-vorspann}


\renewcommand{\erwaehntePersonen}{Personen: Caroline Kotter, Elisabeth Kotter, Paul Salten, Heinrich Schnitzler}
\renewcommand{\erwaehnteOrte}{Orte: Schlosspark Schönbrunn, Wien, XIII., Hietzing}
\renewcommand{\erwaehnteWerke}{}
\section[ Felix Salten an Arthur Schnitzler, 14. {[}10. 1903{]}]{Felix Salten an Arthur Schnitzler, 14. {[}10. 1903{]}}
\nopagebreak\mylabel{v}
\rehead{ }\normalsize\beginnumbering\briefempfaengerindex{Schnitzler, Arthur@\textsc{Schnitzler, Arthur}!zzzSalten, Felix@\emph{von Felix Salten}!1903-10-141@{14. {[}10. 1903{]}}|(be}
\toendnotes[C]{\smallbreak\pagebreak[2]}\Standort{CUL, Schnitzler, B 89, A 2.}
\physDesc{Karte, 541 Zeichen
\newline{}Handschrift: Bleistift, lateinische Kurrent
\newline{}Schnitzler: mit Bleistift die Monatsangabe verdeutlicht und die Jahreszahl ergänzt: »X 90\textcolor{gray}{3}« 
\newline{}Ordnung: mit Bleistift von unbekannter Hand nummeriert: »{\pb}173« }\toendnotes[C]{\smallbreak}
\pstart
           \raggedleft{}14. \textcolor{gray}{X}.\pend
           
\pstart
           {\pb}Lieber, ich muß leider auch für Freitag absagen. Ich bin diese Woche zu sehr in Anspruch genommen. Aber
                  \label{K_L03358-1v}\edtext{Mittwoch}{\lemma{\textnormal{\emph{Mittwoch}}}\Cendnote{\textnormal{siehe A. S.: \emph{Tagebuch}, 21. 10. 1903}}}\label{K_L03358-1h} ganz \uline{bestimmt}. Hoffentlich passt Ihnen dieser
               Tag. Wenn \label{K_L03358-2v}\edtext{Sonntag}{\lemma{\textnormal{\emph{Sonntag}}}\Cendnote{\textnormal{siehe A. S.: \emph{Tagebuch}, 18. 10. 1903}}}\label{K_L03358-2h} schönes Wetter ist, fahren wir Vormittag schon irgendwo hinaus,
               um im Freien zu essen. Am liebsten nach \textcolor{pink}{Hietzing}{}\ledrightnote{\textcolor{pink}{XIII., Hietzing}},
               weil ich meinem \label{K_L03358-3v}\edtext{\textcolor{blue}{Mäderl}{}\ledrightnote{{$\rightarrow$}\textcolor{blue}{Caroline Kotter}}}{\lemma{\textnormal{\emph{Mäderl}}}\Cendnote{\textnormal{\textcolor{blue}{Caroline Kotter}, \textcolor{blue}{Salten}s Tochter mit \textcolor{blue}{Elisabeth Kotter}, die er kürzlich bei sich aufgenommen hatte}}}\label{K_L03358-3h}{ }\textcolor{pink}{Schönbrunn}{}\ledrightnote{\textcolor{pink}{Schlosspark Schönbrunn}} zeigen möchte. Wir würden uns sehr
               freuen, wenn Sie mir uns beisammen sein könnten.\pend
           \pstart herzlichst Ihr \spacefill\mbox{S.}\pend{}
\pstart
           \noindent{}Wir nehmen \textcolor{gray}{auch} den \textcolor{blue}{Paul}{}\ledrightnote{\textcolor{blue}{Paul Salten}}
                  mit, und hätten mit \textcolor{blue}{Heinrich}{}\ledrightnote{\textcolor{blue}{Heinrich Schnitzler}} eine Freude.
                  Wagen? Die Omnibus C\textsuperscript{o} stellt vis a vis Wagen.
                     Gummi{[},{]} sehr billig!\pend
           \endnumbering\briefempfaengerindex{Schnitzler, Arthur@\textsc{Schnitzler, Arthur}!zzzSalten, Felix@\emph{von Felix Salten}!1903-10-141@{14. {[}10. 1903{]}}|)be}\mylabel{h}  \normalsize

\doendnotes{C}
\bigskip
\vfill

\clearpage

\footnotesize

\lohead{\textsc{register}}

% Definiere theindex-Environment komplett neu ohne reledmac
\makeatletter
\renewenvironment{theindex}{%
  \section*{\indexname}%
  \setlength{\parindent}{0pt}%
  \setlength{\parskip}{0pt plus 0.3pt}%
  \let\item\@idxitem
}{%
  \clearpage
}
\makeatother

\IfFileExists{\jobname-pw.ind}{\input{\jobname-pw.ind}}{}

\end{document}

      