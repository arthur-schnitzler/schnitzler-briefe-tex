%% latex-korrekturansicht-vorspann.tex
%% Vorspann für die Korrekturansicht.
%% Lädt die gemeinsame Datei latex-vorspann.tex mit gesetztem Schalter.

\newif\ifkorrekturansicht
\korrekturansichttrue

\input{../tex-inputs/latex-vorspann}


\renewcommand{\erwaehntePersonen}{Personen: Hugo von Hofmannsthal, Felix Salten}
\renewcommand{\erwaehnteOrte}{Orte: Kaltenleutgeben, Salzburg, Wien}
\renewcommand{\erwaehnteWerke}{Werke: Der Schleier der Beatrice. Schauspiel in fünf Akten, Erklärung [Schleier der Beatrice]}
\section[ Arthur Schnitzler an Felix Salten, {[}26. 6. 1902{]}]{Arthur Schnitzler an Felix Salten, {[}26. 6. 1902{]}}
\nopagebreak\mylabel{v}
\rehead{ }\normalsize\beginnumbering\briefempfaengerindex{Salten, Felix@\textsc{Salten, Felix}!zzzSchnitzler, Arthur@\emph{von Arthur Schnitzler}!1902-06-262@{{[}26. 6. 1902{]}}|(be}
\toendnotes[C]{\smallbreak\pagebreak[2]}\Standort{Wienbibliothek im Rathaus, ZPH 1681, 2.1.516.}
\physDesc{Brief, 1 Blatt, 2 Seiten, 420 Zeichen
\newline{}Handschrift: Bleistift, deutsche Kurrent
\newline{}Ordnung: 1) mit Bleistift von unbekannter Hand Nummerierung der Dppelseiten des Konvoluts:
                                    »8«–»9«  2) mit Bleistift datiert: »[26. 06. 1902]«}\toendnotes[C]{\smallbreak}
\pstart
           \noindent{}{\pb}lieber Freund, wieder hat mich geſtern – ſchon auf dem Weg, das gräßliche Wetter abgehalten Sie in \textsc{\textcolor{pink}{Kaltenl.}{}\ledrightnote{\textcolor{pink}{Kaltenleutgeben}}} zu beſuchen. Nun ſeh ich Sie wohl erſt, nach meiner Rückkehr, etwa gegen den
                  10. Juli. Ich fahre {\pb}\label{K_L02976-1v}\edtext{morgen}{\lemma{\textnormal{\emph{morgen}}}\Cendnote{\textnormal{Das erlaubt die Datierung des
                  undatierten Korrespondenzstücks, vgl. A. S.: \emph{Tagebuch}, 27. 6. 1902. \textcolor{blue}{Schnitzler} kehrte am 8. 7. 1902 nach \textcolor{pink}{Wien} zurück und
                  sah \textcolor{blue}{Salten} nachweislich am 8. 7. 1902
                  wieder.}}}\label{K_L02976-1h}{ }\textcolor{pink}{Salzburg}{}\ledrightnote{\textcolor{pink}{Salzburg}}, \textcolor{blue}{Hugo}{}\ledrightnote{\textcolor{blue}{Hugo von Hofmannsthal}} dürfte übermorgen nachko{\geminationm}en.– Briefe werden mir aus \textcolor{pink}{Wien}{}\ledrightnote{\textcolor{pink}{Wien}} nachgeſchickt. Die \label{K_L02976-2v}\edtext{\textcolor{green}{\textcolor{green}{\textsc{Bea.}}{}\ledrightnote{\textcolor{green}{Der Schleier der Beatrice. Schauspiel in fünf Akten}}-Sache}{}\ledrightnote{{$\rightarrow$}\textcolor{green}{Erklärung [Schleier der Beatrice]}}}{\lemma{\textnormal{\emph{Bea.-Sache}}}\Cendnote{\textnormal{siehe A. S.: \emph{Tagebuch}, 17. 7. 1902}}}\label{K_L02976-2h} ka{\geminationn} ich wohl nach meiner Rückkehr noch ſehen,
               nicht wahr? Wie lange denken Sie in \textcolor{pink}{K.}{}\ledrightnote{\textcolor{pink}{Kaltenleutgeben}}{ }{\pb}zu
               bleiben?\pend
           
\pstart
           Ich grüße Sie herzlich {\\[\baselineskip]}Ihr {\\[\baselineskip]}\spacefill\mbox{A.}\pend
           \leftskip=0em{}\endnumbering\briefempfaengerindex{Salten, Felix@\textsc{Salten, Felix}!zzzSchnitzler, Arthur@\emph{von Arthur Schnitzler}!1902-06-262@{{[}26. 6. 1902{]}}|)be}\mylabel{h}  \normalsize

\doendnotes{C}
\bigskip
\vfill

\clearpage

\footnotesize

\lohead{\textsc{register}}

% Definiere theindex-Environment komplett neu ohne reledmac
\makeatletter
\renewenvironment{theindex}{%
  \section*{\indexname}%
  \setlength{\parindent}{0pt}%
  \setlength{\parskip}{0pt plus 0.3pt}%
  \let\item\@idxitem
}{%
  \clearpage
}
\makeatother

\IfFileExists{\jobname-pw.ind}{\input{\jobname-pw.ind}}{}

\end{document}

      