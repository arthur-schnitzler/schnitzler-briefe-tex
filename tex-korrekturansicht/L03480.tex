%% latex-korrekturansicht-vorspann.tex
%% Vorspann für die Korrekturansicht.
%% Lädt die gemeinsame Datei latex-vorspann.tex mit gesetztem Schalter.

\newif\ifkorrekturansicht
\korrekturansichttrue

\input{../tex-inputs/latex-vorspann}


\renewcommand{\erwaehntePersonen}{Personen: Marie Glümer}
\renewcommand{\erwaehnteOrte}{Orte: Berlin, München, Salzburg, Wien}
\renewcommand{\erwaehnteWerke}{}
\section[ Paul Goldmann an Arthur Schnitzler, 20. 11. 1925]{Paul Goldmann an Arthur Schnitzler, 20. 11. 1925}
\nopagebreak\mylabel{v}
\rehead{ }\normalsize\beginnumbering\briefempfaengerindex{Schnitzler, Arthur@\textsc{Schnitzler, Arthur}!zzzGoldmann, Paul@\emph{von Paul Goldmann}!1925-11-201@{20. 11. 1925}|(be}
\toendnotes[C]{\smallbreak\pagebreak[2]}\Standort{DLA, A:Schnitzler, HS.NZ85.1.3176.}
\physDesc{Brief, 1 Blatt, 2 Seiten, 796 Zeichen
\newline{}Handschrift: lila Tinte, deutsche Kurrent
\newline{}Schnitzler: 1) mit Bleistift Vermerk »\textcolor{blue}{Goldm{[}ann{]}}«  2) mit rotem Buntstift zwei Unterstreichungen}\toendnotes[C]{\smallbreak}
\pstart
           {\pb}\textcolor{pink}{Berlin}{}\ledrightnote{\textcolor{pink}{Berlin}}, 20. 11. 25.\pend
           
\pstart{}Lieber Freund,\pend
\pstart
           Mit großer Bewegung leſe ich ſoeben ein \label{K_L03480-1v}\edtext{\textcolor{pink}{München}{}\ledrightnote{\textcolor{pink}{München}}er Telegramm}{\lemma{\textnormal{\emph{Münchener Telegramm}}}\Cendnote{\textnormal{Gemeint ist eine in einer Zeitung abgedruckte Kurzmeldung,
                  die telegrafisch übermittelt wurde.}}}\label{K_L03480-1h}, das den \label{K_L03480-2v}\edtext{Tod von \textsc{\textcolor{blue}{Marie Glümer}{}\ledrightnote{\textcolor{blue}{Marie Glümer}}}}{\lemma{\textnormal{\emph{Tod von Marie Glümer}}}\Cendnote{\textnormal{\textcolor{blue}{Marie Glümer}, \textcolor{blue}{Schnitzler}s wichtigste Partnerin in der Zeit der ersten
                  Bekanntschaft mit \textcolor{blue}{Goldmann}, war am 16. 11. 1925 verstorben. Siehe A. S.: \emph{Tagebuch}, 17. 11. 1925.}}}\label{K_L03480-2h} meldet.
               Alte Zeiten werden wieder lebendig, Bilder aus ferner Vergangenheit ſteigen auf. Ich
               ſehe das junge Mädchen, das die \textcolor{blue}{Verſtorbene}{}\ledrightnote{{$\rightarrow$}\textcolor{blue}{Marie Glümer}} einſt war, ſehe Dich, ihren Freund, den jungen Arzt u. Dichter,
               ſehe mich im Beiſammenſein mit euch \textcolor{blue}{Beiden}{}\ledrightnote{{$\rightarrow$}\textcolor{blue}{Marie Glümer}}. Geſpräche, die ich damals mit Dir geführt, wachen
               wieder auf, – ich erinnere mich an \textcolor{pink}{Wien}{}\ledrightnote{\textcolor{pink}{Wien}}, an
                  \label{K_L03480-3v}\edtext{\textcolor{pink}{Salzburg}{}\ledrightnote{\textcolor{pink}{Salzburg}}}{\lemma{\textnormal{\emph{Salzburg}}}\Cendnote{\textnormal{vgl. Paul Goldmann an Arthur Schnitzler, 1. 10. 1890}}}\label{K_L03480-3h}.\pend
           
\pstart
           Die \textcolor{blue}{Frau}{}\ledrightnote{{$\rightarrow$}\textcolor{blue}{Marie Glümer}}, die dahingegangen
               iſt, war längſt aus Deinem Leben ausgeſchieden. Aber ſie hat Dir einſt viel bedeutet.
                  \strikeout{Ich habe an jenem Teile Deines Lebens
                     \textcolor{gray}{te}} Ich habe an alledem {\pb}teilgenommen u. will Dir nur ſagen, daß ich
               deſſen eingedenk bin u. daß mich der Tod Deiner einſtigen \textcolor{blue}{Freundin}{}\ledrightnote{{$\rightarrow$}\textcolor{blue}{Marie Glümer}} ſehr ergriffen hat.\pend
           
\pstart
           Mit herzlichem Gruß {\\[\baselineskip]}Dein {\\[\baselineskip]}\spacefill\mbox{Paul Goldmann.}\pend
           \leftskip=0em{}\endnumbering\briefempfaengerindex{Schnitzler, Arthur@\textsc{Schnitzler, Arthur}!zzzGoldmann, Paul@\emph{von Paul Goldmann}!1925-11-201@{20. 11. 1925}|)be}\mylabel{h}  \normalsize

\doendnotes{C}
\bigskip
\vfill

\clearpage

\footnotesize

\lohead{\textsc{register}}

% Definiere theindex-Environment komplett neu ohne reledmac
\makeatletter
\renewenvironment{theindex}{%
  \section*{\indexname}%
  \setlength{\parindent}{0pt}%
  \setlength{\parskip}{0pt plus 0.3pt}%
  \let\item\@idxitem
}{%
  \clearpage
}
\makeatother

\IfFileExists{\jobname-pw.ind}{\input{\jobname-pw.ind}}{}

\end{document}

      