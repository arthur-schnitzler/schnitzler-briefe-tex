%% latex-korrekturansicht-vorspann.tex
%% Vorspann für die Korrekturansicht.
%% Lädt die gemeinsame Datei latex-vorspann.tex mit gesetztem Schalter.

\newif\ifkorrekturansicht
\korrekturansichttrue

\input{../tex-inputs/latex-vorspann}


               \section[Arthur Schnitzler an Hermann Bahr, 10. –12. 9. 1901]{ Arthur Schnitzler an Hermann Bahr, 10. –12. 9. 1901}\nopagebreak\mylabel{v}\rehead{ }\normalsize\beginnumbering\briefempfaengerindex{Bahr, Hermann@\textsc{Bahr, Hermann}!zzzSchnitzler, Arthur@\emph{von Arthur Schnitzler}!1901-09-121@{10. –12. 9. 1901}|(be} \toendnotes[C]{\smallbreak\pagebreak[2]} \Standort{TMW, HS AM 23343 Ba.}
\physDesc{Brief, 1 Blatt, 4 Seiten
\newline{}Handschrift: schwarze Tinte, deutsche Kurrent\newline{}Ordnung: 1) Lochung 2) mit Bleistift von unbekannter Hand (falsch) datiert):
                                    »16. 5. 01«}\buchAbdrucke{\weitereDrucke{1) \emph{10., 12. 9. 1901.} In: Arthur Schnitzler: \emph{The Letters of Arthur Schnitzler to Hermann Bahr}. Edited, annotated, and with an introduction, by Donald G.
                        Daviau. Chapel Hill: \emph{The University of North Carolina Press} 1978, S. 70 (University of North Carolina studies in the Germanic languages
                        and literatures, 89).} \weitereDrucke{2) Hermann Bahr, Arthur Schnitzler: \emph{Briefwechsel, Aufzeichnungen, Dokumente (1891–1931)}. Hg. Kurt Ifkovits und Martin Anton Müller. Göttingen: \emph{Wallstein} 2018, S. 213–214.} }\toendnotes[C]{\smallbreak}\pstart
           \noindent{}{\pb}mein lieber
                  Hermann, ich ſchicke dir \introOben{}heute\introOben{} die \textcolor{green}{3 Einakter}{}\ledrightnote{→\textcolor{green}{Die Frau mit dem Dolche}{\newline}→\textcolor{green}{Literatur}{\newline}→\textcolor{green}{Lebendige Stunden}}. Mein
               Bedenken, die \label{K_L01171_1v}\edtext{Kürze des Abends
                  betreffend}{\lemma{\textnormal{\emph{Kürze … betreffend}}}\Cendnote{\textnormal{vgl. A. S.: \emph{Tagebuch}, 6. 9. 1901}}}\label{K_L01171_1h}, iſt wieder
               rege geworden; und ich habe die Abſicht, einen \textcolor{green}{vierten Einakter}{}\ledrightnote{→\textcolor{green}{Der Puppenspieler}}, der mir geſtern einfiel und in Sinn und
               Form zu den bis jetzt vorliegenden paſſt, zu ſchreiben. Ob ich gleich die rechte
                  Sti{\geminationm}ung dafür finden werde, iſt natürlich noch nicht
               ausgemacht. Jedenfalls bitt’ {\pb}ich dich, vor allem
               einmal dieſe 3 \textcolor{green}{Stücke}{}\ledrightnote{→\textcolor{green}{Die Frau mit dem Dolche}{\newline}→\textcolor{green}{Literatur}{\newline}→\textcolor{green}{Lebendige Stunden}} zu leſen, u. zw. in der Reihenfolge »\introOben{}1)\introOben{}{ }\textcolor{green}{Die Frau mit dem Dolch}{}\ledrightnote{\textcolor{green}{Die Frau mit dem Dolche}}«. 2) \textcolor{green}{Lebendige Stunden}{}\ledrightnote{\textcolor{green}{Lebendige Stunden}}\damage{.} 3.) \textcolor{green}{Literatur}{}\ledrightnote{\textcolor{green}{Literatur}}. Es wäre ſchade, wenn der
               Abend an einem ſo äußerlichen Moment, wie dem der Kürze ſcheitern ſollte. Allerdings
               glaube ich, dſs dieſes Bedenken weniger für \textcolor{pink}{Wien}{}\ledrightnote{\textcolor{pink}{Wien}} als
               für \textcolor{pink}{Berlin}{}\ledrightnote{\textcolor{pink}{Berlin}} in Betracht käme.\pend
           \pstart
           Wenns dir recht iſt, ko{\geminationm} ich wieder {\pb}einmal in den
               Vormittagſtunden zu dir hinaus, ſobald du die Sachen geleſen haſt; es eilt \uline{durchaus nicht}.\pend
           \pstart
           herzlich grüßt dich{\\[\baselineskip]}dein \spacefill\mbox{Arthur}\pend
           \leftskip=0em{}\pstart
           \textcolor{pink}{Wien}{}\ledrightnote{\textcolor{pink}{Wien}}{ }10. 9. 901\pend
           \pstart
           \noindent{}Der Zufall fügte es, daſs ich, durch ein teleph. Erſuchen \textcolor{blue}{Kadelburg}{}\ledrightnote{\textcolor{blue}{Heinrich Kadelburg}}s veranlaſſt, die \textcolor{green}{Stücke}{}\ledrightnote{→\textcolor{green}{Die Frau mit dem Dolche}{\newline}→\textcolor{green}{Literatur}{\newline}→\textcolor{green}{Lebendige Stunden}} in der Direktion
               überreichte. Ich bat, daſs man ſie {\pb}dir zukommen ließe, was
               wohl bereits geſchehen iſt\pend
           \pstart
           Indeſs hab ich den \textcolor{green}{vierten
                  Einakter}{}\ledrightnote{→\textcolor{green}{Der Puppenspieler}} zu ſchreiben begonnen und hoffe, daſs er ſich, wie vielleicht noch
               ein \label{K_L01171_2v}\edtext{\textcolor{green}{fünfter}{}\ledrightnote{→\textcolor{green}{Die letzten Masken}}}{\lemma{\textnormal{\emph{fünfter}}}\Cendnote{\textnormal{\emph{\textcolor{green}{Die letzten Masken}}; am 6. 9. 1901{ }schreibt er an diesem und am \emph{\textcolor{green}{Puppenspieler}}. Die Unterscheidung zwischen den zwei Stoffen
                  ergibt sich aus der Formulierung »gestern einfiel« in diesem Brief,
                  da bereits im Frühjahr eine erste dramatische Fassung der \emph{\textcolor{green}{Letzten Masken}} entstanden war. (Vgl. Arthur Schnitzler an Hermann Bahr, [14. 3.? 1901].) Die Arbeit geht schnell voran, so dass am
                     22. die \emph{\textcolor{green}{Masken}} vorliegen,
                  während \emph{\textcolor{green}{Der Puppenspieler}} »noch auf ein
                     oder zwei gute Stunden zur Vollendung« wartet (\emph{Briefwechsel} Schnitzler/Brahm 95).}}}\label{K_L01171_2h} dem
               Cyklus gut einfügen wird\pend
           \pstart herzlichſt \spacefill\mbox{A.}\pend{}\pstart
           12. 9. 901. \pend
           \endnumbering\briefempfaengerindex{Bahr, Hermann@\textsc{Bahr, Hermann}!zzzSchnitzler, Arthur@\emph{von Arthur Schnitzler}!1901-09-121@{10. –12. 9. 1901}|)be}\mylabel{h}  \normalsize

\doendnotes{C}
\bigskip
\vfill

\clearpage

\footnotesize

\lohead{\textsc{register}}

% Definiere theindex-Environment komplett neu ohne reledmac
\makeatletter
\renewenvironment{theindex}{%
  \section*{\indexname}%
  \setlength{\parindent}{0pt}%
  \setlength{\parskip}{0pt plus 0.3pt}%
  \let\item\@idxitem
}{%
  \clearpage
}
\makeatother

\IfFileExists{\jobname-pw.ind}{\input{\jobname-pw.ind}}{}

\end{document}

      