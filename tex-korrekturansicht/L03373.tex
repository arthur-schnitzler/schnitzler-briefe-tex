%% latex-korrekturansicht-vorspann.tex
%% Vorspann für die Korrekturansicht.
%% Lädt die gemeinsame Datei latex-vorspann.tex mit gesetztem Schalter.

\newif\ifkorrekturansicht
\korrekturansichttrue

\input{../tex-inputs/latex-vorspann}


\renewcommand{\erwaehntePersonen}{Personen: Otto Brahm, Julius Elias, Johannes Gaulke, Felix Paul Greve, Georg Hirschfeld, Paul Jonas, Elly Petersen, Theodore Rottenberg, Olga Schnitzler, Irene Triesch, Oscar Wilde}
\renewcommand{\erwaehnteInstitutionen}{Institutionen: Deutsches Theater Berlin, J. C. C. Bruns, Verlag Max Spohr}
\renewcommand{\erwaehnteOrte}{Orte: Berlin, Dessauer Straße, Italien, Leipzig, Minden, Schauspielhaus Leipzig, Südtirol, Wien}
\renewcommand{\erwaehnteWerke}{Werke: Berliner Theater. (»Lebendige Stunden« von Arthur Schnitzler.), Der Schleier der Beatrice. Schauspiel in fünf Akten, Dorian Gray, Dorian Grays Bildnis, Fink und Fliederbusch. Komödie in drei Akten, Lebendige Stunden. Vier Einakter, Professor Bernhardi. Komödie in fünf Akten, The picture of Dorian Gray}
\section[ Paul Goldmann an Arthur Schnitzler, 22. 5. {[}1903{]}]{Paul Goldmann an Arthur Schnitzler, 22. 5. {[}1903{]}}
\nopagebreak\mylabel{v}
\rehead{ }\normalsize\beginnumbering\briefempfaengerindex{Schnitzler, Arthur@\textsc{Schnitzler, Arthur}!zzzGoldmann, Paul@\emph{von Paul Goldmann}!1903-05-221@{22. 5. {[}1903{]}}|(be}
\toendnotes[C]{\smallbreak\pagebreak[2]}\Standort{DLA, A:Schnitzler, HS.NZ85.1.3173.}
\physDesc{Brief, 1 Blatt, 4 Seiten
\newline{}Handschrift: blaue Tinte, deutsche Kurrent
\newline{}Schnitzler: 1) mit Bleistift das Jahr »{[}1{]}903« vermerkt  2) mit rotem Buntstift drei Unterstreichungen}\toendnotes[C]{\smallbreak}
\pstart
           \noindent{}\raggedleft{}{\pb}\textcolor{gray}{\textbf{\textcolor{pink}{DESSAUERSTRASSE 19}{}\ledrightnote{\textcolor{pink}{Dessauer Straße}}}}\pend
           
\pstart
           \textcolor{pink}{Berlin}{}\ledrightnote{\textcolor{pink}{Berlin}}, 2\textcolor{gray}{2}. Mai.\pend
           
\pstart{}Mein lieber Freund,\pend
\pstart
           Dein lieber Brief hat mich ſehr erfreut. Ich war über das Ausbleiben Deiner
               Nachrichten bereits in Sorge. Auch \textsc{\textcolor{blue}{Olga}{}\ledrightnote{\textcolor{blue}{Olga Schnitzler}}s} Brief war ſehr charmant; und ich
               bitte Dich (bis ich Zeit finde, ihn zu beantworten), ihr einſtweilen in meinem Namen
               zu danken.\pend
           
\pstart
           Heut nur in aller Eile: Ich war geſtern{ }Abend bei \textsc{Dr. \textcolor{blue}{Elias}{}\ledrightnote{\textcolor{blue}{Julius Elias}}}. {\pb}Sonſt anweſend \textsc{\textcolor{blue}{Brahm}{}\ledrightnote{\textcolor{blue}{Otto Brahm}}} (der mir immer ſympathiſcher wird), \textsc{\textcolor{blue}{Georg Hirschfeld}{}\ledrightnote{\textcolor{blue}{Georg Hirschfeld}}} und \textcolor{blue}{Frau}{}\ledrightnote{{$\rightarrow$}\textcolor{blue}{Elly Petersen}}, \textsc{Dr. \textcolor{blue}{Jonas}{}\ledrightnote{\textcolor{blue}{Paul Jonas}} etc}.
               Allgemeines Fragen nach Dir. Ich konnte keine Auskunft erteilen. \textsc{\textcolor{blue}{Brahm}{}\ledrightnote{\textcolor{blue}{Otto Brahm}}} ſagte: Du habeſt ihm \label{K_L03373-3v}\edtext{mitgeteilt}{\lemma{\textnormal{\emph{mitgeteilt}}}\Cendnote{\textnormal{Am 29. 4. 1903 notierte
                     \textcolor{blue}{Schnitzler} den Plan des »\emph{\textcolor{green}{Journalistenstück}}«s (\emph{\textcolor{green}{Flink
                     und Fiederbusch}}), zwei Tage später, am 1. 5. 1903 erzählte er \textcolor{blue}{Brahm} davon.}}}\label{K_L03373-3h}, es ſei Dir ein \textcolor{green}{Luſtſpiel}{}\ledrightnote{{$\rightarrow$}\textcolor{green}{Fink und Fliederbusch. Komödie in drei Akten}} eingefallen. Darüber freuten ſich
               Alle (\label{K_L03373-4v}\edtext{ich beſonders}{\lemma{\textnormal{\emph{ich beſonders}}}\Cendnote{\textnormal{\textcolor{blue}{Goldmann} hatte \textcolor{blue}{Schnitzler} bereits mehrmals dazu aufgefordert, ein
                  Lustspiel zu schreiben, siehe Paul Goldmann an Arthur Schnitzler, 2. 5. [1900]. Auch in seinem \emph{\textcolor{green}{Feuilleton}} zu \emph{\textcolor{green}{Lebendige Stunden}}
                  nannte er neben dem historischen Stück die Gattung des Lustspiels als \textcolor{blue}{Schnitzler}s eigentliche Spezialität.}}}\label{K_L03373-4h}),
               und Alle (ich beſonders) hoffen, daß Du den Plan ausführen wirſt.\pend
           
\pstart
           Wenn ich Deine und \textsc{\textcolor{blue}{Olga}{}\ledrightnote{\textcolor{blue}{Olga Schnitzler}}s}{ }{\pb}Andeutungen recht verſtehe, wollt Ihr im Herbſt
                  \label{K_L03373-5v}\edtext{heirathen}{\lemma{\textnormal{\emph{heirathen}}}\Cendnote{\textnormal{\textcolor{blue}{Schnitzler} und \textcolor{blue}{Olga Gussmann} heirateten am 26. 8. 1903.}}}\label{K_L03373-5h}. Das iſt ſehr geſcheit, und
               ich denke, die Legaliſirung des Zuſtandes wird in jeder Beziehung von ſegensreichen
               Folgen ſein.\pend
           
\pstart
           Auch von Euren \label{K_L03373-11v}\edtext{Reiſeplänen}{\lemma{\textnormal{\emph{Reiſeplänen}}}\Cendnote{\textnormal{\textcolor{blue}{Schnitzler} und \textcolor{blue}{Olga Gussmann} reisten zwischen 28. 5. 1903 und 15. 6. 1903 nach \textcolor{pink}{Italien} und \textcolor{pink}{Südtirol}.}}}\label{K_L03373-11h} habe ich mit Vergnügen vernommen; meine beſten Wünſche
               begleiten Euch nach dem ſchönen Süden.\pend
           
\pstart
           \strikeout{D\textcolor{gray}{a}} Da Du ſicherlich Luſt bekommen wirſt, \label{K_L03373-23v}\edtext{mehr von \textsc{\textcolor{blue}{Wilde}{}\ledrightnote{\textcolor{blue}{Oscar Wilde}}}}{\lemma{\textnormal{\emph{mehr von Wilde}}}\Cendnote{\textnormal{siehe Paul Goldmann an Arthur Schnitzler, 13. 5. [1903]}}}\label{K_L03373-23h} zu \strikeout{leſen,} leſen, ſo lies \label{K_L03373-32v}\edtext{»\textcolor{green}{\textsc{Dorian {\pb}Grays}
                  Bildniß}{}\ledrightnote{\textcolor{green}{Dorian Grays Bildnis}}« (in der Überſetzung von \textsc{\textcolor{blue}{Greve}{}\ledrightnote{\textcolor{blue}{Felix Paul Greve}}})}{\lemma{\textnormal{\emph{»Dorian … Greve)}}}\Cendnote{\textnormal{\textcolor{blue}{Oscar Wilde}: \emph{\textcolor{green}{Dorian Grays Bildnis}}. Übers. v. \textcolor{blue}{Felix Paul Greve}. \textcolor{pink}{Minden}: \emph{\textcolor{brown}{J. C. C. Bruns’ Verlag}}
                        [1902]. \textcolor{blue}{Schnitzler} las \emph{\textcolor{green}{The picture of Dorian
                     Gray}} (1890) am 30. 6. 1904 in der \textcolor{green}{Übersetzung} von \textcolor{blue}{Johannes Gaulke}: \textcolor{blue}{Oscar Wilde}: \emph{\textcolor{green}{Dorian Gray}}. Übers. v. \textcolor{blue}{Johannes Gaulke}. \textcolor{pink}{Leipzig}: \emph{\textcolor{brown}{Verlag von Max Spohr}}
                        [1901] (vgl. A. S.: \emph{Lektüren}, England).}}}\label{K_L03373-32h}.\pend
           
\pstart
           Ich habe nichts vergeſſen, nichts überwunden; habe nach meiner \label{K_L03373-56v}\edtext{Rückkehr aus \textcolor{pink}{Wien}{}\ledrightnote{\textcolor{pink}{Wien}}}{\lemma{\textnormal{\emph{Rückkehr aus Wien}}}\Cendnote{\textnormal{siehe Paul Goldmann an Arthur Schnitzler, 13. 5. [1903]}}}\label{K_L03373-56h} wieder eine ſchreckliche Kriſis durchgemacht; und verbringe mein Leben in
                  \label{K_L03373-77v}\edtext{Reue und Sehnſucht}{\lemma{\textnormal{\emph{Reue und Sehnſucht}}}\Cendnote{\textnormal{vermutlich Bezug auf \textcolor{blue}{Goldmann}s Liebeskummer wegen \textcolor{blue}{Theodore Rottenberg}, die ihn Anfang 1903 verlassen hatte (vgl. Paul Goldmann an Arthur Schnitzler, 3. 1. [1903])}}}\label{K_L03373-77h}, hoffnungsloſer Sehnſucht{\dotsfour}\pend
           
\pstart
           Nächſtens mehr! Viele herzliche Grüße Dir und \textsc{\textcolor{blue}{Olga}{}\ledrightnote{\textcolor{blue}{Olga Schnitzler}}}! {\\[\baselineskip]}Dein \spacefill\mbox{Paul Goldm}\pend
           \leftskip=0em{}
\pstart
           \noindent{}\label{T_L03373-1v}\edtext{{\pb}Die \textsc{\textcolor{blue}{Triesch}{}\ledrightnote{\textcolor{blue}{Irene Triesch}}}, die geſtern{ }Abend auch da war, ſagte, daß ſie nach \label{K_L03373-7v}\edtext{\textcolor{pink}{Leipzig}{}\ledrightnote{\textcolor{pink}{Leipzig}}}{\lemma{\textnormal{\emph{Leipzig}}}\Cendnote{\textnormal{Das \emph{\textcolor{brown}{Deutsche Theater Berlin}} hatte ein Gastspiel am \textcolor{pink}{Schauspielhaus Leipzig}. Die Premiere von \emph{\textcolor{green}{Der Schleier der Beatrice}}, mit \textcolor{blue}{Irene Triesch} in der Hauptrolle, fand am
                        24. 5. 1903 statt.}}}\label{K_L03373-7h} geht, um dort den
                     »\textcolor{green}{Schleier der \textsc{Beatrice}}{}\ledrightnote{\textcolor{green}{Der Schleier der Beatrice. Schauspiel in fünf Akten}}« zu ſpielen.}{\lemma{\textnormal{\emph{Die … ſpielen.}}}\Cendnote{\textnormal{kopfüber im oberen
                     rechten Eck der ersten Seite}}}\label{T_L03373-1h}\pend
           \endnumbering\briefempfaengerindex{Schnitzler, Arthur@\textsc{Schnitzler, Arthur}!zzzGoldmann, Paul@\emph{von Paul Goldmann}!1903-05-221@{22. 5. {[}1903{]}}|)be}\mylabel{h}
\begin{anhang}
\end{anhang}\normalsize

\doendnotes{C}
\bigskip
\vfill

\clearpage

\footnotesize

\lohead{\textsc{register}}

% Definiere theindex-Environment komplett neu ohne reledmac
\makeatletter
\renewenvironment{theindex}{%
  \section*{\indexname}%
  \setlength{\parindent}{0pt}%
  \setlength{\parskip}{0pt plus 0.3pt}%
  \let\item\@idxitem
}{%
  \clearpage
}
\makeatother

\IfFileExists{\jobname-pw.ind}{\input{\jobname-pw.ind}}{}

\end{document}

      