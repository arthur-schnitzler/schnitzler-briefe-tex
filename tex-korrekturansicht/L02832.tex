%% latex-korrekturansicht-vorspann.tex
%% Vorspann für die Korrekturansicht.
%% Lädt die gemeinsame Datei latex-vorspann.tex mit gesetztem Schalter.

\newif\ifkorrekturansicht
\korrekturansichttrue

\input{../tex-inputs/latex-vorspann}


               \section[ Paul Goldmann an Arthur Schnitzler, 24. 11. {[}1897{]}]{Paul Goldmann an Arthur Schnitzler, 24. 11. {[}1897{]}}\nopagebreak\mylabel{v}\rehead{ }\normalsize\beginnumbering\briefempfaengerindex{Schnitzler, Arthur@\textsc{Schnitzler, Arthur}!zzzGoldmann, Paul@\emph{von Paul Goldmann}!1897-11-241@{24. 11. {[}1897{]}}|(be} \toendnotes[C]{\smallbreak\pagebreak[2]} \Standort{DLA, A:Schnitzler, HS.NZ85.1.3167.}
\physDesc{Brief, 1 Blatt, 3 Seiten
\newline{}Handschrift: blaue Tinte, deutsche Kurrent
\newline{}Schnitzler: 1) mit Bleistift das Jahr »97« vermerkt 2) mit rotem Buntstift eine Unterstreichung}\toendnotes[C]{\smallbreak}\pstart
           \noindent{}{\pb}\textcolor{gray}{\textbf{\textbf{\textcolor{brown}{Frankfurter Zeitung}{}\ledrightnote{\textcolor{brown}{Frankfurter Zeitung}}}}}\pend
           \pstart
           \textcolor{gray}{\textbf{(\textcolor{brown}{\begin{otherlanguage}{french}Gazette de Francfort\end{otherlanguage}}{}\ledrightnote{\textcolor{brown}{Frankfurter Zeitung}}).}}\pend
           \pstart
           \textcolor{gray}{\textbf{\textbf{\begin{otherlanguage}{french}Fondateur M.\end{otherlanguage}{ }\textcolor{blue}{L. Sonnemann}{}\ledrightnote{\textcolor{blue}{Leopold Sonnemann}}.}}}\pend
           \pstart
           \begin{otherlanguage}{french}\textcolor{gray}{\textbf{Journal politique, financier,}}\end{otherlanguage}\pend
           \pstart
           \begin{otherlanguage}{french}\textcolor{gray}{\textbf{commercial et littéraire.}}\end{otherlanguage}\pend
           \pstart
           \begin{otherlanguage}{french}\textcolor{gray}{\textbf{\textbf{Paraissant trois fois par jour.}}}\end{otherlanguage}\hfill \textsc{\textcolor{pink}{Paris}{}\ledrightnote{\textcolor{pink}{Paris}}}, 24. November.\pend
           \pstart
           \begin{otherlanguage}{french}\textcolor{gray}{\textbf{\textbf{Bureau à \textcolor{pink}{Paris}{}\ledrightnote{\textcolor{pink}{Paris}}}}}\end{otherlanguage}\pend
           \pstart
           \begin{otherlanguage}{french}\textcolor{gray}{\textbf{\textbf{\textcolor{pink}{10 Rue de la Bourse}{}\ledrightnote{\textcolor{pink}{rue de la Bourse}}.}}}\end{otherlanguage}\pend
           \pstart\center{}Mein lieber Freund,\pend\pstart
           Ich hoffe, die kleine \label{K_L02832-1v}\edtext{Reiſe}{\lemma{\textnormal{\emph{Reiſe}}}\Cendnote{\textnormal{\textcolor{blue}{Schnitzler} hielt sich von 24. 11. 1897 bis 28. 11. 1897 in \textcolor{pink}{Prag} auf. Am 25. 11. 1897 las er im gut besuchten \textcolor{pink}{Deutschen Casino} und am
                     27. 11. 1897 fand
                  die Premiere von \emph{\textcolor{green}{Freiwild}} statt – ein
                     »Erfolg; anfangs sehr stark, gegen Schluss sich schwächend.«
                     (A. S.: \emph{Tagebuch}, 27. 11. 1897)}}}\label{K_L02832-1h} wird Dir gut
               anſchlagen und Dich aus Deinen Hypochondrien herausreißen. Auch gibt es hoffentlich
               in \textsc{\textcolor{pink}{Prag}{}\ledrightnote{\textcolor{pink}{Prag}}} neue Erfolge. Wenigſtens wünſche ich das von Herzen.\pend
           \pstart
           Als ich heut Deinen Brief erhielt, bekam ich eine \strikeout{S\textcolor{gray}{o}} ſolche Sehnſucht nach Heimath und Freunden und Ruhe! Und ich hatte eine ſolche
               Luſt, all’ dieſe undankbare Arbeit hier hinzuwerfen, die mir meine Geſundheit
               zerrüttet und mich um mein Leben beſtiehlt!\pend
           \pstart
           {\pb}Was bin ich doch für ein armer Sklave! Und wie biſt
               Du glücklich gegen mich, ſelbſt mit \label{K_L02832-3v}\edtext{Ohrenklingen}{\lemma{\textnormal{\emph{Ohrenklingen}}}\Cendnote{\textnormal{Bezug auf \textcolor{blue}{Schnitzler}s Otosklerose – einer Verknöcherung
                  des Innenohrs mit zunehmender Schwerhörigkeit –, an der er seit
                     Herbst 1896 litt}}}\label{K_L02832-3h}. Ich wünſchte, mir kl\substVorne{}\textsuperscript{\textcolor{gray}{×}}\substDazwischen{}\textcolor{gray}{ä}\substHinten{}ngen die Ohren ſo wie Dir!\pend
           \pstart
           Dein \textcolor{green}{Stück}{}\ledrightnote{→\textcolor{green}{Das Vermächtnis. Schauspiel in drei Akten}} wird ſich ſchon aus
               dem Unklaren heraus arbeiten. Kein Wunder, daß es nicht gleich \label{K_L02832-5v}\edtext{auf den erſten Wurf gelungen}{\lemma{\textnormal{\emph{auf … gelungen}}}\Cendnote{\textnormal{siehe A. S.: \emph{Tagebuch}, 21. 11. 1897}}}\label{K_L02832-5h} iſt, bei all’ den Aufregungen, welche Du haſt durchmachen müſſen. Auch haſt
               Du ja ſtets Deine Stücke mehrmals geſchrieben. Und wenn \strikeout{\textcolor{gray}{e}s} gar ſo Talent dazu gehörte, einen {\pb}guten erſten Akt zu ſchreiben, ſo gäbe es mehr gute
               erſte Akte, als es gibt.\pend
           \pstart
           Warum Du von Deiner trüben Zukunft ſprichſt, begreife ich auch nicht. Ich finde das
               genaue Gegentheil.\pend
           \pstart
           Alſo erhole Dich recht und genieße die \textcolor{pink}{prag}{}\ledrightnote{→\textcolor{pink}{Prag}}er Tage!\pend
           \pstart
           Und \label{K_L02832-11v}\edtext{ſieh’ Dir das liebe Geſicht des
               kleinen \textcolor{blue}{Mädchen}{}\ledrightnote{→\textcolor{blue}{Alice Ziegler}}s an}{\lemma{\textnormal{\emph{ſieh’ … an}}}\Cendnote{\textnormal{siehe Paul Goldmann an Arthur Schnitzler, 19. 11. [1897]}}}\label{K_L02832-11h} und ſage mir, was darin ſteht.\pend
           \pstart
           Berichte mir \strikeout{b\textcolor{gray}{a}} bald und viel!\pend
           \pstart
           Von Herzen {\\[\baselineskip]}Dein {\\[\baselineskip]}\spacefill\mbox{Paul Goldmnn}\pend
           \leftskip=0em{}\pstart
           \noindent{}\label{T_L02832-1v}\edtext{{\pb}Ich hoffe, es kommt zur \label{K_L02832-44v}\edtext{Reviſion des Prozeſſes \textsc{\textcolor{blue}{Dreyfus}{}\ledrightnote{\textcolor{blue}{Alfred Dreyfus}}}}{\lemma{\textnormal{\emph{Reviſion … Dreyfus}}}\Cendnote{\textnormal{Zu einem weiteren Gerichtsprozess in
                     der \textcolor{blue}{Dreyfus}-Affäre kam es erst am 10. und 11. 1. 1898.
                        \textcolor{blue}{Ferdinand Walsin-Esterházy}, der das
                     Gerichtsverfahren gegen sich selbst beantragt hatte, wurde dort freigesprochen.
                     Eigentlich war aber er – und nicht \textcolor{blue}{Alfred
                        Dreyfus} – schuldig, hatte er \textcolor{blue}{Maximilian von Schwartzkoppen} doch (gegen Geld) die geheimen
                     militärischen Dokumente gegeben, die die \textcolor{blue}{Dreyfus}-Affäre ausgelöst hatten.}}}\label{K_L02832-44h}. Der \textsc{\textcolor{blue}{Esterhazy}{}\ledrightnote{\textcolor{blue}{Ferdinand Walsin-Esterházy}}} iſt wohl ſchuldig. Aber weſſen? Des Verraths? Der Fälſchung? Dunkel,
                     dunkel!}{\lemma{\textnormal{\emph{Ich … dunkel!}}}\Cendnote{\textnormal{kopfüber am oberen Rand der
                     ersten Seite}}}\label{T_L02832-1h}\pend
           \endnumbering\briefempfaengerindex{Schnitzler, Arthur@\textsc{Schnitzler, Arthur}!zzzGoldmann, Paul@\emph{von Paul Goldmann}!1897-11-241@{24. 11. {[}1897{]}}|)be}\mylabel{h}\begin{anhang}\end{anhang}\normalsize

\doendnotes{C}
\bigskip
\vfill

\clearpage

\footnotesize

\lohead{\textsc{register}}

% Definiere theindex-Environment komplett neu ohne reledmac
\makeatletter
\renewenvironment{theindex}{%
  \section*{\indexname}%
  \setlength{\parindent}{0pt}%
  \setlength{\parskip}{0pt plus 0.3pt}%
  \let\item\@idxitem
}{%
  \clearpage
}
\makeatother

\IfFileExists{\jobname-pw.ind}{\input{\jobname-pw.ind}}{}

\end{document}

      