%% latex-korrekturansicht-vorspann.tex
%% Vorspann für die Korrekturansicht.
%% Lädt die gemeinsame Datei latex-vorspann.tex mit gesetztem Schalter.

\newif\ifkorrekturansicht
\korrekturansichttrue

\input{../tex-inputs/latex-vorspann}


               \section[Arthur Schnitzler an Richard Beer-Hofmann, {[}16. 6. 1902?{]}]{ Arthur Schnitzler an Richard Beer-Hofmann, {[}16. 6. 1902?{]}}\nopagebreak\mylabel{v}\rehead{ }\normalsize\beginnumbering\briefempfaengerindex{Beer-Hofmann, Richard@\textsc{Beer-Hofmann, Richard}!zzzSchnitzler, Arthur@\emph{von Arthur Schnitzler}!1902-06-161@{{[}16. 6. 1902?{]}}|(be} \toendnotes[C]{\smallbreak\pagebreak[2]} \Standort{YCGL, MSS 31.}
\physDesc{Brief, 1 Blatt, 1 Seite
\newline{}Handschrift: Bleistift, deutsche Kurrent}\toendnotes[C]{\smallbreak}\pstart
           \noindent{}{\pb}\label{K_L01193-1v}\edtext{Auf Wiederſehen}{\lemma{\textnormal{\emph{Auf Wiederſehen}}}\Cendnote{\textnormal{Das undatierte gefaltete Blatt wird unter den
                  Korrespondenzstücken des Jahres 1902 aufbewahrt. Der Falz in der
                  Mitte könnte darauf hinweisen, dass es sich um eine Art Umschlag handelte, in den
                  etwas eingelegt war. Innerhalb des betreffenden Jahres fügt es sich in der
                  erhaltenen Korrespondenz nur als Folgeschreiben zu Arthur Schnitzler an Richard Beer-Hofmann, 13. 6. 1902 ein, zur Übermittlung der versprochenen Karten.}}}\label{K_L01193-1h},\pend
           \pstart
           Herzlichſt{\\[\baselineskip]}Ihr{\\[\baselineskip]}\spacefill\mbox{ArthS}\pend
           \leftskip=0em{}\endnumbering\briefempfaengerindex{Beer-Hofmann, Richard@\textsc{Beer-Hofmann, Richard}!zzzSchnitzler, Arthur@\emph{von Arthur Schnitzler}!1902-06-161@{{[}16. 6. 1902?{]}}|)be}\mylabel{h}  \normalsize

\doendnotes{C}
\bigskip
\vfill

\clearpage

\footnotesize

\lohead{\textsc{register}}

% Definiere theindex-Environment komplett neu ohne reledmac
\makeatletter
\renewenvironment{theindex}{%
  \section*{\indexname}%
  \setlength{\parindent}{0pt}%
  \setlength{\parskip}{0pt plus 0.3pt}%
  \let\item\@idxitem
}{%
  \clearpage
}
\makeatother

\IfFileExists{\jobname-pw.ind}{\input{\jobname-pw.ind}}{}

\end{document}

      