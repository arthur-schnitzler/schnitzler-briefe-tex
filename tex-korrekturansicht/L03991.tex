%% latex-korrekturansicht-vorspann.tex
%% Vorspann für die Korrekturansicht.
%% Lädt die gemeinsame Datei latex-vorspann.tex mit gesetztem Schalter.

\newif\ifkorrekturansicht
\korrekturansichttrue

\input{../tex-inputs/latex-vorspann}


\section[Arthur Schnitzler: Widmungsexemplar von Casanovas Heimfahrt an Berta Zuckerkandl, Jänner 1919]{L03991 Arthur Schnitzler: Widmungsexemplar von Casanovas Heimfahrt an Berta
               Zuckerkandl, Jänner 1919}
\nopagebreak\mylabel{L03991v}
\rehead{ }\normalsize\beginnumbering\briefempfaengerindex{Zuckerkandl, Berta@\textsc{Zuckerkandl, Berta}!zzzSchnitzler, Arthur@\emph{von Arthur Schnitzler}!1919-01-311@{Jänner 1919}|(be}
\toendnotes[C]{\smallbreak\pagebreak[2]}
\correspDesc{Versand  durch Arthur Schnitzler im Zeitraum Jänner 1919 in Wien
\newline{}Erhalt  durch Berta Zuckerkandl im Zeitraum Jänner 1919 \textbf{Ort fehlend} }\toendnotes[C]{\smallbreak}
\Standort{Wien, Österreichische Nationalbibliothek, ZUC.5.1.SchCas LIT MAG.}
\physDesc{Widmung am Schmutztitel, 125 Zeichen
\newline{}Handschrift: schwarze Tinte, lateinische Kurrent}\toendnotes[C]{\smallbreak}
\pstart
           {\pb}{[}\textcolor{brown}{S. Fischer Verlag}\orgindex{S. Fischer Verlag@S. Fischer Verlag|pw}{}\ledrightnote{\textcolor{brown}{S. Fischer Verlag}}{]}\hfill {[}\textcolor{brown}{S. Fischer Verlag}\orgindex{S. Fischer Verlag – Filiale Wien@S. Fischer Verlag – Filiale Wien|pw}{}\ledrightnote{\textcolor{brown}{S. Fischer Verlag – Filiale Wien}}{]}\pend
           
\pstart
           \textcolor{gray}{\textbf{\textcolor{pink}{BERLIN}\oindex{Berlin@\textbf{Berlin}, \emph{Hauptstadt}|pw}{}\ledrightnote{\textcolor{pink}{Berlin}}}}\hfill \textcolor{gray}{\textbf{\textcolor{pink}{WIEN}\oindex{Wien@\textbf{Wien}, \emph{Verwaltungsgebiet}|pw}{}\ledrightnote{\textcolor{pink}{Wien}}}}\pend
           \vspace{0.5em}
\pstart
           Frau Hofrätin Bertha Zuckerkandl\pend
           \pstart in herzlicher Verehrung \spacefill\mbox{ArthurSchnitzler}\pend{}
\pstart
           \textcolor{pink}{Wien}\oindex{Wien@\textbf{Wien}, \emph{Verwaltungsgebiet}|pw}{}\ledrightnote{\textcolor{pink}{Wien}}, \label{K_L03990-1v}\edtext{Januar 1919}{\lemma{\textnormal{\emph{Januar 1919}}}\Cendnote{\textnormal{Weil Papiermangel herrschte, betreute in den Monaten nach
                     der Erstausgabe die \textcolor{pink}{Wiener}\oindex{Wien@\textbf{Wien}, \emph{Verwaltungsgebiet}|pwk} Filiale des \emph{\textcolor{brown}{S.-Fischer-Verlags}\orgindex{S. Fischer Verlag – Filiale Wien@S. Fischer Verlag – Filiale Wien|pwk}} die weiteren
                     Auflagen.}}}\label{K_L03990-1}.\pend
           {\vspace{1\baselineskip}}\selectlanguage{ngerman}\vspace{1em}{\vspace{1\baselineskip}}
\pstart
           \centering{}{\pb}\textcolor{gray}{\textbf{\textcolor{green}{CASANOVAS HEIMFAHRT}\pwindex{Schnitzler, Arthur 15. 5. 1862 Wien – 21. 10. 1931 ebd.@\textsc{Schnitzler, Arthur} (15. 5. 1862 Wien – 21. 10. 1931 ebd.), \emph{Schriftsteller, Mediziner}!Casanovas Heimfahrt@\strich\emph{Casanovas Heimfahrt}|pw}{}\ledrightnote{\textcolor{green}{Casanovas Heimfahrt}}}}\pend
           
\pstart
           \centering{}\textcolor{gray}{\textbf{NOVELLE}}\pend
           
\pstart
           \centering{}\textcolor{gray}{\textbf{VON}}\pend
           
\pstart
           \centering{}\textcolor{gray}{\textbf{ARTHUR SCHNITZLER}}\pend
           {\vspace{1\baselineskip}}
\pstart
           \centering{}\textcolor{gray}{\textbf{1919}}\pend
           
\pstart
           \centering{}\textcolor{gray}{\textbf{\textcolor{brown}{S. FISCHER, VERLAG}\orgindex{S. Fischer Verlag@S. Fischer Verlag|pw}{}\ledrightnote{\textcolor{brown}{S. Fischer Verlag}}, \textcolor{pink}{BERLIN}\oindex{Berlin@\textbf{Berlin}, \emph{Hauptstadt}|pw}{}\ledrightnote{\textcolor{pink}{Berlin}}}}\pend
           \selectlanguage{ngerman}\endnumbering\briefempfaengerindex{Zuckerkandl, Berta@\textsc{Zuckerkandl, Berta}!zzzSchnitzler, Arthur@\emph{von Arthur Schnitzler}!1919-01-011@{Jänner 1919}|)be}\mylabel{L03991h}
\begin{anhang}
\end{anhang}\normalsize

\doendnotes{C}
\bigskip
\vfill

\clearpage

\footnotesize

\lohead{\textsc{register}}

% Definiere theindex-Environment komplett neu ohne reledmac
\makeatletter
\renewenvironment{theindex}{%
  \section*{\indexname}%
  \setlength{\parindent}{0pt}%
  \setlength{\parskip}{0pt plus 0.3pt}%
  \let\item\@idxitem
}{%
  \clearpage
}
\makeatother

\IfFileExists{\jobname-pw.ind}{\input{\jobname-pw.ind}}{}

\end{document}

      