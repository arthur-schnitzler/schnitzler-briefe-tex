%% latex-korrekturansicht-vorspann.tex
%% Vorspann für die Korrekturansicht.
%% Lädt die gemeinsame Datei latex-vorspann.tex mit gesetztem Schalter.

\newif\ifkorrekturansicht
\korrekturansichttrue

\input{../tex-inputs/latex-vorspann}


               \section[Therese Rie-Andro an Arthur Schnitzler, 6. 2. 1912]{ Therese Rie-Andro an Arthur Schnitzler, 6. 2. 1912}\nopagebreak\mylabel{v}\rehead{ }\normalsize\beginnumbering\briefempfaengerindex{Schnitzler, Arthur@\textsc{Schnitzler, Arthur}!zzzRie, Therese@\emph{von Therese Rie}!1912-02-061@{6. 2. 1912}|(be} \toendnotes[C]{\smallbreak\pagebreak[2]} \Standort{DLA, A:Schnitzler, 85.1.4310.}
\physDesc{Brief, 1 Blatt, 3 Seiten
\newline{}Handschrift: blaue Tinte, lateinische Kurrent
\newline{}Schnitzler: 1) mit Bleistift beschriftet: »\textsc{Andro}« 2) mit rotem Buntstift eine Unterstreichung}\toendnotes[C]{\smallbreak}\pstart
           \raggedleft{}{\pb}\textcolor{pink}{Wien}{}\ledrightnote{\textcolor{pink}{Wien}}, d. 6. Februar
                        1912.\pend
           \pstart
           \raggedleft{}\textcolor{pink}{IV, Schönburgstr. 48}{}\ledrightnote{\textcolor{pink}{Schönburgstraße}}.\pend
           \pstart{}Sehr geehrter Herr,\pend\pstart
           \textcolor{blue}{Hans Pfitzner}{}\ledrightnote{\textcolor{blue}{Hans Pfitzner}} sendet Ihnen durch mich die
                    Dichtung zu seine\substVorne{}\textsuperscript{r}\substDazwischen{}m\substHinten{} neuesten \substVorne{}\textsuperscript{Arbeit}{\allowbreak}\substDazwischen{}Musikdrama\substHinten{} »\textcolor{green}{Palestrina}{}\ledrightnote{\textcolor{green}{Palestrina. Musikalische Legende in drei Akten}}«, zugleich seinen ersten
                        \uline{dichterischen}{ }\textcolor{green}{Versuch}{}\ledrightnote{→\textcolor{green}{Palestrina. Musikalische Legende in drei Akten}}, und bittet Sie, als
                    einen der ganz Wenigen, an dessen Urteil ihm gelegen ist, sie zu lesen.\pend
           \pstart
           Wenn er selbst sich nicht direkt an Sie wendet, liegt es zum Teil an seiner
                    Ueberbürdung mit Arbeit (er iſt, wie Sie vielleicht wissen, Direktor der \textcolor{brown}{Oper}{}\ledrightnote{\textcolor{brown}{Oper Straßburg}} und des \textcolor{brown}{Konservatoriums}{}\ledrightnote{\textcolor{brown}{Städtisches Konservatorium}} in \textcolor{pink}{Straßburg}{}\ledrightnote{\textcolor{pink}{Straßburg}} und
                    Leiter der \textcolor{brown}{Orchesterkonzerte}{}\ledrightnote{\textcolor{brown}{Straßburger Philharmoniker}}), zum Teil an
                    einer gewissen Scheu dem Briefschreiben gegenüber, die er mit {\pb}manchen seiner großen Kollegen gemeinsam hat, und \substVorne{}\textsuperscript{\textcolor{gray}{lieber}}{\allowbreak}\substDazwischen{}wobei\substHinten{} er lieber seine »Jünger« ins Treffen schickt.\pend
           \pstart
           \textcolor{blue}{Pfitzner}{}\ledrightnote{\textcolor{blue}{Hans Pfitzner}} weiß, daß Sie seinen Schöpfungen Ihr
                    Interesse nicht entsagt haben, wenn sie – leider viel zu wenig! – in \textcolor{pink}{Wien}{}\ledrightnote{\textcolor{pink}{Wien}} zu hören waren. Vielleicht aber wissen Sie,
                    sehr geehrter Herr Doctor, nicht, daß er zu Ihren wärmsten Bewunderern zählt; er
                    hat sich unter anderm jahrelang mit Ihrem »\textcolor{green}{Parazelsus}{}\ledrightnote{\textcolor{green}{Paracelsus. Versspiel in einem Akt}}« beschäftigt und ich kann es nicht genug beklagen, daß
                    seine Liebe für dieses eminent »musikalische« Werk sich nicht zu Musik
                    verdichtet hat. Ich denke i{\geminationm}er, einmal wird das
                    noch werden.\pend
           \pstart
           \textcolor{blue}{Pfitzner}{}\ledrightnote{\textcolor{blue}{Hans Pfitzner}} hat seine \textcolor{green}{Dichtung}{}\ledrightnote{→\textcolor{green}{Palestrina. Musikalische Legende in drei Akten}} – die Partitur ist erst in den
                    allerersten Anfängen vorhanden – in ganz wenigen Exemplaren für Freunde drucken
                    lassen. Er hat mich ermächtigt, Ihnen das meine zu senden und ich bitte Sie, es
                    ruhig so lange zu behalten, als es Ihnen lieb iſt. Doch bittet mich \textcolor{blue}{Pfitzner}{}\ledrightnote{\textcolor{blue}{Hans Pfitzner}} sehr, \substVorne{}\textsuperscript{\textcolor{gray}{seine}}\substDazwischen{}die Ueber\substHinten{}sendung seiner \textcolor{green}{Dichtung}{}\ledrightnote{→\textcolor{green}{Palestrina. Musikalische Legende in drei Akten}} als einen Akt des innigsten persönlichen Vertrauens
                    aufzufassen und auch zu Freunden nicht drüber zu sprechen, ehe nicht auch der
                    musikalische Teil der Arbeit vollendet ist.\pend
           \pstart
           Verzeihen Sie, sehr geehrter Herr, wenn ich Ihnen diese ein wenig drakonischen
                        Besti{\geminationm}ungen des \textcolor{blue}{Meisters}{}\ledrightnote{→\textcolor{blue}{Hans Pfitzner}} völlig ungeschminkt übermittle; allein ich
                    bin es gewöhnt, mich seinen künstlerischen Wünschen unbedingt unterzuordnen und
                    überzeugt, daß diese auch bei Ihnen das {\pb}\substVorne{}\textsuperscript{\textcolor{gray}{äußerste}}{\allowbreak}\substDazwischen{}absoluteste\substHinten{} Verständnis finden werden.\pend
           \pstart
           Ich begrüße Sie in herzlicher Bewunderung.{\\[\baselineskip]}\spacefill\mbox{L. Andro. (\label{K_L02569-1v}\edtext{R.}{\lemma{\textnormal{\emph{R.}}}\Cendnote{\textnormal{für »Risa«}}}\label{K_L02569-1h} Rie.)}\pend
           \leftskip=0em{}\endnumbering\briefempfaengerindex{Schnitzler, Arthur@\textsc{Schnitzler, Arthur}!zzzRie, Therese@\emph{von Therese Rie}!1912-02-061@{6. 2. 1912}|)be}\mylabel{h}  \normalsize

\doendnotes{C}
\bigskip
\vfill

\clearpage

\footnotesize

\lohead{\textsc{register}}

% Definiere theindex-Environment komplett neu ohne reledmac
\makeatletter
\renewenvironment{theindex}{%
  \section*{\indexname}%
  \setlength{\parindent}{0pt}%
  \setlength{\parskip}{0pt plus 0.3pt}%
  \let\item\@idxitem
}{%
  \clearpage
}
\makeatother

\IfFileExists{\jobname-pw.ind}{\input{\jobname-pw.ind}}{}

\end{document}

      