%% latex-korrekturansicht-vorspann.tex
%% Vorspann für die Korrekturansicht.
%% Lädt die gemeinsame Datei latex-vorspann.tex mit gesetztem Schalter.

\newif\ifkorrekturansicht
\korrekturansichttrue

\input{../tex-inputs/latex-vorspann}


               \section[Paul Goldmann an Arthur Schnitzler, 23. 12. {[}1893{]}]{ Paul Goldmann an Arthur Schnitzler, 23. 12. {[}1893{]}}\nopagebreak\mylabel{v}\rehead{ }\normalsize\beginnumbering\briefempfaengerindex{Schnitzler, Arthur@\textsc{Schnitzler, Arthur}!zzzGoldmann, Paul@\emph{von Paul Goldmann}!1893-12-231@{23. 12. {[}1893{]}}|(be} \toendnotes[C]{\smallbreak\pagebreak[2]} \Standort{DLA, A:Schnitzler, HS.NZ85.1.3163.}
\physDesc{Brief, 1 Blatt, 4 Seiten
\newline{}Handschrift: schwarze Tinte, deutsche Kurrent
\newline{}Schnitzler: 1) mit Bleistift das Jahr »93« vermerkt 2) mit rotem Buntstift zwei Unterstreichungen}\toendnotes[C]{\smallbreak}\pstart
           \raggedleft{}{\pb}\textsc{\textcolor{pink}{Paris}{}\ledrightnote{\textcolor{pink}{Paris}}}, 23. December.\pend
           \pstart\center{}Mein lieber Freund!\pend\pstart
           Dein letzter Brief und die ſich daran ſchließenden Zeilen der \label{K_L02724-1v}\edtext{\textcolor{blue}{Freunde}{}\ledrightnote{→\textcolor{blue}{Hugo von Hofmannsthal}{\newline}→\textcolor{blue}{Richard Beer-Hofmann}{\newline}→\textcolor{blue}{Felix Salten}{\newline}→\textcolor{blue}{Gustav Schwarzkopf}}}{\lemma{\textnormal{\emph{Freunde}}}\Cendnote{\textnormal{\textcolor{blue}{Schnitzler}s Brief könnte am 10. 12. 1893 abgefasst gewesen sein, als er die
                  mit \textcolor{blue}{Goldmann} bekannten \textcolor{blue}{Freunde}{ }\textcolor{blue}{Hugo von Hofmannsthal}, \textcolor{blue}{Richard Beer-Hofmann}, \textcolor{blue}{Felix Salten} und \textcolor{blue}{Gustav
                     Schwarzkopf} traf.}}}\label{K_L02724-1h} haben mir eine unendliche Freude bereitet. Mir
               ſind die Thränen in die Augen gekommen, als ich all’ das las. Und ich war einen
               ganzen Tag lang glücklich, ſo viel Freundſchaft und Treue verdient zu haben. Gern
               hätte ich Dir, dem lieben Anſtifter der Freudengabe, und allen \textcolor{blue}{Betheiligten}{}\ledrightnote{→\textcolor{blue}{Hugo von Hofmannsthal}{\newline}→\textcolor{blue}{Richard Beer-Hofmann}{\newline}→\textcolor{blue}{Felix Salten}{\newline}→\textcolor{blue}{Gustav Schwarzkopf}} ſofort gedankt. Da kam die \label{K_L02724-2v}\edtext{Bombe in der \textcolor{brown}{Kammer}{}\ledrightnote{\textcolor{brown}{Nationalversammlung}}}{\lemma{\textnormal{\emph{Bombe in der Kammer}}}\Cendnote{\textnormal{Am 9. 12. 1893 verübte der
                  Anarchist \textcolor{blue}{Auguste Vaillant} ein
                  Bombenattentat auf die \emph{\textcolor{brown}{Französische
                     Nationalversammlung}}, bei dem um die 50 Personen verletzt wurden.}}}\label{K_L02724-2h}
               und ſonſt Allerlei und warf mich weit ab von Euch und all’ den frohen Gedanken. {\pb}Inzwiſchen kam auch Dein liebes \label{K_L02724-3v}\edtext{Bild}{\lemma{\textnormal{\emph{Bild}}}\Cendnote{\textnormal{Wohl das von \textcolor{blue}{Carl
                     Pietzner} erstellte \textcolor{green}{Porträtfoto} von \textcolor{blue}{Schnitzler}, vgl. Arthur Schnitzler an Hermann Bahr, 2. 12. 1893.}}}\label{K_L02724-3h}. Dank, innigen Dank für
               die Sendung. Ich habe es auf meinem Schreibtiſch aufgeſtellt und tauſche mit Dir
               manch’ einen Blick und verſinke in manch’ eine Träumerei während irgendeines
               politiſchen Artikels. Es iſt eine vorzügliche Aufnahme – wenngleich Du freilich in
               Wirklichkeit nie ſo hübſch geweſen. Auch zeige ich Dich Allen, die mich beſuchen
               kommen, und Du haſt viel Erfolg. Neulich war \textsc{\textcolor{blue}{Jean Thorel}{}\ledrightnote{\textcolor{blue}{Jean Thorel}}} bei mir und ſagte: \label{K_L02724-4v}\edtext{»\begin{otherlanguage}{french}Je jurerais, que c’est un monsieur, qui écrit des
                  comédies.\end{otherlanguage}«}{\lemma{\textnormal{\emph{»Je … comédies.«}}}\Cendnote{\textnormal{französisch: Ich
                  könnte schwören, dass das ein Herr ist, der Lustspiele schreibt.}}}\label{K_L02724-4h} Wenn Du
               jetzt {\pb}\strikeout{\textcolor{gray}{noc}} noch \label{K_L02724-5v}\edtext{keine Luſtſpiele ſchreiben
                  willſt}{\lemma{\textnormal{\emph{keine … willſt}}}\Cendnote{\textnormal{vgl. Paul Goldmann an Arthur Schnitzler, 8. 12. [1893]}}}\label{K_L02724-5h}{\dotsfour}!\pend
           \pstart
           Bitte liebſter Freund, ſchreib’ mir ein ausführlicheres Wort über Deine Pläne. Die
               Idee mit dem \label{K_L02724-22v}\edtext{ſüßen Wiener \textcolor{green}{Stück}{}\ledrightnote{→\textcolor{green}{Liebelei. Schauspiel in drei Akten}}}{\lemma{\textnormal{\emph{ſüßen Wiener Stück}}}\Cendnote{\textnormal{\emph{\textcolor{green}{Liebelei}}, das unter dem Titel »Armes Mädl«
                  als Volksstück geplant war. Das »süß« dürfte sich auf das »süße Mädl« beziehen,
                  dass schon früher in den Briefen \textcolor{blue}{Goldmann}s
                  Thema war (Paul Goldmann an Arthur Schnitzler, 18. 8. 1890). Die Popularisierung des
                  Begriffs wird häufig \textcolor{blue}{Schnitzler}
                  zugeschrieben und der Erfolg der \emph{\textcolor{green}{Liebelei}}
                  spielt dabei eine zentrale Rolle. Diese Briefstelle legt nahe, dass schon bei der
                  Konzeption der \emph{\textcolor{green}{Liebelei}} das Vorhaben eine
                  zentrale Rolle spielte, den Typus »unkomplizierte Frau für eine sexuelle Beziehung
                  ohne längerfristige Bindung« auf die Bühne zu bringen.}}}\label{K_L02724-22h} gefällt mir ſehr.
               Das müßte Dir ganz ausnehmend liegen. Und ſchreib’ vor allen Dingen ein Stück ohne
               Dich. Was macht dein \textcolor{green}{Roman}{}\ledrightnote{→\textcolor{green}{Sterben. Novelle}}?
               Brinſgt Du ihn nirgends an? Sende mir auch, wenn möglich, ein oder zwei Exemplare \textsc{\textcolor{green}{Anatol}{}\ledrightnote{\textcolor{green}{Anatol}}} zu Progaganda-Zwecken. In \textcolor{pink}{Paris}{}\ledrightnote{\textcolor{pink}{Paris}} bekommt man
               nämlich nie ein Buch wieder, wenn man es wegborgt. Ich hoffe doch {\pb}noch etwas für Dich hier durchzuſetzen. Die Übergabe
               Deiner \label{K_L02724-566v}\edtext{Novellen}{\lemma{\textnormal{\emph{Novellen}}}\Cendnote{\textnormal{Von \textcolor{blue}{Schnitzler} existierte zu dieser Zeit keine Buchausgabe seiner Novellen,
                  welche hier für die Vermittlung vorgesehen waren, lässt sich nicht
                  bestimmen.}}}\label{K_L02724-566h} an eine \label{K_L02724-7v}\edtext{\textcolor{blue}{Mitarbeiterin}{}\ledrightnote{→\textcolor{blue}{?? [Mitarbeiterin von La Vie Parisienne]}}}{\lemma{\textnormal{\emph{Mitarbeiterin}}}\Cendnote{\textnormal{nicht identifiziert}}}\label{K_L02724-7h} der \textsc{\textcolor{brown}{Vie Parisienne}{}\ledrightnote{\textcolor{brown}{La Vie Parisienne}}} habe ich doch nicht in’s Werk ſetzen wollen. Gewiſſe \label{K_L02724-8v}\edtext{Erfahrungen der letzten Zeit haben mich gelehrt, daß
               möglicher Weiſe Deine Novelle Aufnahme gefunden hätte, aber nicht unter Deinem
                  Namen}{\lemma{\textnormal{\emph{Erfahrungen … Namen}}}\Cendnote{\textnormal{Auf welchen Plagiatsvorwurf \textcolor{blue}{Goldmann} anspielt, ist unklar.}}}\label{K_L02724-8h}, – Du
               verſtehſt? \pend
           \pstart
           Schreib’ mir auch, \label{K_L02724-9v}\edtext{was mit \textsc{\textcolor{blue}{Bahr}{}\ledrightnote{\textcolor{blue}{Hermann Bahr}}} vorgegangen iſt? Warum der Austritt aus der »\textcolor{brown}{Deutſchen Ztg}{}\ledrightnote{\textcolor{brown}{Deutsche Zeitung}}}{\lemma{\textnormal{\emph{was … Ztg}}}\Cendnote{\textnormal{Am
                     21. 12. 1893 stand in der \emph{\textcolor{green}{Deutschen Zeitung}}, dass \textcolor{blue}{Bahr} die \textcolor{brown}{Redaktion des
                     Blattes} verlassen habe (Nr. 7898, S. 5). Offizielle
                  Begründung gab es keine. \textcolor{blue}{Bahr} betreute seit
                     September 1892 die Theaterkritik und kündigte, nachdem zweimal in
                  Kritiken von ihm eingegriffen worden war, einmal um ein Lob, einmal um eine
                  kritische Äußerung zu streichen.}}}\label{K_L02724-9h}«? Wird das \textcolor{brown}{Blatt}{}\ledrightnote{→\textcolor{brown}{Deutsche Zeitung}} eingehen?\pend
           \pstart
           Fröhliche Feiertage, mein lieber Freund, und nochmals Dank Dir und den \textcolor{blue}{Andern}{}\ledrightnote{→\textcolor{blue}{Hugo von Hofmannsthal}{\newline}→\textcolor{blue}{Richard Beer-Hofmann}{\newline}→\textcolor{blue}{Felix Salten}{\newline}→\textcolor{blue}{Gustav Schwarzkopf}} und viele treue Grüße an
               Euch Alle.\pend
           \pstart Dein \spacefill\mbox{Paul Goldm}\pend{}\endnumbering\briefempfaengerindex{Schnitzler, Arthur@\textsc{Schnitzler, Arthur}!zzzGoldmann, Paul@\emph{von Paul Goldmann}!1893-12-231@{23. 12. {[}1893{]}}|)be}\mylabel{h}\begin{anhang}\end{anhang}\normalsize

\doendnotes{C}
\bigskip
\vfill

\clearpage

\footnotesize

\lohead{\textsc{register}}

% Definiere theindex-Environment komplett neu ohne reledmac
\makeatletter
\renewenvironment{theindex}{%
  \section*{\indexname}%
  \setlength{\parindent}{0pt}%
  \setlength{\parskip}{0pt plus 0.3pt}%
  \let\item\@idxitem
}{%
  \clearpage
}
\makeatother

\IfFileExists{\jobname-pw.ind}{\input{\jobname-pw.ind}}{}

\end{document}

      