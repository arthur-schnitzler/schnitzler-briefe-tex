%% latex-korrekturansicht-vorspann.tex
%% Vorspann für die Korrekturansicht.
%% Lädt die gemeinsame Datei latex-vorspann.tex mit gesetztem Schalter.

\newif\ifkorrekturansicht
\korrekturansichttrue

\input{../tex-inputs/latex-vorspann}


\renewcommand{\erwaehntePersonen}{Personen: Herbert Ginsberg, Ottilie Salten, Olga Schnitzler}
\renewcommand{\erwaehnteInstitutionen}{Institutionen: B.Z. am Mittag}
\renewcommand{\erwaehnteOrte}{Orte: Berlin, Griechenland, Kairo, Kochstraße, Wien}
\renewcommand{\erwaehnteWerke}{Werke: Reisen der Jahre 1893–1909}
\section[ Felix Salten an Arthur Schnitzler, 8. 4. 1906]{Felix Salten an Arthur Schnitzler, 8. 4. 1906}
\nopagebreak\mylabel{v}
\rehead{ }\normalsize\beginnumbering\briefempfaengerindex{Schnitzler, Arthur@\textsc{Schnitzler, Arthur}!zzzSalten, Felix@\emph{von Felix Salten}!1906-04-081@{8. 4. 1906}|(be}
\toendnotes[C]{\smallbreak\pagebreak[2]}\Standort{CUL, Schnitzler, B 89, B 1.}
\physDesc{Brief, 1 Blatt, 1 Seite, 732 Zeichen
\newline{}Handschrift: schwarze Tinte, lateinische Kurrent
\newline{}Ordnung: mit Bleistift von unbekannter Hand nummeriert: »208« }\toendnotes[C]{\smallbreak}
\pstart
           \noindent{}{\pb}\textcolor{brown}{\textcolor{gray}{\textbf{\emph{B. Z. am Mittag}}}}{}\ledrightnote{\textcolor{brown}{B.Z. am Mittag}}\hfill \textcolor{gray}{\textbf{\emph{\textcolor{pink}{BERLIN SW}{}\ledrightnote{\textcolor{pink}{Berlin}},}}}{ }8. IV. 06\pend
           
\pstart
           \textcolor{gray}{\textbf{\emph{Chefredaktion}}}\hfill \textcolor{pink}{\textcolor{gray}{\textbf{\emph{Kochstr. 23–25}}}}{}\ledrightnote{\textcolor{pink}{Kochstraße}}\pend
           
\pstart
           Lieber, erlauben Sie, dass ich Ihnen Herrn D\textsuperscript{r}{ }\textcolor{blue}{Herbert Ginsberg}{}\ledrightnote{\textcolor{blue}{Herbert Ginsberg}} vorstelle, den ich gerne bei
               Ihnen einführen möchte. Er kommt – studienhalber – für ein paar Monate nach \textcolor{pink}{Wien}{}\ledrightnote{\textcolor{pink}{Wien}}. Wenn Sie ihn freundlich aufnehmen wollen,
               werden Sie mich sehr verbinden und – gewiss – die lebhafte Sympathie, die ich für ihn
               habe, sehr bald teilen. Eine nähere Personalbeschreibung kann ich mir wol sparen.
               Aber unter manchen anderen Anknüpfungspunkten ist vielleicht der zu erwähnen, dass
               Herr D\textsuperscript{r} \textcolor{blue}{Ginsberg}{}\ledrightnote{\textcolor{blue}{Herbert Ginsberg}}
               viel gereist ist, (ich lernte ihn bei meinem Ausflug nach \label{K_L03417-1v}\edtext{\textcolor{pink}{Kairo}{}\ledrightnote{\textcolor{pink}{Kairo}}}{\lemma{\textnormal{\emph{Kairo}}}\Cendnote{\textnormal{Siehe Felix Salten an Arthur Schnitzler, 8. 3. 1904. Das Journal \emph{\textcolor{green}{Reisen
                  der Jahre 1893–1909}} von \textcolor{blue}{Ginsberg} ist online
                  einzusehen. Darin finden sich sowohl für den Aufenthalt in \textcolor{pink}{Kairo} als auch für die beiden Begegnungen mit \textcolor{blue}{Schnitzler} Aufzeichnungen (13. 4. 1906, S. 98,
                  und 12. 6. 1906,
                  S. 112),
                  https://archive.org/details/gilbertfamily01reel05/page/n443.}}}\label{K_L03417-1h}
               kennen) und Ihnen gewiss über einige Gegenden, die Sie interessiren, z. B. \textcolor{pink}{Griechenland}{}\ledrightnote{\textcolor{pink}{Griechenland}}, interessante Aufschlüße zu geben
               weiß.\pend
           
\pstart
           Herzlichste Grüße von \textcolor{blue}{Otti}{}\ledrightnote{\textcolor{blue}{Ottilie Salten}} und mir an Sie \textcolor{blue}{Beide}{}\ledrightnote{{$\rightarrow$}\textcolor{blue}{Olga Schnitzler}}.\pend
           \pstart Ihr \spacefill\mbox{Salten}\pend{}\endnumbering\briefempfaengerindex{Schnitzler, Arthur@\textsc{Schnitzler, Arthur}!zzzSalten, Felix@\emph{von Felix Salten}!1906-04-081@{8. 4. 1906}|)be}\mylabel{h}  \normalsize

\doendnotes{C}
\bigskip
\vfill

\clearpage

\footnotesize

\lohead{\textsc{register}}

% Definiere theindex-Environment komplett neu ohne reledmac
\makeatletter
\renewenvironment{theindex}{%
  \section*{\indexname}%
  \setlength{\parindent}{0pt}%
  \setlength{\parskip}{0pt plus 0.3pt}%
  \let\item\@idxitem
}{%
  \clearpage
}
\makeatother

\IfFileExists{\jobname-pw.ind}{\input{\jobname-pw.ind}}{}

\end{document}

      