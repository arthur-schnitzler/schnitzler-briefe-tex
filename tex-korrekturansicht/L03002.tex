%% latex-korrekturansicht-vorspann.tex
%% Vorspann für die Korrekturansicht.
%% Lädt die gemeinsame Datei latex-vorspann.tex mit gesetztem Schalter.

\newif\ifkorrekturansicht
\korrekturansichttrue

\input{../tex-inputs/latex-vorspann}


\renewcommand{\erwaehntePersonen}{Personen: Hermann Bahr, Otto Brahm, Felix Salten, Albert von Speidel}
\renewcommand{\erwaehnteInstitutionen}{Institutionen: Nationaltheater München}
\renewcommand{\erwaehnteOrte}{Orte: Berlin, Edmund-Weiß-Gasse 7, Kantstraße, Lessing-Theater, Wien}
\renewcommand{\erwaehnteWerke}{Werke: Der Ruf des Lebens. Schauspiel in drei Akten}
\section[ Arthur Schnitzler an Felix Salten, 30. 1. 1906]{Arthur Schnitzler an Felix Salten, 30. 1. 1906}
\nopagebreak\mylabel{v}
\rehead{ }\normalsize\beginnumbering\briefempfaengerindex{Salten, Felix@\textsc{Salten, Felix}!zzzSchnitzler, Arthur@\emph{von Arthur Schnitzler}!1906-01-302@{30. 1. 1906}|(be}
\toendnotes[C]{\smallbreak\pagebreak[2]}\Standort{Wienbibliothek im Rathaus, ZPH 1681, 2.1.516.}
\physDesc{Brief, 1 Blatt, 3 Seiten, 848 Zeichen
\newline{}Handschrift: schwarze Tinte, deutsche Kurrent
\newline{}Ordnung: mit Bleistift von unbekannter Hand Nummerierung der Blätter des Konvoluts:
                                    »24«–»25« }\toendnotes[C]{\smallbreak}
\pstart
           \noindent{}{\pb}\textcolor{gray}{\textbf{Dr. Arthur Schnitzler}}\hfill 30. 1. 906\pend
           
\pstart
           \textcolor{gray}{\textbf{\textcolor{pink}{Wien, XVIII. Spoettelgasse 7}{}\ledrightnote{\textcolor{pink}{Edmund-Weiß-Gasse 7}}.}}\pend
           
\pstart
           lieber, zum Einzug in \textcolor{pink}{Berlin}{}\ledrightnote{\textcolor{pink}{Berlin}} und
               in die neue \textcolor{pink}{Wohnung}{}\ledrightnote{{$\rightarrow$}\textcolor{pink}{Kantstraße}} wünſchen
               wir Ihnen Alles erdenkliche gute u ſchöne. \label{K_L03002-1v}\edtext{Am 17. etwa}{\lemma{\textnormal{\emph{Am 17. etwa}}}\Cendnote{\textnormal{Die Abreise fand am Abend des
                     16. 2. 1906
                  statt.}}}\label{K_L03002-1h} denken wir nach \textcolor{pink}{Berlin}{}\ledrightnote{\textcolor{pink}{Berlin}} zu
               fahren, wo die \label{K_L03002-2v}\edtext{Pr. des »\textcolor{green}{Ruf}{}\ledrightnote{\textcolor{green}{Der Ruf des Lebens. Schauspiel in drei Akten}}« am 24.}{\lemma{\textnormal{\emph{Pr. des »Ruf« am 24.}}}\Cendnote{\textnormal{Am 24. 2. 1906 fand die deutschsprachige
                  Uraufführung von \emph{\textcolor{green}{Der Ruf des Lebens}} am \textcolor{pink}{Lessing-Theater} statt.}}}\label{K_L03002-2h} ſtattfinden ſoll;
               ſehr möglich aber wär es, daſs ich {\pb}\label{K_L03002-3v}\edtext{um den 5. Feber}{\lemma{\textnormal{\emph{um den 5. Feber}}}\Cendnote{\textnormal{Am 3. 2. 1906 fuhr \textcolor{blue}{Schnitzler} nach \textcolor{pink}{Berlin}, am 5. 2. 1906 und am
                     Folgetag fanden die \textcolor{green}{Arrangierproben} statt. Am 7. 2. 1906 fuhr \textcolor{blue}{Schnitzler} retour.}}}\label{K_L03002-3h} herum auf einige Tage hinfahre, theils zu den
               Arrangirproben, theils zu \label{K_L03002-4v}\edtext{\textcolor{blue}{Brahm}{}\ledrightnote{\textcolor{blue}{Otto Brahm}}s
               fünfzigſtem}{\lemma{\textnormal{\emph{Brahms
               fünfzigſtem}}}\Cendnote{\textnormal{vgl. A. S.: \emph{Tagebuch}, 5. 2. 1906}}}\label{K_L03002-4h}.\pend
           
\pstart
           – Von \label{K_L03002-5v}\edtext{\textcolor{blue}{Bahr}{}\ledrightnote{\textcolor{blue}{Hermann Bahr}} erhielt ich geſtern Nachricht}{\lemma{\textnormal{\emph{Bahr … Nachricht}}}\Cendnote{\textnormal{siehe Hermann Bahr an Arthur Schnitzler, 29. 1. 1906}}}\label{K_L03002-5h}, daſs ihm der \label{K_L03002-6v}\edtext{\textcolor{blue}{Intendant}{}\ledrightnote{{$\rightarrow$}\textcolor{blue}{Albert von Speidel}} die Genehmigung
               zur Annahme}{\lemma{\textnormal{\emph{Intendant … Annahme}}}\Cendnote{\textnormal{\textcolor{blue}{Bahr} war zum Oberregisseur des \emph{\textcolor{brown}{Münchener Hoftheater}}s ernannt worden. Aufgrund
                  von öffentlichem konservativem Gegenwind kam es zur Vertragsauflösung.}}}\label{K_L03002-6h} des
                  »\textcolor{green}{Ruf}{}\ledrightnote{\textcolor{green}{Der Ruf des Lebens. Schauspiel in drei Akten}}« (die er dringend verlangt hatte)
               verweigert hat. Er fügt hinzu: »Es iſt das nur ein Glied in der Kette, {\pb}von kleinen Gemeinheiten, durch welche man
               mich jetzt aus meinem Contract hinausekeln will, was vermuthlich gelingen wird.«
               (bitte das vorläufig als vertraulich zu behandeln, ich meine natürlich gegenüber \textcolor{pink}{Berlin}{}\ledrightnote{\textcolor{pink}{Berlin}}er Bekannten).\pend
           
\pstart
           Wenn ich komme, melde ich mich natürlich gleich. {\\[\baselineskip]}Von Herzen, mit Grüßen von
                  \textcolor{pink}{Spöttel}{}\ledrightnote{\textcolor{pink}{Edmund-Weiß-Gasse 7}} nach \label{K_L03002-7v}\edtext{\textcolor{pink}{Kant}{}\ledrightnote{\textcolor{pink}{Kantstraße}}}{\lemma{\textnormal{\emph{Kant}}}\Cendnote{\textnormal{\textcolor{blue}{Salten} hatte in \textcolor{pink}{Berlin} eine Unterkunft in der \textcolor{pink}{Kantstraße 34} bezogen, vgl. Felix Salten an Arthur Schnitzler, 29. 1. 1906.}}}\label{K_L03002-7h} Ihr \spacefill\mbox{A.}\pend
           \leftskip=0em{}\endnumbering\briefempfaengerindex{Salten, Felix@\textsc{Salten, Felix}!zzzSchnitzler, Arthur@\emph{von Arthur Schnitzler}!1906-01-302@{30. 1. 1906}|)be}\mylabel{h}  \normalsize

\doendnotes{C}
\bigskip
\vfill

\clearpage

\footnotesize

\lohead{\textsc{register}}

% Definiere theindex-Environment komplett neu ohne reledmac
\makeatletter
\renewenvironment{theindex}{%
  \section*{\indexname}%
  \setlength{\parindent}{0pt}%
  \setlength{\parskip}{0pt plus 0.3pt}%
  \let\item\@idxitem
}{%
  \clearpage
}
\makeatother

\IfFileExists{\jobname-pw.ind}{\input{\jobname-pw.ind}}{}

\end{document}

      