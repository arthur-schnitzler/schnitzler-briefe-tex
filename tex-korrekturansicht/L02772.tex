%% latex-korrekturansicht-vorspann.tex
%% Vorspann für die Korrekturansicht.
%% Lädt die gemeinsame Datei latex-vorspann.tex mit gesetztem Schalter.

\newif\ifkorrekturansicht
\korrekturansichttrue

\input{../tex-inputs/latex-vorspann}


               \section[Paul Goldmann an Arthur Schnitzler, Paul Goldmann an Arthur Schnitzler, 29. 4. {[}1896{]}]{ Paul Goldmann an Arthur Schnitzler, 29. 4. {[}1896{]}}\nopagebreak\mylabel{v}\rehead{ }\normalsize\beginnumbering\briefempfaengerindex{Schnitzler, Arthur@\textsc{Schnitzler, Arthur}!zzzGoldmann, Paul@\emph{von Paul Goldmann}!1896-04-291@{29. 4. {[}1896{]}}|(be} \toendnotes[C]{\smallbreak\pagebreak[2]} \Standort{DLA, A:Schnitzler, HS.NZ85.1.3166.}
\physDesc{Brief, 2 Blätter, 8 Seiten
\newline{}Handschrift: blaue Tinte, deutsche Kurrent
\newline{}Schnitzler: 1) mit Bleistift das Jahr »96« vermerkt, sowie »\noindent{}\textcolor{blue}{\textsc{Kerr}}?{ / }\textsc{\textcolor{blue}{Altenb}?}{ / }\textcolor{gray}{Brief}« vermerkt 2) mit rotem Buntstift drei Unterstreichungen}\toendnotes[C]{\smallbreak}\pstart
           \noindent{}{\pb}\textcolor{gray}{\textbf{\textbf{\textcolor{brown}{Frankfurter Zeitung}{}\ledrightnote{\textcolor{brown}{Frankfurter Zeitung}}}}}\pend
           \pstart
           \textcolor{gray}{\textbf{(\textcolor{brown}{\begin{otherlanguage}{french}Gazette de Francfort\end{otherlanguage}}{}\ledrightnote{\textcolor{brown}{Frankfurter Zeitung}}).}}\pend
           \pstart
           \textcolor{gray}{\textbf{\textbf{\begin{otherlanguage}{french}Fondateur M.\end{otherlanguage}{ }\textcolor{blue}{L. Sonnemann}{}\ledrightnote{\textcolor{blue}{Leopold Sonnemann}}.}}}\pend
           \pstart
           \begin{otherlanguage}{french}\textcolor{gray}{\textbf{\textcolor{green}{Journal}{}\ledrightnote{→\textcolor{green}{Frankfurter Zeitung}} politique,
                        financier,}}\end{otherlanguage}\pend
           \pstart
           \begin{otherlanguage}{french}\textcolor{gray}{\textbf{commercial et littéraire.}}\end{otherlanguage}\pend
           \pstart
           \begin{otherlanguage}{french}\textcolor{gray}{\textbf{\textbf{Paraissant trois fois par jour.}}}\end{otherlanguage}\pend
           \pstart
           \begin{otherlanguage}{french}\textcolor{gray}{\textbf{\textbf{Bureau à \textcolor{pink}{Paris}{}\ledrightnote{\textcolor{pink}{Paris}}}}}\end{otherlanguage}\pend
           \pstart
           \begin{otherlanguage}{french}\textcolor{gray}{\textbf{\textbf{\textcolor{pink}{24. Rue Feydeau}{}\ledrightnote{\textcolor{pink}{rue Feydeau}}.}}}\end{otherlanguage}\hfill \textsc{\textcolor{pink}{Paris}{}\ledrightnote{\textcolor{pink}{Paris}}}, 29. April.\pend
           \pstart\center{}Mein lieber Freund,\pend\pstart
           Ich war 14 Tage in \textcolor{pink}{Frankfurt}{}\ledrightnote{\textcolor{pink}{Frankfurt am Main}}, habe geruht und
               neue Kräfte zu gewinnen geſtrebt. Nöthig wars. Zur Feier meiner Rückkunft fand eine
               feſtliche \label{K_L02772-1v}\edtext{\textcolor{blue}{Miniſter}{}\ledrightnote{→\textcolor{blue}{Léon Bourgeois}}kriſis}{\lemma{\textnormal{\emph{Miniſterkriſis}}}\Cendnote{\textnormal{Mit dem 29. 4. 1896 endete das Ministerium von \textcolor{blue}{Léon Bourgeois}.}}}\label{K_L02772-1h} ſtatt. Ich ſtecke bis über die
               Ohren in Arbeit, und ſo komme ich erſt heut dazu, Dir
               für Deinen ſo überaus lieben Brief zu danken, den ich noch in \textcolor{pink}{Frankfurt}{}\ledrightnote{\textcolor{pink}{Frankfurt am Main}} empfing. Als ich in \textcolor{pink}{Frankfurt}{}\ledrightnote{\textcolor{pink}{Frankfurt am Main}} war, wurde gerade dein \textcolor{green}{Stück}{}\ledrightnote{→\textcolor{green}{Liebelei. Schauspiel in drei Akten}} in \textcolor{pink}{Köln}{}\ledrightnote{\textcolor{pink}{Köln}} aufgeführt,
               und in der {\pb}\textcolor{green}{Frankf. Zeit.}{}\ledrightnote{\textcolor{green}{Frankfurter Zeitung}} erſchien eine kleine \label{K_L02772-2v}\edtext{\textcolor{green}{Beſprechung}{}\ledrightnote{→\textcolor{green}{[Man schreibt uns aus Köln]}}}{\lemma{\textnormal{\emph{Beſprechung}}}\Cendnote{\textnormal{\emph{\textcolor{green}{[Man schreibt uns aus Köln]}}. In: \emph{\textcolor{green}{Frankfurter Zeitung}}, Jg. 40, Nr. 103,
                        13. 4. 1896, Abendblatt, S. 2.}}}\label{K_L02772-2h}, die ich hier
               einfüge, da Du ſie vielleicht überſehen haſt.\pend
           {\bigskip}\pstart
           \noindent{}\textcolor{gray}{\textbf{Man ſchreibt uns aus \textcolor{pink}{\so{Köln}}{}\ledrightnote{\textcolor{pink}{Köln}}, 11. April: Schnitzler’s \textcolor{green}{Schauſpiel}{}\ledrightnote{→\textcolor{green}{Liebelei. Schauspiel in drei Akten}} »\textcolor{green}{\so{Liebelei}}{}\ledrightnote{\textcolor{green}{Liebelei. Schauspiel in drei Akten}}« ging geſtern zum erſten Mal in Szene und erzielte einen ſehr
                  ſtarken Erfolg. Die Mitwirkenden wurden nach dem letzten \textcolor{green}{Akt}{}\ledrightnote{→\textcolor{green}{Liebelei. Schauspiel in drei Akten}} fünfmal gerufen. Die Darſtellung war
                  im Ganzen recht befriedigend. Die \textcolor{green}{Chriſtine}{}\ledrightnote{→\textcolor{green}{Liebelei. Schauspiel in drei Akten}} wußte Frau \textcolor{blue}{Doré}{}\ledrightnote{\textcolor{blue}{Adele Doré}} in ergreifender Weise zu geſtalten. In der \textcolor{green}{Mizi}{}\ledrightnote{→\textcolor{green}{Liebelei. Schauspiel in drei Akten}} des Frl. \textcolor{blue}{\so{Glümer}}{}\ledrightnote{\textcolor{blue}{Marie Glümer}} und in dem \textcolor{green}{Theodor}{}\ledrightnote{→\textcolor{green}{Liebelei. Schauspiel in drei Akten}}
                  des Hrn. \textcolor{blue}{\so{Leyrer}}{}\ledrightnote{\textcolor{blue}{Rudolf Leyrer}} fand die \textcolor{pink}{Wien}{}\ledrightnote{\textcolor{pink}{Wien}}er Leichtlebigkeit ihre
                  angemeſſene Vertretung. Fein und discret gab Herr \textcolor{blue}{\so{Beck}}{}\ledrightnote{\textcolor{blue}{Beck}} den \textcolor{green}{alten Musiker}{}\ledrightnote{→\textcolor{green}{Liebelei. Schauspiel in drei Akten}};
                  auch der \textcolor{green}{Fritz}{}\ledrightnote{→\textcolor{green}{Liebelei. Schauspiel in drei Akten}} des Hrn. \textcolor{blue}{\so{Monnard}}{}\ledrightnote{\textcolor{blue}{Heinz Monnard}} war nicht ohne tiefere Wirkung. –}}\pend
           {\bigskip}\pstart
           \noindent{}auch lege ich einen \label{K_L02772-3v}\edtext{Brief}{\lemma{\textnormal{\emph{Brief}}}\Cendnote{\textnormal{\textcolor{blue}{Goldmann} vergaß, ihn beizulegen, siehe Paul Goldmann an Arthur Schnitzler, 3. 4. [1895]}}}\label{K_L02772-3h} des Herrn \textsc{\textcolor{blue}{Christian Schefer}{}\ledrightnote{\textcolor{blue}{Christian Schefer}}} bei, den ich noch in \textcolor{pink}{Frankfurt}{}\ledrightnote{\textcolor{pink}{Frankfurt am Main}} erhielt.
               Schicke ihm ein Exemplar von »\textsc{\textcolor{green}{Mourir}{}\ledrightnote{\textcolor{green}{Mourir. Roman}}}«, ebenſo eines an \textsc{\textcolor{blue}{Lalo}{}\ledrightnote{\textcolor{blue}{Pierre Lalo}}}, ein drittes an \textsc{M. \textcolor{blue}{de
                     Wyzewa}{}\ledrightnote{\textcolor{blue}{Théodore de Wyzewa}}, \textcolor{pink}{9. Rue Coëtlogon}{}\ledrightnote{\textcolor{pink}{Rue Coëtlogon}}}. Auch ſchicke mir noch zwei oder {\pb}drei \substVorne{}\textsuperscript{\textcolor{gray}{Büche}}\substDazwischen{}\textcolor{green}{Exemplare}{}\ledrightnote{→\textcolor{green}{Mourir. Roman}}\substHinten{} zur Propaganda. Das \textcolor{green}{Buch}{}\ledrightnote{→\textcolor{green}{Mourir. Roman}} iſt ſehr gut ausgeſtattet und ſieht recht vornehm aus. Ferner ſende ich
               Dir die Briefe des Herrn \textsc{\textcolor{blue}{de Riaz}{}\ledrightnote{\textcolor{blue}{Henri de Riaz}}} zurück. Laß’ die \label{K_L02772-99v}\edtext{\textcolor{green}{Überſetzung}{}\ledrightnote{→\textcolor{green}{Liebelei. Schauspiel in drei Akten}}s-Angelegenheit}{\lemma{\textnormal{\emph{Überſetzungs-Angelegenheit}}}\Cendnote{\textnormal{siehe Paul Goldmann an Arthur Schnitzler, 5. 12. [1895]}}}\label{K_L02772-99h} noch ruhn und antworte aufſchiebend. Endlich finde ich noch in meinen
               Papieren die \label{K_L02772-4v}\edtext{\textcolor{green}{Kritik}{}\ledrightnote{→\textcolor{green}{Burgtheater [Rechte der Seele, Liebelei]}}}{\lemma{\textnormal{\emph{Kritik}}}\Cendnote{\textnormal{\textcolor{blue}{Alfred Freiherr von Berger}: \emph{\textcolor{green}{Burgtheater}}. In: \emph{\textcolor{green}{Montags-Revue}}, Jg. XXXX, Nr. XXXX, 14. 10. 1895, S. XXXX.}}}\label{K_L02772-4h} des Baron \textcolor{blue}{\textsc{Berger}}{}\ledrightnote{\textcolor{blue}{Alfred von Berger}}, die ich Dir
               gleichfalls zurückſende.\pend
           \pstart
           Zu erzählen habe ich Dir nichts. Mein Leben iſt vollſtändig unintereſſant. Es gibt
               nichts Neues und wird nie etwas Neues geben, {\pb}außer
               irgend einem definitiven Unglück. Intereſſant iſt nur Dein Leben, und ich möchte ſehr
               viel darüber wiſſen. Haſt Du alſo \label{K_L02772-5v}\edtext{zum
               dritten Mal angeſangen, das \textcolor{green}{Stück}{}\ledrightnote{→\textcolor{green}{Freiwild. Schauspiel in 3 Akten}} zu ſchreiben}{\lemma{\textnormal{\emph{zum … ſchreiben}}}\Cendnote{\textnormal{siehe A. S.: \emph{Tagebuch}, 27. 4. 1896}}}\label{K_L02772-5h}? Könnte man nicht doch das \textcolor{green}{Manuſcript}{}\ledrightnote{→\textcolor{green}{Freiwild. Schauspiel in 3 Akten}} ſehen? Wirſt Du \label{K_L02772-6v}\edtext{in
               die »\textcolor{brown}{Zeit}{}\ledrightnote{\textcolor{brown}{Die Zeit. Wiener Wochenschrift}}« eintreten}{\lemma{\textnormal{\emph{in
               die »Zeit« eintreten}}}\Cendnote{\textnormal{nicht geschehen}}}\label{K_L02772-6h}, jetzt nach \textsc{\textcolor{blue}{Kanner}{}\ledrightnote{\textcolor{blue}{Heinrich Kanner}}s} Rückkehr? Und wie iſt ſonſt
               Daſeinsführung und Stimmung?\pend
           \pstart
           Recht geärgert habe an mich, als ich Deinen {\pb}\label{K_L02772-7v}\edtext{Namen im »\textcolor{green}{\textsc{Simplicissimus}}{}\ledrightnote{\textcolor{green}{Simplicissimus}}«}{\lemma{\textnormal{\emph{Namen im »Simplicissimus«}}}\Cendnote{\textnormal{\textcolor{blue}{Arthur Schnitzler}: \emph{\textcolor{green}{Die überspannte Person}}. In: \emph{\textcolor{green}{Simplicissimus}}, Jg. 1, H. 3, 18. 4. 1896, S. 3 u. 6.}}}\label{K_L02772-7h} fand. Dieſer Lausbub’ \textsc{\textcolor{blue}{Langen}{}\ledrightnote{\textcolor{blue}{Albert Langen}}}, der mir i\substVorne{}\textsuperscript{m}\substDazwischen{}n\substHinten{}{ }\textsc{\textcolor{pink}{Paris}{}\ledrightnote{\textcolor{pink}{Paris}}}, wenn ich ihn dazu drängte, Deine Bücher in Verlag zu nehmen, ſtets antwortete:
               Du könnteſt \label{K_L02772-88v}\edtext{nicht deutſch
                  ſchreiben}{\lemma{\textnormal{\emph{nicht deutſch
                  ſchreiben}}}\Cendnote{\textnormal{eventuell auf die
                  Verwendung von Austriazismen gemünzt?}}}\label{K_L02772-88h}, – iſt jetzt in der Lage, ſein neues
                  \textcolor{brown}{Unternehmen}{}\ledrightnote{→\textcolor{brown}{Simplicissimus}} mit Deinem
               jungen Rénommée aufzuputzen. Das hat er wahrlich nicht verdient. Warum haſt {\pb}Du ihm den \textcolor{green}{Beitrag}{}\ledrightnote{→\textcolor{green}{Die überspannte Person}} gegeben\damage{?} Ich bekam in \textcolor{pink}{Deutſchland}{}\ledrightnote{\textcolor{pink}{Deutschland}} durch Zufall
               das \textcolor{green}{Heft}{}\ledrightnote{→\textcolor{green}{Die Zukunft}} der »\textcolor{green}{Zukunft}{}\ledrightnote{\textcolor{green}{Die Zukunft}}« in die Hand, das \label{K_L02772-8v}\edtext{\textsc{\textcolor{blue}{Harden}{}\ledrightnote{\textcolor{blue}{Maximilian Harden}}s}{ }\textcolor{green}{Kritik}{}\ledrightnote{→\textcolor{green}{Theaternotizen [Liebelei]}} über »\textcolor{green}{Liebelei}{}\ledrightnote{\textcolor{green}{Liebelei. Schauspiel in drei Akten}}«}{\lemma{\textnormal{\emph{Hardens … »Liebelei«}}}\Cendnote{\textnormal{\textcolor{blue}{Maximilian Harden}: \emph{\textcolor{green}{Theaternotizen}}. In: \emph{\textcolor{green}{Die
                        Zukunft}}, Jg. 5, Bd. 14, 14. 3. 1896,
                     S. 527–528.}}}\label{K_L02772-8h} enthält. Das iſt doch eine recht unverſtändige \textcolor{green}{Kritik}{}\ledrightnote{\textcolor{green}{Theaternotizen [Liebelei]}}, die Dich völlig unterſchätzt. Biſt Du
               trotzdem bei Deiner großen Meinung über \textsc{\textcolor{blue}{Harden}{}\ledrightnote{\textcolor{blue}{Maximilian Harden}}} geblieben?\pend
           \pstart
           Aber ich will nicht fragen, und Du ſollſt den \strikeout{Ih\textcolor{gray}{a}} Inhalt des nächſten Briefes nach {\pb}\damage{freier} Wahl zuſammen\damage{th}un. Schreib’ mir nur recht viel über Dich.\pend
           \pstart
           Und wie gehts dem \textsc{\textcolor{blue}{Richard}{}\ledrightnote{\textcolor{blue}{Richard Beer-Hofmann}}}? Er bringts wirklich fertig, mir keine Zeile zu ſchreiben. Erwartet hab’ ichs,
               aber es erſtaunt mich doch. Es iſt immerhin der ſchönſte Fall von Faulheit, der mir
               in meinem Leben vorgekommen iſt.\pend
           \pstart
           Gern ginge ich mit früh im August{ }{\pb}nach \label{K_L02772-9v}\edtext{\textcolor{pink}{Dänemark}{}\ledrightnote{\textcolor{pink}{Dänemark}}}{\lemma{\textnormal{\emph{Dänemark}}}\Cendnote{\textnormal{Von 5. 8. 1896 bis 21. 8. 1896 waren \textcolor{blue}{Schnitzler},
                     \textcolor{blue}{Goldmann}, \textcolor{blue}{Richard} und \textcolor{blue}{Paula
                     Beer-Hofmann} gemeinsam in \textcolor{pink}{Skodsborg}.
               }}}\label{K_L02772-9h}, w\damage{enn} ich Geld hätte, w\damage{as} noch zweifelhaft iſt. Ich würde dann \label{K_L02772-11v}\edtext{über \textcolor{pink}{Berlin}{}\ledrightnote{\textcolor{pink}{Berlin}}
                  zurückreiſen}{\lemma{\textnormal{\emph{über Berlin
                  zurückreiſen}}}\Cendnote{\textnormal{siehe A. S.: \emph{Tagebuch}, 26. 8. 1896}}}\label{K_L02772-11h}, wo mich meine \textcolor{blue}{Mutter}{}\ledrightnote{→\textcolor{blue}{Clementine Goldmann}} und mein \textcolor{blue}{Onkel}{}\ledrightnote{→\textcolor{blue}{Hermann Mamroth}} erwarten.\pend
           \pstart
           Grüß’ Dich Gott, mein lieber Freund, und ſchreib’ mir bald!\pend
           \pstart
           Dein treuer {\\[\baselineskip]}\spacefill\mbox{Paul Goldmann}\pend
           \leftskip=0em{}\endnumbering\briefempfaengerindex{Schnitzler, Arthur@\textsc{Schnitzler, Arthur}!zzzGoldmann, Paul@\emph{von Paul Goldmann}!1896-04-291@{29. 4. {[}1896{]}}|)be}\mylabel{h}  \normalsize

\doendnotes{C}
\bigskip
\vfill

\clearpage

\footnotesize

\lohead{\textsc{register}}

% Definiere theindex-Environment komplett neu ohne reledmac
\makeatletter
\renewenvironment{theindex}{%
  \section*{\indexname}%
  \setlength{\parindent}{0pt}%
  \setlength{\parskip}{0pt plus 0.3pt}%
  \let\item\@idxitem
}{%
  \clearpage
}
\makeatother

\IfFileExists{\jobname-pw.ind}{\input{\jobname-pw.ind}}{}

\end{document}

      