%% latex-korrekturansicht-vorspann.tex
%% Vorspann für die Korrekturansicht.
%% Lädt die gemeinsame Datei latex-vorspann.tex mit gesetztem Schalter.

\newif\ifkorrekturansicht
\korrekturansichttrue

\input{../tex-inputs/latex-vorspann}


\renewcommand{\erwaehntePersonen}{Personen: Elvira Leontine Hervay von Kirchberg, Anna Loew, Hansi Niese, Felix Salten, Adele Sandrock, Olga Schnitzler}
\renewcommand{\erwaehnteOrte}{Orte: IX., Alsergrund, Porzellangasse, VIII., Josefstadt, Wien}
\renewcommand{\erwaehnteWerke}{}
\section[ Arthur Schnitzler an Felix Salten, 12. 1. 1905]{Arthur Schnitzler an Felix Salten, 12. 1. 1905}
\nopagebreak\mylabel{v}
\rehead{ }\normalsize\beginnumbering\briefempfaengerindex{Salten, Felix@\textsc{Salten, Felix}!zzzSchnitzler, Arthur@\emph{von Arthur Schnitzler}!1905-01-121@{12. 1. 1905}|(be}
\toendnotes[C]{\smallbreak\pagebreak[2]}\Standort{Wienbibliothek im Rathaus, ZPH 1681, 2.1.516.}
\physDesc{Kartenbrief, 364 Zeichen
\newline{}Handschrift: schwarze Tinte, deutsche Kurrent
\newline{}Versand: Stempel: »\nobreak{}\oindex{VIII., Josefstadt@\textbf{VIII., Josefstadt}, \emph{A.ADM3}|pwk}18/1 Wien 110, 12. II. 16, XI\nobreak{}«.  
\newline{}Ordnung: mit Bleistift von unbekannter Hand Nummerierung der Blätter des Konvoluts:
                                    »19« }\toendnotes[C]{\smallbreak}\pstart{}{\pb}\textsc{Herrn Felix Salten}\pend{}\pstart{}\textcolor{pink}{Wien IX}{}\ledrightnote{\textcolor{pink}{IX., Alsergrund}}\pend{}\pstart{}\textsc{\textcolor{pink}{Porzellang. 45}{}\ledrightnote{\textcolor{pink}{Porzellangasse}}.}\pend{}
{\bigskip}
\pstart
           \raggedleft{}{\pb}\label{K_L02996-1v}\edtext{12. 1. 905}{\lemma{\textnormal{\emph{12. 1. 905}}}\Cendnote{\textnormal{Der Poststempel war demnach völlig
                     verdreht.}}}\label{K_L02996-1h}\pend
           
\pstart
           lieber, herzlichen Dank für Ihren \label{K_L02996-2v}\edtext{Brief}{\lemma{\textnormal{\emph{Brief}}}\Cendnote{\textnormal{Felix Salten an Arthur Schnitzler, 11. 1. 1905}}}\label{K_L02996-2h}. Ich habe der \textsc{\textcolor{blue}{Hervay}{}\ledrightnote{\textcolor{blue}{Elvira Leontine Hervay von Kirchberg}}} ſelbſt geſchrieben, ganz ehrlich, die Gründe; warum ich überhaupt, nicht nur
               für ſie, nicht leſe.\pend
           
\pstart
           Die \textsc{\textcolor{blue}{S.}{}\ledrightnote{\textcolor{blue}{Adele Sandrock}}} ſchrieb mir geſtern, dſs die \textcolor{blue}{Nieſe}{}\ledrightnote{\textcolor{blue}{Hansi Niese}} auch mit Freuden zugeſagt habe.\pend
           
\pstart
           Auf bald. Bei uns im Hauſe lag oder liegt alles; jetzt die \textcolor{blue}{Kinderfrau}{}\ledrightnote{{$\rightarrow$}\textcolor{blue}{Anna Loew}} und \textcolor{blue}{Olga}{}\ledrightnote{\textcolor{blue}{Olga Schnitzler}}.\pend
           
\pstart
           Herzlichſt Ihr {\\[\baselineskip]}\spacefill\mbox{A.}\pend
           \leftskip=0em{}\endnumbering\briefempfaengerindex{Salten, Felix@\textsc{Salten, Felix}!zzzSchnitzler, Arthur@\emph{von Arthur Schnitzler}!1905-01-121@{12. 1. 1905}|)be}\mylabel{h}  \normalsize

\doendnotes{C}
\bigskip
\vfill

\clearpage

\footnotesize

\lohead{\textsc{register}}

% Definiere theindex-Environment komplett neu ohne reledmac
\makeatletter
\renewenvironment{theindex}{%
  \section*{\indexname}%
  \setlength{\parindent}{0pt}%
  \setlength{\parskip}{0pt plus 0.3pt}%
  \let\item\@idxitem
}{%
  \clearpage
}
\makeatother

\IfFileExists{\jobname-pw.ind}{\input{\jobname-pw.ind}}{}

\end{document}

      