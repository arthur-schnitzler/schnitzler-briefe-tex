%% latex-korrekturansicht-vorspann.tex
%% Vorspann für die Korrekturansicht.
%% Lädt die gemeinsame Datei latex-vorspann.tex mit gesetztem Schalter.

\newif\ifkorrekturansicht
\korrekturansichttrue

\input{../tex-inputs/latex-vorspann}


\renewcommand{\erwaehntePersonen}{Personen: Felix Salten, Michael Emil Salzmann, Marie Salzmann, Olga Schnitzler}
\renewcommand{\erwaehnteOrte}{Orte: Edmund-Weiß-Gasse 7, Seis am Schlern, Wien}
\renewcommand{\erwaehnteWerke}{}
\section[ Arthur Schnitzler an Felix Salten, 29. 6. 1908]{Arthur Schnitzler an Felix Salten, 29. 6. 1908}
\nopagebreak\mylabel{v}
\rehead{ }\normalsize\beginnumbering\briefempfaengerindex{Salten, Felix@\textsc{Salten, Felix}!zzzSchnitzler, Arthur@\emph{von Arthur Schnitzler}!1908-06-292@{29. 6. 1908}|(be}
\toendnotes[C]{\smallbreak\pagebreak[2]}\Standort{Wienbibliothek im Rathaus, ZPH 1681, 2.1.516.}
\physDesc{Karte, 527 Zeichen
\newline{}Handschrift: schwarze Tinte, deutsche Kurrent
\newline{}Ordnung: mit Bleistift von unbekannter Hand nummeriert: »5« }\toendnotes[C]{\smallbreak}
\pstart
           \noindent{}{\pb}\textcolor{gray}{\textbf{Dr. Arthur Schnitzler}}\hfill am 29. Juni 908\pend
           
\pstart
           \textcolor{gray}{\textbf{\textcolor{pink}{Wien XVIII. Spoettelgasse 7}{}\ledrightnote{\textcolor{pink}{Edmund-Weiß-Gasse 7}}.}}\hfill \textsc{\textcolor{pink}{Seis am Schlern}{}\ledrightnote{\textcolor{pink}{Seis am Schlern}}}.\pend
           
\pstart
           lieber, ich leſe eben, daſs Ihr \label{K_L03014-1v}\edtext{\textcolor{blue}{Bruder}{}\ledrightnote{{$\rightarrow$}\textcolor{blue}{Michael Emil Salzmann}} geſtorben}{\lemma{\textnormal{\emph{Bruder geſtorben}}}\Cendnote{\textnormal{\textcolor{blue}{(Michael) Emil Salzmann} starb am 26. 6. 1908 an einer Neurasthenie. Er war
                  das älteste Geschwister und die wichtigste familiäre Bezugsperson \textcolor{blue}{Salten}s. Er hatte zeitlebens unverheiratet bei der
                     \textcolor{blue}{Mutter} gelebt.}}}\label{K_L03014-1h} iſt,
               und bin um ſo tiefer ergriffen, als ich nicht wußte, daſs ſein Befinden ſich in der
               letzten Zeit verſchli{\geminationm}ert hatte. Glauben Sie mir, daſs
               ich an Ihrem Schmerze den herzlichſten Antheil nehme und ſagen Sie es auch den
               Ihrigen, vor allem Ihrer \textcolor{blue}{Mutter}{}\ledrightnote{{$\rightarrow$}\textcolor{blue}{Marie Salzmann}}, wie ſehr ich das {\pb}frühe Ende
               dieſes liebenswerthen \textcolor{blue}{Menſchen}{}\ledrightnote{{$\rightarrow$}\textcolor{blue}{Michael Emil Salzmann}} beklage. Auch \textcolor{blue}{Olga}{}\ledrightnote{\textcolor{blue}{Olga Schnitzler}} bittet Sie
               ihres Mitgefühls verſichert zu ſein. \textcolor{blue}{Wir}{}\ledrightnote{{$\rightarrow$}\textcolor{blue}{Olga Schnitzler}} grüßen vielmals und hoffen baldmöglichſt
                  \textcolor{gray}{wieder} von Ihnen zu hören.\pend
           
\pstart
           Ihr {\\[\baselineskip]}\spacefill\mbox{Arthur}\pend
           \leftskip=0em{}\endnumbering\briefempfaengerindex{Salten, Felix@\textsc{Salten, Felix}!zzzSchnitzler, Arthur@\emph{von Arthur Schnitzler}!1908-06-292@{29. 6. 1908}|)be}\mylabel{h}  \normalsize

\doendnotes{C}
\bigskip
\vfill

\clearpage

\footnotesize

\lohead{\textsc{register}}

% Definiere theindex-Environment komplett neu ohne reledmac
\makeatletter
\renewenvironment{theindex}{%
  \section*{\indexname}%
  \setlength{\parindent}{0pt}%
  \setlength{\parskip}{0pt plus 0.3pt}%
  \let\item\@idxitem
}{%
  \clearpage
}
\makeatother

\IfFileExists{\jobname-pw.ind}{\input{\jobname-pw.ind}}{}

\end{document}

      