%% latex-korrekturansicht-vorspann.tex
%% Vorspann für die Korrekturansicht.
%% Lädt die gemeinsame Datei latex-vorspann.tex mit gesetztem Schalter.

\newif\ifkorrekturansicht
\korrekturansichttrue

\input{../tex-inputs/latex-vorspann}


               \section[Olga Schnitzler an Richard Beer-Hofmann, {[}25. 10. 1907{]}]{ Olga Schnitzler an Richard Beer-Hofmann,
               {[}25. 10. 1907{]}}\nopagebreak\mylabel{v}\rehead{ }\normalsize\beginnumbering\briefempfaengerindex{Beer-Hofmann, Richard@\textsc{Beer-Hofmann, Richard}!zzzSchnitzler, Olga@\emph{von Olga Schnitzler}!1907-10-251@{{[}25. 10. 1907{]}}|(be} \toendnotes[C]{\smallbreak\pagebreak[2]} \Standort{YCGL, MSS 31.}
\physDesc{Briefkarte, Umschlag
\newline{}Handschrift: schwarze Tinte, lateinische Kurrent\newline{}Versand: ohne postalischen Übermittlungsvermerk }\toendnotes[C]{\smallbreak}\pstart{}{\pb}\textcolor{gray}{\textbf{O. S.}}\pend{}{\bigskip}\pstart{}{\pb}Herrn D\textsuperscript{r} Richard
                  Beer-Hofmann \pend{}{\bigskip}\pstart
           \noindent{}{\pb}\textcolor{gray}{\textbf{O. S.}}\pend
           \pstart
           Freitag.\pend
           \pstart
           Lieber Herr Doctor, \label{K_L01726_1v}\edtext{Montag}{\lemma{\textnormal{\emph{Montag}}}\Cendnote{\textnormal{vgl. A. S.: \emph{Tagebuch}, 28. 10. 1907}}}\label{K_L01726_1h}{ }Abend kommen \textcolor{blue}{Saltens}{}\ledrightnote{\textcolor{blue}{Felix Salten}{\newline}\textcolor{blue}{Ottilie Salten}} zum
               Nachtmal, – es wäre sehr lieb, wenn Sie \textcolor{blue}{Beide}{}\ledrightnote{→\textcolor{blue}{Paula Beer-Hofmann}}
               auch {\pb}kommen wollten. Bitte um Antwort. Morgen wollen
               wir bis Montag Früh auf den \textcolor{pink}{Semmering}{}\ledrightnote{\textcolor{pink}{Semmering}}, nächste Woche
               beginnen meine Stunden bei \textcolor{blue}{Ress}{}\ledrightnote{\textcolor{blue}{Johann Ress}}.\pend
           \pstart
           Die herzlichsten Grüsse!{\\[\baselineskip]}\spacefill\mbox{OlgaSch.}\pend
           \leftskip=0em{}\endnumbering\briefempfaengerindex{Beer-Hofmann, Richard@\textsc{Beer-Hofmann, Richard}!zzzSchnitzler, Olga@\emph{von Olga Schnitzler}!1907-10-251@{{[}25. 10. 1907{]}}|)be}\mylabel{h}  \normalsize

\doendnotes{C}
\bigskip
\vfill

\clearpage

\footnotesize

\lohead{\textsc{register}}

% Definiere theindex-Environment komplett neu ohne reledmac
\makeatletter
\renewenvironment{theindex}{%
  \section*{\indexname}%
  \setlength{\parindent}{0pt}%
  \setlength{\parskip}{0pt plus 0.3pt}%
  \let\item\@idxitem
}{%
  \clearpage
}
\makeatother

\IfFileExists{\jobname-pw.ind}{\input{\jobname-pw.ind}}{}

\end{document}

      