%% latex-korrekturansicht-vorspann.tex
%% Vorspann für die Korrekturansicht.
%% Lädt die gemeinsame Datei latex-vorspann.tex mit gesetztem Schalter.

\newif\ifkorrekturansicht
\korrekturansichttrue

\input{../tex-inputs/latex-vorspann}


\renewcommand{\erwaehntePersonen}{Personen: Richard Beer-Hofmann, Gabriele D’Annunzio,  Frimmel, Ludwig Ganghofer, Wilhelm von Scholz}
\renewcommand{\erwaehnteInstitutionen}{Institutionen: Residenztheater München}
\renewcommand{\erwaehnteOrte}{Orte: Berlin, Friedrichstraße, Haupttelegrafenamt, Hotel Savoy, München, Residenztheater München}
\renewcommand{\erwaehnteWerke}{Werke: Der grüne Kakadu. Groteske in einem Akt, Mein Fürst, Traum eines Frühlingsmorgens}
\section[Ludwig Ganghofer an Arthur Schnitzler, 30. 4. {[}1899{]}]{Ludwig Ganghofer an Arthur Schnitzler, 30. 4. {[}1899{]}}
\nopagebreak\mylabel{v}
\rehead{ }\normalsize\beginnumbering\briefempfaengerindex{Schnitzler, Arthur@\textsc{Schnitzler, Arthur}!zzzGanghofer, Ludwig@\emph{von Ludwig Ganghofer}!1899-04-301@{30. 4. {[}1899{]}}|(be}
\toendnotes[C]{\smallbreak\pagebreak[2]}\Standort{CUL, Schnitzler, B 775.}
\physDesc{Telegramm, 373 Zeichen (Vordruck: »\textcolor{gray}{\textbf{\textbf{\textsc{\textcolor{pink}{Berlin},}}{ }\textcolor{pink}{Haupt-Telegraphenamt}}}.«)
\newline{}maschinell
\newline{}Versand: 1) mit Bleistift rückseitiger Vermerk: »\noindent{}{\pb}Adrſ. wohnt \textcolor{pink}{Savoy-Hôtel}{ }\textcolor{pink}{Friedrichſtr}{ / }Bote \textcolor{blue}{Frimmel}«  2) mit rotem Buntstift vier Unterstreichungen und »\textcolor{gray}{K}« für \textcolor{green}{Kakadu}?}\toendnotes[C]{\smallbreak}
\pstart
           \noindent{}\centering{}{\pb}fr \textcolor{pink}{muenchen}{}\ledrightnote{\textcolor{pink}{München}} tel 55 30/4{ }9 m =\pend
           
\pstart
           kann jhnen zu meiner freude mitteilen dass \textcolor{green}{gruener
                  kakadu}{}\ledrightnote{\textcolor{green}{Der grüne Kakadu. Groteske in einem Akt}}{ }gestern{ }abend bei wirklich musterhafter \label{K_L03543-1v}\edtext{auffuehrung}{\lemma{\textnormal{\emph{auffuehrung}}}\Cendnote{\textnormal{Am
                     29. 4. 1899 hatten am \textcolor{pink}{Residenztheater} in \textcolor{pink}{München} die Premieren von \emph{\textcolor{green}{Traum eines
                     Frühlingsmorgens}} von \textcolor{blue}{Gabriele
                     d’Annunzio}, \emph{\textcolor{green}{Mein Fürst}} von \textcolor{blue}{Wilhelm von Scholz} und \textcolor{blue}{Schnitzler}s \emph{\textcolor{green}{Der grüne
                     Kakadu}} stattgefunden.}}}\label{K_L03543-1h} durch die ersten kraefte der \textcolor{brown}{hofbuehne}{}\ledrightnote{\textcolor{brown}{Residenztheater München}} einen so stuermischen erfolg errang wie ihn das \textcolor{pink}{residenztheater}{}\ledrightnote{\textcolor{pink}{Residenztheater München}} seit jahren nicht erlebte. nach
               schluss des \textcolor{green}{stueck}{}\ledrightnote{{$\rightarrow$}\textcolor{green}{Der grüne Kakadu. Groteske in einem Akt}}es wurden
               die darsteller ein dutzend mal hervorgejubelt mit bestem gruss =\pend
           \pstart \spacefill\mbox{ludwig ganghofer .+}\pend{}\endnumbering\briefempfaengerindex{Schnitzler, Arthur@\textsc{Schnitzler, Arthur}!zzzGanghofer, Ludwig@\emph{von Ludwig Ganghofer}!1899-04-301@{30. 4. {[}1899{]}}|)be}\mylabel{h}  \normalsize

\doendnotes{C}
\bigskip
\vfill

\clearpage

\footnotesize

\lohead{\textsc{register}}

% Definiere theindex-Environment komplett neu ohne reledmac
\makeatletter
\renewenvironment{theindex}{%
  \section*{\indexname}%
  \setlength{\parindent}{0pt}%
  \setlength{\parskip}{0pt plus 0.3pt}%
  \let\item\@idxitem
}{%
  \clearpage
}
\makeatother

\IfFileExists{\jobname-pw.ind}{\input{\jobname-pw.ind}}{}

\end{document}

      