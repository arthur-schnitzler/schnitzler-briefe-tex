%% latex-korrekturansicht-vorspann.tex
%% Vorspann für die Korrekturansicht.
%% Lädt die gemeinsame Datei latex-vorspann.tex mit gesetztem Schalter.

\newif\ifkorrekturansicht
\korrekturansichttrue

\input{../tex-inputs/latex-vorspann}


               \section[ Paul Goldmann an Arthur Schnitzler, 16. 10. {[}1898{]}]{Paul Goldmann an Arthur Schnitzler, 16. 10. {[}1898{]}}\nopagebreak\mylabel{v}\rehead{ }\normalsize\beginnumbering\briefempfaengerindex{Schnitzler, Arthur@\textsc{Schnitzler, Arthur}!zzzGoldmann, Paul@\emph{von Paul Goldmann}!1898-10-161@{16. 10. {[}1898{]}}|(be} \toendnotes[C]{\smallbreak\pagebreak[2]} \Standort{DLA, A:Schnitzler, HS.NZ85.1.3168.}
\physDesc{Brief, 2 Blätter, 8 Seiten
\newline{}Handschrift: blaue Tinte, deutsche Kurrent
\newline{}Schnitzler: mit Bleistift das Jahr »98« vermerkt }\toendnotes[C]{\smallbreak}\pstart
           \raggedleft{}{\pb}\textsc{\textcolor{pink}{Peking}{}\ledrightnote{\textcolor{pink}{Peking}}}, 16. Oktober.\pend
           \pstart\center{}Mein lieber Freund,\pend\pstart
           Alle Deine Karten von unterwegs, ſowie Deinen lieben Brief aus \label{K_L02861-1v}\edtext{\textsc{\textcolor{pink}{Luzern}{}\ledrightnote{\textcolor{pink}{Luzern}}}}{\lemma{\textnormal{\emph{Luzern}}}\Cendnote{\textnormal{\textcolor{blue}{Schnitzler} hielt sich zwischen 21. 8. 1898 und 28. 8. 1898 in \textcolor{pink}{Luzern} auf. Siehe zu \textcolor{blue}{Schnitzler}s Sommerreise im Jahr 1898 auch Paul Goldmann an Arthur Schnitzler, 16. 5. 1898.}}}\label{K_L02861-1h} habe
               ich erhalten.\pend
           \pstart
           Ich freue mich, zu erfahren, daß der Sommer ſo angenehm für Dich verlaufen iſt.
               Hoffentlich bleibt von dem guten Reſultat etwas für den Winter zurück. \strikeout{\textcolor{gray}{×}} Ich kann Dich immer nur wieder darauf hinweiſen: Wenn Alles, was Dich quält,
               ſich auf Reiſen ſo ganz verliert, kann es doch unmöglich materielle Geſtalt haben. Im
               Übrigen hoffe ich, daß die viele Arbeit, die Du vorhaſt, ein gutes Heilmittel gegen
               die Hypochondrie ſein wird. Schon aus dem Grunde {\pb}bin ich ſehr froh über alle Deine neuen Pläne, von denen Du ſchreibſt. Aber auch
               ſonſt (unberufen!) iſt es prächtig, wie ſich ſo \strikeout{\textcolor{gray}{V}} Vieles in Dir regt und wie es aus Dir ſo reich herausblüht!\pend
           \pstart
           Was Du über die Disciplin beim Schaffen ſagſt, iſt ſehr ſchön, aber ich meine, es
               ſtimmt nicht. Man ſoll ſich nicht ſo fataliſtiſch hinſetzen\strikeout{,} und einfach das aus ſich herausfließen laſſen, was in Einem liegt. Was in
               Dir liegt, iſt zu einem Zehntel vielleicht Natur, zu \strikeout{neun} neun Zehnteln aber das, was Du in Dich hineingelegt haſt. Der
               Schriftſteller iſt doch ein Product aus Natur und aus ſich ſelbſt. Er iſt in
               fortwährender Entwickelung begriffen; und {\pb}während
               er an einem Werke arbeitet, \strikeout{\textcolor{gray}{arb}} arbeitet er zugleich ebenſo an ſich ſelbſt. Gewiß ſoll Jeder nur ſchaffen, was
               er vermag. Aber Jeder ſoll auch beſtrebt ſein, \strikeout{im}
               immer mehr zu vermögen. Gewiß darf Keiner aus ſeiner Art herauswollen. Doch in ſeiner
               Art kann Jeder Alles anſtreben, und auf allen Arten kann man zum Höchſten kommen, wie
                  \strikeout{ja} ja alle Wege zum ſelben Bergesgipfel führen.
               Blaſe Du nur ruhig Deine Flöte, die ſo liebe Klänge gibt. Ich meine \strikeout{nicht} nicht, daß Du auf einmal anfangen ſollſt, die
               Geige zu ſtreichen. Aber ich möchte, daß Du auf Deiner Flöte auch \strikeout{ein} einmal ein \label{K_L02861-2v}\edtext{\uline{anderes}{ }{\pb}Lied}{\lemma{\textnormal{\emph{anderes Lied}}}\Cendnote{\textnormal{siehe Paul Goldmann an Arthur Schnitzler, 26. 6. [1898]}}}\label{K_L02861-2h} ſpielſt. Die Gleichniſſe ſind alle falſch. Laſſen wir alſo die Gleichniſſe!
               Ich meine: Aus Deinen \textcolor{green}{Novellen}{}\ledrightnote{→\textcolor{green}{Die Frau des Weisen. Novelletten}}
               ſehe ich wieder, \strikeout{wie \textcolor{gray}{×}\-\textcolor{gray}{×}\-\textcolor{gray}{×}\-\textcolor{gray}{×}\-\textcolor{gray}{×}\-\textcolor{gray}{×}{ }\textcolor{gray}{×}\-\textcolor{gray}{×}\-\textcolor{gray}{×}\-\textcolor{gray}{×}\-\textcolor{gray}{×}\-\textcolor{gray}{×}{ }\textcolor{gray}{Du}} wie Du große menſchliche Töne zu finden vermagſt. Nur ſteckt das immer in
               einer Liebesgeſchichte gleichſam als Epiſode drin. Warum nicht die Liebesgeſchichte
               einmal weglaſſen und das große Menſchliche \strikeout{\textcolor{gray}{al}}{ }\strikeout{ſch} allein ſchreiben, ohne alle Liebe? Oder meinſt Du
               wirklich, daß Du ein »\textsc{\textcolor{green}{Erotiker}{}\ledrightnote{→\textcolor{green}{Briefe an eine Dame}}}« biſt, wie dieſes Rindvieh \textsc{\textcolor{blue}{Lothar}{}\ledrightnote{\textcolor{blue}{Rudolf Lothar}}}{ }\label{K_L02861-3v}\edtext{\textcolor{green}{geſchrieben}{}\ledrightnote{→\textcolor{green}{Briefe an eine Dame}}}{\lemma{\textnormal{\emph{geſchrieben}}}\Cendnote{\textnormal{\textcolor{blue}{Rudolf Lothar}: \emph{\textcolor{green}{Briefe an eine Dame}}. In: \emph{\textcolor{green}{Die Wage. Eine Wiener Wochenschrift}}, Jg. 1, Nr. 26,
                        25. 6. 1898, S. 439–440, hier:
                  S. 439.}}}\label{K_L02861-3h} hat?\pend
           \pstart
           Ich bin ſchon ungemein geſpannt auf Dein neues \label{K_L02861-23v}\edtext{\textcolor{green}{Stück}{}\ledrightnote{→\textcolor{green}{Das Vermächtnis. Schauspiel in drei Akten}}}{\lemma{\textnormal{\emph{Stück}}}\Cendnote{\textnormal{\emph{\textcolor{green}{Das
                     Vermächtnis}} wurde am 8. 10. 1898 am \textcolor{pink}{Deutschen Theater} in
                     \textcolor{pink}{Berlin} uraufgeführt.}}}\label{K_L02861-23h} – mehr auf das
                  \textcolor{green}{Stück}{}\ledrightnote{→\textcolor{green}{Das Vermächtnis. Schauspiel in drei Akten}}{ }{\pb}ſelbſt, als auf das, was das Publicum dazu ſagt.
               Die Idee iſt vortrefflich, und ich ſtelle mir ein ſehr zu Herzen gehendes Drama vor{\dotsfive}\pend
           \pstart
           Ich bin nun ſchon faſt drei Wochen in \textsc{\textcolor{pink}{Peking}{}\ledrightnote{\textcolor{pink}{Peking}}}, dem grauenhafteſten Schmutzneſt der Welt\substVorne{}\textsuperscript{.}\substDazwischen{},\substHinten{} habe aber manches Intereſſante miterlebt, bin auch einmal beinahe dem \textcolor{pink}{chin}{}\ledrightnote{→\textcolor{pink}{China}}eſiſchen Pöbel in die Hände
               gerathen, was ſehr ſchlecht hätte ablaufen können. Aber auch die Gefahr hat ihren
               Reiz – beſonders \strikeout{\textcolor{gray}{×}\-\textcolor{gray}{×}\-\textcolor{gray}{×}} nachher. Zugleich iſt ſie eine gute Lection: Man lernt, ruhig und entſchloſſen
               ſich zu benehmen. Morgen fahre {\pb}ich wieder nach \textsc{\textcolor{pink}{Tientsin}{}\ledrightnote{\textcolor{pink}{Tianjin}}}, von da nach \textsc{\textcolor{pink}{Shanghai}{}\ledrightnote{\textcolor{pink}{Shanghai}}} zurück. Was dann werden wird, iſt unklar; und dunkel iſt auch, was nach meiner
               Rückkehr geſchehen ſoll. In \textcolor{pink}{Wien}{}\ledrightnote{\textcolor{pink}{Wien}} bleiben? Was ſoll
               ich in einem \textcolor{pink}{Lande}{}\ledrightnote{→\textcolor{pink}{Österreich}} machen, wo
               man die Leute einſperrt, wenn ſie vor dem Sakrament nicht den Hut abnehmen? Ich
               glaube, in vier Wochen wäre ich ausgewieſen oder im Gefängniß. Und wem fehle ich in
                  \textcolor{pink}{Wien}{}\ledrightnote{\textcolor{pink}{Wien}}? Dir? Es iſt ſehr lieb, daß Du das ſagſt.
               Aber ich \strikeout{\textcolor{gray}{×}} weiß nicht, ob es gut wäre, wenn wir wieder in einer
               Stadt {\pb}zuſammenlebten. Wir kennen eigentlich nur
               unſere guten Eigenſchaften und haben unſere ſchlechten vergeſſen. Wer weiß, 
                  \strikeout{ob}{ }
               ob dieſe uns nicht jetzt, wo wir nicht mehr die
               Anpaſſungs-Fähigkeit von ehedem haben, ſehr ſtören \strikeout{und} würden. Wer weiß, \strikeout{was bei} wieviel
               Trennendes ſich bei einem dauernden Zuſammenleben zwiſchen uns plötzlich aufrichtig
               würde! Und wem fehle ich ſonſt in \textcolor{pink}{Wien}{}\ledrightnote{\textcolor{pink}{Wien}}? Keinem
               Menſchen, nicht einmal dem \textsc{\textcolor{blue}{Richard}{}\ledrightnote{\textcolor{blue}{Richard Beer-Hofmann}}}. Wo ſoll überhaupt in dieſer \textcolor{pink}{Stadt}{}\ledrightnote{→\textcolor{pink}{Wien}} für mich ein Platz ſein? Ich kann ihn nirgends entdecken{\dotsfive}\pend
           \pstart
           Ich \strikeout{bat} bat Dich ſchon, Deine {\pb}lieben Briefe fortan an meine \textcolor{blue}{Mutter}{}\ledrightnote{→\textcolor{blue}{Clementine Goldmann}} zu ſenden, welche telegraphiſch
               meine neue Adreſſe erfahren wird. Ich ſelbſt kann Dir einſtweilen keine angeben.\pend
           \pstart
           Empfiehl’ mich Deiner \textcolor{blue}{Freundin}{}\ledrightnote{→\textcolor{blue}{Marie Reinhard}} und ſei Du ſelbſt von Herzen begrüßt!\pend
           \pstart
           Dein treuer {\\[\baselineskip]}\spacefill\mbox{Paul Goldmann.}\pend
           \leftskip=0em{}\pstart
           \noindent{}Bitte, ſage dem \label{K_L02861-67v}\edtext{\textcolor{blue}{Herrn}{}\ledrightnote{→\textcolor{blue}{Louis Philipp Friedmann}}}{\lemma{\textnormal{\emph{Herrn}}}\Cendnote{\textnormal{\textcolor{blue}{Louis Friedmann}, vgl. Paul Goldmann an Arthur Schnitzler, 16. 5. 1898}}}\label{K_L02861-67h}, der mir die Empfehlung an den \textsc{Dr. 
                     \textcolor{blue}{von Rosthorn}{}\ledrightnote{\textcolor{blue}{Arthur Rosthorn}}} überſandt hat, daß ich keine Zeit hatte, ſie abzugeben. Es liegt mir daran,
                  daß Du ihm das ſagſt. Ich erkläre es Dir ſpäter einmal.\pend
           \pstart
           In einem 
                  \textcolor{pink}{fran}{}\ledrightnote{→\textcolor{pink}{Frankreich}}zöſiſchen Blatte las ich Berichte
                  über den \label{K_L02861-11v}\edtext{Zioniſten-Congreß}{\lemma{\textnormal{\emph{Zioniſten-Congreß}}}\Cendnote{\textnormal{Der zweite Zionistenkongress fand
                     zwischen 28. 8. 1898 und 31. 8. 1898 unter dem Vorsitz \textcolor{blue}{Theodor Herzl}s in \textcolor{pink}{Basel} statt. \textcolor{blue}{Goldmann} äußerte bereits in
                     vorangegangenen Briefen an \textcolor{blue}{Schnitzler}
                     Kritik an \textcolor{blue}{Herzl} und dem Zionismus. Siehe
                     etwa Paul Goldmann an Arthur Schnitzler, 29. 7. [1895] und Paul Goldmann an Arthur Schnitzler, 7. 3. [1898].}}}\label{K_L02861-11h}. Das wird doch ein recht
                  widerlicher Unfug!\pend
           \endnumbering\briefempfaengerindex{Schnitzler, Arthur@\textsc{Schnitzler, Arthur}!zzzGoldmann, Paul@\emph{von Paul Goldmann}!1898-10-161@{16. 10. {[}1898{]}}|)be}\mylabel{h}\begin{anhang}\end{anhang}\normalsize

\doendnotes{C}
\bigskip
\vfill

\clearpage

\footnotesize

\lohead{\textsc{register}}

% Definiere theindex-Environment komplett neu ohne reledmac
\makeatletter
\renewenvironment{theindex}{%
  \section*{\indexname}%
  \setlength{\parindent}{0pt}%
  \setlength{\parskip}{0pt plus 0.3pt}%
  \let\item\@idxitem
}{%
  \clearpage
}
\makeatother

\IfFileExists{\jobname-pw.ind}{\input{\jobname-pw.ind}}{}

\end{document}

      