%% latex-korrekturansicht-vorspann.tex
%% Vorspann für die Korrekturansicht.
%% Lädt die gemeinsame Datei latex-vorspann.tex mit gesetztem Schalter.

\newif\ifkorrekturansicht
\korrekturansichttrue

\input{../tex-inputs/latex-vorspann}


\renewcommand{\erwaehntePersonen}{Personen: Richard Beer-Hofmann,  Elisabeth von Österreich-Ungarn,  Franz Joseph I. von Österreich-Ungarn, Paul Goldmann, Hugo von Hofmannsthal, Richard Metzl, Louise Metzl, Ottilie Salten, Adele Sandrock}
\renewcommand{\erwaehnteInstitutionen}{Institutionen: Franz-Joseph-Orden}
\renewcommand{\erwaehnteOrte}{Orte: London, Ostsee, Paris, Riga, Russland, Wien}
\renewcommand{\erwaehnteWerke}{}
\section[ Felix Salten an Arthur Schnitzler, 23. 5. 1897]{Felix Salten an Arthur Schnitzler, 23. 5. 1897}
\nopagebreak\mylabel{v}
\rehead{ }\normalsize\beginnumbering\briefempfaengerindex{Schnitzler, Arthur@\textsc{Schnitzler, Arthur}!zzzSalten, Felix@\emph{von Felix Salten}!1897-05-231@{23. 5. 1897}|(be}
\toendnotes[C]{\smallbreak\pagebreak[2]}\Standort{CUL, Schnitzler, B 89, A 2.}
\physDesc{Brief, 2 Blätter, 6 Seiten, 5216 Zeichen
\newline{}Handschrift: schwarze Tinte, lateinische Kurrent
\newline{}Ordnung: mit Bleistift von unbekannter Hand nummeriert: »89« }\toendnotes[C]{\smallbreak}
\pstart
           \raggedleft{}{\pb}\textcolor{pink}{Wien}{}\ledrightnote{\textcolor{pink}{Wien}}, 23. Mai 97\pend
           
\pstart
           Lieber Arthur! Unsere Briefe haben sich gekreuzt. Am Tage, nach
               welchem \label{K_L03266-1v}\edtext{meiner}{\lemma{\textnormal{\emph{meiner}}}\Cendnote{\textnormal{Felix Salten an Arthur Schnitzler, 16. 5. 1897}}}\label{K_L03266-1h} abgesendet war, empfing ich den \label{K_L03266-2v}\edtext{Ihrigen}{\lemma{\textnormal{\emph{Ihrigen}}}\Cendnote{\textnormal{nicht überliefert}}}\label{K_L03266-2h}. Ich konnte nicht mehr erwarten, von
               Ihnen Nachricht zu erhalten, denn da Alle Anderen Briefe von Ihnen bekamen, musste
               ich denken, mein Schreiben Sei verloren gegangen.\pend
           
\pstart
           Die gute Stimmung, in der ich \label{K_L03266-3v}\edtext{kürzlich}{\lemma{\textnormal{\emph{kürzlich}}}\Cendnote{\textnormal{vgl. Felix Salten an Arthur Schnitzler, 5. 5. 1897}}}\label{K_L03266-3h} an Sie geschrieben, läßt nach. Vierzehn Tage Regenwetter liegen dazwischen,
               und der Frühling hat nun ein Ende. Ich fühle mich nach und nach wieder beschwert von
               allen Gewichten, die sonst immer mein Wesen drücken. Alle meine Wünsche concentriren
               sich nun darauf Etwas fertig zu bringen. Wenn das geschehen könnte, wäre ich
               wesentlich gefestigter. Aber Sie können sich nicht denken, wie sehr ich an dem Gefühl
               der Unwichtigkeit leide, sobald ich mir meine {\pb}Arbeiten fertig vorstelle und
               gedruckt und unter allen Anderem wirkend, was es in der Kunst gibt. Es ist das
               niederdrückendste Gefühl, und man ist wie gelähmt, wenn diese Empfindung Einem
               vorzurechnen beginnt, in wie vieler Beziehung man mit Allem, was man machen möchte
               und könnte, entbehrlich sei. Dass ich damit allein bleibe, ist mir oft schwer genug.
               Und ich bin jetzt ganz allein. Ich möchte das Einmal ganz klar aussprechen, damit ich
               später nicht mehr in Andeutungen darauf zurückkommen brauche. Ich möchte es umso
               eher, als ich es jetzt ohne die mindeste Bitterkeit thun kann u. meine sonstige
               starke Empfindlichkeit ungerechnet, über Empfindlichkeiten hinweg bin: Ich habe außer
               zu Ihnen, weder zu \textcolor{blue}{Hugo}{}\ledrightnote{\textcolor{blue}{Hugo von Hofmannsthal}} und noch viel weniger
               zu \textcolor{blue}{Beer-Hofmann}{}\ledrightnote{\textcolor{blue}{Richard Beer-Hofmann}} Beziehungen irgend welcher
               Art. Sie suchen mich nicht und nirgends und ich sie nicht. {\pb}Bei der großen Schätzung, auf
               welcher mein Verkehr mit ihnen beruhte, war ich zunächst darauf hingewiesen, die
               Schuld an dieser Wandlung in mir allein zu suchen. Das hat mir manche, sehr verzagte
               Stunde verursacht. Jetzt bin ich mir über die inneren und äußeren Gründe, über die
               Art, in welcher diese Gründe mit dem Leben im Allgemeinen und mit den \introOben{}beteiligten\introOben{} Personen im Besonderen zusammenhängen vollständig
               klar, und deshalb spreche ich es – wie um der Ordnung willen – aus. Ich thue \introOben{}es\introOben{} übrigens auch, weil ich nicht weiß, ob nicht in ähnlicher
               oder anderer Weise diese Angelegenheit mit Ihnen besprochen wurde, und weil ich in
               Ihnen gewiss nicht das Gefühl erhalten wissen möchte, als hätten Sie mir etwas
               schonend zu verschweigen. Schließlich bitte ich Sie, zu glauben, dass ich durchaus
               keine gegentheiligen Versicherungen, auch keine Confidencen provoziren wollte. Nicht
               wahr, das glauben Sie mir?\pend
           
\pstart
           {\pb}Mit meinen Arbeiten geht es
               mir merkwürdig. So vielerlei durcheinander, so viele neue Ausblicke durch alte
               Stoffe, so viele neue Pläne haben mich selten auf einmal beschäftigt. Und wenn das
               Gefühl der grossen Unwichtigkeit mich nicht hinderte, käme ich wol rascher vorwärts.
               Schließlich wird ja doch der Todesgedanke, der sich immer mehr und mehr meiner
               bemächtigt, seine Wirkung ausüben, und mich an ein Ziel führen. Ist es nicht
               sonderbar, dass ich an den Tod unabläßiger denn je, aber ohne Qual und ohne Angst, ja
               beinahe mit Neugierde denke? Ich bilde mir aus mehrfachen Gründen ein, dass ich
                  \label{K_L03266-4v}\edtext{mit fünfunddreißig Jahren an einem
               Märztag weggehen}{\lemma{\textnormal{\emph{mit … weggehen}}}\Cendnote{\textnormal{\textcolor{blue}{Salten} starb im Alter von 76 Jahren am 8. 10. 1945.}}}\label{K_L03266-4h} werde, und ich denke daran, wie an
               ein Unternehmen, dessen Zustandekommen zum Theil in meiner Macht liegt. Es ist nicht
               Selbstmord, warum sonst die fünfundreißig? Aber es ist so, als müsste ich in diesen
                  {\pb}acht Jahren, die es bis
               dahin noch sind, alles erledigen, und als wäre ich dann eben \label{K_L03266-5v}\edtext{\begin{otherlanguage}{french}à jour\end{otherlanguage}}{\lemma{\textnormal{\emph{à jour}}}\Cendnote{\textnormal{französisch: auf dem neuesten
                  Stand}}}\label{K_L03266-5h}. Meine Gesundheit, – der alte Bronchialkatarrh, der ja doch einmal
               die Lunge angreifen muss, – meine Arbeiten, mein Lieben, alles scheint mir so, als
               könne es nicht länger vorhalten als bis zu jenem Märztag im Jahre 1905. Jedesfalls hängt meine Sorglosigkeit und der Glaube an eine Wendung
               der Dinge, die nun bald eintreten müsse{[},{]} mit dieser Vorstellung
               zusammen, und ich habe wenigstens die Zuversicht davon, nicht einen Tag früher zu
               sterben.\pend
           
\pstart
           Gestern erhielt ich von Frl. \textcolor{blue}{Sandrock}{}\ledrightnote{\textcolor{blue}{Adele Sandrock}} einen wunderschönen Brief, so echt, wie ich noch
               nichts von ihr gehört: »Ich liege an der \textcolor{pink}{Ostsee}{}\ledrightnote{\textcolor{pink}{Ostsee}}
               und blicke mit meinen blauen Augen zum blauen Himmel empor,« so beginnt es, und es
               ist, als ob sie \strikeout{\textcolor{gray}{ih}} in ihrem gelösten Wesen mit diesen ironischen Worten eine directe Verbindung
               zwischen sich und der Welt gefunden hätte. Dann kommen Sätze: »Gestern
               trat ich hier auf, und heute liegt \textcolor{pink}{Riga}{}\ledrightnote{\textcolor{pink}{Riga}}{ }\label{K_L03266-6v}\edtext{in Fraisen}{\lemma{\textnormal{\emph{in Fraisen}}}\Cendnote{\textnormal{veralteter Ausdruck von: in tödlichen Krämpfen, an einem
                  gefährlichen Anfall leidend}}}\label{K_L03266-6h}.« Oder: »Was geht in Dir vor? gehe ich {\pb}Dir ab? (Ich habe sie zwei
               Monate nicht gesehen und nichts von ihr gehört) Fällst Du nicht todt vom Boden bei
               dem Gedanken mich bald wieder zu sehen?« ec. Schließlich läuft das Ganze auf die
               Bitte hinaus, ihre Triumphe in die Zeitung zu geben. – Der \textcolor{blue}{Bruder}{}\ledrightnote{{$\rightarrow$}\textcolor{blue}{Richard Metzl}} des Fräulein \textcolor{blue}{M.}{}\ledrightnote{\textcolor{blue}{Ottilie Salten}} hat in \textcolor{pink}{Rußland}{}\ledrightnote{\textcolor{pink}{Russland}}
               anläßlich der Anwesenheit unseres \textcolor{blue}{Kaiser}{}\ledrightnote{{$\rightarrow$}\textcolor{blue}{Franz Joseph I. von Österreich-Ungarn}}s den \textcolor{brown}{Franz Josefs-Orden}{}\ledrightnote{\textcolor{brown}{Franz-Joseph-Orden}}
               erhalten. Heute war er hier, im Smoking und schwarzer
               Binde, mit der Rosette im Knopfloch bei seiner \textcolor{blue}{Mutter}{}\ledrightnote{{$\rightarrow$}\textcolor{blue}{Louise Metzl}} zu Besuch. Gespräch: Thema: unsere \textcolor{blue}{Kaiserin}{}\ledrightnote{{$\rightarrow$}\textcolor{blue}{Elisabeth von Österreich-Ungarn}}. \uline{Ich}: Sie lebt wunderschön, – so allein, von nichts bekümmert. \uline{Er}: Sie könnte es noch schöner haben. \uline{Ich}: Wieso denn? \uline{Er}:
               Sie könnte Protektorin aller Wohltätigkeitsvereine sein! (Wörtlich)\pend
           
\pstart
           Ich weiß wieder nicht, ob dieser Brief Sie in \textcolor{pink}{Paris}{}\ledrightnote{\textcolor{pink}{Paris}} noch trifft. Wenn Sie ihn erhalten, dann bitte zeigen Sie’s mir, wenn
               auch nur auf einer Postkarte, an.\pend
           
\pstart
            Herzlich Ihr {\\[\baselineskip]}\spacefill\mbox{Salten}\pend
           \leftskip=0em{}\endnumbering\briefempfaengerindex{Schnitzler, Arthur@\textsc{Schnitzler, Arthur}!zzzSalten, Felix@\emph{von Felix Salten}!1897-05-231@{23. 5. 1897}|)be}\mylabel{h}  \normalsize

\doendnotes{C}
\bigskip
\vfill

\clearpage

\footnotesize

\lohead{\textsc{register}}

% Definiere theindex-Environment komplett neu ohne reledmac
\makeatletter
\renewenvironment{theindex}{%
  \section*{\indexname}%
  \setlength{\parindent}{0pt}%
  \setlength{\parskip}{0pt plus 0.3pt}%
  \let\item\@idxitem
}{%
  \clearpage
}
\makeatother

\IfFileExists{\jobname-pw.ind}{\input{\jobname-pw.ind}}{}

\end{document}

      