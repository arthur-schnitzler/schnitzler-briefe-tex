%% latex-korrekturansicht-vorspann.tex
%% Vorspann für die Korrekturansicht.
%% Lädt die gemeinsame Datei latex-vorspann.tex mit gesetztem Schalter.

\newif\ifkorrekturansicht
\korrekturansichttrue

\input{../tex-inputs/latex-vorspann}


\renewcommand{\erwaehntePersonen}{Personen: Clementine Goldmann, Theodore Rottenberg, Paul Schlenther, Olga Schnitzler, Ida d’Albert}
\renewcommand{\erwaehnteInstitutionen}{Institutionen: Bauernfeld-Preis, Burgtheater, Deutsches Theater Berlin, Neue Freie Presse}
\renewcommand{\erwaehnteOrte}{Orte: Berlin, Dessauer Straße, Deutsches Theater Berlin, Volkstheater, Wien}
\renewcommand{\erwaehnteWerke}{Werke: Berliner Theater. (»Der Schleier der Beatrice« von Arthur Schnitzler.), Berliner Theater. (»Lebendige Stunden« von Arthur Schnitzler.), Der Schleier der Beatrice. Schauspiel in fünf Akten, Lebendige Stunden. Vier Einakter, Neue Freie Presse, Tagebuch}
\section[ Paul Goldmann an Arthur Schnitzler, 17. 3. {[}1903{]}]{Paul Goldmann an Arthur Schnitzler, 17. 3. {[}1903{]}}
\nopagebreak\mylabel{v}
\rehead{ }\normalsize\beginnumbering\briefempfaengerindex{Schnitzler, Arthur@\textsc{Schnitzler, Arthur}!zzzGoldmann, Paul@\emph{von Paul Goldmann}!1903-03-171@{17. 3. {[}1903{]}}|(be}
\toendnotes[C]{\smallbreak\pagebreak[2]}\Standort{DLA, A:Schnitzler, HS.NZ85.1.3173.}
\physDesc{Brief, 1 Blatt, 4 Seiten
\newline{}Handschrift: blaue Tinte, deutsche Kurrent
\newline{}Schnitzler: 1) mit Bleistift das Jahr »{[}1{]}903« vermerkt  2) mit rotem Buntstift eine Unterstreichung}\toendnotes[C]{\smallbreak}
\pstart
           \noindent{}\raggedleft{}{\pb}\textcolor{gray}{\textbf{\textcolor{pink}{DESSAUERSTRASSE 19}{}\ledrightnote{\textcolor{pink}{Dessauer Straße}}}}\pend
           
\pstart
           \textcolor{pink}{Berlin}{}\ledrightnote{\textcolor{pink}{Berlin}}, 17. März.\pend
           
\pstart\center{}Mein lieber Freund,\pend
\pstart
           Ich habe mit großer Freude \strikeout{\textcolor{gray}{ver}} geleſen, daß Du den \label{K_L03369-1v}\edtext{\textcolor{brown}{\textsc{Bauernfeld}-Preis}{}\ledrightnote{\textcolor{brown}{Bauernfeld-Preis}}}{\lemma{\textnormal{\emph{Bauernfeld-Preis}}}\Cendnote{\textnormal{Den \emph{\textcolor{brown}{Bauernfeld-Preis}} erhielt \textcolor{blue}{Schnitzler}
                  am 17. 3. 1903 für
                  seinen Einakterzyklus \emph{\textcolor{green}{Lebendige Stunden}}.
                     1899 hatte er den \textcolor{brown}{Literaturpreis} schon einmal erhalten.}}}\label{K_L03369-1h} erhalten
               haſt, u. beglückwünſche Dich (auch im Namen meiner \textcolor{blue}{Mutter}{}\ledrightnote{{$\rightarrow$}\textcolor{blue}{Clementine Goldmann}}) auf das Herzlichſte.\pend
           
\pstart
           Auch höre ich, daß die \label{K_L03369-2v}\edtext{»\textsc{\textcolor{green}{Beatrice}{}\ledrightnote{\textcolor{green}{Der Schleier der Beatrice. Schauspiel in fünf Akten}}}«}{\lemma{\textnormal{\emph{»Beatrice«}}}\Cendnote{\textnormal{am \emph{\textcolor{brown}{Deutschen Theater Berlin}}}}}\label{K_L03369-2h} gut geht. Frau \textsc{\textcolor{blue}{Fulda}{}\ledrightnote{\textcolor{blue}{Ida d’Albert}}} ſagte es mir; ſie fügte hinzu, Sonntag ſei das
                  \textcolor{pink}{Haus}{}\ledrightnote{{$\rightarrow$}\textcolor{pink}{Deutsches Theater Berlin}} ausverkauft geweſen. {\pb}Auch das freut mich von Herzen.\pend
           
\pstart
           Heut habe ich nun endlich mein \label{K_L03369-3v}\edtext{\textcolor{green}{Feuilleton}{}\ledrightnote{{$\rightarrow$}\textcolor{green}{Berliner Theater. (»Der Schleier der Beatrice« von Arthur Schnitzler.)}}}{\lemma{\textnormal{\emph{Feuilleton}}}\Cendnote{\textnormal{\textcolor{blue}{Paul Goldmann}: \emph{\textcolor{green}{Berliner Theater. (»Der Schleier der Beatrice« von Arthur
                        Schnitzler.)}}. In: \emph{\textcolor{green}{Neue Freie
                        Presse}}, Nr. 13.851, 19. 3. 1903,
                     Morgenblatt, S. 1–5. Dieses äußerst negativ ausfallende \textcolor{green}{Feuilleton} stellt eine Zäsur in der
                  Beziehung zwischen \textcolor{blue}{Goldmann} und \textcolor{blue}{Schnitzler} dar. Nach \textcolor{blue}{Goldmann}s kritischem \textcolor{green}{Feuilleton} zu \emph{\textcolor{green}{Lebendige
                     Stunden}} im Jahr zuvor war es in den folgenden
                  Jahren der zweite zentrale Punkt in deren Streit. In \textcolor{blue}{Schnitzler}s \emph{\textcolor{green}{Tagebuch}}
                  finden sich ab dem 19. 3. 1903 mehrfach Notizen dazu.}}}\label{K_L03369-3h} abgeſandt. Ich habe zehn
               Tage lang damit gerungen – wahrhaft gerungen – habe allein den Anfang vier Mal neu
               geſchrieben. Das \textcolor{green}{Stück}{}\ledrightnote{{$\rightarrow$}\textcolor{green}{Der Schleier der Beatrice. Schauspiel in fünf Akten}} hat
               mir, je mehr ich darauf einging, immer weniger gefallen. Ich finde es, bei allen
               dichteriſchen Eigenſchaften, innerlich klein. Nun habe ich mich aufs Äußerſte
               angeſtrengt, {\pb}gerecht zu ſein, mit jedem Worte. Mein
               Gewiſſen ſagt mir, daß ich es geweſen bin. Was Du ſagen wirſt, weiß ich nicht. Aber
               ich verwünſche mein Schickſal und ich frage mich, ob man dazu einen einzigen nahen
               und lieben Freund hat, um gegen ihn – öffentlich, vor allen Leuten – gerecht zu ſein?
               Vielleicht übrigens mißfällt das \textcolor{green}{Feuilleton}{}\ledrightnote{{$\rightarrow$}\textcolor{green}{Berliner Theater. (»Der Schleier der Beatrice« von Arthur Schnitzler.)}} in der \textcolor{brown}{Redaktion}{}\ledrightnote{{$\rightarrow$}\textcolor{brown}{Neue Freie Presse}} und es erſcheint {\pb}gar nicht. Das
               wäre mir das Liebſte.\pend
           
\pstart
           Auch zu dem \label{K_L03369-4v}\edtext{Erfolge der »\textcolor{green}{Lebendigen St.}{}\ledrightnote{\textcolor{green}{Lebendige Stunden. Vier Einakter}}« in \textcolor{pink}{Wien}{}\ledrightnote{\textcolor{pink}{Wien}}}{\lemma{\textnormal{\emph{Erfolge … Wien}}}\Cendnote{\textnormal{\emph{\textcolor{green}{Lebendige Stunden}} hatte am 14. 3. 1903 am \textcolor{pink}{Deutschen Volkstheater} in \textcolor{pink}{Wien} Premiere.}}}\label{K_L03369-4h} beglückwünſche ich Dich auf das
               Herzlichſte. Wird nun der Herr \label{K_L03369-5v}\edtext{\textsc{\textcolor{blue}{Schlenther}{}\ledrightnote{\textcolor{blue}{Paul Schlenther}}}}{\lemma{\textnormal{\emph{Schlenther}}}\Cendnote{\textnormal{Bezug auf die \emph{\textcolor{green}{Beatrice}}-Affaire (1899–1900) bzw. dem damit einhergehenden fünfjährigen Ausschluss
                     \textcolor{blue}{Schnitzler}s vom \emph{\textcolor{brown}{Burgtheater}}}}}\label{K_L03369-5h} ſich nicht endlich rühren?\pend
           
\pstart
           Dank für Deine lieben Zeilen aus \textcolor{pink}{Wien}{}\ledrightnote{\textcolor{pink}{Wien}}! Ich bin
               traurig, wie zuvor. Mein ganzes Leben iſt voll von dieſer \label{K_L03369-7v}\edtext{\textcolor{blue}{Frau}{}\ledrightnote{{$\rightarrow$}\textcolor{blue}{Theodore Rottenberg}}}{\lemma{\textnormal{\emph{Frau}}}\Cendnote{\textnormal{\textcolor{blue}{Theodore Rottenberg}, die \textcolor{blue}{Goldmann} Anfang 1903 verlassen
                  hatte (vgl. Paul Goldmann an Arthur Schnitzler, 3. 1. [1903])}}}\label{K_L03369-7h}, die
               mich längſt vergeſſen hat.\pend
           
\pstart
           Leb’ wohl, mein lieber Freund! Grüße \textsc{\textcolor{blue}{Olga}{}\ledrightnote{\textcolor{blue}{Olga Schnitzler}}} u. ſei Du ſelbſt vielmals gegrüßt von Deinem {\\[\baselineskip]}getreuen \spacefill\mbox{Paul
                  Goldm}\pend
           \leftskip=0em{}\endnumbering\briefempfaengerindex{Schnitzler, Arthur@\textsc{Schnitzler, Arthur}!zzzGoldmann, Paul@\emph{von Paul Goldmann}!1903-03-171@{17. 3. {[}1903{]}}|)be}\mylabel{h}
\begin{anhang}
\end{anhang}\normalsize

\doendnotes{C}
\bigskip
\vfill

\clearpage

\footnotesize

\lohead{\textsc{register}}

% Definiere theindex-Environment komplett neu ohne reledmac
\makeatletter
\renewenvironment{theindex}{%
  \section*{\indexname}%
  \setlength{\parindent}{0pt}%
  \setlength{\parskip}{0pt plus 0.3pt}%
  \let\item\@idxitem
}{%
  \clearpage
}
\makeatother

\IfFileExists{\jobname-pw.ind}{\input{\jobname-pw.ind}}{}

\end{document}

      