%% latex-korrekturansicht-vorspann.tex
%% Vorspann für die Korrekturansicht.
%% Lädt die gemeinsame Datei latex-vorspann.tex mit gesetztem Schalter.

\newif\ifkorrekturansicht
\korrekturansichttrue

\input{../tex-inputs/latex-vorspann}


               \section[Richard Beer-Hofmann an Arthur Schnitzler, 22. 7. 1896]{ Richard Beer-Hofmann an Arthur Schnitzler, 22. 7. 1896}\nopagebreak\mylabel{v}\rehead{ }\normalsize\beginnumbering\briefempfaengerindex{Schnitzler, Arthur@\textsc{Schnitzler, Arthur}!zzzBeer-Hofmann, Richard@\emph{von Richard Beer-Hofmann}!1896-07-221@{22. 7. 1896}|(be} \toendnotes[C]{\smallbreak\pagebreak[2]} \Standort{CUL, Schnitzler, B 8.}
\physDesc{Telegramm
\newline{}Handschrift einer Schreibkraft: Bleistift, lateinische Kurrent\newline{}Versand: »\noindent{}\textcolor{gray}{\textbf{\textbf{Optaget} fra}} 39 \textcolor{gray}{\textbf{den}}{ }21\textcolor{gray}{\textbf{/}}7{ }4,45 \textcolor{gray}{\textbf{midd. af}}{ }\textcolor{gray}{M}« \newline{}Ordnung: mit Bleistift von unbekannter Hand nummeriert:
                                    »76« }\buchAbdrucke{\weitereDrucke{Arthur Schnitzler, Richard Beer-Hofmann: \emph{Briefwechsel 1891–1931}. Hg. Konstanze Fliedl. Wien, Zürich: \emph{Europaverlag} 1992, S. 93.} }\pstart{}{\pb}\textsc{Doktor Arthur Schnitzler +}\pend{}\pstart{}\textsc{poste restante \textcolor{pink}{Thiem}{}\ledrightnote{\textcolor{pink}{Trondheim}}}\pend{}{\bigskip}\pstart
           \noindent{}{\pb}\textcolor{gray}{\textbf{Telegram fra}}{ }\textcolor{pink}{Salzburg}{}\ledrightnote{\textcolor{pink}{Salzburg}}{ }\textcolor{gray}{\textbf{No.}} 501\textcolor{gray}{\textbf{, Ord}} 20\textcolor{gray}{\textbf{, den}}{ }22\textcolor{gray}{\textbf{/}}7{ }\textcolor{gray}{\textbf{189}}6{ }\textcolor{gray}{\textbf{Kl.}} 11,10\textcolor{gray}{\textbf{midd.}}\pend
           \pstart
           Reise heute \textcolor{pink}{Salzburg}{}\ledrightnote{\textcolor{pink}{Salzburg}} ab, über \textcolor{pink}{München}{}\ledrightnote{\textcolor{pink}{München}}{ }\textcolor{pink}{Berlin}{}\ledrightnote{\textcolor{pink}{Berlin}} bin 25{ }\textcolor{pink}{Kopenhagen}{}\ledrightnote{\textcolor{pink}{Kopenhagen}} erwarte Nachricht herzlichst\pend
           \pstart \spacefill\mbox{Richard}\pend{}\endnumbering\briefempfaengerindex{Schnitzler, Arthur@\textsc{Schnitzler, Arthur}!zzzBeer-Hofmann, Richard@\emph{von Richard Beer-Hofmann}!1896-07-221@{22. 7. 1896}|)be}\mylabel{h}  \normalsize

\doendnotes{C}
\bigskip
\vfill

\clearpage

\footnotesize

\lohead{\textsc{register}}

% Definiere theindex-Environment komplett neu ohne reledmac
\makeatletter
\renewenvironment{theindex}{%
  \section*{\indexname}%
  \setlength{\parindent}{0pt}%
  \setlength{\parskip}{0pt plus 0.3pt}%
  \let\item\@idxitem
}{%
  \clearpage
}
\makeatother

\IfFileExists{\jobname-pw.ind}{\input{\jobname-pw.ind}}{}

\end{document}

      