%% latex-korrekturansicht-vorspann.tex
%% Vorspann für die Korrekturansicht.
%% Lädt die gemeinsame Datei latex-vorspann.tex mit gesetztem Schalter.

\newif\ifkorrekturansicht
\korrekturansichttrue

\input{../tex-inputs/latex-vorspann}


               \section[Arthur Schnitzler an Hugo von Hofmannsthal, {[}25.? 11. 1902{]}]{ Arthur Schnitzler an Hugo von Hofmannsthal, {[}25.? 11. 1902{]}}\nopagebreak\mylabel{v}\rehead{ }\normalsize\beginnumbering\briefempfaengerindex{Hofmannsthal, Hugo von@\textsc{Hofmannsthal, Hugo von}!zzzSchnitzler, Arthur@\emph{von Arthur Schnitzler}!1902-11-251@{{[}25.? 11. 1902{]}}|(be} \toendnotes[C]{\smallbreak\pagebreak[2]} \Standort{FDH, Hs-30885,100.}
\physDesc{Brief, 1 Blatt, 3 Seiten
\newline{}Handschrift: Bleistift, deutsche Kurrent\newline{}Ordnung: mit Bleistift von unbekannter Hand datiert: »1906??« }\buchAbdrucke{\weitereDrucke{Hugo von Hofmannsthal, Arthur Schnitzler: \emph{Briefwechsel}. Hg. Therese Nickl und Heinrich Schnitzler. Frankfurt am Main: \emph{S. Fischer} 1964, S. 164.} }\toendnotes[C]{\smallbreak}\pstart
           \noindent{}{\pb}lieber Hugo, ich habe, da auch ich keine andre Adreſſe weiſs,
                    den Brief in die Direktion des \textcolor{pink}{Burg. Th.}{}\ledrightnote{\textcolor{pink}{Burgtheater}}
                    geſchickt.\pend
           \pstart
           – Es iſt jetzt mit dem Landfahren, beſonders abends \strikeout{übrigens} keine ſehr begeiſternde Sache; es wäre mir ſchon lieber,
                        we{\geminationn} ich Sie, gelegentlich einer \textcolor{pink}{Wien}{}\ledrightnote{\textcolor{pink}{Wien}}fahrt, vorerſt einmal hier zu ſehen u zu ſprechen
                    bekäme. – Natürlich fahr ich, we{\geminationn}{ }\substVorne{}\textsuperscript{ich}\substDazwischen{}die\substHinten{}{ }\textcolor{blue}{Hauptma{\geminationn}}{}\ledrightnote{\textcolor{blue}{Gerhart Hauptmann}}geſchichte zu Stande ko{\geminationm}t, mit ihm zu Ihnen
                        {\pb}hinaus. –\pend
           \pstart
           Ich freue mich auf Ihr \textcolor{green}{Stück}{}\ledrightnote{→\textcolor{green}{Das gerettete Venedig. Trauerspiel in fünf Aufzügen}}. – Ich habe geſtern die \textcolor{green}{Skizze des meinen}{}\ledrightnote{→\textcolor{green}{Der einsame Weg. Schauspiel in fünf Akten}} – de{\geminationn}
                    ich ka{\geminationn} es in keiner Weiſe ausgeführt nennen, – zu
                    Ende dictirt, und ein ſchwerer \label{K_L01252_1v}\edtext{Grundfehler}{\lemma{\textnormal{\emph{Grundfehler}}}\Cendnote{\textnormal{siehe A. S.: \emph{Tagebuch}, 25. 11. 1902}}}\label{K_L01252_1h} des
                    ganzen, der nun mit Evidenz zu Tage trat, hat mich auffallend tief verſtimmt; –
                    mich in die Nacht und in meine Träume wie ein wirkliches Unglück ver{\pb}folgt. Solche Dinge haben natürlich i{\geminationm}er ihren Sinn: Mängel eines Werks, die man \uline{ſo}{ }ſchmerzlich empfindet, ſind i{\geminationm}er Mängel des eigenen Weſens, auf die man in
                    dieſer geheimnisvollen Weiſe geleitet wird.\pend
           \pstart
           – Leben Sie wohl. Auf bald.\pend
           \pstart
           Herzlichſt Ihr{\\[\baselineskip]}\spacefill\mbox{A.}\pend
           \leftskip=0em{}\endnumbering\briefempfaengerindex{Hofmannsthal, Hugo von@\textsc{Hofmannsthal, Hugo von}!zzzSchnitzler, Arthur@\emph{von Arthur Schnitzler}!1902-11-251@{{[}25.? 11. 1902{]}}|)be}\mylabel{h}  \normalsize

\doendnotes{C}
\bigskip
\vfill

\clearpage

\footnotesize

\lohead{\textsc{register}}

% Definiere theindex-Environment komplett neu ohne reledmac
\makeatletter
\renewenvironment{theindex}{%
  \section*{\indexname}%
  \setlength{\parindent}{0pt}%
  \setlength{\parskip}{0pt plus 0.3pt}%
  \let\item\@idxitem
}{%
  \clearpage
}
\makeatother

\IfFileExists{\jobname-pw.ind}{\input{\jobname-pw.ind}}{}

\end{document}

      