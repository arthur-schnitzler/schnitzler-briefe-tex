%% latex-korrekturansicht-vorspann.tex
%% Vorspann für die Korrekturansicht.
%% Lädt die gemeinsame Datei latex-vorspann.tex mit gesetztem Schalter.

\newif\ifkorrekturansicht
\korrekturansichttrue

\input{../tex-inputs/latex-vorspann}


\renewcommand{\erwaehntePersonen}{Personen: Charlotte Pohl-Glas,  Reisner, Adele Reisner}
\renewcommand{\erwaehnteInstitutionen}{Institutionen: Berliner Neueste Nachrichten, Münchener General-Anzeiger}
\renewcommand{\erwaehnteOrte}{Orte: Hörlgasse, Wien}
\renewcommand{\erwaehnteWerke}{Werke: Tagebuch}
\section[Felix Salten an Arthur Schnitzler, {[}zwischen 4. und 13. 9.? 1894{]}]{Felix Salten an Arthur Schnitzler,
               {[}zwischen 4. und 13. 9.? 1894{]}}
\nopagebreak\mylabel{v}
\rehead{ }\normalsize\beginnumbering\briefempfaengerindex{Schnitzler, Arthur@\textsc{Schnitzler, Arthur}!zzzSalten, Felix@\emph{von Felix Salten}!1894-09-041@{{[}zwischen 4. und 13. 9.? 1894{]}}|(be}
\toendnotes[C]{\smallbreak\pagebreak[2]}\Standort{CUL, Schnitzler, B 89, A 1.}
\physDesc{Visitenkarte, 307 Zeichen
\newline{}Handschrift: Bleistift, lateinische Kurrent
\newline{}Schnitzler: mit Bleistift datiert: »\textcolor{gray}{94}« 
\newline{}Ordnung: mit Bleistift von unbekannter Hand nummeriert: »36a« }\toendnotes[C]{\smallbreak}
\pstart
           \noindent{}\centering{}{\pb}\textcolor{gray}{\textbf{FELIX SALTEN}}\pend
           
\pstart
           \noindent{}\textcolor{gray}{\textbf{\textcolor{pink}{WIEN}{}\ledrightnote{\textcolor{pink}{Wien}},}}\hfill \textcolor{gray}{\textbf{»\textcolor{brown}{Berliner Neueste
                        Nachrichten}{}\ledrightnote{\textcolor{brown}{Berliner Neueste Nachrichten}}.«}}\pend
           
\pstart
           \textcolor{gray}{\textbf{\textcolor{pink}{IX., Hörlgasse 16}{}\ledrightnote{\textcolor{pink}{Hörlgasse}}.}}\hfill \textcolor{gray}{\textbf{ »\textcolor{brown}{Münchener
                        General-Anzeiger}{}\ledrightnote{\textcolor{brown}{Münchener General-Anzeiger}}.«}}\pend
           
\pstart
           {\pb}Lieber Frd, ich habe jetzt \label{K_L03134-1v}\edtext{\textcolor{blue}{Rendezvous}{}\ledrightnote{{$\rightarrow$}\textcolor{blue}{Charlotte Pohl-Glas}}}{\lemma{\textnormal{\emph{Rendezvous}}}\Cendnote{\textnormal{Da diese Visitenkarte \textcolor{blue}{Salten}s nur für den Zeitraum vom [6. 9. 1894] bis zum 15. 9. 189[4?] belegt ist, ist
                  es wahrscheinlich, dass auch diese Karte nach \textcolor{blue}{Schnitzler}s Heimkehr nach \textcolor{pink}{Wien} im
                     September 1894 übermittelt wurde. Nimmt man zudem
                  an, dass ein »Rendezvous« \textcolor{blue}{Salten}s mit \textcolor{blue}{Lotte Glas} gemeint ist,
                  so schränkt sich der Zeitraum weiter ein, denn diese hatte am [11. 9. 1894] ihre Haftstrafe
                  angetreten.}}}\label{K_L03134-1h} und kann deshalb nicht ko{\geminationm}en. Es
               ist möglich, dass wir, dh. ich u. »\textcolor{blue}{sie}{}\ledrightnote{{$\rightarrow$}\textcolor{blue}{Charlotte Pohl-Glas}}« mit der \label{K_L03134-2v}\edtext{\textcolor{blue}{Reisner}{}\ledrightnote{\textcolor{blue}{Reisner}}}{\lemma{\textnormal{\emph{Reisner}}}\Cendnote{\textnormal{Obzwar die \textcolor{blue}{Person} bislang nicht genauer
                  identifiziert werden konnte, ist anzunehmen, dass damit nicht die im Register des
                     \emph{\textcolor{green}{Tagebuch}}s angeführte \textcolor{blue}{Adele Reisner} gemeint war, da diese zu diesem Zeitpunkt
                  noch nicht einmal 12 Jahre alt war. Wahrscheinlicher ist, dass sich auch die
                  Einträge zu \textcolor{blue}{Adele Reisner} im \emph{\textcolor{green}{Tagebuch}} auf die vorliegende \textcolor{blue}{Person} bezogen.}}}\label{K_L03134-2h} zusammen
               soupiren, für diesen Fall telephonire ich Sie an, oder bitte laßen Sie mir sagen, wo
               ich Sie zwischen ½ 8 u. ½ 9 treffen kann. Ohne dass Sie
               sich binden, natürlich.\pend
           
\pstart
           Herzlichst {\\[\baselineskip]}\spacefill\mbox{Salten}\pend
           \leftskip=0em{}\endnumbering\briefempfaengerindex{Schnitzler, Arthur@\textsc{Schnitzler, Arthur}!zzzSalten, Felix@\emph{von Felix Salten}!1894-09-041@{{[}zwischen 4. und 13. 9.? 1894{]}}|)be}\mylabel{h}  \normalsize

\doendnotes{C}
\bigskip
\vfill

\clearpage

\footnotesize

\lohead{\textsc{register}}

% Definiere theindex-Environment komplett neu ohne reledmac
\makeatletter
\renewenvironment{theindex}{%
  \section*{\indexname}%
  \setlength{\parindent}{0pt}%
  \setlength{\parskip}{0pt plus 0.3pt}%
  \let\item\@idxitem
}{%
  \clearpage
}
\makeatother

\IfFileExists{\jobname-pw.ind}{\input{\jobname-pw.ind}}{}

\end{document}

      