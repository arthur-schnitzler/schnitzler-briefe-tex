%% latex-korrekturansicht-vorspann.tex
%% Vorspann für die Korrekturansicht.
%% Lädt die gemeinsame Datei latex-vorspann.tex mit gesetztem Schalter.

\newif\ifkorrekturansicht
\korrekturansichttrue

\input{../tex-inputs/latex-vorspann}


               \section[Hermann Bahr an Arthur Schnitzler, 19. {[}10. 1904{]}]{ Hermann Bahr an Arthur Schnitzler, 19. {[}10. 1904{]}}\nopagebreak\mylabel{v}\rehead{ }\normalsize\beginnumbering\briefempfaengerindex{Schnitzler, Arthur@\textsc{Schnitzler, Arthur}!zzzBahr, Hermann@\emph{von Hermann Bahr}!1904-10-191@{19. {[}10. 1904{]}}|(be} \toendnotes[C]{\smallbreak\pagebreak[2]} \Standort{CUL, Schnitzler, B 5b.}
\physDesc{Brief, 1 Blatt, 1 Seite
\newline{}Handschrift: schwarze Tinte, deutsche Kurrent
\newline{}Schnitzler: mit Bleistift Monats- und Jahresangabe ergänzt: »10. 904« ergänzt \newline{}Ordnung: mit Bleistift von unbekannter Hand nummeriert:
                              »121« }\buchAbdrucke{\weitereDrucke{Hermann Bahr, Arthur Schnitzler: \emph{Briefwechsel, Aufzeichnungen, Dokumente (1891–1931)}. Hg. Kurt Ifkovits und Martin Anton Müller. Göttingen: \emph{Wallstein} 2018, S. 325.} }\toendnotes[C]{\smallbreak}\pstart
           \raggedleft{}{\pb}19. früh\pend
           \pstart\center{}Lieber Arthur!\pend\pstart
           Leider reiſe ich eben nach \label{K_L01458_1v}\edtext{\textcolor{pink}{\textsc{Ragusa}}{}\ledrightnote{\textcolor{pink}{Dubrovnik}}}{\lemma{\textnormal{\emph{Ragusa}}}\Cendnote{\textnormal{\textcolor{blue}{Bahr} war vom 19. bis
                     23. 10. 1904 in \textcolor{pink}{Dalmatien}.}}}\label{K_L01458_1h}. Hoffentlich nächſtens einmal.\pend
           \pstart
           Deine \textcolor{blue}{Frau}{}\ledrightnote{→\textcolor{blue}{Olga Schnitzler}}, die ich herzlichſt
               grüße, ſoll jedenfalls zu den \textcolor{blue}{Schweſtern
                  Flöge}{}\ledrightnote{\textcolor{blue}{Pauline Flöge}{\newline}\textcolor{blue}{Helene Flöge}{\newline}\textcolor{blue}{Emilie Flöge}} gehen \textcolor{pink}{Mariahilferſtr. 1}{}\ledrightnote{\textcolor{pink}{Mariahilferstraße}} (\textsc{\textcolor{pink}{Casa \textcolor{gray}{p}iccola}{}\ledrightnote{\textcolor{pink}{Casa Piccola}}}), die für die \textcolor{blue}{meinige}{}\ledrightnote{→\textcolor{blue}{Rosa Bahr}}
               herrlich gearbeitet haben.\pend
           \pstart
           Herzlichſt{\\[\baselineskip]}\spacefill\mbox{H.}\pend
           \leftskip=0em{}\endnumbering\briefempfaengerindex{Schnitzler, Arthur@\textsc{Schnitzler, Arthur}!zzzBahr, Hermann@\emph{von Hermann Bahr}!1904-10-191@{19. {[}10. 1904{]}}|)be}\mylabel{h}  \normalsize

\doendnotes{C}
\bigskip
\vfill

\clearpage

\footnotesize

\lohead{\textsc{register}}

% Definiere theindex-Environment komplett neu ohne reledmac
\makeatletter
\renewenvironment{theindex}{%
  \section*{\indexname}%
  \setlength{\parindent}{0pt}%
  \setlength{\parskip}{0pt plus 0.3pt}%
  \let\item\@idxitem
}{%
  \clearpage
}
\makeatother

\IfFileExists{\jobname-pw.ind}{\input{\jobname-pw.ind}}{}

\end{document}

      