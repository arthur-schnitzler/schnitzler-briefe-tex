%% latex-korrekturansicht-vorspann.tex
%% Vorspann für die Korrekturansicht.
%% Lädt die gemeinsame Datei latex-vorspann.tex mit gesetztem Schalter.

\newif\ifkorrekturansicht
\korrekturansichttrue

\input{../tex-inputs/latex-vorspann}


               \section[Georg Brandes an Arthur Schnitzler, 13. 6. 1920]{ Georg Brandes an Arthur Schnitzler, 13. 6. 1920}\nopagebreak\mylabel{v}\rehead{ }\normalsize\beginnumbering\briefempfaengerindex{Schnitzler, Arthur@\textsc{Schnitzler, Arthur}!zzzBrandes, Georg@\emph{von Georg Brandes}!1920-06-131@{13. 6. 1920}|(be} \toendnotes[C]{\smallbreak\pagebreak[2]} \Standort{CUL, Schnitzler, B 17.}
\physDesc{Brief, 1 Blatt, 4 Seiten
\newline{}Handschrift: schwarze Tinte, lateinische Kurrent
\newline{}Schnitzler: mit rotem Buntstift vereinzelte Unterstreichungen \newline{}Ordnung: von unbekannter Hand nummeriert: »50« }\buchAbdrucke{\weitereDrucke{Georg Brandes, Arthur Schnitzler: \emph{Ein Briefwechsel}. Hg. Kurt Bergel. Bern: \emph{Francke} 1956, S. 126–127.} }\toendnotes[C]{\smallbreak}\pstart
           \raggedleft{}{\pb}\textcolor{pink}{Kopenhagen}{}\ledrightnote{\textcolor{pink}{Kopenhagen}} (genügende Adresse){\\}13 Juni 20\pend
           \pstart{}Verehrter und lieber Freund \pend\pstart
           Kennen Sie die unverständlichen inneren Hindernisse, die \label{T_L02342_1v}\edtext{es uns}{\lemma{\textnormal{\emph{es uns}}}\Cendnote{\textnormal{mit Hilfe
                  einer Schleife umgestellt aus »uns es«}}}\label{T_L02342_1h} unmöglich machen,
               einen Brief zu schreiben? Es gibt täglich so viel zu thun, dass wenn ein Augenblick
               der geistigen Frische sich einfindet, man es als Pflicht und Notwendigkeit fühlt,
               diesen Augenblick für die Arbeit zu verwenden. Und dann liegt es vielleicht daran,
               dass man tausend Dinge sich zu sagen hätte, und nicht weiss, was herauszugreifen für
               einen elenden Brief. Sie, wie auch unser gemeinsamer Freund \textcolor{blue}{Beer-Hofmann}{}\ledrightnote{\textcolor{blue}{Richard Beer-Hofmann}}, sind mir in einem Menschenalter treu geblieben,
               und ich gebe Ihnen nicht ein Lebenszeichen, nicht einmal wenn Sie mir Ihre Werke
               schenken. Das Lächerliche dabei und das Unglaubliche ist, {\pb}dass ich immer und immer wieder an
               Sie \label{T_L02342_2v}\edtext{dachte}{\lemma{\textnormal{\emph{dachte}}}\Cendnote{\textnormal{das Wort wohl wegen der Lesbarkeit durchgestrichen und erneut
                  über die Zeile geschrieben}}}\label{T_L02342_2h} und mir sagte: An Schnitzler will ich
               schreiben, und kam nicht dazu.\pend
           \pstart
           Ich glaube, dass wir, als \textcolor{blue}{Peter}{}\ledrightnote{\textcolor{blue}{Peter Nansen}} starb, ein Paar
               Briefe wechselten, aber es ist lange her. Er starb Ende Juli 18. Gesehen
               haben wir uns nicht seit December 12, und was ist nicht in der Welt
               geschehen seit jener Zeit!\pend
           \pstart
           Ich weiss ja augenblicklich Nichts über Sie, nicht einmal, ob Sie in \textcolor{pink}{Wien}{}\ledrightnote{\textcolor{pink}{Wien}} weilen, sie haben wol eher Ihre Zuflucht zu irgend einer
               Villa genommen; aber der Brief wird Sie hoffentlich finden.\pend
           \pstart
           In irgend einer Zeitung sah ich mit Freuden, dass \textcolor{green}{\uline{Die Schwestern}}{}\ledrightnote{\textcolor{green}{Die Schwestern oder Casanova in Spa. Lustspiel in Versen}} einen grossen Bühnenerfolg gehabt haben. Ich finde das Stück sehr fein, sehr
               unterhaltend und echt, bin leise erstaunt, {\pb}dass Sie in so trauriger Zeit sich
               den Muth und die Spannkraft bewahrt haben, ein Lustspiel zu schreiben. Ich kann nicht
               glauben, dass was ich über die niederschlagenden Zustände in \textcolor{pink}{Oesterreich}{}\ledrightnote{\textcolor{pink}{Österreich}} erfahren habe, übertrieben sei. Die Wandlung von dem
               Zustand vor dem Krieg zu dem jetzigen ist für uns alle, auch für die früheren
               Neutralen, furchtbar, doch am allermeisten für die bedauernswerthe Städte \textcolor{pink}{Wien}{}\ledrightnote{\textcolor{pink}{Wien}} und \textcolor{pink}{Budapest}{}\ledrightnote{\textcolor{pink}{Budapest}},
                  \textcolor{pink}{Petersburg}{}\ledrightnote{\textcolor{pink}{Sankt Petersburg}} und \textcolor{pink}{Moskau}{}\ledrightnote{\textcolor{pink}{Moskau}}. Die paar \textcolor{pink}{russischen}{}\ledrightnote{\textcolor{pink}{Russland}} Freunde und
               Freundinnen, die ich hatte, sind nach \textcolor{pink}{Constantinopel}{}\ledrightnote{\textcolor{pink}{Istanbul}}
               versprengt, und leben dort in Armuth; in \textcolor{pink}{Deutschland}{}\ledrightnote{\textcolor{pink}{Deutschland}}
               ist Alles unsicher und in Auflösung; in \textcolor{pink}{Frankreich}{}\ledrightnote{\textcolor{pink}{Frankreich}}
               und \textcolor{pink}{England}{}\ledrightnote{\textcolor{pink}{England}} sind mehrere meiner besten Freunde
                  \label{K_L02342_1v}\edtext{Jingo}{\lemma{\textnormal{\emph{Jingo}}}\Cendnote{\textnormal{Ausdruck für übersteigerten englischen Patriotismus.}}}\label{K_L02342_1h}’s
               geworden und aller Vernunft verschlossen. Das grosse Publicum ist dort, wie überall,
               der ewige Dummkopf, der \uline{man} genannt wird! {\pb}Ich hatte hier einen flüchtigen
               aber recht angenehmen Besuch von einem \textcolor{pink}{österreichischen}{}\ledrightnote{\textcolor{pink}{Österreich}} Obersten Namens \textcolor{blue}{\uline{Kreutz}}{}\ledrightnote{\textcolor{blue}{Rudolf Jeremias Kreutz}}, der ein gutes Buch \textcolor{green}{\uline{Die grosse Phrase}}{}\ledrightnote{\textcolor{green}{Die große Phrase}} geschrieben hat, und danach einige weniger gute, oder wiederholende.\pend
           \pstart
           Mein Leben ist einsam; ich arbeite viel, habe wieder nachdem ich die \textcolor{green}{zwei Bände über \textcolor{blue}{\uline{Cäsar}}{}\ledrightnote{\textcolor{blue}{Gaius Iulius Caesar}}}{}\ledrightnote{→\textcolor{green}{Gaius Julius Cæsar}} herausgab, eine grosse \textcolor{green}{Maschine}{}\ledrightnote{→\textcolor{green}{Michelangelo Buonarotti}} in Arbeit; ich bin \introOben{}seit anderthalb Jahren\introOben{}
               in der \textcolor{pink}{italiänischen}{}\ledrightnote{\textcolor{pink}{Italien}} Renaissance vertieft. Ob es
               was wird, weiss ich nicht. Ich habe ja mehrere Altersgrenzen hinter mir.\pend
           \pstart
           \textcolor{blue}{Beer-Hofmann}{}\ledrightnote{\textcolor{blue}{Richard Beer-Hofmann}}s merkwürdige \textcolor{green}{Mysterie}{}\ledrightnote{→\textcolor{green}{Jaákobs Traum. Ein Vorspiel}} verstehe ich als \uline{seine} Antwort auf die immer mehr anschwellende Bewegung des
               Judenhasses in \textcolor{pink}{Europa}{}\ledrightnote{\textcolor{pink}{Europa}}. Diese Bewegung hat auch den
                  \textcolor{pink}{Norden}{}\ledrightnote{\textcolor{pink}{Skandinavien}} erreicht, und mich zum Einsiedler gemacht.
               Früher war ich \textcolor{pink}{Däne}{}\ledrightnote{\textcolor{pink}{Dänemark}} und wurde so aufgefasst;
               plötzlich werde ich Jude genannt, und war es nie. Unmöglich, irgend etwas der \label{K_L02342_2v}\edtext{Krapüle}{\lemma{\textnormal{\emph{Krapüle}}}\Cendnote{\textnormal{französisch crapule: Gesindel}}}\label{K_L02342_2h} verständlich zu machen.\pend
           \pstart
           Ich hoffe, dass es Ihrer Frau \textcolor{blue}{Gemahlin}{}\ledrightnote{→\textcolor{blue}{Olga Schnitzler}} und Ihren \textcolor{blue}{Kindern}{}\ledrightnote{→\textcolor{blue}{Heinrich Schnitzler}{\newline}→\textcolor{blue}{Lili Schnitzler}} nicht übel geht. Ich drücke Ihnen von Herzen die Hand.\pend
           \pstart Ihr \spacefill\mbox{Georg Brandes}\pend{}\endnumbering\briefempfaengerindex{Schnitzler, Arthur@\textsc{Schnitzler, Arthur}!zzzBrandes, Georg@\emph{von Georg Brandes}!1920-06-131@{13. 6. 1920}|)be}\mylabel{h}  \normalsize

\doendnotes{C}
\bigskip
\vfill

\clearpage

\footnotesize

\lohead{\textsc{register}}

% Definiere theindex-Environment komplett neu ohne reledmac
\makeatletter
\renewenvironment{theindex}{%
  \section*{\indexname}%
  \setlength{\parindent}{0pt}%
  \setlength{\parskip}{0pt plus 0.3pt}%
  \let\item\@idxitem
}{%
  \clearpage
}
\makeatother

\IfFileExists{\jobname-pw.ind}{\input{\jobname-pw.ind}}{}

\end{document}

      