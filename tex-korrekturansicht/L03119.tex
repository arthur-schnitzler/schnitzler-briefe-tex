%% latex-korrekturansicht-vorspann.tex
%% Vorspann für die Korrekturansicht.
%% Lädt die gemeinsame Datei latex-vorspann.tex mit gesetztem Schalter.

\newif\ifkorrekturansicht
\korrekturansichttrue

\input{../tex-inputs/latex-vorspann}


\renewcommand{\erwaehntePersonen}{Personen: Richard Beer-Hofmann, Amalie Joachim, Karl Kraus, Marie Pichler, Max Reinhardt, Richard Specht}
\renewcommand{\erwaehnteInstitutionen}{Institutionen: Volkstheater in Rudolfsheim}
\renewcommand{\erwaehnteOrte}{Orte: Café Griensteidl, Café Pfob, Gürtel, Volkstheater in Rudolfsheim, Wien, XV., Rudolfsheim-Fünfhaus}
\renewcommand{\erwaehnteWerke}{Werke: Die Räuber. Ein Schauspiel, Medea. Trauerspiel in fünf Aufzügen}
\section[Felix Salten an Arthur Schnitzler, {[}7. 2. 1893?{]}]{Felix Salten an Arthur Schnitzler, {[}7. 2. 1893?{]}}
\nopagebreak\mylabel{v}
\rehead{ }\normalsize\beginnumbering\briefempfaengerindex{Schnitzler, Arthur@\textsc{Schnitzler, Arthur}!zzzSalten, Felix@\emph{von Felix Salten}!1893-02-071@{{[}7. 2. 1893?{]}}|(be}
\toendnotes[C]{\smallbreak\pagebreak[2]}\Standort{CUL, Schnitzler, B 89, A 1.}
\physDesc{Brief, 1 Blatt, 2 Seiten, 601 Zeichen
\newline{}Handschrift: Bleistift, lateinische Kurrent
\newline{}Schnitzler: mit Bleistift datiert: »92« 
\newline{}Ordnung: mit Bleistift von unbekannter Hand nummeriert: »22« }\toendnotes[C]{\smallbreak}
\pstart
           \noindent{}{\pb}Lieber Freund! Ich habe allerdings eine Verständigung
               erhalten, bin aber nicht sehr aufgelegt \label{K_L03119-1v}\edtext{hinauszufahren}{\lemma{\textnormal{\emph{hinauszufahren}}}\Cendnote{\textnormal{Das \textcolor{pink}{Volkstheater in Rudolfsheim}, um das es hier
                  höchstwahrscheinlich ging, befand sich im \textcolor{pink}{15. Wiener
                     Gemeindebezirk} und damit außerhalb der ›Linie‹ – dem \textcolor{pink}{Gürtel} –, die die inneren Wohnbezirke von den äußeren
                  trennte.}}}\label{K_L03119-1h}, um so mehr als ich eine \label{K_L03119-2v}\edtext{Karte zur \textcolor{blue}{Joachim}{}\ledrightnote{\textcolor{blue}{Amalie Joachim}}}{\lemma{\textnormal{\emph{Karte zur Joachim}}}\Cendnote{\textnormal{Das Korrespondenzstück ist undatiert und
                  von \textcolor{blue}{Schnitzler} nur grob – im Jahr 1892 – verortet. Im Oktober 1892
                  gab \textcolor{blue}{Amalie Joachim} drei Konzerte in \textcolor{pink}{Wien}, am 3.,
                     5. und 7.{ }\textcolor{blue}{Schnitzler} war auf keinem der drei und zu
                  dieser Zeit auch nicht im \emph{\textcolor{brown}{Volkstheater in
                     Rudolfsheim}}. Die nächten drei Auftritte in \textcolor{pink}{Wien} gab \textcolor{blue}{Joachim} am
                     5., 7. und 11. 2. 1892. Da \textcolor{blue}{Schnitzler} am 7. 2. 1893 im \textcolor{pink}{Volkstheater in Rudolfsheim} die Aufführung von \emph{\textcolor{green}{Medea}} besuchte, dürfte dies der Tag dieses Schreibens
                  sein.}}}\label{K_L03119-2h} habe, wovon ich Ihnen auch eine zur Verfügung stellen kann, falls Sie
               doch nicht nach \textcolor{pink}{Rudolfsheim}{}\ledrightnote{\textcolor{pink}{Volkstheater in Rudolfsheim}} fahren.\pend
           
\pstart
           Ich gehe jetzt zu \textcolor{blue}{Beer-Hofmann}{}\ledrightnote{\textcolor{blue}{Richard Beer-Hofmann}} und frage ihn
               was er beschließt. Auf jeden Fall {\pb}haben Sie dann bestimmte
               Nachricht im \textcolor{pink}{Griensteidl}{}\ledrightnote{\textcolor{pink}{Café Griensteidl}} noch vor
                  6 Uhr.\pend
           
\pstart
           Ehrlich, ist mir diese \label{K_L03119-3v}\edtext{\textcolor{blue}{Person}{}\ledrightnote{{$\rightarrow$}\textcolor{blue}{Marie Pichler}}}{\lemma{\textnormal{\emph{Person}}}\Cendnote{\textnormal{Zuletzt waren \textcolor{blue}{Schnitzler} und \textcolor{blue}{Salten} am 14. 1. 1893 im
                  \textcolor{pink}{Volkstheater Rudolfsheim} in der Aufführung von \emph{\textcolor{green}{Die Räuber}}, 
                  an der \textcolor{blue}{Karl Kraus} und \textcolor{blue}{Max Reinhardt} mitwirkten.
                  Wenn es sich bei der »Person« um eine Schauspielerin handeln sollte, dürfte \textcolor{blue}{Marie Pichler}
                  gemeint sein, der einzigen Schauspielerin, die am Theaterzettel von \emph{\textcolor{green}{Die Räuber}} stand.}}}\label{K_L03119-3h} ziemlich
               uninteressant, und glaube ich, dass wir uns ein 2\textsuperscript{tes} Mal
               sehr langweilen werden.\pend
           
\pstart
           Herzlichst Ihr {\\[\baselineskip]}treuer {\\[\baselineskip]}\spacefill\mbox{Salten}\pend
           \leftskip=0em{}
\pstart
           \noindent{}\label{K_L03119-4v}\edtext{\textcolor{blue}{Specht}{}\ledrightnote{\textcolor{blue}{Richard Specht}}, werde ich wegen \textcolor{pink}{Pfob}{}\ledrightnote{\textcolor{pink}{Café Pfob}} avisiren}{\lemma{\textnormal{\emph{Specht, … avisiren}}}\Cendnote{\textnormal{Für
                     diese Zeit ist kein gemeinsamer Besuch im \textcolor{pink}{Café
                        Pfob} belegt.}}}\label{K_L03119-4h}, da er \uline{gewiss} nicht
                  nach \textcolor{pink}{Rdlfshm}{}\ledrightnote{\textcolor{pink}{Volkstheater in Rudolfsheim}} fährt.\pend
           \endnumbering\briefempfaengerindex{Schnitzler, Arthur@\textsc{Schnitzler, Arthur}!zzzSalten, Felix@\emph{von Felix Salten}!1893-02-071@{{[}7. 2. 1893?{]}}|)be}\mylabel{h}  \normalsize

\doendnotes{C}
\bigskip
\vfill

\clearpage

\footnotesize

\lohead{\textsc{register}}

% Definiere theindex-Environment komplett neu ohne reledmac
\makeatletter
\renewenvironment{theindex}{%
  \section*{\indexname}%
  \setlength{\parindent}{0pt}%
  \setlength{\parskip}{0pt plus 0.3pt}%
  \let\item\@idxitem
}{%
  \clearpage
}
\makeatother

\IfFileExists{\jobname-pw.ind}{\input{\jobname-pw.ind}}{}

\end{document}

      