%% latex-korrekturansicht-vorspann.tex
%% Vorspann für die Korrekturansicht.
%% Lädt die gemeinsame Datei latex-vorspann.tex mit gesetztem Schalter.

\newif\ifkorrekturansicht
\korrekturansichttrue

\input{../tex-inputs/latex-vorspann}


\renewcommand{\erwaehntePersonen}{Personen:  ?? [Vater von Peter Dorners Verlobter],  ?? [Verlobte von Peter Dorner], Eduard Bacher, Moriz Benedikt, Peter Dorner, Theodor Herzl, Hugo von Hofmannsthal,  Papst Pius X., Olga Schnitzler,  Wilhelm II. von Preußen}
\renewcommand{\erwaehnteInstitutionen}{Institutionen: H. Gladenbeck {\kaufmannsund}  Sohn Bildgießerei, Neue Freie Presse}
\renewcommand{\erwaehnteOrte}{Orte: Berlin, Dessauer Straße, Tirol, Welsberg-Taisten, Wien}
\renewcommand{\erwaehnteWerke}{Werke: Tagebuch}
\section[ Paul Goldmann an Arthur Schnitzler, 4. 8. {[}1904{]}]{Paul Goldmann an Arthur Schnitzler, 4. 8. {[}1904{]}}
\nopagebreak\mylabel{v}
\rehead{ }\normalsize\beginnumbering\briefempfaengerindex{Schnitzler, Arthur@\textsc{Schnitzler, Arthur}!zzzGoldmann, Paul@\emph{von Paul Goldmann}!1904-08-041@{4. 8. {[}1904{]}}|(be}
\toendnotes[C]{\smallbreak\pagebreak[2]}\Standort{DLA, A:Schnitzler, HS.NZ85.1.3174.}
\physDesc{Brief, 2 Blätter, 6 Seiten, 2217 Zeichen
\newline{}Handschrift: blaue Tinte, deutsche Kurrent
\newline{}Schnitzler: 1) mit Bleistift das Jahr »{[}1{]}904« vermerkt  2) mit rotem Buntstift drei Unterstreichungen}\toendnotes[C]{\smallbreak}
\pstart
           \noindent{}\raggedleft{}{\pb}\textcolor{gray}{\textbf{\textcolor{pink}{DESSAUERSTRASSE 19}{}\ledrightnote{\textcolor{pink}{Dessauer Straße}}}}\pend
           
\pstart
           \textcolor{pink}{Berlin}{}\ledrightnote{\textcolor{pink}{Berlin}}, 4. Auguſt.\pend
           
\pstart{}Mein lieber Freund,\pend
\pstart
           \textsc{\textcolor{blue}{Peter Dorner}{}\ledrightnote{\textcolor{blue}{Peter Dorner}}s} Verlobung mag im
               Zuſammenhang mit ſeinem \label{K_L03449-1v}\edtext{»Ruhm«}{\lemma{\textnormal{\emph{»Ruhm«}}}\Cendnote{\textnormal{\textcolor{blue}{Peter Dorner} war ein Kunstschmied, der
                  wegen seiner Vorliebe, Schlangen darzustellen, als »Schlangenschmied von \textcolor{pink}{Welsberg}« bekannt war. Am
                     28. 4. 1904 hatte er erstmals in \textcolor{pink}{Berlin} seine Arbeiten ausgestellt, in der \emph{\textcolor{brown}{Gießerei Gladenbeck}}.}}}\label{K_L03449-1h} ſtehen. Sicherlich aber macht er eine \label{K_L03449-2v}\edtext{»\textcolor{blue}{gute Parthie}{}\ledrightnote{{$\rightarrow$}\textcolor{blue}{?? [Verlobte von Peter Dorner]}}«}{\lemma{\textnormal{\emph{»gute Parthie«}}}\Cendnote{\textnormal{nicht ermittelt
                  }}}\label{K_L03449-2h}.
               Das Haus des \textcolor{blue}{Schwiegervater}{}\ledrightnote{{$\rightarrow$}\textcolor{blue}{?? [Vater von Peter Dorners Verlobter]}}s auf der Karte, die er mir geſchickt hat, \strikeout{ſpricht} läßt das mit aller Deutlichkeit erkennen. Ich
               habe dieſen \textsc{\textcolor{blue}{Peter Dorner}{}\ledrightnote{\textcolor{blue}{Peter Dorner}}}, den »weltverlorenen \strikeout{\textcolor{gray}{×}} Bauern«{[},{]}{ }\textcolor{pink}{hier}{}\ledrightnote{{$\rightarrow$}\textcolor{pink}{Berlin}} als einen \strikeout{G\textcolor{gray}{e} Ge\textcolor{gray}{f}\textcolor{gray}{×}\-\textcolor{gray}{×}\-\textcolor{gray}{×}\-\textcolor{gray}{×}\-\textcolor{gray}{×}\-\textcolor{gray}{×}} Gefühlsmann kennen gelernt, der die geriſſenſten Börſenjuden übertrifft.\pend
           
\pstart
           Sehr bedauert habe ich, zu erfahren, {\pb} daß du \strikeout{\textcolor{gray}{×}\-\textcolor{gray}{×}\-\textcolor{gray}{×}}\introOben{}14\introOben{} Tage \label{K_L03449-3v}\edtext{krank}{\lemma{\textnormal{\emph{krank}}}\Cendnote{\textnormal{vgl. \emph{\textcolor{green}{Tagebuch}}}}}\label{K_L03449-3h} warſt. Hoffentlich haſt Du, außer einiger »Gelbheit«, keine großen
               Beſchwerden gehabt, und ich freue mich, daß Du wiederhergeſtellt und arbeitsluſtig
               und arbeitskräftig biſt.\pend
           
\pstart
           Der \label{K_L03449-4v}\edtext{Tod \textsc{\textcolor{blue}{Herzl}{}\ledrightnote{\textcolor{blue}{Theodor Herzl}}s}}{\lemma{\textnormal{\emph{Tod Herzls}}}\Cendnote{\textnormal{\textcolor{blue}{Theodor Herzl} war am 3. 7. 1904 an Herzleiden verstorben.}}}\label{K_L03449-4h} hat auch
               mich ſehr ergriffen. Er war der Anſtändigſten und Begabteſten \strikeout{e\textcolor{gray}{i}} einer, und \strikeout{\textcolor{gray}{×}} man ſchätzt ihn umſo höher, wenn man \strikeout{bed\textcolor{gray}{enkt, was nac}h} ihn mit dem Nachwuchs vergleicht.
               Nur was ſeinen \label{K_L03449-5v}\edtext{zioniſtiſchen
                  Lebensplan}{\lemma{\textnormal{\emph{zioniſtiſchen
                  Lebensplan}}}\Cendnote{\textnormal{siehe zu \textcolor{blue}{Goldmann}s Ablehnung gegenüber \textcolor{blue}{Herzl}s zionistischen Visionen etwa Paul Goldmann an Arthur Schnitzler, 29. 7. [1895], 1. 4. [1896] und 7. 3. [1898]}}}\label{K_L03449-5h} anlangt, ſo iſt {\pb}er, glaube ich, zur rechten
               Zeit geſtorben. Denn die Bewegung ſtand, wie ich höre, am Vorabend ſchwerer \label{K_L03449-6v}\edtext{Kriſen}{\lemma{\textnormal{\emph{Kriſen}}}\Cendnote{\textnormal{womöglich Bezug auf die wiederholte Ablehnung eines jüdischen
                  Staats durch Autoritäten wie \textcolor{blue}{Papst Pius X.}
                  und \textcolor{blue}{Kaiser Wilhelm II.}}}}\label{K_L03449-6h}.\pend
           
\pstart
           Daß ich ſein \label{K_L03449-7v}\edtext{\textcolor{brown}{Nachfolger}{}\ledrightnote{{$\rightarrow$}\textcolor{brown}{Neue Freie Presse}}}{\lemma{\textnormal{\emph{Nachfolger}}}\Cendnote{\textnormal{als Feuilletonredakteur der \emph{\textcolor{brown}{Neuen Freien Presse}}}}}\label{K_L03449-7h} werde, halte ich für ausgeſchloſſen. Die \textcolor{blue}{Herausgeber}{}\ledrightnote{{$\rightarrow$}\textcolor{blue}{Eduard Bacher}{\newline}{$\rightarrow$}\textcolor{blue}{Moriz Benedikt}} machen keine Anſtalten, mir die
               Stellung anzubieten, und ich habe nicht die Abſicht, mich darum zu bewerben, da die
               Stellung mir nicht die Freiheit gewährt, zu leiſten, was ich leiſten möchte, und da
               außerdem meine Luſt, nach \textcolor{pink}{Wien}{}\ledrightnote{\textcolor{pink}{Wien}} zurückzukehren,
               immer geringer wird.\pend
           
\pstart
           {\pb}Meine \label{K_L03449-8v}\edtext{Äußerung}{\lemma{\textnormal{\emph{Äußerung}}}\Cendnote{\textnormal{siehe Paul Goldmann an Arthur Schnitzler, 23. 6. [1904] und Hugo von Hofmannsthal an Arthur Schnitzler, 1[9?]. 6. [1904]}}}\label{K_L03449-8h} über \textsc{\textcolor{blue}{Hoffmannsthal}{}\ledrightnote{\textcolor{blue}{Hugo von Hofmannsthal}}} haſt Du wieder einmal mißverſtanden. Mich hat es nicht überraſcht, daß Du die
               Fehler, die Deine Freunde begehen, offen als ſolche bezeichneſt (ich kenne Deine
               Offenheit ſehr wohl und ſchätze ſie ſehr hoch), ſondern mich hat es überraſcht, daß
               Du einen Fehler \textsc{\textcolor{blue}{Hoffmannsthal}{}\ledrightnote{\textcolor{blue}{Hugo von Hofmannsthal}}s} als ſolchen erkannt
               haſt, da Du ſonſt, meiner Anſicht nach, \textsc{\textcolor{blue}{Hoffmannsthal}{}\ledrightnote{\textcolor{blue}{Hugo von Hofmannsthal}}}{ }{\pb}nicht richtig beurtheilſt. Im Übrigen überraſcht
               mich wieder der Ausdruck »Eſelei«, den Du gebrauchſt. Jemanden, der ſich abfällig
               über einen Schriftſteller geäußert hat und dieſe Äußerung dann beſtreitet, \strikeout{nenn} nenne ich nicht einen Eſel, ſondern einen
               Lügner.\pend
           
\pstart
           Ich trete Ende dieſer Woche meinen Urlaub an. Wohin ich gehe, weiß {\pb}ich immer noch nicht. \strikeout{Wa} Wahrſcheinlich gehe ich nach \textcolor{pink}{Tirol}{}\ledrightnote{\textcolor{pink}{Tirol}},
               über \textcolor{pink}{Wien}{}\ledrightnote{\textcolor{pink}{Wien}}, und in dieſem Falle werde ich ſehr
               freuen, Dir \label{K_L03449-9v}\edtext{nächſte Woche}{\lemma{\textnormal{\emph{nächſte Woche}}}\Cendnote{\textnormal{\textcolor{blue}{Goldmann} war jedenfalls am 10. 8. 1904 und am 11. 8. 1904 in \textcolor{pink}{Wien}. Am 11. 8. 1904 besuchte er \textcolor{blue}{Arthur und Olga Schnitzler}.}}}\label{K_L03449-9h} die Hand zu drücken.\pend
           
\pstart
           Herzliche Grüße Dir und Deiner \textcolor{blue}{Frau}{}\ledrightnote{{$\rightarrow$}\textcolor{blue}{Olga Schnitzler}}! {\\[\baselineskip]}Dein {\\[\baselineskip]}\spacefill\mbox{Paul Goldmn}\pend
           \leftskip=0em{}\endnumbering\briefempfaengerindex{Schnitzler, Arthur@\textsc{Schnitzler, Arthur}!zzzGoldmann, Paul@\emph{von Paul Goldmann}!1904-08-041@{4. 8. {[}1904{]}}|)be}\mylabel{h}
\begin{anhang}
\end{anhang}\normalsize

\doendnotes{C}
\bigskip
\vfill

\clearpage

\footnotesize

\lohead{\textsc{register}}

% Definiere theindex-Environment komplett neu ohne reledmac
\makeatletter
\renewenvironment{theindex}{%
  \section*{\indexname}%
  \setlength{\parindent}{0pt}%
  \setlength{\parskip}{0pt plus 0.3pt}%
  \let\item\@idxitem
}{%
  \clearpage
}
\makeatother

\IfFileExists{\jobname-pw.ind}{\input{\jobname-pw.ind}}{}

\end{document}

      