%% latex-korrekturansicht-vorspann.tex
%% Vorspann für die Korrekturansicht.
%% Lädt die gemeinsame Datei latex-vorspann.tex mit gesetztem Schalter.

\newif\ifkorrekturansicht
\korrekturansichttrue

\input{../tex-inputs/latex-vorspann}


\renewcommand{\erwaehntePersonen}{Personen: Felix Salten, Ottilie Salten}
\renewcommand{\erwaehnteOrte}{Orte: Heiligenstadt, Wien}
\renewcommand{\erwaehnteWerke}{Werke: Andreas Thameyers letzter Brief, Das Schicksal des Freiherrn von Leisenbohg. Novellette, Das neue Lied, Dämmerseele, Dämmerseelen. Novellen}
\section[ Felix Salten an Arthur Schnitzler, 10. 3. 1907]{Felix Salten an Arthur Schnitzler, 10. 3. 1907}
\nopagebreak\mylabel{v}
\rehead{ }\normalsize\beginnumbering\briefempfaengerindex{Schnitzler, Arthur@\textsc{Schnitzler, Arthur}!zzzSalten, Felix@\emph{von Felix Salten}!1907-03-101@{10. 3. 1907}|(be}
\toendnotes[C]{\smallbreak\pagebreak[2]}\Standort{CUL, Schnitzler, B 89, B 1.}
\physDesc{Brief, 1 Blatt, 1 Seite, 618 Zeichen
\newline{}Handschrift: schwarze Tinte, lateinische Kurrent
\newline{}Ordnung: mit Bleistift von unbekannter Hand nummeriert: »227« }\toendnotes[C]{\smallbreak}
\pstart
           \raggedleft{}{\pb}\textcolor{pink}{Heiligenstadt}{}\ledrightnote{\textcolor{pink}{Heiligenstadt}}. 10. III. 07\pend
           
\pstart
           Lieber, danke schön für Ihr neues \label{K_L03436-1v}\edtext{\textcolor{green}{Buch}{}\ledrightnote{{$\rightarrow$}\textcolor{green}{Dämmerseelen. Novellen}}}{\lemma{\textnormal{\emph{Buch}}}\Cendnote{\textnormal{siehe Arthur Schnitzler: Widmungsexemplar Dämmerseelen für Felix
               Salten, 2. 3. 1907}}}\label{K_L03436-1h}. Es kam heute{ }früh, ich hab es vormittag gleich gelesen und es hat wieder
               sehr auf mich gewirkt. Am meisten der \textcolor{green}{Leisenbohg}{}\ledrightnote{\textcolor{green}{Das Schicksal des Freiherrn von Leisenbohg. Novellette}}
               und der \textcolor{green}{Thameyer}{}\ledrightnote{\textcolor{green}{Andreas Thameyers letzter Brief}}. Dann noch »\textcolor{green}{die Fremde}{}\ledrightnote{\textcolor{green}{Dämmerseele}}«. Gegen das »\textcolor{green}{neue
                  Lied}{}\ledrightnote{\textcolor{green}{Das neue Lied}}« hätte ich einiges zu sagen. Zunächst scheint mir das Anekdotische darin
               nicht ganz überwunden. Ein Roman, dessen Art aus dem \textcolor{green}{Leisenbohg}{}\ledrightnote{\textcolor{green}{Das Schicksal des Freiherrn von Leisenbohg. Novellette}}, der \textcolor{green}{Fremden}{}\ledrightnote{\textcolor{green}{Dämmerseele}}, und \textcolor{green}{Thameyer}{}\ledrightnote{\textcolor{green}{Andreas Thameyers letzter Brief}} sich zusammensetzte, der diese Farben
               und Schatten brächte, müßte etwas ganz Unvergleichliches sein. Hoffentlich \label{K_L03436-2v}\edtext{sehen wir uns bald}{\lemma{\textnormal{\emph{sehen wir uns bald}}}\Cendnote{\textnormal{Nachweisbar sahen sich die beiden am 16. 3. 1907
                  wieder.}}}\label{K_L03436-2h}. Es ist noch manches über das \textcolor{green}{Buch}{}\ledrightnote{{$\rightarrow$}\textcolor{green}{Dämmerseelen. Novellen}} zu sagen.\pend
           
\pstart
           Viele Grüße von \textcolor{blue}{uns}{}\ledrightnote{{$\rightarrow$}\textcolor{blue}{Ottilie Salten}} zu
               Ihnen. {\\[\baselineskip]}Ihr {\\[\baselineskip]}\spacefill\mbox{Felix Salten}\pend
           \leftskip=0em{}\endnumbering\briefempfaengerindex{Schnitzler, Arthur@\textsc{Schnitzler, Arthur}!zzzSalten, Felix@\emph{von Felix Salten}!1907-03-101@{10. 3. 1907}|)be}\mylabel{h}  \normalsize

\doendnotes{C}
\bigskip
\vfill

\clearpage

\footnotesize

\lohead{\textsc{register}}

% Definiere theindex-Environment komplett neu ohne reledmac
\makeatletter
\renewenvironment{theindex}{%
  \section*{\indexname}%
  \setlength{\parindent}{0pt}%
  \setlength{\parskip}{0pt plus 0.3pt}%
  \let\item\@idxitem
}{%
  \clearpage
}
\makeatother

\IfFileExists{\jobname-pw.ind}{\input{\jobname-pw.ind}}{}

\end{document}

      