%% latex-korrekturansicht-vorspann.tex
%% Vorspann für die Korrekturansicht.
%% Lädt die gemeinsame Datei latex-vorspann.tex mit gesetztem Schalter.

\newif\ifkorrekturansicht
\korrekturansichttrue

\input{../tex-inputs/latex-vorspann}


               \section[Karl Kraus an Arthur Schnitzler, 27. 2. 1893]{ Karl Kraus an Arthur Schnitzler, 27. 2. 1893}\nopagebreak\mylabel{v}\rehead{ }\normalsize\beginnumbering\briefempfaengerindex{Schnitzler, Arthur@\textsc{Schnitzler, Arthur}!zzzKraus, Karl@\emph{von Karl Kraus}!1893-02-271@{27. 2. 1893}|(be} \toendnotes[C]{\smallbreak\pagebreak[2]} \Standort{CUL, Schnitzler, B 55.}
\physDesc{Postkarte
\newline{}Handschrift: Bleistift, deutsche Kurrent\newline{}Versand: 1) Stempel: »\nobreak{}\oindex{Berlin@\textbf{Berlin}, \emph{https://www.geonames.org/ontologyP.PPLC}|pwk}Berlin. N.W. 66, 27/02 93, 3–4 N\nobreak{}«.  2) Stempel: »\nobreak{}Wien 1/1, 28. 2. 93, 5–6½ N\nobreak{}«. }\buchAbdrucke{\weitereDrucke{\emph{Karl Kraus und Arthur Schnitzler. Eine Dokumentation.} Hg. Reinhard Urbach. In: \emph{Literatur und Kritik}, Bd. 49, Oktober 1970, S. 515.} }\toendnotes[C]{\smallbreak}\pstart{}{\pb}Herrn Schriftſteller\pend{}\pstart{}D\textsuperscript{r} Arthur Schnitzler,\pend{}\pstart{}\textcolor{pink}{Wien I}{}\ledrightnote{\textcolor{pink}{I., Innere Stadt}}\pend{}\pstart{}\textcolor{pink}{Grillparzerſtr 7}{}\ledrightnote{\textcolor{pink}{Grillparzerstraße}}\pend{}{\bigskip}\pstart
           {\pb}\textcolor{pink}{Berlin}{}\ledrightnote{\textcolor{pink}{Berlin}}, Montag, \textcolor{gray}{2}7/2 93, \textcolor{pink}{Restaurant Schultheiß}{}\ledrightnote{\textcolor{pink}{Schultheiß}}.\pend
           \pstart
           Liebſter Doctor! Mir geht’s hier famos! Geſtern war Matinée im
                        »\textcolor{pink}{Neuen Theater}{}\ledrightnote{\textcolor{pink}{Neues Theater}}«: »\textcolor{brown}{Freie Bühne}{}\ledrightnote{\textcolor{brown}{Freie Bühne}}« – \textcolor{green}{\uuline{Weber}}{}\ledrightnote{\textcolor{green}{Die Weber. Schauspiel aus den vierziger Jahren}}! \uuline{Colossaler} Erfolg. \textcolor{blue}{Hauptmann}{}\ledrightnote{\textcolor{blue}{Gerhart Hauptmann}} war ganz glückſeelig. Im »\textcolor{brown}{Magazin}{}\ledrightnote{\textcolor{brown}{Magazin für die Literatur des Auslandes}}« (25. Feber) iſt von mir ein
                        \label{K_L00183_1v}\edtext{\textcolor{green}{Artikel}{}\ledrightnote{→\textcolor{green}{Wiener Lyriker}}}{\lemma{\textnormal{\emph{Artikel}}}\Cendnote{\textnormal{\textcolor{blue}{Karl Kraus}: \emph{\textcolor{green}{Wiener Lyriker. »Sensationen« von Felix Dörmann (Wien: L. Weiß) und »Gedichte« von Richard
                        Specht (München: Seitz {\kaufmannsund}
                        Schauer)}}. In: \emph{\textcolor{green}{Das Magazin für
                                Litteratur}}, Jg. 62, Nr. 8, 25. 1. 1893,
                            S. 128.}}}\label{K_L00183_1h} über \textcolor{blue}{Dörmann}{}\ledrightnote{\textcolor{blue}{Felix Dörmann}} und \textcolor{blue}{Specht}{}\ledrightnote{\textcolor{blue}{Richard Specht}}. Jetzt geh ich mir das Honorar
                    eincaſſieren.\pend
           \pstart
           Ach, in \textcolor{pink}{Berlin}{}\ledrightnote{\textcolor{pink}{Berlin}} ist’s herrlich!! Grüßen Sie mir den \textcolor{blue}{\uline{Salten}}{}\ledrightnote{\textcolor{blue}{Felix Salten}} u D\textsuperscript{r}{ }\textcolor{blue}{\uline{Beer-Hofmann}}{}\ledrightnote{\textcolor{blue}{Richard Beer-Hofmann}}; \textcolor{blue}{Dörmann}{}\ledrightnote{\textcolor{blue}{Felix Dörmann}}, \textcolor{blue}{Fannjungs}{}\ledrightnote{\textcolor{blue}{Leo Van-Jung}{\newline}\textcolor{blue}{Boris Van-Jung}}, \textcolor{blue}{Fiſcher}{}\ledrightnote{\textcolor{blue}{Georg Fischer}{\newline}\textcolor{blue}{Robert Fischer}} etc. ganz
                        \textcolor{pink}{Grienſteidl}{}\ledrightnote{\textcolor{pink}{Café Griensteidl}}. Ja, wenn ich hier Ihr
                        »\textcolor{green}{\uline{Märchen}}{}\ledrightnote{\textcolor{green}{Das Märchen. Schauspiel in drei Aufzügen}}« im \textcolor{brown}{Leſſingtheater}{}\ledrightnote{\textcolor{brown}{Lessing-Theater}}{ }ſehen könnte!
                    Viele Grüße\pend
           \pstart Ihr \spacefill\mbox{Karl Kraus}\pend{}\pstart
           \noindent{}\strikeout{p. A.}{ }\textcolor{pink}{Berlin S. O. Waldemarstr 3}{}\ledrightnote{\textcolor{pink}{Waldemarstraße}}/\textsuperscript{II} p. A. \textcolor{blue}{Carl
                            Buſſe}{}\ledrightnote{\textcolor{blue}{Carl Busse}}. Schreiben Sie bald!\pend
           \endnumbering\briefempfaengerindex{Schnitzler, Arthur@\textsc{Schnitzler, Arthur}!zzzKraus, Karl@\emph{von Karl Kraus}!1893-02-271@{27. 2. 1893}|)be}\mylabel{h}  \normalsize

\doendnotes{C}
\bigskip
\vfill

\clearpage

\footnotesize

\lohead{\textsc{register}}

% Definiere theindex-Environment komplett neu ohne reledmac
\makeatletter
\renewenvironment{theindex}{%
  \section*{\indexname}%
  \setlength{\parindent}{0pt}%
  \setlength{\parskip}{0pt plus 0.3pt}%
  \let\item\@idxitem
}{%
  \clearpage
}
\makeatother

\IfFileExists{\jobname-pw.ind}{\input{\jobname-pw.ind}}{}

\end{document}

      