%% latex-korrekturansicht-vorspann.tex
%% Vorspann für die Korrekturansicht.
%% Lädt die gemeinsame Datei latex-vorspann.tex mit gesetztem Schalter.

\newif\ifkorrekturansicht
\korrekturansichttrue

\input{../tex-inputs/latex-vorspann}


\renewcommand{\erwaehntePersonen}{Personen:  ?? [Hausmeister von Felix Salten in der Kochgasse 1901], Paul Goldmann, Alfred Kerr, Marcell Salzer, Frank Wedekind}
\renewcommand{\erwaehnteInstitutionen}{Institutionen: Jung-Wiener Theater zum Lieben Augustin}
\renewcommand{\erwaehnteOrte}{Orte: Berlin, Theater an der Wien, Wien}
\renewcommand{\erwaehnteWerke}{}
\section[ Felix Salten an Arthur Schnitzler, 21. 9. 1901]{Felix Salten an Arthur Schnitzler, 21. 9. 1901}
\nopagebreak\mylabel{v}
\rehead{ }\normalsize\beginnumbering\briefempfaengerindex{Schnitzler, Arthur@\textsc{Schnitzler, Arthur}!zzzSalten, Felix@\emph{von Felix Salten}!1901-09-212@{21. 9. 1901}|(be}
\toendnotes[C]{\smallbreak\pagebreak[2]}\Standort{CUL, Schnitzler, B 89, A 2.}
\physDesc{Briefkarte, 444 Zeichen
\newline{}Handschrift: schwarze Tinte, lateinische Kurrent
\newline{}Ordnung: mit Bleistift von unbekannter Hand nummeriert: »143« }\toendnotes[C]{\smallbreak}
\pstart
           \noindent{}{\pb}\textcolor{gray}{\textbf{\textcolor{brown}{Jung-Wiener Theater}{}\ledrightnote{\textcolor{brown}{Jung-Wiener Theater zum Lieben Augustin}}}}\hfill \substVorne{}\textsuperscript{\textcolor{gray}{\textbf{\textcolor{pink}{Wien}{}\ledrightnote{\textcolor{pink}{Wien}}}}}\substDazwischen{}\textcolor{pink}{Berlin}{}\ledrightnote{\textcolor{pink}{Berlin}}\substHinten{}\textcolor{gray}{\textbf{,}}{ }21. Septemb. \textcolor{gray}{\textbf{190}}1\pend
           
\pstart
           \textcolor{gray}{\textbf{\textcolor{brown}{Zum lieben Augustin}{}\ledrightnote{\textcolor{brown}{Jung-Wiener Theater zum Lieben Augustin}}.}}\hfill \textcolor{gray}{\textbf{(\textcolor{pink}{Theater a. d.
                        Wien}{}\ledrightnote{\textcolor{pink}{Theater an der Wien}})}}\pend
           
\pstart
           \textcolor{gray}{\textbf{Direction.}}\pend
           
\pstart
           Lieber Freund, bin seit einigen Tagen hier, und werde nach meiner
               Rückkehr das \label{K_L03319-1v}\edtext{verl. Manuscript}{\lemma{\textnormal{\emph{verl. Manuscript}}}\Cendnote{\textnormal{siehe Arthur Schnitzler an Felix Salten, 16. 9. 1901}}}\label{K_L03319-1h} zum \textcolor{blue}{Hausbesorger}{}\ledrightnote{{$\rightarrow$}\textcolor{blue}{?? [Hausmeister von Felix Salten in der Kochgasse 1901]}}
               legen. Da ich bis jetzt krank und ziemlich unmöglich war habe ich weder \textcolor{blue}{Goldmann}{}\ledrightnote{\textcolor{blue}{Paul Goldmann}} noch {\pb}\textcolor{blue}{Kerr}{}\ledrightnote{\textcolor{blue}{Alfred Kerr}} bisher aufgesucht. \textcolor{blue}{Wedekind}{}\ledrightnote{\textcolor{blue}{Frank Wedekind}} hat mir eben \label{K_L03319-2v}\edtext{für \textcolor{brown}{\textcolor{pink}{Wien}{}\ledrightnote{\textcolor{pink}{Wien}}}{}\ledrightnote{{$\rightarrow$}\textcolor{brown}{Jung-Wiener Theater zum Lieben Augustin}}}{\lemma{\textnormal{\emph{für Wien}}}\Cendnote{\textnormal{für das \emph{\textcolor{brown}{Jung-Wiener Theater zum Lieben Augustin}}}}}\label{K_L03319-2h} zugesagt. \textcolor{pink}{Hier}{}\ledrightnote{{$\rightarrow$}\textcolor{pink}{Berlin}} werde
               ich wol kaum etwas finden. Das ist ein Niveau hier – ganz unwahrscheinlich. Und \textcolor{blue}{Salzer}{}\ledrightnote{\textcolor{blue}{Marcell Salzer}} nicht das Schlimmste dabei!! Donnerstag bin ich wieder in \textcolor{pink}{Wien}{}\ledrightnote{\textcolor{pink}{Wien}}.\pend
           
\pstart
           Herzlichst Ihr {\\[\baselineskip]}\spacefill\mbox{Salten}\pend
           \leftskip=0em{}\endnumbering\briefempfaengerindex{Schnitzler, Arthur@\textsc{Schnitzler, Arthur}!zzzSalten, Felix@\emph{von Felix Salten}!1901-09-212@{21. 9. 1901}|)be}\mylabel{h}  \normalsize

\doendnotes{C}
\bigskip
\vfill

\clearpage

\footnotesize

\lohead{\textsc{register}}

% Definiere theindex-Environment komplett neu ohne reledmac
\makeatletter
\renewenvironment{theindex}{%
  \section*{\indexname}%
  \setlength{\parindent}{0pt}%
  \setlength{\parskip}{0pt plus 0.3pt}%
  \let\item\@idxitem
}{%
  \clearpage
}
\makeatother

\IfFileExists{\jobname-pw.ind}{\input{\jobname-pw.ind}}{}

\end{document}

      