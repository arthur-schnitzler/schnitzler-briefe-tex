%% latex-korrekturansicht-vorspann.tex
%% Vorspann für die Korrekturansicht.
%% Lädt die gemeinsame Datei latex-vorspann.tex mit gesetztem Schalter.

\newif\ifkorrekturansicht
\korrekturansichttrue

\input{../tex-inputs/latex-vorspann}


\renewcommand{\erwaehntePersonen}{Personen: Lou Andreas-Salomé, Richard Beer-Hofmann, Siegfried Bing, Emma Fr., Vincent van Gogh, Paul Goldmann, Hugo von Hofmannsthal, Franz Irresberger, Karl Kraus, Charlotte Pohl-Glas, Adele Sandrock}
\renewcommand{\erwaehnteOrte}{Orte: Asien, Bad Ischl, Gmunden, Japan, München, Paris, Salzburg, Schafberg (St. Gilgen), Unterach am Attersee, Wien, Österreichischer Hof}
\renewcommand{\erwaehnteWerke}{Werke: Erinnerungen, Ischler Brief. (Wiener Dichter auf der Esplanade.), Wiener Familien-Journal}
\section[ Felix Salten an Arthur Schnitzler, 30. 8. 1895]{Felix Salten an Arthur Schnitzler, 30. 8. 1895}
\nopagebreak\mylabel{v}
\rehead{ }\normalsize\beginnumbering\briefempfaengerindex{Schnitzler, Arthur@\textsc{Schnitzler, Arthur}!zzzSalten, Felix@\emph{von Felix Salten}!1895-08-301@{30. 8. 1895}|(be}
\toendnotes[C]{\smallbreak\pagebreak[2]}\Standort{CUL, Schnitzler, B 89, A 1.}
\physDesc{Brief, 1 Blatt, 3 Seiten, 1772 Zeichen
\newline{}Handschrift: schwarze Tinte, lateinische Kurrent
\newline{}Ordnung: mit Bleistift von unbekannter Hand nummeriert: »64« }\toendnotes[C]{\smallbreak}
\pstart
           \noindent{}\centering{}{\pb}\textcolor{gray}{\textbf{\textsc{\textcolor{pink}{Hôtel oesterreichischer Hof}{}\ledrightnote{\textcolor{pink}{Österreichischer Hof}}}.}}\pend
           
\pstart
           \noindent{}\centering{}\textcolor{gray}{\textbf{\textcolor{blue}{Franz Irresberger}{}\ledrightnote{\textcolor{blue}{Franz Irresberger}}}}\pend
           
\pstart
           \noindent{}\centering{}\textcolor{gray}{\textbf{\textcolor{pink}{SALZBURG}{}\ledrightnote{\textcolor{pink}{Salzburg}}.}}\pend
           
\pstart
           \raggedleft{}30. VIII. 95\pend
           
\pstart
           Lieber Freund, ich habe bei meiner Ankunft nur \uline{die Hälfte} des so bestimmt erwarteten \label{K_L03164-1v}\edtext{Betrages}{\lemma{\textnormal{\emph{Betrages}}}\Cendnote{\textnormal{Offenbar
                  hatte \textcolor{blue}{Salten} Geld von \textcolor{blue}{Schnitzler} geliehen und konnte es nicht rechtzeitig
                  zurückzahlen.}}}\label{K_L03164-1h} erhalten und auf meine telegrafische Urgenz ist bis jetzt
               noch nichts eingelangt, so dass ich wegen der Rückreise selbst in arger Verlegenheit
               bin. Seien Sie mir deshalb nicht böse, wenn in der Sache eine Verzögerung von einigen
               Tagen eintritt, ich empfinde das ohnedies peinlich genug und leide darunter, dass
               auch unsere \label{K_L03164-2v}\edtext{2\textsuperscript{te} Bicycle Tour}{\lemma{\textnormal{\emph{2\textsuperscript{te} … Tour}}}\Cendnote{\textnormal{Schon zwischen
                     24. 8. 1895 und
                     27. 8. 1895
                  hatten \textcolor{blue}{Salten} und \textcolor{blue}{Schnitzler} eine Radtour von \textcolor{pink}{Salzburg} nach \textcolor{pink}{München}
                  unternommen.}}}\label{K_L03164-2h} mit einem solchen Nachspiel endet. Sollte ich aber heute oder morgen noch das
               Erhoffte bekommen, dann sende ich es Ihnen \uline{sofort}, wo
               nicht, \uline{gleich} nach meiner Rückkehr nach \textcolor{pink}{Wien}{}\ledrightnote{\textcolor{pink}{Wien}}. Das ist ganz sicher.\pend
           
\pstart
           \textcolor{blue}{L.}{}\ledrightnote{\textcolor{blue}{Charlotte Pohl-Glas}} kam \textcolor{pink}{hier}{}\ledrightnote{\textcolor{pink}{Salzburg}}
               an voll Erbitterung und ich lebe schwere Tage. Irgend {\pb}ein Mensch, – wer, das bringe
               ich noch nicht heraus, – hat ihr in \textcolor{pink}{Gmunden}{}\ledrightnote{\textcolor{pink}{Gmunden}} oder
                  \textcolor{pink}{Ischl}{}\ledrightnote{\textcolor{pink}{Bad Ischl}} erzählt, dass ich das erste mal in \textcolor{pink}{Ischl}{}\ledrightnote{\textcolor{pink}{Bad Ischl}} war, ferner, dass ich voriges Jahr, als sie
               hieherkam, \label{K_L03164-3v}\edtext{auch in \textcolor{pink}{Ischl}{}\ledrightnote{\textcolor{pink}{Bad Ischl}} gewesen}{\lemma{\textnormal{\emph{auch in Ischl gewesen}}}\Cendnote{\textnormal{In
                  seinen \emph{\textcolor{green}{Erinnerungen}} schreibt \textcolor{blue}{Salten}: »Ich beredete meinen damaligen
                     Freund \textcolor{blue}{Karl Kraus}, dass er \textcolor{blue}{Lotte} an der Bahn erwarte und ihr sagen
                     solle, ich sei über eine \textcolor{pink}{Schafbergpartie}
                     nach \textcolor{pink}{Unterach} und von dort auf dem Wege
                     nach \textcolor{pink}{Wien}. Womöglich solle er sie zur
                     sofortigen Abreise nach \textcolor{pink}{Wien} überreden. \textcolor{blue}{Kraus} war gerne dazu bereit und ich
                     wollte mich von Lottens Abreise persönlich überzeugen. Damals gab es in \textcolor{pink}{Ischl} noch Sänften. Also liess ich mich in
                     einer Sänfte auf den Perron tragen, liess dort die Sänfte an die Wand stellen
                     und beobachtete durch einen Vorhangsspalt, wie \textcolor{blue}{Lotte} von \textcolor{blue}{Kraus} überredet den
                     Zug bestieg um wieder nach \textcolor{pink}{Wien} zu fahren.
                     Als der Zug abgegangen war verliess ich die Sänfte und erschien vor dem
                     entgeistert dastehenden \textcolor{blue}{Kraus}.«
                     (\emph{Wienbibliothek im Rathaus}, Nachlass Salten, ZPH 1681/1 1.1.1.9.1, [S. 47].)}}}\label{K_L03164-3h}, hat ihr sonst
               allerhand Geschichten von Frau \label{K_L03164-4v}\edtext{\textcolor{blue}{Fr.}{}\ledrightnote{\textcolor{blue}{Emma Fr.}} ferner von Frl. \textcolor{blue}{S.}{}\ledrightnote{\textcolor{blue}{Adele Sandrock}}}{\lemma{\textnormal{\emph{Fr. ferner von Frl. S.}}}\Cendnote{\textnormal{\textcolor{blue}{Emma Fr.} und \textcolor{blue}{Adele Sandrock}, die beide mit \textcolor{blue}{Salten} in einer intimen Beziehung standen}}}\label{K_L03164-4h} erzählt, –
               kurz, Sie können sich denken wie das arme \textcolor{blue}{Mädel}{}\ledrightnote{{$\rightarrow$}\textcolor{blue}{Charlotte Pohl-Glas}} zugerichtet war. So hatte ich hier zu thun, und habe es
               noch, um alles wieder ins Gleichgewicht zu bringen.\pend
           
\pstart
           Außerdem hat man ihr erzählt, wir seien in \textcolor{pink}{Salzburg}{}\ledrightnote{\textcolor{pink}{Salzburg}} mit einer \label{K_L03164-5v}\edtext{»\textcolor{blue}{jungen chic\substVorne{}\textsuperscript{k}\substDazwischen{}e\substHinten{}n Blondine}{}\ledrightnote{{$\rightarrow$}\textcolor{blue}{Lou Andreas-Salomé}}«}{\lemma{\textnormal{\emph{»jungen chicen Blondine«}}}\Cendnote{\textnormal{\textcolor{blue}{Lou Andreas-Salomé}, siehe A. S.: \emph{Tagebuch}, 20. 8. 1895.}}}\label{K_L03164-5h}
               »umhergelaufen«. Dass sie mir viel Tratsch über Sie, \textcolor{blue}{Beer-Hofmann}{}\ledrightnote{\textcolor{blue}{Richard Beer-Hofmann}} und mich mitgebracht, gehört {\pb}wol mit dazu. Von \textcolor{blue}{Kraus}{}\ledrightnote{\textcolor{blue}{Karl Kraus}} ist im \textcolor{green}{Familien-Journal}{}\ledrightnote{\textcolor{green}{Wiener Familien-Journal}} eine Geschichte erschienen, »\label{K_L03164-6v}\edtext{\textcolor{green}{Esplanade-Dichter}{}\ledrightnote{{$\rightarrow$}\textcolor{green}{Ischler Brief. (Wiener Dichter auf der Esplanade.)}}}{\lemma{\textnormal{\emph{Esplanade-Dichter}}}\Cendnote{\textnormal{\textcolor{blue}{Crêpedechine} [ = \textcolor{blue}{Karl Kraus}]: \emph{\textcolor{green}{Ischler
                        Brief. (Wiener Dichter auf der Esplanade.)}}. In: \emph{\textcolor{green}{Wiener Familien-Journal}}, Nr. 230, 23. 8. 1895, S. 914–915. Während die
                  satirischen Bemerkungen über \textcolor{blue}{Beer-Hofmann}
                     (»ein junger Dichter, der die beſten Erfolge auf dem Gebiete der Mode
                     aufzuweiſen hat«) und \textcolor{blue}{Hofmannsthal} (»{[}e{]}in \textcolor{pink}{Wien}er Dichter, der
                     den Schulſchluß abwarten muß, um nach \textcolor{pink}{Iſchl}
                     gehen zu können«) gut zuordenbar scheinen, lässt sich im \textcolor{green}{Text} keine unzweifelhafte
                  Spitze gegen \textcolor{blue}{Schnitzler} ausmachen.}}}\label{K_L03164-6h}«,
               das sind \textcolor{blue}{Beer Hofmann}{}\ledrightnote{\textcolor{blue}{Richard Beer-Hofmann}} und Sie, und sollen
               »Eure Affectationen und Posen« darin mit vielem Witz »gegeißelt« worden sein. Ich
               habs nicht gesehen.\pend
           
\pstart
           Bitte, sagen Sie an Hr. D\textsuperscript{r}{ }\textcolor{blue}{Goldmann}{}\ledrightnote{\textcolor{blue}{Paul Goldmann}}, er möge Ihnen die Adresse von
                  \label{K_L03164-7v}\edtext{\textcolor{blue}{Bing}{}\ledrightnote{\textcolor{blue}{Siegfried Bing}}}{\lemma{\textnormal{\emph{Bing}}}\Cendnote{\textnormal{Gemeint dürfte der in \textcolor{pink}{Paris} lebende Kunsthändler \textcolor{blue}{Siegfried Bing} sein, der sich auf \textcolor{pink}{japan}ische und \textcolor{pink}{asia}tische Kunst spezialisiert hatte. \textcolor{blue}{Vincent van Gogh} frequentierte seine Sammlung. \textcolor{blue}{Goldmann} hielt sich zu diesem Zeitpunkt mit \textcolor{blue}{Schnitzler} in \textcolor{pink}{München} auf.}}}\label{K_L03164-7h} oder \textcolor{blue}{Bingen}{}\ledrightnote{\textcolor{blue}{Siegfried Bing}}, das ist der \textcolor{pink}{Japaner}{}\ledrightnote{{$\rightarrow$}\textcolor{pink}{Japan}}{[},{]} mittheilen, und schreiben Sie mir nach \textcolor{pink}{Wien}{}\ledrightnote{\textcolor{pink}{Wien}}, wo\strikeout{hin} ich ohnedies bald
               einen Brief von Ihnen erwarte.\pend
           
\pstart
           Mit vielen Empfehlungen an D\textsuperscript{r}{ }\textcolor{blue}{G.}{}\ledrightnote{\textcolor{blue}{Paul Goldmann}} herzlichst\pend
           
\pstart
           Ihr {\\[\baselineskip]}\spacefill\mbox{Salten}\pend
           \leftskip=0em{}\endnumbering\briefempfaengerindex{Schnitzler, Arthur@\textsc{Schnitzler, Arthur}!zzzSalten, Felix@\emph{von Felix Salten}!1895-08-301@{30. 8. 1895}|)be}\mylabel{h}  \normalsize

\doendnotes{C}
\bigskip
\vfill

\clearpage

\footnotesize

\lohead{\textsc{register}}

% Definiere theindex-Environment komplett neu ohne reledmac
\makeatletter
\renewenvironment{theindex}{%
  \section*{\indexname}%
  \setlength{\parindent}{0pt}%
  \setlength{\parskip}{0pt plus 0.3pt}%
  \let\item\@idxitem
}{%
  \clearpage
}
\makeatother

\IfFileExists{\jobname-pw.ind}{\input{\jobname-pw.ind}}{}

\end{document}

      