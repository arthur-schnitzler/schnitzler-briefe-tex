%% latex-korrekturansicht-vorspann.tex
%% Vorspann für die Korrekturansicht.
%% Lädt die gemeinsame Datei latex-vorspann.tex mit gesetztem Schalter.

\newif\ifkorrekturansicht
\korrekturansichttrue

\input{../tex-inputs/latex-vorspann}


\renewcommand{\erwaehnteOrte}{Orte: Berlin, Frankfurt am Main, Prag, Wien}
\renewcommand{\erwaehnteWerke}{}
\section[ Paul Goldmann und Theodore Rottenberg an Arthur Schnitzler, {[}1{]}9. 3. {[}1902{]}]{Paul Goldmann und Theodore Rottenberg an Arthur Schnitzler, {[}1{]}9. 3. {[}1902{]}}
\nopagebreak\mylabel{v}
\rehead{ }\normalsize\beginnumbering\briefempfaengerindex{Schnitzler, Arthur@\textsc{Schnitzler, Arthur}!zzzRottenberg, Theodore@\emph{von Theodore Rottenberg}!1902-03-191@{{[}1{]}9. 3. {[}1902{]}}|(be}\briefempfaengerindex{Schnitzler, Arthur@\textsc{Schnitzler, Arthur}!zzzGoldmann, Paul@\emph{von Paul Goldmann}!1902-03-191@{{[}1{]}9. 3. {[}1902{]}}|(be}
\toendnotes[C]{\smallbreak\pagebreak[2]}\Standort{DLA, A:Schnitzler, HS.NZ85.1.3172.}
\physDesc{Brief, 1 Blatt, 1 Seite
\newline{}Handschrift Paul Goldmann: schwarze Tinte, deutsche Kurrent
\newline{}Handschrift Theodore Rottenberg: schwarze Tinte, deutsche Kurrent
\newline{}Schnitzler: mit Bleistift das Jahr »{[}1{]}90\textcolor{gray}{2}« vermerkt }\toendnotes[C]{\smallbreak}
\pstart
           \raggedleft{}{\pb}\textcolor{pink}{Berlin}{}\ledrightnote{\textcolor{pink}{Berlin}}, \label{K_L03199-55v}\edtext{\textcolor{gray}{1}9. März}{\lemma{\textnormal{\emph{19. März}}}\Cendnote{\textnormal{Die erste Ziffer der Datumsangabe ist
                     nicht mit Sicherheit zu lesen. Da \textcolor{blue}{Goldmann} am 20. 3. [1902] von 
                     der Anwesenheit \textcolor{blue}{Rottenbergs} in \textcolor{pink}{Berlin} schreibt und
                     sich am 29. 3. 1902 in \textcolor{pink}{Prag} aufhält, scheint
                  die Datierung mit »19.« aber verlässlich.}}}\label{K_L03199-55h}.\pend
           
\pstart
           Es grüßen Dich, mein lieber Freund, Zwei, die ſich lieb haben, nämlich\pend
           
\pstart
           1.) \textsc{Paul Goldmann}\pend
           
\pstart
           2.) {[}hs. Rottenberg:{]} Frau {\dotsfour} aus \textcolor{pink}{Frankfurt}{}\ledrightnote{\textcolor{pink}{Frankfurt am Main}}. (Sie wissen ſchon!)\pend
           \endnumbering\briefempfaengerindex{Schnitzler, Arthur@\textsc{Schnitzler, Arthur}!zzzRottenberg, Theodore@\emph{von Theodore Rottenberg}!1902-03-191@{{[}1{]}9. 3. {[}1902{]}}|)be}\briefempfaengerindex{Schnitzler, Arthur@\textsc{Schnitzler, Arthur}!zzzGoldmann, Paul@\emph{von Paul Goldmann}!1902-03-191@{{[}1{]}9. 3. {[}1902{]}}|)be}\mylabel{h}
\begin{anhang}
\end{anhang}\normalsize

\doendnotes{C}
\bigskip
\vfill

\clearpage

\footnotesize

\lohead{\textsc{register}}

% Definiere theindex-Environment komplett neu ohne reledmac
\makeatletter
\renewenvironment{theindex}{%
  \section*{\indexname}%
  \setlength{\parindent}{0pt}%
  \setlength{\parskip}{0pt plus 0.3pt}%
  \let\item\@idxitem
}{%
  \clearpage
}
\makeatother

\IfFileExists{\jobname-pw.ind}{\input{\jobname-pw.ind}}{}

\end{document}

      