%% latex-korrekturansicht-vorspann.tex
%% Vorspann für die Korrekturansicht.
%% Lädt die gemeinsame Datei latex-vorspann.tex mit gesetztem Schalter.

\newif\ifkorrekturansicht
\korrekturansichttrue

\input{../tex-inputs/latex-vorspann}


\renewcommand{\erwaehntePersonen}{Personen:  ?? [Liebhaberin von Felix Salten, Ende März 1892]}
\renewcommand{\erwaehnteOrte}{Orte: Café Kremser, Kettenbrücke, Wien}
\renewcommand{\erwaehnteWerke}{}
\section[Felix Salten an Arthur Schnitzler, {[}31. 3. 1892{]}]{Felix Salten an Arthur Schnitzler, {[}31. 3. 1892{]}}
\nopagebreak\mylabel{v}
\rehead{ }\normalsize\beginnumbering\briefempfaengerindex{Schnitzler, Arthur@\textsc{Schnitzler, Arthur}!zzzSalten, Felix@\emph{von Felix Salten}!1892-03-311@{{[}31. 3. 1892{]}}|(be}
\toendnotes[C]{\smallbreak\pagebreak[2]}\Standort{CUL, Schnitzler, B 89, A 1.}
\physDesc{Brief, 1 Blatt, 4 Seiten, 1311 Zeichen
\newline{}Handschrift: schwarze Tinte, lateinische Kurrent
\newline{}Schnitzler: mit Bleistift datiert: »31/3 92« 
\newline{}Ordnung: mit Bleistift von unbekannter Hand nummeriert: »9« }\toendnotes[C]{\smallbreak}
\pstart
           \noindent{}{\pb}Lieber Arthur! Soeben bin ich für immer von der
                  \label{K_L03108-1v}\edtext{»schönsten Pflicht des
                  Bürgers«}{\lemma{\textnormal{\emph{»schönsten … Bürgers«}}}\Cendnote{\textnormal{Wehrdienst}}}\label{K_L03108-1h}
               freigesprochen worden, und mir ist, als hätte ich eben mich selbst zum Geschenk
               erhalten. Ich bin in einer so guten, leichten Stimmung, dass ich meine, man hätte mir
               in der Welt kein schöneres Präsent machen können. Der Aufenthalt {\pb}im Aussenlokale mitten
               unter diesen Anderen ist etwas Entsetzliches. Man ist wie diese hier, und wird als
               dasselbe angesehen und behandelt wie der vertrottelte Schuster, besoffene
               Maurergeselle, arrogante Commis ec. ec. 1529, – der Schuster – 1530 – der
               Maurergehilfe, – 1531 – ich, 1532 – der Commis u. s. w. aber man kann niemandem einen
               Vorwurf daraus machen, der Staat richtet {\pb}sich hierin nach der
               Natur, die ja für uns nicht die Ehre hat, – Sie wissen schon, und die \uline{uns} weder ein längeres Leben noch andere Nerven
                  gibt\textcolor{gray}{.–} Der Maurergehilfe leb\substVorne{}\textsuperscript{\textcolor{gray}{st}}\substDazwischen{}t\substHinten{} sicher länger als ich, und der Commis wird mich vermutlich mit meiner
               Geliebten betrügen, weil er eine vielversprechendere Nase hat als ich.\pend
           
\pstart
           Auf der Herreise habe ich eine kleine Novelle erlebt, reizend sage {\pb}ich Ihnen. Ganz ohne Handlung,
               denn das Rendezvous auf der \textcolor{pink}{Kettenbrücke}{}\ledrightnote{\textcolor{pink}{Kettenbrücke}} werde
               ich heute N. M. kaum einhalten. Es ist nicht mehr nothwendig. Ich kenn’
                  \label{K_L03108-2v}\edtext{\textcolor{blue}{sie}{}\ledrightnote{{$\rightarrow$}\textcolor{blue}{?? [Liebhaberin von Felix Salten, Ende März 1892]}}}{\lemma{\textnormal{\emph{sie}}}\Cendnote{\textnormal{nicht ermittelt}}}\label{K_L03108-2h} schon, also –
               abtreten.\pend
           
\pstart
           Leben Sie wol. Vielleicht erst \label{K_L03108-3v}\edtext{Samstag{ }Abend{ }\uline{\textcolor{pink}{Café Kremser}{}\ledrightnote{\textcolor{pink}{Café Kremser}}}}{\lemma{\textnormal{\emph{Samstag … Kremser}}}\Cendnote{\textnormal{nicht nachweisbar}}}\label{K_L03108-3h}\pend
           
\pstart
           Herzlich Ihr {\\[\baselineskip]}\spacefill\mbox{Felix Salten}\pend
           \leftskip=0em{}\endnumbering\briefempfaengerindex{Schnitzler, Arthur@\textsc{Schnitzler, Arthur}!zzzSalten, Felix@\emph{von Felix Salten}!1892-03-311@{{[}31. 3. 1892{]}}|)be}\mylabel{h}  \normalsize

\doendnotes{C}
\bigskip
\vfill

\clearpage

\footnotesize

\lohead{\textsc{register}}

% Definiere theindex-Environment komplett neu ohne reledmac
\makeatletter
\renewenvironment{theindex}{%
  \section*{\indexname}%
  \setlength{\parindent}{0pt}%
  \setlength{\parskip}{0pt plus 0.3pt}%
  \let\item\@idxitem
}{%
  \clearpage
}
\makeatother

\IfFileExists{\jobname-pw.ind}{\input{\jobname-pw.ind}}{}

\end{document}

      