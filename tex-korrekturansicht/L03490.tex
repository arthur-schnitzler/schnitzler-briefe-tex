%% latex-korrekturansicht-vorspann.tex
%% Vorspann für die Korrekturansicht.
%% Lädt die gemeinsame Datei latex-vorspann.tex mit gesetztem Schalter.

\newif\ifkorrekturansicht
\korrekturansichttrue

\input{../tex-inputs/latex-vorspann}


\renewcommand{\erwaehntePersonen}{Personen: Felix Salten, Ottilie Salten, Olga Schnitzler}
\renewcommand{\erwaehnteInstitutionen}{Institutionen: Burgtheater, Die Zeit, Franz-Grillparzer-Preis}
\renewcommand{\erwaehnteOrte}{Orte: Heiligenstadt, Wien}
\renewcommand{\erwaehnteWerke}{Werke: Der Schleier der Beatrice. Schauspiel in fünf Akten, Verleihung des Grillparzer-Preises an Artur Schnitzler, Zwischenspiel. Komödie in drei Akten}
\section[ Felix Salten an Arthur Schnitzler, 15. 1. 1908]{Felix Salten an Arthur Schnitzler, 15. 1. 1908}
\nopagebreak\mylabel{v}
\rehead{ }\normalsize\beginnumbering\briefempfaengerindex{Schnitzler, Arthur@\textsc{Schnitzler, Arthur}!zzzSalten, Felix@\emph{von Felix Salten}!1908-01-151@{15. 1. 1908}|(be}
\toendnotes[C]{\smallbreak\pagebreak[2]}\Standort{CUL, Schnitzler, B 89, B 1.}
\physDesc{Brief, 1 Blatt, 1 Seite, 672 Zeichen
\newline{}Handschrift: schwarze Tinte, lateinische Kurrent
\newline{}Schnitzler: mit Bleistift Vermerk: »\textsc{Salten}« 
\newline{}Ordnung: mit Bleistift von unbekannter Hand nummeriert: »239« }\toendnotes[C]{\smallbreak}
\pstart
           \raggedleft{}{\pb}\textcolor{pink}{Heiligenstadt}{}\ledrightnote{\textcolor{pink}{Heiligenstadt}}, 15. I. 08\pend
           
\pstart{}Lieber,\pend
\pstart
           eben wird mir aus der \textcolor{brown}{Redaktion}{}\ledrightnote{{$\rightarrow$}\textcolor{brown}{Die Zeit}}
               telefonirt, dass Ihr »\textcolor{green}{Zwischenspiel}{}\ledrightnote{\textcolor{green}{Zwischenspiel. Komödie in drei Akten}}« den \label{K_L03490-1v}\edtext{\textcolor{brown}{Grillparzer-Preis}{}\ledrightnote{\textcolor{brown}{Franz-Grillparzer-Preis}}}{\lemma{\textnormal{\emph{Grillparzer-Preis}}}\Cendnote{\textnormal{Das Auswahlkomitee hatte am 15. 1. 1908
                  entschieden, dass \textcolor{blue}{Schnitzler} für seine
                  Komödie \emph{\textcolor{green}{Zwischenspiel}} der mit 5.000 Kronen
                  dotierte \emph{\textcolor{brown}{Grillparzer-Preis}} verliehen wird. In
                  den Jahren zuvor war er zwar immer wieder als Favorit gehandelt worden, doch
                  stellte das Zerwürfnis mit dem \emph{\textcolor{brown}{Burgtheater}} in
                  Folge der Rückgabe von \emph{\textcolor{green}{Der Schleier der
                     Beatrice}} (1901) ein Hindernis dar. Seit Sommer 1905 war der Konflikt behoben und \textcolor{blue}{Schnitzler} konnte wieder bei der \textcolor{brown}{Preisvergabe} berücksichtigt
                  werden.}}}\label{K_L03490-1h} bekam. Ich habe eine große Freude drüber, und sende Ihnen meinen
               herzlichen Glückwunsch. Es war das Beste, was die Herren tun konnten, – wenn es ihnen
               auch, wie’s scheint, \label{K_L03490-2v}\edtext{nicht so bald
               eingefallen ist}{\lemma{\textnormal{\emph{nicht … ist}}}\Cendnote{\textnormal{\textcolor{blue}{Salten} kannte also bereits das \textcolor{green}{Interview}, das am nächsten Tag in seiner Zeitung
                  erscheinen sollte: A. S.: \emph{»Das Zeitlose ist von kürzester Dauer«}, [Karl Werkmann]: Verleihung des Grillparzer-Preises an Artur Schnitzler, 16. 1. 1908.
               }}}\label{K_L03490-2h} – und hoffentlich kommt diese Freude auch in einem guten Moment, und es
                  \label{K_L03490-3v}\edtext{geht Ihrer \textcolor{blue}{Frau}{}\ledrightnote{{$\rightarrow$}\textcolor{blue}{Olga Schnitzler}} immer besser und besser}{\lemma{\textnormal{\emph{geht … besser}}}\Cendnote{\textnormal{vgl. Felix Salten an Arthur Schnitzler, [10. 12. 1907]}}}\label{K_L03490-3h}.\pend
           
\pstart
           Wir sind alle krank. Influenza. Und wir liegen auch alle seit Samstag im Bett. \textcolor{blue}{Otti}{}\ledrightnote{\textcolor{blue}{Ottilie Salten}} hat sogar
               eine Blinddarmreizung. Aber wir hoffen, dass nächste Woche alles wieder gut ist.\pend
           
\pstart
           Nochmals herzliche Glückwünsche, und viele Güße an Sie u. Frau \textcolor{blue}{Olga}{}\ledrightnote{\textcolor{blue}{Olga Schnitzler}}.\pend
           
\pstart
           Ihr {\\[\baselineskip]}\spacefill\mbox{Salten}\pend
           \leftskip=0em{}\endnumbering\briefempfaengerindex{Schnitzler, Arthur@\textsc{Schnitzler, Arthur}!zzzSalten, Felix@\emph{von Felix Salten}!1908-01-151@{15. 1. 1908}|)be}\mylabel{h}  \normalsize

\doendnotes{C}
\bigskip
\vfill

\clearpage

\footnotesize

\lohead{\textsc{register}}

% Definiere theindex-Environment komplett neu ohne reledmac
\makeatletter
\renewenvironment{theindex}{%
  \section*{\indexname}%
  \setlength{\parindent}{0pt}%
  \setlength{\parskip}{0pt plus 0.3pt}%
  \let\item\@idxitem
}{%
  \clearpage
}
\makeatother

\IfFileExists{\jobname-pw.ind}{\input{\jobname-pw.ind}}{}

\end{document}

      