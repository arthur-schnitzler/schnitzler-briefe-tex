%% latex-korrekturansicht-vorspann.tex
%% Vorspann für die Korrekturansicht.
%% Lädt die gemeinsame Datei latex-vorspann.tex mit gesetztem Schalter.

\newif\ifkorrekturansicht
\korrekturansichttrue

\input{../tex-inputs/latex-vorspann}


\renewcommand{\erwaehntePersonen}{Personen: Hugo von Hofmannsthal, Adele Sandrock, Richard Specht}
\renewcommand{\erwaehnteOrte}{Orte: Café Griensteidl, Wien}
\renewcommand{\erwaehnteWerke}{Werke: Tagebuch}
\section[ Felix Salten an Arthur Schnitzler, {[}10. 2. 1895{]}]{Felix Salten an Arthur Schnitzler, {[}10. 2. 1895{]}}
\nopagebreak\mylabel{v}
\rehead{ }\normalsize\beginnumbering\briefempfaengerindex{Schnitzler, Arthur@\textsc{Schnitzler, Arthur}!zzzSalten, Felix@\emph{von Felix Salten}!1895-02-101@{{[}10. 2. 1895{]}}|(be}
\toendnotes[C]{\smallbreak\pagebreak[2]}\Standort{CUL, Schnitzler, B 89, A 1.}
\physDesc{Brief, 1 Blatt, 1 Seite, 210 Zeichen
\newline{}Handschrift: Bleistift, lateinische Kurrent
\newline{}Schnitzler: mit Bleistift datiert: »10/2 95« 
\newline{}Ordnung: mit Bleistift von unbekannter Hand nummeriert: »52« }\toendnotes[C]{\smallbreak}
\pstart
           \noindent{}{\pb}Lieber Freund, ich bin zum \label{K_L03151-1v}\edtext{Souper bei \textcolor{blue}{Specht}{}\ledrightnote{\textcolor{blue}{Richard Specht}}}{\lemma{\textnormal{\emph{Souper bei Specht}}}\Cendnote{\textnormal{Die Datierung des Korrespondenzstücks
                  ist mit Vorbehalt zu betrachten, da \textcolor{blue}{Salten}
                  dem \emph{\textcolor{green}{Tagebuch}}{ }\textcolor{blue}{Schnitzler}s zufolge an diesem
                     Abend bei \textcolor{blue}{Adele Sandrock}
                  war.}}}\label{K_L03151-1h}, wo Sie mich, falls es nötig wäre, a\textcolor{gray}{n}rufen können
               (Telefon N\textsuperscript{o} 526 – (Genau\textcolor{gray}{!} nicht?) So
               gegen ¾ 11 komme ich ins \textcolor{pink}{Griensteidl}{}\ledrightnote{\textcolor{pink}{Café Griensteidl}}.
               Auch \textcolor{blue}{Hugo}{}\ledrightnote{\textcolor{blue}{Hugo von Hofmannsthal}} kommt eventuell her.\pend
           
\pstart
           Herzlichst {\\[\baselineskip]}Ihr {\\[\baselineskip]}\spacefill\mbox{Salten}\pend
           \leftskip=0em{}\endnumbering\briefempfaengerindex{Schnitzler, Arthur@\textsc{Schnitzler, Arthur}!zzzSalten, Felix@\emph{von Felix Salten}!1895-02-101@{{[}10. 2. 1895{]}}|)be}\mylabel{h}  \normalsize

\doendnotes{C}
\bigskip
\vfill

\clearpage

\footnotesize

\lohead{\textsc{register}}

% Definiere theindex-Environment komplett neu ohne reledmac
\makeatletter
\renewenvironment{theindex}{%
  \section*{\indexname}%
  \setlength{\parindent}{0pt}%
  \setlength{\parskip}{0pt plus 0.3pt}%
  \let\item\@idxitem
}{%
  \clearpage
}
\makeatother

\IfFileExists{\jobname-pw.ind}{\input{\jobname-pw.ind}}{}

\end{document}

      