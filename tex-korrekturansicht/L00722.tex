%% latex-korrekturansicht-vorspann.tex
%% Vorspann für die Korrekturansicht.
%% Lädt die gemeinsame Datei latex-vorspann.tex mit gesetztem Schalter.

\newif\ifkorrekturansicht
\korrekturansichttrue

\input{../tex-inputs/latex-vorspann}


               \section[Arthur Schnitzler an Richard Beer-Hofmann, {[}17. 9. 1897?{]}]{ Arthur Schnitzler an Richard Beer-Hofmann, {[}17. 9. 1897?{]}}\nopagebreak\mylabel{v}\rehead{ }\normalsize\beginnumbering\briefempfaengerindex{Beer-Hofmann, Richard@\textsc{Beer-Hofmann, Richard}!zzzSchnitzler, Arthur@\emph{von Arthur Schnitzler}!1897-09-171@{{[}zwischen 15. 11. 1893 und 1. 5. 1901{]}}|(be} \toendnotes[C]{\smallbreak\pagebreak[2]} \Standort{YCGL, MSS 31.}
\physDesc{Brief, 1 Blatt (Briefpapier mit Trauerrand), 2 Seiten, Umschlag
\newline{}Handschrift: Bleistift, deutsche Kurrent\newline{}Versand: ohne postalischen Übermittlungsvermerk }\toendnotes[C]{\smallbreak}\pstart{}{\pb}\textsc{Dr. Arth Schnitzler \textcolor{pink}{IX
                        Frankg 1}{}\ledrightnote{\textcolor{pink}{Frankgasse}}}.\pend{}{\bigskip}\pstart{}{\pb}\textsc{Herrn Dr. Rich. Beer-Hofmann}\pend{}\pstart{}\textcolor{pink}{Wien}{}\ledrightnote{\textcolor{pink}{Wien}}\pend{}\pstart{}\textsc{\textcolor{pink}{I. Wollzeile 15}{}\ledrightnote{\textcolor{pink}{Wollzeile}}}\pend{}{\bigskip}\pstart{}{\pb}Lieber Richard,\pend\pstart
           wir ſind nur 3 in der Loge u meine \label{K_L00722_1v}\edtext{\textcolor{blue}{Mama}{}\ledrightnote{→\textcolor{blue}{Louise Schnitzler}}}{\lemma{\textnormal{\emph{Mama}}}\Cendnote{\textnormal{Das Korrespondenzstück ist undatiert.
                  Zeitlich setzen die Adressen Grenzen: Am 15. 11. 1893 zog \textcolor{blue}{Schnitzler} in die \textcolor{pink}{Frankgasse}, ab
                     1. 5. 1901 wohnte \textcolor{blue}{Beer-Hofmann}
                  in der \textcolor{pink}{Willergasse}. Das \emph{\textcolor{green}{Tagebuch}} erwähnt nur einen Theaterbesuch mit \textcolor{blue}{Louise Schnitzler}, die Aufführung von \emph{\textcolor{green}{Die Meistersinger von Nürnberg}}, gemeinsam mit \textcolor{blue}{Rosa Freudenthal} am 17. 9. 1897.}}}\label{K_L00722_1h} lädt Sie »dringend« {\pb}zu uns ein, alſo bitte ko{\geminationm}en Sie!\pend
           \pstart
           Herzlichſt Ihr{\\[\baselineskip]}\spacefill\mbox{Arthur}\pend
           \leftskip=0em{}\pstart
           \noindent{}\introOben{}2. Stock.\introOben{}\pend
           \pstart
           Nr 2, links\pend
           \pstart
           \label{T_L00722_1v}\edtext{\uline{\label{K_L00722-1v}\edtext{Dſtm}{\lemma{\textnormal{\emph{Dſtm}}}\Cendnote{\textnormal{Dienstmann}}}\label{K_L00722-1h}. iſt
                     bezahlt.}}{\lemma{\textnormal{\emph{Dſtm. iſt
                     bezahlt.}}}\Cendnote{\textnormal{auf dem Umschlag neben der
                     Adresse}}}\label{T_L00722_1h}\pend
           \endnumbering\briefempfaengerindex{Beer-Hofmann, Richard@\textsc{Beer-Hofmann, Richard}!zzzSchnitzler, Arthur@\emph{von Arthur Schnitzler}!1897-09-171@{{[}zwischen 15. 11. 1893 und 1. 5. 1901{]}}|)be}\mylabel{h}  \normalsize

\doendnotes{C}
\bigskip
\vfill

\clearpage

\footnotesize

\lohead{\textsc{register}}

% Definiere theindex-Environment komplett neu ohne reledmac
\makeatletter
\renewenvironment{theindex}{%
  \section*{\indexname}%
  \setlength{\parindent}{0pt}%
  \setlength{\parskip}{0pt plus 0.3pt}%
  \let\item\@idxitem
}{%
  \clearpage
}
\makeatother

\IfFileExists{\jobname-pw.ind}{\input{\jobname-pw.ind}}{}

\end{document}

      