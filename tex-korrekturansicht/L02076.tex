%% latex-korrekturansicht-vorspann.tex
%% Vorspann für die Korrekturansicht.
%% Lädt die gemeinsame Datei latex-vorspann.tex mit gesetztem Schalter.

\newif\ifkorrekturansicht
\korrekturansichttrue

\input{../tex-inputs/latex-vorspann}


               \section[Hugo von Hofmannsthal an Olga Schnitzler, 5. 7. {[}1912{]}]{ Hugo von Hofmannsthal an Olga Schnitzler, 5. 7. {[}1912{]}}\nopagebreak\mylabel{v}\rehead{ }\normalsize\beginnumbering\briefempfaengerindex{Schnitzler, Olga@\textsc{Schnitzler, Olga}!zzzHofmannsthal, Hugo von@\emph{von Hugo von Hofmannsthal}!1912-07-051@{5. 7. {[}1912{]}}|(be} \toendnotes[C]{\smallbreak\pagebreak[2]} \Standort{CUL, Schnitzler, B 43.}
\physDesc{Briefkarte
\newline{}Handschrift: schwarze Tinte, deutsche Kurrent\newline{}Ordnung: 1) von Schnitzler mit Bleistift die
                           Jahreszahl ergänzt: »912« und beschriftet: »\textsc{Hugo}« 2) mit Bleistift von unbekannter Hand nummeriert: »\strikeout{328}«3) mit Bleistift von unbekannter Hand nummeriert: »338«}\buchAbdrucke{\weitereDrucke{Hugo von Hofmannsthal, Arthur Schnitzler: \emph{Briefwechsel}. Hg. Therese Nickl und Heinrich Schnitzler. Frankfurt am Main: \emph{S. Fischer} 1964, S. 385.} }\toendnotes[C]{\smallbreak}\pstart
           \raggedleft{}{\pb}\textcolor{pink}{Rodaun}{}\ledrightnote{\textcolor{pink}{Rodaun}}{ }\substVorne{}\textsuperscript{6}\substDazwischen{}5\substHinten{}. VII.\pend
           \pstart{}liebe Olga,\pend\pstart
           gerade geſtern Abend fand ich einen ſehr netten Brief von \textcolor{blue}{Steinrück}{}\ledrightnote{\textcolor{blue}{Albert Steinrück}} aus \textcolor{pink}{Tutzing}{}\ledrightnote{\textcolor{pink}{Tutzing}}, alſo
               liegt kein Grund vor, ihn zu erziehen.\hspace*{1.5em}Ich ſchicke
               Ihnen demnächst \textcolor{green}{Ariadne}{}\ledrightnote{\textcolor{green}{Ariadne auf Naxos. Oper in einem Aufzug}} und den \textcolor{green}{Sa{\geminationm}elband meiner jugendlichen
                  Arbeiten}{}\ledrightnote{\textcolor{green}{Die Gedichte und kleinen Dramen}} und würde mich ſehr freuen wenn Sie beides in den So{\geminationm}er mitnähmen.\pend
           \pstart
           {\pb}Man ſieht ſich gar ſo ſelten! Das
               Leben iſt ſo kurz, auf einmal wird man todt ſein und es dann ſehr bedauern. Ko{\geminationm}t Ihr beide oder ko{\geminationm}t \textcolor{blue}{Arthur}{}\ledrightnote{} doch noch nächſte Woche für 1–1½ Tage nach
                  \textcolor{pink}{Vöslau}{}\ledrightnote{\textcolor{pink}{Bad Vöslau}}{ }ſo würde ich ſehr gern von der \textcolor{pink}{Hinterbrühl}{}\ledrightnote{\textcolor{pink}{Hinterbrühl}} hinüberfahren für eine Stunde \label{K_L02076_1v}\edtext{Zuſa{\geminationm}enſein}{\lemma{\textnormal{\emph{Zuſaenſein}}}\Cendnote{\textnormal{siehe A. S.: \emph{Tagebuch}, 10. 7. 1912}}}\label{K_L02076_1h}.\pend
           \pstart
           Erbitte alſo eventuell Depeſche \textsc{\textcolor{pink}{Villa \textcolor{blue}{Louis
                        Friedmann}{}\ledrightnote{\textcolor{blue}{Louis Philipp Friedmann}}}{}\ledrightnote{\textcolor{pink}{Villa Friedmann}}}.\pend
           \pstart \label{T_L02076_1v}\edtext{Freundschaftlich
                  Ihr}{\lemma{\textnormal{\emph{Freundschaftlich
                  Ihr}}}\Cendnote{\textnormal{quer am
                  linken Rand}}}\label{T_L02076_1h}\spacefill\mbox{Hugo}\pend{}\endnumbering\briefempfaengerindex{Schnitzler, Olga@\textsc{Schnitzler, Olga}!zzzHofmannsthal, Hugo von@\emph{von Hugo von Hofmannsthal}!1912-07-051@{5. 7. {[}1912{]}}|)be}\mylabel{h}  \normalsize

\doendnotes{C}
\bigskip
\vfill

\clearpage

\footnotesize

\lohead{\textsc{register}}

% Definiere theindex-Environment komplett neu ohne reledmac
\makeatletter
\renewenvironment{theindex}{%
  \section*{\indexname}%
  \setlength{\parindent}{0pt}%
  \setlength{\parskip}{0pt plus 0.3pt}%
  \let\item\@idxitem
}{%
  \clearpage
}
\makeatother

\IfFileExists{\jobname-pw.ind}{\input{\jobname-pw.ind}}{}

\end{document}

      