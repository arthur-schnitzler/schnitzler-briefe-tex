%% latex-korrekturansicht-vorspann.tex
%% Vorspann für die Korrekturansicht.
%% Lädt die gemeinsame Datei latex-vorspann.tex mit gesetztem Schalter.

\newif\ifkorrekturansicht
\korrekturansichttrue

\input{../tex-inputs/latex-vorspann}


\renewcommand{\erwaehntePersonen}{Personen: Paul Goldmann}
\renewcommand{\erwaehnteOrte}{Orte: Paris, Riva del Garda, Wien, rue de Maubeuge, rue de la Bourse}
\renewcommand{\erwaehnteWerke}{}
\section[ Felix Salten an Arthur Schnitzler, 16. 5. 1897]{Felix Salten an Arthur Schnitzler, 16. 5. 1897}
\nopagebreak\mylabel{v}
\rehead{ }\normalsize\beginnumbering\briefempfaengerindex{Schnitzler, Arthur@\textsc{Schnitzler, Arthur}!zzzSalten, Felix@\emph{von Felix Salten}!1897-05-161@{16. 5. 1897}|(be}
\toendnotes[C]{\smallbreak\pagebreak[2]}\Standort{CUL, Schnitzler, B 89, A 2.}
\physDesc{Brief, 1 Blatt, 1 Seite, 739 Zeichen
\newline{}Handschrift: schwarze Tinte, lateinische Kurrent
\newline{}Ordnung: mit Bleistift von unbekannter Hand nummeriert: »88« }\toendnotes[C]{\smallbreak}
\pstart
           \raggedleft{}{\pb}\textcolor{pink}{Wien}{}\ledrightnote{\textcolor{pink}{Wien}}, 16. Mai 97\pend
           
\pstart
           Lieber Arthur,{ }gestern{ }Abend erfuhr ich durch Zufall Ihre jetzige \label{K_L03265-1v}\edtext{Adresse}{\lemma{\textnormal{\emph{Adresse}}}\Cendnote{\textnormal{\textcolor{blue}{Salten} bringt hier mehrere Dinge
                  durcheinander. \textcolor{blue}{Schnitzler}s Adresse in \textcolor{pink}{Paris} lag in der \textcolor{pink}{rue de Maubeuge}, wohin \textcolor{blue}{Salten} am
                     5. 5. 1897 geschrieben
                  haben dürfte. \textcolor{blue}{Schnitzler} dürfte das
                  Schreiben auch erhalten haben, jedenfalls ist es im Nachlass \textcolor{blue}{Schnitzler}s überliefert. In der \textcolor{pink}{rue de la Bourse}, was \textcolor{blue}{Salten} im vorliegenden Brief als Adresse nennt, wohnte \textcolor{blue}{Paul Goldmann}. Das war die Adresse, die \textcolor{blue}{Schnitzler} zu verwenden bat, vgl. Arthur Schnitzler an Richard Beer-Hofmann, 19. 4. 1897.}}}\label{K_L03265-1h}, und erklärte mir daraus, weshalb Sie
               mir wol bis heute nicht geantwortet haben. Offenbar
               haben Sie meinen \label{K_L03265-2v}\edtext{Brief nicht erhalten,
               den ich Ihnen vor mehr als vierzehn Tagen}{\lemma{\textnormal{\emph{Brief … Tagen}}}\Cendnote{\textnormal{Nachdem der Brief vom 5. 5. 1897 erhalten ist, dürfte \textcolor{blue}{Schnitzler}
                  ihn regulär erhalten haben, aber auf eine unmittelbare Antwort verzichtet haben –
                  oder diese ging verlustig. Jedenfalls irrt \textcolor{blue}{Salten}, sein Schreiben lag noch keine zwei Wochen zurück.}}}\label{K_L03265-2h} schrieb.
               Ich kam Ende April aus \textcolor{pink}{Riva}{}\ledrightnote{\textcolor{pink}{Riva del Garda}} zurück und fand Ihre Karte und Ihren Brief. Darauf habe ich ziemlich
               ausführlich erwiedert und, da Sie es zu wünschen schienen, über mein Leben und meine
               Arbeiten ec. berichtet. Auf die Adresse schrieb ich nach Ihrer Angabe \textcolor{pink}{rue de la Bourse}{}\ledrightnote{\textcolor{pink}{rue de la Bourse}}. Offenbar haben Sie dieses
               Schreiben nicht erhalten, und da ich hier mit Niemandem verkehre, habe ich erst gestern{ }Abend Ihre neue Wohnung erfahren und glaube, Ihnen das zur Aufklärung
               sagen zu müßen.\pend
           
\pstart
           Herzlich {\\[\baselineskip]}\spacefill\mbox{Salten}\pend
           \leftskip=0em{}\endnumbering\briefempfaengerindex{Schnitzler, Arthur@\textsc{Schnitzler, Arthur}!zzzSalten, Felix@\emph{von Felix Salten}!1897-05-161@{16. 5. 1897}|)be}\mylabel{h}  \normalsize

\doendnotes{C}
\bigskip
\vfill

\clearpage

\footnotesize

\lohead{\textsc{register}}

% Definiere theindex-Environment komplett neu ohne reledmac
\makeatletter
\renewenvironment{theindex}{%
  \section*{\indexname}%
  \setlength{\parindent}{0pt}%
  \setlength{\parskip}{0pt plus 0.3pt}%
  \let\item\@idxitem
}{%
  \clearpage
}
\makeatother

\IfFileExists{\jobname-pw.ind}{\input{\jobname-pw.ind}}{}

\end{document}

      