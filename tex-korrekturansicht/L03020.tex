%% latex-korrekturansicht-vorspann.tex
%% Vorspann für die Korrekturansicht.
%% Lädt die gemeinsame Datei latex-vorspann.tex mit gesetztem Schalter.

\newif\ifkorrekturansicht
\korrekturansichttrue

\input{../tex-inputs/latex-vorspann}


\renewcommand{\erwaehntePersonen}{Personen: Lili Cappellini, Felix Salten, Heinrich Schnitzler, Olga Schnitzler,  Voltaire}
\renewcommand{\erwaehnteOrte}{Orte: Attersee, Baden-Baden, Berghof, Oberösterreich, Salzburg, Schweiz, Sternwartestraße 71, Unterach am Attersee, Wien, XVIII., Währing}
\renewcommand{\erwaehnteWerke}{Werke: Neue Freie Presse, Voltaire}
\section[ Arthur Schnitzler an Felix Salten, 22. 7. 1923]{Arthur Schnitzler an Felix Salten, 22. 7. 1923}
\nopagebreak\mylabel{v}
\rehead{ }\normalsize\beginnumbering\briefempfaengerindex{Salten, Felix@\textsc{Salten, Felix}!zzzSchnitzler, Arthur@\emph{von Arthur Schnitzler}!1923-07-221@{22. 7. 1923}|(be}
\toendnotes[C]{\smallbreak\pagebreak[2]}\Standort{Wienbibliothek im Rathaus, ZPH 1681, 2.1.516.}
\physDesc{Postkarte, 472 Zeichen
\newline{}Handschrift: Bleistift, lateinische Kurrent
\newline{}Versand: Stempel: »\nobreak{}18/\textsubscript{1} Wien 11\textcolor{gray}{0}, 24. VII. 23, 9\nobreak{}«.  
\newline{}Ordnung: mit Bleistift von unbekannter Hand nummeriert: »5« }
\buchAbdrucke{\weitereDrucke{Arthur Schnitzler: \emph{Briefe 1913–1931}. Hg. Peter Michael Braunwarth, Richard Miklin, Susanne Pertlik und Heinrich Schnitzler. Frankfurt am Main: \emph{S. Fischer} 1984, S. 322–323.} }\toendnotes[C]{\smallbreak}\pstart{}{\pb}\label{T_L03020-1v}\edtext{\textcolor{gray}{\textbf{A. S.}}}{\lemma{\textnormal{\emph{A. S.}}}\Cendnote{\textnormal{ovaler Absenderkleber}}}\label{T_L03020-1h}\pend{}\pstart{}\textcolor{pink}{\textcolor{gray}{\textbf{WIEN, XVIII.}}}{}\ledrightnote{\textcolor{pink}{XVIII., Währing}}\pend{}\pstart{}\textcolor{pink}{\textcolor{gray}{\textbf{STERNWARTESTR. 71}}}{}\ledrightnote{\textcolor{pink}{Sternwartestraße 71}}\pend{}
{\bigskip}\pstart{}\textcolor{pink}{Ob. Oe.}{}\ledrightnote{\textcolor{pink}{Oberösterreich}}\pend{}\pstart{}Herrn\pend{}\pstart{}Felix Salten\pend{}\pstart{}\textcolor{pink}{Unterach}{}\ledrightnote{\textcolor{pink}{Unterach am Attersee}} am \textcolor{pink}{Attersee}{}\ledrightnote{\textcolor{pink}{Attersee}}\pend{}\pstart{}\textcolor{pink}{Berghof}{}\ledrightnote{\textcolor{pink}{Berghof}}\pend{}
{\bigskip}
\pstart
           \raggedleft{}{\pb}\textcolor{pink}{Wien}{}\ledrightnote{\textcolor{pink}{Wien}}, 22. 7. 23\pend
           
\pstart
           lieber, lassen Sie sich die Hand drücken für Ihr \textcolor{gray}{schönes}{ }\label{K_L03020-1v}\edtext{\textcolor{green}{\textcolor{blue}{Voltaire}{}\ledrightnote{\textcolor{blue}{Voltaire}}
                  Feu{[}i{]}lleton}{}\ledrightnote{{$\rightarrow$}\textcolor{green}{Voltaire}}}{\lemma{\textnormal{\emph{Voltaire
                  Feuilleton}}}\Cendnote{\textnormal{\textcolor{blue}{Felix Salten}: \emph{\textcolor{green}{Voltaire}}. In: \emph{\textcolor{green}{Neue
                        Freie Presse}}, Nr. 21.144, 22. 7. 1923,
                     Morgenblatt, S. 1–3.}}}\label{K_L03020-1h} – u rechnen Sie nicht nach, wie viele
               ähnliche Händedrucke ich Ihnen schuldig bin!\pend
           
\pstart
           Ich lebe ziemlich stille Tage in \textcolor{pink}{Wien}{}\ledrightnote{\textcolor{pink}{Wien}}, und werde
                  Anfang August, vermutlich \label{K_L03020-2v}\edtext{über \textcolor{pink}{Baden Baden}{}\ledrightnote{\textcolor{pink}{Baden-Baden}}, wo die
                  \textcolor{blue}{Kinder}{}\ledrightnote{{$\rightarrow$}\textcolor{blue}{Lili Cappellini}{\newline}{$\rightarrow$}\textcolor{blue}{Heinrich Schnitzler}} bei \textcolor{blue}{Olga}{}\ledrightnote{\textcolor{blue}{Olga Schnitzler}}{ }{\pb}sommerweilen, in die \textcolor{pink}{Schweiz}{}\ledrightnote{\textcolor{pink}{Schweiz}} – oder sonstwohin fahren}{\lemma{\textnormal{\emph{über … fahren}}}\Cendnote{\textnormal{\textcolor{blue}{Schnitzler} reiste am 3. 8. 1923 nach \textcolor{pink}{Salzburg} ab und kam am 6. 8. 1923 in \textcolor{pink}{Baden-Baden} an. Am 15. 8. 1923 reiste er
                  weiter in die \textcolor{pink}{Schweiz}.}}}\label{K_L03020-2h}.\pend
           
\pstart
           Lassen Sie mich wissen, wies Ihnen und den Ihren geht u ob Sie arbeiten.\pend
           \pstart Herzlichst Ihr \spacefill\mbox{Arthur}\pend{}\endnumbering\briefempfaengerindex{Salten, Felix@\textsc{Salten, Felix}!zzzSchnitzler, Arthur@\emph{von Arthur Schnitzler}!1923-07-221@{22. 7. 1923}|)be}\mylabel{h}  \normalsize

\doendnotes{C}
\bigskip
\vfill

\clearpage

\footnotesize

\lohead{\textsc{register}}

% Definiere theindex-Environment komplett neu ohne reledmac
\makeatletter
\renewenvironment{theindex}{%
  \section*{\indexname}%
  \setlength{\parindent}{0pt}%
  \setlength{\parskip}{0pt plus 0.3pt}%
  \let\item\@idxitem
}{%
  \clearpage
}
\makeatother

\IfFileExists{\jobname-pw.ind}{\input{\jobname-pw.ind}}{}

\end{document}

      