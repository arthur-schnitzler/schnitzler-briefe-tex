%% latex-korrekturansicht-vorspann.tex
%% Vorspann für die Korrekturansicht.
%% Lädt die gemeinsame Datei latex-vorspann.tex mit gesetztem Schalter.

\newif\ifkorrekturansicht
\korrekturansichttrue

\input{../tex-inputs/latex-vorspann}


\renewcommand{\erwaehntePersonen}{Personen: Marceline Desbordes, Anatolij W. Lunačarski, Kote Marjanishvili, Hans Steinhoff, Leo N. von Tolstoi, Stefan Zweig}
\renewcommand{\erwaehnteInstitutionen}{Institutionen: Wremja}
\renewcommand{\erwaehnteOrte}{Orte: Paschinger Schlössl, Russland, Salzburg, Wien}
\renewcommand{\erwaehnteWerke}{Werke: Amok. Novelly, Angst, Angst, Brennendes Geheimnis, Das Haus am Meer. Ein Schauspiel in zwei Teilen (drei Aufzügen), Der Amokläufer, Marceline Desbordes-Valmore. Das Lebensbild einer Dichterin, Mutter, Dein Kind ruft{\rufezeichen}, Smjatenie Chusto, Sternstunden der Menschheit, Zach. Novella, Žgučaja tajna. Pervye pereživanija}
\section[Stefan Zweig an Arthur Schnitzler, 14. 10. 1927]{Stefan Zweig an Arthur Schnitzler, 14. 10. 1927}
\nopagebreak\mylabel{v}
\rehead{ }\normalsize\beginnumbering\briefempfaengerindex{Schnitzler, Arthur@\textsc{Schnitzler, Arthur}!zzzZweig, Stefan@\emph{von Stefan Zweig}!1927-10-141@{14. 10. 1927}|(be}
\toendnotes[C]{\smallbreak\pagebreak[2]}\Standort{CUL, Schnitzler, B 118.}
\physDesc{Brief, 1 Blatt, 2 Seiten, 1758 Zeichen
\newline{}Schreibmaschine
\newline{}Handschrift: blaue Tinte, lateinische Kurrent (\noindent{}Unterschrift)
\newline{}Schnitzler: 1) mit Bleistift beschriftet: »\textsc{Zweig}«  2) mit rotem Buntstift sechs Unterstreichungen}
\buchAbdrucke{\weitereDrucke{Stefan Zweig: \emph{Briefwechsel mit Hermann Bahr, Sigmund Freud, Rainer Maria
                        Rilke und Arthur Schnitzler}. Hg. Jeffrey B. Berlin, Hans-Ulrich Lindken und Donald A. Prater. Frankfurt am Main: \emph{S. Fischer} 1987, S. 431–432.} }\toendnotes[C]{\smallbreak}
\pstart
           {\pb}\textcolor{gray}{\textbf{SZ}}\hfill \textcolor{gray}{\textbf{\textcolor{pink}{SALZBURG}{}\ledrightnote{\textcolor{pink}{Salzburg}},}}\pend
           
\pstart
           \raggedleft{}\textcolor{gray}{\textbf{\textcolor{pink}{KAPUZINERBERG 5}{}\ledrightnote{\textcolor{pink}{Paschinger Schlössl}}}}\pend
           
\pstart
           \raggedleft{}14. Oktober 1927.\pend
           
\pstart{}Lieber, verehrter Herr Doktor!\pend\vspace{0.5em}
\pstart
           Ihre Handschrift erweckt immer freudiges Gefühl in mir und ich eile mich, Ihnen zu
               antworten, freilich nicht unbeschämt, denn meine Auskunft ist unverantwortlich
               ungenau. Ich bin in allen Honorardingen geradezu tölpisch leichtsinnig, kümmere mich
               um gar nichts und die Honorare, die ich bislang für \label{K_L03687-1v}\edtext{Verfilmungen meiner \textcolor{green}{Novellen}{}\ledrightnote{{$\rightarrow$}\textcolor{green}{Brennendes Geheimnis}{\newline}{$\rightarrow$}\textcolor{green}{Angst}{\newline}{$\rightarrow$}\textcolor{green}{Der Amokläufer}}}{\lemma{\textnormal{\emph{Verfilmungen … Novellen}}}\Cendnote{\textnormal{1923 wurde \textcolor{blue}{Zweigs} Novelle \emph{\textcolor{green}{Brennendes Geheimnis}} unter dem Titel \emph{\textcolor{green}{Mutter, Dein Kind ruft!}} verfilmt,
                     1924 sein Schauspiel \emph{\textcolor{green}{Das Haus am
                     Meer}}, 1927 die Novelle \emph{\textcolor{green}{Der
                     Amokläufer}} unter der Regie von \textcolor{blue}{Kote
                     Marjanishvili} in \textcolor{pink}{Russland} und im
                  Folgejahr die Novelle \emph{\textcolor{green}{Angst}} als \textcolor{green}{Stummfilm} unter der Regie
                  von \textcolor{blue}{Hans Steinhoff}.}}}\label{} erhielt, haben
               die Heiterkeit der Fachleute herausgefordert. So habe ich auch in \textcolor{pink}{Russland}{}\ledrightnote{\textcolor{pink}{Russland}} glattweg die Vorschläge angenommen, die mir die »\textcolor{brown}{Wremja}{}\ledrightnote{\textcolor{brown}{Wremja}}« stellte und die ich gar nicht mehr
               auswendig weiss. Ich kann nur feststellen, dass der Ertrag sich \label{K_L03687-2v}\edtext{bei dem letzten \textcolor{green}{Buche}{}\ledrightnote{{$\rightarrow$}\textcolor{green}{Smjatenie Chusto}{\newline}{$\rightarrow$}\textcolor{green}{Zach. Novella}}}{\lemma{\textnormal{\emph{bei dem letzten Buche}}}\Cendnote{\textnormal{Der Verlag \emph{\textcolor{brown}{Wremla}} hatte ohne \textcolor{blue}{Zweigs} Zustimmung 1925{ }\emph{\textcolor{green}{Erstes Erlebnis}} und 1926{ }\emph{\textcolor{green}{Amok}} auf \textcolor{pink}{russisch} publiziert. Nach der Kontaktaufnahme erschienen mit \textcolor{blue}{Zweigs} Zustimmung zwei Novellen unter dem
                  Titel \emph{\textcolor{green}{Smjatenie Chusto}} und die Novelle \emph{\textcolor{green}{Angst}}. Auf einen der letzten beiden Titel
                  bezieht sich Zweig hier.}}}\label{} etwa auf 150 Dollar belief, bin aber gewiss, dass
               Sie das Vierfache erzielen können. Die Buchpreise sind ja drüben nicht sehr
               wesentlich, aber nach den neuen Vereinbarungen, deren Text ich noch nicht kenne, hat
                  \textcolor{blue}{Lunatscharski}{}\ledrightnote{\textcolor{blue}{Anatolij W. Lunačarski}} auch von den unerlaubten
               Nachdrucken jetzt eine gewisse Quote für den ausländischen Autor festgesetzt. Ob sie
               gezahlt wird, ist eine andere Sache. Ich persönlich würde Ihnen raten, sich \textcolor{pink}{Russland}{}\ledrightnote{\textcolor{pink}{Russland}} gegenüber nicht auf Perzente
               einzulassen, weil man ja jeder Kontrollmöglichkeit entzogen ist, und eine einmalige
               Dollarsumme zu fordern: es ist ja ohnehin ein Wunder, wenn man etwas aus \textcolor{pink}{Russland}{}\ledrightnote{\textcolor{pink}{Russland}} herausbekommt. Ich hoffe, Sie
               allerdings in sechs Monaten viel besser informieren zu können, denn ich möchte sehr
               gerne im März mir für vier Wochen die Sache \label{K_L03687-3v}\edtext{persönlich anschauen}{\lemma{\textnormal{\emph{persönlich anschauen}}}\Cendnote{\textnormal{\textcolor{blue}{Stefan Zweig} reiste erst vom
                     7. bis zum 20. 9. 1928 nach \textcolor{pink}{Russland}, um an der Gedenkfeier zum 100. Geburtsag \textcolor{blue}{Tolstois} teilzunehmen.}}}\label{}.\pend
           
\pstart
           {\pb}Ich beglückwünsche Sie sehr dazu, so
               rasch und fleissig ein schöpferisches Buch dem anderen nachzusenden, was mir leider
               nicht gelingen will. Ich habe nur Kleineres zu bieten und dies mögen Sie heute mit
               der \textcolor{green}{Biographie der \textcolor{blue}{Desbordes-Valmore}{}\ledrightnote{\textcolor{blue}{Marceline Desbordes}}}{}\ledrightnote{\textcolor{green}{Marceline Desbordes-Valmore. Das Lebensbild einer Dichterin}} und den essayistischen \textcolor{green}{Miniaturen}{}\ledrightnote{{$\rightarrow$}\textcolor{green}{Sternstunden der Menschheit}} freundlich empfangen.\pend
           
\pstart
           In getreuer Liebe und Verehrung Ihr{\\[\baselineskip]}\spacefill\mbox{{[}hs.:{]} Stefan Zweig}\pend
           \leftskip=0em{}\endnumbering\briefempfaengerindex{Schnitzler, Arthur@\textsc{Schnitzler, Arthur}!zzzZweig, Stefan@\emph{von Stefan Zweig}!1927-10-141@{14. 10. 1927}|)be}\mylabel{h}
\begin{anhang}
\end{anhang}\normalsize

\doendnotes{C}
\bigskip
\vfill

\clearpage

\footnotesize

\lohead{\textsc{register}}

% Definiere theindex-Environment komplett neu ohne reledmac
\makeatletter
\renewenvironment{theindex}{%
  \section*{\indexname}%
  \setlength{\parindent}{0pt}%
  \setlength{\parskip}{0pt plus 0.3pt}%
  \let\item\@idxitem
}{%
  \clearpage
}
\makeatother

\IfFileExists{\jobname-pw.ind}{\input{\jobname-pw.ind}}{}

\end{document}

      