%% latex-korrekturansicht-vorspann.tex
%% Vorspann für die Korrekturansicht.
%% Lädt die gemeinsame Datei latex-vorspann.tex mit gesetztem Schalter.

\newif\ifkorrekturansicht
\korrekturansichttrue

\input{../tex-inputs/latex-vorspann}


               \section[ Paul Goldmann an Arthur Schnitzler, 21. 10. 1898]{Paul Goldmann an Arthur Schnitzler, 21. 10. 1898}\nopagebreak\mylabel{v}\rehead{ }\normalsize\beginnumbering\briefempfaengerindex{, @\textsc{, }!zzzGoldmann, Paul@\emph{von Paul Goldmann}!1898-10-211@{21. 10. 1898}|(be} \toendnotes[C]{\smallbreak\pagebreak[2]} \Standort{DLA, A:Schnitzler, HS.NZ85.1.3168.}
\physDesc{Bildpostkarte
\newline{}Handschrift: 1) schwarze Tinte, deutsche Kurrent\hspace{1em}2) schwarze Tinte, lateinische Kurrent (\noindent{}Adresse)\hspace{1em}\newline{}Versand: 1) Stempel: »\nobreak{}\oindex{Yantai@\textbf{Yantai}, \emph{Besiedelter Ort (A.BSO)}|pwk}{[}Ch{]}efoo, 21 Oct 98\nobreak{}«.  2) Stempel: »\nobreak{}\oindex{Shanghai@\textbf{Shanghai}, \emph{http://www.geonames.org/ontologyP.PPLA}|pwk}{[}Shanghai{]}, 23 Oct 98\nobreak{}«. 3) Stempel: »\nobreak{}Oc 24 98\nobreak{}«. 4) Stempel: »\nobreak{}Oc 27 98\nobreak{}«. 5) Stempel: »\nobreak{}{[}Wien 9/3 72{]}, 26. 11. 98, 11.V, Best{[}ellt{]}\nobreak{}«. }\pstart{}{\pb}\begin{otherlanguage}{english}\textcolor{pink}{Austria}{}\ledrightnote{\textcolor{pink}{Österreich}}\end{otherlanguage}\pend{}\pstart{}Herrn\pend{}\pstart{}Dr. Arthur Schnitzler\pend{}\pstart{}\textcolor{pink}{Wien}{}\ledrightnote{\textcolor{pink}{Wien}}\pend{}\pstart{}\textcolor{pink}{IX. Frankgaße 1}{}\ledrightnote{\textcolor{pink}{Frankgasse}}.\pend{}{\bigskip}\pstart
           \noindent{}\centering{}{\pb}\textcolor{gray}{\textbf{GRUSS AUS \textcolor{pink}{KIAUTSCHOU}{}\ledrightnote{\textcolor{pink}{Kiautschou}}}}\pend
           \pstart
           \centering{}21. Oktober.\pend
           \pstart
           Herzlichen Gruß!\pend
           \pstart
           Dein treuer {\\[\baselineskip]}\spacefill\mbox{P. Goldmann}\pend
           \leftskip=0em{}\endnumbering\briefempfaengerindex{, @\textsc{, }!zzzGoldmann, Paul@\emph{von Paul Goldmann}!1898-10-211@{21. 10. 1898}|)be}\mylabel{h}  \normalsize

\doendnotes{C}
\bigskip
\vfill

\clearpage

\footnotesize

\lohead{\textsc{register}}

% Definiere theindex-Environment komplett neu ohne reledmac
\makeatletter
\renewenvironment{theindex}{%
  \section*{\indexname}%
  \setlength{\parindent}{0pt}%
  \setlength{\parskip}{0pt plus 0.3pt}%
  \let\item\@idxitem
}{%
  \clearpage
}
\makeatother

\IfFileExists{\jobname-pw.ind}{\input{\jobname-pw.ind}}{}

\end{document}

      