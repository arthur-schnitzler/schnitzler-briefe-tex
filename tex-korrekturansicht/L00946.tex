%% latex-korrekturansicht-vorspann.tex
%% Vorspann für die Korrekturansicht.
%% Lädt die gemeinsame Datei latex-vorspann.tex mit gesetztem Schalter.

\newif\ifkorrekturansicht
\korrekturansichttrue

\input{../tex-inputs/latex-vorspann}


               \section[Richard Beer-Hofmann an Arthur Schnitzler, 18. 7. 1899]{ Richard Beer-Hofmann an Arthur Schnitzler, 18. 7. 1899}\nopagebreak\mylabel{v}\rehead{ }\normalsize\beginnumbering\briefempfaengerindex{Schnitzler, Arthur@\textsc{Schnitzler, Arthur}!zzzBeer-Hofmann, Richard@\emph{von Richard Beer-Hofmann}!1899-07-181@{18. 7. 1899}|(be} \toendnotes[C]{\smallbreak\pagebreak[2]} \Standort{CUL, Schnitzler, B 8.}
\physDesc{Kartenbrief
\newline{}Handschrift: Bleistift, lateinische Kurrent\newline{}Versand: 1) Stempel: »\nobreak{}\oindex{Seeboden@\textbf{Seeboden}, \emph{http://www.geonames.org/ontologyA.ADM3}|pwk}Seeboden, 18{[}. 7. 1899{]}\nobreak{}«.  2) Stempel: »\nobreak{}\oindex{Velden@\textbf{Velden}, \emph{Besiedelter Ort (A.BSO)}|pwk}Velden am Wörthersee, 19 7 99, 18.F\nobreak{}«. \newline{}Ordnung: mit Bleistift von unbekannter Hand nummeriert: »134« }\toendnotes[C]{\smallbreak}\pstart{}{\pb}D\textsuperscript{r}
                  Arthur Schnitzler\pend{}\pstart{}\textcolor{pink}{Velden a Wörthersee}{}\ledrightnote{\textcolor{pink}{Velden}}\pend{}\pstart{}\textcolor{pink}{Pension Pundschu}{}\ledrightnote{\textcolor{pink}{Pension Pundschu}}\pend{}{\bigskip}\pstart
           \raggedleft{}{\pb}\textcolor{pink}{Seeboden}{}\ledrightnote{\textcolor{pink}{Seeboden}}{ }18/VII früh\pend
           \pstart
           Lieber Arthur! Ich hoffe Ende dieser Woche fertig zu werden. Auch
               wenn ich aber nicht \textcolor{green}{fertig}{}\ledrightnote{→\textcolor{green}{Der Tod Georgs}} bin
                  ko{\geminationm} ich Sonntag oder Montag
               zu Ihnen. Jedenfalls telegrafire ich früher.\pend
           \pstart
           Was unsere Tour anlangt, habe ich außer irgendeinem \textcolor{pink}{Tauern}{}\ledrightnote{\textcolor{pink}{Hohe Tauern}}übergang keine besondere Wünsche.\pend
           \pstart
           Grüßen Sie \textcolor{blue}{Wassermann}{}\ledrightnote{\textcolor{blue}{Jakob Wassermann}}.\pend
           \pstart
           Herzlichst{\\[\baselineskip]}\spacefill\mbox{Richard}\pend
           \leftskip=0em{}\endnumbering\briefempfaengerindex{Schnitzler, Arthur@\textsc{Schnitzler, Arthur}!zzzBeer-Hofmann, Richard@\emph{von Richard Beer-Hofmann}!1899-07-181@{18. 7. 1899}|)be}\mylabel{h}  \normalsize

\doendnotes{C}
\bigskip
\vfill

\clearpage

\footnotesize

\lohead{\textsc{register}}

% Definiere theindex-Environment komplett neu ohne reledmac
\makeatletter
\renewenvironment{theindex}{%
  \section*{\indexname}%
  \setlength{\parindent}{0pt}%
  \setlength{\parskip}{0pt plus 0.3pt}%
  \let\item\@idxitem
}{%
  \clearpage
}
\makeatother

\IfFileExists{\jobname-pw.ind}{\input{\jobname-pw.ind}}{}

\end{document}

      