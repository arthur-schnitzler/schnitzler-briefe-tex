%% latex-korrekturansicht-vorspann.tex
%% Vorspann für die Korrekturansicht.
%% Lädt die gemeinsame Datei latex-vorspann.tex mit gesetztem Schalter.

\newif\ifkorrekturansicht
\korrekturansichttrue

\input{../tex-inputs/latex-vorspann}


         
         \renewcommand{\erwaehntePersonen}{Personen: Theodor Herzl, Theodor Loewe}
         \renewcommand{\erwaehnteInstitutionen}{Institutionen: Neue Freie Presse}
         \renewcommand{\erwaehnteOrte}{Orte: Berlin, Dessauer Straße, Wien}
         \renewcommand{\erwaehnteWerke}{Werke: Der Schleier der Beatrice. Schauspiel in fünf Akten, Lieutenant Gustl. Novelle, Neue Freie Presse, [Man telegraphirt uns aus Breslau…]}
               \section[ Paul Goldmann an Arthur Schnitzler, 11. 12. {[}1900{]}]{Paul Goldmann an Arthur Schnitzler, 11. 12. {[}1900{]}}\nopagebreak\mylabel{v}\rehead{ }\normalsize\beginnumbering\briefempfaengerindex{Schnitzler, Arthur@\textsc{Schnitzler, Arthur}!zzzGoldmann, Paul@\emph{von Paul Goldmann}!1900-12-111@{11. 12. {[}1900{]}}|(be} \toendnotes[C]{\smallbreak\pagebreak[2]} \Standort{DLA, A:Schnitzler, HS.NZ85.1.3170.}
\physDesc{Brief, 1 Blatt, 3 Seiten
\newline{}Handschrift: blaue Tinte, deutsche Kurrent
\newline{}Schnitzler: 1) mit Bleistift das Jahr »{[}1{]}900« vermerkt  2) mit rotem Buntstift eine Unterstreichung}\toendnotes[C]{\smallbreak}\pstart
           \noindent{}\raggedleft{}{\pb}\textcolor{pink}{\textcolor{gray}{\textbf{DESSAUERSTRASSE 19}}}{}\ledrightnote{\textcolor{pink}{Dessauer Straße}}\pend
           \pstart
           \textcolor{pink}{Berlin}{}\ledrightnote{\textcolor{pink}{Berlin}}, 11. December.\pend
           \pstart\center{}Mein lieber Freund,\pend\pstart
           Gewiß, die \textcolor{brown}{N. Fr. Pr.}{}\ledrightnote{\textcolor{brown}{Neue Freie Presse}} hat ſich \label{K_L02945-1v}\edtext{niederträchtig benommen}{\lemma{\textnormal{\emph{niederträchtig benommen}}}\Cendnote{\textnormal{Bezug auf die \textcolor{green}{Berichterstattung} der \emph{\textcolor{brown}{Neuen Freien Presse}} über die Uraufführung von \emph{\textcolor{green}{Der Schleier der Beatrice}}, vgl. Paul Goldmann an Arthur Schnitzler, 3. 12. [1900] und 9. 12. [1900]}}}\label{K_L02945-1h}. Ob man dagegen nichts thun kann? \strikeout{Jawoh}
               Jawohl. Beiſpielsweiſe: Schreib’ an das \textcolor{brown}{Blatt}{}\ledrightnote{{$\rightarrow$}\textcolor{brown}{Neue Freie Presse}} einen Brief, worin Du mittheilſt, daß Du wegen der Dir
               gegenüber bewieſenen niederträchtigen Parteilichkeit die für die {\pb}\textcolor{green}{Weihnachtsnummer}{}\ledrightnote{{$\rightarrow$}\textcolor{green}{Neue Freie Presse}} beſtimmte \textcolor{green}{Novelle}{}\ledrightnote{{$\rightarrow$}\textcolor{green}{Lieutenant Gustl. Novelle}}{ }\label{K_L02945-2v}\edtext{zurückziehſt}{\lemma{\textnormal{\emph{zurückziehſt}}}\Cendnote{\textnormal{In der Korrespondenz mit \textcolor{blue}{Theodor Herzl}, mit
               dem \textcolor{blue}{Schnitzler} die Aufnahme von
                  \emph{\textcolor{green}{Lieutenant Gustl}} verhandelte, ist von
                  der \textcolor{green}{Berichterstattung} über die Uraufführung von \emph{\textcolor{green}{Der Schleier der Beatrice}} nicht die Rede.}}}\label{K_L02945-2h}. Das wäre eine Lektion. Aber wenn Ihr
               Unabhängigen nichts gegen das \textcolor{brown}{Blatt}{}\ledrightnote{{$\rightarrow$}\textcolor{brown}{Neue Freie Presse}} thun wollt, was ſollen dann wir Abhängigen thun?\pend
           \pstart
           Die Streichung in dem Telegramm iſt offenbar erfolgt, weil man dem Herrn \label{K_L02945-12v}\edtext{\textcolor{blue}{\textsc{Loewe}}{}\ledrightnote{\textcolor{blue}{Theodor Loewe}} nicht wehthun}{\lemma{\textnormal{\emph{Loewe nicht wehthun}}}\Cendnote{\textnormal{siehe Paul Goldmann an Arthur Schnitzler, 14. 10. [1900]}}}\label{K_L02945-12h} wollte. Da hat man lieber {\pb}den Sachverhalt
               gefälſcht und den Autor geſchädigt.\pend
           \pstart
           Viele treue Grüße! {\\[\baselineskip]}Dein {\\[\baselineskip]}\spacefill\mbox{Paul Goldmann.}\pend
           \leftskip=0em{}\endnumbering\briefempfaengerindex{Schnitzler, Arthur@\textsc{Schnitzler, Arthur}!zzzGoldmann, Paul@\emph{von Paul Goldmann}!1900-12-111@{11. 12. {[}1900{]}}|)be}\mylabel{h}  \normalsize

\doendnotes{C}
\bigskip
\vfill

\clearpage

\footnotesize

\lohead{\textsc{register}}

% Definiere theindex-Environment komplett neu ohne reledmac
\makeatletter
\renewenvironment{theindex}{%
  \section*{\indexname}%
  \setlength{\parindent}{0pt}%
  \setlength{\parskip}{0pt plus 0.3pt}%
  \let\item\@idxitem
}{%
  \clearpage
}
\makeatother

\IfFileExists{\jobname-pw.ind}{\input{\jobname-pw.ind}}{}

\end{document}

      