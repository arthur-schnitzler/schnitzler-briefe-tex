%% latex-korrekturansicht-vorspann.tex
%% Vorspann für die Korrekturansicht.
%% Lädt die gemeinsame Datei latex-vorspann.tex mit gesetztem Schalter.

\newif\ifkorrekturansicht
\korrekturansichttrue

\input{../tex-inputs/latex-vorspann}


         
         \renewcommand{\erwaehntePersonen}{Personen: Alfred Kerr}
         \renewcommand{\erwaehnteOrte}{Orte: Bad Ischl, Berlin, Dessauer Straße, Toblach, Wien}
         \renewcommand{\erwaehnteWerke}{}
               \section[ Paul Goldmann an Arthur Schnitzler, 2. 8. {[}1900{]}]{Paul Goldmann an Arthur Schnitzler, 2. 8. {[}1900{]}}\nopagebreak\mylabel{v}\rehead{ }\normalsize\beginnumbering\briefempfaengerindex{Schnitzler, Arthur@\textsc{Schnitzler, Arthur}!zzzGoldmann, Paul@\emph{von Paul Goldmann}!1900-08-022@{2. 8. {[}1900{]}}|(be} \toendnotes[C]{\smallbreak\pagebreak[2]} \Standort{DLA, A:Schnitzler, HS.NZ85.1.3170.}
\physDesc{Brief, 1 Blatt, 2 Seiten
\newline{}Handschrift: blaue Tinte, deutsche Kurrent
\newline{}Schnitzler: mit Bleistift das Jahr »{[}1{]}900« vermerkt }\toendnotes[C]{\smallbreak}\pstart
           \noindent{}\textcolor{pink}{Berlin}{}\ledrightnote{\textcolor{pink}{Berlin}}, 2. Auguſt.\hfill {\pb}\textcolor{pink}{\textcolor{gray}{\textbf{DESSAUERSTRASSE 19}}}{}\ledrightnote{\textcolor{pink}{Dessauer Straße}}\pend
           \pstart\center{}Mein lieber Freund,\pend\pstart
           Ich bin mit dem \label{K_L02926-1v}\edtext{Reiſeprogramm}{\lemma{\textnormal{\emph{Reiſeprogramm}}}\Cendnote{\textnormal{siehe Paul Goldmann an Arthur Schnitzler, 16. 6. [1900]}}}\label{K_L02926-1h} einverſtanden und hoffe, am 10. Auguſt hier
               wegzukönnen. Aber beſimmt iſt es noch nicht. \strikeout{Di\textcolor{gray}{e}} Es kann früher und auch ſpäter ſein. Der Termin verſchiebt ſich jeden Tag.
               Bleibſt Du bis dahin in \label{K_L02926-2v}\edtext{\textsc{\textcolor{pink}{Ischl}{}\ledrightnote{\textcolor{pink}{Bad Ischl}}}}{\lemma{\textnormal{\emph{Ischl}}}\Cendnote{\textnormal{\textcolor{blue}{Schnitzler} war von 1. 8. 1900 bis 10. 8. 1900 in \textcolor{pink}{Ischl}.}}}\label{K_L02926-2h}? Damit ich Dich von meiner
               Abreiſe verſtändigen kann.\pend
           \pstart
           Ich muß eine Nacht in \textcolor{pink}{Wien}{}\ledrightnote{\textcolor{pink}{Wien}}{ }{\pb}bleiben. In welchem billigen \textsc{Hotel} kann ich wohnen?\pend
           \pstart
           Viele Grüße! {\\[\baselineskip]}Dein {\\[\baselineskip]}\spacefill\mbox{Paul Goldmnn}\pend
           \leftskip=0em{}\pstart
           \noindent{}Deinen Brief ſchicke ich an \textsc{\textcolor{blue}{Kerr}{}\ledrightnote{\textcolor{blue}{Alfred Kerr}}}, deſſen letzte Adreſſe \textsc{\textcolor{pink}{Toblach}{}\ledrightnote{\textcolor{pink}{Toblach}}{ }\begin{otherlanguage}{french}Poste restante\end{otherlanguage}} iſt.\pend
           \endnumbering\briefempfaengerindex{Schnitzler, Arthur@\textsc{Schnitzler, Arthur}!zzzGoldmann, Paul@\emph{von Paul Goldmann}!1900-08-022@{2. 8. {[}1900{]}}|)be}\mylabel{h}\begin{anhang}\end{anhang}\normalsize

\doendnotes{C}
\bigskip
\vfill

\clearpage

\footnotesize

\lohead{\textsc{register}}

% Definiere theindex-Environment komplett neu ohne reledmac
\makeatletter
\renewenvironment{theindex}{%
  \section*{\indexname}%
  \setlength{\parindent}{0pt}%
  \setlength{\parskip}{0pt plus 0.3pt}%
  \let\item\@idxitem
}{%
  \clearpage
}
\makeatother

\IfFileExists{\jobname-pw.ind}{\input{\jobname-pw.ind}}{}

\end{document}

      