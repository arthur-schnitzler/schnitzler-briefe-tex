%% latex-korrekturansicht-vorspann.tex
%% Vorspann für die Korrekturansicht.
%% Lädt die gemeinsame Datei latex-vorspann.tex mit gesetztem Schalter.

\newif\ifkorrekturansicht
\korrekturansichttrue

\input{../tex-inputs/latex-vorspann}


\renewcommand{\erwaehntePersonen}{Personen: Ludwig Fulda, Paul Goldmann, Karl Goldmann, Elisabeth von Heyking, Edmund Friedrich Gustav von Heyking, Theodore Rottenberg, Olga Schnitzler, Heinrich Schnitzler, Friedrich Zelnik, Ida d’Albert}
\renewcommand{\erwaehnteInstitutionen}{Institutionen: Akademisch-dramatischer Verein München, Gebrüder Paetel Verlag, Reichstag}
\renewcommand{\erwaehnteOrte}{Orte: Berlin, China, Dessauer Straße, Deutschland, Edmund-Weiß-Gasse 7, Italien, Lavarone, München, Riva del Garda, Südtirol, Trient, Wien, XVIII., Währing}
\renewcommand{\erwaehnteWerke}{Werke: Briefe, die ihn nicht erreichten, Briefe, die ihn nicht erreichten..., Die Tragödie des Triumphes, Fink und Fliederbusch. Komödie in drei Akten, Reigen. Zehn Dialoge, Tägliche Rundschau}
\section[ Paul Goldmann an Arthur Schnitzler, 27. 6. {[}1903{]}]{Paul Goldmann an Arthur Schnitzler, 27. 6. {[}1903{]}}
\nopagebreak\mylabel{v}
\rehead{ }\normalsize\beginnumbering\briefempfaengerindex{Schnitzler, Arthur@\textsc{Schnitzler, Arthur}!zzzGoldmann, Paul@\emph{von Paul Goldmann}!1903-06-271@{27. 6. {[}1903{]}}|(be}
\toendnotes[C]{\smallbreak\pagebreak[2]}\Standort{DLA, A:Schnitzler, HS.NZ85.1.3173.}
\physDesc{Brief, 1 Blatt, 4 Seiten, 1568 Zeichen
\newline{}Handschrift: blaue Tinte, deutsche Kurrent
\newline{}Schnitzler: 1) mit Bleistift das Jahr »903« und »\textsc{Nest\textcolor{gray}{l}}« vermerkt  2) mit rotem Buntstift eine einfache und eine doppelte Unterstreichung}\toendnotes[C]{\smallbreak}
\pstart
           \noindent{}\raggedleft{}{\pb}\textcolor{gray}{\textbf{\textcolor{pink}{DESSAUERSTRASSE 19}{}\ledrightnote{\textcolor{pink}{Dessauer Straße}}}}\pend
           
\pstart
           \textcolor{pink}{Berlin}{}\ledrightnote{\textcolor{pink}{Berlin}}, 27. Juni\pend
           
\pstart{}Mein lieber Freund,\pend
\pstart
           Ich habe mit den \label{K_L03375-1v}\edtext{Wahlen}{\lemma{\textnormal{\emph{Wahlen}}}\Cendnote{\textnormal{Gemeint war die \emph{\textcolor{brown}{Reichstag}}swahl am 16. 6. 1903.}}}\label{K_L03375-1h} ſchrecklich viel zu thun und kann daher erſt heut Dir und \textsc{\textcolor{blue}{Olga}{}\ledrightnote{\textcolor{blue}{Olga Schnitzler}}} für Eure lieben Grüße von \label{K_L03375-2v}\edtext{unterwegs}{\lemma{\textnormal{\emph{unterwegs}}}\Cendnote{\textnormal{siehe Paul Goldmann an Arthur Schnitzler, 2[2?]. 5. [1903]}}}\label{K_L03375-2h} vielmals danken. Alſo im Herbſt werdet Ihr Eure kleine \label{K_L03375-3v}\edtext{Wohnung}{\lemma{\textnormal{\emph{Wohnung}}}\Cendnote{\textnormal{Am 2. 9. 1903 zogen \textcolor{blue}{Olga} und \textcolor{blue}{Heinrich} in eine Wohnung in der \textcolor{pink}{Spöttelgasse 7} (heute \textcolor{pink}{Edmund-Weiß-Gasse}) im \textcolor{pink}{18. Wiener Gemeindebezirk}. Zehn Tage später, am 2. 9. 1903, zog \textcolor{blue}{Schnitzler} ein.}}}\label{K_L03375-3h} beziehen? Sie muß ſehr
               traulich und ſehr reizend ſein, nach Deiner Schilderung, und ich hoffe ſehr, daß Ihr
               darin glückliche Tage und Jahre verleben werdet.\pend
           
\pstart
           Die \label{K_L03375-4v}\edtext{»\textcolor{green}{Komödie}{}\ledrightnote{{$\rightarrow$}\textcolor{green}{Fink und Fliederbusch. Komödie in drei Akten}}«}{\lemma{\textnormal{\emph{»Komödie«}}}\Cendnote{\textnormal{\emph{\textcolor{green}{Flink und Fliederbusch}}, vgl. Paul Goldmann an Arthur Schnitzler, 2[2?]. 5. [1903]}}}\label{K_L03375-4h} wird hoffentlich
               noch feſte Geſtalt annehmen. {\pb}Wenn Dich gar nichts
               Anderes reizt, ſo denke an das »Geſchäft«, das mit einem luſtigen Stück heut zu
               machen wäre. Alle Theater würden danach greifen.\pend
           
\pstart
           Der \label{K_L03375-5v}\edtext{\textsc{\textcolor{blue}{Goldmann}{}\ledrightnote{\textcolor{blue}{Karl Goldmann}}} von der »\textcolor{green}{Tragödie des Triumphes}{}\ledrightnote{\textcolor{green}{Die Tragödie des Triumphes}}}{\lemma{\textnormal{\emph{Goldmann … Triumphes}}}\Cendnote{\textnormal{\emph{\textcolor{green}{Die Tragödie des Triumphes}} von \textcolor{blue}{Karl Goldmann} wurde am 25. 6. 1903 gemeinsam mit einzelnen Szenen aus dem \emph{\textcolor{green}{Reigen}} in \textcolor{pink}{München} in einer geschlossenen Aufführung des \emph{\textcolor{brown}{Akademisch-dramatischen Verein}}s gegeben. Unmittelbare
                  Folge der Aufführung der \emph{\textcolor{green}{Reigen}}-Szenen war
                  die Auflösung des seit 1890 bestehenden \textcolor{brown}{Verein}s. Diese Briefstelle belegt, dass
                     \textcolor{blue}{Schnitzler} bereits vorab von der
                  Inszenierung wusste.}}}\label{K_L03375-5h}« bin nicht ich. Wie man Deinen »\textcolor{green}{Reigen}{}\ledrightnote{\textcolor{green}{Reigen. Zehn Dialoge}}« aufführen will, – namentlich die \strikeout{\textcolor{gray}{r}}{ }\label{K_L03375-6v}\edtext{Gedankenſtriche}{\lemma{\textnormal{\emph{Gedankenſtriche}}}\Cendnote{\textnormal{Jede der zehn Szenen im \emph{\textcolor{green}{Reigen}} besteht aus Gesprächen vor und nach dem Geschlechtsverkehr der
                  Dialogpartnerinnen und -partner. Der Geschlechtsverkehr selbst ist in der
                  gedruckten Ausgabe mit Gedankenstrichen markiert.}}}\label{K_L03375-6h} – darauf bin ich ſehr
               neugierig. Das \textcolor{green}{Buch}{}\ledrightnote{{$\rightarrow$}\textcolor{green}{Reigen. Zehn Dialoge}} wird auch
               hier allgemein geleſen und erregt großes Entzücken.\pend
           
\pstart
           Sommerpläne habe ich noch nicht. Ich ſehe mit Schrecken meinen Urlaub herankommen.
               Mir {\pb}grauſt davor, einen Entſchluß zu faſſen. Wohin
               ſoll ich gehen? Die Welt iſt leer, und Niemand wartet auf mich.\pend
           
\pstart
           Vielleicht komme ich Anfang Auguſt nach \textcolor{pink}{Wien}{}\ledrightnote{\textcolor{pink}{Wien}} und fahre mit Dir nach \label{K_L03375-7v}\edtext{\textcolor{pink}{Südtirol}{}\ledrightnote{\textcolor{pink}{Südtirol}}}{\lemma{\textnormal{\emph{Südtirol}}}\Cendnote{\textnormal{\textcolor{blue}{Goldmann} war von 8. 8. 1903 bis 11. 8. 1903 in \textcolor{pink}{Wien} (vgl. Paul Goldmann an Arthur Schnitzler, 7. 8. [1903] und 11. 8. 1903). \textcolor{blue}{Schnitzler} traf er am 9. 8. 1903 und 11. 8. 1903. Danach reiste \textcolor{blue}{Goldmann} nach \textcolor{pink}{Südtirol} und \textcolor{pink}{Italien}, wo er mit \textcolor{blue}{Theodore Rottenberg} zusammentraf, mit der es zur
                  Versöhnung gekommen war (vgl. Paul Goldmann an Arthur Schnitzler, 11. 8. 1903). In Folge trafen sich die drei zumindest am 18. 8. 1903 in \textcolor{pink}{Riva del Garda} (vgl. Paul Goldmann und Theodore Rottenberg an Arthur
               Schnitzler, 18. 8. [1903]), am Folgetag dann wieder in \textcolor{pink}{Trient}, von wo sie nach einer Übernachtung zu dritt nach \textcolor{pink}{Lavarone} gingen. Am 21. 8. 1903 trennte
                  sich \textcolor{blue}{Schnitzler} von den beiden und fuhr über
                     \textcolor{pink}{Trient} wieder nach \textcolor{pink}{Wien}.}}}\label{K_L03375-7h}.\pend
           
\pstart
           Die \label{K_L03375-8v}\edtext{\textsc{\textcolor{blue}{Fulda}{}\ledrightnote{\textcolor{blue}{Ludwig Fulda}{\newline}\textcolor{blue}{Ida d’Albert}}}’ſche Eheſcheidung}{\lemma{\textnormal{\emph{Fulda’ſche Eheſcheidung}}}\Cendnote{\textnormal{siehe Paul Goldmann an Arthur Schnitzler, 15. 6. [1903]}}}\label{K_L03375-8h} geht ihren Gang. \textcolor{blue}{Sie}{}\ledrightnote{{$\rightarrow$}\textcolor{blue}{Ida d’Albert}}
               hat ihren \textcolor{blue}{Mann}{}\ledrightnote{{$\rightarrow$}\textcolor{blue}{Ludwig Fulda}} ſo lange
               gequält, bis er es nicht mehr aushielt\strikeout{,} und auf
               Scheidung klagte. Es iſt eine große Dummheit von ihr, daß ſie es ſo weit kommen ließ;
                  {\pb}denn \textcolor{blue}{ſie}{}\ledrightnote{{$\rightarrow$}\textcolor{blue}{Ida d’Albert}} wird den Sturz von der ſocialen Höhe, auf der ſie \strikeout{\textcolor{gray}{ſte}h\textcolor{gray}{t,}} bisher ſtand, doch nicht vertragen.\pend
           
\pstart
           Lies: \label{K_L03375-9v}\edtext{»\textcolor{green}{Briefe, die ihn nicht erreichten}{}\ledrightnote{\textcolor{green}{Briefe, die ihn nicht erreichten}}«}{\lemma{\textnormal{\emph{»Briefe, … erreichten«}}}\Cendnote{\textnormal{[\textcolor{blue}{Elisabeth von Heyking}]: \emph{\textcolor{green}{Briefe, die ihn nicht erreichten}}. \textcolor{pink}{Berlin}: \emph{\textcolor{brown}{Gebrüder Paetel}}{ }1903, Vorabdruck in der \emph{\textcolor{green}{Täglichen Rundschau}}{ }1902. Eine Lektüre durch \textcolor{blue}{Schnitzler} ist
                  nicht belegt. Am 14. 10. 1925 sah er die gleichnamige \textcolor{green}{Verfilmung} von \textcolor{blue}{Friedrich
                     Zelnik}.}}}\label{K_L03375-9h}. Verfaſſerin iſt die Baronin \textsc{\textcolor{blue}{Heyking}{}\ledrightnote{\textcolor{blue}{Elisabeth von Heyking}}}, die Frau des ehemaligen \textcolor{pink}{deutſch}{}\ledrightnote{{$\rightarrow$}\textcolor{pink}{Deutschland}}en \textcolor{blue}{Geſandten}{}\ledrightnote{{$\rightarrow$}\textcolor{blue}{Edmund Friedrich Gustav von Heyking}} in \textcolor{pink}{China}{}\ledrightnote{\textcolor{pink}{China}}.\pend
           
\pstart
           Grüße \textsc{\textcolor{blue}{Olga}{}\ledrightnote{\textcolor{blue}{Olga Schnitzler}}} vielmals und ſei auch Du herzlichſt gegrüßt von Deinem {\\[\baselineskip]}\spacefill\mbox{Paul Goldm}\pend
           \leftskip=0em{}\endnumbering\briefempfaengerindex{Schnitzler, Arthur@\textsc{Schnitzler, Arthur}!zzzGoldmann, Paul@\emph{von Paul Goldmann}!1903-06-271@{27. 6. {[}1903{]}}|)be}\mylabel{h}  \normalsize

\doendnotes{C}
\bigskip
\vfill

\clearpage

\footnotesize

\lohead{\textsc{register}}

% Definiere theindex-Environment komplett neu ohne reledmac
\makeatletter
\renewenvironment{theindex}{%
  \section*{\indexname}%
  \setlength{\parindent}{0pt}%
  \setlength{\parskip}{0pt plus 0.3pt}%
  \let\item\@idxitem
}{%
  \clearpage
}
\makeatother

\IfFileExists{\jobname-pw.ind}{\input{\jobname-pw.ind}}{}

\end{document}

      