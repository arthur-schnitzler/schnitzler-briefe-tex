%% latex-korrekturansicht-vorspann.tex
%% Vorspann für die Korrekturansicht.
%% Lädt die gemeinsame Datei latex-vorspann.tex mit gesetztem Schalter.

\newif\ifkorrekturansicht
\korrekturansichttrue

\input{../tex-inputs/latex-vorspann}


               \section[Richard Beer-Hofmann an Arthur Schnitzler, 24. 12. 1898]{ Richard Beer-Hofmann an Arthur Schnitzler, 24. 12. 1898}\nopagebreak\mylabel{v}\rehead{ }\normalsize\beginnumbering\briefempfaengerindex{Schnitzler, Arthur@\textsc{Schnitzler, Arthur}!zzzBeer-Hofmann, Richard@\emph{von Richard Beer-Hofmann}!1898-12-241@{24. 12. 1898}|(be} \toendnotes[C]{\smallbreak\pagebreak[2]} \Standort{CUL, Schnitzler, B 8.}
\physDesc{Brief, 1 Blatt, 2 Seiten
\newline{}Handschrift: schwarze Tinte, lateinische Kurrent\newline{}Ordnung: mit Bleistift von unbekannter Hand nummeriert: »125« }\buchAbdrucke{\weitereDrucke{Arthur Schnitzler, Richard Beer-Hofmann: \emph{Briefwechsel 1891–1931}. Hg. Konstanze Fliedl. Wien, Zürich: \emph{Europaverlag} 1992, S. 126.} }\pstart
           \raggedleft{}{\pb}24/XII 98\pend
           \pstart
           Da Sie mir die Wahl lassen, lieber Arthur – so betrachte ich es als Hochzeitsgeschenk
               damit ich erst bei Ihrer Hochzeit Ihnen ein Geschenk machen muß, als
               Geschmacklosigkeit, »no ja weil’s wahr ist«. Diese Vase ist {\pb}»\textcolor{blue}{Clement Massier}{}\ledrightnote{\textcolor{blue}{Clément Massier}}. \textcolor{pink}{Golf St. Juan}{}\ledrightnote{\textcolor{pink}{Golfe-Juan}} bei \textcolor{pink}{Nizza}{}\ledrightnote{\textcolor{pink}{Nizza}},
               Reflêt metallic (que?){[}«{]}. Sie müßen aber nicht glauben daß das
               was Besonderes ist.\pend
           \pstart
           Von Herzen Ihr{\\[\baselineskip]}\spacefill\mbox{Richard}\pend
           \leftskip=0em{}\endnumbering\briefempfaengerindex{Schnitzler, Arthur@\textsc{Schnitzler, Arthur}!zzzBeer-Hofmann, Richard@\emph{von Richard Beer-Hofmann}!1898-12-241@{24. 12. 1898}|)be}\mylabel{h}  \normalsize

\doendnotes{C}
\bigskip
\vfill

\clearpage

\footnotesize

\lohead{\textsc{register}}

% Definiere theindex-Environment komplett neu ohne reledmac
\makeatletter
\renewenvironment{theindex}{%
  \section*{\indexname}%
  \setlength{\parindent}{0pt}%
  \setlength{\parskip}{0pt plus 0.3pt}%
  \let\item\@idxitem
}{%
  \clearpage
}
\makeatother

\IfFileExists{\jobname-pw.ind}{\input{\jobname-pw.ind}}{}

\end{document}

      