%% latex-korrekturansicht-vorspann.tex
%% Vorspann für die Korrekturansicht.
%% Lädt die gemeinsame Datei latex-vorspann.tex mit gesetztem Schalter.

\newif\ifkorrekturansicht
\korrekturansichttrue

\input{../tex-inputs/latex-vorspann}


\renewcommand{\erwaehnteOrte}{Orte: Bad Reichenhall, Genf, Ungarn, Wattmanngasse, Wien}
\renewcommand{\erwaehnteWerke}{}
\section[ Felix Salten an Arthur Schnitzler, 5. 9. 1898]{Felix Salten an Arthur Schnitzler, 5. 9. 1898}
\nopagebreak\mylabel{v}
\rehead{ }\normalsize\beginnumbering\briefempfaengerindex{Schnitzler, Arthur@\textsc{Schnitzler, Arthur}!zzzSalten, Felix@\emph{von Felix Salten}!1898-09-052@{5. 9. 1898}|(be}
\toendnotes[C]{\smallbreak\pagebreak[2]}\Standort{CUL, Schnitzler, B 89, A 2.}
\physDesc{Brief, 1 Blatt, 1 Seite, 425 Zeichen
\newline{}Handschrift: Bleistift, lateinische Kurrent
\newline{}Ordnung: mit Bleistift von unbekannter Hand nummeriert: »105« }\toendnotes[C]{\smallbreak}
\pstart
           \raggedleft{}{\pb}\textcolor{pink}{Hietzing, Wattmanngaße 11}{}\ledrightnote{\textcolor{pink}{Wattmanngasse}}{\\}5. Septemb.\pend
           
\pstart
           Lieber Arthur, ich war die ganze Zeit, vom 4. August bis zum 28., \label{K_L03281-1v}\edtext{fort}{\lemma{\textnormal{\emph{fort}}}\Cendnote{\textnormal{siehe Felix Salten an Arthur Schnitzler, 30. 7. 1898}}}\label{K_L03281-1h}. Theils in \textcolor{pink}{Ungarn}{}\ledrightnote{\textcolor{pink}{Ungarn}}, theils \textcolor{pink}{Reichenhall}{}\ledrightnote{\textcolor{pink}{Bad Reichenhall}}, und bekam nichts nachgesendet. Am
                  28\textsuperscript{ten} aber war es auch für Ihre \label{K_L03281-2v}\edtext{\textcolor{pink}{Genf}{}\ledrightnote{\textcolor{pink}{Genf}}er Adreße}{\lemma{\textnormal{\emph{Genfer Adreße}}}\Cendnote{\textnormal{\textcolor{blue}{Schnitzler} war von 16. 8. 1898 bis 18. 8. 1898 in \textcolor{pink}{Genf} gewesen.}}}\label{K_L03281-2h} schon zu spät. Also
               entschuldigen Sie, dass ich nichts hören ließ, und erst heute für Ihre lieben Karten danke. Wenn Sie \label{K_L03281-3v}\edtext{schon in \textcolor{pink}{Wien}{}\ledrightnote{\textcolor{pink}{Wien}}}{\lemma{\textnormal{\emph{schon in Wien}}}\Cendnote{\textnormal{\textcolor{blue}{Schnitzler} war am 3. 9. 1898 nach \textcolor{pink}{Wien} zurückgekehrt.}}}\label{K_L03281-3h} sind, senden Sie mir
               eine Zeile, wann wir uns sehen können. herzlichst Ihr {\\}\spacefill\mbox{Salten}\pend
           \endnumbering\briefempfaengerindex{Schnitzler, Arthur@\textsc{Schnitzler, Arthur}!zzzSalten, Felix@\emph{von Felix Salten}!1898-09-052@{5. 9. 1898}|)be}\mylabel{h}  \normalsize

\doendnotes{C}
\bigskip
\vfill

\clearpage

\footnotesize

\lohead{\textsc{register}}

% Definiere theindex-Environment komplett neu ohne reledmac
\makeatletter
\renewenvironment{theindex}{%
  \section*{\indexname}%
  \setlength{\parindent}{0pt}%
  \setlength{\parskip}{0pt plus 0.3pt}%
  \let\item\@idxitem
}{%
  \clearpage
}
\makeatother

\IfFileExists{\jobname-pw.ind}{\input{\jobname-pw.ind}}{}

\end{document}

      