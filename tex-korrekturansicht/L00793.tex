%% latex-korrekturansicht-vorspann.tex
%% Vorspann für die Korrekturansicht.
%% Lädt die gemeinsame Datei latex-vorspann.tex mit gesetztem Schalter.

\newif\ifkorrekturansicht
\korrekturansichttrue

\input{../tex-inputs/latex-vorspann}


               \section[Fritz Schlesinger u. a. an Hermann Bahr, 21. 4. 1898]{ Fritz Schlesinger u. a. an Hermann Bahr, 21. 4. 1898}\nopagebreak\mylabel{v}\rehead{ }\normalsize\beginnumbering\briefempfaengerindex{Bahr, Hermann@\textsc{Bahr, Hermann}!zzzFranckenstein, Georg von@\emph{von Georg von Franckenstein}!1898-04-211@{21. 4. 1898}|(be}\briefempfaengerindex{Bahr, Hermann@\textsc{Bahr, Hermann}!zzzSchnitzler, Arthur@\emph{von Arthur Schnitzler}!1898-04-211@{21. 4. 1898}|(be}\briefempfaengerindex{Bahr, Hermann@\textsc{Bahr, Hermann}!zzzHofmannsthal, Gertrude von@\emph{von Gertrude von Hofmannsthal}!1898-04-211@{21. 4. 1898}|(be}\briefempfaengerindex{Bahr, Hermann@\textsc{Bahr, Hermann}!zzzSchlesinger, Friedrich@\emph{von Friedrich Schlesinger}!1898-04-211@{21. 4. 1898}|(be} \toendnotes[C]{\smallbreak\pagebreak[2]} \Standort{TMW, HS AM 57775 Ba.}
\physDesc{Postkarte
\newline{}Handschrift Friedrich Schlesinger: Bleistift, lateinische Kurrent\newline{}Handschrift Gertrude von Hofmannsthal: Bleistift, lateinische Kurrent\newline{}Handschrift Arthur Schnitzler: Bleistift, deutsche Kurrent\newline{}Handschrift Georg von Franckenstein: Bleistift, lateinische Kurrent\newline{}Versand: 1) Stempel: »\nobreak{}\oindex{Breitenfurt bei Wien@\textbf{Breitenfurt bei Wien}, \emph{Besiedelter Ort (A.BSO)}|pwk}\textcolor{gray}{Brei}\damage{ten}furt, 21 4 98\nobreak{}«.  2) Stempel: »\nobreak{}Bestellt, \oindex{IX., Alsergrund@\textbf{IX., Alsergrund}, \emph{Bezirk (A.BZK)}|pwk}Wien 9/2, 22 4. 98, 2 \textcolor{gray}{½} N\nobreak{}«. }\buchAbdrucke{\weitereDrucke{Hermann Bahr, Arthur Schnitzler: \emph{Briefwechsel, Aufzeichnungen, Dokumente (1891–1931)}. Hg. Kurt Ifkovits und Martin Anton Müller. Göttingen: \emph{Wallstein} 2018, S. 162.} }\toendnotes[C]{\smallbreak}\pstart{}{\pb}Herrn Hermann
                  Bahr\pend{}\pstart{}\textcolor{pink}{IX. Porzellangasse 37}{}\ledrightnote{\textcolor{pink}{Porzellangasse}}\pend{}\pstart{}\textcolor{pink}{Wien}{}\ledrightnote{\textcolor{pink}{Wien}}\pend{}{\bigskip}\pstart{[}Abbildung{]}\pend\pstart
           \noindent{}{\pb}\uline{\textcolor{pink}{Breitenfurth}{}\ledrightnote{\textcolor{pink}{Breitenfurt bei Wien}}}.\pend
           \pstart
           Der Dichter ist oft sehr zerstreut\pend
           \pstart
           Was sein Bicycle nicht erfreut\pend
           \pstart
           Die Bremse wohl sehr wichtig ist\pend
           \pstart
           Weil sonst man in den Graben schießt. \introOben{}\label{K_L00793_1v}\edtext{\textcolor{blue}{Hugo}{}\ledrightnote{\textcolor{blue}{Hugo von Hofmannsthal}}}{\lemma{\textnormal{\emph{Hugo}}}\Cendnote{\textnormal{Als Beschriftung der stürzenden
                     Person auf der Bleistiftzeichnung gewertet. Es ließe sich auch als Unterschrift
                        \textcolor{blue}{Hofmannsthals} deuten. Im \emph{\textcolor{green}{Tagebuch}} nennt \textcolor{blue}{Schnitzler} diesen und zusätzlich die Mutter \textcolor{blue}{Franziska Schlesinger} als weitere Teilnehmer der Radtour,
                     übergeht jedoch \textcolor{blue}{Fritz Schlesinger}.}}}\label{K_L00793_1h}\introOben{}\pend
           \pstart
           \spacefill\mbox{Fritz Schlesinger}{\\[\baselineskip]}\spacefill\mbox{{[}hs. Franckenstein:{]} G Franckenste\damage{in}}{\\[\baselineskip]}{[}hs. Hofmannsthal:{]} Beneiden Sie uns ein bisserl, ja?\hspace*{2.5em}\spacefill\mbox{Gerty}{\\[\baselineskip]}{[}hs. Schnitzler:{]} HerzGruß\hspace*{1.5em}\spacefill\mbox{ArthSchnitzler}\pend
           \leftskip=0em{}\endnumbering\briefempfaengerindex{Bahr, Hermann@\textsc{Bahr, Hermann}!zzzFranckenstein, Georg von@\emph{von Georg von Franckenstein}!1898-04-211@{21. 4. 1898}|)be}\briefempfaengerindex{Bahr, Hermann@\textsc{Bahr, Hermann}!zzzSchnitzler, Arthur@\emph{von Arthur Schnitzler}!1898-04-211@{21. 4. 1898}|)be}\briefempfaengerindex{Bahr, Hermann@\textsc{Bahr, Hermann}!zzzHofmannsthal, Gertrude von@\emph{von Gertrude von Hofmannsthal}!1898-04-211@{21. 4. 1898}|)be}\briefempfaengerindex{Bahr, Hermann@\textsc{Bahr, Hermann}!zzzSchlesinger, Friedrich@\emph{von Friedrich Schlesinger}!1898-04-211@{21. 4. 1898}|)be}\mylabel{h}  \normalsize

\doendnotes{C}
\bigskip
\vfill

\clearpage

\footnotesize

\lohead{\textsc{register}}

% Definiere theindex-Environment komplett neu ohne reledmac
\makeatletter
\renewenvironment{theindex}{%
  \section*{\indexname}%
  \setlength{\parindent}{0pt}%
  \setlength{\parskip}{0pt plus 0.3pt}%
  \let\item\@idxitem
}{%
  \clearpage
}
\makeatother

\IfFileExists{\jobname-pw.ind}{\input{\jobname-pw.ind}}{}

\end{document}

      