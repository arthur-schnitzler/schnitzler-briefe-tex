%% latex-korrekturansicht-vorspann.tex
%% Vorspann für die Korrekturansicht.
%% Lädt die gemeinsame Datei latex-vorspann.tex mit gesetztem Schalter.

\newif\ifkorrekturansicht
\korrekturansichttrue

\input{../tex-inputs/latex-vorspann}


\renewcommand{\erwaehntePersonen}{Personen: Samuel Fischer, Hedwig Fischer, Alois Pahler, Felix Salten, Ottilie Salten, Olga Schnitzler, Heinrich Schnitzler, Jakob Wassermann}
\renewcommand{\erwaehnteInstitutionen}{Institutionen: Südbahnstrecke}
\renewcommand{\erwaehnteOrte}{Orte: Croda Rossa, Edlach, Hôtel Dürrenstein, Höhlenstein, Niederösterreich, Prato Piazza, Südtirol}
\renewcommand{\erwaehnteWerke}{}
\section[Felix Salten u. a. an Arthur Schnitzler, {[}zwischen 19. und 30. 7.? 1909{]}]{Felix Salten u. a. an Arthur Schnitzler,
               {[}zwischen 19. und 30. 7.? 1909{]}}
\nopagebreak\mylabel{v}
\rehead{ }\normalsize\beginnumbering\briefempfaengerindex{Schnitzler, Arthur@\textsc{Schnitzler, Arthur}!zzzFischer, Hedwig@\emph{von Hedwig Fischer}!1909-07-301@{{[}zwischen 19. und 30. 7.? 1909{]}}|(be}\briefempfaengerindex{Schnitzler, Arthur@\textsc{Schnitzler, Arthur}!zzzFischer, Samuel@\emph{von Samuel Fischer}!1909-07-301@{{[}zwischen 19. und 30. 7.? 1909{]}}|(be}\briefempfaengerindex{Schnitzler, Arthur@\textsc{Schnitzler, Arthur}!zzzWassermann, Jakob@\emph{von Jakob Wassermann}!1909-07-301@{{[}zwischen 19. und 30. 7.? 1909{]}}|(be}\briefempfaengerindex{Schnitzler, Arthur@\textsc{Schnitzler, Arthur}!zzzSalten, Ottilie@\emph{von Ottilie Salten}!1909-07-301@{{[}zwischen 19. und 30. 7.? 1909{]}}|(be}\briefempfaengerindex{Schnitzler, Arthur@\textsc{Schnitzler, Arthur}!zzzSalten, Felix@\emph{von Felix Salten}!1909-07-301@{{[}zwischen 19. und 30. 7.? 1909{]}}|(be}
\toendnotes[C]{\smallbreak\pagebreak[2]}\Standort{CUL, Schnitzler, B 89, B 1.}
\physDesc{Bildpostkarte, 450 Zeichen
\newline{}Handschrift Felix Salten: Bleistift, lateinische Kurrent
\newline{}Handschrift Ottilie Salten: Bleistift, deutsche Kurrent
\newline{}Handschrift Hedwig Fischer: Bleistift, deutsche Kurrent
\newline{}Handschrift Jakob Wassermann: Bleistift, lateinische Kurrent
\newline{}Handschrift Samuel Fischer: Bleistift, lateinische Kurrent
\newline{}Versand: 1) Stempel: »\nobreak{}\oindex{Hôtel Duerrenstein@\textbf{Hôtel Dürrenstein}, \emph{Hotel (K.HTL)}|pwk}Hôtel Dürrenstein, 2000 M. \textcolor{pink}{Plätzwiese} 2000 M.\textcolor{blue}{Alois Pahler}\nobreak{}«.   2) Stempel: »\nobreak{}\oindex{Hoehlenstein@\textbf{Höhlenstein}, \emph{P.PPLQ}|pwk}{[}L{]}andro, 8\nobreak{}«. 
\newline{}Schnitzler: mit Bleistift Vermerk: »\textsc{Salten}« 
\newline{}Ordnung: mit Bleistift von unbekannter Hand nummeriert: »254« }\toendnotes[C]{\smallbreak}\pstart{}{\pb}Herrn\pend{}\pstart{}D\textsuperscript{r} Arthur Schnitzler\pend{}\pstart{}\textcolor{pink}{Edlach \textsuperscript{b}/Reichenau}{}\ledrightnote{\textcolor{pink}{Edlach}}\pend{}\pstart{}\textcolor{brown}{Südbahn}{}\ledrightnote{\textcolor{brown}{Südbahnstrecke}}\pend{}\pstart{}\textcolor{pink}{Nied\textcolor{gray}{.} Öst}{}\ledrightnote{\textcolor{pink}{Niederösterreich}}\pend{}
{\bigskip}
\pstart
           \noindent{}{\pb}\textcolor{gray}{\textbf{\textcolor{pink}{Plätzwiesen}{}\ledrightnote{\textcolor{pink}{Prato Piazza}} (2003 m) mit \textcolor{pink}{Hoher Gaisl}{}\ledrightnote{\textcolor{pink}{Croda Rossa}} (3148 m).}}\hfill \textcolor{gray}{\textbf{\textcolor{pink}{Tirol}{}\ledrightnote{\textcolor{pink}{Südtirol}}.}}\pend
           
\pstart
           {\pb}Schöner Weg – schönes Ausruhen und herzliches Gedenken der Entfernten. Hoffentlich
               geht es Ihrer \label{K_L03504-1v}\edtext{\textcolor{blue}{Frau}{}\ledrightnote{{$\rightarrow$}\textcolor{blue}{Olga Schnitzler}} dauernd gut u. \textcolor{blue}{Heini}{}\ledrightnote{\textcolor{blue}{Heinrich Schnitzler}} ist ganz gesund}{\lemma{\textnormal{\emph{Frau … gesund}}}\Cendnote{\textnormal{Die Karte ist undatiert und lässt sich zeitlich nur anhand
                  einiger Indizien einem Zeitraum zuordnen: \textcolor{blue}{Olga
                     Schnitzler} war schwanger und hatte zeitweise Beschwerden, vgl. A. S.: \emph{Tagebuch}, 26. 6. 1909. \textcolor{blue}{Heinrich}s Keuchhusten heilte Anfang Juli 1909 aus. \textcolor{blue}{Samuel Fischer} schrieb am 20. 7. 1909 aus \textcolor{pink}{Landro} an \textcolor{blue}{Schnitzler} (vgl. \emph{Briefwechsel mit Autoren}, S. 84). Nachdem in der Karte \textcolor{blue}{Salten}s vom 18. 7. 1909 die Anwesenheit \textcolor{blue}{Fischer}s nicht erwähnt wird und \textcolor{blue}{Heinrich}s Keuchhusten erst »besser« geworden ist, dürfte die vorliegende Karte
                  danach abgefasst worden sein – und vor dem Monatsende, da in der Karte vom 31. 7. 1909 nicht mehr nach
                  dem Befinden \textcolor{blue}{Heinrich}s gefragt
               wird.}}}\label{K_L03504-1h}. Alles herzliche von uns zu Ihnen\pend
           \pstart Ihr \spacefill\mbox{Salten}\pend{}
\pstart
           \noindent{}{[}hs. Ottilie Salten:{]} Viele ſchöne Grüße {\\}\spacefill\mbox{Otti}\pend
           
\pstart
           \noindent{}{[}hs. Hedwig Fischer:{]} herzliche Grüße und viele gute Wünsche für Frau \textcolor{blue}{\textsc{Schnitzler}}{}\ledrightnote{\textcolor{blue}{Olga Schnitzler}} u. \textcolor{blue}{\textsc{Heini}}{}\ledrightnote{\textcolor{blue}{Heinrich Schnitzler}}{ }{\\}\spacefill\mbox{Hedwig Fischer.}\pend
           
\pstart
           \noindent{}{[}hs. Wassermann:{]} Herzlich grüsst Ihr \spacefill\mbox{JakobWassermann}\pend
           
\pstart
           \noindent{}{[}hs. Samuel Fischer:{]} Herzliche Grüße Ihr \spacefill\mbox{SFischer}\pend
           \endnumbering\briefempfaengerindex{Schnitzler, Arthur@\textsc{Schnitzler, Arthur}!zzzFischer, Hedwig@\emph{von Hedwig Fischer}!1909-07-191@{{[}zwischen 19. und 30. 7.? 1909{]}}|)be}\briefempfaengerindex{Schnitzler, Arthur@\textsc{Schnitzler, Arthur}!zzzFischer, Samuel@\emph{von Samuel Fischer}!1909-07-191@{{[}zwischen 19. und 30. 7.? 1909{]}}|)be}\briefempfaengerindex{Schnitzler, Arthur@\textsc{Schnitzler, Arthur}!zzzWassermann, Jakob@\emph{von Jakob Wassermann}!1909-07-191@{{[}zwischen 19. und 30. 7.? 1909{]}}|)be}\briefempfaengerindex{Schnitzler, Arthur@\textsc{Schnitzler, Arthur}!zzzSalten, Ottilie@\emph{von Ottilie Salten}!1909-07-191@{{[}zwischen 19. und 30. 7.? 1909{]}}|)be}\briefempfaengerindex{Schnitzler, Arthur@\textsc{Schnitzler, Arthur}!zzzSalten, Felix@\emph{von Felix Salten}!1909-07-191@{{[}zwischen 19. und 30. 7.? 1909{]}}|)be}\mylabel{h}  \normalsize

\doendnotes{C}
\bigskip
\vfill

\clearpage

\footnotesize

\lohead{\textsc{register}}

% Definiere theindex-Environment komplett neu ohne reledmac
\makeatletter
\renewenvironment{theindex}{%
  \section*{\indexname}%
  \setlength{\parindent}{0pt}%
  \setlength{\parskip}{0pt plus 0.3pt}%
  \let\item\@idxitem
}{%
  \clearpage
}
\makeatother

\IfFileExists{\jobname-pw.ind}{\input{\jobname-pw.ind}}{}

\end{document}

      