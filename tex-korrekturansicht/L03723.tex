%% latex-korrekturansicht-vorspann.tex
%% Vorspann für die Korrekturansicht.
%% Lädt die gemeinsame Datei latex-vorspann.tex mit gesetztem Schalter.

\newif\ifkorrekturansicht
\korrekturansichttrue

\input{../tex-inputs/latex-vorspann}


\section[Elsa Plessner an Arthur Schnitzler, 9. 1. 1900]{L03723 Elsa Plessner an Arthur Schnitzler, 9. 1. 1900}
\nopagebreak\mylabel{L03723v}
\rehead{ }\normalsize\beginnumbering\briefempfaengerindex{Schnitzler, Arthur@\textsc{Schnitzler, Arthur}!zzzPlessner, Elsa@\emph{von Elsa Plessner}!1900-01-091@{9. 1. 1900}|(be}
\toendnotes[C]{\smallbreak\pagebreak[2]}
\correspDesc{Versand  durch Elsa Plessner am 9. 1. 1900 in Wien
\newline{}Erhalt  durch Arthur Schnitzler im Zeitraum [9. 1. 1900
                  – 12. 1. 1900?] in Wien}\toendnotes[C]{\smallbreak}
\Standort{DLA, A:Schnitzler, HS.1985.1.419.}
\physDesc{Brief, 1 Blatt, 4 Seiten, 1645 Zeichen
\newline{}Handschrift: schwarze Tinte, lateinische Kurrent}\toendnotes[C]{\smallbreak}
\pstart
           \raggedleft{}{\pb}\textcolor{pink}{Wien I. Kärnthnerstraße N\textsuperscript{o} 10}\oindex{Wien@\textbf{Wien}!I., Innere Stadt@\textbf{I., Innere Stadt}!Kärntner Straße 10@\textbf{Kärntner Straße 10}, \emph{Wohngebäude}|pw}{}\ledrightnote{\textcolor{pink}{Kärntner Straße 10}}\pend
           
\pstart
           \raggedleft{}den 9. Januar 1900\pend
           
\pstart{}Verehrter Herr Doctor!\pend\vspace{0.5em}
\pstart
           Nach langer Zeit erlaube ich mir heute wieder einmal Sie an mein
               bescheidenes Vorhandensein zu erinnern und Ihr Urtheil über eine \textcolor{green}{Arbeit}\pwindex{Plessner, Elsa 22.\,8.\,1875 Wien – 7.\,5.\,1932 Alicante@\textsc{Plessner, Elsa} (22.\,8.\,1875 Wien – 7.\,5.\,1932 Alicante), \emph{Schriftstellerin}!erste Kapitel. Schauspiel in drei Akten@\strich\emph{Das erste Kapitel. Schauspiel in drei Akten}|pwv}{}\ledrightnote{{$\rightarrow$}\emph{\textcolor{green}{Das erste Kapitel. Schauspiel in drei Akten}}} zu erbitten. Es ist wiederum ein
                  \textcolor{green}{Stück}\pwindex{Plessner, Elsa 22.\,8.\,1875 Wien – 7.\,5.\,1932 Alicante@\textsc{Plessner, Elsa} (22.\,8.\,1875 Wien – 7.\,5.\,1932 Alicante), \emph{Schriftstellerin}!erste Kapitel. Schauspiel in drei Akten@\strich\emph{Das erste Kapitel. Schauspiel in drei Akten}|pwv}{}\ledrightnote{{$\rightarrow$}\emph{\textcolor{green}{Das erste Kapitel. Schauspiel in drei Akten}}}, ein dreiactiges
               Schauspiel, das ich Ihrer Kritik unterbreite, indem ich die leise Hoffnung hege, dass
               dieses – mein letztes \textcolor{green}{Opus}\pwindex{Plessner, Elsa 22.\,8.\,1875 Wien – 7.\,5.\,1932 Alicante@\textsc{Plessner, Elsa} (22.\,8.\,1875 Wien – 7.\,5.\,1932 Alicante), \emph{Schriftstellerin}!erste Kapitel. Schauspiel in drei Akten@\strich\emph{Das erste Kapitel. Schauspiel in drei Akten}|pwv}{}\ledrightnote{{$\rightarrow$}\emph{\textcolor{green}{Das erste Kapitel. Schauspiel in drei Akten}}} –
               Ihnen nicht allzusehr missfallen dürfte.\pend
           
\pstart
           {\pb}Ich kann es noch immer nicht verschmerzen, dass ich Ihnen – der Sie doch
               seinerzeit einige Hoffnungen auf mein Talent setzten – fortwährend Enttäuschungen
               bereitet habe und wenn ich auch ein Jahr lang von mir nichts habe hören lassen, so
               dürfen Sie nicht glauben, dass mich \label{K_L03723-1v}\edtext{Ihr abfälliges Urtheil}{\lemma{\textnormal{\emph{Ihr abfälliges Urtheil}}}\Cendnote{\textnormal{\textcolor{blue}{Schnitzlers} Brief aus dem Januar 1899 ist
                  nicht überliefert, zu \textcolor{blue}{Plessners}\pwindex{Plessner, Elsa 22.\,8.\,1875 Wien – 7.\,5.\,1932 Alicante@\textsc{Plessner, Elsa} (22.\,8.\,1875 Wien – 7.\,5.\,1932 Alicante), \emph{Schriftstellerin}|pwk} Reaktion auf
                  seine Kritik vgl. Elsa Plessner an Arthur Schnitzler, 26. 1. 1899.}}}\label{K_L03723-1}
               über das letzte \textcolor{green}{Stück}\pwindex{Plessner, Elsa 22.\,8.\,1875 Wien – 7.\,5.\,1932 Alicante@\textsc{Plessner, Elsa} (22.\,8.\,1875 Wien – 7.\,5.\,1932 Alicante), \emph{Schriftstellerin}!Ehrlosen. Schauspiel in drei Acten@\strich\emph{Die Ehrlosen. Schauspiel in drei Acten}|pwv}{}\ledrightnote{{$\rightarrow$}\emph{\textcolor{green}{Die Ehrlosen. Schauspiel in drei Acten}}}
               abgeschreckt habe. Welches äußerliche Schicksal dieses \textcolor{green}{Schauspiel}\pwindex{Plessner, Elsa 22.\,8.\,1875 Wien – 7.\,5.\,1932 Alicante@\textsc{Plessner, Elsa} (22.\,8.\,1875 Wien – 7.\,5.\,1932 Alicante), \emph{Schriftstellerin}!erste Kapitel. Schauspiel in drei Akten@\strich\emph{Das erste Kapitel. Schauspiel in drei Akten}|pwv}{}\ledrightnote{{$\rightarrow$}\emph{\textcolor{green}{Das erste Kapitel. Schauspiel in drei Akten}}} auch erleben möge – Sie bleiben
               doch die Verkörperung meines besseren, literarischen Ich – d. h. ich habe so eine {\pb}dunkle Empfindung als ob ich Ihnen Rechenschaft schuldig wäre, über die
               Verwaltung meines Talentes – . Diese meine Ansicht ist ja vielleicht etwas zeitraubend für Sie – aber wenn etwas
               Ersprießliches dabei herauskommen sollte, glaube ich doch, dass es Ihnen ein wenig
               Spaß macht. –\pend
           
\pstart
           Beifolgende feingeäderte \textcolor{green}{Seelengeschichte}\pwindex{Plessner, Elsa 22.\,8.\,1875 Wien – 7.\,5.\,1932 Alicante@\textsc{Plessner, Elsa} (22.\,8.\,1875 Wien – 7.\,5.\,1932 Alicante), \emph{Schriftstellerin}!erste Kapitel. Schauspiel in drei Akten@\strich\emph{Das erste Kapitel. Schauspiel in drei Akten}|pwv}{}\ledrightnote{{$\rightarrow$}\emph{\textcolor{green}{Das erste Kapitel. Schauspiel in drei Akten}}} zartester Structur dürfte Ihnen beweisen, dass ich mich
               bemühe, nicht zu verflachen. Ob mein Bemühen auch {\pb}Erfolg hatte – das
               bitte ich Sie, mir zu sagen. Besonders der dritte Act liegt mir am Herzen und ich
               erwarte Ihr Urtheil über »\textcolor{green}{das erste Capitel}\pwindex{Plessner, Elsa 22.\,8.\,1875 Wien – 7.\,5.\,1932 Alicante@\textsc{Plessner, Elsa} (22.\,8.\,1875 Wien – 7.\,5.\,1932 Alicante), \emph{Schriftstellerin}!erste Kapitel. Schauspiel in drei Akten@\strich\emph{Das erste Kapitel. Schauspiel in drei Akten}|pw}{}\ledrightnote{\textcolor{green}{Das erste Kapitel. Schauspiel in drei Akten}}«
               mit unbeschreiblicher Spannung.\pend
           
\pstart
           Seien Sie mir nicht böse über diesen neuerlichen Überfall und üben Sie – wie immer –
               strenges Recht.\pend
           
\pstart
           Mit vorzüglicher Hochachtung{\\[\baselineskip]}\spacefill\mbox{Elsa Plessner.}\pend
           \leftskip=0em{}\selectlanguage{ngerman}\endnumbering\briefempfaengerindex{Schnitzler, Arthur@\textsc{Schnitzler, Arthur}!zzzPlessner, Elsa@\emph{von Elsa Plessner}!1900-01-091@{9. 1. 1900}|)be}\mylabel{L03723h}  \normalsize

\doendnotes{C}
\bigskip
\vfill

\clearpage

\footnotesize

\lohead{\textsc{register}}

% Definiere theindex-Environment komplett neu ohne reledmac
\makeatletter
\renewenvironment{theindex}{%
  \section*{\indexname}%
  \setlength{\parindent}{0pt}%
  \setlength{\parskip}{0pt plus 0.3pt}%
  \let\item\@idxitem
}{%
  \clearpage
}
\makeatother

\IfFileExists{\jobname-pw.ind}{\input{\jobname-pw.ind}}{}

\end{document}

      