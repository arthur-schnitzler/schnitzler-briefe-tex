%% latex-korrekturansicht-vorspann.tex
%% Vorspann für die Korrekturansicht.
%% Lädt die gemeinsame Datei latex-vorspann.tex mit gesetztem Schalter.

\newif\ifkorrekturansicht
\korrekturansichttrue

\input{../tex-inputs/latex-vorspann}


\section[Stefan Zweig an Arthur Schnitzler, {[}zwischen 5. 4. 1915–9. 4. 1915?{]}]{L03653 Stefan Zweig an Arthur Schnitzler, {[}zwischen 5. 4. 1915–9. 4. 1915?{]}}
\nopagebreak\mylabel{L03653v}
\rehead{ }\normalsize\beginnumbering\briefempfaengerindex{, @\textsc{, }!zzz, @\emph{von  }!1915-04-091@{{[}zwischen 5. 4. 1915–9. 4. 1915?{]}}|(be}
\toendnotes[C]{\smallbreak\pagebreak[2]}\Standort{CUL, Schnitzler, B 118.}
\physDesc{Brief, 1 Blatt, 2 Seiten, 1448 Zeichen
\newline{}Handschrift: lila Tinte, lateinische Kurrent
\newline{}Schnitzler: 1) mit Bleistift »\textsc{Zweig}«  2) mit rotem Buntstift eine Unterstreichung}
\buchAbdrucke{\weitereDrucke{Stefan Zweig: \emph{Briefwechsel mit Hermann Bahr, Sigmund Freud, Rainer Maria
                        Rilke und Arthur Schnitzler}. Herausgegeben von Jeffrey B. Berlin,  Hans-Ulrich Lindken und  Donald A. Prater. Frankfurt am Main: \emph{S. Fischer} 1987, S. 394–395.} }\toendnotes[C]{\smallbreak}
\pstart
           {\pb}\textcolor{gray}{\textbf{SZ}}\hfill \textcolor{gray}{\textbf{\textcolor{pink}{VIII. KOCHGASSE}\oindex{Wien@\textbf{Wien}!VIII., Josefstadt@\textbf{VIII., Josefstadt}!Kochgasse 8@\textbf{Kochgasse 8}, \emph{Wohngebäude}|pw}{}\ledrightnote{\textcolor{pink}{Kochgasse 8}}}}\pend
           
\pstart
           \raggedleft{}\textcolor{gray}{\textbf{\textcolor{pink}{WIEN}\oindex{Wien@\textbf{Wien}, \emph{Verwaltungsgebiet}|pw}{}\ledrightnote{\textcolor{pink}{Wien}},}}\pend
           \vspace{0.5em}
\pstart
           Sehr verehrter lieber Herr Doktor, ich wäre sehr froh, wenn Sie
               nächstens einmal mir wieder eine \label{K_L03653-1v}\edtext{Stunde mit Ihnen}{\lemma{\textnormal{\emph{Stunde mit Ihnen}}}\Cendnote{\textnormal{Das Korrespondenzstück
                  ist undatiert, \textcolor{blue}{Schnitzlers} Antwort dürfte
                  dessen Schreiben vom 9. 4. 1915 darstellen, so dass eine zeitliche Begrenzung nach hinten
                  vorliegt. Das gewünschte Treffen wäre folglich jenes am 11. 4. 1915.}}}\label{K_L03653-1}
               verstatten wollten: ich hätte gerne mit Ihnen über die \label{K_L03653-2v}\edtext{Angelegenheit Unseres gemeinsamen Freundes \textcolor{blue}{Rosenbaum}\pwindex{Rosenbaum, Richard 4.\,11.\,1867 Žikov – 25.\,6.\,1942 Konzentrationslager Theresienstadt@\textsc{Rosenbaum, Richard} (4.\,11.\,1867 Žikov – 25.\,6.\,1942 Konzentrationslager Theresienstadt), \emph{Dramaturg, Verleger}|pw}{}\ledrightnote{\textcolor{blue}{Richard Rosenbaum}}}{\lemma{\textnormal{\emph{Angelegenheit … Rosenbaum}}}\Cendnote{\textnormal{\textcolor{blue}{Richard Rosenbaum}\pwindex{Rosenbaum, Richard 4.\,11.\,1867 Žikov – 25.\,6.\,1942 Konzentrationslager Theresienstadt@\textsc{Rosenbaum, Richard} (4.\,11.\,1867 Žikov – 25.\,6.\,1942 Konzentrationslager Theresienstadt), \emph{Dramaturg, Verleger}|pwk} war
                     »literarisch-artistischer Sekretär« des \emph{\textcolor{brown}{Burgtheaters}\orgindex{Burgtheater@Burgtheater|pwk}} und seit Jahren ein zentraler
                  Verantwortungsträger des \emph{\textcolor{brown}{Burgtheaters}\orgindex{Burgtheater@Burgtheater|pwk}}. Bei der
                  Besetzung des Postens des Direktors wurde er wegen seiner jüdischen Abstammung
                  nicht in Betracht gezogen. Die Konflikte mit dem seit 1912 mit der
                  Leitung betreuten \textcolor{blue}{Hugo Thimig}\pwindex{Thimig, Hugo 16.\,6.\,1854 Dresden – 24.\,9.\,1944 Wien@\textsc{Thimig, Hugo} (16.\,6.\,1854 Dresden – 24.\,9.\,1944 Wien), \emph{Theaterleiter, Schauspieler}|pwk} waren seither
                  zunehmends eskaliert, so dass dieser die Entlassung \textcolor{blue}{Rosenbaums}\pwindex{Rosenbaum, Richard 4.\,11.\,1867 Žikov – 25.\,6.\,1942 Konzentrationslager Theresienstadt@\textsc{Rosenbaum, Richard} (4.\,11.\,1867 Žikov – 25.\,6.\,1942 Konzentrationslager Theresienstadt), \emph{Dramaturg, Verleger}|pwk} herbeiführte. \textcolor{blue}{Schnitzler} war seit dem 31. 3. 1915 über das Ultimatum von \textcolor{blue}{Hugo Thimig}\pwindex{Thimig, Hugo 16.\,6.\,1854 Dresden – 24.\,9.\,1944 Wien@\textsc{Thimig, Hugo} (16.\,6.\,1854 Dresden – 24.\,9.\,1944 Wien), \emph{Theaterleiter, Schauspieler}|pwk} informiert, wonach \textcolor{blue}{Rosenbaum}\pwindex{Rosenbaum, Richard 4.\,11.\,1867 Žikov – 25.\,6.\,1942 Konzentrationslager Theresienstadt@\textsc{Rosenbaum, Richard} (4.\,11.\,1867 Žikov – 25.\,6.\,1942 Konzentrationslager Theresienstadt), \emph{Dramaturg, Verleger}|pwk} entweder freiwillig als
                  zurücktreten könne – oder mit seiner Entlassung rechnen müsse.}}}\label{K_L03653-2} gesprochen.
               Immerhin sind Wir – wenn auch machtlos gegen solche Entschliessungen — \substVorne{}\textsuperscript{ein}\substDazwischen{}der\substHinten{}
               wesentlichste Teil der Interessierten und es ist die Frage, ob wir Uns ganz
               unbeteiligt zu einer solchen brutalen Entscheidung stellen sollten. Bis zu einem
               gewissen Grade glaube ich die »\textcolor{brown}{Reichspost}\orgindex{Reichspost@Reichspost|pw}{}\ledrightnote{\textcolor{brown}{Reichspost}}« in
               dieser Sache zu spüren – inwieweit D\textsuperscript{r}{ }\textcolor{blue}{R.}\pwindex{Rosenbaum, Richard 4.\,11.\,1867 Žikov – 25.\,6.\,1942 Konzentrationslager Theresienstadt@\textsc{Rosenbaum, Richard} (4.\,11.\,1867 Žikov – 25.\,6.\,1942 Konzentrationslager Theresienstadt), \emph{Dramaturg, Verleger}|pw}{}\ledrightnote{\textcolor{blue}{Richard Rosenbaum}} im seiner \label{K_L03653-3v}\edtext{Offenheit des Wortes}{\lemma{\textnormal{\emph{Offenheit des Wortes}}}\Cendnote{\textnormal{Vgl. A. S.: \emph{Tagebuch}, 17. 4. 1915.
               }}}\label{K_L03653-3} Etwas verschuldet hat, vermag ich nicht zu entscheiden – und vielleicht wäre
               eine Form der moralischen Satisfaction für diesen vortrefflichen {\pb}Menschen zu finden, der nach siebzehn
               Jahren Tätigkeit \label{K_L03653-4v}\edtext{cum infamia}{\lemma{\textnormal{\emph{cum infamia}}}\Cendnote{\textnormal{lateinisch: mit Schimpf und Schande}}}\label{K_L03653-4}
               weggejagt werden soll. Ich weiss nicht, wie Sie in dieser Sache denken, doch ich
               zweifle nicht, dass sie auch Sie seelisch beschäftigt hat: mir scheint sie nicht
               bloss ein Einzelfall, sondern das Symptom einer Gesinnung, die sich jetzt schon
               mitten im Kriege entfaltet um dann nachher agitatorisch und aggressiv zu werden und
               der man vielleicht heute schon in Parade entgegentreten sollte.\pend
           
\pstart
           D\textsuperscript{r}{ }\textcolor{blue}{R.}\pwindex{Rosenbaum, Richard 4.\,11.\,1867 Žikov – 25.\,6.\,1942 Konzentrationslager Theresienstadt@\textsc{Rosenbaum, Richard} (4.\,11.\,1867 Žikov – 25.\,6.\,1942 Konzentrationslager Theresienstadt), \emph{Dramaturg, Verleger}|pw}{}\ledrightnote{\textcolor{blue}{Richard Rosenbaum}} weiss selbstverständlich nichts von diesem
               Brief. Er tut mir sehr leid: das \textcolor{brown}{Burgtheater}\orgindex{Burgtheater@Burgtheater|pw}{}\ledrightnote{\textcolor{brown}{Burgtheater}} war
               schon so sehr der Sinn seiner Existenz und seines Fühlens geworden, dass er sich kaum
               jemals wird wieder ganz finden können.\pend
           
\pstart
           In herzlicher Liebe und Verehrung Ihr getreuer{\\[\baselineskip]}\spacefill\mbox{Stefan Zweig}\pend
           \leftskip=0em{}\selectlanguage{ngerman}\endnumbering\briefempfaengerindex{, @\textsc{, }!zzz, @\emph{von  }!1915-04-051@{{[}zwischen 5. 4. 1915–9. 4. 1915?{]}}|)be}\mylabel{L03653h}  \normalsize

\doendnotes{C}
\bigskip
\vfill

\clearpage

\footnotesize

\lohead{\textsc{register}}

% Definiere theindex-Environment komplett neu ohne reledmac
\makeatletter
\renewenvironment{theindex}{%
  \section*{\indexname}%
  \setlength{\parindent}{0pt}%
  \setlength{\parskip}{0pt plus 0.3pt}%
  \let\item\@idxitem
}{%
  \clearpage
}
\makeatother

\IfFileExists{\jobname-pw.ind}{\input{\jobname-pw.ind}}{}

\end{document}

      