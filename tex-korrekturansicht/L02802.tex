%% latex-korrekturansicht-vorspann.tex
%% Vorspann für die Korrekturansicht.
%% Lädt die gemeinsame Datei latex-vorspann.tex mit gesetztem Schalter.

\newif\ifkorrekturansicht
\korrekturansichttrue

\input{../tex-inputs/latex-vorspann}


               \section[Paul Goldmann an Arthur Schnitzler, Paul Goldmann an Arthur Schnitzler, 9. 2. {[}1897{]}]{ Paul Goldmann an Arthur Schnitzler, 9. 2. {[}1897{]}}\nopagebreak\mylabel{v}\rehead{ }\normalsize\beginnumbering\briefempfaengerindex{Schnitzler, Arthur@\textsc{Schnitzler, Arthur}!zzzGoldmann, Paul@\emph{von Paul Goldmann}!1897-02-093@{9. 2. {[}1897{]}}|(be} \toendnotes[C]{\smallbreak\pagebreak[2]} \Standort{DLA, A:Schnitzler, HS.NZ85.1.3167.}
\physDesc{Brief, 1 Blatt, 4 Seiten
\newline{}Handschrift: blaue Tinte, deutsche Kurrent\newline{}Beilage: handschriftlicher Brief: 1 Blatt, 2 Seiten 
\newline{}Schnitzler: 1) mit Bleistift das Jahr »97« vermerkt 2) mit rotem Buntstift eine Unterstreichung}\toendnotes[C]{\smallbreak}\pstart
           \noindent{}{\pb}\textcolor{gray}{\textbf{\textbf{\textcolor{brown}{Frankfurter Zeitung}{}\ledrightnote{\textcolor{brown}{Frankfurter Zeitung}}}}}\pend
           \pstart
           \textcolor{gray}{\textbf{(\textcolor{brown}{\begin{otherlanguage}{french}Gazette de Francfort\end{otherlanguage}}{}\ledrightnote{\textcolor{brown}{Frankfurter Zeitung}}).}}\pend
           \pstart
           \textcolor{gray}{\textbf{\textbf{\begin{otherlanguage}{french}Fondateur M.\end{otherlanguage}{ }\textcolor{blue}{L. Sonnemann}{}\ledrightnote{\textcolor{blue}{Leopold Sonnemann}}.}}}\pend
           \pstart
           \begin{otherlanguage}{french}\textcolor{gray}{\textbf{Journal politique, financier,}}\end{otherlanguage}\pend
           \pstart
           \begin{otherlanguage}{french}\textcolor{gray}{\textbf{commercial et littéraire.}}\end{otherlanguage}\pend
           \pstart
           \begin{otherlanguage}{french}\textcolor{gray}{\textbf{\textbf{Paraissant trois fois par jour.}}}\end{otherlanguage}\hfill \textsc{\textcolor{pink}{Paris}{}\ledrightnote{\textcolor{pink}{Paris}}}, 9. Februar.\pend
           \pstart
           \begin{otherlanguage}{french}\textcolor{gray}{\textbf{\textbf{Bureau à \textcolor{pink}{Paris}{}\ledrightnote{\textcolor{pink}{Paris}}}}}\end{otherlanguage}\pend
           \pstart
           \begin{otherlanguage}{french}\textcolor{gray}{\textbf{\textbf{\textcolor{pink}{24. Rue Feydeau}{}\ledrightnote{\textcolor{pink}{rue Feydeau}}.}}}\end{otherlanguage}\pend
           \pstart\center{}Mein lieber Freund,\pend\pstart
           Dein lieber Brief, den ich mit Ungeduld \strikeout{\textcolor{gray}{e}\textcolor{gray}{×}} erwartet habe, hat mich ein wenig erregt und beunruhigt. In einem Augenblick,
               wo ſo wichtige Dinge in Deinem Leben vorgehen, biſt Du gar wortkarg; und Du ahnſt
               nicht, wie ſehr dieſe allgemeinen Andeutungen, die man zu errathen verſuchen muß,
               denjenigen quälen können, der in der Ferne liebevollen Antheil an Dir nimmt und nicht
               weiß was vorgeht. Was gibts eigentlich? Sags doch heraus mit drei klaren Worten!
               Worin liegt vor allen Dingen der »Ernſt« {\pb}der
               Verhältniſſe, von dem Du ſprichſt? Biſt Du bedroht in irgend einer Weise? Du wirſt
               Dich doch nicht etwa mit Jemandem ſchlagen müſſen? Dann ſetze \uline{ich} mich in den Zug und komme nach \textcolor{pink}{Wien}{}\ledrightnote{\textcolor{pink}{Wien}}. Und was ſoll dieſe \label{K_L02802-1v}\edtext{»Flucht«}{\lemma{\textnormal{\emph{»Flucht«}}}\Cendnote{\textnormal{Bereits am 21. 1. 1897 schrieb
                     \textcolor{blue}{Schnitzler} im \emph{\textcolor{green}{Tagebuch}} von einem »Reiseplan«, den er
                     »[g]anz ernstlich mit \textcolor{blue}{Mz. Rh.}« (für eine versteckte Entbindung) erwog. Es wurden verschiedene Orte in
                  Betracht gezogen. Ab Anfang August 1897 (jedenfalls ab
                  dem 6. 8. 1897) war
                     \textcolor{blue}{Marie Reinhard} schließlich im \textcolor{pink}{Wien}er Vorort \textcolor{pink}{Mauer}, wo sie deren \textcolor{blue}{Sohn} tot gebar.}}}\label{K_L02802-1h}? Wohin willſt Du gehen? Komm\textcolor{gray}{,:}
               wenigſtens \label{K_L02802-4v}\edtext{nach \textsc{\textcolor{pink}{Paris}{}\ledrightnote{\textcolor{pink}{Paris}}}}{\lemma{\textnormal{\emph{nach Paris}}}\Cendnote{\textnormal{\textcolor{blue}{Schnitzler} reiste im Frühjahr 1897 gemeinsam mit \textcolor{blue}{Marie
                     Reinhard} nach \textcolor{pink}{Paris}. Am 12. 4. 1897 kamen sie
                  dort an. \textcolor{blue}{Schnitzler} blieb bis zum 24. 5. 1897 und reiste
                  dann weiter nach \textcolor{pink}{London}.}}}\label{K_L02802-4h}, Liebſter, –
               hier kannſt Du in irgend einem Vorort wunderſchön und billig wohnen, ohne daß ein
               Menſch von Deiner Anweſenheit etwas zu ahnen braucht. Und wir ſollen uns im \label{K_L02802-23v}\edtext{Sommer}{\lemma{\textnormal{\emph{Sommer}}}\Cendnote{\textnormal{Zwischen 19. 8. 1897 und 30. 8. 1897 sahen sich \textcolor{blue}{Schnitzler} und \textcolor{blue}{Goldmann} mehrmals
                  in \textcolor{pink}{Bad Ischl} wieder.}}}\label{K_L02802-23h} nicht wiederſehen?
               Ja, liebes Kind, willſt Du denn nach \textcolor{pink}{Auſtralien}{}\ledrightnote{\textcolor{pink}{Australien}}
               gehen? Und Du glaubſt, daß ich {\pb}nach ſolchen
               Vorgängen auf eine Ausſprache mit Dir verzichten werde, nachdem ich Dich bisher in
               jedem gleichgiltigen Sommer anzutreffen geſucht? Wo immer und mit wem immer Du biſt,
               – ich komme hin. Und wenn Du mir dieſes Freundſchafts-Recht verſagen wollteſt, würde
               ich das ſehr bitter empfinden. Und die äußeren Unannehmlichkeiten, von denen Du
               ſprichſt, – kann ich Dir da nicht wenigſtens etwas tragen helfen? Kannſt Du nicht
               irgend etwas auf mich ſchieben? Ich habe einen breiten Rücken.\pend
           \pstart
           {\pb}Den Anlaß zu allen dieſen Vorgängen verſtehe ich
               natürlich; von dem Übrigen habe ich keine Ahnung, da ich die Verhältniſſe nicht
               kenne. Ich bitte dringend um zwei Zeilen Aufklärung.\pend
           \pstart
           Ich ſende Dir anbei einen Brief von \textsc{\textcolor{blue}{Thorel}{}\ledrightnote{\textcolor{blue}{Jean Thorel}}}, den ich auf eine Anfrage bei dieſem bekam.\pend
           \pstart
           Haſt Du noch ein Exemplar von »\textsc{\textcolor{green}{Mourir}{}\ledrightnote{\textcolor{green}{Mourir. Roman}}}«? Bitte, ſende es\strikeout{,}{ }\strikeout{mit} an \textsc{Madame \textcolor{blue}{J. Marnière}{}\ledrightnote{\textcolor{blue}{Jeanne Marni}}}, \textsc{\textcolor{pink}{68. rue Jouffroy, Paris}{}\ledrightnote{\textcolor{pink}{Rue Jouffroy d'Abbans}}}. Schreibe hinein: \label{K_L02802-25v}\edtext{\begin{otherlanguage}{french}\textsc{À Madame \textcolor{blue}{J. \strikeout{M\textcolor{gray}{ar}}}{}\ledrightnote{→\textcolor{blue}{Jeanne Marni}}{ }\textcolor{blue}{Marni}{}\ledrightnote{\textcolor{blue}{Jeanne Marni}}, hommage respectueux}\end{otherlanguage}}{\lemma{\textnormal{\emph{À … respectueux}}}\Cendnote{\textnormal{An Frau \textcolor{blue}{J. Marni}, respektvolle Anerkennung}}}\label{K_L02802-25h}, und Deinen
               Namen. Es iſt eine geiſtvolle und liebenswürdige \textsc{\label{K_L02802-56v}\edtext{\begin{otherlanguage}{french}\textcolor{blue}{femme de lettres}{}\ledrightnote{→\textcolor{blue}{Jeanne Marni}}\end{otherlanguage}}{\lemma{\textnormal{\emph{femme de lettres}}}\Cendnote{\textnormal{französisch: Literatin}}}\label{K_L02802-56h}{ } (\label{K_L02802-67v}\edtext{\textcolor{blue}{E. Voilà}{}\ledrightnote{\textcolor{blue}{Jeanne Marni}}}{\lemma{\textnormal{\emph{E. Voilà}}}\Cendnote{\textnormal{Pseudonym}}}\label{K_L02802-67h}} der »\textsc{\textcolor{brown}{Vie Parisienne}{}\ledrightnote{\textcolor{brown}{La Vie Parisienne}}}«), der ich von Dir geſprochen habe.\pend
           \pstart
           Tauſend Grüße! Dein {\\[\baselineskip]}\spacefill\mbox{Paul Goldm}\pend
           \leftskip=0em{}{\bigskip}\pstart
           \raggedleft{}{\pb}{[}hs. Thorel:{]} \textcolor{pink}{12 rue de Milan}{}\ledrightnote{\textcolor{pink}{Rue de Milan}}\pend
           \pstart\center{}Cher monsieur Goldmann.\pend\pstart
           \label{K_L02802-33v}\edtext{Non, sur de nouveau. Il fallait
               laiſser \textcolor{blue}{Carré}{}\ledrightnote{\textcolor{blue}{Albert Carré}} quelques semains. Je les lui ai
               laiſsé. Maintenant, je vais le relancer aſsez souvent. J'ai commencé vendredi
                  dernier. Et je continuerai, en rapprochant de plus emplis les distance. Il
               faut traquer les directeurs de théâtre comme on traque les cerfs à la chasse.}{\lemma{\textnormal{\emph{Non, … chasse.}}}\Cendnote{\textnormal{französisch: Lieber Herr \textcolor{blue}{Goldmann}, Nein, von neuem. Es war nötig, \textcolor{blue}{Carré} ein paar Wochen Zeit zu geben. Ich habe sie ihm
                  gegeben. Jetzt werde ich ihn recht oft noch einmal ansprechen. Ich habe
                     letzten Freitag damit angefangen. Und ich werde weitermachen,
                  indem ich die Abstände immer kleiner lassen werde. Man muss Theaterdirektoren
                  aufspüren, wie man Hirsche auf der Jagd aufspürt. Bitte weisen Sie \textcolor{blue}{Schnitzler} auf \textcolor{blue}{Wyzewa}s \textcolor{green}{Artikel} in \emph{\textcolor{green}{Le Temps}} vom 27. Januar hin{[}.{]} Ich hatte \textcolor{blue}{Wyzewa} gesagt, dass ich \textcolor{blue}{Schnitzler}{ }\textcolor{green}{übersetzte}, und so
                  versuchte er, mir mit ein paar äußerst schmeichelhaften \textcolor{green}{Zeilen}, die er \textcolor{blue}{Schnitzler} widmete, einen Gefallen zu tun. Ich werde Sie
                  auf dem Laufenden halten. Ihr ergebener Freund \textcolor{blue}{Jean Thorel}}}}\label{K_L02802-33h}\pend
           \pstart
           Signalez, donc à Schnitzler, \label{K_L02802-28v}\edtext{\textcolor{green}{l'article}{}\ledrightnote{→\textcolor{green}{Un vaudevilliste viennois}} de \textcolor{blue}{Wyzewa}{}\ledrightnote{\textcolor{blue}{Théodore de Wyzewa}}}{\lemma{\textnormal{\emph{l'article de Wyzewa}}}\Cendnote{\textnormal{\textcolor{blue}{Théodore de Wyzewa}: \emph{\textcolor{green}{Un vaudevilliste viennois}}. In: \emph{\textcolor{green}{Le Temps}}, Jg. 37, Nr. 13023, 27. 1. 1897, S. 2.}}}\label{K_L02802-28h} dans le \uline{\textcolor{green}{Temps}{}\ledrightnote{\textcolor{green}{Le Temps}}} du 27 janvier{[}.{]} J’avais dit à \textcolor{blue}{Wyzewa}{}\ledrightnote{\textcolor{blue}{Théodore de Wyzewa}}
               que je \textcolor{green}{traduisais}{}\ledrightnote{→\textcolor{green}{Amourette. Pièce en trois actes. Adaptée de Arthur Schnitzler}} de
               Schnitzler, et il a ainsi cherché {\pb}à me rendre service
               par les quelques \textcolor{green}{lignes}{}\ledrightnote{→\textcolor{green}{Un vaudevilliste viennois}}
               entrêmement flatteurs qu’il a consacrés à Schnitzler –\pend
           \pstart
           Je vous tiendrai au courant.\pend
           \pstart
           Votre bien devoué {\\[\baselineskip]}\spacefill\mbox{\textcolor{blue}{Jean Thorel}{}\ledrightnote{\textcolor{blue}{Jean Thorel}}}\pend
           \leftskip=0em{}\endnumbering\briefempfaengerindex{Schnitzler, Arthur@\textsc{Schnitzler, Arthur}!zzzGoldmann, Paul@\emph{von Paul Goldmann}!1897-02-093@{9. 2. {[}1897{]}}|)be}\mylabel{h}\begin{anhang}\end{anhang}\normalsize

\doendnotes{C}
\bigskip
\vfill

\clearpage

\footnotesize

\lohead{\textsc{register}}

% Definiere theindex-Environment komplett neu ohne reledmac
\makeatletter
\renewenvironment{theindex}{%
  \section*{\indexname}%
  \setlength{\parindent}{0pt}%
  \setlength{\parskip}{0pt plus 0.3pt}%
  \let\item\@idxitem
}{%
  \clearpage
}
\makeatother

\IfFileExists{\jobname-pw.ind}{\input{\jobname-pw.ind}}{}

\end{document}

      