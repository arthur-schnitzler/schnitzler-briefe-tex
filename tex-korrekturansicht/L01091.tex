%% latex-korrekturansicht-vorspann.tex
%% Vorspann für die Korrekturansicht.
%% Lädt die gemeinsame Datei latex-vorspann.tex mit gesetztem Schalter.

\newif\ifkorrekturansicht
\korrekturansichttrue

\input{../tex-inputs/latex-vorspann}


               \section[Hugo von Hofmannsthal an Arthur Schnitzler, 1{[}6?{]} 1. 1901]{ Hugo von Hofmannsthal an Arthur Schnitzler, 1{[}6?{]} 1. 1901}\nopagebreak\mylabel{v}\rehead{ }\normalsize\beginnumbering\briefempfaengerindex{Schnitzler, Arthur@\textsc{Schnitzler, Arthur}!zzzHofmannsthal, Hugo von@\emph{von Hugo von Hofmannsthal}!1901-01-161@{1{[}6?{]} 1. 1901}|(be} \toendnotes[C]{\smallbreak\pagebreak[2]} \Standort{CUL, Schnitzler, B 43.}
\physDesc{Brief, 1 Blatt, 3 Seiten
\newline{}Handschrift: schwarze Tinte, deutsche Kurrent
\newline{}Schnitzler: mit schwarzer Tinte datiert: »Januar 901« \newline{}Ordnung: mit Bleistift von unbekannter Hand nummeriert:
                                    »171« und frühere Nummerierungen unkenntlich
                                 gemacht }\buchAbdrucke{\weitereDrucke{Hugo von Hofmannsthal, Arthur Schnitzler: \emph{Briefwechsel}. Hg. Therese Nickl und Heinrich Schnitzler. Frankfurt am Main: \emph{S. Fischer} 1964, S. 145–146.} }\toendnotes[C]{\smallbreak}\pstart{}{\pb}lieber, \pend\pstart
           hier iſt das Bild für die Schauſpielerinnen. Habe aus Neugierde den \label{K_L01091_1v}\edtext{erſten Theil}{\lemma{\textnormal{\emph{erſten Theil}}}\Cendnote{\textnormal{Die Datierung dieses Korrespondenzstücks gelingt durch implizite
                  Faktoren: Die \emph{\textcolor{green}{Neue Deutsche Rundschau}} erschien
                  üblicherweise zur Monatsmitte, was die früheste Möglichkeit der Lektüre von \emph{\textcolor{green}{Frau Bertha Garlan}} ergibt. Nachdem der Brief vom
                     17. 1. 1901 bereits auf die
                  stattgefundene Lektüre verweist, ist dieser davor anzusetzen.}}}\label{K_L01091_1h} von »\textcolor{green}{Frau Bertha \textsc{Garlan}}{}\ledrightnote{\textcolor{green}{Frau Bertha Garlan. Roman}}« geleſen und finde es wunderſchön, ſo reif, reich und leicht, voll Ruhe und
               Fülle, in zarten Farben, voll Luft, \uuline{ſehr}{ }ſchön. {\pb}Trotzdem bleibt der Schluſs des
                  »\textcolor{green}{blinden Geronimo}{}\ledrightnote{\textcolor{green}{Der blinde Geronimo und sein Bruder}}« in der gegenwärtigen Form
               mangelhaft, enttäuſchend. Es muſs aber ſehr leicht zu ändern ſein. Aber ich irre mich
               nicht, denn ich habs wieder \substVorne{}\textsuperscript{geſehen}{\allowbreak}\substDazwischen{}geleſen\substHinten{}.\pend
           \pstart
           Ich hätte eine große Bitte: Daſs am \label{K_L01091_2v}\edtext{Sonntag}{\lemma{\textnormal{\emph{Sonntag}}}\Cendnote{\textnormal{vgl. A. S.: \emph{Tagebuch}, 20. 1. 1901}}}\label{K_L01091_2h} mit dem Leſen ſchon um ½ 5 begonnen {\pb}wird. Ich freue mich ſeit langem
               mit der \textcolor{blue}{Gerty}{}\ledrightnote{\textcolor{blue}{Gertrude von Hofmannsthal}}, die nie ein \textcolor{green}{Stück}{}\ledrightnote{→\textcolor{green}{Henry IV, Part 1}} von \textcolor{blue}{\textsc{Shakespeare}}{}\ledrightnote{\textcolor{blue}{William Shakespeare}} geſehen hat, in eines zu gehen und ſo haben wir für Sonntag eine
               Loge für \textcolor{green}{\textsc{Heinrich IV.}}{}\ledrightnote{\textcolor{green}{Henry IV, Part 1}} beſtellt.\pend
           \pstart
           Ich hoffe, es läſst ſich durchführen und werde {\pb}pünktlich ½ 5 bei
               Ihnen ſein.\pend
           \pstart
           Herzlich{\\}\spacefill\mbox{Hugo.}\pend
           \endnumbering\briefempfaengerindex{Schnitzler, Arthur@\textsc{Schnitzler, Arthur}!zzzHofmannsthal, Hugo von@\emph{von Hugo von Hofmannsthal}!1901-01-161@{1{[}6?{]} 1. 1901}|)be}\mylabel{h}  \normalsize

\doendnotes{C}
\bigskip
\vfill

\clearpage

\footnotesize

\lohead{\textsc{register}}

% Definiere theindex-Environment komplett neu ohne reledmac
\makeatletter
\renewenvironment{theindex}{%
  \section*{\indexname}%
  \setlength{\parindent}{0pt}%
  \setlength{\parskip}{0pt plus 0.3pt}%
  \let\item\@idxitem
}{%
  \clearpage
}
\makeatother

\IfFileExists{\jobname-pw.ind}{\input{\jobname-pw.ind}}{}

\end{document}

      