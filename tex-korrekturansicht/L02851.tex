%% latex-korrekturansicht-vorspann.tex
%% Vorspann für die Korrekturansicht.
%% Lädt die gemeinsame Datei latex-vorspann.tex mit gesetztem Schalter.

\newif\ifkorrekturansicht
\korrekturansichttrue

\input{../tex-inputs/latex-vorspann}


               \section[ Paul Goldmann an Arthur Schnitzler, 27. 4. 1898]{Paul Goldmann an Arthur Schnitzler, 27. 4. 1898}\nopagebreak\mylabel{v}\rehead{ }\normalsize\beginnumbering\briefempfaengerindex{Schnitzler, Arthur@\textsc{Schnitzler, Arthur}!zzzGoldmann, Paul@\emph{von Paul Goldmann}!1898-04-271@{27. 4. 1898}|(be} \toendnotes[C]{\smallbreak\pagebreak[2]} \Standort{DLA, A:Schnitzler, HS.NZ85.1.3168.}
\physDesc{Postkarte
\newline{}Handschrift: schwarze Tinte, lateinische Kurrent\newline{}Versand: 1) Stempel: »\nobreak{}\oindex{Singapur@\textbf{Singapur}, \emph{Land (A.LND)}|pwk}Singapore, Ap 27 98\nobreak{}«.  2) Stempel: »\nobreak{}Wien 9/3 72, 21. 5. {[}9{]}8, 8. V, Bestellt\nobreak{}«. 
\newline{}Schnitzler: mit Bleistift das Jahr »98« und das Datum »27/4 98« vermerkt }\toendnotes[C]{\smallbreak}\pstart{}{\pb}\textsc{\textcolor{pink}{Austria}{}\ledrightnote{\textcolor{pink}{Österreich}}}.\pend{}\pstart{}\textsc{Herrn Dr. Arthur Schnitzler}\pend{}\pstart{}\textsc{\textcolor{pink}{Wien}{}\ledrightnote{\textcolor{pink}{Wien}}}\pend{}\pstart{}\textsc{\textcolor{pink}{IX. Frankgaße 1}{}\ledrightnote{\textcolor{pink}{Frankgasse}}}.\pend{}{\bigskip}\pstart
           {\pb}\textsc{\textcolor{pink}{Singapore}{}\ledrightnote{\textcolor{pink}{Singapur}}}, 27. \textsc{April}.\pend
           \pstart
           Hier brennt die Tropenſonne erbarmungsloſer, als je, und ich bin faſt verrückt vor
               Hitze. \textcolor{pink}{Aſien}{}\ledrightnote{\textcolor{pink}{Asien}} iſt eine ſeltſame Welt, aber wie
               ſchön iſt es zu Haue, in einem grünen \textcolor{pink}{Wiener-Wald}{}\ledrightnote{\textcolor{pink}{Wienerwald}}-Thale. Und alle Palmen von \textsc{\textcolor{pink}{Singapore}{}\ledrightnote{\textcolor{pink}{Singapur}}} gäbe ich um \label{T_L02851-1v}\edtext{eine{[}n{]}}{\lemma{\textnormal{\emph{einen}}}\Cendnote{\textnormal{\textcolor{blue}{Goldmann} schrieb
                  »einer«.}}}\label{T_L02851-1h} einzigen lieben Menſchen von daheim. Viele treue
               Grüße Dir u. \textsc{\textcolor{blue}{Richard}{}\ledrightnote{\textcolor{blue}{Richard Beer-Hofmann}}}.\pend
           \pstart Dein \spacefill\mbox{P. G.}\pend{}\endnumbering\briefempfaengerindex{Schnitzler, Arthur@\textsc{Schnitzler, Arthur}!zzzGoldmann, Paul@\emph{von Paul Goldmann}!1898-04-271@{27. 4. 1898}|)be}\mylabel{h}\begin{anhang}\end{anhang}\normalsize

\doendnotes{C}
\bigskip
\vfill

\clearpage

\footnotesize

\lohead{\textsc{register}}

% Definiere theindex-Environment komplett neu ohne reledmac
\makeatletter
\renewenvironment{theindex}{%
  \section*{\indexname}%
  \setlength{\parindent}{0pt}%
  \setlength{\parskip}{0pt plus 0.3pt}%
  \let\item\@idxitem
}{%
  \clearpage
}
\makeatother

\IfFileExists{\jobname-pw.ind}{\input{\jobname-pw.ind}}{}

\end{document}

      