%% latex-korrekturansicht-vorspann.tex
%% Vorspann für die Korrekturansicht.
%% Lädt die gemeinsame Datei latex-vorspann.tex mit gesetztem Schalter.

\newif\ifkorrekturansicht
\korrekturansichttrue

\input{../tex-inputs/latex-vorspann}


               \section[Georg Brandes an Arthur Schnitzler, 5. 1. 1922]{ Georg Brandes an Arthur Schnitzler, 5. 1. 1922}\nopagebreak\mylabel{v}\rehead{ }\normalsize\beginnumbering\briefempfaengerindex{Schnitzler, Arthur@\textsc{Schnitzler, Arthur}!zzzBrandes, Georg@\emph{von Georg Brandes}!1922-01-051@{5. 1. 1922}|(be} \toendnotes[C]{\smallbreak\pagebreak[2]} \Standort{CUL, Schnitzler, B 17.}
\physDesc{Brief, 1 Blatt, 4 Seiten
\newline{}Handschrift: schwarze Tinte, lateinische Kurrent
\newline{}Schnitzler: mit rotem Buntstift vereinzelte Unterstreichungen \newline{}Ordnung: mit Bleistift von unbekannter Hand nummeriert:
                                    »52« }\buchAbdrucke{\weitereDrucke{Georg Brandes, Arthur Schnitzler: \emph{Ein Briefwechsel}. Hg. Kurt Bergel. Bern: \emph{Francke} 1956, S. 132–133.} }\toendnotes[C]{\smallbreak}\pstart
           \raggedleft{}{\pb}\textcolor{pink}{Kopenhagen}{}\ledrightnote{\textcolor{pink}{Kopenhagen}}{ }5 Januar 22\pend
           \pstart{}Verehrter lieber Freund\pend\pstart
           Es war mir eine Freude, von Ihnen zu hören, eine noch grössere, dass Sie jenes schon
               alte \textcolor{green}{Buch}{}\ledrightnote{→\textcolor{green}{Wolfgang Goethe}}, das ich seit
                  1915 nie wieder angesehen habe, mit Befriedigung gelesen. Welcher
               Fluch für mich, eine Sprache zu schreiben, die Niemand versteht. Ich möchte Ihnen so
               gern die späteren Bücher, \textcolor{green}{\textcolor{blue}{Voltaire}{}\ledrightnote{\textcolor{blue}{Voltaire}}}{}\ledrightnote{→\textcolor{green}{Voltaire und sein Jahrhundert}}, \textcolor{green}{\textcolor{blue}{Cäsar}{}\ledrightnote{\textcolor{blue}{Gaius Iulius Caesar}}}{}\ledrightnote{→\textcolor{green}{Gaius Julius Cæsar}}, \textcolor{green}{\textcolor{blue}{Michelangelo}{}\ledrightnote{\textcolor{blue}{Buonarroti Michelangelo}}}{}\ledrightnote{→\textcolor{green}{Michelangelo Buonarotti}} zugeschickt haben. Auch was ich in der letzten Zeit \label{K_L02373_1v}\edtext{über \textcolor{green}{\textcolor{blue}{\uline{Homer}}{}\ledrightnote{\textcolor{blue}{Homer}}}{}\ledrightnote{→\textcolor{green}{Homer}} geschrieben}{\lemma{\textnormal{\emph{über Homer geschrieben}}}\Cendnote{\textnormal{Die Rede, die er bei
                  der in Folge erwähnten großen Feier an der \textcolor{brown}{Universität} am 3. 11. 1921 über \textcolor{blue}{Homer} hielt, erschien zuerst als eigener Druck (\emph{\textcolor{green}{Homer}}. Hg. \emph{\textcolor{brown}{Studentersamfundet}} (in Kommission bei
                        \emph{Gyldendalske}) 1921) und wurde in
                  Folge in sein Buch \emph{\textcolor{green}{Hellas}} (Kopenhagen:
                        \emph{Gyldendal, Nordisk forlag}{ }1925) integriert. Auf deutsch erschien der Aufsatz über \textcolor{blue}{Homer}{ }1927 im \emph{\textcolor{brown}{Philipp
                  Reclam}}-Verlag.}}}\label{K_L02373_1h}.\pend
           \pstart
           Ich weiss nicht, ob Ihre Zeitungen davon gesprochen, dass (weil es am
                  3. November 50 Jahre her war, dass ich meine ersten Vorträge an der
                  \textcolor{brown}{Kopenhagener Universität}{}\ledrightnote{\textcolor{brown}{Københavns Universitet}} hielt) hier grosse Feier
               waren, Fackelzug der Studenten {\pb}und anderes. Es würde mich vor 40 Jahren sehr erfreut haben.\pend
           \pstart
           Am 15. Januar soll ich vor der Aufführung von \textcolor{green}{Tartufe}{}\ledrightnote{\textcolor{green}{Tartuffe}} von der Bühne des \textcolor{pink}{Dagmar-Teaters}{}\ledrightnote{\textcolor{pink}{Dagmar Teatret}} über \textcolor{blue}{Molière}{}\ledrightnote{\textcolor{blue}{Molière}} reden. Am
                  19 wieder an die \textcolor{pink}{russischen}{}\ledrightnote{\textcolor{pink}{Russland}}
               Schauspieler \textcolor{pink}{französisch}{}\ledrightnote{\textcolor{pink}{Frankreich}} reden.\pend
           \pstart
           Dann verschwinde ich Ende dieses Monats für einige Zeit. Ich will mich wahrlich nicht
               zu meinem 80 Geburtstag Glück wünschen lassen. Die Lächerlichkeit wäre zu gross.\pend
           \pstart
           Ich las hier einmal im Herbst in einer Zeitung ein
               \textcolor{green}{Interview}{}\ledrightnote{→\textcolor{green}{?? [nicht ermitteltes dänisches Interview]}} eines mir
               unbekannten { }\textcolor{blue}{dänischen Journalisten}{}\ledrightnote{→\textcolor{blue}{?? [dänischer Journalist, der Schnitzler interviewt]}} mit
               Ihnen, worin Sie sehr \label{K_L02373-1v}\edtext{freundliche
               Worte}{\lemma{\textnormal{\emph{freundliche
               Worte}}}\Cendnote{\textnormal{nicht nachgewiesen}}}\label{K_L02373-1h} über mich
               sagten, ich glaube die freundlichsten, die in jenem Blatte je über mich gestanden
               haben.\pend
           \pstart
           Ich bleibe Ihnen immer verpflichtet und {\pb}verbunden. Der Genuss, den ich
               durch das Lesen Ihrer Werke gehabt habe, ist hundert Mal grösser als das mögliche
               Vergnügen, das Sie durch meine nur belehrenden Bücher gehabt haben können.\pend
           \pstart
           Ich sah durch dies \textcolor{green}{Interview}{}\ledrightnote{→\textcolor{green}{Arthur Schnitzler. En Samtale med en beremt Wiener [1921]}}, wie
               viel Unannehmlichkeiten Sie durch das alte, nur scherzhafte und witzige, \textcolor{green}{\uline{Reigen}}{}\ledrightnote{\textcolor{green}{Reigen. Zehn Dialoge}} gehabt haben. Der jetzt überall glühende Antisemitismus und die Tugendbolderei
                  \introOben{}geben\introOben{} im Verein \strikeout{geben}
               solche Resultate. Als ob die Menschen durch die Umstände dieser Zeit nicht genug
               litten, gehen sie mit zehnfachem Eifer darauf los, sich gegenseitig das Leben noch
               saurer zu machen.\pend
           \pstart
           Ich habe immer \textcolor{pink}{Wien}{}\ledrightnote{\textcolor{pink}{Wien}} in meinen Gedanken, immer mit
               Mitleid, Trauer und Dankbarkeit. Können Sie verstehen, das unser Freund \textcolor{blue}{Beer-Hofmann}{}\ledrightnote{\textcolor{blue}{Richard Beer-Hofmann}} sich {\pb}mit solcher Leidenschaft an das
               Judenthum krampft. Es hat mich im Grunde nie interessiert; nur wenn die Juden
               verfolgt wurden, und wenn sie es werden, habe ich für sie heisses Mitgefühl, wie für
               alle ungerecht unterdrückten. Ich kenne nicht einen einzigen hebräischen
               Buchstaben. – Es scheint mir auch von ihm so \uline{gewollt}.\pend
           \pstart
           Ich denke mir, Sie haben sich in den späteren Jahren mit \textcolor{blue}{Casanova}{}\ledrightnote{\textcolor{blue}{Giacomo Girolamo Casanova}} beschäftigt, am meisten um sich nicht mit dem
               Gegenwärtigen herumzuschlagen. Wenn der Einzelne seine Ohnmacht fühlt, nützt es ja
               nichts mitzureden. Deshalb schweige ich selbst, wo ich viel zu sagen hätte. Ich habe
               nicht \textcolor{blue}{Frithiof Nansens}{}\ledrightnote{\textcolor{blue}{Fridtjof Nansen}} praktische Begabung so
               wenig wie sein Ansehen. Er ist durch den Krieg sehr gewachsen.\pend
           \pstart
           Ich bitte Sie Ihrer Frau \textcolor{blue}{Gemahlin}{}\ledrightnote{→\textcolor{blue}{Olga Schnitzler}} meine Huldigung, Ihren \textcolor{blue}{Kindern}{}\ledrightnote{→\textcolor{blue}{Heinrich Schnitzler}{\newline}→\textcolor{blue}{Lili Schnitzler}} meine Sympathie zu überbringen.\pend
           \pstart
           Ihr Freund{\\[\baselineskip]}\spacefill\mbox{Georg Brandes}\pend
           \leftskip=0em{}\endnumbering\briefempfaengerindex{Schnitzler, Arthur@\textsc{Schnitzler, Arthur}!zzzBrandes, Georg@\emph{von Georg Brandes}!1922-01-051@{5. 1. 1922}|)be}\mylabel{h}  \normalsize

\doendnotes{C}
\bigskip
\vfill

\clearpage

\footnotesize

\lohead{\textsc{register}}

% Definiere theindex-Environment komplett neu ohne reledmac
\makeatletter
\renewenvironment{theindex}{%
  \section*{\indexname}%
  \setlength{\parindent}{0pt}%
  \setlength{\parskip}{0pt plus 0.3pt}%
  \let\item\@idxitem
}{%
  \clearpage
}
\makeatother

\IfFileExists{\jobname-pw.ind}{\input{\jobname-pw.ind}}{}

\end{document}

      