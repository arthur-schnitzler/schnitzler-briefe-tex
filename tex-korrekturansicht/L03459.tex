%% latex-korrekturansicht-vorspann.tex
%% Vorspann für die Korrekturansicht.
%% Lädt die gemeinsame Datei latex-vorspann.tex mit gesetztem Schalter.

\newif\ifkorrekturansicht
\korrekturansichttrue

\input{../tex-inputs/latex-vorspann}


\renewcommand{\erwaehnteOrte}{Orte: Edmund-Weiß-Gasse, Gossensass, Hühnerspiel, Hühnerspielalm, Wien}
\renewcommand{\erwaehnteWerke}{}
\section[Paul Goldmann, Markus Hajek, Gisela Hajek und Margot Vallo an Arthur Schnitzler, {[}29. 8. 1907?{]}]{Paul Goldmann, Markus Hajek, Gisela Hajek und Margot Vallo an Arthur
               Schnitzler, {[}29. 8. 1907?{]}}
\nopagebreak\mylabel{v}
\rehead{ }\normalsize\beginnumbering\briefempfaengerindex{Schnitzler, Arthur@\textsc{Schnitzler, Arthur}!zzzVallo, Margot@\emph{von Margot Vallo}!1907-08-292@{{[}29. 8. 1907?{]}}|(be}\briefempfaengerindex{Schnitzler, Arthur@\textsc{Schnitzler, Arthur}!zzzHajek, Gisela@\emph{von Gisela Hajek}!1907-08-292@{{[}29. 8. 1907?{]}}|(be}\briefempfaengerindex{Schnitzler, Arthur@\textsc{Schnitzler, Arthur}!zzzHajek, Markus@\emph{von Markus Hajek}!1907-08-292@{{[}29. 8. 1907?{]}}|(be}\briefempfaengerindex{Schnitzler, Arthur@\textsc{Schnitzler, Arthur}!zzzGoldmann, Paul@\emph{von Paul Goldmann}!1907-08-292@{{[}29. 8. 1907?{]}}|(be}
\toendnotes[C]{\smallbreak\pagebreak[2]}\Standort{DLA, A:Schnitzler, HS.NZ85.1.3175.}
\physDesc{Bildpostkarte, 245 Zeichen
\newline{}Handschrift Paul Goldmann: 1) Bleistift, deutsche Kurrent\hspace{1em}2) Bleistift, lateinische Kurrent (\noindent{}Adresse)\hspace{1em}
\newline{}Handschrift Markus Hajek: Bleistift, lateinische Kurrent
\newline{}Handschrift Gisela Hajek: Bleistift, deutsche Kurrent
\newline{}Handschrift Margot Vallo: Bleistift, deutsche Kurrent
\newline{}Versand: Stempel: »\nobreak{}\oindex{Gossensass@\textbf{Gossensass}, \emph{P.PPLA3}|pwk}Goss{[}ensa{]}ss, 2\textcolor{gray}{9} 8 \textcolor{gray}{07}\nobreak{}«.  }\pstart{}{\pb}Herrn\pend{}\pstart{}Dr. Arthur Schnitzler\pend{}\pstart{}\textcolor{pink}{Wien}{}\ledrightnote{\textcolor{pink}{Wien}}\pend{}\pstart{}\textcolor{pink}{XVIII. Spöttelgaſse 7}{}\ledrightnote{\textcolor{pink}{Edmund-Weiß-Gasse}}.\pend{}
{\bigskip}
\pstart
           \noindent{}{\pb}\textcolor{gray}{\textbf{Gruss von der \textcolor{pink}{Hühnerspielalm}{}\ledrightnote{\textcolor{pink}{Hühnerspielalm}} (\textcolor{pink}{Amthorspitze}{}\ledrightnote{\textcolor{pink}{Hühnerspiel}}) 2746 m}}\pend
           
\pstart
           {\pb}Wir denken Deiner in einer unwahrſcheinlichen Höhe
               über dem Meer u. ſenden Dir herzliche, teils verwandtſchaftliche, teils
               freundſchaftliche Grüße. \spacefill\mbox{Paul Goldmann.}\pend
           
\pstart
           \spacefill\mbox{{[}hs. Markus Hajek:{]} Dr. M. Hajek}\pend
           
\pstart
           {[}hs. Gisela Hajek:{]} Herzlichſt {\\}\spacefill\mbox{Giſa}\pend
           
\pstart
           \spacefill\mbox{{[}hs. Vallo:{]} Margot}\pend
           \endnumbering\briefempfaengerindex{Schnitzler, Arthur@\textsc{Schnitzler, Arthur}!zzzVallo, Margot@\emph{von Margot Vallo}!1907-08-292@{{[}29. 8. 1907?{]}}|)be}\briefempfaengerindex{Schnitzler, Arthur@\textsc{Schnitzler, Arthur}!zzzHajek, Gisela@\emph{von Gisela Hajek}!1907-08-292@{{[}29. 8. 1907?{]}}|)be}\briefempfaengerindex{Schnitzler, Arthur@\textsc{Schnitzler, Arthur}!zzzHajek, Markus@\emph{von Markus Hajek}!1907-08-292@{{[}29. 8. 1907?{]}}|)be}\briefempfaengerindex{Schnitzler, Arthur@\textsc{Schnitzler, Arthur}!zzzGoldmann, Paul@\emph{von Paul Goldmann}!1907-08-292@{{[}29. 8. 1907?{]}}|)be}\mylabel{h}  \normalsize

\doendnotes{C}
\bigskip
\vfill

\clearpage

\footnotesize

\lohead{\textsc{register}}

% Definiere theindex-Environment komplett neu ohne reledmac
\makeatletter
\renewenvironment{theindex}{%
  \section*{\indexname}%
  \setlength{\parindent}{0pt}%
  \setlength{\parskip}{0pt plus 0.3pt}%
  \let\item\@idxitem
}{%
  \clearpage
}
\makeatother

\IfFileExists{\jobname-pw.ind}{\input{\jobname-pw.ind}}{}

\end{document}

      