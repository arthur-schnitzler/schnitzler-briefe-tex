%% latex-korrekturansicht-vorspann.tex
%% Vorspann für die Korrekturansicht.
%% Lädt die gemeinsame Datei latex-vorspann.tex mit gesetztem Schalter.

\newif\ifkorrekturansicht
\korrekturansichttrue

\input{../tex-inputs/latex-vorspann}


\renewcommand{\erwaehntePersonen}{Personen: Giuseppe Giacosa, Ludmilla Karplus, Siegmund Karplus}
\renewcommand{\erwaehnteOrte}{Orte: Burgtheater, Ronacher, Wien}
\renewcommand{\erwaehnteWerke}{Werke: Liebelei. Schauspiel in drei Akten, Rechte der Seele. Schauspiel in einem Act}
\section[ Felix Salten an Arthur Schnitzler, {[}16. 11. 1895{]}]{Felix Salten an Arthur Schnitzler, {[}16. 11. 1895{]}}
\nopagebreak\mylabel{v}
\rehead{ }\normalsize\beginnumbering\briefempfaengerindex{Schnitzler, Arthur@\textsc{Schnitzler, Arthur}!zzzSalten, Felix@\emph{von Felix Salten}!1895-11-162@{{[}16. 11. 1895{]}}|(be}
\toendnotes[C]{\smallbreak\pagebreak[2]}\Standort{CUL, Schnitzler, B 89, A 1.}
\physDesc{Brief, 1 Blatt, 1 Seite, 149 Zeichen
\newline{}Handschrift: Bleistift, lateinische Kurrent
\newline{}Schnitzler: mit Bleistift datiert: »16/11 95« 
\newline{}Ordnung: mit Bleistift von unbekannter Hand nummeriert: »66« }\toendnotes[C]{\smallbreak}
\pstart
           \noindent{}{\pb}Ich will Ihnen nur sagen:\pend
           \settowidth{\longeste}{Sonntag, denx24.}\settowidth{\longestz}{»Rechte der Seele«}\settowidth{\longestd}{}\settowidth{\longestv}{}\settowidth{\longestf}{}\addtolength\longeste{1em}
        \addtolength\longestz{1em}
      \pstart\noindent\makebox[\the\longeste][l]{\label{K_L03166-1v}\edtext{Sonntag, den 24.}{\lemma{\textnormal{\emph{Sonntag, den 24.}}}\Cendnote{\textnormal{Seit dem 9. 10. 1895 wurden \textcolor{blue}{Giuseppe
                           Giacosa}s \emph{\textcolor{green}{Rechte der Seele}} und
                           \textcolor{blue}{Schnitzler}s \emph{\textcolor{green}{Liebelei}} am \textcolor{pink}{Burgtheater} gemeinsam gespielt. Am 24. 11. 1895 wurde die \emph{\textcolor{green}{Liebelei}} zum elften Mal gegeben.}}}\label{K_L03166-1h}}\makebox[\the\longestz][l]{»\textcolor{green}{Rechte der Seele}{}\ledrightnote{\textcolor{green}{Rechte der Seele. Schauspiel in einem Act}}«}
                  \pend\pstart\noindent\makebox[\the\longeste][l]{}\makebox[\the\longestz][l]{»\textcolor{green}{Liebelei}{}\ledrightnote{\textcolor{green}{Liebelei. Schauspiel in drei Akten}}« –}
                  \pend
\pstart
           Über so was kann ich mich \label{T_L03166-1v}\edtext{riesig}{\lemma{\textnormal{\emph{riesig}}}\Cendnote{\textnormal{»riesig« dürfte absichtlich mit größerer
            Schrift geschrieben sein}}}\label{T_L03166-1h}{ }\label{K_L03166-4v}\edtext{amusiren}{\lemma{\textnormal{\emph{amusiren}}}\Cendnote{\textnormal{Eventuell 
            fand er die Paarung der Titel im Sinne von »Liebelei« als »Recht der Seele« vergnüglich?}}}\label{K_L03166-4h}. Ihr {\\}\spacefill\mbox{Salten}\pend
           
\pstart
           \noindent{}Wie ist’s \label{K_L03166-2v}\edtext{heute mit \textcolor{pink}{Ronacher}{}\ledrightnote{\textcolor{pink}{Ronacher}}}{\lemma{\textnormal{\emph{heute mit Ronacher}}}\Cendnote{\textnormal{\textcolor{blue}{Schnitzler} besuchte an diesem Abend den
                     Polterabend von \textcolor{blue}{Ludmilla Kaufmann}, die
                     am Folgetag 
                     den Rechtsanwalt \textcolor{blue}{Siegmund Karplus}
                     heiratete. Ein Besuch der Hochzeit erwähnt \textcolor{blue}{Schnitzler} nicht, stattdessen 
                     besuchte er am 17. 11. 1895 das \textcolor{pink}{Ronacher}, so dass das Korrespondenzstück auch in der Nacht vom 16. auf den 17. gelaufen sein 
                     und sich auf den 17. beziehen könnte.
                  }}}\label{K_L03166-2h}?\pend
           \endnumbering\briefempfaengerindex{Schnitzler, Arthur@\textsc{Schnitzler, Arthur}!zzzSalten, Felix@\emph{von Felix Salten}!1895-11-162@{{[}16. 11. 1895{]}}|)be}\mylabel{h}  \normalsize

\doendnotes{C}
\bigskip
\vfill

\clearpage

\footnotesize

\lohead{\textsc{register}}

% Definiere theindex-Environment komplett neu ohne reledmac
\makeatletter
\renewenvironment{theindex}{%
  \section*{\indexname}%
  \setlength{\parindent}{0pt}%
  \setlength{\parskip}{0pt plus 0.3pt}%
  \let\item\@idxitem
}{%
  \clearpage
}
\makeatother

\IfFileExists{\jobname-pw.ind}{\input{\jobname-pw.ind}}{}

\end{document}

      