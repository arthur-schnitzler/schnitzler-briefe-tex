%% latex-korrekturansicht-vorspann.tex
%% Vorspann für die Korrekturansicht.
%% Lädt die gemeinsame Datei latex-vorspann.tex mit gesetztem Schalter.

\newif\ifkorrekturansicht
\korrekturansichttrue

\input{../tex-inputs/latex-vorspann}


               \section[Paul Goldmann an Arthur Schnitzler, Paul Goldmann an Arthur Schnitzler, 17. 10. {[}1896{]}]{ Paul Goldmann an Arthur Schnitzler, 17. 10. {[}1896{]}}\nopagebreak\mylabel{v}\rehead{ }\normalsize\beginnumbering\briefempfaengerindex{Schnitzler, Arthur@\textsc{Schnitzler, Arthur}!zzzGoldmann, Paul@\emph{von Paul Goldmann}!1896-10-172@{17. 10. {[}1896{]}}|(be} \toendnotes[C]{\smallbreak\pagebreak[2]} \Standort{DLA, A:Schnitzler, HS.NZ85.1.3166.}
\physDesc{Brief, 1 Blatt, 3 Seiten
\newline{}Handschrift: blaue Tinte, deutsche Kurrent\newline{}Beilage: Zeitungsausschnitt, beschnitten, am Spaltenumbruch
                                 zusammengeklebt. Mit blauem Buntstift den Beginn markiert. Die
                                 Rückseite ist offensichtlich nicht relevant. 
\newline{}Schnitzler: 1) mit Bleistift das Jahr »96« vermerkt 2) mit rotem Buntstift zwei Unterstreichungen}\toendnotes[C]{\smallbreak}\pstart
           \noindent{}{\pb}\textcolor{gray}{\textbf{\textbf{\textcolor{brown}{Frankfurter Zeitung}{}\ledrightnote{\textcolor{brown}{Frankfurter Zeitung}}}}}\pend
           \pstart
           \textcolor{gray}{\textbf{(\textcolor{brown}{\begin{otherlanguage}{french}Gazette de Francfort\end{otherlanguage}}{}\ledrightnote{\textcolor{brown}{Frankfurter Zeitung}}).}}\pend
           \pstart
           \textcolor{gray}{\textbf{\textbf{\begin{otherlanguage}{french}Fondateur M.\end{otherlanguage}{ }\textcolor{blue}{L. Sonnemann}{}\ledrightnote{\textcolor{blue}{Leopold Sonnemann}}.}}}\pend
           \pstart
           \begin{otherlanguage}{french}\textcolor{gray}{\textbf{\textcolor{green}{Journal}{}\ledrightnote{→\textcolor{green}{Frankfurter Zeitung}} politique,
                        financier,}}\end{otherlanguage}\pend
           \pstart
           \begin{otherlanguage}{french}\textcolor{gray}{\textbf{commercial et littéraire.}}\end{otherlanguage}\pend
           \pstart
           \begin{otherlanguage}{french}\textcolor{gray}{\textbf{\textbf{Paraissant trois fois par jour.}}}\end{otherlanguage}\hfill \textsc{\textcolor{pink}{Paris}{}\ledrightnote{\textcolor{pink}{Paris}}}, 17. October.\pend
           \pstart
           \begin{otherlanguage}{french}\textcolor{gray}{\textbf{\textbf{Bureau à \textcolor{pink}{Paris}{}\ledrightnote{\textcolor{pink}{Paris}}}}}\end{otherlanguage}\pend
           \pstart
           \begin{otherlanguage}{french}\textcolor{gray}{\textbf{\textbf{\textcolor{pink}{24. Rue Feydeau}{}\ledrightnote{\textcolor{pink}{rue Feydeau}}.}}}\end{otherlanguage}\pend
           \pstart{}Mein lieber Freund,\pend\pstart
           Warum höre ich ſo gar nichts mehr von Dir? Deine lieben Nachrichten fehlen mir ſehr.
               Eine ſo lange Pauſe haſt Du noch nie gemacht. Ich bin in Sorgen. Biſt Du unwohl? Oder
               iſt Dir ſonſt etwas Verſtimmendes \label{K_L02787-1v}\edtext{zugeſtoßen}{\lemma{\textnormal{\emph{zugeſtoßen}}}\Cendnote{\textnormal{keine Vorkommnisse
                  bekannt}}}\label{K_L02787-1h}? Du mußt mir gleich ſchreiben.\pend
           \pstart
           Anbei eine \label{K_L02787-2v}\edtext{Beſcheinigung}{\lemma{\textnormal{\emph{Beſcheinigung}}}\Cendnote{\textnormal{Beilage nicht erhalten}}}\label{K_L02787-2h} von \textsc{\textcolor{blue}{Thorel}{}\ledrightnote{\textcolor{blue}{Jean Thorel}}}, dem ich die 500 \textsc{Fr.} ausgehändigt. Dieſe
               Beſcheinigung habe ich mir ausſtelten laſſen, um gegenüber der \textsc{\textcolor{brown}{Société des Auteurs Dramatiques}{}\ledrightnote{\textcolor{brown}{Société des Auteurs et Compositeurs Dramatiques}}} (durch welche hier das \textsc{Tantièmen}-Geſchäft geht) {\pb}den \uline{Darlehens}-Character des von Dir gezahlten Betrages zu conſtatiren. Heb’ Dir das
               Billet gut auf!\pend
           \pstart
           Die \textcolor{green}{Überſetzung}{}\ledrightnote{→\textcolor{green}{Amourette. Pièce en trois actes}} iſt ſeit geſtern in meinen Händen. Ich will ſie ein wenig
               durchſchauen, dann ſoll ſie copirt werden, und dann bekommſt Du die Copie. Große Schwierigkeiten macht uns das
                  \label{K_L02787-3v}\edtext{»\textcolor{brown}{Joſefſtädter Theater}{}\ledrightnote{\textcolor{brown}{Theater in der Josefstadt}}«}{\lemma{\textnormal{\emph{»Joſefſtädter Theater«}}}\Cendnote{\textnormal{das in der
                     \emph{\textcolor{green}{Liebelei}} mehrmals namentlich erwähnt
                  wird}}}\label{K_L02787-3h}. In \textsc{\textcolor{pink}{Paris}{}\ledrightnote{\textcolor{pink}{Paris}}} hat natürlich kein Menſch eine Ahnung, was für ein Ding das iſt? Wie ſoll man
               das alſo im Franzöſiſchen umſchreiben, um dem Publicum den Eindruck des {\pb}Vorſtadt-\textsc{Milieus} zu geben?
               Vielleicht einfach: »\label{K_L02787-4v}\edtext{\begin{otherlanguage}{french}\textsc{un \textcolor{brown}{théâtre
                        du faubourg}{}\ledrightnote{→\textcolor{brown}{Theater in der Josefstadt}}}\end{otherlanguage}}{\lemma{\textnormal{\emph{un théâtre
                        du faubourg}}}\Cendnote{\textnormal{französisch, etwa:
                  Vorstadttheater}}}\label{K_L02787-4h}«? Oder fällt Dir was Beſſeres ein.\pend
           \pstart
           Anbei auch ein \label{K_L02787-18v}\edtext{\textcolor{green}{Ausſchnitt}{}\ledrightnote{→\textcolor{green}{Tages-Rundschau [Offizier hat Bürger niedergestochen]}}}{\lemma{\textnormal{\emph{Ausſchnitt}}}\Cendnote{\textnormal{siehe unten}}}\label{K_L02787-18h} aus unſerem \textcolor{green}{Blatte}{}\ledrightnote{→\textcolor{green}{Frankfurter Zeitung}} über eine dieſer Tage
               vorgefallene Säbel-Affaire. Wenn Du \strikeout{\textcolor{gray}{h}} das noch nicht geleſen haſt, wirds Dich intereſſiren.\pend
           \pstart
           Wie ſtehts mit \label{K_L02787-5v}\edtext{\textcolor{pink}{Berlin}{}\ledrightnote{\textcolor{pink}{Berlin}}}{\lemma{\textnormal{\emph{Berlin}}}\Cendnote{\textnormal{Siehe dazu vor allem \emph{Der Briefwechsel Arthur Schnitzler — Otto Brahm}.
                     Vollständige Ausgabe. Herausgegeben, eingeleitet und erläutert von Oskar
                     Seidlin. Tübingen: \emph{Niemeyer}{ }1975, S. 14 ff.}}}\label{K_L02787-5h}?\pend
           \pstart
           Durch die verfluchten \label{K_L02787-7v}\edtext{\textcolor{pink}{Ruſſ}{}\ledrightnote{\textcolor{pink}{Russland}}enfeſte}{\lemma{\textnormal{\emph{Ruſſenfeſte}}}\Cendnote{\textnormal{\textcolor{blue}{Goldmann} bezog sich hier wohl auf den \textcolor{pink}{Frankreich}-Besuch des Zaren \textcolor{blue}{Nikolaus II.} und der Kaiserin \textcolor{blue}{Alexandra Fjodorowna} zwischen dem 5. und 9. 10. 1896 und den damit
                  einhergehenden »Zarentagen« in \textcolor{pink}{Paris}.}}}\label{K_L02787-7h}
               habe ich noch keine Zeit gehabt, zu \textsc{\textcolor{blue}{Forain}{}\ledrightnote{\textcolor{blue}{Jean-Louis Forain}}} zu gehen. Das bleibt für nächſte Woche.\pend
           \pstart
           Viele treue Grüße!\pend
           \pstart
           Dein {\\[\baselineskip]}\spacefill\mbox{Paul Goldmann}\pend
           \leftskip=0em{}\pstart
           \noindent{}\textsc{\textcolor{blue}{Leo Fanjung}{}\ledrightnote{\textcolor{blue}{Leo Van-Jung}}} war hier, mit dem ich mich rieſig geſreut habe. Welch’
                     {[}ein{]} liebes Kind!\pend
           {\bigskip}\pstart
           \noindent{}{\pb}\label{K_L02787-999v}\edtext{Wie ſchon mitgetheilt}{\lemma{\textnormal{\emph{Wie ſchon mitgetheilt}}}\Cendnote{\textnormal{Der beiliegende Ausschnitt ist gedruckt
                  und aus der \emph{\textcolor{green}{Frankfurter Zeitung}}
                  ausgeschnitten: \emph{\textcolor{green}{Tages-Rundschau}}. In: \emph{\textcolor{green}{Frankfurter Zeitung}}, Jg. 41, Nr. 286,
                        14. 10. 1896, Abendblatt, S. 1.}}}\label{K_L02787-999h} wurde, hat in \textcolor{pink}{\so{Karlsruhe}}{}\ledrightnote{\textcolor{pink}{Karlsruhe}} ein \textcolor{blue}{\so{Offizier}}{}\ledrightnote{→\textcolor{blue}{Henning von Brüsewitz}} einen \textcolor{blue}{\so{Bürger}}{}\ledrightnote{→\textcolor{blue}{Theodor Siepmann}} ohne jede Veranlaſſung \so{niedergeſtochen}. Ueber den
               traurigen Vorgang erhalten wir von einem Augenzeugen zugleich nach den Mittheilungen
               weiterer Augenzeugen eine Darſtellung, die durchaus den Eindruck der Glaubwürdigkeit
               macht. Wir geben ſie nachſtehend wieder, da der Vorgang zu einigen Bemerkungen an
               dieſer Stelle Veranlaſſung gibt. Der Augenzeuge ſchreibt:\pend
           \pstart
           Premierlieutenant \textcolor{blue}{v. Brüſewitz}{}\ledrightnote{\textcolor{blue}{Henning von Brüsewitz}} begann mit \textcolor{blue}{Siepmann}{}\ledrightnote{\textcolor{blue}{Theodor Siepmann}} einen Wortwechſel, weil dieſer
               angeblich beim Niederſitzen an ſeinen Stuhl geſtoßen ſein ſoll, was übrigens ſelbſt
               von den mit \textcolor{blue}{Siepmann}{}\ledrightnote{\textcolor{blue}{Theodor Siepmann}} am gleichen Tiſche
               ſitzenden Perſonen nicht bemerkt wurde. \textcolor{blue}{Siepmann}{}\ledrightnote{\textcolor{blue}{Theodor Siepmann}} erwiderte, er wiſſe nichts davon, daß er \textcolor{blue}{v. Brüſewitz}{}\ledrightnote{\textcolor{blue}{Henning von Brüsewitz}} angerempelt habe. Dieſer rief hierauf den
                  \label{K_L02787-19v}\edtext{\textcolor{blue}{Wirth}{}\ledrightnote{→\textcolor{blue}{?? [Wirt im Café Tannhäuser]}}}{\lemma{\textnormal{\emph{Wirth}}}\Cendnote{\textnormal{nicht identifiziert}}}\label{K_L02787-19h} und forderte
               ihn auf, \textcolor{blue}{Siepmann}{}\ledrightnote{\textcolor{blue}{Theodor Siepmann}} hinauszuweiſen, der nicht
               wiſſe, wie er ſich zu betragen habe. Der \textcolor{blue}{Wirth}{}\ledrightnote{→\textcolor{blue}{?? [Wirt im Café Tannhäuser]}} ſuchte die Beiden durch Zureden zu beruhigen was ihm
               anſcheinend auch gelang. \textcolor{blue}{Siepmann}{}\ledrightnote{\textcolor{blue}{Theodor Siepmann}} verließ
               dann das \textcolor{pink}{Lokal}{}\ledrightnote{→\textcolor{pink}{Café Tannhäuser}}, kam aber
               gleich darauf wieder herein und ſetzte ſich. Nach kurzer Zeit rief \textcolor{blue}{v. Brüſewitz}{}\ledrightnote{\textcolor{blue}{Henning von Brüsewitz}} ſehr laut: »Sie haben mich in brüſker Weiſe
               angerempelt und ſich nicht entſchuldigt.« \textcolor{blue}{Siepmann}{}\ledrightnote{\textcolor{blue}{Theodor Siepmann}} erwiderte: »Ich weiß nichts davon.« Daraufhin ſprang \textcolor{blue}{v. Brüſewitz}{}\ledrightnote{\textcolor{blue}{Henning von Brüsewitz}} auf, ſtellte ſich vor \textcolor{blue}{Siepmann}{}\ledrightnote{\textcolor{blue}{Theodor Siepmann}} hin und ſchrie: »Wollen Sie mich um
               Entſchuldigung bitten, ja oder nein, ja oder nein, ja oder nein?« \textcolor{blue}{Siepmann}{}\ledrightnote{\textcolor{blue}{Theodor Siepmann}} blieb ruhig ſitzen und erwiderte ſchließlich:
               »Keine Antwort wird Ihnen auch genügen.« Daraufhin trat \textcolor{blue}{v. Brüſewitz}{}\ledrightnote{\textcolor{blue}{Henning von Brüsewitz}} 2 bis 3 Schritte zurück, ſchrie: »\so{Nein, das genügt mir ganz und gar nicht}«, riß den Säbel aus
               der Scheide und wollte mit hochgeſchwungener Waffe auf \textcolor{blue}{Siepmann}{}\ledrightnote{\textcolor{blue}{Theodor Siepmann}} eindringen. Der \textcolor{blue}{Wirth}{}\ledrightnote{→\textcolor{blue}{?? [Wirt im Café Tannhäuser]}} und der Kellner fielen ihm jedoch in den Arm und
               hielten ihn feſt, während \textcolor{blue}{Siepmann}{}\ledrightnote{\textcolor{blue}{Theodor Siepmann}} das Lokal
               verließ und auf den Hof ging. \textcolor{blue}{v. Brüſewitz}{}\ledrightnote{\textcolor{blue}{Henning von Brüsewitz}}
               ſteckte ſeinen Säbel ein, ſetzte die Mütze auf, zog den Mantel an und rief dabei »\so{Meine Ehre iſt kaput, ich bin ein todter Mann; morgen kann ich meinen Abſchied einreichen.}« Mit dieſen Worten verließ er das \textcolor{pink}{Lokal}{}\ledrightnote{→\textcolor{pink}{Café Tannhäuser}} durch die nach der \textcolor{pink}{Karlſtraße}{}\ledrightnote{\textcolor{pink}{Karlstraße}} führende Thür. Dort ſtand ein \textcolor{blue}{Schutzmann}{}\ledrightnote{→\textcolor{blue}{Theodor Siepmann}}, bei dem ſich \textcolor{blue}{v. Brüſewitz}{}\ledrightnote{\textcolor{blue}{Henning von Brüsewitz}} erkundigte, ob \textcolor{blue}{Siepmann}{}\ledrightnote{\textcolor{blue}{Theodor Siepmann}} das \textcolor{pink}{Lokal}{}\ledrightnote{→\textcolor{pink}{Café Tannhäuser}} verlaſſen habe. Als dieſer das verneinte, ſagte \textcolor{blue}{v. Brüſewitz}{}\ledrightnote{\textcolor{blue}{Henning von Brüsewitz}}: »den muß ich abpaſſen.« \so{Er holte dann zwei Feldwebel herbei}, denen er befahl, an
               der Thüre zu bleiben, \so{da er bedroht ſei}. Er ſelbſt ging
               von der \textcolor{pink}{Kaiſerſtraße}{}\ledrightnote{\textcolor{pink}{Kaiserstraße}} aus wieder in den zu den
               vordern Lokalen führenden Gang hinein. Inzwiſchen hatten der \textcolor{blue}{Wirth}{}\ledrightnote{→\textcolor{blue}{?? [Wirt im Café Tannhäuser]}} und ein anderer Herr dem \textcolor{blue}{Siepmann}{}\ledrightnote{\textcolor{blue}{Theodor Siepmann}} im Hofe zugeredet, er ſolle, um die
               Sache gütlich zu erledigen, am andern Tage zu \textcolor{blue}{v.
                  Brüſewitz}{}\ledrightnote{\textcolor{blue}{Henning von Brüsewitz}} gehen und ſich entſchuldigen, wozu er auch bereit ſchien. Er bat
               den \textcolor{blue}{Wirth}{}\ledrightnote{→\textcolor{blue}{?? [Wirt im Café Tannhäuser]}}, ihm ſeinen Hut
               zu holen. Der \textcolor{blue}{Wirth}{}\ledrightnote{→\textcolor{blue}{?? [Wirt im Café Tannhäuser]}} holte
               den Hut, und wollte \textcolor{blue}{Siepmann}{}\ledrightnote{\textcolor{blue}{Theodor Siepmann}} vom Hofe auf den
               nach der \textcolor{pink}{Kaiſerſtraße}{}\ledrightnote{\textcolor{pink}{Kaiserstraße}} führenden Hausflur
               laſſen. Als er die Thür öffnete, ſtand \textcolor{blue}{v.
                  Brüſewitz}{}\ledrightnote{\textcolor{blue}{Henning von Brüsewitz}} direkt vor der Thür und wollte mit den Worten: »Wo iſt der Schuft?«
               in den Hof eindringen. Der \textcolor{blue}{Wirth}{}\ledrightnote{→\textcolor{blue}{?? [Wirt im Café Tannhäuser]}} faßte ihn am Arme und rief ihm laut zu: »Herr \textcolor{blue}{Lieutenant}{}\ledrightnote{→\textcolor{blue}{Henning von Brüsewitz}}, der \textcolor{blue}{Mann}{}\ledrightnote{→\textcolor{blue}{Theodor Siepmann}} will ſich ja entſchuldigen.« \textcolor{blue}{Von Brüſewitz}{}\ledrightnote{\textcolor{blue}{Henning von Brüsewitz}} erwiderte nichts, zog, als er
                  \textcolor{blue}{Siepmann}{}\ledrightnote{\textcolor{blue}{Theodor Siepmann}} erblickte, den Säbel und ging auf
               ihn los. \textcolor{blue}{Siepmann}{}\ledrightnote{\textcolor{blue}{Theodor Siepmann}} ergriff die Flucht und
               rief: »Ich bitte um Verzeihung, verzeihen Sie mir.« Am Ende des nur wenige Schritte
               langen Hofes, holte \textcolor{blue}{v. Brüſewitz}{}\ledrightnote{\textcolor{blue}{Henning von Brüsewitz}} den \textcolor{blue}{Siepmann,}{}\ledrightnote{\textcolor{blue}{Theodor Siepmann}} der die Thüre zum \textcolor{pink}{Lokal}{}\ledrightnote{→\textcolor{pink}{Café Tannhäuser}} nicht fand, ein und stach ihn
               nieder. Als er die blutige Waffe wieder einſteckte, ſagte er: »\so{So, jetzt iſt meine Ehre gerettet,}« und begab ſich dann durch das \textcolor{pink}{Lokal}{}\ledrightnote{→\textcolor{pink}{Café Tannhäuser}} ungehindert auf die
               Straße. \textcolor{blue}{Siepmann}{}\ledrightnote{\textcolor{blue}{Theodor Siepmann}} wurde von einigen Herren in
               die Portierſtube auf ein Bett gebracht, wo er nach etwa einer halben Stunde
               verſchied. Der Säbel war auf der rechten Seite ungefähr 30 cm tief eingedrungen und
               hatte die Leber und wahrſcheinlich noch andere Organe durchbohrt. Die Wunde war
               abſolut tödtlich, und die ärztliche Hilfe war vergeblich.\pend
           \endnumbering\briefempfaengerindex{Schnitzler, Arthur@\textsc{Schnitzler, Arthur}!zzzGoldmann, Paul@\emph{von Paul Goldmann}!1896-10-172@{17. 10. {[}1896{]}}|)be}\mylabel{h}  \normalsize

\doendnotes{C}
\bigskip
\vfill

\clearpage

\footnotesize

\lohead{\textsc{register}}

% Definiere theindex-Environment komplett neu ohne reledmac
\makeatletter
\renewenvironment{theindex}{%
  \section*{\indexname}%
  \setlength{\parindent}{0pt}%
  \setlength{\parskip}{0pt plus 0.3pt}%
  \let\item\@idxitem
}{%
  \clearpage
}
\makeatother

\IfFileExists{\jobname-pw.ind}{\input{\jobname-pw.ind}}{}

\end{document}

      