%% latex-korrekturansicht-vorspann.tex
%% Vorspann für die Korrekturansicht.
%% Lädt die gemeinsame Datei latex-vorspann.tex mit gesetztem Schalter.

\newif\ifkorrekturansicht
\korrekturansichttrue

\input{../tex-inputs/latex-vorspann}


\renewcommand{\erwaehntePersonen}{Personen: Louise Schnitzler, Olga Schnitzler}
\renewcommand{\erwaehnteOrte}{Orte: Berlin, Edmund-Weiß-Gasse, Wien}
\renewcommand{\erwaehnteWerke}{}
\section[ Paul Goldmann an Arthur Schnitzler, 12. 5. 1905]{Paul Goldmann an Arthur Schnitzler, 12. 5. 1905}
\nopagebreak\mylabel{v}
\rehead{ }\normalsize\beginnumbering\briefempfaengerindex{Schnitzler, Arthur@\textsc{Schnitzler, Arthur}!zzzGoldmann, Paul@\emph{von Paul Goldmann}!1905-05-121@{12. 5. 1905}|(be}
\toendnotes[C]{\smallbreak\pagebreak[2]}\Standort{DLA, A:Schnitzler, HS.NZ85.1.3175.}
\physDesc{Postkarte
\newline{}Handschrift: 1) blaue Tinte, deutsche Kurrent\hspace{1em}2) blaue Tinte, lateinische Kurrent (\noindent{}Adresse)\hspace{1em}
\newline{}Versand: 1) Stempel: »\nobreak{}\oindex{Berlin@\textbf{Berlin}, \emph{https://www.geonames.org/ontologyP.PPLC}|pwk}Berlin S. W. 11, 12. 5. 05, 4–5 N\nobreak{}«.   2) Stempel: »\nobreak{}Wien 110, {[}1{]}\textcolor{gray}{3}. 5. 05, Bestellt\nobreak{}«. 
\newline{}Schnitzler: mit Bleistift das Jahr »{[}19{]}05« vermerkt }\toendnotes[C]{\smallbreak}\pstart{}{\pb}Herrn\pend{}\pstart{}Dr. Arthur Schnitzler\pend{}\pstart{}\textcolor{pink}{Wien}{}\ledrightnote{\textcolor{pink}{Wien}}\pend{}\pstart{}\textcolor{pink}{XVIII. Spöttelgaſse 7}{}\ledrightnote{\textcolor{pink}{Edmund-Weiß-Gasse}}.\pend{}
{\bigskip}
\pstart
           \noindent{}{\pb}\textcolor{pink}{Berlin}{}\ledrightnote{\textcolor{pink}{Berlin}}, 12. Mai.
                  Lieber Freund, Ich habe ſehr bedauert, Dich in
                  \label{K_L03232-1v}\edtext{\textcolor{pink}{Wien}{}\ledrightnote{\textcolor{pink}{Wien}}}{\lemma{\textnormal{\emph{Wien}}}\Cendnote{\textnormal{siehe Paul Goldmann an Arthur Schnitzler, 3. 5. 1905}}}\label{K_L03232-1h} nicht angetroffen zu haben, und danke
               Dir nachträglich für die Einladung, die mich nicht erreicht hat. Hoffentlich gibt mir
               der \label{K_L03232-2v}\edtext{Sommer}{\lemma{\textnormal{\emph{Sommer}}}\Cendnote{\textnormal{Am 31. 7. 1905 besuchte \textcolor{blue}{Goldmann}{ }\textcolor{blue}{Schnitzler} in \textcolor{pink}{Wien}.}}}\label{K_L03232-2h} Gelegenheit, Dich zu ſehen. Laß’ mich jedenfalls wiſſen, wo
               Du biſt. Mit Deiner \textcolor{blue}{Mutter}{}\ledrightnote{{$\rightarrow$}\textcolor{blue}{Louise Schnitzler}}
               habe ich ſo halb und halb ein Zuſammentreffen verabredet. Herzliche Grüße an Dich und
               Deine \textcolor{blue}{Frau}{}\ledrightnote{{$\rightarrow$}\textcolor{blue}{Olga Schnitzler}} von Deinem {\\}\spacefill\mbox{Paul Goldmann.}\pend
           
\pstart
           \noindent{}Haſt Du nicht dieſer Tage Deinen \label{K_L03232-3v}\edtext{Geburtstag}{\lemma{\textnormal{\emph{Geburtstag}}}\Cendnote{\textnormal{\textcolor{blue}{Schnitzler} wurde am 15. 5. 1905 43 Jahre alt.}}}\label{K_L03232-3h}? Wenn ja, ſo gratulire ich \strikeout{D\textcolor{gray}{ic}h} Dir herzlich.\pend
           \endnumbering\briefempfaengerindex{Schnitzler, Arthur@\textsc{Schnitzler, Arthur}!zzzGoldmann, Paul@\emph{von Paul Goldmann}!1905-05-121@{12. 5. 1905}|)be}\mylabel{h}  \normalsize

\doendnotes{C}
\bigskip
\vfill

\clearpage

\footnotesize

\lohead{\textsc{register}}

% Definiere theindex-Environment komplett neu ohne reledmac
\makeatletter
\renewenvironment{theindex}{%
  \section*{\indexname}%
  \setlength{\parindent}{0pt}%
  \setlength{\parskip}{0pt plus 0.3pt}%
  \let\item\@idxitem
}{%
  \clearpage
}
\makeatother

\IfFileExists{\jobname-pw.ind}{\input{\jobname-pw.ind}}{}

\end{document}

      