%% latex-korrekturansicht-vorspann.tex
%% Vorspann für die Korrekturansicht.
%% Lädt die gemeinsame Datei latex-vorspann.tex mit gesetztem Schalter.

\newif\ifkorrekturansicht
\korrekturansichttrue

\input{../tex-inputs/latex-vorspann}


               \section[Eduard Michael Kafka an Arthur Schnitzler, 12. 1. 1893]{ Eduard Michael Kafka an Arthur Schnitzler, 12. 1. 1893}\nopagebreak\mylabel{v}\rehead{ }\normalsize\beginnumbering\briefempfaengerindex{Schnitzler, Arthur@\textsc{Schnitzler, Arthur}!zzzKafka, Eduard Michael@\emph{von Eduard Michael Kafka}!1893-01-121@{12. 1. 1893}|(be} \toendnotes[C]{\smallbreak\pagebreak[2]} \Standort{DLA, A:Schnitzler, HS.NZ85.1.3604.}
\physDesc{Brief, 1 Blatt, 4 Seiten
\newline{}Handschrift: schwarze Tinte, deutsche Kurrent
\newline{}Schnitzler: mit rotem Buntstift mehrere Unterstreichungen }\toendnotes[C]{\smallbreak}\pstart
           \raggedleft{}{\pb}12/1 93.\pend
           \pstart{}Lieber Freund,\pend\pstart
           vorgeſtern – bei einer Soiree des Rechtsanwalts D\textsuperscript{r}{ }\textcolor{blue}{Grelling}{}\ledrightnote{\textcolor{blue}{Richard Grelling}} in \textcolor{pink}{\textsc{Berlin}}{}\ledrightnote{\textcolor{pink}{Berlin}} – wurde Ihre »\textcolor{green}{Frage an das Schickſal}{}\ledrightnote{\textcolor{green}{Die Frage an das Schicksal}}«
                    aufgeführt. \textcolor{blue}{Reicher}{}\ledrightnote{\textcolor{blue}{Emanuel Reicher}} brillirte als \textcolor{green}{Anatol}{}\ledrightnote{\textcolor{green}{Anatol}} – ich kann Ihnen nicht ſchildern, wie
                    vorzüglich er war: einfach ganz \uline{einzig}, der \textcolor{green}{Anatol}{}\ledrightnote{\textcolor{green}{Anatol}}{ }\textsc{par excellence}. – Es hat mich ungemein gefreut, daſs
                    ich der Aufführung Ihres Stückes – in ſo meiſterlicher Darſtellung – habe
                    perſönlich beiwohnen können. Es waren mehr {\pb}als 100 Perſonen anweſend; die
                    hervorragendſten \textsc{literarischen} u künſtleriſchen Kreiſe
                    waren vertreten: von \textcolor{blue}{Sudermann}{}\ledrightnote{\textcolor{blue}{Hermann Sudermann}} bis \textcolor{blue}{Träger}{}\ledrightnote{\textcolor{blue}{Albert Traeger}}. \textcolor{blue}{Sudermann}{}\ledrightnote{\textcolor{blue}{Hermann Sudermann}}\introOben{}inſonderheit\introOben{} war ganz entzückt u. wurde nicht müde,
                    ſeinen Beifall in der allerlebhafteſten Weiſe, durch beſtändige Zwischenrufe \substVorne{}\textsuperscript{\textcolor{gray}{von}}\substDazwischen{}aufrichtiger\substHinten{} Bewunderung, Ausdruck zu geben.\pend
           \pstart
           \textcolor{blue}{Reicher}{}\ledrightnote{\textcolor{blue}{Emanuel Reicher}} läßt Sie grüßen. Er bat mich Ihnen
                        \introOben{}zugleich\introOben{} mitzuteilen, daſs \textcolor{blue}{Blumenthal}{}\ledrightnote{\textcolor{blue}{Oskar Blumenthal}}{ }\substVorne{}\textsuperscript{\textcolor{gray}{angeg}}\substDazwischen{}bezüglich\substHinten{} der Aufführung des »\textcolor{green}{Märchen}{}\ledrightnote{\textcolor{green}{Das Märchen. Schauspiel in drei Aufzügen}}« darauf
                    {\pb}hinweiſt, daſs Sie ihm ſeinerzeit
                    geſagt hätten, das Stück werde in \textcolor{pink}{Prag}{}\ledrightnote{\textcolor{pink}{Prag}} gegeben
                    werden. \textcolor{blue}{Er}{}\ledrightnote{→\textcolor{blue}{Oskar Blumenthal}} möchte erst
                    dieſe Aufführung abwarten, – Sie ſollen daher zuſehen, daſs Sie die \textcolor{pink}{Prag}{}\ledrightnote{\textcolor{pink}{Prag}}er Première beſchleunigen. – Notabene,
                    Lieber Freund, – dieſes \textcolor{pink}{Berlin}{}\ledrightnote{\textcolor{pink}{Berlin}} iſt eine
                    herrliche Stadt: ich fühle mich hier, obwol ich erſt einige Tage da bin, ſo
                    heimiſch, als wäre {\pb}ich \substVorne{}\textsuperscript{hier}\substDazwischen{}dort\substHinten{} geboren. Wir wiſſen in \textcolor{pink}{Wien}{}\ledrightnote{\textcolor{pink}{Wien}} nicht, was
                    geiſtiges u künſtleriſches Leben bedeutet: man muſs hieher kommen, wenn man dies
                    erfahren will.\pend
           \pstart
           Raten Sie, bitte, ſchleunigſt allen unſeren lieben Freunden: Sie ſollen ohne
                    Zaudern, ohne eine Minute zu verlieren, ihr Bündel packen und nach \textcolor{pink}{Berlin}{}\ledrightnote{\textcolor{pink}{Berlin}} ko{\geminationm}en –
                    Alle, – es iſt hier Boden genug für ſie u. in \textcolor{pink}{Wien}{}\ledrightnote{\textcolor{pink}{Wien}} werden ſie \introOben{}ja\introOben{} doch alle verkü{\geminationm}ern! \pend
           \pstart
           Herzlichſt Ihr{\\[\baselineskip]}EMKafka\pend
           \leftskip=0em{}\pstart
           \noindent{}\label{T_L00158_1v}\edtext{\textcolor{pink}{Hotel \textsc{Wienerhof}}{}\ledrightnote{\textcolor{pink}{Wienerhof}}, \textcolor{pink}{Marienstraße}{}\ledrightnote{\textcolor{pink}{Marienstraße}} 20}{\lemma{\textnormal{\emph{Hotel … 20}}}\Cendnote{\textnormal{quer am Rand der letzten Seite}}}\label{T_L00158_1h}\pend
           \endnumbering\briefempfaengerindex{Schnitzler, Arthur@\textsc{Schnitzler, Arthur}!zzzKafka, Eduard Michael@\emph{von Eduard Michael Kafka}!1893-01-121@{12. 1. 1893}|)be}\mylabel{h}  \normalsize

\doendnotes{C}
\bigskip
\vfill

\clearpage

\footnotesize

\lohead{\textsc{register}}

% Definiere theindex-Environment komplett neu ohne reledmac
\makeatletter
\renewenvironment{theindex}{%
  \section*{\indexname}%
  \setlength{\parindent}{0pt}%
  \setlength{\parskip}{0pt plus 0.3pt}%
  \let\item\@idxitem
}{%
  \clearpage
}
\makeatother

\IfFileExists{\jobname-pw.ind}{\input{\jobname-pw.ind}}{}

\end{document}

      