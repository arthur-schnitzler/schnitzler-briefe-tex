%% latex-korrekturansicht-vorspann.tex
%% Vorspann für die Korrekturansicht.
%% Lädt die gemeinsame Datei latex-vorspann.tex mit gesetztem Schalter.

\newif\ifkorrekturansicht
\korrekturansichttrue

\input{../tex-inputs/latex-vorspann}


               \section[Paul Goldmann an Arthur Schnitzler, 27. 4. {[}1902?{]}]{ Paul Goldmann an Arthur Schnitzler, 27. 4. {[}1902?{]}}\nopagebreak\mylabel{v}\rehead{ }\normalsize\beginnumbering\briefempfaengerindex{Schnitzler, Arthur@\textsc{Schnitzler, Arthur}!zzzGoldmann, Paul@\emph{von Paul Goldmann}!1902-04-271@{27. 4. {[}1902?{]}}|(be} \toendnotes[C]{\smallbreak\pagebreak[2]} \Standort{DLA, A:Schnitzler, HS.NZ85.1.3170.}
\physDesc{Telegramm
\newline{}maschinell\newline{}Versand: 1) Stempel: »\nobreak{}\oindex{IX., Alsergrund@\textbf{IX., Alsergrund}, \emph{Bezirk (A.BZK)}|pwk}{[}Wien{]} 9/2, 27 IV 00\nobreak{}«.  2) »\noindent{}\textcolor{gray}{\textbf{\textit{27 Apr}}}{ / }\textcolor{gray}{\textbf{\textit{\textcolor{blue}{Zaunegger}}}}{ / }\textcolor{gray}{\textbf{\textit{Ausgefertigt
                                             27 Apr{ }8\textsubscript{10>}}}}«\newline{}Ordnung: beschnitten }\toendnotes[C]{\smallbreak}\pstart{}{\pb}= arthur schnitzler \textcolor{pink}{wien}{}\ledrightnote{\textcolor{pink}{Wien}}\pend{}\pstart{}\textcolor{pink}{neuntbezirk frankgasse}{}\ledrightnote{\textcolor{pink}{Frankgasse}} =\pend{}{\bigskip}\pstart
           {\pb}v{[}on{]}{ }\textcolor{pink}{berlin}{}\ledrightnote{\textcolor{pink}{Berlin}} 68646
                     24 276 33 S\pend
           \pstart
           ich glaube nicht dasz die \label{K_L02633-1v}\edtext{\textcolor{green}{notizen}{}\ledrightnote{→\textcolor{green}{Ein litterarisch-dramatisches Hochstapler-Stücklein}}}{\lemma{\textnormal{\emph{notizen}}}\Cendnote{\textnormal{Dieses Telegramm ist im Nachlass den Korrespondenzstücken des
               Jahres 1900 zugeordnet. Die Datierung dürfte auf den abgeschnitten überlieferten Stempel 
                  zurückgehen, der sichtbar die Zeichenfolge »27 IV 00« enthält. Ob es sich
               dabei um einen falsch eingestellten Stempel handelt oder ob es hier um Reste der Uhrzeit geht, bleibt
               unklar. Das Telegramm dürfte jedenfalls zu jenem des Vortags (Paul Goldmann an Arthur Schnitzler, 26. 4. 1902) gehören.}}}\label{K_L02633-1h}
               irgendwelche folgen haben werden; sie sind nur taktlos und albern. herzlichst =
                  \spacefill\mbox{goldmann}\pend
           \endnumbering\briefempfaengerindex{Schnitzler, Arthur@\textsc{Schnitzler, Arthur}!zzzGoldmann, Paul@\emph{von Paul Goldmann}!1902-04-271@{27. 4. {[}1902?{]}}|)be}\mylabel{h}  \normalsize

\doendnotes{C}
\bigskip
\vfill

\clearpage

\footnotesize

\lohead{\textsc{register}}

% Definiere theindex-Environment komplett neu ohne reledmac
\makeatletter
\renewenvironment{theindex}{%
  \section*{\indexname}%
  \setlength{\parindent}{0pt}%
  \setlength{\parskip}{0pt plus 0.3pt}%
  \let\item\@idxitem
}{%
  \clearpage
}
\makeatother

\IfFileExists{\jobname-pw.ind}{\input{\jobname-pw.ind}}{}

\end{document}

      