%% latex-korrekturansicht-vorspann.tex
%% Vorspann für die Korrekturansicht.
%% Lädt die gemeinsame Datei latex-vorspann.tex mit gesetztem Schalter.

\newif\ifkorrekturansicht
\korrekturansichttrue

\input{../tex-inputs/latex-vorspann}


\renewcommand{\erwaehntePersonen}{Personen: Josef Kainz, Margarethe Kainz, Anna Katharina Rehmann, Felix Salten, Paul Salten, Olga Schnitzler, Heinrich Schnitzler}
\renewcommand{\erwaehnteInstitutionen}{Institutionen: Burgtheater}
\renewcommand{\erwaehnteOrte}{Orte: Grado, Lido, Venedig, Villa Bauer, Wien}
\renewcommand{\erwaehnteWerke}{Werke: Der neue Vertrag Josef Kainz’, Neue Freie Presse}
\section[ Felix Salten an Arthur Schnitzler, 29. 6. 1909]{Felix Salten an Arthur Schnitzler, 29. 6. 1909}
\nopagebreak\mylabel{v}
\rehead{ }\normalsize\beginnumbering\briefempfaengerindex{Schnitzler, Arthur@\textsc{Schnitzler, Arthur}!zzzSalten, Felix@\emph{von Felix Salten}!1909-06-291@{29. 6. 1909}|(be}
\toendnotes[C]{\smallbreak\pagebreak[2]}\Standort{CUL, Schnitzler, B 89, B 1.}
\physDesc{Brief, 1 Blatt, 1 Seite, 1555 Zeichen
\newline{}Handschrift: schwarze Tinte, lateinische Kurrent
\newline{}Schnitzler: mit Bleistift Vermerk: »\textsc{Salten}« 
\newline{}Ordnung: mit Bleistift von unbekannter Hand nummeriert: »251« }\toendnotes[C]{\smallbreak}
\pstart
           \noindent{}{\pb}\textcolor{pink}{\textcolor{gray}{\textbf{\textsc{Villa Bauer}}}}{}\ledrightnote{\textcolor{pink}{Villa Bauer}}\hfill \textcolor{pink}{\textcolor{gray}{\textbf{\textsc{Grado}}}}{}\ledrightnote{\textcolor{pink}{Grado}}\pend
           
\pstart
           \raggedleft{}\textcolor{gray}{\textbf{\textsc{Küstenland}}}\pend
           
\pstart
           \raggedleft{}29. VI. 09\pend
           
\pstart{}Lieber,\pend
\pstart
           wir sind heute aus \textcolor{pink}{Venedig}{}\ledrightnote{\textcolor{pink}{Venedig}} zurückgekommen, und ich finde Ihren Brief vom 22. Das letzte, was mir \textcolor{blue}{Kainz}{}\ledrightnote{\textcolor{blue}{Josef Kainz}} sagte, war etwa zwei Tage vor meiner Abreise; und da meinte er, er
               wolle es seiner \textcolor{blue}{Frau}{}\ledrightnote{{$\rightarrow$}\textcolor{blue}{Margarethe Kainz}}
               überlaßen, im Dezember darüber zu verfügen. Deshalb
               glaube ich, die \label{K_L03501-1v}\edtext{\textcolor{green}{Zeitungsnotiz}{}\ledrightnote{{$\rightarrow$}\textcolor{green}{Der neue Vertrag Josef Kainz’}}}{\lemma{\textnormal{\emph{Zeitungsnotiz}}}\Cendnote{\textnormal{In den Tagen rund um \textcolor{blue}{Schnitzler}s (nicht erhaltenem) Brief an \textcolor{blue}{Salten} vom 22. 6. 1909 waren die
                  Vertragsverhandlungen des \emph{\textcolor{brown}{Burgtheaters}} mit \textcolor{blue}{Josef Kainz} in den \textcolor{pink}{Wien}er Tageszeitungen ein großes Thema. Wenngleich unklar
                  ist, auf welche Meldung sich \textcolor{blue}{Schnitzler}
                  genau bezieht, dürfte es inhaltlich dieser entsprechen, die aber erst nach dem
                  Brief an \textcolor{blue}{Salten} erschien: [O. V.]: \emph{\textcolor{green}{Der neue Vertrag Josef Kainz’}}. In: \emph{\textcolor{green}{Neue Freie Presse}}, Nr. 16.105, 23. 6. 1909, Abendblatt, S. 5. Darin wurde
                  vom neuen Vertrag von \textcolor{blue}{Josef Kainz} mit dem
                     \emph{\textcolor{brown}{Burgtheater}} berichtet und kolportiert, \textcolor{blue}{Kainz} sei nur noch zwei Monate im Semester
                  in \textcolor{pink}{Wien} und löse deshalb seinen Haushalt auf.
                     \textcolor{blue}{Schnitzler} hatte Interesse an dieser
                  Wohnung, siehe A. S.: \emph{Tagebuch}, 23. 6. 1909.}}}\label{K_L03501-1h} dürfte nicht ganz stimmen. Auch scheint es mir nicht wahrscheinlich,
               dass \textcolor{blue}{Kainz}{}\ledrightnote{\textcolor{blue}{Josef Kainz}}, müde wie er jetzt ist, sich vor
               den Ferien mit der »Auflösung des Hausstandes« befaßen wird. Es müßte denn inzwischen
               seine \textcolor{blue}{Frau}{}\ledrightnote{{$\rightarrow$}\textcolor{blue}{Margarethe Kainz}} irgend etwas
               veranlaßt haben. Aber auch das halte ich nicht für wahrscheinlich. Sollte es dennoch
               der Fall sein, dann bezieht es sich wol nur auf den Termin, wann die Wohnung geräumt
               wird. Wenn Sie wollen, frage ich direkt bei Frau \textcolor{blue}{Kainz}{}\ledrightnote{\textcolor{blue}{Margarethe Kainz}} an. Sicherlich wird sie mir dann gegen den 7. od. 8. Juli Bescheid geben, sobald sie
               mit ihm zusammentrifft. Oder Sie schreiben ihm ein paar Zeilen. Ich bleibe jetzt
               voraussichtlich bis 15. Juli ununterbrochen \textcolor{pink}{hier}{}\ledrightnote{{$\rightarrow$}\textcolor{pink}{Grado}}. In \textcolor{pink}{Venedig}{}\ledrightnote{\textcolor{pink}{Venedig}} war es sehr schön, und den \textcolor{pink}{Lido}{}\ledrightnote{\textcolor{pink}{Lido}} fanden wir in allen Verhältnissen, Strand, Bad, Capanne,
               ec. um so viel komfortabler, dass wir nächstes Jahr wol hingehen werden, falls wir
               wieder ans Meer wollen. Den \textcolor{blue}{Kinder}{}\ledrightnote{{$\rightarrow$}\textcolor{blue}{Paul Salten}{\newline}{$\rightarrow$}\textcolor{blue}{Anna Katharina Rehmann}}n ist hier bis jetzt und unberufen sehr wol. Sie
               haben nichts von den kleinen Übeln bekommen, die für gefährlich profezeiht werden.
               Ich hatte den Sonnenbrand und Fieber, aber das Fieber war von der Erkältung, die ich
               mit her brachte. Und jetzt ist auch das längst vorbei. Ich häute mich nur an Nase,
               Armen und Beinen wie ein Molch. Alles Gute für Sie, Frau \textcolor{blue}{Olga}{}\ledrightnote{\textcolor{blue}{Olga Schnitzler}} u. \textcolor{blue}{Heini}{}\ledrightnote{\textcolor{blue}{Heinrich Schnitzler}}!\pend
           
\pstart
           Viele herzliche Grüße von uns zu Ihnen {\\[\baselineskip]}Ihr \spacefill\mbox{Salten}\pend
           \leftskip=0em{}\endnumbering\briefempfaengerindex{Schnitzler, Arthur@\textsc{Schnitzler, Arthur}!zzzSalten, Felix@\emph{von Felix Salten}!1909-06-291@{29. 6. 1909}|)be}\mylabel{h}  \normalsize

\doendnotes{C}
\bigskip
\vfill

\clearpage

\footnotesize

\lohead{\textsc{register}}

% Definiere theindex-Environment komplett neu ohne reledmac
\makeatletter
\renewenvironment{theindex}{%
  \section*{\indexname}%
  \setlength{\parindent}{0pt}%
  \setlength{\parskip}{0pt plus 0.3pt}%
  \let\item\@idxitem
}{%
  \clearpage
}
\makeatother

\IfFileExists{\jobname-pw.ind}{\input{\jobname-pw.ind}}{}

\end{document}

      