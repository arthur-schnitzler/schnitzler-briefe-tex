%% latex-korrekturansicht-vorspann.tex
%% Vorspann für die Korrekturansicht.
%% Lädt die gemeinsame Datei latex-vorspann.tex mit gesetztem Schalter.

\newif\ifkorrekturansicht
\korrekturansichttrue

\input{../tex-inputs/latex-vorspann}


\renewcommand{\erwaehntePersonen}{Personen: Clementine Goldmann}
\renewcommand{\erwaehnteOrte}{Orte: Berlin, Bethmannstraße, Central-Hotel, Deutsches Theater Berlin, Frankfurt am Main}
\renewcommand{\erwaehnteWerke}{Werke: Lebendige Stunden. Vier Einakter}
\section[ Paul Goldmann an Arthur Schnitzler, 29. 12. {[}1901{]}]{Paul Goldmann an Arthur Schnitzler, 29. 12. {[}1901{]}}
\nopagebreak\mylabel{v}
\rehead{ }\normalsize\beginnumbering\briefempfaengerindex{Schnitzler, Arthur@\textsc{Schnitzler, Arthur}!zzzGoldmann, Paul@\emph{von Paul Goldmann}!1901-12-291@{29. 12. {[}1901{]}}|(be}
\toendnotes[C]{\smallbreak\pagebreak[2]}\Standort{DLA, A:Schnitzler, HS.NZ85.1.3171.}
\physDesc{Brief, 1 Blatt, 1 Seite
\newline{}Handschrift: blaue Tinte, deutsche Kurrent
\newline{}Schnitzler: mit Bleistift das Jahr »{[}1{]}901.« vermerkt }\toendnotes[C]{\smallbreak}
\pstart
           {\pb}\textcolor{pink}{Frankfurt}{}\ledrightnote{\textcolor{pink}{Frankfurt am Main}}, 29. Dezember.\pend
           
\pstart{}Mein lieber Freund,\pend
\pstart
           Zu Deinem \label{K_L03098-1v}\edtext{Eintreffen in \textcolor{pink}{Berlin}{}\ledrightnote{\textcolor{pink}{Berlin}}}{\lemma{\textnormal{\emph{Eintreffen in Berlin}}}\Cendnote{\textnormal{\textcolor{blue}{Schnitzler} war seit dem Vortag, dem 28. 12. 1901 in \textcolor{pink}{Berlin} und blieb bis zum 6. 1. 1902.}}}\label{K_L03098-1h}
               wünſche ich Dir alles gute Glück.\pend
           
\pstart
           Bitte, ſchreib’ mir gleich (Adreſſe: \textsc{\textcolor{pink}{Hotel Central}{}\ledrightnote{\textcolor{pink}{Central-Hotel}}}, \textsc{\textcolor{pink}{Bethmannstraße}{}\ledrightnote{\textcolor{pink}{Bethmannstraße}}}), wie es auf den \label{K_L03098-2v}\edtext{\textcolor{green}{Proben}{}\ledrightnote{\textcolor{green}{Lebendige Stunden. Vier Einakter}}}{\lemma{\textnormal{\emph{Proben}}}\Cendnote{\textnormal{siehe A. S.: \emph{Tagebuch}, 28. 12. 1901, A. S.: \emph{Tagebuch}, 3. 1. 1902 und Paul Goldmann an Arthur Schnitzler, 31. 12. [1901]}}}\label{K_L03098-2h} geht.\pend
           
\pstart
           Ich werde \label{K_L03098-11v}\edtext{Samſtag{ }früh von hier wegfahren, um zu Deiner \textsc{\textcolor{green}{Première}{}\ledrightnote{\textcolor{green}{Lebendige Stunden. Vier Einakter}}} in \textsc{\textcolor{pink}{Berlin}{}\ledrightnote{\textcolor{pink}{Berlin}}}}{\lemma{\textnormal{\emph{Samſtag … Berlin}}}\Cendnote{\textnormal{Am Samstag, dem 4. 1. 1902, fand am
                     \textcolor{pink}{Deutschen Theater Berlin} die Uraufführung der
                  vier Einakter \emph{\textcolor{green}{Lebendige Stunden}} statt.}}}\label{K_L03098-11h}
               zu ſein.\pend
           
\pstart
           Bitte, ſorge dafür, daß ich in meiner Wohnung ein Billet vorfinde.\pend
           
\pstart
           Meine \textcolor{blue}{Mutter}{}\ledrightnote{{$\rightarrow$}\textcolor{blue}{Clementine Goldmann}} (die Dich
               grüßen läßt) iſt auch in \textcolor{pink}{Frankfurt}{}\ledrightnote{\textcolor{pink}{Frankfurt am Main}}.\pend
           
\pstart
           Es thut mir unendlich leid, daß Deine \strikeout{An} Anweſenheit
               in \textcolor{pink}{Berlin}{}\ledrightnote{\textcolor{pink}{Berlin}} gerade in die Zeit meiner Abweſenheit
               fällt.\pend
           
\pstart
           Viele treue Grüße! {\\[\baselineskip]}Dein {\\[\baselineskip]}\spacefill\mbox{Paul Goldmn}\pend
           \leftskip=0em{}\endnumbering\briefempfaengerindex{Schnitzler, Arthur@\textsc{Schnitzler, Arthur}!zzzGoldmann, Paul@\emph{von Paul Goldmann}!1901-12-291@{29. 12. {[}1901{]}}|)be}\mylabel{h}
\begin{anhang}
\end{anhang}\normalsize

\doendnotes{C}
\bigskip
\vfill

\clearpage

\footnotesize

\lohead{\textsc{register}}

% Definiere theindex-Environment komplett neu ohne reledmac
\makeatletter
\renewenvironment{theindex}{%
  \section*{\indexname}%
  \setlength{\parindent}{0pt}%
  \setlength{\parskip}{0pt plus 0.3pt}%
  \let\item\@idxitem
}{%
  \clearpage
}
\makeatother

\IfFileExists{\jobname-pw.ind}{\input{\jobname-pw.ind}}{}

\end{document}

      