%% latex-korrekturansicht-vorspann.tex
%% Vorspann für die Korrekturansicht.
%% Lädt die gemeinsame Datei latex-vorspann.tex mit gesetztem Schalter.

\newif\ifkorrekturansicht
\korrekturansichttrue

\input{../tex-inputs/latex-vorspann}


               \section[Paul Goldmann an Arthur Schnitzler, Paul Goldmann an Arthur Schnitzler, 7. 9. {[}1896{]}]{ Paul Goldmann an Arthur Schnitzler, 7. 9. {[}1896{]}}\nopagebreak\mylabel{v}\rehead{ }\normalsize\beginnumbering\briefempfaengerindex{Schnitzler, Arthur@\textsc{Schnitzler, Arthur}!zzzGoldmann, Paul@\emph{von Paul Goldmann}!1896-09-072@{7. 9. {[}1896{]}}|(be} \toendnotes[C]{\smallbreak\pagebreak[2]} \Standort{DLA, A:Schnitzler, HS.NZ85.1.3166.}
\physDesc{Brief, 3 Blätter, 12 Seiten
\newline{}Handschrift: schwarze Tinte, deutsche Kurrent
\newline{}Schnitzler: 1) mit Bleistift das Jahr »96« vermerkt 2) mit rotem Buntstift zwölf Unterstreichungen}\toendnotes[C]{\smallbreak}\pstart
           \noindent{}{\pb}\textcolor{gray}{\textbf{\textbf{\textcolor{brown}{Frankfurter Zeitung}{}\ledrightnote{\textcolor{brown}{Frankfurter Zeitung}}}}}\pend
           \pstart
           \textcolor{gray}{\textbf{(\textcolor{brown}{\begin{otherlanguage}{french}Gazette de Francfort\end{otherlanguage}}{}\ledrightnote{\textcolor{brown}{Frankfurter Zeitung}}).}}\pend
           \pstart
           \textcolor{gray}{\textbf{\textbf{\begin{otherlanguage}{french}Fondateur M.\end{otherlanguage}{ }\textcolor{blue}{L. Sonnemann}{}\ledrightnote{\textcolor{blue}{Leopold Sonnemann}}.}}}\pend
           \pstart
           \begin{otherlanguage}{french}\textcolor{gray}{\textbf{\textcolor{green}{Journal}{}\ledrightnote{→\textcolor{green}{Frankfurter Zeitung}} politique,
                        financier,}}\end{otherlanguage}\pend
           \pstart
           \begin{otherlanguage}{french}\textcolor{gray}{\textbf{commercial et littéraire.}}\end{otherlanguage}\pend
           \pstart
           \begin{otherlanguage}{french}\textcolor{gray}{\textbf{\textbf{Paraissant trois fois par jour.}}}\end{otherlanguage}\pend
           \pstart
           \begin{otherlanguage}{french}\textcolor{gray}{\textbf{\textbf{Bureau à \textcolor{pink}{Paris}{}\ledrightnote{\textcolor{pink}{Paris}}}}}\end{otherlanguage}\pend
           \pstart
           \begin{otherlanguage}{french}\textcolor{gray}{\textbf{\textbf{\textcolor{pink}{24. Rue Feydeau}{}\ledrightnote{\textcolor{pink}{rue Feydeau}}.}}}\end{otherlanguage}\hfill \textsc{\textcolor{pink}{Berlin}{}\ledrightnote{\textcolor{pink}{Berlin}}}, 7. September.\pend
           \pstart\center{}Mein lieber Freund,\pend\pstart
           Morgen, Dinſtag, \label{K_L02784-1v}\edtext{fahre ich heim}{\lemma{\textnormal{\emph{fahre ich heim}}}\Cendnote{\textnormal{\textcolor{blue}{Schnitzler} war bereits am 26. 8. 1896 von \textcolor{pink}{Berlin} über 
                     \textcolor{pink}{München} nach \textcolor{pink}{Wien} 
                     gereit, wo er am 29. 8. 1896 ankam.}}}\label{K_L02784-1h} (»heim«
               iſt gut!), und Dein lieber Brief iſt das letzte Angenehme, das mir hier
               widerfährt.\pend
           \pstart
           Ich freue mich, daß Du glücklich wieder in \textcolor{pink}{Wien}{}\ledrightnote{\textcolor{pink}{Wien}}
               biſt und dort Alles beim Rechten gefunden haſt.\pend
           \pstart
           \label{K_L02784-2v}\edtext{\textsc{\textcolor{blue}{Burckhardt}{}\ledrightnote{\textcolor{blue}{Max Eugen Burckhard}}}s Begeiſterung}{\lemma{\textnormal{\emph{Burckhardts Begeiſterung}}}\Cendnote{\textnormal{siehe A. S.: \emph{Tagebuch}, 4. 9. 1896}}}\label{K_L02784-2h} für Dein \textcolor{green}{Stück}{}\ledrightnote{→\textcolor{green}{Freiwild. Schauspiel in 3 Akten}} iſt ein
               weiteres gutes \textsc{omen}. Daß das \textcolor{green}{Werk}{}\ledrightnote{→\textcolor{green}{Freiwild. Schauspiel in 3 Akten}} den Theaterleuten ſo gefällt, iſt das
               ſtärkſte Zeugniß für die Theater-Wirkung, die man {\pb}davon erwarten kann. Warum \textcolor{blue}{B.}{}\ledrightnote{→\textcolor{blue}{Max Eugen Burckhard}} ſämmtliche noch überlebenden Perſonen des \textcolor{green}{Stück}{}\ledrightnote{→\textcolor{green}{Freiwild. Schauspiel in 3 Akten}}es \strikeout{\textcolor{gray}{von d\textcolor{gray}{×}\-\textcolor{gray}{×}}} umbringen will, iſt mir nicht recht begreiflich. Dieſe Abänderungs-Vorſchläge
               ſind ſehr komiſch. Da wüßte ich viel beſſere: \textsc{\textcolor{green}{Anna}{}\ledrightnote{→\textcolor{green}{Freiwild. Schauspiel in 3 Akten}}} ſoll den Kaſſierer \textsc{\textcolor{green}{Kohn}{}\ledrightnote{→\textcolor{green}{Freiwild. Schauspiel in 3 Akten}}} heirathen und \textsc{\textcolor{green}{Vogel}{}\ledrightnote{→\textcolor{green}{Freiwild. Schauspiel in 3 Akten}}} ſoll in dem Theater-Director ſeinen verloren geglaubten Vater wiederfinden{\dotsfive}\pend
           \pstart
           Die \label{K_L02784-5v}\edtext{Äußerung des allerhöchſten \textcolor{blue}{Herrn}{}\ledrightnote{→\textcolor{blue}{Franz Joseph I. von Österreich-Ungarn}} über »\textcolor{green}{Lielelei}{}\ledrightnote{\textcolor{green}{Liebelei. Schauspiel in drei Akten}}«}{\lemma{\textnormal{\emph{Äußerung … »Lielelei«}}}\Cendnote{\textnormal{siehe A. S.: \emph{Tagebuch}, 5. 9. 1896}}}\label{K_L02784-5h} iſt köſtlich. Ich hoffe, Seine \textcolor{blue}{Majeſtät}{}\ledrightnote{→\textcolor{blue}{Franz Joseph I. von Österreich-Ungarn}} verſteht vom Regieren mehr, wie von der Kunſt, {\pb}ſonſt müßte man mit großer Beſorgniß in die Zukunft
                  \textcolor{pink}{Öſterreich}{}\ledrightnote{\textcolor{pink}{Österreich}}s blicken. \label{K_L02784-3v}\edtext{\textsc{\textcolor{blue}{Mitterwurzer}{}\ledrightnote{\textcolor{blue}{Friedrich Mitterwurzer}}} iſt ſo der rechte \textcolor{blue}{Sau-Komödiant}{}\ledrightnote{→\textcolor{blue}{Friedrich Mitterwurzer}}}{\lemma{\textnormal{\emph{Mitterwurzer … Sau-Komödiant}}}\Cendnote{\textnormal{siehe A. S.: \emph{Tagebuch}, 5. 9. 1896}}}\label{K_L02784-3h}. Schreib’ \strikeout{ihn} ihm einmal eine Rolle, in der
               er Erfolg hat, und er wird Dich als das erſte Genie der Welt ausſchreien.\pend
           \pstart
           Von \textsc{\textcolor{blue}{Richard}{}\ledrightnote{\textcolor{blue}{Richard Beer-Hofmann}}} weiß ich Dir wenig zu jagen. Er muß ſchon \label{K_L02784-6v}\edtext{in \textsc{\textcolor{pink}{Baden}{}\ledrightnote{\textcolor{pink}{Kaiser-Franz-Ring}}}}{\lemma{\textnormal{\emph{in Baden}}}\Cendnote{\textnormal{siehe Richard Beer-Hofmann an Arthur Schnitzler, 5. 9. 1896}}}\label{K_L02784-6h} ſein. Während der letzten Tage ſeines Hierſeins war er nervös und verging
               ſich in unangenehmen Betrachtungen über die »guten Menſchen«. \textsc{\textcolor{blue}{Paula}{}\ledrightnote{\textcolor{blue}{Paula Beer-Hofmann}}} hat er {\pb}fortgeſchickt; ſie wollte natürlich
               zum Schluß durchaus noch dableiben weil ſie bei \label{K_L02784-7v}\edtext{\textsc{\textcolor{pink}{Hagenbeck}{}\ledrightnote{\textcolor{pink}{Hagenbecks Tierpark}}}}{\lemma{\textnormal{\emph{Hagenbeck}}}\Cendnote{\textnormal{\textcolor{pink}{Hamburg}er \textcolor{pink}{Tierpark}}}}\label{K_L02784-7h} ſo ſchöne Affen und Raubthiere geſehen hatte.\pend
           \pstart
           Was mich anlangt, ſo ſind mir die Tage in \textcolor{pink}{Berlin}{}\ledrightnote{\textcolor{pink}{Berlin}}
               recht angenehm verfloſſen. Der liebſte unter den Menſchen, die ich hier kennen
               gelernt, iſt mir Dr. \textsc{\textcolor{blue}{Bie}{}\ledrightnote{\textcolor{blue}{Oskar Bie}}}. Er iſt ehrlich und gut. Wir verſtehen uns und haben uns wohl auch gern. \textsc{\textcolor{blue}{Kerr}{}\ledrightnote{\textcolor{blue}{Alfred Kerr}}} mag ich weniger. Ich wittere in ihm {\pb}den
                  \label{K_L02784-8v}\edtext{\begin{otherlanguage}{french}\textsc{froid ambitieux}\end{otherlanguage}}{\lemma{\textnormal{\emph{froid ambitieux}}}\Cendnote{\textnormal{französisch: kühler Ehrgeizling}}}\label{K_L02784-8h}.
               Mit \textsc{\textcolor{blue}{Brahm}{}\ledrightnote{\textcolor{blue}{Otto Brahm}}}, \textsc{\textcolor{blue}{Rittner}{}\ledrightnote{\textcolor{blue}{Rudolf Rittner}}} und \textsc{\textcolor{blue}{Richard}{}\ledrightnote{\textcolor{blue}{Richard Beer-Hofmann}}} verbrachte ich einen Abend. \textsc{\textcolor{blue}{Rittner}{}\ledrightnote{\textcolor{blue}{Rudolf Rittner}}} gefiel auch mir ausnehmend. \textsc{\textcolor{blue}{Brahm}{}\ledrightnote{\textcolor{blue}{Otto Brahm}}} forderte mich auf, ihm noch einmal Rendezvous für einen Abend zu geben. Ich
               hab’ es aber nicht gethan; ich glaub’ nicht, daß ihn irgend etwas an mir liegt. \textsc{\textcolor{blue}{Fischer}{}\ledrightnote{\textcolor{blue}{Samuel Fischer}}} hat ſofort \strikeout{\textcolor{gray}{×}} in mir einen ausutzbaren Mann geſehen, hat \strikeout{mich} ſich von mir einige Stunden über \textsc{\textcolor{pink}{Paris}{}\ledrightnote{\textcolor{pink}{Paris}}} erzählen {\pb}laſſen, hat mich auch zum Abendeſſen
               geladen. \strikeout{Das} Die Herausgabe der Humoriſten hat er
               natürlich abgelehnt. Hingegen wird ſeine \textcolor{blue}{Frau}{}\ledrightnote{→\textcolor{blue}{Hedwig Fischer}} wohl einen oder den anderen von dieſen Leuten jetzt
               überſetzen, angeregt durch die Lectüre meiner \label{K_L02784-9v}\edtext{Feuilletons}{\lemma{\textnormal{\emph{Feuilletons}}}\Cendnote{\textnormal{\textcolor{blue}{Goldmann} hat in seiner Feuilletonreihe
                  »Neue französische Humoristen« in der \emph{\textcolor{green}{Frankfurter
                     Zeitung}} verschiedene Literaturschaffende vorgestellt, jeweils mit einer
                  kurzen Einleitung und einer kleinen Übersetzung. Während die ersten Beiträge
                  nachgewiesen wurde, muss offen bleiben, wie viele Beiträge in Folge an die
                  angeführten noch erschienen sind. \emph{\textcolor{green}{Alphonse Allais}}, 3. 9. 1893; \emph{\textcolor{green}{Georges Courteline}},
                        31. 12. 1893 und 1. 1. 1894; \emph{\textcolor{green}{L. Xanrof}}, 25. 3. 1894, \emph{\textcolor{green}{Pierre Veber}}, 11. 5. 1894
                     und 13. 5. 1894; \emph{\textcolor{green}{Narcisse Lebeau}}, 5. 10. 1894; \emph{\textcolor{green}{Tristan Bernard. – Georges Auriol. – Bill
                        Sharp. – Maurice O’Reilly}}, 14. 4. 1894 und
                        17. 4. 1894. Übersetzungen von diesen \textcolor{blue}{Autoren} durch \textcolor{blue}{Hedwig
                     Fischer} konnten nicht nachgewiesen werden.}}}\label{K_L02784-9h}! Das mindert nicht den
               Freundſchaftsdienſt, den Du mir haſt leiſten wollen, und ich danke Dir von ganzem
               Herzen dafür. Die \label{K_L02784-11v}\edtext{Zeichnung von \textsc{\textcolor{blue}{Forain}{}\ledrightnote{\textcolor{blue}{Jean-Louis Forain}}}}{\lemma{\textnormal{\emph{Zeichnung von Forain}}}\Cendnote{\textnormal{nicht ermittelt}}}\label{K_L02784-11h}{ }{\pb}konnte ich ihm nicht zeigen. Ich habe ſie dem \textsc{\textcolor{blue}{Richard }{}\ledrightnote{\textcolor{blue}{Richard Beer-Hofmann}}} für Dich mitgegeben. Derſelbe hat auch Deinen \textsc{\textcolor{blue}{\textcolor{green}{Altenberg}{}\ledrightnote{→\textcolor{green}{Wie ich es sehe}}}{}\ledrightnote{\textcolor{blue}{Peter Altenberg}}}. Sag’ ihm, bitte, daß ich ihm \label{K_L02784-12v}\edtext{den \textsc{\textcolor{blue}{Gregorovius}{}\ledrightnote{\textcolor{blue}{Ferdinand Gregorovius}}}}{\lemma{\textnormal{\emph{den Gregorovius}}}\Cendnote{\textnormal{nicht ermittelt}}}\label{K_L02784-12h} ſofort nach
               meiner Ankunft in \textsc{\textcolor{pink}{Paris}{}\ledrightnote{\textcolor{pink}{Paris}}} ſchicken werde. Ich habe \strikeout{di\textcolor{gray}{e}} den Brief mit ſeiner \textcolor{pink}{Baden}{}\ledrightnote{\textcolor{pink}{Baden bei Wien}}er Adreſſe
               verloren, und auch ſeine \textcolor{pink}{Wien}{}\ledrightnote{\textcolor{pink}{Wien}}er Adreſſe finde ich
               erſt in \textsc{\textcolor{pink}{Paris}{}\ledrightnote{\textcolor{pink}{Paris}}}.\pend
           \pstart
           Sonſt hat mir \textsc{\textcolor{pink}{Berlin}{}\ledrightnote{\textcolor{pink}{Berlin}}} beſſer gefallen, als ich erwartet. Aber lieb {\pb}gewinnen könnte ich die \textcolor{pink}{Stadt}{}\ledrightnote{→\textcolor{pink}{Berlin}}
               wohl nicht. Im Großen und Ganzen macht ſie den Eindruck\strikeout{\textcolor{gray}{,}} einer raſch und billig hergeſtellten Großſtadt. Aber überall fehlt Cultur\textcolor{gray}{\textcolor{gray}{×}} und Schönheit. Immerhin iſt Vieles impoſant; und die Leute ſitzen da und
               hören Einem \strikeout{zu, oh} ſogar zu, als ahnten ſie, daß es
               noch etwas jenſeits ihres Horizontes gibt – was mich überraſcht hat. Freilich das
               ſind {\pb}doch wohl flüchtige und vielleicht falſche
               Eindrücke.\pend
           \pstart
           Meine arme \textcolor{blue}{Mama}{}\ledrightnote{→\textcolor{blue}{Clementine Goldmann}} iſt geſtern unter vielen Thränen nach \textcolor{pink}{Frankfurkt}{}\ledrightnote{\textcolor{pink}{Frankfurt am Main}} gefahren. Was daraus werden ſoll, weiß ich nicht.
               Einſtweilen muß ich meine Monatsrate erhöhen. Ich kanns natürlich nicht, aber ich muß
               es.\pend
           \pstart
           Mir grauſt vor \textsc{\textcolor{pink}{Paris}{}\ledrightnote{\textcolor{pink}{Paris}}} – das heißt vor der Arbeit, die \strikeout{ich} mich {\pb}dort erwartet und auch an dieſer Arbeit iſt nur
               ſchrecklich, daß ſie ſo ganz vergeblich iſt. Ich ſehe es \strikeout{\textcolor{gray}{×}} klarer wie je: Alles, was ich dort arbeite, kommt nur meinem \textcolor{blue}{Chef}{}\ledrightnote{→\textcolor{blue}{Leopold Sonnemann}} zu gute, nicht mir. All’ dieſe
               Rieſen-Anſtrengung da drüben zählt nicht, und ich müßte \strikeout{\textcolor{gray}{×}\-\textcolor{gray}{×}\-\textcolor{gray}{×}\textcolor{gray}{h}} noch nach dem ermüdenden Arbeitstage Zeit und Kraft finden, um das Eigentliche
               zu arbeiten, das erſt zählen würde. Unter {\pb}dieſen
               Umſtänden muß man müde und muthlos werden.\pend
           \pstart
           Grüß’ Dich Gott, mein lieber Arthur, und hab’ Dank für Deine Treue und Freundſchaft
               und für die ſchönen \label{K_L02784-13v}\edtext{Tage von \textsc{\textcolor{pink}{Skodsborg}{}\ledrightnote{\textcolor{pink}{Skodsborg}}}}{\lemma{\textnormal{\emph{Tage von Skodsborg}}}\Cendnote{\textnormal{Nachdem \textcolor{blue}{Goldmann} von \textcolor{blue}{Schnitzler}, \textcolor{blue}{Richard Beer-Hofmann}
                  und vermutlich auch \textcolor{blue}{Paula Beer-Hofmann} am
                     5. 8. 1896 in \textcolor{pink}{Kopenhagen} abgeholt wurde (vgl. A. S.: \emph{Tagebuch}, 8. 8. 1896), dürfte er bis um den 20. 8. 1896 mit ihnen in \textcolor{pink}{Skodsborg} gewesen sein. Am 21. 8. 1896 war er jedenfalls, wenn auch
                  womöglich nur für einen Tag, wieder in \textcolor{pink}{Kopenhagen}, zu Besuch bei \textcolor{blue}{Peter
                     und Betty Nansen}.}}}\label{K_L02784-13h} (nicht wahr, ſie waren ſchön?)\pend
           \pstart
           Empfiehl’ mich Deiner Frau \textcolor{blue}{Mutter}{}\ledrightnote{→\textcolor{blue}{Louise Schnitzler}}, deinem \textcolor{blue}{Bruder}{}\ledrightnote{→\textcolor{blue}{Julius Schnitzler}}, deiner \textcolor{blue}{Schwägerin}{}\ledrightnote{→\textcolor{blue}{Helene Schnitzler}}, Deine\substVorne{}\textsuperscript{\textcolor{gray}{m}}\substDazwischen{}r\substHinten{}{ }\textcolor{blue}{Schweſter}{}\ledrightnote{→\textcolor{blue}{Gisela Hajek}} und {\pb}Deinem \textcolor{blue}{Schwager}{}\ledrightnote{→\textcolor{blue}{Markus Hajek}}.\pend
           \pstart
           Empfiehl’ mich auch der \textcolor{blue}{Dame}{}\ledrightnote{→\textcolor{blue}{Marie Reinhard}}, die mir den \textsc{\textcolor{blue}{\textcolor{green}{Altenberg}{}\ledrightnote{→\textcolor{green}{Wie ich es sehe}}}{}\ledrightnote{\textcolor{blue}{Peter Altenberg}}} überſandt hat.\pend
           \pstart
           In Treue {\\[\baselineskip]}Dein {\\[\baselineskip]}\spacefill\mbox{Paul Goldmann}\pend
           \leftskip=0em{}\pstart
           \noindent{}Schreib’ mir bald nach \textsc{\textcolor{pink}{Paris}{}\ledrightnote{\textcolor{pink}{Paris}}}.\pend
           \pstart
           Wann gehſt Du \label{K_L02784-14v}\edtext{nach \textsc{\textcolor{pink}{Berlin}{}\ledrightnote{\textcolor{pink}{Berlin}}}}{\lemma{\textnormal{\emph{nach Berlin}}}\Cendnote{\textnormal{\textcolor{blue}{Schnitzler} war bereits von 22. 8. 1896 bis
                        26. 8. 1896 in
                        \textcolor{pink}{Berlin}. Das nächste Mal war er dort
                     zwischen 26. 10. 1896 und 9. 11. 1896.}}}\label{K_L02784-14h}?\pend
           \endnumbering\briefempfaengerindex{Schnitzler, Arthur@\textsc{Schnitzler, Arthur}!zzzGoldmann, Paul@\emph{von Paul Goldmann}!1896-09-072@{7. 9. {[}1896{]}}|)be}\mylabel{h}\begin{anhang}\end{anhang}\normalsize

\doendnotes{C}
\bigskip
\vfill

\clearpage

\footnotesize

\lohead{\textsc{register}}

% Definiere theindex-Environment komplett neu ohne reledmac
\makeatletter
\renewenvironment{theindex}{%
  \section*{\indexname}%
  \setlength{\parindent}{0pt}%
  \setlength{\parskip}{0pt plus 0.3pt}%
  \let\item\@idxitem
}{%
  \clearpage
}
\makeatother

\IfFileExists{\jobname-pw.ind}{\input{\jobname-pw.ind}}{}

\end{document}

      