%% latex-korrekturansicht-vorspann.tex
%% Vorspann für die Korrekturansicht.
%% Lädt die gemeinsame Datei latex-vorspann.tex mit gesetztem Schalter.

\newif\ifkorrekturansicht
\korrekturansichttrue

\input{../tex-inputs/latex-vorspann}


\section[Stefan Zweig an Arthur Schnitzler, {[}27. 10. 1912?{]}]{L03640 Stefan Zweig an Arthur Schnitzler, {[}27. 10. 1912?{]}}
\nopagebreak\mylabel{L03640v}
\rehead{ }\normalsize\beginnumbering\briefempfaengerindex{Schnitzler, Arthur@\textsc{Schnitzler, Arthur}!zzzZweig, Stefan@\emph{von Stefan Zweig}!1912-10-272@{27. 10. 1912}|(be}
\toendnotes[C]{\smallbreak\pagebreak[2]}\Standort{CUL, Schnitzler, B 118.}
\physDesc{Briefkarte, 1 Blatt, 2 Seiten, 821 Zeichen
\newline{}Handschrift: blaue Tinte, lateinische Kurrent
\newline{}Schnitzler: mit Bleistift »\textsc{Zweig}« }
\buchAbdrucke{\weitereDrucke{1) Stefan Zweig: \emph{Briefwechsel mit Hermann Bahr, Sigmund Freud, Rainer Maria
                        Rilke und Arthur Schnitzler}. Frankfurt am Main: \emph{S. Fischer} 1987, S. 369–370.} \weitereDrucke{2) Stefan Zweig: \emph{Briefe. Bd. I: 1897–1914}. Frankfurt am Main: \emph{S. Fischer} 1995, S. 264.} }\toendnotes[C]{\smallbreak}
\pstart
           {\pb}\textcolor{gray}{\textbf{SZ}}\hfill \textcolor{gray}{\textbf{\textcolor{pink}{VIII. KOCHGASSE 8}\oindex{Kochgasse 8@\textbf{Kochgasse 8}, \emph{Wohngebäude (K.WHS)}|pw}{}\ledrightnote{\textcolor{pink}{Kochgasse 8}}}}\pend
           
\pstart
           \raggedleft{}\textcolor{gray}{\textbf{\textcolor{pink}{WIEN}\oindex{Wien@\textbf{Wien}, \emph{A.ADM2}|pw}{}\ledrightnote{\textcolor{pink}{Wien}},}}\pend
           \vspace{0.5em}
\pstart
           Verehrter lieber Herr Doktor, empfangen Sie meinen innigsten Dank
               für Ihre guten Worte. Mir ist's mit allem nur um die Zustimmung der Besten zu tun und
                  \label{K_L03640-1v}\edtext{gestern}{\lemma{\textnormal{\emph{gestern}}}\Cendnote{\textnormal{Am
                     26. 10. 1912 wurde \textcolor{blue}{Zweigs}\pwindex{Zweig, Stefan 28.11.1881 – 23.02.1942@\textsc{Zweig, Stefan} (28.11.1881 – 23.02.1942), \emph{Schriftsteller/Schriftstellerin}|pwk}
                  Schauspiel \emph{\textcolor{green}{Das Haus am Meer}\pwindex{Haus am Meer. Ein Schauspiel in zwei Teilen (drei Aufzuegen)@\emph{Das Haus am Meer. Ein Schauspiel in zwei Teilen (drei Aufzügen)}|pwk}} am \textcolor{pink}{Wiener}\oindex{Wien@\textbf{Wien}, \emph{A.ADM2}|pwk}{ }\textcolor{pink}{Burgtheater}\oindex{Burgtheater@\textbf{Burgtheater}, \emph{S.THTR}|pwk} uraufgeführt. \textcolor{blue}{Zweigs}\pwindex{Zweig, Stefan 28.11.1881 – 23.02.1942@\textsc{Zweig, Stefan} (28.11.1881 – 23.02.1942), \emph{Schriftsteller/Schriftstellerin}|pwk} undatiertes Schreiben ist somit am 27. 10. 1912 abgefasst worden.}}}\label{K_L03640-1} hat mich bei der Aufführung nichts so
               beglückt, als ein spontanes \label{K_L03640-2v}\edtext{Telegramm}{\lemma{\textnormal{\emph{Telegramm}}}\Cendnote{\textnormal{Das Glückwunschtelegramm
                     \textcolor{blue}{Gehart Hauptmanns}\pwindex{Hauptmann, Gerhart 15.11.1862 – 06.06.1946@\textsc{Hauptmann, Gerhart} (15.11.1862 – 06.06.1946), \emph{Schriftsteller/Schriftstellerin}|pwk} und seine Freude
                  darüber während der Premiere hebt \textcolor{blue}{Zweig}\pwindex{Zweig, Stefan 28.11.1881 – 23.02.1942@\textsc{Zweig, Stefan} (28.11.1881 – 23.02.1942), \emph{Schriftsteller/Schriftstellerin}|pwk}
                  auch im Tagebucheintrag vom 26. 10. 1912 hervor (\emph{\textcolor{green}{Tagebuch September 1912 und Frühjahr 1913
                        (Paris)}\pwindex{Tagebuch September 1912 und Fruehjahr 1913 (Paris)@\emph{Tagebuch September 1912 und Frühjahr 1913 (Paris)}|pwk}}, SZ-AAP/L1. SZ-AAP/L1).}}}\label{K_L03640-2}{ }\textcolor{blue}{Gerhardt Hauptmanns}\pwindex{Hauptmann, Gerhart 15.11.1862 – 06.06.1946@\textsc{Hauptmann, Gerhart} (15.11.1862 – 06.06.1946), \emph{Schriftsteller/Schriftstellerin}|pw}{}\ledrightnote{\textcolor{blue}{Gerhart Hauptmann}}. Sie wissen ja, wie ich
               das Klaffende des \textcolor{green}{Stückes}\pwindex{Haus am Meer. Ein Schauspiel in zwei Teilen (drei Aufzuegen)@\emph{Das Haus am Meer. Ein Schauspiel in zwei Teilen (drei Aufzügen)}|pwv}{}\ledrightnote{{$\rightarrow$}\emph{\textcolor{green}{Das Haus am Meer. Ein Schauspiel in zwei Teilen (drei Aufzügen)}}}
               selber fühlte, aber ich durfte die Gelegenheit nicht vorübergehen lassen, ein mal an
               solcher Stelle zu erscheinen und ich habe – {\pb}das fühle ich – viel an den Erfahrungen
               und selbst der Kritik gelernt. Erhalten Sie mir, verehrter Herr Doktor, Ihre gute
               Gesinnung: sie ist mir wertvoller, als Sie vielleicht vermuten, und gibt, so
               freundlich sie auch nur sein mag, nur unvollkommen die Stärke des Gefühls zurück, das
               ich Ihnen von je – und Jahr um Jahr verstärkt – freudig entgegenbringe. In Verehrung
               getreut Ihr\pend
           \pstart \spacefill\mbox{Stefan Zweig}\pend{}\selectlanguage{ngerman}\endnumbering\briefempfaengerindex{Schnitzler, Arthur@\textsc{Schnitzler, Arthur}!zzzZweig, Stefan@\emph{von Stefan Zweig}!1912-10-272@{27. 10. 1912}|)be}\mylabel{L03640h}  \normalsize

\doendnotes{C}
\bigskip
\vfill

\clearpage

\footnotesize

\lohead{\textsc{register}}

% Definiere theindex-Environment komplett neu ohne reledmac
\makeatletter
\renewenvironment{theindex}{%
  \section*{\indexname}%
  \setlength{\parindent}{0pt}%
  \setlength{\parskip}{0pt plus 0.3pt}%
  \let\item\@idxitem
}{%
  \clearpage
}
\makeatother

\IfFileExists{\jobname-pw.ind}{\input{\jobname-pw.ind}}{}

\end{document}

      