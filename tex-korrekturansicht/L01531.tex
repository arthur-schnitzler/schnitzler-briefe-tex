%% latex-korrekturansicht-vorspann.tex
%% Vorspann für die Korrekturansicht.
%% Lädt die gemeinsame Datei latex-vorspann.tex mit gesetztem Schalter.

\newif\ifkorrekturansicht
\korrekturansichttrue

\input{../tex-inputs/latex-vorspann}


               \section[Arthur Schnitzler an Richard Beer-Hofmann, 17. 7. 1905]{ Arthur Schnitzler an Richard Beer-Hofmann, 17. 7. 1905}\nopagebreak\mylabel{v}\rehead{ }\normalsize\beginnumbering\briefempfaengerindex{Beer-Hofmann, Richard@\textsc{Beer-Hofmann, Richard}!zzzSchnitzler, Arthur@\emph{von Arthur Schnitzler}!1905-07-171@{17. 7. 1905}|(be} \toendnotes[C]{\smallbreak\pagebreak[2]} \Standort{YCGL, MSS 31.}
\physDesc{Kartenbrief
\newline{}Handschrift: schwarze Tinte, deutsche Kurrent\newline{}Versand: Stempel: »\nobreak{}\oindex{Reichenau an der Rax@\textbf{Reichenau an der Rax}, \emph{Besiedelter Ort (A.BSO)}|pwk}Reichenau bei Payerbach, 17 \textcolor{gray}{7} 0\textcolor{gray}{5}, 2–6N\nobreak{}«.  }\pstart{}{\pb}Herrn \textsc{Dr.
                  Richard}\pend{}\pstart{}\textsc{Beer-Hofmann}\pend{}\pstart{}\textsc{\textcolor{pink}{Rodaun{\\}bei Liesing}{}\ledrightnote{\textcolor{pink}{Rodaun}}}\pend{}\pstart{}\textcolor{pink}{\textsc{Liesingerstraße 2}}{}\ledrightnote{\textcolor{pink}{Liesingerstraße}}\pend{}{\bigskip}\pstart
           \raggedleft{}{\pb}\textcolor{pink}{\textsc{Reichenau, Kurhaus}}{}\ledrightnote{\textcolor{pink}{Kurhaus Rudolfsbad}}{\\}Am 17. 7. 905\pend
           \pstart
           lieber Richard, bitte theilen Sie uns mit, wie es \textcolor{blue}{Paula}{}\ledrightnote{\textcolor{blue}{Paula Beer-Hofmann}} geht. Was iſt de{\geminationn} eigentlich
               geſchehn? Und wie ſtehen Sie jetzt mit Ihren vielen So{\geminationm}erplänen?\pend
           \pstart
           Vielleicht ko{\geminationm}t man doch irgendwa{\geminationn} u irgendwo zuſammen? Wir bleiben noch 2–3 Wochen da,
               kleinere (2–3 tägige) Fußpartien meinerſeits abgerechnet\pend
           \pstart
           Herzlichſt Ihr{\\}\spacefill\mbox{Arthur}\pend
           \endnumbering\briefempfaengerindex{Beer-Hofmann, Richard@\textsc{Beer-Hofmann, Richard}!zzzSchnitzler, Arthur@\emph{von Arthur Schnitzler}!1905-07-171@{17. 7. 1905}|)be}\mylabel{h}  \normalsize

\doendnotes{C}
\bigskip
\vfill

\clearpage

\footnotesize

\lohead{\textsc{register}}

% Definiere theindex-Environment komplett neu ohne reledmac
\makeatletter
\renewenvironment{theindex}{%
  \section*{\indexname}%
  \setlength{\parindent}{0pt}%
  \setlength{\parskip}{0pt plus 0.3pt}%
  \let\item\@idxitem
}{%
  \clearpage
}
\makeatother

\IfFileExists{\jobname-pw.ind}{\input{\jobname-pw.ind}}{}

\end{document}

      