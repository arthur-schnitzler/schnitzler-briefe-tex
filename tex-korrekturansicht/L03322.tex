%% latex-korrekturansicht-vorspann.tex
%% Vorspann für die Korrekturansicht.
%% Lädt die gemeinsame Datei latex-vorspann.tex mit gesetztem Schalter.

\newif\ifkorrekturansicht
\korrekturansichttrue

\input{../tex-inputs/latex-vorspann}


\renewcommand{\erwaehntePersonen}{Personen: Siegfried Trebitsch}
\renewcommand{\erwaehnteOrte}{Orte: Deutsches Theater Berlin, Wien}
\renewcommand{\erwaehnteWerke}{Werke: Die Frau mit dem Dolche, Lebendige Stunden. Vier Einakter}
\section[ Felix Salten an Arthur Schnitzler, {[}12. 1. 1902{]}]{Felix Salten an Arthur Schnitzler, {[}12. 1. 1902{]}}
\nopagebreak\mylabel{v}
\rehead{ }\normalsize\beginnumbering\briefempfaengerindex{Schnitzler, Arthur@\textsc{Schnitzler, Arthur}!zzzSalten, Felix@\emph{von Felix Salten}!1902-01-121@{{[}12. 1. 1902{]}}|(be}
\toendnotes[C]{\smallbreak\pagebreak[2]}\Standort{CUL, Schnitzler, B 89, A 2.}
\physDesc{Brief, 1 Blatt, 1 Seite, 280 Zeichen
\newline{}Handschrift: Bleistift, lateinische Kurrent
\newline{}Schnitzler: mit Bleistift datiert: »12/1 902« 
\newline{}Ordnung: mit Bleistift von unbekannter Hand nummeriert: »146« }\toendnotes[C]{\smallbreak}
\pstart
           \raggedleft{}{\pb}Sonntag\pend
           
\pstart
           Lieber, danke herzlich für die \label{K_L03322-1v}\edtext{»\textcolor{green}{lebendigen
                  Stunden}{}\ledrightnote{\textcolor{green}{Lebendige Stunden. Vier Einakter}}«}{\lemma{\textnormal{\emph{»lebendigen
                  Stunden«}}}\Cendnote{\textnormal{siehe Arthur Schnitzler: Widmungsexemplar Lebendige Stunden für Felix
               Salten, [11.?] 1. 1902}}}\label{K_L03322-1h}, die ich eben bekam. Hörte von \textcolor{blue}{Trebitsch}{}\ledrightnote{\textcolor{blue}{Siegfried Trebitsch}}, dass Sie wieder in \textcolor{pink}{Wien}{}\ledrightnote{\textcolor{pink}{Wien}} sind.
               Ich habe mich sehr über den großen \label{K_L03322-2v}\edtext{Erfolg}{\lemma{\textnormal{\emph{Erfolg}}}\Cendnote{\textnormal{der Uraufführung des
                  Einakterzyklus’ \emph{\textcolor{green}{Lebendige Stunden}} (\textcolor{pink}{Deutsches Theater Berlin}, 4. 1. 1902)}}}\label{K_L03322-2h}
               gefreut, besonders darüber, dass die \label{K_L03322-3v}\edtext{»\textcolor{green}{Frau mit dem Dolch}{}\ledrightnote{\textcolor{green}{Die Frau mit dem Dolche}}« uns Recht gegeben}{\lemma{\textnormal{\emph{»Frau … gegeben}}}\Cendnote{\textnormal{\textcolor{blue}{Salten} dürfte seine Meinung zu diesem \textcolor{green}{Einakter} geändert haben,
                     vgl. A. S.: \emph{Tagebuch}, 4. 9. 1901.}}}\label{K_L03322-3h}.
               Hoffentlich \label{K_L03322-4v}\edtext{sehe ich Sie bald}{\lemma{\textnormal{\emph{sehe ich Sie bald}}}\Cendnote{\textnormal{Nachweislich sahen sich \textcolor{blue}{Salten} und \textcolor{blue}{Schnitzler}
                  am 26. 1. 1902
                  wieder.}}}\label{K_L03322-4h}.\pend
           
\pstart
           Ihr {\\[\baselineskip]}\spacefill\mbox{Salten}\pend
           \leftskip=0em{}\endnumbering\briefempfaengerindex{Schnitzler, Arthur@\textsc{Schnitzler, Arthur}!zzzSalten, Felix@\emph{von Felix Salten}!1902-01-121@{{[}12. 1. 1902{]}}|)be}\mylabel{h}  \normalsize

\doendnotes{C}
\bigskip
\vfill

\clearpage

\footnotesize

\lohead{\textsc{register}}

% Definiere theindex-Environment komplett neu ohne reledmac
\makeatletter
\renewenvironment{theindex}{%
  \section*{\indexname}%
  \setlength{\parindent}{0pt}%
  \setlength{\parskip}{0pt plus 0.3pt}%
  \let\item\@idxitem
}{%
  \clearpage
}
\makeatother

\IfFileExists{\jobname-pw.ind}{\input{\jobname-pw.ind}}{}

\end{document}

      