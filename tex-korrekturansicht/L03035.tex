%% latex-korrekturansicht-vorspann.tex
%% Vorspann für die Korrekturansicht.
%% Lädt die gemeinsame Datei latex-vorspann.tex mit gesetztem Schalter.

\newif\ifkorrekturansicht
\korrekturansichttrue

\input{../tex-inputs/latex-vorspann}


\renewcommand{\erwaehntePersonen}{Personen: Josef Kaufmann, Marie Reinhard, Felix Salten}
\renewcommand{\erwaehnteInstitutionen}{Institutionen: Jung-Wiener Theater zum Lieben Augustin}
\renewcommand{\erwaehnteOrte}{Orte: Raimund-Theater, Wien}
\renewcommand{\erwaehnteWerke}{Werke: Tagebuch}
\section[ Arthur Schnitzler an Felix Salten, {[}vor dem 16. 5. 1896?{]}]{Arthur Schnitzler an Felix Salten, {[}vor dem 16. 5. 1896?{]}}
\nopagebreak\mylabel{v}
\rehead{ }\normalsize\beginnumbering\briefempfaengerindex{Salten, Felix@\textsc{Salten, Felix}!zzzSchnitzler, Arthur@\emph{von Arthur Schnitzler}!1@{{[}vor dem
                  16. 5. 1896?{]}}|(be}
\toendnotes[C]{\smallbreak\pagebreak[2]}\Standort{Wienbibliothek im Rathaus, ZPH 1681, 2.1.516.}
\physDesc{Brief, 1 Blatt, 2 Seiten, 374 Zeichen
\newline{}Handschrift: Bleistift, deutsche Kurrent
\newline{}Ordnung: mit Bleistift von unbekannter Hand Nummerierung der Blätter des Konvoluts:
                                    »2« }\toendnotes[C]{\smallbreak}
\pstart
           \noindent{}{\pb}lieber, wenn es Ihnen alſo keine Umſtände macht, bitte
               ſehr, laſſen Sie mir folgendes für den \label{K_L03035-1v}\edtext{16.}{\lemma{\textnormal{\emph{16.}}}\Cendnote{\textnormal{Das Korrespondenzstück ist undatiert. Vier enthaltene
                  Details geben Hinweise für eine mögliche Datierung. 1.) Es werden (Theater-)Karten für eine
                  Aufführung am 16. eines Monats erbeten. 2.) Dieser Tag ist ein Samstag. 3.) Das 
                  Theater enthält eine »2. Gallerie«. 4.) \textcolor{blue}{Schnitzler}
                  möchte mit jemandem ins Theater gehen, ohne gemeinsam aufzutreten. Die letzten beiden Punkte 
                  machen es unwahrscheinlich,
                  dass die am 16. 11. 1901 stattfindende
                  Premiere des von \textcolor{blue}{Salten} geleiteten Kabaretts \emph{\textcolor{brown}{Jung-Wiener Theater zum Lieben Augustin}}
                  gemeint ist. (Auch besuchte \textcolor{blue}{Schnitzler} die Generalprobe, so dass
                  potentiell auch dies hier mitdiskutiert werden müsste). Berücksichtigt man ausschließlich Theater, die
                  über mehrere Gallerien verfügen, so reduziert sich die Zahl möglicher Termine stark. Eine Anwesenheit von
                  \textcolor{blue}{Salten} kann nur zu einem der in Frage kommenden Termine belegt werden
                  (16. 10. 1897), doch spricht die Erwähnung
                  mehrerer anderer Kollegen in und nach dem Theater im \emph{\textcolor{green}{Tagebuch}}-Eintrag 
                  dagegen. So bleibt \textcolor{blue}{Schnitzler}s Besuch
                  im \textcolor{pink}{Raimund-Theater} am 16. 5. 1896.
                  Einer der vier Sitze in der anderen Reihe wäre dann für \textcolor{blue}{Schnitzler}s Partnerin \textcolor{blue}{Marie Reinhard}
                  gedacht. Ein weiterer für \textcolor{blue}{Josef Kaufmann}. 
                  Für wen der vierte Sitz gedacht ist, bleibt offen.}}}\label{K_L03035-1h} reſerviren\pend
           
\pstart
           \uline{2. Gallerie}, 1. Reihe\pend
           
\pstart
           \textcolor{gray}{W}e{\geminationn} irgend möglich Mittelgang
               Ecke 2 Sitze und \introOben{}(etwa)\introOben{} gleich dahinter 2. Reihe – noch 2,
               alſo {\pb}im ganzen 4 Sitze.\pend
           
\pstart
           Vielleicht ſtecken Sie die Sitze zu ſich? oder ſchicken Sie mir? oder ich hol ſie ab?
               oder Sie bringen ſie mir Samſtag –?\pend
           
\pstart
           Herzlichſt Ihr {\\[\baselineskip]}\spacefill\mbox{Arthur}\pend
           \leftskip=0em{}\endnumbering\briefempfaengerindex{Salten, Felix@\textsc{Salten, Felix}!zzzSchnitzler, Arthur@\emph{von Arthur Schnitzler}!1896-05-141@{{[}vor dem
                  16. 5. 1896?{]}}|)be}\mylabel{h}  \normalsize

\doendnotes{C}
\bigskip
\vfill

\clearpage

\footnotesize

\lohead{\textsc{register}}

% Definiere theindex-Environment komplett neu ohne reledmac
\makeatletter
\renewenvironment{theindex}{%
  \section*{\indexname}%
  \setlength{\parindent}{0pt}%
  \setlength{\parskip}{0pt plus 0.3pt}%
  \let\item\@idxitem
}{%
  \clearpage
}
\makeatother

\IfFileExists{\jobname-pw.ind}{\input{\jobname-pw.ind}}{}

\end{document}

      