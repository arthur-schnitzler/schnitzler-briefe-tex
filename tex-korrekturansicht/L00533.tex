%% latex-korrekturansicht-vorspann.tex
%% Vorspann für die Korrekturansicht.
%% Lädt die gemeinsame Datei latex-vorspann.tex mit gesetztem Schalter.

\newif\ifkorrekturansicht
\korrekturansichttrue

\input{../tex-inputs/latex-vorspann}


               \section[Arthur Schnitzler an Hermann Bahr, 7. 2. 1896]{ Arthur Schnitzler an Hermann Bahr, 7. 2. 1896}\nopagebreak\mylabel{v}\rehead{ }\normalsize\beginnumbering\briefempfaengerindex{Bahr, Hermann@\textsc{Bahr, Hermann}!zzzSchnitzler, Arthur@\emph{von Arthur Schnitzler}!1896-02-071@{7. 2. 1896}|(be} \toendnotes[C]{\smallbreak\pagebreak[2]} \Standort{TMW, HS AM 23325 Ba.}
\physDesc{Brief, 1 Blatt, 3 Seiten
\newline{}Handschrift: schwarze Tinte, deutsche Kurrent\newline{}Ordnung: Lochung }\buchAbdrucke{\weitereDrucke{1) \emph{7. 2. 1896.} In: Arthur Schnitzler: \emph{The Letters of Arthur Schnitzler to Hermann Bahr}. Edited, annotated, and with an introduction, by Donald G.
                        Daviau. Chapel Hill: \emph{The University of North Carolina Press} 1978, S. 58–59 (University of North Carolina studies in the Germanic languages
                        and literatures, 89).} \weitereDrucke{2) Hermann Bahr, Arthur Schnitzler: \emph{Briefwechsel, Aufzeichnungen, Dokumente (1891–1931)}. Hg. Kurt Ifkovits und Martin Anton Müller. Göttingen: \emph{Wallstein} 2018, S. 117.} }\toendnotes[C]{\smallbreak}\pstart{}{\pb}Lieber Hermann,\pend\pstart
           herzlichen Dank für deine freundlichen Glückwünſche.\pend
           \pstart
           Was dich intereſſieren wird: \label{K_L00533_1v}\edtext{\textcolor{green}{verriſſsen}{}\ledrightnote{→\textcolor{green}{Deutsches Theater}} hat mich nur einer,
               nemlich Herr \textcolor{blue}{Peſchkau}{}\ledrightnote{\textcolor{blue}{Emil Peschkau}} in den \textcolor{brown}{Berl. Neueſten Nachrichten}{}\ledrightnote{\textcolor{brown}{Berliner Neueste Nachrichten}}}{\lemma{\textnormal{\emph{verriſſsen … Nachrichten}}}\Cendnote{\textnormal{»›\textcolor{green}{Man dramatisirt Zustände, indem
                        man Menschen in sie bringt, die sich ihnen widersetzen; dort, wo sich die
                        Menschen mit den Dingen entzweien, fängt das Drama erst an. Aber seine
                        Menschen, die nichts wollen, sitzen unbeweglich in ihren Zuständen drin, wie
                        Chamäleons, die immer die Farbe ihrer Umgebung haben;}‹« (\textcolor{blue}{E. Peschkau}: \emph{\textcolor{green}{Deutsches Theater}}. In: \emph{\textcolor{green}{Berliner Neueste
                        Nachrichten}}, Jg. 16, Nr. 59, 5. 2. 1896, S. 2–3, hier:
                     S. 3).}}}\label{K_L00533_1h}, u weißt du, was er zu dieſem
               Behufe gethan hat? einfach \uline{wörtlich} citirt (mit
               Anführung der Quelle), was du über mich ſagſt und daraus zwingend bewieſen, daſs ich
               weder {\pb}ein Dramatiker noch ein
               Dichter bin, sondern daſs mir ſelbſt die Elementarkenntniſſe zu dieſen beiden ſchönen
               Stellungen fehlen. – \pend
           \pstart
           Sehr erfreulich waren mir Deine Mittheilungen über das \textcolor{green}{Märchen}{}\ledrightnote{\textcolor{green}{Das Märchen. Schauspiel in drei Aufzügen}} und \textcolor{blue}{Langka{\geminationm}ers}{}\ledrightnote{\textcolor{blue}{Karl Langkammer}} Urtheil. Aber ich habe wieder ſehr lebhafte
               Bedenken betreffs einer eventuellen Aufführung bekommen. Ich werde ja wohl bald
               Gelegenheit {[}haben{]}, ſowohl mit dir als mit \textcolor{blue}{Langkammer}{}\ledrightnote{\textcolor{blue}{Karl Langkammer}}{ }{\pb}darüber zu reden. Bis dahin
               beſte Grüße und nochmals vielen Dank.\pend
           \pstart Dein \spacefill\mbox{ArthSchn}\pend{}\pstart
           \textsc{\textcolor{pink}{Berlin}{}\ledrightnote{\textcolor{pink}{Berlin}}}{ }\substVorne{}\textsuperscript{6}\substDazwischen{}7\substHinten{}. 2. 96.\pend
           \endnumbering\briefempfaengerindex{Bahr, Hermann@\textsc{Bahr, Hermann}!zzzSchnitzler, Arthur@\emph{von Arthur Schnitzler}!1896-02-071@{7. 2. 1896}|)be}\mylabel{h}  \normalsize

\doendnotes{C}
\bigskip
\vfill

\clearpage

\footnotesize

\lohead{\textsc{register}}

% Definiere theindex-Environment komplett neu ohne reledmac
\makeatletter
\renewenvironment{theindex}{%
  \section*{\indexname}%
  \setlength{\parindent}{0pt}%
  \setlength{\parskip}{0pt plus 0.3pt}%
  \let\item\@idxitem
}{%
  \clearpage
}
\makeatother

\IfFileExists{\jobname-pw.ind}{\input{\jobname-pw.ind}}{}

\end{document}

      