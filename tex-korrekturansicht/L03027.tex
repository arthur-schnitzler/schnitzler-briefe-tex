%% latex-korrekturansicht-vorspann.tex
%% Vorspann für die Korrekturansicht.
%% Lädt die gemeinsame Datei latex-vorspann.tex mit gesetztem Schalter.

\newif\ifkorrekturansicht
\korrekturansichttrue

\input{../tex-inputs/latex-vorspann}


\renewcommand{\erwaehntePersonen}{Personen: Aura Hertwig, Ottilie Salten}
\renewcommand{\erwaehnteOrte}{Orte: Wien}
\renewcommand{\erwaehnteWerke}{Werke: Schnitzler [Griff an den Bart, 1905]}
\section[ Arthur Schnitzler an Ottilie Salten, 14. 1. 1906]{Arthur Schnitzler an Ottilie Salten, 14. 1. 1906}
\nopagebreak\mylabel{v}
\rehead{ }\normalsize\beginnumbering\briefempfaengerindex{Salten, Ottilie@\textsc{Salten, Ottilie}!zzzSchnitzler, Arthur@\emph{von Arthur Schnitzler}!1906-01-141@{14. 1. 1906}|(be}
\toendnotes[C]{\smallbreak\pagebreak[2]}\Standort{Wienbibliothek im Rathaus, Nachlass Salten, ZPH 1681, 17.3.11.11.40.6.}
\physDesc{Fotografie, 80 Zeichen
\newline{}Handschrift: schwarze Tinte, deutsche Kurrent
\newline{}Editorischer Hinweis: mit rotem Buntstift auf der Fotografie die handschriftliche Signatur »\textcolor{blue}{\textsc{Aura Hertwig}}« und »1905« }\toendnotes[C]{\smallbreak}\begin{figure}[H]\centering\includegraphics[width=10cm]{../tex-inputs/img/ZPH1681_Box_17_3_11_11_40_5_0001_1.jpg}\end{figure}
\pstart
           \noindent{}{\pb}\textcolor{green}{\textsc{Frau Otti Salten}{ }{\\} zur \label{K_L03027-1v}\edtext{freundlichen
                     Erinnerung}{\lemma{\textnormal{\emph{freundlichen
                     Erinnerung}}}\Cendnote{\textnormal{Es handelte sich um das
                     Abschiedstreffen, weil \textcolor{blue}{Salten} eine
                     Stellung beim \emph{\textcolor{brown}{Ullstein-Konzern}} in \textcolor{pink}{Berlin} übernahm, siehe Felix Salten an Arthur Schnitzler, 29. 1. 1906. »Abd. \textcolor{blue}{Salten’s} bei uns. […]
                        Photographien angeschaut, ihm und ihr etliche gegeben.― Auf meine schrieb
                        ich ihm ›M. l. \textcolor{blue}{F. S.} nach 15
                        Jahren für alle weitern in Freundschaft herzlichst
                        A. S.‹ ―« A. S.: \emph{Tagebuch}, 14. 1. 1906}}}\label{K_L03027-1h}}{}\ledrightnote{{$\rightarrow$}\textcolor{green}{Schnitzler [Griff an den Bart, 1905]}}\pend
           \pstart Herzlichſt \spacefill\mbox{Arth Schnitzler}\pend{}
\pstart
           14. 1. 906.\pend
           \endnumbering\briefempfaengerindex{Salten, Ottilie@\textsc{Salten, Ottilie}!zzzSchnitzler, Arthur@\emph{von Arthur Schnitzler}!1906-01-141@{14. 1. 1906}|)be}\mylabel{h}  \normalsize

\doendnotes{C}
\bigskip
\vfill

\clearpage

\footnotesize

\lohead{\textsc{register}}

% Definiere theindex-Environment komplett neu ohne reledmac
\makeatletter
\renewenvironment{theindex}{%
  \section*{\indexname}%
  \setlength{\parindent}{0pt}%
  \setlength{\parskip}{0pt plus 0.3pt}%
  \let\item\@idxitem
}{%
  \clearpage
}
\makeatother

\IfFileExists{\jobname-pw.ind}{\input{\jobname-pw.ind}}{}

\end{document}

      