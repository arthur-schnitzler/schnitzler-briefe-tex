%% latex-korrekturansicht-vorspann.tex
%% Vorspann für die Korrekturansicht.
%% Lädt die gemeinsame Datei latex-vorspann.tex mit gesetztem Schalter.

\newif\ifkorrekturansicht
\korrekturansichttrue

\input{../tex-inputs/latex-vorspann}


               \section[Paul Goldmann an Arthur Schnitzler, 27. 9. 1901]{ Paul Goldmann an Arthur Schnitzler, 27. 9. 1901}\nopagebreak\mylabel{v}\rehead{ }\normalsize\beginnumbering\briefempfaengerindex{Schnitzler, Arthur@\textsc{Schnitzler, Arthur}!zzzGoldmann, Paul@\emph{von Paul Goldmann}!1901-09-271@{27. 9. 1901}|(be} \toendnotes[C]{\smallbreak\pagebreak[2]} \Standort{DLA, A:Schnitzler, HS.NZ85.1.3171.}
\physDesc{Telegramm
\newline{}maschinell\newline{}Ordnung: beschnitten }\toendnotes[C]{\smallbreak}\pstart
           {\pb}de \textcolor{pink}{berlin}{}\ledrightnote{\textcolor{pink}{Berlin}}
                  44846 22 27 9 15m=\pend
           \pstart
           \textcolor{green}{lokalanzeiger}{}\ledrightnote{\textcolor{green}{Berliner Lokal-Anzeiger}}{ }\label{K_L02656-1v}\edtext{\textcolor{green}{meldet}{}\ledrightnote{→\textcolor{green}{Die neueste Verlobung in Berliner Theaterkreisen}}}{\lemma{\textnormal{\emph{meldet}}}\Cendnote{\textnormal{»\textcolor{green}{\textbf{Die neueste Verlobung in \textcolor{pink}{Berlin}er Theaterkreisen}, die sich in aller Stille vollzogen
                        hat, betrifft \textcolor{blue}{\so{Paul Martin}}, den Mitdirector des ›\textcolor{brown}{Neuen
                           Theater}s‹ und Fräulein \textcolor{blue}{\so{Marie Glümer}}. Die Vermählung soll schon in den nächsten Wochen stattfinden. Demnach
                        hat ›\textcolor{green}{Das Ewig-Weibliche}› bei Director
                           \textcolor{blue}{Martin} in dieser Saison bereits
                        einen zweiten Erfolg aufzuweisen.}« (\emph{\textcolor{green}{}}\emph{\textcolor{green}{Berliner Lokal-Anzeiger}}, Jg. 19, Nr. 453,
                        27. 9. 1901, Morgenblatt, 1. Ausgabe, S. 2.)}}}\label{K_L02656-1h}
               verlobung von fraeulein \textcolor{blue}{marie gluemer}{}\ledrightnote{\textcolor{blue}{Marie Glümer}} mit
               direktor \textcolor{blue}{martin}{}\ledrightnote{\textcolor{blue}{Paul Martin Marton}}. \label{K_L02656-2v}\edtext{weisst du etwas darueber}{\lemma{\textnormal{\emph{weisst du etwas darueber}}}\Cendnote{\textnormal{siehe A. S.: \emph{Tagebuch}, 27. 9. 1901}}}\label{K_L02656-2h}?\pend
           \pstart gruesse = \spacefill\mbox{goldmann .+}\pend{}\endnumbering\briefempfaengerindex{Schnitzler, Arthur@\textsc{Schnitzler, Arthur}!zzzGoldmann, Paul@\emph{von Paul Goldmann}!1901-09-271@{27. 9. 1901}|)be}\mylabel{h}  \normalsize

\doendnotes{C}
\bigskip
\vfill

\clearpage

\footnotesize

\lohead{\textsc{register}}

% Definiere theindex-Environment komplett neu ohne reledmac
\makeatletter
\renewenvironment{theindex}{%
  \section*{\indexname}%
  \setlength{\parindent}{0pt}%
  \setlength{\parskip}{0pt plus 0.3pt}%
  \let\item\@idxitem
}{%
  \clearpage
}
\makeatother

\IfFileExists{\jobname-pw.ind}{\input{\jobname-pw.ind}}{}

\end{document}

      