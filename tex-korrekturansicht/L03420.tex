%% latex-korrekturansicht-vorspann.tex
%% Vorspann für die Korrekturansicht.
%% Lädt die gemeinsame Datei latex-vorspann.tex mit gesetztem Schalter.

\newif\ifkorrekturansicht
\korrekturansichttrue

\input{../tex-inputs/latex-vorspann}


\renewcommand{\erwaehntePersonen}{Personen: Otto Brahm,  Matter, Emanuel Reicher, Felix Salten}
\renewcommand{\erwaehnteOrte}{Orte: Berlin, Charlottenburg, I., Innere Stadt, Wien}
\renewcommand{\erwaehnteWerke}{Werke: B.Z. am Mittag, Der einsame Weg. Schauspiel in fünf Akten, Theater. Der einsame Weg. – Othello. – Die Mitschuldigen. Der Tartüffe. – Der Fall Reinhardt}
\section[ Felix Salten an Arthur Schnitzler, 21. 4. {[}1906{]}]{Felix Salten an Arthur Schnitzler, 21. 4. {[}1906{]}}
\nopagebreak\mylabel{v}
\rehead{ }\normalsize\beginnumbering\briefempfaengerindex{Schnitzler, Arthur@\textsc{Schnitzler, Arthur}!zzzSalten, Felix@\emph{von Felix Salten}!1906-04-211@{21. 4. {[}1906{]}}|(be}
\toendnotes[C]{\smallbreak\pagebreak[2]}\Standort{CUL, Schnitzler, B 89, B 1.}
\physDesc{Telegramm, 2 Blätter, 2 Seiten, 405 Zeichen
\newline{}maschinell
\newline{}Versand: 1) mit Bleistift abgeschnittener Vermerk des Namens des für die Transkription
                                 verantwortlichen Postbeamten bzw. der Postbeamtin: »\textcolor{blue}{\textsc{\textcolor{gray}{M}atter}}«  2) Stempel: »\nobreak{}\oindex{I., Innere Stadt@\textbf{I., Innere Stadt}, \emph{A.ADM3}|pwk}{\pb}{[}Wi{]}\textcolor{gray}{e}n 1/1\nobreak{}«. Stempel: »\nobreak{}21 Apr, 5 \textsuperscript{40}\textsubscript{41}, Ausgefertigt\nobreak{}«. 
\newline{}Ordnung: mit Bleistift von unbekannter Hand nummeriert: »210a« }\toendnotes[C]{\smallbreak}
\pstart
           \centering{}{\pb},+ de \textcolor{pink}{charlottenburg}{}\ledrightnote{\textcolor{pink}{Charlottenburg}} 2454 61/60 21 4/25– s .=\pend
           
\pstart
           \textcolor{blue}{reicher}{}\ledrightnote{\textcolor{blue}{Emanuel Reicher}}{ }\label{K_L03420-1v}\edtext{\textcolor{green}{julian}{}\ledrightnote{{$\rightarrow$}\textcolor{green}{Der einsame Weg. Schauspiel in fünf Akten}} so vollstaendig vergriffen}{\lemma{\textnormal{\emph{julian … vergriffen}}}\Cendnote{\textnormal{Zur Wiederaufnahme von \emph{\textcolor{green}{Der
                     einsame Weg}}{ }siehe Felix Salten u. a. an Arthur Schnitzler, 19. 4. 1906. Während \textcolor{blue}{Brahm} am 19. 4. 1906 von einer »miserabeln Aufführung« schrieb,
                  dürfte dieses Telegramm \textcolor{blue}{Salten}s \textcolor{blue}{Schnitzler} zu einem Brief an \textcolor{blue}{Brahm} motiviert haben. Am
                     22. 4. 1906 antwortete \textcolor{blue}{Brahm} jedenfalls auf Kritik an der Besetzung von \textcolor{green}{Julian Fichtner} mit \textcolor{blue}{Emanuel Reicher}. vgl. \emph{Briefwechsel
                        Schnitzler/Brahm}, 225–226. In dieser Antwort geht \textcolor{blue}{Brahm} auch explizit auf \textcolor{blue}{Salten} ein: »So scheint mir das Raisonnabelste, den
                        \textcolor{green}{Julian} des \textcolor{blue}{Reicher}, der übrigens auch bessere und
                     feinere Momente hat und den nur ein in Extravaganzen und \textcolor{blue}{Salten}-Mortale geübter Kompetenter unerträglich und
                     höchst gefahrvoll finden wird – es scheint mir, daß wir versuchen müssen, den
                        \textcolor{blue}{Reicher} besser zu machen.«
                     (S. 226.)}}}\label{K_L03420-1h} und falsch ausserdem im \textcolor{green}{text}{}\ledrightnote{{$\rightarrow$}\textcolor{green}{Der einsame Weg. Schauspiel in fünf Akten}} so unsicher dass ich es vorzog \label{K_L03420-2v}\edtext{ueberhaupt nichts ueber reprise}{\lemma{\textnormal{\emph{ueberhaupt … reprise}}}\Cendnote{\textnormal{\textcolor{blue}{Felix Salten}: \emph{\textcolor{green}{Theater. Der einsame Weg. – Othello. – Die Mitschuldigen.
                        Der Tartüffe. – Der Fall Reinhardt}}. In: \emph{\textcolor{green}{B. Z. am Mittag}}, Jg. 30, Nr. 99, 28. 4. 1906, S. 2 u. 7. \textcolor{blue}{Salten} behandelt vor allem die Bedeutung, die \textcolor{pink}{Berlin}er Inszenierungen mittlerweile für die \textcolor{pink}{Wien}erinnen und \textcolor{pink}{Wien}er haben, um Bekanntschaft mit \textcolor{pink}{Wien}er Autoren auf der Bühne zu bekommen.}}}\label{K_L03420-2h} zu \textcolor{green}{schreiben}{}\ledrightnote{{$\rightarrow$}\textcolor{green}{Theater. Der einsame Weg. – Othello. – Die Mitschuldigen. Der Tartüffe. – Der Fall Reinhardt}} . halte einen anderen,
               vielleicht minder namhaften aber frischen schauspieler fuer {\pb}\textcolor{pink}{wien}{}\ledrightnote{\textcolor{pink}{Wien}} noch geeigneter als \textcolor{blue}{reicher}{}\ledrightnote{\textcolor{blue}{Emanuel Reicher}} der die \textcolor{green}{figur}{}\ledrightnote{{$\rightarrow$}\textcolor{green}{Der einsame Weg. Schauspiel in fünf Akten}} vom grund aus faelscht und viele schoenheiten der
               dichtung in wuesten umwandelt. herzlichst \spacefill\mbox{salten ,+}\pend
           \endnumbering\briefempfaengerindex{Schnitzler, Arthur@\textsc{Schnitzler, Arthur}!zzzSalten, Felix@\emph{von Felix Salten}!1906-04-211@{21. 4. {[}1906{]}}|)be}\mylabel{h}  \normalsize

\doendnotes{C}
\bigskip
\vfill

\clearpage

\footnotesize

\lohead{\textsc{register}}

% Definiere theindex-Environment komplett neu ohne reledmac
\makeatletter
\renewenvironment{theindex}{%
  \section*{\indexname}%
  \setlength{\parindent}{0pt}%
  \setlength{\parskip}{0pt plus 0.3pt}%
  \let\item\@idxitem
}{%
  \clearpage
}
\makeatother

\IfFileExists{\jobname-pw.ind}{\input{\jobname-pw.ind}}{}

\end{document}

      