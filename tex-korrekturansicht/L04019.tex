%% latex-korrekturansicht-vorspann.tex
%% Vorspann für die Korrekturansicht.
%% Lädt die gemeinsame Datei latex-vorspann.tex mit gesetztem Schalter.

\newif\ifkorrekturansicht
\korrekturansichttrue

\input{../tex-inputs/latex-vorspann}


\section[Arthur Schnitzler an Gustav Schwarzkopf, 7. 10. 1905]{L04019 Arthur Schnitzler an Gustav Schwarzkopf, 7. 10. 1905}
\nopagebreak\mylabel{L04019v}
\rehead{ }\normalsize\beginnumbering\briefempfaengerindex{Schwarzkopf, Gustav@\textsc{Schwarzkopf, Gustav}!zzzSchnitzler, Arthur@\emph{von Arthur Schnitzler}!1905-10-071@{7. 10. 1905}|(be}
\toendnotes[C]{\smallbreak\pagebreak[2]}
\correspDesc{Versand  durch Arthur Schnitzler am 7. 10. 1905 in Wien
\newline{}Erhalt  durch Gustav Schwarzkopf im Zeitraum [7. 10. 1905 – 10. 10. 1905?] in Wien}\toendnotes[C]{\smallbreak}
\Standort{CUL, Schnitzler, B 96.}
\physDesc{Postkarte, 316 Zeichen
\newline{}Handschrift: schwarze Tinte, deutsche Kurrent
\newline{}Versand: 1) Rohrpost  2) Stempel: »\nobreak{}\oindex{XVIII., Währing@\textbf{XVIII., Währing}, \emph{Verwaltungsgebiet}|pwk}Wien 18/\textcolor{gray}{1}
                                       110, 7 X {[}05{]}, 4\textsuperscript{10}N \nobreak{}«.  3) Stempel: »\nobreak{}\oindex{I., Innere Stadt@\textbf{I., Innere Stadt}, \emph{Verwaltungsgebiet}|pwk}Wien 1/1, 7 X 05, 4 30N\nobreak{}«. }\toendnotes[C]{\smallbreak}\pstart{}{\pb}Herrn Guſtav
                  Schwarzkopf\pend{}\pstart{}\textcolor{pink}{Wien I}\oindex{I., Innere Stadt@\textbf{I., Innere Stadt}, \emph{Verwaltungsgebiet}|pw}{}\ledrightnote{\textcolor{pink}{I., Innere Stadt}}\pend{}\pstart{}\textcolor{pink}{Tiefer Graben 23}\oindex{Wien@\textbf{Wien}!I., Innere Stadt@\textbf{I., Innere Stadt}!Tiefer Graben 23@\textbf{Tiefer Graben 23}, \emph{Wohngebäude}|pw}{}\ledrightnote{\textcolor{pink}{Tiefer Graben 23}}.\pend{}{\bigskip}\vspace{1em}
\pstart
           \noindent{}{\pb}lieber Guſtav, da \textcolor{blue}{Mama}\pwindex{Schnitzler, Louise 8.\,7.\,1840 Kőszeg – 9.\,9.\,1911 Wien@\textsc{Schnitzler, Louise} (8.\,7.\,1840 Kőszeg – 9.\,9.\,1911 Wien)|pw}{}\ledrightnote{\textcolor{blue}{Louise Schnitzler}} nun in
               meine \textcolor{violet}{Loge}\eventindex{Burgtheater@\textbf{Burgtheater}!Uraufführung von Zwischenspiel, 12.10.1905@Uraufführung von Zwischenspiel, 12.10.1905|pwv}{}\ledrightnote{{$\rightarrow$}\emph{\textcolor{violet}{Uraufführung von Zwischenspiel, 12.10.1905}}}, und \textcolor{blue}{Hajeks}\pwindex{Hajek, Markus 25.\,11.\,1861 Vršac – 4.\,4.\,1941 London@\textsc{Hajek, Markus} (25.\,11.\,1861 Vršac – 4.\,4.\,1941 London), \emph{Mediziner, Laryngologe}|pw}\pwindex{Hajek, Gisela 20.\,12.\,1867 Wien – 3.\,2.\,1953 Cambridge@\textsc{Hajek, Gisela} (20.\,12.\,1867 Wien – 3.\,2.\,1953 Cambridge)|pw}{}\ledrightnote{\textcolor{blue}{Markus Hajek}{\newline}\textcolor{blue}{Gisela Hajek}} u. \textcolor{blue}{Schnitzlers}\pwindex{Schnitzler, Julius 13.\,7.\,1865 Wien – 29.\,6.\,1939 ebd.@\textsc{Schnitzler, Julius} (13.\,7.\,1865 Wien – 29.\,6.\,1939 ebd.), \emph{Chirurg}|pw}\pwindex{Schnitzler, Helene 16.\,7.\,1871 Budapest – September 1941 Atlantischer Ozean@\textsc{Schnitzler, Helene} (16.\,7.\,1871 Budapest – September 1941 Atlantischer Ozean)|pw}{}\ledrightnote{\textcolor{blue}{Julius Schnitzler}{\newline}\textcolor{blue}{Helene Schnitzler}} (\textcolor{blue}{Julius}\pwindex{Schnitzler, Julius 13.\,7.\,1865 Wien – 29.\,6.\,1939 ebd.@\textsc{Schnitzler, Julius} (13.\,7.\,1865 Wien – 29.\,6.\,1939 ebd.), \emph{Chirurg}|pw}{}\ledrightnote{\textcolor{blue}{Julius Schnitzler}}) vermut in \uline{eine} andere Loge gehn,
               muſs ich heute verzichten, Sie in der Autorenloge zu ſehn\footnote{\noindent{}zu beherbergen mein ich, – ſeh’n hoffentlich
                  dort.}, und frage an, ob ich Ihnen und Dr. \textcolor{blue}{Max}\pwindex{Schwarzkopf, Max 12.\,6.\,1857 Wien – 14.\,4.\,1928 ebd.@\textsc{Schwarzkopf, Max} (12.\,6.\,1857 Wien – 14.\,4.\,1928 ebd.), \emph{Rechtsanwalt}|pw}{}\ledrightnote{\textcolor{blue}{Max Schwarzkopf}} für den \textcolor{violet}{2. Abend}\eventindex{Burgtheater@\textbf{Burgtheater}!2. Aufführung von Zwischenspiel, 13.10.1905@2. Aufführung von Zwischenspiel, 13.10.1905|pwv}{}\ledrightnote{{$\rightarrow$}\emph{\textcolor{violet}{2. Aufführung von Zwischenspiel, 13.10.1905}}} meine Autorenſitze \label{K_L04019-1v}\edtext{ſchicken}{\lemma{\textnormal{\emph{ſchicken}}}\Cendnote{\textnormal{Siehe Arthur Schnitzler an Max Schwarzkopf, 13. 10. 1905.}}}\label{K_L04019-1} darf?\pend
           
\pstart
           Herzlich grüßen\textcolor{gray}{d} Ihr{\\[\baselineskip]}\spacefill\mbox{ArthSch}\pend
           \leftskip=0em{}\selectlanguage{ngerman}\endnumbering\briefempfaengerindex{Schwarzkopf, Gustav@\textsc{Schwarzkopf, Gustav}!zzzSchnitzler, Arthur@\emph{von Arthur Schnitzler}!1905-10-071@{7. 10. 1905}|)be}\mylabel{L04019h}
\begin{anhang}
\end{anhang}\normalsize

\doendnotes{C}
\bigskip
\vfill

\clearpage

\footnotesize

\lohead{\textsc{register}}

% Definiere theindex-Environment komplett neu ohne reledmac
\makeatletter
\renewenvironment{theindex}{%
  \section*{\indexname}%
  \setlength{\parindent}{0pt}%
  \setlength{\parskip}{0pt plus 0.3pt}%
  \let\item\@idxitem
}{%
  \clearpage
}
\makeatother

\IfFileExists{\jobname-pw.ind}{\input{\jobname-pw.ind}}{}

\end{document}

      