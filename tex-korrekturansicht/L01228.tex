%% latex-korrekturansicht-vorspann.tex
%% Vorspann für die Korrekturansicht.
%% Lädt die gemeinsame Datei latex-vorspann.tex mit gesetztem Schalter.

\newif\ifkorrekturansicht
\korrekturansichttrue

\input{../tex-inputs/latex-vorspann}


               \section[Hugo von Hofmannsthal und Arthur Schnitzler an Richard Beer-Hofmann, 3. 7. 1902]{ Hugo von Hofmannsthal und Arthur Schnitzler an Richard Beer-Hofmann,
               3. 7. 1902}\nopagebreak\mylabel{v}\rehead{ }\normalsize\beginnumbering\briefempfaengerindex{Beer-Hofmann, Richard@\textsc{Beer-Hofmann, Richard}!zzzSchnitzler, Arthur@\emph{von Arthur Schnitzler}!1902-07-032@{3. 7. 1902}|(be}\briefempfaengerindex{Beer-Hofmann, Richard@\textsc{Beer-Hofmann, Richard}!zzzHofmannsthal, Hugo von@\emph{von Hugo von Hofmannsthal}!1902-07-032@{3. 7. 1902}|(be} \toendnotes[C]{\smallbreak\pagebreak[2]} \Standort{YCGL, MSS 31.}
\physDesc{Bildpostkarte
\newline{}Handschrift Hugo von Hofmannsthal: Bleistift, deutsche Kurrent\newline{}Handschrift Arthur Schnitzler: Bleistift, deutsche Kurrent\newline{}Versand: 1) Stempel: »\nobreak{}\oindex{Matrei am Brenner@\textbf{Matrei am Brenner}, \emph{Besiedelter Ort (A.BSO)}|pwk}Deutsch-\textcolor{gray}{Matrei}, 3/{[}7 1902{]}\nobreak{}«.  2) Stempel: »\nobreak{}\oindex{Rodaun@\textbf{Rodaun}, \emph{Teil eines besiedelten Ortes (A.BSOX)}|pwk}\textcolor{gray}{Rod}aun, 4. 7. 02, 9–12V\nobreak{}«. \newline{}Ordnung: mit Bleistift von unbekannter Hand datiert: »3. 7.« }\pstart{}{\pb}Hrn \strikeout{\textcolor{blue}{\textsc{Gustav Schwarzkopf}}{}\ledrightnote{\textcolor{blue}{Gustav Schwarzkopf}}}\pend{}\pstart{}\textsc{Dr Richard Beer-Hofmann}\pend{}\pstart{}\textcolor{pink}{\textsc{Rodaun bei Wien}}{}\ledrightnote{\textcolor{pink}{Rodaun}}\pend{}\pstart{}\textcolor{pink}{\textsc{Liesingerstr 2}}{}\ledrightnote{\textcolor{pink}{Liesingerstraße}}. \pend{}{\bigskip}\pstart
           \noindent{}\centering{}\textcolor{gray}{\textbf{{\pb}\textcolor{pink}{MATREI}{}\ledrightnote{\textcolor{pink}{Matrei am Brenner}}.}}\pend
           \pstart
           3. 7. 902.\pend
           \pstart
           Herzlichen Gruß!\pend
           \pstart Ihr \spacefill\mbox{Arthur}\pend{}\pstart
           \noindent{}{[}hs. Hofmannsthal:{]} Jajaſibär!\pend
           \endnumbering\briefempfaengerindex{Beer-Hofmann, Richard@\textsc{Beer-Hofmann, Richard}!zzzSchnitzler, Arthur@\emph{von Arthur Schnitzler}!1902-07-032@{3. 7. 1902}|)be}\briefempfaengerindex{Beer-Hofmann, Richard@\textsc{Beer-Hofmann, Richard}!zzzHofmannsthal, Hugo von@\emph{von Hugo von Hofmannsthal}!1902-07-032@{3. 7. 1902}|)be}\mylabel{h}  \normalsize

\doendnotes{C}
\bigskip
\vfill

\clearpage

\footnotesize

\lohead{\textsc{register}}

% Definiere theindex-Environment komplett neu ohne reledmac
\makeatletter
\renewenvironment{theindex}{%
  \section*{\indexname}%
  \setlength{\parindent}{0pt}%
  \setlength{\parskip}{0pt plus 0.3pt}%
  \let\item\@idxitem
}{%
  \clearpage
}
\makeatother

\IfFileExists{\jobname-pw.ind}{\input{\jobname-pw.ind}}{}

\end{document}

      