%% latex-korrekturansicht-vorspann.tex
%% Vorspann für die Korrekturansicht.
%% Lädt die gemeinsame Datei latex-vorspann.tex mit gesetztem Schalter.

\newif\ifkorrekturansicht
\korrekturansichttrue

\input{../tex-inputs/latex-vorspann}


               \section[Max Burckhard an Arthur Schnitzler, 24. {[}12. 1904{]}]{ Max Burckhard an Arthur Schnitzler, 24. {[}12. 1904{]}}\nopagebreak\mylabel{v}\rehead{ }\normalsize\beginnumbering\briefempfaengerindex{Schnitzler, Arthur@\textsc{Schnitzler, Arthur}!zzzBurckhard, Max Eugen@\emph{von Max Eugen Burckhard}!1904-12-241@{24. {[}12. 1904{]}}|(be} \toendnotes[C]{\smallbreak\pagebreak[2]} \Standort{CUL, Schnitzler, B 20.}
\physDesc{Telegramm
\newline{}maschinell\newline{}Versand: aufgenommen von »\textcolor{gray}{\textbf{\textit{\textcolor{blue}{C. Neuberg\textcolor{gray}{er}}}}}« 
\newline{}Schnitzler: mit Bleistift datiert: »24/12 904« }\pstart
           {\pb}fr \textcolor{pink}{st gilgen}{}\ledrightnote{\textcolor{pink}{St. Gilgen}} 35 13 24{ }2 35 n\pend
           \pstart
           wetter herrlich. – was ists?
                    herzliche weihnachtsgruesse\pend
           \pstart \spacefill\mbox{= burckhard. +}\pend{}\endnumbering\briefempfaengerindex{Schnitzler, Arthur@\textsc{Schnitzler, Arthur}!zzzBurckhard, Max Eugen@\emph{von Max Eugen Burckhard}!1904-12-241@{24. {[}12. 1904{]}}|)be}\mylabel{h}  \normalsize

\doendnotes{C}
\bigskip
\vfill

\clearpage

\footnotesize

\lohead{\textsc{register}}

% Definiere theindex-Environment komplett neu ohne reledmac
\makeatletter
\renewenvironment{theindex}{%
  \section*{\indexname}%
  \setlength{\parindent}{0pt}%
  \setlength{\parskip}{0pt plus 0.3pt}%
  \let\item\@idxitem
}{%
  \clearpage
}
\makeatother

\IfFileExists{\jobname-pw.ind}{\input{\jobname-pw.ind}}{}

\end{document}

      