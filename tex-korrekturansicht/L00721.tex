%% latex-korrekturansicht-vorspann.tex
%% Vorspann für die Korrekturansicht.
%% Lädt die gemeinsame Datei latex-vorspann.tex mit gesetztem Schalter.

\newif\ifkorrekturansicht
\korrekturansichttrue

\input{../tex-inputs/latex-vorspann}


               \section[Arthur Schnitzler an Richard Beer-Hofmann, 31. 8. 1897]{ Arthur Schnitzler an Richard Beer-Hofmann, 31. 8. 1897}\nopagebreak\mylabel{v}\rehead{ }\normalsize\beginnumbering\briefempfaengerindex{Beer-Hofmann, Richard@\textsc{Beer-Hofmann, Richard}!zzzSchnitzler, Arthur@\emph{von Arthur Schnitzler}!1897-08-313@{31. 8. 1897}|(be} \toendnotes[C]{\smallbreak\pagebreak[2]} \Standort{YCGL, MSS 31.}
\physDesc{Briefkarte, Umschlag
\newline{}Handschrift: Bleistift, deutsche Kurrent\newline{}Versand: 1) Rohrpost 2) Stempel: »\nobreak{}\oindex{VIII., Josefstadt@\textbf{VIII., Josefstadt}, \emph{Bezirk (A.BZK)}|pwk}Wien 8/1, \textcolor{gray}{1} IX 97, 9 10V\nobreak{}«. 3) Stempel: »\nobreak{}\oindex{I., Innere Stadt@\textbf{I., Innere Stadt}, \emph{Bezirk (A.BZK)}|pwk}Wien 1/1, 1 XI 97, 9 30V\nobreak{}«. }\toendnotes[C]{\smallbreak}\pstart{}{\pb}\damage{Herrn Dr.}{ }\textsc{Richard Beer Hofmann}\pend{}\pstart{}\textcolor{pink}{Wien}{}\ledrightnote{\textcolor{pink}{Wien}}\pend{}\pstart{}\textcolor{pink}{\textsc{I. Wollzeile 15}}{}\ledrightnote{\textcolor{pink}{Wollzeile}}.\pend{}{\bigskip}\pstart
           \noindent{}{\pb}Lieber Richard, Ihren Brief erhielt ich
               um \label{K_L00721_1v}\edtext{¾ 10}{\lemma{\textnormal{\emph{¾ 10}}}\Cendnote{\textnormal{21 Uhr 45}}}\label{K_L00721_1h} im \textcolor{pink}{Arkaden}{}\ledrightnote{\textcolor{pink}{Café Arkaden}}. War zu müd Sie zu erwarten. Morgen
                  (Mittwoch) hab ich keine Sekunde für mich; denkbar wäre ſehr ſpät \textcolor{pink}{\textsc{Arkadencafé}}{}\ledrightnote{\textcolor{pink}{Café Arkaden}}. Do{\geminationn}erſtag{ }ſchreib ich Ihnen. Ich bin ſehr, ſehr \label{K_L00721_2v}\edtext{nervös}{\lemma{\textnormal{\emph{nervös}}}\Cendnote{\textnormal{womöglich wegen der bevorstehenden Entbindung seiner
                  Lebensgefährtin \textcolor{blue}{Marie Reinhard}. Am
                     24. 9. 1897 kam ein \textcolor{blue}{Kind} tot auf die Welt.}}}\label{K_L00721_2h}.\pend
           \pstart
           {\pb}Bei Ihnen geht doch \label{K_L00721_3v}\edtext{alles gut}{\lemma{\textnormal{\emph{alles gut}}}\Cendnote{\textnormal{Am
                     4. 9. 1897 kam die Tochter \textcolor{blue}{Mirjam
                     Beer-Hofmann} zur Welt.}}}\label{K_L00721_3h}?\pend
           \pstart Herzlich Ihr \spacefill\mbox{Arthur}\pend{}\endnumbering\briefempfaengerindex{Beer-Hofmann, Richard@\textsc{Beer-Hofmann, Richard}!zzzSchnitzler, Arthur@\emph{von Arthur Schnitzler}!1897-08-313@{31. 8. 1897}|)be}\mylabel{h}  \normalsize

\doendnotes{C}
\bigskip
\vfill

\clearpage

\footnotesize

\lohead{\textsc{register}}

% Definiere theindex-Environment komplett neu ohne reledmac
\makeatletter
\renewenvironment{theindex}{%
  \section*{\indexname}%
  \setlength{\parindent}{0pt}%
  \setlength{\parskip}{0pt plus 0.3pt}%
  \let\item\@idxitem
}{%
  \clearpage
}
\makeatother

\IfFileExists{\jobname-pw.ind}{\input{\jobname-pw.ind}}{}

\end{document}

      