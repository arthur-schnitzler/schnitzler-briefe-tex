%% latex-korrekturansicht-vorspann.tex
%% Vorspann für die Korrekturansicht.
%% Lädt die gemeinsame Datei latex-vorspann.tex mit gesetztem Schalter.

\newif\ifkorrekturansicht
\korrekturansichttrue

\input{../tex-inputs/latex-vorspann}


               \section[Arthur Schnitzler an Hugo von Hofmannsthal, 30. 6. 1904]{ Arthur Schnitzler an Hugo von Hofmannsthal, 30. 6. 1904}\nopagebreak\mylabel{v}\rehead{ }\normalsize\beginnumbering\briefempfaengerindex{Hofmannsthal, Hugo von@\textsc{Hofmannsthal, Hugo von}!zzzSchnitzler, Arthur@\emph{von Arthur Schnitzler}!1904-06-301@{30. 6. 1904}|(be} \toendnotes[C]{\smallbreak\pagebreak[2]} \Standort{FDH, Hs-30885,109.}
\physDesc{Kartenbrief
\newline{}Handschrift: schwarze Tinte, deutsche Kurrent\newline{}Versand: 1) Stempel: »\nobreak{}\oindex{XVIII., Waehring@\textbf{XVIII., Währing}, \emph{Bezirk (A.BZK)}|pwk}18/1 Wien, 3{[}0. 6. 1904{]}\nobreak{}«.  2) Stempel: »\nobreak{}\oindex{Rodaun@\textbf{Rodaun}, \emph{Teil eines besiedelten Ortes (A.BSOX)}|pwk}Rodaun, 1{[}. 7. 1904{]}\nobreak{}«. }\buchAbdrucke{\weitereDrucke{Hugo von Hofmannsthal, Arthur Schnitzler: \emph{Briefwechsel}. Hg. Therese Nickl und Heinrich Schnitzler. Frankfurt am Main: \emph{S. Fischer} 1964, S. 190.} }\toendnotes[C]{\smallbreak}\pstart{}{\pb}Herrn Dr Hugo von Hofmannsthal\pend{}\pstart{}\textsc{\label{K_L01412_1v}\edtext{\textcolor{pink}{Rodaun}{}\ledrightnote{\textcolor{pink}{Rodaun}}}{\lemma{\textnormal{\emph{Rodaun}}}\Cendnote{\textnormal{\textcolor{blue}{Schnitzler} begann die Zeile mit einem
                        »W«, das von einem »R« überschrieben wurde. Zur Sicherheit schrieb er am
                        oberen Rand noch einmal »\textcolor{pink}{Rodaun}«.}}}\label{K_L01412_1h}}\pend{}\pstart{}\textsc{bei \textcolor{pink}{Liesing}{}\ledrightnote{\textcolor{pink}{XXIII., Liesing}}}\pend{}{\bigskip}\pstart
           \raggedleft{}{\pb}30. 6. 904\pend
           \pstart
           mein lieber Hugo, es geht mir noch recht gelb aber doch im ganzen
               beſſer, daſs Sie bald kommen wollen, iſt ſehr lieb, ich ſchlage Ihnen z. B. vor
                  Mittwoch{ }Mittag bei uns zu ſpeiſen, vielleicht ka{\geminationn}
               ich da auch ſchon ein bischen ſpaziren gehen. Für die »\textcolor{green}{Kunſt}{}\ledrightnote{\textcolor{green}{Kunst und Künstler}}« ſchönen Dank. Antworten Sie recht bald. Auch jeder andre Tag geht
               natürlich.\pend
           \pstart
           Herzlichſt{\\[\baselineskip]}Ihr \spacefill\mbox{A.}\pend
           \leftskip=0em{}\endnumbering\briefempfaengerindex{Hofmannsthal, Hugo von@\textsc{Hofmannsthal, Hugo von}!zzzSchnitzler, Arthur@\emph{von Arthur Schnitzler}!1904-06-301@{30. 6. 1904}|)be}\mylabel{h}  \normalsize

\doendnotes{C}
\bigskip
\vfill

\clearpage

\footnotesize

\lohead{\textsc{register}}

% Definiere theindex-Environment komplett neu ohne reledmac
\makeatletter
\renewenvironment{theindex}{%
  \section*{\indexname}%
  \setlength{\parindent}{0pt}%
  \setlength{\parskip}{0pt plus 0.3pt}%
  \let\item\@idxitem
}{%
  \clearpage
}
\makeatother

\IfFileExists{\jobname-pw.ind}{\input{\jobname-pw.ind}}{}

\end{document}

      