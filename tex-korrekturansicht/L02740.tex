%% latex-korrekturansicht-vorspann.tex
%% Vorspann für die Korrekturansicht.
%% Lädt die gemeinsame Datei latex-vorspann.tex mit gesetztem Schalter.

\newif\ifkorrekturansicht
\korrekturansichttrue

\input{../tex-inputs/latex-vorspann}


               \section[Paul Goldmann an Arthur Schnitzler, 9. 7. 1895]{ Paul Goldmann an Arthur Schnitzler, 9. 7. 1895}\nopagebreak\mylabel{v}\rehead{ }\normalsize\beginnumbering\briefempfaengerindex{Schnitzler, Arthur@\textsc{Schnitzler, Arthur}!zzzGoldmann, Paul@\emph{von Paul Goldmann}!1895-07-092@{9. 7. 1895}|(be} \toendnotes[C]{\smallbreak\pagebreak[2]} \Standort{DLA, A:Schnitzler, HS.NZ85.1.3165.}
\physDesc{Brief, 1 Blatt, 2 Seiten
\newline{}Handschrift: schwarze Tinte, deutsche Kurrent\newline{}Beilage: handschriftlicher Brief, 1 Blatt, 1 Seite, schwarze Tinte,
                                 Lateinschrift 
\newline{}Schnitzler: 1) mit schwarzer Tinte das Jahr »95« vermerkt 2) mit rotem Buntstift eine Unterstreichung}\toendnotes[C]{\smallbreak}\pstart
           \noindent{}{\pb}\textcolor{gray}{\textbf{\textbf{\textcolor{brown}{Frankfurter Zeitung}{}\ledrightnote{\textcolor{brown}{Frankfurter Zeitung}}}}}\pend
           \pstart
           \textcolor{gray}{\textbf{(\textcolor{brown}{\begin{otherlanguage}{french}Gazette de Francfort\end{otherlanguage}}{}\ledrightnote{\textcolor{brown}{Frankfurter Zeitung}}). }}\pend
           \pstart
           \textcolor{gray}{\textbf{\textbf{\begin{otherlanguage}{french}Fondateur M. \textcolor{blue}{L.
                              Sonnemann}{}\ledrightnote{\textcolor{blue}{Leopold Sonnemann}}\end{otherlanguage}.}}}\pend
           \pstart
           \begin{otherlanguage}{french}\textcolor{gray}{\textbf{\textcolor{green}{Journal}{}\ledrightnote{→\textcolor{green}{Frankfurter Zeitung}} politique,
                        financier,}}\end{otherlanguage}\pend
           \pstart
           \begin{otherlanguage}{french}\textcolor{gray}{\textbf{commercial et littéraire.}}\end{otherlanguage}\pend
           \pstart
           \begin{otherlanguage}{french}\textcolor{gray}{\textbf{\textbf{Paraissant trois fois par jour.}}}\end{otherlanguage}\hfill \textsc{\textcolor{pink}{Paris}{}\ledrightnote{\textcolor{pink}{Paris}}}, 9. Juli.\pend
           \pstart
           \begin{otherlanguage}{french}\textcolor{gray}{\textbf{\textbf{Bureau à \textcolor{pink}{Paris}{}\ledrightnote{\textcolor{pink}{Paris}}:}}}\end{otherlanguage}\pend
           \pstart
           \begin{otherlanguage}{french}\textcolor{gray}{\textbf{\textbf{\textcolor{pink}{24. Rue Feydeau}{}\ledrightnote{\textcolor{pink}{rue Feydeau}}.}}}\end{otherlanguage}\pend
           \pstart\center{}Mein lieber Freund,\pend\pstart
           Eben erhalte ich den beifolgenden Brief von \textsc{\textcolor{blue}{Henri Becque}{}\ledrightnote{\textcolor{blue}{Henry Becque}}} über »\textcolor{green}{Sterben}{}\ledrightnote{\textcolor{green}{Sterben. Novelle}}«. Nun wollen wir weiter
               ſehen.\pend
           \pstart
           Herzlichſt {\\[\baselineskip]}Dein {\\[\baselineskip]}\spacefill\mbox{Paul Goldmann.}\pend
           \leftskip=0em{}{\bigskip}\pstart{}{\pb}{[}hs. Becque:{]} \label{K_L02740-2v}\edtext{Mon cher Goldmann}{\lemma{\textnormal{\emph{Mon cher Goldmann}}}\Cendnote{\textnormal{französisch: »Mein lieber Goldmann«}}}\label{K_L02740-2h}\pend\pstart
           \label{K_L02740-3v}\edtext{\begin{otherlanguage}{french}Je viens de lire le \textcolor{green}{roman}{}\ledrightnote{→\textcolor{green}{Sterben. Novelle}} de votre ami. C’est très douloureux et \label{K_L02740-1v}\edtext{toût à fait remarquable}{\lemma{\textnormal{\emph{toût à fait remarquable}}}\Cendnote{\textnormal{vgl. A. S.: \emph{Tagebuch}, 15. 7. 1895}}}\label{K_L02740-1h}. Pourquoi m’avez vous demandé d’un \textcolor{gray}{prendre}{ }\textcolor{gray}{conveniance}?\end{otherlanguage}}{\lemma{\textnormal{\emph{Je … conveniance?}}}\Cendnote{\textnormal{französisch: »Ich habe eben den \textcolor{green}{Roman} Ihres Freundes gelesen. Es ist sehr schmerzhaft und vollständig bemerkenswert. Warum
                  haben Sie mich gefragt, XXXX«}}}\label{K_L02740-3h}\pend
           \pstart
           \label{K_L02740-4v}\edtext{\begin{otherlanguage}{french}Bien à vous\end{otherlanguage}}{\lemma{\textnormal{\emph{Bien à vous}}}\Cendnote{\textnormal{Der Ihre}}}\label{K_L02740-4h}{\\[\baselineskip]}\spacefill\mbox{\textcolor{blue}{Henry Becque}{}\ledrightnote{\textcolor{blue}{Henry Becque}}}\pend
           \leftskip=0em{}\endnumbering\briefempfaengerindex{Schnitzler, Arthur@\textsc{Schnitzler, Arthur}!zzzGoldmann, Paul@\emph{von Paul Goldmann}!1895-07-092@{9. 7. 1895}|)be}\mylabel{h}  \normalsize

\doendnotes{C}
\bigskip
\vfill

\clearpage

\footnotesize

\lohead{\textsc{register}}

% Definiere theindex-Environment komplett neu ohne reledmac
\makeatletter
\renewenvironment{theindex}{%
  \section*{\indexname}%
  \setlength{\parindent}{0pt}%
  \setlength{\parskip}{0pt plus 0.3pt}%
  \let\item\@idxitem
}{%
  \clearpage
}
\makeatother

\IfFileExists{\jobname-pw.ind}{\input{\jobname-pw.ind}}{}

\end{document}

      