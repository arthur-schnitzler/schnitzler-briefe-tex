%% latex-korrekturansicht-vorspann.tex
%% Vorspann für die Korrekturansicht.
%% Lädt die gemeinsame Datei latex-vorspann.tex mit gesetztem Schalter.

\newif\ifkorrekturansicht
\korrekturansichttrue

\input{../tex-inputs/latex-vorspann}


               \section[Friedrich M. Fels an Arthur Schnitzler, 25. 5. 1894]{ Friedrich M. Fels an Arthur Schnitzler, 25. 5. 1894}\nopagebreak\mylabel{v}\rehead{ }\normalsize\beginnumbering\briefempfaengerindex{Schnitzler, Arthur@\textsc{Schnitzler, Arthur}!zzzFels, Friedrich Michael@\emph{von Friedrich Michael Fels}!1894-05-251@{25. 5. 1894}|(be} \toendnotes[C]{\smallbreak\pagebreak[2]} \Standort{DLA, A:Schnitzler, HS.NZ85.1.2956.}
\physDesc{Kartenbrief
\newline{}Handschrift: schwarze Tinte, lateinische Kurrent\newline{}Versand: 1) Stempel: »\nobreak{}W{[}ien{]}
                                                  110, 25. 5. 1894, 8–9V\nobreak{}«.  2) Stempel: »\nobreak{}\oindex{IX., Alsergrund@\textbf{IX., Alsergrund}, \emph{Bezirk (A.BZK)}|pwk}Wien 9/\textcolor{gray}{3}, 25. 5. 94, 10.V, Bestellt\nobreak{}«. 
\newline{}Schnitzler: mit Bleistift datiert: »25/5 94« und nummeriert: »14« }\toendnotes[C]{\smallbreak}\pstart{}{\pb}Herrn Dr. Arthur Schnitzler\pend{}\pstart{}\textcolor{pink}{Wien}{}\ledrightnote{\textcolor{pink}{Wien}}\pend{}\pstart{}\textcolor{pink}{IX, Frankgaſse 1}{}\ledrightnote{\textcolor{pink}{Frankgasse}}\pend{}{\bigskip}\pstart
           \noindent{}\raggedleft{}{\pb}\textcolor{pink}{Wien XVIII, Exnergasse 3}{}\ledrightnote{\textcolor{pink}{Krütznergasse}}\textsuperscript{III. St. Th. 22}\pend
           \pstart
           Lieber Dr Schnitzler! Habe von Dr \textcolor{blue}{Beer-Hofma{\geminationn}}{}\ledrightnote{\textcolor{blue}{Richard Beer-Hofmann}} noch nichts empfangen und muss zum Überfluss noch wohl ein paar Tage zu
                    Hause bleiben, da ich schreckliche Zahnschmerzen habe und wieder ein Geschwür zu
                        beko{\geminationm}en scheine. Wären Sie vielleicht so
                    freundlich, mir eine Kleinigkeit zu senden, da es ganz unbesti{\geminationm}t ist, ob und wa{\geminationn}{ }\textcolor{blue}{Beer-Hofma{\geminationn}}{}\ledrightnote{\textcolor{blue}{Richard Beer-Hofmann}} es thun wird. Seien Sie mir nicht böse und bestens gegrüsst von Ihrem\pend
           \pstart \spacefill\mbox{Fels}\pend{}\pstart
           \noindent{}\label{K_L00329_1v}\edtext{scripsit in tormentis}{\lemma{\textnormal{\emph{scripsit in tormentis}}}\Cendnote{\textnormal{lat. geschrieben unter
                            Qualen.}}}\label{K_L00329_1h}\pend
           \endnumbering\briefempfaengerindex{Schnitzler, Arthur@\textsc{Schnitzler, Arthur}!zzzFels, Friedrich Michael@\emph{von Friedrich Michael Fels}!1894-05-251@{25. 5. 1894}|)be}\mylabel{h}  \normalsize

\doendnotes{C}
\bigskip
\vfill

\clearpage

\footnotesize

\lohead{\textsc{register}}

% Definiere theindex-Environment komplett neu ohne reledmac
\makeatletter
\renewenvironment{theindex}{%
  \section*{\indexname}%
  \setlength{\parindent}{0pt}%
  \setlength{\parskip}{0pt plus 0.3pt}%
  \let\item\@idxitem
}{%
  \clearpage
}
\makeatother

\IfFileExists{\jobname-pw.ind}{\input{\jobname-pw.ind}}{}

\end{document}

      