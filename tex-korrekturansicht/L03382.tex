%% latex-korrekturansicht-vorspann.tex
%% Vorspann für die Korrekturansicht.
%% Lädt die gemeinsame Datei latex-vorspann.tex mit gesetztem Schalter.

\newif\ifkorrekturansicht
\korrekturansichttrue

\input{../tex-inputs/latex-vorspann}


\renewcommand{\erwaehntePersonen}{Personen: Paul Goldmann, Theodore Rottenberg, Olga Schnitzler}
\renewcommand{\erwaehnteOrte}{Orte: Berlin, Dessauer Straße, Dresden, Frankgasse 1, Grand Hotel Wien, Podmokly, Prag, Wien}
\renewcommand{\erwaehnteWerke}{}
\section[ Paul Goldmann an Arthur Schnitzler, 7. 8. {[}1903{]}]{Paul Goldmann an Arthur Schnitzler, 7. 8. {[}1903{]}}
\nopagebreak\mylabel{v}
\rehead{ }\normalsize\beginnumbering\briefempfaengerindex{Schnitzler, Arthur@\textsc{Schnitzler, Arthur}!zzzGoldmann, Paul@\emph{von Paul Goldmann}!1903-08-072@{7. 8. {[}1903{]}}|(be}
\toendnotes[C]{\smallbreak\pagebreak[2]}\Standort{DLA, A:Schnitzler, HS.NZ85.1.3173.}
\physDesc{Brief, 1 Blatt, 2 Seiten, 593 Zeichen
\newline{}Handschrift: blaue Tinte, deutsche Kurrent
\newline{}Schnitzler: mit Bleistift das Jahr »903« vermerkt }\toendnotes[C]{\smallbreak}
\pstart
           \noindent{}\raggedleft{}{\pb}\textcolor{gray}{\textbf{\textcolor{pink}{DESSAUERSTRASSE 19}{}\ledrightnote{\textcolor{pink}{Dessauer Straße}}}}\pend
           
\pstart
           \textcolor{pink}{Berlin}{}\ledrightnote{\textcolor{pink}{Berlin}}, 7. Auguſt.\pend
           
\pstart
           Tauſend Dank für Deinen lieben Brief, mein lieber und \label{K_L03382-1v}\edtext{»egoiſtiſcher«}{\lemma{\textnormal{\emph{»egoiſtiſcher«}}}\Cendnote{\textnormal{Auch wenn es sich aller Wahrscheinlichkeit nach nur um
                     eine Aussage \textcolor{blue}{Schnitzler}s vom Typ ›aus
                     Eigeninteresse freue ich mich über Dein Kommen‹ im nicht erhaltenen Brief
                     gehandelt haben dürfte, wurde diese Anmerkung doch in zeitlicher Nähe zu einer
                     ausführlicheren Erklärung \textcolor{blue}{Schnitzler}s
                     über seinen lange Zeit egoistischen Zugang bei Werkkonzeptionen verfasst (vgl. A. S.: \emph{Tagebuch}, 8. 8. 1903). Es ist
                     zumindest vorstellbar, dass er diese Selbstkritik \textcolor{blue}{Goldmann} mitgeteilt hatte.}}}\label{K_L03382-1h} Freund!{ }Geſtern hatte ich Nachricht von \label{K_L03382-2v}\edtext{»\textcolor{blue}{ihr}{}\ledrightnote{{$\rightarrow$}\textcolor{blue}{Theodore Rottenberg}}«}{\lemma{\textnormal{\emph{»ihr«}}}\Cendnote{\textnormal{siehe Paul Goldmann an Arthur Schnitzler, 27. 6. [1903]}}}\label{K_L03382-2h}, daß ſie mit mir kommt. Heut wieder das
               Gegentheil. So geht es ſeit zehn Tagen! Ich kann nicht mehr, und ich habe
               beſchloſſen, morgen, Samſtag, früh nach
                  \textcolor{pink}{Wien}{}\ledrightnote{\textcolor{pink}{Wien}} zu fahren. Ich komme \label{K_L03382-3v}\edtext{über \textsc{\textcolor{pink}{Bodenbach}{}\ledrightnote{\textcolor{pink}{Podmokly}}}}{\lemma{\textnormal{\emph{über Bodenbach}}}\Cendnote{\textnormal{über die Zugstrecke \textcolor{pink}{Dresden}–\textcolor{pink}{Prag}}}}\label{K_L03382-3h} um 10 Uhr 15 (glaube ich) an. Wenn Du Abends ſo
               lange aufbleibſt, ſo hinterlaß’ mir im \textsc{\textcolor{pink}{Grand Hotel}{}\ledrightnote{\textcolor{pink}{Grand Hotel Wien}}} einen Brief, in welchem \textsc{Café} ich Dich \label{K_L03382-4v}\edtext{finden}{\lemma{\textnormal{\emph{finden}}}\Cendnote{\textnormal{\textcolor{blue}{Schnitzler} und \textcolor{blue}{Olga Gussmann} verbrachten den Abend des 8. 8. 1903{ }\textcolor{pink}{zu Hause}. \textcolor{blue}{Goldmann} traf \textcolor{blue}{Schnitzler} am 9. 8. 1903.}}}\label{K_L03382-4h} kann. Bitte, laß’ Dich aber
                  \uline{nicht im Geringſten} ſtören! Höre ich
                  Abends{ }{\pb}nicht von Dir, ſo bin ich Sonntag{ }Vormittag bei Dir.\pend
           
\pstart
           Herzlichſt Dein {\\[\baselineskip]}\spacefill\mbox{Paul Goldmann}\pend
           \leftskip=0em{}\endnumbering\briefempfaengerindex{Schnitzler, Arthur@\textsc{Schnitzler, Arthur}!zzzGoldmann, Paul@\emph{von Paul Goldmann}!1903-08-072@{7. 8. {[}1903{]}}|)be}\mylabel{h}  \normalsize

\doendnotes{C}
\bigskip
\vfill

\clearpage

\footnotesize

\lohead{\textsc{register}}

% Definiere theindex-Environment komplett neu ohne reledmac
\makeatletter
\renewenvironment{theindex}{%
  \section*{\indexname}%
  \setlength{\parindent}{0pt}%
  \setlength{\parskip}{0pt plus 0.3pt}%
  \let\item\@idxitem
}{%
  \clearpage
}
\makeatother

\IfFileExists{\jobname-pw.ind}{\input{\jobname-pw.ind}}{}

\end{document}

      