%% latex-korrekturansicht-vorspann.tex
%% Vorspann für die Korrekturansicht.
%% Lädt die gemeinsame Datei latex-vorspann.tex mit gesetztem Schalter.

\newif\ifkorrekturansicht
\korrekturansichttrue

\input{../tex-inputs/latex-vorspann}


               \section[Richard Dehmel an Arthur Schnitzler, 14. 11. 1908]{ Richard Dehmel an Arthur Schnitzler, 14. 11. 1908}\nopagebreak\mylabel{v}\rehead{ }\normalsize\beginnumbering\briefempfaengerindex{Schnitzler, Arthur@\textsc{Schnitzler, Arthur}!zzzDehmel, Richard@\emph{von Richard Dehmel}!1908-11-141@{14. 11. 1908}|(be} \toendnotes[C]{\smallbreak\pagebreak[2]} \Standort{CUL, Schnitzler, B 26.}
\physDesc{Postkarte
\newline{}Handschrift: blaue Tinte, lateinische Kurrent\newline{}Versand: 1) Rohrpost 2) Stempel: »\nobreak{}\oindex{I., Innere Stadt@\textbf{I., Innere Stadt}, \emph{Bezirk (A.BZK)}|pwk}Wien 1/1, 14 XI 08, X 20\nobreak{}«. 3) Stempel: »\nobreak{}\oindex{XVIII., Waehring@\textbf{XVIII., Währing}, \emph{Bezirk (A.BZK)}|pwk}18/1 Wien, 14 XI 08, 11 10V\nobreak{}«. 4) Stempel: »\nobreak{}\oindex{XVIII., Waehring@\textbf{XVIII., Währing}, \emph{Bezirk (A.BZK)}|pwk}18/1 Wien, 14 XI 08, XI 10\nobreak{}«. 
\newline{}Schnitzler: mit Bleistift zweimal beschriftet: »\textsc{Dehm}« }\pstart{}{\pb}Herrn\pend{}\pstart{}Dr. Arthur Schnitzler.\pend{}\pstart{}\textcolor{pink}{Wien. XVIII, 1.}{}\ledrightnote{\textcolor{pink}{Wien}}\pend{}\pstart{}\textcolor{pink}{Spöttelgasse 7}{}\ledrightnote{\textcolor{pink}{Edmund-Weiß-Gasse}}.\pend{}{\bigskip}\pstart
           \noindent{}{\pb}Verehrter Herr Schnitzler,
               wann treffe ich Sie morgen (Sonntag) zu \uline{Hause}? Am liebsten wäre mir eine Stunde tagsüber, weil ich Abends \strikeout{ziemlich früh} nicht lange \uline{bei} Ihnen bleiben könnte wegen meiner Weiterreise. Aber nur wenn es der
               gnädigen \textcolor{blue}{Frau}{}\ledrightnote{\textcolor{blue}{Olga Schnitzler}} in die Hausordnung paßt.\pend
           \pstart
           Mit Wiedersehensgrüßen{\\[\baselineskip]}\spacefill\mbox{R. Dehmel, \textcolor{pink}{Hotel
                  Bristol}{}\ledrightnote{\textcolor{pink}{Hotel Bristol}}}.\pend
           \leftskip=0em{}\endnumbering\briefempfaengerindex{Schnitzler, Arthur@\textsc{Schnitzler, Arthur}!zzzDehmel, Richard@\emph{von Richard Dehmel}!1908-11-141@{14. 11. 1908}|)be}\mylabel{h}  \normalsize

\doendnotes{C}
\bigskip
\vfill

\clearpage

\footnotesize

\lohead{\textsc{register}}

% Definiere theindex-Environment komplett neu ohne reledmac
\makeatletter
\renewenvironment{theindex}{%
  \section*{\indexname}%
  \setlength{\parindent}{0pt}%
  \setlength{\parskip}{0pt plus 0.3pt}%
  \let\item\@idxitem
}{%
  \clearpage
}
\makeatother

\IfFileExists{\jobname-pw.ind}{\input{\jobname-pw.ind}}{}

\end{document}

      