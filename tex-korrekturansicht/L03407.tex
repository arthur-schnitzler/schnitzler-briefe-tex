%% latex-korrekturansicht-vorspann.tex
%% Vorspann für die Korrekturansicht.
%% Lädt die gemeinsame Datei latex-vorspann.tex mit gesetztem Schalter.

\newif\ifkorrekturansicht
\korrekturansichttrue

\input{../tex-inputs/latex-vorspann}


\renewcommand{\erwaehntePersonen}{Personen: Berta Czegka, Heinrich Kanner, Olga Schnitzler, Isidor Singer}
\renewcommand{\erwaehnteInstitutionen}{Institutionen: Die Zeit}
\renewcommand{\erwaehnteOrte}{Orte: Wien, Wipplingerstraße}
\renewcommand{\erwaehnteWerke}{Werke: Schiller-Feier, Schiller-Zeit 1805 * 1905, Zum großen Wurstel. Burleske in einem Akt}
\section[ Felix Salten an Arthur Schnitzler, 11. 4. 1905]{Felix Salten an Arthur Schnitzler, 11. 4. 1905}
\nopagebreak\mylabel{v}
\rehead{ }\normalsize\beginnumbering\briefempfaengerindex{Schnitzler, Arthur@\textsc{Schnitzler, Arthur}!zzzSalten, Felix@\emph{von Felix Salten}!1905-04-112@{11. 4. 1905}|(be}
\toendnotes[C]{\smallbreak\pagebreak[2]}\Standort{CUL, Schnitzler, B 89, B 1.}
\physDesc{Briefkarte, 903 Zeichen
\newline{}Handschrift: schwarze Tinte, lateinische Kurrent
\newline{}Ordnung: mit Bleistift von unbekannter Hand nummeriert: »199« }\toendnotes[C]{\smallbreak}
\pstart
           \noindent{}{\pb}\textcolor{gray}{\textbf{DIE}}\pend
           
\pstart
           \textcolor{gray}{\textbf{\textcolor{brown}{ZEIT}{}\ledrightnote{\textcolor{brown}{Die Zeit}}}}\hfill \textcolor{gray}{\textbf{\emph{\textcolor{pink}{WIEN}{}\ledrightnote{\textcolor{pink}{Wien}}}}}{ }11. IV. 05\pend
           
\pstart
           \textcolor{gray}{\textbf{\textcolor{pink}{Wien}{}\ledrightnote{\textcolor{pink}{Wien}}er Tageszeitung}}\hfill \textcolor{gray}{\textbf{\emph{\textcolor{pink}{I. Wipplingerstrasse 38}{}\ledrightnote{\textcolor{pink}{Wipplingerstraße}}}}}\pend
           
\pstart
           \textcolor{gray}{\textbf{Herausgeber:}}\pend
           
\pstart
           \textcolor{gray}{\textbf{\textbf{Prof. Dr. \textcolor{blue}{I. Singer}{}\ledrightnote{\textcolor{blue}{Isidor Singer}}}}}\pend
           
\pstart
           \textcolor{gray}{\textbf{\textbf{Dr. \textcolor{blue}{Heinrich Kanner}{}\ledrightnote{\textcolor{blue}{Heinrich Kanner}}}}}\pend
           
\pstart
           \textcolor{gray}{\textbf{\textbf{Feuilleton-Redaktion}}}\pend
           
\pstart
           Lieber, vielen Dank für den \textcolor{green}{Beitrag}{}\ledrightnote{{$\rightarrow$}\textcolor{green}{Schiller-Feier}} zur \textcolor{green}{Schiller-Nu{\geminationm}er}{}\ledrightnote{{$\rightarrow$}\textcolor{green}{Schiller-Zeit 1805 * 1905}}. \label{K_L03407-1v}\edtext{Den \textcolor{green}{großen
                  Wurstel}{}\ledrightnote{\textcolor{green}{Zum großen Wurstel. Burleske in einem Akt}}}{\lemma{\textnormal{\emph{Den großen
                  Wurstel}}}\Cendnote{\textnormal{siehe Arthur Schnitzler an Felix Salten, 8. 2. 1905}}}\label{K_L03407-1h} wollen wir noch für die \textcolor{green}{Osternummer}{}\ledrightnote{{$\rightarrow$}\textcolor{green}{Schiller-Zeit 1805 * 1905}} bringen, und schlage ich Ihnen Frl. \textcolor{blue}{Berta Czegka}{}\ledrightnote{\textcolor{blue}{Berta Czegka}} als Zeichnerin vor, die ich für sehr begabt
               halte. Ich hatte sie schon in der vergangenen Woche zu mir bitten wollen, konnte aber
               mit Niemandem ordentlich sprechen, und war nur immer sehr flüchtig in der \textcolor{pink}{Redaction}{}\ledrightnote{{$\rightarrow$}\textcolor{pink}{Wipplingerstraße}}. Nun kommt sie wegen
               des \textcolor{green}{großen Wurstel}{}\ledrightnote{\textcolor{green}{Zum großen Wurstel. Burleske in einem Akt}}{ }morgen gegen 2 – od. ½ 3 zu mir, und ich will
               sie bitten, \label{K_L03407-2v}\edtext{am Donnerstag um 4\textsuperscript{h}–5\textsuperscript{h.} bei Ihnen}{\lemma{\textnormal{\emph{am … Ihnen}}}\Cendnote{\textnormal{Dazu kam es am
                  Dienstag, dem 13. 4. 1905.}}}\label{K_L03407-2h} zu sein. Sie arbeitet sehr flink; aber man muß
               ihr alles genau erklären. Wie Sie mir \label{K_L03407-3v}\edtext{s. Z.}{\lemma{\textnormal{\emph{s. Z.}}}\Cendnote{\textnormal{seiner Zeit}}}\label{K_L03407-3h}{ }\label{K_L03407-4v}\edtext{schrieben}{\lemma{\textnormal{\emph{schrieben}}}\Cendnote{\textnormal{siehe Arthur Schnitzler an Felix Salten, 8. 2. 1905}}}\label{K_L03407-4h}, verlangen Sie 600 Kronen für den Abdruck; und wird das Honorar am 1. Mai an Sie gesendet. Selbstverständlich erhalten Sie
               von \uline{beiden}{ }{\pb}\textcolor{green}{Manuscripten}{}\ledrightnote{{$\rightarrow$}\textcolor{green}{Schiller-Feier}{\newline}{$\rightarrow$}\textcolor{green}{Zum großen Wurstel. Burleske in einem Akt}} Autoren
               Correctur.\pend
           
\pstart
           Ich habe sehr bedauert, Sie \textcolor{blue}{Beide}{}\ledrightnote{{$\rightarrow$}\textcolor{blue}{Olga Schnitzler}}{ }\label{K_L03407-5v}\edtext{neulich}{\lemma{\textnormal{\emph{neulich}}}\Cendnote{\textnormal{siehe A. S.: \emph{Tagebuch}, 7. 4. 1905}}}\label{K_L03407-5h} verfehlt zu haben u. danke Ihnen noch nachträglich für Ihren Besuch.
               Hoffentlich auf bald.\pend
           
\pstart
           Herzlichst {\\[\baselineskip]}Ihr {\\[\baselineskip]}\spacefill\mbox{Salten}\pend
           \leftskip=0em{}\endnumbering\briefempfaengerindex{Schnitzler, Arthur@\textsc{Schnitzler, Arthur}!zzzSalten, Felix@\emph{von Felix Salten}!1905-04-112@{11. 4. 1905}|)be}\mylabel{h}  \normalsize

\doendnotes{C}
\bigskip
\vfill

\clearpage

\footnotesize

\lohead{\textsc{register}}

% Definiere theindex-Environment komplett neu ohne reledmac
\makeatletter
\renewenvironment{theindex}{%
  \section*{\indexname}%
  \setlength{\parindent}{0pt}%
  \setlength{\parskip}{0pt plus 0.3pt}%
  \let\item\@idxitem
}{%
  \clearpage
}
\makeatother

\IfFileExists{\jobname-pw.ind}{\input{\jobname-pw.ind}}{}

\end{document}

      