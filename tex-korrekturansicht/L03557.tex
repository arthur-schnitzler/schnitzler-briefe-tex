%% latex-korrekturansicht-vorspann.tex
%% Vorspann für die Korrekturansicht.
%% Lädt die gemeinsame Datei latex-vorspann.tex mit gesetztem Schalter.

\newif\ifkorrekturansicht
\korrekturansichttrue

\input{../tex-inputs/latex-vorspann}


\renewcommand{\erwaehntePersonen}{Personen: Lili Cappellini, Paul Goldmann, Gerhart Hauptmann, Felix Salten, Olga Schnitzler, Heinrich Schnitzler, Louise Wolff}
\renewcommand{\erwaehnteOrte}{Orte: Berghof, Berlin, Brijuni, Dresden, Landshut, Leipzig, Prag, Salzkammergut, Sternwartestraße 71, Unterach am Attersee, Weimar, Wien}
\renewcommand{\erwaehnteWerke}{Werke: Eine Gerhart Hauptmann-Première in Lauchstedt. (»Gabriel Schillings Flucht.«), Gabriel Schillings Flucht. Drama, Neue Freie Presse, Tagebuch}
\section[ Felix Salten an Arthur Schnitzler, 2. 7. 1912]{Felix Salten an Arthur Schnitzler, 2. 7. 1912}
\nopagebreak\mylabel{v}
\rehead{ }\normalsize\beginnumbering\briefempfaengerindex{Schnitzler, Arthur@\textsc{Schnitzler, Arthur}!zzzSalten, Felix@\emph{von Felix Salten}!1912-07-021@{2. 7. 1912}|(be}
\toendnotes[C]{\smallbreak\pagebreak[2]}\Standort{CUL, Schnitzler, B 89, B 2.}
\physDesc{Bildpostkarte, 812 Zeichen
\newline{}Handschrift: schwarze Tinte, lateinische Kurrent
\newline{}Versand: Stempel: »\nobreak{}\oindex{Unterach am Attersee@\textbf{Unterach am Attersee}, \emph{P.PPL}|pwk}Unterach
                                          \textcolor{gray}{Atters}e\textcolor{gray}{e}, 2. VII. 12\nobreak{}«.  
\newline{}Ordnung: mit Bleistift von unbekannter Hand nummeriert: »272« }\toendnotes[C]{\smallbreak}\pstart{}{\pb}Herrn D\textsuperscript{r} Arthur Schnitzler\pend{}\pstart{}\textcolor{pink}{Wien}{}\ledrightnote{\textcolor{pink}{Wien}}\pend{}\pstart{}\textcolor{pink}{XVIII. Sternwartestraße 71}{}\ledrightnote{\textcolor{pink}{Sternwartestraße 71}}\pend{}
{\bigskip}
\pstart
           \noindent{}\centering{}{\pb}\textcolor{gray}{\textbf{\textcolor{pink}{Salzkammergut}{}\ledrightnote{\textcolor{pink}{Salzkammergut}}. \textcolor{pink}{Berghof}{}\ledrightnote{\textcolor{pink}{Berghof}} bei \textcolor{pink}{Unterach}{}\ledrightnote{\textcolor{pink}{Unterach am Attersee}}.}}\pend
           
\pstart
           {\pb}Vielen Dank für die \label{K_L03557-1v}\edtext{\textcolor{pink}{Prag}{}\ledrightnote{\textcolor{pink}{Prag}}er Karte}{\lemma{\textnormal{\emph{Prager Karte}}}\Cendnote{\textnormal{\textcolor{blue}{Schnitzler} hielt sich am 14. 6. 1912 für einen
                  Tag in \textcolor{pink}{Prag} auf.}}}\label{K_L03557-1h}. Ich bin vorgestern über \textcolor{pink}{Landshut}{}\ledrightnote{\textcolor{pink}{Landshut}}, \textcolor{pink}{Leipzig}{}\ledrightnote{\textcolor{pink}{Leipzig}}, \textcolor{pink}{Weimar}{}\ledrightnote{\textcolor{pink}{Weimar}}, \textcolor{pink}{Berlin}{}\ledrightnote{\textcolor{pink}{Berlin}} u. \textcolor{pink}{Dresden}{}\ledrightnote{\textcolor{pink}{Dresden}} wieder \textcolor{pink}{hier}{}\ledrightnote{{$\rightarrow$}\textcolor{pink}{Unterach am Attersee}} gelandet. War drei Wochen fort, und freue mich jetzt,
               wieder hier zu sein. Wenn gehen Sie nach \label{K_L03557-2v}\edtext{\textcolor{pink}{Brioni}{}\ledrightnote{\textcolor{pink}{Brijuni}}}{\lemma{\textnormal{\emph{Brioni}}}\Cendnote{\textnormal{\textcolor{blue}{Schnitzler} reiste mit seiner Familie am 20. 7. 1912
                  aus \textcolor{pink}{Wien} ab und war am nächsten Tag in \textcolor{pink}{Brijuni}.
                  Hier blieben sie den ganzen Sommer bis zum 24. 8. 1912.}}}\label{K_L03557-2h}? Sie haben, glaub’ ich,
               sehr gut gewählt damit. Denn hier regnet es sich wieder tüchtig ein, und möchte ein
               nasser Sommer werden. Wie geht es Frau \textcolor{blue}{Olga}{}\ledrightnote{\textcolor{blue}{Olga Schnitzler}} und
               den \textcolor{blue}{Kinder}{}\ledrightnote{{$\rightarrow$}\textcolor{blue}{Heinrich Schnitzler}{\newline}{$\rightarrow$}\textcolor{blue}{Lili Cappellini}}n? In \textcolor{pink}{Berlin}{}\ledrightnote{\textcolor{pink}{Berlin}} hörte ich, Frau \textcolor{blue}{Wolf}{}\ledrightnote{\textcolor{blue}{Louise Wolff}} sei verreist gewesen, und habe durch
               Krankheitsfälle in der Familie böse Zeiten gehabt; wolle aber Ihrer \textcolor{blue}{Frau}{}\ledrightnote{{$\rightarrow$}\textcolor{blue}{Olga Schnitzler}} nun endlich schreiben. Über \textcolor{pink}{Landshut}{}\ledrightnote{\textcolor{pink}{Landshut}}{ }\textcolor{gray}{etc}. wäre viel zu erzählen. Ihrem \label{K_L03557-3v}\edtext{Urteil über das \textcolor{green}{Stück}{}\ledrightnote{{$\rightarrow$}\textcolor{green}{Gabriel Schillings Flucht. Drama}}}{\lemma{\textnormal{\emph{Urteil über das Stück}}}\Cendnote{\textnormal{\textcolor{blue}{Salten} dürfte sich in Folge auf die zuletzt
                  erschienene Theaterkritik von \textcolor{blue}{Paul Goldmann}
                  bezogen haben, die eine Aufführung von \textcolor{blue}{Gerhart
                     Hauptmann}s \emph{\textcolor{green}{Gabriel Schillings Flucht}}
                  behandelte: \textcolor{blue}{Paul Goldmann}: \emph{\textcolor{green}{Eine Gerhart Hauptmann-Première in Lauchstedt. (»Gabriel
                        Schillings Flucht.«)}}. In: \emph{\textcolor{green}{Neue Freie
                        Presse}}, Nr. 17.185, 27. 6. 1912,
                     Morgenblatt, S. 1–4. Für den 2. 2. 1912 führt \textcolor{blue}{Schnitzler}s \emph{\textcolor{green}{Tagebuch}}
                  eine Diskussion mit \textcolor{blue}{Salten} über das \textcolor{green}{Stück} an.}}}\label{K_L03557-3h} bin ich ein
               wenig näher geko{\geminationm}en, seit ich es auf der Bühne sah. \textcolor{blue}{Paul Goldmann}{}\ledrightnote{\textcolor{blue}{Paul Goldmann}} war wieder »\textcolor{green}{fein}{}\ledrightnote{{$\rightarrow$}\textcolor{green}{Eine Gerhart Hauptmann-Première in Lauchstedt. (»Gabriel Schillings Flucht.«)}}«!\pend
           
\pstart
           Alles Herzlichste von uns allen Sie alle! Ihr {\\[\baselineskip]}\spacefill\mbox{Salten}\pend
           \leftskip=0em{}
\pstart
           \textcolor{pink}{Berghof}{}\ledrightnote{\textcolor{pink}{Berghof}}, 2. Juli 12\pend
           \endnumbering\briefempfaengerindex{Schnitzler, Arthur@\textsc{Schnitzler, Arthur}!zzzSalten, Felix@\emph{von Felix Salten}!1912-07-021@{2. 7. 1912}|)be}\mylabel{h}  \normalsize

\doendnotes{C}
\bigskip
\vfill

\clearpage

\footnotesize

\lohead{\textsc{register}}

% Definiere theindex-Environment komplett neu ohne reledmac
\makeatletter
\renewenvironment{theindex}{%
  \section*{\indexname}%
  \setlength{\parindent}{0pt}%
  \setlength{\parskip}{0pt plus 0.3pt}%
  \let\item\@idxitem
}{%
  \clearpage
}
\makeatother

\IfFileExists{\jobname-pw.ind}{\input{\jobname-pw.ind}}{}

\end{document}

      