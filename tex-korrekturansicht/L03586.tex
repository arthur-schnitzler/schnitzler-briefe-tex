%% latex-korrekturansicht-vorspann.tex
%% Vorspann für die Korrekturansicht.
%% Lädt die gemeinsame Datei latex-vorspann.tex mit gesetztem Schalter.

\newif\ifkorrekturansicht
\korrekturansichttrue

\input{../tex-inputs/latex-vorspann}


\renewcommand{\erwaehntePersonen}{Personen: Felix Salten}
\renewcommand{\erwaehnteInstitutionen}{Institutionen: Georg Müller Verlag, Wiener Verlag}
\renewcommand{\erwaehnteOrte}{Orte: Wien}
\renewcommand{\erwaehnteWerke}{Werke: Der Schrei der Liebe. Novelle, Die Gedenktafel der Prinzessin Anna. Der Schrei der Liebe. Zwei Novellen}
\section[ Felix Salten an Arthur Schnitzler, 11. 4. 1928]{Felix Salten an Arthur Schnitzler, 11. 4. 1928}
\nopagebreak\mylabel{v}
\rehead{ }\normalsize\beginnumbering\briefempfaengerindex{Schnitzler, Arthur@\textsc{Schnitzler, Arthur}!zzzSalten, Felix@\emph{von Felix Salten}!1928-04-111@{11. 4. 1928}|(be}
\toendnotes[C]{\smallbreak\pagebreak[2]}\Standort{CUL, Schnitzler, B 89, B 2.}
\physDesc{Karte, 471 Zeichen
\newline{}Handschrift: Bleistift, lateinische Kurrent
\newline{}Schnitzler: mit rotem Buntstift zwei Unterstreichungen 
\newline{}Ordnung: mit Bleistift von unbekannter Hand nummeriert: »299« }\toendnotes[C]{\smallbreak}
\pstart
           \raggedleft{}{\pb}11. IV. 28\pend
           
\pstart{}Lieber,\pend
\pstart
           vielleicht haben Sie noch \label{K_L03586-1v}\edtext{mein Buch
                  »\textcolor{green}{Schrei der Liebe}{}\ledrightnote{\textcolor{green}{Der Schrei der Liebe. Novelle}}}{\lemma{\textnormal{\emph{mein … Liebe}}}\Cendnote{\textnormal{erschienen im \emph{\textcolor{brown}{Wiener Verlag}} im Oktober 1904,
                     siehe Felix Salten: Widmungsexemplar Der Schrei der Liebe für Arthur
               Schnitzler, 22. 10. 1904}}}\label{K_L03586-1h}«; ich gab es Ihnen damals, als es erschien. Auch die \label{K_L03586-2v}\edtext{zweite Ausgabe bei \textcolor{brown}{Georg Müller}{}\ledrightnote{\textcolor{brown}{Georg Müller Verlag}}, zusammen mit der »\textcolor{green}{Gedenktafel}{}\ledrightnote{\textcolor{green}{Die Gedenktafel der Prinzessin Anna. Der Schrei der Liebe. Zwei Novellen}}«, hab’ ich Ihnen dediziert}{\lemma{\textnormal{\emph{zweite … dediziert}}}\Cendnote{\textnormal{Das Widmungsexemplar der Ausgabe von 1913 ist 
                     nicht überliefert.}}}\label{K_L03586-2h}. Jetzt ist das kleine \textcolor{green}{Buch}{}\ledrightnote{{$\rightarrow$}\textcolor{green}{Der Schrei der Liebe. Novelle}} total vergriffen, ich \label{K_L03586-3v}\edtext{brauche es dringend}{\lemma{\textnormal{\emph{brauche es dringend}}}\Cendnote{\textnormal{\textcolor{blue}{Salten} benötigte das \textcolor{green}{Buch} als Grundlage für eine Neuausgabe,
                     vgl. Felix Salten: Widmungsexemplar Der Schrei der Liebe für Arthur
               Schnitzler, Juli 1928.}}}\label{K_L03586-3h} und kann
               es nirgendwo kriegen. Wenn Sie es noch haben und so gut sein wollen, es mir für zwei
               Wochen zu leihen, wäre sich sehr dankbar. Sie bekommen es unversehrt zurück.\pend
           
\pstart
           Herzlichst {\\[\baselineskip]}Ihr {\\[\baselineskip]}\spacefill\mbox{Felix Salten}\pend
           \leftskip=0em{}\endnumbering\briefempfaengerindex{Schnitzler, Arthur@\textsc{Schnitzler, Arthur}!zzzSalten, Felix@\emph{von Felix Salten}!1928-04-111@{11. 4. 1928}|)be}\mylabel{h}  \normalsize

\doendnotes{C}
\bigskip
\vfill

\clearpage

\footnotesize

\lohead{\textsc{register}}

% Definiere theindex-Environment komplett neu ohne reledmac
\makeatletter
\renewenvironment{theindex}{%
  \section*{\indexname}%
  \setlength{\parindent}{0pt}%
  \setlength{\parskip}{0pt plus 0.3pt}%
  \let\item\@idxitem
}{%
  \clearpage
}
\makeatother

\IfFileExists{\jobname-pw.ind}{\input{\jobname-pw.ind}}{}

\end{document}

      