%% latex-korrekturansicht-vorspann.tex
%% Vorspann für die Korrekturansicht.
%% Lädt die gemeinsame Datei latex-vorspann.tex mit gesetztem Schalter.

\newif\ifkorrekturansicht
\korrekturansichttrue

\input{../tex-inputs/latex-vorspann}


               \section[Arthur und Olga Schnitzler an Richard und Paula Beer-Hofmann, 11. 5. 1908]{ Arthur und Olga Schnitzler an Richard und Paula Beer-Hofmann,
               11. 5. 1908}\nopagebreak\mylabel{v}\rehead{ }\normalsize\beginnumbering\briefempfaengerindex{Beer-Hofmann, Paula@\textsc{Beer-Hofmann, Paula}!zzzSchnitzler, Olga@\emph{von Olga Schnitzler}!1908-05-111@{11. 5. 1908}|(be}\briefempfaengerindex{Beer-Hofmann, Paula@\textsc{Beer-Hofmann, Paula}!zzzSchnitzler, Arthur@\emph{von Arthur Schnitzler}!1908-05-111@{11. 5. 1908}|(be}\briefempfaengerindex{Beer-Hofmann, Richard@\textsc{Beer-Hofmann, Richard}!zzzSchnitzler, Olga@\emph{von Olga Schnitzler}!1908-05-111@{11. 5. 1908}|(be}\briefempfaengerindex{Beer-Hofmann, Richard@\textsc{Beer-Hofmann, Richard}!zzzSchnitzler, Arthur@\emph{von Arthur Schnitzler}!1908-05-111@{11. 5. 1908}|(be} \toendnotes[C]{\smallbreak\pagebreak[2]} \Standort{YCGL, MSS 31.}
\physDesc{Bildpostkarte
\newline{}Handschrift Arthur Schnitzler: Bleistift, deutsche Kurrent\newline{}Handschrift Olga Schnitzler: Bleistift\newline{}Versand: Stempel: »\nobreak{}\oindex{Garmisch-Partenkirchen@\textbf{Garmisch-Partenkirchen}, \emph{Besiedelter Ort (A.BSO)}|pwk}Garmisch, 11 Mai 08, 6–7Nm\nobreak{}«.  }\buchAbdrucke{\weitereDrucke{Arthur Schnitzler, Richard Beer-Hofmann: \emph{Briefwechsel 1891–1931}. Hg. Konstanze Fliedl. Wien, Zürich: \emph{Europaverlag} 1992, S. 190.} }\pstart{}{\pb}\textsc{Dr. Richard Beer-Hofmann}\pend{}\pstart{}und Frau
               \pend{}\pstart{}\textcolor{pink}{\textsc{Wien XVIII}}{}\ledrightnote{\textcolor{pink}{XVIII., Währing}}\pend{}\pstart{}\textcolor{pink}{\textsc{Hasenauerstr. 59}}{}\ledrightnote{\textcolor{pink}{Hasenauerstraße}}.\pend{}{\bigskip}\pstart
           \noindent{}\centering{}{\pb}\textcolor{gray}{\textbf{\textcolor{pink}{Garmisch}{}\ledrightnote{\textcolor{pink}{Garmisch-Partenkirchen}} mit \textcolor{pink}{Alpspitze}{}\ledrightnote{\textcolor{pink}{Alpspitze}}, \textcolor{pink}{Waxenstein}{}\ledrightnote{\textcolor{pink}{Waxenstein}} und \textcolor{pink}{Zugspitze}{}\ledrightnote{\textcolor{pink}{Zugspitze}}.}}\pend
           \pstart
           {\pb}11. 5. 08\pend
           \pstart
           Wir ſind im Autonachtmobilkaſtl von \textcolor{pink}{München}{}\ledrightnote{\textcolor{pink}{München}} hieher
               gefahren und grüßen herzlichſt!\pend
           \pstart
           Ihr{\\[\baselineskip]}\spacefill\mbox{Arthur}{\\[\baselineskip]}\spacefill\mbox{{[}hs. O. Schnitzler:{]} Olga.}\pend
           \leftskip=0em{}\endnumbering\briefempfaengerindex{Beer-Hofmann, Paula@\textsc{Beer-Hofmann, Paula}!zzzSchnitzler, Olga@\emph{von Olga Schnitzler}!1908-05-111@{11. 5. 1908}|)be}\briefempfaengerindex{Beer-Hofmann, Paula@\textsc{Beer-Hofmann, Paula}!zzzSchnitzler, Arthur@\emph{von Arthur Schnitzler}!1908-05-111@{11. 5. 1908}|)be}\briefempfaengerindex{Beer-Hofmann, Richard@\textsc{Beer-Hofmann, Richard}!zzzSchnitzler, Olga@\emph{von Olga Schnitzler}!1908-05-111@{11. 5. 1908}|)be}\briefempfaengerindex{Beer-Hofmann, Richard@\textsc{Beer-Hofmann, Richard}!zzzSchnitzler, Arthur@\emph{von Arthur Schnitzler}!1908-05-111@{11. 5. 1908}|)be}\mylabel{h}  \normalsize

\doendnotes{C}
\bigskip
\vfill

\clearpage

\footnotesize

\lohead{\textsc{register}}

% Definiere theindex-Environment komplett neu ohne reledmac
\makeatletter
\renewenvironment{theindex}{%
  \section*{\indexname}%
  \setlength{\parindent}{0pt}%
  \setlength{\parskip}{0pt plus 0.3pt}%
  \let\item\@idxitem
}{%
  \clearpage
}
\makeatother

\IfFileExists{\jobname-pw.ind}{\input{\jobname-pw.ind}}{}

\end{document}

      