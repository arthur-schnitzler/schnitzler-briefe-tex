%% latex-korrekturansicht-vorspann.tex
%% Vorspann für die Korrekturansicht.
%% Lädt die gemeinsame Datei latex-vorspann.tex mit gesetztem Schalter.

\newif\ifkorrekturansicht
\korrekturansichttrue

\input{../tex-inputs/latex-vorspann}


\renewcommand{\erwaehntePersonen}{Personen: Hermann Mamroth}
\renewcommand{\erwaehnteOrte}{Orte: Berlin, Bernburger Straße, Dessauer Straße, Deutsches Theater Berlin, Frankfurt am Main, Hotel Savoy}
\renewcommand{\erwaehnteWerke}{Werke: Lebendige Stunden. Vier Einakter, Tagebuch}
\section[ Paul Goldmann an Arthur Schnitzler, 1. 1. {[}1902{]}]{Paul Goldmann an Arthur Schnitzler, 1. 1. {[}1902{]}}
\nopagebreak\mylabel{v}
\rehead{ }\normalsize\beginnumbering\briefempfaengerindex{Schnitzler, Arthur@\textsc{Schnitzler, Arthur}!zzzGoldmann, Paul@\emph{von Paul Goldmann}!1902-01-014@{1. 1. {[}1902{]}}|(be}
\toendnotes[C]{\smallbreak\pagebreak[2]}\Standort{DLA, A:Schnitzler, HS.NZ85.1.3172.}
\physDesc{Brief, 1 Blatt, 1 Seite
\newline{}Handschrift: blaue Tinte, deutsche Kurrent
\newline{}Schnitzler: mit Bleistift das Jahr »{[}1{]}902« vermerkt }\toendnotes[C]{\smallbreak}
\pstart
           \centering{}{\pb}\textcolor{pink}{Frankfurt}{}\ledrightnote{\textcolor{pink}{Frankfurt am Main}}{ }1. Januar.\pend
           
\pstart\center{}Mein lieber Freund,\pend
\pstart
           Bitte, nimm’ den \label{K_L03189-1v}\edtext{Sitz}{\lemma{\textnormal{\emph{Sitz}}}\Cendnote{\textnormal{Für die Uraufführung von \emph{\textcolor{green}{Lebendige Stunden}} am \textcolor{pink}{Deutschen Theater Berlin}. Danach war \textcolor{blue}{Schnitzler} im \textcolor{pink}{Hotel
                           Savoy}. Dem \emph{\textcolor{green}{Tagebuch}} ist nicht zu
               entnehmen, ob \textcolor{blue}{Goldmann} und \textcolor{blue}{Hermann Mamroth} auch dort
               waren.
               }}}\label{K_L03189-1h}, den Du neben dem meinigen (N\textsuperscript{o} 95, 10. Reihe)
               haſt reſerviren laſſen und ſende ihn an meinen \textcolor{blue}{Onkel}{}\ledrightnote{{$\rightarrow$}\textcolor{blue}{Hermann Mamroth}}, Herrn \textsc{\textcolor{blue}{Hermann Mamroth}{}\ledrightnote{\textcolor{blue}{Hermann Mamroth}}}, \textsc{\textcolor{pink}{Berlin S. W.}{}\ledrightnote{\textcolor{pink}{Berlin}}}, \textcolor{pink}{\textsc{Bernburgerstraße} 28}{}\ledrightnote{\textcolor{pink}{Bernburger Straße}}. Wir verrechnen uns nach meiner
               Rückkunft.\pend
           
\pstart
           Bitte, ſchreibe mir nach meiner \textcolor{pink}{Berliner Wohnung}{}\ledrightnote{{$\rightarrow$}\textcolor{pink}{Dessauer Straße}} ein Wort, wo ich Dich am Samſtag nach der 
               \textcolor{green}{Vorſtellung}{}\ledrightnote{{$\rightarrow$}\textcolor{green}{Lebendige Stunden. Vier Einakter}}
                finde.\pend
           
\pstart
           Viele treue Grüße! Und nochmals Glück zum neuen Jahr!
               {\\[\baselineskip]}Dein {\\[\baselineskip]}\spacefill\mbox{Paul Goldm}\pend
           \leftskip=0em{}\endnumbering\briefempfaengerindex{Schnitzler, Arthur@\textsc{Schnitzler, Arthur}!zzzGoldmann, Paul@\emph{von Paul Goldmann}!1902-01-014@{1. 1. {[}1902{]}}|)be}\mylabel{h}
\begin{anhang}
\end{anhang}\normalsize

\doendnotes{C}
\bigskip
\vfill

\clearpage

\footnotesize

\lohead{\textsc{register}}

% Definiere theindex-Environment komplett neu ohne reledmac
\makeatletter
\renewenvironment{theindex}{%
  \section*{\indexname}%
  \setlength{\parindent}{0pt}%
  \setlength{\parskip}{0pt plus 0.3pt}%
  \let\item\@idxitem
}{%
  \clearpage
}
\makeatother

\IfFileExists{\jobname-pw.ind}{\input{\jobname-pw.ind}}{}

\end{document}

      