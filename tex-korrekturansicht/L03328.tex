%% latex-korrekturansicht-vorspann.tex
%% Vorspann für die Korrekturansicht.
%% Lädt die gemeinsame Datei latex-vorspann.tex mit gesetztem Schalter.

\newif\ifkorrekturansicht
\korrekturansichttrue

\input{../tex-inputs/latex-vorspann}


\renewcommand{\erwaehntePersonen}{Personen: Ottilie Salten, Siegfried Trebitsch}
\renewcommand{\erwaehnteOrte}{Orte: Wien}
\renewcommand{\erwaehnteWerke}{}
\section[ Felix Salten an Arthur Schnitzler, {[}11?. 4. 1902{]}]{Felix Salten an Arthur Schnitzler, {[}11?. 4. 1902{]}}
\nopagebreak\mylabel{v}
\rehead{ }\normalsize\beginnumbering\briefempfaengerindex{Schnitzler, Arthur@\textsc{Schnitzler, Arthur}!zzzSalten, Felix@\emph{von Felix Salten}!1902-04-111@{{[}11?. 4. 1902{]}}|(be}
\toendnotes[C]{\smallbreak\pagebreak[2]}\Standort{CUL, Schnitzler, B 89, A 2.}
\physDesc{Brief, 1 Blatt, 2 Seiten, 387 Zeichen
\newline{}Handschrift: Bleistift, lateinische Kurrent
\newline{}Schnitzler: mit Bleistift datiert: »10. 4. {[}1{]}902« 
\newline{}Ordnung: mit Bleistift von unbekannter Hand nummeriert: »152« }\toendnotes[C]{\smallbreak}
\pstart
           \noindent{}{\pb}Lieber Freund, also \uline{doch}{ }\label{K_L03328-1v}\edtext{Sonntag}{\lemma{\textnormal{\emph{Sonntag}}}\Cendnote{\textnormal{Am Sonntag, dem 13. 4. 1902, fand die Hochzeit von \textcolor{blue}{Ottilie Metzl} und \textcolor{blue}{Felix
                        Salten} statt. Die Trauzeugen waren \textcolor{blue}{Schnitzler} und \textcolor{blue}{Siegfried
                     Trebitsch}.}}}\label{K_L03328-1h}. Könnten Sie dabei sein, wäre es
               mir \uline{sehr} lieb, u. a. auch deswegen, weil ich es sonst
               Niemandem anzeigen will, nicht einmal in meiner Familie. Wäre aber sehr dankbar, wenn
               Sie Sonntag um 5\textsuperscript{h} zu mir kämen. herzlichst {\\}\spacefill\mbox{Salten}\pend
           
\pstart
           \noindent{}Holen Sie mich bitte \label{K_L03328-2v}\edtext{morgen{ }N. M.}{\lemma{\textnormal{\emph{morgen N. M.}}}\Cendnote{\textnormal{Das deutet darauf hin, dass sich \textcolor{blue}{Schnitzler} bei seiner Datierung um einen
                     Tag vertan hat. \textcolor{blue}{Salten} wusste, dass die
                     Impfung am Samstag, dem 12. 4. 1902, stattfinden sollte (vgl. Arthur Schnitzler an Felix Salten, [10. 4. 1902]).}}}\label{K_L03328-2h} zum {\pb}\label{K_L03328-3v}\edtext{Impfen}{\lemma{\textnormal{\emph{Impfen}}}\Cendnote{\textnormal{siehe A. S.: \emph{Tagebuch}, 12. 4. 1902}}}\label{K_L03328-3h} ab? Und sind Sie \label{K_L03328-4v}\edtext{heut{ }Abend im Caféhaus}{\lemma{\textnormal{\emph{heut Abend im Caféhaus}}}\Cendnote{\textnormal{nicht
                     nachweisbar}}}\label{K_L03328-4h}? Wenn Ja, senden Sie mir ein Wort, sonst geh ich garnicht
                  hin.\pend
           \endnumbering\briefempfaengerindex{Schnitzler, Arthur@\textsc{Schnitzler, Arthur}!zzzSalten, Felix@\emph{von Felix Salten}!1902-04-111@{{[}11?. 4. 1902{]}}|)be}\mylabel{h}  \normalsize

\doendnotes{C}
\bigskip
\vfill

\clearpage

\footnotesize

\lohead{\textsc{register}}

% Definiere theindex-Environment komplett neu ohne reledmac
\makeatletter
\renewenvironment{theindex}{%
  \section*{\indexname}%
  \setlength{\parindent}{0pt}%
  \setlength{\parskip}{0pt plus 0.3pt}%
  \let\item\@idxitem
}{%
  \clearpage
}
\makeatother

\IfFileExists{\jobname-pw.ind}{\input{\jobname-pw.ind}}{}

\end{document}

      