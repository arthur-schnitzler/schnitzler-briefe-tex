%% latex-korrekturansicht-vorspann.tex
%% Vorspann für die Korrekturansicht.
%% Lädt die gemeinsame Datei latex-vorspann.tex mit gesetztem Schalter.

\newif\ifkorrekturansicht
\korrekturansichttrue

\input{../tex-inputs/latex-vorspann}


\renewcommand{\erwaehntePersonen}{Personen: Heinrich Kanner, Ernst Raupach, Isidor Singer}
\renewcommand{\erwaehnteInstitutionen}{Institutionen: Die Zeit}
\renewcommand{\erwaehnteOrte}{Orte: Raimund-Theater, Wien, Wipplingerstraße}
\renewcommand{\erwaehnteWerke}{Werke: Der Müller und sein Kind. Volksdrama in fünf Aufzügen}
\section[ Felix Salten an Arthur Schnitzler, {[}zwischen 26. und 30. 10. 1903{]}]{Felix Salten an Arthur
               Schnitzler, {[}zwischen 26. und 30. 10. 1903{]}}
\nopagebreak\mylabel{v}
\rehead{ }\normalsize\beginnumbering\briefempfaengerindex{Schnitzler, Arthur@\textsc{Schnitzler, Arthur}!zzzSalten, Felix@\emph{von Felix Salten}!1@{{[}zwischen 26. und 30. 10. 1903{]}}|(be}
\toendnotes[C]{\smallbreak\pagebreak[2]}\Standort{CUL, Schnitzler, B 89, A 2.}
\physDesc{Brief, 1 Blatt, 1 Seite, 88 Zeichen
\newline{}Handschrift: Bleistift, lateinische Kurrent
\newline{}Schnitzler: mit Bleistift datiert: »Oct 903« 
\newline{}Ordnung: mit Bleistift von unbekannter Hand nummeriert: »174« }\toendnotes[C]{\smallbreak}
\pstart
           \noindent{}{\pb}\textcolor{gray}{\textbf{DIE}}\pend
           
\pstart
           \textcolor{gray}{\textbf{\textcolor{brown}{ZEIT}{}\ledrightnote{\textcolor{brown}{Die Zeit}}}}\hfill \textcolor{gray}{\textbf{\textcolor{pink}{\emph{WIEN}}{}\ledrightnote{\textcolor{pink}{Wien}}}}\pend
           
\pstart
           \textcolor{gray}{\textbf{\textcolor{pink}{Wien}{}\ledrightnote{\textcolor{pink}{Wien}}er Tageszeitung}}\hfill \textcolor{gray}{\textbf{\emph{\textcolor{pink}{I. Wipplingerstrasse 38}{}\ledrightnote{\textcolor{pink}{Wipplingerstraße}}}}}\pend
           
\pstart
           \textcolor{gray}{\textbf{Herausgeber:}}\pend
           
\pstart
           \textcolor{gray}{\textbf{\textbf{Prof. Dr. \textcolor{blue}{I. Singer}{}\ledrightnote{\textcolor{blue}{Isidor Singer}}}}}\pend
           
\pstart
           \textcolor{gray}{\textbf{\textbf{Dr. \textcolor{blue}{Heinrich Kanner}{}\ledrightnote{\textcolor{blue}{Heinrich Kanner}}}}}\pend
           
\pstart
           \textcolor{gray}{\textbf{\textbf{Redaction}}}\pend
           
\pstart
           \textcolor{gray}{\textbf{Telegramm-Adresse: \textcolor{brown}{\so{Zeit}}{}\ledrightnote{\textcolor{brown}{Die Zeit}}\so{,}{ }\textcolor{pink}{\so{Wien}}{}\ledrightnote{\textcolor{pink}{Wien}}}}\pend
           
\pstart
           \textcolor{gray}{\textbf{Interurbanes Telephon Nr. 15.988}}\pend
           
\pstart
           \textcolor{gray}{\textbf{= Telephone Nr. 17.040, 17.041 =}}\pend
           
\pstart
           Lieber, wir kommen also (mit \label{K_L03348-1v}\edtext{fourage}{\lemma{\textnormal{\emph{fourage}}}\Cendnote{\textnormal{eigentlich Pferdefutter, hier im Sinne von: mitgebrachtes Essen}}}\label{K_L03348-1h}) Sonntag nach dem »\label{K_L03348-2v}\edtext{\textcolor{green}{Müller}{}\ledrightnote{\textcolor{green}{Der Müller und sein Kind. Volksdrama in fünf Aufzügen}}}{\lemma{\textnormal{\emph{Müller}}}\Cendnote{\textnormal{\emph{\textcolor{green}{Der Müller und sein Kind. Volksdrama in fünf
                     Aufzügen}} von \textcolor{blue}{Ernst Raupach} wurde am
                     1. 11. 1903 am \textcolor{pink}{Raimundtheater} als Nachmittagsvorstellung (Beginnzeit
                     halb 3 Uhr) gegeben. Das erlaubt die Datierung des
                  Korrespondenzstücks in die Woche vor dem Sonntag, dem 1. 11. 1903. Der Brief
                   [zwischen 27. und 31. 10. 1903] wiederum
                  folgt auf den vorliegenden und ist ebenfalls vor dem Sonntag zu datieren.}}}\label{K_L03348-2h}«
               zu Ihnen.\pend
           
\pstart
           Herzlichst {\\[\baselineskip]}Ihr {\\[\baselineskip]}\spacefill\mbox{Salten}\pend
           \leftskip=0em{}\endnumbering\briefempfaengerindex{Schnitzler, Arthur@\textsc{Schnitzler, Arthur}!zzzSalten, Felix@\emph{von Felix Salten}!1903-10-261@{{[}zwischen 26. und 30. 10. 1903{]}}|)be}\mylabel{h}  \normalsize

\doendnotes{C}
\bigskip
\vfill

\clearpage

\footnotesize

\lohead{\textsc{register}}

% Definiere theindex-Environment komplett neu ohne reledmac
\makeatletter
\renewenvironment{theindex}{%
  \section*{\indexname}%
  \setlength{\parindent}{0pt}%
  \setlength{\parskip}{0pt plus 0.3pt}%
  \let\item\@idxitem
}{%
  \clearpage
}
\makeatother

\IfFileExists{\jobname-pw.ind}{\input{\jobname-pw.ind}}{}

\end{document}

      