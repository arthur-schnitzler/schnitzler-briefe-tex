%% latex-korrekturansicht-vorspann.tex
%% Vorspann für die Korrekturansicht.
%% Lädt die gemeinsame Datei latex-vorspann.tex mit gesetztem Schalter.

\newif\ifkorrekturansicht
\korrekturansichttrue

\input{../tex-inputs/latex-vorspann}


\renewcommand{\erwaehntePersonen}{Personen: Georg Hirschfeld}
\renewcommand{\erwaehnteInstitutionen}{Institutionen: Burgtheater}
\renewcommand{\erwaehnteOrte}{Orte: Bad Ischl, Hotel und Pension Rudolfshöhe (Leopold Petter), Salzburg, Wien}
\renewcommand{\erwaehnteWerke}{Werke: Agnes Jordan. Schauspiel in fünf Akten, Theater, Kunst und Literatur [Agnes Jordan nicht am Burgtheater], Wiener Allgemeine Zeitung}
\section[ Felix Salten an Arthur Schnitzler, 22. 7. 1897]{Felix Salten an Arthur Schnitzler, 22. 7. 1897}
\nopagebreak\mylabel{v}
\rehead{ }\normalsize\beginnumbering\briefempfaengerindex{Schnitzler, Arthur@\textsc{Schnitzler, Arthur}!zzzSalten, Felix@\emph{von Felix Salten}!1897-07-222@{22. 7. 1897}|(be}
\toendnotes[C]{\smallbreak\pagebreak[2]}\Standort{CUL, Schnitzler, B 89, A 2.}
\physDesc{Postkarte, 500 Zeichen
\newline{}Handschrift: Bleistift, lateinische Kurrent
\newline{}Versand: Stempel: »\nobreak{}Wien 8/1 a 64, 23\textcolor{gray}{.} 7. 97, 3–4 \textcolor{gray}{N}\nobreak{}«. Stempel: »\nobreak{}\oindex{Bad Ischl@\textbf{Bad Ischl}, \emph{P.PPL}|pwk}{[}Ischl{]}, {[}24. 7. 97{]}, 6—\textcolor{gray}{7 V}\nobreak{}«.  
\newline{}Ordnung: mit Bleistift von unbekannter Hand nummeriert: »93« }\toendnotes[C]{\smallbreak}\pstart{}{\pb}Herrn D\textsuperscript{r} Arthur Schnitzler\pend{}\pstart{}\textcolor{pink}{Ischl}{}\ledrightnote{\textcolor{pink}{Bad Ischl}}\pend{}\pstart{}\textcolor{pink}{Kaltenbach, Pension Petter}{}\ledrightnote{\textcolor{pink}{Hotel und Pension Rudolfshöhe (Leopold Petter)}}\pend{}
{\bigskip}
\pstart
           \noindent{}{\pb}Lieber Freund, ich lese soeben \textcolor{gray}{i}m \textcolor{green}{6-Uhr Blatt}{}\ledrightnote{\textcolor{green}{Wiener Allgemeine Zeitung}} die \label{K_L03270-1v}\edtext{\textcolor{green}{Notiz}{}\ledrightnote{{$\rightarrow$}\textcolor{green}{Theater, Kunst und Literatur [Agnes Jordan nicht am Burgtheater]}}}{\lemma{\textnormal{\emph{Notiz}}}\Cendnote{\textnormal{ »– Wie wir aus verläßlicher
                     Quelle erfahren, ist die Direction des \textcolor{brown}{\so{Hofburgtheaters}} von der Absicht, \textcolor{blue}{Georg \so{Hirschfeld}}’s neues Drama ›\textcolor{green}{\so{Agnes Jordan}}‹ nächste Saison zur Aufführung zu bringen, abgekommen.«
                     ([O. V.]: \emph{\textcolor{green}{Theater, Kunst und
                        Literatur}}. In: \emph{\textcolor{green}{Wiener Allgemeine
                        Zeitung}}, Nr. 5.818, 23. 7. 1897,
                     S. 3.)}}}\label{K_L03270-1h} von \textcolor{green}{Agnes Jordan}{}\ledrightnote{\textcolor{green}{Agnes Jordan. Schauspiel in fünf Akten}}. Ich
               brauche Ihnen wol nicht erst zu sagen, dass ich derselben vollständig ferne stehe.
               Ich weiß absolut nicht \label{K_L03270-2v}\edtext{durch wen}{\lemma{\textnormal{\emph{durch wen}}}\Cendnote{\textnormal{siehe Felix Salten an Arthur Schnitzler, 23. 7. 1897}}}\label{K_L03270-2h} man das erfahren hat. Morgen{ }Abend reise ich nach \textcolor{pink}{Salzburg}{}\ledrightnote{\textcolor{pink}{Salzburg}}, für
               ein paar Tage – Vielleicht \label{K_L03270-3v}\edtext{kommen Sie
                  hin}{\lemma{\textnormal{\emph{kommen Sie
                  hin}}}\Cendnote{\textnormal{nicht geschehen, siehe Felix Salten an Arthur Schnitzler, 13. 7. 1897}}}\label{K_L03270-3h}, ehe Sie nach \textcolor{pink}{Wien}{}\ledrightnote{\textcolor{pink}{Wien}} fahren. Wir reisen dann
               zusammen nach \textcolor{pink}{Wien}{}\ledrightnote{\textcolor{pink}{Wien}} zurück. Nachricht trifft mich
               in \textcolor{pink}{Salzburg}{}\ledrightnote{\textcolor{pink}{Salzburg}} poste restante. Herzlich
                  \spacefill\mbox{Salten}\pend
           
\pstart
           \label{T_L03270-1v}\edtext{22./7. 97\textcolor{gray}{.}{ }½ 12 Nachm im Café.}{\lemma{\textnormal{\emph{22./7. 97. … Café.}}}\Cendnote{\textnormal{am linken Rand, quer zum Text}}}\label{T_L03270-1h}\pend
           \endnumbering\briefempfaengerindex{Schnitzler, Arthur@\textsc{Schnitzler, Arthur}!zzzSalten, Felix@\emph{von Felix Salten}!1897-07-222@{22. 7. 1897}|)be}\mylabel{h}  \normalsize

\doendnotes{C}
\bigskip
\vfill

\clearpage

\footnotesize

\lohead{\textsc{register}}

% Definiere theindex-Environment komplett neu ohne reledmac
\makeatletter
\renewenvironment{theindex}{%
  \section*{\indexname}%
  \setlength{\parindent}{0pt}%
  \setlength{\parskip}{0pt plus 0.3pt}%
  \let\item\@idxitem
}{%
  \clearpage
}
\makeatother

\IfFileExists{\jobname-pw.ind}{\input{\jobname-pw.ind}}{}

\end{document}

      