%% latex-korrekturansicht-vorspann.tex
%% Vorspann für die Korrekturansicht.
%% Lädt die gemeinsame Datei latex-vorspann.tex mit gesetztem Schalter.

\newif\ifkorrekturansicht
\korrekturansichttrue

\input{../tex-inputs/latex-vorspann}


\renewcommand{\erwaehntePersonen}{Personen: Bjørnstjerne Bjørnson}
\renewcommand{\erwaehnteOrte}{Orte: Berlin, Dessauer Straße, Palasthotel Berlin, Residenztheater Berlin}
\renewcommand{\erwaehnteWerke}{Werke: Leonarda}
\section[ Paul Goldmann an Arthur Schnitzler, 21. 2. {[}1903{]}]{Paul Goldmann an Arthur Schnitzler, 21. 2. {[}1903{]}}
\nopagebreak\mylabel{v}
\rehead{ }\normalsize\beginnumbering\briefempfaengerindex{Schnitzler, Arthur@\textsc{Schnitzler, Arthur}!zzzGoldmann, Paul@\emph{von Paul Goldmann}!1903-02-211@{21. 2. {[}1903{]}}|(be}
\toendnotes[C]{\smallbreak\pagebreak[2]}\Standort{DLA, A:Schnitzler, HS.NZ85.1.3173.}
\physDesc{Brief, 1 Blatt, 2 Seiten
\newline{}Handschrift: blaue Tinte, deutsche Kurrent
\newline{}Schnitzler: mit Bleistift das Jahr »{[}1{]}903« vermerkt }\toendnotes[C]{\smallbreak}
\pstart
           \noindent{}\raggedleft{}{\pb}\textcolor{gray}{\textbf{\textcolor{pink}{DESSAUERSTRASSE 19}{}\ledrightnote{\textcolor{pink}{Dessauer Straße}}}}\pend
           
\pstart
           \textcolor{pink}{Berlin}{}\ledrightnote{\textcolor{pink}{Berlin}}, 21. Februar\pend
           
\pstart\center{}Mein lieber Freund,\pend
\pstart
           Herzlichſt \label{K_L03364-1v}\edtext{willkommen}{\lemma{\textnormal{\emph{willkommen}}}\Cendnote{\textnormal{\textcolor{blue}{Schnitzler} kam am nächsten Tag in \textcolor{pink}{Berlin} an.}}}\label{K_L03364-1h}! Ich muß leider zu einer \textsc{Matinée} ins \textcolor{pink}{Reſidenztheater}{}\ledrightnote{\textcolor{pink}{Residenztheater Berlin}} (»\textsc{\textcolor{green}{Leonarda}{}\ledrightnote{\textcolor{green}{Leonarda}}}« von \textsc{\textcolor{blue}{Björnson}{}\ledrightnote{\textcolor{blue}{Bjørnstjerne Bjørnson}}}). Anbei ein Billet, für den Fall, daß Du \label{K_L03364-2v}\edtext{mitkommen}{\lemma{\textnormal{\emph{mitkommen}}}\Cendnote{\textnormal{\textcolor{blue}{Schnitzler} kam nicht mit, vgl. A. S.: \emph{Tagebuch}, 22. 2. 1903.}}}\label{K_L03364-2h} willſt.
               Wenn nicht, ſo {\pb}komme ich
               zwiſchen 3 und \strikeout{1/} 3 ½ Uhr
               zu Dir ins \textsc{\textcolor{pink}{Hôtel}{}\ledrightnote{{$\rightarrow$}\textcolor{pink}{Palasthotel Berlin}}}. Bis 5 Uhr bin ich frei.\pend
           
\pstart
           Herzlichſt {\\[\baselineskip]}Dein {\\[\baselineskip]}\spacefill\mbox{Paul Goldm}\pend
           \leftskip=0em{}\endnumbering\briefempfaengerindex{Schnitzler, Arthur@\textsc{Schnitzler, Arthur}!zzzGoldmann, Paul@\emph{von Paul Goldmann}!1903-02-211@{21. 2. {[}1903{]}}|)be}\mylabel{h}
\begin{anhang}
\end{anhang}\normalsize

\doendnotes{C}
\bigskip
\vfill

\clearpage

\footnotesize

\lohead{\textsc{register}}

% Definiere theindex-Environment komplett neu ohne reledmac
\makeatletter
\renewenvironment{theindex}{%
  \section*{\indexname}%
  \setlength{\parindent}{0pt}%
  \setlength{\parskip}{0pt plus 0.3pt}%
  \let\item\@idxitem
}{%
  \clearpage
}
\makeatother

\IfFileExists{\jobname-pw.ind}{\input{\jobname-pw.ind}}{}

\end{document}

      