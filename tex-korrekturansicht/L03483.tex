%% latex-korrekturansicht-vorspann.tex
%% Vorspann für die Korrekturansicht.
%% Lädt die gemeinsame Datei latex-vorspann.tex mit gesetztem Schalter.

\newif\ifkorrekturansicht
\korrekturansichttrue

\input{../tex-inputs/latex-vorspann}


\renewcommand{\erwaehntePersonen}{Personen: Eva Marie Goldmann}
\renewcommand{\erwaehnteOrte}{Orte: Berlin, Gmunden, Hotel Excelsior, Kärnten, Sternwartestraße, Velden am Wörthersee, XVIII., Währing}
\renewcommand{\erwaehnteWerke}{}
\section[ Paul Goldmann an Arthur Schnitzler, 22. 8. 1931]{Paul Goldmann an Arthur Schnitzler, 22. 8. 1931}
\nopagebreak\mylabel{v}
\rehead{ }\normalsize\beginnumbering\briefempfaengerindex{Schnitzler, Arthur@\textsc{Schnitzler, Arthur}!zzzGoldmann, Paul@\emph{von Paul Goldmann}!1931-08-221@{22. 8. 1931}|(be}
\toendnotes[C]{\smallbreak\pagebreak[2]}\Standort{DLA, A:Schnitzler, HS.NZ85.1.3176.}
\physDesc{Bildpostkarte, 579 Zeichen
\newline{}Handschrift: schwarze Tinte, deutsche Kurrent
\newline{}Versand: Stempel: »\nobreak{}\oindex{Velden am Woerthersee@\textbf{Velden am Wörthersee}, \emph{P.PPL}|pwk}Velden \textcolor{gray}{am}
                                          Wörthersee \textcolor{gray}{3}, 22. VIII. 31\nobreak{}«.  }\toendnotes[C]{\smallbreak}\pstart{}{\pb}Herrn Dr.\pend{}\pstart{}Arthur Schnitzler\pend{}\pstart{}\textcolor{pink}{Wien XVIII.}{}\ledrightnote{\textcolor{pink}{XVIII., Währing}}\pend{}\pstart{}\textcolor{pink}{Sternwartstr. 71}{}\ledrightnote{\textcolor{pink}{Sternwartestraße}}.\pend{}
{\bigskip}
\pstart
           \noindent{}{\pb}\textcolor{gray}{\textbf{Blick auf \textcolor{pink}{Velden am
                        Wörthersee}{}\ledrightnote{\textcolor{pink}{Velden am Wörthersee}}.}}\pend
           
\pstart
           \textcolor{pink}{Velden a. Wörtherſee}{}\ledrightnote{\textcolor{pink}{Velden am Wörthersee}}, \textsc{\textcolor{pink}{Hotel Excelsior}{}\ledrightnote{\textcolor{pink}{Hotel Excelsior}}}, den 22. 8.\pend
           
\pstart
           Lieber Freund, Mit großer Verſpätung erreicht Deine
               nach \textcolor{pink}{Berlin}{}\ledrightnote{\textcolor{pink}{Berlin}} geſandte Karte mich hier in \textcolor{pink}{Velden}{}\ledrightnote{\textcolor{pink}{Velden am Wörthersee}}. Meine \textcolor{blue}{Frau}{}\ledrightnote{{$\rightarrow$}\textcolor{blue}{Eva Marie Goldmann}} wollte abſolut an einen See gehen; für mich hat es wenig
               Sinn, da ich nicht ſchwimme, auch iſt mir die Luft zu lau u. zu ſchlapp. Aber das \textsc{\textcolor{pink}{Hotel Excelsior}{}\ledrightnote{\textcolor{pink}{Hotel Excelsior}}} iſt vorzüglich. Hoffentlich verbringſt Du \label{K_L03483-1v}\edtext{an einem ſchönen Ort einen angenehmen Sommer}{\lemma{\textnormal{\emph{an … Sommer}}}\Cendnote{\textnormal{\textcolor{blue}{Schnitzler} war seit 7. 8. 1931 und noch
                  bis 25. 8. 1931 in
                     \textcolor{pink}{Gmunden}.}}}\label{K_L03483-1h}. \textcolor{blue}{Wir}{}\ledrightnote{{$\rightarrow$}\textcolor{blue}{Eva Marie Goldmann}} haben hier faſt immer schönes Wetter,
               das iſt der große Vorzug von \textcolor{pink}{Kärnten}{}\ledrightnote{\textcolor{pink}{Kärnten}}. Alles
               Herzliche von meiner \textcolor{blue}{Frau}{}\ledrightnote{{$\rightarrow$}\textcolor{blue}{Eva Marie Goldmann}} u.
               mir! Dein \spacefill\mbox{Paul Goldmann.}\pend
           \endnumbering\briefempfaengerindex{Schnitzler, Arthur@\textsc{Schnitzler, Arthur}!zzzGoldmann, Paul@\emph{von Paul Goldmann}!1931-08-221@{22. 8. 1931}|)be}\mylabel{h}
\begin{anhang}
\end{anhang}\normalsize

\doendnotes{C}
\bigskip
\vfill

\clearpage

\footnotesize

\lohead{\textsc{register}}

% Definiere theindex-Environment komplett neu ohne reledmac
\makeatletter
\renewenvironment{theindex}{%
  \section*{\indexname}%
  \setlength{\parindent}{0pt}%
  \setlength{\parskip}{0pt plus 0.3pt}%
  \let\item\@idxitem
}{%
  \clearpage
}
\makeatother

\IfFileExists{\jobname-pw.ind}{\input{\jobname-pw.ind}}{}

\end{document}

      