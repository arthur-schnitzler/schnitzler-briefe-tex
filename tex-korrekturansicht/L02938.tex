%% latex-korrekturansicht-vorspann.tex
%% Vorspann für die Korrekturansicht.
%% Lädt die gemeinsame Datei latex-vorspann.tex mit gesetztem Schalter.

\newif\ifkorrekturansicht
\korrekturansichttrue

\input{../tex-inputs/latex-vorspann}


         
         \renewcommand{\erwaehntePersonen}{Personen: Erich Freund}
         \renewcommand{\erwaehnteInstitutionen}{Institutionen: Neue Freie Presse}
         \renewcommand{\erwaehnteOrte}{Orte: Berlin, Breslau, Dessauer Straße, Wien}
         \renewcommand{\erwaehnteWerke}{Werke: Der Schleier der Beatrice. Schauspiel in fünf Akten}
               \section[ Paul Goldmann an Arthur Schnitzler, 12. 11. {[}1900{]}]{Paul Goldmann an Arthur Schnitzler, 12. 11. {[}1900{]}}\nopagebreak\mylabel{v}\rehead{ }\normalsize\beginnumbering\briefempfaengerindex{Schnitzler, Arthur@\textsc{Schnitzler, Arthur}!zzzGoldmann, Paul@\emph{von Paul Goldmann}!1900-11-121@{12. 11. {[}1900{]}}|(be} \toendnotes[C]{\smallbreak\pagebreak[2]} \Standort{DLA, A:Schnitzler, HS.NZ85.1.3170.}
\physDesc{Brief, 1 Blatt, 2 Seiten
\newline{}Handschrift: blaue Tinte, deutsche Kurrent
\newline{}Schnitzler: mit Bleistift das Jahr »{[}1{]}90\textcolor{gray}{0}« vermerkt }\toendnotes[C]{\smallbreak}{\bigskip}\pstart
           \noindent{}{\pb}\textcolor{pink}{Berlin}{}\ledrightnote{\textcolor{pink}{Berlin}}, 12. November.\hfill \textcolor{pink}{\textcolor{gray}{\textbf{DESSAUERSTRASSE 19}}}{}\ledrightnote{\textcolor{pink}{Dessauer Straße}}\pend
           \pstart\center{}Mein lieber Freund,\pend\pstart
           Ich will Dir nur in aller Eile Glück zur \label{K_L02938-1v}\edtext{Reiſe}{\lemma{\textnormal{\emph{Reiſe}}}\Cendnote{\textnormal{\textcolor{blue}{Schnitzler} hielt sich von 22. 11. 1900 bis 24. 11. 1900 und von
                     29. 11. 1900 bis
                     2. 12. 1900 in
                     \textcolor{pink}{Breslau} auf.}}}\label{K_L02938-1h} wünſchen. Es iſt
               wirklich ſehr beklagenswerth, daß ich nicht nach \textcolor{pink}{Breslau}{}\ledrightnote{\textcolor{pink}{Breslau}} kommen kann. Wo wirſt Du in \textcolor{pink}{Breslau}{}\ledrightnote{\textcolor{pink}{Breslau}} wohnen? Willſt Du ſo lieb ſein, mir am \label{K_L02938-4v}\edtext{Tage nach der \begin{otherlanguage}{french}\textsc{\textcolor{green}{Première}{}\ledrightnote{{$\rightarrow$}\textcolor{green}{Der Schleier der Beatrice. Schauspiel in fünf Akten}}}\end{otherlanguage}}{\lemma{\textnormal{\emph{Tage nach der Première}}}\Cendnote{\textnormal{Ursprünglich war die Uraufführung von
                     \emph{\textcolor{green}{Der Schleier der Beatrice}} für den 17. 11. 1900 geplant. Die Premiere wurde jedoch auf
                  den 1. 12. 1900
                  verschoben.}}}\label{K_L02938-4h} ein Wort zu telegraphiren?\pend
           \pstart
           Die \textcolor{brown}{N. Fr. Pr.}{}\ledrightnote{\textcolor{brown}{Neue Freie Presse}} hat meinen Vorſchlag, das \label{K_L02938-2v}\edtext{Referat}{\lemma{\textnormal{\emph{Referat}}}\Cendnote{\textnormal{siehe Paul Goldmann an Arthur Schnitzler, 30. 10. [1900] und 3. 12. [1900]}}}\label{K_L02938-2h} dem \textsc{Dr.}{ }{\pb}\textsc{\textcolor{blue}{Erich Freund}{}\ledrightnote{\textcolor{blue}{Erich Freund}}} zu übertragen, angenommen. So wird wenigſtens ein anſtändiger Menſch über \textcolor{green}{Dich}{}\ledrightnote{{$\rightarrow$}\textcolor{green}{Der Schleier der Beatrice. Schauspiel in fünf Akten}} berichten. Das iſt
               einſtweilen Alles, was ich thun konnte.\pend
           \pstart
           Auf frohes Wiederſehn in \textcolor{pink}{Berlin}{}\ledrightnote{\textcolor{pink}{Berlin}}!\pend
           \pstart
           Sei von Herzen gegrüßt {\\[\baselineskip]}von Deinem {\\[\baselineskip]}treuen {\\[\baselineskip]}\spacefill\mbox{Paul Goldmann.}\pend
           \leftskip=0em{}\endnumbering\briefempfaengerindex{Schnitzler, Arthur@\textsc{Schnitzler, Arthur}!zzzGoldmann, Paul@\emph{von Paul Goldmann}!1900-11-121@{12. 11. {[}1900{]}}|)be}\mylabel{h}\begin{anhang}\end{anhang}\normalsize

\doendnotes{C}
\bigskip
\vfill

\clearpage

\footnotesize

\lohead{\textsc{register}}

% Definiere theindex-Environment komplett neu ohne reledmac
\makeatletter
\renewenvironment{theindex}{%
  \section*{\indexname}%
  \setlength{\parindent}{0pt}%
  \setlength{\parskip}{0pt plus 0.3pt}%
  \let\item\@idxitem
}{%
  \clearpage
}
\makeatother

\IfFileExists{\jobname-pw.ind}{\input{\jobname-pw.ind}}{}

\end{document}

      