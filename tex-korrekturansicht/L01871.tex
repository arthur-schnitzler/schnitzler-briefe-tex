%% latex-korrekturansicht-vorspann.tex
%% Vorspann für die Korrekturansicht.
%% Lädt die gemeinsame Datei latex-vorspann.tex mit gesetztem Schalter.

\newif\ifkorrekturansicht
\korrekturansichttrue

\input{../tex-inputs/latex-vorspann}


               \section[Gerty von Hofmannsthal an Olga Schnitzler, 13. {[}9.{]} 1909]{ Gerty von Hofmannsthal an Olga Schnitzler, 13. {[}9.{]} 1909}\nopagebreak\mylabel{v}\rehead{ }\normalsize\beginnumbering\briefempfaengerindex{Schnitzler, Olga@\textsc{Schnitzler, Olga}!zzzHofmannsthal, Gertrude von@\emph{von Gertrude von Hofmannsthal}!1909-09-131@{13. {[}9.{]} 1909}|(be} \toendnotes[C]{\smallbreak\pagebreak[2]} \Standort{CUL, Schnitzler, B 43.}
\physDesc{Bildpostkarte
\newline{}Handschrift: schwarze Tinte, lateinische Kurrent\newline{}Versand: Stempel: »\nobreak{}\oindex{Bad Aussee@\textbf{Bad Aussee}, \emph{Besiedelter Ort (A.BSO)}|pwk}\textcolor{gray}{Aussee} in der
                                                  Steiermark, 13. {[}9.{]} 09\nobreak{}«.  
\newline{}Schnitzler: mit Bleistift beschriftet: »\textsc{Hofm}« \newline{}Ordnung: 1) mit Bleistift von unbekannter Hand nummeriert:
                                                »379« 2) mit Bleistift von unbekannter Hand nummeriert: »309«}\toendnotes[C]{\smallbreak}\pstart{}{\pb}Frau Olga
                        Schnitzler\pend{}\pstart{}\textcolor{pink}{Wien}{}\ledrightnote{\textcolor{pink}{Wien}}\pend{}\pstart{}\textcolor{pink}{XVIII Spöttlgasse 7}{}\ledrightnote{\textcolor{pink}{Edmund-Weiß-Gasse}}\pend{}{\bigskip}\pstart
           \noindent{}\centering{}{\pb}{[}\textcolor{blue}{Hugo}{}\ledrightnote{\textcolor{blue}{Hugo von Hofmannsthal}} und \textcolor{blue}{Christiane von Hofmannsthal}{}\ledrightnote{\textcolor{blue}{Christiane von Hofmannsthal}} auf einer
                            Wiese.{]}\pend
           \pstart
           {\pb}Liebe Olga, ich danke Ihnen herzlichst für Ihren lieben Brief
                    und für die Auskunft. Die Anfälle bei der \textcolor{blue}{Kleinen}{}\ledrightnote{→\textcolor{blue}{Christiane von Hofmannsthal}} sind gottlob so dass es noch nicht
                    entschieden ist, ob es der \label{K_L01871_1v}\edtext{Keuchhusten}{\lemma{\textnormal{\emph{Keuchhusten}}}\Cendnote{\textnormal{Die Monatsangabe
                        ist am Poststempel nicht zu erkennen. Aber da \textcolor{blue}{Christiane}s Erkrankung auch in einem Brief \textcolor{blue}{Hugo von Hofmannsthal}s an \textcolor{blue}{Helene von Nostitz-Wallwitz} vom
                            12. 9. 1909 Erwähnung findet, kann die Karte datiert
                        werden. (\emph{Hugo von Hofmannsthal – Helene von Nostitz.
                                Briefwechsel.} Herausgegeben von Oswalt von Nostitz.
                            Frankfurt am Main: \emph{\textcolor{brown}{Fischer}}{ }1965, S. 87)}}}\label{K_L01871_1h} ist. Es ko{\geminationm}t einen Abend und in der Nacht, so dass sie am
                    Tag ganz frei davon ist. Ich lasse sie alle drei beisa{\geminationm}en. Ich denke jetzt {\pb}viel an Sie und wir sind sehr
                    traurig, dass wir Sie heuer \introOben{}im Sommer\introOben{} gar nicht
                    gesehen haben, vom \textcolor{blue}{Hugo}{}\ledrightnote{\textcolor{blue}{Hugo von Hofmannsthal}} viele Grüsse an \textcolor{blue}{Arthur}{}\ledrightnote{} und Sie und gute Wünsche\pend
           \pstart
           Ihre{\\[\baselineskip]}\spacefill\mbox{Gerty}\pend
           \leftskip=0em{}\endnumbering\briefempfaengerindex{Schnitzler, Olga@\textsc{Schnitzler, Olga}!zzzHofmannsthal, Gertrude von@\emph{von Gertrude von Hofmannsthal}!1909-09-131@{13. {[}9.{]} 1909}|)be}\mylabel{h}  \normalsize

\doendnotes{C}
\bigskip
\vfill

\clearpage

\footnotesize

\lohead{\textsc{register}}

% Definiere theindex-Environment komplett neu ohne reledmac
\makeatletter
\renewenvironment{theindex}{%
  \section*{\indexname}%
  \setlength{\parindent}{0pt}%
  \setlength{\parskip}{0pt plus 0.3pt}%
  \let\item\@idxitem
}{%
  \clearpage
}
\makeatother

\IfFileExists{\jobname-pw.ind}{\input{\jobname-pw.ind}}{}

\end{document}

      