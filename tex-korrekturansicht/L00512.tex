%% latex-korrekturansicht-vorspann.tex
%% Vorspann für die Korrekturansicht.
%% Lädt die gemeinsame Datei latex-vorspann.tex mit gesetztem Schalter.

\newif\ifkorrekturansicht
\korrekturansichttrue

\input{../tex-inputs/latex-vorspann}


               \section[Arthur Schnitzler an Richard Beer-Hofmann, 7. 11. 1895]{ Arthur Schnitzler an Richard Beer-Hofmann, 7. 11. 1895}\nopagebreak\mylabel{v}\rehead{ }\normalsize\beginnumbering\briefempfaengerindex{Beer-Hofmann, Richard@\textsc{Beer-Hofmann, Richard}!zzzSchnitzler, Arthur@\emph{von Arthur Schnitzler}!1895-11-071@{7. 11. 1895}|(be} \toendnotes[C]{\smallbreak\pagebreak[2]} \Standort{YCGL, MSS 31.}
\physDesc{Postkarte
\newline{}Handschrift: Bleistift, deutsche Kurrent\newline{}Versand: Stempel: »\nobreak{}\oindex{I., Innere Stadt@\textbf{I., Innere Stadt}, \emph{Bezirk (A.BZK)}|pwk}Wien 1/1, 7. 11. 95, 8–9 N\nobreak{}«.  }\toendnotes[C]{\smallbreak}\pstart{}{\pb}Herrn \textsc{Dr. Richard
                     Beer-Hofmann}\pend{}\pstart{}\textcolor{pink}{Wien}{}\ledrightnote{\textcolor{pink}{Wien}}. \pend{}\pstart{}\textsc{\textcolor{pink}{I Wollzeile 15}{}\ledrightnote{\textcolor{pink}{Wollzeile}}}\pend{}{\bigskip}\pstart
           \noindent{}{\pb}Lieber Richard, für den Fall,
               daß ich Sie früher nicht ſehe: den Sitz zu \label{K_L00512_1v}\edtext{\textcolor{green}{Goldene Herzen}{}\ledrightnote{\textcolor{green}{Goldene Herzen. Wiener Volksstück in 4 Akten}}}{\lemma{\textnormal{\emph{Goldene Herzen}}}\Cendnote{\textnormal{Uraufführung am
                     9. 11. 1895, auch \textcolor{blue}{Schnitzler} besuchte die Vorstellung.}}}\label{K_L00512_1h} erhalten Sie
               Samſtag zugeſandt. Auch \textcolor{blue}{\textsc{Zelzer}}{}\ledrightnote{\textcolor{blue}{Georg Zelzer}} hab ich ſchon wegen des \label{K_L00512_2v}\edtext{\textcolor{green}{Winkelglücks}{}\ledrightnote{\textcolor{green}{Das Glück im Winkel}}}{\lemma{\textnormal{\emph{Winkelglücks}}}\Cendnote{\textnormal{Karten für die Uraufführung am
                     9. 11. 1895 am \textcolor{pink}{Burgtheater}}}}\label{K_L00512_2h} geſprochen.\pend
           \pstart
           Herzlich Ihr{\\[\baselineskip]}\spacefill\mbox{Arthur}\pend
           \leftskip=0em{}\endnumbering\briefempfaengerindex{Beer-Hofmann, Richard@\textsc{Beer-Hofmann, Richard}!zzzSchnitzler, Arthur@\emph{von Arthur Schnitzler}!1895-11-071@{7. 11. 1895}|)be}\mylabel{h}  \normalsize

\doendnotes{C}
\bigskip
\vfill

\clearpage

\footnotesize

\lohead{\textsc{register}}

% Definiere theindex-Environment komplett neu ohne reledmac
\makeatletter
\renewenvironment{theindex}{%
  \section*{\indexname}%
  \setlength{\parindent}{0pt}%
  \setlength{\parskip}{0pt plus 0.3pt}%
  \let\item\@idxitem
}{%
  \clearpage
}
\makeatother

\IfFileExists{\jobname-pw.ind}{\input{\jobname-pw.ind}}{}

\end{document}

      