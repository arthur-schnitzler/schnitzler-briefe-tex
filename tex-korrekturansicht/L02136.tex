%% latex-korrekturansicht-vorspann.tex
%% Vorspann für die Korrekturansicht.
%% Lädt die gemeinsame Datei latex-vorspann.tex mit gesetztem Schalter.

\newif\ifkorrekturansicht
\korrekturansichttrue

\input{../tex-inputs/latex-vorspann}


               \section[Robert Adam an Arthur Schnitzler, 7. 5. 1913]{ Robert Adam an Arthur Schnitzler, 7. 5. 1913}\nopagebreak\mylabel{v}\rehead{ }\normalsize\beginnumbering\briefempfaengerindex{Schnitzler, Arthur@\textsc{Schnitzler, Arthur}!zzzAdam, Robert@\emph{von Robert Adam}!1913-05-071@{7. 5. 1913}|(be} \toendnotes[C]{\smallbreak\pagebreak[2]} \Standort{DLA, A:Schnitzler, HS.NZ85.1.4230,7.}
\physDesc{Brief, 1 Blatt, 2 Seiten
\newline{}Handschrift: schwarze Tinte, deutsche Kurrent
\newline{}Schnitzler: mit Bleistift beschriftet: »\textsc{Adam}« }\Standort{Wien, Österreichische Nationalbibliothek, Cod.ser. 52.266, 166.}
\physDesc{handschriftliche Abschrift
\newline{}Handschrift: schwarze Tinte, Gabelsberger Kurzschrift}\Standort{Wien, Österreichische Nationalbibliothek, Cod.ser. 52.266, 166.}
\physDesc{maschinelle Abschrift
\newline{}Schreibmaschine}\toendnotes[C]{\smallbreak}\pstart
           \raggedleft{}{\pb}\textcolor{pink}{Ziſtersdorf}{}\ledrightnote{\textcolor{pink}{Zistersdorf}}, am 7. Mai 1913.
                    \pend
           \pstart{}Hochverehrter Herr Doktor!\pend\pstart
           Nehmen Sie meinen herzlichen Dank für die freundlichen Zeilen, welche die
                    Rückſendung des \textcolor{green}{Manuſkripts}{}\ledrightnote{→\textcolor{green}{Fatme}}
                    begleiteten.\pend
           \pstart
           Trotz ihrer kann ich die Befürchtung nicht abwehren, daß meine krauſe und, wie
                    ich einſehe, mißratene Studie Ihren Beifall nicht gefunden habe. Ich begreife
                    ſehr gut, daß ſie Ihren Künſtlerſinn, deſſen wunderbare Reife ich zuletzt in der
                        \textcolor{green}{Frau Beate}{}\ledrightnote{\textcolor{green}{Frau Beate und ihr Sohn. Novelle}} bewundern durfte, geradezu
                    beleidigt haben muß.\pend
           \pstart
           Vielleicht iſt es mir noch vergönnt, künftighin wieder einmal mit einem
                    ausgeglichenen Produkt vor Sie hinzutreten.\pend
           \pstart
           Genehmigen Sie, hochverehrter Herr {\pb}Doktor, den Ausdruck
                    meiner unbegrenzten Verehrung und meines Dankes!\pend
           \pstart
           Ihr ergebener{\\[\baselineskip]}\spacefill\mbox{Robert Adam}\pend
           \leftskip=0em{}\endnumbering\briefempfaengerindex{Schnitzler, Arthur@\textsc{Schnitzler, Arthur}!zzzAdam, Robert@\emph{von Robert Adam}!1913-05-071@{7. 5. 1913}|)be}\mylabel{h}  \normalsize

\doendnotes{C}
\bigskip
\vfill

\clearpage

\footnotesize

\lohead{\textsc{register}}

% Definiere theindex-Environment komplett neu ohne reledmac
\makeatletter
\renewenvironment{theindex}{%
  \section*{\indexname}%
  \setlength{\parindent}{0pt}%
  \setlength{\parskip}{0pt plus 0.3pt}%
  \let\item\@idxitem
}{%
  \clearpage
}
\makeatother

\IfFileExists{\jobname-pw.ind}{\input{\jobname-pw.ind}}{}

\end{document}

      