%% latex-korrekturansicht-vorspann.tex
%% Vorspann für die Korrekturansicht.
%% Lädt die gemeinsame Datei latex-vorspann.tex mit gesetztem Schalter.

\newif\ifkorrekturansicht
\korrekturansichttrue

\input{../tex-inputs/latex-vorspann}


               \section[Hermann Bahr und Siegfried Trebitsch an Arthur Schnitzler, 2{[}8?{]}. 12. 1903]{ Hermann Bahr und Siegfried Trebitsch an Arthur Schnitzler,
               2{[}8?{]}. 12. 1903}\nopagebreak\mylabel{v}\rehead{ }\normalsize\beginnumbering\briefempfaengerindex{Schnitzler, Arthur@\textsc{Schnitzler, Arthur}!zzzTrebitsch, Siegfried@\emph{von Siegfried Trebitsch}!1903-12-281@{2{[}8?{]}. 12. 1903}|(be}\briefempfaengerindex{Schnitzler, Arthur@\textsc{Schnitzler, Arthur}!zzzBahr, Hermann@\emph{von Hermann Bahr}!1903-12-281@{2{[}8?{]}. 12. 1903}|(be} \toendnotes[C]{\smallbreak\pagebreak[2]} \Standort{CUL, Schnitzler, B 5b.}
\physDesc{Bildpostkarte
\newline{}Handschrift Hermann Bahr: Bleistift, deutsche Kurrent\newline{}Handschrift Siegfried Trebitsch: Bleistift, deutsche Kurrent\newline{}Versand: Stempel: »\nobreak{}\oindex{Semmering@\textbf{Semmering}, \emph{Besiedelter Ort (A.BSO)}|pwk}Semmering, 2\textcolor{gray}{8}. 12. 03, 12–1N\nobreak{}«.  \newline{}Ordnung: mit Bleistift von unbekannter Hand nummeriert: »106« }\buchAbdrucke{\weitereDrucke{Hermann Bahr, Arthur Schnitzler: \emph{Briefwechsel, Aufzeichnungen, Dokumente (1891–1931)}. Hg. Kurt Ifkovits und Martin Anton Müller. Göttingen: \emph{Wallstein} 2018, S. 287.} }\pstart{}{\pb}\textsc{Herrn D\textsuperscript{r}
                        Arthur Schnitzler
                  }\pend{}\pstart{}\textcolor{pink}{\textsc{Wien X\strikeout{X}VIII}}{}\ledrightnote{\textcolor{pink}{XVIII., Währing}}\pend{}\pstart{}\textcolor{pink}{Spöttelgaſſe 7}{}\ledrightnote{\textcolor{pink}{Edmund-Weiß-Gasse}}\pend{}{\bigskip}\pstart
           \noindent{}\centering{}\textcolor{gray}{\textbf{{\pb}Kapelle. \textcolor{pink}{Semmering}{}\ledrightnote{\textcolor{pink}{Semmering}}.}}\pend
           \pstart
           {\pb}Dich u. Deine liebe \textcolor{blue}{Frau}{}\ledrightnote{\textcolor{blue}{Olga Schnitzler}} grüßen nämlich: herzlichſt{\\}\spacefill\mbox{Hermann Bahr}\pend
           \pstart
           {[}hs. Trebitsch:{]} Nicht minder\hspace*{1.5em}Ihr
                  \spacefill\mbox{Trebitsch}\pend
           \endnumbering\briefempfaengerindex{Schnitzler, Arthur@\textsc{Schnitzler, Arthur}!zzzTrebitsch, Siegfried@\emph{von Siegfried Trebitsch}!1903-12-281@{2{[}8?{]}. 12. 1903}|)be}\briefempfaengerindex{Schnitzler, Arthur@\textsc{Schnitzler, Arthur}!zzzBahr, Hermann@\emph{von Hermann Bahr}!1903-12-281@{2{[}8?{]}. 12. 1903}|)be}\mylabel{h}  \normalsize

\doendnotes{C}
\bigskip
\vfill

\clearpage

\footnotesize

\lohead{\textsc{register}}

% Definiere theindex-Environment komplett neu ohne reledmac
\makeatletter
\renewenvironment{theindex}{%
  \section*{\indexname}%
  \setlength{\parindent}{0pt}%
  \setlength{\parskip}{0pt plus 0.3pt}%
  \let\item\@idxitem
}{%
  \clearpage
}
\makeatother

\IfFileExists{\jobname-pw.ind}{\input{\jobname-pw.ind}}{}

\end{document}

      