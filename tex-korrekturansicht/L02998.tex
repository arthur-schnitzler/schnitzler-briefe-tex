%% latex-korrekturansicht-vorspann.tex
%% Vorspann für die Korrekturansicht.
%% Lädt die gemeinsame Datei latex-vorspann.tex mit gesetztem Schalter.

\newif\ifkorrekturansicht
\korrekturansichttrue

\input{../tex-inputs/latex-vorspann}


\renewcommand{\erwaehntePersonen}{Personen: Berta Czegka, Felix Salten, Friedrich von Schiller}
\renewcommand{\erwaehnteOrte}{Orte: Edmund-Weiß-Gasse 7, Wien}
\renewcommand{\erwaehnteWerke}{Werke: Die Zeit, Schiller-Feier, Schiller-Zeit 1805 * 1905, Zum großen Wurstel. Burleske in einem Akt}
\section[ Arthur Schnitzler an Felix Salten, 11. 4. 1905]{Arthur Schnitzler an Felix Salten, 11. 4. 1905}
\nopagebreak\mylabel{v}
\rehead{ }\normalsize\beginnumbering\briefempfaengerindex{Salten, Felix@\textsc{Salten, Felix}!zzzSchnitzler, Arthur@\emph{von Arthur Schnitzler}!1905-04-111@{11. 4. 1905}|(be}
\toendnotes[C]{\smallbreak\pagebreak[2]}\Standort{Wienbibliothek im Rathaus, ZPH 1681, 2.1.516.}
\physDesc{Brief, 1 Blatt, 3 Seiten, 595 Zeichen
\newline{}Handschrift: schwarze Tinte, deutsche Kurrent
\newline{}Ordnung: mit Bleistift von unbekannter Hand Nummerierung der Blätter des Konvoluts:
                                    »26«–»27« }
\buchAbdrucke{\weitereDrucke{Arthur Schnitzler: \emph{Briefe 1875–1912}. Hg. Therese Nickl und Heinrich Schnitzler. Frankfurt am Main: \emph{S. Fischer} 1981, S. 513.} }\toendnotes[C]{\smallbreak}
\pstart
           \noindent{}\textcolor{gray}{\textbf{{\pb}Dr. Arthur Schnitzler}}\hfill 11. 4. 905\pend
           
\pstart
           \textcolor{gray}{\textbf{\textcolor{pink}{Wien, XVIII. Spoettelgasse 7}{}\ledrightnote{\textcolor{pink}{Edmund-Weiß-Gasse 7}}.}}\pend
           
\pstart
           lieber, hiebei etliche \label{K_L02998-1v}\edtext{\textcolor{green}{Diſtichen}{}\ledrightnote{{$\rightarrow$}\textcolor{green}{Schiller-Feier}} für Ihre \textcolor{green}{\textcolor{blue}{Schiller}{}\ledrightnote{\textcolor{blue}{Friedrich von Schiller}}nummer}{}\ledrightnote{\textcolor{green}{Schiller-Zeit 1805 * 1905}}}{\lemma{\textnormal{\emph{Diſtichen … Schillernummer}}}\Cendnote{\textnormal{\textcolor{blue}{Arthur Schnitzler}: \emph{\textcolor{green}{Schiller-Feier}}. In: \emph{\textcolor{green}{Die
                        Zeit}}, Jg. 4, Nr. 926, 23. 4. 1905,
                     Beilage: \emph{\textcolor{green}{Die Schiller-Zeit}},
                     S. VI. Siehe auch A. S.: \emph{»Das Zeitlose ist von kürzester Dauer«}, Schiller-Feier, 23. 4. 1905.}}}\label{K_L02998-1h}, wenn Sie ſie brauchen können.–\pend
           
\pstart
           Werden Sie den \label{K_L02998-2v}\edtext{\textcolor{green}{Wurſtelſpaſs}{}\ledrightnote{{$\rightarrow$}\textcolor{green}{Zum großen Wurstel. Burleske in einem Akt}}}{\lemma{\textnormal{\emph{Wurſtelſpaſs}}}\Cendnote{\textnormal{siehe Arthur Schnitzler an Felix Salten, 8. 2. 1905}}}\label{K_L02998-2h} zu Oſtern bringen? Ich ſchlug Ihnen bei Zuſand
               vor, Bilder dazu machen zu laſſen und wollte mit dem ev. \textcolor{blue}{Illuſtrator}{}\ledrightnote{{$\rightarrow$}\textcolor{blue}{Berta Czegka}} ſelbſt reden. Vielleicht haben
               Sie die Stelle überleſen, ſti{\geminationm}en aber jetzt {\pb}der Bilder\substVorne{}\textsuperscript{\textcolor{gray}{illu}}\substDazwischen{}idee\substHinten{} bei, in welchem Fall man die Sache bis \uline{Pfingſten} laſſen könnte?–\pend
           
\pstart
           Die \uline{Correcturen} erhalte ich doch in jedem Falle?–\pend
           
\pstart
           Herzlichſt {\\[\baselineskip]}Ihr {\\[\baselineskip]}\spacefill\mbox{A.}\pend
           \leftskip=0em{}
\pstart
           Iſt es zu viel verlangt, wenn ich Sie bitte mir auch eine Correctur der \textcolor{green}{Diſtichen}{}\ledrightnote{{$\rightarrow$}\textcolor{green}{Schiller-Feier}} ſchicken zu laſſen?
               In Verſen leiſten die Setzer {\pb}manchmal
               ſeltſames.\pend
           \endnumbering\briefempfaengerindex{Salten, Felix@\textsc{Salten, Felix}!zzzSchnitzler, Arthur@\emph{von Arthur Schnitzler}!1905-04-111@{11. 4. 1905}|)be}\mylabel{h}  \normalsize

\doendnotes{C}
\bigskip
\vfill

\clearpage

\footnotesize

\lohead{\textsc{register}}

% Definiere theindex-Environment komplett neu ohne reledmac
\makeatletter
\renewenvironment{theindex}{%
  \section*{\indexname}%
  \setlength{\parindent}{0pt}%
  \setlength{\parskip}{0pt plus 0.3pt}%
  \let\item\@idxitem
}{%
  \clearpage
}
\makeatother

\IfFileExists{\jobname-pw.ind}{\input{\jobname-pw.ind}}{}

\end{document}

      