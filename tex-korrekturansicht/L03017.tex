%% latex-korrekturansicht-vorspann.tex
%% Vorspann für die Korrekturansicht.
%% Lädt die gemeinsame Datei latex-vorspann.tex mit gesetztem Schalter.

\newif\ifkorrekturansicht
\korrekturansichttrue

\input{../tex-inputs/latex-vorspann}


\renewcommand{\erwaehntePersonen}{Personen: Samuel Fischer, Hedwig Fischer, Felix Salten, Ottilie Salten, Louise Schnitzler}
\renewcommand{\erwaehnteOrte}{Orte: Bad Ischl, Edmund-Weiß-Gasse 7, Frankfurt am Main, Hotel und Pension Rudolfshöhe (Leopold Petter), München, Partenkirchen, Salzburg, Unterach am Attersee}
\renewcommand{\erwaehnteWerke}{}
\section[ Arthur Schnitzler an Felix Salten, 30. 8. 1910]{Arthur Schnitzler an Felix Salten, 30. 8. 1910}
\nopagebreak\mylabel{v}
\rehead{ }\normalsize\beginnumbering\briefempfaengerindex{Salten, Felix@\textsc{Salten, Felix}!zzzSchnitzler, Arthur@\emph{von Arthur Schnitzler}!1910-08-301@{30. 8. 1910}|(be}
\toendnotes[C]{\smallbreak\pagebreak[2]}\Standort{Wienbibliothek im Rathaus, ZPH 1681, 2.1.516.}
\physDesc{Brief, 1 Blatt, 2 Seiten, 517 Zeichen
\newline{}Handschrift: Bleistift, deutsche Kurrent
\newline{}Ordnung: mit Bleistift von unbekannter Hand Nummerierung der Blätter des Konvoluts:
                                    »9« }\toendnotes[C]{\smallbreak}
\pstart
           \noindent{}{\pb}\textcolor{gray}{\textbf{Dr Arthur Schnitzler}}\hfill 30/8 1910\pend
           
\pstart
           \textcolor{gray}{\textbf{\textcolor{pink}{Wien XVIII. Spoettelgasse 7}{}\ledrightnote{\textcolor{pink}{Edmund-Weiß-Gasse 7}}.}}\hfill \textsc{\textcolor{pink}{Ischl}{}\ledrightnote{\textcolor{pink}{Bad Ischl}}, \textcolor{pink}{Pens. Petter}{}\ledrightnote{\textcolor{pink}{Hotel und Pension Rudolfshöhe (Leopold Petter)}}}\pend
           
\pstart
           lieber,{ }\label{K_L03017-1v}\edtext{\textcolor{pink}{Frankfurt}{}\ledrightnote{\textcolor{pink}{Frankfurt am Main}} iſt verſchoben, ſo ſind wir alſo doch von \textsc{\textcolor{pink}{Partenkirchen}{}\ledrightnote{\textcolor{pink}{Partenkirchen}}} über \textsc{\textcolor{pink}{München}{}\ledrightnote{\textcolor{pink}{München}}} – \textsc{\textcolor{pink}{Salzburg}{}\ledrightnote{\textcolor{pink}{Salzburg}}}{ }\textcolor{pink}{hieher}{}\ledrightnote{{$\rightarrow$}\textcolor{pink}{Bad Ischl}}}{\lemma{\textnormal{\emph{Frankfurt … hieher}}}\Cendnote{\textnormal{Die
                  Uraufführung der \emph{\textcolor{green}{Liebelei-Oper}}, vertont durch
                     \textcolor{blue}{Franz Neumann}, wurde auf den 18. 9. 1910
                  verschoben. \textcolor{blue}{Schnitzler} hielt sich dafür
                  zwischen 15. 9. 1910
                  und 19. 9. 1910 in
                     \textcolor{pink}{Frankfurt am Main} auf. In \textcolor{pink}{Bad Ischl} war er zwischen 29. 8. 1910 und 5. 9. 1910.}}}\label{K_L03017-1h}, wo wir ein paar Tage (bei
                  \textcolor{blue}{Mama}{}\ledrightnote{{$\rightarrow$}\textcolor{blue}{Louise Schnitzler}}) bleiben wollen. Zu
               größeren Ausflügen fühlen wir uns nicht friſch genug, nach den macherlei Erregungen
               der letzten Zeit, und ſchlagen Ihnen vor, {\pb}ob
               Sie nicht \textcolor{blue}{Beide}{}\ledrightnote{{$\rightarrow$}\textcolor{blue}{Ottilie Salten}} dieſer Tage,
               etwa \label{K_L03017-2v}\edtext{Donnerſtag oder Freitag{[},{]} zu uns herüber ko{\geminationm}en}{\lemma{\textnormal{\emph{Donnerſtag … kommen}}}\Cendnote{\textnormal{siehe A. S.: \emph{Tagebuch}, 1. 9. 1910}}}\label{K_L03017-2h} möchten?
               Und ob ſich nicht \textcolor{blue}{Fiſchers}{}\ledrightnote{\textcolor{blue}{Samuel Fischer}{\newline}\textcolor{blue}{Hedwig Fischer}}
               anſchließen wollten? Wir würden uns ſehr freuen. Laſſen Sie baldigſt ein Wort
               hören.\pend
           
\pstart
           Herzlichſt Ihr {\\[\baselineskip]}\spacefill\mbox{A.}\pend
           \leftskip=0em{}\endnumbering\briefempfaengerindex{Salten, Felix@\textsc{Salten, Felix}!zzzSchnitzler, Arthur@\emph{von Arthur Schnitzler}!1910-08-301@{30. 8. 1910}|)be}\mylabel{h}  \normalsize

\doendnotes{C}
\bigskip
\vfill

\clearpage

\footnotesize

\lohead{\textsc{register}}

% Definiere theindex-Environment komplett neu ohne reledmac
\makeatletter
\renewenvironment{theindex}{%
  \section*{\indexname}%
  \setlength{\parindent}{0pt}%
  \setlength{\parskip}{0pt plus 0.3pt}%
  \let\item\@idxitem
}{%
  \clearpage
}
\makeatother

\IfFileExists{\jobname-pw.ind}{\input{\jobname-pw.ind}}{}

\end{document}

      