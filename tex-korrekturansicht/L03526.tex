%% latex-korrekturansicht-vorspann.tex
%% Vorspann für die Korrekturansicht.
%% Lädt die gemeinsame Datei latex-vorspann.tex mit gesetztem Schalter.

\newif\ifkorrekturansicht
\korrekturansichttrue

\input{../tex-inputs/latex-vorspann}


\renewcommand{\erwaehntePersonen}{Personen: Paul Goldmann, Paul Lindau, Paul Marx, Olga Schnitzler, Elisabeth Steinrück}
\renewcommand{\erwaehnteInstitutionen}{Institutionen: Berliner Theater, Konservatorium der Gesellschaft der Musikfreunde}
\renewcommand{\erwaehnteOrte}{Orte: Berlin, Dessauer Straße, Wien, Überbrettl}
\renewcommand{\erwaehnteWerke}{Werke: Marionetten. Drei Einakter, Tagebuch, Zum großen Wurstel. Burleske in einem Akt, [Portraitfoto von Olga Gussmann]}
\section[ Paul Goldmann an Olga Gussmann, 9. 3. {[}1901{]}]{Paul Goldmann an Olga Gussmann, 9. 3. {[}1901{]}}
\nopagebreak\mylabel{v}
\rehead{ }\normalsize\beginnumbering\briefempfaengerindex{Schnitzler, Olga@\textsc{Schnitzler, Olga}!zzzGoldmann, Paul@\emph{von Paul Goldmann}!1901-03-091@{9. 3. {[}1901{]}}|(be}
\toendnotes[C]{\smallbreak\pagebreak[2]}\Standort{DLA, A:Schnitzler, HS.NZ85.1.5247.}
\physDesc{Brief, 1 Blatt, 4 Seiten, 1536 Zeichen
\newline{}Handschrift: blaue Tinte, deutsche Kurrent
\newline{}Ordnung: 1) mit Bleistift von \textcolor{blue}{Arthur Schnitzler} das
                                 Jahr »1901« vermerkt  2) mit rotem Buntstift von \textcolor{blue}{Arthur Schnitzler} den
                                 ersten Absatz fast vollständig unterstrichen und mit »\textsc{\textcolor{green}{Marionetten}}« annotiert sowie eine weitere Unterstreichung}\toendnotes[C]{\smallbreak}
\pstart
           \noindent{}\raggedleft{}{\pb}\textcolor{gray}{\textbf{\textcolor{pink}{DESSAUERSTRASSE 19}{}\ledrightnote{\textcolor{pink}{Dessauer Straße}}}}\pend
           
\pstart
           \textcolor{pink}{Berlin}{}\ledrightnote{\textcolor{pink}{Berlin}}, 9. März.\pend
           
\pstart\center{}Liebes Fräulein \textsc{Olga},\pend
\pstart
           \textsc{Dr. \textcolor{blue}{Schnitzlers}{}\ledrightnote{}}{ }\label{K_L03526-1v}\edtext{\textcolor{green}{Stück}{}\ledrightnote{{$\rightarrow$}\textcolor{green}{Zum großen Wurstel. Burleske in einem Akt}}}{\lemma{\textnormal{\emph{Stück}}}\Cendnote{\textnormal{\emph{\textcolor{green}{Zum großen Wurstel}} aus dem \emph{\textcolor{green}{Marionetten}}-Zyklus, am 8. 3. 1901 am \textcolor{pink}{Berlin}er \emph{\textcolor{brown}{Überbrettl}}
                  uraufgeführt}}}\label{K_L03526-1h} kam infolge unzureichender Darſtellung nicht zur rechten
               Wirkung. Auch hatte man die Unverſchämtheit und Taktloſigkeit, es ganz zuletzt, um
                  \strikeout{\textcolor{gray}{½}}{ }½ 11 Uhr Abends, nachdem das Publikum bereits durch ein überlanges
               Programm ermüdet war, aufzuführen.\pend
           
\pstart
           \textsc{Dr. \textcolor{blue}{Schnitzlers}{}\ledrightnote{}}{ }\label{K_L03526-2v}\edtext{Anweſenheit}{\lemma{\textnormal{\emph{Anweſenheit}}}\Cendnote{\textnormal{\textcolor{blue}{Schnitzler} war zwischen 3. 3. 1901 und 10. 3. 1901 in \textcolor{pink}{Berlin}. \textcolor{blue}{Goldmann} traf er nachweislich am 6. 3. 1901, 7. 3. 1901, 8. 3. 1901 und 10. 3. 1901.}}}\label{K_L03526-2h}{ }\strikeout{h\textcolor{gray}{ier}} thut mir ſehr wohl, und ich werde mich nachher nur umſo einſamer fühlen.\pend
           
\pstart
           Ich gratulire Ihnen zu Ihren \label{K_L03526-3v}\edtext{ſchauſpieleriſchen Erfolgen}{\lemma{\textnormal{\emph{ſchauſpieleriſchen Erfolgen}}}\Cendnote{\textnormal{\textcolor{blue}{Gussmann} studierte Schauspiel am \emph{\textcolor{brown}{Konservatorium}}.}}}\label{K_L03526-3h}, von denen Sie mir mit
               ſo überzeugender Beredſamkeit berichten. {\pb}Selbſtverſtändlich werde ich bei \label{K_L03526-4v}\edtext{\textsc{\textcolor{blue}{Lindau}{}\ledrightnote{\textcolor{blue}{Paul Lindau}}}}{\lemma{\textnormal{\emph{Lindau}}}\Cendnote{\textnormal{\textcolor{blue}{Paul Lindau} leitete das \emph{\textcolor{brown}{Berliner Theater}}. Siehe auch A. S.: \emph{Tagebuch}, 3. 8. 1901 und Paul Goldmann an Arthur Schnitzler, 18. 2. [1901].}}}\label{K_L03526-4h}, ſoweit es in meinen
               ſchwachen Kräften ſteht, Ihnen behilflich ſein.\pend
           
\pstart
           Zerbrechen Sie ſich nicht den Kopf über das Künftige. Erſtens nützt es doch nichts,
               und zweitens kommt das Künftige ſchon von ſelbſt, wenn man jung iſt und Talent
               hat.\pend
           
\pstart
           Ich würde mich freuen, wenn Sie nach \textcolor{pink}{Berlin}{}\ledrightnote{\textcolor{pink}{Berlin}}
               kämen. Dann hätte auch ich »doch wenigſtens eine bekannte Seele in der \textcolor{pink}{Stadt}{}\ledrightnote{{$\rightarrow$}\textcolor{pink}{Berlin}}« (wie Sie ſich in Bezug
               auf mich ausdrücken).\pend
           
\pstart
           {\pb}Hoffentlich ſind Sie wieder in guter \label{K_L03526-5v}\edtext{Stimmung}{\lemma{\textnormal{\emph{Stimmung}}}\Cendnote{\textnormal{Womöglich handelt es sich an dieser Stelle um eine Anspielung auf \textcolor{blue}{Olga Gussmanns} Eifersucht, die sie \textcolor{blue}{Schnitzler} gegenüber, wie dem \emph{\textcolor{green}{Tagebuch}} zu entnehmen ist, in dieser Zeit mehrfach äußerte.}}}\label{K_L03526-5h}, wenn
               dieſer Brief ankommt. Iſt das Leben wirklich ſo bitter? Ich finde aber, alle
               Bitterkeit macht auch nichts, wenn es \strikeout{\textcolor{gray}{richtig}}{ }\introOben{}nur\introOben{} hier und da einen ſüßen Schluck gibt. Nur ganz ohne \strikeout{\textcolor{gray}{Schluck}} ſüßen Schluck iſt es ſchwer zu tragen.\pend
           
\pstart
           Ihr \label{K_L03526-6v}\edtext{\textcolor{green}{Bild}{}\ledrightnote{{$\rightarrow$}\textcolor{green}{[Portraitfoto von Olga Gussmann]}}}{\lemma{\textnormal{\emph{Bild}}}\Cendnote{\textnormal{Siehe Paul Goldmann an Olga Gussmann, 10. 5. [1901].
               }}}\label{K_L03526-6h} ſoll willkommen ſein.\pend
           
\pstart
           Ich habe Ihnen lange nicht geantwortet, weil ich wenig Zeit zum Schreiben habe und
               weil – weil – weil ich nicht recht wußte, {\pb}was ich
               Ihnen antworten ſollte.\pend
           
\pstart
           Grüßen Sie Ihr \textcolor{blue}{Schweſterlein}{}\ledrightnote{{$\rightarrow$}\textcolor{blue}{Elisabeth Steinrück}} und ſeien Sie ſelbſt recht herzlich gegrüßt von {\\[\baselineskip]}Ihrem
               ergebenen {\\[\baselineskip]}\spacefill\mbox{Dr. Paul Goldmann}\pend
           \leftskip=0em{}
\pstart
           \noindent{}Grüße an Herrn \label{K_L03526-7v}\edtext{\textsc{\textcolor{blue}{Paul}{}\ledrightnote{\textcolor{blue}{Paul Marx}}}}{\lemma{\textnormal{\emph{Paul}}}\Cendnote{\textnormal{\textcolor{blue}{Paul Marx} war zwischen 1900 und 1903 der Partner von \textcolor{blue}{Olgas} Schwester \textcolor{blue}{Elisabeth Gussmann} und, wie \textcolor{blue}{Olga und Elisabeth}, Schüler am \emph{\textcolor{brown}{Konservatorium}}.}}}\label{K_L03526-7h}!\pend
           \endnumbering\briefempfaengerindex{Schnitzler, Olga@\textsc{Schnitzler, Olga}!zzzGoldmann, Paul@\emph{von Paul Goldmann}!1901-03-091@{9. 3. {[}1901{]}}|)be}\mylabel{h}  \normalsize

\doendnotes{C}
\bigskip
\vfill

\clearpage

\footnotesize

\lohead{\textsc{register}}

% Definiere theindex-Environment komplett neu ohne reledmac
\makeatletter
\renewenvironment{theindex}{%
  \section*{\indexname}%
  \setlength{\parindent}{0pt}%
  \setlength{\parskip}{0pt plus 0.3pt}%
  \let\item\@idxitem
}{%
  \clearpage
}
\makeatother

\IfFileExists{\jobname-pw.ind}{\input{\jobname-pw.ind}}{}

\end{document}

      