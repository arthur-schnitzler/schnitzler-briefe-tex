%% latex-korrekturansicht-vorspann.tex
%% Vorspann für die Korrekturansicht.
%% Lädt die gemeinsame Datei latex-vorspann.tex mit gesetztem Schalter.

\newif\ifkorrekturansicht
\korrekturansichttrue

\input{../tex-inputs/latex-vorspann}


\renewcommand{\erwaehntePersonen}{Personen: Richard Metzl, Felix Salten, Ottilie Salten}
\renewcommand{\erwaehnteOrte}{Orte: Adriatisches Meer, Bansin, Dresden, Edmund-Weiß-Gasse 7, Eisenach, Hamburg, Kantstraße, Lübeck, Reichenau an der Rax, Venedig, Weimar, Wien, XVIII., Währing}
\renewcommand{\erwaehnteWerke}{}
\section[ Felix Salten an Arthur Schnitzler, 23. 8. 1906]{Felix Salten an Arthur Schnitzler, 23. 8. 1906}
\nopagebreak\mylabel{v}
\rehead{ }\normalsize\beginnumbering\briefempfaengerindex{Schnitzler, Arthur@\textsc{Schnitzler, Arthur}!zzzSalten, Felix@\emph{von Felix Salten}!1906-08-231@{23. 8. 1906}|(be}
\toendnotes[C]{\smallbreak\pagebreak[2]}\Standort{CUL, Schnitzler, B 89, B 1.}
\physDesc{Postkarte, 968 Zeichen
\newline{}Handschrift: schwarze Tinte, lateinische Kurrent
\newline{}Versand: Stempel: »\nobreak{}\oindex{Bansin@\textbf{Bansin}, \emph{P.PPL}|pwk}Seebad Ba\textcolor{gray}{nsin}, 23. 8. 06\nobreak{}«. Stempel: »\nobreak{}\oindex{XVIII., Waehring@\textbf{XVIII., Währing}, \emph{A.ADM3}|pwk}18/\textsubscript{1} Wien 110, 25. VIII. 06, X, Bestellt\nobreak{}«.  
\newline{}Ordnung: mit Bleistift von unbekannter Hand nummeriert: »224« }\toendnotes[C]{\smallbreak}\pstart{}{\pb}Herrn D\textsuperscript{r} Arthur Schnitzler\pend{}\pstart{}\textcolor{pink}{Wien}{}\ledrightnote{\textcolor{pink}{Wien}}\pend{}\pstart{}\textcolor{pink}{XVIII. Spoettelgasse 7}{}\ledrightnote{\textcolor{pink}{Edmund-Weiß-Gasse 7}}\pend{}
{\bigskip}
\pstart
           \raggedleft{}{\pb}\textcolor{pink}{Bansin}{}\ledrightnote{\textcolor{pink}{Bansin}}, 23. VIII. 06. \pend
           
\pstart
           Lieber, schönen Dank für Ihre Karten aus \label{K_L03433-1v}\edtext{\textcolor{pink}{Weimar}{}\ledrightnote{\textcolor{pink}{Weimar}}}{\lemma{\textnormal{\emph{Weimar}}}\Cendnote{\textnormal{\textcolor{blue}{Schnitzler}s Aufenthalt in \textcolor{pink}{Weimar} fand zwischen 12. 8. 1906 und 16. 8. 1906 statt.}}}\label{K_L03433-1h}. Wir bleiben noch
               ca 10–12 Tage \textcolor{pink}{hier}{}\ledrightnote{{$\rightarrow$}\textcolor{pink}{Bansin}}, gehen
               dann nach \textcolor{pink}{Lübeck}{}\ledrightnote{\textcolor{pink}{Lübeck}} u. \textcolor{pink}{Hamburg}{}\ledrightnote{\textcolor{pink}{Hamburg}}, dann nach \textcolor{pink}{Weimar}{}\ledrightnote{\textcolor{pink}{Weimar}}
               und \textcolor{pink}{Eisenach}{}\ledrightnote{\textcolor{pink}{Eisenach}}. Zuletzt begleitet mich \textcolor{blue}{Otti}{}\ledrightnote{\textcolor{blue}{Ottilie Salten}} nach \textcolor{pink}{Dresden}{}\ledrightnote{\textcolor{pink}{Dresden}}. Ich bin gegen den 10. Septb. in \textcolor{pink}{Wien}{}\ledrightnote{\textcolor{pink}{Wien}}, und fahre – wahrscheinlich – zu den
               Flottenmanövern in der \textcolor{pink}{Adria}{}\ledrightnote{\textcolor{pink}{Adriatisches Meer}}. Von da noch ein
               paar Tage \textcolor{pink}{Venedig}{}\ledrightnote{\textcolor{pink}{Venedig}}, dann \label{K_L03433-2v}\edtext{definitiv \textcolor{pink}{Wien}{}\ledrightnote{\textcolor{pink}{Wien}}}{\lemma{\textnormal{\emph{definitiv Wien}}}\Cendnote{\textnormal{Der seit Jahresbeginn dauernde Aufenthalt
                     in \textcolor{pink}{Berlin} beendet und der Wohnsitz wieder nach
                     \textcolor{pink}{Wien} verlegt, vgl. A. S.: \emph{Tagebuch}, 2. 8. 1906.}}}\label{K_L03433-2h}. Wenn das Wetter schön bleibt, könnten Sie
               wegen eines Tennisplatzes (Vormittag) etwas veranlaßen. Mein Schwager \textcolor{blue}{Richard}{}\ledrightnote{\textcolor{blue}{Richard Metzl}}, der in \textcolor{pink}{Reichenau}{}\ledrightnote{\textcolor{pink}{Reichenau an der Rax}} mit uns spielte, spielt jetzt noch schärfer und wird ein guter
               Partner sein. \textcolor{blue}{Otti}{}\ledrightnote{\textcolor{blue}{Ottilie Salten}} übersiedelt, Sack und Pack,
               am 14. September. Wir sind unsere Wohnung in der \textcolor{pink}{Kantstraße}{}\ledrightnote{\textcolor{pink}{Kantstraße}} los; müßen sie am 14. schon räumen. Eine Chance! Denn ich hätte sonst die
               ganze Miete für die restliche Vertragszeit, also 5000.- M. vor meiner Abreise
               deponiren müßen, u. hätte dann wer weiß wie viel verloren. Auf bald. Herzliche Grüße
               von uns zu Ihnen. {\\}Ihr {\\}\spacefill\mbox{Salten}\pend
           \endnumbering\briefempfaengerindex{Schnitzler, Arthur@\textsc{Schnitzler, Arthur}!zzzSalten, Felix@\emph{von Felix Salten}!1906-08-231@{23. 8. 1906}|)be}\mylabel{h}  \normalsize

\doendnotes{C}
\bigskip
\vfill

\clearpage

\footnotesize

\lohead{\textsc{register}}

% Definiere theindex-Environment komplett neu ohne reledmac
\makeatletter
\renewenvironment{theindex}{%
  \section*{\indexname}%
  \setlength{\parindent}{0pt}%
  \setlength{\parskip}{0pt plus 0.3pt}%
  \let\item\@idxitem
}{%
  \clearpage
}
\makeatother

\IfFileExists{\jobname-pw.ind}{\input{\jobname-pw.ind}}{}

\end{document}

      