%% latex-korrekturansicht-vorspann.tex
%% Vorspann für die Korrekturansicht.
%% Lädt die gemeinsame Datei latex-vorspann.tex mit gesetztem Schalter.

\newif\ifkorrekturansicht
\korrekturansichttrue

\input{../tex-inputs/latex-vorspann}


               \section[Paul Goldmann an Arthur Schnitzler, Paul Goldmann an Arthur Schnitzler, 2. 12. {[}1896{]}]{ Paul Goldmann an Arthur Schnitzler, 2. 12. {[}1896{]}}\nopagebreak\mylabel{v}\rehead{ }\normalsize\beginnumbering\briefempfaengerindex{Schnitzler, Arthur@\textsc{Schnitzler, Arthur}!zzzGoldmann, Paul@\emph{von Paul Goldmann}!1896-12-021@{2. 12. {[}1896{]}}|(be} \toendnotes[C]{\smallbreak\pagebreak[2]} \Standort{DLA, A:Schnitzler, HS.NZ85.1.3166.}
\physDesc{Brief, 3 Blätter, 9 Seiten
\newline{}Handschrift: blaue Tinte, deutsche Kurrent\newline{}Beilage: aufgeklebter Ausschnitt aus dem Brief von \textcolor{blue}{Recha Wolff} an \textcolor{blue}{Theodor Wolff}, schwarze Tinte, deutsche
                                 Kurrent 
\newline{}Schnitzler: 1) mit Bleistift das Jahr »96« vermerkt 2) mir rotem Buntstift vier Unterstreichungen}\toendnotes[C]{\smallbreak}\pstart
           \noindent{}{\pb}\textcolor{gray}{\textbf{\textbf{\textcolor{brown}{Frankfurter Zeitung}{}\ledrightnote{\textcolor{brown}{Frankfurter Zeitung}}}}}\pend
           \pstart
           \textcolor{gray}{\textbf{(\textcolor{brown}{\begin{otherlanguage}{french}Gazette de Francfort\end{otherlanguage}}{}\ledrightnote{\textcolor{brown}{Frankfurter Zeitung}}).}}\pend
           \pstart
           \textcolor{gray}{\textbf{\textbf{\begin{otherlanguage}{french}Fondateur M.\end{otherlanguage}{ }\textcolor{blue}{L. Sonnemann}{}\ledrightnote{\textcolor{blue}{Leopold Sonnemann}}.}}}\pend
           \pstart
           \begin{otherlanguage}{french}\textcolor{gray}{\textbf{\textcolor{green}{Journal}{}\ledrightnote{→\textcolor{green}{Frankfurter Zeitung}} politique,
                        financier,}}\end{otherlanguage}\pend
           \pstart
           \begin{otherlanguage}{french}\textcolor{gray}{\textbf{commercial et littéraire.}}\end{otherlanguage}\pend
           \pstart
           \begin{otherlanguage}{french}\textcolor{gray}{\textbf{\textbf{Paraissant trois fois par jour.}}}\end{otherlanguage}\hfill \textsc{\textcolor{pink}{Paris}{}\ledrightnote{\textcolor{pink}{Paris}}}, 2. December.\pend
           \pstart
           \begin{otherlanguage}{french}\textcolor{gray}{\textbf{\textbf{Bureau à \textcolor{pink}{Paris}{}\ledrightnote{\textcolor{pink}{Paris}}}}}\end{otherlanguage}\pend
           \pstart
           \begin{otherlanguage}{french}\textcolor{gray}{\textbf{\textbf{\textcolor{pink}{24. Rue Feydeau}{}\ledrightnote{\textcolor{pink}{rue Feydeau}}.}}}\end{otherlanguage}\pend
           \pstart\center{}Mein lieber Freund,\pend\pstart
           Mir ſcheint, in meinen letzten Brief hat ſich ſehr gegen meinen Willen ein falſcher
               Ton eingeſchlichen. Du haſt etwas vom »Berühmtwerden« herausgehört? Ich ſchwöre Dir,
               ich bin durchdrungen von der Nichtigkeit und \label{K_L02794-111v}\edtext{Unbedeutenheit}{\lemma{\textnormal{\emph{Unbedeutenheit}}}\Cendnote{\textnormal{zu der Zeit längst veraltete Form von
                  »Unbedeutendheit«}}}\label{K_L02794-111h} aller jener Vorgänge. Ich habe mich ſogar im Verdacht,
               daß ich ein \strikeout{we\textcolor{gray}{n}} wenig Komödie geſpielt habe. Ich \strikeout{\textcolor{gray}{×}} glaube, ich hätte mich vielleicht doch nicht geſchlagen, wenn ich nicht gar ſo
               ſicher darauf gerechnet hätte, der Andere werde mich nicht erſchießen. Du wirſt ja
               ſelbſt auch ſehen, wie raſch das Alles vergeſſen werden {\pb}wird, wie bald ich in mein Dunkel zurückkehren
               werde, nachdem ein flüchtiger Lichtſtrahl von draußen auf mich gefallen. Ich glaube
               ſogar, ich habe es von Anfang an ein wenig auf dieſen Lichtſtrahl angelegt. Ich habe
               für Gerechtigkeit eintreten und zugleich nur etwas \strikeout{Rekla} Reklame machen wollen. Ich habe mit ſchlauer Berechnung von Anfang an
               geſehen, daß die ganze Angelegenheit ein gutes Mittel ſei, auf anſtändige Weiſe von
               mir reden zu machen. Gewiß war auch die Empörung über das Unrecht dabei. Ich will
               mich nicht ſchlechter machen, als ich bin, aber Du machſt {\pb}mich viel zu gut. Etwas Derartiges, wie Deinen
               entzückenden Glückwunſchbrief von neulich habe ich nicht verdient. So wie ich Dirs
               eben geſagt, ſtehen die Dinge und nicht anders, und ich möchte nicht, daß es einen
               Schatten von Unehrlichkeit gebe zwiſchen Dir und mir.\pend
           \pstart
           Jetzt will ich Dir noch ſagen, daß ich geſtern einen
               Brief von \textsc{\textcolor{blue}{Georg Brandes}{}\ledrightnote{\textcolor{blue}{Georg Brandes}}} erhielt, worin er mir, zu meiner freudigen Überraſchung, ſchreibt, er habe mich
                  \label{K_L02794-1v}\edtext{in \textsc{\textcolor{pink}{Kopenhagen}{}\ledrightnote{\textcolor{pink}{Kopenhagen}}} liebgewonnen}{\lemma{\textnormal{\emph{in … liebgewonnen}}}\Cendnote{\textnormal{Im Rahmen der
                  Skandinavien-Reise im Sommer 1896 traf \textcolor{blue}{Goldmann} auch auf \textcolor{blue}{Georg
                     Brandes}, jedenfalls am 21. 8. 1896.}}}\label{K_L02794-1h}; will Dir außerdem ſagen, daß ich \label{K_L02794-2v}\edtext{\textsc{\textcolor{blue}{Herzl}{}\ledrightnote{\textcolor{blue}{Theodor Herzl}}s} Art, mich jetzt zu {\pb}überſchätzen}{\lemma{\textnormal{\emph{Herzls … überſchätzen}}}\Cendnote{\textnormal{gemeint ist wohl: nach dem Pistolenduell}}}\label{K_L02794-2h}, ebenſo
               lächerlich finde, wie ſeine bisherige Art, mich zu unterſchätzen (der Mann iſt immer
               urtheilslos, ſo oder ſo); und will Dich erſuchen, dem \label{K_L02794-3v}\edtext{\textcolor{green}{Artikel}{}\ledrightnote{→\textcolor{green}{Verschiedenes [Goldmann und Millevoye]}} des »\textsc{\textcolor{green}{Figaro}{}\ledrightnote{\textcolor{green}{Le Figaro}}}«, den Du im \strikeout{Bo}{ }\textcolor{green}{Börſen-Courier}{}\ledrightnote{→\textcolor{green}{Berliner Börsen-Zeitung}} gefunden}{\lemma{\textnormal{\emph{Artikel … gefunden}}}\Cendnote{\textnormal{\textcolor{blue}{Maurice Leudet}: \emph{\textcolor{green}{L’Affaire Millevoye-Goldmann}}. In: \emph{\textcolor{green}{Le Figaro}}, Jg. 42, Nr. 326, 21. 11. 1896, S. 1–2. [o. V.]: \emph{\textcolor{green}{Verschiedenes}}. In: \emph{\textcolor{green}{Berliner Börsen-Zeitung}}, Jg. 42, Nr. 531, 24. 11. 1896, Morgen-Ausgabe, S. 12.}}}\label{K_L02794-3h}, nicht das
               mindeſte Gewicht beizulegen. Im »\textsc{\textcolor{green}{Figaro}{}\ledrightnote{\textcolor{green}{Le Figaro}}}« werden ſolche Dinge nur gedruckt, wenn man ſie bezahlt. Der \textcolor{blue}{Mann}{}\ledrightnote{→\textcolor{blue}{Maurice Leudet}}, der dieſen \textcolor{green}{Artikel}{}\ledrightnote{→\textcolor{green}{Verschiedenes [Goldmann und Millevoye]}} geſchrieben, iſt ein erbärmliches
               Subject, unfähig, irgend Jemandem aus freien Stücken Gerechtigkeit zu erweiſen. Ich
               vermuthe, daß der \textcolor{green}{Artikel}{}\ledrightnote{→\textcolor{green}{Verschiedenes [Goldmann und Millevoye]}} von
               der Familie \textsc{\textcolor{blue}{Dreyfus}{}\ledrightnote{→\textcolor{blue}{Alfred Dreyfus}}} herrührt, {\pb}und wenn man ihn aufmerkſam lieſt,
               ſo iſt er \strikeout{\textcolor{gray}{ein}}, unter dem Vorwand \strikeout{\textcolor{gray}{v}} von mir zu ſprechen, ein geſchicktes \textsc{\textcolor{green}{Plaidoyer}{}\ledrightnote{→\textcolor{green}{Verschiedenes [Goldmann und Millevoye]}}} für den \strikeout{Verurt}{ }\textcolor{blue}{Verurtheilten}{}\ledrightnote{→\textcolor{blue}{Alfred Dreyfus}}. Und nun wollen
               wir kein Wort mehr von der ganzen Geſchichte reden, nicht wahr?\pend
           \pstart
           Nach \strikeout{alle} Allem, was in den letzten Wochen \label{K_L02794-44v}\edtext{zwiſchen mir und mir}{\lemma{\textnormal{\emph{zwiſchen mir und mir}}}\Cendnote{\textnormal{vermutlich eine wörtliche Übersetzung von
                     »entre moi et moi-même«,}}}\label{K_L02794-44h} geſtanden, bin ich jetzt wieder
               allein \label{K_L02794-6v}\edtext{\begin{otherlanguage}{french}\textsc{en tête-à-tête avec moi-même}\end{otherlanguage}}{\lemma{\textnormal{\emph{en … moi-même}}}\Cendnote{\textnormal{französisch: mit mir selbst von
                  Angesicht zu Angesicht}}}\label{K_L02794-6h}. Und da ſehe ich erſt ganz deutlich, daß alles
               Äußere Schwindel war, und daß ich unfähig bin {\pb}zur
               wahren Leiſtung: ein gutes Buch, ein gutes Stück. Und nicht einmal die Liebe will
               kommen. Nie, nie ein geliebtes Weſen in die Arme geſchloſſen! Und \label{K_L02794-89v}\edtext{morgen iſt die
               Jugend zu Ende}{\lemma{\textnormal{\emph{morgen … Ende}}}\Cendnote{\textnormal{metaphorisch gemeint, er hatte nicht Geburtstag.}}}\label{K_L02794-89h}! Und es will nicht kommen! Das iſt troſtlos; und dann gehts recht
               ſchlimm mit meinen Augen, und ich fürchte, blind zu werden{\dots}\pend
           \pstart
           Entſchuldige, daß ich Dir gar ſo viel von mir ſpreche. Ich freue mich, zu hören, daß
               Du wieder arbeiteſt und daß Dir die Arbeit ſeeliſch gut thut. Die \label{K_L02794-7v}\edtext{\textcolor{green}{Sachen}{}\ledrightnote{→\textcolor{green}{Reigen. Zehn Dialoge}}, mit denen Du
               beſchäftigt biſt}{\lemma{\textnormal{\emph{Sachen, … biſt}}}\Cendnote{\textnormal{Am 23. 11. 1896 begann \textcolor{blue}{Schnitzler} am \emph{\textcolor{green}{Reigen}} zu schreiben. Enthusiasmus für dieses neue \textcolor{green}{Stück} klingt etwa im \emph{\textcolor{green}{Tagebuch}}-Eintrag vom 27. 11. 1896 durch: »Schrieb mit Laune
                        die \textcolor{green}{4. Scene} des \textcolor{green}{Hemic}.«.}}}\label{K_L02794-7h},
               dürften {\pb}Dir ſehr »liegen«. Wie denkſt Du aber doch
               über das hiſtoriſche \strikeout{Wie}{ }\label{K_L02794-9v}\edtext{\textcolor{pink}{Wien}{}\ledrightnote{\textcolor{pink}{Wien}}er Stück}{\lemma{\textnormal{\emph{Wiener Stück}}}\Cendnote{\textnormal{siehe A. S.: \emph{Tagebuch}, 22. 11. 1896}}}\label{K_L02794-9h}? Vielleicht mit einem jungen Componiſten, der ein Bischen alte und neue \textcolor{pink}{Wien}{}\ledrightnote{\textcolor{pink}{Wien}}er Muſik dazu machen würde? Würde Dich dieſe
               Abwechſelung nicht einmal reizen? Oder willſt Du fürs Erſte überhaupt kein größeres
               Stück ſchreiben? Auch das würde ich ſehr billigen. Und wann kommt Dein \label{K_L02794-11v}\edtext{Buch bei \textsc{\textcolor{brown}{Fischer}{}\ledrightnote{\textcolor{brown}{S. Fischer Verlag}}}}{\lemma{\textnormal{\emph{Buch bei Fischer}}}\Cendnote{\textnormal{Im August 1896 vereinbarten 
                  \textcolor{blue}{S. Fischer} und \textcolor{blue}{Schnitzler}
                  eine Sammlung seiner Novelletten als Buch zu veröffentlichen. \emph{\textcolor{green}{Die Frau des
                  Weisen}} erschien aber erst am 3. 5. 1898.}}}\label{K_L02794-11h}?\pend
           \pstart
           Wer iſt dieſer \textsc{\textcolor{blue}{Stephan Grossmann}{}\ledrightnote{\textcolor{blue}{Stefan Großmann}}}, den Du mir geſchickt haſt? Ich habe mich für ihn verwendet und heut wird mir ein \label{K_L02794-12v}\edtext{Zeitungs-Ausſchnitt}{\lemma{\textnormal{\emph{Zeitungs-Ausſchnitt}}}\Cendnote{\textnormal{Mehrere Tageszeitungen berichteten über die Verhaftung des
                  Anarchisten und Journalisten \textcolor{blue}{Stephan
                     Großmann} in \textcolor{pink}{Berlin}. Siehe etwa
                     o. V.: \emph{\textcolor{green}{Verhaftung eines Wiener
                        Anarchisten in Berlin}}. In: \emph{\textcolor{green}{Arbeiter-Zeitung}}, Jg. 8, Nr. 297, 28. 10. 1896, Morgenblatt, S. 5–6.}}}\label{K_L02794-12h} geſchickt, worin
               ſteht, daß {\pb}er ſich der \textcolor{pink}{Berlin}{}\ledrightnote{\textcolor{pink}{Berlin}}er \textcolor{brown}{Polizei}{}\ledrightnote{→\textcolor{brown}{Polizeidirektion Berlin}} als Spitzel angeboten habe. \strikeout{H\textcolor{gray}{×}\-\textcolor{gray}{×}} Ich habe ihm geſagt, daß er, da er mit einer Empfehlung von Dir bei mir
               erſchienen iſt, \strikeout{v\textcolor{gray}{o}n
                     v\textcolor{gray}{o}} in meinen Augen von Vornherein gegen alle Zeitungen Recht hat. Aber er hat
               ſich \strikeout{\textcolor{gray}{m}i\textcolor{gray}{s}} ungeſchickt gerechtfertigt; das kann freilich auch Befangenheit ſein; \strikeout{i\textcolor{gray}{mme}} darum möchte ich gern in zwei Worten hören, wie Du über den Fall denkſt?\pend
           \pstart
           Iſt es wahr, daß die \label{K_L02794-14v}\edtext{»\textcolor{brown}{Allgemeine Zeitung}{}\ledrightnote{→\textcolor{brown}{Wiener Allgemeine Zeitung}}« in andere Hände}{\lemma{\textnormal{\emph{»Allgemeine … Hände}}}\Cendnote{\textnormal{Mit Jahresende 1896 übergab der mit einer Cousine \textcolor{blue}{Schnitzler}s
                  verheiratete \textcolor{blue}{Julius Gans-Ludassy} die Herausgabe der \emph{\textcolor{brown}{Wiener Allgemeinen Zeitung}} an \textcolor{blue}{August Krawani},
                  der zu diesem Zeitpunkt beinahe siebzig Jahre alt war. Seit einem Jahr im Amt war \textcolor{blue}{Julian Sternberg} als
                  Chefredakteur, er wurde am
                  30. 6. 1897 von \textcolor{blue}{Josef Münz} abgelöst. Die Personalwechsel 
                  bedeuteten für \textcolor{blue}{Salten}, der seit 1894 am Blatt mitarbeitete,
                  zu verschiedenen Zeiten unterschiedliche Aufgaben, er verlor aber
                  seine Stelle nicht.}}}\label{K_L02794-14h} übergeht? Was wird aus \textsc{\textcolor{blue}{Salten}{}\ledrightnote{\textcolor{blue}{Felix Salten}}}?{\dots}\pend
           \pstart
           Sei nochmals von ganzem Herzen bedankt für Deine treue Antheilnahme an den letzten
               Vorgängen. Tauſend herzliche Grüße! Dein \spacefill\mbox{Paul Goldmann}\pend
           \pstart
           \label{T_L02794-1v}\edtext{Grüße \textsc{\textcolor{blue}{Richard}{}\ledrightnote{\textcolor{blue}{Richard Beer-Hofmann}}} und \textsc{\textcolor{blue}{Leo}{}\ledrightnote{\textcolor{blue}{Leo Van-Jung}}}! Und schreib’ mir recht bald!}{\lemma{\textnormal{\emph{Grüße … bald!}}}\Cendnote{\textnormal{seitlich am linken Rand}}}\label{T_L02794-1h}\pend
           \pstart
           \label{T_L02794-2v}\edtext{Die \label{K_L02794-23v}\edtext{Kritiken}{\lemma{\textnormal{\emph{Kritiken}}}\Cendnote{\textnormal{Rezensionen der Uraufführung von \emph{\textcolor{green}{Freiwild}}}}}\label{K_L02794-23h} ſende ich Dir demnächſt zurück}{\lemma{\textnormal{\emph{Die … zurück}}}\Cendnote{\textnormal{kopfüber am oberen Rand}}}\label{T_L02794-2h}\pend
           {\bigskip}\pstart
           \noindent{}{\pb}\label{K_L02790-8765v}\edtext{Dies iſt ein Ausſchnitt}{\lemma{\textnormal{\emph{Dies iſt ein Ausſchnitt}}}\Cendnote{\textnormal{Die Ergänzung dieses undatierten Blattes
                  zu diesem Brief muss gerechtfertigt werden. Als eigenes Korrespondenzstück wirkt
                  es zu zusammenhanglos und entspricht es auch nicht den sonstigen Usancen der
                  Korrespondenz, solche Petitessen separat zu senden. Die inhärente Datierung des
                  Briefs von \textcolor{blue}{Recha Wolff} auf den Tag nach
                  einer \textcolor{pink}{Berlin}er Aufführung von \emph{\textcolor{green}{Freiwild}} erlaubt, ihren Brief zeitlich einzugrenzen, da
                  das Stück zwischen 3. 11. 1896 und dem 16. 11. 1896 am \textcolor{pink}{Deutschen
                     Theater} am Programm stand. Entsprechend könnte der Ausschnitt jedem der
                  Briefstücke des Novembers 1896 zugeordnet werden, doch unterscheidet
                  sich der Farbton der von \textcolor{blue}{Goldmann}
                  verwendeten Tinte deutlich von jener, die auf diesem Blatt zum Einsatz kommt.
                  Dieser stimmt mit dem vorliegenden Brief weitgehend überein, dass damit die
                  Zuordnung vorgenommen wurde.}}}\label{K_L02790-8765h} aus einem Briefe, den mein College \textsc{\textcolor{blue}{Th. Wolff}{}\ledrightnote{\textcolor{blue}{Theodor Wolff}}} dieſer Tage von ſeiner \textcolor{blue}{Mutter}{}\ledrightnote{→\textcolor{blue}{Recha Wolff}} erhalten hat:\pend
           \pstart
           \noindent{}{[}hs. Wolff:{]} recht zu ſagen. Gestern war ich mit \textsc{\textcolor{blue}{Martha}{}\ledrightnote{\textcolor{blue}{Marta Wolff}}} am \textcolor{pink}{Deutſchen Theater}{}\ledrightnote{\textcolor{pink}{Deutsches Theater Berlin}}, \textcolor{gray}{wo}
               wir einen wirklichen Genuß hatten. »\textcolor{green}{Freiwild}{}\ledrightnote{\textcolor{green}{Freiwild. Schauspiel in 3 Akten}}«
               von Schnitzler iſt das Schönſte, was ich ſeit lange geſehen, und geſpielt wurde
               geradezu vollendet\pend
           \endnumbering\briefempfaengerindex{Schnitzler, Arthur@\textsc{Schnitzler, Arthur}!zzzGoldmann, Paul@\emph{von Paul Goldmann}!1896-12-021@{2. 12. {[}1896{]}}|)be}\mylabel{h}  \normalsize

\doendnotes{C}
\bigskip
\vfill

\clearpage

\footnotesize

\lohead{\textsc{register}}

% Definiere theindex-Environment komplett neu ohne reledmac
\makeatletter
\renewenvironment{theindex}{%
  \section*{\indexname}%
  \setlength{\parindent}{0pt}%
  \setlength{\parskip}{0pt plus 0.3pt}%
  \let\item\@idxitem
}{%
  \clearpage
}
\makeatother

\IfFileExists{\jobname-pw.ind}{\input{\jobname-pw.ind}}{}

\end{document}

      