%% latex-korrekturansicht-vorspann.tex
%% Vorspann für die Korrekturansicht.
%% Lädt die gemeinsame Datei latex-vorspann.tex mit gesetztem Schalter.

\newif\ifkorrekturansicht
\korrekturansichttrue

\input{../tex-inputs/latex-vorspann}


\section[Elsa Plessner an Arthur Schnitzler, 18. 3. 1896]{L03699 Elsa Plessner an Arthur Schnitzler, 18. 3. 1896}
\nopagebreak\mylabel{L03699v}
\rehead{ }\normalsize\beginnumbering\briefempfaengerindex{Schnitzler, Arthur@\textsc{Schnitzler, Arthur}!zzzPlessner, Elsa@\emph{von Elsa Plessner}!1896-03-181@{18. 3. 1896}|(be}
\toendnotes[C]{\smallbreak\pagebreak[2]}
\correspDesc{Versand  durch Elsa Plessner am 18. 3. 1896 in Wien
\newline{}Erhalt  durch Arthur Schnitzler im Zeitraum [18. 3. 1896
                  – 21. 3. 1896?] in Wien}\toendnotes[C]{\smallbreak}
\Standort{DLA, A:Schnitzler, HS.1985.1.419.}
\physDesc{Brief, 1 Blatt, 2 Seiten, 1114 Zeichen
\newline{}Handschrift: schwarze Tinte, lateinische Kurrent
\newline{}Schnitzler: mit rotem Buntstift eine Unterstreichung }\toendnotes[C]{\smallbreak}
\pstart
           {\pb}\textcolor{pink}{Bäckerstrasse N\textsuperscript{o} 1}\oindex{Wien@\textbf{Wien}!I., Innere Stadt@\textbf{I., Innere Stadt}!Bäckerstraße 1@\textbf{Bäckerstraße 1}, \emph{Wohngebäude}|pw}{}\ledrightnote{\textcolor{pink}{Bäckerstraße 1}}, den 18. III. 96.\pend
           
\pstart\center{}Verehrter Herr Doctor!\pend\vspace{0.5em}
\pstart
           Und es herrschte Freude und eitel Sonnenschein und siehe, eine unpässlich zu Bett
               liegende junge Dame wurde vor lauter Vergnügen plötzlich gesund. Das hat Ihr
               liebenswürdiger \label{K_L03699-1v}\edtext{Brief}{\lemma{\textnormal{\emph{Brief}}}\Cendnote{\textnormal{nicht überliefert}}}\label{K_L03699-1} verursacht, für
               den, sowie für die beispiellose bewundernswürdige Schnelligkeit, mit der Sie meine
               Bitte erfüllt haben, ich Ihnen auf das Herzlichste danke. – \pend
           
\pstart
           Wenn Ihre Spannung auf meine ferneren Arbeiten wohl kaum den Grad je erreichen
               dürfte, wie die meine auf Ihr Urtheil war, so will ich doch Gleiches mit Gleichem
               vergelten und Ihnen als Dank ungesäumt drei andere Arbeiten zur gütigen Durchsicht
               übersenden. N\textsuperscript{o} 1. »\label{K_L03699-2v}\edtext{\textcolor{green}{Pierettes Tagebuch}\pwindex{Plessner, Elsa 22.\,8.\,1875 Wien – 7.\,5.\,1932 Alicante@\textsc{Plessner, Elsa} (22.\,8.\,1875 Wien – 7.\,5.\,1932 Alicante), \emph{Schriftstellerin}!Pierettes Tagebuch [19 unveröffentlichte Gedichte]@\strich\emph{Pierettes Tagebuch [19 unveröffentlichte Gedichte]}|pw}{}\ledrightnote{\textcolor{green}{Pierettes Tagebuch [19 unveröffentlichte Gedichte]}}}{\lemma{\textnormal{\emph{Pierettes Tagebuch}}}\Cendnote{\textnormal{Beilage nicht erhalten; die lyrische Zusammenstellung \emph{\textcolor{green}{Pierettes
                     Tagebuch}\pwindex{Plessner, Elsa 22.\,8.\,1875 Wien – 7.\,5.\,1932 Alicante@\textsc{Plessner, Elsa} (22.\,8.\,1875 Wien – 7.\,5.\,1932 Alicante), \emph{Schriftstellerin}!Pierettes Tagebuch [19 unveröffentlichte Gedichte]@\strich\emph{Pierettes Tagebuch [19 unveröffentlichte Gedichte]}|pwk}} wurde nie publiziert und ist verschollen.}}}\label{K_L03699-2}«, 19 Nummern {\pb}Lyrik, in einer
                  \textcolor{green}{Novelle}\pwindex{Plessner, Elsa 22.\,8.\,1875 Wien – 7.\,5.\,1932 Alicante@\textsc{Plessner, Elsa} (22.\,8.\,1875 Wien – 7.\,5.\,1932 Alicante), \emph{Schriftstellerin}!Pierettes Tagebuch@\strich\emph{Pierettes Tagebuch}|pwv}{}\ledrightnote{{$\rightarrow$}\emph{\textcolor{green}{Pierettes Tagebuch}}} verstreut gewesene
                  \textcolor{green}{Gedichte}\pwindex{Plessner, Elsa 22.\,8.\,1875 Wien – 7.\,5.\,1932 Alicante@\textsc{Plessner, Elsa} (22.\,8.\,1875 Wien – 7.\,5.\,1932 Alicante), \emph{Schriftstellerin}!Pierettes Tagebuch [19 unveröffentlichte Gedichte]@\strich\emph{Pierettes Tagebuch [19 unveröffentlichte Gedichte]}|pwv}{}\ledrightnote{{$\rightarrow$}\emph{\textcolor{green}{Pierettes Tagebuch [19 unveröffentlichte Gedichte]}}}, die nun für sich
               allein stehen sollen, da die \textcolor{green}{Novelle}\pwindex{Plessner, Elsa 22.\,8.\,1875 Wien – 7.\,5.\,1932 Alicante@\textsc{Plessner, Elsa} (22.\,8.\,1875 Wien – 7.\,5.\,1932 Alicante), \emph{Schriftstellerin}!Pierettes Tagebuch@\strich\emph{Pierettes Tagebuch}|pwv}{}\ledrightnote{{$\rightarrow$}\emph{\textcolor{green}{Pierettes Tagebuch}}} unbrauchbar war.\pend
           
\pstart
           \label{K_L03699-3v}\edtext{\textcolor{green}{N\textsuperscript{o} 2 und
                  3}\pwindex{Plessner, Elsa 22.\,8.\,1875 Wien – 7.\,5.\,1932 Alicante@\textsc{Plessner, Elsa} (22.\,8.\,1875 Wien – 7.\,5.\,1932 Alicante), \emph{Schriftstellerin}!Baby@\strich\emph{Baby}|pwuv}\pwindex{Begräbnißtag@\emph{Der Begräbnißtag}|pwuv}{}\ledrightnote{{$\rightarrow$}\emph{\textcolor{green}{Baby}}{\newline}{$\rightarrow$}\emph{\textcolor{green}{Der Begräbnißtag}}}}{\lemma{\textnormal{\emph{N\textsuperscript{o} 2 und 3}}}\Cendnote{\textnormal{Auch diese Beilagen sind nicht überliefert. Zieht man
                     \textcolor{blue}{Plessners}\pwindex{Plessner, Elsa 22.\,8.\,1875 Wien – 7.\,5.\,1932 Alicante@\textsc{Plessner, Elsa} (22.\,8.\,1875 Wien – 7.\,5.\,1932 Alicante), \emph{Schriftstellerin}|pwk} Aufstellung ihrer eigenen Arbeiten (Elsa Plessner an Arthur Schnitzler, 12. 10. 1900)
                     heran, so dürfte es sich um \emph{\textcolor{green}{Baby}\pwindex{Plessner, Elsa 22.\,8.\,1875 Wien – 7.\,5.\,1932 Alicante@\textsc{Plessner, Elsa} (22.\,8.\,1875 Wien – 7.\,5.\,1932 Alicante), \emph{Schriftstellerin}!Baby@\strich\emph{Baby}|pwk}} und \emph{\textcolor{green}{Der Begräbnißtag}\pwindex{Begräbnißtag@\emph{Der Begräbnißtag}|pwk}}
                     handeln.}}}\label{K_L03699-3} kleine Skizzen, Federspritzer, wie ich sie sehr gern schreibe.
               Wenn das kritische Verfahren wieder nur annähernd so kurze Zeit in Anspruch nimmt,
               wie das erstemal, so bauen Sie sich eine weitere Staffel ins Himmelreich und einen
               Dankaltar in meinem Herzen. –\pend
           
\pstart
           Mit vorzüglicher Hochachtung{\\[\baselineskip]}\spacefill\mbox{Elsa Plessner}. \pend
           \leftskip=0em{}\selectlanguage{ngerman}\endnumbering\briefempfaengerindex{Schnitzler, Arthur@\textsc{Schnitzler, Arthur}!zzzPlessner, Elsa@\emph{von Elsa Plessner}!1896-03-181@{18. 3. 1896}|)be}\mylabel{L03699h}  \normalsize

\doendnotes{C}
\bigskip
\vfill

\clearpage

\footnotesize

\lohead{\textsc{register}}

% Definiere theindex-Environment komplett neu ohne reledmac
\makeatletter
\renewenvironment{theindex}{%
  \section*{\indexname}%
  \setlength{\parindent}{0pt}%
  \setlength{\parskip}{0pt plus 0.3pt}%
  \let\item\@idxitem
}{%
  \clearpage
}
\makeatother

\IfFileExists{\jobname-pw.ind}{\input{\jobname-pw.ind}}{}

\end{document}

      