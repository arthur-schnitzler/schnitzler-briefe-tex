%% latex-korrekturansicht-vorspann.tex
%% Vorspann für die Korrekturansicht.
%% Lädt die gemeinsame Datei latex-vorspann.tex mit gesetztem Schalter.

\newif\ifkorrekturansicht
\korrekturansichttrue

\input{../tex-inputs/latex-vorspann}


               \section[Friedrich M. Fels an Arthur Schnitzler, 20. 4. 1893]{ Friedrich M. Fels an Arthur Schnitzler, 20. 4. 1893}\nopagebreak\mylabel{v}\rehead{ }\normalsize\beginnumbering\briefempfaengerindex{Schnitzler, Arthur@\textsc{Schnitzler, Arthur}!zzzFels, Friedrich Michael@\emph{von Friedrich Michael Fels}!1893-04-201@{20. 4. 1893}|(be} \toendnotes[C]{\smallbreak\pagebreak[2]} \Standort{DLA, A:Schnitzler, HS.NZ85.1.2956.}
\physDesc{Brief, 1 Blatt, 2 Seiten
\newline{}Handschrift: schwarze Tinte, lateinische Kurrent
\newline{}Schnitzler: mit Bleistift nummeriert: »10« }\toendnotes[C]{\smallbreak}\pstart
           \raggedleft{}{\pb}\textcolor{pink}{Meran-Obermais, Erzh. Rainer}{}\ledrightnote{\textcolor{pink}{Erzherzog Rainer}}{\\}20. April \label{K_L00198_1v}\edtext{1892}{\lemma{\textnormal{\emph{1892}}}\Cendnote{\textnormal{Die falsche
                                Jahresangabe von \textcolor{blue}{Schnitzler} durch
                                    »3« ersetzt.}}}\label{K_L00198_1h}\pend
           \pstart\center{}Lieber Dr Schnitzler!\pend\pstart
           Entschuldigen Sie, bitte, daſs ich so lange nichts von mir hören lieſs; we{\geminationn} ich wieder in \textcolor{pink}{Wien}{}\ledrightnote{\textcolor{pink}{Wien}}{ }ſein werde, werde ich Ihnen des
                    ausführlicheren über die Gründe meines höchst unliebenswürdigen und undankbaren
                    Schweigens sprechen. Ende dieses Monats werde ich zurückkehren, nachdem ich
                    vollständig genesen bin. Da aber zuvor die Angelegenheit mit der Rechnung
                    geordnet werden muſs, hätte ich folgende Bitte an Sie: Wollen Sie so freundlich
                    sein, bei den Herren der \textcolor{brown}{Deutschen Zeitung}{}\ledrightnote{\textcolor{brown}{Deutsche Zeitung}} –
                    daſs meine Anstellung ganz sicher sei, darüber hat mir \textcolor{blue}{Loris}{}\ledrightnote{\textcolor{blue}{Hugo von Hofmannsthal}} geschrieben – vielleicht zu veranlaſsen, daſs ich
                    vom 1. Mai ab eintreten ka{\geminationn} und \strikeout{zug} daſs mir, we{\geminationn}
                    das der Fall ist, umgehend eine Schrift zugeschickt werde, wodurch die \textcolor{brown}{D. Ztg.}{}\ledrightnote{\textcolor{brown}{Deutsche Zeitung}} erklärt, dem \textcolor{blue}{Hotelier}{}\ledrightnote{→\textcolor{blue}{Josef Drassl}} des \textcolor{pink}{Erzh. Rainer}{}\ledrightnote{\textcolor{pink}{Erzherzog Rainer}}, bis zur Befriedigung seiner Ansprüche,
                    monatlich eine besti{\geminationm}te Su{\geminationm}e etwa ¼ \introOben{}oder ⅓\introOben{} meines
                    Gehaltes zuzusenden. We{\geminationn} ich nicht in kürzester
                    Kürze diese Schrift oder eine andere Sicherstellung \substVorne{}\textsuperscript{erhalten}{\allowbreak}\substDazwischen{}bieten ka{\geminationn}\substHinten{}{ }{\pb}werde ich in sehr unangenehme
                    Verwickelungen geraten und wahrscheinlich noch etwas früher, als hier sonst der
                    Fall wäre, die Strafe für all meine Thaten erhalten.\pend
           \pstart
           Bitte, grüſsen Sie mir alle Beka{\geminationn}ten, die etwa noch
                    geneigt sein sollten, einen Gruſs von mir zu empfangen, und seien Sie selbst
                    herzl. gegrüſst\pend
           \pstart
           von{\\[\baselineskip]}\spacefill\mbox{Fels}\pend
           \leftskip=0em{}\endnumbering\briefempfaengerindex{Schnitzler, Arthur@\textsc{Schnitzler, Arthur}!zzzFels, Friedrich Michael@\emph{von Friedrich Michael Fels}!1893-04-201@{20. 4. 1893}|)be}\mylabel{h}  \normalsize

\doendnotes{C}
\bigskip
\vfill

\clearpage

\footnotesize

\lohead{\textsc{register}}

% Definiere theindex-Environment komplett neu ohne reledmac
\makeatletter
\renewenvironment{theindex}{%
  \section*{\indexname}%
  \setlength{\parindent}{0pt}%
  \setlength{\parskip}{0pt plus 0.3pt}%
  \let\item\@idxitem
}{%
  \clearpage
}
\makeatother

\IfFileExists{\jobname-pw.ind}{\input{\jobname-pw.ind}}{}

\end{document}

      