%% latex-korrekturansicht-vorspann.tex
%% Vorspann für die Korrekturansicht.
%% Lädt die gemeinsame Datei latex-vorspann.tex mit gesetztem Schalter.

\newif\ifkorrekturansicht
\korrekturansichttrue

\input{../tex-inputs/latex-vorspann}


\renewcommand{\erwaehntePersonen}{Personen:  ?? [Hausmeister von Kärntnerring 12/Bösendorferstraße 11],  ?? [Hausmeisterin von Kärntnerring 12/Bösendorferstraße 11], Marie Glümer, Cesare Lombroso}
\renewcommand{\erwaehnteInstitutionen}{Institutionen: J. Deibler, Buchhandel und Antiquariat}
\renewcommand{\erwaehnteOrte}{Orte: Allgemeine Poliklinik, Herrengasse, Kärntnerring 12/Bösendorferstraße 11, Unterach am Attersee, Wien}
\renewcommand{\erwaehnteWerke}{Werke: Genie und Irrsinn in ihren Beziehungen zum Gesetz, zur Kritik und zur Geschichte}
\section[Felix Salten an Arthur Schnitzler, 10. 8. 1892]{Felix Salten an Arthur Schnitzler, 10. 8. 1892}
\nopagebreak\mylabel{v}
\rehead{ }\normalsize\beginnumbering\briefempfaengerindex{Schnitzler, Arthur@\textsc{Schnitzler, Arthur}!zzzSalten, Felix@\emph{von Felix Salten}!1892-08-102@{10. 8. 1892}|(be}
\toendnotes[C]{\smallbreak\pagebreak[2]}\Standort{CUL, Schnitzler, B 89, A 1.}
\physDesc{Kartenbrief, 719 Zeichen
\newline{}Handschrift: schwarze Tinte, lateinische Kurrent
\newline{}Versand: Stempel: »\nobreak{}\oindex{Unterach am Attersee@\textbf{Unterach am Attersee}, \emph{P.PPL}|pwk}Unterach am
                                       \textcolor{gray}{Att}ersee, 10/8 92\nobreak{}«.  
\newline{}Ordnung: mit Bleistift von unbekannter Hand nummeriert: »15« }\toendnotes[C]{\smallbreak}\pstart{}{\pb}Herrn D\textsuperscript{r} Arthur Schnitzler\pend{}\pstart{}\textcolor{pink}{\strikeout{Unterach}}{}\ledrightnote{\textcolor{pink}{Unterach am Attersee}}{ }\textcolor{pink}{Wien}{}\ledrightnote{\textcolor{pink}{Wien}}\pend{}\pstart{}\textcolor{pink}{I. Kärntnerring 12}{}\ledrightnote{\textcolor{pink}{Kärntnerring 12/Bösendorferstraße 11}}\pend{}
{\bigskip}
\pstart
           \raggedleft{}{\pb}\textcolor{pink}{Unterach}{}\ledrightnote{\textcolor{pink}{Unterach am Attersee}}, 10. VIII. 92.\pend
           
\pstart
           Ich habe viele Menschen, die mir werth sind, die ich schätze und die mir sympathisch
               sind, ich habe aber nur \uline{einen}{[},{]}
               den ich \uline{wirklich liebe} und nur
                  einen{[},{]} dem \uline{ich wirklich}
               Freund bin, und das sind Sie! Bitte Sie \label{K_L03186-1v}\edtext{\uline{aufrichtigst} schreiben Sie mir umgehend \uline{Alles}}{\lemma{\textnormal{\emph{aufrichtigst … Alles}}}\Cendnote{\textnormal{In undatierten Erinnerungen \textcolor{blue}{Schnitzler}s wird der Hintergrund beleuchtet:
                     \textcolor{blue}{Salten} hatte zu dieser Zeit die
                  Erlaubnis, ohne Rücksprache in \textcolor{blue}{Schnitzler}s
                  \textcolor{pink}{Wohnung} zu übernachten,
                  wenn er die letzte Straßenbahn versäumt hatte. »Ich verlasse das Haus
                        meist früher als er, da ich auf die \textcolor{pink}{Poliklinik} muss. Er schläft weiter. Bald merke ich, dass mir
                        allerlei wegkommt, ein Ring, eine Nadel, auffallend viel Bücher.{ / }\textcolor{blue}{M. G.} hat sofort einen Verdacht,
                        den ich ohne rechte Ueberzeugung bekämpfe.{ / }Er leiht sich Bücher aus, ohne sie mir zurückzugeben.{ / }Indess geht meinem \textcolor{blue}{Hausmeister} sein junges \textcolor{blue}{Weibchen} mit einem Liebhaber durch und
                           jener{[},{]} von den Diebstählen bei mir in Kenntnis
                        gesetzt, spricht den Verdacht aus, dass seine \textcolor{blue}{Ungetreue} wie allerlei aus des
                           \textcolor{blue}{Ehegatten}
                        Wohnung auch manches aus der meinen entwendet haben könnte, die sie
                        aufzuräumen pflegte.{ / }Einmal beim \textcolor{brown}{Antiquariat} in der \textcolor{pink}{Herrengasse}{[},{]}{ }\textcolor{brown}{Deibler}{[},{]}
                        entdecke ich ein \textcolor{green}{Buch} von \textcolor{blue}{Lombroso}, das ich \textcolor{blue}{F. S.}
                        geliehen. Um mich zu vergewissern, lasse ich mir das \textcolor{green}{Buch} zeigen und
                        entdecke gewisse Schriftzeichen, die ich bei Gelegenheit kritischer
                        Besprechung eingetragen, so dass ein Irrtum ausgeschlossen ist.{ / }Ich begebe mich zu \textcolor{blue}{F. S.}, er
                        liegt noch zu Bett, ich ersuche ihn um Rückgabe meiner Bücher, insbesondere
                        des \textcolor{blue}{\textcolor{green}{Lo{[}m{]}broso}}, er erklärt,
                        dass er leider seinen Schlüssel verloren habe. Es kommt wohl nicht zu einer
                        Aussprache, doch zu einer Unterredung, in der er meinen begründeten Verdacht
                        zu verkennen nicht mehr in der Lage ist. Er schreibt mir einen halb
                        aufrichtigen, halb reuigen Brief, den er Jahre später von mir
                        zurückerbittet, von dem ich mir aber eine Abschrift behalte.«
                        (\emph{DLA}, A:Schnitzler, Verschiedenes
                     Autobiographisches, »Felix Salten«, HS.NZ85.1.116)}}}\label{K_L03186-1h}, was Sie mir
               gegenüber auf der Seele haben, schreiben Sie es mir bitte Gleich, denn ich werde hier
               nicht ruhig sein, bis ich nicht Alles von Ihnen gehört. Dass ich meine Abreise nicht
               dennoch um einen Tag hinausgeschoben{[},{]} thut mir jetzt \uuline{sehr}{ }\uuline{leid}. Ich hoffe Sie nehmen sich die halbe Stunde
               Zeit, damit wir wieder in klare Luft kommen. Das ist nun mein ungeduldiger Wun\damage{sch}\pend
           \pstart Ihr aufrichtig ergebe\damage{ner}\pend{}\endnumbering\briefempfaengerindex{Schnitzler, Arthur@\textsc{Schnitzler, Arthur}!zzzSalten, Felix@\emph{von Felix Salten}!1892-08-102@{10. 8. 1892}|)be}\mylabel{h}  \normalsize

\doendnotes{C}
\bigskip
\vfill

\clearpage

\footnotesize

\lohead{\textsc{register}}

% Definiere theindex-Environment komplett neu ohne reledmac
\makeatletter
\renewenvironment{theindex}{%
  \section*{\indexname}%
  \setlength{\parindent}{0pt}%
  \setlength{\parskip}{0pt plus 0.3pt}%
  \let\item\@idxitem
}{%
  \clearpage
}
\makeatother

\IfFileExists{\jobname-pw.ind}{\input{\jobname-pw.ind}}{}

\end{document}

      