%% latex-korrekturansicht-vorspann.tex
%% Vorspann für die Korrekturansicht.
%% Lädt die gemeinsame Datei latex-vorspann.tex mit gesetztem Schalter.

\newif\ifkorrekturansicht
\korrekturansichttrue

\input{../tex-inputs/latex-vorspann}


\renewcommand{\erwaehntePersonen}{Personen: Hermann Bahr, Felix Salten, Ottilie Salten}
\renewcommand{\erwaehnteOrte}{Orte: Armbrustergasse, Edmund-Weiß-Gasse 7, Heiligenstadt, Wien, Österreich}
\renewcommand{\erwaehnteWerke}{Werke: Burgtheater. »Husarenfieber.« Schwank in vier Akten von Gustav Kadelburg und Richard Skowronnek. – Zum erstenmal: am 17. Januar 1907, Die Zeit}
\section[ Arthur Schnitzler an Felix Salten, 18. 1. 1907]{Arthur Schnitzler an Felix Salten, 18. 1. 1907}
\nopagebreak\mylabel{v}
\rehead{ }\normalsize\beginnumbering\briefempfaengerindex{Salten, Felix@\textsc{Salten, Felix}!zzzSchnitzler, Arthur@\emph{von Arthur Schnitzler}!1907-01-181@{18. 1. 1907}|(be}
\toendnotes[C]{\smallbreak\pagebreak[2]}\Standort{Wienbibliothek im Rathaus, ZPH 1681, 2.1.516.}
\physDesc{Postkarte, 324 Zeichen
\newline{}Handschrift: 1) schwarze Tinte, deutsche Kurrent\hspace{1em}2) schwarze Tinte, lateinische Kurrent (\noindent{}Adresse)\hspace{1em}
\newline{}Versand: Stempel: »\nobreak{}\textcolor{gray}{Wien} 3a, 18. 1. 07, 9\nobreak{}«. Stempel: »\nobreak{}Wien \textcolor{gray}{118}, 19. 1. 07, 8. V, Bestellt\nobreak{}«.  
\newline{}Ordnung: mit Bleistift von unbekannter Hand nummeriert: »13« }
\buchAbdrucke{\weitereDrucke{1) Arthur Schnitzler: \emph{Briefe 1875–1912}. Hg. Therese Nickl und Heinrich Schnitzler. Frankfurt am Main: \emph{S. Fischer} 1981, S. 550.} \weitereDrucke{2) Hermann Bahr, Arthur Schnitzler: \emph{Briefwechsel, Aufzeichnungen, Dokumente (1891–1931)}. Hg. Kurt Ifkovits und Martin Anton Müller. Göttingen: \emph{Wallstein} 2018, S. 388.} }\toendnotes[C]{\smallbreak}\pstart{}{\pb}Herrn Felix Salten\pend{}\pstart{}\textcolor{pink}{Wien Heiligenstadt}{}\ledrightnote{\textcolor{pink}{Heiligenstadt}}\pend{}\pstart{}\textcolor{pink}{Armbrusterstr. 6}{}\ledrightnote{\textcolor{pink}{Armbrustergasse}}.\pend{}
{\bigskip}
\pstart
           \noindent{}{\pb}\textcolor{gray}{\textbf{Dr. Arthur Schnitzler}}\hfill 18/1 907\pend
           
\pstart
           \textcolor{gray}{\textbf{\textcolor{pink}{Wien, XVIII. Spoettelgasse 7}{}\ledrightnote{\textcolor{pink}{Edmund-Weiß-Gasse 7}}.}}\pend
           
\pstart
           lieber,{ }\textcolor{blue}{Bahr}{}\ledrightnote{\textcolor{blue}{Hermann Bahr}} ko{\geminationm}t erſt
                  \label{K_L03007-1v}\edtext{½ 2}{\lemma{\textnormal{\emph{½ 2}}}\Cendnote{\textnormal{13 Uhr 30}}}\label{K_L03007-1h}, wir ſpeiſen alſo erſt
                  \label{K_L03007-2v}\edtext{¾ 2}{\lemma{\textnormal{\emph{¾ 2}}}\Cendnote{\textnormal{13 Uhr 45}}}\label{K_L03007-2h}, was ich zu Ordnung
               eventueller Hungerangelegenheiten gebührend mittheile. Aber \label{K_L03007-3v}\edtext{ko{\geminationm}en Sie u \textcolor{blue}{Otti}{}\ledrightnote{\textcolor{blue}{Ottilie Salten}}}{\lemma{\textnormal{\emph{kommen Sie u Otti}}}\Cendnote{\textnormal{siehe A. S.: \emph{Tagebuch}, 19. 1. 1907}}}\label{K_L03007-3h} deswegen nicht ſpäter.\pend
           
\pstart
           herzlich {\\[\baselineskip]}\spacefill\mbox{A.}\pend
           \leftskip=0em{}
\pstart
           \noindent{}Ihr \label{K_L03007-4v}\edtext{\textcolor{green}{Huſarenfieberfeu{[}i{]}ll}{}\ledrightnote{\textcolor{green}{Burgtheater. »Husarenfieber.« Schwank in vier Akten von Gustav Kadelburg und Richard Skowronnek. – Zum erstenmal: am 17. Januar 1907}}}{\lemma{\textnormal{\emph{Huſarenfieberfeuill}}}\Cendnote{\textnormal{\textcolor{blue}{Felix Salten}: \emph{\textcolor{green}{Burgtheater. »Husarenfieber.« Schwank in vier Akten von
                           Gustav Kadelburg und Richard Skowronnek. – Zum erstenmal: am 17. Januar
                           1907}}. In: \emph{\textcolor{green}{Die Zeit}}, Jg. 6,
                        Nr. 1.552, 18. 1. 1907, S. 1–2.}}}\label{K_L03007-4h}
                  erſter Rang. Was hilft’s? \textcolor{pink}{Oeſterreich}{}\ledrightnote{\textcolor{pink}{Österreich}} iſt
                  das Land des Verhallens.\pend
           \endnumbering\briefempfaengerindex{Salten, Felix@\textsc{Salten, Felix}!zzzSchnitzler, Arthur@\emph{von Arthur Schnitzler}!1907-01-181@{18. 1. 1907}|)be}\mylabel{h}  \normalsize

\doendnotes{C}
\bigskip
\vfill

\clearpage

\footnotesize

\lohead{\textsc{register}}

% Definiere theindex-Environment komplett neu ohne reledmac
\makeatletter
\renewenvironment{theindex}{%
  \section*{\indexname}%
  \setlength{\parindent}{0pt}%
  \setlength{\parskip}{0pt plus 0.3pt}%
  \let\item\@idxitem
}{%
  \clearpage
}
\makeatother

\IfFileExists{\jobname-pw.ind}{\input{\jobname-pw.ind}}{}

\end{document}

      