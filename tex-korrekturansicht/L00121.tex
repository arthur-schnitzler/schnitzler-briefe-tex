%% latex-korrekturansicht-vorspann.tex
%% Vorspann für die Korrekturansicht.
%% Lädt die gemeinsame Datei latex-vorspann.tex mit gesetztem Schalter.

\newif\ifkorrekturansicht
\korrekturansichttrue

\input{../tex-inputs/latex-vorspann}


               \section[Hugo von Hofmannsthal an Arthur Schnitzler, 7. 9. {[}1892{]}]{ Hugo von Hofmannsthal an Arthur Schnitzler, 7. 9. {[}1892{]}}\nopagebreak\mylabel{v}\rehead{ }\normalsize\beginnumbering\briefempfaengerindex{Schnitzler, Arthur@\textsc{Schnitzler, Arthur}!zzzHofmannsthal, Hugo von@\emph{von Hugo von Hofmannsthal}!1892-09-071@{7. 9. {[}1892{]}}|(be} \toendnotes[C]{\smallbreak\pagebreak[2]} \Standort{CUL, Schnitzler, B 43.}
\physDesc{Brief, 1 Blatt, 3 Seiten
\newline{}Handschrift: blaue Tinte, deutsche Kurrent
\newline{}Schnitzler: mit Bleistift nummeriert: »\strikeout{30}
                                 31« und die Jahreszahl ergänzt: »92« }\buchAbdrucke{\weitereDrucke{1) Hugo von Hofmannsthal, Arthur Schnitzler: \emph{Briefwechsel}. Hg. Therese Nickl und Heinrich Schnitzler. Frankfurt am Main: \emph{S. Fischer} 1964, S. 28–29.} \weitereDrucke{2) Hermann Bahr, Arthur Schnitzler: \emph{Briefwechsel, Aufzeichnungen, Dokumente (1891–1931)}. Hg. Kurt Ifkovits und Martin Anton Müller. Göttingen: \emph{Wallstein} 2018, S. 27.} }\toendnotes[C]{\smallbreak}\pstart
           \noindent{}{\pb}\textcolor{pink}{\textsc{Lélex. (Ain)}}{}\ledrightnote{\textcolor{pink}{Lélex}}\hfill \textsc{7. sept.}\pend
           \pstart
           Fünf Stunden von der Eiſenbahn. Keine Zeitung. Kühe. \textsc{Monsieur le
                  curé qui fait des enfants aux jolies paysannes.} Der Gendarm: \textsc{\textcolor{green}{Pandore}{}\ledrightnote{\textcolor{green}{Pandore}}}. Die alten Fliegenſchimmel des Wirths: \textsc{\textcolor{green}{Pyrame et Thisbé}{}\ledrightnote{\textcolor{green}{Ein Sommernachtstraum}}}. Die Hauskatze: \textsc{Madeleine}. Der Nachttopf: \textsc{Monsieur Jules}.\pend
           \pstart
           \centering{}– – – –\pend
           \pstart
           \noindent{}Lange grüne Hochplateaus mit Farrnkraut und Jurakalk; dahinter der große See und der
                  \textsc{Montblanc} und Herr \textcolor{blue}{\textsc{\textcolor{blue}{Edouard Rod}{}\ledrightnote{\textcolor{blue}{Édouard Rod}}}}{}\ledrightnote{\textcolor{blue}{Édouard Rod}}.\pend
           \pstart
           {\pb}Gang der Handlung: Ich werde
               behandelt, wie der \label{K_L00121_1v}\edtext{\textcolor{blue}{kleine Dauphin}{}\ledrightnote{→\textcolor{blue}{Louis Charles de Bourbon}} beim böſen
               Schuſter \textcolor{blue}{\textsc{Simon}}{}\ledrightnote{\textcolor{blue}{Antoine Simon}}}{\lemma{\textnormal{\emph{kleine … Simon}}}\Cendnote{\textnormal{1793 wurde der ehemalige Thronfolger \textcolor{blue}{Louis Charles de Bourbon} dem Schuster \textcolor{blue}{Alain Simon} zur ›Erziehung‹ überantwortet.}}}\label{K_L00121_1h}. Man giebt
               mir mehr grüne und gelbe \label{K_L00121_2v}\edtext{Chartreuſe}{\lemma{\textnormal{\emph{Chartreuſe}}}\Cendnote{\textnormal{Kräuterlikör}}}\label{K_L00121_2h} zu
               trinken, als einem Steinklopfer, und dann muſs ich Lieder im Patois lernen und
               ſingen, z. B.\pend
           \stanza{}\label{K_L00121_3v}\edtext{\textsc{Z’ame les bouguettes}\newverse{}\textsc{Et les matafans}\newverse{}\textsc{Et les dsones feuilles}\newverse{}\textsc{Qu’ont lo tétés blancs!}\newverse{}– – – – –\newverse{}\hspace*{2.5em}(unanständig)}{\lemma{\textnormal{\emph{Z’ame … (unanständig)}}}\Cendnote{\textnormal{Es handelt sich um ein Lied, mit dem nach
                  Bougettes (eine herausgebackene Speise aus Ei, Mehl und Kartoffeln) und Matafans
                  (einer dem Crêpe verwandten, herausgebackenen Speise aus Mehl und Kartoffeln)
                  verlangt wurde. Die letzten beiden Verse besagen, dass der Sänger zudem eine
                  Vorliebe für weiße Brüste besitzt.}}}\label{K_L00121_3h}\stanzaend{}\pstart
           \textsc{{\pb}\label{K_L00121_4v}\edtext{Voilà ce qu’on ap\strikeout{p}elle se dépayser}{\lemma{\textnormal{\emph{Voilà … dépayser}}}\Cendnote{\textnormal{sinngemäß: Das heißt es, sich in ein fremdes Land zu
                     begeben.}}}\label{K_L00121_4h}}; \label{LL417-1v}ſiehe \textcolor{blue}{Hermann Bahr}{}\ledrightnote{\textcolor{blue}{Hermann Bahr}}, ges. Werke, \textsc{passim}\label{LL417-1h} »über die rechte Art in fremden Ländern zu reiſen«. Dienstag beginnt
               eigentlich meine \textcolor{green}{Reiſe in die
                  Provinzen des mittäglichen \textcolor{pink}{Frankreich}{}\ledrightnote{\textcolor{pink}{Frankreich}}}{}\ledrightnote{→\textcolor{green}{Reise in die mittäglichen Provinzen von Frankreich}}.\pend
           \pstart
           Schreiben Sie, bitte, zwiſchen 10. und 16. nach \textcolor{pink}{\textsc{Arles, Bouches-du-Rhône}}{}\ledrightnote{\textcolor{pink}{Arles}}{ }\textsc{poste rest.}\pend
           \pstart
           \uline{\textsc{via \textcolor{pink}{Buchs}{}\ledrightnote{\textcolor{pink}{Buchs}}{ }\textcolor{pink}{Genève}{}\ledrightnote{\textcolor{pink}{Genf}}}}\pend
           \pstart \spacefill\mbox{Hugo.}\pend{}\endnumbering\briefempfaengerindex{Schnitzler, Arthur@\textsc{Schnitzler, Arthur}!zzzHofmannsthal, Hugo von@\emph{von Hugo von Hofmannsthal}!1892-09-071@{7. 9. {[}1892{]}}|)be}\mylabel{h}  \normalsize

\doendnotes{C}
\bigskip
\vfill

\clearpage

\footnotesize

\lohead{\textsc{register}}

% Definiere theindex-Environment komplett neu ohne reledmac
\makeatletter
\renewenvironment{theindex}{%
  \section*{\indexname}%
  \setlength{\parindent}{0pt}%
  \setlength{\parskip}{0pt plus 0.3pt}%
  \let\item\@idxitem
}{%
  \clearpage
}
\makeatother

\IfFileExists{\jobname-pw.ind}{\input{\jobname-pw.ind}}{}

\end{document}

      