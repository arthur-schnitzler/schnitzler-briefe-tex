%% latex-korrekturansicht-vorspann.tex
%% Vorspann für die Korrekturansicht.
%% Lädt die gemeinsame Datei latex-vorspann.tex mit gesetztem Schalter.

\newif\ifkorrekturansicht
\korrekturansichttrue

\input{../tex-inputs/latex-vorspann}


               \section[Arthur Schnitzler an Richard Beer-Hofmann, 4. 11. 1896]{ Arthur Schnitzler an Richard Beer-Hofmann, 4. 11. 1896}\nopagebreak\mylabel{v}\rehead{ }\normalsize\beginnumbering\briefempfaengerindex{Beer-Hofmann, Richard@\textsc{Beer-Hofmann, Richard}!zzzSchnitzler, Arthur@\emph{von Arthur Schnitzler}!1896-11-042@{4. 11. 1896}|(be} \toendnotes[C]{\smallbreak\pagebreak[2]} \Standort{YCGL, MSS 31.}
\physDesc{Telegramm
\newline{}maschinell\newline{}Versand: 1) Stempel: »\nobreak{}\oindex{I., Innere Stadt@\textbf{I., Innere Stadt}, \emph{Bezirk (A.BZK)}|pwk}Wien 1/1, 14 XI 96, 8 40 N\nobreak{}«.  2) mit dem Namen des empfangenden Telegrafenbeamten bestempelt: »\noindent{}\textcolor{gray}{\textbf{\textit{11 \textcolor{pink}{Bln.} 42}}}{ / }\textcolor{gray}{\textbf{\textit{\textcolor{blue}{Krejči}}}}.«}\toendnotes[C]{\smallbreak}\pstart{}{\pb}richard beer-hofmann \textcolor{pink}{wien}{}\ledrightnote{\textcolor{pink}{Wien}}\pend{}\pstart{}\textcolor{pink}{wollzeile 15}{}\ledrightnote{\textcolor{pink}{Wollzeile}}\pend{}{\bigskip}\pstart
           \noindent{}{\pb}\textcolor{pink}{Wien}{}\ledrightnote{\textcolor{pink}{Wien}} de \textcolor{pink}{berlin}{}\ledrightnote{\textcolor{pink}{Berlin}} 1407 15 7 36 =\pend
           \pstart
           = herzlichen dank ihnen und den \textcolor{blue}{andern}{}\ledrightnote{→\textcolor{blue}{Gustav Schwarzkopf}{\newline}→\textcolor{blue}{Max Schwarzkopf}{\newline}→\textcolor{blue}{Emil Schwarzkopf}{\newline}→\textcolor{blue}{Victor Léon}{\newline}→\textcolor{blue}{Leo Feld}{\newline}→\textcolor{blue}{Alexander Engel}} viele
               gruesse \spacefill\mbox{arthur. +}\pend
           \endnumbering\briefempfaengerindex{Beer-Hofmann, Richard@\textsc{Beer-Hofmann, Richard}!zzzSchnitzler, Arthur@\emph{von Arthur Schnitzler}!1896-11-042@{4. 11. 1896}|)be}\mylabel{h}  \normalsize

\doendnotes{C}
\bigskip
\vfill

\clearpage

\footnotesize

\lohead{\textsc{register}}

% Definiere theindex-Environment komplett neu ohne reledmac
\makeatletter
\renewenvironment{theindex}{%
  \section*{\indexname}%
  \setlength{\parindent}{0pt}%
  \setlength{\parskip}{0pt plus 0.3pt}%
  \let\item\@idxitem
}{%
  \clearpage
}
\makeatother

\IfFileExists{\jobname-pw.ind}{\input{\jobname-pw.ind}}{}

\end{document}

      