%% latex-korrekturansicht-vorspann.tex
%% Vorspann für die Korrekturansicht.
%% Lädt die gemeinsame Datei latex-vorspann.tex mit gesetztem Schalter.

\newif\ifkorrekturansicht
\korrekturansichttrue

\input{../tex-inputs/latex-vorspann}


               \section[Hermann Bahr an Arthur Schnitzler, 22. 11. 1910]{ Hermann Bahr an Arthur Schnitzler, 22. 11. 1910}\nopagebreak\mylabel{v}\rehead{ }\normalsize\beginnumbering\briefempfaengerindex{Schnitzler, Arthur@\textsc{Schnitzler, Arthur}!zzzBahr, Hermann@\emph{von Hermann Bahr}!1910-11-222@{22. 11. 1910}|(be} \toendnotes[C]{\smallbreak\pagebreak[2]} \Standort{CUL, Schnitzler, B 5b.}
\physDesc{Bildpostkarte
\newline{}Handschrift: schwarze Tinte, deutsche Kurrent\newline{}Versand: 1) Stempel: »\nobreak{}\oindex{Koeln@\textbf{Köln}, \emph{Besiedelter Ort (A.BSO)}|pwk}\oindex{Frankfurt am Main@\textbf{Frankfurt am Main}, \emph{Besiedelter Ort (A.BSO)}|pwk}Cöln – Frankfurt (M.) Bahnpost, 22. 11. 10\nobreak{}«.  2) mit Bleistift von unbekannter Hand die Straße der Anschrift gestrichen\newline{}Ordnung: mit Bleistift von unbekannter Hand nummeriert: »170« }\buchAbdrucke{\weitereDrucke{Hermann Bahr, Arthur Schnitzler: \emph{Briefwechsel, Aufzeichnungen, Dokumente (1891–1931)}. Hg. Kurt Ifkovits und Martin Anton Müller. Göttingen: \emph{Wallstein} 2018, S. 446.} }\toendnotes[C]{\smallbreak}\pstart{}{\pb}\textsc{Arthur
                  Schnitzler}\pend{}\pstart{}\textsc{\textcolor{pink}{Wien XIII}{}\ledrightnote{\textcolor{pink}{XIII., Hietzing}}}\pend{}\pstart{}\textsc{\textcolor{pink}{Spöttelgasse 7}{}\ledrightnote{\textcolor{pink}{Edmund-Weiß-Gasse}}}\pend{}{\bigskip}\pstart
           \noindent{}\centering{}\textcolor{gray}{\textbf{{\pb}Blick in das \textcolor{pink}{\textcolor{blue}{Beethoven}{}\ledrightnote{\textcolor{blue}{Ludwig van Beethoven}}museum}{}\ledrightnote{\textcolor{pink}{Beethovenmuseum}} im 2.
                  Stockwerk mit dem Flügel und den Streichinstrumenten des
                  Meisters.}}\pend
           \pstart
           {\pb}Schönſten Dank für Deinen lieben Brief u. die
               herzlichſten Grüße an Dich u Deine \textcolor{blue}{Frau}{}\ledrightnote{→\textcolor{blue}{Olga Schnitzler}}\textcolor{gray}{!}\pend
           \pstart Herzlichſt \spacefill\mbox{Hermann}\pend{}\pstart
           22. 11. 10\pend
           \endnumbering\briefempfaengerindex{Schnitzler, Arthur@\textsc{Schnitzler, Arthur}!zzzBahr, Hermann@\emph{von Hermann Bahr}!1910-11-222@{22. 11. 1910}|)be}\mylabel{h}  \normalsize

\doendnotes{C}
\bigskip
\vfill

\clearpage

\footnotesize

\lohead{\textsc{register}}

% Definiere theindex-Environment komplett neu ohne reledmac
\makeatletter
\renewenvironment{theindex}{%
  \section*{\indexname}%
  \setlength{\parindent}{0pt}%
  \setlength{\parskip}{0pt plus 0.3pt}%
  \let\item\@idxitem
}{%
  \clearpage
}
\makeatother

\IfFileExists{\jobname-pw.ind}{\input{\jobname-pw.ind}}{}

\end{document}

      