%% latex-korrekturansicht-vorspann.tex
%% Vorspann für die Korrekturansicht.
%% Lädt die gemeinsame Datei latex-vorspann.tex mit gesetztem Schalter.

\newif\ifkorrekturansicht
\korrekturansichttrue

\input{../tex-inputs/latex-vorspann}


\section[Stefan Zweig an Arthur Schnitzler, 26. 4. 1916]{L03658 Stefan Zweig an Arthur Schnitzler, 26. 4. 1916}
\nopagebreak\mylabel{L03658v}
\rehead{ }\normalsize\beginnumbering\briefempfaengerindex{, @\textsc{, }!zzz, @\emph{von  }!1916-04-261@{26. 4. 1916}|(be}
\toendnotes[C]{\smallbreak\pagebreak[2]}\Standort{CUL, Schnitzler, B 118.}
\physDesc{Briefkarte, 431 Zeichen
\newline{}Handschrift: lila Tinte, lateinische Kurrent}
\buchAbdrucke{\weitereDrucke{Stefan Zweig: \emph{Briefwechsel mit Hermann Bahr, Sigmund Freud, Rainer Maria
                        Rilke und Arthur Schnitzler}. Herausgegeben von Jeffrey B. Berlin,  Hans-Ulrich Lindken und  Donald A. Prater. Frankfurt am Main: \emph{S. Fischer} 1987, S. 399.} }\toendnotes[C]{\smallbreak}
\pstart
           \raggedleft{}{\pb}26. IV 1916\pend
           
\pstart
           \textcolor{gray}{\textbf{SZ}}\hfill \textcolor{gray}{\textbf{\textcolor{pink}{VIII. KOCHGASSE 8}\oindex{Wien@\textbf{Wien}!VIII., Josefstadt@\textbf{VIII., Josefstadt}!Kochgasse 8@\textbf{Kochgasse 8}, \emph{Wohngebäude}|pw}{}\ledrightnote{\textcolor{pink}{Kochgasse 8}}.}}\pend
           \vspace{0.5em}
\pstart
           Lieber verehrter Herr Doktor, ich wollte es Ihnen seit langem sagen,
               dass Sie es nicht falsch verstehen mögen, wenn ich mich gar nicht bei Ihnen zeigte
               und anfragte – ich habe mich in die \label{K_L03658-1v}\edtext{\textcolor{pink}{Nähe Wiens}\oindex{Wien@\textbf{Wien}!XXIII., Liesing@\textbf{XXIII., Liesing}!Haselbrunnerstraße 12@\textbf{Haselbrunnerstraße 12}, \emph{Wohngebäude}|pwv}{}\ledrightnote{{$\rightarrow$}\emph{\textcolor{pink}{Haselbrunnerstraße 12}}}}{\lemma{\textnormal{\emph{Nähe Wiens}}}\Cendnote{\textnormal{\textcolor{blue}{Zweig}\pwindex{Zweig, Stefan 28.\,11.\,1881 Wien – 23.\,2.\,1942 Petrópolis@\textsc{Zweig, Stefan} (28.\,11.\,1881 Wien – 23.\,2.\,1942 Petrópolis), \emph{Schriftsteller}|pwk} hatte sich zwei \textcolor{pink}{Pavillons}\oindex{Wien@\textbf{Wien}!XXIII., Liesing@\textbf{XXIII., Liesing}!Haselbrunnerstraße 12@\textbf{Haselbrunnerstraße 12}, \emph{Wohngebäude}|pwkv} in \textcolor{pink}{Kalksburg}\oindex{Wien@\textbf{Wien}!XXIII., Liesing@\textbf{XXIII., Liesing}!Kalksburg@\textbf{Kalksburg}, \emph{Region}|pwk} gemietet, die er mit seiner \textcolor{blue}{Frau}\pwindex{Zweig, Friderike Maria 4.\,12.\,1882 Wien – 18.\,1.\,1971 Stamford@\textsc{Zweig, Friderike Maria} (4.\,12.\,1882 Wien – 18.\,1.\,1971 Stamford), \emph{Schriftstellerin}|pwkv} in der warmen Jahreszeit
                  bewohnte.}}}\label{K_L03658-1} zurückgezogen, um von meinem zerstückelten Leben den armen Rest
               für Arbeit nutzen zu können. Umsomehr freue ich mich, \label{K_L03658-2v}\edtext{Ihre liebe \textcolor{blue}{Frau}\pwindex{Schnitzler, Olga 17.\,1.\,1882 Wien – 13.\,1.\,1970 Lugano@\textsc{Schnitzler, Olga} (17.\,1.\,1882 Wien – 13.\,1.\,1970 Lugano), \emph{Schauspielerin, Sängerin}|pwv}{}\ledrightnote{{$\rightarrow$}\emph{\textcolor{blue}{Olga Schnitzler}}}}{\lemma{\textnormal{\emph{Ihre liebe Frau}}}\Cendnote{\textnormal{Am 29. 4. 1916 sang \textcolor{blue}{Olga}\pwindex{Zweig, Friderike Maria 4.\,12.\,1882 Wien – 18.\,1.\,1971 Stamford@\textsc{Zweig, Friderike Maria} (4.\,12.\,1882 Wien – 18.\,1.\,1971 Stamford), \emph{Schriftstellerin}|pwk} ein Wohltätigkeitskonzert in einem Hörsaal der \textcolor{pink}{Allgemeinen Poliklinik.}\oindex{Wien@\textbf{Wien}!IX., Alsergrund@\textbf{IX., Alsergrund}!Allgemeine Poliklinik [neues Gebäude]@\textbf{Allgemeine Poliklinik [neues Gebäude]}, \emph{Krankenhaus}|pwk}}}}\label{K_L03658-2}{ }Samstag zu hören und hoffentlich {\pb}Sie auch sehen zu dürfen. In Verehrung
               getreu Ihr\pend
           \pstart \spacefill\mbox{Stefan Zweig}\pend{}\selectlanguage{ngerman}\endnumbering\briefempfaengerindex{, @\textsc{, }!zzz, @\emph{von  }!1916-04-261@{26. 4. 1916}|)be}\mylabel{L03658h}  \normalsize

\doendnotes{C}
\bigskip
\vfill

\clearpage

\footnotesize

\lohead{\textsc{register}}

% Definiere theindex-Environment komplett neu ohne reledmac
\makeatletter
\renewenvironment{theindex}{%
  \section*{\indexname}%
  \setlength{\parindent}{0pt}%
  \setlength{\parskip}{0pt plus 0.3pt}%
  \let\item\@idxitem
}{%
  \clearpage
}
\makeatother

\IfFileExists{\jobname-pw.ind}{\input{\jobname-pw.ind}}{}

\end{document}

      