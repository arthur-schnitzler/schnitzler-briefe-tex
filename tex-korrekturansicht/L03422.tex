%% latex-korrekturansicht-vorspann.tex
%% Vorspann für die Korrekturansicht.
%% Lädt die gemeinsame Datei latex-vorspann.tex mit gesetztem Schalter.

\newif\ifkorrekturansicht
\korrekturansichttrue

\input{../tex-inputs/latex-vorspann}


\renewcommand{\erwaehntePersonen}{Personen:  Alfons XIII., Oskar Bie, Hedwig Fischer, Samuel Fischer, Julius von Gans-Ludassy, Herbert Ginsberg, Leopold Godowsky, Maximilian Harden, Karl August von Hardenberg, Moritz Heimann, Theodor Herzl, Hugo von Hofmannsthal, Siegfried Jacobsohn, Alfred Kerr, Fritz Kreisler, Richard Metzl, Arthur Nikisch, Felix Poppenberg, Anna Katharina Rehmann, Max Reinhardt, Rudolf Rittner, Vasilij Ilʹič Safonov, Felix Salten, Paul Salten, Ottilie Salten, Olga Schnitzler, Heinrich Schnitzler, Heinrich Friedrich Karl vom und zum Stein,  Victoria Eugénie von Spanien, Ida d’Albert}
\renewcommand{\erwaehnteOrte}{Orte: Auerstedt, Bansin, Berlin, Cadiz, Friedrichstraße, Gibraltar, Granada, Heringsdorf, Italien, Jena, Lissabon, Madrid, Ostsee, Preußen, Sevilla, Skodsborg, Spanien, Tanger, Tiergarten, Toledo, Wien, Świnoujście}
\renewcommand{\erwaehnteWerke}{Werke: Die neue Rundschau, Herr Wenzel auf Rehberg. Novelle}
\section[ Felix Salten an Arthur Schnitzler, 1. 5. 1906]{Felix Salten an Arthur Schnitzler, 1. 5. 1906}
\nopagebreak\mylabel{v}
\rehead{ }\normalsize\beginnumbering\briefempfaengerindex{Schnitzler, Arthur@\textsc{Schnitzler, Arthur}!zzzSalten, Felix@\emph{von Felix Salten}!1906-05-011@{1. 5. 1906}|(be}
\toendnotes[C]{\smallbreak\pagebreak[2]}\Standort{CUL, Schnitzler, B 89, B 1.}
\physDesc{Brief, 1 Blatt, 2 Seiten, 3410 Zeichen
\newline{}Handschrift: schwarze Tinte, lateinische Kurrent
\newline{}Schnitzler: mit rotem Buntstift eine Unterstreichung 
\newline{}Ordnung: mit Bleistift von unbekannter Hand nummeriert: »212« }\toendnotes[C]{\smallbreak}
\pstart
           \raggedleft{}{\pb}\textcolor{pink}{Berlin}{}\ledrightnote{\textcolor{pink}{Berlin}}, 1. Mai 06.\pend
           
\pstart
           Lieber, die \label{K_L03422-1v}\edtext{Radpartie}{\lemma{\textnormal{\emph{Radpartie}}}\Cendnote{\textnormal{siehe Felix Salten an Arthur Schnitzler, 28. 3. 1906}}}\label{K_L03422-1h}, ja, wenn ich heute nur wüßte, wie und was in drei, vier Wochen sein wird.
               Ich fürchte, die Radpartie wird sich nicht machen laßen. Vorläufig nämlich ist es
               beschloßen, dass ich am 20. od. 21. nach \textcolor{pink}{Madrid}{}\ledrightnote{\textcolor{pink}{Madrid}}
               fahre, zur \label{K_L03422-2v}\edtext{\textcolor{blue}{Königshochzeit}{}\ledrightnote{{$\rightarrow$}\textcolor{blue}{Alfons XIII.}{\newline}{$\rightarrow$}\textcolor{blue}{Victoria Eugénie von Spanien}}}{\lemma{\textnormal{\emph{Königshochzeit}}}\Cendnote{\textnormal{Am 17. 5. 1906 heirateten in \textcolor{pink}{Madrid}
                  König \textcolor{blue}{Alfonso XIII. von Spanien} und \textcolor{blue}{Victoria Eugénie von Battenberg}.}}}\label{K_L03422-2h}. Da
               käme ich erst am 10. Juni wieder zurück, weil ich
               natürlich \textcolor{pink}{Toledo}{}\ledrightnote{\textcolor{pink}{Toledo}}, \textcolor{pink}{Sevilla}{}\ledrightnote{\textcolor{pink}{Sevilla}}, \textcolor{pink}{Cadiz}{}\ledrightnote{\textcolor{pink}{Cadiz}}, \textcolor{pink}{Tanger}{}\ledrightnote{\textcolor{pink}{Tanger}}, \textcolor{pink}{Gibraltar}{}\ledrightnote{\textcolor{pink}{Gibraltar}}, \textcolor{pink}{Granada}{}\ledrightnote{\textcolor{pink}{Granada}} mitnehme, und der
               Weg zurück über \textcolor{pink}{Lissabon}{}\ledrightnote{\textcolor{pink}{Lissabon}} führe. Da gäbe es dann
               – ausser dem contractlichen Urlaub – keine Absenz mehr. Und die vier Wochen im Juli will ich still an einem Fleck sitzen, Tennis spielen
               und arbeiten. (Ich bin im Begriff, die \label{K_L03422-3v}\edtext{\textcolor{blue}{Herzl}{}\ledrightnote{\textcolor{blue}{Theodor Herzl}}-Biographie}{\lemma{\textnormal{\emph{Herzl-Biographie}}}\Cendnote{\textnormal{Eine Biografie \textcolor{blue}{Herzl}s
                  wurde von \textcolor{blue}{Salten} nie geschrieben.}}}\label{K_L03422-3h} zu
               übernehmen, was ich mir als eine Art von Denkmal-Portrait sehr schön denke.) Mit dem
                  \label{K_L03422-4v}\edtext{Seebad}{\lemma{\textnormal{\emph{Seebad}}}\Cendnote{\textnormal{siehe Felix Salten an Arthur Schnitzler, 28. 3. 1906}}}\label{K_L03422-4h} ist das so: wir müßen doch im Juni schon aufs
               Land, der \textcolor{blue}{Kinder}{}\ledrightnote{{$\rightarrow$}\textcolor{blue}{Anna Katharina Rehmann}{\newline}{$\rightarrow$}\textcolor{blue}{Paul Salten}}
               wegen. \textcolor{blue}{Otti}{}\ledrightnote{\textcolor{blue}{Ottilie Salten}} und die \textcolor{blue}{Kinder}{}\ledrightnote{{$\rightarrow$}\textcolor{blue}{Anna Katharina Rehmann}{\newline}{$\rightarrow$}\textcolor{blue}{Paul Salten}} gehen Juni, Juli, August, bis Mitte September an die \textcolor{pink}{See}{}\ledrightnote{{$\rightarrow$}\textcolor{pink}{Ostsee}}. Da wird eine Wohnung
               genommen und Wirtschaft geführt. Möglichst nahe, damit ich über Sonntag einmal hin,
                  \textcolor{blue}{Otti}{}\ledrightnote{\textcolor{blue}{Ottilie Salten}} manchmal zu mir in die \textcolor{pink}{Stadt}{}\ledrightnote{{$\rightarrow$}\textcolor{pink}{Berlin}} kommen kann. Also \textcolor{pink}{Bansin}{}\ledrightnote{\textcolor{pink}{Bansin}}, \textcolor{pink}{Swinemünde}{}\ledrightnote{\textcolor{pink}{Świnoujście}} oder \textcolor{pink}{Heringsdorf}{}\ledrightnote{\textcolor{pink}{Heringsdorf}}. \uline{Deshalb} kann ich dann für den Juli nicht alles nach \textcolor{pink}{Skodsborg}{}\ledrightnote{\textcolor{pink}{Skodsborg}}
               verlegen. Es ist einfach eine Sache des Geldes. Und bin ich selbst frei, möchte ich
               doch bei den \textcolor{blue}{Kindern}{}\ledrightnote{{$\rightarrow$}\textcolor{blue}{Anna Katharina Rehmann}{\newline}{$\rightarrow$}\textcolor{blue}{Paul Salten}} sein.\pend
           
\pstart
           Wenn sich die \textcolor{pink}{spani}{}\ledrightnote{{$\rightarrow$}\textcolor{pink}{Spanien}}sche Reise
               nun doch nicht macht, schreibe ich Ihnen rechtzeitig wegen der Radtour.\pend
           
\pstart
           Mein \label{K_L03422-5v}\edtext{Brief an \textcolor{blue}{Hugo}{}\ledrightnote{\textcolor{blue}{Hugo von Hofmannsthal}} mit der starken Verstimmung}{\lemma{\textnormal{\emph{Brief … Verstimmung}}}\Cendnote{\textnormal{»Ich habe alle die Fremdheiten dieses Landes jetzt zu
                     verdauen, und alle die Bräuche, Zustände u. s. w. durch die es mich enttäuscht,
                     irgendwie zur Kenntnis zu nehmen. Thatsächlich lebt man hier in russischen
                     Verhältnissen, lebt in einem Polizeistaat, in welchem die Menschen auf eine
                     ekelerregende Weise von Demut zur Frechheit, von Furcht zur Rohheit taumeln.
                     Alle führen die Worte: ›Zuverläßigkeit‹, ›Wahrheit‹, ›Treue‹ u. s. w. beständig
                     im Mund, und alle sind unzuverläßig, verlogen, treulos. Es ist ein \textcolor{pink}{Preussen}, wie es \uline{vor}{ }\textcolor{blue}{Hardenberg} und \textcolor{blue}{Stein}, wie es vor \textcolor{pink}{Jena} und \textcolor{pink}{Auerstädt} gewesen:
                     corrupt, niedrig, schandbar.« Felix Salten an \textcolor{blue}{Hugo von Hofmannsthal},
                        9. 3. 1906, \emph{Freies Deutsches
                        Hochstift}, Hs-30865,25. Zit. n. Marcel Atze: \emph{»Unser aller Feldmarschall mit der Feder«. Felix Saltens halbes Jahrhundert
                        als Journalist.} In: Marcel Atze, unter Mitarbeit von Tanja Gausterer
                     (Hg.): \emph{Im Schatten von Bambi. Felix Salten entdeckt die Wiener
                        Moderne. Leben und Werk}.
                     Salzburg/Wien:
                        \emph{Residenz}{ }2020, S. 260–289, hier 281.}}}\label{K_L03422-5h}
               gegen \textcolor{pink}{Berlin}{}\ledrightnote{\textcolor{pink}{Berlin}} datirt weit zurück, war im März noch geschrieben, während er in \textcolor{pink}{Italien}{}\ledrightnote{\textcolor{pink}{Italien}} war. Seither hat sich die Sache genau um die
               Frühlingssonne verbessert. Ich schreibe selten, weil ich mit organisatorischen
               Arbeiten beschäftigt bin, weil ich productiv einiges componire, und die \textcolor{pink}{Stadt}{}\ledrightnote{{$\rightarrow$}\textcolor{pink}{Berlin}} noch zu wenig als
               publizistische Anregung fühle. Es würden Reisebriefe werden, und das wäre falsch. Ich
               bin froh, dass mich meine Selbstcontrolle {\pb}vor solchen Verfehlungen ebenso
               wie vor allzufrühen, taktlosen Vertraulichkeiten mit dieser \textcolor{pink}{Stadt}{}\ledrightnote{{$\rightarrow$}\textcolor{pink}{Berlin}} bewahrt.\pend
           
\pstart
           Wie \label{K_L03422-6v}\edtext{\textcolor{green}{Herr Wenzel}{}\ledrightnote{\textcolor{green}{Herr Wenzel auf Rehberg. Novelle}}}{\lemma{\textnormal{\emph{Herr Wenzel}}}\Cendnote{\textnormal{\textcolor{blue}{Felix Salten}: \emph{\textcolor{green}{Herr Wenzel auf Rehberg. Novelle}}. In: \emph{\textcolor{green}{Die neue Rundschau}}, Jg. 17, H. 5, Mai 1906, S. 544–576.}}}\label{K_L03422-6h} aufgenommen wird, bin ich
               neugierig. Es ist das erstemal, dass ich eine Novelle von mir in der Correctur ohne
               Desperation und tiefe Niedergeschlagenheit lesen konnte.\pend
           
\pstart
           Mein Verkehr hier? Ab und zu \textcolor{blue}{Heimann}{}\ledrightnote{\textcolor{blue}{Moritz Heimann}}, \textcolor{blue}{Jakobsohn}{}\ledrightnote{\textcolor{blue}{Siegfried Jacobsohn}}. Dann \textcolor{blue}{Rittner}{}\ledrightnote{\textcolor{blue}{Rudolf Rittner}}. Und \textcolor{blue}{Fischers}{}\ledrightnote{\textcolor{blue}{Hedwig Fischer}{\newline}\textcolor{blue}{Samuel Fischer}}, die mir aus der Nähe immer sympathischer werden. Selten \textcolor{blue}{Reinhardt}{}\ledrightnote{\textcolor{blue}{Max Reinhardt}} und seine Leute, manchmal \textcolor{blue}{Bie}{}\ledrightnote{\textcolor{blue}{Oskar Bie}} (sehr lieb und fein) und \textcolor{blue}{Poppenberg}{}\ledrightnote{\textcolor{blue}{Felix Poppenberg}}, zwei, drei lange Gespräche mit \textcolor{blue}{Kerr}{}\ledrightnote{\textcolor{blue}{Alfred Kerr}}; fast garnicht mehr \textcolor{blue}{Harden}{}\ledrightnote{\textcolor{blue}{Maximilian Harden}}. Dazwischen die Gesellschaften, denen sich nicht ausweichen läßt. Bei
               meinem \textcolor{blue}{Schwager}{}\ledrightnote{{$\rightarrow$}\textcolor{blue}{Richard Metzl}} Musikleute:
                  \textcolor{blue}{Safonoff}{}\ledrightnote{\textcolor{blue}{Vasilij Ilʹič Safonov}}, \textcolor{blue}{Godowski}{}\ledrightnote{\textcolor{blue}{Leopold Godowsky}}, \textcolor{blue}{Nikisch}{}\ledrightnote{\textcolor{blue}{Arthur Nikisch}}, \textcolor{blue}{Kreisler}{}\ledrightnote{\textcolor{blue}{Fritz Kreisler}}. Hie und da eine ärgerliche, manchmal eine nette
               Stunde mit Frau \textcolor{blue}{Fulda}{}\ledrightnote{\textcolor{blue}{Ida d’Albert}}. Das ist alles; ist
               genug, ist – gelegentlich sogar zu viel. Ich will lieber lesen, will jetzt viel, sehr
               viel lesen; lerne ein bischen spanisch und gehe mit \textcolor{blue}{Otti}{}\ledrightnote{\textcolor{blue}{Ottilie Salten}} im \textcolor{pink}{Thiergarten}{}\ledrightnote{\textcolor{pink}{Tiergarten}} spazieren, wo es –
               unglaublich aber wahr – gerade jetzt einfach märchenhaft schön ist.\pend
           
\pstart
           \textcolor{blue}{Otti}{}\ledrightnote{\textcolor{blue}{Ottilie Salten}} läßt Frau \textcolor{blue}{Olga}{}\ledrightnote{\textcolor{blue}{Olga Schnitzler}} um Entschuldigung bitten, weil sie ihren lieben Brief
               noch nicht beantworten konnte. Sie hat sich erst die linke Hand verbrannt, und kaum
               die halbwegs gut war, wieder die rechte verbrüht. Da wir nicht hoffen, dass sie jetzt
               wieder von vorne anfängt, rechnen wir darauf, dass sie bald wieder den Gebrauch all
               ihrer Gliedmaßen erlangt. Die \textcolor{blue}{Kinder}{}\ledrightnote{{$\rightarrow$}\textcolor{blue}{Anna Katharina Rehmann}{\newline}{$\rightarrow$}\textcolor{blue}{Paul Salten}} sind reizend, und wir alle grüßen \textcolor{blue}{Sie alle}{}\ledrightnote{{$\rightarrow$}\textcolor{blue}{Olga Schnitzler}{\newline}{$\rightarrow$}\textcolor{blue}{Heinrich Schnitzler}} aufs Herzlichste.\pend
           \pstart Ihr \spacefill\mbox{Salten}\pend{}
\pstart
           \noindent{}\label{K_L03422-7v}\edtext{\textcolor{gray}{NB}}{\lemma{\textnormal{\emph{NB}}}\Cendnote{\textnormal{nota bene; lateinisch: merke
                     wohl}}}\label{K_L03422-7h}. Heute sahen wir \textcolor{blue}{Ludaßy}{}\ledrightnote{\textcolor{blue}{Julius von Gans-Ludassy}} in der \textcolor{pink}{Friedrichstraße}{}\ledrightnote{\textcolor{pink}{Friedrichstraße}}. Wir haben sehr gestaunt, weil wir dachten, er sei – wie
                  lange schon! – gestorben.\pend
           
\pstart
           D\textsuperscript{r} \textcolor{blue}{Ginsberg}{}\ledrightnote{\textcolor{blue}{Herbert Ginsberg}}
                  schrieb mir sehr entzückt über die freundl. \label{K_L03422-8v}\edtext{Aufnahme}{\lemma{\textnormal{\emph{Aufnahme}}}\Cendnote{\textnormal{siehe Felix Salten an Arthur Schnitzler, 8. 4. 1906}}}\label{K_L03422-8h} bei Ihnen. Vielen Dank!\pend
           \endnumbering\briefempfaengerindex{Schnitzler, Arthur@\textsc{Schnitzler, Arthur}!zzzSalten, Felix@\emph{von Felix Salten}!1906-05-011@{1. 5. 1906}|)be}\mylabel{h}  \normalsize

\doendnotes{C}
\bigskip
\vfill

\clearpage

\footnotesize

\lohead{\textsc{register}}

% Definiere theindex-Environment komplett neu ohne reledmac
\makeatletter
\renewenvironment{theindex}{%
  \section*{\indexname}%
  \setlength{\parindent}{0pt}%
  \setlength{\parskip}{0pt plus 0.3pt}%
  \let\item\@idxitem
}{%
  \clearpage
}
\makeatother

\IfFileExists{\jobname-pw.ind}{\input{\jobname-pw.ind}}{}

\end{document}

      