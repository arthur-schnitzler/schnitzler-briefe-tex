%% latex-korrekturansicht-vorspann.tex
%% Vorspann für die Korrekturansicht.
%% Lädt die gemeinsame Datei latex-vorspann.tex mit gesetztem Schalter.

\newif\ifkorrekturansicht
\korrekturansichttrue

\input{../tex-inputs/latex-vorspann}


\section[Arthur Schnitzler an Stefan Zweig, 22. 1. 1923]{L03753 Arthur Schnitzler an Stefan Zweig, 22. 1. 1923}
\nopagebreak\mylabel{L03753v}
\rehead{ }\normalsize\beginnumbering\briefempfaengerindex{, @\textsc{, }!zzz, @\emph{von  }!1923-01-221@{22. 1. 1923}|(be}
\toendnotes[C]{\smallbreak\pagebreak[2]}\Standort{Jerusalem, National Library of Israel, ARC. Ms. Var. 305 1 58 Stefan Zweig Collection.}
\physDesc{Briefkarte, 840 Zeichen
\newline{}Schreibmaschine
\newline{}Handschrift: Bleistift, lateinische Kurrent (\noindent{}minimale Korrekturen, Unterschrift)}\toendnotes[C]{\smallbreak}
\pstart
           {\pb}\textcolor{gray}{\textbf{D\textsuperscript{R} ARTHUR SCHNITZLER}}\hfill {\pb}22. 1. 1923.\pend
           
\pstart
           \textcolor{gray}{\textbf{\textcolor{pink}{WIEN, XVIII.
                        STERNWARTESTRASSE 71}\oindex{Wien@\textbf{Wien}!XVIII., Währing@\textbf{XVIII., Währing}!Sternwartestraße 71@\textbf{Sternwartestraße 71}, \emph{Wohngebäude}|pw}{}\ledrightnote{\textcolor{pink}{Sternwartestraße 71}}.}}\pend
           
\pstart{}Lieber und verehrter Herr Doktor.\pend\vspace{0.5em}
\pstart
           Herr \textcolor{blue}{Alzir Hella}\pwindex{Hella, Alzir 30.\,12.\,1881 Vieux Condé – 14.\,7.\,1953 Paris@\textsc{Hella, Alzir} (30.\,12.\,1881 Vieux Condé – 14.\,7.\,1953 Paris), \emph{Übersetzer}|pw}{}\ledrightnote{\textcolor{blue}{Alzir Hella}} hatte sich schon \label{K_L03753-1v}\edtext{an \textcolor{blue}{Fischer}\pwindex{Fischer, Samuel 24.\,12.\,1859 Liptovský Mikuláš – 15.\,10.\,1934 Berlin@\textsc{Fischer, Samuel} (24.\,12.\,1859 Liptovský Mikuláš – 15.\,10.\,1934 Berlin), \emph{Verleger}|pw}{}\ledrightnote{\textcolor{blue}{Samuel Fischer}} gewandt}{\lemma{\textnormal{\emph{an Fischer gewandt}}}\Cendnote{\textnormal{Am
                     3. 1. 1923 schrieb \textcolor{blue}{Fischer}\pwindex{Fischer, Samuel 24.\,12.\,1859 Liptovský Mikuláš – 15.\,10.\,1934 Berlin@\textsc{Fischer, Samuel} (24.\,12.\,1859 Liptovský Mikuláš – 15.\,10.\,1934 Berlin), \emph{Verleger}|pwk} an \textcolor{blue}{Schnitzler}: »Was
                     die Erzählungen für \textcolor{pink}{Frankreich}\oindex{Frankreich@\textbf{Frankreich}|pw} anbetrifft,
                     so handelt es sich um die Anfrage eines Herrn \textcolor{blue}{Alzir Hella}\pwindex{Hella, Alzir 30.\,12.\,1881 Vieux Condé – 14.\,7.\,1953 Paris@\textsc{Hella, Alzir} (30.\,12.\,1881 Vieux Condé – 14.\,7.\,1953 Paris), \emph{Übersetzer}|pw} in \textcolor{pink}{Paris}\oindex{Paris@\textbf{Paris}, \emph{Hauptstadt}|pw}, die Sache
                     taugt wohl nicht allzuviel. Ich werde mit dem Herrn korrespondieren und Ihnen
                     dann eventuell Weiteres mitteilen.« \textcolor{blue}{Schnitzler} schrieb am 22. 1. 1923 an \textcolor{blue}{Leo Greiner}\pwindex{Greiner, Leo 1.\,4.\,1876 Brünn – 22.\,8.\,1928 Berlin@\textsc{Greiner, Leo} (1.\,4.\,1876 Brünn – 22.\,8.\,1928 Berlin), \emph{Schriftsteller, Verlagslektor}|pwk} vom \emph{\textcolor{brown}{Fischer-Verlag}\orgindex{S. Fischer Verlag@S. Fischer Verlag|pwk}}: »Herr \textcolor{blue}{Alzir
                        Hella}\pwindex{Hella, Alzir 30.\,12.\,1881 Vieux Condé – 14.\,7.\,1953 Paris@\textsc{Hella, Alzir} (30.\,12.\,1881 Vieux Condé – 14.\,7.\,1953 Paris), \emph{Übersetzer}|pw} hat sich nun, von \textcolor{blue}{Stefan
                        Zweig}\pwindex{Zweig, Stefan 28.\,11.\,1881 Wien – 23.\,2.\,1942 Petrópolis@\textsc{Zweig, Stefan} (28.\,11.\,1881 Wien – 23.\,2.\,1942 Petrópolis), \emph{Schriftsteller}|pw} empfohlen, direkt an mich gewandt und ich werde ihm auch direkt
                     schreiben. Um Missverständnissen ein für alle Mal vorzubeugen, möchte ich heute
                     nur prinzipiell feststellen: der Umstand, dass sich Leute mit Anfragen über
                     meine Werke öfters an den \textcolor{brown}{Verlag Fischer}\orgindex{S. Fischer Verlag@S. Fischer Verlag|pw}
                     wenden, in Unkenntnis, dass ihm für Abschlüsse mit dem Ausland nur im Falle
                     ausdrücklicher Genehmigung meinerseits, also von Fall zu Fall ein Recht
                     zusteht, verleiht dem \textcolor{brown}{Verlag Fischer}\orgindex{S. Fischer Verlag@S. Fischer Verlag|pw}
                     natürlich nicht, sozusagen automatisch sich in einem solchen Fall als meinen
                     Vertreter zu betrachten.« \textcolor{blue}{Alzir
                     Hella}\pwindex{Hella, Alzir 30.\,12.\,1881 Vieux Condé – 14.\,7.\,1953 Paris@\textsc{Hella, Alzir} (30.\,12.\,1881 Vieux Condé – 14.\,7.\,1953 Paris), \emph{Übersetzer}|pwk}}}}\label{K_L03753-1}, aber es ist mir im Grunde lieber mit ihm persönlich zu verhandeln. »\textcolor{green}{Casanovas Heimfahrt}\pwindex{Schnitzler, Arthur 15. 5. 1862 Wien – 21. 10. 1931 ebd.@\textsc{Schnitzler, Arthur} (15. 5. 1862 Wien – 21. 10. 1931 ebd.), \emph{Schriftsteller, Mediziner}!Casanovas Heimfahrt@\strich\emph{Casanovas Heimfahrt}|pw}{}\ledrightnote{\textcolor{green}{Casanovas Heimfahrt}}« ist schon halb und halb
               vergeben, »\textcolor{green}{Frau Beate}\pwindex{Schnitzler, Arthur 15. 5. 1862 Wien – 21. 10. 1931 ebd.@\textsc{Schnitzler, Arthur} (15. 5. 1862 Wien – 21. 10. 1931 ebd.), \emph{Schriftsteller, Mediziner}!Frau Beate und ihr Sohn. Novelle@\strich\emph{Frau Beate und ihr Sohn. Novelle}|pw}{}\ledrightnote{\textcolor{green}{Frau Beate und ihr Sohn. Novelle}}« ist noch frei und ich
               wäre gern geneigt sie zur Uebersetzung ins Französische dem von Ihnen empfohlenen
               Herrn \textcolor{blue}{Hella}\pwindex{Hella, Alzir 30.\,12.\,1881 Vieux Condé – 14.\,7.\,1953 Paris@\textsc{Hella, Alzir} (30.\,12.\,1881 Vieux Condé – 14.\,7.\,1953 Paris), \emph{Übersetzer}|pw}{}\ledrightnote{\textcolor{blue}{Alzir Hella}} zu überlassen, wenn der Verleger
               sich zu einer Garantie und für einen bestimmten Termin verpflichtet\introOben{}e\introOben{}. Sonst sind alle diese Sachen gar zu unsicher. Vielleicht
               ist es das Richtigste, wenn Sie, lieber Herr Doktor, der ja mit \textcolor{blue}{Hella}\pwindex{Hella, Alzir 30.\,12.\,1881 Vieux Condé – 14.\,7.\,1953 Paris@\textsc{Hella, Alzir} (30.\,12.\,1881 Vieux Condé – 14.\,7.\,1953 Paris), \emph{Übersetzer}|pw}{}\ledrightnote{\textcolor{blue}{Alzir Hella}} in Verbindung zu stehen scheint, ihm das gelegentlich mitteilt\substVorne{}\textsuperscript{.}\substDazwischen{}?\substHinten{} Oder halten sie es für richtiger, dass \label{K_L03753-2v}\edtext{ich ihm direkt schreibe}{\lemma{\textnormal{\emph{ich ihm direkt schreibe}}}\Cendnote{\textnormal{Im Nachlass \textcolor{blue}{Schnitzlers} befindet sich der Durchschlag eines Briefes an \textcolor{blue}{Hella}\pwindex{Hella, Alzir 30.\,12.\,1881 Vieux Condé – 14.\,7.\,1953 Paris@\textsc{Hella, Alzir} (30.\,12.\,1881 Vieux Condé – 14.\,7.\,1953 Paris), \emph{Übersetzer}|pwk}, das mit dem Vortag datiert ist. Die
                  Formulierung im vorliegenden Brief lässt es aber als unklar erscheinen, ob das
                  Schreiben an \textcolor{blue}{Hella}\pwindex{Hella, Alzir 30.\,12.\,1881 Vieux Condé – 14.\,7.\,1953 Paris@\textsc{Hella, Alzir} (30.\,12.\,1881 Vieux Condé – 14.\,7.\,1953 Paris), \emph{Übersetzer}|pwk} überhaupt abgeschickt
                  wurde. »19. 2. 1923{ / }Sehr geehrter Herr \textcolor{blue}{Hella}\pwindex{Hella, Alzir 30.\,12.\,1881 Vieux Condé – 14.\,7.\,1953 Paris@\textsc{Hella, Alzir} (30.\,12.\,1881 Vieux Condé – 14.\,7.\,1953 Paris), \emph{Übersetzer}|pw}.{ / }In den nächsten Tagen kommt Frau Hofrätin \textcolor{blue}{Bertha
                           Zuckerkandl}\pwindex{Zuckerkandl, Berta 13.\,4.\,1864 Wien – 16.\,10.\,1945 Paris@\textsc{Zuckerkandl, Berta} (13.\,4.\,1864 Wien – 16.\,10.\,1945 Paris), \emph{Journalistin, Übersetzerin}|pw} nach \textcolor{pink}{Paris}\oindex{Paris@\textbf{Paris}, \emph{Hauptstadt}|pw} und wird
                        dort bei ihrer Schwester, \textcolor{blue}{MMe. Paul
                           Clemenceau}\pwindex{Clemenceau, Sophie 25.\,5.\,1862 – 24.\,9.\,1937@\textsc{Clemenceau, Sophie} (25.\,5.\,1862 – 24.\,9.\,1937)|pw}, \textcolor{pink}{12, Avenue d’Eylau}\oindex{12, Avenue d’Eylau@\textbf{12, Avenue d’Eylau}, \emph{Wohngebäude}|pw} wohnen.
                        Darf ich Sie bitten sich mit ihr in Verbindung zu
                           setzen{[},{]} ich habe ihr von Ihrem freundlichen Antrag
                        Mitteilung gemacht und sie ermächtigt mit Ihnen weiter darüber zu
                        unterhandeln. Es wäre mir natürlich sehr willkommen, wenn eine meiner
                        Novellen in ›\textcolor{green}{Monde Nouveau}\pwindex{Monde nouveau@\emph{Monde nouveau}|pw}‹ zum Abdruck käme.
                        ›\textcolor{green}{Casanovas Heimfahrt}\pwindex{Schnitzler, Arthur 15. 5. 1862 Wien – 21. 10. 1931 ebd.@\textsc{Schnitzler, Arthur} (15. 5. 1862 Wien – 21. 10. 1931 ebd.), \emph{Schriftsteller, Mediziner}!Casanovas Heimfahrt@\strich\emph{Casanovas Heimfahrt}|pw}‹ ist nicht frei, aber
                        vielleicht erlange ich mein Rechte auch auf diese Novelle wirder zurück, da
                        der Bewerber bisher meines Wissens die
                        Uebersetzung nicht in Angriff genommen hat. Ueber die Honorarbedingungen
                        wird Frau Hofrätin \textcolor{blue}{Zuckerkandl}\pwindex{Zuckerkandl, Berta 13.\,4.\,1864 Wien – 16.\,10.\,1945 Paris@\textsc{Zuckerkandl, Berta} (13.\,4.\,1864 Wien – 16.\,10.\,1945 Paris), \emph{Journalistin, Übersetzerin}|pw} mit Ihnen
                        reden.{ / }Mit verbindlichem Dank für Ihr freundliches Interesse und Ihre
                        liebenswürdigen Worte{ / }Ihr sehr ergebener{ / }{[}Raum für die Unterschrift{]}{ / }Herrn \textcolor{blue}{Alzir Hella}\pwindex{Hella, Alzir 30.\,12.\,1881 Vieux Condé – 14.\,7.\,1953 Paris@\textsc{Hella, Alzir} (30.\,12.\,1881 Vieux Condé – 14.\,7.\,1953 Paris), \emph{Übersetzer}|pw}, \textcolor{pink}{Paris, 18, rue de l’Odéon}\oindex{18, rue de l’Odéon@\textbf{18, rue de l’Odéon}, \emph{Wohngebäude}|pw}.« \textcolor{blue}{Hella}\pwindex{Hella, Alzir 30.\,12.\,1881 Vieux Condé – 14.\,7.\,1953 Paris@\textsc{Hella, Alzir} (30.\,12.\,1881 Vieux Condé – 14.\,7.\,1953 Paris), \emph{Übersetzer}|pwk} übersetzte (gemeinsam mit
                     \textcolor{blue}{Olivier Bournac}\pwindex{Bournac, Olivier 13.\,8.\,1885 Saint-Amans-du-Pech – Anfang Januar 1931 Toulon@\textsc{Bournac, Olivier} (13.\,8.\,1885 Saint-Amans-du-Pech – Anfang Januar 1931 Toulon), \emph{Schriftsteller, Übersetzer}|pwk}) von \textcolor{blue}{Schnitzler} drei Texte. Als erstes erschien
                     1925 mit \emph{\textcolor{green}{Mourir}\pwindex{Schnitzler, Arthur 15. 5. 1862 Wien – 21. 10. 1931 ebd.@\textsc{Schnitzler, Arthur} (15. 5. 1862 Wien – 21. 10. 1931 ebd.), \emph{Schriftsteller, Mediziner}!Mourir. Roman [1925]@\strich\emph{Mourir. Roman [1925]}|pwk}} eine
                  Neuübersetzung von \emph{\textcolor{green}{Sterben}\pwindex{Schnitzler, Arthur 15. 5. 1862 Wien – 21. 10. 1931 ebd.@\textsc{Schnitzler, Arthur} (15. 5. 1862 Wien – 21. 10. 1931 ebd.), \emph{Schriftsteller, Mediziner}!Sterben. Novelle@\strich\emph{Sterben. Novelle}|pwk}}, danach kamen noch
                     \emph{\textcolor{green}{Madame Beate et son fils}\pwindex{Schnitzler, Arthur 15. 5. 1862 Wien – 21. 10. 1931 ebd.@\textsc{Schnitzler, Arthur} (15. 5. 1862 Wien – 21. 10. 1931 ebd.), \emph{Schriftsteller, Mediziner}!Madame Beate et son fils@\strich\emph{Madame Beate et son fils}|pwk}}
                     (Oktober–November 1928) und \emph{\textcolor{green}{Le
                     Célibataire}\pwindex{Schnitzler, Arthur 15. 5. 1862 Wien – 21. 10. 1931 ebd.@\textsc{Schnitzler, Arthur} (15. 5. 1862 Wien – 21. 10. 1931 ebd.), \emph{Schriftsteller, Mediziner}!Le Célibataire@\strich\emph{Le Célibataire}|pwk}} (\emph{\textcolor{green}{Der Tod des
                  Junggesellen}\pwindex{Schnitzler, Arthur 15. 5. 1862 Wien – 21. 10. 1931 ebd.@\textsc{Schnitzler, Arthur} (15. 5. 1862 Wien – 21. 10. 1931 ebd.), \emph{Schriftsteller, Mediziner}!Tod des Junggesellen. Novelle@\strich\emph{Der Tod des Junggesellen. Novelle}|pwk}}, März 1929). }}}\label{K_L03753-2}?\pend
           
\pstart
           Seien Sie vielmals gegrüsst, auf baldiges Wiedersehen!{\\[\baselineskip]}Ihr herzlich ergebener{\\[\baselineskip]}\spacefill\mbox{{[}hs.:{]} Arthur Schnitzler}\pend
           \leftskip=0em{}
\pstart
           \noindent{}{[}ms.:{]} Herrn Dr. Stefan Zweig,{\\}\textcolor{pink}{Salzburg, Kapuzinerberg 5}\oindex{Paschinger Schlössl@\textbf{Paschinger Schlössl}, \emph{Wohngebäude}|pw}{}\ledrightnote{\textcolor{pink}{Paschinger Schlössl}}.\pend
           \selectlanguage{ngerman}\endnumbering\briefempfaengerindex{, @\textsc{, }!zzz, @\emph{von  }!1923-01-221@{22. 1. 1923}|)be}\mylabel{L03753h}  \normalsize

\doendnotes{C}
\bigskip
\vfill

\clearpage

\footnotesize

\lohead{\textsc{register}}

% Definiere theindex-Environment komplett neu ohne reledmac
\makeatletter
\renewenvironment{theindex}{%
  \section*{\indexname}%
  \setlength{\parindent}{0pt}%
  \setlength{\parskip}{0pt plus 0.3pt}%
  \let\item\@idxitem
}{%
  \clearpage
}
\makeatother

\IfFileExists{\jobname-pw.ind}{\input{\jobname-pw.ind}}{}

\end{document}

      