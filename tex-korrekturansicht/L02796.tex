%% latex-korrekturansicht-vorspann.tex
%% Vorspann für die Korrekturansicht.
%% Lädt die gemeinsame Datei latex-vorspann.tex mit gesetztem Schalter.

\newif\ifkorrekturansicht
\korrekturansichttrue

\input{../tex-inputs/latex-vorspann}


               \section[Arthur Schnitzler an Albert Ehrenstein, 5. 6. 1906]{ Arthur Schnitzler an Albert Ehrenstein, 5. 6. 1906}\nopagebreak\mylabel{v}\rehead{ }\normalsize\beginnumbering\briefempfaengerindex{Ehrenstein, Albert@\textsc{Ehrenstein, Albert}!zzzSchnitzler, Arthur@\emph{von Arthur Schnitzler}!1906-06-051@{5. 6. 1906}|(be} \toendnotes[C]{\smallbreak\pagebreak[2]} \Standort{New York, Leo Baeck Institute, Gertrude Lobbenberg Collection (AR 11130 / MF 1582), Autograph letters, 1846-1937.}
\physDesc{Kartenbrief
\newline{}Handschrift: schwarze Tinte, deutsche Kurrent\newline{}Versand: 1) Stempel: »\nobreak{}\oindex{XVIII., Waehring@\textbf{XVIII., Währing}, \emph{Bezirk (A.BZK)}|pwk}18/1 Wien 110, 6. VI. 06, XI\nobreak{}«.  2) Stempel: »\nobreak{}\oindex{XVIII., Waehring@\textbf{XVIII., Währing}, \emph{Bezirk (A.BZK)}|pwk}Wien 18, \textcolor{gray}{6}. VI. 06, 3.N, Bestellt\nobreak{}«. \newline{}Ordnung: mit Bleistift von unbekannter Hand nummeriert:
                                 »2« }\pstart{}{\pb}Herrn \textsc{studphil}\pend{}\pstart{}\textsc{Albert Ehrenstein}\pend{}\pstart{}\textcolor{pink}{Wien XVII}{}\ledrightnote{\textcolor{pink}{XVII., Hernals}}\pend{}\pstart{}\textcolor{pink}{Ottakringerstr. 114}{}\ledrightnote{\textcolor{pink}{Ottakringerstraße}}.\pend{}{\bigskip}\pstart{}{\pb}ſehr geehrter Herr Ehrenſtein,\pend\pstart
           wollen Sie ſich am Freitag zwiſchen ½ 4 und 4 Ihre Gedichte von mir
               abholen?\pend
           \pstart
           mit beſten Grüßen Ihres{\\[\baselineskip]}\spacefill\mbox{ArthSchnitzler}\pend
           \leftskip=0em{}\pstart
           5. 6. 906.\pend
           \endnumbering\briefempfaengerindex{Ehrenstein, Albert@\textsc{Ehrenstein, Albert}!zzzSchnitzler, Arthur@\emph{von Arthur Schnitzler}!1906-06-051@{5. 6. 1906}|)be}\mylabel{h}\begin{anhang}\end{anhang}\normalsize

\doendnotes{C}
\bigskip
\vfill

\clearpage

\footnotesize

\lohead{\textsc{register}}

% Definiere theindex-Environment komplett neu ohne reledmac
\makeatletter
\renewenvironment{theindex}{%
  \section*{\indexname}%
  \setlength{\parindent}{0pt}%
  \setlength{\parskip}{0pt plus 0.3pt}%
  \let\item\@idxitem
}{%
  \clearpage
}
\makeatother

\IfFileExists{\jobname-pw.ind}{\input{\jobname-pw.ind}}{}

\end{document}

      