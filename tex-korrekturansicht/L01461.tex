%% latex-korrekturansicht-vorspann.tex
%% Vorspann für die Korrekturansicht.
%% Lädt die gemeinsame Datei latex-vorspann.tex mit gesetztem Schalter.

\newif\ifkorrekturansicht
\korrekturansichttrue

\input{../tex-inputs/latex-vorspann}


               \section[Arthur Schnitzler an Richard Beer-Hofmann, 28. 10. 1904]{ Arthur Schnitzler an Richard Beer-Hofmann, 28. 10. 1904}\nopagebreak\mylabel{v}\rehead{ }\normalsize\beginnumbering\briefempfaengerindex{Beer-Hofmann, Richard@\textsc{Beer-Hofmann, Richard}!zzzSchnitzler, Arthur@\emph{von Arthur Schnitzler}!1904-10-281@{28. 10. 1904}|(be} \toendnotes[C]{\smallbreak\pagebreak[2]} \Standort{YCGL, MSS 31.}
\physDesc{Kartenbrief
\newline{}Handschrift: schwarze Tinte, deutsche Kurrent\newline{}Versand: 1) Stempel: »\nobreak{}2\textcolor{gray}{8}. X. 04, XI\nobreak{}«.  2) Stempel: »\nobreak{}\oindex{Rodaun@\textbf{Rodaun}, \emph{Teil eines besiedelten Ortes (A.BSOX)}|pwk}Ro{[}dau{]}n, 28 {[}10.{]} 0\textcolor{gray}{4}\nobreak{}«. 
\newline{}Beer-Hofmann: mit blauem Buntstift beschriftet: »30/X beantw.« }\toendnotes[C]{\smallbreak}\pstart{}{\pb}Herrn \textsc{Dr. Richard Beer
                     Hofmann}\pend{}\pstart{}\textcolor{pink}{\textsc{Rodaun} bei Wien}{}\ledrightnote{\textcolor{pink}{Rodaun}}\pend{}\pstart{}\textsc{\textcolor{pink}{Liesinger Straße 2}{}\ledrightnote{\textcolor{pink}{Liesingerstraße}}.}\pend{}{\bigskip}\pstart
           \raggedleft{}{\pb}\textcolor{pink}{Wien}{}\ledrightnote{\textcolor{pink}{Wien}}, 28. X. 904\pend
           \pstart
           lieber Richard, ſind Sie ſchon da? Könnte man nicht nächſte Woche,
               an einem der erſten \label{K_L01461_1v}\edtext{Abende}{\lemma{\textnormal{\emph{Abende}}}\Cendnote{\textnormal{siehe A. S.: \emph{Tagebuch}, 31. 10. 1904}}}\label{K_L01461_1h} in \textcolor{pink}{Hietzing}{}\ledrightnote{\textcolor{pink}{XIII., Hietzing}} miteinander nachtmahlen?
               Schreiben Sie mir ein oder noch lieber mehrere Worte. Auch wann Sie wieder nach \textcolor{pink}{Berlin}{}\ledrightnote{\textcolor{pink}{Berlin}} fahren u. ſ. w.\pend
           \pstart
           Herzlichen Gruß{\\[\baselineskip]}Ihr{\\[\baselineskip]}\spacefill\mbox{A.}\pend
           \leftskip=0em{}\endnumbering\briefempfaengerindex{Beer-Hofmann, Richard@\textsc{Beer-Hofmann, Richard}!zzzSchnitzler, Arthur@\emph{von Arthur Schnitzler}!1904-10-281@{28. 10. 1904}|)be}\mylabel{h}  \normalsize

\doendnotes{C}
\bigskip
\vfill

\clearpage

\footnotesize

\lohead{\textsc{register}}

% Definiere theindex-Environment komplett neu ohne reledmac
\makeatletter
\renewenvironment{theindex}{%
  \section*{\indexname}%
  \setlength{\parindent}{0pt}%
  \setlength{\parskip}{0pt plus 0.3pt}%
  \let\item\@idxitem
}{%
  \clearpage
}
\makeatother

\IfFileExists{\jobname-pw.ind}{\input{\jobname-pw.ind}}{}

\end{document}

      