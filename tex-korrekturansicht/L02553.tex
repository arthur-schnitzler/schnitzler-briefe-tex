%% latex-korrekturansicht-vorspann.tex
%% Vorspann für die Korrekturansicht.
%% Lädt die gemeinsame Datei latex-vorspann.tex mit gesetztem Schalter.

\newif\ifkorrekturansicht
\korrekturansichttrue

\input{../tex-inputs/latex-vorspann}


               \section[Olga Schnitzler an Paula Beer-Hofmann, 23. 1. 1910]{ Olga Schnitzler an Paula Beer-Hofmann, 23. 1. 1910}\nopagebreak\mylabel{v}\rehead{ }\normalsize\beginnumbering\briefempfaengerindex{Beer-Hofmann, Paula@\textsc{Beer-Hofmann, Paula}!zzzSchnitzler, Olga@\emph{von Olga Schnitzler}!1910-01-231@{23. 1. 1910}|(be} \toendnotes[C]{\smallbreak\pagebreak[2]} \Standort{YCGL, MSS 31.}
\physDesc{Bildpostkarte
\newline{}Handschrift: schwarze Tinte, lateinische Kurrent\newline{}Versand: Stempel: »\nobreak{}\oindex{Dresden@\textbf{Dresden}, \emph{Besiedelter Ort (A.BSO)}|pwk}Dresden, 23. 1. 10, 5–6N\nobreak{}«.  }\toendnotes[C]{\smallbreak}\pstart{}{\pb}Frau Paula Beer-Hofmann\pend{}\pstart{}\textcolor{pink}{Wien XIX}{}\ledrightnote{\textcolor{pink}{XIX., Döbling}}\pend{}\pstart{}\textcolor{pink}{Hasenauerstrasse 59}{}\ledrightnote{\textcolor{pink}{Hasenauerstraße}}.\pend{}{\bigskip}\pstart
           \noindent{}\centering{}{\pb}\textcolor{gray}{\textbf{\textsc{\textcolor{pink}{Hotel Bellevue}{}\ledrightnote{\textcolor{pink}{Hotel Bellevue}}, \textcolor{pink}{Dresden}{}\ledrightnote{\textcolor{pink}{Dresden}}}}}\pend
           \pstart
           \noindent{}\centering{}\textcolor{gray}{\textbf{Great sitting-room\hspace*{1.5em}Grosser Salon\hspace*{1.5em}Grand
                     Salon}}\pend
           \pstart
           Liebe geliebte Paula, es war so schön hier, ungeheuer erfrischend.
               Ein \label{K_mmL02553-1v}\edtext{grosser \textcolor{green}{Erfolg}{}\ledrightnote{→\textcolor{green}{Der Schleier der Pierrette}}}{\lemma{\textnormal{\emph{grosser Erfolg}}}\Cendnote{\textnormal{die Uraufführung von \emph{\textcolor{green}{Der
                     Schleier der Pierrette}} am 22. 1. 1910}}}\label{K_mmL02553-1h}, \textcolor{blue}{Schuch}{}\ledrightnote{\textcolor{blue}{Ernst von Schuch}}
               ist prachtvoll, die Musik entzückend. Wie schön wär’s gewesen wenn Ihr auch da
               gewesen wärt. Von \textcolor{blue}{Dohnányi}{}\ledrightnote{\textcolor{blue}{Ernst von Dohnányi}} sind \textcolor{blue}{Freunde}{}\ledrightnote{→\textcolor{blue}{Otto Carl Waldemar Benzon}} sogar aus \textcolor{pink}{Dänemark}{}\ledrightnote{\textcolor{pink}{Dänemark}} hier. Schönes Wetter, freudige Stimmung.\pend
           \pstart
           Leben Sie wol, auf Wiedersehen! Grüssen Sie den \textcolor{blue}{Herrn D\textsuperscript{r}}{}\ledrightnote{→\textcolor{blue}{Richard Beer-Hofmann}} u. die \textcolor{blue}{Kinder}{}\ledrightnote{→\textcolor{blue}{Gabriel Beer-Hofmann}{\newline}→\textcolor{blue}{Mirjam Beer-Hofmann}{\newline}→\textcolor{blue}{Naëmah Beer-Hofmann}}.{\\[\baselineskip]}Ihre\spacefill\mbox{Olga.}\pend
           \leftskip=0em{}\pstart
           23. 1. 10.\pend
           \endnumbering\briefempfaengerindex{Beer-Hofmann, Paula@\textsc{Beer-Hofmann, Paula}!zzzSchnitzler, Olga@\emph{von Olga Schnitzler}!1910-01-231@{23. 1. 1910}|)be}\mylabel{h}  \normalsize

\doendnotes{C}
\bigskip
\vfill

\clearpage

\footnotesize

\lohead{\textsc{register}}

% Definiere theindex-Environment komplett neu ohne reledmac
\makeatletter
\renewenvironment{theindex}{%
  \section*{\indexname}%
  \setlength{\parindent}{0pt}%
  \setlength{\parskip}{0pt plus 0.3pt}%
  \let\item\@idxitem
}{%
  \clearpage
}
\makeatother

\IfFileExists{\jobname-pw.ind}{\input{\jobname-pw.ind}}{}

\end{document}

      