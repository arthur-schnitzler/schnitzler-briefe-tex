%% latex-korrekturansicht-vorspann.tex
%% Vorspann für die Korrekturansicht.
%% Lädt die gemeinsame Datei latex-vorspann.tex mit gesetztem Schalter.

\newif\ifkorrekturansicht
\korrekturansichttrue

\input{../tex-inputs/latex-vorspann}


               \section[Hugo von Hofmannsthal an Arthur Schnitzler, {[}Anfang Oktober 1903{]}]{ Hugo von Hofmannsthal an Arthur Schnitzler,
               {[}Anfang Oktober 1903{]}}\nopagebreak\mylabel{v}\rehead{ }\normalsize\beginnumbering\briefempfaengerindex{Schnitzler, Arthur@\textsc{Schnitzler, Arthur}!zzzHofmannsthal, Hugo von@\emph{von Hugo von Hofmannsthal}!1903-10-011@{{[}Anfang Oktober 1903{]}}|(be} \toendnotes[C]{\smallbreak\pagebreak[2]} \Standort{CUL, Schnitzler, B 43.}
\physDesc{Brief, 1 Blatt, 1 Seite
\newline{}Handschrift: Bleistift, deutsche Kurrent
\newline{}Schnitzler: mit Bleistift datiert: »Anf Oct. 903« \newline{}Ordnung: 1) mit Bleistift von unbekannter Hand nummeriert: »\strikeout{214}« 2) mit Bleistift von unbekannter Hand nummeriert:
                                    »203«}\buchAbdrucke{\weitereDrucke{Hugo von Hofmannsthal, Arthur Schnitzler: \emph{Briefwechsel}. Hg. Therese Nickl und Heinrich Schnitzler. Frankfurt am Main: \emph{S. Fischer} 1964, S. 175.} }\toendnotes[C]{\smallbreak}\pstart
           \noindent{}{\pb}Bitte \label{K_L01322_1v}\edtext{durchzuleſen}{\lemma{\textnormal{\emph{durchzuleſen}}}\Cendnote{\textnormal{vermutlich \emph{\textcolor{green}{Elektra}}}}}\label{K_L01322_1h} und dann zu ſchicken
               an meinen \textcolor{blue}{Papa}{}\ledrightnote{→\textcolor{blue}{Hugo August von Hofmannsthal}}\pend
           \leftskip=3em{}\pstart
           \noindent{}\textcolor{pink}{III Salesianergasse 12}{}\ledrightnote{\textcolor{pink}{Salesianergasse}}.\pend
           \leftskip=0em{}\pstart \spacefill\mbox{Hugo.}\pend{}\pstart
           \noindent{}(\textcolor{green}{Aufführung}{}\ledrightnote{→\textcolor{green}{Elektra. Tragödie in einem Aufzug}} anſcheinend
                     \label{K_L01322_2v}\edtext{Ende October}{\lemma{\textnormal{\emph{Ende October}}}\Cendnote{\textnormal{am 30. 10. 1903 wurde
                        \emph{\textcolor{green}{Elektra}} am \textcolor{pink}{Kleinen Theater} aufgeführt}}}\label{K_L01322_2h}.)\pend
           \endnumbering\briefempfaengerindex{Schnitzler, Arthur@\textsc{Schnitzler, Arthur}!zzzHofmannsthal, Hugo von@\emph{von Hugo von Hofmannsthal}!1903-10-011@{{[}Anfang Oktober 1903{]}}|)be}\mylabel{h}  \normalsize

\doendnotes{C}
\bigskip
\vfill

\clearpage

\footnotesize

\lohead{\textsc{register}}

% Definiere theindex-Environment komplett neu ohne reledmac
\makeatletter
\renewenvironment{theindex}{%
  \section*{\indexname}%
  \setlength{\parindent}{0pt}%
  \setlength{\parskip}{0pt plus 0.3pt}%
  \let\item\@idxitem
}{%
  \clearpage
}
\makeatother

\IfFileExists{\jobname-pw.ind}{\input{\jobname-pw.ind}}{}

\end{document}

      