%% latex-korrekturansicht-vorspann.tex
%% Vorspann für die Korrekturansicht.
%% Lädt die gemeinsame Datei latex-vorspann.tex mit gesetztem Schalter.

\newif\ifkorrekturansicht
\korrekturansichttrue

\input{../tex-inputs/latex-vorspann}


\renewcommand{\erwaehntePersonen}{Personen:  ?? [Anwalt von Julius Gans-Ludassy],  ?? [Kinderarzt von Paul Salten], Julius von Gans-Ludassy, Gustav Harpner, Josef Svatopluk Machar, Anna Katharina Rehmann, Felix Salten, Paul Salten}
\renewcommand{\erwaehnteOrte}{Orte: Wien}
\renewcommand{\erwaehnteWerke}{}
\section[ Felix Salten an Arthur Schnitzler, {[}20.? 10. 1906{]}]{Felix Salten an Arthur Schnitzler, {[}20.? 10. 1906{]}}
\nopagebreak\mylabel{v}
\rehead{ }\normalsize\beginnumbering\briefempfaengerindex{Schnitzler, Arthur@\textsc{Schnitzler, Arthur}!zzzSalten, Felix@\emph{von Felix Salten}!1906-10-201@{{[}20.? 10. 1906{]}}|(be}
\toendnotes[C]{\smallbreak\pagebreak[2]}\Standort{CUL, Schnitzler, B 89, B 1.}
\physDesc{Brief, 1 Blatt, 1 Seite, 541 Zeichen
\newline{}Handschrift: schwarze Tinte, lateinische Kurrent
\newline{}Schnitzler: mit Bleistift datiert: »Nov 90\textcolor{gray}{6}« 
\newline{}Ordnung: mit Bleistift von unbekannter Hand nummeriert: »226« }\toendnotes[C]{\smallbreak}
\pstart
           \raggedleft{}{\pb}\label{K_L03435-1v}\edtext{Samstag}{\lemma{\textnormal{\emph{Samstag}}}\Cendnote{\textnormal{Die Datierung dieses Korrespondenstücks ist im Abgleich
                        mit dem vorangehenden (Felix Salten an Arthur Schnitzler, [18.? 10. 1906])
                        möglich, doch widerspricht das der Einordnung \textcolor{blue}{Schnitzler}s in den November. Siehe zum Prozess auch Felix Salten an Arthur Schnitzler, 9. 3. 1906.}}}\label{K_L03435-1h}.\pend
           
\pstart{}Lieber,\pend
\pstart
           die Verhandlung \textcolor{blue}{Ludaßy}{}\ledrightnote{\textcolor{blue}{Julius von Gans-Ludassy}} am Montag entfällt, da der \label{K_L03435-2v}\edtext{\textcolor{blue}{Advokat}{}\ledrightnote{{$\rightarrow$}\textcolor{blue}{?? [Anwalt von Julius Gans-Ludassy]}} des \uline{\textcolor{blue}{Kläger}{}\ledrightnote{{$\rightarrow$}\textcolor{blue}{Julius von Gans-Ludassy}}s}}{\lemma{\textnormal{\emph{Advokat des Klägers}}}\Cendnote{\textnormal{\textcolor{blue}{Julius von Gans-Ludassy} wurde von \textcolor{blue}{Josef Svatopluk Machar} vertreten.}}}\label{K_L03435-2h}
               meinen \textcolor{blue}{Vertreter}{}\ledrightnote{{$\rightarrow$}\textcolor{blue}{Gustav Harpner}} bat, es
               möchte die Sache aussergerichtlich beigelegt werden, und D\textsuperscript{r}{ }\textcolor{blue}{Harpner}{}\ledrightnote{\textcolor{blue}{Gustav Harpner}} leider, ohne mich zu fragen, in eine
               einstweilige Vertagung gewilligt hat. Ich danke Ihnen jedenfalls herzlich, für Ihre
               Bereitwilligkeit, auszusagen.\pend
           
\pstart
           Die \textcolor{blue}{Kinder}{}\ledrightnote{{$\rightarrow$}\textcolor{blue}{Anna Katharina Rehmann}} sind krank. \textcolor{blue}{Paul}{}\ledrightnote{\textcolor{blue}{Paul Salten}} hat eine starke Angina. Der \label{K_L03435-3v}\edtext{\textcolor{blue}{Arzt}{}\ledrightnote{{$\rightarrow$}\textcolor{blue}{?? [Kinderarzt von Paul Salten]}}}{\lemma{\textnormal{\emph{Arzt}}}\Cendnote{\textnormal{nicht ermittelt}}}\label{K_L03435-3h} fürchtete zuerst
               Scharlach. Vorsichtigerweise kann ich mich jetzt weder auf dem Tennisplatz noch sonst
               wo in die Nähe eines Kindesvaters wagen.\pend
           
\pstart
           Aufrichtig Ihr {\\[\baselineskip]}\spacefill\mbox{Felix Salten}\pend
           \leftskip=0em{}\endnumbering\briefempfaengerindex{Schnitzler, Arthur@\textsc{Schnitzler, Arthur}!zzzSalten, Felix@\emph{von Felix Salten}!1906-10-201@{{[}20.? 10. 1906{]}}|)be}\mylabel{h}  \normalsize

\doendnotes{C}
\bigskip
\vfill

\clearpage

\footnotesize

\lohead{\textsc{register}}

% Definiere theindex-Environment komplett neu ohne reledmac
\makeatletter
\renewenvironment{theindex}{%
  \section*{\indexname}%
  \setlength{\parindent}{0pt}%
  \setlength{\parskip}{0pt plus 0.3pt}%
  \let\item\@idxitem
}{%
  \clearpage
}
\makeatother

\IfFileExists{\jobname-pw.ind}{\input{\jobname-pw.ind}}{}

\end{document}

      