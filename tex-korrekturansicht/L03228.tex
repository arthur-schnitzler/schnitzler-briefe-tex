%% latex-korrekturansicht-vorspann.tex
%% Vorspann für die Korrekturansicht.
%% Lädt die gemeinsame Datei latex-vorspann.tex mit gesetztem Schalter.

\newif\ifkorrekturansicht
\korrekturansichttrue

\input{../tex-inputs/latex-vorspann}


\renewcommand{\erwaehntePersonen}{Personen: Gerhart Hauptmann, Maurice Maeterlinck, Heinrich Schnitzler, Olga Schnitzler, Agnes Sorma}
\renewcommand{\erwaehnteInstitutionen}{Institutionen: Berliner Theater, Buchhandlung L. Rosner, Deutsches Theater Berlin}
\renewcommand{\erwaehnteOrte}{Orte: Agnetendorf, Berlin, Dessauer Straße, Deutsches Theater Berlin, Grunewald, Haus Wiesenstein, Wien}
\renewcommand{\erwaehnteWerke}{Werke: Der Schleier der Beatrice. Schauspiel in fünf Akten, Die »neue Richtung«. Polemische Aufsätze über Berliner Theater-Aufführungen, Monna Vanna. Schauspiel in drei Akten}
\section[ Paul Goldmann an Arthur Schnitzler, 26. 10. {[}1902{]}]{Paul Goldmann an Arthur Schnitzler, 26. 10. {[}1902{]}}
\nopagebreak\mylabel{v}
\rehead{ }\normalsize\beginnumbering\briefempfaengerindex{Schnitzler, Arthur@\textsc{Schnitzler, Arthur}!zzzGoldmann, Paul@\emph{von Paul Goldmann}!1902-10-261@{26. 10. {[}1902{]}}|(be}
\toendnotes[C]{\smallbreak\pagebreak[2]}\Standort{DLA, A:Schnitzler, HS.NZ85.1.3172.}
\physDesc{Brief, 1 Blatt, 4 Seiten
\newline{}Handschrift: blaue Tinte, deutsche Kurrent
\newline{}Schnitzler: mit Bleistift das Jahr »{[}1{]}902« vermerkt }\toendnotes[C]{\smallbreak}
\pstart
           \noindent{}\raggedleft{}{\pb}\textcolor{pink}{\textcolor{gray}{\textbf{DESSAUERSTRASSE 19}}}{}\ledrightnote{\textcolor{pink}{Dessauer Straße}}\pend
           
\pstart
           \textcolor{pink}{Berlin}{}\ledrightnote{\textcolor{pink}{Berlin}}, 26. Oktober.\pend
           
\pstart\center{}Mein lieber Freund,\pend
\pstart
           Ich danke Dir vielmals für Deinen lieben Brief, der mich ſehr erfreut hat. Was Du von
                  \label{K_L03228-1v}\edtext{\textcolor{pink}{Agnetendorf}{}\ledrightnote{\textcolor{pink}{Agnetendorf}}}{\lemma{\textnormal{\emph{Agnetendorf}}}\Cendnote{\textnormal{Bezug auf \textcolor{blue}{Schnitzler}s Besuch bei \textcolor{blue}{Gerhart Hauptmann} am 19. 10. 1902 und 20. 10. 1902}}}\label{K_L03228-1h} erzählſt, hat mich natürlich ganz beſonders intereſſirt. Es thut mir
               aufrichtig leid, daß ich einen \textcolor{blue}{Mann}{}\ledrightnote{{$\rightarrow$}\textcolor{blue}{Gerhart Hauptmann}}, den Du als ſo ſympathiſch ſchilderſt, \label{K_L03228-2v}\edtext{öffentlich bekämpfen}{\lemma{\textnormal{\emph{öffentlich bekämpfen}}}\Cendnote{\textnormal{Symbolisch dafür ist seine kurz davor erschienene
                  Feuilletonsammlung: \textcolor{blue}{Paul Goldmann}: \emph{\textcolor{green}{Die »neue Richtung«. Polemische Aufsätze über Berliner
                        Theater-Aufführungen}}. \textcolor{pink}{Wien}: \emph{\textcolor{brown}{Buchhandlung L. Rosner}}{ }1902, vordatiert auf 1903.}}}\label{K_L03228-2h} und dadurch manchmal kränken muß.\pend
           
\pstart
           {\pb}Daß die \label{K_L03228-3v}\edtext{\textsc{\textcolor{blue}{Sorma}{}\ledrightnote{\textcolor{blue}{Agnes Sorma}}}}{\lemma{\textnormal{\emph{Sorma}}}\Cendnote{\textnormal{höchstwahrscheinlich für die
                  Inszenierung von \emph{\textcolor{green}{Der Schleier der Beatrice}} am
                     \textcolor{pink}{Deutschen Theater Berlin} – \textcolor{blue}{Agnes Sorma} gastierte zur Zeit der Premiere, Anfang März 1903, am \emph{\textcolor{brown}{Berliner
                     Theater}}; siehe auch A. S.: \emph{Tagebuch}, 19. 10. 1902}}}\label{K_L03228-3h} nicht zu haben iſt, iſt ſehr bedauerlich. Jetzt rathe ich \strikeout{\textcolor{gray}{ſelbſt}} ganz entſchieden zum \label{K_L03228-4v}\edtext{»\textcolor{brown}{Deutſchen Theater}{}\ledrightnote{\textcolor{brown}{Deutsches Theater Berlin}}«}{\lemma{\textnormal{\emph{»Deutſchen Theater«}}}\Cendnote{\textnormal{für die \textcolor{pink}{Berlin}er Premiere
                  von \emph{\textcolor{green}{Der Schleier der Beatrice}}, wo sie am 7. 3. 1903 auch
                  stattfand; siehe zu \emph{\textcolor{green}{Monna Vanna}} auch A. S.: \emph{Tagebuch}, 24. 11. 1902 und 12. 12. 1902}}}\label{K_L03228-4h}. Da Du ſelbſt die Proben leiten wirſt, iſt eine Chance mehr, daß die
               Aufführung beſſer wird als die der »\textsc{\textcolor{green}{Monna Vanna}{}\ledrightnote{\textcolor{green}{Monna Vanna. Schauspiel in drei Akten}}}«, bei deren Vorbereitung der \textcolor{blue}{Dichter}{}\ledrightnote{{$\rightarrow$}\textcolor{blue}{Maurice Maeterlinck}} nicht mitgewirkt hat. Komm’ nur zu den \label{K_L03228-5v}\edtext{Proben}{\lemma{\textnormal{\emph{Proben}}}\Cendnote{\textnormal{\textcolor{blue}{Schnitzler} kam am 22. 2. 1903 in \textcolor{pink}{Berlin} an. Zwischen 23. 2. 1903 und 6. 3. 1903 war er,
                  abgesehen von einer Pause am Sonntag und Mittwoch vor der \textcolor{green}{Premiere}, täglich bei den Proben.}}}\label{K_L03228-5h} recht bald nach
                  \textcolor{pink}{Berlin}{}\ledrightnote{\textcolor{pink}{Berlin}} und {\pb}bringe Dir gleich das Geld mit, um Dir die gewiſſe \strikeout{klei\textcolor{gray}{e}} kleine \label{K_L03228-45v}\edtext{\textcolor{pink}{Villa}{}\ledrightnote{{$\rightarrow$}\textcolor{pink}{Haus Wiesenstein}} im \textcolor{pink}{Grunewald}{}\ledrightnote{\textcolor{pink}{Grunewald}}}{\lemma{\textnormal{\emph{Villa im Grunewald}}}\Cendnote{\textnormal{Anspielung auf \textcolor{blue}{Gerhart Hauptmann}s \textcolor{pink}{Wohnhaus}}}}\label{K_L03228-45h} zu kaufen.\pend
           
\pstart
           Daß Dein \textcolor{blue}{Sohn}{}\ledrightnote{{$\rightarrow$}\textcolor{blue}{Heinrich Schnitzler}} gedeiht, freut
               mich zu hören. Wenn er ſo viel Symptome von Intelligenz zeigt, wird er ſicherlich ein
               Kritiker werden und gegen die »\textcolor{green}{neue Richtung}{}\ledrightnote{{$\rightarrow$}\textcolor{green}{Die »neue Richtung«. Polemische Aufsätze über Berliner Theater-Aufführungen}}« auftreten. Grüße ihn und ſeine \textcolor{blue}{Mutter}{}\ledrightnote{{$\rightarrow$}\textcolor{blue}{Olga Schnitzler}} vielmals von mir.\pend
           
\pstart
           Beſprechungen über mein \textcolor{green}{Buch}{}\ledrightnote{{$\rightarrow$}\textcolor{green}{Die »neue Richtung«. Polemische Aufsätze über Berliner Theater-Aufführungen}}
               kann ich Dir nicht {\pb}ſchicken, weil keine erſcheinen.
               Es wird todtgeſchwiegen, von den Gegnern wie von den Freunden.\pend
           
\pstart
           Viele herzliche Grüße! {\\[\baselineskip]}Dein {\\[\baselineskip]}\spacefill\mbox{Paul Goldm }\pend
           \leftskip=0em{}\endnumbering\briefempfaengerindex{Schnitzler, Arthur@\textsc{Schnitzler, Arthur}!zzzGoldmann, Paul@\emph{von Paul Goldmann}!1902-10-261@{26. 10. {[}1902{]}}|)be}\mylabel{h}
\begin{anhang}
\end{anhang}\normalsize

\doendnotes{C}
\bigskip
\vfill

\clearpage

\footnotesize

\lohead{\textsc{register}}

% Definiere theindex-Environment komplett neu ohne reledmac
\makeatletter
\renewenvironment{theindex}{%
  \section*{\indexname}%
  \setlength{\parindent}{0pt}%
  \setlength{\parskip}{0pt plus 0.3pt}%
  \let\item\@idxitem
}{%
  \clearpage
}
\makeatother

\IfFileExists{\jobname-pw.ind}{\input{\jobname-pw.ind}}{}

\end{document}

      