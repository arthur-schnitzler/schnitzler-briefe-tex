%% latex-korrekturansicht-vorspann.tex
%% Vorspann für die Korrekturansicht.
%% Lädt die gemeinsame Datei latex-vorspann.tex mit gesetztem Schalter.

\newif\ifkorrekturansicht
\korrekturansichttrue

\input{../tex-inputs/latex-vorspann}


               \section[Hermann Bahr an Arthur Schnitzler, 3. 11. 1893]{ Hermann Bahr an Arthur Schnitzler, 3. 11. 1893}\nopagebreak\mylabel{v}\rehead{ }\normalsize\beginnumbering\briefempfaengerindex{Schnitzler, Arthur@\textsc{Schnitzler, Arthur}!zzzBahr, Hermann@\emph{von Hermann Bahr}!1893-11-031@{3. 11. 1893}|(be} \toendnotes[C]{\smallbreak\pagebreak[2]} \Standort{CUL, Schnitzler, B 5b.}
\physDesc{Brief, 1 Blatt, 1 Seite
\newline{}Handschrift Hermann Bahr: schwarze Tinte, deutsche Kurrent (\noindent{}Unterschrift)\newline{}Handschrift  : schwarze Tinte, deutsche Kurrent\newline{}Ordnung: mit rotem Buntstift von unbekannter Hand und mit Bleistift
                                 jeweils nummeriert: »16« }\buchAbdrucke{\weitereDrucke{Hermann Bahr, Arthur Schnitzler: \emph{Briefwechsel, Aufzeichnungen, Dokumente (1891–1931)}. Hg. Kurt Ifkovits und Martin Anton Müller. Göttingen: \emph{Wallstein} 2018, S. 46.} }\toendnotes[C]{\smallbreak}\pstart
           \noindent{}{\pb}\textcolor{gray}{\textbf{\textcolor{brown}{Deutſche Zeitung}{}\ledrightnote{\textcolor{brown}{Deutsche Zeitung}}}}\hfill \uline{\textcolor{pink}{Wien}{}\ledrightnote{\textcolor{pink}{Wien}}}, 3. Novbr. 1893.\pend
           \pstart
           \textcolor{gray}{\textbf{\textcolor{pink}{Wien}{}\ledrightnote{\textcolor{pink}{Wien}}}}\hfill \textcolor{pink}{III. Saleſianerg. 12}{}\ledrightnote{\textcolor{pink}{Salesianergasse}}\pend
           \pstart
           \textcolor{gray}{\textbf{\textcolor{pink}{IX., Pelikangaſſse 4}{}\ledrightnote{\textcolor{pink}{Pelikangasse}}.}}\pend
           \pstart{}Lieber Freund!\pend\pstart
           Wenn Sie mir nichts anderes geben, will ich es verſuchen den \textsc{\textcolor{green}{Artifex}{}\ledrightnote{\textcolor{green}{Artifex}}} durchzuſetzen. Doch wäre mir aufrichtig geſagt etwas anderes lieber. Aber das
               Wichtigſte bleibt, daſz Sie mir endlich etwas für den \textcolor{green}{Wiener Spiegel}{}\ledrightnote{→\textcolor{green}{Spaziergang}}{ }ſenden – nun haben Sie einmal verſprochen, nun
               hilft Ihnen nichts mehr Sie müſſen in den ſauren Apfel beiſzen und bitte vergeſzen
               Sie mir auch nicht das \label{K_L00278_1v}\edtext{Feuilleton}{\lemma{\textnormal{\emph{Feuilleton}}}\Cendnote{\textnormal{nicht erschienen}}}\label{K_L00278_1h} über \textcolor{blue}{\textsc{Schönlein}}{}\ledrightnote{\textcolor{blue}{Johann Lukas Schönlein}} zu beſorgen.\pend
           \pstart
           Mit herzlichen Grüſzen Ihr treuer{\\[\baselineskip]}\spacefill\mbox{{[}hs. Bahr:{]} Hermann Bahr}\pend
           \leftskip=0em{}\endnumbering\briefempfaengerindex{Schnitzler, Arthur@\textsc{Schnitzler, Arthur}!zzzBahr, Hermann@\emph{von Hermann Bahr}!1893-11-031@{3. 11. 1893}|)be}\mylabel{h}  \normalsize

\doendnotes{C}
\bigskip
\vfill

\clearpage

\footnotesize

\lohead{\textsc{register}}

% Definiere theindex-Environment komplett neu ohne reledmac
\makeatletter
\renewenvironment{theindex}{%
  \section*{\indexname}%
  \setlength{\parindent}{0pt}%
  \setlength{\parskip}{0pt plus 0.3pt}%
  \let\item\@idxitem
}{%
  \clearpage
}
\makeatother

\IfFileExists{\jobname-pw.ind}{\input{\jobname-pw.ind}}{}

\end{document}

      