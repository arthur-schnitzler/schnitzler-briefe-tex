%% latex-korrekturansicht-vorspann.tex
%% Vorspann für die Korrekturansicht.
%% Lädt die gemeinsame Datei latex-vorspann.tex mit gesetztem Schalter.

\newif\ifkorrekturansicht
\korrekturansichttrue

\input{../tex-inputs/latex-vorspann}


               \section[Hermann Bahr: Widmungsexemplar Rezensionen für Arthur Schnitzler, {[}21.?{]} 6. 1903]{ Hermann Bahr: Widmungsexemplar Rezensionen für Arthur Schnitzler,
               {[}21.?{]} 6. 1903}\nopagebreak\mylabel{v}\rehead{ }\normalsize\beginnumbering\briefempfaengerindex{Schnitzler, Arthur@\textsc{Schnitzler, Arthur}!zzzBahr, Hermann@\emph{von Hermann Bahr}!1903-06-211@{{[}21.?{]} 6. 1903}|(be} \toendnotes[C]{\smallbreak\pagebreak[2]} \Standort{DLA, G:Schnitzler, Arthur (Sammlung Heinrich Schnitzler).}
\physDesc{Widmung am Vortitel
\newline{}Handschrift: schwarze Tinte, deutsche Kurrent\newline{}Ordnung: bei der Enteignung des Exemplars 1938 von
                                 unbekannter Hand mit Bleistift ergänzte Informationen:
                                    »Dubl. zu 426.317-B« }\buchAbdrucke{\weitereDrucke{Hermann Bahr, Arthur Schnitzler: \emph{Briefwechsel, Aufzeichnungen, Dokumente (1891–1931)}. Hg. Kurt Ifkovits und Martin Anton Müller. Göttingen: \emph{Wallstein} 2018, S. 266.} }\toendnotes[C]{\smallbreak}\pstart
           \noindent{}{\pb}Meinem lieben Arthur\pend
           \pstart
           herzlichſt{\\[\baselineskip]}\spacefill\mbox{HermannB.}\pend
           \leftskip=0em{}\pstart
           \noindent{}Juni 1903.\pend
           {\bigskip}\pstart
           \noindent{}\centering{}\textcolor{gray}{\textbf{\textcolor{green}{Rezenſionen}{}\ledrightnote{\textcolor{green}{Rezensionen. Wiener Theater 1901 bis 1903}}}}\pend
           {\bigskip}\pstart
           \noindent{}\centering{}{\pb}\textcolor{gray}{\textbf{Hermann Bahr}}\pend
           \pstart
           \noindent{}\centering{}\textcolor{gray}{\textbf{\textcolor{green}{\textbf{Rezenſionen}}{}\ledrightnote{\textcolor{green}{Rezensionen. Wiener Theater 1901 bis 1903}}}}\pend
           \pstart
           \noindent{}\centering{}\textcolor{gray}{\textbf{Wiener Theater}}\pend
           \pstart
           \noindent{}\centering{}\textcolor{gray}{\textbf{1901 bis 1903}}\pend
           {\bigskip}\pstart
           \noindent{}\raggedleft{}\textcolor{gray}{\textbf{\textcolor{green}{»Wenn die Leute glauben,{\\}ich
                     wäre noch in \textcolor{pink}{Weimar}{}\ledrightnote{\textcolor{pink}{Weimar}},{\\} dann bin ich ſchon
                     in \textcolor{pink}{Erfurt}{}\ledrightnote{\textcolor{pink}{Erfurt}}.«}{}\ledrightnote{→\textcolor{green}{Gespräche}}}}\pend
           \pstart
           \noindent{}\raggedleft{}\textcolor{gray}{\textbf{\textcolor{blue}{\so{Goethe}.}{}\ledrightnote{\textcolor{blue}{Johann Wolfgang von Goethe}}}}\pend
           {\bigskip}\pstart
           \noindent{}\centering{}\textcolor{gray}{\textbf{\textcolor{pink}{\textbf{Berlin}}{}\ledrightnote{\textcolor{pink}{Berlin}}{ }1903}}\pend
           \pstart
           \noindent{}\centering{}\textcolor{gray}{\textbf{\textcolor{brown}{\textbf{S. Fiſcher, Verlag}}{}\ledrightnote{\textcolor{brown}{S. Fischer Verlag}}}}\pend
           \endnumbering\briefempfaengerindex{Schnitzler, Arthur@\textsc{Schnitzler, Arthur}!zzzBahr, Hermann@\emph{von Hermann Bahr}!1903-06-211@{{[}21.?{]} 6. 1903}|)be}\mylabel{h}  \normalsize

\doendnotes{C}
\bigskip
\vfill

\clearpage

\footnotesize

\lohead{\textsc{register}}

% Definiere theindex-Environment komplett neu ohne reledmac
\makeatletter
\renewenvironment{theindex}{%
  \section*{\indexname}%
  \setlength{\parindent}{0pt}%
  \setlength{\parskip}{0pt plus 0.3pt}%
  \let\item\@idxitem
}{%
  \clearpage
}
\makeatother

\IfFileExists{\jobname-pw.ind}{\input{\jobname-pw.ind}}{}

\end{document}

      