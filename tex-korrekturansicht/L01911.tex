%% latex-korrekturansicht-vorspann.tex
%% Vorspann für die Korrekturansicht.
%% Lädt die gemeinsame Datei latex-vorspann.tex mit gesetztem Schalter.

\newif\ifkorrekturansicht
\korrekturansichttrue

\input{../tex-inputs/latex-vorspann}


               \section[Richard Beer-Hofmann an Arthur Schnitzler, 18. 1. 1910]{ Richard Beer-Hofmann an Arthur Schnitzler, 18. 1. 1910}\nopagebreak\mylabel{v}\rehead{ }\normalsize\beginnumbering\briefempfaengerindex{Schnitzler, Arthur@\textsc{Schnitzler, Arthur}!zzzBeer-Hofmann, Richard@\emph{von Richard Beer-Hofmann}!1910-01-181@{18. 1. 1910}|(be} \toendnotes[C]{\smallbreak\pagebreak[2]} \Standort{CUL, Schnitzler, B 8.}
\physDesc{Kartenbrief, 1 Blatt, 4 Seiten
\newline{}Handschrift: Bleistift, lateinische Kurrent\newline{}Versand: ohne postalischen Übermittlungsvermerk 
\newline{}Schnitzler: mit Bleistift beschriftet: »\textsc{BH}« \newline{}Ordnung: mit Bleistift von unbekannter Hand nummeriert:
                              »227« }\toendnotes[C]{\smallbreak}\pstart{}{\pb}Herrn\pend{}\pstart{}Arthur Schnitzler\pend{}\pstart{}\textcolor{pink}{Spöttelgasse}{}\ledrightnote{\textcolor{pink}{Edmund-Weiß-Gasse}} 7\pend{}{\bigskip}\pstart
           \raggedleft{}{\pb}18/I 10\pend
           \pstart{}Lieber Arthur!\pend\pstart
           Bitte, veranlassen Sie, dass das bewusste \label{KLL01911_AS-v}\edtext{\textcolor{blue}{Fräulein}{}\ledrightnote{→\textcolor{blue}{Anna Reiter}}}{\lemma{\textnormal{\emph{Fräulein}}}\Cendnote{\textnormal{vgl. A. S.: \emph{Tagebuch}, 19. 1. 1910}}}\label{KLL01911_AS-h}{ }\uline{nicht} zwischen halbdrei – halbvier,
               sondern erst wenn Sie von Ihnen weggeht – also zwischen 6 und
                  7 zu uns kommt{[}.{]}{ }{\pb}Sie collidirt sonst mit den
               Fräuleins die wir von \textcolor{blue}{Schallingers}{}\ledrightnote{\textcolor{blue}{Schallinger}{\newline}\textcolor{blue}{Schallinger}}
               erwarten.\pend
           \pstart
           Herzlichst mit allen guten Wünschen für \label{KLL01911_AS-1v}\edtext{\textcolor{pink}{Dresden}{}\ledrightnote{\textcolor{pink}{Dresden}}}{\lemma{\textnormal{\emph{Dresden}}}\Cendnote{\textnormal{\textcolor{blue}{Schnitzler} reiste am 20. 1. 1910 zur Premiere
                  von \emph{\textcolor{green}{Der Schleier der Pierrette}}.}}}\label{KLL01911_AS-1h}{\\[\baselineskip]}Ihr{\\[\baselineskip]}\spacefill\mbox{Richard}\pend
           \leftskip=0em{}\endnumbering\briefempfaengerindex{Schnitzler, Arthur@\textsc{Schnitzler, Arthur}!zzzBeer-Hofmann, Richard@\emph{von Richard Beer-Hofmann}!1910-01-181@{18. 1. 1910}|)be}\mylabel{h}  \normalsize

\doendnotes{C}
\bigskip
\vfill

\clearpage

\footnotesize

\lohead{\textsc{register}}

% Definiere theindex-Environment komplett neu ohne reledmac
\makeatletter
\renewenvironment{theindex}{%
  \section*{\indexname}%
  \setlength{\parindent}{0pt}%
  \setlength{\parskip}{0pt plus 0.3pt}%
  \let\item\@idxitem
}{%
  \clearpage
}
\makeatother

\IfFileExists{\jobname-pw.ind}{\input{\jobname-pw.ind}}{}

\end{document}

      