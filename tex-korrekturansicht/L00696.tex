%% latex-korrekturansicht-vorspann.tex
%% Vorspann für die Korrekturansicht.
%% Lädt die gemeinsame Datei latex-vorspann.tex mit gesetztem Schalter.

\newif\ifkorrekturansicht
\korrekturansichttrue

\input{../tex-inputs/latex-vorspann}


               \section[Arthur Schnitzler an Georg Brandes, 9. 7. 1897]{ Arthur Schnitzler an Georg Brandes, 9. 7. 1897}\nopagebreak\mylabel{v}\rehead{ }\normalsize\beginnumbering\briefempfaengerindex{Brandes, Georg@\textsc{Brandes, Georg}!zzzSchnitzler, Arthur@\emph{von Arthur Schnitzler}!1897-07-091@{9. 7. 1897}|(be} \toendnotes[C]{\smallbreak\pagebreak[2]} \Standort{Kopenhagen, Det Kongelige Bibliotek, Georg Brandes Arkiv, box 125.}
\physDesc{Brief, 1 Blatt, 4 Seiten
\newline{}Handschrift: schwarze Tinte, deutsche Kurrent\newline{}Ordnung: mit Bleistift von unbekannter Hand nummeriert: »8. Schnitzler« }\buchAbdrucke{\weitereDrucke{Georg Brandes, Arthur Schnitzler: \emph{Ein Briefwechsel}. Hg. Kurt Bergel. Bern: \emph{Francke} 1956, S. 63.} }\toendnotes[C]{\smallbreak}\pstart
           \raggedleft{}{\pb}\textcolor{pink}{\textsc{Ischl, Rudolfshöhe}}{}\ledrightnote{\textcolor{pink}{Hotel und Pension Rudolfshöhe (Leopold Petter)}}{ }9. 7. 97\pend
           \pstart{}Verehrteſter Herr Brandes,\pend\pstart
           hier fällt mir ein \label{K_L00696_1v}\edtext{\textcolor{green}{Zeitungsblatt}{}\ledrightnote{→\textcolor{green}{[Krankheit von Georg Brandes]}}}{\lemma{\textnormal{\emph{Zeitungsblatt}}}\Cendnote{\textnormal{Eine entsprechende Meldung über eine
                            »\textcolor{green}{ungünstige Wendung}« einer
                        Lungenentzündung findet sich etwa in der \emph{\textcolor{brown}{Agramer
                            Zeitung}} vom 9. 7. 1897 (Jg. 72, Nr. 154,
                            S. 6).}}}\label{K_L00696_1h} in die Hand, das von Ihrem Befinden ſchreibt, und
                    aus dem ich nicht klug werde. Sie wiſſen, wie ſehr wir Sie lieben (ich ſpreche
                    noch im Namen einiger anderer Menſchen), und ein Wort, das Sie mir ſchrieben,
                    oder, wenn Sie wirklich noch lei{\pb}dend
                    ſind, mir ſchreiben ließen, brächte viel Beruhigung. Iſt es viel verlangt, wenn
                    ich Sie herzlich bitte, dieſe Zeilen nicht ganz ohne Antwort zu laſſen?\pend
           \pstart
           Ich \introOben{}bin\introOben{} eben im letzten Drittel Ihres \textcolor{green}{\textsc{Shakespeare}}{}\ledrightnote{\textcolor{green}{William Shakespeare}}; langſam und mit einer tiefen Freude an dem wunderbaren {\pb}Entwicklungsgang, den Sie erzählen und
                    einer gleichen Freude an dem unvergleichlichen Erzähler, leſe ich dieſes ſchöne
                    Buch. Was ich immer ſo ſehr an Ihnen bewundre, hier iſt es wieder: wenn Sie ein
                    Werk erklären, ſteigt der Menſch auf, der es geſchaffen; we{\geminationn}
               Sie einen Menſchen ſchildern, ſeine ganze Zeit,
                    und {\pb}\strikeout{und}{ }ſo ko{\geminationm}t aus
                    allem, was Sie geben, der Schein und das Tönen des Lebens über die, welche es
                    faſſen können. Vor ein paar Monaten haben Sie mich gefragt, wie mir Ihr
                        \textcolor{green}{\textsc{Shakespeare}}{}\ledrightnote{\textcolor{green}{William Shakespeare}} gefalle – ſo darf ich Ihnen das alſo ſagen, ohne zudringlich zu
                    ſcheinen. –\pend
           \pstart Ich hoffe ſehr, gutes von Ihnen zu hören, und bald. Meine innigſten Wünſche
                    ſind um Sie. Ihr dankbarer \spacefill\mbox{ArthurSchnitzler.}\pend{}\endnumbering\briefempfaengerindex{Brandes, Georg@\textsc{Brandes, Georg}!zzzSchnitzler, Arthur@\emph{von Arthur Schnitzler}!1897-07-091@{9. 7. 1897}|)be}\mylabel{h}  \normalsize

\doendnotes{C}
\bigskip
\vfill

\clearpage

\footnotesize

\lohead{\textsc{register}}

% Definiere theindex-Environment komplett neu ohne reledmac
\makeatletter
\renewenvironment{theindex}{%
  \section*{\indexname}%
  \setlength{\parindent}{0pt}%
  \setlength{\parskip}{0pt plus 0.3pt}%
  \let\item\@idxitem
}{%
  \clearpage
}
\makeatother

\IfFileExists{\jobname-pw.ind}{\input{\jobname-pw.ind}}{}

\end{document}

      