%% latex-korrekturansicht-vorspann.tex
%% Vorspann für die Korrekturansicht.
%% Lädt die gemeinsame Datei latex-vorspann.tex mit gesetztem Schalter.

\newif\ifkorrekturansicht
\korrekturansichttrue

\input{../tex-inputs/latex-vorspann}


               \section[Robert Adam an Arthur Schnitzler, 22. 7. 1929]{ Robert Adam an Arthur Schnitzler, 22. 7. 1929}\nopagebreak\mylabel{v}\rehead{ }\normalsize\beginnumbering\briefempfaengerindex{Schnitzler, Arthur@\textsc{Schnitzler, Arthur}!zzzAdam, Robert@\emph{von Robert Adam}!1929-07-221@{22. 7. 1929}|(be} \toendnotes[C]{\smallbreak\pagebreak[2]} \Standort{CUL, Schnitzler, B 1.}
\physDesc{Brief, 1 Blatt, 2 Seiten
\newline{}Handschrift: schwarze Tinte, deutsche Kurrent
\newline{}Schnitzler: 1) mit rotem Buntstift beschriftet: »\textsc{Adam}« und »\textsc{Vortrag}« 2) mit rotem Buntstift vereinzelte Unterstreichungen\newline{}Ordnung: von unbekannter Hand nummeriert: »22« }\Standort{Wien, Österreichische Nationalbibliothek, Cod.ser. 52.269, 42 recto.}
\physDesc{handschriftliche Abschrift
\newline{}Handschrift: schwarze Tinte, Gabelsberger Kurzschrift}\Standort{Wien, Österreichische Nationalbibliothek, Cod.ser. 52.269, 43 recto.}
\physDesc{maschinelle Abschrift
\newline{}Schreibmaschine}\toendnotes[C]{\smallbreak}\pstart
           \raggedleft{}{\pb}\textcolor{pink}{Wien}{}\ledrightnote{\textcolor{pink}{Wien}}, den 22. Juli
                        1929.\pend
           \pstart{}Hochverehrter Herr Doktor!\pend\pstart
           Fräulein \textcolor{blue}{Frieda Pollak}{}\ledrightnote{\textcolor{blue}{Frieda Pollak}} hat mich durch
                    Übergabe Ihres »\textcolor{green}{Profeſſor Bernhardi}{}\ledrightnote{\textcolor{green}{Professor Bernhardi. Komödie in fünf Akten}}« und Ihrer
                    freundlichen Gedenkworte überraſcht. Nehmen Sie, bitte, hiefür meinen
                    herzlichſten Dank!\pend
           \pstart
           Wenn ich mir erlaube, dieſen Zeilen den \label{K_L02514_1v}\edtext{\textcolor{green}{Abdruck eines Vortrags}{}\ledrightnote{→\textcolor{green}{Zur Frage des Laienrichtertums beim Handelsgericht}}}{\lemma{\textnormal{\emph{Abdruck eines Vortrags}}}\Cendnote{\textnormal{Überliefert in der \emph{\textcolor{brown}{Österreichischen Nationalbibliothek}}, Cod. Ser. n.
                            52263, Beilage.}}}\label{K_L02514_1h} anzuſchließen, den ich im \textcolor{brown}{Verein der Laienrichter}{}\ledrightnote{\textcolor{brown}{Vereinigung der fachmännischen Laienrichter Österreichs}} hielt, ſo tue ich es mit einigem
                    Zagen und \textsc{faute de mieux}. Während ſonſt bekanntermaßen
                    niemand etwas drucken will, das von mir ſtammt und deſſen Drucklegung mir am
                    Herzen läge, ſo wurde mir diesmal das Manuſkript plötzlich für dieſen im
                    konkreten Fall verruchten Zweck abgefordert und ich wurde nicht ganz nach meinem
                    Wunſch zu einem wenn auch nicht populären, ſo doch populariſirenden Autor
                    kreiert. Die Übersendung soll nur beſagen, daß ich den Wunſch hege, Ihnen einmal
                    mit einem ganzen Buche {\pb}vor Augen
                    treten zu dürfen; ſie iſt als Surrogat dieſer Wunſcherfüllung gewiſſermaßen
                    ſymboliſcher Natur.\pend
           \pstart
           Vielleicht gelingt es mir doch noch einmal, eine Arbeit zuſtande zu bringen, die
                    ich Ihnen mit gutem Gewiſſen vorlegen kann. Sooft ich dem Urlaub nahe bin, hebt
                    ſich die Hoffnung auf Muße, Nervenruhe und Arbeitsluſt und -fähigkeit; ich weiß
                    nur leider aus Erfahrung, daß ſchon die erſte Urlaubswoche eine Enttäuſchung
                    bringt.\pend
           \pstart
           Nehmen Sie, hochverehrter Herr Doktor, nochmals meinen beſten Dank und den
                    Ausdruck meiner tiefen Ergebenheit!\pend
           \pstart \spacefill\mbox{D\textsuperscript{r}RAdam}\pend{}\endnumbering\briefempfaengerindex{Schnitzler, Arthur@\textsc{Schnitzler, Arthur}!zzzAdam, Robert@\emph{von Robert Adam}!1929-07-221@{22. 7. 1929}|)be}\mylabel{h}  \normalsize

\doendnotes{C}
\bigskip
\vfill

\clearpage

\footnotesize

\lohead{\textsc{register}}

% Definiere theindex-Environment komplett neu ohne reledmac
\makeatletter
\renewenvironment{theindex}{%
  \section*{\indexname}%
  \setlength{\parindent}{0pt}%
  \setlength{\parskip}{0pt plus 0.3pt}%
  \let\item\@idxitem
}{%
  \clearpage
}
\makeatother

\IfFileExists{\jobname-pw.ind}{\input{\jobname-pw.ind}}{}

\end{document}

      