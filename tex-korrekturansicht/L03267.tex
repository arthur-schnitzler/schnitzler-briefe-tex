%% latex-korrekturansicht-vorspann.tex
%% Vorspann für die Korrekturansicht.
%% Lädt die gemeinsame Datei latex-vorspann.tex mit gesetztem Schalter.

\newif\ifkorrekturansicht
\korrekturansichttrue

\input{../tex-inputs/latex-vorspann}


\renewcommand{\erwaehnteOrte}{Orte: Café Pucher, London, Wien}
\renewcommand{\erwaehnteWerke}{}
\section[ Felix Salten an Arthur Schnitzler, 1. {[}6.{]} 1897]{Felix Salten an Arthur Schnitzler, 1. {[}6.{]} 1897}
\nopagebreak\mylabel{v}
\rehead{ }\normalsize\beginnumbering\briefempfaengerindex{Schnitzler, Arthur@\textsc{Schnitzler, Arthur}!zzzSalten, Felix@\emph{von Felix Salten}!1897-06-011@{1. {[}6.{]} 1897}|(be}
\toendnotes[C]{\smallbreak\pagebreak[2]}\Standort{CUL, Schnitzler, B 89, A 2.}
\physDesc{Brief, 1 Blatt, 1 Seite, 437 Zeichen
\newline{}Handschrift: schwarze Tinte, lateinische Kurrent
\newline{}Schnitzler: mit Bleistift bei der Datierung der Monatsangabe das »l« durch
                                 ein »n« ersetzt 
\newline{}Ordnung: mit Bleistift von unbekannter Hand nummeriert: »90« }\toendnotes[C]{\smallbreak}
\pstart
           \raggedleft{}{\pb}1. Juli 97\pend
           
\pstart
           Lieber Arthur, eben kommt Ihre \label{K_L03267-1v}\edtext{\textcolor{pink}{London}{}\ledrightnote{\textcolor{pink}{London}}er Karte}{\lemma{\textnormal{\emph{Londoner Karte}}}\Cendnote{\textnormal{Arthur Schnitzler an Felix Salten, 29. 5. 1897}}}\label{K_L03267-1h} und ich freue mich herzlich, Sie so bald wieder zu \label{K_L03267-2v}\edtext{sehen}{\lemma{\textnormal{\emph{sehen}}}\Cendnote{\textnormal{Sie sahen
                  sich gleich am nächsten Tag, vgl. A. S.: \emph{Tagebuch}, 3. 6. 1897.}}}\label{K_L03267-2h}. Ich bin, wenn Sie mir Nachricht geben wollen, wo wir
               uns treffen, jeden Tag bis 4 oder 5 zu Hause. Für die Abschaffung des \textcolor{pink}{Pucher}{}\ledrightnote{\textcolor{pink}{Café Pucher}} bin ich auch. Ohnedies war ich in der
               letzten Zeit nur sehr unregelmäßig dort und wenn wir eine frühe Stunde fürs
               Schlafengehen von Anfang gleich festhalten, ist’s um so besser.\pend
           
\pstart
           Also auf bald {\\[\baselineskip]}Ihr {\\[\baselineskip]}\spacefill\mbox{Salten}\pend
           \leftskip=0em{}\endnumbering\briefempfaengerindex{Schnitzler, Arthur@\textsc{Schnitzler, Arthur}!zzzSalten, Felix@\emph{von Felix Salten}!1897-06-011@{1. {[}6.{]} 1897}|)be}\mylabel{h}  \normalsize

\doendnotes{C}
\bigskip
\vfill

\clearpage

\footnotesize

\lohead{\textsc{register}}

% Definiere theindex-Environment komplett neu ohne reledmac
\makeatletter
\renewenvironment{theindex}{%
  \section*{\indexname}%
  \setlength{\parindent}{0pt}%
  \setlength{\parskip}{0pt plus 0.3pt}%
  \let\item\@idxitem
}{%
  \clearpage
}
\makeatother

\IfFileExists{\jobname-pw.ind}{\input{\jobname-pw.ind}}{}

\end{document}

      