%% latex-korrekturansicht-vorspann.tex
%% Vorspann für die Korrekturansicht.
%% Lädt die gemeinsame Datei latex-vorspann.tex mit gesetztem Schalter.

\newif\ifkorrekturansicht
\korrekturansichttrue

\input{../tex-inputs/latex-vorspann}


               \section[Richard Beer-Hofmann an Arthur Schnitzler, 15. 6. 1898]{ Richard Beer-Hofmann an Arthur Schnitzler, 15. 6. 1898}\nopagebreak\mylabel{v}\rehead{ }\normalsize\beginnumbering\briefempfaengerindex{Schnitzler, Arthur@\textsc{Schnitzler, Arthur}!zzzBeer-Hofmann, Richard@\emph{von Richard Beer-Hofmann}!1898-06-151@{15. 6. 1898}|(be} \toendnotes[C]{\smallbreak\pagebreak[2]} \Standort{CUL, Schnitzler, B 8.}
\physDesc{Brief, 1 Blatt, 3 Seiten
\newline{}Handschrift: Bleistift, lateinische Kurrent\newline{}Ordnung: mit Bleistift von unbekannter Hand nummeriert: »115« }\buchAbdrucke{\weitereDrucke{Arthur Schnitzler, Richard Beer-Hofmann: \emph{Briefwechsel 1891–1931}. Hg. Konstanze Fliedl. Wien, Zürich: \emph{Europaverlag} 1992, S. 118.} }\toendnotes[C]{\smallbreak}\pstart
           \raggedleft{}{\pb}\textcolor{pink}{Steindorf}{}\ledrightnote{\textcolor{pink}{Steindorf am Ossiacher See}}{ }15/VI 1898\pend
           \pstart
           Lieber Arthur! Ich sende Ihnen beiliegend den Korrecturbogen \uline{des} \textcolor{green}{Gedichtes}{}\ledrightnote{→\textcolor{green}{Schlaflied für Mirjam}}. Bitte senden Sie ihn mir baldmöglichst zurück und sagen Sie mir
               ob Sie irgend etwas in der Interpunction stört oder nicht. Außerdem: IV. Zeile, II.
               Strophe:\pend
           \pstart
           »Dir, und auch mir, und uns\textcolor{gray}{«} etc. –\pend
           \leftskip=3em{}\pstart
           \noindent{}oder\pend
           \leftskip=0em{}\pstart
           \noindent{}Dir, und mir, und uns etc. –?\pend
           \pstart
           {\pb}Wenn Sie \textcolor{blue}{Hugo}{}\ledrightnote{\textcolor{blue}{Hugo von Hofmannsthal}} sehen fragen Sie ihn darum; aber bestellen Sie ihn wegen
               dieser wichtigen Angelegenheit nicht ins Caffée da er ja jetzt zu arbeiten hat. –\pend
           \pstart
           Hier gießt es »noch so sehr, und wie geht es Ihnen«?\pend
           \pstart
           Bleibt \textcolor{blue}{Schlenther}{}\ledrightnote{\textcolor{blue}{Paul Schlenther}}?\pend
           \pstart
           Brief, Carton, haben Sie ja wol erhalten? Bitte reco{\geminationm}andiren Sie den Brief mit dem Korrecturbo{\pb}gen. \textcolor{blue}{Hugo}{}\ledrightnote{\textcolor{blue}{Hugo von Hofmannsthal}}, \textcolor{blue}{Schwarzkopf}{}\ledrightnote{\textcolor{blue}{Gustav Schwarzkopf}}, \textcolor{blue}{Leo}{}\ledrightnote{\textcolor{blue}{Leo Van-Jung}} grüßen Sie herzlich von mir.\pend
           \pstart
           Herzlichst Ihr{\\[\baselineskip]}\spacefill\mbox{Richard}\pend
           \leftskip=0em{}\pstart
           \noindent{}Soeben erhalte ich Brief von \textcolor{blue}{Hugo}{}\ledrightnote{\textcolor{blue}{Hugo von Hofmannsthal}} – \pend
           \endnumbering\briefempfaengerindex{Schnitzler, Arthur@\textsc{Schnitzler, Arthur}!zzzBeer-Hofmann, Richard@\emph{von Richard Beer-Hofmann}!1898-06-151@{15. 6. 1898}|)be}\mylabel{h}  \normalsize

\doendnotes{C}
\bigskip
\vfill

\clearpage

\footnotesize

\lohead{\textsc{register}}

% Definiere theindex-Environment komplett neu ohne reledmac
\makeatletter
\renewenvironment{theindex}{%
  \section*{\indexname}%
  \setlength{\parindent}{0pt}%
  \setlength{\parskip}{0pt plus 0.3pt}%
  \let\item\@idxitem
}{%
  \clearpage
}
\makeatother

\IfFileExists{\jobname-pw.ind}{\input{\jobname-pw.ind}}{}

\end{document}

      