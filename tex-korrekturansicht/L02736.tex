%% latex-korrekturansicht-vorspann.tex
%% Vorspann für die Korrekturansicht.
%% Lädt die gemeinsame Datei latex-vorspann.tex mit gesetztem Schalter.

\newif\ifkorrekturansicht
\korrekturansichttrue

\input{../tex-inputs/latex-vorspann}


               \section[Paul Goldmann an Arthur Schnitzler, 7. 6. {[}1895{]}]{ Paul Goldmann an Arthur Schnitzler, 7. 6. {[}1895{]}}\nopagebreak\mylabel{v}\rehead{ }\normalsize\beginnumbering\briefempfaengerindex{Schnitzler, Arthur@\textsc{Schnitzler, Arthur}!zzzGoldmann, Paul@\emph{von Paul Goldmann}!1895-06-072@{7. 6. {[}1895{]}}|(be} \toendnotes[C]{\smallbreak\pagebreak[2]} \Standort{DLA, A:Schnitzler, HS.NZ85.1.3165.}
\physDesc{Brief, 2 Blätter, 8 Seiten
\newline{}Handschrift: schwarze Tinte, deutsche Kurrent
\newline{}Schnitzler: 1) mit Bleistift das Jahr »95« vermerkt 2) mit rotem Buntstift fünf Unterstreichungen}\toendnotes[C]{\smallbreak}\pstart
           \noindent{}{\pb}\textcolor{gray}{\textbf{\textbf{\textcolor{brown}{Frankfurter Zeitung}{}\ledrightnote{\textcolor{brown}{Frankfurter Zeitung}}}}}\pend
           \pstart
           \textcolor{gray}{\textbf{(\textcolor{brown}{\begin{otherlanguage}{french}Gazette de Francfort\end{otherlanguage}}{}\ledrightnote{\textcolor{brown}{Frankfurter Zeitung}}). }}\pend
           \pstart
           \textcolor{gray}{\textbf{\textbf{\begin{otherlanguage}{french}Fondateur M. \textcolor{blue}{L.
                              Sonnemann}{}\ledrightnote{\textcolor{blue}{Leopold Sonnemann}}\end{otherlanguage}.}}}\pend
           \pstart
           \begin{otherlanguage}{french}\textcolor{gray}{\textbf{\textcolor{green}{Journal}{}\ledrightnote{→\textcolor{green}{Frankfurter Zeitung}} politique,
                        financier,}}\end{otherlanguage}\pend
           \pstart
           \begin{otherlanguage}{french}\textcolor{gray}{\textbf{commercial et littéraire.}}\end{otherlanguage}\pend
           \pstart
           \begin{otherlanguage}{french}\textcolor{gray}{\textbf{\textbf{Paraissant trois fois par jour.}}}\end{otherlanguage}\hfill \textsc{\textcolor{pink}{Paris}{}\ledrightnote{\textcolor{pink}{Paris}}}, 7. Juni.\pend
           \pstart
           \begin{otherlanguage}{french}\textcolor{gray}{\textbf{\textbf{Bureau à \textcolor{pink}{Paris}{}\ledrightnote{\textcolor{pink}{Paris}}:}}}\end{otherlanguage}\pend
           \pstart
           \begin{otherlanguage}{french}\textcolor{gray}{\textbf{\textbf{\textcolor{pink}{24. Rue Feydeau}{}\ledrightnote{\textcolor{pink}{rue Feydeau}}.}}}\end{otherlanguage}\pend
           \pstart\center{}Mein lieber Freund,\pend\pstart
           Noch immer nicht der große Brief. Ich bin zu lebensmüde, zu hoffnungslos. Von allen
               Seiten wird es enge um mich, und kein Ausweg, keiner!\pend
           \pstart
           Nur Folgendes: \textsc{\textcolor{blue}{Isidor Fuchs}{}\ledrightnote{\textcolor{blue}{Isidor Fuchs}}}, der ein verläßlicher \textcolor{blue}{Vertrauensmann}{}\ledrightnote{→\textcolor{blue}{Isidor Fuchs}} iſt, frug mich um Dein \textcolor{green}{Stück}{}\ledrightnote{→\textcolor{green}{Liebelei. Schauspiel in drei Akten}}. Ich ſagte ihm, die Schwierig{\pb}keiten, die ſich ihm bisher entgegengeſtellt, lagen
               wohl in den Kühnheiten, die es hat. Worauf \textsc{\textcolor{blue}{Fuchs}{}\ledrightnote{\textcolor{blue}{Isidor Fuchs}}} ſolgenden Vorſchlag machte: Man ſolle es zuerſt in einer jener Vorſtellungen
               zum Benefiz der »\textsc{\textcolor{brown}{Concordia}{}\ledrightnote{\textcolor{brown}{Concordia}}}« geben, bei denen die \textcolor{brown}{Burg}{}\ledrightnote{→\textcolor{brown}{Burgtheater}}ſchauſpieler alljährlich mitwirken. Präcedenzfälle ſind da, wo ein \textcolor{blue}{\textcolor{brown}{Burgtheater}{}\ledrightnote{\textcolor{brown}{Burgtheater}}-Direktor}{}\ledrightnote{→\textcolor{blue}{Max Eugen Burckhard}} ein Stück auf dieſe
               Weiſe zuerſt dem Publikum vorführte, {\pb}gleichſam
               probeweiſe, um \strikeout{den} die Stimmung des Publikums zu
               ſondiren. \textsc{\textcolor{blue}{Fuchs}{}\ledrightnote{\textcolor{blue}{Isidor Fuchs}}}, der, wie Du weißt, ein einflußreiches \textcolor{blue}{Mitglied}{}\ledrightnote{→\textcolor{blue}{Isidor Fuchs}} der »\textsc{\textcolor{brown}{Concordia}{}\ledrightnote{\textcolor{brown}{Concordia}}}« iſt, will Dir gern die Sache bei \label{K_L02736-3v}\edtext{\textsc{\textcolor{blue}{Spigl}{}\ledrightnote{\textcolor{blue}{Edgar von Spiegl-Thurnsee}}}}{\lemma{\textnormal{\emph{Spigl}}}\Cendnote{\textnormal{\textcolor{blue}{Edgar von Spiegl-Thurnsee}, \textcolor{blue}{Vizepräsident} der \emph{\textcolor{brown}{Concordia}}. Es sind keine Bemühungen um eine
                  Aufführung der \emph{\textcolor{green}{Liebelei}} bei einer \emph{\textcolor{brown}{Concordia}}-Veranstaltung bekannt.}}}\label{K_L02736-3h}
               richten. Er meint, auch \textsc{\textcolor{blue}{Burckhardt}{}\ledrightnote{\textcolor{blue}{Max Eugen Burckhard}}} würde mit Freuden zustimmen, und ſo könnte man am Besten ein weiteres
               Hinausſchieben der \textcolor{green}{Aufführung}{}\ledrightnote{→\textcolor{green}{Liebelei. Schauspiel in drei Akten}}
               verhindern. Außerdem gibt eine \textsc{\textcolor{brown}{Concordia}{}\ledrightnote{\textcolor{brown}{Concordia}}}-Vor{\pb}ſtellung eine gewiſſe Garantie für
               günſtige Referate. Was ſagſt Du zu dem Vorſchlag? Du ſollteſt ihn meiner Anſicht nach
               freilich nur annehmen, wenn Du nicht ein \uline{bindendes}
               Verſprechen von \textsc{\textcolor{blue}{Burckhardt}{}\ledrightnote{\textcolor{blue}{Max Eugen Burckhard}}} erhalten könnteſt, \textcolor{green}{Dich}{}\ledrightnote{→\textcolor{green}{Liebelei. Schauspiel in drei Akten}}{ }\uline{bald} aufzuführen. Es wäre aber nur eine Brücke für
               die \textcolor{blue}{Director}{}\ledrightnote{→\textcolor{blue}{Max Eugen Burckhard}}en-Feigheit.\pend
           \pstart
           Die \textsc{\textcolor{blue}{Sorma}{}\ledrightnote{\textcolor{blue}{Agnes Sorma}}} iſt in \textsc{\textcolor{pink}{Paris}{}\ledrightnote{\textcolor{pink}{Paris}}}. \textsc{\textcolor{blue}{Th. Wolff}{}\ledrightnote{\textcolor{blue}{Theodor Wolff}}}, der hier \textsc{\textcolor{blue}{Correſpondent}{}\ledrightnote{→\textcolor{blue}{Theodor Wolff}}}{ }{\pb}des »\textcolor{brown}{Berliner
                  Tageblatt}{}\ledrightnote{\textcolor{brown}{Berliner Tageblatt}}« iſt, wird mich ihr vorſtellen, und ich werde ihr von Dir
               ſprechen.\pend
           \pstart
           \textsc{\begin{otherlanguage}{french}À propos\end{otherlanguage}{ }\textcolor{blue}{Wolff}{}\ledrightnote{\textcolor{blue}{Theodor Wolff}}}: er hat in \textcolor{pink}{Berlin}{}\ledrightnote{\textcolor{pink}{Berlin}} eine \textcolor{blue}{Geliebte}{}\ledrightnote{→\textcolor{blue}{Mizi Rosner}}{ }\strikeout{\textcolor{gray}{f}} gehabt, die ihm lieber war, als alle andern: \label{K_L02736-1v}\edtext{\textsc{\textcolor{blue}{Mizzi Rosner}{}\ledrightnote{\textcolor{blue}{Mizi Rosner}}}}{\lemma{\textnormal{\emph{Mizzi Rosner}}}\Cendnote{\textnormal{\textcolor{blue}{Schauspielerin} und
                  ehemalige \textcolor{blue}{Geliebte}{ }\textcolor{blue}{Schnitzler}s}}}\label{K_L02736-1h}. Die Fäden, die
               Fäden!\pend
           \pstart
           Und \textsc{\textcolor{blue}{Nordau}{}\ledrightnote{\textcolor{blue}{Max Nordau}}s}{ }{\pb}\label{K_L02736-4v}\edtext{\textcolor{green}{Debüt}{}\ledrightnote{→\textcolor{green}{Die Kunst in den elysäischen Feldern}{\newline}→\textcolor{green}{Marsfeldsalon-Typen}}}{\lemma{\textnormal{\emph{Debüt}}}\Cendnote{\textnormal{Im Mai 1895 erschienen zwei
                  Feuilletons von \textcolor{blue}{Max Nordau} in der \emph{\textcolor{green}{Neuen Freie Presse}}: \emph{\textcolor{green}{ Marsfeldsalon-Typen}}. In: \emph{\textcolor{green}{Neue Freie Presse}}, Nr. 11.027, 7. 5. 1895,
                     Morgenblatt, S. 1–4 und \emph{\textcolor{green}{Die Kunst in den elysäischen Feldern}}. In:
                        \emph{\textcolor{green}{Neue Freie Presse}}, Nr. 11.038,
                        18. 5. 1895, Morgenblatt, S. 1–3. }}}\label{K_L02736-4h} in der »\textcolor{green}{Neuen Freien Presse}{}\ledrightnote{\textcolor{green}{Neue Freie Presse}}«? \strikeout{D\textcolor{gray}{a}} Die langſame Vorbereitung zu \label{K_L02736-2v}\edtext{\textsc{\textcolor{blue}{Herzl}{}\ledrightnote{\textcolor{blue}{Theodor Herzl}}s}{ }\textcolor{blue}{Nachfolger}{}\ledrightnote{→\textcolor{blue}{Max Nordau}}ſchaft}{\lemma{\textnormal{\emph{Herzls Nachfolgerſchaft}}}\Cendnote{\textnormal{\textcolor{blue}{Nordau} wurde \textcolor{pink}{Paris}er \textcolor{blue}{Kultur-Korrespondent} der \emph{\textcolor{brown}{Neuen Freien
                     Presse}}. Im Mai 1895 erschienen in dieser Zeitung neben \textcolor{blue}{Nordau}s Feuilleton auch zwei ausführliche
                  Theaterberichte \textcolor{blue}{Herzl}s.}}}\label{K_L02736-2h}. Du ahnſt
               gar nicht, was für frecher Blödſinn in dieſen Kunſtartikeln ſtand. Aber er iſt der
               große \textcolor{blue}{Schriftſteller}{}\ledrightnote{→\textcolor{blue}{Max Nordau}}, \textsc{\textcolor{blue}{Herzl}{}\ledrightnote{\textcolor{blue}{Theodor Herzl}}} ſelbſt hat ihn candidirt, ich bin ein guter Reporter und zähle nicht mit. Von
                  \textsc{\textcolor{blue}{Herzl}{}\ledrightnote{\textcolor{blue}{Theodor Herzl}}} überraſcht mich das nicht. {\pb}Trotz aller
               äußeren Collegialitäts-Tünche haben wir uns im Grunde immer gehaßt, und ich habe auch
               nichts gemeinſam mit dieſem engherzigen, doktrinär vernagelten Menſchen von echt
                  rabbiniſtiſchem Spitz- und Dörr-Geiſte.\pend
           \pstart
           Nur thut es eben gar ſo weh, ſich ſo übergangen zu ſehen {\pb}und immer und ewig der Menſch zweiten oder dritten
               Ranges zu ſein.\pend
           \pstart
           Grüß’ Dich Gott, mein lieber Freund, und laß wieder von Dir hören!\pend
           \pstart
           Dein {\\[\baselineskip]}treuer {\\[\baselineskip]}\spacefill\mbox{Paul Goldmann}\pend
           \leftskip=0em{}\endnumbering\briefempfaengerindex{Schnitzler, Arthur@\textsc{Schnitzler, Arthur}!zzzGoldmann, Paul@\emph{von Paul Goldmann}!1895-06-072@{7. 6. {[}1895{]}}|)be}\mylabel{h}  \normalsize

\doendnotes{C}
\bigskip
\vfill

\clearpage

\footnotesize

\lohead{\textsc{register}}

% Definiere theindex-Environment komplett neu ohne reledmac
\makeatletter
\renewenvironment{theindex}{%
  \section*{\indexname}%
  \setlength{\parindent}{0pt}%
  \setlength{\parskip}{0pt plus 0.3pt}%
  \let\item\@idxitem
}{%
  \clearpage
}
\makeatother

\IfFileExists{\jobname-pw.ind}{\input{\jobname-pw.ind}}{}

\end{document}

      