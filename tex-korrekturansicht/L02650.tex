%% latex-korrekturansicht-vorspann.tex
%% Vorspann für die Korrekturansicht.
%% Lädt die gemeinsame Datei latex-vorspann.tex mit gesetztem Schalter.

\newif\ifkorrekturansicht
\korrekturansichttrue

\input{../tex-inputs/latex-vorspann}


               \section[Paul Goldmann an Arthur Schnitzler, 25. 9. 1890]{ Paul Goldmann an Arthur Schnitzler, 25. 9. 1890}\nopagebreak\mylabel{v}\rehead{ }\normalsize\beginnumbering\briefempfaengerindex{Schnitzler, Arthur@\textsc{Schnitzler, Arthur}!zzzGoldmann, Paul@\emph{von Paul Goldmann}!1890-09-251@{25. 9. 1890}|(be} \toendnotes[C]{\smallbreak\pagebreak[2]} \Standort{DLA, A:Schnitzler, HS.NZ85.1.3162.}
\physDesc{Brief, 1 Blatt, 2 Seiten
\newline{}Handschrift: blaue Tinte, deutsche Kurrent}\toendnotes[C]{\smallbreak}\pstart
           \noindent{}\centering{}{\pb}\textcolor{gray}{\textbf{\textbf{Adminiſtration: \textcolor{pink}{VII.
                           Seidengaſſe 7}{}\ledrightnote{\textcolor{pink}{Seidengasse}}} (\textcolor{brown}{Jos. Eberle {\kaufmannsund} Co.}{}\ledrightnote{\textcolor{brown}{Josef Eberle  Stein-, Buch und Musikaliendruckerei}})}}\pend
           \pstart
           \noindent{}\centering{}\textcolor{gray}{\textbf{\textcolor{brown}{An der Schönen Blauen Donau}{}\ledrightnote{\textcolor{brown}{An der schönen blauen Donau}}}}\pend
           \pstart
           \noindent{}\centering{}\textcolor{gray}{\textbf{Chef-Redacteur: Dr. \textcolor{blue}{F.
                        Mamroth}{}\ledrightnote{\textcolor{blue}{Fedor Mamroth}}. – Redaction: \textcolor{pink}{IX.,
                        Berggaſſe 31}{}\ledrightnote{\textcolor{pink}{Berggasse}}.}}\pend
           \pstart
           \raggedleft{}\textcolor{gray}{\textbf{\textcolor{pink}{Wien}{}\ledrightnote{\textcolor{pink}{Wien}}, den}}{ }25. September \textcolor{gray}{\textbf{18}}90.\pend
           \pstart\center{}Mein lieber Arthur!\pend\pstart
           Es hat ſich ſo getroffen, daß ich erſt heut nach \textcolor{pink}{Salzburg}{}\ledrightnote{\textcolor{pink}{Salzburg}} fahre. Ich \label{K_L02650-1v}\edtext{ſuche
               Dich in den nächſten Tagen auf}{\lemma{\textnormal{\emph{ſuche … auf}}}\Cendnote{\textnormal{\textcolor{blue}{Schnitzler} hielt sich vom 18. 9. 1890 bis zum 4. 10. 1890 in \textcolor{pink}{Salzburg} auf, um hier ein paar Tage mit \textcolor{blue}{Marie Glümer} verbringen zu können.}}}\label{K_L02650-1h}
               und bitte Dich, täglich im \textcolor{pink}{Hotel}{}\ledrightnote{→\textcolor{pink}{Österreichischer Hof}} eine Notiz zu hinterlaſſen, wo Du \label{K_L02650-12v}\edtext{zu finden biſt}{\lemma{\textnormal{\emph{zu finden biſt}}}\Cendnote{\textnormal{Sie trafen sich am 27. 9. 1890, 28. 9. 1890 und 29. 9. 1890.}}}\label{K_L02650-12h}, das heißt wenigſtens zu gewiſſen Hauptzeiten
               des Tages, zum Mittag- und Nachtmahl. Erſt muß ich nämlich mit meinem \label{K_L02650-2v}\edtext{\textcolor{blue}{Onkel}{}\ledrightnote{→\textcolor{blue}{Fedor Mamroth}}}{\lemma{\textnormal{\emph{Onkel}}}\Cendnote{\textnormal{Auch \textcolor{blue}{Fedor Mamroth} reiste mit nach \textcolor{pink}{Salzburg}.}}}\label{K_L02650-2h} das Viele, was vorliegt, beſprechen, und dann kann ich erſt
               zu Dir.\pend
           \pstart
           {\pb}Da ich die wenigen Stunden vor meiner Abreiſe alle
               Hände voll zu thun habe, kann ich \label{K_L02650-11v}\edtext{Deinen lieben Brief}{\lemma{\textnormal{\emph{Deinen lieben Brief}}}\Cendnote{\textnormal{Der Inhalt des
                  Briefes ist unklar. Aus der verspäteten Antwort, die \textcolor{blue}{Goldmann} hier rechtfertigt, geht zumindest hervor, dass er
                     \textcolor{blue}{Schnitzler} ins Vertrauen über eine
                  Krankheit gesetzt habe, an der er leide. Genaueres lässt sich nicht bestimmen,
                  doch dürfte es sich eher um eine psychische Disposition als um etwas Behandelbares
                  gehandelt haben. (Paul Goldmann an Arthur Schnitzler, 1. 10. 1890)}}}\label{K_L02650-11h} nicht beantworten, ſo ſehr \strikeout{ich} es mich dazu drängt. Mündlich läßt ſich das aber nicht ſagen, wie Du
               mit feinem Tact herausgefühlt. Ich denke alſo, wir betrachten ihn für die Stunden
               unſeres jetzigen Zuſammenſeins als nicht geſchrieben und reden nicht davon. Willſt Du
               aber doch davon reden, ſo fang’ Du an. Sonſt ſchreibe ich Dir all’ das Viele, was ich
               darauf zu bemerken habe, nach meiner Rückkehr. Einſtweilen danke ich Dir für die
               männliche und offene Rede!\pend
           \pstart
           Gott zum Gruß! Auf Wiederſehen! {\\[\baselineskip]} Dein {\\[\baselineskip]}\spacefill\mbox{Paul Goldmann.}\pend
           \leftskip=0em{}\endnumbering\briefempfaengerindex{Schnitzler, Arthur@\textsc{Schnitzler, Arthur}!zzzGoldmann, Paul@\emph{von Paul Goldmann}!1890-09-251@{25. 9. 1890}|)be}\mylabel{h}  \normalsize

\doendnotes{C}
\bigskip
\vfill

\clearpage

\footnotesize

\lohead{\textsc{register}}

% Definiere theindex-Environment komplett neu ohne reledmac
\makeatletter
\renewenvironment{theindex}{%
  \section*{\indexname}%
  \setlength{\parindent}{0pt}%
  \setlength{\parskip}{0pt plus 0.3pt}%
  \let\item\@idxitem
}{%
  \clearpage
}
\makeatother

\IfFileExists{\jobname-pw.ind}{\input{\jobname-pw.ind}}{}

\end{document}

      