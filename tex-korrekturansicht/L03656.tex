%% latex-korrekturansicht-vorspann.tex
%% Vorspann für die Korrekturansicht.
%% Lädt die gemeinsame Datei latex-vorspann.tex mit gesetztem Schalter.

\newif\ifkorrekturansicht
\korrekturansichttrue

\input{../tex-inputs/latex-vorspann}


\section[Stefan Zweig an Arthur Schnitzler, {[}17. 7. 1915?{]}]{L03656 Stefan Zweig an Arthur Schnitzler, {[}17. 7. 1915?{]}}
\nopagebreak\mylabel{L03656v}
\rehead{ }\normalsize\beginnumbering\briefempfaengerindex{, @\textsc{, }!zzz, @\emph{von  }!1915-07-171@{{[}17. 7. 1915?{]}}|(be}
\toendnotes[C]{\smallbreak\pagebreak[2]}\Standort{CUL, Schnitzler, B 118.}
\physDesc{Bildpostkarte, 494 Zeichen
\newline{}Handschrift: lila Tinte, lateinische Kurrent
\newline{}Versand: Stempel: »\nobreak{}K. u. K. MILITÄRZENSUR\nobreak{}«.  }
\buchAbdrucke{\weitereDrucke{Stefan Zweig: \emph{Briefwechsel mit Hermann Bahr, Sigmund Freud, Rainer Maria
                        Rilke und Arthur Schnitzler}. Herausgegeben von Jeffrey B. Berlin,  Hans-Ulrich Lindken und  Donald A. Prater. Frankfurt am Main: \emph{S. Fischer} 1987, S. 395.} }\toendnotes[C]{\smallbreak}\pstart{}Stefan Zweig, Einj. Freiw. Feldwebel\pend{}\pstart{}zugeteilt dem \textcolor{brown}{K. u. K Kriegsarchiv}\orgindex{Kriegsarchiv@Kriegsarchiv|pw}{}\ledrightnote{\textcolor{brown}{Kriegsarchiv}}\pend{}\pstart{}auf Dienstreise derzeit \textcolor{pink}{Przemysl}\oindex{Przemyśl@\textbf{Przemyśl}, \emph{Hauptstadt}|pw}{}\ledrightnote{\textcolor{pink}{Przemyśl}}\pend{}{\bigskip}\pstart{}{\pb}\textcolor{gray}{\textbf{Pocztówka – }}\substVorne{}\textsuperscript{\textcolor{gray}{\textbf{Post}}}\substDazwischen{}Feldpost\substHinten{}\textcolor{gray}{\textbf{karte}}\pend{}\pstart{}D\textsuperscript{r} Artur Schnitzler\pend{}\pstart{}\textcolor{pink}{Wien – Cottage}\oindex{Wien@\textbf{Wien}!XVIII., Währing@\textbf{XVIII., Währing}!Währinger Cottage@\textbf{Währinger Cottage}, \emph{Teil eines besiedelten Ortes}|pw}{}\ledrightnote{\textcolor{pink}{Währinger Cottage}}\pend{}\pstart{}\textcolor{pink}{\label{K_L03656-1v}\edtext{Sternwartestrasse 72}{\lemma{\textnormal{\emph{Sternwartestrasse 72}}}\Cendnote{\textnormal{\textcolor{blue}{Zweig}\pwindex{Zweig, Stefan 28.\,11.\,1881 Wien – 23.\,2.\,1942 Petrópolis@\textsc{Zweig, Stefan} (28.\,11.\,1881 Wien – 23.\,2.\,1942 Petrópolis), \emph{Schriftsteller}|pwk} wechselt bei der Adressierung
                        seiner Schreiben an \textcolor{blue}{Schnitzler} immer
                        wieder zwischen der falschen Hausnummer »72« und der
                        richtigen »71«.}}}\label{K_L03656-1}}\oindex{Wien@\textbf{Wien}!XVIII., Währing@\textbf{XVIII., Währing}!Sternwartestraße 71@\textbf{Sternwartestraße 71}, \emph{Wohngebäude}|pw}{}\ledrightnote{\textcolor{pink}{Sternwartestraße 71}}\pend{}{\bigskip}
\pstart
           \noindent{}\centering{}{\pb}\textcolor{gray}{\textbf{\textcolor{pink}{Przemyśl, ul. Mickiewicza}\oindex{Adama Mickiewicza@\textbf{Adama Mickiewicza}, \emph{Straße}|pw}{}\ledrightnote{\textcolor{pink}{Adama Mickiewicza}} – \textcolor{pink}{Przemyśl, Mickiewiczstrasse}\oindex{Adama Mickiewicza@\textbf{Adama Mickiewicza}, \emph{Straße}|pw}{}\ledrightnote{\textcolor{pink}{Adama Mickiewicza}}.}}\pend
           \vspace{1em}
\pstart
           \noindent{}{\pb}Lieber verehrter Herr Doktor, ich habe in diesen \label{K_L03656-2v}\edtext{\textcolor{pink}{galizischen}\oindex{Galizien@\textbf{Galizien}|pw}{}\ledrightnote{\textcolor{pink}{Galizien}} Tagen}{\lemma{\textnormal{\emph{galizischen Tagen}}}\Cendnote{\textnormal{\textcolor{blue}{Zweig}\pwindex{Zweig, Stefan 28.\,11.\,1881 Wien – 23.\,2.\,1942 Petrópolis@\textsc{Zweig, Stefan} (28.\,11.\,1881 Wien – 23.\,2.\,1942 Petrópolis), \emph{Schriftsteller}|pwk} war vom 13. 7. 1915 bis
                  zum 26. 7. 1915 im Kriegsgebiet von \textcolor{pink}{Galizien}\oindex{Galizien@\textbf{Galizien}|pwk}, kurz nachdem die \textcolor{pink}{russische}\oindex{Russland@\textbf{Russland}|pwk} Armee zurückgedrängt worden war. \textcolor{blue}{Zweigs}\pwindex{Zweig, Stefan 28.\,11.\,1881 Wien – 23.\,2.\,1942 Petrópolis@\textsc{Zweig, Stefan} (28.\,11.\,1881 Wien – 23.\,2.\,1942 Petrópolis), \emph{Schriftsteller}|pwk} Reise begann in \textcolor{pink}{Krakau}\oindex{Krakau@\textbf{Krakau}, \emph{Verwaltungsgebiet}|pwk} und endete in \textcolor{pink}{Budapest}\oindex{Budapest@\textbf{Budapest}, \emph{Hauptstadt}|pwk}.
                        (\emph{\textcolor{green}{Tagebuch aus dem Kriegsjahr 1915.
                        Zweiter Band}\pwindex{Zweig, Stefan 28.\,11.\,1881 Wien – 23.\,2.\,1942 Petrópolis@\textsc{Zweig, Stefan} (28.\,11.\,1881 Wien – 23.\,2.\,1942 Petrópolis), \emph{Schriftsteller}!Tagebuch aus dem Kriegsjahr 1915. Zweiter Band@\strich\emph{Tagebuch aus dem Kriegsjahr 1915. Zweiter Band}|pwk}}, SZ-AAP/L3. SZ-AAP/L3) Ein Tag vor Reisebeginn
                  nannte er in einem Brief an \textcolor{blue}{Franz Karl Ginzkey}\pwindex{Ginzkey, Franz Karl 8.\,9.\,1871 Pula – 11.\,4.\,1963 Wien@\textsc{Ginzkey, Franz Karl} (8.\,9.\,1871 Pula – 11.\,4.\,1963 Wien), \emph{Schriftsteller}|pwk} die
                  Route: »\textcolor{pink}{Tarnow}\oindex{Tarnów@\textbf{Tarnów}, \emph{Hauptstadt}|pw}, \textcolor{pink}{Przemysl}\oindex{Przemyśl@\textbf{Przemyśl}, \emph{Hauptstadt}|pw}, \textcolor{pink}{Lemberg}\oindex{Lviv@\textbf{Lviv}|pw}, \textcolor{pink}{Stryi}\oindex{Stryj@\textbf{Stryj}|pw}, \textcolor{pink}{Drohobycz}\oindex{Drohobych@\textbf{Drohobych}, \emph{Hauptstadt}|pw}, \textcolor{pink}{Ungvar}\oindex{Uzhhorod@\textbf{Uzhhorod}|pw}«.
                  (\textcolor{blue}{Stefan Zweig}\pwindex{Zweig, Stefan 28.\,11.\,1881 Wien – 23.\,2.\,1942 Petrópolis@\textsc{Zweig, Stefan} (28.\,11.\,1881 Wien – 23.\,2.\,1942 Petrópolis), \emph{Schriftsteller}|pwk}: \emph{Briefe. Bd. II: 1914–1919}. Herausgegeben von Knut Beck, Jeffrey B. Berlin und Natascha
                     Weschenbach-Feggeler. Frankfurt am Main: \emph{S. Fischer}{ }1998, S. 76.)
                  Von den Herausgebern auf den Folgetag umdatiert ist ein von \textcolor{blue}{Zweig}\pwindex{Zweig, Stefan 28.\,11.\,1881 Wien – 23.\,2.\,1942 Petrópolis@\textsc{Zweig, Stefan} (28.\,11.\,1881 Wien – 23.\,2.\,1942 Petrópolis), \emph{Schriftsteller}|pwk}
                  mit »16. Juli 1915« datiertes Schreiben aus \textcolor{pink}{Przemyśl}\oindex{Przemyśl@\textbf{Przemyśl}, \emph{Hauptstadt}|pwk}
                  an \textcolor{blue}{Raoul Auernheimer}\pwindex{Auernheimer, Raoul 15.\,4.\,1876 Wien – 6.\,1.\,1948 Oakland@\textsc{Auernheimer, Raoul} (15.\,4.\,1876 Wien – 6.\,1.\,1948 Oakland), \emph{Schriftsteller, Journalist, Kritiker}|pwk}, das in der Eröffnung 
                  ebenfalls die »\textcolor{pink}{galizischen}\oindex{Galizien@\textbf{Galizien}|pw} Tage[]« erwähnt: »Lieber Freund, ich habe jetzt heiße und herrliche Tage hier in \textcolor{pink}{Galizien}\oindex{Galizien@\textbf{Galizien}|pw}«
                   (ebd., S. 77). Die dichte Reiseroute spricht dafür, dass sich \textcolor{blue}{Zweig}\pwindex{Zweig, Stefan 28.\,11.\,1881 Wien – 23.\,2.\,1942 Petrópolis@\textsc{Zweig, Stefan} (28.\,11.\,1881 Wien – 23.\,2.\,1942 Petrópolis), \emph{Schriftsteller}|pwk} nur
                  kurz in \textcolor{pink}{Przemyśl}\oindex{Przemyśl@\textbf{Przemyśl}, \emph{Hauptstadt}|pwk} aufhielt und diese Karte am selben Tag
                  wie der Brief an \textcolor{blue}{Auernheimer}\pwindex{Auernheimer, Raoul 15.\,4.\,1876 Wien – 6.\,1.\,1948 Oakland@\textsc{Auernheimer, Raoul} (15.\,4.\,1876 Wien – 6.\,1.\,1948 Oakland), \emph{Schriftsteller, Journalist, Kritiker}|pwk} verfasst wurde.}}}\label{K_L03656-2}
               Unendliches gesehn: den ungeheuren Gang dieser gewaltigen Centripetalmaschine, die
               alle Kraft eines Reiches mit Wucht nach aussen schleudert und dann eine tragische
               aber doch schöne Welt: \textcolor{pink}{Galizien}\oindex{Galizien@\textbf{Galizien}|pw}{}\ledrightnote{\textcolor{pink}{Galizien}}. Ich werde Ihnen
               viel zu erzählen haben. \pend
           
\pstart
           In Verehrung getreu Ihr{\\[\baselineskip]}\spacefill\mbox{Stefan Zweig}\pend
           \leftskip=0em{}\selectlanguage{ngerman}\endnumbering\briefempfaengerindex{, @\textsc{, }!zzz, @\emph{von  }!1915-07-171@{{[}17. 7. 1915?{]}}|)be}\mylabel{L03656h}  \normalsize

\doendnotes{C}
\bigskip
\vfill

\clearpage

\footnotesize

\lohead{\textsc{register}}

% Definiere theindex-Environment komplett neu ohne reledmac
\makeatletter
\renewenvironment{theindex}{%
  \section*{\indexname}%
  \setlength{\parindent}{0pt}%
  \setlength{\parskip}{0pt plus 0.3pt}%
  \let\item\@idxitem
}{%
  \clearpage
}
\makeatother

\IfFileExists{\jobname-pw.ind}{\input{\jobname-pw.ind}}{}

\end{document}

      