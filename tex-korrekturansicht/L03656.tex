%% latex-korrekturansicht-vorspann.tex
%% Vorspann für die Korrekturansicht.
%% Lädt die gemeinsame Datei latex-vorspann.tex mit gesetztem Schalter.

\newif\ifkorrekturansicht
\korrekturansichttrue

\input{../tex-inputs/latex-vorspann}


\renewcommand{\erwaehntePersonen}{Personen: Stefan Zweig}
\renewcommand{\erwaehnteInstitutionen}{Institutionen: Kriegsarchiv}
\renewcommand{\erwaehnteOrte}{Orte: Adama Mickiewicza, Budapest, Galizien, Gorlice, Gródek nad Dunajcem, Krakau, Lviv, Przemyśl, Russland, Sternwartestraße 71, Tarnów, Wien, Währinger Cottage}
\renewcommand{\erwaehnteWerke}{Werke: Aus den Tagen des deutschen Vormarsches in Galizien, Kriegszeitung der 4. Armee}
\section[Stefan Zweig an Arthur Schnitzler, {[}17. 7. 1915?{]}]{Stefan Zweig an Arthur Schnitzler, {[}17. 7. 1915?{]}}
\nopagebreak\mylabel{v}
\rehead{ }\normalsize\beginnumbering\briefempfaengerindex{Schnitzler, Arthur@\textsc{Schnitzler, Arthur}!zzzZweig, Stefan@\emph{von Stefan Zweig}!1914-07-171@{{[}17. 7. 1915?{]}}|(be}
\toendnotes[C]{\smallbreak\pagebreak[2]}\Standort{CUL, Schnitzler, B 118.}
\physDesc{494 Zeichen
\newline{}Handschrift: blaue Tinte, lateinische Kurrent
\newline{}Versand: Stempel: »\nobreak{}K. u. K. MILITÄRZENSUR\nobreak{}«.  }\toendnotes[C]{\smallbreak}\pstart{}Stefan Zweig, Einj. Freiw. Feldwebel\pend{}\pstart{}zugeteilt dem \textcolor{brown}{K. u. K Kriegsarchiv}{}\ledrightnote{\textcolor{brown}{Kriegsarchiv}}\pend{}\pstart{}auf Dienstreise derzeit \textcolor{pink}{Przemysl}{}\ledrightnote{\textcolor{pink}{Przemyśl}}\pend{}
{\bigskip}\pstart{}{\pb}\substVorne{}\textsuperscript{\textcolor{gray}{\textbf{Postkarte}}}{\allowbreak}\substDazwischen{}Feldpost\substHinten{}\pend{}\pstart{}D\textsuperscript{r} Artur Schnitzler\pend{}\pstart{}\textcolor{pink}{Wien – Cottage}{}\ledrightnote{\textcolor{pink}{Währinger Cottage}}\pend{}\pstart{}\textcolor{pink}{\label{K_L03656-1v}\edtext{Sternwartestrasse 72}{\lemma{\textnormal{\emph{Sternwartestrasse 72}}}\Cendnote{\textnormal{\textcolor{blue}{Zweig} wechselt bei der Adressierung
                        seiner Schreiben an \textcolor{blue}{Schnitzler} immer
                        wieder zwischen der falschen Hausnummer »72« und der
                        richtigen »71«.}}}\label{K_L03656-1h}}{}\ledrightnote{\textcolor{pink}{Sternwartestraße 71}}\pend{}
{\bigskip}
\pstart
           \noindent{}\centering{}{\pb}\textcolor{gray}{\textbf{\textcolor{pink}{Przemyśl, ul. Mickiewicza}{}\ledrightnote{\textcolor{pink}{Adama Mickiewicza}} – \textcolor{pink}{Przemyśl, Mickiewiczstrasse}{}\ledrightnote{\textcolor{pink}{Adama Mickiewicza}}.}}\pend
           
\pstart
           \noindent{}{\pb}Lieber verehrter Herr Doktor, ich habe in diesen \label{K_L03656-2v}\edtext{\textcolor{pink}{galizischen}{}\ledrightnote{\textcolor{pink}{Galizien}} Tagen}{\lemma{\textnormal{\emph{galizischen Tagen}}}\Cendnote{\textnormal{\textcolor{blue}{Zweig} war vom 13. 7. 1915 bis
                  zum 26. 7. 1915 im Kriegsgebiet von \textcolor{pink}{Galizien}, kurz nachdem die \textcolor{pink}{russische} Armee zurückgedrängt worden war. \textcolor{blue}{Zweigs} Reise begann in \textcolor{pink}{Krakau} und endete in \textcolor{pink}{Budapest}.
                        (\emph{\textcolor{green}{Tagebuch aus dem Kriegsjahr 1915.
                        Zweiter Band}}, SZ-AAP/L3. SZ-AAP/L3) Ein Tag vor Reisebeginn
                  nannte er in einem Brief an \textcolor{blue}{Franz Karl Ginzkey} die
                  Route: »\textcolor{pink}{Tarnow}, \textcolor{pink}{Przemysl}, \textcolor{pink}{Lemberg}, \textcolor{pink}{Stryi}, \textcolor{pink}{Drohobycz}, \textcolor{pink}{Ungvar}«.
                  (\textcolor{blue}{Stefan Zweig}: \emph{Briefe. Bd. II: 1914–1919}. Herausgegeben von Knut Beck, Jeffrey B. Berlin und Natascha
                     Weschenbach-Feggeler. Frankfurt am Main: \emph{S. Fischer}{ }1998, S. 76.)
                  Von den Herausgebern auf den Folgetag umdatiert ist ein von \textcolor{blue}{Zweig}
                  mit 16. Juli 1915 datiertes Schreiben aus \textcolor{pink}{Przemyśl}
                  an \textcolor{blue}{Raoul Auernheimer}, das in der Eröffnung 
                  ebenfalls die »\textcolor{pink}{galizische[]} Tagen« erwähnt: »Lieber Freund, ich habe jetzt heiße und herrliche Tage hier in \textcolor{pink}{Galizien}«
                  erwähnt. (Ebd., S. 77.) Die dichte Reiseroute spricht dafür, dass sich \textcolor{blue}{Zweig} nur
                  kurz in \textcolor{pink}{Przemyśl} aufhielt und diese Karte am selben Tag
                  wie der Brief an \textcolor{blue}{Auernheimer} verfasst wurde.}}}\label{K_L03656-2h}
               Unendliches gesehn: den ungeheuren Gang dieser gewaltigen Centripetalmaschine, die
               alle Kraft eines Reiches mit Wucht nach aussen schleudert und dann eine tragische
               aber doch schöne Welt: \textcolor{pink}{Galizien}{}\ledrightnote{\textcolor{pink}{Galizien}}. Ich werde Ihnen
               viel zu erzählen haben. \pend
           
\pstart
           In Verehrung getreu Ihr{\\[\baselineskip]}\spacefill\mbox{Stefan Zweig}\pend
           \leftskip=0em{}\endnumbering\briefempfaengerindex{Schnitzler, Arthur@\textsc{Schnitzler, Arthur}!zzzZweig, Stefan@\emph{von Stefan Zweig}!1914-07-171@{{[}17. 7. 1915?{]}}|)be}\mylabel{h}
\begin{anhang}
\end{anhang}\normalsize

\doendnotes{C}
\bigskip
\vfill

\clearpage

\footnotesize

\lohead{\textsc{register}}

% Definiere theindex-Environment komplett neu ohne reledmac
\makeatletter
\renewenvironment{theindex}{%
  \section*{\indexname}%
  \setlength{\parindent}{0pt}%
  \setlength{\parskip}{0pt plus 0.3pt}%
  \let\item\@idxitem
}{%
  \clearpage
}
\makeatother

\IfFileExists{\jobname-pw.ind}{\input{\jobname-pw.ind}}{}

\end{document}

      