%% latex-korrekturansicht-vorspann.tex
%% Vorspann für die Korrekturansicht.
%% Lädt die gemeinsame Datei latex-vorspann.tex mit gesetztem Schalter.

\newif\ifkorrekturansicht
\korrekturansichttrue

\input{../tex-inputs/latex-vorspann}


\section[Arthur Schnitzler an Stefan Zweig, 29. 11. 1922]{L03756 Arthur Schnitzler an Stefan Zweig, 29. 11. 1922}
\nopagebreak\mylabel{L03756v}
\rehead{ }\normalsize\beginnumbering\briefempfaengerindex{, @\textsc{, }!zzz, @\emph{von  }!1922-11-291@{29. 11. 1922}|(be}
\toendnotes[C]{\smallbreak\pagebreak[2]}\Standort{Jerusalem, National Library of Israel, ARC. Ms. Var. 305 1 58 Stefan Zweig Collection.}
\physDesc{Postkarte, 501 Zeichen
\newline{}Handschrift: Bleistift, lateinische Kurrent
\newline{}Versand: Stempel: »\nobreak{}\oindex{XVIII., Währing@\textbf{XVIII., Währing}, \emph{Verwaltungsgebiet}|pwk}\textcolor{gray}{18}/\textcolor{gray}{1} Wien 110, 29. XI. 22, \textcolor{gray}{3}\nobreak{}«.  
\newline{}Zweig: mit Bleistift beschriftet: »Schnitzler« }\toendnotes[C]{\smallbreak}\pstart{}{\pb}Herrn Dr.\pend{}\pstart{}Stefan Zweig.\pend{}\pstart{}\textcolor{pink}{Salzburg}\oindex{Salzburg@\textbf{Salzburg}, \emph{Verwaltungsgebiet}|pw}{}\ledrightnote{\textcolor{pink}{Salzburg}}\pend{}\pstart{}\textcolor{pink}{Kapuzinerberg 5}\oindex{Paschinger Schlössl@\textbf{Paschinger Schlössl}, \emph{Wohngebäude}|pw}{}\ledrightnote{\textcolor{pink}{Paschinger Schlössl}}\pend{}{\bigskip}\vspace{1em}
\pstart{}{\pb}lieber und verehrter Herr Doctor,\pend\vspace{0.5em}
\pstart
           vielen Dank für die liebenswürdige Übersendg Ihrer \textcolor{green}{Novellen}\pwindex{Zweig, Stefan 28.\,11.\,1881 Wien – 23.\,2.\,1942 Petrópolis@\textsc{Zweig, Stefan} (28.\,11.\,1881 Wien – 23.\,2.\,1942 Petrópolis), \emph{Schriftsteller}!Amok@\strich\emph{Amok}|pwv}{}\ledrightnote{{$\rightarrow$}\emph{\textcolor{green}{Amok}}}, mit denen eine wertvolle Beka{\geminationn}tschaft zu
               erneuern ich mich herzlich freue. Hab ich Ihnen das schon gesagt, wie besonders schön
               ich Ihren \label{K_L03756-1v}\edtext{\textcolor{green}{Geburtstagsartikel}\pwindex{Arthur Schnitzler. Zu seinem sechzigsten Geburtstag (15. Mai 1922)@\emph{Arthur Schnitzler. Zu seinem sechzigsten Geburtstag (15. Mai 1922)}|pwv}{}\ledrightnote{{$\rightarrow$}\emph{\textcolor{green}{Arthur Schnitzler. Zu seinem sechzigsten Geburtstag (15. Mai 1922)}}}}{\lemma{\textnormal{\emph{Geburtstagsartikel}}}\Cendnote{\textnormal{\textcolor{blue}{Raoul Auernheimer}\pwindex{Auernheimer, Raoul 15.\,4.\,1876 Wien – 6.\,1.\,1948 Oakland@\textsc{Auernheimer, Raoul} (15.\,4.\,1876 Wien – 6.\,1.\,1948 Oakland), \emph{Schriftsteller, Journalist, Kritiker}|pwk}, \textcolor{blue}{Hermann Bahr}\pwindex{Bahr, Hermann 19.\,7.\,1863 Linz – 15.\,1.\,1934 München@\textsc{Bahr, Hermann} (19.\,7.\,1863 Linz – 15.\,1.\,1934 München), \emph{Schriftsteller, Kritiker}|pwk}, \textcolor{blue}{Franz
                        Blei}\pwindex{Blei, Franz 18.\,1.\,1871 Wien – 10.\,7.\,1942 Westbury@\textsc{Blei, Franz} (18.\,1.\,1871 Wien – 10.\,7.\,1942 Westbury), \emph{Schriftsteller}|pwk}, \textcolor{blue}{Samuel Fischer}\pwindex{Fischer, Samuel 24.\,12.\,1859 Liptovský Mikuláš – 15.\,10.\,1934 Berlin@\textsc{Fischer, Samuel} (24.\,12.\,1859 Liptovský Mikuláš – 15.\,10.\,1934 Berlin), \emph{Verleger}|pwk}, \textcolor{blue}{Otto Flake}\pwindex{Flake, Otto 29.\,10.\,1880 Metz – 10.\,11.\,1963 Baden-Baden@\textsc{Flake, Otto} (29.\,10.\,1880 Metz – 10.\,11.\,1963 Baden-Baden), \emph{Schriftsteller}|pwk}, \textcolor{blue}{Egon Friedell}\pwindex{Friedell, Egon 21.\,1.\,1878 Wien – 16.\,3.\,1938 ebd.@\textsc{Friedell, Egon} (21.\,1.\,1878 Wien – 16.\,3.\,1938 ebd.), \emph{Schriftsteller, Journalist, Kulturphilosoph}|pwk}, \textcolor{blue}{Gerhart
                        Hauptmann}\pwindex{Hauptmann, Gerhart 15.\,11.\,1862 Szczawno-Zdrój – 6.\,6.\,1946 Jagniątków@\textsc{Hauptmann, Gerhart} (15.\,11.\,1862 Szczawno-Zdrój – 6.\,6.\,1946 Jagniątków), \emph{Schriftsteller}|pwk}, \textcolor{blue}{Hugo von
                     Hofmannsthal}\pwindex{Hofmannsthal, Hugo von 1.\,2.\,1874 Wien – 15.\,7.\,1929 Rodaun@\textsc{Hofmannsthal, Hugo von} (1.\,2.\,1874 Wien – 15.\,7.\,1929 Rodaun), \emph{Schriftsteller}|pwk}, \textcolor{blue}{Felix Hollaender}\pwindex{Hollaender, Felix 1.\,11.\,1867 Głubczyce – 29.\,5.\,1931 Berlin@\textsc{Hollaender, Felix} (1.\,11.\,1867 Głubczyce – 29.\,5.\,1931 Berlin), \emph{Schriftsteller, Theaterleiter, Regisseur}|pwk}, \textcolor{blue}{Alfred Kerr}\pwindex{Kerr, Alfred 25.\,12.\,1867 Breslau – 12.\,10.\,1948 Hamburg@\textsc{Kerr, Alfred} (25.\,12.\,1867 Breslau – 12.\,10.\,1948 Hamburg), \emph{Schriftsteller, Kritiker}|pwk}, \textcolor{blue}{Heinrich Mann}\pwindex{Mann, Heinrich 27.\,3.\,1871 Lübeck – 11.\,3.\,1950 Santa Monica@\textsc{Mann, Heinrich} (27.\,3.\,1871 Lübeck – 11.\,3.\,1950 Santa Monica), \emph{Schriftsteller}|pwk}, \textcolor{blue}{Thomas
                        Mann}\pwindex{Mann, Thomas 6.\,6.\,1875 Lübeck – 12.\,8.\,1955 Zürich@\textsc{Mann, Thomas} (6.\,6.\,1875 Lübeck – 12.\,8.\,1955 Zürich), \emph{Schriftsteller}|pwk}, \textcolor{blue}{Jakob Wassermann}\pwindex{Wassermann, Jakob 10.\,3.\,1873 Fürth – 1.\,1.\,1934 Altaussee@\textsc{Wassermann, Jakob} (10.\,3.\,1873 Fürth – 1.\,1.\,1934 Altaussee), \emph{Schriftsteller}|pwk}, \textcolor{blue}{Franz Werfel}\pwindex{Werfel, Franz 10.\,9.\,1890 Prag – 26.\,8.\,1945 Beverly Hills@\textsc{Werfel, Franz} (10.\,9.\,1890 Prag – 26.\,8.\,1945 Beverly Hills), \emph{Schriftsteller}|pwk}, \textcolor{blue}{Stefan Zweig}\pwindex{Zweig, Stefan 28.\,11.\,1881 Wien – 23.\,2.\,1942 Petrópolis@\textsc{Zweig, Stefan} (28.\,11.\,1881 Wien – 23.\,2.\,1942 Petrópolis), \emph{Schriftsteller}|pwk}: \emph{\textcolor{green}{Arthur
                        Schnitzler. Zu seinem sechzigsten Geburtstag (15. Mai 1922)}\pwindex{Arthur Schnitzler. Zu seinem sechzigsten Geburtstag (15. Mai 1922)@\emph{Arthur Schnitzler. Zu seinem sechzigsten Geburtstag (15. Mai 1922)}|pwk}}. In: \emph{\textcolor{green}{Die neue Rundschau}\pwindex{neue Rundschau@\emph{Die neue Rundschau}|pwk}}, Jg. 33,
                     Nr. 5, 1. 5. 1922, S. 498–513. }}}\label{K_L03756-1}
               in der \textcolor{green}{n. R.}\pwindex{neue Rundschau@\emph{Die neue Rundschau}|pw}{}\ledrightnote{\textcolor{green}{Die neue Rundschau}} gefunden habe? Jetzt les ich Ihr
                  ausge{\pb}zeichnetes \textcolor{green}{\textcolor{blue}{Desbord-Valmore}\pwindex{Desbordes-Valmore, Marceline 20.\,6.\,1786 Douai – 23.\,7.\,1859 Paris@\textsc{Desbordes-Valmore, Marceline} (20.\,6.\,1786 Douai – 23.\,7.\,1859 Paris), \emph{Schauspielerin, Sängerin, Schriftstellerin}|pw}{}\ledrightnote{\textcolor{blue}{Marceline Desbordes-Valmore}}
                  Buch}\pwindex{Zweig, Stefan 28.\,11.\,1881 Wien – 23.\,2.\,1942 Petrópolis@\textsc{Zweig, Stefan} (28.\,11.\,1881 Wien – 23.\,2.\,1942 Petrópolis), \emph{Schriftsteller}!Marceline Desbordes-Valmore. Das Lebensbild einer Dichterin@\strich\emph{Marceline Desbordes-Valmore. Das Lebensbild einer Dichterin}|pwv}{}\ledrightnote{{$\rightarrow$}\emph{\textcolor{green}{Marceline Desbordes-Valmore. Das Lebensbild einer Dichterin}}}, das mir ein dafür schwärmender \textcolor{blue}{Vetter}\pwindex{?? [Vetter von Schnitzler, der ihm Buch schenkt] @\textsc{?? [Vetter von Schnitzler, der ihm Buch schenkt]}|pwv}{}\ledrightnote{{$\rightarrow$}\emph{\textcolor{blue}{?? [Vetter von Schnitzler, der ihm Buch schenkt]}}} zum Geschenk gemacht hat. Auf Wiedersehen!\pend
           \pstart Mit vielen Grüßen Ihr sehr ergebener \spacefill\mbox{Arth Schnitzler}\pend{}\selectlanguage{ngerman}\endnumbering\briefempfaengerindex{, @\textsc{, }!zzz, @\emph{von  }!1922-11-291@{29. 11. 1922}|)be}\mylabel{L03756h}
\begin{anhang}
\end{anhang}\normalsize

\doendnotes{C}
\bigskip
\vfill

\clearpage

\footnotesize

\lohead{\textsc{register}}

% Definiere theindex-Environment komplett neu ohne reledmac
\makeatletter
\renewenvironment{theindex}{%
  \section*{\indexname}%
  \setlength{\parindent}{0pt}%
  \setlength{\parskip}{0pt plus 0.3pt}%
  \let\item\@idxitem
}{%
  \clearpage
}
\makeatother

\IfFileExists{\jobname-pw.ind}{\input{\jobname-pw.ind}}{}

\end{document}

      