%% latex-korrekturansicht-vorspann.tex
%% Vorspann für die Korrekturansicht.
%% Lädt die gemeinsame Datei latex-vorspann.tex mit gesetztem Schalter.

\newif\ifkorrekturansicht
\korrekturansichttrue

\input{../tex-inputs/latex-vorspann}


               \section[Hugo Hofmannsthal an Arthur Schnitzler, 15. 5. 1923]{ Hugo Hofmannsthal an Arthur Schnitzler, 15. 5. 1923}\nopagebreak\mylabel{v}\rehead{ }\normalsize\beginnumbering\briefempfaengerindex{Schnitzler, Arthur@\textsc{Schnitzler, Arthur}!zzzHofmannsthal, Hugo von@\emph{von Hugo von Hofmannsthal}!1923-05-151@{15. 5. 1923}|(be} \toendnotes[C]{\smallbreak\pagebreak[2]} \Standort{CUL, Schnitzler, B 43.}
\physDesc{Bildpostkarte
\newline{}Handschrift: Bleistift, lateinische Kurrent\newline{}Versand: Stempel: »\nobreak{}\oindex{Magglingen@\textbf{Magglingen}, \emph{https://www.geonames.org/ontologyP.PPL}|pwk}Macolin (Magglingen), 16. V. 23\nobreak{}«.  \newline{}Ordnung: 1) mit Bleistift von \textcolor{blue}{Frieda
                                    Pollak} (?) mit dem Buchstaben »A«
                                 (Abgeschrieben/Abschrift) gekennzeichnet 2) mit Bleistift von unbekannter Hand nummeriert: »\strikeout{373}«3) mit Bleistift von unbekannter Hand nummeriert:
                                    »377«}\buchAbdrucke{\weitereDrucke{Hugo von Hofmannsthal, Arthur Schnitzler: \emph{Briefwechsel}. Hg. Therese Nickl und Heinrich Schnitzler. Frankfurt am Main: \emph{S. Fischer} 1964, S. 298.} }\toendnotes[C]{\smallbreak}\pstart{}{\pb}Herrn D\textsuperscript{r} Arthur Schnitzler\pend{}\pstart{}\textcolor{pink}{Wien}{}\ledrightnote{\textcolor{pink}{Wien}}\pend{}\pstart{}\textcolor{pink}{XVIII Sternwartestrasse 71}{}\ledrightnote{\textcolor{pink}{Sternwartestraße}}\pend{}{\bigskip}\pstart
           \noindent{}\centering{}{\pb}\textcolor{gray}{\textbf{Nr. 6508 \textcolor{pink}{Biel – Bienne}{}\ledrightnote{\textcolor{pink}{Biel}}}}\pend
           \pstart
           \raggedleft{}{\pb}\textcolor{pink}{Biel}{}\ledrightnote{\textcolor{pink}{Biel}} den 15\textsuperscript{ten} Mai\pend
           \pstart{}mein lieber Arthur \pend\pstart
           hier sind wir nämlich vor 25 Jahren (am \label{K_L02399_1v}\edtext{20\textsuperscript{ten} oder 21\textsuperscript{ten} August 1898}{\lemma{\textnormal{\emph{20ten … 1898}}}\Cendnote{\textnormal{Es dürfte sich um den
                     13. 8. 1898 gehandelt haben, vgl. A. S.: \emph{Tagebuch}, 13. 8. 1898}}}\label{K_L02399_1h}) miteinander gesessen!\pend
           \pstart
           Das ist seltsam und geisterhaft.\pend
           \pstart
           Ich schicke Ihnen viele freundschaftliche Gedanken! \pend
           \pstart
           Ihr{\\[\baselineskip]}\spacefill\mbox{Hugo}\pend
           \leftskip=0em{}\endnumbering\briefempfaengerindex{Schnitzler, Arthur@\textsc{Schnitzler, Arthur}!zzzHofmannsthal, Hugo von@\emph{von Hugo von Hofmannsthal}!1923-05-151@{15. 5. 1923}|)be}\mylabel{h}  \normalsize

\doendnotes{C}
\bigskip
\vfill

\clearpage

\footnotesize

\lohead{\textsc{register}}

% Definiere theindex-Environment komplett neu ohne reledmac
\makeatletter
\renewenvironment{theindex}{%
  \section*{\indexname}%
  \setlength{\parindent}{0pt}%
  \setlength{\parskip}{0pt plus 0.3pt}%
  \let\item\@idxitem
}{%
  \clearpage
}
\makeatother

\IfFileExists{\jobname-pw.ind}{\input{\jobname-pw.ind}}{}

\end{document}

      