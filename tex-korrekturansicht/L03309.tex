%% latex-korrekturansicht-vorspann.tex
%% Vorspann für die Korrekturansicht.
%% Lädt die gemeinsame Datei latex-vorspann.tex mit gesetztem Schalter.

\newif\ifkorrekturansicht
\korrekturansichttrue

\input{../tex-inputs/latex-vorspann}


\renewcommand{\erwaehntePersonen}{Personen: Oskar Mayer}
\renewcommand{\erwaehnteOrte}{Orte: Bad Ischl, Karlsbad, Salzburg}
\renewcommand{\erwaehnteWerke}{}
\section[ Felix Salten an Arthur Schnitzler, 8. 8. 1900]{Felix Salten an Arthur Schnitzler, 8. 8. 1900}
\nopagebreak\mylabel{v}
\rehead{ }\normalsize\beginnumbering\briefempfaengerindex{Schnitzler, Arthur@\textsc{Schnitzler, Arthur}!zzzSalten, Felix@\emph{von Felix Salten}!1900-08-081@{8. 8. 1900}|(be}
\toendnotes[C]{\smallbreak\pagebreak[2]}\Standort{CUL, Schnitzler, B 89, A 2.}
\physDesc{Brief, 1 Blatt, 1 Seite, 406 Zeichen
\newline{}Handschrift: schwarze Tinte, lateinische Kurrent
\newline{}Ordnung: mit Bleistift von unbekannter Hand nummeriert: »133« }\toendnotes[C]{\smallbreak}
\pstart
           \raggedleft{}{\pb}\textcolor{pink}{Karlsbad}{}\ledrightnote{\textcolor{pink}{Karlsbad}}, 8./VIII. 00.\pend
           
\pstart
           Lieber Arthur, ich bitte, \label{K_L03309-1v}\edtext{eingeschloßenen Brief}{\lemma{\textnormal{\emph{eingeschloßenen Brief}}}\Cendnote{\textnormal{Beilage nicht erhalten; Bezug unklar, siehe evtl. Richard Beer-Hofmann an Arthur Schnitzler, 13. 7. 1900}}}\label{K_L03309-1h} an \textcolor{blue}{Osc. Mayer}{}\ledrightnote{\textcolor{blue}{Oskar Mayer}} gütigst befördern zu
               wollen, dessen Adreße mir leider nicht bekannt ist, und der ja wol noch in \textcolor{pink}{Ischl}{}\ledrightnote{\textcolor{pink}{Bad Ischl}} oder mit Ihnen in \textcolor{pink}{Salzburg}{}\ledrightnote{\textcolor{pink}{Salzburg}} sich befindet. Es handelt sich um eine mir ganz
               unerfindliche Geschichte, die ich gerne so oder so aufgeklärt sähe.\pend
           
\pstart
           Vielen Dank und herzlichste Grüße. Ich bin vermuthlich Sonntag oder Montag in \label{K_L03309-2v}\edtext{\textcolor{pink}{Ischl}{}\ledrightnote{\textcolor{pink}{Bad Ischl}}}{\lemma{\textnormal{\emph{Ischl}}}\Cendnote{\textnormal{siehe Felix Salten an Arthur Schnitzler, 7. 8. 1900}}}\label{K_L03309-2h}.\pend
           
\pstart
           Ihr {\\[\baselineskip]}\spacefill\mbox{Salten}\pend
           \leftskip=0em{}\endnumbering\briefempfaengerindex{Schnitzler, Arthur@\textsc{Schnitzler, Arthur}!zzzSalten, Felix@\emph{von Felix Salten}!1900-08-081@{8. 8. 1900}|)be}\mylabel{h}  \normalsize

\doendnotes{C}
\bigskip
\vfill

\clearpage

\footnotesize

\lohead{\textsc{register}}

% Definiere theindex-Environment komplett neu ohne reledmac
\makeatletter
\renewenvironment{theindex}{%
  \section*{\indexname}%
  \setlength{\parindent}{0pt}%
  \setlength{\parskip}{0pt plus 0.3pt}%
  \let\item\@idxitem
}{%
  \clearpage
}
\makeatother

\IfFileExists{\jobname-pw.ind}{\input{\jobname-pw.ind}}{}

\end{document}

      