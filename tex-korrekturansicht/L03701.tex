%% latex-korrekturansicht-vorspann.tex
%% Vorspann für die Korrekturansicht.
%% Lädt die gemeinsame Datei latex-vorspann.tex mit gesetztem Schalter.

\newif\ifkorrekturansicht
\korrekturansichttrue

\input{../tex-inputs/latex-vorspann}


\section[Elsa Plessner an Arthur Schnitzler, 15. 9. 1896]{L03701 Elsa Plessner an Arthur Schnitzler, 15. 9. 1896}
\nopagebreak\mylabel{L03701v}
\rehead{ }\normalsize\beginnumbering\briefempfaengerindex{Schnitzler, Arthur@\textsc{Schnitzler, Arthur}!zzzPlessner, Elsa@\emph{von Elsa Plessner}!1896-09-152@{15. 9. 1896}|(be}
\toendnotes[C]{\smallbreak\pagebreak[2]}
\correspDesc{Versand  durch Elsa Plessner am 15. 9. 1896 in Wien
\newline{}Erhalt  durch Arthur Schnitzler im Zeitraum [16. 9. 1896
                  – 20. 9. 1896?] in Wien}\toendnotes[C]{\smallbreak}
\Standort{DLA, A:Schnitzler, HS.1985.1.419.}
\physDesc{Brief, 1 Blatt, 3 Seiten, 770 Zeichen
\newline{}Handschrift: schwarze Tinte, lateinische Kurrent
\newline{}Schnitzler: mit rotem Buntstift eine Unterstreichung }\toendnotes[C]{\smallbreak}
\pstart
           {\pb}\textcolor{pink}{I. Bäckerstraße N\textsuperscript{o}
                     1}\oindex{Wien@\textbf{Wien}!I., Innere Stadt@\textbf{I., Innere Stadt}!Bäckerstraße 1@\textbf{Bäckerstraße 1}, \emph{Wohngebäude}|pw}{}\ledrightnote{\textcolor{pink}{Bäckerstraße 1}}, den 14. 9. 96. \pend
           
\pstart\center{}Hochverehrter Herr!\pend\vspace{0.5em}
\pstart
           Der Plagegeist, der Sie im vergangenen Winter mit \label{K_L03701-1v}\edtext{Manuscripten}{\lemma{\textnormal{\emph{Manuscripten}}}\Cendnote{\textnormal{Im Frühjahr des Jahres 1896 hatte \textcolor{blue}{Elsa Plessner}\pwindex{Plessner, Elsa 22.\,8.\,1875 Wien – 7.\,5.\,1932 Alicante@\textsc{Plessner, Elsa} (22.\,8.\,1875 Wien – 7.\,5.\,1932 Alicante), \emph{Schriftstellerin}|pwk}{ }\textcolor{blue}{Schnitzler} ihre Schauspiel \emph{\textcolor{green}{Heimkehr}\pwindex{Plessner, Elsa 22.\,8.\,1875 Wien – 7.\,5.\,1932 Alicante@\textsc{Plessner, Elsa} (22.\,8.\,1875 Wien – 7.\,5.\,1932 Alicante), \emph{Schriftstellerin}!Heimweh [dreiaktige Tragikomödie]@\strich\emph{Heimweh [dreiaktige Tragikomödie]}|pwk}} (14. 3. 1896), neunzehn Gedichte unter dem Titel \emph{\textcolor{green}{Pierettes
                     Tagebuch}\pwindex{Plessner, Elsa 22.\,8.\,1875 Wien – 7.\,5.\,1932 Alicante@\textsc{Plessner, Elsa} (22.\,8.\,1875 Wien – 7.\,5.\,1932 Alicante), \emph{Schriftstellerin}!Pierettes Tagebuch [19 unveröffentlichte Gedichte]@\strich\emph{Pierettes Tagebuch [19 unveröffentlichte Gedichte]}|pwk}}, \emph{\textcolor{green}{Baby}\pwindex{Plessner, Elsa 22.\,8.\,1875 Wien – 7.\,5.\,1932 Alicante@\textsc{Plessner, Elsa} (22.\,8.\,1875 Wien – 7.\,5.\,1932 Alicante), \emph{Schriftstellerin}!Baby@\strich\emph{Baby}|pwk}} und \emph{\textcolor{green}{Der Begräbnißtag}\pwindex{Begräbnißtag@\emph{Der Begräbnißtag}|pwk}} (18. 3. 1896) sowie den Entwurf zur Novelle \emph{\textcolor{green}{Warten}\pwindex{Plessner, Elsa 22.\,8.\,1875 Wien – 7.\,5.\,1932 Alicante@\textsc{Plessner, Elsa} (22.\,8.\,1875 Wien – 7.\,5.\,1932 Alicante), \emph{Schriftstellerin}!Warten. Novelle@\strich\emph{Warten. Novelle}|pwk}} (14. 4. 1896) gesandt.}}}\label{K_L03701-1}
               bombardirt hat und dem Sie in himmlischer Geduld mehrmals schriftlich \label{K_L03701-2v}\edtext{Rede und Antwort}{\lemma{\textnormal{\emph{Rede und Antwort}}}\Cendnote{\textnormal{\textcolor{blue}{Schnitzlers} Briefe an \textcolor{blue}{Elsa Plessner}\pwindex{Plessner, Elsa 22.\,8.\,1875 Wien – 7.\,5.\,1932 Alicante@\textsc{Plessner, Elsa} (22.\,8.\,1875 Wien – 7.\,5.\,1932 Alicante), \emph{Schriftstellerin}|pwk} sind nicht überliefert.}}}\label{K_L03701-2} – will sagen
               Urtheil – standen, erlaubt sich hiemit die höfl. Anfrage, ob und wann Sie ihm in
               einer für ihn außerordentlich wichtigen Angelegenheit eine Audienz bewilligen. Es
               handelt sich um das Ihnen bekannte drei-actige \textcolor{green}{Drama}\pwindex{Plessner, Elsa 22.\,8.\,1875 Wien – 7.\,5.\,1932 Alicante@\textsc{Plessner, Elsa} (22.\,8.\,1875 Wien – 7.\,5.\,1932 Alicante), \emph{Schriftstellerin}!Heimweh [dreiaktige Tragikomödie]@\strich\emph{Heimweh [dreiaktige Tragikomödie]}|pwv}{}\ledrightnote{{$\rightarrow$}\emph{\textcolor{green}{Heimweh [dreiaktige Tragikomödie]}}}. –\pend
           
\pstart
           Wenn sie die große Liebenswürdigkeit haben wollten, mir mitzutheilen, wann Sie die
               noch größere besitzen {\pb}werden, für mich zu sprechen zu
               sein so bringen Sie das Maß Ihrer engelhaften Güte mir gegenüber zum Überfließen.
               –\pend
           
\pstart
           Und harrend der freudigen Botschaft zeichnet mit neuem Dank im Voraus – und
               alter, hochachtungsvoller Verehrung{\\[\baselineskip]}\spacefill\mbox{Elsa Plessner}\pend
           \leftskip=0em{}\selectlanguage{ngerman}\endnumbering\briefempfaengerindex{Schnitzler, Arthur@\textsc{Schnitzler, Arthur}!zzzPlessner, Elsa@\emph{von Elsa Plessner}!1896-09-152@{15. 9. 1896}|)be}\mylabel{L03701h}  \normalsize

\doendnotes{C}
\bigskip
\vfill

\clearpage

\footnotesize

\lohead{\textsc{register}}

% Definiere theindex-Environment komplett neu ohne reledmac
\makeatletter
\renewenvironment{theindex}{%
  \section*{\indexname}%
  \setlength{\parindent}{0pt}%
  \setlength{\parskip}{0pt plus 0.3pt}%
  \let\item\@idxitem
}{%
  \clearpage
}
\makeatother

\IfFileExists{\jobname-pw.ind}{\input{\jobname-pw.ind}}{}

\end{document}

      