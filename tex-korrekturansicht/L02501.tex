%% latex-korrekturansicht-vorspann.tex
%% Vorspann für die Korrekturansicht.
%% Lädt die gemeinsame Datei latex-vorspann.tex mit gesetztem Schalter.

\newif\ifkorrekturansicht
\korrekturansichttrue

\input{../tex-inputs/latex-vorspann}


               \section[Thomas Mann an Arthur Schnitzler, 28. 5. 1928]{ Thomas Mann an Arthur Schnitzler, 28. 5. 1928}\nopagebreak\mylabel{v}\rehead{ }\normalsize\beginnumbering\briefempfaengerindex{Schnitzler, Arthur@\textsc{Schnitzler, Arthur}!zzzMann, Thomas@\emph{von Thomas Mann}!1928-05-281@{28. 5. 1928}|(be} \toendnotes[C]{\smallbreak\pagebreak[2]} \Standort{CUL, Schnitzler, B 67.}
\physDesc{Briefkarte
\newline{}Handschrift: schwarze Tinte, deutsche Kurrent
\newline{}Schnitzler: mit rotem Buntstift beschrieben: »\textsc{\textcolor{green}{Therese}}« }\buchAbdrucke{\weitereDrucke{Hertha Krotkoff: \emph{Arthur Schnitzler – Thomas Mann: Briefe.} In: \emph{Modern Austrian Literature}, Jg. 7 (1974) Nr. 1/2, S. 25.} }\toendnotes[C]{\smallbreak}\pstart
           \noindent{}{\pb}\textcolor{gray}{\textbf{DR. THOMAS MANN}}\hfill \textcolor{gray}{\textbf{\textcolor{pink}{MÜNCHEN}{}\ledrightnote{\textcolor{pink}{München}} den}}{ }28. V. 28.\pend
           \pstart
           \raggedleft{}\textcolor{gray}{\textbf{\textcolor{pink}{POSCHINGERSTR. 1}{}\ledrightnote{\textcolor{pink}{Poschingerstraße}}}}\pend
           \pstart{}Lieber, verehrter Arthur Schnitzler,\pend\pstart
           ich muß Ihnen ſagen, wie ſehr ich Ihre »\textcolor{green}{Thereſe}{}\ledrightnote{\textcolor{green}{Therese. Chronik eines Frauenlebens}}« liebe, dieſen Roman, der, wie alle Guten und Wichtigen heute,
                    keiner mehr iſt, und in den ich in langſamer, inniger Lektüre in mich
                    aufgenommen habe. Was ich ſo bewundere, iſt die Conception des Buches, das
                    Große, Einfache, Wahre, durchaus Lebensgemäße, die dauernde ſtille und tiefe
                    Erſchütterung durch das {\pb}Menſchliche,
                    ohne Aufwand, ohne Spannung, Konflikte, »Knotenſchürzung«, »Erfindung«, – lauter
                    Dinge, die als läppiſch zu empfinden dies \textcolor{green}{Buch}{}\ledrightnote{→\textcolor{green}{Therese. Chronik eines Frauenlebens}} wie kein anderes zu lehren geeignet iſt. Und Sie
                    haben dem Menſchenleben, wie es iſt, wie es meiſtens iſt, eine Sprache zu finden
                    gewußt, ſchlicht und rein und wahr wiederum, wahr, treffend und ſcheinbar
                    unbewegt, aber von ſo zwingender Melodik dabei, daß man nach den erſten paar
                    Sätzen weiß: Das leſe ich mit Luſt zu Ende. Haben Sie vielen Dank und
                    aufrichtigen Glückwunſch!\pend
           \pstart
           Ihr ergebener{\\[\baselineskip]}\spacefill\mbox{Thomas Mann.}\pend
           \leftskip=0em{}\endnumbering\briefempfaengerindex{Schnitzler, Arthur@\textsc{Schnitzler, Arthur}!zzzMann, Thomas@\emph{von Thomas Mann}!1928-05-281@{28. 5. 1928}|)be}\mylabel{h}  \normalsize

\doendnotes{C}
\bigskip
\vfill

\clearpage

\footnotesize

\lohead{\textsc{register}}

% Definiere theindex-Environment komplett neu ohne reledmac
\makeatletter
\renewenvironment{theindex}{%
  \section*{\indexname}%
  \setlength{\parindent}{0pt}%
  \setlength{\parskip}{0pt plus 0.3pt}%
  \let\item\@idxitem
}{%
  \clearpage
}
\makeatother

\IfFileExists{\jobname-pw.ind}{\input{\jobname-pw.ind}}{}

\end{document}

      