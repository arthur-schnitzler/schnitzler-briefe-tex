%% latex-korrekturansicht-vorspann.tex
%% Vorspann für die Korrekturansicht.
%% Lädt die gemeinsame Datei latex-vorspann.tex mit gesetztem Schalter.

\newif\ifkorrekturansicht
\korrekturansichttrue

\input{../tex-inputs/latex-vorspann}


\renewcommand{\erwaehntePersonen}{Personen: Stefan Zweig}
\renewcommand{\erwaehnteOrte}{Orte: Kochgasse 8, Wien}
\renewcommand{\erwaehnteWerke}{Werke: Zwischenspiel. Komödie in drei Akten}
\section[Stefan Zweig an Arthur Schnitzler, 15. 1. 1908]{Stefan Zweig an Arthur Schnitzler, 15. 1. 1908}
\nopagebreak\mylabel{v}
\rehead{ }\normalsize\beginnumbering\briefempfaengerindex{Schnitzler, Arthur@\textsc{Schnitzler, Arthur}!zzzZweig, Stefan@\emph{von Stefan Zweig}!1908-01-151@{15. 1. 1908}|(be}
\toendnotes[C]{\smallbreak\pagebreak[2]}\Standort{CUL, Schnitzler, B 118.}
\physDesc{Brief, 1 Blatt, 2 Seiten, 708 Zeichen
\newline{}Handschrift: schwarze Tinte, lateinische Kurrent
\newline{}Schnitzler: 1) mit rotem Buntstift eine Markierung  2) mit Bleistift »\textsc{Zweig}«}\toendnotes[C]{\smallbreak}
\pstart
           {\pb}\textcolor{pink}{Wien VIII Kochgasse 8}{}\ledrightnote{\textcolor{pink}{Kochgasse 8}}\pend
           
\pstart
           \label{K_L03621-1v}\edtext{15. Januar 1907}{\lemma{\textnormal{\emph{15. Januar 1907}}}\Cendnote{\textnormal{Mit der Jahreszahl 1907 unterlief \textcolor{blue}{Zweig} ein Schreibfehler: Der Brief stammt
                     vom 15. Januar 1908, wie aus der in ihm ausgesprochenen Gratulation zum
                     Grillparzer-Preis hervorgeht, der \textcolor{blue}{Schnitzler} an diesem Tag verliehen wurde.}}}\label{K_L03621-1h}. \pend
           
\pstart{}Sehr verehrter Herr Doktor,\pend
\pstart
            gestatten Sie mir als persönlich Unbekanntem Ihnen \label{K_L03621-2v}\edtext{heute}{\lemma{\textnormal{\emph{heute}}}\Cendnote{\textnormal{Am 15. 1. 1908erhielt \textcolor{blue}{Schnitzler} den Grillparzer-Preis für seine
                  Kommödie \emph{\textcolor{green}{Zwischenspiel}}.}}}\label{K_L03621-2h}heute meine
               aufrichtigen Glückwünsche zu übermitteln. Ich glaube, für uns jüngere Leute, die wir
               in der Bewunderung Ihres Werkes gewissermassen aufgewachsen sind, kann es keine
               grössere Freude geben, als zu sehen, wie Ihnen nun auch aus den älteren kälteren
               Kreisen endlich die grosse Zustimmung wird, die wir so lange schon als ein
               Selbstverständliches ersehnen. Und so einen Tag wollte ich nicht vorrübergehen zu
               lassen, ohne Ihnen zu sagen, dass {\pb}es
               für uns ein Tag der freudigsten Genugtuung gewesen ist, unsere Liebe bestätigt zu
               wissen.\pend
           
\pstart
           In Verehrung getreu{\\[\baselineskip]} Ihr sehr ergebener \spacefill\mbox{Stefan Zweig}\pend
           \leftskip=0em{}\endnumbering\briefempfaengerindex{Schnitzler, Arthur@\textsc{Schnitzler, Arthur}!zzzZweig, Stefan@\emph{von Stefan Zweig}!1908-01-151@{15. 1. 1908}|)be}\mylabel{h}
\begin{anhang}
\end{anhang}\normalsize

\doendnotes{C}
\bigskip
\vfill

\clearpage

\footnotesize

\lohead{\textsc{register}}

% Definiere theindex-Environment komplett neu ohne reledmac
\makeatletter
\renewenvironment{theindex}{%
  \section*{\indexname}%
  \setlength{\parindent}{0pt}%
  \setlength{\parskip}{0pt plus 0.3pt}%
  \let\item\@idxitem
}{%
  \clearpage
}
\makeatother

\IfFileExists{\jobname-pw.ind}{\input{\jobname-pw.ind}}{}

\end{document}

      