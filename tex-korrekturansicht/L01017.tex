%% latex-korrekturansicht-vorspann.tex
%% Vorspann für die Korrekturansicht.
%% Lädt die gemeinsame Datei latex-vorspann.tex mit gesetztem Schalter.

\newif\ifkorrekturansicht
\korrekturansichttrue

\input{../tex-inputs/latex-vorspann}


               \section[Arthur Schnitzler an Richard Beer-Hofmann, 2. 3. 1900]{ Arthur Schnitzler an Richard Beer-Hofmann, 2. 3. 1900}\nopagebreak\mylabel{v}\rehead{ }\normalsize\beginnumbering\briefempfaengerindex{Beer-Hofmann, Richard@\textsc{Beer-Hofmann, Richard}!zzzSchnitzler, Arthur@\emph{von Arthur Schnitzler}!1900-03-021@{2. 3. 1900}|(be} \toendnotes[C]{\smallbreak\pagebreak[2]} \Standort{YCGL, MSS 31.}
\physDesc{Brief, 3 Blätter, 10 Seiten
\newline{}Handschrift: Bleistift, deutsche Kurrent}\buchAbdrucke{\weitereDrucke{1) Arthur Schnitzler: \emph{Briefe 1875–1912}. Hg. Therese Nickl und Heinrich Schnitzler. Frankfurt am Main: \emph{S. Fischer} 1981, S. 380–382.} \weitereDrucke{2) Arthur Schnitzler, Richard Beer-Hofmann: \emph{Briefwechsel 1891–1931}. Hg. Konstanze Fliedl. Wien, Zürich: \emph{Europaverlag} 1992, S. 144–145.} }\toendnotes[C]{\smallbreak}\pstart
           \raggedleft{}{\pb}\textcolor{pink}{\textsc{Edlacher Hof}}{}\ledrightnote{\textcolor{pink}{Hotel Edlacherhof}},\pend
           \pstart
           \raggedleft{}2. März 1900.\pend
           \pstart
           mein lieber Richard, vorgeſtern Abend bin ich hier angeko{\geminationm}en, ich wollte dem Frühling entgegenfahren – und ſeit
               geſtern ſchneit und friert es. I{\geminationm}erhin iſt es in den
               Mittagſtunden ſchön. Heut ſowohl als geſtern bin ich nahezu 6 Stunden ſpazieren
               gegangen. Weniger lang {\pb}ſchrieb ich an der \textcolor{green}{Novelle}{}\ledrightnote{→\textcolor{green}{Frau Bertha Garlan. Roman}}, für die ich keinen Namen
               habe.\pend
           \pstart
           \textcolor{green}{Ihre}{}\ledrightnote{→\textcolor{green}{Der Tod Georgs}} hab’ ich in 2 Etappen
               geleſen, die erſten 2 Capitel in der Eiſenbahn, die letzten 2 geſtern Abend auf
               meinem Zimmer (3. außer 4. im Bett) Also glauben Sie mir: es iſt ein
               wundervolles \textcolor{green}{Buch}{}\ledrightnote{→\textcolor{green}{Der Tod Georgs}}. Man hat
               allerdings das Gefühl, als wenn die aneinandergereihten Edelſteine nicht auf einer
               Schnur, {\pb}ſondern auf einem Zwirnsfaden – oder gar nur
               in der Luft aneinandergereiht wären – aber man muſs nicht alles als Kette um den Hals
               tragen können. Im vierten \textcolor{green}{Kapitel}{}\ledrightnote{→\textcolor{green}{Der Tod Georgs}}{ }ſteckt übrigens irgend wo ein frecher Schwindel –
               das dürfte Ihnen nicht unbekannt ſein. Sie ſetzen ſich ſozuſagen plötzlich an eine
               andre Orgel, die auch herrlich klingt – {\pb}aber das
               beweiſt nichts. – Nicht überall ſcheint es mir geglückt, daſs gegenwärtiges und
               erinnertes ſich gegeneinander abhebt, wie es ſoll; daſs man das Bedürfnis hat, das
                  \textcolor{green}{Buch}{}\ledrightnote{→\textcolor{green}{Der Tod Georgs}} wieder zu leſen \strikeout{dagegen} iſt ja ſehr ſchön; aber dſs man es entschieden
               2–3 Mal lesen \uline{muſs}, iſt vielleicht ein Fehler. Ihre
               Bilderpracht ſchreit nach Jamben {\pb}und nach Drama. Ja
               es verlangt mich geradezu, einige von Ihren Vergleichen in Ihren Stücken
               wiederzufinden und ſie auf der Bühne ſprechen zu hören. – Wunderbar iſt, wie
               ſcheinbar belangloſe Details zu ihrer Zeit ausgenützt und nachträglich voll Belang
               erſcheinen. Das gibt den gewiſſen Schauer. Überhaupt: meiner {\pb}Empfindg nach ſteckt viel mehr Dichteriſches in dem
                  \textcolor{green}{Buch}{}\ledrightnote{→\textcolor{green}{Der Tod Georgs}} als, wie gewiſs vielfach
               behauptet werden wird, Verſtand. Sie wiſſen wie ich das meine. So geſcheidt iſt bald
               einer – aber die Dinge \uline{ſo} sagen – ! Um \textcolor{blue}{Goethe}{}\ledrightnote{\textcolor{blue}{Johann Wolfgang von Goethe}} zu variiren: \textcolor{green}{Alles gescheidte iſt ſchon einmal geſagt worden: man muſs nur
                  verſuchen, es \label{K_L01017_1v}\edtext{– ganz anders zu
                     sagen.}{\lemma{\textnormal{\emph{– ganz anders zu
                     sagen.}}}\Cendnote{\textnormal{Bei \textcolor{blue}{Goethe} endet es: »es noch einmal zu
                        denken.«}}}\label{K_L01017_1h}}{}\ledrightnote{→\textcolor{green}{Wilhelm Meisters Wanderjahre}}{ }{\pb}Und »\label{K_L01017_2v}\edtext{\textsc{ma foi}}{\lemma{\textnormal{\emph{ma foi}}}\Cendnote{\textnormal{französisch: meiner Treu}}}\label{K_L01017_2h}« das haben Sie
               gethan. –\pend
           \pstart
           Während ich dieſes ſchreibe ſitze ich allein im Speiſeſaal, abends
               9 Uhr. Außer mir lebt hier nemlich nur ein (noch) älterer Herr. Montag fahr
               ich wohl wieder nach \textcolor{pink}{Wien}{}\ledrightnote{\textcolor{pink}{Wien}}. Ich ſehn mich nach
               niemandem – niemand ſehnt ſich nach mir. Das iſt nicht ſenti{\pb}mental – ſondern das iſt eben ſo. Heut vor einem Jahr
               war alles noch ſo anders – und doch \label{K_L01017_3v}\edtext{ſchwebte es ſchon über uns}{\lemma{\textnormal{\emph{ſchwebte … uns}}}\Cendnote{\textnormal{der Tod \textcolor{blue}{Marie Reinhard}s}}}\label{K_L01017_3h}{\dotstwo} Ja ja, es ſchwebt immer{\dots} »\textcolor{green}{Zeit iſt nur ein Wort —}{}\ledrightnote{→\textcolor{green}{Der Schleier der Beatrice. Schauspiel in fünf Akten}}« Könnte
               von Ihnen, von \textcolor{blue}{Hugo}{}\ledrightnote{\textcolor{blue}{Hugo von Hofmannsthal}} und von mir \introOben{}(und etlichen andern)\introOben{} ſein. Zufällig ſagt es \textcolor{green}{\textsc{Beatrice}}{}\ledrightnote{→\textcolor{green}{Der Schleier der Beatrice. Schauspiel in fünf Akten}}. –\pend
           \pstart
           Wie lang denken Sie noch auf Reiſen zu ſein? Ich ſchicke {\pb}dieſen Brief nach \textcolor{pink}{Florenz}{}\ledrightnote{\textcolor{pink}{Florenz}}, wo ich Sie, glücklicher und wenn Sie wünſchen weniger witzig als in
                  \textcolor{pink}{\textsc{Sanremo}}{}\ledrightnote{\textcolor{pink}{Sanremo}} vermuthe. – \textcolor{blue}{Mirjam}{}\ledrightnote{\textcolor{blue}{Mirjam Beer-Hofmann}} hoff ich ſo luſtig als
               ſie war und Ihre \textcolor{blue}{Frau}{}\ledrightnote{→\textcolor{blue}{Paula Beer-Hofmann}} ſo
               erholt, als man es von \textcolor{pink}{italieniſcher}{}\ledrightnote{\textcolor{pink}{Italien}} Luft erwarten
               sollte. –\pend
           \pstart
           Von \textcolor{blue}{Hugo}{}\ledrightnote{\textcolor{blue}{Hugo von Hofmannsthal}} weiſs ich noch immer nichts, und \textcolor{blue}{Gustav}{}\ledrightnote{\textcolor{blue}{Gustav Schwarzkopf}}{ }{\pb}hab ich von Ihnen gegrüßt. Thun Sie das gleiche von
               mir an \textcolor{blue}{\textsc{Mayer}}{}\ledrightnote{\textcolor{blue}{Oskar Mayer}}, we{\geminationn} er ſchon mit Ihnen zuſa{\geminationm}engeſtoßen iſt (– was hoffentlich nicht weh gethan
               hat.)\pend
           \pstart
           Leben Sie wohl!\pend
           \pstart Ihr \spacefill\mbox{Arthur}\pend{}\endnumbering\briefempfaengerindex{Beer-Hofmann, Richard@\textsc{Beer-Hofmann, Richard}!zzzSchnitzler, Arthur@\emph{von Arthur Schnitzler}!1900-03-021@{2. 3. 1900}|)be}\mylabel{h}  \normalsize

\doendnotes{C}
\bigskip
\vfill

\clearpage

\footnotesize

\lohead{\textsc{register}}

% Definiere theindex-Environment komplett neu ohne reledmac
\makeatletter
\renewenvironment{theindex}{%
  \section*{\indexname}%
  \setlength{\parindent}{0pt}%
  \setlength{\parskip}{0pt plus 0.3pt}%
  \let\item\@idxitem
}{%
  \clearpage
}
\makeatother

\IfFileExists{\jobname-pw.ind}{\input{\jobname-pw.ind}}{}

\end{document}

      