%% latex-korrekturansicht-vorspann.tex
%% Vorspann für die Korrekturansicht.
%% Lädt die gemeinsame Datei latex-vorspann.tex mit gesetztem Schalter.

\newif\ifkorrekturansicht
\korrekturansichttrue

\input{../tex-inputs/latex-vorspann}


\section[Arthur Schnitzler an Gustav Schwarzkopf, 7. 12. 1906]{L04039 Arthur Schnitzler an Gustav Schwarzkopf, 7. 12. 1906}
\nopagebreak\mylabel{L04039v}
\rehead{ }\normalsize\beginnumbering\briefempfaengerindex{Schwarzkopf, Gustav@\textsc{Schwarzkopf, Gustav}!zzzSchnitzler, Arthur@\emph{von Arthur Schnitzler}!1906-12-071@{7. 12. 1906}|(be}
\toendnotes[C]{\smallbreak\pagebreak[2]}
\correspDesc{Versand  durch Arthur Schnitzler am 7. 12. 1906 in Wien
\newline{}Erhalt  durch Gustav Schwarzkopf im Zeitraum [7. 12. 1906 –
                  8. 12. 1906?] in Wien}\toendnotes[C]{\smallbreak}
\Standort{CUL, Schnitzler, B 96.}
\physDesc{Briefkarte, 296 Zeichen
\newline{}Handschrift: schwarze Tinte, deutsche Kurrent}\toendnotes[C]{\smallbreak}
\pstart
           {\pb}\textcolor{gray}{\textbf{Dr. Arthur Schnitzler}}\hfill 7. 12. 906\pend
           
\pstart
           \textcolor{gray}{\textbf{\textcolor{pink}{Wien,
                        XVIII. Spoettelgasse 7}\oindex{Wien@\textbf{Wien}!XVIII., Währing@\textbf{XVIII., Währing}!Edmund-Weiß-Gasse@\textbf{Edmund-Weiß-Gasse}, \emph{Straße}|pw}{}\ledrightnote{\textcolor{pink}{Edmund-Weiß-Gasse}}.}}\pend
           \vspace{0.5em}
\pstart
           lieber Guſtav, wollen Sie \label{K_L04039-1v}\edtext{Montag bei uns (\textcolor{pink}{Spoettelgaſſe}\oindex{Wien@\textbf{Wien}!XVIII., Währing@\textbf{XVIII., Währing}!Edmund-Weiß-Gasse 7@\textbf{Edmund-Weiß-Gasse 7}, \emph{Wohngebäude}|pw}{}\ledrightnote{\textcolor{pink}{Edmund-Weiß-Gasse 7}}) nachtmahlen}{\lemma{\textnormal{\emph{Montag … nachtmahlen}}}\Cendnote{\textnormal{Siehe A. S.: \emph{Tagebuch}, 10. 12. 1906.}}}\label{K_L04039-1}, ſo würde es uns ſehr freuen. \textcolor{blue}{\textsc{Gerty}}\pwindex{Hofmannsthal, Gertrude von 16.\,3.\,1880 Wien – 9.\,11.\,1959 Paddington@\textsc{Hofmannsthal, Gertrude von} (16.\,3.\,1880 Wien – 9.\,11.\,1959 Paddington)|pw}{}\ledrightnote{\textcolor{blue}{Gertrude von Hofmannsthal}} kommt jedenfalls; vielleicht
               auch {\pb}\textsc{\textcolor{blue}{Fred}\pwindex{W. Fred 29.\,6.\,1879 Wien – 23.\,10.\,1922 Berlin@\textsc{W. Fred} (29.\,6.\,1879 Wien – 23.\,10.\,1922 Berlin), \emph{Schriftsteller, Journalist}|pw}{}\ledrightnote{\textcolor{blue}{W. Fred}}}, der Ihnen glaub ich ſympathiſch ſein wird. Kommen Sie nicht ſpät, um auch noch
               was von \textcolor{blue}{Heini}\pwindex{Schnitzler, Heinrich 9.\,8.\,1902 Hinterbrühl – 12.\,7.\,1982 Wien@\textsc{Schnitzler, Heinrich} (9.\,8.\,1902 Hinterbrühl – 12.\,7.\,1982 Wien), \emph{Regisseur, Schauspieler}|pw}{}\ledrightnote{\textcolor{blue}{Heinrich Schnitzler}} zu genießen.\pend
           
\pstart
           Herzlichſt, mit Grüßen von uns \textcolor{blue}{Beiden}\pwindex{Schnitzler, Olga 17.\,1.\,1882 Wien – 13.\,1.\,1970 Lugano@\textsc{Schnitzler, Olga} (17.\,1.\,1882 Wien – 13.\,1.\,1970 Lugano), \emph{Schauspielerin, Sängerin}|pwv}{}\ledrightnote{{$\rightarrow$}\emph{\textcolor{blue}{Olga Schnitzler}}}{\\[\baselineskip]} Ihr{\\[\baselineskip]}\spacefill\mbox{A. S.}\pend
           \leftskip=0em{}\selectlanguage{ngerman}\endnumbering\briefempfaengerindex{Schwarzkopf, Gustav@\textsc{Schwarzkopf, Gustav}!zzzSchnitzler, Arthur@\emph{von Arthur Schnitzler}!1906-12-071@{7. 12. 1906}|)be}\mylabel{L04039h}
\begin{anhang}
\end{anhang}\normalsize

\doendnotes{C}
\bigskip
\vfill

\clearpage

\footnotesize

\lohead{\textsc{register}}

% Definiere theindex-Environment komplett neu ohne reledmac
\makeatletter
\renewenvironment{theindex}{%
  \section*{\indexname}%
  \setlength{\parindent}{0pt}%
  \setlength{\parskip}{0pt plus 0.3pt}%
  \let\item\@idxitem
}{%
  \clearpage
}
\makeatother

\IfFileExists{\jobname-pw.ind}{\input{\jobname-pw.ind}}{}

\end{document}

      