%% latex-korrekturansicht-vorspann.tex
%% Vorspann für die Korrekturansicht.
%% Lädt die gemeinsame Datei latex-vorspann.tex mit gesetztem Schalter.

\newif\ifkorrekturansicht
\korrekturansichttrue

\input{../tex-inputs/latex-vorspann}


               \section[Paul Goldmann an Arthur Schnitzler, 13. 11. {[}1895{]}]{ Paul Goldmann an Arthur Schnitzler, 13. 11. {[}1895{]}}\nopagebreak\mylabel{v}\rehead{ }\normalsize\beginnumbering\briefempfaengerindex{Schnitzler, Arthur@\textsc{Schnitzler, Arthur}!zzzGoldmann, Paul@\emph{von Paul Goldmann}!1895-11-131@{13. 11. {[}1895{]}}|(be} \toendnotes[C]{\smallbreak\pagebreak[2]} \Standort{DLA, A:Schnitzler, HS.NZ85.1.3165.}
\physDesc{Brief, 1 Blatt, 4 Seiten
\newline{}Handschrift: blaue Tinte, deutsche Kurrent
\newline{}Schnitzler: 1) mit Bleistift das Jahr » 95« vermerkt 2) mit rotem Buntstift drei Unterstreichungen}\toendnotes[C]{\smallbreak}\pstart
           \noindent{}{\pb}\textcolor{gray}{\textbf{\textbf{\textcolor{brown}{Frankfurter Zeitung}{}\ledrightnote{\textcolor{brown}{Frankfurter Zeitung}}}}}\pend
           \pstart
           \textcolor{gray}{\textbf{(\textcolor{brown}{\begin{otherlanguage}{french}Gazette de Francfort\end{otherlanguage}}{}\ledrightnote{\textcolor{brown}{Frankfurter Zeitung}}). }}\pend
           \pstart
           \textcolor{gray}{\textbf{\textbf{\begin{otherlanguage}{french}Fondateur M. \textcolor{blue}{L.
                              Sonnemann}{}\ledrightnote{\textcolor{blue}{Leopold Sonnemann}}\end{otherlanguage}.}}}\pend
           \pstart
           \begin{otherlanguage}{french}\textcolor{gray}{\textbf{\textcolor{green}{Journal}{}\ledrightnote{→\textcolor{green}{Frankfurter Zeitung}} politique,
                           financier,}}\end{otherlanguage}\hfill \textsc{\textcolor{pink}{Paris}{}\ledrightnote{\textcolor{pink}{Paris}}}, 13. November.\pend
           \pstart
           \begin{otherlanguage}{french}\textcolor{gray}{\textbf{commercial et littéraire.}}\end{otherlanguage}\pend
           \pstart
           \begin{otherlanguage}{french}\textcolor{gray}{\textbf{\textbf{Paraissant trois fois par jour.}}}\end{otherlanguage}\pend
           \pstart
           \begin{otherlanguage}{french}\textcolor{gray}{\textbf{\textbf{Bureau à \textcolor{pink}{Paris}{}\ledrightnote{\textcolor{pink}{Paris}}:}}}\end{otherlanguage}\pend
           \pstart
           \begin{otherlanguage}{french}\textcolor{gray}{\textbf{\textbf{\textcolor{pink}{24. Rue Feydeau}{}\ledrightnote{\textcolor{pink}{rue Feydeau}}.}}}\end{otherlanguage}\pend
           \pstart\center{}Mein lieber Freund,\pend\pstart
           Die Arbeit dauert fort, und den großen Brief kann ich noch immer nicht ſchreiben.
               Alſo den kleinen.\pend
           \pstart
           1.) Die »\textcolor{green}{Kleine Komödie}{}\ledrightnote{\textcolor{green}{Die kleine Komödie}}« iſt fertig \textcolor{green}{überſetzt}{}\ledrightnote{→\textcolor{green}{La petite comédie. Mœurs viennois}}, dem \textsc{\begin{otherlanguage}{french}\textcolor{blue}{Directeur}{}\ledrightnote{→\textcolor{blue}{Jules Frank}}\end{otherlanguage}} der »\textsc{\textcolor{brown}{Liberté}{}\ledrightnote{\textcolor{brown}{La Liberté}}}« überreicht u. von dieſem geſtern acceptirt
               worden. Sie dürfte nächſte Woche zu \label{K_L02755-1v}\edtext{erſcheinen}{\lemma{\textnormal{\emph{erſcheinen}}}\Cendnote{\textnormal{\textcolor{blue}{Arthur Schnitzler}: \emph{\textcolor{green}{La Petite comédie. Mœurs viennois}}. Übersetzt von \textcolor{blue}{Mme. Georges Aubry}. In: \emph{\textcolor{green}{La Liberté}}, Jg. 30, Nr. 11.327,
                        19. 11. 1895 bis Nr. 11.336, 28. 11. 1895. (acht
                     Teile)}}}\label{K_L02755-1h} beginnen. Außer \textsc{\textcolor{blue}{Sudermann}{}\ledrightnote{\textcolor{blue}{Hermann Sudermann}}} biſt Du ſeit Jahren der einzige deutſche Autor, von dem eine Arbeit im
               Roman-Feuilleton eines großen \textcolor{pink}{Pariſ}{}\ledrightnote{\textcolor{pink}{Paris}}er \textcolor{green}{Tagesblatt}{}\ledrightnote{→\textcolor{green}{La Liberté}}es erſcheint. Ein
               neuer kleiner Erfolg, zu dem ich Dir gratulire.\pend
           \pstart
           {\pb}2.) Wann erſcheint die »\textcolor{green}{Liebelei}{}\ledrightnote{\textcolor{green}{Liebelei. Schauspiel in drei Akten}}« als \label{K_L02755-2v}\edtext{Buch}{\lemma{\textnormal{\emph{Buch}}}\Cendnote{\textnormal{Die erste Buchausgabe erschien
                  in den Folgetagen nach der \textcolor{pink}{Berlin}er Premiere
                  der \emph{\textcolor{green}{Liebelei}} am 4. 2. 1896 bei \emph{\textcolor{brown}{S.
                     Fischer}}.}}}\label{K_L02755-2h}? Ich erbitte mehrere Exemplare, und eines ſendeſt Du wohl
               mit einer freundlichen Widmung an \textsc{\textcolor{blue}{Pierre Lalo}{}\ledrightnote{\textcolor{blue}{Pierre Lalo}}}, \strikeout{(19. \textsc{Bvd}}{ }\strikeout{(19} (\textsc{\textcolor{pink}{19. Boulevard de Courcelles}{}\ledrightnote{\textcolor{pink}{Boulevard de Courcelles}}}), der mich dieſer Tage danach fragte u. uns hoffentlich im »\textsc{\textcolor{green}{Journal des Débats}{}\ledrightnote{\textcolor{green}{Journal des débats. Politiques et littéraires}}}« einen \label{K_L02755-3v}\edtext{Bericht}{\lemma{\textnormal{\emph{Bericht}}}\Cendnote{\textnormal{Dazu kam es nicht, aber die Buchausgabe
                  wurde angezeigt: [O. V.]: \emph{\textcolor{green}{Courrier des
                        Théatres}}. In: \emph{\textcolor{green}{Journal des débats
                        politiques et littéraires}}, Jg. 108, Nr. 43, 13. 2. 1895, S. 3.}}}\label{K_L02755-3h} darüber ſchreiben wird.\pend
           \pstart
           3.) Ich bitte Dich oder \textsc{\textcolor{blue}{Richard}{}\ledrightnote{\textcolor{blue}{Richard Beer-Hofmann}}} um eine gute Einführung bei \textsc{\textcolor{blue}{Johann Strauss}{}\ledrightnote{\textcolor{blue}{Johann Strauss}}}, der dieſer Tage nach \textsc{\textcolor{pink}{Paris}{}\ledrightnote{\textcolor{pink}{Paris}}} kommt. Hier wird ihn natürlich \textsc{\textcolor{blue}{Feldmann}{}\ledrightnote{\textcolor{blue}{Siegmund Feldmann}}} in Beſchlag nehmen, und ich will {\pb}mich von
               dieſem Menſchen nicht glücklich machen laſſen. Müßt mir aber die Empfehlung bald
               ſchicken.\pend
           \pstart
           4.) \textsc{\textcolor{blue}{Hoffmannsthal}{}\ledrightnote{\textcolor{blue}{Hugo von Hofmannsthal}}}s \label{K_L02755-34v}\edtext{\textcolor{green}{Erzählung}{}\ledrightnote{→\textcolor{green}{Das Märchen der 672. Nacht}}}{\lemma{\textnormal{\emph{Erzählung}}}\Cendnote{\textnormal{\textcolor{blue}{Hugo von Hofmannsthal}: \emph{\textcolor{green}{Das Märchen der 672. Nacht. Geschichte des jungen
                        Kaufmannssohnes und seiner vier Diener}}. In: \emph{\textcolor{green}{Die Zeit}}, Bd. 5, Nr. 57, 2. 11. 1895, S. 79–80; Nr. 58, 9. 11. 1895, S. 95–96; Nr. 59, 16. 11. 1895, S. 111–112.}}}\label{K_L02755-34h} in der »\textcolor{green}{Zeit}{}\ledrightnote{\textcolor{green}{Die Zeit. Wiener Wochenschrift}}« mißfällt mir ſehr.\pend
           \pstart
           5.) Wer iſt der \textcolor{blue}{Maler}{}\ledrightnote{→\textcolor{blue}{Leonhard Fanto}}{ }\textsc{\textcolor{blue}{Fanto}{}\ledrightnote{\textcolor{blue}{Leonhard Fanto}}}? Er iſt zu mir gekommen mit einer Empfehlung von \textsc{\textcolor{blue}{Bahr}{}\ledrightnote{\textcolor{blue}{Hermann Bahr}}}, was bereits ſehr gegen ihn ſpricht. Auch mag ich ihn perſönlich nicht, es
               ſteckt in ihm viel mit Wohlwollen umwickelter Neid. Kann der \textcolor{blue}{Burſche}{}\ledrightnote{→\textcolor{blue}{Leonhard Fanto}} was?\pend
           \pstart
           6.) Wüßte ich nur, {\pb}\strikeout{wies} wie’s Dir geht!\pend
           \pstart
           8.) Grüß’ Dich Gott!\pend
           \pstart
           In Treue {\\[\baselineskip]}Dein {\\[\baselineskip]}\spacefill\mbox{Paul Goldmann}\pend
           \leftskip=0em{}\endnumbering\briefempfaengerindex{Schnitzler, Arthur@\textsc{Schnitzler, Arthur}!zzzGoldmann, Paul@\emph{von Paul Goldmann}!1895-11-131@{13. 11. {[}1895{]}}|)be}\mylabel{h}  \normalsize

\doendnotes{C}
\bigskip
\vfill

\clearpage

\footnotesize

\lohead{\textsc{register}}

% Definiere theindex-Environment komplett neu ohne reledmac
\makeatletter
\renewenvironment{theindex}{%
  \section*{\indexname}%
  \setlength{\parindent}{0pt}%
  \setlength{\parskip}{0pt plus 0.3pt}%
  \let\item\@idxitem
}{%
  \clearpage
}
\makeatother

\IfFileExists{\jobname-pw.ind}{\input{\jobname-pw.ind}}{}

\end{document}

      