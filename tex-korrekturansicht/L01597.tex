%% latex-korrekturansicht-vorspann.tex
%% Vorspann für die Korrekturansicht.
%% Lädt die gemeinsame Datei latex-vorspann.tex mit gesetztem Schalter.

\newif\ifkorrekturansicht
\korrekturansichttrue

\input{../tex-inputs/latex-vorspann}


               \section[Richard Beer-Hofmann an Arthur Schnitzler, {[}zum 15.?{]} 5. 1906]{ Richard Beer-Hofmann an Arthur Schnitzler, {[}zum
               15.?{]} 5. 1906}\nopagebreak\mylabel{v}\rehead{ }\normalsize\beginnumbering\briefempfaengerindex{Schnitzler, Arthur@\textsc{Schnitzler, Arthur}!zzzBeer-Hofmann, Richard@\emph{von Richard Beer-Hofmann}!1906-05-151@{{[}zum 15.?{]} 5. 1906}|(be} \toendnotes[C]{\smallbreak\pagebreak[2]} \Standort{CUL, Schnitzler, B 8.}
\physDesc{Brief, 1 Blatt, 1 Seite
\newline{}Handschrift: schwarze Tinte, lateinische Kurrent\newline{}Ordnung: mit Bleistift von unbekannter Hand nummeriert: »205a« }\buchAbdrucke{\weitereDrucke{Arthur Schnitzler, Richard Beer-Hofmann: \emph{Briefwechsel 1891–1931}. Hg. Konstanze Fliedl. Wien, Zürich: \emph{Europaverlag} 1992, S. 178.} }\toendnotes[C]{\smallbreak}\pstart
           \noindent{}\centering{}{\pb}\uline{»\textcolor{green}{Der
                     einsame Weg}{}\ledrightnote{→\textcolor{green}{Der einsame Weg. Schauspiel in fünf Akten}}«}\pend
           \pstart
           \noindent{}\raggedleft{}\uline{An Arthur Schnitzler}\pend
           \stanza{}Alle Wege die wir treten\newverse{}Münden in die Einsamkeit,\newverse{}Nimmermüde Stunden jäten\newverse{}Aus, was wuchs, an Lust und Leid.\newverse{}\newverse{}Alles Glück, und alles Elend\newverse{}Blasst zu fernem Wi\strikeout{e}derschein,\newverse{}Was beseeligend, was quälend,\newverse{}Geht – lässt uns, mit uns allein.\newverse{}\newverse{}Schritt ich eben nicht im \textcolor{green}{Reigen}{}\ledrightnote{→\textcolor{green}{Reigen. Zehn Dialoge}}?\newverse{}Und was traf, das traf gemeinsam!\newverse{}Bietet keine Hand sich? – Schweigen\newverse{}Sieht mich an – der Weg wird einsam.\newverse{}\newverse{}Ob ich stieg von Glückesthronen,\newverse{}Ob ich klomm aus Leidensgründen –\newverse{}Dort, wohin ich geh zu wohnen,\newverse{}Wird sich keines zu mir finden!\newverse{}\newverse{}Ein Erkennen nur, mit klaaren\newverse{}Augen, will mich hingeleiten:\newverse{}Dass, auch vorher, um mich waren,\newverse{}– Unerkannt – nur Einsamkeiten!\stanzaend{}\pstart \spacefill\mbox{R. B-H.}\pend{}\pstart
           \textcolor{pink}{Rodaun}{}\ledrightnote{\textcolor{pink}{Rodaun}}, \label{K_L01597_1v}\edtext{Mai 1906}{\lemma{\textnormal{\emph{Mai 1906}}}\Cendnote{\textnormal{Am
                        15. 5. 1906 feierte \textcolor{blue}{Schnitzler} seinen 44. Geburtstag.}}}\label{K_L01597_1h}\pend
           \endnumbering\briefempfaengerindex{Schnitzler, Arthur@\textsc{Schnitzler, Arthur}!zzzBeer-Hofmann, Richard@\emph{von Richard Beer-Hofmann}!1906-05-151@{{[}zum 15.?{]} 5. 1906}|)be}\mylabel{h}  \normalsize

\doendnotes{C}
\bigskip
\vfill

\clearpage

\footnotesize

\lohead{\textsc{register}}

% Definiere theindex-Environment komplett neu ohne reledmac
\makeatletter
\renewenvironment{theindex}{%
  \section*{\indexname}%
  \setlength{\parindent}{0pt}%
  \setlength{\parskip}{0pt plus 0.3pt}%
  \let\item\@idxitem
}{%
  \clearpage
}
\makeatother

\IfFileExists{\jobname-pw.ind}{\input{\jobname-pw.ind}}{}

\end{document}

      