%% latex-korrekturansicht-vorspann.tex
%% Vorspann für die Korrekturansicht.
%% Lädt die gemeinsame Datei latex-vorspann.tex mit gesetztem Schalter.

\newif\ifkorrekturansicht
\korrekturansichttrue

\input{../tex-inputs/latex-vorspann}


               \section[Arthur Schnitzler an Hugo Hofmannsthal, 3. 9. 1923]{ Arthur Schnitzler an Hugo Hofmannsthal, 3. 9. 1923}\nopagebreak\mylabel{v}\rehead{ }\normalsize\beginnumbering\briefempfaengerindex{Hofmannsthal, Hugo von@\textsc{Hofmannsthal, Hugo von}!zzzSchnitzler, Arthur@\emph{von Arthur Schnitzler}!1923-09-031@{3. 9. 1923}|(be} \toendnotes[C]{\smallbreak\pagebreak[2]} \Standort{FDH, Hs-30885,150.}
\physDesc{Postkarte
\newline{}Handschrift: Bleistift, lateinische Kurrent\newline{}Versand: 1) nachgesandt nach \textcolor{pink}{Bad-Aussee} 2) Stempel: »\nobreak{}\oindex{Celerina@\textbf{Celerina}, \emph{Besiedelter Ort (A.BSO)}|pwk}Celerina (Graubünden), 3. IX. 23, 12\nobreak{}«. 3) Stempel: »\nobreak{}\oindex{Rodaun@\textbf{Rodaun}, \emph{Teil eines besiedelten Ortes (A.BSOX)}|pwk}Rodaun, \textcolor{gray}{6} 9 23\nobreak{}«. }\buchAbdrucke{\weitereDrucke{Hugo von Hofmannsthal, Arthur Schnitzler: \emph{Briefwechsel}. Hg. Therese Nickl und Heinrich Schnitzler. Frankfurt am Main: \emph{S. Fischer} 1964, S. 298.} }\toendnotes[C]{\smallbreak}\pstart{}{\pb}Herrn\pend{}\pstart{}Dr. Hugo von Hofmannsthal\pend{}\pstart{}\textcolor{pink}{Rodaun}{}\ledrightnote{\textcolor{pink}{Rodaun}}\pend{}\pstart{}bei \textcolor{pink}{Wien}{}\ledrightnote{\textcolor{pink}{Wien}}\pend{}\pstart{}(\textcolor{pink}{Südbahn}{}\ledrightnote{\textcolor{pink}{Südbahnhof}}{[}){]}\pend{}\pstart{}\textcolor{pink}{NiederOesterreich}{}\ledrightnote{\textcolor{pink}{Niederösterreich}}\pend{}{\bigskip}\pstart
           \noindent{}\centering{}\textcolor{gray}{\textbf{{\pb}\textcolor{pink}{Celerina}{}\ledrightnote{\textcolor{pink}{Celerina}}}}\pend
           \pstart
           \raggedleft{}{\pb}3/9. 23.\pend
           \pstart
           mein lieber Hugo – Ihr letztes Lebenszeichen hab ich vor \label{K_L02403_1v}\edtext{Monaten}{\lemma{\textnormal{\emph{Monaten}}}\Cendnote{\textnormal{siehe Hugo Hofmannsthal an Arthur Schnitzler, 15. 5. 1923}}}\label{K_L02403_1h} aus der \textcolor{pink}{Schweiz}{}\ledrightnote{\textcolor{pink}{Schweiz}} erhalten – und heut
               erst, auch aus der \textcolor{pink}{Schweiz}{}\ledrightnote{\textcolor{pink}{Schweiz}}, aus \textcolor{pink}{Celerina}{}\ledrightnote{\textcolor{pink}{Celerina}}, wo mich vor 9 Jahren der Krieg überrascht hat
                  un\textcolor{gray}{d} ich \introOben{}heuer\introOben{} ein paar gute Wochen
               allein verlebt habe, {\pb}erwider ich Ihren lieben Gruſs.
               Heute reis ich ab, seh mir noch im \textcolor{pink}{Engadin}{}\ledrightnote{\textcolor{pink}{Engadin}} einiges
               an, und geh da{\geminationn} an den \textcolor{pink}{Bodensee}{}\ledrightnote{\textcolor{pink}{Bodensee}} (\textcolor{pink}{Bregenz}{}\ledrightnote{\textcolor{pink}{Bregenz}}), von wo ich \textcolor{blue}{Lili}{}\ledrightnote{\textcolor{blue}{Lili Schnitzler}} abhole. Auf Wiedersehen hoffentlich! \pend
           \pstart Ihr \spacefill\mbox{Arthur}\pend{}\endnumbering\briefempfaengerindex{Hofmannsthal, Hugo von@\textsc{Hofmannsthal, Hugo von}!zzzSchnitzler, Arthur@\emph{von Arthur Schnitzler}!1923-09-031@{3. 9. 1923}|)be}\mylabel{h}  \normalsize

\doendnotes{C}
\bigskip
\vfill

\clearpage

\footnotesize

\lohead{\textsc{register}}

% Definiere theindex-Environment komplett neu ohne reledmac
\makeatletter
\renewenvironment{theindex}{%
  \section*{\indexname}%
  \setlength{\parindent}{0pt}%
  \setlength{\parskip}{0pt plus 0.3pt}%
  \let\item\@idxitem
}{%
  \clearpage
}
\makeatother

\IfFileExists{\jobname-pw.ind}{\input{\jobname-pw.ind}}{}

\end{document}

      