%% latex-korrekturansicht-vorspann.tex
%% Vorspann für die Korrekturansicht.
%% Lädt die gemeinsame Datei latex-vorspann.tex mit gesetztem Schalter.

\newif\ifkorrekturansicht
\korrekturansichttrue

\input{../tex-inputs/latex-vorspann}


\renewcommand{\erwaehntePersonen}{Personen: Julius von Gans-Ludassy, Victorien Sardou}
\renewcommand{\erwaehnteInstitutionen}{Institutionen: Wiener Allgemeine Zeitung}
\renewcommand{\erwaehnteOrte}{Orte: Burgtheater, Café Griensteidl, Universitätsstraße, Wien, Wollzeile}
\renewcommand{\erwaehnteWerke}{Werke: Die alten Junggesellen}
\section[ Felix Salten an Arthur Schnitzler, {[}27. 4. 1896{]}]{Felix Salten an Arthur Schnitzler, {[}27. 4. 1896{]}}
\nopagebreak\mylabel{v}
\rehead{ }\normalsize\beginnumbering\briefempfaengerindex{Schnitzler, Arthur@\textsc{Schnitzler, Arthur}!zzzSalten, Felix@\emph{von Felix Salten}!1896-04-271@{{[}27. 4. 1896{]}}|(be}
\toendnotes[C]{\smallbreak\pagebreak[2]}\Standort{CUL, Schnitzler, B 89, A 1.}
\physDesc{Brief, 1 Blatt, 1 Seite, 149 Zeichen
\newline{}Handschrift: Bleistift, lateinische Kurrent
\newline{}Schnitzler: mit Bleistift auf der Vorlage datiert: »27/4 \textcolor{gray}{\textbf{189}}6« 
\newline{}Ordnung: mit Bleistift von unbekannter Hand nummeriert: »70« }\toendnotes[C]{\smallbreak}
\pstart
           \noindent{}{\pb}\textcolor{gray}{\textbf{\textbf{»\textcolor{brown}{Wiener Allgemeine
                        Zeitung}{}\ledrightnote{\textcolor{brown}{Wiener Allgemeine Zeitung}}«}}}\pend
           
\pstart
           \textcolor{gray}{\textbf{Redaction:}}\pend
           
\pstart
           \textcolor{gray}{\textbf{\textbf{\textcolor{pink}{IX/3, Univerſitätsstraße Nr. 6}{}\ledrightnote{\textcolor{pink}{Universitätsstraße}}.}}}\pend
           
\pstart
           \textcolor{gray}{\textbf{Administration:}}\hfill \textcolor{gray}{\textbf{\textcolor{pink}{Wien}{}\ledrightnote{\textcolor{pink}{Wien}}, am ..........{ }189{\dots}}}\pend
           
\pstart
           \textcolor{gray}{\textbf{\textbf{\textcolor{pink}{I. Wollzeile Nr. 5}{}\ledrightnote{\textcolor{pink}{Wollzeile}}} (im Durchhauſe).}}\pend
           
\pstart
           \textcolor{gray}{\textbf{Telegramm-Adreſſe: »Allgemeine, \textcolor{pink}{Wien}{}\ledrightnote{\textcolor{pink}{Wien}}«.}}\pend
           
\pstart
           \textcolor{gray}{\textbf{Telephon der Redaction: Nr. 805 u. 2180.}}\pend
           
\pstart
           \textcolor{gray}{\textbf{\hspace*{1.5em}„\hspace*{1.5em}„\hspace*{1.5em} Adminiſtration: Nr. 1024.}}\pend
           
\pstart
           lieber Arthur.{ }\textcolor{blue}{Ludaßy}{}\ledrightnote{\textcolor{blue}{Julius von Gans-Ludassy}} hat die \label{K_L03171-1v}\edtext{\textcolor{pink}{Loge}{}\ledrightnote{{$\rightarrow$}\textcolor{pink}{Burgtheater}}}{\lemma{\textnormal{\emph{Loge}}}\Cendnote{\textnormal{wohl für die Aufführung von \textcolor{blue}{Victorien Sardou}s \emph{\textcolor{green}{Die alten Junggesellen}} im \textcolor{pink}{Burgtheater}, vgl. A. S.: \emph{Tagebuch}, 27. 4. 1896}}}\label{K_L03171-1h} im letzten Momente \label{K_L03171-2v}\edtext{mit
               Beschlag gelegt}{\lemma{\textnormal{\emph{mit
               Beschlag gelegt}}}\Cendnote{\textnormal{im Sinne von: für sich in
                  Anspruch genommen}}}\label{K_L03171-2h}.\pend
           
\pstart
           Ich werde heute im \textcolor{pink}{Griensteidl}{}\ledrightnote{\textcolor{pink}{Café Griensteidl}} sein. Gegen die \textcolor{pink}{Loge}{}\ledrightnote{{$\rightarrow$}\textcolor{pink}{Burgtheater}} kann ich nichts machen.\pend
           \pstart Ihr \spacefill\mbox{Salten.}\pend{}\endnumbering\briefempfaengerindex{Schnitzler, Arthur@\textsc{Schnitzler, Arthur}!zzzSalten, Felix@\emph{von Felix Salten}!1896-04-271@{{[}27. 4. 1896{]}}|)be}\mylabel{h}  \normalsize

\doendnotes{C}
\bigskip
\vfill

\clearpage

\footnotesize

\lohead{\textsc{register}}

% Definiere theindex-Environment komplett neu ohne reledmac
\makeatletter
\renewenvironment{theindex}{%
  \section*{\indexname}%
  \setlength{\parindent}{0pt}%
  \setlength{\parskip}{0pt plus 0.3pt}%
  \let\item\@idxitem
}{%
  \clearpage
}
\makeatother

\IfFileExists{\jobname-pw.ind}{\input{\jobname-pw.ind}}{}

\end{document}

      