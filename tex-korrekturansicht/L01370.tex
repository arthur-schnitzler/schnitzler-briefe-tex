%% latex-korrekturansicht-vorspann.tex
%% Vorspann für die Korrekturansicht.
%% Lädt die gemeinsame Datei latex-vorspann.tex mit gesetztem Schalter.

\newif\ifkorrekturansicht
\korrekturansichttrue

\input{../tex-inputs/latex-vorspann}


               \section[Hermann Bahr an Arthur Schnitzler, 5. 2. 1904]{ Hermann Bahr an Arthur Schnitzler, 5. 2. 1904}\nopagebreak\mylabel{v}\rehead{ }\normalsize\beginnumbering\briefempfaengerindex{Schnitzler, Arthur@\textsc{Schnitzler, Arthur}!zzzBahr, Hermann@\emph{von Hermann Bahr}!1904-02-051@{5. 2. 1904}|(be} \toendnotes[C]{\smallbreak\pagebreak[2]} \Standort{CUL, Schnitzler, B 5b.}
\physDesc{Brief, 1 Blatt, 2 Seiten
\newline{}Handschrift: schwarze Tinte, deutsche Kurrent\newline{}Ordnung: mit Bleistift von unbekannter Hand nummeriert: »109« }\buchAbdrucke{\weitereDrucke{Hermann Bahr, Arthur Schnitzler: \emph{Briefwechsel, Aufzeichnungen, Dokumente (1891–1931)}. Hg. Kurt Ifkovits und Martin Anton Müller. Göttingen: \emph{Wallstein} 2018, S. 295–296.} }\toendnotes[C]{\smallbreak}\pstart
           \raggedleft{}{\pb}5. 2. 04\pend
           \pstart{}Lieber Arthur!\pend\pstart
           Mich berührt natürlich der \textcolor{green}{Fichtner}{}\ledrightnote{→\textcolor{green}{Der einsame Weg. Schauspiel in fünf Akten}} am meiſten, in welchem
               ich unheimlich viel von mir finde (meine Sachen ließen sich kritiſch gar nicht beſſer
               bezeichnen als damit daß ich mich leider auch in ihnen ſozusagen nur vorübergehend
                  aufhielt). Ich verſtehe auch das \damage{Ver}hältnis \textcolor{green}{Julian –
               Wegrath}{}\ledrightnote{→\textcolor{green}{Der einsame Weg. Schauspiel in fünf Akten}}, ebenſo das \textcolor{green}{Julian –
                  Felix}{}\ledrightnote{→\textcolor{green}{Der einsame Weg. Schauspiel in fünf Akten}}{ }ſo gut, während ich mir das \textcolor{green}{Sala – Johanna}{}\ledrightnote{→\textcolor{green}{Der einsame Weg. Schauspiel in fünf Akten}} nicht ganz erklären und mich
               darin nicht zurechtfinden kann. Außerdem miſcht ſich jetzt bei mir Perſönliches in
               alles, ſo die Neugierde, die mich plagt, ob \textcolor{green}{Sala}{}\ledrightnote{→\textcolor{green}{Der einsame Weg. Schauspiel in fünf Akten}} nicht vollkommen meinen Herzzuſtand hat und wie der
               Arzt dann \textcolor{gray}{denn} doch ſeinen Tod faſt auf den Tag zu wiſſen glauben
               kann – was ſehr albern von mir iſt.\pend
           \pstart
           {\pb}Kritiſch möcht ich ſagen: Daß in dem \textcolor{green}{Stück}{}\ledrightnote{→\textcolor{green}{Der einsame Weg. Schauspiel in fünf Akten}} viel mehr angeſchlagen und
               aufgeregt als zuletzt ausgelöſt wird, was ich weniger problematiſch als muſikaliſch
               meine. Für mein Gefühl iſt das \textcolor{green}{Stück}{}\ledrightnote{→\textcolor{green}{Der einsame Weg. Schauspiel in fünf Akten}} aus, bevor es ſeine Stimmungsmotive naturgemäß hat aus- und ablaufen
               laſſen.\pend
           \pstart
           \label{K_L01370_1v}\edtext{Prachtvoll find ich den \textcolor{green}{\textsc{Cassian}}{}\ledrightnote{\textcolor{green}{Der einsame Weg. Schauspiel in fünf Akten}}}{\lemma{\textnormal{\emph{Prachtvoll … Cassian}}}\Cendnote{\textnormal{Erstdruck: \emph{\textcolor{green}{Der tapfere Cassian. Burleske in einem Akt}}.
                     In: \emph{\textcolor{green}{Neue Rundschau}}, Jg. 15, H. 2,
                        1. 2. 1904, S. 227–247.}}}\label{K_L01370_1h} und bedaure nur, daß die
               blöden \textcolor{pink}{Deutſchen}{}\ledrightnote{\textcolor{pink}{Deutschland}} für ſolchen argloſen und rein
               ſinnlichen und darum künſtleriſch reinen Humor nun einmal keine Organe
                  habe{[}n{]}.\pend
           \pstart
           Da ich mich ſehr ſchlecht fühle, iſt es möglich, daß ich ſchon ſehr bald hier
               weggehe, vielleicht nach \textcolor{pink}{Abbazia}{}\ledrightnote{\textcolor{pink}{Opatija}}. Jedenfalls lockt
               mich der Gedanke, Dich im April in \textcolor{pink}{Taormina}{}\ledrightnote{\textcolor{pink}{Taormina}} zu
               finden, ſehr. Hoffentlich.\pend
           \pstart
           Grüß Deine \textcolor{blue}{Frau}{}\ledrightnote{→\textcolor{blue}{Olga Schnitzler}}, \textcolor{blue}{Brahm}{}\ledrightnote{\textcolor{blue}{Otto Brahm}}, den ſtark von \textcolor{blue}{Reinhardt}{}\ledrightnote{\textcolor{blue}{Marie Reinhard}} bekümmerten \textcolor{blue}{Trebitſch}{}\ledrightnote{\textcolor{blue}{Siegfried Trebitsch}} und –
               ich wär ſehr froh, wenn der »\textcolor{green}{einſame Weg}{}\ledrightnote{\textcolor{green}{Der einsame Weg. Schauspiel in fünf Akten}}« ein
               großer Erfolg würde!\pend
           \pstart
           Herzlichſt{\\[\baselineskip]}\spacefill\mbox{Hermann}\pend
           \leftskip=0em{}\endnumbering\briefempfaengerindex{Schnitzler, Arthur@\textsc{Schnitzler, Arthur}!zzzBahr, Hermann@\emph{von Hermann Bahr}!1904-02-051@{5. 2. 1904}|)be}\mylabel{h}  \normalsize

\doendnotes{C}
\bigskip
\vfill

\clearpage

\footnotesize

\lohead{\textsc{register}}

% Definiere theindex-Environment komplett neu ohne reledmac
\makeatletter
\renewenvironment{theindex}{%
  \section*{\indexname}%
  \setlength{\parindent}{0pt}%
  \setlength{\parskip}{0pt plus 0.3pt}%
  \let\item\@idxitem
}{%
  \clearpage
}
\makeatother

\IfFileExists{\jobname-pw.ind}{\input{\jobname-pw.ind}}{}

\end{document}

      