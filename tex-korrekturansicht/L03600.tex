%% latex-korrekturansicht-vorspann.tex
%% Vorspann für die Korrekturansicht.
%% Lädt die gemeinsame Datei latex-vorspann.tex mit gesetztem Schalter.

\newif\ifkorrekturansicht
\korrekturansichttrue

\input{../tex-inputs/latex-vorspann}


\renewcommand{\erwaehntePersonen}{Personen: Felix Salten}
\renewcommand{\erwaehnteInstitutionen}{Institutionen: E. Pierson’s Verlag}
\renewcommand{\erwaehnteOrte}{Orte: Dresden, Leipzig, Wien}
\renewcommand{\erwaehnteWerke}{Werke: Börsenblatt für den Deutschen Buchhandel, Das Märchen. Schauspiel in drei Aufzügen}
\section[Arthur Schnitzler: Widmungsexemplar Das Märchen für Felix Salten, 8. 5. 1894]{Arthur Schnitzler: Widmungsexemplar Das Märchen für Felix Salten,
               8. 5. 1894}
\nopagebreak\mylabel{v}
\rehead{ }\normalsize\beginnumbering\briefempfaengerindex{Salten, Felix@\textsc{Salten, Felix}!zzzSchnitzler, Arthur@\emph{von Arthur Schnitzler}!1894-05-083@{8. 5. 94}|(be}
\toendnotes[C]{\smallbreak\pagebreak[2]}\Standort{Wienbibliothek im Rathaus, A-57094/2.Ex., DS-2018-9503.}
\physDesc{Widmung am Vorsatzblatt, 59 Zeichen
\newline{}Handschrift: schwarze Tinte, deutsche Kurrent
\newline{}Ordnung: 1) mit schwarzer Tinte am linken Vorsatzblatt gestrichene Regalerfassung: »\noindent{}IN\textsuperscript{o} 2468 WN\textsuperscript{o} 1537{ / }XI b«  2) mit schwarzer Tinte ausgefüllter Stempel: »\noindent{}\textcolor{gray}{\textbf{\textit{Felix
                                 Salten}}}{ / }\textcolor{gray}{\textbf{\textit{Inv.Nr.}}}{ }4464{ / }\textcolor{gray}{\textbf{\textit{Werk Nr.}}}{ }2192{ / }\textcolor{gray}{\textbf{\textit{Schrank}}}{ }XIV A.Z \textcolor{gray}{\textbf{\textit{Fach}}} b«
                              }\toendnotes[C]{\smallbreak}
\pstart
           \noindent{}{\pb}Meinem lieben \textsc{Felix Salten}\pend
           
\pstart
           herzlichſt{\\[\baselineskip]}\spacefill\mbox{Arth Sch}\pend
           \leftskip=0em{}
\pstart
           \textcolor{pink}{Wien}{}\ledrightnote{\textcolor{pink}{Wien}}, 8. 5. 94.\pend
           {\bigskip}
\pstart
           \noindent{}\centering{}{\pb}\textcolor{gray}{\textbf{\textcolor{green}{Das Märchen}{}\ledrightnote{\textcolor{green}{Das Märchen. Schauspiel in drei Aufzügen}}.}}\pend
           
\pstart
           \noindent{}\centering{}\textcolor{gray}{\textbf{\textbf{Schauſpiel in drei Aufzügen}}}{\\}\textcolor{gray}{\textbf{von}}{\\}\textcolor{gray}{\textbf{\textbf{Arthur Schnitzler.}}}\pend
           {\bigskip}
\pstart
           \noindent{}\centering{}\textcolor{gray}{\textbf{\textcolor{pink}{\textbf{Dresden}}{}\ledrightnote{\textcolor{pink}{Dresden}} und \textcolor{pink}{\textbf{Leipzig}}{}\ledrightnote{\textcolor{pink}{Leipzig}}}}\pend
           
\pstart
           \noindent{}\centering{}\textcolor{gray}{\textbf{\textcolor{brown}{\so{E. Pierſon’s Verlag}}{}\ledrightnote{\textcolor{brown}{E. Pierson’s Verlag}}}}\pend
           
\pstart
           \noindent{}\centering{}\textcolor{gray}{\textbf{\label{K_L03600-1v}\edtext{1894}{\lemma{\textnormal{\emph{1894}}}\Cendnote{\textnormal{\emph{\textcolor{green}{Das Märchen}} war am
                           5. 5. 1894 vom \emph{\textcolor{green}{Börsenblatt für den deutschen Buchhandel}} als
                        Neuerscheinung gemeldet worden.}}}\label{K_L03600-1h}.}}\pend
           \endnumbering\briefempfaengerindex{Salten, Felix@\textsc{Salten, Felix}!zzzSchnitzler, Arthur@\emph{von Arthur Schnitzler}!1894-05-083@{8. 5. 94}|)be}\mylabel{h}
\begin{anhang}
\end{anhang}\normalsize

\doendnotes{C}
\bigskip
\vfill

\clearpage

\footnotesize

\lohead{\textsc{register}}

% Definiere theindex-Environment komplett neu ohne reledmac
\makeatletter
\renewenvironment{theindex}{%
  \section*{\indexname}%
  \setlength{\parindent}{0pt}%
  \setlength{\parskip}{0pt plus 0.3pt}%
  \let\item\@idxitem
}{%
  \clearpage
}
\makeatother

\IfFileExists{\jobname-pw.ind}{\input{\jobname-pw.ind}}{}

\end{document}

      