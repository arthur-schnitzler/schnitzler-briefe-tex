%% latex-korrekturansicht-vorspann.tex
%% Vorspann für die Korrekturansicht.
%% Lädt die gemeinsame Datei latex-vorspann.tex mit gesetztem Schalter.

\newif\ifkorrekturansicht
\korrekturansichttrue

\input{../tex-inputs/latex-vorspann}


               \section[Hugo von Hofmannsthal an Arthur Schnitzler, 30. 9. 1905]{ Hugo von Hofmannsthal an Arthur Schnitzler, 30. 9. 1905}\nopagebreak\mylabel{v}\rehead{ }\normalsize\beginnumbering\briefempfaengerindex{Schnitzler, Arthur@\textsc{Schnitzler, Arthur}!zzzHofmannsthal, Hugo von@\emph{von Hugo von Hofmannsthal}!1905-09-301@{30. 9. 1905}|(be} \toendnotes[C]{\smallbreak\pagebreak[2]} \Standort{CUL, Schnitzler, B 43.}
\physDesc{Postkarte
\newline{}Handschrift: schwarze Tinte, deutsche Kurrent\newline{}Versand: 1) Stempel: »\nobreak{}\oindex{Rodaun@\textbf{Rodaun}, \emph{Teil eines besiedelten Ortes (A.BSOX)}|pwk}{[}Rodau{]}n, 1 10 \textcolor{gray}{05}\nobreak{}«.  2) Stempel: »\nobreak{}\oindex{XVIII., Waehring@\textbf{XVIII., Währing}, \emph{Bezirk (A.BZK)}|pwk}18/1 Wien 110, 2 X 05, VIII, Bestellt\nobreak{}«. \newline{}Ordnung: 1) mit Bleistift von unbekannter Hand nummeriert:
                                    »253« 2) mit Bleistift von unbekannter Hand nummeriert: »258a«}\buchAbdrucke{\weitereDrucke{Hugo von Hofmannsthal, Arthur Schnitzler: \emph{Briefwechsel}. Hg. Therese Nickl und Heinrich Schnitzler. Frankfurt am Main: \emph{S. Fischer} 1964, S. 215.} }\toendnotes[C]{\smallbreak}\pstart{}{\pb}\textsc{Herrn D\textsuperscript{r} Arthur Schnitzler}\pend{}\pstart{}\textcolor{pink}{\textsc{Wien}}{}\ledrightnote{\textcolor{pink}{Wien}}\pend{}\pstart{}\textcolor{pink}{\textsc{XVIII Spöttelgasse 7}.}{}\ledrightnote{\textcolor{pink}{Edmund-Weiß-Gasse}}\pend{}{\bigskip}\pstart
           \raggedleft{}{\pb}Samstg 30/9 905\pend
           \pstart
           lieber, ich bin ſchon über eine Woche zurück, arbeite aber vor- und
               nachmittg, wenn ich nicht, wie zufällig heute, unwohl bin. Ich höre von Bahr, daſs
               der »\textcolor{green}{Ruf des Lebens}{}\ledrightnote{\textcolor{green}{Der Ruf des Lebens. Schauspiel in drei Akten}}« ſchon in irgend einer Form
               lesbar vorliegt. Ich wäre ſehr froh, es im Ganzen zu leſen. Dem »\textcolor{green}{Zwiſchenſpiel}{}\ledrightnote{\textcolor{green}{Zwischenspiel. Komödie in drei Akten}}« bewahre ich die ſchönſte Erinnerung und würde mich
               auf die \label{K_L01553_1v}\edtext{Aufführung}{\lemma{\textnormal{\emph{Aufführung}}}\Cendnote{\textnormal{Die Uraufführung fand am
                     12. 10. 1905 statt.}}}\label{K_L01553_1h} ſehr freuen, wäre nicht \textcolor{blue}{\uuline{Witt}}{}\ledrightnote{\textcolor{blue}{Lotte Witt}}! Unbegreiflich! Unerklärlich!\pend
           \pstart Ihr \spacefill\mbox{Hugo}\pend{}\pstart
           \noindent{}\label{T_L01553_1v}\edtext{\textsc{Frl. \textcolor{blue}{W.}{}\ledrightnote{\textcolor{blue}{Lotte Witt}}} iſt für mich eines der unangenehmſten Geſchöpfe der deutſchen Bühnen.}{\lemma{\textnormal{\emph{Frl. … Bühnen.}}}\Cendnote{\textnormal{quer am linken Rand}}}\label{T_L01553_1h}\pend
           \endnumbering\briefempfaengerindex{Schnitzler, Arthur@\textsc{Schnitzler, Arthur}!zzzHofmannsthal, Hugo von@\emph{von Hugo von Hofmannsthal}!1905-09-301@{30. 9. 1905}|)be}\mylabel{h}  \normalsize

\doendnotes{C}
\bigskip
\vfill

\clearpage

\footnotesize

\lohead{\textsc{register}}

% Definiere theindex-Environment komplett neu ohne reledmac
\makeatletter
\renewenvironment{theindex}{%
  \section*{\indexname}%
  \setlength{\parindent}{0pt}%
  \setlength{\parskip}{0pt plus 0.3pt}%
  \let\item\@idxitem
}{%
  \clearpage
}
\makeatother

\IfFileExists{\jobname-pw.ind}{\input{\jobname-pw.ind}}{}

\end{document}

      