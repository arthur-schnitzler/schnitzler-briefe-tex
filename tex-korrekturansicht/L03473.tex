%% latex-korrekturansicht-vorspann.tex
%% Vorspann für die Korrekturansicht.
%% Lädt die gemeinsame Datei latex-vorspann.tex mit gesetztem Schalter.

\newif\ifkorrekturansicht
\korrekturansichttrue

\input{../tex-inputs/latex-vorspann}


\renewcommand{\erwaehntePersonen}{Personen: Olga Schnitzler, Heinrich Schnitzler, Lili Schnitzler}
\renewcommand{\erwaehnteOrte}{Orte: Hotel Sacher, Wien}
\renewcommand{\erwaehnteWerke}{Werke: Das weite Land. Tragikomödie in fünf Akten}
\section[ Paul Goldmann an Arthur Schnitzler, 31. 12. 1910]{Paul Goldmann an Arthur Schnitzler, 31. 12. 1910}
\nopagebreak\mylabel{v}
\rehead{ }\normalsize\beginnumbering\briefempfaengerindex{Schnitzler, Arthur@\textsc{Schnitzler, Arthur}!zzzGoldmann, Paul@\emph{von Paul Goldmann}!1910-12-312@{31. 12. 1910}|(be}
\toendnotes[C]{\smallbreak\pagebreak[2]}\Standort{DLA, A:Schnitzler, HS.NZ85.1.3175.}
\physDesc{Brief, 1 Blatt, 1 Seite, 245 Zeichen
\newline{}Handschrift: schwarze Tinte, deutsche Kurrent}\toendnotes[C]{\smallbreak}
\pstart
           \noindent{}\centering{}{\pb}\textcolor{gray}{\textbf{\textcolor{pink}{Hotel Sacher}{}\ledrightnote{\textcolor{pink}{Hotel Sacher}}}}\pend
           
\pstart
           \noindent{}\textcolor{gray}{\textbf{Telefon Nr 8008.}}\hfill \textcolor{gray}{\textbf{\textcolor{pink}{Wien I.}{}\ledrightnote{\textcolor{pink}{Wien}}}}\pend
           
\pstart
           31. 12. 10.\pend
           
\pstart{}Lieber Freund,\pend
\pstart
           Ich danke Dir herzlichſt für die Überſendung der Exemplars von Deinem \textcolor{green}{Drama}{}\ledrightnote{{$\rightarrow$}\textcolor{green}{Das weite Land. Tragikomödie in fünf Akten}}, habe mich aufrichtig darüber gefreut
               u. wünſche Dir, Deiner \textcolor{blue}{Frau}{}\ledrightnote{{$\rightarrow$}\textcolor{blue}{Olga Schnitzler}}
               u. Deinen \textcolor{blue}{Kinder}{}\ledrightnote{{$\rightarrow$}\textcolor{blue}{Heinrich Schnitzler}{\newline}{$\rightarrow$}\textcolor{blue}{Lili Schnitzler}}n
               ein glückliches neues Jahr! – Mit vielen Grüßen {\\}Dein {\\}\spacefill\mbox{Paul Goldmann.}\pend
           \endnumbering\briefempfaengerindex{Schnitzler, Arthur@\textsc{Schnitzler, Arthur}!zzzGoldmann, Paul@\emph{von Paul Goldmann}!1910-12-312@{31. 12. 1910}|)be}\mylabel{h}
\begin{anhang}
\end{anhang}\normalsize

\doendnotes{C}
\bigskip
\vfill

\clearpage

\footnotesize

\lohead{\textsc{register}}

% Definiere theindex-Environment komplett neu ohne reledmac
\makeatletter
\renewenvironment{theindex}{%
  \section*{\indexname}%
  \setlength{\parindent}{0pt}%
  \setlength{\parskip}{0pt plus 0.3pt}%
  \let\item\@idxitem
}{%
  \clearpage
}
\makeatother

\IfFileExists{\jobname-pw.ind}{\input{\jobname-pw.ind}}{}

\end{document}

      