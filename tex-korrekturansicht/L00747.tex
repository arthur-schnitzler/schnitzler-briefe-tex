%% latex-korrekturansicht-vorspann.tex
%% Vorspann für die Korrekturansicht.
%% Lädt die gemeinsame Datei latex-vorspann.tex mit gesetztem Schalter.

\newif\ifkorrekturansicht
\korrekturansichttrue

\input{../tex-inputs/latex-vorspann}


               \section[Arthur Schnitzler an Hermann Bahr, 8. 12. 1897]{ Arthur Schnitzler an Hermann Bahr, 8. 12. 1897}\nopagebreak\mylabel{v}\rehead{ }\normalsize\beginnumbering\briefempfaengerindex{Bahr, Hermann@\textsc{Bahr, Hermann}!zzzSchnitzler, Arthur@\emph{von Arthur Schnitzler}!1897-12-081@{8. 12. 1897}|(be} \toendnotes[C]{\smallbreak\pagebreak[2]} \Standort{TMW, HS AM 23332 Ba.}
\physDesc{Brief, 1 Blatt, 2 Seiten
\newline{}Handschrift: schwarze Tinte, deutsche Kurrent\newline{}Ordnung: 1) Lochung 2) mit Bleistift von unbekannter Hand datiert: »8. XII. 97«}\buchAbdrucke{\weitereDrucke{1) \emph{8. 12. 1897.} In: Arthur Schnitzler: \emph{The Letters of Arthur Schnitzler to Hermann Bahr}. Edited, annotated, and with an introduction, by Donald G.
                        Daviau. Chapel Hill: \emph{The University of North Carolina Press} 1978, S. 63 (University of North Carolina studies in the Germanic languages
                        and literatures, 89).} \weitereDrucke{2) Hermann Bahr, Arthur Schnitzler: \emph{Briefwechsel, Aufzeichnungen, Dokumente (1891–1931)}. Hg. Kurt Ifkovits und Martin Anton Müller. Göttingen: \emph{Wallstein} 2018, S. 159.} }\toendnotes[C]{\smallbreak}\pstart
           \noindent{}{\pb}Lieber Hermann, ich werde erſucht, dich zu bitten, auch dein
               werthvolles Autogramm auf dieſe Karte zu ſetzen. Für wen – weiſs ich nicht. Angeblich
               ſoll es eine »reizende \textcolor{blue}{Autographensa{\geminationm}lerin}{}\ledrightnote{→\textcolor{blue}{?? [Autografensammlerin]}}« ſein. Schicke mir die
               Karte freundlichſt zurück.\pend
           \pstart
           Sag mir auch bei dieſer Gelegenheit, {\pb}wie ich mein \textcolor{green}{\textcolor{green}{\textsc{Cosmopolis}-Heft}{}\ledrightnote{\textcolor{green}{Cosmopolis}}}{}\ledrightnote{→\textcolor{green}{Die Toten schweigen}}, mein einziges, von der Polizei zurück beko{\geminationm}en
               kann? \pend
           \pstart Herzlich dein \spacefill\mbox{ArthSchn}\pend{}\pstart
           8. 12. 97.\pend
           \endnumbering\briefempfaengerindex{Bahr, Hermann@\textsc{Bahr, Hermann}!zzzSchnitzler, Arthur@\emph{von Arthur Schnitzler}!1897-12-081@{8. 12. 1897}|)be}\mylabel{h}  \normalsize

\doendnotes{C}
\bigskip
\vfill

\clearpage

\footnotesize

\lohead{\textsc{register}}

% Definiere theindex-Environment komplett neu ohne reledmac
\makeatletter
\renewenvironment{theindex}{%
  \section*{\indexname}%
  \setlength{\parindent}{0pt}%
  \setlength{\parskip}{0pt plus 0.3pt}%
  \let\item\@idxitem
}{%
  \clearpage
}
\makeatother

\IfFileExists{\jobname-pw.ind}{\input{\jobname-pw.ind}}{}

\end{document}

      