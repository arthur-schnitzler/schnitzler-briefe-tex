%% latex-korrekturansicht-vorspann.tex
%% Vorspann für die Korrekturansicht.
%% Lädt die gemeinsame Datei latex-vorspann.tex mit gesetztem Schalter.

\newif\ifkorrekturansicht
\korrekturansichttrue

\input{../tex-inputs/latex-vorspann}


               \section[Richard Beer-Hofmann an Arthur Schnitzler, {[}zwischen 8. und 28.? 2. 1892{]}]{ Richard Beer-Hofmann an Arthur Schnitzler, {[}zwischen 8. und
               28.? 2. 1892{]}}\nopagebreak\mylabel{v}\rehead{ }\normalsize\beginnumbering\briefempfaengerindex{Schnitzler, Arthur@\textsc{Schnitzler, Arthur}!zzzBeer-Hofmann, Richard@\emph{von Richard Beer-Hofmann}!1892-02-081@{{[}zwischen 8. und
                  28.? 2. 1892{]}}|(be} \toendnotes[C]{\smallbreak\pagebreak[2]} \Standort{CUL, Schnitzler, B 8.}
\physDesc{Brief, 1 Blatt (Briefpapier mit Trauerrand), 3 Seiten
\newline{}Handschrift: blauer Buntstift, lateinische Kurrent
\newline{}Schnitzler: mit Bleistift datiert: »Febe 92« und nummeriert: »6« }\buchAbdrucke{\weitereDrucke{Arthur Schnitzler, Richard Beer-Hofmann: \emph{Briefwechsel 1891–1931}. Hg. Konstanze Fliedl. Wien, Zürich: \emph{Europaverlag} 1992, S. 33.} }\toendnotes[C]{\smallbreak}\pstart
           \noindent{}{\pb}\textcolor{gray}{\textbf{RB}}\pend
           \pstart
           Soviel ich weiß sollt Ihr zu mir ko{\geminationm}en; wurde \label{K_L00071_1v}\edtext{gestern}{\lemma{\textnormal{\emph{gestern}}}\Cendnote{\textnormal{Die durch \textcolor{blue}{Schnitzler}
                  vorgenommene Datierung in den Februar 1892 (der auch das in dieser
                  Zeit verwendete Briefpapier mit Trauerrand entspricht) ist nicht genauer
                  einzugrenzen. Einzig der Monatsanfang scheint auszufallen, da hier die anderen
                  brieflichen Zeugnisse dieses Dokument nicht ohne Verrenkungen eingliedern
                  lassen.}}}\label{K_L00071_1h}{ }\uline{ausdrücklich} besprochen; ich warte seit {\pb}4 Uhr; \textcolor{blue}{Dörmann}{}\ledrightnote{\textcolor{blue}{Felix Dörmann}} ist bei mir; \strikeout{Ger} zuerst werden wir jausen, und \uline{dann vielleicht} ko{\geminationm}en.\pend
           \pstart
           {\pb}Eure Rücksichtslosigkeit ist
               unverantwortlich\pend
           \pstart \spacefill\mbox{R.}\pend{}\endnumbering\briefempfaengerindex{Schnitzler, Arthur@\textsc{Schnitzler, Arthur}!zzzBeer-Hofmann, Richard@\emph{von Richard Beer-Hofmann}!1892-02-081@{{[}zwischen 8. und
                  28.? 2. 1892{]}}|)be}\mylabel{h}  \normalsize

\doendnotes{C}
\bigskip
\vfill

\clearpage

\footnotesize

\lohead{\textsc{register}}

% Definiere theindex-Environment komplett neu ohne reledmac
\makeatletter
\renewenvironment{theindex}{%
  \section*{\indexname}%
  \setlength{\parindent}{0pt}%
  \setlength{\parskip}{0pt plus 0.3pt}%
  \let\item\@idxitem
}{%
  \clearpage
}
\makeatother

\IfFileExists{\jobname-pw.ind}{\input{\jobname-pw.ind}}{}

\end{document}

      