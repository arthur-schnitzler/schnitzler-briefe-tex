%% latex-korrekturansicht-vorspann.tex
%% Vorspann für die Korrekturansicht.
%% Lädt die gemeinsame Datei latex-vorspann.tex mit gesetztem Schalter.

\newif\ifkorrekturansicht
\korrekturansichttrue

\input{../tex-inputs/latex-vorspann}


               \section[Arthur Schnitzler an Richard Beer-Hofmann, 17. 7. 1896]{ Arthur Schnitzler an Richard Beer-Hofmann,
               17. 7. 1896}\nopagebreak\mylabel{v}\rehead{ }\normalsize\beginnumbering\briefempfaengerindex{Beer-Hofmann, Richard@\textsc{Beer-Hofmann, Richard}!zzzSchnitzler, Arthur@\emph{von Arthur Schnitzler}!1896-07-181@{18. 7. 1896}|(be} \toendnotes[C]{\smallbreak\pagebreak[2]} \Standort{YCGL, MSS 31.}
\physDesc{Postkarte
\newline{}Handschrift: Bleistift, deutsche Kurrent\newline{}Versand: 1) Stempel: »\nobreak{}\oindex{Tromsø@\textbf{Tromsø}, \emph{http://www.geonames.org/ontologyP.PPLA}|pwk}Tromsø, 18. VII. 96\nobreak{}«.  2) Stempel: »\nobreak{}\oindex{Kopenhagen@\textbf{Kopenhagen}, \emph{Besiedelter Ort (A.BSO)}|pwk}Kjobenhavn, 23. 7. 96, 50M8\nobreak{}«. }\pstart{}{\pb}\textsc{Dr. Richard
                     Beer-Hofmann}\pend{}\pstart{}\textcolor{pink}{\textsc{Kopenhagen}}{}\ledrightnote{\textcolor{pink}{Kopenhagen}}\pend{}\pstart{}\textsc{post rest}\pend{}\pstart{}\textsc{\textcolor{pink}{Dänemark}{}\ledrightnote{\textcolor{pink}{Dänemark}}}\pend{}{\bigskip}\pstart
           \noindent{}{\pb}Mein lieber Richard, ich
               nehme an Sie beko{\geminationm}en dieſe Karte am 24. Da
               ſchreiben Sie mir \uline{gleich} nach \textcolor{pink}{\textsc{Stockholm}}{}\ledrightnote{\textcolor{pink}{Stockholm}}. (\textsc{post rest} natürlich) Ich werde wahrſcheinlich
                  24. 25. 26. in \textcolor{pink}{\textsc{Krist.}}{}\ledrightnote{\textcolor{pink}{Oslo}}{ }ſein, dann bis etwa
                  30 od 31{ }\textcolor{pink}{\textsc{Stockholm}}{}\ledrightnote{\textcolor{pink}{Stockholm}}. Es wäre
               wunderſchön, wenn Sie doch wenigſtens nach \textcolor{pink}{\textsc{Stockh}}{}\ledrightnote{\textcolor{pink}{Stockholm}}. hinüberko{\geminationm}en \substVorne{}\textsuperscript{w}\substDazwischen{}mö\substHinten{}chten. Oder nach \textcolor{pink}{\textsc{Goetheborg}}{}\ledrightnote{\textcolor{pink}{Göteborg}} mir entgegen. Überlegen Sie ſichs. Bitte laſſen
               Sie mich nicht ohne Nachricht. –\pend
           \pstart
           Herzlich der Ihre{\\[\baselineskip]}\spacefill\mbox{ArthSch}\pend
           \leftskip=0em{}\pstart
           \noindent{}an Bord der \textsc{Sig Jarl}{ }17/7 96.\pend
           \endnumbering\briefempfaengerindex{Beer-Hofmann, Richard@\textsc{Beer-Hofmann, Richard}!zzzSchnitzler, Arthur@\emph{von Arthur Schnitzler}!1896-07-181@{18. 7. 1896}|)be}\mylabel{h}  \normalsize

\doendnotes{C}
\bigskip
\vfill

\clearpage

\footnotesize

\lohead{\textsc{register}}

% Definiere theindex-Environment komplett neu ohne reledmac
\makeatletter
\renewenvironment{theindex}{%
  \section*{\indexname}%
  \setlength{\parindent}{0pt}%
  \setlength{\parskip}{0pt plus 0.3pt}%
  \let\item\@idxitem
}{%
  \clearpage
}
\makeatother

\IfFileExists{\jobname-pw.ind}{\input{\jobname-pw.ind}}{}

\end{document}

      