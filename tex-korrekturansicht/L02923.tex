%% latex-korrekturansicht-vorspann.tex
%% Vorspann für die Korrekturansicht.
%% Lädt die gemeinsame Datei latex-vorspann.tex mit gesetztem Schalter.

\newif\ifkorrekturansicht
\korrekturansichttrue

\input{../tex-inputs/latex-vorspann}


         
         \renewcommand{\erwaehntePersonen}{Personen: Richard Beer-Hofmann, Ludwig Fulda, Alfred Kerr, Leo Van-Jung}
         \renewcommand{\erwaehnteOrte}{Orte: Berlin, China, Dessauer Straße, Innsbruck, Reichenau an der Rax}
         \renewcommand{\erwaehnteWerke}{}
               \section[ Paul Goldmann an Arthur Schnitzler, 5. 7. {[}1900{]}]{Paul Goldmann an Arthur Schnitzler, 5. 7. {[}1900{]}}\nopagebreak\mylabel{v}\rehead{ }\normalsize\beginnumbering\briefempfaengerindex{Schnitzler, Arthur@\textsc{Schnitzler, Arthur}!zzzGoldmann, Paul@\emph{von Paul Goldmann}!1900-07-051@{5. 7. {[}1900{]}}|(be} \toendnotes[C]{\smallbreak\pagebreak[2]} \Standort{DLA, A:Schnitzler, HS.NZ85.1.3170.}
\physDesc{Brief, 1 Blatt, 2 Seiten
\newline{}Handschrift: blaue Tinte, deutsche Kurrent
\newline{}Schnitzler: mit Bleistift das Jahr »{[}1{]}900« vermerkt }\toendnotes[C]{\smallbreak}\pstart
           \noindent{}{\pb}\textcolor{pink}{\textcolor{gray}{\textbf{DESSAUERSTRASSE 19}}}{}\ledrightnote{\textcolor{pink}{Dessauer Straße}}\hfill \textcolor{pink}{Berlin}{}\ledrightnote{\textcolor{pink}{Berlin}}, 5. Juli.\pend
           \pstart\center{}Mein lieber Freund,\pend\pstart
           Mit dem \label{K_L02923-1v}\edtext{Rendevous in \textcolor{pink}{Innsbruck}{}\ledrightnote{\textcolor{pink}{Innsbruck}} Mitte Auguſt}{\lemma{\textnormal{\emph{Rendevous … Auguſt}}}\Cendnote{\textnormal{siehe Paul Goldmann an Arthur Schnitzler, 16. 6. [1900] und A. S.: \emph{Tagebuch}, 16. 8. 1900}}}\label{K_L02923-1h} behufs Antritts der Fußparthie wäre ich einverſtanden. Freilich wird es durch
               die \label{K_L02923-3v}\edtext{\textcolor{pink}{chin}{}\ledrightnote{{$\rightarrow$}\textcolor{pink}{China}}eſiſchen Ereigniſſe}{\lemma{\textnormal{\emph{chineſiſchen Ereigniſſe}}}\Cendnote{\textnormal{Bezug auf den sich im Sommer 1900 zuspitzenden \textcolor{pink}{chin}esischen Boxeraufstand}}}\label{K_L02923-3h} immer fraglicher, ob ich
               überhaupt fort kann. Es wäre ſehr ſchön, wenn \textsc{\textcolor{blue}{Leo}{}\ledrightnote{\textcolor{blue}{Leo Van-Jung}}} und \textsc{\textcolor{blue}{Richard}{}\ledrightnote{\textcolor{blue}{Richard Beer-Hofmann}}} mit kämen. Wohin wollen wir wandern? Und wie lange, glaubſt Du, wird das
               dauern?\pend
           \pstart
           Wie geht es Dir? Bitte, laß’ bald wieder von Dir {\pb}hören. Haſt Du von \label{K_L02923-2v}\edtext{\textsc{\textcolor{blue}{Fulda}{}\ledrightnote{\textcolor{blue}{Ludwig Fulda}}}}{\lemma{\textnormal{\emph{Fulda}}}\Cendnote{\textnormal{vermutlich Bezug auf gemeinsame
                  Sommerpläne (siehe A. S.: \emph{Tagebuch}, 27. 8. 1900,
                     28. 8. 1900 und
                     29. 8. 1900)}}}\label{K_L02923-2h} ſchon Beſcheid?\pend
           \pstart
           \textsc{\textcolor{blue}{Kerr}{}\ledrightnote{\textcolor{blue}{Alfred Kerr}}} dürfte Mitte Auguſt auch mithalten.\pend
           \pstart
           Viele treue Grüße! {\\[\baselineskip]}Dein {\\[\baselineskip]}\spacefill\mbox{Paul Goldmnn.}\pend
           \leftskip=0em{}\endnumbering\briefempfaengerindex{Schnitzler, Arthur@\textsc{Schnitzler, Arthur}!zzzGoldmann, Paul@\emph{von Paul Goldmann}!1900-07-051@{5. 7. {[}1900{]}}|)be}\mylabel{h}  \normalsize

\doendnotes{C}
\bigskip
\vfill

\clearpage

\footnotesize

\lohead{\textsc{register}}

% Definiere theindex-Environment komplett neu ohne reledmac
\makeatletter
\renewenvironment{theindex}{%
  \section*{\indexname}%
  \setlength{\parindent}{0pt}%
  \setlength{\parskip}{0pt plus 0.3pt}%
  \let\item\@idxitem
}{%
  \clearpage
}
\makeatother

\IfFileExists{\jobname-pw.ind}{\input{\jobname-pw.ind}}{}

\end{document}

      