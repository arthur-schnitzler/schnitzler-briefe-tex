%% latex-korrekturansicht-vorspann.tex
%% Vorspann für die Korrekturansicht.
%% Lädt die gemeinsame Datei latex-vorspann.tex mit gesetztem Schalter.

\newif\ifkorrekturansicht
\korrekturansichttrue

\input{../tex-inputs/latex-vorspann}


\renewcommand{\erwaehntePersonen}{Personen: Jakob Wassermann, Stefan Zweig}
\renewcommand{\erwaehnteOrte}{Orte: Paschinger Schlössl, Salzburg}
\renewcommand{\erwaehnteWerke}{Werke: ?? [Novelle, in der ein kleiner Geldbetrag ein Schicksal entscheidet], Die neue Rundschau, Fräulein Else, Ulrike Woytich. Roman}
\section[Stefan Zweig an Arthur Schnitzler, 4. 11. 1924]{Stefan Zweig an Arthur Schnitzler, 4. 11. 1924}
\nopagebreak\mylabel{v}
\rehead{ }\normalsize\beginnumbering\briefempfaengerindex{Schnitzler, Arthur@\textsc{Schnitzler, Arthur}!zzzZweig, Stefan@\emph{von Stefan Zweig}!1924-11-041@{4. 11. 1924}|(be}
\toendnotes[C]{\smallbreak\pagebreak[2]}\Standort{CUL, Schnitzler, B 118.}
\physDesc{Brief, 1 Blatt, 2 Seiten, 1369 Zeichen
\newline{}Handschrift: blaue Tinte, lateinische Kurrent
\newline{}Schnitzler: 1) mit Bleistift »\textsc{Zweig}«  2) mit rotem Buntstift drei Unterstreichungen}\toendnotes[C]{\smallbreak}
\pstart
           {\pb}\textcolor{gray}{\textbf{SZ}}\hfill \textcolor{gray}{\textbf{\textcolor{pink}{KAPUZINERBERG 5}{}\ledrightnote{\textcolor{pink}{Paschinger Schlössl}}}}\pend
           
\pstart
           \raggedleft{}\textcolor{gray}{\textbf{\textcolor{pink}{SALZBURG}{}\ledrightnote{\textcolor{pink}{Salzburg}}}}{ }4. Nov. 1924\pend
           
\pstart
           Lieber verehrter Herr Doktor, ich bin schwer in Arbeit – aber ich
               muss mich für eine Minute unterbrechen, um Ihnen zu sagen, \uline{wie ausserordentlich} ich ihre \label{K_L03669-1v}\edtext{\textcolor{green}{Novelle}{}\ledrightnote{{$\rightarrow$}\textcolor{green}{Fräulein Else}}}{\lemma{\textnormal{\emph{Novelle}}}\Cendnote{\textnormal{\emph{\textcolor{green}{Fräulein Else}}. Novelle von \textcolor{blue}{Arthur
                        Schnitzler}. In: \emph{\textcolor{green}{Die neue
                     Rundschau}}, Jg. 35, H. 10, Oktober 1924,
                     S. 993–1051.}}}\label{K_L03669-1h} in der »\textcolor{green}{Neuen
                  Rundschau}{}\ledrightnote{\textcolor{green}{Die neue Rundschau}}« finde: eine \label{K_L03669-2v}\edtext{trouvaille}{\lemma{\textnormal{\emph{trouvaille}}}\Cendnote{\textnormal{französisch: wertvolles Fundstück, Schmuckstück}}}\label{K_L03669-2h} in der Technik der Novelle, spannend,
               aufwühlend, ganz ins Tragische aus kleinen Präludium aufsteigend. Ich wüsste kein
               Wort darin zu ändern – einzig für die Buchausgabe eine \uline{Zahl}. 50 000 Gulden 100.000 Friedenskronen – das war eine Summe, die ein
               Rotschild kaum seinem Brüder \label{K_L03669-3v}\edtext{a fond
                  perdu}{\lemma{\textnormal{\emph{a fond
                  perdu}}}\Cendnote{\textnormal{französisch: ohne
                  Rückzahlungsverpflichtung}}}\label{K_L03669-3h} lieh. Gieng es Ihnen nicht da wie \textcolor{blue}{Jacob Wassermann}{}\ledrightnote{\textcolor{blue}{Jakob Wassermann}} in der \textcolor{green}{Ulrike Woytech}{}\ledrightnote{\textcolor{green}{Ulrike Woytich. Roman}}, dass unsere Erinnerungsgefühle an Geld auch
               schon inflationiert sind? Gerade weil es ein entfernter Bekannter aus dem \uline{Mittelstand} ist, schien mir die Summe grotest hoch –
               ich verstehe, dass {\pb}Sie für die seelische
               Motivation eine \uline{hohe} Summe brauchten – uns klingt
               10000 Kronen heute wie ein »Fetzen« war aber doch schon als Leihgeld unerhört viel.
               Ich kam auf diesen Kleinkram zu reden, weil ich selbst bei einer (unveröffentlichten)
                  \label{K_L03669-4v}\edtext{\textcolor{green}{Arbeit}{}\ledrightnote{{$\rightarrow$}\textcolor{green}{?? [Novelle, in der ein kleiner Geldbetrag ein Schicksal entscheidet]}}}{\lemma{\textnormal{\emph{Arbeit}}}\Cendnote{\textnormal{nicht
                  identifiziert}}}\label{K_L03669-4h} den Widersinn spürte, zehn Kronen zu einer Entscheidung über
               ein Schicksal zu machen: aber es gab damals Katastrofen wegen fünfzig Heller. Wo ist
               die Zeit! Wie habe ich mich gefreut an Ihrem \textcolor{green}{Werk}{}\ledrightnote{{$\rightarrow$}\textcolor{green}{Fräulein Else}}, wie an der Überraschung, die mir trotz aller alten
               Liebe, alles guten Vertrauens, dieser Aufstieg war!\pend
           
\pstart
           Seien Sie innigst beglückwünscht von Ihrem ergebenen{\\[\baselineskip]}\spacefill\mbox{Stefan Zweig}\pend
           \leftskip=0em{}\endnumbering\briefempfaengerindex{Schnitzler, Arthur@\textsc{Schnitzler, Arthur}!zzzZweig, Stefan@\emph{von Stefan Zweig}!1924-11-041@{4. 11. 1924}|)be}\mylabel{h}
\begin{anhang}
\end{anhang}\normalsize

\doendnotes{C}
\bigskip
\vfill

\clearpage

\footnotesize

\lohead{\textsc{register}}

% Definiere theindex-Environment komplett neu ohne reledmac
\makeatletter
\renewenvironment{theindex}{%
  \section*{\indexname}%
  \setlength{\parindent}{0pt}%
  \setlength{\parskip}{0pt plus 0.3pt}%
  \let\item\@idxitem
}{%
  \clearpage
}
\makeatother

\IfFileExists{\jobname-pw.ind}{\input{\jobname-pw.ind}}{}

\end{document}

      