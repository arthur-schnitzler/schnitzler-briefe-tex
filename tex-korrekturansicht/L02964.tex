%% latex-korrekturansicht-vorspann.tex
%% Vorspann für die Korrekturansicht.
%% Lädt die gemeinsame Datei latex-vorspann.tex mit gesetztem Schalter.

\newif\ifkorrekturansicht
\korrekturansichttrue

\input{../tex-inputs/latex-vorspann}


\renewcommand{\erwaehntePersonen}{Personen: Paul Goldmann, Felix Salten}
\renewcommand{\erwaehnteOrte}{Orte: Café Griensteidl, Café Pucher, England, Forest Hill, Hörlgasse, IX., Alsergrund, London, Paris, Wien}
\renewcommand{\erwaehnteWerke}{}
\section[ Arthur Schnitzler an Felix Salten, 29. 5. 1897]{Arthur Schnitzler an Felix Salten, 29. 5. 1897}
\nopagebreak\mylabel{v}
\rehead{ }\normalsize\beginnumbering\briefempfaengerindex{Salten, Felix@\textsc{Salten, Felix}!zzzSchnitzler, Arthur@\emph{von Arthur Schnitzler}!1897-05-292@{29. 5. 1897}|(be}
\toendnotes[C]{\smallbreak\pagebreak[2]}\Standort{Wienbibliothek im Rathaus, ZPH 1681, 2.1.516.}
\physDesc{Postkarte, 554 Zeichen
\newline{}Handschrift: 1) schwarze Tinte, deutsche Kurrent\hspace{1em}2) schwarze Tinte, lateinische Kurrent (\noindent{}Adresse)\hspace{1em}
\newline{}Versand: 1) Stempel: »\nobreak{}\oindex{Forest Hill@\textbf{Forest Hill}, \emph{Bezirk (A.BZK)}|pwk}Forest-Hill S.E., MY 29 97\nobreak{}«.   2) Stempel: »\nobreak{}\oindex{IX., Alsergrund@\textbf{IX., Alsergrund}, \emph{A.ADM3}|pwk}Wien 9/1, 1/6. 9\textcolor{gray}{7}, 8–9½ V., Bestellt\nobreak{}«. 
\newline{}Ordnung: mit Bleistift von unbekannter Hand Nummerierung der Blätter des Konvoluts:
                                    »75« }\toendnotes[C]{\smallbreak}\pstart{}{\pb}Austria\pend{}\pstart{}Mr. Felix Salten\pend{}\pstart{}\textcolor{pink}{Wien}{}\ledrightnote{\textcolor{pink}{Wien}}\pend{}\pstart{}\textcolor{pink}{IX. Hoerlgasse 16}{}\ledrightnote{\textcolor{pink}{Hörlgasse}}\pend{}
{\bigskip}
\pstart
           \noindent{}{\pb}Lieber Freund, Ihr lieber \label{K_L02964-1v}\edtext{Brief, den ich nicht mehr ſo ausführlich beantworten kann,
               als ich ſollte u möchte, iſt mir \textcolor{pink}{hieher}{}\ledrightnote{{$\rightarrow$}\textcolor{pink}{London}}}{\lemma{\textnormal{\emph{Brief, … hieher}}}\Cendnote{\textnormal{\textcolor{blue}{Schnitzler} reiste am 24. 5. 1897 von \textcolor{pink}{Paris} weiter nach \textcolor{pink}{London}. \textcolor{blue}{Goldmann}
                  sandte ihm am 26. 5. [1897]
                  einen Brief nach, vermutlich Felix Salten an Arthur Schnitzler, 23. 5. 1897.}}}\label{K_L02964-1h} nachgeſchickt worden. Es wird ſich ja ſehr bald in \textcolor{pink}{Wien}{}\ledrightnote{\textcolor{pink}{Wien}} zu allerlei Ausſprache Gelegenheit \introOben{}er\introOben{}geben. Werde hoffentlich \label{K_L02964-2v}\edtext{Mittwoch}{\lemma{\textnormal{\emph{Mittwoch}}}\Cendnote{\textnormal{\textcolor{blue}{Schnitzler} kehrte am Mittwoch, dem 2. 6. 1897, nach \textcolor{pink}{Wien} zurück.}}}\label{K_L02964-2h}{ }Abd{ }\textsc{resp.}{ }Do{\geminationn}erſtag in \textcolor{pink}{Wien}{}\ledrightnote{\textcolor{pink}{Wien}} ſein. Finde vielleicht ein Wort von Ihnen.–
               Jetzt eben hab ich mir ein Rad beſtellt – glauben Sie mir, daſs es echt \textcolor{pink}{engliſch}{}\ledrightnote{{$\rightarrow$}\textcolor{pink}{England}} ſein wird? – Ich möchte
                  \label{K_L02964-3v}\edtext{\textcolor{pink}{Pucher}{}\ledrightnote{\textcolor{pink}{Café Pucher}}}{\lemma{\textnormal{\emph{Pucher}}}\Cendnote{\textnormal{Die Stelle bleibt weitgehend kryptisch.
                  Die wahrscheinlichste Erklärung besitzt aber einige Relevanz. Am 21. 1. 1897 hatte das
                     \textcolor{pink}{Café Griensteidl} geschlossen, folglich
                  musste in Folge ein neues Stammkaffeehaus gefunden werden. Eventuell war dies in
                  den ersten Tagen bis zu \textcolor{blue}{Schnitzler}s Abreise
                  das \textcolor{pink}{Café Pucher}?}}}\label{K_L02964-3h} womöglich ganz
               aufgeben.– Auf frohes Wiederſehen. Herzlich Ihr\pend
           \pstart \spacefill\mbox{Arthur Sch}\pend{}
\pstart
           \textcolor{pink}{London}{}\ledrightnote{\textcolor{pink}{London}}{ }29. 5. 97.\pend
           \endnumbering\briefempfaengerindex{Salten, Felix@\textsc{Salten, Felix}!zzzSchnitzler, Arthur@\emph{von Arthur Schnitzler}!1897-05-292@{29. 5. 1897}|)be}\mylabel{h}  \normalsize

\doendnotes{C}
\bigskip
\vfill

\clearpage

\footnotesize

\lohead{\textsc{register}}

% Definiere theindex-Environment komplett neu ohne reledmac
\makeatletter
\renewenvironment{theindex}{%
  \section*{\indexname}%
  \setlength{\parindent}{0pt}%
  \setlength{\parskip}{0pt plus 0.3pt}%
  \let\item\@idxitem
}{%
  \clearpage
}
\makeatother

\IfFileExists{\jobname-pw.ind}{\input{\jobname-pw.ind}}{}

\end{document}

      