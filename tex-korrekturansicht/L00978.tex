%% latex-korrekturansicht-vorspann.tex
%% Vorspann für die Korrekturansicht.
%% Lädt die gemeinsame Datei latex-vorspann.tex mit gesetztem Schalter.

\newif\ifkorrekturansicht
\korrekturansichttrue

\input{../tex-inputs/latex-vorspann}


               \section[Richard Beer-Hofmann und Hugo von Hofmannsthal an Arthur Schnitzler, 19. 9. 1899]{ Richard Beer-Hofmann und Hugo von Hofmannsthal an Arthur Schnitzler,
               19. 9. 1899}\nopagebreak\mylabel{v}\rehead{ }\normalsize\beginnumbering\briefempfaengerindex{Schnitzler, Arthur@\textsc{Schnitzler, Arthur}!zzzHofmannsthal, Hugo von@\emph{von Hugo von Hofmannsthal}!1899-09-192@{19. 9. 1899}|(be}\briefempfaengerindex{Schnitzler, Arthur@\textsc{Schnitzler, Arthur}!zzzBeer-Hofmann, Richard@\emph{von Richard Beer-Hofmann}!1899-09-192@{19. 9. 1899}|(be} \toendnotes[C]{\smallbreak\pagebreak[2]} \Standort{CUL, Schnitzler, B 8.}
\physDesc{Bildpostkarte
\newline{}Handschrift Hugo von Hofmannsthal: schwarze Tinte\newline{}Handschrift Richard Beer-Hofmann: schwarze Tinte, lateinische Kurrent\newline{}Versand: 1) Stempel: »\nobreak{}20. 9. 99\nobreak{}«.  2) Stempel: »\nobreak{}\oindex{Frankfurt am Main@\textbf{Frankfurt am Main}, \emph{Besiedelter Ort (A.BSO)}|pwk}Frankfurt (Main), 22. 9. 99, 7–8V\nobreak{}«. \newline{}Ordnung: mit Bleistift von unbekannter Hand nummeriert:
                                    »145« }\buchAbdrucke{\weitereDrucke{Arthur Schnitzler, Richard Beer-Hofmann: \emph{Briefwechsel 1891–1931}. Hg. Konstanze Fliedl. Wien, Zürich: \emph{Europaverlag} 1992, S. 137.} }\toendnotes[C]{\smallbreak}\pstart{}{\pb}D\textsuperscript{r}
                  Arthur Schnitzler\pend{}\pstart{}\textcolor{pink}{Frankfurt a. Main}{}\ledrightnote{\textcolor{pink}{Frankfurt am Main}}\pend{}\pstart{}Poste restante\pend{}{\bigskip}\pstart
           \noindent{}\centering{}\textcolor{gray}{\textbf{{\pb}Künstler-Postkarte.}}\pend
           \pstart
           \noindent{}\centering{}\textcolor{gray}{\textbf{\textcolor{blue}{E. Klingebeil}{}\ledrightnote{\textcolor{blue}{Eduard Klingebeil}}: \textcolor{green}{Zweierlei Pegasus}{}\ledrightnote{\textcolor{green}{Zweierlei Pegasus}}.}}\pend
           \pstart
           \raggedleft{}\textcolor{pink}{Vahrn}{}\ledrightnote{\textcolor{pink}{Vahrn}}{ }19/IX 1899\pend
           \pstart
           \label{K_L00978_1v}\edtext{\textcolor{blue}{Adolf Pichler}{}\ledrightnote{\textcolor{blue}{Adolf Pichler}}}{\lemma{\textnormal{\emph{Adolf Pichler}}}\Cendnote{\textnormal{Die Gegenüberstellung der beiden
                  Schriftsteller \textcolor{blue}{Schnitzler} und \textcolor{blue}{Adolf Pichler} möchte nicht nur durch die Zuordnung zu den zwei auf
                  der Karte dargestellten Poeten – der eine reitet mit einer Lyra auf einem Pegasus zum
                  Himmel, der andere mit einem Leierkasten und einer Tänzerin auf einem Schwein
                  durch den Dreck – witzig sein, sondern zieht den Humor auch aus dem
                  Altersunterschied: \textcolor{blue}{Pichler} wurde am
                     4. 9. 1899 achtzig.}}}\label{K_L00978_1h}\hspace*{2.5em} Arthur S.\pend
           \pstart
           Dies wünschen Ihnen\pend
           \pstart
           \spacefill\mbox{Richard}{\\[\baselineskip]}\spacefill\mbox{{[}hs. Hofmannsthal:{]} Hugo}\pend
           \leftskip=0em{}\endnumbering\briefempfaengerindex{Schnitzler, Arthur@\textsc{Schnitzler, Arthur}!zzzHofmannsthal, Hugo von@\emph{von Hugo von Hofmannsthal}!1899-09-192@{19. 9. 1899}|)be}\briefempfaengerindex{Schnitzler, Arthur@\textsc{Schnitzler, Arthur}!zzzBeer-Hofmann, Richard@\emph{von Richard Beer-Hofmann}!1899-09-192@{19. 9. 1899}|)be}\mylabel{h}  \normalsize

\doendnotes{C}
\bigskip
\vfill

\clearpage

\footnotesize

\lohead{\textsc{register}}

% Definiere theindex-Environment komplett neu ohne reledmac
\makeatletter
\renewenvironment{theindex}{%
  \section*{\indexname}%
  \setlength{\parindent}{0pt}%
  \setlength{\parskip}{0pt plus 0.3pt}%
  \let\item\@idxitem
}{%
  \clearpage
}
\makeatother

\IfFileExists{\jobname-pw.ind}{\input{\jobname-pw.ind}}{}

\end{document}

      