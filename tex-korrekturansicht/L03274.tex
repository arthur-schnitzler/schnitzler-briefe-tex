%% latex-korrekturansicht-vorspann.tex
%% Vorspann für die Korrekturansicht.
%% Lädt die gemeinsame Datei latex-vorspann.tex mit gesetztem Schalter.

\newif\ifkorrekturansicht
\korrekturansichttrue

\input{../tex-inputs/latex-vorspann}


\renewcommand{\erwaehntePersonen}{Personen:  ?? [Totgeborener Sohn von Arthur Schnitzler und Marie Reinhard], Paul Goldmann, Marie Reinhard, Leo Van-Jung}
\renewcommand{\erwaehnteOrte}{Orte: Café Tomaselli, Hotel Erzherzog Karl, Salzburg, Wien}
\renewcommand{\erwaehnteWerke}{}
\section[ Felix Salten an Arthur Schnitzler, 3. 9. {[}1897{]}]{Felix Salten an Arthur Schnitzler, 3. 9. {[}1897{]}}
\nopagebreak\mylabel{v}
\rehead{ }\normalsize\beginnumbering\briefempfaengerindex{Schnitzler, Arthur@\textsc{Schnitzler, Arthur}!zzzSalten, Felix@\emph{von Felix Salten}!1897-09-031@{3. 9. {[}1897{]}}|(be}
\toendnotes[C]{\smallbreak\pagebreak[2]}\Standort{CUL, Schnitzler, B 89, A 2.}
\physDesc{Brief, 1 Blatt, 2 Seiten, 1333 Zeichen
\newline{}Handschrift: Bleistift, lateinische Kurrent
\newline{}Schnitzler: mit Bleistift die Jahreszahl ergänzt: »97« 
\newline{}Ordnung: mit Bleistift von unbekannter Hand nummeriert: »97« }\toendnotes[C]{\smallbreak}
\pstart
           \noindent{}{\pb}\textcolor{gray}{\textbf{\textcolor{pink}{Café Tomaselli}{}\ledrightnote{\textcolor{pink}{Café Tomaselli}}}}\hfill den 3. September\pend
           
\pstart
           \textcolor{gray}{\textbf{\textcolor{pink}{SALZBURG}{}\ledrightnote{\textcolor{pink}{Salzburg}}}}\pend
           
\pstart
           \textcolor{gray}{\textbf{\textbf{gegründet 1753.}}}\pend
           
\pstart
           lieber Arthur, es ist so schönes Wetter, dass ich noch ein paar Tage
                  \textcolor{pink}{hier}{}\ledrightnote{{$\rightarrow$}\textcolor{pink}{Salzburg}} geblieben bin. So habe
               ich noch \label{K_L03274-1v}\edtext{\textcolor{blue}{Leo Fan-Jung}{}\ledrightnote{\textcolor{blue}{Leo Van-Jung}}}{\lemma{\textnormal{\emph{Leo Fan-Jung}}}\Cendnote{\textnormal{vgl. Richard Beer-Hofmann an Arthur Schnitzler, [31. 8. 1897]}}}\label{K_L03274-1h} und
                  \label{K_L03274-2v}\edtext{\textcolor{blue}{Goldmann}{}\ledrightnote{\textcolor{blue}{Paul Goldmann}} gesehen}{\lemma{\textnormal{\emph{Goldmann gesehen}}}\Cendnote{\textnormal{siehe Paul Goldmann an Arthur Schnitzler, 15. 10. [1897]}}}\label{K_L03274-2h}. \textcolor{blue}{G.}{}\ledrightnote{\textcolor{blue}{Paul Goldmann}} habe ich unverändert gefunden und
               er hat wieder einen schönen Eindruck gemacht. Das ist doch Einer, von dem man sagen
               kann, er sei ein absolut guter Mensch. Er war sehr lieb zu mir, was mir wolgethan
               hat. Im Allgemeinen ist meine Sti{\geminationm}ung nicht gut. Ich
               sehe von diesem schönen Platz aus nach \textcolor{pink}{Wien}{}\ledrightnote{\textcolor{pink}{Wien}}\textcolor{gray}{,} wie in einen dunkeln, unangenehmen Nebel hinein.
               Ich weiß nicht, was werden wird, und fühle meine Sorgen, auch wenn mir am wohlsten
               ist, wie man den leisen Druck permanenter Kopfschmerzen immer spürt und sich
               schließlich daran gewöhnt. Doch möchte ich gerne einmal freier athmen können, – ich
               glaube, {\pb}es käme da noch
               Manches heraus, was gut an mir ist. Für den Winter mache ich mir die strengsten
               Pläne, und denke sie auch auszuführen. Der \label{K_L03274-3v}\edtext{Gedanke ans Sterben}{\lemma{\textnormal{\emph{Gedanke ans Sterben}}}\Cendnote{\textnormal{siehe Felix Salten an Arthur Schnitzler, 23. 5. 1897}}}\label{K_L03274-3h}, der mir, wie Sie wissen, eine zeitlang abhanden gekommen, ist jetzt wieder
               so lebhaft in mir. Ich finde, dass das in vielen Beziehungen gut ist, der macht uns
               das Leben leichter, und macht es bewußter. Darüber wäre noch viel zu sagen.\pend
           
\pstart
           Wie geht es bei Ihnen? Arbeiten \textcolor{gray}{S}ie? Und \label{K_L03274-4v}\edtext{verläuft die Sache glatt}{\lemma{\textnormal{\emph{verläuft die Sache glatt}}}\Cendnote{\textnormal{Er dürfte sich auf die bevorstehende Niederkunft von \textcolor{blue}{Marie Reinhard} mit einem todgeborenen \textcolor{blue}{Kind} am 24. 9. 1897. }}}\label{K_L03274-4h}?
               Schreiben Sie mir ein Wort darüber. Ich bin voraussichtlich Dienstag in \textcolor{pink}{Wien}{}\ledrightnote{\textcolor{pink}{Wien}}. Herzliche Grüße\pend
           
\pstart
           Ihr{\\[\baselineskip]}\spacefill\mbox{Salten}\pend
           \leftskip=0em{}
\pstart
           \noindent{}Ich wohne jetzt: \textcolor{pink}{Erzherzog Karl}{}\ledrightnote{\textcolor{pink}{Hotel Erzherzog Karl}}\pend
           \endnumbering\briefempfaengerindex{Schnitzler, Arthur@\textsc{Schnitzler, Arthur}!zzzSalten, Felix@\emph{von Felix Salten}!1897-09-031@{3. 9. {[}1897{]}}|)be}\mylabel{h}  \normalsize

\doendnotes{C}
\bigskip
\vfill

\clearpage

\footnotesize

\lohead{\textsc{register}}

% Definiere theindex-Environment komplett neu ohne reledmac
\makeatletter
\renewenvironment{theindex}{%
  \section*{\indexname}%
  \setlength{\parindent}{0pt}%
  \setlength{\parskip}{0pt plus 0.3pt}%
  \let\item\@idxitem
}{%
  \clearpage
}
\makeatother

\IfFileExists{\jobname-pw.ind}{\input{\jobname-pw.ind}}{}

\end{document}

      