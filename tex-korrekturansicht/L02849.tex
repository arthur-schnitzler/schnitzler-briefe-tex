%% latex-korrekturansicht-vorspann.tex
%% Vorspann für die Korrekturansicht.
%% Lädt die gemeinsame Datei latex-vorspann.tex mit gesetztem Schalter.

\newif\ifkorrekturansicht
\korrekturansichttrue

\input{../tex-inputs/latex-vorspann}


               \section[ Paul Goldmann an Arthur Schnitzler, {[}28.–31. 3. 1898?{]}]{Paul Goldmann an Arthur Schnitzler, {[}28.–31. 3. 1898?{]}}\nopagebreak\mylabel{v}\rehead{ }\normalsize\beginnumbering\briefempfaengerindex{Schnitzler, Arthur@\textsc{Schnitzler, Arthur}!zzzGoldmann, Paul@\emph{von Paul Goldmann}!1898-03-281@{{[}28.–31. 3. 1898?{]}}|(be} \toendnotes[C]{\smallbreak\pagebreak[2]} \Standort{DLA, A:Schnitzler, HS.NZ85.1.3168.}
\physDesc{Brief, 1 Blatt, 4 Seiten
\newline{}Handschrift: schwarze Tinte, lateinische Kurrent
\newline{}Schnitzler: 1) mit Bleistift »Ende \textsc{März 98}« vermerkt 2) mit rotem Buntstift zwei Unterstreichungen}\toendnotes[C]{\smallbreak}\pstart
           \noindent{}\raggedleft{}{\pb}\textcolor{gray}{\textbf{\strikeout{\textcolor{pink}{Große Eſchenheimerſtraße 1}{}\ledrightnote{\textcolor{pink}{Große Eschenheimer Straße}}.}}}\pend
           \pstart{}Mein lieber Freund,\pend\pstart
           Ich danke Dir für Deinen lieben Brief, den ich \textcolor{pink}{hier}{}\ledrightnote{→\textcolor{pink}{Frankfurt am Main}} fand.\pend
           \pstart
           Es geht nicht, nach \textsc{\textcolor{pink}{Wien}{}\ledrightnote{\textcolor{pink}{Wien}}} zu kommen. Die Zeit reicht nicht aus. Es thut mir unendlich leid, daß ich ſo
               hinausfahren ſoll, ohne einen guten Händedruck von Dir mitzunehmen.\pend
           \pstart
           \label{K_L02849-1v}\edtext{Samſtag}{\lemma{\textnormal{\emph{Samſtag}}}\Cendnote{\textnormal{Damit dürfte der
                  Samstag 2. 4. 1898 gemeint sein. \textcolor{blue}{Goldmann} kam spätestens
                  am 4. 4. 1898 in \textcolor{pink}{Genua} an (vgl. Paul Goldmann an Arthur Schnitzler, 4. 4. 1898). \textcolor{blue}{Schnitzler} datiert den Brief vorliegenden Brief auf »Ende \textsc{März 98}«, was den Schluss zulässt, dass er zwischen  Montag, 28. und Donnerstag, 31. 3. 1898
                  verfasst wurde.}}}\label{K_L02849-1h}{ }früh fahre ich von \textcolor{pink}{hier}{}\ledrightnote{→\textcolor{pink}{Frankfurt am Main}} nach \textsc{\textcolor{pink}{Genua}{}\ledrightnote{\textcolor{pink}{Genua}}}. Am 5. ſteige ich dort aufs Schiff. {\pb}Ich habe viel Angſt vor der Seekrankheit und noch
               mehr davor, daß ich den \strikeout{\textcolor{gray}{G}} Aufgaben meiner Reiſe \label{K_L02849-2v}\edtext{journaliſtiſch-ſchriftſtelleriſch nicht gewachſen}{\lemma{\textnormal{\emph{journaliſtiſch-ſchriftſtelleriſch nicht gewachſen}}}\Cendnote{\textnormal{Derweil entstand genau in dieser Zeit das einzige
                  selbstständige Werk \textcolor{blue}{Goldmann}s: \emph{\textcolor{green}{Ein Sommer in China}} (\textcolor{pink}{Frankfurt a. M.}, 1899, 2 Bde.).}}}\label{K_L02849-2h} ſein werde.\pend
           \pstart
           Es freut mich unendlich, daß Du \label{K_L02849-3v}\edtext{arbeiteſt}{\lemma{\textnormal{\emph{arbeiteſt}}}\Cendnote{\textnormal{womöglich Bezug auf die
                  Fertigstellung von \emph{\textcolor{green}{Die Gefährtin}}, vgl. A. S.: \emph{Tagebuch}, 28. 3. 1898}}}\label{K_L02849-3h}. Laß’ Deine Stimmung ſein, wie ſie will, und arbeite weiter. Dadurch wird am
               Ende auch die Stimmung beſſer werden. Alle Mißſtimmung kommt ja doch nur daher, daß
                  {\pb}man \strikeout{\textcolor{gray}{×}} über ſich nachdenkt. Das muß man unter allen Umſtänden vermeiden, und Arbeit
               iſt das beſte Mittel hierzu.\pend
           \pstart
           Schreib’ mir, bitte, noch ein Wort über Dein Ergehen nach \textsc{\textcolor{pink}{Genova}{}\ledrightnote{\textcolor{pink}{Genua}}}, \label{K_L02849-4v}\edtext{\begin{otherlanguage}{italian}\textsc{ferma in posta}\end{otherlanguage}}{\lemma{\textnormal{\emph{ferma in posta}}}\Cendnote{\textnormal{italienisch: postlagernd}}}\label{K_L02849-4h}. Auch
               während ich unterwegs bin, mußt Du mir regelmäßig über Dich berichten. Ich theile Dir
               noch das Nähere über Adreſſen u. Sonſtiges mit. \pend
           \pstart
           {\pb}Vor meiner Abreiſe aus \textsc{\textcolor{pink}{Paris}{}\ledrightnote{\textcolor{pink}{Paris}}} war ich noch ein oder zwei Mal mit \textsc{\label{K_L02849-11v}\edtext{Frau \strikeout{Bahr}\textcolor{blue}{Bahr}{}\ledrightnote{\textcolor{blue}{Rosa Bahr}}} zuſammen (Saumenſch}{\lemma{\textnormal{\emph{XXXX Lemmafehler}}}\Cendnote{\textnormal{Der Umgang mit \textcolor{blue}{Rosa Bahr} wurde
               von mehreren Seiten als schwierig geschildert.}}}\label{K_L02849-11h}!) \pend
           \pstart
           Die Meinigen haben Alle viel nach Dir gefragt und grüßen Dich herzlich.\pend
           \pstart
           Grüße mir den \textsc{\textcolor{blue}{Richard}{}\ledrightnote{\textcolor{blue}{Richard Beer-Hofmann}}} und den \textsc{\textcolor{blue}{Leo}{}\ledrightnote{\textcolor{blue}{Leo Van-Jung}}} und ſei Du ſelbſt von Herzen gegrüßt!\pend
           \pstart
           Dein treuer {\\[\baselineskip]}\spacefill\mbox{Paul Goldmann}\pend
           \leftskip=0em{}\endnumbering\briefempfaengerindex{Schnitzler, Arthur@\textsc{Schnitzler, Arthur}!zzzGoldmann, Paul@\emph{von Paul Goldmann}!1898-03-281@{{[}28.–31. 3. 1898?{]}}|)be}\mylabel{h}\begin{anhang}\end{anhang}\normalsize

\doendnotes{C}
\bigskip
\vfill

\clearpage

\footnotesize

\lohead{\textsc{register}}

% Definiere theindex-Environment komplett neu ohne reledmac
\makeatletter
\renewenvironment{theindex}{%
  \section*{\indexname}%
  \setlength{\parindent}{0pt}%
  \setlength{\parskip}{0pt plus 0.3pt}%
  \let\item\@idxitem
}{%
  \clearpage
}
\makeatother

\IfFileExists{\jobname-pw.ind}{\input{\jobname-pw.ind}}{}

\end{document}

      