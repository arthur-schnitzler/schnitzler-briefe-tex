%% latex-korrekturansicht-vorspann.tex
%% Vorspann für die Korrekturansicht.
%% Lädt die gemeinsame Datei latex-vorspann.tex mit gesetztem Schalter.

\newif\ifkorrekturansicht
\korrekturansichttrue

\input{../tex-inputs/latex-vorspann}


               \section[ Paul Goldmann an Arthur Schnitzler, 21. 4. {[}1898{]}]{Paul Goldmann an Arthur Schnitzler, 21. 4. {[}1898{]}}\nopagebreak\mylabel{v}\rehead{ }\normalsize\beginnumbering\briefempfaengerindex{Schnitzler, Arthur@\textsc{Schnitzler, Arthur}!zzzGoldmann, Paul@\emph{von Paul Goldmann}!1898-04-212@{21. 4. {[}1898{]}}|(be} \toendnotes[C]{\smallbreak\pagebreak[2]} \Standort{DLA, A:Schnitzler, HS.NZ85.1.3168.}
\physDesc{Brief, 2 Blätter, 6 Seiten
\newline{}Handschrift: schwarze Tinte, lateinische Kurrent
\newline{}Schnitzler: mit Bleistift das Datum »21/4 98« vermerkt }\toendnotes[C]{\smallbreak}\pstart
           \noindent{}\centering{}{\pb}\textcolor{gray}{\textbf{\textcolor{brown}{DAMPFER »PREUSSEN«}{}\ledrightnote{\textcolor{brown}{Preussen}}}}\pend
           \pstart
           \noindent{}\textcolor{gray}{\textbf{\textcolor{brown}{NORDDEUTSCHER LLOYD}{}\ledrightnote{\textcolor{brown}{Norddeutscher Lloyd}} * \textcolor{pink}{BREMEN>}{}\ledrightnote{\textcolor{pink}{Bremen}}}}\pend
           \pstart
           \raggedleft{}21. April, \textcolor{pink}{Indiſcher Ocean}{}\ledrightnote{\textcolor{pink}{Indischer Ozean}}.\pend
           \pstart\center{}Mein lieber Freund,\pend\pstart
           Morgen iſt Poſtanſchluß in \textsc{\textcolor{pink}{Ceylon}{}\ledrightnote{\textcolor{pink}{Sri Lanka}}}, und ich will Dir einen herzlichen Gruß ſenden.\pend
           \pstart
           Die Reiſe iſt bisher wenig erfreulich. Ich leide abwechſelnd unter der Seekrankheit
               und unter der namenloſen Hitze. Das geht ſo ſeit dem \textcolor{pink}{Rothen Meer}{}\ledrightnote{\textcolor{pink}{Rotes Meer}}, alſo ſeit zehn Tagen {\pb}und es wird täglich ſchlimmer, je mehr wir an den
                  \textsc{Aequator} herankommen. Heut haben wir 36 Grad (Celsius), und dazu nicht ein Lüftchen Wind. In der
               Nacht gibt es keine Abkühlung, und die enge Cabine iſt ein entſetzlicher Aufenthalt.
               An Schlafen iſt kaum zu denken. Man dämmert ein paar Stunden hin zwiſchen Wachen
                  u. Schlaf und ſpr\textcolor{gray}{i}ngt beim erſten
               Lichtſtrahl wieder auf die Beine, froh aus {\pb}dem
               dumpfen Kerkerloch herauszukommen. Dazu habe ich einen \strikeout{\textcolor{gray}{du}} durch Seekrankheit u. heißes Trinken unheilbar verdorbenen Magen. Und in \textsc{\textcolor{pink}{China}{}\ledrightnote{\textcolor{pink}{China}}} ſollen wir in den heißen Sommer hineinkommen! Das kann gut werden. Das
               Schlimmſte aber iſt, daß mir das Arbeiten ſo ſchlecht von der Hand geht. Ich zwinge
               mich dazu mit Aufwendung aller meiner Energie. {\pb}Jeden Satz quäle ich nur heraus, und es iſt ſchrecklich, wie unlebendig,
               unperſönlich und conventionell Alles herauskommt. Ich reihe mühſam Eindrückchen an
               Eindrückchen, und ich fühle, daß das Ganze kein Bild gibt. Das iſt tief verſtimmend,
               und ich fürchte, meine Reiſe wird journaliſtiſch ein \textsc{Fiasco}.\pend
           \pstart
           Sehr fehlen mir auch Deine lieben Nachrichten. Ich bitte Dich, mir gleich {\pb}nach \textsc{\textcolor{pink}{Shanghai}{}\ledrightnote{\textcolor{pink}{Shanghai}}}, \textsc{\textcolor{brown}{Deutsches Post-Amt}{}\ledrightnote{→\textcolor{brown}{Deutsche Post in China}}, Poste
                  Restante} zu ſchreiben u. dieſe Adreſſe auch für ſpäter beizubehalten, bis ich
               Dir Gegentheiliges angebe.\pend
           \pstart
           Was wirſt Du dieſen \label{K_L02846-1v}\edtext{Sommer
                  unternehmen}{\lemma{\textnormal{\emph{Sommer
                  unternehmen}}}\Cendnote{\textnormal{siehe Paul Goldmann an Arthur Schnitzler, 16. 5. 1898}}}\label{K_L02846-1h}? \textsc{\textcolor{pink}{Ischl}{}\ledrightnote{\textcolor{pink}{Bad Ischl}}}? Der Gedanke an einen \textsc{\textcolor{pink}{Ischl}{}\ledrightnote{\textcolor{pink}{Bad Ischl}}er} Tannen-Wald \strikeout{i\textcolor{gray}{n}} iſt {\pb}wahrhaft ſchmerzlich an einem verſengenden
                  \textcolor{pink}{Indiſchen-Ocean}{}\ledrightnote{\textcolor{pink}{Indischer Ozean}}-Tage, wo man nach Luft und
               Kühlung ſchmachtet. Warum bin ich auch auf dieſes verfluchte \textcolor{pink}{Meer}{}\ledrightnote{→\textcolor{pink}{Indischer Ozean}} hinausgefahren!\pend
           \pstart
           Ich grüße Dich u. den lieben \textsc{\textcolor{blue}{Richard}{}\ledrightnote{\textcolor{blue}{Richard Beer-Hofmann}}} von ganzem Herzen.\pend
           \pstart
           Dein treuer {\\[\baselineskip]}\spacefill\mbox{Paul Goldmn}\pend
           \leftskip=0em{}\pstart
           \noindent{}Herzlichen Gruß an Deine \textcolor{blue}{Freundin}{}\ledrightnote{→\textcolor{blue}{Marie Reinhard}}!\pend
           \endnumbering\briefempfaengerindex{Schnitzler, Arthur@\textsc{Schnitzler, Arthur}!zzzGoldmann, Paul@\emph{von Paul Goldmann}!1898-04-212@{21. 4. {[}1898{]}}|)be}\mylabel{h}  \normalsize

\doendnotes{C}
\bigskip
\vfill

\clearpage

\footnotesize

\lohead{\textsc{register}}

% Definiere theindex-Environment komplett neu ohne reledmac
\makeatletter
\renewenvironment{theindex}{%
  \section*{\indexname}%
  \setlength{\parindent}{0pt}%
  \setlength{\parskip}{0pt plus 0.3pt}%
  \let\item\@idxitem
}{%
  \clearpage
}
\makeatother

\IfFileExists{\jobname-pw.ind}{\input{\jobname-pw.ind}}{}

\end{document}

      