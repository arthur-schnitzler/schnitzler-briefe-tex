%% latex-korrekturansicht-vorspann.tex
%% Vorspann für die Korrekturansicht.
%% Lädt die gemeinsame Datei latex-vorspann.tex mit gesetztem Schalter.

\newif\ifkorrekturansicht
\korrekturansichttrue

\input{../tex-inputs/latex-vorspann}


\renewcommand{\erwaehntePersonen}{Personen: Paul Goldmann, Eva Marie Goldmann, Franziska Goldmann, Olga Schnitzler}
\renewcommand{\erwaehnteOrte}{Orte: Berlin, Dorotheenstraße, Marienbad, Seis am Schlern, Villa Heufler, Villa Kamecke, Wien}
\renewcommand{\erwaehnteWerke}{}
\section[ Paul Goldmann an Arthur Schnitzler, 29. 7. 1908]{Paul Goldmann an Arthur Schnitzler, 29. 7. 1908}
\nopagebreak\mylabel{v}
\rehead{ }\normalsize\beginnumbering\briefempfaengerindex{Schnitzler, Arthur@\textsc{Schnitzler, Arthur}!zzzGoldmann, Paul@\emph{von Paul Goldmann}!1908-07-291@{29. 7. 1908}|(be}
\toendnotes[C]{\smallbreak\pagebreak[2]}\Standort{DLA, A:Schnitzler, HS.NZ85.1.3175.}
\physDesc{Bildpostkarte, 426 Zeichen
\newline{}Handschrift: 1) blaue Tinte, deutsche Kurrent\hspace{1em}2) blaue Tinte, lateinische Kurrent (\noindent{}Adresse)\hspace{1em}
\newline{}Versand: Stempel: »\nobreak{}\oindex{Berlin@\textbf{Berlin}, \emph{P.PPLC}|pwk}Berlin, S.
                                       W\textcolor{gray}{.} 11, 29. 7. 08, 11–12V.\nobreak{}«.  }\toendnotes[C]{\smallbreak}\pstart{}{\pb}Herrn\pend{}\pstart{}Dr. Arthur Schnitzler\pend{}\pstart{}\textcolor{pink}{Seis am Schlern}{}\ledrightnote{\textcolor{pink}{Seis am Schlern}}\pend{}\pstart{}\textcolor{pink}{Villa Heufler}{}\ledrightnote{\textcolor{pink}{Villa Heufler}}\pend{}\pstart{}\textcolor{pink}{Tirol}{}\ledrightnote{\textcolor{pink}{Seis am Schlern}}.\pend{}
{\bigskip}
\pstart
           \noindent{}\centering{}{\pb}\textcolor{gray}{\textbf{\textbf{Alt-\textcolor{pink}{Berlin}{}\ledrightnote{\textcolor{pink}{Berlin}}}}}\hspace*{2.5em}\textcolor{gray}{\textbf{\textcolor{pink}{Loge Royal-York}{}\ledrightnote{\textcolor{pink}{Villa Kamecke}} in der \textcolor{pink}{Dorotheenſtraße}{}\ledrightnote{\textcolor{pink}{Dorotheenstraße}} im Jahre 1833.}}\pend
           
\pstart
           {\pb}29. 7. 08.\pend
           
\pstart
           Lieber Freund, Ich danke Dir für Deine Karte u. habe mich ſehr
               gefreut, daß Du wieder einmal meiner gedacht haſt. Ich bin noch in \textcolor{pink}{Berlin}{}\ledrightnote{\textcolor{pink}{Berlin}}\strikeout{,} u. \label{K_L03463-1v}\edtext{verheirate}{\lemma{\textnormal{\emph{verheirate}}}\Cendnote{\textnormal{\textcolor{blue}{Goldmann} und \textcolor{blue}{Eva Marie Kobler, geb. Fränkel}, heirateten am 4. 8. 1908. \textcolor{blue}{Schnitzler} war die aus \textcolor{pink}{Wien} stammende
                  Braut spätestens seit 9. 8. 1900 bekannt. Das Paar hatte eine gemeinsame Tochter, die am
                     29. 5. 1911 zur Welt kam: \textcolor{blue}{Franziska Goldmann}.}}}\label{K_L03463-1h} mich hier nächſte Woche mit
               Frau \textsc{\textcolor{blue}{Kobler}{}\ledrightnote{\textcolor{blue}{Eva Marie Goldmann}}}. Wir gehen dann zunächſt nach \textcolor{pink}{Marienbad}{}\ledrightnote{\textcolor{pink}{Marienbad}},
               vielleicht ſpäter noch an die See. Meine zukünftige \textcolor{blue}{Frau}{}\ledrightnote{{$\rightarrow$}\textcolor{blue}{Eva Marie Goldmann}} u. ich ſenden Dir u. Deiner \textcolor{blue}{Frau}{}\ledrightnote{{$\rightarrow$}\textcolor{blue}{Olga Schnitzler}} herzliche Grüße.\pend
           
\pstart
           Dein getreuer {\\[\baselineskip]}\spacefill\mbox{Paul Goldmann.}\pend
           \leftskip=0em{}\endnumbering\briefempfaengerindex{Schnitzler, Arthur@\textsc{Schnitzler, Arthur}!zzzGoldmann, Paul@\emph{von Paul Goldmann}!1908-07-291@{29. 7. 1908}|)be}\mylabel{h}  \normalsize

\doendnotes{C}
\bigskip
\vfill

\clearpage

\footnotesize

\lohead{\textsc{register}}

% Definiere theindex-Environment komplett neu ohne reledmac
\makeatletter
\renewenvironment{theindex}{%
  \section*{\indexname}%
  \setlength{\parindent}{0pt}%
  \setlength{\parskip}{0pt plus 0.3pt}%
  \let\item\@idxitem
}{%
  \clearpage
}
\makeatother

\IfFileExists{\jobname-pw.ind}{\input{\jobname-pw.ind}}{}

\end{document}

      