%% latex-korrekturansicht-vorspann.tex
%% Vorspann für die Korrekturansicht.
%% Lädt die gemeinsame Datei latex-vorspann.tex mit gesetztem Schalter.

\newif\ifkorrekturansicht
\korrekturansichttrue

\input{../tex-inputs/latex-vorspann}


\renewcommand{\erwaehnteOrte}{Orte: Café Griensteidl, Prater, Wien}
\renewcommand{\erwaehnteWerke}{}
\section[ Felix Salten an Arthur Schnitzler, {[}30. 4.? 1895{]}]{Felix Salten an Arthur Schnitzler, {[}30. 4.? 1895{]}}
\nopagebreak\mylabel{v}
\rehead{ }\normalsize\beginnumbering\briefempfaengerindex{Schnitzler, Arthur@\textsc{Schnitzler, Arthur}!zzzSalten, Felix@\emph{von Felix Salten}!1895-04-301@{{[}30. 4.? 1895{]}}|(be}
\toendnotes[C]{\smallbreak\pagebreak[2]}\Standort{CUL, Schnitzler, B 89, A 1.}
\physDesc{Brief, 1 Blatt, 1 Seite, 188 Zeichen
\newline{}Handschrift: Bleistift, lateinische Kurrent
\newline{}Schnitzler: mit Bleistift datiert: »30/\textcolor{gray}{4} 95« 
\newline{}Ordnung: mit Bleistift von unbekannter Hand nummeriert: »54b(?)« }\toendnotes[C]{\smallbreak}
\pstart
           \noindent{}{\pb}Lieber Arthur, ich bin Abends von
                  ½ 8 an im \textcolor{pink}{Griensteidl}{}\ledrightnote{\textcolor{pink}{Café Griensteidl}} und würde
               gerne in den \textcolor{pink}{Prater}{}\ledrightnote{\textcolor{pink}{Prater}}, – ich mache aber, wenns sein
               muss auch was Anderes. Bitte telefoniren Sie mir, wenn Sie \label{K_L03154-1v}\edtext{ko{\geminationm}en}{\lemma{\textnormal{\emph{kommen}}}\Cendnote{\textnormal{Für diesen Tag ist kein
                  Treffen der beiden nachweisbar.}}}\label{K_L03154-1h}.\pend
           
\pstart
           Herzlichst {\\[\baselineskip]}\spacefill\mbox{Salten}\pend
           \leftskip=0em{}\endnumbering\briefempfaengerindex{Schnitzler, Arthur@\textsc{Schnitzler, Arthur}!zzzSalten, Felix@\emph{von Felix Salten}!1895-04-301@{{[}30. 4.? 1895{]}}|)be}\mylabel{h}  \normalsize

\doendnotes{C}
\bigskip
\vfill

\clearpage

\footnotesize

\lohead{\textsc{register}}

% Definiere theindex-Environment komplett neu ohne reledmac
\makeatletter
\renewenvironment{theindex}{%
  \section*{\indexname}%
  \setlength{\parindent}{0pt}%
  \setlength{\parskip}{0pt plus 0.3pt}%
  \let\item\@idxitem
}{%
  \clearpage
}
\makeatother

\IfFileExists{\jobname-pw.ind}{\input{\jobname-pw.ind}}{}

\end{document}

      