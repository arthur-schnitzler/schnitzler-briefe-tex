%% latex-korrekturansicht-vorspann.tex
%% Vorspann für die Korrekturansicht.
%% Lädt die gemeinsame Datei latex-vorspann.tex mit gesetztem Schalter.

\newif\ifkorrekturansicht
\korrekturansichttrue

\input{../tex-inputs/latex-vorspann}


\section[Arthur Schnitzler an Stefan Zweig, 9. 10. 1917]{L03775 Arthur Schnitzler an Stefan Zweig, 9. 10. 1917}
\nopagebreak\mylabel{L03775v}
\rehead{ }\normalsize\beginnumbering\briefempfaengerindex{, @\textsc{, }!zzz, @\emph{von  }!1917-10-091@{9. 10. 1917}|(be}
\toendnotes[C]{\smallbreak\pagebreak[2]}\Standort{Jerusalem, National Library of Israel, ARC. Ms. Var. 305 1 58 Stefan Zweig Collection.}
\physDesc{Briefkarte, 726 Zeichen
\newline{}Handschrift: schwarze Tinte, lateinische Kurrent}\toendnotes[C]{\smallbreak}
\pstart
           \textcolor{gray}{\textbf{Dr. Arthur Schnitzler}}\hfill {\pb}9. X. 917\pend
           
\pstart
           \textcolor{gray}{\textbf{\textcolor{pink}{Wien XVIII. Sternwartestrasse 71}\oindex{Wien@\textbf{Wien}!XVIII., Währing@\textbf{XVIII., Währing}!Sternwartestraße 71@\textbf{Sternwartestraße 71}, \emph{Wohngebäude}|pw}{}\ledrightnote{\textcolor{pink}{Sternwartestraße 71}}}}\pend
           
\pstart{}lieber und verehrter Herr Doctor,\pend\vspace{0.5em}
\pstart
           nun hab ich Ihren \textcolor{green}{Jeremias}\pwindex{Zweig, Stefan 28.\,11.\,1881 Wien – 23.\,2.\,1942 Petrópolis@\textsc{Zweig, Stefan} (28.\,11.\,1881 Wien – 23.\,2.\,1942 Petrópolis), \emph{Schriftsteller}!Jeremias. Ein dramatische Dichtung in neun Bildern@\strich\emph{Jeremias. Ein dramatische Dichtung in neun Bildern}|pw}{}\ledrightnote{\textcolor{green}{Jeremias. Ein dramatische Dichtung in neun Bildern}} gelesen, mit
               stärkster sich von Bild zu Bild erhöhender Theilnahme, und insbesondere den Schluss,
               nicht nur von den dichterischen Schönheiten, sondern auch von der menschlichen Wärme
               ergriffen, die Ihr \textcolor{green}{Werk}\pwindex{Zweig, Stefan 28.\,11.\,1881 Wien – 23.\,2.\,1942 Petrópolis@\textsc{Zweig, Stefan} (28.\,11.\,1881 Wien – 23.\,2.\,1942 Petrópolis), \emph{Schriftsteller}!Jeremias. Ein dramatische Dichtung in neun Bildern@\strich\emph{Jeremias. Ein dramatische Dichtung in neun Bildern}|pwv}{}\ledrightnote{{$\rightarrow$}\emph{\textcolor{green}{Jeremias. Ein dramatische Dichtung in neun Bildern}}}
               ausstrahlt. Auf der Bühne wird es meiner Überzeugung nach – in der gedrängten Form,
               die Sie für diesen Zweck Ihrem dramatischen {\pb}\textcolor{green}{Gedicht}\pwindex{Zweig, Stefan 28.\,11.\,1881 Wien – 23.\,2.\,1942 Petrópolis@\textsc{Zweig, Stefan} (28.\,11.\,1881 Wien – 23.\,2.\,1942 Petrópolis), \emph{Schriftsteller}!Jeremias. Ein dramatische Dichtung in neun Bildern@\strich\emph{Jeremias. Ein dramatische Dichtung in neun Bildern}|pwv}{}\ledrightnote{{$\rightarrow$}\emph{\textcolor{green}{Jeremias. Ein dramatische Dichtung in neun Bildern}}} wahrscheinlich geben
               werden, seine Wirkung gleichfalls nicht verfehlen, und ich wünschte sehr, Sie recht
               bald zu seinem ersten Theatererfolg beglückwünschen zu können.\pend
           
\pstart
           Meine \textcolor{blue}{Frau}\pwindex{Schnitzler, Olga 17.\,1.\,1882 Wien – 13.\,1.\,1970 Lugano@\textsc{Schnitzler, Olga} (17.\,1.\,1882 Wien – 13.\,1.\,1970 Lugano), \emph{Schauspielerin, Sängerin}|pwv}{}\ledrightnote{{$\rightarrow$}\emph{\textcolor{blue}{Olga Schnitzler}}}, die den gleichen
               Eindruck von \textcolor{green}{Jeremias}\pwindex{Zweig, Stefan 28.\,11.\,1881 Wien – 23.\,2.\,1942 Petrópolis@\textsc{Zweig, Stefan} (28.\,11.\,1881 Wien – 23.\,2.\,1942 Petrópolis), \emph{Schriftsteller}!Jeremias. Ein dramatische Dichtung in neun Bildern@\strich\emph{Jeremias. Ein dramatische Dichtung in neun Bildern}|pw}{}\ledrightnote{\textcolor{green}{Jeremias. Ein dramatische Dichtung in neun Bildern}} erhalten, dankt Ihnen,
               lieber Herr Doctor und grüßt Sie so herzlich wie ich.\pend
           
\pstart
           Ihr ergebner{\\[\baselineskip]}\spacefill\mbox{Arthur Schnitzler}\pend
           \leftskip=0em{}\selectlanguage{ngerman}\endnumbering\briefempfaengerindex{, @\textsc{, }!zzz, @\emph{von  }!1917-10-091@{9. 10. 1917}|)be}\mylabel{L03775h}
\begin{anhang}
\end{anhang}\normalsize

\doendnotes{C}
\bigskip
\vfill

\clearpage

\footnotesize

\lohead{\textsc{register}}

% Definiere theindex-Environment komplett neu ohne reledmac
\makeatletter
\renewenvironment{theindex}{%
  \section*{\indexname}%
  \setlength{\parindent}{0pt}%
  \setlength{\parskip}{0pt plus 0.3pt}%
  \let\item\@idxitem
}{%
  \clearpage
}
\makeatother

\IfFileExists{\jobname-pw.ind}{\input{\jobname-pw.ind}}{}

\end{document}

      