%% latex-korrekturansicht-vorspann.tex
%% Vorspann für die Korrekturansicht.
%% Lädt die gemeinsame Datei latex-vorspann.tex mit gesetztem Schalter.

\newif\ifkorrekturansicht
\korrekturansichttrue

\input{../tex-inputs/latex-vorspann}


\renewcommand{\erwaehntePersonen}{Personen: Felix Salten, Ottilie Salten, Olga Schnitzler, Heinrich Schnitzler}
\renewcommand{\erwaehnteOrte}{Orte: Berlin, Charlottenburg, Dänemark, Helsingør, Heringsdorf, Kopenhagen, Kurhotellet, Marienlyst, Wien, Zoologischer Garten Berlin}
\renewcommand{\erwaehnteWerke}{}
\section[ Felix und Ottilie Salten an Arthur Schnitzler, 27. 7. 1906]{Felix und Ottilie Salten an Arthur Schnitzler, 27. 7. 1906}
\nopagebreak\mylabel{v}
\rehead{ }\normalsize\beginnumbering\briefempfaengerindex{Schnitzler, Arthur@\textsc{Schnitzler, Arthur}!zzzSalten, Ottilie@\emph{von Ottilie Salten}!1906-07-272@{27. 7. 1906}|(be}\briefempfaengerindex{Schnitzler, Arthur@\textsc{Schnitzler, Arthur}!zzzSalten, Felix@\emph{von Felix Salten}!1906-07-272@{27. 7. 1906}|(be}
\toendnotes[C]{\smallbreak\pagebreak[2]}\Standort{CUL, Schnitzler, B 89, B 1.}
\physDesc{Bildpostkarte, 301 Zeichen
\newline{}Handschrift Felix Salten: schwarze Tinte, lateinische Kurrent
\newline{}Handschrift Ottilie Salten: schwarze Tinte, lateinische Kurrent
\newline{}Versand: 1) Stempel: »\nobreak{}\oindex{Charlottenburg@\textbf{Charlottenburg}, \emph{P.PPLX}|pwk}Charlottenburg 2, 27. 7. 06, 11–12 N\textcolor{gray}{.}\nobreak{}«.   2) Stempel: »\nobreak{}\oindex{Helsingør@\textbf{Helsingør}, \emph{P.PPLA2}|pwk}Helsingør, 28. 7. 06, 10–11 E\nobreak{}«. 
\newline{}Ordnung: mit Bleistift von unbekannter Hand nummeriert: »223« }\toendnotes[C]{\smallbreak}\pstart{}{\pb}Herrn D\textsuperscript{r} Arthur Schnitzler\pend{}\pstart{}\textcolor{pink}{Kurhaus}{}\ledrightnote{\textcolor{pink}{Kurhotellet}}\pend{}\pstart{}\textcolor{pink}{Marienlyst}{}\ledrightnote{\textcolor{pink}{Marienlyst}} bei \textcolor{pink}{Kopenhagen}{}\ledrightnote{\textcolor{pink}{Kopenhagen}}\pend{}\pstart{}\textcolor{pink}{Dänemark}{}\ledrightnote{\textcolor{pink}{Dänemark}}\pend{}
{\bigskip}
\pstart
           \noindent{}\centering{}{\pb}\textcolor{gray}{\textbf{Gruss aus dem \textcolor{pink}{zoologischen
                        Garten}{}\ledrightnote{\textcolor{pink}{Zoologischer Garten Berlin}} zu \textcolor{pink}{Berlin}{}\ledrightnote{\textcolor{pink}{Berlin}}.}}\pend
           
\pstart
           27. 7. 06.\pend
           
\pstart
           Abds.\pend
           
\pstart
           Lieber, heute aus \textcolor{pink}{Wien}{}\ledrightnote{\textcolor{pink}{Wien}} zurück, mit Otti hier Rendezvous. Dienstag{ }früh ab \textcolor{pink}{Heringsdorf}{}\ledrightnote{\textcolor{pink}{Heringsdorf}}, \textcolor{pink}{Kopenhagen}{}\ledrightnote{\textcolor{pink}{Kopenhagen}}, sind wir Donnerstag in \textcolor{pink}{Marienlyst}{}\ledrightnote{\textcolor{pink}{Marienlyst}}; freuen uns
               auf das \label{K_L03432-1v}\edtext{Wiedersehen}{\lemma{\textnormal{\emph{Wiedersehen}}}\Cendnote{\textnormal{siehe A. S.: \emph{Tagebuch}, 2. 8. 1906}}}\label{K_L03432-1h} und grüßen Sie \textcolor{blue}{Beide}{}\ledrightnote{{$\rightarrow$}\textcolor{blue}{Olga Schnitzler}}
               und \textcolor{blue}{Heini}{}\ledrightnote{\textcolor{blue}{Heinrich Schnitzler}} herzlichst {\\}\spacefill\mbox{Salten}\pend
           
\pstart
           \noindent{}{[}hs. Ottilie Salten:{]} Viele herzliche Grüße \spacefill\mbox{Otti S.}\pend
           \endnumbering\briefempfaengerindex{Schnitzler, Arthur@\textsc{Schnitzler, Arthur}!zzzSalten, Ottilie@\emph{von Ottilie Salten}!1906-07-272@{27. 7. 1906}|)be}\briefempfaengerindex{Schnitzler, Arthur@\textsc{Schnitzler, Arthur}!zzzSalten, Felix@\emph{von Felix Salten}!1906-07-272@{27. 7. 1906}|)be}\mylabel{h}  \normalsize

\doendnotes{C}
\bigskip
\vfill

\clearpage

\footnotesize

\lohead{\textsc{register}}

% Definiere theindex-Environment komplett neu ohne reledmac
\makeatletter
\renewenvironment{theindex}{%
  \section*{\indexname}%
  \setlength{\parindent}{0pt}%
  \setlength{\parskip}{0pt plus 0.3pt}%
  \let\item\@idxitem
}{%
  \clearpage
}
\makeatother

\IfFileExists{\jobname-pw.ind}{\input{\jobname-pw.ind}}{}

\end{document}

      