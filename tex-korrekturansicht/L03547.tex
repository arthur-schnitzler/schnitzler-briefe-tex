%% latex-korrekturansicht-vorspann.tex
%% Vorspann für die Korrekturansicht.
%% Lädt die gemeinsame Datei latex-vorspann.tex mit gesetztem Schalter.

\newif\ifkorrekturansicht
\korrekturansichttrue

\input{../tex-inputs/latex-vorspann}


\renewcommand{\erwaehntePersonen}{Personen: Felix Salten, Olga Schnitzler}
\renewcommand{\erwaehnteOrte}{Orte: Edmund-Weiß-Gasse 7, Triest, Wien}
\renewcommand{\erwaehnteWerke}{Werke: Monument für Domenico Rossetti}
\section[ Felix Salten an Arthur Schnitzler, 3. 4. 1910]{Felix Salten an Arthur Schnitzler, 3. 4. 1910}
\nopagebreak\mylabel{v}
\rehead{ }\normalsize\beginnumbering\briefempfaengerindex{Schnitzler, Arthur@\textsc{Schnitzler, Arthur}!zzzSalten, Felix@\emph{von Felix Salten}!1910-04-031@{3. 4. 1910}|(be}
\toendnotes[C]{\smallbreak\pagebreak[2]}\Standort{CUL, Schnitzler, B 89, B 2.}
\physDesc{Bildpostkarte, 88 Zeichen
\newline{}Handschrift: schwarze Tinte, lateinische Kurrent
\newline{}Versand: Stempel: »\nobreak{}\oindex{Triest@\textbf{Triest}, \emph{A.ADM3}|pwk}\textcolor{gray}{Trieste}, 3. IV. 10, VI\nobreak{}«.  
\newline{}Ordnung: mit Bleistift von unbekannter Hand nummeriert: »262« }\toendnotes[C]{\smallbreak}\pstart{}{\pb}Herrn D\textsuperscript{r} Arthur Schnitzler\pend{}\pstart{}\textcolor{pink}{Wien}{}\ledrightnote{\textcolor{pink}{Wien}}\pend{}\pstart{}\textcolor{pink}{XVIII. Spöttelgaße 7}{}\ledrightnote{\textcolor{pink}{Edmund-Weiß-Gasse 7}}\pend{}
{\bigskip}
\pstart
           \noindent{}{\pb}\textcolor{gray}{\textbf{\textcolor{pink}{Trieste}{}\ledrightnote{\textcolor{pink}{Triest}}.}}\hfill \textcolor{gray}{\textbf{\textcolor{green}{Monumento a Domenico Rossetti}{}\ledrightnote{\textcolor{green}{Monument für Domenico Rossetti}}.}}\pend
           
\pstart
           {\pb}Viele schöne Grüße an Sie \textcolor{blue}{Beide}{}\ledrightnote{{$\rightarrow$}\textcolor{blue}{Olga Schnitzler}}\pend
           
\pstart
           Ihr {\\[\baselineskip]}\spacefill\mbox{Salten}\pend
           \leftskip=0em{}\endnumbering\briefempfaengerindex{Schnitzler, Arthur@\textsc{Schnitzler, Arthur}!zzzSalten, Felix@\emph{von Felix Salten}!1910-04-031@{3. 4. 1910}|)be}\mylabel{h}  \normalsize

\doendnotes{C}
\bigskip
\vfill

\clearpage

\footnotesize

\lohead{\textsc{register}}

% Definiere theindex-Environment komplett neu ohne reledmac
\makeatletter
\renewenvironment{theindex}{%
  \section*{\indexname}%
  \setlength{\parindent}{0pt}%
  \setlength{\parskip}{0pt plus 0.3pt}%
  \let\item\@idxitem
}{%
  \clearpage
}
\makeatother

\IfFileExists{\jobname-pw.ind}{\input{\jobname-pw.ind}}{}

\end{document}

      