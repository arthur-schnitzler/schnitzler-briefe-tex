%% latex-korrekturansicht-vorspann.tex
%% Vorspann für die Korrekturansicht.
%% Lädt die gemeinsame Datei latex-vorspann.tex mit gesetztem Schalter.

\newif\ifkorrekturansicht
\korrekturansichttrue

\input{../tex-inputs/latex-vorspann}


               \section[Arthur Schnitzler an Wilhelm Bölsche, 8. 7. 1893]{ Arthur Schnitzler an Wilhelm Bölsche, 8. 7. 1893}\nopagebreak\mylabel{v}\rehead{ }\normalsize\beginnumbering\briefempfaengerindex{Boelsche, Wilhelm@\textsc{Bölsche, Wilhelm}!zzzSchnitzler, Arthur@\emph{von Arthur Schnitzler}!1893-07-081@{8. 7. 1893}|(be} \toendnotes[C]{\smallbreak\pagebreak[2]} \Standort{Wrocław, Biblioteka Uniwersytecka, Böl.Pis 1770.}
\physDesc{Brief, 1 Blatt (Briefpapier mit Trauerrand), 3 Seiten
\newline{}Handschrift: schwarze Tinte, deutsche Kurrent
\newline{}Bölsche: als »Erl{[}edigt{]}« gezeichnet }\buchAbdrucke{\weitereDrucke{1) Alois Woldan: \emph{Arthur Schnitzler – Briefe an Wilhelm Bölsche.} In: \emph{Germanica Wratislaviensia} (1987) Nr. 77, S. 463–464.} \weitereDrucke{2) Wilhelm Bölsche: \emph{Briefwechsel. Mit Autoren der Freien Bühne}. Hg. Gerd-Hermann Susen. Berlin: \emph{Weidler} 2010, S. 692 (Werke und Briefe. Wissenschaftliche Ausgabe, Briefe I).} }\toendnotes[C]{\smallbreak}\pstart{}{\pb}Sehr geehrter Herr Doktor,\pend\pstart
           erlauben Sie mir nunmehr die folgende Frage: Kö{\geminationn}ten
                    Sie \textcolor{green}{Das Märchen}{}\ledrightnote{\textcolor{green}{Das Märchen. Schauspiel in drei Aufzügen}} nach \textcolor{blue}{\textsc{Halbe}}{}\ledrightnote{\textcolor{blue}{Max Halbe}}’s \textcolor{green}{neuem Stück}{}\ledrightnote{→\textcolor{green}{Der Amerikafahrer}}, alſo
                    etwa im Oktober oder November bringen, \textsc{resp.} kö{\geminationn}te ich darauf
                    rechnen? – {\pb}Ich glaube annehmen zu können, dß es im
                        \textcolor{brown}{\textsc{Lessingtheater}}{}\ledrightnote{\textcolor{brown}{Lessing-Theater}} im Oktober dranko{\geminationm}t. Falls Sie
                    mein Ihnen gewidmetes \textcolor{green}{Exemplar}{}\ledrightnote{→\textcolor{green}{Das Märchen. Schauspiel in drei Aufzügen}} verlegt haben, will ich Ihnen zur Durchſicht gern ein andres
                    ſchicken. Daſs es ſich für Ihr \textcolor{green}{Blatt}{}\ledrightnote{→\textcolor{green}{Freie Bühne für den Entwickelungskampf der Zeit}}{ }{\pb}eignet, iſt kaum zu bezweifeln. –\pend
           \pstart
           Hochachtungsvoll{\\[\baselineskip]}\spacefill\mbox{Dr. Arthur Schnitzler}\pend
           \leftskip=0em{}\pstart
           \textcolor{pink}{\textsc{Ischl}}{}\ledrightnote{\textcolor{pink}{Bad Ischl}}, 8. 7. 93.\pend
           \pstart
           (Adreſſe nach wie vor \textcolor{pink}{\textsc{Wien I Grillparzerstr 7}}{}\ledrightnote{\textcolor{pink}{Grillparzerstraße}}.)\pend
           \pstart
           \raggedleft{}\spacefill\mbox{Sch}\pend
           \endnumbering\briefempfaengerindex{Boelsche, Wilhelm@\textsc{Bölsche, Wilhelm}!zzzSchnitzler, Arthur@\emph{von Arthur Schnitzler}!1893-07-081@{8. 7. 1893}|)be}\mylabel{h}  \normalsize

\doendnotes{C}
\bigskip
\vfill

\clearpage

\footnotesize

\lohead{\textsc{register}}

% Definiere theindex-Environment komplett neu ohne reledmac
\makeatletter
\renewenvironment{theindex}{%
  \section*{\indexname}%
  \setlength{\parindent}{0pt}%
  \setlength{\parskip}{0pt plus 0.3pt}%
  \let\item\@idxitem
}{%
  \clearpage
}
\makeatother

\IfFileExists{\jobname-pw.ind}{\input{\jobname-pw.ind}}{}

\end{document}

      