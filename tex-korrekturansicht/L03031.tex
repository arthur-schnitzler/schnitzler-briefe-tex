%% latex-korrekturansicht-vorspann.tex
%% Vorspann für die Korrekturansicht.
%% Lädt die gemeinsame Datei latex-vorspann.tex mit gesetztem Schalter.

\newif\ifkorrekturansicht
\korrekturansichttrue

\input{../tex-inputs/latex-vorspann}


\renewcommand{\erwaehntePersonen}{Personen: Felix Salten, Johann Schnitzler}
\renewcommand{\erwaehnteInstitutionen}{Institutionen: Internationale Ausstellung für Musik und Theaterwesen}
\renewcommand{\erwaehnteOrte}{Orte: Burgring, Kärntnerring 12/Bösendorferstraße 11, Ordination Dr. Johann Schnitzler Burgring 1, Hinterhaus, 2. Stock, Wien}
\renewcommand{\erwaehnteWerke}{}
\section[Arthur Schnitzler an Felix Salten, {[}7. 5. 1892?{]}]{Arthur Schnitzler an Felix Salten, {[}7. 5. 1892?{]}}
\nopagebreak\mylabel{v}
\rehead{ }\normalsize\beginnumbering\briefempfaengerindex{Salten, Felix@\textsc{Salten, Felix}!zzzSchnitzler, Arthur@\emph{von Arthur Schnitzler}!1892-05-072@{{[}7. 5. 1892?{]}}|(be}
\toendnotes[C]{\smallbreak\pagebreak[2]}\Standort{Wienbibliothek im Rathaus, ZPH 1681, 2.1.516.}
\physDesc{Briefkarte, 284 Zeichen
\newline{}Handschrift: Bleistift, deutsche Kurrent
\newline{}Ordnung: mit Bleistift von unbekannter Hand Nummerierung der Blätter des Konvoluts:
                                    »37« }\toendnotes[C]{\smallbreak}
\pstart
           \noindent{}{\pb}Lieber Freund, ich ko{\geminationn}te
                  geſtern nicht ko{\geminationm}en u nicht abſagen –
               Pardon! – Heute hab ich Sitze für Sie, d h für uns beide geno{\geminationm}en, bitte ſehr, erwarten Sie mich {\pb}4 Uhr in meiner \label{K_L03031-1v}\edtext{Wohnung
                  \textsc{\textcolor{pink}{Giselastraße}{}\ledrightnote{\textcolor{pink}{Kärntnerring 12/Bösendorferstraße 11}}}}{\lemma{\textnormal{\emph{Wohnung
                  Giselastraße}}}\Cendnote{\textnormal{Nach hinten kann das undatierte
                  Korrespondenzstück durch den Zeitraum eingegrenzt werden, in dem \textcolor{blue}{Schnitzler} an dieser Adresse wohnte (14. 10. 1892). Im Zuge
                  der \emph{\textcolor{brown}{Wiener Musik- und Theaterausstellung 1892}}
                  sind häufige gemeinsame Theaterbesuche nachgewiesen. Der erste Tag der \textcolor{brown}{Ausstellung}, der 7. 5. 1892, dürfte
                  auch der Versandtag dieses Schreibens sein, da \textcolor{blue}{Schnitzler} am [7. 5. 1892] seinen erkrankten \textcolor{blue}{Vater} in der \textcolor{pink}{Ordination} am \textcolor{pink}{Burgring 1} vertrat.}}}\label{K_L03031-1h} – we{\geminationn} Sie nicht eventuell ſchon früher 
               \textcolor{pink}{Burgring}{}\ledrightnote{\textcolor{pink}{Burgring}}
                  ko{\geminationm}en können. Aber treffen müſſen wir
               uns.\pend
           \pstart Ihr \spacefill\mbox{Arth Sch}\pend{}\endnumbering\briefempfaengerindex{Salten, Felix@\textsc{Salten, Felix}!zzzSchnitzler, Arthur@\emph{von Arthur Schnitzler}!1892-05-072@{{[}7. 5. 1892?{]}}|)be}\mylabel{h}  \normalsize

\doendnotes{C}
\bigskip
\vfill

\clearpage

\footnotesize

\lohead{\textsc{register}}

% Definiere theindex-Environment komplett neu ohne reledmac
\makeatletter
\renewenvironment{theindex}{%
  \section*{\indexname}%
  \setlength{\parindent}{0pt}%
  \setlength{\parskip}{0pt plus 0.3pt}%
  \let\item\@idxitem
}{%
  \clearpage
}
\makeatother

\IfFileExists{\jobname-pw.ind}{\input{\jobname-pw.ind}}{}

\end{document}

      