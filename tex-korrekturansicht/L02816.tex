%% latex-korrekturansicht-vorspann.tex
%% Vorspann für die Korrekturansicht.
%% Lädt die gemeinsame Datei latex-vorspann.tex mit gesetztem Schalter.

\newif\ifkorrekturansicht
\korrekturansichttrue

\input{../tex-inputs/latex-vorspann}


               \section[ Paul Goldmann an Arthur Schnitzler, 2. 7. {[}1897{]}]{Paul Goldmann an Arthur Schnitzler, 2. 7. {[}1897{]}}\nopagebreak\mylabel{v}\rehead{ }\normalsize\beginnumbering\briefempfaengerindex{Schnitzler, Arthur@\textsc{Schnitzler, Arthur}!zzzGoldmann, Paul@\emph{von Paul Goldmann}!1897-07-021@{2. 7. {[}1897{]}}|(be} \toendnotes[C]{\smallbreak\pagebreak[2]} \Standort{DLA, A:Schnitzler, HS.NZ85.1.3167.}
\physDesc{Brief, 1 Blatt, 4 Seiten
\newline{}Handschrift: blaue Tinte, deutsche Kurrent
\newline{}Schnitzler: 1) mit Bleistift das Jahr »97« vermerkt 2) mit rotem Buntstift fünf Unterstreichungen}\toendnotes[C]{\smallbreak}\pstart
           \noindent{}{\pb}\textcolor{gray}{\textbf{\textbf{\textcolor{brown}{Frankfurter Zeitung}{}\ledrightnote{\textcolor{brown}{Frankfurter Zeitung}}}}}\pend
           \pstart
           \textcolor{gray}{\textbf{(\textcolor{brown}{\begin{otherlanguage}{french}Gazette de Francfort\end{otherlanguage}}{}\ledrightnote{\textcolor{brown}{Frankfurter Zeitung}}).}}\pend
           \pstart
           \textcolor{gray}{\textbf{\textbf{\begin{otherlanguage}{french}Fondateur M.\end{otherlanguage}{ }\textcolor{blue}{L. Sonnemann}{}\ledrightnote{\textcolor{blue}{Leopold Sonnemann}}.}}}\pend
           \pstart
           \begin{otherlanguage}{french}\textcolor{gray}{\textbf{Journal politique, financier,}}\end{otherlanguage}\hfill \textsc{\textcolor{pink}{Paris}{}\ledrightnote{\textcolor{pink}{Paris}}}, 2. Juli.\pend
           \pstart
           \begin{otherlanguage}{french}\textcolor{gray}{\textbf{commercial et littéraire.}}\end{otherlanguage}\pend
           \pstart
           \begin{otherlanguage}{french}\textcolor{gray}{\textbf{\textbf{Paraissant trois fois par jour.}}}\end{otherlanguage}\pend
           \pstart
           \begin{otherlanguage}{french}\textcolor{gray}{\textbf{\textbf{Bureau à \textcolor{pink}{Paris}{}\ledrightnote{\textcolor{pink}{Paris}}}}}\end{otherlanguage}\pend
           \pstart
           \begin{otherlanguage}{french}\textcolor{gray}{\textbf{\textbf{\textcolor{pink}{10 Rue de la Bourse}{}\ledrightnote{\textcolor{pink}{rue de la Bourse}}.}}}\end{otherlanguage}\pend
           \pstart{}Mein lieber Freund,\pend\pstart
           Ich danke Dir für Deinen lieben Brief und Deine Correſpondenz-Karte. All’ dieſe Tage
               konnte ich nicht die Zeit zur Antwort finden. Auch bin ich krank und mißmuthig.\pend
           \pstart
           \label{K_L02816-87v}\edtext{Aus der \textcolor{pink}{Schweiz}{}\ledrightnote{\textcolor{pink}{Schweiz}}}{\lemma{\textnormal{\emph{Aus der Schweiz}}}\Cendnote{\textnormal{\textcolor{blue}{Marie Reinhard} war zu dieser Zeit bei ihrer
                     \textcolor{blue}{Mutter} in \textcolor{pink}{Andermatt}. Siehe A. S.: \emph{Tagebuch}, 13. 7. 1897.}}}\label{K_L02816-87h} habe ich plötzlich die \label{K_L02816-12v}\edtext{\textcolor{green}{\textsc{\textcolor{blue}{Wagner}{}\ledrightnote{\textcolor{blue}{Richard Wagner}}}-Biographie}{}\ledrightnote{→\textcolor{green}{Richard Wagner}}}{\lemma{\textnormal{\emph{Wagner-Biographie}}}\Cendnote{\textnormal{vermutlich: \textcolor{blue}{Houston Stewart Chamberlain}: \emph{\textcolor{green}{Richard Wagner}}. Mit zahlreichen Porträts,
                     Faksimiles, Illustrationen und Beilagen. München:
                        \emph{Verlagssanstalt für Kunst und Wissenschaft (vormals Friedrich
                        Bruckmann)}{ }1896 [vordatiert von Oktober 1895].}}}\label{K_L02816-12h} erhalten. \textcolor{blue}{Ihr}{}\ledrightnote{→\textcolor{blue}{Marie Reinhard}} ſeid
               wirklich zu lieb und gut! Ich hoffte ſchon, Ihr hättet e\textcolor{gray}{s}
               vergeſſen. Ich freue mich ſehr über das ſchöne \textcolor{green}{Buch}{}\ledrightnote{→\textcolor{green}{Richard Wagner}}. Bitte, theile mir die \textcolor{pink}{Schweiz}{}\ledrightnote{\textcolor{pink}{Schweiz}}er Adreſſe mit, damit ich danken kann. Und was wird aus dem \label{K_L02816-88v}\edtext{Opernglas}{\lemma{\textnormal{\emph{Opernglas}}}\Cendnote{\textnormal{Obzwar \textcolor{blue}{Goldmann}
                  bereits früher für \textcolor{blue}{Schnitzler} ein Opernglas
                  besorgt hatte (siehe Paul Goldmann an Arthur Schnitzler, 11. 1. [1896]),
                  dürfte sich dies hier auf eine neuerliche Bitte beziehen.}}}\label{K_L02816-88h}? Willſt Du mich
               denn unter allen Umſtänden zwingen, die 10 \textsc{Francs}, die Du
               mir dafür gegeben haſt, zu unterſchlagen? Bitte, laß’ mir die Redlichkeit meiner
               Seele, mein einziges Gut.\pend
           \pstart
           {\pb}Wenn ich daran denke, daß Du noch vor Kurzem hier
               geweſen biſt, ſo will ich es gar nicht glauben. Das iſt ſo fern, und ich bin ſo
               einſam!\pend
           \pstart
           Brauche ich Dir zu ſagen, daß es mein Herzenswunſch iſt, Dich in dieſem Sommer
                  \label{K_L02816-2v}\edtext{noch ein paar Tage zu ſehen}{\lemma{\textnormal{\emph{noch … ſehen}}}\Cendnote{\textnormal{Zwischen 19. 8. 1897 und 30. 8. 1897 sahen sich \textcolor{blue}{Schnitzler} und \textcolor{blue}{Goldmann} noch mehrmals in \textcolor{pink}{Bad Ischl}
                  wieder.}}}\label{K_L02816-2h}? Aber die Reiſe nach \textsc{\textcolor{pink}{Ischl}{}\ledrightnote{\textcolor{pink}{Bad Ischl}}} iſt ſo weit und theuer. Für die Hin- und Rückfahrt geht allein \strikeout{\textcolor{gray}{50 fl}} der größere Theil des Geldes
               drauf, das ich ausgeben \strikeout{k} kann. Ich kann noch gar
               nichts Beſtimmtes ſagen. Was würde mich die Penſion in \textsc{\textcolor{pink}{Ischl}{}\ledrightnote{\textcolor{pink}{Bad Ischl}}} pro Tag koſten? Natürlich dürfte das Zimmer nicht allzu ſchlecht ſein.\pend
           \pstart
           Fahre ich nach \textsc{\textcolor{pink}{Ischl}{}\ledrightnote{\textcolor{pink}{Bad Ischl}}}, ſo gehe ich über \label{K_L02816-3v}\edtext{\textsc{\textcolor{pink}{Bayreuth}{}\ledrightnote{\textcolor{pink}{Bayreuth}}} zu einer der \textsc{\textcolor{green}{Parſifal}{}\ledrightnote{\textcolor{green}{Parsifal}}}-Vorſtellungen{ }}{\lemma{\textnormal{\emph{Bayreuth … Parſifal-Vorſtellungen}}}\Cendnote{\textnormal{siehe Paul Goldmann an Arthur Schnitzler, 15. 6. [1897]}}}\label{K_L02816-3h}{\pb}am 8, 9, oder 11 Auguſt. Wenn Du
               ſchon nicht hinkommen kannſt, vielleicht kann \textsc{\textcolor{blue}{Richard}{}\ledrightnote{\textcolor{blue}{Richard Beer-Hofmann}}} auf ein paar Tage herüberfahren? Es iſt nicht unmöglich, daß von hier aus \textsc{\textcolor{blue}{Maxime Dethomas}{}\ledrightnote{\textcolor{blue}{Maxime Dethomas}}} mitkommt. \textsc{\textcolor{blue}{Leo}{}\ledrightnote{\textcolor{blue}{Leo Van-Jung}}} wiederzuſehen würde mich unendlich freuen. Von \textsc{\textcolor{blue}{Hugo}{}\ledrightnote{\textcolor{blue}{Hugo von Hofmannsthal}}} mag ich nichts wiſſen, ganz und gar nichts. Ich mag mir auch nicht die Mühe
               nehmen, ihn wiederzufinden. Er hätte mich ja blos nicht zu verlieren brauchen.\pend
           \pstart
           Vorgeſtern habe ich bei \textsc{Madame
                     \textcolor{blue}{Marni}{}\ledrightnote{\textcolor{blue}{Jeanne Marni}}} in \textsc{\textcolor{pink}{Louveciennes}{}\ledrightnote{\textcolor{pink}{Louveciennes}}} gefrühſtückt. Sie hat ſich ſehr mit Deinen Grüßen gefreut und ſich
               angelegentlich nach Dir erkundigt.\pend
           \pstart
           Ich hoffe, es geht Dir gut in \label{K_L02816-5v}\edtext{\textsc{\textcolor{pink}{Ischl}{}\ledrightnote{\textcolor{pink}{Bad Ischl}}}}{\lemma{\textnormal{\emph{Ischl}}}\Cendnote{\textnormal{\textcolor{blue}{Schnitzler} hielt sich von 26. 6. 1897 bis 24. 7. 1897 in \textcolor{pink}{Ischl} auf.}}}\label{K_L02816-5h}. Mit beſonderer Freude habe
               ich vernommen, daß das {\pb}neue \label{K_L02816-7v}\edtext{\textcolor{green}{Stück}{}\ledrightnote{→\textcolor{green}{Das Vermächtnis. Schauspiel in drei Akten}}}{\lemma{\textnormal{\emph{Stück}}}\Cendnote{\textnormal{der Dreiakter \emph{\textcolor{green}{Das Vermächtnis}}, an dem \textcolor{blue}{Schnitzler} seit dem 26. 6. 1897 arbeitete}}}\label{K_L02816-7h} zum Leben erwacht. Trag’ es nur mit Dir
               herum, bis die gewünſchte Klarheit da iſt. Und wenn Du Dich jetzt nicht zum Arbeiten
               geſtimmt fühlſt, ſo überſtürze es nicht und laß’ Dir Ruhe. Es iſt durchaus nicht
               nöthig, daß Du für die nächſte Saiſon gleich wieder mit einem neuen Stücke da
               biſt.\pend
           \pstart
           Schreib’ mir, bitte, recht bald und recht ausführlich: 1.) Wie es Dir geht (\label{K_L02816-9v}\edtext{körperlich}{\lemma{\textnormal{\emph{körperlich}}}\Cendnote{\textnormal{\textcolor{blue}{Schnitzler} notierte zu dieser Zeit keine
                  akuten Beschwerden im \emph{\textcolor{green}{Tagebuch}}, hatte aber
                  seit Herbst 1896 fortlaufend mit seiner Otosklerose zu
                  kämpfen.}}}\label{K_L02816-9h} auch)? 2.) Wie Du \label{K_L02816-11v}\edtext{\textsc{\textcolor{green}{\textcolor{blue}{Richard}{}\ledrightnote{\textcolor{blue}{Richard Beer-Hofmann}}}{}\ledrightnote{→\textcolor{green}{Der Tod Georgs}}} gefunden}{\lemma{\textnormal{\emph{Richard gefunden}}}\Cendnote{\textnormal{\textcolor{blue}{Richard Beer-Hofmann} arbeitete an der
                  Erzählung \emph{\textcolor{green}{Der Tod Georgs}}, damals noch unter
                  dem Titel \textcolor{green}{Der
                     Götterliebling}. \textcolor{blue}{Beer-Hofmann} hatte
                     \textcolor{blue}{Schnitzler} daraus bereits am 1. 1. 1897
                  vorgelesen, kurz nach diesem Brief auch am 17. 7. 1897. Womöglich kam es im
                     Sommer 1897 zu weiteren Vorlesungen, die jedoch nicht belegt
                  sind.}}}\label{K_L02816-11h} haſt? 3.) Welche Nachrichten Du aus der \textcolor{pink}{Schweiz}{}\ledrightnote{\textcolor{pink}{Schweiz}} haſt? Und was weiter geſchehen wird?\pend
           \pstart
           Ich begrüße Dich von Herzen und in Treue {\\[\baselineskip]}Dein{\\[\baselineskip]}\spacefill\mbox{Paul
                  Goldmann.}\pend
           \leftskip=0em{}\pstart
           \noindent{}Wenn Deine Frau \textcolor{blue}{Mutter}{}\ledrightnote{→\textcolor{blue}{Louise Schnitzler}}{ }\label{K_L02816-67v}\edtext{mit Dir iſt}{\lemma{\textnormal{\emph{mit Dir iſt}}}\Cendnote{\textnormal{\textcolor{blue}{Louise Schnitzler} kam am 3. 7. 1897 in \textcolor{pink}{Bad Ischl} an.}}}\label{K_L02816-67h}, ſo empfiehl’ mich,
                  bitte.\pend
           \endnumbering\briefempfaengerindex{Schnitzler, Arthur@\textsc{Schnitzler, Arthur}!zzzGoldmann, Paul@\emph{von Paul Goldmann}!1897-07-021@{2. 7. {[}1897{]}}|)be}\mylabel{h}\begin{anhang}\end{anhang}\normalsize

\doendnotes{C}
\bigskip
\vfill

\clearpage

\footnotesize

\lohead{\textsc{register}}

% Definiere theindex-Environment komplett neu ohne reledmac
\makeatletter
\renewenvironment{theindex}{%
  \section*{\indexname}%
  \setlength{\parindent}{0pt}%
  \setlength{\parskip}{0pt plus 0.3pt}%
  \let\item\@idxitem
}{%
  \clearpage
}
\makeatother

\IfFileExists{\jobname-pw.ind}{\input{\jobname-pw.ind}}{}

\end{document}

      