%% latex-korrekturansicht-vorspann.tex
%% Vorspann für die Korrekturansicht.
%% Lädt die gemeinsame Datei latex-vorspann.tex mit gesetztem Schalter.

\newif\ifkorrekturansicht
\korrekturansichttrue

\input{../tex-inputs/latex-vorspann}


               \section[Paul Goldmann an Arthur Schnitzler, Paul Goldmann an Arthur Schnitzler, 1. 11. {[}1896{]}]{ Paul Goldmann an Arthur Schnitzler, 1. 11. {[}1896{]}}\nopagebreak\mylabel{v}\rehead{ }\normalsize\beginnumbering\briefempfaengerindex{Schnitzler, Arthur@\textsc{Schnitzler, Arthur}!zzzGoldmann, Paul@\emph{von Paul Goldmann}!1896-11-011@{1. 11. {[}1896{]}}|(be} \toendnotes[C]{\smallbreak\pagebreak[2]} \Standort{DLA, A:Schnitzler, HS.NZ85.1.3166.}
\physDesc{Brief, 1 Blatt, 4 Seiten
\newline{}Handschrift: blaue Tinte, deutsche Kurrent
\newline{}Schnitzler: mit Bleistift das Jahr »96« vermerkt }\toendnotes[C]{\smallbreak}\pstart
           \noindent{}{\pb}\textcolor{gray}{\textbf{\textbf{\textcolor{brown}{Frankfurter Zeitung}{}\ledrightnote{\textcolor{brown}{Frankfurter Zeitung}}}}}\pend
           \pstart
           \textcolor{gray}{\textbf{(\textcolor{brown}{\begin{otherlanguage}{french}Gazette de Francfort\end{otherlanguage}}{}\ledrightnote{\textcolor{brown}{Frankfurter Zeitung}}).}}\pend
           \pstart
           \textcolor{gray}{\textbf{\textbf{\begin{otherlanguage}{french}Fondateur M.\end{otherlanguage}{ }\textcolor{blue}{L. Sonnemann}{}\ledrightnote{\textcolor{blue}{Leopold Sonnemann}}.}}}\pend
           \pstart
           \begin{otherlanguage}{french}\textcolor{gray}{\textbf{\textcolor{green}{Journal}{}\ledrightnote{→\textcolor{green}{Frankfurter Zeitung}} politique,
                        financier,}}\end{otherlanguage}\pend
           \pstart
           \begin{otherlanguage}{french}\textcolor{gray}{\textbf{commercial et littéraire.}}\end{otherlanguage}\pend
           \pstart
           \begin{otherlanguage}{french}\textcolor{gray}{\textbf{\textbf{Paraissant trois fois par jour.}}}\end{otherlanguage}\hfill \textsc{\textcolor{pink}{Paris}{}\ledrightnote{\textcolor{pink}{Paris}}}, 1. November.\pend
           \pstart
           \begin{otherlanguage}{french}\textcolor{gray}{\textbf{\textbf{Bureau à \textcolor{pink}{Paris}{}\ledrightnote{\textcolor{pink}{Paris}}}}}\end{otherlanguage}\pend
           \pstart
           \begin{otherlanguage}{french}\textcolor{gray}{\textbf{\textbf{\textcolor{pink}{24. Rue Feydeau}{}\ledrightnote{\textcolor{pink}{rue Feydeau}}.}}}\end{otherlanguage}\pend
           \pstart{}Mein lieber Freund,\pend\pstart
           Es iſt ſehr lieb von Dir, daß Du inmitten all Deiner Obliegenheiten in \textcolor{pink}{Berlin}{}\ledrightnote{\textcolor{pink}{Berlin}} noch Zeit gefunden, mir zu ſchreiben. Ich
               danke Dir und \label{K_L02789-11v}\edtext{ſende Dir dieſe
                  Zeilen}{\lemma{\textnormal{\emph{ſende Dir dieſe
                  Zeilen}}}\Cendnote{\textnormal{\textcolor{blue}{Goldmann} geht also davon aus, dass ein am Sonntag in \textcolor{pink}{Paris} abgeschickter Brief am Morgen des
                  Dienstags in \textcolor{pink}{Berlin} vorliegt.}}}\label{K_L02789-11h} nur,
               damit Du am Morgen des entſcheidenden \label{K_L02789-1v}\edtext{Tag}{\lemma{\textnormal{\emph{Tag}}}\Cendnote{\textnormal{die Uraufführung des \emph{\textcolor{green}{Freiwild}}s am 3. 11. 1896 am \textcolor{pink}{Deutschen Theater} in \textcolor{pink}{Berlin}}}}\label{K_L02789-1h}es einen Gruß von mir bekommſt. Das heißt: entſcheiden wird der Tag gar nichts. Alles Weſentliche iſt entſchieden. Wir
               wiſſen Alle, wer Du biſt; und Dein neues \textcolor{green}{Stück}{}\ledrightnote{→\textcolor{green}{Freiwild. Schauspiel in 3 Akten}}, \label{K_L02789-2v}\edtext{wenn es
               Erfolg hat}{\lemma{\textnormal{\emph{wenn es
               Erfolg hat}}}\Cendnote{\textnormal{\emph{\textcolor{green}{Freiwild}} war nicht ansatzweise so erfolgreich
                  wie die \emph{\textcolor{green}{Liebelei}}.}}}\label{K_L02789-2h}, kann uns {\pb}nichts Neues lehren, – wenn \strikeout{\textcolor{gray}{ſ}} ſein Erfolg beſtritten wird, kann es an der bereits beſtehenden Thatſache
               nichts ändern, daß \textsc{Arthur Schnitzler} in der gegenwärtigen
               deutſchen dramatiſchen Bewegung eine der wenigen bemerkenswerthen Erſcheinungen iſt.
               Ich ſehe alſo dem 3. November lange nicht mit
               derſelben Spannung entgegen, wie dem Tage der \textsc{Première} der »\textcolor{green}{Liebelei}{}\ledrightnote{\textcolor{green}{Liebelei. Schauspiel in drei Akten}}«. Ein neuer Erfolg wäre ſehr ſchön, aber nöthig iſt er gerade nicht.
               Die »\textcolor{green}{Liebelei}{}\ledrightnote{\textcolor{green}{Liebelei. Schauspiel in drei Akten}}« \uline{mußte} Erfolg haben; denn {\pb}darin \strikeout{\textcolor{gray}{la}g \textcolor{gray}{la}g} lag Deine ganze Art, und es
               war die große, ein für alle Mal entſcheidende Frage: \strikeout{ob} ob das Publicum »Ja« oder »Nein« dazu ſagen würde. Was das \textcolor{pink}{Berlin}{}\ledrightnote{\textcolor{pink}{Berlin}}er Publicum zu »\textcolor{green}{Freiwild}{}\ledrightnote{\textcolor{green}{Freiwild. Schauspiel in 3 Akten}}« ſagt, iſt \strikeout{\textcolor{gray}{wig}} wichtig mit Rückſicht auf die materiellen Conſequenzen – für das \uline{Weſentliche} aber iſt es ganz gleichgiltig. Daß ich Dir
               trotzdem für ein Telegramm am Mittwoch{ }Vormittag von Herzen dankbar ſein werde, verſteht ſich von ſelbſt.\pend
           \pstart
           {\pb}Schade, daß Du das \label{K_L02789-3v}\edtext{»befreiende« Wort}{\lemma{\textnormal{\emph{»befreiende« Wort}}}\Cendnote{\textnormal{siehe Paul Goldmann an Arthur Schnitzler, 27. 10. [1896]}}}\label{K_L02789-3h} nicht findeſt. \strikeout{Laß} Eigentlich iſt es \strikeout{eigentlich} ſchon enthalten in dem Ausſpruch: »\textcolor{green}{Solche
               Leute haben im Frieden gar keine Exiſtenz-Berechtigung}{}\ledrightnote{→\textcolor{green}{Freiwild. Schauspiel in 3 Akten}}«. Laß’ den \textcolor{blue}{Schauſpieler}{}\ledrightnote{→\textcolor{blue}{Hermann Nissen}} das nur recht kräftig und
               deutungsvoll ſagen!\pend
           \pstart
           Ich hab’ einen Augenblick mit der Idee geliebäugelt, hier auf drei Tage durchzugehen
               und zur \textsc{Première} zu kommen. Aber, wie gewöhnlich, fehlte das
               Geld; auch bin ich doch nicht mehr jung genug für ſolche Huſarenſtücklein. Ich muß alſo wieder aus
               der Ferne zuſchauen. Statt meiner kommen meine Wünſche; ſie ſollen Dir alle{[}s{]} Liebe,
               Gute, Frohe für Dienſtag{ }Abend bringen. Ich umarme Dich von Herzen.\pend
           \pstart
           Dein treuer {\\[\baselineskip]}\spacefill\mbox{Paul Goldm}\pend
           \leftskip=0em{}\pstart
           \noindent{}\label{T_L02789-1v}\edtext{Du ſchreibſt mir wohl noch ein Wort aus \textcolor{pink}{Berlin}{}\ledrightnote{\textcolor{pink}{Berlin}}?}{\lemma{\textnormal{\emph{Du … Berlin?}}}\Cendnote{\textnormal{entlang des
               Seitenrands der letzten Seite, quer zum Text}}}\label{T_L02789-1h}\pend
           \endnumbering\briefempfaengerindex{Schnitzler, Arthur@\textsc{Schnitzler, Arthur}!zzzGoldmann, Paul@\emph{von Paul Goldmann}!1896-11-011@{1. 11. {[}1896{]}}|)be}\mylabel{h}  \normalsize

\doendnotes{C}
\bigskip
\vfill

\clearpage

\footnotesize

\lohead{\textsc{register}}

% Definiere theindex-Environment komplett neu ohne reledmac
\makeatletter
\renewenvironment{theindex}{%
  \section*{\indexname}%
  \setlength{\parindent}{0pt}%
  \setlength{\parskip}{0pt plus 0.3pt}%
  \let\item\@idxitem
}{%
  \clearpage
}
\makeatother

\IfFileExists{\jobname-pw.ind}{\input{\jobname-pw.ind}}{}

\end{document}

      