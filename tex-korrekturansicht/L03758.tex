%% latex-korrekturansicht-vorspann.tex
%% Vorspann für die Korrekturansicht.
%% Lädt die gemeinsame Datei latex-vorspann.tex mit gesetztem Schalter.

\newif\ifkorrekturansicht
\korrekturansichttrue

\input{../tex-inputs/latex-vorspann}


\section[Arthur Schnitzler an Stefan Zweig, 29. 5. 1923]{L03758 Arthur Schnitzler an Stefan Zweig, 29. 5. 1923}
\nopagebreak\mylabel{L03758v}
\rehead{ }\normalsize\beginnumbering\briefempfaengerindex{, @\textsc{, }!zzz, @\emph{von  }!1923-05-291@{29. 5. 1923}|(be}
\toendnotes[C]{\smallbreak\pagebreak[2]}\Standort{Jerusalem, National Library of Israel, ARC. Ms. Var. 305 1 58 Stefan Zweig Collection.}
\physDesc{Postkarte, 803 Zeichen
\newline{}Schreibmaschine
\newline{}Handschrift: Bleistift (\noindent{}Unterschrift)
\newline{}Versand: Stempel: »\nobreak{}\oindex{IX., Alsergrund@\textbf{IX., Alsergrund}, \emph{Verwaltungsgebiet}|pwk}9/\textcolor{gray}{×} Wien 72, 30. V. 23, VIII\nobreak{}«.  }\toendnotes[C]{\smallbreak}\pstart{}{\pb}\label{T_L03758-1v}\edtext{\textcolor{gray}{\textbf{A. S.}}}{\lemma{\textnormal{\emph{A. S.}}}\Cendnote{\textnormal{ovaler Absenderkleber}}}\label{T_L03758-1}\pend{}\pstart{}\textcolor{pink}{\textcolor{gray}{\textbf{WIEN, XVIII.}}}\oindex{XVIII., Währing@\textbf{XVIII., Währing}, \emph{Verwaltungsgebiet}|pw}{}\ledrightnote{\textcolor{pink}{XVIII., Währing}}\pend{}\pstart{}\textcolor{pink}{\textcolor{gray}{\textbf{STERNWARTESTR. 71}}}\oindex{Wien@\textbf{Wien}!XVIII., Währing@\textbf{XVIII., Währing}!Sternwartestraße 71@\textbf{Sternwartestraße 71}, \emph{Wohngebäude}|pw}{}\ledrightnote{\textcolor{pink}{Sternwartestraße 71}}\pend{}{\bigskip}\pstart{}Herrn\pend{}\pstart{}Dr. Stefan Zweig\pend{}\pstart{}\textcolor{pink}{Salzburg}\oindex{Salzburg@\textbf{Salzburg}, \emph{Verwaltungsgebiet}|pw}{}\ledrightnote{\textcolor{pink}{Salzburg}}.\pend{}\pstart{}\textcolor{pink}{Kapuzinerberg 5}\oindex{Paschinger Schlössl@\textbf{Paschinger Schlössl}, \emph{Wohngebäude}|pw}{}\ledrightnote{\textcolor{pink}{Paschinger Schlössl}}.\pend{}{\bigskip}\vspace{1em}
\pstart
           \raggedleft{}{\pb}29. 5. 1923. \pend
           
\pstart{}Lieber Herr Doktor.\pend\vspace{0.5em}
\pstart
           Vielen Dank, dass Sie mich auf diese bevorstehende Versteigerung aufmerksam gemacht
               haben. Das \textcolor{green}{Buch}\pwindex{Wassermann, Jakob 10.\,3.\,1873 Fürth – 1.\,1.\,1934 Altaussee@\textsc{Wassermann, Jakob} (10.\,3.\,1873 Fürth – 1.\,1.\,1934 Altaussee), \emph{Schriftsteller}!Gänsemännchen. Roman@\strich\emph{Das Gänsemännchen. Roman}|pwv}{}\ledrightnote{{$\rightarrow$}\emph{\textcolor{green}{Das Gänsemännchen. Roman}}} ist mir
               offenbar gestohlen worden. Ich habe gleich an das betreffende \textcolor{brown}{Antiquariat}\orgindex{Antiquariat Emil Hirsch@Antiquariat Emil Hirsch|pwv}{}\ledrightnote{{$\rightarrow$}\emph{\textcolor{brown}{Antiquariat Emil Hirsch}}} geschrieben und das \textcolor{green}{Buch}\pwindex{Wassermann, Jakob 10.\,3.\,1873 Fürth – 1.\,1.\,1934 Altaussee@\textsc{Wassermann, Jakob} (10.\,3.\,1873 Fürth – 1.\,1.\,1934 Altaussee), \emph{Schriftsteller}!Gänsemännchen. Roman@\strich\emph{Das Gänsemännchen. Roman}|pwv}{}\ledrightnote{{$\rightarrow$}\emph{\textcolor{green}{Das Gänsemännchen. Roman}}} zurückgefordert.\pend
           
\pstart
           \label{K_L03758-1v}\edtext{Eben}{\lemma{\textnormal{\emph{Eben}}}\Cendnote{\textnormal{Er kam am 27. 5. 1923 wieder in \textcolor{pink}{Wien}\oindex{Wien@\textbf{Wien}, \emph{Verwaltungsgebiet}|pwk}
                  an.}}}\label{K_L03758-1} komme ich von einer sehr schönen Reise nach \textcolor{pink}{Dänemark}\oindex{Dänemark@\textbf{Dänemark}|pw}{}\ledrightnote{\textcolor{pink}{Dänemark}} und \textcolor{pink}{Schweden}\oindex{Schweden@\textbf{Schweden}|pw}{}\ledrightnote{\textcolor{pink}{Schweden}}
               zurück und habe nun erst Ihren lieben Brief vorgefunden. Seien Sie herzlichst
               gegrüsst und lassen Sie mich hoffen, dass ich bald wieder das Vergnügen habe Sie
               persönlich wiederzusehen und ausführlicher {\pb}mit Ihnen zu
               reden.\pend
           
\pstart
           Sehr gefreut habe ich mich unter manchem anderm, was ich in der letzten Zeit
               von Ihnen las, an Ihrem warmen \label{K_L03758-2v}\edtext{\textcolor{green}{Worten}\pwindex{L. Andro (Therese Rie), »Der Klimenole«, Deutsche Verlagsanstalt, Stuttgart]@\emph{[L. Andro (Therese Rie), »Der Klimenole«, Deutsche Verlagsanstalt, Stuttgart]}|pwv}{}\ledrightnote{{$\rightarrow$}\emph{\textcolor{green}{[L. Andro (Therese Rie), »Der Klimenole«, Deutsche Verlagsanstalt, Stuttgart]}}}}{\lemma{\textnormal{\emph{Worten}}}\Cendnote{\textnormal{\textcolor{blue}{st. z.}\pwindex{Zweig, Stefan 28.\,11.\,1881 Wien – 23.\,2.\,1942 Petrópolis@\textsc{Zweig, Stefan} (28.\,11.\,1881 Wien – 23.\,2.\,1942 Petrópolis), \emph{Schriftsteller}|pwk} [= \textcolor{blue}{Stefan Zweig}\pwindex{Zweig, Stefan 28.\,11.\,1881 Wien – 23.\,2.\,1942 Petrópolis@\textsc{Zweig, Stefan} (28.\,11.\,1881 Wien – 23.\,2.\,1942 Petrópolis), \emph{Schriftsteller}|pwk}]: \emph{\textcolor{green}{[L. Andro (Therese Rie), »Der Klimenole«, Deutsche
                        Verlagsanstalt, Stuttgart]}\pwindex{L. Andro (Therese Rie), »Der Klimenole«, Deutsche Verlagsanstalt, Stuttgart]@\emph{[L. Andro (Therese Rie), »Der Klimenole«, Deutsche Verlagsanstalt, Stuttgart]}|pwk}}. In: \emph{\textcolor{green}{Neue
                        Freie Presse}\pwindex{Neue Freie Presse@\emph{Neue Freie Presse}|pwk}}, Nr. 21.088, 27. 5. 1923, Morgenblatt,
                     S. 33. }}}\label{K_L03758-2} über das neue \textcolor{green}{Buch}\pwindex{Rie, Therese 1.\,1.\,1878 Wien – 23.\,7.\,1934 ebd.@\textsc{Rie, Therese} (1.\,1.\,1878 Wien – 23.\,7.\,1934 ebd.), \emph{Schriftstellerin}!Klimenole. Roman@\strich\emph{Der Klimenole. Roman}|pwv}{}\ledrightnote{{$\rightarrow$}\emph{\textcolor{green}{Der Klimenole. Roman}}} von \textcolor{blue}{L. Andro}\pwindex{Rie, Therese 1.\,1.\,1878 Wien – 23.\,7.\,1934 ebd.@\textsc{Rie, Therese} (1.\,1.\,1878 Wien – 23.\,7.\,1934 ebd.), \emph{Schriftstellerin}|pw}{}\ledrightnote{\textcolor{blue}{Therese Rie}},
               der ich auch \label{K_L03758-3v}\edtext{neulich schrieb}{\lemma{\textnormal{\emph{neulich schrieb}}}\Cendnote{\textnormal{Das Korrespondenzstück ist nicht erhalten,
                     vgl. Therese Rie-Andro an Arthur Schnitzler, 3. 5. 1923.}}}\label{K_L03758-3}, ohne sie
               persönlich zu kennen. Ihr\pend
           \pstart \spacefill\mbox{{[}hs.:{]} Arthur Schnitzler}\pend{}\selectlanguage{ngerman}\endnumbering\briefempfaengerindex{, @\textsc{, }!zzz, @\emph{von  }!1923-05-291@{29. 5. 1923}|)be}\mylabel{L03758h}
\begin{anhang}
\end{anhang}\normalsize

\doendnotes{C}
\bigskip
\vfill

\clearpage

\footnotesize

\lohead{\textsc{register}}

% Definiere theindex-Environment komplett neu ohne reledmac
\makeatletter
\renewenvironment{theindex}{%
  \section*{\indexname}%
  \setlength{\parindent}{0pt}%
  \setlength{\parskip}{0pt plus 0.3pt}%
  \let\item\@idxitem
}{%
  \clearpage
}
\makeatother

\IfFileExists{\jobname-pw.ind}{\input{\jobname-pw.ind}}{}

\end{document}

      