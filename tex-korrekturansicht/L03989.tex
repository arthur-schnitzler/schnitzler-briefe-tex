%% latex-korrekturansicht-vorspann.tex
%% Vorspann für die Korrekturansicht.
%% Lädt die gemeinsame Datei latex-vorspann.tex mit gesetztem Schalter.

\newif\ifkorrekturansicht
\korrekturansichttrue

\input{../tex-inputs/latex-vorspann}


\section[Arthur Schnitzler an Theodor Herzl, 10. 12. 1899]{L03989 Arthur Schnitzler an Theodor Herzl, 10. 12. 1899}
\nopagebreak\mylabel{L03989v}
\rehead{ }\normalsize\beginnumbering\briefempfaengerindex{Herzl, Theodor@\textsc{Herzl, Theodor}!zzzSchnitzler, Arthur@\emph{von Arthur Schnitzler}!1899-12-101@{10. 12. 1899}|(be}
\toendnotes[C]{\smallbreak\pagebreak[2]}
\correspDesc{Versand  durch Arthur Schnitzler am 10. 12. 1899 in Wien
\newline{}Erhalt  durch Theodor Herzl im Zeitraum [10. 12. 1899 –
                  12. 12. 1899?] in Wien}\toendnotes[C]{\smallbreak}
\Standort{Wien, Österreichische Nationalbibliothek, Lit 571/B303/1.}
\physDesc{Brief, 1 Blatt, 2 Seiten, 506 Zeichen
\newline{}Handschrift: schwarze Tinte, deutsche Kurrent}\toendnotes[C]{\smallbreak}
\pstart{}{\pb}lieber Freund,\pend\vspace{0.5em}
\pstart
           wie lange geben Sie mir noch Friſt? Noch i{\geminationm}er bin ich,
               dichtend, feilend mit dem \textcolor{green}{Stück}\pwindex{Schnitzler, Arthur 15. 5. 1862 Wien – 21. 10. 1931 ebd.@\textsc{Schnitzler, Arthur} (15. 5. 1862 Wien – 21. 10. 1931 ebd.), \emph{Schriftsteller, Mediziner}!Schleier der Beatrice. Schauspiel in fünf Akten@\strich\emph{Der Schleier der Beatrice. Schauspiel in fünf Akten}|pwv}{}\ledrightnote{{$\rightarrow$}\emph{\textcolor{green}{Der Schleier der Beatrice. Schauspiel in fünf Akten}}}
               beſchäftigt, aber noch i{\geminationm}er hoff ich dſs ich Ihnen
               irgend war werde ſchicken können. Natürlich möcht ich ſo wenig als Sie, dſs es ein
                  {\pb}absoluter Schmarrn iſt. Iſt es zu ſpät, wenn Sie das \textcolor{green}{Manuscript}\pwindex{Schnitzler, Arthur 15. 5. 1862 Wien – 21. 10. 1931 ebd.@\textsc{Schnitzler, Arthur} (15. 5. 1862 Wien – 21. 10. 1931 ebd.), \emph{Schriftsteller, Mediziner}!Um eine Stunde@\strich\emph{Um eine Stunde}|pwv}{}\ledrightnote{{$\rightarrow$}\emph{\textcolor{green}{Um eine Stunde}}} (groſs wird es ja keineswegs ſein) am 18. oder
                  19. bekommen? Haben Sie es da nicht, ſo halten Sie mich für einen
               meineidigen – aber darin nicht minder für Ihren aufrichtg un\textcolor{gray}{d}
               herzlich ergebnen \pend
           \pstart \spacefill\mbox{ArthurSchnitzler}\pend{}
\pstart
           10/12 99.\pend
           \selectlanguage{ngerman}\endnumbering\briefempfaengerindex{Herzl, Theodor@\textsc{Herzl, Theodor}!zzzSchnitzler, Arthur@\emph{von Arthur Schnitzler}!1899-12-101@{10. 12. 1899}|)be}\mylabel{L03989h}
\begin{anhang}
\end{anhang}\normalsize

\doendnotes{C}
\bigskip
\vfill

\clearpage

\footnotesize

\lohead{\textsc{register}}

% Definiere theindex-Environment komplett neu ohne reledmac
\makeatletter
\renewenvironment{theindex}{%
  \section*{\indexname}%
  \setlength{\parindent}{0pt}%
  \setlength{\parskip}{0pt plus 0.3pt}%
  \let\item\@idxitem
}{%
  \clearpage
}
\makeatother

\IfFileExists{\jobname-pw.ind}{\input{\jobname-pw.ind}}{}

\end{document}

      