%% latex-korrekturansicht-vorspann.tex
%% Vorspann für die Korrekturansicht.
%% Lädt die gemeinsame Datei latex-vorspann.tex mit gesetztem Schalter.

\newif\ifkorrekturansicht
\korrekturansichttrue

\input{../tex-inputs/latex-vorspann}


\renewcommand{\erwaehntePersonen}{Personen: Paul Goldmann}
\renewcommand{\erwaehnteInstitutionen}{Institutionen: S. Fischer Verlag}
\renewcommand{\erwaehnteOrte}{Orte: Berlin, Wien}
\renewcommand{\erwaehnteWerke}{Werke: Das weite Land. Tragikomödie in fünf Akten}
\section[ Arthur Schnitzler an Paul Goldmann, 9. 3. 1925]{Arthur Schnitzler an Paul Goldmann, 9. 3. 1925}
\nopagebreak\mylabel{v}
\rehead{ }\normalsize\beginnumbering\briefempfaengerindex{Goldmann, Paul@\textsc{Goldmann, Paul}!zzzSchnitzler, Arthur@\emph{von Arthur Schnitzler}!1925-03-092@{9. 3. 1925}|(be}
\toendnotes[C]{\smallbreak\pagebreak[2]}\Standort{DLA, A:Schnitzler, HS.1985.1.857.}
\physDesc{Brief, Durchschlag, 1 Blatt, 1 Seite, 2169 Zeichen
\newline{}Schreibmaschine
\newline{}Handschrift Arthur Schnitzler: roter Buntstift, deutsche Kurrent (\noindent{}»\uline{\textcolor{blue}{Goldmann}}«, am zweiten Blatt die Datumsangabe »9/3 \textcolor{gray}{25}« wiederholt, eine Unterstreichung)
\newline{}Handschrift Schreibkraft: roter Buntstift, lateinische Kurrent (\noindent{}Vermerk »K{[}opie{]}«)}
\buchAbdrucke{\weitereDrucke{Arthur Schnitzler: \emph{Briefe 1913–1931}. Hg. Peter Michael Braunwarth, Richard Miklin, Susanne Pertlik und Heinrich Schnitzler. Frankfurt am Main: \emph{S. Fischer} 1984, S. 395–397.} }\toendnotes[C]{\smallbreak}
\pstart
           \raggedleft{}{\pb}\textcolor{pink}{Wien}{}\ledrightnote{\textcolor{pink}{Wien}}, 9. 3. 1925.\pend
           
\pstart{}Mein lieber Freund.\pend
\pstart
           Mir ist, als hättest Du den eigentlichen Sinn und Zweck meines \label{K_L03519-1v}\edtext{Glückwunschschreibens}{\lemma{\textnormal{\emph{Glückwunschschreibens}}}\Cendnote{\textnormal{siehe Paul Goldmann an Arthur Schnitzler, 16. 2. 1925}}}\label{K_L03519-1h} missverstanden. Es war a priori nicht anzunehmen, dass wir, Du und ich über
               uns selbst und über einander als Sechzigjährige wesentlich anders denken sollten, als
               wir vor 10 oder 15 Jahren gedacht haben; – und es ist möglich, dass meine Ansicht
               über die Art und das Ausmass Deiner Begabung so wenig zutrifft, als das Deine über
               mich und meine Werke.\pend
           
\pstart
           Jedesfalls liegt die Entscheidung darüber nicht bei uns Beiden und es liegt mir ferne
                  heute über diese Fragen eine Diskussion zu
               eröffnen, die doch aller Voraussicht nach nicht zu einer Einigung führen dürfte.\pend
           
\pstart
           Ob Dir eine Arbeit von mir mehr oder weniger gelungen scheint; – ob ich Deinen
               menschlichen Wert und Deine schriftstellerische Bedeutung darin ausgedrückt finde,
               was man gemeiniglich poetisches Talent nennt, oder in andern an sich nicht minder
               hochzuschätzenden Elementen Deines Wesens und Deiner Begabung, – das kommt für meine
               Empfindung im gegenwärtigen Moment unseres Lebens nicht mehr in Betracht.\pend
           
\pstart
           Was ich in meinem Brief sagen oder wenigstens anzudeuten versuchte, – das ist: dass
                  \so{über} unseren Meinungen und Urteilen, mögen sie nun
               irrtümlich sein oder nicht, zwischen Dir und mir eine Beziehung bestand und für mein
               Gefühl noch immer besteht, die in einer seelischen und geistigen Gemeinsamkeit
               unserer Jugendjahr wurzelt – und somit als »Idee« unzerstörbar ist, mag sie auch {\pb}für die äussere Gestaltung unseres Verhältnisses zu
               meinem Bedauern keine genügende aufbauende Kraft mehr besitzen.\pend
           
\pstart
           Trotzdem (oder deswegen) könnte auch ich mich versucht fühlen ein \label{K_L03519-2v}\edtext{Wort aus einem meiner Stücke}{\lemma{\textnormal{\emph{Wort … Stücke}}}\Cendnote{\textnormal{»Ja. Solche Dinge hängen nämlich nie
                     von dem ab, was man miteinander {\dots} für Erfahrungen
                     macht. Sonſt täten ja Enttäuſchungen nicht weh {\dots}
                     wenn damit die innern Beziehungen einfach aus wären. Aber daß man doch immer
                     aneinander hängen bleibt {\dots} das {\dots}! {\dots} Es gibt nur ewige Liebe und ewige
                     Freundſchaft. Und der Mauer iſt und bleibt mein einziger Freund. Das steht feſt
                     {\dots} Auch wenn er mich einmal erſchießen ſollte, es
                     wird nicht anders.« \textcolor{blue}{Arthur Schnitzler}: \emph{\textcolor{green}{Das weite Land. Tragikomödie in fünf Akten}}.
                     Berlin: \emph{\textcolor{brown}{S. Fischer}}{ }1911, S. 138 (4. Akt, Erna und Hofreiter). }}}\label{K_L03519-2h} zu
               zitieren, wie Du es getan – \textcolor{green}{Hofreiter}{}\ledrightnote{{$\rightarrow$}\textcolor{green}{Das weite Land. Tragikomödie in fünf Akten}} spricht es im »\textcolor{green}{Weiten Land}{}\ledrightnote{\textcolor{green}{Das weite Land. Tragikomödie in fünf Akten}}«
               aus, dass es nämlich überhaupt \uline{nur ewige} Liebe und
                  \uline{ewige} Freundschaft gebe (auch wenn die Freunde
               durch die Macht der Umstände gedrungen sein sollten sich gegenseitig totzuschiessen
               (ich zitiere ungenau)); – und so musst Du es Dir schon gefallen lassen, dass ich mich
               auch weiterhin mit neuerlichen Glückwünschen und Grüssen Deinen Freund nenne, wie in
               fernen Jugendtagen – (ohne mörderische und ohne sentimentale Consequenzen)\pend
           
\pstart
           Herzlichst Dein {\\[\baselineskip]}\spacefill\mbox{A.S.}\pend
           \leftskip=0em{}\endnumbering\briefempfaengerindex{Goldmann, Paul@\textsc{Goldmann, Paul}!zzzSchnitzler, Arthur@\emph{von Arthur Schnitzler}!1925-03-092@{9. 3. 1925}|)be}\mylabel{h}  \normalsize

\doendnotes{C}
\bigskip
\vfill

\clearpage

\footnotesize

\lohead{\textsc{register}}

% Definiere theindex-Environment komplett neu ohne reledmac
\makeatletter
\renewenvironment{theindex}{%
  \section*{\indexname}%
  \setlength{\parindent}{0pt}%
  \setlength{\parskip}{0pt plus 0.3pt}%
  \let\item\@idxitem
}{%
  \clearpage
}
\makeatother

\IfFileExists{\jobname-pw.ind}{\input{\jobname-pw.ind}}{}

\end{document}

      