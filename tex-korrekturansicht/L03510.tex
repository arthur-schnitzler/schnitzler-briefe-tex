%% latex-korrekturansicht-vorspann.tex
%% Vorspann für die Korrekturansicht.
%% Lädt die gemeinsame Datei latex-vorspann.tex mit gesetztem Schalter.

\newif\ifkorrekturansicht
\korrekturansichttrue

\input{../tex-inputs/latex-vorspann}


\renewcommand{\erwaehntePersonen}{Personen:  ?? [Bedienstete im Hotel Quisisana],  ?? [Mädchen, das Tennis spielt], Julius Elias, Siegfried Jacobsohn, Eduard Pötzl, Anna Katharina Rehmann, Felix Salten, Ottilie Salten, Paul Salten, Paul Schlenther}
\renewcommand{\erwaehnteInstitutionen}{Institutionen: Die Zeit}
\renewcommand{\erwaehnteOrte}{Orte: ?? [Delikatessenhandlung in Marienbad], Ambrosiusquelle, Berlin, Café Egerländer, Café Forstwarte, Café Nimrod, Ferdinandquelle, Hotel Quisisana, Hotel Rübezahl, Kreuzbrunnen, Lago di Garda, Lviv, Mannheim, Marienbad, München, Schneeberg, Semmering, Welsberg-Taisten, Wien}
\renewcommand{\erwaehnteWerke}{Werke: ?? [Englische Reise], Auferstehung. Komödie in einem Akt, Das gelobte Wien, Der Ernst des Lebens. Schauspiel in einem Akt, Der Graf. Komödie in einem Akt, Der Hund von Florenz, Der Wiener Korrespondent, Morgen. Wochenschrift für deutsche Kultur, Rotkäppchen, Vom andern Ufer. Einakter}
\section[ Felix Salten an Arthur Schnitzler, 15. 8. 1907]{Felix Salten an Arthur Schnitzler, 15. 8. 1907}
\nopagebreak\mylabel{v}
\rehead{ }\normalsize\beginnumbering\briefempfaengerindex{Schnitzler, Arthur@\textsc{Schnitzler, Arthur}!zzzSalten, Felix@\emph{von Felix Salten}!1907-08-151@{15. 8. 1907}|(be}
\toendnotes[C]{\smallbreak\pagebreak[2]}\Standort{CUL, Schnitzler, B 89, B 1.}
\physDesc{Brief, 1 Blatt, 3 Seiten, 4004 Zeichen
\newline{}Handschrift: schwarze Tinte, lateinische Kurrent
\newline{}Ordnung: mit Bleistift von unbekannter Hand nummeriert: »233« }\toendnotes[C]{\smallbreak}
\pstart
           \raggedleft{}{\pb}\textcolor{pink}{Marienbad}{}\ledrightnote{\textcolor{pink}{Marienbad}}, 15. August 07\pend
           
\pstart
           \raggedleft{}\textcolor{pink}{Haus Quisisana}{}\ledrightnote{\textcolor{pink}{Hotel Quisisana}}.\pend
           
\pstart
           Lieber, wir sind jetzt bald eine Woche da. \textcolor{blue}{Otti}{}\ledrightnote{\textcolor{blue}{Ottilie Salten}} braucht die Kur. \textcolor{pink}{Kreuzbrunnen}{}\ledrightnote{\textcolor{pink}{Kreuzbrunnen}} und \textcolor{pink}{Ferdinandsquelle}{}\ledrightnote{\textcolor{pink}{Ferdinandquelle}},
               Moorbäder und Kohlensäure; sie befindet sich dabei sehr wol, und ihre Genesung macht
               sichtlich Fortschritte. Ich habe auch mit einer Kur begonnen, aber nur einen Tag
               ausgehalten. Um 5Uhr aufstehen und um neun erst frühstücken
               könnte ich nur dann vertragen, wenn ich von hier aus erst noch auf vier Wochen
               anderswohin zu Erholung ginge. Da ich mich aber ausruhen muss, hat es keinen Sinn,
               wenn ich mich jetzt quäle, und dann vielleicht noch matter und noch nervöser nach \textcolor{pink}{Wien}{}\ledrightnote{\textcolor{pink}{Wien}} zurückkomme. Den \textcolor{blue}{Kindern}{}\ledrightnote{{$\rightarrow$}\textcolor{blue}{Paul Salten}{\newline}{$\rightarrow$}\textcolor{blue}{Anna Katharina Rehmann}} tut \textcolor{pink}{Mbd.}{}\ledrightnote{\textcolor{pink}{Marienbad}} unglaublich gut. Sie essen hier, dass wir eine Freude
               haben. Und sie lernen endlich weite Spaziergänge machen, was man an der See weniger
               übt, und wozu sie – durch unseren Garten – in \textcolor{pink}{Wien}{}\ledrightnote{\textcolor{pink}{Wien}}
               nie gelangt sind. Hier sind die Wälder herrlich, und die vielen Jausenorte, die
               überall auf den kleinen Berggipfeln und Hochplateaus liegen, sind wirklich famos. Wir
               wohnen ganz ausserhalb von \textcolor{pink}{Marienbad}{}\ledrightnote{\textcolor{pink}{Marienbad}} in einer
               Straße, die nur auf der einen Seite Häuser, auf der anderen den Wald hat, zahlen für
               zwei hübsche Zimmer 25fl die Woche, was sehr billig ist, haben das Mittagessen – und
               was für ein Mittagessen! – für 60 Kreuzer die Person auf dem Zimmer. Das Frühstück
               macht das \label{K_L03510-1v}\edtext{\textcolor{blue}{Fräulein}{}\ledrightnote{\textcolor{blue}{?? [Bedienstete im Hotel Quisisana]}}}{\lemma{\textnormal{\emph{Fräulein}}}\Cendnote{\textnormal{nicht ermittelt}}}\label{K_L03510-1h}, gejaust wird
               irgendwo auf einem Berg. (\textcolor{pink}{Rübezahl}{}\ledrightnote{\textcolor{pink}{Hotel Rübezahl}}{ }\textcolor{pink}{Forstwarte}{}\ledrightnote{\textcolor{pink}{Café Forstwarte}}, \textcolor{pink}{Nimrod}{}\ledrightnote{\textcolor{pink}{Café Nimrod}}, \textcolor{pink}{Egerländer}{}\ledrightnote{\textcolor{pink}{Café Egerländer}} u. s. w.) Und
               Nachtmahl holt man sich in der \label{K_L03510-2v}\edtext{\textcolor{pink}{Delikatessenhandlung}{}\ledrightnote{\textcolor{pink}{?? [Delikatessenhandlung in Marienbad]}}}{\lemma{\textnormal{\emph{Delikatessenhandlung}}}\Cendnote{\textnormal{nicht ermittelt}}}\label{K_L03510-2h}, die hier alle
               Begriffe, die man sich in einer Delikatessenhandlung macht \textcolor{gray}{h}och
               übertrifft. Ich verstehe, warum \textcolor{blue}{Elias}{}\ledrightnote{\textcolor{blue}{Julius Elias}} von \textcolor{pink}{Marienbad}{}\ledrightnote{\textcolor{pink}{Marienbad}} so begeistert {\pb}ist. Die Tennisplätze sind die
               schönsten, die ich kenne. Man spielt eine halbe Stunde nach dem Regen. Wir haben eine
               ganz gute Partie, ein taubstummes junges \label{K_L03510-3v}\edtext{\textcolor{blue}{Mädchen}{}\ledrightnote{{$\rightarrow$}\textcolor{blue}{?? [Mädchen, das Tennis spielt]}}}{\lemma{\textnormal{\emph{Mädchen}}}\Cendnote{\textnormal{nicht ermittelt}}}\label{K_L03510-3h}, die sehr nett
               ist und sehr scharf spielt. Morgen{ }früh kommt \textcolor{blue}{Siegfried Jacobsohn}{}\ledrightnote{\textcolor{blue}{Siegfried Jacobsohn}}{ }\textcolor{pink}{hier}{}\ledrightnote{{$\rightarrow$}\textcolor{pink}{Marienbad}} an, von den \textcolor{blue}{Kindern}{}\ledrightnote{{$\rightarrow$}\textcolor{blue}{Paul Salten}{\newline}{$\rightarrow$}\textcolor{blue}{Anna Katharina Rehmann}} Onkel \textcolor{blue}{Japottsohn}{}\ledrightnote{\textcolor{blue}{Siegfried Jacobsohn}} genannt. Er bleibt bis Mittwoch und geht dann nach \textcolor{pink}{Wien}{}\ledrightnote{\textcolor{pink}{Wien}}. Hier sind natürlich eine Menge Menschen, denen man nicht immer
               ausweichen kann. Wir waren denn auch die ersten Tage in einem Gebrodel von \textcolor{pink}{Berlin}{}\ledrightnote{\textcolor{pink}{Berlin}}er, \textcolor{pink}{Lemberg}{}\ledrightnote{\textcolor{pink}{Lviv}}er, \textcolor{pink}{Wien}{}\ledrightnote{\textcolor{pink}{Wien}}er, \textcolor{pink}{München}{}\ledrightnote{\textcolor{pink}{München}}er und \textcolor{pink}{Mannheim}{}\ledrightnote{\textcolor{pink}{Mannheim}}er
               Leuten, von Wagenfahrten, Automobilpartien, u. s. w. Aber wir haben schnell gebremst
               und leben jetzt ruhig. Wenn \textcolor{blue}{Otti}{}\ledrightnote{\textcolor{blue}{Ottilie Salten}} nicht früh
               und Abend zum Brunnen müßte, würden wir noch weniger Verkehr haben. Die \textcolor{blue}{Kinder}{}\ledrightnote{{$\rightarrow$}\textcolor{blue}{Paul Salten}{\newline}{$\rightarrow$}\textcolor{blue}{Anna Katharina Rehmann}} trinken \textcolor{pink}{Ambrosiusquelle}{}\ledrightnote{\textcolor{pink}{Ambrosiusquelle}} (Eisen){[},{]}
               was immer ein großer Spass ist. Dann fahren sie Eselwagen, und da sie jetzt \label{K_L03510-4v}\edtext{nacheinander Geburtstag}{\lemma{\textnormal{\emph{nacheinander Geburtstag}}}\Cendnote{\textnormal{\textcolor{blue}{Paul} war am 11. 8. 1907 vier Jahre alt geworden. \textcolor{blue}{Annerl}s dritter Geburtstag stand am 18. 8. 1907 bevor.}}}\label{K_L03510-4h} feiern, ist ihr Jubel groß. \textcolor{blue}{Annerl}{}\ledrightnote{\textcolor{blue}{Anna Katharina Rehmann}} hat fabelhafte Erfolge, während die tieferen Naturen
                  \textcolor{blue}{Pauli}{}\ledrightnote{\textcolor{blue}{Paul Salten}} schätzen. Neulich haben die \textcolor{blue}{Kinder}{}\ledrightnote{{$\rightarrow$}\textcolor{blue}{Paul Salten}{\newline}{$\rightarrow$}\textcolor{blue}{Anna Katharina Rehmann}} im Wald
               Theater gespielt und \textcolor{green}{Rothkäppchen}{}\ledrightnote{\textcolor{green}{Rotkäppchen}} aufgeführt.
               Sie waren förmlich betrunken davon, dass da ein wirklicher Wald war, und man kann
               sagen, dass es auch sonst eine vortreffliche Aufführung gewesen ist. – Wir haben
               manchmal auch schon \textcolor{blue}{Schlenther}{}\ledrightnote{\textcolor{blue}{Paul Schlenther}} gesehen. Er
               sieht aus, als ob er heimliche Balggeschwülste und Drüsen hätte.\pend
           
\pstart
           \textcolor{pink}{Hier}{}\ledrightnote{{$\rightarrow$}\textcolor{pink}{Marienbad}} arbeite ich nur
               Kleinigkeiten, die von der \textcolor{brown}{Redaktion}{}\ledrightnote{{$\rightarrow$}\textcolor{brown}{Die Zeit}} verlangt werden, sonst nichts. Ich habe in \textcolor{pink}{Wien}{}\ledrightnote{\textcolor{pink}{Wien}} allerlei gemacht. Darunter die \label{K_L03510-5v}\edtext{drei kleinen \textcolor{green}{Stücke}{}\ledrightnote{{$\rightarrow$}\textcolor{green}{Vom andern Ufer. Einakter}{\newline}{$\rightarrow$}\textcolor{green}{Auferstehung. Komödie in einem Akt}{\newline}{$\rightarrow$}\textcolor{green}{Der Graf. Komödie in einem Akt}{\newline}{$\rightarrow$}\textcolor{green}{Der Ernst des Lebens. Schauspiel in einem Akt}}}{\lemma{\textnormal{\emph{drei kleinen Stücke}}}\Cendnote{\textnormal{\emph{\textcolor{green}{Auferstehung}}, \emph{\textcolor{green}{Der Graf}} und \emph{\textcolor{green}{Ernst des
                     Lebens}}, versammelt in \emph{\textcolor{green}{Vom andern Ufer}}}}}\label{K_L03510-5h}, die nun in Maschinschrift vorliegen. Wenn ich sie im Herbst noch erträglich
               finde, \label{K_L03510-6v}\edtext{les’ ich sie vielleicht {\pb}vor}{\lemma{\textnormal{\emph{les’ … vor}}}\Cendnote{\textnormal{\textcolor{blue}{Schnitzler} las sie am 5. 10. 1907
                  selbst.}}}\label{K_L03510-6h}. Im September schreibe ich den »\textcolor{green}{Hund v. Florenz}{}\ledrightnote{\textcolor{green}{Der Hund von Florenz}}«. Er ist jetzt ganz fertig dazu
               und vielfach verändert. Könnte ich die Zeitung los sein, wäre ich froh und vermöchte
               vielleicht einiges Gute zustande zu bringen. Mir wird die Zeitungschreiberei immer
               leerer und leerer. Bin ich wirklich im September mit dem
                  »\textcolor{green}{Hund}{}\ledrightnote{\textcolor{green}{Der Hund von Florenz}}« fertig, dann mache ich die Seereise.
               Der \textcolor{pink}{Gardasee}{}\ledrightnote{\textcolor{pink}{Lago di Garda}} genügt mir davor wirklich nicht.
               Im Übrigen wissen Sie ja, wie es mit meinen Plänen geht. Von zwanzig projektirten
               Reisen werden zwei verwirklicht. Am 1. Septbr. bin ich
               jedenfalls in \textcolor{pink}{Wien}{}\ledrightnote{\textcolor{pink}{Wien}}. Vorher zwei, drei, Tage \textcolor{pink}{Semmering}{}\ledrightnote{\textcolor{pink}{Semmering}} oder \textcolor{pink}{Schneeberg}{}\ledrightnote{\textcolor{pink}{Schneeberg}}.\pend
           
\pstart
           Auf Wiedersehen, und viele herzliche Grüße von \textcolor{blue}{uns}{}\ledrightnote{{$\rightarrow$}\textcolor{blue}{Ottilie Salten}} zu Ihnen. Schreiben Sie mir bald
               wieder. {\\[\baselineskip]}Aufrichtig {\\[\baselineskip]}Ihr {\\[\baselineskip]}\spacefill\mbox{Salten}\pend
           \leftskip=0em{}
\pstart
           \noindent{}Hier das \label{K_L03510-7v}\edtext{\textcolor{green}{Feuill.}{}\ledrightnote{{$\rightarrow$}\textcolor{green}{Der Wiener Korrespondent}}}{\lemma{\textnormal{\emph{Feuill.}}}\Cendnote{\textnormal{\textcolor{blue}{Felix Salten}: \emph{\textcolor{green}{Der Wiener Korrespondent}}. In: \emph{\textcolor{green}{Der Morgen}}, Jg. 1, H. 4, 5. 7. 1907, S. 113–116. Vgl. Arthur Schnitzler an Felix Salten, 5. 8. 1907.}}}\label{K_L03510-7h} aus dem »\textcolor{green}{Morgen}{}\ledrightnote{\textcolor{green}{Morgen. Wochenschrift für deutsche Kultur}}« das Sie wünschten. Die \label{K_L03510-8v}\edtext{»\textcolor{green}{engl. Reise}{}\ledrightnote{\textcolor{green}{?? [Englische Reise]}}«}{\lemma{\textnormal{\emph{»engl. Reise«}}}\Cendnote{\textnormal{nicht ermittelt; womöglich die in \textcolor{blue}{Schnitzler}s Brief vom 5. 8. 1907 erwähnte
                     Feuilletonsammlung oder ein Teil davon?}}}\label{K_L03510-8h} suche ich selbst schon seit
                  Monaten vergebens. Sonst hätten Sie sie schon. \textcolor{blue}{\textcolor{green}{Pötzl}{}\ledrightnote{{$\rightarrow$}\textcolor{green}{Das gelobte Wien}}}{}\ledrightnote{\textcolor{blue}{Eduard Pötzl}} habe ich nicht zur Hand.\pend
           \endnumbering\briefempfaengerindex{Schnitzler, Arthur@\textsc{Schnitzler, Arthur}!zzzSalten, Felix@\emph{von Felix Salten}!1907-08-151@{15. 8. 1907}|)be}\mylabel{h}  \normalsize

\doendnotes{C}
\bigskip
\vfill

\clearpage

\footnotesize

\lohead{\textsc{register}}

% Definiere theindex-Environment komplett neu ohne reledmac
\makeatletter
\renewenvironment{theindex}{%
  \section*{\indexname}%
  \setlength{\parindent}{0pt}%
  \setlength{\parskip}{0pt plus 0.3pt}%
  \let\item\@idxitem
}{%
  \clearpage
}
\makeatother

\IfFileExists{\jobname-pw.ind}{\input{\jobname-pw.ind}}{}

\end{document}

      