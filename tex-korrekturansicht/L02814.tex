%% latex-korrekturansicht-vorspann.tex
%% Vorspann für die Korrekturansicht.
%% Lädt die gemeinsame Datei latex-vorspann.tex mit gesetztem Schalter.

\newif\ifkorrekturansicht
\korrekturansichttrue

\input{../tex-inputs/latex-vorspann}


               \section[ Paul Goldmann an Arthur Schnitzler, 15. 6. {[}1897{]}]{Paul Goldmann an Arthur Schnitzler, 15. 6. {[}1897{]}}\nopagebreak\mylabel{v}\rehead{ }\normalsize\beginnumbering\briefempfaengerindex{Schnitzler, Arthur@\textsc{Schnitzler, Arthur}!zzzGoldmann, Paul@\emph{von Paul Goldmann}!1897-06-152@{15. 6. {[}1897{]}}|(be} \toendnotes[C]{\smallbreak\pagebreak[2]} \Standort{DLA, A:Schnitzler, HS.NZ85.1.3167.}
\physDesc{Brief, 1 Blatt, 3 Seiten
\newline{}Handschrift: blaue Tinte, deutsche Kurrent
\newline{}Schnitzler: 1) mit Bleistift das Jahr »97« vermerkt 2) mit rotem Buntstift zwei Unterstreichungen}\toendnotes[C]{\smallbreak}\pstart
           \noindent{}{\pb}\textcolor{gray}{\textbf{\textbf{\textcolor{brown}{Frankfurter Zeitung}{}\ledrightnote{\textcolor{brown}{Frankfurter Zeitung}}}}}\pend
           \pstart
           \textcolor{gray}{\textbf{(\textcolor{brown}{\begin{otherlanguage}{french}Gazette de Francfort\end{otherlanguage}}{}\ledrightnote{\textcolor{brown}{Frankfurter Zeitung}}).}}\pend
           \pstart
           \textcolor{gray}{\textbf{\textbf{\begin{otherlanguage}{french}Fondateur M.\end{otherlanguage}{ }\textcolor{blue}{L. Sonnemann}{}\ledrightnote{\textcolor{blue}{Leopold Sonnemann}}.}}}\pend
           \pstart
           \begin{otherlanguage}{french}\textcolor{gray}{\textbf{Journal politique, financier,}}\end{otherlanguage}\pend
           \pstart
           \begin{otherlanguage}{french}\textcolor{gray}{\textbf{commercial et littéraire.}}\end{otherlanguage}\pend
           \pstart
           \begin{otherlanguage}{french}\textcolor{gray}{\textbf{\textbf{Paraissant trois fois par jour.}}}\end{otherlanguage}\hfill \textsc{\textcolor{pink}{Paris}{}\ledrightnote{\textcolor{pink}{Paris}}}, 15. Juni.\pend
           \pstart
           \begin{otherlanguage}{french}\textcolor{gray}{\textbf{\textbf{Bureau à \textcolor{pink}{Paris}{}\ledrightnote{\textcolor{pink}{Paris}}}}}\end{otherlanguage}\pend
           \pstart
           \begin{otherlanguage}{french}\textcolor{gray}{\textbf{\textbf{\textcolor{pink}{10 Rue de la Bourse}{}\ledrightnote{\textcolor{pink}{rue de la Bourse}}.}}}\end{otherlanguage}\pend
           \pstart\center{}Mein lieber Freund,\pend\pstart
           Ich wollte Dir immerfort ſchon ſchreiben; aber ich habe wieder ſo hunderterlei zu
               thun gehabt, und von Tag zu Tage mußte ich das Project verſchieben, bis endlich Dein
               Brief kam.\pend
           \pstart
           In der erſten Zeit nach Deiner Abreiſe haſt Du mir an allen Ecken und Enden gefehlt.
               Nur ſchwer habe ich mich wieder an das Alleinſein mit \strikeout{f\textcolor{gray}{re}mde} all’ den fremden Menſchen gewöhnen können.\pend
           \pstart
           Geſtern habe ich endlich auch eine halbe Stunde Zeit
               gefunden, um zu \textsc{Madame 
                  \textcolor{blue}{Marni}{}\ledrightnote{\textcolor{blue}{Jeanne Marni}}} zu gehen. Sie ſprach ſehr warm von Dir und {\pb}hat Dich offenbar \label{K_L02814-1v}\edtext{ſehr gut
                  verſtanden}{\lemma{\textnormal{\emph{ſehr gut
                  verſtanden}}}\Cendnote{\textnormal{\textcolor{blue}{Schnitzler} traf \textcolor{blue}{Jeanne Marni} gemeinsam mit deren Tochter \textcolor{blue}{Emmy Fournier} und \textcolor{blue}{Goldmann} am 14. 5. 1897. \textcolor{blue}{Schnitzler}
                  notierte im \emph{\textcolor{green}{Tagebuch}} einen äußerst positiven
                  Eindruck von ihr. Am 19. 5. 1897 traf er sie mit \textcolor{blue}{Goldmann} und \textcolor{blue}{Paul Hermann} noch
                  einmal.}}}\label{K_L02814-1h}. Deine Roſen haben ſie ſehr entzückt. Sie hätte Dir gern gedankt,
               wenn ſie Deine Adreſſe gewußt hätte.\pend
           \pstart
           Daß ich Ende Juli nicht fortkann, iſt ſo gut wie ſicher.
               Ich muß jetzt auch mit der \label{K_L02814-2v}\edtext{\textcolor{pink}{ruſſ}{}\ledrightnote{→\textcolor{pink}{Russland}}iſchen Reiſe des \textcolor{blue}{Präſident}{}\ledrightnote{→\textcolor{blue}{Félix Faure}}en}{\lemma{\textnormal{\emph{ruſſiſchen … Präſidenten}}}\Cendnote{\textnormal{Nachdem \textcolor{blue}{Nikolaus II.}{ }\textcolor{pink}{Frankreich} im Vorjahr besucht hatte, reiste
                  der \textcolor{pink}{fran}zösische Präsident
                     \textcolor{blue}{Félix Faure} auf Einladung des \textcolor{blue}{Zaren} im August 1897 nach \textcolor{pink}{Russland}.}}}\label{K_L02814-2h} rechnen, während deren ich in \textsc{\textcolor{pink}{Paris}{}\ledrightnote{\textcolor{pink}{Paris}}} bleiben muß wegen möglicher Zwiſchenfälle. Könnteſt Du Dir es nicht ſo
               einrichten, daß Du \uline{Mitte}{ }Auguſt auf 8 bis 10 Tage nach \label{K_L02814-3v}\edtext{\textsc{\textcolor{pink}{Ischl}{}\ledrightnote{\textcolor{pink}{Bad Ischl}}}}{\lemma{\textnormal{\emph{Ischl}}}\Cendnote{\textnormal{Im August 1897 war \textcolor{blue}{Schnitzler} von
                     19. 8. 1897 bis
                     30. 8. 1897 in
                     \textcolor{pink}{Bad Ischl}. \textcolor{blue}{Goldmann} war zu dieser Zeit auch dort.}}}\label{K_L02814-3h} kommſt? Wenn
               nicht, ſo werde ich wohl kaum mich dorthin begeben. Immerhin iſt das Alles noch nicht
               endgiltig. Meine definitiven Dispoſitionen hängen vom Gang der Ereigniſſe ab.\pend
           \pstart
           An meine \textcolor{blue}{Mutter}{}\ledrightnote{→\textcolor{blue}{Clementine Goldmann}} habe ich {\pb}mindeſtens dreimal geſchrieben, daß ſie Dir den \textsc{\textcolor{blue}{Nansen}{}\ledrightnote{\textcolor{blue}{Peter Nansen}}schen}\label{K_L02814-4v}\edtext{\textcolor{green}{Artikel}{}\ledrightnote{→\textcolor{green}{?? [Artikel von Peter Nansen, Mai/Juni 1897]}}}{\lemma{\textnormal{\emph{Artikel}}}\Cendnote{\textnormal{nicht nachgewiesen}}}\label{K_L02814-4h} ſchicken möge. Hoffentlich haſt Du ihn jetzt endlich erhalten.\pend
           \pstart
           Daß Frau \textsc{\textcolor{blue}{Olga}{}\ledrightnote{\textcolor{blue}{Olga Waissnix}}} die ſchwere \label{K_L02814-7v}\edtext{Operation}{\lemma{\textnormal{\emph{Operation}}}\Cendnote{\textnormal{\textcolor{blue}{Olga Waissnix} wurde im Mai 1897 zwei Mal im \textcolor{pink}{Sanatorium Loew}
                  in der \textcolor{pink}{Mariannengasse} operiert. Die erste
                  Operation, bei der womöglich eine Krebserkrankung festgestellt wurde, fand am
                     16. 5. 1897 statt, die zweite am 25. 5. 1897. Ihr wurden die Gebärmutter sowie die
                  Eierstöcke entfernt. Vgl. \emph{Arthur Schnitzler, Olga Waissnix. Liebe, die starb vor der
                        Zeit. Ein Briefwechsel}. Mit einem Vorwort von Hans Weigel. Hg. von
                     Therese Nickl und Heinrich Schnitzler. Wien,
                     München, Zürich: \emph{Fritz
                        Molden}{ }1970, S. 322, 324 und Elisabeth-Joe
                     Harriet: \emph{Die unvollendete Geliebte. Olga Waissnix {\kaufmannsund} Arthur
                        Schnitzler}. Wien:
                        \emph{Almathea}{ }2015, S. 369–371. [E-Book]}}}\label{K_L02814-7h} glücklich
               überſtanden hat, freut mich von Herzen. Es iſt ſchön, daß ſie ſich meiner noch
                  \label{K_L02814-9v}\edtext{erinnert}{\lemma{\textnormal{\emph{erinnert}}}\Cendnote{\textnormal{Im
                   Brief von \textcolor{blue}{Olga Waissnix} an \textcolor{blue}{Schnitzler} vom 13. 5. 1897 bat sie ihn,
                     \textcolor{blue}{Goldmann} zu grüßen. Vgl. 
                     \emph{Arthur Schnitzler, Olga Waissnix. Liebe, die starb vor der
                        Zeit. Ein Briefwechsel}. Mit einem Vorwort von Hans Weigel. Hg. von
                     Therese Nickl und Heinrich Schnitzler. Wien,
                     München, Zürich: \emph{Fritz
                        Molden}{ }1970, S. 322.}}}\label{K_L02814-9h}. Empfiehl’ mich ihr,
               bitte, und ſag’ ihr, ſie ſolle \strikeout{\textcolor{gray}{e}} eine \label{K_L02814-11v}\edtext{Reconvalescenz}{\lemma{\textnormal{\emph{Reconvalescenz}}}\Cendnote{\textnormal{Genesung}}}\label{K_L02814-11h}-Reiſe nach \textsc{\textcolor{pink}{Paris}{}\ledrightnote{\textcolor{pink}{Paris}}} machen.\pend
           \pstart
           Die Klatſcherei von \label{K_L02814-12v}\edtext{\textsc{\textcolor{blue}{M. B.}{}\ledrightnote{→\textcolor{blue}{Hermine von Schaffgotsch}}}}{\lemma{\textnormal{\emph{M. B.}}}\Cendnote{\textnormal{\textcolor{blue}{Schnitzler} erfuhr am 8. 6. 1897, dass \textcolor{blue}{Minnie Benedict} in Anwesenheit verschiedener anderer Leute
                  »erzählte, dass ich mit einem \textcolor{blue}{Vorstadtmädel}, wegen \textcolor{blue}{Kind} etc. nach \textcolor{pink}{Paris} gereist«.}}}\label{K_L02814-12h}
               iſt widerwärtig. Oh dieſe \textcolor{pink}{iſrael}{}\ledrightnote{\textcolor{pink}{Israel}}itiſchen
                  Jungfrauen! {\dotsfour}\pend
           \pstart
           Ich ſchlafe ſchlecht, bin unzufrieden und mißmuthig{\dotsfive}\pend
           \pstart
           Könnteſt Du nicht am 9. oder 11. Auguſt zum \label{K_L02814-56v}\edtext{\textsc{\textcolor{green}{Parsifal}{}\ledrightnote{\textcolor{green}{Parsifal}}} nach \textsc{\textcolor{pink}{Bayreuth}{}\ledrightnote{\textcolor{pink}{Bayreuth}}}}{\lemma{\textnormal{\emph{Parsifal nach Bayreuth}}}\Cendnote{\textnormal{Dazu kam es nicht. Die \emph{\textcolor{brown}{Bayreuther Festspiele}} wurden 1897 von \textcolor{blue}{Cosima Wagner} geleitet.}}}\label{K_L02814-56h} kommen?\pend
           \pstart
           Grüße Deine \textcolor{blue}{Freundin}{}\ledrightnote{→\textcolor{blue}{Marie Reinhard}} recht herzlich und ſei Du ſelbſt vielmals gegrüßt von
               {\\[\baselineskip]}Deinem treuen \spacefill\mbox{Paul Goldm}\pend
           \leftskip=0em{}\endnumbering\briefempfaengerindex{Schnitzler, Arthur@\textsc{Schnitzler, Arthur}!zzzGoldmann, Paul@\emph{von Paul Goldmann}!1897-06-152@{15. 6. {[}1897{]}}|)be}\mylabel{h}\begin{anhang}\end{anhang}\normalsize

\doendnotes{C}
\bigskip
\vfill

\clearpage

\footnotesize

\lohead{\textsc{register}}

% Definiere theindex-Environment komplett neu ohne reledmac
\makeatletter
\renewenvironment{theindex}{%
  \section*{\indexname}%
  \setlength{\parindent}{0pt}%
  \setlength{\parskip}{0pt plus 0.3pt}%
  \let\item\@idxitem
}{%
  \clearpage
}
\makeatother

\IfFileExists{\jobname-pw.ind}{\input{\jobname-pw.ind}}{}

\end{document}

      