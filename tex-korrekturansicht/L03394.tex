%% latex-korrekturansicht-vorspann.tex
%% Vorspann für die Korrekturansicht.
%% Lädt die gemeinsame Datei latex-vorspann.tex mit gesetztem Schalter.

\newif\ifkorrekturansicht
\korrekturansichttrue

\input{../tex-inputs/latex-vorspann}


\renewcommand{\erwaehntePersonen}{Personen: Peter Altenberg, Ottilie Salten, Paul Salten, Olga Schnitzler}
\renewcommand{\erwaehnteOrte}{Orte: Edmund-Weiß-Gasse 7, I., Innere Stadt, Kahlenberg, Riedhof, Semmering, Wien, XVIII., Währing}
\renewcommand{\erwaehnteWerke}{Werke: Die Zeit, Mattachich}
\section[ Felix Salten an Arthur Schnitzler, 30. 3. 1904]{Felix Salten an Arthur Schnitzler, 30. 3. 1904}
\nopagebreak\mylabel{v}
\rehead{ }\normalsize\beginnumbering\briefempfaengerindex{Schnitzler, Arthur@\textsc{Schnitzler, Arthur}!zzzSalten, Felix@\emph{von Felix Salten}!1904-03-301@{30. 3. 1904}|(be}
\toendnotes[C]{\smallbreak\pagebreak[2]}\Standort{CUL, Schnitzler, B 89, B 1.}
\physDesc{Kartenbrief, 1067 Zeichen
\newline{}Handschrift: schwarze Tinte, lateinische Kurrent
\newline{}Versand: 1) Stempel: »\nobreak{}\oindex{I., Innere Stadt@\textbf{I., Innere Stadt}, \emph{A.ADM3}|pwk}Wien 1/2 C.R. r, 30 III 04, 2 30N\nobreak{}«.   2) Stempel: »\nobreak{}Wien 18/1 11\textcolor{gray}{1}, 30 \textcolor{gray}{III 04}, 3 10N\nobreak{}«. 
\newline{}Schnitzler: mit Bleistift datiert: »30. 3. 904.–« 
\newline{}Ordnung: mit Bleistift von unbekannter Hand nummeriert: »186« }\toendnotes[C]{\smallbreak}\pstart{}{\pb}Herrn D\textsuperscript{r} Arthur Schnitzler\pend{}\pstart{}\textcolor{pink}{Wien XVIII.}{}\ledrightnote{\textcolor{pink}{XVIII., Währing}}\pend{}\pstart{}\textcolor{pink}{Spöttelgaße 7}{}\ledrightnote{\textcolor{pink}{Edmund-Weiß-Gasse 7}}\pend{}
{\bigskip}
\pstart
           \raggedleft{}{\pb}Mittwoch\pend
           
\pstart
           Lieber Freund, vielen Dank für Ihren Brief, über den ich mich sehr
               gefreut habe. Es geht ja oft wunderlich mit diesen kleinen Arbeiten: \label{K_L03394-1v}\edtext{\textcolor{green}{diese}{}\ledrightnote{{$\rightarrow$}\textcolor{green}{Mattachich}}}{\lemma{\textnormal{\emph{diese}}}\Cendnote{\textnormal{\textcolor{blue}{Felix Salten}: \emph{\textcolor{green}{Mattachich}}. In: \emph{\textcolor{green}{Die
                        Zeit}}, Jg. 3, Nr. 538, 27. 3. 1904,
                     Morgenblatt, S. 1–3.}}}\label{K_L03394-1h} letzte mußte ich, schläfrig, müd und eilig,
               in drei Stunden fertigmachen, und wenn wirklich was dran zu loben ist, dann war es
               eben doch wol der »Schmiß« (kann – falls das Wort zu minder erscheint, etwa durch
               »Elan« ersetzt werden). Nicht wenig bin ich über \label{K_L03394-2v}\edtext{\textcolor{blue}{P. A.}{}\ledrightnote{\textcolor{blue}{Peter Altenberg}}}{\lemma{\textnormal{\emph{P. A.}}}\Cendnote{\textnormal{\textcolor{blue}{Peter Altenberg}}}}\label{K_L03394-2h} erschrocken. Habe gleich überall nach ihm gesucht, aber nichts gefunden. Wo
               denn? Dass ich manchmal in Satzmelodien falle, die mir lieb sind, weiß ich, und
               glaube, das hängt mit minder musikalischer Empfänglichkeit zusammen. Aber \textcolor{blue}{A.}{}\ledrightnote{\textcolor{blue}{Peter Altenberg}}’s Sätze waren mir nie angenehm, haben nichts
               in mir dauernd berührt, und ich könnte es mir also nicht erklären.\pend
           
\pstart
           \textcolor{blue}{Otti}{}\ledrightnote{\textcolor{blue}{Ottilie Salten}}, \textcolor{blue}{Paul}{}\ledrightnote{\textcolor{blue}{Paul Salten}} und ich wollen Samstag früh über Ostern
               auf den \textcolor{pink}{Kahlenberg}{}\ledrightnote{\textcolor{pink}{Kahlenberg}}. (\label{K_L03394-3v}\edtext{\textcolor{pink}{Privat-Semmering}{}\ledrightnote{{$\rightarrow$}\textcolor{pink}{Semmering}}}{\lemma{\textnormal{\emph{Privat-Semmering}}}\Cendnote{\textnormal{siehe Felix Salten an Arthur Schnitzler, 27. 11. 1903}}}\label{K_L03394-3h}) Wenn es Ihnen recht ist, \label{K_L03394-4v}\edtext{kommen wir morgen Donnerstag oder übermorgen Freitag um ½ 7–7 zu Ihnen}{\lemma{\textnormal{\emph{kommen … Ihnen}}}\Cendnote{\textnormal{siehe A. S.: \emph{Tagebuch}, 1. 4. 1904}}}\label{K_L03394-4h}. Ich schlage vor, dass wir dann im \textcolor{pink}{Riedhof}{}\ledrightnote{\textcolor{pink}{Riedhof}}
               nachtmahlen.\pend
           
\pstart
           herzlichste Grüße an \textcolor{blue}{Olga}{}\ledrightnote{\textcolor{blue}{Olga Schnitzler}} u. Sie {\\[\baselineskip]}Ihr {\\[\baselineskip]}\spacefill\mbox{Salten}\pend
           \leftskip=0em{}\endnumbering\briefempfaengerindex{Schnitzler, Arthur@\textsc{Schnitzler, Arthur}!zzzSalten, Felix@\emph{von Felix Salten}!1904-03-301@{30. 3. 1904}|)be}\mylabel{h}  \normalsize

\doendnotes{C}
\bigskip
\vfill

\clearpage

\footnotesize

\lohead{\textsc{register}}

% Definiere theindex-Environment komplett neu ohne reledmac
\makeatletter
\renewenvironment{theindex}{%
  \section*{\indexname}%
  \setlength{\parindent}{0pt}%
  \setlength{\parskip}{0pt plus 0.3pt}%
  \let\item\@idxitem
}{%
  \clearpage
}
\makeatother

\IfFileExists{\jobname-pw.ind}{\input{\jobname-pw.ind}}{}

\end{document}

      