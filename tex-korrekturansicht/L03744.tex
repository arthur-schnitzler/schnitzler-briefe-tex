%% latex-korrekturansicht-vorspann.tex
%% Vorspann für die Korrekturansicht.
%% Lädt die gemeinsame Datei latex-vorspann.tex mit gesetztem Schalter.

\newif\ifkorrekturansicht
\korrekturansichttrue

\input{../tex-inputs/latex-vorspann}


\section[Arthur Schnitzler an Stefan Zweig, 28. 5. 1927]{L03744 Arthur Schnitzler an Stefan Zweig, 28. 5. 1927}
\nopagebreak\mylabel{L03744v}
\rehead{ }\normalsize\beginnumbering\briefempfaengerindex{Zweig, Stefan@\textsc{Zweig, Stefan}!zzzSchnitzler, Arthur@\emph{von Arthur Schnitzler}!1927-05-282@{28. 5. 1927}|(be}
\toendnotes[C]{\smallbreak\pagebreak[2]}
\correspDesc{Versand  durch Arthur Schnitzler am 28. 5. 1927 in Wien
\newline{}Erhalt  durch Stefan Zweig im Zeitraum [29. 5. 1927 – 2. 6. 1927?] in Salzburg}\toendnotes[C]{\smallbreak}
\Standort{Jerusalem, National Library of Israel, ARC. Ms. Var. 305 1 58 Stefan Zweig Collection.}
\physDesc{Brief, 1 Blatt, 2 Seiten, 1254 Zeichen
\newline{}Handschrift: schwarze Tinte, lateinische Kurrent}\toendnotes[C]{\smallbreak}
\pstart
           \raggedleft{}{\pb}\textcolor{pink}{Wien}\oindex{Wien@\textbf{Wien}, \emph{Verwaltungsgebiet}|pw}{}\ledrightnote{\textcolor{pink}{Wien}},
                        28. 5. 927\pend
           \vspace{0.5em}
\pstart
           lieber Stefan Zweig, daſs und wie Sie mir bei jeder Gelegenheit Ihre
               Sympathie und Ihre Antheilnahme kundgeben – anläßlich Älterwerdens,
               Novellenschreibens und Nichtaufgeführtwerdens, – rührt mich geradezu und so hab ich
               Ihnen auch für Ihren letzten lieben \label{K_L03744-1v}\edtext{Brief}{\lemma{\textnormal{\emph{Brief}}}\Cendnote{\textnormal{Stefan Zweig an Arthur Schnitzler, 18. 5. 1927.
               }}}\label{K_L03744-1} wärmstens zu danken. Mit Ihrem Bedenken gegen die Höhe des Betrags haben Sie wahrscheinlich recht, wie im
               Fall \textcolor{green}{Else}\pwindex{Schnitzler, Arthur 15. 5. 1862 Wien – 21. 10. 1931 ebd.@\textsc{Schnitzler, Arthur} (15. 5. 1862 Wien – 21. 10. 1931 ebd.), \emph{Schriftsteller, Mediziner}!Fräulein Else@\strich\emph{Fräulein Else}|pw}{}\ledrightnote{\textcolor{green}{Fräulein Else}}; nach der Aufführung des »\textcolor{green}{Gangs}\pwindex{Schnitzler, Arthur 15. 5. 1862 Wien – 21. 10. 1931 ebd.@\textsc{Schnitzler, Arthur} (15. 5. 1862 Wien – 21. 10. 1931 ebd.), \emph{Schriftsteller, Mediziner}!Gang zum Weiher. Dramatische Dichtung@\strich\emph{Der Gang zum Weiher. Dramatische Dichtung}|pw}{}\ledrightnote{\textcolor{green}{Der Gang zum Weiher. Dramatische Dichtung}}« sehn ich mich, unter den gegenwärtigen
               Umständen, selbst nicht sonderlich; – und \strikeout{daſs} das
               Alter – um nicht zu sagen Altwerden – ist (wie die \textcolor{blue}{Sandrock}\pwindex{Sandrock, Adele 19.\,8.\,1863 Rotterdam – 30.\,8.\,1937 Berlin@\textsc{Sandrock, Adele} (19.\,8.\,1863 Rotterdam – 30.\,8.\,1937 Berlin), \emph{Schauspielerin}|pw}{}\ledrightnote{\textcolor{blue}{Adele Sandrock}} einmal vom Tod behauptet hat) ein Element gegen das sich nichts
               sagen läßt. {\pb}Pathetisch oder resignirt geno{\geminationm}en – unsere
               Erwiderung bleibt immer nur »\textcolor{green}{\label{K_L03744-2v}\edtext{Allons travailler}{\lemma{\textnormal{\emph{Allons travailler}}}\Cendnote{\textnormal{französisch: machen wir uns an die
                     Arbeit. Es handelt sich um die letzten Worte von \emph{\textcolor{green}{L’Œuvre}\pwindex{Zola, Émile 2.\,4.\,1840 Paris – 29.\,9.\,1902 ebd.@\textsc{Zola, Émile} (2.\,4.\,1840 Paris – 29.\,9.\,1902 ebd.), \emph{Schriftsteller, Journalist}!œuvre@\strich\emph{L’œuvre}|pwk}} (1886) von \textcolor{blue}{Émile Zola}\pwindex{Zola, Émile 2.\,4.\,1840 Paris – 29.\,9.\,1902 ebd.@\textsc{Zola, Émile} (2.\,4.\,1840 Paris – 29.\,9.\,1902 ebd.), \emph{Schriftsteller, Journalist}|pwk}.}}}\label{K_L03744-2}}\pwindex{Zola, Émile 2.\,4.\,1840 Paris – 29.\,9.\,1902 ebd.@\textsc{Zola, Émile} (2.\,4.\,1840 Paris – 29.\,9.\,1902 ebd.), \emph{Schriftsteller, Journalist}!œuvre@\strich\emph{L’œuvre}|pwv}{}\ledrightnote{{$\rightarrow$}\emph{\textcolor{green}{L’œuvre}}}« (\textcolor{blue}{wer}\pwindex{Zola, Émile 2.\,4.\,1840 Paris – 29.\,9.\,1902 ebd.@\textsc{Zola, Émile} (2.\,4.\,1840 Paris – 29.\,9.\,1902 ebd.), \emph{Schriftsteller, Journalist}|pwv}{}\ledrightnote{{$\rightarrow$}\emph{\textcolor{blue}{Émile Zola}}} hat es nur gesagt?)\pend
           
\pstart
           Ich bleibe vorläufig in \textcolor{pink}{Wien}\oindex{Wien@\textbf{Wien}, \emph{Verwaltungsgebiet}|pw}{}\ledrightnote{\textcolor{pink}{Wien}} (we{\geminationn} nicht das
               Wetter zu ausgedehnten Ausflügen locken sollte) vor dem So{\geminationm}er noch, Sie haben es
               wohl gelesen, \label{K_L03744-3v}\edtext{heiratet meine \textcolor{blue}{Tochter}\pwindex{Cappellini, Lili 13.\,9.\,1909 Wien – 26.\,7.\,1928 Venedig@\textsc{Cappellini, Lili} (13.\,9.\,1909 Wien – 26.\,7.\,1928 Venedig)|pwv}{}\ledrightnote{{$\rightarrow$}\emph{\textcolor{blue}{Lili Cappellini}}}}{\lemma{\textnormal{\emph{heiratet meine Tochter}}}\Cendnote{\textnormal{Vgl. A. S.: \emph{Kulturveranstaltungen}, 30. 6. 1927.}}}\label{K_L03744-3} nach \textcolor{pink}{Venedig}\oindex{Venedig@\textbf{Venedig}|pw}{}\ledrightnote{\textcolor{pink}{Venedig}} (die \textcolor{pink}{Wohnung}\oindex{Campo Sant’Agostin 2545@\textbf{Campo Sant’Agostin 2545}, \emph{Wohngebäude}|pwv}{}\ledrightnote{{$\rightarrow$}\emph{\textcolor{pink}{Campo Sant’Agostin 2545}}} dort, in \textcolor{pink}{Fari}\oindex{Santa Maria Gloriosa dei Frari@\textbf{Santa Maria Gloriosa dei Frari}, \emph{Kirche}|pw}{}\ledrightnote{\textcolor{pink}{Santa Maria Gloriosa dei Frari}}-Nähe steht schon bereit) die Eintheilung
               meiner »Ferien« (die oft meine beste Arbeitszeit sind) wird dazu ein wenig abhängen.
               Noch steht mein Progra{\geminationm} nicht fest – in jedem Fall hoff ich wir begegnen einander
               bald wieder – es ist mir immer eine Freude wie Sie wissen.\pend
           
\pstart
           Herzlichst grüßt Sie Ihr{\\[\baselineskip]}\spacefill\mbox{ArthurSchnitzler}\pend
           \leftskip=0em{}\selectlanguage{ngerman}\endnumbering\briefempfaengerindex{Zweig, Stefan@\textsc{Zweig, Stefan}!zzzSchnitzler, Arthur@\emph{von Arthur Schnitzler}!1927-05-282@{28. 5. 1927}|)be}\mylabel{L03744h}  \normalsize

\doendnotes{C}
\bigskip
\vfill

\clearpage

\footnotesize

\lohead{\textsc{register}}

% Definiere theindex-Environment komplett neu ohne reledmac
\makeatletter
\renewenvironment{theindex}{%
  \section*{\indexname}%
  \setlength{\parindent}{0pt}%
  \setlength{\parskip}{0pt plus 0.3pt}%
  \let\item\@idxitem
}{%
  \clearpage
}
\makeatother

\IfFileExists{\jobname-pw.ind}{\input{\jobname-pw.ind}}{}

\end{document}

      