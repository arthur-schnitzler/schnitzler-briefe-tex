%% latex-korrekturansicht-vorspann.tex
%% Vorspann für die Korrekturansicht.
%% Lädt die gemeinsame Datei latex-vorspann.tex mit gesetztem Schalter.

\newif\ifkorrekturansicht
\korrekturansichttrue

\input{../tex-inputs/latex-vorspann}


\renewcommand{\erwaehntePersonen}{Personen: Felix Salten}
\renewcommand{\erwaehnteInstitutionen}{Institutionen: Paul Zsolnay Verlag}
\renewcommand{\erwaehnteOrte}{Orte: Berlin, Leipzig, Wien}
\renewcommand{\erwaehnteWerke}{Werke: Geister der Zeit. Erlebnisse}
\section[Felix Salten: Widmungsexemplar Geister der Zeit für Arthur Schnitzler, Februar 1925]{Felix Salten: Widmungsexemplar Geister der Zeit für Arthur Schnitzler,
               Februar 1925}
\nopagebreak\mylabel{v}
\rehead{ }\normalsize\beginnumbering\briefempfaengerindex{Schnitzler, Arthur@\textsc{Schnitzler, Arthur}!zzzSalten, Felix@\emph{von Felix Salten}!1925-02-011@{Februar 1925}|(be}
\toendnotes[C]{\smallbreak\pagebreak[2]}\Standort{DLA, G:Schnitzler, Arthur (Sammlung Heinrich Schnitzler).}
\physDesc{Widmung am Titelblatt, 60 Zeichen
\newline{}Handschrift: schwarze Tinte, lateinische Kurrent}
\pstart
           \noindent{}\centering{}{\pb}\textcolor{gray}{\textbf{FELIX SALTEN}}\pend
           
\pstart
           \noindent{}\centering{}\textcolor{gray}{\textbf{\textcolor{green}{GEISTER DER ZEIT}{}\ledrightnote{\textcolor{green}{Geister der Zeit. Erlebnisse}}}}\pend
           
\pstart
           \noindent{}\centering{}\textcolor{gray}{\textbf{ERLEBNISSE}}\pend
           {\bigskip}
\pstart
           \noindent{}Arthur Schnitzler\pend
           \leftskip=3em{}
\pstart
           \noindent{}in aller Herzlichkeit {\\}\spacefill\mbox{Felix Salten}\pend
           \leftskip=0em{}
\pstart
           Feber 25\pend
           {\bigskip}
\pstart
           \noindent{}\centering{}\textcolor{gray}{\textbf{1924}}\pend
           
\pstart
           \noindent{}\centering{}\textcolor{gray}{\textbf{\textcolor{brown}{PAUL ZSOLNAY VERLAG}{}\ledrightnote{\textcolor{brown}{Paul Zsolnay Verlag}}}}\pend
           
\pstart
           \noindent{}\centering{}\textcolor{gray}{\textbf{\textcolor{pink}{BERLIN}{}\ledrightnote{\textcolor{pink}{Berlin}} / \textcolor{pink}{WIEN}{}\ledrightnote{\textcolor{pink}{Wien}} / \textcolor{pink}{LEIPZIG}{}\ledrightnote{\textcolor{pink}{Leipzig}}}}\pend
           \endnumbering\briefempfaengerindex{Schnitzler, Arthur@\textsc{Schnitzler, Arthur}!zzzSalten, Felix@\emph{von Felix Salten}!1925-02-011@{Februar 1925}|)be}\mylabel{h}  \normalsize

\doendnotes{C}
\bigskip
\vfill

\clearpage

\footnotesize

\lohead{\textsc{register}}

% Definiere theindex-Environment komplett neu ohne reledmac
\makeatletter
\renewenvironment{theindex}{%
  \section*{\indexname}%
  \setlength{\parindent}{0pt}%
  \setlength{\parskip}{0pt plus 0.3pt}%
  \let\item\@idxitem
}{%
  \clearpage
}
\makeatother

\IfFileExists{\jobname-pw.ind}{\input{\jobname-pw.ind}}{}

\end{document}

      