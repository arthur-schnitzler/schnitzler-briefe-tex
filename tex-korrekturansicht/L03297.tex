%% latex-korrekturansicht-vorspann.tex
%% Vorspann für die Korrekturansicht.
%% Lädt die gemeinsame Datei latex-vorspann.tex mit gesetztem Schalter.

\newif\ifkorrekturansicht
\korrekturansichttrue

\input{../tex-inputs/latex-vorspann}


\renewcommand{\erwaehntePersonen}{Personen: Richard Beer-Hofmann, Houston Stewart Chamberlain, Theodor Herzl, Ottilie Salten, Franz Servaes, Karl von Thaler, Jakob Wassermann}
\renewcommand{\erwaehnteInstitutionen}{Institutionen: Bruckmann Verlag, Neue Freie Presse}
\renewcommand{\erwaehnteOrte}{Orte: Bad Ischl, Dubrovnik, München, Seeboden, Wien}
\renewcommand{\erwaehnteWerke}{Werke: Decadence-Romane, Die Grundlagen des Neunzehnten Jahrhunderts. 2 Bde., Die Welt (Wien), Neue Freie Presse, Tagebuch, »Das fremde Volk«, »Das fremde Volk«. I., »Das fremde Volk«. II., »Das fremde Volk«. III.}
\section[ Felix Salten an Arthur Schnitzler, {[}17. 8. 1899{]}]{Felix Salten an Arthur Schnitzler, {[}17. 8. 1899{]}}
\nopagebreak\mylabel{v}
\rehead{ }\normalsize\beginnumbering\briefempfaengerindex{Schnitzler, Arthur@\textsc{Schnitzler, Arthur}!zzzSalten, Felix@\emph{von Felix Salten}!1899-08-172@{{[}17. 8. 1899{]}}|(be}
\toendnotes[C]{\smallbreak\pagebreak[2]}\Standort{CUL, Schnitzler, B 89, A 2.}
\physDesc{Karte, 1172 Zeichen
\newline{}Handschrift: Bleistift, lateinische Kurrent
\newline{}Schnitzler: mit Bleistift datiert: »17/8 99.« 
\newline{}Ordnung: mit Bleistift von unbekannter Hand nummeriert: »121« }\toendnotes[C]{\smallbreak}
\pstart
           \noindent{}{\pb}Lieber Freund, den Gedanken an eine Radtour scheinen Sie selbst
               aufgegeben zu haben, – nun, ich hätte auch nur sehr schwer abkommen können, und es
               ist mir ganz recht. Sind Sie dafür im September oder
               halben October vielleicht für \label{K_L03297-1v}\edtext{\textcolor{pink}{Ragusa}{}\ledrightnote{\textcolor{pink}{Dubrovnik}}}{\lemma{\textnormal{\emph{Ragusa}}}\Cendnote{\textnormal{nicht geschehen}}}\label{K_L03297-1h} zu haben? Ich
               möchte gerne auf acht Tage dahin gehen. In der nächsten Woche ko{\geminationm}e ich vermutlich auf einen od. zwei Tage nach \label{K_L03297-2v}\edtext{\textcolor{pink}{Ischl}{}\ledrightnote{\textcolor{pink}{Bad Ischl}}}{\lemma{\textnormal{\emph{Ischl}}}\Cendnote{\textnormal{\textcolor{blue}{Salten} kam am 22. 8. 1899 in \textcolor{pink}{Ischl} an.}}}\label{K_L03297-2h}. Ich zeige Ihnen das
               jedenfalls noch genau an. Haben Sie heute das \textcolor{green}{Feuilleton}{}\ledrightnote{{$\rightarrow$}\textcolor{green}{Decadence-Romane}} von \textcolor{blue}{Franz Servaes}{}\ledrightnote{\textcolor{blue}{Franz Servaes}} gelesen? »\label{K_L03297-3v}\edtext{\textcolor{green}{Decadence Romane}{}\ledrightnote{\textcolor{green}{Decadence-Romane}}}{\lemma{\textnormal{\emph{Decadence Romane}}}\Cendnote{\textnormal{\textcolor{blue}{Franz Servaes}: \emph{\textcolor{green}{Decadence-Romane}}. In: \emph{\textcolor{green}{Neue Freie Presse}}, Nr. 12.566, 17. 8. 1899, Morgenblatt, S. 1–3.}}}\label{K_L03297-3h}« – – Die \textcolor{brown}{Neue freie Presse}{}\ledrightnote{\textcolor{brown}{Neue Freie Presse}} brauchte für den alternden \textcolor{blue}{Karl v. Thaler}{}\ledrightnote{\textcolor{blue}{Karl von Thaler}} einen Ersatz und hat ihn in \textcolor{blue}{Servaes}{}\ledrightnote{\textcolor{blue}{Franz Servaes}} gefunden, nur dass mir \textcolor{blue}{Servaes}{}\ledrightnote{\textcolor{blue}{Franz Servaes}} mit seinem Orientirtsein noch eckel{\pb}hafter ist. Wo befindet sich
                  \label{K_L03297-4v}\edtext{\textcolor{blue}{Beer-Hofmann}{}\ledrightnote{\textcolor{blue}{Richard Beer-Hofmann}}}{\lemma{\textnormal{\emph{Beer-Hofmann}}}\Cendnote{\textnormal{\textcolor{blue}{Beer-Hofmann} reiste nach der gemeinsamen
                  Wanderung mit \textcolor{blue}{Schnitzler} und \textcolor{blue}{Jakob Wassermann} wieder nach \textcolor{pink}{Seeboden}.}}}\label{K_L03297-4h} jetzt?\pend
           
\pstart
           \textcolor{blue}{Otti}{}\ledrightnote{\textcolor{blue}{Ottilie Salten}} ist in \textcolor{pink}{Ischl}{}\ledrightnote{\textcolor{pink}{Bad Ischl}}. Wahrscheinlich haben Sie sie schon \label{K_L03297-5v}\edtext{gesehen}{\lemma{\textnormal{\emph{gesehen}}}\Cendnote{\textnormal{\textcolor{blue}{Schnitzler} war seit 15. 8. 1899 in \textcolor{pink}{Ischl}. Eine Begegnung mit \textcolor{blue}{Ottilie Metzl} ist in seinem  \emph{\textcolor{green}{Tagebuch}} nur gemeinsam mit \textcolor{blue}{Salten} am 24. 8. 1899 festgehalten.}}}\label{K_L03297-5h}. Sie hat noch
                  \label{K_L03297-6v}\edtext{kein Engagement}{\lemma{\textnormal{\emph{kein Engagement}}}\Cendnote{\textnormal{vgl. Felix Salten an Arthur Schnitzler, 28. 4. 1899}}}\label{K_L03297-6h}, ist aber im Ganzen ruhiger. Ich bin die ganze Zeit
               schlecht aufgelegt, aber ich arbeite viel. \label{K_L03297-7v}\edtext{»\textcolor{green}{Die Grundlagen des
                  Jahrhunderts}{}\ledrightnote{\textcolor{green}{Die Grundlagen des Neunzehnten Jahrhunderts. 2 Bde.}}« von \textcolor{blue}{Chamberlain}{}\ledrightnote{\textcolor{blue}{Houston Stewart Chamberlain}}}{\lemma{\textnormal{\emph{»Die … Chamberlain}}}\Cendnote{\textnormal{\textcolor{blue}{Houston Stewart Chamberlain}: \emph{\textcolor{green}{Die Grundlagen des Neunzehnten Jahrhunderts. 2
                        Bde.}}{ }\textcolor{pink}{München}: \emph{\textcolor{brown}{Verlagsanstalt F. Bruckmann A.-G.}}{ }1899. Eine Lektüre durch \textcolor{blue}{Schnitzler} ist
                  nicht nachweisbar.}}}\label{K_L03297-7h} ist ein sehr interessantes Buch. Ich gebe es Ihnen, wenn
               Sie zurückkommen. Ich schreibe augenblicklich darüber eine Anzahl von \label{K_L03297-8v}\edtext{\textcolor{green}{Entgegnungen}{}\ledrightnote{{$\rightarrow$}\textcolor{green}{»Das fremde Volk«. I.}{\newline}{$\rightarrow$}\textcolor{green}{»Das fremde Volk«. II.}{\newline}{$\rightarrow$}\textcolor{green}{»Das fremde Volk«. III.}}}{\lemma{\textnormal{\emph{Entgegnungen}}}\Cendnote{\textnormal{Die Reihe \textcolor{green}{»Das fremde Volk«} erschien in den
                  Nummern 35–37 des dritten Jahrgangs von \textcolor{blue}{Theodor
                     Herzl}s zionistischer Zeitschrift \emph{\textcolor{green}{Die
                     Welt}}: \textcolor{blue}{F. S.} [ = \textcolor{blue}{Felix Salten}]: \emph{\textcolor{green}{»Das fremde Volk«. I}}. In: \emph{\textcolor{green}{Die Welt}}, Jg. 3, Nr. 35, 1. 9. 1899, S. 6–7; \textcolor{blue}{F. S.}: \emph{\textcolor{green}{»Das fremde Volk«. II}}. In: \emph{\textcolor{green}{Die Welt}}, Jg. 3, Nr. 36, 8. 9. 1899, S. 13–14 und \textcolor{blue}{F. S.}: \emph{\textcolor{green}{»Das fremde Volk«. III}}. In: \emph{\textcolor{green}{Die Welt}}, Jg. 3, Nr. 37, 15. 9. 1899, S. 13–14.}}}\label{K_L03297-8h} für »\textcolor{green}{Die Welt}{}\ledrightnote{\textcolor{green}{Die Welt (Wien)}}«.\pend
           
\pstart
           Senden Sie mir bald wieder eine Zeile. – Die Zeitungen bringe ich Ihnen selbst
               mit.\pend
           
\pstart
           Herzlichst Ihr {\\[\baselineskip]}\spacefill\mbox{Salten}\pend
           \leftskip=0em{}\endnumbering\briefempfaengerindex{Schnitzler, Arthur@\textsc{Schnitzler, Arthur}!zzzSalten, Felix@\emph{von Felix Salten}!1899-08-172@{{[}17. 8. 1899{]}}|)be}\mylabel{h}  \normalsize

\doendnotes{C}
\bigskip
\vfill

\clearpage

\footnotesize

\lohead{\textsc{register}}

% Definiere theindex-Environment komplett neu ohne reledmac
\makeatletter
\renewenvironment{theindex}{%
  \section*{\indexname}%
  \setlength{\parindent}{0pt}%
  \setlength{\parskip}{0pt plus 0.3pt}%
  \let\item\@idxitem
}{%
  \clearpage
}
\makeatother

\IfFileExists{\jobname-pw.ind}{\input{\jobname-pw.ind}}{}

\end{document}

      