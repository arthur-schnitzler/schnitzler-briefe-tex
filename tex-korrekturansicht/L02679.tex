%% latex-korrekturansicht-vorspann.tex
%% Vorspann für die Korrekturansicht.
%% Lädt die gemeinsame Datei latex-vorspann.tex mit gesetztem Schalter.

\newif\ifkorrekturansicht
\korrekturansichttrue

\input{../tex-inputs/latex-vorspann}


               \section[Paul Goldmann an Arthur Schnitzler, {[}21. 3. 1899?{]}]{ Paul Goldmann an Arthur Schnitzler, {[}21. 3. 1899?{]}}\nopagebreak\mylabel{v}\rehead{ }\normalsize\beginnumbering\briefempfaengerindex{Schnitzler, Arthur@\textsc{Schnitzler, Arthur}!zzzGoldmann, Paul@\emph{von Paul Goldmann}!1899-03-211@{{[}21. 3. 1899?{]}}|(be} \toendnotes[C]{\smallbreak\pagebreak[2]} \Standort{DLA, A:Schnitzler, HS.NZ85.1.3169.}
\physDesc{Telegramm
\newline{}maschinell
\newline{}Schnitzler: mit Bleistift datiert: »März 99« \newline{}Ordnung: beschnitten }\toendnotes[C]{\smallbreak}\pstart
           \noindent{}\centering{}{\pb}fr \textcolor{pink}{frankfurtmain}{}\ledrightnote{\textcolor{pink}{Frankfurt am Main}}
               9+ 73219 \label{K_L02679-1v}\edtext{21 3}{\lemma{\textnormal{\emph{21 3}}}\Cendnote{\textnormal{Inhaltlich dürfte sich das Telegramm
                  auf die geplante Mitarbeit \textcolor{blue}{Goldmann}s bei der
                  \emph{\textcolor{brown}{Neuen Freien Presse}} beziehen. Der Brief {XXXX ref}, in
                  dem er ausführlicher auf die Unsicherheit eingeht, dürfte nach diesem Telegramm verfasst sein.}}}\label{K_L02679-1h}1{ }1 20=\pend
           \pstart
           \noindent{}situation wieder vollstaendig ins schwanken gerathen +\pend
           \pstart
           sobald etwas definitives entschieden schrejbe ich dir =\pend
           \pstart grusz \spacefill\mbox{goldmann +}\pend{}\endnumbering\briefempfaengerindex{Schnitzler, Arthur@\textsc{Schnitzler, Arthur}!zzzGoldmann, Paul@\emph{von Paul Goldmann}!1899-03-211@{{[}21. 3. 1899?{]}}|)be}\mylabel{h}  \normalsize

\doendnotes{C}
\bigskip
\vfill

\clearpage

\footnotesize

\lohead{\textsc{register}}

% Definiere theindex-Environment komplett neu ohne reledmac
\makeatletter
\renewenvironment{theindex}{%
  \section*{\indexname}%
  \setlength{\parindent}{0pt}%
  \setlength{\parskip}{0pt plus 0.3pt}%
  \let\item\@idxitem
}{%
  \clearpage
}
\makeatother

\IfFileExists{\jobname-pw.ind}{\input{\jobname-pw.ind}}{}

\end{document}

      