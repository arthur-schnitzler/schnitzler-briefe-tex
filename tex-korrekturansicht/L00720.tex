%% latex-korrekturansicht-vorspann.tex
%% Vorspann für die Korrekturansicht.
%% Lädt die gemeinsame Datei latex-vorspann.tex mit gesetztem Schalter.

\newif\ifkorrekturansicht
\korrekturansichttrue

\input{../tex-inputs/latex-vorspann}


               \section[Richard Beer-Hofmann an Arthur Schnitzler, {[}31. 8. 1897{]}]{ Richard Beer-Hofmann an Arthur Schnitzler, {[}31. 8. 1897{]}}\nopagebreak\mylabel{v}\rehead{ }\normalsize\beginnumbering\briefempfaengerindex{Schnitzler, Arthur@\textsc{Schnitzler, Arthur}!zzzBeer-Hofmann, Richard@\emph{von Richard Beer-Hofmann}!1897-08-312@{{[}31. 8. 1897{]}}|(be} \toendnotes[C]{\smallbreak\pagebreak[2]} \Standort{CUL, Schnitzler, B 8.}
\physDesc{Brief, 1 Blatt, 1 Seite
\newline{}Handschrift: Bleistift, lateinische Kurrent
\newline{}Schnitzler: mit Bleistift datiert: »31/8 97« }\toendnotes[C]{\smallbreak}\pstart
           \raggedleft{}{\pb}8 Uhr Abend\pend
           \pstart
           Lieber Arthur! Vor einer Stunde Ihre Karte erhalten; \textcolor{blue}{Leo}{}\ledrightnote{\textcolor{blue}{Leo Van-Jung}}, der \textcolor{blue}{Goldmann}{}\ledrightnote{\textcolor{blue}{Paul Goldmann}}
               morgen noch in \textcolor{pink}{Salzburg}{}\ledrightnote{\textcolor{pink}{Salzburg}} treffen möchte, reist
               heute Abends nach \textcolor{pink}{Salzburg}{}\ledrightnote{\textcolor{pink}{Salzburg}}, möchte
               Sie gerne auch noch sprechen und packt momentan. Bis 9 Uhr
                sind wir
               hier \textcolor{pink}{Kolingasse 9}{}\ledrightnote{\textcolor{pink}{Kolingasse}}, dann begleite ich \textcolor{blue}{Leo}{}\ledrightnote{\textcolor{blue}{Leo Van-Jung}} zur Bahn und kann erst um 11 ½ im
                  \textcolor{pink}{Caffee}{}\ledrightnote{→\textcolor{pink}{Café Arkaden}}
                sein. Hoffe Sie zu treffen.\pend
           \pstart
           Ihr{\\[\baselineskip]}\spacefill\mbox{Richard}\pend
           \leftskip=0em{}\endnumbering\briefempfaengerindex{Schnitzler, Arthur@\textsc{Schnitzler, Arthur}!zzzBeer-Hofmann, Richard@\emph{von Richard Beer-Hofmann}!1897-08-312@{{[}31. 8. 1897{]}}|)be}\mylabel{h}  \normalsize

\doendnotes{C}
\bigskip
\vfill

\clearpage

\footnotesize

\lohead{\textsc{register}}

% Definiere theindex-Environment komplett neu ohne reledmac
\makeatletter
\renewenvironment{theindex}{%
  \section*{\indexname}%
  \setlength{\parindent}{0pt}%
  \setlength{\parskip}{0pt plus 0.3pt}%
  \let\item\@idxitem
}{%
  \clearpage
}
\makeatother

\IfFileExists{\jobname-pw.ind}{\input{\jobname-pw.ind}}{}

\end{document}

      