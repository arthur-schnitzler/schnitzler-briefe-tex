%% latex-korrekturansicht-vorspann.tex
%% Vorspann für die Korrekturansicht.
%% Lädt die gemeinsame Datei latex-vorspann.tex mit gesetztem Schalter.

\newif\ifkorrekturansicht
\korrekturansichttrue

\input{../tex-inputs/latex-vorspann}


\renewcommand{\erwaehntePersonen}{Personen: Stefanie Bachrach, Felix Salten, Olga Schnitzler, Oscar Wettstein}
\renewcommand{\erwaehnteOrte}{Orte: Cottagegasse, Salzburg, Wien, XVIII., Währing}
\renewcommand{\erwaehnteWerke}{}
\section[ Arthur Schnitzler an Felix Salten, 17. 5. 1917]{Arthur Schnitzler an Felix Salten, 17. 5. 1917}
\nopagebreak\mylabel{v}
\rehead{ }\normalsize\beginnumbering\briefempfaengerindex{Salten, Felix@\textsc{Salten, Felix}!zzzSchnitzler, Arthur@\emph{von Arthur Schnitzler}!1917-05-171@{17. 5. 1917}|(be}
\toendnotes[C]{\smallbreak\pagebreak[2]}\Standort{Wienbibliothek im Rathaus, ZPH 1681, 2.1.516.}
\physDesc{Kartenbrief, 502 Zeichen
\newline{}Handschrift: 1) schwarze Tinte, lateinische Kurrent\hspace{1em}2) schwarze Tinte, deutsche Kurrent (\noindent{}Adresse)\hspace{1em}
\newline{}Ordnung: mit Bleistift von unbekannter Hand nummeriert: »8« }\toendnotes[C]{\smallbreak}\pstart{}{\pb}Herrn \textsc{Felix
                     Salten}.\pend{}\pstart{}\textcolor{pink}{Wien XVIII}{}\ledrightnote{\textcolor{pink}{XVIII., Währing}}\pend{}\pstart{}\textcolor{pink}{\textsc{Cottagegasse 37}}{}\ledrightnote{\textcolor{pink}{Cottagegasse}}.\pend{}
{\bigskip}
\pstart
           \raggedleft{}{\pb}17/5 17\textcolor{gray}{.}\pend
           
\pstart
           \raggedleft{}früh.\pend
           
\pstart
           lieber, von einer \label{K_L03019-1v}\edtext{Reise (\textcolor{pink}{Salzburg}{}\ledrightnote{\textcolor{pink}{Salzburg}})}{\lemma{\textnormal{\emph{Reise (Salzburg)}}}\Cendnote{\textnormal{\textcolor{blue}{Arthur} und \textcolor{blue}{Olga Schnitzler} waren am 14. 5. 1917 nach \textcolor{pink}{Salzburg} abgereist. Am 16. 5. 1917 fuhren sie zurück nach \textcolor{pink}{Wien}.}}}\label{K_L03019-1h} heimgekehrt, die durch die
               Nachricht vom \label{K_L03019-2v}\edtext{Tode unsrer Freundin
                  \textcolor{blue}{Stephi Bachrach}{}\ledrightnote{\textcolor{blue}{Stefanie Bachrach}}}{\lemma{\textnormal{\emph{Tode … Bachrach}}}\Cendnote{\textnormal{\textcolor{blue}{Stefanie Bachrach} war am 16. 5. 1917
                  verstorben.}}}\label{K_L03019-2h} jäh unterbrochen wurde, finde ich Ihre freundliche Einladg zu
               dem \textcolor{blue}{Wettstein}{}\ledrightnote{\textcolor{blue}{Oscar Wettstein}} Souper und
                  bitte\textcolor{gray}{,} Sie zugleich mein Fernbleiben mit Rücksicht auf diesen
               Trauerfall zu entschuldigen, der mich sehr tief bewegt.\pend
           
\pstart
           Die Einladg zu dem Vortrag, auf die Sie sich beziehen, ist übrigens nicht an mich
               gelangt.\pend
           
\pstart
           Auf Wiedersehen und herzlichen Dank. {\\[\baselineskip]}Ihr {\\[\baselineskip]}\spacefill\mbox{Arth Sch}\pend
           \leftskip=0em{}\endnumbering\briefempfaengerindex{Salten, Felix@\textsc{Salten, Felix}!zzzSchnitzler, Arthur@\emph{von Arthur Schnitzler}!1917-05-171@{17. 5. 1917}|)be}\mylabel{h}  \normalsize

\doendnotes{C}
\bigskip
\vfill

\clearpage

\footnotesize

\lohead{\textsc{register}}

% Definiere theindex-Environment komplett neu ohne reledmac
\makeatletter
\renewenvironment{theindex}{%
  \section*{\indexname}%
  \setlength{\parindent}{0pt}%
  \setlength{\parskip}{0pt plus 0.3pt}%
  \let\item\@idxitem
}{%
  \clearpage
}
\makeatother

\IfFileExists{\jobname-pw.ind}{\input{\jobname-pw.ind}}{}

\end{document}

      