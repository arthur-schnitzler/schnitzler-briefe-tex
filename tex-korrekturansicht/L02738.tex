%% latex-korrekturansicht-vorspann.tex
%% Vorspann für die Korrekturansicht.
%% Lädt die gemeinsame Datei latex-vorspann.tex mit gesetztem Schalter.

\newif\ifkorrekturansicht
\korrekturansichttrue

\input{../tex-inputs/latex-vorspann}


               \section[Paul Goldmann an Arthur Schnitzler, 29. 6. {[}1895{]}]{ Paul Goldmann an Arthur Schnitzler, 29. 6. {[}1895{]}}\nopagebreak\mylabel{v}\rehead{ }\normalsize\beginnumbering\briefempfaengerindex{Schnitzler, Arthur@\textsc{Schnitzler, Arthur}!zzzGoldmann, Paul@\emph{von Paul Goldmann}!1895-06-291@{29. 6. {[}1895{]}}|(be} \toendnotes[C]{\smallbreak\pagebreak[2]} \Standort{DLA, A:Schnitzler, HS.NZ85.1.3165.}
\physDesc{Brief, 2 Blätter, 8 Seiten
\newline{}Handschrift: schwarze Tinte, deutsche Kurrent
\newline{}Schnitzler: 1) mit schwarzer Tinte das Jahr »\textcolor{gray}{9}5« vermerkt 2) mit rotem Buntstift zwei Unterstreichungen}\toendnotes[C]{\smallbreak}\pstart
           \noindent{}{\pb}\textcolor{gray}{\textbf{\textbf{\textcolor{brown}{Frankfurter Zeitung}{}\ledrightnote{\textcolor{brown}{Frankfurter Zeitung}}}}}\hfill \textsc{\textcolor{pink}{Paris}{}\ledrightnote{\textcolor{pink}{Paris}}}, 29. Juni.\pend
           \pstart
           \textcolor{gray}{\textbf{(\textcolor{brown}{\begin{otherlanguage}{french}Gazette de Francfort\end{otherlanguage}}{}\ledrightnote{\textcolor{brown}{Frankfurter Zeitung}}). }}\pend
           \pstart
           \textcolor{gray}{\textbf{\textbf{\begin{otherlanguage}{french}Fondateur M. \textcolor{blue}{L.
                              Sonnemann}{}\ledrightnote{\textcolor{blue}{Leopold Sonnemann}}\end{otherlanguage}.}}}\pend
           \pstart
           \begin{otherlanguage}{french}\textcolor{gray}{\textbf{\textcolor{green}{Journal}{}\ledrightnote{→\textcolor{green}{Frankfurter Zeitung}} politique,
                        financier,}}\end{otherlanguage}\pend
           \pstart
           \begin{otherlanguage}{french}\textcolor{gray}{\textbf{commercial et littéraire.}}\end{otherlanguage}\pend
           \pstart
           \begin{otherlanguage}{french}\textcolor{gray}{\textbf{\textbf{Paraissant trois fois par jour.}}}\end{otherlanguage}\pend
           \pstart
           \begin{otherlanguage}{french}\textcolor{gray}{\textbf{\textbf{Bureau à \textcolor{pink}{Paris}{}\ledrightnote{\textcolor{pink}{Paris}}:}}}\end{otherlanguage}\pend
           \pstart
           \begin{otherlanguage}{french}\textcolor{gray}{\textbf{\textbf{\textcolor{pink}{24. Rue Feydeau}{}\ledrightnote{\textcolor{pink}{rue Feydeau}}.}}}\end{otherlanguage}\pend
           \pstart\center{}Mein lieber Freund,\pend\pstart
           Noch weiß ich nichts ganz Genaues über meinen Urlaub; aber die Sache wird ungefähr ſo
               ſein: zwiſchen dem 10. und 15.
                  Auguſt gehe ich nach \textsc{\textcolor{pink}{Toelz}{}\ledrightnote{\textcolor{pink}{Bad Tölz}}}, das 2 Stunden Bahnfahrt von \textsc{\textcolor{pink}{Muenchen}{}\ledrightnote{\textcolor{pink}{München}}} entfernt iſt, u. gebrauche dort die Kur, drei oder vier Wochen, je nach
               ärztlicher Vorſchrift. {\pb}Dann wird mein Urlaub wohl
               zu Ende ſein. Immerhin hoffe ich doch ſo um den 5. September herum acht Tage in \textcolor{pink}{München}{}\ledrightnote{\textcolor{pink}{München}} verbringen zu können. Du kannſt Dir denken, wie leid es mir thut,
               Dir diesmal nicht mehr entgegenkommen zu können; denn auch mein liebſter Wunſch für
               dieſen Sommer wäre, dich zu treffen. Aber ich muß {\pb}etwas für die Geſundheit (?!) thun, denn ich bin gar ſehr elend: Wie alſo, wenn Du
               Deine \label{K_L02738-1v}\edtext{Bicycle-\textsc{Tour}}{\lemma{\textnormal{\emph{Bicycle-Tour}}}\Cendnote{\textnormal{Am 24. 8. 1895 startete \textcolor{blue}{Schnitzler} mit \textcolor{blue}{Felix
                     Salten} eine Radtour in \textcolor{pink}{Salzburg}. Am
                     25. 8. 1895 kam \textcolor{blue}{Schnitzler} in \textcolor{pink}{Bad Tölz} an, wo er den nächsten Tag mit \textcolor{blue}{Goldmann} verbrachte. Am 27. 8. 1895 fuhren \textcolor{blue}{Schnitzler}
                  und \textcolor{blue}{Salten} weiter nach \textcolor{pink}{München}, wohin auch \textcolor{blue}{Goldmann} nachreiste.}}}\label{K_L02738-1h} nach \textcolor{pink}{\textsc{Muenchen}}{}\ledrightnote{\textcolor{pink}{München}} auf den \substVorne{}\textsuperscript{December}{\allowbreak}\substDazwischen{}September\substHinten{} ließeſt, etwa \strikeout{z\textcolor{gray}{u}} nach Rückkehr \strikeout{\textcolor{gray}{v}} von \textcolor{pink}{Kopenhagen}{}\ledrightnote{\textcolor{pink}{Kopenhagen}}? Oder ſonſt, wie Du
               willſt. Beſtimme, und ich werde ſuchen, mich nach Dir zu richten.\pend
           \pstart
           Von der Frau \textsc{\textcolor{blue}{Andreas}{}\ledrightnote{\textcolor{blue}{Lou Andreas-Salomé}}} hatte ich {\pb}\label{K_L02738-2v}\edtext{folgende kurzen Zeilen}{\lemma{\textnormal{\emph{folgende kurzen Zeilen}}}\Cendnote{\textnormal{siehe Lou Andreas-Salomé an Arthur Schnitzler, 25. 5. 1895}}}\label{K_L02738-2h}, die ich Dir ſende. Liebenswürdig, aber unnnatürlich und gekünſtelt. Die \strikeout{Doppel} Doppel-Adjektive »tief und deutlich empfand ich«
               ſind das beſte Zeichen dafür, daß man gar nichts empfindet. Oder nein? {\dotsfour}\pend
           \pstart
           Nochmals von Herzen glückliche Reiſe, liebſter Freund! Ich freue mich, daß Dir {\pb}der Sommer diesmal ein ſo reiches Programm bringt.
               Wie denkſt Du über eine Rückreiſe von \textsc{\textcolor{pink}{Kopenhagen}{}\ledrightnote{\textcolor{pink}{Kopenhagen}} via \textcolor{pink}{Paris}{}\ledrightnote{\textcolor{pink}{Paris}}}?\pend
           \pstart
           Die \textcolor{green}{Aufführung}{}\ledrightnote{→\textcolor{green}{Liebelei. Schauspiel in drei Akten}}s- Chancen
               machen mir doch jetzt einen recht ernſten Eindruck. \label{K_L02738-3v}\edtext{\textsc{\textcolor{blue}{Sonnenthal}{}\ledrightnote{\textcolor{blue}{Adolf von Sonnenthal}}}, \textsc{\textcolor{blue}{Mitterwurzer}{}\ledrightnote{\textcolor{blue}{Friedrich Mitterwurzer}}}}{\lemma{\textnormal{\emph{Sonnenthal, Mitterwurzer}}}\Cendnote{\textnormal{Bei der Uraufführung der \emph{\textcolor{green}{Liebelei}} am 9. 10. 1895 im \textcolor{pink}{Burgtheater} spielte \textcolor{blue}{Adolf von
                     Sonnenthal} den alten \textcolor{green}{Weiring}, \textcolor{blue}{Friedrich Mitterwurzer}
                  den \textcolor{green}{Herrn} und \textcolor{blue}{Adele Sandrock} die \textcolor{green}{Christine}.}}}\label{K_L02738-3h}, das wäre herrlich.
               Aber \strikeout{w\textcolor{gray}{e}} wer gibt das \textcolor{green}{Mädel}{}\ledrightnote{→\textcolor{green}{Liebelei. Schauspiel in drei Akten}}? Und
                  {\pb}was hörſt Du aus \textsc{\textcolor{pink}{Berlin}{}\ledrightnote{\textcolor{pink}{Berlin}}}?\pend
           \pstart
           Auch dieſe \label{K_L02738-4v}\edtext{reichliche Production}{\lemma{\textnormal{\emph{reichliche Production}}}\Cendnote{\textnormal{Zuletzt arbeitete \textcolor{blue}{Schnitzler} an \emph{\textcolor{green}{Freiwild}}, \emph{\textcolor{green}{Die Frau des Weisen}} und \emph{\textcolor{green}{Der Empfindsame}}.}}}\label{K_L02738-4h} iſt ſchön. Man ſoll
               aber gar nicht darüber reden, ums nicht zu berufen. Ich ſage eben nur, daß es ſchön
               iſt.\pend
           \pstart
           Verleger? Schreib’ ruhig an den \label{K_L02738-5v}\edtext{\textcolor{blue}{Mann}{}\ledrightnote{→\textcolor{blue}{Louis Debarge}}}{\lemma{\textnormal{\emph{Mann}}}\Cendnote{\textnormal{\textcolor{blue}{Louis Debarge}, der Gründer und Herausgeber
                  der \emph{\textcolor{brown}{Semaine Littéraire}}. Seine Briefe an \textcolor{blue}{Schnitzler} liegen heute im \emph{Deutschen Literaturarchiv Marbach},
                  HS.1985.1.2728.}}}\label{K_L02738-5h} von der »\textsc{\textcolor{brown}{Semaine littéraire}{}\ledrightnote{\textcolor{brown}{La Semaine Littéraire}}}.« Du brauchſt ja von der \label{K_L02738-6v}\edtext{\textsc{\textcolor{green}{Mercure}{}\ledrightnote{\textcolor{green}{Mercure de France}}}-\textcolor{green}{Notiz}{}\ledrightnote{→\textcolor{green}{Journaux et Revues. [Le dernier numéro]}}}{\lemma{\textnormal{\emph{Mercure-Notiz}}}\Cendnote{\textnormal{\textcolor{blue}{Henri Albert}: \emph{\textcolor{green}{Journaux et Revues. [Le dernier numéro]}}. In: \emph{\textcolor{green}{Mercure de France}}, Jg. 12, Nr. 66, 1. 6. 1895, S. 371–372, hier: S. 372. Darin
                  berichtet \textcolor{blue}{Albert}, von \textcolor{blue}{Schnitzler} um ein paar Worte anlässlich des Abdrucks von
                     \emph{\textcolor{green}{Mourir}} in der \emph{\textcolor{brown}{Semaine littéraire}} gebeten worden zu sein. Da ihm der \textcolor{blue}{Leiter} der \emph{\textcolor{brown}{Semaine littéraire}} aber geschrieben habe, er dürfe nicht
                  erwähnen, dass das Liebespaar in \emph{\textcolor{green}{Sterben}}
                  nicht verheiratet sei, habe er dankend abgelehnt.}}}\label{K_L02738-6h} gar nichts zu wiſſen. Ich
               hab’ ſie {\pb}übrigens auch recht überflüſſig gefunden.
               Aber das iſt ſo \textcolor{pink}{Pariſ}{}\ledrightnote{\textcolor{pink}{Paris}}er Art: immer nur von ſich
               reden. Alle haben ſie hier was von \textsc{\textcolor{blue}{Hermann Bahr}{}\ledrightnote{\textcolor{blue}{Hermann Bahr}}} an ſich.\pend
           \pstart
           Mit \label{K_L02738-7v}\edtext{\textsc{\textcolor{brown}{\textcolor{blue}{Langen}{}\ledrightnote{\textcolor{blue}{Albert Langen}}}{}\ledrightnote{\textcolor{brown}{Albert Langen}}}}{\lemma{\textnormal{\emph{Langen}}}\Cendnote{\textnormal{Siehe Paul Goldmann an Arthur Schnitzler, 3. 4. [1895]}}}\label{K_L02738-7h} wird nichts zu machen ſein. Er iſt ein blödſinniger Idiot. Er haßt mich, weil er weiß, daß ich
               weiß, daß er ein Idiot iſt;
               und er {\pb}haßt Dich, weil Du mein Freund biſt. Auch
               gibt er keine franzöſiſchen Bücher mehr heraus. Aber ich will einmal etwas Anderes
               durch \textsc{\textcolor{blue}{Henri Becque}{}\ledrightnote{\textcolor{blue}{Henry Becque}}} verſuchen.\pend
           \pstart
           Soll’ ich Dir die \textcolor{pink}{franzöſiſch}{}\ledrightnote{→\textcolor{pink}{Frankreich}}en
               Blätter, die ich für Dich ſammle, auch nach unterwegs ſchicken? Es macht mir gar
               nichts, denn ich ſammele ſo wie ſo.\pend
           \pstart
           Viele treue Grüße Dir und \textsc{\textcolor{blue}{Richard}{}\ledrightnote{\textcolor{blue}{Richard Beer-Hofmann}}}. Von Herzen\pend
           \pstart
           Dein {\\[\baselineskip]}\spacefill\mbox{Paul Goldmann.}\pend
           \leftskip=0em{}\endnumbering\briefempfaengerindex{Schnitzler, Arthur@\textsc{Schnitzler, Arthur}!zzzGoldmann, Paul@\emph{von Paul Goldmann}!1895-06-291@{29. 6. {[}1895{]}}|)be}\mylabel{h}\begin{anhang}\end{anhang}\normalsize

\doendnotes{C}
\bigskip
\vfill

\clearpage

\footnotesize

\lohead{\textsc{register}}

% Definiere theindex-Environment komplett neu ohne reledmac
\makeatletter
\renewenvironment{theindex}{%
  \section*{\indexname}%
  \setlength{\parindent}{0pt}%
  \setlength{\parskip}{0pt plus 0.3pt}%
  \let\item\@idxitem
}{%
  \clearpage
}
\makeatother

\IfFileExists{\jobname-pw.ind}{\input{\jobname-pw.ind}}{}

\end{document}

      