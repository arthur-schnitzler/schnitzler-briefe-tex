%% latex-korrekturansicht-vorspann.tex
%% Vorspann für die Korrekturansicht.
%% Lädt die gemeinsame Datei latex-vorspann.tex mit gesetztem Schalter.

\newif\ifkorrekturansicht
\korrekturansichttrue

\input{../tex-inputs/latex-vorspann}


               \section[Lou Andreas-Salomé an Arthur Schnitzler, 28. 4. 1895]{ Lou Andreas-Salomé an Arthur Schnitzler, 28. 4. 1895}\nopagebreak\mylabel{v}\rehead{ }\normalsize\beginnumbering\briefempfaengerindex{Schnitzler, Arthur@\textsc{Schnitzler, Arthur}!zzzAndreas-Salome, Lou@\emph{von Lou Andreas-Salomé}!1895-04-281@{28. 4. 1895}|(be} \toendnotes[C]{\smallbreak\pagebreak[2]} \Standort{CUL, Schnitzler, B 3.}
\physDesc{Briefkarte
\newline{}Handschrift: schwarze Tinte, deutsche Kurrent\newline{}Ordnung: mit rotem Buntstift von unbekannter Hand
                                    nummeriert: »2« }\pstart{}{\pb}Sehr geehrter Herr \textsc{D\textsuperscript{r}},\pend\pstart
           ich bin für kurze Zeit in \textcolor{pink}{\textsc{Wien}}{}\ledrightnote{\textcolor{pink}{Wien}}; kann ich Sie perſönlich kennen lernen?\pend
           \pstart
           In ausgezeichneter Hochachtung{\\[\baselineskip]}\spacefill\mbox{Lou Andreas-Salomé.}\pend
           \leftskip=0em{}\pstart
           \noindent{}\textsc{\textcolor{pink}{Hôtel Royal}{}\ledrightnote{\textcolor{pink}{Hotel Royal}}}\pend
           \pstart
           \textsc{am \textcolor{pink}{Stephansplatz}{}\ledrightnote{\textcolor{pink}{Stephansplatz}}.}\pend
           \pstart
           28. IV. 95\pend
           \endnumbering\briefempfaengerindex{Schnitzler, Arthur@\textsc{Schnitzler, Arthur}!zzzAndreas-Salome, Lou@\emph{von Lou Andreas-Salomé}!1895-04-281@{28. 4. 1895}|)be}\mylabel{h}  \normalsize

\doendnotes{C}
\bigskip
\vfill

\clearpage

\footnotesize

\lohead{\textsc{register}}

% Definiere theindex-Environment komplett neu ohne reledmac
\makeatletter
\renewenvironment{theindex}{%
  \section*{\indexname}%
  \setlength{\parindent}{0pt}%
  \setlength{\parskip}{0pt plus 0.3pt}%
  \let\item\@idxitem
}{%
  \clearpage
}
\makeatother

\IfFileExists{\jobname-pw.ind}{\input{\jobname-pw.ind}}{}

\end{document}

      