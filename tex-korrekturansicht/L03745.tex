%% latex-korrekturansicht-vorspann.tex
%% Vorspann für die Korrekturansicht.
%% Lädt die gemeinsame Datei latex-vorspann.tex mit gesetztem Schalter.

\newif\ifkorrekturansicht
\korrekturansichttrue

\input{../tex-inputs/latex-vorspann}


\section[Arthur Schnitzler an Stefan Zweig, 12. 10. 1927]{L03745 Arthur Schnitzler an Stefan Zweig, 12. 10. 1927}
\nopagebreak\mylabel{L03745v}
\rehead{ }\normalsize\beginnumbering\briefempfaengerindex{Zweig, Stefan@\textsc{Zweig, Stefan}!zzzSchnitzler, Arthur@\emph{von Arthur Schnitzler}!1927-10-121@{12. 10. 1927}|(be}
\toendnotes[C]{\smallbreak\pagebreak[2]}
\correspDesc{Versand  durch Arthur Schnitzler am 12. 10. 1927 in Wien
\newline{}Erhalt  durch Stefan Zweig im Zeitraum [13. 10. 1927 – 17. 10. 1927?] in Salzburg}\toendnotes[C]{\smallbreak}
\Standort{Jerusalem, National Library of Israel, ARC. Ms. Var. 305 1 58 Stefan Zweig Collection.}
\physDesc{Brief, 1 Blatt, 2 Seiten, 1082 Zeichen
\newline{}Handschrift: schwarze Tinte, lateinische Kurrent}\toendnotes[C]{\smallbreak}
\pstart
           \raggedleft{}{\pb}\textcolor{pink}{Wien}\oindex{Wien@\textbf{Wien}, \emph{Verwaltungsgebiet}|pw}{}\ledrightnote{\textcolor{pink}{Wien}}, 12. Oct. 927\pend
           
\pstart{}lieber Doctor Zweig,\pend\vspace{0.5em}
\pstart
           es besteht eine Möglichkeit für mich, meine nächsten Sachen, vor Erscheinen in \textcolor{pink}{Deutschland}\oindex{Deutschland@\textbf{Deutschland}|pw}{}\ledrightnote{\textcolor{pink}{Deutschland}} an eine \textcolor{pink}{russ.}\oindex{Russland@\textbf{Russland}|pw}{}\ledrightnote{\textcolor{pink}{Russland}}{ }Verlagsanstalt zu verkaufen. Wie ich höre, haben
               Sie Ihr letztes \label{K_L03745-1v}\edtext{\textcolor{green}{Novellenbuch}\pwindex{Zweig, Stefan 28.\,11.\,1881 Wien – 23.\,2.\,1942 Petrópolis@\textsc{Zweig, Stefan} (28.\,11.\,1881 Wien – 23.\,2.\,1942 Petrópolis), \emph{Schriftsteller}!Smjatenie Chusto@\strich\emph{Smjatenie Chusto}|pwv}{}\ledrightnote{{$\rightarrow$}\emph{\textcolor{green}{Smjatenie Chusto}}}}{\lemma{\textnormal{\emph{Novellenbuch}}}\Cendnote{\textnormal{Der Verlag \emph{\textcolor{brown}{Wremla}\orgindex{Wremja@Wremja|pwk}} hatte ohne \textcolor{blue}{Zweigs}\pwindex{Zweig, Stefan 28.\,11.\,1881 Wien – 23.\,2.\,1942 Petrópolis@\textsc{Zweig, Stefan} (28.\,11.\,1881 Wien – 23.\,2.\,1942 Petrópolis), \emph{Schriftsteller}|pwk} Zustimmung 1925{ }\emph{\textcolor{green}{Erstes Erlebnis}\pwindex{Zweig, Stefan 28.\,11.\,1881 Wien – 23.\,2.\,1942 Petrópolis@\textsc{Zweig, Stefan} (28.\,11.\,1881 Wien – 23.\,2.\,1942 Petrópolis), \emph{Schriftsteller}!Žgučaja tajna. Pervye pereživanija@\strich\emph{Žgučaja tajna. Pervye pereživanija}|pwk}} und 1926{ }\emph{\textcolor{green}{Amok}\pwindex{Zweig, Stefan 28.\,11.\,1881 Wien – 23.\,2.\,1942 Petrópolis@\textsc{Zweig, Stefan} (28.\,11.\,1881 Wien – 23.\,2.\,1942 Petrópolis), \emph{Schriftsteller}!Amok. Novelly@\strich\emph{Amok. Novelly}|pwk}} auf \textcolor{pink}{russisch}\oindex{Russland@\textbf{Russland}|pwk} publiziert. Nach der Kontaktaufnahme erschienen mit \textcolor{blue}{Zweigs}\pwindex{Zweig, Stefan 28.\,11.\,1881 Wien – 23.\,2.\,1942 Petrópolis@\textsc{Zweig, Stefan} (28.\,11.\,1881 Wien – 23.\,2.\,1942 Petrópolis), \emph{Schriftsteller}|pwk} Zustimmung die zwei Novellen \emph{\textcolor{green}{Verwirrung der Gefühle}\pwindex{Zweig, Stefan 28.\,11.\,1881 Wien – 23.\,2.\,1942 Petrópolis@\textsc{Zweig, Stefan} (28.\,11.\,1881 Wien – 23.\,2.\,1942 Petrópolis), \emph{Schriftsteller}!Verwirrung der Gefühle@\strich\emph{Verwirrung der Gefühle}|pwk}} und \emph{\textcolor{green}{Brief einer Unbekannten}\pwindex{Zweig, Stefan 28.\,11.\,1881 Wien – 23.\,2.\,1942 Petrópolis@\textsc{Zweig, Stefan} (28.\,11.\,1881 Wien – 23.\,2.\,1942 Petrópolis), \emph{Schriftsteller}!Brief einer Unbekannten@\strich\emph{Brief einer Unbekannten}|pwk}} unter dem Titel \emph{\textcolor{green}{Smjatenie Chusto}\pwindex{Zweig, Stefan 28.\,11.\,1881 Wien – 23.\,2.\,1942 Petrópolis@\textsc{Zweig, Stefan} (28.\,11.\,1881 Wien – 23.\,2.\,1942 Petrópolis), \emph{Schriftsteller}!Smjatenie Chusto@\strich\emph{Smjatenie Chusto}|pwk}}.}}}\label{K_L03745-1} auch nach \textcolor{pink}{Russland}\oindex{Russland@\textbf{Russland}|pw}{}\ledrightnote{\textcolor{pink}{Russland}} verkauft, und es wäre mir sehr erwünscht zu wissen 
                  (we{\geminationn} Sie über solche Dinge nicht principiell
               schweigen) welche Summe Ihnen bezahlt worden ist resp. unter welchen Bedingungen Sie
               abgeschlossen haben. Pauschalsummen\substVorne{}\textsuperscript{?}\substDazwischen{}\textcolor{gray}{,}\substHinten{} Perzente? Vorschuſs u. Perzente? U. s. w. Sie
               erweisen mir einen rechten Gefallen, we{\geminationn} Sie mich {\pb}aufklärten. Es handelt sich um einen Roman, der eben
               fertig geworden ist. »\label{K_L03745-2v}\edtext{\textcolor{green}{Therese, Chronik eines Frauenlebens}\pwindex{Schnitzler, Arthur 15. 5. 1862 Wien – 21. 10. 1931 ebd.@\textsc{Schnitzler, Arthur} (15. 5. 1862 Wien – 21. 10. 1931 ebd.), \emph{Schriftsteller, Mediziner}!Therese. Chronik eines Frauenlebens@\strich\emph{Therese. Chronik eines Frauenlebens}|pw}{}\ledrightnote{\textcolor{green}{Therese. Chronik eines Frauenlebens}}}{\lemma{\textnormal{\emph{Therese, … Frauenlebens}}}\Cendnote{\textnormal{Zu Lebzeiten \textcolor{blue}{Schnitzlers} kam es zu keiner \textcolor{pink}{russischen}\oindex{Russland@\textbf{Russland}|pwk} Übersetzung des Romans.}}}\label{K_L03745-2}«.\pend
           
\pstart
           Sie haben
               hoffentlich einen schönen So{\geminationm}er gehabt. Was mich
               anbelangt so war ich \label{K_L03745-3v}\edtext{in den \textcolor{pink}{Dolomiten}\oindex{Dolomiten@\textbf{Dolomiten}, \emph{Gebirge}|pw}{}\ledrightnote{\textcolor{pink}{Dolomiten}}}{\lemma{\textnormal{\emph{in den Dolomiten}}}\Cendnote{\textnormal{\textcolor{blue}{Schnitzler} war zwischen 11. 8. 1927 und 5. 9. 1927 an
                  verschiedenen Orten in \textcolor{pink}{Südtirol}\oindex{Südtirol@\textbf{Südtirol}, \emph{Verwaltungsgebiet}|pwk} und \textcolor{pink}{Norditalien}\oindex{Italien@\textbf{Italien}|pwk}. Am letztgenannten Tag langte er
                  in \textcolor{pink}{Venedig}\oindex{Venedig@\textbf{Venedig}|pwk} an, wo er bis zum 15. 9. 1927 blieb.}}}\label{K_L03745-3}
               und zum Schluſs am \textcolor{pink}{Lido}\oindex{Lido@\textbf{Lido}|pw}{}\ledrightnote{\textcolor{pink}{Lido}}, resp in \textcolor{pink}{Venedig}\oindex{Venedig@\textbf{Venedig}|pw}{}\ledrightnote{\textcolor{pink}{Venedig}} wo meine \textcolor{blue}{Tochter}\pwindex{Cappellini, Lili 13.\,9.\,1909 Wien – 26.\,7.\,1928 Venedig@\textsc{Cappellini, Lili} (13.\,9.\,1909 Wien – 26.\,7.\,1928 Venedig)|pwv}{}\ledrightnote{{$\rightarrow$}\emph{\textcolor{blue}{Lili Cappellini}}} verheiratet mit dem Capitano \textcolor{blue}{Arnoldo Cappellini}\pwindex{Cappellini, Arnoldo 10.\,8.\,1889 Venedig – 8.\,5.\,1954 Rom@\textsc{Cappellini, Arnoldo} (10.\,8.\,1889 Venedig – 8.\,5.\,1954 Rom)|pw}{}\ledrightnote{\textcolor{blue}{Arnoldo Cappellini}}, in der Nähe der \textcolor{pink}{Frari Kirche}\oindex{Santa Maria Gloriosa dei Frari@\textbf{Santa Maria Gloriosa dei Frari}, \emph{Kirche}|pw}{}\ledrightnote{\textcolor{pink}{Santa Maria Gloriosa dei Frari}} lebt. Zurück bin ich \label{K_L03745-4v}\edtext{geflogen}{\lemma{\textnormal{\emph{geflogen}}}\Cendnote{\textnormal{vgl. A. S.: \emph{Tagebuch}, 15. 9. 1927. }}}\label{K_L03745-4}. Das
               ist ein Erlebnis, das über alle Begriffe und sogar über alle
                  Feu{[}i{]}lletons geht.\pend
           
\pstart
           Ich hoffe wir sehen uns bald wieder.{\\[\baselineskip]}Sehr herzlich Ihr freundschaftlich ergebner{\\[\baselineskip]}\spacefill\mbox{ArthSchnitzler}\pend
           \leftskip=0em{}\selectlanguage{ngerman}\endnumbering\briefempfaengerindex{Zweig, Stefan@\textsc{Zweig, Stefan}!zzzSchnitzler, Arthur@\emph{von Arthur Schnitzler}!1927-10-121@{12. 10. 1927}|)be}\mylabel{L03745h}  \normalsize

\doendnotes{C}
\bigskip
\vfill

\clearpage

\footnotesize

\lohead{\textsc{register}}

% Definiere theindex-Environment komplett neu ohne reledmac
\makeatletter
\renewenvironment{theindex}{%
  \section*{\indexname}%
  \setlength{\parindent}{0pt}%
  \setlength{\parskip}{0pt plus 0.3pt}%
  \let\item\@idxitem
}{%
  \clearpage
}
\makeatother

\IfFileExists{\jobname-pw.ind}{\input{\jobname-pw.ind}}{}

\end{document}

      