%% latex-korrekturansicht-vorspann.tex
%% Vorspann für die Korrekturansicht.
%% Lädt die gemeinsame Datei latex-vorspann.tex mit gesetztem Schalter.

\newif\ifkorrekturansicht
\korrekturansichttrue

\input{../tex-inputs/latex-vorspann}


\renewcommand{\erwaehntePersonen}{Personen: Hermann Bahr, Alfred Friedmann, Paul Schlenther, Olga Schnitzler, Elisabeth Steinrück}
\renewcommand{\erwaehnteInstitutionen}{Institutionen: Akademischer Verein für Kunst und Literatur, Reichstag, Volkstheater}
\renewcommand{\erwaehnteOrte}{Orte: Berlin, Dessauer Straße, Deutsches Theater Berlin, Frankfurt am Main, Volkstheater, Wien}
\renewcommand{\erwaehnteWerke}{Werke: Der Schleier der Beatrice. Schauspiel in fünf Akten, Lebendige Stunden. Vier Einakter, Neue Freie Presse, Theater- und Kunstnachrichten [Uraufführung von Lebendige Stunden]}
\section[ Paul Goldmann an Arthur Schnitzler, 13. 12. {[}1901{]}]{Paul Goldmann an Arthur Schnitzler, 13. 12. {[}1901{]}}
\nopagebreak\mylabel{v}
\rehead{ }\normalsize\beginnumbering\briefempfaengerindex{Schnitzler, Arthur@\textsc{Schnitzler, Arthur}!zzzGoldmann, Paul@\emph{von Paul Goldmann}!1901-12-131@{13. 12. {[}1901{]}}|(be}
\toendnotes[C]{\smallbreak\pagebreak[2]}\Standort{DLA, A:Schnitzler, HS.NZ85.1.3171.}
\physDesc{Brief, 1 Blatt, 4 Seiten
\newline{}Handschrift: blaue Tinte, deutsche Kurrent
\newline{}Schnitzler: 1) mit Bleistift das Jahr »{[}1{]}901« vermerkt  2) mit rotem Buntstift eine Unterstreichung}\toendnotes[C]{\smallbreak}
\pstart
           \noindent{}\raggedleft{}{\pb}\textcolor{pink}{\textcolor{gray}{\textbf{DESSAUERSTRASSE 19}}}{}\ledrightnote{\textcolor{pink}{Dessauer Straße}}\pend
           
\pstart
           \textcolor{pink}{Berlin}{}\ledrightnote{\textcolor{pink}{Berlin}}, 13. Dezember.\pend
           
\pstart\center{}Mein lieber Freund,\pend
\pstart
           Das \label{K_L03095-1v}\edtext{Verhalten des \textcolor{brown}{Volkstheater}{}\ledrightnote{\textcolor{brown}{Volkstheater}}s}{\lemma{\textnormal{\emph{Verhalten des Volkstheaters}}}\Cendnote{\textnormal{hinsichtlich einer möglichen Aufführung der \emph{\textcolor{green}{Lebendigen Stunden}} am \textcolor{pink}{Volkstheater};
                     siehe A. S.: \emph{Tagebuch}, 6. 12. 1901, A. S.: \emph{Tagebuch}, 10. 12. 1901, Hermann Bahr an Arthur Schnitzler, 27. 10. [1901], Arthur Schnitzler an Hermann Bahr, 28. 10. 1901 und Arthur Schnitzler an Hermann Bahr, 11. 12. 1901.}}}\label{K_L03095-1h} iſt ſkandalös, und Dein
                  \label{K_L03095-22v}\edtext{Brief}{\lemma{\textnormal{\emph{Brief}}}\Cendnote{\textnormal{siehe Bahr/Schnitzler, L041651}}}\label{K_L03095-22h} iſt unter dieſen Umſtänden nur der Ausdruck legitimer Entrüſtung. Ob es aber
                  \uline{klug} war, die Beziehungen ganz abzulehnen, kann
               ich von hier aus nicht beurtheilen. Dazu bedarf ich Deiner mündlichen Aufklärungen.
               Herr \textsc{\textcolor{blue}{Bahr}{}\ledrightnote{\textcolor{blue}{Hermann Bahr}}} ſcheint da wieder eine feine Rolle geſpielt zu haben. Wie aber wird die Zukunft
               ſein? Wenn Du \label{K_L03095-2v}\edtext{in \textcolor{pink}{Wien}{}\ledrightnote{\textcolor{pink}{Wien}} kein Theater mehr}{\lemma{\textnormal{\emph{in … mehr}}}\Cendnote{\textnormal{Bezug auf die \textcolor{blue}{Schlenther}-Affäre im vorangegangenen Jahr}}}\label{K_L03095-2h} haſt, wirſt Du, ſo denke
               ich mir, nach \textcolor{pink}{Berlin}{}\ledrightnote{\textcolor{pink}{Berlin}} überſiedeln. Hier wirſt Du
               die Stellung finden, die man Dir in \textcolor{pink}{Wien}{}\ledrightnote{\textcolor{pink}{Wien}} verſagt.
               Und {\pb}auch \strikeout{\textcolor{gray}{h}\textcolor{gray}{×}} Deine Weiterentwickelung könnte nur günſtig \strikeout{beein} beeinflußt werden, wenn Du die engen \textcolor{pink}{Wien}{}\ledrightnote{\textcolor{pink}{Wien}}er Verhältniſſe verließeſt und in die große Welt hinauszögeſt.\pend
           
\pstart
           Die \label{K_L03095-87v}\edtext{Karte die wir Dir ſandten}{\lemma{\textnormal{\emph{Karte … ſandten}}}\Cendnote{\textnormal{nicht
                  ermittelt}}}\label{K_L03095-87h}, war in der That bei \label{K_L03095-67v}\edtext{Dr. \textsc{\textcolor{blue}{Friedmann}{}\ledrightnote{\textcolor{blue}{Alfred Friedmann}}}}{\lemma{\textnormal{\emph{Dr. Friedmann}}}\Cendnote{\textnormal{möglicherweise der Schriftsteller \textcolor{blue}{Alfred Friedmann}, der in \textcolor{pink}{Berlin} wohnte?}}}\label{K_L03095-67h} geſchrieben.\pend
           
\pstart
           Warum führt der \textcolor{brown}{Akademiſch-Literariſche Verein}{}\ledrightnote{\textcolor{brown}{Akademischer Verein für Kunst und Literatur}},
               der ſich in \textcolor{pink}{Wien}{}\ledrightnote{\textcolor{pink}{Wien}} begründet hat, nicht den »\textcolor{green}{Schleier der \textsc{Beatrice}}{}\ledrightnote{\textcolor{green}{Der Schleier der Beatrice. Schauspiel in fünf Akten}}« auf?\pend
           
\pstart
           Ich hoffe um Weihnachten herum etwa 14 Tage in \textcolor{pink}{Frankfurt}{}\ledrightnote{\textcolor{pink}{Frankfurt am Main}} bleiben zu können bis zur Wiedererö\textcolor{gray}{f}fnung des
                  \textcolor{brown}{Reichstag}{}\ledrightnote{\textcolor{brown}{Reichstag}}s (\label{K_L03095-54v}\edtext{8. Jänner}{\lemma{\textnormal{\emph{8. Jänner}}}\Cendnote{\textnormal{\textcolor{blue}{Goldmann} war ab dem 4. 1. 1902 wieder in \textcolor{pink}{Berlin} (vgl. Paul Goldmann an Arthur Schnitzler, 29. 12. [1901].}}}\label{K_L03095-54h}). Ich bin {\pb}unbeſchreiblich heruntergearbeitet und bedarf der
               Ruhe und Erholung. Daß Deine \label{K_L03095-14v}\edtext{\textsc{\textcolor{green}{Première}{}\ledrightnote{{$\rightarrow$}\textcolor{green}{Lebendige Stunden. Vier Einakter}}}}{\lemma{\textnormal{\emph{Première}}}\Cendnote{\textnormal{Am 4. 1. 1902 fand am \textcolor{pink}{Deutschen Theater} in \textcolor{pink}{Berlin} die
                  Uraufführung der vier Einakter \emph{\textcolor{green}{Lebendige
                     Stunden}} statt. Zu der von \textcolor{blue}{Goldmann}
                  vorgeschlagenen Verschiebung kam es nicht.}}}\label{K_L03095-14h} in meine kurze Ferienzeit fällt,
               iſt ein Zuſammentreffen, das ſich ausnimmt, als ſei \strikeout{von
                  irge} dieſe Unordnung von einer feindſeligen Hand getroffen worden. Ich werde
               von Dir nicht verlangen, daß Du meinetwegen Deine \textsc{\textcolor{green}{Première}{}\ledrightnote{{$\rightarrow$}\textcolor{green}{Lebendige Stunden. Vier Einakter}}} verſchiebſt. Aber mit Rückſicht auf das \label{K_L03095-77v}\edtext{\textcolor{green}{Referat}{}\ledrightnote{{$\rightarrow$}\textcolor{green}{Theater- und Kunstnachrichten [Uraufführung von Lebendige Stunden]}}}{\lemma{\textnormal{\emph{Referat}}}\Cendnote{\textnormal{[O. V.] [=\textcolor{blue}{Paul Goldmann}]: \emph{\textcolor{green}{Theater- und Kunstnachrichten. [Zur
                        Uraufführung von Lebendige Stunden]}}. In: \emph{\textcolor{green}{Neue Freie Presse}}, Nr. 13422, 5. 1. 1902, Morgenblatt, S. 8–9.}}}\label{K_L03095-77h} in der \textcolor{green}{N. Fr. Pr.}{}\ledrightnote{\textcolor{green}{Neue Freie Presse}}, das doch von großer Wichtigkeit ſein
               wird, könnteſt Du ſchon eine Verſchiebung um ein paar Tage vornehmen, unter igend
               einem Vorwande. Ich werde ſehen, ob ich hier einen anſtändigen und verläßlichen {\pb}Vertreter finden kann. Wenn nicht, ſo werde ich meinen
               Urlaub abkürzen und zur \textsc{\textcolor{green}{Première }{}\ledrightnote{{$\rightarrow$}\textcolor{green}{Lebendige Stunden. Vier Einakter}}} zurückkommen.\pend
           
\pstart
           Viele herzliche Grüße Dir und den \textcolor{blue}{Mädeln}{}\ledrightnote{{$\rightarrow$}\textcolor{blue}{Olga Schnitzler}{\newline}{$\rightarrow$}\textcolor{blue}{Elisabeth Steinrück}}! {\\[\baselineskip]}Dein {\\[\baselineskip]}\spacefill\mbox{Paul Goldmn}\pend
           \leftskip=0em{}\endnumbering\briefempfaengerindex{Schnitzler, Arthur@\textsc{Schnitzler, Arthur}!zzzGoldmann, Paul@\emph{von Paul Goldmann}!1901-12-131@{13. 12. {[}1901{]}}|)be}\mylabel{h}
\begin{anhang}
\end{anhang}\normalsize

\doendnotes{C}
\bigskip
\vfill

\clearpage

\footnotesize

\lohead{\textsc{register}}

% Definiere theindex-Environment komplett neu ohne reledmac
\makeatletter
\renewenvironment{theindex}{%
  \section*{\indexname}%
  \setlength{\parindent}{0pt}%
  \setlength{\parskip}{0pt plus 0.3pt}%
  \let\item\@idxitem
}{%
  \clearpage
}
\makeatother

\IfFileExists{\jobname-pw.ind}{\input{\jobname-pw.ind}}{}

\end{document}

      