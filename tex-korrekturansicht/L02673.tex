%% latex-korrekturansicht-vorspann.tex
%% Vorspann für die Korrekturansicht.
%% Lädt die gemeinsame Datei latex-vorspann.tex mit gesetztem Schalter.

\newif\ifkorrekturansicht
\korrekturansichttrue

\input{../tex-inputs/latex-vorspann}


               \section[Paul Goldmann an Arthur Schnitzler, 3. 12. 1891]{ Paul Goldmann an Arthur Schnitzler, 3. 12. 1891}\nopagebreak\mylabel{v}\rehead{ }\normalsize\beginnumbering\briefempfaengerindex{Schnitzler, Arthur@\textsc{Schnitzler, Arthur}!zzzGoldmann, Paul@\emph{von Paul Goldmann}!1891-12-031@{3. 12. 1891}|(be} \toendnotes[C]{\smallbreak\pagebreak[2]} \Standort{DLA, A:Schnitzler, HS.NZ85.1.3162.}
\physDesc{Kartenbrief
\newline{}Handschrift: 1) schwarze Tinte, deutsche Kurrent\hspace{1em}2) schwarze Tinte, lateinische Kurrent (\noindent{}Adresse)\hspace{1em}\newline{}Versand: 1) Stempel: »\nobreak{}\oindex{Place de la Bourse@\textbf{Place de la Bourse}, \emph{Platz (K.PLT)}|pwk}Paris 1 Pl. de la Bourse, 3 Dec 91, 7\textsuperscript{E}\nobreak{}«.  2) Stempel: »\nobreak{}\oindex{I., Innere Stadt@\textbf{I., Innere Stadt}, \emph{Bezirk (A.BZK)}|pwk}Wien \textcolor{gray}{1/1}, 5{[}.{]} 12. 91, 8–9½V.\nobreak{}«. }\toendnotes[C]{\smallbreak}\pstart{}{\pb}\textsc{\textcolor{pink}{\begin{otherlanguage}{french}Autriche\end{otherlanguage}}{}\ledrightnote{\textcolor{pink}{Österreich}}}!\pend{}\pstart{}\begin{otherlanguage}{french}\textcolor{gray}{\textbf{M\textsc{}}}\textsc{onsieur le docteur}\end{otherlanguage}\pend{}\pstart{}\textsc{Arthur Schnitzler}\pend{}\pstart{}\textsc{\textcolor{pink}{\begin{otherlanguage}{french}Vienne\end{otherlanguage}}{}\ledrightnote{\textcolor{pink}{Wien}}}\pend{}\pstart{}\textsc{\textcolor{pink}{I. Giselastraße 11.}{}\ledrightnote{\textcolor{pink}{Bösendorferstraße}}}\pend{}{\bigskip}\pstart
           \raggedleft{}{\pb}\textcolor{pink}{Paris}{}\ledrightnote{\textcolor{pink}{Paris}}, 3. Dezember.\pend
           \pstart\center{}Mein lieber Arthur!\pend\pstart
           Ich bin in \textcolor{pink}{Paris}{}\ledrightnote{\textcolor{pink}{Paris}}, das iſt nicht mehr zu leugnen,
               und in den erſten äußeren Eindrücken habe ich beſtätigt gefunden, was Du mir
               geſchrieben: Das iſt eher \label{K_L02673-2v}\edtext{heimlich}{\lemma{\textnormal{\emph{heimlich}}}\Cendnote{\textnormal{im Sinne von: heimatlich
                  (das Gegenteil von ›unheimlich‹)}}}\label{K_L02673-2h} als fremd, viel weniger fremd als \textcolor{pink}{Brüſſel}{}\ledrightnote{\textcolor{pink}{Brüssel}}; das iſt im Weſentlichen \textcolor{pink}{Wien}{}\ledrightnote{\textcolor{pink}{Wien}}, nur farbiger und lebensvoller. Freilich, was mich hier im
                  \textcolor{brown}{Büreau}{}\ledrightnote{→\textcolor{brown}{Pariser Büro der Frankfurter Zeitung}} erwartetete, war
               geeignet, alle freundlichen Eindrücke des Anfangs zu verwiſchen. Ich ſehe es jetzt
               klar, \label{K_L02673-5v}\edtext{was ich Dir ſchrieb}{\lemma{\textnormal{\emph{was ich Dir ſchrieb}}}\Cendnote{\textnormal{siehe Paul Goldmann an Arthur Schnitzler, 15. 11. 1891}}}\label{K_L02673-5h}: zu meinem Beſten hat man mich nicht hergeſandt; es wird
               ein wilder Kampf werden, ſolange ich die Kräfte habe; und auf die Dauer iſt die
               Stellung unhaltbar. Dies unter uns. Wundre Dich nicht, wenn ich Dir in der erſten
               Zeit wenig ſchreibe. Meine Arbeitslaſt hat ſich verfünffacht. Mein Arbeitstag iſt
               von 7 Uhr Morgens bis 1 Uhr Nachts. Viele Grüße an Dich, \textsc{\textcolor{blue}{Kapper}{}\ledrightnote{\textcolor{blue}{Friedrich Kapper}}}, \textsc{\textcolor{blue}{Richard}{}\ledrightnote{\textcolor{blue}{Richard Beer-Hofmann}}} u. \textsc{\textcolor{blue}{Loris}{}\ledrightnote{\textcolor{blue}{Hugo von Hofmannsthal}}}. Dein \spacefill\mbox{P. G.}\pend
           \pstart
           \noindent{}\label{T_L02673-1v}\edtext{Adreſſe: \textcolor{brown}{\textsc{\textcolor{pink}{51. Rue Vivienne}{}\ledrightnote{\textcolor{pink}{rue Vivienne}}}, »\textsc{\textcolor{brown}{Gazette de Francfort}{}\ledrightnote{\textcolor{brown}{Frankfurter Zeitung}}}}{}\ledrightnote{→\textcolor{brown}{Pariser Büro der Frankfurter Zeitung}}«.}{\lemma{\textnormal{\emph{Adreſſe: … Francfort«.}}}\Cendnote{\textnormal{kopfüber am oberen Rand}}}\label{T_L02673-1h}\pend
           \endnumbering\briefempfaengerindex{Schnitzler, Arthur@\textsc{Schnitzler, Arthur}!zzzGoldmann, Paul@\emph{von Paul Goldmann}!1891-12-031@{3. 12. 1891}|)be}\mylabel{h}  \normalsize

\doendnotes{C}
\bigskip
\vfill

\clearpage

\footnotesize

\lohead{\textsc{register}}

% Definiere theindex-Environment komplett neu ohne reledmac
\makeatletter
\renewenvironment{theindex}{%
  \section*{\indexname}%
  \setlength{\parindent}{0pt}%
  \setlength{\parskip}{0pt plus 0.3pt}%
  \let\item\@idxitem
}{%
  \clearpage
}
\makeatother

\IfFileExists{\jobname-pw.ind}{\input{\jobname-pw.ind}}{}

\end{document}

      