%% latex-korrekturansicht-vorspann.tex
%% Vorspann für die Korrekturansicht.
%% Lädt die gemeinsame Datei latex-vorspann.tex mit gesetztem Schalter.

\newif\ifkorrekturansicht
\korrekturansichttrue

\input{../tex-inputs/latex-vorspann}


               \section[Richard Beer-Hofmann an Arthur Schnitzler, {[}11. 12. 1907{]}]{ Richard Beer-Hofmann an Arthur Schnitzler, {[}11. 12. 1907{]}}\nopagebreak\mylabel{v}\rehead{ }\normalsize\beginnumbering\briefempfaengerindex{Schnitzler, Arthur@\textsc{Schnitzler, Arthur}!zzzBeer-Hofmann, Richard@\emph{von Richard Beer-Hofmann}!1907-12-111@{{[}11. 12. 1907{]}}|(be} \toendnotes[C]{\smallbreak\pagebreak[2]} \Standort{CUL, Schnitzler, B 8.}
\physDesc{Brief, 2 Blätter (Briefpapier mit Trauerrand. Die Blätter nummeriert: »I« beziehungsweise »II«), 6 Seiten
\newline{}Handschrift: Bleistift, lateinische Kurrent
\newline{}Schnitzler: mit Bleistift datiert: »11/\textcolor{gray}{1}2 907« \newline{}Ordnung: mit Bleistift von unbekannter Hand nummeriert:
                                    »161?« }\buchAbdrucke{\weitereDrucke{Arthur Schnitzler, Richard Beer-Hofmann: \emph{Briefwechsel 1891–1931}. Hg. Konstanze Fliedl. Wien, Zürich: \emph{Europaverlag} 1992, S. 187–188.} }\toendnotes[C]{\smallbreak}\pstart
           \raggedleft{}{\pb}Mittwoch\pend
           \pstart
           Lieber Arthur! \textcolor{blue}{Paula}{}\ledrightnote{\textcolor{blue}{Paula Beer-Hofmann}} hat vorgestern bei Ihrer \textcolor{blue}{Mama}{}\ledrightnote{→\textcolor{blue}{Louise Schnitzler}} angefragt und – (allerdings nicht direkt
               durch Ihre \textcolor{blue}{Mama}{}\ledrightnote{→\textcolor{blue}{Louise Schnitzler}}) erfahren,
               das es ein leichter Scharlachfall ist, und daß das Fieber zurückgeht.
               Heute habe ich telephonisch mit Ihrer \textcolor{blue}{Mama}{}\ledrightnote{→\textcolor{blue}{Louise Schnitzler}} selbst gesprochen, und erfahren daß \uuline{S}ie selbst beunruhigt sind weil trotz des Zurückgehen des
               Fiebers noch i{\geminationm}er Delieriren vorhanden ist. Ihre \textcolor{blue}{Mama}{}\ledrightnote{→\textcolor{blue}{Louise Schnitzler}} versichert {\pb}mich, daß der behandelnde \textcolor{blue}{Arzt}{}\ledrightnote{→\textcolor{blue}{Jacob Pollak}} erklärt hat, daß trotz
               dieser unangenehmen Begleiterscheinung \uline{kein} Anlass zu
               Besorgnis ist. Ich schreibe Ihnen dies Alles, weil ich {\pb}nicht weiss ob Sie dem \textcolor{blue}{Arzt}{}\ledrightnote{→\textcolor{blue}{Jacob Pollak}} glauben. \uline{Mir} gegenüber ist kein Grund zum Schönfärben
               vorhanden. Ich wünsche vor Allem von ganzem Herzen daß es bald und rasch {\pb}besser geht, dann daß Sie sich
               nicht in nutzlosem Schwarzsehen \strikeout{sich} verzehren. Ich
               weiss ich weiss – ich habe leicht reden – aber vielleicht beruhigt es Sie doch ein
                  \uline{ganz klein} wenig, daß man mir den
               Krankheitsverlauf, als unangenehm, – als überflüssig complicirt – aber \uline{nicht} als gefährlich dargestellt hat.\pend
           \pstart
           {\pb}Nur darum schreib ich Ihnen, und
               weil ich denke, daß mitten unter Wichtigerem, dies kleine unwichtige – dass ich und
                  \textcolor{blue}{Paula}{}\ledrightnote{\textcolor{blue}{Paula Beer-Hofmann}}{ }{\pb}oft im Tage an Sie Beide denken,
               und starke und gute Wünsche für Sie im Herzen haben – weil es vielleicht doch für
               eine Sekunde Ihnen angenehm sein könnte.\pend
           \pstart
           Von Herzen wie immer{\\[\baselineskip]}Ihr \spacefill\mbox{Richard}\pend
           \leftskip=0em{}\endnumbering\briefempfaengerindex{Schnitzler, Arthur@\textsc{Schnitzler, Arthur}!zzzBeer-Hofmann, Richard@\emph{von Richard Beer-Hofmann}!1907-12-111@{{[}11. 12. 1907{]}}|)be}\mylabel{h}  \normalsize

\doendnotes{C}
\bigskip
\vfill

\clearpage

\footnotesize

\lohead{\textsc{register}}

% Definiere theindex-Environment komplett neu ohne reledmac
\makeatletter
\renewenvironment{theindex}{%
  \section*{\indexname}%
  \setlength{\parindent}{0pt}%
  \setlength{\parskip}{0pt plus 0.3pt}%
  \let\item\@idxitem
}{%
  \clearpage
}
\makeatother

\IfFileExists{\jobname-pw.ind}{\input{\jobname-pw.ind}}{}

\end{document}

      