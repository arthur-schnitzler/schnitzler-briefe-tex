%% latex-korrekturansicht-vorspann.tex
%% Vorspann für die Korrekturansicht.
%% Lädt die gemeinsame Datei latex-vorspann.tex mit gesetztem Schalter.

\newif\ifkorrekturansicht
\korrekturansichttrue

\input{../tex-inputs/latex-vorspann}


\section[Arthur Schnitzler an Carl Sternheim, 27. 12. 1911]{L04014 Arthur Schnitzler an Carl Sternheim, 27. 12. 1911}
\nopagebreak\mylabel{L04014v}
\rehead{ }\normalsize\beginnumbering\briefempfaengerindex{Sternheim, Carl@\textsc{Sternheim, Carl}!zzzSchnitzler, Arthur@\emph{von Arthur Schnitzler}!1911-12-271@{27. 12. 1911}|(be}
\toendnotes[C]{\smallbreak\pagebreak[2]}
\correspDesc{Versand  durch Arthur Schnitzler am 27. 12. 1911 in Wien
\newline{}Erhalt  durch Carl Sternheim im Zeitraum [28. 12. 1911 – 1. 1. 1912?] in Pullach im Isartal}\toendnotes[C]{\smallbreak}
\Standort{DLA, A:Schnitzler, HS.1985.1.2002.}
\physDesc{Brief, Durchschlag, 1 Blatt, 1 Seite, 603 Zeichen
\newline{}Schreibmaschine}
\buchAbdrucke{\weitereDrucke{Arthur Schnitzler: \emph{Briefe 1875–1912}. Herausgegeben von Therese Nickl und Heinrich Schnitzler. Frankfurt am Main: \emph{S. Fischer} 1981, S. 687.} }\toendnotes[C]{\smallbreak}
\pstart
           \raggedleft{}{\pb}27. 12. 1911.\pend
           
\pstart{}Sehr geehrter Herr Sternheim.\pend\vspace{0.5em}
\pstart
           Der \textcolor{green}{Aufruf}\pwindex{Sternheim, Carl 1.\,4.\,1878 Leipzig – 3.\,11.\,1942 Brüssel@\textsc{Sternheim, Carl} (1.\,4.\,1878 Leipzig – 3.\,11.\,1942 Brüssel), \emph{Schriftsteller}!Aufruf [Aus der Mitte des Publikums…]@\strich\emph{Aufruf [Aus der Mitte des Publikums…]}|pwv}\pwindex{Wedekind, Frank 24.\,7.\,1864 Hannover – 9.\,3.\,1918 München@\textsc{Wedekind, Frank} (24.\,7.\,1864 Hannover – 9.\,3.\,1918 München), \emph{Schriftsteller, Schauspieler, Schriftsteller}!Aufruf [Aus der Mitte des Publikums…]@\strich\emph{Aufruf [Aus der Mitte des Publikums…]}|pwv}\pwindex{Borngräber, Otto 19.\,11.\,1874 Stendal – 19.\,10.\,1916 Lugano@\textsc{Borngräber, Otto} (19.\,11.\,1874 Stendal – 19.\,10.\,1916 Lugano), \emph{Schriftsteller}!Aufruf [Aus der Mitte des Publikums…]@\strich\emph{Aufruf [Aus der Mitte des Publikums…]}|pwv}\pwindex{Eulenberg, Herbert 25.\,1.\,1876 Mülheim [Köln] – 4.\,9.\,1949 Düsseldorf@\textsc{Eulenberg, Herbert} (25.\,1.\,1876 Mülheim [Köln] – 4.\,9.\,1949 Düsseldorf), \emph{Schriftsteller}!Aufruf [Aus der Mitte des Publikums…]@\strich\emph{Aufruf [Aus der Mitte des Publikums…]}|pwv}{}\ledrightnote{{$\rightarrow$}\emph{\textcolor{green}{Aufruf [Aus der Mitte des Publikums…]}}}, den Sie im Namen von \textcolor{blue}{Wedekind}\pwindex{Wedekind, Frank 24.\,7.\,1864 Hannover – 9.\,3.\,1918 München@\textsc{Wedekind, Frank} (24.\,7.\,1864 Hannover – 9.\,3.\,1918 München), \emph{Schriftsteller, Schauspieler, Schriftsteller}|pw}{}\ledrightnote{\textcolor{blue}{Frank Wedekind}}, \textcolor{blue}{Eulenberg}\pwindex{Eulenberg, Herbert 25.\,1.\,1876 Mülheim [Köln] – 4.\,9.\,1949 Düsseldorf@\textsc{Eulenberg, Herbert} (25.\,1.\,1876 Mülheim [Köln] – 4.\,9.\,1949 Düsseldorf), \emph{Schriftsteller}|pw}{}\ledrightnote{\textcolor{blue}{Herbert Eulenberg}}, \textcolor{blue}{Borngräber}\pwindex{Borngräber, Otto 19.\,11.\,1874 Stendal – 19.\,10.\,1916 Lugano@\textsc{Borngräber, Otto} (19.\,11.\,1874 Stendal – 19.\,10.\,1916 Lugano), \emph{Schriftsteller}|pw}{}\ledrightnote{\textcolor{blue}{Otto Borngräber}} und in Ihrem eigenen mir zur Unterzeichnung zuzusenden so
               freundlich sind, widerspricht meinen Erfahrungen, meinen Ansichten und meinem Gefühl
               vom Verhältnis des Dichters zum Publikum so sehr, dass ich mich ausserstande erklären
               muss, ihn zu unterschreiben, trotzdem ich mich in meinem Widerwillen gegen die Zensur
               und gegen deren wahrhaft aufreizende Übergriffe, insbesondere in der letzten Zeit,
               mit Ihnen allen, meine Herren, völlig eines Sinnes weiss.\pend
           
\pstart
           Mit vorzüglicher Hochachtung{\\[\baselineskip]} Ihr sehr ergebener\pend
           \leftskip=0em{}\selectlanguage{ngerman}\endnumbering\briefempfaengerindex{Sternheim, Carl@\textsc{Sternheim, Carl}!zzzSchnitzler, Arthur@\emph{von Arthur Schnitzler}!1911-12-271@{27. 12. 1911}|)be}\mylabel{L04014h}
\begin{anhang}
\end{anhang}\normalsize

\doendnotes{C}
\bigskip
\vfill

\clearpage

\footnotesize

\lohead{\textsc{register}}

% Definiere theindex-Environment komplett neu ohne reledmac
\makeatletter
\renewenvironment{theindex}{%
  \section*{\indexname}%
  \setlength{\parindent}{0pt}%
  \setlength{\parskip}{0pt plus 0.3pt}%
  \let\item\@idxitem
}{%
  \clearpage
}
\makeatother

\IfFileExists{\jobname-pw.ind}{\input{\jobname-pw.ind}}{}

\end{document}

      