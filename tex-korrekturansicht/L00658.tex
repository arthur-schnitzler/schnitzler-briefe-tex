%% latex-korrekturansicht-vorspann.tex
%% Vorspann für die Korrekturansicht.
%% Lädt die gemeinsame Datei latex-vorspann.tex mit gesetztem Schalter.

\newif\ifkorrekturansicht
\korrekturansichttrue

\input{../tex-inputs/latex-vorspann}


               \section[Arthur Schnitzler an Hermann Bahr, 23. 3. 1897]{ Arthur Schnitzler an Hermann Bahr, 23. 3. 1897}\nopagebreak\mylabel{v}\rehead{ }\normalsize\beginnumbering\briefempfaengerindex{Bahr, Hermann@\textsc{Bahr, Hermann}!zzzSchnitzler, Arthur@\emph{von Arthur Schnitzler}!1897-03-231@{23. 3. 1897}|(be} \toendnotes[C]{\smallbreak\pagebreak[2]} \Standort{TMW, HS AM 23329 Ba.}
\physDesc{Brief, 1 Blatt, 4 Seiten
\newline{}Handschrift: Bleistift, deutsche Kurrent\newline{}Ordnung: Lochung }\buchAbdrucke{\weitereDrucke{1) \emph{23. 3. 1897.} In: Arthur Schnitzler: \emph{The Letters of Arthur Schnitzler to Hermann Bahr}. Edited, annotated, and with an introduction, by Donald G.
                        Daviau. Chapel Hill: \emph{The University of North Carolina Press} 1978, S. 60–61 (University of North Carolina studies in the Germanic languages
                        and literatures, 89).} \weitereDrucke{2) Hermann Bahr, Arthur Schnitzler: \emph{Briefwechsel, Aufzeichnungen, Dokumente (1891–1931)}. Hg. Kurt Ifkovits und Martin Anton Müller. Göttingen: \emph{Wallstein} 2018, S. 139–140.} }\toendnotes[C]{\smallbreak}\pstart
           \noindent{}{\pb}Lieber Hermann, wie ka{\geminationn} ich dir den
               Titel ſagen, wenn ich noch nicht weiſs was ich leſe? Das zu entſcheiden ko{\geminationm}en wir ja morgen zuſa{\geminationm}en.
               Wahrſcheinlich eine \label{K_L00658_1v}\edtext{\textcolor{green}{Novellette}{}\ledrightnote{→\textcolor{green}{Der Ehrentag}}}{\lemma{\textnormal{\emph{Novellette}}}\Cendnote{\textnormal{\emph{\textcolor{green}{Der Ehrentag}} (Erstdruck in: \emph{\textcolor{brown}{Die Romanwelt}}, Jg. 5 (1897/1898),
                     H. 16, {[}15.{]} 1. 1898, S. 507–516).}}}\label{K_L00658_1h}, die ich
               vorgeſtern zu Ende geſchrieben, {\pb}vielleicht \label{K_L00658_2v}\edtext{\textcolor{green}{eine, die morgen fertig wird}{}\ledrightnote{→\textcolor{green}{Die Toten schweigen}}}{\lemma{\textnormal{\emph{eine, … wird}}}\Cendnote{\textnormal{\emph{\textcolor{green}{Die Toten schweigen}} (Erstdruck in: \emph{\textcolor{green}{Cosmopolis}}, Jg. 2, Bd. 8, Nr. 22,
                        1. 10. 1897, S. 193–211).}}}\label{K_L00658_2h} – am Ende was ganz
               anderes. Es iſt nemlich zu bedenken dſs du, \textcolor{blue}{Hirſchfeld}{}\ledrightnote{\textcolor{blue}{Georg Hirschfeld}} und ich Novelletten leſen, (\textcolor{blue}{Hugo}{}\ledrightnote{\textcolor{blue}{Hugo von Hofmannsthal}} wirkt nicht mit) – daſs alſo das Progra{\geminationm}
               von einer beiſpielloſen Langwei{\pb}ligkeit ſein wird.
               Meine Hoffnung iſt, dſs uns morgen Abend doch noch was geſcheidtes einfällt. – \textcolor{blue}{Hirſchfelds}{}\ledrightnote{\textcolor{blue}{Georg Hirschfeld}} Geſchichte heißt: »\label{K_L00658_3v}\edtext{\textcolor{green}{Bei beiden}{}\ledrightnote{\textcolor{green}{Bei Beiden}}}{\lemma{\textnormal{\emph{Bei beiden}}}\Cendnote{\textnormal{Erstdruck in: \emph{\textcolor{green}{Neue deutsche Rundschau}},
                     Jg. 5, H. 10, 1. 10. 1894, S. 919–927, Erstausgabe in
                        \emph{\textcolor{green}{Dämon Kleist. Novellen}}. Berlin: \emph{\textcolor{brown}{S. Fischer}}{ }1895, S. 152–179.}}}\label{K_L00658_3h}.« Von mir ka{\geminationn}ſt du ſagen, daſs ich eine ungedruckte Novellette
               vorleſen werde. We{\geminationn} das Programm Freitag gedruckt wird,
               iſt Zeit genug, meiner Ansicht nach. Sterben {\pb}ſterb’ \damage{ich}, aber hetzen l\damage{a}ſs ich mich nicht.\pend
           \pstart Herzlich dein \spacefill\mbox{Arthur}\pend{}\pstart
           23. 3. 97.\pend
           \pstart
           Der \label{K_L00658_4v}\edtext{Donnerſtag Notiz}{\lemma{\textnormal{\emph{Donnerſtag Notiz}}}\Cendnote{\textnormal{nicht nachgewiesen}}}\label{K_L00658_4h} wäre
                  jedenfalls mehr Geſchmack zu wünſchen als \label{K_L00658_5v}\edtext{\textcolor{green}{die von Sonntag}{}\ledrightnote{→\textcolor{green}{[Ankündigung der Vorlesung]}}}{\lemma{\textnormal{\emph{die von Sonntag}}}\Cendnote{\textnormal{\emph{\textcolor{green}{}}Etwa in: \emph{\textcolor{brown}{Neue
                           Freie Presse}}, 21. 3. 1897, S. 9: »– Am
                        Sonntag den 28. d., Abends, findet im \textcolor{pink}{Bösendorfer-Saale} eine Vorlesung statt, die von vier der
                        bekanntesten Vertreter jungdeutscher Literatur zu wohlthätigem Zwecke
                        veranstaltet wird. Am Vorlesertische werden erscheinen als Interpreten ihrer
                        eigenen Werke: \textcolor{blue}{Hermann \so{Bahr}}, der erst jüngst anläßlich der Aufführung seines ›\textcolor{green}{Tschaperl}‹ so vielbesprochene Führer Jung-\textcolor{pink}{Wien}s; \textcolor{blue}{Arthur \so{Schnitzler}}, der Verfasser der ›\textcolor{green}{Liebelei}‹; \textcolor{blue}{Hugo \so{v. Hoffmannsthal}} (\textcolor{blue}{Loris}), ein interessantes Talent
                        des modernen \textcolor{pink}{Oesterreich}, und \textcolor{blue}{Georg \so{Hirschfeld}}, dessen ›\textcolor{green}{Mütter}‹ vor Kurzem am \textcolor{pink}{Deutschen Volkstheater} einen
                        Sensations-Erfolg errangen. Bürgen schon die Namen der Vorleser für den
                        interessanten Verlauf des Abends, so noch mehr der Umstand, daß die vier
                        Herren fast durchwegs neue oder mindestens für \textcolor{pink}{Wien} neue Dichtungen zum Vortrage bringen werden. Der Kartenverkauf
                        für diesen originellen literarischen Abend findet bei \textcolor{pink}{Bösendorfer}{ }statt.«}}}\label{K_L00658_5h} verrieth. Wir ſind
                  ja nicht Mitglieder des Vereins »\textcolor{brown}{Gemütliche
                     Harmonie}{}\ledrightnote{\textcolor{brown}{Gemütliche Harmonie}}«, daſs man uns durch \label{K_L00658_6v}\edtext{\textsc{Epitheta}}{\lemma{\textnormal{\emph{Epitheta}}}\Cendnote{\textnormal{schmückende Beiworte}}}\label{K_L00658_6h} erklären
                  muſs.\pend
           \endnumbering\briefempfaengerindex{Bahr, Hermann@\textsc{Bahr, Hermann}!zzzSchnitzler, Arthur@\emph{von Arthur Schnitzler}!1897-03-231@{23. 3. 1897}|)be}\mylabel{h}  \normalsize

\doendnotes{C}
\bigskip
\vfill

\clearpage

\footnotesize

\lohead{\textsc{register}}

% Definiere theindex-Environment komplett neu ohne reledmac
\makeatletter
\renewenvironment{theindex}{%
  \section*{\indexname}%
  \setlength{\parindent}{0pt}%
  \setlength{\parskip}{0pt plus 0.3pt}%
  \let\item\@idxitem
}{%
  \clearpage
}
\makeatother

\IfFileExists{\jobname-pw.ind}{\input{\jobname-pw.ind}}{}

\end{document}

      