%% latex-korrekturansicht-vorspann.tex
%% Vorspann für die Korrekturansicht.
%% Lädt die gemeinsame Datei latex-vorspann.tex mit gesetztem Schalter.

\newif\ifkorrekturansicht
\korrekturansichttrue

\input{../tex-inputs/latex-vorspann}


\renewcommand{\erwaehntePersonen}{Personen: Richard Beer-Hofmann}
\renewcommand{\erwaehnteOrte}{Orte: Berggasse, Grillparzerstraße, I., Innere Stadt, Wien}
\renewcommand{\erwaehnteWerke}{}
\section[Felix Salten an Arthur Schnitzler, 30. 4. 1893]{Felix Salten an Arthur Schnitzler, 30. 4. 1893}
\nopagebreak\mylabel{v}
\rehead{ }\normalsize\beginnumbering\briefempfaengerindex{Schnitzler, Arthur@\textsc{Schnitzler, Arthur}!zzzSalten, Felix@\emph{von Felix Salten}!1893-04-302@{30. 4. 1893}|(be}
\toendnotes[C]{\smallbreak\pagebreak[2]}\Standort{CUL, Schnitzler, B 89, A 1.}
\physDesc{Postkarte, 293 Zeichen
\newline{}Handschrift: Bleistift, lateinische Kurrent
\newline{}Versand: 1) Stempel: »\nobreak{}\oindex{I., Innere Stadt@\textbf{I., Innere Stadt}, \emph{A.ADM3}|pwk}Wien 1/1, 30 IV 93, 7 30V\nobreak{}«.   2) Stempel: »\nobreak{}\oindex{I., Innere Stadt@\textbf{I., Innere Stadt}, \emph{A.ADM3}|pwk}Wien 1/1 10, \textcolor{gray}{30 IV 93}\nobreak{}«. 
\newline{}Ordnung: mit Bleistift von unbekannter Hand nummeriert: »24« }\toendnotes[C]{\smallbreak}\pstart{}{\pb}Herrn D\textsuperscript{r} Arthur Schnitzler\pend{}\pstart{}\textcolor{pink}{I. Grillparzerstraße}{}\ledrightnote{\textcolor{pink}{Grillparzerstraße}}\pend{}\pstart{}N\textsuperscript{o} 7\pend{}
{\bigskip}
\pstart
           \noindent{}{\pb}lieber Arthur!{ }\textcolor{blue}{BH.}{}\ledrightnote{\textcolor{blue}{Richard Beer-Hofmann}} kann sich für 
               morgen, d. i. heute{ }\label{K_L03121-1v}\edtext{Nachmittag}{\lemma{\textnormal{\emph{Nachmittag}}}\Cendnote{\textnormal{siehe A. S.: \emph{Tagebuch}, 30. 4. 1893}}}\label{K_L03121-1h} nicht binden, ein Rendezvous bei ihm also ausgeschloßen. – Ich bin von
                  ½ 4 Uhr an frei: erwarte zu Mittag Nachricht von Ihnen,
               eventuell besuchen Sie mich Vormittag im \textcolor{pink}{Bureau}{}\ledrightnote{{$\rightarrow$}\textcolor{pink}{Berggasse}}.\pend
           
\pstart
           Herzlichst {\\[\baselineskip]}\spacefill\mbox{Salten}\pend
           \leftskip=0em{}\endnumbering\briefempfaengerindex{Schnitzler, Arthur@\textsc{Schnitzler, Arthur}!zzzSalten, Felix@\emph{von Felix Salten}!1893-04-302@{30. 4. 1893}|)be}\mylabel{h}  \normalsize

\doendnotes{C}
\bigskip
\vfill

\clearpage

\footnotesize

\lohead{\textsc{register}}

% Definiere theindex-Environment komplett neu ohne reledmac
\makeatletter
\renewenvironment{theindex}{%
  \section*{\indexname}%
  \setlength{\parindent}{0pt}%
  \setlength{\parskip}{0pt plus 0.3pt}%
  \let\item\@idxitem
}{%
  \clearpage
}
\makeatother

\IfFileExists{\jobname-pw.ind}{\input{\jobname-pw.ind}}{}

\end{document}

      