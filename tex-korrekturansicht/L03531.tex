%% latex-korrekturansicht-vorspann.tex
%% Vorspann für die Korrekturansicht.
%% Lädt die gemeinsame Datei latex-vorspann.tex mit gesetztem Schalter.

\newif\ifkorrekturansicht
\korrekturansichttrue

\input{../tex-inputs/latex-vorspann}


\renewcommand{\erwaehntePersonen}{Personen:  Albert von Sachsen, Paul Goldmann, Raphael Löwenfeld, Paul Marx, Olga Schnitzler, Elisabeth Steinrück}
\renewcommand{\erwaehnteInstitutionen}{Institutionen: Deutsches Theater Berlin}
\renewcommand{\erwaehnteOrte}{Orte: Berlin, Dessauer Straße, Dresden, Wien}
\renewcommand{\erwaehnteWerke}{}
\section[ Paul Goldmann an Olga Gussmann, 20. 6. {[}1902{]}]{Paul Goldmann an Olga Gussmann, 20. 6. {[}1902{]}}
\nopagebreak\mylabel{v}
\rehead{ }\normalsize\beginnumbering\briefempfaengerindex{Schnitzler, Olga@\textsc{Schnitzler, Olga}!zzzGoldmann, Paul@\emph{von Paul Goldmann}!1902-06-202@{20. 6. {[}1902{]}}|(be}
\toendnotes[C]{\smallbreak\pagebreak[2]}\Standort{DLA, A:Schnitzler, HS.NZ85.1.5247.}
\physDesc{Brief, 1 Blatt, 1 Seite, 389 Zeichen
\newline{}Handschrift: blaue Tinte, deutsche Kurrent}\toendnotes[C]{\smallbreak}
\pstart
           \noindent{}\raggedleft{}{\pb}\textcolor{gray}{\textbf{\textcolor{pink}{DESSAUERSTRASSE 19}{}\ledrightnote{\textcolor{pink}{Dessauer Straße}}}}\pend
           
\pstart
           \textcolor{pink}{Berlin}{}\ledrightnote{\textcolor{pink}{Berlin}}, 20. Juni.\pend
           
\pstart\center{}Liebe Freundin,\pend
\pstart
           Eben bekomme ich \label{K_L03531-1v}\edtext{Marſchordre}{\lemma{\textnormal{\emph{Marſchordre}}}\Cendnote{\textnormal{\textcolor{blue}{Goldmann} verwendet das französische Wort »ordre«, wechselt aber nicht, 
            wie zu erwarten wäre, für das Fremdwort in lateinische Kurrentschrift.}}}\label{K_L03531-1h} nach \label{K_L03531-2v}\edtext{\textcolor{pink}{Dresden}{}\ledrightnote{\textcolor{pink}{Dresden}}}{\lemma{\textnormal{\emph{Dresden}}}\Cendnote{\textnormal{Die fehlende Jahresangabe des Korrespondenzstücks lässt sich über den 
                  Inhalt erschließen. \textcolor{blue}{Albert von
                     Sachsen} starb am 19. 6. 1902, am 23. 6. 1902 wurde er in \textcolor{pink}{Dresden} beerdigt.
                  \textcolor{blue}{Goldmann} hielt sich nachweislich am 24. 6. 1902 in der Stadt auf.}}}\label{K_L03531-2h} (Beerdigung des \textcolor{blue}{König}{}\ledrightnote{{$\rightarrow$}\textcolor{blue}{Albert von Sachsen}}s). In fliegender Eile alſo: vielen
               Dank für Ihren lieben Brief! Sorgen Sie, bitte, dafür, daß \textsc{\textcolor{blue}{Liesl}{}\ledrightnote{\textcolor{blue}{Elisabeth Steinrück}}} die \label{K_L03531-3v}\edtext{Angelegenheit mit \textsc{\textcolor{blue}{Löwenfeld}{}\ledrightnote{\textcolor{blue}{Raphael Löwenfeld}}}}{\lemma{\textnormal{\emph{Angelegenheit mit Löwenfeld}}}\Cendnote{\textnormal{siehe Paul Goldmann an Arthur Schnitzler, 16. 6. [1902]}}}\label{K_L03531-3h} nicht verſchlampt. \textsc{\textcolor{blue}{Paul}{}\ledrightnote{\textcolor{blue}{Paul Marx}}} werde ich in meine \label{K_L03531-4v}\edtext{Obhut}{\lemma{\textnormal{\emph{Obhut}}}\Cendnote{\textnormal{Das steht womöglich in Zusammenhang mit dem
                  im September des Jahres beginnenden Engagement \textcolor{blue}{Paul Marx}’ am \emph{\textcolor{brown}{Deutschen Theater Berlin}}.}}}\label{K_L03531-4h} nehmen. Ihnen wünſche ich von Herzen das
               Allerbeſte und ſende Ihnen viele Grüße.\pend
           
\pstart
           Ihr getreuer, halb todt gehetzter {\\[\baselineskip]}\spacefill\mbox{Paul Goldmann.}\pend
           \leftskip=0em{}\endnumbering\briefempfaengerindex{Schnitzler, Olga@\textsc{Schnitzler, Olga}!zzzGoldmann, Paul@\emph{von Paul Goldmann}!1902-06-202@{20. 6. {[}1902{]}}|)be}\mylabel{h}  \normalsize

\doendnotes{C}
\bigskip
\vfill

\clearpage

\footnotesize

\lohead{\textsc{register}}

% Definiere theindex-Environment komplett neu ohne reledmac
\makeatletter
\renewenvironment{theindex}{%
  \section*{\indexname}%
  \setlength{\parindent}{0pt}%
  \setlength{\parskip}{0pt plus 0.3pt}%
  \let\item\@idxitem
}{%
  \clearpage
}
\makeatother

\IfFileExists{\jobname-pw.ind}{\input{\jobname-pw.ind}}{}

\end{document}

      