%% latex-korrekturansicht-vorspann.tex
%% Vorspann für die Korrekturansicht.
%% Lädt die gemeinsame Datei latex-vorspann.tex mit gesetztem Schalter.

\newif\ifkorrekturansicht
\korrekturansichttrue

\input{../tex-inputs/latex-vorspann}


\section[Theodor Herzl an Arthur Schnitzler, 19. 5. 1895]{L03862 Theodor Herzl an Arthur Schnitzler, 19. 5. 1895}
\nopagebreak\mylabel{L03862v}
\rehead{ }\normalsize\beginnumbering\briefempfaengerindex{Schnitzler, Arthur@\textsc{Schnitzler, Arthur}!zzzHerzl, Theodor@\emph{von Theodor Herzl}!1895-05-192@{19. 5. 1895}|(be}
\toendnotes[C]{\smallbreak\pagebreak[2]}
\correspDesc{Versand  durch Theodor Herzl am 19. 5. 1895 in Paris
\newline{}Erhalt  durch Arthur Schnitzler im Zeitraum [20. 5. 1895
                  – 24. 5. 1895?] in Wien}\toendnotes[C]{\smallbreak}
\Standort{CUL, Schnitzler, B 39.}
\physDesc{Brief, 1 Blatt, 3 Seiten, 1118 Zeichen
\newline{}Handschrift: schwarze Tinte, lateinische Kurrent
\newline{}Schnitzler: mit Bleistift datiert: »19/5 85« 
\newline{}Ordnung: mit Bleistift von unbekannter Hand nummeriert: »41« }
\buchAbdrucke{\weitereDrucke{Theodor Herzl: \emph{Briefe Anfang Mai 1895 – Anfang Dezember 1898}. Bearbeitet von Barbara Schäfer in Zusammenarbeit mit Sofia Gelmann, Chaya Harel, Ines Rubin und Daisy Ticho. Berlin, Frankfurt am Main, Wien: \emph{Propyläen} 1990, S. 41–42 (Briefe und Tagebücher. Herausgegeben von Alex Bein, Hermann Greive, Moshe Schaerf, Julius H. Schoeps und Johannes Wachten, 4).} }\toendnotes[C]{\smallbreak}
\pstart{}{\pb}Theurer Freund!\pend\vspace{0.5em}
\pstart
           \textcolor{blue}{Teweles}\pwindex{Teweles, Heinrich 13.\,11.\,1856 Prag – 9.\,8.\,1927 Prein an der Rax@\textsc{Teweles, Heinrich} (13.\,11.\,1856 Prag – 9.\,8.\,1927 Prein an der Rax), \emph{Schriftsteller, Journalist, Theaterleiter}|pw}{}\ledrightnote{\textcolor{blue}{Heinrich Teweles}} hat sich, wie nicht zu bezweifeln war,
               zur Discretion verpflichtet. Der \textcolor{blue}{Mensch}\pwindex{Teweles, Heinrich 13.\,11.\,1856 Prag – 9.\,8.\,1927 Prein an der Rax@\textsc{Teweles, Heinrich} (13.\,11.\,1856 Prag – 9.\,8.\,1927 Prein an der Rax), \emph{Schriftsteller, Journalist, Theaterleiter}|pwv}{}\ledrightnote{{$\rightarrow$}\emph{\textcolor{blue}{Heinrich Teweles}}} hat mir dazu \label{K_L03862-1v}\edtext{einen
                  Brief}{\lemma{\textnormal{\emph{einen
                  Brief}}}\Cendnote{\textnormal{\textcolor{blue}{Heinrich Teweles}\pwindex{Teweles, Heinrich 13.\,11.\,1856 Prag – 9.\,8.\,1927 Prein an der Rax@\textsc{Teweles, Heinrich} (13.\,11.\,1856 Prag – 9.\,8.\,1927 Prein an der Rax), \emph{Schriftsteller, Journalist, Theaterleiter}|pwk} an \textcolor{blue}{Theodor Herzl}\pwindex{Herzl, Theodor 2.\,5.\,1860 Budapest – 3.\,7.\,1904 Edlach@\textsc{Herzl, Theodor} (2.\,5.\,1860 Budapest – 3.\,7.\,1904 Edlach), \emph{Schriftsteller, Journalist}|pwk}, 16. 5. 1895, \emph{Central Zionist
                              Archives Jerusalem}, H1:1985-2.}}}\label{K_L03862-1} geschrieben, der mich rührte, so viel gute Freundschaft
               spricht daraus. Merkwürdig, ich habe Freunde! Ich übertreibe nicht. \pend
           
\pstart
           Ich bitte Sie also, sich noch in folgender Weise für mich zu ruiniren: Schicken Sie
               den \label{K_L03862-2v}\edtext{beiliegenden Brief}{\lemma{\textnormal{\emph{beiliegenden Brief}}}\Cendnote{\textnormal{\emph{Theodor Herzl an Heinrich Teweles, 19. 5. 1895}. In:
                        \emph{Briefe Anfang Mai 1895 – 1898},
                  S. 38–41.}}}\label{K_L03862-2}{ }\label{K_L03862-3v}\edtext{recommandirt}{\lemma{\textnormal{\emph{recommandirt}}}\Cendnote{\textnormal{per Einschreiben}}}\label{K_L03862-3} an \textcolor{blue}{Teweles}\pwindex{Teweles, Heinrich 13.\,11.\,1856 Prag – 9.\,8.\,1927 Prein an der Rax@\textsc{Teweles, Heinrich} (13.\,11.\,1856 Prag – 9.\,8.\,1927 Prein an der Rax), \emph{Schriftsteller, Journalist, Theaterleiter}|pw}{}\ledrightnote{\textcolor{blue}{Heinrich Teweles}}. (\label{K_L03862-4v}\edtext{Er schreibt mir
               nämlich dass man meine Handschrift erkannt hat, auf dem Couvert meines ersten
                  Briefes}{\lemma{\textnormal{\emph{Er … Briefes}}}\Cendnote{\textnormal{Der Brief von \textcolor{blue}{Teweles}\pwindex{Teweles, Heinrich 13.\,11.\,1856 Prag – 9.\,8.\,1927 Prein an der Rax@\textsc{Teweles, Heinrich} (13.\,11.\,1856 Prag – 9.\,8.\,1927 Prein an der Rax), \emph{Schriftsteller, Journalist, Theaterleiter}|pwk} beginnt folgendermaßen: »Lieber
                     Freund! An der Aufſchrift erkannte ich Ihre Schrift und freute mich ſchon. Sie
                     haben mich alſo in \textcolor{pink}{Paris}\oindex{Paris@\textbf{Paris}, \emph{Hauptstadt}|pw} nicht vergeſſen
                     und das iſt doch wol ſchwerer, als daß ich Sie hier in \textcolor{pink}{Prag}\oindex{Prag@\textbf{Prag}, \emph{Land}|pw} vergeſſe.« }}}\label{K_L03862-4}) \pend
           
\pstart
           Lassen Sie ferner durch \textcolor{blue}{Schick}\pwindex{Schik, Friedrich *~6.\,9.\,1857 Wien@\textsc{Schik, Friedrich} (*~6.\,9.\,1857 Wien), \emph{Notar, Journalist, Dramaturg}|pw}{}\ledrightnote{\textcolor{blue}{Friedrich Schik}} das Manuscript
               des \textcolor{green}{Stückes}\pwindex{Herzl, Theodor 2.\,5.\,1860 Budapest – 3.\,7.\,1904 Edlach@\textsc{Herzl, Theodor} (2.\,5.\,1860 Budapest – 3.\,7.\,1904 Edlach), \emph{Schriftsteller, Journalist}!neue Ghetto. Schauspiel in vier Acten@\strich\emph{Das neue Ghetto. Schauspiel in vier Acten}|pwv}{}\ledrightnote{{$\rightarrow$}\emph{\textcolor{green}{Das neue Ghetto. Schauspiel in vier Acten}}} unter der Adresse:
                  {\pb}Löbliche Direction des \textcolor{brown}{Königl. deutschen Landestheaters}\orgindex{Ständetheater@Ständetheater|pw}{}\ledrightnote{\textcolor{brown}{Ständetheater}}{ }\pend
           
\pstart
           \centering{}\textcolor{pink}{\uline{Prag}}\oindex{Prag@\textbf{Prag}, \emph{Land}|pw}{}\ledrightnote{\textcolor{pink}{Prag}}\pend
           
\pstart
           absenden.\pend
           
\pstart
           Der an die \textcolor{pink}{Berliner}\oindex{Berlin@\textbf{Berlin}, \emph{Hauptstadt}|pw}{}\ledrightnote{\textcolor{pink}{Berlin}}{ }\textcolor{blue}{Directoren}\pwindex{Blumenthal, Oskar 13.\,3.\,1852 Berlin – 24.\,4.\,1917 ebd.@\textsc{Blumenthal, Oskar} (13.\,3.\,1852 Berlin – 24.\,4.\,1917 ebd.), \emph{Schriftsteller, Journalist, Theaterleiter}|pwv}\pwindex{Brahm, Otto 5.\,2.\,1856 Hamburg – 28.\,11.\,1912 Berlin@\textsc{Brahm, Otto} (5.\,2.\,1856 Hamburg – 28.\,11.\,1912 Berlin), \emph{Theaterleiter, Regisseur}|pwv}{}\ledrightnote{{$\rightarrow$}\emph{\textcolor{blue}{Oskar Blumenthal}}{\newline}{$\rightarrow$}\emph{\textcolor{blue}{Otto Brahm}}}
               gerichtete \label{K_L03862-5v}\edtext{Vorwortbrief}{\lemma{\textnormal{\emph{Vorwortbrief}}}\Cendnote{\textnormal{S. die Beilage von siehe Theodor Herzl an Arthur Schnitzler, 1. 1. 1895.}}}\label{K_L03862-5} ist natürlich
               herauszuschneiden.\pend
           
\pstart
           Folgenden Begleitbrief soll \textcolor{blue}{Schick}\pwindex{Schik, Friedrich *~6.\,9.\,1857 Wien@\textsc{Schik, Friedrich} (*~6.\,9.\,1857 Wien), \emph{Notar, Journalist, Dramaturg}|pw}{}\ledrightnote{\textcolor{blue}{Friedrich Schik}}
               schreiben:\pend
           
\pstart
           Geehrte Direction!\pend
           
\pstart
           Beifolgend mein 4 actiges Schauspiel »\textcolor{green}{das
               Ghetto}\pwindex{Herzl, Theodor 2.\,5.\,1860 Budapest – 3.\,7.\,1904 Edlach@\textsc{Herzl, Theodor} (2.\,5.\,1860 Budapest – 3.\,7.\,1904 Edlach), \emph{Schriftsteller, Journalist}!neue Ghetto. Schauspiel in vier Acten@\strich\emph{Das neue Ghetto. Schauspiel in vier Acten}|pw}{}\ledrightnote{\textcolor{green}{Das neue Ghetto. Schauspiel in vier Acten}}«.\pend
           
\pstart
           Ich stelle nur folgende Bedingungen: baldige unveränderte Aufführung noch in dieser
               Spielzeit. An Tantièmen wollen Sie den üblichen {\pb}Satz entrichten. Alle Abmachungen trifft
               mein Vertreter Herr \textcolor{blue}{Fr. Schick}\pwindex{Schik, Friedrich *~6.\,9.\,1857 Wien@\textsc{Schik, Friedrich} (*~6.\,9.\,1857 Wien), \emph{Notar, Journalist, Dramaturg}|pw}{}\ledrightnote{\textcolor{blue}{Friedrich Schik}} in \textcolor{pink}{Wien}\oindex{Wien@\textbf{Wien}, \emph{Verwaltungsgebiet}|pw}{}\ledrightnote{\textcolor{pink}{Wien}}.\pend
           
\pstart
           \centering{}Hochachtungsvoll\pend
           
\pstart
           \raggedleft{}Albert Schnabel\pend
           
\pstart
           Für heute in Eile nur herzliche Grüsse{\\[\baselineskip]} von Ihrem getreuen{\\[\baselineskip]}\spacefill\mbox{Th H.}\pend
           \leftskip=0em{}
\pstart
           19 mai 95\pend
           \selectlanguage{ngerman}\endnumbering\briefempfaengerindex{Schnitzler, Arthur@\textsc{Schnitzler, Arthur}!zzzHerzl, Theodor@\emph{von Theodor Herzl}!1895-05-192@{19. 5. 1895}|)be}\mylabel{L03862h}
\begin{anhang}
\end{anhang}\normalsize

\doendnotes{C}
\bigskip
\vfill

\clearpage

\footnotesize

\lohead{\textsc{register}}

% Definiere theindex-Environment komplett neu ohne reledmac
\makeatletter
\renewenvironment{theindex}{%
  \section*{\indexname}%
  \setlength{\parindent}{0pt}%
  \setlength{\parskip}{0pt plus 0.3pt}%
  \let\item\@idxitem
}{%
  \clearpage
}
\makeatother

\IfFileExists{\jobname-pw.ind}{\input{\jobname-pw.ind}}{}

\end{document}

      