%% latex-korrekturansicht-vorspann.tex
%% Vorspann für die Korrekturansicht.
%% Lädt die gemeinsame Datei latex-vorspann.tex mit gesetztem Schalter.

\newif\ifkorrekturansicht
\korrekturansichttrue

\input{../tex-inputs/latex-vorspann}


               \section[Arthur Schnitzler an Robert Adam, 19. 8. 1918]{ Arthur Schnitzler an Robert Adam, 19. 8. 1918}\nopagebreak\mylabel{v}\rehead{ }\normalsize\beginnumbering\briefempfaengerindex{Adam, Robert@\textsc{Adam, Robert}!zzzSchnitzler, Arthur@\emph{von Arthur Schnitzler}!1918-08-191@{19. 8. 1918}|(be} \toendnotes[C]{\smallbreak\pagebreak[2]} \Standort{DLA, 96.34.2/12.}
\physDesc{Brief, 1 Blatt, 2 Seiten, Umschlag
\newline{}Schreibmaschine
\newline{}Handschrift: schwarze Tinte, deutsche Kurrent (\noindent{}Korrektur und Nachschrift)\newline{}Versand: Stempel: »\nobreak{}Wien, 19. VIII. 18, 3\nobreak{}«.  }\Standort{DLA, A:Schnitzler, 85.1.1621.}
\physDesc{Brief, 2 Blätter, 2 Seiten, Umschlag, maschineller Durchschlag
\newline{}Schreibmaschine
\newline{}Handschrift: Bleistift, lateinische Kurrent (\noindent{}Beschriftung »Adam« und »Kr{[}itik{]}«)}\toendnotes[C]{\smallbreak}\pstart{}{\pb}\textcolor{gray}{\textbf{Dr. Arthur Schnitzler}}\pend{}\pstart{}\textcolor{pink}{\textcolor{gray}{\textbf{Wien, XVIII. Sternwartestrasse 71}}}{}\ledrightnote{\textcolor{pink}{Sternwartestraße}}\pend{}{\bigskip}\pstart{}{\pb}Herrn Robert Adam Pollak\pend{}\pstart{}\textcolor{pink}{\so{Wien XII}}{}\ledrightnote{\textcolor{pink}{XII., Meidling}}.\pend{}\pstart{}\textcolor{pink}{Meidlinger Hauptstrasse 58}{}\ledrightnote{\textcolor{pink}{Meidlinger Hauptstraße}}.\pend{}{\bigskip}\pstart
           {\pb}\textcolor{gray}{\textbf{Dr. Arthur Schnitzler}}\hfill 19. 8. 1918.\pend
           \pstart
           \textcolor{gray}{\textbf{\textcolor{pink}{Wien XVIII. Sternwartestrasse 71}{}\ledrightnote{\textcolor{pink}{Sternwartestraße}}}}\pend
           \pstart\center{}Verehrtester Herr Doktor.\pend\pstart
           Bei der Lektüre Ihres »\textcolor{green}{Yppl}{}\ledrightnote{\textcolor{green}{Yppl. Idylle in fünf Akten}}« habe ich mich recht
               wohlbehagt. Die Milieuschilderung ist hübsch gelungen, vielleicht etwas zu sehr
               biedermeierisch geraten, wenn auch nicht ganz ohne moderne Durchleuchtung. Die
               Charakteristik ist fein, nur der Held kommt, wie das ja so häufig der Fall ist, etwas
               blässlich heraus. Die Chargen sind am besten, besonders der \textcolor{green}{Almeseder}{}\ledrightnote{→\textcolor{green}{Yppl. Idylle in fünf Akten}}, auch der \textcolor{blue}{Hans Sachs}{}\ledrightnote{\textcolor{blue}{Hans Sachs}}\substVorne{}\textsuperscript{sche}\substDazwischen{}hafte\substHinten{} Präsident hat mir ganz wohl gefallen.\pend
           \pstart
           Ob sich die Idylle auf dem Theater würde behaupten können, ist schwer vorher zu
               sagen. Dazu hat sie vielleicht doch nicht Eigenart und Kraft genug. Auch bin ich
               zweifelhaft, ob die Wiederholung der Situation des 2. Aktes im 4. (Probe) glückliche
               Wirkung tun möchte. Immerhin sollten Sie einen Versuch mit dem {\pb}\textcolor{green}{Stück}{}\ledrightnote{→\textcolor{green}{Yppl. Idylle in fünf Akten}} machen und vielleicht
               könnte man eine kleine Bühne – ich meine eine räumlich kleine wie etwa die \textcolor{pink}{Kammerspiele}{}\ledrightnote{\textcolor{pink}{Kammerspiele Wien}} – dafür interessieren. Wenn es Ihnen
               Recht ist, will ich gerne den Regisseur Dr. \textcolor{blue}{Rosenthal}{}\ledrightnote{\textcolor{blue}{Friedrich Rosenthal}} auf Ihr \textcolor{green}{Stück}{}\ledrightnote{→\textcolor{green}{Yppl. Idylle in fünf Akten}}
               aufmerksam machen, das ich Ihnen hiemit mit bestem Danke zurückstelle. Wir reden wohl
               noch ausführlicher darüber. Von Mitte September an stehe ich gerne zur
               Verfügung.\pend
           \pstart
           Herzlichst grüssend{\\[\baselineskip]}Ihr \pend
           \leftskip=0em{}\pstart
           \noindent{}Das \textcolor{green}{Stück}{}\ledrightnote{→\textcolor{green}{Yppl. Idylle in fünf Akten}} liegt Ihrem Wunsch
                  gemäss zum Abholen bei mir bereit.\pend
           \pstart
           \noindent{}{[}hs.:{]} Vielen Dank für das Verzeichnis. Wie viel Mühe haben Sie ſich gemacht –
               ich bin ganz gerührt. Einige der Bücher würden mich ſehr intereſſieren, – beſonders
                  \label{K_L02298_1v}\edtext{\textcolor{blue}{\textsc{Mönckenmüller}}{}\ledrightnote{\textcolor{blue}{Otto Mönkemöller}}}{\lemma{\textnormal{\emph{Mönckenmüller}}}\Cendnote{\textnormal{Vermutlich: \emph{\textcolor{green}{Geistesstörung und Verbrechen im Kindesalter}}
                     von Dr. \textcolor{blue}{Mönkemöller}, Oberarzt an der \textcolor{pink}{Provinzial-Heil- und Pflegeanstalt Osnabrück}.
                     Berlin: \emph{\textcolor{brown}{Verlag von Reuther {\kaufmannsund} Reichard}}{ }1903.}}}\label{K_L02298_1h} u \label{K_L02298_2v}\edtext{\textcolor{blue}{\textsc{Ferrioni}}{}\ledrightnote{\textcolor{blue}{Lino Ferriani}}}{\lemma{\textnormal{\emph{Ferrioni}}}\Cendnote{\textnormal{Vermutlich: \emph{\textcolor{green}{Minderjährige Verbrecher. (Versuch einer
                        strafgerichtlichen Psychologie) mit Original-Gutachten von Berenini – Brusa
                        – Colajanni – Negri – Nordau – Pierantoni}}. Von Cav. \textcolor{blue}{Lino Ferriani}, Staatsanwalt in Como. Deutsch von \textcolor{blue}{Alfred Ruhemann}. Autorisierte Ausgabe.
                     Berlin: \emph{\textcolor{brown}{Siegfried Cronbach}}{ }1896.}}}\label{K_L02298_2h} – dazu nächſtens. \spacefill\mbox{A. S.}\pend
           \endnumbering\briefempfaengerindex{Adam, Robert@\textsc{Adam, Robert}!zzzSchnitzler, Arthur@\emph{von Arthur Schnitzler}!1918-08-191@{19. 8. 1918}|)be}\mylabel{h}  \normalsize

\doendnotes{C}
\bigskip
\vfill

\clearpage

\footnotesize

\lohead{\textsc{register}}

% Definiere theindex-Environment komplett neu ohne reledmac
\makeatletter
\renewenvironment{theindex}{%
  \section*{\indexname}%
  \setlength{\parindent}{0pt}%
  \setlength{\parskip}{0pt plus 0.3pt}%
  \let\item\@idxitem
}{%
  \clearpage
}
\makeatother

\IfFileExists{\jobname-pw.ind}{\input{\jobname-pw.ind}}{}

\end{document}

      