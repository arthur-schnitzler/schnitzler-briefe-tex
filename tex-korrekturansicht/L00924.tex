%% latex-korrekturansicht-vorspann.tex
%% Vorspann für die Korrekturansicht.
%% Lädt die gemeinsame Datei latex-vorspann.tex mit gesetztem Schalter.

\newif\ifkorrekturansicht
\korrekturansichttrue

\input{../tex-inputs/latex-vorspann}


               \section[Georg Brandes an Arthur Schnitzler, 12. 6. 1899]{ Georg Brandes an Arthur Schnitzler, 12. 6. 1899}\nopagebreak\mylabel{v}\rehead{ }\normalsize\beginnumbering\briefempfaengerindex{Schnitzler, Arthur@\textsc{Schnitzler, Arthur}!zzzBrandes, Georg@\emph{von Georg Brandes}!1899-06-121@{12. 6. 1899}|(be} \toendnotes[C]{\smallbreak\pagebreak[2]} \Standort{CUL, Schnitzler, B 17.}
\physDesc{Postkarte
\newline{}Handschrift: blaue Tinte, lateinische Kurrent\newline{}Versand: 1) Stempel: »\nobreak{}\oindex{Kopenhagen@\textbf{Kopenhagen}, \emph{Besiedelter Ort (A.BSO)}|pwk}Kopenhagen, 12. 6. 99, 6–7 E\nobreak{}«.  2) Stempel: »\nobreak{}\oindex{I., Innere Stadt@\textbf{I., Innere Stadt}, \emph{Bezirk (A.BZK)}|pwk}{[}Wien 1/1{]}, 14. 6 {[}99{]}\nobreak{}«. \newline{}Ordnung: mit Bleistift von unbekannter Hand nummeriert: »16« }\buchAbdrucke{\weitereDrucke{Georg Brandes, Arthur Schnitzler: \emph{Ein Briefwechsel}. Hg. Kurt Bergel. Bern: \emph{Francke} 1956, S. 78.} }\toendnotes[C]{\smallbreak}\pstart{}{\pb}Herrn Dr. Arthur
                        Schnitzler\pend{}\pstart{}\textcolor{pink}{Frankgasse 1}{}\ledrightnote{\textcolor{pink}{Frankgasse}}\pend{}\pstart{}\textcolor{pink}{Wien IX}{}\ledrightnote{\textcolor{pink}{Frankgasse}}\pend{}{\bigskip}\pstart
           \raggedleft{}{\pb}Den
                            12 Juni 99\pend
           \pstart
           Verehrter Freund!\hspace*{2.5em}Ich bin willig Alles zu thun
                    was Sie von mir wünschen. Ich bemerke nur, dass ich \textcolor{blue}{Antoine}{}\ledrightnote{\textcolor{blue}{André Antoine}} gar nicht kenne, ihn nicht gesehen habe, nicht ahne, \uline{ob er meinen Namen je gehört hat}. Seien Sie aber
                    nur so freundlich, mir seinen \uline{Vornamen} und seine
                        \uline{Adresse} auf einer \uline{Karte} zu schicken. Dann werde ich ihm mit Vergnügen schreiben, es wird
                    ja nicht meine Schuld sein, falls er von meinem Brief keine Notiz nimmt. Ich las
                    Ihre \textcolor{green}{Stücke}{}\ledrightnote{→\textcolor{green}{Der grüne Kakadu – Paracelsus – Die Gefährtin. Drei Einakter}} mit grossem Vergnügen, habe zwar
                    einige kritische Bedenken, die Sie gelegentlich hören können. Ein \uline{halbes Jahr} habe ich im Bette verbracht; in diesen
                    Tagen aufgestanden. Ihr ergebener \spacefill\mbox{G. B.}\pend
           \endnumbering\briefempfaengerindex{Schnitzler, Arthur@\textsc{Schnitzler, Arthur}!zzzBrandes, Georg@\emph{von Georg Brandes}!1899-06-121@{12. 6. 1899}|)be}\mylabel{h}  \normalsize

\doendnotes{C}
\bigskip
\vfill

\clearpage

\footnotesize

\lohead{\textsc{register}}

% Definiere theindex-Environment komplett neu ohne reledmac
\makeatletter
\renewenvironment{theindex}{%
  \section*{\indexname}%
  \setlength{\parindent}{0pt}%
  \setlength{\parskip}{0pt plus 0.3pt}%
  \let\item\@idxitem
}{%
  \clearpage
}
\makeatother

\IfFileExists{\jobname-pw.ind}{\input{\jobname-pw.ind}}{}

\end{document}

      