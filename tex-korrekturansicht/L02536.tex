%% latex-korrekturansicht-vorspann.tex
%% Vorspann für die Korrekturansicht.
%% Lädt die gemeinsame Datei latex-vorspann.tex mit gesetztem Schalter.

\newif\ifkorrekturansicht
\korrekturansichttrue

\input{../tex-inputs/latex-vorspann}


               \section[Arthur Schnitzler an Robert Adam, 12. 6. 1930]{ Arthur Schnitzler an Robert Adam, 12. 6. 1930}\nopagebreak\mylabel{v}\rehead{ }\normalsize\beginnumbering\briefempfaengerindex{Adam, Robert@\textsc{Adam, Robert}!zzzSchnitzler, Arthur@\emph{von Arthur Schnitzler}!1930-06-121@{12. 6. 1930}|(be} \toendnotes[C]{\smallbreak\pagebreak[2]} \Standort{DLA, 96.34.1/6.}
\physDesc{Brief, 1 Blatt, 1 Seite, Umschlag
\newline{}Handschrift: schwarze Tinte, lateinische Kurrent\newline{}Versand: Stempel: »\nobreak{}Wien, 13. 6. 1930, 11\nobreak{}«.  }\toendnotes[C]{\smallbreak}\pstart{}{\pb}\label{T_L02536-1v}\edtext{\textcolor{gray}{\textbf{A. S.}}}{\lemma{\textnormal{\emph{A. S.}}}\Cendnote{\textnormal{ovaler Absenderkleber}}}\label{T_L02536-1h}\pend{}\pstart{}\textcolor{pink}{\textcolor{gray}{\textbf{WIEN, XVIII.}}}{}\ledrightnote{\textcolor{pink}{XVIII., Währing}}\pend{}\pstart{}\textcolor{pink}{\textcolor{gray}{\textbf{STERNWARTESTR. 71}}}{}\ledrightnote{\textcolor{pink}{Sternwartestraße}}\pend{}{\bigskip}\pstart{}{\pb}Hrn Ob. Landesgerichts
                        Rath\pend{}\pstart{}Dr. Robert Adam Pollak,\pend{}\pstart{}Vicepraesident des \textcolor{brown}{Handelsgerichts}{}\ledrightnote{\textcolor{brown}{Handelsgericht Wien}}\pend{}\pstart{}\textcolor{pink}{Wien XII}{}\ledrightnote{\textcolor{pink}{XII., Meidling}}\pend{}\pstart{}\textcolor{pink}{Meidlinger Hptstr 56}{}\ledrightnote{\textcolor{pink}{Meidlinger Hauptstraße}}.\pend{}{\bigskip}\pstart
           \raggedleft{}{\pb}\textcolor{pink}{Wien}{}\ledrightnote{\textcolor{pink}{Wien}}, 12. 6. 930\pend
           \pstart{}Verehrter Herr Doctor,\pend\pstart
           lassen Sie mich Ihnen zur Erne{\geminationn}ung zum
                    Vicepraesidenten des \textcolor{brown}{Handelsgerichtes}{}\ledrightnote{\textcolor{brown}{Handelsgericht Wien}}
                    herzlichst gratulieren; – zugleich zur Annahme der \textcolor{green}{Margot}{}\ledrightnote{\textcolor{green}{Margot und das Jugendgericht}} in \textcolor{pink}{Frankfurt}{}\ledrightnote{\textcolor{pink}{Frankfurt am Main}}; – ich freue
                    mich, daſs der Erfolg von ein paar Seiten zugleich heranko{\geminationm}t; wenigen wünsch ich so überzeugt ein
                    freundschaftliches Gelingen auf jedem Gebiet wie Ihnen – de{\geminationn} wenige verdienen es wie Sie.\pend
           \pstart
           Ich grüße Sie \textcolor{gray}{schönstens} und hoffe wir sehen und sprechen
                    einander wieder.{\\[\baselineskip]}Ihr{\\[\baselineskip]}\spacefill\mbox{ArthSchnitzler}\pend
           \leftskip=0em{}\endnumbering\briefempfaengerindex{Adam, Robert@\textsc{Adam, Robert}!zzzSchnitzler, Arthur@\emph{von Arthur Schnitzler}!1930-06-121@{12. 6. 1930}|)be}\mylabel{h}  \normalsize

\doendnotes{C}
\bigskip
\vfill

\clearpage

\footnotesize

\lohead{\textsc{register}}

% Definiere theindex-Environment komplett neu ohne reledmac
\makeatletter
\renewenvironment{theindex}{%
  \section*{\indexname}%
  \setlength{\parindent}{0pt}%
  \setlength{\parskip}{0pt plus 0.3pt}%
  \let\item\@idxitem
}{%
  \clearpage
}
\makeatother

\IfFileExists{\jobname-pw.ind}{\input{\jobname-pw.ind}}{}

\end{document}

      