%% latex-korrekturansicht-vorspann.tex
%% Vorspann für die Korrekturansicht.
%% Lädt die gemeinsame Datei latex-vorspann.tex mit gesetztem Schalter.

\newif\ifkorrekturansicht
\korrekturansichttrue

\input{../tex-inputs/latex-vorspann}


               \section[Hermann Bahr an Arthur Schnitzler, 16. 4. 1913]{ Hermann Bahr an Arthur Schnitzler, 16. 4. 1913}\nopagebreak\mylabel{v}\rehead{ }\normalsize\beginnumbering\briefempfaengerindex{Schnitzler, Arthur@\textsc{Schnitzler, Arthur}!zzzBahr, Hermann@\emph{von Hermann Bahr}!1913-04-161@{16. 4. 1913}|(be} \toendnotes[C]{\smallbreak\pagebreak[2]} \Standort{CUL, Schnitzler, B 5b.}
\physDesc{Kartenbrief
\newline{}Handschrift: schwarze Tinte, deutsche Kurrent\newline{}Versand: Stempel: »\nobreak{}\oindex{Salzburg@\textbf{Salzburg}, \emph{Besiedelter Ort (A.BSO)}|pwk}Sa{[}lzburg{]}, 16. IV. 13, 10\nobreak{}«.  
\newline{}Schnitzler: mit Bleistift ergänzt »Bahr« \newline{}Ordnung: mit Bleistift von unbekannter Hand nummeriert:
                              »176« }\buchAbdrucke{\weitereDrucke{Hermann Bahr, Arthur Schnitzler: \emph{Briefwechsel, Aufzeichnungen, Dokumente (1891–1931)}. Hg. Kurt Ifkovits und Martin Anton Müller. Göttingen: \emph{Wallstein} 2018, S. 482.} }\toendnotes[C]{\smallbreak}\pstart{}{\pb}Abſ. \textsc{Hermann Bahr}\pend{}\pstart{}\textsc{\textcolor{pink}{Salzburg}{}\ledrightnote{\textcolor{pink}{Salzburg}}}\pend{}{\bigskip}\pstart{}
                  Herrn \textsc{D\textsuperscript{r} Arthur
                  Schnitzler}\pend{}\pstart{}\textsc{\textcolor{pink}{Wien XVIII}{}\ledrightnote{\textcolor{pink}{XVIII., Währing}}}\pend{}\pstart{}\textcolor{pink}{Sternwarteſtraße 71}{}\ledrightnote{\textcolor{pink}{Sternwartestraße}}\pend{}{\bigskip}\pstart
           \raggedleft{}{\pb}\textcolor{pink}{Salzburg}{}\ledrightnote{\textcolor{pink}{Salzburg}}{ }16. 4. 13\pend
           \pstart
           Lieber Arthur! Ich erhielt eben einen etwas verworrenen Brief \textcolor{blue}{Peter Altenbergs}{}\ledrightnote{\textcolor{blue}{Peter Altenberg}}, worin er mich anfleht, ihn zu
               retten, der im \textcolor{pink}{Steinhof}{}\ledrightnote{\textcolor{pink}{Otto-Wagner-Spital}} »wie ein giftiges
               irrſinniges Tier« behandelt und zu Tod gequält werde. Es iſt möglich, daß das
               »Einbildungen« ſind. Es iſt ebenſo möglich, daß es wahr iſt. Ich weiß gar nicht, was
               ich von hier aus tun soll, und weiß auch nicht, wie ich mir, in \textcolor{pink}{Wien}{}\ledrightnote{\textcolor{pink}{Wien}} angekommen, den Eintritt im \textcolor{pink}{Steinhof}{}\ledrightnote{\textcolor{pink}{Otto-Wagner-Spital}} erzwingen könnte. Du biſt »Arzt«, Du wirſt eher wiſſen, ob und wie
               man helfen könnte. Willſt Du Dich der Sache annehmen? Und mir dann ſagen, ob Du
               glaubſt, daß ich was tun kann? Ich bin natürlich gern zu allem bereit – Mordsſkandal
               in der Öffentlichkeit oder auch gewaltſame Entführung, die ja mit Geld dort leicht zu
               bewerkſtelligen ſein wird. Bitte ſchreib bald\pend
           \pstart
           Deinem alten{\\[\baselineskip]}\spacefill\mbox{Hermann}\pend
           \leftskip=0em{}\pstart
           \noindent{}Grüße an \textcolor{blue}{Olga}{}\ledrightnote{\textcolor{blue}{Olga Schnitzler}} u die \textcolor{blue}{Kinder}{}\ledrightnote{→\textcolor{blue}{Heinrich Schnitzler}{\newline}→\textcolor{blue}{Lili Schnitzler}}!\pend
           \endnumbering\briefempfaengerindex{Schnitzler, Arthur@\textsc{Schnitzler, Arthur}!zzzBahr, Hermann@\emph{von Hermann Bahr}!1913-04-161@{16. 4. 1913}|)be}\mylabel{h}  \normalsize

\doendnotes{C}
\bigskip
\vfill

\clearpage

\footnotesize

\lohead{\textsc{register}}

% Definiere theindex-Environment komplett neu ohne reledmac
\makeatletter
\renewenvironment{theindex}{%
  \section*{\indexname}%
  \setlength{\parindent}{0pt}%
  \setlength{\parskip}{0pt plus 0.3pt}%
  \let\item\@idxitem
}{%
  \clearpage
}
\makeatother

\IfFileExists{\jobname-pw.ind}{\input{\jobname-pw.ind}}{}

\end{document}

      