%% latex-korrekturansicht-vorspann.tex
%% Vorspann für die Korrekturansicht.
%% Lädt die gemeinsame Datei latex-vorspann.tex mit gesetztem Schalter.

\newif\ifkorrekturansicht
\korrekturansichttrue

\input{../tex-inputs/latex-vorspann}


\renewcommand{\erwaehntePersonen}{Personen: Richard Beer-Hofmann, Samuel Fischer, Anna Katharina Rehmann, Felix Salten, Ottilie Salten, Paul Salten, Olga Schnitzler, Heinrich Schnitzler, Louise Schnitzler}
\renewcommand{\erwaehnteInstitutionen}{Institutionen: Franz-Grillparzer-Preis}
\renewcommand{\erwaehnteOrte}{Orte: Semmering, Wien}
\renewcommand{\erwaehnteWerke}{Werke: Der Weg ins Freie. Roman}
\section[ Arthur Schnitzler an Felix Salten, 25. 1. 1908]{Arthur Schnitzler an Felix Salten, 25. 1. 1908}
\nopagebreak\mylabel{v}
\rehead{ }\normalsize\beginnumbering\briefempfaengerindex{Salten, Felix@\textsc{Salten, Felix}!zzzSchnitzler, Arthur@\emph{von Arthur Schnitzler}!1908-01-252@{25. 1. 1908}|(be}
\toendnotes[C]{\smallbreak\pagebreak[2]}\Standort{Wienbibliothek im Rathaus, ZPH 1681, 2.1.516.}
\physDesc{Brief, 1 Blatt, 4 Seiten, 2045 Zeichen
\newline{}Handschrift: blaue Tinte, lateinische Kurrent
\newline{}Ordnung: mit Bleistift von unbekannter Hand nummeriert: »6« }\toendnotes[C]{\smallbreak}
\pstart
           \raggedleft{}{\pb}25. 1. 908\pend
           
\pstart
           lieber, es ist \label{K_L03011-1v}\edtext{desinfizirt}{\lemma{\textnormal{\emph{desinfizirt}}}\Cendnote{\textnormal{siehe Felix Salten an Arthur Schnitzler, [10. 12. 1907]}}}\label{K_L03011-1h}, Wohnung, Kleider, \textcolor{blue}{Olga}{}\ledrightnote{\textcolor{blue}{Olga Schnitzler}} ist meist außer
               Bett, also die Zustände sind annähernd zur Norm zurückgekehrt. Der \label{K_L03011-2v}\edtext{\textcolor{blue}{Bub}{}\ledrightnote{{$\rightarrow$}\textcolor{blue}{Heinrich Schnitzler}} ist noch nicht
                  daheim}{\lemma{\textnormal{\emph{Bub … daheim}}}\Cendnote{\textnormal{\textcolor{blue}{Heinrich} war während der Erkrankung seiner
                     \textcolor{blue}{Mutter} bei seiner \textcolor{blue}{Großmutter}
                  väterlicherseits.}}}\label{K_L03011-2h}, doch hab ich mit ihm Zusa{\geminationm}enkünfte, auch macht er uns \label{K_L03011-3v}\edtext{Fensterpromenaden}{\lemma{\textnormal{\emph{Fensterpromenaden}}}\Cendnote{\textnormal{Heißt: Er spaziert
                  am Fenster vorbei und winkt seiner \textcolor{blue}{Mutter}, die weiterhin in Quarantäne ist.}}}\label{K_L03011-3h}. Wir wollen
               in etwa 10 Tagen, bis \textcolor{blue}{Olga}{}\ledrightnote{\textcolor{blue}{Olga Schnitzler}} ganz gehtüchtig und
               die Influenzagerüchte – oder -wahrheiten vom \textcolor{pink}{Semmering}{}\ledrightnote{\textcolor{pink}{Semmering}} geschwunden sind, \label{K_L03011-4v}\edtext{auf besagten \textcolor{pink}{Südbahngipfel}{}\ledrightnote{{$\rightarrow$}\textcolor{pink}{Semmering}}
                  reisen}{\lemma{\textnormal{\emph{auf … reisen}}}\Cendnote{\textnormal{\textcolor{blue}{Arthur} und \textcolor{blue}{Olga Schnitzler} reisten am 4. 2. 1908 auf den \textcolor{pink}{Semmering} und trafen dabei im Zug auf \textcolor{blue}{Salten}. Am 22. 2. 1908 reisten sie zurück nach \textcolor{pink}{Wien}.}}}\label{K_L03011-4h} und dort mit \textcolor{blue}{Heini}{}\ledrightnote{\textcolor{blue}{Heinrich Schnitzler}} etwa 8 Tage verbringen. Dies unser Progra{\geminationm}. Da{\geminationn} erst gedenk ich
               Freundes- und andre Häuser wieder zu betreten und das {\pb}unsre zu eröffnen.\pend
           
\pstart
           Trotzdem möcht ich Sie gerne sehen{[},{]} früher sehen; – we{\geminationn} Sie nicht (was ich Ihnen beim Hi{\geminationm}el keinen Moment lang verübeln kö{\geminationn}te!) zu ängstlich sind. Jedenfalls schreiben Sie mir
               zum Trost\textcolor{gray}{,} wie es Ihnen Allen geht; von \textcolor{blue}{Richard}{}\ledrightnote{\textcolor{blue}{Richard Beer-Hofmann}} hört ich, dass Sie sich noch i{\geminationm}er nicht ganz wohl befinden.\pend
           
\pstart
           Hinsichtlich des Vorausdrucks des \textcolor{green}{Roman}{}\ledrightnote{{$\rightarrow$}\textcolor{green}{Der Weg ins Freie. Roman}}s hab ich mit \textcolor{blue}{Fischer}{}\ledrightnote{\textcolor{blue}{Samuel Fischer}} schon vor
               Monaten correspondirt; aus \textcolor{gray}{irgend}welchen techn. Gründen läßt sich
               die Sache nicht machen. Ich habe in den letzten Wochen noch viel daran corrigirt, so
               daß die Manuscripte immer ungastlicher aussehen, überdies werden Sie lieber kein {\pb}Papierconvolut aus unsrer Wohnung in Ihre
               hinübernehmen wollen – was bleibt mir also übrig? Sie bitten, das \textcolor{green}{Ding}{}\ledrightnote{{$\rightarrow$}\textcolor{green}{Der Weg ins Freie. Roman}} nicht in Forsetzungen zu lesen,
               sondern warten, bis das \textcolor{green}{Buch}{}\ledrightnote{{$\rightarrow$}\textcolor{green}{Der Weg ins Freie. Roman}}
               da ist, um es, womöglich an einem – zwei schönen Sommertagen in \uline{einem} Zug (eventuell auch in einem \uline{Zug},
               aber besser, im Freien) hinunterzuschlucken. Der Nachgeschmack wird kein übler sein;
               heut trau ich mich es zu sagen.–\pend
           
\pstart
           Ich danke Ihnen sehr für Ihre lieben \textcolor{brown}{Grillparzerpreis}{}\ledrightnote{\textcolor{brown}{Franz-Grillparzer-Preis}}glückwünsche. Anfangs war ich sehr erstaunt, da{\geminationn} eher (aus allerlei, complicirten und oberflächlichen
               Gründen) heruntergesti{\geminationm}t – jetzt überwiegt die Freude,
               woran die {\pb}\label{K_L03011-5v}\edtext{5 Mille}{\lemma{\textnormal{\emph{5 Mille}}}\Cendnote{\textnormal{5000 Kronen im Jahr 1908 entsprechen
                     2023 etwa 38.000 Euro.}}}\label{K_L03011-5h} nicht ganz unbetheiligt
               sind. Nach dem Arbeiten sehn ich mich, hab manches vorbereitet und \substVorne{}\textsuperscript{\textcolor{gray}{au}}\substDazwischen{}bin\substHinten{} neugierig, was zuerst fertig sein wird. So stellt man sich frech wieder
               mitten ins Leben hinein.\pend
           
\pstart
           Seien Sie, \textcolor{blue}{Otti}{}\ledrightnote{\textcolor{blue}{Ottilie Salten}} und die \textcolor{blue}{Kinder}{}\ledrightnote{{$\rightarrow$}\textcolor{blue}{Paul Salten}{\newline}{$\rightarrow$}\textcolor{blue}{Anna Katharina Rehmann}} herzlichst gegrüßt und
               lassen mindestens was von sich \uline{hören}. Auch von \textcolor{blue}{Olga}{}\ledrightnote{\textcolor{blue}{Olga Schnitzler}} alles schöne.\pend
           
\pstart
           Ihr {\\[\baselineskip]}\spacefill\mbox{Arthur}\pend
           \leftskip=0em{}\endnumbering\briefempfaengerindex{Salten, Felix@\textsc{Salten, Felix}!zzzSchnitzler, Arthur@\emph{von Arthur Schnitzler}!1908-01-252@{25. 1. 1908}|)be}\mylabel{h}  \normalsize

\doendnotes{C}
\bigskip
\vfill

\clearpage

\footnotesize

\lohead{\textsc{register}}

% Definiere theindex-Environment komplett neu ohne reledmac
\makeatletter
\renewenvironment{theindex}{%
  \section*{\indexname}%
  \setlength{\parindent}{0pt}%
  \setlength{\parskip}{0pt plus 0.3pt}%
  \let\item\@idxitem
}{%
  \clearpage
}
\makeatother

\IfFileExists{\jobname-pw.ind}{\input{\jobname-pw.ind}}{}

\end{document}

      