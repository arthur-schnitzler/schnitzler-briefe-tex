%% latex-korrekturansicht-vorspann.tex
%% Vorspann für die Korrekturansicht.
%% Lädt die gemeinsame Datei latex-vorspann.tex mit gesetztem Schalter.

\newif\ifkorrekturansicht
\korrekturansichttrue

\input{../tex-inputs/latex-vorspann}


\renewcommand{\erwaehntePersonen}{Personen: Moriz Benedikt, Olga Schnitzler}
\renewcommand{\erwaehnteInstitutionen}{Institutionen: Neue Freie Presse}
\renewcommand{\erwaehnteOrte}{Orte: Berlin, Dessauer Straße, Deutschland, Salzburg, Österreich}
\renewcommand{\erwaehnteWerke}{Werke: Leutnant Gustl. Parodie, Lieutenant Gustl. Novelle, Neue Freie Presse, Wien, 20. Juni}
\section[ Paul Goldmann an Arthur Schnitzler, 21. 6. {[}1901{]}]{Paul Goldmann an Arthur Schnitzler, 21. 6. {[}1901{]}}
\nopagebreak\mylabel{v}
\rehead{ }\normalsize\beginnumbering\briefempfaengerindex{Schnitzler, Arthur@\textsc{Schnitzler, Arthur}!zzzGoldmann, Paul@\emph{von Paul Goldmann}!1901-06-211@{21. 6. {[}1901{]}}|(be}
\toendnotes[C]{\smallbreak\pagebreak[2]}\Standort{DLA, A:Schnitzler, HS.NZ85.1.3171.}
\physDesc{Brief, 1 Blatt, 4 Seiten
\newline{}Handschrift: blaue Tinte, deutsche Kurrent
\newline{}Schnitzler: mit Bleistift das Jahr »{[}1{]}901« vermerkt }\toendnotes[C]{\smallbreak}
\pstart
           \noindent{}\raggedleft{}{\pb}\textcolor{pink}{\textcolor{gray}{\textbf{DESSAUERSTRASSE 19}}}{}\ledrightnote{\textcolor{pink}{Dessauer Straße}}\pend
           
\pstart
           \textcolor{pink}{Berlin}{}\ledrightnote{\textcolor{pink}{Berlin}}, 21. Juni.\pend
           
\pstart\center{}Mein lieber Freund,\pend
\pstart
           Wir haben heut hier telegraphiſch die Kunde erhalten,
               daß Du \label{K_L03070-1v}\edtext{aus dem Offizierſtande
                  geſtrichen}{\lemma{\textnormal{\emph{aus … geſtrichen}}}\Cendnote{\textnormal{Für die Veröffentlichung
                  von \emph{\textcolor{green}{Lieutenant Gustl}} wurde \textcolor{blue}{Schnitzler} am 21. 6. 1901 der Offiziersrang aberkannt.}}}\label{K_L03070-1h}
               biſt. \strikeout{Es iſt} Ich weiß, es wird Dir ſchrecklich ſein,
               daß Du künftig den bewaffneten Schaaren nicht als Heerführer voranziehen ſollſt, aber
               Du wirſt das Unglück zu tragen wiſſen. Die \label{K_L03070-2v}\edtext{Begründung}{\lemma{\textnormal{\emph{Begründung}}}\Cendnote{\textnormal{Siehe
                  etwa den \textcolor{green}{Leitartikel} der
                     \emph{\textcolor{green}{Neuen Freien Presse}} zum Thema:
                     [O. V. = \textcolor{blue}{Moriz Benedikt}]: \emph{\textcolor{green}{Wien, 20. Juni}}. In: \emph{\textcolor{green}{Neue Freie Presse}}, Nr. 13.226, 21. 6. 1901, Morgenblatt, S. 1–2.}}}\label{K_L03070-2h} des ehrenräthlichen
               Erkenntniſſes iſt perfid und verräth gute jeſuitiſche Schulung. Wenn Du noch eines
               Mittels bedurft {\pb}hätteſt, um in ganz \textcolor{pink}{Deutſchland}{}\ledrightnote{\textcolor{pink}{Deutschland}} und \textcolor{pink}{Öſterreich}{}\ledrightnote{\textcolor{pink}{Österreich}} Sympathien zu gewinnen, ſo wäre dieſer Streich jedenfalls das
               beſte Mittel dieſer Art. Immerhin werden die Sympathien, die \substVorne{}\textsuperscript{man}\substDazwischen{}man\substHinten{} für Dich hegt, überall an Herzlichkeit zunehmen, und die Herren vom
               Ehrenrathe haben durch ihr Verdikt für Deine Perſon und Deine Werke eine ſehr
               löbliche Propaganda gemacht. Da ſie aber das Gegentheil beabſichtigt haben, {\pb}ſo wirſt Du hoffentlich die Antwort \strikeout{\textcolor{gray}{n}} nicht ſchuldig bleiben. Eine kräftige und doch vornehme Abſage an den \strikeout{\textcolor{gray}{ge}} Ehrenrath und an Militarismus überhaupt wäre wohl angemeſſen, und die »\textcolor{brown}{Neue Freie Preſſe}{}\ledrightnote{\textcolor{brown}{Neue Freie Presse}}« könnte einer ſolchen \label{K_L03070-4v}\edtext{Antwort}{\lemma{\textnormal{\emph{Antwort}}}\Cendnote{\textnormal{Eine solche Antwort gab es nie. \textcolor{blue}{Schnitzler} verfasste jedoch eine fünfseitige, zu Lebzeiten
                  nicht veröffentlichte \textcolor{green}{Parodie} auf seine \textcolor{green}{Novelle}, betitelt \emph{\textcolor{green}{Leutnant Gustl}}.
                  Darin wird \textcolor{green}{Gustl}
                  übertrieben sittlich-korrekt dargestellt und die antisemitisch geprägte
                  Berichterstattung humorvoll thematisiert.}}}\label{K_L03070-4h} aus Deiner Feder die Aufnahme
               kaum verweigern.\pend
           
\pstart
           Ich drücke Dir herzlichſt die \strikeout{\textcolor{gray}{H}} Hand und grüße Dich in Treue, – obwohl ich es für meinen Theil lebhaft
               bedaure, {\pb}nicht mehr einen k. u. k. Regimentsrat,
               ſondern einen ganz gemeinen Reſerviſten als Freund zu beſitzen. {\\[\baselineskip]}Dein {\\[\baselineskip]}\spacefill\mbox{Paul Goldmann.}\pend
           \leftskip=0em{}
\pstart
           \noindent{}Herzlichen Gruß an Fräulein \textsc{\textcolor{blue}{Olga}{}\ledrightnote{\textcolor{blue}{Olga Schnitzler}}}!\pend
           \endnumbering\briefempfaengerindex{Schnitzler, Arthur@\textsc{Schnitzler, Arthur}!zzzGoldmann, Paul@\emph{von Paul Goldmann}!1901-06-211@{21. 6. {[}1901{]}}|)be}\mylabel{h}
\begin{anhang}
\end{anhang}\normalsize

\doendnotes{C}
\bigskip
\vfill

\clearpage

\footnotesize

\lohead{\textsc{register}}

% Definiere theindex-Environment komplett neu ohne reledmac
\makeatletter
\renewenvironment{theindex}{%
  \section*{\indexname}%
  \setlength{\parindent}{0pt}%
  \setlength{\parskip}{0pt plus 0.3pt}%
  \let\item\@idxitem
}{%
  \clearpage
}
\makeatother

\IfFileExists{\jobname-pw.ind}{\input{\jobname-pw.ind}}{}

\end{document}

      