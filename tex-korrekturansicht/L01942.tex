%% latex-korrekturansicht-vorspann.tex
%% Vorspann für die Korrekturansicht.
%% Lädt die gemeinsame Datei latex-vorspann.tex mit gesetztem Schalter.

\newif\ifkorrekturansicht
\korrekturansichttrue

\input{../tex-inputs/latex-vorspann}


               \section[Richard Beer-Hofmann an Arthur Schnitzler, 7. 7. 1910]{ Richard Beer-Hofmann an Arthur Schnitzler, 7. 7. 1910}\nopagebreak\mylabel{v}\rehead{ }\normalsize\beginnumbering\briefempfaengerindex{Schnitzler, Arthur@\textsc{Schnitzler, Arthur}!zzzBeer-Hofmann, Richard@\emph{von Richard Beer-Hofmann}!1910-07-071@{7. 7. 1910}|(be} \toendnotes[C]{\smallbreak\pagebreak[2]} \Standort{CUL, Schnitzler, B 8.}
\physDesc{Brief, 2 Blätter, 5 Seiten
\newline{}Handschrift: schwarze Tinte, lateinische Kurrent\newline{}Beilage: maschinschriftliche Abschrift, undatiert und von unbekannter
                                 Hand mit »278c« nummeriert. Die Zuordnung ist eine
                                 Mutmaßung, basierend auf der Erwähnung im Text. Die Abschrift ist
                                 um die letzte Strophe gekürzt \newline{}Ordnung: mit Bleistift von unbekannter Hand nummeriert:
                                    »234« }\buchAbdrucke{\weitereDrucke{Arthur Schnitzler, Richard Beer-Hofmann: \emph{Briefwechsel 1891–1931}. Hg. Konstanze Fliedl. Wien, Zürich: \emph{Europaverlag} 1992, S. 209–210.} }\toendnotes[C]{\smallbreak}\pstart
           \raggedleft{}{\pb}\textcolor{pink}{Ischl}{}\ledrightnote{\textcolor{pink}{Bad Ischl}}{ }7./VII. 10\pend
           \pstart{}Lieber Arthur!\pend\pstart
           Ihr \label{K_L01942-1v}\edtext{Brief}{\lemma{\textnormal{\emph{Brief}}}\Cendnote{\textnormal{siehe Arthur Schnitzler an Richard Beer-Hofmann, 29. 6. 1910}}}\label{K_L01942-1h} ist mit seiner neuen Adressirung gestern angelangt. Nun weiss es der
               Briefträger – glaube ich – auch schon wo »\textcolor{pink}{Steinfeld 6}{}\ledrightnote{\textcolor{pink}{Steinfeld}}« ist.\pend
           \pstart
           Hier, die gewünschte \textcolor{green}{Abschrift}{}\ledrightnote{→\textcolor{green}{Schlaflied für Mirjam}},
               des »\label{K_L01942_1v}\edtext{\textcolor{green}{einen schönen Verses}{}\ledrightnote{→\textcolor{green}{Die Welt in der man sich langweilt}}}{\lemma{\textnormal{\emph{einen schönen Verses}}}\Cendnote{\textnormal{In \emph{\textcolor{green}{Die
                     Welt, in der man sich langweilt}} wird von einem Dichter eine Tragödie
                  vorgetragen, die »einen schönen Vers« enthält.}}}\label{K_L01942_1h}« aus der »\textcolor{green}{Welt i. d. m. sich langweilt}{}\ledrightnote{\textcolor{green}{Die Welt in der man sich langweilt}}«. Bitte, lassen Sie die
               Verse vielleicht von Fräulein \textcolor{blue}{Pollak}{}\ledrightnote{\textcolor{blue}{Frieda Pollak}} abtypiren,
               und schicken Sie mir die Abschrift zurück; ich brauche sie, um sie dem \textcolor{blue}{Übersetzer}{}\ledrightnote{→\textcolor{blue}{?? [Übersetzer Deutsch zu Ungarisch]}} ins \textcolor{pink}{Ungarische}{}\ledrightnote{\textcolor{pink}{Ungarn}} (dem ich sie vor Monaten versprach) zu schicken.\pend
           \pstart
           »\textcolor{green}{Das weite Land}{}\ledrightnote{\textcolor{green}{Das weite Land. Tragikomödie in fünf Akten}}« habe ich auf der Fahrt mit vieler
               Freude gelesen. Es hat den denkbar schlanksten Aufbau, und das bewusste
               Nichtverkleiden des Constructiven wirkt am Ende – wo einem die Führung klar wird –
               wie ein neuer Reiz. Sie haben, – glaube ich – bisher noch nie so straff die Zügel
               aller Ihrer {\pb}Figuren gehalten, und
               man empfindet alles, was feinfühlige Kritiker »Beiwerk« nennen als woltuend, um
               scharf gespanntes ein wenig zu lockern. Schon im »\textcolor{green}{Medardus}{}\ledrightnote{\textcolor{green}{Der junge Medardus. Dramatische Historie in einem Vorspiel und fünf Aufzügen}}« schien mir die Richtung erkennbar, nach der Sie \label{T_L01942-1v}\edtext{sich}{\lemma{\textnormal{\emph{sich}}}\Cendnote{\textnormal{Er schreibt »Sich«.}}}\label{T_L01942-1h} nun wenden.\pend
           \pstart
           Es scheint, als genügte es Ihnen nicht, und wäre nicht in Ihren Absichten gelegen,
               die stärksten Wirkungen von den einzelnen Menschen Ihres \textcolor{green}{Stückes}{}\ledrightnote{→\textcolor{green}{Das weite Land. Tragikomödie in fünf Akten}}, und ihren Schicksalen ausgehen zu lassen, sondern
               als strebten Sie, bewusst, dahin, Einzelschicksale derart miteinander zu verknoten,
               dass jedes Theilschicksal nur ein sich unterordnender Zug, eine Runzel, ein Grinsen,
               ein Blick einer einzigen Schicksalsmaske würde, deren Ausdruck, am Ende des \textcolor{green}{Stückes}{}\ledrightnote{→\textcolor{green}{Das weite Land. Tragikomödie in fünf Akten}},
               das wäre, was einem als Wesentlichstes haften bliebe. {\pb}Sie können freilich sagen, – dahin
               gienge endlich alles dramatische Gestalten. Nur, scheint mir jetzt – ich möchte sagen
               – die Art Ihres Vortrags, Ihre Stimme zärtlicher und liebender zu sein, wenn es um
               Verschlingungen von Schicksalen geht, als um das Fühlen der Einzelnen.\pend
           \pstart
           Übrigens ist die Figur des »\textcolor{green}{Hofreiter}{}\ledrightnote{→\textcolor{green}{Das weite Land. Tragikomödie in fünf Akten}}« so stark herausgekommen, dass es mir wie Kindern geht, denen die
               Bösewichte des Stückes nie genug geprügelt werden. Den »\textcolor{green}{Hofreiter}{}\ledrightnote{→\textcolor{green}{Das weite Land. Tragikomödie in fünf Akten}}« an den Sie \label{K_L01942-4v}\edtext{\textcolor{blue}{dachten}{}\ledrightnote{→\textcolor{blue}{Louis Philipp Friedmann}}}{\lemma{\textnormal{\emph{dachten}}}\Cendnote{\textnormal{Vorlage war \textcolor{blue}{Louis Friedmann}, einen Fabrikanten, den er bereits im
                  Gymnasium kennengelernt hatte, vgl. A. S.: \emph{Tagebuch}, 21. 8. 1908}}}\label{K_L01942-4h}, kenne ich nur sehr oberflächlich aber dieser »Charmeur« war mir in seiner
               halbfrechen, halb \label{K_L01942-3v}\edtext{minaudirenden}{\lemma{\textnormal{\emph{minaudirenden}}}\Cendnote{\textnormal{französisch: affektiert, manieriert}}}\label{K_L01942-3h}
               Koketterie immer unerträglich. Alles was er sagte und tat, war ein Versuch einen zu
               beschwätzen, oder zu brutalisieren. Ich glaube i{\geminationm}er die
               Art wie er seine Liebe an die Frau bringt, muss ein Mittelding \substVorne{}\textsuperscript{von}\substDazwischen{}zwischen\substHinten{} der Energie eines {\pb}Handlungsreisenden und der eines Erpressers \strikeout{haben.}
               sein. Für Menschen dieses Schlages wäre eine Hölle leicht zu erfinden: Der Ort, wo
               Alles, um seiner selbst willen gesagt und getan wird, und wo nichts sich spiegeln
               kann. Ich begreife, dass Frauen die Existenz von \textcolor{green}{Hofreiters}{}\ledrightnote{→\textcolor{green}{Das weite Land. Tragikomödie in fünf Akten}} als eine einzige grossartige Reverenz vor ihrer
               Sexualität empfinden, aber ich verarge – Ihnen, lieber Arthur – sehr, dass Frau \textcolor{green}{Genia}{}\ledrightnote{→\textcolor{green}{Das weite Land. Tragikomödie in fünf Akten}} ihn liebt. Ich glaube immer,
                  \strikeout{a} Sie haben, aus gemeinsamer Jugend her, noch mehr
               Sympathie für Herrn \textcolor{green}{\textcolor{blue}{Fried}{}\ledrightnote{→\textcolor{blue}{Louis Philipp Friedmann}}– – –rich
                  Hofreiter}{}\ledrightnote{→\textcolor{green}{Das weite Land. Tragikomödie in fünf Akten}} als er verdient. Wenn schon – dann ziehe ich die \textcolor{green}{Aigner}{}\ledrightnote{→\textcolor{green}{Das weite Land. Tragikomödie in fünf Akten}}s vor. Bei denen ist es animalischer,
               mehr um der Sache selbst willen, und, wie Alles Sachliche, zuletzt, nicht
               hässlich.\pend
           \pstart
           Übrigens ist das »\textcolor{green}{Und man {\pb}kann doch nicht
               Jeden – – –}{}\ledrightnote{→\textcolor{green}{Das weite Land. Tragikomödie in fünf Akten}}« \textcolor{green}{Hofreiters}{}\ledrightnote{→\textcolor{green}{Das weite Land. Tragikomödie in fünf Akten}}, in
               der letzten Scene, prachtvoll. Hier wirkt er doch grösser, und hat ein anderes
               Gesicht als die kleinlich verknitterten Züge einer lüsternen Maus (über die, von den
               klein sich kräuselnden Haaren, ein Schatten Judenthums fällt) – an die mich das
               Original immer erinnerte.\pend
           \pstart
           Missrathenes Halbblut, das einen – nicht mich – nachdenklich machen könnte!\pend
           \pstart
           Eine einzige Stelle im Stück würde ich gerne vermissen: Ende des \textcolor{green}{III. Aktes}{}\ledrightnote{→\textcolor{green}{Das weite Land. Tragikomödie in fünf Akten}}. Die Worte \textcolor{green}{Ernas}{}\ledrightnote{→\textcolor{green}{Das weite Land. Tragikomödie in fünf Akten}}: »\textcolor{green}{Und ich ahne, es giebt noch schönre Stunden, als die dort oben
                  war auf dem Aignerturm.}{}\ledrightnote{→\textcolor{green}{Das weite Land. Tragikomödie in fünf Akten}}«\pend
           \pstart
           Hier – noch dazu in Association mit der \label{K_L01942-5v}\edtext{Table d’hôte}{\lemma{\textnormal{\emph{Table d’hôte}}}\Cendnote{\textnormal{französisch, wörtlich:
                  Tisch des Gastgebers; gemeint ist eine Menüfolge, bei der die Speisen vorgegeben
                  sind.}}}\label{K_L01942-5h} – wirkt das nicht wie ruhige Offenheit, sondern es wird daraus ein
               komisch-pedantisches, sich an den Tisch der Liebe setzen, und auf den letzten Gang
               freuen.\pend
           \pstart
           {\pb}Als ich hier ankam, und vor dem
                  »\textcolor{pink}{Hôtel Post}{}\ledrightnote{\textcolor{pink}{Hotel Post}}« auf mein Gepäck wartete, war
                  \label{K_L01942_2v}\edtext{Ihr »\textcolor{blue}{\textcolor{green}{Gustl Wahl}{}\ledrightnote{→\textcolor{green}{Das weite Land. Tragikomödie in fünf Akten}}}{}\ledrightnote{→\textcolor{blue}{Friedrich Eckstein}}«}{\lemma{\textnormal{\emph{Ihr »Gustl Wahl«}}}\Cendnote{\textnormal{\textcolor{blue}{Friedrich Eckstein} ist in der Ischler Kurliste
                  vom 6. 7. 1910 als im \textcolor{pink}{Hôtel Post}
                  wohnhaft gelistet.}}}\label{K_L01942_2h} das erste bekannte Gesicht, das ich sah. Er wird meine
               grosse Heiterkeit bei seinem Anblick nicht verstanden haben.\pend
           \pstart
           Lieber Arthur: Ich danke Ihnen für die schöne Nachmittagsvorstellung die Sie mir
               verschafften, bin sicher, dass Sie noch sehr viel Freude an Ihrer \textcolor{green}{Tragikomödie}{}\ledrightnote{→\textcolor{green}{Das weite Land. Tragikomödie in fünf Akten}} haben werden, habe Ihnen noch
               eine ganze Menge darüber zu sagen: (hoffentlich ko{\geminationm}en
               Sie bald hieher) – und grüsse – mit \textcolor{blue}{Paula}{}\ledrightnote{\textcolor{blue}{Paula Beer-Hofmann}} zusa{\geminationm}en – Sie und Ihre \textcolor{blue}{Frau}{}\ledrightnote{→\textcolor{blue}{Olga Schnitzler}} herzlichst\pend
           \pstart
           Ihr{\\[\baselineskip]}\spacefill\mbox{Richard}\pend
           \leftskip=0em{}{\bigskip}\pstart
           \noindent{}\centering{}{\pb}\uline{Schlaflied für \textcolor{blue}{Mirjam}{}\ledrightnote{\textcolor{blue}{Mirjam Beer-Hofmann}}.}\pend
           {\bigskip}\stanza{}Schlaf mein Kind – schlaf, es ist spät!\newverse{}Sieh, wie die Sonne zur Ruhe dort geht,\newverse{}Hinter den Bergen stirbt sie im Rot.\newverse{}Du – du weisst nichts von Sonne und Tod,\newverse{}Wendest die Augen zum Licht und zum Schein,\newverse{}Schlaf, es sind soviel Sonnen noch dein,\newverse{}Schlaf mein Kind – mein Kind, schlaf ein!\stanzaend{}\stanza{}Schlaf mein Kind – der Abendwind weht;\newverse{}Weiss man, woher er kommt, wohin er geht?\newverse{}Dunkel, verborgen die Wege hier sind,\newverse{}Dir, und auch mir, und uns allen mein Kind!\newverse{}Blinde – so gehn wir und gehen allein,\newverse{}Keiner kann Keinem Gefährte hier sein –\newverse{}Schlaf mein Kind – mein Kind, schlaf ein!\stanzaend{}\stanza{}Schlaf mein Kind und horch nicht auf mich!\newverse{}Sinn hat’s für mich nur, und Schall ist’s für dich;\newverse{}Schall nur, wie Windeswehn, Wassergerinn,\newverse{}Worte – vielleicht eines Lebens Gewinn!\newverse{}Was ich gewonnen, gräbt mit mir man ein,\newverse{}Keiner kann Keinem ein Erbe hier sein –\newverse{}Schlaf mein Kind – mein Kind, schlaf ein!\stanzaend{}\pstart
           \spacefill\mbox{Richard Beer-Hofmann.}\pend
           \endnumbering\briefempfaengerindex{Schnitzler, Arthur@\textsc{Schnitzler, Arthur}!zzzBeer-Hofmann, Richard@\emph{von Richard Beer-Hofmann}!1910-07-071@{7. 7. 1910}|)be}\mylabel{h}  \normalsize

\doendnotes{C}
\bigskip
\vfill

\clearpage

\footnotesize

\lohead{\textsc{register}}

% Definiere theindex-Environment komplett neu ohne reledmac
\makeatletter
\renewenvironment{theindex}{%
  \section*{\indexname}%
  \setlength{\parindent}{0pt}%
  \setlength{\parskip}{0pt plus 0.3pt}%
  \let\item\@idxitem
}{%
  \clearpage
}
\makeatother

\IfFileExists{\jobname-pw.ind}{\input{\jobname-pw.ind}}{}

\end{document}

      