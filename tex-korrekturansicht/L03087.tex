%% latex-korrekturansicht-vorspann.tex
%% Vorspann für die Korrekturansicht.
%% Lädt die gemeinsame Datei latex-vorspann.tex mit gesetztem Schalter.

\newif\ifkorrekturansicht
\korrekturansichttrue

\input{../tex-inputs/latex-vorspann}


\renewcommand{\erwaehntePersonen}{Personen: Marie Glümer, Auguste Glümer, Maxim Gorkij, Paul Martin Marton, Olga Schnitzler, Elisabeth Steinrück}
\renewcommand{\erwaehnteOrte}{Orte: Berlin, Dessauer Straße, Russland, Salzburg, Wien}
\renewcommand{\erwaehnteWerke}{Werke: Berliner Lokal-Anzeiger, Die neueste Verlobung in Berliner Theaterkreisen, [Artikel über eine russische Judenverfolgung]}
\section[ Paul Goldmann an Arthur Schnitzler, 28. 9. {[}1901{]}]{Paul Goldmann an Arthur Schnitzler, 28. 9. {[}1901{]}}
\nopagebreak\mylabel{v}
\rehead{ }\normalsize\beginnumbering\briefempfaengerindex{Schnitzler, Arthur@\textsc{Schnitzler, Arthur}!zzzGoldmann, Paul@\emph{von Paul Goldmann}!1901-09-281@{28. 9. {[}1901{]}}|(be}
\toendnotes[C]{\smallbreak\pagebreak[2]}\Standort{DLA, A:Schnitzler, HS.NZ85.1.3171.}
\physDesc{Brief, 1 Blatt, 2 Seiten
\newline{}Handschrift: blaue Tinte, deutsche Kurrent
\newline{}Schnitzler: 1) mit Bleistift das Jahr »{[}1{]}901« vermerkt  2) mit rotem Buntstift drei Unterstreichungen}\toendnotes[C]{\smallbreak}
\pstart
           \noindent{}\raggedleft{}{\pb}\textcolor{pink}{\textcolor{gray}{\textbf{DESSAUERSTRASSE 19}}}{}\ledrightnote{\textcolor{pink}{Dessauer Straße}}\pend
           
\pstart
           \textcolor{pink}{Berlin}{}\ledrightnote{\textcolor{pink}{Berlin}}, 28. September.\pend
           
\pstart\center{}Mein lieber Freund,\pend
\pstart
           Dank für Deinen lieben Brief!\pend
           
\pstart
           Zu \label{K_L03087-1v}\edtext{\textsc{\textcolor{blue}{Glümers}{}\ledrightnote{\textcolor{blue}{Marie Glümer}{\newline}\textcolor{blue}{Auguste Glümer}}}}{\lemma{\textnormal{\emph{Glümers}}}\Cendnote{\textnormal{Bezug auf die Verlobung \textcolor{blue}{Marie Glümer}s mit \textcolor{blue}{Paul Martin Blaustein}, siehe A. S.: \emph{Tagebuch}, 27. 9. 1901}}}\label{K_L03087-1h} werde ich nicht hingehen. Ich betrachte überhaupt meine Beziehungen mit ihnen
               als abgeſchloſſen. Es iſt empörend, daß ich, nachdem mir zwei Jahre aufs
               Freundſchaftlichſte verkehrt haben, die \textcolor{green}{Verlobungs-Nachricht}{}\ledrightnote{{$\rightarrow$}\textcolor{green}{Die neueste Verlobung in Berliner Theaterkreisen}} aus dem \label{K_L03087-2v}\edtext{»\textcolor{green}{Lokalanzeiger}{}\ledrightnote{\textcolor{green}{Berliner Lokal-Anzeiger}}«}{\lemma{\textnormal{\emph{»Lokalanzeiger«}}}\Cendnote{\textnormal{[O. V.]: \emph{\textcolor{green}{[Die neueste Verlobung in Berliner
                        Theaterkreisen]}}. In: \emph{\textcolor{green}{Berliner
                        Lokal-Anzeiger}}, Jg. 19, Nr. 453, 27. 9. 1901, Morgenblatt, 1. Ausgabe, S. 2.}}}\label{K_L03087-2h} erfahren
               muß!\pend
           
\pstart
           Im Übrigen iſt es wirklich das Beſte. Der Herr \textcolor{blue}{Direktor}{}\ledrightnote{{$\rightarrow$}\textcolor{blue}{Paul Martin Marton}} mag ein {\pb}Schwindler ſein, – für ſeine \textcolor{blue}{Frau}{}\ledrightnote{{$\rightarrow$}\textcolor{blue}{Marie Glümer}} wird er ſchon ſorgen.
               Vielleicht ſchwindelt er ſich auch hinauf. Jedenfalls kommt das arme \textcolor{blue}{Mädel}{}\ledrightnote{{$\rightarrow$}\textcolor{blue}{Marie Glümer}} aus den ſchlimmſten
               Exiſtenzforgen heraus.\pend
           
\pstart
           Ich ſehe ſie noch in \textcolor{pink}{Salzburg}{}\ledrightnote{\textcolor{pink}{Salzburg}}, wie ich ſie mit Dir
               zuſammen \label{K_L03087-4v}\edtext{beſuchte}{\lemma{\textnormal{\emph{beſuchte}}}\Cendnote{\textnormal{vermutlich Bezug auf den gemeinsamen \textcolor{pink}{Salzburg}aufenthalt Ende September 1890}}}\label{K_L03087-4h}. Wer hätte damals das Alles geahnt?\pend
           
\pstart
           Ich ſende Dir heut einen \label{K_L03087-3v}\edtext{\textcolor{green}{Artikel}{}\ledrightnote{{$\rightarrow$}\textcolor{green}{[Artikel über eine russische Judenverfolgung]}}}{\lemma{\textnormal{\emph{Artikel}}}\Cendnote{\textnormal{nicht ermittelt, Beilage nicht
                  erhalten}}}\label{K_L03087-3h} von \textsc{\textcolor{blue}{Gorki}{}\ledrightnote{\textcolor{blue}{Maxim Gorkij}}}, der mich tief ergriffen hat, – die Schilderung einer \textcolor{pink}{ruſſ}{}\ledrightnote{{$\rightarrow$}\textcolor{pink}{Russland}}iſchen Judenverfolgung.\pend
           
\pstart
           Viele treue Grüße! {\\[\baselineskip]}Dein \spacefill\mbox{Paul Goldmann}\pend
           \leftskip=0em{}
\pstart
           \noindent{}Alles Liebe den beiden \textcolor{blue}{Schweſtern}{}\ledrightnote{{$\rightarrow$}\textcolor{blue}{Olga Schnitzler}{\newline}{$\rightarrow$}\textcolor{blue}{Elisabeth Steinrück}}!\pend
           \endnumbering\briefempfaengerindex{Schnitzler, Arthur@\textsc{Schnitzler, Arthur}!zzzGoldmann, Paul@\emph{von Paul Goldmann}!1901-09-281@{28. 9. {[}1901{]}}|)be}\mylabel{h}
\begin{anhang}
\end{anhang}\normalsize

\doendnotes{C}
\bigskip
\vfill

\clearpage

\footnotesize

\lohead{\textsc{register}}

% Definiere theindex-Environment komplett neu ohne reledmac
\makeatletter
\renewenvironment{theindex}{%
  \section*{\indexname}%
  \setlength{\parindent}{0pt}%
  \setlength{\parskip}{0pt plus 0.3pt}%
  \let\item\@idxitem
}{%
  \clearpage
}
\makeatother

\IfFileExists{\jobname-pw.ind}{\input{\jobname-pw.ind}}{}

\end{document}

      