%% latex-korrekturansicht-vorspann.tex
%% Vorspann für die Korrekturansicht.
%% Lädt die gemeinsame Datei latex-vorspann.tex mit gesetztem Schalter.

\newif\ifkorrekturansicht
\korrekturansichttrue

\input{../tex-inputs/latex-vorspann}


               \section[Paul Goldmann an Arthur Schnitzler, 23. 8. {[}1893{]}]{ Paul Goldmann an Arthur Schnitzler, 23. 8. {[}1893{]}}\nopagebreak\mylabel{v}\rehead{ }\normalsize\beginnumbering\briefempfaengerindex{Schnitzler, Arthur@\textsc{Schnitzler, Arthur}!zzzGoldmann, Paul@\emph{von Paul Goldmann}!1893-08-231@{23. 8. {[}1893{]}}|(be} \toendnotes[C]{\smallbreak\pagebreak[2]} \Standort{DLA, A:Schnitzler, HS.NZ85.1.3163.}
\physDesc{Brief, 1 Blatt, 4 Seiten
\newline{}Handschrift: schwarze Tinte, deutsche Kurrent
\newline{}Schnitzler: 1) mit Bleistift das Jahr »93« vermerkt 2) mit rotem Buntstift eine Unterstreichung}\toendnotes[C]{\smallbreak}\pstart
           \noindent{}{\pb}\textcolor{gray}{\textbf{\textbf{\textcolor{brown}{Frankfurter Zeitung}{}\ledrightnote{\textcolor{brown}{Frankfurter Zeitung}}.}}}\pend
           \pstart
           \textcolor{gray}{\textbf{\textbf{(\textcolor{brown}{\begin{otherlanguage}{french}Gazette de Francfort\end{otherlanguage}}{}\ledrightnote{\textcolor{brown}{Frankfurter Zeitung}}.)}}}\pend
           \pstart
           \textcolor{gray}{\textbf{\begin{otherlanguage}{french}\textcolor{blue}{Directeur}{}\ledrightnote{→\textcolor{blue}{Leopold Sonnemann}}\end{otherlanguage}{ }\textbf{M. \textcolor{blue}{L. Sonnemann}{}\ledrightnote{\textcolor{blue}{Leopold Sonnemann}}.}}}\hfill \textsc{\textcolor{pink}{Paris}{}\ledrightnote{\textcolor{pink}{Paris}}}, 23. August.\pend
           \pstart
           \begin{otherlanguage}{french}\textcolor{gray}{\textbf{\textcolor{green}{Journal}{}\ledrightnote{\textcolor{green}{Frankfurter Zeitung}} politique, financier,}}\end{otherlanguage}\pend
           \pstart
           \begin{otherlanguage}{french}\textcolor{gray}{\textbf{commercial et litteraire.}}\end{otherlanguage}\pend
           \pstart
           \begin{otherlanguage}{french}\textcolor{gray}{\textbf{\textbf{Paraissant trois fois par jour}}}\end{otherlanguage}\pend
           \pstart
           \begin{otherlanguage}{french}\textcolor{gray}{\textbf{\textbf{Bureaux à \textcolor{pink}{Paris}{}\ledrightnote{\textcolor{pink}{Paris}}:}}}\end{otherlanguage}\pend
           \pstart
           \begin{otherlanguage}{french}\textcolor{gray}{\textbf{\textbf{\textcolor{pink}{rue Richelieu 75}{}\ledrightnote{\textcolor{pink}{rue Richelieu}}.}}}\end{otherlanguage}\pend
           \pstart\center{}Mein lieber Arthur!\pend\pstart
           Ich könnte eigentlich jetzt ſchon fort. Aber eine unbezwingliche Geldverlegenheit
               hält mich noch zurück. Ich muß ſehen, irgendwo noch ein paar hundert \textsc{Frcs} aufzutreiben. Wenn mir das gelingt, will ich Montag ſortgehen. Aus verſchiedenen Gründen will und muß
               ich auf ein paar Tage zunächſt in die \textcolor{pink}{Schweiz}{}\ledrightnote{\textcolor{pink}{Schweiz}}.
               Du biſt im \label{K_L02713-1v}\edtext{\textsc{\textcolor{pink}{Pusterthal}{}\ledrightnote{\textcolor{pink}{Pustertal}}}}{\lemma{\textnormal{\emph{Pusterthal}}}\Cendnote{\textnormal{Zu einem gemeinsamen Aufenthalt in der
                     \textcolor{pink}{Schweiz} kam es nicht. \textcolor{blue}{Schnitzler} und \textcolor{blue}{Goldmann} sahen sich erst am 17. 9. 1893 und 18. 9. 1893 in \textcolor{pink}{Salzburg} wieder.}}}\label{K_L02713-1h}, alſo nicht allzuweit davon. Könnten wir nicht
               die nächſte Woche mitſammen {\pb}in der \textcolor{pink}{Schweiz}{}\ledrightnote{\textcolor{pink}{Schweiz}} verbringen? Wir träfen uns z. B. an einem der Tage
               der nächſten Woche irgendwo da unten, und ich reiſte am Ende mit Dir nach \textcolor{pink}{Salzburg}{}\ledrightnote{\textcolor{pink}{Salzburg}} in der Richtung \textsc{\textcolor{pink}{Wien}{}\ledrightnote{\textcolor{pink}{Wien}}} zurück. Hältſt Du dieſen Plan für durchführbar, ſo ſei ſo gut mir \uline{telegraphiſch} eine Nachricht nach \textsc{\textcolor{pink}{Paris}{}\ledrightnote{\textcolor{pink}{Paris}}} zu geben. (Adreſſe: \textsc{Goldmann}, \textcolor{pink}{\textsc{Paris, 75. Richelieu}}{}\ledrightnote{\textcolor{pink}{rue Richelieu}}). Theile mir eine telegraphiſche Antwortadreſſe mit, und vielleicht wird auf
               dieſe Weiſe der kühne Plan zur Wahrheit. Ich warte jedenfalls auf Dein\strikeout{\textcolor{gray}{e}} Telegramm noch Dienſtag und Mittwoch, da ich nicht {\pb}weiß, ob Du meinen Brief rechtzeitig erhältſt. In einem Tage können alle
               Verabredungen getroffen ſein.\pend
           \pstart
           Folgendes iſt ein Gerücht, für das ich nicht die mindeſte Bürgſchaft übernehme, da
               mein \label{K_L02713-2v}\edtext{Gewährsmann}{\lemma{\textnormal{\emph{Gewährsmann}}}\Cendnote{\textnormal{nicht identifiziert}}}\label{K_L02713-2h} ebenſogut
               gelogen haben kann, um mir ein Vergnügen zu machen. Anderſeits möchte ich es Dir doch
               nicht vorenthalten: Ein von \textcolor{pink}{Berlin}{}\ledrightnote{\textcolor{pink}{Berlin}}
               zurückkommender College ſagte auf meine Frage, er habe dort gehört, \label{K_L02713-3v}\edtext{\textsc{\textcolor{blue}{Blumenthal}{}\ledrightnote{\textcolor{blue}{Oskar Blumenthal}}} wolle das \textsc{Schnitzler}’ſche \textcolor{green}{Stück}{}\ledrightnote{→\textcolor{green}{Das Märchen. Schauspiel in drei Aufzügen}} im Herbſt gleich nach dem von \label{K_L02713-13v}\edtext{\textsc{\textcolor{green}{\textcolor{blue}{Skowronek}{}\ledrightnote{\textcolor{blue}{Richard Skowronnek}}}{}\ledrightnote{→\textcolor{green}{Eine Palastrevolution}}}}{\lemma{\textnormal{\emph{Skowronek}}}\Cendnote{\textnormal{\textcolor{blue}{Richard Skowronnek}s vieraktiges Lustspiel
                     \emph{\textcolor{green}{Der erste seines Stammes}} feierte am \textcolor{pink}{Berlin}er \emph{\textcolor{brown}{Lessing-Theater}} am 11. 11. 1893 Uraufführung.}}}\label{K_L02713-13h}
                  aufführen}{\lemma{\textnormal{\emph{Blumenthal … aufführen}}}\Cendnote{\textnormal{\textcolor{blue}{Oskar Blumenthal}, \textcolor{blue}{Leiter} des \emph{\textcolor{brown}{Lessing-Theater}}s in \textcolor{pink}{Berlin}, hatte \textcolor{blue}{Schnitzler} am 12. 8. 1893 bereits brieflich
                  mitgeteilt, dass nichts am Gerücht dran sei. (Oscar Blumenthal an Arthur Schnitzler, 12. 8. 1893)}}}\label{K_L02713-3h}. Nochmals: ohne \uline{jede}{ }{\pb}Garantie. Nur ein Möglichkeits-Spahn, um ihn mit
               Urlaubshoffnungen zu umſpinnen{\dotsfour}\pend
           \pstart
           Wird aus der Reiſe nichts, ſo erhältſt Du nach 1. September Nachricht von mir in \textcolor{pink}{Wien}{}\ledrightnote{\textcolor{pink}{Wien}}.\pend
           \pstart
           Viele treue Grüße! {\\[\baselineskip]}Dein {\\[\baselineskip]}\spacefill\mbox{Paul Goldm.}\pend
           \leftskip=0em{}\endnumbering\briefempfaengerindex{Schnitzler, Arthur@\textsc{Schnitzler, Arthur}!zzzGoldmann, Paul@\emph{von Paul Goldmann}!1893-08-231@{23. 8. {[}1893{]}}|)be}\mylabel{h}  \normalsize

\doendnotes{C}
\bigskip
\vfill

\clearpage

\footnotesize

\lohead{\textsc{register}}

% Definiere theindex-Environment komplett neu ohne reledmac
\makeatletter
\renewenvironment{theindex}{%
  \section*{\indexname}%
  \setlength{\parindent}{0pt}%
  \setlength{\parskip}{0pt plus 0.3pt}%
  \let\item\@idxitem
}{%
  \clearpage
}
\makeatother

\IfFileExists{\jobname-pw.ind}{\input{\jobname-pw.ind}}{}

\end{document}

      