%% latex-korrekturansicht-vorspann.tex
%% Vorspann für die Korrekturansicht.
%% Lädt die gemeinsame Datei latex-vorspann.tex mit gesetztem Schalter.

\newif\ifkorrekturansicht
\korrekturansichttrue

\input{../tex-inputs/latex-vorspann}


\renewcommand{\erwaehntePersonen}{Personen: Ludwig Fulda, Olga Schnitzler, Ida d’Albert}
\renewcommand{\erwaehnteOrte}{Orte: Baden-Baden, Berlin, Dessauer Straße, Italien, Südtirol, Wien}
\renewcommand{\erwaehnteWerke}{}
\section[ Paul Goldmann an Arthur Schnitzler, 15. 6. {[}1903{]}]{Paul Goldmann an Arthur Schnitzler, 15. 6. {[}1903{]}}
\nopagebreak\mylabel{v}
\rehead{ }\normalsize\beginnumbering\briefempfaengerindex{Schnitzler, Arthur@\textsc{Schnitzler, Arthur}!zzzGoldmann, Paul@\emph{von Paul Goldmann}!1903-06-151@{15. 6. {[}1903{]}}|(be}
\toendnotes[C]{\smallbreak\pagebreak[2]}\Standort{DLA, A:Schnitzler, HS.NZ85.1.3173.}
\physDesc{Brief, 1 Blatt, 2 Seiten
\newline{}Handschrift: blaue Tinte, deutsche Kurrent
\newline{}Schnitzler: 1) mit Bleistift das Jahr »{[}1{]}903.« vermerkt  2) mit rotem Buntstift eine Unterstreichung}\toendnotes[C]{\smallbreak}
\pstart
           \noindent{}\raggedleft{}{\pb}\textcolor{gray}{\textbf{\textcolor{pink}{DESSAUERSTRASSE 19}{}\ledrightnote{\textcolor{pink}{Dessauer Straße}}}}\pend
           
\pstart
           \textcolor{pink}{Berlin}{}\ledrightnote{\textcolor{pink}{Berlin}}, 15. Juni.\pend
           
\pstart{}Mein lieber Freund,\pend
\pstart
           Ich danke Dir für Deine lieben \label{K_L03374-8v}\edtext{Karten}{\lemma{\textnormal{\emph{Karten}}}\Cendnote{\textnormal{\textcolor{blue}{Schnitzler} und \textcolor{blue}{Olga Gussmann} reisten zwischen 28. 5. 1903 und 15. 6. 1903 nach \textcolor{pink}{Italien} und \textcolor{pink}{Südtirol}.}}}\label{K_L03374-8h} und bitte Dich, \textsc{\textcolor{blue}{Olga}{}\ledrightnote{\textcolor{blue}{Olga Schnitzler}}} für ihre Grüße zu danken.\pend
           
\pstart
           Ich habe wahnſinnig viel zu thun und kann daher \textsc{\textcolor{blue}{Olga}{}\ledrightnote{\textcolor{blue}{Olga Schnitzler}}s} Brief \label{K_L03374-1v}\edtext{noch immer nicht}{\lemma{\textnormal{\emph{noch immer nicht}}}\Cendnote{\textnormal{siehe Paul Goldmann an Arthur Schnitzler, 22. 5. [1903]}}}\label{K_L03374-1h} beantworten.\pend
           
\pstart
           \textsc{\textcolor{blue}{Fuldas}{}\ledrightnote{\textcolor{blue}{Ludwig Fulda}{\newline}\textcolor{blue}{Ida d’Albert}}} laſſen ſich, wie ich höre, diesmal ernſtlich {\pb}\label{K_L03374-2v}\edtext{ſcheiden}{\lemma{\textnormal{\emph{ſcheiden}}}\Cendnote{\textnormal{\textcolor{blue}{Schnitzler} hatte bereits von der Scheidung
                  von \textcolor{blue}{Ludwig} und \textcolor{blue}{Ida Fulda} gewusst, vgl. A. S.: \emph{Tagebuch}, 28. 4. 1903. Sie waren seit 1893 verheiratet. Siehe auch Paul Goldmann an Arthur Schnitzler, 27. 6. [1903].}}}\label{K_L03374-2h}; die Scheidungsklage ſoll bereits eingereicht ſein.
               Weißt Du etwas davon? \textcolor{blue}{Er}{}\ledrightnote{{$\rightarrow$}\textcolor{blue}{Ludwig Fulda}} iſt
               in \textcolor{pink}{Baden Baden}{}\ledrightnote{\textcolor{pink}{Baden-Baden}}, \textcolor{blue}{ſie}{}\ledrightnote{{$\rightarrow$}\textcolor{blue}{Ida d’Albert}}, glaube ich, in \textcolor{pink}{Berlin}{}\ledrightnote{\textcolor{pink}{Berlin}}.\pend
           
\pstart
           Herzlichſte Grüße Dir und \textsc{\textcolor{blue}{Olga}{}\ledrightnote{\textcolor{blue}{Olga Schnitzler}}}! Und weiter: \label{K_L03374-3v}\edtext{glückliche
                  Fahrt}{\lemma{\textnormal{\emph{glückliche
                  Fahrt}}}\Cendnote{\textnormal{siehe Paul Goldmann an Arthur Schnitzler, 22. 5. [1903]}}}\label{K_L03374-3h}! {\\[\baselineskip]}Dein {\\[\baselineskip]}\spacefill\mbox{Paul Goldm}\pend
           \leftskip=0em{}\endnumbering\briefempfaengerindex{Schnitzler, Arthur@\textsc{Schnitzler, Arthur}!zzzGoldmann, Paul@\emph{von Paul Goldmann}!1903-06-151@{15. 6. {[}1903{]}}|)be}\mylabel{h}  \normalsize

\doendnotes{C}
\bigskip
\vfill

\clearpage

\footnotesize

\lohead{\textsc{register}}

% Definiere theindex-Environment komplett neu ohne reledmac
\makeatletter
\renewenvironment{theindex}{%
  \section*{\indexname}%
  \setlength{\parindent}{0pt}%
  \setlength{\parskip}{0pt plus 0.3pt}%
  \let\item\@idxitem
}{%
  \clearpage
}
\makeatother

\IfFileExists{\jobname-pw.ind}{\input{\jobname-pw.ind}}{}

\end{document}

      