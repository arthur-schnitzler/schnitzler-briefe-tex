%% latex-korrekturansicht-vorspann.tex
%% Vorspann für die Korrekturansicht.
%% Lädt die gemeinsame Datei latex-vorspann.tex mit gesetztem Schalter.

\newif\ifkorrekturansicht
\korrekturansichttrue

\input{../tex-inputs/latex-vorspann}


               \section[Arthur Schnitzler an Hugo von Hofmannsthal, 4. 11. 1903]{ Arthur Schnitzler an Hugo von Hofmannsthal, 4. 11. 1903}\nopagebreak\mylabel{v}\rehead{ }\normalsize\beginnumbering\briefempfaengerindex{Hofmannsthal, Hugo von@\textsc{Hofmannsthal, Hugo von}!zzzSchnitzler, Arthur@\emph{von Arthur Schnitzler}!1903-11-041@{4. 11. 1903}|(be} \toendnotes[C]{\smallbreak\pagebreak[2]} \Standort{FDH, Hs-30885,105.}
\physDesc{Brief, 1 Blatt, 4 Seiten
\newline{}Handschrift: Bleistift, deutsche Kurrent}\buchAbdrucke{\weitereDrucke{Hugo von Hofmannsthal, Arthur Schnitzler: \emph{Briefwechsel}. Hg. Therese Nickl und Heinrich Schnitzler. Frankfurt am Main: \emph{S. Fischer} 1964, S. 176.} }\toendnotes[C]{\smallbreak}\pstart
           \raggedleft{}{\pb}\textcolor{pink}{\textsc{Spöttelgasse 7}}{}\ledrightnote{\textcolor{pink}{Edmund-Weiß-Gasse}}.
                     4. 11. 903\pend
           \pstart{}lieber Hugo, \pend\pstart
           über \textcolor{green}{Elektra}{}\ledrightnote{\textcolor{green}{Elektra. Tragödie in einem Aufzug}} hab ich mich ſehr gefreut, und das \textcolor{blue}{Goldma{\geminationn}}{}\ledrightnote{\textcolor{blue}{Paul Goldmann}}sche \textcolor{green}{Telegramm}{}\ledrightnote{→\textcolor{green}{Aus Berlin [Elektra-Premiere]}} gehört zu dem Übrigen. Denken
               Sie, daſs er \strikeout{mir}, ſeit er \textcolor{pink}{Wien}{}\ledrightnote{\textcolor{pink}{Wien}} verlaſſen \strikeout{hat}, Mitte
                  September, keine Zeile an mich geſchrieben hat.\pend
           \pstart
           – Das \textcolor{green}{Stück}{}\ledrightnote{→\textcolor{green}{Der einsame Weg. Schauspiel in fünf Akten}} iſt ſchon an \textcolor{blue}{Brahm}{}\ledrightnote{\textcolor{blue}{Otto Brahm}} abgegangen. Freitag gehn wir
                  {\pb}auf ein paar Tage auf den \textcolor{pink}{Semmering}{}\ledrightnote{\textcolor{pink}{Semmering}}. Mitte nächſter Woche möchte ich \label{K_L01335_1v}\edtext{\textcolor{green}{vorleſen}{}\ledrightnote{→\textcolor{green}{Der einsame Weg. Schauspiel in fünf Akten}}}{\lemma{\textnormal{\emph{vorleſen}}}\Cendnote{\textnormal{vgl. A. S.: \emph{Tagebuch}, 12. 11. 1903}}}\label{K_L01335_1h}. Sagen Sie mir bitte, ob Ihnen
                  Dienſtag{ }Abend ½ 7 recht wäre. Fragen Sie auch gleich den \textcolor{blue}{Richard}{}\ledrightnote{\textcolor{blue}{Richard Beer-Hofmann}}.\pend
           \pstart
           Dieſer Tage iſt die \label{K_L01335_2v}\edtext{\textcolor{green}{\textsc{Kakadu}}{}\ledrightnote{\textcolor{green}{Der grüne Kakadu. Groteske in einem Akt}}\textsc{première} in \textcolor{pink}{Paris}{}\ledrightnote{\textcolor{pink}{Paris}}}{\lemma{\textnormal{\emph{Kakadupremière in Paris}}}\Cendnote{\textnormal{am
                     7. 11. 1903}}}\label{K_L01335_2h}; \textcolor{blue}{\textsc{Antoine}}{}\ledrightnote{\textcolor{blue}{André Antoine}}{ }ſcheint ſich nach einem Brief von ihm und von
               einigen andern, die Proben geſehen haben, viel {\pb}zu
               verſprechen.\pend
           \pstart
           Grüßen Sie von uns beiden herzlich \textcolor{blue}{\textsc{Gerty}}{}\ledrightnote{\textcolor{blue}{Gertrude von Hofmannsthal}} und \textcolor{blue}{Hofmannsthal den
                  Winzigen}{}\ledrightnote{→\textcolor{blue}{Franz von Hofmannsthal}}. Sich ſelber desgleichen.\pend
           \pstart
           – Hat ſich die \textcolor{brown}{Burg}{}\ledrightnote{\textcolor{brown}{Burgtheater}} um die ihrer \textcolor{green}{Hoheit entkleidete Griechin}{}\ledrightnote{→\textcolor{green}{Elektra. Tragödie in einem Aufzug}} beworben?{\dotstwo} Aus dem alten \textcolor{blue}{\textcolor{green}{\textsc{Sophokles}}{}\ledrightnote{→\textcolor{green}{Elektra. Tragödie}}}{}\ledrightnote{\textcolor{blue}{Sophokles}} ein Zugstück zu machen! Echt {\pb}jüdiſch.\pend
           \pstart
           Ihr{\\[\baselineskip]}\spacefill\mbox{A.}\pend
           \leftskip=0em{}\endnumbering\briefempfaengerindex{Hofmannsthal, Hugo von@\textsc{Hofmannsthal, Hugo von}!zzzSchnitzler, Arthur@\emph{von Arthur Schnitzler}!1903-11-041@{4. 11. 1903}|)be}\mylabel{h}  \normalsize

\doendnotes{C}
\bigskip
\vfill

\clearpage

\footnotesize

\lohead{\textsc{register}}

% Definiere theindex-Environment komplett neu ohne reledmac
\makeatletter
\renewenvironment{theindex}{%
  \section*{\indexname}%
  \setlength{\parindent}{0pt}%
  \setlength{\parskip}{0pt plus 0.3pt}%
  \let\item\@idxitem
}{%
  \clearpage
}
\makeatother

\IfFileExists{\jobname-pw.ind}{\input{\jobname-pw.ind}}{}

\end{document}

      