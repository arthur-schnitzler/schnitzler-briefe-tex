%% latex-korrekturansicht-vorspann.tex
%% Vorspann für die Korrekturansicht.
%% Lädt die gemeinsame Datei latex-vorspann.tex mit gesetztem Schalter.

\newif\ifkorrekturansicht
\korrekturansichttrue

\input{../tex-inputs/latex-vorspann}


               \section[Hermann Bahr an Arthur Schnitzler, {[}1?. 2. 1892{]}]{ Hermann Bahr an Arthur Schnitzler, {[}1?. 2. 1892{]}}\nopagebreak\mylabel{v}\rehead{ }\normalsize\beginnumbering\briefempfaengerindex{Schnitzler, Arthur@\textsc{Schnitzler, Arthur}!zzzBahr, Hermann@\emph{von Hermann Bahr}!1892-02-011@{{[}1?. 2. 1892{]}}|(be} \toendnotes[C]{\smallbreak\pagebreak[2]} \Standort{CUL, Schnitzler, B 5b.}
\physDesc{Visitenkarte
\newline{}Handschrift: Bleistift, deutsche Kurrent
\newline{}Schnitzler: mit rotem Buntstift datiert: »92« \newline{}Ordnung: mit Bleistift und rotem Buntstift von unbekannter Hand 
                           nummeriert: »2« }\buchAbdrucke{\weitereDrucke{Hermann Bahr, Arthur Schnitzler: \emph{Briefwechsel, Aufzeichnungen, Dokumente (1891–1931)}. Hg. Kurt Ifkovits und Martin Anton Müller. Göttingen: \emph{Wallstein} 2018, S. 20.} }\toendnotes[C]{\smallbreak}\pstart
           \noindent{}\centering{}\textcolor{gray}{\textbf{{\pb}Hermann Bahr.}}\pend
           \pstart
           \noindent{}bittet Sie, ihm mitzuteilen, ob er Ihnen eine Einladg {\pb}zu \label{K_L00066_1v}\edtext{\textsc{Matinee}{ }\textcolor{blue}{\textsc{Reicher}}{}\ledrightnote{\textcolor{blue}{Emanuel Reicher}}}{\lemma{\textnormal{\emph{Matinee Reicher}}}\Cendnote{\textnormal{Die Matinée fand am 7. 2. 1892{ }statt; \textcolor{blue}{Schnitzler} nahm
                  teil.}}}\label{K_L00066_1h} bei \textcolor{blue}{Goldſchmid}{}\ledrightnote{\textcolor{blue}{Adalbert von Goldschmidt}} beſorgen ſoll\pend
           \pstart
           Herzlichſt{\\[\baselineskip]}\spacefill\mbox{H}\pend
           \leftskip=0em{}\endnumbering\briefempfaengerindex{Schnitzler, Arthur@\textsc{Schnitzler, Arthur}!zzzBahr, Hermann@\emph{von Hermann Bahr}!1892-02-011@{{[}1?. 2. 1892{]}}|)be}\mylabel{h}  \normalsize

\doendnotes{C}
\bigskip
\vfill

\clearpage

\footnotesize

\lohead{\textsc{register}}

% Definiere theindex-Environment komplett neu ohne reledmac
\makeatletter
\renewenvironment{theindex}{%
  \section*{\indexname}%
  \setlength{\parindent}{0pt}%
  \setlength{\parskip}{0pt plus 0.3pt}%
  \let\item\@idxitem
}{%
  \clearpage
}
\makeatother

\IfFileExists{\jobname-pw.ind}{\input{\jobname-pw.ind}}{}

\end{document}

      