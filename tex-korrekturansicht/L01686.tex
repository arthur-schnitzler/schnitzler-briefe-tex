%% latex-korrekturansicht-vorspann.tex
%% Vorspann für die Korrekturansicht.
%% Lädt die gemeinsame Datei latex-vorspann.tex mit gesetztem Schalter.

\newif\ifkorrekturansicht
\korrekturansichttrue

\input{../tex-inputs/latex-vorspann}


               \section[Arthur Schnitzler an Hugo von Hofmannsthal, 25. 6. 1907]{ Arthur Schnitzler an Hugo von Hofmannsthal, 25. 6. 1907}\nopagebreak\mylabel{v}\rehead{ }\normalsize\beginnumbering\briefempfaengerindex{Hofmannsthal, Hugo von@\textsc{Hofmannsthal, Hugo von}!zzzSchnitzler, Arthur@\emph{von Arthur Schnitzler}!1907-06-251@{25. 6. 1907}|(be} \toendnotes[C]{\smallbreak\pagebreak[2]} \Standort{FDH, Hs-30885,128.}
\physDesc{Brief, 1 Blatt, 3 Seiten
\newline{}Handschrift: schwarze Tinte, deutsche Kurrent}\buchAbdrucke{\weitereDrucke{Hugo von Hofmannsthal, Arthur Schnitzler: \emph{Briefwechsel}. Hg. Therese Nickl und Heinrich Schnitzler. Frankfurt am Main: \emph{S. Fischer} 1964, S. 229–230.} }\toendnotes[C]{\smallbreak}\pstart
           \raggedleft{}{\pb}\textcolor{pink}{Wien}{}\ledrightnote{\textcolor{pink}{Wien}}{ }25. 6. 907\pend
           \pstart{}Mein lieber Hugo, \pend\pstart
           morgen fahren wir nach \textcolor{pink}{Villach}{}\ledrightnote{\textcolor{pink}{Villach}}; – von dort aus
               wollen wir uns umſehen, ob wir irgd was (\textsc{\textcolor{pink}{Veldes}{}\ledrightnote{\textcolor{pink}{Veldes}}? \textcolor{pink}{Wochein}{}\ledrightnote{\textcolor{pink}{Die Wochein}}}? oder
               ſonſt wo) – we{\geminationn}s gut geht, zu längerem Aufenthalt finden. Den \textcolor{blue}{Buben}{}\ledrightnote{→\textcolor{blue}{Heinrich Schnitzler}} laſſen wir erſt nachko{\geminationm}en we{\geminationn} wir wiſſen, wo
               unſres Bleibens. Der \textcolor{green}{Roman}{}\ledrightnote{→\textcolor{green}{Der Weg ins Freie. Roman}}, den
               ich nun tüchtig durchfeile, zum großen Theil natürlich neu ſchreibe, zieht mit. Das
                  \textcolor{green}{Winterſtück}{}\ledrightnote{→\textcolor{green}{Das Wort. Tragikomödie in fünf Akten}}{ }{\pb}hab ich weggeschmiſſen; nicht
               weggelegt, da ich in ein ſchlechtes Verhältnis dazu gerieth. Irgend ein Wurzelfehler
               war da, ſo daſs ich durch corrigiren nicht weiter kam. Vielleicht muſs der Stoff in
               andre Erde geſetzt werden, doch weiſs ich noch nicht in welche. Vorläufig gehn mir
               andre theatralische Einfälle näher. – Wir haben in der letzten Zeit viele Leute
               geſehen; es gab manche ſehr gute Stunden, mit \textcolor{blue}{Richard}{}\ledrightnote{\textcolor{blue}{Richard Beer-Hofmann}}, \textcolor{blue}{\textsc{Wasserman\textcolor{gray}{n}}}{}\ledrightnote{\textcolor{blue}{Jakob Wassermann}}, \textcolor{blue}{Kainz}{}\ledrightnote{\textcolor{blue}{Josef Kainz}}, \introOben{}\textcolor{blue}{\textsc{Fred}}{}\ledrightnote{\textcolor{blue}{W. Fred}},
                  und and\textcolor{gray}{re}\introOben{}; auch das \textsc{Tennis} war ſchön – nur lockt
               es mich {\pb}doch ins einſamere. Der Gräfin \textcolor{blue}{Thun}{}\ledrightnote{\textcolor{blue}{Christiane von Thun-Hohenstein-Salm-Reifferscheidt}} hab ich die \textcolor{green}{Dä{\geminationm}erſeelen}{}\ledrightnote{\textcolor{green}{Dämmerseelen. Novellen}} geſchickt; ſie hat in einem ſehr
               liebenswürdg Telegra{\geminationm} gedankt. Wie lange bleiben Sie
               noch am \textcolor{pink}{Lido}{}\ledrightnote{\textcolor{pink}{Lido}}? Von endgiltigem Zeltaufſchlag
               verſtändige ich Sie gleich. Ich hoffe Sie leſen im September was
               wundervolles vor.\pend
           \pstart
           Seien Sie, un\textcolor{gray}{d}{ }\textcolor{blue}{Gerty}{}\ledrightnote{\textcolor{blue}{Gertrude von Hofmannsthal}} herzlichſt gegrüßt, von \textcolor{blue}{\textcolor{gray}{O}lga}{}\ledrightnote{\textcolor{blue}{Olga Schnitzler}} u mir.\pend
           \pstart
           Ihr{\\[\baselineskip]}\spacefill\mbox{Arthur}\pend
           \leftskip=0em{}\endnumbering\briefempfaengerindex{Hofmannsthal, Hugo von@\textsc{Hofmannsthal, Hugo von}!zzzSchnitzler, Arthur@\emph{von Arthur Schnitzler}!1907-06-251@{25. 6. 1907}|)be}\mylabel{h}  \normalsize

\doendnotes{C}
\bigskip
\vfill

\clearpage

\footnotesize

\lohead{\textsc{register}}

% Definiere theindex-Environment komplett neu ohne reledmac
\makeatletter
\renewenvironment{theindex}{%
  \section*{\indexname}%
  \setlength{\parindent}{0pt}%
  \setlength{\parskip}{0pt plus 0.3pt}%
  \let\item\@idxitem
}{%
  \clearpage
}
\makeatother

\IfFileExists{\jobname-pw.ind}{\input{\jobname-pw.ind}}{}

\end{document}

      