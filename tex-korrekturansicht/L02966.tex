%% latex-korrekturansicht-vorspann.tex
%% Vorspann für die Korrekturansicht.
%% Lädt die gemeinsame Datei latex-vorspann.tex mit gesetztem Schalter.

\newif\ifkorrekturansicht
\korrekturansichttrue

\input{../tex-inputs/latex-vorspann}


\renewcommand{\erwaehntePersonen}{Personen: Camilla Gerzhofer, Carl Karlweis, Felix Salten, Ottilie Salten}
\renewcommand{\erwaehnteOrte}{Orte: Volkstheater, Wien}
\renewcommand{\erwaehnteWerke}{Werke: Das Vermächtnis. Schauspiel in drei Akten, Das liebe Ich}
\section[ Arthur Schnitzler an Felix Salten, 24. 9. 1898]{Arthur Schnitzler an Felix Salten, 24. 9. 1898}
\nopagebreak\mylabel{v}
\rehead{ }\normalsize\beginnumbering\briefempfaengerindex{Salten, Felix@\textsc{Salten, Felix}!zzzSchnitzler, Arthur@\emph{von Arthur Schnitzler}!1898-09-241@{24. 9. 1898}|(be}
\toendnotes[C]{\smallbreak\pagebreak[2]}\Standort{Wienbibliothek im Rathaus, ZPH 1681, 2.1.516.}
\physDesc{Brief, 1 Blatt, 3 Seiten, 512 Zeichen
\newline{}Handschrift: Bleistift, deutsche Kurrent
\newline{}Ordnung: mit Bleistift von unbekannter Hand Nummerierung der Blätter des Konvoluts:
                                    »73«–»74« }
\buchAbdrucke{\weitereDrucke{Arthur Schnitzler: \emph{Briefe 1875–1912}. Hg. Therese Nickl und Heinrich Schnitzler. Frankfurt am Main: \emph{S. Fischer} 1981, S. 354.} }\toendnotes[C]{\smallbreak}
\pstart
           \raggedleft{}{\pb}24. 9. 98\pend
           
\pstart{}Lieber Freund,\pend
\pstart
           den \label{K_L02966-1v}\edtext{\textcolor{green}{Lu\textcolor{gray}{lu}}{}\ledrightnote{{$\rightarrow$}\textcolor{green}{Das Vermächtnis. Schauspiel in drei Akten}}}{\lemma{\textnormal{\emph{Lulu}}}\Cendnote{\textnormal{siehe Felix Salten an Arthur Schnitzler, 23. 9. 1898}}}\label{K_L02966-1h} wird die kleine \textcolor{blue}{Gerzhofer}{}\ledrightnote{\textcolor{blue}{Camilla Gerzhofer}}, alſo ein
               wirkliches Kind ſpielen, welche Eventual. wir noch gar nicht in Betracht gezogen
               hatten, und was mir {\pb}doch das weitaus beſte
               zu ſein ſcheint. We{\geminationn} Sie das Fräulein \textcolor{blue}{Metzl}{}\ledrightnote{\textcolor{blue}{Ottilie Salten}} ſagen, wird ſie gewiſs nicht im mindeſten verletzt
               ſein. Sie wiſſen, daſs unter den wirklichen Schauſpielerin\textcolor{gray}{nen} für
               mich nur \textsc{Frl. \textcolor{blue}{Metzl}{}\ledrightnote{\textcolor{blue}{Ottilie Salten}}} in {\pb}Betracht kam; aber das wirkliche
                  \uline{Kind}, das Talent hat, iſt in der Rolle entſchieden
               vorzuziehen.\pend
           
\pstart
           Ich ſehe Sie hoffentlich \label{K_L02966-2v}\edtext{heut{ }Abd}{\lemma{\textnormal{\emph{heut Abd}}}\Cendnote{\textnormal{\textcolor{blue}{Schnitzler} besuchte am Abend
                  des 24. 9. 1898 die
                  Premiere von \textcolor{blue}{Carl Karlweis}’ \emph{\textcolor{green}{Das liebe Ich}} im \textcolor{pink}{Volkstheater}. \textcolor{blue}{Salten}s Anwesenheit
                  ist nicht nachweisbar.}}}\label{K_L02966-2h}\pend
           
\pstart
           HerzlGrü\textcolor{gray}{ſse}{ }{\\[\baselineskip]}Ihr
                  \spacefill\mbox{ArthS.}\pend
           \leftskip=0em{}\endnumbering\briefempfaengerindex{Salten, Felix@\textsc{Salten, Felix}!zzzSchnitzler, Arthur@\emph{von Arthur Schnitzler}!1898-09-241@{24. 9. 1898}|)be}\mylabel{h}  \normalsize

\doendnotes{C}
\bigskip
\vfill

\clearpage

\footnotesize

\lohead{\textsc{register}}

% Definiere theindex-Environment komplett neu ohne reledmac
\makeatletter
\renewenvironment{theindex}{%
  \section*{\indexname}%
  \setlength{\parindent}{0pt}%
  \setlength{\parskip}{0pt plus 0.3pt}%
  \let\item\@idxitem
}{%
  \clearpage
}
\makeatother

\IfFileExists{\jobname-pw.ind}{\input{\jobname-pw.ind}}{}

\end{document}

      