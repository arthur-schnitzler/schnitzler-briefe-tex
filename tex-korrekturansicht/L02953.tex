%% latex-korrekturansicht-vorspann.tex
%% Vorspann für die Korrekturansicht.
%% Lädt die gemeinsame Datei latex-vorspann.tex mit gesetztem Schalter.

\newif\ifkorrekturansicht
\korrekturansichttrue

\input{../tex-inputs/latex-vorspann}


\renewcommand{\erwaehntePersonen}{Personen: Richard Beer-Hofmann, Hugo von Hofmannsthal, Felix Salten}
\renewcommand{\erwaehnteOrte}{Orte: Wien}
\renewcommand{\erwaehnteWerke}{Werke: Tagebuch}
\section[Arthur Schnitzler an Felix Salten, {[}27. 6. 1891?{]}]{Arthur Schnitzler an Felix Salten, {[}27. 6. 1891?{]}}
\nopagebreak\mylabel{v}
\rehead{ }\normalsize\beginnumbering\briefempfaengerindex{Salten, Felix@\textsc{Salten, Felix}!zzzSchnitzler, Arthur@\emph{von Arthur Schnitzler}!1891-06-271@{{[}27. 6. 1891?{]}}|(be}
\toendnotes[C]{\smallbreak\pagebreak[2]}\Standort{Wienbibliothek im Rathaus, ZPH 1681, 2.1.516.}
\physDesc{Brief, 1 Blatt, 2 Seiten, 345 Zeichen
\newline{}Handschrift: Bleistift, deutsche Kurrent
\newline{}Ordnung: mit Bleistift von unbekannter Hand Nummerierung der Seiten des Konvoluts:
                                    »17«–»18« }\toendnotes[C]{\smallbreak}
\pstart{}{\pb}Lieber Freund,\pend
\pstart
           \textcolor{blue}{Loris}{}\ledrightnote{\textcolor{blue}{Hugo von Hofmannsthal}} war ſehr ärgerlich, als ich ihm ſagte,
               dſs Sie morgen möglicherweiſe nicht ko{\geminationm}en; behauptet, er
               habe ſich extra Ihretwegen frei{\pb}gemacht;
               ſchwört, er ſagt Ihnen nicht Adieu wenn Sie wegfahren – was aus alldem folgt, iſt nur
               die längſt beka{\geminationn}te Thatſache, daſs Sie \label{K_L02953-1v}\edtext{morgen So{\geminationn}tag}{\lemma{\textnormal{\emph{morgen Sonntag}}}\Cendnote{\textnormal{Das Korrespondenzstück ist undatiert.
                  Die Hinweise, die sich ihm entnehmen lassen, besagen, dass es an einem Samstag
                  verfasst wurde, sich \textcolor{blue}{Schnitzler} und \textcolor{blue}{Hofmannsthal} am Sonntag nachmittag treffen
                  wollten und möglicherweise eine Abreise \textcolor{blue}{Salten}s bevorstand. Durch die Verwendung von »\textcolor{blue}{Loris}« als Name ist es mit ziemlicher Wahrscheinlichkeit vor 1893 einzuordnen. Ein offensichtlicher Sonntag, an dem es zu einem
                  Zusammentreffen aller drei an einem Nachmittag kam, bietet sich im \emph{\textcolor{green}{Tagebuch}}{ }\textcolor{blue}{Schnitzler}s nicht an. Für Sonntag, den 21. 6. 1891, ist ein
                  besonderes Zusammentreffen zwischen \textcolor{blue}{Hofmannsthal} und \textcolor{blue}{Salten}
                  dokumentiert, durch das es nachvollziehbar scheint, dass \textcolor{blue}{Hofmannsthal} an eine Fortsetzung des Gespräches lebhaftes
                  Interesse hatte: »Vorm. \textcolor{blue}{Loris} und
                        \textcolor{blue}{Salten} bei mir (letztrer hatte bei mir
                     geschlafen). Wir ›sprühten‹. \textcolor{blue}{Loris} ist
                     einfach stupend! ―« In \textcolor{blue}{Salten}s 
                  Nachlass ist ein ›Protokoll‹ der Gespräche überliefert (\emph{Wienbibliothek},
                  Nachlass Salten, ZPH 1681, Schachtel 5, 1.2.10). In den folgenden Tagen
                  begegneten sich \textcolor{blue}{Schnitzler} und \textcolor{blue}{Salten}
                  mehrfach, vermutlich aber nicht am Samstag, dem 26. 6. 1891, für
                  den \textcolor{blue}{Schnitzler} keinen Eintrag anlegte. Am Folgetag, dem Sonntag,
                  kam es am Abend zu einem gemeinsamen Essen von \textcolor{blue}{Schnitzler}, 
                  \textcolor{blue}{Hofmannsthal} und \textcolor{blue}{Beer-Hofmann},
                  so dass es möglich scheint, dass dazu auch \textcolor{blue}{Salten} geladen
                  gewesen wäre, aber abgesagt hatte.}}}\label{K_L02953-1h}{ }5 Uhr ſicher von mir erwartet werden \pend
           
\pstart
           Herzlich Ihr {\\[\baselineskip]}\spacefill\mbox{Arthur}\pend
           \leftskip=0em{}\endnumbering\briefempfaengerindex{Salten, Felix@\textsc{Salten, Felix}!zzzSchnitzler, Arthur@\emph{von Arthur Schnitzler}!1891-06-271@{{[}27. 6. 1891?{]}}|)be}\mylabel{h}  \normalsize

\doendnotes{C}
\bigskip
\vfill

\clearpage

\footnotesize

\lohead{\textsc{register}}

% Definiere theindex-Environment komplett neu ohne reledmac
\makeatletter
\renewenvironment{theindex}{%
  \section*{\indexname}%
  \setlength{\parindent}{0pt}%
  \setlength{\parskip}{0pt plus 0.3pt}%
  \let\item\@idxitem
}{%
  \clearpage
}
\makeatother

\IfFileExists{\jobname-pw.ind}{\input{\jobname-pw.ind}}{}

\end{document}

      