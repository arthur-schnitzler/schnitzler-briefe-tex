%% latex-korrekturansicht-vorspann.tex
%% Vorspann für die Korrekturansicht.
%% Lädt die gemeinsame Datei latex-vorspann.tex mit gesetztem Schalter.

\newif\ifkorrekturansicht
\korrekturansichttrue

\input{../tex-inputs/latex-vorspann}


\renewcommand{\erwaehntePersonen}{Personen:  ?? [Anstandsdame von Anna und Clara Loeb], Marianne Benedict, Anna Epstein, Clara Katharina Pollaczek, Felix Salten}
\renewcommand{\erwaehnteOrte}{Orte: Café Arkaden, Wien}
\renewcommand{\erwaehnteWerke}{Werke: Tagebuch}
\section[Arthur Schnitzler an Felix Salten, {[}18. 12. 1896?{]}]{Arthur Schnitzler an Felix Salten, {[}18. 12. 1896?{]}}
\nopagebreak\mylabel{v}
\rehead{ }\normalsize\beginnumbering\briefempfaengerindex{Salten, Felix@\textsc{Salten, Felix}!zzzSchnitzler, Arthur@\emph{von Arthur Schnitzler}!1896-12-181@{{[}18. 12. 1896?{]}}|(be}
\toendnotes[C]{\smallbreak\pagebreak[2]}\Standort{Wienbibliothek im Rathaus, ZPH 1681, 2.1.516.}
\physDesc{Brief, 1 Blatt, 3 Seiten, 431 Zeichen
\newline{}Handschrift: Bleistift, deutsche Kurrent
\newline{}Ordnung: mit Bleistift von unbekannter Hand Nummerierung der Doppelseiten des
                                 Konvoluts: »15«–»16« }\toendnotes[C]{\smallbreak}
\pstart
           \noindent{}{\pb}Lieber, ich habe \label{K_L03036-1v}\edtext{\textsc{\textcolor{blue}{Mademoiselle}{}\ledrightnote{{$\rightarrow$}\textcolor{blue}{?? [Anstandsdame von Anna und Clara Loeb]}}} und die \textcolor{blue}{2
                  Mädel}{}\ledrightnote{{$\rightarrow$}\textcolor{blue}{Clara Katharina Pollaczek}{\newline}{$\rightarrow$}\textcolor{blue}{Anna Epstein}}}{\lemma{\textnormal{\emph{Mademoiselle … Mädel}}}\Cendnote{\textnormal{Die Datierung dieses Korrespondenzstücks
                  gelingt durch die Identifizierung der beiden jungen Frauen als die Schwestern \textcolor{blue}{Clara} und \textcolor{blue}{Anna Loeb}. Am 17. 12. 1896 hatten sie auf einer Soirée bei \textcolor{blue}{Marianne Benedict} mit \textcolor{blue}{Schnitzler} geplaudert, für den Folgetag wird im \emph{\textcolor{green}{Tagebuch}} am Nachmittag deren ›\textcolor{blue}{Anstandsdame}‹
                  erwähnt.}}}\label{K_L03036-1h} eine viertel Minute vor Ihnen getroffen –\pend
           
\pstart
           \textsc{\textcolor{blue}{Cl.}{}\ledrightnote{\textcolor{blue}{Clara Katharina Pollaczek}}} fragt mich, warum ich \uline{nicht} telephonirt habe?
               ich: ich ka{\geminationn}{ }heut nicht ko{\geminationm}en\textcolor{gray}{!}{ }\textsc{\textcolor{blue}{Cl}{}\ledrightnote{\textcolor{blue}{Clara Katharina Pollaczek}}}: Schade, {\pb}zu ſprechen, wir ſind allein.
                  \textcolor{blue}{Anna}{}\ledrightnote{\textcolor{blue}{Anna Epstein}}: Sehn Sie S.? Ich: Ich ka{\geminationn} ihm ſchreiben. \uline{\textcolor{blue}{Anna}{}\ledrightnote{\textcolor{blue}{Anna Epstein}}}: Er ſoll beſti{\geminationm}t um ½ 5 zu uns ko{\geminationm}en.\pend
           
\pstart
           – Gehn Sie vielleicht {\pb}auf eine halbe Stunde
               hinauf? –\pend
           
\pstart
           Ja, »\label{K_L03036-11v}\edtext{angfangt iſt leicht}{\lemma{\textnormal{\emph{angfangt iſt leicht}}}\Cendnote{\textnormal{Redewendung: anfangen ist leicht, beharren
                  eine Kunst}}}\label{K_L03036-11h}«!\pend
           
\pstart
           Ich hoff Sie Abends im \textcolor{pink}{Arkaden}{}\ledrightnote{\textcolor{pink}{Café Arkaden}},
               nicht ſpät, zu ſehen. Herzlichſt\pend
           \pstart Ihr \spacefill\mbox{Arth}\pend{}\endnumbering\briefempfaengerindex{Salten, Felix@\textsc{Salten, Felix}!zzzSchnitzler, Arthur@\emph{von Arthur Schnitzler}!1896-12-181@{{[}18. 12. 1896?{]}}|)be}\mylabel{h}  \normalsize

\doendnotes{C}
\bigskip
\vfill

\clearpage

\footnotesize

\lohead{\textsc{register}}

% Definiere theindex-Environment komplett neu ohne reledmac
\makeatletter
\renewenvironment{theindex}{%
  \section*{\indexname}%
  \setlength{\parindent}{0pt}%
  \setlength{\parskip}{0pt plus 0.3pt}%
  \let\item\@idxitem
}{%
  \clearpage
}
\makeatother

\IfFileExists{\jobname-pw.ind}{\input{\jobname-pw.ind}}{}

\end{document}

      