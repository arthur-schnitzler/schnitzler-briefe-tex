%% latex-korrekturansicht-vorspann.tex
%% Vorspann für die Korrekturansicht.
%% Lädt die gemeinsame Datei latex-vorspann.tex mit gesetztem Schalter.

\newif\ifkorrekturansicht
\korrekturansichttrue

\input{../tex-inputs/latex-vorspann}


               \section[Hugo von Hofmannsthal an Arthur Schnitzler, {[}13. 6. 1914{]}]{ Hugo von Hofmannsthal an Arthur Schnitzler, {[}13. 6. 1914{]}}\nopagebreak\mylabel{v}\rehead{ }\normalsize\beginnumbering\briefempfaengerindex{Schnitzler, Arthur@\textsc{Schnitzler, Arthur}!zzzHofmannsthal, Hugo von@\emph{von Hugo von Hofmannsthal}!1914-06-131@{{[}13. 6. 1914{]}}|(be} \toendnotes[C]{\smallbreak\pagebreak[2]} \Standort{CUL, Schnitzler, B 43.}
\physDesc{Brief, 1 Blatt, 4 Seiten
\newline{}Handschrift: schwarze Tinte, deutsche Kurrent
\newline{}Schnitzler: mit Bleistift datiert: »Juni 914« und beschriftet: »\textsc{Hugo}« \newline{}Ordnung: 1) mit Bleistift von unbekannter Hand nummeriert: »\strikeout{337}« 2) mit Bleistift von unbekannter Hand nummeriert:
                                    »350«}\buchAbdrucke{\weitereDrucke{Hugo von Hofmannsthal, Arthur Schnitzler: \emph{Briefwechsel}. Hg. Therese Nickl und Heinrich Schnitzler. Frankfurt am Main: \emph{S. Fischer} 1964, S. 275.} }\toendnotes[C]{\smallbreak}\pstart
           \raggedleft{}{\pb}\textcolor{pink}{Rodaun}{}\ledrightnote{\textcolor{pink}{Rodaun}}, Samstag\pend
           \pstart{}mein lieber Arthur \pend\pstart
           ich höre, Ihr ſeid von Eurer großen \label{K_L02182_1v}\edtext{Reiſe}{\lemma{\textnormal{\emph{Reiſe}}}\Cendnote{\textnormal{Sie waren von 1. 5. 1914 bis zum 7. 6. 1914 unterwegs, die
                  meiste Zeit mit dem Schiff von \textcolor{pink}{Italien} in die
                     \textcolor{pink}{Niederlande}.}}}\label{K_L02182_1h} wohlbehalten zurück, und
               wir haben den herzlichen Wunſch Euch zu ſehen! \pend
           \pstart
           Ich war indeſſen \label{K_L02182_2v}\edtext{in \textcolor{pink}{Paris}{}\ledrightnote{\textcolor{pink}{Paris}}}{\lemma{\textnormal{\emph{in Paris}}}\Cendnote{\textnormal{von 9. 5. 1914 bis zum
                     20. 5. 1914, wobei die Heimkehr erst am
                     30. 5. 1914 stattfand}}}\label{K_L02182_2h}, hatte dort recht trübe
               niedergeſchlagene Tage (von innen heraus, und in ſolchen Zeiten iſt mir eine große
               fremde Stadt nicht günſtig), traf dann meinen \textcolor{blue}{Vater}{}\ledrightnote{→\textcolor{blue}{Hugo August von Hofmannsthal}} in \textcolor{pink}{Frankfurt}{}\ledrightnote{\textcolor{pink}{Frankfurt am Main}},
               brachte ihn nach \textcolor{pink}{Nauheim}{}\ledrightnote{\textcolor{pink}{Bad Nauheim}}, wo die Cur ihm, wie es
               ſcheint, recht wohl tut. – Wie {\pb}könnten wir uns ſehen, Arthur? Wir ſind ſicher noch die ganze Woche da \label{K_L02182_3v}\edtext{bis zum 22\textsuperscript{ten} etwa}{\lemma{\textnormal{\emph{bis zum 22ten etwa}}}\Cendnote{\textnormal{Erst eine Woche danach
                  übersiedelten sie nach \textcolor{pink}{Aussee}.}}}\label{K_L02182_3h}.\hspace*{1.5em}Wir haben aber keine Möglichkeit des Übernachtens mehr
               in der Stadt.\hspace*{1.5em}Wenn Ihr wie neulich die \textcolor{blue}{Bären}{}\ledrightnote{\textcolor{blue}{Richard Beer-Hofmann}{\newline}\textcolor{blue}{Paula Beer-Hofmann}}, zu einem gemeinſamen Nachtmahl nach
                  \textcolor{pink}{Hietzing}{}\ledrightnote{\textcolor{pink}{XIII., Hietzing}} kämet – und etwa ſchon um 7
               oder ſo dort wäret, \textsc{rendezvous}{ }\uline{vor} dem \textcolor{pink}{Parkhôtel}{}\ledrightnote{\textcolor{pink}{Parkhotel Schönbrunn}},
               daſs man {\pb}vorher eine Stunde
               miteinander im \textcolor{pink}{Schönbrunner Park}{}\ledrightnote{\textcolor{pink}{Schloß Schönbrunn}} herumginge oder
               ſäße – das wäre ſehr ſchön. Schreiben Sie eine Zeile, jeder Tag wird uns recht
               ſein.\pend
           \pstart
           Noch eines, da Sie ja \uline{mein} eigentlicher Hausarzt
               ſind. In der (irrigen) Idee von etwas Gicht ließ ich eine Analyſe machen; ſie ergab
               nichts Pathologiſches, nur: Traubenzucker, \uuline{nur} in
               Spuren, {\pb}\uline{quantitativ nicht nachweisbar}. Mein hieſiger \textcolor{blue}{Landarzt}{}\ledrightnote{→\textcolor{blue}{Maximilian Wimmer}}, der recht geſcheidt,
               nur etwas ſummariſch iſt, ſagt, das käme bei vielen Leuten vor, habe gar nichts auf
               ſich, bedeute durchaus nicht einen Anfang oder eine Andeutung dieſer Krankheit.\hspace*{1.5em}Iſt das richtig?\pend
           \pstart
           Von Herzen Ihr{\\[\baselineskip]}\spacefill\mbox{Hugo.}\pend
           \leftskip=0em{}\pstart
           \noindent{}\textsc{PS}. Meine oben gemeldete Niedergeſchlagenheit hat nichts
                  mit Hypochondrien zu tun, die mich durchaus nicht beſchäftigen; obige Analyſe kam
                  mir erst geſtern vor Augen.\pend
           \endnumbering\briefempfaengerindex{Schnitzler, Arthur@\textsc{Schnitzler, Arthur}!zzzHofmannsthal, Hugo von@\emph{von Hugo von Hofmannsthal}!1914-06-131@{{[}13. 6. 1914{]}}|)be}\mylabel{h}  \normalsize

\doendnotes{C}
\bigskip
\vfill

\clearpage

\footnotesize

\lohead{\textsc{register}}

% Definiere theindex-Environment komplett neu ohne reledmac
\makeatletter
\renewenvironment{theindex}{%
  \section*{\indexname}%
  \setlength{\parindent}{0pt}%
  \setlength{\parskip}{0pt plus 0.3pt}%
  \let\item\@idxitem
}{%
  \clearpage
}
\makeatother

\IfFileExists{\jobname-pw.ind}{\input{\jobname-pw.ind}}{}

\end{document}

      