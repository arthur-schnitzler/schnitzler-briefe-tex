%% latex-korrekturansicht-vorspann.tex
%% Vorspann für die Korrekturansicht.
%% Lädt die gemeinsame Datei latex-vorspann.tex mit gesetztem Schalter.

\newif\ifkorrekturansicht
\korrekturansichttrue

\input{../tex-inputs/latex-vorspann}


               \section[Olga Schnitzler an Richard und Paula Beer-Hofmann, {[}15. 12. 1909?{]}]{ Olga Schnitzler an Richard und Paula Beer-Hofmann,
               {[}15. 12. 1909?{]}}\nopagebreak\mylabel{v}\rehead{ }\normalsize\beginnumbering\briefempfaengerindex{Beer-Hofmann, Paula@\textsc{Beer-Hofmann, Paula}!zzzSchnitzler, Olga@\emph{von Olga Schnitzler}!1909-12-151@{{[}15. 12. 1909{]}?}|(be}\briefempfaengerindex{Beer-Hofmann, Richard@\textsc{Beer-Hofmann, Richard}!zzzSchnitzler, Olga@\emph{von Olga Schnitzler}!1909-12-151@{{[}15. 12. 1909{]}?}|(be} \toendnotes[C]{\smallbreak\pagebreak[2]} \Standort{YCGL, MSS 31.}
\physDesc{Brief, 1 Blatt, 1 Seite
\newline{}Handschrift: schwarze Tinte, lateinische Kurrent}\toendnotes[C]{\smallbreak}\pstart
           \noindent{}{\pb}Meine Lieben, \textcolor{blue}{Arth.}{}\ledrightnote{} lässt Euch um
               einen Gefallen bitten: \textcolor{blue}{Andrian}{}\ledrightnote{\textcolor{blue}{Leopold von Andrian-Werburg}} kommt \label{K_L02560-1v}\edtext{heut{ }Abend}{\lemma{\textnormal{\emph{heut Abend}}}\Cendnote{\textnormal{Das Korrespondenzstück ist undatiert.
                  Die Datierung folgt der Annahme, dass es sich um das im \emph{\textcolor{green}{Tagebuch}} vom 15. 12. 1909 erwähnte Treffen handelt. \textcolor{blue}{Beer-Hofmann} wäre demnach nicht gekommen. Da das Korrespondenzstück im
                  Nachlass im Ordner 1909 abgelegt ist, wird es dem Tagebuch folgend
                  datiert. Möglich wäre aber auch der 3. 4. 1910; auch in diesem Fall war \textcolor{blue}{Beer-Hofmann} nicht bei \textcolor{blue}{Schnitzler} zum Abendessen. Es ist mit den derzeit zu
                  überblickenden Korrespondenzstücken nicht möglich, eine definitive Datierung
                  vorzunehmen.}}}\label{K_L02560-1h} zum Nachtmal, wir bitten Euch, auch zu kommen, oder falls die
                  \textcolor{blue}{Paula}{}\ledrightnote{\textcolor{blue}{Paula Beer-Hofmann}} zu müde ist, so bitten wir Sie, lieber
               Herr Doctor, ev. nach dem Nachtmal. Aber am schönsten wär’s, Ihr kämet beide. Herzl.
               Grüsse,\pend
           \pstart \spacefill\mbox{Olga.}\pend{}\endnumbering\briefempfaengerindex{Beer-Hofmann, Paula@\textsc{Beer-Hofmann, Paula}!zzzSchnitzler, Olga@\emph{von Olga Schnitzler}!1909-12-151@{{[}15. 12. 1909{]}?}|)be}\briefempfaengerindex{Beer-Hofmann, Richard@\textsc{Beer-Hofmann, Richard}!zzzSchnitzler, Olga@\emph{von Olga Schnitzler}!1909-12-151@{{[}15. 12. 1909{]}?}|)be}\mylabel{h}  \normalsize

\doendnotes{C}
\bigskip
\vfill

\clearpage

\footnotesize

\lohead{\textsc{register}}

% Definiere theindex-Environment komplett neu ohne reledmac
\makeatletter
\renewenvironment{theindex}{%
  \section*{\indexname}%
  \setlength{\parindent}{0pt}%
  \setlength{\parskip}{0pt plus 0.3pt}%
  \let\item\@idxitem
}{%
  \clearpage
}
\makeatother

\IfFileExists{\jobname-pw.ind}{\input{\jobname-pw.ind}}{}

\end{document}

      