%% latex-korrekturansicht-vorspann.tex
%% Vorspann für die Korrekturansicht.
%% Lädt die gemeinsame Datei latex-vorspann.tex mit gesetztem Schalter.

\newif\ifkorrekturansicht
\korrekturansichttrue

\input{../tex-inputs/latex-vorspann}


               \section[Arthur Schnitzler an Richard Beer-Hofmann, 28. 7. 1922]{ Arthur Schnitzler an Richard Beer-Hofmann, 28. 7. 1922}\nopagebreak\mylabel{v}\rehead{ }\normalsize\beginnumbering\briefempfaengerindex{Beer-Hofmann, Richard@\textsc{Beer-Hofmann, Richard}!zzzSchnitzler, Arthur@\emph{von Arthur Schnitzler}!1922-07-281@{28. 7. 1922}|(be} \toendnotes[C]{\smallbreak\pagebreak[2]} \Standort{DLA, A:Schnitzler, HS.NZ85.1.342, S. 156.}
\physDesc{maschinelle Abschrift}\toendnotes[C]{\smallbreak}\pstart
           \raggedleft{}{\pb}\textcolor{pink}{Feldafing}{}\ledrightnote{\textcolor{pink}{Feldafing}}, 28. 7. 1922.\pend
           \pstart
           \raggedleft{}\textcolor{pink}{Kaiserin Elisabeth}{}\ledrightnote{\textcolor{pink}{Hotel Kaiserin Elisabeth}}\pend
           \pstart
           Lieber Richard – leider konnte ich Sie nicht vor meiner \label{K_L02554-2v}\edtext{Abreise}{\lemma{\textnormal{\emph{Abreise}}}\Cendnote{\textnormal{siehe A. S.: \emph{Tagebuch}, 25. 7. 1922}}}\label{K_L02554-2h} noch einmal sehn. Hier \label{K_L02554-1v}\edtext{sitze
               ich zwar unbesorgt}{\lemma{\textnormal{\emph{sitze
               ich zwar unbesorgt}}}\Cendnote{\textnormal{siehe Arthur Schnitzler an Richard Beer-Hofmann, 27. 8. 1895, Arthur Schnitzler an Richard Beer-Hofmann, 5. 8. 1912}}}\label{K_L02554-1h} (soweit es das gibt), aber in Regen und Nebeln. Doch ist es ein höchst
               behagliches \textcolor{pink}{Hotel}{}\ledrightnote{→\textcolor{pink}{Hotel Kaiserin Elisabeth}} – und
               keineswegs sehnt man sich in die verhängten Berge. Seien Sie und die Ihren herzlichst
               gegrüsst, eine glückliche Reise! Ihr \spacefill\mbox{A.}\pend
           \pstart
           \noindent{}(nach \textcolor{pink}{Wien}{}\ledrightnote{\textcolor{pink}{Wien}})\pend
           \endnumbering\briefempfaengerindex{Beer-Hofmann, Richard@\textsc{Beer-Hofmann, Richard}!zzzSchnitzler, Arthur@\emph{von Arthur Schnitzler}!1922-07-281@{28. 7. 1922}|)be}\mylabel{h}  \normalsize

\doendnotes{C}
\bigskip
\vfill

\clearpage

\footnotesize

\lohead{\textsc{register}}

% Definiere theindex-Environment komplett neu ohne reledmac
\makeatletter
\renewenvironment{theindex}{%
  \section*{\indexname}%
  \setlength{\parindent}{0pt}%
  \setlength{\parskip}{0pt plus 0.3pt}%
  \let\item\@idxitem
}{%
  \clearpage
}
\makeatother

\IfFileExists{\jobname-pw.ind}{\input{\jobname-pw.ind}}{}

\end{document}

      