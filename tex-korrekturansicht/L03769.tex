%% latex-korrekturansicht-vorspann.tex
%% Vorspann für die Korrekturansicht.
%% Lädt die gemeinsame Datei latex-vorspann.tex mit gesetztem Schalter.

\newif\ifkorrekturansicht
\korrekturansichttrue

\input{../tex-inputs/latex-vorspann}


\section[Arthur Schnitzler an Stefan Zweig, 5.  5. 1915]{L03769 Arthur Schnitzler an Stefan Zweig, 5.  5. 1915}
\nopagebreak\mylabel{L03769v}
\rehead{ }\normalsize\beginnumbering\briefempfaengerindex{, @\textsc{, }!zzz, @\emph{von  }!1915-05-051@{5.  5. 1915}|(be}
\toendnotes[C]{\smallbreak\pagebreak[2]}\Standort{Jerusalem, National Library of Israel, ARC. Ms. Var. 305 1 58 Stefan Zweig Collection.}
\physDesc{Briefkarte, 304 Zeichen
\newline{}Schreibmaschine
\newline{}Handschrift: schwarze Tinte, deutsche Kurrent (\noindent{}Unterschrift)}\toendnotes[C]{\smallbreak}
\pstart
           {\pb}\textcolor{gray}{\textbf{Dr. Arthur Schnitzler}}\hfill 5. 5. 1915. \pend
           
\pstart
           \textcolor{gray}{\textbf{\textcolor{pink}{Wien XVIII.
                        Sternwartestrasse 71}\oindex{Wien@\textbf{Wien}!XVIII., Währing@\textbf{XVIII., Währing}!Sternwartestraße 71@\textbf{Sternwartestraße 71}, \emph{Wohngebäude}|pw}{}\ledrightnote{\textcolor{pink}{Sternwartestraße 71}}}}\pend
           
\pstart\center{}Lieber Herr Doktor!\pend\vspace{0.5em}
\pstart
           Beigeschlossen der \label{K_L03769-1v}\edtext{\textcolor{green}{Brief}\pwindex{Schnitzler, Arthur 15. 5. 1862 Wien – 21. 10. 1931 ebd.@\textsc{Schnitzler, Arthur} (15. 5. 1862 Wien – 21. 10. 1931 ebd.), \emph{Schriftsteller, Mediziner}!Rücktritt des Burgtheatersekretärs Dr. Rosenbaum@\strich\emph{Der Rücktritt des Burgtheatersekretärs Dr. Rosenbaum}|pwv}\pwindex{Hauptmann, Gerhart 15.\,11.\,1862 Szczawno-Zdrój – 6.\,6.\,1946 Jagniątków@\textsc{Hauptmann, Gerhart} (15.\,11.\,1862 Szczawno-Zdrój – 6.\,6.\,1946 Jagniątków), \emph{Schriftsteller}!Rücktritt des Burgtheatersekretärs Dr. Rosenbaum@\strich\emph{Der Rücktritt des Burgtheatersekretärs Dr. Rosenbaum}|pwv}{}\ledrightnote{{$\rightarrow$}\emph{\textcolor{green}{Der Rücktritt des Burgtheatersekretärs Dr. Rosenbaum}}} an Dr. \textcolor{blue}{Rosenbaum}\pwindex{Rosenbaum, Richard 4.\,11.\,1867 Žikov – 25.\,6.\,1942 Konzentrationslager Theresienstadt@\textsc{Rosenbaum, Richard} (4.\,11.\,1867 Žikov – 25.\,6.\,1942 Konzentrationslager Theresienstadt), \emph{Dramaturg, Verleger}|pw}{}\ledrightnote{\textcolor{blue}{Richard Rosenbaum}} für die \textcolor{green}{Neue freie
                  Presse}\pwindex{Neue Freie Presse@\emph{Neue Freie Presse}|pw}{}\ledrightnote{\textcolor{green}{Neue Freie Presse}}}{\lemma{\textnormal{\emph{Brief … Presse}}}\Cendnote{\textnormal{Siehe A. S.: \emph{»Das Zeitlose ist von kürzester Dauer«}, Der Rücktritt des Burgtheatersekretärs Dr. Rosenbaum, 16. 5. 1915.}}}\label{K_L03769-1}. Da Sie so
               gütig sind ihn selbst dorthin zu geben haben Sie vielleicht auch die Freundlichkeit
               mir eine Korrektur zu verschaffen, die ich in diesem Fall besonders gern lesen
               möchte.\pend
           
\pstart
           Herzlichst{\\[\baselineskip]}Ihr{\\[\baselineskip]}\spacefill\mbox{{[}hs.:{]} Arthur Schnitzler}\pend
           \leftskip=0em{}\selectlanguage{ngerman}\endnumbering\briefempfaengerindex{, @\textsc{, }!zzz, @\emph{von  }!1915-05-051@{5.  5. 1915}|)be}\mylabel{L03769h}
\begin{anhang}
\end{anhang}\normalsize

\doendnotes{C}
\bigskip
\vfill

\clearpage

\footnotesize

\lohead{\textsc{register}}

% Definiere theindex-Environment komplett neu ohne reledmac
\makeatletter
\renewenvironment{theindex}{%
  \section*{\indexname}%
  \setlength{\parindent}{0pt}%
  \setlength{\parskip}{0pt plus 0.3pt}%
  \let\item\@idxitem
}{%
  \clearpage
}
\makeatother

\IfFileExists{\jobname-pw.ind}{\input{\jobname-pw.ind}}{}

\end{document}

      