%% latex-korrekturansicht-vorspann.tex
%% Vorspann für die Korrekturansicht.
%% Lädt die gemeinsame Datei latex-vorspann.tex mit gesetztem Schalter.

\newif\ifkorrekturansicht
\korrekturansichttrue

\input{../tex-inputs/latex-vorspann}


\renewcommand{\erwaehntePersonen}{Personen: Olga Schnitzler, Elisabeth Steinrück}
\renewcommand{\erwaehnteOrte}{Orte: Dolomiten, Lago di Garda, Pörtschach, Vahrn, Valle d’Ampezzo, Wien}
\renewcommand{\erwaehnteWerke}{}
\section[ Paul Goldmann an Arthur Schnitzler, 1. 8. {[}1901{]}]{Paul Goldmann an Arthur Schnitzler, 1. 8. {[}1901{]}}
\nopagebreak\mylabel{v}
\rehead{ }\normalsize\beginnumbering\briefempfaengerindex{Schnitzler, Arthur@\textsc{Schnitzler, Arthur}!zzzGoldmann, Paul@\emph{von Paul Goldmann}!1901-08-012@{1. 8. {[}1901{]}}|(be}
\toendnotes[C]{\smallbreak\pagebreak[2]}\Standort{DLA, A:Schnitzler, HS.NZ85.1.3171.}
\physDesc{Brief, 1 Blatt, 2 Seiten
\newline{}Handschrift: blaue Tinte, deutsche Kurrent
\newline{}Schnitzler: mit Bleistift das Jahr »{[}1{]}901« vermerkt }\toendnotes[C]{\smallbreak}
\pstart
           \raggedleft{}{\pb}\textsc{\textcolor{pink}{Pörtschach}{}\ledrightnote{\textcolor{pink}{Pörtschach}}}, 1. Auguſt.\pend
           
\pstart\center{}Mein lieber Freund,\pend
\pstart
           Dank für Deinen lieben Brief.\pend
           
\pstart
           Ich muß fort von hier, denn ich kann nicht ſchlafen. Die warme, matte Luft bekommt
               mir ſchlecht. In \label{K_L03076-1v}\edtext{\textsc{\textcolor{pink}{Vahrn}{}\ledrightnote{\textcolor{pink}{Vahrn}}}}{\lemma{\textnormal{\emph{Vahrn}}}\Cendnote{\textnormal{\textcolor{blue}{Schnitzler} dürfte vorgeschlagen haben, dass
                     \textcolor{blue}{Goldmann} nach \textcolor{pink}{Vahrn} komme, wo er sich seit 13. 7. 1901 und noch bis 12. 8. 1901
                  aufhielt.}}}\label{K_L03076-1h} wäre es dieſelbe Geſchichte. Ich muß höher hinauf, in ſtarke und
               kühle Luft. \textcolor{blue}{Euch}{}\ledrightnote{{$\rightarrow$}\textcolor{blue}{Olga Schnitzler}}
               wiederzuſehen wäre ſchön. Aber Wochen lang keine Nacht ſchlafen, iſt kein Spaß. Da Du
               alſo noch nichts Hohes gefunden haſt, muß ich ſelbſt ſuchen. Ich gehe von hier in die
                  \textcolor{pink}{Dolomiten}{}\ledrightnote{\textcolor{pink}{Dolomiten}}. Werde das \textcolor{pink}{\textsc{Ampezzo}-Thal}{}\ledrightnote{\textcolor{pink}{Valle d’Ampezzo}} durchprobiren. Wo ich ſchlafen kann,
               bleibe ich ein paar Tage. Es wird ſich alſo leider ſo fügen, {\pb}daß ich erſt den Schluß meines
               Urlaubs mit \textcolor{blue}{Euch}{}\ledrightnote{{$\rightarrow$}\textcolor{blue}{Olga Schnitzler}} verbringen
               kann, wenn Ihr in \textsc{\textcolor{pink}{Vahrn}{}\ledrightnote{\textcolor{pink}{Vahrn}}} bleibt. \strikeout{Ende} Ende Auguſt muß ich in \textcolor{pink}{Wien}{}\ledrightnote{\textcolor{pink}{Wien}} ſein. Samſtag{ }früh fahre ich von hier ab. Da ich nicht weiß, wo ich bleiben werde,
               kann ich Dir noch keine Adreſſe \strikeout{g} geben. Aber das muß
               ſich Sonntag oder Montag entſcheiden. Ich ſchreibe Dir dann ſofort. Laß’ alſo das Suchen
               ſein. Da Du Dich in \textsc{\textcolor{pink}{Vahrn}{}\ledrightnote{\textcolor{pink}{Vahrn}}} wohl fühlſt, bleibe dort. Wenn ich meine Nerven zur Raiſon gebracht haben
               werde, komme ich zu \textcolor{blue}{Euch}{}\ledrightnote{{$\rightarrow$}\textcolor{blue}{Olga Schnitzler}}, –
               dorthin oder an den \textcolor{pink}{Gardaſee}{}\ledrightnote{\textcolor{pink}{Lago di Garda}}. Einſtweilen geht
               es mir recht elend. Es iſt eine ganz verfluchte Geſchichte, wenn man nicht ſchläft.
               Viele treue Grüße Dir und den lieben \textcolor{blue}{Mädchen}{}\ledrightnote{{$\rightarrow$}\textcolor{blue}{Olga Schnitzler}{\newline}{$\rightarrow$}\textcolor{blue}{Elisabeth Steinrück}}!\pend
           \pstart Dein \spacefill\mbox{Paul Goldmann}\pend{}\endnumbering\briefempfaengerindex{Schnitzler, Arthur@\textsc{Schnitzler, Arthur}!zzzGoldmann, Paul@\emph{von Paul Goldmann}!1901-08-012@{1. 8. {[}1901{]}}|)be}\mylabel{h}  \normalsize

\doendnotes{C}
\bigskip
\vfill

\clearpage

\footnotesize

\lohead{\textsc{register}}

% Definiere theindex-Environment komplett neu ohne reledmac
\makeatletter
\renewenvironment{theindex}{%
  \section*{\indexname}%
  \setlength{\parindent}{0pt}%
  \setlength{\parskip}{0pt plus 0.3pt}%
  \let\item\@idxitem
}{%
  \clearpage
}
\makeatother

\IfFileExists{\jobname-pw.ind}{\input{\jobname-pw.ind}}{}

\end{document}

      