%% latex-korrekturansicht-vorspann.tex
%% Vorspann für die Korrekturansicht.
%% Lädt die gemeinsame Datei latex-vorspann.tex mit gesetztem Schalter.

\newif\ifkorrekturansicht
\korrekturansichttrue

\input{../tex-inputs/latex-vorspann}


               \section[Arthur Schnitzler an Richard Beer-Hofmann, {[}18. 11. 1898?{]}]{ Arthur Schnitzler an Richard Beer-Hofmann, {[}18. 11. 1898?{]}}\nopagebreak\mylabel{v}\rehead{ }\normalsize\beginnumbering\briefempfaengerindex{Beer-Hofmann, Richard@\textsc{Beer-Hofmann, Richard}!zzzSchnitzler, Arthur@\emph{von Arthur Schnitzler}!1898-11-181@{{[}18. 11. 1898?{]}}|(be} \toendnotes[C]{\smallbreak\pagebreak[2]} \Standort{YCGL, MSS 31.}
\physDesc{Brief, Umschlag
\newline{}Handschrift: Bleistift, deutsche Kurrent\newline{}Versand: ohne postalischen Übermittlungsvermerk }\toendnotes[C]{\smallbreak}\pstart{}{\pb}\textsc{Herrn Dr. Richard Beer-Hofmann }\pend{}\pstart{}\textcolor{pink}{Wien}{}\ledrightnote{\textcolor{pink}{Wien}}\pend{}\pstart{}\textsc{\textcolor{pink}{I. Wollzeile 15}{}\ledrightnote{\textcolor{pink}{Wollzeile}}}.\pend{}{\bigskip}\pstart
           \noindent{}{\pb}Lieber Richard, Sie erwieſen mir einen Gefallen, we{\geminationn}{ }Sie heut mit mir auf dieſen Sitz im \label{K_L00858_1v}\edtext{2. Stock}{\lemma{\textnormal{\emph{2. Stock}}}\Cendnote{\textnormal{Sofern die archivalische Überlieferung, die dieses undatierte
                  Korrespondenzstück in der Mappe für das Jahr 1898 überliefert,
                  verlässlich ist, ergibt sich mit dem \emph{\textcolor{green}{Tagebuch}}
                  eine mögliche genauere Bestimmung. In diesem Jahr besuchte \textcolor{blue}{Schnitzler} viermal Aufführungen im \textcolor{pink}{Raimundtheater}. Nur an einem Abend, bei der \emph{\textcolor{green}{Juana}} von \textcolor{blue}{Hermann Bahr} und
                  sein eigenes \emph{\textcolor{green}{Abschiedssouper}} gemeinsam gegeben
                  wurden, lässt sich die Anwesenheit von \textcolor{blue}{Beer-Hofmann} belegen, siehe A. S.: \emph{Tagebuch}, 18. 11. 1898.}}}\label{K_L00858_1h}, \textcolor{pink}{Rmdtheater}{}\ledrightnote{\textcolor{pink}{Raimund-Theater}} kämen. We{\geminationn}{ }Sie nicht wollen, ſenden Sie mir ihn raſch zurück,
               bitte.\pend
           \pstart Herzlichſt Ihr \spacefill\mbox{Arthur}\pend{}\endnumbering\briefempfaengerindex{Beer-Hofmann, Richard@\textsc{Beer-Hofmann, Richard}!zzzSchnitzler, Arthur@\emph{von Arthur Schnitzler}!1898-11-181@{{[}18. 11. 1898?{]}}|)be}\mylabel{h}  \normalsize

\doendnotes{C}
\bigskip
\vfill

\clearpage

\footnotesize

\lohead{\textsc{register}}

% Definiere theindex-Environment komplett neu ohne reledmac
\makeatletter
\renewenvironment{theindex}{%
  \section*{\indexname}%
  \setlength{\parindent}{0pt}%
  \setlength{\parskip}{0pt plus 0.3pt}%
  \let\item\@idxitem
}{%
  \clearpage
}
\makeatother

\IfFileExists{\jobname-pw.ind}{\input{\jobname-pw.ind}}{}

\end{document}

      