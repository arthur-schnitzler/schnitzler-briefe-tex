%% latex-korrekturansicht-vorspann.tex
%% Vorspann für die Korrekturansicht.
%% Lädt die gemeinsame Datei latex-vorspann.tex mit gesetztem Schalter.

\newif\ifkorrekturansicht
\korrekturansichttrue

\input{../tex-inputs/latex-vorspann}


               \section[Hermann Bahr an Arthur Schnitzler, 21. 2. 1892]{ Hermann Bahr an Arthur Schnitzler, 21. 2. 1892}\nopagebreak\mylabel{v}\rehead{ }\normalsize\beginnumbering\briefempfaengerindex{Schnitzler, Arthur@\textsc{Schnitzler, Arthur}!zzzBahr, Hermann@\emph{von Hermann Bahr}!1892-02-211@{21. 2. 1892}|(be} \toendnotes[C]{\smallbreak\pagebreak[2]} \Standort{CUL, Schnitzler, B 5b.}
\physDesc{Postkarte
\newline{}Handschrift: schwarze Tinte, deutsche Kurrent\newline{}Versand: Stempel: »\nobreak{}\oindex{I., Innere Stadt@\textbf{I., Innere Stadt}, \emph{Bezirk (A.BZK)}|pwk}Wien 1/1, 22{[}.{]} 2. 92, 8–9 V\nobreak{}«.  
\newline{}Schnitzler: mit Bleistift datiert: »22/2 92« \newline{}Ordnung: 1) mit rotem Buntstift von unbekannter Hand nummeriert:
                                       »\strikeout{6}« 2) mit Bleistift von unbekannter Hand nummeriert:
                                 »5«}\buchAbdrucke{\weitereDrucke{Hermann Bahr, Arthur Schnitzler: \emph{Briefwechsel, Aufzeichnungen, Dokumente (1891–1931)}. Hg. Kurt Ifkovits und Martin Anton Müller. Göttingen: \emph{Wallstein} 2018, S. 22.} }\toendnotes[C]{\smallbreak}\pstart{}{\pb}Herrn D\textsuperscript{r} A.
                  Schnitzler \pend{}\pstart{}\textcolor{pink}{Kärntnerring 12}{}\ledrightnote{\textcolor{pink}{Kärntnerring}}\pend{}\pstart{}\textcolor{pink}{Wien I}{}\ledrightnote{\textcolor{pink}{I., Innere Stadt}}\pend{}{\bigskip}\pstart
           \raggedleft{}\label{K_L00074_1v}\edtext{{\pb}So{\geminationn}tag}{\lemma{\textnormal{\emph{Sotag}}}\Cendnote{\textnormal{geschrieben am 
                     Sonntag, den 21. 2. 1892, aber erst am Folgetag abgeschickt}}}\label{K_L00074_1h}
                  Mittag.\pend
           \pstart{} Lieber Freund!\pend\pstart
           Das \textcolor{blue}{\label{K_L00074_2v}\edtext{Mauſerl}{\lemma{\textnormal{\emph{Mauſerl}}}\Cendnote{\textnormal{\textcolor{blue}{Ilka Pálmay}}}}\label{K_L00074_2h}}{}\ledrightnote{→\textcolor{blue}{Ilka Pálmay}} will nicht, abſolut nicht. Alles mögliche Schöne u Gute könnte man von ihr
               haben – nur gerade das eine nicht, was wir brauchen. Sie ſagt übrigens ſehr
               vernünftige Gründe u. i{\geminationn}erlich muß ich ihr Recht
               geben.\pend
           \pstart herzlichſt\spacefill\mbox{Bahr}\pend{}\endnumbering\briefempfaengerindex{Schnitzler, Arthur@\textsc{Schnitzler, Arthur}!zzzBahr, Hermann@\emph{von Hermann Bahr}!1892-02-211@{21. 2. 1892}|)be}\mylabel{h}  \normalsize

\doendnotes{C}
\bigskip
\vfill

\clearpage

\footnotesize

\lohead{\textsc{register}}

% Definiere theindex-Environment komplett neu ohne reledmac
\makeatletter
\renewenvironment{theindex}{%
  \section*{\indexname}%
  \setlength{\parindent}{0pt}%
  \setlength{\parskip}{0pt plus 0.3pt}%
  \let\item\@idxitem
}{%
  \clearpage
}
\makeatother

\IfFileExists{\jobname-pw.ind}{\input{\jobname-pw.ind}}{}

\end{document}

      