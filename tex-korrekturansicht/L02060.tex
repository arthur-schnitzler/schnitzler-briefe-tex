%% latex-korrekturansicht-vorspann.tex
%% Vorspann für die Korrekturansicht.
%% Lädt die gemeinsame Datei latex-vorspann.tex mit gesetztem Schalter.

\newif\ifkorrekturansicht
\korrekturansichttrue

\input{../tex-inputs/latex-vorspann}


               \section[Hermann Bahr an Olga Schnitzler, 27. 4. 1912]{ Hermann Bahr an Olga Schnitzler, 27. 4. 1912}\nopagebreak\mylabel{v}\rehead{ }\normalsize\beginnumbering\briefempfaengerindex{Schnitzler, Olga@\textsc{Schnitzler, Olga}!zzzBahr, Hermann@\emph{von Hermann Bahr}!1912-04-271@{27. 4. 1912}|(be} \toendnotes[C]{\smallbreak\pagebreak[2]} \Standort{CUL, Schnitzler, B 5b.}
\physDesc{Brief, 1 Blatt, 2 Seiten
\newline{}Handschrift: schwarze Tinte, deutsche Kurrent
\newline{}Schnitzler: mit Bleistift ergänzt »\textsc{Bahr}« \newline{}Ordnung: mit Bleistift von unbekannter Hand nummeriert:
                                                »172« }\buchAbdrucke{\weitereDrucke{Hermann Bahr, Arthur Schnitzler: \emph{Briefwechsel, Aufzeichnungen, Dokumente
                                (1891–1931)}. Hg. Kurt Ifkovits und Martin Anton Müller. Göttingen: \emph{Wallstein} 2018, S. 470.} }\toendnotes[C]{\smallbreak}\pstart
           \raggedleft{}{\pb}27. 4. 12\pend
           \pstart\center{}Sehr verehrte liebe gnädige Frau!\pend\pstart
           Meine \textcolor{blue}{Frau}{}\ledrightnote{→\textcolor{blue}{Anna Bahr-Mildenburg}} dankt Ihnen
                    herzlichſt für Ihre liebe Einladung, der ſie ſo gern folgen würde, wenns nur
                    irgend ging! Es geht aber leider nicht, weil ſie gerade jetzt von den
                    ſämmtlichen Freundinnen oder Bekannten, die ſie ſich in den \strikeout{zwölf}{ }\label{K_L02060_1v}\edtext{vierzehn \textcolor{pink}{Wiener}{}\ledrightnote{\textcolor{pink}{Wien}} Jahren}{\lemma{\textnormal{\emph{vierzehn Wiener Jahren}}}\Cendnote{\textnormal{Am
                            1. 6. 1898 wurde sie Ensemblemitglied der \emph{\textcolor{brown}{Wiener Hofoper}}.}}}\label{K_L02060_1h} angesammelt hat, dringend
                    aufgefordert wird, ſie müßte nun bevor wir \textcolor{pink}{Wien}{}\ledrightnote{\textcolor{pink}{Wien}}
                    verlaſſen, noch einmal zu ihnen kommen; ſie hätte alſo vierzehn Tage rein mit
                    Beſuchen zuzubringen, da ſagt ſie lieber allen Nein. Nun können Sie ſich aber
                    vorſtellen, wie eiferſüchtig {\pb}dieſe
                    ſämmtlichen Freundinnen darüber wachen, daß ſie wenigſtens auch bei den anderen
                    nicht erſcheint, und Sie können ſich den Lärm vorſtellen, w\substVorne{}\textsuperscript{ie}\substDazwischen{}enn\substHinten{}{ }\textcolor{blue}{ſie}{}\ledrightnote{→\textcolor{blue}{Anna Bahr-Mildenburg}} auch nur eine einzige
                    Ausnahme machte. Da Sie ja ſelbſt ſo glücklich ſind, weiblichen Geſchlechts zu
                    ſein, werden Sie ja dieſe femininen Feinheiten beſſer zu würdigen verſtehen als
                    ich ſelbſt und ſich Donnerstag mit mir begnügen, der ſich unendlich
                    freut, mit Ihnen \textcolor{blue}{beiden}{}\ledrightnote{→}
                    zuſammen zu ſein.\pend
           \pstart
           Mit den ſchönsten Grüßen von Haus zu Haus{\\[\baselineskip]}immer Ihr alternder{\\[\baselineskip]}\spacefill\mbox{HermannBahr}\pend
           \leftskip=0em{}\endnumbering\briefempfaengerindex{Schnitzler, Olga@\textsc{Schnitzler, Olga}!zzzBahr, Hermann@\emph{von Hermann Bahr}!1912-04-271@{27. 4. 1912}|)be}\mylabel{h}  \normalsize

\doendnotes{C}
\bigskip
\vfill

\clearpage

\footnotesize

\lohead{\textsc{register}}

% Definiere theindex-Environment komplett neu ohne reledmac
\makeatletter
\renewenvironment{theindex}{%
  \section*{\indexname}%
  \setlength{\parindent}{0pt}%
  \setlength{\parskip}{0pt plus 0.3pt}%
  \let\item\@idxitem
}{%
  \clearpage
}
\makeatother

\IfFileExists{\jobname-pw.ind}{\input{\jobname-pw.ind}}{}

\end{document}

      