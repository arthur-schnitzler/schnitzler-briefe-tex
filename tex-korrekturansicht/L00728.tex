%% latex-korrekturansicht-vorspann.tex
%% Vorspann für die Korrekturansicht.
%% Lädt die gemeinsame Datei latex-vorspann.tex mit gesetztem Schalter.

\newif\ifkorrekturansicht
\korrekturansichttrue

\input{../tex-inputs/latex-vorspann}


               \section[Hugo von Hofmannsthal an Arthur Schnitzler, 3. 10. {[}1897{]}]{ Hugo von Hofmannsthal an Arthur Schnitzler, 3. 10. {[}1897{]}}\nopagebreak\mylabel{v}\rehead{ }\normalsize\beginnumbering\briefempfaengerindex{Schnitzler, Arthur@\textsc{Schnitzler, Arthur}!zzzHofmannsthal, Hugo von@\emph{von Hugo von Hofmannsthal}!1897-10-031@{3. 10. {[}1897{]}}|(be} \toendnotes[C]{\smallbreak\pagebreak[2]} \Standort{CUL, Schnitzler, B 43.}
\physDesc{Brief, 1 Blatt, 3 Seiten
\newline{}Handschrift: schwarze Tinte, deutsche Kurrent
\newline{}Schnitzler: mit Bleistift die Jahreszahl ergänzt: »97« \newline{}Ordnung: 1) mit Bleistift von unbekannter Hand nummeriert: »\strikeout{103}« 2) mit Bleistift von unbekannter Hand nummeriert:
                                    »96«}\buchAbdrucke{\weitereDrucke{Hugo von Hofmannsthal, Arthur Schnitzler: \emph{Briefwechsel}. Hg. Therese Nickl und Heinrich Schnitzler. Frankfurt am Main: \emph{S. Fischer} 1964, S. 95–96.} }\toendnotes[C]{\smallbreak}\pstart
           \raggedleft{}{\pb}\textcolor{pink}{Hinterbrühl}{}\ledrightnote{\textcolor{pink}{Hinterbrühl}}{ }3\textsuperscript{ten} X\textsuperscript{ten}.\pend
           \pstart{}mein lieber Arthur\pend\pstart
           Ihr Geſicht iſt mir \label{K_L00728_1v}\edtext{neulich}{\lemma{\textnormal{\emph{neulich}}}\Cendnote{\textnormal{Am 26., 28. und
                     30. 9. 1897 besuchte \textcolor{blue}{Schnitzler}
                  das Gastspiel von \textcolor{blue}{Ermete Zacconi} im \textcolor{pink}{Carl-Theater}. \textcolor{blue}{Hofmannsthal} hielt sich am Land auf, konnte aber in die Stadt reisen und
                  war aber nachweislich in der Vorstellung des \emph{\textcolor{green}{König
                     Lear}} am letzten der genannten Tage (Brief an die \textcolor{blue}{Eltern}).}}}\label{K_L00728_1h} ſchon von der Loge aus ſehr ernſt und
               traurig erſchienen, ich bin dann zu \textcolor{blue}{Richard}{}\ledrightnote{\textcolor{blue}{Richard Beer-Hofmann}}
               gegangen, er hat mir alles \label{K_L00728_2v}\edtext{erzählt}{\lemma{\textnormal{\emph{erzählt}}}\Cendnote{\textnormal{\textcolor{blue}{Marie Reinhard} und er betrauerten gerade ein am
                     24. 9. 1897 totgeborenes Kind.}}}\label{K_L00728_2h} und deshalb habe ich Ihnen
               unter den vielen fremden Leuten nur die Hand gegeben und nichts geſagt. Ich weiß
               Ihnen {\pb}nichts tröſtliches zu ſagen
               und ob Ihnen meine Zuneigung und Anhänglichkeit irgend eine wirkliche Freude macht,
               weiß ich nicht, deshalb will ich auch nicht davon ſprechen. Ich hoffe von Herzen,
               daſs Sie bald wieder oder ſchon wieder arbeiten können. Ich werde {\pb}wohl die nächſte Woche nach \textcolor{pink}{Wien}{}\ledrightnote{\textcolor{pink}{Wien}} kommen und hätte Ihnen und dem \textcolor{blue}{Richard}{}\ledrightnote{\textcolor{blue}{Richard Beer-Hofmann}}, wenn Sie beide aufgelegt ſind, recht viel
               vorzuleſen.\pend
           \pstart
           Herzlich{\\[\baselineskip]}Ihr{\\[\baselineskip]}\spacefill\mbox{Hugo.}\pend
           \leftskip=0em{}\endnumbering\briefempfaengerindex{Schnitzler, Arthur@\textsc{Schnitzler, Arthur}!zzzHofmannsthal, Hugo von@\emph{von Hugo von Hofmannsthal}!1897-10-031@{3. 10. {[}1897{]}}|)be}\mylabel{h}  \normalsize

\doendnotes{C}
\bigskip
\vfill

\clearpage

\footnotesize

\lohead{\textsc{register}}

% Definiere theindex-Environment komplett neu ohne reledmac
\makeatletter
\renewenvironment{theindex}{%
  \section*{\indexname}%
  \setlength{\parindent}{0pt}%
  \setlength{\parskip}{0pt plus 0.3pt}%
  \let\item\@idxitem
}{%
  \clearpage
}
\makeatother

\IfFileExists{\jobname-pw.ind}{\input{\jobname-pw.ind}}{}

\end{document}

      