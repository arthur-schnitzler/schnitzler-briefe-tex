%% latex-korrekturansicht-vorspann.tex
%% Vorspann für die Korrekturansicht.
%% Lädt die gemeinsame Datei latex-vorspann.tex mit gesetztem Schalter.

\newif\ifkorrekturansicht
\korrekturansichttrue

\input{../tex-inputs/latex-vorspann}


\renewcommand{\erwaehntePersonen}{Personen: Stefan Zweig}
\renewcommand{\erwaehnteOrte}{Orte: Kochgasse 8, Salzburg, Wien}
\renewcommand{\erwaehnteWerke}{}
\section[Stefan Zweig an Arthur Schnitzler, 10. 11. 1926]{Stefan Zweig an Arthur Schnitzler, 10. 11. 1926}
\nopagebreak\mylabel{v}
\rehead{ }\normalsize\beginnumbering\briefempfaengerindex{Schnitzler, Arthur@\textsc{Schnitzler, Arthur}!zzzZweig, Stefan@\emph{von Stefan Zweig}!1926-11-101@{10. 11. 1926}|(be}
\toendnotes[C]{\smallbreak\pagebreak[2]}\Standort{CUL, Schnitzler, B 118.}
\physDesc{Briefkarte, 1 Blatt, 2 Seiten, 980 Zeichen
\newline{}Handschrift: blaue Tinte, lateinische Kurrent
\newline{}Schnitzler: mit Bleistift »\textsc{Zweig}« }
\pstart
           {\pb}\textcolor{gray}{\textbf{SZ}}\hfill \substVorne{}\textsuperscript{\textcolor{gray}{\textbf{\textcolor{pink}{VIII.
                              KOCHGASSE 8}{}\ledrightnote{\textcolor{pink}{Kochgasse 8}}}}}{\allowbreak}\substDazwischen{}\textcolor{pink}{Salzburg}{}\ledrightnote{\textcolor{pink}{Salzburg}}{ }10. Nov 26\substHinten{}\pend
           
\pstart
           \raggedleft{}\textcolor{gray}{\textbf{\textcolor{pink}{WIEN}{}\ledrightnote{\textcolor{pink}{Wien}},}}\pend
           
\pstart
           Lieber verehrter Herr Doktor, gewisse Beziehungen vermag die Zeit
               nicht zu ändern – ich bin jetzt, Gottseisgeklagt, 45 Jahre alt, aber dennoch, wenn
               Sie zu mir sprechen, bin ich noch immer der schüchterne flaumbärtige Bursch, der
               rückwärts ins Parterre gedrückt \strikeout{auf} zu dem berühmten
               Dichter auf der Bühne \introOben{}empor\introOben{}sah. Ein zustimmendes Wort von
               Ihnen macht mich noch genau so beglückt und all die Freundschaft, die stotz gefühlte
                  {\pb}Sicherheit Ihrer Neigung kann nichts
               ändern an diesem dankbaren Aufblick. Und eigentlich möchte ich’s nicht anders. Fast
               alle, zu denen ich einst aufgeblickt, haben mich enttäuscht durch ihr Werk oder durch
               ihre menschliche Haltung – darum bin ich so froh, dass sich gerade an Ihnen meine
               Stellung, meine wirklich aufschauende, niemals änderte und niemals ändern wird. Ich
               liebe Sie sehr und bin froh, dass Sie es wissen: vielleicht kann ich das alles einmal
               besser ausdrücken als gerade Blick in Blick.\pend
           
\pstart
           Dankbarst, treulichst Ihr{\\[\baselineskip]}\spacefill\mbox{Stefan Zweig}\pend
           \leftskip=0em{}\endnumbering\briefempfaengerindex{Schnitzler, Arthur@\textsc{Schnitzler, Arthur}!zzzZweig, Stefan@\emph{von Stefan Zweig}!1926-11-101@{10. 11. 1926}|)be}\mylabel{h}
\begin{anhang}
\end{anhang}\normalsize

\doendnotes{C}
\bigskip
\vfill

\clearpage

\footnotesize

\lohead{\textsc{register}}

% Definiere theindex-Environment komplett neu ohne reledmac
\makeatletter
\renewenvironment{theindex}{%
  \section*{\indexname}%
  \setlength{\parindent}{0pt}%
  \setlength{\parskip}{0pt plus 0.3pt}%
  \let\item\@idxitem
}{%
  \clearpage
}
\makeatother

\IfFileExists{\jobname-pw.ind}{\input{\jobname-pw.ind}}{}

\end{document}

      