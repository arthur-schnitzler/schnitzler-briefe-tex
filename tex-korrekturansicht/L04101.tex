%% latex-korrekturansicht-vorspann.tex
%% Vorspann für die Korrekturansicht.
%% Lädt die gemeinsame Datei latex-vorspann.tex mit gesetztem Schalter.

\newif\ifkorrekturansicht
\korrekturansichttrue

\input{../tex-inputs/latex-vorspann}


\section[Arthur Schnitzler an Gustav Schwarzkopf, 31. 5. 1893]{L04101 Arthur Schnitzler an Gustav Schwarzkopf, 31. 5. 1893}
\nopagebreak\mylabel{L04101v}
\rehead{ }\normalsize\beginnumbering\briefempfaengerindex{Schwarzkopf, Gustav@\textsc{Schwarzkopf, Gustav}!zzzSchnitzler, Arthur@\emph{von Arthur Schnitzler}!1893-05-311@{31. 5. 1893}|(be}
\toendnotes[C]{\smallbreak\pagebreak[2]}
\correspDesc{Versand  durch Arthur Schnitzler am 31. 5. 1893 in Wien
\newline{}Erhalt  durch Gustav Schwarzkopf im Zeitraum [31. 5. 1893 – 3. 6. 1893?] in Wien}\toendnotes[C]{\smallbreak}
\Standort{CUL, Schnitzler, B 96.}
\physDesc{Kartenbrief, 456 Zeichen
\newline{}Handschrift: Bleistift, deutsche Kurrent
\newline{}Versand: Stempel: »\nobreak{}\oindex{VIII., Josefstadt@\textbf{VIII., Josefstadt}, \emph{Verwaltungsgebiet}|pwk}Wien 8/1 64, 31 5 1893, 10–11V\nobreak{}«.  }\pstart{}{\pb}Herrn \textsc{Gustav Schwarzkopf}\pend{}\pstart{}\textcolor{pink}{Wien}\oindex{Wien@\textbf{Wien}, \emph{Verwaltungsgebiet}|pw}{}\ledrightnote{\textcolor{pink}{Wien}}\pend{}\pstart{}\textcolor{pink}{\textsc{I. Tiefer Graben 23}}\oindex{Wien@\textbf{Wien}!I., Innere Stadt@\textbf{I., Innere Stadt}!Tiefer Graben 23@\textbf{Tiefer Graben 23}, \emph{Wohngebäude}|pw}{}\ledrightnote{\textcolor{pink}{Tiefer Graben 23}}\pend{}{\bigskip}\vspace{1em}
\pstart
           \noindent{}{\pb}Verehrteſter Freund, wir ſind jetzt Abends
      (u. zw. meiſt ſehr früh,) im \textsc{\textcolor{pink}{Café Auböck}\oindex{Wien@\textbf{Wien}!I., Innere Stadt@\textbf{I., Innere Stadt}!Café Reichsrath (Inh. Karl Auböck)@\textbf{Café Reichsrath (Inh. Karl Auböck)}, \emph{Kaffeehaus}|pw}{}\ledrightnote{\textcolor{pink}{Café Reichsrath (Inh. Karl Auböck)}}}, unter den Arkaden,
         \uline{hinter dem \textcolor{pink}{Parlament}\oindex{Wien@\textbf{Wien}!I., Innere Stadt@\textbf{I., Innere Stadt}!Österreichisches Parlament@\textbf{Österreichisches Parlament}, \emph{Regierungsgebäude}|pw}{}\ledrightnote{\textcolor{pink}{Österreichisches Parlament}}}, (alſo
               nicht \textcolor{pink}{\textsc{Cafe} Wien}\oindex{Wien@\textbf{Wien}!I., Innere Stadt@\textbf{I., Innere Stadt}!Café Arkaden@\textbf{Café Arkaden}, \emph{Kaffeehaus}|pw}{}\ledrightnote{\textcolor{pink}{Café Arkaden}} oder \textsc{\textcolor{pink}{Arkadencafé}\oindex{Wien@\textbf{Wien}!I., Innere Stadt@\textbf{I., Innere Stadt}!Café Arkaden@\textbf{Café Arkaden}, \emph{Kaffeehaus}|pw}{}\ledrightnote{\textcolor{pink}{Café Arkaden}}}.) –
      Gute Luft! Es duftet nach
      Flieder und Abgeordneten!
      Man ſitzt in Freien, ka{\geminationn} aber
      auch drin ſitzen. Sie ſehen,
      unzählige Vorzüge. – Ich bin faſt
      jeden Abend dort, auch \textsc{\textcolor{blue}{Salten}\pwindex{Salten, Felix 6.\,9.\,1869 Budapest – 8.\,10.\,1945 Zürich@\textsc{Salten, Felix} (6.\,9.\,1869 Budapest – 8.\,10.\,1945 Zürich), \emph{Schriftsteller, Journalist, Chefredakteur}|pw}{}\ledrightnote{\textcolor{blue}{Felix Salten}}}, auch
            \textcolor{blue}{Richard}\pwindex{Beer-Hofmann, Richard 11.\,7.\,1866 Wien – 26.\,9.\,1945 New York City@\textsc{Beer-Hofmann, Richard} (11.\,7.\,1866 Wien – 26.\,9.\,1945 New York City), \emph{Schriftsteller}|pw}{}\ledrightnote{\textcolor{blue}{Richard Beer-Hofmann}} u. \textsc{\textcolor{blue}{Loris}\pwindex{Hofmannsthal, Hugo von 1.\,2.\,1874 Wien – 15.\,7.\,1929 Rodaun@\textsc{Hofmannsthal, Hugo von} (1.\,2.\,1874 Wien – 15.\,7.\,1929 Rodaun), \emph{Schriftsteller}|pw}{}\ledrightnote{\textcolor{blue}{Hugo von Hofmannsthal}}} ko{\geminationm}en zuweilen,
               auch \textcolor{blue}{Fanjung}\pwindex{Van-Jung, Leo 15.\,10.\,1866 Odessa – 2.\,7.\,1939 Riga@\textsc{Van-Jung, Leo} (15.\,10.\,1866 Odessa – 2.\,7.\,1939 Riga), \emph{Gesangspädagoge, Mathematiker}|pw}{}\ledrightnote{\textcolor{blue}{Leo Van-Jung}} u. ſ. w. –\pend
           
\pstart
           Herzlich{\\[\baselineskip]} Ihr{\\[\baselineskip]}\spacefill\mbox{Artur Sch}\pend
           \leftskip=0em{}\selectlanguage{ngerman}\endnumbering\briefempfaengerindex{Schwarzkopf, Gustav@\textsc{Schwarzkopf, Gustav}!zzzSchnitzler, Arthur@\emph{von Arthur Schnitzler}!1893-05-311@{31. 5. 1893}|)be}\mylabel{L04101h}
\begin{anhang}
\end{anhang}\normalsize

\doendnotes{C}
\bigskip
\vfill

\clearpage

\footnotesize

\lohead{\textsc{register}}

% Definiere theindex-Environment komplett neu ohne reledmac
\makeatletter
\renewenvironment{theindex}{%
  \section*{\indexname}%
  \setlength{\parindent}{0pt}%
  \setlength{\parskip}{0pt plus 0.3pt}%
  \let\item\@idxitem
}{%
  \clearpage
}
\makeatother

\IfFileExists{\jobname-pw.ind}{\input{\jobname-pw.ind}}{}

\end{document}

      