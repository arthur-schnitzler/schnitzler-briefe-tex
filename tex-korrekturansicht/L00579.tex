%% latex-korrekturansicht-vorspann.tex
%% Vorspann für die Korrekturansicht.
%% Lädt die gemeinsame Datei latex-vorspann.tex mit gesetztem Schalter.

\newif\ifkorrekturansicht
\korrekturansichttrue

\input{../tex-inputs/latex-vorspann}


               \section[Arthur Schnitzler an Hugo von Hofmannsthal, 7. 8. 1896]{ Arthur Schnitzler an Hugo von Hofmannsthal, 7. 8. 1896}\nopagebreak\mylabel{v}\rehead{ }\normalsize\beginnumbering\briefempfaengerindex{Hofmannsthal, Hugo von@\textsc{Hofmannsthal, Hugo von}!zzzSchnitzler, Arthur@\emph{von Arthur Schnitzler}!1896-08-072@{7. 8. 1896}|(be} \toendnotes[C]{\smallbreak\pagebreak[2]} \Standort{FDH, Hs-30885,51.}
\physDesc{Brief, 2 Blätter (Auch das zweite Blatt von Schnitzler datiert), 7 Seiten
\newline{}Handschrift: Bleistift, deutsche Kurrent}\buchAbdrucke{\weitereDrucke{1) Hugo von Hofmannsthal, Arthur Schnitzler: \emph{Briefwechsel}. Hg. Therese Nickl und Heinrich Schnitzler. Frankfurt am Main: \emph{S. Fischer} 1964, S. 70–72.} \weitereDrucke{2) Arthur Schnitzler: \emph{Briefe 1875–1912}. Hg. Therese Nickl und Heinrich Schnitzler. Frankfurt am Main: \emph{S. Fischer} 1981, S. 290–292.} }\toendnotes[C]{\smallbreak}\pstart
           \raggedleft{}{\pb}\textcolor{pink}{\textsc{Skodsborg}}{}\ledrightnote{\textcolor{pink}{Skodsborg}},
                            7. 8. 96\pend
           \pstart
           Lieber Hugo, ſeit So{\geminationn}tag bin ich mit \textcolor{blue}{Richard}{}\ledrightnote{\textcolor{blue}{Richard Beer-Hofmann}} (und
                        \textcolor{blue}{Paula}{}\ledrightnote{\textcolor{blue}{Paula Beer-Hofmann}}) zuſa{\geminationm}en; ſeit vorgeſtern iſt auch \textcolor{blue}{Paul Goldmann}{}\ledrightnote{\textcolor{blue}{Paul Goldmann}}
                    da, und wir ſind in einem angenehmen Hotel, am Meer, hinter den Häuſern gleich
                    ein wunderſchöner Wald mit Buchen und Tannen, im Wald kleine faſt verſteckte
                    Teiche, und we{\geminationn} man eine halbe Stunde weiter \substVorne{}\textsuperscript{\textcolor{gray}{läuft}}\substDazwischen{}geht\substHinten{}, das freundliche Thal mit lieben kleinen Häuſern und Ort\substVorne{}\textsuperscript{en}\substDazwischen{}ſchaften\substHinten{} (wo wir aber noch nie geweſen ſind). Heute Vormittag ſind wir nach
                    einer kleinen \textcolor{pink}{ſchwe{\pb}diſchen Inſel}{}\ledrightnote{→\textcolor{pink}{Ven}} hinübergeſegelt, wo nicht
                    viele Menſchen wohnen, ſind in dem netten Haus des \textcolor{blue}{Leuchtthurmwächters}{}\ledrightnote{→\textcolor{blue}{?? [Leuchtturmwärter]}} geweſen, und wie wir
                    von dem niedern Thurm herunterſtiegen, fanden wir im Wohnzimmer ein leiſes
                    Harmonium, eine freundliche \textcolor{blue}{Hausfrau}{}\ledrightnote{→\textcolor{blue}{?? [Frau des Leuchtturmwärters]}} und \substVorne{}\textsuperscript{eine}\substDazwischen{}im\substHinten{} Vorzimmer saſs die vierzehnjährige \textcolor{blue}{Tochter}{}\ledrightnote{→\textcolor{blue}{?? [Teenagertochter eines Leuchtturmwärters]}} des Hauſes, regungslos in einer Ecke des
                    Divans, ſah uns mit prachtvollen braunen Augen an, {\pb}ſtrickte und hatte nur einen Schuh an. Dafür war der andere Strumpf an den
                    Zehen zerriſſen. Das war die junge \textcolor{blue}{Dame}{}\ledrightnote{→\textcolor{blue}{?? [Teenagertochter eines Leuchtturmwärters]}} von \textcolor{pink}{\textsc{Hven}}{}\ledrightnote{\textcolor{pink}{Ven}}{\dotstwo}{ }\substVorne{}\textsuperscript{D}\substDazwischen{}I\substHinten{}m Zurückfahren gab es ſo hohe Wellen, daſs man die \textcolor{pink}{Oſtſee}{}\ledrightnote{\textcolor{pink}{Ostsee}} als Meer erkennen durfte; bisher war ſie immer ſo
                    ſtill, daſs man ſich an einem See hätte glauben können. \textcolor{blue}{Paula}{}\ledrightnote{\textcolor{blue}{Paula Beer-Hofmann}} iſt ſogar ſeekrank geweſen. – Wir werden hier wohl
                    alle bis etwa zum 20. Auguſt bleiben. Nachmittags
                    pflege ich zu arbeiten. Vorher bin ich {\pb}wenig
                        dazugeko{\geminationm}en; nur ein paar Regentage oder
                    -ſtunden auf der \textcolor{pink}{Nordcap}{}\ledrightnote{\textcolor{pink}{Nordkap}}tour bin ich in
                    meiner Kajüte geſeſſen und habe am 2. \textcolor{green}{Akt}{}\ledrightnote{→\textcolor{green}{Freiwild. Schauspiel in 3 Akten}} allerlei verſucht. Immerhin ſcheint’s mir, als
                        we{\geminationn} ich theilweiſe in den Intentionen Ihres
                    Briefs, den ich in \textcolor{pink}{\textsc{Trondjhem}}{}\ledrightnote{\textcolor{pink}{Trondheim}} bei meiner Rückkehr gefunden habe, verfahren wäre;
                    denn vor allem hatte ich das Bedürfnis die Scene zwiſchen Ihm
                        un\textcolor{gray}{d} Ihr mit mehr Leben anzufüllen. Ich weiſs noch nicht,
                    ob mir das {\pb}und manches andre, das ich am \textcolor{green}{2.}{}\ledrightnote{→\textcolor{green}{Freiwild. Schauspiel in 3 Akten}} und in den letzten Tagen
                    am \textcolor{green}{3. Akt}{}\ledrightnote{→\textcolor{green}{Freiwild. Schauspiel in 3 Akten}} gearbeitet habe,
                    gelungen iſt; in ein paar Tagen les’ ich die ganze Sache dem \textcolor{blue}{Paul}{}\ledrightnote{\textcolor{blue}{Paul Goldmann}} und dem \textcolor{blue}{Richard}{}\ledrightnote{\textcolor{blue}{Richard Beer-Hofmann}}
                    wieder vor. So wie ichs haben will, bring ichs doch wohl nie zuſa{\geminationm}en. –\pend
           \pstart
           \textcolor{blue}{Richard}{}\ledrightnote{\textcolor{blue}{Richard Beer-Hofmann}} hat mir von Ihrer \textcolor{green}{Novelle}{}\ledrightnote{→\textcolor{green}{Geschichte der beiden Liebespaare}} erzählt; auch dſs er Ihnen
                    gerathen, Sie drucken zu laſſen. Solange muſs ich wohl warten bis
                    ich ſie zu leſen bekomme. Wohin werden Sie ſie geben? –\pend
           \pstart
           Meine Reiſe iſt im ganzen ſehr ſchön geweſen; vielleicht iſt die Zeit nur {\pb}etwas zu kurz geweſen, um ſoviel in ſich
                    aufzunehmen.\pend
           \pstart
           Auf der See hab ich merkwürdg viel Kopfſchmerzen gehabt. Von Städten hat mir \textcolor{pink}{\textsc{Gothenburg}}{}\ledrightnote{\textcolor{pink}{Göteborg}} den
                    ſtärkſten Eindruck gemacht; wahrſcheinlich weil ich dort ganz allein (auch nicht
                    mit zufälligen Bekannten von der Reiſe) herumgegangen bin und am tiefſten
                    geſpürt habe: Wie fremd – wie fern – und dann weil ich nur ein paar Stunden dort
                    geweſen bin und bei jedem Haus, jedem Menſchen {\pb}wußte – dich ſeh ich zum letzten Mal.\pend
           \pstart
           – In \textcolor{pink}{\textsc{Christ}.}{}\ledrightnote{\textcolor{pink}{Oslo}} hab ich \textcolor{blue}{\textsc{Ibsen}}{}\ledrightnote{\textcolor{blue}{Henrik Ibsen}} geſprochen, der mehr zuhörte als redete aber ſehr liebenswürdg war; in \textcolor{pink}{\textsc{Kopenhagen}}{}\ledrightnote{\textcolor{pink}{Kopenhagen}}{ }ſind wir (\textcolor{blue}{Richard}{}\ledrightnote{\textcolor{blue}{Richard Beer-Hofmann}} u ich) mit \textcolor{blue}{\textsc{Nansen}}{}\ledrightnote{\textcolor{blue}{Peter Nansen}} beim Frühſtück geſeſſen, den wir wohl noch ſehen werden. –\pend
           \pstart
           – Bis zum 20. treffen mich Nachrichten hier, \textcolor{pink}{Badehotel}{}\ledrightnote{\textcolor{pink}{Badehotel}}. Es möcht mich freuen, noch zwei Worte von Ihnen
                    zu hören.\pend
           \pstart Leben Sie wohl! Mit vielen herzlichen Grüßen Ihr \spacefill\mbox{ArthSch}\pend{}\pstart
           \textcolor{pink}{\textsc{Skodsborg}}{}\ledrightnote{\textcolor{pink}{Skodsborg}}{ }7/8 96. \pend
           \pstart
           Nach 20. (–25.) \textcolor{pink}{\textsc{Berlin}}{}\ledrightnote{\textcolor{pink}{Berlin}}, aber ſchreiben Sie nach \textcolor{pink}{Wien}{}\ledrightnote{\textcolor{pink}{Wien}}.\pend
           \endnumbering\briefempfaengerindex{Hofmannsthal, Hugo von@\textsc{Hofmannsthal, Hugo von}!zzzSchnitzler, Arthur@\emph{von Arthur Schnitzler}!1896-08-072@{7. 8. 1896}|)be}\mylabel{h}  \normalsize

\doendnotes{C}
\bigskip
\vfill

\clearpage

\footnotesize

\lohead{\textsc{register}}

% Definiere theindex-Environment komplett neu ohne reledmac
\makeatletter
\renewenvironment{theindex}{%
  \section*{\indexname}%
  \setlength{\parindent}{0pt}%
  \setlength{\parskip}{0pt plus 0.3pt}%
  \let\item\@idxitem
}{%
  \clearpage
}
\makeatother

\IfFileExists{\jobname-pw.ind}{\input{\jobname-pw.ind}}{}

\end{document}

      