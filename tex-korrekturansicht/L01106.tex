%% latex-korrekturansicht-vorspann.tex
%% Vorspann für die Korrekturansicht.
%% Lädt die gemeinsame Datei latex-vorspann.tex mit gesetztem Schalter.

\newif\ifkorrekturansicht
\korrekturansichttrue

\input{../tex-inputs/latex-vorspann}


               \section[Arthur Schnitzler an Richard Beer-Hofmann, 6. 4. 1901]{ Arthur Schnitzler an Richard Beer-Hofmann, 6. 4. 1901}\nopagebreak\mylabel{v}\rehead{ }\normalsize\beginnumbering\briefempfaengerindex{Beer-Hofmann, Richard@\textsc{Beer-Hofmann, Richard}!zzzSchnitzler, Arthur@\emph{von Arthur Schnitzler}!1901-04-061@{6. 4. 1901}|(be} \toendnotes[C]{\smallbreak\pagebreak[2]} \Standort{YCGL, MSS 31.}
\physDesc{Bildpostkarte
\newline{}Handschrift: schwarze Tinte, deutsche Kurrent\newline{}Versand: 1) Stempel: »\nobreak{}\oindex{Rom@\textbf{Rom}, \emph{Besiedelter Ort (A.BSO)}|pwk}\textcolor{gray}{Ro}ma\nobreak{}«.  2) Stempel: »\nobreak{}\oindex{I., Innere Stadt@\textbf{I., Innere Stadt}, \emph{Bezirk (A.BZK)}|pwk}Wien \textcolor{gray}{1/1}, 8 4. 1901, 9–10½V, Bestellt\nobreak{}«. \newline{}Ordnung: mit Bleistift von unbekannter Hand datiert: »6. 4.« }\toendnotes[C]{\smallbreak}\pstart{}{\pb}Dr. \textsc{Rich.
                            Beer-Hofmann}\pend{}\pstart{}\textcolor{pink}{Wien}{}\ledrightnote{\textcolor{pink}{Wien}}\pend{}\pstart{}\textcolor{pink}{\textsc{I. Wollzeile 15}}{}\ledrightnote{\textcolor{pink}{Wollzeile}}.\pend{}\pstart{}\textcolor{pink}{\textsc{Austria}}{}\ledrightnote{\textcolor{pink}{Österreich}}\pend{}{\bigskip}\pstart
           \noindent{}\centering{}{\pb}\textcolor{gray}{\textbf{SOUVENIR DES \textcolor{pink}{CATACOMBES DE S\textsuperscript{T} CALLISTE}{}\ledrightnote{\textcolor{pink}{Calixtus-Katakombe}}}}\pend
           \pstart
           \noindent{}\centering{}\textcolor{gray}{\textbf{\textcolor{pink}{Via Appia Antica 33 ⋅ ROMA}{}\ledrightnote{\textcolor{pink}{Via Appia}}}}\pend
           \pstart
           \label{T_L01106_1v}\edtext{Es}{\lemma{\textnormal{\emph{Es}}}\Cendnote{\textnormal{Ein Pfeil weist auf »Via« in \textcolor{pink}{Via Appia}.}}}\label{T_L01106_1h} iſt offenbar nicht ſo wie Sie meinen.\pend
           \pstart
           Herzlichſt Ihr{\\[\baselineskip]}\spacefill\mbox{ArthSch}\pend
           \leftskip=0em{}\pstart
           6. 4. 901.\pend
           \endnumbering\briefempfaengerindex{Beer-Hofmann, Richard@\textsc{Beer-Hofmann, Richard}!zzzSchnitzler, Arthur@\emph{von Arthur Schnitzler}!1901-04-061@{6. 4. 1901}|)be}\mylabel{h}  \normalsize

\doendnotes{C}
\bigskip
\vfill

\clearpage

\footnotesize

\lohead{\textsc{register}}

% Definiere theindex-Environment komplett neu ohne reledmac
\makeatletter
\renewenvironment{theindex}{%
  \section*{\indexname}%
  \setlength{\parindent}{0pt}%
  \setlength{\parskip}{0pt plus 0.3pt}%
  \let\item\@idxitem
}{%
  \clearpage
}
\makeatother

\IfFileExists{\jobname-pw.ind}{\input{\jobname-pw.ind}}{}

\end{document}

      