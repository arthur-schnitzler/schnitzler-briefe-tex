%% latex-korrekturansicht-vorspann.tex
%% Vorspann für die Korrekturansicht.
%% Lädt die gemeinsame Datei latex-vorspann.tex mit gesetztem Schalter.

\newif\ifkorrekturansicht
\korrekturansichttrue

\input{../tex-inputs/latex-vorspann}


\renewcommand{\erwaehntePersonen}{Personen: Felix Salten}
\renewcommand{\erwaehnteInstitutionen}{Institutionen: S. Fischer Verlag}
\renewcommand{\erwaehnteOrte}{Orte: Edmund-Weiß-Gasse 7, Semmering, Wien, XIX., Döbling, XVIII., Währing}
\renewcommand{\erwaehnteWerke}{Werke: Der Weg ins Freie. Roman, Die neue Rundschau}
\section[ Felix Salten an Arthur Schnitzler, 16. 1. 1908]{Felix Salten an Arthur Schnitzler, 16. 1. 1908}
\nopagebreak\mylabel{v}
\rehead{ }\normalsize\beginnumbering\briefempfaengerindex{Schnitzler, Arthur@\textsc{Schnitzler, Arthur}!zzzSalten, Felix@\emph{von Felix Salten}!1908-01-164@{16. 1. 1908}|(be}
\toendnotes[C]{\smallbreak\pagebreak[2]}\Standort{CUL, Schnitzler, B 89, B 1.}
\physDesc{Postkarte, 490 Zeichen
\newline{}Handschrift: schwarze Tinte, lateinische Kurrent
\newline{}Versand: Stempel: »\nobreak{}\oindex{XIX., Doebling@\textbf{XIX., Döbling}, \emph{A.ADM3}|pwk}19/2 Wien 119 b, 18. 1. 08, VI\nobreak{}«.  
\newline{}Ordnung: mit Bleistift von unbekannter Hand nummeriert: »240« }\toendnotes[C]{\smallbreak}\pstart{}{\pb}Herrn D\textsuperscript{r} Arthur Schnitzler\pend{}\pstart{}\textcolor{pink}{Wien XVIII.}{}\ledrightnote{\textcolor{pink}{XVIII., Währing}}\pend{}\pstart{}\textcolor{pink}{Spöttelgasse 7}{}\ledrightnote{\textcolor{pink}{Edmund-Weiß-Gasse 7}}\pend{}
{\bigskip}
\pstart
           \raggedleft{}16. I. \textcolor{gray}{08}\pend
           
\pstart{}Lieber,\pend
\pstart
           ich vergaß, Ihnen folgendes zu schreiben: Wird Ihr \label{K_L03509-1v}\edtext{\textcolor{green}{Roman}{}\ledrightnote{{$\rightarrow$}\textcolor{green}{Der Weg ins Freie. Roman}} jetzt auf längere
                  Strecken}{\lemma{\textnormal{\emph{Roman … Strecken}}}\Cendnote{\textnormal{Der erste Teil von \emph{\textcolor{green}{Der Weg ins Freie}} erschien im ersten Heft von
                     \emph{\textcolor{green}{Die neue Rundschau}} (Jg. 19, H. 1,
                        Januar 1908). Es folgten fünf weitere Teile. Der sechste und letzte Teil erschien um
                  Anfang Juni 1908. Zeitgleich mit dem letzten Abdruck
                  erschien die Buchausgabe bei \emph{\textcolor{brown}{S. Fischer}}.}}}\label{K_L03509-1h} als auf eine Monatsrate gesetzt? Und wenn er’s wird,
               könnten oder wollten Sie mir von \textcolor{brown}{Fischer}{}\ledrightnote{\textcolor{brown}{S. Fischer Verlag}} etwa
               einen Abzug senden laßen? (den ich natürlich wie ein Manuscript geheimhalten würde).
               Ich bin durch den Influenza-Anfall, durch nervöse Darmstörungen ec. sehr herunter und
               werde voraussichtlich Sonntag oder Montag auf den \textcolor{pink}{Semmering}{}\ledrightnote{\textcolor{pink}{Semmering}}.\pend
           \pstart Herzlichst Ihr \spacefill\mbox{Salten}\pend{}\endnumbering\briefempfaengerindex{Schnitzler, Arthur@\textsc{Schnitzler, Arthur}!zzzSalten, Felix@\emph{von Felix Salten}!1908-01-164@{16. 1. 1908}|)be}\mylabel{h}  \normalsize

\doendnotes{C}
\bigskip
\vfill

\clearpage

\footnotesize

\lohead{\textsc{register}}

% Definiere theindex-Environment komplett neu ohne reledmac
\makeatletter
\renewenvironment{theindex}{%
  \section*{\indexname}%
  \setlength{\parindent}{0pt}%
  \setlength{\parskip}{0pt plus 0.3pt}%
  \let\item\@idxitem
}{%
  \clearpage
}
\makeatother

\IfFileExists{\jobname-pw.ind}{\input{\jobname-pw.ind}}{}

\end{document}

      