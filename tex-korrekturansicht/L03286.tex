%% latex-korrekturansicht-vorspann.tex
%% Vorspann für die Korrekturansicht.
%% Lädt die gemeinsame Datei latex-vorspann.tex mit gesetztem Schalter.

\newif\ifkorrekturansicht
\korrekturansichttrue

\input{../tex-inputs/latex-vorspann}


\renewcommand{\erwaehnteOrte}{Orte: Café Pfob, Wien}
\renewcommand{\erwaehnteWerke}{Werke: Wiener Allgemeine Zeitung, »Franz Joseph I. und seine Zeit.« (Culturhistorischer Rückblick auf die Francisco-Josephinische Epoche. – Unter dem Protectorate des Erzherzogs Franz Ferdinand, herausgegeben von J. Schnitzer. Wien, bei R. Lechner.)}
\section[ Felix Salten an Arthur Schnitzler, {[}28. 1. 1899{]}]{Felix Salten an Arthur Schnitzler, {[}28. 1. 1899{]}}
\nopagebreak\mylabel{v}
\rehead{ }\normalsize\beginnumbering\briefempfaengerindex{Schnitzler, Arthur@\textsc{Schnitzler, Arthur}!zzzSalten, Felix@\emph{von Felix Salten}!1899-01-281@{{[}28. 1. 1899{]}}|(be}
\toendnotes[C]{\smallbreak\pagebreak[2]}\Standort{CUL, Schnitzler, B 89, A 2.}
\physDesc{Karte, 198 Zeichen
\newline{}Handschrift: schwarze Tinte, lateinische Kurrent
\newline{}Schnitzler: mit Bleistift datiert: »28/1 99« 
\newline{}Ordnung: mit Bleistift von unbekannter Hand nummeriert: »110« }\toendnotes[C]{\smallbreak}
\pstart
           \noindent{}{\pb}lieber Arthur, wenn Sie eine verfügbare halbe Stunde
               haben, lesen Sie, bitte, meine »\label{K_L03286-1v}\edtext{\textcolor{green}{Literatur}{}\ledrightnote{{$\rightarrow$}\textcolor{green}{»Franz Joseph I. und seine Zeit.« (Culturhistorischer Rückblick auf die Francisco-Josephinische Epoche. – Unter dem Protectorate des Erzherzogs Franz Ferdinand, herausgegeben von J. Schnitzer. Wien, bei R. Lechner.)}}}{\lemma{\textnormal{\emph{Literatur}}}\Cendnote{\textnormal{Wohl: \textcolor{blue}{–X.–} [ = \textcolor{blue}{Felix Salten}]: \emph{\textcolor{green}{»Franz Joseph I. und
                        seine Zeit.« (Culturhistorischer Rückblick auf die Francisco-Josephinische
                        Epoche. – Unter dem Protectorate des Erzherzogs Franz Ferdinand,
                        herausgegeben von J. Schnitzer. Wien, bei R. Lechner.)}} In: \emph{\textcolor{green}{Wiener Allgemeine Zeitung}}, Nr. 6.272, 28. 1. 1899, S. 2.}}}\label{K_L03286-1h}«. Ich bin heute{ }Abend
               im \textcolor{pink}{Schrangl}{}\ledrightnote{\textcolor{pink}{Café Pfob}}, und es ist mir natürlich sehr um Ihre
               Meinung zu thun.\pend
           
\pstart
           Herzlich Ihr {\\[\baselineskip]}\spacefill\mbox{Salten}\pend
           \leftskip=0em{}\endnumbering\briefempfaengerindex{Schnitzler, Arthur@\textsc{Schnitzler, Arthur}!zzzSalten, Felix@\emph{von Felix Salten}!1899-01-281@{{[}28. 1. 1899{]}}|)be}\mylabel{h}  \normalsize

\doendnotes{C}
\bigskip
\vfill

\clearpage

\footnotesize

\lohead{\textsc{register}}

% Definiere theindex-Environment komplett neu ohne reledmac
\makeatletter
\renewenvironment{theindex}{%
  \section*{\indexname}%
  \setlength{\parindent}{0pt}%
  \setlength{\parskip}{0pt plus 0.3pt}%
  \let\item\@idxitem
}{%
  \clearpage
}
\makeatother

\IfFileExists{\jobname-pw.ind}{\input{\jobname-pw.ind}}{}

\end{document}

      