%% latex-korrekturansicht-vorspann.tex
%% Vorspann für die Korrekturansicht.
%% Lädt die gemeinsame Datei latex-vorspann.tex mit gesetztem Schalter.

\newif\ifkorrekturansicht
\korrekturansichttrue

\input{../tex-inputs/latex-vorspann}


               \section[Arthur Schnitzler: Widmungsexemplar Anatol für Hermann Bahr, {[}29. 10.?{]} 1892]{ Arthur Schnitzler: Widmungsexemplar Anatol für Hermann Bahr,
               {[}29. 10.?{]} 1892}\nopagebreak\mylabel{v}\rehead{ }\normalsize\beginnumbering\briefempfaengerindex{Bahr, Hermann@\textsc{Bahr, Hermann}!zzzSchnitzler, Arthur@\emph{von Arthur Schnitzler}!1892-10-291@{{[}29. 10.?{]} 1892}|(be} \toendnotes[C]{\smallbreak\pagebreak[2]} \Standort{Salzburg, Universitätsbibliothek, 32342-I.}
\physDesc{Widmung am Titelblatt
\newline{}Handschrift: schwarze Tinte, deutsche Kurrent}\buchAbdrucke{\weitereDrucke{Hermann Bahr, Arthur Schnitzler: \emph{Briefwechsel, Aufzeichnungen, Dokumente (1891–1931)}. Hg. Kurt Ifkovits und Martin Anton Müller. Göttingen: \emph{Wallstein} 2018, S. 28.} }\toendnotes[C]{\smallbreak}\pstart
           \noindent{}{\pb}Herrn \textsc{Hermann Bahr}{\\}freundſchaftlich u verehrungsvoll\pend
           \pstart \spacefill\mbox{ArthSch}\pend{}{\bigskip}\pstart
           \noindent{}\centering{}\textcolor{gray}{\textbf{\textbf{Arthur Schnitzler.}}}\pend
           \pstart
           \noindent{}\centering{}\textcolor{gray}{\textbf{\textcolor{green}{Anatol}{}\ledrightnote{\textcolor{green}{Anatol}}.}}\pend
           {\bigskip}\pstart
           \noindent{}\centering{}\textcolor{gray}{\textbf{\textbf{\textcolor{pink}{Berlin}{}\ledrightnote{\textcolor{pink}{Berlin}},{ }\label{K_L00129_1v}\edtext{1893}{\lemma{\textnormal{\emph{1893}}}\Cendnote{\textnormal{Vgl. A. S.: \emph{Tagebuch}, 29. 10. 1892. Am
                           3. 11. 1892 vom \emph{\textcolor{green}{Börsenblatt für den deutschen
                              Buchhandel}} als Neuerscheinung gemeldet}}}\label{K_L00129_1h}.}}}\pend
           \pstart
           \noindent{}\centering{}\textcolor{gray}{\textbf{\textcolor{brown}{Verlag des Bibliographiſchen Bureaus}{}\ledrightnote{\textcolor{brown}{Bibliographisches Bureau}}.}}\pend
           \pstart
           \noindent{}\centering{}\textcolor{gray}{\textbf{\textcolor{pink}{Alexanderſtraße 2}{}\ledrightnote{\textcolor{pink}{Alexanderstraße}}.}}\pend
           \endnumbering\briefempfaengerindex{Bahr, Hermann@\textsc{Bahr, Hermann}!zzzSchnitzler, Arthur@\emph{von Arthur Schnitzler}!1892-10-291@{{[}29. 10.?{]} 1892}|)be}\mylabel{h}  \normalsize

\doendnotes{C}
\bigskip
\vfill

\clearpage

\footnotesize

\lohead{\textsc{register}}

% Definiere theindex-Environment komplett neu ohne reledmac
\makeatletter
\renewenvironment{theindex}{%
  \section*{\indexname}%
  \setlength{\parindent}{0pt}%
  \setlength{\parskip}{0pt plus 0.3pt}%
  \let\item\@idxitem
}{%
  \clearpage
}
\makeatother

\IfFileExists{\jobname-pw.ind}{\input{\jobname-pw.ind}}{}

\end{document}

      