%% latex-korrekturansicht-vorspann.tex
%% Vorspann für die Korrekturansicht.
%% Lädt die gemeinsame Datei latex-vorspann.tex mit gesetztem Schalter.

\newif\ifkorrekturansicht
\korrekturansichttrue

\input{../tex-inputs/latex-vorspann}


\renewcommand{\erwaehntePersonen}{Personen: Samuel Fischer, Hedwig Fischer, Margarethe Kainz, Josef Kainz, Anna Katharina Rehmann, Felix Salten, Ottilie Salten, Paul Salten, Paula Schlenther, Paul Schlenther, Olga Schnitzler, Heinrich Schnitzler,  W. Fred}
\renewcommand{\erwaehnteInstitutionen}{Institutionen: Franz-Grillparzer-Preis}
\renewcommand{\erwaehnteOrte}{Orte: Semmering, Südbahnhotel, Wien, Österreich}
\renewcommand{\erwaehnteWerke}{Werke: Der Weg ins Freie. Roman, Die neue Rundschau, Tagebuch}
\section[ Felix Salten an Arthur Schnitzler, 26. 1. 1908]{Felix Salten an Arthur Schnitzler, 26. 1. 1908}
\nopagebreak\mylabel{v}
\rehead{ }\normalsize\beginnumbering\briefempfaengerindex{Schnitzler, Arthur@\textsc{Schnitzler, Arthur}!zzzSalten, Felix@\emph{von Felix Salten}!1908-01-261@{26. 1. 1908}|(be}
\toendnotes[C]{\smallbreak\pagebreak[2]}\Standort{CUL, Schnitzler, B 89, B 1.}
\physDesc{Brief, 1 Blatt, 2 Seiten, 2083 Zeichen
\newline{}Handschrift: Bleistift, lateinische Kurrent
\newline{}Schnitzler: mit Bleistift Vermerk »\textsc{Salt\textcolor{gray}{en}}« 
\newline{}Ordnung: mit Bleistift von unbekannter Hand nummeriert: »241« }\toendnotes[C]{\smallbreak}
\pstart
           \noindent{}{\pb}\textcolor{gray}{\textbf{\textcolor{pink}{Südbahn-Hôtel}{}\ledrightnote{\textcolor{pink}{Südbahnhotel}}}}\pend
           
\pstart
           \textcolor{gray}{\textbf{\textcolor{pink}{Semmering}{}\ledrightnote{\textcolor{pink}{Semmering}}}}\hfill 26./1. 08\pend
           
\pstart
           \textcolor{gray}{\textbf{\textcolor{pink}{Austria}{}\ledrightnote{\textcolor{pink}{Österreich}}.}}\pend
           
\pstart
           \textcolor{gray}{\textbf{\textbf{TELEGRAMME:}}}\pend
           
\pstart
           \textcolor{gray}{\textbf{\textbf{\textcolor{pink}{SÜDBAHNHÔTEL SEMMERING}{}\ledrightnote{\textcolor{pink}{Südbahnhotel}}.}}}\pend
           
\pstart
           \textcolor{gray}{\textbf{TELEPHON:}}\pend
           
\pstart
           \textcolor{gray}{\textbf{HÔTEL {\dotsfour} NR. 5.}}\pend
           
\pstart
           \textcolor{gray}{\textbf{DEPENDANCE NR. 6.}}\pend
           
\pstart{}Lieber,\pend
\pstart
           danke sehr für Ihren ausführlichen \label{K_L03491-1v}\edtext{Brief}{\lemma{\textnormal{\emph{Brief}}}\Cendnote{\textnormal{Arthur Schnitzler an Felix Salten, 25. 1. 1908}}}\label{K_L03491-1h}, der mich sehr gefreut hat. Den letzten Satz, da wo Sie sagen, dass Sie sich
               wieder »\label{K_L03491-2v}\edtext{keck mitten ins Leben}{\lemma{\textnormal{\emph{keck mitten ins Leben}}}\Cendnote{\textnormal{\textcolor{blue}{Schnitzler} schrieb »frech wieder mitten ins Leben hinein«.}}}\label{K_L03491-2h}« u. s. w. habe ich, wie ich Ihnen gestehen muss, mit
               einer plötzlich aufsteigenden, sehr starken Ergriffenheit gelesen. Denn aus ihm sah
               ich erst ganz deutlich, \uline{wo} Sie in dieser letzten Zeit
               mit Ihren Gedanken und Sorgen gewesen sind, und was Sie durchgemacht haben. Nun aber
               dürfen Sie sich wol freuen und Ihre Freunde mit Ihnen. Wundervoll ist es ja, wie
               diese \label{K_L03491-3v}\edtext{Gefahr}{\lemma{\textnormal{\emph{Gefahr}}}\Cendnote{\textnormal{siehe Felix Salten an Arthur Schnitzler, [10. 12. 1907]}}}\label{K_L03491-3h} an Ihnen u. Ihrer \textcolor{blue}{Frau}{}\ledrightnote{{$\rightarrow$}\textcolor{blue}{Olga Schnitzler}} vorbeigeschwebt ist, und wie dann mit dem \textcolor{brown}{Grillparzer Preis}{}\ledrightnote{\textcolor{brown}{Franz-Grillparzer-Preis}} etwas zu Ihnen kam, was schließlich doch im
               Tiefsten so etwas wie einen Schimmer von Glück bedeutet. Wir gehen dem Frühling
               entgegen, und Ihre \textcolor{blue}{Frau}{}\ledrightnote{{$\rightarrow$}\textcolor{blue}{Olga Schnitzler}} wird
               sich hoffentlich rasch erholen. Man sagt ja, dass nach dem Scharlach die Gesundheit
               intensiver wird, und so wird Frau \textcolor{blue}{Olga}{}\ledrightnote{\textcolor{blue}{Olga Schnitzler}} jetzt in
               ein schönes Genesen und Glühen kommen, und mit der Jahreszeit gehen. Besseres läßt
               sich kaum denken. Ihren \label{K_L03491-4v}\edtext{\textcolor{green}{Roman}{}\ledrightnote{{$\rightarrow$}\textcolor{green}{Der Weg ins Freie. Roman}} las ich nun doch in den
               ersten zwei Fortsetzungen}{\lemma{\textnormal{\emph{Roman … Fortsetzungen}}}\Cendnote{\textnormal{vgl. Felix Salten an Arthur Schnitzler, 16. 1. 1908. \textcolor{blue}{Salten}s Lektüre der
                  ersten Fortsetzung bedeutet, dass das Monatsheft des Februar bereits vorzeitig ausgeliefert wurde.}}}\label{K_L03491-4h}. Sie werden meine Neugierde begreifen u. entschuldigen. \label{K_L03491-5v}\edtext{Sagen kann ich jetzt natürlich noch
                  nichts}{\lemma{\textnormal{\emph{Sagen … nichts}}}\Cendnote{\textnormal{Nachdem sie sich wenige Tage
                  später, am 4. 2. 1908, auf dem Weg zum \textcolor{pink}{Semmering} getroffen
                  hatten, notierte \textcolor{blue}{Schnitzler} in seinem \emph{\textcolor{green}{Tagebuch}}: »Er [ = \textcolor{blue}{Salten}] sagt über einen \textcolor{green}{Roman}, dessen 2 erste Theile (Jänner-, Feber\textcolor{green}{heft}) er gelesen: Sehr
                  lebendige Gestalten. Dann (zögernd) … ›Aber es hat mir erst recht leid gethan,
                  dass ich’s nicht im Manuscript gelesen … es sind stilistische (Fehler?) Mängel,
                  Härten (erinner mich des Worts nicht) – wie sie natürlich bei einem so großen \textcolor{green}{Werk} nicht zu vermeiden
                  sind.–‹ Es ärgerte, ja empörte mich beinahe – obwohl, oder weil ich darauf
                  vorbereitet war.– ›Er wird nicht wollen‹ sagte ich neulich.– Wer wird wollen –?‹«
                  Diese Kritik \textcolor{blue}{Salten}s sollte \textcolor{blue}{Schnitzler} noch lange beschäftigen. Siehe
                  etwa A. S.: \emph{Tagebuch}, 28. 4. 1908.}}}\label{K_L03491-5h}, ahne auch nur
               von weitem, wohin der \textcolor{green}{Weg ins
                  Freie}{}\ledrightnote{{$\rightarrow$}\textcolor{green}{Der Weg ins Freie. Roman}} führt. Aber eine Menge Menschen wird mir jetzt schon sehr lebendig und
               das Abreißen der Fortsetzung mir freilich je mehr zur Qual, je näher einem diese
               Menschen kommen.\pend
           
\pstart
           {\pb}Ich bin seit \label{K_L03491-6v}\edtext{Donnerstag voriger Woche}{\lemma{\textnormal{\emph{Donnerstag voriger Woche}}}\Cendnote{\textnormal{\textcolor{blue}{Salten} dürfte seine Pläne kurzfristig
                  geändert haben, hatte er doch am 16. 1. 1908 noch geschrieben, dass er erst »voraussichtlich Sonntag oder Montag auf
                  den \textcolor{pink}{Semmering}« fahren wolle. Der 23. 1. 1908 kann durch die folgenden Ausführungen
                  ausgeschlossen werden.}}}\label{K_L03491-6h}{ }\textcolor{pink}{hier oben}{}\ledrightnote{{$\rightarrow$}\textcolor{pink}{Semmering}}; traf hier Frau \textcolor{blue}{Kainz}{}\ledrightnote{\textcolor{blue}{Margarethe Kainz}} mit Frau \textcolor{blue}{Schlenther}{}\ledrightnote{\textcolor{blue}{Paula Schlenther}}, mit der ich komischerweise sehr sympathisirte. (Nett hat sich
                  \label{K_L03491-7v}\edtext{\textcolor{blue}{Schlenther}{}\ledrightnote{\textcolor{blue}{Paul Schlenther}} in der \textcolor{brown}{Preis}{}\ledrightnote{\textcolor{brown}{Franz-Grillparzer-Preis}}-Angelegenheit}{\lemma{\textnormal{\emph{Schlenther … Preis-Angelegenheit}}}\Cendnote{\textnormal{siehe Felix Salten an Arthur Schnitzler, 15. 1. 1908}}}\label{K_L03491-7h} benommen) Samstag kam \textcolor{blue}{Otti}{}\ledrightnote{\textcolor{blue}{Ottilie Salten}} mit den \textcolor{blue}{Kindern}{}\ledrightnote{{$\rightarrow$}\textcolor{blue}{Paul Salten}{\newline}{$\rightarrow$}\textcolor{blue}{Anna Katharina Rehmann}}, Sonntag kamen \textcolor{blue}{Fischers}{}\ledrightnote{\textcolor{blue}{Samuel Fischer}{\newline}\textcolor{blue}{Hedwig Fischer}}, gestern u. heute ist der \textcolor{blue}{Kainz}{}\ledrightnote{\textcolor{blue}{Josef Kainz}} dagewesen, und Herr \textcolor{blue}{Fred}{}\ledrightnote{\textcolor{blue}{W. Fred}} ist immer da. Ich arbeite ein bischen und spüre noch immer meine
               Darmzustände. – Hoffentlich sehen wir uns \textcolor{pink}{hier oben}{}\ledrightnote{{$\rightarrow$}\textcolor{pink}{Semmering}} oder in \textcolor{pink}{Wien}{}\ledrightnote{\textcolor{pink}{Wien}}.
               Ängstlich bin ich ja, das gebe ich zu. Sie wißen doch, dass ich wegen meiner \textcolor{blue}{Kinder}{}\ledrightnote{{$\rightarrow$}\textcolor{blue}{Paul Salten}{\newline}{$\rightarrow$}\textcolor{blue}{Anna Katharina Rehmann}} beständig in
               einer halbtollen Furcht lebe. Aber ich denke, wenn Sie \textcolor{blue}{Heini}{}\ledrightnote{\textcolor{blue}{Heinrich Schnitzler}} bei sich haben, ist wol nichts mehr zu
                  besorgen.\pend
           
\pstart
           Also vieles Gute und Herzliche von \textcolor{blue}{uns}{}\ledrightnote{{$\rightarrow$}\textcolor{blue}{Ottilie Salten}} zu Ihnen. \textcolor{blue}{Otti}{}\ledrightnote{\textcolor{blue}{Ottilie Salten}} u.
               ich laßen Frau \textcolor{blue}{Olga}{}\ledrightnote{\textcolor{blue}{Olga Schnitzler}} besonders grüßen.\pend
           
\pstart
           Ihr {\\[\baselineskip]}\spacefill\mbox{Salten}\pend
           \leftskip=0em{}\endnumbering\briefempfaengerindex{Schnitzler, Arthur@\textsc{Schnitzler, Arthur}!zzzSalten, Felix@\emph{von Felix Salten}!1908-01-261@{26. 1. 1908}|)be}\mylabel{h}  \normalsize

\doendnotes{C}
\bigskip
\vfill

\clearpage

\footnotesize

\lohead{\textsc{register}}

% Definiere theindex-Environment komplett neu ohne reledmac
\makeatletter
\renewenvironment{theindex}{%
  \section*{\indexname}%
  \setlength{\parindent}{0pt}%
  \setlength{\parskip}{0pt plus 0.3pt}%
  \let\item\@idxitem
}{%
  \clearpage
}
\makeatother

\IfFileExists{\jobname-pw.ind}{\input{\jobname-pw.ind}}{}

\end{document}

      