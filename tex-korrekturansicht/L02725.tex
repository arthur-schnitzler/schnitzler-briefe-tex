%% latex-korrekturansicht-vorspann.tex
%% Vorspann für die Korrekturansicht.
%% Lädt die gemeinsame Datei latex-vorspann.tex mit gesetztem Schalter.

\newif\ifkorrekturansicht
\korrekturansichttrue

\input{../tex-inputs/latex-vorspann}


               \section[Paul Goldmann an Arthur Schnitzler, {[}28./29.?{]} 12. 1893]{ Paul Goldmann an Arthur Schnitzler, {[}28./29.?{]} 12. 1893}\nopagebreak\mylabel{v}\rehead{ }\normalsize\beginnumbering\briefempfaengerindex{Schnitzler, Arthur@\textsc{Schnitzler, Arthur}!zzzGoldmann, Paul@\emph{von Paul Goldmann}!1893-12-281@{{[}28./29.?{]} 12. 1893}|(be} \toendnotes[C]{\smallbreak\pagebreak[2]} \Standort{DLA, A:Schnitzler, HS.NZ85.1.3163.}
\physDesc{Postkarte
\newline{}Handschrift: 1) schwarze Tinte, deutsche Kurrent\hspace{1em}2) schwarze Tinte, lateinische Kurrent (\noindent{}Adresse)\hspace{1em}\newline{}Versand: 1) Stempel: »\nobreak{}\oindex{Place de la Bourse@\textbf{Place de la Bourse}, \emph{Platz (K.PLT)}|pwk}Par{[}is{]} Pl. de la
                                          Bour{[}se{]}, \textcolor{gray}{×}\-\textcolor{gray}{×} \begin{otherlanguage}{french}Dec\end{otherlanguage}{[}.{]} 93, 4\textsuperscript{E}\nobreak{}«.  2) Stempel: »\nobreak{}Wien 9/3 72, 31. 12. 93, 8.V, Bestellt\nobreak{}«. }\toendnotes[C]{\smallbreak}\pstart{}{\pb}\textcolor{pink}{Autriche}{}\ledrightnote{\textcolor{pink}{Österreich}}.\pend{}\pstart{}Herrn Dr. Arthur Schnitzler\pend{}\pstart{}\textcolor{pink}{IX. Frankgaße 1}{}\ledrightnote{\textcolor{pink}{Frankgasse}}\pend{}\pstart{}\textcolor{pink}{Wien}{}\ledrightnote{\textcolor{pink}{Wien}}. \pend{}{\bigskip}\pstart
           \noindent{}{\pb}1.) Leſen: \label{K_L02725-1v}\edtext{\textsc{\textcolor{blue}{Albrecht Dürer}{}\ledrightnote{\textcolor{blue}{Albrecht Dürer}}s}{ }\textcolor{green}{Briefe und Tagebücher}{}\ledrightnote{\textcolor{green}{Dürers Briefe, Tagebücher und Reime}}}{\lemma{\textnormal{\emph{Albrecht … Tagebücher}}}\Cendnote{\textnormal{\emph{\textcolor{green}{\textcolor{blue}{Dürer}s Briefe, Tagebücher und Reime nebst einem
                        Anhange von Zuschriften an und für Dürer}}. Übersetzt und mit Einleitung,
                     Anmerkungen, Personenverzeichniss und einer Reisekarte vershehen von \textcolor{blue}{Moritz Thausing}. Wien:
                        \emph{\textcolor{brown}{Braumüller}}{ }1872. (Quellenschriften für Kunstgeschichte und
                     Kunsttechnik des Mittelalters und der Renaissance 3)}}}\label{K_L02725-1h} (\textsc{\textcolor{brown}{Braumueller}{}\ledrightnote{\textcolor{brown}{Verlag Wilhelm Braumüller}}, \textcolor{pink}{Wien}{}\ledrightnote{\textcolor{pink}{Wien}}}, 1872).\pend
           \pstart
           2.) Mir ſchreiben.\pend
           \pstart
           3.) Fröhliches Neujahr Dir und den Freunden.\pend
           \pstart \spacefill\mbox{P. G.}\pend{}\endnumbering\briefempfaengerindex{Schnitzler, Arthur@\textsc{Schnitzler, Arthur}!zzzGoldmann, Paul@\emph{von Paul Goldmann}!1893-12-281@{{[}28./29.?{]} 12. 1893}|)be}\mylabel{h}  \normalsize

\doendnotes{C}
\bigskip
\vfill

\clearpage

\footnotesize

\lohead{\textsc{register}}

% Definiere theindex-Environment komplett neu ohne reledmac
\makeatletter
\renewenvironment{theindex}{%
  \section*{\indexname}%
  \setlength{\parindent}{0pt}%
  \setlength{\parskip}{0pt plus 0.3pt}%
  \let\item\@idxitem
}{%
  \clearpage
}
\makeatother

\IfFileExists{\jobname-pw.ind}{\input{\jobname-pw.ind}}{}

\end{document}

      