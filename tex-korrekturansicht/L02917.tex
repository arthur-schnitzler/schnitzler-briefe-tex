%% latex-korrekturansicht-vorspann.tex
%% Vorspann für die Korrekturansicht.
%% Lädt die gemeinsame Datei latex-vorspann.tex mit gesetztem Schalter.

\newif\ifkorrekturansicht
\korrekturansichttrue

\input{../tex-inputs/latex-vorspann}


         
         \renewcommand{\erwaehntePersonen}{Personen: Richard Beer-Hofmann, Otto Brahm, Erich Freund, Clementine Goldmann, Paul Schlenther}
         \renewcommand{\erwaehnteInstitutionen}{Institutionen: Burgtheater, Freie literarische Vereinigung zu Breslau}
         \renewcommand{\erwaehnteOrte}{Orte: Alpen, Berlin, Breslau, Dessauer Straße, Dänemark, Kopenhagen, Puchberg am Schneeberg, Schneeberg, Wien}
         \renewcommand{\erwaehnteWerke}{Werke: Der Schleier der Beatrice. Schauspiel in fünf Akten}
               \section[ Paul Goldmann an Arthur Schnitzler, 29. 5. {[}1900{]}]{Paul Goldmann an Arthur Schnitzler, 29. 5. {[}1900{]}}\nopagebreak\mylabel{v}\rehead{ }\normalsize\beginnumbering\briefempfaengerindex{Schnitzler, Arthur@\textsc{Schnitzler, Arthur}!zzzGoldmann, Paul@\emph{von Paul Goldmann}!1900-05-291@{29. 5. {[}1900{]}}|(be} \toendnotes[C]{\smallbreak\pagebreak[2]} \Standort{DLA, A:Schnitzler, HS.NZ85.1.3170.}
\physDesc{Brief, 1 Blatt, 4 Seiten
\newline{}Handschrift: blaue Tinte, deutsche Kurrent
\newline{}Schnitzler: 1) mit Bleistift das Jahr »{[}1{]}900« vermerkt  2) mit rotem Buntstift drei Unterstreichungen}\toendnotes[C]{\smallbreak}\pstart
           {\pb}\textcolor{pink}{\textcolor{gray}{\textbf{DESSAUERSTRASSE 19}}}{}\ledrightnote{\textcolor{pink}{Dessauer Straße}}\hfill \textcolor{pink}{Berlin}{}\ledrightnote{\textcolor{pink}{Berlin}}, 29. Mai.\pend
           \pstart\center{}Mein lieber Freund,\pend\pstart
           Unſere Briefe haben ſich wieder einmal gekreuzt. Es iſt ſchön, daß Du in den \textcolor{pink}{Bergen}{}\ledrightnote{{$\rightarrow$}\textcolor{pink}{Alpen}} biſt, in guter Luft und
               in Ruhe. Wie der \label{K_L02917-2v}\edtext{\textcolor{pink}{Ort}{}\ledrightnote{{$\rightarrow$}\textcolor{pink}{Puchberg am Schneeberg}} am Fuße des \textcolor{pink}{Schneeberg}{}\ledrightnote{\textcolor{pink}{Schneeberg}}s}{\lemma{\textnormal{\emph{Ort … Schneebergs}}}\Cendnote{\textnormal{Es handelte sich wohl um \textcolor{pink}{Puchberg am Schneeberg}. \textcolor{blue}{Schnitzler}
                  hielt sich dort von 24. 5. 1900 bis 27. 5. 1900 auf.}}}\label{K_L02917-2h} heißt, habe ich nicht enziffern können. Über
                  \label{K_L02917-1v}\edtext{\textsc{\textcolor{blue}{Schlenther}{}\ledrightnote{\textcolor{blue}{Paul Schlenther}}}}{\lemma{\textnormal{\emph{Schlenther}}}\Cendnote{\textnormal{\textcolor{blue}{Schlenther} ruderte von der Zusage,
                  die Uraufführung von \emph{\textcolor{green}{Der Schleier der Beatrice}}
                  zu übernehmen, zurück, 
                  siehe Bahr/Schnitzler, T030017}}}\label{K_L02917-1h} ärgere
               Dich nicht. Aufführen muß er \textcolor{green}{Dich}{}\ledrightnote{{$\rightarrow$}\textcolor{green}{Der Schleier der Beatrice. Schauspiel in fünf Akten}} ja doch, ob er will oder nicht. \strikeout{Üb} Im Übrigen iſt er ein erbärmlicher Kerl und wird
                  \label{K_L02917-3v}\edtext{nicht mehr lange das \textcolor{brown}{Burgtheater}{}\ledrightnote{\textcolor{brown}{Burgtheater}} dirigiren}{\lemma{\textnormal{\emph{nicht … dirigiren}}}\Cendnote{\textnormal{\textcolor{blue}{Paul Schlenther} blieb bis 1910 Direktor des \emph{\textcolor{brown}{Burgtheater}}s.}}}\label{K_L02917-3h}. Daß \textsc{\textcolor{blue}{Brahm}{}\ledrightnote{\textcolor{blue}{Otto Brahm}}}{ }\textcolor{green}{Dich}{}\ledrightnote{{$\rightarrow$}\textcolor{green}{Der Schleier der Beatrice. Schauspiel in fünf Akten}} bisher
               nicht aufgeführt hat, iſt begreiflich. Er iſt ein Geſchäftsmann und will zuerſt ſeine
               neuen Stücke bringen, die beſſere {\pb}Einnahmen
               verſprechen, als die ſchon bekannten.\pend
           \pstart
           Ich habe jetzt wieder eine Zeit relativer Ruhe, konnte für mich arbeiten, zermartere
               mir den Kopf und bringe nicht \uline{einen} Gedanken heraus.
               Das verſtimmt mich tief. Ich bin eben offenbar doch nur ein Journaliſt\strikeout{,} und habe kein Recht zu höheren Prätentionen.\pend
           \pstart
           Der \textcolor{blue}{Leiter}{}\ledrightnote{{$\rightarrow$}\textcolor{blue}{Erich Freund}} der \textcolor{pink}{Breslau}{}\ledrightnote{\textcolor{pink}{Breslau}}er \textcolor{brown}{Freien
                  Literariſchen Vereinigung}{}\ledrightnote{\textcolor{brown}{Freie literarische Vereinigung zu Breslau}}, \textsc{Dr. \textcolor{blue}{Erich Freund}{}\ledrightnote{\textcolor{blue}{Erich Freund}}}, der, wie Du weißt, ein Jugendfreund von mir iſt, weilt gegenwärtig in \textcolor{pink}{Berlin}{}\ledrightnote{\textcolor{pink}{Berlin}} und hat mich gebeten, Dich {\pb}zu fragen, ob Du nicht in dieſem Winter einmal
                  \label{K_L02917-4v}\edtext{in \textcolor{pink}{Breslau}{}\ledrightnote{\textcolor{pink}{Breslau}} leſen}{\lemma{\textnormal{\emph{in Breslau leſen}}}\Cendnote{\textnormal{nicht
                  geschehen}}}\label{K_L02917-4h} möchteſt? Die \textcolor{brown}{Leute}{}\ledrightnote{{$\rightarrow$}\textcolor{brown}{Freie literarische Vereinigung zu Breslau}} haben ein ſehr vornehmes Vortrags-Programm, zahlen von 150 \textsc{MK} aufwärts und wären ſehr glücklich, Dich einmal zu
               haben.\pend
           \pstart
           Sommerpläne? Wie ich Dir ſchon geſchrieben habe: Ich wüßte mir natürlich nichts
               Beſſeres, als mit Dir und \textsc{\textcolor{blue}{Richard}{}\ledrightnote{\textcolor{blue}{Richard Beer-Hofmann}}} zuſammen zu ſein, aber ich werde kein Geld haben. Meine Haushalt-Ausgaben
               laufen fort, ob ich hier bin oder nicht, meine \textcolor{blue}{Mutter}{}\ledrightnote{{$\rightarrow$}\textcolor{blue}{Clementine Goldmann}} muß aufs Land, endlich muß ich, wenn ich \textcolor{pink}{hier}{}\ledrightnote{{$\rightarrow$}\textcolor{pink}{Berlin}}{ }{\pb}weggehe, mir einen Vertreter zahlen. Es iſt ſehr
               lieb von Dir, daß Du mir etwas borgen willſt. Aber ich ſehe keine Möglichkeit, wie
               ich Dir das wiedergeben ſoll, und überdies ſchulde ich Dir noch 100 \textsc{Kronen} von \label{K_L02917-5v}\edtext{\textcolor{pink}{Kopenhagen}{}\ledrightnote{\textcolor{pink}{Kopenhagen}}}{\lemma{\textnormal{\emph{Kopenhagen}}}\Cendnote{\textnormal{Die gemeinsame \textcolor{pink}{Dänemark}-Reise im Sommer 1896,
                     siehe Paul Goldmann an Arthur Schnitzler, 7. 9. [1896]}}}\label{K_L02917-5h} her. Wenn alſo bis zum \textsc{Auguſt} nicht ein Wunder geſchieht, werde ich in \textcolor{pink}{Berlin}{}\ledrightnote{\textcolor{pink}{Berlin}} bleiben müſſen.\pend
           \pstart
           Schreib’ mir bald und ſei von Herzen gegrüßt!\pend
           \pstart
           Dein treuer {\\[\baselineskip]}\spacefill\mbox{Paul Goldmann.}\pend
           \leftskip=0em{}\endnumbering\briefempfaengerindex{Schnitzler, Arthur@\textsc{Schnitzler, Arthur}!zzzGoldmann, Paul@\emph{von Paul Goldmann}!1900-05-291@{29. 5. {[}1900{]}}|)be}\mylabel{h}  \normalsize

\doendnotes{C}
\bigskip
\vfill

\clearpage

\footnotesize

\lohead{\textsc{register}}

% Definiere theindex-Environment komplett neu ohne reledmac
\makeatletter
\renewenvironment{theindex}{%
  \section*{\indexname}%
  \setlength{\parindent}{0pt}%
  \setlength{\parskip}{0pt plus 0.3pt}%
  \let\item\@idxitem
}{%
  \clearpage
}
\makeatother

\IfFileExists{\jobname-pw.ind}{\input{\jobname-pw.ind}}{}

\end{document}

      