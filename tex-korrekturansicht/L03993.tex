%% latex-korrekturansicht-vorspann.tex
%% Vorspann für die Korrekturansicht.
%% Lädt die gemeinsame Datei latex-vorspann.tex mit gesetztem Schalter.

\newif\ifkorrekturansicht
\korrekturansichttrue

\input{../tex-inputs/latex-vorspann}


\section[Arthur Schnitzler: Widmungsexemplar von Buch der Sprüche und Bedenken an Berta Zuckerkandl, 17. 12. 1927]{L03993 Arthur Schnitzler: Widmungsexemplar von Buch der Sprüche und Bedenken an Berta
               Zuckerkandl, 17. 12. 1927}
\nopagebreak\mylabel{L03993v}
\rehead{ }\normalsize\beginnumbering\briefempfaengerindex{Zuckerkandl, Berta@\textsc{Zuckerkandl, Berta}!zzzSchnitzler, Arthur@\emph{von Arthur Schnitzler}!1927-12-172@{17. 12. 1927}|(be}
\toendnotes[C]{\smallbreak\pagebreak[2]}
\correspDesc{Versand  durch Arthur Schnitzler am 17. 12. 1927 in Wien
\newline{}Erhalt  durch Berta Zuckerkandl im Zeitraum [17. 12. 1927 – 20. 12. 1927?] in Wien}\toendnotes[C]{\smallbreak}
\Standort{Wien, Österreichische Nationalbibliothek, ZUC.5.1.SchFin LIT MAG.}
\physDesc{Widmung am Schmutztitel, 73 Zeichen
\newline{}Handschrift: schwarze Tinte, lateinische Kurrent}
\pstart
           \noindent{}{\pb}Fr. Hofrätin Bertha Zuckerkandl\pend
           
\pstart
           herzlichſt {\\[\baselineskip]}\spacefill\mbox{ArthSchnitzler}\pend
           \leftskip=0em{}
\pstart
           \textcolor{pink}{Wien}\oindex{Wien@\textbf{Wien}, \emph{Verwaltungsgebiet}|pw}{}\ledrightnote{\textcolor{pink}{Wien}},
                  17. 12. 927\pend
           \selectlanguage{ngerman}\vspace{1em}{\vspace{1\baselineskip}}
\pstart
           \centering{}{\pb}\textcolor{gray}{\textbf{\so{ARTHUR SCHNITZLER}}}\pend
           
\pstart
           \centering{}\textcolor{gray}{\textbf{\textcolor{green}{BUCH DER SPRÜCHE UND BEDENKEN}\pwindex{Schnitzler, Arthur 15. 5. 1862 Wien – 21. 10. 1931 ebd.@\textsc{Schnitzler, Arthur} (15. 5. 1862 Wien – 21. 10. 1931 ebd.), \emph{Schriftsteller, Mediziner}!Buch der Sprüche und Bedenken@\strich\emph{Buch der Sprüche und Bedenken}|pw}{}\ledrightnote{\textcolor{green}{Buch der Sprüche und Bedenken}}}}\pend
           
\pstart
           \centering{}\textcolor{gray}{\textbf{\so{APHORISMEN UND FRAGMENTE}}}\pend
           {\vspace{1\baselineskip}}
\pstart
           \centering{}\textcolor{gray}{\textbf{IM \textcolor{brown}{PHAIDON-VERLAG}\orgindex{Phaidon-Verlag@Phaidon-Verlag|pw}{}\ledrightnote{\textcolor{brown}{Phaidon-Verlag}} ⋅ \textcolor{pink}{WIEN}\oindex{Wien@\textbf{Wien}, \emph{Verwaltungsgebiet}|pw}{}\ledrightnote{\textcolor{pink}{Wien}} ⋅ MCMXXVII}}\pend
           \selectlanguage{ngerman}\endnumbering\briefempfaengerindex{Zuckerkandl, Berta@\textsc{Zuckerkandl, Berta}!zzzSchnitzler, Arthur@\emph{von Arthur Schnitzler}!1927-12-172@{17. 12. 1927}|)be}\mylabel{L03993h}
\begin{anhang}
\end{anhang}\normalsize

\doendnotes{C}
\bigskip
\vfill

\clearpage

\footnotesize

\lohead{\textsc{register}}

% Definiere theindex-Environment komplett neu ohne reledmac
\makeatletter
\renewenvironment{theindex}{%
  \section*{\indexname}%
  \setlength{\parindent}{0pt}%
  \setlength{\parskip}{0pt plus 0.3pt}%
  \let\item\@idxitem
}{%
  \clearpage
}
\makeatother

\IfFileExists{\jobname-pw.ind}{\input{\jobname-pw.ind}}{}

\end{document}

      