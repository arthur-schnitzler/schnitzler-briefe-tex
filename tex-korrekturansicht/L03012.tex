%% latex-korrekturansicht-vorspann.tex
%% Vorspann für die Korrekturansicht.
%% Lädt die gemeinsame Datei latex-vorspann.tex mit gesetztem Schalter.

\newif\ifkorrekturansicht
\korrekturansichttrue

\input{../tex-inputs/latex-vorspann}


\renewcommand{\erwaehntePersonen}{Personen: Felix Salten}
\renewcommand{\erwaehnteOrte}{Orte: Edmund-Weiß-Gasse 7, Wien}
\renewcommand{\erwaehnteWerke}{Werke: Der Weg ins Freie. Roman, Die Zeit, Die neue Rundschau, Schnitzlers Wiener Roman, Tagebuch}
\section[ Arthur Schnitzler an Felix Salten, 30. 5. 1908]{Arthur Schnitzler an Felix Salten, 30. 5. 1908}
\nopagebreak\mylabel{v}
\rehead{ }\normalsize\beginnumbering\briefempfaengerindex{Salten, Felix@\textsc{Salten, Felix}!zzzSchnitzler, Arthur@\emph{von Arthur Schnitzler}!1908-05-301@{30. 5. 1908}|(be}
\toendnotes[C]{\smallbreak\pagebreak[2]}\Standort{Wienbibliothek im Rathaus, ZPH 1681, 2.1.516.}
\physDesc{Brief, 1 Blatt, 2 Seiten, 321 Zeichen
\newline{}Handschrift: schwarze Tinte, deutsche Kurrent
\newline{}Ordnung: mit Bleistift von unbekannter Hand nummeriert: »\textcolor{gray}{1}8« }
\buchAbdrucke{\weitereDrucke{Arthur Schnitzler: \emph{Briefe 1875–1912}. Hg. Therese Nickl und Heinrich Schnitzler. Frankfurt am Main: \emph{S. Fischer} 1981, S. 578.} }\toendnotes[C]{\smallbreak}
\pstart
           \noindent{}{\pb}\textcolor{gray}{\textbf{Dr Arthur Schnitzler}}\hfill 30. 5. 908. \pend
           
\pstart
           \textcolor{gray}{\textbf{\textcolor{pink}{Wien XVIII. Spoettelgasse 7}{}\ledrightnote{\textcolor{pink}{Edmund-Weiß-Gasse 7}}.}}\pend
           
\pstart
           mein lieber, ich ka{\geminationn} Ihnen gar nicht
               ſagen, \uline{wie} ich mich \label{K_L03012-1v}\edtext{gefreut}{\lemma{\textnormal{\emph{gefreut}}}\Cendnote{\textnormal{\textcolor{blue}{Salten} hatte die allererste Rezension von
                     \emph{\textcolor{green}{Der Weg ins Freie}} verfasst: \textcolor{blue}{Felix Salten}: \emph{\textcolor{green}{Schnitzlers Wiener Roman}}. In: \emph{\textcolor{green}{Die Zeit}}, Jg. 7, Nr. 2.042, 30. 5. 1908, Morgenblatt, S. 1–2. Die Rezension verweist auf
                  die Buchausgabe, die ihm aber zu diesem Zeitpunkt höchstens als Vorabexemplar
                  vorgelegen haben dürfte. Wahrscheinlicher ist, dass ihm \textcolor{blue}{Schnitzler} den Text des 6. und (letzten) Teils des
                  Vorabdrucks in der \emph{\textcolor{green}{Der neuen Rundschau}} oder
                  sonst eine Druckfahne zur Verfügung gestellt hatte (vgl. Felix Salten an Arthur Schnitzler, 16. 1. 1908). \textcolor{blue}{Schnitzler} zeigt sich jedenfalls im \emph{\textcolor{green}{Tagebuch}} gerührt: »In der \textcolor{green}{Zeit}{ }\textcolor{green}{Feuilleton}{ }\textcolor{blue}{Salten}’s über den \textcolor{green}{Roman}. Sehr schön; fast ergreifend –
                     ohne Einschränkung. – Schrieb ihm.«}}}\label{K_L03012-1h} habe. Aber Sie können ſichs
               ja denken. Daſs Sie der Erſte ſind, der ſich \textcolor{green}{vernehmen lieſs}{}\ledrightnote{{$\rightarrow$}\textcolor{green}{Schnitzlers Wiener Roman}}, und ſo, gerade ſo, bedeutet mir {\pb}viel – vielleicht \label{K_L03012-2v}\edtext{mehr als Sie vermuthen}{\lemma{\textnormal{\emph{mehr als Sie vermuthen}}}\Cendnote{\textnormal{ x vgl. Felix Salten an Arthur Schnitzler, 26. 1. 1908}}}\label{K_L03012-2h}. An gewiſſen Stellen ſind mir Thränen geko{\geminationm}en.
               »Naja {\dotstwo} weil’s wahr is {\dotstwo}«\pend
           
\pstart
           Von Herzen {\\[\baselineskip]}Ihr \spacefill\mbox{Arthur}\pend
           \leftskip=0em{}\endnumbering\briefempfaengerindex{Salten, Felix@\textsc{Salten, Felix}!zzzSchnitzler, Arthur@\emph{von Arthur Schnitzler}!1908-05-301@{30. 5. 1908}|)be}\mylabel{h}  \normalsize

\doendnotes{C}
\bigskip
\vfill

\clearpage

\footnotesize

\lohead{\textsc{register}}

% Definiere theindex-Environment komplett neu ohne reledmac
\makeatletter
\renewenvironment{theindex}{%
  \section*{\indexname}%
  \setlength{\parindent}{0pt}%
  \setlength{\parskip}{0pt plus 0.3pt}%
  \let\item\@idxitem
}{%
  \clearpage
}
\makeatother

\IfFileExists{\jobname-pw.ind}{\input{\jobname-pw.ind}}{}

\end{document}

      