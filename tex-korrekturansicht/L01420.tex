%% latex-korrekturansicht-vorspann.tex
%% Vorspann für die Korrekturansicht.
%% Lädt die gemeinsame Datei latex-vorspann.tex mit gesetztem Schalter.

\newif\ifkorrekturansicht
\korrekturansichttrue

\input{../tex-inputs/latex-vorspann}


               \section[Arthur Schnitzler an Richard Beer-Hofmann, 28. 7. 1904]{ Arthur Schnitzler an Richard Beer-Hofmann, 28. 7. 1904}\nopagebreak\mylabel{v}\rehead{ }\normalsize\beginnumbering\briefempfaengerindex{Beer-Hofmann, Richard@\textsc{Beer-Hofmann, Richard}!zzzSchnitzler, Arthur@\emph{von Arthur Schnitzler}!1904-07-281@{28. 7. 1904}|(be} \toendnotes[C]{\smallbreak\pagebreak[2]} \Standort{YCGL, MSS 31.}
\physDesc{Brief, 1 Blatt, 4 Seiten, Umschlag
\newline{}Handschrift: Bleistift, deutsche Kurrent\newline{}Versand: 1) Stempel: »\nobreak{}\oindex{I., Innere Stadt@\textbf{I., Innere Stadt}, \emph{Bezirk (A.BZK)}|pwk}Wien 1/1, 28. VII. 04, 12\nobreak{}«.  2) Stempel: »\nobreak{}\oindex{Bad Aussee@\textbf{Bad Aussee}, \emph{Besiedelter Ort (A.BSO)}|pwk}Aussee in Steiermark, 29 7 04\nobreak{}«. }\buchAbdrucke{\weitereDrucke{Arthur Schnitzler, Richard Beer-Hofmann: \emph{Briefwechsel 1891–1931}. Hg. Konstanze Fliedl. Wien, Zürich: \emph{Europaverlag} 1992, S. 164–165.} }\toendnotes[C]{\smallbreak}\pstart{}{\pb}\textsc{A. Schn. XIII \textcolor{pink}{Spöttelg. 7}{}\ledrightnote{\textcolor{pink}{Edmund-Weiß-Gasse}}}\pend{}{\bigskip}\pstart{}{\pb}\textsc{Dr Richard Beer-Hofma{\geminationn}}\pend{}\pstart{}\textsc{\textcolor{pink}{Markt Aussee}{}\ledrightnote{\textcolor{pink}{Bad Aussee}}}\pend{}\pstart{}\textcolor{pink}{\textsc{Villa Frühling}}{}\ledrightnote{\textcolor{pink}{Villa Frühling}}\pend{}{\bigskip}\pstart
           \raggedleft{}{\pb}28. 7. 904\pend
           \pstart
           lieber Richard – ich hatte mir wirklich ſchon eingebildet – es
               könnte ein Brief ſein – aber auch für den Theaterzettel mit Gruſs und Spaſs danke ich
               Ihnen herzlich. Wir waren etwa 14 Tage \introOben{}(\introOben{}mit \textcolor{blue}{Mama}{}\ledrightnote{→\textcolor{blue}{Louise Schnitzler}}\introOben{})\introOben{} in \textcolor{pink}{Reichenau}{}\ledrightnote{\textcolor{pink}{Reichenau an der Rax}}, ſind Samſtag zurück; es war wunderſchön,
                  {\pb}ich war im \textcolor{pink}{Naßwald}{}\ledrightnote{\textcolor{pink}{Nasswald}} und endlich ſogar auf der Rax,
               habe etliches gearbeitet, und was meine Geſundheit anbelangt, ſo iſt ſie eigentlich
                  ko{\geminationm}t mir vor beſſer als \uline{vor} der Gelbſucht. Nun bleiben wir wahrſcheinlich (\introOben{}von\introOben{} Ausflüg\textcolor{gray}{en} von ein paar Tagen abge{\pb}ſehen) bis Ende Auguſt hier und fahren
                  da{\geminationn} vielleicht auf 10–14 Tage nach \textcolor{pink}{Iſchl}{}\ledrightnote{\textcolor{pink}{Bad Ischl}} bei welcher Gelegenheit ich Sie hoffentlich ſehen und –
               als letzter unter den {\dots} »Näheren« das \textcolor{green}{Stück}{}\ledrightnote{→\textcolor{green}{Der Graf von Charolais. Ein Trauerspiel}} hören werde, von dem mir \textcolor{blue}{Salten}{}\ledrightnote{\textcolor{blue}{Felix Salten}} vorgeſtern höchſt begeiſtert ſprach. Ich denke, {\pb}Sie ſind bald fertig? –\pend
           \pstart
           Schreiben Sie mir bald, we{\geminationn} auch nur eine Zeile, auch
               wie es Ihnen allen geht. –\pend
           \pstart
           Mein Balkon iſt ein Luftkurort (heute übrigens beinah ein Sturmkurort)\pend
           \pstart
           Wir grüßen Sie Beide\footnote{\noindent{}Subjekt} Beide\footnote{\noindent{}Objekt.}\pend
           \pstart
           Von Herzen{\\[\baselineskip]}Ihr \spacefill\mbox{A.}\pend
           \leftskip=0em{}\endnumbering\briefempfaengerindex{Beer-Hofmann, Richard@\textsc{Beer-Hofmann, Richard}!zzzSchnitzler, Arthur@\emph{von Arthur Schnitzler}!1904-07-281@{28. 7. 1904}|)be}\mylabel{h}  \normalsize

\doendnotes{C}
\bigskip
\vfill

\clearpage

\footnotesize

\lohead{\textsc{register}}

% Definiere theindex-Environment komplett neu ohne reledmac
\makeatletter
\renewenvironment{theindex}{%
  \section*{\indexname}%
  \setlength{\parindent}{0pt}%
  \setlength{\parskip}{0pt plus 0.3pt}%
  \let\item\@idxitem
}{%
  \clearpage
}
\makeatother

\IfFileExists{\jobname-pw.ind}{\input{\jobname-pw.ind}}{}

\end{document}

      