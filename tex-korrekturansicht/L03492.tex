%% latex-korrekturansicht-vorspann.tex
%% Vorspann für die Korrekturansicht.
%% Lädt die gemeinsame Datei latex-vorspann.tex mit gesetztem Schalter.

\newif\ifkorrekturansicht
\korrekturansichttrue

\input{../tex-inputs/latex-vorspann}


\renewcommand{\erwaehntePersonen}{Personen: Hugo von Hofmannsthal, Margarethe Kainz, Felix Salten, Julius Schnitzler, Olga Schnitzler, Louise Schnitzler, Jakob Wassermann}
\renewcommand{\erwaehnteOrte}{Orte: Biberstraße, Brünn, Heiligenstadt, I., Innere Stadt, Ordination Julius Schnitzler, Semmering, Südbahnhotel, Wien}
\renewcommand{\erwaehnteWerke}{}
\section[ Felix Salten an Arthur Schnitzler, 8. 2. 1908]{Felix Salten an Arthur Schnitzler, 8. 2. 1908}
\nopagebreak\mylabel{v}
\rehead{ }\normalsize\beginnumbering\briefempfaengerindex{Schnitzler, Arthur@\textsc{Schnitzler, Arthur}!zzzSalten, Felix@\emph{von Felix Salten}!1908-02-081@{8. 2. 1908}|(be}
\toendnotes[C]{\smallbreak\pagebreak[2]}\Standort{CUL, Schnitzler, B 89, B 1.}
\physDesc{Postkarte, 699 Zeichen
\newline{}Handschrift: schwarze Tinte, lateinische Kurrent
\newline{}Versand: 1) Stempel: »\nobreak{}\oindex{I., Innere Stadt@\textbf{I., Innere Stadt}, \emph{A.ADM3}|pwk}1/\textsubscript{1} Wien 1, 8. II. 08, 12\nobreak{}«.   2) mit Bleistift von unbekannter Hand der Vorname \textcolor{blue}{Schnitzler}s in der Adressangabe gestrichen
\newline{}Schnitzler: mit Bleistift datiert: »8/2 908« 
\newline{}Ordnung: mit Bleistift von unbekannter Hand nummeriert: »{\pb}242« }\toendnotes[C]{\smallbreak}\pstart{}{\pb}Herrn D\textsuperscript{r} Arthur Schnitzler\pend{}\pstart{}\textcolor{pink}{Semmering}{}\ledrightnote{\textcolor{pink}{Semmering}}\pend{}\pstart{}\textcolor{pink}{Südbahnhotel}{}\ledrightnote{\textcolor{pink}{Südbahnhotel}}\pend{}
{\bigskip}
\pstart
           \noindent{}Lieber, wir waren erst gegen 2\textsuperscript{h} in \textcolor{pink}{Wien}{}\ledrightnote{\textcolor{pink}{Wien}}, \label{K_L03492-1v}\edtext{¾ 3}{\lemma{\textnormal{\emph{¾ 3}}}\Cendnote{\textnormal{14 Uhr 45}}}\label{K_L03492-1h} in \textcolor{pink}{Heiligenstadt}{}\ledrightnote{\textcolor{pink}{Heiligenstadt}}, wo wir
               essen mußten. Wir haben Ihrem Herrn \textcolor{blue}{Bruder}{}\ledrightnote{{$\rightarrow$}\textcolor{blue}{Julius Schnitzler}} gleich telefonirt, fuhren auch ohne Verzögerung in
               die \textcolor{pink}{Stadt}{}\ledrightnote{{$\rightarrow$}\textcolor{pink}{I., Innere Stadt}}, aber bei dem heftigen
               Sturm kamen die Pferde nur schwer vorwärts. Und als wir mit einer Verspätung um \uline{10} Minuten in die \label{K_L03492-2v}\edtext{\textcolor{pink}{Biberstraße}{}\ledrightnote{\textcolor{pink}{Biberstraße}}}{\lemma{\textnormal{\emph{Biberstraße}}}\Cendnote{\textnormal{In der \textcolor{pink}{Biberstraße 8} befand sich \textcolor{blue}{Julius
                     Schnitzler}s chirurgische \textcolor{pink}{Ordination}.}}}\label{K_L03492-2h} kamen, wurden wir nicht mehr angenommen. Mir that es sehr
               leid, umso mehr, als ich ja eigens wegen dieser Consultation um 10.17
               vom \textcolor{pink}{Semmering}{}\ledrightnote{\textcolor{pink}{Semmering}} weg bin und nicht mit dem
               Schnell-Zug.\pend
           
\pstart
           Vielleicht komme ich am Montag{ }früh, oder um 2\textsuperscript{h.} von \textcolor{pink}{Brünn}{}\ledrightnote{\textcolor{pink}{Brünn}} aus noch einmal für einen Tag
                  \textcolor{pink}{hinauf}{}\ledrightnote{{$\rightarrow$}\textcolor{pink}{Semmering}}. Grüßen Sie \uline{Alle}, Ihre \textcolor{blue}{Frau}{}\ledrightnote{{$\rightarrow$}\textcolor{blue}{Olga Schnitzler}}, Ihre \textcolor{blue}{Mama}{}\ledrightnote{{$\rightarrow$}\textcolor{blue}{Louise Schnitzler}}, \textcolor{blue}{Hofmannsthal}{}\ledrightnote{\textcolor{blue}{Hugo von Hofmannsthal}},
                  \textcolor{blue}{Wassermann}{}\ledrightnote{\textcolor{blue}{Jakob Wassermann}} u. Frau \textcolor{blue}{Kainz}{}\ledrightnote{\textcolor{blue}{Margarethe Kainz}}. Herzlichst Ihr \spacefill\mbox{Salten}\pend
           \endnumbering\briefempfaengerindex{Schnitzler, Arthur@\textsc{Schnitzler, Arthur}!zzzSalten, Felix@\emph{von Felix Salten}!1908-02-081@{8. 2. 1908}|)be}\mylabel{h}  \normalsize

\doendnotes{C}
\bigskip
\vfill

\clearpage

\footnotesize

\lohead{\textsc{register}}

% Definiere theindex-Environment komplett neu ohne reledmac
\makeatletter
\renewenvironment{theindex}{%
  \section*{\indexname}%
  \setlength{\parindent}{0pt}%
  \setlength{\parskip}{0pt plus 0.3pt}%
  \let\item\@idxitem
}{%
  \clearpage
}
\makeatother

\IfFileExists{\jobname-pw.ind}{\input{\jobname-pw.ind}}{}

\end{document}

      