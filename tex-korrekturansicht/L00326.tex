%% latex-korrekturansicht-vorspann.tex
%% Vorspann für die Korrekturansicht.
%% Lädt die gemeinsame Datei latex-vorspann.tex mit gesetztem Schalter.

\newif\ifkorrekturansicht
\korrekturansichttrue

\input{../tex-inputs/latex-vorspann}


               \section[Friedrich M. Fels an Arthur Schnitzler, {[}17. 5. 1894{]}]{ Friedrich M. Fels an Arthur Schnitzler, {[}17. 5. 1894{]}}\nopagebreak\mylabel{v}\rehead{ }\normalsize\beginnumbering\briefempfaengerindex{Schnitzler, Arthur@\textsc{Schnitzler, Arthur}!zzzFels, Friedrich Michael@\emph{von Friedrich Michael Fels}!1894-05-171@{{[}17. 5. 1894{]}}|(be} \toendnotes[C]{\smallbreak\pagebreak[2]} \Standort{DLA, A:Schnitzler, HS.NZ85.1.2956.}
\physDesc{Brief, 1 Blatt, 2 Seiten
\newline{}Handschrift: schwarze Tinte, lateinische Kurrent
\newline{}Schnitzler: mit Bleistift datiert »17/5 94« und nummeriert: »12« }\toendnotes[C]{\smallbreak}\pstart{}{\pb}Lieber Dr. Schnitzler!\pend\pstart
           I. Verzeihen Sie mir den unfrankierten Brief; aber we{\geminationn} ich mich auf den Kopf stelle, ko{\geminationm}en keine 3 Kr
                    zum Vorschein. Ich müsste also höchstens Ihr »\textcolor{green}{Mährchen}{}\ledrightnote{\textcolor{green}{Das Märchen. Schauspiel in drei Aufzügen}}« zum Antiquar tragen – und da zahlen Sie jedenfalls lieber
                    Strafporto. Verzeihen Sie ferner das kaum recht dicke Papier; aber {\dots} Grund wie vorhin.\pend
           \pstart
           II. Da Sie die Liebenswürdigkeit hatten, \textcolor{blue}{Beer-Hofma{\geminationn}}{}\ledrightnote{\textcolor{blue}{Richard Beer-Hofmann}} zu schreiben, haben Sie vielleicht die grössere Liebenswürdigkeit, ihm
                    noch einmal zu schreiben. Ganz abgesehen davon, dass ich, im Vertrauen auf ihn,
                    so leichtgläubig war, vorgestern ordentlich zu essen und den ganzen von Ihnen
                    erhaltenen Gulden aufzubrauchen, dass ich also seit vorgestern gar nichts zum
                    Leben habe, wäre es mir wirklich unangenehm und ein Verlust, we{\geminationn} ich nicht baldmöglichst in die Kunstausstellung
                    und am Samstag zum \textcolor{brown}{Augartenfest}{}\ledrightnote{\textcolor{brown}{Augartenfest}}
                    gehen kö{\geminationn}te. Also bitte, schreiben Sie \textcolor{blue}{Beer-Hofma{\geminationn}}{}\ledrightnote{\textcolor{blue}{Richard Beer-Hofmann}} nochmals und entschuldigen Sie mir die Mühe, die ich Ihnen verursache. Ich
                    wollte Sie heute früh aufsuchen; doch Ihre Betten hingen bereits {\pb}unter dem Fenster, dass Sie kaum zu Hause waren;
                    auch wollte die elektrische Klingel durchaus nicht »thun«.\pend
           \pstart
           III. Um die Annehmlichkeiten meines Lebens voll zu machen, scheint meine \textcolor{blue}{Hauswirthin}{}\ledrightnote{→\textcolor{blue}{?? [Vermieterin]}} im Sterben zu
                    liegen. Offen gestanden, ich fühle kein Mitleid mit dem armen, jungen Weib, viel
                    eher ein bischen Neid auf \substVorne{}\textsuperscript{S}\substDazwischen{}s\substHinten{}ie.\pend
           \pstart
           Bestens grüsst{\\[\baselineskip]}Ihr{\\[\baselineskip]}dankbarergebener{\\[\baselineskip]}\spacefill\mbox{Fels}\pend
           \leftskip=0em{}\pstart
           \noindent{}\textcolor{pink}{Wien XVIII, Exnergasse 3\textsuperscript{III. St. Th. 22}}{}\ledrightnote{\textcolor{pink}{Krütznergasse}}\pend
           \pstart
           N. B. Ich merke jetzt, dass der \uline{letzte} Satz
                        sehr nach Pose ausschaut; aber, nach gründlicher Gewissenserforschung, muss
                        ich sagen, dass ich, als ich ihn niederschrieb; durchaus nicht an Pose
                        gedacht \label{T_L00326_1v}\edtext{habe}{\lemma{\textnormal{\emph{habe}}}\Cendnote{\textnormal{Er
                            schreibt: »haben«.}}}\label{T_L00326_1h}. Bitte, von dieser
                        Rechtfertigung Notiz zu nehmen. \spacefill\mbox{F}\pend
           \endnumbering\briefempfaengerindex{Schnitzler, Arthur@\textsc{Schnitzler, Arthur}!zzzFels, Friedrich Michael@\emph{von Friedrich Michael Fels}!1894-05-171@{{[}17. 5. 1894{]}}|)be}\mylabel{h}  \normalsize

\doendnotes{C}
\bigskip
\vfill

\clearpage

\footnotesize

\lohead{\textsc{register}}

% Definiere theindex-Environment komplett neu ohne reledmac
\makeatletter
\renewenvironment{theindex}{%
  \section*{\indexname}%
  \setlength{\parindent}{0pt}%
  \setlength{\parskip}{0pt plus 0.3pt}%
  \let\item\@idxitem
}{%
  \clearpage
}
\makeatother

\IfFileExists{\jobname-pw.ind}{\input{\jobname-pw.ind}}{}

\end{document}

      