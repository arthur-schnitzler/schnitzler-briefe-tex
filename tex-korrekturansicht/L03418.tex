%% latex-korrekturansicht-vorspann.tex
%% Vorspann für die Korrekturansicht.
%% Lädt die gemeinsame Datei latex-vorspann.tex mit gesetztem Schalter.

\newif\ifkorrekturansicht
\korrekturansichttrue

\input{../tex-inputs/latex-vorspann}


\renewcommand{\erwaehntePersonen}{Personen: Ernst Moritz Geyger}
\renewcommand{\erwaehnteOrte}{Orte: Berlin, Edmund-Weiß-Gasse 7, Garten von Sanssouci, Schloss Sanssouci, Wien}
\renewcommand{\erwaehnteWerke}{Werke: Bogenschütze}
\section[ Felix Salten an Arthur Schnitzler, 15. 4. 1906]{Felix Salten an Arthur Schnitzler, 15. 4. 1906}
\nopagebreak\mylabel{v}
\rehead{ }\normalsize\beginnumbering\briefempfaengerindex{Schnitzler, Arthur@\textsc{Schnitzler, Arthur}!zzzSalten, Felix@\emph{von Felix Salten}!1906-04-151@{15. 4. 1906}|(be}
\toendnotes[C]{\smallbreak\pagebreak[2]}\Standort{CUL, Schnitzler, B 89, B 1.}
\physDesc{Bildpostkarte, 126 Zeichen
\newline{}Handschrift: schwarze Tinte, lateinische Kurrent
\newline{}Versand: Stempel: »\nobreak{}\oindex{Berlin@\textbf{Berlin}, \emph{P.PPLC}|pwk}Berlin\textcolor{gray}{,}
                                       N. W. f, 16. 4. 06, 3–4 N.\nobreak{}«.  
\newline{}Ordnung: mit Bleistift von unbekannter Hand nummeriert: »209« }\pstart{}{\pb}Herrn D\textsuperscript{r} Arthur Schnitzler\pend{}\pstart{}\textcolor{pink}{Wien XVIII.}{}\ledrightnote{\textcolor{pink}{Berlin}}\pend{}\pstart{}\textcolor{pink}{Spöttelgasse 7}{}\ledrightnote{\textcolor{pink}{Edmund-Weiß-Gasse 7}}\pend{}
{\bigskip}
\pstart
           \noindent{}{\pb}\textcolor{gray}{\textbf{Gruss aus \textcolor{pink}{Sanssouci-Potsdam}{}\ledrightnote{\textcolor{pink}{Schloss Sanssouci}}}}\hfill \textcolor{gray}{\textbf{\textcolor{pink}{Sicilianischer Garten}{}\ledrightnote{\textcolor{pink}{Garten von Sanssouci}} mit \textcolor{green}{Bogenschützen}{}\ledrightnote{\textcolor{green}{Bogenschütze}} v. Prof. \textcolor{blue}{Geiger}{}\ledrightnote{\textcolor{blue}{Ernst Moritz Geyger}}}}\pend
           
\pstart
           Auf einer schönen Automobiltour\pend
           
\pstart
           herzliche Grüße{\\[\baselineskip]}\spacefill\mbox{Salten}\pend
           \leftskip=0em{}
\pstart
           {\pb}Ostersonntag 15. IV. 06.\pend
           \endnumbering\briefempfaengerindex{Schnitzler, Arthur@\textsc{Schnitzler, Arthur}!zzzSalten, Felix@\emph{von Felix Salten}!1906-04-151@{15. 4. 1906}|)be}\mylabel{h}  \normalsize

\doendnotes{C}
\bigskip
\vfill

\clearpage

\footnotesize

\lohead{\textsc{register}}

% Definiere theindex-Environment komplett neu ohne reledmac
\makeatletter
\renewenvironment{theindex}{%
  \section*{\indexname}%
  \setlength{\parindent}{0pt}%
  \setlength{\parskip}{0pt plus 0.3pt}%
  \let\item\@idxitem
}{%
  \clearpage
}
\makeatother

\IfFileExists{\jobname-pw.ind}{\input{\jobname-pw.ind}}{}

\end{document}

      