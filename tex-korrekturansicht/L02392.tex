%% latex-korrekturansicht-vorspann.tex
%% Vorspann für die Korrekturansicht.
%% Lädt die gemeinsame Datei latex-vorspann.tex mit gesetztem Schalter.

\newif\ifkorrekturansicht
\korrekturansichttrue

\input{../tex-inputs/latex-vorspann}


               \section[Thomas Mann an Arthur Schnitzler, 4. 9. 1922]{ Thomas Mann an Arthur Schnitzler, 4. 9. 1922}\nopagebreak\mylabel{v}\rehead{ }\normalsize\beginnumbering\briefempfaengerindex{Schnitzler, Arthur@\textsc{Schnitzler, Arthur}!zzzMann, Thomas@\emph{von Thomas Mann}!1922-09-041@{4. 9. 1922}|(be} \toendnotes[C]{\smallbreak\pagebreak[2]} \Standort{CUL, Schnitzler, B 67.}
\physDesc{Brief, 1 Blatt, 3 Seiten
\newline{}Handschrift: schwarze Tinte, deutsche Kurrent
\newline{}Schnitzler: 1) mit Bleistift beschriftet: »\textsc{Thomas Mann}«, von unbekannter Hand »abg.« (für:
                                 abgeschrieben) 2) mit rotem Buntstift mehrere Unterstreichungen}\buchAbdrucke{\weitereDrucke{1) Thomas Mann: \emph{Briefe 1889–1936}. Mann, Erika. Frankfurt am Main: \emph{S. Fischer} 1961, S. 199.} \weitereDrucke{2) Hertha Krotkoff: \emph{Arthur Schnitzler – Thomas Mann: Briefe.} In: \emph{Modern Austrian Literature}, Jg. 7 (1974) Nr. 1/2, S. 18–19.} }\toendnotes[C]{\smallbreak}\pstart
           \raggedleft{}{\pb}\textcolor{pink}{München}{}\ledrightnote{\textcolor{pink}{München}} den 4. IX. 22.\pend
           \pstart{}Verehrter Herr Dr. Schnitzler,\pend\pstart
           ich habe Ihnen noch zu danken für die gütigen Zeilen, die mir Mr. \textcolor{blue}{Thayer}{}\ledrightnote{\textcolor{blue}{Scofield Thayer}}, ein wirklich ſehr ſympathiſcher junger Mann, von Ihnen
               überbrachte. Es haben ſich aus dieſer Bekanntſchaft geſchäftliche Abmachungen
               ergeben, die mir als hochgradigem Familienvater höchſt angenehm ſein müſſen.\pend
           \pstart
           Eine große Freude war es mir, bei Gelegenheit Ihres 60. Geburtstags von der Liebe zu
                  \textcolor{green}{zeugen}{}\ledrightnote{→\textcolor{green}{Arthur Schnitzler zu seinem sechzigsten Geburtstag}}, mit der ich Ihrem
               bezaubernden Lebenswerk anhänge. Eben leſe ich \textcolor{green}{Caſanovas
                  Heimkehr}{}\ledrightnote{\textcolor{green}{Casanovas Heimfahrt}} – die Novelle war mir ſonderbarer Weiſe bisher unbekannt geblieben –
               und kann die tiefe Zufriedenheit nicht ſchildern, mit der ich {\pb}mich von Ihrer Erzählungskunſt tragen
               laſſe.\pend
           \pstart
           Im \label{K_L02392_1v}\edtext{Oktober-Heft}{\lemma{\textnormal{\emph{Oktober-Heft}}}\Cendnote{\textnormal{Es wurde November
                  (S. 1072–1106).}}}\label{K_L02392_1h} der \textcolor{green}{Neuen Rundſchau}{}\ledrightnote{\textcolor{green}{Die neue Rundschau}} werden
               Sie einen größeren Beitrag von mir finden, einen Aufſatz, betitelt »\textcolor{green}{Von deutſcher Republik}{}\ledrightnote{\textcolor{green}{Von deutscher Republik. Gerhart Hauptmann zum sechzigsten Geburtstag}}«, der vielleicht gar durch zwei Hefte wird
               fortgeſetzt werden müſſen. Ich ermahne darin die renitenten Teile unſerer Jugend und
               unſeres Bürgertums ſich endlich vorbehaltlos in den Dienſt der Republik und der
               Humanität zu ſtellen, – eine Tendenz, über die Sie vielleicht erſtaunt ſein werden.
               Aber gerade als Verfaſſer der »\textcolor{green}{Betrachtungen eines
                  Unpolitiſchen}{}\ledrightnote{\textcolor{green}{Betrachtungen eines Unpolitischen}}« glaubte ich meinem Lande ein ſolches Manifeſt in dieſem
               Augenblick ſchuldig zu ſein. Und was die Verliebtheit in den Gedanken der Humanität
               betrifft, die ich ſeit einiger Zeit bei mir feſtstelle, ſo mag ſie mit dem \textcolor{green}{Roman}{}\ledrightnote{→\textcolor{green}{Der Zauberberg. Roman}} zuſammenhängen, an dem ich
               ſchon {\pb}allzu lange ſchreibe, einer Art von
               Bildungsgeſchichte und \textcolor{green}{Wilhelm
                  Meiſter}{}\ledrightnote{→\textcolor{green}{Wilhelm Meister}}iade, worin ein junger Menſch (vor dem Kriege) durch das Erlebnis der
               Krankheit und des Todes zur Idee des Menſchen und des Staates geführt wird. –
               Verzeihen Sie die unerbetene Vertraulichkeit! – –\pend
           \pstart
           Im Oktober werde ich Ihren Spuren in \textcolor{pink}{Holland}{}\ledrightnote{\textcolor{pink}{Niederlande}} folgen. Im Januar{ }ſoll ich \textcolor{pink}{Wien}{}\ledrightnote{\textcolor{pink}{Wien}}
               wiederſehen und damit, ſo hoffe ich, Sie. Ich freue mich ſehr darauf.\pend
           \pstart
           In herzlicher Ehrerbietung Sie grüßend bin ich, lieber Herr Dr.
               Schnitzler,{\\[\baselineskip]}Ihr ergebenſter{\\[\baselineskip]}\spacefill\mbox{Thomas Mann.}\pend
           \leftskip=0em{}\endnumbering\briefempfaengerindex{Schnitzler, Arthur@\textsc{Schnitzler, Arthur}!zzzMann, Thomas@\emph{von Thomas Mann}!1922-09-041@{4. 9. 1922}|)be}\mylabel{h}  \normalsize

\doendnotes{C}
\bigskip
\vfill

\clearpage

\footnotesize

\lohead{\textsc{register}}

% Definiere theindex-Environment komplett neu ohne reledmac
\makeatletter
\renewenvironment{theindex}{%
  \section*{\indexname}%
  \setlength{\parindent}{0pt}%
  \setlength{\parskip}{0pt plus 0.3pt}%
  \let\item\@idxitem
}{%
  \clearpage
}
\makeatother

\IfFileExists{\jobname-pw.ind}{\input{\jobname-pw.ind}}{}

\end{document}

      