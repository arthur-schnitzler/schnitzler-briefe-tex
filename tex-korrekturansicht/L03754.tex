%% latex-korrekturansicht-vorspann.tex
%% Vorspann für die Korrekturansicht.
%% Lädt die gemeinsame Datei latex-vorspann.tex mit gesetztem Schalter.

\newif\ifkorrekturansicht
\korrekturansichttrue

\input{../tex-inputs/latex-vorspann}


\section[Arthur Schnitzler an Stefan Zweig, 6. 11. 1924]{L03754 Arthur Schnitzler an Stefan Zweig, 6. 11. 1924}
\nopagebreak\mylabel{L03754v}
\rehead{ }\normalsize\beginnumbering\briefempfaengerindex{Zweig, Stefan@\textsc{Zweig, Stefan}!zzzSchnitzler, Arthur@\emph{von Arthur Schnitzler}!1924-11-062@{6. 11. 1924}|(be}
\toendnotes[C]{\smallbreak\pagebreak[2]}
\correspDesc{Versand  durch Arthur Schnitzler am 6. 11. 1924 in Wien
\newline{}Erhalt  durch Stefan Zweig im Zeitraum [7. 11. 1924 – 11. 11. 1924?] in Salzburg}\toendnotes[C]{\smallbreak}
\Standort{Jerusalem, National Library of Israel, ARC. Ms. Var. 305 1 58 Stefan Zweig Collection.}
\physDesc{Brief, 1 Blatt, 1 Seite, 1509 Zeichen
\newline{}Schreibmaschine
\newline{}Handschrift: Bleistift, lateinische Kurrent (\noindent{}minimale Korrekturen, Schlussformel, Unterschrift)}
\buchAbdrucke{\weitereDrucke{Arthur Schnitzler: \emph{Briefe 1913–1931}. Herausgegeben von Peter Michael Braunwarth, Richard Miklin, Susanne Pertlik und Heinrich Schnitzler. Frankfurt am Main: \emph{S. Fischer} 1984, S. 372–373.} }\toendnotes[C]{\smallbreak}
\pstart
           {\pb}\textcolor{gray}{\textbf{D\textsuperscript{R} ARTHUR SCHNITZLER}}\hfill {\pb}6. 11. 1924.\pend
           
\pstart
           \textcolor{gray}{\textbf{\textcolor{pink}{WIEN, XVIII.
                           STERNWARTESTRASSE 71}\oindex{Wien@\textbf{Wien}!XVIII., Währing@\textbf{XVIII., Währing}!Sternwartestraße 71@\textbf{Sternwartestraße 71}, \emph{Wohngebäude}|pw}{}\ledrightnote{\textcolor{pink}{Sternwartestraße 71}}.}}\pend
           
\pstart{}Lieber Herr Dr. Zweig.\pend\vspace{0.5em}
\pstart
           Es freut mich herzlich, dass Ihnen das »\textcolor{green}{Fräulein
                  Else}\pwindex{Schnitzler, Arthur 15. 5. 1862 Wien – 21. 10. 1931 ebd.@\textsc{Schnitzler, Arthur} (15. 5. 1862 Wien – 21. 10. 1931 ebd.), \emph{Schriftsteller, Mediziner}!Fräulein Else@\strich\emph{Fräulein Else}|pw}{}\ledrightnote{\textcolor{green}{Fräulein Else}}« so wohlgefällt. Eine trouvaille ist es \strikeout{ja} eigentlich nicht, dieselbe
               Technik habe ich ja im »\textcolor{green}{Leutnant Gustl}\pwindex{Schnitzler, Arthur 15. 5. 1862 Wien – 21. 10. 1931 ebd.@\textsc{Schnitzler, Arthur} (15. 5. 1862 Wien – 21. 10. 1931 ebd.), \emph{Schriftsteller, Mediziner}!Lieutenant Gustl. Novelle@\strich\emph{Lieutenant Gustl. Novelle}|pw}{}\ledrightnote{\textcolor{green}{Lieutenant Gustl. Novelle}}« schon
               angewandt. Es ist eigentlich merkwürdig, dass sie seitdem so selten benützt wurde, da
               sie ganz ausserordentliche Möglichkeiten bietet. Freilich eignen sich nur wenige
               Sujets dazu, sonst hätte wahrscheinlich vor allem ich selbst von dieser Form öfters
               Gebrauch gemacht. Als der »\textcolor{green}{Leutnant Gustl}\pwindex{Schnitzler, Arthur 15. 5. 1862 Wien – 21. 10. 1931 ebd.@\textsc{Schnitzler, Arthur} (15. 5. 1862 Wien – 21. 10. 1931 ebd.), \emph{Schriftsteller, Mediziner}!Lieutenant Gustl. Novelle@\strich\emph{Lieutenant Gustl. Novelle}|pw}{}\ledrightnote{\textcolor{green}{Lieutenant Gustl. Novelle}}« neu
               war sagte man mir, dass in einer Novelle von \textcolor{blue}{Dujardin}\pwindex{Dujardin, Édouard 10.\,10.\,1861 Saint-Gervais-la-Forêt – 31.\,10.\,1949 Paris@\textsc{Dujardin, Édouard} (10.\,10.\,1861 Saint-Gervais-la-Forêt – 31.\,10.\,1949 Paris), \emph{Schriftsteller}|pw}{}\ledrightnote{\textcolor{blue}{Édouard Dujardin}} »\textcolor{green}{Les Lau\strikeout{r}riers sont coupé\strikeout{é}s}\pwindex{Dujardin, Édouard 10.\,10.\,1861 Saint-Gervais-la-Forêt – 31.\,10.\,1949 Paris@\textsc{Dujardin, Édouard} (10.\,10.\,1861 Saint-Gervais-la-Forêt – 31.\,10.\,1949 Paris), \emph{Schriftsteller}!lauriers sont coupés@\strich\emph{Les lauriers sont coupés}|pw}{}\ledrightnote{\textcolor{green}{Les lauriers sont coupés}}«
               eine ähnliche Technik angewandt worden sei; die Angabe stimmte nicht ganz. Nach
                  \label{K_L03754-1v}\edtext{\textcolor{blue}{Georg Brandes}\pwindex{Brandes, Georg 4.\,2.\,1842 Kopenhagen – 19.\,2.\,1927 ebd.@\textsc{Brandes, Georg} (4.\,2.\,1842 Kopenhagen – 19.\,2.\,1927 ebd.)|pw}{}\ledrightnote{\textcolor{blue}{Georg Brandes}} sollte die »\textcolor{green}{Krotkaja}\pwindex{Dostojevskij, Fjodor Mihajlovič 11.\,11.\,1821 Moskau – 9.\,2.\,1881 Sankt Petersburg@\textsc{Dostojevskij, Fjodor Mihajlovič} (11.\,11.\,1821 Moskau – 9.\,2.\,1881 Sankt Petersburg), \emph{Schriftsteller}!Sanfte@\strich\emph{Die Sanfte}|pw}{}\ledrightnote{\textcolor{green}{Die Sanfte}}}{\lemma{\textnormal{\emph{Georg … »Krotkaja}}}\Cendnote{\textnormal{Siehe Georg Brandes an Arthur Schnitzler, 16. 6. 1901.
               }}}\label{K_L03754-1}« von \textcolor{blue}{Dostojewsky}\pwindex{Dostojevskij, Fjodor Mihajlovič 11.\,11.\,1821 Moskau – 9.\,2.\,1881 Sankt Petersburg@\textsc{Dostojevskij, Fjodor Mihajlovič} (11.\,11.\,1821 Moskau – 9.\,2.\,1881 Sankt Petersburg), \emph{Schriftsteller}|pw}{}\ledrightnote{\textcolor{blue}{Fjodor Mihajlovič Dostojevskij}}
               sich der gleichen Technik bedienen, aber auch das trifft eigentlich nicht zu. \pend
           
\pstart
           Ihr Bedenken wegen der Summe kann ich wohl verstehen. Es ist schon möglich, dass ich,
               wie die übrigen \textcolor{pink}{österreichischen}\oindex{Österreich@\textbf{Österreich}|pw}{}\ledrightnote{\textcolor{pink}{Österreich}} Millionäre in
               unserem Nullenwahnsinn \label{K_L03754-2v}\edtext{a priori}{\lemma{\textnormal{\emph{a priori}}}\Cendnote{\textnormal{lateinisch: von vornherein}}}\label{K_L03754-2} falsch eingestellt war; andererseits gebe ich ihnen
               zu erwägen, dass Dors\substVorne{}\textsuperscript{t}\substDazwischen{}d\substHinten{}ay immerhin an einem Bild achtzigtausend Gulden verdient hatte, was schon
               damals vorkam; ferner dass durch die Höhe der Summe auch seine Forderung für das
               Publikum gewissermassen entschuldbarer wird; – und endlich spielten gewisse
               persönliche Jugenderinnerungen in die finanzielle Partie meiner \textcolor{green}{Novelle}\pwindex{Schnitzler, Arthur 15. 5. 1862 Wien – 21. 10. 1931 ebd.@\textsc{Schnitzler, Arthur} (15. 5. 1862 Wien – 21. 10. 1931 ebd.), \emph{Schriftsteller, Mediziner}!Fräulein Else@\strich\emph{Fräulein Else}|pwv}{}\ledrightnote{{$\rightarrow$}\emph{\textcolor{green}{Fräulein Else}}} hinein, nach denen sich die von mir genannte
               Summe durchaus im Bereich des Wahrscheinlichen bewegt. \pend
           
\pstart
           Nochmals herzlichen Dank, viele Grüsse und auf baldiges Wiedersehen{\\[\baselineskip]}{[}hs.:{]} Ihr
                  \spacefill\mbox{Arthur Schnitzler}\pend
           \leftskip=0em{}\selectlanguage{ngerman}\endnumbering\briefempfaengerindex{Zweig, Stefan@\textsc{Zweig, Stefan}!zzzSchnitzler, Arthur@\emph{von Arthur Schnitzler}!1924-11-062@{6. 11. 1924}|)be}\mylabel{L03754h}
\begin{anhang}
\end{anhang}\normalsize

\doendnotes{C}
\bigskip
\vfill

\clearpage

\footnotesize

\lohead{\textsc{register}}

% Definiere theindex-Environment komplett neu ohne reledmac
\makeatletter
\renewenvironment{theindex}{%
  \section*{\indexname}%
  \setlength{\parindent}{0pt}%
  \setlength{\parskip}{0pt plus 0.3pt}%
  \let\item\@idxitem
}{%
  \clearpage
}
\makeatother

\IfFileExists{\jobname-pw.ind}{\input{\jobname-pw.ind}}{}

\end{document}

      