%% latex-korrekturansicht-vorspann.tex
%% Vorspann für die Korrekturansicht.
%% Lädt die gemeinsame Datei latex-vorspann.tex mit gesetztem Schalter.

\newif\ifkorrekturansicht
\korrekturansichttrue

\input{../tex-inputs/latex-vorspann}


               \section[Hugo von Hofmannsthal an Arthur Schnitzler, 3. 10. 1908]{ Hugo von Hofmannsthal an Arthur Schnitzler, 3. 10. 1908}\nopagebreak\mylabel{v}\rehead{ }\normalsize\beginnumbering\briefempfaengerindex{Schnitzler, Arthur@\textsc{Schnitzler, Arthur}!zzzHofmannsthal, Hugo von@\emph{von Hugo von Hofmannsthal}!1908-10-031@{3. 10. 1908}|(be} \toendnotes[C]{\smallbreak\pagebreak[2]} \Standort{CUL, Schnitzler, B 43.}
\physDesc{Postkarte
\newline{}Handschrift: schwarze Tinte, deutsche Kurrent\newline{}Versand: Stempel: »\nobreak{}\oindex{Semmering@\textbf{Semmering}, \emph{Besiedelter Ort (A.BSO)}|pwk}Semmering 1, 3. X 08, 3\nobreak{}«.  
\newline{}Schnitzler: mit Bleistift datiert: »3. X 08« und beschriftet: »Hofmannsthal« \newline{}Ordnung: 1) mit Bleistift von unbekannter Hand nummeriert: »\strikeout{297}« 2) mit Bleistift von unbekannter Hand nummeriert: »301«}\buchAbdrucke{\weitereDrucke{Hugo von Hofmannsthal, Arthur Schnitzler: \emph{Briefwechsel}. Hg. Therese Nickl und Heinrich Schnitzler. Frankfurt am Main: \emph{S. Fischer} 1964, S. 241.} }\toendnotes[C]{\smallbreak}\pstart{}{\pb}\textsc{Herrn D\textsuperscript{r} Arthur Schnitzler}\pend{}\pstart{}\textcolor{pink}{\textsc{Wien}}{}\ledrightnote{\textcolor{pink}{Wien}}\pend{}\pstart{}\textsc{\textcolor{pink}{XVIII Spöttelgasse 7}{}\ledrightnote{\textcolor{pink}{Edmund-Weiß-Gasse}}}\pend{}{\bigskip}\pstart
           \raggedleft{}\textsc{Se{\geminationm}ering}\hspace*{1.5em}3 X.\pend
           \pstart
           mein lieber, ich bin hier für unbeſti{\geminationm}te Dauer um meinen \textcolor{green}{4\textsuperscript{ten} Act}{}\ledrightnote{→\textcolor{green}{Cristinas Heimreise. Komödie}} zu machen – und den Anfang {\pb}vom \textcolor{green}{erſten}{}\ledrightnote{→\textcolor{green}{Cristinas Heimreise. Komödie}}, und ein Stückel vom \textcolor{green}{dritten}{}\ledrightnote{→\textcolor{green}{Cristinas Heimreise. Komödie}}.\hspace*{1.5em}Ko{\geminationm}en Sie nicht mit Ihrem Arbeiterl ein biſſerl herauf?
               wie nett wäre das. Es iſt ſo ein ſchöner Moment in der Landſchaft.\pend
           \pstart
           Ihr{\\[\baselineskip]}\spacefill\mbox{Hugo}\pend
           \leftskip=0em{}\pstart
           \noindent{}\label{K_L01791_1v}\edtext{\textsc{\textcolor{blue}{L’arbre des roses}{}\ledrightnote{\textcolor{blue}{Karl Rostler}}, assis dans sa loge, lit
                     toujours avec une mine transfigurée »\textcolor{green}{le chemin à
                        la liberté!}{}\ledrightnote{\textcolor{green}{Der Weg ins Freie. Roman}}« C’est absolument touchant à voir.}}{\lemma{\textnormal{\emph{L’arbre … voir.}}}\Cendnote{\textnormal{»\textcolor{blue}{Rosenbaum}, in seiner Loge sitzend, liest immer mit
                     verklärter Mine ›\emph{\textcolor{green}{Der Weg ins Freie}}‹. Es ist
                     zutiefst rührend anzusehen.« Das Postskript wohl französisch, weil die Karte an
                     besagten Hotelportier \textcolor{blue}{Rosenbaum/Rostler} zur
                     Weiterleitung übermittelt wurde.}}}\label{K_L01791_1h}\pend
           \endnumbering\briefempfaengerindex{Schnitzler, Arthur@\textsc{Schnitzler, Arthur}!zzzHofmannsthal, Hugo von@\emph{von Hugo von Hofmannsthal}!1908-10-031@{3. 10. 1908}|)be}\mylabel{h}  \normalsize

\doendnotes{C}
\bigskip
\vfill

\clearpage

\footnotesize

\lohead{\textsc{register}}

% Definiere theindex-Environment komplett neu ohne reledmac
\makeatletter
\renewenvironment{theindex}{%
  \section*{\indexname}%
  \setlength{\parindent}{0pt}%
  \setlength{\parskip}{0pt plus 0.3pt}%
  \let\item\@idxitem
}{%
  \clearpage
}
\makeatother

\IfFileExists{\jobname-pw.ind}{\input{\jobname-pw.ind}}{}

\end{document}

      