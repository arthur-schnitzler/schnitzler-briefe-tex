%% latex-korrekturansicht-vorspann.tex
%% Vorspann für die Korrekturansicht.
%% Lädt die gemeinsame Datei latex-vorspann.tex mit gesetztem Schalter.

\newif\ifkorrekturansicht
\korrekturansichttrue

\input{../tex-inputs/latex-vorspann}


\renewcommand{\erwaehntePersonen}{Personen: Ludwig Fulda, Paul Goldmann, Gerhart Hauptmann, Hugo von Hofmannsthal, Rudolf Lothar}
\renewcommand{\erwaehnteOrte}{Orte: Deutschland, Wien}
\renewcommand{\erwaehnteWerke}{Werke: Berliner Theater. »Mensch und Übermensch« von Bernard Shaw, Der blinde Geronimo und sein Bruder, Der einsame Weg. Schauspiel in fünf Akten, Frau Bertha Garlan. Roman, Neue Freie Presse, Reigen. Zehn Dialoge, »Michael Kramer.«}
\section[Arthur Schnitzler an Paul Goldmann, nicht abgesandt, 28. 1. 1907]{Arthur Schnitzler an Paul Goldmann, nicht abgesandt,
               28. 1. 1907}
\nopagebreak\mylabel{v}
\rehead{ }\normalsize\beginnumbering\briefempfaengerindex{Goldmann, Paul@\textsc{Goldmann, Paul}!zzzSchnitzler, Arthur@\emph{von Arthur Schnitzler}!1907-01-281@{28. 1. 1907}|(be}
\toendnotes[C]{\smallbreak\pagebreak[2]}\Standort{CUL, Schnitzler, A 20,6.}
\physDesc{Brief, Maschinenschriftliche Abschrift, 5 Blätter, 5 Seiten, 4178 Zeichen
\newline{}Schreibmaschine
\newline{}Handschrift Arthur Schnitzler: 1) Bleistift (\noindent{}eine Unterstreichung)\hspace{1em}2) Bleistift, lateinische Kurrent (\noindent{}Vermerk »(An \textcolor{blue}{Goldmann}.)« und zwei sprachliche Eingriffe)\hspace{1em}
\newline{}Handschrift Schreibkraft: Bleistift, lateinische Kurrent (\noindent{}Vermerk
                                       »n{[}icht{]}. a{[}bgesandt{]}.«,
                                 marginale Korrekturen, Paginierung der 1. Seite)}\toendnotes[C]{\smallbreak}
\pstart
           \raggedleft{}{\pb}28. I. 907.\pend
           
\pstart
           \label{K_L03520-1v}\edtext{Auch Schweigen wäre
                  Unaufrichtigkeit.}{\lemma{\textnormal{\emph{Auch … Unaufrichtigkeit.}}}\Cendnote{\textnormal{Der unmittelbare
                  Anlass für diesen nicht abgesandten Briefentwurf ist unklar, womöglich das letzte
                  Feuilleton \textcolor{blue}{Goldmann}s: \emph{\textcolor{green}{Berliner Theater. »Mensch und Übermensch« von Bernard
                     Shaw}} (\emph{\textcolor{green}{Neue Freie Presse}}, Nr. 15.241,
                        25. 1. 1907, Morgenblatt, S. 1–4). Zumindest mit der
                  Wiederaufnahme des Begriffs »polemisieren« aus dem vorangegangenen (erhaltenen)
                  Brief \textcolor{blue}{Goldmann}s vom 18. 9. [1906] bettet sich
                  dieses Schreiben in die Korrespondenz ein. Zugleich verfügt er durch die Unterscheidung
                  zwischen Dichter und Literat (ersterer arbeitet unter Einsatz seiner Persönlichkeit),
                  eine bedeutsame biografisch-werkästhetische Aussage für \textcolor{blue}{Schnitzler}. Das dürfte dieser
                     selbst so gesehen haben, denn dieser Text wird
                  (zusammen mit den Abschriften seiner Briefe an \textcolor{blue}{Goldmann}, vgl. auch Arthur Schnitzler an Paul Goldmann, 1. 2. 1911)
                  nicht bei den restlichen Briefen im \emph{Deutschen Literaturarchiv in
                        Marbach} aufbewahrt, sondern findet sich in den
                  literarischen Nachlass in der \emph{Cambridge University
                        Library}.}}}\label{K_L03520-1h} Ich muss es Dir wieder einmal sagen.
                  \label{T_L03520-2v}\edtext{Seit Jahren{[},{]} Du
               weisst es, verfolge ich Deine Feuilletons mit wachsendem Widerstand}{\lemma{\textnormal{\emph{Seit … Widerstand}}}\Cendnote{\textnormal{Die Unterstreichung, die sich hier an der
                  Vorlage befindet, dürfte von \textcolor{blue}{Schnitzler}
                  stammen und einem archivalischen Zweck dienen, nicht der Textauszeichnung des
                  Originals.}}}\label{T_L03520-2h}. Es braucht nicht erst gesagt zu werden, dass auch meinem
               Geschmack \label{T_L03520-1v}\edtext{Einzelheiten}{\lemma{\textnormal{\emph{Einzelheiten}}}\Cendnote{\textnormal{korrigiert aus
                  »einzelheiten«}}}\label{T_L03520-1h} zusagen. Dass Du in einzelnem Recht hast.
               Aber als ganzes verwerf ich sie durchaus. Gesinnung und Ton. Ich wünsche nicht mit
               Dir zu polemisieren{[},{]} vielmehr{[},{]} ich betone
               ausdrücklich, dass ich Unrecht haben kann\label{T_L03520-11v}\edtext{,}{\lemma{\textnormal{\emph{,}}}\Cendnote{\textnormal{korrigiert aus
                     ».,«}}}\label{T_L03520-11h} dass Du
               sachlich Recht
               haben, dass Du sogar gut schreiben magst. Alles das ist möglich. Aber erst in einer
               fernen Zukunft wird das zu entscheiden sein. Und wir haben keine Zeit das abzuwarten.
               Das Wesentliche ist nur, dass Du und ich wie zwei fremde Welten einander gegenüber
               stehen. Dass unser Verhältnis zu dem, was heute gesagt, gedacht, geschrieben wird, in
               den wesentlichsten Punkten völlig von einander verschieden ist. Wir sind vor fünf {\pb}Jahren anlässlich Deiner \label{K_L03520-3v}\edtext{\textcolor{green}{Stellungna{[}h{]}me}{}\ledrightnote{{$\rightarrow$}\textcolor{green}{»Michael Kramer.«}}{ }\textcolor{blue}{Hauptmann}{}\ledrightnote{\textcolor{blue}{Gerhart Hauptmann}}}{\lemma{\textnormal{\emph{Stellungnahme Hauptmann}}}\Cendnote{\textnormal{höchstwahrscheinlich Bezug auf ein
                  älteres \textcolor{green}{Feuilleton},
                  nämlich \textcolor{blue}{Paul Goldmann}: \emph{\textcolor{green}{»Michael Kramer.«}}. In: \emph{\textcolor{green}{Neue Freie Presse}}, Nr. 13055, 28. 12. 1900, Morgenblatt, S. 1–3, bzw. auf darauf folgende
                  Feuilletons und damit einhergehende Auseinandersetzungen vgl. Paul Goldmann an Arthur Schnitzler, 31. 12. [1900], 9. 11. [1901] und 23. 11. [1901]}}}\label{K_L03520-3h} gegenüber zum erstenmal brieflich an einander geraten. Ich habe es
                  vorgezogen{[},{]} eine Disskussion abzubrechen, deren
               Hoffnungslosigkeit vom ersten \substVorne{}\textsuperscript{Tag}\substDazwischen{}Augenblick\substHinten{} an klar zu Tage lag. Der Verdacht, den später einmal andre, die mich nicht
                  kennen{[},{]} äussern könnten, dass erst persönliche
               Empfindlichkeit mich die Verschiederheit unserer Anschauungen\introOben{},\introOben{} unserer Naturen entdecken liess, fällt damit fort. Nun bin ich aber fern
               davon zu glauben, dass es zu den lebhaften inneren Differenzen gekommen wäre, wie sie
               nun bestehen, wenn nicht auch meine rein persönliche Sache zur Verhandlung stünde.
               Auch hier schalt ich gleich die Frage des Recht- oder Unrechthabens aus. Vielleicht
               wird Dir die Zukunft beistimmen und wird bei allen Dichtern deutscher Sprache, die
               heute leben und schaffen{[},{]} konstatieren{[},{]} was
               Du heute konstatierst, dass sie Dramen schreiben, in denen alles mangelt, {\pb}was einem Gedanken auch nur von fern
               ähnlich sieht. Und dass man überall in \textcolor{pink}{Deutschland}{}\ledrightnote{\textcolor{pink}{Deutschland}} Ideen finden kann{[},{]} nur nicht im modernen
               deutschen Drama. Sehr möglich, dass Du recht hast. Jedenfalls steht für mich die
               Sache so, dass ich nicht umhin kann{[},{]} mich mit den Dingen, die
               ich schreibe zu identifizieren. Es ist mir selbstverständlich bis heute noch nicht
               gelungen mich und meine Welt völlig zum Ausdruck zu bringen, aber die Arbeit\introOben{}en\introOben{} der letzten Zeit enthalten so viel von mir, \label{T_L03520-22v}\edtext{dass der,}{\lemma{\textnormal{\emph{dass der,}}}\Cendnote{\textnormal{korrigiert aus »dass, der«}}}\label{T_L03520-22h} der sie
                  \label{K_L03520-4v}\edtext{ablehnt}{\lemma{\textnormal{\emph{ablehnt}}}\Cendnote{\textnormal{Bezug auf diverse kritische Feuilletons \textcolor{blue}{Goldmann}s aus den vorherigen
                  Jahren}}}\label{K_L03520-4h}{[},{]} von mir als Ganze\substVorne{}\textsuperscript{n}\substDazwischen{}m\substHinten{} sich abwenden muss. Das hat nichts mit persönlicher Eitelkeit zu tun. Es
               gibt Schriftsteller bei denen es möglich ist ihr Schaffen von ihrem Dasein zu
               trennen. Ich gehöre nicht zu ihnen. Ich vermeide es mich hinter der Legende von einer
               Persönlichkeit zu verstecken, die es verschmäht oder nicht imstande ist, ihr bestes,
               ihr Eigenstes in ihren Werken zum Ausdruck zu bringen. Man kann es zum Beispiel {\pb}bei \textcolor{blue}{Lothar}{}\ledrightnote{\textcolor{blue}{Rudolf Lothar}} trennen, was er ist und was er schreibt, kann es vielleicht in anderm
                  Sin{[}n{]} bei \textcolor{blue}{Hofmannsthal}{}\ledrightnote{\textcolor{blue}{Hugo von Hofmannsthal}},
               wieder in anderm bei \textcolor{blue}{Fulda}{}\ledrightnote{\textcolor{blue}{Ludwig Fulda}}, gerade bei mir
               kann man es nicht. Ich bin\introOben{},\introOben{} was wieder die Zukunft zu
               entscheiden haben wird, vielleicht ein niederträchtiger Dichter, aber ich bin ein
               Dichter und kein Literat. Und übernehme die Verantwortung so gut für den \textcolor{green}{Reigen}{}\ledrightnote{\textcolor{green}{Reigen. Zehn Dialoge}}, wie für den \textcolor{green}{einsamen Weg}{}\ledrightnote{\textcolor{green}{Der einsame Weg. Schauspiel in fünf Akten}}, für den \textcolor{green}{blinden
                  Geronimo}{}\ledrightnote{\textcolor{green}{Der blinde Geronimo und sein Bruder}}, wie für die \textcolor{green}{Berta Garlan}{}\ledrightnote{\textcolor{green}{Frau Bertha Garlan. Roman}} u. s.
               w. Natürlich weiss ich sehr gut, dass mir formal einiges mehr,
                  a{[}n{]}deres minder gelungen ist und verstehe ohne weiters, dass
               auch jemand\introOben{}em\introOben{}, der mich schätzt, das eine oder das andre
               meiner Werke zuwider ist. Aber ich bestreite es, dass irgend ein Mensch, der beinah
               zu keinem dieser Werke ein Verhältnis zu finden imstande ist (und ihren Gehalt nicht
               spüren heisst für mich\introOben{}:\introOben{} kein Verhältnis zu ihnen finden) zu
               mir persönlich in irgend einem wirklichen Verhältnis zu stehen imstande {\pb}ist. Sind diese Werke ideenlos und
                  gering{[},{]} so muss ich es selbst auch sein. Und es ist nur ein
               Gebot der Selbstachtung{[},{]} eine menschliche \label{T_L03520-3v}\edtext{Beziehung}{\lemma{\textnormal{\emph{Beziehung}}}\Cendnote{\textnormal{korrigiert aus »Beziehung,«}}}\label{T_L03520-3h} jener
               schönen Lüge zu entkleiden, die sie durch die \label{T_L03520-8v}\edtext{Ursupierung}{\lemma{\textnormal{\emph{Ursupierung}}}\Cendnote{\textnormal{korrigiert aus: »Ursopierung«}}}\label{T_L03520-8h} des \strikeout{schönen} Wortes Freundschaft um die Schultern schlägt. Und die Erinnerung
               unserer früheren Freundschaft steht mir zu hoch, als dass ich die Illusion aufrecht
               erhalten dürfte, zwei Menschen{[},{]} die so ziemlich über alle Dinge
               der Welt so verschieden denken, wie ich und Du könnten Freunde bleiben oder weiter
               Freunde heissen.\pend
           \endnumbering\briefempfaengerindex{Goldmann, Paul@\textsc{Goldmann, Paul}!zzzSchnitzler, Arthur@\emph{von Arthur Schnitzler}!1907-01-281@{28. 1. 1907}|)be}\mylabel{h}  \normalsize

\doendnotes{C}
\bigskip
\vfill

\clearpage

\footnotesize

\lohead{\textsc{register}}

% Definiere theindex-Environment komplett neu ohne reledmac
\makeatletter
\renewenvironment{theindex}{%
  \section*{\indexname}%
  \setlength{\parindent}{0pt}%
  \setlength{\parskip}{0pt plus 0.3pt}%
  \let\item\@idxitem
}{%
  \clearpage
}
\makeatother

\IfFileExists{\jobname-pw.ind}{\input{\jobname-pw.ind}}{}

\end{document}

      