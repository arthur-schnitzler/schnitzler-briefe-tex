%% latex-korrekturansicht-vorspann.tex
%% Vorspann für die Korrekturansicht.
%% Lädt die gemeinsame Datei latex-vorspann.tex mit gesetztem Schalter.

\newif\ifkorrekturansicht
\korrekturansichttrue

\input{../tex-inputs/latex-vorspann}


               \section[Hermann Bahr: Widmungsexemplar Wenn es Euch gefällt für Arthur Schnitzler, 21. 5. 1899]{ Hermann Bahr: Widmungsexemplar Wenn es Euch gefällt für Arthur Schnitzler,
               21. 5. 1899}\nopagebreak\mylabel{v}\rehead{ }\normalsize\beginnumbering\briefempfaengerindex{Schnitzler, Arthur@\textsc{Schnitzler, Arthur}!zzzBahr, Hermann@\emph{von Hermann Bahr}!1899-05-211@{21. 5. 1899}|(be} \toendnotes[C]{\smallbreak\pagebreak[2]} \Standort{DLA, G:Schnitzler, Arthur (Sammlung Heinrich Schnitzler).}
\physDesc{Widmung am Vorsatzblatt
\newline{}Handschrift: schwarze Tinte, deutsche Kurrent\newline{}Ordnung: bei der Enteignung des Exemplars 1938 von unbekannter Hand mit Bleistift ergänzte
                           Information: »Dublette zu 101.290-B« }\pstart
           \noindent{}{\pb}Seinem lieben Freunde{\\}Arthur
               Schnitzler\pend
           \pstart
           herzlichſt{\\[\baselineskip]}\spacefill\mbox{HermannBahr}\pend
           \leftskip=0em{}\pstart
           \noindent{}Pfingsten 1899\pend
           \pstart
           \centering{}\textcolor{gray}{\textbf{\textcolor{green}{\textbf{Wenn es Euch gefällt}}{}\ledrightnote{\textcolor{green}{Wenn es euch gefällt. Wiener Revue in drei Bildern und einem Vorspiel}}.}}\pend
           {\bigskip}\pstart
           \noindent{}\centering{}{\pb}\textcolor{gray}{\textbf{\textcolor{green}{\textbf{Wenn es Euch gefällt}}{}\ledrightnote{\textcolor{green}{Wenn es euch gefällt. Wiener Revue in drei Bildern und einem Vorspiel}}.}}\pend
           \pstart
           \noindent{}\centering{}\textcolor{gray}{\textbf{\textbf{\textcolor{pink}{Wien}{}\ledrightnote{\textcolor{pink}{Wien}}er
               Revue}}}\pend
           \pstart
           \noindent{}\centering{}\textcolor{gray}{\textbf{in}}\pend
           \pstart
           \noindent{}\centering{}\textcolor{gray}{\textbf{\textbf{drei Bildern und einem Vorſpiel}}}\pend
           \pstart
           \noindent{}\centering{}\textcolor{gray}{\textbf{von}}\pend
           \pstart
           \noindent{}\centering{}\textcolor{gray}{\textbf{\textbf{Hermann Bahr} und \textcolor{blue}{\textbf{C.
                     Karlweis}}{}\ledrightnote{\textcolor{blue}{Carl Karlweis}}.}}\pend
           {\bigskip}\pstart
           \noindent{}\centering{}\textcolor{gray}{\textbf{\textcolor{pink}{\textbf{Wien}}{}\ledrightnote{\textcolor{pink}{Wien}}.}}\pend
           \pstart
           \noindent{}\centering{}\textcolor{gray}{\textbf{\textcolor{brown}{\so{Verlag von Carl Konegen}}{}\ledrightnote{\textcolor{brown}{Carl Konegen}}}}\pend
           \pstart
           \noindent{}\centering{}\textcolor{gray}{\textbf{1899.}}\pend
           \endnumbering\briefempfaengerindex{Schnitzler, Arthur@\textsc{Schnitzler, Arthur}!zzzBahr, Hermann@\emph{von Hermann Bahr}!1899-05-211@{21. 5. 1899}|)be}\mylabel{h}  \normalsize

\doendnotes{C}
\bigskip
\vfill

\clearpage

\footnotesize

\lohead{\textsc{register}}

% Definiere theindex-Environment komplett neu ohne reledmac
\makeatletter
\renewenvironment{theindex}{%
  \section*{\indexname}%
  \setlength{\parindent}{0pt}%
  \setlength{\parskip}{0pt plus 0.3pt}%
  \let\item\@idxitem
}{%
  \clearpage
}
\makeatother

\IfFileExists{\jobname-pw.ind}{\input{\jobname-pw.ind}}{}

\end{document}

      