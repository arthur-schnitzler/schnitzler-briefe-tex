%% latex-korrekturansicht-vorspann.tex
%% Vorspann für die Korrekturansicht.
%% Lädt die gemeinsame Datei latex-vorspann.tex mit gesetztem Schalter.

\newif\ifkorrekturansicht
\korrekturansichttrue

\input{../tex-inputs/latex-vorspann}


               \section[Arthur Schnitzler an Richard Beer-Hofmann, 31. 7. 1909]{ Arthur Schnitzler an Richard Beer-Hofmann, 31. 7. 1909}\nopagebreak\mylabel{v}\rehead{ }\normalsize\beginnumbering\briefempfaengerindex{Beer-Hofmann, Richard@\textsc{Beer-Hofmann, Richard}!zzzSchnitzler, Arthur@\emph{von Arthur Schnitzler}!1909-07-311@{31. 7. 1909}|(be} \toendnotes[C]{\smallbreak\pagebreak[2]} \Standort{YCGL, MSS 31.}
\physDesc{Brief, 1 Blatt, 4 Seiten, Umschlag
\newline{}Handschrift: schwarze Tinte, lateinische Kurrent\newline{}Versand: Stempel: »\nobreak{}\oindex{Edlach@\textbf{Edlach}, \emph{Besiedelter Ort (A.BSO)}|pwk}Edlach \textcolor{gray}{bei Reichenau N.Ö.}, XII\nobreak{}«.  
\newline{}Beer-Hofmann: mit rotem Buntstift mit dem Datum der Beantwortung beschriftet:
               »B 4/VIII 09« }\buchAbdrucke{\weitereDrucke{Arthur Schnitzler, Richard Beer-Hofmann: \emph{Briefwechsel 1891–1931}. Hg. Konstanze Fliedl. Wien, Zürich: \emph{Europaverlag} 1992, S. 194.} }\toendnotes[C]{\smallbreak}\pstart{}{\pb}\textcolor{gray}{\textbf{Dr. Arthur Schnitzler}}\pend{}\pstart{}\textcolor{gray}{\textbf{\textcolor{pink}{Wien XVIII. Spoettelgasse 7}{}\ledrightnote{\textcolor{pink}{Edmund-Weiß-Gasse}}.}}\pend{}{\bigskip}\pstart{}{\pb}Herrn Dr Richard Beer Hofmann\pend{}\pstart{}\textcolor{pink}{Wien XVIII}{}\ledrightnote{\textcolor{pink}{XVIII., Währing}}\pend{}\pstart{}\textcolor{pink}{Hasenauerstr. 59}{}\ledrightnote{\textcolor{pink}{Hasenauerstraße}}.\pend{}{\bigskip}\pstart
           \noindent{}{\pb}\textcolor{gray}{\textbf{Dr. Arthur Schnitzler}}\hfill \textcolor{pink}{Edlach}{}\ledrightnote{\textcolor{pink}{Edlach}}, \textcolor{pink}{Edlacher
                        Hof}{}\ledrightnote{\textcolor{pink}{Hotel Edlacherhof}}\pend
           \pstart
           \textcolor{gray}{\textbf{\textcolor{pink}{Wien XVIII. Spoettelgasse 7}{}\ledrightnote{\textcolor{pink}{Edmund-Weiß-Gasse}}.}}\hfill 31. 7. 09.\pend
           \pstart
           lieber Richard, Ihnen allen innig theilnahmsvollen Gruß und
               Händedruck, auch von \textcolor{blue}{Olga}{}\ledrightnote{\textcolor{blue}{Olga Schnitzler}}. Wir wissen, wie gern
               Sie diese \label{K_L01862-1v}\edtext{\textcolor{blue}{Frau}{}\ledrightnote{→\textcolor{blue}{Agnes Beer}}}{\lemma{\textnormal{\emph{Frau}}}\Cendnote{\textnormal{Am 27. 7. 1909 starb seine Tante \textcolor{blue}{Agnes Beer} in ihrer Wohnung in \textcolor{pink}{Wien}.}}}\label{K_L01862-1h} gehabt haben; es müssen traurige Tage für Sie sein.
               Schreiben Sie mir doch bald ein Wort, {\pb}wie lange Sie
               in \textcolor{pink}{Wien}{}\ledrightnote{\textcolor{pink}{Wien}} bleiben werden. Möchten Sie sich nicht doch
               entschliessen hieher zu ko{\geminationm}en? Wir würden uns so sehr
               freuen und ich glaube, für Sie alle wäre die Luft hier, trotz gelegentlicher
               Mittagsschwüle (Abends immer kühl) sehr angenehm. Die Spaziergänge charmant,
               vielfältig, jeder {\pb}Art von Ansprüchen gemäß. –\pend
           \pstart
           – Wir denken bis Ende August zu bleiben, doch wäre es sehr möglich, daß
               ich in der zweiten August Hälfte auf ca 8 Tage nach \textcolor{pink}{München}{}\ledrightnote{\textcolor{pink}{München}} gehe (aus praktischen \textcolor{blue}{Reinhardt}{}\ledrightnote{\textcolor{blue}{Max Reinhardt}}
               Gründen.)\pend
           \pstart
           Lassen Sie doch recht bald hören, wie’s Ihnen Allen geht. Bei uns gut; der \textcolor{blue}{Bub}{}\ledrightnote{→\textcolor{blue}{Heinrich Schnitzler}} schon {\pb}ganz gesund.\pend
           \pstart
           Herzlichst Ihr{\\[\baselineskip]}\spacefill\mbox{Arthur.}\pend
           \leftskip=0em{}\endnumbering\briefempfaengerindex{Beer-Hofmann, Richard@\textsc{Beer-Hofmann, Richard}!zzzSchnitzler, Arthur@\emph{von Arthur Schnitzler}!1909-07-311@{31. 7. 1909}|)be}\mylabel{h}  \normalsize

\doendnotes{C}
\bigskip
\vfill

\clearpage

\footnotesize

\lohead{\textsc{register}}

% Definiere theindex-Environment komplett neu ohne reledmac
\makeatletter
\renewenvironment{theindex}{%
  \section*{\indexname}%
  \setlength{\parindent}{0pt}%
  \setlength{\parskip}{0pt plus 0.3pt}%
  \let\item\@idxitem
}{%
  \clearpage
}
\makeatother

\IfFileExists{\jobname-pw.ind}{\input{\jobname-pw.ind}}{}

\end{document}

      