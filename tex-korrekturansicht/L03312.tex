%% latex-korrekturansicht-vorspann.tex
%% Vorspann für die Korrekturansicht.
%% Lädt die gemeinsame Datei latex-vorspann.tex mit gesetztem Schalter.

\newif\ifkorrekturansicht
\korrekturansichttrue

\input{../tex-inputs/latex-vorspann}


\renewcommand{\erwaehnteOrte}{Orte: Bad Ischl, Bozen, Italien, Meran, Ponte Adige, Pontresina, Strobl, Venedig, Wien}
\renewcommand{\erwaehnteWerke}{}
\section[ Felix Salten an Arthur Schnitzler, 20. 8. 1900]{Felix Salten an Arthur Schnitzler, 20. 8. 1900}
\nopagebreak\mylabel{v}
\rehead{ }\normalsize\beginnumbering\briefempfaengerindex{Schnitzler, Arthur@\textsc{Schnitzler, Arthur}!zzzSalten, Felix@\emph{von Felix Salten}!1900-08-201@{20. 8. 1900}|(be}
\toendnotes[C]{\smallbreak\pagebreak[2]}\Standort{CUL, Schnitzler, B 89, A 2.}
\physDesc{Postkarte, 485 Zeichen
\newline{}Handschrift: schwarze Tinte, lateinische Kurrent
\newline{}Versand: Stempel: »\nobreak{}\oindex{Strobl@\textbf{Strobl}, \emph{A.ADM3}|pwk}Strobl, 20 8 00\nobreak{}«. Stempel: »\nobreak{}\oindex{Pontresina@\textbf{Pontresina}, \emph{P.PPL}|pwk}Pontresina, 23. VIII. 00., 4\nobreak{}«. Stempel: »\nobreak{}\oindex{Pontresina@\textbf{Pontresina}, \emph{P.PPL}|pwk}Pontresina, 23. VIII. 00., XI\nobreak{}«.  
\newline{}Schnitzler: mit Bleistift datiert: »20/8 90\textcolor{gray}{0}« 
\newline{}Ordnung: mit Bleistift von unbekannter Hand nummeriert: »136« }\toendnotes[C]{\smallbreak}\pstart{}{\pb}Herrn D\textsuperscript{r} Arthur Schnitzler\pend{}\pstart{}\textcolor{pink}{Pontresina}{}\ledrightnote{\textcolor{pink}{Pontresina}}\pend{}\pstart{}Poste restante.\pend{}
{\bigskip}
\pstart
           \noindent{}{\pb}Lieber,{ }heute erhielt ich Ihre Carte. Ich möchte von \textcolor{pink}{Ischl}{}\ledrightnote{\textcolor{pink}{Bad Ischl}}{ }\uline{so} fortfahren, dass ich gleichzeitig mit Ihnen in
                  \label{K_L03312-1v}\edtext{\textcolor{pink}{Bozen}{}\ledrightnote{\textcolor{pink}{Bozen}}}{\lemma{\textnormal{\emph{Bozen}}}\Cendnote{\textnormal{\textcolor{blue}{Schnitzler} und \textcolor{blue}{Salten} trafen sich am 28. 8. 1900 in \textcolor{pink}{Meran}. Am 30. 8. 1900 fuhren sie womöglich gemeinsam weiter nach \textcolor{pink}{Bozen} und \textcolor{pink}{Ponte Adige}.}}}\label{K_L03312-1h} bin. Bitte, sagen Sie mir also, wann Sie dort sind, –
               ungefähr wenigstens. Ferner: Ich müßte am 1.
               spätestens am 3. September in \textcolor{pink}{Wien}{}\ledrightnote{\textcolor{pink}{Wien}} sein. Endlich: welche Tour machen wir? \label{K_L03312-2v}\edtext{Ob. \textcolor{pink}{Italien}{}\ledrightnote{\textcolor{pink}{Italien}}}{\lemma{\textnormal{\emph{Ob. Italien}}}\Cendnote{\textnormal{Oberes \textcolor{pink}{Italien}}}}\label{K_L03312-2h} u. \textcolor{pink}{Venedig}{}\ledrightnote{\textcolor{pink}{Venedig}} ist vielleicht
                  \textcolor{gray}{do}ch zu heiß u. hat jetzt zu viel Mosquitos. Übrigens ist es
               mir ziemlich egal, wohin wir fahren.\pend
           
\pstart
           Auf Wiedersehen, {\\[\baselineskip]}herzl. {\\[\baselineskip]}\spacefill\mbox{Salten.}\pend
           \leftskip=0em{}\endnumbering\briefempfaengerindex{Schnitzler, Arthur@\textsc{Schnitzler, Arthur}!zzzSalten, Felix@\emph{von Felix Salten}!1900-08-201@{20. 8. 1900}|)be}\mylabel{h}  \normalsize

\doendnotes{C}
\bigskip
\vfill

\clearpage

\footnotesize

\lohead{\textsc{register}}

% Definiere theindex-Environment komplett neu ohne reledmac
\makeatletter
\renewenvironment{theindex}{%
  \section*{\indexname}%
  \setlength{\parindent}{0pt}%
  \setlength{\parskip}{0pt plus 0.3pt}%
  \let\item\@idxitem
}{%
  \clearpage
}
\makeatother

\IfFileExists{\jobname-pw.ind}{\input{\jobname-pw.ind}}{}

\end{document}

      