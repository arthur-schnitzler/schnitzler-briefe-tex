%% latex-korrekturansicht-vorspann.tex
%% Vorspann für die Korrekturansicht.
%% Lädt die gemeinsame Datei latex-vorspann.tex mit gesetztem Schalter.

\newif\ifkorrekturansicht
\korrekturansichttrue

\input{../tex-inputs/latex-vorspann}


               \section[Paul Goldmann an Arthur Schnitzler, 20. 12. 1890]{ Paul Goldmann an Arthur Schnitzler, 20. 12. 1890}\nopagebreak\mylabel{v}\rehead{ }\normalsize\beginnumbering\briefempfaengerindex{Schnitzler, Arthur@\textsc{Schnitzler, Arthur}!zzzGoldmann, Paul@\emph{von Paul Goldmann}!1890-12-201@{20. 12. 1890}|(be} \toendnotes[C]{\smallbreak\pagebreak[2]} \Standort{DLA, A:Schnitzler, HS.NZ85.1.3162.}
\physDesc{Brief, 1 Blatt, 2 Seiten
\newline{}Handschrift: schwarze Tinte, deutsche Kurrent}\toendnotes[C]{\smallbreak}\pstart
           {\pb}\textcolor{pink}{Wien}{}\ledrightnote{\textcolor{pink}{Wien}} den \textsuperscript{20}/\textsubscript{12}
                  1890.\pend
           \pstart
           Lieber Arthur! Ich ſchreibe dieſe Zeilen in fliegender
               Eile in einem \textsc{Café} auf der \textcolor{pink}{Mariahilferſtraße}{}\ledrightnote{\textcolor{pink}{Mariahilferstraße}}. Soeben iſt ein ſcharfer Conflict zwiſchen dem \label{K_L02652-1v}\edtext{bisherigen \textcolor{blue}{Verleger}{}\ledrightnote{→\textcolor{blue}{Joseph Eberle}}}{\lemma{\textnormal{\emph{bisherigen Verleger}}}\Cendnote{\textnormal{Die ersten fünf Jahrgänge von \emph{\textcolor{brown}{An der schönen blauen Donau}} wurden von der
                  Druckerei \emph{\textcolor{brown}{Josef Eberle}} in der \textcolor{pink}{Seidengasse} nahe der \textcolor{pink}{Mariahilferſtraße} hergestellt. Mit dem 6. Jahrgang übernahm ab
                     1891 die Druckerei der Tageszeitung \emph{\textcolor{brown}{Die Presse}} die Produktion.}}}\label{K_L02652-1h} der »\textcolor{brown}{Blauen Donau}{}\ledrightnote{\textcolor{brown}{An der schönen blauen Donau}}« und der »\textcolor{brown}{Preſſe}{}\ledrightnote{\textcolor{brown}{Die Presse}}« zum
               Ausbruch gekommen. \textcolor{brown}{Erſteren}{}\ledrightnote{→\textcolor{brown}{Josef Eberle  Stein-, Buch und Musikaliendruckerei}}
               verärgert die Ausfolgung des Materials; ich habe ſoeben mit ihm und ſeinem \label{K_L02652-2v}\edtext{\textcolor{blue}{Advocaten}{}\ledrightnote{→\textcolor{blue}{?? [Anwalt der Buchdruckerei Eberle, 1891]}}}{\lemma{\textnormal{\emph{Advocaten}}}\Cendnote{\textnormal{nicht identifiziert}}}\label{K_L02652-2h} conferirt und
               muß ſofort wieder einer zweiten Conferenz beiwohnen. Theile dies, bitte, deiner Frau
                  \textcolor{blue}{Schweſter}{}\ledrightnote{→\textcolor{blue}{Gisela Hajek}} u. Deinem Herrn
                  \textcolor{blue}{Schwager}{}\ledrightnote{→\textcolor{blue}{Markus Hajek}} – unter
               Discretion – mit! Unter dieſen Umſtänden {\pb}werden ſie mein Nichterſcheinen
               wohl entſchuldigen. Ich bedaure unendlich, daß mir die Freude verſtört wird, dieſen
               Abend bei ihnen zubringen zu können. Und wie verſtört! Näheres mündlich!\pend
           \pstart
           Ich habe auch nicht früher ſchreiben können, weil ſich die ganze Geſchichte erſt um
                  7 Uhr Abends begeben hat.\pend
           \pstart
           Viele Grüße!{\\[\baselineskip]}Dein{\\[\baselineskip]}\spacefill\mbox{Paul.}\pend
           \leftskip=0em{}\endnumbering\briefempfaengerindex{Schnitzler, Arthur@\textsc{Schnitzler, Arthur}!zzzGoldmann, Paul@\emph{von Paul Goldmann}!1890-12-201@{20. 12. 1890}|)be}\mylabel{h}  \normalsize

\doendnotes{C}
\bigskip
\vfill

\clearpage

\footnotesize

\lohead{\textsc{register}}

% Definiere theindex-Environment komplett neu ohne reledmac
\makeatletter
\renewenvironment{theindex}{%
  \section*{\indexname}%
  \setlength{\parindent}{0pt}%
  \setlength{\parskip}{0pt plus 0.3pt}%
  \let\item\@idxitem
}{%
  \clearpage
}
\makeatother

\IfFileExists{\jobname-pw.ind}{\input{\jobname-pw.ind}}{}

\end{document}

      