%% latex-korrekturansicht-vorspann.tex
%% Vorspann für die Korrekturansicht.
%% Lädt die gemeinsame Datei latex-vorspann.tex mit gesetztem Schalter.

\newif\ifkorrekturansicht
\korrekturansichttrue

\input{../tex-inputs/latex-vorspann}


\renewcommand{\erwaehntePersonen}{Personen: Irene Triesch}
\renewcommand{\erwaehnteOrte}{Orte: Berlin, Deutsches Theater Berlin, Frankfurt am Main}
\renewcommand{\erwaehnteWerke}{Werke: Lebendige Stunden. Vier Einakter}
\section[ Paul Goldmann an Arthur Schnitzler, 31. 12. {[}1901{]}]{Paul Goldmann an Arthur Schnitzler, 31. 12. {[}1901{]}}
\nopagebreak\mylabel{v}
\rehead{ }\normalsize\beginnumbering\briefempfaengerindex{Schnitzler, Arthur@\textsc{Schnitzler, Arthur}!zzzGoldmann, Paul@\emph{von Paul Goldmann}!1901-12-311@{31. 12. {[}1901{]}}|(be}
\toendnotes[C]{\smallbreak\pagebreak[2]}\Standort{DLA, A:Schnitzler, HS.NZ85.1.3171.}
\physDesc{Brief, 1 Blatt, 1 Seite
\newline{}Handschrift: blaue Tinte, deutsche Kurrent
\newline{}Schnitzler: 1) mit Bleistift das Jahr »{[}1{]}901.« vermerkt  2) mit rotem Buntstift eine Unterstreichung}\toendnotes[C]{\smallbreak}
\pstart
           \centering{}{\pb}\textcolor{pink}{Frankfurt}{}\ledrightnote{\textcolor{pink}{Frankfurt am Main}}{ }31. Dezember\pend
           
\pstart\center{}Mein lieber Freund,\pend
\pstart
           Dank für das Billet! Ich freue mich ſehr über den guten Fortgang der \textcolor{green}{Proben}{}\ledrightnote{{$\rightarrow$}\textcolor{green}{Lebendige Stunden. Vier Einakter}}. \label{K_L03099-1v}\edtext{Samſtag{ }Abend bin ich im \textcolor{pink}{Theater}{}\ledrightnote{{$\rightarrow$}\textcolor{pink}{Deutsches Theater Berlin}}}{\lemma{\textnormal{\emph{Samſtag … Theater}}}\Cendnote{\textnormal{Am Samstag, dem 4. 1. 1902, fand am
                     \textcolor{pink}{Deutschen Theater Berlin} die Uraufführung der
                  vier Einakter \emph{\textcolor{green}{Lebendige Stunden}}
               statt.}}}\label{K_L03099-1h}. Vorher werde ich Dich kaum ſehen, da ich erſt ſpät ankomme. Paß’ bei
               den Proben nur auf die \textsc{\textcolor{blue}{Triesch}{}\ledrightnote{\textcolor{blue}{Irene Triesch}}} auf, daß ſie nicht zu viel thut! Sie iſt bei aller Begabung von einer
               unglaublichen Geſchmackloſigkeit. Laß’ es Dir in \textcolor{pink}{Berlin}{}\ledrightnote{\textcolor{pink}{Berlin}} gut gehen! Gückliches neues Jahr! Viele
               treue Grüße!\pend
           
\pstart
           Dein {\\[\baselineskip]}\spacefill\mbox{Paul Goldmann}\pend
           \leftskip=0em{}\endnumbering\briefempfaengerindex{Schnitzler, Arthur@\textsc{Schnitzler, Arthur}!zzzGoldmann, Paul@\emph{von Paul Goldmann}!1901-12-311@{31. 12. {[}1901{]}}|)be}\mylabel{h}  \normalsize

\doendnotes{C}
\bigskip
\vfill

\clearpage

\footnotesize

\lohead{\textsc{register}}

% Definiere theindex-Environment komplett neu ohne reledmac
\makeatletter
\renewenvironment{theindex}{%
  \section*{\indexname}%
  \setlength{\parindent}{0pt}%
  \setlength{\parskip}{0pt plus 0.3pt}%
  \let\item\@idxitem
}{%
  \clearpage
}
\makeatother

\IfFileExists{\jobname-pw.ind}{\input{\jobname-pw.ind}}{}

\end{document}

      