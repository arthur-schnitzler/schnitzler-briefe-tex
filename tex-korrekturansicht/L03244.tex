%% latex-korrekturansicht-vorspann.tex
%% Vorspann für die Korrekturansicht.
%% Lädt die gemeinsame Datei latex-vorspann.tex mit gesetztem Schalter.

\newif\ifkorrekturansicht
\korrekturansichttrue

\input{../tex-inputs/latex-vorspann}


\renewcommand{\erwaehntePersonen}{Personen: Vincenz Czerny, Fedor Mamroth, Josef Rosengart, Julius Schnitzler}
\renewcommand{\erwaehnteOrte}{Orte: Frankfurt am Main, Heidelberg, Wien}
\renewcommand{\erwaehnteWerke}{}
\section[ Paul Goldmann an Arthur Schnitzler, 20. 4. {[}1906{]}]{Paul Goldmann an Arthur Schnitzler, 20. 4. {[}1906{]}}
\nopagebreak\mylabel{v}
\rehead{ }\normalsize\beginnumbering\briefempfaengerindex{Schnitzler, Arthur@\textsc{Schnitzler, Arthur}!zzzGoldmann, Paul@\emph{von Paul Goldmann}!1906-04-201@{20. 4. {[}1906{]}}|(be}
\toendnotes[C]{\smallbreak\pagebreak[2]}\Standort{DLA, A:Schnitzler, HS.NZ85.1.3175.}
\physDesc{Brief, 1 Blatt, 2 Seiten
\newline{}Handschrift: schwarze Tinte, deutsche Kurrent
\newline{}Schnitzler: mit Bleistift das Jahr »{[}1{]}906« vermerkt }\toendnotes[C]{\smallbreak}
\pstart
           \centering{}{\pb}\textcolor{pink}{Frankfurt}{}\ledrightnote{\textcolor{pink}{Frankfurt am Main}}{ }20. April.\pend
           
\pstart
           Lieber Freund, Ich danke Dir \textcolor{gray}{und}
               Deinem \textcolor{blue}{Bruder}{}\ledrightnote{{$\rightarrow$}\textcolor{blue}{Julius Schnitzler}} auf das
               Herzlichſte für die raſche Antwort. Daß eine Autorität \strikeout{\textcolor{gray}{×}} wie Dein \textcolor{blue}{Bruder}{}\ledrightnote{{$\rightarrow$}\textcolor{blue}{Julius Schnitzler}} zur
                  \label{K_L03244-11v}\edtext{Operation}{\lemma{\textnormal{\emph{Operation}}}\Cendnote{\textnormal{siehe Paul Goldmann an Arthur Schnitzler, 9. 4. [1906] und Paul Goldmann an Arthur Schnitzler, 16. 4. [1906]}}}\label{K_L03244-11h}{ }\strikeout{r} rät, iſt für uns wichtig zu
               wiſſen, und ich habe von meinem \textcolor{blue}{Schwager}{}\ledrightnote{{$\rightarrow$}\textcolor{blue}{Josef Rosengart}}, der ſich ſchon entſchloſſen hatte, nichts weiter zu tun,
               wenigſtens erreicht, daß er nach \textcolor{pink}{Heidelberg}{}\ledrightnote{\textcolor{pink}{Heidelberg}}
               fahren wird, um ſich mit \textsc{\textcolor{blue}{Czerny}{}\ledrightnote{\textcolor{blue}{Vincenz Czerny}}} zu beſprechen. Der Sitz des \textsc{Tumors} iſt allerdings {\pb}ein derartiger, daß eine Operation faſt unmöglich
               erſcheint. Auch ſprechen ſtarke pſychiſche Gründe dagegen, indem man den \textcolor{blue}{Kranken}{}\ledrightnote{{$\rightarrow$}\textcolor{blue}{Fedor Mamroth}} nicht noch einmal zur
               Operation veranlaſſen kann, ohne ihm die volle Wahrheit zu ſagen. Immerhin, \textsc{\textcolor{blue}{Czerny}{}\ledrightnote{\textcolor{blue}{Vincenz Czerny}}} ſoll entſcheiden.\pend
           
\pstart
           Dir und Deinem \textcolor{blue}{Bruder}{}\ledrightnote{\textcolor{blue}{Julius Schnitzler}} tauſend Dank für
               den Freundſchaftsdienſt, den Ihr mir geleiſtet habt, und viele treue Grüße! {\\[\baselineskip]}Dein
                  \spacefill\mbox{Paul Goldmnn}\pend
           \leftskip=0em{}\endnumbering\briefempfaengerindex{Schnitzler, Arthur@\textsc{Schnitzler, Arthur}!zzzGoldmann, Paul@\emph{von Paul Goldmann}!1906-04-201@{20. 4. {[}1906{]}}|)be}\mylabel{h}  \normalsize

\doendnotes{C}
\bigskip
\vfill

\clearpage

\footnotesize

\lohead{\textsc{register}}

% Definiere theindex-Environment komplett neu ohne reledmac
\makeatletter
\renewenvironment{theindex}{%
  \section*{\indexname}%
  \setlength{\parindent}{0pt}%
  \setlength{\parskip}{0pt plus 0.3pt}%
  \let\item\@idxitem
}{%
  \clearpage
}
\makeatother

\IfFileExists{\jobname-pw.ind}{\input{\jobname-pw.ind}}{}

\end{document}

      