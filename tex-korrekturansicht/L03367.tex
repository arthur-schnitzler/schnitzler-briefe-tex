%% latex-korrekturansicht-vorspann.tex
%% Vorspann für die Korrekturansicht.
%% Lädt die gemeinsame Datei latex-vorspann.tex mit gesetztem Schalter.

\newif\ifkorrekturansicht
\korrekturansichttrue

\input{../tex-inputs/latex-vorspann}


\renewcommand{\erwaehntePersonen}{Personen: Elisabeth Steinrück}
\renewcommand{\erwaehnteOrte}{Orte: Berlin, Café Josty, Palasthotel Berlin, Potsdamer Platz}
\renewcommand{\erwaehnteWerke}{}
\section[ Paul Goldmann an Arthur Schnitzler, 28. 2. 1903]{Paul Goldmann an Arthur Schnitzler, 28. 2. 1903}
\nopagebreak\mylabel{v}
\rehead{ }\normalsize\beginnumbering\briefempfaengerindex{Schnitzler, Arthur@\textsc{Schnitzler, Arthur}!zzzGoldmann, Paul@\emph{von Paul Goldmann}!1903-02-281@{28. 2. 1903}|(be}
\toendnotes[C]{\smallbreak\pagebreak[2]}\Standort{DLA, A:Schnitzler, HS.NZ85.1.3173.}
\physDesc{Postkarte
\newline{}Handschrift: 1) blaue Tinte, deutsche Kurrent\hspace{1em}2) blaue Tinte, lateinische Kurrent (\noindent{}Adresse)\hspace{1em}
\newline{}Versand: Stempel: »\nobreak{}\oindex{Berlin@\textbf{Berlin}, \emph{https://www.geonames.org/ontologyP.PPLC}|pwk}Berlin, S. W. 11, 28. 2. 03., 11\textsuperscript{20} V.\nobreak{}«. Stempel: »\nobreak{}\oindex{Berlin@\textbf{Berlin}, \emph{https://www.geonames.org/ontologyP.PPLC}|pwk}Berlin, S. W. 11, 28. 2. 03., 11–12 V.\nobreak{}«. Stempel: »\nobreak{}\oindex{Berlin@\textbf{Berlin}, \emph{https://www.geonames.org/ontologyP.PPLC}|pwk}Berlin, W. P9 (R6), 28 II 03, 11\textsuperscript{30} V.\nobreak{}«.  
\newline{}Schnitzler: mit Bleistift datiert: »28/2 {[}1{]}90\textcolor{gray}{3}.« }\toendnotes[C]{\smallbreak}\pstart{}{\pb}Herrn\pend{}\pstart{}Dr. Arthur Schnitzler\pend{}\pstart{}\textcolor{pink}{Palasthotel}{}\ledrightnote{\textcolor{pink}{Palasthotel Berlin}}\pend{}
{\bigskip}
\pstart
           {\pb}Samſtag.\pend
           
\pstart{}Liebſter Freund,\pend
\pstart
           Ich werde heut{ }Abend zwiſchen 10 u. 10 ½ Uhr bei \textsc{\textcolor{pink}{Josty}{}\ledrightnote{\textcolor{pink}{Café Josty}}}, \textsc{\textcolor{pink}{Potsdamer Platz}{}\ledrightnote{\textcolor{pink}{Potsdamer Platz}}}, nachſchauen, ob Du \label{K_L03367-1v}\edtext{dort
                  biſt}{\lemma{\textnormal{\emph{dort
                  biſt}}}\Cendnote{\textnormal{\textcolor{blue}{Goldmann} und \textcolor{blue}{Schnitzler} waren – womöglich in Folge dieser Verabredung –
                  am 28. 2. 1902 bei
                     \textcolor{blue}{Elisabeth Gussmann}.}}}\label{K_L03367-1h}. \uline{Du biſt}{ }\strikeout{aber}{ }\uline{aber nicht im Mindeſten gebunden.} Treffen wir uns
                  heut nicht, ſo erwarte ich morgen{ }Vormittag bis 11 ½ Uhr eine Verſtändigung\pend
           
\pstart
           Herzlichſt Dein {\\[\baselineskip]}\spacefill\mbox{P. G.}\pend
           \leftskip=0em{}\endnumbering\briefempfaengerindex{Schnitzler, Arthur@\textsc{Schnitzler, Arthur}!zzzGoldmann, Paul@\emph{von Paul Goldmann}!1903-02-281@{28. 2. 1903}|)be}\mylabel{h}  \normalsize

\doendnotes{C}
\bigskip
\vfill

\clearpage

\footnotesize

\lohead{\textsc{register}}

% Definiere theindex-Environment komplett neu ohne reledmac
\makeatletter
\renewenvironment{theindex}{%
  \section*{\indexname}%
  \setlength{\parindent}{0pt}%
  \setlength{\parskip}{0pt plus 0.3pt}%
  \let\item\@idxitem
}{%
  \clearpage
}
\makeatother

\IfFileExists{\jobname-pw.ind}{\input{\jobname-pw.ind}}{}

\end{document}

      