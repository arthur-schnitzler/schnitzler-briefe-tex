%% latex-korrekturansicht-vorspann.tex
%% Vorspann für die Korrekturansicht.
%% Lädt die gemeinsame Datei latex-vorspann.tex mit gesetztem Schalter.

\newif\ifkorrekturansicht
\korrekturansichttrue

\input{../tex-inputs/latex-vorspann}


\section[Elsa Plessner an Arthur Schnitzler, 22. 10. 1898]{L03717 Elsa Plessner an Arthur Schnitzler, 22. 10. 1898}
\nopagebreak\mylabel{L03717v}
\rehead{ }\normalsize\beginnumbering\briefempfaengerindex{Schnitzler, Arthur@\textsc{Schnitzler, Arthur}!zzzPlessner, Elsa@\emph{von Elsa Plessner}!1898-10-221@{22. 10. 1898}|(be}
\toendnotes[C]{\smallbreak\pagebreak[2]}
\correspDesc{Versand  durch Elsa Plessner am 22. 10. 1898 in Wien
\newline{}Erhalt  durch Arthur Schnitzler im Zeitraum [23. 10. 1898 – 27. 10. 1898?] in Wien}\toendnotes[C]{\smallbreak}
\Standort{DLA, A:Schnitzler, HS.1985.1.419.}
\physDesc{Visitenkarte, 420 Zeichen
\newline{}Handschrift: schwarze Tinte, lateinische Kurrent}\toendnotes[C]{\smallbreak}
\pstart
           \centering{}{\pb}den 22./10. 98.\pend
           
\pstart{}Verehrter Herr Doctor!\pend\vspace{0.5em}
\pstart
           Bitte, seien Sie so lieb wie immer und theilen Sie mir \label{K_L03717-1v}\edtext{gfl.}{\lemma{\textnormal{\emph{gfl.}}}\Cendnote{\textnormal{gefällig}}}\label{K_L03717-1} mit, wie Ihre Ansicht über
               die \label{K_L03717-2v}\edtext{beifolgende Geschichte}{\lemma{\textnormal{\emph{beifolgende Geschichte}}}\Cendnote{\textnormal{Beilage nicht erhalten. Um welchen ihrer Texte es sich 
                  gehandelt hat, ist
                  nicht zu rekonstruieren.}}}\label{K_L03717-2} ausfällt. – – – Sie wissen ja, wieviel mir stets an
               Ihrem Urtheil {\pb}liegt! –\pend
           
\pstart
           In einer der nächsten Nummern der »\textcolor{green}{Wage}\pwindex{Wage. Eine Wiener Wochenschrift@\emph{Die Wage. Eine Wiener Wochenschrift}|pw}{}\ledrightnote{\textcolor{green}{Die Wage. Eine Wiener Wochenschrift}}« werden
               Sie \label{K_L03717-3v}\edtext{eine größere \textcolor{green}{Novelle}\pwindex{Plessner, Elsa 22.\,8.\,1875 Wien – 7.\,5.\,1932 Alicante@\textsc{Plessner, Elsa} (22.\,8.\,1875 Wien – 7.\,5.\,1932 Alicante), \emph{Schriftstellerin}!neue Lehrer. Novelle@\strich\emph{Der neue Lehrer. Novelle}|pwuv}{}\ledrightnote{{$\rightarrow$}\emph{\textcolor{green}{Der neue Lehrer. Novelle}}}}{\lemma{\textnormal{\emph{eine größere Novelle}}}\Cendnote{\textnormal{\textcolor{blue}{Elsa Plessner}\pwindex{Plessner, Elsa 22.\,8.\,1875 Wien – 7.\,5.\,1932 Alicante@\textsc{Plessner, Elsa} (22.\,8.\,1875 Wien – 7.\,5.\,1932 Alicante), \emph{Schriftstellerin}|pwk} zog den
                  Text zurück, wie aus dem Brief vom 2. 1. 1899 hervorgeht. Vermutlich handelte es sich um die Novelle \emph{\textcolor{green}{Der neue Lehrer}\pwindex{Plessner, Elsa 22.\,8.\,1875 Wien – 7.\,5.\,1932 Alicante@\textsc{Plessner, Elsa} (22.\,8.\,1875 Wien – 7.\,5.\,1932 Alicante), \emph{Schriftstellerin}!neue Lehrer. Novelle@\strich\emph{Der neue Lehrer. Novelle}|pwk}}, deren Titel \textcolor{blue}{Plessner}\pwindex{Plessner, Elsa 22.\,8.\,1875 Wien – 7.\,5.\,1932 Alicante@\textsc{Plessner, Elsa} (22.\,8.\,1875 Wien – 7.\,5.\,1932 Alicante), \emph{Schriftstellerin}|pwk} im Brief vom 19. 1. 1899 erstmals erwähnt und die ihren längsten
                  überlieferten Prosatext darstellt.}}}\label{K_L03717-3} von mir finden, deren \substVorne{}\textsuperscript{U}\substDazwischen{}Beu\substHinten{}rtheil\introOben{}ung\introOben{} von Ihrer Seite mich schon jetzt
               außerordentlich interessirt. – Besten herzlichen Dank im Voraus!\pend
           
\pstart
           \centering{}\textcolor{gray}{\textbf{ELSA PLESSNER}}\pend
           \selectlanguage{ngerman}\endnumbering\briefempfaengerindex{Schnitzler, Arthur@\textsc{Schnitzler, Arthur}!zzzPlessner, Elsa@\emph{von Elsa Plessner}!1898-10-221@{22. 10. 1898}|)be}\mylabel{L03717h}  \normalsize

\doendnotes{C}
\bigskip
\vfill

\clearpage

\footnotesize

\lohead{\textsc{register}}

% Definiere theindex-Environment komplett neu ohne reledmac
\makeatletter
\renewenvironment{theindex}{%
  \section*{\indexname}%
  \setlength{\parindent}{0pt}%
  \setlength{\parskip}{0pt plus 0.3pt}%
  \let\item\@idxitem
}{%
  \clearpage
}
\makeatother

\IfFileExists{\jobname-pw.ind}{\input{\jobname-pw.ind}}{}

\end{document}

      