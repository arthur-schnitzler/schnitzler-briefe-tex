%% latex-korrekturansicht-vorspann.tex
%% Vorspann für die Korrekturansicht.
%% Lädt die gemeinsame Datei latex-vorspann.tex mit gesetztem Schalter.

\newif\ifkorrekturansicht
\korrekturansichttrue

\input{../tex-inputs/latex-vorspann}


\renewcommand{\erwaehnteInstitutionen}{Institutionen: Stadttheater Teplitz}
\renewcommand{\erwaehnteOrte}{Orte: Frankgasse 1, IX., Alsergrund, Teplice, Wien}
\renewcommand{\erwaehnteWerke}{}
\section[ Felix Salten an Arthur Schnitzler, 1{[}3{]}. 5. 1899]{Felix Salten an Arthur Schnitzler, 1{[}3{]}. 5. 1899}
\nopagebreak\mylabel{v}
\rehead{ }\normalsize\beginnumbering\briefempfaengerindex{Schnitzler, Arthur@\textsc{Schnitzler, Arthur}!zzzSalten, Felix@\emph{von Felix Salten}!1899-05-131@{13. 5. 1899}|(be}
\toendnotes[C]{\smallbreak\pagebreak[2]}\Standort{CUL, Schnitzler, B 89, A 2.}
\physDesc{Kartenbrief, 331 Zeichen
\newline{}Handschrift: Bleistift, lateinische Kurrent
\newline{}Versand: Stempel: »\nobreak{}1/1 Wien, 1{[}3{]}. 5. 99, 11–12 N\nobreak{}«. Stempel: »\nobreak{}Wien 9/3 72, 14. 5. 99, 9. V, Bestellt\nobreak{}«.  
\newline{}Schnitzler: mit Bleistift datiert: »13/5 99« 
\newline{}Ordnung: mit Bleistift von unbekannter Hand nummeriert: »116« }\toendnotes[C]{\smallbreak}\pstart{}{\pb}Herrn D\textsuperscript{r} Arthur Schnitzler\pend{}\pstart{}\textcolor{pink}{Wien IX.}{}\ledrightnote{\textcolor{pink}{IX., Alsergrund}}\pend{}\pstart{}\textcolor{pink}{Frankgaße N\textsuperscript{o}. 1}{}\ledrightnote{\textcolor{pink}{Frankgasse 1}}\pend{}
{\bigskip}
\pstart{}{\pb}Lieber,\pend
\pstart
           ich fahre jetzt nach \textcolor{pink}{Teplitz}{}\ledrightnote{\textcolor{pink}{Teplice}} – \label{K_L03292-1v}\edtext{vielleicht glückt es mir diesmal doch}{\lemma{\textnormal{\emph{vielleicht … doch}}}\Cendnote{\textnormal{siehe Felix Salten an Arthur Schnitzler, 6. 5. 1899}}}\label{K_L03292-1h}. Das Geld hab ich mir theilweise aufgetrieben. Ich weiß nicht, soll ich mir
                  \strikeout{diesmal} das \textcolor{brown}{Theater}{}\ledrightnote{{$\rightarrow$}\textcolor{brown}{Stadttheater Teplitz}} wünschen oder nicht.\pend
           
\pstart
           Montag bin ich wieder in \textcolor{pink}{Wien}{}\ledrightnote{\textcolor{pink}{Wien}}, u. Montag ist auch schon alles
               entschieden.\pend
           
\pstart
           Herzlichstes von Ihrem {\\[\baselineskip]}\spacefill\mbox{Salten}\pend
           \leftskip=0em{}\endnumbering\briefempfaengerindex{Schnitzler, Arthur@\textsc{Schnitzler, Arthur}!zzzSalten, Felix@\emph{von Felix Salten}!1899-05-131@{13. 5. 1899}|)be}\mylabel{h}  \normalsize

\doendnotes{C}
\bigskip
\vfill

\clearpage

\footnotesize

\lohead{\textsc{register}}

% Definiere theindex-Environment komplett neu ohne reledmac
\makeatletter
\renewenvironment{theindex}{%
  \section*{\indexname}%
  \setlength{\parindent}{0pt}%
  \setlength{\parskip}{0pt plus 0.3pt}%
  \let\item\@idxitem
}{%
  \clearpage
}
\makeatother

\IfFileExists{\jobname-pw.ind}{\input{\jobname-pw.ind}}{}

\end{document}

      