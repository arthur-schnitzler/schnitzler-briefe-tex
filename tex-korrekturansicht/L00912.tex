%% latex-korrekturansicht-vorspann.tex
%% Vorspann für die Korrekturansicht.
%% Lädt die gemeinsame Datei latex-vorspann.tex mit gesetztem Schalter.

\newif\ifkorrekturansicht
\korrekturansichttrue

\input{../tex-inputs/latex-vorspann}


               \section[Hugo von Hofmannsthal an Arthur Schnitzler, {[}5. 4. 1899{]}]{ Hugo von Hofmannsthal an Arthur Schnitzler, {[}5. 4. 1899{]}}\nopagebreak\mylabel{v}\rehead{ }\normalsize\beginnumbering\briefempfaengerindex{Schnitzler, Arthur@\textsc{Schnitzler, Arthur}!zzzHofmannsthal, Hugo von@\emph{von Hugo von Hofmannsthal}!1899-04-051@{{[}5. 4. 1899{]}}|(be} \toendnotes[C]{\smallbreak\pagebreak[2]} \Standort{CUL, Schnitzler, B 43.}
\physDesc{Brief, 1 Blatt, 1 Seite
\newline{}Handschrift: Bleistift, deutsche Kurrent
\newline{}Schnitzler: mit Bleistift datiert: »5/4 99« \newline{}Ordnung: 1) mit Bleistift von unbekannter Hand nummeriert: »\strikeout{145}« 2) mit Bleistift von unbekannter Hand nummeriert: »142«}\buchAbdrucke{\weitereDrucke{Hugo von Hofmannsthal, Arthur Schnitzler: \emph{Briefwechsel}. Hg. Therese Nickl und Heinrich Schnitzler. Frankfurt am Main: \emph{S. Fischer} 1964, S. 122.} }\toendnotes[C]{\smallbreak}\pstart{}{\pb}lieber\pend\pstart
           wenn es eine Stunde giebt, wo man Sie untertags trifft und nicht ſtört, ſo
                    ſchreiben Sie mir ſie. Ich reiſe kaum vor Montag wegen der armen
                        \label{K_L00912_1v}\edtext{\textcolor{blue}{Familie S.}{}\ledrightnote{\textcolor{blue}{Franziska Schlesinger}{\newline}\textcolor{blue}{Gertrude von Hofmannsthal}{\newline}\textcolor{blue}{Emil Schlesinger}}}{\lemma{\textnormal{\emph{Familie S.}}}\Cendnote{\textnormal{\textcolor{blue}{Emil Schlesinger} wird am
                            31. 5. 1899
                   sterben.}}}\label{K_L00912_1h}\pend
           \pstart
           Von Herzen Ihr{\\[\baselineskip]}\spacefill\mbox{Hugo.}\pend
           \leftskip=0em{}\endnumbering\briefempfaengerindex{Schnitzler, Arthur@\textsc{Schnitzler, Arthur}!zzzHofmannsthal, Hugo von@\emph{von Hugo von Hofmannsthal}!1899-04-051@{{[}5. 4. 1899{]}}|)be}\mylabel{h}  \normalsize

\doendnotes{C}
\bigskip
\vfill

\clearpage

\footnotesize

\lohead{\textsc{register}}

% Definiere theindex-Environment komplett neu ohne reledmac
\makeatletter
\renewenvironment{theindex}{%
  \section*{\indexname}%
  \setlength{\parindent}{0pt}%
  \setlength{\parskip}{0pt plus 0.3pt}%
  \let\item\@idxitem
}{%
  \clearpage
}
\makeatother

\IfFileExists{\jobname-pw.ind}{\input{\jobname-pw.ind}}{}

\end{document}

      