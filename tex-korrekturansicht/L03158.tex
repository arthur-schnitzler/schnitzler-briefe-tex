%% latex-korrekturansicht-vorspann.tex
%% Vorspann für die Korrekturansicht.
%% Lädt die gemeinsame Datei latex-vorspann.tex mit gesetztem Schalter.

\newif\ifkorrekturansicht
\korrekturansichttrue

\input{../tex-inputs/latex-vorspann}


\renewcommand{\erwaehntePersonen}{Personen: Richard Beer-Hofmann, Paul Goldmann, Hugo von Hofmannsthal, Charlotte Pohl-Glas, Theodor Zasche}
\renewcommand{\erwaehnteOrte}{Orte: Bad Ischl, Kopenhagen, Skandinavien, Wien}
\renewcommand{\erwaehnteWerke}{Werke: Pan, Quer durch den Wurstelprater, Terzinen, Wiener Allgemeine Zeitung}
\section[ Felix Salten an Arthur Schnitzler, 16. 7. {[}1895{]}]{Felix Salten an Arthur Schnitzler, 16. 7. {[}1895{]}}
\nopagebreak\mylabel{v}
\rehead{ }\normalsize\beginnumbering\briefempfaengerindex{Schnitzler, Arthur@\textsc{Schnitzler, Arthur}!zzzSalten, Felix@\emph{von Felix Salten}!1895-07-161@{16. 7. {[}1895{]}}|(be}
\toendnotes[C]{\smallbreak\pagebreak[2]}\Standort{CUL, Schnitzler, B 89, A 1.}
\physDesc{Brief, 1 Blatt, 4 Seiten, 1050 Zeichen
\newline{}Handschrift: Bleistift, lateinische Kurrent
\newline{}Schnitzler: mit Bleistift die Jahreszahl ergänzt: »95« 
\newline{}Ordnung: mit Bleistift von unbekannter Hand nummeriert: »57« }\toendnotes[C]{\smallbreak}
\pstart
           \raggedleft{}{\pb}Montag, 16. VII.\pend
           
\pstart
           Lieber Arthur, so viel ich zu sagen hätte, so wenig
               hab’ ich zu schreiben, wie ja Sie auch. Nur so viel, dass es mir leidlich geht, dass
               ich einiges arbeite, und hie und da aufs Land fahre. Von \textcolor{blue}{Hugo}{}\ledrightnote{\textcolor{blue}{Hugo von Hofmannsthal}} habe ich ein paarmal schöne Briefe gehabt, und habe ihm
               das zweite Heft des \textcolor{green}{Pan}{}\ledrightnote{\textcolor{green}{Pan}} gesendet, welches soeben
                  {\pb}erschienen, seine \label{K_L03158-1v}\edtext{\textcolor{green}{Terzinen}{}\ledrightnote{\textcolor{green}{Terzinen}}}{\lemma{\textnormal{\emph{Terzinen}}}\Cendnote{\textnormal{\textcolor{blue}{Loris}: \emph{\textcolor{green}{Terzinen}}. In: \emph{\textcolor{green}{Pan}}, H. 2,
                        Juni, Juli, August 1895, S. 86–88.}}}\label{K_L03158-1h} bringt. Ich
               mühe mich in \label{K_L03158-2v}\edtext{Umständen}{\lemma{\textnormal{\emph{Umständen}}}\Cendnote{\textnormal{Bezugnahme auf die die schwierige Beziehung
                  mit \textcolor{blue}{Charlotte Glas} (vgl. Felix Salten an Arthur Schnitzler, 22. 7. 1895)?}}}\label{K_L03158-2h}, die Sie ja
               kennen, und trachte \strikeout{\textcolor{gray}{nur}}, so wenig Kräfte zu verbrauchen als möglich. Das hindert nicht, dass mir
               darüber manche Stunden vergehen, die ich besser hätte anwenden
                  können\textcolor{gray}{.}\pend
           
\pstart
           Ich möchten gerne wissen, wie es mit \label{K_L03158-3v}\edtext{\textcolor{pink}{Kopenhagen}{}\ledrightnote{\textcolor{pink}{Kopenhagen}}}{\lemma{\textnormal{\emph{Kopenhagen}}}\Cendnote{\textnormal{Zu \textcolor{blue}{Schnitzler}s erster \textcolor{pink}{Skandinavien}reise
                  kam es erst ein Jahr später, im August 1896, aber ohne
                     \textcolor{blue}{Salten}, dafür mit \textcolor{blue}{Paul Goldmann} und \textcolor{blue}{Richard Beer-Hofmann}.}}}\label{K_L03158-3h}{ }{\pb}steht. Ich möchte das gerne
               bald und genau wissen, weil ich mich danach einrichten muss. Vielleicht können Sie
               mir jetzt schon etwas darüber mitttheilen. Fährt \textcolor{blue}{B-H.}{}\ledrightnote{\textcolor{blue}{Richard Beer-Hofmann}}, von dem ich Nichts höre, auch? Ich habe ihm, \strikeout{auf} wie die \label{K_L03158-4v}\edtext{\textcolor{blue}{L.}{}\ledrightnote{\textcolor{blue}{Charlotte Pohl-Glas}}}{\lemma{\textnormal{\emph{L.}}}\Cendnote{\textnormal{\textcolor{blue}{Lotte
                     Glas}?}}}\label{K_L03158-4h} mir ausgerichtet, den \label{K_L03158-5v}\edtext{\textcolor{green}{Wurstelprater}{}\ledrightnote{\textcolor{green}{Quer durch den Wurstelprater}}}{\lemma{\textnormal{\emph{Wurstelprater}}}\Cendnote{\textnormal{\textcolor{blue}{Felix Salten}:
                           \emph{\textcolor{green}{Quer durch den Wurstelprater}}. In:
                           \emph{\textcolor{green}{Wiener Allgemeine Zeitung}}, Jg. 16, Nr. 5.174,
                           2. 6. 1895, Pfingst-Beilage, S. [1]–[4] und Nr. 5.179,
                           9. 6. 1895, S. 2–4. 
                           (Mit Illustrationen von
                           \textcolor{blue}{Theo Zasche}). }}}\label{K_L03158-5h} geschickt, aber ich weiss nicht, ob er {\pb}ihn erhalten hat. Also bitte,
               theilen Sie mir mit, ob es mit \textcolor{pink}{Kphg}{}\ledrightnote{\textcolor{pink}{Kopenhagen}}. etwas ist,
               weil ich ja doch etwas anfangen möchte.\pend
           
\pstart
           Grüßen Sie \textcolor{blue}{Beer-Hofmann}{}\ledrightnote{\textcolor{blue}{Richard Beer-Hofmann}}\textcolor{gray}{,} herzlichst Ihr {\\[\baselineskip]}\spacefill\mbox{Salten}\pend
           \leftskip=0em{}\endnumbering\briefempfaengerindex{Schnitzler, Arthur@\textsc{Schnitzler, Arthur}!zzzSalten, Felix@\emph{von Felix Salten}!1895-07-161@{16. 7. {[}1895{]}}|)be}\mylabel{h}  \normalsize

\doendnotes{C}
\bigskip
\vfill

\clearpage

\footnotesize

\lohead{\textsc{register}}

% Definiere theindex-Environment komplett neu ohne reledmac
\makeatletter
\renewenvironment{theindex}{%
  \section*{\indexname}%
  \setlength{\parindent}{0pt}%
  \setlength{\parskip}{0pt plus 0.3pt}%
  \let\item\@idxitem
}{%
  \clearpage
}
\makeatother

\IfFileExists{\jobname-pw.ind}{\input{\jobname-pw.ind}}{}

\end{document}

      