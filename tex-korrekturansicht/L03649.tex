%% latex-korrekturansicht-vorspann.tex
%% Vorspann für die Korrekturansicht.
%% Lädt die gemeinsame Datei latex-vorspann.tex mit gesetztem Schalter.

\newif\ifkorrekturansicht
\korrekturansichttrue

\input{../tex-inputs/latex-vorspann}


\section[Stefan Zweig an Arthur Schnitzler, {[}29. 11. 1914{]}]{L03649 Stefan Zweig an Arthur Schnitzler, {[}29. 11. 1914{]}}
\nopagebreak\mylabel{L03649v}
\rehead{ }\normalsize\beginnumbering\briefempfaengerindex{Schnitzler, Arthur@\textsc{Schnitzler, Arthur}!zzzZweig, Stefan@\emph{von Stefan Zweig}!1914-11-291@{{[}29. 11. 1914{]}}|(be}
\toendnotes[C]{\smallbreak\pagebreak[2]}\Standort{CUL, Schnitzler, B 118.}
\physDesc{Brief, 1 Blatt, 4 Seiten, 2488 Zeichen
\newline{}Handschrift: lila Tinte, lateinische Kurrent
\newline{}Schnitzler: 1) mit Bleistift datiert: »29/11 914« und beschriftet: »\textsc{Zweig}«  2) mit rotem Buntstift eine Unterstreichung}
\buchAbdrucke{\weitereDrucke{Stefan Zweig: \emph{Briefwechsel mit Hermann Bahr, Sigmund Freud, Rainer Maria
                        Rilke und Arthur Schnitzler}. Frankfurt am Main: \emph{S. Fischer} 1987, S. 384–385.} }\toendnotes[C]{\smallbreak}
\pstart
           {\pb}\textcolor{gray}{\textbf{SZ}}\hfill \textcolor{gray}{\textbf{\textcolor{pink}{VIII. KOCHGASSE}\oindex{Kochgasse 8@\textbf{Kochgasse 8}, \emph{Wohngebäude (K.WHS)}|pw}{}\ledrightnote{\textcolor{pink}{Kochgasse 8}}}}\pend
           
\pstart
           \raggedleft{}\textcolor{gray}{\textbf{\textcolor{pink}{WIEN}\oindex{Wien@\textbf{Wien}, \emph{A.ADM2}|pw}{}\ledrightnote{\textcolor{pink}{Wien}},}}\pend
           \vspace{0.5em}
\pstart
           Verehrter lieber Herr Doktor, Sie sind so gütig, meine bescheidene
               Meinung in dieser Sache anzufragen und ich sage sie aufrichtigst. Ich glaube nur der
               erste Teil der \textcolor{green}{Berichtigung}\pwindex{Une protestation DArthur Schnitzler@\emph{Une protestation d’Arthur Schnitzler}|pwv}{}\ledrightnote{{$\rightarrow$}\emph{\textcolor{green}{Une protestation d’Arthur Schnitzler}}}
               ist \uline{notwendig}, der zweite bloss eben nur eine
               Richtigstellung einer Veränderung, die niemanden beleidigt. Und im ersten Teile hätte
               ich so gerne von einem Manne Ihrer Gerechtigkeit eines gesehen: ein Wort des
               Positiven, der Bejahung. Ich glaube, nie war eine Zeit besser für das Bekennen, nie
               es notwendiger, die Unerschütterlichkeit unserer innern Überzeugungen gegen gewisse
               Versuche aufrechtzuerhalten, den {\pb}politischen Constellationen unsere künstlerischen Empfindungen preiszugeben. Ich
               meine: es wäre schön und vorbildlich gewesen (und zugleich die stärkste, die
               schlagendste Berichtigung jeder Entstellung), \strikeout{\textcolor{gray}{Sie}} wenn Sie an einer Stelle sagten, wie sehr Sie \textcolor{blue}{Tolstoi}\pwindex{Tolstoi, Leo N. von 09.09.1828 – 20.11.1910@\textsc{Tolstoi, Leo N. von} (09.09.1828 – 20.11.1910), \emph{Schriftsteller, Schriftsteller, Krimiautor}|pw}{}\ledrightnote{\textcolor{blue}{Leo N. von Tolstoi}} bewundern und auch Ihr Verhältnis zu \textcolor{blue}{France}\pwindex{France, Anatole 16.04.1844 – 12.10.1924@\textsc{France, Anatole} (16.04.1844 – 12.10.1924), \emph{Schriftsteller}|pw}{}\ledrightnote{\textcolor{blue}{Anatole France}} und \textcolor{blue}{Maeterlinck}\pwindex{Maeterlinck, Maurice 29.08.1862 – 06.05.1949@\textsc{Maeterlinck, Maurice} (29.08.1862 – 06.05.1949), \emph{Schriftsteller}|pw}{}\ledrightnote{\textcolor{blue}{Maurice Maeterlinck}} in künstlerischer Bejahung andeuteten. Ich glaube, wir müssen
               ein Beispiel bei jedem Anlass geben, zu zeigen, dass unsere Neigungen nicht ein
               Tauschgeschäft auf Gegenliebe sind, sondern unerschütterlich selbst durch Hass und
               Anfeindung. Gerade weil Einige versuchen, jeden, der gegen Deutschland heute
               auftritt, zu negieren, statt seine Argumente zu befeinden, müssen wir unsere
               Unabhängigkeit in der eigensten engsten Welt unseres Standes und Wirkens {\pb}mit sichtbarem Willen betonen. Nichts
               ist gemäßer in diesen Tagen als Wahrhaftigkeit, die sich nicht einschüchtern lässt
               durch die Reden am Markt: ich glaube, wir sollen heute \substVorne{}\textsuperscript{\textcolor{gray}{je} als \textcolor{gray}{mehr}}\substDazwischen{}unentwegt\substHinten{}{ }\textcolor{blue}{Tolstoi}\pwindex{Tolstoi, Leo N. von 09.09.1828 – 20.11.1910@\textsc{Tolstoi, Leo N. von} (09.09.1828 – 20.11.1910), \emph{Schriftsteller, Schriftsteller, Krimiautor}|pw}{}\ledrightnote{\textcolor{blue}{Leo N. von Tolstoi}} einen der wirklichsten Menschen aller
               Zeiten nennen und brauchen nicht zu zögern mit Ehrerbietung vor der Leistung eines
                  \textcolor{blue}{Anatole France}\pwindex{France, Anatole 16.04.1844 – 12.10.1924@\textsc{France, Anatole} (16.04.1844 – 12.10.1924), \emph{Schriftsteller}|pw}{}\ledrightnote{\textcolor{blue}{Anatole France}}. Ein Vermeiden dieser
               Höflichkeitsbezeugung und dieser freien Zustimmung zu ihren Werken (die längst vor
               diesen Tagen entstanden) könnte leicht darauf deuten, wenn schon nicht eine Äusserung
               so sei doch Ihre Gesinnung jenen feindlich. Und das ist doch nicht Ihre Absicht.\pend
           
\pstart
           Ich
               wage natürlich nicht, diese meine Empfindung zur Ihren machen zu wollen: es ist nur
               eine Antwort auf Ihre gütige Frage. Gerne expediere ich den {\pb}\textcolor{green}{Brief}\pwindex{Une protestation DArthur Schnitzler@\emph{Une protestation d’Arthur Schnitzler}|pwv}{}\ledrightnote{{$\rightarrow$}\emph{\textcolor{green}{Une protestation d’Arthur Schnitzler}}} in dieser Fassung wie in jeder andern an \textcolor{blue}{R. R.}\pwindex{Rolland, Romain 29.01.1866 – 30.12.1944@\textsc{Rolland, Romain} (29.01.1866 – 30.12.1944), \emph{Schriftsteller}|pw}{}\ledrightnote{\textcolor{blue}{Romain Rolland}}, es wird ihm eine grosse Freude sein, Sie
               unter den Wenigen zu wissen, die heute, mitten im Kampf, schon an die Versöhnung
               denken.\pend
           
\pstart
           Ich bin morgen Montag nach dem \textcolor{brown}{Bureau}\orgindex{Kriegsarchiv@Kriegsarchiv|pwv}{}\ledrightnote{{$\rightarrow$}\emph{\textcolor{brown}{Kriegsarchiv}}} bestimmt zwischen 4–5 zu hause und freute mich sehr Ihres Aurufes.
               Vielen vielen Dunk für Ihr Vertrauen und alles Herzliche Ihnen und den Ihren!
               Treulichst\pend
           \pstart \spacefill\mbox{Stefan Zweig}\pend{}\selectlanguage{ngerman}\endnumbering\briefempfaengerindex{Schnitzler, Arthur@\textsc{Schnitzler, Arthur}!zzzZweig, Stefan@\emph{von Stefan Zweig}!1914-11-291@{{[}29. 11. 1914{]}}|)be}\mylabel{L03649h}  \normalsize

\doendnotes{C}
\bigskip
\vfill

\clearpage

\footnotesize

\lohead{\textsc{register}}

% Definiere theindex-Environment komplett neu ohne reledmac
\makeatletter
\renewenvironment{theindex}{%
  \section*{\indexname}%
  \setlength{\parindent}{0pt}%
  \setlength{\parskip}{0pt plus 0.3pt}%
  \let\item\@idxitem
}{%
  \clearpage
}
\makeatother

\IfFileExists{\jobname-pw.ind}{\input{\jobname-pw.ind}}{}

\end{document}

      