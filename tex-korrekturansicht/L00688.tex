%% latex-korrekturansicht-vorspann.tex
%% Vorspann für die Korrekturansicht.
%% Lädt die gemeinsame Datei latex-vorspann.tex mit gesetztem Schalter.

\newif\ifkorrekturansicht
\korrekturansichttrue

\input{../tex-inputs/latex-vorspann}


               \section[Arthur Schnitzler an Adalbert Seligmann, 15. 6. 1897]{ Arthur Schnitzler an Adalbert Seligmann,
                    15. 6. 1897}\nopagebreak\mylabel{v}\rehead{ }\normalsize\beginnumbering\briefempfaengerindex{Seligmann, Adalbert Franz@\textsc{Seligmann, Adalbert Franz}!zzzSchnitzler, Arthur@\emph{von Arthur Schnitzler}!1897-06-151@{15. 6. 1897}|(be} \toendnotes[C]{\smallbreak\pagebreak[2]} \Standort{Wienbibliothek im Rathaus, H.I.N.-96445.}
\physDesc{Visitenkarte
\newline{}Handschrift: schwarze Tinte, deutsche Kurrent}\toendnotes[C]{\smallbreak}\pstart
           \noindent{}{\pb}Herzlichſten Dank! Wirklich köſtlich.
                    Eine Bemerkung geſtatten Sie mir. So wunderbar der \textcolor{blue}{\textsc{Burckhard}}{}\ledrightnote{\textcolor{blue}{Max Eugen Burckhard}}ſche Stil getroffen;
                    die \textcolor{green}{Satire}{}\ledrightnote{→\textcolor{green}{Timon Sums, Bekenntnisse einer schönen Seele. (3798. Fortsetzung und Schluss.)}} auf ſein \uline{Weſen} geht manchmal ſehr daneben. Sie haben eine
                    Seite von ihm als das ganze genommen und ihm dadurch, ſcheint mir, in gewiſſem
                    Sinn Unrecht gethan. {\pb}Ich ſage Ihnen
                    das, weil ich das \textcolor{green}{Buch}{}\ledrightnote{→\textcolor{green}{Hinter dem Leben}} ſonſt
                    ſo wunderbar finde.\pend
           \pstart Herzlichen Gruß Ihr ſehr ergebener\pend{}\pstart
           \centering{}\textcolor{gray}{\textbf{D\textsuperscript{r} Arthur
                        Schnitzler}}\pend
           \pstart
           \textcolor{pink}{Wien}{}\ledrightnote{\textcolor{pink}{Wien}}{ }15. 6. 97.\pend
           \endnumbering\briefempfaengerindex{Seligmann, Adalbert Franz@\textsc{Seligmann, Adalbert Franz}!zzzSchnitzler, Arthur@\emph{von Arthur Schnitzler}!1897-06-151@{15. 6. 1897}|)be}\mylabel{h}  \normalsize

\doendnotes{C}
\bigskip
\vfill

\clearpage

\footnotesize

\lohead{\textsc{register}}

% Definiere theindex-Environment komplett neu ohne reledmac
\makeatletter
\renewenvironment{theindex}{%
  \section*{\indexname}%
  \setlength{\parindent}{0pt}%
  \setlength{\parskip}{0pt plus 0.3pt}%
  \let\item\@idxitem
}{%
  \clearpage
}
\makeatother

\IfFileExists{\jobname-pw.ind}{\input{\jobname-pw.ind}}{}

\end{document}

      