%% latex-korrekturansicht-vorspann.tex
%% Vorspann für die Korrekturansicht.
%% Lädt die gemeinsame Datei latex-vorspann.tex mit gesetztem Schalter.

\newif\ifkorrekturansicht
\korrekturansichttrue

\input{../tex-inputs/latex-vorspann}


\renewcommand{\erwaehntePersonen}{Personen: Marie Barthel, Paul Lindau, Felix Salten, Olga Schnitzler}
\renewcommand{\erwaehnteOrte}{Orte: Edmund-Weiß-Gasse 7, Wartburg, Wien, XVIII., Währing}
\renewcommand{\erwaehnteWerke}{}
\section[Felix Salten, Paul Lindau und Marie Barthel an Arthur Schnitzler, 9. 5. 1906]{Felix Salten, Paul Lindau und Marie Barthel an Arthur
               Schnitzler, 9. 5. 1906}
\nopagebreak\mylabel{v}
\rehead{ }\normalsize\beginnumbering\briefempfaengerindex{Schnitzler, Arthur@\textsc{Schnitzler, Arthur}!zzzBarthel, Marie@\emph{von Marie Barthel}!1906-05-091@{9. 5. 1906}|(be}\briefempfaengerindex{Schnitzler, Arthur@\textsc{Schnitzler, Arthur}!zzzLindau, Paul@\emph{von Paul Lindau}!1906-05-091@{9. 5. 1906}|(be}\briefempfaengerindex{Schnitzler, Arthur@\textsc{Schnitzler, Arthur}!zzzSalten, Felix@\emph{von Felix Salten}!1906-05-091@{9. 5. 1906}|(be}
\toendnotes[C]{\smallbreak\pagebreak[2]}\Standort{CUL, Schnitzler, B 89, B 1.}
\physDesc{Bildpostkarte, 143 Zeichen
\newline{}Handschrift Felix Salten: schwarze Tinte, lateinische Kurrent
\newline{}Handschrift Paul Lindau: schwarze Tinte, deutsche Kurrent
\newline{}Handschrift Marie Barthel: schwarze Tinte
\newline{}Versand: Stempel: »\nobreak{}\oindex{Wartburg@\textbf{Wartburg}, \emph{S.CSTL}|pwk}Wartburg, 9. 5. 06, 6–7 \textcolor{gray}{N}\nobreak{}«.  
\newline{}Schnitzler: mit Bleistift datiert: »9. 5. 90\textcolor{gray}{6}« 
\newline{}Ordnung: mit Bleistift von unbekannter Hand nummeriert: »213« }\pstart{}{\pb}Herrn D\textsuperscript{r} Arthur Schnitzler\pend{}\pstart{}\textcolor{pink}{Wien XVIII.}{}\ledrightnote{\textcolor{pink}{XVIII., Währing}}\pend{}\pstart{}\textcolor{pink}{Spöttelgaſse 7}{}\ledrightnote{\textcolor{pink}{Edmund-Weiß-Gasse 7}}\pend{}
{\bigskip}
\pstart
           \noindent{}{\pb}\textcolor{gray}{\textbf{\textcolor{pink}{Wartburg}{}\ledrightnote{\textcolor{pink}{Wartburg}} von Süd-West.}}\pend
           
\pstart
           {\pb}Viele herzliche Grüße an Sie u.
               Frau \textcolor{blue}{Olga}{}\ledrightnote{\textcolor{blue}{Olga Schnitzler}}.\pend
           \pstart Ihr \spacefill\mbox{Salten}\pend{}
\pstart
           \noindent{}{[}hs. Lindau:{]} Herzlich grüßend\pend
           
\pstart
           Ihr \spacefill\mbox{Paul Lindau}{\\[\baselineskip]}{[}hs. Barthel:{]} \spacefill\mbox{Marie Barthel}\pend
           \leftskip=0em{}\endnumbering\briefempfaengerindex{Schnitzler, Arthur@\textsc{Schnitzler, Arthur}!zzzBarthel, Marie@\emph{von Marie Barthel}!1906-05-091@{9. 5. 1906}|)be}\briefempfaengerindex{Schnitzler, Arthur@\textsc{Schnitzler, Arthur}!zzzLindau, Paul@\emph{von Paul Lindau}!1906-05-091@{9. 5. 1906}|)be}\briefempfaengerindex{Schnitzler, Arthur@\textsc{Schnitzler, Arthur}!zzzSalten, Felix@\emph{von Felix Salten}!1906-05-091@{9. 5. 1906}|)be}\mylabel{h}  \normalsize

\doendnotes{C}
\bigskip
\vfill

\clearpage

\footnotesize

\lohead{\textsc{register}}

% Definiere theindex-Environment komplett neu ohne reledmac
\makeatletter
\renewenvironment{theindex}{%
  \section*{\indexname}%
  \setlength{\parindent}{0pt}%
  \setlength{\parskip}{0pt plus 0.3pt}%
  \let\item\@idxitem
}{%
  \clearpage
}
\makeatother

\IfFileExists{\jobname-pw.ind}{\input{\jobname-pw.ind}}{}

\end{document}

      