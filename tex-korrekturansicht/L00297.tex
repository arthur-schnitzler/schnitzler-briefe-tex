%% latex-korrekturansicht-vorspann.tex
%% Vorspann für die Korrekturansicht.
%% Lädt die gemeinsame Datei latex-vorspann.tex mit gesetztem Schalter.

\newif\ifkorrekturansicht
\korrekturansichttrue

\input{../tex-inputs/latex-vorspann}


               \section[Karl Kraus u.a. an Richard Dehmel, 10. 2. 1894]{ Karl Kraus u.a. an Richard Dehmel, 10. 2. 1894}\nopagebreak\mylabel{v}\rehead{ }\normalsize\beginnumbering\briefempfaengerindex{Dehmel, Richard@\textsc{Dehmel, Richard}!zzzHofmannsthal, Hugo von@\emph{von Hugo von Hofmannsthal}!1894-02-101@{10. 2. 1894}|(be}\briefempfaengerindex{Dehmel, Richard@\textsc{Dehmel, Richard}!zzzBeer-Hofmann, Richard@\emph{von Richard Beer-Hofmann}!1894-02-101@{10. 2. 1894}|(be}\briefempfaengerindex{Dehmel, Richard@\textsc{Dehmel, Richard}!zzzSchnitzler, Arthur@\emph{von Arthur Schnitzler}!1894-02-101@{10. 2. 1894}|(be}\briefempfaengerindex{Dehmel, Richard@\textsc{Dehmel, Richard}!zzzKraus, Karl@\emph{von Karl Kraus}!1894-02-101@{10. 2. 1894}|(be} \toendnotes[C]{\smallbreak\pagebreak[2]} \Standort{Hamburg, Staats- und Universitätsbibliothek, DA:Br:K:282.}
\physDesc{Kartenbrief
\newline{}Handschrift Karl Kraus: schwarze Tinte, deutsche Kurrent\newline{}Handschrift Arthur Schnitzler: schwarze Tinte, deutsche Kurrent\newline{}Handschrift Richard Beer-Hofmann: schwarze Tinte, deutsche Kurrent\newline{}Handschrift Hugo von Hofmannsthal: schwarze Tinte, deutsche Kurrent\newline{}Versand: 1) Stempel: »\nobreak{}\oindex{I., Innere Stadt@\textbf{I., Innere Stadt}, \emph{Bezirk (A.BZK)}|pwk}Wien 1/1, 11. 2. 94, 8–9V\nobreak{}«.  2) Stempel: »\nobreak{}\oindex{Berlin-Pankow@\textbf{Berlin-Pankow}, \emph{Bezirk (A.BZK)}|pwk}Pankow bei
                                                  Berlin, 12. 2. 94, 10–11V.\nobreak{}«. }\buchAbdrucke{\weitereDrucke{Joachim Kersten, Friedrich Pfäfflin: \emph{Detlev von Liliencron entdeckt, gefeiert und gelesen
                                von Karl Kraus}. Göttingen: \emph{Wallstein} 2016, S. 116–117.} }\toendnotes[C]{\smallbreak}\pstart{}{\pb}Abſender: Karl Kraus, \textcolor{pink}{I. Maximilianſtr 13}{}\ledrightnote{\textcolor{pink}{Mahlerstraße}}. \pend{}\pstart{}\textcolor{pink}{Wien}{}\ledrightnote{\textcolor{pink}{Wien}}\pend{}\pstart{}Loris\pend{}\pstart{}Schnitzler\pend{}\pstart{}Beer-Hofmann\pend{}{\bigskip}\pstart{}Herrn\pend{}\pstart{}Richard Dehmel\pend{}\pstart{}\textcolor{pink}{Pankow bei Berlin}{}\ledrightnote{\textcolor{pink}{Berlin-Pankow}}, \textcolor{pink}{Parkstr. 25.}{}\ledrightnote{\textcolor{pink}{Parkstraße}}\pend{}{\bigskip}\pstart
           {\pb}\textcolor{pink}{Wien}{}\ledrightnote{\textcolor{pink}{Wien}}, \label{K_L00297_1v}\edtext{10. II. 93}{\lemma{\textnormal{\emph{10. II. 93}}}\Cendnote{\textnormal{Die Datierung ist, wie aus den
                            Poststempeln ersichtlich wird, um ein Jahr falsch.}}}\label{K_L00297_1h}.\pend
           \pstart
           \textcolor{pink}{Café Central}{}\ledrightnote{\textcolor{pink}{Café Central}} – die Secession\introOben{}isten\introOben{} der Secession
                        (\uline{nicht mehr} das altberühmte \textcolor{pink}{Café Grienſteidl}{}\ledrightnote{\textcolor{pink}{Café Griensteidl}} oder »Steinkrügl«, wie \textcolor{blue}{Liliencron}{}\ledrightnote{\textcolor{blue}{Detlev von Liliencron}}{ }ſagt)\pend
           \pstart
           Liebſter Dehmel, viele ſchöne Grüße, Sie welttiefer Völkerpsycholog. Meinen Brief
                    haben Sie wohl ſchon!\pend
           \pstart
           Gruß an \textcolor{blue}{Bierbaum}{}\ledrightnote{\textcolor{blue}{Otto Julius Bierbaum}}, \textcolor{blue}{Schlaf}{}\ledrightnote{\textcolor{blue}{Johannes Schlaf}}, \textcolor{blue}{Scheerbart}{}\ledrightnote{\textcolor{blue}{Paul Scheerbart}}, \textcolor{blue}{Halbe}{}\ledrightnote{\textcolor{blue}{Max Halbe}}! Ihr \spacefill\mbox{Karl Kraus.}\pend
           \pstart
           \spacefill\mbox{{[}hs. Hofmannsthal:{]} Richard Beer-Hofmann}\footnote{\noindent{}\emph{\textcolor{green}{Novellen}}. \textcolor{pink}{Berlin}{ }\emph{\textcolor{brown}{Freund {\kaufmannsund}
                                    Jäckel}}{ }1893}\footnote{\noindent{}dieser Dichter hat nicht selbst unterschrieben, weil er nicht schreiben
                            kann aber er sitzt auch da. Loris.}\pend
           \pstart
           \spacefill\mbox{Loris}\pend
           \pstart
           {[}hs. Schnitzler:{]} Herzliche Grüße \spacefill\mbox{Arthur Schnitzler}\pend
           \endnumbering\briefempfaengerindex{Dehmel, Richard@\textsc{Dehmel, Richard}!zzzHofmannsthal, Hugo von@\emph{von Hugo von Hofmannsthal}!1894-02-101@{10. 2. 1894}|)be}\briefempfaengerindex{Dehmel, Richard@\textsc{Dehmel, Richard}!zzzBeer-Hofmann, Richard@\emph{von Richard Beer-Hofmann}!1894-02-101@{10. 2. 1894}|)be}\briefempfaengerindex{Dehmel, Richard@\textsc{Dehmel, Richard}!zzzSchnitzler, Arthur@\emph{von Arthur Schnitzler}!1894-02-101@{10. 2. 1894}|)be}\briefempfaengerindex{Dehmel, Richard@\textsc{Dehmel, Richard}!zzzKraus, Karl@\emph{von Karl Kraus}!1894-02-101@{10. 2. 1894}|)be}\mylabel{h}  \normalsize

\doendnotes{C}
\bigskip
\vfill

\clearpage

\footnotesize

\lohead{\textsc{register}}

% Definiere theindex-Environment komplett neu ohne reledmac
\makeatletter
\renewenvironment{theindex}{%
  \section*{\indexname}%
  \setlength{\parindent}{0pt}%
  \setlength{\parskip}{0pt plus 0.3pt}%
  \let\item\@idxitem
}{%
  \clearpage
}
\makeatother

\IfFileExists{\jobname-pw.ind}{\input{\jobname-pw.ind}}{}

\end{document}

      