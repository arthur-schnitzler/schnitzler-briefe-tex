%% latex-korrekturansicht-vorspann.tex
%% Vorspann für die Korrekturansicht.
%% Lädt die gemeinsame Datei latex-vorspann.tex mit gesetztem Schalter.

\newif\ifkorrekturansicht
\korrekturansichttrue

\input{../tex-inputs/latex-vorspann}


               \section[Robert Adam an Arthur Schnitzler, 22. 12. 1929]{ Robert Adam an Arthur Schnitzler, 22. 12. 1929}\nopagebreak\mylabel{v}\rehead{ }\normalsize\beginnumbering\briefempfaengerindex{Schnitzler, Arthur@\textsc{Schnitzler, Arthur}!zzzAdam, Robert@\emph{von Robert Adam}!1929-12-221@{22. 12. 1929}|(be} \toendnotes[C]{\smallbreak\pagebreak[2]} \Standort{CUL, Schnitzler, B 1.}
\physDesc{Brief, 1 Blatt, 2 Seiten
\newline{}Handschrift: schwarze Tinte, deutsche Kurrent
\newline{}Schnitzler: 1) mit rotem Buntstift beschriftet: »\textcolor{green}{\textsc{So{\geminationm}erlüfte}}« 2) mit rotem Buntstift vereinzelte Unterstreichungen\newline{}Ordnung: mit Bleistift von unbekannter Hand nummeriert:
                                        »23« }\Standort{Wien, Österreichische Nationalbibliothek, Cod.ser. 52.269, 149 recto.}
\physDesc{handschriftliche Abschrift
\newline{}Handschrift: schwarze Tinte, Gabelsberger Kurzschrift}\Standort{Wien, Österreichische Nationalbibliothek, Cod.ser. 52.269, 43.}
\physDesc{maschinelle Abschrift
\newline{}Schreibmaschine}\toendnotes[C]{\smallbreak}\pstart
           \raggedleft{}{\pb}\textcolor{pink}{Wien}{}\ledrightnote{\textcolor{pink}{Wien}}, am 22. Dezember 1929\pend
           \pstart{}Hochverehrter Herr Doktor!\pend\pstart
           Nehmen Sie meinen herzlichſten Dank für die Überſendung Ihrer Komödie »\textcolor{green}{Im Spiel der Sommerlüfte}{}\ledrightnote{\textcolor{green}{Im Spiel der Sommerlüfte. In drei Aufzügen}}« entgegen!\pend
           \pstart
           Wenn ich ſo meine eigenen Produkte, auch die letzten und auch die noch gar nicht
                    geſchriebenen, ſondern erſt geplanten – es gibt leider ſolche noch immer –, im
                    Geiſt Revue paſſieren laſſe und Ihr Stück danebenhalte, dann erkenne ich ſo
                    recht, wie tief ich im Dilettantiſmus und in der Barbarei ſtecke: denn ich
                    verkenne gar nicht, daß allen meinen Hervorbringungen, und mögen ſie ſich noch
                    ſo kultiviert gehaben, etwas Barbariſches, das nun einmal mit meinem innerſten
                    Weſen verbunden ſein mag und vielleicht eine gewiſſe Eigenheit bewirkt, immerzu
                    anhaftet.\pend
           \pstart
           Wie wundervoll rein und klar iſt wieder Ihr neues \textcolor{green}{Stück}{}\ledrightnote{→\textcolor{green}{Im Spiel der Sommerlüfte. In drei Aufzügen}} gefügt und auf {\pb}welch einheitlichem Niveau ſtehen
                    und gebahren ſich Ihre Menſchen! Wie jugendfriſch betaut iſt alles, vor und nach
                    dem Gewitter, das die Luft von Leidenſchaften reinigt! Und welch geiſtreiche
                    Ergänzung der von Ihnen geſchaffenen Welt iſt dieſes Eindringen der im Kaplan
                    verkörperten religiöſen Idee in die Weltlichkeit des \textcolor{green}{Weiten Land}{}\ledrightnote{\textcolor{green}{Das weite Land. Tragikomödie in fünf Akten}}s! Man möchte, wenn man den Kreis Ihrer Menſchen
                    verlaſſen muß, noch einmal wiederholen: »\textcolor{green}{Ich werd’ oft zurückdenken an den Garten, an das liebe
                        Haus, an die Landſchaft}{}\ledrightnote{→\textcolor{green}{Im Spiel der Sommerlüfte. In drei Aufzügen}}« und an die, die drin lebten.\pend
           \pstart
           Indem ich Ihnen freudige Weihnachtsfeiertage von Herzen wünſche, verbleibe ich
                    mit vielem Dank und vielen Empfehlungen\pend
           \pstart
           Ihr ergebener{\\[\baselineskip]}\spacefill\mbox{D\textsuperscript{r}Adam}\pend
           \leftskip=0em{}\endnumbering\briefempfaengerindex{Schnitzler, Arthur@\textsc{Schnitzler, Arthur}!zzzAdam, Robert@\emph{von Robert Adam}!1929-12-221@{22. 12. 1929}|)be}\mylabel{h}  \normalsize

\doendnotes{C}
\bigskip
\vfill

\clearpage

\footnotesize

\lohead{\textsc{register}}

% Definiere theindex-Environment komplett neu ohne reledmac
\makeatletter
\renewenvironment{theindex}{%
  \section*{\indexname}%
  \setlength{\parindent}{0pt}%
  \setlength{\parskip}{0pt plus 0.3pt}%
  \let\item\@idxitem
}{%
  \clearpage
}
\makeatother

\IfFileExists{\jobname-pw.ind}{\input{\jobname-pw.ind}}{}

\end{document}

      