%% latex-korrekturansicht-vorspann.tex
%% Vorspann für die Korrekturansicht.
%% Lädt die gemeinsame Datei latex-vorspann.tex mit gesetztem Schalter.

\newif\ifkorrekturansicht
\korrekturansichttrue

\input{../tex-inputs/latex-vorspann}


\renewcommand{\erwaehntePersonen}{Personen: Paul Goldmann, Laura Marholm}
\renewcommand{\erwaehnteOrte}{Orte: Bad Ischl, Hotel und Pension Rudolfshöhe (Leopold Petter), III., Landstraße, Wien}
\renewcommand{\erwaehnteWerke}{Werke: Das Märchen. Schauspiel in drei Aufzügen, Die Zukunft, Ein Märchen}
\section[ Felix Salten an Arthur Schnitzler, 30. 8. 1894]{Felix Salten an Arthur Schnitzler, 30. 8. 1894}
\nopagebreak\mylabel{v}
\rehead{ }\normalsize\beginnumbering\briefempfaengerindex{Schnitzler, Arthur@\textsc{Schnitzler, Arthur}!zzzSalten, Felix@\emph{von Felix Salten}!1894-08-301@{30. 8. 1894}|(be}
\toendnotes[C]{\smallbreak\pagebreak[2]}\Standort{CUL, Schnitzler, B 89, A 1.}
\physDesc{Postkarte, 364 Zeichen
\newline{}Handschrift: Bleistift, lateinische Kurrent
\newline{}Versand: 1) Stempel: »\nobreak{}\oindex{III., Landstrasse@\textbf{III., Landstraße}, \emph{A.ADM3}|pwk}Wien 3/1 66, 30. 8. 94, 11–12V\nobreak{}«.   2) Stempel: »\nobreak{}\oindex{Bad Ischl@\textbf{Bad Ischl}, \emph{P.PPL}|pwk}Ischl, 31/8 94, 7–F\nobreak{}«. 
\newline{}Ordnung: mit Bleistift von unbekannter Hand nummeriert: »44« }\toendnotes[C]{\smallbreak}\pstart{}{\pb}Herrn D\textsuperscript{r} Arthur Schnitzler\pend{}\pstart{}\textcolor{pink}{Ischl}{}\ledrightnote{\textcolor{pink}{Bad Ischl}}\pend{}\pstart{}\textcolor{pink}{Pension Leopold Petter.}{}\ledrightnote{\textcolor{pink}{Hotel und Pension Rudolfshöhe (Leopold Petter)}}\pend{}
{\bigskip}
\pstart
           \noindent{}{\pb}Lieber Freund, ich höre soeben, dass im letzten Heft
               der »\textcolor{green}{Zukunft}{}\ledrightnote{\textcolor{green}{Die Zukunft}}« ein \label{K_L03143-1v}\edtext{\textcolor{green}{Artikel}{}\ledrightnote{{$\rightarrow$}\textcolor{green}{Ein Märchen}}}{\lemma{\textnormal{\emph{Artikel}}}\Cendnote{\textnormal{\textcolor{blue}{Laura Marholm}: \emph{\textcolor{green}{Ein Märchen}}. In: \emph{\textcolor{green}{Die
                        Zukunft}}, Jg. 8, 25. 8. 1894,
                     S. 368–371.}}}\label{K_L03143-1h} der \textcolor{blue}{Laura
                  Marholm}{}\ledrightnote{\textcolor{blue}{Laura Marholm}} über Ihr »\textcolor{green}{Märchen}{}\ledrightnote{\textcolor{green}{Das Märchen. Schauspiel in drei Aufzügen}}« steht. Falls
               Sie’s noch nicht gehört haben, zeige ich’s Ihnen an. Der \textcolor{green}{Aufsatz}{}\ledrightnote{{$\rightarrow$}\textcolor{green}{Ein Märchen}} soll \label{K_L03143-2v}\edtext{sehr schön u. anerkennend}{\lemma{\textnormal{\emph{sehr … anerkennend}}}\Cendnote{\textnormal{siehe A. S.: \emph{Tagebuch}, 4. 9. 1894}}}\label{K_L03143-2h} sein, Ich werde mich jedenfalls drum kümmern.\pend
           
\pstart
           Grüssen sie Herrn Doktor \label{K_L03143-3v}\edtext{\textcolor{blue}{Goldmann}{}\ledrightnote{\textcolor{blue}{Paul Goldmann}}}{\lemma{\textnormal{\emph{Goldmann}}}\Cendnote{\textnormal{\textcolor{blue}{Schnitzler} und \textcolor{blue}{Goldmann} hielten sich beide in \textcolor{pink}{Ischl} auf.}}}\label{K_L03143-3h}.\pend
           \pstart Herzlichst \spacefill\mbox{Salten.}\pend{}\endnumbering\briefempfaengerindex{Schnitzler, Arthur@\textsc{Schnitzler, Arthur}!zzzSalten, Felix@\emph{von Felix Salten}!1894-08-301@{30. 8. 1894}|)be}\mylabel{h}  \normalsize

\doendnotes{C}
\bigskip
\vfill

\clearpage

\footnotesize

\lohead{\textsc{register}}

% Definiere theindex-Environment komplett neu ohne reledmac
\makeatletter
\renewenvironment{theindex}{%
  \section*{\indexname}%
  \setlength{\parindent}{0pt}%
  \setlength{\parskip}{0pt plus 0.3pt}%
  \let\item\@idxitem
}{%
  \clearpage
}
\makeatother

\IfFileExists{\jobname-pw.ind}{\input{\jobname-pw.ind}}{}

\end{document}

      