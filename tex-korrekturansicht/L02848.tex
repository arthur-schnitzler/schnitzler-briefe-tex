%% latex-korrekturansicht-vorspann.tex
%% Vorspann für die Korrekturansicht.
%% Lädt die gemeinsame Datei latex-vorspann.tex mit gesetztem Schalter.

\newif\ifkorrekturansicht
\korrekturansichttrue

\input{../tex-inputs/latex-vorspann}


               \section[ Paul Goldmann an Arthur Schnitzler, 26. 6. {[}1898{]}]{Paul Goldmann an Arthur Schnitzler, 26. 6. {[}1898{]}}\nopagebreak\mylabel{v}\rehead{ }\normalsize\beginnumbering\briefempfaengerindex{Schnitzler, Arthur@\textsc{Schnitzler, Arthur}!zzzGoldmann, Paul@\emph{von Paul Goldmann}!1898-06-261@{26. 6. {[}1898{]}}|(be} \toendnotes[C]{\smallbreak\pagebreak[2]} \Standort{DLA, A:Schnitzler, HS.NZ85.1.3168.}
\physDesc{Brief, 3 Blätter, 12 Seiten
\newline{}Handschrift: blaue Tinte, lateinische Kurrent
\newline{}Schnitzler: mit Bleistift das Jahr »98« vermerkt }\toendnotes[C]{\smallbreak}\pstart
           \raggedleft{}{\pb}\textsc{\textcolor{pink}{Shanghai}{}\ledrightnote{\textcolor{pink}{Shanghai}}}, 26. Juni.\pend
           \pstart{}Mein lieber Freund,\pend\pstart
           Ich danke Dir für Deinen lieben Brief (vom 17. Mai)
               und alle die Nachrichten, die er enthält. \label{K_L02848-1v}\edtext{\textsc{\textcolor{blue}{Richard}{}\ledrightnote{\textcolor{blue}{Richard Beer-Hofmann}}s} Verheirathung}{\lemma{\textnormal{\emph{Richards Verheirathung}}}\Cendnote{\textnormal{\textcolor{blue}{Richard Beer-Hofmann} und \textcolor{blue}{Paula Lissy} heirateten am 14. 5. 1898. \textcolor{blue}{Schnitzler} war Trauzeuge.}}}\label{K_L02848-1h} hat mich nicht wenig überraſcht. Ich
               denke auch, er wird ſein Glück \strikeout{da\textcolor{gray}{nn}} dabei finden, und das iſt ja der einzige Geſichtspunkt, unter dem wir die
               Sache zu beurtheilen haben.\pend
           \pstart
           Aus Deinen letzten Briefen, liebſter Freund, ſehe ich nicht ohne Sorge, wie {\pb}unruhig und \label{K_L02848-2v}\edtext{verdüſtert Deine Gemüths-\strikeout{St\textcolor{gray}{ü}}Stimmung iſt und wie Du, weil es Dir im Ohre klingt}{\lemma{\textnormal{\emph{verdüſtert … klingt}}}\Cendnote{\textnormal{vgl. A. S.: \emph{Tagebuch}, 15. 5. 1898: »zu
                     Hause in tiefer Verstimmung; stets mit dem Gedanken an mein Ohr
                     beschäftigt« }}}\label{K_L02848-2h}, all’ das Herrliche mißachteſt, was ſonſt Dein
               Leben bietet. Es iſt unerhört, wenn ein Menſch, wie Du, in der Blüthe des Daſeins,
               auf der Höhe des Lebens, das Wort »verzweifelt« ausſpricht. Ich kann mir vorſtellen,
               wie läſtig die Symptome ſein mögen, die Du ſchilderſt. Bedenklich ſind ſie in keiner Weiſe\substVorne{}\textsuperscript{.}\substDazwischen{},\substHinten{} das weiß ich aus einer beſſeren Quelle, als von Dir {\pb}(\strikeout{\textcolor{gray}{ni}} nimm’ mir das nicht übel!). Ich finde, Du biſt zu nachgiebig gegen Deine
               Hypochondrie. Krankheit! Aber um des Himmels Willen, wer iſt nicht krank? Die
               körperlichen Übel ſind eine Lebens-Erſcheinung, wie alle anderen, und da ſie nicht zu
               vermeiden ſind, handelt es ſich nur darum, ihnen nicht zu erlauben, daß ſie gar zu
               viel Macht \strikeout{zu} über uns gewinnen. Ich verſichere Dich,
               daß man mit alledem fertig werden kann. Du müßteſt {\pb}Deine Lebensweiſe ändern, müßteſt nicht zu viel allein ſein, und vor allen Dingen,
               das kann ich Dir nicht oft genug ſagen, müßteſt Du aus Deinem \textcolor{pink}{Wien}{}\ledrightnote{\textcolor{pink}{Wien}}er Trübſals-Winkel hinaus in die helle und große Welt. Ich
               hoffe, die \label{K_L02848-3v}\edtext{Sommer-Reiſe}{\lemma{\textnormal{\emph{Sommer-Reiſe}}}\Cendnote{\textnormal{siehe Paul Goldmann an Arthur Schnitzler, 16. 5. 1898}}}\label{K_L02848-3h} wird Dir gut thun; und der Sommer-Reiſe müßte eine Winter-Reiſe folgen; und
               dann, hoffe ich, werde ich Dich wieder einmal ſehen {\pb}und Dich recht tüchtig auslachen, daß Du ſo \strikeout{d} dumm
               biſt, Dein Leben Dir zu vergrämen, während Du doch, den Thatſachen nach, der Froheſte
               und Ruhigſte von uns Allen ſein könnteſt und müßteſt{\dotsseven}\pend
           \pstart
           Am \strikeout{\textcolor{gray}{A}}{ }\label{K_L02848-4v}\edtext{15. Mai}{\lemma{\textnormal{\emph{15. Mai}}}\Cendnote{\textnormal{\textcolor{blue}{Schnitzler}s 36. Geburtstag}}}\label{K_L02848-4h} habe auch
               ich in Freundſchaft Deiner gedacht. Aber war es wirklich ſo ſchön \label{K_L02848-56v}\edtext{vor einem Jahre}{\lemma{\textnormal{\emph{vor einem Jahre}}}\Cendnote{\textnormal{Den 35. Geburtstag hatten sie gemeinsam in \textcolor{pink}{Paris} verbracht.}}}\label{K_L02848-56h}? Ich glaube, Du hatteſt an jenem{ }Abend{ }\label{K_L02848-5v}\edtext{Kopfſchmerzen}{\lemma{\textnormal{\emph{Kopfſchmerzen}}}\Cendnote{\textnormal{Sowohl die Kopfschmerzen, als auch, dass es ein perfekter
                  Geburtstag war, sind im \emph{\textcolor{green}{Tagebuch}} vermerkt,
                     vgl. A. S.: \emph{Tagebuch}, 15. 5. 1897}}}\label{K_L02848-5h} und warſt verſtimmt. Das haſt Du ſchon wieder {\pb}vergeſſen, und ſo wirſt Du wahrſcheinlich auch in
               einem Jahre wieder vergeſſen haben, was Dich jetzt quält.\pend
           \pstart
           Dein \label{K_L02848-111v}\edtext{\textcolor{green}{Buch}{}\ledrightnote{→\textcolor{green}{Die Frau des Weisen. Novelletten}}}{\lemma{\textnormal{\emph{Buch}}}\Cendnote{\textnormal{\textcolor{blue}{Schnitzler}s erste Sammlung von Prosatexten,
                     \emph{\textcolor{green}{Die Frau des Weisen. Novelletten}}, erschien
                  am 3. 5. 1898.}}}\label{K_L02848-111h} habe ich geleſen. Es ſind herrliche Seiten
               darin. Der »\textcolor{green}{Ehrentag}{}\ledrightnote{\textcolor{green}{Der Ehrentag}}« iſt mir das Liebſte
               daraus. Aber wenn man ſchon einmal im Stande iſt, dieſe erſchütternde Figur des
                  \label{K_L02848-6v}\edtext{\begin{otherlanguage}{french}\textsc{raté}\end{otherlanguage}}{\lemma{\textnormal{\emph{raté}}}\Cendnote{\textnormal{französisch: Versager; gemeint war die
                  Figur des \textcolor{green}{August von Witte}}}}\label{K_L02848-6h} zu zeichnen, warum das Alles nur gleichſam als Epiſode hineinzwängen in eine
                  \textcolor{green}{Liebesgeſchichte}{}\ledrightnote{→\textcolor{green}{Der Ehrentag}} zwiſchen
               einem Theater-{\pb}Menſch und \strikeout{\textcolor{gray}{×}} einem düſteren \label{K_L02848-8v}\edtext{\begin{otherlanguage}{french}\textsc{Poseur}\end{otherlanguage}}{\lemma{\textnormal{\emph{Poseur}}}\Cendnote{\textnormal{französisch: Angeber}}}\label{K_L02848-8h} von \textsc{\textcolor{green}{August}{}\ledrightnote{→\textcolor{green}{Der Ehrentag}}}? Warum hat nicht die Rohheit des Directors den »\textcolor{green}{Ehrentag}{}\ledrightnote{\textcolor{green}{Der Ehrentag}}« angeſtiftet, ſtatt der Eiferſucht eines Liebhabers? Ich glaube,
               das würde die \textcolor{green}{Geſchichte}{}\ledrightnote{→\textcolor{green}{Der Ehrentag}} noch
               mehr vertieft und vermenſchlicht haben. Ich meine auch, Du ſollteſt Dich jetzt eine
               Zeit lang zwingen, \uline{keine Liebesgeſchichten mehr} zu
               ſchreiben. Tief ergeifend iſt auch der {\pb}»\textcolor{green}{Abſchied}{}\ledrightnote{\textcolor{green}{Ein Abschied}}«. Nur die letzten zwanzig Zeilen ſtimmen
               mir nicht recht zum Ganzen\textcolor{gray}{,} ich weiß nicht warum? Die »\textcolor{green}{Frau des Weiſen}{}\ledrightnote{\textcolor{green}{Die Frau des Weisen. Erzählung}}« mag ich nicht, die letzte \textcolor{green}{Geſchichte}{}\ledrightnote{→\textcolor{green}{Die Toten schweigen}} auch nicht ſehr,
               trotz der meiſterhaften Darſtellung (\label{K_L02848-55v}\edtext{ſie iſt doch eine dumme Gans, daß ſie dem Manne Alles ſagt}{\lemma{\textnormal{\emph{ſie … ſagt}}}\Cendnote{\textnormal{\emph{\textcolor{green}{Die Toten schweigen}} endet damit, dass die
                  Frau zu einem Zeitpunkt, an dem ihre außereheliche Affäre nicht mehr entdeckt
                  werden kann, beschließt ihrem Ehemann die Wahrheit zu sagen.}}}\label{K_L02848-55h}!). Der Erfolg
               Deines \textcolor{green}{Buch}{}\ledrightnote{→\textcolor{green}{Die Frau des Weisen. Novelletten}}es freut mich von
               Herzen. Er iſt redlich verdient, {\pb}denn ich glaube
               nicht, daß ſeit Langem in \textcolor{pink}{Deutſchland}{}\ledrightnote{\textcolor{pink}{Deutschland}} eine
               Sammlung ſo guter Novellen erſchienen iſt. Du biſt ein beneidenswerther Menſch, daß
               Du zu ſolchen Leiſtungen fähig biſt. Aber nein, ich vergaß, Du haſt Ohrenklingen, Du
               biſt der Unglücklichſte der Unglücklichen!\pend
           \pstart
           Mach’ Dich darauf gefaßt, daß meine theure \textcolor{blue}{Tante}{}\ledrightnote{→\textcolor{blue}{Jenny Mamroth}} in der \textcolor{green}{Frankfurter
                  Ztg.}{}\ledrightnote{\textcolor{green}{Frankfurter Zeitung}} auf Dein \textcolor{green}{Buch}{}\ledrightnote{→\textcolor{green}{Die Frau des Weisen. Novelletten}}{ }\label{K_L02848-7v}\edtext{ſchimpft}{\lemma{\textnormal{\emph{ſchimpft}}}\Cendnote{\textnormal{Eine Rezension in der \emph{\textcolor{green}{Frankfurter Zeitung}} ist nicht belegt.}}}\label{K_L02848-7h}.\pend
           \pstart
           Welches iſt das \label{K_L02848-11v}\edtext{\textcolor{green}{Stück}{}\ledrightnote{→\textcolor{green}{Das Vermächtnis. Schauspiel in drei Akten}}}{\lemma{\textnormal{\emph{Stück}}}\Cendnote{\textnormal{\emph{\textcolor{green}{Das Vermächtnis}} wurde am 8. 10. 1898 am \textcolor{pink}{Deutschen Theater} in \textcolor{pink}{Berlin} uraufgeführt.}}}\label{K_L02848-11h}, {\pb}das im Herbſt das »\textcolor{brown}{Deutſche Theater}{}\ledrightnote{\textcolor{brown}{Deutsches Theater Berlin}}« herausbringen ſoll? Sehr traurig oder ein wenig luſtig?
               Viel Handlung? Viel Perſonen? Viel Pſychologie? Bitte, ſchreib’ mir ein Wort darüber.
               Ich weiß gar nichts davon.\pend
           \pstart
           Ich ſehe viel Seltſames, aber die Schönheit fehlt in dieſem \textcolor{pink}{Lande}{}\ledrightnote{→\textcolor{pink}{China}}. Ich ſehne mich unendlich nach ein
               paar Wochen \textcolor{pink}{Italien}{}\ledrightnote{\textcolor{pink}{Italien}}, nach Paläſten und alten
               Bildern! {\pb}Die Reiſe zieht ſich ſehr in die Länge.
               Ich arbeite ſchwer, leide unſäglich unter meiner Impotenz\strikeout{,} dieſer neuen Welt gegenüber, habe Wochen lang Kopfſchmerzen, bin nervöſer
               als je und fühle mich\strikeout{,} mehr noch als früher aus dem
               Geleiſe geworfen. Heut{ }Abend fahre ich den \textsc{\textcolor{pink}{Yang-tse}{}\ledrightnote{\textcolor{pink}{Jangtsekiang}}} hinauf (100 Grad \textsc{Fahrenheit} im Schatten). Meine
               Adreſſe bleibt \textsc{\textcolor{pink}{Shanghai}{}\ledrightnote{\textcolor{pink}{Shanghai}}}, {\pb}\textcolor{pink}{deutſches Poſtamt}{}\ledrightnote{→\textcolor{pink}{Deutsches Postamt in Shanghai}}. Bitte, ſag’
               dem \textsc{\textcolor{blue}{Richard}{}\ledrightnote{\textcolor{blue}{Richard Beer-Hofmann}}}, daß ich ihm nach \textcolor{pink}{\textsc{Wollzeile} 15}{}\ledrightnote{\textcolor{pink}{Wollzeile}} einen Brief und ein Paket geſandt
               habe. \pend
           \pstart
           \strikeout{B\textcolor{gray}{it}} Grüße mir Deine \textcolor{blue}{Freundin}{}\ledrightnote{→\textcolor{blue}{Marie Reinhard}} recht herzlich und ſei ſelbſt tauſend Mal gegrüßt von Deinem
               treuen{\\[\baselineskip]}\spacefill\mbox{Paul Goldmnn}\pend
           \leftskip=0em{}\endnumbering\briefempfaengerindex{Schnitzler, Arthur@\textsc{Schnitzler, Arthur}!zzzGoldmann, Paul@\emph{von Paul Goldmann}!1898-06-261@{26. 6. {[}1898{]}}|)be}\mylabel{h}  \normalsize

\doendnotes{C}
\bigskip
\vfill

\clearpage

\footnotesize

\lohead{\textsc{register}}

% Definiere theindex-Environment komplett neu ohne reledmac
\makeatletter
\renewenvironment{theindex}{%
  \section*{\indexname}%
  \setlength{\parindent}{0pt}%
  \setlength{\parskip}{0pt plus 0.3pt}%
  \let\item\@idxitem
}{%
  \clearpage
}
\makeatother

\IfFileExists{\jobname-pw.ind}{\input{\jobname-pw.ind}}{}

\end{document}

      