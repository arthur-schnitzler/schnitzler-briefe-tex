%% latex-korrekturansicht-vorspann.tex
%% Vorspann für die Korrekturansicht.
%% Lädt die gemeinsame Datei latex-vorspann.tex mit gesetztem Schalter.

\newif\ifkorrekturansicht
\korrekturansichttrue

\input{../tex-inputs/latex-vorspann}


\renewcommand{\erwaehntePersonen}{Personen: W. Bogdanovits, Jens Peter Jacobsen, Vladimir Galaktionovič Korolenko}
\renewcommand{\erwaehnteInstitutionen}{Institutionen: »Phönix« Versicherung}
\renewcommand{\erwaehnteOrte}{Orte: Hotel Erzherzog Karl, Kärntner Straße, Kärntnerring 12/Bösendorferstraße 11, Wien, Währingerstraße}
\renewcommand{\erwaehnteWerke}{}
\section[Felix Salten an Arthur Schnitzler, {[}24?. 10. 1893{]}]{Felix Salten an Arthur Schnitzler, {[}24?. 10. 1893{]}}
\nopagebreak\mylabel{v}
\rehead{ }\normalsize\beginnumbering\briefempfaengerindex{Schnitzler, Arthur@\textsc{Schnitzler, Arthur}!zzzSalten, Felix@\emph{von Felix Salten}!1893-10-241@{{[}24?. 10. 1893{]}}|(be}
\toendnotes[C]{\smallbreak\pagebreak[2]}\Standort{CUL, Schnitzler, B 89, A 1.}
\physDesc{Brief, 1 Blatt, 3 Seiten, 348 Zeichen
\newline{}Handschrift: Bleistift, lateinische Kurrent
\newline{}Schnitzler: 1) mit Bleistift datiert: »2\substVorne{}\textsuperscript{\textcolor{gray}{5}}\substDazwischen{}\textcolor{gray}{4}\substHinten{}/X 93«  2) mit Bleistift auf der vierten Seite vermerkt: »{\pb}Dr. \textcolor{blue}{v. Bogdanovits}{ }\textcolor{pink}{Erzh. Karl}{ }\textcolor{pink}{Kärnt.}«
\newline{}Ordnung: mit Bleistift von unbekannter Hand nummeriert: »33« }\toendnotes[C]{\smallbreak}
\pstart
           \noindent{}{\pb}lieber Arthur, vom \textcolor{brown}{Bureau}{}\ledrightnote{{$\rightarrow$}\textcolor{brown}{»Phönix« Versicherung}} musste ich nach \textcolor{pink}{Hause}{}\ledrightnote{{$\rightarrow$}\textcolor{pink}{Währingerstraße}} gehen, und liege im Bette. Bitte, seien Sie nicht \label{K_L03130-1v}\edtext{bös’}{\lemma{\textnormal{\emph{bös’}}}\Cendnote{\textnormal{Bezug unklar}}}\label{K_L03130-1h}, aber mein Knie thut mir weh, sehr weh.
               Wenn Sie können, so {\pb}\label{K_L03130-2v}\edtext{schauen Sie im Lauf des Tages zu mir}{\lemma{\textnormal{\emph{schauen … mir}}}\Cendnote{\textnormal{Das kann als Indiz dafür genommen werden, dass die bei der Tagesziffer nicht
                  verlässlich lesbare Datierung durch \textcolor{blue}{Schnitzler} stimmt, da er am 24. 10. 1893 bei \textcolor{blue}{Salten}{ }\textcolor{pink}{zu Hause} war.}}}\label{K_L03130-2h}. Sind
               Sie bei diesem Brief \textcolor{gray}{gu}t! \textcolor{pink}{zu Hause}{}\ledrightnote{{$\rightarrow$}\textcolor{pink}{Kärntnerring 12/Bösendorferstraße 11}}, so senden Sie mir bitte irgend einen Roma\substVorne{}\textsuperscript{m}\substDazwischen{}n\substHinten{}, \textcolor{blue}{Korolenko}{}\ledrightnote{\textcolor{blue}{Vladimir Galaktionovič Korolenko}}, oder \textcolor{blue}{Jacobsen}{}\ledrightnote{\textcolor{blue}{Jens Peter Jacobsen}} oder {\pb}so etwas. Auf
               Wiedersehen.\pend
           
\pstart
           Herzlichst {\\[\baselineskip]}Ihr {\\[\baselineskip]}\spacefill\mbox{Salten}\pend
           \leftskip=0em{}\endnumbering\briefempfaengerindex{Schnitzler, Arthur@\textsc{Schnitzler, Arthur}!zzzSalten, Felix@\emph{von Felix Salten}!1893-10-241@{{[}24?. 10. 1893{]}}|)be}\mylabel{h}  \normalsize

\doendnotes{C}
\bigskip
\vfill

\clearpage

\footnotesize

\lohead{\textsc{register}}

% Definiere theindex-Environment komplett neu ohne reledmac
\makeatletter
\renewenvironment{theindex}{%
  \section*{\indexname}%
  \setlength{\parindent}{0pt}%
  \setlength{\parskip}{0pt plus 0.3pt}%
  \let\item\@idxitem
}{%
  \clearpage
}
\makeatother

\IfFileExists{\jobname-pw.ind}{\input{\jobname-pw.ind}}{}

\end{document}

      