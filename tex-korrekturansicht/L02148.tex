%% latex-korrekturansicht-vorspann.tex
%% Vorspann für die Korrekturansicht.
%% Lädt die gemeinsame Datei latex-vorspann.tex mit gesetztem Schalter.

\newif\ifkorrekturansicht
\korrekturansichttrue

\input{../tex-inputs/latex-vorspann}


               \section[Richard Beer-Hofmann an Arthur Schnitzler, 10. 8. 1913]{ Richard Beer-Hofmann an Arthur Schnitzler, 10. 8. 1913}\nopagebreak\mylabel{v}\rehead{ }\normalsize\beginnumbering\briefempfaengerindex{Schnitzler, Arthur@\textsc{Schnitzler, Arthur}!zzzBeer-Hofmann, Richard@\emph{von Richard Beer-Hofmann}!1913-08-101@{10. 8. 1913}|(be} \toendnotes[C]{\smallbreak\pagebreak[2]} \Standort{CUL, Schnitzler, B 8.}
\physDesc{Bildpostkarte
\newline{}Handschrift: Bleistift, lateinische Kurrent\newline{}Versand: Stempel: »\nobreak{}\oindex{Santa Maria Elisabetta@\textbf{Santa Maria Elisabetta}, \emph{Bezirk (A.BZK)}|pwk}S. Elisabetta di Lido (Venezia), 10. 8. 13.\nobreak{}«.  \newline{}Ordnung: mit Bleistift von unbekannter Hand nummeriert: »253« }\buchAbdrucke{\weitereDrucke{Arthur Schnitzler, Richard Beer-Hofmann: \emph{Briefwechsel 1891–1931}. Hg. Konstanze Fliedl. Wien, Zürich: \emph{Europaverlag} 1992, S. 218.} }\toendnotes[C]{\smallbreak}\pstart{}{\pb}Herrn\pend{}\pstart{}Arthur Schnitzler\pend{}\pstart{}Insel \textcolor{pink}{Brioni}{}\ledrightnote{\textcolor{pink}{Brijuni}}\pend{}\pstart{}\textcolor{pink}{Austria}{}\ledrightnote{\textcolor{pink}{Österreich}}.\pend{}{\bigskip}\pstart
           \noindent{}\centering{}{\pb}\textcolor{gray}{\textbf{\textcolor{pink}{Lido}{}\ledrightnote{\textcolor{pink}{Lido}} – \textcolor{pink}{Venezia}{}\ledrightnote{\textcolor{pink}{Venedig}}.
                     \textcolor{pink}{Hôtel des Bains}{}\ledrightnote{\textcolor{pink}{Grand Hotel des Bains}}.}}\pend
           \pstart
           \noindent{}{\pb}Lieber Arthur! Dies ist nun unsere Nordsee! \textcolor{pink}{Brioni}{}\ledrightnote{\textcolor{pink}{Brijuni}} würde ich gerne sehen – mit Ihnen als Staffage – aber es
               würde die kurze Zeit die ich hierbleibe zersplittern. \textcolor{blue}{Bubis}{}\ledrightnote{\textcolor{blue}{Gabriel Beer-Hofmann}} wegen – der Aufnahmsprüfung in die III machen und dazu vorbereitet
               werden muss soll ich schon am 1 Sept in \textcolor{pink}{Wien}{}\ledrightnote{\textcolor{pink}{Wien}}{ }sein. Wann sind Sie zurück?\pend
           \pstart
           \label{T_L02148-1v}\edtext{Ihnen, Ihrer \textcolor{blue}{Frau}{}\ledrightnote{→\textcolor{blue}{Olga Schnitzler}}}{\lemma{\textnormal{\emph{Ihnen, Ihrer Frau}}}\Cendnote{\textnormal{ab hier oberhalb und verkehrt zum Text}}}\label{T_L02148-1h} u. d. \textcolor{blue}{Kindern}{}\ledrightnote{→\textcolor{blue}{Heinrich Schnitzler}{\newline}→\textcolor{blue}{Lili Schnitzler}} herzliche Grüsse von uns
                  Allen!\spacefill\mbox{R.}\pend
           \endnumbering\briefempfaengerindex{Schnitzler, Arthur@\textsc{Schnitzler, Arthur}!zzzBeer-Hofmann, Richard@\emph{von Richard Beer-Hofmann}!1913-08-101@{10. 8. 1913}|)be}\mylabel{h}  \normalsize

\doendnotes{C}
\bigskip
\vfill

\clearpage

\footnotesize

\lohead{\textsc{register}}

% Definiere theindex-Environment komplett neu ohne reledmac
\makeatletter
\renewenvironment{theindex}{%
  \section*{\indexname}%
  \setlength{\parindent}{0pt}%
  \setlength{\parskip}{0pt plus 0.3pt}%
  \let\item\@idxitem
}{%
  \clearpage
}
\makeatother

\IfFileExists{\jobname-pw.ind}{\input{\jobname-pw.ind}}{}

\end{document}

      