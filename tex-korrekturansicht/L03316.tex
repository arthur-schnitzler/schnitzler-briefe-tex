%% latex-korrekturansicht-vorspann.tex
%% Vorspann für die Korrekturansicht.
%% Lädt die gemeinsame Datei latex-vorspann.tex mit gesetztem Schalter.

\newif\ifkorrekturansicht
\korrekturansichttrue

\input{../tex-inputs/latex-vorspann}


\renewcommand{\erwaehntePersonen}{Personen: Friedrich von Bodenstedt, Hugo Felix, Mirzä Şäfi Vazeh, Josef Willomitzer, Bogumil Zepler}
\renewcommand{\erwaehnteOrte}{Orte: Bad Ischl, Berlin, Jung-Wiener Theater zum Lieben Augustin, Lago di Garda, München, Vahrn, Venedig, Verona, Wien}
\renewcommand{\erwaehnteWerke}{Werke: Der einsame Weg. Schauspiel in fünf Akten, Die Gedenktafel der Prinzessin Anna, Die Insel. Monatsschrift mit Buchschmuck und Illustrationen, Hafisa, Neue Loreley}
\section[ Felix Salten an Arthur Schnitzler, 28. 7. 1901]{Felix Salten an Arthur Schnitzler, 28. 7. 1901}
\nopagebreak\mylabel{v}
\rehead{ }\normalsize\beginnumbering\briefempfaengerindex{Schnitzler, Arthur@\textsc{Schnitzler, Arthur}!zzzSalten, Felix@\emph{von Felix Salten}!1901-07-281@{28. 7. 1901}|(be}
\toendnotes[C]{\smallbreak\pagebreak[2]}\Standort{CUL, Schnitzler, B 89, A 2.}
\physDesc{Brief, 1 Blatt, 2 Seiten, 937 Zeichen
\newline{}Handschrift: schwarze Tinte, lateinische Kurrent
\newline{}Ordnung: mit Bleistift von unbekannter Hand nummeriert: »140« }\toendnotes[C]{\smallbreak}
\pstart
           \raggedleft{}{\pb}\textcolor{pink}{Ischl}{}\ledrightnote{\textcolor{pink}{Bad Ischl}}, 28. Juli 01\pend
           
\pstart
           Lieber Freund.{ }Dienstag gehe ich nach \textcolor{pink}{Wien}{}\ledrightnote{\textcolor{pink}{Wien}} zurück. Bleibe dort ein paar Wochen, dann muß ich freilich wieder
               hierher. Dann habe ich noch ein paar Fahrten nach \textcolor{pink}{München}{}\ledrightnote{\textcolor{pink}{München}}{ }{\kaufmannsund} nach \textcolor{pink}{Berlin}{}\ledrightnote{\textcolor{pink}{Berlin}} zu
               machen, aber erst im September. Vielleicht ist es nöthig,
               dass ich vorher, Ende August, od. Anfangs Septemb. noch mit \textcolor{blue}{Felix}{}\ledrightnote{\textcolor{blue}{Hugo Felix}} zusammentreffe. Er schlägt \textcolor{pink}{Verona}{}\ledrightnote{\textcolor{pink}{Verona}} vor, ich \textcolor{pink}{Venedig}{}\ledrightnote{\textcolor{pink}{Venedig}}.
               Wenn Sie nun diese Zeit am \label{K_L03316-1v}\edtext{\textcolor{pink}{Gardasee}{}\ledrightnote{\textcolor{pink}{Lago di Garda}}}{\lemma{\textnormal{\emph{Gardasee}}}\Cendnote{\textnormal{Das war nicht der Fall.}}}\label{K_L03316-1h} sind,
               könnten wir, falls es Ihnen recht ist dorthin, oder doch in die Nähe kommen. Vor
               wenigen Tagen war \textcolor{blue}{Bogumil Zepler}{}\ledrightnote{\textcolor{blue}{Bogumil Zepler}} da, mit
               hübschen neuen \label{K_L03316-2v}\edtext{\textcolor{green}{Sachen}{}\ledrightnote{{$\rightarrow$}\textcolor{green}{Neue Loreley}{\newline}{$\rightarrow$}\textcolor{green}{Hafisa}}}{\lemma{\textnormal{\emph{Sachen}}}\Cendnote{\textnormal{Zwei \textcolor{green}{Lieder} lassen sich nachweisen, wobei nur das
                  zweite bei der Premiere am 16. 11. 1901 im \textcolor{pink}{Jung-Wiener Theater zum
                     Lieben Augustin} aufgeführt wurde: \emph{\textcolor{green}{Neue
                     Loreley}} (Balladentext von \textcolor{blue}{Josef
                     Willomitzer}) und \emph{\textcolor{green}{Hafisa}} nach einer
                  Vorlage von \textcolor{blue}{Mirzä Şäfi Vazeh} in der
                  Übersetzung von \textcolor{blue}{Friedrich von
                  Bodenstedt}.}}}\label{K_L03316-2h}, die ich erworben habe. Von den \textcolor{pink}{Wien}{}\ledrightnote{\textcolor{pink}{Wien}}er Leuten ist nichts, aber auch noch garnichts da, was die
               Sache allerdings nicht erleichtert. Doch war ich darauf {\pb}so ziemlich vorbereitet.\pend
           
\pstart
           Dass wir im selben Zug fuhren und uns nicht sahen? Von wo –? und bis wohin?\pend
           
\pstart
           Gratuliere zum neuen \label{K_L03316-3v}\edtext{\textcolor{green}{Stück}{}\ledrightnote{{$\rightarrow$}\textcolor{green}{Der einsame Weg. Schauspiel in fünf Akten}}}{\lemma{\textnormal{\emph{Stück}}}\Cendnote{\textnormal{siehe Arthur Schnitzler an Felix Salten, 10. 8. 1901}}}\label{K_L03316-3h} und bin sehr neugierig. Die \label{K_L03316-4v}\edtext{\textcolor{green}{Prinzessin Anna}{}\ledrightnote{\textcolor{green}{Die Gedenktafel der Prinzessin Anna}} ist erschienen. Soll ich Ihnen
               das Heft der »\textcolor{green}{Insel}{}\ledrightnote{\textcolor{green}{Die Insel. Monatsschrift mit Buchschmuck und Illustrationen}}}{\lemma{\textnormal{\emph{Prinzessin … »Insel}}}\Cendnote{\textnormal{\textcolor{blue}{Felix Salten}: \emph{\textcolor{green}{Die Gedenktafel der Prinzessin Anna}}. In: \emph{\textcolor{green}{Die Insel}}, Jg. 2, Quartal 4, Nr. 10, Juli 1901, S. 67–117.}}}\label{K_L03316-4h}« schicken?\pend
           
\pstart
           Herzlichst {\\[\baselineskip]}Ihr {\\[\baselineskip]}\spacefill\mbox{Salten}\pend
           \leftskip=0em{}\endnumbering\briefempfaengerindex{Schnitzler, Arthur@\textsc{Schnitzler, Arthur}!zzzSalten, Felix@\emph{von Felix Salten}!1901-07-281@{28. 7. 1901}|)be}\mylabel{h}  \normalsize

\doendnotes{C}
\bigskip
\vfill

\clearpage

\footnotesize

\lohead{\textsc{register}}

% Definiere theindex-Environment komplett neu ohne reledmac
\makeatletter
\renewenvironment{theindex}{%
  \section*{\indexname}%
  \setlength{\parindent}{0pt}%
  \setlength{\parskip}{0pt plus 0.3pt}%
  \let\item\@idxitem
}{%
  \clearpage
}
\makeatother

\IfFileExists{\jobname-pw.ind}{\input{\jobname-pw.ind}}{}

\end{document}

      