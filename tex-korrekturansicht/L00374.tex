%% latex-korrekturansicht-vorspann.tex
%% Vorspann für die Korrekturansicht.
%% Lädt die gemeinsame Datei latex-vorspann.tex mit gesetztem Schalter.

\newif\ifkorrekturansicht
\korrekturansichttrue

\input{../tex-inputs/latex-vorspann}


               \section[Arthur Schnitzler an Richard Beer-Hofmann, 29. 9. 1894]{ Arthur Schnitzler an Richard Beer-Hofmann, 29. 9. 1894}\nopagebreak\mylabel{v}\rehead{ }\normalsize\beginnumbering\briefempfaengerindex{Beer-Hofmann, Richard@\textsc{Beer-Hofmann, Richard}!zzzSchnitzler, Arthur@\emph{von Arthur Schnitzler}!1894-09-291@{29. 9. 1894}|(be} \toendnotes[C]{\smallbreak\pagebreak[2]} \Standort{CUL, Schnitzler, B 8.1, S. 23–24.}
\physDesc{maschinelle Abschrift
\newline{}Schreibmaschine\newline{}Ordnung: von unbekannter Hand nummeriert: »42« }\buchAbdrucke{\weitereDrucke{Arthur Schnitzler, Richard Beer-Hofmann: \emph{Briefwechsel 1891–1931}. Hg. Konstanze Fliedl. Wien, Zürich: \emph{Europaverlag} 1992, S. 60–61.} }\toendnotes[C]{\smallbreak}\pstart
           \raggedleft{}{\pb}\textcolor{pink}{Wien}{}\ledrightnote{\textcolor{pink}{Wien}}, 29. 9. 94.\pend
           \pstart
           Lieber Richard, \uline{zwei} (due) Karten hab ich Ihnen nach \textcolor{pink}{Pallanza}{}\ledrightnote{\textcolor{pink}{Pallanza}} geschrieben – das ist doch mehr als Mau? – Sie sind
               offenbar verloren gegangen.\pend
           \pstart
           (Wer, – ich? (\textcolor{blue}{Leon}{}\ledrightnote{\textcolor{blue}{Victor Léon}} und \textcolor{blue}{Waldberg}{}\ledrightnote{\textcolor{blue}{Heinrich von Waldberg}}, \textcolor{blue}{Blumenthal}{}\ledrightnote{\textcolor{blue}{Oskar Blumenthal}} und
                  \textcolor{blue}{Kadelburg}{}\ledrightnote{\textcolor{blue}{Gustav Kadelburg}}, \textcolor{blue}{Brociner}{}\ledrightnote{\textcolor{blue}{Marco Brociner}} und \textcolor{blue}{Gerhard}{}\ledrightnote{\textcolor{blue}{Leopold Geiringer}})). –\pend
           \pstart
           Gestern Eröffnung \textcolor{pink}{Josefstadt}{}\ledrightnote{\textcolor{pink}{Theater in der Josefstadt}}; mit Dank des Herrn \textcolor{blue}{Léon}{}\ledrightnote{\textcolor{blue}{Victor Léon}} im Frack, mit gekränkter Miene. Sehr amüsant,
               abgesehn vom \textcolor{green}{1. Akt}{}\ledrightnote{→\textcolor{green}{Tata-Toto}}. –\pend
           \pstart
           Mein \textcolor{green}{Stück}{}\ledrightnote{→\textcolor{green}{Liebelei. Schauspiel in drei Akten}} – zwei Akte bis auf
               letzte Feile (exclus.) vollendet. Wohl in acht Tagen fertig, – bühnenfertig in etwa
               4 Wochen, bühnenwirksam – wann? –\pend
           \pstart
           Wie fühlen Sie sich? »Fliesst die Arbeit munter fort?« –\pend
           \pstart
           {\pb}»\textcolor{green}{Zeit}{}\ledrightnote{\textcolor{green}{Die Zeit. Wiener Wochenschrift}}«
               soll besorgt werden. – Bitte schreiben Sie häufiger – die Gemäldegalerie, die so
               hoffnungsvoll begonnen, hat rasch geendet. –\pend
           \pstart
           Herzlich der Ihre{\\[\baselineskip]}\spacefill\mbox{\strikeout{Richard} entschuldigen – Arthur.}\pend
           \leftskip=0em{}\pstart
           \noindent{}»Aeh, Kamerad, und was machen Weiber?« (\textcolor{green}{Carricaturen}{}\ledrightnote{\textcolor{green}{Wiener Caricaturen}}, \textcolor{green}{Floh}{}\ledrightnote{\textcolor{green}{Der Floh}}, \textcolor{green}{Bombe}{}\ledrightnote{\textcolor{green}{Die Bombe}}, \textcolor{green}{Wiener
                  Witzblatt}{}\ledrightnote{\textcolor{green}{Wiener Witzblatt}}).\pend
           \stanza{}Und jene schöne, die vor Zeiten Euch\newverse{}Das Wasser auf den Nachttisch Abends stellte –\newverse{}Mit der Madonna holdem Lächeln – denkt\newverse{}Ihr dieses guten Mädchens manchmal noch, –\newverse{}Das sicher manches gegen die Empfängnis,\newverse{}Doch gegen das Beflecktsein gar nichts hatte –?\stanzaend{}\pstart
           Der Obige, was ich leider nicht auf jenes Mädchen beziehn kann.\pend
           \pstart
           \spacefill\mbox{A.}\pend
           \pstart
           (nach \textcolor{pink}{Florenz}{}\ledrightnote{\textcolor{pink}{Florenz}} a posta ferma)\pend
           \endnumbering\briefempfaengerindex{Beer-Hofmann, Richard@\textsc{Beer-Hofmann, Richard}!zzzSchnitzler, Arthur@\emph{von Arthur Schnitzler}!1894-09-291@{29. 9. 1894}|)be}\mylabel{h}  \normalsize

\doendnotes{C}
\bigskip
\vfill

\clearpage

\footnotesize

\lohead{\textsc{register}}

% Definiere theindex-Environment komplett neu ohne reledmac
\makeatletter
\renewenvironment{theindex}{%
  \section*{\indexname}%
  \setlength{\parindent}{0pt}%
  \setlength{\parskip}{0pt plus 0.3pt}%
  \let\item\@idxitem
}{%
  \clearpage
}
\makeatother

\IfFileExists{\jobname-pw.ind}{\input{\jobname-pw.ind}}{}

\end{document}

      