%% latex-korrekturansicht-vorspann.tex
%% Vorspann für die Korrekturansicht.
%% Lädt die gemeinsame Datei latex-vorspann.tex mit gesetztem Schalter.

\newif\ifkorrekturansicht
\korrekturansichttrue

\input{../tex-inputs/latex-vorspann}


\section[Arthur Schnitzler an Richard Beer-Hofmann, 29. 9. 1894]{L00374 Arthur Schnitzler an Richard Beer-Hofmann, 29. 9. 1894}
\nopagebreak\mylabel{L00374v}
\rehead{ }\normalsize\beginnumbering\briefempfaengerindex{, @\textsc{, }!zzz, @\emph{von  }!1894-09-291@{29. 9. 1894}|(be}
\toendnotes[C]{\smallbreak\pagebreak[2]}\Standort{CUL, Schnitzler, B 8.1, S. 23–24.}
\physDesc{Brief, maschinenschriftliche Abschrift, 1 Blatt, 1 Seite, 1150 Zeichen
\newline{}Schreibmaschine
\newline{}Ordnung: von unbekannter Hand nummeriert: »42« }
\buchAbdrucke{\weitereDrucke{Arthur Schnitzler, Richard Beer-Hofmann: \emph{Briefwechsel 1891–1931}. Herausgegeben von Konstanze Fliedl. Wien, Zürich: \emph{Europaverlag} 1992, S. 60–61.} }\toendnotes[C]{\smallbreak}
\pstart
           \raggedleft{}{\pb}\textcolor{pink}{Wien}\oindex{Wien@\textbf{Wien}, \emph{Verwaltungsgebiet}|pw}{}\ledrightnote{\textcolor{pink}{Wien}}, 29. 9. 94.\pend
           \vspace{0.5em}
\pstart
           Lieber Richard,{ }\uline{zwei} (due) Karten hab ich Ihnen nach \textcolor{pink}{Pallanza}\oindex{Pallanza@\textbf{Pallanza}|pw}{}\ledrightnote{\textcolor{pink}{Pallanza}} geschrieben – das ist doch mehr als
               Mau? – Sie sind offenbar verloren gegangen.\pend
           
\pstart
           (Wer, – ich? (\textcolor{blue}{Leon}\pwindex{Léon, Victor 4.\,1.\,1858 Senica – 23.\,2.\,1940 Wien@\textsc{Léon, Victor} (4.\,1.\,1858 Senica – 23.\,2.\,1940 Wien), \emph{Schriftsteller, Dramaturg}|pw}{}\ledrightnote{\textcolor{blue}{Victor Léon}} und \textcolor{blue}{Waldberg}\pwindex{Waldberg, Heinrich von 2.\,3.\,1860 Iași – 20.\,10.\,1942 Konzentrationslager Theresienstadt@\textsc{Waldberg, Heinrich von} (2.\,3.\,1860 Iași – 20.\,10.\,1942 Konzentrationslager Theresienstadt), \emph{Schriftsteller}|pw}{}\ledrightnote{\textcolor{blue}{Heinrich von Waldberg}}, \textcolor{blue}{Blumenthal}\pwindex{Blumenthal, Oskar 13.\,3.\,1852 Berlin – 24.\,4.\,1917 ebd.@\textsc{Blumenthal, Oskar} (13.\,3.\,1852 Berlin – 24.\,4.\,1917 ebd.), \emph{Schriftsteller, Journalist, Theaterleiter}|pw}{}\ledrightnote{\textcolor{blue}{Oskar Blumenthal}} und
                  \textcolor{blue}{Kadelburg}\pwindex{Kadelburg, Gustav 26.\,7.\,1851 Budapest – 11.\,9.\,1925 Berlin@\textsc{Kadelburg, Gustav} (26.\,7.\,1851 Budapest – 11.\,9.\,1925 Berlin), \emph{Schriftsteller, Schauspieler}|pw}{}\ledrightnote{\textcolor{blue}{Gustav Kadelburg}}, \textcolor{blue}{Brociner}\pwindex{Brociner, Marco 20.\,10.\,1852 Iași – 12.\,4.\,1942 Wien@\textsc{Brociner, Marco} (20.\,10.\,1852 Iași – 12.\,4.\,1942 Wien), \emph{Schriftsteller, Journalist, Kritiker}|pw}{}\ledrightnote{\textcolor{blue}{Marco Brociner}} und \textcolor{blue}{Gerhard}\pwindex{Geiringer, Leopold 27.\,6.\,1851 Wien – 29.\,5.\,1900 ebd.@\textsc{Geiringer, Leopold} (27.\,6.\,1851 Wien – 29.\,5.\,1900 ebd.), \emph{Schriftsteller, Dramaturg}|pw}{}\ledrightnote{\textcolor{blue}{Leopold Geiringer}})). –\pend
           
\pstart
           Gestern Eröffnung \textcolor{pink}{Josefstadt}\oindex{Wien@\textbf{Wien}!VIII., Josefstadt@\textbf{VIII., Josefstadt}!Theater in der Josefstadt@\textbf{Theater in der Josefstadt}, \emph{Theater}|pw}{}\ledrightnote{\textcolor{pink}{Theater in der Josefstadt}}; mit Dank des Herrn
                  \textcolor{blue}{Léon}\pwindex{Léon, Victor 4.\,1.\,1858 Senica – 23.\,2.\,1940 Wien@\textsc{Léon, Victor} (4.\,1.\,1858 Senica – 23.\,2.\,1940 Wien), \emph{Schriftsteller, Dramaturg}|pw}{}\ledrightnote{\textcolor{blue}{Victor Léon}} im Frack, mit gekränkter Miene. Sehr
               amüsant, abgesehn vom \textcolor{green}{1.
               Akt}\pwindex{Bilhaud, Paul 31.\,12.\,1854 Allichamps – 8.\,1.\,1932@\textsc{Bilhaud, Paul} (31.\,12.\,1854 Allichamps – 8.\,1.\,1932), \emph{Schriftsteller}!Tata-Toto. Vaudeville in drei Akten@\strich\emph{Tata-Toto. Vaudeville in drei Akten}|pwv}\pwindex{Barré, Albert 29.\,12.\,1854 Paris – 31.\,5.\,1910 ebd.@\textsc{Barré, Albert} (29.\,12.\,1854 Paris – 31.\,5.\,1910 ebd.), \emph{Schriftsteller}!Tata-Toto. Vaudeville in drei Akten@\strich\emph{Tata-Toto. Vaudeville in drei Akten}|pwv}{}\ledrightnote{{$\rightarrow$}\emph{\textcolor{green}{Tata-Toto. Vaudeville in drei Akten}}}. –\pend
           
\pstart
           Mein \textcolor{green}{Stück}\pwindex{Schnitzler, Arthur 15. 5. 1862 Wien – 21. 10. 1931 ebd.@\textsc{Schnitzler, Arthur} (15. 5. 1862 Wien – 21. 10. 1931 ebd.), \emph{Schriftsteller, Mediziner}!Liebelei. Schauspiel in drei Akten@\strich\emph{Liebelei. Schauspiel in drei Akten}|pwv}{}\ledrightnote{{$\rightarrow$}\emph{\textcolor{green}{Liebelei. Schauspiel in drei Akten}}} – zwei Akte bis auf
               letzte Feile (exclus.) vollendet. Wohl in acht Tagen fertig, – bühnenfertig in etwa
               4 Wochen, bühnenwirksam – wann? –\pend
           
\pstart
           Wie fühlen Sie sich? »Fliesst die Arbeit munter fort?« –\pend
           
\pstart
           {\pb}»\textcolor{green}{Zeit}\pwindex{Zeit. Wiener Wochenschrift@\emph{Die Zeit. Wiener Wochenschrift}|pw}{}\ledrightnote{\textcolor{green}{Die Zeit. Wiener Wochenschrift}}« soll besorgt werden. – Bitte schreiben Sie häufiger – die
               Gemäldegalerie, die so hoffnungsvoll begonnen, hat rasch geendet. –\pend
           
\pstart
           Herzlich der Ihre{\\[\baselineskip]}\spacefill\mbox{\strikeout{Richard} entschuldigen – Arthur.}\pend
           \leftskip=0em{}
\pstart
           \noindent{}»Aeh, Kamerad, und was machen Weiber?« (\textcolor{green}{Carricaturen}\pwindex{Wiener Caricaturen@\emph{Wiener Caricaturen}|pw}{}\ledrightnote{\textcolor{green}{Wiener Caricaturen}}, \textcolor{green}{Floh}\pwindex{Franzos, Karl Emil 25.\,10.\,1848 Tschortkiw – 28.\,1.\,1904 Berlin@\textsc{Franzos, Karl Emil} (25.\,10.\,1848 Tschortkiw – 28.\,1.\,1904 Berlin), \emph{Schriftsteller, Journalist}!Floh@\strich\emph{Der Floh}|pw}\pwindex{Hevesi, Ludwig 20.\,12.\,1843 Heves – 27.\,2.\,1910 Wien@\textsc{Hevesi, Ludwig} (20.\,12.\,1843 Heves – 27.\,2.\,1910 Wien), \emph{Schriftsteller, Journalist}!Floh@\strich\emph{Der Floh}|pw}\pwindex{Bauer, Julius 15.\,10.\,1853 Szigetvár – 11.\,6.\,1941 Wien@\textsc{Bauer, Julius} (15.\,10.\,1853 Szigetvár – 11.\,6.\,1941 Wien), \emph{Schriftsteller, Journalist, Kritiker}!Floh@\strich\emph{Der Floh}|pw}\pwindex{Herzl, Theodor 2.\,5.\,1860 Budapest – 3.\,7.\,1904 Edlach@\textsc{Herzl, Theodor} (2.\,5.\,1860 Budapest – 3.\,7.\,1904 Edlach), \emph{Schriftsteller, Journalist}!Floh@\strich\emph{Der Floh}|pw}\pwindex{Landesberg, Alexander 15.\,7.\,1848 Oradea – 14.\,6.\,1916 Wien@\textsc{Landesberg, Alexander} (15.\,7.\,1848 Oradea – 14.\,6.\,1916 Wien), \emph{Schriftsteller, Journalist}!Floh@\strich\emph{Der Floh}|pw}\pwindex{Schlesinger, Sigmund 15.\,6.\,1832 Nové Mesto nad Váhom – 7.\,3.\,1918 Wien@\textsc{Schlesinger, Sigmund} (15.\,6.\,1832 Nové Mesto nad Váhom – 7.\,3.\,1918 Wien), \emph{Schriftsteller}!Floh@\strich\emph{Der Floh}|pw}\pwindex{Gans-Ludassy, Julius von 13.\,4.\,1858 Wien – 30.\,9.\,1922 ebd.@\textsc{Gans-Ludassy, Julius von} (13.\,4.\,1858 Wien – 30.\,9.\,1922 ebd.), \emph{Schriftsteller, Journalist, Herausgeber}!Floh@\strich\emph{Der Floh}|pw}\pwindex{Walzel, Camillo 11.\,2.\,1829 Magdeburg – 17.\,3.\,1895 Wien@\textsc{Walzel, Camillo} (11.\,2.\,1829 Magdeburg – 17.\,3.\,1895 Wien), \emph{Schriftsteller, Theaterleiter}!Floh@\strich\emph{Der Floh}|pw}{}\ledrightnote{\textcolor{green}{Der Floh}}, \textcolor{green}{Bombe}\pwindex{Bombe@\emph{Die Bombe}|pw}{}\ledrightnote{\textcolor{green}{Die Bombe}}, \textcolor{green}{Wiener
                     Witzblatt}\pwindex{Wiener Witzblatt@\emph{Wiener Witzblatt}|pw}{}\ledrightnote{\textcolor{green}{Wiener Witzblatt}}).\pend
           \stanza{}Und jene schöne, die vor Zeiten EuchDas Wasser auf den Nachttisch Abends stellte –Mit der Madonna holdem Lächeln – denktIhr dieses guten Mädchens manchmal noch, –Das sicher manches gegen die Empfängnis,Doch gegen das Beflecktsein gar nichts hatte –?\stanzaend{}
\pstart
           Der Obige, was ich leider nicht auf jenes Mädchen beziehn kann.\pend
           
\pstart
           \spacefill\mbox{A.}\pend
           
\pstart
           (nach \textcolor{pink}{Florenz}\oindex{Florenz@\textbf{Florenz}|pw}{}\ledrightnote{\textcolor{pink}{Florenz}} a posta ferma)\pend
           \selectlanguage{ngerman}\endnumbering\briefempfaengerindex{, @\textsc{, }!zzz, @\emph{von  }!1894-09-291@{29. 9. 1894}|)be}\mylabel{L00374h}  \normalsize

\doendnotes{C}
\bigskip
\vfill

\clearpage

\footnotesize

\lohead{\textsc{register}}

% Definiere theindex-Environment komplett neu ohne reledmac
\makeatletter
\renewenvironment{theindex}{%
  \section*{\indexname}%
  \setlength{\parindent}{0pt}%
  \setlength{\parskip}{0pt plus 0.3pt}%
  \let\item\@idxitem
}{%
  \clearpage
}
\makeatother

\IfFileExists{\jobname-pw.ind}{\input{\jobname-pw.ind}}{}

\end{document}

      