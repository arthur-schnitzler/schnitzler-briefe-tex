%% latex-korrekturansicht-vorspann.tex
%% Vorspann für die Korrekturansicht.
%% Lädt die gemeinsame Datei latex-vorspann.tex mit gesetztem Schalter.

\newif\ifkorrekturansicht
\korrekturansichttrue

\input{../tex-inputs/latex-vorspann}


               \section[Paul Goldmann an Arthur Schnitzler, Paul Goldmann an Arthur Schnitzler, 4. 7. {[}1896{]}]{ Paul Goldmann an Arthur Schnitzler, 4. 7. {[}1896{]}}\nopagebreak\mylabel{v}\rehead{ }\normalsize\beginnumbering\briefempfaengerindex{Schnitzler, Arthur@\textsc{Schnitzler, Arthur}!zzzGoldmann, Paul@\emph{von Paul Goldmann}!1896-07-041@{4. 7. {[}1896{]}}|(be} \toendnotes[C]{\smallbreak\pagebreak[2]} \Standort{DLA, A:Schnitzler, HS.NZ85.1.3166.}
\physDesc{Brief, 1 Blatt, 3 Seiten
\newline{}Handschrift: blaue Tinte, deutsche Kurrent
\newline{}Schnitzler: mit Bleistift das Jahr »96« vermerkt }\toendnotes[C]{\smallbreak}\pstart
           \noindent{}{\pb}\textcolor{gray}{\textbf{\textbf{\textcolor{brown}{Frankfurter Zeitung}{}\ledrightnote{\textcolor{brown}{Frankfurter Zeitung}}}}}\pend
           \pstart
           \textcolor{gray}{\textbf{(\textcolor{brown}{\begin{otherlanguage}{french}Gazette de Francfort\end{otherlanguage}}{}\ledrightnote{\textcolor{brown}{Frankfurter Zeitung}}).}}\pend
           \pstart
           \textcolor{gray}{\textbf{\textbf{\begin{otherlanguage}{french}Fondateur M.\end{otherlanguage}{ }\textcolor{blue}{L. Sonnemann}{}\ledrightnote{\textcolor{blue}{Leopold Sonnemann}}.}}}\pend
           \pstart
           \begin{otherlanguage}{french}\textcolor{gray}{\textbf{\textcolor{green}{Journal}{}\ledrightnote{→\textcolor{green}{Frankfurter Zeitung}} politique,
                        financier,}}\end{otherlanguage}\pend
           \pstart
           \begin{otherlanguage}{french}\textcolor{gray}{\textbf{commercial et littéraire.}}\end{otherlanguage}\pend
           \pstart
           \begin{otherlanguage}{french}\textcolor{gray}{\textbf{\textbf{Paraissant trois fois par jour.}}}\end{otherlanguage}\hfill \textsc{\textcolor{pink}{Paris}{}\ledrightnote{\textcolor{pink}{Paris}}}, 4. Juli.\pend
           \pstart
           \begin{otherlanguage}{french}\textcolor{gray}{\textbf{\textbf{Bureau à \textcolor{pink}{Paris}{}\ledrightnote{\textcolor{pink}{Paris}}}}}\end{otherlanguage}\pend
           \pstart
           \begin{otherlanguage}{french}\textcolor{gray}{\textbf{\textbf{\textcolor{pink}{24. Rue Feydeau}{}\ledrightnote{\textcolor{pink}{rue Feydeau}}.}}}\end{otherlanguage}\pend
           \pstart\center{}Mein lieber Freund,\pend\pstart
           Alſo ſchön willkommen in \label{K_L02780-1v}\edtext{\textcolor{pink}{Hamburg}{}\ledrightnote{\textcolor{pink}{Hamburg}}}{\lemma{\textnormal{\emph{Hamburg}}}\Cendnote{\textnormal{\textcolor{blue}{Schnitzler} hielt sich von 4. 7. 1896 bis 7. 7. 1896 im \textcolor{pink}{Hamburg} auf, bevor er nach \textcolor{pink}{Norwegen} weiterreiste.}}}\label{K_L02780-1h} und von Herzen frohe
               Fahrt!\pend
           \pstart
           Dieſer Brief ſoll Dir nur einen Gruß von mir \strikeout{\textcolor{gray}{×}\-\textcolor{gray}{×}\-\textcolor{gray}{×}} bringen.\pend
           \pstart
           Neues weiß ich nicht. Auch hab’ ich keine Ahnung, wann ich von hier fortkomme. Die
               verfluchten Schwätzer im {\pb}\label{K_L02780-3v}\edtext{\textsc{\textcolor{pink}{Palais Bourbon}{}\ledrightnote{\textcolor{pink}{Palais Bourbon}}}}{\lemma{\textnormal{\emph{Palais Bourbon}}}\Cendnote{\textnormal{Sitz der \emph{\textcolor{brown}{französischen Nationalversammlung}}}}}\label{K_L02780-3h} machen keiner\textcolor{gray}{l}lei Anſtalten, in die Ferien zu gehen. Auch
               ſonſt erſcheint mir meine Reiſe im dunkelſten Nebel.\pend
           \pstart
           Ich ſchreibe \strikeout{\textcolor{gray}{nac}h} nach \textcolor{pink}{Hamburg}{}\ledrightnote{\textcolor{pink}{Hamburg}},
               weil das noch im Bereich der Vorſtellungs-Möglichkeit liegt. Aber kannſt Du Dir,
               ehrlich geſagt, ein \textsc{Poste restante}-Büreau in \textsc{\textcolor{pink}{Trondjhem}{}\ledrightnote{→\textcolor{pink}{Trondheim}}} vorſtellen? Ich nicht.\pend
           \pstart
           Wie alle Jahre habe ich natürlich Furcht, Dich {\pb}wiederzuſehen, – diesmal aber mehr als je.\pend
           \pstart
           Gott befohlen, mein lieber Freund, und möge Dir der \textcolor{pink}{ſchwed}{}\ledrightnote{→\textcolor{pink}{Schweden}}iſche Himmel hold ſein (wenn es
               überhaupt in dieſem \textcolor{pink}{Lande}{}\ledrightnote{→\textcolor{pink}{Schweden}},
               das ſeit \label{K_L02780-4v}\edtext{\textcolor{blue}{Guſtav Adolph}{}\ledrightnote{\textcolor{blue}{Gustav II. Adolf}}}{\lemma{\textnormal{\emph{Guſtav Adolph}}}\Cendnote{\textnormal{\textcolor{pink}{schwed}ischer \textcolor{blue}{König} zwischen 1611 und 1632}}}\label{K_L02780-4h} jede Exiſtenzberechtigung verloren hat, ſo etwas gibt, wie einen Himmel).\pend
           \pstart
           Viele treue Grüße! {\\[\baselineskip]}Dein {\\[\baselineskip]}\spacefill\mbox{Paul Goldmann}\pend
           \leftskip=0em{}\endnumbering\briefempfaengerindex{Schnitzler, Arthur@\textsc{Schnitzler, Arthur}!zzzGoldmann, Paul@\emph{von Paul Goldmann}!1896-07-041@{4. 7. {[}1896{]}}|)be}\mylabel{h}\begin{anhang}\end{anhang}\normalsize

\doendnotes{C}
\bigskip
\vfill

\clearpage

\footnotesize

\lohead{\textsc{register}}

% Definiere theindex-Environment komplett neu ohne reledmac
\makeatletter
\renewenvironment{theindex}{%
  \section*{\indexname}%
  \setlength{\parindent}{0pt}%
  \setlength{\parskip}{0pt plus 0.3pt}%
  \let\item\@idxitem
}{%
  \clearpage
}
\makeatother

\IfFileExists{\jobname-pw.ind}{\input{\jobname-pw.ind}}{}

\end{document}

      