%% latex-korrekturansicht-vorspann.tex
%% Vorspann für die Korrekturansicht.
%% Lädt die gemeinsame Datei latex-vorspann.tex mit gesetztem Schalter.

\newif\ifkorrekturansicht
\korrekturansichttrue

\input{../tex-inputs/latex-vorspann}


               \section[Arthur Schnitzler an Richard Beer-Hofmann, 29. 7. 1907]{ Arthur Schnitzler an Richard Beer-Hofmann, 29. 7. 1907}\nopagebreak\mylabel{v}\rehead{ }\normalsize\beginnumbering\briefempfaengerindex{Beer-Hofmann, Richard@\textsc{Beer-Hofmann, Richard}!zzzSchnitzler, Arthur@\emph{von Arthur Schnitzler}!1907-07-291@{29. 7. 1907}|(be} \toendnotes[C]{\smallbreak\pagebreak[2]} \Standort{YCGL, MSS 31.}
\physDesc{Brief, 1 Blatt, 2 Seiten, Umschlag (auf der Rückseite Fotografien von »Hôtel Pension BavariaMeran-Obermais« und »Hôtel Pension Wildbad-Waldbrunn, Pustertal (Tirol)«.)
\newline{}Handschrift: schwarze Tinte, lateinische Kurrent\newline{}Versand: 1) Stempel: »\nobreak{}\oindex{Wildbad Waldbrunn@\textbf{Wildbad Waldbrunn}, \emph{Hotel (K.HTL)}|pwk}Wildbad Waldbrunn, 29. Jul. 1907\nobreak{}«.  2) Stempel: »\nobreak{}\oindex{Welsberg-Taisten@\textbf{Welsberg-Taisten}, \emph{Besiedelter Ort (A.BSO)}|pwk}{[}Wel{]}sb{[}erg{]}, 29. 7. 07\nobreak{}«. 3) Stempel: »\nobreak{}\oindex{Maria Schutz@\textbf{Maria Schutz}, \emph{https://www.geonames.org/ontologyP.PPL}|pwk}{\pb}Maria
                                          {[}Schu{]}tz, 0\textcolor{gray}{3.} 7. 07, 8–9N\nobreak{}«. 
\newline{}Beer-Hofmann: mit blauem Buntstift den Zeitpunkt der Beantwortung
                                 festgehalten: »B. 30/VII 07« }\buchAbdrucke{\weitereDrucke{Arthur Schnitzler, Richard Beer-Hofmann: \emph{Briefwechsel 1891–1931}. Hg. Konstanze Fliedl. Wien, Zürich: \emph{Europaverlag} 1992, S. 181–182.} }\toendnotes[C]{\smallbreak}\pstart{}{\pb}\textcolor{gray}{\textbf{\textcolor{pink}{HÔTEL {\kaufmannsund} PENSION
                           BAVARIA}{}\ledrightnote{\textcolor{pink}{Hotel {\kaufmannsund} Pension Bavaria}}{ }\textcolor{pink}{MERAN-OBERMAIS}{}\ledrightnote{\textcolor{pink}{Obermais}}}}\pend{}\pstart{}\textcolor{gray}{\textbf{HÔTEL {\kaufmannsund} PENSION \textcolor{pink}{WILDBAD WALDBRUNN}{}\ledrightnote{\textcolor{pink}{Wildbad Waldbrunn}}, \textcolor{pink}{PUSTERTAL}{}\ledrightnote{\textcolor{pink}{Pustertal}}}}\pend{}\pstart{}\textcolor{gray}{\textbf{BESITZER: \textcolor{blue}{JOS. BÖHM}{}\ledrightnote{\textcolor{blue}{Josef Böhm}}}}\pend{}{\bigskip}\pstart{}Dr. Richard Beer-Hofmann\pend{}\pstart{}\textcolor{pink}{Maria Schutz}{}\ledrightnote{\textcolor{pink}{Maria Schutz}}\pend{}\pstart{}am \textcolor{pink}{Semmering}{}\ledrightnote{\textcolor{pink}{Semmering}}\pend{}{\bigskip}\pstart
           \noindent{}{\pb}\textcolor{gray}{\textbf{Telegramm-Adresse: Böhm – Welsberg}}\pend
           \pstart
           \textcolor{gray}{\textbf{Hôtel {\kaufmannsund} Pension \textcolor{pink}{Wildbad Waldbrunn}{}\ledrightnote{\textcolor{pink}{Wildbad Waldbrunn}}}}\pend
           \pstart
           \textcolor{gray}{\textbf{bei \textcolor{pink}{Welsberg}{}\ledrightnote{\textcolor{pink}{Welsberg-Taisten}}
                     (Eilzughaltestelle)}}\pend
           \pstart
           \textcolor{gray}{\textbf{1150 M. \textsuperscript{ü}/Meer.\hspace*{1.5em}\textcolor{pink}{Hochpusterthal}{}\ledrightnote{\textcolor{pink}{Pustertal}} (\textcolor{pink}{Tirol}{}\ledrightnote{\textcolor{pink}{Tirol}})}}\pend
           \pstart
           \textcolor{gray}{\textbf{Heilkräftiges altbekanntes Bad in prachtvoller Lage.}}\pend
           \pstart
           \textcolor{gray}{\textbf{Ausgezeichnete Trinkquelle.}}\pend
           \pstart
           \textcolor{gray}{\textbf{70 mit allem Comfort eingerichtete Zimmer.}}\pend
           \pstart
           \raggedleft{}\textcolor{gray}{\textbf{\textcolor{pink}{Waldbrunn}{}\ledrightnote{\textcolor{pink}{Welsberg-Taisten}}, den}}{ }29. 7. \textcolor{gray}{\textbf{190}}7\pend
           \pstart{}lieber Richard,\pend\pstart
           im \textcolor{pink}{Lidohotel}{}\ledrightnote{\textcolor{pink}{Palast Hotel Lido}}, \textcolor{pink}{Riva}{}\ledrightnote{\textcolor{pink}{Riva del Garda}}, war ich ein oder \label{K_L01695_1v}\edtext{zwei
                  Tage}{\lemma{\textnormal{\emph{zwei
                  Tage}}}\Cendnote{\textnormal{siehe A. S.: \emph{Tagebuch}, 18. 8. 1903}}}\label{K_L01695_1h}, vor 4 Jahren, Ende August, mit \textcolor{blue}{Goldmann}{}\ledrightnote{\textcolor{blue}{Paul Goldmann}}; es war ein comfortables, sogar elegantes Hotel, das Essen damals
               mäßig; eine Badeanstalt we{\geminationn} ich mich recht erinnere im
               Park. Würde wohl wieder dort absteigen. Gerühmt wurde mir s. Z. (weiß nicht mehr von
               wem) »\textcolor{pink}{Hotel Riva}{}\ledrightnote{\textcolor{pink}{Grand Hotel Riva}}«. \textcolor{pink}{Hotel du lac}{}\ledrightnote{\textcolor{pink}{Hotel du Lac}} ke{\geminationn} ich nicht. \textcolor{pink}{Torbole}{}\ledrightnote{\textcolor{pink}{Torbole sul Garda}} kenn ich nur ausflugs- nicht aufenthaltsweise.\pend
           \pstart
           Auch wir haben \textcolor{pink}{Dolomitenstraßenpläne}{}\ledrightnote{\textcolor{pink}{Große Dolomitenstraße}}, \textcolor{pink}{Pordoi}{}\ledrightnote{\textcolor{pink}{Pordoijoch}}, \textcolor{pink}{Karersee}{}\ledrightnote{\textcolor{pink}{Karersee}}
               – wollen in \textcolor{pink}{Meran}{}\ledrightnote{\textcolor{pink}{Meran}} einige Zeit verweilen,
               vielleicht auch am \textcolor{pink}{Gardasee}{}\ledrightnote{\textcolor{pink}{Lago di Garda}}. Vielleicht fügt es
               sich, dass wir zusa{\geminationm}en paßwandern? Wir sind hier (denk
               ich) bis nach dem 20–25. etwa – wenn Sie von Ihrem \textcolor{pink}{Kär{[}n{]}thner}{}\ledrightnote{\textcolor{pink}{Kärnten}}see kommen, könnten Sie
               hier, in \textcolor{pink}{Welsberg}{}\ledrightnote{\textcolor{pink}{Welsberg-Taisten}} ein paar Tage {\pb}Station machen; vielleicht daß wir gemeinsam
               aufbrechen, südwärts –?\pend
           \pstart
           \textcolor{blue}{Hugo}{}\ledrightnote{\textcolor{blue}{Hugo von Hofmannsthal}{\newline}\textcolor{blue}{Gertrude von Hofmannsthal}}’s waren 10 Tage da, wie Sie schon
               wissen; es sind wirklich schattenlos angenehme gewesen. – Wir fühlen uns alle hier
               recht wohl; ich arbeite nicht unfleißig und hoffe mit dem \textcolor{green}{Roman}{}\ledrightnote{→\textcolor{green}{Der Weg ins Freie. Roman}}, eh wir von hier abfahren, recht, sehr
               weit zu gedeihen. Waldspaziergänge lassen sich täglich neue entdecken; das Essen ist
               gut, mein Zimmer ein von mir längst gesuchtes Ideal; Balkon, Blick ins freie, vom
               Hotel nichts zu sehn (so vorgebaut); die Gesellschaft indifferent, da man sich
               absolut nicht umeinander kümmert.\pend
           \pstart
           Lassen Sie sehr bald von sich hören; dass wir Hoffnung haben, Sie beide unter
                  Reisehi{\geminationm}el, und wenn sich alles gut fügt, für nicht
               gar zu kurze Zeit wiederzusehen, freut uns riesig.\pend
           \pstart
           Von Herzen, mit Grüßen von Höhe zu Höhe, Haus zu Haus\pend
           \pstart
           Ihr{\\[\baselineskip]}\spacefill\mbox{Arthur}\pend
           \leftskip=0em{}\endnumbering\briefempfaengerindex{Beer-Hofmann, Richard@\textsc{Beer-Hofmann, Richard}!zzzSchnitzler, Arthur@\emph{von Arthur Schnitzler}!1907-07-291@{29. 7. 1907}|)be}\mylabel{h}  \normalsize

\doendnotes{C}
\bigskip
\vfill

\clearpage

\footnotesize

\lohead{\textsc{register}}

% Definiere theindex-Environment komplett neu ohne reledmac
\makeatletter
\renewenvironment{theindex}{%
  \section*{\indexname}%
  \setlength{\parindent}{0pt}%
  \setlength{\parskip}{0pt plus 0.3pt}%
  \let\item\@idxitem
}{%
  \clearpage
}
\makeatother

\IfFileExists{\jobname-pw.ind}{\input{\jobname-pw.ind}}{}

\end{document}

      