%% latex-korrekturansicht-vorspann.tex
%% Vorspann für die Korrekturansicht.
%% Lädt die gemeinsame Datei latex-vorspann.tex mit gesetztem Schalter.

\newif\ifkorrekturansicht
\korrekturansichttrue

\input{../tex-inputs/latex-vorspann}


               \section[Arthur Schnitzler an Georg Engländer, 20. 4. 1913]{ Arthur Schnitzler an Georg Engländer,
                    20. 4. 1913}\nopagebreak\mylabel{v}\rehead{ }\normalsize\beginnumbering\briefempfaengerindex{Englaender, Georg@\textsc{Engländer, Georg}!zzzSchnitzler, Arthur@\emph{von Arthur Schnitzler}!1913-04-201@{20. 4. 1913}|(be} \toendnotes[C]{\smallbreak\pagebreak[2]} \Standort{Wien, Österreichische Nationalbibliothek, 228/B8/1-3 LIT MAG.}
\physDesc{Postkarte
\newline{}Handschrift: schwarze Tinte, deutsche Kurrent\newline{}Versand: Stempel: »\nobreak{}\oindex{XVIII., Waehring@\textbf{XVIII., Währing}, \emph{Bezirk (A.BZK)}|pwk}18/1 WIEN 111, \textcolor{gray}{20. 4.} 13, 4\nobreak{}«.  }\pstart{}{\pb}\textcolor{gray}{\textbf{Dr. Arthur Schnitzler}}\pend{}\pstart{}\textcolor{gray}{\textbf{\textcolor{pink}{Wien XVIII. Sternwartestrasse 71}{}\ledrightnote{\textcolor{pink}{Sternwartestraße}}}}\pend{}{\bigskip}\pstart{}Herrn\pend{}\pstart{}\textsc{Georg Engländer}\pend{}\pstart{}\textcolor{pink}{Wien IX}{}\ledrightnote{\textcolor{pink}{IX., Alsergrund}}\pend{}\pstart{}\textcolor{pink}{Nußdorferstr \strikeout{\textcolor{gray}{4}} 10.}{}\ledrightnote{\textcolor{pink}{Nussdorfer Straße}}\pend{}{\bigskip}\pstart
           \raggedleft{}{\pb}20. 4. 913\pend
           \pstart{}Sehr geehrter Herr,\pend\pstart
           in Angelegenheit von \textcolor{blue}{Pet.}{}\ledrightnote{\textcolor{blue}{Peter Altenberg}} den ich heute ſprach
                    (auch \textsc{Primar} Dr. \textcolor{blue}{R.}{}\ledrightnote{\textcolor{blue}{Karl Richter}}{ }ſprach ich) möchte ich gern mit Ihnen reden.
                        We{\geminationn} ich Sie morgen Montag um
                        1 Uhr oder um ½ 7–7 Abend erwarten darf, bedarf es
                    keiner Antwort.\pend
           \pstart Ihr ſehr ergebner \spacefill\mbox{A. S.}\pend{}\endnumbering\briefempfaengerindex{Englaender, Georg@\textsc{Engländer, Georg}!zzzSchnitzler, Arthur@\emph{von Arthur Schnitzler}!1913-04-201@{20. 4. 1913}|)be}\mylabel{h}  \normalsize

\doendnotes{C}
\bigskip
\vfill

\clearpage

\footnotesize

\lohead{\textsc{register}}

% Definiere theindex-Environment komplett neu ohne reledmac
\makeatletter
\renewenvironment{theindex}{%
  \section*{\indexname}%
  \setlength{\parindent}{0pt}%
  \setlength{\parskip}{0pt plus 0.3pt}%
  \let\item\@idxitem
}{%
  \clearpage
}
\makeatother

\IfFileExists{\jobname-pw.ind}{\input{\jobname-pw.ind}}{}

\end{document}

      