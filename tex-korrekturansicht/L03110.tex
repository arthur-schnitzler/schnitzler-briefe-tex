%% latex-korrekturansicht-vorspann.tex
%% Vorspann für die Korrekturansicht.
%% Lädt die gemeinsame Datei latex-vorspann.tex mit gesetztem Schalter.

\newif\ifkorrekturansicht
\korrekturansichttrue

\input{../tex-inputs/latex-vorspann}


\renewcommand{\erwaehntePersonen}{Personen: Richard Beer-Hofmann, Hugo von Hofmannsthal, Julie Laska, Philipp Salzmann}
\renewcommand{\erwaehnteInstitutionen}{Institutionen: Internationale Ausstellung für Musik und Theaterwesen}
\renewcommand{\erwaehnteOrte}{Orte: Café Kremser, Café Schneider, Wien}
\renewcommand{\erwaehnteWerke}{Werke: Muza}
\section[Felix Salten an Arthur Schnitzler, {[}4. 6. 1892{]}]{Felix Salten an Arthur Schnitzler, {[}4. 6. 1892{]}}
\nopagebreak\mylabel{v}
\rehead{ }\normalsize\beginnumbering\briefempfaengerindex{Schnitzler, Arthur@\textsc{Schnitzler, Arthur}!zzzSalten, Felix@\emph{von Felix Salten}!1892-06-041@{{[}4. 6. 1892{]}}|(be}
\toendnotes[C]{\smallbreak\pagebreak[2]}\Standort{CUL, Schnitzler, B 89, A 1.}
\physDesc{Brief, 1 Blatt, 4 Seiten, 803 Zeichen
\newline{}Handschrift: Bleistift, lateinische Kurrent
\newline{}Schnitzler: mit Bleistift datiert: »4/6 92« 
\newline{}Ordnung: mit Bleistift von unbekannter Hand nummeriert: »12« }\toendnotes[C]{\smallbreak}
\pstart
           \noindent{}{\pb}Lieber Freund! Vom \textcolor{blue}{Beer-Hofmann}{}\ledrightnote{\textcolor{blue}{Richard Beer-Hofmann}} keine Nachricht. Er hat mich auch gestern, als er mich zur \textcolor{blue}{Laska}{}\ledrightnote{\textcolor{blue}{Julie Laska}}
               abholen sollte, – ohne abzuschreiben – sitzen lassen. Auch von \textcolor{blue}{Loris}{}\ledrightnote{\textcolor{blue}{Hugo von Hofmannsthal}} keine Zeile. Ich verstehe das nicht.\pend
           
\pstart
           Heute{ }Abend, wenn’s nicht {\pb}fortfährt zu
                  regnen{[},{]} beim \textcolor{pink}{Schneider}{}\ledrightnote{\textcolor{pink}{Café Schneider}}
               in der \textcolor{brown}{Ausstellung}{}\ledrightnote{\textcolor{brown}{Internationale Ausstellung für Musik und Theaterwesen}}.\pend
           
\pstart
           In Anbetracht Ihres gestrigen \label{K_L03110-1v}\edtext{Spielverlustes}{\lemma{\textnormal{\emph{Spielverlustes}}}\Cendnote{\textnormal{vermutlich beim
                  Pokern, vgl. A. S.: \emph{Tagebuch}, 5. 6. 1892}}}\label{K_L03110-1h} fällt es mir schwer, Sie anzupumpen, doch kann ich Ihnen, da {\pb}ich von \textcolor{blue}{Papa}{}\ledrightnote{{$\rightarrow$}\textcolor{blue}{Philipp Salzmann}} vor seiner Abreise am Montag Geld bekomme, vielleicht auch morgen schon das selbe zurückgeben. Wenn es Ihnen also
               möglich ist, würde ich Sie sehr um 3 f. bitten.\pend
           
\pstart
           Was soll ich {\pb}mit \textcolor{blue}{Beer-Hofmann}{}\ledrightnote{\textcolor{blue}{Richard Beer-Hofmann}} anfangen und mit \textcolor{blue}{Loris}{}\ledrightnote{\textcolor{blue}{Hugo von Hofmannsthal}}? Eigentlich ist’s mir ja lieber, wenn nicht gelesen
               wird, da ich jetzt wieder verbu{\geminationm}elt bin, u. \textcolor{green}{Muza}{}\ledrightnote{\textcolor{green}{Muza}} nicht fertig. Also entweder \textcolor{pink}{Schneider}{}\ledrightnote{\textcolor{pink}{Café Schneider}}, oder im \uline{Regen}{ }\uuline{\textcolor{pink}{Kremser}{}\ledrightnote{\textcolor{pink}{Café Kremser}}}, \label{K_L03110-2v}\edtext{heute noch}{\lemma{\textnormal{\emph{heute noch}}}\Cendnote{\textnormal{siehe A. S.: \emph{Tagebuch}, 4. 6. 1892}}}\label{K_L03110-2h}, weil \textcolor{blue}{Richard}{}\ledrightnote{\textcolor{blue}{Richard Beer-Hofmann}} nun kommen könnte.\pend
           \pstart Herzlich \spacefill\mbox{FelixS\textcolor{gray}{a}}\pend{}\endnumbering\briefempfaengerindex{Schnitzler, Arthur@\textsc{Schnitzler, Arthur}!zzzSalten, Felix@\emph{von Felix Salten}!1892-06-041@{{[}4. 6. 1892{]}}|)be}\mylabel{h}  \normalsize

\doendnotes{C}
\bigskip
\vfill

\clearpage

\footnotesize

\lohead{\textsc{register}}

% Definiere theindex-Environment komplett neu ohne reledmac
\makeatletter
\renewenvironment{theindex}{%
  \section*{\indexname}%
  \setlength{\parindent}{0pt}%
  \setlength{\parskip}{0pt plus 0.3pt}%
  \let\item\@idxitem
}{%
  \clearpage
}
\makeatother

\IfFileExists{\jobname-pw.ind}{\input{\jobname-pw.ind}}{}

\end{document}

      