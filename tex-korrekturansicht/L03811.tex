%% latex-korrekturansicht-vorspann.tex
%% Vorspann für die Korrekturansicht.
%% Lädt die gemeinsame Datei latex-vorspann.tex mit gesetztem Schalter.

\newif\ifkorrekturansicht
\korrekturansichttrue

\input{../tex-inputs/latex-vorspann}


\section[Arthur Schnitzler an Stefan Zweig, 7. 11. 1926]{L03811 Arthur Schnitzler an Stefan Zweig, 7. 11. 1926}
\nopagebreak\mylabel{L03811v}
\rehead{ }\normalsize\beginnumbering\briefempfaengerindex{Zweig, Stefan@\textsc{Zweig, Stefan}!zzzSchnitzler, Arthur@\emph{von Arthur Schnitzler}!1926-11-071@{7. 11. 1926}|(be}
\toendnotes[C]{\smallbreak\pagebreak[2]}
\correspDesc{Versand  durch Arthur Schnitzler am 7. 11. 1926 in Wien
\newline{}Erhalt  durch Stefan Zweig im Zeitraum [8. 11. 1926
                  – 12. 11. 1926?] in Salzburg}\toendnotes[C]{\smallbreak}
\Standort{Jerusalem, National Library of Israel, ARC. Ms. Var. 305 1 58 Stefan Zweig Collection.}
\physDesc{Brief, 1 Blatt, 2 Seiten, 825 Zeichen
\newline{}Handschrift: schwarze Tinte, deutsche Kurrent}\toendnotes[C]{\smallbreak}
\pstart
           \raggedleft{}{\pb}\textcolor{pink}{Wien}\oindex{Wien@\textbf{Wien}, \emph{Verwaltungsgebiet}|pw}{}\ledrightnote{\textcolor{pink}{Wien}}, 7. 11. 926\pend
           
\pstart{}lieber Herr Doctor,\pend\vspace{0.5em}
\pstart
           ich dachte Sie doch wenigstens bei der \label{K_L03811-1v}\edtext{\textcolor{violet}{Generalprobe}\eventindex{Burgtheater@\textbf{Burgtheater}!Generalprobe von Volpone oder Der Fuchs, 5.11.1926@Generalprobe von Volpone oder Der Fuchs, 5.11.1926|pwv}{}\ledrightnote{{$\rightarrow$}\emph{\textcolor{violet}{Generalprobe von Volpone oder Der Fuchs, 5.11.1926}}}}{\lemma{\textnormal{\emph{Generalprobe}}}\Cendnote{\textnormal{\textcolor{blue}{Schnitzler} besuchte am 5. 11. 1926 die \textcolor{violet}{Generalprobe von \emph{\textcolor{green}{Ben Jonsons »Volpone«}\pwindex{Zweig, Stefan 28.\,11.\,1881 Wien – 23.\,2.\,1942 Petrópolis@\textsc{Zweig, Stefan} (28.\,11.\,1881 Wien – 23.\,2.\,1942 Petrópolis), \emph{Schriftsteller}!Ben Jonsons »Volpone«.  Eine lieblose Komödie in drei Akten@\strich\emph{Ben Jonsons »Volpone«. Eine lieblose Komödie in drei Akten}|pwk}}}\eventindex{Burgtheater@\textbf{Burgtheater}!Generalprobe von Volpone oder Der Fuchs, 5.11.1926@Generalprobe von Volpone oder Der Fuchs, 5.11.1926|pwkv} im \textcolor{pink}{Burgtheater}\oindex{Wien@\textbf{Wien}!I., Innere Stadt@\textbf{I., Innere Stadt}!Burgtheater@\textbf{Burgtheater}, \emph{Theater}|pwk}.}}}\label{K_L03811-1} zu sehn; nun muss
               ich, da Sie wohl wieder abgereist sind, Ihnen Glückwunsch u Dank nach \textcolor{pink}{Salzburg}\oindex{Salzburg@\textbf{Salzburg}, \emph{Verwaltungsgebiet}|pw}{}\ledrightnote{\textcolor{pink}{Salzburg}} senden. Ich finde Sie haben das \textcolor{green}{Stück}\pwindex{Jonson, Ben 11.\,6.\,1572 London – 6.\,8.\,1637 ebd.@\textsc{Jonson, Ben} (11.\,6.\,1572 London – 6.\,8.\,1637 ebd.), \emph{Schriftsteller}!Volpone, or, the foxe@\strich\emph{Volpone, or, the foxe}|pwv}{}\ledrightnote{{$\rightarrow$}\emph{\textcolor{green}{Volpone, or, the foxe}}} von \textcolor{blue}{Ben Jonson}\pwindex{Jonson, Ben 11.\,6.\,1572 London – 6.\,8.\,1637 ebd.@\textsc{Jonson, Ben} (11.\,6.\,1572 London – 6.\,8.\,1637 ebd.), \emph{Schriftsteller}|pw}{}\ledrightnote{\textcolor{blue}{Ben Jonson}} in jedem Sinne höher gebracht als der
               Original-Autor gethan, – Sie haben es nicht nur für das Theater, sondern auch als
               Dichtung (für meinen Geschmack) erst lebendig gemacht. Ich las (gewissenhafter Weise)
               den \label{K_L03811-2v}\edtext{\textcolor{blue}{\textcolor{green}{Ben Jonson}\pwindex{Jonson, Ben 11.\,6.\,1572 London – 6.\,8.\,1637 ebd.@\textsc{Jonson, Ben} (11.\,6.\,1572 London – 6.\,8.\,1637 ebd.), \emph{Schriftsteller}!Volpone, or, the foxe@\strich\emph{Volpone, or, the foxe}|pwv}{}\ledrightnote{{$\rightarrow$}\emph{\textcolor{green}{Volpone, or, the foxe}}}}\pwindex{Jonson, Ben 11.\,6.\,1572 London – 6.\,8.\,1637 ebd.@\textsc{Jonson, Ben} (11.\,6.\,1572 London – 6.\,8.\,1637 ebd.), \emph{Schriftsteller}|pw}{}\ledrightnote{\textcolor{blue}{Ben Jonson}} (\textcolor{green}{deutsch}\pwindex{Jonson, Ben 11.\,6.\,1572 London – 6.\,8.\,1637 ebd.@\textsc{Jonson, Ben} (11.\,6.\,1572 London – 6.\,8.\,1637 ebd.), \emph{Schriftsteller}!Herr von Fuchs. Ein Lustspiel in drei Aufzügen@\strich\emph{Herr von Fuchs. Ein Lustspiel in drei Aufzügen}|pwv}\pwindex{Jonson, Ben 11.\,6.\,1572 London – 6.\,8.\,1637 ebd.@\textsc{Jonson, Ben} (11.\,6.\,1572 London – 6.\,8.\,1637 ebd.), \emph{Schriftsteller}!Volpone, oder, Der Fuchs@\strich\emph{Volpone, oder, Der Fuchs}|pwv}{}\ledrightnote{{$\rightarrow$}\emph{\textcolor{green}{Herr von Fuchs. Ein Lustspiel in drei Aufzügen}}{\newline}{$\rightarrow$}\emph{\textcolor{green}{Volpone, oder, Der Fuchs}}})}{\lemma{\textnormal{\emph{Ben Jonson (deutsch)}}}\Cendnote{\textnormal{\emph{\textcolor{green}{Volpone, or, the foxe}\pwindex{Jonson, Ben 11.\,6.\,1572 London – 6.\,8.\,1637 ebd.@\textsc{Jonson, Ben} (11.\,6.\,1572 London – 6.\,8.\,1637 ebd.), \emph{Schriftsteller}!Volpone, or, the foxe@\strich\emph{Volpone, or, the foxe}|pwk}} wurde bis dahin zweimal
                  auf deutsch übersetzt. Die erste Übersetzung (\emph{\textcolor{green}{Herr von Fuchs. Ein Lustspiel in drei Aufzügen}\pwindex{Jonson, Ben 11.\,6.\,1572 London – 6.\,8.\,1637 ebd.@\textsc{Jonson, Ben} (11.\,6.\,1572 London – 6.\,8.\,1637 ebd.), \emph{Schriftsteller}!Herr von Fuchs. Ein Lustspiel in drei Aufzügen@\strich\emph{Herr von Fuchs. Ein Lustspiel in drei Aufzügen}|pwk}}, 1793)
                  stammte von \textcolor{blue}{Ludwig Tieck}\pwindex{Tieck, Ludwig 31.\,5.\,1773 Berlin – 28.\,4.\,1853 ebd.@\textsc{Tieck, Ludwig} (31.\,5.\,1773 Berlin – 28.\,4.\,1853 ebd.), \emph{Schriftsteller}|pwk}. Die bis dato
                  neueste (\emph{\textcolor{green}{Volpone, oder, Der Fuchs}\pwindex{Jonson, Ben 11.\,6.\,1572 London – 6.\,8.\,1637 ebd.@\textsc{Jonson, Ben} (11.\,6.\,1572 London – 6.\,8.\,1637 ebd.), \emph{Schriftsteller}!Volpone, oder, Der Fuchs@\strich\emph{Volpone, oder, Der Fuchs}|pwk}},
                     1912) von \textcolor{blue}{Margarethe
                     Mauthner}\pwindex{Mauthner, Margarete 7.\,7.\,1863 Berlin – 24.\,4.\,1947 Johannesburg@\textsc{Mauthner, Margarete} (7.\,7.\,1863 Berlin – 24.\,4.\,1947 Johannesburg), \emph{Übersetzerin, Kunstsammlerin}|pwk}. Welche von beiden Übersetzungen er las, dürfte aus einem Fehler
                  ersichtlich werden, der sich in seiner \emph{\textcolor{green}{Leseliste}\pwindex{Schnitzler, Arthur 15. 5. 1862 Wien – 21. 10. 1931 ebd.@\textsc{Schnitzler, Arthur} (15. 5. 1862 Wien – 21. 10. 1931 ebd.), \emph{Schriftsteller, Mediziner}!Notizen zu Lektüre und Theaterbesuchen (1879-1927)@\strich\emph{Notizen zu Lektüre und Theaterbesuchen (1879-1927)}|pwk}} befindet (A. S.: \emph{Lektüren}, England).
                  Hier findet sich \emph{\textcolor{green}{Volpone}\pwindex{Jonson, Ben 11.\,6.\,1572 London – 6.\,8.\,1637 ebd.@\textsc{Jonson, Ben} (11.\,6.\,1572 London – 6.\,8.\,1637 ebd.), \emph{Schriftsteller}!Volpone, or, the foxe@\strich\emph{Volpone, or, the foxe}|pwk}} unter den Titeln von
                     \textcolor{blue}{Philip Massinger}\pwindex{Massinger, Philip 1.\,1.\,1583 Salisbury – 18.\,3.\,1640 London@\textsc{Massinger, Philip} (1.\,1.\,1583 Salisbury – 18.\,3.\,1640 London), \emph{Schriftsteller}|pwk}. Hier wäre alphabetisch
                  die Übersetzerin \textcolor{blue}{Margarethe Mauthner}\pwindex{Mauthner, Margarete 7.\,7.\,1863 Berlin – 24.\,4.\,1947 Johannesburg@\textsc{Mauthner, Margarete} (7.\,7.\,1863 Berlin – 24.\,4.\,1947 Johannesburg), \emph{Übersetzerin, Kunstsammlerin}|pwk}
                  einzuordnen.}}}\label{K_L03811-2}, eh ich zur \textcolor{violet}{General{\pb}probe}\eventindex{Burgtheater@\textbf{Burgtheater}!Generalprobe von Volpone oder Der Fuchs, 5.11.1926@Generalprobe von Volpone oder Der Fuchs, 5.11.1926|pwv}{}\ledrightnote{{$\rightarrow$}\emph{\textcolor{violet}{Generalprobe von Volpone oder Der Fuchs, 5.11.1926}}} ging; – ich war
               von der Wirkung und dem Geist der Bearbeitung aufs angenehmste überrascht.
               Insbesondre den (– Ihren) dritten \textcolor{green}{Akt}\pwindex{Zweig, Stefan 28.\,11.\,1881 Wien – 23.\,2.\,1942 Petrópolis@\textsc{Zweig, Stefan} (28.\,11.\,1881 Wien – 23.\,2.\,1942 Petrópolis), \emph{Schriftsteller}!Ben Jonsons »Volpone«.  Eine lieblose Komödie in drei Akten@\strich\emph{Ben Jonsons »Volpone«. Eine lieblose Komödie in drei Akten}|pwv}{}\ledrightnote{{$\rightarrow$}\emph{\textcolor{green}{Ben Jonsons »Volpone«.  Eine lieblose Komödie in drei Akten}}} fand ich glänzend.\pend
           
\pstart
           Und nun will ich Ihnen noch herzlich für die lieben Worte danken, die Sie mir in das
                  \textcolor{green}{Buch}\pwindex{Zweig, Stefan 28.\,11.\,1881 Wien – 23.\,2.\,1942 Petrópolis@\textsc{Zweig, Stefan} (28.\,11.\,1881 Wien – 23.\,2.\,1942 Petrópolis), \emph{Schriftsteller}!Ben Jonsons »Volpone«.  Eine lieblose Komödie in drei Akten@\strich\emph{Ben Jonsons »Volpone«. Eine lieblose Komödie in drei Akten}|pwv}{}\ledrightnote{{$\rightarrow$}\emph{\textcolor{green}{Ben Jonsons »Volpone«.  Eine lieblose Komödie in drei Akten}}} geschrieben haben.
               Ich freue mich Ihrer Sympathie und erwidre sie von Herzen.\pend
           
\pstart
           Schönste Grüße, Ihr{\\[\baselineskip]}\spacefill\mbox{ArthurSchnitzler}\pend
           \leftskip=0em{}\selectlanguage{ngerman}\endnumbering\briefempfaengerindex{Zweig, Stefan@\textsc{Zweig, Stefan}!zzzSchnitzler, Arthur@\emph{von Arthur Schnitzler}!1926-11-071@{7. 11. 1926}|)be}\mylabel{L03811h}  \normalsize

\doendnotes{C}
\bigskip
\vfill

\clearpage

\footnotesize

\lohead{\textsc{register}}

% Definiere theindex-Environment komplett neu ohne reledmac
\makeatletter
\renewenvironment{theindex}{%
  \section*{\indexname}%
  \setlength{\parindent}{0pt}%
  \setlength{\parskip}{0pt plus 0.3pt}%
  \let\item\@idxitem
}{%
  \clearpage
}
\makeatother

\IfFileExists{\jobname-pw.ind}{\input{\jobname-pw.ind}}{}

\end{document}

      