%% latex-korrekturansicht-vorspann.tex
%% Vorspann für die Korrekturansicht.
%% Lädt die gemeinsame Datei latex-vorspann.tex mit gesetztem Schalter.

\newif\ifkorrekturansicht
\korrekturansichttrue

\input{../tex-inputs/latex-vorspann}


               \section[Paul Goldmann an Arthur Schnitzler, 24. 6. {[}1895{]}]{ Paul Goldmann an Arthur Schnitzler, 24. 6. {[}1895{]}}\nopagebreak\mylabel{v}\rehead{ }\normalsize\beginnumbering\briefempfaengerindex{Schnitzler, Arthur@\textsc{Schnitzler, Arthur}!zzzGoldmann, Paul@\emph{von Paul Goldmann}!1895-06-242@{24. 6. {[}1895{]}}|(be} \toendnotes[C]{\smallbreak\pagebreak[2]} \Standort{DLA, A:Schnitzler, HS.NZ85.1.3165.}
\physDesc{Brief, 3 Blätter, 12 Seiten
\newline{}Handschrift: schwarze Tinte, deutsche Kurrent
\newline{}Schnitzler: 1) mit Bleistift das Jahr »95« vermerkt 2) mit rotem Buntstift fünf Unterstreichungen}\toendnotes[C]{\smallbreak}\pstart
           \noindent{}{\pb}\textcolor{gray}{\textbf{\textbf{\textcolor{brown}{Frankfurter Zeitung}{}\ledrightnote{\textcolor{brown}{Frankfurter Zeitung}}}}}\pend
           \pstart
           \textcolor{gray}{\textbf{(\textcolor{brown}{\begin{otherlanguage}{french}Gazette de Francfort\end{otherlanguage}}{}\ledrightnote{\textcolor{brown}{Frankfurter Zeitung}}). }}\pend
           \pstart
           \textcolor{gray}{\textbf{\textbf{\begin{otherlanguage}{french}Fondateur M. \textcolor{blue}{L.
                                 Sonnemann}{}\ledrightnote{\textcolor{blue}{Leopold Sonnemann}}\end{otherlanguage}.}}}\hfill \textsc{\textcolor{pink}{Paris}{}\ledrightnote{\textcolor{pink}{Paris}}}, 24. Juni.\pend
           \pstart
           \begin{otherlanguage}{french}\textcolor{gray}{\textbf{\textcolor{green}{Journal}{}\ledrightnote{→\textcolor{green}{Frankfurter Zeitung}} politique,
                        financier,}}\end{otherlanguage}\pend
           \pstart
           \begin{otherlanguage}{french}\textcolor{gray}{\textbf{commercial et littéraire.}}\end{otherlanguage}\pend
           \pstart
           \begin{otherlanguage}{french}\textcolor{gray}{\textbf{\textbf{Paraissant trois fois par jour.}}}\end{otherlanguage}\pend
           \pstart
           \begin{otherlanguage}{french}\textcolor{gray}{\textbf{\textbf{Bureau à \textcolor{pink}{Paris}{}\ledrightnote{\textcolor{pink}{Paris}}:}}}\end{otherlanguage}\pend
           \pstart
           \begin{otherlanguage}{french}\textcolor{gray}{\textbf{\textbf{\textcolor{pink}{24. Rue Feydeau}{}\ledrightnote{\textcolor{pink}{rue Feydeau}}.}}}\end{otherlanguage}\pend
           \pstart\center{}Mein lieber Freund,\pend\pstart
           Eben bekomme ich Deinen lieben Brief. Nur raſch ein paar Zeilen. Mit Deinen Urtheilen
               über die geſandten \label{K_L02737-55v}\edtext{\textcolor{green}{Druckſachen}{}\ledrightnote{→\textcolor{green}{»L’Assaut malicieux«}{\newline}→\textcolor{green}{L’amour s’amuse. Saynète}{\newline}→\textcolor{green}{Les Funérailles}{\newline}→\textcolor{green}{[Salon-Feuilleton]}}}{\lemma{\textnormal{\emph{Druckſachen}}}\Cendnote{\textnormal{siehe Paul Goldmann an Arthur Schnitzler, 19. 5. [1895]}}}\label{K_L02737-55h} – es iſt wirklich zu viel Mühe, daß Du mir lange darüber ſchreibſt – bin ich
               Wort für Wort einverſtanden. Du mußt bedenken, daß {\pb}ich Dir kunterbunt durcheinander ſchicke, was mir intereſſant erſcheint – Einiges
               wegen ſtyliſtiſcher Schönheiten oder origineller Anſchauungen – Anderes wieder nur,
               weil es ein beachtenswerther Abſurditäts-Fall iſt (z. B. \label{K_L02737-1v}\edtext{\textsc{\textcolor{blue}{Rochefort}{}\ledrightnote{\textcolor{blue}{Victor Henri de Rochefort}}}). oder Fall \textsc{\textcolor{green}{\textcolor{blue}{Wilde}{}\ledrightnote{\textcolor{blue}{Oscar Wilde}}}{}\ledrightnote{→\textcolor{green}{»L’Assaut malicieux«}}}}{\lemma{\textnormal{\emph{Rochefort). … Wilde}}}\Cendnote{\textnormal{Der Polemiker \textcolor{blue}{Victor Henri de Rochefort}, der nach seiner politischen
                  Verfolgung sechs Jahre im \textcolor{pink}{London}er Exil lebte,
                  wurde im Februar 1895 amnestiert und kehrte ruhmvoll
                  nach \textcolor{pink}{Paris} zurück, wo er sich unter anderem
                  zur \textcolor{blue}{Dreyfus}-Affäre zu Wort meldete. Es ist
                  unklar, auf welchen Text \textcolor{blue}{Goldmann} hier
                  Bezug nimmt. \textcolor{blue}{Oscar Wilde} wurde wegen
                  »Unzucht« am 25. 5. 1895 zu zwei Jahren Zuchthaus
                  mit schwerer Zwangsarbeit verurteilt, vgl. den gesandten Text von \textcolor{blue}{Paul Adam}: \emph{\textcolor{green}{»L’Assaut malicieux«}}. In: \emph{\textcolor{green}{La Revue blanche}}, Jg. 8, Nr. 47, 15. 5. 1895, 15. 5. 1895, S. 458–462}}}\label{K_L02737-1h} empört mich ſchon lange. Das \textcolor{pink}{engliſch}{}\ledrightnote{→\textcolor{pink}{England}}e Zuchthaus begreife ich {\pb}übrigens
               zur Noth, das ſind dumme heuchleriſche \begin{otherlanguage}{french}\textsc{Bourgeois}\end{otherlanguage}, in \textcolor{pink}{England}{}\ledrightnote{\textcolor{pink}{England}} – damit hat man ſich
               abgefunden. Aber da gibt es dieſen Kerl, den \textsc{\textcolor{blue}{Dr. Nordau}{}\ledrightnote{\textcolor{blue}{Max Nordau}}}, der nach dem Urtheil an \textcolor{pink}{franzöſiſch}{}\ledrightnote{→\textcolor{pink}{Frankreich}}e und \textcolor{pink}{deutsch}{}\ledrightnote{→\textcolor{pink}{Deutschland}}e Blätter Briefe richtet, um zu ſagen: man möge nur in ſeinem Briefe
               nachleſen, wie er das \label{K_L02737-2v}\edtext{Schickſal \textsc{\textcolor{blue}{Wilde}{}\ledrightnote{\textcolor{blue}{Oscar Wilde}}s} voraus{\pb}berechnet}{\lemma{\textnormal{\emph{Schickſal … vorausberechnet}}}\Cendnote{\textnormal{\textcolor{blue}{Max Nordau} beschäftigte sich bereits in
                  seinem zweibändigen \textcolor{green}{Buch}{ }\emph{\textcolor{green}{Entartung}} (1892–1893) mit \textcolor{blue}{Oscar Wilde}, dessen vermeintliche Degeneration er
                  analysierte. Dass er damit den »Fall \textcolor{blue}{Wilde}«
                  hervorgesagt habe, betonte er beispielsweise in einem \textcolor{green}{Interview}: \textcolor{blue}{Paul Roche}: \emph{\textcolor{green}{\textcolor{blue}{Oscar Wilde} judgé par le docteur \textcolor{blue}{Max Nordau}}}. In: \emph{\textcolor{green}{Le Gaulois}}, Jg. 29, Nr. 5443,
                        10. 4. 1895, S. 1–2.}}}\label{K_L02737-2h} – um alſo
               aus dem Schickſal dieſes \textcolor{blue}{Bemitleidenswerthen}{}\ledrightnote{→\textcolor{blue}{Oscar Wilde}} ſich eine Reklame für ſeinen Dekadenz-Schwindel zu
               machen. \uline{Das} macht mir das Blut kochen – da möchte ich
               losprügeln können mit Fäuſten und Stiefel-Abſätzen.\pend
           \pstart
           Über einen \textcolor{pink}{franzöſiſch}{}\ledrightnote{→\textcolor{pink}{Frankreich}}en
                  \label{K_L02737-3v}\edtext{Verleger}{\lemma{\textnormal{\emph{Verleger}}}\Cendnote{\textnormal{Die \emph{\textcolor{green}{Liebelei}} wurde 1896 und 1897 von \textcolor{blue}{Jean Thorel} ins Französische übersetzt, jedoch erst in der
                  Übersetzung von \textcolor{blue}{Suzanne Clauser} im Jahr
                     1933 unter dem Titel \emph{\textcolor{green}{Liebelei (\begin{otherlanguage}{french}amourette\end{otherlanguage})}} gedruckt.}}}\label{K_L02737-3h}
               aus einer Aufführung {\pb}in \textsc{\textcolor{pink}{Paris}{}\ledrightnote{\textcolor{pink}{Paris}}} denke ich ſeit Empfang Deines letzten lieben Briefes nach. Das wird aber ſchwer
               ſein. Die \textcolor{pink}{Pariſ}{}\ledrightnote{\textcolor{pink}{Paris}}er Verleger ſind noch ſchlimmeres
               Geſindel als die \textcolor{pink}{deutſch}{}\ledrightnote{→\textcolor{pink}{Deutschland}}en. Die
                  \textcolor{pink}{deutsch}{}\ledrightnote{→\textcolor{pink}{Deutschland}}en zahlen nur nichts,
               die \textcolor{pink}{franzöſiſch}{}\ledrightnote{→\textcolor{pink}{Frankreich}}en verlangen,
               daß man ihnen {\pb}zahlt. Wärſt Du dazu bereit? Eine \textcolor{green}{Aufführung}{}\ledrightnote{→\textcolor{green}{Liebelei. Schauspiel in drei Akten}} wäre eher möglich –
               aber erſt \uline{nach} einer \textcolor{green}{Aufführung}{}\ledrightnote{→\textcolor{green}{Liebelei. Schauspiel in drei Akten}} in \textcolor{pink}{Berlin}{}\ledrightnote{\textcolor{pink}{Berlin}} oder \textcolor{pink}{Wien}{}\ledrightnote{\textcolor{pink}{Wien}}, nicht
               zugleich. Wir reden noch darüber. Ich hab’ die Sache ſchon lange im Auge und hab’
               auch ſchon einige Schritte gethan.\pend
           \pstart
           {\pb}Das iſt aber immer noch nicht der große Brief – nur
               ein paar raſche Worte, ehe die \strikeout{\textcolor{brown}{K\textcolor{gray}{a}}{}\ledrightnote{→\textcolor{brown}{Französische Abgeordnetenkammer}}}{ }\textcolor{brown}{Kammer}{}\ledrightnote{→\textcolor{brown}{Französische Abgeordnetenkammer}} beginnt. Darum ſchreibe
               ich nicht über allerlei Perſönliches, das ich längſt berühren möchte.\pend
           \pstart
           Es wäre mir eine große Freude, könnt’ ich Dich im Sommer ſehen; aber ich möchte keine
                  {\pb}Störung bringen in Deine Reiſe-Pläne. \strikeout{\textcolor{gray}{×}} Ich muß nach \textsc{\textcolor{pink}{Toelz}{}\ledrightnote{\textcolor{pink}{Bad Tölz}}} gehen u. muß dort vier Wochen bleiben. Das iſt nicht weit von \textsc{\textcolor{pink}{Muenchen}{}\ledrightnote{\textcolor{pink}{München}}}. Wie machen wirs alſo?\pend
           \pstart
           Reiſe glücklich, liebſter Freund! Ich weiß, wie gern Du hinausfährſt, und freue mich
               für Dich. Laß’ die \strikeout{Hypoch\textcolor{gray}{ond}}{ }{\pb}\label{K_L02737-4v}\edtext{Hypochondrien}{\lemma{\textnormal{\emph{Hypochondrien}}}\Cendnote{\textnormal{\textcolor{blue}{Schnitzler} notierte 1895 immer wieder hypochondrische Zustände im \emph{\textcolor{green}{Tagebuch}}, zuletzt am 22. 6. 1895.}}}\label{K_L02737-4h} in \textcolor{pink}{Wien}{}\ledrightnote{\textcolor{pink}{Wien}}! Die Welt iſt ſchön, und Du biſt jung und ein glücklicher
               Menſch, – ja, glaub’ mir, ein glücklicher Menſch.\pend
           \pstart
           Ich höre wohl Deine Unterwegs-Adreſſe.\pend
           \pstart
           \textsc{\textcolor{blue}{Burckhardt}{}\ledrightnote{\textcolor{blue}{Max Eugen Burckhard}}} iſt \label{K_L02737-5v}\edtext{unglaublich}{\lemma{\textnormal{\emph{unglaublich}}}\Cendnote{\textnormal{Am 15. 6. 1895 schrieb \textcolor{blue}{Schnitzler}
                  an \textcolor{blue}{Richard Beer-Hofmann} von dem Gerücht,
                  die \emph{\textcolor{green}{Liebelei}} würde am \emph{\textcolor{brown}{Burgtheater}} nicht mehr aufgeführt werden. \textcolor{blue}{Schnitzler} konfrontierte \textcolor{blue}{Max Burckhard} damit, doch der machte deutlich, dass er es
                  unter allen Umständen aufführen werde, vgl. A. S.: \emph{Tagebuch}, 16. 6. 1895.}}}\label{K_L02737-5h}. Es wäre {\pb}ſogar ſchon komiſch, wenns Dich nicht gerade träfe.
               Aber auch ich bin feſt überzeugt: das \textcolor{green}{Stück}{}\ledrightnote{→\textcolor{green}{Liebelei. Schauspiel in drei Akten}}{ }\uline{wird} aufgeführt.\pend
           \pstart
           Dem \textsc{\textcolor{blue}{Fuchs}{}\ledrightnote{\textcolor{blue}{Isidor Fuchs}}} thatſt \strikeout{\textcolor{gray}{o}h} Du Unrecht. Er iſt kein \textsc{\textcolor{brown}{Concordia}{}\ledrightnote{\textcolor{brown}{Concordia}}}-Literat mehr, ſondern ein lieber, neidloſer, treuer, einfacher Menſch, der alt
               wird und gut wird. Als Mensch {\pb}tauſendmal mehr
               werth, wie \textsc{\textcolor{blue}{Herzl}{}\ledrightnote{\textcolor{blue}{Theodor Herzl}}}.\pend
           \pstart
           \textsc{\textcolor{blue}{Herzl}{}\ledrightnote{\textcolor{blue}{Theodor Herzl}}} ſchreibt einen \label{K_L02737-7v}\edtext{Roman}{\lemma{\textnormal{\emph{Roman}}}\Cendnote{\textnormal{Im Sommer 1895,
                  kurz vor seiner Rückkehr nach \textcolor{pink}{Wien}, spielte \textcolor{blue}{Theodor Herzl} mit der Idee, einen
                  politischen Roman zu schreiben. Vgl. Shlomo Avineri: \textcolor{blue}{Herzl}. \textcolor{blue}{Theodor
                        Herzl} und die Gründung des jüdischen Staates. \textcolor{pink}{Berlin}: eBook Jüdischer Verlag im \emph{\textcolor{brown}{Suhrkamp}} Verlag 2016,
                     S. 181.}}}\label{K_L02737-7h}.\pend
           \pstart
           Was macht \textsc{\textcolor{blue}{\strikeout{Ric}}{}\ledrightnote{→\textcolor{blue}{Richard Beer-Hofmann}}}{ }\textsc{\textcolor{blue}{Richard}{}\ledrightnote{\textcolor{blue}{Richard Beer-Hofmann}}}? Schreibt er was? Und ſehe ich ihn?\pend
           \pstart
           Wie geht die »\textcolor{brown}{Zeit}{}\ledrightnote{\textcolor{brown}{Die Zeit. Wiener Wochenschrift}}«?\pend
           \pstart
           Die \textcolor{green}{Überſetzung}{}\ledrightnote{→\textcolor{green}{Mourir}} von »\textcolor{green}{Sterben}{}\ledrightnote{\textcolor{green}{Sterben. Novelle}}« iſt nicht übel. Dank für die
               Zuſendung.\pend
           \pstart
           {\pb}\textsc{\textcolor{blue}{Bahr}{}\ledrightnote{\textcolor{blue}{Hermann Bahr}}} hat hierher geſchrieben, um die Unterſchriften der \textcolor{pink}{franzöſiſch}{}\ledrightnote{→\textcolor{pink}{Frankreich}}en Schriftſteller-Welt \strikeout{zur} zum Verlangen einer Aufführung eines \label{K_L02737-8v}\edtext{\textsc{\textcolor{blue}{Goldschmidt}{}\ledrightnote{\textcolor{blue}{Adalbert von Goldschmidt}}schen}{ }\textcolor{green}{Muſik-Dramas}{}\ledrightnote{→\textcolor{green}{Gaea. Musikdrama}}}{\lemma{\textnormal{\emph{Goldschmidtschen Muſik-Dramas}}}\Cendnote{\textnormal{Das monumentale Musikdrama \emph{\textcolor{green}{Gäa}} von \textcolor{blue}{Adalbert von Goldschmidt} wurde seit 1892 von \textcolor{blue}{Bahr} für die Aufführung propagiert (\textcolor{blue}{Hermann Bahr}: \emph{\textcolor{green}{Adalbert von Goldschmidt}}. In: \emph{\textcolor{green}{Deutsche Zeitung}}, Jg. 22, Nr. 7.490,
                        4. 11. 1892, Morgen-Ausgabe, S. 6). Erster Anlass war
                  dazu das Erscheinen einer französischen Übersetzung durch \textcolor{blue}{Catulle Mendès} (\emph{\textcolor{green}{Ghea. Poeme dramatique}}. Mis en Français
                     par \textcolor{blue}{Catulle Mendès}.
                     Paris: \emph{\textcolor{brown}{G. Charpentier et E.
                        Fasquelle}}{ }1893.) Eine vollständige Inszenierung würde drei Tage
                  dauern. Auf Initiative von \textcolor{blue}{Bahr} entstanden
                  Komitees in \textcolor{pink}{Wien}, \textcolor{pink}{Berlin} und \textcolor{pink}{Paris}, die
                  die Aufführung bewerkstelligen sollten. \textcolor{blue}{Goldmann} irrte sich jedoch in der Bereitwilligkeit von französischen
                  Kulturgrößen, ihren Namen herzugeben. Im März 1896 erschien eine
                  Petition, die die Aufführung forderte (\emph{\textcolor{green}{»Gäa«}}. In: \emph{\textcolor{green}{Neuen Deutschen Rundschau}}, Jg. 7, H. 3, März 1896,
                     S. 303. Sie war unterzeichnet von: \textcolor{blue}{Julius Bauer}, \textcolor{blue}{Reinhold Begas}, \textcolor{blue}{Alfred von Berger}, \textcolor{blue}{Otto Julius Bierbaum}, \textcolor{blue}{Max Eugen Burckhard}, \textcolor{blue}{Alphonse
                     Daudet}, \textcolor{blue}{Georg Davidsohn}, \textcolor{blue}{Max Halbe}, \textcolor{blue}{Wilhelm Kienzl}, \textcolor{blue}{Wilhelm von
                  Knigge}, \textcolor{blue}{Maurice Kufferath}, \textcolor{blue}{Charles Lamoureux}, \textcolor{blue}{Eduard Lassen}, \textcolor{blue}{Ruggero
                     Leoncavallo}, \textcolor{blue}{Arthur Levysohn}, \textcolor{blue}{Josef Lewinsky}, \textcolor{blue}{Detlev von Liliencron}, \textcolor{blue}{Paul Lindau}, \textcolor{blue}{Rudolf Lothar}, \textcolor{blue}{Maurice Maeterlinck}, \textcolor{blue}{Jules Massenet}, \textcolor{blue}{Catulle
                     Mendès}, \textcolor{blue}{Moritz Moszkowski}, \textcolor{blue}{Felix Mottl}, \textcolor{blue}{Vittorio Pica}, \textcolor{blue}{Emanuel
                     Reicher}, \textcolor{blue}{Marcel Schwob}, \textcolor{blue}{Johann Strauss}, \textcolor{blue}{Hermann Sudermann}, \textcolor{blue}{Viktor Oskar Tilgner}, \textcolor{blue}{Ernest Van
                     Dyck}, \textcolor{blue}{Sidney Whitman}, \textcolor{blue}{Hermann Wolff} und \textcolor{blue}{Émile Zola}.}}}\label{K_L02737-8h} zu erhalten, das er, wenn ich nicht
               irre, als das größte dieſes Jahrhunderts bezeichnet. Man hat ihn ausgelacht. Aber iſt
               das nicht ekelhaft?\pend
           \pstart
           Grüß’ Dich Gott, mein lieber Freund, und ſchreib’ mir bald.\pend
           \pstart
           Dein treuer{\\[\baselineskip]}\spacefill\mbox{Paul Goldmann.}\pend
           \leftskip=0em{}\endnumbering\briefempfaengerindex{Schnitzler, Arthur@\textsc{Schnitzler, Arthur}!zzzGoldmann, Paul@\emph{von Paul Goldmann}!1895-06-242@{24. 6. {[}1895{]}}|)be}\mylabel{h}\begin{anhang}\end{anhang}\normalsize

\doendnotes{C}
\bigskip
\vfill

\clearpage

\footnotesize

\lohead{\textsc{register}}

% Definiere theindex-Environment komplett neu ohne reledmac
\makeatletter
\renewenvironment{theindex}{%
  \section*{\indexname}%
  \setlength{\parindent}{0pt}%
  \setlength{\parskip}{0pt plus 0.3pt}%
  \let\item\@idxitem
}{%
  \clearpage
}
\makeatother

\IfFileExists{\jobname-pw.ind}{\input{\jobname-pw.ind}}{}

\end{document}

      