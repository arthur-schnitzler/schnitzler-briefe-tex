%% latex-korrekturansicht-vorspann.tex
%% Vorspann für die Korrekturansicht.
%% Lädt die gemeinsame Datei latex-vorspann.tex mit gesetztem Schalter.

\newif\ifkorrekturansicht
\korrekturansichttrue

\input{../tex-inputs/latex-vorspann}


               \section[Hugo von Hofmannsthal an Arthur Schnitzler, {[}28. 10. 1899{]}]{ Hugo von Hofmannsthal an Arthur Schnitzler, {[}28. 10. 1899{]}}\nopagebreak\mylabel{v}\rehead{ }\normalsize\beginnumbering\briefempfaengerindex{Schnitzler, Arthur@\textsc{Schnitzler, Arthur}!zzzHofmannsthal, Hugo von@\emph{von Hugo von Hofmannsthal}!1899-10-281@{{[}28. 10. 1899{]}}|(be} \toendnotes[C]{\smallbreak\pagebreak[2]} \Standort{CUL, Schnitzler, B 43.}
\physDesc{Brief, 1 Blatt, 1 Seite
\newline{}Handschrift: schwarze Tinte, deutsche Kurrent
\newline{}Schnitzler: mit Bleistift datiert: »28/X 97« \newline{}Ordnung: 1) mit Bleistift von unbekannter Hand nummeriert: »\strikeout{93}« 2) mit Bleistift von unbekannter Hand nummeriert: »\strikeout{87}«3) mit Bleistift von unbekannter Hand nummeriert: »98«}\buchAbdrucke{\weitereDrucke{Hugo von Hofmannsthal, Arthur Schnitzler: \emph{Briefwechsel}. Hg. Therese Nickl und Heinrich Schnitzler. Frankfurt am Main: \emph{S. Fischer} 1964, S. 133.} }\toendnotes[C]{\smallbreak}\pstart
           \raggedleft{}{\pb}Samstag\pend
           \pstart
           Falls Sie etwa für ſich in irgendwelcher Form für \label{K_L00993_1v}\edtext{Montag}{\lemma{\textnormal{\emph{Montag}}}\Cendnote{\textnormal{\textcolor{blue}{Schnitzler} besuchte an diesem Tag – dem
                        30. 10. 1899 – die Vorstellung im \textcolor{pink}{Burgtheater}.}}}\label{K_L00993_1h} (\textcolor{green}{Eſther}{}\ledrightnote{\textcolor{green}{Esther}}) Sitz beſtellen, bitte auch einen für mich.\pend
           \pstart
           Also \label{K_L00993_2v}\edtext{morgen}{\lemma{\textnormal{\emph{morgen}}}\Cendnote{\textnormal{Am 29. 10. 1899 las \textcolor{blue}{Hofmannsthal} bei \textcolor{blue}{Beer-Hofmann}{ }\emph{\textcolor{green}{Das Bergwerk zu Falun}} vor.}}}\label{K_L00993_2h}{ }\textcolor{blue}{Richard}{}\ledrightnote{\textcolor{blue}{Richard Beer-Hofmann}}, 6\textsuperscript{h}.\pend
           \pstart
           Ihr{\\[\baselineskip]}\spacefill\mbox{Hugo}\pend
           \leftskip=0em{}\endnumbering\briefempfaengerindex{Schnitzler, Arthur@\textsc{Schnitzler, Arthur}!zzzHofmannsthal, Hugo von@\emph{von Hugo von Hofmannsthal}!1899-10-281@{{[}28. 10. 1899{]}}|)be}\mylabel{h}  \normalsize

\doendnotes{C}
\bigskip
\vfill

\clearpage

\footnotesize

\lohead{\textsc{register}}

% Definiere theindex-Environment komplett neu ohne reledmac
\makeatletter
\renewenvironment{theindex}{%
  \section*{\indexname}%
  \setlength{\parindent}{0pt}%
  \setlength{\parskip}{0pt plus 0.3pt}%
  \let\item\@idxitem
}{%
  \clearpage
}
\makeatother

\IfFileExists{\jobname-pw.ind}{\input{\jobname-pw.ind}}{}

\end{document}

      