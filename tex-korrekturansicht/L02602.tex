%% latex-korrekturansicht-vorspann.tex
%% Vorspann für die Korrekturansicht.
%% Lädt die gemeinsame Datei latex-vorspann.tex mit gesetztem Schalter.

\newif\ifkorrekturansicht
\korrekturansichttrue

\input{../tex-inputs/latex-vorspann}


               \section[Paul Goldmann an Arthur Schnitzler, {[}Mitte? August 1894{]}]{ Paul Goldmann an Arthur Schnitzler, {[}Mitte? August 1894{]}}\nopagebreak\mylabel{v}\rehead{ }\normalsize\beginnumbering\briefempfaengerindex{Schnitzler, Arthur@\textsc{Schnitzler, Arthur}!zzzGoldmann, Paul@\emph{von Paul Goldmann}!1894-08-151@{{[}Mitte? August 1894{]}}|(be} \toendnotes[C]{\smallbreak\pagebreak[2]} \Standort{DLA, A:Schnitzler, HS.NZ85.1.3164.}
\physDesc{Telegramm
\newline{}maschinell
\newline{}Schnitzler: mit Bleistift Vermerk »\textsc{August 94}« \newline{}Ordnung: beschnitten }\toendnotes[C]{\smallbreak}\pstart
           \noindent{}{\pb}erbitte drahtantwort \textcolor{pink}{genf}{}\ledrightnote{\textcolor{pink}{Genf}} poste \label{T_L02602-1v}\edtext{restant{[}e{]}}{\lemma{\textnormal{\emph{restante}}}\Cendnote{\textnormal{Übermittlungsfehler:
                     »restantf«}}}\label{T_L02602-1h} ob ich euch \label{K_L02602-1v}\edtext{\textcolor{pink}{ischl}{}\ledrightnote{\textcolor{pink}{Bad Ischl}} treffe}{\lemma{\textnormal{\emph{ischl treffe}}}\Cendnote{\textnormal{Zwischen dem letzten Schreiben \textcolor{blue}{Goldmann}s vom 9. 8. [1894] und der Ankunft in \textcolor{pink}{Ischl} am 23. 8. 1894 sandte er drei Telegramme, die um die
                  Übermittlungszeilen beschnitten sind. Dieses ist das erste und dürfte einige Tage
                  nach dem letzten Brief verfasst sein, wofür auch der implizite Hinweis auf die
                  Abreise aus \textcolor{pink}{Paris} spricht.}}}\label{K_L02602-1h}\pend
           \pstart \spacefill\mbox{= goldmann .+}\pend{}\endnumbering\briefempfaengerindex{Schnitzler, Arthur@\textsc{Schnitzler, Arthur}!zzzGoldmann, Paul@\emph{von Paul Goldmann}!1894-08-151@{{[}Mitte? August 1894{]}}|)be}\mylabel{h}  \normalsize

\doendnotes{C}
\bigskip
\vfill

\clearpage

\footnotesize

\lohead{\textsc{register}}

% Definiere theindex-Environment komplett neu ohne reledmac
\makeatletter
\renewenvironment{theindex}{%
  \section*{\indexname}%
  \setlength{\parindent}{0pt}%
  \setlength{\parskip}{0pt plus 0.3pt}%
  \let\item\@idxitem
}{%
  \clearpage
}
\makeatother

\IfFileExists{\jobname-pw.ind}{\input{\jobname-pw.ind}}{}

\end{document}

      