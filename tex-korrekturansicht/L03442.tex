%% latex-korrekturansicht-vorspann.tex
%% Vorspann für die Korrekturansicht.
%% Lädt die gemeinsame Datei latex-vorspann.tex mit gesetztem Schalter.

\newif\ifkorrekturansicht
\korrekturansichttrue

\input{../tex-inputs/latex-vorspann}


\renewcommand{\erwaehntePersonen}{Personen: Charlotte Bondy, Fritz Freund, Theodore Rottenberg, Olga Schnitzler, Heinrich Schnitzler, Moritz von Schwind, Giovanni Battista Tiepolo}
\renewcommand{\erwaehnteInstitutionen}{Institutionen: Wiener Verlag}
\renewcommand{\erwaehnteOrte}{Orte: Bamberg, Domberg (Bamberg), Domplatz (Bamberg), Eisenach, Hotel Marienbad, München, Opletalova, Prag, Regensburg, Regensburger Dom, Sizilien, Wartburg, Weimar, Wien, Würzburg}
\renewcommand{\erwaehnteWerke}{Werke: Deckenfresko im Treppenhaus der Würzburger Residenz, Reigen. Zehn Dialoge, Schwindsche Wartburgfresken}
\section[ Paul Goldmann an Arthur Schnitzler, 8. 4. {[}1904{]}]{Paul Goldmann an Arthur Schnitzler, 8. 4. {[}1904{]}}
\nopagebreak\mylabel{v}
\rehead{ }\normalsize\beginnumbering\briefempfaengerindex{Schnitzler, Arthur@\textsc{Schnitzler, Arthur}!zzzGoldmann, Paul@\emph{von Paul Goldmann}!1904-04-081@{8. 4. {[}1904{]}}|(be}
\toendnotes[C]{\smallbreak\pagebreak[2]}\Standort{DLA, A:Schnitzler, HS.NZ85.1.3174.}
\physDesc{Brief, 1 Blatt, 4 Seiten
\newline{}Handschrift: schwarze Tinte, deutsche Kurrent
\newline{}Schnitzler: 1) mit Bleistift das Jahr »{[}1{]}904« vermerkt  2) mit rotem Buntstift eine Unterstreichung}\toendnotes[C]{\smallbreak}
\pstart
           \raggedleft{}{\pb}\textcolor{pink}{München}{}\ledrightnote{\textcolor{pink}{München}}{ }8. April.\pend
           
\pstart{}Mein lieber Freund,\pend
\pstart
           Dein lieber Brief (mit dem ich mich ſehr gefreut habe) und Deine Karte wurden mir
               hierher nachgeſandt (\label{K_L03442-12v}\edtext{Frau \textsc{\textcolor{blue}{Bondy}{}\ledrightnote{\textcolor{blue}{Charlotte Bondy}}}: \textsc{\textcolor{pink}{Prag}{}\ledrightnote{\textcolor{pink}{Prag}}}, \textcolor{pink}{\textsc{Mariengaſse} 45}{}\ledrightnote{\textcolor{pink}{Opletalova}}}{\lemma{\textnormal{\emph{Frau … Mariengaſse 45}}}\Cendnote{\textnormal{Bezug unklar}}}\label{K_L03442-12h}). Ich habe eine
               kleine Erholungsreiſe gemacht, bei der ich mich freilich wenig erholt habe. Ins
               Gebirge konnte ich nicht wegen des ſchlechten Wetters. So bin ich in Etappen nach \textcolor{pink}{München}{}\ledrightnote{\textcolor{pink}{München}} gefahren: \textcolor{pink}{Weimar}{}\ledrightnote{\textcolor{pink}{Weimar}}, \textcolor{pink}{Eiſenach}{}\ledrightnote{\textcolor{pink}{Eisenach}} (mit der reizend
               gelegenen und wegen der \textcolor{green}{Fresken}{}\ledrightnote{{$\rightarrow$}\textcolor{green}{Schwindsche Wartburgfresken}}{ }\textsc{\textcolor{blue}{Schwindt}{}\ledrightnote{\textcolor{blue}{Moritz von Schwind}}s} überaus ſehenswerthen \textcolor{pink}{Wartburg}{}\ledrightnote{\textcolor{pink}{Wartburg}}), \textcolor{pink}{Würzburg}{}\ledrightnote{\textcolor{pink}{Würzburg}} (herrliche \textcolor{green}{Fresken}{}\ledrightnote{{$\rightarrow$}\textcolor{green}{Deckenfresko im Treppenhaus der Würzburger Residenz}} von \textsc{\textcolor{blue}{Tiepolo}{}\ledrightnote{\textcolor{blue}{Giovanni Battista Tiepolo}}}), \textcolor{pink}{Bamberg}{}\ledrightnote{\textcolor{pink}{Bamberg}} (ein großartiger \textcolor{pink}{Domplatz}{}\ledrightnote{\textcolor{pink}{Domplatz (Bamberg)}} auf einem \textcolor{pink}{Berge}{}\ledrightnote{{$\rightarrow$}\textcolor{pink}{Domberg (Bamberg)}}), \textcolor{pink}{Regensburg}{}\ledrightnote{\textcolor{pink}{Regensburg}}{ }{\pb}ſchöner gothiſcher \textcolor{pink}{Dom}{}\ledrightnote{\textcolor{pink}{Regensburger Dom}}) und \textcolor{pink}{München}{}\ledrightnote{\textcolor{pink}{München}}. Ich
               wohne wieder im \textsc{\textcolor{pink}{Hotel Marienbad}{}\ledrightnote{\textcolor{pink}{Hotel Marienbad}}} und gedenke \strikeout{D\textcolor{gray}{ein}} der ſchönen Tage, die wir \label{K_L03442-1v}\edtext{vor
                  Jahren}{\lemma{\textnormal{\emph{vor
                  Jahren}}}\Cendnote{\textnormal{nicht rekonstruierbar}}}\label{K_L03442-1h}
               hier verbracht haben.\pend
           
\pstart
           Daß das \label{K_L03442-2v}\edtext{Verbot des »\textcolor{green}{Reigen}{}\ledrightnote{\textcolor{green}{Reigen. Zehn Dialoge}}«}{\lemma{\textnormal{\emph{Verbot des »Reigen«}}}\Cendnote{\textnormal{siehe Paul Goldmann an Arthur Schnitzler, 19. 3. [1904]}}}\label{K_L03442-2h} Dir keinen Schaden gethan hat, freut mich ſehr. Auch haſt Du ganz Recht, daß
               Du vorläufig in der Öffentlichkeit nichts darüber verlauten laſſen willſt. Wenn es
               zum Prozeß kommen ſollte, wird dazu immer noch Zeit ſein, – falls es überhaupt
               nothwendig werden ſollte. Immerhin iſt es wichtig, daß in dem Prozeß Dein \textcolor{brown}{\textcolor{blue}{Verleger}{}\ledrightnote{\textcolor{blue}{Fritz Freund}}}{}\ledrightnote{{$\rightarrow$}\textcolor{brown}{Wiener Verlag}} durch einen tüchtigen Anwalt vertreten wird, der\strikeout{i\textcolor{gray}{m Sta}} fähig {\pb}iſt, die Angelegenheit von einem
               höheren Standpunkte aus zu erörtern.\pend
           
\pstart
           \textcolor{blue}{Eure}{}\ledrightnote{{$\rightarrow$}\textcolor{blue}{Olga Schnitzler}}{ }\label{K_L03442-17v}\edtext{Frühjahrsreiſe}{\lemma{\textnormal{\emph{Frühjahrsreiſe}}}\Cendnote{\textnormal{siehe Paul Goldmann an Arthur Schnitzler, 14. 3. [1904]}}}\label{K_L03442-17h} nach \textcolor{pink}{Sizilien}{}\ledrightnote{\textcolor{pink}{Sizilien}} wird ſehr ſchön werden.
               Durch den Aufſchub iſt Euch das ſchlechte Wetter erſpart geblieben. Ich wünſche Euch
               den ſchönſten Sonnenſchein{[}.{]} Nur follteſt Du länger als einen
               Monat bleiben. In einer Woche iſt die Reiſe vielleicht etwas anſtrengend.\pend
           
\pstart
           Meiner \textcolor{blue}{Freundin}{}\ledrightnote{{$\rightarrow$}\textcolor{blue}{Theodore Rottenberg}} geht es,
               nachdem die drohende \label{K_L03442-13v}\edtext{Gefahr}{\lemma{\textnormal{\emph{Gefahr}}}\Cendnote{\textnormal{siehe Paul Goldmann an Arthur Schnitzler, 14. 3. [1904]}}}\label{K_L03442-13h}{ }\strikeout{abg} glücklich abgewendet iſt, recht gut. Sie hat mir
                  \strikeout{mehrsm} mehrmals Grüße für Dich aufgetragen. Wie
               ſich unſere Zukunft geſtalten wird, weiß Gott allein. Wenn {\pb}\strikeout{ſ\textcolor{gray}{ie}} ich \textcolor{blue}{ſie}{}\ledrightnote{{$\rightarrow$}\textcolor{blue}{Theodore Rottenberg}} nicht habe,
               wie jetzt, ſo ſehne ich mich nach ihr; war ich aber vier Wochen mit ihr zuſammen, ſo
               habe ich, wenn ſie wegfährt, ein Gefühl, \strikeout{a\textcolor{gray}{bſ}} der Freiheit. Es ſcheint, daß man von einer Frau niemals gerade ſo viel hat,
               als man \substVorne{}\textsuperscript{b\textcolor{gray}{raucht},}{\allowbreak}\substDazwischen{}braucht,\substHinten{} ſondern immer nur entweder zu wenig oder zu viel.\pend
           
\pstart
           Ich leide ſeit einer Woche an Kopfſchmerzen, die ich mir durch Zuviel-Sehen und
               Zuviel-Herumreiſen zugezogen habe. Nimm’ Dir ein warnendes Beiſpiel für \textcolor{pink}{Sizilien}{}\ledrightnote{\textcolor{pink}{Sizilien}}!\pend
           
\pstart
           Schreib’ mir bald wieder und ſei, ſammt \textcolor{blue}{Frau}{}\ledrightnote{{$\rightarrow$}\textcolor{blue}{Olga Schnitzler}} und \textcolor{blue}{Kind}{}\ledrightnote{{$\rightarrow$}\textcolor{blue}{Heinrich Schnitzler}} (was macht \textcolor{blue}{Heinrich}{}\ledrightnote{\textcolor{blue}{Heinrich Schnitzler}}?) herzlichſt
               gegrüßt von Deinem getreuen {\\[\baselineskip]}\spacefill\mbox{Paul Goldmann}\pend
           \leftskip=0em{}\endnumbering\briefempfaengerindex{Schnitzler, Arthur@\textsc{Schnitzler, Arthur}!zzzGoldmann, Paul@\emph{von Paul Goldmann}!1904-04-081@{8. 4. {[}1904{]}}|)be}\mylabel{h}
\begin{anhang}
\end{anhang}\normalsize

\doendnotes{C}
\bigskip
\vfill

\clearpage

\footnotesize

\lohead{\textsc{register}}

% Definiere theindex-Environment komplett neu ohne reledmac
\makeatletter
\renewenvironment{theindex}{%
  \section*{\indexname}%
  \setlength{\parindent}{0pt}%
  \setlength{\parskip}{0pt plus 0.3pt}%
  \let\item\@idxitem
}{%
  \clearpage
}
\makeatother

\IfFileExists{\jobname-pw.ind}{\input{\jobname-pw.ind}}{}

\end{document}

      