%% latex-korrekturansicht-vorspann.tex
%% Vorspann für die Korrekturansicht.
%% Lädt die gemeinsame Datei latex-vorspann.tex mit gesetztem Schalter.

\newif\ifkorrekturansicht
\korrekturansichttrue

\input{../tex-inputs/latex-vorspann}


\renewcommand{\erwaehntePersonen}{Personen: Richard Beer-Hofmann, Hugo von Hofmannsthal, Richard Mandl, Felix Salten}
\renewcommand{\erwaehnteOrte}{Orte: Paris, Wien}
\renewcommand{\erwaehnteWerke}{Werke: ?? [Romanprojekt]}
\section[Arthur Schnitzler an Felix Salten, {[}25. 9. 1893?{]}]{Arthur Schnitzler an Felix Salten, {[}25. 9. 1893?{]}}
\nopagebreak\mylabel{v}
\rehead{ }\normalsize\beginnumbering\briefempfaengerindex{Salten, Felix@\textsc{Salten, Felix}!zzzSchnitzler, Arthur@\emph{von Arthur Schnitzler}!1893-09-251@{{[}25. 9. 1893?{]}}|(be}
\toendnotes[C]{\smallbreak\pagebreak[2]}\Standort{Wienbibliothek im Rathaus, ZPH 1681, 2.1.516.}
\physDesc{Brief, 1 Blatt, 3 Seiten, 422 Zeichen (Briefpapier mit Trauerrand)
\newline{}Handschrift: Bleistift, deutsche Kurrent
\newline{}Ordnung: mit Bleistift von unbekannter Hand die erste und dritte Seite paginiert:
                                    »13«–»14« }\toendnotes[C]{\smallbreak}
\pstart{}{\pb}Hochverehrter Herr von Salten!\pend
\pstart
           \label{K_L02962-1v}\edtext{Morgen Dinſtag}{\lemma{\textnormal{\emph{Morgen Dinſtag}}}\Cendnote{\textnormal{siehe A. S.: \emph{Tagebuch}, 26. 9. 1893}}}\label{K_L02962-1h}{ }Nachmittag 4 Uhr ko{\geminationm}en \textsc{\textcolor{blue}{Loris}{}\ledrightnote{\textcolor{blue}{Hugo von Hofmannsthal}}} u. \textcolor{blue}{Richard}{}\ledrightnote{\textcolor{blue}{Richard Beer-Hofmann}} zu mir, und außerdem Herr \textsc{\textcolor{blue}{Richard Mandl}{}\ledrightnote{\textcolor{blue}{Richard Mandl}}}, (Componiſt, \textcolor{pink}{Paris}{}\ledrightnote{\textcolor{pink}{Paris}}) {\pb}der uns auf dem Piano artige Dinge zu ſpielen
               gedenkt, welches ich Ihnen mittheile, um Sie zu bewegen, mir gleichfalls die Ehre
               Ihres Beſuches zu ſchenken, der mir denn {\pb}ſicherlich höflich willko\textcolor{gray}{{\geminationm}}en ſein wird.\pend
           
\pstart
           Leben Sie wohl und ſagen mir bald gute Nachricht von Ihrem \label{K_L02962-2v}\edtext{\textcolor{green}{Roman}{}\ledrightnote{{$\rightarrow$}\textcolor{green}{?? [Romanprojekt]}}}{\lemma{\textnormal{\emph{Roman}}}\Cendnote{\textnormal{nicht ermittelt}}}\label{K_L02962-2h}.\pend
           \pstart Ihr \spacefill\mbox{ArthS}\pend{}
\pstart
           Montag.\pend
           \endnumbering\briefempfaengerindex{Salten, Felix@\textsc{Salten, Felix}!zzzSchnitzler, Arthur@\emph{von Arthur Schnitzler}!1893-09-251@{{[}25. 9. 1893?{]}}|)be}\mylabel{h}  \normalsize

\doendnotes{C}
\bigskip
\vfill

\clearpage

\footnotesize

\lohead{\textsc{register}}

% Definiere theindex-Environment komplett neu ohne reledmac
\makeatletter
\renewenvironment{theindex}{%
  \section*{\indexname}%
  \setlength{\parindent}{0pt}%
  \setlength{\parskip}{0pt plus 0.3pt}%
  \let\item\@idxitem
}{%
  \clearpage
}
\makeatother

\IfFileExists{\jobname-pw.ind}{\input{\jobname-pw.ind}}{}

\end{document}

      