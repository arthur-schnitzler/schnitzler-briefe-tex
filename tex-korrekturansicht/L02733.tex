%% latex-korrekturansicht-vorspann.tex
%% Vorspann für die Korrekturansicht.
%% Lädt die gemeinsame Datei latex-vorspann.tex mit gesetztem Schalter.

\newif\ifkorrekturansicht
\korrekturansichttrue

\input{../tex-inputs/latex-vorspann}


               \section[Paul Goldmann an Arthur Schnitzler, 3. 4. {[}1895{]}]{ Paul Goldmann an Arthur Schnitzler, 3. 4. {[}1895{]}}\nopagebreak\mylabel{v}\rehead{ }\normalsize\beginnumbering\briefempfaengerindex{Schnitzler, Arthur@\textsc{Schnitzler, Arthur}!zzzGoldmann, Paul@\emph{von Paul Goldmann}!1895-04-031@{3. 4. {[}1895{]}}|(be} \toendnotes[C]{\smallbreak\pagebreak[2]} \Standort{DLA, A:Schnitzler, HS.NZ85.1.3165.}
\physDesc{Brief, 1 Blatt, 2 Seiten
\newline{}Handschrift: schwarze Tinte, deutsche Kurrent
\newline{}Schnitzler: 1) mit schwarzer Tinte das Jahr »95« vermerkt 2) mit rotem Buntstift eine Unterstreichung}\toendnotes[C]{\smallbreak}\pstart
           \noindent{}{\pb}\textcolor{gray}{\textbf{\textbf{\textcolor{brown}{Frankfurter Zeitung}{}\ledrightnote{\textcolor{brown}{Frankfurter Zeitung}}}}}\pend
           \pstart
           \textcolor{gray}{\textbf{(\textcolor{brown}{\begin{otherlanguage}{french}Gazette de Francfort\end{otherlanguage}}{}\ledrightnote{\textcolor{brown}{Frankfurter Zeitung}}). }}\pend
           \pstart
           \textcolor{gray}{\textbf{\textbf{\begin{otherlanguage}{french}Fondateur M. \textcolor{blue}{L. Sonnemann}{}\ledrightnote{\textcolor{blue}{Leopold Sonnemann}}\end{otherlanguage}.}}}\pend
           \pstart
           \begin{otherlanguage}{french}\textcolor{gray}{\textbf{\textcolor{green}{Journal}{}\ledrightnote{\textcolor{green}{Frankfurter Zeitung}} politique, financier,}}\end{otherlanguage}\pend
           \pstart
           \begin{otherlanguage}{french}\textcolor{gray}{\textbf{commercial et littéraire.}}\end{otherlanguage}\pend
           \pstart
           \begin{otherlanguage}{french}\textcolor{gray}{\textbf{\textbf{Paraissant trois fois par jour.}}}\end{otherlanguage}\hfill \textsc{\textcolor{pink}{Paris}{}\ledrightnote{\textcolor{pink}{Paris}}}, 3. April.\pend
           \pstart
           \begin{otherlanguage}{french}\textcolor{gray}{\textbf{\textbf{Bureau à \textcolor{pink}{Paris}{}\ledrightnote{\textcolor{pink}{Paris}}:}}}\end{otherlanguage}\pend
           \pstart
           \begin{otherlanguage}{french}\textcolor{gray}{\textbf{\textbf{\textcolor{pink}{24. Rue Feydeau}{}\ledrightnote{\textcolor{pink}{rue Feydeau}}.}}}\end{otherlanguage}\pend
           \pstart\center{}Mein lieber Freund,\pend\pstart
           In Eile: Dieſen \label{K_L02733-1v}\edtext{\textcolor{blue}{Mann}{}\ledrightnote{→\textcolor{blue}{Gaspard Vallette}}}{\lemma{\textnormal{\emph{Mann}}}\Cendnote{\textnormal{Womöglich handelte es sich um \textcolor{blue}{Gaspard Vallette}, der \emph{\textcolor{green}{Sterben}} ins Französische übersetzte. Nur wenige Tage vor
                  der Entstehung dieses Briefs, am 31. 3. 1895, notierte sich \textcolor{blue}{Schnitzler} die Anfrage zur \textcolor{green}{Übersetzung} im \emph{\textcolor{green}{Tagebuch}}.}}}\label{K_L02733-1h} in \textsc{\textcolor{pink}{Cannes}{}\ledrightnote{\textcolor{pink}{Cannes}}} kenne ich nicht, und Niemand kennt ihn, den ich hier befragt. Die Adreſſe
               deutet auf einen \label{K_L02733-2v}\edtext{\textsc{\begin{otherlanguage}{french}homme cossu\end{otherlanguage}}}{\lemma{\textnormal{\emph{homme cossu}}}\Cendnote{\textnormal{französisch: wohlhabender Mann}}}\label{K_L02733-2h}
               hin. Ob er Franzöſiſch kann? Denn es ſcheint kein \textcolor{pink}{Franzoſ}{}\ledrightnote{→\textcolor{pink}{Frankreich}}e zu ſein. Immerhin gib’ ihm die Autoriſation. Eine
               franzöſiſche \textcolor{green}{Überſetzung}{}\ledrightnote{→\textcolor{green}{Mourir}}, die
               Du noch dazu nicht zu bezahlen brauchſt, iſt beſſer als gar keine. Mache aber aus,
               daß er die \textcolor{green}{Sache}{}\ledrightnote{→\textcolor{green}{Mourir}} nicht
               veröffentlicht ohne daß Du die \textcolor{green}{Überſetzung}{}\ledrightnote{→\textcolor{green}{Mourir}}{\pb}geſehen und Deine Zuſtimmung gegeben haſt. Du wirſt
               ſie dann mir zuſenden, und wir werden ſehen.\pend
           \pstart
           Die Idee, daß \textsc{\textcolor{brown}{\textcolor{blue}{Langen}{}\ledrightnote{\textcolor{blue}{Albert Langen}}}{}\ledrightnote{\textcolor{brown}{Albert Langen}}} Deine \label{K_L02733-3v}\edtext{\textcolor{green}{Novelle}{}\ledrightnote{→\textcolor{green}{Sterben. Novelle von Arthur Schnitzler}}}{\lemma{\textnormal{\emph{Novelle}}}\Cendnote{\textnormal{\emph{\textcolor{green}{Sterben}} in französischer Übersetzung}}}\label{K_L02733-3h}
               verlegen ſoll, iſt nicht übel. Laß’ mich nur machen. Vielleicht kommt übrigens der
                  \textcolor{blue}{Lausbube}{}\ledrightnote{→\textcolor{blue}{Albert Langen}} nach \textsc{\textcolor{pink}{Wien}{}\ledrightnote{\textcolor{pink}{Wien}}}. \strikeout{D} Dann will ich Dir vorher Inſtruktionen
               geben.\pend
           \pstart
           Grüß Dich Gott! {\\[\baselineskip]}Dein {\\[\baselineskip]}\spacefill\mbox{Paul Goldmann}\pend
           \leftskip=0em{}\endnumbering\briefempfaengerindex{Schnitzler, Arthur@\textsc{Schnitzler, Arthur}!zzzGoldmann, Paul@\emph{von Paul Goldmann}!1895-04-031@{3. 4. {[}1895{]}}|)be}\mylabel{h}\begin{anhang}\end{anhang}\normalsize

\doendnotes{C}
\bigskip
\vfill

\clearpage

\footnotesize

\lohead{\textsc{register}}

% Definiere theindex-Environment komplett neu ohne reledmac
\makeatletter
\renewenvironment{theindex}{%
  \section*{\indexname}%
  \setlength{\parindent}{0pt}%
  \setlength{\parskip}{0pt plus 0.3pt}%
  \let\item\@idxitem
}{%
  \clearpage
}
\makeatother

\IfFileExists{\jobname-pw.ind}{\input{\jobname-pw.ind}}{}

\end{document}

      