%% latex-korrekturansicht-vorspann.tex
%% Vorspann für die Korrekturansicht.
%% Lädt die gemeinsame Datei latex-vorspann.tex mit gesetztem Schalter.

\newif\ifkorrekturansicht
\korrekturansichttrue

\input{../tex-inputs/latex-vorspann}


               \section[Paul Goldmann an Arthur Schnitzler, Paul Goldmann an Arthur Schnitzler, 2. 4. {[}1896{]}]{ Paul Goldmann an Arthur Schnitzler, 2. 4. {[}1896{]}}\nopagebreak\mylabel{v}\rehead{ }\normalsize\beginnumbering\briefempfaengerindex{Schnitzler, Arthur@\textsc{Schnitzler, Arthur}!zzzGoldmann, Paul@\emph{von Paul Goldmann}!1896-04-021@{2. 4. {[}1896{]}}|(be} \toendnotes[C]{\smallbreak\pagebreak[2]} \Standort{DLA, A:Schnitzler, HS.NZ85.1.3166.}
\physDesc{Brief, 1 Blatt, 4 Seiten
\newline{}Handschrift: blaue Tinte, deutsche Kurrent
\newline{}Schnitzler: 1) mit Bleistift das Jahr »96« vermerkt 2) mit rotem Buntstift eine Unterstreichung}\toendnotes[C]{\smallbreak}\pstart
           \noindent{}{\pb}\textcolor{gray}{\textbf{\textbf{\textcolor{brown}{Frankfurter Zeitung}{}\ledrightnote{\textcolor{brown}{Frankfurter Zeitung}}}}}\pend
           \pstart
           \textcolor{gray}{\textbf{(\textcolor{brown}{\begin{otherlanguage}{french}Gazette de Francfort\end{otherlanguage}}{}\ledrightnote{\textcolor{brown}{Frankfurter Zeitung}}).}}\pend
           \pstart
           \textcolor{gray}{\textbf{\textbf{\begin{otherlanguage}{french}Fondateur M.\end{otherlanguage}{ }\textcolor{blue}{L. Sonnemann}{}\ledrightnote{\textcolor{blue}{Leopold Sonnemann}}.}}}\pend
           \pstart
           \begin{otherlanguage}{french}\textcolor{gray}{\textbf{\textcolor{green}{Journal}{}\ledrightnote{→\textcolor{green}{Frankfurter Zeitung}} politique,
                        financier,}}\end{otherlanguage}\pend
           \pstart
           \begin{otherlanguage}{french}\textcolor{gray}{\textbf{commercial et littéraire.}}\end{otherlanguage}\pend
           \pstart
           \begin{otherlanguage}{french}\textcolor{gray}{\textbf{\textbf{Paraissant trois fois par jour.}}}\end{otherlanguage}\pend
           \pstart
           \begin{otherlanguage}{french}\textcolor{gray}{\textbf{\textbf{Bureau à \textcolor{pink}{Paris}{}\ledrightnote{\textcolor{pink}{Paris}}}}}\end{otherlanguage}\pend
           \pstart
           \begin{otherlanguage}{french}\textcolor{gray}{\textbf{\textbf{\textcolor{pink}{24. Rue Feydeau}{}\ledrightnote{\textcolor{pink}{rue Feydeau}}.}}}\end{otherlanguage}\hfill \textsc{\textcolor{pink}{Paris}{}\ledrightnote{\textcolor{pink}{Paris}}}, 2. April.\pend
           \pstart\center{}Mein lieber Freund,\pend\pstart
           Das iſt auch noch nicht der lange Brief, ſondern nur eine Nachſchrift zum geſtrigen.
               Ich empfing geſtern{ }Nachmittag den Beſuch des \textsc{M. 
                  \textcolor{blue}{Schefer}{}\ledrightnote{\textcolor{blue}{Christian Schefer}}}, eines geſcheiten und Vornehmen \textcolor{blue}{Mann}{}\ledrightnote{→\textcolor{blue}{Christian Schefer}}es (\textcolor{blue}{Profeſſor}{}\ledrightnote{→\textcolor{blue}{Christian Schefer}} an der \textsc{\textcolor{brown}{École des Sciences Politiques}{}\ledrightnote{\textcolor{brown}{École libre des sciences politiques}} etc.}), der
               in der »\textsc{\textcolor{green}{Nouvelle Revue}{}\ledrightnote{\textcolor{green}{La Nouvelle Revue}}}«, die zu den angeſehenſten und geleſenſten Revuen gehört, nächſtens eine Rubrik
               über auswärtige Literatur eröffnen wird. Er will das nicht ſo oberflächlich machen,
               wie dies ſonſt hier geſchieht, will gründlich auf die Sache {\pb}eingehen und alle Zuſammenhänge beleuchten. Er frug
               mich um Rath wegen des deutſchen Geiſteslebens und wollte den Namen eines neuen
               Talents wiſſen, mit dem er ſeine Beſprechungen über deutſche Literatur einleiten
               könnte. Du kannſt Dir denken, daß ich eifrig die Gelegenheit ergriff, um ihm von Dir
               zu ſprechen. Es ſcheint, daß Du gerade das biſt, was er braucht, er war ganz Feuer
               und Flamme, nahm mir mein Exemplar von der »\textcolor{green}{Liebelei}{}\ledrightnote{\textcolor{green}{Liebelei. Schauspiel in drei Akten}}« weg (was er lieber hätte nicht thun {\pb}ſollen), ließ ſich Deine Photographie zeigen und
               erwartet Deine \textcolor{green}{Bücher}{}\ledrightnote{→\textcolor{green}{Sterben. Novelle}{\newline}→\textcolor{green}{Liebelei. Schauspiel in drei Akten}{\newline}→\textcolor{green}{Anatol}}, deren Zuſendung ich ihm in Deinem Namen verſprochen habe. Bitte,
               ſchicke ihm alſo: 1.) \textcolor{green}{Sterben}{}\ledrightnote{\textcolor{green}{Sterben. Novelle}} 2.) \textcolor{green}{Liebelei}{}\ledrightnote{\textcolor{green}{Liebelei. Schauspiel in drei Akten}} 3.) \textsc{\textcolor{green}{Anatol}{}\ledrightnote{\textcolor{green}{Anatol}}}. Schreibe in eines der \textcolor{green}{Bücher}{}\ledrightnote{→\textcolor{green}{Sterben. Novelle}{\newline}→\textcolor{green}{Liebelei. Schauspiel in drei Akten}{\newline}→\textcolor{green}{Anatol}} (oder in alle) \label{K_L02770-1v}\edtext{\begin{otherlanguage}{french}\textsc{À Monsieur \textcolor{blue}{Schefer,}{}\ledrightnote{\textcolor{blue}{Christian Schefer}}
                     Hommage de l’auteur}\end{otherlanguage}}{\lemma{\textnormal{\emph{À … l’auteur}}}\Cendnote{\textnormal{französisch: an Herrn \textcolor{blue}{Schefer}, Würdigung des Autors}}}\label{K_L02770-1h}, mit Unterzeichnung
               Deines Namens. Ich hoffe, das wird gute \label{K_L02770-2v}\edtext{Früchte tragen}{\lemma{\textnormal{\emph{Früchte tragen}}}\Cendnote{\textnormal{\textcolor{blue}{Christian Schefer}: \emph{\textcolor{green}{Un jeune écrivain viennois: M. Arthur Schnitzler}}. In:
                        \emph{\textcolor{green}{La Nouvelle Revue}}, Jg. 18, Nr. 100,
                        Mai–Juni 1896,
                     S. 855–859.}}}\label{K_L02770-2h}; auch eröffnet mir das eine neue Perſpective für die
                  \textcolor{green}{Überſetzung}{}\ledrightnote{→\textcolor{green}{Amourette. Pièce en trois actes}}s-Angelegenheit,
               und wir wollen daher dieß, wenns Dir recht iſt, {\pb}noch ein wenig aufſchieben. Adreſſe: \textsc{M. \textcolor{blue}{Schefer}{}\ledrightnote{\textcolor{blue}{Christian Schefer}}, »\textcolor{brown}{Nouvelle
                     Revue}{}\ledrightnote{\textcolor{brown}{Nouvelle Revue}}«, \textcolor{pink}{18. Boulevard Montmartre,
                     Paris}{}\ledrightnote{\textcolor{pink}{Boulevard Montmartre}}}. (\uline{Kein} Begleitbrief.)\pend
           \pstart
           Grüß’ Dich Gott, liebſter Freund!\pend
           \pstart
           Dein {\\[\baselineskip]}\spacefill\mbox{Paul Goldmann}\pend
           \leftskip=0em{}\endnumbering\briefempfaengerindex{Schnitzler, Arthur@\textsc{Schnitzler, Arthur}!zzzGoldmann, Paul@\emph{von Paul Goldmann}!1896-04-021@{2. 4. {[}1896{]}}|)be}\mylabel{h}  \normalsize

\doendnotes{C}
\bigskip
\vfill

\clearpage

\footnotesize

\lohead{\textsc{register}}

% Definiere theindex-Environment komplett neu ohne reledmac
\makeatletter
\renewenvironment{theindex}{%
  \section*{\indexname}%
  \setlength{\parindent}{0pt}%
  \setlength{\parskip}{0pt plus 0.3pt}%
  \let\item\@idxitem
}{%
  \clearpage
}
\makeatother

\IfFileExists{\jobname-pw.ind}{\input{\jobname-pw.ind}}{}

\end{document}

      