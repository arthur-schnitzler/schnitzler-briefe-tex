%% latex-korrekturansicht-vorspann.tex
%% Vorspann für die Korrekturansicht.
%% Lädt die gemeinsame Datei latex-vorspann.tex mit gesetztem Schalter.

\newif\ifkorrekturansicht
\korrekturansichttrue

\input{../tex-inputs/latex-vorspann}


               \section[Arthur Schnitzler an Richard Beer-Hofmann, 25. 8. 1903]{ Arthur Schnitzler an Richard Beer-Hofmann, 25. 8. 1903}\nopagebreak\mylabel{v}\rehead{ }\normalsize\beginnumbering\briefempfaengerindex{Beer-Hofmann, Richard@\textsc{Beer-Hofmann, Richard}!zzzSchnitzler, Arthur@\emph{von Arthur Schnitzler}!1903-08-251@{25. 8. 1903}|(be} \toendnotes[C]{\smallbreak\pagebreak[2]} \Standort{YCGL, MSS 31.}
\physDesc{Telegramm
\newline{}Handschrift einer Schreibkraft: Bleistift, deutsche Kurrent\newline{}Versand: »\noindent{}\textcolor{gray}{\textbf{Gattung des Telegrammes.}} P.{ / }\textcolor{gray}{\textbf{Aufgegeben am {\dots} 190{\dots}{ }um}}{ }4 \textcolor{gray}{\textbf{Uhr}} 45 \textcolor{gray}{\textbf{Min.}} n \textcolor{gray}{\textbf{Mittag}}{ / }\textcolor{gray}{\textbf{Eingelangt von}} H. \textcolor{gray}{\textbf{auf Leitung Nr.}}{ }535/43 \textcolor{gray}{\textbf{am}}{ }25/8 \textcolor{gray}{\textbf{190}}3 { }\textcolor{gray}{\textbf{um}}{ }5 \textcolor{gray}{\textbf{Uhr}} 30 \textcolor{gray}{\textbf{Min.}} n \textcolor{gray}{\textbf{Mittag}}{ / }\textcolor{gray}{\textbf{Aufgenommen durch}}{ }\textcolor{gray}{H}{ / }\textcolor{gray}{\textbf{Von}}{ }\textcolor{pink}{Wien 72}{ }\textcolor{gray}{\textbf{Aufgabe-Nr.}} 249402{ }\textcolor{gray}{\textbf{mit}} 14 \textcolor{gray}{\textbf{Taxworten ({\dots} Worten {\dots} Chiffern)}}« \newline{}Ordnung: mit Bleistift von unbekannter Hand datiert: »25. 8. 1903« }\pstart{}{\pb}\textsc{Richard Beerhofmann}\pend{}\pstart{}\textsc{\textcolor{pink}{Rodaun}{}\ledrightnote{\textcolor{pink}{Rodaun}}{ }\textcolor{pink}{Liesingerstrasse 2}{}\ledrightnote{\textcolor{pink}{Liesingerstraße}}}\pend{}{\bigskip}\pstart
           \noindent{}{\pb}Nicht um halb eins ſondern Punkt
                  eins\pend
           \pstart Herzlichſt \spacefill\mbox{Arthur}\pend{}\endnumbering\briefempfaengerindex{Beer-Hofmann, Richard@\textsc{Beer-Hofmann, Richard}!zzzSchnitzler, Arthur@\emph{von Arthur Schnitzler}!1903-08-251@{25. 8. 1903}|)be}\mylabel{h}  \normalsize

\doendnotes{C}
\bigskip
\vfill

\clearpage

\footnotesize

\lohead{\textsc{register}}

% Definiere theindex-Environment komplett neu ohne reledmac
\makeatletter
\renewenvironment{theindex}{%
  \section*{\indexname}%
  \setlength{\parindent}{0pt}%
  \setlength{\parskip}{0pt plus 0.3pt}%
  \let\item\@idxitem
}{%
  \clearpage
}
\makeatother

\IfFileExists{\jobname-pw.ind}{\input{\jobname-pw.ind}}{}

\end{document}

      