%% latex-korrekturansicht-vorspann.tex
%% Vorspann für die Korrekturansicht.
%% Lädt die gemeinsame Datei latex-vorspann.tex mit gesetztem Schalter.

\newif\ifkorrekturansicht
\korrekturansichttrue

\input{../tex-inputs/latex-vorspann}


               \section[Paul Goldmann an Arthur Schnitzler, 5. 1. {[}1895{]}]{ Paul Goldmann an Arthur Schnitzler, 5. 1. {[}1895{]}}\nopagebreak\mylabel{v}\rehead{ }\normalsize\beginnumbering\briefempfaengerindex{Schnitzler, Arthur@\textsc{Schnitzler, Arthur}!zzzGoldmann, Paul@\emph{von Paul Goldmann}!1895-01-051@{5. 1. {[}1895{]}}|(be} \toendnotes[C]{\smallbreak\pagebreak[2]} \Standort{DLA, A:Schnitzler, HS.NZ85.1.3165.}
\physDesc{Brief, 3 Blätter, 12 Seiten
\newline{}Handschrift: schwarze Tinte, deutsche Kurrent
\newline{}Schnitzler: 1) mit rotem Buntstift »\textsc{\textcolor{green}{Liebelei}}« vermerkt 2) mit Bleistift das Jahr »1895« vermerkt}\toendnotes[C]{\smallbreak}\pstart
           \noindent{}{\pb}\textcolor{gray}{\textbf{\textbf{\textcolor{brown}{Frankfurter Zeitung}{}\ledrightnote{\textcolor{brown}{Frankfurter Zeitung}}.}}}\pend
           \pstart
           \textcolor{gray}{\textbf{(\textcolor{brown}{\begin{otherlanguage}{french}Gazette de Francfort\end{otherlanguage}}{}\ledrightnote{\textcolor{brown}{Frankfurter Zeitung}}.) }}\hfill \textcolor{pink}{\textsc{Paris}}{}\ledrightnote{\textcolor{pink}{Paris}}, 5. Januar.\pend
           \pstart
           \textcolor{gray}{\textbf{\textbf{\begin{otherlanguage}{french}Fondateur M. \textcolor{blue}{L.
                              Sonnemann}{}\ledrightnote{\textcolor{blue}{Leopold Sonnemann}}\end{otherlanguage}.}}}\pend
           \pstart
           \begin{otherlanguage}{french}\textcolor{gray}{\textbf{\textcolor{green}{Journal}{}\ledrightnote{\textcolor{green}{Frankfurter Zeitung}} politique, financier,}}\end{otherlanguage}\pend
           \pstart
           \begin{otherlanguage}{french}\textcolor{gray}{\textbf{commercial et littéraire.}}\end{otherlanguage}\pend
           \pstart
           \begin{otherlanguage}{french}\textcolor{gray}{\textbf{\textbf{Paraissant trois fois par jour.}}}\end{otherlanguage}\pend
           \pstart
           \begin{otherlanguage}{french}\textcolor{gray}{\textbf{\textbf{Bureaux à \textcolor{pink}{Paris}{}\ledrightnote{\textcolor{pink}{Paris}}:}}}\end{otherlanguage}\pend
           \pstart
           \begin{otherlanguage}{french}\textcolor{gray}{\textbf{\textbf{\textcolor{pink}{24. Rue Feydeau}{}\ledrightnote{\textcolor{pink}{rue Feydeau}}.}}}\end{otherlanguage}\pend
           \pstart\center{}Mein lieber Freund,\pend\pstart
           Ich danke Dir von Herzen, daß Du meine Bitte ſo raſch erfüllt haſt. Entſchuldige nur
               die \label{K_L02726-12v}\edtext{großen Koſten}{\lemma{\textnormal{\emph{großen Koſten}}}\Cendnote{\textnormal{\textcolor{blue}{Schnitzler} hatte am 1. 1. 1895 eine Abschrift der \emph{\textcolor{green}{Liebelei}} geschickt.}}}\label{K_L02726-12h}, die ich Dir verurſacht; aber Du haſt mir eine
               große Freude gemacht. Mittags bekam ich \textcolor{green}{es}{}\ledrightnote{→\textcolor{green}{Liebelei. Schauspiel in drei Akten}}, in einer Stunde war es geleſen, und am
               ſelben Tage ſende ich es Dir noch zurück.\pend
           \pstart
           Da ich ſofort ſchreiben muß, bin ich meiner Eindrücke noch nicht ganz ſicher. Der
               erſte \textcolor{green}{Akt}{}\ledrightnote{→\textcolor{green}{Liebelei. Schauspiel in drei Akten}} iſt voll Anmuth,
               voll Bewegung, er endet aufs {\pb}Packendſte. Ich
               glaube, er wird ſehr gut geſpielt werden müſſen. Die zwangloſe, natürliche
               Fröhlichkeit ſtellt den Komödianten keine leichte Aufgabe. Auch möchte ich gleich
               hier ſagen, daß ich beſonders dieſe einfache Sprache überall bewundert habe. \strikeout{Das} Die Leute ſprechen im \textcolor{green}{Stück}{}\ledrightnote{→\textcolor{green}{Liebelei. Schauspiel in drei Akten}}, wie im Leben. Welch’ eine Kunſt da
               drinſteckt! Im zweiten \textcolor{green}{Akt}{}\ledrightnote{→\textcolor{green}{Liebelei. Schauspiel in drei Akten}} –
               und auch ſonſt – hätte ich gern, daß der alte \textsc{\textcolor{green}{Weiring}{}\ledrightnote{→\textcolor{green}{Liebelei. Schauspiel in drei Akten}}} etwas mehr \strikeout{her} hervorträte, als blos mit ein
               wenig Profil. Ich hätte ihn etwas ausführlicher gewünſcht, \strikeout{umſomehr als ich} eine kleine Scene rührender Vaterliebe zwiſchen ihm und
               dem Mädel hätte das \textcolor{green}{Ende}{}\ledrightnote{→\textcolor{green}{Liebelei. Schauspiel in drei Akten}}{ }{\pb}noch um eine \textsc{Nuance}
               tragiſcher gemacht. »Ich alter Mann\strikeout{,} habe nur noch
               Dich.« Es gibt nichts mehr zum Weinen, als hilfloſes, verlaſſenes Alter. Zudem bin
               ich überzeugt, daß der \label{K_L02726-34v}\edtext{\textcolor{blue}{Herr}{}\ledrightnote{→\textcolor{blue}{Josef von Bezecný}}, der von
                  Cenſur-Schwierigkeiten}{\lemma{\textnormal{\emph{Herr, … Cenſur-Schwierigkeiten}}}\Cendnote{\textnormal{siehe A. S.: \emph{Tagebuch}, 26. 12. 1894}}}\label{K_L02726-34h} ſprach,
               gerade die Reden \textsc{\textcolor{green}{Weiring}{}\ledrightnote{→\textcolor{green}{Liebelei. Schauspiel in drei Akten}}s } über Tugend
               und Behütung von Glück gemeint hat. Das iſt zwar eine Hauptſache, ein Grundgedanke
               des \textcolor{green}{Stück}{}\ledrightnote{→\textcolor{green}{Liebelei. Schauspiel in drei Akten}}es. Das liegt aber
               den Trotteln wenig auf. Niemals wird man im kaiſerlichen \strikeout{Hofteh}{ }\textcolor{brown}{Hoftheater}{}\ledrightnote{→\textcolor{brown}{Burgtheater}} ſo etwas ſagen
               laſſen. Sonſt iſt die \textcolor{green}{Scene}{}\ledrightnote{→\textcolor{green}{Liebelei. Schauspiel in drei Akten}}
               ergreifend. Die \textcolor{green}{Abſchiedsſcene}{}\ledrightnote{→\textcolor{green}{Liebelei. Schauspiel in drei Akten}}
               hätte ich auch {\pb}noch um einen Grad kräftiger
               gewünſcht, mit etwas mehr Betonung darauf, daß es der \uline{Abſchied} ist. \introOben{}Auch ſollte er einmal vom Sterben ſprechen
                  und Angſt zeigen.\introOben{} Sonſt iſt ſie entzückend. Der \textcolor{green}{Schluß}{}\ledrightnote{→\textcolor{green}{Liebelei. Schauspiel in drei Akten}} mit der letzten Umarmung \strikeout{m} wird ungeheuer wirken. Einfach, aber ſo ſchön! Der
               dritte \textcolor{green}{Akt}{}\ledrightnote{→\textcolor{green}{Liebelei. Schauspiel in drei Akten}} iſt der Höhepunkt;
               überhaupt iſt das \textcolor{green}{Stück}{}\ledrightnote{→\textcolor{green}{Liebelei. Schauspiel in drei Akten}}
               vorzüglich gebaut, es wächſt ſo allmälig ins große Dramatiſche hinein. Bewundert habe
               ich nebenbei die Kunſt, mit der Du all’ die techniſchen Schwierigkeiten für den
               dritten \textcolor{green}{Akt}{}\ledrightnote{→\textcolor{green}{Liebelei. Schauspiel in drei Akten}} bewältigt haſt,
               von denen Du in \label{K_L02726-2v}\edtext{\textsc{\textcolor{pink}{Ischl}{}\ledrightnote{\textcolor{pink}{Bad Ischl}}}}{\lemma{\textnormal{\emph{Ischl}}}\Cendnote{\textnormal{Zwischen 23. 8. 1894 und 3. 9. 1894 verbrachten \textcolor{blue}{Schnitzler} und \textcolor{blue}{Goldmann} einige Zeit gemeinsam in \textcolor{pink}{Ischl}. Am 30. 8. 1894 sowie am 1. 9. 1894 diskutierten sie »fruchtbar« über die \emph{\textcolor{green}{Liebelei}}, die damals noch den Titel \emph{\textcolor{green}{Armes Mädl}} hatte.}}}\label{K_L02726-2h} ſprachſt. Man kann
               ſich keinen zwangloſeren und natürlicheren {\pb}Vorgang
               denken. Beſonders daß die Sache »übermorgen« ſpielt, iſt zugleich techniſch fein und
               dramatiſch wirkſam. Nun möchte ich auf eine kleine Gefahr aufmerkſam machen: daß man
               nämlich den \textsc{\textcolor{green}{Theodor}{}\ledrightnote{→\textcolor{green}{Liebelei. Schauspiel in drei Akten}}}, wenn er nicht \strikeout{vortrefflich} ſehr geſchickt
               geſpielt wird, im Publikum zuerſt komiſch nehmen kann. Er iſt auch gar zu ſehr
                  »\label{K_L02726-1v}\edtext{\begin{otherlanguage}{french}mufle\end{otherlanguage}}{\lemma{\textnormal{\emph{mufle}}}\Cendnote{\textnormal{französisch: Rüpel}}}\label{K_L02726-1h}«. Insbeſondere
               möchte ich, daß er das von dem Fallen im Duell nicht gar zu trocken herausſagt. Ich
               weiß wohl, was Du damit willſt: mit {\pb}dem Mädel macht
               man eben keine Umſtände. Aber ſo ein roher Kerl iſt der \textsc{\textcolor{green}{Theodor}{}\ledrightnote{→\textcolor{green}{Liebelei. Schauspiel in drei Akten}}} doch nicht. Er ſollte wenigſtens verlegen ſein, zu umſchreiben verſuchen:
               Unfall {\dotsfour} ſchwer verwundet {\dotsfour} und
                  \strikeout{lan} dann erſt das Duell herausbringen. Die Tragik,
               die dann mit elementarer Gewalt lospraſſelt, – die Reden des Mädels – das iſt ein \textcolor{green}{Meiſterſtück}{}\ledrightnote{→\textcolor{green}{Liebelei. Schauspiel in drei Akten}}. Mich hats bereits
               beim Leſen in der Kehle gewürgt. Auf dem Theater kann dem kein Menſch wiederſtehen.
               Herrlich und tief ergreifend! Der \textcolor{green}{Schluß}{}\ledrightnote{→\textcolor{green}{Liebelei. Schauspiel in drei Akten}} gefällt mir nicht. Ich möchte nicht, daß ſie ſich umbringt. Das iſt
                  {\pb}gar nicht nöthig. Laß’ dem dummen Publikum
               wenigſtens den kleinen Troſt, daß ſie leben bleibt. Es kann viel erſchütternder
               enden. Sinkt dem Vater weinend an die Bruſt und der hebt ſchluchzend ſeinen
               zitternden Arm und ſchreit zu \textsc{\textcolor{green}{Theodor}{}\ledrightnote{→\textcolor{green}{Liebelei. Schauspiel in drei Akten}}}, dem Repräſentanten der »Welt draußen«: »Ihr habt mir mein Mädel umgebracht.«
               Oder ſo was. Aber kein Weglaufen. Man verhindert \strikeout{\textcolor{gray}{ſod}} ſie auch ans Grab zu gehen, damit baſta! Die Fenſter-Hinausſchreierei iſt
               verſehlt. Die \textcolor{green}{Hauptperſon}{}\ledrightnote{→\textcolor{green}{Liebelei. Schauspiel in drei Akten}} muß
               auf der Bühne bleiben. Und dann ſo unwahrſcheinlich. {\pb}Er holt ſie ja doch ein\strikeout{;} bis zum Kirchhof, braucht
               ſich nur einen Fiaker zu nehmen, um ihr zuvorzukommen. Oder die \textsc{\textcolor{green}{Mizzi}{}\ledrightnote{→\textcolor{green}{Liebelei. Schauspiel in drei Akten}}} ſchreit aus dem Fenſter den Paſſanten zu: »Haltets auf!« Das \uline{mußt} Du ändern. Es iſt ein Fehler, das Ende hinter die
               Couliſſen zu verlegen.\pend
           \pstart
           Im Ganzen: ein edles und reifes \textcolor{green}{Werk}{}\ledrightnote{→\textcolor{green}{Liebelei. Schauspiel in drei Akten}}. Ich beglückwünſche Dich dazu von ganzem Herzen. Ich kenne zur Zeit
               Niemanden, der ſo etwas ſchreiben könnte, auch hier in \textcolor{pink}{Frankreich}{}\ledrightnote{\textcolor{pink}{Frankreich}} nicht. Es iſt die \textcolor{green}{Krönung}{}\ledrightnote{→\textcolor{green}{Liebelei. Schauspiel in drei Akten}} Deines bisherigen Lebens und Schaffens, {\pb}und wird es erſt einmal aufgeführt, ſo wird die Welt
               mit Erſtaunen ſehen, daß Du ein Dichter biſt{\dots}\pend
           \pstart
           Gräulich iſt, nochmals, der \textcolor{green}{Titel}{}\ledrightnote{→\textcolor{green}{Liebelei. Schauspiel in drei Akten}}. Wenn Du einen hätteſt wählen wollen, der alle ſchlimmen Vorurtheile
               gegen das \textcolor{green}{Stück}{}\ledrightnote{→\textcolor{green}{Liebelei. Schauspiel in drei Akten}} erwecken
               ſollte, ſo hätteſt Du keinen beſſern finden können. Du mußt es umtaufen. Kannſt und
               willſt Du es nicht »\textcolor{green}{Eine
                  Liebſchaft}{}\ledrightnote{→\textcolor{green}{Liebelei. Schauspiel in drei Akten}}« nennen – das wäre das weitaus Beſte – ſo {\pb}möchte ich Dir vorſchlagen: »\textcolor{green}{Arme Liebe}{}\ledrightnote{→\textcolor{green}{Liebelei. Schauspiel in drei Akten}}«. Leicht \strikeout{kan} kannſt Du der \textcolor{green}{Chriſtine}{}\ledrightnote{→\textcolor{green}{Liebelei. Schauspiel in drei Akten}} im dritten \textcolor{green}{Akt}{}\ledrightnote{→\textcolor{green}{Liebelei. Schauspiel in drei Akten}} noch zehn Worte in den Mund legen, die dieſen \textcolor{green}{Titel}{}\ledrightnote{→\textcolor{green}{Liebelei. Schauspiel in drei Akten}} erklären\substVorne{}\textsuperscript{!}\substDazwischen{};\substHinten{} oder noch beſſer der Vater ſoll es zum \textcolor{green}{Schluß}{}\ledrightnote{→\textcolor{green}{Liebelei. Schauspiel in drei Akten}} ſagen: »Wein’ Dich aus, \strikeout{armes} Kind. Wenn arme Leute lieben, ſo dürfen ſie nichts beanſpruchen als
               Thränen.« \strikeout{D} In der Größe ſeines Schmerzes wird der
               Alte aphoriſtiſch – ein einziges Mal. Das wäre umſo wirkſamer. Und denk’ Dir nur, was
                  \strikeout{ſich} für eine {\pb}große allgemeine Perſpektive ſich am Schluß durch dieſe Worte noch öffnen würde.
               Das wäre doch beſſer, als die Fenſter-Geſchichten {\dotsfive}\pend
           \pstart
           Vielen, vielen Dank, mein lieber Freund, für den großen Genuß, den Du mir verſchafft
               haſt. Wie ſtehts nun mit der \label{K_L02726-3v}\edtext{\textcolor{green}{Aufführung}{}\ledrightnote{→\textcolor{green}{Liebelei. Schauspiel in drei Akten}}}{\lemma{\textnormal{\emph{Aufführung}}}\Cendnote{\textnormal{Die »\emph{\textcolor{green}{Liebelei}}« wurde am 9. 10. 1895 am \textcolor{pink}{Wien}er \textcolor{pink}{Burgtheater} uraufgeführt.}}}\label{K_L02726-3h}? Schreib’ mir
               bald und ausführlich.\pend
           \pstart
           Zwei Bitten: Erſtens. Ich habe zum Neujahr ein ſchönes
                  \label{K_L02726-4v}\edtext{Alt-\textcolor{pink}{Wien}{}\ledrightnote{\textcolor{pink}{Wien}}er Bild}{\lemma{\textnormal{\emph{Alt-Wiener Bild}}}\Cendnote{\textnormal{nicht ermittelt; mit
                     Alt-\textcolor{pink}{Wien} gemeint ist, eine Motiv oder eine
                  Darstellung aus der Zeit vor der Schleifung der Basteien und dem \textcolor{pink}{Ringstraße}nbau.}}}\label{K_L02726-4h} erhalten, von \textsc{\textcolor{brown}{Artaria}{}\ledrightnote{\textcolor{brown}{Artaria {\kaufmannsund} Co.}}}, mit dem ich mich unbändig gefreut habe. Aber ohne {\pb}Begleitbrief. Ein ſo zartſinniges, von Herzen zu
               Herzen gehendes Geſchenk kann nur von \label{K_L02726-67v}\edtext{\textcolor{blue}{Jemandem}{}\ledrightnote{→\textcolor{blue}{Julius Schnitzler}{\newline}→\textcolor{blue}{Helene Schnitzler}} aus
               Deinem Kreiſe}{\lemma{\textnormal{\emph{Jemandem … Kreiſe}}}\Cendnote{\textnormal{es kam von \textcolor{blue}{Schnitzler}s Bruder \textcolor{blue}{Julius} und dessen Frau \textcolor{blue}{Helene}, vgl. Paul Goldmann an Arthur Schnitzler, 2. 3. [1895]}}}\label{K_L02726-67h}
               herkommen. Sag’ mir, wer der Spender iſt.\pend
           \pstart
           Zweitens. Schreib’ mir \label{K_L02726-78v}\edtext{\textcolor{blue}{Torresani}{}\ledrightnote{\textcolor{blue}{Carl von Torresani-Lanzenfeld}}s Adreſſe}{\lemma{\textnormal{\emph{Torresanis Adreſſe}}}\Cendnote{\textnormal{\textcolor{blue}{Torresani}
               scheint im Adressbuch \emph{\textcolor{green}{Lehmann}} für das Jahr 1891 zum letzten Mal als wohnhaft in \textcolor{pink}{Wien} auf. Danach reiste
               er jahrelang.}}}\label{K_L02726-78h}.\pend
           \pstart
           Viele treue Grüße! {\\[\baselineskip]}Dein {\\[\baselineskip]}\spacefill\mbox{Paul Goldmann}\pend
           \leftskip=0em{}\endnumbering\briefempfaengerindex{Schnitzler, Arthur@\textsc{Schnitzler, Arthur}!zzzGoldmann, Paul@\emph{von Paul Goldmann}!1895-01-051@{5. 1. {[}1895{]}}|)be}\mylabel{h}  \normalsize

\doendnotes{C}
\bigskip
\vfill

\clearpage

\footnotesize

\lohead{\textsc{register}}

% Definiere theindex-Environment komplett neu ohne reledmac
\makeatletter
\renewenvironment{theindex}{%
  \section*{\indexname}%
  \setlength{\parindent}{0pt}%
  \setlength{\parskip}{0pt plus 0.3pt}%
  \let\item\@idxitem
}{%
  \clearpage
}
\makeatother

\IfFileExists{\jobname-pw.ind}{\input{\jobname-pw.ind}}{}

\end{document}

      