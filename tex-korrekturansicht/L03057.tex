%% latex-korrekturansicht-vorspann.tex
%% Vorspann für die Korrekturansicht.
%% Lädt die gemeinsame Datei latex-vorspann.tex mit gesetztem Schalter.

\newif\ifkorrekturansicht
\korrekturansichttrue

\input{../tex-inputs/latex-vorspann}


\renewcommand{\erwaehntePersonen}{Personen: Richard Beer-Hofmann, Gabriel Beer-Hofmann, Eduard Douwes Dekker, Marie Glümer, Louise Schnitzler, Olga Schnitzler, Elisabeth Steinrück, Ida d’Albert}
\renewcommand{\erwaehnteInstitutionen}{Institutionen: Breslauer Zeitung}
\renewcommand{\erwaehnteOrte}{Orte: Berlin, Bologna, Dessauer Straße, Florenz, Genua, Griechenland, Italien, Niederlande, Pisa, Rom, Rotensterngasse, Ungarn, Wien}
\renewcommand{\erwaehnteWerke}{Werke: Arbeiter-Zeitung, Tagebuch, Tagesneuigkeiten. Richtig}
\section[ Paul Goldmann an Arthur Schnitzler, 12. 2. {[}1901{]}]{Paul Goldmann an Arthur Schnitzler, 12. 2. {[}1901{]}}
\nopagebreak\mylabel{v}
\rehead{ }\normalsize\beginnumbering\briefempfaengerindex{Schnitzler, Arthur@\textsc{Schnitzler, Arthur}!zzzGoldmann, Paul@\emph{von Paul Goldmann}!1901-02-121@{12. 2. {[}1901{]}}|(be}
\toendnotes[C]{\smallbreak\pagebreak[2]}\Standort{DLA, A:Schnitzler, HS.NZ85.1.3171.}
\physDesc{Brief, 1 Blatt, 3 Seiten
\newline{}Handschrift: blaue Tinte, deutsche Kurrent
\newline{}Beilage: ein Zeitungsausschnitt, beschnitten und aufgeklebt 
\newline{}Schnitzler: 1) mit Bleistift das Jahr »1901« vermerkt  2) mit rotem Buntstift drei Unterstreichungen}\toendnotes[C]{\smallbreak}
\pstart
           \noindent{}\raggedleft{}{\pb}\textcolor{pink}{\textcolor{gray}{\textbf{DESSAUERSTRASSE 19}}}{}\ledrightnote{\textcolor{pink}{Dessauer Straße}}\pend
           
\pstart
           \textcolor{pink}{Berlin}{}\ledrightnote{\textcolor{pink}{Berlin}}, 12. Februar.\pend
           
\pstart\center{}Mein lieber Freund,\pend
\pstart
           Wie gehts?\pend
           
\pstart
           Nach \label{K_L03057-1v}\edtext{\textcolor{pink}{Italien}{}\ledrightnote{\textcolor{pink}{Italien}}}{\lemma{\textnormal{\emph{Italien}}}\Cendnote{\textnormal{Bezug auf \textcolor{blue}{Schnitzler}s \textcolor{pink}{Italien}reise nach \textcolor{pink}{Genua}, \textcolor{pink}{Pisa}, \textcolor{pink}{Rom}, \textcolor{pink}{Florenz} und \textcolor{pink}{Bologna} zwischen 26. 3. 1901 und 18. 4. 1901}}}\label{K_L03057-1h} kann ich ſelbſtverſtändlich
               nicht mitkommen. Aber es iſt ſchön, daß Du hingehſt.\pend
           
\pstart
           Frau \textcolor{blue}{Fulda}{}\ledrightnote{\textcolor{blue}{Ida d’Albert}} (welche ein geiſt- und herzloſes
               Weib iſt und mir immer weniger ſympathiſch wird) ſuchte dieſer Tage aus mir
               herauszubekommen, ob Du in \label{K_L03057-2v}\edtext{weiblicher
                  Geſellſchaft}{\lemma{\textnormal{\emph{weiblicher
                  Geſellſchaft}}}\Cendnote{\textnormal{\textcolor{blue}{Schnitzler} wurde auf seiner Reise, abgesehen von seiner
                  Mutter \textcolor{blue}{Louise}, die am 11. 4. 1901 in \textcolor{pink}{Florenz} ankam, sowie vereinzelten Begegnungen,
                  von keiner Frau begleitet.}}}\label{K_L03057-2h} nach \textcolor{pink}{Italien}{}\ledrightnote{\textcolor{pink}{Italien}} gehſt? Ich ſagte: nein.\pend
           
\pstart
           {\pb}Was macht die \label{K_L03057-3v}\edtext{\textcolor{blue}{\textcolor{pink}{Rotheſterngaſſe}{}\ledrightnote{\textcolor{pink}{Rotensterngasse}}}{}\ledrightnote{{$\rightarrow$}\textcolor{blue}{Olga Schnitzler}{\newline}{$\rightarrow$}\textcolor{blue}{Elisabeth Steinrück}}}{\lemma{\textnormal{\emph{Rotheſterngaſſe}}}\Cendnote{\textnormal{wohl hauptsächlich Bezug auf \textcolor{blue}{Schnitzler}s spätere Ehefrau \textcolor{blue}{Olga Schnitzler}, die in der \textcolor{pink}{Rotensterngasse} wohnte}}}\label{K_L03057-3h}?\pend
           
\pstart
           Bitte, lies \label{K_L03057-4v}\edtext{\textsc{\textcolor{blue}{Multatuli}{}\ledrightnote{\textcolor{blue}{Eduard Douwes Dekker}}}}{\lemma{\textnormal{\emph{Multatuli}}}\Cendnote{\textnormal{Pseudonym des \textcolor{pink}{niederländ}ischen Autors \textcolor{blue}{Eduard Douwes Dekker}; Lektüre mittels \emph{\textcolor{green}{Tagebuch}} und Leseliste belegbar, vgl. A. S.: \emph{Lektüren}, Norden sowie A. S.: \emph{Tagebuch}, 28. 11. 1907, 30. 11. 1907, 12. 1. 1908, 26. 1. 1908}}}\label{K_L03057-4h}!\pend
           
\pstart
           \textsc{\textcolor{blue}{Richard}{}\ledrightnote{\textcolor{blue}{Richard Beer-Hofmann}}} hat ſich in der That nicht dazu aufſchwingen können, mir die \label{K_L03057-6v}\edtext{Geburt ſeines \textcolor{blue}{Sohn}{}\ledrightnote{{$\rightarrow$}\textcolor{blue}{Gabriel Beer-Hofmann}}es}{\lemma{\textnormal{\emph{Geburt ſeines Sohnes}}}\Cendnote{\textnormal{\textcolor{blue}{Gabriel Beer-Hofmann}
                  wurde am 9. 1. 1901 in \textcolor{pink}{Wien} geboren.}}}\label{K_L03057-6h} anzuzeigen. Ich habe keine Worte mehr
               für dieſes Benehmen. Nichtsdeſtoweniger ſchicke ich ihm die nachfolgende \label{K_L03057-11v}\edtext{\textcolor{green}{Zeitungsnotiz}{}\ledrightnote{{$\rightarrow$}\textcolor{green}{Tagesneuigkeiten. Richtig}}}{\lemma{\textnormal{\emph{Zeitungsnotiz}}}\Cendnote{\textnormal{Die Meldung wurde Ende Januar 1901 in diversen
                  Zeitungen gebracht, etwa: [O. V.]: \emph{\textcolor{green}{Tagesneuigkeiten. Richtig}}. In: \emph{\textcolor{green}{Arbeiter-Zeitung}}, Jg. 13, Nr. 28, 28. 1. 1901, Mittagsblatt, S. 3.}}}\label{K_L03057-11h}:\pend
           
\pstart
           \textcolor{gray}{\textbf{\textbf{Die verkannte Muſe.} Dem Briefkasten eines \label{K_L03057-12v}\edtext{ſüd\textcolor{pink}{ungar}{}\ledrightnote{{$\rightarrow$}\textcolor{pink}{Ungarn}}iſchen Blattes}{\lemma{\textnormal{\emph{ſüdungariſchen Blattes}}}\Cendnote{\textnormal{nicht ermittelt}}}\label{K_L03057-12h} entnimmt die »\textcolor{brown}{Bresl. Ztg.}{}\ledrightnote{\textcolor{brown}{Breslauer Zeitung}}« folgende merkwürdige Antwort: »Alter
                  Abonnent. Sie haben Ihre Wette gewonnen. \label{K_L03057-9v}\edtext{Terpſichore}{\lemma{\textnormal{\emph{Terpſichore}}}\Cendnote{\textnormal{eine der neun Musen aus der \textcolor{pink}{griech}ischen Mythologie, die stellvertretend für die Chorlyrik, den
                     Tanz und die Wissenschaften steht; unklarer Bezug zu \textcolor{blue}{Beer-Hofmann}}}}\label{K_L03057-9h} iſt kein jüdiſcher
               Feiertag«}}\pend
           
\pstart
           {\pb}Frl. \textsc{\textcolor{blue}{Mizzi Glümer}{}\ledrightnote{\textcolor{blue}{Marie Glümer}}} hatte wieder einen \label{K_L03057-7v}\edtext{Rückfall}{\lemma{\textnormal{\emph{Rückfall}}}\Cendnote{\textnormal{siehe Paul Goldmann an Arthur Schnitzler, 22. 1. [1901]}}}\label{K_L03057-7h}, nachdem ſie ſich bereits ganz geneſen
               geglaubt. Es iſt ein Jammer mit dem \textcolor{blue}{Mädel}{}\ledrightnote{{$\rightarrow$}\textcolor{blue}{Marie Glümer}}. Kann das wirklich nur \label{K_L03057-8v}\edtext{\textsc{Neuralgie}}{\lemma{\textnormal{\emph{Neuralgie}}}\Cendnote{\textnormal{Nervenschmerzen; siehe A. S.: \emph{Tagebuch}, 22. 2. 1901, 3. 3. 1901, 5. 3. 1901}}}\label{K_L03057-8h} ſein? Oder was ſonſt?\pend
           
\pstart
           Schreib’ mir bald!\pend
           
\pstart
           Viele treue Grüße! {\\[\baselineskip]}Dein {\\[\baselineskip]}\spacefill\mbox{Paul Goldmnn}\pend
           \leftskip=0em{}\endnumbering\briefempfaengerindex{Schnitzler, Arthur@\textsc{Schnitzler, Arthur}!zzzGoldmann, Paul@\emph{von Paul Goldmann}!1901-02-121@{12. 2. {[}1901{]}}|)be}\mylabel{h}
\begin{anhang}
\end{anhang}\normalsize

\doendnotes{C}
\bigskip
\vfill

\clearpage

\footnotesize

\lohead{\textsc{register}}

% Definiere theindex-Environment komplett neu ohne reledmac
\makeatletter
\renewenvironment{theindex}{%
  \section*{\indexname}%
  \setlength{\parindent}{0pt}%
  \setlength{\parskip}{0pt plus 0.3pt}%
  \let\item\@idxitem
}{%
  \clearpage
}
\makeatother

\IfFileExists{\jobname-pw.ind}{\input{\jobname-pw.ind}}{}

\end{document}

      