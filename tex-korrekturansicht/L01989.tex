%% latex-korrekturansicht-vorspann.tex
%% Vorspann für die Korrekturansicht.
%% Lädt die gemeinsame Datei latex-vorspann.tex mit gesetztem Schalter.

\newif\ifkorrekturansicht
\korrekturansichttrue

\input{../tex-inputs/latex-vorspann}


               \section[Hugo von Hofmannsthal an Arthur Schnitzler, 2. 12. 1910]{ Hugo von Hofmannsthal an Arthur Schnitzler, 2. 12. 1910}\nopagebreak\mylabel{v}\rehead{ }\normalsize\beginnumbering\briefempfaengerindex{Schnitzler, Arthur@\textsc{Schnitzler, Arthur}!zzzHofmannsthal, Hugo von@\emph{von Hugo von Hofmannsthal}!1910-12-021@{2. 12. 1910}|(be} \toendnotes[C]{\smallbreak\pagebreak[2]} \Standort{CUL, Schnitzler, B 43.}
\physDesc{Briefkarte
\newline{}Handschrift: schwarze Tinte, deutsche Kurrent
\newline{}Schnitzler: mit Bleistift beschriftet: »Hugo« \newline{}Ordnung: 1) mit Bleistift von unbekannter Hand nummeriert: »\strikeout{309}« 2) mit Bleistift von unbekannter Hand nummeriert:
                                    »327«}\buchAbdrucke{\weitereDrucke{Hugo von Hofmannsthal, Arthur Schnitzler: \emph{Briefwechsel}. Hg. Therese Nickl und Heinrich Schnitzler. Frankfurt am Main: \emph{S. Fischer} 1964, S. 260.} }\toendnotes[C]{\smallbreak}\pstart
           \raggedleft{}{\pb}\textcolor{pink}{Rodaun}{}\ledrightnote{\textcolor{pink}{Rodaun}}{ }2 XII. 10.\pend
           \pstart{}mein lieber Arthur\pend\pstart
           verzeihen Sie die elende Schlamperei, Ihnen bei \label{K_L01989_1v}\edtext{2 Begegnungen}{\lemma{\textnormal{\emph{2 Begegnungen}}}\Cendnote{\textnormal{siehe A. S.: \emph{Tagebuch}, 29. 11. 1910, 1. 12. 1910}}}\label{K_L01989_1h} das ausgelegte Geld für die \textcolor{green}{2 Plätze}{}\ledrightnote{→\textcolor{green}{Der junge Medardus. Dramatische Historie in einem Vorspiel und fünf Aufzügen}} nicht rückerſtattet zu haben.\hspace*{1.5em}–
               Haben Sie gute Tage \label{K_L01989_2v}\edtext{in \textcolor{pink}{München}{}\ledrightnote{\textcolor{pink}{München}}}{\lemma{\textnormal{\emph{in München}}}\Cendnote{\textnormal{Von 8. 12. 1910 an war er für eine Vorlesung
                  eigener Stücke sowie einer Premiere mehrerer Einakter in \textcolor{pink}{München}.}}}\label{K_L01989_2h}.\hspace*{1.5em}Vielleicht
               verbringen wir {\pb}doch noch vor
               Weihnachten ein paar Tage auf dem \textcolor{pink}{Semmering}{}\ledrightnote{\textcolor{pink}{Semmering}}, das
               wäre ſehr ſchön.\hspace*{1.5em}Daſs Sie in der \label{K_L01989_3v}\edtext{\textcolor{blue}{Goldmann}{}\ledrightnote{\textcolor{blue}{Paul Goldmann}}ſache}{\lemma{\textnormal{\emph{Goldmannſache}}}\Cendnote{\textnormal{Er ärgerte sich über das Feuilleton \emph{\textcolor{green}{Berliner Theater. ›König Oedipus‹ im Zirkus Schumann}} (\emph{\textcolor{green}{Neue Freie Presse}}, Nr. 16618,
                        26. 11. 1910, Morgenblatt, S. 1–3), vgl. A. S.: \emph{Tagebuch}, 1. 12. 1910}}}\label{K_L01989_3h} eine Unannehmlichkeit die hauptſächlich mich trifft, ſo ſtark fühlen, iſt mir
               unendlich woltuend, und für mich das einzig Reale an der läſtigen, aber eigentlich
                  \label{T_L01989_1v}\edtext{weſenloſen Angelegenheit}{\lemma{\textnormal{\emph{weſenloſen Angelegenheit}}}\Cendnote{\textnormal{ab hier weiter quer am linken
               Rand}}}\label{T_L01989_1h}.\pend
           \pstart Von Herzen Ihr\spacefill\mbox{Hugo.}\pend{}\endnumbering\briefempfaengerindex{Schnitzler, Arthur@\textsc{Schnitzler, Arthur}!zzzHofmannsthal, Hugo von@\emph{von Hugo von Hofmannsthal}!1910-12-021@{2. 12. 1910}|)be}\mylabel{h}  \normalsize

\doendnotes{C}
\bigskip
\vfill

\clearpage

\footnotesize

\lohead{\textsc{register}}

% Definiere theindex-Environment komplett neu ohne reledmac
\makeatletter
\renewenvironment{theindex}{%
  \section*{\indexname}%
  \setlength{\parindent}{0pt}%
  \setlength{\parskip}{0pt plus 0.3pt}%
  \let\item\@idxitem
}{%
  \clearpage
}
\makeatother

\IfFileExists{\jobname-pw.ind}{\input{\jobname-pw.ind}}{}

\end{document}

      