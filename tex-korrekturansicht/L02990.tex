%% latex-korrekturansicht-vorspann.tex
%% Vorspann für die Korrekturansicht.
%% Lädt die gemeinsame Datei latex-vorspann.tex mit gesetztem Schalter.

\newif\ifkorrekturansicht
\korrekturansichttrue

\input{../tex-inputs/latex-vorspann}


\renewcommand{\erwaehntePersonen}{Personen: Felix Salten}
\renewcommand{\erwaehnteOrte}{Orte: Wien}
\renewcommand{\erwaehnteWerke}{}
\section[ Arthur Schnitzler an Felix Salten, 12. 11. 1903]{Arthur Schnitzler an Felix Salten, 12. 11. 1903}
\nopagebreak\mylabel{v}
\rehead{ }\normalsize\beginnumbering\briefempfaengerindex{Salten, Felix@\textsc{Salten, Felix}!zzzSchnitzler, Arthur@\emph{von Arthur Schnitzler}!1903-11-121@{12. 11. 1903}|(be}
\toendnotes[C]{\smallbreak\pagebreak[2]}\Standort{Wienbibliothek im Rathaus, ZPH 1681, 2.1.516.}
\physDesc{Brief, 1 Blatt, 1 Seite, 257 Zeichen
\newline{}Handschrift: Bleistift, deutsche Kurrent
\newline{}Ordnung: mit Bleistift von unbekannter Hand Nummerierung der Blätter des Konvoluts:
                                    »54« }
\buchAbdrucke{\weitereDrucke{Arthur Schnitzler: \emph{Briefe 1875–1912}. Hg. Therese Nickl und Heinrich Schnitzler. Frankfurt am Main: \emph{S. Fischer} 1981, S. 475–476.} }\toendnotes[C]{\smallbreak}
\pstart
           \raggedleft{}{\pb}\textcolor{pink}{Wien}{}\ledrightnote{\textcolor{pink}{Wien}}{ }12. 11. 903.\pend
           
\pstart
           lieber, ich denke wir ſind beide um eine Kleinigkeit empfindlicher
               als dringend notwendig ist{\dots} und wären es weniger, we{\geminationn} einer von uns der Gedanke unſchmerzlich wäre, den
               andern zu verlieren. Auf Wiederſehen \label{K_L02990-1v}\edtext{heute{ }Nachmittag}{\lemma{\textnormal{\emph{heute Nachmittag}}}\Cendnote{\textnormal{siehe A. S.: \emph{Tagebuch}, 12. 11. 1903}}}\label{K_L02990-1h}. Herzlichſt Ihr\pend
           \pstart \spacefill\mbox{A.}\pend{}\endnumbering\briefempfaengerindex{Salten, Felix@\textsc{Salten, Felix}!zzzSchnitzler, Arthur@\emph{von Arthur Schnitzler}!1903-11-121@{12. 11. 1903}|)be}\mylabel{h}  \normalsize

\doendnotes{C}
\bigskip
\vfill

\clearpage

\footnotesize

\lohead{\textsc{register}}

% Definiere theindex-Environment komplett neu ohne reledmac
\makeatletter
\renewenvironment{theindex}{%
  \section*{\indexname}%
  \setlength{\parindent}{0pt}%
  \setlength{\parskip}{0pt plus 0.3pt}%
  \let\item\@idxitem
}{%
  \clearpage
}
\makeatother

\IfFileExists{\jobname-pw.ind}{\input{\jobname-pw.ind}}{}

\end{document}

      