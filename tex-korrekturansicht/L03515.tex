%% latex-korrekturansicht-vorspann.tex
%% Vorspann für die Korrekturansicht.
%% Lädt die gemeinsame Datei latex-vorspann.tex mit gesetztem Schalter.

\newif\ifkorrekturansicht
\korrekturansichttrue

\input{../tex-inputs/latex-vorspann}


\renewcommand{\erwaehntePersonen}{Personen: Franziska Goldmann, Eva Marie Goldmann}
\renewcommand{\erwaehnteInstitutionen}{Institutionen: Neue Freie Presse, Phaidon-Verlag, Österreichische Journal A.G.}
\renewcommand{\erwaehnteOrte}{Orte: Akademietheater, Bendlerstraße, Berlin, Wien}
\renewcommand{\erwaehnteWerke}{Werke: Buch der Sprüche und Bedenken, Es ist mein Wille{\rufezeichen} Eine unwahrscheinliche Begebenheit aus dem 18. Jahrhundert in einem Akt}
\section[ Paul Goldmann an Arthur Schnitzler, 23. 12. 1927]{Paul Goldmann an Arthur Schnitzler, 23. 12. 1927}
\nopagebreak\mylabel{v}
\rehead{ }\normalsize\beginnumbering\briefempfaengerindex{Schnitzler, Arthur@\textsc{Schnitzler, Arthur}!zzzGoldmann, Paul@\emph{von Paul Goldmann}!1927-12-231@{23. 12. 1927}|(be}
\toendnotes[C]{\smallbreak\pagebreak[2]}\Standort{DLA, A:Schnitzler, HS.NZ85.1.3176.}
\physDesc{Brief, 1 Blatt, 1 Seite, 886 Zeichen
\newline{}Schreibmaschine
\newline{}Handschrift: lila Tinte, lateinische Kurrent (\noindent{}zwei Korrekturen und Unterschrift)
\newline{}Schnitzler: mit rotem Buntstift »Aph{[}orismen{]}« vermerkt und vier
                                 Unterstreichungen }\toendnotes[C]{\smallbreak}
\pstart
           \noindent{}{\pb}\textcolor{gray}{\textbf{Dr. Paul Goldmann}}\hfill \textcolor{gray}{\textbf{\textcolor{pink}{Berlin W. 10}{}\ledrightnote{\textcolor{pink}{Berlin}}}}\pend
           
\pstart
           \textcolor{gray}{\textbf{Vertreter der »\textcolor{brown}{Neuen Freien
                           Presse}{}\ledrightnote{\textcolor{brown}{Neue Freie Presse}}«}}\hfill \textcolor{gray}{\textbf{\textcolor{pink}{Bendlerſtraße 36}{}\ledrightnote{\textcolor{pink}{Bendlerstraße}}.}}\pend
           
\pstart
           \raggedleft{}\textcolor{gray}{\textbf{Tel. Lützow 9142}}\pend
           
\pstart
           \raggedleft{}23. 12. 27.\pend
           
\pstart\center{}Lieber Arthur,\pend
\pstart
           In unser aller Namen danke ich Dir herzlichst für Dein neues \label{K_L03515-1v}\edtext{\textcolor{green}{Buch}{}\ledrightnote{{$\rightarrow$}\textcolor{green}{Buch der Sprüche und Bedenken}}}{\lemma{\textnormal{\emph{Buch}}}\Cendnote{\textnormal{Die Aphorismensammlung \emph{\textcolor{green}{Buch der Sprüche und Bedenken}} war am 17. 10. 1927 im \textcolor{pink}{Wien}er \emph{\textcolor{brown}{Phaidon-Verlag}}
                  erschienen.}}}\label{K_L03515-1h}. Einiges von \substVorne{}\textsuperscript{D}\substDazwischen{}s\substHinten{}einem Inhalt kenne ich bereits aus Zeitungen und Zeitschriften, das übrige
               freue ich mich, im \textcolor{green}{Buche}{}\ledrightnote{{$\rightarrow$}\textcolor{green}{Buch der Sprüche und Bedenken}} zu
               lesen. Meine \textcolor{blue}{Tochter}{}\ledrightnote{{$\rightarrow$}\textcolor{blue}{Franziska Goldmann}} ist
               bereits in Deine Spruchweisheit vertieft, – während der Feiertage werde ich \substVorne{}\textsuperscript{I}\substDazwischen{}i\substHinten{}hr das \textcolor{green}{Buch}{}\ledrightnote{{$\rightarrow$}\textcolor{green}{Buch der Sprüche und Bedenken}}
               entreissen. Es war sehr lieb von Dir, dass Du unser gedacht hast.\pend
           
\pstart
           Infolge der \label{K_L03515-2v}\edtext{Verschiebung der \textcolor{green}{Première}{}\ledrightnote{{$\rightarrow$}\textcolor{green}{Es ist mein Wille! Eine unwahrscheinliche Begebenheit aus dem 18. Jahrhundert in einem Akt}} im \textcolor{pink}{Akademietheater}{}\ledrightnote{\textcolor{pink}{Akademietheater}}}{\lemma{\textnormal{\emph{Verschiebung … Akademietheater}}}\Cendnote{\textnormal{Die ursprünglich für Mitte Dezember 1927 angesetzte Uraufführung von \textcolor{blue}{Goldmann}s Einakter \emph{\textcolor{green}{Es ist mein Wille! Eine unwahrscheinliche Begebenheit aus
                     dem 18. Jahrhundert in einem Akt}} fand am 5. 1. 1928 im \textcolor{pink}{Wien}er \textcolor{pink}{Akademietheater} statt. Bereits 1924 war das \textcolor{green}{Stück} als Sonderdruck der \emph{\textcolor{brown}{Neuen Freien Presse}} in der \emph{\textcolor{brown}{Österreichischen Journal A. G.}} erschienen.}}}\label{K_L03515-2h} hat sich auch meine
               Reise nach \textcolor{pink}{Wien}{}\ledrightnote{\textcolor{pink}{Wien}} verschoben. Das \textcolor{green}{Stück}{}\ledrightnote{{$\rightarrow$}\textcolor{green}{Es ist mein Wille! Eine unwahrscheinliche Begebenheit aus dem 18. Jahrhundert in einem Akt}} soll angeblich Anfang Januar herauskommen, – ob ich dann werde meinen \textcolor{pink}{Berlin}{}\ledrightnote{\textcolor{pink}{Berlin}}er Posten verlassen können, ist noch
               ungewiss. Wenn ich nach \textcolor{pink}{Wien}{}\ledrightnote{\textcolor{pink}{Wien}} komme und wenn mein
               Aufenthalt nicht allzu kurz bemessen ist, werde ich Dich natürlich dort \label{K_L03515-3v}\edtext{wiedersehen}{\lemma{\textnormal{\emph{wiedersehen}}}\Cendnote{\textnormal{\textcolor{blue}{Schnitzler} besuchte die Aufführung von \emph{\textcolor{green}{Es ist mein Wille!}} am 8. 1. 1928. Er traf
                     \textcolor{blue}{Goldmann} auch am 10. 1. 1928, wo er ihm
                  mitteilte, dass ihm das \textcolor{green}{Stück} nicht gefallen hatte.}}}\label{K_L03515-3h}. Inzwischen wünsche ich Dir, auch im
               Namen von \textcolor{blue}{Frau}{}\ledrightnote{{$\rightarrow$}\textcolor{blue}{Eva Marie Goldmann}} und \textcolor{blue}{Tochter}{}\ledrightnote{{$\rightarrow$}\textcolor{blue}{Franziska Goldmann}}, frohe Feiertage und
               ein glückliches neues Jahr. Wir alle grüssen Dich
               herzlichst.\pend
           
\pstart
           {[}hs. Goldmann:{]} Dein {\\[\baselineskip]}\spacefill\mbox{Paul Goldmann.}\pend
           \leftskip=0em{}\endnumbering\briefempfaengerindex{Schnitzler, Arthur@\textsc{Schnitzler, Arthur}!zzzGoldmann, Paul@\emph{von Paul Goldmann}!1927-12-231@{23. 12. 1927}|)be}\mylabel{h}
\begin{anhang}
\end{anhang}\normalsize

\doendnotes{C}
\bigskip
\vfill

\clearpage

\footnotesize

\lohead{\textsc{register}}

% Definiere theindex-Environment komplett neu ohne reledmac
\makeatletter
\renewenvironment{theindex}{%
  \section*{\indexname}%
  \setlength{\parindent}{0pt}%
  \setlength{\parskip}{0pt plus 0.3pt}%
  \let\item\@idxitem
}{%
  \clearpage
}
\makeatother

\IfFileExists{\jobname-pw.ind}{\input{\jobname-pw.ind}}{}

\end{document}

      