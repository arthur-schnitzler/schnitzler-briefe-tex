%% latex-korrekturansicht-vorspann.tex
%% Vorspann für die Korrekturansicht.
%% Lädt die gemeinsame Datei latex-vorspann.tex mit gesetztem Schalter.

\newif\ifkorrekturansicht
\korrekturansichttrue

\input{../tex-inputs/latex-vorspann}


\section[Stefan Zweig an Arthur Schnitzler, {[}28. 5. 1913?{]}]{L03643 Stefan Zweig an Arthur Schnitzler, {[}28. 5. 1913?{]}}
\nopagebreak\mylabel{L03643v}
\rehead{ }\normalsize\beginnumbering\briefempfaengerindex{Schnitzler, Arthur@\textsc{Schnitzler, Arthur}!zzzZweig, Stefan@\emph{von Stefan Zweig}!1913-05-281@{{[}28. 5. 1913?{]}}|(be}
\toendnotes[C]{\smallbreak\pagebreak[2]}\Standort{CUL, Schnitzler, B 118.}
\physDesc{Bildpostkarte, 412 Zeichen
\newline{}Handschrift: blaue Tinte, lateinische Kurrent
\newline{}Versand: Stempel: »\nobreak{}\oindex{VIII., Josefstadt@\textbf{VIII., Josefstadt}, \emph{A.ADM3}|pwk}8/\textcolor{gray}{×} Wien, 28. {[}5. 1913{]}\nobreak{}«.  
\newline{}Schnitzler: mit rotem Buntstift eine Unterstreichung }
\buchAbdrucke{\weitereDrucke{1) Stefan Zweig: \emph{Briefwechsel mit Hermann Bahr, Sigmund Freud, Rainer Maria
                        Rilke und Arthur Schnitzler}. Frankfurt am Main: \emph{S. Fischer} 1987, S. 379.} \weitereDrucke{2) Hermann Bahr, Arthur Schnitzler: \emph{Briefwechsel, Aufzeichnungen, Dokumente (1891–1931)}. Göttingen: \emph{Wallstein} 2018, S. 487.} }\toendnotes[C]{\smallbreak}\pstart{}{\pb}D\textsuperscript{r} Artur
                  Schnitzler\pend{}\pstart{}\textcolor{pink}{Wien – Cottage}\oindex{Waehringer Cottage@\textbf{Währinger Cottage}, \emph{Teil eines besiedelten Ortes (A.BSOX)}|pw}{}\ledrightnote{\textcolor{pink}{Währinger Cottage}}\pend{}\pstart{}\textcolor{pink}{\label{K_L03643-1v}\edtext{Sternwartestrasse 72}{\lemma{\textnormal{\emph{Sternwartestrasse 72}}}\Cendnote{\textnormal{\textcolor{blue}{Zweig}\pwindex{Zweig, Stefan 28.11.1881 – 23.02.1942@\textsc{Zweig, Stefan} (28.11.1881 – 23.02.1942), \emph{Schriftsteller}|pwk} wechselt bei der Adressierung
                        seiner Schreiben an \textcolor{blue}{Schnitzler} immer
                        wieder zwischen der falschen Hausnummer »72« und der
                        richtigen »71«.}}}\label{K_L03643-1}}\oindex{Sternwartestrasse 71@\textbf{Sternwartestraße 71}, \emph{Wohngebäude (K.WHS)}|pw}{}\ledrightnote{\textcolor{pink}{Sternwartestraße 71}}\pend{}{\bigskip}
\pstart
           \noindent{}\centering{}{\pb}\textcolor{gray}{\textbf{\textcolor{pink}{Wien}\oindex{Wien@\textbf{Wien}, \emph{A.ADM2}|pw}{}\ledrightnote{\textcolor{pink}{Wien}}.}}\pend
           
\pstart
           \centering{}\textcolor{pink}{Kirche Maria am Gestade}\oindex{Maria am Gestade@\textbf{Maria am Gestade}, \emph{Kirche (K.KRC)}|pw}{}\ledrightnote{\textcolor{pink}{Maria am Gestade}}\pend
           
\pstart
           \centering{}\textcolor{gray}{\textbf{\textcolor{blue}{Erwin Pendl}\pwindex{Pendl, Erwin August 18.10.1875 – 04.08.1945@\textsc{Pendl, Erwin August} (18.10.1875 – 04.08.1945), \emph{Schriftsteller, Maler}|pw}{}\ledrightnote{\textcolor{blue}{Erwin August Pendl}} pinx.}}\pend
           \vspace{1em}
\pstart{}{\pb}Verehrter Herr Doktor,\pend\vspace{0.5em}
\pstart
           vielen Dank für Ihre \label{K_L03643-2v}\edtext{guten Worte}{\lemma{\textnormal{\emph{guten Worte}}}\Cendnote{\textnormal{Arthur Schnitzler an Stefan Zweig, 2[4?]. 5. 1913.}}}\label{K_L03643-2}. Meine \label{K_L03643-3v}\edtext{\textcolor{violet}{\textcolor{green}{\textcolor{blue}{Bahr}\pwindex{Bahr, Hermann 19.07.1863 – 15.01.1934@\textsc{Bahr, Hermann} (19.07.1863 – 15.01.1934), \emph{Schriftsteller, Kritiker}|pw}{}\ledrightnote{\textcolor{blue}{Hermann Bahr}}-Rede}\pwindex{Hermann Bahr, der Fuenfzigjaehrige. (Eine Rede im Akademischen Verband fuer Literatur)@\emph{Hermann Bahr, der Fünfzigjährige. (Eine Rede im Akademischen Verband für Literatur)}|pw}{}\ledrightnote{\textcolor{green}{Hermann Bahr, der Fünfzigjährige. (Eine Rede im Akademischen Verband für Literatur)}}}\eventindex{Elektrotechnisches Institut der Technischen Universitaet@\textbf{Elektrotechnisches Institut der Technischen Universität}!Hermann-Bahr-Feier, 26.5.1913@Hermann-Bahr-Feier, 26.5.1913|pwv}{}\ledrightnote{{$\rightarrow$}\emph{\textcolor{violet}{Hermann-Bahr-Feier, 26.5.1913}}}}{\lemma{\textnormal{\emph{Bahr-Rede}}}\Cendnote{\textnormal{\textcolor{blue}{Stefan Zweig}\pwindex{Zweig, Stefan 28.11.1881 – 23.02.1942@\textsc{Zweig, Stefan} (28.11.1881 – 23.02.1942), \emph{Schriftsteller}|pwk} hatte am
                     26. 5. 1913 aus Anlass von \textcolor{blue}{Hermann Bahrs}\pwindex{Bahr, Hermann 19.07.1863 – 15.01.1934@\textsc{Bahr, Hermann} (19.07.1863 – 15.01.1934), \emph{Schriftsteller, Kritiker}|pwk} 50. Geburtstag am 19. 5. 1913 eine \textcolor{violet}{Rede im \emph{\textcolor{brown}{Akademischen Verband für Literatur}\orgindex{Akademischer Verband fuer Literatur und Musik in Wien@Akademischer Verband für Literatur und Musik in Wien|pwk}}}\eventindex{Elektrotechnisches Institut der Technischen Universitaet@\textbf{Elektrotechnisches Institut der Technischen Universität}!Hermann-Bahr-Feier, 26.5.1913@Hermann-Bahr-Feier, 26.5.1913|pwkv}
                  gehalten.}}}\label{K_L03643-3}{ }\label{K_L03643-4v}\edtext{in der \textcolor{green}{N. F. P.}\pwindex{Neue Freie Presse@\emph{Neue Freie Presse}|pw}{}\ledrightnote{\textcolor{green}{Neue Freie Presse}}}{\lemma{\textnormal{\emph{in der N. F. P.}}}\Cendnote{\textnormal{\textcolor{blue}{Stefan Zweig}\pwindex{Zweig, Stefan 28.11.1881 – 23.02.1942@\textsc{Zweig, Stefan} (28.11.1881 – 23.02.1942), \emph{Schriftsteller}|pwk}: \emph{\textcolor{green}{Hermann Bahr, der Fünfzigjährige. (Eine Rede im Akademischen
                        Verband für Literatur)}\pwindex{Hermann Bahr, der Fuenfzigjaehrige. (Eine Rede im Akademischen Verband fuer Literatur)@\emph{Hermann Bahr, der Fünfzigjährige. (Eine Rede im Akademischen Verband für Literatur)}|pwk}}. In: \emph{\textcolor{brown}{Neue Freie
                        Presse}\orgindex{Neue Freie Presse@Neue Freie Presse|pwk}}, Nr. 17.513, 27. 5. 1913, Morgenblatt,
                     S. 1–3.}}}\label{K_L03643-4} war stark frisiert und geschoren, ich hoffe, dass sie in
               Wirklichkeit intensiver war und mehr von seinem Rytmus hatte. Ich würde mich sehr
               freuen, Sie im Juli sehen zu dürfen und wünsche Ihnen inzwischen für
                  \label{K_L03643-5v}\edtext{Ihre Fahrt}{\lemma{\textnormal{\emph{Ihre Fahrt}}}\Cendnote{\textnormal{Den Sommer verbrachte \textcolor{blue}{Arthur
                     Schnitzler} ab dem 24. 7. 1913 mit seiner \textcolor{blue}{Frau Olga}\pwindex{Schnitzler, Olga 17.01.1882 – 13.01.1970@\textsc{Schnitzler, Olga} (17.01.1882 – 13.01.1970), \emph{Schauspielerin, Sängerin}|pwk} und den Kindern \textcolor{blue}{Heinrich}\pwindex{Schnitzler, Heinrich 09.08.1902 – 12.07.1982@\textsc{Schnitzler, Heinrich} (09.08.1902 – 12.07.1982), \emph{Regisseur, Schauspieler}|pwk} und \textcolor{blue}{Lili}\pwindex{Cappellini, Lili 13.09.1909 – 26.07.1928@\textsc{Cappellini, Lili} (13.09.1909 – 26.07.1928)|pwk} auf \textcolor{pink}{Brioni}\oindex{Brijuni@\textbf{Brijuni}, \emph{P.PPL}|pwk} und reiste im Anschluss nach \textcolor{pink}{Venedig}\oindex{Venedig@\textbf{Venedig}, \emph{P.PPLA}|pwk}, \textcolor{pink}{Sankt
                     Moritz}\oindex{St. Moritz@\textbf{St. Moritz}, \emph{P.PPL}|pwk} und \textcolor{pink}{München}\oindex{Muenchen@\textbf{München}, \emph{P.PPLA}|pwk}, bevor er am
                     12. 9. 1913 nach \textcolor{pink}{Wien}\oindex{Wien@\textbf{Wien}, \emph{A.ADM2}|pwk}
                  zurückkehrte.}}}\label{K_L03643-5} alles Schöne.\pend
           \pstart Ihr ergebener \spacefill\mbox{Stefan Zweig}\pend{}\selectlanguage{ngerman}\endnumbering\briefempfaengerindex{Schnitzler, Arthur@\textsc{Schnitzler, Arthur}!zzzZweig, Stefan@\emph{von Stefan Zweig}!1913-05-281@{{[}28. 5. 1913?{]}}|)be}\mylabel{L03643h}  \normalsize

\doendnotes{C}
\bigskip
\vfill

\clearpage

\footnotesize

\lohead{\textsc{register}}

% Definiere theindex-Environment komplett neu ohne reledmac
\makeatletter
\renewenvironment{theindex}{%
  \section*{\indexname}%
  \setlength{\parindent}{0pt}%
  \setlength{\parskip}{0pt plus 0.3pt}%
  \let\item\@idxitem
}{%
  \clearpage
}
\makeatother

\IfFileExists{\jobname-pw.ind}{\input{\jobname-pw.ind}}{}

\end{document}

      