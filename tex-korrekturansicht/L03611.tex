%% latex-korrekturansicht-vorspann.tex
%% Vorspann für die Korrekturansicht.
%% Lädt die gemeinsame Datei latex-vorspann.tex mit gesetztem Schalter.

\newif\ifkorrekturansicht
\korrekturansichttrue

\input{../tex-inputs/latex-vorspann}


\renewcommand{\erwaehntePersonen}{Personen: Hermann Bahr, Felix Salten}
\renewcommand{\erwaehnteInstitutionen}{Institutionen: S. Fischer Verlag}
\renewcommand{\erwaehnteOrte}{Orte: Berlin, Wien}
\renewcommand{\erwaehnteWerke}{Werke: Marionetten. Drei Einakter}
\section[ Arthur Schnitzler: Widmungsexemplar Marionetten für Felix Salten, {[}23.?{]} 3. 1906]{Arthur Schnitzler: Widmungsexemplar Marionetten für Felix
               Salten, {[}23.?{]} 3. 1906}
\nopagebreak\mylabel{v}
\rehead{ }\normalsize\beginnumbering\briefempfaengerindex{Salten, Felix@\textsc{Salten, Felix}!zzzSchnitzler, Arthur@\emph{von Arthur Schnitzler}!1906-03-233@{{[}23.?{]} 3. 1906}|(be}
\toendnotes[C]{\smallbreak\pagebreak[2]}\Standort{Wienbibliothek im Rathaus, A-44844/2.Ex., DS-2019-4241.}
\physDesc{Widmung am Vorsatzblatt, 52 Zeichen
\newline{}Handschrift: schwarze Tinte, deutsche Kurrent}\toendnotes[C]{\smallbreak}
\pstart
           \noindent{}{\pb}Meinem lieben Felix Salten\pend
           \pstart \spacefill\mbox{Arthur Sch}\pend{}
\pstart
           \textcolor{pink}{Wien}{}\ledrightnote{\textcolor{pink}{Wien}}{ }\label{K_L03611-1v}\edtext{März 906}{\lemma{\textnormal{\emph{März 906}}}\Cendnote{\textnormal{Die Datierung folgt der Widmung an
                     \textcolor{blue}{Bahr}, 23. 3. 1906.}}}\label{K_L03611-1h}.\pend
           {\bigskip}
\pstart
           \noindent{}\centering{}{\pb}\textcolor{gray}{\textbf{\textcolor{green}{\so{MARIONETTEN}}{}\ledrightnote{\textcolor{green}{Marionetten. Drei Einakter}}}}\pend
           
\pstart
           \noindent{}\centering{}\textcolor{gray}{\textbf{Drei Einakter von}}{\\}\textcolor{gray}{\textbf{\so{Arthur Schnitzler}}}\pend
           {\bigskip}
\pstart
           \noindent{}\centering{}\textcolor{gray}{\textbf{\textcolor{brown}{\so{S. Fischer, Verlag}}{}\ledrightnote{\textcolor{brown}{S. Fischer Verlag}}\so{,{ }}\textcolor{pink}{\so{Berlin}}{}\ledrightnote{\textcolor{pink}{Berlin}}}}\pend
           
\pstart
           \noindent{}\centering{}\textcolor{gray}{\textbf{1906}}\pend
           \endnumbering\briefempfaengerindex{Salten, Felix@\textsc{Salten, Felix}!zzzSchnitzler, Arthur@\emph{von Arthur Schnitzler}!1906-03-233@{{[}23.?{]} 3. 1906}|)be}\mylabel{h}  \normalsize

\doendnotes{C}
\bigskip
\vfill

\clearpage

\footnotesize

\lohead{\textsc{register}}

% Definiere theindex-Environment komplett neu ohne reledmac
\makeatletter
\renewenvironment{theindex}{%
  \section*{\indexname}%
  \setlength{\parindent}{0pt}%
  \setlength{\parskip}{0pt plus 0.3pt}%
  \let\item\@idxitem
}{%
  \clearpage
}
\makeatother

\IfFileExists{\jobname-pw.ind}{\input{\jobname-pw.ind}}{}

\end{document}

      