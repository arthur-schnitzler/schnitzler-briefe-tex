%% latex-korrekturansicht-vorspann.tex
%% Vorspann für die Korrekturansicht.
%% Lädt die gemeinsame Datei latex-vorspann.tex mit gesetztem Schalter.

\newif\ifkorrekturansicht
\korrekturansichttrue

\input{../tex-inputs/latex-vorspann}


\renewcommand{\erwaehntePersonen}{Personen: Fedor Mamroth, Olga Schnitzler, Heinrich Schnitzler, Franz Joseph Österreicher}
\renewcommand{\erwaehnteOrte}{Orte: Bahnhof Trient, Brenta (Gebirge), Cima Tosa, Dolomiten, Engadin, Frankfurt am Main, Gotthardpass, Grand Hotel des Alpes, Imperial Hotel Trento, Italien, Madonna di Campiglio, Monte Spinale, Riva del Garda, Südtirol, Tirol, Trient, Wien}
\renewcommand{\erwaehnteWerke}{}
\section[ Paul Goldmann an Arthur Schnitzler, 31. 8. 1904]{Paul Goldmann an Arthur Schnitzler, 31. 8. 1904}
\nopagebreak\mylabel{v}
\rehead{ }\normalsize\beginnumbering\briefempfaengerindex{Schnitzler, Arthur@\textsc{Schnitzler, Arthur}!zzzGoldmann, Paul@\emph{von Paul Goldmann}!1904-08-311@{31. 8. 1904}|(be}
\toendnotes[C]{\smallbreak\pagebreak[2]}\Standort{DLA, A:Schnitzler, HS.NZ85.1.3174.}
\physDesc{Brief, 1 Blatt, 2 Seiten, 633 Zeichen
\newline{}Handschrift: schwarze Tinte, deutsche Kurrent
\newline{}Schnitzler: mit rotem Buntstift eine Unterstreichung }\toendnotes[C]{\smallbreak}
\pstart
           \noindent{}\raggedleft{}{\pb}\textcolor{gray}{\textbf{\textcolor{pink}{BRENTA-DOLOMITEN}{}\ledrightnote{\textcolor{pink}{Brenta (Gebirge)}} (\textcolor{pink}{CIMA TOSA}{}\ledrightnote{\textcolor{pink}{Cima Tosa}} 3176 M) VOM \textcolor{pink}{M. SPINALE}{}\ledrightnote{\textcolor{pink}{Monte Spinale}} 2021 M.}}\pend
           
\pstart
           \noindent{}\raggedleft{}\textcolor{gray}{\textbf{\textcolor{pink}{MADONNA DI CAMPIGLIO}{}\ledrightnote{\textcolor{pink}{Madonna di Campiglio}} 1553 MTR.}}\pend
           
\pstart
           \noindent{}\textcolor{gray}{\textbf{\textcolor{pink}{Imperial Hôtel Trento}{}\ledrightnote{\textcolor{pink}{Imperial Hotel Trento}}}}\pend
           
\pstart
           \textcolor{gray}{\textbf{\begin{otherlanguage}{french}vis-à-vis de la \textcolor{pink}{Gare}{}\ledrightnote{\textcolor{pink}{Bahnhof Trient}}\end{otherlanguage}.}}\hfill \textcolor{gray}{\textbf{Saison Juni–September.}}\pend
           
\pstart
           \textcolor{gray}{\textbf{\textsc{\textcolor{pink}{Trient}{}\ledrightnote{\textcolor{pink}{Trient}} (\textcolor{pink}{Südtirol}{}\ledrightnote{\textcolor{pink}{Südtirol}})}}}\pend
           
\pstart
           \textcolor{gray}{\textbf{MÊME MAISON:}}{\\}\textcolor{gray}{\textbf{\textsc{\textcolor{pink}{Grand Hôtel des Alpes}{}\ledrightnote{\textcolor{pink}{Grand Hotel des Alpes}}}}}{\\}\textcolor{gray}{\textbf{\textsc{\textcolor{pink}{Madonna di Campiglio}{}\ledrightnote{\textcolor{pink}{Madonna di Campiglio}}.}}}\pend
           
\pstart
           \textcolor{gray}{\textbf{\emph{\textcolor{blue}{F. J. Oesterreicher}{}\ledrightnote{\textcolor{blue}{Franz Joseph Österreicher}}, Prop\textsuperscript{re}}}}\pend
           
\pstart
           \raggedleft{}\textcolor{gray}{\textbf{\textbf{\textcolor{pink}{Madonna di Campiglio}{}\ledrightnote{\textcolor{pink}{Madonna di Campiglio}}},}}{\\}31. Auguſt \textcolor{gray}{\textbf{19}}04.\pend
           
\pstart\center{}Mein lieber Freund,\pend
\pstart
           Es regnet in \textsc{\textcolor{pink}{Campiglio}{}\ledrightnote{\textcolor{pink}{Madonna di Campiglio}}}, und ich fahre morgen nach \textsc{\textcolor{pink}{Riva}{}\ledrightnote{\textcolor{pink}{Riva del Garda}}}. Von da wahrſcheinlich weiter nach Ober\textcolor{pink}{italien}{}\ledrightnote{\textcolor{pink}{Italien}}. Das Nähere hängt von meinem \textcolor{blue}{Onkel}{}\ledrightnote{{$\rightarrow$}\textcolor{blue}{Fedor Mamroth}} ab, der aus dem \textcolor{pink}{Engadin}{}\ledrightnote{\textcolor{pink}{Engadin}} an einen von ihm noch zu beſtimmenden Ort kommt. Im \textcolor{pink}{Gebirge}{}\ledrightnote{{$\rightarrow$}\textcolor{pink}{Dolomiten}} werde ich Dich alſo wohl nicht
               ſehen können, – rathe Dir auch {\pb}dringend ab, nach \textcolor{pink}{Tirol}{}\ledrightnote{\textcolor{pink}{Tirol}{\newline}\textcolor{pink}{Südtirol}} zu kommen, ehe das Wetter ſich
               gebeſſert hat (wozu anſcheinend wenig Ausſicht.) Aber wenn ich über \textcolor{pink}{Wien}{}\ledrightnote{\textcolor{pink}{Wien}} zurückkehre (es iſt allerdings auch möglich, daß ich \textsc{\textcolor{pink}{Gotthardt}{}\ledrightnote{\textcolor{pink}{Gotthardpass}}}–\textcolor{pink}{Frankfurt}{}\ledrightnote{\textcolor{pink}{Frankfurt am Main}} fahre) hoffe ich ſehr, Dir dort
               die \label{K_L03454-1v}\edtext{Hand drücken}{\lemma{\textnormal{\emph{Hand drücken}}}\Cendnote{\textnormal{\textcolor{blue}{Goldmann} kehrte über \textcolor{pink}{Wien}
                  zurück, am
                     21. 9. 1904
                  besuchte er \textcolor{blue}{Schnitzler}.}}}\label{K_L03454-1h} zu können.\pend
           
\pstart
           Mit herzlichen Grüßen an Dich, \textcolor{blue}{Frau}{}\ledrightnote{{$\rightarrow$}\textcolor{blue}{Olga Schnitzler}} und \textcolor{blue}{Kind}{}\ledrightnote{{$\rightarrow$}\textcolor{blue}{Heinrich Schnitzler}} bin ich {\\[\baselineskip]}Dein getreuer {\\[\baselineskip]}\spacefill\mbox{Paul Goldmann.}\pend
           \leftskip=0em{}\endnumbering\briefempfaengerindex{Schnitzler, Arthur@\textsc{Schnitzler, Arthur}!zzzGoldmann, Paul@\emph{von Paul Goldmann}!1904-08-311@{31. 8. 1904}|)be}\mylabel{h}
\begin{anhang}
\end{anhang}\normalsize

\doendnotes{C}
\bigskip
\vfill

\clearpage

\footnotesize

\lohead{\textsc{register}}

% Definiere theindex-Environment komplett neu ohne reledmac
\makeatletter
\renewenvironment{theindex}{%
  \section*{\indexname}%
  \setlength{\parindent}{0pt}%
  \setlength{\parskip}{0pt plus 0.3pt}%
  \let\item\@idxitem
}{%
  \clearpage
}
\makeatother

\IfFileExists{\jobname-pw.ind}{\input{\jobname-pw.ind}}{}

\end{document}

      