%% latex-korrekturansicht-vorspann.tex
%% Vorspann für die Korrekturansicht.
%% Lädt die gemeinsame Datei latex-vorspann.tex mit gesetztem Schalter.

\newif\ifkorrekturansicht
\korrekturansichttrue

\input{../tex-inputs/latex-vorspann}


               \section[Richard Beer-Hofmann an Arthur Schnitzler, 31. 7. 1899]{ Richard Beer-Hofmann an Arthur Schnitzler, 31. 7. 1899}\nopagebreak\mylabel{v}\rehead{ }\normalsize\beginnumbering\briefempfaengerindex{Schnitzler, Arthur@\textsc{Schnitzler, Arthur}!zzzBeer-Hofmann, Richard@\emph{von Richard Beer-Hofmann}!1899-07-312@{31. 7. 1899}|(be} \toendnotes[C]{\smallbreak\pagebreak[2]} \Standort{CUL, Schnitzler, B 8.}
\physDesc{Bildpostkarte
\newline{}Handschrift: schwarze Tinte, lateinische Kurrent\newline{}Versand: 1) Stempel: »\nobreak{}\oindex{Seeboden@\textbf{Seeboden}, \emph{http://www.geonames.org/ontologyA.ADM3}|pwk}Seeboden, 1 8 \textcolor{gray}{99}\nobreak{}«.  2) Stempel: »\nobreak{}\oindex{Toblach@\textbf{Toblach}, \emph{Besiedelter Ort (A.BSO)}|pwk}Toblach Bhf, 2. 8. 99\nobreak{}«. \newline{}Ordnung: mit Bleistift von unbekannter Hand nummeriert: »135« }\buchAbdrucke{\weitereDrucke{Arthur Schnitzler, Richard Beer-Hofmann: \emph{Briefwechsel 1891–1931}. Hg. Konstanze Fliedl. Wien, Zürich: \emph{Europaverlag} 1992, S. 133.} }\toendnotes[C]{\smallbreak}\pstart{}{\pb}D\textsuperscript{r}
                  Arthur Schnitzler\pend{}\pstart{}\textcolor{pink}{Toblach}{}\ledrightnote{\textcolor{pink}{Toblach}}\pend{}\pstart{}\textcolor{pink}{Südbahnhôtel}{}\ledrightnote{\textcolor{pink}{Südbahnhotel}}\pend{}{\bigskip}\pstart
           \noindent{}\centering{}\textcolor{gray}{\textbf{{\pb}Menagerie-Allée im \textcolor{pink}{k. k.
                     Schloßgarten Schönbrunn}{}\ledrightnote{\textcolor{pink}{Schloß Schönbrunn}}}}\pend
           \pstart
           \raggedleft{}\textcolor{pink}{Seeboden}{}\ledrightnote{\textcolor{pink}{Seeboden}}{\\}31/VII. 1899\pend
           \pstart
           \strikeout{\textcolor{gray}{\textbf{Beſten Gruß aus \textcolor{pink}{Wien}{}\ledrightnote{\textcolor{pink}{Wien}}
                      sendet}}}\pend
           \pstart
           \textcolor{green}{Fertig}{}\ledrightnote{→\textcolor{green}{Der Tod Georgs}}.\pend
           \endnumbering\briefempfaengerindex{Schnitzler, Arthur@\textsc{Schnitzler, Arthur}!zzzBeer-Hofmann, Richard@\emph{von Richard Beer-Hofmann}!1899-07-312@{31. 7. 1899}|)be}\mylabel{h}  \normalsize

\doendnotes{C}
\bigskip
\vfill

\clearpage

\footnotesize

\lohead{\textsc{register}}

% Definiere theindex-Environment komplett neu ohne reledmac
\makeatletter
\renewenvironment{theindex}{%
  \section*{\indexname}%
  \setlength{\parindent}{0pt}%
  \setlength{\parskip}{0pt plus 0.3pt}%
  \let\item\@idxitem
}{%
  \clearpage
}
\makeatother

\IfFileExists{\jobname-pw.ind}{\input{\jobname-pw.ind}}{}

\end{document}

      