%% latex-korrekturansicht-vorspann.tex
%% Vorspann für die Korrekturansicht.
%% Lädt die gemeinsame Datei latex-vorspann.tex mit gesetztem Schalter.

\newif\ifkorrekturansicht
\korrekturansichttrue

\input{../tex-inputs/latex-vorspann}


               \section[Arthur Schnitzler an Richard Beer-Hofmann, 17. 12. 1897]{ Arthur Schnitzler an Richard Beer-Hofmann, 17. 12. 1897}\nopagebreak\mylabel{v}\rehead{ }\normalsize\beginnumbering\briefempfaengerindex{Beer-Hofmann, Richard@\textsc{Beer-Hofmann, Richard}!zzzSchnitzler, Arthur@\emph{von Arthur Schnitzler}!1897-12-171@{17. 12. 1897}|(be} \toendnotes[C]{\smallbreak\pagebreak[2]} \Standort{YCGL, MSS 31.}
\physDesc{Briefkarte, Umschlag
\newline{}Handschrift: Bleistift, deutsche Kurrent\newline{}Versand: 1) Stempel: »\nobreak{}\oindex{IX., Alsergrund@\textbf{IX., Alsergrund}, \emph{Bezirk (A.BZK)}|pwk}Wien 9/3, 17. 12. 97, 11–12V\nobreak{}«.  2) Stempel: »\nobreak{}\oindex{I., Innere Stadt@\textbf{I., Innere Stadt}, \emph{Bezirk (A.BZK)}|pwk}{\pb}Wien 1/1, 17/12 97, 1–2½N, Bestellt\nobreak{}«. }\buchAbdrucke{\weitereDrucke{Arthur Schnitzler, Richard Beer-Hofmann: \emph{Briefwechsel 1891–1931}. Hg. Konstanze Fliedl. Wien, Zürich: \emph{Europaverlag} 1992, S. 114.} }\toendnotes[C]{\smallbreak}\pstart{}{\pb}Herrn \textsc{Dr. Richard
                     Beer-Hofmann}\pend{}\pstart{}\textcolor{pink}{Wien}{}\ledrightnote{\textcolor{pink}{Wien}}\pend{}\pstart{}\textcolor{pink}{\textsc{I. Wollzeile 15}}{}\ledrightnote{\textcolor{pink}{Wollzeile}}.\pend{}{\bigskip}\pstart
           \noindent{}{\pb}Lieber Richard, bitte ſenden Sie mir gelegentlich »\textcolor{green}{Die Todten ſchweigen}{}\ledrightnote{\textcolor{green}{Die Toten schweigen}}«.\pend
           \pstart Herzlichſt Ihr \spacefill\mbox{Arthur –}\pend{}\pstart
           \noindent{}(wiſſen Sie, der in der \textcolor{pink}{Frank{\pb}gaſſe}{}\ledrightnote{\textcolor{pink}{Frankgasse}} wohnt – gelegentlich auch bei Notaren
                     \label{K_L00750_1v}\edtext{Zeugenſchaft}{\lemma{\textnormal{\emph{Zeugenſchaft}}}\Cendnote{\textnormal{\textcolor{blue}{Schnitzler} war sowohl Zeuge für die am 4. 9. 1897 geborene Tochter
                     \textcolor{blue}{Mirjam} und Trauzeuge bei der Hochzeit von
                        \textcolor{blue}{Beer-Hofmann} und \textcolor{blue}{Paula Lissy} am 14. 5. 1898.}}}\label{K_L00750_1h} ablegt – der bekannte Arzt des \textcolor{blue}{Verfaſſer}{}\ledrightnote{→\textcolor{blue}{Leopold von Andrian-Werburg}}s des \textcolor{green}{Gartens der Erkenntnis}{}\ledrightnote{→\textcolor{green}{Der Garten der Erkenntnis}} – na,
                  Sie werden ſich ſchon erinnern.)\pend
           \endnumbering\briefempfaengerindex{Beer-Hofmann, Richard@\textsc{Beer-Hofmann, Richard}!zzzSchnitzler, Arthur@\emph{von Arthur Schnitzler}!1897-12-171@{17. 12. 1897}|)be}\mylabel{h}  \normalsize

\doendnotes{C}
\bigskip
\vfill

\clearpage

\footnotesize

\lohead{\textsc{register}}

% Definiere theindex-Environment komplett neu ohne reledmac
\makeatletter
\renewenvironment{theindex}{%
  \section*{\indexname}%
  \setlength{\parindent}{0pt}%
  \setlength{\parskip}{0pt plus 0.3pt}%
  \let\item\@idxitem
}{%
  \clearpage
}
\makeatother

\IfFileExists{\jobname-pw.ind}{\input{\jobname-pw.ind}}{}

\end{document}

      