%% latex-korrekturansicht-vorspann.tex
%% Vorspann für die Korrekturansicht.
%% Lädt die gemeinsame Datei latex-vorspann.tex mit gesetztem Schalter.

\newif\ifkorrekturansicht
\korrekturansichttrue

\input{../tex-inputs/latex-vorspann}


         
         \renewcommand{\erwaehntePersonen}{Personen: Cesare Borgia, Gerhart Hauptmann, D. W. Schröder, William Shakespeare}
         \renewcommand{\erwaehnteInstitutionen}{Institutionen: Deutsches Theater Berlin}
         \renewcommand{\erwaehnteOrte}{Orte: Berlin, Bologna, Hotel Saxonia, Potsdamer Platz, Stresemannstraße, Tiergarten, Wien}
         \renewcommand{\erwaehnteWerke}{Werke: Der Schleier der Beatrice. Schauspiel in fünf Akten, Schluck und Jau}
               \section[ Paul Goldmann an Arthur Schnitzler, 11. 2. 1900]{Paul Goldmann an Arthur Schnitzler, 11. 2. 1900}\nopagebreak\mylabel{v}\rehead{ }\normalsize\beginnumbering\briefempfaengerindex{Schnitzler, Arthur@\textsc{Schnitzler, Arthur}!zzzGoldmann, Paul@\emph{von Paul Goldmann}!1900-02-112@{11. 2. 1900}|(be} \toendnotes[C]{\smallbreak\pagebreak[2]} \Standort{DLA, A:Schnitzler, HS.NZ85.1.3170.}
\physDesc{Brief, 2 Blätter, 8 Seiten
\newline{}Handschrift: schwarze Tinte, deutsche Kurrent}\toendnotes[C]{\smallbreak}\pstart
           \noindent{}\centering{}{\pb}\textcolor{gray}{\textbf{\textbf{\textcolor{pink}{HOTEL SAXONIA}{}\ledrightnote{\textcolor{pink}{Hotel Saxonia}}}}}\pend
           \pstart
           \noindent{}\raggedleft{}\textcolor{gray}{\textbf{am \textcolor{pink}{Potsdamer Platz}{}\ledrightnote{\textcolor{pink}{Potsdamer Platz}} und
                     \textcolor{pink}{Thiergarten}{}\ledrightnote{\textcolor{pink}{Tiergarten}}.
                  }}\pend
           \pstart
           \noindent{}\centering{}\textcolor{gray}{\textbf{\textcolor{blue}{D. W. SCHRÖDER}{}\ledrightnote{\textcolor{blue}{D. W. Schröder}}.}}\pend
           \pstart
           \noindent{}\textcolor{gray}{\textbf{Fernsprecher:}}\pend
           \pstart
           \textcolor{gray}{\textbf{\textbf{Amt VI. No. 2838.}}}\pend
           \pstart
           \raggedleft{}\textcolor{gray}{\textbf{\textcolor{pink}{BERLIN W.}{}\ledrightnote{\textcolor{pink}{Berlin}}, den}}{ }11. Februar \textcolor{gray}{\textbf{1}}900.
               \pend
           \pstart
           \raggedleft{}\textcolor{gray}{\textbf{\textcolor{pink}{Königgrätzerstrasse 10}{}\ledrightnote{\textcolor{pink}{Stresemannstraße}}.}}\pend
           \pstart{}Mein lieber Freund,\pend\pstart
           Ich danke Dir von Herzen für Dein \textcolor{green}{Stück}{}\ledrightnote{{$\rightarrow$}\textcolor{green}{Der Schleier der Beatrice. Schauspiel in fünf Akten}}. In den Nächten, die auf die ſchwere Arbeit dieſer Tage folgten, habe
               ich es geleſen.\pend
           \pstart
           Ich glaube, es iſt das Bedeutendſte, was Du geſchrieben haſt. Die Sprache, Poeſie und
               Proſa, iſt prachtvoll. Die \textcolor{green}{Verſe}{}\ledrightnote{{$\rightarrow$}\textcolor{green}{Der Schleier der Beatrice. Schauspiel in fünf Akten}} namentlich find von einer goldenen Reife, – zum Theil von wunderbarer
               Schönheit. Und dabei ganz {\pb}Du ſelbſt. Kein Ton von
               einem Andern (Ich denke dabei an \textsc{\textcolor{blue}{Gerhart Hauptmann}{}\ledrightnote{\textcolor{blue}{Gerhart Hauptmann}}}, den ich erſt vor Kurzem gehört habe, wie er \label{K_L02904-1v}\edtext{\textsc{\textcolor{blue}{Shakespeare}{}\ledrightnote{\textcolor{blue}{William Shakespeare}}} nachſtammelte}{\lemma{\textnormal{\emph{Shakespeare nachſtammelte}}}\Cendnote{\textnormal{\textcolor{blue}{Goldmann} dürfte sich auf \textcolor{blue}{Hauptmann}s Komödie \emph{\textcolor{green}{Schluck
                        und Jau}}, die am 3. 2. 1900 am \emph{\textcolor{brown}{Deutschen Theater Berlin}} uraufgeführt worden
                  war und die von \textcolor{blue}{Shakespeare} inspiriert war.}}}\label{K_L02904-1h}.)\pend
           \pstart
           Was die Bühnenwirkung anlangt, ſo habe ich noch nie vor einem Drama ſo rathlos
               geſtanden. Vielleicht wird es mir bei längerem Nachdenken klarer. Denn ich bin eben
               erſt zu Ende. Es ſind \textcolor{green}{Szenen}{}\ledrightnote{{$\rightarrow$}\textcolor{green}{Der Schleier der Beatrice. Schauspiel in fünf Akten}}
               darin, die Einem ſchon beim Leſen den dramatiſchen Schauer geben, – die ergreifendſte
               iſt ſicherlich die zwiſchen \textsc{\textcolor{green}{Filippo}{}\ledrightnote{{$\rightarrow$}\textcolor{green}{Der Schleier der Beatrice. Schauspiel in fünf Akten}}} und \textsc{\textcolor{green}{Beatrice}{}\ledrightnote{{$\rightarrow$}\textcolor{green}{Der Schleier der Beatrice. Schauspiel in fünf Akten}}} am Schluß des dritten \textcolor{green}{Akt}{}\ledrightnote{{$\rightarrow$}\textcolor{green}{Der Schleier der Beatrice. Schauspiel in fünf Akten}}s. Aber einige \textcolor{green}{Charaktere}{}\ledrightnote{{$\rightarrow$}\textcolor{green}{Der Schleier der Beatrice. Schauspiel in fünf Akten}} verſtehe ich nicht. Und ich weiß nicht: werden ſie auf der Bühne,
               von bedeutenden Künſtlern {\pb}dargeſtellt, \strikeout{\textcolor{gray}{es}} erſt \strikeout{\textcolor{gray}{in}} zu Leben und Wahrheit erwachſen, oder werden ſie auf der Bühne erſt recht
               unbegreiflich ſcheinen, weil die feinen pſychologiſchen \textsc{Nuancen} auf dem Theater ſo
               gut wie unſichtbar \strikeout{\textcolor{gray}{wer}} werden? In dieſer Frage ruht, meiner Anſicht nach, die Frage der
               Bühnenwirkſamkeit des \textcolor{green}{Stück}{}\ledrightnote{{$\rightarrow$}\textcolor{green}{Der Schleier der Beatrice. Schauspiel in fünf Akten}}es.
               Und ich bin außer Stande, ſie zu beantworten.\pend
           \pstart
           Die \textsc{\textcolor{green}{Beatrice}{}\ledrightnote{{$\rightarrow$}\textcolor{green}{Der Schleier der Beatrice. Schauspiel in fünf Akten}}} verſtehe ich \strikeout{z\textcolor{gray}{×}\-\textcolor{gray}{×}} noch ganz gut. Kann die weibliche \label{K_L02904-2v}\edtext{\begin{otherlanguage}{french}\textsc{inconscience}\end{otherlanguage}}{\lemma{\textnormal{\emph{inconscience}}}\Cendnote{\textnormal{französisch: Gedankenlosigkeit, Unbewusstsein}}}\label{K_L02904-2h} ſo
               weit gehen? Ich würde es nicht für möglich halten, aber es wird durch das \strikeout{\textcolor{gray}{Dr}}{ }\textcolor{green}{Drama}{}\ledrightnote{{$\rightarrow$}\textcolor{green}{Der Schleier der Beatrice. Schauspiel in fünf Akten}} beinahe wahrſcheinlich.
               Ich beuge mich vor der Geſtaltungskraft des Dichters, obwohl im Grunde meines Herzens
               einige {\pb}Zweifel verbleiben. Aber den \textsc{\textcolor{green}{Filippo}{}\ledrightnote{{$\rightarrow$}\textcolor{green}{Der Schleier der Beatrice. Schauspiel in fünf Akten}}} verſtehe ich nicht. Wie? \strikeout{W\textcolor{gray}{enn}
                     di\textcolor{gray}{e}} Die Heißgeliebte und Heißerſehnte kommt, und man ſchickt ſie wieder weg –
               wegen eines Traumes? Wenn ich mein Mädchen \introOben{}heut\introOben{} in den Armen
               halte, kann ſie \strikeout{\textcolor{gray}{×}\-\textcolor{gray}{×}} geſtern geträumt haben, was ſie will. Und dann kommt ſie wieder, – kommt
               wieder aus dem Brautgemach des \textcolor{green}{Herzog}{}\ledrightnote{{$\rightarrow$}\textcolor{green}{Der Schleier der Beatrice. Schauspiel in fünf Akten}}s heraus. \textsc{\textcolor{green}{Filippo}{}\ledrightnote{{$\rightarrow$}\textcolor{green}{Der Schleier der Beatrice. Schauspiel in fünf Akten}}} will mit ihr ſterben. Sie hat Furcht vor dem Tode und will am Leben bleiben.
               Schön! Aber warum bringt \uline{er} ſich dann um? Sie iſt
               menſchlich und wahr. Und er ſieht das nicht ein, – er, der ein Dichter iſt? Man kann
               Jemanden immer noch ungeheuer lieb haben, ſelbſt wenn man nicht mit ihm ſterben will.
               Es geht {\pb}nun einmal nicht ſo leicht mit dem Sterben.
               Das Alles ſagt \textsc{\textcolor{green}{Filippo}{}\ledrightnote{{$\rightarrow$}\textcolor{green}{Der Schleier der Beatrice. Schauspiel in fünf Akten}}} ſelber mit den herrlichſten Worten. Und auf einmal bringt er ſich um. Weshalb?
               Ich kann es nicht begreifen. Und ich finde, wenn man ein ſchönes Liebchen hat, und
               wenn ſie in der Nacht zu Einem kommt, und wenn man nicht weiß, was morgen ſein wird,
               ſo greift man, weiß Gott, nicht zum Giftbecher. \strikeout{Ich
                  mag} Ich mag die jungen {\pb}Leute nicht, die
               ſich aus Pſychologie vergiften.\pend
           \pstart
           Auch den \textcolor{green}{Herzog}{}\ledrightnote{{$\rightarrow$}\textcolor{green}{Der Schleier der Beatrice. Schauspiel in fünf Akten}} verſtehe ich
               nicht. Ich hätte ihn verſtanden, wenn die Trauung mit \textsc{\textcolor{green}{Beatrice}{}\ledrightnote{{$\rightarrow$}\textcolor{green}{Der Schleier der Beatrice. Schauspiel in fünf Akten}}}{ }\strikeout{\textcolor{gray}{die wirkl}ich} ein \label{K_L02904-3v}\edtext{Faſtnachts-Scherz}{\lemma{\textnormal{\emph{Faſtnachts-Scherz}}}\Cendnote{\textnormal{traditioneller Scherz zur Fastnacht (Fasching, Karneval)}}}\label{K_L02904-3h} geweſen wäre\substVorne{}\textsuperscript{,}\substDazwischen{}.\substHinten{} Aber ich begreife nicht, daß dieſer Renaiſſance-Despot ſentimental genug
               iſt, das Mädchen wirklich zu heirathen. \strikeout{\textcolor{gray}{Überhaupt iſt}{ }\textcolor{gray}{[4 Zeilen unleserlich{]} }}
                Gewiß, es iſt nur für eine Nacht, und man weiß nicht, was morgen ſein wird.
               Und doch hat er unverkennbar ſentimentale Anwandlungen, und die {\pb}paſſen nicht zum Bilde eines Mannes, der
               entſchloſſen iſt, das Leben in ſeiner Fülle zu genießen. \strikeout{\textcolor{gray}{×}\-\textcolor{gray}{×}\-\textcolor{gray}{×}\-\textcolor{gray}{×}\-\textcolor{gray}{×}\-\textcolor{gray}{×}\-\textcolor{gray}{×}\-\textcolor{gray}{×}\-\textcolor{gray}{×}\-\textcolor{gray}{×}\-\textcolor{gray}{×}\-\textcolor{gray}{×}\-\textcolor{gray}{×}\-\textcolor{gray}{×}\-\textcolor{gray}{×}}{ }\strikeout{\textcolor{gray}{×}\-\textcolor{gray}{×}\-\textcolor{gray}{×}\-\textcolor{gray}{×}\-\textcolor{gray}{×}\-\textcolor{gray}{×} ſei\textcolor{gray}{.}}\pend
           \pstart
           Bewundernswürdig aber iſt wieder die Fülle der \introOben{}andern\introOben{}
               Figuren, die \uline{Alle} leben, die \substVorne{}\textsuperscript{G}\substDazwischen{}g\substHinten{}roßen und die kleinen. Den \textsc{\textcolor{green}{Francesco}{}\ledrightnote{{$\rightarrow$}\textcolor{green}{Der Schleier der Beatrice. Schauspiel in fünf Akten}}} mag ich freilich auch nicht und es kommt mir vor, als ſei er nur da, damit ſich
               am Schluß doch noch Jemand finde, welcher die \textsc{\textcolor{green}{Beatrice}{}\ledrightnote{{$\rightarrow$}\textcolor{green}{Der Schleier der Beatrice. Schauspiel in fünf Akten}}} erſticht. Ob es unumgänglich iſt, \strikeout{\textcolor{gray}{da}} daß ſie erſtochen wird, iſt mir ebenfalls nicht klar.\pend
           \pstart
           Höchſt eindrucksvoll iſt es, wie ſich alle dieſe Ereigniſſe in der \uline{einen} Nacht zuſammendrängen und wie während {\pb}des \strikeout{g\textcolor{gray}{roß}} ganzen \textcolor{green}{Drama}{}\ledrightnote{{$\rightarrow$}\textcolor{green}{Der Schleier der Beatrice. Schauspiel in fünf Akten}}s \textsc{\textcolor{blue}{\textcolor{green}{Cesar Borgia}{}\ledrightnote{{$\rightarrow$}\textcolor{green}{Der Schleier der Beatrice. Schauspiel in fünf Akten}}}{}\ledrightnote{\textcolor{blue}{Cesare Borgia}}} vor den Thoren von \textsc{\textcolor{pink}{Bologna}{}\ledrightnote{\textcolor{pink}{Bologna}}} ſteht. Auch habe ich auf mancher Seite des \textcolor{green}{Buch}{}\ledrightnote{{$\rightarrow$}\textcolor{green}{Der Schleier der Beatrice. Schauspiel in fünf Akten}}es die Kraft und die Fülle der Zeit empfunden, in welche
               die Handlung verlegt iſt{\dotsfive}\pend
           \pstart
           Das ſind wenige, flüchtige Worte, – mit müdem und ſchmerzendem Kopfe geſchrieben.\pend
           \pstart
           Ich grüße Dich von Herzen {\\[\baselineskip]}Dein {\\[\baselineskip]}\spacefill\mbox{Paul Goldmann.}\pend
           \leftskip=0em{}\endnumbering\briefempfaengerindex{Schnitzler, Arthur@\textsc{Schnitzler, Arthur}!zzzGoldmann, Paul@\emph{von Paul Goldmann}!1900-02-112@{11. 2. 1900}|)be}\mylabel{h}  \normalsize

\doendnotes{C}
\bigskip
\vfill

\clearpage

\footnotesize

\lohead{\textsc{register}}

% Definiere theindex-Environment komplett neu ohne reledmac
\makeatletter
\renewenvironment{theindex}{%
  \section*{\indexname}%
  \setlength{\parindent}{0pt}%
  \setlength{\parskip}{0pt plus 0.3pt}%
  \let\item\@idxitem
}{%
  \clearpage
}
\makeatother

\IfFileExists{\jobname-pw.ind}{\input{\jobname-pw.ind}}{}

\end{document}

      