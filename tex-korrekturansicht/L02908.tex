%% latex-korrekturansicht-vorspann.tex
%% Vorspann für die Korrekturansicht.
%% Lädt die gemeinsame Datei latex-vorspann.tex mit gesetztem Schalter.

\newif\ifkorrekturansicht
\korrekturansichttrue

\input{../tex-inputs/latex-vorspann}


         
         \renewcommand{\erwaehntePersonen}{Personen: Caroline Burger,  Dante Alighieri, Karl Federn, Hugo von Hofmannsthal}
         \renewcommand{\erwaehnteOrte}{Orte: Berlin, Dessauer Straße, Wien}
         \renewcommand{\erwaehnteWerke}{Werke: Dante, Vorspiel zur Antigone des Sophokles}
               \section[ Paul Goldmann an Arthur Schnitzler, 24. 3. {[}1900{]}]{Paul Goldmann an Arthur Schnitzler, 24. 3. {[}1900{]}}\nopagebreak\mylabel{v}\rehead{ }\normalsize\beginnumbering\briefempfaengerindex{Schnitzler, Arthur@\textsc{Schnitzler, Arthur}!zzzGoldmann, Paul@\emph{von Paul Goldmann}!1900-03-241@{24. 3. {[}1900{]}}|(be} \toendnotes[C]{\smallbreak\pagebreak[2]} \Standort{DLA, A:Schnitzler, HS.NZ85.1.3170.}
\physDesc{Brief, 1 Blatt, 1 Seite
\newline{}Handschrift: blaue Tinte, deutsche Kurrent
\newline{}Schnitzler: 1) mit Bleistift das Jahr »{[}1{]}900« vermerkt  2) mit rotem Buntstift eine Unterstreichung}\toendnotes[C]{\smallbreak}\pstart
           \noindent{}{\pb}\textcolor{pink}{\textcolor{gray}{\textbf{DESSAUERSTRASSE 19}}}{}\ledrightnote{\textcolor{pink}{Dessauer Straße}}\pend
           \pstart
           \raggedleft{}\textcolor{pink}{Berlin}{}\ledrightnote{\textcolor{pink}{Berlin}}, 24. März.\pend
           \pstart\center{}Mein lieber Freund,\pend\pstart
           Ich danke Dir für die Überſendung des \label{K_L02908-3v}\edtext{\textsc{\textcolor{blue}{Hoffmannsthal}{}\ledrightnote{\textcolor{blue}{Hugo von Hofmannsthal}}}’ſchen \textcolor{green}{Vorſpiel}{}\ledrightnote{{$\rightarrow$}\textcolor{green}{Vorspiel zur Antigone des Sophokles}}s}{\lemma{\textnormal{\emph{Hoffmannsthal’ſchen Vorſpiels}}}\Cendnote{\textnormal{\textcolor{blue}{Hugo von Hofmannsthal} hatte \textcolor{blue}{Schnitzler} gebeten, sein \emph{\textcolor{green}{Vorspiel zur Antigone des Sophokles}} an \textcolor{blue}{Goldmann} zu übersenden. Vgl. Hugo von Hofmannsthal an Arthur Schnitzler,
               15. 3. [1900], Hugo August von Hofmannsthal an Arthur Schnitzler,
                    22. 3. 1900
                  und Arthur Schnitzler an Hugo von Hofmannsthal, 23. 3. 1900.}}}\label{K_L02908-3h}. Ich finde es
               abſcheulich.\pend
           \pstart
           Haſt Du meinen \label{K_L02908-1v}\edtext{Brief von \strikeout{\textcolor{gray}{×}}{ }vorgeſtern}{\lemma{\textnormal{\emph{Brief von  vorgeſtern}}}\Cendnote{\textnormal{Paul Goldmann an Arthur Schnitzler, 22. 3. [1900]}}}\label{K_L02908-1h} nicht erhalten?\pend
           \pstart
           Ich danke Dir für die \label{K_L02908-4v}\edtext{Mittheilung}{\lemma{\textnormal{\emph{Mittheilung}}}\Cendnote{\textnormal{Bezug unklar}}}\label{K_L02908-4h} der Äußerung der \textcolor{blue}{Frau \textsc{Bürger}}{}\ledrightnote{{$\rightarrow$}\textcolor{blue}{Caroline Burger}}, die mich ſehr gefreut hat.\pend
           \pstart
           Haſt Du die prachtvolle \label{K_L02908-2v}\edtext{\textcolor{green}{\textsc{\textcolor{blue}{Dante}{}\ledrightnote{\textcolor{blue}{Dante Alighieri}}}-Biographie}{}\ledrightnote{\textcolor{green}{Dante}} von \textsc{\textcolor{blue}{Federn}{}\ledrightnote{\textcolor{blue}{Karl Federn}}}}{\lemma{\textnormal{\emph{Dante-Biographie von Federn}}}\Cendnote{\textnormal{\textcolor{blue}{Schnitzler} las \textcolor{blue}{Karl Federn}s \textcolor{green}{\textcolor{blue}{Dante}-Biographie} (zuerst unter dem
                  Titel \emph{\textcolor{green}{Dante}} erschienen, später auch unter \emph{\textcolor{green}{Dante und seine Zeit}}) im Mai 1900 (vgl. Arthur Schnitzler an Georg Brandes, 3. 5. 1900).}}}\label{K_L02908-2h} ſchon geleſen?\pend
           \pstart
           Viele treue Grüße! {\\[\baselineskip]}Dein {\\[\baselineskip]}\spacefill\mbox{Paul Goldmann}\pend
           \leftskip=0em{}\endnumbering\briefempfaengerindex{Schnitzler, Arthur@\textsc{Schnitzler, Arthur}!zzzGoldmann, Paul@\emph{von Paul Goldmann}!1900-03-241@{24. 3. {[}1900{]}}|)be}\mylabel{h}  \normalsize

\doendnotes{C}
\bigskip
\vfill

\clearpage

\footnotesize

\lohead{\textsc{register}}

% Definiere theindex-Environment komplett neu ohne reledmac
\makeatletter
\renewenvironment{theindex}{%
  \section*{\indexname}%
  \setlength{\parindent}{0pt}%
  \setlength{\parskip}{0pt plus 0.3pt}%
  \let\item\@idxitem
}{%
  \clearpage
}
\makeatother

\IfFileExists{\jobname-pw.ind}{\input{\jobname-pw.ind}}{}

\end{document}

      