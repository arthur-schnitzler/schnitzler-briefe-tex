%% latex-korrekturansicht-vorspann.tex
%% Vorspann für die Korrekturansicht.
%% Lädt die gemeinsame Datei latex-vorspann.tex mit gesetztem Schalter.

\newif\ifkorrekturansicht
\korrekturansichttrue

\input{../tex-inputs/latex-vorspann}


               \section[Arthur Schnitzler an Michael Georg Conrad, 11. 3. 1891]{ Arthur Schnitzler an Michael Georg Conrad, 11. 3. 1891}\nopagebreak\mylabel{v}\rehead{ }\normalsize\beginnumbering\briefempfaengerindex{Conrad, Michael Georg@\textsc{Conrad, Michael Georg}!zzzSchnitzler, Arthur@\emph{von Arthur Schnitzler}!1891-03-111@{11. 3. 1891}|(be} \toendnotes[C]{\smallbreak\pagebreak[2]} \Standort{München, Monacensia, Schnitzler, Arthur A I/1.}
\physDesc{Brief, 1 Blatt, 2 Seiten
\newline{}Handschrift: schwarze Tinte, deutsche Kurrent}\toendnotes[C]{\smallbreak}\pstart
           \raggedleft{}{\pb}\textcolor{pink}{Wien}{}\ledrightnote{\textcolor{pink}{Wien}}, 11. März 1891\pend
           \pstart
           Erlauben Sie mir, ſehr verehrter Herr, Ihnen hiemit \label{K_L00009_1v}\edtext{\textcolor{green}{Alkandi’s Lied}{}\ledrightnote{\textcolor{green}{Alkandi’s Lied}}}{\lemma{\textnormal{\emph{Alkandi’s Lied}}}\Cendnote{\textnormal{Schnitzler hatte das \textcolor{green}{Stück} bereits im Herbst 1889
                  vollendet, vgl. A. S.: \emph{Tagebuch}, 15. 11. 1889}}}\label{K_L00009_1h},
               ein dramatiſches Gedicht zu überſenden. Vielleicht haben Sie einmal eine halbe
               Stunde, es durchzuleſen. Ihr Urtheil wäre mir ſehr werthvoll. Halten Sie das \textcolor{green}{Stück}{}\ledrightnote{→\textcolor{green}{Alkandi’s Lied}} für aufführbar? Kö{\geminationn}ten Sie mir rathen, es der \label{K_L00009_2v}\edtext{\textcolor{brown}{Münchner Bühne}{}\ledrightnote{\textcolor{brown}{Königliche Hof- und Nationaltheater München}}}{\lemma{\textnormal{\emph{Münchner Bühne}}}\Cendnote{\textnormal{Schnitzler bezieht sich auf das \emph{\textcolor{brown}{Kgl. Hof- und Nationaltheater und das Kgl.
                     Residenz-Theater}}; General-Intendant war \textcolor{blue}{Karl Freiherr von Perfall}; zur Beziehung \textcolor{blue}{Conrad}s zu den Königlichen Bühnen vgl. Michael Georg Conrad an Arthur Schnitzler, 28. 3. 1893}}}\label{K_L00009_2h} einzuſenden? Wie ſehr möchte ich Ihnen für eine kurze
               Beantwortung dieſer {\pb}Fragen danken!\pend
           \pstart
           In aufrichtiger Verehrung{\\[\baselineskip]}Ihr ſehr ergebener{\\[\baselineskip]}\spacefill\mbox{Dr. Arthur Schnitzler}\pend
           \leftskip=0em{}\pstart
           \noindent{}\textsc{\textcolor{pink}{Wien, I. Giselastraße 11}{}\ledrightnote{\textcolor{pink}{Wien}}.}\pend
           \endnumbering\briefempfaengerindex{Conrad, Michael Georg@\textsc{Conrad, Michael Georg}!zzzSchnitzler, Arthur@\emph{von Arthur Schnitzler}!1891-03-111@{11. 3. 1891}|)be}\mylabel{h}  \normalsize

\doendnotes{C}
\bigskip
\vfill

\clearpage

\footnotesize

\lohead{\textsc{register}}

% Definiere theindex-Environment komplett neu ohne reledmac
\makeatletter
\renewenvironment{theindex}{%
  \section*{\indexname}%
  \setlength{\parindent}{0pt}%
  \setlength{\parskip}{0pt plus 0.3pt}%
  \let\item\@idxitem
}{%
  \clearpage
}
\makeatother

\IfFileExists{\jobname-pw.ind}{\input{\jobname-pw.ind}}{}

\end{document}

      