%% latex-korrekturansicht-vorspann.tex
%% Vorspann für die Korrekturansicht.
%% Lädt die gemeinsame Datei latex-vorspann.tex mit gesetztem Schalter.

\newif\ifkorrekturansicht
\korrekturansichttrue

\input{../tex-inputs/latex-vorspann}


               \section[Paul Goldmann an Arthur Schnitzler, 5. 12. {[}1893{]}]{ Paul Goldmann an Arthur Schnitzler, 5. 12. {[}1893{]}}\nopagebreak\mylabel{v}\rehead{ }\normalsize\beginnumbering\briefempfaengerindex{Schnitzler, Arthur@\textsc{Schnitzler, Arthur}!zzzGoldmann, Paul@\emph{von Paul Goldmann}!1893-12-051@{5. 12. {[}1893{]}}|(be} \toendnotes[C]{\smallbreak\pagebreak[2]} \Standort{DLA, A:Schnitzler, HS.NZ85.1.3163.}
\physDesc{Brief, 1 Blatt, 4 Seiten
\newline{}Handschrift: schwarze Tinte, deutsche Kurrent
\newline{}Schnitzler: 1) mit Bleistift das Jahr »93« vermerkt 2) mit rotem Buntstift vier Unterstreichungen}\toendnotes[C]{\smallbreak}\pstart
           \noindent{}{\pb}\textcolor{gray}{\textbf{\textbf{\textcolor{brown}{Frankfurter Zeitung}{}\ledrightnote{\textcolor{brown}{Frankfurter Zeitung}}.}}}\pend
           \pstart
           \textcolor{gray}{\textbf{\textbf{(\textcolor{brown}{\begin{otherlanguage}{french}Gazette de Francfort\end{otherlanguage}}{}\ledrightnote{\textcolor{brown}{Frankfurter Zeitung}}.)}}}\pend
           \pstart
           \textcolor{gray}{\textbf{\begin{otherlanguage}{french}\textcolor{blue}{Directeur}{}\ledrightnote{→\textcolor{blue}{Leopold Sonnemann}}\end{otherlanguage}{ }\textbf{M. \textcolor{blue}{L. Sonnemann}{}\ledrightnote{\textcolor{blue}{Leopold Sonnemann}}.}}}\hfill \textsc{\textcolor{pink}{Paris}{}\ledrightnote{\textcolor{pink}{Paris}}}, 5. December.\pend
           \pstart
           \begin{otherlanguage}{french}\textcolor{gray}{\textbf{\textcolor{green}{Journal}{}\ledrightnote{\textcolor{green}{Frankfurter Zeitung}} politique, financier,}}\end{otherlanguage}\pend
           \pstart
           \begin{otherlanguage}{french}\textcolor{gray}{\textbf{commercial et litteraire.}}\end{otherlanguage}\pend
           \pstart
           \begin{otherlanguage}{french}\textcolor{gray}{\textbf{\textbf{Paraissant trois fois par jour}}}\end{otherlanguage}\pend
           \pstart
           \begin{otherlanguage}{french}\textcolor{gray}{\textbf{\textbf{Bureaux à \textcolor{pink}{Paris}{}\ledrightnote{\textcolor{pink}{Paris}}:}}}\end{otherlanguage}\pend
           \pstart
           \begin{otherlanguage}{french}\textcolor{gray}{\textbf{\textbf{\textcolor{pink}{rue Richelieu 75}{}\ledrightnote{\textcolor{pink}{rue Richelieu}}.}}}\end{otherlanguage}\pend
           \pstart\center{}Mein lieber Freund!\pend\pstart
           Nachdem ich bisher vergeblich auf die verſprochenen Kritiken oder wenigſtens auf eine
               briefliche Mittheilung über die \textsc{\textcolor{green}{Premièren}{}\ledrightnote{→\textcolor{green}{Das Märchen. Schauspiel in drei Aufzügen}}}-Eindrücke gewartet, habe ich mir das Nöthige von \textcolor{pink}{Frankfurt}{}\ledrightnote{\textcolor{pink}{Frankfurt am Main}} kommen laſſen und bitte Dich, Dich nun nicht mehr zu bemühen.\pend
           \pstart
           Wenn ich aus der Sammlung der Kritiken, die mir vorliegt, die dummen Jungen weglaſſe
               – \label{K_L02721-12v}\edtext{\introOben{}\textcolor{green}{Neue Freie Preſſe}{}\ledrightnote{→\textcolor{green}{Theater- und Kunstnachrichten [Uraufführung Das Märchen]}}\introOben{}}{\lemma{\textnormal{\emph{Neue Freie Preſſe}}}\Cendnote{\textnormal{[\emph{\textcolor{green}{Friedrich Schütz}}]: \emph{\textcolor{green}{Theater- und Kunstnachrichten}}. In: \emph{\textcolor{green}{Neue Freie Presse}}, Jg. 30, Nr. 10.518, 2. 12. 1893, S. 7.}}}\label{K_L02721-12h}{ }\label{K_L02721-23v}\edtext{\textcolor{green}{Neues Wiener Tagblatt}{}\ledrightnote{→\textcolor{green}{Theater, Kunst und Literatur [Uraufführung Das Märchen]}}}{\lemma{\textnormal{\emph{Neues Wiener Tagblatt}}}\Cendnote{\textnormal{\textcolor{blue}{l. h. [=Ludwig Held]}: \emph{\textcolor{green}{Theater, Kunst und Literatur}}. In: \emph{\textcolor{green}{Neues Wiener Tagblatt}}, Jg. 27, Nr. 333, 2. 12. 1893, S. 8.}}}\label{K_L02721-23h}, \label{K_L02721-56v}\edtext{\textcolor{green}{Volksblatt}{}\ledrightnote{→\textcolor{green}{Theater, Kunst und Literatur [Uraufführung Das Märchen]}}}{\lemma{\textnormal{\emph{Volksblatt}}}\Cendnote{\textnormal{\textcolor{blue}{H. P.}: \emph{\textcolor{green}{Theater, Kunst und Literatur}}. In: \emph{\textcolor{green}{Deutsches Volksblatt}}, Jg. 5, Nr. 1.768, 2. 12. 1893, S. 6–7.}}}\label{K_L02721-56h}, \label{K_L02721-78v}\edtext{\textcolor{green}{Vaterland}{}\ledrightnote{→\textcolor{green}{Theater und Kunst [Uraufführung Das Märchen]}}}{\lemma{\textnormal{\emph{Vaterland}}}\Cendnote{\textnormal{\textcolor{blue}{–r–}: \emph{\textcolor{green}{Theater und Kunst}}. In: \emph{\textcolor{green}{Das
                        Vaterland}}, Jg. 34, Nr. 333, 2. 12. 1893,
                     S. 7.}}}\label{K_L02721-78h}{ }\textsc{etc.} – und mich nur an {\pb}die Zurechnungsfähigen halte, wie \textsc{\label{K_L02721-44v}\edtext{\textcolor{green}{\textcolor{blue}{Uhl}{}\ledrightnote{\textcolor{blue}{Friedrich Uhl}}}{}\ledrightnote{→\textcolor{green}{Feuilleton. Theater [Uraufführung Das Märchen]}}}{\lemma{\textnormal{\emph{Uhl}}}\Cendnote{\textnormal{[\textcolor{blue}{Friedrich Uhl}]: \emph{\textcolor{green}{Feuilleton. Theater}}. In: \emph{\textcolor{green}{Wiener Abendpost. Beilage zur Wiener Zeitung}},
                        Jg. 190, Nr. 276, 2. 12. 1893,
                     S. 1–2.}}}\label{K_L02721-44h}}, \textsc{\label{K_L02721-55v}\edtext{\textcolor{green}{\textcolor{blue}{Bahr}{}\ledrightnote{\textcolor{blue}{Hermann Bahr}}}{}\ledrightnote{→\textcolor{green}{Das Märchen (Schauspiel in drei Aufzügen von Arthur Schnitzler)}}}{\lemma{\textnormal{\emph{Bahr}}}\Cendnote{\textnormal{\textcolor{blue}{Hermann Bahr}: \emph{\textcolor{green}{Das Märchen (Schauspiel in drei Aufzügen von Arthur
                           Schnitzler. Zum ersten Male aufgeführt am Deutschen Volkstheater den 1.
                           December)}}. In: \emph{\textcolor{green}{Deutsche
                        Zeitung}}, Jg. 23, Nr. 7.879, 2. 12. 1893,
                        Morgen-Ausgabe, S. 1–3.}}}\label{K_L02721-55h}} und \label{K_L02721-66v}\edtext{\textsc{\textcolor{green}{\textcolor{blue}{Brociner}{}\ledrightnote{\textcolor{blue}{Marco Brociner}}}{}\ledrightnote{→\textcolor{green}{»Das Märchen.« (Schauspiel in 3 Aufzügen von Arthur Schnitzler. Zum erstenmale im Deutschen Volkstheater aufgeführt am 1. Dezember.)}}}}{\lemma{\textnormal{\emph{Brociner}}}\Cendnote{\textnormal{\textcolor{blue}{Marco Brociner}: \emph{\textcolor{green}{»Das Märchen.« (Schauspiel in 3 Aufzügen von Arthur
                        Schnitzler. Zum erstenmale im Deutschen Volkstheater aufgeführt am 1.
                        Dezember.)}} In: \emph{\textcolor{green}{Wiener Tagblatt}},
                     Jg. XXXX, Nr. YYYY, 2. 12. 1893, S. 1–2.}}}\label{K_L02721-66h}, ſo finde
               ich, daß man Dich hier auch mehrfach mißverſteht, daß man Dir aber auch vielerlei
               Richtiges und Beherzigenswerthes ſagt. Beſonders \textsc{\textcolor{green}{\textcolor{blue}{Uhl}{}\ledrightnote{\textcolor{blue}{Friedrich Uhl}}}{}\ledrightnote{→\textcolor{green}{Feuilleton. Theater [Uraufführung Das Märchen]}}} halte ich für im Weſentlichen richtig urtheilend. Du erinnerſt Dich, wir haben
               oft im Streit gelegen, Du und ich, und ich meine noch heute, heute erſt recht, daß
               Deinem glänzenden Talent beim Produciren die Disciplin fehlt. Auch beim Produciren
               denkſt Du ein wenig zu ſehr an Dich und zu wenig an das Andere, an die Forderungen
               der Kunſtform. Du ſchreibſt Deinem Herzeleid zuliebe und nicht {\pb}dem Drama zuliebe. Das iſt falſch. Ich komme immer
               mehr dahinter, daß das Produciren ein Streben nach möglichſter Objectivirung ſein
               muß, am allermeiſten aber das dramatiſche Produciren. Ich habe das in \textsc{\textcolor{pink}{Paris}{}\ledrightnote{\textcolor{pink}{Paris}}} noch mehr gelernt, habe daraufhin das »\textcolor{green}{Märchen}{}\ledrightnote{\textcolor{green}{Das Märchen. Schauspiel in drei Aufzügen}}« nochmals geleſen und meine Ausſtellungen von früher noch mehr
               beſtätigt gefunden. Erinnere Dich auch, was ich Dir ſtets über den \label{K_L02721-98v}\edtext{dritten \textcolor{green}{Act}{}\ledrightnote{→\textcolor{green}{Das Märchen. Schauspiel in drei Aufzügen}}}{\lemma{\textnormal{\emph{dritten Act}}}\Cendnote{\textnormal{vgl. Paul Goldmann an Arthur Schnitzler, 12. 12. [1891], Paul Goldmann an Arthur Schnitzler, 18. 12. [1891]}}}\label{K_L02721-98h} geſagt! Im Allgemeinen aber denke ich, daß Du mit
               Deinem \textcolor{green}{Debüt}{}\ledrightnote{→\textcolor{green}{Das Märchen. Schauspiel in drei Aufzügen}} nicht unzufrieden
               ſein darfſt. Du biſt den Kennern ſignaliſirt; alle Leute, die es verſtehen, haben
               Dein großes {\pb}Talent erkannt; die dumme Bande
               Publicum wirſt Du jetzt raſch gewinnen. Aber jetzt ſofort weiter ſchreiben! Vieles
               lernen aus den drei zurechnungsfähigen \textcolor{green}{Kritiken}{}\ledrightnote{→\textcolor{green}{Feuilleton. Theater [Uraufführung Das Märchen]}{\newline}→\textcolor{green}{Das Märchen (Schauspiel in drei Aufzügen von Arthur Schnitzler)}{\newline}→\textcolor{green}{»Das Märchen.« (Schauspiel in 3 Aufzügen von Arthur Schnitzler. Zum erstenmale im Deutschen Volkstheater aufgeführt am 1. Dezember.)}}. Und ein Drama machen, keine
               Beichte, kein Tagebuch! Das koſtet nur eine Willensanſtrengung. Denn Du biſt, ich
               weiß es genau, ein Dramatiker allererſten Ranges. Mach’ auch einen \label{K_L02721-3v}\edtext{neuen Verſuch mit dem \textsc{\textcolor{green}{Alkandi}{}\ledrightnote{\textcolor{green}{Alkandi’s Lied}}}}{\lemma{\textnormal{\emph{neuen … Alkandi}}}\Cendnote{\textnormal{vgl. Ferdinand von Saar an Arthur Schnitzler, 5. 2. 1894 und A. S.: \emph{Tagebuch}, 8. 3. 1894}}}\label{K_L02721-3h}, nachdem Du vorher den Schluß verſtärk\substVorne{}\textsuperscript{t}\substDazwischen{}end\substHinten{} umgearbeitet haſt. An \textsc{\textcolor{blue}{Uhl}{}\ledrightnote{\textcolor{blue}{Friedrich Uhl}}} hatte ich geſchrieben, damit er Dich nicht \label{K_L02721-4v}\edtext{in der \textcolor{green}{\textcolor{green}{Frkf. Ztg.}{}\ledrightnote{\textcolor{green}{Frankfurter Zeitung}}}{}\ledrightnote{→\textcolor{green}{Wiener Brief}}}{\lemma{\textnormal{\emph{in der Frkf. Ztg.}}}\Cendnote{\textnormal{[\textcolor{blue}{Friedrich Uhl}]: \emph{\textcolor{green}{Wiener Brief}}. In:
                        \emph{\textcolor{green}{Frankfurter Zeitung}}, Jg. 37, Nr. XXXX,
                        DatumXXXX, S. XXXX.}}}\label{K_L02721-4h} etwa ſchlecht behandle. Ich
               glaube, er wer ganz anſtändig?\pend
           \pstart
           Treue Grüße! Dein \spacefill\mbox{P. G.}\pend
           \endnumbering\briefempfaengerindex{Schnitzler, Arthur@\textsc{Schnitzler, Arthur}!zzzGoldmann, Paul@\emph{von Paul Goldmann}!1893-12-051@{5. 12. {[}1893{]}}|)be}\mylabel{h}\begin{anhang}\end{anhang}\normalsize

\doendnotes{C}
\bigskip
\vfill

\clearpage

\footnotesize

\lohead{\textsc{register}}

% Definiere theindex-Environment komplett neu ohne reledmac
\makeatletter
\renewenvironment{theindex}{%
  \section*{\indexname}%
  \setlength{\parindent}{0pt}%
  \setlength{\parskip}{0pt plus 0.3pt}%
  \let\item\@idxitem
}{%
  \clearpage
}
\makeatother

\IfFileExists{\jobname-pw.ind}{\input{\jobname-pw.ind}}{}

\end{document}

      