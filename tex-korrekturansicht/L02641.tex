%% latex-korrekturansicht-vorspann.tex
%% Vorspann für die Korrekturansicht.
%% Lädt die gemeinsame Datei latex-vorspann.tex mit gesetztem Schalter.

\newif\ifkorrekturansicht
\korrekturansichttrue

\input{../tex-inputs/latex-vorspann}


               \section[Paul Goldmann an Arthur Schnitzler, 25. 6. 1889]{ Paul Goldmann an Arthur Schnitzler, 25. 6. 1889}\nopagebreak\mylabel{v}\rehead{ }\normalsize\beginnumbering\briefempfaengerindex{Schnitzler, Arthur@\textsc{Schnitzler, Arthur}!zzzGoldmann, Paul@\emph{von Paul Goldmann}!1889-06-251@{25. 6. 1889}|(be} \toendnotes[C]{\smallbreak\pagebreak[2]} \Standort{DLA, A:Schnitzler, HS.NZ85.1.3162.}
\physDesc{Brief, 1 Blatt, 2 Seiten
\newline{}Handschrift: blaue Tinte, deutsche Kurrent
\newline{}Schnitzler: mit rotem Buntstift eine Unterstreichung }\toendnotes[C]{\smallbreak}\pstart
           \noindent{}\centering{}{\pb}\textcolor{gray}{\textbf{\textbf{Adminiſtration: \textcolor{pink}{VII.
                           Seidengaſſe 7}{}\ledrightnote{\textcolor{pink}{Seidengasse}}} (\textcolor{brown}{Jos. Eberle {\kaufmannsund} Co.}{}\ledrightnote{\textcolor{brown}{Josef Eberle  Stein-, Buch und Musikaliendruckerei}})}}\pend
           \pstart
           \noindent{}\centering{}\textcolor{gray}{\textbf{\textcolor{brown}{An der Schönen Blauen Donau}{}\ledrightnote{\textcolor{brown}{An der schönen blauen Donau}}}}\pend
           \pstart
           \noindent{}\centering{}\textcolor{gray}{\textbf{Chef-Redacteur: Dr. \textcolor{blue}{F.
                        Mamroth}{}\ledrightnote{\textcolor{blue}{Fedor Mamroth}}. – Redaction: \textcolor{pink}{IX.,
                        Berggaſſe 31}{}\ledrightnote{\textcolor{pink}{Berggasse}}.}}\pend
           \pstart
           \raggedleft{}\textcolor{gray}{\textbf{\textcolor{pink}{Wien}{}\ledrightnote{\textcolor{pink}{Wien}}, den}}{ }25. Juni \textcolor{gray}{\textbf{18}}89.\pend
           \pstart\center{}Sehr geehrter Herr Doctor!\pend\pstart
           Herr \textsc{Dr. \textcolor{blue}{Spitzer}{}\ledrightnote{\textcolor{blue}{Alfred Spitzer}}}, der geſtern in \textcolor{pink}{Wien}{}\ledrightnote{\textcolor{pink}{Wien}} war, bittet Sie und mich, morgen,
                  Mittwoch, zu ihm nach \textcolor{pink}{Baden}{}\ledrightnote{\textcolor{pink}{Baden bei Wien}} zu kommen,
               und hat mich erſucht, Sie zu verſtändigen. Ich bitte Sie daher, mir freundlichſt
                  morgen im Laufe des Vormittags
               mittheilen zu {\pb}wollen, ob es Ihnen
               möglich iſt, morgen{ }Nachmittag mit mir \label{K_L02641-1v}\edtext{hinauszufahren}{\lemma{\textnormal{\emph{hinauszufahren}}}\Cendnote{\textnormal{aus dieser Zeit  ist kein Besuch bei \textcolor{blue}{Alfred Spitzer} 
                  nachweisbar}}}\label{K_L02641-1h}, und im bejahenden Falle Herrn \textsc{Dr. \textcolor{blue}{Spitzer}{}\ledrightnote{\textcolor{blue}{Alfred Spitzer}}} zu verſtändigen.\pend
           \pstart
           Ich empfehle mich Ihnen mit beſten Grüßen {\\[\baselineskip]}Hochachtungsvoll {\\[\baselineskip]}Ihr
               ergebener {\\[\baselineskip]}\spacefill\mbox{Dr. Goldmann}\pend
           \leftskip=0em{}\endnumbering\briefempfaengerindex{Schnitzler, Arthur@\textsc{Schnitzler, Arthur}!zzzGoldmann, Paul@\emph{von Paul Goldmann}!1889-06-251@{25. 6. 1889}|)be}\mylabel{h}  \normalsize

\doendnotes{C}
\bigskip
\vfill

\clearpage

\footnotesize

\lohead{\textsc{register}}

% Definiere theindex-Environment komplett neu ohne reledmac
\makeatletter
\renewenvironment{theindex}{%
  \section*{\indexname}%
  \setlength{\parindent}{0pt}%
  \setlength{\parskip}{0pt plus 0.3pt}%
  \let\item\@idxitem
}{%
  \clearpage
}
\makeatother

\IfFileExists{\jobname-pw.ind}{\input{\jobname-pw.ind}}{}

\end{document}

      