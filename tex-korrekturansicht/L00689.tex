%% latex-korrekturansicht-vorspann.tex
%% Vorspann für die Korrekturansicht.
%% Lädt die gemeinsame Datei latex-vorspann.tex mit gesetztem Schalter.

\newif\ifkorrekturansicht
\korrekturansichttrue

\input{../tex-inputs/latex-vorspann}


               \section[Richard Beer-Hofmann an Arthur Schnitzler, 22. 6. 1897]{ Richard Beer-Hofmann an Arthur Schnitzler,
               22. 6. 1897}\nopagebreak\mylabel{v}\rehead{ }\normalsize\beginnumbering\briefempfaengerindex{Schnitzler, Arthur@\textsc{Schnitzler, Arthur}!zzzBeer-Hofmann, Richard@\emph{von Richard Beer-Hofmann}!1897-06-221@{22. 6. 1897}|(be} \toendnotes[C]{\smallbreak\pagebreak[2]} \Standort{CUL, Schnitzler, B 8.}
\physDesc{Brief, 1 Blatt, 4 Seiten
\newline{}Handschrift: blauer Buntstift, lateinische Kurrent\newline{}Ordnung: mit Bleistift von unbekannter Hand nummeriert: »100« }\buchAbdrucke{\weitereDrucke{Arthur Schnitzler, Richard Beer-Hofmann: \emph{Briefwechsel 1891–1931}. Hg. Konstanze Fliedl. Wien, Zürich: \emph{Europaverlag} 1992, S. 110.} }\toendnotes[C]{\smallbreak}\pstart
           \centering{}{\pb}\textcolor{pink}{Ischl}{}\ledrightnote{\textcolor{pink}{Bad Ischl}}{ }22/VI 97\pend
           \pstart
           Lieber Arthur, sie haben meinen letzten Brief nicht
               beantwortet und ko{\geminationm}en daher wol sehr bald. Bitte
               besorgen Sie mir – ohne Nervosität Folgendes:\pend
           \pstart
           I. Eine Pincette – vernickelt oder {\pb}in Silber.\pend
           \pstart
           2.) Im Durchhaus in der \textcolor{pink}{Wollzeile}{}\ledrightnote{\textcolor{pink}{Wollzeile}} das auf den alten
                  \textcolor{pink}{Universitätsplatz}{}\ledrightnote{\textcolor{pink}{Universitätsplatz}} führt ist ein \textcolor{brown}{Tierhändler}{}\ledrightnote{→\textcolor{brown}{G. Findeis}}; dort kaufen Sie um circa \label{K_L00689_1v}\edtext{50 xr}{\lemma{\textnormal{\emph{50 xr}}}\Cendnote{\textnormal{50 Kreuzer}}}\label{K_L00689_1h} Vogelfutter für Wellenpapageie.\pend
           \pstart
           3.) Im Durchhaus \textcolor{pink}{Graben}{}\ledrightnote{\textcolor{pink}{Graben}}{ }\textcolor{pink}{Goldschmidt{\pb}gasse}{}\ledrightnote{\textcolor{pink}{Goldschmiedgasse}} die \label{K_L00689_2v}\edtext{Dinge}{\lemma{\textnormal{\emph{Dinge}}}\Cendnote{\textnormal{Kondome}}}\label{K_L00689_2h} die Sie auch dort kaufen.\pend
           \pstart
           \strikeout{4.) \strikeout{Wi} Im Verlag der »\textcolor{brown}{Wiener
                     Mode}{}\ledrightnote{\textcolor{brown}{Wiener Mode}}« ist ein \textcolor{green}{Pro}{}\ledrightnote{\textcolor{green}{Pro und Contra. Eine hygienische Studie über das Radfahren}}}
               überflüssig.\pend
           \pstart
           Ich bin da es viel regnet erst einmal auf der Strasse gefahren. Hoffe wenn Sie ko{\geminationm}en {\pb}öfters. \textcolor{blue}{Schwarzkopf}{}\ledrightnote{\textcolor{blue}{Gustav Schwarzkopf}} viele Grüße – ko{\geminationm}t er?\pend
           \pstart
           Auf Wiedersehen{\\[\baselineskip]}\spacefill\mbox{Richard}\pend
           \leftskip=0em{}\endnumbering\briefempfaengerindex{Schnitzler, Arthur@\textsc{Schnitzler, Arthur}!zzzBeer-Hofmann, Richard@\emph{von Richard Beer-Hofmann}!1897-06-221@{22. 6. 1897}|)be}\mylabel{h}  \normalsize

\doendnotes{C}
\bigskip
\vfill

\clearpage

\footnotesize

\lohead{\textsc{register}}

% Definiere theindex-Environment komplett neu ohne reledmac
\makeatletter
\renewenvironment{theindex}{%
  \section*{\indexname}%
  \setlength{\parindent}{0pt}%
  \setlength{\parskip}{0pt plus 0.3pt}%
  \let\item\@idxitem
}{%
  \clearpage
}
\makeatother

\IfFileExists{\jobname-pw.ind}{\input{\jobname-pw.ind}}{}

\end{document}

      