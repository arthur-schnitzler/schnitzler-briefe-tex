%% latex-korrekturansicht-vorspann.tex
%% Vorspann für die Korrekturansicht.
%% Lädt die gemeinsame Datei latex-vorspann.tex mit gesetztem Schalter.

\newif\ifkorrekturansicht
\korrekturansichttrue

\input{../tex-inputs/latex-vorspann}


\renewcommand{\erwaehntePersonen}{Personen: Olga Schnitzler}
\renewcommand{\erwaehnteOrte}{Orte: Berlin, Deutsches Theater Berlin, Edmund-Weiß-Gasse, Semmering, Wien}
\renewcommand{\erwaehnteWerke}{Werke: Der einsame Weg. Schauspiel in fünf Akten}
\section[ Paul Goldmann an Arthur Schnitzler, 18. 1. 1904]{Paul Goldmann an Arthur Schnitzler, 18. 1. 1904}
\nopagebreak\mylabel{v}
\rehead{ }\normalsize\beginnumbering\briefempfaengerindex{Schnitzler, Arthur@\textsc{Schnitzler, Arthur}!zzzGoldmann, Paul@\emph{von Paul Goldmann}!1904-01-181@{18. 1. 1904}|(be}
\toendnotes[C]{\smallbreak\pagebreak[2]}\Standort{DLA, A:Schnitzler, HS.NZ85.1.3174.}
\physDesc{Postkarte
\newline{}Handschrift: 1) blaue Tinte, deutsche Kurrent\hspace{1em}2) blaue Tinte, lateinische Kurrent (\noindent{}Adresse)\hspace{1em}
\newline{}Versand: Stempel: »\nobreak{}\oindex{Berlin@\textbf{Berlin}, \emph{https://www.geonames.org/ontologyP.PPLC}|pwk}Berlin, W. 10, 16. 1. 04, 12–1N.\nobreak{}«. Stempel: »\nobreak{}18/1 Wien 110, 18. 1. 1904, 8.V, Bestellt\nobreak{}«.  
\newline{}Schnitzler: mit Bleistift das Jahr »{[}1{]}904« vermerkt }\toendnotes[C]{\smallbreak}\pstart{}{\pb}Herrn\pend{}\pstart{}Dr. Arthur Schnitzler\pend{}\pstart{}\textcolor{pink}{Wien}{}\ledrightnote{\textcolor{pink}{Wien}}\pend{}\pstart{}\textcolor{pink}{XVIII. Spöttelgaſse 7}{}\ledrightnote{\textcolor{pink}{Edmund-Weiß-Gasse}}.\pend{}
{\bigskip}
\pstart
           \noindent{}{\pb}\textcolor{pink}{Berlin}{}\ledrightnote{\textcolor{pink}{Berlin}}, 16. Januar. Herzlichſten Dank Dir, mein lieber
                  Freund, und Deiner \textcolor{blue}{Frau}{}\ledrightnote{{$\rightarrow$}\textcolor{blue}{Olga Schnitzler}} für Eure Grüße vom \label{K_L03439-1v}\edtext{\textcolor{pink}{Semmering}{}\ledrightnote{\textcolor{pink}{Semmering}}}{\lemma{\textnormal{\emph{Semmering}}}\Cendnote{\textnormal{\textcolor{blue}{Arthur und Olga Schnitzler} waren
                  zwischen 9. 1. 1904
                  und 14. 1. 1904 am
                     \textcolor{pink}{Semmering} gewesen.}}}\label{K_L03439-1h}. Hoffentlich
               bleibt es bei Eurer \label{K_L03439-2v}\edtext{\textcolor{pink}{Berlin}{}\ledrightnote{\textcolor{pink}{Berlin}}er Reiſe im
               Februar}{\lemma{\textnormal{\emph{Berliner … Februar}}}\Cendnote{\textnormal{Für die Uraufführung von \emph{\textcolor{green}{Der
                     einsame Weg}} am 13. 2. 1904 im \textcolor{pink}{Deutschen Theater
                           Berlin} war \textcolor{blue}{Schnitzler} von 5. 2. 1904 bis 17. 2. 1904 in \textcolor{pink}{Berlin}. Am 8. 2. 1904 kam \textcolor{blue}{Olga Schnitzler} nach. \textcolor{blue}{Goldmann} und
                  \textcolor{blue}{Schnitzler} trafen sich am 6. 2. 1904, 10. 2. 1904, 11. 2. 1904 und am 16. 2. 1904.}}}\label{K_L03439-2h}. Ich freue mich ſehr, Euch hier zu ſehen.
               Entſchuldige, daß ich Deinen letzten Brief noch nicht beantwortet habe. Ich erſaufe
               in Arbeit.\pend
           
\pstart
           Herzlichſt Dein getreuer {\\[\baselineskip]}\spacefill\mbox{Paul Goldmann}\pend
           \leftskip=0em{}\endnumbering\briefempfaengerindex{Schnitzler, Arthur@\textsc{Schnitzler, Arthur}!zzzGoldmann, Paul@\emph{von Paul Goldmann}!1904-01-181@{18. 1. 1904}|)be}\mylabel{h}  \normalsize

\doendnotes{C}
\bigskip
\vfill

\clearpage

\footnotesize

\lohead{\textsc{register}}

% Definiere theindex-Environment komplett neu ohne reledmac
\makeatletter
\renewenvironment{theindex}{%
  \section*{\indexname}%
  \setlength{\parindent}{0pt}%
  \setlength{\parskip}{0pt plus 0.3pt}%
  \let\item\@idxitem
}{%
  \clearpage
}
\makeatother

\IfFileExists{\jobname-pw.ind}{\input{\jobname-pw.ind}}{}

\end{document}

      