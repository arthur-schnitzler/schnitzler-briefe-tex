%% latex-korrekturansicht-vorspann.tex
%% Vorspann für die Korrekturansicht.
%% Lädt die gemeinsame Datei latex-vorspann.tex mit gesetztem Schalter.

\newif\ifkorrekturansicht
\korrekturansichttrue

\input{../tex-inputs/latex-vorspann}


\renewcommand{\erwaehntePersonen}{Personen: Victor Klemperer}
\renewcommand{\erwaehnteOrte}{Orte: Berlin, Schöneberger Ufer, Wien}
\renewcommand{\erwaehnteWerke}{}
\section[ Paul Goldmann an Arthur Schnitzler, 19. 4. 1910]{Paul Goldmann an Arthur Schnitzler, 19. 4. 1910}
\nopagebreak\mylabel{v}
\rehead{ }\normalsize\beginnumbering\briefempfaengerindex{Schnitzler, Arthur@\textsc{Schnitzler, Arthur}!zzzGoldmann, Paul@\emph{von Paul Goldmann}!1910-04-191@{19. 4. 1910}|(be}
\toendnotes[C]{\smallbreak\pagebreak[2]}\Standort{DLA, A:Schnitzler, HS.NZ85.1.3175.}
\physDesc{Brief, 1 Blatt, 1 Seite, 481 Zeichen
\newline{}Handschrift: blaue Tinte, lateinische Kurrent}\toendnotes[C]{\smallbreak}
\pstart
           \raggedleft{}{\pb}19. 4. 10\pend
           
\pstart
           \raggedleft{}\textcolor{gray}{\textbf{\textcolor{pink}{W. Schöneberger-Ufer 34}{}\ledrightnote{\textcolor{pink}{Schöneberger Ufer}}.}}\pend
           
\pstart{}Lieber Freund,\pend
\pstart
           Herr \textcolor{blue}{Victor Klemperer}{}\ledrightnote{\textcolor{blue}{Victor Klemperer}}, der Dir aus seinen
               vortrefflichen literarischen Essais ja schon literarisch bekannt ist, möchte Dich
               auch persönlich \label{K_L03470-1v}\edtext{kennen lernen}{\lemma{\textnormal{\emph{kennen lernen}}}\Cendnote{\textnormal{\textcolor{blue}{Schnitzler} und \textcolor{blue}{Victor Klemperer} lernten sich folglich am 27. 4. 1910 kennen.
                  Auch wenn \textcolor{blue}{Klemperer} über \textcolor{blue}{Schnitzler} schrieb, verfasste er nie ein Buch über
                  ihn.}}}\label{K_L03470-1h} und hat mich um eine Einführung bei Dir gebeten, die ich ihm mit
               Vergnügen gebe. Ich höre, daß er beabsichtigt, auch ein Buch über Dich zu schreiben,
               und ich würde mich freuen, wenn dieses Werk zustande käme. Ich bitte Dich, Herrn \textcolor{blue}{Klemperer}{}\ledrightnote{\textcolor{blue}{Victor Klemperer}} freundlich aufzunehmen, und begrüße
               Dich herzlich.\pend
           
\pstart
           Dein {\\[\baselineskip]}\spacefill\mbox{Paul Goldmann.}\pend
           \leftskip=0em{}\endnumbering\briefempfaengerindex{Schnitzler, Arthur@\textsc{Schnitzler, Arthur}!zzzGoldmann, Paul@\emph{von Paul Goldmann}!1910-04-191@{19. 4. 1910}|)be}\mylabel{h}
\begin{anhang}
\end{anhang}\normalsize

\doendnotes{C}
\bigskip
\vfill

\clearpage

\footnotesize

\lohead{\textsc{register}}

% Definiere theindex-Environment komplett neu ohne reledmac
\makeatletter
\renewenvironment{theindex}{%
  \section*{\indexname}%
  \setlength{\parindent}{0pt}%
  \setlength{\parskip}{0pt plus 0.3pt}%
  \let\item\@idxitem
}{%
  \clearpage
}
\makeatother

\IfFileExists{\jobname-pw.ind}{\input{\jobname-pw.ind}}{}

\end{document}

      