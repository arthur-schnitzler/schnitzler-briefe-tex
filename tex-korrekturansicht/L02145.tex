%% latex-korrekturansicht-vorspann.tex
%% Vorspann für die Korrekturansicht.
%% Lädt die gemeinsame Datei latex-vorspann.tex mit gesetztem Schalter.

\newif\ifkorrekturansicht
\korrekturansichttrue

\input{../tex-inputs/latex-vorspann}


               \section[Arthur Schnitzler an Richard Beer-Hofmann, 23. 7. 1913]{ Arthur Schnitzler an Richard Beer-Hofmann, 23. 7. 1913}\nopagebreak\mylabel{v}\rehead{ }\normalsize\beginnumbering\briefempfaengerindex{Beer-Hofmann, Richard@\textsc{Beer-Hofmann, Richard}!zzzSchnitzler, Arthur@\emph{von Arthur Schnitzler}!1913-07-231@{23. 7. 1913}|(be} \toendnotes[C]{\smallbreak\pagebreak[2]} \Standort{YCGL, MSS 31.}
\physDesc{Bildpostkarte
\newline{}Handschrift: schwarze Tinte, deutsche Kurrent\newline{}Versand: Stempel: »\nobreak{}Wien 110, {[}23.{]} VII. 13, 3\nobreak{}«.  \newline{}Zusatz: Postkartenmotiv mit \textcolor{blue}{Olga} und
                                    \textcolor{blue}{Heinrich} links vor dem Haus
                                 und Schnitzler und \textcolor{blue}{Lili} auf dem
                                 Söller }\buchAbdrucke{\weitereDrucke{Arthur Schnitzler, Richard Beer-Hofmann: \emph{Briefwechsel 1891–1931}. Hg. Konstanze Fliedl. Wien, Zürich: \emph{Europaverlag} 1992, S. 218.} }\pstart{}{\pb}Herrn \textsc{Dr. Richard Beer
                     Hofmann}\pend{}\pstart{}aus \textcolor{pink}{Wien}{}\ledrightnote{\textcolor{pink}{Wien}}\pend{}\pstart{}\textsc{\textcolor{pink}{St. Moritz}{}\ledrightnote{\textcolor{pink}{Sankt Moritz}}}\pend{}\pstart{}\textsc{im \textcolor{pink}{Engadin}{}\ledrightnote{\textcolor{pink}{Engadin}}}\pend{}\pstart{}\textsc{\textcolor{pink}{Hotel du lac}{}\ledrightnote{\textcolor{pink}{Hotel du Lac}}.}\pend{}{\bigskip}\pstart
           \noindent{}\centering{}{\pb}\textcolor{gray}{\textbf{\textcolor{pink}{Wien, XVIII, Sternwartestr. 71}{}\ledrightnote{\textcolor{pink}{Sternwartestraße}}.}}\pend
           \pstart
           {\pb}23. 7. 913\pend
           \pstart{}lieber Richard,\pend\pstart
           heute fahren wir nach \textcolor{pink}{Brioni}{}\ledrightnote{\textcolor{pink}{Brijuni}}, \textcolor{blue}{Lili}{}\ledrightnote{\textcolor{blue}{Lili Schnitzler}} iſt ſchon ſeit ein paar Tagen dort – ſchreiben Sie mir
               doch ein Wort dahin. Es war eine etwas unruhig-ruhige Zeit. Sie haben wohl kein
               ſchönes Wetter gehabt bis heut – {\pb}nun
                  wir\textcolor{gray}{ds} hoffentlich beſſer, – überall. Wir grüßen Sie alle
               herzlichſt\pend
           \pstart
           Ihr{\\[\baselineskip]}\spacefill\mbox{A.}\pend
           \leftskip=0em{}\endnumbering\briefempfaengerindex{Beer-Hofmann, Richard@\textsc{Beer-Hofmann, Richard}!zzzSchnitzler, Arthur@\emph{von Arthur Schnitzler}!1913-07-231@{23. 7. 1913}|)be}\mylabel{h}  \normalsize

\doendnotes{C}
\bigskip
\vfill

\clearpage

\footnotesize

\lohead{\textsc{register}}

% Definiere theindex-Environment komplett neu ohne reledmac
\makeatletter
\renewenvironment{theindex}{%
  \section*{\indexname}%
  \setlength{\parindent}{0pt}%
  \setlength{\parskip}{0pt plus 0.3pt}%
  \let\item\@idxitem
}{%
  \clearpage
}
\makeatother

\IfFileExists{\jobname-pw.ind}{\input{\jobname-pw.ind}}{}

\end{document}

      