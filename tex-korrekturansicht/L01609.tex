%% latex-korrekturansicht-vorspann.tex
%% Vorspann für die Korrekturansicht.
%% Lädt die gemeinsame Datei latex-vorspann.tex mit gesetztem Schalter.

\newif\ifkorrekturansicht
\korrekturansichttrue

\input{../tex-inputs/latex-vorspann}


               \section[Georg Brandes an Arthur Schnitzler, 11. 7. 1906]{ Georg Brandes an Arthur Schnitzler, 11. 7. 1906}\nopagebreak\mylabel{v}\rehead{ }\normalsize\beginnumbering\briefempfaengerindex{Schnitzler, Arthur@\textsc{Schnitzler, Arthur}!zzzBrandes, Georg@\emph{von Georg Brandes}!1906-07-112@{11. 7. 1906}|(be} \toendnotes[C]{\smallbreak\pagebreak[2]} \Standort{CUL, Schnitzler, B 17.}
\physDesc{Brief, 1 Blatt, 3 Seiten
\newline{}Handschrift: schwarze Tinte, lateinische Kurrent\newline{}Ordnung: mit Bleistift von unbekannter Hand nummeriert:
                                    »31« }\buchAbdrucke{\weitereDrucke{Georg Brandes, Arthur Schnitzler: \emph{Ein Briefwechsel}. Hg. Kurt Bergel. Bern: \emph{Francke} 1956, S. 93.} }\toendnotes[C]{\smallbreak}\pstart
           \raggedleft{}{\pb}\strikeout{2}{ }11 Juli 06\pend
           \pstart{}Verehrter Freund\pend\pstart
           Wenn ich Ihre Karte einigermassen richtig dechiffrire – die Schrift ist
                    räthselhaft – so fragen Sie nach meinem Befinden und sagen mir dass {\dots} Jemand mich grüssen lässt.\pend
           \pstart
           Ich bin heute aus dem Spital heraus, nur noch sehr, sehr matt, stolpere aber
                    umher, um mich zum Gehen wieder zu gewöhnen.\pend
           \pstart
           Ich wurde sehr gerührt, dass {\pb}Sie meiner gedacht hatten. \textcolor{blue}{Hoffmannsthal}{}\ledrightnote{\textcolor{blue}{Hugo von Hofmannsthal}}
               schickte mir \textcolor{green}{Thor und Tod}{}\ledrightnote{\textcolor{green}{Der Thor und der Tod}}. Es ist schön und fein, machte
                    mir aber lange nicht den Eindruck wie die zwei antikisirenden \textcolor{green}{Schauspiele}{}\ledrightnote{→\textcolor{green}{Elektra. Tragödie in einem Aufzug}{\newline}→\textcolor{green}{Oedipus und die Sphinx. Tragödie in drei Aufzügen}}.\pend
           \pstart
           Ueber Ihre eigenen Arbeiten kam ich das letzte \label{K_L01609_1v}\edtext{Mal}{\lemma{\textnormal{\emph{Mal}}}\Cendnote{\textnormal{\textcolor{blue}{Schnitzler} hatte \textcolor{blue}{Brandes} am 2. 7. 1906 im \textcolor{pink}{Kommunehospitalet} besucht.}}}\label{K_L01609_1h} gar nicht dazu, mit
                    Ihnen zu reden, wollte es doch sehr gern.\pend
           \pstart
           Ich komme wohl eines Tages nach \textcolor{pink}{Helsingør}{}\ledrightnote{\textcolor{pink}{Helsingør}} und
                    versuche an Ihre Thür zu klopfen. Aber etwas kräftiger muss {\pb}ich erst sein.\pend
           \pstart
           Vorläufig soll ich arme Sau Empfangsrede an das \label{K_L01609_2v}\edtext{Allthing}{\lemma{\textnormal{\emph{Allthing}}}\Cendnote{\textnormal{Am
                            17. 8. 1906 hielt \textcolor{blue}{Brandes} eine Festrede für die Mitglieder des \emph{\textcolor{brown}{Althing}}, für die \textcolor{pink}{isländischen} Volksvertreter, die sich in \textcolor{pink}{Kopenhagen} aufhielten.}}}\label{K_L01609_2h} halten.\pend
           \pstart
           Ihr ergebener{\\[\baselineskip]}\spacefill\mbox{Georg Brandes}\pend
           \leftskip=0em{}\endnumbering\briefempfaengerindex{Schnitzler, Arthur@\textsc{Schnitzler, Arthur}!zzzBrandes, Georg@\emph{von Georg Brandes}!1906-07-112@{11. 7. 1906}|)be}\mylabel{h}  \normalsize

\doendnotes{C}
\bigskip
\vfill

\clearpage

\footnotesize

\lohead{\textsc{register}}

% Definiere theindex-Environment komplett neu ohne reledmac
\makeatletter
\renewenvironment{theindex}{%
  \section*{\indexname}%
  \setlength{\parindent}{0pt}%
  \setlength{\parskip}{0pt plus 0.3pt}%
  \let\item\@idxitem
}{%
  \clearpage
}
\makeatother

\IfFileExists{\jobname-pw.ind}{\input{\jobname-pw.ind}}{}

\end{document}

      