%% latex-korrekturansicht-vorspann.tex
%% Vorspann für die Korrekturansicht.
%% Lädt die gemeinsame Datei latex-vorspann.tex mit gesetztem Schalter.

\newif\ifkorrekturansicht
\korrekturansichttrue

\input{../tex-inputs/latex-vorspann}


\renewcommand{\erwaehntePersonen}{Personen:  ?? [intime Partnerin von Georg Brandes, Dresden, Oktober 1900], Hermann Bahr, Georg Brandes, Otto Erich Hartleben, Siegfried Loewy, Malwida von Meysenbug, Friedrich Nietzsche, Wilhelm Ernst Oswalt, Maria Stona}
\renewcommand{\erwaehnteInstitutionen}{Institutionen: Rütten {\kaufmannsund}  Loening}
\renewcommand{\erwaehnteOrte}{Orte: Berlin, Deutsches Theater Berlin, Donau, Dresden, Frankfurt am Main, Frankreich, Volkstheater, Wien}
\renewcommand{\erwaehnteWerke}{Werke: ?? [Kurzkritik von Wienerinnen von Hermann Bahr], Berliner Börsen-Courier, Berliner Tageblatt, Der erste Nietzsche, Die Erziehung zur Ehe. (»Die Lore«, Plauderei in einem Act von Otto Erich Hartleben; »Die Erziehung zur Ehe«, Satire in drei Acten von Otto Erich Hartleben. Zum ersten Mal aufgeführt im Deutschen Volkstheater am 11. September 1897), Die Zeit. Wiener Wochenschrift, Ein Sommer in China. Reisebilder, Ein Sommer in China. Reisebilder. Zweite, durchgesehene und vermehrte Auflage, Kritische Tagebuchblätter, Liebelei. Schauspiel in drei Akten, Neue Freie Presse, Rosenmontag, Theater und Musik [Wienerinnen], Theaterchronik [Wienerinnen], Vossische Zeitung, Wienerinnen. Lustspiel in drei Akten}
\section[ Paul Goldmann an Arthur Schnitzler, 4. 10. {[}1900{]}]{Paul Goldmann an Arthur Schnitzler, 4. 10. {[}1900{]}}
\nopagebreak\mylabel{v}
\rehead{ }\normalsize\beginnumbering\briefempfaengerindex{Schnitzler, Arthur@\textsc{Schnitzler, Arthur}!zzzGoldmann, Paul@\emph{von Paul Goldmann}!1900-10-042@{4. 10. {[}1900{]}}|(be}
\toendnotes[C]{\smallbreak\pagebreak[2]}\Standort{DLA, A:Schnitzler, HS.NZ85.1.3170.}
\physDesc{Brief, 1 Blatt, 4 Seiten
\newline{}Handschrift: blaue Tinte, deutsche Kurrent
\newline{}Beilage: drei aufgeklebte, beschnittene Zeitungsausschnitte 
\newline{}Schnitzler: 1) mit Bleistift das Jahr »{[}1{]}900« vermerkt  2) mit rotem Buntstift drei Unterstreichungen}\toendnotes[C]{\smallbreak}
{\bigskip}
\pstart
           \noindent{}{\pb}\textcolor{pink}{Berlin}{}\ledrightnote{\textcolor{pink}{Berlin}}, 4. Oktober.\hfill \textcolor{gray}{\textbf{DESSAUERSTRASSE 19}}\pend
           
\pstart\center{}Mein lieber Freund,\pend
\pstart
           Ich danke Dir von Herzen für Deine lieben Briefe, insbeſondere für den wunderſchönen
               von neulich, den \strikeout{d\textcolor{gray}{×}\-\textcolor{gray}{×}} ich ausführlich beantworten werde, ſobald ich Zeit finde.\pend
           
\pstart
           Die \textcolor{green}{zweite Auflage}{}\ledrightnote{{$\rightarrow$}\textcolor{green}{Ein Sommer in China. Reisebilder. Zweite, durchgesehene und vermehrte
                  Auflage}} meines \textcolor{green}{Buch}{}\ledrightnote{{$\rightarrow$}\textcolor{green}{Ein Sommer in China. Reisebilder}}es erſcheint erſt \label{K_L02934-1v}\edtext{in einigen Wochen}{\lemma{\textnormal{\emph{in einigen Wochen}}}\Cendnote{\textnormal{Die \textcolor{green}{zweite
                     Auflage} von \emph{\textcolor{green}{Ein Sommer in China}}
                  erschien am 22. 11. 1900.}}}\label{K_L02934-1h}. Der Idiot von
                  \label{K_L02934-2v}\edtext{\textcolor{blue}{Verleger}{}\ledrightnote{{$\rightarrow$}\textcolor{blue}{Wilhelm Ernst Oswalt}}}{\lemma{\textnormal{\emph{Verleger}}}\Cendnote{\textnormal{vermutlich \textcolor{blue}{Wilhelm Ernst Oswalt} vom \textcolor{pink}{Frankfurt}er Verlag \emph{\textcolor{brown}{Rütten {\kaufmannsund} Loening}}}}}\label{K_L02934-2h} kann mit der Drucklegung nicht fertig werden. Selbſtverſtändlich geht ein \textcolor{green}{Exemplar}{}\ledrightnote{{$\rightarrow$}\textcolor{green}{Ein Sommer in China. Reisebilder. Zweite, durchgesehene und vermehrte
                  Auflage}} an die angegebene
               Adreſſe.\pend
           
\pstart
           {\pb}Geſtern hatten wir hier \label{K_L02934-3v}\edtext{»\textcolor{green}{Roſenmontag}{}\ledrightnote{\textcolor{green}{Rosenmontag}}« von \textsc{\textcolor{blue}{Hartleben}{}\ledrightnote{\textcolor{blue}{Otto Erich Hartleben}}}}{\lemma{\textnormal{\emph{»Roſenmontag« von Hartleben}}}\Cendnote{\textnormal{im \textcolor{pink}{Deutschen Theater}}}}\label{K_L02934-3h}. \strikeout{\textcolor{gray}{W}}{ }\label{K_L02934-7v}\edtext{»\textcolor{green}{Unſer \textsc{\textcolor{blue}{Otto Erich}{}\ledrightnote{\textcolor{blue}{Otto Erich Hartleben}}}.}{}\ledrightnote{{$\rightarrow$}\textcolor{green}{Kritische Tagebuchblätter}}«}{\lemma{\textnormal{\emph{»Unſer Otto Erich.«}}}\Cendnote{\textnormal{zur stehenden Wendung
                  gewordene Phrase, die womöglich auf eine Rezension von \textcolor{blue}{Bahr} zurückgeht (vgl. \textcolor{blue}{Hermann Bahr}: \emph{\textcolor{green}{Die
                        Erziehung zur Ehe. (»Die Lore«, Plauderei in einem Act von Otto Erich
                        Hartleben; »Die Erziehung zur Ehe«, Satire in drei Acten von Otto Erich
                        Hartleben. Zum ersten Mal aufgeführt im Deutschen Volkstheater am 11.
                        September 1897)}}. In: \emph{\textcolor{green}{Die Zeit}},
                     Jg. 12, Nr. 155, 18. 9. 1897, S. 188–189)}}}\label{K_L02934-7h} Guter
               erſter \textcolor{green}{Akt}{}\ledrightnote{{$\rightarrow$}\textcolor{green}{Rosenmontag}}. Sobald das \introOben{}eigentliche\introOben{}{ }\textcolor{green}{Drama}{}\ledrightnote{{$\rightarrow$}\textcolor{green}{Rosenmontag}} anfängt, eine von \strikeout{A\textcolor{gray}{k}} Akt zu Akt troſtloſer werdende Unfähigkeit und Leere. So ein \textcolor{blue}{Burſch}{}\ledrightnote{{$\rightarrow$}\textcolor{blue}{Otto Erich Hartleben}} ohne \strikeout{Wärme} Wärme und Poeſie, der ſich als Dichter aufſpielt, weil es in der
               deutſchen Literatur zufällig an ſolchen mangelte!\pend
           
\pstart
           \textsc{\textcolor{blue}{Bahr}{}\ledrightnote{\textcolor{blue}{Hermann Bahr}}} ſcheint auch ein liebes \textcolor{green}{Stück}{}\ledrightnote{{$\rightarrow$}\textcolor{green}{Wienerinnen. Lustspiel in drei Akten}} geſchrieben zu haben. Wir haben hier folgende \textcolor{green}{Berichte}{}\ledrightnote{{$\rightarrow$}\textcolor{green}{Theater und Musik [Wienerinnen]}{\newline}{$\rightarrow$}\textcolor{green}{Theaterchronik [Wienerinnen]}{\newline}{$\rightarrow$}\textcolor{green}{?? [Kurzkritik von Wienerinnen von Hermann Bahr]}}{ }{\pb}erhalten:\pend
           
\pstart
           \substVorne{}\textsuperscript{B\textcolor{gray}{er}}\substDazwischen{}\textcolor{green}{Vo}{}\ledrightnote{{$\rightarrow$}\textcolor{green}{Vossische Zeitung}}\substHinten{}\textcolor{green}{ſſiſche Zeitung}{}\ledrightnote{{$\rightarrow$}\textcolor{green}{Vossische Zeitung}}:\pend
           
\pstart
           \label{K_L02934-5v}\edtext{\textcolor{green}{\textcolor{gray}{\textbf{Im \textcolor{pink}{\so{Deutſchen Volkstheater}}{}\ledrightnote{\textcolor{pink}{Volkstheater}} hatte heute ein neues Stück \textbf{»\textcolor{green}{Die Wienerinnen}{}\ledrightnote{\textcolor{green}{Wienerinnen. Lustspiel in drei Akten}}«} von
                        \textcolor{blue}{\so{Hermann Bahr}}{}\ledrightnote{\textcolor{blue}{Hermann Bahr}} einen durchſchlagenden Erfolg.}}}{}\ledrightnote{{$\rightarrow$}\textcolor{green}{Theater und Musik [Wienerinnen]}}}{\lemma{\textnormal{\emph{Im … Erfolg.}}}\Cendnote{\textnormal{Auszug aus [O. V.]: \emph{\textcolor{green}{Theater und Musik}}. In: \emph{\textcolor{green}{Vossische Zeitung}}, Nr. 464, 4. 10. 1900, Morgen-Ausgabe, S. [16]}}}\label{K_L02934-5h}\pend
           {\bigskip}
\pstart
           \noindent{}\textcolor{green}{Berliner Tageblatt}{}\ledrightnote{\textcolor{green}{Berliner Tageblatt}}:\pend
           
\pstart
           \label{K_L02934-6v}\edtext{\textcolor{green}{\textcolor{gray}{\textbf{Aus \textcolor{pink}{\textbf{Wien}}{}\ledrightnote{\textcolor{pink}{Wien}} meldet uns ein Privat-Telegramm: \textcolor{blue}{Hermann Bahr}{}\ledrightnote{\textcolor{blue}{Hermann Bahr}}s Luſtſpiel »\textcolor{green}{\so{Wienerinnen}}{}\ledrightnote{\textcolor{green}{Wienerinnen. Lustspiel in drei Akten}}« hatte einen kompleten Mißerfolg.}}}{}\ledrightnote{{$\rightarrow$}\textcolor{green}{Theaterchronik [Wienerinnen]}}}{\lemma{\textnormal{\emph{Aus … Mißerfolg.}}}\Cendnote{\textnormal{Auszug aus [O. V.]: \emph{\textcolor{green}{Theaterchronik}}. In: \emph{\textcolor{green}{Berliner Tageblatt}}, Jg. 29, Nr. 504, 4. 10. 1900, Morgen-Ausgabe, S. {[}3{]}.
               }}}\label{K_L02934-6h}\pend
           {\bigskip}
\pstart
           \noindent{}Dieſe \label{K_L02934-21v}\edtext{zwei Kritiker}{\lemma{\textnormal{\emph{zwei Kritiker}}}\Cendnote{\textnormal{nicht ermittelt}}}\label{K_L02934-21h} ſcheinen das neue
                  \textcolor{green}{Werk}{}\ledrightnote{{$\rightarrow$}\textcolor{green}{Wienerinnen. Lustspiel in drei Akten}} von verſchiedenen
               Geſichtspunkten aus zu betrachten. Im »\textcolor{green}{Börſencourier}{}\ledrightnote{\textcolor{green}{Berliner Börsen-Courier}}« aber ſchmückt \textsc{\textcolor{blue}{Siegfried Löwy}{}\ledrightnote{\textcolor{blue}{Siegfried Loewy}}} ſich folgendermaßen aus:\pend
           
\pstart
           \label{K_L02934-9v}\edtext{\textcolor{green}{{\pb}\textcolor{gray}{\textbf{Das »süße \textcolor{pink}{Wien}{}\ledrightnote{\textcolor{pink}{Wien}}er Mädel«
                     iſt durch Arthur Schnitzler’s farbenſatte Schilderung mit ihrer ergreifenden
                     Wendung in’s Tragiſche in ſeiner ganzen Echtheit \label{K_L02934-17v}\edtext{in »\textcolor{green}{Liebelei}{}\ledrightnote{\textcolor{green}{Liebelei. Schauspiel in drei Akten}}« zum
                     erſten Male auf die Bühne gebracht}{\lemma{\textnormal{\emph{in … gebracht}}}\Cendnote{\textnormal{siehe zum Begriff »süßel Mädel« auch Paul Goldmann an Arthur Schnitzler, 4. 10. [1900]}}}\label{K_L02934-17h} worden, das Mädel aus dem Volke, die kleine, liebe \label{K_L02934-12v}\edtext{Griſette}{\lemma{\textnormal{\emph{Griſette}}}\Cendnote{\textnormal{unverheiratete junge Frau niederen Standes, die etwa
                        als Modistin, Fabrikarbeiterin, Näherin oder Wäscherin ihren Unterhalt
                        selbst finanziert (bekannt aus der \textcolor{pink}{fran}zösischen Literatur des 19. Jahrhunderts)}}}\label{K_L02934-12h},
                     die ja ſchließlich nicht blos in \textcolor{pink}{Wien}{}\ledrightnote{\textcolor{pink}{Wien}} zu
                     finden iſt, der aber die \textcolor{pink}{Wien}{}\ledrightnote{\textcolor{pink}{Wien}}er Art, der \textcolor{pink}{Wien}{}\ledrightnote{\textcolor{pink}{Wien}}er Humor ſo ganz beſonders gut zu Geſicht
                     ſteht. Ein gründlicher Kenner der \textcolor{pink}{Wien}{}\ledrightnote{\textcolor{pink}{Wien}}er
                     Verhältniſſe, ein geiſtreicher Spottvogel, \textcolor{blue}{Hermann \so{Bahr}}{}\ledrightnote{\textcolor{blue}{Hermann Bahr}}, hat nun in ſeinem ſoeben aufgeführten Luſtſpiel »\textcolor{green}{\so{Wienerinnen}}{}\ledrightnote{\textcolor{green}{Wienerinnen. Lustspiel in drei Akten}}« einen anderen Typus der mit dem Waſſer der blauen \textcolor{pink}{Donau}{}\ledrightnote{\textcolor{pink}{Donau}} getauften – manchmal auch nicht getauften
                     weiblichen Jugend von heute gezeichnet.}}}{}\ledrightnote{{$\rightarrow$}\textcolor{green}{?? [Kurzkritik von Wienerinnen von Hermann Bahr]}}}{\lemma{\textnormal{\emph{Das … gezeichnet.}}}\Cendnote{\textnormal{\textcolor{blue}{Siegfried Löwy}: \emph{\textcolor{green}{XXXX}}. In: \emph{\textcolor{green}{Berliner
                        Börſencourier}}, Jg. YY, Nr. YY, 4.YYY 10. 1900,
                     S. YY. }}}\label{K_L02934-9h}\pend
           {\bigskip}
\pstart
           \noindent{}Bitte, liebſter Freund, wenn Du eine Minute Zeit haſt, ſchreib’ mir in drei Worten
               die Wahrheit!\pend
           
\pstart
           Was haſt Du zu den herrlichen \label{K_L02934-18v}\edtext{\textcolor{green}{\textsc{\textcolor{blue}{Nietzsche}{}\ledrightnote{\textcolor{blue}{Friedrich Nietzsche}}}-Briefe}{}\ledrightnote{{$\rightarrow$}\textcolor{green}{Der erste Nietzsche}}n in der \textcolor{green}{N. Fr. Pr.}{}\ledrightnote{\textcolor{green}{Neue Freie Presse}}}{\lemma{\textnormal{\emph{Nietzsche-Briefen … Pr.}}}\Cendnote{\textnormal{Bezug auf die Feuilletonreihe \emph{\textcolor{green}{Der erste Nietzsche}} von \textcolor{blue}{Malwida von Meysenbug}, die zwischen 18. 9. 1900 (Nr. 12956) und 28. 9. 1900 (Nr. 12966) in der \emph{\textcolor{green}{Neuen
                     Freien Presse}} erschienen war}}}\label{K_L02934-18h} geſagt?\pend
           
\pstart
           Viele treue Grüße! {\\[\baselineskip]}Dein {\\[\baselineskip]}\spacefill\mbox{Paul Goldmann}\pend
           \leftskip=0em{}
\pstart
           \noindent{}\textsc{\textcolor{blue}{Brandes}{}\ledrightnote{\textcolor{blue}{Georg Brandes}}} war \textcolor{pink}{hier}{}\ledrightnote{{$\rightarrow$}\textcolor{pink}{Berlin}} und iſt zu
                  einem \label{K_L02934-19v}\edtext{ weiblichen \textcolor{blue}{Rendezvous}{}\ledrightnote{{$\rightarrow$}\textcolor{blue}{?? [intime Partnerin von Georg Brandes, Dresden, Oktober 1900]}}}{\lemma{\textnormal{\emph{ weiblichen Rendezvous}}}\Cendnote{\textnormal{jedenfalls nicht \textcolor{blue}{Maria Stona}, die enttäuscht war, dass \textcolor{blue}{Georg Brandes} nicht auch zu ihr
                     reiste (vgl. Martin Pelc: \emph{Maria Stona und ihr Salon in
                           Strzebowitz. Kultur am Rande der Monarchie, der Republik und des
                           Kanons}. Opava: \emph{Europäischer
                           Strukturfonds}/\emph{Schlesische Universität}{ }2014, S. 126)}}}\label{K_L02934-19h}, wie er ſelbſt
                  mittheilt, nach \textcolor{pink}{Dresden}{}\ledrightnote{\textcolor{pink}{Dresden}} gefahren.\pend
           \endnumbering\briefempfaengerindex{Schnitzler, Arthur@\textsc{Schnitzler, Arthur}!zzzGoldmann, Paul@\emph{von Paul Goldmann}!1900-10-042@{4. 10. {[}1900{]}}|)be}\mylabel{h}  \normalsize

\doendnotes{C}
\bigskip
\vfill

\clearpage

\footnotesize

\lohead{\textsc{register}}

% Definiere theindex-Environment komplett neu ohne reledmac
\makeatletter
\renewenvironment{theindex}{%
  \section*{\indexname}%
  \setlength{\parindent}{0pt}%
  \setlength{\parskip}{0pt plus 0.3pt}%
  \let\item\@idxitem
}{%
  \clearpage
}
\makeatother

\IfFileExists{\jobname-pw.ind}{\input{\jobname-pw.ind}}{}

\end{document}

      