%% latex-korrekturansicht-vorspann.tex
%% Vorspann für die Korrekturansicht.
%% Lädt die gemeinsame Datei latex-vorspann.tex mit gesetztem Schalter.

\newif\ifkorrekturansicht
\korrekturansichttrue

\input{../tex-inputs/latex-vorspann}


\renewcommand{\erwaehntePersonen}{Personen: Emanuel Reicher, Rudolf Rittner, Felix Salten}
\renewcommand{\erwaehnteInstitutionen}{Institutionen: Lessing-Theater}
\renewcommand{\erwaehnteOrte}{Orte: Berlin, Theater an der Wien, Wien}
\renewcommand{\erwaehnteWerke}{Werke: Der einsame Weg. Schauspiel in fünf Akten}
\section[ Felix Salten an Arthur Schnitzler, 14. 5. 1906]{Felix Salten an Arthur Schnitzler, 14. 5. 1906}
\nopagebreak\mylabel{v}
\rehead{ }\normalsize\beginnumbering\briefempfaengerindex{Schnitzler, Arthur@\textsc{Schnitzler, Arthur}!zzzSalten, Felix@\emph{von Felix Salten}!1906-05-141@{14. 5. 1906}|(be}
\toendnotes[C]{\smallbreak\pagebreak[2]}\Standort{CUL, Schnitzler, B 89, B 1.}
\physDesc{Brief, 1 Blatt, 1 Seite, 271 Zeichen
\newline{}Handschrift: schwarze Tinte, lateinische Kurrent
\newline{}Ordnung: mit Bleistift von unbekannter Hand nummeriert: »215« }\toendnotes[C]{\smallbreak}
\pstart
           \raggedleft{}{\pb}\textcolor{pink}{Berlin}{}\ledrightnote{\textcolor{pink}{Berlin}}, 14. V. 06.\pend
           
\pstart{}Lieber Freund,\pend
\pstart
           \label{K_L03474-1v}\edtext{morgen spielen sie in \textcolor{pink}{Wien}{}\ledrightnote{\textcolor{pink}{Wien}} Ihren »\textcolor{green}{Einsamen}{}\ledrightnote{\textcolor{green}{Der einsame Weg. Schauspiel in fünf Akten}}{ }\damage{\textcolor{green}{Weg}{}\ledrightnote{\textcolor{green}{Der einsame Weg. Schauspiel in fünf Akten}}«.}}{\lemma{\textnormal{\emph{morgen … Weg«.}}}\Cendnote{\textnormal{Das \textcolor{green}{Gastspiel} des \emph{\textcolor{brown}{Lessing-Theater}}s fand im \textcolor{pink}{Theater an der
                     Wien} statt. Siehe A. S.: \emph{Tagebuch}, 15. 5. 1906.}}}\label{K_L03474-1h} Irgendwie habe ich dabei das Gefühl, dass ich mir
               selbst (und viellei\damage{cht} auch Ihnen ein wenig) dort \label{K_L03474-2v}\edtext{fehle}{\lemma{\textnormal{\emph{fehle}}}\Cendnote{\textnormal{\textcolor{blue}{Salten} fühlte
               sich womöglich auch deswegen involviert, weil er im Voraus \textcolor{blue}{Schnitzler} empfohlen hatte, eine 
                  Umbesetzung von \textcolor{blue}{Emanuel
                     Reicher} zu \textcolor{blue}{Rudolf Rittner} zu erwirken, vgl. Felix Salten u. a. an Arthur Schnitzler, 19. 4. 1906, 
                  vgl. Felix Salten an Arthur Schnitzler, 21. 4. [1906].
               }}}\label{K_L03474-2h}.
               Jedenfalls möchte ich, dass Sie an diesem Tag einen Gruß von mir haben.\pend
           
\pstart
           herzlichst {\\[\baselineskip]}Ihr \spacefill\mbox{Salten}\pend
           \leftskip=0em{}\endnumbering\briefempfaengerindex{Schnitzler, Arthur@\textsc{Schnitzler, Arthur}!zzzSalten, Felix@\emph{von Felix Salten}!1906-05-141@{14. 5. 1906}|)be}\mylabel{h}  \normalsize

\doendnotes{C}
\bigskip
\vfill

\clearpage

\footnotesize

\lohead{\textsc{register}}

% Definiere theindex-Environment komplett neu ohne reledmac
\makeatletter
\renewenvironment{theindex}{%
  \section*{\indexname}%
  \setlength{\parindent}{0pt}%
  \setlength{\parskip}{0pt plus 0.3pt}%
  \let\item\@idxitem
}{%
  \clearpage
}
\makeatother

\IfFileExists{\jobname-pw.ind}{\input{\jobname-pw.ind}}{}

\end{document}

      