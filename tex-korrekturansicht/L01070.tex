%% latex-korrekturansicht-vorspann.tex
%% Vorspann für die Korrekturansicht.
%% Lädt die gemeinsame Datei latex-vorspann.tex mit gesetztem Schalter.

\newif\ifkorrekturansicht
\korrekturansichttrue

\input{../tex-inputs/latex-vorspann}


               \section[Hugo von Hofmannsthal an Arthur Schnitzler, {[}3.?{]} 9. 1900]{ Hugo von Hofmannsthal an Arthur Schnitzler, {[}3.?{]} 9. 1900}\nopagebreak\mylabel{v}\rehead{ }\normalsize\beginnumbering\briefempfaengerindex{Schnitzler, Arthur@\textsc{Schnitzler, Arthur}!zzzHofmannsthal, Hugo von@\emph{von Hugo von Hofmannsthal}!1900-09-031@{{[}3.?{]} 9. 1900}|(be} \toendnotes[C]{\smallbreak\pagebreak[2]} \Standort{CUL, Schnitzler, B 43.}
\physDesc{Bildpostkarte
\newline{}Handschrift: Bleistift, deutsche Kurrent\newline{}Versand: 1) Stempel: »\nobreak{}\oindex{Sambir@\textbf{Sambir}, \emph{http://www.geonames.org/ontologyP.PPLA2}|pwk}{[}Sambor{]}\nobreak{}«.  2) Stempel: »\nobreak{}\oindex{IX., Alsergrund@\textbf{IX., Alsergrund}, \emph{Bezirk (A.BZK)}|pwk}Wien 9/3, 4. 9. 00, 5.{[}N{]}, Bestellt\nobreak{}«. 
\newline{}Schnitzler: mit Bleistift datiert: »Aug 900« \newline{}Ordnung: mit Bleistift von unbekannter Hand doppelt
                           nummeriert: »165« und »\strikeout{172}« }\buchAbdrucke{\weitereDrucke{Hugo von Hofmannsthal, Arthur Schnitzler: \emph{Briefwechsel}. Hg. Therese Nickl und Heinrich Schnitzler. Frankfurt am Main: \emph{S. Fischer} 1964, S. 144.} }\pstart{}{\pb}\textsc{Herrn D\textsuperscript{r} Arthur Schnitzler}\pend{}\pstart{}\textcolor{pink}{\textsc{Wien}}{}\ledrightnote{\textcolor{pink}{Wien}}\pend{}\pstart{}\textcolor{pink}{IX Franckgasse 1}{}\ledrightnote{\textcolor{pink}{Frankgasse}}.\pend{}{\bigskip}\pstart
           \noindent{}\centering{}{\pb}\textcolor{gray}{\textbf{\textcolor{pink}{Sambor}{}\ledrightnote{\textcolor{pink}{Sambir}}}}\pend
           \pstart
           \noindent{}\centering{}\textcolor{gray}{\textbf{z wiezy koscielnej widziany.}}\pend
           \pstart
           \noindent{}\centering{}\textcolor{gray}{\textbf{von dem Kirchenthurme gesehen.}}\pend
           \pstart
           {\pb}Was Sie machen? Ich bin
               10 Stunden im Sattel, ſchlafe im Heu, jeden Tag in einem andern Neſt, und bin
               eigentlich ſehr zufrieden und gut aufgelegt.\pend
           \pstart
           Von Herzen{\\[\baselineskip]}Ihr\spacefill\mbox{Hugo.}\pend
           \leftskip=0em{}\endnumbering\briefempfaengerindex{Schnitzler, Arthur@\textsc{Schnitzler, Arthur}!zzzHofmannsthal, Hugo von@\emph{von Hugo von Hofmannsthal}!1900-09-031@{{[}3.?{]} 9. 1900}|)be}\mylabel{h}  \normalsize

\doendnotes{C}
\bigskip
\vfill

\clearpage

\footnotesize

\lohead{\textsc{register}}

% Definiere theindex-Environment komplett neu ohne reledmac
\makeatletter
\renewenvironment{theindex}{%
  \section*{\indexname}%
  \setlength{\parindent}{0pt}%
  \setlength{\parskip}{0pt plus 0.3pt}%
  \let\item\@idxitem
}{%
  \clearpage
}
\makeatother

\IfFileExists{\jobname-pw.ind}{\input{\jobname-pw.ind}}{}

\end{document}

      