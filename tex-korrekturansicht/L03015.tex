%% latex-korrekturansicht-vorspann.tex
%% Vorspann für die Korrekturansicht.
%% Lädt die gemeinsame Datei latex-vorspann.tex mit gesetztem Schalter.

\newif\ifkorrekturansicht
\korrekturansichttrue

\input{../tex-inputs/latex-vorspann}


\renewcommand{\erwaehntePersonen}{Personen: Leopold von Andrian-Werburg, Anton Bettelheim, Hedwig Bleibtreu, Felix Salten}
\renewcommand{\erwaehnteInstitutionen}{Institutionen: Die Zeit}
\renewcommand{\erwaehnteOrte}{Orte: Sternwartestraße 71, Wien}
\renewcommand{\erwaehnteWerke}{Werke: Tagebuch}
\section[ Arthur Schnitzler an Felix Salten, {[}14. 4. 1910?{]}]{Arthur Schnitzler an Felix Salten, {[}14. 4. 1910?{]}}
\nopagebreak\mylabel{v}
\rehead{ }\normalsize\beginnumbering\briefempfaengerindex{Salten, Felix@\textsc{Salten, Felix}!zzzSchnitzler, Arthur@\emph{von Arthur Schnitzler}!1910-04-141@{{[}14. 4. 1910?{]}}|(be}
\toendnotes[C]{\smallbreak\pagebreak[2]}\Standort{Wienbibliothek im Rathaus, ZPH 1681, 2.1.516.}
\physDesc{Brief, 1 Blatt, 3 Seiten, 502 Zeichen
\newline{}Handschrift: Bleistift, deutsche Kurrent
\newline{}Ordnung: mit Bleistift von unbekannter Hand Nummerierung der Doppelseiten des Konvoluts:
                                    »5«–»6« }\toendnotes[C]{\smallbreak}
\pstart
           \noindent{}{\pb}lieber, ich weiſs nun nicht, wa{\geminationn} ich
                  \label{K_L03015-1v}\edtext{in den nächſten Tagen zu Ihnen ko{\geminationm}en}{\lemma{\textnormal{\emph{in … kommen}}}\Cendnote{\textnormal{\textcolor{blue}{Schnitzler} war am 15. 4. 1910 und am
                     21. 4. 1910 bei
                     \textcolor{blue}{Salten}.}}}\label{K_L03015-1h} ka{\geminationn}; u muſs Sie nur etwas fragen: wie Ihre Sache mit der
                  »\textcolor{brown}{\uline{Zeit}}{}\ledrightnote{\textcolor{brown}{Die Zeit}}« ſteht. Es hat mich nemlich {\pb}\label{K_L03015-2v}\edtext{\textcolor{blue}{jemand}{}\ledrightnote{{$\rightarrow$}\textcolor{blue}{Leopold von Andrian-Werburg}{\newline}{$\rightarrow$}\textcolor{blue}{Anton Bettelheim}}}{\lemma{\textnormal{\emph{jemand}}}\Cendnote{\textnormal{Sofern es sich um jemanden handelt, der
                  am 14. 4. 1910 im
                     \emph{\textcolor{green}{Tagebuch}} genannt wird, könnten \textcolor{blue}{Leopold Andrian} oder \textcolor{blue}{Anton Bettelheim} gemeint gewesen sein.}}}\label{K_L03015-2h}, den ich
               nicht nennen darf, um meine Intervention für die Stellung eines
                  Feu{[}i{]}lleton Redacteurs erſucht, u ich habe vorläufg abgelehnt,
               da ich nicht weiſs, ob Sie noch in Verhandlung {\pb}ſtehn \textsc{etc.} (Habe
               natürlich Ihren Namen nicht genannt.) Bitte ſagen Sie mir ein Wort. Was fehlt Ihnen
               eigentlich?\pend
           
\pstart
           herzlichſt Ihr {\\[\baselineskip]}\spacefill\mbox{Arthur}\pend
           \leftskip=0em{}
\pstart
           \noindent{}\label{K_L03015-3v}\edtext{Endlich hab ich die \textcolor{pink}{Villa}{}\ledrightnote{{$\rightarrow$}\textcolor{pink}{Sternwartestraße 71}}}{\lemma{\textnormal{\emph{Endlich … Villa}}}\Cendnote{\textnormal{Am 14. 4. 1910 hatte \textcolor{blue}{Schnitzler} den Kaufvertrag für das bis dahin im Eigentum
                     von \textcolor{blue}{Hedwig Bleibtreu-Römpler} stehende
                     Haus in der \textcolor{pink}{Sternwartestrasse 71}
                     unterschrieben. Damit kann das undatierte Korrespondenzstück zeitlich nach
                     vorne abgegrenzt werden. Da sich \textcolor{blue}{Salten}
                     und \textcolor{blue}{Schnitzler} am Folgetag, dem 15. 4. 1910,
                     bereits ausführlich sprachen, ist auch zeitlich nach hinten eine Grenze zu
                     ziehen.}}}\label{K_L03015-3h}\pend
           \endnumbering\briefempfaengerindex{Salten, Felix@\textsc{Salten, Felix}!zzzSchnitzler, Arthur@\emph{von Arthur Schnitzler}!1910-04-141@{{[}14. 4. 1910?{]}}|)be}\mylabel{h}  \normalsize

\doendnotes{C}
\bigskip
\vfill

\clearpage

\footnotesize

\lohead{\textsc{register}}

% Definiere theindex-Environment komplett neu ohne reledmac
\makeatletter
\renewenvironment{theindex}{%
  \section*{\indexname}%
  \setlength{\parindent}{0pt}%
  \setlength{\parskip}{0pt plus 0.3pt}%
  \let\item\@idxitem
}{%
  \clearpage
}
\makeatother

\IfFileExists{\jobname-pw.ind}{\input{\jobname-pw.ind}}{}

\end{document}

      