%% latex-korrekturansicht-vorspann.tex
%% Vorspann für die Korrekturansicht.
%% Lädt die gemeinsame Datei latex-vorspann.tex mit gesetztem Schalter.

\newif\ifkorrekturansicht
\korrekturansichttrue

\input{../tex-inputs/latex-vorspann}


\renewcommand{\erwaehntePersonen}{Personen: Richard Beer-Hofmann, Alois Hofmann, Gustav Mahler, Maria Anna Mahler, Felix Salten, Ottilie Salten, Michael Emil Salzmann, Olga Schnitzler}
\renewcommand{\erwaehnteInstitutionen}{Institutionen: Neue Freie Presse}
\renewcommand{\erwaehnteOrte}{Orte: Edlach, Südtirol, Welsberg-Taisten, Wien, Wildbad Waldbrunn, Wocheiner See, XIX., Döbling}
\renewcommand{\erwaehnteWerke}{Werke: Kindertotenlieder}
\section[ Felix Salten an Arthur Schnitzler, 15. 7. 1907]{Felix Salten an Arthur Schnitzler, 15. 7. 1907}
\nopagebreak\mylabel{v}
\rehead{ }\normalsize\beginnumbering\briefempfaengerindex{Schnitzler, Arthur@\textsc{Schnitzler, Arthur}!zzzSalten, Felix@\emph{von Felix Salten}!1907-07-152@{15. 7. 1907}|(be}
\toendnotes[C]{\smallbreak\pagebreak[2]}\Standort{CUL, Schnitzler, B 89, B 1.}
\physDesc{Postkarte, 978 Zeichen
\newline{}Handschrift: schwarze Tinte, lateinische Kurrent
\newline{}Versand: 1) Stempel: »\nobreak{}\oindex{XIX., Doebling@\textbf{XIX., Döbling}, \emph{A.ADM3}|pwk}19/2 Wien 119, 15. VII. 07, 6\nobreak{}«.   2) Stempel: »\nobreak{}\oindex{Welsberg-Taisten@\textbf{Welsberg-Taisten}, \emph{A.ADM3}|pwk}Welsberg, 16\textcolor{gray}{.} 7. 07\nobreak{}«. 
\newline{}Ordnung: mit Bleistift von unbekannter Hand nummeriert: »231« }\toendnotes[C]{\smallbreak}\pstart{}{\pb}Herrn D\textsuperscript{r} Arthur Schnitzler\pend{}\pstart{}\textcolor{pink}{Wildbad Waldbrunn}{}\ledrightnote{\textcolor{pink}{Wildbad Waldbrunn}}{ }\textsuperscript{b}/\textcolor{pink}{Welsberg i
                     Pustertal}{}\ledrightnote{\textcolor{pink}{Welsberg-Taisten}}\pend{}\pstart{}\textcolor{pink}{Tirol}{}\ledrightnote{\textcolor{pink}{Südtirol}}\pend{}
{\bigskip}
\pstart
           \noindent{}{\pb}Lieber, für die \label{K_L03488-1v}\edtext{\textcolor{pink}{Wochein}{}\ledrightnote{{$\rightarrow$}\textcolor{pink}{Wocheiner See}}er Pläne ist \textcolor{pink}{Waldbrunn}{}\ledrightnote{\textcolor{pink}{Wildbad Waldbrunn}}}{\lemma{\textnormal{\emph{Wocheiner … Waldbrunn}}}\Cendnote{\textnormal{\textcolor{blue}{Arthur} und \textcolor{blue}{Olga Schnitzler} waren zwischen 28. 6. 1907 und 30. 6. 1907 am \textcolor{pink}{Wocheiner See}. Seit 4. 7. 1907 und bis zum
                  26. 8. 1907 hielten sie sich im \textcolor{pink}{Wildbad Waldbrunn}
                  in \textcolor{pink}{Welsberg} auf.}}}\label{K_L03488-1h} immerhin ein
               überraschendes Resultat. Aber \textcolor{pink}{Welsberg}{}\ledrightnote{\textcolor{pink}{Welsberg-Taisten}} ist sehr
               schön. – Was haben Sie denn für Wetter dort? Bei uns geht man im Winterrock, was die
                  \textcolor{brown}{Neue freie Presse}{}\ledrightnote{\textcolor{brown}{Neue Freie Presse}} veranlaßt, ihre
               Sonntagsfeuilletonisten über Hitzschläge plaudern zu laßen. – Gestern wurde \label{K_L03488-2v}\edtext{\textcolor{blue}{Beer-Hofmann}{}\ledrightnote{\textcolor{blue}{Richard Beer-Hofmann}}s \textcolor{blue}{Vater}{}\ledrightnote{{$\rightarrow$}\textcolor{blue}{Alois Hofmann}} begraben}{\lemma{\textnormal{\emph{Beer-Hofmanns Vater begraben}}}\Cendnote{\textnormal{\textcolor{blue}{Alois Hofmann} starb am 11. 7. 1907.}}}\label{K_L03488-2h}, der furchtbar gelitten
               haben soll. \label{K_L03488-3v}\edtext{\textcolor{blue}{Mahler}{}\ledrightnote{\textcolor{blue}{Gustav Mahler}}’s \textcolor{blue}{Kind}{}\ledrightnote{{$\rightarrow$}\textcolor{blue}{Maria Anna Mahler}}}{\lemma{\textnormal{\emph{Mahler’s Kind}}}\Cendnote{\textnormal{\textcolor{blue}{Maria Anna Mahler} starb am 12. 7. 1907.}}}\label{K_L03488-3h} – hat mich so
               ergriffen, dass ich garnicht zur Ruhe kommen konnte. – Erinnern Sie sich, dass ich
               seine \textcolor{green}{Kindertotenlieder}{}\ledrightnote{\textcolor{green}{Kindertotenlieder}} nicht hören konnte? –
               Überhaupt ist es ein lieblicher Sommer: mit meinem \label{K_L03488-4v}\edtext{Bruder \textcolor{blue}{Emil}{}\ledrightnote{\textcolor{blue}{Michael Emil Salzmann}}}{\lemma{\textnormal{\emph{Bruder Emil}}}\Cendnote{\textnormal{siehe Arthur Schnitzler an Felix Salten, 29. 6. 1908 und Felix Salten an Arthur Schnitzler, 5. 7. 1908}}}\label{K_L03488-4h} hatte ich noch manchen Schrecken auszustehen. Doch geht’s ihm jetzt in \textcolor{pink}{Edlach}{}\ledrightnote{\textcolor{pink}{Edlach}} besser. \textcolor{blue}{Otti}{}\ledrightnote{\textcolor{blue}{Ottilie Salten}} ist dauernd leidend und muß dieser Tage eine Operation überstehen.
               Lauter angenehme Dinge. Ob wir dann noch fortreisen, weiss ich nicht. Sehr weit
               schwerlich. Laßen Sie bald wieder was hören und seien Sie alle von uns herzlichst
               gegrüßt\pend
           \pstart Ihr \spacefill\mbox{Salten}\pend{}
\pstart
           15. 7. 07.\pend
           \endnumbering\briefempfaengerindex{Schnitzler, Arthur@\textsc{Schnitzler, Arthur}!zzzSalten, Felix@\emph{von Felix Salten}!1907-07-152@{15. 7. 1907}|)be}\mylabel{h}  \normalsize

\doendnotes{C}
\bigskip
\vfill

\clearpage

\footnotesize

\lohead{\textsc{register}}

% Definiere theindex-Environment komplett neu ohne reledmac
\makeatletter
\renewenvironment{theindex}{%
  \section*{\indexname}%
  \setlength{\parindent}{0pt}%
  \setlength{\parskip}{0pt plus 0.3pt}%
  \let\item\@idxitem
}{%
  \clearpage
}
\makeatother

\IfFileExists{\jobname-pw.ind}{\input{\jobname-pw.ind}}{}

\end{document}

      