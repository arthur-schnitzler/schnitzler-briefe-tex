%% latex-korrekturansicht-vorspann.tex
%% Vorspann für die Korrekturansicht.
%% Lädt die gemeinsame Datei latex-vorspann.tex mit gesetztem Schalter.

\newif\ifkorrekturansicht
\korrekturansichttrue

\input{../tex-inputs/latex-vorspann}


\renewcommand{\erwaehntePersonen}{Personen: Felix Salten}
\renewcommand{\erwaehnteOrte}{Orte: Gasthof zur Post, Kaltenleutgeben, Kaltwasserheilanstalt Winternitz, Lofer, Niederösterreich, Wien}
\renewcommand{\erwaehnteWerke}{}
\section[ Arthur Schnitzler und Hugo von Hofmannsthal an Felix Salten, {[}1. 7. 1902{]}]{Arthur Schnitzler und Hugo von Hofmannsthal an Felix
               Salten, {[}1. 7. 1902{]}}
\nopagebreak\mylabel{v}
\rehead{ }\normalsize\beginnumbering\briefempfaengerindex{Salten, Felix@\textsc{Salten, Felix}!zzzHofmannsthal, Hugo von@\emph{von Hugo von Hofmannsthal}!1902-07-011@{{[}1. 7. 1902{]}}|(be}\briefempfaengerindex{Salten, Felix@\textsc{Salten, Felix}!zzzSchnitzler, Arthur@\emph{von Arthur Schnitzler}!1902-07-011@{{[}1. 7. 1902{]}}|(be}
\toendnotes[C]{\smallbreak\pagebreak[2]}\Standort{Wienbibliothek im Rathaus, ZPH 1681, 2.1.516.}
\physDesc{Bildpostkarte, 136 Zeichen
\newline{}Handschrift Arthur Schnitzler: Bleistift, deutsche Kurrent
\newline{}Handschrift Hugo von Hofmannsthal: Bleistift, lateinische Kurrent
\newline{}Versand: Stempel: »\nobreak{}\oindex{Kaltenleutgeben@\textbf{Kaltenleutgeben}, \emph{P.PPLA3}|pwk}Kaltenleutgeben, 2. 7. 02, 12–1N, Bestellt\nobreak{}«.  
\newline{}Ordnung: mit Bleistift von unbekannter Hand Nummerierung der Blätter des Konvoluts: »\textcolor{gray}{4}« }\toendnotes[C]{\smallbreak}\pstart{}{\pb}\textsc{\textcolor{pink}{N.Oe.}{}\ledrightnote{\textcolor{pink}{Niederösterreich}}}\pend{}\pstart{}Herrn Felix Salten\pend{}\pstart{}\textsc{\textcolor{pink}{Kaltenleutgeben}{}\ledrightnote{\textcolor{pink}{Kaltenleutgeben}}}\pend{}\pstart{}\textsc{bei}{ }\textcolor{pink}{Wien}{}\ledrightnote{\textcolor{pink}{Wien}}\pend{}\pstart{}\textsc{\textcolor{pink}{Anstalt Winternitz}{}\ledrightnote{\textcolor{pink}{Kaltwasserheilanstalt Winternitz}}}\pend{}
{\bigskip}
\pstart
           \noindent{}\centering{}{\pb}\textcolor{gray}{\textbf{\textcolor{pink}{Gasthof zur Post}{}\ledrightnote{\textcolor{pink}{Gasthof zur Post}}}}\pend
           
\pstart
           \noindent{}\centering{}\textcolor{gray}{\textbf{\textcolor{pink}{Lofer}{}\ledrightnote{\textcolor{pink}{Lofer}}}}\pend
           
\pstart
           Herzl\textcolor{gray}{.} Grüße von demſelben Ort, wo wir vor \label{K_L02977-1v}\edtext{7 Jahren}{\lemma{\textnormal{\emph{7 Jahren}}}\Cendnote{\textnormal{siehe A. S.: \emph{Tagebuch}, 24. 8. 1895}}}\label{K_L02977-1h}{ }\textsc{etc\textcolor{gray}{.}}\pend
           \pstart Ihr \spacefill\mbox{Arthur}\pend{}
\pstart
           \noindent{}{[}hs. Hofmannsthal:{]} Gruss \spacefill\mbox{Hugo}\pend
           \endnumbering\briefempfaengerindex{Salten, Felix@\textsc{Salten, Felix}!zzzHofmannsthal, Hugo von@\emph{von Hugo von Hofmannsthal}!1902-07-011@{{[}1. 7. 1902{]}}|)be}\briefempfaengerindex{Salten, Felix@\textsc{Salten, Felix}!zzzSchnitzler, Arthur@\emph{von Arthur Schnitzler}!1902-07-011@{{[}1. 7. 1902{]}}|)be}\mylabel{h}  \normalsize

\doendnotes{C}
\bigskip
\vfill

\clearpage

\footnotesize

\lohead{\textsc{register}}

% Definiere theindex-Environment komplett neu ohne reledmac
\makeatletter
\renewenvironment{theindex}{%
  \section*{\indexname}%
  \setlength{\parindent}{0pt}%
  \setlength{\parskip}{0pt plus 0.3pt}%
  \let\item\@idxitem
}{%
  \clearpage
}
\makeatother

\IfFileExists{\jobname-pw.ind}{\input{\jobname-pw.ind}}{}

\end{document}

      