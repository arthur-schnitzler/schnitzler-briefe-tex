%% latex-korrekturansicht-vorspann.tex
%% Vorspann für die Korrekturansicht.
%% Lädt die gemeinsame Datei latex-vorspann.tex mit gesetztem Schalter.

\newif\ifkorrekturansicht
\korrekturansichttrue

\input{../tex-inputs/latex-vorspann}


\renewcommand{\erwaehntePersonen}{Personen: Gabriele D’Annunzio, Heinrich Kanner, Isidor Singer, Karl Gustav Vollmoeller}
\renewcommand{\erwaehnteInstitutionen}{Institutionen: Die Zeit}
\renewcommand{\erwaehnteOrte}{Orte: Edmund-Weiß-Gasse, I., Innere Stadt, Wien, Wipplingerstraße}
\renewcommand{\erwaehnteWerke}{Werke: Die Zeit, Giulia. Drama in einem Akt}
\section[Felix Salten an Arthur Schnitzler, 2. 7. 1904]{Felix Salten an Arthur Schnitzler, 2. 7. 1904}
\nopagebreak\mylabel{v}
\rehead{ }\normalsize\beginnumbering\briefempfaengerindex{Schnitzler, Arthur@\textsc{Schnitzler, Arthur}!zzzSalten, Felix@\emph{von Felix Salten}!1904-07-022@{2. 7. 1904}|(be}
\toendnotes[C]{\smallbreak\pagebreak[2]}\Standort{CUL, Schnitzler, B 89, B 1.}
\physDesc{Brief, 1 Blatt, 1 Seite
\newline{}maschinenschriftlich
\newline{}Handschrift: schwarze Tinte, lateinische Kurrent (\noindent{}ein Wortabstand eingefügt, Unterschrift und Nachschrift)
\newline{}Ordnung: mit Bleistift von unbekannter Hand nummeriert:
                                    »190« }\toendnotes[C]{\smallbreak}
\pstart
           \noindent{}{\pb}\textcolor{gray}{\textbf{DIE}}\hfill \textcolor{gray}{\textbf{\textcolor{pink}{WIEN, I.}{}\ledrightnote{\textcolor{pink}{I., Innere Stadt}}}}{ }2. Juli 1904\pend
           
\pstart
           \textcolor{gray}{\textbf{\textcolor{brown}{ZEIT}{}\ledrightnote{\textcolor{brown}{Die Zeit}}}}\hfill \textcolor{gray}{\textbf{\textcolor{pink}{Wipplingerstrasse 38}{}\ledrightnote{\textcolor{pink}{Wipplingerstraße}}}}\pend
           
\pstart
           \textcolor{gray}{\textbf{\textcolor{pink}{WIEN}{}\ledrightnote{\textcolor{pink}{Wien}}ER
                     TAGESZEITUNG}}\pend
           
\pstart
           \textcolor{gray}{\textbf{Herausgeber: }}\pend
           
\pstart
           \textcolor{gray}{\textbf{Prof. Dr. \textcolor{blue}{I. Singer}{}\ledrightnote{\textcolor{blue}{Isidor Singer}}}}\pend
           
\pstart
           \textcolor{gray}{\textbf{Dr. \textcolor{blue}{Heinrich Kanner}{}\ledrightnote{\textcolor{blue}{Heinrich Kanner}}}}\pend
           
\pstart
           \textcolor{gray}{\textbf{Redaction}}\pend
           
\pstart
           \textcolor{gray}{\textbf{Telegramm-Adresse: \textcolor{brown}{\so{Zeit}}{}\ledrightnote{\textcolor{brown}{Die Zeit}}\so{,{ }}\textcolor{pink}{\so{Wien}}{}\ledrightnote{\textcolor{pink}{Wien}}}}\pend
           
\pstart
           \textcolor{gray}{\textbf{Interurbanes Telephon Nr. 15.988}}\pend
           
\pstart
           \textcolor{gray}{\textbf{= Telephone Nr. 17.040, 17.041 =}}\pend
           
\pstart\center{}Lieber Freund!\pend
\pstart
           Den Einakter »\textcolor{green}{Giulia}{}\ledrightnote{\textcolor{green}{Giulia. Drama in einem Akt}}« von \textcolor{blue}{Artur Vollmöller}{}\ledrightnote{\textcolor{blue}{Karl Gustav Vollmoeller}} kann ich leider in der »\textcolor{green}{Zeit}{}\ledrightnote{\textcolor{green}{Die Zeit}}« nicht bringen. Die Situation lässt sich unmöglich vom
               Bett aus auf ein anderes Möbelstück verlegen. Das wäre aber noch das
                  wenigste{[}.{]} Ich kann der ganzen Arbeit keinen Geschmack
               abgewinnen; sie erscheint mir forciert, vollständig dem \textcolor{blue}{D’Annunzio}{}\ledrightnote{\textcolor{blue}{Gabriele D’Annunzio}} nachgebildet und unnötig. Ich glaube, dass \textcolor{blue}{Vollmöller}{}\ledrightnote{\textcolor{blue}{Karl Gustav Vollmoeller}} zuletzt doch eine Enttäuschung sein
               wird, ausser, man hat sich von ihm überhaupt nichts versprochen. \pend
           
\pstart
           Hoffentlich sind Sie bald wieder ganz gesund, ich schaue jedenfalls dieser Tage noch
               einmal zu Ihnen. \pend
           \pstart Herzlichst Ihr \spacefill\mbox{{[}hs.:{]} Salten}\pend{}
\pstart
           \noindent{}{[}ms.:{]} Herrn Dr. Arthur Schnitzler\pend
           
\pstart
           \textcolor{pink}{Wien, XVIII. Spöttelgasse 7}{}\ledrightnote{\textcolor{pink}{Edmund-Weiß-Gasse}}\pend
           
\pstart
           {[}hs.:{]} \label{K_L03398-1v}\edtext{1 Manuscript}{\lemma{\textnormal{\emph{1 Manuscript}}}\Cendnote{\textnormal{Beilage nicht erhalten}}}\label{K_L03398-1h}\pend
           \endnumbering\briefempfaengerindex{Schnitzler, Arthur@\textsc{Schnitzler, Arthur}!zzzSalten, Felix@\emph{von Felix Salten}!1904-07-022@{2. 7. 1904}|)be}\mylabel{h}
\begin{anhang}
\end{anhang}\normalsize

\doendnotes{C}
\bigskip
\vfill

\clearpage

\footnotesize

\lohead{\textsc{register}}

% Definiere theindex-Environment komplett neu ohne reledmac
\makeatletter
\renewenvironment{theindex}{%
  \section*{\indexname}%
  \setlength{\parindent}{0pt}%
  \setlength{\parskip}{0pt plus 0.3pt}%
  \let\item\@idxitem
}{%
  \clearpage
}
\makeatother

\IfFileExists{\jobname-pw.ind}{\input{\jobname-pw.ind}}{}

\end{document}

      