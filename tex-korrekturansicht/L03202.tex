%% latex-korrekturansicht-vorspann.tex
%% Vorspann für die Korrekturansicht.
%% Lädt die gemeinsame Datei latex-vorspann.tex mit gesetztem Schalter.

\newif\ifkorrekturansicht
\korrekturansichttrue

\input{../tex-inputs/latex-vorspann}


\renewcommand{\erwaehnteOrte}{Orte: Frankgasse, Hotel Blauer Stern, Prag, Prager Burg, Wien}
\renewcommand{\erwaehnteWerke}{}
\section[ Paul Goldmann an Arthur Schnitzler, 29. 3. 1902]{Paul Goldmann an Arthur Schnitzler, 29. 3. 1902}
\nopagebreak\mylabel{v}
\rehead{ }\normalsize\beginnumbering\briefempfaengerindex{Schnitzler, Arthur@\textsc{Schnitzler, Arthur}!zzzGoldmann, Paul@\emph{von Paul Goldmann}!1902-03-292@{29. 3. 1902}|(be}
\toendnotes[C]{\smallbreak\pagebreak[2]}\Standort{DLA, A:Schnitzler, HS.NZ85.1.3172.}
\physDesc{Bildpostkarte
\newline{}Handschrift: 1) schwarze Tinte, deutsche Kurrent\hspace{1em}2) schwarze Tinte, lateinische Kurrent (\noindent{}Adresse)\hspace{1em}
\newline{}Versand: 1) Stempel: »\nobreak{}\oindex{Prag@\textbf{Prag}, \emph{Besiedelter Ort (A.BSO)}|pwk}Praha 1 * Prag 1, 29. \textcolor{gray}{I}II. 0\textcolor{gray}{2}\nobreak{}«.   2) Stempel: »\nobreak{}9/3 Wien 7{[}2{]}, 30. 3{[}. 02{]}, 9. V, Beste{[}llt{]}\nobreak{}«. }\toendnotes[C]{\smallbreak}\pstart{}{\pb}Herrn Dr. Arthur Schnitzler\pend{}\pstart{}\textcolor{pink}{Wien}{}\ledrightnote{\textcolor{pink}{Wien}}\pend{}\pstart{}\textcolor{pink}{IX. Frankgaſse 1}{}\ledrightnote{\textcolor{pink}{Frankgasse}}.\pend{}
{\bigskip}
\pstart
           \noindent{}{\pb}\textcolor{gray}{\textbf{Gruss aus \textcolor{pink}{Prag}{}\ledrightnote{\textcolor{pink}{Prag}}!}}\hfill \textcolor{gray}{\textbf{Totalansicht von der \textcolor{pink}{Burg}{}\ledrightnote{{$\rightarrow$}\textcolor{pink}{Prager Burg}} aus – 61.}}\pend
           
\pstart
           29. 3. 1902.\pend
           
\pstart
           Fröhliche Feiertage, mein lieber Freund, und viele
               Grüße! Es iſt ſehr ſchade, daß Du nicht hier biſt. Dein treuer {\\}\spacefill\mbox{Paul Goldmann}\pend
           
\pstart
           \noindent{}\textsc{\textcolor{pink}{Hotel »Blauer Stern«}{}\ledrightnote{\textcolor{pink}{Hotel Blauer Stern}}}\textcolor{gray}{.}\pend
           \endnumbering\briefempfaengerindex{Schnitzler, Arthur@\textsc{Schnitzler, Arthur}!zzzGoldmann, Paul@\emph{von Paul Goldmann}!1902-03-292@{29. 3. 1902}|)be}\mylabel{h}
\begin{anhang}
\end{anhang}\normalsize

\doendnotes{C}
\bigskip
\vfill

\clearpage

\footnotesize

\lohead{\textsc{register}}

% Definiere theindex-Environment komplett neu ohne reledmac
\makeatletter
\renewenvironment{theindex}{%
  \section*{\indexname}%
  \setlength{\parindent}{0pt}%
  \setlength{\parskip}{0pt plus 0.3pt}%
  \let\item\@idxitem
}{%
  \clearpage
}
\makeatother

\IfFileExists{\jobname-pw.ind}{\input{\jobname-pw.ind}}{}

\end{document}

      