%% latex-korrekturansicht-vorspann.tex
%% Vorspann für die Korrekturansicht.
%% Lädt die gemeinsame Datei latex-vorspann.tex mit gesetztem Schalter.

\newif\ifkorrekturansicht
\korrekturansichttrue

\input{../tex-inputs/latex-vorspann}


               \section[Engelbert Pernerstorfer und Stefan Großmann an Arthur Schnitzler, 14. 3. 1911]{ Engelbert Pernerstorfer und Stefan Großmann an Arthur Schnitzler,
                    14. 3. 1911}\nopagebreak\mylabel{v}\rehead{ }\normalsize\beginnumbering\briefempfaengerindex{Schnitzler, Arthur@\textsc{Schnitzler, Arthur}!zzzGrossmann, Stefan@\emph{von Stefan Großmann}!1911-03-141@{14. 3. 1911}|(be}\briefempfaengerindex{Schnitzler, Arthur@\textsc{Schnitzler, Arthur}!zzzPernerstorfer, Engelbert@\emph{von Engelbert Pernerstorfer}!1911-03-141@{14. 3. 1911}|(be} \toendnotes[C]{\smallbreak\pagebreak[2]} \Standort{CUL, Schnitzler, B 34.}
\physDesc{Brief, 1 Blatt, 1 Seite
\newline{}Schreibmaschine
\newline{}Handschrift Engelbert Pernerstorfer: schwarze Tinte\newline{}Handschrift Stefan Großmann: schwarze Tinte
\newline{}Schnitzler: 1) mit rotem Buntstift eine Unterstreichung 2) mit Bleistift beschriftet: »\textsc{Großma{\geminationn}}« und mit einer – nur unsicher lesbaren – Skizze der Antwort versehen: »\noindent{}{[}({]}bed ſehr, durch Arbeit in Anſpruch gen ein ſptr Zeitp für
                     \textcolor{gray}{Mit}arb von mir).«\newline{}Ordnung: mit Bleistift von unbekannter Hand nummeriert:
                                                »10« }\toendnotes[C]{\smallbreak}\pstart
           \noindent{}{\pb}\textcolor{gray}{\textbf{\textcolor{brown}{FREIE VOLKSBÜHNE}{}\ledrightnote{\textcolor{brown}{Wiener Freie Volksbühne}}}}\pend
           \pstart
           \textcolor{gray}{\textbf{SEKRETARIAT: \textcolor{pink}{V/2,
                                    SCHÖNBRUNNERSTRASSE 124}{}\ledrightnote{\textcolor{pink}{Schönbrunnerstraße}}}}\pend
           \pstart
           \textcolor{gray}{\textbf{Kanzleistunden: (nur an Wochentagen): Vom 1. September
                            bis 31. Mai von 9 bis 12 Uhr vormittags und von 4 bis 8 Uhr abends. Vom
                            1. Juni bis 31. August von 9 Uhr vormittags bis 4 Uhr nachmittags}}\pend
           \pstart
           \textcolor{gray}{\textbf{Telephon: Nr. 7582}}\hfill \textcolor{gray}{\textbf{Postsparkassen-Konto: Nr. 87.544}}\pend
           \pstart
           \raggedleft{}\textcolor{gray}{\textbf{\textcolor{pink}{WIEN}{}\ledrightnote{\textcolor{pink}{Wien}}, den}}{ }14. März \textcolor{gray}{\textbf{191}}1.\pend
           \pstart\center{}Sehr verehrter Herr!\pend\pstart
           Die \textcolor{brown}{Freie Volksbühne}{}\ledrightnote{\textcolor{brown}{Wiener Freie Volksbühne}} will ihre
                    kunst-pädagogische Tätigkeit dadurch ergänzen, dass sie ihren Mitgliedern
                    regelmässig für 10 hl eine \textcolor{green}{Zeitschrift}{}\ledrightnote{→\textcolor{green}{Der Strom. Organ der Wiener Freien Volksbühne}} in die Hand gibt, die für die stille Wirkung im eigenen
                    Heim des Mitgliedes bestimmt ist. Sie würden unsere Ziele fördern, wenn Sie uns
                    schon für die ersten Hefte irgend einen Beitrag novellistischen Charakters, oder
                    auch ein Gedicht zur Verfügung stellten. Die besten Namen Deutschlands stehen
                    uns in dieser Arbeit zur Seite und wir haben die feste Hoffnung, dass auch Sie
                    verehrter Herr uns bei diesem neuen Zweige unserer Tätigkeit helfen werden.\pend
           \pstart
           Das erste \textcolor{green}{Heft}{}\ledrightnote{→\textcolor{green}{Der Strom. Organ der Wiener Freien Volksbühne}} soll
                        Ende März erscheinen und aus diesem Grunde erbitten wir eine
                    umgehende Antwort unseres Briefes.\pend
           \pstart
           Mit aufrichtiger Hochschätzung{\\[\baselineskip]} sehr ergeben{\\[\baselineskip]}\spacefill\mbox{Pernerstorfer}\hspace*{1.5em}\spacefill\mbox{Stefan Großmann}\pend
           \leftskip=0em{}\endnumbering\briefempfaengerindex{Schnitzler, Arthur@\textsc{Schnitzler, Arthur}!zzzGrossmann, Stefan@\emph{von Stefan Großmann}!1911-03-141@{14. 3. 1911}|)be}\briefempfaengerindex{Schnitzler, Arthur@\textsc{Schnitzler, Arthur}!zzzPernerstorfer, Engelbert@\emph{von Engelbert Pernerstorfer}!1911-03-141@{14. 3. 1911}|)be}\mylabel{h}  \normalsize

\doendnotes{C}
\bigskip
\vfill

\clearpage

\footnotesize

\lohead{\textsc{register}}

% Definiere theindex-Environment komplett neu ohne reledmac
\makeatletter
\renewenvironment{theindex}{%
  \section*{\indexname}%
  \setlength{\parindent}{0pt}%
  \setlength{\parskip}{0pt plus 0.3pt}%
  \let\item\@idxitem
}{%
  \clearpage
}
\makeatother

\IfFileExists{\jobname-pw.ind}{\input{\jobname-pw.ind}}{}

\end{document}

      