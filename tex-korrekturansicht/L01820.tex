%% latex-korrekturansicht-vorspann.tex
%% Vorspann für die Korrekturansicht.
%% Lädt die gemeinsame Datei latex-vorspann.tex mit gesetztem Schalter.

\newif\ifkorrekturansicht
\korrekturansichttrue

\input{../tex-inputs/latex-vorspann}


               \section[Arthur Schnitzler an Richard Beer-Hofmann, 7. 1. 1909]{ Arthur Schnitzler an Richard Beer-Hofmann, 7. 1. 1909}\nopagebreak\mylabel{v}\rehead{ }\normalsize\beginnumbering\briefempfaengerindex{Beer-Hofmann, Richard@\textsc{Beer-Hofmann, Richard}!zzzSchnitzler, Arthur@\emph{von Arthur Schnitzler}!1909-01-071@{7. 1. 1909}|(be} \toendnotes[C]{\smallbreak\pagebreak[2]} \Standort{YCGL, MSS 31.}
\physDesc{Brief, 1 Blatt (Briefpapier mit Trauerrand), 1 Seite, Umschlag
\newline{}Handschrift: schwarze Tinte, deutsche Kurrent\newline{}Beilage: Visitenkarte, schwarze Tinte \newline{}Versand: ohne postalischen Übermittlungsvermerk }\buchAbdrucke{\weitereDrucke{Arthur Schnitzler, Richard Beer-Hofmann: \emph{Briefwechsel 1891–1931}. Hg. Konstanze Fliedl. Wien, Zürich: \emph{Europaverlag} 1992, S. 193.} }\toendnotes[C]{\smallbreak}\pstart{}{\pb}\textcolor{gray}{\textbf{Dr. Arthur Schnitzler}}\pend{}\pstart{}\textcolor{gray}{\textbf{\textcolor{pink}{Wien XVIII. Spoettelgasse 7}{}\ledrightnote{\textcolor{pink}{Edmund-Weiß-Gasse}}.}}\pend{}{\bigskip}\pstart{}{\pb}Herrn \textsc{Dr. Richard Beer
                     Hofmann}\pend{}\pstart{}\textcolor{pink}{\textsc{Wien XVIII}}{}\ledrightnote{\textcolor{pink}{XVIII., Währing}}\pend{}\pstart{}\textcolor{pink}{\textsc{Hasenauerstr 59}}{}\ledrightnote{\textcolor{pink}{Hasenauerstraße}}.\pend{}{\bigskip}\pstart
           \noindent{}\centering{}{\pb}\textcolor{gray}{\textbf{D\textsuperscript{r} Arthur Schnitzler}}\pend
           \pstart
           \noindent{}\raggedleft{}\textcolor{gray}{\textbf{\textcolor{pink}{Wien}{}\ledrightnote{\textcolor{pink}{Wien}}}}\pend
           \pstart
           \noindent{}{\pb}Einladg zur \uline{\textcolor{green}{\label{K_L01820_1v}\edtext{Generalprobe}{\lemma{\textnormal{\emph{Generalprobe}}}\Cendnote{\textnormal{\emph{\textcolor{green}{Anatols Hochzeitsmorgen}} wurde am 10. 1. 1909 im Zuge
                        einer Matinée gegeben.}}}\label{K_L01820_1h}}{}\ledrightnote{→\textcolor{green}{Anatols Hochzeitsmorgen}} im \textcolor{pink}{\textsc{Johann Strauss} Theater}{}\ledrightnote{\textcolor{pink}{Johann Strauß-Theater}}}{ }\uline{Samſtag den 9. 1.} für Dr. \textsc{Richard Beer-Hofmann} und Frau \textcolor{blue}{Gemahlin}{}\ledrightnote{→\textcolor{blue}{Paula Beer-Hofmann}}.\pend
           {\bigskip}\pstart
           \noindent{}{\pb}\textcolor{gray}{\textbf{Dr. Arthur Schnitzler}}\hfill 7. 1. 09.\pend
           \pstart
           \textcolor{gray}{\textbf{\textcolor{pink}{Wien XVIII. Spoettelgasse 7}{}\ledrightnote{\textcolor{pink}{Edmund-Weiß-Gasse}}.}}\pend
           \pstart
           Dies, lieber Richard genügt u eigentlich brauchen Sie auch das nicht. \textcolor{green}{Beginn}{}\ledrightnote{→\textcolor{green}{Anatols Hochzeitsmorgen}} (angeblich)
               10. Eingang Bühnenthürl. Herzlichſt Ihr\pend
           \pstart \spacefill\mbox{Arthur.}\pend{}\endnumbering\briefempfaengerindex{Beer-Hofmann, Richard@\textsc{Beer-Hofmann, Richard}!zzzSchnitzler, Arthur@\emph{von Arthur Schnitzler}!1909-01-071@{7. 1. 1909}|)be}\mylabel{h}  \normalsize

\doendnotes{C}
\bigskip
\vfill

\clearpage

\footnotesize

\lohead{\textsc{register}}

% Definiere theindex-Environment komplett neu ohne reledmac
\makeatletter
\renewenvironment{theindex}{%
  \section*{\indexname}%
  \setlength{\parindent}{0pt}%
  \setlength{\parskip}{0pt plus 0.3pt}%
  \let\item\@idxitem
}{%
  \clearpage
}
\makeatother

\IfFileExists{\jobname-pw.ind}{\input{\jobname-pw.ind}}{}

\end{document}

      