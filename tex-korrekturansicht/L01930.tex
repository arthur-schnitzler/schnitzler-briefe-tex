%% latex-korrekturansicht-vorspann.tex
%% Vorspann für die Korrekturansicht.
%% Lädt die gemeinsame Datei latex-vorspann.tex mit gesetztem Schalter.

\newif\ifkorrekturansicht
\korrekturansichttrue

\input{../tex-inputs/latex-vorspann}


               \section[Arthur Schnitzler an Hugo von Hofmannsthal, 15. 5. 1910]{ Arthur Schnitzler an Hugo von Hofmannsthal, 15. 5. 1910}\nopagebreak\mylabel{v}\rehead{ }\normalsize\beginnumbering\briefempfaengerindex{Hofmannsthal, Hugo von@\textsc{Hofmannsthal, Hugo von}!zzzSchnitzler, Arthur@\emph{von Arthur Schnitzler}!1910-05-151@{5. 5. 1910}|(be} \toendnotes[C]{\smallbreak\pagebreak[2]} \Standort{FDH, Hs-30885,137.}
\physDesc{Kartenbrief
\newline{}Handschrift: schwarze Tinte, deutsche Kurrent\newline{}Versand: 1) Stempel: »\nobreak{}\oindex{IX., Alsergrund@\textbf{IX., Alsergrund}, \emph{Bezirk (A.BZK)}|pwk}9/4 Wien, 15. V. 10, 6\nobreak{}«.  2) Stempel: »\nobreak{}\oindex{Rodaun@\textbf{Rodaun}, \emph{Teil eines besiedelten Ortes (A.BSOX)}|pwk}Rodaun, 16. V. 10, 6\nobreak{}«. }\buchAbdrucke{\weitereDrucke{Hugo von Hofmannsthal, Arthur Schnitzler: \emph{Briefwechsel}. Hg. Therese Nickl und Heinrich Schnitzler. Frankfurt am Main: \emph{S. Fischer} 1964, S. 250.} }\toendnotes[C]{\smallbreak}\pstart{}{\pb}Herrn \textsc{Dr Hugo von
                     Hofmannsthal}\pend{}\pstart{}\textcolor{pink}{Rodaun}{}\ledrightnote{\textcolor{pink}{Rodaun}}\pend{}\pstart{}\textcolor{pink}{Badgaſſe 5}{}\ledrightnote{\textcolor{pink}{Badgasse}}.\pend{}{\bigskip}\pstart
           \raggedleft{}{\pb}15/5 910\pend
           \pstart{}lieber Hugo, \pend\pstart
           ich gratulire herzlich; es war ein ſchöner \label{K_L01930_1v}\edtext{Abend}{\lemma{\textnormal{\emph{Abend}}}\Cendnote{\textnormal{vgl. A. S.: \emph{Tagebuch}, 13. 5. 1910}}}\label{K_L01930_1h}. Die \textcolor{green}{Umarbeitung}{}\ledrightnote{→\textcolor{green}{Cristinas Heimreise. Komödie}} find ich in der
               Anlage famos, aber an einzelnen Stellen noch nicht vollko{\geminationm}en fertig. Vielleicht iſt es nur ein halbes Dutzend Worte der \textcolor{green}{\textsc{Cristina}}{}\ledrightnote{→\textcolor{green}{Cristinas Heimreise. Komödie}}, die mir
               fehlen – und vielleicht fehlen ſie mir nur, weil ich von dieſer anmutvollen Geſtalt
               noch irgend etwas vernehmen möchte, eh ſie aus der ſchönen Welt dieſer \textcolor{green}{Komödie}{}\ledrightnote{→\textcolor{green}{Cristinas Heimreise. Komödie}}{ }ſcheidet.\pend
           \pstart
           Wir reiſen Dinſtag in die \textcolor{pink}{Schweiz}{}\ledrightnote{\textcolor{pink}{Schweiz}} auf
               circa 3 Wochen. Und ſehen \substVorne{}\textsuperscript{uns}\substDazwischen{}Sie\substHinten{} hoffentlich bald nach unſrer Rückkehr.\pend
           \pstart Viele Grüße von Haus zu Haus Ihr \spacefill\mbox{A.}\pend{}\endnumbering\briefempfaengerindex{Hofmannsthal, Hugo von@\textsc{Hofmannsthal, Hugo von}!zzzSchnitzler, Arthur@\emph{von Arthur Schnitzler}!1910-05-151@{5. 5. 1910}|)be}\mylabel{h}  \normalsize

\doendnotes{C}
\bigskip
\vfill

\clearpage

\footnotesize

\lohead{\textsc{register}}

% Definiere theindex-Environment komplett neu ohne reledmac
\makeatletter
\renewenvironment{theindex}{%
  \section*{\indexname}%
  \setlength{\parindent}{0pt}%
  \setlength{\parskip}{0pt plus 0.3pt}%
  \let\item\@idxitem
}{%
  \clearpage
}
\makeatother

\IfFileExists{\jobname-pw.ind}{\input{\jobname-pw.ind}}{}

\end{document}

      