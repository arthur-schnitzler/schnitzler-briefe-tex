%% latex-korrekturansicht-vorspann.tex
%% Vorspann für die Korrekturansicht.
%% Lädt die gemeinsame Datei latex-vorspann.tex mit gesetztem Schalter.

\newif\ifkorrekturansicht
\korrekturansichttrue

\input{../tex-inputs/latex-vorspann}


\renewcommand{\erwaehntePersonen}{Personen:  Raffaello Sanzio da Urbino, Felix Salten}
\renewcommand{\erwaehnteOrte}{Orte: Bologna, Edmund-Weiß-Gasse 7, Florenz, Pinacoteca Nazionale di Bologna, Ravenna, Rimini, Wien, Österreich}
\renewcommand{\erwaehnteWerke}{Werke: Der Schleier der Beatrice. Schauspiel in fünf Akten, Die Verzückung der Heiligen Cäcilia, Die Zeit, Unsichere Reise}
\section[ Felix Salten an Arthur Schnitzler, 25. 4. 1908]{Felix Salten an Arthur Schnitzler, 25. 4. 1908}
\nopagebreak\mylabel{v}
\rehead{ }\normalsize\beginnumbering\briefempfaengerindex{Schnitzler, Arthur@\textsc{Schnitzler, Arthur}!zzzSalten, Felix@\emph{von Felix Salten}!1908-04-251@{25. 4. 1908}|(be}
\toendnotes[C]{\smallbreak\pagebreak[2]}\Standort{CUL, Schnitzler, B 89, B 1.}
\physDesc{Bildpostkarte, 188 Zeichen
\newline{}Handschrift: schwarze Tinte, lateinische Kurrent
\newline{}Versand: Stempel: »\nobreak{}\oindex{Bologna@\textbf{Bologna}, \emph{P.PPLA}|pwk}\textcolor{gray}{Bologna}, 25{[}. 4. 1908{]}\nobreak{}«.  
\newline{}Ordnung: mit Bleistift von unbekannter Hand nummeriert: »244« }\toendnotes[C]{\smallbreak}\pstart{}{\pb}\textcolor{pink}{Vienna}{}\ledrightnote{\textcolor{pink}{Wien}}{ }\textcolor{pink}{Austria}{}\ledrightnote{\textcolor{pink}{Österreich}}\pend{}\pstart{}Herrn D\textsuperscript{r} Arthur Schnitzler\pend{}\pstart{}\textcolor{pink}{Wien}{}\ledrightnote{\textcolor{pink}{Wien}}\pend{}\pstart{}\textcolor{pink}{XVIII. Spöttelgaße 7}{}\ledrightnote{\textcolor{pink}{Edmund-Weiß-Gasse 7}}\pend{}
{\bigskip}
\pstart
           \noindent{}\centering{}{\pb}\textcolor{gray}{\textbf{\textcolor{pink}{BOLOGNA – R. Pinacoteca}{}\ledrightnote{\textcolor{pink}{Pinacoteca Nazionale di Bologna}}. \textcolor{green}{S. Cecilia}{}\ledrightnote{\textcolor{green}{Die Verzückung der Heiligen Cäcilia}} (\textcolor{blue}{Raffaello
                        Sanzio}{}\ledrightnote{\textcolor{blue}{Raffaello Sanzio da Urbino}})}}\pend
           
\pstart
           {\pb}»\textcolor{green}{Das Leben ist die Fülle, nicht die Zeit {\dotstwo}}{}\ledrightnote{{$\rightarrow$}\textcolor{green}{Der Schleier der Beatrice. Schauspiel in fünf Akten}}« \pend
           
\pstart
           Aus einem \label{K_L03495-1v}\edtext{\textcolor{green}{Drama}{}\ledrightnote{{$\rightarrow$}\textcolor{green}{Der Schleier der Beatrice. Schauspiel in fünf Akten}}}{\lemma{\textnormal{\emph{Drama}}}\Cendnote{\textnormal{\emph{\textcolor{green}{Der Schleier der Beatrice}}; das Zitat sind die
                  Schlussworte des \textcolor{green}{Herzog}s}}}\label{K_L03495-1h}, das hier in \label{K_L03495-2v}\edtext{\textcolor{pink}{Bologna}{}\ledrightnote{\textcolor{pink}{Bologna}}}{\lemma{\textnormal{\emph{Bologna}}}\Cendnote{\textnormal{Am Ende seines
                     Feuilletons \emph{\textcolor{green}{Unsichere Reise}} (\textcolor{blue}{Felix Salten}: \emph{\textcolor{green}{Unsichere Reise}}. In: \emph{\textcolor{green}{Die Zeit}}, 
                        Jg. 7, Nr. 2.008, 26. 4. 1908, Morgenblatt, S. 1–3, hier 3.) überlegt der Erzähler/\textcolor{blue}{Salten} noch,
                     ob er tatsächlich weiter nach \textcolor{pink}{Bologna} und \textcolor{pink}{Florenz} soll und nicht eine andere Route wählen und
                     nach \textcolor{pink}{Ravenna} und \textcolor{pink}{Rimini} zu gehen,
                    wo er noch
                  nie war.}}}\label{K_L03495-2h}
               spielt, mit herzlichen Grüßen {\\}Ihr {\\}\spacefill\mbox{Salten}\pend
           
\pstart
           25./4. 08\pend
           \endnumbering\briefempfaengerindex{Schnitzler, Arthur@\textsc{Schnitzler, Arthur}!zzzSalten, Felix@\emph{von Felix Salten}!1908-04-251@{25. 4. 1908}|)be}\mylabel{h}  \normalsize

\doendnotes{C}
\bigskip
\vfill

\clearpage

\footnotesize

\lohead{\textsc{register}}

% Definiere theindex-Environment komplett neu ohne reledmac
\makeatletter
\renewenvironment{theindex}{%
  \section*{\indexname}%
  \setlength{\parindent}{0pt}%
  \setlength{\parskip}{0pt plus 0.3pt}%
  \let\item\@idxitem
}{%
  \clearpage
}
\makeatother

\IfFileExists{\jobname-pw.ind}{\input{\jobname-pw.ind}}{}

\end{document}

      