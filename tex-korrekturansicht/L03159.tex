%% latex-korrekturansicht-vorspann.tex
%% Vorspann für die Korrekturansicht.
%% Lädt die gemeinsame Datei latex-vorspann.tex mit gesetztem Schalter.

\newif\ifkorrekturansicht
\korrekturansichttrue

\input{../tex-inputs/latex-vorspann}


\renewcommand{\erwaehntePersonen}{Personen: Richard Beer-Hofmann, Paul Goldmann, Hugo von Hofmannsthal, Caroline Kotter, Maria Charlotte Lamberg, Alexander Wilhelm Neuman, Charlotte Pohl-Glas, Moritz Rosenthal, Hans Schnitzler, Julius Schnitzler, Helene Schnitzler}
\renewcommand{\erwaehnteInstitutionen}{Institutionen: Frankfurter Zeitung}
\renewcommand{\erwaehnteOrte}{Orte: Bad Ischl, Glaspalast, Hotel und Pension Rudolfshöhe (Leopold Petter), Kopenhagen, München, Paris, Rügen, Sylt, Szczecin, Wien}
\renewcommand{\erwaehnteWerke}{Werke: Der Tod Georgs, Die Münchener Kunstausstellungen. I. Im königl. Glaspalast, Die Münchener Kunstausstellungen. II. Im königl. Glaspalast, Münchener Brief. (Orig.-Corr. der »Wiener Allg. Ztg.«), Quer durch den Wurstelprater, Wiener Allgemeine Zeitung}
\section[ Felix Salten an Arthur Schnitzler, 22. 7. 1895]{Felix Salten an Arthur Schnitzler, 22. 7. 1895}
\nopagebreak\mylabel{v}
\rehead{ }\normalsize\beginnumbering\briefempfaengerindex{Schnitzler, Arthur@\textsc{Schnitzler, Arthur}!zzzSalten, Felix@\emph{von Felix Salten}!1895-07-221@{22. 7. 1895}|(be}
\toendnotes[C]{\smallbreak\pagebreak[2]}\Standort{CUL, Schnitzler, B 89, A 1.}
\physDesc{Brief, 1 Blatt, 4 Seiten, 3989 Zeichen
\newline{}Handschrift: Bleistift, lateinische Kurrent
\newline{}Ordnung: mit Bleistift von unbekannter Hand nummeriert: »58« }\toendnotes[C]{\smallbreak}
\pstart
           \raggedleft{}{\pb}22. Juli 95.\pend
           
\pstart
           Lieber Freund! Das war sehr lieb von Ihnen, dass Sie
               mir mittheilten, ich werde oft gelobt, es hat mich sehr gefreut, denn ich begreife
               immer mehr, dass der \textcolor{blue}{Hugo}{}\ledrightnote{\textcolor{blue}{Hugo von Hofmannsthal}} recht hat, wenn er
               sagt: »Ich möcht mehr g’lobt werd’n.« Sie können sich vorstellen, welches Gewicht ich
               auf das Urtheil von \textcolor{blue}{Neumann}{}\ledrightnote{\textcolor{blue}{Alexander Wilhelm Neuman}} lege. Jetzt erst
               glaube ich, dass ich doch etwas kann. Ich habe mir jetzt meine \label{K_L03159-1v}\edtext{\textcolor{green}{Feuilletons}{}\ledrightnote{{$\rightarrow$}\textcolor{green}{Die Münchener Kunstausstellungen. I. Im königl. Glaspalast}{\newline}{$\rightarrow$}\textcolor{green}{Die Münchener Kunstausstellungen. II. Im königl. Glaspalast}{\newline}{$\rightarrow$}\textcolor{green}{Münchener Brief. (Orig.-Corr. der »Wiener Allg. Ztg.«)}}}{\lemma{\textnormal{\emph{Feuilletons}}}\Cendnote{\textnormal{Er dürfte sich im Besonderen auf die
                  aktuellen Texte über \textcolor{pink}{München}er
                  Kunstausstellungen im \textcolor{pink}{Glaspalast} beziehen: \textcolor{blue}{f. s.}: \emph{\textcolor{green}{Münchener Brief. (Orig.-Corr. der »Wiener Allg. Ztg.«)}}. In: \emph{\textcolor{green}{Wiener Allgemeinen Zeitung}}, Nr. 5.200,
                        6. 7. 1895, S. 8. \textcolor{blue}{Felix Salten}: \emph{\textcolor{green}{Die Münchener Kunstausstellungen. I. Im königl.
                        Glaspalast}}. In: \emph{\textcolor{green}{Wiener Allgemeinen
                        Zeitung}}, Nr. 5.215, 24. 7. 1895,
                     S. 2. \textcolor{blue}{Felix Salten}: \textcolor{green}{Die Münchener Kunstausstellungen. II. Im
                        königl. Glaspalast}. In: \emph{\textcolor{green}{Wiener
                        Allgemeinen Zeitung}}, Nr. 5.216, 25. 7. 1895, S. 2–3. Die Zusammenstellung für \textcolor{blue}{Goldmann} dürfte erfolgen, weil dieser als
                  Korrespondent für die \emph{\textcolor{brown}{Frankfurter Zeitung}} in
                     \textcolor{pink}{Paris} tätig war.}}}\label{K_L03159-1h} zusammenstellen
               laßen, und schicke sie Ihnen morgen. Wählen Sie davon
               welche aus, und senden Sie das an \textcolor{blue}{Goldmann}{}\ledrightnote{\textcolor{blue}{Paul Goldmann}}
               weiter, ja? Dass ich \label{K_L03159-2v}\edtext{\textcolor{blue}{Beer Hofmann}{}\ledrightnote{\textcolor{blue}{Richard Beer-Hofmann}} nichts geschrieben}{\lemma{\textnormal{\emph{Beer … geschrieben}}}\Cendnote{\textnormal{Der Versand des Zeitungsabdrucks \emph{\textcolor{green}{Quer durch den Wurstelprater}}, Felix Salten an Arthur Schnitzler, 16. 7. [1895] war also ohne Begleitschreiben
                  erfolgt.}}}\label{K_L03159-2h} habe, soll nicht missdeutet werden. Zu einem Brief lag rein
               äußerlich nichts vor, und ihm auf den \textcolor{green}{Wurstelprater}{}\ledrightnote{\textcolor{green}{Quer durch den Wurstelprater}} eine Widmung schreiben, mochte ich nicht, weil ich ja nicht
               wusste, wie ihm der \textcolor{green}{Wurstelprater}{}\ledrightnote{\textcolor{green}{Quer durch den Wurstelprater}} gefallen
               werde, und weil, – nun Sie wissen ja dass ich da vielleicht ein bisschen zu sehr
               empfindlich bin. Ich weiss ja auch heute nicht, ob er
                  \textcolor{gray}{w}as davon hält, und so konnte ich ihm bis heute nichts schreiben. Übrigens vermuthe ich, dass er
               ihm nicht gefallen hat, weil Sie mir das sonst sicher geschrieben hätten. Dabei kann
               ich aber nicht begreifen, {\pb}seit wann wir uns das \uline{nicht} mittheilen. Das Sie
               einen kleinen \label{K_L03159-3v}\edtext{\textcolor{blue}{Neffe}{}\ledrightnote{{$\rightarrow$}\textcolor{blue}{Hans Schnitzler}}n }{\lemma{\textnormal{\emph{Neffen }}}\Cendnote{\textnormal{\textcolor{blue}{Hans Schnitzler}, der gemeinsame Sohn von
                     \textcolor{blue}{Julius} und \textcolor{blue}{Helene Schnitzler}, war am 11. 7. 1895 geboren worden.}}}\label{K_L03159-3h} haben, wusste ich, aber das kann mir
               doch nicht imponiren, da ich doch zwei \textcolor{blue}{Töchter}{}\ledrightnote{{$\rightarrow$}\textcolor{blue}{Maria Charlotte Lamberg}{\newline}{$\rightarrow$}\textcolor{blue}{Caroline Kotter}} habe! Übrigens habe ich jetzt wieder acht
               Schreckenstage mitgemacht. Ich bin nämlich einmal doch erlegen, und so kamen dann die
               acht langen Tage. Endlich erschien die Gefahr doch beseitigt und ich atmete auf. Es
               wäre wirklich zu schrecklich gewesen. Übrigens verbringe ich nach dieser Seite hin
               arge Tage. Scenen, Scenen, Scenen. Wie einem \uline{da} zu
               Muthe wird, können nicht einmal Sie recht wissen. Es gibt gegenwärtig, besonders aber
                  heute, keine Frau, die mir unausstehlicher wäre als
                  \label{K_L03159-4v}\edtext{meine \textcolor{blue}{Geliebte}{}\ledrightnote{{$\rightarrow$}\textcolor{blue}{Charlotte Pohl-Glas}}}{\lemma{\textnormal{\emph{meine Geliebte}}}\Cendnote{\textnormal{wohl \textcolor{blue}{Charlotte Glas}}}}\label{K_L03159-4h}. Sie hat übrigens gestern, als wir eine Stunde
               lang wortlos und wüthend nebeneinander saßen, plötzlich gesagt: »Uns sollte man mit
               Knütteln auseinander jagen.« O, wie recht! Wir sind übrigens in ein Stadium getreten,
               in welchem jeder Streit sofort ausartet und nicht wieder gutzumachende gegenseitige
               Beschimpfungen hervorruft, ich thue nichts, um das zu mildern\textcolor{gray}{,}{ }{\pb}und könnte es auch nicht.
               Intensiv denke ich ans Fortreisen, wo ich denn durch Ruhe und lieberen Umgang mich zu
               erholen, und \textcolor{blue}{ihr}{}\ledrightnote{{$\rightarrow$}\textcolor{blue}{Charlotte Pohl-Glas}} durch Briefe unsere Nichtzusammengehörigkeit eindringlich vorzustellen
               beabsichtige. Dass \textcolor{blue}{B.-H.}{}\ledrightnote{\textcolor{blue}{Richard Beer-Hofmann}} erst \label{K_L03159-5v}\edtext{Anfangs September fahren}{\lemma{\textnormal{\emph{Anfangs September fahren}}}\Cendnote{\textnormal{Bezug auf die gemeinsame Sommerreise, siehe Felix Salten an Arthur Schnitzler, 16. 7. [1895]}}}\label{K_L03159-5h} will ist fatal, aber da er den »\textcolor{green}{Götterliebling}{}\ledrightnote{\textcolor{green}{Der Tod Georgs}}« fertig macht, läßt sich nichts thun, das ist jedenfalls
               wichtiger, und wenn er im Herbste erscheinen will soll er doch dazu schauen, noch
               diesen Monat (August) fertig zu werden. Mit mir steht die
               Sache so: Ich kann den 13. od. 14. August fort; \uline{muss} aber jedesfalls den
                  1. September zurück sein. Wenn wir zusammen reisen,
               dann müssten Sie sich \uline{längstens} bis 1. Aug. entschloßen haben, damit ich mich danach
               einrichten kann. Für diesen Fall käme ich \uline{nicht} nach
                  \textcolor{pink}{Ischl}{}\ledrightnote{\textcolor{pink}{Bad Ischl}}, sondern wir träfen uns entweder in \textcolor{pink}{Wien}{}\ledrightnote{\textcolor{pink}{Wien}}, oder \substVorne{}\textsuperscript{i\textcolor{gray}{n}}\substDazwischen{}am\substHinten{}{ }16. Aug. in \textcolor{pink}{Stettin}{}\ledrightnote{\textcolor{pink}{Szczecin}}, da ich auf 1 Tag nach der Insel \textcolor{pink}{Rügen}{}\ledrightnote{\textcolor{pink}{Rügen}} muss. Nun aber folgendes: \textcolor{blue}{Moriz
                  Rosenthal}{}\ledrightnote{\textcolor{blue}{Moritz Rosenthal}}, den ich heute sprach, sagte mir, er
               könne nicht \uline{dringend genug} vor \textcolor{pink}{Kopenhagen}{}\ledrightnote{\textcolor{pink}{Kopenhagen}} warnen. Es sei weder schön noch gut dort, ferner
               theuer, schlechte Bäder ec. Er räth \textcolor{pink}{Rügen}{}\ledrightnote{\textcolor{pink}{Rügen}} an,
               oder \textcolor{pink}{Sylt}{}\ledrightnote{\textcolor{pink}{Sylt}}, \uline{gewiss
                  nicht}{ }\textcolor{pink}{Kopenhagen}{}\ledrightnote{\textcolor{pink}{Kopenhagen}}. Geht es noch, dass daran gerüttelt
               wird? Ferner: Wenn Sie nicht sehr gerne von {\pb}\label{K_L03159-6v}\edtext{\textcolor{pink}{Ischl}{}\ledrightnote{\textcolor{pink}{Bad Ischl}}}{\lemma{\textnormal{\emph{Ischl}}}\Cendnote{\textnormal{\textcolor{blue}{Schnitzler} war, abgesehen von einer kurzen
                  Unterbrechung, zwischen 15. 7. 1895 und 19. 8. 1895 in \textcolor{pink}{Ischl}. Danach
                  machte er mit \textcolor{blue}{Salten} eine Radtour nach \textcolor{pink}{München}, wo er bis 6. 8. 1895
                  blieb.}}}\label{K_L03159-6h} früher weggingen, als bis \textcolor{blue}{BH.}{}\ledrightnote{\textcolor{blue}{Richard Beer-Hofmann}}
               fährt, oder auch \label{K_L03159-7v}\edtext{die \textcolor{blue}{Anderen}{}\ledrightnote{{$\rightarrow$}\textcolor{blue}{Paul Goldmann}}}{\lemma{\textnormal{\emph{die Anderen}}}\Cendnote{\textnormal{jedenfalls Bezug auf \textcolor{blue}{Paul Goldmann}}}}\label{K_L03159-7h} in \textcolor{pink}{Kphg.}{}\ledrightnote{\textcolor{pink}{Kopenhagen}} eintreffen, bin ich auch
               bereit auf die Reise zu verzichten. Für diesen Fall \substVorne{}\textsuperscript{\textcolor{gray}{könnte}}{\allowbreak}\substDazwischen{}käme\substHinten{} ich dann am 13. oder 14. Aug. einfach nach \textcolor{pink}{Ischl}{}\ledrightnote{\textcolor{pink}{Bad Ischl}}, ginge zum
                  \textcolor{pink}{Leopold}{}\ledrightnote{\textcolor{pink}{Hotel und Pension Rudolfshöhe (Leopold Petter)}}, nähme mein Bicycle mit, und bliebe
               ruhig bis 1. September dort. Wie es Ihnen angenehmer
               ist, mögen Sie nun entscheiden. Ich muß gestehen, dass es mir im Grunde gleich ist,
               wie u. wo ich die 14 Tage verbringe, ich möchte nur gerne \uline{rechtzeitig} wissen, (also bis 1. Aug.) was
               geschieht. Mir kommt es in meiner momentanen Verfassung lediglich darauf an überhaupt
               nur fort zu \substVorne{}\textsuperscript{\textcolor{gray}{ko}}\substDazwischen{}fa\substHinten{}hren, und ein bischen Ruhe zu haben.\pend
           
\pstart
           Schreiben \substVorne{}\textsuperscript{sich g}{\allowbreak}\substDazwischen{}Sie b\substHinten{}ald und \uline{leben} Sie recht wol. Ich grüße \textcolor{blue}{Beer Hofmann}{}\ledrightnote{\textcolor{blue}{Richard Beer-Hofmann}} und Sie {\\[\baselineskip]}herzlichst Ihr {\\[\baselineskip]}\spacefill\mbox{Salten}\pend
           \leftskip=0em{}\endnumbering\briefempfaengerindex{Schnitzler, Arthur@\textsc{Schnitzler, Arthur}!zzzSalten, Felix@\emph{von Felix Salten}!1895-07-221@{22. 7. 1895}|)be}\mylabel{h}  \normalsize

\doendnotes{C}
\bigskip
\vfill

\clearpage

\footnotesize

\lohead{\textsc{register}}

% Definiere theindex-Environment komplett neu ohne reledmac
\makeatletter
\renewenvironment{theindex}{%
  \section*{\indexname}%
  \setlength{\parindent}{0pt}%
  \setlength{\parskip}{0pt plus 0.3pt}%
  \let\item\@idxitem
}{%
  \clearpage
}
\makeatother

\IfFileExists{\jobname-pw.ind}{\input{\jobname-pw.ind}}{}

\end{document}

      