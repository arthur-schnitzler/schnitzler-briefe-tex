%% latex-korrekturansicht-vorspann.tex
%% Vorspann für die Korrekturansicht.
%% Lädt die gemeinsame Datei latex-vorspann.tex mit gesetztem Schalter.

\newif\ifkorrekturansicht
\korrekturansichttrue

\input{../tex-inputs/latex-vorspann}


               \section[Arthur Schnitzler an Hermann Bahr, 19. 7. 1903]{ Arthur Schnitzler an Hermann Bahr, 19. 7. 1903}\nopagebreak\mylabel{v}\rehead{ }\normalsize\beginnumbering\briefempfaengerindex{Bahr, Hermann@\textsc{Bahr, Hermann}!zzzSchnitzler, Arthur@\emph{von Arthur Schnitzler}!1903-07-191@{19. 7. 1903}|(be} \toendnotes[C]{\smallbreak\pagebreak[2]} \Standort{TMW, FS PK277797 alt.}
\physDesc{Foto Aura HertwigBerlin1903
\newline{}Handschrift: schwarze Tinte, deutsche Kurrent\newline{}Zusatz: Passepartoutreste weisen auf eine frühere Rahmung }\buchAbdrucke{\weitereDrucke{Hermann Bahr, Arthur Schnitzler: \emph{Briefwechsel, Aufzeichnungen, Dokumente (1891–1931)}. Hg. Kurt Ifkovits und Martin Anton Müller. Göttingen: \emph{Wallstein} 2018, S. 268.} }\toendnotes[C]{\smallbreak}\pstart{[}Abbildung{]}\pend\pstart
           \noindent{}{\pb}Erinner dich, wie oft
               du ſchon alt warſt, und freu dich, wie oft du noch jung ſein wirſt!\pend
           \pstart
           \label{K_L01304_1v}\edtext{Zum 19. Juli 1903}{\lemma{\textnormal{\emph{Zum 19. Juli 1903}}}\Cendnote{\textnormal{\textcolor{blue}{Bahr}s 40. Geburtstag.}}}\label{K_L01304_1h}{\\}meinem lieben Hermann\pend
           \pstart \spacefill\mbox{Arthur Schn}\pend{}\endnumbering\briefempfaengerindex{Bahr, Hermann@\textsc{Bahr, Hermann}!zzzSchnitzler, Arthur@\emph{von Arthur Schnitzler}!1903-07-191@{19. 7. 1903}|)be}\mylabel{h}  \normalsize

\doendnotes{C}
\bigskip
\vfill

\clearpage

\footnotesize

\lohead{\textsc{register}}

% Definiere theindex-Environment komplett neu ohne reledmac
\makeatletter
\renewenvironment{theindex}{%
  \section*{\indexname}%
  \setlength{\parindent}{0pt}%
  \setlength{\parskip}{0pt plus 0.3pt}%
  \let\item\@idxitem
}{%
  \clearpage
}
\makeatother

\IfFileExists{\jobname-pw.ind}{\input{\jobname-pw.ind}}{}

\end{document}

      