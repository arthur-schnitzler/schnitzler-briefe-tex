%% latex-korrekturansicht-vorspann.tex
%% Vorspann für die Korrekturansicht.
%% Lädt die gemeinsame Datei latex-vorspann.tex mit gesetztem Schalter.

\newif\ifkorrekturansicht
\korrekturansichttrue

\input{../tex-inputs/latex-vorspann}


\renewcommand{\erwaehntePersonen}{Personen:  Aristophanes, Paul Goldmann, Friedrich Hebbel, Alfred Klaar, Paul Marx, Olga Schnitzler, Elisabeth Steinrück, Leo N. von Tolstoi, Adolf von Wilbrandt}
\renewcommand{\erwaehnteInstitutionen}{Institutionen: Bohemia, Deutsches Theater Berlin, Schauspielhaus Berlin, Vossische Zeitung}
\renewcommand{\erwaehnteOrte}{Orte: Berlin, Berliner Theater, Café Josty, Dessauer Straße, Rotensterngasse, Wien}
\renewcommand{\erwaehnteWerke}{Werke: Agnes Bernauer, Die Macht der Finsternis, Frauenherrschaft. Lustspiel in vier Aufzügen nach Aristophanes’ »Ekklesiazusen« und »Lysistrate«}
\section[ Paul Goldmann an Olga Gussmann, 20. 12. {[}1900{]}]{Paul Goldmann an Olga Gussmann, 20. 12. {[}1900{]}}
\nopagebreak\mylabel{v}
\rehead{ }\normalsize\beginnumbering\briefempfaengerindex{Schnitzler, Olga@\textsc{Schnitzler, Olga}!zzzGoldmann, Paul@\emph{von Paul Goldmann}!1900-12-201@{20. 12. {[}1900{]}}|(be}
\toendnotes[C]{\smallbreak\pagebreak[2]}\Standort{DLA, A:Schnitzler, HS.NZ85.1.5247.}
\physDesc{Brief, 1 Blatt, 4 Seiten, 2177 Zeichen
\newline{}Handschrift: blaue Tinte, deutsche Kurrent}\toendnotes[C]{\smallbreak}
\pstart
           \noindent{}\raggedleft{}{\pb}\textcolor{gray}{\textbf{\textcolor{pink}{DESSAUERSTRASSE 19}{}\ledrightnote{\textcolor{pink}{Dessauer Straße}}}}\pend
           
\pstart
           \textcolor{pink}{Berlin}{}\ledrightnote{\textcolor{pink}{Berlin}}, 20. Dezember.\pend
           
\pstart\center{}Verehrtes und liebes Fräulein,\pend
\pstart
           Die Briefe, die Sie und Ihr \textcolor{blue}{Schweſterchen}{}\ledrightnote{{$\rightarrow$}\textcolor{blue}{Elisabeth Steinrück}} mir geſchrieben, haben mir \strikeout{g\textcolor{gray}{×}\-\textcolor{gray}{×}} große Freude bereitet. Seit Wochen liegen ſie auf dem Schreibtiſch – ganz
               obenauf, um raſch zur Hand zu ſein für den Fall, daß die Stunde des Briefſchreibens
               kommen ſollte. Aber die Stunde iſt bisher nicht gekommen, wird auch wohl ſo bald
               nicht kommen in meinem vielgeplagten Berichterſtatter-Daſein, und das, was ich Ihnen
                  heut ſchreibe, iſt eigentlich kein Brief, ſondern
               es ſind nur drei kurze Worte des Dankes und des herzlichen Gedenkens, die doch
               endlich einmal geſagt werden mußten, Ihnen {\pb}ſowohl,
               wie dem Fräulein \textsc{\textcolor{blue}{Liesl}{}\ledrightnote{\textcolor{blue}{Elisabeth Steinrück}}}.\pend
           
\pstart
           Inzwiſchen war \textsc{Dr. \textcolor{blue}{Schnitzler}{}\ledrightnote{}} in \strikeout{\textcolor{pink}{Wien}{}\ledrightnote{\textcolor{pink}{Wien}}}{ }\label{K_L03537-1v}\edtext{\textcolor{pink}{Berlin}{}\ledrightnote{\textcolor{pink}{Berlin}}}{\lemma{\textnormal{\emph{Berlin}}}\Cendnote{\textnormal{\textcolor{blue}{Schnitzler} war zwischen dem 24. 11. 1900 und dem 28. 11. 1900 in \textcolor{pink}{Berlin} gewesen und hatte \textcolor{blue}{Goldmann} dort täglich getroffen.}}}\label{K_L03537-1h} und hat mir
               Mancherlei über die \textcolor{pink}{\textcolor{blue}{Rothe-Sterngaſſe}{}\ledrightnote{{$\rightarrow$}\textcolor{blue}{Elisabeth Steinrück}}}{}\ledrightnote{\textcolor{pink}{Rotensterngasse}} berichtet. Insbeſondere, daß es Ihnen gut geht und daß Sie tüchtig vorwärts
               ſtreben, was ja die Hauptſache iſt. Ich wäre gern, gern wieder einmal mit Ihnen
               zuſammen. \textcolor{pink}{Berlin}{}\ledrightnote{\textcolor{pink}{Berlin}} iſt eine große Stadt, aber
                  \label{K_L03537-2v}\edtext{eine \textcolor{pink}{Rothe-Sterngaſſe}{}\ledrightnote{\textcolor{pink}{Rotensterngasse}} gibt es hier nicht}{\lemma{\textnormal{\emph{eine … nicht}}}\Cendnote{\textnormal{Die Stelle lässt sich auch im Kontext von \textcolor{blue}{Goldmann}s (unerwiderter) Schwärmerei für \textcolor{blue}{Elisabeth Gussmann} lesen, vgl.
                     deren Korrespondenz: \emph{DLA}, HS.1985.1.5246.
               }}}\label{K_L03537-2h}. Und ich bin ſehr einſam.\pend
           
\pstart
           Sie ſollen mir bald wieder ſchreiben, Sie und Ihr Fräulein \textcolor{blue}{Schweſter}{}\ledrightnote{{$\rightarrow$}\textcolor{blue}{Elisabeth Steinrück}}, das Sie ſelbſt die »kleine
               Beſtie« nennen. (Ich wage kaum, es niederzuſchreiben). Auch ſollten Sie \textcolor{blue}{Beide}{}\ledrightnote{{$\rightarrow$}\textcolor{blue}{Elisabeth Steinrück}} nach \textcolor{pink}{Berlin}{}\ledrightnote{\textcolor{pink}{Berlin}} kommen. Ich werde Sie fürſtlich aufnehmen, {\pb}und Sie dürfen bei \textsc{\textcolor{pink}{Josty}{}\ledrightnote{\textcolor{pink}{Café Josty}}} einen ganzen Tag lang Indianerkrapfen mit Schlagobers eſſen.\pend
           
\pstart
           Im \label{K_L03537-3v}\edtext{Theater}{\lemma{\textnormal{\emph{Theater}}}\Cendnote{\textnormal{\textcolor{blue}{Friedrich Hebbel}s \emph{\textcolor{green}{Agnes Bernauer}} wurde am \textcolor{pink}{Berlin}er \emph{\textcolor{brown}{Schauspielhaus}} gegeben. \textcolor{blue}{Tolstoi}s \emph{\textcolor{green}{Die
                     Macht der Finsternis}} stand am Spielplan des \emph{\textcolor{brown}{Deutschen Theaters}}. Am \emph{\textcolor{brown}{Berliner
                     Theater}} wurde \emph{\textcolor{green}{Frauenherrschaft. Lustspiel
                     in vier Aufzügen nach Aristophanes’ »Ekklesiazusen« und »Lysistrate«}} von
                     \textcolor{blue}{Adolf von Wilbrandt} gespielt.}}}\label{K_L03537-3h}
               erleben wir allerlei Gutes: \textsc{\textcolor{blue}{Tolstoi}{}\ledrightnote{\textcolor{blue}{Leo N. von Tolstoi}}s} »\textcolor{green}{Macht der Finſterniß}{}\ledrightnote{\textcolor{green}{Die Macht der Finsternis}}«, \textsc{\textcolor{blue}{Hebbel}{}\ledrightnote{\textcolor{blue}{Friedrich Hebbel}}\textcolor{gray}{’}s} herrliche »\textsc{\textcolor{green}{Agnes Bernauer}{}\ledrightnote{\textcolor{green}{Agnes Bernauer}}}«, ein wenig \textsc{\textcolor{blue}{\textcolor{green}{Aristophanes}{}\ledrightnote{{$\rightarrow$}\textcolor{green}{Frauenherrschaft. Lustspiel in vier Aufzügen nach Aristophanes’ »Ekklesiazusen« und »Lysistrate«}}}{}\ledrightnote{\textcolor{blue}{Aristophanes}} etc}.\pend
           
\pstart
           Wenn Sie unſeren lieben \textsc{Dr. \textcolor{blue}{Arthur Schnitzler}{}\ledrightnote{}} ſehen, ſo ſagen Sie ihm: 1.) daß er mir eine Ewigkeit nicht geſchrieben hat und
               daß dies eine Infamie iſt 2.) daß \textsc{\textcolor{blue}{Alfred Klaar}{}\ledrightnote{\textcolor{blue}{Alfred Klaar}}}, der ehemalige Kritiker der »\textsc{\textcolor{brown}{Bohemia}{}\ledrightnote{\textcolor{brown}{Bohemia}}}«, ein Schmock in Reincultur, der ödeſte und blödeſte Schwätzer der
                  Jetztzeit{[},{]} Theaterkritiker und Feuilleton-Redakteur der »\textcolor{brown}{Voſſiſchen Zeitung}{}\ledrightnote{\textcolor{brown}{Vossische Zeitung}}« geworden iſt. Auch ich hatte
               mich für die Stelle gemeldet, {\pb}bekam aber nicht einmal
               eine Antwort. Ich bin nämlich (aber ſagen Sie es nicht weiter!) \label{K_L03537-4v}\edtext{nicht »literariſch«}{\lemma{\textnormal{\emph{nicht »literariſch«}}}\Cendnote{\textnormal{Diesen vermeintlichen Vorbehalt gegenüber seiner Person und
                  dem Beruf des Kritikers an sich hatte \textcolor{blue}{Goldmann} in Briefen an \textcolor{blue}{Schnitzler}
                  bereits mehrmals thematisiert, beispielsweise Paul Goldmann an Arthur Schnitzler, 29. 5. [1900].}}}\label{K_L03537-4h}.\pend
           
\pstart
           Ich wünſche Ihnen und dem Fräulein \textsc{\textcolor{blue}{Liesl}{}\ledrightnote{\textcolor{blue}{Elisabeth Steinrück}}} frohe Weihnachten, bitte Sie, meinen
               Namensvetter \textsc{\textcolor{blue}{Paul}{}\ledrightnote{\textcolor{blue}{Paul Marx}}} zu grüßen, hoffe, bald wieder durch einen Brief erfreut zu werden, und küſſe
               Ihnen \textcolor{blue}{Beiden}{}\ledrightnote{{$\rightarrow$}\textcolor{blue}{Elisabeth Steinrück}} je eine Hand.
               {\\[\baselineskip]}Ihr freundſchaftlich ergebener {\\[\baselineskip]}\spacefill\mbox{Dr. Paul Goldmann.}\pend
           \leftskip=0em{}\endnumbering\briefempfaengerindex{Schnitzler, Olga@\textsc{Schnitzler, Olga}!zzzGoldmann, Paul@\emph{von Paul Goldmann}!1900-12-201@{20. 12. {[}1900{]}}|)be}\mylabel{h}  \normalsize

\doendnotes{C}
\bigskip
\vfill

\clearpage

\footnotesize

\lohead{\textsc{register}}

% Definiere theindex-Environment komplett neu ohne reledmac
\makeatletter
\renewenvironment{theindex}{%
  \section*{\indexname}%
  \setlength{\parindent}{0pt}%
  \setlength{\parskip}{0pt plus 0.3pt}%
  \let\item\@idxitem
}{%
  \clearpage
}
\makeatother

\IfFileExists{\jobname-pw.ind}{\input{\jobname-pw.ind}}{}

\end{document}

      