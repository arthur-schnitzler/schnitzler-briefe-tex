%% latex-korrekturansicht-vorspann.tex
%% Vorspann für die Korrekturansicht.
%% Lädt die gemeinsame Datei latex-vorspann.tex mit gesetztem Schalter.

\newif\ifkorrekturansicht
\korrekturansichttrue

\input{../tex-inputs/latex-vorspann}


\renewcommand{\erwaehntePersonen}{Personen: Ottilie Salten}
\renewcommand{\erwaehnteOrte}{Orte: Starkfriedgassse, Wien, Zum weißen Lamm}
\renewcommand{\erwaehnteWerke}{}
\section[ Felix Salten an Arthur Schnitzler, 3. 6. 1904]{Felix Salten an Arthur Schnitzler, 3. 6. 1904}
\nopagebreak\mylabel{v}
\rehead{ }\normalsize\beginnumbering\briefempfaengerindex{Schnitzler, Arthur@\textsc{Schnitzler, Arthur}!zzzSalten, Felix@\emph{von Felix Salten}!1904-06-031@{3. 6. 1904}|(be}
\toendnotes[C]{\smallbreak\pagebreak[2]}\Standort{CUL, Schnitzler, B 89, B 1.}
\physDesc{Brief, 1 Blatt, 1 Seite, 250 Zeichen
\newline{}Handschrift: Bleistift, lateinische Kurrent
\newline{}Schnitzler: mit Bleistift die Adresse vermerkt: »\textcolor{pink}{Starkfriedg 12}« 
\newline{}Ordnung: mit Bleistift von unbekannter Hand nummeriert: »189« }\toendnotes[C]{\smallbreak}
\pstart
           \raggedleft{}{\pb}3. VI. 04\pend
           
\pstart
           Lieber, wir könnten, wenn es Ihnen recht ist, \label{K_L03397-1v}\edtext{an einem der nächsten Nachmittage in
               unserem Garten sein, oder im Wald spazieren gehen und dann beim \textcolor{pink}{Straßer}{}\ledrightnote{\textcolor{pink}{Zum weißen Lamm}} (lieber aber bei \textcolor{blue}{uns}{}\ledrightnote{{$\rightarrow$}\textcolor{blue}{Ottilie Salten}}) nachtmahlen}{\lemma{\textnormal{\emph{an … nachtmahlen}}}\Cendnote{\textnormal{nicht nachweisbar}}}\label{K_L03397-1h}. Schreiben Sie mir nur vorher eine Zeile.\pend
           
\pstart
           herzlichst {\\[\baselineskip]}Ihr \spacefill\mbox{S.}\pend
           \leftskip=0em{}\endnumbering\briefempfaengerindex{Schnitzler, Arthur@\textsc{Schnitzler, Arthur}!zzzSalten, Felix@\emph{von Felix Salten}!1904-06-031@{3. 6. 1904}|)be}\mylabel{h}  \normalsize

\doendnotes{C}
\bigskip
\vfill

\clearpage

\footnotesize

\lohead{\textsc{register}}

% Definiere theindex-Environment komplett neu ohne reledmac
\makeatletter
\renewenvironment{theindex}{%
  \section*{\indexname}%
  \setlength{\parindent}{0pt}%
  \setlength{\parskip}{0pt plus 0.3pt}%
  \let\item\@idxitem
}{%
  \clearpage
}
\makeatother

\IfFileExists{\jobname-pw.ind}{\input{\jobname-pw.ind}}{}

\end{document}

      