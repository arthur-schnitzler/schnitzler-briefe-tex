%% latex-korrekturansicht-vorspann.tex
%% Vorspann für die Korrekturansicht.
%% Lädt die gemeinsame Datei latex-vorspann.tex mit gesetztem Schalter.

\newif\ifkorrekturansicht
\korrekturansichttrue

\input{../tex-inputs/latex-vorspann}


\renewcommand{\erwaehntePersonen}{Personen: Romain Rolland, Olga Schnitzler, Paul Seippel, Stefan Zweig}
\renewcommand{\erwaehnteOrte}{Orte: Frankreich, Hôtel Beau-Séjour, Kochgasse 8, Russland, Schweiz, Wien}
\renewcommand{\erwaehnteWerke}{Werke: ?? [Gedicht, das Stefan Zweig für Olga Schnitzler übersetzen soll], Journal de Genève, Neue Zürcher Zeitung, Une protestation d’Arthur Schnitzler}
\section[Stefan Zweig an Arthur Schnitzler, 12. 12. {[}1914{]}]{Stefan Zweig an Arthur Schnitzler, 12. 12. {[}1914{]}}
\nopagebreak\mylabel{v}
\rehead{ }\normalsize\beginnumbering\briefempfaengerindex{Schnitzler, Arthur@\textsc{Schnitzler, Arthur}!zzzZweig, Stefan@\emph{von Stefan Zweig}!1914-12-121@{12. 12. {[}1914{]}}|(be}
\toendnotes[C]{\smallbreak\pagebreak[2]}\Standort{CUL, Schnitzler, B 118.}
\physDesc{Brief, 1 Blatt, 2 Seiten, 1050 Zeichen
\newline{}Handschrift: lila Tinte, lateinische Kurrent
\newline{}Schnitzler: mit rotem Buntstift eine Unterstreichung }
\buchAbdrucke{\weitereDrucke{Stefan Zweig: \emph{Briefwechsel mit Hermann Bahr, Sigmund Freud, Rainer Maria
                        Rilke und Arthur Schnitzler}. Hg. Jeffrey B. Berlin, Hans-Ulrich Lindken und Donald A. Prater. Frankfurt am Main: \emph{S. Fischer} 1987, S. 388–389.} }\toendnotes[C]{\smallbreak}
\pstart
           {\pb}\textcolor{gray}{\textbf{SZ}}\hfill \textcolor{gray}{\textbf{\textcolor{pink}{VIII. KOCHGASSE}{}\ledrightnote{\textcolor{pink}{Kochgasse 8}}}}\pend
           
\pstart
           \raggedleft{}\textcolor{gray}{\textbf{\textcolor{pink}{WIEN}{}\ledrightnote{\textcolor{pink}{Wien}},{ }12, XII}}\pend
           
\pstart
           Verehrter Herr Doktor,{ }\textcolor{blue}{Romain Rolland}{}\ledrightnote{\textcolor{blue}{Romain Rolland}} schreibt mir soeben »\label{K_L03648-1v}\edtext{Je recois le noble ècrit de Arthur
               Schnitzler. Je le traduirai avec plaisir et je prierai \textcolor{blue}{Seippel}{}\ledrightnote{\textcolor{blue}{Paul Seippel}} de le faire paraître dans le \textcolor{green}{Journal de Genèvre}{}\ledrightnote{\textcolor{green}{Journal de Genève}}. (Envoyez moi un second exemplaire pour un
                  \textcolor{green}{journal}{}\ledrightnote{{$\rightarrow$}\textcolor{green}{Neue Zürcher Zeitung}} de la \textcolor{pink}{Suisse Allemande}{}\ledrightnote{\textcolor{pink}{Schweiz}}.) Je crains seulement qu’on
               n’objecte que personne, ici ni en \textcolor{pink}{France}{}\ledrightnote{\textcolor{pink}{Frankreich}}, n’a
               entendu parler de ces mensonges; personne chez nons, n’a élevé, ni pensé a elever des
               accusations semblables contre A. S., ni contre aucun des principans écrivains
                  allemands.}{\lemma{\textnormal{\emph{Je … allemands.}}}\Cendnote{\textnormal{\textcolor{blue}{Romain Rolland} an \textcolor{blue}{Stefan
                     Zweig}, 9. 12. 1914: »Ich erhalte den hochherzigen Text
                     von \textcolor{blue}{Arthur Schnitzler}. Ich werde ihn gern
                     übersetzen und \textcolor{blue}{Seippel} bitten, dass er
                     ihn im ›\textcolor{green}{Journal de Genèvre}‹
                     veröffentlicht. (Schicken Sie mir noch ein zweites Exemplar für eine \textcolor{green}{Zeitung} in der \textcolor{pink}{deutschsprachigen Schweiz}.) Nur glaube
                     ich, dass niemand von diesen Lügen etwas gehört hat, weder hier noch in \textcolor{pink}{Frankreich}; es ist bei uns niemandem in den
                     Sinn gekommen, derartige Anschuldigungen gegen \textcolor{blue}{Arthur Schnitzler} oder irgendeinen anderen großen deutschen
                     Schriftsteller zu erheben.«, zitiert nach: \textcolor{blue}{Romain Rolland}, \textcolor{blue}{Stefan Zweig}: \emph{Von Welt zu Welt. Briefe
                        einer Freundschaft 1914–1918}. Mit einem Begleitwort von Peter
                     Handke. Aus dem Französischen von Eva und Gerhard Schwewe (Briefe Rollands) und
                     Christel Gersch (Briefe Zweigs). Berlin: \emph{Aufbau
                        Verlag}{ }2014.}}}\label{K_L03648-1h}«\pend
           
\pstart
           Ich freue mich für Sie, dass die Lügen also kurze Beine hatten und vorläufig nicht
               über \textcolor{pink}{Russland}{}\ledrightnote{\textcolor{pink}{Russland}} hinausgelaufen sind. Das Dementi
               kann aber doch nur {\pb}von Vorteil sein.
               Wenn Sie noch ein \textcolor{green}{Exemplar}{}\ledrightnote{{$\rightarrow$}\textcolor{green}{Une protestation d’Arthur Schnitzler}}
               haben, so senden Sie es am besten direct an \textcolor{blue}{Romain
                  Rolland}{}\ledrightnote{\textcolor{blue}{Romain Rolland}}{ }\textcolor{pink}{Genf, Hôtel Beau Sejour}{}\ledrightnote{\textcolor{pink}{Hôtel Beau-Séjour}}.\pend
           
\pstart
           Das kleine \label{K_L03648-2v}\edtext{\textcolor{green}{Gedicht}{}\ledrightnote{{$\rightarrow$}\textcolor{green}{?? [Gedicht, das Stefan Zweig für Olga Schnitzler übersetzen soll]}}}{\lemma{\textnormal{\emph{Gedicht}}}\Cendnote{\textnormal{nicht ermittelt. Die Details dürften beim letzten Treffen am
                     10. 12. 1914
                  mündlich besprochen worden sein.}}}\label{K_L03648-2h} für das Lied Ihrer Frau \textcolor{blue}{Gemahlin}{}\ledrightnote{{$\rightarrow$}\textcolor{blue}{Olga Schnitzler}} leistet der \uline{guten} Verdeutschung hartnäckigen Widerstand. Hier wie überall offenbart
               sich’s neuerlich, dass das Einfachste immer auch das Schwerste ist.\pend
           
\pstart
           Ich bleibe in treuer Ergebenheit und Verehrung Ihr{\\[\baselineskip]}\spacefill\mbox{Stefan Zweig}\pend
           \leftskip=0em{}\endnumbering\briefempfaengerindex{Schnitzler, Arthur@\textsc{Schnitzler, Arthur}!zzzZweig, Stefan@\emph{von Stefan Zweig}!1914-12-121@{12. 12. {[}1914{]}}|)be}\mylabel{h}
\begin{anhang}
\end{anhang}\normalsize

\doendnotes{C}
\bigskip
\vfill

\clearpage

\footnotesize

\lohead{\textsc{register}}

% Definiere theindex-Environment komplett neu ohne reledmac
\makeatletter
\renewenvironment{theindex}{%
  \section*{\indexname}%
  \setlength{\parindent}{0pt}%
  \setlength{\parskip}{0pt plus 0.3pt}%
  \let\item\@idxitem
}{%
  \clearpage
}
\makeatother

\IfFileExists{\jobname-pw.ind}{\input{\jobname-pw.ind}}{}

\end{document}

      