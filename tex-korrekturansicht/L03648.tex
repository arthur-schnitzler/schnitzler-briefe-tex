%% latex-korrekturansicht-vorspann.tex
%% Vorspann für die Korrekturansicht.
%% Lädt die gemeinsame Datei latex-vorspann.tex mit gesetztem Schalter.

\newif\ifkorrekturansicht
\korrekturansichttrue

\input{../tex-inputs/latex-vorspann}


\section[Stefan Zweig an Arthur Schnitzler, 12. 12. {[}1914{]}]{L03648 Stefan Zweig an Arthur Schnitzler, 12. 12. {[}1914{]}}
\nopagebreak\mylabel{L03648v}
\rehead{ }\normalsize\beginnumbering\briefempfaengerindex{, @\textsc{, }!zzz, @\emph{von  }!1914-12-121@{12. 12. {[}1914{]}}|(be}
\toendnotes[C]{\smallbreak\pagebreak[2]}\Standort{CUL, Schnitzler, B 118.}
\physDesc{Brief, 1 Blatt, 2 Seiten, 1056 Zeichen
\newline{}Handschrift: lila Tinte, lateinische Kurrent
\newline{}Schnitzler: 1) mit rotem Buntstift eine Unterstreichung  2) mit Bleistift Vermerk: »\textsc{Zweig}«}
\buchAbdrucke{\weitereDrucke{Stefan Zweig: \emph{Briefwechsel mit Hermann Bahr, Sigmund Freud, Rainer Maria
                        Rilke und Arthur Schnitzler}. Frankfurt am Main: \emph{S. Fischer} 1987, S. 388–389.} }\toendnotes[C]{\smallbreak}
\pstart
           {\pb}\textcolor{gray}{\textbf{SZ}}\hfill \textcolor{gray}{\textbf{\textcolor{pink}{VIII. KOCHGASSE}\oindex{Wien@\textbf{Wien}!VIII., Josefstadt@\textbf{VIII., Josefstadt}!Kochgasse 8@\textbf{Kochgasse 8}, \emph{Wohngebäude}|pw}{}\ledrightnote{\textcolor{pink}{Kochgasse 8}}}}\pend
           
\pstart
           \raggedleft{}\textcolor{gray}{\textbf{\textcolor{pink}{WIEN}\oindex{Wien@\textbf{Wien}, \emph{Verwaltungsgebiet}|pw}{}\ledrightnote{\textcolor{pink}{Wien}},}}{ }12. XII\pend
           \vspace{0.5em}
\pstart
           Verehrter Herr Doktor,{ }\textcolor{blue}{Romain Rolland}\pwindex{Rolland, Romain 29.\,1.\,1866 Clamecy – 30.\,12.\,1944 Vézelay@\textsc{Rolland, Romain} (29.\,1.\,1866 Clamecy – 30.\,12.\,1944 Vézelay), \emph{Schriftsteller}|pw}{}\ledrightnote{\textcolor{blue}{Romain Rolland}} schreibt mir soeben »\label{K_L03648-1v}\edtext{\begin{otherlanguage}{french}Je recois le noble \textcolor{green}{écrit}\pwindex{Schnitzler, Arthur 15. 5. 1862 Wien – 21. 10. 1931 ebd.@\textsc{Schnitzler, Arthur} (15. 5. 1862 Wien – 21. 10. 1931 ebd.), \emph{Schriftsteller, Mediziner}!Une protestation d’Arthur Schnitzler@\strich\emph{Une protestation d’Arthur Schnitzler}|pwv}\pwindex{Rolland, Romain 29.\,1.\,1866 Clamecy – 30.\,12.\,1944 Vézelay@\textsc{Rolland, Romain} (29.\,1.\,1866 Clamecy – 30.\,12.\,1944 Vézelay), \emph{Schriftsteller}!Une protestation d’Arthur Schnitzler@\strich\emph{Une protestation d’Arthur Schnitzler}|pwv}{}\ledrightnote{{$\rightarrow$}\emph{\textcolor{green}{Une protestation d’Arthur Schnitzler}}} de Arthur
               Schnitzler. Je le traduirai avec plaisir et je prierai \textcolor{blue}{Seippel}\pwindex{Seippel, Paul 24.\,4.\,1858 Gingins – 13.\,3.\,1926 Chêne-Bourg@\textsc{Seippel, Paul} (24.\,4.\,1858 Gingins – 13.\,3.\,1926 Chêne-Bourg), \emph{Herausgeber, Romanist}|pw}{}\ledrightnote{\textcolor{blue}{Paul Seippel}} de le faire paraître dans le \textcolor{green}{Journal de Genèvre}\pwindex{Journal de Genève@\emph{Journal de Genève}|pw}{}\ledrightnote{\textcolor{green}{Journal de Genève}}. (Envoyez moi un second exemplaire pour un
                  \textcolor{green}{journal}\pwindex{Neue Zürcher Zeitung@\emph{Neue Zürcher Zeitung}|pwv}{}\ledrightnote{{$\rightarrow$}\emph{\textcolor{green}{Neue Zürcher Zeitung}}} de la \textcolor{pink}{Suisse Allemande}\oindex{Schweiz@\textbf{Schweiz}|pw}{}\ledrightnote{\textcolor{pink}{Schweiz}}.) Je crains seulement qu’on
               n’objecte que personne, ici ni en \textcolor{pink}{France}\oindex{Frankreich@\textbf{Frankreich}|pw}{}\ledrightnote{\textcolor{pink}{Frankreich}}, n’a
               entendu parler de ces mensonges; personne chez nous, n’a élevé, ni pensé a elever des
               accusations semblables contre A. S., ni contre aucun des principaux écrivains
                  allemands.\end{otherlanguage}}{\lemma{\textnormal{\emph{Je … allemands.}}}\Cendnote{\textnormal{\textcolor{blue}{Romain Rolland}\pwindex{Rolland, Romain 29.\,1.\,1866 Clamecy – 30.\,12.\,1944 Vézelay@\textsc{Rolland, Romain} (29.\,1.\,1866 Clamecy – 30.\,12.\,1944 Vézelay), \emph{Schriftsteller}|pwk} an \textcolor{blue}{Stefan
                     Zweig}\pwindex{Zweig, Stefan 28.\,11.\,1881 Wien – 23.\,2.\,1942 Petrópolis@\textsc{Zweig, Stefan} (28.\,11.\,1881 Wien – 23.\,2.\,1942 Petrópolis), \emph{Schriftsteller}|pwk}, 9. 12. 1914: »Ich erhalte den hochherzigen \textcolor{green}{Text}\pwindex{Schnitzler, Arthur 15. 5. 1862 Wien – 21. 10. 1931 ebd.@\textsc{Schnitzler, Arthur} (15. 5. 1862 Wien – 21. 10. 1931 ebd.), \emph{Schriftsteller, Mediziner}!Une protestation d’Arthur Schnitzler@\strich\emph{Une protestation d’Arthur Schnitzler}|pwv}\pwindex{Rolland, Romain 29.\,1.\,1866 Clamecy – 30.\,12.\,1944 Vézelay@\textsc{Rolland, Romain} (29.\,1.\,1866 Clamecy – 30.\,12.\,1944 Vézelay), \emph{Schriftsteller}!Une protestation d’Arthur Schnitzler@\strich\emph{Une protestation d’Arthur Schnitzler}|pwv}
                     von \textcolor{blue}{Arthur Schnitzler}. Ich werde ihn gern
                     übersetzen und \textcolor{blue}{Seippel}\pwindex{Seippel, Paul 24.\,4.\,1858 Gingins – 13.\,3.\,1926 Chêne-Bourg@\textsc{Seippel, Paul} (24.\,4.\,1858 Gingins – 13.\,3.\,1926 Chêne-Bourg), \emph{Herausgeber, Romanist}|pw} bitten, dass er
                     ihn im ›\textcolor{green}{Journal de Genève}\pwindex{Journal de Genève@\emph{Journal de Genève}|pw}‹
                     veröffentlicht. (Schicken Sie mir noch ein zweites Exemplar für eine \textcolor{green}{Zeitung}\pwindex{Neue Zürcher Zeitung@\emph{Neue Zürcher Zeitung}|pwv} in der \textcolor{pink}{deutschsprachigen Schweiz}\oindex{Schweiz@\textbf{Schweiz}|pw}.) Nur glaube
                     ich, dass niemand von diesen Lügen etwas gehört hat, weder hier noch in \textcolor{pink}{Frankreich}\oindex{Frankreich@\textbf{Frankreich}|pw}; es ist bei uns niemandem in den
                     Sinn gekommen, derartige Anschuldigungen gegen \textcolor{blue}{Arthur Schnitzler} oder irgendeinen anderen großen deutschen
                     Schriftsteller zu erheben.«, zitiert nach: \textcolor{blue}{Romain Rolland}\pwindex{Rolland, Romain 29.\,1.\,1866 Clamecy – 30.\,12.\,1944 Vézelay@\textsc{Rolland, Romain} (29.\,1.\,1866 Clamecy – 30.\,12.\,1944 Vézelay), \emph{Schriftsteller}|pwk}, \textcolor{blue}{Stefan Zweig}\pwindex{Zweig, Stefan 28.\,11.\,1881 Wien – 23.\,2.\,1942 Petrópolis@\textsc{Zweig, Stefan} (28.\,11.\,1881 Wien – 23.\,2.\,1942 Petrópolis), \emph{Schriftsteller}|pwk}: \emph{Von Welt zu Welt. Briefe
                        einer Freundschaft 1914–1918}. Mit einem Begleitwort von Peter
                     Handke. Aus dem Französischen von Eva und Gerhard Schwewe (Briefe Rollands) und
                     Christel Gersch (Briefe Zweigs). Berlin: \emph{Aufbau
                        Verlag}{ }2014.}}}\label{K_L03648-1}«\pend
           
\pstart
           Ich freue mich für Sie, dass die Lügen also kurze Beine hatten und vorläufig nicht
               über \textcolor{pink}{Russland}\oindex{Russland@\textbf{Russland}|pw}{}\ledrightnote{\textcolor{pink}{Russland}} hinausgelaufen sind. Das Dementi
               kann aber doch nur {\pb}von Vorteil sein.
               Wenn Sie noch ein \textcolor{green}{Exemplar}\pwindex{Schnitzler, Arthur 15. 5. 1862 Wien – 21. 10. 1931 ebd.@\textsc{Schnitzler, Arthur} (15. 5. 1862 Wien – 21. 10. 1931 ebd.), \emph{Schriftsteller, Mediziner}!Une protestation d’Arthur Schnitzler@\strich\emph{Une protestation d’Arthur Schnitzler}|pwv}\pwindex{Rolland, Romain 29.\,1.\,1866 Clamecy – 30.\,12.\,1944 Vézelay@\textsc{Rolland, Romain} (29.\,1.\,1866 Clamecy – 30.\,12.\,1944 Vézelay), \emph{Schriftsteller}!Une protestation d’Arthur Schnitzler@\strich\emph{Une protestation d’Arthur Schnitzler}|pwv}{}\ledrightnote{{$\rightarrow$}\emph{\textcolor{green}{Une protestation d’Arthur Schnitzler}}}
               haben, so senden Sie es am besten direct an \textcolor{blue}{Romain
                  Rolland}\pwindex{Rolland, Romain 29.\,1.\,1866 Clamecy – 30.\,12.\,1944 Vézelay@\textsc{Rolland, Romain} (29.\,1.\,1866 Clamecy – 30.\,12.\,1944 Vézelay), \emph{Schriftsteller}|pw}{}\ledrightnote{\textcolor{blue}{Romain Rolland}}{ }\textcolor{pink}{Genf, Hôtel Beau Sejour}\oindex{Hôtel Beau-Séjour@\textbf{Hôtel Beau-Séjour}, \emph{Hotel}|pw}{}\ledrightnote{\textcolor{pink}{Hôtel Beau-Séjour}}.\pend
           
\pstart
           Das kleine \label{K_L03648-2v}\edtext{\textcolor{green}{Gedicht}\pwindex{?? [Gedicht, das Stefan Zweig für Olga Schnitzler übersetzen soll]@\emph{?? [Gedicht, das Stefan Zweig für Olga Schnitzler übersetzen soll]}|pwv}{}\ledrightnote{{$\rightarrow$}\emph{\textcolor{green}{?? [Gedicht, das Stefan Zweig für Olga Schnitzler übersetzen soll]}}}}{\lemma{\textnormal{\emph{Gedicht}}}\Cendnote{\textnormal{nicht ermittelt. Es dürfte sich um ein Lied 
                  handeln, das \textcolor{blue}{Olga Schnitzler}\pwindex{Schnitzler, Olga 17.\,1.\,1882 Wien – 13.\,1.\,1970 Lugano@\textsc{Schnitzler, Olga} (17.\,1.\,1882 Wien – 13.\,1.\,1970 Lugano), \emph{Schauspielerin, Sängerin}|pwk} anlässlich des \emph{\textcolor{violet}{\textcolor{blue}{Liliencron}\pwindex{Liliencron, Detlev von 3.\,6.\,1844 Kiel – 22.\,7.\,1909 Rahlstedt@\textsc{Liliencron, Detlev von} (3.\,6.\,1844 Kiel – 22.\,7.\,1909 Rahlstedt), \emph{Schriftsteller, Dichter, Dramatiker}|pwk}-Abends}\eventindex{Volkshochschule Ottakring@\textbf{Volkshochschule Ottakring}!Konzert Olga Schnitzler, 3.1.1915@Konzert Olga Schnitzler, 3.1.1915|pwk}}
                  am 3. 1. 1915 vortragen sollte. Die Details könnten beim letzten Treffen am
                     10. 12. 1914
                  mündlich besprochen worden sein.}}}\label{K_L03648-2} für das Lied Ihrer Frau \textcolor{blue}{Gemahlin}\pwindex{Schnitzler, Olga 17.\,1.\,1882 Wien – 13.\,1.\,1970 Lugano@\textsc{Schnitzler, Olga} (17.\,1.\,1882 Wien – 13.\,1.\,1970 Lugano), \emph{Schauspielerin, Sängerin}|pwv}{}\ledrightnote{{$\rightarrow$}\emph{\textcolor{blue}{Olga Schnitzler}}} leistet der \uline{guten} Verdeutschung hartnäckigen Widerstand. Hier wie überall offenbart
               sich’s neuerlich, dass das Einfachste immer auch das Schwerste ist.\pend
           
\pstart
           Ich bleibe in treuer Ergebenheit und Verehrung Ihr{\\[\baselineskip]}\spacefill\mbox{Stefan Zweig}\pend
           \leftskip=0em{}\selectlanguage{ngerman}\endnumbering\briefempfaengerindex{, @\textsc{, }!zzz, @\emph{von  }!1914-12-121@{12. 12. {[}1914{]}}|)be}\mylabel{L03648h}  \normalsize

\doendnotes{C}
\bigskip
\vfill

\clearpage

\footnotesize

\lohead{\textsc{register}}

% Definiere theindex-Environment komplett neu ohne reledmac
\makeatletter
\renewenvironment{theindex}{%
  \section*{\indexname}%
  \setlength{\parindent}{0pt}%
  \setlength{\parskip}{0pt plus 0.3pt}%
  \let\item\@idxitem
}{%
  \clearpage
}
\makeatother

\IfFileExists{\jobname-pw.ind}{\input{\jobname-pw.ind}}{}

\end{document}

      