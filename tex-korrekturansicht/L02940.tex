%% latex-korrekturansicht-vorspann.tex
%% Vorspann für die Korrekturansicht.
%% Lädt die gemeinsame Datei latex-vorspann.tex mit gesetztem Schalter.

\newif\ifkorrekturansicht
\korrekturansichttrue

\input{../tex-inputs/latex-vorspann}


         
         \renewcommand{\erwaehntePersonen}{Personen: Moritz Coschell, Marie Glümer, Alfred Kerr, Paul Martin Marton}
         \renewcommand{\erwaehnteOrte}{Orte: Berlin, Dessauer Straße, Hotel Kaiserhof, Reichstag}
         \renewcommand{\erwaehnteWerke}{}
               \section[ Paul Goldmann an Arthur Schnitzler, 24. 11. {[}1900{]}]{Paul Goldmann an Arthur Schnitzler, 24. 11. {[}1900{]}}\nopagebreak\mylabel{v}\rehead{ }\normalsize\beginnumbering\briefempfaengerindex{Schnitzler, Arthur@\textsc{Schnitzler, Arthur}!zzzGoldmann, Paul@\emph{von Paul Goldmann}!1900-11-241@{24. 11. {[}1900{]}}|(be} \toendnotes[C]{\smallbreak\pagebreak[2]} \Standort{DLA, A:Schnitzler, HS.NZ85.1.3170.}
\physDesc{Brief, 1 Blatt, 2 Seiten
\newline{}Handschrift: blaue Tinte, deutsche Kurrent
\newline{}Schnitzler: mit Bleistift das Jahr »{[}1{]}900« vermerkt }\toendnotes[C]{\smallbreak}\pstart
           \noindent{}\raggedleft{}{\pb}\textcolor{pink}{\textcolor{gray}{\textbf{DESSAUERSTRASSE 19}}}{}\ledrightnote{\textcolor{pink}{Dessauer Straße}}\pend
           \pstart
           \textcolor{pink}{Berlin}{}\ledrightnote{\textcolor{pink}{Berlin}}, 24. November.\pend
           \pstart\center{}Mein lieber Freund,\pend\pstart
           Ich kann Dich leider nicht begrüßen kommen, denn ich habe den ganzen Nachmittag im
                  \label{K_L02940-2v}\edtext{\textcolor{pink}{Reichstage}{}\ledrightnote{\textcolor{pink}{Reichstag}}}{\lemma{\textnormal{\emph{Reichstage}}}\Cendnote{\textnormal{siehe Paul Goldmann an Arthur Schnitzler, 30. 10. [1900]}}}\label{K_L02940-2h} zu thun. Einſtweilen alſo heiße ich Dich auf dieſem Wege \label{K_L02940-1v}\edtext{herzlichſt willkommen}{\lemma{\textnormal{\emph{herzlichſt willkommen}}}\Cendnote{\textnormal{\textcolor{blue}{Schnitzler} hielt sich von 24. 11. 1900 bis 28. 11. 1900 in \textcolor{pink}{Berlin} auf.}}}\label{K_L02940-1h}. Abends
                  zwiſchen 9 und 10 Uhr hoffe ich mit meiner Arbeit fertig zu ſein. Bitte,
               ſende mir alſo eine Nachricht in meine Wohnung, wo ich Dich um dieſe Zeit treffen {\pb}kann? Am Beſten wäre es, Du kämeſt
                  zwiſchen 9 und 10 Uhr ſelbſt \label{K_L02940-3v}\edtext{zu mir}{\lemma{\textnormal{\emph{zu mir}}}\Cendnote{\textnormal{Am 24. 11. 1900 trafen
                  sich \textcolor{blue}{Goldmann} und \textcolor{blue}{Schnitzler} mit \textcolor{blue}{Marie
                     Glümer}, \textcolor{blue}{Paul Martin Marton} und \textcolor{blue}{Moritz Coschell} im \textcolor{pink}{Hotel Kaiserhof}. Am 25. 11. 1900 war \textcolor{blue}{Schnitzler} tatsächlich zu Mittag bei \textcolor{blue}{Goldmann} essen und traf ihn abends noch
                  einmal gemeinsam mit \textcolor{blue}{Moritz Coschell} und
                     \textcolor{blue}{Alfred Kerr}.}}}\label{K_L02940-3h}. Und morgen{ }Mittag biſt Du natürlich bei mir zu Tiſch.\pend
           \pstart
           Herzlichſt {\\[\baselineskip]}Dein {\\[\baselineskip]}\spacefill\mbox{Paul Goldmann.}\pend
           \leftskip=0em{}\endnumbering\briefempfaengerindex{Schnitzler, Arthur@\textsc{Schnitzler, Arthur}!zzzGoldmann, Paul@\emph{von Paul Goldmann}!1900-11-241@{24. 11. {[}1900{]}}|)be}\mylabel{h}\begin{anhang}\end{anhang}\normalsize

\doendnotes{C}
\bigskip
\vfill

\clearpage

\footnotesize

\lohead{\textsc{register}}

% Definiere theindex-Environment komplett neu ohne reledmac
\makeatletter
\renewenvironment{theindex}{%
  \section*{\indexname}%
  \setlength{\parindent}{0pt}%
  \setlength{\parskip}{0pt plus 0.3pt}%
  \let\item\@idxitem
}{%
  \clearpage
}
\makeatother

\IfFileExists{\jobname-pw.ind}{\input{\jobname-pw.ind}}{}

\end{document}

      