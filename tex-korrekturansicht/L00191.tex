%% latex-korrekturansicht-vorspann.tex
%% Vorspann für die Korrekturansicht.
%% Lädt die gemeinsame Datei latex-vorspann.tex mit gesetztem Schalter.

\newif\ifkorrekturansicht
\korrekturansichttrue

\input{../tex-inputs/latex-vorspann}


\section[Karl Kraus an Arthur Schnitzler, 19. 3. 1893]{L00191 Karl Kraus an Arthur Schnitzler, 19. 3. 1893}
\nopagebreak\mylabel{L00191v}
\rehead{ }\normalsize\beginnumbering\briefempfaengerindex{Schnitzler, Arthur@\textsc{Schnitzler, Arthur}!zzzKraus, Karl@\emph{von Karl Kraus}!1893-03-191@{19. 3. 1893}|(be}
\toendnotes[C]{\smallbreak\pagebreak[2]}
\correspDesc{Versand  durch Karl Kraus am 19. 3. 1893 in Wien
\newline{}Erhalt  durch Arthur Schnitzler im Zeitraum [19. 3. 1893
                  – 23. 3. 1893?] in Wien}\toendnotes[C]{\smallbreak}
\Standort{CUL, Schnitzler, B 55.}
\physDesc{Brief, 1 Blatt, 4 Seiten, 4247 Zeichen
\newline{}Handschrift: schwarze Tinte, deutsche Kurrent}
\buchAbdrucke{\weitereDrucke{\emph{Karl Kraus und Arthur Schnitzler. Eine Dokumentation.}Herausgegeben von Reinhard Urbach In: \emph{Literatur und Kritik}, Bd. 49, Oktober 1970, S. 516–517.} }\toendnotes[C]{\smallbreak}
\pstart
           {\pb}\textcolor{gray}{\textbf{Karl Kraus}}\hfill \textcolor{gray}{\textbf{\textcolor{pink}{Wien}\oindex{Wien@\textbf{Wien}, \emph{Verwaltungsgebiet}|pw}{}\ledrightnote{\textcolor{pink}{Wien}}, am }}{ }19. 3. \textcolor{gray}{\textbf{189}}3\pend
           
\pstart
           \textcolor{gray}{\textbf{\textcolor{pink}{Wien}\oindex{Wien@\textbf{Wien}, \emph{Verwaltungsgebiet}|pw}{}\ledrightnote{\textcolor{pink}{Wien}}}}\pend
           
\pstart
           \textcolor{gray}{\textbf{\textcolor{pink}{I., Maximilianstrasse 13}\oindex{Wien@\textbf{Wien}!I., Innere Stadt@\textbf{I., Innere Stadt}!Mahlerstraße@\textbf{Mahlerstraße}, \emph{Straße}|pw}{}\ledrightnote{\textcolor{pink}{Mahlerstraße}}.}}\pend
           
\pstart{}Sehr verehrter Herr Doctor!\pend\vspace{0.5em}
\pstart
           Leider ſehe ich mich genöthigt, mich in einer Angelegenheit an Sie zu wenden, mit der
               Sie gewiss nicht gerne belästigt werden. Aber, da ich \uline{Sie}, lieber Herr, ſtets hochgeſchätzt und geachtet habe, ſo will ich \introOben{}mich\introOben{} auch Ihnen \strikeout{mich} ganz offenbaren.
               Sie können ermeſſen, wie ſehr es mich kränkten muſste, daſs Sie mir vorgeſtern im \textcolor{pink}{Grienſteidl}\oindex{Wien@\textbf{Wien}!I., Innere Stadt@\textbf{I., Innere Stadt}!Café Griensteidl@\textbf{Café Griensteidl}, \emph{Kaffeehaus}|pw}{}\ledrightnote{\textcolor{pink}{Café Griensteidl}}, nachdem wir uns 4 Wochen nicht geſehen
               hatten, mit ſichtlicher Kälte und – ich möchte ſagen – »ceremonieller« Höflichkeit
               begegneten. \pend
           
\pstart
           Und weil es mir nun ganz enorm furchtbar und rieſig daran liegt, daſs \uline{Sie}, liebſter Herr D \textsuperscript{r.}
                Schnitzler, von mir \uline{gut} denken oder ſo denken, wie
               über mich zu denken iſt, ſo will ich \uline{Ihnen}, damit \uline{Sie}{ } ſich \introOben{}nicht\introOben{} durch nichtige
               Redereien beſtimmen laſſen, mir böſe zu ſein und mich quasi für einen »Ausſätzigen«
               anzuſehen, folgende Thatſachen mittheilen: \pend
           
\pstart
           Meine in N\textsuperscript{o} 8 des »\textcolor{green}{Magazin}\pwindex{Magazin für die Literatur des Auslandes@\emph{Magazin für die Literatur des Auslandes}|pw}{}\ledrightnote{\textcolor{green}{Magazin für die Literatur des Auslandes}}« enthaltene »\textcolor{green}{\textcolor{blue}{Dörmann}\pwindex{Dörmann, Felix 29.\,5.\,1870 Wien – 26.\,10.\,1928 ebd.@\textsc{Dörmann, Felix} (29.\,5.\,1870 Wien – 26.\,10.\,1928 ebd.), \emph{Schriftsteller}|pw}{}\ledrightnote{\textcolor{blue}{Felix Dörmann}} – \textcolor{blue}{Specht}\pwindex{Specht, Richard 7.\,12.\,1870 Wien – 18.\,3.\,1932 ebd.@\textsc{Specht, Richard} (7.\,12.\,1870 Wien – 18.\,3.\,1932 ebd.), \emph{Schriftsteller, Journalist, Kritiker}|pw}{}\ledrightnote{\textcolor{blue}{Richard Specht}}«-Recenſion}\pwindex{Kraus, Karl 28.\,4.\,1874 Jičín – 12.\,6.\,1936 Wien@\textsc{Kraus, Karl} (28.\,4.\,1874 Jičín – 12.\,6.\,1936 Wien), \emph{Schriftsteller, Publizist, Schriftsteller}!Wiener Lyriker@\strich\emph{Wiener Lyriker}|pwv}{}\ledrightnote{{$\rightarrow$}\emph{\textcolor{green}{Wiener Lyriker}}} iſt \uline{in dieſer Form}
               bereits vor Monaten entſtanden. Herr \textcolor{blue}{Richard
                  Specht}\pwindex{Specht, Richard 7.\,12.\,1870 Wien – 18.\,3.\,1932 ebd.@\textsc{Specht, Richard} (7.\,12.\,1870 Wien – 18.\,3.\,1932 ebd.), \emph{Schriftsteller, Journalist, Kritiker}|pw}{}\ledrightnote{\textcolor{blue}{Richard Specht}}{ } ſandte mir im November od.
                  December, (ich weiß nicht genau, wann) ſeine \textcolor{green}{Gedichte}\pwindex{Specht, Richard 7.\,12.\,1870 Wien – 18.\,3.\,1932 ebd.@\textsc{Specht, Richard} (7.\,12.\,1870 Wien – 18.\,3.\,1932 ebd.), \emph{Schriftsteller, Journalist, Kritiker}!Gedichte@\strich\emph{Gedichte}|pw}{}\ledrightnote{\textcolor{green}{Gedichte}}. Ich ſchrieb ſofort (nach 2–3 Tagen) eine Kritik, \uline{ diese \textcolor{green}{Kritik}\pwindex{Kraus, Karl 28.\,4.\,1874 Jičín – 12.\,6.\,1936 Wien@\textsc{Kraus, Karl} (28.\,4.\,1874 Jičín – 12.\,6.\,1936 Wien), \emph{Schriftsteller, Publizist, Schriftsteller}!Wiener Lyriker@\strich\emph{Wiener Lyriker}|pwv}{}\ledrightnote{{$\rightarrow$}\emph{\textcolor{green}{Wiener Lyriker}}}} (mit \textcolor{blue}{Dörmann}\pwindex{Dörmann, Felix 29.\,5.\,1870 Wien – 26.\,10.\,1928 ebd.@\textsc{Dörmann, Felix} (29.\,5.\,1870 Wien – 26.\,10.\,1928 ebd.), \emph{Schriftsteller}|pw}{}\ledrightnote{\textcolor{blue}{Felix Dörmann}} zuſammen beſprach ich ihn;
                  \textcolor{blue}{F. D.}\pwindex{Dörmann, Felix 29.\,5.\,1870 Wien – 26.\,10.\,1928 ebd.@\textsc{Dörmann, Felix} (29.\,5.\,1870 Wien – 26.\,10.\,1928 ebd.), \emph{Schriftsteller}|pw}{}\ledrightnote{\textcolor{blue}{Felix Dörmann}} »\textcolor{green}{Senſationen}\pwindex{Dörmann, Felix 29.\,5.\,1870 Wien – 26.\,10.\,1928 ebd.@\textsc{Dörmann, Felix} (29.\,5.\,1870 Wien – 26.\,10.\,1928 ebd.), \emph{Schriftsteller}!Sensationen@\strich\emph{Sensationen}|pw}{}\ledrightnote{\textcolor{green}{Sensationen}}« ſandte mir gerade vorher \textcolor{blue}{L.
                  Weiß}\pwindex{Weiß, Leopold *~21.\,11.\,1853 Bernartice@\textsc{Weiß, Leopold} (*~21.\,11.\,1853 Bernartice), \emph{Verleger, Buchhändler}|pw}{}\ledrightnote{\textcolor{blue}{Leopold Weiß}} zur Recenſion). \textcolor{blue}{Dörmann}\pwindex{Dörmann, Felix 29.\,5.\,1870 Wien – 26.\,10.\,1928 ebd.@\textsc{Dörmann, Felix} (29.\,5.\,1870 Wien – 26.\,10.\,1928 ebd.), \emph{Schriftsteller}|pw}{}\ledrightnote{\textcolor{blue}{Felix Dörmann}}{ }\uline{kannte ich damals} noch nicht; den lernte ich erſt
               ſpäter durch Vermittelung D\textsuperscript{r.}{ }\textcolor{blue}{Beer-Hofmann}\pwindex{Beer-Hofmann, Richard 11.\,7.\,1866 Wien – 26.\,9.\,1945 New York City@\textsc{Beer-Hofmann, Richard} (11.\,7.\,1866 Wien – 26.\,9.\,1945 New York City), \emph{Schriftsteller}|pw}{}\ledrightnote{\textcolor{blue}{Richard Beer-Hofmann}}’s perſönlich kennen. \pend
           
\pstart
           Die Kritik gab ich dem »\textcolor{brown}{Tagblatt}\orgindex{Wiener Tagblatt@Wiener Tagblatt|pw}{}\ledrightnote{\textcolor{brown}{Wiener Tagblatt}}«. \textcolor{blue}{Alexander Landesberg}\pwindex{Landesberg, Alexander 15.\,7.\,1848 Oradea – 14.\,6.\,1916 Wien@\textsc{Landesberg, Alexander} (15.\,7.\,1848 Oradea – 14.\,6.\,1916 Wien), \emph{Schriftsteller, Journalist}|pw}{}\ledrightnote{\textcolor{blue}{Alexander Landesberg}} behielt ſie volle 2 Monate
               bei ſich, ohne ſich zu entſcheiden. Endlich gieng ich hin. Er erklärte, dieſer Sache
               keinen ſo breiten Raum gewähren zu können. Er ſuchte sie heraus, fand ſie nach langem
               Suchen und gab ſie mir – {\pb}Nun ſchickte ich
               die Arbeit \introOben{}(\uline{Dieſelbe!! In dieſer
                     Form!!})\introOben{} – auf’s Geratewohl – an’s »\textcolor{brown}{Magazin}\orgindex{Magazin für die Literatur des Auslandes@Magazin für die Literatur des Auslandes|pw}{}\ledrightnote{\textcolor{brown}{Magazin für die Literatur des Auslandes}}«. Nach 8 Tagen ſchrieb mir \textcolor{blue}{Paul
                     Sch\strikeout{l}ettler}\pwindex{Schettler, Paul 11.\,11.\,1864 Dobrovol'sk – 1.\,2.\,1948 Hahlen@\textsc{Schettler, Paul} (11.\,11.\,1864 Dobrovol'sk – 1.\,2.\,1948 Hahlen), \emph{Redakteur}|pw}{}\ledrightnote{\textcolor{blue}{Paul Schettler}} für die Redaction: »Ihre
               Besprechung der beiden \textcolor{pink}{Wien}\oindex{Wien@\textbf{Wien}, \emph{Verwaltungsgebiet}|pw}{}\ledrightnote{\textcolor{pink}{Wien}} er ›Neurotiker‹
               acceptiert das ›\textcolor{brown}{Magazin}\orgindex{Magazin für die Literatur des Auslandes@Magazin für die Literatur des Auslandes|pw}{}\ledrightnote{\textcolor{brown}{Magazin für die Literatur des Auslandes}}‹ mit Vergnügen.«\pend
           
\pstart
            Als ich nach \textcolor{pink}{Berlin}\oindex{Berlin@\textbf{Berlin}, \emph{Hauptstadt}|pw}{}\ledrightnote{\textcolor{pink}{Berlin}} kam, machte man mich auf die
               bereits erſchienene \textcolor{green}{Kritik}\pwindex{Kraus, Karl 28.\,4.\,1874 Jičín – 12.\,6.\,1936 Wien@\textsc{Kraus, Karl} (28.\,4.\,1874 Jičín – 12.\,6.\,1936 Wien), \emph{Schriftsteller, Publizist, Schriftsteller}!Wiener Lyriker@\strich\emph{Wiener Lyriker}|pwv}{}\ledrightnote{{$\rightarrow$}\emph{\textcolor{green}{Wiener Lyriker}}}
               aufmerkſam. Ich war dem \textcolor{brown}{Tgbl.}\orgindex{Wiener Tagblatt@Wiener Tagblatt|pw}{}\ledrightnote{\textcolor{brown}{Wiener Tagblatt}} vom Herzen dankbar,
               daſs es die \textcolor{green}{Kritik}\pwindex{Kraus, Karl 28.\,4.\,1874 Jičín – 12.\,6.\,1936 Wien@\textsc{Kraus, Karl} (28.\,4.\,1874 Jičín – 12.\,6.\,1936 Wien), \emph{Schriftsteller, Publizist, Schriftsteller}!Wiener Lyriker@\strich\emph{Wiener Lyriker}|pwv}{}\ledrightnote{{$\rightarrow$}\emph{\textcolor{green}{Wiener Lyriker}}}{ } retournierte. Denn durch dieſe \textcolor{green}{Kritik}\pwindex{Kraus, Karl 28.\,4.\,1874 Jičín – 12.\,6.\,1936 Wien@\textsc{Kraus, Karl} (28.\,4.\,1874 Jičín – 12.\,6.\,1936 Wien), \emph{Schriftsteller, Publizist, Schriftsteller}!Wiener Lyriker@\strich\emph{Wiener Lyriker}|pwv}{}\ledrightnote{{$\rightarrow$}\emph{\textcolor{green}{Wiener Lyriker}}}, die \textcolor{blue}{Otto Neumann-Hofer}\pwindex{Neumann-Hofer, Gilbert Otto 4.\,2.\,1857 Bol’shiye Berezhki – 14.\,4.\,1941 Detmold@\textsc{Neumann-Hofer, Gilbert Otto} (4.\,2.\,1857 Bol’shiye Berezhki – 14.\,4.\,1941 Detmold), \emph{Kritiker, Theaterleiter}|pw}{}\ledrightnote{\textcolor{blue}{Gilbert Otto Neumann-Hofer}} und die andern Herren \introOben{}(\uline{ auch Baron \textcolor{blue}{Liliencron}\pwindex{Liliencron, Detlev von 3.\,6.\,1844 Kiel – 22.\,7.\,1909 Rahlstedt@\textsc{Liliencron, Detlev von} (3.\,6.\,1844 Kiel – 22.\,7.\,1909 Rahlstedt), \emph{Schriftsteller, Dichter, Dramatiker}|pw}{}\ledrightnote{\textcolor{blue}{Detlev von Liliencron}}})\introOben{} außerordentlich lobten, ſchuf ich mir feſte Position im »\textcolor{brown}{Magazin}\orgindex{Magazin für die Literatur des Auslandes@Magazin für die Literatur des Auslandes|pw}{}\ledrightnote{\textcolor{brown}{Magazin für die Literatur des Auslandes}}«. Die Sache wurde ſofort honoriert und
               weitere Artikel (über \textcolor{pink}{Wien}\oindex{Wien@\textbf{Wien}, \emph{Verwaltungsgebiet}|pw}{}\ledrightnote{\textcolor{pink}{Wien}} er Litteratur,
               »Decadence« etc) – ſozuſagen – »beſtellt«.\pend
           
\pstart
            Ich glaube, es ſind ſchon 4 Monate her, daſs mir Herr \textcolor{blue}{Specht}\pwindex{Specht, Richard 7.\,12.\,1870 Wien – 18.\,3.\,1932 ebd.@\textsc{Specht, Richard} (7.\,12.\,1870 Wien – 18.\,3.\,1932 ebd.), \emph{Schriftsteller, Journalist, Kritiker}|pw}{}\ledrightnote{\textcolor{blue}{Richard Specht}}{ } ſein \textcolor{green}{Büchlein}\pwindex{Specht, Richard 7.\,12.\,1870 Wien – 18.\,3.\,1932 ebd.@\textsc{Specht, Richard} (7.\,12.\,1870 Wien – 18.\,3.\,1932 ebd.), \emph{Schriftsteller, Journalist, Kritiker}!Gedichte@\strich\emph{Gedichte}|pwv}{}\ledrightnote{{$\rightarrow$}\emph{\textcolor{green}{Gedichte}}}{ } ſchickte, circa { }\uline{4 Monate} alſo ſeit Abfaſſung des vor 2–3 Wochen
               erſchienenen \textcolor{green}{Artikels}\pwindex{Kraus, Karl 28.\,4.\,1874 Jičín – 12.\,6.\,1936 Wien@\textsc{Kraus, Karl} (28.\,4.\,1874 Jičín – 12.\,6.\,1936 Wien), \emph{Schriftsteller, Publizist, Schriftsteller}!Wiener Lyriker@\strich\emph{Wiener Lyriker}|pwv}{}\ledrightnote{{$\rightarrow$}\emph{\textcolor{green}{Wiener Lyriker}}} !!
               Deshalb iſt entſtanden \strikeout{,}{ }\uline{lange, lange}, bevor ich Herrn \textcolor{blue}{Specht}\pwindex{Specht, Richard 7.\,12.\,1870 Wien – 18.\,3.\,1932 ebd.@\textsc{Specht, Richard} (7.\,12.\,1870 Wien – 18.\,3.\,1932 ebd.), \emph{Schriftsteller, Journalist, Kritiker}|pw}{}\ledrightnote{\textcolor{blue}{Richard Specht}} den wirklich mit Müh und Not beſchafften
               »Sündentraum«beleg ſchickte und da \substVorne{}\textsuperscript{bei}\substDazwischen{}zu\substHinten{} jenen ominösen, aber durch und durch freundlichen Brief ſchrieb, der den
               harmloſen Witz (»\textcolor{blue}{Dör-mannbar}\pwindex{Dörmann, Felix 29.\,5.\,1870 Wien – 26.\,10.\,1928 ebd.@\textsc{Dörmann, Felix} (29.\,5.\,1870 Wien – 26.\,10.\,1928 ebd.), \emph{Schriftsteller}|pwv}{}\ledrightnote{{$\rightarrow$}\emph{\textcolor{blue}{Felix Dörmann}}}«
               enthielt) ſie iſt entſtanden, \uline{lange} bevor ich Herrn
                  \textcolor{blue}{Dörmann}\pwindex{Dörmann, Felix 29.\,5.\,1870 Wien – 26.\,10.\,1928 ebd.@\textsc{Dörmann, Felix} (29.\,5.\,1870 Wien – 26.\,10.\,1928 ebd.), \emph{Schriftsteller}|pw}{}\ledrightnote{\textcolor{blue}{Felix Dörmann}} perſönlich kennen lernte, ſo daſs
               alſo weder von einem perſönlichen Gefühle {\pb}Herrn \textcolor{blue}{Specht}\pwindex{Specht, Richard 7.\,12.\,1870 Wien – 18.\,3.\,1932 ebd.@\textsc{Specht, Richard} (7.\,12.\,1870 Wien – 18.\,3.\,1932 ebd.), \emph{Schriftsteller, Journalist, Kritiker}|pw}{}\ledrightnote{\textcolor{blue}{Richard Specht}} gegenüber noch von einer
               »Beeinfluſſung durch \textcolor{blue}{Dörmann}\pwindex{Dörmann, Felix 29.\,5.\,1870 Wien – 26.\,10.\,1928 ebd.@\textsc{Dörmann, Felix} (29.\,5.\,1870 Wien – 26.\,10.\,1928 ebd.), \emph{Schriftsteller}|pw}{}\ledrightnote{\textcolor{blue}{Felix Dörmann}}« die Rede ſein
               kann! \pend
           
\pstart
           \uuline{Das beſchwöre ich} ! \pend
           
\pstart
           \textcolor{blue}{\uline{Alexander Landesberg}}\pwindex{Landesberg, Alexander 15.\,7.\,1848 Oradea – 14.\,6.\,1916 Wien@\textsc{Landesberg, Alexander} (15.\,7.\,1848 Oradea – 14.\,6.\,1916 Wien), \emph{Schriftsteller, Journalist}|pw}{}\ledrightnote{\textcolor{blue}{Alexander Landesberg}}, \textcolor{blue}{Alexander Engel}\pwindex{Engel, Alexander 10.\,4.\,1868 Necpaly – 17.\,11.\,1940 Wien@\textsc{Engel, Alexander} (10.\,4.\,1868 Necpaly – 17.\,11.\,1940 Wien), \emph{Schriftsteller, Journalist}|pw}{}\ledrightnote{\textcolor{blue}{Alexander Engel}}, \textcolor{blue}{Anton Lindner}\pwindex{Lindner, Anton 14.\,12.\,1874 Lviv – 30.\,12.\,1928 Wandsbek@\textsc{Lindner, Anton} (14.\,12.\,1874 Lviv – 30.\,12.\,1928 Wandsbek), \emph{Schriftsteller}|pw}{}\ledrightnote{\textcolor{blue}{Anton Lindner}} etc etc andere Freunde ſind Zeugen!! \pend
           
\pstart
            Die \textcolor{green}{Kritik}\pwindex{Kraus, Karl 28.\,4.\,1874 Jičín – 12.\,6.\,1936 Wien@\textsc{Kraus, Karl} (28.\,4.\,1874 Jičín – 12.\,6.\,1936 Wien), \emph{Schriftsteller, Publizist, Schriftsteller}!Wiener Lyriker@\strich\emph{Wiener Lyriker}|pwv}{}\ledrightnote{{$\rightarrow$}\emph{\textcolor{green}{Wiener Lyriker}}} ( \uline{ganz} in der jetzigen Geſtalt!!) iſt – vor Monaten –
               aus einer ehrlichen, vollſten, ureigenſten Überzeugung heraus entſtanden. Nichts
               liegt mir ferner als Unehrlichkeit, als »Rachegefühl« und jüdiſches
               Tagſschreiberthum. Man hüte ſich, mich in dieſer niederträchtigen Weise zu
               verleumden!! \pend
           
\pstart
            Ich haſſe und haſste diese falſche, erlogene »Decadence«, die artig mit ſich ſelbst
               coquettiert; ich bekämpfe und werde immer bekämpfen: die posierte, krankhafte,
               onanierte Poeſie! {\pb}Und \uline{dieſer Haſs} war das Kritikmotiv! \pend
           
\pstart
           \strikeout{Glauben} Sie werden vielleicht, verehrter Herr D \textsuperscript{r.}, ſich denken: Aha, wer ſich \uline{ſo} vertheidigt, \uline{muſs}{ } ſich wohl verteidigen!? \strikeout{und} Nein, ſeien Sie versichert, die ganze Litanei hab ich auch nur \uline{Ihnen}\footnote{\noindent{} Auch dem verehrten Herrn D \textsuperscript{r.}{ }\textcolor{blue}{B-Hofmann}\pwindex{Beer-Hofmann, Richard 11.\,7.\,1866 Wien – 26.\,9.\,1945 New York City@\textsc{Beer-Hofmann, Richard} (11.\,7.\,1866 Wien – 26.\,9.\,1945 New York City), \emph{Schriftsteller}|pw} hätte ich’s geſagt! } hergeſagt, weil mir an \uline{Ihrer} Meinung \strikeout{etw} viel liegt. Den andern gegenüber hab’ ich es
               Gottſseidank nicht nöthig, mich zu vertheidigen! \pend
           
\pstart
           Wenn ich Sie beläſtigt habe, verzeihen Sie.\pend
           
\pstart
           \textcolor{blue}{Otto Erich Hartleben}\pwindex{Hartleben, Otto Erich 3.\,6.\,1864 Clausthal-Zellerfeld – 11.\,2.\,1905 Salò@\textsc{Hartleben, Otto Erich} (3.\,6.\,1864 Clausthal-Zellerfeld – 11.\,2.\,1905 Salò), \emph{Schriftsteller}|pw}{}\ledrightnote{\textcolor{blue}{Otto Erich Hartleben}} grüßt Sie durch mich. \pend
           
\pstart
            Für »\textcolor{brown}{Neue litt. Bl}\orgindex{Neue litterarische Blätter@Neue litterarische Blätter|pw}{}\ledrightnote{\textcolor{brown}{Neue litterarische Blätter}}« \introOben{} ( \textcolor{pink}{Bremen}\oindex{Bremen@\textbf{Bremen}|pw}{}\ledrightnote{\textcolor{pink}{Bremen}} ) \introOben{} wäre ich mit \strikeout{mit}{ }\textcolor{green}{Anatol}\pwindex{Schnitzler, Arthur 15. 5. 1862 Wien – 21. 10. 1931 ebd.@\textsc{Schnitzler, Arthur} (15. 5. 1862 Wien – 21. 10. 1931 ebd.), \emph{Schriftsteller, Mediziner}!Anatol@\strich\emph{Anatol}|pw}{}\ledrightnote{\textcolor{green}{Anatol}} zu ſpät gekommen, da das dort in \label{K_L00191-1v}\edtext{Einläufe }{\lemma{\textnormal{\emph{Einläufe }}}\Cendnote{\textnormal{\emph{\textcolor{green}{Neue litterarische Blätter}\pwindex{Neue litterarische Blätter@\emph{Neue litterarische Blätter}|pwk}}, Jg. 1, H. 5/6,
                        1. 3. 1893, S. 66 . }}}\label{K_L00191-1} verzeichnete \textcolor{green}{Buch}\pwindex{Schnitzler, Arthur 15. 5. 1862 Wien – 21. 10. 1931 ebd.@\textsc{Schnitzler, Arthur} (15. 5. 1862 Wien – 21. 10. 1931 ebd.), \emph{Schriftsteller, Mediziner}!Anatol@\strich\emph{Anatol}|pwv}{}\ledrightnote{{$\rightarrow$}\emph{\textcolor{green}{Anatol}}} bereits an einen andern \textcolor{blue}{Mitarbeiter}\pwindex{Schmid-Braunfels, Josef 29.\,11.\,1871 Ryžoviště – 22.\,11.\,1911 ebd.@\textsc{Schmid-Braunfels, Josef} (29.\,11.\,1871 Ryžoviště – 22.\,11.\,1911 ebd.), \emph{Schriftsteller, Veterinärmediziner}|pwv}{}\ledrightnote{{$\rightarrow$}\emph{\textcolor{blue}{Josef Schmid-Braunfels}}} zur \textcolor{green}{Recension}\pwindex{Schmid-Braunfels, Josef 29.\,11.\,1871 Ryžoviště – 22.\,11.\,1911 ebd.@\textsc{Schmid-Braunfels, Josef} (29.\,11.\,1871 Ryžoviště – 22.\,11.\,1911 ebd.), \emph{Schriftsteller, Veterinärmediziner}!Arthur Schnitzler: Anatol@\strich\emph{Arthur Schnitzler: Anatol}|pwv}{}\ledrightnote{{$\rightarrow$}\emph{\textcolor{green}{Arthur Schnitzler: Anatol}}} abgegeben wurde. \pend
           
\pstart
            Sonſt ſtehe ich Ihnen mit aufrichtigem Vergnügen ſtets zu Dienſten u bin (Sie noch
                  \uline{um paar Zeilen bittend} !) Ihr \uline{Sie vollkommen hochachtender}\pend
           
\pstart
            Herzlichſt grüſſend {\\[\baselineskip]}\spacefill\mbox{Karl Kraus}\pend
           \leftskip=0em{}\selectlanguage{ngerman}\endnumbering\briefempfaengerindex{Schnitzler, Arthur@\textsc{Schnitzler, Arthur}!zzzKraus, Karl@\emph{von Karl Kraus}!1893-03-191@{19. 3. 1893}|)be}\mylabel{L00191h}  \normalsize

\doendnotes{C}
\bigskip
\vfill

\clearpage

\footnotesize

\lohead{\textsc{register}}

% Definiere theindex-Environment komplett neu ohne reledmac
\makeatletter
\renewenvironment{theindex}{%
  \section*{\indexname}%
  \setlength{\parindent}{0pt}%
  \setlength{\parskip}{0pt plus 0.3pt}%
  \let\item\@idxitem
}{%
  \clearpage
}
\makeatother

\IfFileExists{\jobname-pw.ind}{\input{\jobname-pw.ind}}{}

\end{document}

      