%% latex-korrekturansicht-vorspann.tex
%% Vorspann für die Korrekturansicht.
%% Lädt die gemeinsame Datei latex-vorspann.tex mit gesetztem Schalter.

\newif\ifkorrekturansicht
\korrekturansichttrue

\input{../tex-inputs/latex-vorspann}


               \section[Hermann Bahr an Arthur Schnitzler, 25. 10. 1893]{ Hermann Bahr an Arthur Schnitzler, 25. 10. 1893}\nopagebreak\mylabel{v}\rehead{ }\normalsize\beginnumbering\briefempfaengerindex{Schnitzler, Arthur@\textsc{Schnitzler, Arthur}!zzzBahr, Hermann@\emph{von Hermann Bahr}!1893-10-251@{25. 10. 1893}|(be} \toendnotes[C]{\smallbreak\pagebreak[2]} \Standort{CUL, Schnitzler, B 5b.}
\physDesc{Brief, 1 Blatt, 2 Seiten
\newline{}Handschrift  : schwarze Tinte, deutsche Kurrent\newline{}Handschrift Hermann Bahr: schwarze Tinte, deutsche Kurrent (\noindent{}Unterschrift)\newline{}Ordnung: 1) mit rotem Buntstift von unbekannter Hand nummeriert:
                                    »15« 2) mit Bleistift von unbekannter Hand nummeriert:
                                    »15«}\buchAbdrucke{\weitereDrucke{Hermann Bahr, Arthur Schnitzler: \emph{Briefwechsel, Aufzeichnungen, Dokumente (1891–1931)}. Hg. Kurt Ifkovits und Martin Anton Müller. Göttingen: \emph{Wallstein} 2018, S. 45.} }\toendnotes[C]{\smallbreak}\pstart
           \noindent{}{\pb}\textcolor{gray}{\textbf{\textcolor{brown}{Deutſche Zeitung}{}\ledrightnote{\textcolor{brown}{Deutsche Zeitung}}}}\hfill \uline{\textcolor{pink}{Wien}{}\ledrightnote{\textcolor{pink}{Wien}}}, 25. Octbr. 1893\pend
           \pstart
           \textcolor{gray}{\textbf{\textcolor{pink}{Wien}{}\ledrightnote{\textcolor{pink}{Wien}}}}\hfill \textcolor{pink}{III. Saleſianerg. 12}{}\ledrightnote{\textcolor{pink}{Salesianergasse}}\pend
           \pstart
           \textcolor{gray}{\textbf{\textcolor{pink}{IX., Pelikangaſſe 4}{}\ledrightnote{\textcolor{pink}{Pelikangasse}}.}}\pend
           \pstart{}Verehrter Freund!\pend\pstart
           Der Mann um den es ſich handelt heißt \textcolor{blue}{Johann Lukas \textsc{Schönlein}}{}\ledrightnote{\textcolor{blue}{Johann Lukas Schönlein}}. Er iſt der Begründer der ſog. naturhyſteriſchen Schule in der Therapie. Am
                  30. November{ }ſind es hundert Jahre, daß er geboren wurde und ich
               brauche alſo für dieſen Tag ein nicht über ſechs Spalten langes, populäres, \label{K_L00276_1v}\edtext{byografiſches Feuilleton}{\lemma{\textnormal{\emph{byografiſches Feuilleton}}}\Cendnote{\textnormal{nicht erschienen}}}\label{K_L00276_1h}. Können Sie mir das
               verſchaffen?\pend
           \pstart
           Dabei wiederhole ich die bereits \label{K_L00276_2v}\edtext{neulich}{\lemma{\textnormal{\emph{neulich}}}\Cendnote{\textnormal{vermutlich beim Besuch \textcolor{blue}{Hofmannsthals} am 22. 10. 1893}}}\label{K_L00276_2h} durch \textcolor{blue}{\textsc{Loris}}{}\ledrightnote{\textcolor{blue}{Hugo von Hofmannsthal}} vermittelte Bitte um irgend eine Novellette, ſo kurz als möglich, die ich am
                  \label{K_L00276_3v}\edtext{Tage Ihrer Premiere}{\lemma{\textnormal{\emph{Tage Ihrer Premiere}}}\Cendnote{\textnormal{Am 1. 12. 1893 Premiere von \emph{\textcolor{green}{Das Märchen}}; an diesem Tag erschien nichts von
                     \textcolor{blue}{Schnitzler}.}}}\label{K_L00276_3h} bringen will.\pend
           \pstart
           {\pb}Kann ich bis längſtens Ende der nächſten Woche auf
               den \label{K_L00276_4v}\edtext{erſten}{\lemma{\textnormal{\emph{erſten}}}\Cendnote{\textnormal{\textcolor{blue}{Arthur Schnitzler}: \emph{\textcolor{green}{Spaziergang}}. In: \emph{\textcolor{green}{Deutsche
                        Zeitung}}, Jg. 23, Nr. 7883, 6. 12. 1893, Morgen-Ausgabe,
                     S. 1–2 (heute in A. S. \emph{Entworfenes und Verworfenes} 152–156).}}}\label{K_L00276_4h} der verſprochenen \label{K_L00276_5v}\edtext{\textcolor{green}{Beiträge zur Entdeckung von \textcolor{pink}{\textsc{Wien}}{}\ledrightnote{\textcolor{pink}{Wien}}}{}\ledrightnote{→\textcolor{green}{Spaziergang}}}{\lemma{\textnormal{\emph{Beiträge … Wien}}}\Cendnote{\textnormal{\emph{\textcolor{green}{Spaziergang}} eröffnete die Serie, die unter dem
                  Titel »\textcolor{pink}{Wien}er Spiegel« laufen sollte. Dem ersten
                  Beitrag war eine erklärende Fußnote beigesellt: »Der ›\textcolor{pink}{\so{Wien}}\so{er Spiegel}‹ soll in losen Skizzen die \textcolor{pink}{Wien}er Welt, oben und unten, Gesellschaft und Volk, Salon
                     und Straße bringen. Das ganze \textcolor{pink}{Wien}er Leben will
                     er Stück für Stück allmälig erzählen. Beiträge haben \textcolor{blue}{Ferdinand v. \so{Saar}}, \textcolor{blue}{Emil \so{Marriot}}, \textcolor{blue}{Ada \so{Christen}}, \textcolor{blue}{C. \so{Karlweis}}, \textcolor{blue}{Gustav \so{Schwarzkopf}}, \textcolor{blue}{Vincenz \so{Chiavacci}}, \textcolor{blue}{Karl \so{Rabis}}, \textcolor{blue}{Theodor \so{Taube}}, \textcolor{blue}{Hugo v. \so{Hofmannsthal}}, \textcolor{blue}{Arthur \so{Schnitzler}}, Dr. \textcolor{blue}{\so{Beer-Hofmann}}, \textcolor{blue}{Hermann \so{Bahr}} und Andere versprochen. Anmerkung der Redaction.« Nach dem
                  zweiten Teil, \emph{\textcolor{green}{Heunt is Sunntag!}} von \textcolor{blue}{Taube} (Nr. 7887,
                        10. 12. 1893, Sonntags-Ausgabe, S. 1–2), und \textcolor{blue}{Bahrs} Ausscheiden aus der \emph{\textcolor{brown}{Deutschen Zeitung}} wurde sie eingestellt.}}}\label{K_L00276_5h} beſtimmt
               rechnen?\pend
           \pstart
           In herzlicher Freundſchaft{\\[\baselineskip]}\spacefill\mbox{{[}hs. Bahr:{]} HermannBahr}\pend
           \leftskip=0em{}\endnumbering\briefempfaengerindex{Schnitzler, Arthur@\textsc{Schnitzler, Arthur}!zzzBahr, Hermann@\emph{von Hermann Bahr}!1893-10-251@{25. 10. 1893}|)be}\mylabel{h}  \normalsize

\doendnotes{C}
\bigskip
\vfill

\clearpage

\footnotesize

\lohead{\textsc{register}}

% Definiere theindex-Environment komplett neu ohne reledmac
\makeatletter
\renewenvironment{theindex}{%
  \section*{\indexname}%
  \setlength{\parindent}{0pt}%
  \setlength{\parskip}{0pt plus 0.3pt}%
  \let\item\@idxitem
}{%
  \clearpage
}
\makeatother

\IfFileExists{\jobname-pw.ind}{\input{\jobname-pw.ind}}{}

\end{document}

      