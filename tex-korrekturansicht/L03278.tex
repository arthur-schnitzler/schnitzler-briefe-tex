%% latex-korrekturansicht-vorspann.tex
%% Vorspann für die Korrekturansicht.
%% Lädt die gemeinsame Datei latex-vorspann.tex mit gesetztem Schalter.

\newif\ifkorrekturansicht
\korrekturansichttrue

\input{../tex-inputs/latex-vorspann}


\renewcommand{\erwaehnteOrte}{Orte: Burgtheater, Wien}
\renewcommand{\erwaehnteWerke}{Werke: Hanneles Himmelfahrt. Traumdichtung in zwei Teilen, König Ödipus. Tragödie in einem Aufzuge, Tagebuch}
\section[ Felix Salten an Arthur Schnitzler, {[}4. 1.? 1898{]}]{Felix Salten an Arthur Schnitzler, {[}4. 1.? 1898{]}}
\nopagebreak\mylabel{v}
\rehead{ }\normalsize\beginnumbering\briefempfaengerindex{Schnitzler, Arthur@\textsc{Schnitzler, Arthur}!zzzSalten, Felix@\emph{von Felix Salten}!1898-01-042@{{[}4. 1.? 1898{]}}|(be}
\toendnotes[C]{\smallbreak\pagebreak[2]}\Standort{CUL, Schnitzler, B 89, A 2.}
\physDesc{Brief, 1 Blatt, 1 Seite, 161 Zeichen
\newline{}Handschrift: Bleistift, lateinische Kurrent
\newline{}Schnitzler: mit Bleistift auf das Jahr »98« datiert 
\newline{}Ordnung: mit Bleistift von unbekannter Hand nummeriert: »101« }\toendnotes[C]{\smallbreak}
\pstart
           \noindent{}{\pb}Lieber Arthur, ich kann Ihnen den \label{K_L03278-1v}\edtext{Sitz}{\lemma{\textnormal{\emph{Sitz}}}\Cendnote{\textnormal{Das
                  Korrespondenzstück ist von \textcolor{blue}{Schnitzler} nur im
                  Jahr 1898 verortet. Gleicht man die 22 in diesem Jahr nachweisbaren Besuche \textcolor{blue}{Schnitzler}s im \textcolor{pink}{Burgtheater} (»\textcolor{pink}{Vestibül}«) mit den Erwähnungen \textcolor{blue}{Salten}s im \emph{\textcolor{green}{Tagebuch}} in
                  dieser Zeit ab, so ergibt sich nur ein gemeinsamer Besuch, für den \textcolor{blue}{Salten} die Karten besorgt haben könnte. Es
                  dürfte sich also um die Aufführung von \emph{\textcolor{green}{König
                     Oidipus}} und \emph{\textcolor{green}{Hanneles Himmelfahrt}} am 4. 1. 1898
                  handeln. }}}\label{K_L03278-1h} jetzt nicht schicken, weil der Diener eine Dummheit gemacht hat.
               Treffen wir uns also Abends um \label{K_L03278-2v}\edtext{¼ 8}{\lemma{\textnormal{\emph{¼ 8}}}\Cendnote{\textnormal{7 Uhr 15}}}\label{K_L03278-2h} im \textcolor{pink}{Vestibül}{}\ledrightnote{{$\rightarrow$}\textcolor{pink}{Burgtheater}}.\pend
           
\pstart
           Herzlich Ihr {\\[\baselineskip]}\spacefill\mbox{Salten}\pend
           \leftskip=0em{}\endnumbering\briefempfaengerindex{Schnitzler, Arthur@\textsc{Schnitzler, Arthur}!zzzSalten, Felix@\emph{von Felix Salten}!1898-01-042@{{[}4. 1.? 1898{]}}|)be}\mylabel{h}  \normalsize

\doendnotes{C}
\bigskip
\vfill

\clearpage

\footnotesize

\lohead{\textsc{register}}

% Definiere theindex-Environment komplett neu ohne reledmac
\makeatletter
\renewenvironment{theindex}{%
  \section*{\indexname}%
  \setlength{\parindent}{0pt}%
  \setlength{\parskip}{0pt plus 0.3pt}%
  \let\item\@idxitem
}{%
  \clearpage
}
\makeatother

\IfFileExists{\jobname-pw.ind}{\input{\jobname-pw.ind}}{}

\end{document}

      