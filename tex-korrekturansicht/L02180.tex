%% latex-korrekturansicht-vorspann.tex
%% Vorspann für die Korrekturansicht.
%% Lädt die gemeinsame Datei latex-vorspann.tex mit gesetztem Schalter.

\newif\ifkorrekturansicht
\korrekturansichttrue

\input{../tex-inputs/latex-vorspann}


               \section[Arthur und Olga Schnitzler an Richard und Paula Beer-Hofmann, 21. 5. 1914]{ Arthur und Olga Schnitzler an Richard und Paula Beer-Hofmann,
               21. 5. 1914}\nopagebreak\mylabel{v}\rehead{ }\normalsize\beginnumbering\briefempfaengerindex{Beer-Hofmann, Paula@\textsc{Beer-Hofmann, Paula}!zzzSchnitzler, Olga@\emph{von Olga Schnitzler}!1914-05-211@{21. 5. 1914}|(be}\briefempfaengerindex{Beer-Hofmann, Paula@\textsc{Beer-Hofmann, Paula}!zzzSchnitzler, Arthur@\emph{von Arthur Schnitzler}!1914-05-211@{21. 5. 1914}|(be}\briefempfaengerindex{Beer-Hofmann, Richard@\textsc{Beer-Hofmann, Richard}!zzzSchnitzler, Olga@\emph{von Olga Schnitzler}!1914-05-211@{21. 5. 1914}|(be}\briefempfaengerindex{Beer-Hofmann, Richard@\textsc{Beer-Hofmann, Richard}!zzzSchnitzler, Arthur@\emph{von Arthur Schnitzler}!1914-05-211@{21. 5. 1914}|(be} \toendnotes[C]{\smallbreak\pagebreak[2]} \Standort{YCGL, MSS 31.}
\physDesc{Bildpostkarte
\newline{}Handschrift Arthur Schnitzler: Bleistift, deutsche Kurrent\newline{}Handschrift Olga Schnitzler: Bleistift, lateinische Kurrent\newline{}Versand: Stempel: »\nobreak{}\oindex{Antwerpen@\textbf{Antwerpen}, \emph{Besiedelter Ort (A.BSO)}|pwk}\textcolor{gray}{Antw}erpe\textcolor{gray}{n}
                                       Anvers, 21. 5. 1914, 1\textcolor{gray}{0} 20\nobreak{}«.  }\pstart{}{\pb}Herrn u Frau \textsc{Dr.
                     Rich.}\pend{}\pstart{}\textsc{Beer}\substVorne{}\textsuperscript{h}\substDazwischen{}\textsc{H}\substHinten{}\textsc{ofmann}\pend{}\pstart{}\textcolor{pink}{Wien XVIII}{}\ledrightnote{\textcolor{pink}{XVIII., Währing}}\pend{}\pstart{}\textcolor{pink}{\textsc{Hasenauerstr. 59}}{}\ledrightnote{\textcolor{pink}{Hasenauerstraße}}. \pend{}\pstart{}\textcolor{pink}{\textsc{Austria}}{}\ledrightnote{\textcolor{pink}{Österreich}}\pend{}{\bigskip}\pstart
           \noindent{}\centering{}{\pb}\textcolor{gray}{\textbf{\textcolor{blue}{REMBRANDT}{}\ledrightnote{\textcolor{blue}{Rembrandt van Rijn}}. – \textcolor{green}{Portrait d’un Vieux Juif}{}\ledrightnote{\textcolor{green}{Alter Mann}}.}}\pend
           \pstart
           \noindent{}\centering{}\textcolor{gray}{\textbf{\textcolor{brown}{MUSÉE ROYAL D’ANVERS}{}\ledrightnote{\textcolor{brown}{Königliches Museum der Schönen Künste}}}}\pend
           \pstart
           {\pb}Herzliche Grüße!\pend
           \pstart
           \textcolor{pink}{\textsc{Antwerpen}}{}\ledrightnote{\textcolor{pink}{Antwerpen}}{ }2\substVorne{}\textsuperscript{0}\substDazwischen{}1\substHinten{}/5 914\pend
           \pstart Ihr \spacefill\mbox{Arthur}\pend{}\pstart
           \noindent{}{[}hs. O. Schnitzler:{]} Es ist hier viel wärmer wie in \textcolor{pink}{Algier}{}\ledrightnote{\textcolor{pink}{Algiers}}. Wir hatten immer schönes Wetter, immer gute Fahrt.\pend
           \pstart
           Herzlichst{\\[\baselineskip]}Ihre \spacefill\mbox{O.}\pend
           \leftskip=0em{}\endnumbering\briefempfaengerindex{Beer-Hofmann, Paula@\textsc{Beer-Hofmann, Paula}!zzzSchnitzler, Olga@\emph{von Olga Schnitzler}!1914-05-211@{21. 5. 1914}|)be}\briefempfaengerindex{Beer-Hofmann, Paula@\textsc{Beer-Hofmann, Paula}!zzzSchnitzler, Arthur@\emph{von Arthur Schnitzler}!1914-05-211@{21. 5. 1914}|)be}\briefempfaengerindex{Beer-Hofmann, Richard@\textsc{Beer-Hofmann, Richard}!zzzSchnitzler, Olga@\emph{von Olga Schnitzler}!1914-05-211@{21. 5. 1914}|)be}\briefempfaengerindex{Beer-Hofmann, Richard@\textsc{Beer-Hofmann, Richard}!zzzSchnitzler, Arthur@\emph{von Arthur Schnitzler}!1914-05-211@{21. 5. 1914}|)be}\mylabel{h}  \normalsize

\doendnotes{C}
\bigskip
\vfill

\clearpage

\footnotesize

\lohead{\textsc{register}}

% Definiere theindex-Environment komplett neu ohne reledmac
\makeatletter
\renewenvironment{theindex}{%
  \section*{\indexname}%
  \setlength{\parindent}{0pt}%
  \setlength{\parskip}{0pt plus 0.3pt}%
  \let\item\@idxitem
}{%
  \clearpage
}
\makeatother

\IfFileExists{\jobname-pw.ind}{\input{\jobname-pw.ind}}{}

\end{document}

      