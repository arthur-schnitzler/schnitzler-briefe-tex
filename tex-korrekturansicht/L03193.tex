%% latex-korrekturansicht-vorspann.tex
%% Vorspann für die Korrekturansicht.
%% Lädt die gemeinsame Datei latex-vorspann.tex mit gesetztem Schalter.

\newif\ifkorrekturansicht
\korrekturansichttrue

\input{../tex-inputs/latex-vorspann}


\renewcommand{\erwaehntePersonen}{Personen: Heinrich Kanner, Alfred de Musset, Olga Schnitzler, Siegfried Trebitsch, Alice Ziegler, Arnost Ziegler}
\renewcommand{\erwaehnteInstitutionen}{Institutionen: Deutsches Theater Berlin}
\renewcommand{\erwaehnteOrte}{Orte: Berlin, Carl-Theater, Dessauer Straße, Wien}
\renewcommand{\erwaehnteWerke}{Werke: Berliner Theater. (»Lebendige Stunden« von Arthur Schnitzler.), Die Zeit, Die Zeit. Wiener Wochenschrift, Kleine Chronik. [Das Wiener Gastspiel des Berliner Deutschen Theaters.], Lebendige Stunden. Vier Einakter, Lorenzaccio. Drame romantique en cinq actes}
\section[ Paul Goldmann an Arthur Schnitzler, 16. 1. {[}1902{]}]{Paul Goldmann an Arthur Schnitzler, 16. 1. {[}1902{]}}
\nopagebreak\mylabel{v}
\rehead{ }\normalsize\beginnumbering\briefempfaengerindex{Schnitzler, Arthur@\textsc{Schnitzler, Arthur}!zzzGoldmann, Paul@\emph{von Paul Goldmann}!1902-01-161@{16. 1. {[}1902{]}}|(be}
\toendnotes[C]{\smallbreak\pagebreak[2]}\Standort{DLA, A:Schnitzler, HS.NZ85.1.3172.}
\physDesc{Brief, 1 Blatt, 2 Seiten
\newline{}Handschrift: blaue Tinte, deutsche Kurrent
\newline{}Schnitzler: 1) mit Bleistift das Jahr »{[}1{]}902« vermerkt  2) mit rotem Buntstift vier Unterstreichungen}\toendnotes[C]{\smallbreak}
\pstart
           \noindent{}\raggedleft{}{\pb}\textcolor{pink}{\textcolor{gray}{\textbf{DESSAUERSTRASSE 19}}}{}\ledrightnote{\textcolor{pink}{Dessauer Straße}}\pend
           
\pstart
           \textcolor{pink}{Berlin}{}\ledrightnote{\textcolor{pink}{Berlin}}, 16. Januar.\pend
           
\pstart{}Mein lieber Freund,\pend
\pstart
           Diesmal haſt \uline{Du} mich, wie ich glaube, \label{K_L03193-1v}\edtext{mißverſtanden}{\lemma{\textnormal{\emph{mißverſtanden}}}\Cendnote{\textnormal{\textcolor{blue}{Schnitzler} dürfte
                  entweder durch \textcolor{blue}{Goldmann}s abwägende Worte
                  hinsichtlich der \emph{\textcolor{green}{Notiz}} in der \emph{\textcolor{green}{Neuen Freien Presse}} zum \textcolor{green}{Gastspiel} des \emph{\textcolor{brown}{Deutschen Theaters Berlin}} am \textcolor{pink}{Wien}er \textcolor{pink}{Carl-Theater} verstört gewesen
                  sein, oder durch die »eiſige[] Kälte«, mit der dieser am \textcolor{green}{Feuilleton} über \emph{\textcolor{green}{Lebendige Stunden}} arbeitete, siehe Paul Goldmann an Arthur Schnitzler, 14. 1. [1902]}}}\label{K_L03193-1h}. Deine Standrede hat mich \strikeout{daher} überraſcht,
               weil mein letzter Brief ganz harmlos gemeint war. Aber ich mag nicht darauf erwidern.
               Ich habe keine Zeit zur Polemik; ich ſchreibe lieber an dem \textsc{\textcolor{green}{Feuilleton}{}\ledrightnote{{$\rightarrow$}\textcolor{green}{Berliner Theater. (»Lebendige Stunden« von Arthur Schnitzler.)}}} über Deine \textcolor{green}{Stücke}{}\ledrightnote{{$\rightarrow$}\textcolor{green}{Lebendige Stunden. Vier Einakter}}
               weiter. Bin ich wirklich ſo koloſſal empfindlich? Ich finde, es iſt bequem, – \strikeout{\textcolor{gray}{die}{ }\textcolor{gray}{×}\-\textcolor{gray}{×}\-\textcolor{gray}{×}\-\textcolor{gray}{×}\-\textcolor{gray}{×}\-\textcolor{gray}{×} an} irgendwelche
               Differenzen durch die Empfindlichkeit des Anderen zu erklären. Man erſpart ſich
               ſelbſt dadurch jedes Gefühl der Verantwortung. Aber es gäbe vielleicht auch eine
               andere Erklärung. Beiſpielsweiſe die, daß von Dir zu mir nicht Alles in Ordnung iſt –
               vielleicht ſchon ſeit Jahren nicht in Ordnung iſt. Außer über meine Empfindlichkeit
               ſollteſt Du auch darüber einmal nachdenken.\pend
           
\pstart
           Du haſt gewünſcht, wir ſollten grob zu einander ſein. Bin ich grob genug? Aber laſſen
               wir es dabei {\pb}bewenden. Dieſe Diskuſſionen führen zu
               nichts.\pend
           
\pstart
           Ich wäre Dir ſehr dankbar, wenn Du \label{K_L03193-2v}\edtext{\textsc{\textcolor{blue}{Trebitsch}{}\ledrightnote{\textcolor{blue}{Siegfried Trebitsch}}} bewegen}{\lemma{\textnormal{\emph{Trebitsch bewegen}}}\Cendnote{\textnormal{unklar, ob das geschehen
                  ist; jedenfalls gibt es keine veröffentlichte Übersetzung von \textcolor{blue}{Musset}s \emph{\textcolor{green}{Lorenzaccio}}
                  durch \emph{\textcolor{green}{Siegfried Trebitsch}}}}}\label{K_L03193-2h} könnteſt, von der \textsc{\textcolor{green}{Lorenzaccio}{}\ledrightnote{{$\rightarrow$}\textcolor{green}{Lorenzaccio. Drame romantique en cinq actes}}}-Überſetzung abzuſehen. Vielleicht mache ich mich \label{K_L03193-4v}\edtext{doch noch einmal}{\lemma{\textnormal{\emph{doch noch einmal}}}\Cendnote{\textnormal{siehe Paul Goldmann an Arthur Schnitzler, 2. [1.? 1897]}}}\label{K_L03193-4h} an dieſe Arbeit.\pend
           
\pstart
           \textsc{\textcolor{blue}{Kanner}{}\ledrightnote{\textcolor{blue}{Heinrich Kanner}}}, der in \textcolor{pink}{\textsc{Berlin}}{}\ledrightnote{\textcolor{pink}{Berlin}}
               weilt, war bei mir. Die \label{K_L03193-6v}\edtext{Umwandlung der
                  »\textcolor{green}{Zeit}{}\ledrightnote{\textcolor{green}{Die Zeit. Wiener Wochenschrift}}« in ein \textcolor{green}{Tagesblatt}{}\ledrightnote{{$\rightarrow$}\textcolor{green}{Die Zeit}}}{\lemma{\textnormal{\emph{Umwandlung … Tagesblatt}}}\Cendnote{\textnormal{siehe Paul Goldmann an Arthur Schnitzler und Olga
               Gussmann, 7. 7. [1901]}}}\label{K_L03193-6h} iſt beſchloſſene Sache.\pend
           
\pstart
           \textsc{\textcolor{blue}{Alice Bondy}{}\ledrightnote{\textcolor{blue}{Alice Ziegler}}} zeigt mir ihre \label{K_L03193-8v}\edtext{Verlobung}{\lemma{\textnormal{\emph{Verlobung}}}\Cendnote{\textnormal{\textcolor{blue}{Ernst Ziegler} und \textcolor{blue}{Alice Bondy} heirateten am 7. 5. 1902. In den späten 1890er-Jahren hatte
                     \textcolor{blue}{Goldmann} für die damals knapp unter \textcolor{blue}{20-Jährige} geschwärmt,
                     siehe Paul Goldmann an Arthur Schnitzler, 10. 12. [1897], 19. 1. [1898] und 30. 8. 1899.}}}\label{K_L03193-8h} mit
               einem \textsc{\textcolor{blue}{Dr. Ziegler}{}\ledrightnote{\textcolor{blue}{Arnost Ziegler}}} an.\pend
           
\pstart
           Es thut mir unendlich leid, daß \textsc{\textcolor{blue}{Olga}{}\ledrightnote{\textcolor{blue}{Olga Schnitzler}}} ſich ſo \label{K_L03193-9v}\edtext{plagen}{\lemma{\textnormal{\emph{plagen}}}\Cendnote{\textnormal{womöglich verursacht durch die
                  Schwangerschaft, siehe A. S.: \emph{Tagebuch}, 4. 1. 1902 und 8. 1. 1902}}}\label{K_L03193-9h} muß. Verſichere ſie meiner herzlichſten Antheilnahme und grüße ſie
               vielmals.\pend
           
\pstart
           Auch Du ſei von Herzen gegrüßt. {\\[\baselineskip]}Dein {\\[\baselineskip]}\spacefill\mbox{Paul Goldm}\pend
           \leftskip=0em{}\endnumbering\briefempfaengerindex{Schnitzler, Arthur@\textsc{Schnitzler, Arthur}!zzzGoldmann, Paul@\emph{von Paul Goldmann}!1902-01-161@{16. 1. {[}1902{]}}|)be}\mylabel{h}
\begin{anhang}
\end{anhang}\normalsize

\doendnotes{C}
\bigskip
\vfill

\clearpage

\footnotesize

\lohead{\textsc{register}}

% Definiere theindex-Environment komplett neu ohne reledmac
\makeatletter
\renewenvironment{theindex}{%
  \section*{\indexname}%
  \setlength{\parindent}{0pt}%
  \setlength{\parskip}{0pt plus 0.3pt}%
  \let\item\@idxitem
}{%
  \clearpage
}
\makeatother

\IfFileExists{\jobname-pw.ind}{\input{\jobname-pw.ind}}{}

\end{document}

      