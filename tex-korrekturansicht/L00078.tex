%% latex-korrekturansicht-vorspann.tex
%% Vorspann für die Korrekturansicht.
%% Lädt die gemeinsame Datei latex-vorspann.tex mit gesetztem Schalter.

\newif\ifkorrekturansicht
\korrekturansichttrue

\input{../tex-inputs/latex-vorspann}


               \section[Richard Beer-Hofmann an Arthur Schnitzler, 10. 3. 1892]{ Richard Beer-Hofmann an Arthur Schnitzler, 10. 3. 1892}\nopagebreak\mylabel{v}\rehead{ }\normalsize\beginnumbering\briefempfaengerindex{Schnitzler, Arthur@\textsc{Schnitzler, Arthur}!zzzBeer-Hofmann, Richard@\emph{von Richard Beer-Hofmann}!1892-03-101@{10. 3. 1892}|(be} \toendnotes[C]{\smallbreak\pagebreak[2]} \Standort{CUL, Schnitzler, B 8.}
\physDesc{Brief, 1 Blatt (Briefpapier mit Trauerrand), 4 Seiten
\newline{}Handschrift: blauer Buntstift, lateinische Kurrent
\newline{}Schnitzler: mit Bleistift nummeriert: »8« }\buchAbdrucke{\weitereDrucke{Arthur Schnitzler, Richard Beer-Hofmann: \emph{Briefwechsel 1891–1931}. Hg. Konstanze Fliedl. Wien, Zürich: \emph{Europaverlag} 1992, S. 33.} }\toendnotes[C]{\smallbreak}\pstart
           \noindent{}{\pb}\textcolor{gray}{\textbf{RB}}\pend
           \pstart{}Lieber Arthur!\pend\pstart
           Ich wohne \uline{\textcolor{pink}{Pension Quisisana}{}\ledrightnote{\textcolor{pink}{Pension Quisisana}}}; was machen Sie, \textcolor{blue}{Loris}{}\ledrightnote{\textcolor{blue}{Hugo von Hofmannsthal}}, \textcolor{blue}{Salten}{}\ledrightnote{\textcolor{blue}{Felix Salten}}?\pend
           \pstart
           Wird etwas aus der Vorstellung, hat \textcolor{blue}{Kaffka}{}\ledrightnote{\textcolor{blue}{Eduard Michael Kafka}}
               Nachrichten von der »\textcolor{brown}{freien Bühne}{}\ledrightnote{\textcolor{brown}{»Freie Bühne« Verein für moderne Literatur}}« wegen »\textcolor{green}{Camelias}{}\ledrightnote{\textcolor{green}{Camelias}}«?\pend
           \pstart
           {\pb}Ich faullenze und langweile mich;
               keine gesunde erquiquende ruhige Langeweile, sondern eine pretentiöse, lärmende mit
               Gesprächen, und Gesellschaft; ausserdem regnet es heute auch noch. Ist \label{K_L00078_1v}\edtext{mein Artikel}{\lemma{\textnormal{\emph{mein Artikel}}}\Cendnote{\textnormal{Er hatte über \textcolor{blue}{Maximilian
                     Harden} ein \textcolor{green}{Feuilleton}
                  verfasst. Dieses erschien als \emph{\textcolor{green}{Maximilian Harden}}
                  am 30. 4. 1892 in der \emph{\textcolor{green}{Wiener Allgemeinen
                     Zeitung}}.}}}\label{K_L00078_1h} in der »\textcolor{green}{Frankfurter}{}\ledrightnote{\textcolor{green}{Frankfurter Zeitung}}«
               erschienen? {\pb}Ich glaube nicht;
               schon wegen der \introOben{}letzten\introOben{}{ }\label{K_L00078_2v}\edtext{Confiscation}{\lemma{\textnormal{\emph{Confiscation}}}\Cendnote{\textnormal{ Die Morgenausgabe der \emph{\textcolor{green}{Frankfurter
                     Zeitung}} vom 1. 3. 1893 war wegen des Beitrags \emph{\textcolor{green}{Gekrönte Worte}} von \textcolor{blue}{Maximilian Harden} beschlagnahmt worden. Dieser hatte sich darin abfällig
                  über eine Rede des deutschen Kaisers \textcolor{blue}{Wilhelm II.} geäußert.}}}\label{K_L00078_2h}{ }\textcolor{blue}{Harden}{}\ledrightnote{\textcolor{blue}{Maximilian Harden}}s nicht!\pend
           \pstart
           \textcolor{blue}{Julius Bauer}{}\ledrightnote{\textcolor{blue}{Julius Bauer}} ist seit 3 Tagen hier; und spielt
               Piquet. Wir bleiben mindestens eine Woche noch hier, dann vielleicht \textcolor{pink}{Venedig}{}\ledrightnote{\textcolor{pink}{Venedig}}. Bitte schreiben Sie mir \uline{recht viel}; wissen Sie: »Glühende Kohlen«.\pend
           \pstart
           {\pb}ich selbst bin hier mehr als je
               der launeverderbende »Miesmacher{[}«,{]} würde Hermann \textcolor{blue}{Cagliostro}{}\ledrightnote{→\textcolor{blue}{Alessandro von Cagliostro}} (\textcolor{blue}{Bahr}{}\ledrightnote{\textcolor{blue}{Hermann Bahr}}) sagen.\pend
           \pstart
           Ich grüße Sie von Herzen.{\\[\baselineskip]}\spacefill\mbox{Richard}\pend
           \leftskip=0em{}\pstart
           10/III 92{ }\textcolor{pink}{Abbazia}{}\ledrightnote{\textcolor{pink}{Opatija}}\pend
           \endnumbering\briefempfaengerindex{Schnitzler, Arthur@\textsc{Schnitzler, Arthur}!zzzBeer-Hofmann, Richard@\emph{von Richard Beer-Hofmann}!1892-03-101@{10. 3. 1892}|)be}\mylabel{h}  \normalsize

\doendnotes{C}
\bigskip
\vfill

\clearpage

\footnotesize

\lohead{\textsc{register}}

% Definiere theindex-Environment komplett neu ohne reledmac
\makeatletter
\renewenvironment{theindex}{%
  \section*{\indexname}%
  \setlength{\parindent}{0pt}%
  \setlength{\parskip}{0pt plus 0.3pt}%
  \let\item\@idxitem
}{%
  \clearpage
}
\makeatother

\IfFileExists{\jobname-pw.ind}{\input{\jobname-pw.ind}}{}

\end{document}

      