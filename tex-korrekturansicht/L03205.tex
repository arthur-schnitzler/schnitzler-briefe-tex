%% latex-korrekturansicht-vorspann.tex
%% Vorspann für die Korrekturansicht.
%% Lädt die gemeinsame Datei latex-vorspann.tex mit gesetztem Schalter.

\newif\ifkorrekturansicht
\korrekturansichttrue

\input{../tex-inputs/latex-vorspann}


\renewcommand{\erwaehntePersonen}{Personen: Ernest von Gréger-Jurco, Alfred Spitzer, Karl Strecker}
\renewcommand{\erwaehnteInstitutionen}{Institutionen: Berliner Tageblatt, Karl Weiß-Theater, Staatsanwaltschaft, Tägliche Rundschau}
\renewcommand{\erwaehnteOrte}{Orte: Berlin, Deutschland, Wien, Österreich}
\renewcommand{\erwaehnteWerke}{Werke: Das angebliche Telegramm Arthur Schnitzlers, Die Kinder der Armen, Ein litterarisch-dramatisches Hochstapler-Stücklein, Tägliche Rundschau}
\section[ Paul Goldmann an Arthur Schnitzler, 29. 4. {[}1902{]}]{Paul Goldmann an Arthur Schnitzler, 29. 4. {[}1902{]}}
\nopagebreak\mylabel{v}
\rehead{ }\normalsize\beginnumbering\briefempfaengerindex{Schnitzler, Arthur@\textsc{Schnitzler, Arthur}!zzzGoldmann, Paul@\emph{von Paul Goldmann}!1902-04-291@{29. 4. {[}1902{]}}|(be}
\toendnotes[C]{\smallbreak\pagebreak[2]}\Standort{DLA, A:Schnitzler, HS.NZ85.1.3172.}
\physDesc{Brief, 1 Blatt, 4 Seiten
\newline{}Handschrift: blaue Tinte, deutsche Kurrent
\newline{}Schnitzler: 1) mit Bleistift das Jahr »{[}1{]}902« vermerkt  2) mit rotem Buntstift eine Unterstreichung}\toendnotes[C]{\smallbreak}
\pstart
           \centering{}{\pb}\textcolor{pink}{Berlin}{}\ledrightnote{\textcolor{pink}{Berlin}}, 29. April.\pend
           
\pstart{}Mein lieber Freund,\pend
\pstart
           Die »\textcolor{brown}{Tägliche Rundſchau}{}\ledrightnote{\textcolor{brown}{Tägliche Rundschau}}« hat auch heut{ }Morgen noch nicht für nöthig befunden, nachdem ſie in überaus taktloſer
               Weiſe Deinen \label{K_L03205-1v}\edtext{Namen \textcolor{green}{genannt}{}\ledrightnote{{$\rightarrow$}\textcolor{green}{Ein litterarisch-dramatisches Hochstapler-Stücklein}}}{\lemma{\textnormal{\emph{Namen genannt}}}\Cendnote{\textnormal{siehe Paul Goldmann an Arthur Schnitzler, 26. 4. 1902}}}\label{K_L03205-1h} und ſogar von einem »\textcolor{green}{Fall
                     \textsc{Schnitzler}}{}\ledrightnote{{$\rightarrow$}\textcolor{green}{Ein litterarisch-dramatisches Hochstapler-Stücklein}}« geſprochen hat, von Deinem \label{K_L03205-2v}\edtext{\textsc{\textcolor{green}{Dementi}{}\ledrightnote{{$\rightarrow$}\textcolor{green}{Das angebliche Telegramm Arthur Schnitzlers}}}}{\lemma{\textnormal{\emph{Dementi}}}\Cendnote{\textnormal{\textcolor{blue}{Goldmann} dürfte diese \textcolor{green}{Notiz} übersehen haben: \textcolor{blue}{Karl Strecker}: \emph{\textcolor{green}{Das angebliche Telegramm Arthur Schnitzlers}}. In: \emph{\textcolor{green}{Tägliche Rundschau}}, Jg. 22, Nr. 194, 26. 4. 1902, Abend-Blatt, Unterhaltungsbeilage,
                     Nr. 97, S. 388. Siehe A. S.: \emph{»Das Zeitlose ist von kürzester Dauer«}, Karl Strecker: Das angebliche Telegramm Arthur Schnitzlers, 26. 4. 1902}}}\label{K_L03205-2h} Notiz zu nehmen. Die »\textcolor{green}{Tgl. Rundſchau}{}\ledrightnote{\textcolor{green}{Tägliche Rundschau}}« iſt ein alldeutſches und antiſemitiſches
                  Blatt\strikeout{, gilt} und gilt für ſehr »literariſch«,
               ebenſo wie der Herr \textsc{\textcolor{blue}{Karl Strecker}{}\ledrightnote{\textcolor{blue}{Karl Strecker}}} (der ein \strikeout{germa} germaniſtiſcher Schwätzer iſt)
               für einen »vornehmen Kritiker« gilt. Es iſt möglich, daß das {\pb}Schweigen der \textcolor{brown}{Tgl.
                  Rdſch.}{}\ledrightnote{\textcolor{brown}{Tägliche Rundschau}} nur Schlamperei iſt, daß der Herr \textsc{\textcolor{blue}{Strecker}{}\ledrightnote{\textcolor{blue}{Karl Strecker}}} vielleicht die Angelegenheit in ſeinem nächſten Referat berühren will. Aber
               ſchon dieſes Warten, nachdem er das Maul ſo voll genommen und eine »\textcolor{green}{offene Frage}{}\ledrightnote{{$\rightarrow$}\textcolor{green}{Ein litterarisch-dramatisches Hochstapler-Stücklein}}« an Dich gerichtet hat, iſt
               unanſtändig. Ich bitte Dich daher, ihm in gemeſſenem Ton einen Brief zu ſchreiben,
               Dein Erſtaunen über ſein ganzes Vorgehen, Dein noch größeres Erſtaunen über \strikeout{das} die Nichtveröffentlichung Deiner \textcolor{green}{Antwort}{}\ledrightnote{{$\rightarrow$}\textcolor{green}{Das angebliche Telegramm Arthur Schnitzlers}} auszudrücken, ihn um ſofortige
               Publikation Deiner \textcolor{green}{Antwort}{}\ledrightnote{{$\rightarrow$}\textcolor{green}{Das angebliche Telegramm Arthur Schnitzlers}} zu
               erſuchen und die Hoffnung auszuſprechen, daß {\pb}\textcolor{blue}{er}{}\ledrightnote{{$\rightarrow$}\textcolor{blue}{Karl Strecker}} Dich nicht dazu nöthigen
               wird, die Veröffentlichung dieſer \textcolor{green}{Antwort}{}\ledrightnote{{$\rightarrow$}\textcolor{green}{Das angebliche Telegramm Arthur Schnitzlers}}, die eine ſchlicht literariſchen Anſtandes iſt, auf andere Weiſe zu
               erzwingen. Wenn das nicht hilft, wirſt Du das \textcolor{brown}{Blatt}{}\ledrightnote{{$\rightarrow$}\textcolor{brown}{Tägliche Rundschau}} ſelbſtverſtändlich klagen. \textcolor{pink}{Hier}{}\ledrightnote{{$\rightarrow$}\textcolor{pink}{Deutschland}} liegen die Verhältniſſe anders als in
                  \textcolor{pink}{Öſterreich}{}\ledrightnote{\textcolor{pink}{Österreich}}, und jedes Gericht wird Dir Recht
               geben. Ich übernehme die Angelegenheit und beſorge Dir einen guten \label{K_L03205-12v}\edtext{Advokaten}{\lemma{\textnormal{\emph{Advokaten}}}\Cendnote{\textnormal{\textcolor{blue}{Schnitzler} sprach am 5. 5. 1902 jedenfalls
                  mit dem Rechtsanwalt \textcolor{blue}{Alfred Spitzer} über
                  die Angelegenheit.}}}\label{K_L03205-12h}. Ebenſo würde ich rathen, daß Du bei der \textcolor{pink}{Wien}{}\ledrightnote{\textcolor{pink}{Wien}}er \textcolor{brown}{Staatsanwaltſchaft}{}\ledrightnote{\textcolor{brown}{Staatsanwaltschaft}} Anzeige erſtatteſt. Dieſem ſauberen Herrn von \label{K_L03205-5v}\edtext{\textsc{\textcolor{blue}{Jurco}{}\ledrightnote{\textcolor{blue}{Ernest von Gréger-Jurco}}}}{\lemma{\textnormal{\emph{Jurco}}}\Cendnote{\textnormal{\textcolor{blue}{Ernest von Jurco-Gréger}, dessen Stück \emph{\textcolor{green}{Die Kinder der Armen}} in dem gefälschten \textcolor{green}{Telegramm}{ }\textcolor{blue}{Schnitzler}s empfohlen worden war}}}\label{K_L03205-5h} muß
               doch das Handwerk gelegt werden. Auch an die {\pb}Direktion des \label{K_L03205-6v}\edtext{\textcolor{brown}{\textsc{Carl Weiss} Theater}{}\ledrightnote{\textcolor{brown}{Karl Weiß-Theater}}}{\lemma{\textnormal{\emph{Carl Weiss Theater}}}\Cendnote{\textnormal{nicht nachweisbar}}}\label{K_L03205-6h}s ſollteſt Du
               ſchreiben und Dir die Nennung des wirklichen Namens des Herrn \textsc{\textcolor{blue}{von Jurco}{}\ledrightnote{\textcolor{blue}{Ernest von Gréger-Jurco}}} erbitten. Die \textcolor{brown}{Direktion}{}\ledrightnote{{$\rightarrow$}\textcolor{brown}{Karl Weiß-Theater}}
               hat dem \strikeout{H\textcolor{gray}{e}}{ }\textcolor{brown}{Berliner Tageblatt}{}\ledrightnote{\textcolor{brown}{Berliner Tageblatt}}{ }\strikeout{\textcolor{gray}{×}} auf eine telephoniſche Anfrage geantwortet, daß \strikeout{ſih} ſich \strikeout{\textcolor{gray}{u}n} unter dieſem Pſeudonym ein \textcolor{blue}{Autor}{}\ledrightnote{{$\rightarrow$}\textcolor{blue}{Ernest von Gréger-Jurco}} aus »guter \textcolor{pink}{Wien}{}\ledrightnote{\textcolor{pink}{Wien}}er Familie« verberge, deſſen Namen allerdings die \textcolor{brown}{Direktion}{}\ledrightnote{{$\rightarrow$}\textcolor{brown}{Karl Weiß-Theater}} nicht nennen
               könne.\pend
           
\pstart
           Hebe Dir (für den Fall, daß es zum Prozeß kommt) alle \textcolor{pink}{Berlin}{}\ledrightnote{\textcolor{pink}{Berlin}}er Zeitungen auf, die ich Dir ſchicke, ſ\textcolor{gray}{ende} eine
                  \label{K_L03205-8v}\edtext{Copie Deines Briefes an \textsc{\textcolor{blue}{Strecker}{}\ledrightnote{\textcolor{blue}{Karl Strecker}}}}{\lemma{\textnormal{\emph{Copie … Strecker}}}\Cendnote{\textnormal{nicht nachweisbar}}}\label{K_L03205-8h}.\pend
           
\pstart
           Viele treue Grüße! {\\[\baselineskip]}Dein \spacefill\mbox{Paul Goldmann}\pend
           \leftskip=0em{}\endnumbering\briefempfaengerindex{Schnitzler, Arthur@\textsc{Schnitzler, Arthur}!zzzGoldmann, Paul@\emph{von Paul Goldmann}!1902-04-291@{29. 4. {[}1902{]}}|)be}\mylabel{h}
\begin{anhang}
\end{anhang}\normalsize

\doendnotes{C}
\bigskip
\vfill

\clearpage

\footnotesize

\lohead{\textsc{register}}

% Definiere theindex-Environment komplett neu ohne reledmac
\makeatletter
\renewenvironment{theindex}{%
  \section*{\indexname}%
  \setlength{\parindent}{0pt}%
  \setlength{\parskip}{0pt plus 0.3pt}%
  \let\item\@idxitem
}{%
  \clearpage
}
\makeatother

\IfFileExists{\jobname-pw.ind}{\input{\jobname-pw.ind}}{}

\end{document}

      