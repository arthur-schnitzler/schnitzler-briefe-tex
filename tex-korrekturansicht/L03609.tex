%% latex-korrekturansicht-vorspann.tex
%% Vorspann für die Korrekturansicht.
%% Lädt die gemeinsame Datei latex-vorspann.tex mit gesetztem Schalter.

\newif\ifkorrekturansicht
\korrekturansichttrue

\input{../tex-inputs/latex-vorspann}


\renewcommand{\erwaehntePersonen}{Personen: Felix Salten}
\renewcommand{\erwaehnteInstitutionen}{Institutionen: S. Fischer Verlag}
\renewcommand{\erwaehnteOrte}{Orte: Berlin, Wien}
\renewcommand{\erwaehnteWerke}{Werke: Die Schwestern oder Casanova in Spa. Lustspiel in Versen}
\section[ Arthur Schnitzler: Widmungsexemplar Die Schwestern für Felix Salten, 2. 12. 1919]{Arthur Schnitzler: Widmungsexemplar Die Schwestern für Felix
               Salten, 2. 12. 1919}
\nopagebreak\mylabel{v}
\rehead{ }\normalsize\beginnumbering\briefempfaengerindex{Salten, Felix@\textsc{Salten, Felix}!zzzSchnitzler, Arthur@\emph{von Arthur Schnitzler}!1919-12-021@{2. 12. 1919}|(be}
\toendnotes[C]{\smallbreak\pagebreak[2]}\Standort{Wienbibliothek im Rathaus, A-67087/2. Ex., DS-2019-4203.}
\physDesc{Widmung am Schmutztitel, 82 Zeichen
\newline{}Handschrift: schwarze Tinte, deutsche Kurrent
\newline{}Salten: mit schwarzer Tinte ausgefüllter Stempel: »\noindent{}\textcolor{gray}{\textbf{\textit{Felix Salten}}}{ / }\textcolor{gray}{\textbf{\textit{Inv. Nr.}}}{ }4481{ / }\textcolor{gray}{\textbf{\textit{Werk Nr.}}}{ }2209{ / }\textcolor{gray}{\textbf{\textit{Schrank}}}{ }XIV A. Z \textcolor{gray}{\textbf{\textit{Fach}}} b« }
\pstart
           \noindent{}{\pb}Meinem lieben Felix Salten {\\}in alter Herzlichkeit\pend
           \pstart \spacefill\mbox{Arthur Schnitzler}\pend{}
\pstart
           \textcolor{pink}{Wien}{}\ledrightnote{\textcolor{pink}{Wien}}{ }2. 12. 919\pend
           {\bigskip}
\pstart
           \noindent{}\centering{}{\pb}\textcolor{gray}{\textbf{\textcolor{green}{DIE SCHWESTERN}{}\ledrightnote{\textcolor{green}{Die Schwestern oder Casanova in Spa. Lustspiel in Versen}}}}\pend
           
\pstart
           \noindent{}\centering{}\textcolor{gray}{\textbf{\textcolor{green}{ODER}{}\ledrightnote{\textcolor{green}{Die Schwestern oder Casanova in Spa. Lustspiel in Versen}}}}\pend
           
\pstart
           \noindent{}\centering{}\textcolor{gray}{\textbf{\textcolor{green}{CASANOVA IN SPA}{}\ledrightnote{\textcolor{green}{Die Schwestern oder Casanova in Spa. Lustspiel in Versen}}}}\pend
           {\bigskip}
\pstart
           \noindent{}\centering{}\textcolor{gray}{\textbf{EIN LUSTSPIEL IN VERSEN}}\pend
           
\pstart
           \noindent{}\centering{}\textcolor{gray}{\textbf{DREI AKTE IN EINEM}}\pend
           
\pstart
           \noindent{}\centering{}\textcolor{gray}{\textbf{VON}}\pend
           
\pstart
           \noindent{}\centering{}\textcolor{gray}{\textbf{ARTHUR SCHNITZLER}}\pend
           {\bigskip}
\pstart
           \noindent{}\centering{}\textcolor{gray}{\textbf{1919}}\pend
           
\pstart
           \noindent{}\centering{}\textcolor{gray}{\textbf{\textcolor{brown}{S. FISCHER ⋅ VERLAG}{}\ledrightnote{\textcolor{brown}{S. Fischer Verlag}} ⋅ \textcolor{pink}{BERLIN}{}\ledrightnote{\textcolor{pink}{Berlin}}}}\pend
           \endnumbering\briefempfaengerindex{Salten, Felix@\textsc{Salten, Felix}!zzzSchnitzler, Arthur@\emph{von Arthur Schnitzler}!1919-12-021@{2. 12. 1919}|)be}\mylabel{h}  \normalsize

\doendnotes{C}
\bigskip
\vfill

\clearpage

\footnotesize

\lohead{\textsc{register}}

% Definiere theindex-Environment komplett neu ohne reledmac
\makeatletter
\renewenvironment{theindex}{%
  \section*{\indexname}%
  \setlength{\parindent}{0pt}%
  \setlength{\parskip}{0pt plus 0.3pt}%
  \let\item\@idxitem
}{%
  \clearpage
}
\makeatother

\IfFileExists{\jobname-pw.ind}{\input{\jobname-pw.ind}}{}

\end{document}

      