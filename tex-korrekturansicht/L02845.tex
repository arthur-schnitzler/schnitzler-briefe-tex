%% latex-korrekturansicht-vorspann.tex
%% Vorspann für die Korrekturansicht.
%% Lädt die gemeinsame Datei latex-vorspann.tex mit gesetztem Schalter.

\newif\ifkorrekturansicht
\korrekturansichttrue

\input{../tex-inputs/latex-vorspann}


               \section[ Paul Goldmann an Arthur Schnitzler, 16. 5. 1898]{Paul Goldmann an Arthur Schnitzler, 16. 5. 1898}\nopagebreak\mylabel{v}\rehead{ }\normalsize\beginnumbering\briefempfaengerindex{Schnitzler, Arthur@\textsc{Schnitzler, Arthur}!zzzGoldmann, Paul@\emph{von Paul Goldmann}!1898-05-162@{16. 5. 1898}|(be} \toendnotes[C]{\smallbreak\pagebreak[2]} \Standort{DLA, A:Schnitzler, HS.NZ85.1.3168.}
\physDesc{Brief, 2 Blätter, 6 Seiten
\newline{}Handschrift: schwarze Tinte, lateinische Kurrent
\newline{}Schnitzler: mit rotem Buntstift eine Unterstreichung }\toendnotes[C]{\smallbreak}\pstart
           \noindent{}\centering{}{\pb}\textcolor{gray}{\textbf{\textcolor{pink}{HONG KONG HOTEL}{}\ledrightnote{\textcolor{pink}{Hongkong Hotel}}}}\pend
           \pstart
           \noindent{}\textcolor{gray}{\textbf{A.B.C. CODE.}}\pend
           \pstart
           \textcolor{gray}{\textbf{Telegraphic Address.}}\pend
           \pstart
           \textcolor{gray}{\textbf{»KREMLIN«}}\pend
           \pstart
           \raggedleft{}\textcolor{pink}{\textcolor{gray}{\textbf{Hong Kong}}}{}\ledrightnote{\textcolor{pink}{Hong Kong}},{ }16. Mai \textcolor{gray}{\textbf{189}}8.\pend
           \pstart\center{}Mein lieber Freund,\pend\pstart
           Deinen erſten Brief nach \textsc{\textcolor{pink}{Shanghai}{}\ledrightnote{\textcolor{pink}{Shanghai}}} habe ich ſchon \textcolor{pink}{hier}{}\ledrightnote{→\textcolor{pink}{Hong Kong}}
               erhalten, und er iſt das erſte Wort, das ich hier in der Ferne von zu Hauſe u. von
               lieben Menſchen höre. Herzlichſten Dank dafür, ſowie für die beigelegte \label{K_L02845-1v}\edtext{Empfehlung}{\lemma{\textnormal{\emph{Empfehlung}}}\Cendnote{\textnormal{siehe Paul Goldmann an Arthur Schnitzler, 10. 3. [1898]. Eine nachweisbare Verbindung
                     \textcolor{blue}{Schnitzler}s nach \textcolor{pink}{China}
                     verläuft über seinen Klassenkameraden \textcolor{blue}{Louis Friedmann},
                     der mit \textcolor{blue}{Rose Rosthorn} verheiratet war. Ihr Bruder \textcolor{blue}{Arthur Rosthorn}
                     leitete zwischen 1895 und 1898 die österreichische
                     Gesandtschaft in \textcolor{pink}{Bejing}. 
               }}}\label{K_L02845-1h}!\pend
           \pstart
           Ich habe in der letzten {\pb}Zeit viel merkwürdige Dinge
               geſehen, namentlich \textsc{\textcolor{pink}{Canton}{}\ledrightnote{\textcolor{pink}{Guangzhou}}}, das einfach aller Beſchreibung ſpottet.\pend
           \pstart
           Aber Alles in Allem wünſchte ich, ich wäre ſchon wieder zu Haufe. Das Reiſen hier iſt
               mit unſäglichen Strapazen und Entbehrungen verknüpft. Eſſen u. Wohnen ſind ſchlecht,
               die Hitze iſt \strikeout{\textcolor{gray}{×}} unmenſchlich, hält auch in der Nacht an, macht infolgedeſſen das Schlafen
               unmöglich. Die \textcolor{pink}{Deutſch}{}\ledrightnote{→\textcolor{pink}{Deutschland}}en hier
               ſind von einer {\pb}Gaſtfreundſchaft, die man zu Hauſe
               kaum ahnt; und doch ſind es nicht Leute \uline{unſerer} Art,
               und überhaupt liegt Alles, was uns betrifft u. unſer Leben ausmacht, in \textcolor{pink}{Europa}{}\ledrightnote{\textcolor{pink}{Europa}}. \strikeout{Ma} Man
               kann nicht Monate lang allein vom \textsc{Pittoresken} leben. Das
               iſt zu dünne Nahrung. Das Alles hier geſehen zu \uline{haben}, iſt ſchön; \strikeout{aber} aber es zu ſehen,
               erfordert mehr \strikeout{Selb}{ }{\pb}Selbſtüberwindung, Energie u. Entſagung, als man
               glauben möchte.\pend
           \pstart
           Ich ſende Dir anbei meine \label{K_L02845-2v}\edtext{\textcolor{green}{Photographie}{}\ledrightnote{\textcolor{green}{Paul Goldmann}}}{\lemma{\textnormal{\emph{Photographie}}}\Cendnote{\textnormal{Beilage nicht
                  erhalten}}}\label{K_L02845-2h} als Erforſcher \strikeout{fr\textcolor{gray}{e}} fremder Welttheile, gemacht vom \label{K_L02845-3v}\edtext{\textcolor{pink}{chin}{}\ledrightnote{→\textcolor{pink}{China}}eſiſchen \textcolor{blue}{Photographen}{}\ledrightnote{\textcolor{blue}{?? [Chinesischer Fotograf]}}}{\lemma{\textnormal{\emph{chineſiſchen Photographen}}}\Cendnote{\textnormal{nicht ermittelt}}}\label{K_L02845-3h}. Ich hoffe, baldigſt
               wieder von Dir zu hören, (Adreſſe bleibt: \textcolor{pink}{\textsc{Shanghai}}{}\ledrightnote{\textcolor{pink}{Shanghai}}, \textsc{\textcolor{pink}{Kais. Deutsches Postamt}{}\ledrightnote{→\textcolor{pink}{Deutsches Postamt in Shanghai}}}), wünſche Dir von Herzen Glück auf die \label{K_L02845-4v}\edtext{Sommer-Reiſe}{\lemma{\textnormal{\emph{Sommer-Reiſe}}}\Cendnote{\textnormal{Am
                     11. 7. 1898
                  begann \textcolor{blue}{Schnitzler}s große
                     »Sommer-Reiſe«. Zuerst fuhr er mit \textcolor{blue}{Marie Reinhard} nach \textcolor{pink}{Graz}, machte in der Umgebung wieder Radausflüge und kam am 20. 7. 1898 in \textcolor{pink}{Bad Gastein} an. Am 26. 7. 1898 ging es
                  für ihn weiter nach \textcolor{pink}{Salzburg} und am 31. 7. 1898 über \textcolor{pink}{München} nach \textcolor{pink}{Tegernsee}. Wieder über \textcolor{pink}{München} fuhr
                  er am 9. 8. 1898
                  weiter in die \textcolor{pink}{Schweiz}, wo er u. a. mit \textcolor{blue}{Hugo von Hofmannsthal} Rad fuhr. Am 28. 8. 1898 reiste \textcolor{blue}{Schnitzler} weiter nach \textcolor{pink}{Italien}, am 3. 9. 1898 kehrte er nach \textcolor{pink}{Wien} zurück.}}}\label{K_L02845-4h}, {\pb}\strikeout{\textcolor{gray}{g}} gute Stimmung (warum ſo \label{K_L02845-5v}\edtext{düſter}{\lemma{\textnormal{\emph{düſter}}}\Cendnote{\textnormal{Verstimmungen sind dem \emph{\textcolor{green}{Tagebuch}} in dieser Zeit (\textcolor{blue}{Goldmann} bezog sich wohl auf einen Brief \textcolor{blue}{Schnitzler}s von vor einigen Wochen) häufig zu entnehmen,
                  siehe z. B. A. S.: \emph{Tagebuch}, 13. 4. 1898.}}}\label{K_L02845-5h}, liebes
               Kind? warum Dich ſo unnütz quälen?) und frohe Erlebniſſe, bitte Dich, Deine \textcolor{blue}{Freundin}{}\ledrightnote{→\textcolor{blue}{Marie Reinhard}} recht herzlich zu
               grüßen, mich den Deinen zu empfehlen u. bin in Treue\pend
           \pstart
           Dein {\\[\baselineskip]}\spacefill\mbox{Paul Goldmann}\pend
           \leftskip=0em{}\pstart
           \noindent{}Viele Grüße an \textsc{\textcolor{blue}{Richard}{}\ledrightnote{\textcolor{blue}{Richard Beer-Hofmann}}} und \textsc{\textcolor{blue}{Leo}{}\ledrightnote{\textcolor{blue}{Leo Van-Jung}}}! \pend
           \pstart
           \textsc{\uuline{\label{K_L02845-22v}\edtext{verte}{\lemma{\textnormal{\emph{verte}}}\Cendnote{\textnormal{lateinisch: (Blatt)
                           wenden}}}\label{K_L02845-22h}}}\pend
           \pstart
           {\pb}Hörſt Du irgend etwas von dem kleinen \label{K_L02845-7v}\edtext{\textcolor{blue}{Mädchen}{}\ledrightnote{→\textcolor{blue}{Alice Ziegler}} aus \textsc{\textcolor{pink}{Prag}{}\ledrightnote{\textcolor{pink}{Prag}}}}{\lemma{\textnormal{\emph{Mädchen aus Prag}}}\Cendnote{\textnormal{siehe Paul Goldmann an Arthur Schnitzler, 19. 11. [1897]}}}\label{K_L02845-7h}? \strikeout{Gl\textcolor{gray}{au}} Wirſt Du ſie \label{K_L02845-11v}\edtext{dieſen Sommer
                     ſehen}{\lemma{\textnormal{\emph{dieſen Sommer
                     ſehen}}}\Cendnote{\textnormal{dazu kam es
                  nicht}}}\label{K_L02845-11h}?\pend
           \endnumbering\briefempfaengerindex{Schnitzler, Arthur@\textsc{Schnitzler, Arthur}!zzzGoldmann, Paul@\emph{von Paul Goldmann}!1898-05-162@{16. 5. 1898}|)be}\mylabel{h}\begin{anhang}\end{anhang}\normalsize

\doendnotes{C}
\bigskip
\vfill

\clearpage

\footnotesize

\lohead{\textsc{register}}

% Definiere theindex-Environment komplett neu ohne reledmac
\makeatletter
\renewenvironment{theindex}{%
  \section*{\indexname}%
  \setlength{\parindent}{0pt}%
  \setlength{\parskip}{0pt plus 0.3pt}%
  \let\item\@idxitem
}{%
  \clearpage
}
\makeatother

\IfFileExists{\jobname-pw.ind}{\input{\jobname-pw.ind}}{}

\end{document}

      