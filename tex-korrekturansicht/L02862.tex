%% latex-korrekturansicht-vorspann.tex
%% Vorspann für die Korrekturansicht.
%% Lädt die gemeinsame Datei latex-vorspann.tex mit gesetztem Schalter.

\newif\ifkorrekturansicht
\korrekturansichttrue

\input{../tex-inputs/latex-vorspann}


               \section[ Paul Goldmann an Arthur Schnitzler, 18. 10. {[}1898{]}]{Paul Goldmann an Arthur Schnitzler, 18. 10. {[}1898{]}}\nopagebreak\mylabel{v}\rehead{ }\normalsize\beginnumbering\briefempfaengerindex{Schnitzler, Arthur@\textsc{Schnitzler, Arthur}!zzzGoldmann, Paul@\emph{von Paul Goldmann}!1898-10-181@{18. 10. {[}1898{]}}|(be} \toendnotes[C]{\smallbreak\pagebreak[2]} \Standort{DLA, A:Schnitzler, HS.NZ85.1.3168.}
\physDesc{Brief, 1 Blatt, 2 Seiten
\newline{}Handschrift: blaue Tinte, deutsche Kurrent
\newline{}Schnitzler: mit Bleistift das Jahr »98« vermerkt }\toendnotes[C]{\smallbreak}\pstart
           \raggedleft{}{\pb}18. Oktober. An Bord der »\textsc{\textcolor{brown}{Anping}{}\ledrightnote{\textcolor{brown}{Anping Maru}}}«, zwiſchen \textsc{\textcolor{pink}{Taku}{}\ledrightnote{\textcolor{pink}{Taku Shi}}} und \textsc{\textcolor{pink}{Tschifu}{}\ledrightnote{\textcolor{pink}{Yantai}}}.\pend
           \pstart\center{}Mein lieber Freund,\pend\pstart
           Da ich fürchte, daß Dir beifolgendes \label{K_L02862-1v}\edtext{\textcolor{green}{Feuilleton}{}\ledrightnote{→\textcolor{green}{[Feuilleton über die Renaissance]}}}{\lemma{\textnormal{\emph{Feuilleton}}}\Cendnote{\textnormal{XXXX.}}}\label{K_L02862-1h} entgangen iſt, ſende ich es Dir der Sicherheit halber
               zu. Ich denke mir, es wird Dir recht kommen jetzt wo Du mit einer \textcolor{green}{Arbeit}{}\ledrightnote{→\textcolor{green}{Der Schleier der Beatrice. Schauspiel in fünf Akten}} über die \textsc{Renaissance} beſchäftigt biſt. Ich habe ſeit Langem nichts ſo Schönes über
               dieſe Zeit geleſen. Auch iſt eine Definition des »Styls« von \textsc{\textcolor{blue}{Feuerbach}{}\ledrightnote{\textcolor{blue}{Anselm Feuerbach}}} darin citirt, derentwegen allein es ſich ſchon verlohnt, Dir dieſes \textcolor{green}{Feuilleton}{}\ledrightnote{→\textcolor{green}{[Feuilleton über die Renaissance]}} der \textcolor{brown}{Frankfurter Zeitung}{}\ledrightnote{\textcolor{brown}{Frankfurter Zeitung}} auf Dem Umweg über das \textcolor{pink}{Gelbe Meer}{}\ledrightnote{\textcolor{pink}{Gelbes Meer}} nach \textcolor{pink}{Wien}{}\ledrightnote{\textcolor{pink}{Wien}} zu
               ſchicken. {\pb}Vergleiche insbeſondere die einfache und
               tiefe Schreibweiſe dieſes unbekannten \textcolor{blue}{Gelehrten}{}\ledrightnote{→\textcolor{blue}{?? [Verfasser eines Feuilletons über die Renaissance]}} mit dem \strikeout{\textcolor{gray}{unv}} unverſtändlichen Kauderwelſch, das die »Dichter« \textsc{\textcolor{blue}{Loris}{}\ledrightnote{\textcolor{blue}{Hugo von Hofmannsthal}}} und Genoſſen anzuwenden ſich befleißen, wenn ſie über die \textsc{Renaissance} ſchreiben.\pend
           \pstart
           Ich werde in einer halben Stunde wieder ſehr ſeekrank ſein.\pend
           \pstart
           Grüß’ Dich Gott, liebſter Freund!\pend
           \pstart
           Dein treuer {\\[\baselineskip]}\spacefill\mbox{Paul Goldmann}\pend
           \leftskip=0em{}\pstart
           \noindent{}Empfehlungen an Deine \textcolor{blue}{Freundin}{}\ledrightnote{→\textcolor{blue}{Marie Reinhard}}!\pend
           \endnumbering\briefempfaengerindex{Schnitzler, Arthur@\textsc{Schnitzler, Arthur}!zzzGoldmann, Paul@\emph{von Paul Goldmann}!1898-10-181@{18. 10. {[}1898{]}}|)be}\mylabel{h}\begin{anhang}\end{anhang}\normalsize

\doendnotes{C}
\bigskip
\vfill

\clearpage

\footnotesize

\lohead{\textsc{register}}

% Definiere theindex-Environment komplett neu ohne reledmac
\makeatletter
\renewenvironment{theindex}{%
  \section*{\indexname}%
  \setlength{\parindent}{0pt}%
  \setlength{\parskip}{0pt plus 0.3pt}%
  \let\item\@idxitem
}{%
  \clearpage
}
\makeatother

\IfFileExists{\jobname-pw.ind}{\input{\jobname-pw.ind}}{}

\end{document}

      