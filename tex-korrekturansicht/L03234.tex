%% latex-korrekturansicht-vorspann.tex
%% Vorspann für die Korrekturansicht.
%% Lädt die gemeinsame Datei latex-vorspann.tex mit gesetztem Schalter.

\newif\ifkorrekturansicht
\korrekturansichttrue

\input{../tex-inputs/latex-vorspann}


\renewcommand{\erwaehntePersonen}{Personen: Siegfried Jacobsohn}
\renewcommand{\erwaehnteInstitutionen}{Institutionen: Neue Freie Presse}
\renewcommand{\erwaehnteOrte}{Orte: Berlin, Hotel Continental (Berlin)}
\renewcommand{\erwaehnteWerke}{Werke: Zwischenspiel. Komödie in drei Akten}
\section[Paul Goldmann an Arthur Schnitzler, 20. 11. {[}1905{]}]{Paul Goldmann an Arthur Schnitzler, 20. 11. {[}1905{]}}
\nopagebreak\mylabel{v}
\rehead{ }\normalsize\beginnumbering\briefempfaengerindex{Schnitzler, Arthur@\textsc{Schnitzler, Arthur}!zzzGoldmann, Paul@\emph{von Paul Goldmann}!1905-11-202@{20. 11. {[}1905{]}}|(be}
\toendnotes[C]{\smallbreak\pagebreak[2]}\Standort{DLA, A:Schnitzler, HS.NZ85.1.3175.}
\physDesc{Visitenkarte
\newline{}Handschrift: Bleistift, deutsche Kurrent
\newline{}Versand: Stempel: »\nobreak{}\oindex{Hotel Continental (Berlin)@\textbf{Hotel Continental (Berlin)}, \emph{Hotel (K.HTL)}|pwk}Continental Hotel Berlin, {[}N{]}ov 20, 1\textsubscript{56}AM\nobreak{}«.  
\newline{}Schnitzler: mit Bleistift das Datum »20/11 {[}1{]}90\textcolor{gray}{5}« vermerkt }\toendnotes[C]{\smallbreak}
\pstart{}{\pb}Lieber Freund,\pend
\pstart
           Ich habe heut nach verſchiedenen Richtungen vergeblich
               nach Dir telephonirt u. Dich jetzt ebenſo vergeblich im \label{K-L03234-1v}\edtext{\textcolor{pink}{Hotel}{}\ledrightnote{\textcolor{pink}{Hotel Continental (Berlin)}}}{\lemma{\textnormal{\emph{Hotel}}}\Cendnote{\textnormal{Am 20. 11. 1905 hatte \textcolor{blue}{Schnitzler}
                  einer Probe von \emph{\textcolor{green}{Zwischenspiel}} beigewohnt, den
                  Nachmittag und Abend hatte er mit \textcolor{blue}{Siegfried
                     Jacobsohn} verbracht. Siehe A. S.: \emph{Tagebuch}, 20. 11. 1905.}}}\label{K-L03234-1h} geſucht. Heut habe ich
               leider keine Zeit mehr. Wenn Du {\pb}aber morgen um 7 Uhr{ }\introOben{}abends\introOben{}{ }\label{K-L03234-2v}\edtext{bei mir vorbeikommen }{\lemma{\textnormal{\emph{bei mir vorbeikommen }}}\Cendnote{\textnormal{\textcolor{blue}{Schnitzler} traf \textcolor{blue}{Goldmann} am 21. 11. 1905.}}}\label{K-L03234-2h} könnteſt, würde ich mich ſehr
               freuen, Dir die Hand zu drücken. Kannſt Du nicht kommen, ſo erbitte ich morgen{ }zwiſchen 6 u 7 Uhr abend\textcolor{gray}{s} telephoniſche
               Verſtändigung.\pend
           \pstart Herzlichen Gruß! \pend{}
\pstart
           \centering{}\textcolor{gray}{\textbf{D\textsuperscript{r}
                  Paul Goldmann }}\pend
           
\pstart
           \noindent{}\raggedleft{}\textcolor{gray}{\textbf{»\textcolor{brown}{Neue Freie Presse}{}\ledrightnote{\textcolor{brown}{Neue Freie Presse}}.«}}\pend
           \endnumbering\briefempfaengerindex{Schnitzler, Arthur@\textsc{Schnitzler, Arthur}!zzzGoldmann, Paul@\emph{von Paul Goldmann}!1905-11-202@{20. 11. {[}1905{]}}|)be}\mylabel{h}  \normalsize

\doendnotes{C}
\bigskip
\vfill

\clearpage

\footnotesize

\lohead{\textsc{register}}

% Definiere theindex-Environment komplett neu ohne reledmac
\makeatletter
\renewenvironment{theindex}{%
  \section*{\indexname}%
  \setlength{\parindent}{0pt}%
  \setlength{\parskip}{0pt plus 0.3pt}%
  \let\item\@idxitem
}{%
  \clearpage
}
\makeatother

\IfFileExists{\jobname-pw.ind}{\input{\jobname-pw.ind}}{}

\end{document}

      