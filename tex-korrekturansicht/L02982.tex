%% latex-korrekturansicht-vorspann.tex
%% Vorspann für die Korrekturansicht.
%% Lädt die gemeinsame Datei latex-vorspann.tex mit gesetztem Schalter.

\newif\ifkorrekturansicht
\korrekturansichttrue

\input{../tex-inputs/latex-vorspann}


\renewcommand{\erwaehntePersonen}{Personen: Emilie Mewes-Béha, Felix Salten}
\renewcommand{\erwaehnteOrte}{Orte: Edmund-Weiß-Gasse 7, Wien, XVIII., Währing}
\renewcommand{\erwaehnteWerke}{Werke: ?? [Journalistenstück], Der Schrei der Liebe. Novelle, Die Zeit, Studie, Unser Geburtstag}
\section[ Arthur Schnitzler an Felix Salten, 28. {[}9.{]} 1903]{Arthur Schnitzler an Felix Salten, 28. {[}9.{]} 1903}
\nopagebreak\mylabel{v}
\rehead{ }\normalsize\beginnumbering\briefempfaengerindex{Salten, Felix@\textsc{Salten, Felix}!zzzSchnitzler, Arthur@\emph{von Arthur Schnitzler}!1903-09-281@{28. {[}9.{]} 1903}|(be}
\toendnotes[C]{\smallbreak\pagebreak[2]}\Standort{Wienbibliothek im Rathaus, ZPH 1681, 2.1.516.}
\physDesc{Brief, 1 Blatt, 2 Seiten, 400 Zeichen
\newline{}Handschrift: Bleistift, deutsche Kurrent
\newline{}Ordnung: mit Bleistift von unbekannter Hand Nummerierung der Blätter des Konvoluts:
                                    »21« }\toendnotes[C]{\smallbreak}
\pstart
           \noindent{}\raggedleft{}{\pb}\textsc{\textcolor{pink}{Wien, XVIII}{}\ledrightnote{\textcolor{pink}{XVIII., Währing}}}{ }\textcolor{pink}{\textsc{Spöttelg. 7}}{}\ledrightnote{\textcolor{pink}{Edmund-Weiß-Gasse 7}}. \pend
           
\pstart
           \raggedleft{}28. \textcolor{gray}{9}. 903\pend
           
\pstart
           lieber, Ihrer freundlichen \label{K_L02982-1v}\edtext{Zuſage}{\lemma{\textnormal{\emph{Zuſage}}}\Cendnote{\textnormal{siehe Felix Salten an Arthur Schnitzler, 11. 8. 1903}}}\label{K_L02982-1h} vertrauend hatte ich an Frau \textcolor{blue}{B.}{}\ledrightnote{\textcolor{blue}{Emilie Mewes-Béha}}
               geſchrieben daſs ihre \label{K_L02982-2v}\edtext{\textcolor{green}{Skizze}{}\ledrightnote{{$\rightarrow$}\textcolor{green}{Studie}}}{\lemma{\textnormal{\emph{Skizze}}}\Cendnote{\textnormal{\textcolor{blue}{E. Mewes-Béha}: \emph{\textcolor{green}{Studie}}. In: \emph{\textcolor{green}{Die
                        Zeit}}, Jg. 2, Nr. 364, 4. 10. 1903, Die
                     Sonntags-Zeit, S. 2–3.}}}\label{K_L02982-2h} beſtimmt am geſtrigen So{\geminationn}tag erſcheint;\pend
           
\pstart
           bitte theilen Sie mir doch mit, ob ſie im nächſten \textcolor{green}{So{\geminationn}tagsheft}{}\ledrightnote{{$\rightarrow$}\textcolor{green}{Die Zeit}} ſicher gedruckt
               wird.\pend
           
\pstart
           {\pb}In Ihrem \label{K_L02982-3v}\edtext{\textcolor{green}{Geburtstagsfeuilleton}{}\ledrightnote{{$\rightarrow$}\textcolor{green}{Unser Geburtstag}}}{\lemma{\textnormal{\emph{Geburtstagsfeuilleton}}}\Cendnote{\textnormal{\textcolor{blue}{Felix Salten}: \emph{\textcolor{green}{Unser Geburtstag}}. In: \emph{\textcolor{green}{Die Zeit}}, Jg. 2, Nr. 357, 27. 9. 1903,
                     S. 1–3.}}}\label{K_L02982-3h} ſtecken die Elemente zu einer Tragikomödie des
               Journalismus. Was macht übrigens Ihr \label{K_L02982-4v}\edtext{\textcolor{green}{Journaliſtenſtück}{}\ledrightnote{{$\rightarrow$}\textcolor{green}{?? [Journalistenstück]}} und der \textcolor{green}{Schrei}{}\ledrightnote{\textcolor{green}{Der Schrei der Liebe. Novelle}}}{\lemma{\textnormal{\emph{Journaliſtenſtück … Schrei}}}\Cendnote{\textnormal{Das »\textcolor{green}{Journaliſtenſtück}« konnte nicht identifiziert werden. Zum \emph{\textcolor{green}{Schrei der Liebe}} siehe auch A. S.: \emph{Tagebuch}, 21. 10. 1903.}}}\label{K_L02982-4h}?\pend
           
\pstart
           Herzlichſt Ihr {\\[\baselineskip]}\spacefill\mbox{A.}\pend
           \leftskip=0em{}\endnumbering\briefempfaengerindex{Salten, Felix@\textsc{Salten, Felix}!zzzSchnitzler, Arthur@\emph{von Arthur Schnitzler}!1903-09-281@{28. {[}9.{]} 1903}|)be}\mylabel{h}  \normalsize

\doendnotes{C}
\bigskip
\vfill

\clearpage

\footnotesize

\lohead{\textsc{register}}

% Definiere theindex-Environment komplett neu ohne reledmac
\makeatletter
\renewenvironment{theindex}{%
  \section*{\indexname}%
  \setlength{\parindent}{0pt}%
  \setlength{\parskip}{0pt plus 0.3pt}%
  \let\item\@idxitem
}{%
  \clearpage
}
\makeatother

\IfFileExists{\jobname-pw.ind}{\input{\jobname-pw.ind}}{}

\end{document}

      