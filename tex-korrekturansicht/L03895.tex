%% latex-korrekturansicht-vorspann.tex
%% Vorspann für die Korrekturansicht.
%% Lädt die gemeinsame Datei latex-vorspann.tex mit gesetztem Schalter.

\newif\ifkorrekturansicht
\korrekturansichttrue

\input{../tex-inputs/latex-vorspann}


\section[Theodor Herzl an Arthur Schnitzler, 12. 9. 1893]{L03895 Theodor Herzl an Arthur Schnitzler, 12. 9. 1893}
\nopagebreak\mylabel{L03895v}
\rehead{ }\normalsize\beginnumbering\briefempfaengerindex{, @\textsc{, }!zzz, @\emph{von  }!1893-09-122@{12. 9. 1893}|(be}
\toendnotes[C]{\smallbreak\pagebreak[2]}\Standort{Wien, Österreichische Gesellschaft für Literatur, Abschrift Herzl.}
\physDesc{Brief, maschinenschriftliche Abschrift, 1 Blatt, 1 Seite, 397 Zeichen
\newline{}maschinell}
\buchAbdrucke{\weitereDrucke{Theodor Herzl: \emph{Briefe und autobiographische Notizen 1866–1895}. Bearbeitet von Johannes Wachten in Zusammenarbeit mit Chaya Harel, Daisy Tycho und Manfred Winkler. Berlin, Frankfurt am Main, Wien: \emph{Propyläen} 1983, S. 538–539 (Briefe und Tagebücher. Herausgegeben von Alex Bein, Hermann Greive, Moshe Schaerf, Julius H. Schoeps und Johannes Wachten, 1).} }\toendnotes[C]{\smallbreak}
\pstart
           {\pb}H 13\pend
           
\pstart
           \raggedleft{}12. 9. 1893. \pend
           
\pstart{}Lieber Freund!\pend\vspace{0.5em}
\pstart
           Bleibe \label{K_L03895-1v}\edtext{\textcolor{pink}{hier}\oindex{Baden bei Wien@\textbf{Baden bei Wien}, \emph{Hauptstadt}|pwv}{}\ledrightnote{{$\rightarrow$}\emph{\textcolor{pink}{Baden bei Wien}}}}{\lemma{\textnormal{\emph{hier}}}\Cendnote{\textnormal{\textcolor{pink}{Baden bei Wien}\oindex{Baden bei Wien@\textbf{Baden bei Wien}, \emph{Hauptstadt}|pwk}, vgl. A. S.: \emph{Tagebuch}, 22. 9. 1893 und 24. 9. 1893.}}}\label{K_L03895-1} ungefähr drei Wochen, werde
               mich sehr freuen Sie hier zu sehen und länger mit Ihnen zu dischkurieren. Ich bin
               meistens hier, selten in \textcolor{pink}{Wien}\oindex{Wien@\textbf{Wien}, \emph{Verwaltungsgebiet}|pw}{}\ledrightnote{\textcolor{pink}{Wien}}. Vorsichtsweise
               zeigen Sie doch Ihren Besuch telegraphisch an. Nächsten \label{K_L03895-2v}\edtext{Samstag}{\lemma{\textnormal{\emph{Samstag}}}\Cendnote{\textnormal{Am 16. 9. 1893
               reiste \textcolor{blue}{Schnitzler} Abends nach \textcolor{pink}{Salzburg}\oindex{Salzburg@\textbf{Salzburg}, \emph{Verwaltungsgebiet}|pwk}. Ein Treffen mit
                  \textcolor{blue}{Herzl}\pwindex{Herzl, Theodor 2.\,5.\,1860 Budapest – 3.\,7.\,1904 Edlach@\textsc{Herzl, Theodor} (2.\,5.\,1860 Budapest – 3.\,7.\,1904 Edlach), \emph{Schriftsteller, Journalist}|pwk} ist für diesen Tag nicht belegt.}}}\label{K_L03895-2} fahre ich nach \textcolor{pink}{Wien}\oindex{Wien@\textbf{Wien}, \emph{Verwaltungsgebiet}|pw}{}\ledrightnote{\textcolor{pink}{Wien}}; wenn ich kann, springe ich einen Augenblick
               zu Ihnen. Nicht sicher. \pend
           
\pstart
           Aber sicher meine herzliche Ergebenheit.{\\[\baselineskip]}Ihr{\\[\baselineskip]}\spacefill\mbox{Th. Herzl.}\pend
           \leftskip=0em{}\selectlanguage{ngerman}\endnumbering\briefempfaengerindex{, @\textsc{, }!zzz, @\emph{von  }!1893-09-122@{12. 9. 1893}|)be}\mylabel{L03895h}
\begin{anhang}
\end{anhang}\normalsize

\doendnotes{C}
\bigskip
\vfill

\clearpage

\footnotesize

\lohead{\textsc{register}}

% Definiere theindex-Environment komplett neu ohne reledmac
\makeatletter
\renewenvironment{theindex}{%
  \section*{\indexname}%
  \setlength{\parindent}{0pt}%
  \setlength{\parskip}{0pt plus 0.3pt}%
  \let\item\@idxitem
}{%
  \clearpage
}
\makeatother

\IfFileExists{\jobname-pw.ind}{\input{\jobname-pw.ind}}{}

\end{document}

      