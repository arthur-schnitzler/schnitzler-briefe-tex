%% latex-korrekturansicht-vorspann.tex
%% Vorspann für die Korrekturansicht.
%% Lädt die gemeinsame Datei latex-vorspann.tex mit gesetztem Schalter.

\newif\ifkorrekturansicht
\korrekturansichttrue

\input{../tex-inputs/latex-vorspann}


\renewcommand{\erwaehntePersonen}{Personen: Felix Salten, Ottilie Salten}
\renewcommand{\erwaehnteInstitutionen}{Institutionen: Semmeringbahn}
\renewcommand{\erwaehnteOrte}{Orte: Hinterbrühl, Hotel Radetzky, Klamm, Kleiner Krausel-Tunnel}
\renewcommand{\erwaehnteWerke}{}
\section[ Felix Salten an Arthur Schnitzler, 29. 5. 1907]{Felix Salten an Arthur Schnitzler, 29. 5. 1907}
\nopagebreak\mylabel{v}
\rehead{ }\normalsize\beginnumbering\briefempfaengerindex{Schnitzler, Arthur@\textsc{Schnitzler, Arthur}!zzzSalten, Felix@\emph{von Felix Salten}!1907-05-292@{29. 5. 1907}|(be}
\toendnotes[C]{\smallbreak\pagebreak[2]}\Standort{CUL, Schnitzler, B 89, B 1.}
\physDesc{Bildpostkarte, 119 Zeichen
\newline{}Handschrift: schwarze Tinte, lateinische Kurrent
\newline{}Versand: 1) Stempel: »\nobreak{}\oindex{Klamm@\textbf{Klamm}, \emph{P.PPL}|pwk}Klamm am Semmering, 29/5 07, 9–\textcolor{gray}{×} V\nobreak{}«.   2) mit Bleistift von unbekannter Hand Vermerk: »N\textsuperscript{o} 40«
\newline{}Ordnung: mit Bleistift von unbekannter Hand nummeriert: »230« }\toendnotes[C]{\smallbreak}\pstart{}{\pb}Herrn D\textsuperscript{r} Arthur Schnitzter\pend{}\pstart{}\textcolor{pink}{\strikeout{Mödling-}Hinterbrühl}{}\ledrightnote{\textcolor{pink}{Hinterbrühl}}\pend{}\pstart{}»\textcolor{pink}{Zum Radetzky}{}\ledrightnote{\textcolor{pink}{Hotel Radetzky}}«\pend{}
{\bigskip}
\pstart
           \noindent{}\centering{}{\pb}\textcolor{gray}{\textbf{\textcolor{brown}{Semmeringbahn}{}\ledrightnote{\textcolor{brown}{Semmeringbahn}}. \textcolor{pink}{Polerustunnel}{}\ledrightnote{\textcolor{pink}{Kleiner Krausel-Tunnel}}.}}\pend
           
\pstart
           {\pb}\textcolor{pink}{Klamm, a. S.}{}\ledrightnote{\textcolor{pink}{Klamm}}{ }29. V. 07\pend
           
\pstart
           Viele schöne Grüße von \textcolor{blue}{uns}{}\ledrightnote{{$\rightarrow$}\textcolor{blue}{Ottilie Salten}}
               zu Ihnen\pend
           \pstart \spacefill\mbox{Salten}\pend{}\endnumbering\briefempfaengerindex{Schnitzler, Arthur@\textsc{Schnitzler, Arthur}!zzzSalten, Felix@\emph{von Felix Salten}!1907-05-292@{29. 5. 1907}|)be}\mylabel{h}  \normalsize

\doendnotes{C}
\bigskip
\vfill

\clearpage

\footnotesize

\lohead{\textsc{register}}

% Definiere theindex-Environment komplett neu ohne reledmac
\makeatletter
\renewenvironment{theindex}{%
  \section*{\indexname}%
  \setlength{\parindent}{0pt}%
  \setlength{\parskip}{0pt plus 0.3pt}%
  \let\item\@idxitem
}{%
  \clearpage
}
\makeatother

\IfFileExists{\jobname-pw.ind}{\input{\jobname-pw.ind}}{}

\end{document}

      