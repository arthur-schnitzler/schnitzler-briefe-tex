%% latex-korrekturansicht-vorspann.tex
%% Vorspann für die Korrekturansicht.
%% Lädt die gemeinsame Datei latex-vorspann.tex mit gesetztem Schalter.

\newif\ifkorrekturansicht
\korrekturansichttrue

\input{../tex-inputs/latex-vorspann}


\section[Arthur Schnitzler an Stefan Zweig, 21. 2. 1927]{L03748 Arthur Schnitzler an Stefan Zweig, 21. 2. 1927}
\nopagebreak\mylabel{L03748v}
\rehead{ }\normalsize\beginnumbering\briefempfaengerindex{Zweig, Stefan@\textsc{Zweig, Stefan}!zzzSchnitzler, Arthur@\emph{von Arthur Schnitzler}!1927-02-211@{21. 2. 1927}|(be}
\toendnotes[C]{\smallbreak\pagebreak[2]}
\correspDesc{Versand  durch Arthur Schnitzler am 21. 2. 1927 in Wien
\newline{}Übermittlung  am 22. 2. 1927 in Wien
\newline{}Erhalt  durch Stefan Zweig im Zeitraum [23. 2. 1927 – 25. 2. 1927?] in Salzburg}\toendnotes[C]{\smallbreak}
\Standort{Jerusalem, National Library of Israel, ARC. Ms. Var. 305 1 58 Stefan Zweig Collection.}
\physDesc{Postkarte, 815 Zeichen
\newline{}Handschrift: schwarze Tinte, lateinische Kurrent
\newline{}Versand: Stempel: »\nobreak{}\oindex{XVIII., Währing@\textbf{XVIII., Währing}, \emph{Verwaltungsgebiet}|pwk}18/\textsubscript{1} Wien 110, 22. II. 27, \textcolor{gray}{9}\nobreak{}«.  }\toendnotes[C]{\smallbreak}\pstart{}{\pb}\label{T_L03748-1v}\edtext{\textcolor{gray}{\textbf{A. S.}}}{\lemma{\textnormal{\emph{A. S.}}}\Cendnote{\textnormal{ovaler Absenderkleber}}}\label{T_L03748-1}\pend{}\pstart{}\textcolor{pink}{\textcolor{gray}{\textbf{WIEN, XVIII.}}}\oindex{XVIII., Währing@\textbf{XVIII., Währing}, \emph{Verwaltungsgebiet}|pw}{}\ledrightnote{\textcolor{pink}{XVIII., Währing}}\pend{}\pstart{}\textcolor{pink}{\textcolor{gray}{\textbf{STERNWARTESTR. 71}}}\oindex{Wien@\textbf{Wien}!XVIII., Währing@\textbf{XVIII., Währing}!Sternwartestraße 71@\textbf{Sternwartestraße 71}, \emph{Wohngebäude}|pw}{}\ledrightnote{\textcolor{pink}{Sternwartestraße 71}}\pend{}{\bigskip}\pstart{}Hrn Doctor Stefan Zweig\pend{}\pstart{}\textcolor{pink}{Salzburg}\oindex{Salzburg@\textbf{Salzburg}, \emph{Verwaltungsgebiet}|pw}{}\ledrightnote{\textcolor{pink}{Salzburg}}.
               \pend{}\pstart{}\textcolor{pink}{Kapuzinerberg 5}\oindex{Paschinger Schlössl@\textbf{Paschinger Schlössl}, \emph{Wohngebäude}|pw}{}\ledrightnote{\textcolor{pink}{Paschinger Schlössl}}.\pend{}{\bigskip}\vspace{1em}
\pstart
           \raggedleft{}{\pb}\textcolor{pink}{Wien}\oindex{Wien@\textbf{Wien}, \emph{Verwaltungsgebiet}|pw}{}\ledrightnote{\textcolor{pink}{Wien}},
                        21. 2. 927\pend
           \vspace{0.5em}
\pstart
           lieber und verehrter Herr Doctor, für Ihre guten und schönen Worte
               anläßlich meiner \textcolor{green}{diagra{\geminationm}atischen
                  Versuche}\pwindex{Schnitzler, Arthur 15. 5. 1862 Wien – 21. 10. 1931 ebd.@\textsc{Schnitzler, Arthur} (15. 5. 1862 Wien – 21. 10. 1931 ebd.), \emph{Schriftsteller, Mediziner}!Geist im Wort und der Geist in der Tat@\strich\emph{Der Geist im Wort und der Geist in der Tat}|pwv}{}\ledrightnote{{$\rightarrow$}\emph{\textcolor{green}{Der Geist im Wort und der Geist in der Tat}}} dank ich Ihnen herzlichst. Dieser Dank reist Ihnen wohl schon in den
               Süden nach, wo die sich in der Arbeit und den wahrhaft verdienten Erfolgen dieses
               letzten Jahres erholen und zu neuen rüsten werden. Ich habe indess den \label{K_L03748-1v}\edtext{\textcolor{green}{Volpone}\pwindex{Zweig, Stefan 28.\,11.\,1881 Wien – 23.\,2.\,1942 Petrópolis@\textsc{Zweig, Stefan} (28.\,11.\,1881 Wien – 23.\,2.\,1942 Petrópolis), \emph{Schriftsteller}!Ben Jonsons »Volpone«.  Eine lieblose Komödie in drei Akten@\strich\emph{Ben Jonsons »Volpone«. Eine lieblose Komödie in drei Akten}|pw}{}\ledrightnote{\textcolor{green}{Ben Jonsons »Volpone«.  Eine lieblose Komödie in drei Akten}} auch in \textcolor{pink}{Berlin}\oindex{Berlin@\textbf{Berlin}, \emph{Hauptstadt}|pw}{}\ledrightnote{\textcolor{pink}{Berlin}}}{\lemma{\textnormal{\emph{Volpone auch in Berlin}}}\Cendnote{\textnormal{\textcolor{blue}{Schnitzler} besuchte die \textcolor{violet}{Theateraufführung von \emph{\textcolor{green}{Ben Jonsons »Volpone« Eine lieblose Komödie}\pwindex{Zweig, Stefan 28.\,11.\,1881 Wien – 23.\,2.\,1942 Petrópolis@\textsc{Zweig, Stefan} (28.\,11.\,1881 Wien – 23.\,2.\,1942 Petrópolis), \emph{Schriftsteller}!Ben Jonsons »Volpone«.  Eine lieblose Komödie in drei Akten@\strich\emph{Ben Jonsons »Volpone«. Eine lieblose Komödie in drei Akten}|pwk}}}\eventindex{Freie Volksbühne@\textbf{Freie Volksbühne}!Aufführung von Volpone, 30.12.1926@Aufführung von Volpone, 30.12.1926|pwk} in der Bearbeitung von \textcolor{blue}{Zweig}\pwindex{Zweig, Stefan 28.\,11.\,1881 Wien – 23.\,2.\,1942 Petrópolis@\textsc{Zweig, Stefan} (28.\,11.\,1881 Wien – 23.\,2.\,1942 Petrópolis), \emph{Schriftsteller}|pwk} am 30. 12. 1926 in der \textcolor{pink}{Freien Volksbühne}\oindex{Freie Volksbühne@\textbf{Freie Volksbühne}, \emph{Theater}|pwk}.
               }}}\label{K_L03748-1} gesehen, in einer Vorstellung, die
               trotz \textcolor{blue}{Steinrücks}\pwindex{Steinrück, Albert 20.\,5.\,1872 Wetterburg – 11.\,2.\,1929 Berlin@\textsc{Steinrück, Albert} (20.\,5.\,1872 Wetterburg – 11.\,2.\,1929 Berlin), \emph{Schauspieler}|pw}{}\ledrightnote{\textcolor{blue}{Albert Steinrück}}{[},{]} im
               ganzen ungleich roher war als \label{K_L03748-2v}\edtext{die im
               \textcolor{brown}{Wiener Burgtheater}\orgindex{Burgtheater@Burgtheater|pw}{}\ledrightnote{\textcolor{brown}{Burgtheater}}}{\lemma{\textnormal{\emph{die … Burgtheater}}}\Cendnote{\textnormal{\textcolor{blue}{Schnitzler} sah die \textcolor{violet}{Generalprobe von \emph{\textcolor{green}{Volpone}\pwindex{Zweig, Stefan 28.\,11.\,1881 Wien – 23.\,2.\,1942 Petrópolis@\textsc{Zweig, Stefan} (28.\,11.\,1881 Wien – 23.\,2.\,1942 Petrópolis), \emph{Schriftsteller}!Ben Jonsons »Volpone«.  Eine lieblose Komödie in drei Akten@\strich\emph{Ben Jonsons »Volpone«. Eine lieblose Komödie in drei Akten}|pwk}}}\eventindex{Burgtheater@\textbf{Burgtheater}!Generalprobe von Volpone oder Der Fuchs, 5.11.1926@Generalprobe von Volpone oder Der Fuchs, 5.11.1926|pwk} am 5. 11. 1926 im \textcolor{pink}{Burgtheater}\oindex{Wien@\textbf{Wien}!I., Innere Stadt@\textbf{I., Innere Stadt}!Burgtheater@\textbf{Burgtheater}, \emph{Theater}|pwk}.
               }}}\label{K_L03748-2} aber von starker Wirkung. Ich freue mich Ihren
               nächsten Werken entgegen und hoffe wir sprechen bald  wieder miteinander – es muſs ja
               nicht gerade in 1600 Meter Höhe sein. \textcolor{pink}{Kapuzinerberg}\oindex{Kapuzinerberg@\textbf{Kapuzinerberg}, \emph{Berg}|pw}{}\ledrightnote{\textcolor{pink}{Kapuzinerberg}} die \textcolor{pink}{Sternwartestraße}\oindex{Wien@\textbf{Wien}!XVIII., Währing@\textbf{XVIII., Währing}!Sternwartestraße@\textbf{Sternwartestraße}, \emph{Straße}|pw}{}\ledrightnote{\textcolor{pink}{Sternwartestraße}} werden auch
               zur Noth genügen\pend
           \pstart {\pb}Auf Wiedersehen also, und alles herzliche von Ihrem
                  \spacefill\mbox{Arthur Schnitzler}\pend{}\selectlanguage{ngerman}\endnumbering\briefempfaengerindex{Zweig, Stefan@\textsc{Zweig, Stefan}!zzzSchnitzler, Arthur@\emph{von Arthur Schnitzler}!1927-02-211@{21. 2. 1927}|)be}\mylabel{L03748h}  \normalsize

\doendnotes{C}
\bigskip
\vfill

\clearpage

\footnotesize

\lohead{\textsc{register}}

% Definiere theindex-Environment komplett neu ohne reledmac
\makeatletter
\renewenvironment{theindex}{%
  \section*{\indexname}%
  \setlength{\parindent}{0pt}%
  \setlength{\parskip}{0pt plus 0.3pt}%
  \let\item\@idxitem
}{%
  \clearpage
}
\makeatother

\IfFileExists{\jobname-pw.ind}{\input{\jobname-pw.ind}}{}

\end{document}

      