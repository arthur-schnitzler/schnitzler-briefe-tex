%% latex-korrekturansicht-vorspann.tex
%% Vorspann für die Korrekturansicht.
%% Lädt die gemeinsame Datei latex-vorspann.tex mit gesetztem Schalter.

\newif\ifkorrekturansicht
\korrekturansichttrue

\input{../tex-inputs/latex-vorspann}


               \section[ Paul Goldmann an Arthur Schnitzler, 13. 9. 1897]{Paul Goldmann an Arthur Schnitzler, 13. 9. 1897}\nopagebreak\mylabel{v}\rehead{ }\normalsize\beginnumbering\briefempfaengerindex{Schnitzler, Arthur@\textsc{Schnitzler, Arthur}!zzzGoldmann, Paul@\emph{von Paul Goldmann}!1897-09-131@{13. 9. 1897}|(be} \toendnotes[C]{\smallbreak\pagebreak[2]} \Standort{DLA, A:Schnitzler, HS.NZ85.1.3167.}
\physDesc{Brief, 1 Blatt, 4 Seiten
\newline{}Handschrift: blaue Tinte, deutsche Kurrent\newline{}Beilage: eigenhändiger Brief, 1 Blatt, 2 Seiten, Handschrift: blaue
                                 Tinte, deutsche Kurrent. Der Brief wurde von \textcolor{blue}{Schnitzler} weitergereicht und findet sich
                                 heute in der \emph{\textcolor{brown}{Houghton Library}},
                                    Harvard, Signatur 825.978 }\toendnotes[C]{\smallbreak}\pstart
           \noindent{}{\pb}\textcolor{brown}{\textcolor{gray}{\textbf{\textsc{Frankfurter Zeitung}}}}{}\ledrightnote{\textcolor{brown}{Frankfurter Zeitung}}\hfill \textcolor{gray}{\textbf{\textcolor{pink}{Frankfurt a. M.}{}\ledrightnote{\textcolor{pink}{Frankfurt am Main}},}}{ }13. September \textcolor{gray}{\textbf{189}}7.\pend
           \pstart
           \textsc{\textcolor{gray}{\textbf{und}}}\pend
           \pstart
           \textcolor{gray}{\textbf{\textsc{Handelsblatt.}}}\pend
           \pstart
           \textcolor{gray}{\textbf{\textsc{\textcolor{brown}{Redaction}{}\ledrightnote{→\textcolor{brown}{Frankfurter Zeitung}}.\footnote{\noindent{}\textcolor{gray}{\textbf{\textsc{Für die \textcolor{brown}{Redaktion} bestimmte Briefe und Sendungen
                                    wolle man \so{nicht} an die Person eines
                                    Redakteurs, sondern stets \textbf{an die Redaktion der
                                          \textcolor{brown}{Frankfurter Zeitung}} adressiren}}}.}}}}\pend
           \pstart
           \textcolor{gray}{\textbf{\textsc{Telegramm-Adresse:}}}\pend
           \pstart
           \textcolor{gray}{\textbf{\textsc{\textcolor{brown}{Zeitung}{}\ledrightnote{→\textcolor{brown}{Frankfurter Zeitung}}{ }\textcolor{pink}{Frankfurt Main}{}\ledrightnote{\textcolor{pink}{Frankfurt am Main}}.}}}\pend
           \pstart\center{}Mein lieber Freund,\pend\pstart
           Erſt ſeit wenigen Stunden bin ich in \textcolor{pink}{Frankfurt}{}\ledrightnote{\textcolor{pink}{Frankfurt am Main}}.
               Ich habe den \label{K_L02823-1v}\edtext{Brief}{\lemma{\textnormal{\emph{Brief}}}\Cendnote{\textnormal{Bezug unklar}}}\label{K_L02823-1h} gleich nach \textsc{\textcolor{pink}{Paris}{}\ledrightnote{\textcolor{pink}{Paris}}} geſandt\strikeout{.} und hoffe, daß die Verzögerung, die
               durch meine verſpätete Ankunft in \textcolor{pink}{Frankfurt}{}\ledrightnote{\textcolor{pink}{Frankfurt am Main}}
               entſtanden iſt, keine ſtörenden Folgen hat.\pend
           \pstart
           Ich danke Dir für die lieben Mittheilungen Deines Briefes. Der \strikeout{\textcolor{gray}{×}\-\textcolor{gray}{×}\-\textcolor{gray}{×}h\textcolor{gray}{×}\-\textcolor{gray}{×}}{ }\textcolor{blue}{Gattin}{}\ledrightnote{→\textcolor{blue}{Rosa Freudenthal}} des \textcolor{blue}{Rechtsgelehrten}{}\ledrightnote{→\textcolor{blue}{Hermann Freudenthal}} geht es hoffentlich \label{K_L02823-2v}\edtext{beſſer}{\lemma{\textnormal{\emph{beſſer}}}\Cendnote{\textnormal{siehe A. S.: \emph{Tagebuch}, 3. 9. 1897}}}\label{K_L02823-2h}. Grüß’ ſie ſchön von mir.\pend
           \pstart
           Du ſelbſt wirſt \strikeout{hof} wohl bald die \strikeout{S} Ruhe zur Arbeit {\pb}finden. Solche Übergangszeiten vom Sommer zum Winter ſind immer etwas unbehaglich
               und bei Dir drängt ſich gerade jetzt außergewöhnlich Vieles zuſammen. Wird ſich ſchon
               Alles lichten und klären.\pend
           \pstart
           Mein \textcolor{blue}{Schwager}{}\ledrightnote{→\textcolor{blue}{Josef Rosengart}} läßt Dich
               grüßen u. Dir ſagen, daß es lächerlich iſt, ſich über \label{K_L02823-3v}\edtext{Ohrenklingen}{\lemma{\textnormal{\emph{Ohrenklingen}}}\Cendnote{\textnormal{\textcolor{blue}{Schnitzler} litt seit
                     Herbst 1896 an Otosklerose – einer Verknöcherung des Innenohrs mit
                  zunehmender Schwerhörigkeit.}}}\label{K_L02823-3h} Sorgen zu machen. Nach ſeiner Erfahrung gibt
               es kaum einen Menſchen, deſſen Ohren ganz in Ordnung wären. Er hat mir geſagt: wenn
               ich darauf achtete, würde ich auch bald Ohrenklingen \strikeout{be\textcolor{gray}{i}} bei mir bemerken, und mir ſcheint in der that\strikeout{,}
               mehrmals am Tage, daß es auch bei mir klingt. {\pb}Wer
               wird ſich aber dabei aufhalten? Schade um jede Stunde Deines ſchönen Lebens, welche
               Du Dir dadurch verbitterſt.\pend
           \pstart
           Mein Fuß iſt geheilt. Ich bleibe wohl noch bis Ende der Woche \textcolor{pink}{hier}{}\ledrightnote{→\textcolor{pink}{Frankfurt am Main}} u. bitte Dich, mir hieher \strikeout{(\textsc{ Rosse}} (\textsc{\textcolor{pink}{Rossertstraſse 15}{}\ledrightnote{\textcolor{pink}{Rossertstraße}}}) zu ſchreiben, falls Du mir noch etwas zu ſagen haſt oder falls Dein \label{K_L02823-11v}\edtext{\textcolor{blue}{Sohn}{}\ledrightnote{→\textcolor{blue}{?? [Totgeborener Sohn von Arthur Schnitzler und Marie Reinhard]}} ankommt}{\lemma{\textnormal{\emph{Sohn ankommt}}}\Cendnote{\textnormal{Der \textcolor{blue}{Sohn} von \textcolor{blue}{Schnitzler}
                  und \textcolor{blue}{Marie Reinhard} wurde am 24. 9. 1897
                  totgeboren.}}}\label{K_L02823-11h}.\pend
           \pstart
           Deine \textcolor{blue}{Freundin}{}\ledrightnote{→\textcolor{blue}{Marie Reinhard}} grüße recht
               herzlich von mir. Ich habe mich ſehr gefreut zu hören, daß es ihr gut geht.\pend
           \pstart
           Ich habe \textsc{\textcolor{blue}{Richard}{}\ledrightnote{\textcolor{blue}{Richard Beer-Hofmann}}s}{ }{\pb}Hausnummer vergeſſen. Du biſt wohl ſo gut, ihm den
               beifolgenden Brief zu übergeben.\pend
           \pstart
           Ich grüße Dich von {\\[\baselineskip]}Herzen Dein treuer {\\[\baselineskip]}\spacefill\mbox{Paul Goldm}\pend
           \leftskip=0em{}{\bigskip}\pstart
           \noindent{}{\pb}\textcolor{brown}{\textcolor{gray}{\textbf{\textsc{Frankfurter Zeitung}}}}{}\ledrightnote{\textcolor{brown}{Frankfurter Zeitung}}\hfill \textcolor{gray}{\textbf{\textcolor{pink}{Frankfurt a. M.}{}\ledrightnote{\textcolor{pink}{Frankfurt am Main}},}}{ }13. September \textcolor{gray}{\textbf{189}}7.\pend
           \pstart
           \textsc{\textcolor{gray}{\textbf{und}}}\pend
           \pstart
           \textcolor{gray}{\textbf{\textsc{Handelsblatt.}}}\pend
           \pstart
           \textcolor{gray}{\textbf{\textsc{\textcolor{brown}{Redaction}{}\ledrightnote{→\textcolor{brown}{Frankfurter Zeitung}}.\footnote{\noindent{}\textcolor{gray}{\textbf{\textsc{Für die \textcolor{brown}{Redaktion} bestimmte Briefe und Sendungen
                                    wolle man \so{nicht} an die Person eines
                                    Redakteurs, sondern stets \textbf{an die Redaktion der
                                          \textcolor{brown}{Frankfurter Zeitung}} adressiren}}}.}}}}\pend
           \pstart
           \textcolor{gray}{\textbf{\textsc{Telegramm-Adresse:}}}\pend
           \pstart
           \textcolor{gray}{\textbf{\textsc{\textcolor{brown}{Zeitung}{}\ledrightnote{→\textcolor{brown}{Frankfurter Zeitung}}{ }\textcolor{pink}{Frankfurt Main}{}\ledrightnote{\textcolor{pink}{Frankfurt am Main}}.}}}\pend
           \pstart\center{}Mein lieber \textcolor{blue}{\textsc{Richard}}{}\ledrightnote{\textcolor{blue}{Richard Beer-Hofmann}},\pend\pstart
           Erſt dieſer Tage haben meine Irrfahrten in Frankfurt geendet. Ich ſand hier Deinen
               lieben Brief vor und \strikeout{ſ} erſah daraus mit inniger
               Freude, daß das große \label{K_L02823-88v}\edtext{Ereigniß}{\lemma{\textnormal{\emph{Ereigniß}}}\Cendnote{\textnormal{Am 4. 9. 1897 war
                     \textcolor{blue}{Mirjam Beer-Hofmann}, das erste Kind von
                     \textcolor{blue}{Richard} und \textcolor{blue}{Paula Beer-Hofmann} auf die Welt gekommen.}}}\label{K_L02823-88h} ſich
               vollzogen hat. Daß es \textcolor{blue}{Mirjam}{}\ledrightnote{\textcolor{blue}{Mirjam Beer-Hofmann}} war und nicht
               Jehoschuah, überraſcht mich nicht. Es mußte ja \textcolor{blue}{Mirjam}{}\ledrightnote{\textcolor{blue}{Mirjam Beer-Hofmann}} ſein.\pend
           \pstart
           Der alte jüdiſche Gott auf den Du ſo große Stücke hältſt, \strikeout{ſ\textcolor{gray}{×}\-\textcolor{gray}{×}} wird hoffentlich einmal an Deinem Kinde zeigen, was er kann. Er ſoll ein {\pb}liebes und frohes Menſchenkind daraus machen. Dir ſelbſt aber möge die
               kleine \textcolor{blue}{Mirjam}{}\ledrightnote{\textcolor{blue}{Mirjam Beer-Hofmann}}{ }\strikeout{\textcolor{gray}{ein}} nur Freuden bringen und Seelenfrieden in den düſteren Stunden des Grübelns und
               der Selbſtquälerei.\pend
           \pstart
           Ich \strikeout{\textcolor{gray}{×}\-\textcolor{gray}{×}\-\textcolor{gray}{×}\-\textcolor{gray}{×}} aber will ſie ſtets ſehr lieb haben.\pend
           \pstart
           Überbringe der \textcolor{blue}{Mutter}{}\ledrightnote{→\textcolor{blue}{Paula Beer-Hofmann}} Deines
               Kinds meine herzlichſten Glückwünſche und Grüße und ſei ſelbſt u. Herzen umarmt.\pend
           \pstart
           Dein treuer{\\[\baselineskip]}\spacefill\mbox{Paul Goldmann}\pend
           \leftskip=0em{}\endnumbering\briefempfaengerindex{Schnitzler, Arthur@\textsc{Schnitzler, Arthur}!zzzGoldmann, Paul@\emph{von Paul Goldmann}!1897-09-131@{13. 9. 1897}|)be}\mylabel{h}  \normalsize

\doendnotes{C}
\bigskip
\vfill

\clearpage

\footnotesize

\lohead{\textsc{register}}

% Definiere theindex-Environment komplett neu ohne reledmac
\makeatletter
\renewenvironment{theindex}{%
  \section*{\indexname}%
  \setlength{\parindent}{0pt}%
  \setlength{\parskip}{0pt plus 0.3pt}%
  \let\item\@idxitem
}{%
  \clearpage
}
\makeatother

\IfFileExists{\jobname-pw.ind}{\input{\jobname-pw.ind}}{}

\end{document}

      