%% latex-korrekturansicht-vorspann.tex
%% Vorspann für die Korrekturansicht.
%% Lädt die gemeinsame Datei latex-vorspann.tex mit gesetztem Schalter.

\newif\ifkorrekturansicht
\korrekturansichttrue

\input{../tex-inputs/latex-vorspann}


               \section[ Paul Goldmann an Arthur Schnitzler, 30. 12. {[}1897{]}]{Paul Goldmann an Arthur Schnitzler, 30. 12. {[}1897{]}}\nopagebreak\mylabel{v}\rehead{ }\normalsize\beginnumbering\briefempfaengerindex{Schnitzler, Arthur@\textsc{Schnitzler, Arthur}!zzzGoldmann, Paul@\emph{von Paul Goldmann}!1897-12-303@{30. 12. {[}1897{]}}|(be} \toendnotes[C]{\smallbreak\pagebreak[2]} \Standort{DLA, A:Schnitzler, HS.NZ85.1.3167.}
\physDesc{Brief, 1 Blatt, 3 Seiten
\newline{}Handschrift: blaue Tinte, deutsche Kurrent
\newline{}Schnitzler: 1) Die obere und untere Seitenkante mutmaßlich beim Öffnen des
                                 Briefes mit Brieföffner abgeschnitten, was auf der zweiten Seite zu
                                 minimaler Textbeschädigung der letzten Zeile führte. 2) mit Bleistift das Jahr »97« vermerkt3) mit rotem Buntstift eine Unterstreichung}\toendnotes[C]{\smallbreak}\pstart
           \noindent{}{\pb}\textcolor{gray}{\textbf{\textbf{\textcolor{brown}{Frankfurter Zeitung}{}\ledrightnote{\textcolor{brown}{Frankfurter Zeitung}}}}}\pend
           \pstart
           \textcolor{gray}{\textbf{(\textcolor{brown}{\begin{otherlanguage}{french}Gazette de Francfort\end{otherlanguage}}{}\ledrightnote{\textcolor{brown}{Frankfurter Zeitung}}).}}\pend
           \pstart
           \textcolor{gray}{\textbf{\textbf{\begin{otherlanguage}{french}Fondateur M.\end{otherlanguage}{ }\textcolor{blue}{L. Sonnemann}{}\ledrightnote{\textcolor{blue}{Leopold Sonnemann}}.}}}\pend
           \pstart
           \begin{otherlanguage}{french}\textcolor{gray}{\textbf{Journal politique, financier,}}\end{otherlanguage}\pend
           \pstart
           \begin{otherlanguage}{french}\textcolor{gray}{\textbf{commercial et littéraire.}}\end{otherlanguage}\pend
           \pstart
           \begin{otherlanguage}{french}\textcolor{gray}{\textbf{\textbf{Paraissant trois fois par jour.}}}\end{otherlanguage}\pend
           \pstart
           \begin{otherlanguage}{french}\textcolor{gray}{\textbf{\textbf{Bureau à \textcolor{pink}{Paris}{}\ledrightnote{\textcolor{pink}{Paris}}}}}\end{otherlanguage}\hfill \textsc{\textcolor{pink}{Paris}{}\ledrightnote{\textcolor{pink}{Paris}}}, 30. December.\pend
           \pstart
           \begin{otherlanguage}{french}\textcolor{gray}{\textbf{\textbf{\textcolor{pink}{10 Rue de la Bourse}{}\ledrightnote{\textcolor{pink}{rue de la Bourse}}.}}}\end{otherlanguage}\pend
           \pstart\center{}Mein lieber Freund,\pend\pstart
           Ich erwarte täglich einen Brief von Dir und bin ſehr traurig, daß er gar nicht kommt.
               Biſt Du unwohl? Oder was geht ſonſt vor? Ich bin recht ungeduldig, es zu wiſſen, denn
               Deine letzten Briefe waren nicht gerade erheiternd.\pend
           \pstart
           Ich will Dir heut nur ein recht glückliches neues Jahr
               wünſchen. Und das Gleiche Deiner \textcolor{blue}{Freundin}{}\ledrightnote{→\textcolor{blue}{Marie Reinhard}}.\pend
           \pstart
           Die \label{K_L02835-77v}\edtext{Adreſſe der Frau \textcolor{blue}{\textsc{Altmann}}{}\ledrightnote{\textcolor{blue}{Emma Altmann}}}{\lemma{\textnormal{\emph{Adreſſe der Frau Altmann}}}\Cendnote{\textnormal{Sie wohnte am \textcolor{pink}{Lobkowitzplatz 1}. Ein Besuch \textcolor{blue}{Schnitzler}s bei ihr ist für die kommenden Tage nicht
                  belegt.}}}\label{K_L02835-77h}, weiß ich nicht. Willſt Du ſo gut ſein, die \label{K_L02835-44v}\edtext{beiliegende Karte}{\lemma{\textnormal{\emph{beiliegende Karte}}}\Cendnote{\textnormal{Beilage nicht erhalten}}}\label{K_L02835-44h} an ſie zu befördern?\pend
           \pstart
           {\pb}In meiner Exiſtenz wird es wohl in
               einiger Zeit \strikeout{\textcolor{gray}{e}i} eine Änderung geben. Ich bin mehr \textsc{\textcolor{pink}{Paris}{}\ledrightnote{\textcolor{pink}{Paris}}}\textcolor{gray}{-}müde als je. Ich habe meinem \textcolor{blue}{Chef}{}\ledrightnote{→\textcolor{blue}{Leopold Sonnemann}} geſchrieben, daß ich nach \textsc{\textcolor{pink}{Berlin}{}\ledrightnote{\textcolor{pink}{Berlin}}} will. Aber es ſcheint, daß das nicht geht, weil unſer \label{K_L02835-12v}\edtext{\textcolor{blue}{\textcolor{pink}{Berlin}{}\ledrightnote{\textcolor{pink}{Berlin}}er politiſcher Correſpondent}{}\ledrightnote{→\textcolor{blue}{[?? polititscher Korrespondent der Frankfurter Zeitung in Berlin 1897]}}}{\lemma{\textnormal{\emph{Berliner … Correſpondent}}}\Cendnote{\textnormal{nicht ermittelt}}}\label{K_L02835-12h}, der meine
               Rivalität fürchtet, gegen mich hetzt. Zur Zeit beſteht das Project, mich auf ein Jahr
               nach \textsc{\textcolor{pink}{China}{}\ledrightnote{\textcolor{pink}{China}}} zu ſchicken. Auch von \textsc{\textcolor{pink}{Wien}{}\ledrightnote{\textcolor{pink}{Wien}}} war die Rede. Aber ſo froh ich wäre, in \textsc{\textcolor{pink}{Wien}{}\ledrightnote{\textcolor{pink}{Wien}}} mit Euch zu leben, ſo ſehe ich doch \strikeout{in
                     \textcolor{gray}{n}} bei kühler Überlegung, daß ich dort keinerlei Zukunft habe. Es gibt dort nur
               die \textcolor{brown}{Neue Freie Preſſe}{}\ledrightnote{\textcolor{brown}{Neue Freie Presse}} und ich bin \strikeout{zu} doch zu gut, um bei \uline{den} Leuten Jahre lang zu \label{K_L02835-2v}\edtext{antichambriren}{\lemma{\textnormal{\emph{antichambriren}}}\Cendnote{\textnormal{sich dienstfertig im Vorzimmer einer mächtigen
                     Person aufhalten, um
                  dadurch Gunst zu erlangen}}}\label{K_L02835-2h}. Auch würde meine Verſetzung nach \textsc{\textcolor{pink}{Wien}{}\ledrightnote{\textcolor{pink}{Wien}}} eine Gehalts-Reduction, beinahe um die Hälfte, bedeuten. Gott weiß, was bei
               alledem noch herauskommen wird! Bitte \label{T_L02835-1v}\edtext{\damage{ſprich zu}}{\lemma{\textnormal{\emph{ſprich zu}}}\Cendnote{\textnormal{am unteren Rand der beschädigten
                  Seite}}}\label{T_L02835-1h} keinem Menſchen darüber!\pend
           \pstart
           {\pb}Dabei wird es mit meinem \label{K_L02835-33v}\edtext{Auge}{\lemma{\textnormal{\emph{Auge}}}\Cendnote{\textnormal{siehe Paul Goldmann an Arthur Schnitzler, 2. [1.? 1897]}}}\label{K_L02835-33h} beinahe täglich ſchlechter.\pend
           \pstart
           Das kleine \textcolor{blue}{Fräulein}{}\ledrightnote{→\textcolor{blue}{Alice Ziegler}} aus \textsc{\textcolor{pink}{Prag}{}\ledrightnote{\textcolor{pink}{Prag}}} hat mir ihre Photographie geſchickt. Was für ein liebes und ſüßes Geſicht!
               Glaubſt Du wirklich, ich ſollte nicht? Glaubſt Du ich \uline{dürſte} überhaupt? Haſt Du übrigens eine Ahnung, ob die \textcolor{blue}{Leute}{}\ledrightnote{→\textcolor{blue}{Charlotte Bondy}{\newline}→\textcolor{blue}{Vít Šalomoun Bondy}{\newline}→\textcolor{blue}{Alice Ziegler}} Geld haben?\pend
           \pstart
           Sei von Herzen gegrüßt, liebſter Freund, und ſchreib’ mir bald!\pend
           \pstart
           Dein treuer {\\[\baselineskip]}\spacefill\mbox{Paul Goldmann.}\pend
           \leftskip=0em{}\pstart
           \noindent{}Deiner Frau \textcolor{blue}{Mutter}{}\ledrightnote{→\textcolor{blue}{Louise Schnitzler}} bitte
                  ich meine ergebenen Neujahrs-Glückwünſche auszurichten.\pend
           \endnumbering\briefempfaengerindex{Schnitzler, Arthur@\textsc{Schnitzler, Arthur}!zzzGoldmann, Paul@\emph{von Paul Goldmann}!1897-12-303@{30. 12. {[}1897{]}}|)be}\mylabel{h}  \normalsize

\doendnotes{C}
\bigskip
\vfill

\clearpage

\footnotesize

\lohead{\textsc{register}}

% Definiere theindex-Environment komplett neu ohne reledmac
\makeatletter
\renewenvironment{theindex}{%
  \section*{\indexname}%
  \setlength{\parindent}{0pt}%
  \setlength{\parskip}{0pt plus 0.3pt}%
  \let\item\@idxitem
}{%
  \clearpage
}
\makeatother

\IfFileExists{\jobname-pw.ind}{\input{\jobname-pw.ind}}{}

\end{document}

      