%% latex-korrekturansicht-vorspann.tex
%% Vorspann für die Korrekturansicht.
%% Lädt die gemeinsame Datei latex-vorspann.tex mit gesetztem Schalter.

\newif\ifkorrekturansicht
\korrekturansichttrue

\input{../tex-inputs/latex-vorspann}


               \section[Arthur Schnitzler an Georg Brandes, 3. 8. 1914]{ Arthur Schnitzler an Georg Brandes, 3. 8. 1914}\nopagebreak\mylabel{v}\rehead{ }\normalsize\beginnumbering\briefempfaengerindex{Brandes, Georg@\textsc{Brandes, Georg}!zzzSchnitzler, Arthur@\emph{von Arthur Schnitzler}!1914-08-031@{3. 8. 1914}|(be} \toendnotes[C]{\smallbreak\pagebreak[2]} \Standort{Kopenhagen, Det Kongelige Bibliotek, Georg Brandes Arkiv, box 125.}
\physDesc{Brief, 2 Blätter, 6 Seiten (Paginierung 1–6)
\newline{}Handschrift: schwarze Tinte, lateinische Kurrent\newline{}Ordnung: mit Bleistift von unbekannter Hand beschriftet:
                                        »A. Schnitzler« und nummeriert
                                        »35.«, das zweite Blatt erneut mit Bleistift
                                    datiert: »3/8 14« }\buchAbdrucke{\weitereDrucke{Georg Brandes, Arthur Schnitzler: \emph{Ein Briefwechsel}. Hg. Kurt Bergel. Bern: \emph{Francke} 1956, S. 107–109.} }\toendnotes[C]{\smallbreak}\pstart
           \noindent{}{\pb}\textcolor{pink}{\textcolor{gray}{\textbf{CRESTA PALACE}}}{}\ledrightnote{\textcolor{pink}{Cresta Palace}}{\\}\textcolor{pink}{\textcolor{gray}{\textbf{CELERINA B. S\textsuperscript{T} MORITZ}}}{}\ledrightnote{\textcolor{pink}{Celerina}}{\\}\textcolor{gray}{\textbf{DIRECTION: \textcolor{blue}{CARL
                                SONDERHOF}{}\ledrightnote{\textcolor{blue}{Carl Sonderhof}}}}\pend
           \pstart
           \raggedleft{}3. August 914.\pend
           \pstart{}mein sehr verehrter Freund,\pend\pstart
           ich erfahre hier, auf einem \textcolor{blue}{Umweg über \textcolor{pink}{Schweden}{}\ledrightnote{\textcolor{pink}{Schweden}}}{}\ledrightnote{→\textcolor{blue}{?? [Schwede, mit dem Arthur Schnitzler über den Nobelpreis spricht]}}, dass der
                    literarische \textcolor{brown}{Nobelpreis}{}\ledrightnote{\textcolor{brown}{Nobelpreis}} dieses Jahr an \textcolor{pink}{Oesterreich}{}\ledrightnote{\textcolor{pink}{Österreich}} fallen soll und daß, unter oder
                    neben anderen, ich hiefür nicht unerheblich in Betracht kommen dürfte. Nun weiß
                    ich aber, daß von \textcolor{pink}{Schweden}{}\ledrightnote{\textcolor{pink}{Schweden}} aus vor Entscheidung
                    der Angelegenheit bei gewissen officiellen Körperschaften des betreffenden
                    Landes, so bei der \textcolor{brown}{Akademie der Wissenschaften}{}\ledrightnote{\textcolor{brown}{Österreichische Akademie der Wissenschaften}}
                    und dem \textcolor{pink}{Unterrichtsministerium}{}\ledrightnote{\textcolor{pink}{Ministerium für Unterricht}} angefragt zu
                    werden pflegt, wie sich diese zu dem {\pb}Vorschlag verhalten, und, we{\geminationn} mich nicht alles
                    trügt, – Eindrücke und Erfahrungen, – erfreu ich mich an diesen Stellen (was Sie
                    vielleicht nicht einmal Wunder nehmen wird) keiner sonderlichen Sympathien; ja
                    ich ka{\geminationn} mir vorstellen, daß eine eventuell
                    beabsichtigte Zuerke{\geminationn}ung jenes Preises an mich
                    gerade in meinem Vaterland bei manchen maßgebenden Factoren auf einen
                    Widerspruch stieße, der nur durch das Wort eines Manns von höchster Bedeutung
                    und weitestem Ruhm paralysirt werden könnte. In diesem Zusa{\geminationm}enhang \substVorne{}\textsuperscript{\textcolor{gray}{d}}\substDazwischen{}a\substHinten{}ber könnt’ ich kaum an jemand andern denken als an Sie, der mir schon
                    ein Schätzer, ein {\pb}Freund gewesen ist zu
                    einer Zeit, da ich von andern Kritikern und Kennern nicht oder kaum bemerkt
                    wurde, und der seither, durch mehr als zwanzig Jahre meinen Weg nicht nur mit
                    künstlerischer, sondern \introOben{}auch\introOben{} was ich sehr wohlthuend
                    empfand, menschlicher Antheilnahme begleitet hat; und habe mich gefragt, ob Sie
                    sich wohl bereit fänden, ein solches Wort für mich, – das sich im Ernstfall
                    wahrscheinlich als sehr notwendig erweisen würde – an der entscheidenden Stelle
                    auszusprechen? Ob Sie \introOben{}\strikeout{nun}\introOben{} diese Anregung nun als eine nicht allzu bescheidne Bitte oder als den
                    erlaubten Versuch auffassen wollen, mich Ihnen in einem geeigneten Moment
                    einfach in Erinnerung zu bringen, {\pb}– was ich
                    hier gesagt habe, gilt natürlich nur für den Fall daß Sie (was mir allerdings
                    selbstverständlich scheint) von dem schwedischen Comité um Ihre Meinung gefragt
                    werden sollten; – andern Falls betrachten Sie bitte diesen Brief hier nur
                    insoweit als vorhanden, als er Ihnen wieder einmal den Ausdruck meines dankbaren
                    Vertrauens, sowie einer Freundschaft übermittelt, die auch in \strikeout{in} Jahren, da man einander nicht begegnete,
                    gleich herzlich weiterlebte.\pend
           \pstart
           Vielleicht mag ein solches Schreiben in so aufgewühlter Zeit, wo aus den
                    bedenklichen Dünsten der Politik allmälig die erhabnen Wolkenbildun{\pb}gen der Weltgeschichte aufzusteigen begi{\geminationn}en, kaum der Beachtung werth erscheinen; aber,
                    wir empfinden und erfahren es doch alle immer wieder, – auch in so bewegten
                    Epochen fordert die Einzelexistenz von Tag zu Tag ihr Recht, – und wie diesmal
                    üblergesinnte meinen könnten, – auch mehr als das. Sie aber, mein verehrter
                    Freund, nehmen mir diesen Ruf, der hoffentlich nicht allzustörend in Ihre
                    Sommerruhe dringt, keinesfalls übel, wo immer er Sie erreichen möge.\pend
           \pstart
           Wir befinden uns unter ziemlich sonderbaren Verhältnissen hier in \textcolor{pink}{Celerina}{}\ledrightnote{\textcolor{pink}{Celerina}}; auch die \textcolor{pink}{Schweiz}{}\ledrightnote{\textcolor{pink}{Schweiz}} hat völligen Kriegszustand; wie die Dinge eben
                    stehen, wäre es mir, mit \textcolor{blue}{Frau}{}\ledrightnote{→\textcolor{blue}{Olga Schnitzler}} und \textcolor{blue}{Kindern}{}\ledrightnote{→\textcolor{blue}{Olga Schnitzler}{\newline}→\textcolor{blue}{Lili Schnitzler}}, nur unter den ärgsten {\pb}Unbequemlichkeiten möglich, über die Grenze in die Heimat zu kommen, – und so
                    bleiben wir denn vorläufig \introOben{}hier\introOben{}, in nicht geringer
                    Unruhe; aber freuen uns doch der Wald-, Wiesen- und Himmelsruhe in diesem
                    schönen Thal – das von dem Lärm der Welt nichts zu wissen scheint, trotz dreier
                        gri{\geminationm}igen Soldaten, die im Wartesaal des
                    Bahnhofs ihr Vaterland schützen und Karten spielen. –\pend
           \pstart
           Wollen Sie mir eine Zeile schreiben, so bitte ich doch jedenfalls an meine \textcolor{pink}{Wien}{}\ledrightnote{\textcolor{pink}{Wien}}er Wohnung (\textcolor{pink}{XVIII. Sternwartestraße 71}{}\ledrightnote{\textcolor{pink}{Sternwartestraße}}) zu adressiren.\pend
           \pstart
           Mit den herzlichsten Grüßen, auch von meiner \textcolor{blue}{Frau}{}\ledrightnote{→\textcolor{blue}{Olga Schnitzler}}\pend
           \pstart
           Ihr stets ergebner{\\[\baselineskip]}\spacefill\mbox{Arthur Schnitzler}\pend
           \leftskip=0em{}\endnumbering\briefempfaengerindex{Brandes, Georg@\textsc{Brandes, Georg}!zzzSchnitzler, Arthur@\emph{von Arthur Schnitzler}!1914-08-031@{3. 8. 1914}|)be}\mylabel{h}  \normalsize

\doendnotes{C}
\bigskip
\vfill

\clearpage

\footnotesize

\lohead{\textsc{register}}

% Definiere theindex-Environment komplett neu ohne reledmac
\makeatletter
\renewenvironment{theindex}{%
  \section*{\indexname}%
  \setlength{\parindent}{0pt}%
  \setlength{\parskip}{0pt plus 0.3pt}%
  \let\item\@idxitem
}{%
  \clearpage
}
\makeatother

\IfFileExists{\jobname-pw.ind}{\input{\jobname-pw.ind}}{}

\end{document}

      