%% latex-korrekturansicht-vorspann.tex
%% Vorspann für die Korrekturansicht.
%% Lädt die gemeinsame Datei latex-vorspann.tex mit gesetztem Schalter.

\newif\ifkorrekturansicht
\korrekturansichttrue

\input{../tex-inputs/latex-vorspann}


\renewcommand{\erwaehntePersonen}{Personen: Richard Beer-Hofmann}
\renewcommand{\erwaehnteOrte}{Orte: Bad Ischl, Berghof, Unterach am Attersee, Weißenbach am Attersee, Wien}
\renewcommand{\erwaehnteWerke}{}
\section[Felix Salten an Arthur Schnitzler, {[}25.? 8. 1892{]}]{Felix Salten an Arthur Schnitzler, {[}25.? 8. 1892{]}}
\nopagebreak\mylabel{v}
\rehead{ }\normalsize\beginnumbering\briefempfaengerindex{Schnitzler, Arthur@\textsc{Schnitzler, Arthur}!zzzSalten, Felix@\emph{von Felix Salten}!1892-08-251@{{[}25.? 8. 1892{]}}|(be}
\toendnotes[C]{\smallbreak\pagebreak[2]}\Standort{CUL, Schnitzler, B 89, A 1.}
\physDesc{Brief, 1 Blatt, 2 Seiten, 610 Zeichen
\newline{}Handschrift: schwarze Tinte, lateinische Kurrent
\newline{}Schnitzler: mit Bleistift datiert: »\substVorne{}\textsuperscript{Anf}\substDazwischen{}En{[}de{]}\substHinten{} Au{[}g{]} 92« 
\newline{}Ordnung: mit Bleistift von unbekannter Hand nummeriert: »18« }\toendnotes[C]{\smallbreak}
\pstart
           \noindent{}{\pb}Verehrtester\textcolor{gray}{!}{ }Besten Dank für \label{K_L03114-1v}\edtext{Ihren Brief}{\lemma{\textnormal{\emph{Ihren Brief}}}\Cendnote{\textnormal{Die
                  grobe Einordnung des undatierten Korrespondenzstücks gelingt durch die Datierung
                     \textcolor{blue}{Schnitzler}s auf »En{[}de{]} Au{[}g{]} 92«. Innerhalb der Korrespondenzstücke dürfte es sich um \textcolor{blue}{Schnitzler}s Reaktion auf das Schreiben vom 23. 8. 1892 handeln, da in
                  diesem noch nicht von einem persönlichen Treffen die Rede war. \textcolor{blue}{Schnitzler} war ab 27. 8. 1892 in \textcolor{pink}{Ischl} – erst damit wurde ein Treffen möglich. Für den 31. 8. 1892 ist eine
                  Zusammenkunft belegt. Dieser Tag bildet also den letzten möglichen Zeitpunkt.
                  Weniger gewiss, aber doch wahrscheinlich ist die Annahme, dass \textcolor{blue}{Schnitzler} vor seiner Ankunft in \textcolor{pink}{Ischl} das Treffen eingefordert hatte und diese Kommunikation
                  noch nach \textcolor{pink}{Wien} lief. Damit wäre der 25. 8. 1892
                  das wahrscheinliche Datum für dieses Korrespondenzstück.}}}\label{K_L03114-1h}. Ob gerade eine
               persönliche breite Aussprache für mich beruhigend wäre, weiss ich nicht, – doch
               darauf ko{\geminationm}t es gewiss nicht an. Ich freue mich
               jedenfalls aufrichtig Sie zu sehen, u bitte Sie mir den Tag zu bestimmen, wann ich
               nach \textcolor{pink}{Ischl}{}\ledrightnote{\textcolor{pink}{Bad Ischl}} kommen kann, oder wann Sie nach \textcolor{pink}{Weissenbach}{}\ledrightnote{\textcolor{pink}{Weißenbach am Attersee}} kommen wollen. Auch {\pb}am \label{K_L03114-2v}\edtext{\textcolor{pink}{Berghof}{}\ledrightnote{\textcolor{pink}{Berghof}}}{\lemma{\textnormal{\emph{Berghof}}}\Cendnote{\textnormal{In dieser Zeit ist kein Besuch \textcolor{blue}{Schnitzler}s am \textcolor{pink}{Berghof} nachweisbar.}}}\label{K_L03114-2h} würde man Sie gerne sehen, und bin ich
               beauftragt, Sie für einen Tag herüberzubitten. Auch \textcolor{blue}{Beer-Hofmann}{}\ledrightnote{\textcolor{blue}{Richard Beer-Hofmann}} soll, wenn er will{[},{]} mitkommen. Dass es mir
               hauptsächlich jetzt um die Aussprache mit Ihnen zu thun ist\textcolor{gray}{,} brauche ich nicht erst zu
               sagen.\pend
           
\pstart
           Also auf Wiedersehen {\\[\baselineskip]}Ihr \spacefill\mbox{Salten}\pend
           \leftskip=0em{}\endnumbering\briefempfaengerindex{Schnitzler, Arthur@\textsc{Schnitzler, Arthur}!zzzSalten, Felix@\emph{von Felix Salten}!1892-08-251@{{[}25.? 8. 1892{]}}|)be}\mylabel{h}  \normalsize

\doendnotes{C}
\bigskip
\vfill

\clearpage

\footnotesize

\lohead{\textsc{register}}

% Definiere theindex-Environment komplett neu ohne reledmac
\makeatletter
\renewenvironment{theindex}{%
  \section*{\indexname}%
  \setlength{\parindent}{0pt}%
  \setlength{\parskip}{0pt plus 0.3pt}%
  \let\item\@idxitem
}{%
  \clearpage
}
\makeatother

\IfFileExists{\jobname-pw.ind}{\input{\jobname-pw.ind}}{}

\end{document}

      