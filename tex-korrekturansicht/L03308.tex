%% latex-korrekturansicht-vorspann.tex
%% Vorspann für die Korrekturansicht.
%% Lädt die gemeinsame Datei latex-vorspann.tex mit gesetztem Schalter.

\newif\ifkorrekturansicht
\korrekturansichttrue

\input{../tex-inputs/latex-vorspann}


\renewcommand{\erwaehnteOrte}{Orte: Bad Ischl, Karlsbad, Pressbaum, Schruns, Vorarlberg, Wien}
\renewcommand{\erwaehnteWerke}{}
\section[ Felix Salten an Arthur Schnitzler, 7. 8. 1900]{Felix Salten an Arthur Schnitzler, 7. 8. 1900}
\nopagebreak\mylabel{v}
\rehead{ }\normalsize\beginnumbering\briefempfaengerindex{Schnitzler, Arthur@\textsc{Schnitzler, Arthur}!zzzSalten, Felix@\emph{von Felix Salten}!1900-08-072@{7. 8. 1900}|(be}
\toendnotes[C]{\smallbreak\pagebreak[2]}\Standort{CUL, Schnitzler, B 89, A 2.}
\physDesc{Brief, 1 Blatt, 2 Seiten, 674 Zeichen
\newline{}Handschrift: schwarze Tinte, lateinische Kurrent
\newline{}Ordnung: mit Bleistift von unbekannter Hand nummeriert: »132« }\toendnotes[C]{\smallbreak}
\pstart
           \raggedleft{}{\pb}\textcolor{pink}{Wien}{}\ledrightnote{\textcolor{pink}{Wien}}, 7. Aug. 00.\pend
           
\pstart
           Lieber, haben Sie meinen \label{K_L03308-1v}\edtext{Brief aus \textcolor{pink}{Pressbaum}{}\ledrightnote{\textcolor{pink}{Pressbaum}}}{\lemma{\textnormal{\emph{Brief aus Pressbaum}}}\Cendnote{\textnormal{Felix Salten an Arthur Schnitzler, 5. 8. 1900}}}\label{K_L03308-1h} nicht bekommen? Ich muß nun heute{ }Abend nach \textcolor{pink}{Karlsbad}{}\ledrightnote{\textcolor{pink}{Karlsbad}} fahren, wodurch
               meine \label{K_L03308-2v}\edtext{Ankunft in \textcolor{pink}{Ischl}{}\ledrightnote{\textcolor{pink}{Bad Ischl}}}{\lemma{\textnormal{\emph{Ankunft in Ischl}}}\Cendnote{\textnormal{\textcolor{blue}{Salten} kam am 14. 8. 1900 in \textcolor{pink}{Ischl} an.}}}\label{K_L03308-2h} sich bis Sonntag
               verzögert. Nach \textcolor{pink}{Vorarlberg}{}\ledrightnote{\textcolor{pink}{Vorarlberg}} komme ich ganz
               gewiss. Bitte, theilen Sie mir nur immer mit, wo Sie sind. Wenn man so gegen 20. od. 22. in \label{K_L03308-3v}\edtext{\textcolor{pink}{Schruns}{}\ledrightnote{\textcolor{pink}{Schruns}}}{\lemma{\textnormal{\emph{Schruns}}}\Cendnote{\textnormal{siehe Felix Salten an Arthur Schnitzler, 5. 8. 1900}}}\label{K_L03308-3h} wäre, das könnte gerade für mich recht sein.\pend
           
\pstart
           Vielleicht ist es möglich, darauf ein wenig Rücksicht zu nehmen.\pend
           
\pstart
           {\pb}Leben Sie recht wol, und
               laßen Sie mir genaue Nachricht zukommen. Am besten \uline{Postlagernd \textcolor{pink}{Ischl}{}\ledrightnote{\textcolor{pink}{Bad Ischl}}}.\pend
           
\pstart
           Solle ich Sie Sonntag, wie ich aus dem heutigen Brief
               vermuthe, \label{K_L03308-4v}\edtext{nicht mehr antreffen}{\lemma{\textnormal{\emph{nicht mehr antreffen}}}\Cendnote{\textnormal{\textcolor{blue}{Schnitzler} reiste am 10. 8. 1900 aus \textcolor{pink}{Ischl} ab.}}}\label{K_L03308-4h}, so hole ich mir die
               Reisedispositionen von der Post.\pend
           
\pstart
           Auf Wiedersehen da oder dort. {\\[\baselineskip]}Herzlichst {\\[\baselineskip]}\spacefill\mbox{Salten.}\pend
           \leftskip=0em{}\endnumbering\briefempfaengerindex{Schnitzler, Arthur@\textsc{Schnitzler, Arthur}!zzzSalten, Felix@\emph{von Felix Salten}!1900-08-072@{7. 8. 1900}|)be}\mylabel{h}  \normalsize

\doendnotes{C}
\bigskip
\vfill

\clearpage

\footnotesize

\lohead{\textsc{register}}

% Definiere theindex-Environment komplett neu ohne reledmac
\makeatletter
\renewenvironment{theindex}{%
  \section*{\indexname}%
  \setlength{\parindent}{0pt}%
  \setlength{\parskip}{0pt plus 0.3pt}%
  \let\item\@idxitem
}{%
  \clearpage
}
\makeatother

\IfFileExists{\jobname-pw.ind}{\input{\jobname-pw.ind}}{}

\end{document}

      