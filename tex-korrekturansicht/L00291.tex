%% latex-korrekturansicht-vorspann.tex
%% Vorspann für die Korrekturansicht.
%% Lädt die gemeinsame Datei latex-vorspann.tex mit gesetztem Schalter.

\newif\ifkorrekturansicht
\korrekturansichttrue

\input{../tex-inputs/latex-vorspann}


               \section[Arthur Schnitzler an Hugo von Hofmannsthal, {[}15. 1. 1894{]}]{ Arthur Schnitzler an Hugo von Hofmannsthal, {[}15. 1. 1894{]}}\nopagebreak\mylabel{v}\rehead{ }\normalsize\beginnumbering\briefempfaengerindex{Hofmannsthal, Hugo von@\textsc{Hofmannsthal, Hugo von}!zzzSchnitzler, Arthur@\emph{von Arthur Schnitzler}!1894-01-151@{{[}15. 1. 1894{]}}|(be} \toendnotes[C]{\smallbreak\pagebreak[2]} \Standort{FDH, Hs-30885,40.}
\physDesc{Brief, 1 Blatt (Briefpapier mit Trauerrand), 3 Seiten
\newline{}Handschrift: schwarze Tinte, deutsche Kurrent\newline{}Ordnung: von unbekannter Hand datiert: »93« }\buchAbdrucke{\weitereDrucke{Hugo von Hofmannsthal, Arthur Schnitzler: \emph{Briefwechsel}. Hg. Therese Nickl und Heinrich Schnitzler. Frankfurt am Main: \emph{S. Fischer} 1964, S. 48–49.} }\toendnotes[C]{\smallbreak}\pstart{}{\pb}Lieber Hugo,\pend\pstart
           \label{K_L00291_1v}\edtext{Sonntag}{\lemma{\textnormal{\emph{Sonntag}}}\Cendnote{\textnormal{\textcolor{blue}{Schnitzler} und \textcolor{blue}{Hofmannsthal} besuchten die angesprochene Aufführung am
                     21. 1. 1894, die im Zuge eines Gastspiels am \textcolor{pink}{Carltheater} stattfand (A. S.: \emph{Tagebuch}, 21. 1. 1893, \textcolor{blue}{Hugo von Hofmannsthal}: \emph{Aufzeichnungen}. Hg. Rudolf Hirsch † und Ellen Ritter † in
                     Zusammenarbeit mit Konrad Heumann und Peter Michael Braunwarth. Frankfurt am
                     Main: \emph{\textcolor{brown}{S. Fischer}}{ }2013, S. 265 (\emph{Sämtliche Werke},
                     XXXIX)).}}}\label{K_L00291_1h} gibt \textcolor{blue}{\textsc{Mounet-Sully}}{}\ledrightnote{\textcolor{blue}{Jean Mounet-Sully}} den \textcolor{green}{\textsc{Hamlet}}{}\ledrightnote{\textcolor{green}{Hamlet}}; da möcht ich gern hineingehn. Sie auch? Soll ich für uns beide Sitze nehmen?
               Was für eine Su{\geminationm}e {\pb}wollen Sie
               eventuell dieſem Zwecke widmen?\pend
           \pstart
           – \label{K_L00291_2v}\edtext{Heut}{\lemma{\textnormal{\emph{Heut}}}\Cendnote{\textnormal{Am 15. 1. 1894 war \textcolor{blue}{Schnitzler} in
                  der Premiere von \emph{\textcolor{green}{Der ungläubige Thomas}} von \textcolor{blue}{Karl Laufs} und \textcolor{blue}{Wilhelm Jacoby} am \textcolor{pink}{Raimundtheater}. (\emph{Cambridge University Library}, A 179)}}}\label{K_L00291_2h} geh
               ich zum \textcolor{green}{ungläubigen \textsc{Thomas}}{}\ledrightnote{\textcolor{green}{Der ungläubige Thomas}}, \label{K_L00291_3v}\edtext{morgen}{\lemma{\textnormal{\emph{morgen}}}\Cendnote{\textnormal{\textcolor{blue}{Victorien Sardou}s \emph{\textcolor{green}{Madame Sans-Gêne}} wurde am 16. 1. 1894 im \textcolor{pink}{Deutschen
                     Volkstheater} gegeben, \textcolor{blue}{Schnitzler} war
                  anwesend. (\emph{Cambridge University Library}, A 179)}}}\label{K_L00291_3h} zu
                  \textcolor{green}{\textsc{Madame Sans-gêne}}{}\ledrightnote{\textcolor{green}{Madame Sans-Gêne}}. Bin äußerſt kunſtſinnig. –\pend
           \pstart
           – Beifolgende ergreifende \label{K_L00291_4v}\edtext{Erzählung}{\lemma{\textnormal{\emph{Erzählung}}}\Cendnote{\textnormal{Nicht
                  identifiziert.}}}\label{K_L00291_4h} iſt mit Andacht zu leſen.\pend
           \pstart {\pb}Herzlich Ihr Arthur, der eine baldige Antwort erwartet. – \pend{}\pstart
           \noindent{}\uline{Montag.}\pend
           \endnumbering\briefempfaengerindex{Hofmannsthal, Hugo von@\textsc{Hofmannsthal, Hugo von}!zzzSchnitzler, Arthur@\emph{von Arthur Schnitzler}!1894-01-151@{{[}15. 1. 1894{]}}|)be}\mylabel{h}  \normalsize

\doendnotes{C}
\bigskip
\vfill

\clearpage

\footnotesize

\lohead{\textsc{register}}

% Definiere theindex-Environment komplett neu ohne reledmac
\makeatletter
\renewenvironment{theindex}{%
  \section*{\indexname}%
  \setlength{\parindent}{0pt}%
  \setlength{\parskip}{0pt plus 0.3pt}%
  \let\item\@idxitem
}{%
  \clearpage
}
\makeatother

\IfFileExists{\jobname-pw.ind}{\input{\jobname-pw.ind}}{}

\end{document}

      