%% latex-korrekturansicht-vorspann.tex
%% Vorspann für die Korrekturansicht.
%% Lädt die gemeinsame Datei latex-vorspann.tex mit gesetztem Schalter.

\newif\ifkorrekturansicht
\korrekturansichttrue

\input{../tex-inputs/latex-vorspann}


               \section[Arthur Schnitzler an Georg Brandes, 8. 6. 1899]{ Arthur Schnitzler an Georg Brandes, 8. 6. 1899}\nopagebreak\mylabel{v}\rehead{ }\normalsize\beginnumbering\briefempfaengerindex{Brandes, Georg@\textsc{Brandes, Georg}!zzzSchnitzler, Arthur@\emph{von Arthur Schnitzler}!1899-06-081@{8. 6. 1899}|(be} \toendnotes[C]{\smallbreak\pagebreak[2]} \Standort{Kopenhagen, Det Kongelige Bibliotek, Georg Brandes Arkiv, box 125.}
\physDesc{Briefkarte
\newline{}Handschrift: schwarze Tinte, deutsche Kurrent\newline{}Ordnung: mit Bleistift von unbekannter Hand nummeriert: »17.« }\buchAbdrucke{\weitereDrucke{Georg Brandes, Arthur Schnitzler: \emph{Ein Briefwechsel}. Hg. Kurt Bergel. Bern: \emph{Francke} 1956, S. 77–78.} }\toendnotes[C]{\smallbreak}\pstart
           \raggedleft{}{\pb}8. 6. 99.\pend
           \pstart
           Verehrteſter Herr Brandes, eine Bitte diesmal, deren Erfüllg
                    Ihnen hoffentlich nicht allzu viel Mühe macht. Ein Herr \label{K_L00923_1v}\edtext{\textcolor{blue}{\textsc{Soutif}}{}\ledrightnote{\textcolor{blue}{Émile Soutif}}}{\lemma{\textnormal{\emph{Soutif}}}\Cendnote{\textnormal{Die Übersetzung ist nicht überliefert.
                        Über \textcolor{blue}{Émile Soutif} ist nur der Eintrag im
                            \emph{\textcolor{green}{Adreßbuch für Dresden und Vororte}} (1899, Theil I, S. 580.) bekannt, in dem er als
                            »Lehrer d. franz. Sprache u. Literat.« ausgewiesen
                        ist.}}}\label{K_L00923_1h} hat eine Überſetzg »des \textcolor{green}{\uline{grünen Kakadu}}{}\ledrightnote{\textcolor{green}{Der grüne Kakadu. Groteske in einem Akt}}« ins franzöſiſche an \textcolor{blue}{\textsc{\uline{Antoine}}}{}\ledrightnote{\textcolor{blue}{André Antoine}} in \textcolor{pink}{Paris}{}\ledrightnote{\textcolor{pink}{Paris}} geſchickt. Ich weiſs
                    nun kaum, ob \textcolor{blue}{\textsc{Antoine}}{}\ledrightnote{\textcolor{blue}{André Antoine}} meinen Namen kennt. Wenn \uline{Sie} aber
                    ihm ein {\pb}Wort ſchreiben, er ſolle das
                    Ding aufmerkſam durchleſen, ſo thut er’s gewiß. Alſo daſs Sie ihm ſagen: »Leſen
                    Sie den ›\textcolor{green}{\textsc{Peroquet vert}}{}\ledrightnote{\textcolor{green}{Der grüne Kakadu. Groteske in einem Akt}}‹«– bitte ich Sie; – nichts anderes, keine »Empfehlung« – oder
                    dergleichen.\pend
           \pstart
           Es iſt doch nicht zu unbeſcheiden, hoff ich?\pend
           \pstart
           Sind Sie nun endlich außer Bett? Und wohl – und heiter? Ihr treuer \spacefill\mbox{Arthur
                        Schnitzler}\pend
           \endnumbering\briefempfaengerindex{Brandes, Georg@\textsc{Brandes, Georg}!zzzSchnitzler, Arthur@\emph{von Arthur Schnitzler}!1899-06-081@{8. 6. 1899}|)be}\mylabel{h}  \normalsize

\doendnotes{C}
\bigskip
\vfill

\clearpage

\footnotesize

\lohead{\textsc{register}}

% Definiere theindex-Environment komplett neu ohne reledmac
\makeatletter
\renewenvironment{theindex}{%
  \section*{\indexname}%
  \setlength{\parindent}{0pt}%
  \setlength{\parskip}{0pt plus 0.3pt}%
  \let\item\@idxitem
}{%
  \clearpage
}
\makeatother

\IfFileExists{\jobname-pw.ind}{\input{\jobname-pw.ind}}{}

\end{document}

      