%% latex-korrekturansicht-vorspann.tex
%% Vorspann für die Korrekturansicht.
%% Lädt die gemeinsame Datei latex-vorspann.tex mit gesetztem Schalter.

\newif\ifkorrekturansicht
\korrekturansichttrue

\input{../tex-inputs/latex-vorspann}


\renewcommand{\erwaehntePersonen}{Personen: Hermann Bahr, Maximilian Harden, Siegfried Jacobsohn, Emil Rosenow, Ottilie Salten, George Bernard Shaw, Elisabeth Steinrück, Siegfried Trebitsch}
\renewcommand{\erwaehnteInstitutionen}{Institutionen: B.Z. am Mittag, Berliner Morgenpost, Die Zukunft, Neue Freie Presse, Neues Theater}
\renewcommand{\erwaehnteOrte}{Orte: Bayreuth, Berlin, Deutschland, Dänemark, Genfer See, Genthin, Kiel, Kochstraße, Lugano, Magdeburg, Marienlyst, Niederlande, Ostsee, Rheinland, Rothenburg ob der Tauber, Schweiz, Südtirol, Thüringen, Tirol, Wien}
\renewcommand{\erwaehnteWerke}{Werke: B.Z. am Mittag, Bund der Bühnendichter. II, Bühnenvertrieb, Cäsar und Cleopatra. Eine historische Komödie, Die Schaubühne, Kater Lampe, Russisches Theater. II, Theater, »Kater Lampe«}
\section[ Felix Salten an Arthur Schnitzler, 28. 3. 1906]{Felix Salten an Arthur Schnitzler, 28. 3. 1906}
\nopagebreak\mylabel{v}
\rehead{ }\normalsize\beginnumbering\briefempfaengerindex{Schnitzler, Arthur@\textsc{Schnitzler, Arthur}!zzzSalten, Felix@\emph{von Felix Salten}!1906-03-281@{28. 3. 1906}|(be}
\toendnotes[C]{\smallbreak\pagebreak[2]}\Standort{CUL, Schnitzler, B 89, B 1.}
\physDesc{Brief, 1 Blatt, 2 Seiten, 3212 Zeichen
\newline{}Handschrift: schwarze Tinte, lateinische Kurrent
\newline{}Ordnung: mit Bleistift von unbekannter Hand nummeriert: »207« }
\buchAbdrucke{\weitereDrucke{Hermann Bahr, Arthur Schnitzler: \emph{Briefwechsel, Aufzeichnungen, Dokumente (1891–1931)}. Hg. Kurt Ifkovits und Martin Anton Müller. Göttingen: \emph{Wallstein} 2018, S. 376–377.} }\toendnotes[C]{\smallbreak}
\pstart
           \noindent{}{\pb}\textcolor{brown}{\textcolor{gray}{\textbf{\emph{B. Z. am Mittag}}}}{}\ledrightnote{\textcolor{brown}{B.Z. am Mittag}}\hfill \textcolor{gray}{\textbf{\emph{\textcolor{pink}{BERLIN SW}{}\ledrightnote{\textcolor{pink}{Berlin}}},}}{ }28. III. 06\pend
           
\pstart
           \textcolor{gray}{\textbf{\emph{Chefredaktion}}}\hfill \textcolor{pink}{\textcolor{gray}{\textbf{\emph{Kochstr. 23–25}}}}{}\ledrightnote{\textcolor{pink}{Kochstraße}}\pend
           
\pstart
           Lieber, dass wir eine \label{K_L03416-1v}\edtext{Radtour machen}{\lemma{\textnormal{\emph{Radtour machen}}}\Cendnote{\textnormal{Diese fand nicht statt,
                     siehe Felix Salten an Arthur Schnitzler, 1. 5. 1906.}}}\label{K_L03416-1h} könnten,
               ist mir heute wie ein absolutes Muß! Es wäre so schön 6–8 Tage irgendwo durch die
               Welt zu gleiten, wo sie schön ist, und wo man wieder einmal so viel Behagen empfinden
               könnte, wie »einst im Mai«{[}.{]} Denken Sie etwas Gutes aus, und
               ziehen Sie dabei in Betracht, ob wir nicht eine Gegend wählen wollen, die wir noch
               nicht kennen. Deutsches Gebirge, \textcolor{pink}{Thüringen}{}\ledrightnote{\textcolor{pink}{Thüringen}}, \textcolor{pink}{Rhein}{}\ledrightnote{\textcolor{pink}{Rheinland}}, u. s. w. Ich bin aber auch mit \textcolor{pink}{Tirol}{}\ledrightnote{\textcolor{pink}{Tirol}{\newline}\textcolor{pink}{Südtirol}} oder \textcolor{pink}{Schweiz}{}\ledrightnote{\textcolor{pink}{Schweiz}} (\textcolor{pink}{Lugano}{}\ledrightnote{\textcolor{pink}{Lugano}} oder \textcolor{pink}{Genfer See}{}\ledrightnote{\textcolor{pink}{Genfer See}}) einverstanden. Ihr Brief kam heute aber auch \label{K_L03416-2v}\edtext{\begin{otherlanguage}{italian}a tempo\end{otherlanguage}}{\lemma{\textnormal{\emph{a tempo}}}\Cendnote{\textnormal{italienisch: zur rechten Zeit}}}\label{K_L03416-2h}: es
               ist \substVorne{}\textsuperscript{\textcolor{gray}{jetzt}}\substDazwischen{}nach\substHinten{} langem Winter wieder die erste Frühlingswärme, die erste Sonne wieder da,
               und alle Reisepläne, alles Reiseverlangen – »Wanderlust« – regt sich. An solchen
               Tagen hat auch \textcolor{pink}{Berlin}{}\ledrightnote{\textcolor{pink}{Berlin}} seine Schönheit. An solchen
               Tagen würde übrigens auch \textcolor{pink}{Magdeburg}{}\ledrightnote{\textcolor{pink}{Magdeburg}} oder \textcolor{pink}{Genthinen}{}\ledrightnote{\textcolor{pink}{Genthin}} nicht ohne Reiz sein. Ich überlege mir
               heute zum 20\textsuperscript{ten} Mal, wie man es macht, sich ein ganz ein
               kleines Automobil zu kaufen. Geht aber leider im Moment nicht. Wenn ich die große
               Zeitung gegründet habe, \textcolor{brown}{Neue freie Presse}{}\ledrightnote{\textcolor{brown}{Neue Freie Presse}} in \textcolor{pink}{Berlin}{}\ledrightnote{\textcolor{pink}{Berlin}}, eine Wochenschrift im \textcolor{brown}{Zukunft}{}\ledrightnote{\textcolor{brown}{Die Zukunft}}-Stil und dann vier Blätter regiere, statt \textcolor{brown}{zwei}{}\ledrightnote{{$\rightarrow$}\textcolor{brown}{Berliner Morgenpost}{\newline}{$\rightarrow$}\textcolor{brown}{B.Z. am Mittag}} (was ich armselig
                  finde){[},{]} dann werde ich gewiss auch das langerflehte Auto
               haben. Inzwischen freu ich mich, wenn nur eine Radtour zustande kommt, und die
               übrigen Dinge, die ich für den Sommer vorhabe (\textcolor{pink}{Holland}{}\ledrightnote{\textcolor{pink}{Niederlande}}, zu Wasser nach \textcolor{pink}{Kiel}{}\ledrightnote{\textcolor{pink}{Kiel}}){[}.{]} Die Radtour könnte auch durch einige \textcolor{pink}{deutsch}{}\ledrightnote{{$\rightarrow$}\textcolor{pink}{Deutschland}}e Städte gemacht werden,
               – \textcolor{pink}{Rothenburg ob. d. Tauber}{}\ledrightnote{\textcolor{pink}{Rothenburg ob der Tauber}} – \textcolor{pink}{Bayreuth}{}\ledrightnote{\textcolor{pink}{Bayreuth}}, wozu man freilich jetzt schon die Sitze bestellen
               müsste. Das \label{K_L03416-3v}\edtext{\textcolor{pink}{dänische Seebad}{}\ledrightnote{\textcolor{pink}{Marienlyst}}}{\lemma{\textnormal{\emph{dänische Seebad}}}\Cendnote{\textnormal{\textcolor{blue}{Schnitzler} war zwischen 28. 6. 1906 und 9. 8. 1906 in \textcolor{pink}{Marienlyst}. \textcolor{blue}{Felix} und \textcolor{blue}{Ottilie Salten} besuchten
                  ihn dort am 2. 8. 1906.}}}\label{K_L03416-3h}, das Sie vorhaben, verdrießt mich – wenn ich
               aufrichtig sein darf – immer. Weil ich {\dotstwo} aus
               wirthschaftlichen Gründen {\dotstwo} nicht hinkann, wenn ich schon
               einmal an der \textcolor{pink}{Ostsee}{}\ledrightnote{\textcolor{pink}{Ostsee}} sitze, und weil ich mir
               denke, wenn uns ein mehrwöchiges Beisammensein schon beschieden sein könnte, dann
               ließe sich vielleicht doch auf \textcolor{pink}{Dänemark}{}\ledrightnote{\textcolor{pink}{Dänemark}}
               verzichten. Der Unterschied ist nicht so groß, und Wälder gibt's auch am diesseitigen
               Strand der \textcolor{pink}{Ostsee}{}\ledrightnote{\textcolor{pink}{Ostsee}}.\pend
           
\pstart
           Augenblicklich ist \textcolor{pink}{Wien}{}\ledrightnote{\textcolor{pink}{Wien}} durch M\textsuperscript{r}{ }\label{K_L03416-4v}\edtext{\textcolor{blue}{Triebeitsch}{}\ledrightnote{\textcolor{blue}{Siegfried Trebitsch}}}{\lemma{\textnormal{\emph{Triebeitsch}}}\Cendnote{\textnormal{Hier findet
                  das Naserümpfen über \textcolor{blue}{Trebitsch} eine Form, in der 
                  die Herabsetzung durch die Imitation einer englischen Aussprache seines Namens erfolgt.}}}\label{K_L03416-4h} vertreten, der in seinem \label{K_L03416-5v}\edtext{\textcolor{green}{Premiere}{}\ledrightnote{{$\rightarrow$}\textcolor{green}{Cäsar und Cleopatra. Eine historische Komödie}}nfieber}{\lemma{\textnormal{\emph{Premierenfieber}}}\Cendnote{\textnormal{Am 31. 3. 1906 fand am \emph{\textcolor{brown}{Neuen Theater}}
                  die deutschsprachige Uraufführung von \emph{\textcolor{green}{Caesar und
                     Cleopatra}} von \textcolor{blue}{George Bernard Shaw} in
                  der Übersetzung von \textcolor{blue}{Siegfried Trebitsch}
                  statt.}}}\label{K_L03416-5h} wegen \textcolor{blue}{Shaw}{}\ledrightnote{\textcolor{blue}{George Bernard Shaw}} das Maß des
               Lächerlichen erreicht. Seine erste Frage, als er \textcolor{pink}{hier}{}\ledrightnote{{$\rightarrow$}\textcolor{pink}{Berlin}} eintraf, war (natürlich per Telefon) was ich von seinem
                  \label{K_L03416-6v}\edtext{\textcolor{green}{Vorschlag}{}\ledrightnote{{$\rightarrow$}\textcolor{green}{Bühnenvertrieb}}}{\lemma{\textnormal{\emph{Vorschlag}}}\Cendnote{\textnormal{\textcolor{blue}{Siegfried Trebitsch}: \emph{\textcolor{green}{Bühnenvertrieb}}. In: \emph{\textcolor{green}{Die
                        Schaubühne}}, Jg. 2, Nr. 12, 22. 3. 1906,
                     S. 348–350. Darin forderte \textcolor{blue}{Trebitsch} die Einrichtung einer Bühnengenossenschaft zur Vertretung von
                  Autorinnen- und Autorenrechten. Das motivierte den Herausgeber der \textcolor{green}{Zeitschrift}, \textcolor{blue}{Siegfried Jacobsohn}, zu einer mehrteiligen Debatte, die
                  sich über Monate streckte. In der \textcolor{green}{zweiten Fortsetzung} findet sich ein Beitrag \textcolor{blue}{Schnitzler}s. Siehe A. S.: \emph{»Das Zeitlose ist von kürzester Dauer«}, Bund der Bühnendichter, 12. 4. 1906.}}}\label{K_L03416-6h} in der »\textcolor{green}{Schaubühne}{}\ledrightnote{\textcolor{green}{Die Schaubühne}}« halte. Ich sagte, dass ich dagegen
               sei. Er ließ seinen erstaunten Klagelaut vernehmen, und meinte dann, {\pb}\uline{Sie} hätten \textcolor{blue}{ihm}{}\ledrightnote{{$\rightarrow$}\textcolor{blue}{Siegfried Trebitsch}} einen »begeisterten« Brief geschrieben. Ich bin wirklich
               nicht sehr für diesen Vorschlag, der nur aus der \label{K_L03416-7v}\edtext{Seidenbranche}{\lemma{\textnormal{\emph{Seidenbranche}}}\Cendnote{\textnormal{Anspielung
                  auf \textcolor{blue}{Trebitsch} großindustriellen Hintergrund}}}\label{K_L03416-7h} kommt; glaube an Ihre
               »Begeisterung« natürlich nicht, und halte die ganze Sache für unwichtig. Auch die
               Dienstboten betrügen uns, und man denkt nicht daran, sie abzuschaffen. Es fragt sich
               immer nur, um wie viel die Agenten die Autoren übervorteilen. Und das ist im Ganzen
               nicht gar so erheblich.\pend
           
\pstart
           Heute schrieb mir \textcolor{blue}{Bahr}{}\ledrightnote{\textcolor{blue}{Hermann Bahr}}, dass er Samstag{ }Abend auf zwei Tage \textcolor{pink}{her}{}\ledrightnote{{$\rightarrow$}\textcolor{pink}{Berlin}}kommt. Das ist mir weitaus angenehmer. Sonst bin ich ziemlich allein;
               kann mir zu \textcolor{blue}{Harden}{}\ledrightnote{\textcolor{blue}{Maximilian Harden}} kein Herz faßen seit jenem
                  \label{K_L03416-8v}\edtext{\textcolor{green}{Artikel}{}\ledrightnote{{$\rightarrow$}\textcolor{green}{Theater}}}{\lemma{\textnormal{\emph{Artikel}}}\Cendnote{\textnormal{siehe Felix Salten an Arthur Schnitzler, 9. 3. 1906}}}\label{K_L03416-8h} und
                  hab’ ihn seither auch nicht gesehen noch gesucht. Heute – es ist überhaupt ein lebhafter Tag – telefonirte
               mir Ihre \textcolor{blue}{Schwägerin}{}\ledrightnote{{$\rightarrow$}\textcolor{blue}{Elisabeth Steinrück}} wegen
               einer Schiffskarte. Ich bat sie, dieser Tage zu \textcolor{blue}{uns}{}\ledrightnote{{$\rightarrow$}\textcolor{blue}{Ottilie Salten}} zu kommen, damit wir alles genauer besprechen.\pend
           
\pstart
           Hier lege ich Ihnen das zweite \label{K_L03416-9v}\edtext{\textcolor{green}{Russenfeuilleton}{}\ledrightnote{{$\rightarrow$}\textcolor{green}{Russisches Theater. II}}}{\lemma{\textnormal{\emph{Russenfeuilleton}}}\Cendnote{\textnormal{\textcolor{blue}{Felix Salten}: \emph{\textcolor{green}{Russisches Theater. II}}. In: \emph{\textcolor{green}{B. Z. am Mittag}}, Jg. 30, Nr. 70, 23. 3. 1906, S. 2–3.}}}\label{K_L03416-9h} bei, und das
               über \label{K_L03416-10v}\edtext{\textcolor{green}{Kater Lampe}{}\ledrightnote{\textcolor{green}{Kater Lampe}}}{\lemma{\textnormal{\emph{Kater Lampe}}}\Cendnote{\textnormal{Das \textcolor{green}{Stück} von \textcolor{blue}{Rosenow}
                  besprochen in: \textcolor{blue}{Felix Salten}: \emph{\textcolor{green}{»Kater Lampe«}}. In: \emph{\textcolor{green}{B.
                        Z. am Mittag}}, Jg. 30, Nr. 72, 26. 3. 1906, S. 2.}}}\label{K_L03416-10h}. Herzliche Grüße von \textcolor{blue}{uns}{}\ledrightnote{{$\rightarrow$}\textcolor{blue}{Ottilie Salten}} zu Ihnen.{\\}Ihr
                  \spacefill\mbox{Salten}\pend
           \endnumbering\briefempfaengerindex{Schnitzler, Arthur@\textsc{Schnitzler, Arthur}!zzzSalten, Felix@\emph{von Felix Salten}!1906-03-281@{28. 3. 1906}|)be}\mylabel{h}  \normalsize

\doendnotes{C}
\bigskip
\vfill

\clearpage

\footnotesize

\lohead{\textsc{register}}

% Definiere theindex-Environment komplett neu ohne reledmac
\makeatletter
\renewenvironment{theindex}{%
  \section*{\indexname}%
  \setlength{\parindent}{0pt}%
  \setlength{\parskip}{0pt plus 0.3pt}%
  \let\item\@idxitem
}{%
  \clearpage
}
\makeatother

\IfFileExists{\jobname-pw.ind}{\input{\jobname-pw.ind}}{}

\end{document}

      