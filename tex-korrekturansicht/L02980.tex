%% latex-korrekturansicht-vorspann.tex
%% Vorspann für die Korrekturansicht.
%% Lädt die gemeinsame Datei latex-vorspann.tex mit gesetztem Schalter.

\newif\ifkorrekturansicht
\korrekturansichttrue

\input{../tex-inputs/latex-vorspann}


\renewcommand{\erwaehntePersonen}{Personen: Otto Brahm, Paul Goldmann, Mirjam Horwitz, Adolf Lantz, Emil Lessing, Theodore Rottenberg, Felix Salten, Ottilie Salten, Olga Schnitzler, Julius Schnitzler, Curt Wigand}
\renewcommand{\erwaehnteInstitutionen}{Institutionen: Die Zeit}
\renewcommand{\erwaehnteOrte}{Orte: Berlin, Wien}
\renewcommand{\erwaehnteWerke}{Werke: Der Schleier der Beatrice. Schauspiel in fünf Akten, Tagebuch}
\section[ Arthur Schnitzler an Felix Salten, 4. 3. 1903]{Arthur Schnitzler an Felix Salten, 4. 3. 1903}
\nopagebreak\mylabel{v}
\rehead{ }\normalsize\beginnumbering\briefempfaengerindex{Salten, Felix@\textsc{Salten, Felix}!zzzSchnitzler, Arthur@\emph{von Arthur Schnitzler}!1903-03-041@{4. 3. 1903}|(be}
\toendnotes[C]{\smallbreak\pagebreak[2]}\Standort{Wienbibliothek im Rathaus, ZPH 1681, 2.1.516.}
\physDesc{Brief, 1 Blatt, 4 Seiten, 1354 Zeichen
\newline{}Handschrift: Bleistift, deutsche Kurrent
\newline{}Ordnung: mit Bleistift von unbekannter Hand Nummerierung der Blätter des Konvoluts:
                                    »55«–»56« }\toendnotes[C]{\smallbreak}
\pstart
           \raggedleft{}{\pb}4. \textcolor{gray}{3}. 903.\pend
           
\pstart
           lieber Freund, mit \label{K_L02980-1v}\edtext{\textsc{\textcolor{blue}{M. H.}{}\ledrightnote{\textcolor{blue}{Mirjam Horwitz}}}}{\lemma{\textnormal{\emph{M. H.}}}\Cendnote{\textnormal{siehe Felix Salten an Arthur Schnitzler, 3. 3. 1903}}}\label{K_L02980-1h} konnte ich bisher kaum hundert Worte unauffällg ſprechen; der Brief, den Sie
               erhalten, iſt natürlich die Reaction auf meine Mittheilg; – in dieſen Tagen habe ich
               jedenfalls weiter Gelegenheit ſie zu \label{K_L02980-2v}\edtext{ſehen (vielleicht heute)}{\lemma{\textnormal{\emph{ſehen (vielleicht heute)}}}\Cendnote{\textnormal{siehe A. S.: \emph{Tagebuch}, 4. 3. 1903}}}\label{K_L02980-2h} und bringe das gewünſchte ſchonend bei. Ich habe nicht den Eindruck, daſs
               Gefahren drohen. Nicht »Verlogenheit«, aber naive Unechtheit ſozuſagen. Glauben Sie
               nicht? –\pend
           
\pstart
           {\pb}– Die \label{K_L02980-3v}\edtext{\textcolor{green}{Proben}{}\ledrightnote{{$\rightarrow$}\textcolor{green}{Der Schleier der Beatrice. Schauspiel in fünf Akten}} haben mir keine
               beſondre Freude gemacht}{\lemma{\textnormal{\emph{Proben … gemacht}}}\Cendnote{\textnormal{siehe A. S.: \emph{Tagebuch}, 23. 2. 1903, 24. 2. 1903 und 26. 2. 1903}}}\label{K_L02980-3h}; i{\geminationm}erhin ko{\geminationm}t
               einiges beſſer heraus als ich dachte. Mit \textcolor{blue}{Leſſing}{}\ledrightnote{\textcolor{blue}{Emil Lessing}} vertrag ich mich ſchlecht. \textcolor{blue}{Brahm}{}\ledrightnote{\textcolor{blue}{Otto Brahm}} iſt klug und quälend wie i{\geminationm}er. \textcolor{blue}{Paul G.}{}\ledrightnote{\textcolor{blue}{Paul Goldmann}} geht als »\label{K_L02980-4v}\edtext{verbloedeter Thor}{\lemma{\textnormal{\emph{verbloedeter Thor}}}\Cendnote{\textnormal{siehe A. S.: \emph{Tagebuch}, 22. 2. 1903}}}\label{K_L02980-4h}« herum. (So ne{\geminationn}t er ſich ſelbſt, in Anſchluſs an
               eine \introOben{}unglückliche\introOben{}{ }\textcolor{blue}{Liebesgeſchichte}{}\ledrightnote{{$\rightarrow$}\textcolor{blue}{Theodore Rottenberg}}, die er in
               ganz \textcolor{pink}{Berlin}{}\ledrightnote{\textcolor{pink}{Berlin}} ſelber erzählt hat.) – \label{K_L02980-5v}\edtext{Heute{ }Abend ko{\geminationm}t \textcolor{blue}{Olga}{}\ledrightnote{\textcolor{blue}{Olga Schnitzler}} an}{\lemma{\textnormal{\emph{Heute … an}}}\Cendnote{\textnormal{siehe A. S.: \emph{Tagebuch}, 4. 3. 1903}}}\label{K_L02980-5h}, { }{\pb}\label{K_L02980-6v}\edtext{Samſtag mein \textcolor{blue}{Bruder}{}\ledrightnote{{$\rightarrow$}\textcolor{blue}{Julius Schnitzler}} (wahrſcheinlich}{\lemma{\textnormal{\emph{Samſtag … (wahrſcheinlich}}}\Cendnote{\textnormal{Er dürfte nicht angereist sein, jedenfalls erwähnt in \textcolor{blue}{Schnitzler} in diesen Tagen nicht im \emph{\textcolor{green}{Tagebuch}}.}}}\label{K_L02980-6h}.) – Ich hoffe \label{K_L02980-7v}\edtext{Dinſtg{ }früh zu Hauſe}{\lemma{\textnormal{\emph{Dinſtg früh zu Hauſe}}}\Cendnote{\textnormal{siehe A. S.: \emph{Tagebuch}, 10. 3. 1903}}}\label{K_L02980-7h} zu ſein und \label{K_L02980-8v}\edtext{ſpreche Sie wohl
                  gleich}{\lemma{\textnormal{\emph{ſpreche Sie wohl
                  gleich}}}\Cendnote{\textnormal{Nachweislich sahen sie sich
                  bereits einen Tag nach \textcolor{blue}{Schnitzler}s Rückkehr,
                     vgl. A. S.: \emph{Tagebuch}, 11. 3. 1903.}}}\label{K_L02980-8h} in
               den erſten Tagen. – Zu dem neuen \label{K_L02980-9v}\edtext{»\begin{otherlanguage}{french}Avancement\end{otherlanguage}«}{\lemma{\textnormal{\emph{»Avancement«}}}\Cendnote{\textnormal{französisch: Beförderung}}}\label{K_L02980-9h} gratulir ich herzlich.
                  \label{K_L02980-10v}\edtext{Herr \textsc{\textcolor{blue}{Wigand}{}\ledrightnote{\textcolor{blue}{Curt Wigand}}} war hier}{\lemma{\textnormal{\emph{Herr Wigand war hier}}}\Cendnote{\textnormal{siehe A. S.: \emph{Tagebuch}, 3. 3. 1903}}}\label{K_L02980-10h} bei mir; ſolang ich nur durch \textsc{\textcolor{blue}{Lantz}{}\ledrightnote{\textcolor{blue}{Adolf Lantz}}} von den \label{K_L02980-888v}\edtext{adminiſtr Zuſtänden der »\textcolor{brown}{Zeit}{}\ledrightnote{\textcolor{brown}{Die Zeit}}}{\lemma{\textnormal{\emph{adminiſtr … »Zeit}}}\Cendnote{\textnormal{Die
                  Unzufriedenheit an der Führung der Tageszeitung dürfte sich auf die Person von \textcolor{blue}{Heinrich Kanner}
                  konzentriert haben, vgl. Felix Salten an Arthur Schnitzler, 9. 3. 1906.}}}\label{K_L02980-888h}«
               erfahren hatte, konnte ich einige für {\pb}unbewußt übertrieben halten, aber nach den Berichten des Hrn \textcolor{blue}{W.}{}\ledrightnote{\textcolor{blue}{Curt Wigand}} find ich das Verhalten des hier in Betracht ko{\geminationm}enden Hinaus\textcolor{gray}{ſ}chmeißer\substVorne{}\textsuperscript{ un\textcolor{gray}{d}}\substDazwischen{},\substHinten{} Gageverkürzer und Proceſsführer einfach ſkandalös. –\pend
           
\pstart
           – Leben Sie wohl, ſeien Sie herzlich gegrüßt, auf Wiederſehn {\\[\baselineskip]}Ich hoffe, Ihre
                  \textcolor{blue}{Frau}{}\ledrightnote{{$\rightarrow$}\textcolor{blue}{Ottilie Salten}} iſt wohl. {\\[\baselineskip]}Ihr {\\[\baselineskip]}\spacefill\mbox{A.}\pend
           \leftskip=0em{}\endnumbering\briefempfaengerindex{Salten, Felix@\textsc{Salten, Felix}!zzzSchnitzler, Arthur@\emph{von Arthur Schnitzler}!1903-03-041@{4. 3. 1903}|)be}\mylabel{h}  \normalsize

\doendnotes{C}
\bigskip
\vfill

\clearpage

\footnotesize

\lohead{\textsc{register}}

% Definiere theindex-Environment komplett neu ohne reledmac
\makeatletter
\renewenvironment{theindex}{%
  \section*{\indexname}%
  \setlength{\parindent}{0pt}%
  \setlength{\parskip}{0pt plus 0.3pt}%
  \let\item\@idxitem
}{%
  \clearpage
}
\makeatother

\IfFileExists{\jobname-pw.ind}{\input{\jobname-pw.ind}}{}

\end{document}

      