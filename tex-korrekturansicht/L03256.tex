%% latex-korrekturansicht-vorspann.tex
%% Vorspann für die Korrekturansicht.
%% Lädt die gemeinsame Datei latex-vorspann.tex mit gesetztem Schalter.

\newif\ifkorrekturansicht
\korrekturansichttrue

\input{../tex-inputs/latex-vorspann}


\renewcommand{\erwaehntePersonen}{Personen: Clementine Goldmann, Olga Schnitzler}
\renewcommand{\erwaehnteOrte}{Orte: Berlin, Tirol, Welsberg-Taisten, Wildbad Waldbrunn}
\renewcommand{\erwaehnteWerke}{}
\section[ Paul Goldmann an Arthur Schnitzler, 16. 8. 1907]{Paul Goldmann an Arthur Schnitzler, 16. 8. 1907}
\nopagebreak\mylabel{v}
\rehead{ }\normalsize\beginnumbering\briefempfaengerindex{Schnitzler, Arthur@\textsc{Schnitzler, Arthur}!zzzGoldmann, Paul@\emph{von Paul Goldmann}!1907-08-161@{16. 8. 1907}|(be}
\toendnotes[C]{\smallbreak\pagebreak[2]}\Standort{DLA, A:Schnitzler, HS.NZ85.1.3175.}
\physDesc{Postkarte
\newline{}Handschrift: blaue Tinte, lateinische Kurrent
\newline{}Versand: 1) Stempel: »\nobreak{}\oindex{Berlin@\textbf{Berlin}, \emph{https://www.geonames.org/ontologyP.PPLC}|pwk}Berlin, W 9, 16. 8. 07, 11-12 V.\nobreak{}«.   2) Stempel: »\nobreak{}\oindex{Welsberg-Taisten@\textbf{Welsberg-Taisten}, \emph{Besiedelter Ort (A.BSO)}|pwk}Wels{[}berg{]}, 1\textcolor{gray}{×}. 8. \textcolor{gray}{×}\textcolor{gray}{7}\nobreak{}«. 
\newline{}Schnitzler: mit Bleistift das Datum »16. 8. {[}19{]}07« vermerkt }\toendnotes[C]{\smallbreak}\pstart{}{\pb}Herrn Dr. Arthur Schnitzler\pend{}\pstart{}\textcolor{pink}{Welsberg im Pustertal}{}\ledrightnote{\textcolor{pink}{Welsberg-Taisten}}\pend{}\pstart{}\textcolor{pink}{Wildbad Waldbrunn}{}\ledrightnote{\textcolor{pink}{Wildbad Waldbrunn}}.\pend{}\pstart{}\textcolor{pink}{Tirol}{}\ledrightnote{\textcolor{pink}{Tirol}}.\pend{}
{\bigskip}
\pstart
           \noindent{}{\pb}Lieber Freund, Ich komme vielleicht nächste Woche mit
               meiner \textcolor{blue}{Mutter}{}\ledrightnote{{$\rightarrow$}\textcolor{blue}{Clementine Goldmann}} nach \label{K_L03256-5v}\edtext{\textcolor{pink}{Welsberg}{}\ledrightnote{\textcolor{pink}{Welsberg-Taisten}}}{\lemma{\textnormal{\emph{Welsberg}}}\Cendnote{\textnormal{siehe Paul Goldmann an Arthur Schnitzler, 18. 8. 1907}}}\label{K_L03256-5h}, kann Dich aber
               natürlich nicht bitten, mich abzuwarten, da der Tag meines Eintreffens noch unbesti{\geminationm}t ist; hingegen bitte ich Dich sehr, für meine \textcolor{blue}{Mutter}{}\ledrightnote{{$\rightarrow$}\textcolor{blue}{Clementine Goldmann}} und mich\strikeout{,} je ein ruhiges und nicht teueres Zimmer, etwa von
                  Donnerstag ab, reservieren zu lassen. Ich hoffe
               sicher, Dir \label{K-L03256-2v}\edtext{im Laufe meiner
               Urlaubsreise die Hand drücken}{\lemma{\textnormal{\emph{im … drücken}}}\Cendnote{\textnormal{nicht
                  geschehen}}}\label{K-L03256-2h} zu können und bin mit herzlichen Grüßen an Dich und Deine \textcolor{blue}{Frau}{}\ledrightnote{{$\rightarrow$}\textcolor{blue}{Olga Schnitzler}}\label{T_L03256-1v}\edtext{Dein}{\lemma{\textnormal{\emph{Dein}}}\Cendnote{\textnormal{in deutscher Kurrentschrift}}}\label{T_L03256-1h}{ }\spacefill\mbox{Paul Goldmann}.\pend
           \endnumbering\briefempfaengerindex{Schnitzler, Arthur@\textsc{Schnitzler, Arthur}!zzzGoldmann, Paul@\emph{von Paul Goldmann}!1907-08-161@{16. 8. 1907}|)be}\mylabel{h}  \normalsize

\doendnotes{C}
\bigskip
\vfill

\clearpage

\footnotesize

\lohead{\textsc{register}}

% Definiere theindex-Environment komplett neu ohne reledmac
\makeatletter
\renewenvironment{theindex}{%
  \section*{\indexname}%
  \setlength{\parindent}{0pt}%
  \setlength{\parskip}{0pt plus 0.3pt}%
  \let\item\@idxitem
}{%
  \clearpage
}
\makeatother

\IfFileExists{\jobname-pw.ind}{\input{\jobname-pw.ind}}{}

\end{document}

      