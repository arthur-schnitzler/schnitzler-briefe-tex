%% latex-korrekturansicht-vorspann.tex
%% Vorspann für die Korrekturansicht.
%% Lädt die gemeinsame Datei latex-vorspann.tex mit gesetztem Schalter.

\newif\ifkorrekturansicht
\korrekturansichttrue

\input{../tex-inputs/latex-vorspann}


               \section[Hermann Bahr an Arthur Schnitzler, 27. 4. 1904]{ Hermann Bahr an Arthur Schnitzler, 27. 4. 1904}\nopagebreak\mylabel{v}\rehead{ }\normalsize\beginnumbering\briefempfaengerindex{Schnitzler, Arthur@\textsc{Schnitzler, Arthur}!zzzBahr, Hermann@\emph{von Hermann Bahr}!1904-04-271@{27. 4. 1904}|(be} \toendnotes[C]{\smallbreak\pagebreak[2]} \Standort{CUL, Schnitzler, B 5b.}
\physDesc{Kartenbrief
\newline{}Handschrift: schwarze Tinte, deutsche Kurrent\newline{}Versand: 1) Rohrpost 2) Stempel: »\nobreak{}\oindex{XIII., Hietzing@\textbf{XIII., Hietzing}, \emph{Bezirk (A.BZK)}|pwk}Wien 13/5, 27{[}.{]} IV. 04, XII\nobreak{}«. 3) Stempel: »\nobreak{}\oindex{XII., Meidling@\textbf{XII., Meidling}, \emph{Bezirk (A.BZK)}|pwk}Wien 12/1, 27 IV 04, 1 N\nobreak{}«. 4) Stempel: »\nobreak{}\oindex{XII., Meidling@\textbf{XII., Meidling}, \emph{Bezirk (A.BZK)}|pwk}Wien 12/1, 27 IV {[}04{]}, 2.30N\nobreak{}«. 
\newline{}Schnitzler: mit Bleistift die Jahreszahl zum Datum ergänzt: »904« \newline{}Ordnung: mit Bleistift von unbekannter Hand nummeriert:
                                    »117« }\buchAbdrucke{\weitereDrucke{Hermann Bahr, Arthur Schnitzler: \emph{Briefwechsel, Aufzeichnungen, Dokumente (1891–1931)}. Hg. Kurt Ifkovits und Martin Anton Müller. Göttingen: \emph{Wallstein} 2018, S. 306.} }\toendnotes[C]{\smallbreak}\pstart{}{\pb}Pneumatisch\pend{}\pstart{}Herrn \textsc{D\textsuperscript{r} Arthur
                     Schnitzler}\pend{}\pstart{}\textcolor{pink}{\textsc{Wien XVIII}}{}\ledrightnote{\textcolor{pink}{XVIII., Währing}}\pend{}\pstart{}\textcolor{pink}{\textsc{Spöttelgasse 7}}{}\ledrightnote{\textcolor{pink}{Edmund-Weiß-Gasse}}\pend{}{\bigskip}\pstart
           \raggedleft{}{\pb}27. 4.\pend
           \pstart{}Lieber Arthur!\pend\pstart
           Herzlichſten Dank für Deinen \label{LL279-1v}Brief\label{LL279-1h},
               der ſich mit meinem an Dich gekreuzt hat. Ich wollte nun heute abends nach \textcolor{pink}{Hietzing}{}\ledrightnote{\textcolor{pink}{XIII., Hietzing}} kommen. Da mir nun aber \label{K_L01396_1v}\edtext{\textcolor{blue}{Gerty}{}\ledrightnote{\textcolor{blue}{Gertrude von Hofmannsthal}}{ }ſchreibt}{\lemma{\textnormal{\emph{Gerty ſchreibt}}}\Cendnote{\textnormal{nicht im \emph{Briefwechsel} Hofmannsthal/Bahr}}}\label{K_L01396_1h}, \textcolor{blue}{Hugo}{}\ledrightnote{\textcolor{blue}{Hugo von Hofmannsthal}}{ }ſei auf dem \textcolor{pink}{Semmering}{}\ledrightnote{\textcolor{pink}{Semmering}}, denke ich, daß Du wol auch nicht kommen wirſt, und bitte um ein
               anderes Rendezvous, da ich Dich ſehr gern vor Deiner \label{K_L01396_2v}\edtext{Abreiſe}{\lemma{\textnormal{\emph{Abreiſe}}}\Cendnote{\textnormal{Am 30. 4. 1904 trat \textcolor{blue}{Schnitzler} eine mehrwöchige \textcolor{pink}{Italien}reise an.}}}\label{K_L01396_2h} noch ſehen möchte.\pend
           \pstart
           Mit den beſten Grüßen an Deine \textcolor{blue}{Frau}{}\ledrightnote{→\textcolor{blue}{Olga Schnitzler}}{\\[\baselineskip]}herzlichſt{\\[\baselineskip]}\spacefill\mbox{HermB.}\pend
           \leftskip=0em{}\endnumbering\briefempfaengerindex{Schnitzler, Arthur@\textsc{Schnitzler, Arthur}!zzzBahr, Hermann@\emph{von Hermann Bahr}!1904-04-271@{27. 4. 1904}|)be}\mylabel{h}  \normalsize

\doendnotes{C}
\bigskip
\vfill

\clearpage

\footnotesize

\lohead{\textsc{register}}

% Definiere theindex-Environment komplett neu ohne reledmac
\makeatletter
\renewenvironment{theindex}{%
  \section*{\indexname}%
  \setlength{\parindent}{0pt}%
  \setlength{\parskip}{0pt plus 0.3pt}%
  \let\item\@idxitem
}{%
  \clearpage
}
\makeatother

\IfFileExists{\jobname-pw.ind}{\input{\jobname-pw.ind}}{}

\end{document}

      