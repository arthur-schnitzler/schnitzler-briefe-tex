%% latex-korrekturansicht-vorspann.tex
%% Vorspann für die Korrekturansicht.
%% Lädt die gemeinsame Datei latex-vorspann.tex mit gesetztem Schalter.

\newif\ifkorrekturansicht
\korrekturansichttrue

\input{../tex-inputs/latex-vorspann}


\renewcommand{\erwaehntePersonen}{Personen: Richard Beer-Hofmann, Ludwig van Beethoven, Max Eugen Burckhard, Heinrich Friedjung, Clementine Goldmann, Heinrich Graetz, Gerhart Hauptmann, Victor Hehn, Georg Hirschfeld, Heinrich Kanner, Max Klinger, Margarethe Manassewitsch, Else Markbreiter, Louise Schnitzler, Olga Schnitzler, Franz Servaes}
\renewcommand{\erwaehnteInstitutionen}{Institutionen: Buchhandlung Gebrüder Bornträger, Buchhandlung L. Rosner, Burgtheater, Die Zeit, J.G. Cotta’sche Buchhandlung Nachfolger, Oskar Leiner GmbH {\kaufmannsund}  Co. KG, Volkstheater, Wiener Secession, Wiener Tierschutz-Verein}
\renewcommand{\erwaehnteOrte}{Orte: Berlin, Dessauer Straße, Leipzig, Prag, Pörtschach, Stuttgart, Tierschutzhaus, Wien}
\renewcommand{\erwaehnteWerke}{Werke: Beethoven, Der Kampf um die Vorherrschaft in Deutschland 1859 bis 1866. 2 Bde., Der Weg zum Licht. Ein Salzburger Märchendrama, Der einsame Weg. Schauspiel in fünf Akten, Der tapfere Cassian. Puppenspiel in einem Akt, Die »neue Richtung«. Polemische Aufsätze über Berliner Theater-Aufführungen, Francesca da Rimini, Gedanken über Goethe, Goethe und das Publikum. Eine Literaturgeschichte im Kleinen, Klinger’s »Beethoven«, Neue Freie Presse, Tagebuch, Volkstümliche Geschichte der Juden in drei Bänden}
\section[ Paul Goldmann an Arthur Schnitzler, 17. 4. {[}1902{]}]{Paul Goldmann an Arthur Schnitzler, 17. 4. {[}1902{]}}
\nopagebreak\mylabel{v}
\rehead{ }\normalsize\beginnumbering\briefempfaengerindex{Schnitzler, Arthur@\textsc{Schnitzler, Arthur}!zzzGoldmann, Paul@\emph{von Paul Goldmann}!1902-04-171@{17. 4. {[}1902{]}}|(be}
\toendnotes[C]{\smallbreak\pagebreak[2]}\Standort{DLA, A:Schnitzler, HS.NZ85.1.3172.}
\physDesc{Brief, 3 Blätter, 10 Seiten
\newline{}Handschrift: blaue Tinte, deutsche Kurrent
\newline{}Schnitzler: 1) mit Bleistift das Jahr »{[}1{]}902« vermerkt  2) mit rotem Buntstift elf Unterstreichungen}\toendnotes[C]{\smallbreak}
\pstart
           \noindent{}\raggedleft{}{\pb}\textcolor{pink}{\textcolor{gray}{\textbf{DESSAUERSTRASSE 19}}}{}\ledrightnote{\textcolor{pink}{Dessauer Straße}}\pend
           
\pstart
           \textcolor{pink}{Berlin}{}\ledrightnote{\textcolor{pink}{Berlin}}, 17. April.\pend
           
\pstart\center{}Mein lieber Freund,\pend
\pstart
           Seit dem Empfang Deines letzten lieben Briefes, der nach meiner \label{K_L03204-1v}\edtext{Rückkehr aus \textsc{\textcolor{pink}{Prag}{}\ledrightnote{\textcolor{pink}{Prag}}}}{\lemma{\textnormal{\emph{Rückkehr aus Prag}}}\Cendnote{\textnormal{siehe Paul Goldmann an Arthur Schnitzler, 1. 4. [1902]}}}\label{K_L03204-1h} eintraf, will ich Dir \uline{täglich} ſchreiben, und
               täglich muß ich darauf verzichten. Es iſt unbeſchreiblich, was jetzt wieder Alles an
               Arbeit, Beſuchen \textsc{etc.} auf mich einſtrömt. Ich bin Dir ſehr dankbar, daß Du meine
               Antwort nicht abgewartet und mich abermals heut durch
               Deine lieben Nachrichten erfreut haſt. Dieſer \label{K_L03204-2v}\edtext{Bernhardiner}{\lemma{\textnormal{\emph{Bernhardiner}}}\Cendnote{\textnormal{\textcolor{blue}{Schnitzler} besaß für kurze Zeit, vermutlich ab
                  dem 23. 3. 1902, einen
                  Bernhardiner namens »Bern«. Im Oktober wurde er in dem im gleichen
                  Monat eröffneten \textcolor{pink}{Tierschutzhaus} des \emph{\textcolor{brown}{Wiener Tierschutz-Vereins}} behandelt; Mitte
                     Dezember erneut. Ab Januar 1903 versucht er ihn zu
                  vermitteln, da wohnt er aber bereits nicht mehr bei ihnen (siehe Arthur Schnitzler an Richard Beer-Hofmann, 14. 1. 1903, siehe Hermann Bahr an Arthur Schnitzler, 4. 4. [1903]). In diesem Jahr finden sich noch drei
                  Erwähnungen im \emph{\textcolor{green}{Tagebuch}} (23. 5. 1903, 18. 6. 1903 und 6. 8. 1903). Vgl.
                     \emph{Briefe} II,118.
               }}}\label{K_L03204-2h} muß herrlich ſein. Ich freue mich ſchon ſehr darauf, ihn kennen zu lernen.
                  {\pb}Was Du über \textsc{\textcolor{blue}{Hirschfeld}{}\ledrightnote{\textcolor{blue}{Georg Hirschfeld}}} ſchreibſt, iſt ſehr ſchön geſagt. Die Freunde und »literariſchen Kritiker«, die
               den unentwickelten \textcolor{blue}{Burſchen}{}\ledrightnote{{$\rightarrow$}\textcolor{blue}{Georg Hirschfeld}}, deſſen Sentimentalität ſie für Poeſie nehmen, zum Dichter
               ausgeſchrieen haben, haben allerdings viel Schuld an dem jämmerlichen Ende, – aber
               doch nicht die einzige. Wer im Stande iſt, ein flaches Machwerk, wie den »\textcolor{green}{Weg zum Licht}{}\ledrightnote{\textcolor{green}{Der Weg zum Licht. Ein Salzburger Märchendrama}}« zu ſchreiben, in dem auch nicht
               die leiſeſte Spur von Perſönlichkeit ſteckt, der hat eben niemals eine Perſönlichkeit
               gehabt. Denn das iſt vollkommen ausgeſchloſſen, daß man aus einem Dichter {\pb}plötzlich ein Flachkopf wird. Der »\textcolor{green}{Weg zum Licht}{}\ledrightnote{\textcolor{green}{Der Weg zum Licht. Ein Salzburger Märchendrama}}« iſt nicht verfehlt, ſondern complet
               talentlos. Das iſt ein Unterſchied.\pend
           
\pstart
           \textsc{\textcolor{blue}{Servaes}{}\ledrightnote{\textcolor{blue}{Franz Servaes}}}{ }\label{K_L03204-11v}\edtext{\textcolor{green}{Feuilleton}{}\ledrightnote{{$\rightarrow$}\textcolor{green}{Klinger’s »Beethoven«}}}{\lemma{\textnormal{\emph{Feuilleton}}}\Cendnote{\textnormal{\textcolor{blue}{Franz Servaes}: \emph{\textcolor{green}{Klinger’s »Beethoven«}}. In: \emph{\textcolor{green}{Neue Freie Presse}}, Nr. 13521, 16. 4. 1902, Morgenblatt, S. 1–3. \textcolor{blue}{Servaes}’ Urteil fiel sehr gut aus.}}}\label{K_L03204-11h} über \textsc{\textcolor{blue}{\textcolor{green}{Klinger}{}\ledrightnote{{$\rightarrow$}\textcolor{green}{Beethoven}}}{}\ledrightnote{\textcolor{blue}{Max Klinger}}}, \strikeout{hat} das ich eben geleſen, hat mir ſehr gut
               gefallen. Aber iſt auch das Urtheil richtig? Oder iſt wieder ein \label{K_L03204-12v}\edtext{\textcolor{brown}{Seceſſion}{}\ledrightnote{\textcolor{brown}{Wiener Secession}}s-Schwindel}{\lemma{\textnormal{\emph{Seceſſions-Schwindel}}}\Cendnote{\textnormal{\textcolor{blue}{Max Klinger}s \emph{\textcolor{green}{Beethovenstatue}} stand im Mittelpunkt der 14. Ausstellung
                  der \emph{\textcolor{brown}{Wiener Secession}}, die \textcolor{blue}{Beethoven} gewidmet war und von 15. 4. 1902 bis 15. 6. 1902
                  stattfand.}}}\label{K_L03204-12h} dabei? Ich kann es mir allerdings kaum denken; ich ahne etwas
               Großes, wenn \textsc{\textcolor{blue}{Klinger}{}\ledrightnote{\textcolor{blue}{Max Klinger}}} einen \textsc{\textcolor{green}{\textcolor{blue}{Beethoven}{}\ledrightnote{\textcolor{blue}{Ludwig van Beethoven}}}{}\ledrightnote{{$\rightarrow$}\textcolor{green}{Beethoven}}} gemacht hat.\pend
           
\pstart
           Ich habe die Idee, etwa zehn meiner Theater-Feuilletons, die ſich mit \textsc{\textcolor{blue}{Hauptmann}{}\ledrightnote{\textcolor{blue}{Gerhart Hauptmann}}} und ſeinen Anhängern beſchäftigen, {\pb}zu ſammeln
               und als Kampf-Buch unter dem ironiſchen Titel \label{K_L03204-14v}\edtext{»\textcolor{green}{Die neue
                  Dichtung}{}\ledrightnote{{$\rightarrow$}\textcolor{green}{Die »neue Richtung«. Polemische Aufsätze über Berliner Theater-Aufführungen}}«}{\lemma{\textnormal{\emph{»Die neue
                  Dichtung«}}}\Cendnote{\textnormal{\textcolor{blue}{Paul Goldmann}: \emph{\textcolor{green}{Die »neue Richtung«. Polemische Aufsätze über Berliner
                        Theater-Aufführungen}}. \textcolor{pink}{Wien}: \emph{\textcolor{brown}{Buchhandlung L. Rosner}}{ }1902, vordatiert auf 1903.}}}\label{K_L03204-14h} herauszugeben. Glaubſt Du, daß ein ſolches Buch Leſer finden
               würde? Oder hängen Theater-Feuilletons nicht doch zu ſehr mit dem Tage zuſammen, als
               daß ſie in ein Buch hineingehörten? Die Idee kam mir, da ich neulich wieder hörte,
               wie ſehr die \textsc{\textcolor{blue}{Hauptmann}{}\ledrightnote{\textcolor{blue}{Gerhart Hauptmann}}-Clique} hier mich haßt. Man
               hat einer Dame Vorwürfe gemacht, daß ſie im Theater freundlich mit mir geſprochen
               hat! Wenn ich ſehe, daß man mit ſolchen Mitteln eine künſtleriſche Überzeugung {\pb}bekämpfen will, ſo habe ich den Drang, meine
               Überzeugung nur umſo ſtärker zu betonen.\pend
           
\pstart
           Was Du mir vom \label{K_L03204-23v}\edtext{Tode der armen \textsc{\textcolor{blue}{Elsa Marktbreiter}{}\ledrightnote{\textcolor{blue}{Else Markbreiter}}}}{\lemma{\textnormal{\emph{Tode … Marktbreiter}}}\Cendnote{\textnormal{\textcolor{blue}{Schnitzler}s Cousine \textcolor{blue}{Else Markbreiter} war am 30. 3. 1902 an Tuberkulose verstorben. Siehe A. S.: \emph{Tagebuch}, 31. 3. 1902.}}}\label{K_L03204-23h} ſchreibſt, iſt ergreifend. Aber
                  \strikeout{was} war es nicht eine Erlöſung? Freilich, das iſt
               auch eine dumme Phraſe. Erlöſt iſt man doch nur, wenn man \uline{weiß}, daß man erlöſt iſt.\pend
           
\pstart
           Ich habe Deiner Frau \textcolor{blue}{Mutter}{}\ledrightnote{{$\rightarrow$}\textcolor{blue}{Louise Schnitzler}}
               nicht kondolirt, weil ich nicht weiß, ob die Verwandtſchaft nahe genug war, um eine
               Condolenz zu rechtfertigen. Wenn ja, ſo {\pb}kondolire,
               bitte, in meinem Namen.\pend
           
\pstart
           Und dieſe arme hübſche \label{K_L03204-32v}\edtext{\textsc{\textcolor{blue}{Grethl Mandl}{}\ledrightnote{\textcolor{blue}{Margarethe Manassewitsch}}}}{\lemma{\textnormal{\emph{Grethl Mandl}}}\Cendnote{\textnormal{\textcolor{blue}{Margarethe Mandl}, ebenso eine Cousine \textcolor{blue}{Schnitzler}s, war, wie er vermutete, an
                  Neuritis erkrankt (vgl. A. S.: \emph{Tagebuch}, 13. 3. 1902). Gestorben ist sie daran nicht.}}}\label{K_L03204-32h}! Wie, um Himmels Willen,
               iſt das ſo plötzlich gekommen? Sie hat mir in \label{K_L03204-112v}\edtext{\textsc{\textcolor{pink}{Pörtschach}{}\ledrightnote{\textcolor{pink}{Pörtschach}}}}{\lemma{\textnormal{\emph{Pörtschach}}}\Cendnote{\textnormal{vermutlich im Sommer 1901}}}\label{K_L03204-112h} noch ſo gut gefallen. Iſt Ausſicht auf Heilung vorhanden?\pend
           
\pstart
           Haſt Du zu \label{K_L03204-332v}\edtext{arbeiten}{\lemma{\textnormal{\emph{arbeiten}}}\Cendnote{\textnormal{\textcolor{blue}{Schnitzler} hatte am 6. 4. 1902 das
                  einaktige Puppenspiel \emph{\textcolor{green}{Der tapfere Cassian}}
                  begonnen. Ebenso hatte er Überlegungen zu seinem Schauspiel \emph{\textcolor{green}{Der einsame Weg}} angestellt (vgl. A. S.: \emph{Tagebuch}, 8. 4. 1902). Hinsichtlich
                     \textcolor{blue}{Goldmann}s wiederholter Forderung, \textcolor{blue}{Schnitzler} solle eine Lustspiel schreiben,
                     siehe Paul Goldmann an Arthur Schnitzler, 2. 5. [1900].}}}\label{K_L03204-332h}
               angefangen? Denkſt Du an das Luſtſpiel? Ich weiß, Du wirſt über dieſe meine Frage
               wieder ſehr aufgebracht ſein, aber Du mußt mich ſchon entſchuldigen, wenn ich unſeren
               einzigen \strikeout{D} Dramatiker, der \strikeout{\textcolor{gray}{h}\textcolor{gray}{×}\-\textcolor{gray}{×}\-\textcolor{gray}{×}\-\textcolor{gray}{×}\-\textcolor{gray}{×}} Humor hat, hier und da danach frage, {\pb}ob er
               nicht ein Luſtſpiel ſchreiben möchte? Du wirſt wieder ſagen: »Es fällt \substVorne{}\textsuperscript{Dir}\substDazwischen{}mir\substHinten{} nichts ein.« \strikeout{Aben} Aber das \strikeout{Schreiben} Schreiben wäre ſehr einfach, wenn wir nur das
               zu ſchreiben brauchten, was uns \strikeout{einfiele}
               einfällt.\pend
           
\pstart
           Wie geht es \textsc{\textcolor{blue}{Olga}{}\ledrightnote{\textcolor{blue}{Olga Schnitzler}}}? Grüße ſie herzlichſt von mir. Ich ſchreibe ihr nächſtens – jawohl, ganz gewiß,
               nächſtens!\pend
           
\pstart
           Lies’ \textsc{\textcolor{blue}{Hehn}{}\ledrightnote{\textcolor{blue}{Victor Hehn}}}: \textcolor{green}{Gedanken über \textsc{Goethe}}{}\ledrightnote{\textcolor{green}{Gedanken über Goethe}}, namentlich den Aufſatz \label{K_L03204-343v}\edtext{\textcolor{green}{\textsc{Goethe} und das Publikum}{}\ledrightnote{\textcolor{green}{Goethe und das Publikum. Eine Literaturgeschichte im Kleinen}}}{\lemma{\textnormal{\emph{Goethe und das Publikum}}}\Cendnote{\textnormal{\textcolor{blue}{Viktor Hehn}: \emph{\textcolor{green}{Goethe und das Publikum. Eine Literaturgeschichte im
                        Kleinen}}. In: \emph{\textcolor{green}{Gedanken über
                     Goethe}}. \textcolor{pink}{Berlin}: \emph{\textcolor{brown}{Gebrüder Borntraeger}}{ }1887, S. 49–185.}}}\label{K_L03204-343h}. Eine Fülle
               intereſſanten Materials in einem wundervoll klaren {\pb}Styl mitgetheilt. Der einzige Fehler iſt\strikeout{\textcolor{gray}{,}} ein irrſinniger Antiſemitismus.\pend
           
\pstart
           \textsc{\textcolor{blue}{Kanner}{}\ledrightnote{\textcolor{blue}{Heinrich Kanner}}} war \textcolor{pink}{hier}{}\ledrightnote{{$\rightarrow$}\textcolor{pink}{Berlin}}. Ich ſoll
                  \label{K_L03204-777v}\edtext{zur »\textcolor{brown}{Zeit}{}\ledrightnote{\textcolor{brown}{Die Zeit}}« als Feuilleton-Redakteur}{\lemma{\textnormal{\emph{zur … Feuilleton-Redakteur}}}\Cendnote{\textnormal{\textcolor{blue}{Heinrich Kanner} dürfte seine Meinung also
                  geändert haben, siehe Paul Goldmann an Arthur Schnitzler, 25. 1. [1902].}}}\label{K_L03204-777h} kommen\substVorne{}\textsuperscript{,}\substDazwischen{}.\substHinten{}{ }\textcolor{brown}{Burgtheater}{}\ledrightnote{\textcolor{brown}{Burgtheater}} und \textcolor{brown}{Volkstheater}{}\ledrightnote{\textcolor{brown}{Volkstheater}} ſind allerdings ſchon an \textsc{\textcolor{blue}{Burckhardt}{}\ledrightnote{\textcolor{blue}{Max Eugen Burckhard}}} vergeben. Ich ſollte alſo nur Redaktions\substVorne{}\textsuperscript{-\textcolor{gray}{Kuli}}\substDazwischen{}-\textsc{Kuli}\substHinten{} ſein und eine rieſige Büreauarbeit leiſten: Kleines und großes Feuilleton,
               eine Sonntagsbeilage \textsc{etc}. Ich glaube nicht, daß ich unter
               dieſen Umſtänden annehmen werde, – umſomehr als meine \textcolor{blue}{Mutter}{}\ledrightnote{{$\rightarrow$}\textcolor{blue}{Clementine Goldmann}} nicht nach \textcolor{pink}{Wien}{}\ledrightnote{\textcolor{pink}{Wien}}{ }{\pb}mitkommen würde\substVorne{}\textsuperscript{.}\substDazwischen{}{ }und ich meinen Hausſtand auflöſen müßte.\substHinten{} Ja, wenn ich verheirathet wäre, ſo wäre das Alles anders. Haſt Du noch immer
               keine Parthie für mich?\pend
           
\pstart
           \label{K_L03204-144v}\edtext{\textsc{\textcolor{blue}{Friedjung}{}\ledrightnote{\textcolor{blue}{Heinrich Friedjung}}s}{ }\textcolor{green}{Buch}{}\ledrightnote{{$\rightarrow$}\textcolor{green}{Der Kampf um die Vorherrschaft in Deutschland 1859 bis 1866. 2 Bde.}}}{\lemma{\textnormal{\emph{Friedjungs Buch}}}\Cendnote{\textnormal{\textcolor{blue}{Heinrich Friedjung}: \emph{\textcolor{green}{Der Kampf um die Vorherrschaft in Deutschland 1859 bis 1866. 2
                     Bde}}. \textcolor{pink}{Stuttgart}: \emph{\textcolor{brown}{Cotta}}{ }1897–1898. \textcolor{blue}{Schnitzler} las das \textcolor{green}{Buch} am 22. 3. 1902.}}}\label{K_L03204-144h} werde ich leſen. Jetzt ſtecke
               ich in \textsc{\textcolor{blue}{Grätz}{}\ledrightnote{\textcolor{blue}{Heinrich Graetz}}}{ }\label{K_L03204-142v}\edtext{»\textcolor{green}{Geſchichte der Juden}{}\ledrightnote{\textcolor{green}{Volkstümliche Geschichte der Juden in drei Bänden}}« (Volksausgabe in drei Bänden)}{\lemma{\textnormal{\emph{»Geſchichte … Bänden)}}}\Cendnote{\textnormal{\textcolor{blue}{H. [=Heinrich] Graetz}: \emph{\textcolor{green}{Volkstümliche Geschichte der Juden in drei Bänden}}. \textcolor{pink}{Leipzig}: \emph{\textcolor{brown}{Oskar Leiner}}{ }1888. Eine Lektüre durch \textcolor{blue}{Schnitzler} ist
                  nicht nachweisbar.}}}\label{K_L03204-142h}. Ein tauſendfach anregendes \textcolor{green}{Buch}{}\ledrightnote{{$\rightarrow$}\textcolor{green}{Volkstümliche Geschichte der Juden in drei Bänden}}. Mußt Du leſen. \label{K_L03204-555v}\edtext{»\textsc{\textcolor{green}{Francesca da Rimini}{}\ledrightnote{\textcolor{green}{Francesca da Rimini}}}«}{\lemma{\textnormal{\emph{»Francesca da Rimini«}}}\Cendnote{\textnormal{siehe A. S.: \emph{Tagebuch}, 2. 4. 1902}}}\label{K_L03204-555h} hat mich bodenlos gelangweilt.\pend
           
\pstart
           Schreib’ mir bald wieder, {\pb}mein lieber Freund, und
               ſei vielmals und von Herzen gegrüßt von {\\}Deinem {\\}\spacefill\mbox{Paul Goldmn}\pend
           
\pstart
           \noindent{}Was macht \textsc{\textcolor{blue}{Richard}{}\ledrightnote{\textcolor{blue}{Richard Beer-Hofmann}}}?\pend
           \endnumbering\briefempfaengerindex{Schnitzler, Arthur@\textsc{Schnitzler, Arthur}!zzzGoldmann, Paul@\emph{von Paul Goldmann}!1902-04-171@{17. 4. {[}1902{]}}|)be}\mylabel{h}
\begin{anhang}
\end{anhang}\normalsize

\doendnotes{C}
\bigskip
\vfill

\clearpage

\footnotesize

\lohead{\textsc{register}}

% Definiere theindex-Environment komplett neu ohne reledmac
\makeatletter
\renewenvironment{theindex}{%
  \section*{\indexname}%
  \setlength{\parindent}{0pt}%
  \setlength{\parskip}{0pt plus 0.3pt}%
  \let\item\@idxitem
}{%
  \clearpage
}
\makeatother

\IfFileExists{\jobname-pw.ind}{\input{\jobname-pw.ind}}{}

\end{document}

      