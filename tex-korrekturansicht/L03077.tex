%% latex-korrekturansicht-vorspann.tex
%% Vorspann für die Korrekturansicht.
%% Lädt die gemeinsame Datei latex-vorspann.tex mit gesetztem Schalter.

\newif\ifkorrekturansicht
\korrekturansichttrue

\input{../tex-inputs/latex-vorspann}


\renewcommand{\erwaehntePersonen}{Personen: Richard Beer-Hofmann, Hugo von Hofmannsthal, Olga Schnitzler, Nanette Speyer, Albert Speyer, Elisabeth Steinrück, Jakob Wassermann}
\renewcommand{\erwaehnteOrte}{Orte: Hotel Baur, Höhlenstein, Monte Cristallo, Vahrn, Wien}
\renewcommand{\erwaehnteWerke}{}
\section[ Paul Goldmann an Arthur Schnitzler, 5. 8. {[}1901{]}]{Paul Goldmann an Arthur Schnitzler, 5. 8. {[}1901{]}}
\nopagebreak\mylabel{v}
\rehead{ }\normalsize\beginnumbering\briefempfaengerindex{Schnitzler, Arthur@\textsc{Schnitzler, Arthur}!zzzGoldmann, Paul@\emph{von Paul Goldmann}!1901-08-051@{5. 8. {[}1901{]}}|(be}
\toendnotes[C]{\smallbreak\pagebreak[2]}\Standort{DLA, A:Schnitzler, HS.NZ85.1.3171.}
\physDesc{Brief, 1 Blatt, 2 Seiten
\newline{}Handschrift: schwarze Tinte, deutsche Kurrent
\newline{}Schnitzler: 1) mit Bleistift das Jahr »{[}1{]}901« vermerkt  2) mit rotem Buntstift vier Unterstreichungen}\toendnotes[C]{\smallbreak}
\pstart
           \centering{}{\pb}\textsc{\textcolor{pink}{Landro}{}\ledrightnote{\textcolor{pink}{Höhlenstein}}}, 5. Auguſt.\pend
           
\pstart{}Mein lieber Freund,\pend
\pstart
           \textsc{\textcolor{blue}{Richard}{}\ledrightnote{\textcolor{blue}{Richard Beer-Hofmann}}s} Telegramm, in dem er mir
               mittheilte, daß Du einen hochgelegenen \textcolor{pink}{Ort}{}\ledrightnote{{$\rightarrow$}\textcolor{pink}{Vahrn}} gefunden, erreichte mich leider zu spät. Ich hatte mich
               bereits in \textsc{\textcolor{pink}{Landro}{}\ledrightnote{\textcolor{pink}{Höhlenstein}}} eingemiethet; ein Zimmer hatte ich in dem \textsc{\textcolor{pink}{Hôtel}{}\ledrightnote{{$\rightarrow$}\textcolor{pink}{Hotel Baur}}} nämlich nur \strikeout{\textcolor{gray}{×}\-\textcolor{gray}{×}} unter der Bedingung bekommen, daß ich mindeſtens eine Woche zu bleiben mich
               verpflichtete. So werde ich alſo nicht vor Ablauf dieſer Woche \label{K_L03077-1v}\edtext{zu Dir kommen}{\lemma{\textnormal{\emph{zu Dir kommen}}}\Cendnote{\textnormal{siehe Paul Goldmann an Arthur Schnitzler, 26. 4. [1901]}}}\label{K_L03077-1h} können, und ich bitte\substVorne{}\textsuperscript{,}\substDazwischen{} D\substHinten{}ich, mich ſogleich von Deinem Aufenthaltsort zu \label{K_L03077-2v}\edtext{{[}v{]}e{[}r{]}ſtändigen}{\lemma{\textnormal{\emph{verſtändigen}}}\Cendnote{\textnormal{\textcolor{blue}{Goldmann} schrieb
                     »beſtändigen«}}}\label{K_L03077-2h}. Die kühle und ſtarke Luft hier bekommt mir
               gut; die trüben Gedanken vermag freilich keine noch ſo kühle Luft zu bannen. Ich
               hatte gehofft, hier ein paar liebe \textcolor{pink}{Wien}{}\ledrightnote{\textcolor{pink}{Wien}}er Mädeln
               zu finden. Aber es iſt nichts vorhanden als die Familie {\pb}\textsc{\textcolor{blue}{Speyer}{}\ledrightnote{\textcolor{blue}{Nanette Speyer}{\newline}\textcolor{blue}{Albert Speyer}}}. Und angeſichts des \textsc{\textcolor{pink}{Monte Cristallo}{}\ledrightnote{\textcolor{pink}{Monte Cristallo}}} ſich über die literariſche Bedeutung von \textsc{\textcolor{blue}{Hoffmannsthal}{}\ledrightnote{\textcolor{blue}{Hugo von Hofmannsthal}}} und \textsc{\textcolor{blue}{Wassermann}{}\ledrightnote{\textcolor{blue}{Jakob Wassermann}}} zu unterhalten, hat keinen beſonderen Reiz. Geſtern bin ich gekommen, und heut möchte
               ich ſchon wieder fort. Aber ich muß bis Sonntag
               feſtſitzen und hoffe nur, daß Du es mir durch Auffindung eines hohen und kühlen
               Aufenthaltsortes dann wenigſtens möglich\strikeout{\textcolor{gray}{ſt}} machſt zu Dir zu kommen.\pend
           
\pstart
           Ich grüße Dich und die \textcolor{blue}{Begleiterinnen}{}\ledrightnote{{$\rightarrow$}\textcolor{blue}{Olga Schnitzler}{\newline}{$\rightarrow$}\textcolor{blue}{Elisabeth Steinrück}} vielmals und herzlichſt. {\\[\baselineskip]}Dein {\\[\baselineskip]}\spacefill\mbox{Paul Goldmann}\pend
           \leftskip=0em{}
\pstart
           \noindent{}Adreſſe: \textsc{\textcolor{pink}{Landro}{}\ledrightnote{\textcolor{pink}{Höhlenstein}}}, \textsc{\textcolor{pink}{Hôtel Baur}{}\ledrightnote{\textcolor{pink}{Hotel Baur}}}.\pend
           \endnumbering\briefempfaengerindex{Schnitzler, Arthur@\textsc{Schnitzler, Arthur}!zzzGoldmann, Paul@\emph{von Paul Goldmann}!1901-08-051@{5. 8. {[}1901{]}}|)be}\mylabel{h}
\begin{anhang}
\end{anhang}\normalsize

\doendnotes{C}
\bigskip
\vfill

\clearpage

\footnotesize

\lohead{\textsc{register}}

% Definiere theindex-Environment komplett neu ohne reledmac
\makeatletter
\renewenvironment{theindex}{%
  \section*{\indexname}%
  \setlength{\parindent}{0pt}%
  \setlength{\parskip}{0pt plus 0.3pt}%
  \let\item\@idxitem
}{%
  \clearpage
}
\makeatother

\IfFileExists{\jobname-pw.ind}{\input{\jobname-pw.ind}}{}

\end{document}

      