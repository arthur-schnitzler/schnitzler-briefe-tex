%% latex-korrekturansicht-vorspann.tex
%% Vorspann für die Korrekturansicht.
%% Lädt die gemeinsame Datei latex-vorspann.tex mit gesetztem Schalter.

\newif\ifkorrekturansicht
\korrekturansichttrue

\input{../tex-inputs/latex-vorspann}


               \section[Arthur Schnitzler an Hermann Bahr, 14. 12. 1904]{ Arthur Schnitzler an Hermann Bahr, 14. 12. 1904}\nopagebreak\mylabel{v}\rehead{ }\normalsize\beginnumbering\briefempfaengerindex{Bahr, Hermann@\textsc{Bahr, Hermann}!zzzSchnitzler, Arthur@\emph{von Arthur Schnitzler}!1904-12-141@{14. 12. 1904}|(be} \toendnotes[C]{\smallbreak\pagebreak[2]} \Standort{TMW, HS AM 23368 Ba.}
\physDesc{Brief, 2 Blätter, 7 Seiten
\newline{}Handschrift: schwarze Tinte, deutsche Kurrent\newline{}Ordnung: Lochung }\buchAbdrucke{\weitereDrucke{1) Arthur Schnitzler: \emph{Briefe 1875–1912}. Hg. Therese Nickl und Heinrich Schnitzler. Frankfurt am Main: \emph{S. Fischer} 1981, S. 504–506.} \weitereDrucke{2) \emph{14. 12. 1904.} In: Arthur Schnitzler: \emph{The Letters of Arthur Schnitzler to Hermann Bahr}. Edited, annotated, and with an introduction, by Donald G.
                        Daviau. Chapel Hill: \emph{The University of North Carolina Press} 1978, S. 86–87 (University of North Carolina studies in the Germanic languages
                        and literatures, 89).} \weitereDrucke{3) Hermann Bahr, Arthur Schnitzler: \emph{Briefwechsel, Aufzeichnungen, Dokumente (1891–1931)}. Hg. Kurt Ifkovits und Martin Anton Müller. Göttingen: \emph{Wallstein} 2018, S. 333–334.} }\toendnotes[C]{\smallbreak}\pstart
           \raggedleft{}{\pb}\textcolor{pink}{Wien}{}\ledrightnote{\textcolor{pink}{Wien}}{ }14. 12. 904\pend
           \pstart
           mein lieber Hermann, es beſchämt mich faſt, daſs du über ein im
               Ganzen doch ziemlich unbeträchtliches Ding wie es der \textcolor{green}{Puppenſpieler}{}\ledrightnote{\textcolor{green}{Der Puppenspieler}} iſt (er gehörte \label{K_L01477_1v}\edtext{in den Cyclus \textcolor{green}{Lebendg Stunden}{}\ledrightnote{\textcolor{green}{Lebendige Stunden. Vier Einakter}}}{\lemma{\textnormal{\emph{in … Stunden}}}\Cendnote{\textnormal{vgl. Arthur Schnitzler an Hermann Bahr, 18. 10. 1901}}}\label{K_L01477_1h}, aber wegen zu großer Länge des Abends mußte er
                  zurückge{[}ſe{]}tzt werden) – ſo ſchöne Worte ſagſt. Vielleicht
               drücke ich mich beſſer aus, we{\geminationn} ich ſage: \uline{anläßlich} des \textcolor{green}{Puppenſpielers}{}\ledrightnote{\textcolor{green}{Der Puppenspieler}}. Denn deiner Auffaſſung des kleinen Stücks muſs ich
               widerſprechen. Vielleicht hab ich nicht das Recht dazu, denn es werden ja doch
               wahrſcheinlich künſt{\pb}leriſche Mängel der Sache ſchuld daran ſein, daſs du eine Lebensanſchauung darin
               findeſt, die ich nicht hineinlegen wollte und die mir perſönlich fremd iſt. Ebenſo
               verhält es ſich mit dem \textcolor{green}{Einſ. Weg}{}\ledrightnote{\textcolor{green}{Der einsame Weg. Schauspiel in fünf Akten}}. Ich ſtehe ſo
               wenig auf Seite des Oboëſpielers, als \introOben{}ich\introOben{} auf Seiten des \textcolor{green}{Profeſſor Wegrath}{}\ledrightnote{→\textcolor{green}{Der einsame Weg. Schauspiel in fünf Akten}} geſtanden habe –
               freilich auch nicht auf der des Julian und des Puppenſpielers. Aber warum? Weil ſie
               eben nicht ganze Kerle ſind, \introOben{}keine Leute\introOben{} die – nach der dir
               bekannten Anekdote von der alten \textcolor{blue}{Streitmann}{}\ledrightnote{\textcolor{blue}{Katharina Streitmann}} –
                  »\label{K_L01477_2v}\edtext{brav genug}{\lemma{\textnormal{\emph{brav genug}}}\Cendnote{\textnormal{\emph{\textcolor{brown}{Berliner Tageblatt}}, Jg. 54, Nr. 227, 14. 5. 1925, Abend-Blatt, S. 2: »\textcolor{blue}{Arthur Schnitzler} unterhält sich mit einem
                     Freund über Leutnants \textcolor{blue}{Bilse}s Schlüsselroman
                        ›\textcolor{green}{Aus einer kleinen Garnison}‹, und es
                     entsteht die Frage, inwieweit ein Autor ein Recht habe, wirkliche Vorkommnisse
                     und Namen in ein Werk aufzunehmen, ›Die Frage‹, sagt \textcolor{blue}{Schnitzler}, ›erinnert mich an eine reizende Episode ans
                     dem Leben des Tenors \textcolor{blue}{Streitmann}; der war
                     nämlich schon ein berühmter Operettenheld, ohne daß ihn seine auf dem Land
                     lebende Mutter je auf den Brettern gesehen hatte. Eines Tages fährt sie nach
                        \textcolor{pink}{Wien}, begibt sich – auf dem Zettel steht die
                        ›\textcolor{green}{Fledermaus}‹ – ins Theater, wo ihr Sohn
                     auftritt. ›Nun?‹ fragt am Ende der Vorstellung \textcolor{blue}{Streitmann}{ }seine \textcolor{blue}{Mutter}, ›wie habe ich dir gefallen?‹ – ›Sehr gut,
                     sehr brav, mein Kind – aber‹, und sie wird bedrückt, ›warum hast du nicht das
                     schöne Lied gesungen: ›\textcolor{green}{Ach, ich
                        hab’ sie ja nur auf die Schulter geküßt}?‹ – ›Aber Mama,‹ sagte der
                     Tenor, ›das kommt ja gar nicht in dieser Operette vor.‹ – ›Schön, kommt nicht
                     vor {\dots} aber warum hast du’s nicht doch gesungen?‹ –
                     ›Aber Mama, verstehst du nicht – ich hätt’ es ja gar nicht singen dürfen.‹
                     Darauf ein langer, mißtrauischer Blick der \textcolor{blue}{Mutter}: ›Wenn man brav ist, mein Kind, darf man
                     alles.‹ ›Das ist‹, fügt \textcolor{blue}{Schnitzler} hinzu,
                     ›auch meine Meinung über den Schlüsselroman.‹«}}}\label{K_L01477_2h}« ſind – um alles
               zu dürfen. Wäre der \textcolor{green}{Puppen{\pb}ſpieler}{}\ledrightnote{→\textcolor{green}{Der Puppenspieler}}
               wirklich ein »Großer«, ſo bräuchte er ſich nicht in Lügen einzuſpinnen, um der
               größere zu bleiben – wäre \textcolor{green}{Julian}{}\ledrightnote{→\textcolor{green}{Der einsame Weg. Schauspiel in fünf Akten}}
               wirklich ein Großer – ſo würde das beſte ſeines Weſens nicht mit seiner Jugend
               auslöſchen. Gegen die Herzöge und gegen die \textsc{\textcolor{green}{Sala}{}\ledrightnote{→\textcolor{green}{Der einsame Weg. Schauspiel in fünf Akten}}}’s hab ich nichts – und vor den »Großen Räubern« ſalutir ich, gleich dir, in
               Ehrfurcht. Du haſt ganz recht: »Entſagung iſt nicht immer Reife.« – – nur ſetze ich
               hinzu: nicht bei allen. Wenn Individuen wie \textcolor{green}{\uline{Wegrath}}{}\ledrightnote{→\textcolor{green}{Der einsame Weg. Schauspiel in fünf Akten}} in irgend einem Moment ihrer Exiſtenz die Grenzen ihrer Begabung erkennen, –
                  {\pb}so iſt \uline{dieſe} Entſagung, wie jede \uline{Erke{\geminationn}tnis} innere Reife, oder wenigſtens ein
               Symptom innerer Reife. Ebenſo iſt für den Oboëſpieler wirklich der »Innere Friede und
               die ſchuldbefreite Bruſt« das einzig erreichbare Glück. Und daſs ein Menſch wie der
                  »\textcolor{green}{Puppenſpieler}{}\ledrightnote{→\textcolor{green}{Der Puppenspieler}}« nicht, wie es
               eben den Beſchränkungen ſeines Weſens angemeſſen wäre, \introOben{}zu\introOben{}
               entſagen im Stande iſt, ſich \introOben{}vielmehr\introOben{} dieſer Entſagung
                  \label{T_L01477_1v}\edtext{ſchämten}{\lemma{\textnormal{\emph{ſchämten}}}\Cendnote{\textnormal{Schreibfehler, das Wort ist deutlich zu entziffern}}}\label{T_L01477_1h} würde
               und daher den andern u ſich ein \introOben{}falſches\introOben{} Eigenſchickſal
               vorſpielt – iſt ein Zeichen, daſs er innere Reife nicht erlangte, welche eben nur in
               Selbſterkenntnis beſtehen kann. \strikeout{Daher} Es iſt {\pb}alſo nur natürlich,
               daſs bei manchen Menſchen, insbeſondre bei klugen, von mäßigem Talente und ſtillem
               Temperamente das was ihnen an innerer Reife überhaupt beſchieden iſt, in einer Art
               von »Entſagung« den entſprechenden Ausdruck findet.\pend
           \pstart
           Wohl denen, die’s nicht nöthig haben, – wohl uns, die wir wie mir ſcheint zu dieſen
               gehören – und hoffentlich nicht allein wegen Mangels an Klugheit. So ſpricht alſo
               nichts dagegen, mein lieber Hermann, daſs wir beide uns an die Arbeit machen, die du
               in meine {\pb}Hände legſt:
                  »\label{LL075-1v}Das Werk von der letzten Nacht einer alten
                  Zeit\label{LL075-1h}« – Und ſchließlich können es auch andre Werke ſein.\pend
           \pstart
           Zu »\label{K_L01477_3v}\edtext{\textcolor{green}{Mahler}{}\ledrightnote{→\textcolor{green}{Symphonie Nr. 3 D-Moll}}« haben wir noch Sitze}{\lemma{\textnormal{\emph{Mahler« … Sitze}}}\Cendnote{\textnormal{\textcolor{blue}{Mahler} dirigierte seine \emph{\textcolor{green}{3. Symphonie}} im \textcolor{pink}{Musikvereinssaal}.}}}\label{K_L01477_3h}{ }\damage{be}kommen, ſo ſeh ich dich hoffentlich auch heute Abend.\pend
           \pstart
           Jedenfalls aber sage oder schreibe mir pneumatiſch, ob du vielleicht Lust hätteſt, am
                  \uline{Samſtag} bei uns zu nachtmahlen.\pend
           \pstart
           Herzlichst der deine{\\[\baselineskip]}\spacefill\mbox{Arthur{\pb}}\pend
           \leftskip=0em{}\pstart
           \noindent{}\textcolor{blue}{Olga}{}\ledrightnote{\textcolor{blue}{Olga Schnitzler}} grüßt dich herzlich und ſagt dir, daſs sie \substVorne{}\textsuperscript{das}\substDazwischen{}von dem\substHinten{} was du anläßlich de\textcolor{gray}{s \textsc{P}.}
                  geſchrieben haſt, erſchüttert war.\pend
           \endnumbering\briefempfaengerindex{Bahr, Hermann@\textsc{Bahr, Hermann}!zzzSchnitzler, Arthur@\emph{von Arthur Schnitzler}!1904-12-141@{14. 12. 1904}|)be}\mylabel{h}  \normalsize

\doendnotes{C}
\bigskip
\vfill

\clearpage

\footnotesize

\lohead{\textsc{register}}

% Definiere theindex-Environment komplett neu ohne reledmac
\makeatletter
\renewenvironment{theindex}{%
  \section*{\indexname}%
  \setlength{\parindent}{0pt}%
  \setlength{\parskip}{0pt plus 0.3pt}%
  \let\item\@idxitem
}{%
  \clearpage
}
\makeatother

\IfFileExists{\jobname-pw.ind}{\input{\jobname-pw.ind}}{}

\end{document}

      