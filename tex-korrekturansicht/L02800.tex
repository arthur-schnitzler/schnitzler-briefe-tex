%% latex-korrekturansicht-vorspann.tex
%% Vorspann für die Korrekturansicht.
%% Lädt die gemeinsame Datei latex-vorspann.tex mit gesetztem Schalter.

\newif\ifkorrekturansicht
\korrekturansichttrue

\input{../tex-inputs/latex-vorspann}


               \section[Paul Goldmann an Arthur Schnitzler, Paul Goldmann an Arthur Schnitzler, 8. 1. {[}1897{]}]{ Paul Goldmann an Arthur Schnitzler, 8. 1. {[}1897{]}}\nopagebreak\mylabel{v}\rehead{ }\normalsize\beginnumbering\briefempfaengerindex{Schnitzler, Arthur@\textsc{Schnitzler, Arthur}!zzzGoldmann, Paul@\emph{von Paul Goldmann}!1897-01-081@{8. 1. {[}1897{]}}|(be} \toendnotes[C]{\smallbreak\pagebreak[2]} \Standort{DLA, A:Schnitzler, HS.NZ85.1.3167.}
\physDesc{Brief, 1 Blatt, 3 Seiten
\newline{}Handschrift: blaue Tinte, deutsche Kurrent
\newline{}Schnitzler: 1) mit Bleistift das Jahr »97« vermerkt 2) mit rotem Buntstift zwei Unterstreichungen}\toendnotes[C]{\smallbreak}\pstart
           \noindent{}{\pb}\textcolor{gray}{\textbf{\textbf{\textcolor{brown}{Frankfurter Zeitung}{}\ledrightnote{\textcolor{brown}{Frankfurter Zeitung}}}}}\pend
           \pstart
           \textcolor{gray}{\textbf{(\textcolor{brown}{\begin{otherlanguage}{french}Gazette de Francfort\end{otherlanguage}}{}\ledrightnote{\textcolor{brown}{Frankfurter Zeitung}}).}}\pend
           \pstart
           \textcolor{gray}{\textbf{\textbf{\begin{otherlanguage}{french}Fondateur M.\end{otherlanguage}{ }\textcolor{blue}{L. Sonnemann}{}\ledrightnote{\textcolor{blue}{Leopold Sonnemann}}.}}}\pend
           \pstart
           \begin{otherlanguage}{french}\textcolor{gray}{\textbf{Journal politique, financier,}}\end{otherlanguage}\pend
           \pstart
           \begin{otherlanguage}{french}\textcolor{gray}{\textbf{commercial et littéraire.}}\end{otherlanguage}\pend
           \pstart
           \begin{otherlanguage}{french}\textcolor{gray}{\textbf{\textbf{Paraissant trois fois par jour.}}}\end{otherlanguage}\hfill \textsc{\textcolor{pink}{Paris}{}\ledrightnote{\textcolor{pink}{Paris}}}, 8. Januar.\pend
           \pstart
           \begin{otherlanguage}{french}\textcolor{gray}{\textbf{\textbf{Bureau à \textcolor{pink}{Paris}{}\ledrightnote{\textcolor{pink}{Paris}}}}}\end{otherlanguage}\pend
           \pstart
           \begin{otherlanguage}{french}\textcolor{gray}{\textbf{\textbf{\textcolor{pink}{24. Rue Feydeau}{}\ledrightnote{\textcolor{pink}{rue Feydeau}}.}}}\end{otherlanguage}\pend
           \pstart\center{}Mein lieber Freund,\pend\pstart
           da ich nicht weiß, ob Du nicht \label{K_L02800-1v}\edtext{beifolgende \textcolor{green}{Notiz}{}\ledrightnote{→\textcolor{green}{[Notiz]}}}{\lemma{\textnormal{\emph{beifolgende Notiz}}}\Cendnote{\textnormal{Beilage nicht erhalten. Es handelte sich
                  um folgende Notiz: XXXX.}}}\label{K_L02800-1h} in der \textcolor{green}{Frankfurter Zeitung}{}\ledrightnote{\textcolor{green}{Frankfurter Zeitung}} überſehen haſt, ſchicke ich ſie Dir der
               Sicherheit halber. Sie iſt natürlich von mir geſchrieben; aber da \textsc{\textcolor{blue}{Bahr}{}\ledrightnote{\textcolor{blue}{Hermann Bahr}}} an eine Vereinbarung zwiſchen Dir und mir glauben würde und ſich wahrſcheinlich
               an Dir bei der erſten Gelegenheit rächen würde, halte ich es für beſſer, ihm
               einſtweilen nichts von meiner Autorſchaft zu ſagen. \strikeout{\textcolor{gray}{E}}{ }{\pb}Einmal mußte man doch gegen den Schwindel\strikeout{s} proteſtiren, den der Kerl treibt.\pend
           \pstart
           Von \textsc{\textcolor{blue}{Brandes}{}\ledrightnote{\textcolor{blue}{Georg Brandes}}} erhielt ich dieſer Tage einen \label{K_L02800-2v}\edtext{Brief}{\lemma{\textnormal{\emph{Brief}}}\Cendnote{\textnormal{\textcolor{blue}{Schnitzler} reagierte, indem er \textcolor{blue}{Brandes} am 11. 1. 1897 einen freundlichen Brief schrieb und ihm
                     \emph{\textcolor{green}{Freiwild}} (noch als Manuskript) zukommen
                  ließ.}}}\label{K_L02800-2h}, den ich Dir ſchicken werde, ſobald ich ihn beantwortet habe. Er
               ſchreibt unter Anderem:\pend
           \pstart
           »\begin{otherlanguage}{french}\textsc{À propos}\end{otherlanguage}, meinem Verſprechen getreu ſandte ich an Herrn \textsc{\textcolor{blue}{Hofmann-Beer}{}\ledrightnote{\textcolor{blue}{Richard Beer-Hofmann}}} meine \strikeout{\textcolor{gray}{nn}} neue Sammlung \label{K_L02800-21v}\edtext{\textsc{\textcolor{green}{Essais}{}\ledrightnote{→\textcolor{green}{Menschen und Werke. Essays}}}}{\lemma{\textnormal{\emph{Essais}}}\Cendnote{\textnormal{\textcolor{blue}{Georg Brandes}: \emph{\textcolor{green}{Menschen und Werke. Essays}}. \textcolor{pink}{Frankfurt am Main}: \emph{\textcolor{brown}{Literarische Anstalt Rütten {\kaufmannsund}
                        Loening}} 1894. Am 14. 1. 1897 schrieb \textcolor{blue}{Beer-Hofmann}
                  an \textcolor{blue}{Brandes}, unter anderem Folgendes: »\textcolor{blue}{Arthur} und ich sprechen
                  oft von Ihnen, und in den Briefen von \textcolor{blue}{Paul Goldmann} kehrt Ihr Nahmen immer
                  wieder. Besonders freut es mich, dass Sie und \textcolor{blue}{Paul} einander manchmal
                     schreiben. Er ist ein Mensch von Klugheit und Güte.–« (\textcolor{blue}{Richard Beer-Hofmann}:
                      \emph{Briefe.
                     1895–1945}. Hg. u. kommentiert v. Alexander Košenina.
                     Oldenburg: \emph{Igel}{ }1999, S. 9–10 (\emph{Große Richard Beer-Hofmann-Ausgabe in sechs Bänden}.
                     Hg. v. Günter Helmes, Michael M. Schardt und Andreas Thomasberger, 7 / Erster
                     Supplementband).}}}\label{K_L02800-21h}, er hat mir aber mit
               keiner Silbe geantwortet. Auch \textsc{Schnitzler} vergißt mich,
               ſandte mir nicht ſein \textcolor{green}{Schauſpiel}{}\ledrightnote{→\textcolor{green}{Freiwild. Schauspiel in 3 Akten}}.«\pend
           \pstart
           {\pb}Du wirſt dem \textcolor{blue}{Manne}{}\ledrightnote{→\textcolor{blue}{Georg Brandes}} gewiß raſch ſchreiben. Aber auch \textsc{\textcolor{blue}{Richard}{}\ledrightnote{\textcolor{blue}{Richard Beer-Hofmann}}} ſollte ihm antworten. Das Nicht-Schreiben iſt ein Verfahren, das ſich nur im
               Verkehr mit Freunden bewährt, das aber ſeine Unzuträglichkeiten hat, wenn man es auch
               gegenüber Fremden \strikeout{an\textcolor{gray}{w}} anwendet.\pend
           \pstart
           Viele herzliche Grüße an Dich und \textsc{\textcolor{blue}{Richard}{}\ledrightnote{\textcolor{blue}{Richard Beer-Hofmann}}}!\pend
           \pstart
           Dein treuer {\\[\baselineskip]}\spacefill\mbox{Paul Goldmann.}\pend
           \leftskip=0em{}\endnumbering\briefempfaengerindex{Schnitzler, Arthur@\textsc{Schnitzler, Arthur}!zzzGoldmann, Paul@\emph{von Paul Goldmann}!1897-01-081@{8. 1. {[}1897{]}}|)be}\mylabel{h}  \normalsize

\doendnotes{C}
\bigskip
\vfill

\clearpage

\footnotesize

\lohead{\textsc{register}}

% Definiere theindex-Environment komplett neu ohne reledmac
\makeatletter
\renewenvironment{theindex}{%
  \section*{\indexname}%
  \setlength{\parindent}{0pt}%
  \setlength{\parskip}{0pt plus 0.3pt}%
  \let\item\@idxitem
}{%
  \clearpage
}
\makeatother

\IfFileExists{\jobname-pw.ind}{\input{\jobname-pw.ind}}{}

\end{document}

      