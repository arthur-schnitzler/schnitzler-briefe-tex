%% latex-korrekturansicht-vorspann.tex
%% Vorspann für die Korrekturansicht.
%% Lädt die gemeinsame Datei latex-vorspann.tex mit gesetztem Schalter.

\newif\ifkorrekturansicht
\korrekturansichttrue

\input{../tex-inputs/latex-vorspann}


\renewcommand{\erwaehntePersonen}{Personen:  ?? [Arzt von Ottilie Salten], Hermann Bahr, Anna Bahr-Mildenburg, Hugo Ganz, Robert Hirschfeld, Eduard Pötzl, Felix Salten, Ottilie Salten, Paul Salten, Olga Schnitzler, Gustav Schwarzkopf, Adalbert Franz Seligmann, Friedrich Viktor Spitzer}
\renewcommand{\erwaehnteOrte}{Orte: Armbrustergasse, Athen, Heiligenstadt, Hotel Quisisana, Hotel Waldmühle, I., Innere Stadt, Karlsbad, Lido, Marienbad, Oper, Pustertal, Südtirol, Welsberg-Taisten, Wien, Wildbad Waldbrunn, XIX., Döbling}
\renewcommand{\erwaehnteWerke}{Werke: Das gelobte Wien, Der Wiener Korrespondent, Morgen. Wochenschrift für deutsche Kultur, Neues Wiener Tagblatt, Wien. Mit acht Vollbildern}
\section[ Felix und Ottilie Salten an Arthur Schnitzler, 3. 8. 1907]{Felix und Ottilie Salten an Arthur Schnitzler, 3. 8. 1907}
\nopagebreak\mylabel{v}
\rehead{ }\normalsize\beginnumbering\briefempfaengerindex{Schnitzler, Arthur@\textsc{Schnitzler, Arthur}!zzzSalten, Ottilie@\emph{von Ottilie Salten}!1907-08-032@{3. 8. 1907}|(be}\briefempfaengerindex{Schnitzler, Arthur@\textsc{Schnitzler, Arthur}!zzzSalten, Felix@\emph{von Felix Salten}!1907-08-032@{3. 8. 1907}|(be}
\toendnotes[C]{\smallbreak\pagebreak[2]}\Standort{CUL, Schnitzler, B 89, B 1.}
\physDesc{Postkarte, 2006 Zeichen
\newline{}Handschrift: schwarze Tinte, lateinische Kurrent
\newline{}Versand: 1) Stempel: »\nobreak{}\oindex{I., Innere Stadt@\textbf{I., Innere Stadt}, \emph{A.ADM3}|pwk}1/1 Wien 13, 3. VIII. 07, 6\nobreak{}«. Stempel: »\nobreak{}\oindex{Welsberg-Taisten@\textbf{Welsberg-Taisten}, \emph{A.ADM3}|pwk}We{[}lsber{]}g, 4. 8. \textcolor{gray}{07}\nobreak{}«.   2) mit Bleistift beschriftet: »III 9–\substVorne{}\textsuperscript{\textcolor{gray}{11}}\substDazwischen{}4\substHinten{}«
\newline{}Schnitzler: mit Bleistift sechs Unterstreichungen 
\newline{}Ordnung: mit Bleistift von unbekannter Hand nummeriert: »232« }
\buchAbdrucke{\weitereDrucke{Hermann Bahr, Arthur Schnitzler: \emph{Briefwechsel, Aufzeichnungen, Dokumente (1891–1931)}. Hg. Kurt Ifkovits und Martin Anton Müller. Göttingen: \emph{Wallstein} 2018, S. 394–395.} }\toendnotes[C]{\smallbreak}\pstart{}{\pb}Salten\pend{}\pstart{}\textcolor{pink}{Wien XIX.}{}\ledrightnote{\textcolor{pink}{XIX., Döbling}}\pend{}\pstart{}\textcolor{pink}{Armbrustergasse 6}{}\ledrightnote{\textcolor{pink}{Armbrustergasse}}\pend{}
{\bigskip}\pstart{}Herrn D\textsuperscript{r} Arthur Schnitzler\pend{}\pstart{}\textcolor{pink}{Wildbald Waldbrunn bei/ Welsberg}{}\ledrightnote{\textcolor{pink}{Wildbad Waldbrunn}}\pend{}\pstart{}i 
                  \textcolor{pink}{Pustertal}{}\ledrightnote{\textcolor{pink}{Pustertal}}\pend{}
{\bigskip}
\pstart
           \raggedleft{}{\pb}\textcolor{pink}{Heiligenstadt}{}\ledrightnote{\textcolor{pink}{Heiligenstadt}}, 3. VIII. 07\pend
           
\pstart
           Lieber, ich habe Ihre letzte Karte nicht gut lesen können, glaube
               aber dass Sie noch in \label{K_L03489-1v}\edtext{\textcolor{pink}{Waldbrunn}{}\ledrightnote{\textcolor{pink}{Wildbad Waldbrunn}}}{\lemma{\textnormal{\emph{Waldbrunn}}}\Cendnote{\textnormal{siehe Felix Salten an Arthur Schnitzler, 15. 7. 1907}}}\label{K_L03489-1h} sind. Uns ist es nicht besonders gegangen. \textcolor{blue}{Otti}{}\ledrightnote{\textcolor{blue}{Ottilie Salten}} mußte operirt werden, was zu Hause geschah. Sie hat sich bis heute noch nicht völlig erholt. Der \textcolor{blue}{Arzt}{}\ledrightnote{{$\rightarrow$}\textcolor{blue}{?? [Arzt von Ottilie Salten]}} will, dass sie jetzt noch eine Kur
               brauchen soll. So gehen wir nächster Tage auf 4 Wochen nach \textcolor{pink}{Marienbad}{}\ledrightnote{\textcolor{pink}{Marienbad}}. Ich komme eben von dort, wo ich Wohnung genommen
               habe. Vorher war ich ein paar Tage in \textcolor{pink}{Karlsbad}{}\ledrightnote{\textcolor{pink}{Karlsbad}}.
               Unsere Adreße ist dann (wahrscheinlich vom 8\textsuperscript{\textcolor{gray}{ten}} an) »\textcolor{pink}{Quisiana}{}\ledrightnote{\textcolor{pink}{Hotel Quisisana}}«.
               Ein sehr hübsches Haus, oben im Wald bei der \textcolor{pink}{Waldmühle}{}\ledrightnote{\textcolor{pink}{Hotel Waldmühle}}. \textcolor{blue}{Paul}{}\ledrightnote{\textcolor{blue}{Paul Salten}} ist dieser Tage auch
               wieder krank gewesen, hoffentlich wird er sich in \textcolor{pink}{Marienbad}{}\ledrightnote{\textcolor{pink}{Marienbad}} vollständig erholen. Wann kommen Sie \label{K_L03489-2v}\edtext{nach \textcolor{pink}{Wien}{}\ledrightnote{\textcolor{pink}{Wien}}}{\lemma{\textnormal{\emph{nach Wien}}}\Cendnote{\textnormal{\textcolor{blue}{Schnitzler} kehrte am 12. 9. 1907 nach \textcolor{pink}{Wien} zurück.}}}\label{K_L03489-2h} zurück? Spielen Sie \textcolor{pink}{dort}{}\ledrightnote{{$\rightarrow$}\textcolor{pink}{Welsberg-Taisten}}{ }\label{K_L03489-3v}\edtext{Tennis}{\lemma{\textnormal{\emph{Tennis}}}\Cendnote{\textnormal{siehe A. S.: \emph{Tagebuch}, 3. 8. 1907 und 5. 8. 1907}}}\label{K_L03489-3h}? Haben Sie gearbeitet? Haben Sie für den \label{K_L03489-4v}\edtext{September Reisepläne}{\lemma{\textnormal{\emph{September Reisepläne}}}\Cendnote{\textnormal{\textcolor{blue}{Arthur} und \textcolor{blue}{Olga Schnitzler} reisten am 26. 8. 1907 von \textcolor{pink}{Welsberg} weiter \textcolor{pink}{Südtirol}.}}}\label{K_L03489-4h}? Ich
               möchte im September irgend eine Meerfahrt machen. \textcolor{pink}{Athen}{}\ledrightnote{\textcolor{pink}{Athen}} oder so was ähnliches. \textcolor{blue}{Bahr}{}\ledrightnote{\textcolor{blue}{Hermann Bahr}} hat mir vom \textcolor{pink}{Lido}{}\ledrightnote{\textcolor{pink}{Lido}}
               einen entrüsteten Brief geschrieben, weil mich der \label{K_L03489-5v}\edtext{\textcolor{blue}{Pötzl}{}\ledrightnote{\textcolor{blue}{Eduard Pötzl}} im \textcolor{green}{Tagblatt}{}\ledrightnote{\textcolor{green}{Neues Wiener Tagblatt}}{ }\textcolor{green}{gelobt}{}\ledrightnote{{$\rightarrow$}\textcolor{green}{Das gelobte Wien}}}{\lemma{\textnormal{\emph{Pötzl im Tagblatt gelobt}}}\Cendnote{\textnormal{\textcolor{blue}{Ed. [ = Eduard] Pötzl}: \emph{\textcolor{green}{Das gelobte Wien}}. In: \emph{\textcolor{green}{Neues Wiener Tagblatt}}, Jg. 41, Nr. 204, 28. 7. 1907, S. 1–3.}}}\label{K_L03489-5h} hat. Und der \textcolor{blue}{Pötzl}{}\ledrightnote{\textcolor{blue}{Eduard Pötzl}} hat mich \textcolor{green}{gelobt}{}\ledrightnote{{$\rightarrow$}\textcolor{green}{Das gelobte Wien}}, weil ich im »\textcolor{green}{Morgen}{}\ledrightnote{\textcolor{green}{Morgen. Wochenschrift für deutsche Kultur}}« \label{K_L03489-6v}\edtext{\textcolor{green}{Wien}{}\ledrightnote{\textcolor{green}{Wien. Mit acht Vollbildern}}{ }\textcolor{green}{gelobt}{}\ledrightnote{{$\rightarrow$}\textcolor{green}{Der Wiener Korrespondent}}}{\lemma{\textnormal{\emph{Wien gelobt}}}\Cendnote{\textnormal{Das Lob für \textcolor{blue}{Bahr}s Abrechnungsbuch \emph{\textcolor{green}{Wien}} findet sich nur implizit in \textcolor{blue}{Felix Salten}: \emph{\textcolor{green}{Der Wiener Korrespondent}}. In: \emph{\textcolor{green}{Der Morgen}}, Jg. 1, H. 4, 5. 7. 1907, S. 113–116.}}}\label{K_L03489-6h} habe. Es ist eine düstere
               Sache, wie Sie sehen. Aber was soll ich thun? Ich zittere, dass mich am Ende
               nächstens auch noch der \textcolor{blue}{Seligmann}{}\ledrightnote{\textcolor{blue}{Adalbert Franz Seligmann}} lobt, oder
               der \textcolor{blue}{Hugo Ganz}{}\ledrightnote{\textcolor{blue}{Hugo Ganz}} und dann wird mich \textcolor{blue}{Bahr}{}\ledrightnote{\textcolor{blue}{Hermann Bahr}} sicherlich total verachten, und komme ich
               einmal in die \textcolor{pink}{Oper}{}\ledrightnote{\textcolor{pink}{Oper}}, wird die \label{K_L03489-7v}\edtext{\textcolor{blue}{M.}{}\ledrightnote{\textcolor{blue}{Anna Bahr-Mildenburg}}}{\lemma{\textnormal{\emph{M.}}}\Cendnote{\textnormal{\textcolor{blue}{Anna Mildenburg}, \textcolor{blue}{Hermann Bahr}s spätere Frau}}}\label{K_L03489-7h} zu singen aufhören, weil
               ich da bin. Mir fehlt zu meinem gänzlichen Untergang nur noch, dass \textcolor{blue}{Robert Hirschfeld}{}\ledrightnote{\textcolor{blue}{Robert Hirschfeld}} ein Feuilleton über mich schreibt, und dem
                  \textcolor{blue}{Gustav S-Kopf}{}\ledrightnote{\textcolor{blue}{Gustav Schwarzkopf}} in einem Aufruf die \textcolor{pink}{Wien}{}\ledrightnote{\textcolor{pink}{Wien}}er einlädt, meine Bücher fleißiger zu kaufen.
               Dann bin ich ganz kaput\textcolor{gray}{,} und kann mich von D\textsuperscript{r}{ }\textcolor{blue}{Spitzer}{}\ledrightnote{\textcolor{blue}{Friedrich Viktor Spitzer}} ehrlicher Weise nicht einmal mehr
               fotografiren laßen. Ich habe trübe Ahnungen und bin auf das Schlimmste gefaßt. Aber,
               wenn’s mir bestimmt ist, kann ich garnichts machen. – Hoffentlich geht es Ihnen allen
               gut.\pend
           
\pstart
           Leben Sie wol und schreiben Sie bald wieder eine Zeile. Herzliche Grüße von uns zu
                  \textcolor{blue}{Ihnen}{}\ledrightnote{{$\rightarrow$}\textcolor{blue}{Olga Schnitzler}}.\pend
           \pstart Ihr \spacefill\mbox{FSalten}\pend{}
\pstart
           \noindent{}{[}hs. Salten:{]} Viele herzliche Grüße \spacefill\mbox{Ottilie S.}\pend
           \endnumbering\briefempfaengerindex{Schnitzler, Arthur@\textsc{Schnitzler, Arthur}!zzzSalten, Ottilie@\emph{von Ottilie Salten}!1907-08-032@{3. 8. 1907}|)be}\briefempfaengerindex{Schnitzler, Arthur@\textsc{Schnitzler, Arthur}!zzzSalten, Felix@\emph{von Felix Salten}!1907-08-032@{3. 8. 1907}|)be}\mylabel{h}  \normalsize

\doendnotes{C}
\bigskip
\vfill

\clearpage

\footnotesize

\lohead{\textsc{register}}

% Definiere theindex-Environment komplett neu ohne reledmac
\makeatletter
\renewenvironment{theindex}{%
  \section*{\indexname}%
  \setlength{\parindent}{0pt}%
  \setlength{\parskip}{0pt plus 0.3pt}%
  \let\item\@idxitem
}{%
  \clearpage
}
\makeatother

\IfFileExists{\jobname-pw.ind}{\input{\jobname-pw.ind}}{}

\end{document}

      