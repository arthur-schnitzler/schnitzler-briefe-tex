%% latex-korrekturansicht-vorspann.tex
%% Vorspann für die Korrekturansicht.
%% Lädt die gemeinsame Datei latex-vorspann.tex mit gesetztem Schalter.

\newif\ifkorrekturansicht
\korrekturansichttrue

\input{../tex-inputs/latex-vorspann}


               \section[Hugo von Hofmannsthal an Arthur Schnitzler, {[}6. 12. 1891{]}]{ Hugo von Hofmannsthal an Arthur Schnitzler, {[}6. 12. 1891{]}}\nopagebreak\mylabel{v}\rehead{ }\normalsize\beginnumbering\briefempfaengerindex{Schnitzler, Arthur@\textsc{Schnitzler, Arthur}!zzzHofmannsthal, Hugo von@\emph{von Hugo von Hofmannsthal}!1891-12-061@{{[}6. 12. 1891{]}}|(be} \toendnotes[C]{\smallbreak\pagebreak[2]} \Standort{CUL, Schnitzler, B 43.}
\physDesc{Brief, 1 Blatt (Briefpapier mit aufgeprägtem Wappen), 1 Seite
\newline{}Handschrift: schwarze Tinte, deutsche Kurrent
\newline{}Schnitzler: mit rotem Buntstift datiert: »6/12 91« \newline{}Ordnung: mit Bleistift von unbekannter Hand nummeriert:
                                    »11« }\buchAbdrucke{\weitereDrucke{1) Hugo von Hofmannsthal, Arthur Schnitzler: \emph{Briefwechsel}. Hg. Therese Nickl und Heinrich Schnitzler. Frankfurt am Main: \emph{S. Fischer} 1964, S. 14.} \weitereDrucke{2) Hermann Bahr, Arthur Schnitzler: \emph{Briefwechsel, Aufzeichnungen, Dokumente (1891–1931)}. Hg. Kurt Ifkovits und Martin Anton Müller. Göttingen: \emph{Wallstein} 2018, S. 14.} }\toendnotes[C]{\smallbreak}\pstart
           \noindent{}{\pb}Soeben ſchickt mir \textcolor{blue}{Bahr}{}\ledrightnote{\textcolor{blue}{Hermann Bahr}} die beiliegende \label{K_L00049_1v}\edtext{Karte}{\lemma{\textnormal{\emph{Karte}}}\Cendnote{\textnormal{Es dürfte
                  sich hier um eine Karte für die Premiere von \textcolor{blue}{Gerhart Hauptmann}s \emph{\textcolor{green}{Einsame Menschen}} im
                     \textcolor{pink}{Burgtheater} am
                     6. 12. 1891 handeln. \textcolor{blue}{Hofmannsthal} war dort, \textcolor{blue}{Schnitzler}
                  nicht.}}}\label{K_L00049_1h}. Ich gehe jedenfalls hin.\pend
           \pstart
           Vielleicht erwarten Sie mich gegen \uline{4 Uhr} bei ſich und
               wir gehen dann zuſammen hin.\pend
           \pstart
           Wenn nicht, hinterlaſſen Sie mir eine Poſt.\pend
           \pstart \spacefill\mbox{Loris.}\pend{}\pstart
           \noindent{}Soll man ihm einen Arzt ſchicken?\pend
           \endnumbering\briefempfaengerindex{Schnitzler, Arthur@\textsc{Schnitzler, Arthur}!zzzHofmannsthal, Hugo von@\emph{von Hugo von Hofmannsthal}!1891-12-061@{{[}6. 12. 1891{]}}|)be}\mylabel{h}  \normalsize

\doendnotes{C}
\bigskip
\vfill

\clearpage

\footnotesize

\lohead{\textsc{register}}

% Definiere theindex-Environment komplett neu ohne reledmac
\makeatletter
\renewenvironment{theindex}{%
  \section*{\indexname}%
  \setlength{\parindent}{0pt}%
  \setlength{\parskip}{0pt plus 0.3pt}%
  \let\item\@idxitem
}{%
  \clearpage
}
\makeatother

\IfFileExists{\jobname-pw.ind}{\input{\jobname-pw.ind}}{}

\end{document}

      