%% latex-korrekturansicht-vorspann.tex
%% Vorspann für die Korrekturansicht.
%% Lädt die gemeinsame Datei latex-vorspann.tex mit gesetztem Schalter.

\newif\ifkorrekturansicht
\korrekturansichttrue

\input{../tex-inputs/latex-vorspann}


\renewcommand{\erwaehntePersonen}{Personen: Hans Rehmann, Anna Katharina Rehmann, Felix Salten, Johann Strauss}
\renewcommand{\erwaehnteOrte}{Orte: Berlin, Grossmünster, Stadttheater Zürich, Sternwartestraße 71, Wasserkirche, Wien, Zürich}
\renewcommand{\erwaehnteWerke}{}
\section[ Felix Salten an Arthur Schnitzler, 6. 11. 1929]{Felix Salten an Arthur Schnitzler, 6. 11. 1929}
\nopagebreak\mylabel{v}
\rehead{ }\normalsize\beginnumbering\briefempfaengerindex{Schnitzler, Arthur@\textsc{Schnitzler, Arthur}!zzzSalten, Felix@\emph{von Felix Salten}!1929-11-061@{6. 11. 1929}|(be}
\toendnotes[C]{\smallbreak\pagebreak[2]}\Standort{CUL, Schnitzler, B 89, B 2.}
\physDesc{Bildpostkarte, 354 Zeichen
\newline{}Handschrift: schwarze Tinte, lateinische Kurrent
\newline{}Versand: Stempel: »\nobreak{}\oindex{Zuerich@\textbf{Zürich}, \emph{P.PPLA}|pwk}Zürich 1, 6 · IX 929, 21–22, Briefversand\nobreak{}«.  
\newline{}Schnitzler: mit Bleistift datiert: »6/11 92\textcolor{gray}{9}« und zwei Unterstreichungen 
\newline{}Ordnung: mit Bleistift von unbekannter Hand nummeriert: »301« }\toendnotes[C]{\smallbreak}\pstart{}{\pb}Herrn D\textsuperscript{r} Arthur Schnitzler\pend{}\pstart{}\textcolor{pink}{Wien}{}\ledrightnote{\textcolor{pink}{Wien}}\pend{}\pstart{}\textcolor{pink}{XVIII. Sternwartestrasse 71}{}\ledrightnote{\textcolor{pink}{Sternwartestraße 71}}\pend{}
{\bigskip}
\pstart
           \noindent{}\centering{}{\pb}\textcolor{gray}{\textbf{\textcolor{pink}{Zürich}{}\ledrightnote{\textcolor{pink}{Zürich}}. \textcolor{pink}{Großmünster}{}\ledrightnote{\textcolor{pink}{Grossmünster}} und \textcolor{pink}{Wasserkirche}{}\ledrightnote{\textcolor{pink}{Wasserkirche}}}}\pend
           
\pstart{}{\pb}Lieber,\pend
\pstart
           \textcolor{pink}{Berlin}{}\ledrightnote{\textcolor{pink}{Berlin}} war diesmal sehr angenehm. Denn \label{K_L03588-1v}\edtext{\textcolor{blue}{Hans Rehmann}{}\ledrightnote{\textcolor{blue}{Hans Rehmann}} gefiel mir ungemein}{\lemma{\textnormal{\emph{Hans … ungemein}}}\Cendnote{\textnormal{\textcolor{blue}{Hans Rehmann} war der zukunftige Ehemann der
                  Tochter \textcolor{blue}{Anna Katharina Salten}.}}}\label{K_L03588-1h} und
               wir verstanden einander bald. Ich glaube, er ist ein wirklicher Mensch und bin
               natürlich froh! Hier muss ich bis Sonntag bleiben, um
               die \label{K_L03588-2v}\edtext{\textcolor{blue}{Johann-Strauss}{}\ledrightnote{\textcolor{blue}{Johann Strauss}}-Rede am Samstag zu wiederholen}{\lemma{\textnormal{\emph{Johann-Strauss-Rede … wiederholen}}}\Cendnote{\textnormal{Bereits
                  am 4. 11. 1929 hatte \textcolor{blue}{Salten} im \textcolor{pink}{Stadttheater}
                  eine Gedenkrede für \textcolor{blue}{Johann-Strauss}
                  gehalten. Am 9. 11. 1929 wurde die Veranstaltung
                  wiederholt.}}}\label{K_L03588-2h}.\pend
           
\pstart
           Herzlichst {\\[\baselineskip]}Ihr {\\[\baselineskip]}\spacefill\mbox{Felix Salten}\pend
           \leftskip=0em{}
\pstart
           \textcolor{pink}{Zürich}{}\ledrightnote{\textcolor{pink}{Zürich}}{ }\textcolor{gray}{6}. XI. 29\pend
           \endnumbering\briefempfaengerindex{Schnitzler, Arthur@\textsc{Schnitzler, Arthur}!zzzSalten, Felix@\emph{von Felix Salten}!1929-11-061@{6. 11. 1929}|)be}\mylabel{h}  \normalsize

\doendnotes{C}
\bigskip
\vfill

\clearpage

\footnotesize

\lohead{\textsc{register}}

% Definiere theindex-Environment komplett neu ohne reledmac
\makeatletter
\renewenvironment{theindex}{%
  \section*{\indexname}%
  \setlength{\parindent}{0pt}%
  \setlength{\parskip}{0pt plus 0.3pt}%
  \let\item\@idxitem
}{%
  \clearpage
}
\makeatother

\IfFileExists{\jobname-pw.ind}{\input{\jobname-pw.ind}}{}

\end{document}

      