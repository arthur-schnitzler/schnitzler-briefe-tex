%% latex-korrekturansicht-vorspann.tex
%% Vorspann für die Korrekturansicht.
%% Lädt die gemeinsame Datei latex-vorspann.tex mit gesetztem Schalter.

\newif\ifkorrekturansicht
\korrekturansichttrue

\input{../tex-inputs/latex-vorspann}


\renewcommand{\erwaehntePersonen}{Personen: Stefan Zweig}
\renewcommand{\erwaehnteOrte}{Orte: Salzburg, Sternwartestraße 71, Wien, Währinger Cottage, Zermatt}
\renewcommand{\erwaehnteWerke}{}
\section[Stefan Zweig an Arthur Schnitzler, {[}4.{]} 10. {[}1926?{]}]{Stefan Zweig an Arthur Schnitzler, {[}4.{]} 10. {[}1926?{]}}
\nopagebreak\mylabel{v}
\rehead{ }\normalsize\beginnumbering\briefempfaengerindex{Schnitzler, Arthur@\textsc{Schnitzler, Arthur}!zzzZweig, Stefan@\emph{von Stefan Zweig}!1926-10-041@{{[}4.{]} 10. {[}1926?{]}}|(be}
\toendnotes[C]{\smallbreak\pagebreak[2]}\Standort{CUL, Schnitzler, B 118.}
\physDesc{Bildpostkarte, 1 Blatt, 2 Seiten, 531 Zeichen
\newline{}Handschrift: blaue Tinte, lateinische Kurrent
\newline{}Versand: Stempel: »\nobreak{}\oindex{Salzburg@\textbf{Salzburg}, \emph{A.ADM2}|pwk}Sa{[}lzbu{]}\textcolor{gray}{rg}, 6. X. {[}1926{]}, 18\nobreak{}«.  
\newline{}Schnitzler: mit Bleistift datiert: »4/10
                                    2{[}6{]}« }\toendnotes[C]{\smallbreak}\pstart{}{\pb}D\textsuperscript{r}
                  Arthur Schnitzler\pend{}\pstart{}\textcolor{pink}{Wien – Cottage}{}\ledrightnote{\textcolor{pink}{Währinger Cottage}}\pend{}\pstart{}\textcolor{pink}{Sternwartestrasse 71}{}\ledrightnote{\textcolor{pink}{Sternwartestraße 71}}\pend{}
{\bigskip}
\pstart
           \noindent{}\centering{}{\pb}\textcolor{gray}{\textbf{\textcolor{pink}{Salzburg}{}\ledrightnote{\textcolor{pink}{Salzburg}}}}\pend
           
\pstart
           \noindent{}{\pb}Lieber verehrter Herr
                  Doktor, von \label{K_L03663-1v}\edtext{\textcolor{pink}{Zermatt}{}\ledrightnote{\textcolor{pink}{Zermatt}} her}{\lemma{\textnormal{\emph{Zermatt her}}}\Cendnote{\textnormal{Die Karte ist sowohl bei den Poststempeln als auch bei der
                  Datierung durch \textcolor{blue}{Schnitzler} bei der letzten
                  Jahresziffer nicht lesbar. Das erwähnte Treffen in \textcolor{pink}{Zermatt} fand am 20. 8. 1926 statt. Folglich muss die Karte danach
                  und wahrscheinlich noch im selben Jahr verfasst sein.}}}\label{K_L03663-1h} weiss ich noch, wie
               gut es ist, mit Ihnen zu sprechen und weiss es nun wieder, wie wohltuend auch Ihr
               geschriebenes \label{K_L03663-2v}\edtext{Wort}{\lemma{\textnormal{\emph{Wort}}}\Cendnote{\textnormal{{XXXX ref} XXXX }}}\label{K_L03663-2h} einen grüsst: Dankbar habe ich Ihren Eindruck und seine mich sehr
               anregende Ausführlichkeit empfangen und Ihre Zustimmung gilt mir noch genau wie vor
               zwanzig Jahren als innere Bestärkung. Ich werde froh sein, bald Ihnen wieder die Hand
               reichen zu dürfen, viel Grüsse in alter Verehrung voraus! \pend
           
\pstart
           Ihr getreuer{\\[\baselineskip]}\spacefill\mbox{Stefan Zweig}\pend
           \leftskip=0em{}\endnumbering\briefempfaengerindex{Schnitzler, Arthur@\textsc{Schnitzler, Arthur}!zzzZweig, Stefan@\emph{von Stefan Zweig}!1926-10-041@{{[}4.{]} 10. {[}1926?{]}}|)be}\mylabel{h}
\begin{anhang}
\end{anhang}\normalsize

\doendnotes{C}
\bigskip
\vfill

\clearpage

\footnotesize

\lohead{\textsc{register}}

% Definiere theindex-Environment komplett neu ohne reledmac
\makeatletter
\renewenvironment{theindex}{%
  \section*{\indexname}%
  \setlength{\parindent}{0pt}%
  \setlength{\parskip}{0pt plus 0.3pt}%
  \let\item\@idxitem
}{%
  \clearpage
}
\makeatother

\IfFileExists{\jobname-pw.ind}{\input{\jobname-pw.ind}}{}

\end{document}

      