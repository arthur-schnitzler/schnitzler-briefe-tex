%% latex-korrekturansicht-vorspann.tex
%% Vorspann für die Korrekturansicht.
%% Lädt die gemeinsame Datei latex-vorspann.tex mit gesetztem Schalter.

\newif\ifkorrekturansicht
\korrekturansichttrue

\input{../tex-inputs/latex-vorspann}


               \section[Arthur Schnitzler an Richard Beer-Hofmann, 5. 8. 1900]{ Arthur Schnitzler an Richard Beer-Hofmann, 5. 8. 1900}\nopagebreak\mylabel{v}\rehead{ }\normalsize\beginnumbering\briefempfaengerindex{Beer-Hofmann, Richard@\textsc{Beer-Hofmann, Richard}!zzzSchnitzler, Arthur@\emph{von Arthur Schnitzler}!1900-08-051@{5. 8. 1900}|(be} \toendnotes[C]{\smallbreak\pagebreak[2]} \Standort{YCGL, MSS 31.}
\physDesc{Postkarte
\newline{}Handschrift: Bleistift, deutsche Kurrent\newline{}Versand: 1) Stempel: »\nobreak{}\oindex{Bad Ischl@\textbf{Bad Ischl}, \emph{Besiedelter Ort (A.BSO)}|pwk}Ischl, 6. 8. 00, 6–7V\nobreak{}«.  2) Stempel: »\nobreak{}\oindex{Altaussee@\textbf{Altaussee}, \emph{http://www.geonames.org/ontologyA.ADM3}|pwk}\textcolor{gray}{Alt-Aussee}, 6. {[}8.{]} \textcolor{gray}{00}\nobreak{}«. }\toendnotes[C]{\smallbreak}\pstart{}{\pb}Hn Dr. \textsc{Richard}\pend{}\pstart{}\textsc{Beer-Hofmann}\pend{}\pstart{}\textcolor{pink}{\textsc{Altaussee}}{}\ledrightnote{\textcolor{pink}{Altaussee}}\pend{}{\bigskip}\pstart
           \raggedleft{}{\pb}5. 8. 900\pend
           \pstart
           lieber Richard, \textcolor{blue}{Guſtav}{}\ledrightnote{\textcolor{blue}{Gustav Schwarzkopf}} ko{\geminationm}t
                    nicht nach \textcolor{pink}{Salzburg}{}\ledrightnote{\textcolor{pink}{Salzburg}}. Von \textcolor{blue}{Salten}{}\ledrightnote{\textcolor{blue}{Felix Salten}} keine Nachricht. Wir können nun wohl Montag
                        13.{ }Nachmittag 3 Uhr endgiltg abreiſen und \textcolor{pink}{S.}{}\ledrightnote{\textcolor{pink}{Salzburg}} feſtſetzen. Vielleicht ko{\geminationm} ich \label{K_L1900-08-05_01_AS_Beer-Hofmann_1v}\edtext{Mittwoch}{\lemma{\textnormal{\emph{Mittwoch}}}\Cendnote{\textnormal{siehe A. S.: \emph{Tagebuch}, 8. 8. 1900}}}\label{K_L1900-08-05_01_AS_Beer-Hofmann_1h} nach \textcolor{pink}{Auſſee}{}\ledrightnote{\textcolor{pink}{Bad Aussee}}, würde Ihnen noch früher
                    ſchreiben \textsc{resp.} telegr.\pend
           \pstart
           Herzlichſt{\\[\baselineskip]}\spacefill\mbox{Arth}\pend
           \leftskip=0em{}\pstart
           \noindent{}\label{T_L1900-08-05_01_AS_Beer-Hofmann_1v}\edtext{\label{K_L1900-08-05_01_AS_Beer-Hofmann_2v}\edtext{Wer wohnt
                            \textcolor{pink}{Puchen}{}\ledrightnote{\textcolor{pink}{Puchen}} Nr. 57?}{\lemma{\textnormal{\emph{Wer wohnt
                            Puchen Nr. 57?}}}\Cendnote{\textnormal{Laut \emph{\textcolor{green}{Ausseer Cur- und Fremden-Liste}} (Nr. 15,
                                    18. 7. 1900, S. 2) »\textcolor{blue}{Frau Henriette Schönwald},
                                Private, mit Familie und Dienerschaft, aus \textcolor{pink}{Wien}« (6 Personen)}}}\label{K_L1900-08-05_01_AS_Beer-Hofmann_2h}}{\lemma{\textnormal{\emph{Wer wohnt
                            Puchen Nr. 57?}}}\Cendnote{\textnormal{am oberen
                            Rand auf dem Kopf}}}\label{T_L1900-08-05_01_AS_Beer-Hofmann_1h}\pend
           \endnumbering\briefempfaengerindex{Beer-Hofmann, Richard@\textsc{Beer-Hofmann, Richard}!zzzSchnitzler, Arthur@\emph{von Arthur Schnitzler}!1900-08-051@{5. 8. 1900}|)be}\mylabel{h}  \normalsize

\doendnotes{C}
\bigskip
\vfill

\clearpage

\footnotesize

\lohead{\textsc{register}}

% Definiere theindex-Environment komplett neu ohne reledmac
\makeatletter
\renewenvironment{theindex}{%
  \section*{\indexname}%
  \setlength{\parindent}{0pt}%
  \setlength{\parskip}{0pt plus 0.3pt}%
  \let\item\@idxitem
}{%
  \clearpage
}
\makeatother

\IfFileExists{\jobname-pw.ind}{\input{\jobname-pw.ind}}{}

\end{document}

      