%% latex-korrekturansicht-vorspann.tex
%% Vorspann für die Korrekturansicht.
%% Lädt die gemeinsame Datei latex-vorspann.tex mit gesetztem Schalter.

\newif\ifkorrekturansicht
\korrekturansichttrue

\input{../tex-inputs/latex-vorspann}


\renewcommand{\erwaehntePersonen}{Personen: Richard Beer-Hofmann, Olga Schnitzler}
\renewcommand{\erwaehnteOrte}{Orte: Berlin, Dessauer Straße, Hinterbrühl, Wien}
\renewcommand{\erwaehnteWerke}{}
\section[ Paul Goldmann an Arthur Schnitzler, 6. 6. {[}1901{]}]{Paul Goldmann an Arthur Schnitzler, 6. 6. {[}1901{]}}
\nopagebreak\mylabel{v}
\rehead{ }\normalsize\beginnumbering\briefempfaengerindex{Schnitzler, Arthur@\textsc{Schnitzler, Arthur}!zzzGoldmann, Paul@\emph{von Paul Goldmann}!1901-06-062@{6. 6. {[}1901{]}}|(be}
\toendnotes[C]{\smallbreak\pagebreak[2]}\Standort{DLA, A:Schnitzler, HS.NZ85.1.3171.}
\physDesc{Brief, 1 Blatt, 2 Seiten
\newline{}Handschrift: blaue Tinte, deutsche Kurrent
\newline{}Schnitzler: mit rotem Buntstift eine Unterstreichung }\toendnotes[C]{\smallbreak}
\pstart
           \noindent{}\raggedleft{}{\pb}\textcolor{pink}{\textcolor{gray}{\textbf{DESSAUERSTRASSE 19}}}{}\ledrightnote{\textcolor{pink}{Dessauer Straße}}\pend
           
\pstart
           \textcolor{pink}{Berlin}{}\ledrightnote{\textcolor{pink}{Berlin}}, 6. Juni.\pend
           
\pstart{}Mein lieber Freund,\pend
\pstart
           Täglich ſehe ich der Poſt mit der Hoffnung entgegen, Nachricht von Dir zu erhalten;
               täglich wird meine Hoffnung getäuſcht. Nicht einmal die bekannte Anſichtspoſtkarte
               trifft ein. \label{K_L03068-1v}\edtext{Wo biſt Du?}{\lemma{\textnormal{\emph{Wo biſt Du?}}}\Cendnote{\textnormal{\textcolor{blue}{Schnitzler} war – mit drei kurzen
                  Unterbrechungen am 2. 6. 1901, von 6. 6. 1901 bis 7. 6. 1901 und am 9. 6. 1901 – von 30. 5. 1901 bis 11. 6. 1901 in der \textcolor{pink}{Hinterbrühl}.}}}\label{K_L03068-1h} Was machſt Du? Was erlebſt Du? Inbezug auf
               Briefſchreiben entwickelſt {\pb}Du Dich
               langſam zu einem \label{K_L03068-2v}\edtext{zweiten \textsc{\textcolor{blue}{Beer-Hofmann}{}\ledrightnote{\textcolor{blue}{Richard Beer-Hofmann}}}}{\lemma{\textnormal{\emph{zweiten Beer-Hofmann}}}\Cendnote{\textnormal{Anspielung auf \textcolor{blue}{Richard Beer-Hofmann}s ausbleibende Briefe an \textcolor{blue}{Goldmann}}}}\label{K_L03068-2h}. Das ſind erfreuliche Ausſichten für die Zukunft.\pend
           
\pstart
           Viele treue Grüße an Dich und, falls Du eine \label{K_L03068-3v}\edtext{Gefährtin}{\lemma{\textnormal{\emph{Gefährtin}}}\Cendnote{\textnormal{\textcolor{blue}{Olga Gussmann} war auch in der \textcolor{pink}{Hinterbrühl}.}}}\label{K_L03068-3h} haſt, auch an dieſe!
               {\\[\baselineskip]}Dein {\\[\baselineskip]}\spacefill\mbox{Paul Goldmann.}\pend
           \leftskip=0em{}\endnumbering\briefempfaengerindex{Schnitzler, Arthur@\textsc{Schnitzler, Arthur}!zzzGoldmann, Paul@\emph{von Paul Goldmann}!1901-06-062@{6. 6. {[}1901{]}}|)be}\mylabel{h}
\begin{anhang}
\end{anhang}\normalsize

\doendnotes{C}
\bigskip
\vfill

\clearpage

\footnotesize

\lohead{\textsc{register}}

% Definiere theindex-Environment komplett neu ohne reledmac
\makeatletter
\renewenvironment{theindex}{%
  \section*{\indexname}%
  \setlength{\parindent}{0pt}%
  \setlength{\parskip}{0pt plus 0.3pt}%
  \let\item\@idxitem
}{%
  \clearpage
}
\makeatother

\IfFileExists{\jobname-pw.ind}{\input{\jobname-pw.ind}}{}

\end{document}

      