%% latex-korrekturansicht-vorspann.tex
%% Vorspann für die Korrekturansicht.
%% Lädt die gemeinsame Datei latex-vorspann.tex mit gesetztem Schalter.

\newif\ifkorrekturansicht
\korrekturansichttrue

\input{../tex-inputs/latex-vorspann}


               \section[Arthur Schnitzler an Richard Beer-Hofmann, 1. 8. 1916]{ Arthur Schnitzler an Richard Beer-Hofmann, 1. 8. 1916}\nopagebreak\mylabel{v}\rehead{ }\normalsize\beginnumbering\briefempfaengerindex{Beer-Hofmann, Richard@\textsc{Beer-Hofmann, Richard}!zzzSchnitzler, Arthur@\emph{von Arthur Schnitzler}!1916-08-011@{1. 8. 1916}|(be} \toendnotes[C]{\smallbreak\pagebreak[2]} \Standort{YCGL, MSS 31.}
\physDesc{Bildpostkarte
\newline{}Handschrift: Bleistift, deutsche Kurrent\newline{}Versand: Stempel: »\nobreak{}\oindex{Altaussee@\textbf{Altaussee}, \emph{http://www.geonames.org/ontologyA.ADM3}|pwk}Alt Aussee, 1. VIII. 16\nobreak{}«.  
\newline{}Beer-Hofmann: mit blauem Buntstift Erhalt
                                 festgehalten: »E.« }\pstart{}{\pb}\textsc{Herrn Dr. Richard}\pend{}\pstart{}\textsc{Beer}\substVorne{}\textsuperscript{h}\substDazwischen{}\textsc{H}\substHinten{}\textsc{ofmann}\pend{}\pstart{}\textcolor{pink}{\textsc{Bad Ischl}}{}\ledrightnote{\textcolor{pink}{Bad Ischl}}\pend{}\pstart{}\textcolor{pink}{\textsc{Grazerstr. 52}}{}\ledrightnote{\textcolor{pink}{Grazer Straße}}\pend{}{\bigskip}\pstart
           \noindent{}\centering{}{\pb}\textcolor{gray}{\textbf{\textcolor{pink}{Salzkammergut}{}\ledrightnote{\textcolor{pink}{Salzkammergut}}. \textcolor{pink}{Alt Aussee}{}\ledrightnote{\textcolor{pink}{Altaussee}} mit dem \textcolor{pink}{Loser}{}\ledrightnote{\textcolor{pink}{Loser}}.}}\pend
           \pstart
           \raggedleft{}{\pb}1. 8. 1916\pend
           \pstart
           lieber Richard, wir haben in der \textcolor{pink}{Kaiſerkr.}{}\ledrightnote{\textcolor{pink}{Hotel Kaiserkrone}} kein Zi{\geminationm}er beko{\geminationm}en, übernachten daher nicht in \textcolor{pink}{Iſchl}{}\ledrightnote{\textcolor{pink}{Bad Ischl}}. Statt Mittwoch gedenke ich Do{\geminationn}erſtag hinüber zu wandern und um
                  ½ 2 (oder etwas früher) beim \textcolor{pink}{\uline{So{\geminationn}enſchein}}{}\ledrightnote{\textcolor{pink}{Restaurant Sonnenschein}} zu eſſen.\strikeout{)}{ }\uline{Ebenſo Abends} ſchon circa ½, ¾ 8, da wir
               um 9 nach \textcolor{pink}{Auſſee}{}\ledrightnote{\textcolor{pink}{Altaussee}} zurückfahren müſſen.
                  \textcolor{blue}{O.}{}\ledrightnote{\textcolor{blue}{Olga Schnitzler}} iſſt zu Mittag in der \textcolor{pink}{Aſchau}{}\ledrightnote{\textcolor{pink}{Aschau}}. In jedem Fall hoffen wir alſo zum Nachtm. mit Ihnen
                  zuſa{\geminationm}en zu ſein; ich melde mich natürlich früher {\pb}bei Ihnen. Ganz unverbindlich, auch für Sie!\pend
           \pstart
           Herzlichſt Ihr{\\[\baselineskip]}\spacefill\mbox{Arthur}\pend
           \leftskip=0em{}\endnumbering\briefempfaengerindex{Beer-Hofmann, Richard@\textsc{Beer-Hofmann, Richard}!zzzSchnitzler, Arthur@\emph{von Arthur Schnitzler}!1916-08-011@{1. 8. 1916}|)be}\mylabel{h}  \normalsize

\doendnotes{C}
\bigskip
\vfill

\clearpage

\footnotesize

\lohead{\textsc{register}}

% Definiere theindex-Environment komplett neu ohne reledmac
\makeatletter
\renewenvironment{theindex}{%
  \section*{\indexname}%
  \setlength{\parindent}{0pt}%
  \setlength{\parskip}{0pt plus 0.3pt}%
  \let\item\@idxitem
}{%
  \clearpage
}
\makeatother

\IfFileExists{\jobname-pw.ind}{\input{\jobname-pw.ind}}{}

\end{document}

      