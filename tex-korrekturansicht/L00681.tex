%% latex-korrekturansicht-vorspann.tex
%% Vorspann für die Korrekturansicht.
%% Lädt die gemeinsame Datei latex-vorspann.tex mit gesetztem Schalter.

\newif\ifkorrekturansicht
\korrekturansichttrue

\input{../tex-inputs/latex-vorspann}


               \section[Arthur Schnitzler an Richard Beer-Hofmann, 29. 5. 1897]{ Arthur Schnitzler an Richard Beer-Hofmann, 29. 5. 1897}\nopagebreak\mylabel{v}\rehead{ }\normalsize\beginnumbering\briefempfaengerindex{Beer-Hofmann, Richard@\textsc{Beer-Hofmann, Richard}!zzzSchnitzler, Arthur@\emph{von Arthur Schnitzler}!1897-05-291@{29. 5. 1897}|(be} \toendnotes[C]{\smallbreak\pagebreak[2]} \Standort{YCGL, MSS 31.}
\physDesc{Postkarte
\newline{}Handschrift: schwarze Tinte, deutsche Kurrent\newline{}Versand: 1) Stempel: »\nobreak{}\oindex{Forest Hill@\textbf{Forest Hill}, \emph{Bezirk (A.BZK)}|pwk}Forest Hill, MY 29 9\textcolor{gray}{7}\nobreak{}«.  2) Stempel: »\nobreak{}\oindex{I., Innere Stadt@\textbf{I., Innere Stadt}, \emph{Bezirk (A.BZK)}|pwk}Wien 1/1, 31 5. 97, 6½–8N, Bestellt\nobreak{}«. 3) mit Bleistift von unbekannter Hand am
                           oberen Rand der Adressseite: »\textcolor{pink}{\textsc{Austria}}«}\buchAbdrucke{\weitereDrucke{Arthur Schnitzler, Richard Beer-Hofmann: \emph{Briefwechsel 1891–1931}. Hg. Konstanze Fliedl. Wien, Zürich: \emph{Europaverlag} 1992, S. 106.} }\pstart{}{\pb}Herrn \textsc{Dr. Richard
                     Beer-Hofmann}\pend{}\pstart{}\textcolor{pink}{\textsc{Wien}}{}\ledrightnote{\textcolor{pink}{Wien}}\pend{}\pstart{}\textcolor{pink}{\textsc{I. Bez. Wollzeile 15}.}{}\ledrightnote{\textcolor{pink}{Wollzeile}}\pend{}{\bigskip}\pstart
           \raggedleft{}{\pb}\textcolor{pink}{London S. E.}{}\ledrightnote{\textcolor{pink}{London}}{\\}29. 5. \textcolor{gray}{97}\pend
           \pstart
           Mein lieber Richard, Ihren Brief hab ich noch in \textcolor{pink}{Paris}{}\ledrightnote{\textcolor{pink}{Paris}}\footnote{\noindent{}Iſt ja gar nicht wahr; in \textcolor{pink}{London} hab ich ihn
                     gefunden.} beko{\geminationm}en. – »Wie ſchätz ich Euch um dieſes
               Ekels willen!«\pend
           \pstart
           Aber es ſcheint wirklich, ich treffe Sie in \textcolor{pink}{Wien}{}\ledrightnote{\textcolor{pink}{Wien}}
               nicht mehr an? – Möchte Mittwoch{ }\introOben{}Ab\introOben{} oder Do{\geminationn}erſtag{ }Früh anlangen. Ich wünſchte eine Zeile von Ihnen vorzufinden. Ja? – Nach
               Hauſe ſehn ich mich wenig; ſehr nach ein biſſel Ruh und Arbeit.\pend
           \pstart Herzlichen Gruſs. Ihr \spacefill\mbox{Arthur.}\pend{}\endnumbering\briefempfaengerindex{Beer-Hofmann, Richard@\textsc{Beer-Hofmann, Richard}!zzzSchnitzler, Arthur@\emph{von Arthur Schnitzler}!1897-05-291@{29. 5. 1897}|)be}\mylabel{h}  \normalsize

\doendnotes{C}
\bigskip
\vfill

\clearpage

\footnotesize

\lohead{\textsc{register}}

% Definiere theindex-Environment komplett neu ohne reledmac
\makeatletter
\renewenvironment{theindex}{%
  \section*{\indexname}%
  \setlength{\parindent}{0pt}%
  \setlength{\parskip}{0pt plus 0.3pt}%
  \let\item\@idxitem
}{%
  \clearpage
}
\makeatother

\IfFileExists{\jobname-pw.ind}{\input{\jobname-pw.ind}}{}

\end{document}

      