%% latex-korrekturansicht-vorspann.tex
%% Vorspann für die Korrekturansicht.
%% Lädt die gemeinsame Datei latex-vorspann.tex mit gesetztem Schalter.

\newif\ifkorrekturansicht
\korrekturansichttrue

\input{../tex-inputs/latex-vorspann}


\renewcommand{\erwaehntePersonen}{Personen: Felix Salten}
\renewcommand{\erwaehnteOrte}{Orte: Wien}
\renewcommand{\erwaehnteWerke}{Werke: Fünfzehn Hasen. Schicksale in Wald und Feld}
\section[ Arthur Schnitzler an Felix Salten, 30. 12. 19{[}29?{]}]{Arthur Schnitzler an Felix Salten, 30. 12. 19{[}29?{]}}
\nopagebreak\mylabel{v}
\rehead{ }\normalsize\beginnumbering\briefempfaengerindex{Salten, Felix@\textsc{Salten, Felix}!zzzSchnitzler, Arthur@\emph{von Arthur Schnitzler}!1929-12-301@{30. 12. 19{[}29?{]}}|(be}
\toendnotes[C]{\smallbreak\pagebreak[2]}\Standort{Wienbibliothek im Rathaus, ZPH 1681, 2.1.516.}
\physDesc{Brief, 1 Blatt, 1 Seite, 289 Zeichen
\newline{}Handschrift: schwarze Tinte, deutsche Kurrent
\newline{}Ordnung: mit Bleistift von unbekannter Hand nummeriert: »1« }\toendnotes[C]{\smallbreak}
\pstart
           \raggedleft{}{\pb}\textcolor{pink}{Wien}{}\ledrightnote{\textcolor{pink}{Wien}}{ }\label{K_L03025-1v}\edtext{30/12 930}{\lemma{\textnormal{\emph{30/12 930}}}\Cendnote{\textnormal{Es dürfte sich bei der Datierung um
                     eine Verwechslung handeln, womöglich motiviert durch den bevorstehenden
                     Jahreswechsel. Am 27. 12. 1929 hatte \textcolor{blue}{Schnitzler}
                     die Neuerscheinung \emph{\textcolor{green}{Fünfzehn Hasen. Schicksale
                        in Wald und Feld}} gelesen.}}}\label{K_L03025-1h}\pend
           
\pstart
           lieber, lassen Sie mich Ihnen sehr herzlich für Ihr erquickendes
               neues \textcolor{green}{Thierbuch}{}\ledrightnote{{$\rightarrow$}\textcolor{green}{Fünfzehn Hasen. Schicksale in Wald und Feld}} danken, das
               ich erst vor wenig\textcolor{gray}{en} Tagen zu Ende gelesen habe. Es ist so
               naturnah und so jung.\pend
           
\pstart
           Auf Wiedersehen – aber wirklich – und alles gute zum neuen Jahr Ihnen und den Ihren. {\\[\baselineskip]}Immer Ihr {\\[\baselineskip]}\spacefill\mbox{Arth Schnitzler}\pend
           \leftskip=0em{}\endnumbering\briefempfaengerindex{Salten, Felix@\textsc{Salten, Felix}!zzzSchnitzler, Arthur@\emph{von Arthur Schnitzler}!1929-12-301@{30. 12. 19{[}29?{]}}|)be}\mylabel{h}  \normalsize

\doendnotes{C}
\bigskip
\vfill

\clearpage

\footnotesize

\lohead{\textsc{register}}

% Definiere theindex-Environment komplett neu ohne reledmac
\makeatletter
\renewenvironment{theindex}{%
  \section*{\indexname}%
  \setlength{\parindent}{0pt}%
  \setlength{\parskip}{0pt plus 0.3pt}%
  \let\item\@idxitem
}{%
  \clearpage
}
\makeatother

\IfFileExists{\jobname-pw.ind}{\input{\jobname-pw.ind}}{}

\end{document}

      