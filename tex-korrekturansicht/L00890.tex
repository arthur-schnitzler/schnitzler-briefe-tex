%% latex-korrekturansicht-vorspann.tex
%% Vorspann für die Korrekturansicht.
%% Lädt die gemeinsame Datei latex-vorspann.tex mit gesetztem Schalter.

\newif\ifkorrekturansicht
\korrekturansichttrue

\input{../tex-inputs/latex-vorspann}


               \section[Arthur Schnitzler an Hermann Bahr, Antwort auf eine Umfrage, 15. 2. 1899]{ Arthur Schnitzler an Hermann Bahr, Antwort auf eine Umfrage,
               15. 2. 1899}\nopagebreak\mylabel{v}\rehead{ }\normalsize\beginnumbering\briefempfaengerindex{Bahr, Hermann@\textsc{Bahr, Hermann}!zzzSchnitzler, Arthur@\emph{von Arthur Schnitzler}!1899-02-151@{15. 2. 1899}|(be} \toendnotes[C]{\smallbreak\pagebreak[2]} \buchAlsQuelle{\pwindex{Zeit. Wiener Wochenschrift@\emph{Die Zeit. Wiener Wochenschrift}|pwk}Arthur Schnitzler: \emph{[Das Erscheinen der Autoren].} In: \emph{Die Zeit}, Bd. 18, Nr. 229, 18. 2. 1899, S. 104–106, hier: S. 105.}\buchAbdrucke{\weitereDrucke{Hermann Bahr, Arthur Schnitzler: \emph{Briefwechsel, Aufzeichnungen, Dokumente (1891–1931)}. Hg. Kurt Ifkovits und Martin Anton Müller. Göttingen: \emph{Wallstein} 2018, S. 167–168.} }\toendnotes[C]{\smallbreak}\pstart
           \raggedleft{}{\pb}\textcolor{pink}{Wien}{}\ledrightnote{\textcolor{pink}{Wien}}, 15. Februar
                     1899. \pend
           \pstart
           \centering{}\so{Lieber Bahr!}\pend
           \pstart
           \label{K_L00890_1v}\edtext{Ob ein gerufener Autor erſcheinen
                  ſoll}{\lemma{\textnormal{\emph{Ob … ſoll}}}\Cendnote{\textnormal{Der Brief erschien zusammen mit
                  weiteren Antworten nach folgender wohl von \textcolor{blue}{Bahr} verfasster Einleitung: »Zu dem Aufsatze ›\textcolor{green}{Premièren}‹ in Nr. 228 der ›\textcolor{green}{Zeit}‹, welcher anregte, dass sich die Autoren bei ihren Premièren
                     nicht mehr dem Publicum zeigen sollen, sind uns folgende Zuschriften
                     zugekommen:« Die anderen Antworten, durchwegs in Form eines an Bahr
                  gerichteten Briefes: \textcolor{blue}{Emerich von Bukovics}, \textcolor{blue}{Ernst Gettke}, \textcolor{blue}{Leo
                     Ebermann}, \textcolor{blue}{Carl Karlweis}, \textcolor{blue}{Philipp Langmann}, \textcolor{blue}{Victor Léon}, \textcolor{blue}{Oskar Blumenthal}, \textcolor{blue}{Ernst von Wildenbruch} und \textcolor{blue}{Otto Erich Hartleben}; die Antwort von \textcolor{blue}{Max Grube} in Gestalt eines Gedichts. Auf eine \textcolor{green}{Reaktion}{ }\textcolor{blue}{Theodor Herzls} in der \emph{\textcolor{green}{Neuen Freien Presse}} vom 12. 2. 1899
                     (Nr. 12384, S. 8) wird hingewiesen.}}}\label{K_L00890_1h} oder nicht? Nichts iſt
               gleichgiltiger für das innere Schickſal der Première; nichts gleichgiltiger für das
               fernere Schickſal des betreffenden Stückes. Jeder Autor möge es daher in jedem Falle
               halten, wie es ihm beliebt. In Geſchmacks- und Stimmungsfragen gibt es keine
               Solidarität.\pend
           \pstart
           Herzlichen Gruß. \label{K_L00890_2v}\edtext{Dein}{\lemma{\textnormal{\emph{Dein}}}\Cendnote{\textnormal{Drei weitere Antworten geben
                  Duzbrüderschaft mit Bahr zu erkennen: \textcolor{blue}{Bukovics},
                     \textcolor{blue}{Ebermann} und \textcolor{blue}{Karlweis}.}}}\label{K_L00890_2h} ergebener{\\[\baselineskip]}\spacefill\mbox{Arthur Schnitzler.}\pend
           \leftskip=0em{}\endnumbering\briefempfaengerindex{Bahr, Hermann@\textsc{Bahr, Hermann}!zzzSchnitzler, Arthur@\emph{von Arthur Schnitzler}!1899-02-151@{15. 2. 1899}|)be}\mylabel{h}  \normalsize

\doendnotes{C}
\bigskip
\vfill

\clearpage

\footnotesize

\lohead{\textsc{register}}

% Definiere theindex-Environment komplett neu ohne reledmac
\makeatletter
\renewenvironment{theindex}{%
  \section*{\indexname}%
  \setlength{\parindent}{0pt}%
  \setlength{\parskip}{0pt plus 0.3pt}%
  \let\item\@idxitem
}{%
  \clearpage
}
\makeatother

\IfFileExists{\jobname-pw.ind}{\input{\jobname-pw.ind}}{}

\end{document}

      