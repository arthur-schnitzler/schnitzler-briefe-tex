%% latex-korrekturansicht-vorspann.tex
%% Vorspann für die Korrekturansicht.
%% Lädt die gemeinsame Datei latex-vorspann.tex mit gesetztem Schalter.

\newif\ifkorrekturansicht
\korrekturansichttrue

\input{../tex-inputs/latex-vorspann}


\section[Elsa Plessner an Arthur Schnitzler, 17. 11. 1896]{L03707 Elsa Plessner an Arthur Schnitzler, 17. 11. 1896}
\nopagebreak\mylabel{L03707v}
\rehead{ }\normalsize\beginnumbering\briefempfaengerindex{Schnitzler, Arthur@\textsc{Schnitzler, Arthur}!zzzPlessner, Elsa@\emph{von Elsa Plessner}!1896-11-171@{17. 11. 1896}|(be}
\toendnotes[C]{\smallbreak\pagebreak[2]}
\correspDesc{Versand  durch Elsa Plessner am 17. 11. 1896 in Meran
\newline{}Erhalt  durch Arthur Schnitzler im Zeitraum [18. 11. 1896 –
            22. 11. 1896?] in Wien}\toendnotes[C]{\smallbreak}
\Standort{DLA, A:Schnitzler, HS.1985.1.419.}
\physDesc{Kartenbrief, 1 Blatt, 1 Seite, 177 Zeichen
\newline{}Handschrift: schwarze Tinte, lateinische Kurrent
\newline{}Schnitzler: eine Unterstreichung }\toendnotes[C]{\smallbreak}
\pstart
           {\pb}\textcolor{pink}{Meran, Pension Wolf}\oindex{Hotel Meranerhof@\textbf{Hotel Meranerhof}, \emph{Hotel}|pw}{}\ledrightnote{\textcolor{pink}{Hotel Meranerhof}}, den
            17.\,11.\,96.\pend
           
\pstart{}Verehrter Herr Doctor!\pend\vspace{0.5em}
\pstart
           \centering{}\textcolor{gray}{\textbf{Elsa Plessner}}\pend
           
\pstart
           sendet, von Reisevorbereitungen früher davon abgehalten \introOben{}nachträglich\introOben{}, herzliche Gratulation zum \label{K_L03707-1v}\edtext{»\textcolor{green}{Freiwild}\pwindex{Schnitzler, Arthur 15. 5. 1862 Wien – 21. 10. 1931 ebd.@\textsc{Schnitzler, Arthur} (15. 5. 1862 Wien – 21. 10. 1931 ebd.), \emph{Schriftsteller, Mediziner}!Freiwild. Schauspiel in 3 Akten@\strich\emph{Freiwild. Schauspiel in 3 Akten}|pw}{}\ledrightnote{\textcolor{green}{Freiwild. Schauspiel in 3 Akten}}«}{\lemma{\textnormal{\emph{»Freiwild«}}}\Cendnote{\textnormal{Die \textcolor{violet}{Uraufführung}\eventindex{Deutsches Theater Berlin@\textbf{Deutsches Theater Berlin}!Uraufführung von Freiwild, 3.11.1896@Uraufführung von Freiwild, 3.11.1896|pwkv} von \textcolor{blue}{Schnitzlers} Schauspiel
                  \emph{\textcolor{green}{Freiwild}\pwindex{Schnitzler, Arthur 15. 5. 1862 Wien – 21. 10. 1931 ebd.@\textsc{Schnitzler, Arthur} (15. 5. 1862 Wien – 21. 10. 1931 ebd.), \emph{Schriftsteller, Mediziner}!Freiwild. Schauspiel in 3 Akten@\strich\emph{Freiwild. Schauspiel in 3 Akten}|pwk}} fand am 3. 11. 1896 am \emph{\textcolor{brown}{Deutschen Theater}\orgindex{Deutsches Theater Berlin@Deutsches Theater Berlin|pwk}} in \textcolor{pink}{Berlin}\oindex{Berlin@\textbf{Berlin}, \emph{Hauptstadt}|pwk} statt. }}}\label{K_L03707-1}. \label{K_L03707-2v}\edtext{Vivat sequentes}{\lemma{\textnormal{\emph{Vivat sequentes}}}\Cendnote{\textnormal{lateinisch: die Folgenden sollen leben}}}\label{K_L03707-2}!!\pend
           \selectlanguage{ngerman}\endnumbering\briefempfaengerindex{Schnitzler, Arthur@\textsc{Schnitzler, Arthur}!zzzPlessner, Elsa@\emph{von Elsa Plessner}!1896-11-171@{17. 11. 1896}|)be}\mylabel{L03707h}
\begin{anhang}
\end{anhang}\normalsize

\doendnotes{C}
\bigskip
\vfill

\clearpage

\footnotesize

\lohead{\textsc{register}}

% Definiere theindex-Environment komplett neu ohne reledmac
\makeatletter
\renewenvironment{theindex}{%
  \section*{\indexname}%
  \setlength{\parindent}{0pt}%
  \setlength{\parskip}{0pt plus 0.3pt}%
  \let\item\@idxitem
}{%
  \clearpage
}
\makeatother

\IfFileExists{\jobname-pw.ind}{\input{\jobname-pw.ind}}{}

\end{document}

      