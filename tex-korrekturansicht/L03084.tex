%% latex-korrekturansicht-vorspann.tex
%% Vorspann für die Korrekturansicht.
%% Lädt die gemeinsame Datei latex-vorspann.tex mit gesetztem Schalter.

\newif\ifkorrekturansicht
\korrekturansichttrue

\input{../tex-inputs/latex-vorspann}


\renewcommand{\erwaehntePersonen}{Personen: Otto Brahm, Max Eugen Burckhard, Felix Salten, Louise Schnitzler, Gustav Schwarzkopf, Hermann Sudermann, Irene Triesch}
\renewcommand{\erwaehnteInstitutionen}{Institutionen: Burgtheater, Deutsches Theater Berlin, Verwaltungsgerichtshof}
\renewcommand{\erwaehnteOrte}{Orte: Berlin, Danzig, Dessauer Straße, Deutsches Theater Berlin, Wien}
\renewcommand{\erwaehnteWerke}{Werke: Die Frau mit dem Dolche, Es lebe das Leben, Lebendige Stunden, Lebendige Stunden. Vier Einakter, Lieutenant Gustl. Novelle, Literatur}
\section[ Paul Goldmann an Arthur Schnitzler, 16. 9. {[}1901{]}]{Paul Goldmann an Arthur Schnitzler, 16. 9. {[}1901{]}}
\nopagebreak\mylabel{v}
\rehead{ }\normalsize\beginnumbering\briefempfaengerindex{Schnitzler, Arthur@\textsc{Schnitzler, Arthur}!zzzGoldmann, Paul@\emph{von Paul Goldmann}!1901-09-162@{16. 9. {[}1901{]}}|(be}
\toendnotes[C]{\smallbreak\pagebreak[2]}\Standort{DLA, A:Schnitzler, HS.NZ85.1.3171.}
\physDesc{Brief, 1 Blatt, 3 Seiten
\newline{}Handschrift: blaue Tinte, deutsche Kurrent
\newline{}Schnitzler: 1) mit Bleistift das Jahr »{[}1{]}901« vermerkt  2) mit rotem Buntstift drei Unterstreichungen}\toendnotes[C]{\smallbreak}
\pstart
           \noindent{}\raggedleft{}{\pb}\textcolor{pink}{\textcolor{gray}{\textbf{DESSAUERSTRASSE 19}}}{}\ledrightnote{\textcolor{pink}{Dessauer Straße}}\pend
           
\pstart
           \textcolor{pink}{Berlin}{}\ledrightnote{\textcolor{pink}{Berlin}}, 16. September.\pend
           
\pstart\center{}Mein lieber Freund,\pend
\pstart
           Bin aus \textcolor{pink}{Danzig}{}\ledrightnote{\textcolor{pink}{Danzig}} zurück, finde Deinen lieben
               Brief, habe ſehr viel zu thun und kann einſtweilen nur in Eile antworten: Habe geſtern die \textsc{\textcolor{blue}{Triesch}{}\ledrightnote{\textcolor{blue}{Irene Triesch}}} geſprochen, die mit Sehnſucht auf Deine \textcolor{green}{Stücke}{}\ledrightnote{{$\rightarrow$}\textcolor{green}{Lebendige Stunden. Vier Einakter}} wartet und auch ſehr gern die \label{K_L03084-1v}\edtext{\textcolor{gray}{L}uſtſpielrolle \textcolor{gray}{im}{ }\textcolor{green}{dritten}{}\ledrightnote{{$\rightarrow$}\textcolor{green}{Literatur}} ſpielen}{\lemma{\textnormal{\emph{Luſtſpielrolle … ſpielen}}}\Cendnote{\textnormal{\textcolor{blue}{Irene Triesch} übernahm bei der Uraufführung von
                     \emph{\textcolor{green}{Lebendige Stunden}} am 4. 1. 1902 am \emph{\textcolor{brown}{Deutschen Theater Berlin}} die Rolle der \textcolor{green}{Margarete} in \emph{\textcolor{green}{Literatur}}.}}}\label{K_L03084-1h} möchte. Außer {\pb}dem neuen \label{K_L03084-12v}\edtext{\textcolor{green}{Stück}{}\ledrightnote{{$\rightarrow$}\textcolor{green}{Es lebe das Leben}}}{\lemma{\textnormal{\emph{Stück}}}\Cendnote{\textnormal{\textcolor{blue}{Hermann Sudermann}s Fünfakter \emph{\textcolor{green}{Es lebe das Leben}} wurde am 1. 2. 1902 am \textcolor{pink}{Deutschen
                     Theater Berlin} uraufgeführt.}}}\label{K_L03084-12h} von \textsc{\textcolor{blue}{Sudermann}{}\ledrightnote{\textcolor{blue}{Hermann Sudermann}}} hat \textsc{\textcolor{blue}{Brahm}{}\ledrightnote{\textcolor{blue}{Otto Brahm}}} gar nichts.\pend
           
\pstart
           Das \label{K_L03084-2v}\edtext{Urtheil}{\lemma{\textnormal{\emph{Urtheil}}}\Cendnote{\textnormal{Zu den Einaktern \emph{\textcolor{green}{Die Frau mit dem Dolche}}
               und \emph{\textcolor{green}{Lebendige Stunden}}, 
                  vgl. A. S.: \emph{Tagebuch}, 4. 9. 1901}}}\label{K_L03084-2h}, das \textsc{\textcolor{blue}{Schwarzkopf}{}\ledrightnote{\textcolor{blue}{Gustav Schwarzkopf}}} und \textsc{\textcolor{blue}{Salten}{}\ledrightnote{\textcolor{blue}{Felix Salten}}} gefällt haben, halte ich für durchaus unrichtig.\pend
           
\pstart
           Dagegen billige ich durchaus den \label{K_L03084-3v}\edtext{Standpunkt, den \textsc{\textcolor{blue}{Burckhart}{}\ledrightnote{\textcolor{blue}{Max Eugen Burckhard}}}}{\lemma{\textnormal{\emph{Standpunkt, den Burckhart}}}\Cendnote{\textnormal{\textcolor{blue}{Max Burckhard}, ehemaliger Direktor des \emph{\textcolor{brown}{Burgtheater}}s, war seit 1901 am \emph{\textcolor{brown}{Verwaltungsgerichtshof}}
                  tätig und beriet \textcolor{blue}{Schnitzler} in der \emph{\textcolor{green}{Lieutenant Gustl}}-Affäre. Er empfahl \textcolor{blue}{Schnitzler},
                  nicht zu reagieren, siehe Bahr/Schnitzler, T030081.}}}\label{K_L03084-3h}{ }\strikeout{ein} in der Militär-Affaire einnimmt. Laß’ die Leute
               nur reden! Und ſchreib’ weiter gute {\pb}Stücke! Das iſt
               die beſte Antwort und ärgert ſie am Meiſten.\pend
           
\pstart
           Ich danke vielmals für die Zuſendung der alten Hoſen, die ich bei Euch vergeſſen
               hatte. Hättet ſie auch behalten können.\pend
           
\pstart
           Empfiehl’ mich Deiner Frau \textcolor{blue}{Mutter}{}\ledrightnote{{$\rightarrow$}\textcolor{blue}{Louise Schnitzler}} und ſei herzlichſt gegrüßt!\pend
           
\pstart
           Dein {\\[\baselineskip]}\spacefill\mbox{Paul Goldmann }\pend
           \leftskip=0em{}\endnumbering\briefempfaengerindex{Schnitzler, Arthur@\textsc{Schnitzler, Arthur}!zzzGoldmann, Paul@\emph{von Paul Goldmann}!1901-09-162@{16. 9. {[}1901{]}}|)be}\mylabel{h}
\begin{anhang}
\end{anhang}\normalsize

\doendnotes{C}
\bigskip
\vfill

\clearpage

\footnotesize

\lohead{\textsc{register}}

% Definiere theindex-Environment komplett neu ohne reledmac
\makeatletter
\renewenvironment{theindex}{%
  \section*{\indexname}%
  \setlength{\parindent}{0pt}%
  \setlength{\parskip}{0pt plus 0.3pt}%
  \let\item\@idxitem
}{%
  \clearpage
}
\makeatother

\IfFileExists{\jobname-pw.ind}{\input{\jobname-pw.ind}}{}

\end{document}

      