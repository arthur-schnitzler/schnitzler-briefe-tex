%% latex-korrekturansicht-vorspann.tex
%% Vorspann für die Korrekturansicht.
%% Lädt die gemeinsame Datei latex-vorspann.tex mit gesetztem Schalter.

\newif\ifkorrekturansicht
\korrekturansichttrue

\input{../tex-inputs/latex-vorspann}


               \section[ Paul Goldmann an Arthur Schnitzler, 27. 10. {[}1897{]}]{Paul Goldmann an Arthur Schnitzler, 27. 10. {[}1897{]}}\nopagebreak\mylabel{v}\rehead{ }\normalsize\beginnumbering\briefempfaengerindex{Schnitzler, Arthur@\textsc{Schnitzler, Arthur}!zzzGoldmann, Paul@\emph{von Paul Goldmann}!1897-10-273@{27. 10. {[}1897{]}}|(be} \toendnotes[C]{\smallbreak\pagebreak[2]} \Standort{DLA, A:Schnitzler, HS.NZ85.1.3167.}
\physDesc{Brief, 1 Blatt, 2 Seiten
\newline{}Handschrift: blaue Tinte, deutsche Kurrent
\newline{}Schnitzler: 1) mit Bleistift das Jahr »97« vermerkt 2) mit rotem Buntstift drei Unterstreichungen}\toendnotes[C]{\smallbreak}\pstart
           \noindent{}{\pb}\textcolor{gray}{\textbf{\textbf{\textcolor{brown}{Frankfurter Zeitung}{}\ledrightnote{\textcolor{brown}{Frankfurter Zeitung}}}}}\pend
           \pstart
           \textcolor{gray}{\textbf{(\textcolor{brown}{\begin{otherlanguage}{french}Gazette de Francfort\end{otherlanguage}}{}\ledrightnote{\textcolor{brown}{Frankfurter Zeitung}}).}}\pend
           \pstart
           \textcolor{gray}{\textbf{\textbf{\begin{otherlanguage}{french}Fondateur M.\end{otherlanguage}{ }\textcolor{blue}{L. Sonnemann}{}\ledrightnote{\textcolor{blue}{Leopold Sonnemann}}.}}}\pend
           \pstart
           \begin{otherlanguage}{french}\textcolor{gray}{\textbf{Journal politique, financier,}}\end{otherlanguage}\pend
           \pstart
           \begin{otherlanguage}{french}\textcolor{gray}{\textbf{commercial et littéraire.}}\end{otherlanguage}\pend
           \pstart
           \begin{otherlanguage}{french}\textcolor{gray}{\textbf{\textbf{Paraissant trois fois par jour.}}}\end{otherlanguage}\pend
           \pstart
           \begin{otherlanguage}{french}\textcolor{gray}{\textbf{\textbf{Bureau à \textcolor{pink}{Paris}{}\ledrightnote{\textcolor{pink}{Paris}}}}}\end{otherlanguage}\hfill \textsc{\textcolor{pink}{Paris}{}\ledrightnote{\textcolor{pink}{Paris}}}, 27. Oktober.\pend
           \pstart
           \begin{otherlanguage}{french}\textcolor{gray}{\textbf{\textbf{\textcolor{pink}{10 Rue de la Bourse}{}\ledrightnote{\textcolor{pink}{rue de la Bourse}}.}}}\end{otherlanguage}\pend
           \pstart
           Bitte, liebſter Freund, laß’ doch endlich wieder
               einmal etwas von Dir hören. Wie gehts Dir? Wie gehts »\textcolor{blue}{ihr}{}\ledrightnote{→\textcolor{blue}{Marie Reinhard}}«? Wie gehts den Freunden?\pend
           \pstart
           Alles ſchweigt um mich herum, und ich bin ganz einſam.\pend
           \pstart
           Ich ſende Dir einen amüſanten \label{K_L02830-1v}\edtext{\textcolor{green}{Artikel}{}\ledrightnote{→\textcolor{green}{[?? Artikel von Rochefort über Jesus]}} von \textsc{\textcolor{blue}{Rochefort}{}\ledrightnote{\textcolor{blue}{Victor Henri de Rochefort}}}}{\lemma{\textnormal{\emph{Artikel von Rochefort}}}\Cendnote{\textnormal{nicht ermittelt}}}\label{K_L02830-1h}, welcher von
               unſerem \textcolor{blue}{Glaubensgenoſſen}{}\ledrightnote{→\textcolor{blue}{Jesus}}
               handelt, der am Kreuz geſtorben iſt{\dots}\pend
           \pstart
           \textsc{\textcolor{blue}{Thorel}{}\ledrightnote{\textcolor{blue}{Jean Thorel}}} ſprach ich. Er müht ſich, das \textcolor{green}{Stück}{}\ledrightnote{→\textcolor{green}{Amourette. Pièce en trois actes. Adaptée de Arthur Schnitzler}} anzubringen (aber vielleicht bemüht er ſich nicht
                  genug?){[}.{]} Die Nachrichten ſind wenig günſtig. \textsc{\textcolor{blue}{Antoine}{}\ledrightnote{\textcolor{blue}{André Antoine}}} hat ſich die Antwort vorbehalten, ſcheint aber nicht ſehr geneigt zur \label{K_L02830-2v}\edtext{\textcolor{green}{Aufführung}{}\ledrightnote{→\textcolor{green}{Amourette. Pièce en trois actes. Adaptée de Arthur Schnitzler}}}{\lemma{\textnormal{\emph{Aufführung}}}\Cendnote{\textnormal{\textcolor{blue}{Jean Thorel} versuchte (erfolglos) seine
                  \textcolor{green}{\emph{\textcolor{green}{Liebelei}}-Übersetzung} dem \emph{\textcolor{brown}{Théâtre Antoine}} (von \textcolor{blue}{André Antoine} geleitet) oder dem \emph{\textcolor{brown}{Odéon}} zu vermitteln.}}}\label{K_L02830-2h}.\pend
           \pstart
           {\pb}Willſt Du Dich mit \textsc{\textcolor{blue}{Molière}{}\ledrightnote{\textcolor{blue}{Molière}}} ganz, aber ganz befreunden? \label{K_L02830-5v}\edtext{Lies
               ſeinen \textsc{\textcolor{green}{Don Juan}{}\ledrightnote{→\textcolor{green}{Don Juan oder Der steinerne Gast}}}}{\lemma{\textnormal{\emph{Lies
               ſeinen Don Juan}}}\Cendnote{\textnormal{Lektüre nicht nachweisbar, jedoch sah
                     \textcolor{blue}{Schnitzler} in späteren Jahren mehrere
                  Inszenierungen von \textcolor{blue}{Molière}s \emph{\textcolor{green}{Don Juan}} (vgl. A. S.: \emph{Tagebuch}, 21. 10. 1915, 2. 2. 1916, 27. 9. 1919).}}}\label{K_L02830-5h}, von ihm genannt »\textsc{\textcolor{green}{Le festin de Pierre}{}\ledrightnote{\textcolor{green}{Dom Juan ou le Festin de pierre}}}.«\pend
           \pstart
           Ich weiß Dir nichts mehr zu ſchreiben, als daß ich namenloſes Heimweh habe nach \textcolor{pink}{Wien}{}\ledrightnote{\textcolor{pink}{Wien}}, nach Freundſchaſt, nach Heimlichkeit und
               Gemüthlichkeit. Von Liebe w\textcolor{gray}{i}ll ich nicht reden. So anſpruchsvoll
               bin ich ſchon längſt nicht mehr. Aber nicht mehr fremd ſein in der Fremde!{\dots}\pend
           \pstart
           Grüß’ Dich Gott, liebſter Freund, und vergiß mich nicht gar ſo ſehr!\pend
           \pstart
           Dein treuer {\\[\baselineskip]}\spacefill\mbox{Paul Goldmann}\pend
           \leftskip=0em{}\pstart
           \noindent{}Deiner \textcolor{blue}{Freundin}{}\ledrightnote{→\textcolor{blue}{Marie Reinhard}} viele
                  herzliche Grüße!\pend
           \endnumbering\briefempfaengerindex{Schnitzler, Arthur@\textsc{Schnitzler, Arthur}!zzzGoldmann, Paul@\emph{von Paul Goldmann}!1897-10-273@{27. 10. {[}1897{]}}|)be}\mylabel{h}  \normalsize

\doendnotes{C}
\bigskip
\vfill

\clearpage

\footnotesize

\lohead{\textsc{register}}

% Definiere theindex-Environment komplett neu ohne reledmac
\makeatletter
\renewenvironment{theindex}{%
  \section*{\indexname}%
  \setlength{\parindent}{0pt}%
  \setlength{\parskip}{0pt plus 0.3pt}%
  \let\item\@idxitem
}{%
  \clearpage
}
\makeatother

\IfFileExists{\jobname-pw.ind}{\input{\jobname-pw.ind}}{}

\end{document}

      