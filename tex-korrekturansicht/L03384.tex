%% latex-korrekturansicht-vorspann.tex
%% Vorspann für die Korrekturansicht.
%% Lädt die gemeinsame Datei latex-vorspann.tex mit gesetztem Schalter.

\newif\ifkorrekturansicht
\korrekturansichttrue

\input{../tex-inputs/latex-vorspann}


\renewcommand{\erwaehntePersonen}{Personen: Paul Goldmann, Theodore Rottenberg}
\renewcommand{\erwaehnteOrte}{Orte: Grand Hotel Lavarone, Hotel Carloni, Lago di Garda, Lavarone, Madonna di Campiglio, Palast Hotel Lido, Riva del Garda, Trient}
\renewcommand{\erwaehnteWerke}{}
\section[ Paul Goldmann und Theodore Rottenberg an Arthur Schnitzler, 18. 8. {[}1903{]}]{Paul Goldmann und Theodore Rottenberg an Arthur
               Schnitzler, 18. 8. {[}1903{]}}
\nopagebreak\mylabel{v}
\rehead{ }\normalsize\beginnumbering\briefempfaengerindex{Schnitzler, Arthur@\textsc{Schnitzler, Arthur}!zzzRottenberg, Theodore@\emph{von Theodore Rottenberg}!1903-08-182@{18. 8. {[}1903{]}}|(be}\briefempfaengerindex{Schnitzler, Arthur@\textsc{Schnitzler, Arthur}!zzzGoldmann, Paul@\emph{von Paul Goldmann}!1903-08-182@{18. 8. {[}1903{]}}|(be}
\toendnotes[C]{\smallbreak\pagebreak[2]}\Standort{DLA, A:Schnitzler, HS.NZ85.1.3173.}
\physDesc{Brief, 1 Blatt, 2 Seiten, 576 Zeichen
\newline{}Handschrift Paul Goldmann: schwarze Tinte, deutsche Kurrent
\newline{}Handschrift Theodore Rottenberg: schwarze Tinte, deutsche Kurrent}\toendnotes[C]{\smallbreak}
\pstart
           \noindent{}{\pb}\textcolor{gray}{\textbf{\textbf{\textcolor{pink}{PALAST HOTEL LIDO}{}\ledrightnote{\textcolor{pink}{Palast Hotel Lido}}}}}\pend
           
\pstart
           \textcolor{gray}{\textbf{\textbf{\textcolor{pink}{RIVA}{}\ledrightnote{\textcolor{pink}{Riva del Garda}}}}}{\\}\textcolor{gray}{\textbf{AM \textcolor{pink}{GARDASEE}{}\ledrightnote{\textcolor{pink}{Lago di Garda}}}}\pend
           
\pstart
           \textcolor{gray}{\textbf{Gleiche Direction:}}\pend
           
\pstart
           \textcolor{gray}{\textbf{\textbf{\textcolor{pink}{Grand Hotel Lavarone}{}\ledrightnote{\textcolor{pink}{Grand Hotel Lavarone}} – \textcolor{pink}{Lavarone}{}\ledrightnote{\textcolor{pink}{Lavarone}}}}}\pend
           
\pstart
           \textcolor{gray}{\textbf{(1200 m. ü. M., Saison Juni bis October)}}\hfill \textcolor{gray}{\textbf{\emph{\textcolor{pink}{Riva}{}\ledrightnote{\textcolor{pink}{Riva del Garda}},}}}{ }18. Auguſt \textcolor{gray}{\textbf{\emph{1903}}}\pend
           
\pstart{}Mein lieber Freund,\pend
\pstart
           Hoffentlich haſt Du meine \label{K_L03384-1v}\edtext{heutige Depeſche}{\lemma{\textnormal{\emph{heutige Depeſche}}}\Cendnote{\textnormal{Gemeint sein dürfte das Telegramm vom Vortag: Paul Goldmann an Arthur Schnitzler, 17. 8. 1903. \textcolor{blue}{Goldmann} dürfte davon ausgegangen sein, dass \textcolor{blue}{Schnitzler} erst am 18. 8. 1903 nach \textcolor{pink}{Madonna di Campiglio} kommen würde, womit sich das
                     »postlagernd« auf dem Telegramm erklärt. \textcolor{blue}{Schnitzler} war aber zum Zeitpunkt des vorliegenden
                  Schreibens bereits auf dem Weg nach \textcolor{pink}{Riva}, wo
                  sie sich am selben Tag noch trafen und miteinander soupierten. Die Abreise von \textcolor{blue}{Goldmann} und \textcolor{blue}{Rottenberg} dürfte also erst am Abend stattgefunden haben –
                  oder gemeinsam mit \textcolor{blue}{Schnitzler} am Folgetag.
                  Vermutlich war dieses Schreiben bereits in \textcolor{pink}{Riva} in \textcolor{blue}{Schnitzler}s Unterkunft
                  hinterlegt, als sie aufeinandertrafen.}}}\label{K_L03384-1h} noch erhalten. Wir { }\uline{müſſen}{ }heut von \textcolor{pink}{hier}{}\ledrightnote{{$\rightarrow$}\textcolor{pink}{Riva del Garda}} fort. \strikeout{\textcolor{gray}{S}} Die \label{K_L03384-2v}\edtext{Gründe}{\lemma{\textnormal{\emph{Gründe}}}\Cendnote{\textnormal{\textcolor{blue}{Rottenberg} war verheiratet. Es dürfte eine
                  ihnen bekannte Person, die hätte ausplaudern können, ebenfalls in der \textcolor{pink}{Stadt} gewesen sein, vgl. Paul Goldmann und Theodore Rottenberg an Arthur
               Schnitzler, 29. 8. 1903.}}}\label{K_L03384-2h} werden wir Dir
               mündlich ſagen, und Du wirſt ſie begreifen. Wir hätten uns Beide unendlich gefreut,
               Dich hier zu ſehen, hoffen aber, das Verſäumte in \textsc{\textcolor{pink}{Trient}{}\ledrightnote{\textcolor{pink}{Trient}}}{ }\strikeout{und} oder \textsc{\textcolor{pink}{Lavarone}{}\ledrightnote{\textcolor{pink}{Lavarone}}}{ }nachzuholen. In \label{K_L03384-3v}\edtext{\textsc{\textcolor{pink}{Trient}{}\ledrightnote{\textcolor{pink}{Trient}}} bleiben \strikeout{b} wir bis morgen{ }Mittag}{\lemma{\textnormal{\emph{Trient … Mittag}}}\Cendnote{\textnormal{Das verschob sich um einen Tag: \textcolor{blue}{Schnitzler} fuhr entweder mit den beiden am
                     19. 8. 1903
                  gemeinsam nach \textcolor{pink}{Trient} oder holte sie dort ein.
                  Die gemeinsame Weiterreise fand dann am 20. 8. 1903 statt.}}}\label{K_L03384-3h} (\textsc{\textcolor{pink}{Hotel Carloni}{}\ledrightnote{\textcolor{pink}{Hotel Carloni}}}), dann \textsc{\textcolor{pink}{Lavarone}{}\ledrightnote{\textcolor{pink}{Lavarone}}} (\textsc{\textcolor{pink}{Grand Hôtel Central}{}\ledrightnote{\textcolor{pink}{Grand Hotel Lavarone}}})\pend
           
\pstart
           {\pb}Sei uns nicht böſe! Und komme uns \uline{ſo bald als möglich} nach!\pend
           
\pstart
           Tauſend Grüße! {\\[\baselineskip]}Dein {\\[\baselineskip]}\spacefill\mbox{Paul Goldmann}\pend
           \leftskip=0em{}
\pstart
           \noindent{}{[}hs. Rottenberg:{]} Viele herzliche Grüße, hoffentlich können Sie ſich
                  die Gründe denken u ko{\geminationm}en uns gleich nach! –\pend
           \endnumbering\briefempfaengerindex{Schnitzler, Arthur@\textsc{Schnitzler, Arthur}!zzzRottenberg, Theodore@\emph{von Theodore Rottenberg}!1903-08-182@{18. 8. {[}1903{]}}|)be}\briefempfaengerindex{Schnitzler, Arthur@\textsc{Schnitzler, Arthur}!zzzGoldmann, Paul@\emph{von Paul Goldmann}!1903-08-182@{18. 8. {[}1903{]}}|)be}\mylabel{h}  \normalsize

\doendnotes{C}
\bigskip
\vfill

\clearpage

\footnotesize

\lohead{\textsc{register}}

% Definiere theindex-Environment komplett neu ohne reledmac
\makeatletter
\renewenvironment{theindex}{%
  \section*{\indexname}%
  \setlength{\parindent}{0pt}%
  \setlength{\parskip}{0pt plus 0.3pt}%
  \let\item\@idxitem
}{%
  \clearpage
}
\makeatother

\IfFileExists{\jobname-pw.ind}{\input{\jobname-pw.ind}}{}

\end{document}

      