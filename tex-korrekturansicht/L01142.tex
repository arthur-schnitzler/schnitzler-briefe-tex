%% latex-korrekturansicht-vorspann.tex
%% Vorspann für die Korrekturansicht.
%% Lädt die gemeinsame Datei latex-vorspann.tex mit gesetztem Schalter.

\newif\ifkorrekturansicht
\korrekturansichttrue

\input{../tex-inputs/latex-vorspann}


               \section[Arthur Schnitzler an Hugo von Hofmannsthal, {[}4.? 7. 1901{]}]{ Arthur Schnitzler an Hugo von Hofmannsthal, {[}4.? 7. 1901{]}}\nopagebreak\mylabel{v}\rehead{ }\normalsize\beginnumbering\briefempfaengerindex{Hofmannsthal, Hugo von@\textsc{Hofmannsthal, Hugo von}!zzzSchnitzler, Arthur@\emph{von Arthur Schnitzler}!1901-07-043@{{[}4.? 7. 1901{]}}|(be} \toendnotes[C]{\smallbreak\pagebreak[2]} \Standort{FDH, Hs-30885,95.}
\physDesc{Brief, 1 Blatt, 4 Seiten
\newline{}Handschrift: schwarze Tinte, deutsche Kurrent\newline{}Ordnung: mit Bleistift von unbekannter Hand datiert: »Juni 1901« }\buchAbdrucke{\weitereDrucke{Hugo von Hofmannsthal, Arthur Schnitzler: \emph{Briefwechsel}. Hg. Therese Nickl und Heinrich Schnitzler. Frankfurt am Main: \emph{S. Fischer} 1964, S. 148–149.} }\toendnotes[C]{\smallbreak}\pstart
           \noindent{}{\pb}Jüdiſcher Millionärsſohn, auf den Geldſäcken ſeiner Ahnen
               herumprotzender Comoediendichter, Freimaurer und Erniedriger des \textcolor{brown}{k. u. k. Hofburgtheaters}{}\ledrightnote{\textcolor{brown}{Burgtheater}}, das hat Ihnen noch gefehlt, daſs Sie
               anonyme Schmähkarten an anſtändige ſich das Brod mühſelig verdienende deutſche
               Dichter ſenden, die zeitlebens gegen die Macht des Kapitals, gegen die Über{\pb}hebung der Großen, gegen den am Mark des Volks zehrenden
               Adel und Militarismus gekämpft haben! Aber ich werde mich nicht abhalten laſſen. Das
               nächſte Jahr geht es nicht mehr gegen die \textcolor{green}{Infanterieleutenants}{}\ledrightnote{→\textcolor{green}{Lieutenant Gustl. Novelle}}, ſondern gegen die
               Cavallerieleutenants, insbeſondre gegen die in der Reſerve! –\pend
           \pstart
           Wie gehts Ihnen? Schade dſs {\pb}wir in \textcolor{pink}{I{\geminationn}sbruck}{}\ledrightnote{\textcolor{pink}{Innsbruck}} nur ſo aneinander
               vorübergesauſt und geſäuſelt ſind. \label{K_L01142_1v}\edtext{Ich
               bin jetzt in \textcolor{pink}{St. Anton}{}\ledrightnote{\textcolor{pink}{St. Anton am Arlberg}}}{\lemma{\textnormal{\emph{Ich … Anton}}}\Cendnote{\textnormal{\textcolor{blue}{Schnitzler} hielt sich von circa
                     4. 7. 1901 bis vermutlich 9. 7. 1901 in \textcolor{pink}{St. Anton am Arlberg} auf. Nachdem er an \textcolor{blue}{Richard Beer-Hofmann} am 4. 7. 1901
                  einen Brief mit teilweise ähnlichem Inhalt sandte, könnte dieses
                  Korrespondenzstück zeitnah entstanden sein.}}}\label{K_L01142_1h}, friere, und hoffe bald in den
               Süden zu radeln. In \textcolor{pink}{Salzburg}{}\ledrightnote{\textcolor{pink}{Salzburg}} hab ich gearbeitet,
               jetzt weniger. Laſſen Sie recht bald von ſich hören aber mehr. (An meine \textcolor{pink}{Wien}{}\ledrightnote{\textcolor{pink}{Wien}}er Adreſſe.) Die \textcolor{blue}{Schweſtern}{}\ledrightnote{→\textcolor{blue}{Olga Schnitzler}{\newline}→\textcolor{blue}{Elisabeth Steinrück}} grüßen Sie. Ich grüße Sie
               herzlich und bitte Sie auch Ihre {\pb}\textcolor{blue}{Frau}{}\ledrightnote{→\textcolor{blue}{Gertrude von Hofmannsthal}} zu grüßen.\pend
           \pstart
           Ihr{\\}\spacefill\mbox{Arthur}\pend
           \endnumbering\briefempfaengerindex{Hofmannsthal, Hugo von@\textsc{Hofmannsthal, Hugo von}!zzzSchnitzler, Arthur@\emph{von Arthur Schnitzler}!1901-07-043@{{[}4.? 7. 1901{]}}|)be}\mylabel{h}  \normalsize

\doendnotes{C}
\bigskip
\vfill

\clearpage

\footnotesize

\lohead{\textsc{register}}

% Definiere theindex-Environment komplett neu ohne reledmac
\makeatletter
\renewenvironment{theindex}{%
  \section*{\indexname}%
  \setlength{\parindent}{0pt}%
  \setlength{\parskip}{0pt plus 0.3pt}%
  \let\item\@idxitem
}{%
  \clearpage
}
\makeatother

\IfFileExists{\jobname-pw.ind}{\input{\jobname-pw.ind}}{}

\end{document}

      