%% latex-korrekturansicht-vorspann.tex
%% Vorspann für die Korrekturansicht.
%% Lädt die gemeinsame Datei latex-vorspann.tex mit gesetztem Schalter.

\newif\ifkorrekturansicht
\korrekturansichttrue

\input{../tex-inputs/latex-vorspann}


               \section[Paul Goldmann an Arthur Schnitzler, 21. 10. 1889]{ Paul Goldmann an Arthur Schnitzler, 21. 10. 1889}\nopagebreak\mylabel{v}\rehead{ }\normalsize\beginnumbering\briefempfaengerindex{Schnitzler, Arthur@\textsc{Schnitzler, Arthur}!zzzGoldmann, Paul@\emph{von Paul Goldmann}!1889-10-211@{21. 10. 1889}|(be} \toendnotes[C]{\smallbreak\pagebreak[2]} \Standort{DLA, A:Schnitzler, HS.NZ85.1.3162.}
\physDesc{Brief, 1 Blatt, 2 Seiten
\newline{}Handschrift: blaue Tinte, deutsche Kurrent
\newline{}Schnitzler: mit rotem Buntstift eine Unterstreichung }\toendnotes[C]{\smallbreak}\pstart
           \noindent{}\centering{}{\pb}\textcolor{gray}{\textbf{\textbf{Adminiſtration: \textcolor{pink}{VII.
                           Seidengaſſe 7}{}\ledrightnote{\textcolor{pink}{Seidengasse}}} (\textcolor{brown}{Jos. Eberle {\kaufmannsund} Co.}{}\ledrightnote{\textcolor{brown}{Josef Eberle  Stein-, Buch und Musikaliendruckerei}})}}\pend
           \pstart
           \noindent{}\centering{}\textcolor{gray}{\textbf{\textcolor{brown}{An der Schönen Blauen Donau}{}\ledrightnote{\textcolor{brown}{An der schönen blauen Donau}}}}\pend
           \pstart
           \noindent{}\centering{}\textcolor{gray}{\textbf{Chef-Redacteur: Dr. \textcolor{blue}{F.
                        Mamroth}{}\ledrightnote{\textcolor{blue}{Fedor Mamroth}}. – Redaction: \textcolor{pink}{IX.,
                        Berggaſſe 31}{}\ledrightnote{\textcolor{pink}{Berggasse}}.}}\pend
           \pstart
           \raggedleft{}\textcolor{gray}{\textbf{\textcolor{pink}{Wien}{}\ledrightnote{\textcolor{pink}{Wien}}, den}}{ }21. October \textcolor{gray}{\textbf{18}}89.\pend
           \pstart\center{}Lieber Herr Doctor!\pend\pstart
           Ich habe den \label{K_L02644-1v}\edtext{\textcolor{green}{Beitrag}{}\ledrightnote{→\textcolor{green}{?? [Abgelehnte Erzählung für An der schönen blauen Donau]}}}{\lemma{\textnormal{\emph{Beitrag}}}\Cendnote{\textnormal{nicht ermittelt}}}\label{K_L02644-1h} Ihres unbekannten
                  \label{K_L02644-2v}\edtext{\textcolor{blue}{Freundes}{}\ledrightnote{→\textcolor{blue}{?? [Verfasser einer abgelehnten Erzählung, 1889]}}}{\lemma{\textnormal{\emph{Freundes}}}\Cendnote{\textnormal{nicht identifiziert}}}\label{K_L02644-2h} mit lebhaftem
               Intereſſe geleſen. Es ſteckt viel Talent in der kleinen Arbeit – ſie\strikeout{\textcolor{gray}{×}} iſt warm und poetiſch empfunden und nicht ohne Gewand\textcolor{gray}{t}heit
               dargeſtellt. Ich hätte ſie gern in unſerem \textcolor{green}{Allerfeelen-Heft}{}\ledrightnote{→\textcolor{green}{An der schönen blauen Donau}} veröffentlicht. Aber leider füllt die \textcolor{green}{Erzählung}{}\ledrightnote{→\textcolor{green}{?? [Abgelehnte Erzählung für An der schönen blauen Donau]}} nicht den vierten
               Theil des räumlichen Ausmaßes aus, das – nach den techniſchen Principien unferes {\pb}\textcolor{green}{Blattes}{}\ledrightnote{→\textcolor{green}{An der schönen blauen Donau}} – ein Feuilleton
               aufweiſen muß. Mit einem Worte: Die hübſche \textcolor{green}{Arbeit}{}\ledrightnote{→\textcolor{green}{?? [Abgelehnte Erzählung für An der schönen blauen Donau]}} iſt zu klein für uns. Vielleicht wächſt ſie ſich bis
               zum nächſten Allerſeelen ein wenig aus. Inzwiſchen
               aber wäre ich Ihnen dankbar, wenn Sie mir bei Gelegenheit eine andere Arbeit von
               Ihrem Schützling verſchaffen wollten. Der junge \textcolor{blue}{Mann}{}\ledrightnote{→\textcolor{blue}{?? [Verfasser einer abgelehnten Erzählung, 1889]}} intereſſirt mich{\dots}\pend
           \pstart
           Ich begrüße Sie herzlichſt! {\\[\baselineskip]}Ihr ergebener {\\[\baselineskip]}\spacefill\mbox{Dr. Paul Goldmann.}\pend
           \leftskip=0em{}\endnumbering\briefempfaengerindex{Schnitzler, Arthur@\textsc{Schnitzler, Arthur}!zzzGoldmann, Paul@\emph{von Paul Goldmann}!1889-10-211@{21. 10. 1889}|)be}\mylabel{h}  \normalsize

\doendnotes{C}
\bigskip
\vfill

\clearpage

\footnotesize

\lohead{\textsc{register}}

% Definiere theindex-Environment komplett neu ohne reledmac
\makeatletter
\renewenvironment{theindex}{%
  \section*{\indexname}%
  \setlength{\parindent}{0pt}%
  \setlength{\parskip}{0pt plus 0.3pt}%
  \let\item\@idxitem
}{%
  \clearpage
}
\makeatother

\IfFileExists{\jobname-pw.ind}{\input{\jobname-pw.ind}}{}

\end{document}

      