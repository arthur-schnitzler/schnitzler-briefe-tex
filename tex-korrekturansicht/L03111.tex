%% latex-korrekturansicht-vorspann.tex
%% Vorspann für die Korrekturansicht.
%% Lädt die gemeinsame Datei latex-vorspann.tex mit gesetztem Schalter.

\newif\ifkorrekturansicht
\korrekturansichttrue

\input{../tex-inputs/latex-vorspann}


\renewcommand{\erwaehntePersonen}{Personen: Hermann Bahr, Richard Beer-Hofmann, Marie Glümer, Hugo von Hofmannsthal, Paul Horn, Gustav Schwarzkopf}
\renewcommand{\erwaehnteOrte}{Orte: Bad Ischl, Berghof, Café Kremser, Deutschland, Unterach am Attersee, Wien}
\renewcommand{\erwaehnteWerke}{}
\section[Felix Salten an Arthur Schnitzler, 8. 8. 1892]{Felix Salten an Arthur Schnitzler, 8. 8. 1892}
\nopagebreak\mylabel{v}
\rehead{ }\normalsize\beginnumbering\briefempfaengerindex{Schnitzler, Arthur@\textsc{Schnitzler, Arthur}!zzzSalten, Felix@\emph{von Felix Salten}!1892-08-081@{8. 8. 1892}|(be}
\toendnotes[C]{\smallbreak\pagebreak[2]}\Standort{CUL, Schnitzler, B 89, A 1.}
\physDesc{Brief, 1 Blatt, 4 Seiten, 857 Zeichen
\newline{}Handschrift: Bleistift, lateinische Kurrent
\newline{}Ordnung: mit Bleistift von unbekannter Hand nummeriert: »14« }
\buchAbdrucke{\weitereDrucke{Hermann Bahr, Arthur Schnitzler: \emph{Briefwechsel, Aufzeichnungen, Dokumente (1891–1931)}. Herausgegeben von Kurt Ifkovits und Martin Anton Müller. Göttingen: \emph{Wallstein} 2018, S. 80.} }\toendnotes[C]{\smallbreak}
\pstart
           \raggedleft{}{\pb}\textcolor{pink}{Unterach}{}\ledrightnote{\textcolor{pink}{Unterach am Attersee}}, 8. VIII. 92.\pend
           
\pstart
           Lieber Freund!{ }Samstag{ }Abend wollte ich ins \textcolor{pink}{Kremser}{}\ledrightnote{\textcolor{pink}{Café Kremser}} kommen u
               ihnen Adieu sagen, da ich erst Sonntag zu reisen
               gedachte. Allein um 8 Uhr Abd. erhielt ich meine Kleider und so fuhr ich
               also zur selbigen Stunde. Seien Sie also nicht böse. Hier ist’s
                  wunderschön\textcolor{gray}{,} u ich denke oft an Sie u. an Ihre Arbeiten.
               Schreiben Sie mir, bitte, bald was Sie treiben. {\pb}Ich hoffe \textcolor{pink}{hier}{}\ledrightnote{{$\rightarrow$}\textcolor{pink}{Unterach am Attersee}} einiges arbeiten zu können, da man ganz
               ungezwungen lebt u tagelang allein sein kann. Nächste Woche will ich zu \textcolor{blue}{Richard}{}\ledrightnote{\textcolor{blue}{Richard Beer-Hofmann}} nach \textcolor{pink}{Ischl}{}\ledrightnote{\textcolor{pink}{Bad Ischl}} hinüber, und werde auch \textcolor{blue}{Loris}{}\ledrightnote{\textcolor{blue}{Hugo von Hofmannsthal}}
               davon verständigen. \textcolor{blue}{Paul{ }{\pb}Horn}{}\ledrightnote{\textcolor{blue}{Paul Horn}} soll heute{ }Nachmittag ankommen. Leben Sie wol u. schreiben Sie bald, auch wie es
               mit jenem \label{K_L03111-1v}\edtext{Engagement nach \textcolor{pink}{Deutschld}{}\ledrightnote{\textcolor{pink}{Deutschland}}}{\lemma{\textnormal{\emph{Engagement nach Deutschld}}}\Cendnote{\textnormal{für \textcolor{blue}{Marie Glümer}}}}\label{K_L03111-1h} steht.\pend
           
\pstart
           Ich werde übrigens auch bald wieder schreiben, sobald ich Ihnen künstlerisch ei{\pb}niges Neue zu sagen habe. Grüßen
               Sie \textcolor{blue}{Schwarzkopf}{}\ledrightnote{\textcolor{blue}{Gustav Schwarzkopf}} u. \textcolor{blue}{Bahr}{}\ledrightnote{\textcolor{blue}{Hermann Bahr}}.\pend
           
\pstart
           Herzlichst Ihr {\\[\baselineskip]}Treuester {\\[\baselineskip]}\spacefill\mbox{Salten}\pend
           \leftskip=0em{}
\pstart
           \noindent{}\textcolor{pink}{Unterach}{}\ledrightnote{\textcolor{pink}{Unterach am Attersee}}, \textcolor{pink}{Berghof}{}\ledrightnote{\textcolor{pink}{Berghof}}.\pend
           \endnumbering\briefempfaengerindex{Schnitzler, Arthur@\textsc{Schnitzler, Arthur}!zzzSalten, Felix@\emph{von Felix Salten}!1892-08-081@{8. 8. 1892}|)be}\mylabel{h}  \normalsize

\doendnotes{C}
\bigskip
\vfill

\clearpage

\footnotesize

\lohead{\textsc{register}}

% Definiere theindex-Environment komplett neu ohne reledmac
\makeatletter
\renewenvironment{theindex}{%
  \section*{\indexname}%
  \setlength{\parindent}{0pt}%
  \setlength{\parskip}{0pt plus 0.3pt}%
  \let\item\@idxitem
}{%
  \clearpage
}
\makeatother

\IfFileExists{\jobname-pw.ind}{\input{\jobname-pw.ind}}{}

\end{document}

      