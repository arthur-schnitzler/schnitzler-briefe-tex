%% latex-korrekturansicht-vorspann.tex
%% Vorspann für die Korrekturansicht.
%% Lädt die gemeinsame Datei latex-vorspann.tex mit gesetztem Schalter.

\newif\ifkorrekturansicht
\korrekturansichttrue

\input{../tex-inputs/latex-vorspann}


               \section[Therese Rie-Andro an Arthur Schnitzler, {[}Anfang Juli 1923{]}]{ Therese Rie-Andro an Arthur Schnitzler, {[}Anfang Juli 1923{]}}\nopagebreak\mylabel{v}\rehead{ }\normalsize\beginnumbering\briefempfaengerindex{Schnitzler, Arthur@\textsc{Schnitzler, Arthur}!zzzRie, Therese@\emph{von Therese Rie}!1923-07-011@{{[}Anfang Juli 1923{]}}|(be} \toendnotes[C]{\smallbreak\pagebreak[2]} \Standort{DLA, A:Schnitzler, 85.1.4310.}
\physDesc{Brief, 1 Blatt, 3 Seiten
\newline{}Handschrift: Bleistift, lateinische Kurrent
\newline{}Schnitzler: 1) mit Bleistift beschriftet: »\textsc{Andro.}«, datiert: »Juli 23« 2) mit rotem Buntstift sechs Unterstreichungen}\pstart
           \noindent{}\raggedleft{}{\pb}\textcolor{pink}{Bernried/Starnbergerseee}{}\ledrightnote{\textcolor{pink}{Bernried}}\pend
           \pstart
           \noindent{}\raggedleft{}\textcolor{pink}{Oberbayern}{}\ledrightnote{\textcolor{pink}{Oberbayern}}\pend
           \pstart
           \noindent{}\raggedleft{}\textcolor{pink}{Altwirt}{}\ledrightnote{\textcolor{pink}{Hotel Seeblick}}\pend
           \pstart{}Verehrter Herr Doktor,\pend\pstart
           Dieser Ort iſt so lieb, ſtill und schön, daſs ich Ihnen von da einen Gruß schicken
                    muß. Vielleicht finden Sie diese Logik nicht zwingend, aber für mich beſteht sie
                    doch. Wahrscheinlich entſpringt sie aus dem Wunsch, daß Sie für Ihre Erholung
                    einen Platz finden möchten, der Ihren Neigungen ebenso entspricht, wie dieser
                    hier den meinen – wo es nichts gibt als See und herrlich bewaldete Ufer und gar
                    keine Städter und die netteſten Schafe, Ziegen und Gänse und gar keine Tinte.
                        {\pb}Das einzige Tintenfaß in der Gegend befindet sich
                    auf de\substVorne{}\textsuperscript{m}\substDazwischen{}r\substHinten{} »Amtstube« des Bürgermeiſters, der mir, als ich mich bei meinem erſten
                    Aufenthalt – ich war ſchon öfters hier – sagte, als ich mich als Ausländerin
                    melden wollte: »Sö san do ka Ausländer, sö reden do wie mir; a \textcolor{pink}{Saupreuß}{}\ledrightnote{\textcolor{pink}{Deutschland}}, \uline{des} is a
                    Ausländer!!«\pend
           \pstart
           Und als ich diesmal sagte, ich käme jetzt selten ins \textcolor{pink}{Reich}{}\ledrightnote{\textcolor{pink}{Deutschland}}, meinte er: »Ja ja, ich ko{\geminationm} auch
                    selten hin!« – – Und das alles gibts \uline{wirklich}
                    und es ist nicht von \textcolor{blue}{Ludwig Thoma}{}\ledrightnote{\textcolor{blue}{Ludwig Thoma}} und es iſt
                    eine Stunde von \textcolor{pink}{München}{}\ledrightnote{\textcolor{pink}{München}}, wo {\pb}es so übel knirscht, daſs man der nächsten Entwicklung
                    der Dinge nur mit Besorgnis folgt.\pend
           \pstart
           Und nun alle guten So{\geminationm}erwünſche für Sie!\pend
           \pstart
           Ihre{\\[\baselineskip]}\spacefill\mbox{Therese Rie.}\pend
           \leftskip=0em{}\endnumbering\briefempfaengerindex{Schnitzler, Arthur@\textsc{Schnitzler, Arthur}!zzzRie, Therese@\emph{von Therese Rie}!1923-07-011@{{[}Anfang Juli 1923{]}}|)be}\mylabel{h}  \normalsize

\doendnotes{C}
\bigskip
\vfill

\clearpage

\footnotesize

\lohead{\textsc{register}}

% Definiere theindex-Environment komplett neu ohne reledmac
\makeatletter
\renewenvironment{theindex}{%
  \section*{\indexname}%
  \setlength{\parindent}{0pt}%
  \setlength{\parskip}{0pt plus 0.3pt}%
  \let\item\@idxitem
}{%
  \clearpage
}
\makeatother

\IfFileExists{\jobname-pw.ind}{\input{\jobname-pw.ind}}{}

\end{document}

      