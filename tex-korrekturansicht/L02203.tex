%% latex-korrekturansicht-vorspann.tex
%% Vorspann für die Korrekturansicht.
%% Lädt die gemeinsame Datei latex-vorspann.tex mit gesetztem Schalter.

\newif\ifkorrekturansicht
\korrekturansichttrue

\input{../tex-inputs/latex-vorspann}


               \section[Arthur Schnitzler an Hermann Bahr, 9. 2. 1915]{ Arthur Schnitzler an Hermann Bahr, 9. 2. 1915}\nopagebreak\mylabel{v}\rehead{ }\normalsize\beginnumbering\briefempfaengerindex{Bahr, Hermann@\textsc{Bahr, Hermann}!zzzSchnitzler, Arthur@\emph{von Arthur Schnitzler}!1915-02-091@{9. 2. 1915}|(be} \toendnotes[C]{\smallbreak\pagebreak[2]} \Standort{TMW, HS AM 60138 Ba.}
\physDesc{Briefkarte
\newline{}Handschrift: schwarze Tinte, deutsche Kurrent}\buchAbdrucke{\weitereDrucke{1) \emph{9. 2. 1915.} In: Arthur Schnitzler: \emph{The Letters of Arthur Schnitzler to Hermann Bahr}. Edited, annotated, and with an introduction, by Donald G.
                        Daviau. Chapel Hill: \emph{The University of North Carolina Press} 1978, S. 114 (University of North Carolina studies in the Germanic languages
                        and literatures, 89).} \weitereDrucke{2) Hermann Bahr, Arthur Schnitzler: \emph{Briefwechsel, Aufzeichnungen, Dokumente (1891–1931)}. Hg. Kurt Ifkovits und Martin Anton Müller. Göttingen: \emph{Wallstein} 2018, S. 497.} }\toendnotes[C]{\smallbreak}\pstart
           \noindent{}{\pb}\textcolor{gray}{\textbf{Dr. Arthur Schnitzler}}\hfill 9. 2. 915\pend
           \pstart
           \textcolor{gray}{\textbf{\textcolor{pink}{Wien XVIII. Sternwartestrasse 71}{}\ledrightnote{\textcolor{pink}{Sternwartestraße}}}}\pend
           \pstart
           lieber Hermann, der Buchhändler \textcolor{blue}{Heller}{}\ledrightnote{\textcolor{blue}{Hugo Heller}} theilt mir mit daſs er deiner verehrten \textcolor{blue}{Gattin}{}\ledrightnote{→\textcolor{blue}{Anna Bahr-Mildenburg}}{ }\label{K_L02203_1v}\edtext{geſchrieben}{\lemma{\textnormal{\emph{geſchrieben}}}\Cendnote{\textnormal{am 6. 2. 1915 (\emph{Theatermuseum Wien}, AM 27957 BaM)}}}\label{K_L02203_1h}, ob ſie
               hier nicht zu einem \label{K_L02203_2v}\edtext{wohlthätigen
                  Zwecke}{\lemma{\textnormal{\emph{wohlthätigen
                  Zwecke}}}\Cendnote{\textnormal{vgl. A. S.: \emph{Tagebuch}, 13. 12. 1915}}}\label{K_L02203_2h}{ }\textcolor{blue}{Schubert}{}\ledrightnote{\textcolor{blue}{Franz Peter Schubert}} Lieder ſingen möchte – und da ich
               daraufhin mich begreiflicherweiſe äußerte: das möcht ich gern hören, – bittet er
               mich, \strikeout{als} dieſen Wunſch, dieſe Sehnſucht {\pb}(ich theile ſie
               wahrſcheinlich mit vielen) dir direct zu übermitteln. Das thu ich – in der Empfindung
               etwas unbeſcheiden – aber doch deiner Nachſicht gewiſs zu sein. Im übrigen wär es,
               auch abgeſehn von den \textcolor{blue}{Schubert}{}\ledrightnote{\textcolor{blue}{Franz Peter Schubert}} Liedern, die deine
                  \textcolor{blue}{Frau}{}\ledrightnote{→\textcolor{blue}{Anna Bahr-Mildenburg}}{ }ſo herrlich ſingen ſoll, ſchön, we{\geminationn} man ſich wieder einmal ſehen und ſprechen kö{\geminationn}te – in dieſer – Zeit, für die das Adjectiv doch erſt
               gefunden werden müſſte!\pend
           \pstart
           Von Herzen mit Grüßen von Haus zu Haus{\\[\baselineskip]}dein \spacefill\mbox{Arthur}\pend
           \leftskip=0em{}\endnumbering\briefempfaengerindex{Bahr, Hermann@\textsc{Bahr, Hermann}!zzzSchnitzler, Arthur@\emph{von Arthur Schnitzler}!1915-02-091@{9. 2. 1915}|)be}\mylabel{h}  \normalsize

\doendnotes{C}
\bigskip
\vfill

\clearpage

\footnotesize

\lohead{\textsc{register}}

% Definiere theindex-Environment komplett neu ohne reledmac
\makeatletter
\renewenvironment{theindex}{%
  \section*{\indexname}%
  \setlength{\parindent}{0pt}%
  \setlength{\parskip}{0pt plus 0.3pt}%
  \let\item\@idxitem
}{%
  \clearpage
}
\makeatother

\IfFileExists{\jobname-pw.ind}{\input{\jobname-pw.ind}}{}

\end{document}

      