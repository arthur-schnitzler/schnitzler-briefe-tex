%% latex-korrekturansicht-vorspann.tex
%% Vorspann für die Korrekturansicht.
%% Lädt die gemeinsame Datei latex-vorspann.tex mit gesetztem Schalter.

\newif\ifkorrekturansicht
\korrekturansichttrue

\input{../tex-inputs/latex-vorspann}


               \section[Arthur Schnitzler an Richard Beer-Hofmann, 2. 7. 1894]{ Arthur Schnitzler an Richard Beer-Hofmann, 2. 7. 1894}\nopagebreak\mylabel{v}\rehead{ }\normalsize\beginnumbering\briefempfaengerindex{Beer-Hofmann, Richard@\textsc{Beer-Hofmann, Richard}!zzzSchnitzler, Arthur@\emph{von Arthur Schnitzler}!1894-07-021@{2. 7. 1894}|(be} \toendnotes[C]{\smallbreak\pagebreak[2]} \Standort{YCGL, MSS 31.}
\physDesc{Brief, 2 Blätter, 7 Seiten, Umschlag
\newline{}Handschrift: Bleistift, deutsche Kurrent\newline{}Versand: 1) Stempel: »\nobreak{}\oindex{IX., Alsergrund@\textbf{IX., Alsergrund}, \emph{Bezirk (A.BZK)}|pwk}Wien 9/3, 2. 7. 94\nobreak{}«.  2) Stempel: »\nobreak{}\oindex{Bad Ischl@\textbf{Bad Ischl}, \emph{Besiedelter Ort (A.BSO)}|pwk}Ischl, 3. 7. 94, 7 F\nobreak{}«. }\buchAbdrucke{\weitereDrucke{1) Arthur Schnitzler, Richard Beer-Hofmann: \emph{Briefwechsel 1891–1931}. Hg. Konstanze Fliedl. Wien, Zürich: \emph{Europaverlag} 1992, S. 56–57.} \weitereDrucke{2) Hermann Bahr, Arthur Schnitzler: \emph{Briefwechsel, Aufzeichnungen, Dokumente
                                (1891–1931)}. Hg. Kurt Ifkovits und Martin Anton Müller. Göttingen: \emph{Wallstein} 2018.} }\toendnotes[C]{\smallbreak}\pstart{}{\pb}Herrn \textsc{Dr. Rich.
                            Beer-Hofmann}\pend{}\pstart{}\textsc{\textcolor{pink}{Ischl}{}\ledrightnote{\textcolor{pink}{Bad Ischl}}}\pend{}\pstart{}\textsc{\textcolor{pink}{Egelmoos 22.}{}\ledrightnote{\textcolor{pink}{Eglmoosgasse}}}\pend{}{\bigskip}\pstart{}{\pb}Lieber Richard,\pend\pstart
           das \textsc{Cachenez} hoffentlich nach Wunſch besorgt. \textcolor{brown}{\textsc{Stoll}}{}\ledrightnote{\textcolor{brown}{Stoll {\kaufmannsund} Uhlig}}{ }ſchickt’s noch
                    heute, ni{\geminationm}t es auf Verlangen auch wieder zurück;
                    ich finde es ſehr ſchön, was keine Suggeſtion ſein ſoll. –\pend
           \pstart
           {\pb}Gratulation ſchicken Sie in die \textcolor{pink}{Frankgaſſe}{}\ledrightnote{\textcolor{pink}{Frankgasse}}, und, \uline{wenn
                        Sie die \textcolor{blue}{Braut}{}\ledrightnote{→\textcolor{blue}{Helene Schnitzler}} kennen}, auch auf
                    den \textcolor{pink}{Lobkowitzplatz}{}\ledrightnote{\textcolor{pink}{Lobkowitzplatz}}. –\pend
           \pstart
           Ich dürfte 13., 14., 15. nach \textcolor{pink}{Iſchl}{}\ledrightnote{\textcolor{pink}{Bad Ischl}} ko{\geminationm}en, bleibe
                    bis 20. und denke da{\geminationn} mit Ihnen u
                        \textcolor{blue}{\textsc{Bahr}}{}\ledrightnote{\textcolor{blue}{Hermann Bahr}}, der uns abholt, nach \textsc{\textcolor{pink}{Salzburg}{}\ledrightnote{\textcolor{pink}{Salzburg}}} zu fahren, {\pb}wohin auch \textcolor{blue}{Hugo}{}\ledrightnote{\textcolor{blue}{Hugo von Hofmannsthal}} von der \textsc{\textcolor{pink}{Fusch}{}\ledrightnote{\textcolor{pink}{Bad Fusch}}} aus ko{\geminationm}en wird. Ich denke, ſo iſt’s gut?
                    –\pend
           \pstart
           \textcolor{blue}{Hugo}{}\ledrightnote{\textcolor{blue}{Hugo von Hofmannsthal}} war Freitag früh auf der Durchreiſe von
                    der \textcolor{pink}{Saleſianergaſſe}{}\ledrightnote{\textcolor{pink}{Salesianergasse}} nach \textcolor{pink}{Döbling}{}\ledrightnote{\textcolor{pink}{XIX., Döbling}} bei mir. –\pend
           \pstart
           Was macht der \textcolor{green}{Götterliebling}{}\ledrightnote{\textcolor{green}{Der Tod Georgs}}? – Ich bin nicht
                        un{\pb}fleißig. \textcolor{blue}{Paul
                        Schulz}{}\ledrightnote{\textcolor{blue}{Paul Schulz}} und die \textcolor{blue}{Kapper’s}{}\ledrightnote{\textcolor{blue}{Friedrich Kapper}{\newline}\textcolor{blue}{Adele Kapper}} laſſen Sie nur alle wie ſie ſind – wenn wir alle
                    Menſchen ändern könnten wie wir wollen, ſo würden ſie uns – ſchrecklich zuwider
                    werden. (Denken Sie nicht drüber nach; es iſt ausſichtslos. Der obige Satz ist
                    nemlich {\pb}in mannigfacher Weiſe zu beenden.)\pend
           \pstart
           Neulich waren \textcolor{blue}{\textsc{Fels}}{}\ledrightnote{\textcolor{blue}{Friedrich Michael Fels}}
                    und \textcolor{blue}{\textsc{Korff}}{}\ledrightnote{\textcolor{blue}{Heinrich von Korff}} auf
                    einmal bei mir. –\pend
           \pstart
           Ich zerbreche mir den Kopf, warum Sie mir geſchrieben haben; ob wegen \textcolor{blue}{Kapper}{}\ledrightnote{\textcolor{blue}{Friedrich Kapper}} oder wegen \textcolor{blue}{Schulz}{}\ledrightnote{\textcolor{blue}{Paul Schulz}} oder wegen meines \textcolor{blue}{Bruders}{}\ledrightnote{→\textcolor{blue}{Julius Schnitzler}}? – Einen Augenblick hatte ich nemlich den ſchändlichen Ver{\pb}dacht, dß – das ſchwarze, ſchwere, weiche, matte
                    Cachenez – Ihres Briefes »erste Schuld und Urſach« wäre. (Ko{\geminationm}t nirgends vor. Wenn man ſich ſchämt, macht man
                    Anführungszeichen.)\pend
           \pstart
           Leben Sie wohl. Ich freue {\pb}mich nicht aufs
                    Siegeln, obwohl ich mehr Grund dazu habe wie Sie. –\pend
           \pstart
           Schreiben Sie mir bald wieder. Herzlichen Gruß{\\[\baselineskip]}Ihr{\\[\baselineskip]}\spacefill\mbox{Arthur}\pend
           \leftskip=0em{}\pstart
           2. Juli 94. \textsc{\textcolor{pink}{Wien}{}\ledrightnote{\textcolor{pink}{Wien}}}\pend
           \endnumbering\briefempfaengerindex{Beer-Hofmann, Richard@\textsc{Beer-Hofmann, Richard}!zzzSchnitzler, Arthur@\emph{von Arthur Schnitzler}!1894-07-021@{2. 7. 1894}|)be}\mylabel{h}  \normalsize

\doendnotes{C}
\bigskip
\vfill

\clearpage

\footnotesize

\lohead{\textsc{register}}

% Definiere theindex-Environment komplett neu ohne reledmac
\makeatletter
\renewenvironment{theindex}{%
  \section*{\indexname}%
  \setlength{\parindent}{0pt}%
  \setlength{\parskip}{0pt plus 0.3pt}%
  \let\item\@idxitem
}{%
  \clearpage
}
\makeatother

\IfFileExists{\jobname-pw.ind}{\input{\jobname-pw.ind}}{}

\end{document}

      