%% latex-korrekturansicht-vorspann.tex
%% Vorspann für die Korrekturansicht.
%% Lädt die gemeinsame Datei latex-vorspann.tex mit gesetztem Schalter.

\newif\ifkorrekturansicht
\korrekturansichttrue

\input{../tex-inputs/latex-vorspann}


               \section[Georg Brandes an Arthur Schnitzler, 30. 4. 1900]{ Georg Brandes an Arthur Schnitzler, 30. 4. 1900}\nopagebreak\mylabel{v}\rehead{ }\normalsize\beginnumbering\briefempfaengerindex{Schnitzler, Arthur@\textsc{Schnitzler, Arthur}!zzzBrandes, Georg@\emph{von Georg Brandes}!1900-04-301@{30. 4. 1900}|(be} \toendnotes[C]{\smallbreak\pagebreak[2]} \Standort{CUL, Schnitzler, B 17.}
\physDesc{Brief, 1 Blatt, 4 Seiten
\newline{}Handschrift: schwarze Tinte, lateinische Kurrent\newline{}Ordnung: mit Bleistift von unbekannter Hand nummeriert:
                                        »20« }\buchAbdrucke{\weitereDrucke{Georg Brandes, Arthur Schnitzler: \emph{Ein Briefwechsel}. Hg. Kurt Bergel. Bern: \emph{Francke} 1956, S. 80–81.} }\toendnotes[C]{\smallbreak}\pstart
           \raggedleft{}{\pb}\textcolor{pink}{Kommunehospitalet}{}\ledrightnote{\textcolor{pink}{Kommunehospitalet}}\pend
           \pstart
           \raggedleft{}Kopenhagen\pend
           \pstart
           \raggedleft{}30 April 1900\pend
           \pstart{}Verehrter Freund\pend\pstart
           Sie wundern sich vielleicht, gar nicht von mir gehört zu haben, da wir doch
                    verabredet hatten, uns zu treffen und uns jedenfalls in \textcolor{pink}{Wien}{}\ledrightnote{\textcolor{pink}{Wien}} zu sehen. Aber eben wie ich eine Reise auf Kosten des
                        \textcolor{pink}{ungarischen}{}\ledrightnote{\textcolor{pink}{Ungarn}}
               Staats durch die \textcolor{pink}{ungarischen}{}\ledrightnote{\textcolor{pink}{Ungarn}} Provinzen antreten sollte, kam
                    meine alte Krankheit, die Venenentzündung, wieder, ich lag erst 3–4 Tage im
                    Hotel reiste dann nach \textcolor{pink}{Kopenhagen}{}\ledrightnote{\textcolor{pink}{Kopenhagen}} und habe
                    also den ganzen Monat verloren. Ich habe mich ins Hospital eingelegt um
                    sorgfältige Pflege zu haben, die Entzündung schien schon zwei Mal erloschen, kam
                    aber dann wieder. Ich liege also vorläufig in dieser gelinden Tortur, das Bein
                    hoch und in der Schiene {\pb}auf
                    dem Rücken immer in derselben Lage ohne mich weder rechts noch links drehen zu
                    können.\pend
           \pstart
           Dies ist der dritte Frühling, den ich nicht sehe (97,
                        99, 1900)\pend
           \pstart
           Die deutschen Blätter haben Dutzende von Schmähartikeln gegen mich enthalten,
                    weil ich in dem \textcolor{brown}{Klub
                        in Budapest}{}\ledrightnote{→\textcolor{brown}{Liberaler Club}}, aufgefordert, eine \textcolor{pink}{französische}{}\ledrightnote{\textcolor{pink}{Frankreich}} Einleitung zu machen (was mir lächerlich vorkam),
                    einfach sagte »Die Sprache, deren ich mich bediene ist nicht die Ihre und nicht
                    die meine, nicht Ihre Lieblingssprache und nicht die meine, doch es ist die,
                    worin wir uns am leichtesten verstehen.« Das wird ein \uline{hämischer} Angriff auf \textcolor{pink}{Deutschland}{}\ledrightnote{\textcolor{pink}{Deutschland}}
                    und die deutsche Kultur genannt. Und zwar von anonymen Bengeln, die nicht mehr
                    Antheil an die deutsche Kultur haben als ein alter Stiefel. \strikeout{V}Die Verachtung, die ich für die Journalisten {\pb}hege, ist nach und nach so
                    gross, dass ich förmlich einen bitteren Geschmack im Munde davon habe, wenn ich
                    daran denke.\pend
           \pstart
           Ich bin Ihnen und \textcolor{blue}{Beer-Hoffmann}{}\ledrightnote{\textcolor{blue}{Richard Beer-Hofmann}} wie
                    gewöhnlich vielen Dank für \textcolor{pink}{Wien}{}\ledrightnote{\textcolor{pink}{Wien}}
               schuldig.\pend
           \pstart
           Sie beiden und \textcolor{blue}{Gomperz}{}\ledrightnote{\textcolor{blue}{Theodor Gomperz}}’s Haus und \textcolor{blue}{Lanckoronski}{}\ledrightnote{\textcolor{blue}{Karl Lanckoroński}} waren dies mal mein \textcolor{pink}{Wien}{}\ledrightnote{\textcolor{pink}{Wien}}. Ich habe Sie sehr lieb und freue mich,
                    dass wir Freunde sind.\pend
           \pstart
           Ich las jetzt im Bett einige Bücher: \textcolor{green}{\uline{Drames de famille}}{}\ledrightnote{\textcolor{green}{Familiendramen}}, die \textcolor{blue}{Bourget}{}\ledrightnote{\textcolor{blue}{Paul Bourget}} mir schickte trotzdem
                    er so katholisch geworden ist; die zwei grossen Erzählungen, die in unsern \textcolor{pink}{nordischen}{}\ledrightnote{→\textcolor{pink}{Skandinavien}} Blättern übel
                    besprochen werden, gefielen mir sehr, wenn auch nicht die moralisirende
                    Schreibweise, doch Stoff und Ausführung. Dann las ich einen deutschen Roman, der
                    mir geschickt wurde und der mir gut scheint, \textcolor{blue}{Wilhelm Hegeler}{}\ledrightnote{\textcolor{blue}{Wilhelm Hegeler}}, \textcolor{green}{\uline{Ingenieur Horstmann}}{}\ledrightnote{\textcolor{green}{Ingenieur Horstmann}}, eine {\pb}sehr tüchtige
                    Leistung. Mit Interesse las ich \textcolor{blue}{Balzacs}{}\ledrightnote{\textcolor{blue}{Honoré de Balzac}}{ }\textcolor{green}{Briefe À L’Etrangère}{}\ledrightnote{\textcolor{green}{Lettres à l’étrangère (1833–1842), (1842–1844)}} d. h. an seine
                    zukünftige \textcolor{blue}{Frau}{}\ledrightnote{→\textcolor{blue}{Ewelina Hańska}} in der
                    ersten \label{K_L01033_1v}\edtext{\uline{vollständigen} Ausgabe}{\lemma{\textnormal{\emph{vollständigen Ausgabe}}}\Cendnote{\textnormal{\textcolor{blue}{H. de Balzac}: \emph{Œuvres posthumes} I. \emph{\textcolor{green}{Lettres à
                                l’étrangère (1833–1842)}}. Paris: \emph{\textcolor{brown}{Calmann-Lévy, Éditeurs}}{ }{[}1899{]}.}}}\label{K_L01033_1h}.\pend
           \pstart
           Es war amüsant, den Ton in \textcolor{blue}{Lanckoronskis}{}\ledrightnote{\textcolor{blue}{Karl Lanckoroński}}{ }\textcolor{green}{\uline{Rund um die Welt}}{}\ledrightnote{\textcolor{green}{Rund um die Erde 1888–89}} mit dem in unseres Freundes \textcolor{green}{\textcolor{blue}{Goldmann}{}\ledrightnote{\textcolor{blue}{Paul Goldmann}}’s}{}\ledrightnote{→\textcolor{green}{Ein Sommer in China}} zu vergleichen. \textcolor{blue}{Goldmann}{}\ledrightnote{\textcolor{blue}{Paul Goldmann}} hat mehr Geist und Herz, der
                    Graf hat viel mehr gesehen und (erstaunlich!) er hat darin ein gutes lyrisches
                    Gedicht geschrieben.\pend
           \pstart
           Es ist trist, so oft und lange krank zu sein. Ich bin ganz ausser Stande, irgend
                    eine ordentliche Arbeit vorzunehmen, meine tägliche Arbeit besteht allein darin,
                    die Ausgabe meiner \textcolor{green}{Sämmtlichen
                        Schriften}{}\ledrightnote{→\textcolor{green}{Samlede Skrifter [Gesammelte Werke]}} zu verbessern und zu corrigiren.\pend
           \pstart
           Ihr Freund{\\[\baselineskip]}\spacefill\mbox{Georg Brandes}\pend
           \leftskip=0em{}\endnumbering\briefempfaengerindex{Schnitzler, Arthur@\textsc{Schnitzler, Arthur}!zzzBrandes, Georg@\emph{von Georg Brandes}!1900-04-301@{30. 4. 1900}|)be}\mylabel{h}  \normalsize

\doendnotes{C}
\bigskip
\vfill

\clearpage

\footnotesize

\lohead{\textsc{register}}

% Definiere theindex-Environment komplett neu ohne reledmac
\makeatletter
\renewenvironment{theindex}{%
  \section*{\indexname}%
  \setlength{\parindent}{0pt}%
  \setlength{\parskip}{0pt plus 0.3pt}%
  \let\item\@idxitem
}{%
  \clearpage
}
\makeatother

\IfFileExists{\jobname-pw.ind}{\input{\jobname-pw.ind}}{}

\end{document}

      