%% latex-korrekturansicht-vorspann.tex
%% Vorspann für die Korrekturansicht.
%% Lädt die gemeinsame Datei latex-vorspann.tex mit gesetztem Schalter.

\newif\ifkorrekturansicht
\korrekturansichttrue

\input{../tex-inputs/latex-vorspann}


               \section[Arthur Schnitzler an Hugo von Hofmannsthal, 12. 3. 1897]{ Arthur Schnitzler an Hugo von Hofmannsthal,
                    12. 3. 1897}\nopagebreak\mylabel{v}\rehead{ }\normalsize\beginnumbering\briefempfaengerindex{Hofmannsthal, Hugo von@\textsc{Hofmannsthal, Hugo von}!zzzSchnitzler, Arthur@\emph{von Arthur Schnitzler}!1897-03-121@{12. 3. 1897}|(be} \toendnotes[C]{\smallbreak\pagebreak[2]} \Standort{FDH, Hs-30885,55.}
\physDesc{Brief, 1 Blatt, 4 Seiten
\newline{}Handschrift: Bleistift, deutsche Kurrent}\buchAbdrucke{\weitereDrucke{1) Hugo von Hofmannsthal, Arthur Schnitzler: \emph{Briefwechsel}. Hg. Therese Nickl und Heinrich Schnitzler. Frankfurt am Main: \emph{S. Fischer} 1964, S. 78.} \weitereDrucke{2) Hermann Bahr, Arthur Schnitzler: \emph{Briefwechsel, Aufzeichnungen, Dokumente
                                (1891–1931)}. Hg. Kurt Ifkovits und Martin Anton Müller. Göttingen: \emph{Wallstein} 2018, S. 137.} }\toendnotes[C]{\smallbreak}\pstart
           \noindent{}{\pb}\textcolor{gray}{\textbf{»\textcolor{brown}{Die
                                    Zeit}{}\ledrightnote{\textcolor{brown}{Die Zeit. Wiener Wochenschrift}}«}}\hfill \textcolor{gray}{\textbf{\textbf{\textcolor{pink}{Wien}{}\ledrightnote{\textcolor{pink}{Wien}}},
                                den }}12/3 97\pend
           \pstart
           \textcolor{gray}{\textbf{Wiener Wochenſchrift}}\hfill \textcolor{gray}{\textbf{\textcolor{pink}{IX/3, Günthergaſſe 1}{}\ledrightnote{\textcolor{pink}{Günthergasse}}.}}\pend
           \pstart
           \textcolor{gray}{\textbf{\textbf{Herausgeber}:}}{\\}\textcolor{gray}{\textbf{Profeſſor Dr. \textcolor{blue}{I. Singer}{}\ledrightnote{\textcolor{blue}{Isidor Singer}},}}\pend
           \pstart
           \textcolor{gray}{\textbf{\textcolor{blue}{Hermann Bahr}{}\ledrightnote{\textcolor{blue}{Hermann Bahr}}, Dr. \textcolor{blue}{Heinrich Kanner}{}\ledrightnote{\textcolor{blue}{Heinrich Kanner}}.}}\pend
           \pstart
           \textcolor{gray}{\textbf{Telephon Nr. 6415.}}\pend
           \pstart
           Lieber Hugo, vielleicht könnten Sie ſich doch entſchließen, bei
                    dieſer Veranſtaltung zu leſen. Ich thät es hundertmal lieber, wenn Sie dabei
                    wären. Das iſt natürlich kein Grund. Aber Sie wiſſen ganz gut, die Leute würd es
                    ſehr intereſſiren {\pb}und, wenn man ſchon von
                    ſolchen Sachen ſprechen ſoll, »ſchaden« werden Sie ſich nicht, ſondern die
                    Menſchen werden nur das Bedürfnis haben, Ihre Gedichte ſchön zu finden, auch
                        we{\geminationn}
               Sie ihnen nicht gefallen. Ich will jetzt
                    eben zu \textcolor{blue}{Hirſchfeld}{}\ledrightnote{\textcolor{blue}{Georg Hirschfeld}} gehen, daſs {\pb}er vielleicht auch vorlieſt – ſchon um das dumme
                    »Jung \textcolor{pink}{Wien}{}\ledrightnote{\textcolor{pink}{Wien}}« Geplauſch zu
                    paralyſiren. –\pend
           \pstart
           Antworten Sie mir vielleicht ein Wort.\pend
           \pstart
           Mir wäre eine Verſchiebung zum So{\geminationm}er lieb. Was ſoll
                        \uline{ich} denn leſen?\pend
           \pstart
           {\pb}Herzlich{\\[\baselineskip]}Ihr \spacefill\mbox{Arthur}\pend
           \leftskip=0em{}\pstart
           \noindent{}\textcolor{blue}{Bahr}{}\ledrightnote{\textcolor{blue}{Hermann Bahr}} grüßt Sie.\pend
           \pstart
           \noindent{}\textcolor{blue}{Hirſchfeld}{}\ledrightnote{\textcolor{blue}{Georg Hirschfeld}} ist einverſtanden.\pend
           \pstart
           \textcolor{gray}{\textbf{\label{T_L00649_1v}\edtext{Alle für »\textcolor{brown}{Die Zeit}{}\ledrightnote{\textcolor{brown}{Die Zeit. Wiener Wochenschrift}}« beſtimmten
                        Zuſchriften und Sendungen ſind an die Redaction der »\textcolor{brown}{Zeit}{}\ledrightnote{\textcolor{brown}{Die Zeit. Wiener Wochenschrift}}« und \textbf{nicht} an die
                        Perſon eines der Herausgeber zu richten.}{\lemma{\textnormal{\emph{Alle … richten.}}}\Cendnote{\textnormal{am unteren Rand der ersten Seite}}}\label{T_L00649_1h}}}\pend
           \endnumbering\briefempfaengerindex{Hofmannsthal, Hugo von@\textsc{Hofmannsthal, Hugo von}!zzzSchnitzler, Arthur@\emph{von Arthur Schnitzler}!1897-03-121@{12. 3. 1897}|)be}\mylabel{h}  \normalsize

\doendnotes{C}
\bigskip
\vfill

\clearpage

\footnotesize

\lohead{\textsc{register}}

% Definiere theindex-Environment komplett neu ohne reledmac
\makeatletter
\renewenvironment{theindex}{%
  \section*{\indexname}%
  \setlength{\parindent}{0pt}%
  \setlength{\parskip}{0pt plus 0.3pt}%
  \let\item\@idxitem
}{%
  \clearpage
}
\makeatother

\IfFileExists{\jobname-pw.ind}{\input{\jobname-pw.ind}}{}

\end{document}

      