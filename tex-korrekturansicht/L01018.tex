%% latex-korrekturansicht-vorspann.tex
%% Vorspann für die Korrekturansicht.
%% Lädt die gemeinsame Datei latex-vorspann.tex mit gesetztem Schalter.

\newif\ifkorrekturansicht
\korrekturansichttrue

\input{../tex-inputs/latex-vorspann}


               \section[Richard Beer-Hofmann an Arthur Schnitzler, 5. 3. 1900]{ Richard Beer-Hofmann an Arthur Schnitzler, 5. 3. 1900}\nopagebreak\mylabel{v}\rehead{ }\normalsize\beginnumbering\briefempfaengerindex{Schnitzler, Arthur@\textsc{Schnitzler, Arthur}!zzzBeer-Hofmann, Richard@\emph{von Richard Beer-Hofmann}!1900-03-051@{5. 3. 1900}|(be} \toendnotes[C]{\smallbreak\pagebreak[2]} \Standort{CUL, Schnitzler, B 8.}
\physDesc{Bildpostkarte
\newline{}Handschrift: schwarze Tinte, lateinische Kurrent\newline{}Versand: 1) Stempel: »\nobreak{}\oindex{Santa Maria Novella@\textbf{Santa Maria Novella}, \emph{Bahnhofsgebäude (K.BHF)}|pwk}Firenze Ferrovia, 5 3 {[}00{]}, 8 S\nobreak{}«.  2) Stempel: »\nobreak{}\oindex{IX., Alsergrund@\textbf{IX., Alsergrund}, \emph{Bezirk (A.BZK)}|pwk}{[}Wien 9/{]}3, 7. 3. 00, 8.V\nobreak{}«. \newline{}Ordnung: mit Bleistift von unbekannter Hand nummeriert: »152« }\pstart{}{\pb}D\textsuperscript{r}
                  Arthur Schnitzler\pend{}\pstart{}\textcolor{pink}{Wien}{}\ledrightnote{\textcolor{pink}{Wien}}\pend{}\pstart{}\textcolor{pink}{IX Frankgasse 1}{}\ledrightnote{\textcolor{pink}{Frankgasse}}\pend{}\pstart{}\textcolor{pink}{Austria}{}\ledrightnote{\textcolor{pink}{Österreich}}\pend{}{\bigskip}\pstart
           \noindent{}\centering{}\textcolor{gray}{\textbf{{\pb}R. \textcolor{pink}{Galleria Uffizi}{}\ledrightnote{\textcolor{pink}{Uffizien}}. – \textcolor{green}{L’incoronazione della Vergine}{}\ledrightnote{\textcolor{green}{Die Krönung der Jungfrau}} (\textcolor{blue}{Botticelli}{}\ledrightnote{\textcolor{blue}{Sandro Botticelli}}) \textcolor{pink}{Firenze}{}\ledrightnote{\textcolor{pink}{Florenz}}}}\pend
           \pstart
           \raggedleft{}5/III 900\pend
           \pstart
           Lieber Arthur! Ich denke am 11, vielleicht schon
                  10. Nachts wieder in \textcolor{pink}{Wien}{}\ledrightnote{\textcolor{pink}{Wien}} zu sein.
               Herzlich\pend
           \pstart Ihr \spacefill\mbox{R.}\pend{}\endnumbering\briefempfaengerindex{Schnitzler, Arthur@\textsc{Schnitzler, Arthur}!zzzBeer-Hofmann, Richard@\emph{von Richard Beer-Hofmann}!1900-03-051@{5. 3. 1900}|)be}\mylabel{h}  \normalsize

\doendnotes{C}
\bigskip
\vfill

\clearpage

\footnotesize

\lohead{\textsc{register}}

% Definiere theindex-Environment komplett neu ohne reledmac
\makeatletter
\renewenvironment{theindex}{%
  \section*{\indexname}%
  \setlength{\parindent}{0pt}%
  \setlength{\parskip}{0pt plus 0.3pt}%
  \let\item\@idxitem
}{%
  \clearpage
}
\makeatother

\IfFileExists{\jobname-pw.ind}{\input{\jobname-pw.ind}}{}

\end{document}

      