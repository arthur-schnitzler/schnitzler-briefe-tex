%% latex-korrekturansicht-vorspann.tex
%% Vorspann für die Korrekturansicht.
%% Lädt die gemeinsame Datei latex-vorspann.tex mit gesetztem Schalter.

\newif\ifkorrekturansicht
\korrekturansichttrue

\input{../tex-inputs/latex-vorspann}


\renewcommand{\erwaehntePersonen}{Personen: Louis Charles de Bourbon, Felix Salten, Ottilie Salten, Olga Schnitzler, Élisabeth Vigée-Lebrun}
\renewcommand{\erwaehnteOrte}{Orte: Berlin, Hamburg, London, Paris, Schloss Versailles, Sternwartestraße 71, Wien, Österreich}
\renewcommand{\erwaehnteWerke}{Werke: Portrait du Dauphin Louis-Charles}
\section[ Felix und Ottilie Salten an Arthur und Olga Schnitzler, 25. 6. 1914]{Felix und Ottilie Salten an Arthur und Olga
               Schnitzler, 25. 6. 1914}
\nopagebreak\mylabel{v}
\rehead{ }\normalsize\beginnumbering\briefempfaengerindex{Schnitzler, Olga@\textsc{Schnitzler, Olga}!zzzSalten, Ottilie@\emph{von Ottilie Salten}!1914-06-251@{25. 6. 1914}|(be}\briefempfaengerindex{Schnitzler, Olga@\textsc{Schnitzler, Olga}!zzzSalten, Felix@\emph{von Felix Salten}!1914-06-251@{25. 6. 1914}|(be}\briefempfaengerindex{Schnitzler, Arthur@\textsc{Schnitzler, Arthur}!zzzSalten, Ottilie@\emph{von Ottilie Salten}!1914-06-251@{25. 6. 1914}|(be}\briefempfaengerindex{Schnitzler, Arthur@\textsc{Schnitzler, Arthur}!zzzSalten, Felix@\emph{von Felix Salten}!1914-06-251@{25. 6. 1914}|(be}
\toendnotes[C]{\smallbreak\pagebreak[2]}\Standort{CUL, Schnitzler, B 89, B 2.}
\physDesc{Bildpostkarte, 311 Zeichen
\newline{}Handschrift Felix Salten: schwarze Tinte, lateinische Kurrent
\newline{}Handschrift Ottilie Salten: schwarze Tinte, deutsche Kurrent
\newline{}Versand: Stempel: »\nobreak{}\oindex{Paris@\textbf{Paris}, \emph{P.PPLC}|pwk}\textcolor{gray}{Paris – 92} Boissy—D’Anglas, 25—6 14, 15 50\nobreak{}«.  
\newline{}Ordnung: mit Bleistift von unbekannter Hand nummeriert: »277« }\toendnotes[C]{\smallbreak}\pstart{}{\pb}\begin{otherlanguage}{french}\textcolor{pink}{Autriche}{}\ledrightnote{\textcolor{pink}{Österreich}}\end{otherlanguage}\pend{}\pstart{}Herrn u. Frau D\textsuperscript{r} Arthur Schnitzler\pend{}\pstart{}\textcolor{pink}{Wien}{}\ledrightnote{\textcolor{pink}{Wien}}\pend{}\pstart{}\textcolor{pink}{XVIII. Sternwartestrasse 71}{}\ledrightnote{\textcolor{pink}{Sternwartestraße 71}}\pend{}
{\bigskip}
\pstart
           \noindent{}\centering{}{\pb}\textcolor{gray}{\textbf{Mme \textcolor{blue}{VIGÉE-LEBRUN}{}\ledrightnote{\textcolor{blue}{Élisabeth Vigée-Lebrun}}. – \textcolor{green}{Portrait du \textcolor{blue}{Dauphin}{}\ledrightnote{\textcolor{blue}{Louis Charles de Bourbon}}}{}\ledrightnote{\textcolor{green}{Portrait du Dauphin Louis-Charles}}.}}\pend
           
\pstart
           \noindent{}\raggedleft{}\textcolor{gray}{\textbf{\textcolor{pink}{MUSÉE DE VERSAILLES}{}\ledrightnote{\textcolor{pink}{Schloss Versailles}}}}\pend
           
\pstart
           {\pb}Wir fahren heute heim. In diesen kurzen Wochen \textcolor{pink}{Berlin}{}\ledrightnote{\textcolor{pink}{Berlin}}, \textcolor{pink}{Hamburg}{}\ledrightnote{\textcolor{pink}{Hamburg}}, \textcolor{pink}{London}{}\ledrightnote{\textcolor{pink}{London}} und \textcolor{pink}{Paris}{}\ledrightnote{\textcolor{pink}{Paris}} war ein bischen viel und wir sind ein wenig müd. Aber es war sehr
               schön! \label{K_L03564-1v}\edtext{Wann kommen Sie nach
                  Hause?}{\lemma{\textnormal{\emph{Wann … Hause?}}}\Cendnote{\textnormal{\textcolor{blue}{Schnitzler} war zu diesem Zeitpunkt bereits wieder in \textcolor{pink}{Wien}.}}}\label{K_L03564-1h}\pend
           
\pstart
           Viele herzliche Grüße Ihnen Beiden {\\[\baselineskip]}Ihr {\\[\baselineskip]}\spacefill\mbox{Salten}\pend
           \leftskip=0em{}
\pstart
           {[}hs. Ottilie Salten:{]} herzliche Grüße {\\[\baselineskip]}\spacefill\mbox{OttilieS.}\pend
           \leftskip=0em{}\endnumbering\briefempfaengerindex{Schnitzler, Olga@\textsc{Schnitzler, Olga}!zzzSalten, Ottilie@\emph{von Ottilie Salten}!1914-06-251@{25. 6. 1914}|)be}\briefempfaengerindex{Schnitzler, Olga@\textsc{Schnitzler, Olga}!zzzSalten, Felix@\emph{von Felix Salten}!1914-06-251@{25. 6. 1914}|)be}\briefempfaengerindex{Schnitzler, Arthur@\textsc{Schnitzler, Arthur}!zzzSalten, Ottilie@\emph{von Ottilie Salten}!1914-06-251@{25. 6. 1914}|)be}\briefempfaengerindex{Schnitzler, Arthur@\textsc{Schnitzler, Arthur}!zzzSalten, Felix@\emph{von Felix Salten}!1914-06-251@{25. 6. 1914}|)be}\mylabel{h}  \normalsize

\doendnotes{C}
\bigskip
\vfill

\clearpage

\footnotesize

\lohead{\textsc{register}}

% Definiere theindex-Environment komplett neu ohne reledmac
\makeatletter
\renewenvironment{theindex}{%
  \section*{\indexname}%
  \setlength{\parindent}{0pt}%
  \setlength{\parskip}{0pt plus 0.3pt}%
  \let\item\@idxitem
}{%
  \clearpage
}
\makeatother

\IfFileExists{\jobname-pw.ind}{\input{\jobname-pw.ind}}{}

\end{document}

      