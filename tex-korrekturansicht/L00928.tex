%% latex-korrekturansicht-vorspann.tex
%% Vorspann für die Korrekturansicht.
%% Lädt die gemeinsame Datei latex-vorspann.tex mit gesetztem Schalter.

\newif\ifkorrekturansicht
\korrekturansichttrue

\input{../tex-inputs/latex-vorspann}


               \section[Arthur Schnitzler an Richard Beer-Hofmann, 21. 6. 1899]{ Arthur Schnitzler an Richard Beer-Hofmann, 21. 6. 1899}\nopagebreak\mylabel{v}\rehead{ }\normalsize\beginnumbering\briefempfaengerindex{Beer-Hofmann, Richard@\textsc{Beer-Hofmann, Richard}!zzzSchnitzler, Arthur@\emph{von Arthur Schnitzler}!1899-06-211@{21. 6. 1899}|(be} \toendnotes[C]{\smallbreak\pagebreak[2]} \Standort{CUL, Schnitzler, B 8.1, S. 79.}
\physDesc{maschinelle Abschrift
\newline{}Schreibmaschine\newline{}Ordnung: mit Bleistift von unbekannter Hand nummeriert:
                                    »136« }\buchAbdrucke{\weitereDrucke{Arthur Schnitzler, Richard Beer-Hofmann: \emph{Briefwechsel 1891–1931}. Hg. Konstanze Fliedl. Wien, Zürich: \emph{Europaverlag} 1992, S. 130.} }\toendnotes[C]{\smallbreak}\pstart
           \raggedleft{}{\pb}\textcolor{pink}{Wien}{}\ledrightnote{\textcolor{pink}{Wien}}, 21. 6. 99.\pend
           \pstart
           Lieber Richard, ich habe gestern Abend mit \textcolor{blue}{Mayer}{}\ledrightnote{\textcolor{blue}{Oskar Mayer}} gesprochen. Wir schlagen folgendes vor: dass wir etwa
                  Mitte Juli zu Ihnen kommen und Sie von dort mitnehmen (etwa 5 Tage
               später). Wohin? Mir wäre ebenso wie \textcolor{blue}{Mayer}{}\ledrightnote{\textcolor{blue}{Oskar Mayer}} eine
               Tour im \textcolor{pink}{südtirol}{}\ledrightnote{\textcolor{pink}{Südtirol}}ischen am sympathischesten (eine
               Zusammenstellung hab ich); ich will nämlich dann, vielleicht mit \textcolor{blue}{Mayer}{}\ledrightnote{\textcolor{blue}{Oskar Mayer}}, an irgend einen hohen Punkt (\textcolor{pink}{San Martino}{}\ledrightnote{\textcolor{pink}{San Martino di Castrozza}}) 2–3 Wochen bleiben, auch länger und dort zu
               arbeiten versuchen. Denn ich fühle, dass mein Organismus nach Höhenluft verlangt.
               Ihrer wahrscheinlich auch. Man hat ja offenbar nie recht, einem Menschen zu sagen, er
               habe keinen Grund verstimmt zu sein; – aber dass es \uline{mir} heuer sehr nahe liegt, Ihnen irgendwas in der Art zu sagen, werden Sie
               verzeihlich finden. Ich hoffe, Sie erholen sich – von was? – Mir kommt vor, ich wär
               an Ihrer Stelle so glücklich, dass mich schauern müsste, aber offenbar irr ich mich.
               Aber im Ernst, was haben Sie? – Mir scheint nun einmal, dass Sie selbst einfach durch
               Willen einiges dazu thun könnten, um wohl zu sein. Sie lassen sich gehn. Freilich,
               auch dagegen scheinen Sie keine Gewalt zu haben.\pend
           \pstart
           Was mich anbelangt, so fühle ich jenes \label{K_L00928_1v}\edtext{Unglück}{\lemma{\textnormal{\emph{Unglück}}}\Cendnote{\textnormal{der Tod \textcolor{blue}{Marie Reinhard}s am 18. 3. 1899}}}\label{K_L00928_1h} mit {\pb}jedem Tag tiefer; der Sommer hat so seine
               eigenen Qualen. – Zu arbeiten hab ich versucht. – Mit \textcolor{blue}{Hugo}{}\ledrightnote{\textcolor{blue}{Hugo von Hofmannsthal}} hab ich gestern eine schöne Radpartie gemacht: \textcolor{pink}{Edlacher Hof}{}\ledrightnote{\textcolor{pink}{Hotel Edlacherhof}} – \textcolor{pink}{Singerin}{}\ledrightnote{\textcolor{pink}{Gasthof zur Singerin}} – \textcolor{pink}{Gutenstein}{}\ledrightnote{\textcolor{pink}{Gutenstein}} – \textcolor{pink}{Pottenstein}{}\ledrightnote{\textcolor{pink}{Pottenstein}} – \textcolor{pink}{Vöslau}{}\ledrightnote{\textcolor{pink}{Bad Vöslau}}.\pend
           \pstart
           Morgen Abend fahr ich nach \textcolor{pink}{Slavonien}{}\ledrightnote{\textcolor{pink}{Slavonija}} und wünsche
               in den letzten Junitagen wieder hier zu sein. Dann bleib ich etwa 10–12
               Tage hier.\pend
           \pstart
           \textcolor{blue}{Paul Goldmann}{}\ledrightnote{\textcolor{blue}{Paul Goldmann}}s Adresse einfach \textcolor{brown}{Frankfurter Zeitung}{}\ledrightnote{\textcolor{brown}{Frankfurter Zeitung}}.\pend
           \pstart
           Die \textcolor{pink}{tirolische}{}\ledrightnote{\textcolor{pink}{Tirol}{\newline}\textcolor{pink}{Südtirol}} Tour ist ungefähr; oder wäre:
                  \textcolor{pink}{Niederdorf}{}\ledrightnote{\textcolor{pink}{Niederdorf}} – \textcolor{pink}{Schluderbach}{}\ledrightnote{\textcolor{pink}{Carbonin}} – \textcolor{pink}{Tre \label{T_L00928_1v}\edtext{Croci}{\lemma{\textnormal{\emph{Croci}}}\Cendnote{\textnormal{In der
                     Abschrift steht: »Croce«.}}}\label{T_L00928_1h}}{}\ledrightnote{\textcolor{pink}{Tre Croci}} – \textcolor{pink}{Cortina}{}\ledrightnote{\textcolor{pink}{Cortina d'Ampezzo}} – \textcolor{pink}{Caprile}{}\ledrightnote{\textcolor{pink}{Alleghe}} – \label{T_L00928_2v}\edtext{\textcolor{pink}{Fedaja}{}\ledrightnote{\textcolor{pink}{Passo Fedaia}}}{\lemma{\textnormal{\emph{Fedaja}}}\Cendnote{\textnormal{In der Abschrift steht:
                  »Tevaja«.}}}\label{T_L00928_2h} – \label{T_L00928_3v}\edtext{\textcolor{pink}{Karersee}{}\ledrightnote{\textcolor{pink}{Karersee}}}{\lemma{\textnormal{\emph{Karersee}}}\Cendnote{\textnormal{In der Abschrift steht
                     »Karrersee«.}}}\label{T_L00928_3h} – \textcolor{pink}{Rollepass}{}\ledrightnote{\textcolor{pink}{Passo Rolle}} – \textcolor{pink}{Martino}{}\ledrightnote{\textcolor{pink}{San Martino di Castrozza}} – \textcolor{pink}{Trient}{}\ledrightnote{\textcolor{pink}{Trient}}.\pend
           \pstart
           Einfacher: \textcolor{pink}{Bozen}{}\ledrightnote{\textcolor{pink}{Bozen}} – \label{T_L00928_4v}\edtext{\textcolor{pink}{Karersee}{}\ledrightnote{\textcolor{pink}{Karersee}}}{\lemma{\textnormal{\emph{Karersee}}}\Cendnote{\textnormal{In der Abschrift steht
                     »Karrersee«.}}}\label{T_L00928_4h} u. s. w.\pend
           \pstart
           Leben Sie wohl, grüssen Sie \textcolor{blue}{Weib}{}\ledrightnote{→\textcolor{blue}{Paula Beer-Hofmann}} und \textcolor{blue}{Kind}{}\ledrightnote{→\textcolor{blue}{Mirjam Beer-Hofmann}{\newline}→\textcolor{blue}{Naëmah Beer-Hofmann}}.\pend
           \pstart Herzlich der Ihre \spacefill\mbox{Arthur.}\pend{}\pstart
           \noindent{}(nach \textcolor{pink}{Seeboden}{}\ledrightnote{\textcolor{pink}{Seeboden}})\pend
           \endnumbering\briefempfaengerindex{Beer-Hofmann, Richard@\textsc{Beer-Hofmann, Richard}!zzzSchnitzler, Arthur@\emph{von Arthur Schnitzler}!1899-06-211@{21. 6. 1899}|)be}\mylabel{h}  \normalsize

\doendnotes{C}
\bigskip
\vfill

\clearpage

\footnotesize

\lohead{\textsc{register}}

% Definiere theindex-Environment komplett neu ohne reledmac
\makeatletter
\renewenvironment{theindex}{%
  \section*{\indexname}%
  \setlength{\parindent}{0pt}%
  \setlength{\parskip}{0pt plus 0.3pt}%
  \let\item\@idxitem
}{%
  \clearpage
}
\makeatother

\IfFileExists{\jobname-pw.ind}{\input{\jobname-pw.ind}}{}

\end{document}

      