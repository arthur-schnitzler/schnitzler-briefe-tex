%% latex-korrekturansicht-vorspann.tex
%% Vorspann für die Korrekturansicht.
%% Lädt die gemeinsame Datei latex-vorspann.tex mit gesetztem Schalter.

\newif\ifkorrekturansicht
\korrekturansichttrue

\input{../tex-inputs/latex-vorspann}


\renewcommand{\erwaehntePersonen}{Personen: Paul Goldmann, Alfred Kerr}
\renewcommand{\erwaehnteInstitutionen}{Institutionen: Neue Freie Presse}
\renewcommand{\erwaehnteOrte}{Orte: Florenz, Frankgasse 1, IX., Alsergrund, Parco delle Cascine, Wien, Österreich}
\renewcommand{\erwaehnteWerke}{Werke: Abschluß, Dämmerseele, Lebendige Stunden. Vier Einakter, Neue Deutsche Rundschau}
\section[ Felix Salten an Arthur Schnitzler, 2{[}3{]}. 5. 1902]{Felix Salten an Arthur Schnitzler, 2{[}3{]}. 5. 1902}
\nopagebreak\mylabel{v}
\rehead{ }\normalsize\beginnumbering\briefempfaengerindex{Schnitzler, Arthur@\textsc{Schnitzler, Arthur}!zzzSalten, Felix@\emph{von Felix Salten}!1902-05-231@{2{[}3{]}. 5. 1902}|(be}
\toendnotes[C]{\smallbreak\pagebreak[2]}\Standort{CUL, Schnitzler, B 89, A 2.}
\physDesc{Bildpostkarte, 254 Zeichen
\newline{}Handschrift: schwarze Tinte, lateinische Kurrent
\newline{}Versand: 1) Stempel: »\nobreak{}\oindex{Florenz@\textbf{Florenz}, \emph{P.PPLA}|pwk}Firenze Ferrovia, 25 5 02, \textcolor{gray}{11}\nobreak{}«.   2) Stempel: »\nobreak{}\oindex{IX., Alsergrund@\textbf{IX., Alsergrund}, \emph{A.ADM3}|pwk}9/3 W\textcolor{gray}{ien 72}, 27. 5. \textcolor{gray}{02}, 8. V, Beste{[}llt{]}\nobreak{}«. 
\newline{}Ordnung: mit Bleistift von unbekannter Hand nummeriert: »156« }\toendnotes[C]{\smallbreak}\pstart{}{\pb}Herrn D\textsuperscript{r} Arthur Schnitzler\pend{}\pstart{}\textcolor{pink}{Wien IX.}{}\ledrightnote{\textcolor{pink}{IX., Alsergrund}}\pend{}\pstart{}\textcolor{pink}{Frankgaße 1}{}\ledrightnote{\textcolor{pink}{Frankgasse 1}}\pend{}\pstart{}\textcolor{pink}{Austria}{}\ledrightnote{\textcolor{pink}{Österreich}}\pend{}
{\bigskip}
\pstart
           \noindent{}\centering{}{\pb}\textcolor{gray}{\textbf{\textcolor{pink}{Firenze}{}\ledrightnote{\textcolor{pink}{Florenz}}}}\hspace*{2.5em}\textcolor{pink}{\textcolor{gray}{\textbf{Passeggiata delle Cascine\hspace*{1.5em}Viale del Re}}}{}\ledrightnote{\textcolor{pink}{Parco delle Cascine}}\pend
           
\pstart
           Vielen Dank für den \label{K_L03331-1v}\edtext{\textcolor{green}{\textcolor{blue}{Kerr}{}\ledrightnote{\textcolor{blue}{Alfred Kerr}}-Ausschnitt}{}\ledrightnote{{$\rightarrow$}\textcolor{green}{Abschluß}}}{\lemma{\textnormal{\emph{Kerr-Ausschnitt}}}\Cendnote{\textnormal{Beilage nicht erhalten. Es handelte sich
                  wohl um diese Sammelrezension über die neuen Theaterstücke des vergangenen
                  Winters: \textcolor{blue}{Alfred Kerr}: \emph{\textcolor{green}{Abschluß}}. In: \emph{\textcolor{green}{Neue
                        Deutsche Rundschau}}, Jg. 13, H. 5, Mai 1902,
                     S. 545–553. Insofern das Wort »Ausschnitt« wörtlich zu
                  nehmen ist, könnte \textcolor{blue}{Schnitzler} auch nur die
                  Seiten 551–553 gesandt haben, die (trotz allgemeinen Lobs für \textcolor{blue}{Schnitzler}) die vier Einakter der \emph{\textcolor{green}{Lebendigen Stunden}} abwertend beurteilen.}}}\label{K_L03331-1h}. Natürlich
               würde ich mich der \textcolor{brown}{N. fr. Pr.}{}\ledrightnote{\textcolor{brown}{Neue Freie Presse}} gegenüber –
               prinzipiell – \uline{nicht} ablehnend verhalten. \label{K_L03331-2v}\edtext{Schrieb Ihnen gestern}{\lemma{\textnormal{\emph{Schrieb Ihnen gestern}}}\Cendnote{\textnormal{Felix Salten an Arthur Schnitzler, 22. 5. 1902. Das erlaubt die Datierung dieser Karte auf Freitag, den 23. 5. 1902. Der
               Versand erfolgte erst nach dem Wochenende, am 25. 5. 1902.}}}\label{K_L03331-2h} wegen »\textcolor{green}{Dämmerseele}{}\ledrightnote{\textcolor{green}{Dämmerseele}}«. herzlichst {\\}\spacefill\mbox{Salten}\pend
           
\pstart
           \noindent{}\label{K_L03331-3v}\edtext{\textcolor{gray}{b}}{\lemma{\textnormal{\emph{b}}}\Cendnote{\textnormal{bitte?}}}\label{K_L03331-3h}. \label{K_L03331-4v}\edtext{Gruß an \textcolor{blue}{P. Goldmann}{}\ledrightnote{\textcolor{blue}{Paul Goldmann}}}{\lemma{\textnormal{\emph{Gruß an P. Goldmann}}}\Cendnote{\textnormal{siehe A. S.: \emph{Tagebuch}, 25. 5. 1902}}}\label{K_L03331-4h}.\pend
           \endnumbering\briefempfaengerindex{Schnitzler, Arthur@\textsc{Schnitzler, Arthur}!zzzSalten, Felix@\emph{von Felix Salten}!1902-05-231@{2{[}3{]}. 5. 1902}|)be}\mylabel{h}  \normalsize

\doendnotes{C}
\bigskip
\vfill

\clearpage

\footnotesize

\lohead{\textsc{register}}

% Definiere theindex-Environment komplett neu ohne reledmac
\makeatletter
\renewenvironment{theindex}{%
  \section*{\indexname}%
  \setlength{\parindent}{0pt}%
  \setlength{\parskip}{0pt plus 0.3pt}%
  \let\item\@idxitem
}{%
  \clearpage
}
\makeatother

\IfFileExists{\jobname-pw.ind}{\input{\jobname-pw.ind}}{}

\end{document}

      