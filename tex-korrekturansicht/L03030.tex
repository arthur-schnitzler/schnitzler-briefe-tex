%% latex-korrekturansicht-vorspann.tex
%% Vorspann für die Korrekturansicht.
%% Lädt die gemeinsame Datei latex-vorspann.tex mit gesetztem Schalter.

\newif\ifkorrekturansicht
\korrekturansichttrue

\input{../tex-inputs/latex-vorspann}


\renewcommand{\erwaehntePersonen}{Personen: Richard Beer-Hofmann, Hugo von Hofmannsthal, Felix Salten, Paul Schlenther, Gustav Schwarzkopf}
\renewcommand{\erwaehnteOrte}{Orte: Wien}
\renewcommand{\erwaehnteWerke}{Werke: Der grüne Kakadu. Groteske in einem Akt, Tagebuch}
\section[ Arthur Schnitzler an Felix Salten, {[}31. 12. 1899?{]}]{Arthur Schnitzler an Felix Salten, {[}31. 12. 1899?{]}}
\nopagebreak\mylabel{v}
\rehead{ }\normalsize\beginnumbering\briefempfaengerindex{Salten, Felix@\textsc{Salten, Felix}!zzzSchnitzler, Arthur@\emph{von Arthur Schnitzler}!1899-12-311@{{[}31. 12. 1899?{]}}|(be}
\toendnotes[C]{\smallbreak\pagebreak[2]}\Standort{Wienbibliothek im Rathaus, ZPH 1681, 2.1.516.}
\physDesc{Karte, 203 Zeichen
\newline{}Handschrift: schwarze Tinte, deutsche Kurrent
\newline{}Ordnung: mit Bleistift von unbekannter Hand Nummerierung der Blätter des Konvoluts:
                                    »33« }\toendnotes[C]{\smallbreak}
\pstart
           \noindent{}{\pb}lieber Freund, ich bin \label{K_L03030-1v}\edtext{morgen (Neujahr) Abend}{\lemma{\textnormal{\emph{morgen (Neujahr) Abend}}}\Cendnote{\textnormal{Das erlaubt die Datierung anhand von \textcolor{blue}{Schnitzler}s \emph{\textcolor{green}{Tagebuch}}, vgl. A. S.: \emph{Tagebuch}, 1. 1. 1900.}}}\label{K_L03030-1h}, we{\geminationn} ich frei
               bin, bei \textcolor{blue}{Richard}{}\ledrightnote{\textcolor{blue}{Richard Beer-Hofmann}}; er läſst Sie bitten, auch zu
               ihm zu ko{\geminationm}en. \textcolor{blue}{Hugo}{}\ledrightnote{\textcolor{blue}{Hugo von Hofmannsthal}} und \textcolor{blue}{Guſt. Schwarzk.}{}\ledrightnote{\textcolor{blue}{Gustav Schwarzkopf}} ſind beſti{\geminationm}t dort.\pend
           \pstart Herzlichſt Ihr \spacefill\mbox{Arthur.}\pend{}
\pstart
           \textcolor{blue}{Schlenther}{}\ledrightnote{\textcolor{blue}{Paul Schlenther}} wieder \label{K_L03030-2v}\edtext{gutge\textcolor{gray}{ſüß}elt}{\lemma{\textnormal{\emph{gutgeſüßelt}}}\Cendnote{\textnormal{Bezug unklar; womöglich in Zusammenhang mit der kürzlich
                  erfolgten Absetzung von \emph{\textcolor{green}{Der grüne Kakadu}}}}}\label{K_L03030-2h}!\pend
           \endnumbering\briefempfaengerindex{Salten, Felix@\textsc{Salten, Felix}!zzzSchnitzler, Arthur@\emph{von Arthur Schnitzler}!1899-12-311@{{[}31. 12. 1899?{]}}|)be}\mylabel{h}  \normalsize

\doendnotes{C}
\bigskip
\vfill

\clearpage

\footnotesize

\lohead{\textsc{register}}

% Definiere theindex-Environment komplett neu ohne reledmac
\makeatletter
\renewenvironment{theindex}{%
  \section*{\indexname}%
  \setlength{\parindent}{0pt}%
  \setlength{\parskip}{0pt plus 0.3pt}%
  \let\item\@idxitem
}{%
  \clearpage
}
\makeatother

\IfFileExists{\jobname-pw.ind}{\input{\jobname-pw.ind}}{}

\end{document}

      