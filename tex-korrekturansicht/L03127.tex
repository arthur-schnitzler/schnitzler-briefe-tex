%% latex-korrekturansicht-vorspann.tex
%% Vorspann für die Korrekturansicht.
%% Lädt die gemeinsame Datei latex-vorspann.tex mit gesetztem Schalter.

\newif\ifkorrekturansicht
\korrekturansichttrue

\input{../tex-inputs/latex-vorspann}


\renewcommand{\erwaehnteOrte}{Orte: Cortina d'Ampezzo, Dölsach, Grillparzerstraße, Höhlenstein, I., Innere Stadt, Kärnten, Pieve di Cadore, Steiermark, Toblach, Wien}
\renewcommand{\erwaehnteWerke}{}
\section[Felix Salten an Arthur Schnitzler, 14. 8. 1893]{Felix Salten an Arthur Schnitzler, 14. 8. 1893}
\nopagebreak\mylabel{v}
\rehead{ }\normalsize\beginnumbering\briefempfaengerindex{Schnitzler, Arthur@\textsc{Schnitzler, Arthur}!zzzSalten, Felix@\emph{von Felix Salten}!1893-08-142@{14. 8. 1893}|(be}
\toendnotes[C]{\smallbreak\pagebreak[2]}\Standort{CUL, Schnitzler, B 89, A 1.}
\physDesc{Bildpostkarte, 567 Zeichen
\newline{}Handschrift: Bleistift, lateinische Kurrent
\newline{}Versand: 1) Stempel: »\nobreak{}\oindex{Cortina d'Ampezzo@\textbf{Cortina d'Ampezzo}, \emph{P.PPLA3}|pwk}Co{[}rti{]}na, 15/8 9\textcolor{gray}{3}\nobreak{}«.   2) Stempel: »\nobreak{}\oindex{I., Innere Stadt@\textbf{I., Innere Stadt}, \emph{A.ADM3}|pwk}Wien 1/1 1, 17/8. 93, 8–9½ V., Bestellt\nobreak{}«. 
\newline{}Ordnung: mit Bleistift von unbekannter Hand nummeriert: »30.« }\toendnotes[C]{\smallbreak}\pstart{}{\pb}Herrn D\textsuperscript{r} Arthur Schnitzler.\pend{}\pstart{}\textcolor{pink}{Wien}{}\ledrightnote{\textcolor{pink}{Wien}}\pend{}\pstart{}\textcolor{pink}{I. Grillparzerstraße 7}{}\ledrightnote{\textcolor{pink}{Grillparzerstraße}}.\pend{}
{\bigskip}
\pstart
           \noindent{}\centering{}{\pb}\textcolor{gray}{\textbf{\textbf{\textcolor{pink}{Cortina d’Ampezzo}{}\ledrightnote{\textcolor{pink}{Cortina d'Ampezzo}}.}}}\pend
           
\pstart
           \raggedleft{}14. 8. 93.\pend
           
\pstart
           Lieber Freund! Die Fahrt \textcolor{pink}{hieher}{}\ledrightnote{{$\rightarrow$}\textcolor{pink}{Cortina d'Ampezzo}} einfach das Herrlichste, was es
               gibt, die Straße von unerhörter Glätte. \label{K_L03127-1v}\edtext{Wenn Sie kommen}{\lemma{\textnormal{\emph{Wenn Sie kommen}}}\Cendnote{\textnormal{\textcolor{blue}{Schnitzler} kam am 23. 8. 1893 in \textcolor{pink}{Dölsach} an, noch am selben Tag ging es für ihn weiter nach \textcolor{pink}{Toblach}. Von dort aus unternahm er Radausflüge, etwa am 24. 8. 1893 nach \textcolor{pink}{Pieve di Cadore}. Später fuhr er weiter nach
                     \textcolor{pink}{Kärnten} und in die \textcolor{pink}{Steiermark}. Am 31. 8. 1893 verzeichnete \textcolor{blue}{Schnitzler} seine Rückkunft in \textcolor{pink}{Wien}. Bei welchen Touren \textcolor{blue}{Salten}
                  mitmachte, ist unklar. Siehe Felix Salten an Arthur Schnitzler, 18. 8. 1893.}}}\label{K_L03127-1h}, fahren wir nach \textcolor{pink}{Piève di Cadore}{}\ledrightnote{\textcolor{pink}{Pieve di Cadore}},
               ja? Es soll gleichfalls herrlich sein. Ich habe die 35 Km. in 1 ½ Stunden gemacht,
               ungerechnet den Aufenthalt in \textcolor{pink}{Landro}{}\ledrightnote{\textcolor{pink}{Höhlenstein}}. Dieses
               Bergabfahren von \textcolor{pink}{Landro}{}\ledrightnote{\textcolor{pink}{Höhlenstein}} an, na, Sie werden sehen.
               Ich habe nach \textcolor{pink}{Cortina}{}\ledrightnote{\textcolor{pink}{Cortina d'Ampezzo}}{ }\textcolor{gray}{dann die Temp}eratur verachtet, u. als ich ankam, war ich
                  \textcolor{gray}{r}ein \textcolor{gray}{e}rstgradig, was ich jetzt eher nicht
               mehr ganz bin. Ich schreibe nochmals genau. Herzlich Ihr \spacefill\mbox{Salten}\pend
           \endnumbering\briefempfaengerindex{Schnitzler, Arthur@\textsc{Schnitzler, Arthur}!zzzSalten, Felix@\emph{von Felix Salten}!1893-08-142@{14. 8. 1893}|)be}\mylabel{h}  \normalsize

\doendnotes{C}
\bigskip
\vfill

\clearpage

\footnotesize

\lohead{\textsc{register}}

% Definiere theindex-Environment komplett neu ohne reledmac
\makeatletter
\renewenvironment{theindex}{%
  \section*{\indexname}%
  \setlength{\parindent}{0pt}%
  \setlength{\parskip}{0pt plus 0.3pt}%
  \let\item\@idxitem
}{%
  \clearpage
}
\makeatother

\IfFileExists{\jobname-pw.ind}{\input{\jobname-pw.ind}}{}

\end{document}

      