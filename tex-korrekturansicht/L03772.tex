%% latex-korrekturansicht-vorspann.tex
%% Vorspann für die Korrekturansicht.
%% Lädt die gemeinsame Datei latex-vorspann.tex mit gesetztem Schalter.

\newif\ifkorrekturansicht
\korrekturansichttrue

\input{../tex-inputs/latex-vorspann}


\section[Arthur Schnitzler an Stefan Zweig, 11. 2. 1915]{L03772 Arthur Schnitzler an Stefan Zweig, 11. 2. 1915}
\nopagebreak\mylabel{L03772v}
\rehead{ }\normalsize\beginnumbering\briefempfaengerindex{, @\textsc{, }!zzz, @\emph{von  }!1915-02-111@{11. 2. 1915}|(be}
\toendnotes[C]{\smallbreak\pagebreak[2]}\Standort{Jerusalem, National Library of Israel, ARC. Ms. Var. 305 1 58 Stefan Zweig Collection.}
\physDesc{Briefkarte, 815 Zeichen
\newline{}Handschrift: schwarze Tinte, deutsche Kurrent}\toendnotes[C]{\smallbreak}
\pstart
           {\pb}\textcolor{gray}{\textbf{Dr. Arthur Schnitzler}}\hfill 11. 2. 915\pend
           
\pstart
           \textcolor{gray}{\textbf{\textcolor{pink}{Wien XVIII. Sternwartestrasse 71}\oindex{Wien@\textbf{Wien}!XVIII., Währing@\textbf{XVIII., Währing}!Sternwartestraße 71@\textbf{Sternwartestraße 71}, \emph{Wohngebäude}|pw}{}\ledrightnote{\textcolor{pink}{Sternwartestraße 71}}}}\pend
           \vspace{0.5em}
\pstart
           lieber Herr Doktor Zweig, vielen Dank für Ihre \label{K_L03772-1v}\edtext{Karte}{\lemma{\textnormal{\emph{Karte}}}\Cendnote{\textnormal{Stefan Zweig an Arthur Schnitzler, [zwischen 7. und
               10. 2. 1915?].
               }}}\label{K_L03772-1}, die mich veranlaßt hat, auch an \textcolor{blue}{Rom.
                  Rolland}\pwindex{Rolland, Romain 29.\,1.\,1866 Clamecy – 30.\,12.\,1944 Vézelay@\textsc{Rolland, Romain} (29.\,1.\,1866 Clamecy – 30.\,12.\,1944 Vézelay), \emph{Schriftsteller}|pw}{}\ledrightnote{\textcolor{blue}{Romain Rolland}} gleich ein \label{K_L03772-2v}\edtext{paar Worte}{\lemma{\textnormal{\emph{paar Worte}}}\Cendnote{\textnormal{nicht nachgewiesen; im Nachlass \textcolor{blue}{Schnitzlers} finden
                     sich zwei maschinschriftliche Briefe an \textcolor{blue}{Rolland}\pwindex{Rolland, Romain 29.\,1.\,1866 Clamecy – 30.\,12.\,1944 Vézelay@\textsc{Rolland, Romain} (29.\,1.\,1866 Clamecy – 30.\,12.\,1944 Vézelay), \emph{Schriftsteller}|pwk} (14. 12. 1914, 7. 1. 1915). Im Umkehrschluss 
                     kann das 
                     als Indiz genommen werden, dass \textcolor{blue}{Schnitzler} das Schreiben mit der Hand 
                     verfasste.}}}\label{K_L03772-2} zu ſchreiben. Bisher haben ſich die \textcolor{green}{Angriffe}\pwindex{Schnitzler erhebt Einspruch@\emph{Schnitzler erhebt Einspruch}|pwv}{}\ledrightnote{{$\rightarrow$}\emph{\textcolor{green}{Schnitzler erhebt Einspruch}}},
               von denen Sie reden, nur in ein paar antiſemitiſchen Blättern gefunden – und ich habe
               nie davon geträumt, daſs gerade dieſes Ja{\geminationm}ervölkchen in Kriegszeiten Gerechtigkeit u
               Anstand {\pb}kennen würde – da ja auch ſonſt von der
               reinigenden Kraft des Kriegs (hinter den Schützengräben) nicht viel zu verſpüren iſt.
               – Im übrigen hab ich, wie Sie mit ſo freundſchaftlichen Worten wünſchen, thatſächlich
               zu arbeiten angefangen – es iſt Pflicht, Rettung, Notwendigkeit, – auch we{\geminationn} für
               ſpäter nicht gar zu viel herauskommen sollte. Und Sie, lieber Herr Doctor, ſind ganz
               in Ihr \textcolor{brown}{Archiv}\orgindex{Kriegsarchiv@Kriegsarchiv|pwv}{}\ledrightnote{{$\rightarrow$}\emph{\textcolor{brown}{Kriegsarchiv}}} vergraben?\pend
           
\pstart
           Wir grüßen Sie herzlichſt, auf baldgs Wiederſehn!{\\[\baselineskip]}Ihr \spacefill\mbox{Arthur
                  Schnitzler}\pend
           \leftskip=0em{}\selectlanguage{ngerman}\endnumbering\briefempfaengerindex{, @\textsc{, }!zzz, @\emph{von  }!1915-02-111@{11. 2. 1915}|)be}\mylabel{L03772h}
\begin{anhang}
\end{anhang}\normalsize

\doendnotes{C}
\bigskip
\vfill

\clearpage

\footnotesize

\lohead{\textsc{register}}

% Definiere theindex-Environment komplett neu ohne reledmac
\makeatletter
\renewenvironment{theindex}{%
  \section*{\indexname}%
  \setlength{\parindent}{0pt}%
  \setlength{\parskip}{0pt plus 0.3pt}%
  \let\item\@idxitem
}{%
  \clearpage
}
\makeatother

\IfFileExists{\jobname-pw.ind}{\input{\jobname-pw.ind}}{}

\end{document}

      