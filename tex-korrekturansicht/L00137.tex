%% latex-korrekturansicht-vorspann.tex
%% Vorspann für die Korrekturansicht.
%% Lädt die gemeinsame Datei latex-vorspann.tex mit gesetztem Schalter.

\newif\ifkorrekturansicht
\korrekturansichttrue

\input{../tex-inputs/latex-vorspann}


               \section[Karl Kraus an Arthur Schnitzler, 22. 11. 1892]{ Karl Kraus an Arthur Schnitzler, 22. 11. 1892}\nopagebreak\mylabel{v}\rehead{ }\normalsize\beginnumbering\briefempfaengerindex{Schnitzler, Arthur@\textsc{Schnitzler, Arthur}!zzzKraus, Karl@\emph{von Karl Kraus}!1892-11-221@{22. 11. 1892}|(be} \toendnotes[C]{\smallbreak\pagebreak[2]} \Standort{CUL, Schnitzler, B 55.}
\physDesc{Postkarte
\newline{}Handschrift: schwarze Tinte, deutsche Kurrent\newline{}Versand: Stempel: »\nobreak{}\oindex{I., Innere Stadt@\textbf{I., Innere Stadt}, \emph{Bezirk (A.BZK)}|pwk}Wien 1/1, 22. 11. 92, 4–5{[}N{]}\nobreak{}«.  }\buchAbdrucke{\weitereDrucke{\emph{Karl Kraus und Arthur Schnitzler. Eine Dokumentation.} Hg. Reinhard Urbach. In: \emph{Literatur und Kritik}, Bd. 49, Oktober 1970, S. 513.} }\toendnotes[C]{\smallbreak}\pstart{}{\pb}Herrn D\textsuperscript{r.} Arthur Schnitzler\pend{}\pstart{}Schriftsteller\pend{}\pstart{}\textcolor{pink}{Wien I}{}\ledrightnote{\textcolor{pink}{I., Innere Stadt}}\pend{}\pstart{}\textcolor{pink}{Grillparzerſtraße, 7}{}\ledrightnote{\textcolor{pink}{Grillparzerstraße}}\pend{}{\bigskip}\pstart
           \raggedleft{}{\pb}Postamt,
                        4 Uhr.\pend
           \pstart{}Sehr verehrter Herr D\textsuperscript{r}\footnote{\noindent{}Bitte, das kann \uline{D}octo\uline{r}{ }\uline{und}{ }\uline{D}ichte\uline{r} heißen!}!\pend\pstart
           Heute nemlich habe ich von der »\textcolor{brown}{Allgemeinen}{}\ledrightnote{\textcolor{brown}{Wiener Allgemeine Zeitung}}«
                    das \textcolor{green}{Manuscript}{}\ledrightnote{→\textcolor{green}{Wiener Lyriker}{\newline}→\textcolor{green}{Arthur Schnitzler, Anatol}} wiedererhalten. Die
                    beiden andern \textcolor{blue}{Autoren}{}\ledrightnote{→\textcolor{blue}{Felix Dörmann}{\newline}→\textcolor{blue}{Richard Specht}}{ }ſind ihnen
                    nicht wichtig genug und über \uline{\textcolor{green}{Anatol}{}\ledrightnote{\textcolor{green}{Anatol}}} haben ſie bereits acceptiert.\pend
           \pstart
           Faſt \uline{4 Wochen} wurde ich ſo \uline{hingehalten}! Noch heute ſende ich \textcolor{green}{\textcolor{green}{Anatol}{}\ledrightnote{→\textcolor{green}{Anatol}}}{}\ledrightnote{\textcolor{green}{Arthur Schnitzler, Anatol}}{ }\uline{allein}{ }\introOben{}\textcolor{green}{\textcolor{blue}{D.}{}\ledrightnote{\textcolor{blue}{Felix Dörmann}}{ }\textcolor{blue}{S.}{}\ledrightnote{\textcolor{blue}{Richard Specht}}}{}\ledrightnote{→\textcolor{green}{Wiener Lyriker}} extra\introOben{} an die »\textcolor{green}{Gesellſch«.}{}\ledrightnote{\textcolor{green}{Die Gesellschaft. Monatsschrift}}\pend
           \pstart
           Freilich ist es ſchon zu ſpät für Dezemberheft. Werde jedenfalls \uline{meinen ganzen Einfluſs geltend}{ }\uline{machen}, daſs es noch ins Decemb.heft kommt. Wenn
                    nicht iſt der Herr \textcolor{blue}{\uline{Osten}}{}\ledrightnote{\textcolor{blue}{Heinrich Osten}}, nicht \uline{ich} daran ſchuld.\pend
           \pstart
           Herzlichſten Gruß Ihr ergeb.{\\[\baselineskip]}\spacefill\mbox{Karl Kraus,}{ }\textcolor{pink}{Maximilianstr. 13}{}\ledrightnote{\textcolor{pink}{Mahlerstraße}}. \pend
           \leftskip=0em{}\endnumbering\briefempfaengerindex{Schnitzler, Arthur@\textsc{Schnitzler, Arthur}!zzzKraus, Karl@\emph{von Karl Kraus}!1892-11-221@{22. 11. 1892}|)be}\mylabel{h}  \normalsize

\doendnotes{C}
\bigskip
\vfill

\clearpage

\footnotesize

\lohead{\textsc{register}}

% Definiere theindex-Environment komplett neu ohne reledmac
\makeatletter
\renewenvironment{theindex}{%
  \section*{\indexname}%
  \setlength{\parindent}{0pt}%
  \setlength{\parskip}{0pt plus 0.3pt}%
  \let\item\@idxitem
}{%
  \clearpage
}
\makeatother

\IfFileExists{\jobname-pw.ind}{\input{\jobname-pw.ind}}{}

\end{document}

      