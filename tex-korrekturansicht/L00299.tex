%% latex-korrekturansicht-vorspann.tex
%% Vorspann für die Korrekturansicht.
%% Lädt die gemeinsame Datei latex-vorspann.tex mit gesetztem Schalter.

\newif\ifkorrekturansicht
\korrekturansichttrue

\input{../tex-inputs/latex-vorspann}


               \section[Arthur Schnitzler an Max Burckhard, {[}Mitte Februar 1894?{]}]{ Arthur Schnitzler an Max Burckhard, {[}Mitte Februar 1894?{]}}\nopagebreak\mylabel{v}\rehead{ }\normalsize\beginnumbering\briefempfaengerindex{Burckhard, Max Eugen@\textsc{Burckhard, Max Eugen}!zzzSchnitzler, Arthur@\emph{von Arthur Schnitzler}!1894-02-141@{{[}Mitte Februar 1894?{]}}|(be} \toendnotes[C]{\smallbreak\pagebreak[2]} \buchAlsQuelle{\pwindex{Schnitzlers Einzug ins Burgtheater@\emph{Schnitzlers Einzug ins Burgtheater}|pwk}\pwindex{Neue Freie Presse@\emph{Neue Freie Presse}|pwk}Karl Glossy: \emph{Schnitzlers Einzug ins Burgtheater. Unbekannte Briefe des Dichters.} In: \emph{Neue Freie Presse}, Nr. 24162, 19. 12. 1931, S. 14.}\buchAbdrucke{\weitereDrucke{1) \pwindex{Schnitzlers Einzug ins Burgtheater@\emph{Schnitzlers Einzug ins Burgtheater}|pwk}Karl Glossy: \emph{Schnitzlers Einzug ins Burgtheater. Unbekannte Briefe des Dichters.} In: \emph{Wiener Studien und Dokumente}. Zum 85. Geburtstag des Verfassers hg. von seinen Freunden. Wien: \emph{Steyrermühl} 1933, S. 166–168.} \weitereDrucke{2) Hans-Ulrich Lindken: \emph{Arthur Schnitzler. Aspekte und Akzente. Materialien zu Leben
                        und Werk}. Frankfurt am Main, Bern, Göttingen: \emph{Peter Lang} 1984, S. 243–246 (Europäische Hochschulschriften, Reihe 1, Deutsche Sprache und
                        Literatur, 754).} }\toendnotes[C]{\smallbreak}\pstart
           \noindent{}{\pb}\so{Schnitzler an Burckhard}, \label{K_L00299_1v}\edtext{1894}{\lemma{\textnormal{\emph{1894}}}\Cendnote{\textnormal{Die Datierung folgt der Annahme, dass \textcolor{blue}{Schnitzler}{ }\emph{\textcolor{green}{Anatol}}, unmittelbar nachdem ihm \textcolor{blue}{Burckhard} mitgeteilt hatte, das Buch nicht erhalten zu haben,
                  neuerlich mit diesem Begleitschreiben zukommen ließ.}}}\label{K_L00299_1h}: »Sehr verehrter Herr
               Direktor! Die drei \textcolor{green}{Stücke}{}\ledrightnote{→\textcolor{green}{Die Frage an das Schicksal}{\newline}→\textcolor{green}{Abschiedssouper}{\newline}→\textcolor{green}{Episode}}, welche ich für aufführbar halte, habe ich bezeichnet. Das letzte,
                  ›\textcolor{green}{Abſchiedsſouper}{}\ledrightnote{\textcolor{green}{Abschiedssouper}}‹, mag allerdings für eine
               Hofbühne nicht geeignet ſein; die beiden anderen werden Sie möglicherweiſe eines
               Verſuchs wert finden. Beſonders geeignet erſchienen ſie mir anläßlich einer Matinée
               im Repertoire zu erſcheinen. Für den Fall aber, daß Sie die anſpruchsloſen Szenen
               nicht für aufführbar halten, will ich wenigſtens hoffen, daß Sie die Lektüre
               derſelben nicht allzuſehr langweilt. Mit ausgezeichneter Hochachtung Ihr ſehr
               ergebener Dr. Arthur Schnitzler.«\pend
           \endnumbering\briefempfaengerindex{Burckhard, Max Eugen@\textsc{Burckhard, Max Eugen}!zzzSchnitzler, Arthur@\emph{von Arthur Schnitzler}!1894-02-141@{{[}Mitte Februar 1894?{]}}|)be}\mylabel{h}  \normalsize

\doendnotes{C}
\bigskip
\vfill

\clearpage

\footnotesize

\lohead{\textsc{register}}

% Definiere theindex-Environment komplett neu ohne reledmac
\makeatletter
\renewenvironment{theindex}{%
  \section*{\indexname}%
  \setlength{\parindent}{0pt}%
  \setlength{\parskip}{0pt plus 0.3pt}%
  \let\item\@idxitem
}{%
  \clearpage
}
\makeatother

\IfFileExists{\jobname-pw.ind}{\input{\jobname-pw.ind}}{}

\end{document}

      