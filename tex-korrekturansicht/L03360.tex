%% latex-korrekturansicht-vorspann.tex
%% Vorspann für die Korrekturansicht.
%% Lädt die gemeinsame Datei latex-vorspann.tex mit gesetztem Schalter.

\newif\ifkorrekturansicht
\korrekturansichttrue

\input{../tex-inputs/latex-vorspann}


\renewcommand{\erwaehntePersonen}{Personen:  ?? [Partner von Theodore Rottenberg, Ende 1902/Anfang 1903], Louis Bolle-Ritz, Heinrich Heine, Elise Krinitz, Theodore Rottenberg, Olga Schnitzler, Heinrich Schnitzler}
\renewcommand{\erwaehnteOrte}{Orte: Frankfurt am Main, Fürstenhof, Kaiserstraße, Monte Carlo, Münchener Straße, Wien}
\renewcommand{\erwaehnteWerke}{Werke: Gedichte an die Mouche}
\section[ Paul Goldmann an Arthur Schnitzler, 3. 1. {[}1903{]}]{Paul Goldmann an Arthur Schnitzler, 3. 1. {[}1903{]}}
\nopagebreak\mylabel{v}
\rehead{ }\normalsize\beginnumbering\briefempfaengerindex{Schnitzler, Arthur@\textsc{Schnitzler, Arthur}!zzzGoldmann, Paul@\emph{von Paul Goldmann}!1903-01-031@{3. 1. {[}1903{]}}|(be}
\toendnotes[C]{\smallbreak\pagebreak[2]}\Standort{DLA, A:Schnitzler, HS.NZ85.1.3173.}
\physDesc{Brief, 1 Blatt, 3 Seiten
\newline{}Handschrift: schwarze Tinte, deutsche Kurrent
\newline{}Schnitzler: 1) mit Bleistift das Jahr »{[}1{]}903.« vermerkt  2) mit rotem Buntstift eine Unterstreichung}\toendnotes[C]{\smallbreak}
\pstart
           \noindent{}{\pb}\textcolor{gray}{\textbf{\textsc{Telephon \textbf{4167.}}}}\hfill \textcolor{gray}{\textbf{\textsc{Telegramm-Adresse:}}}\pend
           
\pstart
           \textcolor{gray}{\textbf{\textsc{und \textbf{3940.}}}}\hfill \textcolor{gray}{\textbf{\textbf{\textsc{\textcolor{pink}{Palast Fürstenhof}{}\ledrightnote{\textcolor{pink}{Fürstenhof}}{ }\textcolor{pink}{Frankfurtmain}{}\ledrightnote{\textcolor{pink}{Frankfurt am Main}}.}}}}\pend
           
\pstart
           \centering{}\textcolor{gray}{\textbf{\textsc{\textbf{\textcolor{pink}{Palast-Hotel}{}\ledrightnote{\textcolor{pink}{Fürstenhof}}}}}}\pend
           
\pstart
           \noindent{}\centering{}\textcolor{gray}{\textbf{\textsc{\textcolor{pink}{Fürstenhof}{}\ledrightnote{\textcolor{pink}{Fürstenhof}}}}}\pend
           
\pstart
           \noindent{}\centering{}\textcolor{gray}{\textbf{\textcolor{blue}{LOUIS BOLLE-RITZ}{}\ledrightnote{\textcolor{blue}{Louis Bolle-Ritz}}.}}\pend
           
\pstart
           \noindent{}\centering{}\textcolor{gray}{\textbf{\textsc{\textcolor{pink}{(Kaiserstrasse}{}\ledrightnote{\textcolor{pink}{Kaiserstraße}} – \textcolor{pink}{Kronprinzenstrasse}{}\ledrightnote{\textcolor{pink}{Münchener Straße}})}}}\pend
           
\pstart
           \raggedleft{}\textcolor{gray}{\textbf{\textcolor{pink}{Frankfurt \textsuperscript{a/}M.}{}\ledrightnote{\textcolor{pink}{Frankfurt am Main}}}}{ }3. Januar.\pend
           
\pstart\center{}Mein lieber Freund,\pend
\pstart
           Dank für Deinen lieben und theilnehmenden Brief. Morgen fahre ich zurück. Es waren
               entſetzliche Tage. Geſtern habe ich \label{K_L03360-1v}\edtext{\textcolor{blue}{ſie}{}\ledrightnote{{$\rightarrow$}\textcolor{blue}{Theodore Rottenberg}}}{\lemma{\textnormal{\emph{ſie}}}\Cendnote{\textnormal{siehe Paul Goldmann an Arthur Schnitzler, 28. 12. [1902]}}}\label{K_L03360-1h}, nach \strikeout{i} inſtändigen Bitten, zum letzten Mal
               geſehen. Ich habe ſie flehentlich gebeten, zu mir zurückzukehren, habe ihr
               verſprochen, ſie zu heirathen. Sie lächelt ſchmerzlich: »zu ſpät«. Sie hat mich nicht
               mehr lieb. Der {\pb}»\textcolor{blue}{Andere}{}\ledrightnote{{$\rightarrow$}\textcolor{blue}{?? [Partner von Theodore Rottenberg, Ende 1902/Anfang 1903]}}« exiſtirt. Er iſt ein rückenmarkskranker Millionär.
               Was \textcolor{blue}{ſie}{}\ledrightnote{{$\rightarrow$}\textcolor{blue}{Theodore Rottenberg}} an ihn feſſelt, iſt
               eine Miſchung von Romantik, Mitleid und Behagen an Geld und Wohlleben. Sie hat ihn
               gern, ſie gefällt ſich in der Rolle der \label{K_L03360-2v}\edtext{»\textsc{\textcolor{green}{\textcolor{blue}{Mouche}{}\ledrightnote{{$\rightarrow$}\textcolor{blue}{Elise Krinitz}}}{}\ledrightnote{{$\rightarrow$}\textcolor{green}{Gedichte an die Mouche}}}«}{\lemma{\textnormal{\emph{»Mouche«}}}\Cendnote{\textnormal{»Mouche« war \textcolor{blue}{Heinrich Heine}s Kosename für seine letzte Geliebte, \textcolor{blue}{Elise Krinitz}. In \textcolor{blue}{Heine}s Nachlass finden sich auch fünf \emph{\textcolor{green}{Gedichte an die Mouche}}.}}}\label{K_L03360-2h}, – und ſie iſt glücklich,
               daß er mit ihr nach \textsc{\textcolor{pink}{Monte Carlo}{}\ledrightnote{\textcolor{pink}{Monte Carlo}}} reiſen wird. Alles Wundervolle und alles Gemeine iſt in dieſer \textcolor{blue}{Frau}{}\ledrightnote{{$\rightarrow$}\textcolor{blue}{Theodore Rottenberg}} gemiſcht. Das gütigſte
               Herz und die ſchamloſeſten dirnenhaften Inſtinkte. Ich müßte, aus moraliſchen und
               Vernunft-Gründen, froh ſein, von ihr loszukommen. Aber was nützen Vernunft und Moral,
               da ich ſie wahnſinnig liebe?\pend
           
\pstart
           Dank für Deine guten Worte! {\pb}Ich glaube nicht, daß
               ich darüber hinwegkommen werde. \strikeout{D\textcolor{gray}{e}r} Was blühend in meinem Leben war, iſt vernichtet, –
               vernichtet durch meine Schuld. Hätte ich erkannt, was ich an ihr beſaß, – hätte ich
               mich ihrer angenommen, – wäre ich nicht ein niederträchtiger Egoiſt geweſen, – ich
               hätte ſie behalten.\pend
           
\pstart
           Adieu, liebſter Freund! Grüße \textcolor{blue}{Olga}{}\ledrightnote{\textcolor{blue}{Olga Schnitzler}} und den
               dicken \textcolor{blue}{Buben}{}\ledrightnote{{$\rightarrow$}\textcolor{blue}{Heinrich Schnitzler}}! {\\[\baselineskip]}Dein
               getreuer {\\[\baselineskip]}\spacefill\mbox{Paul Goldmann}\pend
           \leftskip=0em{}\endnumbering\briefempfaengerindex{Schnitzler, Arthur@\textsc{Schnitzler, Arthur}!zzzGoldmann, Paul@\emph{von Paul Goldmann}!1903-01-031@{3. 1. {[}1903{]}}|)be}\mylabel{h}
\begin{anhang}
\end{anhang}\normalsize

\doendnotes{C}
\bigskip
\vfill

\clearpage

\footnotesize

\lohead{\textsc{register}}

% Definiere theindex-Environment komplett neu ohne reledmac
\makeatletter
\renewenvironment{theindex}{%
  \section*{\indexname}%
  \setlength{\parindent}{0pt}%
  \setlength{\parskip}{0pt plus 0.3pt}%
  \let\item\@idxitem
}{%
  \clearpage
}
\makeatother

\IfFileExists{\jobname-pw.ind}{\input{\jobname-pw.ind}}{}

\end{document}

      