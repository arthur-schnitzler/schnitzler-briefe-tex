%% latex-korrekturansicht-vorspann.tex
%% Vorspann für die Korrekturansicht.
%% Lädt die gemeinsame Datei latex-vorspann.tex mit gesetztem Schalter.

\newif\ifkorrekturansicht
\korrekturansichttrue

\input{../tex-inputs/latex-vorspann}


\renewcommand{\erwaehntePersonen}{Personen: Georg Brandes, Felix Salten}
\renewcommand{\erwaehnteOrte}{Orte: Wien}
\renewcommand{\erwaehnteWerke}{}
\section[ Arthur Schnitzler an Felix Salten, {[}27. 1. 1898?{]}]{Arthur Schnitzler an Felix Salten, {[}27. 1. 1898?{]}}
\nopagebreak\mylabel{v}
\rehead{ }\normalsize\beginnumbering\briefempfaengerindex{Salten, Felix@\textsc{Salten, Felix}!zzzSchnitzler, Arthur@\emph{von Arthur Schnitzler}!1898-01-271@{{[}27. 1. 1898?{]}}|(be}
\toendnotes[C]{\smallbreak\pagebreak[2]}\Standort{Wienbibliothek im Rathaus, ZPH 1681, 2.1.516.}
\physDesc{Karte, 103 Zeichen
\newline{}Handschrift: Bleistift, deutsche Kurrent
\newline{}Ordnung: mit Bleistift von unbekannter Hand Nummerierung der Blätter des Konvoluts:
                                    »36« }\toendnotes[C]{\smallbreak}
\pstart
           \noindent{}{\pb}lieber Salten, bitte ko{\geminationm}en Sie \label{K_L03034-1v}\edtext{morgen Freitag}{\lemma{\textnormal{\emph{morgen Freitag}}}\Cendnote{\textnormal{Dadurch ist das Korrespondenzstück
                  datierbar, vgl. A. S.: \emph{Tagebuch}, 28. 1. 1898.}}}\label{K_L03034-1h}{ }8 Uhr zum Nachtmalhl {\pb}zu uns
                  (\textcolor{blue}{Brandes}{}\ledrightnote{\textcolor{blue}{Georg Brandes}}{ }\textsc{etc.})\pend
           
\pstart
           Herzlichſt {\\[\baselineskip]}Ihr \spacefill\mbox{Arthur}\pend
           \leftskip=0em{}\endnumbering\briefempfaengerindex{Salten, Felix@\textsc{Salten, Felix}!zzzSchnitzler, Arthur@\emph{von Arthur Schnitzler}!1898-01-271@{{[}27. 1. 1898?{]}}|)be}\mylabel{h}  \normalsize

\doendnotes{C}
\bigskip
\vfill

\clearpage

\footnotesize

\lohead{\textsc{register}}

% Definiere theindex-Environment komplett neu ohne reledmac
\makeatletter
\renewenvironment{theindex}{%
  \section*{\indexname}%
  \setlength{\parindent}{0pt}%
  \setlength{\parskip}{0pt plus 0.3pt}%
  \let\item\@idxitem
}{%
  \clearpage
}
\makeatother

\IfFileExists{\jobname-pw.ind}{\input{\jobname-pw.ind}}{}

\end{document}

      