%% latex-korrekturansicht-vorspann.tex
%% Vorspann für die Korrekturansicht.
%% Lädt die gemeinsame Datei latex-vorspann.tex mit gesetztem Schalter.

\newif\ifkorrekturansicht
\korrekturansichttrue

\input{../tex-inputs/latex-vorspann}


               \section[Paul Goldmann an Arthur Schnitzler, Paul Goldmann an Arthur Schnitzler, 1. 2. {[}1896{]}]{ Paul Goldmann an Arthur Schnitzler, 1. 2. {[}1896{]}}\nopagebreak\mylabel{v}\rehead{ }\normalsize\beginnumbering\briefempfaengerindex{Schnitzler, Arthur@\textsc{Schnitzler, Arthur}!zzzGoldmann, Paul@\emph{von Paul Goldmann}!1896-02-011@{1. 2. {[}1896{]}}|(be} \toendnotes[C]{\smallbreak\pagebreak[2]} \Standort{DLA, A:Schnitzler, HS.NZ85.1.3166.}
\physDesc{Brief, 3 Blätter, 12 Seiten
\newline{}Handschrift: blaue Tinte, deutsche Kurrent\newline{}Beilage: handschriftlicher Brief: 1 stark beschnittener Ausschnitt aus
                                 einem Brief von Wally Rosengart an Goldmann, blaue Tinte,
                                 lateinische Kurrentschrift. Auf der Rückseite des Schnippsels
                                 steht: »{\pb}Mein lieber Paul – es fehlt
                                       \damage{uns} leider alles, um d\textcolor{gray}{en}« 
\newline{}Schnitzler: 1) mit Bleistift das Jahr »96« vermerkt 2) mit rotem Buntstift eine Unterstreichung}\toendnotes[C]{\smallbreak}\pstart
           \noindent{}{\pb}\textcolor{gray}{\textbf{\textbf{\textcolor{brown}{Frankfurter Zeitung}{}\ledrightnote{\textcolor{brown}{Frankfurter Zeitung}}}}}\pend
           \pstart
           \textcolor{gray}{\textbf{(\textcolor{brown}{\begin{otherlanguage}{french}Gazette de Francfort\end{otherlanguage}}{}\ledrightnote{\textcolor{brown}{Frankfurter Zeitung}}).}}\pend
           \pstart
           \textcolor{gray}{\textbf{\textbf{\begin{otherlanguage}{french}Fondateur M.\end{otherlanguage}{ }\textcolor{blue}{L. Sonnemann}{}\ledrightnote{\textcolor{blue}{Leopold Sonnemann}}.}}}\pend
           \pstart
           \begin{otherlanguage}{french}\textcolor{gray}{\textbf{\textcolor{green}{Journal}{}\ledrightnote{→\textcolor{green}{Frankfurter Zeitung}} politique,
                        financier,}}\end{otherlanguage}\pend
           \pstart
           \begin{otherlanguage}{french}\textcolor{gray}{\textbf{commercial et littéraire.}}\end{otherlanguage}\pend
           \pstart
           \begin{otherlanguage}{french}\textcolor{gray}{\textbf{\textbf{Paraissant trois fois par jour.}}}\end{otherlanguage}\hfill \textsc{\textcolor{pink}{Paris}{}\ledrightnote{\textcolor{pink}{Paris}}}, 1. Februar.\pend
           \pstart
           \begin{otherlanguage}{french}\textcolor{gray}{\textbf{\textbf{Bureau à \textcolor{pink}{Paris}{}\ledrightnote{\textcolor{pink}{Paris}}}}}\end{otherlanguage}\pend
           \pstart
           \begin{otherlanguage}{french}\textcolor{gray}{\textbf{\textbf{\textcolor{pink}{24. Rue Feydeau}{}\ledrightnote{\textcolor{pink}{rue Feydeau}}.}}}\end{otherlanguage}\pend
           \pstart\center{}Mein lieber Freund,\pend\pstart
           Herzlich willkommen in \label{K_L02766-1v}\edtext{\textcolor{pink}{Berlin}{}\ledrightnote{\textcolor{pink}{Berlin}}}{\lemma{\textnormal{\emph{Berlin}}}\Cendnote{\textnormal{Für die Premiere der \emph{\textcolor{green}{Liebelei}} am \textcolor{pink}{Deutschen
                     Theater} (4. 2. 1896) war \textcolor{blue}{Schnitzler}
                  zwischen 30. 1. 1896
                  und 10. 2. 1896 in
                     \textcolor{pink}{Berlin}.}}}\label{K_L02766-1h}! Möge Dir neues Gute dort
               beſchieden ſein!\pend
           \pstart
           Ich hörte dieſer Tage, »\textcolor{green}{\textcolor{green}{Sterben}{}\ledrightnote{→\textcolor{green}{Mourir. Roman}}}{}\ledrightnote{\textcolor{green}{Sterben. Novelle}}« werde demnächſt hier bei \textsc{\textcolor{brown}{Perrin}{}\ledrightnote{\textcolor{brown}{Éditions Perrin}}} erſcheinen u. \textsc{\textcolor{blue}{Ed. Rod}{}\ledrightnote{\textcolor{blue}{Édouard Rod}}} intereſſire ſich ganz beſonders dafür. Das wird Dir hoffentlich einen großen
                  \label{K_L02766-9v}\edtext{Artikel}{\lemma{\textnormal{\emph{Artikel}}}\Cendnote{\textnormal{nicht geschehen}}}\label{K_L02766-9h} in den »\textsc{\textcolor{green}{Débats}{}\ledrightnote{\textcolor{green}{Journal des débats. Politiques et littéraires}}}« eintragen, zu deſſen Literatur-Referenten \textsc{\textcolor{blue}{Rod}{}\ledrightnote{\textcolor{blue}{Édouard Rod}}} gehört.\pend
           \pstart
           Von der \textcolor{green}{Überſetzung}{}\ledrightnote{→\textcolor{green}{Amourette. Pièce en trois actes}}s-Angelegenheit betreffend die {\pb}»\textcolor{green}{Liebelei}{}\ledrightnote{\textcolor{green}{Liebelei. Schauspiel in drei Akten}}« habe ich einſtweilen wenig
               Erfreuliches zu melden. Ich hatte dieſer Tage Rendezvous mit \textsc{\textcolor{blue}{Thorel}{}\ledrightnote{\textcolor{blue}{Jean Thorel}}}. Er hat Schritte bei \textsc{\textcolor{blue}{Carré}{}\ledrightnote{\textcolor{blue}{Albert Carré}}}, dem \textcolor{blue}{Director}{}\ledrightnote{→\textcolor{blue}{Albert Carré}} des »\textsc{\textcolor{brown}{Vaudeville}{}\ledrightnote{\textcolor{brown}{Théâtre du Vaudeville}}}« gethan; aber \textsc{\textcolor{blue}{Carré}{}\ledrightnote{\textcolor{blue}{Albert Carré}}} hat geantwortet: das \textcolor{pink}{Pariſ}{}\ledrightnote{\textcolor{pink}{Paris}}er Publicum
               intereſſire ſich nicht mehr für fremde Stücke (was wahr iſt), intereſſire ſich nicht
               für \textsc{\label{K_L02766-3v}\edtext{\begin{otherlanguage}{french}moeurs \textcolor{pink}{Vienn}{}\ledrightnote{→\textcolor{pink}{Wien}}oises\end{otherlanguage}}{\lemma{\textnormal{\emph{moeurs Viennoises}}}\Cendnote{\textnormal{französisch: \textcolor{pink}{Wien}er Sitten}}}\label{K_L02766-3h} etc}. Immerhin, wenn \textsc{\textcolor{blue}{Thorel}{}\ledrightnote{\textcolor{blue}{Jean Thorel}}}{ }\strikeout{es} das \textcolor{green}{Stück}{}\ledrightnote{→\textcolor{green}{Liebelei. Schauspiel in drei Akten}} überſetzen wolle, werde er es gern leſen. Das iſt kein
               abſolutes Nein, aber es iſt nicht viel Hoffnung {\pb}in
               der Antwort. Ich denke daran, die \textcolor{green}{Überſetzung}{}\ledrightnote{→\textcolor{green}{Amourette. Pièce en trois actes}} eventuell der \strikeout{\textsc{\textcolor{blue}{Réjane}{}\ledrightnote{\textcolor{blue}{Réjane}}}}{ }\textsc{\textcolor{blue}{Réjane}{}\ledrightnote{\textcolor{blue}{Réjane}}} zu ſenden. Wenn dieſe das \textcolor{green}{Stück}{}\ledrightnote{→\textcolor{green}{Liebelei. Schauspiel in drei Akten}} ſpielen will, iſt die Sache gemacht, trotz der Anſichten \textsc{\textcolor{blue}{Carré}{}\ledrightnote{\textcolor{blue}{Albert Carré}}s} über die \textsc{\begin{otherlanguage}{french}moeurs \textcolor{pink}{Vienn}{}\ledrightnote{→\textcolor{pink}{Wien}}oises\end{otherlanguage}}. Aber dazu muß es erſt überſetzt ſein. Das einzige \introOben{}große\introOben{} Theater, das außer dem \textsc{\textcolor{brown}{Vaudeville}{}\ledrightnote{\textcolor{brown}{Théâtre du Vaudeville}}}{ }\strikeout{ſ} noch in Betracht käme, wäre \textsc{\textcolor{blue}{Sarah Bernhardt}{}\ledrightnote{\textcolor{blue}{Sarah Bernhardt}}s \textcolor{brown}{Renaissance}{}\ledrightnote{\textcolor{brown}{Théâtre de la Renaissance}}}, die \textsc{\textcolor{blue}{Sudermann}{}\ledrightnote{\textcolor{blue}{Hermann Sudermann}}s} »\textcolor{green}{Heimath}{}\ledrightnote{\textcolor{green}{Heimat}}« geſpielt hat. Aber ich glaube, da iſt erſt recht
               keine Ausſicht, denn \textsc{\textcolor{blue}{Sarah}{}\ledrightnote{\textcolor{blue}{Sarah Bernhardt}}} wird kaum ein {\pb}ausländiſches Stück
                  ſpielen\textcolor{gray}{,} das keine Rolle für ſie enthält. Bleiben die freien
               freien \textcolor{brown}{Bühnen}{}\ledrightnote{→\textcolor{brown}{Théâtre de l’Œuvre}{\newline}→\textcolor{brown}{Théâtre Libre}{\newline}→\textcolor{brown}{Théâtre des Escholiers}}: \textsc{\textcolor{brown}{Œuvre}{}\ledrightnote{\textcolor{brown}{Théâtre de l’Œuvre}}, \textcolor{brown}{Théâtre
                     Libre}{}\ledrightnote{\textcolor{brown}{Théâtre Libre}}, \textcolor{brown}{Escholiers}{}\ledrightnote{\textcolor{brown}{Théâtre des Escholiers}} etc}. \strikeout{Hi} Hier ſetzen wir ſo gut wie ſicher eine Aufführung
               durch. Aber wie wird man da Dein ſchönes \textcolor{green}{Stück}{}\ledrightnote{→\textcolor{green}{Liebelei. Schauspiel in drei Akten}} ſpielen!\pend
           \pstart
           Für alle weiteren Schritte iſt es \strikeout{\textcolor{gray}{×}} jedenfalls nothwendig, daß wir eine Überſetzung zur Hand haben. Dieſe iſt aber
               nur zu bekommen, wenn man zahlt. \textsc{\textcolor{blue}{Thorel}{}\ledrightnote{\textcolor{blue}{Jean Thorel}}} iſt ein armer \strikeout{T\textcolor{gray}{e}} Teufel, {\pb}der von ſeiner Feder lebt. Er kann
               ſich nicht an eine größere Arbeit machen, ohne daß man ſie ihm ſofort honorirt. \strikeout{\textcolor{gray}{Wenn}} Der \textcolor{blue}{Herr}{}\ledrightnote{→\textcolor{blue}{?? [Übersetzer]}} in \textsc{\textcolor{pink}{Lyon}{}\ledrightnote{\textcolor{pink}{Lyon}}} würde die Sache vielleicht umſonſt machen, aber nochmals: es wäre barer Unſinn,
               aus \textsc{\textcolor{pink}{Lyon}{}\ledrightnote{\textcolor{pink}{Lyon}}} ſich eine Überſetzung kommen zu laſſen. \strikeout{Die} Was
               aus der Provinz kommt, gilt hier für ſchlecht. Mein Rath iſt einſtweilen der: Warten
               wir die \textcolor{pink}{Berlin}{}\ledrightnote{\textcolor{pink}{Berlin}}er Aufführung \pend
           \pstart
           {[}XXXX Hier fehlt das FAKSIMILE DER RÜCKSEITE{]}\pend
           \pstart
           {\pb}herſtellen. Er ſprach zwar von 200 pro \textcolor{green}{Akt}{}\ledrightnote{→\textcolor{green}{Liebelei. Schauspiel in drei Akten}}, aber ich handle ſchon
               noch 100 herunter. Warten wir alſo einſtweilen noch ein paar Wochen\strikeout{\textcolor{gray}{n}} und reden wir dann weiter über die Sache.\pend
           \pstart
           Ich hoffe, Du ſchreibſt mir ein paar Zeilen über Deine \textcolor{pink}{Berlin}{}\ledrightnote{\textcolor{pink}{Berlin}}er Eindrücke und Erlebniſſe, die gewiß gut und froh ſein werden. In
                  \textcolor{pink}{Berlin}{}\ledrightnote{\textcolor{pink}{Berlin}} habe ich einen \textcolor{blue}{Onkel}{}\ledrightnote{→\textcolor{blue}{Hermann Mamroth}}, den \textcolor{blue}{Bruder}{}\ledrightnote{→\textcolor{blue}{Hermann Mamroth}} meiner \textcolor{blue}{Mutter}{}\ledrightnote{→\textcolor{blue}{Clementine Goldmann}}, einen braven, einfachen und \strikeout{ſeelens} ſeelensguten {\pb}Mann\strikeout{\textcolor{gray}{e}}, der mich erzogen hat. Er heißt \textsc{\textcolor{blue}{Hermann Mamroth}{}\ledrightnote{\textcolor{blue}{Hermann Mamroth}}} und wohnt \textcolor{pink}{\textsc{Bruecken-Allee} 8}{}\ledrightnote{\textcolor{pink}{Bartningallee}}. Wenn es Dir möglich wäre, ihm ein
               Billet zu einer Deiner Aufführungen zu ſchicken oder gar ihn zu \label{K_L02766-7v}\edtext{beſuchen}{\lemma{\textnormal{\emph{beſuchen}}}\Cendnote{\textnormal{nicht geschehen}}}\label{K_L02766-7h}, ſo würdeſt \strikeout{Du} Du ihm und mir eine große Freude machen. Wenn es Dir aber auch nur die
               mindeſten Umſtände macht, ſo laß’ \strikeout{es} es gehen {\pb}und betrachte dieſe Zeilen als nicht geſchrieben{\dotsfive}\pend
           \pstart
           Dein Bericht über die \label{K_L02766-5v}\edtext{Unterredung mit
                  \textsc{\textcolor{blue}{Bahr}{}\ledrightnote{\textcolor{blue}{Hermann Bahr}}}}{\lemma{\textnormal{\emph{Unterredung mit
                  Bahr}}}\Cendnote{\textnormal{siehe A. S.: \emph{Tagebuch}, 21. 1. 1896}}}\label{K_L02766-5h} hat mich ungemein intereſſirt. Aber geh’ mir doch mit all’ der complicirten
               Pſychologie. Setzen wir die einfache Probe, die das Herz erleichtern: \textsc{\textcolor{blue}{Bahr}{}\ledrightnote{\textcolor{blue}{Hermann Bahr}}} iſt ſo zu Dir, \strikeout{weil} weil er ein Schurke iſt,
               und er haßt Dich, weil er neidiſch auf Dich iſt. Das iſt der Kern der Sache. Dem
               kleinen {\pb}\textsc{\textcolor{blue}{Hugo}{}\ledrightnote{\textcolor{blue}{Hugo von Hofmannsthal}}} bin ich ſehr böſe. Man kann ſich wohl über Deine \strikeout{Lau} Launen ärgern, aber man ſchwankt nicht über die \label{K_L02766-6v}\edtext{Stellung zu Dir}{\lemma{\textnormal{\emph{Stellung zu Dir}}}\Cendnote{\textnormal{siehe A. S.: \emph{Tagebuch}, 21. 12. 1895}}}\label{K_L02766-6h}. Leute, die nicht klar ſehen, wer und was Du biſt, haben ſelber einen Defect.
               Ich erwarte mir längſt allerlei Enttäuſchungen \strikeout{über}
               von dem kleinen \textsc{\textcolor{blue}{Hugo}{}\ledrightnote{\textcolor{blue}{Hugo von Hofmannsthal}}} – vor allen \strikeout{Di} Dingen auf der Character-Seite.
               Er iſt viel zu eitel für ſeine jungen Jahre. Der Schurke \textsc{\textcolor{blue}{Bahr}{}\ledrightnote{\textcolor{blue}{Hermann Bahr}}} trägt {\pb}die Hauptſchuld daran, aber auch Ihr
               habt Schuld, denn Ihr habt ihn verziehen helfen. {\dotsfour}\pend
           \pstart
           Wenn Du alſo irgend etwas in \textcolor{pink}{Berlin}{}\ledrightnote{\textcolor{pink}{Berlin}} brauchſt, ſo
               telegraphire. Du haſt Recht, auf alle Empfehlungen zu verzichten. Die beſte
               Empfehlung iſt Dein \textcolor{green}{Stück}{}\ledrightnote{→\textcolor{green}{Liebelei. Schauspiel in drei Akten}}.\pend
           \pstart
           Und nun von Herzen Glück für Dienſtag!\pend
           \pstart
           In Treue{\\[\baselineskip]}Dein {\\[\baselineskip]}\spacefill\mbox{Paul Goldmann}\pend
           \leftskip=0em{}\pstart
           \noindent{}\label{T_L02766-1v}\edtext{Autograph meiner \textcolor{blue}{Schweſter}{}\ledrightnote{→\textcolor{blue}{Vally Rosengart}}, das eben eintrifft:}{\lemma{\textnormal{\emph{Autograph … eintrifft:}}}\Cendnote{\textnormal{Klebespuren legen nahe, dass die Beilage ursprünglich
                  auf die letzte Seite geklebt war.}}}\label{T_L02766-1h}\pend
           \pstart
           {\pb}{[}hs. Rosengart:{]} Schnitzler iſt ein lieber, reizender Mensch\pend
           \endnumbering\briefempfaengerindex{Schnitzler, Arthur@\textsc{Schnitzler, Arthur}!zzzGoldmann, Paul@\emph{von Paul Goldmann}!1896-02-011@{1. 2. {[}1896{]}}|)be}\mylabel{h}  \normalsize

\doendnotes{C}
\bigskip
\vfill

\clearpage

\footnotesize

\lohead{\textsc{register}}

% Definiere theindex-Environment komplett neu ohne reledmac
\makeatletter
\renewenvironment{theindex}{%
  \section*{\indexname}%
  \setlength{\parindent}{0pt}%
  \setlength{\parskip}{0pt plus 0.3pt}%
  \let\item\@idxitem
}{%
  \clearpage
}
\makeatother

\IfFileExists{\jobname-pw.ind}{\input{\jobname-pw.ind}}{}

\end{document}

      