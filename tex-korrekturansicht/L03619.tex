%% latex-korrekturansicht-vorspann.tex
%% Vorspann für die Korrekturansicht.
%% Lädt die gemeinsame Datei latex-vorspann.tex mit gesetztem Schalter.

\newif\ifkorrekturansicht
\korrekturansichttrue

\input{../tex-inputs/latex-vorspann}


\renewcommand{\erwaehntePersonen}{Personen: Karl Emil Franzos, Ottilie Franzos, Johann Schnitzler}
\renewcommand{\erwaehnteInstitutionen}{Institutionen: Adolf Bonz {\kaufmannsund}  Comp., Deutsche Dichtung}
\renewcommand{\erwaehnteOrte}{Orte: Berlin, Kaiserin-Augusta-Straße 71, Stuttgart, Wien}
\renewcommand{\erwaehnteWerke}{Werke: Amerika, Erbschaft, Jugend in Wien, Mein Freund Ypsilon. Aus den Papieren eines Arztes}
\section[Karl Emil Franzos an Arthur Schnitzler, {[}3. 5. 1888 – 11. 5. 1888?{]}]{Karl Emil Franzos an Arthur Schnitzler, {[}3. 5. 1888 –
               11. 5. 1888?{]}}
\nopagebreak\mylabel{v}
\rehead{ }\normalsize\beginnumbering\briefempfaengerindex{Schnitzler, Arthur@\textsc{Schnitzler, Arthur}!zzzFranzos, Karl Emil@\emph{von Karl Emil Franzos}!1888-05-112@{{[}3. 5. 1888 – 11. 5. 1888?{]}}|(be}
\toendnotes[C]{\smallbreak\pagebreak[2]}\Standort{DLA, A:Schnitzler, HS.1985.1.3025.}
\physDesc{Brief, 1 Blatt, 3 Seiten, 2073 Zeichen
\newline{}Handschrift Ottilie Franzos: schwarze Tinte, deutsche Kurrent
\newline{}Handschrift Karl Emil Franzos: schwarze Tinte, deutsche Kurrent (\noindent{}zwei Einfügungen, Unterschrift und Nachschrift)
\newline{}Schnitzler: 1) mit rotem Buntstift eine Unterstreichung  2) mit Bleistift »\textsc{Franzos}«}\toendnotes[C]{\smallbreak}
\pstart
           \centering{}{\pb}\textcolor{gray}{\textbf{Redaction der »\textcolor{brown}{Deutſchen
                        Dichtung}{}\ledrightnote{\textcolor{brown}{Deutsche Dichtung}}«.}}\pend
           
\pstart
           \textcolor{gray}{\textbf{Herausgeber:}}\hfill \textcolor{gray}{\textbf{Verlag:}}\pend
           
\pstart
           \textcolor{blue}{\textcolor{gray}{\textbf{Karl Emil Franzos}}}{}\ledrightnote{\textcolor{blue}{Karl Emil Franzos}}\hfill \textcolor{brown}{\textcolor{gray}{\textbf{Adolf Bonz {\kaufmannsund} Comp.}}}{}\ledrightnote{\textcolor{brown}{Adolf Bonz {\kaufmannsund} Comp.}}\pend
           
\pstart
           \textcolor{gray}{\textbf{\textcolor{pink}{Berlin}{}\ledrightnote{\textcolor{pink}{Berlin}}.}}\hfill \textcolor{gray}{\textbf{\textcolor{pink}{Stuttgart}{}\ledrightnote{\textcolor{pink}{Stuttgart}}.}}\pend
           
\pstart
           \raggedleft{}\textcolor{gray}{\textbf{\textcolor{pink}{Berlin}{}\ledrightnote{\textcolor{pink}{Berlin}},}} den 3. Mai \textcolor{gray}{\textbf{188}}8. \pend
           
\pstart
           \raggedleft{}\textcolor{gray}{\textbf{\textcolor{pink}{W. Kaiserin Auguſtaſtraße 71}{}\ledrightnote{\textcolor{pink}{Kaiserin-Augusta-Straße 71}}.}}\pend
           
\pstart\center{}Geehrter Herr Doctor!\pend
\pstart
           Ein an ſich nicht gerade erfreulicher Umſtand, ein Unwohlſein nämlich, welches mich
               für einige Tage an’s Bett bannte und mir eine unfreiwillige Muße auferlegte, hat mir
               andrerſeits ermöglicht, Ihrem Wunſche, Ihnen meine Anſicht über Ihre beiden \label{K_L03619-1v}\edtext{\textcolor{green}{Novellen}{}\ledrightnote{{$\rightarrow$}\textcolor{green}{Amerika}{\newline}{$\rightarrow$}\textcolor{green}{Mein Freund Ypsilon. Aus den Papieren eines Arztes}{\newline}{$\rightarrow$}\textcolor{green}{Erbschaft}}}{\lemma{\textnormal{\emph{Novellen}}}\Cendnote{\textnormal{Von den erhaltenen Texten, die in diesem
                  Zeitraum entstanden, kommen \emph{\textcolor{green}{Erbschaft}}, \emph{\textcolor{green}{Mein Freund Ypsilon. Aus den Papieren eines
                     Arztes}} und \emph{\textcolor{green}{Amerika}} in Frage, vgl. A. S.: \emph{Tagebuch}, 19. 10. 1887 und \emph{\textcolor{green}{Jugend in Wien}} (\textcolor{blue}{Arthur Schnitzler}: \emph{\textcolor{green}{Jugend in Wien. Eine Autobiographie}}. Mit einem Nachwort
                     von Friedrich Torberg. Wien, München,
                     Zürich, New York:
                        \emph{S. Fischer}{ }1968, S. 320). \textcolor{blue}{Schnitzler} hatte \textcolor{blue}{Franzos} um seine
                  Einschätzung gebeten (vgl. Arthur Schnitzler an Karl Emil Franzos, 29. 4. 1888).}}}\label{K_L03619-1h} zu ſagen, ſchon jetzt entſprechen zu können, mehr aber als eben eine
               ſubjektive Anſchauung beanſpruche ich gewiß nicht zu bieten. Beide \textcolor{green}{Arbeiten}{}\ledrightnote{{$\rightarrow$}\textcolor{green}{Amerika}{\newline}{$\rightarrow$}\textcolor{green}{Mein Freund Ypsilon. Aus den Papieren eines Arztes}{\newline}{$\rightarrow$}\textcolor{green}{Erbschaft}} waren mir
               insbeſondere ihrer Entſtehung \introOben{}{[}hs. Karl Emil Franzos:{]} nach\introOben{} pſychologiſch intereſſant, ſie ſind
               ſichtlich die Erzeugniſſe eines jungen Arztes, welcher den realen Thatſachen ſeines
               Berufs dadurch eine Art idealiſirenden Gegengewichts zu geben verſucht. {\pb}Daraus erklärt ſich das eigenthümliche
               Gegenüberſtehen der beiden Momente, welche ſich in den \textcolor{green}{Novellen}{}\ledrightnote{{$\rightarrow$}\textcolor{green}{Amerika}{\newline}{$\rightarrow$}\textcolor{green}{Mein Freund Ypsilon. Aus den Papieren eines Arztes}{\newline}{$\rightarrow$}\textcolor{green}{Erbschaft}} gleich ſcharf
               vertreten finden, der romantiſchen Erfindung und der realiſtischen Wahl des
               Grundproblems, welches ja in beiden ein rein pathologisches iſt. Es iſt aber eben
               auch nur ein Nebeneinanderſtehen und keine harmoniſche Miſchung, was wohl darin ſeine
               Erklärung findet, daß beide Elemente in ihrer extremſten Ausprägung hier vertreten
               erſcheinen. Einerſeits wird die Romantik in beiden \textcolor{green}{Novellen}{}\ledrightnote{{$\rightarrow$}\textcolor{green}{Amerika}{\newline}{$\rightarrow$}\textcolor{green}{Mein Freund Ypsilon. Aus den Papieren eines Arztes}{\newline}{$\rightarrow$}\textcolor{green}{Erbschaft}} zur
               Hyperromantik \introOben{}{[}hs. Karl Emil Franzos:{]} getrieben\introOben{}, andrerſeits wird das pathologiſche
               Problem ſehr hart und ſtreng betont. Dies iſt meines bescheidenen Ermeſſens jene
               Klippe, welche Sie künftig zu umſchiffen haben werden, denn obwohl beide \textcolor{green}{Novellen}{}\ledrightnote{{$\rightarrow$}\textcolor{green}{Amerika}{\newline}{$\rightarrow$}\textcolor{green}{Mein Freund Ypsilon. Aus den Papieren eines Arztes}{\newline}{$\rightarrow$}\textcolor{green}{Erbschaft}}
               meines Erachtens nicht ſo druckreif ſind, als daß ich einem ernſthaft ſtrebenden
               Manne damit vor die Öffentlichkeit zu treten anrathen könnte, ſo wäre es doch
               zunächſt für Sie und {\pb}wenn Sie die Arbeit ernſthaft
               anfaſſen, wohl nicht für Sie allein Schade, wenn Sie es dabei bewenden laſſen
               wollten.\pend
           
\pstart
           Mit beſten Empfehlungen{\\[\baselineskip]} Ihr ergebenſter {\\[\baselineskip]}\spacefill\mbox{{[}hs. Karl Emil Franzos:{]} Franzos}\pend
           \leftskip=0em{}
\pstart
           \noindent{}{[}hs. Ottilie Franzos:{]} Herrn \textsc{Dr. A. Schnitzler}.\pend
           {\vspace{1\baselineskip}}
\pstart
           {[}hs. Karl Emil Franzos:{]} Geehrter Herr Dr! Der vorſtehende Brief iſt leider
               durch ein \label{K_L03619-3v}\edtext{Überſehen meiner \textcolor{blue}{Gattin}{}\ledrightnote{{$\rightarrow$}\textcolor{blue}{Ottilie Franzos}}}{\lemma{\textnormal{\emph{Überſehen meiner Gattin}}}\Cendnote{\textnormal{Die Involvierung
                  von \textcolor{blue}{Ottilie Franzos} in der Begründung lässt sich als Hinweis lesen, dass sie den vorliegenden Brief
                  auch für ihren Mann geschrieben hat.}}}\label{K_L03619-3h} bis \label{K_L03619-2v}\edtext{heute}{\lemma{\textnormal{\emph{heute}}}\Cendnote{\textnormal{Die Nachschrift ist undatiert und folglich lässt sich nicht
                  mit Sicherheit feststellen, ob \textcolor{blue}{Schnitzler}
                  das Korrespondenzstück noch vor seiner (vorgezogenen) Abreise aus \textcolor{pink}{Berlin} am 12. 5. 1888 erhalten hat – oder es ihm nach \textcolor{pink}{Wien} nachgesandt wurde. Es ist vorstellbar, dass
                     \textcolor{blue}{Franzos} selbst bemerkte, dass seine
                  Antwort liegen geblieben war. Naheliegend ist aber, dass \textcolor{blue}{Schnitzlers} Brief vom 11. 5. 1888{ }\textcolor{blue}{Franzos} an sein nicht abgesandtes Schreiben
                  erinnerte und er die Nachschrift verfasste und schnell noch spedierte, um sie \textcolor{blue}{Schnitzler} noch vor der Abreise zukommen zu
                  lassen.}}}\label{K_L03619-2h} unbeſtellt geblieben. Ich ſende Ihnen den ſelben nun und unſere
               beſten Abſchiedsgrüße dazu. Vergeſſen Sie uns nicht, we{\geminationn}
               Sie Ihr Weg wieder hierher führt und ſagen Sie Ihrem Herrn \textcolor{blue}{Vater}{}\ledrightnote{\textcolor{blue}{Johann Schnitzler}} unſere beſten Empfehlungen. Herzlich grüßend\pend
           \pstart  Ihr \spacefill\mbox{Fr.}\pend{}\endnumbering\briefempfaengerindex{Schnitzler, Arthur@\textsc{Schnitzler, Arthur}!zzzFranzos, Karl Emil@\emph{von Karl Emil Franzos}!1888-05-112@{{[}3. 5. 1888 – 11. 5. 1888?{]}}|)be}\mylabel{h}  \normalsize

\doendnotes{C}
\bigskip
\vfill

\clearpage

\footnotesize

\lohead{\textsc{register}}

% Definiere theindex-Environment komplett neu ohne reledmac
\makeatletter
\renewenvironment{theindex}{%
  \section*{\indexname}%
  \setlength{\parindent}{0pt}%
  \setlength{\parskip}{0pt plus 0.3pt}%
  \let\item\@idxitem
}{%
  \clearpage
}
\makeatother

\IfFileExists{\jobname-pw.ind}{\input{\jobname-pw.ind}}{}

\end{document}

      