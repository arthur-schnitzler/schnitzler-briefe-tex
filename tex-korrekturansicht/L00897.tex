%% latex-korrekturansicht-vorspann.tex
%% Vorspann für die Korrekturansicht.
%% Lädt die gemeinsame Datei latex-vorspann.tex mit gesetztem Schalter.

\newif\ifkorrekturansicht
\korrekturansichttrue

\input{../tex-inputs/latex-vorspann}


               \section[Jakob Julius David an Arthur Schnitzler, 3. 3. 1899]{ Jakob Julius David an Arthur Schnitzler, 3. 3. 1899}\nopagebreak\mylabel{v}\rehead{ }\normalsize\beginnumbering\briefempfaengerindex{Schnitzler, Arthur@\textsc{Schnitzler, Arthur}!zzzDavid, Jakob Julius@\emph{von Jakob Julius David}!1899-03-031@{3. 3. 1899}|(be} \toendnotes[C]{\smallbreak\pagebreak[2]} \Standort{CUL, Schnitzler, B 25.}
\physDesc{Postkarte
\newline{}Handschrift: schwarze Tinte, lateinische Kurrent\newline{}Versand: 1) Stempel: »\nobreak{}\oindex{II., Leopoldstadt@\textbf{II., Leopoldstadt}, \emph{Bezirk (A.BZK)}|pwk}Wien 2/3, 3. 3. 99, 1–4N\nobreak{}«.  2) Stempel: »\nobreak{}\oindex{IX., Alsergrund@\textbf{IX., Alsergrund}, \emph{Bezirk (A.BZK)}|pwk}Wien 9/3, 3. 3. 99, 6.N\nobreak{}«. \newline{}Ordnung: mit Bleistift von unbekannter Hand nummeriert:
                                 »5« }\toendnotes[C]{\smallbreak}\pstart{}{\pb}Herrn D\textsuperscript{r}
                  Arthur Schnitzler\pend{}\pstart{}\textcolor{pink}{IX. Franckgaße N\textsuperscript{o} 1}{}\ledrightnote{\textcolor{pink}{Frankgasse}}.
               \pend{}{\bigskip}\pstart\center{}{\pb}Werther Herr!\pend\pstart
           Schön Dank. Also \label{K_L00897_1v}\edtext{Dienstag}{\lemma{\textnormal{\emph{Dienstag}}}\Cendnote{\textnormal{An diesem Tag fand die vierte Aufführung
                  der drei Einakter \emph{\textcolor{green}{Der grüne Kakadu – Paracelsus – Die
                     Gefährtin}} statt. Er dürfte die am 28. 2. 1899 erbetenen Freikarten bekommen haben.}}}\label{K_L00897_1h}.\pend
           \pstart
            Seither haben Sie ja wohl auch gesehen, daß ich coram publico nicht anders \label{K_L00897_2v}\edtext{\textcolor{green}{schrieb}{}\ledrightnote{→\textcolor{green}{Aus ungleichen Tagen}}}{\lemma{\textnormal{\emph{schrieb}}}\Cendnote{\textnormal{\textcolor{blue}{J. J. David}: \emph{\textcolor{green}{Aus ungleichen Tagen. (»Paracelsus«, Schauspiel; »Die Gefährtin«,
                        Schauspiel; »Der grüne Kakadu«, Groteske. Drei Einacter von Arthur
                        Schnitzler. Im Burgtheater zum erstenmale aufgeführt am 1. März 1899.)}}.
                     In: \emph{\textcolor{green}{Neues Wiener Journal}}, Jg. 7, Nr. 1925,
                        2. 3. 1899, S. 1–2.}}}\label{K_L00897_2h}. Unsere Kritik! Ein feines
               Capitel! \pend
           \pstart
           Bestens Ihr{\\[\baselineskip]}\spacefill\mbox{David}\pend
           \leftskip=0em{}\endnumbering\briefempfaengerindex{Schnitzler, Arthur@\textsc{Schnitzler, Arthur}!zzzDavid, Jakob Julius@\emph{von Jakob Julius David}!1899-03-031@{3. 3. 1899}|)be}\mylabel{h}  \normalsize

\doendnotes{C}
\bigskip
\vfill

\clearpage

\footnotesize

\lohead{\textsc{register}}

% Definiere theindex-Environment komplett neu ohne reledmac
\makeatletter
\renewenvironment{theindex}{%
  \section*{\indexname}%
  \setlength{\parindent}{0pt}%
  \setlength{\parskip}{0pt plus 0.3pt}%
  \let\item\@idxitem
}{%
  \clearpage
}
\makeatother

\IfFileExists{\jobname-pw.ind}{\input{\jobname-pw.ind}}{}

\end{document}

      