%% latex-korrekturansicht-vorspann.tex
%% Vorspann für die Korrekturansicht.
%% Lädt die gemeinsame Datei latex-vorspann.tex mit gesetztem Schalter.

\newif\ifkorrekturansicht
\korrekturansichttrue

\input{../tex-inputs/latex-vorspann}


\renewcommand{\erwaehntePersonen}{Personen: Jacob Isaacs, Stefan Zweig}
\renewcommand{\erwaehnteInstitutionen}{Institutionen: King’s College London, University of London}
\renewcommand{\erwaehnteOrte}{Orte: Vereinigtes Königreich, Wien}
\renewcommand{\erwaehnteWerke}{Werke: Tagebuch}
\section[Stefan Zweig an Arthur Schnitzler, {[}1924–1928{]}]{Stefan Zweig an Arthur Schnitzler, {[}1924–1928{]}}
\nopagebreak\mylabel{v}
\rehead{ }\normalsize\beginnumbering\briefempfaengerindex{Schnitzler, Arthur@\textsc{Schnitzler, Arthur}!zzzZweig, Stefan@\emph{von Stefan Zweig}!1924-01-011@{{[}1924–1928{]}}|(be}
\toendnotes[C]{\smallbreak\pagebreak[2]}\Standort{CUL, Schnitzler, B 118.}
\physDesc{Brief, 1 Blatt, 1 Seite, 326 Zeichen
\newline{}Handschrift: schwarze Tinte, lateinische Kurrent
\newline{}Schnitzler: mit rotem Buntstift eine Unterstreichung }\toendnotes[C]{\smallbreak}
\pstart
           \noindent{}{\pb}Verehrter Herr Doktor, der Dozent der \textcolor{brown}{Universität London}{}\ledrightnote{\textcolor{brown}{University of London}}{ }M\textsuperscript{r}{ }\textcolor{blue}{J. Isaacs}{}\ledrightnote{\textcolor{blue}{Jacob Isaacs}} würde auf seiner Reise nach \textcolor{pink}{Wien}{}\ledrightnote{\textcolor{pink}{Wien}} Sie ungemein gerne sehen: als \label{K_L03676-1v}\edtext{Dozent der \textcolor{pink}{englischen}{}\ledrightnote{\textcolor{pink}{Vereinigtes Königreich}} Literatur}{\lemma{\textnormal{\emph{Dozent … Literatur}}}\Cendnote{\textnormal{Diese
                  Position hatte \textcolor{blue}{Jacob Isaacs} von
                     1924 bis 1928 am \emph{\textcolor{brown}{King’s College}} inne, so dass das vorliegende Schreiben
                  in diesem Zeitraum gelaufen sein dürfte. \textcolor{blue}{Schnitzler} erwähnt kein Treffen im \emph{\textcolor{green}{Tagebuch}}. }}}\label{K_L03676-1h} kann er Ihnen wohl in mancher Auskunft nützlich sein
               und Sie würden, glaube ich, es nicht bedauern, ihm eine Stunde zu schenken.\pend
           
\pstart
           In alter inniger Verehrung Ihr{\\[\baselineskip]}\spacefill\mbox{Stefan Zweig}\pend
           \leftskip=0em{}\endnumbering\briefempfaengerindex{Schnitzler, Arthur@\textsc{Schnitzler, Arthur}!zzzZweig, Stefan@\emph{von Stefan Zweig}!1924-01-011@{{[}1924–1928{]}}|)be}\mylabel{h}
\begin{anhang}
\end{anhang}\normalsize

\doendnotes{C}
\bigskip
\vfill

\clearpage

\footnotesize

\lohead{\textsc{register}}

% Definiere theindex-Environment komplett neu ohne reledmac
\makeatletter
\renewenvironment{theindex}{%
  \section*{\indexname}%
  \setlength{\parindent}{0pt}%
  \setlength{\parskip}{0pt plus 0.3pt}%
  \let\item\@idxitem
}{%
  \clearpage
}
\makeatother

\IfFileExists{\jobname-pw.ind}{\input{\jobname-pw.ind}}{}

\end{document}

      