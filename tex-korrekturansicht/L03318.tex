%% latex-korrekturansicht-vorspann.tex
%% Vorspann für die Korrekturansicht.
%% Lädt die gemeinsame Datei latex-vorspann.tex mit gesetztem Schalter.

\newif\ifkorrekturansicht
\korrekturansichttrue

\input{../tex-inputs/latex-vorspann}


\renewcommand{\erwaehntePersonen}{Personen: Emil Aufricht, Julian Sternberg, Franz Strosse von Hofwehr}
\renewcommand{\erwaehnteInstitutionen}{Institutionen: Jung-Wiener Theater zum Lieben Augustin}
\renewcommand{\erwaehnteOrte}{Orte: Bad Ischl, Berlin, Cieszyn, Hamburg, Theater an der Wien, Welsberg-Taisten, Wien}
\renewcommand{\erwaehnteWerke}{Werke: Der Schrei der Liebe. Novelle, Die Gedenktafel der Prinzessin Anna, Die Insel. Monatsschrift mit Buchschmuck und Illustrationen, Empfängnis, Lieutenant Gustl. Novelle, Neue Freie Presse, Wir erhalten folgende Mittheilung: Das »Jung-Wiener Theater zum lieben Augustin«}
\section[ Felix Salten an Arthur Schnitzler, 18. 8. 1901]{Felix Salten an Arthur Schnitzler, 18. 8. 1901}
\nopagebreak\mylabel{v}
\rehead{ }\normalsize\beginnumbering\briefempfaengerindex{Schnitzler, Arthur@\textsc{Schnitzler, Arthur}!zzzSalten, Felix@\emph{von Felix Salten}!1901-08-181@{18. 8. 1901}|(be}
\toendnotes[C]{\smallbreak\pagebreak[2]}\Standort{CUL, Schnitzler, B 89, A 2.}
\physDesc{Briefkarte, 729 Zeichen
\newline{}Handschrift: Bleistift, lateinische Kurrent
\newline{}Ordnung: mit Bleistift von unbekannter Hand nummeriert: »142« }\toendnotes[C]{\smallbreak}
\pstart
           \noindent{}{\pb}\textcolor{gray}{\textbf{\textcolor{brown}{Jung-Wiener Theater}{}\ledrightnote{\textcolor{brown}{Jung-Wiener Theater zum Lieben Augustin}}}}\hfill \textcolor{gray}{\textbf{\textcolor{pink}{Wien}{}\ledrightnote{\textcolor{pink}{Wien}},}}{ }18. Aug. \textcolor{gray}{\textbf{190}}1\pend
           
\pstart
           \textcolor{gray}{\textbf{\textcolor{brown}{Zum lieben Augustin}{}\ledrightnote{\textcolor{brown}{Jung-Wiener Theater zum Lieben Augustin}}.}}\hfill \textcolor{gray}{\textbf{(\textcolor{pink}{Theater a. d.
                        Wien}{}\ledrightnote{\textcolor{pink}{Theater an der Wien}})}}\pend
           
\pstart
           \textcolor{gray}{\textbf{Direction.}}\pend
           
\pstart
           Lieber Freund, herzl. Dank für Ihre verschiedenen Ansichtskarten.
               Ich war jetzt wieder eine Woche in \textcolor{pink}{Ischl}{}\ledrightnote{\textcolor{pink}{Bad Ischl}} und gehe
               dieser Tage nochmals hin. Im September{ }\textcolor{pink}{Berlin}{}\ledrightnote{\textcolor{pink}{Berlin}}{ }{\kaufmannsund}{ }\textcolor{pink}{Hamburg}{}\ledrightnote{\textcolor{pink}{Hamburg}}. Ein Exemplar der \label{K_L03318-1v}\edtext{\textcolor{green}{\textcolor{green}{Insel}{}\ledrightnote{{$\rightarrow$}\textcolor{green}{Die Gedenktafel der Prinzessin Anna}}}{}\ledrightnote{\textcolor{green}{Die Insel. Monatsschrift mit Buchschmuck und Illustrationen}}}{\lemma{\textnormal{\emph{Insel}}}\Cendnote{\textnormal{siehe Felix Salten an Arthur Schnitzler, 28. 7. 1901}}}\label{K_L03318-1h} kann ich Ihnen doch erst nächste Woche schicken, und da weiß ich nicht, ob’s
               noch dafürsteht. Geben Sie mir, wenn’s noch sein kann, Directe Adreße an, damit es
               keinen {\pb}solchen Umweg macht. Was
               sagen Sie, in welch’ verschämter Weise \textcolor{blue}{st-g}{}\ledrightnote{\textcolor{blue}{Julian Sternberg}} mir
                  \label{K_L03318-2v}\edtext{\textcolor{green}{Reclame}{}\ledrightnote{{$\rightarrow$}\textcolor{green}{Wir erhalten folgende Mittheilung: Das »Jung-Wiener Theater zum lieben Augustin«}}}{\lemma{\textnormal{\emph{Reclame}}}\Cendnote{\textnormal{[\textcolor{blue}{Julian Sternberg}]: \emph{\textcolor{green}{Wir erhalten folgende Mittheilung: Das »Jung-Wiener Theater
                        zum lieben Augustin«}}. In: \emph{\textcolor{green}{Neue Freie
                        Presse}}, Nr. 13.283, 18. 8. 1901,
                     Morgenblatt, S. 9.}}}\label{K_L03318-2h} gemacht hat? Heuer scheint’s im Sommer nur
               lauter \textcolor{green}{Lieutenant Gustl}{}\ledrightnote{{$\rightarrow$}\textcolor{green}{Lieutenant Gustl. Novelle}}’s zu
               geben – (\label{K_L03318-3v}\edtext{\textcolor{pink}{Teschen}{}\ledrightnote{\textcolor{pink}{Cieszyn}}}{\lemma{\textnormal{\emph{Teschen}}}\Cendnote{\textnormal{In \textcolor{pink}{Teschen} war im Juli der Bäckermeister \textcolor{blue}{Emil Aufricht} vom Lieutenant \textcolor{blue}{Franz Strosse, Edler von Hochwehr}, als
                  »Saujud« beschimpft worden. Dieser nannte folglich den anderen entweder
                  unmittelbar oder im Gespräch mit Dritten »Lausbub«. Daraufhin lauerte \textcolor{blue}{Strosse} mit Gefährten dem Bäcker auf. Sie
                  verprügelten ihn, er erlitt schwere Kopfverletzungen und ihm mussten vier Finger
                  amputiert werden.}}}\label{K_L03318-3h} ec.) Neues gibts genug, aber es wär’ zu weitläufig. Leben
               Sie herzlich wol, hoffentlich auf baldiges Wiedersehen.\pend
           
\pstart
           Ihr {\\[\baselineskip]}\spacefill\mbox{Salten}\pend
           \leftskip=0em{}
\pstart
           \noindent{}Ich schreibe eine \label{K_L03318-4v}\edtext{\textcolor{green}{Geschichte}{}\ledrightnote{{$\rightarrow$}\textcolor{green}{Der Schrei der Liebe. Novelle}}}{\lemma{\textnormal{\emph{Geschichte}}}\Cendnote{\textnormal{\emph{\textcolor{green}{Der Schrei der Liebe}} oder der nicht näher
                     bestimmbare Text \emph{\textcolor{green}{Empfängnis}}, den \textcolor{blue}{Salten}{ }\textcolor{blue}{Schnitzler} am 24. 3. 1902
                     vorlas?}}}\label{K_L03318-4h}, die hoffentl. besser ist als die \textcolor{green}{Prinzessin Anna}{}\ledrightnote{\textcolor{green}{Die Gedenktafel der Prinzessin Anna}}.\pend
           \endnumbering\briefempfaengerindex{Schnitzler, Arthur@\textsc{Schnitzler, Arthur}!zzzSalten, Felix@\emph{von Felix Salten}!1901-08-181@{18. 8. 1901}|)be}\mylabel{h}  \normalsize

\doendnotes{C}
\bigskip
\vfill

\clearpage

\footnotesize

\lohead{\textsc{register}}

% Definiere theindex-Environment komplett neu ohne reledmac
\makeatletter
\renewenvironment{theindex}{%
  \section*{\indexname}%
  \setlength{\parindent}{0pt}%
  \setlength{\parskip}{0pt plus 0.3pt}%
  \let\item\@idxitem
}{%
  \clearpage
}
\makeatother

\IfFileExists{\jobname-pw.ind}{\input{\jobname-pw.ind}}{}

\end{document}

      