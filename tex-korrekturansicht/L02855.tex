%% latex-korrekturansicht-vorspann.tex
%% Vorspann für die Korrekturansicht.
%% Lädt die gemeinsame Datei latex-vorspann.tex mit gesetztem Schalter.

\newif\ifkorrekturansicht
\korrekturansichttrue

\input{../tex-inputs/latex-vorspann}


               \section[ Paul Goldmann an Arthur Schnitzler, 5. 8. 1898]{Paul Goldmann an Arthur Schnitzler, 5. 8. 1898}\nopagebreak\mylabel{v}\rehead{ }\normalsize\beginnumbering\briefempfaengerindex{Schnitzler, Arthur@\textsc{Schnitzler, Arthur}!zzzGoldmann, Paul@\emph{von Paul Goldmann}!1898-08-052@{5. 8. 1898}|(be} \toendnotes[C]{\smallbreak\pagebreak[2]} \Standort{DLA, A:Schnitzler, HS.NZ85.1.3168.}
\physDesc{Bildpostkarte
\newline{}Handschrift: blaue Tinte, deutsche Kurrent\newline{}Versand: 1) Stempel: »\nobreak{}\oindex{Qingdao@\textbf{Qingdao}, \emph{Besiedelter Ort (A.BSO)}|pwk}Tsintau China, 5/8 98\nobreak{}«.  2) Stempel: »\nobreak{}Wien 9/3 72, 19. 9. 98, \textcolor{gray}{9}.V, Bestellt\nobreak{}«. 
\newline{}Schnitzler: mit Bleistift das Jahr »98« vermerkt }\pstart{}{\pb}\textcolor{gray}{\textbf{An}}\pend{}\pstart{}\textsc{Herrn}\pend{}\pstart{}\textsc{Dr. Arthur Schnitzler}\pend{}\pstart{}\textcolor{gray}{\textbf{in}}{ }\textsc{\textcolor{pink}{Wien}{}\ledrightnote{\textcolor{pink}{Wien}}}\pend{}\pstart{}\textcolor{pink}{\textsc{IX. Frankgaße 1}}{}\ledrightnote{\textcolor{pink}{Frankgasse}}.\pend{}{\bigskip}\pstart
           \noindent{}\centering{}{\pb}\textcolor{gray}{\textbf{\textcolor{pink}{Tsingtau: Hauptthor des Artillerielagers}{}\ledrightnote{\textcolor{pink}{Artillerielager Tsingtau}}.}}\pend
           \pstart
           Herzlichſten Gruß aus \textsc{\textcolor{pink}{Kiautschou}{}\ledrightnote{\textcolor{pink}{Kiautschou}}}! \spacefill\mbox{Paul Goldmann}\pend
           \pstart
           \textsc{\textcolor{pink}{Tsintau}{}\ledrightnote{\textcolor{pink}{Qingdao}}}, 5. Auguſt\pend
           \pstart
           \centering{}\textcolor{gray}{\textbf{Aufnahme in \textcolor{pink}{China}{}\ledrightnote{\textcolor{pink}{China}} u. Ausführung: \textcolor{brown}{Graph. Gesellschaft}{}\ledrightnote{\textcolor{brown}{Graphische Gesellschaft (Berlin)}}, \textcolor{pink}{Berlin}{}\ledrightnote{\textcolor{pink}{Berlin}}{ }1898.}}\pend
           \endnumbering\briefempfaengerindex{Schnitzler, Arthur@\textsc{Schnitzler, Arthur}!zzzGoldmann, Paul@\emph{von Paul Goldmann}!1898-08-052@{5. 8. 1898}|)be}\mylabel{h}\begin{anhang}\end{anhang}\normalsize

\doendnotes{C}
\bigskip
\vfill

\clearpage

\footnotesize

\lohead{\textsc{register}}

% Definiere theindex-Environment komplett neu ohne reledmac
\makeatletter
\renewenvironment{theindex}{%
  \section*{\indexname}%
  \setlength{\parindent}{0pt}%
  \setlength{\parskip}{0pt plus 0.3pt}%
  \let\item\@idxitem
}{%
  \clearpage
}
\makeatother

\IfFileExists{\jobname-pw.ind}{\input{\jobname-pw.ind}}{}

\end{document}

      