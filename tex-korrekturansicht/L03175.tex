%% latex-korrekturansicht-vorspann.tex
%% Vorspann für die Korrekturansicht.
%% Lädt die gemeinsame Datei latex-vorspann.tex mit gesetztem Schalter.

\newif\ifkorrekturansicht
\korrekturansichttrue

\input{../tex-inputs/latex-vorspann}


\renewcommand{\erwaehntePersonen}{Personen: Richard Beer-Hofmann, Alphonse Daudet, Edmond Huot de Goncourt, Jules Huot de Goncourt, Wilhelmine Mitterwurzer, Max Nordau, Georges Ohnet, Szigfrid Pongrácz, Ottilie Salten, Ilma Seiler-Willborn, Alexander Zeitlin}
\renewcommand{\erwaehnteInstitutionen}{Institutionen: Académie Goncourt, Akademie der Bildenden Künste Wien, Sommertheater Ischl}
\renewcommand{\erwaehnteOrte}{Orte: Bad Aussee, Bad Ischl, Draveil, Frankreich, Haus von Alphonse Daudet, Innsbruck, Jihlava, Kopenhagen, München, Nordkap, Salzburg, Schliersee, Skodsborg, Tegernsee, Trondheim, Valle d’Ampezzo, Wien}
\renewcommand{\erwaehnteWerke}{Werke: Adam und Eva, Die Schülerausstellung der Akademie, Frankfurter Zeitung, Liebelei. Schauspiel in drei Akten, Neue Freie Presse, Wiener Allgemeine Zeitung, † Edmond de Goncourt}
\section[ Felix Salten an Arthur Schnitzler, 21. 7. 1896]{Felix Salten an Arthur Schnitzler, 21. 7. 1896}
\nopagebreak\mylabel{v}
\rehead{ }\normalsize\beginnumbering\briefempfaengerindex{Schnitzler, Arthur@\textsc{Schnitzler, Arthur}!zzzSalten, Felix@\emph{von Felix Salten}!1896-07-211@{21. 7. 1896}|(be}
\toendnotes[C]{\smallbreak\pagebreak[2]}\Standort{CUL, Schnitzler, B 89, A 1.}
\physDesc{Brief, 1 Blatt, 4 Seiten, 2209 Zeichen
\newline{}Handschrift: Bleistift, lateinische Kurrent
\newline{}Schnitzler: mit Bleistift die Jahreszahl »96« ergänzt 
\newline{}Ordnung: mit Bleistift von unbekannter Hand nummeriert: »74« }\toendnotes[C]{\smallbreak}
\pstart
           \raggedleft{}{\pb}\textcolor{pink}{Wien}{}\ledrightnote{\textcolor{pink}{Wien}}, den 21. Juli\pend
           
\pstart
           Lieber Arthur, in dieser Welt geht garnichts vor, und
               es ist ganz gleichgiltig, ob man jetzt in \textcolor{pink}{Iglau}{}\ledrightnote{\textcolor{pink}{Jihlava}}
               lebt oder auf dem \label{K_L03175-1v}\edtext{\textcolor{pink}{Nordcap}{}\ledrightnote{\textcolor{pink}{Nordkap}}}{\lemma{\textnormal{\emph{Nordcap}}}\Cendnote{\textnormal{\textcolor{blue}{Schnitzler} war am 19. 7. 1896 am \textcolor{pink}{Nordkap} gewesen.}}}\label{K_L03175-1h} ist. Auf dem \textcolor{pink}{Nordcap}{}\ledrightnote{\textcolor{pink}{Nordkap}} ist’s besser, das ist das Ganze. Von
               grossen Ereignissen hab ich Ihnen nur zu melden, dass \label{K_L03175-2v}\edtext{Frau \textcolor{blue}{Seiler-Willborn}{}\ledrightnote{\textcolor{blue}{Ilma Seiler-Willborn}}
               plötzlich gestorben}{\lemma{\textnormal{\emph{Frau … gestorben}}}\Cendnote{\textnormal{Die Schauspielerin
                     \textcolor{blue}{Ilma Seiler-Willborn} war am 16. 7. 1896 in \textcolor{pink}{Wien}
                  verstorben.}}}\label{K_L03175-2h} ist, ferner dass man in \textcolor{pink}{Ischl}{}\ledrightnote{\textcolor{pink}{Bad Ischl}} nächstens Ihre \label{K_L03175-3v}\edtext{»\textcolor{green}{Liebelei}{}\ledrightnote{\textcolor{green}{Liebelei. Schauspiel in drei Akten}}« aufführen}{\lemma{\textnormal{\emph{»Liebelei« aufführen}}}\Cendnote{\textnormal{durch das \emph{\textcolor{brown}{Sommertheater
                     Ischl}}}}}\label{K_L03175-3h} wird. Doch dürfte Sie weder der eine noch der andere Unglücksfall zu sehr
               erschüttern. Diesen Sonntag bin ich in \textcolor{pink}{Ischl}{}\ledrightnote{\textcolor{pink}{Bad Ischl}} gewesen, vielmehr in \textcolor{pink}{Aussee}{}\ledrightnote{\textcolor{pink}{Bad Aussee}}, denn ich fuhr gleich in der Früh mit Frl. \textcolor{blue}{M.}{}\ledrightnote{\textcolor{blue}{Ottilie Salten}}{ }{\pb}dahin. Es schüttete in
               Strömen und wir blieben den ganzen Tag bei Frau \textcolor{blue}{Mitterwurzer}{}\ledrightnote{\textcolor{blue}{Wilhelmine Mitterwurzer}}. Ich gehe nun doch nicht ins \textcolor{pink}{Ampezzothal}{}\ledrightnote{\textcolor{pink}{Valle d’Ampezzo}}. Meine Adresse vom 1–7. Aug. ist jetzt \textcolor{pink}{Ischl}{}\ledrightnote{\textcolor{pink}{Bad Ischl}}. Von da an \textcolor{pink}{München}{}\ledrightnote{\textcolor{pink}{München}} bis zum 12. und von da ab \textcolor{pink}{Salzburg}{}\ledrightnote{\textcolor{pink}{Salzburg}} bis zum 20. August. Wir fahren wie
               Sie daraus sehen von \textcolor{pink}{Salzburg}{}\ledrightnote{\textcolor{pink}{Salzburg}} per Rad nach \textcolor{pink}{München}{}\ledrightnote{\textcolor{pink}{München}}, von da über \textcolor{pink}{Schliersee}{}\ledrightnote{\textcolor{pink}{Schliersee}}, \textcolor{pink}{Tegernsee}{}\ledrightnote{\textcolor{pink}{Tegernsee}}
               nach \textcolor{pink}{Innsbruck}{}\ledrightnote{\textcolor{pink}{Innsbruck}} und von dort nach \textcolor{pink}{Salzburg}{}\ledrightnote{\textcolor{pink}{Salzburg}}. Das ist Alles. Indessen bin ich ununterbrochen zu
               Hause, lese und arbeite. \textcolor{blue}{Zeitlin}{}\ledrightnote{\textcolor{blue}{Alexander Zeitlin}} hat keinen
                  \label{K_L03175-4v}\edtext{Preis}{\lemma{\textnormal{\emph{Preis}}}\Cendnote{\textnormal{der \emph{\textcolor{brown}{Akademie der bildenden
                     Künste}} in \textcolor{pink}{Wien}}}}\label{K_L03175-4h} beko{\geminationm}en, \textcolor{blue}{Popper}{}\ledrightnote{\textcolor{blue}{Szigfrid Pongrácz}}, der mit einer geradezu herrlichen Gruppe »\textcolor{green}{Adam und Eva}{}\ledrightnote{\textcolor{green}{Adam und Eva}}« um den {\pb}Rompreis concurrirte,
               wurde mit dem Specialschulpreis abgefunden. Ich schrieb einen \label{K_L03175-5v}\edtext{\textcolor{green}{Leitartikel}{}\ledrightnote{{$\rightarrow$}\textcolor{green}{Die Schülerausstellung der Akademie}}}{\lemma{\textnormal{\emph{Leitartikel}}}\Cendnote{\textnormal{\textcolor{blue}{f. s.} [ = \textcolor{blue}{Felix Salten}]: \emph{\textcolor{green}{Die
                        Schülerausstellung der Akademie}}. In: \emph{\textcolor{green}{Wiener Allgemeine Zeitung}}, Nr. 5.517, 21. 7. 1896, S. 4.}}}\label{K_L03175-5h} über die Zustände an der \textcolor{brown}{Akademie}{}\ledrightnote{\textcolor{brown}{Akademie der Bildenden Künste Wien}}, musste aber zahm sein, da man in kein
               Wespennest stechen will. Doch denke ich mich in der \label{K_L03175-6v}\edtext{\textcolor{green}{Frankft. Ztg}{}\ledrightnote{\textcolor{green}{Frankfurter Zeitung}} weitläufiger über die Sache
                  auszulaßen}{\lemma{\textnormal{\emph{Frankft. … auszulaßen}}}\Cendnote{\textnormal{kein Feuilleton
                  nachgewiesen}}}\label{K_L03175-6h}. Dass \label{K_L03175-7v}\edtext{\textcolor{blue}{Edmond de Goncourt}{}\ledrightnote{\textcolor{blue}{Edmond Huot de Goncourt}} tot}{\lemma{\textnormal{\emph{Edmond de Goncourt tot}}}\Cendnote{\textnormal{Der Schriftsteller \textcolor{blue}{Edmond
                     de Goncourt} war am 16. 7. 1896 in \textcolor{pink}{Draveil} verstorben.}}}\label{K_L03175-7h} ist, werden Sie
               vielleicht schon erfahren haben. Er starb in dem \textcolor{pink}{Schloße}{}\ledrightnote{{$\rightarrow$}\textcolor{pink}{Haus von Alphonse Daudet}} von \textcolor{blue}{Daudet}{}\ledrightnote{\textcolor{blue}{Alphonse Daudet}}.
               Die  \textcolor{pink}{Wien}{}\ledrightnote{\textcolor{pink}{Wien}}er Schornalisten, welche die letzte
                  \label{K_L03175-8v}\edtext{\textcolor{green}{Flegelei}{}\ledrightnote{{$\rightarrow$}\textcolor{green}{† Edmond de Goncourt}}}{\lemma{\textnormal{\emph{Flegelei}}}\Cendnote{\textnormal{[\textcolor{blue}{Max Nordau}]: \emph{\textcolor{green}{† Edmond de Goncourt}}. In: \emph{\textcolor{green}{Neue Freie Presse}}, Nr. 11.457, 17. 7. 1896, Morgenblatt, S. 5.}}}\label{K_L03175-8h}{ }\textcolor{blue}{Nordau}{}\ledrightnote{\textcolor{blue}{Max Nordau}}’s als Quelle über \textcolor{blue}{Goncourt}{}\ledrightnote{\textcolor{blue}{Edmond Huot de Goncourt}} benützten, schrieben in guten \label{K_L03175-9v}\edtext{Notizelach}{\lemma{\textnormal{\emph{Notizelach}}}\Cendnote{\textnormal{Durch Anhang einer jiddischen Endsilbe spielte \textcolor{blue}{Salten} darauf an, dass \textcolor{blue}{Nordau} Jude war und überhaupt die \textcolor{pink}{Wien}er Presselandschaft in Verruf stand, nur von Juden
                  bevölkert zu werden. Da \textcolor{blue}{Salten} selbst
                  jüdischer Abstammung war, dürfte damit weniger ein antisemitischer Reflex gemeint
                  gewesen sein, als eine als jüdisch wahrgenommene Berichterstattung das Ziel seiner
                  Kritik dargestellt haben.}}}\label{K_L03175-9h}, er sei der populärste und \uline{platteste}{ }{\pb}Schriftsteller \textcolor{pink}{Frankreich}{}\ledrightnote{\textcolor{pink}{Frankreich}}s gewesen. Herr \textcolor{blue}{Ohnet}{}\ledrightnote{\textcolor{blue}{Georges Ohnet}} würde sich freuen. Nach seinem Testament wird eine »\textcolor{brown}{freie Akademie}{}\ledrightnote{{$\rightarrow$}\textcolor{brown}{Académie Goncourt}}« gegründet, deren Präsident
                  \textcolor{blue}{Daudet}{}\ledrightnote{\textcolor{blue}{Alphonse Daudet}} ist, und deren einzelne Mitglieder
               eine Rente von 6000 Frcs aus dem Vermögen \textcolor{blue}{Goncourt}{}\ledrightnote{\textcolor{blue}{Jules Huot de Goncourt}{\newline}\textcolor{blue}{Edmond Huot de Goncourt}}s erhalten. Diese Lust der \textcolor{pink}{Fran}{}\ledrightnote{{$\rightarrow$}\textcolor{pink}{Frankreich}}zosen nach Vereinigungen und ihr Verlangen, dass die
               Berühmtheit durch Zeremonien bestätigt werde, hat etwas, wenn auch nicht viel von
               unserem »hohen Orden«, der freilich schöner ist. Schon deshalb weil er nicht
               exisitirt. Schreiben Sie bald und grüßen \textcolor{blue}{Richard}{}\ledrightnote{\textcolor{blue}{Richard Beer-Hofmann}}. Die Zeitungen schicke ich Ihnen nun schon nach \label{K_L03175-10v}\edtext{\textcolor{pink}{Kopenhagen}{}\ledrightnote{\textcolor{pink}{Kopenhagen}}}{\lemma{\textnormal{\emph{Kopenhagen}}}\Cendnote{\textnormal{\textcolor{blue}{Schnitzler} hielt sich von 2. 8. 1896 bis 3. 8. 1896 sowie am
                     22. 8. 1896 in
                     \textcolor{pink}{Kopenhagen} auf. Dazwischen war er in \textcolor{pink}{Skodsborg}. Vermutlich wurden ihm Briefe von \textcolor{pink}{Kopenhagen} aus nachgesandt.}}}\label{K_L03175-10h}.\pend
           
\pstart
           Herzlichst Ihr {\\[\baselineskip]}\spacefill\mbox{Salten}\pend
           \leftskip=0em{}\endnumbering\briefempfaengerindex{Schnitzler, Arthur@\textsc{Schnitzler, Arthur}!zzzSalten, Felix@\emph{von Felix Salten}!1896-07-211@{21. 7. 1896}|)be}\mylabel{h}  \normalsize

\doendnotes{C}
\bigskip
\vfill

\clearpage

\footnotesize

\lohead{\textsc{register}}

% Definiere theindex-Environment komplett neu ohne reledmac
\makeatletter
\renewenvironment{theindex}{%
  \section*{\indexname}%
  \setlength{\parindent}{0pt}%
  \setlength{\parskip}{0pt plus 0.3pt}%
  \let\item\@idxitem
}{%
  \clearpage
}
\makeatother

\IfFileExists{\jobname-pw.ind}{\input{\jobname-pw.ind}}{}

\end{document}

      