%% latex-korrekturansicht-vorspann.tex
%% Vorspann für die Korrekturansicht.
%% Lädt die gemeinsame Datei latex-vorspann.tex mit gesetztem Schalter.

\newif\ifkorrekturansicht
\korrekturansichttrue

\input{../tex-inputs/latex-vorspann}


\renewcommand{\erwaehntePersonen}{Personen: Lili Cappellini, Frieda Pollak, Felix Salten, Ottilie Salten, Heinrich Schnitzler}
\renewcommand{\erwaehnteOrte}{Orte: Böhmen, Olomouc, Sternwartestraße 71, Wien}
\renewcommand{\erwaehnteWerke}{Werke: Der Hund von Florenz, Tagebuch}
\section[ Felix Salten an Arthur Schnitzler, 18. 3. 1921]{Felix Salten an Arthur Schnitzler, 18. 3. 1921}
\nopagebreak\mylabel{v}
\rehead{ }\normalsize\beginnumbering\briefempfaengerindex{Schnitzler, Arthur@\textsc{Schnitzler, Arthur}!zzzSalten, Felix@\emph{von Felix Salten}!1921-03-181@{18. 3. 1921}|(be}
\toendnotes[C]{\smallbreak\pagebreak[2]}\Standort{CUL, Schnitzler, B 89, B 2.}
\physDesc{Postkarte, 821 Zeichen
\newline{}Handschrift: schwarze Tinte, lateinische Kurrent
\newline{}Versand: Stempel: »\nobreak{}\oindex{Olomouc@\textbf{Olomouc}, \emph{P.PPLA}|pwk}Olomouc 3 C.S.P., 19. III. 21, 9\nobreak{}«.  
\newline{}Schnitzler: mit Bleistift die Jahreszahl ergänzt: »21.« 
\newline{}Ordnung: 1) mit Bleistift von \textcolor{blue}{Frieda Pollak} (?) mit
                                 dem Buchstaben »A« (Abgeschrieben/Abschrift)
                                 gekennzeichnet  2) mit Bleistift von unbekannter Hand nummeriert: »282«}\toendnotes[C]{\smallbreak}\pstart{}{\pb}Herrn\pend{}\pstart{}D\textsuperscript{r} Arthur Schnitzler\pend{}\pstart{}\textcolor{pink}{Wien}{}\ledrightnote{\textcolor{pink}{Wien}}\pend{}\pstart{}\textcolor{pink}{XVIII. Sternwartestraße 71}{}\ledrightnote{\textcolor{pink}{Sternwartestraße 71}}\pend{}
{\bigskip}
\pstart
           \raggedleft{}{\pb}\textcolor{pink}{Olmütz}{}\ledrightnote{\textcolor{pink}{Olomouc}}, 18. 3.\pend
           
\pstart{}Lieber,\pend
\pstart
           hoffentlich \label{K_L03569-1v}\edtext{haben Sie von \textcolor{blue}{Otti}{}\ledrightnote{\textcolor{blue}{Ottilie Salten}} schon das Mscpt. meiner \textcolor{green}{Erzählung}{}\ledrightnote{{$\rightarrow$}\textcolor{green}{Der Hund von Florenz}}}{\lemma{\textnormal{\emph{haben … Erzählung}}}\Cendnote{\textnormal{\textcolor{blue}{Schnitzler}s \emph{\textcolor{green}{Tagebuch}} ist zu entnehmen, dass er das Manuskript von \emph{\textcolor{green}{Der Hund von Florenz}} erst am 23. 3. 1921 bekam.}}}\label{K_L03569-1h}. Wenn nicht,
               bitte, verlangen Sie’s. Ich hoffe sehr, dass Sie wohl und mehr und mehr ruhig sind
               und dass Ihnen das Arbeiten von der Hand geht! Und ich hoffe, dass Ihnen der Frühling
               so stark hilft, wie er kann. Das viele Umherfahren, das ich jetzt absolvieren muß,
               meist in Bu{\geminationm}el-Zügen, ist ja nicht angenehm, aber das
               Anschauen der milden, \textcolor{pink}{böhm}{}\ledrightnote{{$\rightarrow$}\textcolor{pink}{Böhmen}}ischen Landschaft, die jetzt, bei dem schönen Wetter, wie neu aussieht,
               beruhigt so angenehm. Auch ist das die vierte Stadt, in der ich seit Sonntag lese. Noch vier folgen. Es geht gut. Ich bin
               zwischendurch doch viel allein, was wohltut, denke viel und denke natürlich auch sehr
               viel an Sie!\pend
           
\pstart
           Alles Herzliche Ihnen und den \textcolor{blue}{Kinder}{}\ledrightnote{{$\rightarrow$}\textcolor{blue}{Heinrich Schnitzler}{\newline}{$\rightarrow$}\textcolor{blue}{Lili Cappellini}}n. {\\[\baselineskip]}Ihr {\\[\baselineskip]}\spacefill\mbox{Felix Salten}\pend
           \leftskip=0em{}\endnumbering\briefempfaengerindex{Schnitzler, Arthur@\textsc{Schnitzler, Arthur}!zzzSalten, Felix@\emph{von Felix Salten}!1921-03-181@{18. 3. 1921}|)be}\mylabel{h}  \normalsize

\doendnotes{C}
\bigskip
\vfill

\clearpage

\footnotesize

\lohead{\textsc{register}}

% Definiere theindex-Environment komplett neu ohne reledmac
\makeatletter
\renewenvironment{theindex}{%
  \section*{\indexname}%
  \setlength{\parindent}{0pt}%
  \setlength{\parskip}{0pt plus 0.3pt}%
  \let\item\@idxitem
}{%
  \clearpage
}
\makeatother

\IfFileExists{\jobname-pw.ind}{\input{\jobname-pw.ind}}{}

\end{document}

      