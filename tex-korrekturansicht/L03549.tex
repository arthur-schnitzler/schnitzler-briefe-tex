%% latex-korrekturansicht-vorspann.tex
%% Vorspann für die Korrekturansicht.
%% Lädt die gemeinsame Datei latex-vorspann.tex mit gesetztem Schalter.

\newif\ifkorrekturansicht
\korrekturansichttrue

\input{../tex-inputs/latex-vorspann}


\renewcommand{\erwaehntePersonen}{Personen: Felix Salten, Ottilie Salten}
\renewcommand{\erwaehnteOrte}{Orte: Berghof, Edmund-Weiß-Gasse 7, Sternwartestraße 71, Unterach am Attersee, Wien}
\renewcommand{\erwaehnteWerke}{Werke: Künstler sollen reden}
\section[ Felix Salten an Arthur Schnitzler, 28. 6. 1910]{Felix Salten an Arthur Schnitzler, 28. 6. 1910}
\nopagebreak\mylabel{v}
\rehead{ }\normalsize\beginnumbering\briefempfaengerindex{Schnitzler, Arthur@\textsc{Schnitzler, Arthur}!zzzSalten, Felix@\emph{von Felix Salten}!1910-06-282@{28. 6. 1910}|(be}
\toendnotes[C]{\smallbreak\pagebreak[2]}\Standort{CUL, Schnitzler, B 89, B 2.}
\physDesc{Postkarte, 468 Zeichen
\newline{}Handschrift: schwarze Tinte, lateinische Kurrent
\newline{}Versand: Stempel: »\nobreak{}\oindex{Unterach am Attersee@\textbf{Unterach am Attersee}, \emph{P.PPL}|pwk}Unterach am Attersee, 28/6 10, 5\nobreak{}«.  
\newline{}Ordnung: mit Bleistift von unbekannter Hand nummeriert: »264« }\toendnotes[C]{\smallbreak}\pstart{}{\pb}Salten.\pend{}\pstart{}\textcolor{pink}{Unterach a. Attersee}{}\ledrightnote{\textcolor{pink}{Unterach am Attersee}}. \textcolor{pink}{Berghof}{}\ledrightnote{\textcolor{pink}{Berghof}}.\pend{}
{\bigskip}\pstart{}Herrn\pend{}\pstart{}D\textsuperscript{r} Arthur Schnitzler\pend{}\pstart{}\textcolor{pink}{Wien}{}\ledrightnote{\textcolor{pink}{Wien}}\pend{}\pstart{}\textcolor{pink}{XVIII. Spöttelgaße 7}{}\ledrightnote{\textcolor{pink}{Edmund-Weiß-Gasse 7}}\pend{}
{\bigskip}
\pstart
           \raggedleft{}{\pb}28. VI. 10\pend
           
\pstart{}Lieber,\pend
\pstart
           vielen Dank! Ich freu mich, dass \textcolor{green}{es}{}\ledrightnote{{$\rightarrow$}\textcolor{green}{Künstler sollen reden}} Ihnen gefallen hat, und bin froh, dass diese \textcolor{green}{Sache}{}\ledrightnote{{$\rightarrow$}\textcolor{green}{Künstler sollen reden}} auch sonst – wie es scheint – \substVorne{}\textsuperscript{I}\substDazwischen{}i\substHinten{}hre Wirkung tut. Wir leben \textcolor{pink}{hier}{}\ledrightnote{{$\rightarrow$}\textcolor{pink}{Unterach am Attersee}} sehr angenehm, sehr still, und ich arbeite viel. Es regnet oft, aber
               das verdirbt uns, wenigstens bisher, den Aufenthalt nicht. Alles Schöne zur Arbeit am
                  \textcolor{pink}{Haus}{}\ledrightnote{{$\rightarrow$}\textcolor{pink}{Sternwartestraße 71}} und zum übrigen
               Arbeiten. Herzliche Grüße von \textcolor{blue}{uns}{}\ledrightnote{{$\rightarrow$}\textcolor{blue}{Ottilie Salten}} zu Ihnen.\pend
           
\pstart
           Ihr {\\[\baselineskip]}\spacefill\mbox{F. S.}\pend
           \leftskip=0em{}\endnumbering\briefempfaengerindex{Schnitzler, Arthur@\textsc{Schnitzler, Arthur}!zzzSalten, Felix@\emph{von Felix Salten}!1910-06-282@{28. 6. 1910}|)be}\mylabel{h}  \normalsize

\doendnotes{C}
\bigskip
\vfill

\clearpage

\footnotesize

\lohead{\textsc{register}}

% Definiere theindex-Environment komplett neu ohne reledmac
\makeatletter
\renewenvironment{theindex}{%
  \section*{\indexname}%
  \setlength{\parindent}{0pt}%
  \setlength{\parskip}{0pt plus 0.3pt}%
  \let\item\@idxitem
}{%
  \clearpage
}
\makeatother

\IfFileExists{\jobname-pw.ind}{\input{\jobname-pw.ind}}{}

\end{document}

      