%% latex-korrekturansicht-vorspann.tex
%% Vorspann für die Korrekturansicht.
%% Lädt die gemeinsame Datei latex-vorspann.tex mit gesetztem Schalter.

\newif\ifkorrekturansicht
\korrekturansichttrue

\input{../tex-inputs/latex-vorspann}


               \section[Hugo Hofmannsthal an Arthur Schnitzler, 30. 7. 1920]{ Hugo Hofmannsthal an Arthur Schnitzler, 30. 7. 1920}\nopagebreak\mylabel{v}\rehead{ }\normalsize\beginnumbering\briefempfaengerindex{Schnitzler, Arthur@\textsc{Schnitzler, Arthur}!zzzHofmannsthal, Hugo von@\emph{von Hugo von Hofmannsthal}!1920-07-301@{30. 7. 1920}|(be} \toendnotes[C]{\smallbreak\pagebreak[2]} \Standort{CUL, Schnitzler, B 43.}
\physDesc{Postkarte
\newline{}Handschrift: schwarze Tinte, deutsche Kurrent\newline{}Versand: Stempel: »\nobreak{}\oindex{Salzburg@\textbf{Salzburg}, \emph{Besiedelter Ort (A.BSO)}|pwk}Salzburg 2, 31. VII. \textcolor{gray}{20}\nobreak{}«.  \newline{}Ordnung: 1) mit Bleistift von \textcolor{blue}{Frieda Pollak} (?) mit dem Buchstaben »A« (Abgeschrieben/Abschrift) gekennzeichnet 2) mit Bleistift von unbekannter Hand nummeriert: »\strikeout{387}«3) mit Bleistift von unbekannter Hand nummeriert: »369«}\buchAbdrucke{\weitereDrucke{Hugo von Hofmannsthal, Arthur Schnitzler: \emph{Briefwechsel}. Hg. Therese Nickl und Heinrich Schnitzler. Frankfurt am Main: \emph{S. Fischer} 1964, S. 294.} }\toendnotes[C]{\smallbreak}\pstart{}{\pb}\textsc{Herrn D\textsuperscript{r} Arthur Schnitzler}\pend{}\pstart{}\textsc{\textcolor{pink}{Wien}{}\ledrightnote{\textcolor{pink}{Wien}}}\pend{}\pstart{}\textsc{\textcolor{pink}{XVIII Sternwartestrasse 71}{}\ledrightnote{\textcolor{pink}{Sternwartestraße}}}\pend{}{\bigskip}\pstart
           \raggedleft{}{\pb}\textcolor{pink}{Salzburg}{}\ledrightnote{\textcolor{pink}{Salzburg}}{ }30 VII 20.\pend
           \pstart{}mein lieber Arthur\pend\pstart
           \label{K_L02352_1v}\edtext{hier}{\lemma{\textnormal{\emph{hier}}}\Cendnote{\textnormal{Er war anlässlich
                  der 1. \emph{\textcolor{brown}{Festspiele}} in \textcolor{pink}{Salzburg}.}}}\label{K_L02352_1h} kann ich nie ſein, ohne Ihrer und ſchöner weit
               entſchwundener Begegnungen, leichter und tiefer Geſpräche und {\pb}unſerer Lebensfreundſchaft mit dem
               undefinierbaren Gefühl, das man mit »Wehmut« oft aber nicht richtig benennt, zu
               gedenken.\pend
           \pstart
           Ihr Rat war, wie immer, ſehr gut; \textcolor{blue}{Heine}{}\ledrightnote{\textcolor{blue}{Albert Heine}} hat das
                  \label{K_L02352_2v}\edtext{\textcolor{green}{Stück}{}\ledrightnote{→\textcolor{green}{Der Schwierige. Lustspiel in drei Akten}}}{\lemma{\textnormal{\emph{Stück}}}\Cendnote{\textnormal{Zu der
                  angedachten Inszenierung von \emph{\textcolor{green}{Der Schwierige}} kam
                  es nicht. Stattdessen erlebte dies am 7. 11. 1921 am \emph{\textcolor{brown}{Münchner Residenztheater}} seine
                  Uraufführung.}}}\label{K_L02352_2h}, als ich es ihm anbot, ohne weiteres angeno{\geminationm}en, er will es als erſte Frühjahrsnovität in \textcolor{pink}{Schönbrunn}{}\ledrightnote{\textcolor{pink}{Schlosstheater Schönbrunn}} ſpielen.\pend
           \pstart
           Von Herzen Ihr{\\[\baselineskip]}\spacefill\mbox{Hugo.}\pend
           \leftskip=0em{}\endnumbering\briefempfaengerindex{Schnitzler, Arthur@\textsc{Schnitzler, Arthur}!zzzHofmannsthal, Hugo von@\emph{von Hugo von Hofmannsthal}!1920-07-301@{30. 7. 1920}|)be}\mylabel{h}  \normalsize

\doendnotes{C}
\bigskip
\vfill

\clearpage

\footnotesize

\lohead{\textsc{register}}

% Definiere theindex-Environment komplett neu ohne reledmac
\makeatletter
\renewenvironment{theindex}{%
  \section*{\indexname}%
  \setlength{\parindent}{0pt}%
  \setlength{\parskip}{0pt plus 0.3pt}%
  \let\item\@idxitem
}{%
  \clearpage
}
\makeatother

\IfFileExists{\jobname-pw.ind}{\input{\jobname-pw.ind}}{}

\end{document}

      