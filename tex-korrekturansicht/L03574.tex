%% latex-korrekturansicht-vorspann.tex
%% Vorspann für die Korrekturansicht.
%% Lädt die gemeinsame Datei latex-vorspann.tex mit gesetztem Schalter.

\newif\ifkorrekturansicht
\korrekturansichttrue

\input{../tex-inputs/latex-vorspann}


\renewcommand{\erwaehntePersonen}{Personen: Frieda Pollak, Felix Salten, Ottilie Salten}
\renewcommand{\erwaehnteOrte}{Orte: Altaussee, Berchtesgaden, Berghof, München, Salzburg, Salzkammergut, Seewirt, Unterach am Attersee}
\renewcommand{\erwaehnteWerke}{}
\section[ Felix und Ottilie Salten an Arthur Schnitzler, 17. {[}8.?{]} 1921]{Felix und Ottilie Salten an Arthur Schnitzler, 17. {[}8.?{]} 1921}
\nopagebreak\mylabel{v}
\rehead{ }\normalsize\beginnumbering\briefempfaengerindex{Schnitzler, Arthur@\textsc{Schnitzler, Arthur}!zzzSalten, Ottilie@\emph{von Ottilie Salten}!1921-08-171@{17. {[}8.?{]} 1921}|(be}\briefempfaengerindex{Schnitzler, Arthur@\textsc{Schnitzler, Arthur}!zzzSalten, Felix@\emph{von Felix Salten}!1921-08-171@{17. {[}8.?{]} 1921}|(be}
\toendnotes[C]{\smallbreak\pagebreak[2]}\Standort{CUL, Schnitzler, B 89, B 2.}
\physDesc{Bildpostkarte, 377 Zeichen
\newline{}Handschrift Felix Salten: schwarze Tinte, lateinische Kurrent
\newline{}Handschrift Ottilie Salten: schwarze Tinte, deutsche Kurrent
\newline{}Versand: Stempel: »\nobreak{}\oindex{Unterach am Attersee@\textbf{Unterach am Attersee}, \emph{P.PPL}|pwk}\textcolor{gray}{Un}terach am Atter\textcolor{gray}{see}\nobreak{}«.  
\newline{}Ordnung: 1) mit Bleistift von \textcolor{blue}{Frieda Pollak} (?) mit
                                 dem Buchstaben »A« (Abgeschrieben/Abschrift)
                                 gekennzeichnet  2) mit Bleistift von unbekannter Hand nummeriert: »287«}\toendnotes[C]{\smallbreak}\pstart{}{\pb}Herrn\pend{}\pstart{}D\textsuperscript{r} Arthur Schnitzler\pend{}\pstart{}\textcolor{pink}{Alt-Aussee}{}\ledrightnote{\textcolor{pink}{Altaussee}}\pend{}\pstart{}\textcolor{pink}{Seewirt}{}\ledrightnote{\textcolor{pink}{Seewirt}}\pend{}
{\bigskip}
\pstart
           \noindent{}\centering{}{\pb}\textcolor{gray}{\textbf{\textcolor{pink}{Salzkammergut}{}\ledrightnote{\textcolor{pink}{Salzkammergut}}. \textbf{\textcolor{pink}{Unterach am Attersee}{}\ledrightnote{\textcolor{pink}{Unterach am Attersee}}.}}}\pend
           
\pstart
           \raggedleft{}{\pb}\textcolor{pink}{Berghof}{}\ledrightnote{\textcolor{pink}{Berghof}}, \label{K_L03574-1v}\edtext{17. \textcolor{gray}{8}. 21}{\lemma{\textnormal{\emph{17. 8. 21}}}\Cendnote{\textnormal{Die Monatsziffer ist nicht eindeutig
                     lesbar, auch ›9‹ wäre möglich. Durch die Adressierung nach \textcolor{pink}{Altaussee} und den Inhalt kann der September 1921 jedoch ausgeschlossen werden.}}}\label{K_L03574-1h}\pend
           
\pstart{}Lieber,\pend
\pstart
           werden Sie also \label{K_L03574-2v}\edtext{auf Ihrem Weg nach \textcolor{pink}{München}{}\ledrightnote{\textcolor{pink}{München}} an uns vorüber-kommen oder vorbei
                  gehen}{\lemma{\textnormal{\emph{auf … gehen}}}\Cendnote{\textnormal{\textcolor{blue}{Schnitzler} reiste über \textcolor{pink}{Salzburg} und \textcolor{pink}{Berchtesgaden} nach \textcolor{pink}{München}, wo er
                  am 28. 8. 1921
                  ankam.}}}\label{K_L03574-2h}? Wir würden uns so \uline{sehr} freuen, wenn
               Sie kämen und zwei, drei, vier Tage blieben. Je länger, je besser! Es ist sehr still
               und einsam hier!\pend
           
\pstart
           Alles Herzliche von uns allen {\\[\baselineskip]}Ihr {\\[\baselineskip]}\spacefill\mbox{F. S.}\pend
           \leftskip=0em{}
\pstart
           \noindent{}{[}hs. Ottilie Salten:{]} Wie ſchön wäre es, wenn Sie kämen! Herzlichſt
                  \spacefill\mbox{Ottilie Salten}\pend
           \endnumbering\briefempfaengerindex{Schnitzler, Arthur@\textsc{Schnitzler, Arthur}!zzzSalten, Ottilie@\emph{von Ottilie Salten}!1921-08-171@{17. {[}8.?{]} 1921}|)be}\briefempfaengerindex{Schnitzler, Arthur@\textsc{Schnitzler, Arthur}!zzzSalten, Felix@\emph{von Felix Salten}!1921-08-171@{17. {[}8.?{]} 1921}|)be}\mylabel{h}  \normalsize

\doendnotes{C}
\bigskip
\vfill

\clearpage

\footnotesize

\lohead{\textsc{register}}

% Definiere theindex-Environment komplett neu ohne reledmac
\makeatletter
\renewenvironment{theindex}{%
  \section*{\indexname}%
  \setlength{\parindent}{0pt}%
  \setlength{\parskip}{0pt plus 0.3pt}%
  \let\item\@idxitem
}{%
  \clearpage
}
\makeatother

\IfFileExists{\jobname-pw.ind}{\input{\jobname-pw.ind}}{}

\end{document}

      