%% latex-korrekturansicht-vorspann.tex
%% Vorspann für die Korrekturansicht.
%% Lädt die gemeinsame Datei latex-vorspann.tex mit gesetztem Schalter.

\newif\ifkorrekturansicht
\korrekturansichttrue

\input{../tex-inputs/latex-vorspann}


               \section[Arthur Schnitzler an Gertrud Rung, 2. 6. 1925]{ Arthur Schnitzler an Gertrud Rung, 2. 6. 1925}\nopagebreak\mylabel{v}\rehead{ }\normalsize\beginnumbering\briefempfaengerindex{Hauschner, Auguste@\textsc{Hauschner, Auguste}!zzzSchnitzler, Arthur@\emph{von Arthur Schnitzler}!1925-06-021@{2. 6. 1925}|(be} \toendnotes[C]{\smallbreak\pagebreak[2]} \Standort{Kopenhagen, Det Kongelige Bibliotek, NKS 3756 4°.}
\physDesc{Postkarte
\newline{}Handschrift: schwarze Tinte, lateinische Kurrent\newline{}Versand: Stempel: »\nobreak{}\oindex{XVIII., Waehring@\textbf{XVIII., Währing}, \emph{Bezirk (A.BZK)}|pwk}18/\textsubscript{1} Wien  \textcolor{gray}{110}, 2. 6. 2\textcolor{gray}{5}, 19\nobreak{}«.  \newline{}Ordnung: mit Bleistift von unbekannter Hand vermerkt: »\textsc{Schnitzler}« }\buchAbdrucke{\weitereDrucke{Arthur Schnitzler: \emph{Arthur Schnitzlers Briefe nach Dänemark}. Hg. Ernst-Ulrich Pinkert. Roskilde: \emph{Center for Østrigsk-Nordiske Kulturstudier} 2006, S. 20.} }\toendnotes[C]{\smallbreak}\pstart{}{\pb}\label{T_L02601-1v}\edtext{\textcolor{gray}{\textbf{A. S.}}}{\lemma{\textnormal{\emph{A. S.}}}\Cendnote{\textnormal{ovaler Absenderkleber}}}\label{T_L02601-1h}\pend{}\pstart{}\textcolor{pink}{\textcolor{gray}{\textbf{WIEN, XVIII.}}}{}\ledrightnote{\textcolor{pink}{XVIII., Währing}}\pend{}\pstart{}\textcolor{pink}{\textcolor{gray}{\textbf{STERNWARTESTR. 71}}}{}\ledrightnote{\textcolor{pink}{Sternwartestraße}}\pend{}{\bigskip}\pstart{}{\pb}Frau \pend{}\pstart{}Gertrud Rung,\pend{}\pstart{}\textcolor{pink}{Oesterr. Hof –}{}\ledrightnote{\textcolor{pink}{Österreichischer Hof}}\pend{}\pstart{}\textcolor{pink}{Salzburg.}{}\ledrightnote{\textcolor{pink}{Salzburg}}\pend{}{\bigskip}\pstart
           \raggedleft{}{\pb}\textcolor{pink}{Wien}{}\ledrightnote{\textcolor{pink}{Wien}},
                        2. 6. 2\textcolor{gray}{5}\pend
           \pstart
           Verehrte Frau Rung, danke sehr für Ihre lieben und
               erfreulichen Nachrichten! Wie lange sind Sie noch in \textcolor{pink}{Salzburg}{}\ledrightnote{\textcolor{pink}{Salzburg}}? Ich ko{\geminationm}e vielleicht mit der Rückreise
               aus \label{K_L02601-1v}\edtext{\textcolor{pink}{Südtirol}{}\ledrightnote{\textcolor{pink}{Südtirol}} (wohin ich etwa am 17. d.
               abreise) gegen Ende Juni nach \textcolor{pink}{Salzburg}{}\ledrightnote{\textcolor{pink}{Salzburg}}}{\lemma{\textnormal{\emph{Südtirol … Salzburg}}}\Cendnote{\textnormal{\textcolor{blue}{Schnitzler} war von 23. 6. 1925 bis 3. 7. 1925 in \textcolor{pink}{Südtirol} und reiste ohne Unterbrechung nach \textcolor{pink}{Wien} durch.}}}\label{K_L02601-1h} – treff ich Sie und \textcolor{blue}{Brandes}{}\ledrightnote{\textcolor{blue}{Georg Brandes}} noch an – ? Grüßen Sie den von mir verehrten u geliebten Freund
               viele Male. Alles herzliche Ihnen.\pend
           \pstart
           {\pb}Auf Wiedersehen
               {\\[\baselineskip]}Ihr \spacefill\mbox{Arthur Schnitzler}\pend
           \leftskip=0em{}\endnumbering\briefempfaengerindex{Hauschner, Auguste@\textsc{Hauschner, Auguste}!zzzSchnitzler, Arthur@\emph{von Arthur Schnitzler}!1925-06-021@{2. 6. 1925}|)be}\mylabel{h}  \normalsize

\doendnotes{C}
\bigskip
\vfill

\clearpage

\footnotesize

\lohead{\textsc{register}}

% Definiere theindex-Environment komplett neu ohne reledmac
\makeatletter
\renewenvironment{theindex}{%
  \section*{\indexname}%
  \setlength{\parindent}{0pt}%
  \setlength{\parskip}{0pt plus 0.3pt}%
  \let\item\@idxitem
}{%
  \clearpage
}
\makeatother

\IfFileExists{\jobname-pw.ind}{\input{\jobname-pw.ind}}{}

\end{document}

      