%% latex-korrekturansicht-vorspann.tex
%% Vorspann für die Korrekturansicht.
%% Lädt die gemeinsame Datei latex-vorspann.tex mit gesetztem Schalter.

\newif\ifkorrekturansicht
\korrekturansichttrue

\input{../tex-inputs/latex-vorspann}


               \section[Max Burckhard an Arthur Schnitzler, 23. 11. 1907]{ Max Burckhard an Arthur Schnitzler, 23. 11. 1907}\nopagebreak\mylabel{v}\rehead{ }\normalsize\beginnumbering\briefempfaengerindex{Schnitzler, Arthur@\textsc{Schnitzler, Arthur}!zzzBurckhard, Max Eugen@\emph{von Max Eugen Burckhard}!1907-11-231@{23. 11. 1907}|(be} \toendnotes[C]{\smallbreak\pagebreak[2]} \Standort{CUL, Schnitzler, B 20.}
\physDesc{Brief, 1 Blatt, 2 Seiten
\newline{}Handschrift: schwarze Tinte, deutsche Kurrent\newline{}Ordnung: mit Bleistift von unbekannter Hand nummeriert: »19« }\toendnotes[C]{\smallbreak}\pstart
           \noindent{}{\pb}\textcolor{gray}{\textbf{D\textsuperscript{r.} Max Burckhard}}\hfill \textcolor{gray}{\textbf{\strikeout{\textcolor{pink}{Wien, IX. Porzellangasse 48}{}\ledrightnote{\textcolor{pink}{Porzellangasse}}}{ }..........}}\pend
           \pstart
           \raggedleft{}\textcolor{gray}{\textbf{\textcolor{pink}{St. Gilgen}{}\ledrightnote{\textcolor{pink}{St. Gilgen}}}}{ }23. XI. 07\pend
           \pstart{}Sehr verehrter lieber Herr Doctor!\pend\pstart
           Ich danke Ihnen herzlichſt für Ihren lieben Brief und Ihre freundſchaftliche
                    Gesinnung und Antheilnahme, und auch Ihrer verehrten Frau \textcolor{blue}{Gemahlin}{}\ledrightnote{→\textcolor{blue}{Olga Schnitzler}} danke ich herzlichſt. Ich kann
                    Ihnen erfreulicherweiſe melden, daſs es mir ſo ziemlich gut geht und daſs ich
                    hoffe, bald in \textcolor{pink}{Wien}{}\ledrightnote{\textcolor{pink}{Wien}} auftauchen zu können, und
                    daſs ich dann gewiſs recht bald in der \textcolor{pink}{Spöttelgasse 7}{}\ledrightnote{\textcolor{pink}{Edmund-Weiß-Gasse}}{ }{\pb}erscheine.\pend
           \pstart
           Bis dahin bleibe ich mit beſten Empfehlungen und Handkuſs an Frau \textcolor{blue}{Olga}{}\ledrightnote{\textcolor{blue}{Olga Schnitzler}} und mit den allerherzlichſten Grüßen Ihr
                    getreuer\pend
           \pstart \spacefill\mbox{D\textsuperscript{r}Burckhard}\pend{}\endnumbering\briefempfaengerindex{Schnitzler, Arthur@\textsc{Schnitzler, Arthur}!zzzBurckhard, Max Eugen@\emph{von Max Eugen Burckhard}!1907-11-231@{23. 11. 1907}|)be}\mylabel{h}  \normalsize

\doendnotes{C}
\bigskip
\vfill

\clearpage

\footnotesize

\lohead{\textsc{register}}

% Definiere theindex-Environment komplett neu ohne reledmac
\makeatletter
\renewenvironment{theindex}{%
  \section*{\indexname}%
  \setlength{\parindent}{0pt}%
  \setlength{\parskip}{0pt plus 0.3pt}%
  \let\item\@idxitem
}{%
  \clearpage
}
\makeatother

\IfFileExists{\jobname-pw.ind}{\input{\jobname-pw.ind}}{}

\end{document}

      