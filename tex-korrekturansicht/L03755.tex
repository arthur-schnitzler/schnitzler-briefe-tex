%% latex-korrekturansicht-vorspann.tex
%% Vorspann für die Korrekturansicht.
%% Lädt die gemeinsame Datei latex-vorspann.tex mit gesetztem Schalter.

\newif\ifkorrekturansicht
\korrekturansichttrue

\input{../tex-inputs/latex-vorspann}


\section[Arthur Schnitzler an Stefan Zweig, 3. 1. 1921]{L03755 Arthur Schnitzler an Stefan Zweig, 3. 1. 1921}
\nopagebreak\mylabel{L03755v}
\rehead{ }\normalsize\beginnumbering\briefempfaengerindex{, @\textsc{, }!zzz, @\emph{von  }!1921-01-031@{3. 1. 1921}|(be}
\toendnotes[C]{\smallbreak\pagebreak[2]}\Standort{Jerusalem, National Library of Israel, ARC. Ms. Var. 305 1 58 Stefan Zweig Collection.}
\physDesc{Bildpostkarte, 281 Zeichen
\newline{}Handschrift: schwarze Tinte, lateinische Kurrent
\newline{}Versand: Stempel: »\nobreak{}5. I. \textcolor{gray}{21}, XI\nobreak{}«.  
\newline{}Zweig: mit rotem Buntstift Vermerk: »\textsc{Schnitzler}« }\toendnotes[C]{\smallbreak}\pstart{}{\pb}Hrn Dr. Stefan Zweig\pend{}\pstart{}\textcolor{pink}{Salzburg}\oindex{Salzburg@\textbf{Salzburg}, \emph{Verwaltungsgebiet}|pw}{}\ledrightnote{\textcolor{pink}{Salzburg}}\pend{}\pstart{}\textcolor{pink}{Kapuzinerberg}\oindex{Kapuzinerberg@\textbf{Kapuzinerberg}, \emph{Berg}|pw}{}\ledrightnote{\textcolor{pink}{Kapuzinerberg}}\pend{}{\bigskip}
\pstart
           \noindent{}\centering{}{\pb}\textcolor{gray}{\textbf{\textcolor{pink}{Wien, XVIII, Sternwartestr. 71}\oindex{Wien@\textbf{Wien}!XVIII., Währing@\textbf{XVIII., Währing}!Sternwartestraße 71@\textbf{Sternwartestraße 71}, \emph{Wohngebäude}|pw}{}\ledrightnote{\textcolor{pink}{Sternwartestraße 71}}}}\pend
           \vspace{1em}
\pstart
           \raggedleft{}{\pb}3. 1. 921\pend
           
\pstart{}lieber und verehrter Herr Doctor,\pend\vspace{0.5em}
\pstart
           eben in der Lecture Ihres schönen \textcolor{green}{\textcolor{blue}{Rolland}\pwindex{Rolland, Romain 29.\,1.\,1866 Clamecy – 30.\,12.\,1944 Vézelay@\textsc{Rolland, Romain} (29.\,1.\,1866 Clamecy – 30.\,12.\,1944 Vézelay), \emph{Schriftsteller}|pw}{}\ledrightnote{\textcolor{blue}{Romain Rolland}}}\pwindex{Zweig, Stefan 28.\,11.\,1881 Wien – 23.\,2.\,1942 Petrópolis@\textsc{Zweig, Stefan} (28.\,11.\,1881 Wien – 23.\,2.\,1942 Petrópolis), \emph{Schriftsteller}!Romain Rolland. Der Mann und das Werk.@\strich\emph{Romain Rolland. Der Mann und das Werk.}|pwv}{}\ledrightnote{{$\rightarrow$}\emph{\textcolor{green}{Romain Rolland. Der Mann und das Werk.}}} Buches begriffen, für dessen freundliche Übersendg ich
               Ihnen herzlichst danke, sende ich Ihnen die besten Grüße u\textcolor{gray}{nd}
               Wünsche zum neuen Jahr. \pend
           \pstart Ihr 
               \spacefill\mbox{Arthur Schnitzler}\pend{}\selectlanguage{ngerman}\endnumbering\briefempfaengerindex{, @\textsc{, }!zzz, @\emph{von  }!1921-01-031@{3. 1. 1921}|)be}\mylabel{L03755h}
\begin{anhang}
\end{anhang}\normalsize

\doendnotes{C}
\bigskip
\vfill

\clearpage

\footnotesize

\lohead{\textsc{register}}

% Definiere theindex-Environment komplett neu ohne reledmac
\makeatletter
\renewenvironment{theindex}{%
  \section*{\indexname}%
  \setlength{\parindent}{0pt}%
  \setlength{\parskip}{0pt plus 0.3pt}%
  \let\item\@idxitem
}{%
  \clearpage
}
\makeatother

\IfFileExists{\jobname-pw.ind}{\input{\jobname-pw.ind}}{}

\end{document}

      