%% latex-korrekturansicht-vorspann.tex
%% Vorspann für die Korrekturansicht.
%% Lädt die gemeinsame Datei latex-vorspann.tex mit gesetztem Schalter.

\newif\ifkorrekturansicht
\korrekturansichttrue

\input{../tex-inputs/latex-vorspann}


               \section[Arthur Schnitzler: Widmungsexemplar Buch der Sprüche und Bedenken für Hugo von Hofmannsthal, 16. 12. 1927]{ Arthur Schnitzler: Widmungsexemplar Buch der Sprüche und Bedenken für
               Hugo von Hofmannsthal, 16. 12. 1927}\nopagebreak\mylabel{v}\rehead{ }\normalsize\beginnumbering\briefempfaengerindex{Hofmannsthal, Hugo von@\textsc{Hofmannsthal, Hugo von}!zzzSchnitzler, Arthur@\emph{von Arthur Schnitzler}!1927-12-161@{16. 12. 1927}|(be} \toendnotes[C]{\smallbreak\pagebreak[2]} \Standort{FDH, FDH 3221.}
\physDesc{Widmung am Vorsatzblatt
\newline{}Handschrift: schwarze Tinte, lateinische Kurrent}\buchAbdrucke{\weitereDrucke{Hugo von Hofmannsthal: \emph{Bibliothek}. Hg. Ellen Ritter † in Zusammenarbeit mit Dalia Bukauskaité und
                        Konrad Heumann. Frankfurt am Main: \emph{S. Fischer} 2011, S. 603 (Sämtliche Werke. Kritische Ausgabe, XL).} }\pstart
           \noindent{}{\pb}Meinem lieben Hugo
                  Hofma{\geminationn}sthal\pend
           \pstart
           herzlichſt{\\[\baselineskip]}\spacefill\mbox{ArthSch}\pend
           \leftskip=0em{}\pstart
           \textcolor{pink}{Wien}{}\ledrightnote{\textcolor{pink}{Wien}}{ }16/12 927\pend
           {\bigskip}\pstart
           \noindent{}\centering{}{\pb}\textcolor{gray}{\textbf{ARTHUR SCHNITZLER}}\pend
           \pstart
           \noindent{}\centering{}\textcolor{gray}{\textbf{\textcolor{green}{BUCH DER SPRÜCHE UND BEDENKEN}{}\ledrightnote{\textcolor{green}{Buch der Sprüche und Bedenken}}}}\pend
           \pstart
           \noindent{}\centering{}\textcolor{gray}{\textbf{APHORISMEN UND FRAGMENTE}}\pend
           {\bigskip}\pstart
           \noindent{}\centering{}\textcolor{gray}{\textbf{IM \textcolor{brown}{PHAIDON-VERLAG}{}\ledrightnote{\textcolor{brown}{Phaidon-Verlag}} ⋅ \textcolor{pink}{WIEN}{}\ledrightnote{\textcolor{pink}{Wien}} ⋅ MCMXXVII}}\pend
           \endnumbering\briefempfaengerindex{Hofmannsthal, Hugo von@\textsc{Hofmannsthal, Hugo von}!zzzSchnitzler, Arthur@\emph{von Arthur Schnitzler}!1927-12-161@{16. 12. 1927}|)be}\mylabel{h}  \normalsize

\doendnotes{C}
\bigskip
\vfill

\clearpage

\footnotesize

\lohead{\textsc{register}}

% Definiere theindex-Environment komplett neu ohne reledmac
\makeatletter
\renewenvironment{theindex}{%
  \section*{\indexname}%
  \setlength{\parindent}{0pt}%
  \setlength{\parskip}{0pt plus 0.3pt}%
  \let\item\@idxitem
}{%
  \clearpage
}
\makeatother

\IfFileExists{\jobname-pw.ind}{\input{\jobname-pw.ind}}{}

\end{document}

      