%% latex-korrekturansicht-vorspann.tex
%% Vorspann für die Korrekturansicht.
%% Lädt die gemeinsame Datei latex-vorspann.tex mit gesetztem Schalter.

\newif\ifkorrekturansicht
\korrekturansichttrue

\input{../tex-inputs/latex-vorspann}


               \section[Paul Goldmann an Arthur Schnitzler, {[}10. 10. 1895?{]}]{ Paul Goldmann an Arthur Schnitzler, {[}10. 10. 1895?{]}}\nopagebreak\mylabel{v}\rehead{ }\normalsize\beginnumbering\briefempfaengerindex{Schnitzler, Arthur@\textsc{Schnitzler, Arthur}!zzzGoldmann, Paul@\emph{von Paul Goldmann}!1895-10-103@{{[}10. 10. 1895?{]}}|(be} \toendnotes[C]{\smallbreak\pagebreak[2]} \Standort{DLA, A:Schnitzler, HS.NZ85.1.3165.}
\physDesc{Telegramm
\newline{}maschinell
\newline{}Schnitzler: mit Bleistift datiert: »Oct 9\textcolor{gray}{5}« \newline{}Ordnung: beschnitten }\toendnotes[C]{\smallbreak}\pstart
           \centering{}{\pb}\damage{\textcolor{gray}{\textcolor{pink}{paris}{}\ledrightnote{\textcolor{pink}{Paris}}}} 45789 \damage{\textcolor{gray}{10 10 113}}\pend
           \pstart
           ob der \label{K_L02693-1v}\edtext{erfolg}{\lemma{\textnormal{\emph{erfolg}}}\Cendnote{\textnormal{Am Vortag, dem 9. 10. 1895, hatte die Uraufführung der \emph{\textcolor{green}{Liebelei}} am \emph{\textcolor{brown}{Burgtheater}}
                  stattgefunden.}}}\label{K_L02693-1h} nachhaelt ist einstweilen \label{T_L02693-1v}\edtext{gleichgiltig}{\lemma{\textnormal{\emph{gleichgiltig}}}\Cendnote{\textnormal{im
                  Original steht: »gleichgittig«}}}\label{T_L02693-1h} wichtig war nur der gestrige{ }abend er ist gut verlaufen folglich ist das \textcolor{green}{werk}{}\ledrightnote{→\textcolor{green}{Liebelei. Schauspiel in drei Akten}} gelungen\pend
           \pstart
           ich danke dir fuer die frohe nachricht und \label{T_L02693-2v}\edtext{beglueckwuensche}{\lemma{\textnormal{\emph{beglueckwuensche}}}\Cendnote{\textnormal{im Original steht: »begluekwensche«}}}\label{T_L02693-2h} dich von ganzem herzen
               es musste so kommen aber es ist doch schoen dass es so kam \pend
           \pstart gruesse = \spacefill\mbox{goldmann}\pend{}\endnumbering\briefempfaengerindex{Schnitzler, Arthur@\textsc{Schnitzler, Arthur}!zzzGoldmann, Paul@\emph{von Paul Goldmann}!1895-10-103@{{[}10. 10. 1895?{]}}|)be}\mylabel{h}\begin{anhang}\end{anhang}\normalsize

\doendnotes{C}
\bigskip
\vfill

\clearpage

\footnotesize

\lohead{\textsc{register}}

% Definiere theindex-Environment komplett neu ohne reledmac
\makeatletter
\renewenvironment{theindex}{%
  \section*{\indexname}%
  \setlength{\parindent}{0pt}%
  \setlength{\parskip}{0pt plus 0.3pt}%
  \let\item\@idxitem
}{%
  \clearpage
}
\makeatother

\IfFileExists{\jobname-pw.ind}{\input{\jobname-pw.ind}}{}

\end{document}

      