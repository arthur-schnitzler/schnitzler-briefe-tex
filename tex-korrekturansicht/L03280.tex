%% latex-korrekturansicht-vorspann.tex
%% Vorspann für die Korrekturansicht.
%% Lädt die gemeinsame Datei latex-vorspann.tex mit gesetztem Schalter.

\newif\ifkorrekturansicht
\korrekturansichttrue

\input{../tex-inputs/latex-vorspann}


\renewcommand{\erwaehnteOrte}{Orte: Bad Reichenhall, Karlsbad, Pressbaum, Salzburg, Wattmanngasse, Wien, Österreich}
\renewcommand{\erwaehnteWerke}{}
\section[ Felix Salten an Arthur Schnitzler, 30. 7. 1898]{Felix Salten an Arthur Schnitzler, 30. 7. 1898}
\nopagebreak\mylabel{v}
\rehead{ }\normalsize\beginnumbering\briefempfaengerindex{Schnitzler, Arthur@\textsc{Schnitzler, Arthur}!zzzSalten, Felix@\emph{von Felix Salten}!1898-07-301@{30. 7. 1898}|(be}
\toendnotes[C]{\smallbreak\pagebreak[2]}\Standort{CUL, Schnitzler, B 89, A 2.}
\physDesc{Brief, 1 Blatt, 2 Seiten, 543 Zeichen
\newline{}Handschrift: schwarze Tinte, lateinische Kurrent
\newline{}Ordnung: mit Bleistift von unbekannter Hand nummeriert: »104« }\toendnotes[C]{\smallbreak}
\pstart
           \raggedleft{}{\pb}\textcolor{pink}{Wien}{}\ledrightnote{\textcolor{pink}{Wien}}, 30. Juli 98.\pend
           
\pstart
           Lieber Arthur, bis heute war ich nicht in \textcolor{pink}{Wien}{}\ledrightnote{\textcolor{pink}{Wien}}. Meine
               Arbeit habe ich in \textcolor{pink}{Pressbaum}{}\ledrightnote{\textcolor{pink}{Pressbaum}} fertig gemacht,
               dann bin ich in \textcolor{pink}{Karlsbad}{}\ledrightnote{\textcolor{pink}{Karlsbad}} gewesen, und jetzt war
               ich wieder in \textcolor{pink}{Pressbaum}{}\ledrightnote{\textcolor{pink}{Pressbaum}}. Ich gehe am 8\textsuperscript{ten} nach \label{K_L03280-1v}\edtext{\textcolor{pink}{Reichenhall}{}\ledrightnote{\textcolor{pink}{Bad Reichenhall}}}{\lemma{\textnormal{\emph{Reichenhall}}}\Cendnote{\textnormal{nicht
                  geschehen}}}\label{K_L03280-1h}, wo ich bis 1. September bleibe.
               Vielleicht kommen Sie einmal vorbei. Dort schreibe ich das \label{K_L03280-2v}\edtext{\textcolor{pink}{österr.}{}\ledrightnote{\textcolor{pink}{Österreich}} Theater}{\lemma{\textnormal{\emph{österr. Theater}}}\Cendnote{\textnormal{nicht ermittelt}}}\label{K_L03280-2h}. Stimmung und Befinden nicht
               hervorragend. In \textcolor{pink}{Karlsbad}{}\ledrightnote{\textcolor{pink}{Karlsbad}} ein hübsches Erlebnis.
               Ab 1. August wohne ich \textcolor{pink}{Hietzing, Wattmanngasse 11}{}\ledrightnote{\textcolor{pink}{Wattmanngasse}}, doch bitte ich mir Briefe nur hieher, damit sie
               mir nachgeschickt werden.\pend
           
\pstart
           {\pb}Viele Grüße herzlichst {\\[\baselineskip]}Ihr {\\[\baselineskip]}\spacefill\mbox{Salten}\pend
           \leftskip=0em{}\endnumbering\briefempfaengerindex{Schnitzler, Arthur@\textsc{Schnitzler, Arthur}!zzzSalten, Felix@\emph{von Felix Salten}!1898-07-301@{30. 7. 1898}|)be}\mylabel{h}  \normalsize

\doendnotes{C}
\bigskip
\vfill

\clearpage

\footnotesize

\lohead{\textsc{register}}

% Definiere theindex-Environment komplett neu ohne reledmac
\makeatletter
\renewenvironment{theindex}{%
  \section*{\indexname}%
  \setlength{\parindent}{0pt}%
  \setlength{\parskip}{0pt plus 0.3pt}%
  \let\item\@idxitem
}{%
  \clearpage
}
\makeatother

\IfFileExists{\jobname-pw.ind}{\input{\jobname-pw.ind}}{}

\end{document}

      