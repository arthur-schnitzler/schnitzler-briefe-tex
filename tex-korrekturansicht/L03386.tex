%% latex-korrekturansicht-vorspann.tex
%% Vorspann für die Korrekturansicht.
%% Lädt die gemeinsame Datei latex-vorspann.tex mit gesetztem Schalter.

\newif\ifkorrekturansicht
\korrekturansichttrue

\input{../tex-inputs/latex-vorspann}


\renewcommand{\erwaehntePersonen}{Personen:  ?? [Partner von Theodore Rottenberg, Ende 1902/Anfang 1903], Fedor Mamroth, Emilio Rizzi, Georges Rodenbach, Josef Rosengart, Theodore Rottenberg, Ludwig Rottenberg, Olga Schnitzler}
\renewcommand{\erwaehnteInstitutionen}{Institutionen: Deutsches Theater Berlin}
\renewcommand{\erwaehnteOrte}{Orte: Berlin, Hotel Centrale Rovereto, Rovereto, Venedig, Wien}
\renewcommand{\erwaehnteWerke}{Werke: Der Puppenspieler. Studie in einem Aufzuge, Die stille Stadt. Schauspiel in 4 Akten}
\section[ Paul Goldmann an Arthur Schnitzler, 7. 9. 1903]{Paul Goldmann an Arthur Schnitzler, 7. 9. 1903}
\nopagebreak\mylabel{v}
\rehead{ }\normalsize\beginnumbering\briefempfaengerindex{Schnitzler, Arthur@\textsc{Schnitzler, Arthur}!zzzGoldmann, Paul@\emph{von Paul Goldmann}!1903-09-071@{7. 9. 1903}|(be}
\toendnotes[C]{\smallbreak\pagebreak[2]}\Standort{DLA, A:Schnitzler, HS.NZ85.1.3173.}
\physDesc{Brief, 1 Blatt, 2 Seiten, 1196 Zeichen
\newline{}Handschrift: schwarze Tinte, deutsche Kurrent}\toendnotes[C]{\smallbreak}
\pstart
           \noindent{}\raggedleft{}{\pb}\textcolor{gray}{\textbf{\textcolor{pink}{Rovereto}{}\ledrightnote{\textcolor{pink}{Rovereto}}{ }\begin{otherlanguage}{italian}li\end{otherlanguage}}}{ }7. September \textcolor{gray}{\textbf{190}}3.\pend
           
\pstart
           \noindent{}\raggedleft{}\textcolor{gray}{\textbf{\textbf{\textcolor{pink}{Hôtel Centrale}{}\ledrightnote{\textcolor{pink}{Hotel Centrale Rovereto}}}}}\pend
           
\pstart
           \noindent{}\raggedleft{}\textcolor{gray}{\textbf{\textbf{\textcolor{blue}{E. RIZZI}{}\ledrightnote{\textcolor{blue}{Emilio Rizzi}}} – \textcolor{pink}{ROVERETO}{}\ledrightnote{\textcolor{pink}{Rovereto}}}}\pend
           
\pstart\center{}Mein lieber Freund,\pend
\pstart
           Wenn Du am \label{K_L03386-11v}\edtext{15. September{ }\textcolor{pink}{Wien}{}\ledrightnote{\textcolor{pink}{Wien}} verlaſſen}{\lemma{\textnormal{\emph{15. September Wien verlaſſen}}}\Cendnote{\textnormal{\textcolor{blue}{Schnitzler} blieb den September über in \textcolor{pink}{Wien}, dürfte aber bis zum 17. [9. 1903?] geplant haben,
                  abzureisen. Vermutlich hatte er überlegt, wenn schon nicht zur Uraufführung seines
                  Einakters \emph{\textcolor{green}{Der Puppenspieler}} am \emph{\textcolor{brown}{Deutschen Theater}} in \textcolor{pink}{Berlin}, dann zumindest zu einer der folgenden Aufführungen zu reisen. Die
                  Uraufführung fand am 12. 9. 1903 gemeinsam mit der Premiere von \emph{\textcolor{green}{Das Trugbild}} von \textcolor{blue}{Georges Rodenbach} statt.}}}\label{K_L03386-11h} willſt, würde ich wohl
               kaum die Freude haben, Dich auf meiner Rückreiſe zu ſehen. Meine \textcolor{blue}{Freundin}{}\ledrightnote{{$\rightarrow$}\textcolor{blue}{Theodore Rottenberg}} iſt vor einigen Tagen
               heimgefahren. Die Briefe des \label{K_L03386-1v}\edtext{\textcolor{blue}{Mann}{}\ledrightnote{{$\rightarrow$}\textcolor{blue}{Ludwig Rottenberg}}}{\lemma{\textnormal{\emph{Mann}}}\Cendnote{\textnormal{\textcolor{blue}{Theodore Rottenberg}s Ehemann \textcolor{blue}{Ludwig Rottenberg}}}}\label{K_L03386-1h}es wurden drohend und ſchienen eine Kataſtrophe anzukündigen. Was nach der
               Heimkehr der armen \textcolor{blue}{Frau}{}\ledrightnote{{$\rightarrow$}\textcolor{blue}{Theodore Rottenberg}}
               geſchehen iſt, weiß ich noch nicht. Auch auf meiner Seite gibt es \strikeout{ung\textcolor{gray}{e}} unvorhergeſehne Complikationen. Ich erhielt einen Brief meines \textcolor{blue}{Schwager}{}\ledrightnote{{$\rightarrow$}\textcolor{blue}{Josef Rosengart}}s, der beſagt, dieſe
                  \textcolor{blue}{Frau}{}\ledrightnote{{$\rightarrow$}\textcolor{blue}{Theodore Rottenberg}} ſei nach den \label{K_L03386-2v}\edtext{Ereigniſſen dieſes Winters}{\lemma{\textnormal{\emph{Ereigniſſen … Winters}}}\Cendnote{\textnormal{Ende 1902 bis Anfang 1903 waren
                     \textcolor{blue}{Goldmann} und \textcolor{blue}{Rottenberg} getrennt und sie mit einem anderen \textcolor{blue}{Mann} in einer Beziehung.
                     Vgl. Paul Goldmann an Arthur Schnitzler, 14. 11. [1903], 28. 12. [1902] und 17. 2. [1903].}}}\label{K_L03386-2h} nicht
               mehr eine Frau, die man heirathet, und der mich vor die Wahl zwiſchen einer Heirath
               und einem Bruch mit meinem \textcolor{blue}{Schwager}{}\ledrightnote{{$\rightarrow$}\textcolor{blue}{Josef Rosengart}} ſtellt. Mein \textcolor{blue}{Onkel}{}\ledrightnote{{$\rightarrow$}\textcolor{blue}{Fedor Mamroth}}, den ich unterwegs getroffen, ſpricht zu mir\strikeout{, \textcolor{gray}{dem}} in dem milden und mitleidigen Tone, in dem man zu Jemandem ſpricht, der im
               Begriff iſt, ſich in ein großes Unheil zu ſtürzen. Ich weiß in dieſem {\pb}Widerſtreit der Empfindungen wieder nicht aus noch
               ein.\pend
           
\pstart
           Heut fahre ich ein paar Tage nach \textcolor{pink}{Venedig}{}\ledrightnote{\textcolor{pink}{Venedig}}. Vor Montag bin ich kaum
               in \label{K_L03386-3v}\edtext{\textcolor{pink}{Wien}{}\ledrightnote{\textcolor{pink}{Wien}}}{\lemma{\textnormal{\emph{Wien}}}\Cendnote{\textnormal{\textcolor{blue}{Goldmann} war spätestens am 17. [9. 1903?] in \textcolor{pink}{Wien} und er und \textcolor{blue}{Schnitzler} sahen sich am 18. 9. 1903, 20. 9. 1903 und 21. 9. 1903.}}}\label{K_L03386-3h}. Natürlich wirſt Du Dich in
               Deinen Reiſedispoſitionen durch mich keineswegs ſtören laſſen. Wenn Du mir etwas
               ſchreiben willſt: \textsc{\textcolor{pink}{Venedig}{}\ledrightnote{\textcolor{pink}{Venedig}}, Poste restante}.\pend
           
\pstart
           Ich grüße Dich und Deine \textcolor{blue}{Frau}{}\ledrightnote{{$\rightarrow$}\textcolor{blue}{Olga Schnitzler}} auf das Herzlichſte. {\\[\baselineskip]}Dein treuer {\\[\baselineskip]}\spacefill\mbox{Paul Goldm}\pend
           \leftskip=0em{}\endnumbering\briefempfaengerindex{Schnitzler, Arthur@\textsc{Schnitzler, Arthur}!zzzGoldmann, Paul@\emph{von Paul Goldmann}!1903-09-071@{7. 9. 1903}|)be}\mylabel{h}
\begin{anhang}
\end{anhang}\normalsize

\doendnotes{C}
\bigskip
\vfill

\clearpage

\footnotesize

\lohead{\textsc{register}}

% Definiere theindex-Environment komplett neu ohne reledmac
\makeatletter
\renewenvironment{theindex}{%
  \section*{\indexname}%
  \setlength{\parindent}{0pt}%
  \setlength{\parskip}{0pt plus 0.3pt}%
  \let\item\@idxitem
}{%
  \clearpage
}
\makeatother

\IfFileExists{\jobname-pw.ind}{\input{\jobname-pw.ind}}{}

\end{document}

      