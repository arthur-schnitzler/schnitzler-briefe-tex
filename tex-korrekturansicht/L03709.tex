%% latex-korrekturansicht-vorspann.tex
%% Vorspann für die Korrekturansicht.
%% Lädt die gemeinsame Datei latex-vorspann.tex mit gesetztem Schalter.

\newif\ifkorrekturansicht
\korrekturansichttrue

\input{../tex-inputs/latex-vorspann}


\section[Elsa Plessner an Arthur Schnitzler, 23. 12. 1896]{L03709 Elsa Plessner an Arthur Schnitzler, 23. 12. 1896}
\nopagebreak\mylabel{L03709v}
\rehead{ }\normalsize\beginnumbering\briefempfaengerindex{Schnitzler, Arthur@\textsc{Schnitzler, Arthur}!zzzPlessner, Elsa@\emph{von Elsa Plessner}!1896-12-232@{23. 12. 1896}|(be}
\toendnotes[C]{\smallbreak\pagebreak[2]}
\correspDesc{Versand  durch Elsa Plessner am 23. 12. 1896 in Meran
\newline{}Erhalt  durch Arthur Schnitzler im Zeitraum [24. 12. 1896 – 28. 12. 1896?] in Wien}\toendnotes[C]{\smallbreak}
\Standort{DLA, A:Schnitzler, HS.1985.1.419.}
\physDesc{Brief, 1 Blatt, 3 Seiten, 2173 Zeichen
\newline{}Handschrift: schwarze Tinte, lateinische Kurrent
\newline{}Schnitzler: mit rotem Buntstift eine Unterstreichung }\toendnotes[C]{\smallbreak}
\pstart
           \raggedleft{}{\pb}\textcolor{pink}{Meran, Pension Wolf}\oindex{Hotel Meranerhof@\textbf{Hotel Meranerhof}, \emph{Hotel}|pw}{}\ledrightnote{\textcolor{pink}{Hotel Meranerhof}}, den 23. Dez. 1896.\pend
           
\pstart
           \raggedleft{}½ 12 Uhr Nachts \pend
           
\pstart\center{}Verehrter Herr Doctor!! – –\pend\vspace{0.5em}
\pstart
           Hallelujah!! – Mit demselben Tintentropfen, mit welchem ich das Wort »Ende« unter
               mein neues \textcolor{green}{Stück}\pwindex{Plessner, Elsa 22.\,8.\,1875 Wien – 7.\,5.\,1932 Alicante@\textsc{Plessner, Elsa} (22.\,8.\,1875 Wien – 7.\,5.\,1932 Alicante), \emph{Schriftstellerin}!Orchideen [Schauspiel in drei Akten]@\strich\emph{Orchideen [Schauspiel in drei Akten]}|pwv}{}\ledrightnote{{$\rightarrow$}\emph{\textcolor{green}{Orchideen [Schauspiel in drei Akten]}}}{ }\uline{soeben} gesetzt habe – erhalten Sie diese Zeilen
               geschmiert – was Sie mir mit Rücksicht auf diese, Ihnen bekannte Stimmung verzeihen
               werden – . (Einen Styl –{ }\introOben{}was?\introOben{}– !!?) Aber das macht nichts!! – Ich freue mich – denn
                  »\textcolor{green}{Orchideen}\pwindex{Plessner, Elsa 22.\,8.\,1875 Wien – 7.\,5.\,1932 Alicante@\textsc{Plessner, Elsa} (22.\,8.\,1875 Wien – 7.\,5.\,1932 Alicante), \emph{Schriftstellerin}!Orchideen [Schauspiel in drei Akten]@\strich\emph{Orchideen [Schauspiel in drei Akten]}|pw}{}\ledrightnote{\textcolor{green}{Orchideen [Schauspiel in drei Akten]}}« Schauspiel in 3 Acten, ist mir
               gelungen – oder ich heiße \textcolor{blue}{Eugenie Marlitt}\pwindex{Marlitt, E. 5.\,12.\,1825 Arnstadt – 22.\,6.\,1887 ebd.@\textsc{Marlitt, E.} (5.\,12.\,1825 Arnstadt – 22.\,6.\,1887 ebd.), \emph{Schriftstellerin, Sängerin}|pw}{}\ledrightnote{\textcolor{blue}{E. Marlitt}}!! –
               Sie erhalten es, sobald Feile und Abschrift {\pb}hinter mir,
               zur gütigen Durchsicht! – Es ist ein unerbittliches \textcolor{green}{Stück}\pwindex{Plessner, Elsa 22.\,8.\,1875 Wien – 7.\,5.\,1932 Alicante@\textsc{Plessner, Elsa} (22.\,8.\,1875 Wien – 7.\,5.\,1932 Alicante), \emph{Schriftstellerin}!Orchideen [Schauspiel in drei Akten]@\strich\emph{Orchideen [Schauspiel in drei Akten]}|pwv}{}\ledrightnote{{$\rightarrow$}\emph{\textcolor{green}{Orchideen [Schauspiel in drei Akten]}}}, von dramatischer Wucht (das ist Thatsache – lachen
               Sie nicht – bitte) und wie ich glaube \uline{echter} Tragik!
               – Thatsache – blos – ich habe \uuline{alles} zusammengekratzt,
               was ich an Können und künstlerischem Wollen besitze – und auch die negativen
               Erfahrungen des »\textcolor{green}{Heimweh}\pwindex{Plessner, Elsa 22.\,8.\,1875 Wien – 7.\,5.\,1932 Alicante@\textsc{Plessner, Elsa} (22.\,8.\,1875 Wien – 7.\,5.\,1932 Alicante), \emph{Schriftstellerin}!Heimweh [dreiaktige Tragikomödie]@\strich\emph{Heimweh [dreiaktige Tragikomödie]}|pw}{}\ledrightnote{\textcolor{green}{Heimweh [dreiaktige Tragikomödie]}}« haben mir genützt –
               und mein zweites \textcolor{green}{Stück}\pwindex{Plessner, Elsa 22.\,8.\,1875 Wien – 7.\,5.\,1932 Alicante@\textsc{Plessner, Elsa} (22.\,8.\,1875 Wien – 7.\,5.\,1932 Alicante), \emph{Schriftstellerin}!Orchideen [Schauspiel in drei Akten]@\strich\emph{Orchideen [Schauspiel in drei Akten]}|pwv}{}\ledrightnote{{$\rightarrow$}\emph{\textcolor{green}{Orchideen [Schauspiel in drei Akten]}}}, fast
                  \uline{2 Jahre} nach dem \textcolor{green}{ersten}\pwindex{Plessner, Elsa 22.\,8.\,1875 Wien – 7.\,5.\,1932 Alicante@\textsc{Plessner, Elsa} (22.\,8.\,1875 Wien – 7.\,5.\,1932 Alicante), \emph{Schriftstellerin}!Heimweh [dreiaktige Tragikomödie]@\strich\emph{Heimweh [dreiaktige Tragikomödie]}|pwv}{}\ledrightnote{{$\rightarrow$}\emph{\textcolor{green}{Heimweh [dreiaktige Tragikomödie]}}} entstanden{[},{]}{ }\uline{muß}
               aufführbar sein – sonst kann ich die Kratzerei an den Nagel hängen!! – Wenn Alles was
               ich besitze nicht genug ist – – – ! – Tausend herzlichen Dank für \label{K_L03709-1v}\edtext{Ihre reizenden Zeilen}{\lemma{\textnormal{\emph{Ihre reizenden Zeilen}}}\Cendnote{\textnormal{nicht überliefert}}}\label{K_L03709-1}, die mir mitten in
               meiner Arbeit ein lieber, anfeuernder Gruß {\pb}erschienen!
               – – – Das Scenarium und die Disposition habe – 5 mal geschmissen und von Grund wieder
               aufgebaut – na – wie steh ich da? – Freilich – wenn es Glück haben sollte – und warum
               soll eine blinde Henne wie ich, nicht einmal ein Körnchen finden – würde das
               Publikum, sagen »Arche (arge) Ideen« hat E. P.
               – (»Witze thu ich auch machen«!!) – – Aber gearbeitet habe ich – wie ein Holzknecht!!
               – Auch \label{K_L03709-2v}\edtext{\begin{otherlanguage}{french}à la\end{otherlanguage}{ }\textcolor{green}{Penelope}\pwindex{Homer @\textsc{Homer}, \emph{Schriftsteller}!Odyssee@\strich\emph{Odyssee}|pw}{}\ledrightnote{\textcolor{green}{Odyssee}}}{\lemma{\textnormal{\emph{à la Penelope}}}\Cendnote{\textnormal{Während Penelope im Epos der \emph{\textcolor{green}{Odyssee}\pwindex{Homer @\textsc{Homer}, \emph{Schriftsteller}!Odyssee@\strich\emph{Odyssee}|pwk}} auf die Rückkehr ihres Gatten Odysseus
                  von Kriegs- und Irrfahrten wartete, trennte sie nachts das Tuch auf, das sie
                  tagsüber webte, um die Freier hinzuhalten, die sie zu einer neuen Hochzeit drängen
                  wollten.}}}\label{K_L03709-2}, denn sehr oft Morgens verbrannt, was Abends geschrieben!! –
               Wenn das meine Ärzte wüssten, die meine »Nerven« nach \textcolor{pink}{Meran}\oindex{Meran@\textbf{Meran}, \emph{Hauptstadt}|pw}{}\ledrightnote{\textcolor{pink}{Meran}} geschickt haben – – \begin{otherlanguage}{french}\label{K_L03709-3v}\edtext{Entre nous}{\lemma{\textnormal{\emph{Entre nous}}}\Cendnote{\textnormal{französisch: unter uns}}}\label{K_L03709-3}{ }\end{otherlanguage}! – Besser sind freilich die hohen Herschaften dadurch nicht geworden – –
               Aber dafür hole ich es jetzt nach und lege mir ein paar Kurkilogramme
               zu! – Aber der Schnee! – Und \uline{die}!! – Hundekälte –!
               Auf meinem Südbalcon kann ich Schlittschuh laufen!! – – – – \begin{otherlanguage}{english}\label{K_L03709-4v}\edtext{Merry Christmas and new years (100)
                  and all the holidays}{\lemma{\textnormal{\emph{Merry … holidays}}}\Cendnote{\textnormal{englisch: frohe
                     Weihnachten und neue Jahre (100) und all die Ferien}}}\label{K_L03709-4}{ }\end{otherlanguage}!!! – \label{K_L03709-5v}\edtext{Gratulire »\textcolor{violet}{\textcolor{green}{Freiwild}\pwindex{Schnitzler, Arthur 15. 5. 1862 Wien – 21. 10. 1931 ebd.@\textsc{Schnitzler, Arthur} (15. 5. 1862 Wien – 21. 10. 1931 ebd.), \emph{Schriftsteller, Mediziner}!Freiwild. Schauspiel in 3 Akten@\strich\emph{Freiwild. Schauspiel in 3 Akten}|pw}{}\ledrightnote{\textcolor{green}{Freiwild. Schauspiel in 3 Akten}}}\eventindex{Deutsches Theater Berlin@\textbf{Deutsches Theater Berlin}!Uraufführung von Freiwild, 3.11.1896@Uraufführung von Freiwild, 3.11.1896|pwv}{}\ledrightnote{{$\rightarrow$}\emph{\textcolor{violet}{Uraufführung von Freiwild, 3.11.1896}}}}{\lemma{\textnormal{\emph{Gratulire »Freiwild}}}\Cendnote{\textnormal{Auch \textcolor{blue}{Schnitzler} verbucht die \textcolor{pink}{Berliner}\oindex{Berlin@\textbf{Berlin}, \emph{Hauptstadt}|pwk}{ }\textcolor{violet}{Erstaufführung von \emph{\textcolor{green}{Freiwild}\pwindex{Schnitzler, Arthur 15. 5. 1862 Wien – 21. 10. 1931 ebd.@\textsc{Schnitzler, Arthur} (15. 5. 1862 Wien – 21. 10. 1931 ebd.), \emph{Schriftsteller, Mediziner}!Freiwild. Schauspiel in 3 Akten@\strich\emph{Freiwild. Schauspiel in 3 Akten}|pwk}}}\eventindex{Deutsches Theater Berlin@\textbf{Deutsches Theater Berlin}!Uraufführung von Freiwild, 3.11.1896@Uraufführung von Freiwild, 3.11.1896|pwk} als Erfolg, vgl. A. S.: \emph{Tagebuch}, 3. 11. 1896.}}}\label{K_L03709-5}« – \label{K_L03709-6v}\edtext{\textcolor{pink}{Breslau}\oindex{Breslau@\textbf{Breslau}|pw}{}\ledrightnote{\textcolor{pink}{Breslau}}}{\lemma{\textnormal{\emph{Breslau}}}\Cendnote{\textnormal{\textcolor{blue}{Schnitzler} war am 26. 10. 1896 über \textcolor{pink}{Breslau}\oindex{Breslau@\textbf{Breslau}|pwk} nach
                     \textcolor{pink}{Berlin}\oindex{Berlin@\textbf{Berlin}, \emph{Hauptstadt}|pwk} gefahren.}}}\label{K_L03709-6}. Fräulein
               \label{K_L03709-7v}\edtext{\textcolor{blue}{Jurberg}\pwindex{Jurberg, Gisela 15.\,7.\,1874 Wien – 20.\,7.\,1942 Hannover@\textsc{Jurberg, Gisela} (15.\,7.\,1874 Wien – 20.\,7.\,1942 Hannover), \emph{Schauspielerin}|pw}{}\ledrightnote{\textcolor{blue}{Gisela Jurberg}}}{\lemma{\textnormal{\emph{Jurberg}}}\Cendnote{\textnormal{Die aus \textcolor{pink}{Wien}\oindex{Wien@\textbf{Wien}, \emph{Verwaltungsgebiet}|pwk} stammende \textcolor{blue}{Gisela Jurberg}\pwindex{Jurberg, Gisela 15.\,7.\,1874 Wien – 20.\,7.\,1942 Hannover@\textsc{Jurberg, Gisela} (15.\,7.\,1874 Wien – 20.\,7.\,1942 Hannover), \emph{Schauspielerin}|pwk} spielte in der
                  Die \textcolor{violet}{Theaterpremiere von \emph{\textcolor{green}{Liebelei. Schauspiel in drei Akten}\pwindex{Schnitzler, Arthur 15. 5. 1862 Wien – 21. 10. 1931 ebd.@\textsc{Schnitzler, Arthur} (15. 5. 1862 Wien – 21. 10. 1931 ebd.), \emph{Schriftsteller, Mediziner}!Liebelei. Schauspiel in drei Akten@\strich\emph{Liebelei. Schauspiel in drei Akten}|pwk}} von \textcolor{blue}{Arthur Schnitzler}}\eventindex{Lobe-Theater@\textbf{Lobe-Theater}!Premiere von Liebelei, 11.2.1896@Premiere von Liebelei, 11.2.1896|pwk} fand am 11. 2. 1896 am \emph{\textcolor{brown}{Lobe-Theater}\orgindex{Lobe-Theater@Lobe-Theater|pwk}} im \textcolor{pink}{Lobe-Theater}\oindex{Lobe-Theater@\textbf{Lobe-Theater}, \emph{Theater}|pwk} statt.
               }}}\label{K_L03709-7} gesehen? – »\label{K_L03709-8v}\edtext{Süsses Mädel}{\lemma{\textnormal{\emph{Süsses Mädel}}}\Cendnote{\textnormal{Eine
                  Wortprägung, die auf \textcolor{blue}{Schnitzler} zurückgeht
                  und die junge Frauen aus einfachen Verhältnissen bezeichnet, mit denen sich
                  wohlsituierte Männer auf Liebschaften einlassen, die von diesen aber niemals für
                  eine Ehe in Betracht gezogen werden.}}}\label{K_L03709-8}« \pend
           
\pstart
           Hochachtungsvolle Grüße{\\[\baselineskip]}your{\\[\baselineskip]}\spacefill\mbox{Elsa Plessner}\pend
           \leftskip=0em{}
\pstart
           \noindent{}\raggedleft{}(\begin{otherlanguage}{english}\label{K_L03709-9v}\edtext{a little foolish}{\lemma{\textnormal{\emph{a little foolish}}}\Cendnote{\textnormal{englisch: ein bisschen töricht}}}\label{K_L03709-9}\end{otherlanguage})\pend
           \selectlanguage{ngerman}\endnumbering\briefempfaengerindex{Schnitzler, Arthur@\textsc{Schnitzler, Arthur}!zzzPlessner, Elsa@\emph{von Elsa Plessner}!1896-12-232@{23. 12. 1896}|)be}\mylabel{L03709h}  \normalsize

\doendnotes{C}
\bigskip
\vfill

\clearpage

\footnotesize

\lohead{\textsc{register}}

% Definiere theindex-Environment komplett neu ohne reledmac
\makeatletter
\renewenvironment{theindex}{%
  \section*{\indexname}%
  \setlength{\parindent}{0pt}%
  \setlength{\parskip}{0pt plus 0.3pt}%
  \let\item\@idxitem
}{%
  \clearpage
}
\makeatother

\IfFileExists{\jobname-pw.ind}{\input{\jobname-pw.ind}}{}

\end{document}

      