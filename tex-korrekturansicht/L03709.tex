%% latex-korrekturansicht-vorspann.tex
%% Vorspann für die Korrekturansicht.
%% Lädt die gemeinsame Datei latex-vorspann.tex mit gesetztem Schalter.

\newif\ifkorrekturansicht
\korrekturansichttrue

\input{../tex-inputs/latex-vorspann}


\renewcommand{\erwaehntePersonen}{Personen: Grete Litzmann, E. Marlitt, Elsa Plessner}
\renewcommand{\erwaehnteOrte}{Orte: Breslau, Hotel Meranerhof, Meran, Wien}
\renewcommand{\erwaehnteWerke}{Werke: Freiwild. Schauspiel in 3 Akten, Heimweh [dreiaktige Tragikomödie], Odyssee, Orchideen [Schauspiel in drei Akten]}
\section[Elsa Plessner an Arthur Schnitzler, 23. 12. 1896]{Elsa Plessner an Arthur Schnitzler, 23. 12. 1896}
\nopagebreak\mylabel{v}
\rehead{ }\normalsize\beginnumbering\briefempfaengerindex{Schnitzler, Arthur@\textsc{Schnitzler, Arthur}!zzzPlessner, Elsa@\emph{von Elsa Plessner}!1896-12-231@{23. 12. 1896}|(be}
\toendnotes[C]{\smallbreak\pagebreak[2]}\Standort{DLA, A:Schnitzler, HS.1985.1.419.}
\physDesc{Brief,  Blätter, 3 Seiten, 2171 Zeichen
\newline{}Handschrift: , lateinische Kurrent
\newline{}Schnitzler: eine Unterstreichung }\toendnotes[C]{\smallbreak}
\pstart
           {\pb}\textcolor{pink}{Meran, Pension Wolf}{}\ledrightnote{\textcolor{pink}{Hotel Meranerhof}}, den 23. Dez.
                     1896.\pend
           
\pstart
           \raggedleft{}½ 12 Uhr Nachts \pend
           
\pstart{}Verehrter Herr Doctor!! – –\pend\vspace{0.5em}
\pstart
           Hallelujah!! – Mit denselben Tintentropfen, mit welchem ich das Wort »Ende« unter
               mein neues \textcolor{green}{Stück}{}\ledrightnote{{$\rightarrow$}\textcolor{green}{Orchideen [Schauspiel in drei Akten]}}{ }\uline{soeben} gesetzt habe – erhalten Sie diese Zeilen
               geschmiert – was Sie mir mit Rücksicht auf diese, Ihnen bekannte Stimmung verzeihen
               werden – . (Einen Styl – – \introOben{}was?\introOben{} !!?) Aber das macht nichts!!
               – Ich freue mich – denn »\textcolor{green}{Orchideen}{}\ledrightnote{\textcolor{green}{Orchideen [Schauspiel in drei Akten]}}« Schauspiel
               in 3 Acten, ist mir gelungen – oder ich heiße \textcolor{blue}{Eugenie Marlitt}{}\ledrightnote{\textcolor{blue}{E. Marlitt}}!! – Sie erhalten es, sobald Feile und Abschrift {\pb}hinter mir, zur gütigen Durchsicht! – Es ist ein unerbittliches \textcolor{green}{Stück}{}\ledrightnote{{$\rightarrow$}\textcolor{green}{Orchideen [Schauspiel in drei Akten]}}, von dramatischer Wucht das ist
               Thatsache – lachen Sie nicht – bitte) und wie ich glaube \uline{echter} Tragik! – Thatsache – blos – ich habe \uuline{alles} zusammengekratzt, was ich an Können und künstlerischem Wollen besitze
               – und auch die negativen Erfahrungen des »\textcolor{green}{Heimweh}{}\ledrightnote{\textcolor{green}{Heimweh [dreiaktige Tragikomödie]}}« haben mir genützt – und mein zweites \textcolor{green}{Stück}{}\ledrightnote{{$\rightarrow$}\textcolor{green}{Orchideen [Schauspiel in drei Akten]}}, fast \uline{2
                  Jahre} nach dem \textcolor{green}{ersten}{}\ledrightnote{{$\rightarrow$}\textcolor{green}{Heimweh [dreiaktige Tragikomödie]}}
               entstanden \uline{muß} aufführbar sein – sonst kann ich die
               Kratzerei an den Nagel hängen!! – Wenn Alles was ich besitze nicht genug ist – – – !
               – Tausend herzlichen Dank für Ihre reizenden \label{K_L03709-1v}\edtext{Zeilen}{\lemma{\textnormal{\emph{Zeilen}}}\Cendnote{\textnormal{nicht
                  überliefert}}}\label{}, die mir mitten in meiner Arbeit ein lieber, anfeuernder Gruß
                  {\pb}erschienen! – – – Das Scenarium und die Disposition habe – 5 mal
               geschmissen und von Grund wieder aufgebaut – na – wie steh ich da? – Freilich – wenn
               es Glück haben sollte – und warum soll eine blinde Henne wie ich, nicht einmal ein
               Körnchen finden – würde das Publikum, sagen »Arche (arge) Ideen« hat \textcolor{blue}{E. P.}{}\ledrightnote{\textcolor{blue}{Elsa Plessner}} – (»Witze thu ich auch machen«!!) – – Aber
               gearbeitet habe ich – wie ein Holzknecht!! – Auch \label{K_L03709-2v}\edtext{\begin{otherlanguage}{french}à la\end{otherlanguage}{ }\textcolor{green}{Penelope}{}\ledrightnote{\textcolor{green}{Odyssee}}}{\lemma{\textnormal{\emph{à la Penelope}}}\Cendnote{\textnormal{Während Penelope im Epos der \emph{\textcolor{green}{Odyssee}} auf die Rückkehr ihres Gatten Odysseus
                  von Kriegs- und Irrfahrten wartete, trennte sie nachts das Tuch auf, das sie
                  tagsüber webte, um die Freier hinzuhalten, die sie zu einer neuen Hochzeit drängen
                  wollten.}}}\label{}, denn sehr oft Morgens verbrannt, was Abends geschrieben!! – –
               Wenn das meine Ärzte wüssten, die meine »Nerven« nach \textcolor{pink}{Meran}{}\ledrightnote{\textcolor{pink}{Meran}} geschickt haben – – \begin{otherlanguage}{french}\label{K_L03709-3v}\edtext{Entre nous}{\lemma{\textnormal{\emph{Entre nous}}}\Cendnote{\textnormal{französisch: unter uns}}}\label{}{ }\end{otherlanguage}! – Besser sind freilich die hohen Herschaften dadurch nicht geworden – –
               Aber dafür hole ich es jetzt nach und lege mir ein paar Kurkilogramme
               zu! – Aber der Schnee! – Und \uline{die}!! – Hundekälte –!
               Auf meinem Südbalcon kann ich Schlittschuh laufen!! – – – – \begin{otherlanguage}{english}\label{K_L03709-4v}\edtext{Merry Christmas and new years (100)
                  and all the holidays}{\lemma{\textnormal{\emph{Merry … holidays}}}\Cendnote{\textnormal{englisch: frohe
                     Weihnachten und neue Jahre (100) und all die Ferien}}}\label{}{ }\end{otherlanguage}!!! – Gratulire \label{K_L03709-5v}\edtext{»\textcolor{green}{Freiwild}{}\ledrightnote{\textcolor{green}{Freiwild. Schauspiel in 3 Akten}}« – \textcolor{pink}{Breslau}{}\ledrightnote{\textcolor{pink}{Breslau}}}{\lemma{\textnormal{\emph{»Freiwild« – Breslau}}}\Cendnote{\textnormal{Auch \textcolor{blue}{Schnitzler} verbucht die \textcolor{pink}{Breslauer}
                  Erstaufführung von \emph{\textcolor{green}{Freiwild}} als Erfolg, vgl. A. S.: \emph{Tagebuch}, 3. 11. 1896.}}}\label{}. Fräulein \textcolor{blue}{\textcolor{gray}{Her}zberg}{}\ledrightnote{\textcolor{blue}{Grete Litzmann}} gesehen? – »\label{K_L03709-22v}\edtext{Süsses Mädel}{\lemma{\textnormal{\emph{Süsses Mädel}}}\Cendnote{\textnormal{Eine Wortprägung, 
                     die auf \textcolor{blue}{Schnitzler} zurückgeht und die junge Frauen aus einfachen Verhältnissen 
                  bezeichnet, mit denen sich wohlsituierte Männer auf Liebschaften einlassen, die von diesen aber niemals
                  für eine Ehe in Betracht gezogen werden. Hugo von Hofmannsthal an Arthur Schnitzler, 1[1?]. 6. 1902}}}\label{}« \pend
           
\pstart
           Hochachtungsvolle Grüße{\\[\baselineskip]}your{\\[\baselineskip]}\spacefill\mbox{Elsa Plessner }\pend
           \leftskip=0em{}
\pstart
           \noindent{}\raggedleft{}(\begin{otherlanguage}{english}\label{K_L03709-6v}\edtext{a little foolish}{\lemma{\textnormal{\emph{a little foolish}}}\Cendnote{\textnormal{englisch: ein bisschen töricht}}}\label{}{ }\end{otherlanguage})\pend
           \endnumbering\briefempfaengerindex{Schnitzler, Arthur@\textsc{Schnitzler, Arthur}!zzzPlessner, Elsa@\emph{von Elsa Plessner}!1896-12-231@{23. 12. 1896}|)be}\mylabel{h}
\begin{anhang}
\end{anhang}\normalsize

\doendnotes{C}
\bigskip
\vfill

\clearpage

\footnotesize

\lohead{\textsc{register}}

% Definiere theindex-Environment komplett neu ohne reledmac
\makeatletter
\renewenvironment{theindex}{%
  \section*{\indexname}%
  \setlength{\parindent}{0pt}%
  \setlength{\parskip}{0pt plus 0.3pt}%
  \let\item\@idxitem
}{%
  \clearpage
}
\makeatother

\IfFileExists{\jobname-pw.ind}{\input{\jobname-pw.ind}}{}

\end{document}

      