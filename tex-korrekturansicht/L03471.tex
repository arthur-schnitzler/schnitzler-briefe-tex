%% latex-korrekturansicht-vorspann.tex
%% Vorspann für die Korrekturansicht.
%% Lädt die gemeinsame Datei latex-vorspann.tex mit gesetztem Schalter.

\newif\ifkorrekturansicht
\korrekturansichttrue

\input{../tex-inputs/latex-vorspann}


\renewcommand{\erwaehntePersonen}{Personen: Moriz Benedikt, Olga Schnitzler, Gustav Schwarzkopf}
\renewcommand{\erwaehnteOrte}{Orte: Hotel Sacher, Semmering, Sternwartestraße, Wien}
\renewcommand{\erwaehnteWerke}{}
\section[ Paul Goldmann an Arthur Schnitzler, 26. 12. 1910]{Paul Goldmann an Arthur Schnitzler, 26. 12. 1910}
\nopagebreak\mylabel{v}
\rehead{ }\normalsize\beginnumbering\briefempfaengerindex{Schnitzler, Arthur@\textsc{Schnitzler, Arthur}!zzzGoldmann, Paul@\emph{von Paul Goldmann}!1910-12-261@{26. 12. 1910}|(be}
\toendnotes[C]{\smallbreak\pagebreak[2]}\Standort{DLA, A:Schnitzler, HS.NZ85.1.3175.}
\physDesc{Kartenbrief, 567 Zeichen
\newline{}Handschrift: 1) schwarze Tinte, deutsche Kurrent\hspace{1em}2) schwarze Tinte, lateinische Kurrent (\noindent{}Adresse)\hspace{1em}
\newline{}Versand: Stempel: »\nobreak{}Wien 1/1 \textcolor{gray}{5 r}, 26 VII 10, 1 40 N\nobreak{}«.  
\newline{}Schnitzler: mit Bleistift »\textcolor{blue}{G{[}oldmann{]}}« vermerkt }\toendnotes[C]{\smallbreak}\pstart{}{\pb}Herrn\pend{}\pstart{}Dr. Arthur Schnitzler\pend{}\pstart{}\textcolor{pink}{Wien}{}\ledrightnote{\textcolor{pink}{Wien}}\pend{}\pstart{}\textcolor{pink}{Sternwartstraſse 71}{}\ledrightnote{\textcolor{pink}{Sternwartestraße}}.\pend{}
{\bigskip}
\pstart
           Montag 26. 12. 10.\pend
           
\pstart{}Lieber Freund,\pend
\pstart
           Darf ich heut{ }Abend mit \textsc{\textcolor{blue}{Schwarzkopf}{}\ledrightnote{\textcolor{blue}{Gustav Schwarzkopf}}} ſo um 8 herum{ }\label{K_L03471-1v}\edtext{zu Dir kommen}{\lemma{\textnormal{\emph{zu Dir kommen}}}\Cendnote{\textnormal{Bei dem Besuch \textcolor{blue}{Goldmann}s – in Begleitung von \textcolor{blue}{Gustav
                     Schwarzkopf} – kam es zu einem Streit, der sich zwei Tage später noch
                  intensivierte (vgl. A. S.: \emph{Tagebuch}, 26. 12. 1910
                  und 28. 12. 1910).
                  Die Themen, die die bereits angeschlagene Beziehung zwischen \textcolor{blue}{Goldmann} und \textcolor{blue}{Schnitzler} nun endgültig ins Wanken brachten, schlagen sich in den
                  folgenden Briefen nieder.}}}\label{K_L03471-1h}? Wenn ja, ſo erbitte ich mir {\pb}Antwort durch Rohrpoſtkarte ins \textsc{\textcolor{pink}{Hotel Sacher}{}\ledrightnote{\textcolor{pink}{Hotel Sacher}}}, wo ich ſie mir gegen 7 Uhr Abends abholen werde. Haſt Du aber
               über den Abend bereits verfügt (was ſehr wahrſcheinlich iſt), ſo
               brauchſt Du gar nicht zu antworten, u. ich verſuche dann in einigen Tagen (morgen muß ich zu \textsc{\textcolor{blue}{Benedikt}{}\ledrightnote{\textcolor{blue}{Moriz Benedikt}}} auf den \textsc{\textcolor{pink}{Semmering}{}\ledrightnote{\textcolor{pink}{Semmering}}}) von Neuem, Dich zu erreichen. Herzliche Grüße Dir u. Deiner \textcolor{blue}{Frau}{}\ledrightnote{{$\rightarrow$}\textcolor{blue}{Olga Schnitzler}} von Deinem \spacefill\mbox{Paul
                  Goldmann.}\pend
           
\pstart
           \noindent{}\label{T_L03471-1v}\edtext{\textsc{\textcolor{blue}{Schwarzkopf}{}\ledrightnote{\textcolor{blue}{Gustav Schwarzkopf}}} verſtändige \uline{ich}.}{\lemma{\textnormal{\emph{Schwarzkopf … ich.}}}\Cendnote{\textnormal{seitlich am rechten Rand, verkehrt zum Text}}}\label{T_L03471-1h}\pend
           \endnumbering\briefempfaengerindex{Schnitzler, Arthur@\textsc{Schnitzler, Arthur}!zzzGoldmann, Paul@\emph{von Paul Goldmann}!1910-12-261@{26. 12. 1910}|)be}\mylabel{h}
\begin{anhang}
\end{anhang}\normalsize

\doendnotes{C}
\bigskip
\vfill

\clearpage

\footnotesize

\lohead{\textsc{register}}

% Definiere theindex-Environment komplett neu ohne reledmac
\makeatletter
\renewenvironment{theindex}{%
  \section*{\indexname}%
  \setlength{\parindent}{0pt}%
  \setlength{\parskip}{0pt plus 0.3pt}%
  \let\item\@idxitem
}{%
  \clearpage
}
\makeatother

\IfFileExists{\jobname-pw.ind}{\input{\jobname-pw.ind}}{}

\end{document}

      