%% latex-korrekturansicht-vorspann.tex
%% Vorspann für die Korrekturansicht.
%% Lädt die gemeinsame Datei latex-vorspann.tex mit gesetztem Schalter.

\newif\ifkorrekturansicht
\korrekturansichttrue

\input{../tex-inputs/latex-vorspann}


               \section[Paul Goldmann an Arthur Schnitzler, 29. 7. {[}1894{]}]{ Paul Goldmann an Arthur Schnitzler, 29. 7. {[}1894{]}}\nopagebreak\mylabel{v}\rehead{ }\normalsize\beginnumbering\briefempfaengerindex{Schnitzler, Arthur@\textsc{Schnitzler, Arthur}!zzzGoldmann, Paul@\emph{von Paul Goldmann}!1894-07-291@{29. 7. {[}1894{]}}|(be} \toendnotes[C]{\smallbreak\pagebreak[2]} \Standort{DLA, A:Schnitzler, HS.NZ85.1.3164.}
\physDesc{Brief, 4 Blätter, 14 Seiten
\newline{}Handschrift: schwarze Tinte, deutsche Kurrent
\newline{}Schnitzler: 1) mit Bleistift auf dem ersten Blatt die Jahreszahl
                                 »94« vermerkt vermerkt 2) mit rotem Buntstift vier Unterstreichungen}\toendnotes[C]{\smallbreak}\pstart
           \noindent{}{\pb}\textcolor{gray}{\textbf{\textcolor{brown}{Frankfurter Zeitung}{}\ledrightnote{\textcolor{brown}{Frankfurter Zeitung}}.}}\hfill \textsc{\textcolor{pink}{Paris}{}\ledrightnote{\textcolor{pink}{Paris}}}, 29. Juli.\pend
           \pstart
           \textcolor{gray}{\textbf{(\textcolor{brown}{Gazette de
                  Francfort}{}\ledrightnote{\textcolor{brown}{Frankfurter Zeitung}}.)}}\pend
           \pstart
           \textcolor{gray}{\textbf{\begin{otherlanguage}{french}Fondateur\end{otherlanguage}{ }\textbf{M. \textcolor{blue}{L.
                  Sonnemann}{}\ledrightnote{\textcolor{blue}{Leopold Sonnemann}}}.}}\pend
           \pstart
           \textcolor{gray}{\textbf{\begin{otherlanguage}{french}Journal politique,
                        financier,\end{otherlanguage}}}\pend
           \pstart
           \textcolor{gray}{\textbf{\begin{otherlanguage}{french}commercial et
                     littéraire.\end{otherlanguage}}}\pend
           \pstart
           \textcolor{gray}{\textbf{\begin{otherlanguage}{french}\textbf{Paraissant trois fois
                           par jour}\end{otherlanguage}}}.\pend
           \pstart
           \textcolor{gray}{\textbf{–}}\pend
           \pstart
           \textcolor{gray}{\textbf{\begin{otherlanguage}{french}\textbf{Bureaux à \textcolor{pink}{Paris}{}\ledrightnote{\textcolor{pink}{Paris}}:}\end{otherlanguage}}}\pend
           \pstart
           \textcolor{gray}{\textbf{\begin{otherlanguage}{french}\textcolor{pink}{\textbf{24. Rue Feydeau}}{}\ledrightnote{\textcolor{pink}{rue Feydeau}}.\end{otherlanguage}}}\pend
           \pstart\center{}Mein lieber Freund,\pend\pstart
           Du haſt ein ſehr ſchönes Siegel.\pend
           \pstart
           Zweitens bitte ich Dich um einen Dienſt: ſei ſo gut und bring mir umgehend die
               Adresse von \textsc{\textcolor{blue}{Hildegard
                  Mitis}{}\ledrightnote{\textcolor{blue}{Hilda von Mitis}}} in Erfahrung. Die Familie wohnt, wie ich glaube, \textsc{\textcolor{pink}{IX. Alserstraße 42}{}\ledrightnote{\textcolor{pink}{Alser Straße}}}. Der
                  \textcolor{blue}{Vater}{}\ledrightnote{→\textcolor{blue}{Maximilian von Mitis}}, der \textcolor{brown}{Landesgerichts-Mitglied}{}\ledrightnote{\textcolor{brown}{Landesgericht für Strafsachen}} iſt, ſteht übrigens ſicher im Adreßbuch.
               Bitte »ſchick« Jemanden hin und ſage: man wolle die Adreſſe der jungen Dame wiſſen,
               um ſie zur Mitarbeiterſchaft an einem Blatte aufzufordern, oder ſo {\pb}etwas! Die Hauptſache iſt, daß Du mir bald einen
               Beſcheid gibſt. Ja?{\dotsfour}\pend
           \pstart
           Mit Deinem \textcolor{blue}{Bruder}{}\ledrightnote{→\textcolor{blue}{Julius Schnitzler}} und Deiner
                  \textcolor{blue}{Schwägerin}{}\ledrightnote{→\textcolor{blue}{Helene Schnitzler}} habe ich ſchöne
               Stunden verlebt. Es iſt aber ſchwer, dieſe Eindrücke zu analyſieren. Es war kein
               Entzücken, ſondern ein langſam entſtehendes Behagen, ein Sich-Zuhauſe-Fühlen bei \substVorne{}\textsuperscript{lieben}{\allowbreak}\substDazwischen{}lieben\substHinten{} Menſchen. Es iſt etwas wie das Gefühl der Treue, das mich mit ihnen
               verbunden hat – obwohl doch dazu eigentlich \textcolor{gray}{nie} lange Zeitdauer
               oder eine Entfernung gehört. Aber ich weiß wirklich nicht, wie ichs nennen ſoll.
               Etwas von Heimaths-Empfindung, wie geſagt, {\pb}war auch
               dabei. Denn die zwei bringen eine Atmoſphäre von Einfachheit, Sanftheit, Güte,
               Gefühlstiefe, Liebenswürdigkeit und Natürlichkeit – das vollendet Wieneriſche mit
               einem Worte – mit, in der ich Vaterlandsloſer allein, \strikeout{\textcolor{gray}{man}} ein Stück Heimat habe. Bei Deinem \textcolor{blue}{Bruder}{}\ledrightnote{→\textcolor{blue}{Julius Schnitzler}} ahne ich das Alles mehr.
               Du weißt, er verſchließt ſich – er hilft Einem nicht dazu, ihn zu verſtehen – und man
               muß ſich ſelbſt auf die Suche machen, um, den verſchiedenen Zügen folgend, die hier
               und da ſeine äußere Maske von Schweigſamkeit und \substVorne{}\textsuperscript{Irone}\substDazwischen{}Ironie\substHinten{} durchdringen, ſich {\pb}das Bild ſeiner, wie
               ich glaube, bedeutenden Individualität zuſammenzufinden. Auch habe ich ihn beſſer
               verſtanden, als er mich. Er geht nicht ſehr auf mich ein – ich bin ihm zu fremd und
               zu verſchieden – auch iſt ja Menſchenſuchen nicht ſein \textsc{Metier}, wie es das meine iſt. Er war mit mir verbunden durch allerlei
               Äußeres – »netter Freund von \textsc{Arthur}« – \substVorne{}\textsuperscript{\textcolor{gray}{A} Amſee}{\allowbreak}\substDazwischen{}\textcolor{pink}{\textsc{\label{K_L02608-1v}\edtext{Almsee}{\lemma{\textnormal{\emph{Almsee}}}\Cendnote{\textnormal{vgl. A. S.: \emph{Tagebuch}, 10. 8. 1889}}}\label{K_L02608-1h}}}{}\ledrightnote{\textcolor{pink}{Almsee}}\substHinten{} – \textcolor{pink}{Pariſ}{}\ledrightnote{\textcolor{pink}{Paris}}er Beiſammenſein. Ich habe ihn aber
               voll zu genießen geſucht und habe ihn ſehr gern. Deine \textcolor{blue}{Schwägerin}{}\ledrightnote{→\textcolor{blue}{Helene Schnitzler}} hingegen iſt eine Seele, in die man klar
               hineinſieht, wie in den lichten Tag. So mild {\pb}und ſo
               gut! So wirklich! So verblüffend geſcheit! Und im Grunde von dieſem lieben kleinen
                  \textcolor{blue}{Ding}{}\ledrightnote{→\textcolor{blue}{Helene Schnitzler}} vermuthe ich eine große
               ſeeliſche Stärke, wie übrigens bei deinem ſtillen \textcolor{blue}{Bruder}{}\ledrightnote{→\textcolor{blue}{Julius Schnitzler}} auch. Die \textcolor{blue}{Beiden}{}\ledrightnote{→\textcolor{blue}{Julius Schnitzler}{\newline}→\textcolor{blue}{Helene Schnitzler}} paſſen zuſammen, als hätte man ſie auf Beſtellung
               füreinander angefertigt. Nur zwiſchen zwei ſolchen Leuten iſt eine anſtändige Ehe
               möglich (obwohl es gewiß nicht immer friedlich bei ihnen zugehen wird, denn ſie ſind
               beide, wie geſagt, ſtolz und ſtark.) {\pb}Mir war es
               eine große, tiefgehende Freude, und der Abſchied hat mir wehgethan (was mir ſchon
               lange nicht vorgekommen).\pend
           \pstart
           Was das Äußere anlangt, ſo muß ich ein Zeugniß ſeltenen Wohlverhaltens ausſtellen.
               Ich habe Deinen \textcolor{blue}{Bruder}{}\ledrightnote{→\textcolor{blue}{Julius Schnitzler}} nicht
               ein einziges Mal den Vornamen ſeiner \textcolor{blue}{Frau}{}\ledrightnote{→\textcolor{blue}{Helene Schnitzler}} ausſprechen gehört. Allerdings war er immer ſehr müde. Dann gäbe es
               noch den Tag in \textsc{\textcolor{pink}{Versailles}{}\ledrightnote{\textcolor{pink}{Versailles}}}, den die \textcolor{blue}{Herrſchaften}{}\ledrightnote{→\textcolor{blue}{Helene Schnitzler}{\newline}→\textcolor{blue}{Julius Schnitzler}}, wenn ich nicht irre, damit verbracht haben,
               ſich Brotkrumen in den Mund zu werfen, ſtatt in die \textsc{\textcolor{pink}{Trianons}{}\ledrightnote{\textcolor{pink}{Petit Trianon}{\newline}\textcolor{pink}{Grand Trianon}}} zu gehen. Auch hat dein \textcolor{blue}{Bruder}{}\ledrightnote{→\textcolor{blue}{Julius Schnitzler}}{ }{\pb}eine nicht immer ganz berechtigte Vorliebe für die
               Dampftramway. Im Übrigen aber muß ich \strikeout{\textcolor{gray}{von}} eine\strikeout{\textcolor{gray}{r}} äußere\strikeout{\textcolor{gray}{n}} Correctheit bekunden, die mich umſomehr überraſcht
               hat, als ich ſie nie vorher bei einem jungen \textcolor{blue}{Ehepaar}{}\ledrightnote{→\textcolor{blue}{Julius Schnitzler}{\newline}→\textcolor{blue}{Helene Schnitzler}} gefunden{\dotsfive}\pend
           \pstart
           Ich danke Dir herzlichſt für Deinen lieben Brief, die \textcolor{green}{Überſetzung}{}\ledrightnote{→\textcolor{green}{Les Emplettes de Noël}} finde ich, unter uns
               geſagt, \label{K_L02608-2v}\edtext{nicht gut}{\lemma{\textnormal{\emph{nicht gut}}}\Cendnote{\textnormal{Auch Schnitzler kommentierte am 21. 7. 1894 in seinem \emph{\textcolor{green}{Tagebuch}}: »Schlecht
                  übersetzt.«}}}\label{K_L02608-2h}. Es fehlt die Farbe. Davon iſt wohl zunächſt die
               Sprache ſchuld, die ſelbſt ſo chauviniſtiſch iſt, daß ſie ſich entſchieden weigert,
               etwas auszudrücken, das nicht ſranzöſiſch iſt. Dann aber auch ein wenig {\pb}der \textcolor{blue}{Überſetzer}{}\ledrightnote{→\textcolor{blue}{Henri Albert}}, obwohl er ſich ehrlich gemüht hat{\dotsfive}\pend
           \pstart
           Am 15. oder 20. August würde ich
               irgendwohin gehen, nach der \textcolor{pink}{Schweiz}{}\ledrightnote{\textcolor{pink}{Schweiz}} oder nach \textcolor{pink}{Tirol}{}\ledrightnote{\textcolor{pink}{Tirol}{\newline}\textcolor{pink}{Südtirol}}, wenn ich irgend ein Ziel hätte. Wäre es
               nicht möglich, Dich ſchon um dieſe Zeit irgendwo zu \label{K_L02608-3v}\edtext{treffen}{\lemma{\textnormal{\emph{treffen}}}\Cendnote{\textnormal{Wahrscheinlich trafen sich Schnitzler und Goldmann erst am 23. 8. 1894 in \textcolor{pink}{Bad Ischl}.}}}\label{K_L02608-3h}?\pend
           \pstart
           Was das Zuſammentreffen mit den \textcolor{blue}{Andren}{}\ledrightnote{→\textcolor{blue}{Hugo von Hofmannsthal}{\newline}→\textcolor{blue}{Richard Beer-Hofmann}} anlangt, ſo grüble ich darüber nach und kann zu
               keinem Schluſſe kommen. Laß’ Dir ein Wort von meinem Gemüthszuftande erzählen: Ich
               habe \textcolor{pink}{Wien}{}\ledrightnote{\textcolor{pink}{Wien}} verlaſſen, und das Leben dort iſt ohne mich
               weitergegangen. Es konnte nicht gut \strikeout{\textcolor{gray}{×}\textcolor{gray}{e}} etwas Anderes {\pb}thun, mir aber bereitet das Schmerz, trotz dieſer
               Einſicht. Über den Platze, auf dem ich geſtanden, iſt Gras geſproſſen – ein wenig
               auch in Euer Mitte (täuſchen wir uns nicht!) Erſt wieder durch das Beiſammenſein mit
               Deinem \textcolor{blue}{Bruder}{}\ledrightnote{→\textcolor{blue}{Julius Schnitzler}} bekam ich ein
               Echo von einem »\textcolor{pink}{Wien}{}\ledrightnote{\textcolor{pink}{Wien}} ohne mich«, – und da ich altes
               dummes Thier mir das, aller Vernunft zum Trotze, anders vorgeſtellt, ſo \strikeout{thun} gab mir das blutende Stiche ins Herz. Man kann {\pb}ſich ſelbſt eben nicht von einem Orte abweſend
               vorſtellen, und die Phantaſie ſpinnt weiter von dem Augenblick an, als man noch da
               war. \textsc{\textcolor{blue}{Hermann Bahr}{}\ledrightnote{\textcolor{blue}{Hermann Bahr}}}
               brachte mir den erſten \substVorne{}\textsuperscript{ſ\textcolor{gray}{a}}\substDazwischen{}ka\substHinten{}lten Wind von Draußen, Dein \textcolor{blue}{Bruder}{}\ledrightnote{→\textcolor{blue}{Julius Schnitzler}} (ohne es zu wiſſen und zu wollen) war der Zweite. Darum fürchte ich
               zunächſt ein Beiſammenſein mit Euch Allen. Ich habe Angſt, ich würde nur den Eindruck
               davon forttragen, \uline{daß ich nicht mehr da bin}. Ich
               fürchte, ich werde mich fremd aus Eurem Kreiſe zurückſpiegeln – nicht ganz fremd,
               gewiß, {\pb}aber doch im tiefſten Innern – und ich
               möchte nicht gern \introOben{}dieſes\introOben{} mein Geſpenſt ſehen. Bleibe ich
               fort, ſo ſagt mir immer noch die Illuſion, daß dies Alles nicht wahr iſt, und ich
               kann mich langſam \strikeout{et} entwöhnen. Dieſes Perſönliche
               verſchmilzt mit dem Materiellen: Es ſprießt da allerlei zukunftsvolles bei Euch in
                  \textcolor{pink}{Wien}{}\ledrightnote{\textcolor{pink}{Wien}} auf. Ich aber bin nicht dabei, bin in einer
               andern fernen Bahn, und Niemand mehr denkt an mich, ich gehöre nirgends mehr hin, zu
               keiner Gruppe, zu den Jungen nicht und nicht zu den Alten. Ich ſtehe ſo {\pb}in der zweiten Reihe und ſehe keine Ausſicht, in die
               erſte zu kommen. Ich könnte vielleicht mehr, als politiſche Correſpondenzen ſchreiben
               und hier und da ein Feuilleton – aber ich bringe nichts zuftande. Die Erfolge, die
               ich erziele, ſtehen in ſchreiendem Mißerhältniß zu dem \textsc{effort}, den ich aufwende. Du weißt, wie mich der Ehrgeiz verzehrt. \strikeout{Unſ} Und ſo fürchte ich bei dieſem Zuſammentreffen auch
               in dieſer Hinſicht allerlei Schmerzliches – unabſichtliche \textsc{Nuancen} natürlich, \strikeout{die} deren leiſe Berührung
               eben nur einer Seele wehthun \strikeout{kön} kann, wie der
               meinigen, der alle Haut abgeſchunden iſt, weil ſie ſich fortwährend an den harten {\pb}äußern Dingen reibt{\dotsfive}\pend
           \pstart
           Dies, mein lieber Freund, ſollſt Du leſen, ohne Zorn und ohne Spott – ſollſt darauf
               eingehen mit Deinem feinen Verſtändniß – und ſollſt mir dann in Kürze ſagen, \strikeout{\textcolor{gray}{×}\-\textcolor{gray}{×}\-\textcolor{gray}{×}\-\textcolor{gray}{×}\-\textcolor{gray}{×}} ob
                  \strikeout{ich} es räthlich für mich iſt zu kommen oder nicht.
               Das ſoll dann die Entſcheidung ſein{\dotsfour}\pend
           \pstart
           Von ganzem Herzen freut es mich, aus Deinen Zeilen eine gewiſſe Befriedigung
               herauszuleſen über das, \label{K_L02608-4v}\edtext{was Du jetzt
                  ſchreibſt}{\lemma{\textnormal{\emph{was Du jetzt
                  ſchreibſt}}}\Cendnote{\textnormal{nicht ermittelt}}}\label{K_L02608-4h}. Wenn
               wir uns treffen, ſo lieſt Du es mir {\pb}natürlich vor.
               Einſtweilen aber berglückwünſche ich Dich, daß Du die Arbeit ſoweit gefördert. Ich
               habe ſo eine unbeſtimmte Ahnung, daß ſie gelungen ſein muß. Denn ich ſehe aus
               Allerlei, daß Deine Kunſt jene Reife und Ruhe gewinnt, welche das Meiſterwerk
               ſchaffen helfen{\dotsfour}\pend
           \pstart
           Sei von Herzen und in Treue begrüßt, mein lieber Arthur!{\\[\baselineskip]} Dein{\\[\baselineskip]}\spacefill\mbox{Paul Goldmann}\pend
           \leftskip=0em{}\pstart
           \noindent{}Teufel, iſt das ein langer Brief!\pend
           \endnumbering\briefempfaengerindex{Schnitzler, Arthur@\textsc{Schnitzler, Arthur}!zzzGoldmann, Paul@\emph{von Paul Goldmann}!1894-07-291@{29. 7. {[}1894{]}}|)be}\mylabel{h}  \normalsize

\doendnotes{C}
\bigskip
\vfill

\clearpage

\footnotesize

\lohead{\textsc{register}}

% Definiere theindex-Environment komplett neu ohne reledmac
\makeatletter
\renewenvironment{theindex}{%
  \section*{\indexname}%
  \setlength{\parindent}{0pt}%
  \setlength{\parskip}{0pt plus 0.3pt}%
  \let\item\@idxitem
}{%
  \clearpage
}
\makeatother

\IfFileExists{\jobname-pw.ind}{\input{\jobname-pw.ind}}{}

\end{document}

      