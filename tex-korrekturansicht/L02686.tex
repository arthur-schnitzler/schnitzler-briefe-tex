%% latex-korrekturansicht-vorspann.tex
%% Vorspann für die Korrekturansicht.
%% Lädt die gemeinsame Datei latex-vorspann.tex mit gesetztem Schalter.

\newif\ifkorrekturansicht
\korrekturansichttrue

\input{../tex-inputs/latex-vorspann}


               \section[Arthur Schnitzler an Paul Goldmann, 22. 11. 1896]{ Arthur Schnitzler an Paul Goldmann, 22. 11. 1896}\nopagebreak\mylabel{v}\rehead{ }\normalsize\beginnumbering\briefempfaengerindex{Goldmann, Paul@\textsc{Goldmann, Paul}!zzzSchnitzler, Arthur@\emph{von Arthur Schnitzler}!1896-11-222@{22. 11. 1896}|(be} \toendnotes[C]{\smallbreak\pagebreak[2]} \Standort{DLA, A:Schnitzler, HS85.1.5681.}
\physDesc{Brief, 2 Blätter, 8 Seiten, Fotokopie, Fragment
\newline{}Handschrift: schwarze Tinte, deutsche Kurrent\newline{}Zusatz: Von den Korrespondenzstücken Schnitzlers an Goldmann fehlt
                                 weitgehend jede Spur. In der Edition von \textcolor{green}{Ritterlichkeit} (1975) schreibt
                                 die Herausgeberin \textcolor{blue}{Rena R.
                                    Schlein}: »Zwei Telegramme und ein Brief Schnitzlers
                                    an Goldmann wurden mir von Dr. \textcolor{blue}{Leo P. Reckford}, der diese Dokumente von der Familie
                                    Goldmanns zum Geschenk bekam, für meine Arbeit zur Verfügung
                                    gestellt« (S. 1). \textcolor{blue}{Reckford} starb 1988, seine Nachkommen haben
                                 keine Kenntnis von diesen (und etwaigen weiteren)
                                 Korrespondenzstücken und sie sind auch nicht auffindbar. \textcolor{blue}{Rena R. Schlein} wäre, wenn
                                 sie noch leben sollte, deutlich über 100 Jahre alt. Ein Kontakt
                                 konnte nicht hergestellt werden. Die vorliegende Kopie besteht aus
                                 einem Doppelblatt mit zwei Seiten, die links die vierte und rechts
                                 die erste Seite des ersten Blattes umfassen. Beim Erstellen der
                                 Kopie wurde der linke Rand der linken Seite nicht ordentlich
                                 aufgelegt und fehlt. Die Kopie dürfte durch \textcolor{blue}{Reckford} oder \textcolor{blue}{Schlein} in den Besitz \textcolor{blue}{Heinrich Schnitzlers} gelangt sein. \newline{}Editorischer Hinweis: Jene Teile des Briefes, die nicht im Fragment erhalten sind,
                                 werden mit Hilfe der Edition in \textcolor{green}{Ritterlichkeit} ergänzt. Die Verwendung des Schaft-s (»ſ«)
                                 wurde entsprechend den amtlichen Regeln auch auf die nicht
                                 erhaltenen Teile übertragen. }\buchAbdrucke{\weitereDrucke{\pwindex{Ritterlichkeit@\emph{Ritterlichkeit}|pwk}Arthur Schnitzler: \emph{Ritterlichkeit. Fragment aus dem Nachlaß}. Bonn: \emph{Bouvier Verlag Herbert Grundmann} 1975, S. 6–7 (Abhandlungen zur Kunst-, Musik- und
                        Literaturwissenschaft, 176).} }\toendnotes[C]{\smallbreak}\pstart
           \noindent{}{\pb}So feſt ich auch von dem glücklichen Ausgang
               überzeugt war, mein liebſter Paul – ich bin doch jetzt froher als geſtern um die
               Zeit. Noch vor Deinem Telegra{\geminationm} haben wir im Kaffehaus
               von einer Redaction {\pb}das Reſultat telephoniſch
               erfahren. Und nun ſage mir ſelbſt – iſt es nicht jämmerlich, daß Menschen wie Du
               ſolchen Möglichkeiten preisgegeben ſind – oder, wie ich faſt lieber ſagen möchte,
               preisgegeben zu ſein glauben? Ich habe von \textcolor{blue}{Leo}{}\ledrightnote{\textcolor{blue}{Leo Van-Jung}}
               manches gehört, ich habe auch Deine \label{K_L02686-1v}\edtext{\textcolor{green}{Artikel}{}\ledrightnote{→\textcolor{green}{Die Enthüllungen über die Affaire Dreyfus}{\newline}→\textcolor{green}{Die Affaire Dreyfus}{\newline}→\textcolor{green}{Dreyfus, die öffentliche Meinung und die deutsche Regierung}}}{\lemma{\textnormal{\emph{Artikel}}}\Cendnote{\textnormal{\emph{\textcolor{green}{Die Enthüllungen über die Affaire Dreyfus}},
                     Jg. 40, Nr. XXXX, 16. 9. 1896, S. XXXX. \emph{\textcolor{green}{Die Affaire Dreyfus}}, Jg. 40, Nr. XXXX,
                        11. 11. 1896, S. XXXX. \emph{\textcolor{green}{Dreyfus, die öffentliche Meinung und die
                        deutsche Regierung}}, Jg. 40, Nr. XXXX, 12. 11. 1896,
                     S. XXXX}}}\label{K_L02686-1h} in der \textcolor{green}{Fkt. Ztg.}{}\ledrightnote{\textcolor{green}{Frankfurter Zeitung}} alle geleſen – Du haſt
               Dich einfach prachtvoll benommen – auf \uline{Dein} Tun und
               Schreiben hin allein müßte das Verfahren gegen \textcolor{blue}{Dreyfus}{}\ledrightnote{\textcolor{blue}{Alfred Dreyfus}} neu aufgenommen werden.\pend
           \pstart
           Wenn in dieſer Sache ein Erfolg erzielt werden wird; Dir wird er zu danken ſein. Eine
               ſchönere Selbſtloſig{\pb}keit hat ſelten ein Mann in
               Deiner Lage bewieſen. Es iſt ebenſo edel als blödſinnig, daſs Du Dich geſchlagen haſt
               – wärſt Du aber erſchoſſen worden, ſo hätte die Ungeheuerlichkeit des Blödſinns alles
               andere verſchlungen. Es iſt vorbei – und ich hoffe, daſs Du keiner neuen Gefahr
               entgegen {\pb}gehſt. Ich wünſche dringend, daß Du Dich
               durch keinen Tropf mehr beleidigt fühlen mögeſt. Und wenn Du genötigt biſt, einen zu
               inſultieren, ſo wirſt Du jedenfalls genau wiſſen, warum Du es tuſt, wirſt alſo immer
               im Recht ſein und kannſt auf die lächerliche Fälſchung verzichten, welche durch einen
               Kugelwechſel in klare Tatſachen hineingetragen wird. Du haſt ja ſchließlich auch
               bewieſen – nachdem das nun einmal notwendig zu ſein ſcheint – daß Du »Mut« haſt; alſo
               auch von dieſer Seite kann man nicht mehr an Dich heran. –\pend
           \pstart
           Vielleicht haſt Du Zeit und Luſt, mir näheres mitzuteilen; Du begreifſt es, daß Deine
               Seelenzuſtände in den verſchiedenen Momenten mich auch aufs lebhafteſte
               intereſſieren, auch darüber ſage mir etwas. –\pend
           \pstart
           Auf Deinen lieben Brief von neulich antworte ich Dir dieſer Tage. Von mir iſt nur in
               Kürze zu melden, daß ich an den alten pſychiſchen Sachen in ſtörend hohem Maße
               leide. –\pend
           \pstart
           Leb wohl, mein lieber Paul, und nochmals tauſend Glückwünſche, tauſend
               Grüße!{\\[\baselineskip]}Dein treuer \spacefill\mbox{Arthur}\pend
           \leftskip=0em{}\pstart
           \textcolor{pink}{Wien}{}\ledrightnote{\textcolor{pink}{Wien}}{ }22. 11. 96.\pend
           \endnumbering\briefempfaengerindex{Goldmann, Paul@\textsc{Goldmann, Paul}!zzzSchnitzler, Arthur@\emph{von Arthur Schnitzler}!1896-11-222@{22. 11. 1896}|)be}\mylabel{h}\begin{anhang}\end{anhang}\normalsize

\doendnotes{C}
\bigskip
\vfill

\clearpage

\footnotesize

\lohead{\textsc{register}}

% Definiere theindex-Environment komplett neu ohne reledmac
\makeatletter
\renewenvironment{theindex}{%
  \section*{\indexname}%
  \setlength{\parindent}{0pt}%
  \setlength{\parskip}{0pt plus 0.3pt}%
  \let\item\@idxitem
}{%
  \clearpage
}
\makeatother

\IfFileExists{\jobname-pw.ind}{\input{\jobname-pw.ind}}{}

\end{document}

      