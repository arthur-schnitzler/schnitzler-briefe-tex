%% latex-korrekturansicht-vorspann.tex
%% Vorspann für die Korrekturansicht.
%% Lädt die gemeinsame Datei latex-vorspann.tex mit gesetztem Schalter.

\newif\ifkorrekturansicht
\korrekturansichttrue

\input{../tex-inputs/latex-vorspann}


\renewcommand{\erwaehntePersonen}{Personen: Hugo von Hofmannsthal}
\renewcommand{\erwaehnteInstitutionen}{Institutionen: Wiener Verlag}
\renewcommand{\erwaehnteOrte}{Orte: Bad Ischl, Wien}
\renewcommand{\erwaehnteWerke}{Werke: Begräbnis, Das Manhard-Zimmer, Der Hinterbliebene, Der Hinterbliebene. Kurze Novellen, Die Zeit. Wiener Wochenschrift, Fernen, Flucht, Heldentod, Lebenszeit, Mährisches Tagblatt, Sedan, Wiener Allgemeine Montags-Zeitung, Wiener Allgemeine Zeitung}
\section[ Felix Salten an Arthur Schnitzler, {[}29. 8. 1899{]}]{Felix Salten an Arthur Schnitzler, {[}29. 8. 1899{]}}
\nopagebreak\mylabel{v}
\rehead{ }\normalsize\beginnumbering\briefempfaengerindex{Schnitzler, Arthur@\textsc{Schnitzler, Arthur}!zzzSalten, Felix@\emph{von Felix Salten}!1899-08-291@{{[}29. 8. 1899{]}}|(be}
\toendnotes[C]{\smallbreak\pagebreak[2]}\Standort{CUL, Schnitzler, B 89, A 2.}
\physDesc{Brief, 1 Blatt, 2 Seiten, 783 Zeichen
\newline{}Handschrift: schwarze Tinte, lateinische Kurrent
\newline{}Schnitzler: mit Bleistift datiert: »29/8 9\textcolor{gray}{9}« 
\newline{}Ordnung: mit Bleistift von unbekannter Hand nummeriert: »123« }\toendnotes[C]{\smallbreak}
\pstart
           \raggedleft{}{\pb}Dienstag.\pend
           
\pstart
           Lieber, ich sende Ihnen gleichzeitig die \label{K_L03299-1v}\edtext{versprochenen \textcolor{green}{Zeitungen}{}\ledrightnote{{$\rightarrow$}\textcolor{green}{Wiener Allgemeine Montags-Zeitung}}}{\lemma{\textnormal{\emph{versprochenen Zeitungen}}}\Cendnote{\textnormal{Es dürfte sich um drei Novellen \textcolor{blue}{Salten}s handeln, die seit dem ersten Heft vom 3. 7. 1899 in der \emph{\textcolor{green}{Wiener Allgemeinen Montags-Zeitung}} erschienen waren, da \textcolor{blue}{Schnitzler} in seinem Antwortschreiben vom 4. 9. 1899 zwei davon
                  direkt anspricht: \emph{\textcolor{green}{Flucht}} (31. 7. 1899, S. 2–3 und 7. 8. 1899, S. 3–4) und \emph{\textcolor{green}{Das
                     Manhard-Zimmer}} (21. 8. 1899, S. 3–4). Zusätzlich war in
                  dem \textcolor{green}{Blatt}{ }\emph{\textcolor{green}{Sedan}} (3. 7. 1899, S. 2) erschienen.}}}\label{K_L03299-1h}, und
               bitte Sie, mir gelegentlich zu sagen, was Sie drüber denken, und wie Sie glauben,
               dass mans besser machen könnte. Haben Sie sich über die \label{K_L03299-2v}\edtext{Pneumatik}{\lemma{\textnormal{\emph{Pneumatik}}}\Cendnote{\textnormal{des
                  Fahrrads, mit dem \textcolor{blue}{Schnitzler} in dieser Zeit
                  viel unterwegs war}}}\label{K_L03299-2h} sehr geärgert? Ich habe mit der \textcolor{green}{Zeitung}{}\ledrightnote{{$\rightarrow$}\textcolor{green}{Wiener Allgemeine Montags-Zeitung}} sehr viel zu thun, arbeite aber
               gleichwol ziemlich viel. Ich denke ernsthaft daran, die \label{K_L03299-3v}\edtext{\textcolor{green}{Novellen}{}\ledrightnote{{$\rightarrow$}\textcolor{green}{Der Hinterbliebene. Kurze Novellen}} herauszugeben}{\lemma{\textnormal{\emph{Novellen herauszugeben}}}\Cendnote{\textnormal{Zu den bereits in der \emph{\textcolor{green}{Wiener Allgemeinen Montags-Zeitung}} erschienenen drei \textcolor{green}{Novellen} fügte \textcolor{blue}{Salten} fünf weitere
                  hinzu und vereinigte sie zum Novellenband \emph{\textcolor{green}{Der
                     Hinterbliebene. Kurze Novellen}}, der 1900 im \emph{\textcolor{brown}{Wiener Verlag}} erschien.}}}\label{K_L03299-3h}: \label{K_L03299-4v}\edtext{\textcolor{green}{Der Hinterbliebene}{}\ledrightnote{\textcolor{green}{Der Hinterbliebene}}}{\lemma{\textnormal{\emph{Der Hinterbliebene}}}\Cendnote{\textnormal{\emph{\textcolor{green}{Die Zeit}}, Bd. 18, Nr. 231, 4. 3. 1899 – Nr. 232, 11. 3. 1899.}}}\label{K_L03299-4h}, \textcolor{green}{Flucht}{}\ledrightnote{\textcolor{green}{Flucht}}, \label{K_L03299-5v}\edtext{\textcolor{green}{Begräbnis}{}\ledrightnote{\textcolor{green}{Begräbnis}}}{\lemma{\textnormal{\emph{Begräbnis}}}\Cendnote{\textnormal{\emph{\textcolor{green}{Wiener Allgemeine Montags-Zeitung}}, 6. 10. 1899; Erstdruck: \emph{\textcolor{green}{Mährisches Tagblatt}}, Jg. 14, Nr. 160,
                        17. 7. 1893, S. 1–2.}}}\label{K_L03299-5h}, \label{K_L03299-6v}\edtext{\textcolor{green}{Heldentod}{}\ledrightnote{\textcolor{green}{Heldentod}}}{\lemma{\textnormal{\emph{Heldentod}}}\Cendnote{\textnormal{\emph{\textcolor{green}{Wiener Allgemeine Zeitung}}, Nr. 5.044,
                        1. 1. 1895, Neujahrs-Beilage,
                  S. 3–4.}}}\label{K_L03299-6h}, \label{K_L03299-7v}\edtext{\textcolor{green}{Fernen}{}\ledrightnote{\textcolor{green}{Fernen}}}{\lemma{\textnormal{\emph{Fernen}}}\Cendnote{\textnormal{\emph{\textcolor{green}{Wiener Allgemeine Zeitung}}, Nr. 5.947,
                        25. 12. 1897, Weihnachts-Beilage,
                     S. [3–4].}}}\label{K_L03299-7h}, \textcolor{green}{Sedan}{}\ledrightnote{\textcolor{green}{Sedan}}, \label{K_L03299-8v}\edtext{\textcolor{green}{Lebenszeit}{}\ledrightnote{\textcolor{green}{Lebenszeit}}}{\lemma{\textnormal{\emph{Lebenszeit}}}\Cendnote{\textnormal{Erstdruck vor der \textcolor{green}{Buchausgabe} unbekannt}}}\label{K_L03299-8h}. Bitte,
               sagen Sie mir, was Sie davon halten, ob nämlich all diese Dinge nicht doch zu
               werthlos sind. (Nicht Affectation) Aber ich glaube, \uline{wenn} ich sie überhaupt als Buch erscheinen laße, dann will ichs jetzt thun,
               denn später, wenn Anderes fertig ist, {\pb}werde ichs gewiss nicht mehr
               wollen.\pend
           
\pstart
           Wann kommen Sie nach \label{K_L03299-9v}\edtext{\textcolor{pink}{Wien}{}\ledrightnote{\textcolor{pink}{Wien}}}{\lemma{\textnormal{\emph{Wien}}}\Cendnote{\textnormal{\textcolor{blue}{Schnitzler} kam erst am 12. 10. 1899 wieder
                  nach \textcolor{pink}{Wien} zurück.}}}\label{K_L03299-9h}?\pend
           
\pstart
           Herzlichst {\\[\baselineskip]}Ihr {\\[\baselineskip]}\spacefill\mbox{Salten}\pend
           \leftskip=0em{}
\pstart
           \noindent{}Grüßen Sie \label{K_L03299-10v}\edtext{\textcolor{blue}{Hugo}{}\ledrightnote{\textcolor{blue}{Hugo von Hofmannsthal}}}{\lemma{\textnormal{\emph{Hugo}}}\Cendnote{\textnormal{\textcolor{blue}{Hugo von Hofmannsthal} war am 22. 8. 1899 in \textcolor{pink}{Ischl} angekommen.}}}\label{K_L03299-10h}.\pend
           \endnumbering\briefempfaengerindex{Schnitzler, Arthur@\textsc{Schnitzler, Arthur}!zzzSalten, Felix@\emph{von Felix Salten}!1899-08-291@{{[}29. 8. 1899{]}}|)be}\mylabel{h}  \normalsize

\doendnotes{C}
\bigskip
\vfill

\clearpage

\footnotesize

\lohead{\textsc{register}}

% Definiere theindex-Environment komplett neu ohne reledmac
\makeatletter
\renewenvironment{theindex}{%
  \section*{\indexname}%
  \setlength{\parindent}{0pt}%
  \setlength{\parskip}{0pt plus 0.3pt}%
  \let\item\@idxitem
}{%
  \clearpage
}
\makeatother

\IfFileExists{\jobname-pw.ind}{\input{\jobname-pw.ind}}{}

\end{document}

      