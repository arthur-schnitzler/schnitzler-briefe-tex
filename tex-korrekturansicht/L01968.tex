%% latex-korrekturansicht-vorspann.tex
%% Vorspann für die Korrekturansicht.
%% Lädt die gemeinsame Datei latex-vorspann.tex mit gesetztem Schalter.

\newif\ifkorrekturansicht
\korrekturansichttrue

\input{../tex-inputs/latex-vorspann}


               \section[Hugo von Hofmannsthal an Arthur Schnitzler, 20. 10. {[}1910{]}]{ Hugo von Hofmannsthal an Arthur Schnitzler, 20. 10. {[}1910{]}}\nopagebreak\mylabel{v}\rehead{ }\normalsize\beginnumbering\briefempfaengerindex{Schnitzler, Arthur@\textsc{Schnitzler, Arthur}!zzzHofmannsthal, Hugo von@\emph{von Hugo von Hofmannsthal}!1910-10-202@{20. 10. 1910}|(be} \toendnotes[C]{\smallbreak\pagebreak[2]} \Standort{CUL, Schnitzler, B 43.}
\physDesc{Brief, 1 Blatt, 4 Seiten
\newline{}Handschrift: schwarze Tinte, deutsche Kurrent
\newline{}Schnitzler: mit Bleistift die Jahreszahl ergänzt: »910« und beschriftet: »\textsc{Hofmannsthal}« \newline{}Ordnung: 1) mit Bleistift von unbekannter Hand nummeriert: »\strikeout{318}« 2) mit Bleistift von unbekannter Hand nummeriert:
                                    »323«}\buchAbdrucke{\weitereDrucke{Hugo von Hofmannsthal, Arthur Schnitzler: \emph{Briefwechsel}. Hg. Therese Nickl und Heinrich Schnitzler. Frankfurt am Main: \emph{S. Fischer} 1964, S. 254.} }\toendnotes[C]{\smallbreak}\pstart
           \raggedleft{}{\pb}\textcolor{pink}{Rod.}{}\ledrightnote{\textcolor{pink}{Rodaun}}{ }20. X\pend
           \pstart
           mein guter Arthur, \hspace*{1.5em}vielmals danke ich Ihnen für Ihren Brief und Ihre
               Depeſche nach \textcolor{pink}{Neubeuern}{}\ledrightnote{\textcolor{pink}{Neubeuern}} (wo wir 2 unvergleichlich
               ſchöne und wirklich ſehr glückerfüllte \label{K_L01968_1v}\edtext{Herbſtwochen}{\lemma{\textnormal{\emph{Herbſtwochen}}}\Cendnote{\textnormal{vom 4. 10. 1910 bis zum
                     16. 10. 1910}}}\label{K_L01968_1h} zubrachten) für Ihre Hilfe in der
               Beſetzungsſache und vor allem für die schönen Stunden, die mir Ihr neues {\pb}\textcolor{green}{Stück}{}\ledrightnote{→\textcolor{green}{Das weite Land. Tragikomödie in fünf Akten}} geſchenkt hat. Ich glaube,
               dieſes »\textcolor{green}{weite Land}{}\ledrightnote{\textcolor{green}{Das weite Land. Tragikomödie in fünf Akten}}« iſt wirklich die allerbeſte
               Arbeit Ihrer an guten Arbeiten ſo reichen zweiten Lebens- oder Arbeitsperiode.\pend
           \pstart
           Das Stück gehört ſo ganz Ihnen, und iſt dabei ſo äußerſt kräftig, ſo wunderſchön
               zuſammengehalten. Alle Ihre nicht leicht in einem Athem aufzuzählenden Vorzüge: das
               ſo ganz perſönliche Lebensgefühl, die höchſt beſondere Scala der Wertungen, {\pb}die zarte und ſichere Geſtaltung,
               die leichte Hand für die Scenenführung, die Melancholie und der Witz, der höchſt
               nötige \textsc{bon sens}, normaler (aber ſeltener) Menſchenverſtand,
               und dazu das tiefere poetiſch-philoſophiſche Zuſammenſehen und Nebeneinanderſehen,
               die Güte, die Erfahrung und zugleich ein entzückender Mangel an Routine, ein
               Friſches, Blühendes, Geſpanntes überall – dies alles ko{\geminationm}t zuſammen, um ein {\pb}Werk
               herzuſtellen, das ſich in unvergleichlicher Weiſe im Gleichgewicht hält, weltlich und
               tief, theatermäßig und philoſophiſch, amüſant und bedeutend iſt.\hspace*{1.5em}Ich freue mich ſehr, es auch noch auf der Bühne zu ſehen – doch hab
               ich es auf der inneren Bühne tadellos beſetzt und ſehr ſchön mir aufgeführt.\pend
           \pstart
           Ko{\geminationm}en Sie vielleicht Samstag zur
               Generalprobe der \label{K_L01968_2v}\edtext{\textcolor{green}{Trauerfeier}{}\ledrightnote{→\textcolor{green}{Der Thor und der Tod}{\newline}→\textcolor{green}{Saul. Ein Tragödienfragment}}}{\lemma{\textnormal{\emph{Trauerfeier}}}\Cendnote{\textnormal{In Erinnerung an \textcolor{blue}{Josef Kainz} am \emph{\textcolor{brown}{Burgtheater}}.
                     \textcolor{blue}{Schnitzler} war sowohl am 22. 10. 1910 bei der
                  Generalprobe, als auch am 23. 10. 1910 bei der Veranstaltung.}}}\label{K_L01968_2h}? Das wäre mir ſehr lieb.
               Ich fahre dann noch für ein paar Tage \label{K_L01968_3v}\edtext{nach \textcolor{pink}{Grätz}{}\ledrightnote{\textcolor{pink}{Hradec nad Moravicí}}}{\lemma{\textnormal{\emph{nach Grätz}}}\Cendnote{\textnormal{vom 25. 10. 1910 bis zum
                     30. 10. 1910.}}}\label{K_L01968_3h} (zu \textcolor{blue}{Lichnowskys}{}\ledrightnote{\textcolor{blue}{Karl Max Lichnowsky}{\newline}\textcolor{blue}{Mechtilde Lichnowsky}}) dann bin ich ganz hier und leſe Euch die \textcolor{green}{Spieloper}{}\ledrightnote{→\textcolor{green}{Der Rosenkavalier}} bei Ihnen, ja?\pend
           \pstart Ihr \spacefill\mbox{Hugo}\pend{}\pstart
           \label{T_L01968_1v}\edtext{\textsc{P.S.} Hab in \textcolor{pink}{Neubeuern}{}\ledrightnote{\textcolor{pink}{Neubeuern}}
               die »\textcolor{green}{Weisſagung}{}\ledrightnote{\textcolor{green}{Die Weissagung}}« vorgeleſen. Sie lieſt ſich
                  wunderſchön.}{\lemma{\textnormal{\emph{P.S. … wunderſchön.}}}\Cendnote{\textnormal{quer am linken Rand der
                  dritten Seite}}}\label{T_L01968_1h}\pend
           \endnumbering\briefempfaengerindex{Schnitzler, Arthur@\textsc{Schnitzler, Arthur}!zzzHofmannsthal, Hugo von@\emph{von Hugo von Hofmannsthal}!1910-10-202@{20. 10. 1910}|)be}\mylabel{h}  \normalsize

\doendnotes{C}
\bigskip
\vfill

\clearpage

\footnotesize

\lohead{\textsc{register}}

% Definiere theindex-Environment komplett neu ohne reledmac
\makeatletter
\renewenvironment{theindex}{%
  \section*{\indexname}%
  \setlength{\parindent}{0pt}%
  \setlength{\parskip}{0pt plus 0.3pt}%
  \let\item\@idxitem
}{%
  \clearpage
}
\makeatother

\IfFileExists{\jobname-pw.ind}{\input{\jobname-pw.ind}}{}

\end{document}

      