%% latex-korrekturansicht-vorspann.tex
%% Vorspann für die Korrekturansicht.
%% Lädt die gemeinsame Datei latex-vorspann.tex mit gesetztem Schalter.

\newif\ifkorrekturansicht
\korrekturansichttrue

\input{../tex-inputs/latex-vorspann}


               \section[Arthur Schnitzler an Richard Beer-Hofmann, 14. 1. 1903]{ Arthur Schnitzler an Richard Beer-Hofmann, 14. 1. 1903}\nopagebreak\mylabel{v}\rehead{ }\normalsize\beginnumbering\briefempfaengerindex{Beer-Hofmann, Richard@\textsc{Beer-Hofmann, Richard}!zzzSchnitzler, Arthur@\emph{von Arthur Schnitzler}!1903-01-141@{14. 1. 1903}|(be} \toendnotes[C]{\smallbreak\pagebreak[2]} \Standort{YCGL, MSS 31.}
\physDesc{Brief, 1 Blatt, 4 Seiten, Umschlag
\newline{}Handschrift: Bleistift, deutsche Kurrent\newline{}Versand: 1) Stempel: »\nobreak{}\oindex{Salzburg@\textbf{Salzburg}, \emph{Besiedelter Ort (A.BSO)}|pwk}Sal\textcolor{gray}{zb}urg, 14. 1. 03, 9–12V\nobreak{}«.  2) Stempel: »\nobreak{}\oindex{Rodaun@\textbf{Rodaun}, \emph{Teil eines besiedelten Ortes (A.BSOX)}|pwk}{\pb}Rodaun, 15. 1. 03, 6–7N\nobreak{}«. \newline{}Ordnung: mit Bleistift von unbekannter Hand datiert:
                                    »14. 1.« }\buchAbdrucke{\weitereDrucke{Arthur Schnitzler, Richard Beer-Hofmann: \emph{Briefwechsel 1891–1931}. Hg. Konstanze Fliedl. Wien, Zürich: \emph{Europaverlag} 1992, S. 159–160.} }\toendnotes[C]{\smallbreak}\pstart{}{\pb}Hr \textsc{Dr Richard
                     Beer-Hofmann}\pend{}\pstart{}\textcolor{pink}{\textsc{Rodaun}}{}\ledrightnote{\textcolor{pink}{Rodaun}}\pend{}\pstart{}\textsc{bei \textcolor{pink}{Liesing}{}\ledrightnote{\textcolor{pink}{XXIII., Liesing}}}\pend{}\pstart{}\textsc{b \textcolor{pink}{Wien}{}\ledrightnote{\textcolor{pink}{Wien}}}\pend{}\pstart{}\textsc{\textcolor{pink}{Liesinger Haupts 2}{}\ledrightnote{\textcolor{pink}{Liesingerstraße}}.}\pend{}{\bigskip}\pstart
           \raggedleft{}{\pb}\textcolor{pink}{\textsc{Salzburg}}{}\ledrightnote{\textcolor{pink}{Salzburg}}{ }14. 1. 903.{\\}\textcolor{pink}{\textsc{Oesterr. Hof}}{}\ledrightnote{\textcolor{pink}{Österreichischer Hof}}. –\pend
           \pstart
           lieber Richard, bei dem Badebeſitzer \textcolor{blue}{\textsc{Schaller}}{}\ledrightnote{\textcolor{blue}{Schaller}} in \textcolor{pink}{Rodaun}{}\ledrightnote{\textcolor{pink}{Rodaun}}, \textcolor{pink}{\textsc{Liesinger}straſſe}{}\ledrightnote{\textcolor{pink}{Liesingerstraße}}, wohnt ſeit einigen Tagen unſer
               Hund, \textsc{\label{K_L01265-1v}\edtext{Bern}{\lemma{\textnormal{\emph{Bern}}}\Cendnote{\textnormal{\textcolor{blue}{Schnitzler} besaß den Bernhardiner nur für
                     kurze Zeit, vermutlich ab dem 23. 3. 1902. Im Oktober wurde er in dem im gleichen
                     Monat eröffneten \textcolor{pink}{Tierschutzhaus} des \emph{\textcolor{brown}{Wiener Tierschutz-Vereins}} behandelt;
                        Mitte Dezember erneut. Nach der Absage \textcolor{blue}{Beer-Hofmann}s sagt im April auch \textcolor{blue}{Bahr} ab. (Hermann Bahr an Arthur Schnitzler, 4. 4. [1903]) Im selben Jahr finden sich noch drei Erwähnungen im \emph{\textcolor{green}{Tagebuch}} (23. 5. 1903, 18. 6. 1903 und 6. 8. 1903) Vgl. \emph{Briefe} II,118.}}}\label{K_L01265-1h}} genannt. Sie wiſſen dſs wir
               in \textcolor{pink}{Wien}{}\ledrightnote{\textcolor{pink}{Wien}} nichts mit ihm anfangen können, und daſs wir
               deshalb jedenfalls auf ſeinen fernern Beſitz verzichten {\pb}müſſen. Wenn Sie ihn daher (ſtatt des \label{K_L01265_1v}\edtext{Flirt}{\lemma{\textnormal{\emph{Flirt}}}\Cendnote{\textnormal{der über zehn Jahre alte Hund \textcolor{blue}{Beer-Hofmann}s}}}\label{K_L01265_1h} zu tragen) von mir annehmen wollen, ſo erweiſen Sie
               mir damit nur einen Gefallen. Überlegen Sie ſichs, denn Eile hat die Sache in keiner
               Weiſe. Das Thier wohnt in Ihrer Nähe, warten Sie, bis ihm wieder {\pb}die Haare gewachſen ſind, und fragen Sie ſich, ob Sie
               ſich mit ihm befreunden können. – Wär ich auf dem Land wie Sie, ich behielte ihn
               gern; unter den gegebenen Umſtänden aber wäre mir der Gedanke, daſs \textsc{Bern} in Ihren Beſitz übergeht, der freundlichſte. –\pend
           \pstart
           {\pb}Ich bin mit \textcolor{blue}{Olga}{}\ledrightnote{\textcolor{blue}{Olga Schnitzler}}
               ſeit vorgeſtern hier; – und freue mich, inmitten des beruhigenden Schneefalls und der
               winterlichen Stille, daſs ich mich wenigſtens zu diesem Entschluſſe aufraffen konnte.
               Bis Ende der Woche hoffen wir zu bleiben.\pend
           \pstart
           Seien Sie herzlichſt gegrüßt\pend
           \pstart
           Ihr{\\[\baselineskip]}\spacefill\mbox{A.}\pend
           \leftskip=0em{}\endnumbering\briefempfaengerindex{Beer-Hofmann, Richard@\textsc{Beer-Hofmann, Richard}!zzzSchnitzler, Arthur@\emph{von Arthur Schnitzler}!1903-01-141@{14. 1. 1903}|)be}\mylabel{h}  \normalsize

\doendnotes{C}
\bigskip
\vfill

\clearpage

\footnotesize

\lohead{\textsc{register}}

% Definiere theindex-Environment komplett neu ohne reledmac
\makeatletter
\renewenvironment{theindex}{%
  \section*{\indexname}%
  \setlength{\parindent}{0pt}%
  \setlength{\parskip}{0pt plus 0.3pt}%
  \let\item\@idxitem
}{%
  \clearpage
}
\makeatother

\IfFileExists{\jobname-pw.ind}{\input{\jobname-pw.ind}}{}

\end{document}

      