%% latex-korrekturansicht-vorspann.tex
%% Vorspann für die Korrekturansicht.
%% Lädt die gemeinsame Datei latex-vorspann.tex mit gesetztem Schalter.

\newif\ifkorrekturansicht
\korrekturansichttrue

\input{../tex-inputs/latex-vorspann}


\renewcommand{\erwaehntePersonen}{Personen: Felix Salten}
\renewcommand{\erwaehnteInstitutionen}{Institutionen: Wiener Literarische Anstalt}
\renewcommand{\erwaehnteOrte}{Orte: Leipzig, Wien}
\renewcommand{\erwaehnteWerke}{Werke: Schauen und Spielen. Studien zur Kritik des modernen Theaters, Schauen und Spielen. Studien zur Kritik des modernen Theaters. Erster Band. Ergebnisse Erlebnisse}
\section[ Felix Salten: Widmungsexemplar Schauen und Spielen für Arthur Schnitzler, 22. 9. 1921]{Felix Salten: Widmungsexemplar Schauen und Spielen für Arthur
               Schnitzler, 22. 9. 1921}
\nopagebreak\mylabel{v}
\rehead{ }\normalsize\beginnumbering\briefempfaengerindex{Schnitzler, Arthur@\textsc{Schnitzler, Arthur}!zzzSalten, Felix@\emph{von Felix Salten}!1921-09-221@{22. 9. 1921}|(be}
\toendnotes[C]{\smallbreak\pagebreak[2]}\Standort{DLA, G:Schnitzler, Arthur (Sammlung Heinrich Schnitzler).}
\physDesc{Widmung am Titelblatt, 70 Zeichen
\newline{}Handschrift: schwarze Tinte, lateinische Kurrent}
\pstart
           \noindent{}{\pb}Meinem lieben Arthur Schnitzler\pend
           
\pstart
           herzlichst {\\[\baselineskip]}\spacefill\mbox{Felix Salten}\pend
           \leftskip=0em{}
\pstart
           \textcolor{pink}{Wien}{}\ledrightnote{\textcolor{pink}{Wien}}, 22. 9. 21\pend
           {\bigskip}
\pstart
           \noindent{}\centering{}\textcolor{gray}{\textbf{\so{Felix Salten}}}\pend
           
\pstart
           \noindent{}\centering{}\textcolor{gray}{\textbf{\textsc{\textbf{\textcolor{green}{Schauen und Spielen}{}\ledrightnote{\textcolor{green}{Schauen und Spielen. Studien zur Kritik des modernen Theaters}}}}}}\pend
           
\pstart
           \noindent{}\centering{}\textcolor{gray}{\textbf{Erſter Band}}\pend
           
\pstart
           \noindent{}\centering{}\textcolor{gray}{\textbf{\textcolor{green}{\so{Ergebniſſe}{ }{\\}\so{Erlebniſſe}}{}\ledrightnote{\textcolor{green}{Schauen und Spielen. Studien zur Kritik des modernen Theaters. Erster Band. Ergebnisse Erlebnisse}}}}\pend
           {\bigskip}
\pstart
           \noindent{}\raggedleft{}\textcolor{gray}{\textbf{Bloße Vernunft, die ſich am Kunſtwerk}}\pend
           
\pstart
           \noindent{}\raggedleft{}\textcolor{gray}{\textbf{reibt, begeht allemal Unzucht.}}\pend
           {\bigskip}
\pstart
           \noindent{}\centering{}\textcolor{gray}{\textbf{\so{1921}}}\pend
           
\pstart
           \noindent{}\centering{}\textcolor{gray}{\textbf{\textcolor{pink}{Wien}{}\ledrightnote{\textcolor{pink}{Wien}} * \textcolor{brown}{WILA}{}\ledrightnote{\textcolor{brown}{Wiener Literarische Anstalt}} * \textcolor{pink}{Leipzig}{}\ledrightnote{\textcolor{pink}{Leipzig}}}}\pend
           \endnumbering\briefempfaengerindex{Schnitzler, Arthur@\textsc{Schnitzler, Arthur}!zzzSalten, Felix@\emph{von Felix Salten}!1921-09-221@{22. 9. 1921}|)be}\mylabel{h}  \normalsize

\doendnotes{C}
\bigskip
\vfill

\clearpage

\footnotesize

\lohead{\textsc{register}}

% Definiere theindex-Environment komplett neu ohne reledmac
\makeatletter
\renewenvironment{theindex}{%
  \section*{\indexname}%
  \setlength{\parindent}{0pt}%
  \setlength{\parskip}{0pt plus 0.3pt}%
  \let\item\@idxitem
}{%
  \clearpage
}
\makeatother

\IfFileExists{\jobname-pw.ind}{\input{\jobname-pw.ind}}{}

\end{document}

      