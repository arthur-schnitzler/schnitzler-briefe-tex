%% latex-korrekturansicht-vorspann.tex
%% Vorspann für die Korrekturansicht.
%% Lädt die gemeinsame Datei latex-vorspann.tex mit gesetztem Schalter.

\newif\ifkorrekturansicht
\korrekturansichttrue

\input{../tex-inputs/latex-vorspann}


               \section[Richard Beer-Hofmann an Arthur Schnitzler, 3. 9. 1904]{ Richard Beer-Hofmann an Arthur Schnitzler, 3. 9. 1904}\nopagebreak\mylabel{v}\rehead{ }\normalsize\beginnumbering\briefempfaengerindex{Schnitzler, Arthur@\textsc{Schnitzler, Arthur}!zzzBeer-Hofmann, Richard@\emph{von Richard Beer-Hofmann}!1904-09-031@{3. 9. 1904}|(be} \toendnotes[C]{\smallbreak\pagebreak[2]} \Standort{CUL, Schnitzler, B 8.}
\physDesc{Brief, 1 Blatt, 1 Seite
\newline{}Handschrift: schwarze Tinte, lateinische Kurrent\newline{}Ordnung: mit Bleistift von unbekannter Hand nummeriert:
                                    »186« }\buchAbdrucke{\weitereDrucke{Arthur Schnitzler, Richard Beer-Hofmann: \emph{Briefwechsel 1891–1931}. Hg. Konstanze Fliedl. Wien, Zürich: \emph{Europaverlag} 1992, S. 165.} }\toendnotes[C]{\smallbreak}\pstart
           \centering{}{\pb}\textcolor{pink}{Aussee}{}\ledrightnote{\textcolor{pink}{Bad Aussee}}{ }3/IX. 04{ }Morgens\pend
           \pstart
           Lieber Arthur! Ich bin seit gestern fertig, und habe nun nur noch
               mit Durchsicht zum Theil Reinschrift einzelner \textcolor{green}{Akte}{}\ledrightnote{→\textcolor{green}{Der Graf von Charolais. Ein Trauerspiel}} zu thun. Ich würde mich sehr freuen wenn Sie und Ihre
                  \textcolor{blue}{Frau}{}\ledrightnote{\textcolor{blue}{Olga Schnitzler}} herüberkämen. Aber, bitte, dann für einen
               ganzen Tag, und kommen Sie rechtzeitig früh; 7 ½ und 9\textsuperscript{h.} gehen Züge von \textcolor{pink}{Ischl}{}\ledrightnote{\textcolor{pink}{Bad Ischl}} ab. Jedenfalls ko{\geminationm}e ich nach \textcolor{pink}{Ischl}{}\ledrightnote{\textcolor{pink}{Bad Ischl}}
               hinüber. Theilen Sie mir auch mit welche Tage den \textcolor{pink}{Lueg}{}\ledrightnote{\textcolor{pink}{Lueg am Wolfgangsee}}ern (von \textcolor{pink}{Lueg}{}\ledrightnote{\textcolor{pink}{Lueg am Wolfgangsee}}) gewidmet sind. Es hätte
               nicht viel Sinn für mich gerade an einem solchen Tag zu ko{\geminationm}en.\pend
           \pstart
           Ich freue mich wieder einmal mit Ihnen beisammen sein zu können, und grüße Sie von
               ganzem Herzen.\pend
           \pstart \spacefill\mbox{Richard}\pend{}\endnumbering\briefempfaengerindex{Schnitzler, Arthur@\textsc{Schnitzler, Arthur}!zzzBeer-Hofmann, Richard@\emph{von Richard Beer-Hofmann}!1904-09-031@{3. 9. 1904}|)be}\mylabel{h}  \normalsize

\doendnotes{C}
\bigskip
\vfill

\clearpage

\footnotesize

\lohead{\textsc{register}}

% Definiere theindex-Environment komplett neu ohne reledmac
\makeatletter
\renewenvironment{theindex}{%
  \section*{\indexname}%
  \setlength{\parindent}{0pt}%
  \setlength{\parskip}{0pt plus 0.3pt}%
  \let\item\@idxitem
}{%
  \clearpage
}
\makeatother

\IfFileExists{\jobname-pw.ind}{\input{\jobname-pw.ind}}{}

\end{document}

      