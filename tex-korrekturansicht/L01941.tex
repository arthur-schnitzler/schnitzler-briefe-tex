%% latex-korrekturansicht-vorspann.tex
%% Vorspann für die Korrekturansicht.
%% Lädt die gemeinsame Datei latex-vorspann.tex mit gesetztem Schalter.

\newif\ifkorrekturansicht
\korrekturansichttrue

\input{../tex-inputs/latex-vorspann}


               \section[Arthur Schnitzler an Richard Beer-Hofmann, 29. 6. 1910]{ Arthur Schnitzler an Richard Beer-Hofmann, 29. 6. 1910}\nopagebreak\mylabel{v}\rehead{ }\normalsize\beginnumbering\briefempfaengerindex{Beer-Hofmann, Richard@\textsc{Beer-Hofmann, Richard}!zzzSchnitzler, Arthur@\emph{von Arthur Schnitzler}!1910-06-291@{29. 6. 1910}|(be} \toendnotes[C]{\smallbreak\pagebreak[2]} \Standort{YCGL, MSS 31.}
\physDesc{Kartenbrief
\newline{}Handschrift: Bleistift, deutsche Kurrent\newline{}Versand: 1) Stempel: »\nobreak{}29. VI. 10, 6\nobreak{}«.  2) Stempel: »\nobreak{}\oindex{Bad Ischl@\textbf{Bad Ischl}, \emph{Besiedelter Ort (A.BSO)}|pwk}Bad Ischl 1, \textcolor{gray}{3}0. \textcolor{gray}{VI.} 10\nobreak{}«. 3) Weil dem Postbediensteten in \textcolor{pink}{Ischl
                              }die Adresse nicht geläufig war, strich dieser mit Bleistift diese Ortsangabe
                                 durch und vermerkte: »\textsc{retour}« und »\textsc{wenden}« (zweiteres
                                 bezieht sich auf die auf der Rückseite angebrachte Absenderangabe)
                                 und das Korrespondenzstück ging wieder nach \textcolor{pink}{Wien}, von wo es neuerlich gesandt wurde und am
                                    6. 7. 1910 den Empfänger erreichte.}\buchAbdrucke{\weitereDrucke{Arthur Schnitzler, Richard Beer-Hofmann: \emph{Briefwechsel 1891–1931}. Hg. Konstanze Fliedl. Wien, Zürich: \emph{Europaverlag} 1992, S. 208.} }\pstart{}{\pb}Abſ.:\pend{}\pstart{}\textsc{Schnitzler}\textcolor{pink}{Wien}{}\ledrightnote{\textcolor{pink}{Wien}}\pend{}\pstart{}\textcolor{pink}{\textsc{XVIII Spoettelg. 7}}{}\ledrightnote{\textcolor{pink}{Edmund-Weiß-Gasse}}\pend{}{\bigskip}\pstart{}{\pb}\textsc{Herrn Dr. Richard Beer-Hofma{\geminationn}}\pend{}\pstart{}\textsc{\strikeout{\textcolor{pink}{Wien}{}\ledrightnote{\textcolor{pink}{Wien}}}}\pend{}\pstart{}\textsc{\textcolor{pink}{Ischl}{}\ledrightnote{\textcolor{pink}{Bad Ischl}}}\pend{}\pstart{}\textsc{\textcolor{pink}{Steinfeld Nr 6}{}\ledrightnote{\textcolor{pink}{Steinfeld}}}.\pend{}{\bigskip}\pstart
           \raggedleft{}{\pb}29. 6. 1910\pend
           \pstart{}lieber Richard,\pend\pstart
           würd es Ihnen Mühe machen, mir geſchwind eine Abſchrift von »\textcolor{green}{\textsc{Mirjams Wiegenlied}}{}\ledrightnote{\textcolor{green}{Schlaflied für Mirjam}}« zu ſenden, um das ich von \textcolor{blue}{Paul Marx}{}\ledrightnote{\textcolor{blue}{Paul Marx}}
               dringend gebeten wurde u das ich nicht beſitze?\pend
           \pstart
           Hoffe Sie wohl am Ort!{\\}Herzlichſt\pend
           \pstart Ihr \spacefill\mbox{A.}\pend{}\endnumbering\briefempfaengerindex{Beer-Hofmann, Richard@\textsc{Beer-Hofmann, Richard}!zzzSchnitzler, Arthur@\emph{von Arthur Schnitzler}!1910-06-291@{29. 6. 1910}|)be}\mylabel{h}  \normalsize

\doendnotes{C}
\bigskip
\vfill

\clearpage

\footnotesize

\lohead{\textsc{register}}

% Definiere theindex-Environment komplett neu ohne reledmac
\makeatletter
\renewenvironment{theindex}{%
  \section*{\indexname}%
  \setlength{\parindent}{0pt}%
  \setlength{\parskip}{0pt plus 0.3pt}%
  \let\item\@idxitem
}{%
  \clearpage
}
\makeatother

\IfFileExists{\jobname-pw.ind}{\input{\jobname-pw.ind}}{}

\end{document}

      