%% latex-korrekturansicht-vorspann.tex
%% Vorspann für die Korrekturansicht.
%% Lädt die gemeinsame Datei latex-vorspann.tex mit gesetztem Schalter.

\newif\ifkorrekturansicht
\korrekturansichttrue

\input{../tex-inputs/latex-vorspann}


               \section[Thomas Mann an Arthur Schnitzler, 20. 2. 1923]{ Thomas Mann an Arthur Schnitzler, 20. 2. 1923}\nopagebreak\mylabel{v}\rehead{ }\normalsize\beginnumbering\briefempfaengerindex{Schnitzler, Arthur@\textsc{Schnitzler, Arthur}!zzzMann, Thomas@\emph{von Thomas Mann}!1923-02-201@{20. 2. 1923}|(be} \toendnotes[C]{\smallbreak\pagebreak[2]} \Standort{CUL, Schnitzler, B 67.}
\physDesc{Brief, 1 Blatt, 3 Seiten
\newline{}Handschrift: schwarze Tinte, deutsche Kurrent
\newline{}Schnitzler: mit rotem Buntstift mehrere Unterstreichungen \newline{}Ordnung: mit Bleistift von \textcolor{blue}{Frieda Pollak} (?) mit dem Buchstaben »A« (Abgeschrieben/Abschrift) gekennzeichnet }\buchAbdrucke{\weitereDrucke{Hertha Krotkoff: \emph{Arthur Schnitzler – Thomas Mann: Briefe.} In: \emph{Modern Austrian Literature}, Jg. 7 (1974) Nr. 1/2, S. 20–21.} }\toendnotes[C]{\smallbreak}\pstart
           \noindent{}{\pb}\textcolor{gray}{\textbf{\textsc{Dr. Thomas Mann}}}\hfill \textcolor{gray}{\textbf{\textcolor{pink}{MÜNCHEN}{}\ledrightnote{\textcolor{pink}{München}}, DEN}}{ }20. II. 23.\pend
           \pstart
           \raggedleft{}\textcolor{gray}{\textbf{\textcolor{pink}{POSCHINGERSTR. 1}{}\ledrightnote{\textcolor{pink}{Poschingerstraße}}}}\pend
           \pstart{}Verehrter Herr Dr. Schnitzler!\pend\pstart
           Für Ihren liebenswürdigen Brief vom Dezember habe ich noch vielmals
                    zu danken. Die Abenteuer \textcolor{green}{\textcolor{blue}{Caſanovas}{}\ledrightnote{→\textcolor{blue}{Giacomo Girolamo Casanova}}}{}\ledrightnote{\textcolor{green}{Casanovas Heimfahrt}} in \textcolor{pink}{Amerika}{}\ledrightnote{\textcolor{pink}{Amerika}} haben mich ſehr amüſiert. Auf
                    die Frage, die Sie daran knüpfen, weiß ich nicht viel zu antworten, denn meine
                    Erfahrungen mit \textcolor{brown}{Kirpatrick + Brandt}{}\ledrightnote{\textcolor{brown}{Brandt {\kaufmannsund} Kirkpatrick}}{ }ſind beſchränkt. Vor Jahr und Tag wurde »\textcolor{green}{Buddenbrooks}{}\ledrightnote{\textcolor{green}{Buddenbrooks}}« nach \textcolor{pink}{Amerika}{}\ledrightnote{\textcolor{pink}{Amerika}} verkauft, das iſt alles. Die Bezahlung war nicht ſchlecht:
                    500 Dollars, wenn ich nicht irre. Aber das Riſiko iſt auch wohl groß, – obgleich
                    der Roman durch einen \textcolor{green}{Buddenbrook-Film}{}\ledrightnote{\textcolor{green}{Die Buddenbrooks}}
                    geſtützt werden wird, den zur Zeit eine \textcolor{pink}{Berlin}{}\ledrightnote{\textcolor{pink}{Berlin}}er Export-Firma mit meiner ſchamloſen Zuſtimmung {\pb}herzuſtellen im Begriffe iſt. Was
                    wollen Sie, – das ist der Krieg!\pend
           \pstart
           Auf Ihre freundlichen Worte über den republikaniſchen \textcolor{green}{Aufſatz}{}\ledrightnote{→\textcolor{green}{Von deutscher Republik. Gerhart Hauptmann zum sechzigsten Geburtstag}} bilde ich mir nicht wenig ein.
                    Seien Sie überzeugt, daß ich Ihre Skepſis in Hinſicht auf die Bedeutung
                    poſitiver Staatsformen vollkommen teile. Hier handelte es ſich für mich um eine
                    rein praktiſche Aktion, mit der ich in gewiſſen Grenzen \uline{genützt} zu haben glaube, denn der \textcolor{green}{Artikel}{}\ledrightnote{→\textcolor{green}{Von deutscher Republik. Gerhart Hauptmann zum sechzigsten Geburtstag}} iſt im Auslande viel excerpiert worden. Aber
                    freilich gegen die Thorheit der \textcolor{pink}{Franzoſen}{}\ledrightnote{\textcolor{pink}{Frankreich}} iſt
                    nicht aufzuko{\geminationm}en. Offenbar haben ſie es ſich in den
                    Kopf geſetzt, jedem das Konzept zu verderben, der verſucht, in \textcolor{pink}{Deutſchland}{}\ledrightnote{\textcolor{pink}{Deutschland}} zum Guten zu reden. Man verſichert, daß die
                        \textsc{Détails} von der \textcolor{pink}{Ruhr}{}\ledrightnote{\textcolor{pink}{Ruhrgebiet}} nicht nur nicht übertrieben ſind, ſon{\pb}dern ſogar noch hinter der Wahrheit
                    zurückbleiben. Der Ingrimm iſt fürchterlich, und man ſieht nicht ab, was einmal
                    daraus werden ſoll.\pend
           \pstart
           Ich wollte Sie um Folgendes bitten. \textcolor{blue}{Felix
                        Salten}{}\ledrightnote{\textcolor{blue}{Felix Salten}} hatte die große Freundlichkeit, mir ſein Buch »\textcolor{green}{Bambi}{}\ledrightnote{\textcolor{green}{Bambi}}« zu ſchicken, – und ich habe ſeine Adreſſe nicht.
                    Wollen Sie es gütigſt übernehmen, ihm in meinem Auftrage zu \uline{danken}? Ich finde dieſe Tier- und Waldgeſchichte reizend,
                    erquickend, voll von Humor und Natur. Sagen Sie ihm das!\pend
           \pstart
           Ich komme Ende März nach \textcolor{pink}{Wien}{}\ledrightnote{\textcolor{pink}{Wien}} (wahrſcheinlich
                    wieder mit meiner \textcolor{blue}{Frau}{}\ledrightnote{→\textcolor{blue}{Katia Mann}})
                    und will hoffen, daß Sie dann noch nicht im Norden ſind.\pend
           \pstart
           Ihr ergebener{\\[\baselineskip]}\spacefill\mbox{Thomas Mann.}\pend
           \leftskip=0em{}\endnumbering\briefempfaengerindex{Schnitzler, Arthur@\textsc{Schnitzler, Arthur}!zzzMann, Thomas@\emph{von Thomas Mann}!1923-02-201@{20. 2. 1923}|)be}\mylabel{h}  \normalsize

\doendnotes{C}
\bigskip
\vfill

\clearpage

\footnotesize

\lohead{\textsc{register}}

% Definiere theindex-Environment komplett neu ohne reledmac
\makeatletter
\renewenvironment{theindex}{%
  \section*{\indexname}%
  \setlength{\parindent}{0pt}%
  \setlength{\parskip}{0pt plus 0.3pt}%
  \let\item\@idxitem
}{%
  \clearpage
}
\makeatother

\IfFileExists{\jobname-pw.ind}{\input{\jobname-pw.ind}}{}

\end{document}

      