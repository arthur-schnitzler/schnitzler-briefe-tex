%% latex-korrekturansicht-vorspann.tex
%% Vorspann für die Korrekturansicht.
%% Lädt die gemeinsame Datei latex-vorspann.tex mit gesetztem Schalter.

\newif\ifkorrekturansicht
\korrekturansichttrue

\input{../tex-inputs/latex-vorspann}


\section[Arthur Schnitzler an Stefan Zweig, 5. 7. 1908]{L03800 Arthur Schnitzler an Stefan Zweig, 5. 7. 1908}
\nopagebreak\mylabel{L03800v}
\rehead{ }\normalsize\beginnumbering\briefempfaengerindex{Zweig, Stefan@\textsc{Zweig, Stefan}!zzzSchnitzler, Arthur@\emph{von Arthur Schnitzler}!1908-07-051@{5. 7. 1908}|(be}
\toendnotes[C]{\smallbreak\pagebreak[2]}\Standort{ISRJerusalemNational Library of IsraelARC. Ms. Var. 305 1 58 Stefan Zweig Collection}
\physDesc{1 Blatt, 2 Seiten, 217 Zeichen
\newline{}Handschrift: schwarze Tinte, deutsche Kurrent
\newline{}Versand: Stempel: »\nobreak{}5. 7. {[}1908{]}\nobreak{}«.  }\toendnotes[C]{\smallbreak}\pstart{}{\pb}\textsc{Dr. Stefan Zweig}\pend{}\pstart{}\textcolor{pink}{Wien VIII}\oindex{VIII., Josefstadt@\textbf{VIII., Josefstadt}|pw}{}\ledrightnote{\textcolor{pink}{VIII., Josefstadt}}\pend{}\pstart{}\textcolor{pink}{\textsc{Kochgasse 8}}\oindex{Kochgasse 8@\textbf{Kochgasse 8}|pw}{}\ledrightnote{\textcolor{pink}{Kochgasse 8}}\pend{}{\bigskip}
\pstart
           {\pb}\textcolor{gray}{\textbf{\textcolor{pink}{Santnerspitze}\oindex{Punta Santner@\textbf{Punta Santner}|pw}{}\ledrightnote{\textcolor{pink}{Punta Santner}}}}\hspace*{2.5em}\textcolor{gray}{\textbf{\textcolor{pink}{Burgstall}\oindex{Burgstall@\textbf{Burgstall}|pw}{}\ledrightnote{\textcolor{pink}{Burgstall}}}}\hfill \textcolor{gray}{\textbf{\textcolor{pink}{Jung-Schlern}\oindex{Piccolo Sciliar@\textbf{Piccolo Sciliar}|pw}{}\ledrightnote{\textcolor{pink}{Piccolo Sciliar}}}}\pend
           
\pstart
           \centering{}\textcolor{gray}{\textbf{\textcolor{pink}{Tirol}\oindex{Suedtirol@\textbf{Südtirol}|pw}{}\ledrightnote{\textcolor{pink}{Südtirol}}: \textcolor{pink}{Seis am Schlern}\oindex{Seis am Schlern@\textbf{Seis am Schlern}|pw}{}\ledrightnote{\textcolor{pink}{Seis am Schlern}}, 1000m. Nach dem \textcolor{green}{Aquarell}\pwindex{Tirol: Seis am Schlern@\emph{Tirol: Seis am Schlern}|pwv}{}\ledrightnote{{$\rightarrow$}\emph{\textcolor{green}{Tirol: Seis am Schlern}}} von \textcolor{blue}{F. A. C. M. Reisch}\pwindex{Reisch, Franz August Carl Maria @\textsc{Reisch, Franz August Carl Maria}|pw}{}\ledrightnote{\textcolor{blue}{Franz August Carl Maria Reisch}},
                     \textcolor{pink}{Meran}\oindex{Meran@\textbf{Meran}|pw}{}\ledrightnote{\textcolor{pink}{Meran}}.}}\pend
           \vspace{1em}
\pstart
           {\pb}5. 7. 08\pend
           \vspace{0.5em}
\pstart
           herzlichen Dank, lieber Herr Doktor, für das \label{K_L03800-1v}\edtext{\textcolor{green}{\textsc{\textcolor{blue}{Balzac}\pwindex{Balzac, Honore de @\textsc{Balzac, Honoré de}|pw}{}\ledrightnote{\textcolor{blue}{Honoré de Balzac}}}-Büchlein}\pwindex{Balzac. Sein Weltbild aus den Werken@\emph{Balzac. Sein Weltbild aus den Werken}|pwv}{}\ledrightnote{{$\rightarrow$}\emph{\textcolor{green}{Balzac. Sein Weltbild aus den Werken}}}}{\lemma{\textnormal{\emph{Balzac-Büchlein}}}\Cendnote{\textnormal{\textcolor{blue}{Stefan Zweig}\pwindex{Zweig, Stefan @\textsc{Zweig, Stefan}|pwk}:
                        \emph{\textcolor{green}{Balzac. Sein Weltbild aus den Werken}\pwindex{Balzac. Sein Weltbild aus den Werken@\emph{Balzac. Sein Weltbild aus den Werken}|pwk}}.
                     Stuttgart: \emph{\textcolor{brown}{Verlag von
                        Robert Lutz}XXXX ORGangabe fehlt}{ }{[}1908{]} (\emph{\textcolor{green}{Aus
                        der Gedankenwelt großer Geister. Eine Sammlung von Auswahlbänden}\pwindex{Aus der Gedankenwelt grosser Geister. Eine Sammlung von Auswahlbaenden@\emph{Aus der Gedankenwelt großer Geister. Eine Sammlung von Auswahlbänden}|pwk}}.
                     Herausgegeben von \textcolor{blue}{Lothar
                        Brieger-Wasservogel}\pwindex{Brieger, Lothar @\textsc{Brieger, Lothar}|pwk}. Band 11: \textcolor{blue}{Balzac}\pwindex{Balzac, Honore de @\textsc{Balzac, Honoré de}|pwk}).}}}\label{K_L03800-1}; es begleitet mich in den Wald, wo ich geſtern
               Ihre ſchöne Vorrede mit Vergnügen geleſen habe.\pend
           
\pstart
           Ihr{\\[\baselineskip]}\spacefill\mbox{Arthur Schnitzler}\pend
           \leftskip=0em{}\selectlanguage{ngerman}\endnumbering\mylabel{L03800h}  \normalsize

\doendnotes{C}
\bigskip
\vfill

\clearpage

\footnotesize

\lohead{\textsc{register}}

% Definiere theindex-Environment komplett neu ohne reledmac
\makeatletter
\renewenvironment{theindex}{%
  \section*{\indexname}%
  \setlength{\parindent}{0pt}%
  \setlength{\parskip}{0pt plus 0.3pt}%
  \let\item\@idxitem
}{%
  \clearpage
}
\makeatother

\IfFileExists{\jobname-pw.ind}{\input{\jobname-pw.ind}}{}

\end{document}

      