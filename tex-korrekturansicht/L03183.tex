%% latex-korrekturansicht-vorspann.tex
%% Vorspann für die Korrekturansicht.
%% Lädt die gemeinsame Datei latex-vorspann.tex mit gesetztem Schalter.

\newif\ifkorrekturansicht
\korrekturansichttrue

\input{../tex-inputs/latex-vorspann}


\renewcommand{\erwaehntePersonen}{Personen: Richard Beer-Hofmann}
\renewcommand{\erwaehnteOrte}{Orte: Wien}
\renewcommand{\erwaehnteWerke}{}
\section[Felix Salten an Arthur Schnitzler, {[}5. 7. 1891{]}]{Felix Salten an Arthur Schnitzler, {[}5. 7. 1891{]}}
\nopagebreak\mylabel{v}
\rehead{ }\normalsize\beginnumbering\briefempfaengerindex{Schnitzler, Arthur@\textsc{Schnitzler, Arthur}!zzzSalten, Felix@\emph{von Felix Salten}!1891-07-052@{{[}5. 7. 1891{]}}|(be}
\toendnotes[C]{\smallbreak\pagebreak[2]}\Standort{CUL, Schnitzler, B 89, A 1.}
\physDesc{Brief, 1 Blatt, 1 Seite, 285 Zeichen
\newline{}Handschrift: schwarze Tinte, lateinische Kurrent
\newline{}Schnitzler: mit Bleistift datiert »5/7 91« 
\newline{}Ordnung: mit Bleistift von unbekannter Hand nummeriert: »4« }\toendnotes[C]{\smallbreak}
\pstart{}{\pb}lieber Freund!\pend
\pstart
           leider ist heute nicht auf mich zu zählen, da ich
               überhaupt keine Verständigung erzielen kann. Seien Sie mir nicht böse, bei mir ist so
               wie so: \label{K_L03183-1v}\edtext{diem perdidi}{\lemma{\textnormal{\emph{diem perdidi}}}\Cendnote{\textnormal{lateinisch: verlorener Tag}}}\label{K_L03183-1h}. \pend
           
\pstart
           Ich hoffe, dass Sie mit \label{K_L03183-2v}\edtext{\textcolor{blue}{Beer-Hoffmann}{}\ledrightnote{\textcolor{blue}{Richard Beer-Hofmann}}}{\lemma{\textnormal{\emph{Beer-Hoffmann}}}\Cendnote{\textnormal{Weder ein Treffen mit \textcolor{blue}{Beer-Hofmann} am 5. 7. 1891 noch eines mit \textcolor{blue}{Salten} am 6. 7. 1891 fand nachweislich statt.}}}\label{K_L03183-2h} beisa{\geminationm}en sein werden und hoffe, Sie morgen im Café zu
               treffen. Herzlich Ihr {\\}\spacefill\mbox{Salten}\pend
           \endnumbering\briefempfaengerindex{Schnitzler, Arthur@\textsc{Schnitzler, Arthur}!zzzSalten, Felix@\emph{von Felix Salten}!1891-07-052@{{[}5. 7. 1891{]}}|)be}\mylabel{h}  \normalsize

\doendnotes{C}
\bigskip
\vfill

\clearpage

\footnotesize

\lohead{\textsc{register}}

% Definiere theindex-Environment komplett neu ohne reledmac
\makeatletter
\renewenvironment{theindex}{%
  \section*{\indexname}%
  \setlength{\parindent}{0pt}%
  \setlength{\parskip}{0pt plus 0.3pt}%
  \let\item\@idxitem
}{%
  \clearpage
}
\makeatother

\IfFileExists{\jobname-pw.ind}{\input{\jobname-pw.ind}}{}

\end{document}

      