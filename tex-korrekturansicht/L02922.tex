%% latex-korrekturansicht-vorspann.tex
%% Vorspann für die Korrekturansicht.
%% Lädt die gemeinsame Datei latex-vorspann.tex mit gesetztem Schalter.

\newif\ifkorrekturansicht
\korrekturansichttrue

\input{../tex-inputs/latex-vorspann}


         
         \renewcommand{\erwaehntePersonen}{Personen: Richard Beer-Hofmann, Robert Hirschfeld, Leo Van-Jung}
         \renewcommand{\erwaehnteInstitutionen}{Institutionen: Houghton Library}
         \renewcommand{\erwaehnteOrte}{Orte: Alpen, Altaussee, Berlin, Dessauer Straße, Salzburg, Sekirn, Südtirol}
         \renewcommand{\erwaehnteWerke}{Werke: Tagebuch}
               \section[ Paul Goldmann an Arthur Schnitzler, 27. 6. {[}1900{]}]{Paul Goldmann an Arthur Schnitzler, 27. 6. {[}1900{]}}\nopagebreak\mylabel{v}\rehead{ }\normalsize\beginnumbering\briefempfaengerindex{Schnitzler, Arthur@\textsc{Schnitzler, Arthur}!zzzGoldmann, Paul@\emph{von Paul Goldmann}!1900-06-271@{27. 6. {[}1900{]}}|(be} \toendnotes[C]{\smallbreak\pagebreak[2]} \Standort{DLA, A:Schnitzler, HS.NZ85.1.3170.}
\physDesc{Brief, 1 Blatt, 3 Seiten
\newline{}Handschrift: blaue Tinte, deutsche Kurrent
\newline{}Schnitzler: 1) mit Bleistift das Jahr »{[}1{]}\textcolor{gray}{900}« vermerkt  2) mit rotem Buntstift zwei Unterstreichungen}\toendnotes[C]{\smallbreak}\pstart
           \noindent{}{\pb}\textcolor{pink}{\textcolor{gray}{\textbf{DESSAUERSTRASSE 19}}}{}\ledrightnote{\textcolor{pink}{Dessauer Straße}}\hfill \textcolor{pink}{Berlin}{}\ledrightnote{\textcolor{pink}{Berlin}}, 27. \substVorne{}\textsuperscript{\textcolor{gray}{×}\-\textcolor{gray}{×}}\substDazwischen{}Ju\substHinten{}ni.\pend
           \pstart\center{}Mein lieber Freund,\pend\pstart
           Für heut nur ein Wort bezüglich der \label{K_L02922-2v}\edtext{Sommerpläne}{\lemma{\textnormal{\emph{Sommerpläne}}}\Cendnote{\textnormal{siehe Paul Goldmann an Arthur Schnitzler, 16. 6. [1900]}}}\label{K_L02922-2h}. Ich möchte bald in’s Klare kommen, da ich mich nach verſchiedenen Seiten
               entſcheiden ſoll. Auf \textsc{\textcolor{blue}{Richard}{}\ledrightnote{\textcolor{blue}{Richard Beer-Hofmann}}} iſt alſo nicht zu rechnen. Überdies habe ich \strikeout{\textcolor{gray}{an}} ihm auch \label{K_L02922-1v}\edtext{direkt
                  geſchrieben}{\lemma{\textnormal{\emph{direkt
                  geſchrieben}}}\Cendnote{\textnormal{wohl der Brief vom 20. 6. {[}1900{]} (\emph{\textcolor{brown}{Houghton Library}}, Harvard (Signatur
                     825.978))}}}\label{K_L02922-1h}, und er antwortet mir nicht. Alſo nicht! Auf Dich
               ſcheint auch nicht {\pb}zu rechnen zu ſein. Bitte; gib’
               mir eine entſcheidende Antwort hierüber. In dieſem Falle würde ich einer Einladung
                  \textsc{\textcolor{blue}{Hirschfeld}{}\ledrightnote{\textcolor{blue}{Robert Hirschfeld}}s} nach \textsc{\textcolor{pink}{Seekirn}{}\ledrightnote{\textcolor{pink}{Sekirn}}} folgen und mit dieſem zuſammen \strikeout{\textcolor{gray}{w}an} eine Wanderung nach \textcolor{pink}{Südtirol}{}\ledrightnote{\textcolor{pink}{Südtirol}} machen, – wenn ich überhaupt fortkomme\strikeout{\textcolor{gray}{n}}, was noch immer zweifelhaft iſt.\pend
           \pstart
           {\pb}In welche\substVorne{}\textsuperscript{\textcolor{gray}{r}}\substDazwischen{}m\substHinten{} Orte wird \label{K_L02922-5v}\edtext{\textsc{\textcolor{blue}{Leo}{}\ledrightnote{\textcolor{blue}{Leo Van-Jung}}} im Auguſt}{\lemma{\textnormal{\emph{Leo im Auguſt}}}\Cendnote{\textnormal{Vor der gemeinsamen \textcolor{pink}{Alpen}wanderung hielt sich \textcolor{blue}{Leo Van-Jung}, wie \textcolor{blue}{Schnitzler}s \emph{\textcolor{green}{Tagebuch}} zu entnehmen ist, in \textcolor{pink}{Salzburg} auf.}}}\label{K_L02922-5h} ſtecken?\pend
           \pstart
           Viele treue Grüße! {\\[\baselineskip]}Dein {\\[\baselineskip]}\spacefill\mbox{Paul Goldmnn}\pend
           \leftskip=0em{}\endnumbering\briefempfaengerindex{Schnitzler, Arthur@\textsc{Schnitzler, Arthur}!zzzGoldmann, Paul@\emph{von Paul Goldmann}!1900-06-271@{27. 6. {[}1900{]}}|)be}\mylabel{h}  \normalsize

\doendnotes{C}
\bigskip
\vfill

\clearpage

\footnotesize

\lohead{\textsc{register}}

% Definiere theindex-Environment komplett neu ohne reledmac
\makeatletter
\renewenvironment{theindex}{%
  \section*{\indexname}%
  \setlength{\parindent}{0pt}%
  \setlength{\parskip}{0pt plus 0.3pt}%
  \let\item\@idxitem
}{%
  \clearpage
}
\makeatother

\IfFileExists{\jobname-pw.ind}{\input{\jobname-pw.ind}}{}

\end{document}

      