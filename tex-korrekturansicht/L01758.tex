%% latex-korrekturansicht-vorspann.tex
%% Vorspann für die Korrekturansicht.
%% Lädt die gemeinsame Datei latex-vorspann.tex mit gesetztem Schalter.

\newif\ifkorrekturansicht
\korrekturansichttrue

\input{../tex-inputs/latex-vorspann}


               \section[Arthur Schnitzler an Hugo von Hofmannsthal, 25. 1. 1908]{ Arthur Schnitzler an Hugo von Hofmannsthal, 25. 1. 1908}\nopagebreak\mylabel{v}\rehead{ }\normalsize\beginnumbering\briefempfaengerindex{Hofmannsthal, Hugo von@\textsc{Hofmannsthal, Hugo von}!zzzSchnitzler, Arthur@\emph{von Arthur Schnitzler}!1908-01-251@{25. 1. 1908}|(be} \toendnotes[C]{\smallbreak\pagebreak[2]} \Standort{FDH, Hs-30885,131.}
\physDesc{Brief, 1 Blatt, 4 Seiten
\newline{}Handschrift: blaue Tinte, deutsche Kurrent}\buchAbdrucke{\weitereDrucke{Hugo von Hofmannsthal, Arthur Schnitzler: \emph{Briefwechsel}. Hg. Therese Nickl und Heinrich Schnitzler. Frankfurt am Main: \emph{S. Fischer} 1964, S. 235–236.} }\toendnotes[C]{\smallbreak}\pstart
           \raggedleft{}{\pb}25. 1 908\pend
           \pstart{}mein lieber Hugo,\pend\pstart
           die Verhältniſſe nähern ſich ſehr allmälig dem \label{K_L01758_1v}\edtext{\textsc{soi disant}}{\lemma{\textnormal{\emph{soi disant}}}\Cendnote{\textnormal{französisch:
                  sogenannt.}}}\label{K_L01758_1h} Normalen. Die Wohnung iſt desinfizirt, \textcolor{blue}{Olga}{}\ledrightnote{\textcolor{blue}{Olga Schnitzler}}{ }ſchon viel außer Bett; \textcolor{blue}{Heini}{}\ledrightnote{\textcolor{blue}{Heinrich Schnitzler}} noch nicht zu Haus; aber ich treffe ihn zuweilen. –\pend
           \pstart
           In etwa 10 Tagen wollen wir auf den \textcolor{pink}{Se{\geminationm}ering}{}\ledrightnote{\textcolor{pink}{Semmering}} (jetzt, heißt es, iſt Influenza oben) und
               etwa 8 Tage oder länger, mit \textcolor{blue}{Heini}{}\ledrightnote{\textcolor{blue}{Heinrich Schnitzler}} oben bleiben –
                  da{\geminationn} erſt oeffnen ſich wieder unſeres Hauſes
               Pforten.\pend
           \pstart
           {\pb}Vielleicht ſieht man ſich vorher ſchon in neutralem
               Gebiet –? Ich möchte gern näheres über Sie, von Ihnen wiſſen, von andern, ſelbſt we{\geminationn} die andern \textcolor{blue}{Richard}{}\ledrightnote{\textcolor{blue}{Richard Beer-Hofmann}}s ſind, erfährt man doch nicht genug.\pend
           \pstart
           Mit edler Geste ſchuppſen Sie mir den \textcolor{brown}{Grillparzerpreis}{}\ledrightnote{\textcolor{brown}{Franz-Grillparzer-Preis}} wieder zurück – i{\geminationm}erhin bin
               ich froh, daſs ich ihn direct bekommen hab – es vereinfacht die Einkaſſirung. Mit
                  »\textcolor{blue}{\textsc{Interviewern}}{}\ledrightnote{→\textcolor{blue}{Karl Werkmann}}« ſoll man natürlich nie ſprechen (we{\geminationn} man ihnen
               nicht dictirt, wie es andere thun) {\pb}ja man ſoll ſie nicht
               empfangen, was aber ſchwer iſt, wenn ſie hinter einem Stubenmädl die öffnet, direct
               ins Zimmer ſtürzen, ohne Meldung abzuwarten, – oder man ſoll ſie hinauswerfen – was
               auch wieder ſchwer iſt, we{\geminationn} man nicht weiſs, wer ſie
               ſind und ſie plötzlich aus heiterm oder vielmehr bewölktem Himmel einem Glückwünſche
               zu unvermutet erſchienenen fünftauſend Kronen (nebſt Ehre, Auszeichnung u Lorbeer) zu
               Füßen legen. Übrigens werd ich Ihnen {\pb}nächſtens noch
               etwas Komiſches vom Vormittag des 15. Januar erzählen.\pend
           \pstart
           Zur Arbeit fühl ich mich ſchon ſehr bereit; an Tagen, da man innerlich u äußerlich
               allerlei ordnen konnte, un\textcolor{gray}{d} ſelbſt an Einfällen hat es \strikeout{mir} nicht gefehlt.\pend
           \pstart
           Wie gehts Ihnen Allen? \textcolor{blue}{Olga}{}\ledrightnote{\textcolor{blue}{Olga Schnitzler}} iſt über die
               prachtvolle Schale ſehr froh. Ich hab sie \introOben{}ihr\introOben{} erſt im
               desinfizirten Raum übergeben.\pend
           \pstart
           Wir grüßen Euch! Laßt was hören!{\\[\baselineskip]}\spacefill\mbox{Arthur }\pend
           \leftskip=0em{}\endnumbering\briefempfaengerindex{Hofmannsthal, Hugo von@\textsc{Hofmannsthal, Hugo von}!zzzSchnitzler, Arthur@\emph{von Arthur Schnitzler}!1908-01-251@{25. 1. 1908}|)be}\mylabel{h}  \normalsize

\doendnotes{C}
\bigskip
\vfill

\clearpage

\footnotesize

\lohead{\textsc{register}}

% Definiere theindex-Environment komplett neu ohne reledmac
\makeatletter
\renewenvironment{theindex}{%
  \section*{\indexname}%
  \setlength{\parindent}{0pt}%
  \setlength{\parskip}{0pt plus 0.3pt}%
  \let\item\@idxitem
}{%
  \clearpage
}
\makeatother

\IfFileExists{\jobname-pw.ind}{\input{\jobname-pw.ind}}{}

\end{document}

      