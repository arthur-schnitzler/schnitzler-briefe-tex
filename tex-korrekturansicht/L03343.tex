%% latex-korrekturansicht-vorspann.tex
%% Vorspann für die Korrekturansicht.
%% Lädt die gemeinsame Datei latex-vorspann.tex mit gesetztem Schalter.

\newif\ifkorrekturansicht
\korrekturansichttrue

\input{../tex-inputs/latex-vorspann}


\renewcommand{\erwaehntePersonen}{Personen: Heinrich Kanner, Ottilie Salten, Olga Schnitzler, Heinrich Schnitzler, Isidor Singer}
\renewcommand{\erwaehnteInstitutionen}{Institutionen: Die Zeit}
\renewcommand{\erwaehnteOrte}{Orte: Edmund-Weiß-Gasse 7, Frankgasse 1, Wien, Wipplingerstraße, XVIII., Währing}
\renewcommand{\erwaehnteWerke}{}
\section[ Felix Salten an Arthur Schnitzler, 17. 9. 1903]{Felix Salten an Arthur Schnitzler, 17. 9. 1903}
\nopagebreak\mylabel{v}
\rehead{ }\normalsize\beginnumbering\briefempfaengerindex{Schnitzler, Arthur@\textsc{Schnitzler, Arthur}!zzzSalten, Felix@\emph{von Felix Salten}!1903-09-172@{17. 9. 1903}|(be}
\toendnotes[C]{\smallbreak\pagebreak[2]}\Standort{CUL, Schnitzler, B 89, A 2.}
\physDesc{Brief, 1 Blatt, 1 Seite, 408 Zeichen
\newline{}Handschrift: schwarze Tinte, lateinische Kurrent
\newline{}Ordnung: mit Bleistift von unbekannter Hand nummeriert: »168« }\toendnotes[C]{\smallbreak}
\pstart
           \noindent{}{\pb}\textcolor{gray}{\textbf{DIE}}\pend
           
\pstart
           \textcolor{gray}{\textbf{\textcolor{brown}{ZEIT}{}\ledrightnote{\textcolor{brown}{Die Zeit}}}}\hfill \textcolor{gray}{\textbf{\textcolor{pink}{\emph{WIEN}}{}\ledrightnote{\textcolor{pink}{Wien}}}}{ }17. IX. 03\pend
           
\pstart
           \textcolor{gray}{\textbf{\textcolor{pink}{Wien}{}\ledrightnote{\textcolor{pink}{Wien}}er Tageszeitung}}\hfill \textcolor{gray}{\textbf{\emph{\textcolor{pink}{I. Wipplingerstrasse 38}{}\ledrightnote{\textcolor{pink}{Wipplingerstraße}}}}}\pend
           
\pstart
           \textcolor{gray}{\textbf{Herausgeber:}}\pend
           
\pstart
           \textcolor{gray}{\textbf{\textbf{Prof. Dr. \textcolor{blue}{I. Singer}{}\ledrightnote{\textcolor{blue}{Isidor Singer}}}}}\pend
           
\pstart
           \textcolor{gray}{\textbf{\textbf{Dr. \textcolor{blue}{Heinrich Kanner}{}\ledrightnote{\textcolor{blue}{Heinrich Kanner}}}}}\pend
           
\pstart
           \textcolor{gray}{\textbf{\textbf{Redaction}}}\pend
           
\pstart
           \textcolor{gray}{\textbf{Telegramm-Adresse: \textcolor{brown}{\so{Zeit}}{}\ledrightnote{\textcolor{brown}{Die Zeit}}\so{,}{ }\textcolor{pink}{\so{Wien}}{}\ledrightnote{\textcolor{pink}{Wien}}}}\pend
           
\pstart
           \textcolor{gray}{\textbf{Interurbanes Telephon Nr. 15.988}}\pend
           
\pstart
           \textcolor{gray}{\textbf{= Telephone Nr. 17.040, 17.041 =}}\pend
           
\pstart
           Lieber, ich weiß nicht, ob Sie \label{K_L03343-1v}\edtext{noch, oder wieder in \textcolor{pink}{Wien}{}\ledrightnote{\textcolor{pink}{Wien}}}{\lemma{\textnormal{\emph{noch, … Wien}}}\Cendnote{\textnormal{\textcolor{blue}{Schnitzler} war in \textcolor{pink}{Wien}.}}}\label{K_L03343-1h} sind, und wundere mich natürlich, nichts von
               Ihnen zu hören. \textcolor{blue}{Otti}{}\ledrightnote{\textcolor{blue}{Ottilie Salten}} ist noch immer nicht ganz
               wol und erholt sich nur langsam.\pend
           
\pstart
           Wenn Sie da sind, möchte ich Sie bald, in \label{K_L03343-2v}\edtext{einer, die »\textcolor{brown}{Zeit}{}\ledrightnote{\textcolor{brown}{Die Zeit}}«
               betreffd. Sache}{\lemma{\textnormal{\emph{einer, … Sache}}}\Cendnote{\textnormal{siehe Felix Salten an Arthur Schnitzler, 19. 9. [1903]}}}\label{K_L03343-2h} sprechen. Mit den schönsten Grüßen von uns \textcolor{blue}{Beiden}{}\ledrightnote{{$\rightarrow$}\textcolor{blue}{Ottilie Salten}} an \textcolor{blue}{Olga}{}\ledrightnote{\textcolor{blue}{Olga Schnitzler}}\pend
           
\pstart
           herzlich {\\[\baselineskip]}Ihr {\\[\baselineskip]}\spacefill\mbox{Salten}\pend
           \leftskip=0em{}
\pstart
           \noindent{}Ich weiß auch Ihre \label{K_L03343-3v}\edtext{neue Adreße}{\lemma{\textnormal{\emph{neue Adreße}}}\Cendnote{\textnormal{Am 2. 9. 1903 waren \textcolor{blue}{Olga Schnitzler} und der Sohn \textcolor{blue}{Heinrich} in die erste gemeinsame Wohnung in einem neu
                     errichteten Haus in der \textcolor{pink}{Spoettelgasse 7}
                     (heute: \textcolor{pink}{Edmund-Weiß-Gasse}) im \textcolor{pink}{18. Wiener Gemeindebezirk} gezogen; am 9. 9. 1903 war \textcolor{blue}{Schnitzler} nachgefolgt.}}}\label{K_L03343-3h} nicht, {\kaufmannsund} sende den Brief deshalb in die \textcolor{pink}{Franckgaße}{}\ledrightnote{\textcolor{pink}{Frankgasse 1}}.\pend
           \endnumbering\briefempfaengerindex{Schnitzler, Arthur@\textsc{Schnitzler, Arthur}!zzzSalten, Felix@\emph{von Felix Salten}!1903-09-172@{17. 9. 1903}|)be}\mylabel{h}  \normalsize

\doendnotes{C}
\bigskip
\vfill

\clearpage

\footnotesize

\lohead{\textsc{register}}

% Definiere theindex-Environment komplett neu ohne reledmac
\makeatletter
\renewenvironment{theindex}{%
  \section*{\indexname}%
  \setlength{\parindent}{0pt}%
  \setlength{\parskip}{0pt plus 0.3pt}%
  \let\item\@idxitem
}{%
  \clearpage
}
\makeatother

\IfFileExists{\jobname-pw.ind}{\input{\jobname-pw.ind}}{}

\end{document}

      