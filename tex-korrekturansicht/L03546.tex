%% latex-korrekturansicht-vorspann.tex
%% Vorspann für die Korrekturansicht.
%% Lädt die gemeinsame Datei latex-vorspann.tex mit gesetztem Schalter.

\newif\ifkorrekturansicht
\korrekturansichttrue

\input{../tex-inputs/latex-vorspann}


\renewcommand{\erwaehntePersonen}{Personen: Richard Rosenbaum, Felix Salten}
\renewcommand{\erwaehnteOrte}{Orte: Wien}
\renewcommand{\erwaehnteWerke}{Werke: Der junge Medardus. Dramatische Historie in einem Vorspiel und fünf Aufzügen}
\section[ Felix Salten an Arthur Schnitzler, {[}22. 11. 1910?{]}]{Felix Salten an Arthur Schnitzler, {[}22. 11. 1910?{]}}
\nopagebreak\mylabel{v}
\rehead{ }\normalsize\beginnumbering\briefempfaengerindex{Schnitzler, Arthur@\textsc{Schnitzler, Arthur}!zzzSalten, Felix@\emph{von Felix Salten}!1910-11-224@{{[}22. 11. 1910?{]}}|(be}
\toendnotes[C]{\smallbreak\pagebreak[2]}\Standort{CUL, Schnitzler, B 89, B 2.}
\physDesc{Briefkarte, 167 Zeichen
\newline{}Handschrift: schwarze Tinte, lateinische Kurrent
\newline{}Ordnung: mit Bleistift von unbekannter Hand nummeriert: »261?« }\toendnotes[C]{\smallbreak}
\pstart
           \noindent{}{\pb}\textcolor{gray}{\textbf{\textsc{Felix Salten}}}\pend
           
\pstart{}Lieber,\pend
\pstart
           darf ich Sie fragen, wann \label{K_L03546-1v}\edtext{morgen die \textcolor{green}{Generalprobe}{}\ledrightnote{{$\rightarrow$}\textcolor{green}{Der junge Medardus. Dramatische Historie in einem Vorspiel und fünf Aufzügen}}}{\lemma{\textnormal{\emph{morgen die Generalprobe}}}\Cendnote{\textnormal{Die Karte ist undatiert. Der gedruckte Briefkopf entspricht der im
                  Korrespondenzstück vom Felix Salten an Arthur Schnitzler, 14. 10. 1910 erstmals nachgewiesenen Gestalt. Das
                  wiederum kann als Indiz genommen werden, dass die im Nachlass überlieferte Einordung unter die Korrespondenzstücke
                  des Jahres 1910 zutrifft. Folglich dürfte es sich um die
                  Generalprobe zur Uraufführung von \emph{\textcolor{green}{Der junge
                     Medardus}} gehandelt haben und die Karte auf den Vortag der Generalprobe zu datieren sein. Diese fand
                  am 23. 11. 1910
                  statt.}}}\label{K_L03546-1h} beginnt? D\textsuperscript{r} \textcolor{blue}{Rosenbaum}{}\ledrightnote{\textcolor{blue}{Richard Rosenbaum}} hat versprochen, mich zu benachrichtigen, läßt aber
               nichts von sich hören.\pend
           
\pstart
           Herzlichst {\\[\baselineskip]}Ihr {\\[\baselineskip]}\spacefill\mbox{Salten}\pend
           \leftskip=0em{}\endnumbering\briefempfaengerindex{Schnitzler, Arthur@\textsc{Schnitzler, Arthur}!zzzSalten, Felix@\emph{von Felix Salten}!1910-11-224@{{[}22. 11. 1910?{]}}|)be}\mylabel{h}  \normalsize

\doendnotes{C}
\bigskip
\vfill

\clearpage

\footnotesize

\lohead{\textsc{register}}

% Definiere theindex-Environment komplett neu ohne reledmac
\makeatletter
\renewenvironment{theindex}{%
  \section*{\indexname}%
  \setlength{\parindent}{0pt}%
  \setlength{\parskip}{0pt plus 0.3pt}%
  \let\item\@idxitem
}{%
  \clearpage
}
\makeatother

\IfFileExists{\jobname-pw.ind}{\input{\jobname-pw.ind}}{}

\end{document}

      