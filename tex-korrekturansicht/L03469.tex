%% latex-korrekturansicht-vorspann.tex
%% Vorspann für die Korrekturansicht.
%% Lädt die gemeinsame Datei latex-vorspann.tex mit gesetztem Schalter.

\newif\ifkorrekturansicht
\korrekturansichttrue

\input{../tex-inputs/latex-vorspann}


\renewcommand{\erwaehntePersonen}{Personen: Margarethe Fränkel, Paul Goldmann, Eva Marie Goldmann, Olga Schnitzler, Hans Tietze, Erica Tietze-Conrat}
\renewcommand{\erwaehnteInstitutionen}{Institutionen: Hotel Sacher}
\renewcommand{\erwaehnteOrte}{Orte: Armbrustergasse, Edmund-Weiß-Gasse 7, Hohe Warte, Hotel Sacher, Sternwartestraße 71, Wien}
\renewcommand{\erwaehnteWerke}{}
\section[ Paul Goldmann und Eva Marie Goldmann an Arthur Schnitzler, 1. 10. 1909]{Paul Goldmann und Eva Marie Goldmann an Arthur
               Schnitzler, 1. 10. 1909}
\nopagebreak\mylabel{v}
\rehead{ }\normalsize\beginnumbering\briefempfaengerindex{Schnitzler, Arthur@\textsc{Schnitzler, Arthur}!zzzGoldmann, Eva Marie@\emph{von Eva Marie Goldmann}!1909-10-011@{1. 10. 1909}|(be}\briefempfaengerindex{Schnitzler, Arthur@\textsc{Schnitzler, Arthur}!zzzGoldmann, Paul@\emph{von Paul Goldmann}!1909-10-011@{1. 10. 1909}|(be}
\toendnotes[C]{\smallbreak\pagebreak[2]}\Standort{DLA, A:Schnitzler, HS.NZ85.1.3175.}
\physDesc{Kartenbrief, 587 Zeichen
\newline{}Handschrift Paul Goldmann: 1) schwarze Tinte, deutsche Kurrent\hspace{1em}2) schwarze Tinte, lateinische Kurrent (\noindent{}Adresse)\hspace{1em}
\newline{}Handschrift Eva Marie Goldmann: schwarze Tinte, lateinische Kurrent (\noindent{}Fußnote)
\newline{}Versand: Stempel: »\nobreak{}\textcolor{gray}{1/1}{ }\textcolor{gray}{Wi}{[}en{]}, 1. X. {[}09{]}, 2\nobreak{}«.  }\toendnotes[C]{\smallbreak}\pstart{}{\pb}Herrn\pend{}\pstart{}Dr. Arthur Schnitzler\pend{}\pstart{}\textcolor{pink}{Wien}{}\ledrightnote{\textcolor{pink}{Wien}}\pend{}\pstart{}\textcolor{pink}{XVIII. Spöttelgaſse 7}{}\ledrightnote{\textcolor{pink}{Edmund-Weiß-Gasse 7}}.\pend{}
{\bigskip}
\pstart
           1. 10. 09.\pend
           
\pstart
           Lieber Freund, Ich fahre heut{ }Mittag{ }\label{K_L03469-1v}\edtext{ab}{\lemma{\textnormal{\emph{ab}}}\Cendnote{\textnormal{aus \textcolor{pink}{Wien}, am 28. 9. 1909 hatte er
                     \textcolor{blue}{Schnitzler} noch besucht}}}\label{K_L03469-1h} u. will
               Dir nur raſch vorher mitteilen, daß meine \textcolor{blue}{Schwägerin}{}\ledrightnote{{$\rightarrow$}\textcolor{blue}{Margarethe Fränkel}}, Frl. \textsc{\textcolor{blue}{Fränkel}{}\ledrightnote{\textcolor{blue}{Margarethe Fränkel}}}, die im \textsc{\textcolor{pink}{Hotel Sacher}{}\ledrightnote{\textcolor{pink}{Hotel Sacher}}} wohnt, gern bereit iſt, Dich in das \textcolor{pink}{Haus}{}\ledrightnote{{$\rightarrow$}\textcolor{pink}{Armbrustergasse}} des \label{K_L03469-2v}\edtext{\textsc{Dr. \textcolor{blue}{Tietze}{}\ledrightnote{\textcolor{blue}{Hans Tietze}}}}{\lemma{\textnormal{\emph{Dr. Tietze}}}\Cendnote{\textnormal{Durch die Schwangerschaft mit dem
                  zweiten Kind würden die Wohnverhältnisse der Familie \textcolor{blue}{Schnitzler} in absehbarer Zeit zu beengt werden.
                  Deswegen fanden sie sich auf Wohnungs- bzw. Haussuche, die am 16. 7. 1910 in der
                  Übersiedelung in die \textcolor{pink}{Sternwartestraße 71}
                  mündete. Ob sie das \textcolor{pink}{Haus}
                  besichtigten, in dem \textcolor{blue}{Hans Tietze} mit seiner
                  Frau \textcolor{blue}{Erica Tietze-Conrat} wohnte, ist nicht
                  geklärt.}}}\label{K_L03469-2h}, der eine \textcolor{blue}{Couſine}{}\ledrightnote{{$\rightarrow$}\textcolor{blue}{Erica Tietze-Conrat}} von ihr {\pb}geheiratet hat, zu führen.
               Du brauchſt \textcolor{blue}{ihr}{}\ledrightnote{{$\rightarrow$}\textcolor{blue}{Margarethe Fränkel}} nur ins \textsc{\textcolor{brown}{Hotel Sacher}{}\ledrightnote{\textcolor{brown}{Hotel Sacher}}} zu telephoniren\footnote{\noindent{}{[}hs. Eva Marie Goldmann:{]} Lieber zu \textcolor{brown}{Sacher} ein paar Zeilen schreiben. Telephoniren ist fast nicht zu
                     machen. {\\}Viele Grüsse \spacefill\mbox{EvaG.}}. Du ſollteſt Dir das \textcolor{pink}{Haus}{}\ledrightnote{{$\rightarrow$}\textcolor{pink}{Armbrustergasse}}, das tatſächlich mit den billigſten Mitteln erbaut iſt u. auf der \textcolor{pink}{Hohen Warte}{}\ledrightnote{\textcolor{pink}{Hohe Warte}}, \textcolor{pink}{Armbruſterſtraße 20}{}\ledrightnote{\textcolor{pink}{Armbrustergasse}}, ſteht, einmal anſehen, ehe Du daran gehſt, die
               Wohnungsfrage zu löſen.\pend
           
\pstart
           Herzliche Grüße Deiner \textcolor{blue}{Frau}{}\ledrightnote{{$\rightarrow$}\textcolor{blue}{Olga Schnitzler}} u. Dir! Dein {\\[\baselineskip]}\spacefill\mbox{Paul Goldmann.}\pend
           \leftskip=0em{}\endnumbering\briefempfaengerindex{Schnitzler, Arthur@\textsc{Schnitzler, Arthur}!zzzGoldmann, Eva Marie@\emph{von Eva Marie Goldmann}!1909-10-011@{1. 10. 1909}|)be}\briefempfaengerindex{Schnitzler, Arthur@\textsc{Schnitzler, Arthur}!zzzGoldmann, Paul@\emph{von Paul Goldmann}!1909-10-011@{1. 10. 1909}|)be}\mylabel{h}  \normalsize

\doendnotes{C}
\bigskip
\vfill

\clearpage

\footnotesize

\lohead{\textsc{register}}

% Definiere theindex-Environment komplett neu ohne reledmac
\makeatletter
\renewenvironment{theindex}{%
  \section*{\indexname}%
  \setlength{\parindent}{0pt}%
  \setlength{\parskip}{0pt plus 0.3pt}%
  \let\item\@idxitem
}{%
  \clearpage
}
\makeatother

\IfFileExists{\jobname-pw.ind}{\input{\jobname-pw.ind}}{}

\end{document}

      