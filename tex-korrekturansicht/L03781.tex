%% latex-korrekturansicht-vorspann.tex
%% Vorspann für die Korrekturansicht.
%% Lädt die gemeinsame Datei latex-vorspann.tex mit gesetztem Schalter.

\newif\ifkorrekturansicht
\korrekturansichttrue

\input{../tex-inputs/latex-vorspann}


\section[Arthur Schnitzler an Stefan Zweig, {[}zwischen 25. und 31.?{]} 5. 1912]{L03781 Arthur Schnitzler an Stefan Zweig, {[}zwischen 25. und 31.?{]} 5. 1912}
\nopagebreak\mylabel{L03781v}
\rehead{ }\normalsize\beginnumbering\briefempfaengerindex{Zweig, Stefan@\textsc{Zweig, Stefan}!zzzSchnitzler, Arthur@\emph{von Arthur Schnitzler}!1912-05-312@{{[}zwischen 25. und 31.?{]} 5. 1912}|(be}
\toendnotes[C]{\smallbreak\pagebreak[2]}\Standort{Jerusalem, National Library of Israel, ARC. Ms. Var. 305 1 58 Stefan Zweig Collection.}
\physDesc{Karte, 1 Blatt, 1 Seite, 160 Zeichen
\newline{}Handschrift: schwarze Tinte, deutsche Kurrent}\toendnotes[C]{\smallbreak}
\pstart
           \noindent{}{\pb}Herzlichsten Dank, und ich möchte Ihnen doch \label{K_L03781-1v}\edtext{noch einmal}{\lemma{\textnormal{\emph{noch einmal}}}\Cendnote{\textnormal{Es gibt, abseits dieser Karte, keine erhaltene Korrespondenz
                  zwischen \textcolor{blue}{Schnitzler} und \textcolor{blue}{Zweig}\pwindex{Zweig, Stefan 28.11.1881 – 23.02.1942@\textsc{Zweig, Stefan} (28.11.1881 – 23.02.1942), \emph{Schriftsteller/Schriftstellerin}|pwk} aus diesem Zeitraum. Am 24. 5. 1912 begegnete
                  man sich (zufällig?) bei \textcolor{blue}{Eugenie Bachrach}\pwindex{Bachrach, Eugenie 04.03.1857 – 04.12.1937@\textsc{Bachrach, Eugenie} (04.03.1857 – 04.12.1937)|pwk}.
                     \textcolor{blue}{Schnitzler} notierte sich im \emph{\textcolor{green}{Tagebuch}\pwindex{Tagebuch@\emph{Tagebuch}|pwk}}: »Es kamen später ›\textcolor{blue}{Gicki}\pwindex{Gruenfeld, Max 1881 – 02.03.1915@\textsc{Grünfeld, Max} (1881 – 02.03.1915), \emph{Rechtsanwalt/Rechtsanwältin}|pw}‹, \textcolor{blue}{Stefan Zweig}\pwindex{Zweig, Stefan 28.11.1881 – 23.02.1942@\textsc{Zweig, Stefan} (28.11.1881 – 23.02.1942), \emph{Schriftsteller/Schriftstellerin}|pw}, der eigentlich wie ich ihm sagte, durch seine Anregung
                     an meinem 50. Geburtstag schuld. (Er hatte mir liebe Verse geschickt und im \textcolor{green}{Merker}\pwindex{Merker. Oesterreichische Zeitschrift fuer Musik und Theater@\emph{Der Merker. Österreichische Zeitschrift für Musik und Theater}|pw} einen warmen \textcolor{green}{Artikel}\pwindex{Schnitzler und die Jugend@\emph{Schnitzler und die Jugend}|pwv} über mich geschrieben.) –« Das an der vorliegenden
                  Stelle gebrauchte »noch einmal« deutet darauf hin, dass die Karte nach dieser 
                  Begegnung abgefasst wurde.}}}\label{K_L03781-1}
               sagen, wie ſehr mich Ihre lieben \label{K_L03781-2v}\edtext{\textcolor{green}{Worte}\pwindex{Schnitzler und die Jugend@\emph{Schnitzler und die Jugend}|pwv}{}\ledrightnote{{$\rightarrow$}\emph{\textcolor{green}{Schnitzler und die Jugend}}}}{\lemma{\textnormal{\emph{Worte}}}\Cendnote{\textnormal{\textcolor{blue}{Stefan Zweig}\pwindex{Zweig, Stefan 28.11.1881 – 23.02.1942@\textsc{Zweig, Stefan} (28.11.1881 – 23.02.1942), \emph{Schriftsteller/Schriftstellerin}|pwk}: \emph{\textcolor{green}{Schnitzler und die Jugend}\pwindex{Schnitzler und die Jugend@\emph{Schnitzler und die Jugend}|pwk}}. In: \emph{\textcolor{green}{Der Merker}\pwindex{Merker. Oesterreichische Zeitschrift fuer Musik und Theater@\emph{Der Merker. Österreichische Zeitschrift für Musik und Theater}|pwk}}, Jg. 3, Nr. 9, 1. 5. 1912,
                     S. 349–350. }}}\label{K_L03781-2} u Ihre ſchöne \label{K_L03781-3v}\edtext{\textcolor{green}{Verſe}\pwindex{?? [Verse zu Arthur Schnitzlers 50. Geburtstag]@\emph{?? [Verse zu Arthur Schnitzlers 50. Geburtstag]}|pw}{}\ledrightnote{\textcolor{green}{?? [Verse zu Arthur Schnitzlers 50. Geburtstag]}}}{\lemma{\textnormal{\emph{Verſe}}}\Cendnote{\textnormal{nicht erhalten}}}\label{K_L03781-3} erfreut
               haben!\pend
           
\pstart
           Ihr{\\[\baselineskip]}\spacefill\mbox{Arthur Schnitzler}\pend
           \leftskip=0em{}
\pstart
           \textcolor{pink}{Wien}\oindex{Wien@\textbf{Wien}|pw}{}\ledrightnote{\textcolor{pink}{Wien}}, im Mai 1912\pend
           \selectlanguage{ngerman}\endnumbering\briefempfaengerindex{Zweig, Stefan@\textsc{Zweig, Stefan}!zzzSchnitzler, Arthur@\emph{von Arthur Schnitzler}!1912-05-252@{{[}zwischen 25. und 31.?{]} 5. 1912}|)be}\mylabel{L03781h}
\begin{anhang}
\end{anhang}\normalsize

\doendnotes{C}
\bigskip
\vfill

\clearpage

\footnotesize

\lohead{\textsc{register}}

% Definiere theindex-Environment komplett neu ohne reledmac
\makeatletter
\renewenvironment{theindex}{%
  \section*{\indexname}%
  \setlength{\parindent}{0pt}%
  \setlength{\parskip}{0pt plus 0.3pt}%
  \let\item\@idxitem
}{%
  \clearpage
}
\makeatother

\IfFileExists{\jobname-pw.ind}{\input{\jobname-pw.ind}}{}

\end{document}

      