%% latex-korrekturansicht-vorspann.tex
%% Vorspann für die Korrekturansicht.
%% Lädt die gemeinsame Datei latex-vorspann.tex mit gesetztem Schalter.

\newif\ifkorrekturansicht
\korrekturansichttrue

\input{../tex-inputs/latex-vorspann}


               \section[ Paul Goldmann an Arthur Schnitzler, 3. 11. {[}1898{]}]{Paul Goldmann an Arthur Schnitzler, 3. 11. {[}1898{]}}\nopagebreak\mylabel{v}\rehead{ }\normalsize\beginnumbering\briefempfaengerindex{, @\textsc{, }!zzzGoldmann, Paul@\emph{von Paul Goldmann}!1898-11-031@{3. 11. {[}1898{]}}|(be} \toendnotes[C]{\smallbreak\pagebreak[2]} \Standort{DLA, A:Schnitzler, HS.NZ85.1.3168.}
\physDesc{Brief, 1 Blatt, 2 Seiten
\newline{}Handschrift: schwarze Tinte, deutsche Kurrent
\newline{}Schnitzler: mit Bleistift das Jahr »98« vermerkt }\toendnotes[C]{\smallbreak}\pstart
           \noindent{}{\pb}\textcolor{gray}{\textbf{\textcolor{pink}{The Oriental Hotel}{}\ledrightnote{\textcolor{pink}{The Oriental Hotel}},}}\pend
           \pstart
           \textcolor{gray}{\textbf{\textcolor{pink}{YOKOHAMA,
                     JAPAN}{}\ledrightnote{\textcolor{pink}{Yokohama}}.}}\pend
           \pstart
           \raggedleft{}\textsc{\textcolor{pink}{Yokohama}{}\ledrightnote{\textcolor{pink}{Yokohama}}}, 3. November.\pend
           \pstart\center{}Mein lieber Freund,\pend\pstart
           Ich habe drei Tage in \textsc{\textcolor{pink}{Kyoto}{}\ledrightnote{\textcolor{pink}{Kyoto}}}, der \textcolor{gray}{a}lten \textcolor{pink}{japan}{}\ledrightnote{\textcolor{pink}{Japan}}iſchen
                  \textcolor{pink}{Hauptſtadt}{}\ledrightnote{→\textcolor{pink}{Kyoto}}, verlebt, die
               zu den ſchönſten meines Lebens gehören. Das einzige Mal, daß ich den Eindruck hatte,
               ganz aus der Wirklichkeit heraus zu ſein! Ich bin gerade ſo kurze Zeit dageweſen, daß
               der Zauber nicht verfliegen konnte. Und ich ſpreche vom Lande allein, \strikeout{von die} nicht von den \label{K_L02864-3v}\edtext{\textsc{Musmes}}{\lemma{\textnormal{\emph{Musmes}}}\Cendnote{\textnormal{junge Mädchen, eventuell
               von \textcolor{blue}{Goldmann} hier als Synonym für »süßes Mädel« gedacht?}}}\label{K_L02864-3h}
               und leichter Liebe, – nein, allein von dem Zauber dieſer herrlichen Berge mit ihren
               Nadelwäldern und herbſtrothen Ahorn-Bäumen, von dem {\pb}Zauber dieſer ſeltſamen, ſeltſamen \textcolor{pink}{Stadt}{}\ledrightnote{→\textcolor{pink}{Kyoto}} mit ihren wundervollen \strikeout{\textcolor{gray}{Tempeln und ihr}}
                Tempeln und den ſtillen Straßen, in denen das ſanfte Flötenſpiel der Prieſter
               klingt, welche Almoſen einſammeln. Keine Feder vermag das zu beſchreiben. Jetzt fällt
               der Regen, und ich ſitze in dem reizloſen kosmopolitiſchen \textsc{\textcolor{pink}{Yokohama}{}\ledrightnote{\textcolor{pink}{Yokohama}}} und ſehne mich nach \textsc{\textcolor{pink}{Kyoto}{}\ledrightnote{\textcolor{pink}{Kyoto}}}, wie ich mich mein ganzes Leben danach ſehnen werde.\pend
           \pstart
           Von Dir habe ich lange nichts gehört. Wie mag es Dir nur gehen?\pend
           \pstart
           Viele treue Grüße! {\\[\baselineskip]}Dein {\\[\baselineskip]}\spacefill\mbox{Paul Goldmann}\pend
           \leftskip=0em{}\pstart
           \noindent{}Grüße an Deine \textcolor{blue}{Freundin}{}\ledrightnote{→\textcolor{blue}{Marie Reinhard}}!\pend
           \endnumbering\briefempfaengerindex{, @\textsc{, }!zzzGoldmann, Paul@\emph{von Paul Goldmann}!1898-11-031@{3. 11. {[}1898{]}}|)be}\mylabel{h}  \normalsize

\doendnotes{C}
\bigskip
\vfill

\clearpage

\footnotesize

\lohead{\textsc{register}}

% Definiere theindex-Environment komplett neu ohne reledmac
\makeatletter
\renewenvironment{theindex}{%
  \section*{\indexname}%
  \setlength{\parindent}{0pt}%
  \setlength{\parskip}{0pt plus 0.3pt}%
  \let\item\@idxitem
}{%
  \clearpage
}
\makeatother

\IfFileExists{\jobname-pw.ind}{\input{\jobname-pw.ind}}{}

\end{document}

      