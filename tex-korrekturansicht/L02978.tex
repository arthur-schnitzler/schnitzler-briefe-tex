%% latex-korrekturansicht-vorspann.tex
%% Vorspann für die Korrekturansicht.
%% Lädt die gemeinsame Datei latex-vorspann.tex mit gesetztem Schalter.

\newif\ifkorrekturansicht
\korrekturansichttrue

\input{../tex-inputs/latex-vorspann}


\renewcommand{\erwaehntePersonen}{Personen: Felix Salten, Émile Zola}
\renewcommand{\erwaehnteOrte}{Orte: Raimund-Theater, Wien}
\renewcommand{\erwaehnteWerke}{Werke: Abschiedssouper, Die Zeit, Die kleine Veronika, La joie de vivre, L’œuvre, Zola’s Lebenswerk}
\section[ Arthur Schnitzler an Felix Salten, 30. 9. 1902]{Arthur Schnitzler an Felix Salten, 30. 9. 1902}
\nopagebreak\mylabel{v}
\rehead{ }\normalsize\beginnumbering\briefempfaengerindex{Salten, Felix@\textsc{Salten, Felix}!zzzSchnitzler, Arthur@\emph{von Arthur Schnitzler}!1902-09-301@{30. 9. 1902}|(be}
\toendnotes[C]{\smallbreak\pagebreak[2]}\Standort{Wienbibliothek im Rathaus, ZPH 1681, 2.1.516.}
\physDesc{Brief, 1 Blatt, 2 Seiten, 421 Zeichen
\newline{}Handschrift: schwarze Tinte, deutsche Kurrent
\newline{}Ordnung: mit Bleistift von unbekannter Hand nummeriert: »66« }\toendnotes[C]{\smallbreak}
\pstart
           \raggedleft{}{\pb}30. 9. 902\pend
           
\pstart{}lieber Freund,\pend
\pstart
           ich konnte leider geſtern nicht länger auf Sie \label{K_L02978-1v}\edtext{warten}{\lemma{\textnormal{\emph{warten}}}\Cendnote{\textnormal{Mutmaßlich im
               Kaffeehaus, nachdem \textcolor{blue}{Schnitzler} im \textcolor{pink}{Raimundtheater} gewesen war, wo er \emph{\textcolor{green}{Abschiedssouper}}
                 sah, vgl. A. S.: \emph{Tagebuch}, 29. 9. 1902}}}\label{K_L02978-1h}. Hatte arge Kopfſchmerzen.\pend
           
\pstart
           Ihr \label{K_L02978-2v}\edtext{\textcolor{green}{\textcolor{blue}{Zola}{}\ledrightnote{\textcolor{blue}{Émile Zola}} Feu{[}i{]}lleton}{}\ledrightnote{{$\rightarrow$}\textcolor{green}{Zola’s Lebenswerk}}}{\lemma{\textnormal{\emph{Zola Feuilleton}}}\Cendnote{\textnormal{\textcolor{blue}{Felix Salten}: \emph{\textcolor{green}{Zola’s Lebenswerk}}. In: \emph{\textcolor{green}{Die Zeit}}, Jg. 1, Nr. 4, 30. 9. 1902,
                     Morgenblatt, S. 1–2.}}}\label{K_L02978-2h} iſt glänzend – insbeſondre freu {\pb}ich mich, daſs Sie \textsc{\textcolor{green}{oeuvre}{}\ledrightnote{\textcolor{green}{L’œuvre}}} und \textcolor{green}{\textsc{joie de vivre}}{}\ledrightnote{\textcolor{green}{La joie de vivre}} als die ewigen unter ſeinen Werken herausgegriffen haben. Und das ganze hat ſo
               einen Schmiſs.\pend
           
\pstart
           – Hoffentlich \label{K_L02978-3v}\edtext{ſeh ich Sie heut{ }Abend im Café}{\lemma{\textnormal{\emph{ſeh … Café}}}\Cendnote{\textnormal{Ein Treffen an diesem Abend ist nicht
                  nachgewiesen.}}}\label{K_L02978-3h} und Sie bringen die \textcolor{green}{kleine \textsc{Veronika}}{}\ledrightnote{\textcolor{green}{Die kleine Veronika}} mit we{\geminationn} ſie ſchon ins Kaffehaus gehen darf.\pend
           
\pstart
           Herzlichſt Ihr {\\[\baselineskip]}\spacefill\mbox{Arth Sch\textcolor{gray}{.}}\pend
           \leftskip=0em{}\endnumbering\briefempfaengerindex{Salten, Felix@\textsc{Salten, Felix}!zzzSchnitzler, Arthur@\emph{von Arthur Schnitzler}!1902-09-301@{30. 9. 1902}|)be}\mylabel{h}  \normalsize

\doendnotes{C}
\bigskip
\vfill

\clearpage

\footnotesize

\lohead{\textsc{register}}

% Definiere theindex-Environment komplett neu ohne reledmac
\makeatletter
\renewenvironment{theindex}{%
  \section*{\indexname}%
  \setlength{\parindent}{0pt}%
  \setlength{\parskip}{0pt plus 0.3pt}%
  \let\item\@idxitem
}{%
  \clearpage
}
\makeatother

\IfFileExists{\jobname-pw.ind}{\input{\jobname-pw.ind}}{}

\end{document}

      