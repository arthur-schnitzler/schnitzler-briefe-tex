%% latex-korrekturansicht-vorspann.tex
%% Vorspann für die Korrekturansicht.
%% Lädt die gemeinsame Datei latex-vorspann.tex mit gesetztem Schalter.

\newif\ifkorrekturansicht
\korrekturansichttrue

\input{../tex-inputs/latex-vorspann}


\renewcommand{\erwaehntePersonen}{Personen: Max Reinhardt, Felix Salten}
\renewcommand{\erwaehnteInstitutionen}{Institutionen: Deutsches Theater Berlin}
\renewcommand{\erwaehnteOrte}{Orte: Altes Palais, Berlin, Edmund-Weiß-Gasse 7, Wien}
\renewcommand{\erwaehnteWerke}{Werke: Der gute König Dagobert. Lustspiel in vier Aufzügen, Der junge Medardus. Dramatische Historie in einem Vorspiel und fünf Aufzügen}
\section[ Felix Salten an Arthur Schnitzler, 17. 1. 1910]{Felix Salten an Arthur Schnitzler, 17. 1. 1910}
\nopagebreak\mylabel{v}
\rehead{ }\normalsize\beginnumbering\briefempfaengerindex{Schnitzler, Arthur@\textsc{Schnitzler, Arthur}!zzzSalten, Felix@\emph{von Felix Salten}!1910-01-171@{17. 1. 1910}|(be}
\toendnotes[C]{\smallbreak\pagebreak[2]}\Standort{CUL, Schnitzler, B 89, B 2.}
\physDesc{Bildpostkarte, 356 Zeichen
\newline{}Handschrift: schwarze Tinte, lateinische Kurrent
\newline{}Versand: Stempel: »\nobreak{}\oindex{Berlin@\textbf{Berlin}, \emph{P.PPLC}|pwk}Berlin W 9, 17. 1. 10, 8–9 N\nobreak{}«.  
\newline{}Ordnung: mit Bleistift von unbekannter Hand nummeriert: »260« }\toendnotes[C]{\smallbreak}\pstart{}{\pb}Herrn D\textsuperscript{r} Artur Schnitzler\pend{}\pstart{}\textcolor{pink}{Wien}{}\ledrightnote{\textcolor{pink}{Wien}}\pend{}\pstart{}\textcolor{pink}{XVIII. Spoettelgaße 7}{}\ledrightnote{\textcolor{pink}{Edmund-Weiß-Gasse 7}}\pend{}
{\bigskip}
\pstart
           \noindent{}{\pb}\textcolor{pink}{\textcolor{gray}{\textbf{\textbf{Berlin}.}}}{}\ledrightnote{\textcolor{pink}{Berlin}}\hfill \textcolor{gray}{\textbf{\textcolor{pink}{Palais Kaiser Wilhelm des Grossen}{}\ledrightnote{\textcolor{pink}{Altes Palais}} mit
                        dem historischen Eckfenster.}}\pend
           
\pstart
           {\pb}Lieber, wenn es etwas gibt, was noch unangenehmer ist, als \textcolor{blue}{Reinhardt}{}\ledrightnote{\textcolor{blue}{Max Reinhardt}} ein Stück einzureichen, dann ist es
               das: \label{K_L03545-1v}\edtext{bei \textcolor{blue}{Reinhardt}{}\ledrightnote{\textcolor{blue}{Max Reinhardt}} aufgeführt werden}{\lemma{\textnormal{\emph{bei … werden}}}\Cendnote{\textnormal{vermutlich Bezug auf \textcolor{blue}{Schnitzler}s Streitigkeiten mit \textcolor{blue}{Max
                     Reinhardt} rund um seine Einreichung von \emph{\textcolor{green}{Der junge Medardus}} (vgl. \emph{Der Briefwechsel Arthur
                        Schnitzlers mit Max Reinhardt und dessen Mitarbeitern}. Herausgegeben
                     von Renate Wagner. Salzburg: \emph{Otto Müller
                        Verlag}{ }1971, S. 60–79) und die Aufführung von \textcolor{blue}{Salten}s Übersetzung von \emph{\textcolor{green}{Der gute König Dagobert}} durch das \emph{\textcolor{brown}{Deutsche Theater Berlin}} (siehe A. S.: \emph{Tagebuch}, 2. 1. 1910)}}}\label{K_L03545-1h}! Ich ärgere mich nicht mehr, aber
               ich habe eben eine Reise getan, und kann etwas erzählen!\pend
           
\pstart
           Hoffentlich bald! Herzliche Grüße von Haus zu Haus {\\[\baselineskip]}Ihr {\\[\baselineskip]}\spacefill\mbox{Felix Salten}\pend
           \leftskip=0em{}
\pstart
           \textcolor{pink}{Berlin}{}\ledrightnote{\textcolor{pink}{Berlin}}{ }1\textcolor{gray}{7}. I. \textcolor{gray}{10}\pend
           \endnumbering\briefempfaengerindex{Schnitzler, Arthur@\textsc{Schnitzler, Arthur}!zzzSalten, Felix@\emph{von Felix Salten}!1910-01-171@{17. 1. 1910}|)be}\mylabel{h}  \normalsize

\doendnotes{C}
\bigskip
\vfill

\clearpage

\footnotesize

\lohead{\textsc{register}}

% Definiere theindex-Environment komplett neu ohne reledmac
\makeatletter
\renewenvironment{theindex}{%
  \section*{\indexname}%
  \setlength{\parindent}{0pt}%
  \setlength{\parskip}{0pt plus 0.3pt}%
  \let\item\@idxitem
}{%
  \clearpage
}
\makeatother

\IfFileExists{\jobname-pw.ind}{\input{\jobname-pw.ind}}{}

\end{document}

      