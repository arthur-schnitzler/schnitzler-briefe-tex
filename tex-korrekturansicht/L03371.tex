%% latex-korrekturansicht-vorspann.tex
%% Vorspann für die Korrekturansicht.
%% Lädt die gemeinsame Datei latex-vorspann.tex mit gesetztem Schalter.

\newif\ifkorrekturansicht
\korrekturansichttrue

\input{../tex-inputs/latex-vorspann}


\renewcommand{\erwaehntePersonen}{Personen: Wilhelm von Hartel, Robert Pattai, Olga Schnitzler}
\renewcommand{\erwaehnteInstitutionen}{Institutionen: Bauernfeld-Preis, Reichsrat}
\renewcommand{\erwaehnteOrte}{Orte: Berlin, Dessauer Straße, Wien}
\renewcommand{\erwaehnteWerke}{}
\section[ Paul Goldmann an Arthur Schnitzler, 1. 4. {[}1903{]}]{Paul Goldmann an Arthur Schnitzler, 1. 4. {[}1903{]}}
\nopagebreak\mylabel{v}
\rehead{ }\normalsize\beginnumbering\briefempfaengerindex{Schnitzler, Arthur@\textsc{Schnitzler, Arthur}!zzzGoldmann, Paul@\emph{von Paul Goldmann}!1903-04-011@{1. 4. {[}1903{]}}|(be}
\toendnotes[C]{\smallbreak\pagebreak[2]}\Standort{DLA, A:Schnitzler, HS.NZ85.1.3173.}
\physDesc{Brief, 1 Blatt, 1 Seite
\newline{}Handschrift: blaue Tinte, deutsche Kurrent
\newline{}Schnitzler: 1) mit Bleistift das Jahr »{[}1{]}903.« vermerkt  2) mit rotem Buntstift eine Unterstreichung}\toendnotes[C]{\smallbreak}
\pstart
           \noindent{}\raggedleft{}{\pb}\textcolor{gray}{\textbf{\textcolor{pink}{DESSAUERSTRASSE 19}{}\ledrightnote{\textcolor{pink}{Dessauer Straße}}}}\pend
           
\pstart
           \textcolor{pink}{Berlin}{}\ledrightnote{\textcolor{pink}{Berlin}}, 1. April.\pend
           
\pstart\center{}Mein lieber Freund,\pend
\pstart
           Die \label{K_L03371-1v}\edtext{Interpellations\textcolor{gray}{-}Beantwortung}{\lemma{\textnormal{\emph{Interpellations-Beantwortung}}}\Cendnote{\textnormal{Der antisemitische Abgeordnete \textcolor{blue}{Robert Pattai} hatte
               am 18. 3. 1903 im \textcolor{brown}{Abgeordnetenhaus} die Zuerkennung des \emph{\textcolor{brown}{Bauernfeld-Preis}}es
               an den »jüdischen Autor« \textcolor{blue}{Schnitzler}
               kritisiert, zumal da dessen ausgezeichnetes Werk \emph{\textcolor{green}{Lebendige Stunden}} von niederer Qualität sei. (Siehe A. S.: \emph{»Das Zeitlose ist von kürzester Dauer«}, [Felix Salten]: Der Bauernfeld-Preis. Eine Interpellation, 19. 3. 1903)
               In der Sitzung des \textcolor{brown}{Abgeordnetenhaus}es am 31. 3. 1903 hatte der Unterrichtsminister \textcolor{blue}{Wilhelm von Hartel} darauf geantwortet. }}}\label{K_L03371-1h} des \textcolor{blue}{Unterrichtsminiſter}{}\ledrightnote{\textcolor{blue}{Wilhelm von Hartel}}s iſt ſehr anſtändig und für
               Dich auch recht ehrenvoll. Ich habe mich darüber ſehr gefreut.\pend
           
\pstart
           Warum ſchreibſt Du mir nicht?\pend
           \pstart Viele herzliche Grüße Dir und \textsc{\textcolor{blue}{Olga}{}\ledrightnote{\textcolor{blue}{Olga Schnitzler}}}! Dein \spacefill\mbox{Paul Goldm}\pend{}\endnumbering\briefempfaengerindex{Schnitzler, Arthur@\textsc{Schnitzler, Arthur}!zzzGoldmann, Paul@\emph{von Paul Goldmann}!1903-04-011@{1. 4. {[}1903{]}}|)be}\mylabel{h}  \normalsize

\doendnotes{C}
\bigskip
\vfill

\clearpage

\footnotesize

\lohead{\textsc{register}}

% Definiere theindex-Environment komplett neu ohne reledmac
\makeatletter
\renewenvironment{theindex}{%
  \section*{\indexname}%
  \setlength{\parindent}{0pt}%
  \setlength{\parskip}{0pt plus 0.3pt}%
  \let\item\@idxitem
}{%
  \clearpage
}
\makeatother

\IfFileExists{\jobname-pw.ind}{\input{\jobname-pw.ind}}{}

\end{document}

      