%% latex-korrekturansicht-vorspann.tex
%% Vorspann für die Korrekturansicht.
%% Lädt die gemeinsame Datei latex-vorspann.tex mit gesetztem Schalter.

\newif\ifkorrekturansicht
\korrekturansichttrue

\input{../tex-inputs/latex-vorspann}


\renewcommand{\erwaehntePersonen}{Personen: Theodor Entsch, Paul Goldmann, Alfred Kerr, P. M.}
\renewcommand{\erwaehnteOrte}{Orte: Berlin, Deutsches Theater Berlin, Hotel Bristol Berlin, I., Innere Stadt, Wien}
\renewcommand{\erwaehnteWerke}{Werke: Der Gemeine. Schauspiel in drei Aufzügen, Lebendige Stunden. Vier Einakter}
\section[ Felix Salten an Arthur Schnitzler, 2. 1. 1902]{Felix Salten an Arthur Schnitzler, 2. 1. 1902}
\nopagebreak\mylabel{v}
\rehead{ }\normalsize\beginnumbering\briefempfaengerindex{Schnitzler, Arthur@\textsc{Schnitzler, Arthur}!zzzSalten, Felix@\emph{von Felix Salten}!1902-01-021@{2. 1. 1902}|(be}
\toendnotes[C]{\smallbreak\pagebreak[2]}\Standort{CUL, Schnitzler, B 89, A 2.}
\physDesc{Postkarte, 463 Zeichen
\newline{}Handschrift: Bleistift, lateinische Kurrent
\newline{}Versand: 1) Stempel: »\nobreak{}\oindex{I., Innere Stadt@\textbf{I., Innere Stadt}, \emph{A.ADM3}|pwk}Wien 1/1 1 a, 2. 1. 02, 8–9 N\nobreak{}«.   2) Stempel: »\nobreak{}\textcolor{gray}{×}.
                                          \textcolor{gray}{1}. 02, Bestellt vom Postamte 64\nobreak{}«. 
\newline{}Schnitzler: mit Bleistift datiert: »2/1 902« 
\newline{}Ordnung: mit Bleistift von unbekannter Hand nummeriert: »145« }\toendnotes[C]{\smallbreak}\pstart{}{\pb}Herrn D\textsuperscript{r} Arthur Schnitzler\pend{}\pstart{}\textcolor{pink}{Berlin \strikeout{W.}}{}\ledrightnote{\textcolor{pink}{Berlin}}\pend{}\pstart{}\textcolor{pink}{Hotel Bristol}{}\ledrightnote{\textcolor{pink}{Hotel Bristol Berlin}}\pend{}
{\bigskip}
\pstart
           \noindent{}{\pb}Lieber, danke für Ihre \label{K_L03321-1v}\edtext{C. C.}{\lemma{\textnormal{\emph{C. C.}}}\Cendnote{\textnormal{Correspondenz-Carte?}}}\label{K_L03321-1h} und für Ihr frdl. Anerbieten. Wenn Sie \label{K_L03321-2v}\edtext{\textcolor{blue}{Entsch}{}\ledrightnote{\textcolor{blue}{Theodor Entsch}} sehen}{\lemma{\textnormal{\emph{Entsch sehen}}}\Cendnote{\textnormal{\textcolor{blue}{Schnitzler} traf \textcolor{blue}{Theodor Entsch}, Theateragent und Verleger, am 6. 1. 1902.}}}\label{K_L03321-2h},
               dann bitte sagen Sie ihm, dass \textcolor{blue}{P. M.}{}\ledrightnote{\textcolor{blue}{P. M.}} mein \textcolor{green}{Stück}{}\ledrightnote{{$\rightarrow$}\textcolor{green}{Der Gemeine. Schauspiel in drei Aufzügen}} gerne los wäre, dass ich
               es aber jedenfalls darauf ankommen laße, dass \uline{er} den
               Contract bricht. Wenn Sie mir \textcolor{blue}{Kerr}{}\ledrightnote{\textcolor{blue}{Alfred Kerr}}’s Adreße
               angeben könnten, wäre ich Ihnen sehr dankbar. Wenn Sie Zeit haben, schreiben Sie mir
               ein paar Zeilen über den Ausgang von \label{K_L03321-3v}\edtext{Samstag{ }Abend}{\lemma{\textnormal{\emph{Samstag Abend}}}\Cendnote{\textnormal{Am Samstag, dem 4. 1. 1902, fand am
                     \textcolor{pink}{Deutschen Theater Berlin} die Uraufführung der
                  vier Einakter \emph{\textcolor{green}{Lebendige Stunden}}
               statt.}}}\label{K_L03321-3h}. Grüßen Sie \textcolor{blue}{Goldmann}{}\ledrightnote{\textcolor{blue}{Paul Goldmann}} ec.\pend
           
\pstart
           Herzlichst Ihr {\\[\baselineskip]}\spacefill\mbox{Salten}\pend
           \leftskip=0em{}\endnumbering\briefempfaengerindex{Schnitzler, Arthur@\textsc{Schnitzler, Arthur}!zzzSalten, Felix@\emph{von Felix Salten}!1902-01-021@{2. 1. 1902}|)be}\mylabel{h}  \normalsize

\doendnotes{C}
\bigskip
\vfill

\clearpage

\footnotesize

\lohead{\textsc{register}}

% Definiere theindex-Environment komplett neu ohne reledmac
\makeatletter
\renewenvironment{theindex}{%
  \section*{\indexname}%
  \setlength{\parindent}{0pt}%
  \setlength{\parskip}{0pt plus 0.3pt}%
  \let\item\@idxitem
}{%
  \clearpage
}
\makeatother

\IfFileExists{\jobname-pw.ind}{\input{\jobname-pw.ind}}{}

\end{document}

      