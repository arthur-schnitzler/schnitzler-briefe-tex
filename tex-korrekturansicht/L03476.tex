%% latex-korrekturansicht-vorspann.tex
%% Vorspann für die Korrekturansicht.
%% Lädt die gemeinsame Datei latex-vorspann.tex mit gesetztem Schalter.

\newif\ifkorrekturansicht
\korrekturansichttrue

\input{../tex-inputs/latex-vorspann}


\renewcommand{\erwaehntePersonen}{Personen: Louise Schnitzler, Karl Seeauer, Helene Seeauer}
\renewcommand{\erwaehnteOrte}{Orte: Bad Ischl, Hotel Kaiserin Elisabeth, Salzkammergut, Wien, Österreich}
\renewcommand{\erwaehnteWerke}{}
\section[ Paul Goldmann an Arthur Schnitzler, 10. 9. 1911]{Paul Goldmann an Arthur Schnitzler, 10. 9. 1911}
\nopagebreak\mylabel{v}
\rehead{ }\normalsize\beginnumbering\briefempfaengerindex{Schnitzler, Arthur@\textsc{Schnitzler, Arthur}!zzzGoldmann, Paul@\emph{von Paul Goldmann}!1911-09-102@{10. 9. 1911}|(be}
\toendnotes[C]{\smallbreak\pagebreak[2]}\Standort{DLA, A:Schnitzler, HS.NZ85.1.3176.}
\physDesc{Brief, 1 Blatt, 2 Seiten, 422 Zeichen
\newline{}Handschrift: schwarze Tinte, deutsche Kurrent
\newline{}Schnitzler: mit Bleistift Vermerk »\textsc{\textcolor{blue}{Goldma\textcolor{gray}{nn}}}« }\toendnotes[C]{\smallbreak}
\pstart
           \noindent{}\centering{}{\pb}\textcolor{gray}{\textbf{\textbf{\textsc{\textcolor{pink}{Hotel Kaiserin Elisabeth}{}\ledrightnote{\textcolor{pink}{Hotel Kaiserin Elisabeth}}}}}}\pend
           
\pstart
           \noindent{}\textcolor{gray}{\textbf{INTERURBAN. TELEPHON No. 12}}\hfill \textcolor{gray}{\textbf{Eigentümer: \textcolor{blue}{K.}{}\ledrightnote{\textcolor{blue}{Karl Seeauer}}{ }{\kaufmannsund}{ }\textcolor{blue}{H. SEEAUER}{}\ledrightnote{\textcolor{blue}{Helene Seeauer}}}}\pend
           
\pstart
           \textcolor{gray}{\textbf{Telegr.-Adr.:}}{ }{\\}\textcolor{gray}{\textbf{\textcolor{pink}{ELISABETHHOTEL}{}\ledrightnote{\textcolor{pink}{Hotel Kaiserin Elisabeth}}, \textcolor{pink}{BAD ISCHL}{}\ledrightnote{\textcolor{pink}{Bad Ischl}}.}}\hfill \textcolor{gray}{\textbf{\textsc{\textcolor{pink}{Bad Ischl}{}\ledrightnote{\textcolor{pink}{Bad Ischl}}}}}\pend
           
\pstart
           \textcolor{gray}{\textbf{LIFT\hspace*{5em}BÄDER.}}\hfill \textcolor{gray}{\textbf{= \textcolor{pink}{Salzkammergut}{}\ledrightnote{\textcolor{pink}{Salzkammergut}}, \textcolor{pink}{Österreich}{}\ledrightnote{\textcolor{pink}{Österreich}}. =}}\pend
           
\pstart
           \centering{}10. 9. 11.\pend
           
\pstart{}Lieber Freund,\pend
\pstart
           Tief ergriffen von der Nachricht, die ich eben erfahre, habe ich das Bedürfnis, Dir
               teilnehmend die Hand zu drücken u. Dir zu ſagen, daß ich den \label{K_L03476-1v}\edtext{Tod Deiner \textcolor{blue}{Mutter}{}\ledrightnote{{$\rightarrow$}\textcolor{blue}{Louise Schnitzler}}}{\lemma{\textnormal{\emph{Tod Deiner Mutter}}}\Cendnote{\textnormal{\textcolor{blue}{Louise Schnitzler} war am 9. 9. 9011
                  verstorben.}}}\label{K_L03476-1h} mit Dir betraure. Das iſt das Schwerſte, das einen Menſchen
               treffen \strikeout{K} kann, u. ich wünſche Dir die Kraft, dieſen
               Schickſalsſchlag zu {\pb}ertagen. Ich werde Deiner
                  \textcolor{blue}{Mutter}{}\ledrightnote{{$\rightarrow$}\textcolor{blue}{Louise Schnitzler}} die Güte u.
               Freundſchaft, die ſie mir erwieſen, nie vergeſſen. {\\}\spacefill\mbox{Paul Goldmann.}\pend
           \endnumbering\briefempfaengerindex{Schnitzler, Arthur@\textsc{Schnitzler, Arthur}!zzzGoldmann, Paul@\emph{von Paul Goldmann}!1911-09-102@{10. 9. 1911}|)be}\mylabel{h}
\begin{anhang}
\end{anhang}\normalsize

\doendnotes{C}
\bigskip
\vfill

\clearpage

\footnotesize

\lohead{\textsc{register}}

% Definiere theindex-Environment komplett neu ohne reledmac
\makeatletter
\renewenvironment{theindex}{%
  \section*{\indexname}%
  \setlength{\parindent}{0pt}%
  \setlength{\parskip}{0pt plus 0.3pt}%
  \let\item\@idxitem
}{%
  \clearpage
}
\makeatother

\IfFileExists{\jobname-pw.ind}{\input{\jobname-pw.ind}}{}

\end{document}

      