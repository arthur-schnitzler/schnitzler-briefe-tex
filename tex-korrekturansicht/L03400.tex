%% latex-korrekturansicht-vorspann.tex
%% Vorspann für die Korrekturansicht.
%% Lädt die gemeinsame Datei latex-vorspann.tex mit gesetztem Schalter.

\newif\ifkorrekturansicht
\korrekturansichttrue

\input{../tex-inputs/latex-vorspann}


\renewcommand{\erwaehntePersonen}{Personen: Hansi Niese, Anna Katharina Rehmann, Ottilie Salten, Alfred von Schik-Markenau, Leo Stein}
\renewcommand{\erwaehnteOrte}{Orte: Raimund-Theater, Wien}
\renewcommand{\erwaehnteWerke}{Werke: Eduard, der Herzensdieb. Posse mit Gesang in fünf Bildern}
\section[ Felix Salten an Arthur Schnitzler, {[}15. 12. 1904{]}]{Felix Salten an Arthur Schnitzler, {[}15. 12. 1904{]}}
\nopagebreak\mylabel{v}
\rehead{ }\normalsize\beginnumbering\briefempfaengerindex{Schnitzler, Arthur@\textsc{Schnitzler, Arthur}!zzzSalten, Felix@\emph{von Felix Salten}!1904-12-151@{{[}15. 12. 1904{]}}|(be}
\toendnotes[C]{\smallbreak\pagebreak[2]}\Standort{CUL, Schnitzler, B 89, B 1.}
\physDesc{Brief, 1 Blatt, 1 Seite, 510 Zeichen
\newline{}Handschrift: schwarze Tinte, lateinische Kurrent
\newline{}Schnitzler: mit Bleistift datiert: »15/12 904« 
\newline{}Ordnung: mit Bleistift von unbekannter Hand nummeriert: »193« }\toendnotes[C]{\smallbreak}
\pstart
           \raggedleft{}{\pb}Donnerstag\pend
           
\pstart
           Lieber, ich hab’ es der \textcolor{blue}{Niese}{}\ledrightnote{\textcolor{blue}{Hansi Niese}}
               leider schon versprechen müssen, dass ich Samstag zu
               der \label{K_L03400-1v}\edtext{\textcolor{green}{Première}{}\ledrightnote{{$\rightarrow$}\textcolor{green}{Eduard, der Herzensdieb. Posse mit Gesang in fünf Bildern}}}{\lemma{\textnormal{\emph{Première}}}\Cendnote{\textnormal{Am 17. 12. 1904 fand die Uraufführung von \emph{\textcolor{green}{Eduard, der Herzensdieb. Posse mit Gesang in fünf Bildern}} von \textcolor{blue}{Leo Stein} und \textcolor{blue}{Alfred von Schik-Markenau} im \textcolor{pink}{Raimund-Theater} statt. \textcolor{blue}{Hansi Niese}
                  gab die weibliche Hauptrolle.}}}\label{K_L03400-1h} gehe. Vielleicht \label{K_L03400-2v}\edtext{sehen wir uns}{\lemma{\textnormal{\emph{sehen wir uns}}}\Cendnote{\textnormal{Nachweislich sahen sich \textcolor{blue}{Salten} und \textcolor{blue}{Schnitzler} erst am 23. 12. 1904
                  wieder.}}}\label{K_L03400-2h} also an einem anderen Abend, Montag oder Dienstag, was ich Ihnen aber
               erst Samstag, wenn das Repertoire da
                  ist{[},{]} sagen kann. \textcolor{blue}{Otti}{}\ledrightnote{\textcolor{blue}{Ottilie Salten}}
               ist schon zurück, wird aber die nächsten Wochen nicht für länger vom Haus fortkönnen,
               weil das \textcolor{blue}{Mäderl}{}\ledrightnote{{$\rightarrow$}\textcolor{blue}{Anna Katharina Rehmann}} geimpft
               wurde, und sie braucht.\pend
           
\pstart
           Was Sie mit dem »\label{K_L03400-3v}\edtext{sich in Schulden
               gestürzt haben}{\lemma{\textnormal{\emph{sich … haben}}}\Cendnote{\textnormal{siehe Arthur Schnitzler an Felix Salten, 13. 12. 1904}}}\label{K_L03400-3h}« meinen, verstehe ich nicht. In \textcolor{pink}{Wien}{}\ledrightnote{\textcolor{pink}{Wien}} sind
               Sie doch eher Gläubiger.\pend
           
\pstart
           herzlich {\\[\baselineskip]}Ihr \spacefill\mbox{Salten}\pend
           \leftskip=0em{}\endnumbering\briefempfaengerindex{Schnitzler, Arthur@\textsc{Schnitzler, Arthur}!zzzSalten, Felix@\emph{von Felix Salten}!1904-12-151@{{[}15. 12. 1904{]}}|)be}\mylabel{h}  \normalsize

\doendnotes{C}
\bigskip
\vfill

\clearpage

\footnotesize

\lohead{\textsc{register}}

% Definiere theindex-Environment komplett neu ohne reledmac
\makeatletter
\renewenvironment{theindex}{%
  \section*{\indexname}%
  \setlength{\parindent}{0pt}%
  \setlength{\parskip}{0pt plus 0.3pt}%
  \let\item\@idxitem
}{%
  \clearpage
}
\makeatother

\IfFileExists{\jobname-pw.ind}{\input{\jobname-pw.ind}}{}

\end{document}

      