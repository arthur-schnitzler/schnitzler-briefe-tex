%% latex-korrekturansicht-vorspann.tex
%% Vorspann für die Korrekturansicht.
%% Lädt die gemeinsame Datei latex-vorspann.tex mit gesetztem Schalter.

\newif\ifkorrekturansicht
\korrekturansichttrue

\input{../tex-inputs/latex-vorspann}


\renewcommand{\erwaehntePersonen}{Personen: Emil Mayer, Felix Salten}
\renewcommand{\erwaehnteInstitutionen}{Institutionen: Verlagsanstalt Brüder Rosenbaum}
\renewcommand{\erwaehnteOrte}{Orte: Leipzig, Wien}
\renewcommand{\erwaehnteWerke}{Werke: Wurstelprater}
\section[ Felix Salten: Widmungsexemplar Wurstelprater für Arthur Schnitzler, 12. 12. 1911]{Felix Salten: Widmungsexemplar Wurstelprater für Arthur
               Schnitzler, 12. 12. 1911}
\nopagebreak\mylabel{v}
\rehead{ }\normalsize\beginnumbering\briefempfaengerindex{Schnitzler, Arthur@\textsc{Schnitzler, Arthur}!zzzSalten, Felix@\emph{von Felix Salten}!1911-12-121@{12. 12. 1911}|(be}
\toendnotes[C]{\smallbreak\pagebreak[2]}\Standort{DLA, G:Schnitzler, Arthur (Sammlung Heinrich Schnitzler).}
\physDesc{Widmung am Titelblatt, 38 Zeichen
\newline{}Handschrift: schwarze Tinte, lateinische Kurrent}\toendnotes[C]{\smallbreak}
\pstart
           \noindent{}\centering{}{\pb}\textcolor{gray}{\textbf{\textbf{\textcolor{green}{\so{WURSTELPRATER}}{}\ledrightnote{\textcolor{green}{Wurstelprater}}}}}\pend
           
\pstart
           \noindent{}\centering{}\textcolor{gray}{\textbf{\so{VON FELIX SALTEN}}}\pend
           
\pstart
           \noindent{}\centering{}\textcolor{gray}{\textbf{MIT 75 ORIGINALAUFNAHMEN}}\pend
           
\pstart
           \noindent{}\centering{}\textcolor{gray}{\textbf{VON D\textsuperscript{R.}{ }\textcolor{blue}{EMIL MAYER}{}\ledrightnote{\textcolor{blue}{Emil Mayer}}}}\pend
           
\pstart
           \noindent{}Meinem \label{K_L03052-1v}\edtext{l.}{\lemma{\textnormal{\emph{l.}}}\Cendnote{\textnormal{lieben}}}\label{K_L03052-1h} Arthur\pend
           
\pstart
           herzlichst{\\[\baselineskip]}\spacefill\mbox{FS.}\pend
           \leftskip=0em{}
\pstart
           12. 12. 11\pend
           {\bigskip}
\pstart
           \noindent{}\centering{}\textcolor{gray}{\textbf{\textcolor{brown}{VERLAG BRÜDER ROSENBAUM}{}\ledrightnote{\textcolor{brown}{Verlagsanstalt Brüder Rosenbaum}}}}\pend
           
\pstart
           \noindent{}\textcolor{gray}{\textbf{\textcolor{pink}{WIEN}{}\ledrightnote{\textcolor{pink}{Wien}}}}\hfill \textcolor{gray}{\textbf{\textcolor{pink}{LEIPZIG}{}\ledrightnote{\textcolor{pink}{Leipzig}}}}\pend
           \endnumbering\briefempfaengerindex{Schnitzler, Arthur@\textsc{Schnitzler, Arthur}!zzzSalten, Felix@\emph{von Felix Salten}!1911-12-121@{12. 12. 1911}|)be}\mylabel{h}  \normalsize

\doendnotes{C}
\bigskip
\vfill

\clearpage

\footnotesize

\lohead{\textsc{register}}

% Definiere theindex-Environment komplett neu ohne reledmac
\makeatletter
\renewenvironment{theindex}{%
  \section*{\indexname}%
  \setlength{\parindent}{0pt}%
  \setlength{\parskip}{0pt plus 0.3pt}%
  \let\item\@idxitem
}{%
  \clearpage
}
\makeatother

\IfFileExists{\jobname-pw.ind}{\input{\jobname-pw.ind}}{}

\end{document}

      