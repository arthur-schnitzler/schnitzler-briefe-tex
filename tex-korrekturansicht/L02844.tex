%% latex-korrekturansicht-vorspann.tex
%% Vorspann für die Korrekturansicht.
%% Lädt die gemeinsame Datei latex-vorspann.tex mit gesetztem Schalter.

\newif\ifkorrekturansicht
\korrekturansichttrue

\input{../tex-inputs/latex-vorspann}


               \section[ Paul Goldmann an Arthur Schnitzler, 13. 6. 1898]{Paul Goldmann an Arthur Schnitzler, 13. 6. 1898}\nopagebreak\mylabel{v}\rehead{ }\normalsize\beginnumbering\briefempfaengerindex{Schnitzler, Arthur@\textsc{Schnitzler, Arthur}!zzzGoldmann, Paul@\emph{von Paul Goldmann}!1898-06-131@{13. 6. 1898}|(be} \toendnotes[C]{\smallbreak\pagebreak[2]} \Standort{DLA, A:Schnitzler, HS.NZ85.1.3168.}
\physDesc{Brief, 1 Blatt, 2 Seiten
\newline{}Handschrift: blaue Tinte, lateinische Kurrent}\toendnotes[C]{\smallbreak}\pstart
           \noindent{}\centering{}{\pb}\textcolor{gray}{\textbf{\begin{otherlanguage}{english}\textcolor{pink}{The Astor House}{}\ledrightnote{\textcolor{pink}{Astor House Hotel [Shanghai]}}\end{otherlanguage}}}\pend
           \pstart
           \noindent{}\centering{}\textcolor{gray}{\textbf{\begin{otherlanguage}{english}MRS.\end{otherlanguage}{ }\textcolor{blue}{E. JANSEN}{}\ledrightnote{\textcolor{blue}{Ellen Jansen}}, \begin{otherlanguage}{english}PROPRIETRESS\end{otherlanguage}.}}\pend
           \pstart
           \raggedleft{}\textcolor{gray}{\textbf{Shanghai}}13. Juni \textcolor{gray}{\textbf{189}}8.\pend
           \pstart\center{}Mein lieber Freund,\pend\pstart
           Warum höre ich ſo gar nichts von Dir? Geſtern erhielt
               ich hier Dein neues \textcolor{green}{Buch}{}\ledrightnote{→\textcolor{green}{Die Frau des Weisen. Novelletten}}.
               Tauſend Dank dafür. Ich will es leſen, aber einen Brief möchte ich auch haben.\pend
           \pstart
           Heute ſende ich ein kleines Poſt-Paket an Dich ab. Du
               findeſt darin: 1.) ein paar goldene Manſchetten-Knöpfe für Dich 2.) eine goldene
               Krawatten-Nadel für \textsc{\textcolor{blue}{Richard}{}\ledrightnote{\textcolor{blue}{Richard Beer-Hofmann}}} 3.) eine Tigerzahn-\strikeout{K}Krawatten-Nadel für \textsc{\textcolor{blue}{Leo}{}\ledrightnote{\textcolor{blue}{Leo Van-Jung}}} 4.) eine \strikeout{ſ\textcolor{gray}{×}}{ }{\pb}ſilberne \textsc{Broche} für Deine
                  \textcolor{blue}{Freundin}{}\ledrightnote{→\textcolor{blue}{Marie Reinhard}}.\pend
           \pstart
           Bitte, übergib den drei \textcolor{blue}{Anderen}{}\ledrightnote{→\textcolor{blue}{Richard Beer-Hofmann}{\newline}→\textcolor{blue}{Leo Van-Jung}{\newline}→\textcolor{blue}{Marie Reinhard}} die für ſie beſtimmten Gegenſtände
               mit vielen Grüßen von mir und nimm’ Dir \strikeout{das}{ }\strikeout{\textcolor{gray}{d}} den Deinigen mit derſelben Beigabe.\pend
           \pstart
           Ich leide furchtbar unter der Hitze, den \textsc{Mosquitos}, dem
               Heimweh, andauernden Kopfſchmerzen und meiner Unfähigkeit, zu ſchreiben.\pend
           \pstart
           Tauſend Grüße!\pend
           \pstart
           Dein {\\[\baselineskip]}\spacefill\mbox{Paul Goldmn}\pend
           \leftskip=0em{}\endnumbering\briefempfaengerindex{Schnitzler, Arthur@\textsc{Schnitzler, Arthur}!zzzGoldmann, Paul@\emph{von Paul Goldmann}!1898-06-131@{13. 6. 1898}|)be}\mylabel{h}\begin{anhang}\end{anhang}\normalsize

\doendnotes{C}
\bigskip
\vfill

\clearpage

\footnotesize

\lohead{\textsc{register}}

% Definiere theindex-Environment komplett neu ohne reledmac
\makeatletter
\renewenvironment{theindex}{%
  \section*{\indexname}%
  \setlength{\parindent}{0pt}%
  \setlength{\parskip}{0pt plus 0.3pt}%
  \let\item\@idxitem
}{%
  \clearpage
}
\makeatother

\IfFileExists{\jobname-pw.ind}{\input{\jobname-pw.ind}}{}

\end{document}

      