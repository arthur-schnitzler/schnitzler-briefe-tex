%% latex-korrekturansicht-vorspann.tex
%% Vorspann für die Korrekturansicht.
%% Lädt die gemeinsame Datei latex-vorspann.tex mit gesetztem Schalter.

\newif\ifkorrekturansicht
\korrekturansichttrue

\input{../tex-inputs/latex-vorspann}


               \section[Arthur Schnitzler an Hugo von Hofmannsthal, 23. 5. 1903]{ Arthur Schnitzler an Hugo von Hofmannsthal, 23. 5. 1903}\nopagebreak\mylabel{v}\rehead{ }\normalsize\beginnumbering\briefempfaengerindex{Hofmannsthal, Hugo von@\textsc{Hofmannsthal, Hugo von}!zzzSchnitzler, Arthur@\emph{von Arthur Schnitzler}!1903-05-231@{23. 5. 1903}|(be} \toendnotes[C]{\smallbreak\pagebreak[2]} \Standort{FDH, Hs-30885,102.}
\physDesc{Brief, 1 Blatt, 3 Seiten
\newline{}Handschrift: Bleistift, deutsche Kurrent}\buchAbdrucke{\weitereDrucke{Hugo von Hofmannsthal, Arthur Schnitzler: \emph{Briefwechsel}. Hg. Therese Nickl und Heinrich Schnitzler. Frankfurt am Main: \emph{S. Fischer} 1964, S. 168–169.} }\pstart
           \raggedleft{}{\pb}23/5 903.\pend
           \pstart
           Was ich Ihnen heute zu ſagen vergaſs, lieber Hugo, ein Frl \textcolor{blue}{\textsc{Maria Luggin}}{}\ledrightnote{\textcolor{blue}{Marie Luggin}} Vorleſerin, früher bei der \textcolor{blue}{\textsc{Ebner Eschenbach}}{}\ledrightnote{\textcolor{blue}{Marie von Ebner-Eschenbach}} glaub ich, jetzt bei der Generalin \textcolor{blue}{\textsc{v. Hueber}}{}\ledrightnote{\textcolor{blue}{Henriette von Hueber}}, von ſehr ſympathiſchem Weſen, will im Herbſt in kleinem Kreiſe
                  (\textcolor{pink}{Saal des wiſſenſch. Club}{}\ledrightnote{\textcolor{pink}{Saal des wissenschaftlichen Clubs}}{[}){]}{ }{\pb}oder ſonſt wo, ungedrucktes (oder möglichſt unbekanntes) von beſſeren \textcolor{pink}{Wien}{}\ledrightnote{\textcolor{pink}{Wien}}ern \textsc{resp}{ }\textcolor{pink}{Oeſterreichern}{}\ledrightnote{\textcolor{pink}{Österreich}} vorleſen; bat
               mich, bei Ihnen für ſie zu reden, was ich ſehr gern thue. Ich geb ihr jedenfalls was
                  we{\geminationn} ich was habe; ka{\geminationn}
               ich ihr in Ihrem {\pb}Namen Hoffnung machen?\pend
           \pstart
           Herzlichſt{\\[\baselineskip]}Ihr \spacefill\mbox{A.}\pend
           \leftskip=0em{}\endnumbering\briefempfaengerindex{Hofmannsthal, Hugo von@\textsc{Hofmannsthal, Hugo von}!zzzSchnitzler, Arthur@\emph{von Arthur Schnitzler}!1903-05-231@{23. 5. 1903}|)be}\mylabel{h}  \normalsize

\doendnotes{C}
\bigskip
\vfill

\clearpage

\footnotesize

\lohead{\textsc{register}}

% Definiere theindex-Environment komplett neu ohne reledmac
\makeatletter
\renewenvironment{theindex}{%
  \section*{\indexname}%
  \setlength{\parindent}{0pt}%
  \setlength{\parskip}{0pt plus 0.3pt}%
  \let\item\@idxitem
}{%
  \clearpage
}
\makeatother

\IfFileExists{\jobname-pw.ind}{\input{\jobname-pw.ind}}{}

\end{document}

      