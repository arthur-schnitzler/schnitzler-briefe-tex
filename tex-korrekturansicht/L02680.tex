%% latex-korrekturansicht-vorspann.tex
%% Vorspann für die Korrekturansicht.
%% Lädt die gemeinsame Datei latex-vorspann.tex mit gesetztem Schalter.

\newif\ifkorrekturansicht
\korrekturansichttrue

\input{../tex-inputs/latex-vorspann}


               \section[Paul Goldmann an Arthur Schnitzler, 20. 3. 1899]{ Paul Goldmann an Arthur Schnitzler, 20. 3. 1899}\nopagebreak\mylabel{v}\rehead{ }\normalsize\beginnumbering\briefempfaengerindex{Schnitzler, Arthur@\textsc{Schnitzler, Arthur}!zzzGoldmann, Paul@\emph{von Paul Goldmann}!1899-03-201@{20. 3. 1899}|(be} \toendnotes[C]{\smallbreak\pagebreak[2]} \Standort{DLA, A:Schnitzler, HS.NZ85.1.3169.}
\physDesc{Telegramm1 Blatt, 1 Seite
\newline{}maschinell\newline{}Versand: 1) Stempel: »\nobreak{}\oindex{IX., Alsergrund@\textbf{IX., Alsergrund}, \emph{Bezirk (A.BZK)}|pwk}\textcolor{gray}{W}ien 9/2, 20 III 99, 11 50V\nobreak{}«.  2) Stempel: »\nobreak{}20 3 1899, \textcolor{blue}{Ulrich}\nobreak{}«. \newline{}Ordnung: beschnitten }\pstart{}{\pb}\textcolor{pink}{neuntbezirk frankgasze 1}{}\ledrightnote{\textcolor{pink}{Frankgasse}}.+\pend{}{\bigskip}\pstart
           \noindent{}\centering{}{\pb}v \textcolor{pink}{frankfurtmain}{}\ledrightnote{\textcolor{pink}{Frankfurt am Main}}
               928 38 20/3{ }9/55,– m =\pend
           \pstart
           \noindent{}tief erschuettert druecke ich dir die hand im innigsten
            bejlejd. es ist furchtbar und ich finde keine worte. und doch darfst
            du selbst jetzt nicht glauben dass alles zu ende ist.\pend
           \pstart \spacefill\mbox{goldmann}\pend{}\endnumbering\briefempfaengerindex{Schnitzler, Arthur@\textsc{Schnitzler, Arthur}!zzzGoldmann, Paul@\emph{von Paul Goldmann}!1899-03-201@{20. 3. 1899}|)be}\mylabel{h}\begin{anhang}\end{anhang}\normalsize

\doendnotes{C}
\bigskip
\vfill

\clearpage

\footnotesize

\lohead{\textsc{register}}

% Definiere theindex-Environment komplett neu ohne reledmac
\makeatletter
\renewenvironment{theindex}{%
  \section*{\indexname}%
  \setlength{\parindent}{0pt}%
  \setlength{\parskip}{0pt plus 0.3pt}%
  \let\item\@idxitem
}{%
  \clearpage
}
\makeatother

\IfFileExists{\jobname-pw.ind}{\input{\jobname-pw.ind}}{}

\end{document}

      