%% latex-korrekturansicht-vorspann.tex
%% Vorspann für die Korrekturansicht.
%% Lädt die gemeinsame Datei latex-vorspann.tex mit gesetztem Schalter.

\newif\ifkorrekturansicht
\korrekturansichttrue

\input{../tex-inputs/latex-vorspann}


\renewcommand{\erwaehntePersonen}{Personen: Hermann Bahr, Elsa Plessner, Jean-Jacques Rousseau, Johanna Schopenhauer}
\renewcommand{\erwaehnteInstitutionen}{Institutionen: Die Zeit. Wiener Wochenschrift}
\renewcommand{\erwaehnteOrte}{Orte: Fröschelgasse 6, London, Wien}
\renewcommand{\erwaehnteWerke}{Werke: Der gläserne Käfig. Eine Parabel, Gabriele. Ein Roman, Julie oder Die neue Heloise, Meine Freundin Klothilde, Sämmtliche Schriften, Warten}
\section[Elsa Plessner an Arthur Schnitzler, 30. 5. {[}1897{]}]{Elsa Plessner an Arthur Schnitzler, 30. 5. {[}1897{]}}
\nopagebreak\mylabel{v}
\rehead{ }\normalsize\beginnumbering\briefempfaengerindex{Schnitzler, Arthur@\textsc{Schnitzler, Arthur}!zzzPlessner, Elsa@\emph{von Elsa Plessner}!1897-05-301@{30. 5. {[}1897{]}}|(be}
\toendnotes[C]{\smallbreak\pagebreak[2]}\Standort{DLA, A:Schnitzler, 85.1.4198.}
\physDesc{1 Blatt, 4 Seiten, 1190 Zeichen
\newline{}Handschrift: , lateinische Kurrent}
\buchAbdrucke{\weitereDrucke{Hermann Bahr, Arthur Schnitzler: \emph{Briefwechsel, Aufzeichnungen, Dokumente
                                (1891–1931)}. Hg. Kurt Ifkovits und Martin Anton Müller. Göttingen: \emph{Wallstein} 2018, S. 144.} }\toendnotes[C]{\smallbreak}
\pstart
           \textcolor{pink}{Wien-Sievering, Fröschelgasse 6}{}\ledrightnote{\textcolor{pink}{Fröschelgasse 6}}
                        den 30. V.\pend
           
\pstart\center{}Verehrter Herr Doctor!\pend\vspace{0.5em}
\pstart
           Herzlichsten Dank für Ihre liebenswürdigen Zeilen bez. »\textcolor{green}{Freundin Clotilde}{}\ledrightnote{\textcolor{green}{Meine Freundin Klothilde}}«. Anweisung ist bereits befolgt und
                    dieses Opus liegt schon in Dialogform vor. –\pend
           
\pstart
           Zweck meines heutigen Schreibens ist, Sie, verehrter Herr Doctor davon zu
                    benachrichtigen, dass \textcolor{blue}{H. Bahr}{}\ledrightnote{\textcolor{blue}{Hermann Bahr}} den »\textcolor{green}{gläsernen Käfig}{}\ledrightnote{\textcolor{green}{Der gläserne Käfig. Eine Parabel}}« für die {\pb}»\textcolor{brown}{Zeit}{}\ledrightnote{\textcolor{brown}{Die Zeit. Wiener Wochenschrift}}« acceptirt hat, was er mit gestern in
                    einer überaus liebenswürdigen Epistel anzeigte, in welcher er auch über »\textcolor{green}{Warten}{}\ledrightnote{\textcolor{green}{Warten}}« sich außerordentlich günstig
                    ausspricht. – – – –\pend
           
\pstart
           Dieses angenehme Resultat verdanke ich wiederum nicht zum kleinsten Theil Ihrer
                    Befürwortung! – – »\label{K_L03696-1v}\edtext{Ich hab’s aber
                    immer gesagt – Sie sind ein Engel!}{\lemma{\textnormal{\emph{Ich … Engel!}}}\Cendnote{\textnormal{Mit der Ergänzung »{\dots} an Gemüth« zu finden in \textcolor{blue}{Johanna Schopenhauer}: \emph{\textcolor{green}{Gabriele}}
                        (1818–1819, \emph{\textcolor{green}{Sämmtliche
                                Schriften}}. Neunter Band: \emph{\textcolor{green}{Gabriele}}. Dritter Theil. Leipzig:
                                \emph{F. A. Brockhaus}, Frankfurt am
                                Main:
                                \emph{J. D. Sauerländer}{ }1830, S. 223), womöglich dort
                        schon ein Reflex auf »\begin{otherlanguage}{french}Ah! je l’ai dit cent
                                fois, tu es un ange du Ciel, ma Julie!\end{otherlanguage}« (\textcolor{blue}{Jean-Jacques Rousseau}: \emph{\textcolor{green}{Julie ou la Nouvelle Héloïse}}, Brief
                        43).}}}\label{}« Pardon – ich freue mich so sehr, darum dieser schauderhaft
                        »\textcolor{gray}{N}\textcolor{gray}{×}\-\textcolor{gray}{×}e«sche Ausspruch!!\pend
           
\pstart
           Ich hatte so ein bisschen Aufmunterung sehr nöthig!!–\pend
           
\pstart
           {\pb}Ich hoffe und wünsche, dass Sie sich in \label{K_L03696-2v}\edtext{\textcolor{pink}{London}{}\ledrightnote{\textcolor{pink}{London}}}{\lemma{\textnormal{\emph{London}}}\Cendnote{\textnormal{Er hielt sich vom 26. 5. 1897 bis
                        zum 1. 6. 1897 in \textcolor{pink}{London} auf.}}}\label{} recht wohl und
                    vergnügt befinden mögen und uns als künstlerische Ausbeute Ihrer Reise recht
                    bald eine Reihe neuer Arbeiten bescheeren mögen, mit denen Sie selbst zufrieden
                    sind.– Das ist das Schönste, was ich einem Künstler wünschen kann! Nicht? –
                    –\pend
           
\pstart
           Also nochmals, herz{\pb}lichsten Dank von Ihrer Sie ehrlich und aufrichtig
               verehrenden{\\[\baselineskip]}\spacefill\mbox{ElsaPlessner.}\pend
           \leftskip=0em{}\endnumbering\briefempfaengerindex{Schnitzler, Arthur@\textsc{Schnitzler, Arthur}!zzzPlessner, Elsa@\emph{von Elsa Plessner}!1897-05-301@{30. 5. {[}1897{]}}|)be}\mylabel{h}
\begin{anhang}
\end{anhang}\normalsize

\doendnotes{C}
\bigskip
\vfill

\clearpage

\footnotesize

\lohead{\textsc{register}}

% Definiere theindex-Environment komplett neu ohne reledmac
\makeatletter
\renewenvironment{theindex}{%
  \section*{\indexname}%
  \setlength{\parindent}{0pt}%
  \setlength{\parskip}{0pt plus 0.3pt}%
  \let\item\@idxitem
}{%
  \clearpage
}
\makeatother

\IfFileExists{\jobname-pw.ind}{\input{\jobname-pw.ind}}{}

\end{document}

      