%% latex-korrekturansicht-vorspann.tex
%% Vorspann für die Korrekturansicht.
%% Lädt die gemeinsame Datei latex-vorspann.tex mit gesetztem Schalter.

\newif\ifkorrekturansicht
\korrekturansichttrue

\input{../tex-inputs/latex-vorspann}


\section[Elsa Plessner an Arthur Schnitzler, 30. 5. {[}1897{]}]{L03696 Elsa Plessner an Arthur Schnitzler, 30. 5. {[}1897{]}}
\nopagebreak\mylabel{L03696v}
\rehead{ }\normalsize\beginnumbering\briefempfaengerindex{Schnitzler, Arthur@\textsc{Schnitzler, Arthur}!zzzPlessner, Elsa@\emph{von Elsa Plessner}!1897-05-301@{30. 5. {[}1897{]}}|(be}
\toendnotes[C]{\smallbreak\pagebreak[2]}
\correspDesc{Versand  durch Elsa Plessner am 30. 5. [1897] in Wien
\newline{}Erhalt  durch Arthur Schnitzler im Zeitraum [31. 5. 1897 – 4. 6. 1897?] in London}\toendnotes[C]{\smallbreak}
\Standort{DLA, A:Schnitzler, 85.1.4198.}
\physDesc{Brief, 1 Blatt, 4 Seiten, 1190 Zeichen
\newline{}Handschrift: schwarze Tinte, lateinische Kurrent
\newline{}Schnitzler: 1) mit Bleistift Jahreszahl beim Datum ergänzt: »97«  2) mit rotem Buntstift eine Unterstreichung}
\buchAbdrucke{\weitereDrucke{Hermann Bahr, Arthur Schnitzler: \emph{Briefwechsel, Aufzeichnungen, Dokumente
                                (1891–1931)}. Herausgegeben von Kurt Ifkovits und Martin Anton Müller. Göttingen: \emph{Wallstein} 2018, S. 144.} }\toendnotes[C]{\smallbreak}
\pstart
           \raggedleft{}{\pb}\textcolor{pink}{Wien-Sievering, Fröschelgasse 6}\oindex{Wien@\textbf{Wien}!XIX., Döbling@\textbf{XIX., Döbling}!Fröschelgasse 6@\textbf{Fröschelgasse 6}, \emph{Wohngebäude}|pw}{}\ledrightnote{\textcolor{pink}{Fröschelgasse 6}}{\\}den
                            30. V.\pend
           
\pstart\center{}Verehrter Herr Doctor!\pend\vspace{0.5em}
\pstart
           Herzlichsten Dank für Ihre \label{K_L03696-1v}\edtext{liebenswürdigen Zeilen}{\lemma{\textnormal{\emph{liebenswürdigen Zeilen}}}\Cendnote{\textnormal{nicht
                        überliefert}}}\label{K_L03696-1} bez. »\textcolor{green}{Freundin
                        Clotilde}\pwindex{Plessner, Elsa 22.\,8.\,1875 Wien – 7.\,5.\,1932 Alicante@\textsc{Plessner, Elsa} (22.\,8.\,1875 Wien – 7.\,5.\,1932 Alicante), \emph{Schriftstellerin}!Meine Freundin Clotilde@\strich\emph{Meine Freundin Clotilde}|pw}{}\ledrightnote{\textcolor{green}{Meine Freundin Clotilde}}«. Anweisung ist bereits befolgt und dieses Opus liegt schon in
                    Dialogform vor. –\pend
           
\pstart
           Zweck meines heutigen Schreibens ist, Sie, verehrter Herr Doctor davon zu
                    benachrichtigen, dass \textcolor{blue}{H. Bahr}\pwindex{Bahr, Hermann 19.\,7.\,1863 Linz – 15.\,1.\,1934 München@\textsc{Bahr, Hermann} (19.\,7.\,1863 Linz – 15.\,1.\,1934 München), \emph{Schriftsteller, Kritiker}|pw}{}\ledrightnote{\textcolor{blue}{Hermann Bahr}} den »\textcolor{green}{gläsernen Käfig}\pwindex{Plessner, Elsa 22.\,8.\,1875 Wien – 7.\,5.\,1932 Alicante@\textsc{Plessner, Elsa} (22.\,8.\,1875 Wien – 7.\,5.\,1932 Alicante), \emph{Schriftstellerin}!gläserne Käfig. Eine Parabel@\strich\emph{Der gläserne Käfig. Eine Parabel}|pw}{}\ledrightnote{\textcolor{green}{Der gläserne Käfig. Eine Parabel}}« für die {\pb}»\textcolor{brown}{Zeit}\orgindex{Zeit. Wiener Wochenschrift@Die Zeit. Wiener Wochenschrift|pw}{}\ledrightnote{\textcolor{brown}{Die Zeit. Wiener Wochenschrift}}«
                    acceptirt hat, was er mit gestern in einer überaus liebenswürdigen Epistel
                    anzeigte, in welcher er auch über »\textcolor{green}{Warten}\pwindex{Plessner, Elsa 22.\,8.\,1875 Wien – 7.\,5.\,1932 Alicante@\textsc{Plessner, Elsa} (22.\,8.\,1875 Wien – 7.\,5.\,1932 Alicante), \emph{Schriftstellerin}!Warten. Novelle@\strich\emph{Warten. Novelle}|pw}{}\ledrightnote{\textcolor{green}{Warten. Novelle}}«
                    sich außerordentlich günstig ausspricht. – – – –\pend
           
\pstart
           Dieses angenehme Resultat verdanke ich wiederum nicht zum kleinsten Theil Ihrer
                    Befürwortung!! – – »\label{K_L03696-2v}\edtext{Ich hab’s aber
                    immer gesagt – Sie sind ein Engel!}{\lemma{\textnormal{\emph{Ich … Engel!}}}\Cendnote{\textnormal{Mit der Ergänzung »{\dots} an Gemüth« zu finden in \textcolor{blue}{Johanna Schopenhauer}\pwindex{Schopenhauer, Johanna 9.\,7.\,1766 Danzig – 16.\,4.\,1838 Jena@\textsc{Schopenhauer, Johanna} (9.\,7.\,1766 Danzig – 16.\,4.\,1838 Jena), \emph{Schriftstellerin}|pwk}: \emph{\textcolor{green}{Gabriele}\pwindex{Schopenhauer, Johanna 9.\,7.\,1766 Danzig – 16.\,4.\,1838 Jena@\textsc{Schopenhauer, Johanna} (9.\,7.\,1766 Danzig – 16.\,4.\,1838 Jena), \emph{Schriftstellerin}!Gabriele. Ein Roman@\strich\emph{Gabriele. Ein Roman}|pwk}} (1818–1819, \emph{\textcolor{green}{Sämmtliche Schriften}\pwindex{Schopenhauer, Johanna 9.\,7.\,1766 Danzig – 16.\,4.\,1838 Jena@\textsc{Schopenhauer, Johanna} (9.\,7.\,1766 Danzig – 16.\,4.\,1838 Jena), \emph{Schriftstellerin}!Sämmtliche Schriften@\strich\emph{Sämmtliche Schriften}|pwk}}. Neunter Band: \emph{\textcolor{green}{Gabriele}\pwindex{Schopenhauer, Johanna 9.\,7.\,1766 Danzig – 16.\,4.\,1838 Jena@\textsc{Schopenhauer, Johanna} (9.\,7.\,1766 Danzig – 16.\,4.\,1838 Jena), \emph{Schriftstellerin}!Gabriele. Ein Roman@\strich\emph{Gabriele. Ein Roman}|pwk}}. Dritter Theil.
                            Leipzig: \emph{F. A. Brockhaus},
                            Frankfurt am
                                Main: \emph{J. D.
                                Sauerländer}{ }1830, S. 223), womöglich dort schon ein Reflex auf »\begin{otherlanguage}{french}Ah! je l’ai dit cent fois, tu es un ange du Ciel,
                                ma Julie!\end{otherlanguage}« (\textcolor{blue}{Jean-Jacques Rousseau}\pwindex{Rousseau, Jean-Jacques 28.\,6.\,1712 Genf – 2.\,7.\,1778 Ermenonville@\textsc{Rousseau, Jean-Jacques} (28.\,6.\,1712 Genf – 2.\,7.\,1778 Ermenonville), \emph{Philosoph}|pwk}: \emph{\textcolor{green}{Julie ou la Nouvelle Héloïse}\pwindex{Rousseau, Jean-Jacques 28.\,6.\,1712 Genf – 2.\,7.\,1778 Ermenonville@\textsc{Rousseau, Jean-Jacques} (28.\,6.\,1712 Genf – 2.\,7.\,1778 Ermenonville), \emph{Philosoph}!Julie oder Die neue Heloise@\strich\emph{Julie oder Die neue Heloise}|pwk}}, Brief
                        43).}}}\label{K_L03696-2}« Pardon – ich freue mich so sehr, darum dieser schauderhaft
                        »\textcolor{gray}{×}nne«sche Ausspruch!!\pend
           
\pstart
           Ich hatte so ein bisschen Aufmunterung sehr nöthig!!–\pend
           
\pstart
           {\pb}Ich hoffe und wünsche, dass Sie sich in \label{K_L03696-3v}\edtext{\textcolor{pink}{London}\oindex{London@\textbf{London}, \emph{Hauptstadt}|pw}{}\ledrightnote{\textcolor{pink}{London}}}{\lemma{\textnormal{\emph{London}}}\Cendnote{\textnormal{Er hielt sich vom 26. 5. 1897 bis
                        zum 1. 6. 1897 in \textcolor{pink}{London}\oindex{London@\textbf{London}, \emph{Hauptstadt}|pwk} auf.}}}\label{K_L03696-3} recht wohl und
                    vergnügt befinden mögen und uns als künstlerische Ausbeute Ihrer Reise recht
                    bald eine Reihe neuer Arbeiten bescheeren mögen, mit denen Sie selbst zufrieden
                    sind.– Das ist das Schönste, was ich einem Künstler wünschen kann! Nicht? –
                    –\pend
           
\pstart
           Also nochmals, herz{\pb}lichsten Dank von Ihrer
                    Sie ehrlich und aufrichtig verehrenden{\\[\baselineskip]}\spacefill\mbox{ElsaPlessner.}\pend
           \leftskip=0em{}\selectlanguage{ngerman}\endnumbering\briefempfaengerindex{Schnitzler, Arthur@\textsc{Schnitzler, Arthur}!zzzPlessner, Elsa@\emph{von Elsa Plessner}!1897-05-301@{30. 5. {[}1897{]}}|)be}\mylabel{L03696h}  \normalsize

\doendnotes{C}
\bigskip
\vfill

\clearpage

\footnotesize

\lohead{\textsc{register}}

% Definiere theindex-Environment komplett neu ohne reledmac
\makeatletter
\renewenvironment{theindex}{%
  \section*{\indexname}%
  \setlength{\parindent}{0pt}%
  \setlength{\parskip}{0pt plus 0.3pt}%
  \let\item\@idxitem
}{%
  \clearpage
}
\makeatother

\IfFileExists{\jobname-pw.ind}{\input{\jobname-pw.ind}}{}

\end{document}

      