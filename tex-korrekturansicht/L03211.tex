%% latex-korrekturansicht-vorspann.tex
%% Vorspann für die Korrekturansicht.
%% Lädt die gemeinsame Datei latex-vorspann.tex mit gesetztem Schalter.

\newif\ifkorrekturansicht
\korrekturansichttrue

\input{../tex-inputs/latex-vorspann}


\renewcommand{\erwaehntePersonen}{Personen: Otto Brahm, Eva Marie Goldmann, Rudolf Gussmann, Amalia Gussmann, Johanna Gussmann, Carl Loewe, Raphael Löwenfeld, Maurice Maeterlinck, Felix Salten, Olga Schnitzler, Elisabeth Steinrück}
\renewcommand{\erwaehnteInstitutionen}{Institutionen: Schiller-Theater}
\renewcommand{\erwaehnteOrte}{Orte: Berlin, Budapest, Dessauer Straße, Deutsches Theater Berlin, Lustspieltheater, Trafoi, Wien}
\renewcommand{\erwaehnteWerke}{Werke: Der Schleier der Beatrice. Schauspiel in fünf Akten, Der einsame Weg. Schauspiel in fünf Akten, Die Weissagung, Die griechische Tänzerin. Novellette, Heinrich der Vogler, Lebendige Stunden. Vier Einakter, Monna Vanna. Schauspiel in drei Akten, Tom der Reimer}
\section[ Paul Goldmann an Arthur Schnitzler, 16. 6. {[}1902{]}]{Paul Goldmann an Arthur Schnitzler, 16. 6. {[}1902{]}}
\nopagebreak\mylabel{v}
\rehead{ }\normalsize\beginnumbering\briefempfaengerindex{Schnitzler, Arthur@\textsc{Schnitzler, Arthur}!zzzGoldmann, Paul@\emph{von Paul Goldmann}!1902-06-162@{16. 6. {[}1902{]}}|(be}
\toendnotes[C]{\smallbreak\pagebreak[2]}\Standort{DLA, A:Schnitzler, HS.NZ85.1.3172.}
\physDesc{Brief, 1 Blatt, 4 Seiten
\newline{}Handschrift: blaue Tinte, deutsche Kurrent
\newline{}Schnitzler: 1) mit Bleistift das Jahr »{[}1{]}902« vermerkt  2) mit rotem Buntstift eine Unterstreichung}\toendnotes[C]{\smallbreak}
\pstart
           \noindent{}\raggedleft{}{\pb}\textcolor{pink}{\textcolor{gray}{\textbf{DESSAUERSTRASSE 19}}}{}\ledrightnote{\textcolor{pink}{Dessauer Straße}}\pend
           
\pstart
           \textcolor{pink}{Berlin}{}\ledrightnote{\textcolor{pink}{Berlin}}, 16. Juni.\pend
           
\pstart\center{}Mein lieber Freund,\pend
\pstart
           Ich habe mich ſehr gefreut, wieder von Dir zu hören. Die \label{K_L03211-1v}\edtext{\textcolor{pink}{Budapeſt}{}\ledrightnote{\textcolor{pink}{Budapest}}er Reiſe}{\lemma{\textnormal{\emph{Budapeſter Reiſe}}}\Cendnote{\textnormal{Auf \textcolor{blue}{Otto Brahm}s
                  Einladung hin war \textcolor{blue}{Schnitzler} am 7. 6. 1902 und 8. 6. 1902 in \textcolor{pink}{Budapest} gewesen, wo im \textcolor{pink}{Lustspielhaus} die \emph{\textcolor{green}{Lebendigen Stunden}} gegeben wurden. Vgl. \emph{Der
                        Briefwechsel Arthur Schnitzler — Otto Brahm}. Vollständige Ausgabe.
                     Herausgegeben, eingeleitet und erläutert von Oskar Seidlin.
                     Tübingen: \emph{Niemeyer}{ }1975, S. 123.}}}\label{K_L03211-1h} muß recht intereſſant
               geweſen ſein. Hat ſich \label{K_L03211-2v}\edtext{\textsc{\textcolor{blue}{Brahm}{}\ledrightnote{\textcolor{blue}{Otto Brahm}}} über die »\textsc{\textcolor{green}{Beatrice}{}\ledrightnote{\textcolor{green}{Der Schleier der Beatrice. Schauspiel in fünf Akten}}}« entſchieden}{\lemma{\textnormal{\emph{Brahm … entſchieden}}}\Cendnote{\textnormal{\emph{\textcolor{green}{Der Schleier der Beatrice}} wurde von \textcolor{blue}{Otto Brahm} für das \textcolor{pink}{Deutsche Theater Berlin} angenommen und feierte dort am 7. 3. 1903
                  Premiere.}}}\label{K_L03211-2h}? Wenn er die \label{K_L03211-3v}\edtext{»\textsc{\textcolor{green}{Monna Vanna}{}\ledrightnote{\textcolor{green}{Monna Vanna. Schauspiel in drei Akten}}}« von \textsc{\textcolor{blue}{Maeterlinck}{}\ledrightnote{\textcolor{blue}{Maurice Maeterlinck}}} gibt}{\lemma{\textnormal{\emph{»Monna … gibt}}}\Cendnote{\textnormal{\textcolor{blue}{Maurice Maeterlinck}s \emph{\textcolor{green}{Monna Vanna}} wurde ab dem 12. 10. 1902 über 100 Mal im \textcolor{pink}{Deutschen
                     Theater Berlin} aufgeführt. Siehe auch \emph{Der
                        Briefwechsel Arthur Schnitzler — Otto Brahm}. Vollständige Ausgabe.
                     Herausgegeben, eingeleitet und erläutert von Oskar Seidlin.
                     Tübingen: \emph{Niemeyer}{ }1975, S. 123–131 und A. S.: \emph{Tagebuch}, 24. 11. 1902.}}}\label{K_L03211-3h}, muß er auch die »\textsc{\textcolor{green}{Beatrice}{}\ledrightnote{\textcolor{green}{Der Schleier der Beatrice. Schauspiel in fünf Akten}}}« geben können. Dein \label{K_L03211-4v}\edtext{\textcolor{green}{Stück}{}\ledrightnote{{$\rightarrow$}\textcolor{green}{Der einsame Weg. Schauspiel in fünf Akten}}}{\lemma{\textnormal{\emph{Stück}}}\Cendnote{\textnormal{\textcolor{blue}{Schnitzler} hatte die Konzeption für \emph{\textcolor{green}{Der einsame Weg}} am 2. 6. 1902
                  abgeschlossen und begann es am 9. 8. 1902 zu schreiben.}}}\label{K_L03211-4h} laß’ nur ruhig noch warten, bis Du
               ordentlich Luſt bekommſt, es zu ſchreiben. Daß Du kurze \label{K_L03211-5v}\edtext{\textcolor{green}{Geſchichten}{}\ledrightnote{{$\rightarrow$}\textcolor{green}{Die griechische Tänzerin. Novellette}{\newline}{$\rightarrow$}\textcolor{green}{Die Weissagung}}}{\lemma{\textnormal{\emph{Geſchichten}}}\Cendnote{\textnormal{Bezug auf \emph{\textcolor{green}{Die griechische Tänzerin}} und \emph{\textcolor{green}{Die Weissagung}}, die \textcolor{blue}{Schnitzler} am 7. 6. 1902 neu begonnen hatte}}}\label{K_L03211-5h} ſchreibſt, gefällt mir ſehr.
               Ich glaube, auf dieſem Gebiete iſt viel für Dich zu holen.\pend
           
\pstart
           Daß ſich der \textcolor{blue}{Vater}{}\ledrightnote{{$\rightarrow$}\textcolor{blue}{Rudolf Gussmann}} der \textcolor{blue}{Mädels}{}\ledrightnote{{$\rightarrow$}\textcolor{blue}{Olga Schnitzler}{\newline}{$\rightarrow$}\textcolor{blue}{Elisabeth Steinrück}}{ }{\pb}\label{K_L03211-6v}\edtext{verheiratet}{\lemma{\textnormal{\emph{verheiratet}}}\Cendnote{\textnormal{\textcolor{blue}{Amalia Gussmann}, die Mutter von \textcolor{blue}{Olga} und \textcolor{blue}{Elisabeth}, war am 14. 11. 1899 verstorben.
                     \textcolor{blue}{Rudolf Gussmann}s zweite Frau war \textcolor{blue}{Johanna Gussmann (geb. Steiner)}. Auch sie
                  verstarb nur wenige Jahre nach der Hochzeit, womöglich im Juni 1905.}}}\label{K_L03211-6h} hat, iſt zugleich komiſch und gemein. Dieſer \textcolor{blue}{Hundsfott}{}\ledrightnote{{$\rightarrow$}\textcolor{blue}{Rudolf Gussmann}}! Wie hat ſich die
                  \label{K_L03211-7v}\edtext{Geſchichte mit dem Advokaten }{\lemma{\textnormal{\emph{Geſchichte … Advokaten}}}\Cendnote{\textnormal{Bezug unklar}}}\label{K_L03211-7h} abgewickelt?\pend
           
\pstart
           Was \textsc{\textcolor{blue}{Liesl}{}\ledrightnote{\textcolor{blue}{Elisabeth Steinrück}}} anlangt, ſo bitte ich Dich, einmal mit einem Donnerwetter dazwiſchenzufahren.
               Den an mich gerichteten \label{K_L03211-8v}\edtext{Brief von \textsc{\textcolor{blue}{Löwenfeld}{}\ledrightnote{\textcolor{blue}{Raphael Löwenfeld}}}}{\lemma{\textnormal{\emph{Brief von Löwenfeld}}}\Cendnote{\textnormal{\emph{Deutsches Literaturarchiv Marbach},
                     HS.1985.1.05246,5. \textcolor{blue}{Elisabeth
                     Gussmann} schloss am 2. 8. 1902 einen
                  Vertrag mit dem \emph{\textcolor{brown}{Schiller-Theater}} ab. Das
                  Beschäftigungsverhältnis dauerte von 1. 9. 1902 bis
                     30. 6. 1907.}}}\label{K_L03211-8h} haſt Du wohl geleſen? Ich
               ſchließe daraus, daß eine Möglichkeit des Engagements am \textcolor{brown}{Schillertheater}{}\ledrightnote{\textcolor{brown}{Schiller-Theater}} beſteht, wenn man nur ein wenig nachhilft.
               Ich bin gern bereit, nachzuhelfen\strikeout{,} und den
               perſönlichen Beſuch zu machen, zu dem er mich auffordert. Aber vorher muß ich wiſſen,
               ob \textsc{\textcolor{blue}{Liesl}{}\ledrightnote{\textcolor{blue}{Elisabeth Steinrück}}} ihm geſchrieben hat, nachdem ſie mir bereits {\pb}einmal \strikeout{geſ} vorgeſchwindelt hat, ſie habe ihm
               geſchrieben, ohne es gethan zu haben. Ich warte alſo auf Antwort und bekomme keine.
               Veranlaſſe doch, \strikeout{\textcolor{gray}{×}\-\textcolor{gray}{×}\-\textcolor{gray}{×}\-\textcolor{gray}{×}\-\textcolor{gray}{×}} daß die junge \strikeout{\textcolor{blue}{Dam\textcolor{gray}{e}}{}\ledrightnote{\textcolor{blue}{Elisabeth Steinrück}}}{ }\textcolor{blue}{Dame}{}\ledrightnote{\textcolor{blue}{Elisabeth Steinrück}} ſich aufrafft und zur Feder greift, und
               ſage ihr, bitte, in meinem Namen, daß ich wüthend bin und daß man mit ſolch’ einer
               verfluchten Schlamperei keine Engagements bekommt!\pend
           
\pstart
           Grüße \textsc{\textcolor{blue}{Olga}{}\ledrightnote{\textcolor{blue}{Olga Schnitzler}}} recht herzlich. Ich hoffe, ſie übt die \textsc{\textcolor{blue}{Löwe}{}\ledrightnote{\textcolor{blue}{Carl Loewe}}}’ſchen \textcolor{green}{Balladen}{}\ledrightnote{{$\rightarrow$}\textcolor{green}{Tom der Reimer}{\newline}{$\rightarrow$}\textcolor{green}{Heinrich der Vogler}} (\textcolor{green}{Tom der Reimer}{}\ledrightnote{\textcolor{green}{Tom der Reimer}}, \textcolor{green}{Heinrich der Vogler}{}\ledrightnote{\textcolor{green}{Heinrich der Vogler}}). Wenn ich nach \textcolor{pink}{Wien}{}\ledrightnote{\textcolor{pink}{Wien}} komme, will ich ſie vorgeſungen haben.\pend
           
\pstart
           {\pb}Meine Pläne bleiben einſtweilen die alten: \label{K_L03211-12v}\edtext{Zwiſchen 20. u. 25. Juli{ }\textcolor{pink}{Wien}{}\ledrightnote{\textcolor{pink}{Wien}}, dann \textsc{\textcolor{pink}{Trafoi}{}\ledrightnote{\textcolor{pink}{Trafoi}}}}{\lemma{\textnormal{\emph{Zwiſchen … Trafoi}}}\Cendnote{\textnormal{nicht geschehen, siehe Paul Goldmann an Arthur Schnitzler, 25. 7. [1902]}}}\label{K_L03211-12h}. Von \label{K_L03211-11v}\edtext{Fräulein \textsc{\textcolor{blue}{F.}{}\ledrightnote{{$\rightarrow$}\textcolor{blue}{Eva Marie Goldmann}}}}{\lemma{\textnormal{\emph{Fräulein F.}}}\Cendnote{\textnormal{womöglich \textcolor{blue}{Goldmann}s spätere Frau \textcolor{blue}{Eva Marie}, geboren \textcolor{blue}{Fränkel},
                  geschieden \textcolor{blue}{Kobler}}}}\label{K_L03211-11h} erhalte ich hier und da einen Brief. Aber das Schreiben iſt eine dumme Sache.
               Die Fäden ſind abgeriſſen. Sie ſchreibt mir übrigens, daß ſie öfter mit \textsc{\textcolor{blue}{Salten}{}\ledrightnote{\textcolor{blue}{Felix Salten}}} zuſammen iſt.\pend
           
\pstart
           Schreib’ mir bald wieder und ſei vielmals und von Herzen gegrüßt! {\\[\baselineskip]}Dein {\\[\baselineskip]}\spacefill\mbox{Paul Goldmnn}\pend
           \leftskip=0em{}\endnumbering\briefempfaengerindex{Schnitzler, Arthur@\textsc{Schnitzler, Arthur}!zzzGoldmann, Paul@\emph{von Paul Goldmann}!1902-06-162@{16. 6. {[}1902{]}}|)be}\mylabel{h}
\begin{anhang}
\end{anhang}\normalsize

\doendnotes{C}
\bigskip
\vfill

\clearpage

\footnotesize

\lohead{\textsc{register}}

% Definiere theindex-Environment komplett neu ohne reledmac
\makeatletter
\renewenvironment{theindex}{%
  \section*{\indexname}%
  \setlength{\parindent}{0pt}%
  \setlength{\parskip}{0pt plus 0.3pt}%
  \let\item\@idxitem
}{%
  \clearpage
}
\makeatother

\IfFileExists{\jobname-pw.ind}{\input{\jobname-pw.ind}}{}

\end{document}

      