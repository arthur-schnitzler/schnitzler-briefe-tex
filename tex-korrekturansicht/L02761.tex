%% latex-korrekturansicht-vorspann.tex
%% Vorspann für die Korrekturansicht.
%% Lädt die gemeinsame Datei latex-vorspann.tex mit gesetztem Schalter.

\newif\ifkorrekturansicht
\korrekturansichttrue

\input{../tex-inputs/latex-vorspann}


\renewcommand{\erwaehntePersonen}{Personen: Ottilie Salten}
\renewcommand{\erwaehnteOrte}{Orte: Felberstraße, Frankgasse 1, IX., Alsergrund, Penzinger Viadukt, Wien}
\renewcommand{\erwaehnteWerke}{}
\section[ Felix Salten an Arthur Schnitzler, 8. 7. 1897]{Felix Salten an Arthur Schnitzler, 8. 7. 1897}
\nopagebreak\mylabel{v}
\rehead{ }\normalsize\beginnumbering\briefempfaengerindex{Schnitzler, Arthur@\textsc{Schnitzler, Arthur}!zzzSalten, Felix@\emph{von Felix Salten}!1898-07-081@{8. 7. 1898}|(be}
\toendnotes[C]{\smallbreak\pagebreak[2]}\Standort{CUL, Schnitzler, B 89, A 2.}
\physDesc{Postkarte, 138 Zeichen
\newline{}Handschrift: Bleistift, lateinische Kurrent
\newline{}Versand: 1) Stempel: »\nobreak{}\oindex{IX., Alsergrund@\textbf{IX., Alsergrund}, \emph{A.ADM3}|pwk}Wien 9/3 7\textcolor{gray}{2}, 8. 7. 98, 6–7 V\nobreak{}«.   2) Stempel: »\nobreak{}\oindex{IX., Alsergrund@\textbf{IX., Alsergrund}, \emph{A.ADM3}|pwk}Wien 9/3 72 a, 8. 7. 98, 7–8 V\nobreak{}«. 
\newline{}Schnitzler: mit Bleistift datiert: »8/7 9\textcolor{gray}{8}« 
\newline{}Ordnung: mit Bleistift von unbekannter Hand nummeriert: »102« }\toendnotes[C]{\smallbreak}\pstart{}{\pb}Herrn D\textsuperscript{r} Arthur Schnitzler \pend{}\pstart{}\textcolor{pink}{Wien}{}\ledrightnote{\textcolor{pink}{Wien}}\pend{}\pstart{}\textcolor{pink}{IX. Franckgaße 1}{}\ledrightnote{\textcolor{pink}{Frankgasse 1}}\pend{}
{\bigskip}
\pstart
           \noindent{}{\pb}lieber, – also \label{K_L02761-1v}\edtext{heute}{\lemma{\textnormal{\emph{heute}}}\Cendnote{\textnormal{siehe A. S.: \emph{Tagebuch}, 8. 7. 1898}}}\label{K_L02761-1h}, um 7\textsuperscript{h.} fahren \textcolor{blue}{wir}{}\ledrightnote{{$\rightarrow$}\textcolor{blue}{Ottilie Salten}} Ihnen zum
                  \textcolor{pink}{Viaduct}{}\ledrightnote{\textcolor{pink}{Penzinger Viadukt}} der \textcolor{pink}{Felberstraße}{}\ledrightnote{\textcolor{pink}{Felberstraße}} entgegen.\pend
           
\pstart
           herzlichst{\\[\baselineskip]}Ihr{\\[\baselineskip]}\spacefill\mbox{Salten}\pend
           \leftskip=0em{}\endnumbering\briefempfaengerindex{Schnitzler, Arthur@\textsc{Schnitzler, Arthur}!zzzSalten, Felix@\emph{von Felix Salten}!1898-07-081@{8. 7. 1898}|)be}\mylabel{h}  \normalsize

\doendnotes{C}
\bigskip
\vfill

\clearpage

\footnotesize

\lohead{\textsc{register}}

% Definiere theindex-Environment komplett neu ohne reledmac
\makeatletter
\renewenvironment{theindex}{%
  \section*{\indexname}%
  \setlength{\parindent}{0pt}%
  \setlength{\parskip}{0pt plus 0.3pt}%
  \let\item\@idxitem
}{%
  \clearpage
}
\makeatother

\IfFileExists{\jobname-pw.ind}{\input{\jobname-pw.ind}}{}

\end{document}

      