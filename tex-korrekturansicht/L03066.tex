%% latex-korrekturansicht-vorspann.tex
%% Vorspann für die Korrekturansicht.
%% Lädt die gemeinsame Datei latex-vorspann.tex mit gesetztem Schalter.

\newif\ifkorrekturansicht
\korrekturansichttrue

\input{../tex-inputs/latex-vorspann}


\renewcommand{\erwaehntePersonen}{Personen: Robert Hirschfeld, Karl Felix Kohler, Olga Schnitzler, Heinrich Schnitzler, Louise Schnitzler}
\renewcommand{\erwaehnteInstitutionen}{Institutionen: Neue Freie Presse}
\renewcommand{\erwaehnteOrte}{Orte: Berlin, Dessauer Straße, Klosters Dorf, Pörtschach, Schweiz, Velden am Wörthersee, Wien, Wörthersee}
\renewcommand{\erwaehnteWerke}{}
\section[ Paul Goldmann an Arthur Schnitzler, 13. 5. {[}1901{]}]{Paul Goldmann an Arthur Schnitzler, 13. 5. {[}1901{]}}
\nopagebreak\mylabel{v}
\rehead{ }\normalsize\beginnumbering\briefempfaengerindex{Schnitzler, Arthur@\textsc{Schnitzler, Arthur}!zzzGoldmann, Paul@\emph{von Paul Goldmann}!1901-05-133@{13. 5. {[}1901{]}}|(be}
\toendnotes[C]{\smallbreak\pagebreak[2]}\Standort{DLA, A:Schnitzler, HS.NZ85.1.3171.}
\physDesc{Brief, 1 Blatt, 4 Seiten
\newline{}Handschrift: blaue Tinte, deutsche Kurrent
\newline{}Schnitzler: 1) mit Bleistift das Jahr »{[}1{]}901« vermerkt  2) mit rotem Buntstift zwei Unterstreichungen}\toendnotes[C]{\smallbreak}
\pstart
           \noindent{}\raggedleft{}{\pb}\textcolor{pink}{\textcolor{gray}{\textbf{DESSAUERSTRASSE 19}}}{}\ledrightnote{\textcolor{pink}{Dessauer Straße}}\pend
           
\pstart
           \textcolor{pink}{Berlin}{}\ledrightnote{\textcolor{pink}{Berlin}}, 13. Mai.\pend
           
\pstart\center{}Mein lieber Freund,\pend
\pstart
           Es thut mir \label{K_L03066-1v}\edtext{unendlich leid}{\lemma{\textnormal{\emph{unendlich leid}}}\Cendnote{\textnormal{\textcolor{blue}{Olga Gussmann} hatte am 10. 5. 1901 das
                  gemeinsame Kind, mit dem sie schwanger war, verloren.}}}\label{K_L03066-1h}, daß es ſo gekommen
               iſt. Da kann man ſich zum Troſt immer nur ſagen: wer weiß, wozu es gut war?
               Jedenfalls ſind auch manche Sorgen dadurch beſeitigt. Und wenn wirklich Anämie daran
               Schuld war, ſo iſt es vielleicht beſſer, wenn die \textcolor{blue}{Mutter}{}\ledrightnote{{$\rightarrow$}\textcolor{blue}{Olga Schnitzler}} erſt einmal ordentlich gekräftigt wird, um {\pb}auch ein kräftiges Kind zur Welt zu bringen. Oder
               iſt das ein naturwiſſenſchaftlicher Unſinn? Schſade, ſchade! Ihr ſcheint Euch \textcolor{blue}{Beide}{}\ledrightnote{{$\rightarrow$}\textcolor{blue}{Olga Schnitzler}} ſehr darauf gefreut zu
               haben. Hoffen wir alſo auf \label{K_L03066-2v}\edtext{das nächſte
                  Mal}{\lemma{\textnormal{\emph{das nächſte
                  Mal}}}\Cendnote{\textnormal{Das nächste Mal wurde \textcolor{blue}{Olga Gussmann} Ende des Jahres schwanger. Am
                     9. 8. 1902 gebar
                  sie \textcolor{blue}{Heinrich Schnitzler}.}}}\label{K_L03066-2h}!\pend
           
\pstart
           Wenn die \label{K_L03066-3v}\edtext{Sommerpläne}{\lemma{\textnormal{\emph{Sommerpläne}}}\Cendnote{\textnormal{siehe Paul Goldmann an Arthur Schnitzler, 26. 4. [1901]}}}\label{K_L03066-3h} gar ſo ſchwankend ſind, ſo iſt es vielleicht am Beſten, daß ich \textsc{\textcolor{blue}{Hirschfeld}{}\ledrightnote{\textcolor{blue}{Robert Hirschfeld}}s} Einladung annehme, zu ihm
               an den \textcolor{pink}{Wörther See}{}\ledrightnote{\textcolor{pink}{Wörthersee}}{ }{\pb}zu kommen. Oder ich gehe nach \textcolor{pink}{Velden}{}\ledrightnote{\textcolor{pink}{Velden am Wörthersee}}{ }\strikeout{\textcolor{gray}{×}\-\textcolor{gray}{×}\-\textcolor{gray}{×}} oder \textcolor{pink}{Pörtſchach}{}\ledrightnote{\textcolor{pink}{Pörtschach}}. Ihr kommt dann hin,
                  \strikeout{\textcolor{gray}{ſoram} Ihr k} ſobald Ihr könnt. Ich wiederhole nochmals:
               ich will diesmal ruhig ſitzen und nicht herumreiſen. Möchte auch in dieſen paar
               Wochen in einer \textcolor{pink}{Wien}{}\ledrightnote{\textcolor{pink}{Wien}}er Sommerfriſche ein Bischen
                  \textcolor{pink}{Wien}{}\ledrightnote{\textcolor{pink}{Wien}}er Leben mitmachen. Iſt Deine Frau \label{K_L03066-4v}\edtext{\textcolor{blue}{Mutter}{}\ledrightnote{{$\rightarrow$}\textcolor{blue}{Louise Schnitzler}} im Auguſt am \textcolor{pink}{Wörtherſee}{}\ledrightnote{\textcolor{pink}{Wörthersee}}}{\lemma{\textnormal{\emph{Mutter … Wörtherſee}}}\Cendnote{\textnormal{\textcolor{blue}{Louise Schnitzler} war im Sommer 1901 höchstwahrscheinlich nicht am \textcolor{pink}{Wörthersee}. Den Briefen \textcolor{blue}{Schnitzler}s an sie ist zu entnehmen, dass sie in \textcolor{pink}{Klosters} (\textcolor{pink}{Schweiz})
                  war.}}}\label{K_L03066-4h}?\pend
           
\pstart
           Ich muß mich jetzt wieder namenlos {\pb}mit der \textcolor{brown}{N. Fr. Pr.}{}\ledrightnote{\textcolor{brown}{Neue Freie Presse}} herumkränken. Dem Herrn \textcolor{blue}{Nachtredakteur}{}\ledrightnote{{$\rightarrow$}\textcolor{blue}{Karl Felix Kohler}} (\textsc{\textcolor{blue}{Kohler}{}\ledrightnote{\textcolor{blue}{Karl Felix Kohler}}}) bin ich antipathiſch. Infolgedeſſen verſchwinden alle meine \textcolor{pink}{Berlin}{}\ledrightnote{\textcolor{pink}{Berlin}}er Theatertelegramme ſpurlos. Wenn ich mich beſchwere,
               heißt es: Raummangel, und dann wird ruhig weiter weggeworfen, was ich ſchicke. Hätte
               ich eine andere Stellung, ich würde meine Demiſſion geben{\dots}\pend
           
\pstart
           Bitte, Fräulein \textsc{\textcolor{blue}{Olga}{}\ledrightnote{\textcolor{blue}{Olga Schnitzler}}} recht herzlich zu grüßen, und ſei auch Du vielmals gegrüßt von Deinem treuen {\\[\baselineskip]}\spacefill\mbox{Paul Goldmnn. }\pend
           \leftskip=0em{}\endnumbering\briefempfaengerindex{Schnitzler, Arthur@\textsc{Schnitzler, Arthur}!zzzGoldmann, Paul@\emph{von Paul Goldmann}!1901-05-133@{13. 5. {[}1901{]}}|)be}\mylabel{h}
\begin{anhang}
\end{anhang}\normalsize

\doendnotes{C}
\bigskip
\vfill

\clearpage

\footnotesize

\lohead{\textsc{register}}

% Definiere theindex-Environment komplett neu ohne reledmac
\makeatletter
\renewenvironment{theindex}{%
  \section*{\indexname}%
  \setlength{\parindent}{0pt}%
  \setlength{\parskip}{0pt plus 0.3pt}%
  \let\item\@idxitem
}{%
  \clearpage
}
\makeatother

\IfFileExists{\jobname-pw.ind}{\input{\jobname-pw.ind}}{}

\end{document}

      