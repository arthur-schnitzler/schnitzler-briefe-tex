%% latex-korrekturansicht-vorspann.tex
%% Vorspann für die Korrekturansicht.
%% Lädt die gemeinsame Datei latex-vorspann.tex mit gesetztem Schalter.

\newif\ifkorrekturansicht
\korrekturansichttrue

\input{../tex-inputs/latex-vorspann}


               \section[Hermann Bahr an Arthur Schnitzler, 22. 3. 1897]{ Hermann Bahr an Arthur Schnitzler, 22. 3. 1897}\nopagebreak\mylabel{v}\rehead{ }\normalsize\beginnumbering\briefempfaengerindex{Schnitzler, Arthur@\textsc{Schnitzler, Arthur}!zzzBahr, Hermann@\emph{von Hermann Bahr}!1897-03-221@{22. 3. 1897}|(be} \toendnotes[C]{\smallbreak\pagebreak[2]} \Standort{CUL, Schnitzler, B 5b.}
\physDesc{Brief, 1 Blatt, 2 Seiten
\newline{}Handschrift: schwarze Tinte, deutsche Kurrent
\newline{}Schnitzler: mit Bleistift die Jahreszahl »7« ergänzt \newline{}Ordnung: mit Bleistift von unbekannter Hand nummeriert:
                                    »51« }\buchAbdrucke{\weitereDrucke{Hermann Bahr, Arthur Schnitzler: \emph{Briefwechsel, Aufzeichnungen, Dokumente (1891–1931)}. Hg. Kurt Ifkovits und Martin Anton Müller. Göttingen: \emph{Wallstein} 2018, S. 139.} }\toendnotes[C]{\smallbreak}\pstart
           \noindent{}{\pb}\textcolor{gray}{\textbf{»\textcolor{brown}{Die Zeit}{}\ledrightnote{\textcolor{brown}{Die Zeit. Wiener Wochenschrift}}«}}\hfill \textcolor{gray}{\textbf{\textbf{\textcolor{pink}{Wien}{}\ledrightnote{\textcolor{pink}{Wien}}}, den }}22. März{ }\textcolor{gray}{\textbf{189{\dotstwo}}}\pend
           \pstart
           \textcolor{gray}{\textbf{Wiener Wochenſchrift}}\hfill \textcolor{gray}{\textbf{\textcolor{pink}{IX/3, Günthergaſſe 1}{}\ledrightnote{\textcolor{pink}{Günthergasse}}.}}\pend
           \pstart
           \textcolor{gray}{\textbf{\textbf{Herausgeber}:}}{\\}\textcolor{gray}{\textbf{Profeſſor Dr. \textcolor{blue}{I. Singer}{}\ledrightnote{\textcolor{blue}{Isidor Singer}},
                        \textcolor{blue}{Hermann Bahr}{}\ledrightnote{\textcolor{blue}{Hermann Bahr}}, Dr. \textcolor{blue}{Heinrich Kanner}{}\ledrightnote{\textcolor{blue}{Heinrich Kanner}}.}}\pend
           \pstart
           \textcolor{gray}{\textbf{Telephon Nr. 6415.}}\pend
           \pstart\center{}Lieber Arthur!\pend\pstart
           \label{K_L00656_1v}\edtext{\textcolor{blue}{Altenberg}{}\ledrightnote{\textcolor{blue}{Peter Altenberg}}}{\lemma{\textnormal{\emph{Altenberg}}}\Cendnote{\textnormal{\textcolor{blue}{Kraus} nannte das Fehlen von \textcolor{blue}{Altenberg} den größten Mangel des Abends (\textcolor{blue}{Karl Kraus}: \emph{\textcolor{green}{Wiener Premièren}}. In: \emph{\textcolor{green}{Breslauer
                        Zeitung}}, Jg. 79, Nr. 255, Abend-Ausgabe, 10. 4. 1897,
                     S. 2).}}}\label{K_L00656_1h} nicht, wenn es nicht ſein muß – bei aller Verehrung ſeiner
               ſchönen Begabung. Aus »Opportunität« nicht. – Ich komme alſo Mittwoch um
                  10 zu Dir. Ich muß aber bis \uline{morgen
                  Dienſtag Abend} die Titel haben, damit Donnerſtag (\label{K_L00656_2v}\edtext{Feiertag}{\lemma{\textnormal{\emph{Feiertag}}}\Cendnote{\textnormal{25. 3.: Mariä Verkündigung.}}}\label{K_L00656_2h}) die Ankündigung in den Blättern
               ſein kann. Schreibe {\pb}mir alſo den Titel von \textcolor{blue}{Hirſchfelds}{}\ledrightnote{\textcolor{blue}{Georg Hirschfeld}}{ }\textcolor{green}{Geſchichte}{}\ledrightnote{→\textcolor{green}{Bei Beiden}}{ }ſowie von \textcolor{green}{Deine\damage{r}}{}\ledrightnote{→\textcolor{green}{Der Ehrentag}}, von \textcolor{blue}{Hugo}{}\ledrightnote{\textcolor{blue}{Hugo von Hofmannsthal}} wollen wir einfach »Gedichte«
               annoncieren. Reihenfolge: \textcolor{blue}{Hirſchfeld}{}\ledrightnote{\textcolor{blue}{Georg Hirschfeld}}, \textcolor{blue}{Hugo}{}\ledrightnote{\textcolor{blue}{Hugo von Hofmannsthal}}, Du, ich – nicht? Programme müſſen Mittwoch
               gedruckt werden.\pend
           \pstart
           Herzlichſt{\\[\baselineskip]}in großer Eile{\\[\baselineskip]}Dein{\\[\baselineskip]}\spacefill\mbox{Hermann}\pend
           \leftskip=0em{}\pstart
           \textcolor{gray}{\textbf{\label{T_L00656_1v}\edtext{Alle für »\textcolor{brown}{Die Zeit}{}\ledrightnote{\textcolor{brown}{Die Zeit. Wiener Wochenschrift}}« beſtimmten Zuſchriften und Sendungen ſind an die
                  Redaction der »\textcolor{brown}{Zeit}{}\ledrightnote{\textcolor{brown}{Die Zeit. Wiener Wochenschrift}}« und \textbf{nicht} an die Perſon eines der Herausgeber zu richten.}{\lemma{\textnormal{\emph{Alle … richten.}}}\Cendnote{\textnormal{am unteren Rand der ersten Seite}}}\label{T_L00656_1h}}}\pend
           \endnumbering\briefempfaengerindex{Schnitzler, Arthur@\textsc{Schnitzler, Arthur}!zzzBahr, Hermann@\emph{von Hermann Bahr}!1897-03-221@{22. 3. 1897}|)be}\mylabel{h}  \normalsize

\doendnotes{C}
\bigskip
\vfill

\clearpage

\footnotesize

\lohead{\textsc{register}}

% Definiere theindex-Environment komplett neu ohne reledmac
\makeatletter
\renewenvironment{theindex}{%
  \section*{\indexname}%
  \setlength{\parindent}{0pt}%
  \setlength{\parskip}{0pt plus 0.3pt}%
  \let\item\@idxitem
}{%
  \clearpage
}
\makeatother

\IfFileExists{\jobname-pw.ind}{\input{\jobname-pw.ind}}{}

\end{document}

      