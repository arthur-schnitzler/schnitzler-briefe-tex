%% latex-korrekturansicht-vorspann.tex
%% Vorspann für die Korrekturansicht.
%% Lädt die gemeinsame Datei latex-vorspann.tex mit gesetztem Schalter.

\newif\ifkorrekturansicht
\korrekturansichttrue

\input{../tex-inputs/latex-vorspann}


               \section[Paul Goldmann an Arthur Schnitzler, 5. 12. {[}1895{]}]{ Paul Goldmann an Arthur Schnitzler, 5. 12. {[}1895{]}}\nopagebreak\mylabel{v}\rehead{ }\normalsize\beginnumbering\briefempfaengerindex{Schnitzler, Arthur@\textsc{Schnitzler, Arthur}!zzzGoldmann, Paul@\emph{von Paul Goldmann}!1895-12-051@{5. 12. {[}1895{]}}|(be} \toendnotes[C]{\smallbreak\pagebreak[2]} \Standort{DLA, A:Schnitzler, HS.NZ85.1.3165.}
\physDesc{Brief, 4 Blätter, 14 Seiten
\newline{}Handschrift: blaue Tinte, deutsche Kurrent
\newline{}Schnitzler: 1) mit Bleistift das Jahr » 95« vermerkt 2) mit rotem Buntstift acht Unterstreichungen und eine seitliche
                                 Markierung}\toendnotes[C]{\smallbreak}\pstart
           \noindent{}{\pb}\textcolor{gray}{\textbf{\textbf{\textcolor{brown}{Frankfurter Zeitung}{}\ledrightnote{\textcolor{brown}{Frankfurter Zeitung}}}}}\pend
           \pstart
           \textcolor{gray}{\textbf{(\textcolor{brown}{\begin{otherlanguage}{french}Gazette de Francfort\end{otherlanguage}}{}\ledrightnote{\textcolor{brown}{Frankfurter Zeitung}}). }}\pend
           \pstart
           \textcolor{gray}{\textbf{\textbf{\begin{otherlanguage}{french}Fondateur M. \textcolor{blue}{L.
                              Sonnemann}{}\ledrightnote{\textcolor{blue}{Leopold Sonnemann}}\end{otherlanguage}.}}}\pend
           \pstart
           \begin{otherlanguage}{french}\textcolor{gray}{\textbf{\textcolor{green}{Journal}{}\ledrightnote{→\textcolor{green}{Frankfurter Zeitung}} politique,
                        financier,}}\end{otherlanguage}\pend
           \pstart
           \begin{otherlanguage}{french}\textcolor{gray}{\textbf{commercial et littéraire.}}\end{otherlanguage}\pend
           \pstart
           \begin{otherlanguage}{french}\textcolor{gray}{\textbf{\textbf{Paraissant trois fois par jour.}}}\end{otherlanguage}\pend
           \pstart
           \begin{otherlanguage}{french}\textcolor{gray}{\textbf{\textbf{Bureau à \textcolor{pink}{Paris}{}\ledrightnote{\textcolor{pink}{Paris}}:}}}\end{otherlanguage}\pend
           \pstart
           \begin{otherlanguage}{french}\textcolor{gray}{\textbf{\textbf{\textcolor{pink}{24. Rue Feydeau}{}\ledrightnote{\textcolor{pink}{rue Feydeau}}.}}}\end{otherlanguage}\hfill \textsc{\textcolor{pink}{Paris}{}\ledrightnote{\textcolor{pink}{Paris}}}, 5. December.\pend
           \pstart\center{}Mein lieber Freund,\pend\pstart
           In Angelegenheit der Aufführung von »\textcolor{green}{Liebelei}{}\ledrightnote{\textcolor{green}{Liebelei. Schauspiel in drei Akten}}«
               in \textsc{\textcolor{pink}{Paris}{}\ledrightnote{\textcolor{pink}{Paris}}} habe ich geſtern einen Schritt gethan, den ich
               längſt thun wollte. Ich war bei \textsc{\textcolor{blue}{Jean Thorel}{}\ledrightnote{\textcolor{blue}{Jean Thorel}}}, deſſen Namen Du gewiß kennſt. Sehr braver u. gewiſſenhafter \textcolor{blue}{Menſch}{}\ledrightnote{→\textcolor{blue}{Jean Thorel}}, wenig Künſtler, großer \textcolor{blue}{Freund \textsc{\textcolor{blue}{Hauptmann}{}\ledrightnote{\textcolor{blue}{Gerhart Hauptmann}}}s}{}\ledrightnote{→\textcolor{blue}{Jean Thorel}}, von dem er die »\textcolor{green}{Weber}{}\ledrightnote{\textcolor{green}{Die Weber. Schauspiel aus den vierziger Jahren}}« u. »\textsc{\textcolor{green}{Hannele}{}\ledrightnote{\textcolor{green}{Hanneles Himmelfahrt. Traumdichtung in zwei Teilen}}}« für die \textcolor{pink}{Pariſ}{}\ledrightnote{\textcolor{pink}{Paris}}er Aufführung überſetzt hat,
                  \textsc{\textcolor{blue}{Intimus}{}\ledrightnote{→\textcolor{blue}{Jean Thorel}}}{ }\strikeout{v} von \textsc{\textcolor{blue}{Antoine}{}\ledrightnote{\textcolor{blue}{André Antoine}}}{ }\textsc{etc.} Ich habe ihm von Deinem \textcolor{green}{Stück}{}\ledrightnote{→\textcolor{green}{Liebelei. Schauspiel in drei Akten}} geſprochen, \label{K_L02758-54v}\edtext{\begin{otherlanguage}{french}\textsc{il est très – emballé là-dessus}\end{otherlanguage}}{\lemma{\textnormal{\emph{il … là-dessus}}}\Cendnote{\textnormal{französisch: er ist sehr dafür
                  eingenommen}}}\label{K_L02758-54h}, will es gern \label{K_L02758-1v}\edtext{überſetzen}{\lemma{\textnormal{\emph{überſetzen}}}\Cendnote{\textnormal{Die Übersetzung wurde,
                  obzwar mit einer Summe von 500 Francs bezahlt, nie fertiggestellt. Am
                     16. 6. 1910 setzte \textcolor{blue}{Schnitzler}{ }\textcolor{blue}{Jean Thorel} davon in Kenntnis, dass er sich
                  nach vierzehn Jahren nicht mehr an frühere Abmachungen gebunden fühle und er
                  nunmehr über das Recht, \emph{\textcolor{green}{Liebelei}} übersetzen
                  und auf die Bühne zu bringen, wieder frei verfüge. (\emph{Deutsches Literaturarchiv Marbach},
                  HS.1985.1.2069)}}}\label{K_L02758-1h}, unter der Bedingung freilich, daß es zur Aufführung
                  {\pb}kommt, will Schritte zur Aufführung bei ernſten
               Theatern thun, verlangt aber baldige Einſendung des \textcolor{green}{Buch}{}\ledrightnote{→\textcolor{green}{Liebelei. Schauspiel in drei Akten}}es, im \strikeout{Druk} Druck oder
               auch im Manuſcript. Wenn es irgend geht, ſende ihm die Sache, mit einem artigen
               Briefe, deutſch geſchrieben, worin Du Dich entſchuldigſt, daß Du wegen mangelnder
               franzöſiſcher Stylgewandtheit ihm nicht franzöſiſch ſchreibſt. Er wird keine
               glänzende Überſetzung machen; eine gute franzöſiſche Überſetzung bekommſt Du
               überhaupt nicht, da alle überſetzenden Franzoſen mehr oder minder plumpe Handwerker
               ſind; aber von Allen, die ich kenne, {\pb}wird er die
               Sache noch am Wenigſten verhunzen. Damit erledigt ſich wohl von ſelbſt der Brief des
               jungen \label{K_L02758-2v}\edtext{\textcolor{blue}{Mann}{}\ledrightnote{→\textcolor{blue}{[Übersetzer]}}es }{\lemma{\textnormal{\emph{Mannes }}}\Cendnote{\textnormal{nicht identifiziert}}}\label{K_L02758-2h} aus \textsc{\textcolor{pink}{Lyon}{}\ledrightnote{\textcolor{pink}{Lyon}}}, der mir ſonſt ſehr gefällt und ſehr ehrlich zu ſein ſcheint. Aber ich habe
               mich nach ihm erkundigt, kein Menſch kennt den Namen ſelbſt die \textsc{\textcolor{pink}{Lyon}{}\ledrightnote{\textcolor{pink}{Lyon}}}er Journaliſten nicht. \strikeout{D\textcolor{gray}{ru}} Darum iſts wohl beſſer, ſich nicht aufs Unſichere einzulaſſen und lieber den
               geraden Weg, d. h. einen bekannten \textcolor{blue}{Überſetzer}{}\ledrightnote{→\textcolor{blue}{Jean Thorel}} zu wählen. Entſchuldige, daß ich den Brief ſolange behalten. Aber
               wüßteſt Du, was Alles in meinen Kopfe rumort hat, ſeitdem!\pend
           \pstart
           {\pb}Haſt Du an \textsc{\textcolor{blue}{Aubry}{}\ledrightnote{\textcolor{blue}{Georges Aubry}}} oder \textcolor{blue}{Frau}{}\ledrightnote{→\textcolor{blue}{[MMe. Georges] Aubry}}
               geſchrieben?\pend
           \pstart
           Die kürzlich zurückgeſandten Druckſachen haben mich intereſſirt, wie alles Übrige.
                  \label{K_L02758-22v}\edtext{\textsc{\textcolor{blue}{\textcolor{green}{Wolter}{}\ledrightnote{→\textcolor{green}{Bei Charlotte Wolter}}}{}\ledrightnote{\textcolor{blue}{Charlotte Wolter}}}}{\lemma{\textnormal{\emph{Wolter}}}\Cendnote{\textnormal{Wahrscheinlich folgende \emph{home story}, die in \textcolor{blue}{Schnitzler}s Zeitungsausschnittsammlung an der \emph{University of Exeter} aufbewahrt wird (5. Liebelei, box 10/1): \textcolor{blue}{Moriz Baumfeld}: \emph{\textcolor{green}{Bei Charlotte Wolter}}. In: \emph{\textcolor{green}{Extrapost}}, Jg. 14, Nr. 718, 21. 10. 1895, S. 1–2. Darin erzählt \textcolor{blue}{Charlotte Wolter}, dass sie nach einem Jahr erstmals wieder
                  im Theater war und das Pech hatte, \emph{\textcolor{green}{Liebelei}}
                  zu sehen – eine, wie sie fand, völlig kunstlose Arbeit.}}}\label{K_L02758-22h}, die dumme \textcolor{blue}{Gans}{}\ledrightnote{→\textcolor{blue}{Charlotte Wolter}}, hat mich beluſtigt,
                  \label{K_L02758-23v}\edtext{\textsc{\textcolor{blue}{\textcolor{green}{Ludassy}{}\ledrightnote{→\textcolor{green}{Burgtheater. »Rechte der Seele«, Schauspiel in einem Acte von Giuseppe Giacosa: deutsch von Otto Eisenschitz. »Liebelei«, Schauspiel in drei Acten von Arthur Schnitzler. Beide zum erstenmale aufgeführt am 9. October 1895}}}{}\ledrightnote{\textcolor{blue}{Julius von Gans-Ludassy}}}}{\lemma{\textnormal{\emph{Ludassy}}}\Cendnote{\textnormal{Es könnte sich um den Nachtrag einer
                  früheren Kritik handeln: \textcolor{blue}{L} [=\textcolor{blue}{Julius von Gans-Ludassy}]: \emph{\textcolor{green}{Burgtheater. »Rechte der Seele«, Schauspiel in einem Acte
                        von Giuseppe Giacosa: deutsch von Otto Eisenschitz. »Liebelei«, Schauspiel
                        in drei Acten von Arthur Schnitzler. Beide zum erstenmale aufgeführt am
                        9. October 1895}}. In: \emph{\textcolor{green}{Wiener Allgemeine
                        Zeitung}}, Jg. XXXX, Nr. XXXX, 10. 10. 1895, S. XXXX}}}\label{K_L02758-23h} mag \strikeout{d} ich gar nicht – auch Einer, der mit dem
               Erfolge geht und Dich bei der erſten Schwierigkeit im Stich laſſen wird. Die kleine
                  \label{K_L02758-24v}\edtext{\textcolor{green}{Parodie}{}\ledrightnote{→\textcolor{green}{[Parodie auf Liebelei / Schnitzler]}}}{\lemma{\textnormal{\emph{Parodie}}}\Cendnote{\textnormal{Eventuell der ungezeichnete Text: \emph{\textcolor{green}{Aus dem Tagebuch einer Weltdame}}. In: \emph{\textcolor{green}{Wiener Caricaturen}}, Jg. 15, Nr. 42,
                        20. 10. 1895, S. 2–3. Nicht so sehr eine Parodie, als
                  eine Satire: Geschildert wird aus der Perspektive einer eher simplen »Dame von
                  Welt«, wie junge Mädchen nicht durch den Besuch der \emph{\textcolor{green}{Liebelei}}, sondern durch Gespräche in der »stillen
                     Häuslichkeit« in sittliche Gefahr geraten.c}}}\label{K_L02758-24h} iſt nicht übel
               gemacht. Daß \label{K_L02758-25v}\edtext{\textsc{\textcolor{blue}{\textcolor{green}{Granichstaedten}{}\ledrightnote{→\textcolor{green}{Deutsches Volkstheater. (»Ein Regentag«, Charakterbild von J. J. David.)}}}{}\ledrightnote{\textcolor{blue}{Emil Granichstaedten}}}}{\lemma{\textnormal{\emph{Granichstaedten}}}\Cendnote{\textnormal{Bezug womöglich auf diese Stelle:
                     »Werden alle die Redlichen, welche das Glück hatten, an \textcolor{blue}{Schnitzler}’s ›\textcolor{green}{Liebelei}‹ Gefallen zu finden, nun auch für \textcolor{blue}{David}’s ›\textcolor{green}{Ein Regentag}‹ das Wort ergreifen und das Lob eines Dichters singen,
                     der sein Werk aus seiner Seele geholt und mit der Beredtsamkeit seines Herzens
                     geschmückt hat? — Mag es gelten, daß man jedes Streben mit Wohlwollen fördern
                     soll. Aber warum offenbart sich dieses Wohlwollen nicht gleich beglückend und
                     gleich allgemein und kräftig bei dem armen Poeten, der nicht die Zeit hat, so
                     viele gewiß redliche Freunde gewiß redlich zu gewinnen, der nicht in der Lage
                     ist, auch in der Gesellschaft als interessanter junger Mann eine Stellung zu
                     haben? Nicht darin liegt die Gefährlichkeit der Camaraderie, daß sie kleine
                     Talente aufbläht, sondern darin, daß sie damit echten Talenten den Weg
                     erschwert, wol auch versperrt. Es ist so leicht, ein ›lieber Kerl‹ zu sein, und
                     die ›lieben Kerle‹ wissen gar nicht, wie viel himmelschreiendes Unrecht sie
                     täglich verschulden.« \textcolor{blue}{Emil Granichstaedten}: \emph{\textcolor{green}{Deutsches Volkstheater. (»Ein Regentag«, Charakterbild von
                        J. J. David.)}}. In: \emph{\textcolor{green}{Die Presse}},
                     Jg. 48, Nr. 283, 15. 10. 1895, S. 1–2, hier:
                     S. 2. }}}\label{K_L02758-25h}{ }\substVorne{}\textsuperscript{jede}\substDazwischen{}jede\substHinten{} nur irgend mögliche Gemeinheit begeht, iſt ſelbſtverſtändlich. Du haſt
               Recht, Dich nicht dabei aufzuhalten. Weiterſchreiben iſt die beſte {\pb}Antwort. Zum Haſſen und zum Bekämpfen ſolcher
               perſönlicher Widerſacher haben nur die unproductiven Leute Zeit. \strikeout{Nur \textcolor{gray}{z} B} Nur den \textsc{\textcolor{blue}{Bahr}{}\ledrightnote{\textcolor{blue}{Hermann Bahr}}} würde ich an Deiner Stelle doch einſalzen. Das iſt nämlich eine Maßnahme von
               Hygiene des alltäglichen Lebens. Der Burſch darf Dir nicht mehr ins Haus, es muß ein
               ordentlicher und klarer Bruch zwiſchen Dir und ihm ſein. Was haſt Du ihm auf das
               infame, \label{K_L02758-4v}\edtext{Billet}{\lemma{\textnormal{\emph{Billet}}}\Cendnote{\textnormal{Gemeint ist die herzliche Gratulation, trotz der mehr als
                  distanzierten \textcolor{green}{Kritik} der
                     \emph{\textcolor{green}{Liebelei}}, siehe Hermann Bahr an Arthur Schnitzler, [12. 10. 1895].}}}\label{K_L02758-4h} geantwortet, das er Dir nach ſeiner \textcolor{green}{Kritik}{}\ledrightnote{→\textcolor{green}{Burgtheater (Liebelei, Schauspiel in drei Acten von Arthur Schnitzler. Rechte der Seele, Schauspiel in einem Act von Guiseppe Giacosa. Zum ersten Mal aufgeführt am 9. October)}}{ }{\pb}zu ſchreiben die Frechheit hatte?\pend
           \pstart
           \label{K_L02758-31v}\edtext{\textsc{\textcolor{blue}{Berger}{}\ledrightnote{\textcolor{blue}{Alfred von Berger}}s}{ }\textcolor{green}{Feuilleton}{}\ledrightnote{→\textcolor{green}{Burgtheater [Rechte der Seele, Liebelei]}}}{\lemma{\textnormal{\emph{Bergers Feuilleton}}}\Cendnote{\textnormal{\textcolor{blue}{Alfred Freiherr von Berger}: \emph{\textcolor{green}{Burgtheater}}. In: \emph{\textcolor{green}{Montags-Revue}}, Jg. 26, Nr. XXXX, 14. 10. 1895, S. XXXX.}}}\label{K_L02758-31h} haſt Du mir
               leider nicht geſchickt.\pend
           \pstart
           Daran, daß die Leute Deinen Erfolg Deinen Freunden und Beziehungen zuſchreiben, wirſt
               Du Dich gewöhnen müſſen. Das Geſindel \strikeout{d} kann doch
               nicht rückhaltslos loben; irgend etwas Geringſchätzendes müſſen ſie einfließen
               laſſen. So haben ſie das gefunden. Beim nächſten Erfolg werden ſie ſchon auf etwas
               Neues kommen. Das Alles hat aber nicht die geringſte Bedeutung, {\pb}und mit all ihrer Gemeinheit, vorn herum oder hinten
               herum, können ſie Dir nichts Weſentliches \strikeout{rauben}
               rauben.\pend
           \pstart
           \textsc{\textcolor{blue}{Herzl}{}\ledrightnote{\textcolor{blue}{Theodor Herzl}}} war bei mir und ſagte über Dich \strikeout{wohl}
               wohlwollend: »Der iſt jetzt der größte Dichter von \textcolor{pink}{Wien}{}\ledrightnote{\textcolor{pink}{Wien}}«. Auch dieſen wirſt Du bald auf der Gegenſeite finden. Oh was für ein
               widerliches Subject! Ich habe nicht die Kraft \strikeout{verhehlt,
                  ihm} gehabt, ihm diesmal den abſtoßenden Eindruck zu verbergen, den er mir
               machte.\pend
           \pstart
           {\pb}Auch \textsc{\textcolor{blue}{Sudermann}{}\ledrightnote{\textcolor{blue}{Hermann Sudermann}}} iſt mir nicht ſympathiſch. Freilich iſt er zu Dir anders wie zu mir. Aber dieſe
               ſeine Einfachheit \strikeout{iſt eine} iſt eine gemachte; und er
               iſt ſogar eitel darauf, der ſchöne Mann zu ſein. Auch bin ich überzeugt, bei \strikeout{Fr} Frauen ſpielt er den Räthſelhaften und
               Dämoniſchen.\pend
           \pstart
           Haſt Du nun wirklich die »\textcolor{green}{Liebelei}{}\ledrightnote{\textcolor{green}{Liebelei. Schauspiel in drei Akten}}« für Dich
               umgearbeitet? Und was macht das neue \textcolor{green}{Stück}{}\ledrightnote{→\textcolor{green}{Freiwild. Schauspiel in 3 Akten}}? Werde ich es im Manuſkript zu ſehen
               bekommen, auf {\pb}einen Tag, wie immer? Und was \label{K_L02758-6v}\edtext{ſchreibſt Du ſonſt}{\lemma{\textnormal{\emph{ſchreibſt Du ſonſt}}}\Cendnote{\textnormal{\textcolor{blue}{Schnitzler} arbeitete am \emph{\textcolor{green}{Freiwild}}, ein \textcolor{green}{Schauspiel}, mit dem er zu diesem Zeitpunkt sehr
                  unzufrieden war, vgl. A. S.: \emph{Tagebuch}, 2. 12. 1895.
                  Am 5. 12. 1895 begann er
                  zudem die \textcolor{green}{Erzählung}{ }\emph{\textcolor{green}{Die Frau des Weisen}} neu.}}}\label{K_L02758-6h}? Und wie und
               mit wem lebſt Du? Was macht die große \textcolor{blue}{Tragödin}{}\ledrightnote{→\textcolor{blue}{Adele Sandrock}}? Wie lange wird die »\textcolor{green}{Liebelei}{}\ledrightnote{\textcolor{green}{Liebelei. Schauspiel in drei Akten}}« noch geſpielt werden? Der Erfolg iſt phänomenal.
               Haſt Du viel Geld verdient? Und das ſparſt Du doch hoffentlich? Haſt Du die ſechs
                  \strikeout{E}{ }\label{K_L02758-88v}\edtext{Ausſchnitte}{\lemma{\textnormal{\emph{Ausſchnitte}}}\Cendnote{\textnormal{Beilage nicht erhalten. Eventuell Teile der bis
                     28. 11. 1895 in acht Folgen abgedruckten Übersetzung von \emph{\textcolor{green}{Die kleine Komödie}}, \emph{\textcolor{green}{La petite comédie}}. }}}\label{K_L02758-88h} aus der »\textsc{\textcolor{green}{Liberté}{}\ledrightnote{\textcolor{green}{La Liberté}}}« erhalten, die ich Dir ſenden
               ließ? Was macht die Frau \textsc{\textcolor{blue}{Lou Andreas}{}\ledrightnote{\textcolor{blue}{Lou Andreas-Salomé}}}? Was {\pb}macht \textsc{\textcolor{blue}{Richard}{}\ledrightnote{\textcolor{blue}{Richard Beer-Hofmann}}}? Arbeitet er? Wird was von ihm erſcheinen?{\dotsfive}\pend
           \pstart
           Wir zwei! In einem Deiner Briefe befindet ſich eine lange und rührende Stelle
               darüber, die mich jetzt beim Wiederleſen nicht weniger bewegt, als beim \strikeout{E\textcolor{gray}{rh}} erſten Mal. Es iſt lieb, daß Du Dir ſolche Mühe gibſt, mir die ſchlimmen Dinge
               auszureden. Sprechen muß ich dir davon, denn ich bin Dir Ehrlichkeit ſchuldig. Von
               Dir aus iſt gewiß nichts zu befürchten. Du wirſt {\pb}Dich nicht ändern, was auch kommen mag, und wirſt einfach und treu bleiben. Aber in
               mir ſitzt das Übel. Ich habe die Empfindung und ſie kehrt immer wieder, trotz allen
               Ankämpfens dagegen – daß Du mir auf einmal ferner gerückt biſt, als je, daß Du und
               ich jetzt auf zwei ganz verſchiedenen Lebensgefilden ſtehen, die weiter auseinander
               liegen, als \strikeout{\textcolor{gray}{ho}}{ }\textcolor{pink}{Wien}{}\ledrightnote{\textcolor{pink}{Wien}} und \textsc{\textcolor{pink}{Paris}{}\ledrightnote{\textcolor{pink}{Paris}}}, und \strikeout{w} durch etwas Weiteres getrennt ſind, als
               durch einen Raum von fünf Jahren. Du und ich, \strikeout{w} wir
               werden jetzt zwei {\pb}verſchiedene Leben führen. Das
                  \textcolor{gray}{×} kommt nicht plötzlich, aber
               ganz \strikeout{all} allmälig, ganz unmerklich. Du wirſt oben
               leben, und ich unten. Derjenige aber, der unten bleibt, bemerkt die Veränderung immer
               zuerſt. Ich \strikeout{\textcolor{gray}{l}} habe die Empfindung,
               daß Du \strikeout{mir} mir langſam entrückt wirſt, und daß ich
               Dir nicht nach kann. Ich denke \strikeout{mich} mir, daß ich ein
               Stadium in Deinem Daſein war, daß ſich Dein Leben von mir weg weiter entwickelt: denn
               mein Leben \strikeout{ent} entwickelt {\pb}nicht, und ich bleibe ſtehen. Ich meine, daß Du mich
               nicht mehr brauchſt, und daß meine Rolle \label{K_L02758-8v}\edtext{\textsc{\begin{otherlanguage}{french}auprès de ta personne\end{otherlanguage}}}{\lemma{\textnormal{\emph{auprès de ta personne}}}\Cendnote{\textnormal{französisch: im Bezug auf Deine Person}}}\label{K_L02758-8h} ausgeſpielt iſt. Ich ſehe Dich weit, weit weg von mir. Schreib’ mir,
               was Du willſt, ich kann mir nicht helfen: ich \uline{ſehe}
               Dich eben ſo. Ich weiß, daß Du die größten Kraftanſtrengungen machen wirſt, um mich
               mit Dir zu nehmen; aber ich weiß, daß 
               {\pb}keine Kraft da nützen
               kann, weil es ein \uline{Geſetz}
               iſt, daß ich zurückbleiben
               muß.\pend
           \pstart
           Ich denke das Alles
               ſchlecht aus. Es iſt heut
               wieder ein ſchlimmer
               Tag. Ich ſitze mit ſchwerem
               Kopfe da, und habe mich
               eine Nacht ſchlaflos
               herumgewälzt, in Seelenqualen. Die Arbeit habe
               ich ſatt. Habs wieder
               einmal mit den
               Leben verſuchen wollen.
               Oh, was für eine Sehnſucht
               ich danach habe, nach
               dem heißen, lebendigen {\pb}Leben! Nicht vorwärtskommen,
            gut! Der Ehrgeiz und
            das Alles iſt doch nur
            künſtlich! Aber leben!
            Und da iſt ein ſüßes
            Kind, die der liebe
            Herrgott für mich
            geſchaffen hat, \textsc{\textcolor{blue}{Grisette}{}\ledrightnote{→\textcolor{blue}{?? [Junge Frau, in die Goldmann Dezember 1895 verliebt ist]}}}
            oder ſo etwas. Aber ſie
            kann mich nicht lieben,
            weil ich nicht jung bin
            und kein ſeuriger Liebhaber.
            Und da es nun nichts wird
            und da alle Sehnſucht
            wieder einmal vergeblich
            war, entdecke ich,
            daß ich im Innern
            ſtets eine Angſt davor {\pb}gehabt habe,
               es könne doch wahr werden und mir doch gelingen! {\dotsfour}\pend
           \pstart
           Grüß’ Dich Gott, mein lieber Freund!\pend
           \pstart
           Dein {\\[\baselineskip]}treuer {\\[\baselineskip]}\spacefill\mbox{Paul Goldmann}\pend
           \leftskip=0em{}\pstart
           \noindent{}Schreib’ bald!\pend
           \endnumbering\briefempfaengerindex{Schnitzler, Arthur@\textsc{Schnitzler, Arthur}!zzzGoldmann, Paul@\emph{von Paul Goldmann}!1895-12-051@{5. 12. {[}1895{]}}|)be}\mylabel{h}  \normalsize

\doendnotes{C}
\bigskip
\vfill

\clearpage

\footnotesize

\lohead{\textsc{register}}

% Definiere theindex-Environment komplett neu ohne reledmac
\makeatletter
\renewenvironment{theindex}{%
  \section*{\indexname}%
  \setlength{\parindent}{0pt}%
  \setlength{\parskip}{0pt plus 0.3pt}%
  \let\item\@idxitem
}{%
  \clearpage
}
\makeatother

\IfFileExists{\jobname-pw.ind}{\input{\jobname-pw.ind}}{}

\end{document}

      