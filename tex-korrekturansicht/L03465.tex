%% latex-korrekturansicht-vorspann.tex
%% Vorspann für die Korrekturansicht.
%% Lädt die gemeinsame Datei latex-vorspann.tex mit gesetztem Schalter.

\newif\ifkorrekturansicht
\korrekturansichttrue

\input{../tex-inputs/latex-vorspann}


\renewcommand{\erwaehntePersonen}{Personen: Eva Marie Goldmann, Olga Schnitzler}
\renewcommand{\erwaehnteOrte}{Orte: Berlin, Edmund-Weiß-Gasse, Kreuzbrunnen, Marienbad, Seis am Schlern, Südtirol, Tirol, Wien}
\renewcommand{\erwaehnteWerke}{}
\section[ Paul Goldmann an Arthur Schnitzler, 17. 8. 1908]{Paul Goldmann an Arthur Schnitzler, 17. 8. 1908}
\nopagebreak\mylabel{v}
\rehead{ }\normalsize\beginnumbering\briefempfaengerindex{Schnitzler, Arthur@\textsc{Schnitzler, Arthur}!zzzGoldmann, Paul@\emph{von Paul Goldmann}!1908-08-171@{17. 8. 1908}|(be}
\toendnotes[C]{\smallbreak\pagebreak[2]}\Standort{DLA, A:Schnitzler, HS.NZ85.1.3175.}
\physDesc{Bildpostkarte, 320 Zeichen
\newline{}Handschrift: 1) schwarze Tinte, deutsche Kurrent\hspace{1em}2) schwarze Tinte, lateinische Kurrent (\noindent{}Adresse)\hspace{1em}
\newline{}Versand: Stempel: »\nobreak{}\oindex{Marienbad@\textbf{Marienbad}, \emph{P.PPL}|pwk}Marienbad 1 \textcolor{gray}{8}, 17. VIII. 08, –1\nobreak{}«.  }\toendnotes[C]{\smallbreak}\pstart{}{\pb}Herrn\pend{}\pstart{}Dr. Arthur Schnitzler\pend{}\pstart{}\textcolor{pink}{Wien}{}\ledrightnote{\textcolor{pink}{Wien}}\pend{}\pstart{}\textcolor{pink}{XVIII. Spöttelgaſse 7}{}\ledrightnote{\textcolor{pink}{Edmund-Weiß-Gasse}}.\pend{}
{\bigskip}
\pstart
           \noindent{}{\pb}\textcolor{gray}{\textbf{\textcolor{pink}{Marienbad}{}\ledrightnote{\textcolor{pink}{Marienbad}}. \textcolor{pink}{Kreuzbrunn}{}\ledrightnote{\textcolor{pink}{Kreuzbrunnen}} Colonnade.}}\pend
           
\pstart
           17. 8. 08.\pend
           
\pstart{}Lieber Freund,\pend
\pstart
           Meine \textcolor{blue}{Frau}{}\ledrightnote{{$\rightarrow$}\textcolor{blue}{Eva Marie Goldmann}} u. ich danken Dir
               herzlich für Deine Karte u. ſenden Deiner \textcolor{blue}{Frau}{}\ledrightnote{{$\rightarrow$}\textcolor{blue}{Olga Schnitzler}} u. Dir herzliche Grüße! \textcolor{pink}{Hier}{}\ledrightnote{{$\rightarrow$}\textcolor{pink}{Marienbad}} gießt es ununterbrochen. Es tut mir
               leid, daß ich nicht auch dieſes Jahr nach \label{K_L03465-1v}\edtext{\textcolor{pink}{Tirol}{}\ledrightnote{\textcolor{pink}{Tirol}{\newline}\textcolor{pink}{Südtirol}}}{\lemma{\textnormal{\emph{Tirol}}}\Cendnote{\textnormal{\textcolor{blue}{Schnitzler} hielt sich im Sommer 1908 in \textcolor{pink}{Südtirol}
                  auf.}}}\label{K_L03465-1h} gegangen bin.\pend
           
\pstart
           Kommſt du dieſen Winter nach \label{K_L03465-2v}\edtext{\textcolor{pink}{Berlin}{}\ledrightnote{\textcolor{pink}{Berlin}}}{\lemma{\textnormal{\emph{Berlin}}}\Cendnote{\textnormal{\textcolor{blue}{Schnitzler} war erst Jahre später wieder in
                  Berlin, zwischen 22. 2. 1911 und 28. 2. 1911.}}}\label{K_L03465-2h}?\pend
           \pstart Dein \spacefill\mbox{Paul Goldmann}\pend{}\endnumbering\briefempfaengerindex{Schnitzler, Arthur@\textsc{Schnitzler, Arthur}!zzzGoldmann, Paul@\emph{von Paul Goldmann}!1908-08-171@{17. 8. 1908}|)be}\mylabel{h}
\begin{anhang}
\end{anhang}\normalsize

\doendnotes{C}
\bigskip
\vfill

\clearpage

\footnotesize

\lohead{\textsc{register}}

% Definiere theindex-Environment komplett neu ohne reledmac
\makeatletter
\renewenvironment{theindex}{%
  \section*{\indexname}%
  \setlength{\parindent}{0pt}%
  \setlength{\parskip}{0pt plus 0.3pt}%
  \let\item\@idxitem
}{%
  \clearpage
}
\makeatother

\IfFileExists{\jobname-pw.ind}{\input{\jobname-pw.ind}}{}

\end{document}

      