%% latex-korrekturansicht-vorspann.tex
%% Vorspann für die Korrekturansicht.
%% Lädt die gemeinsame Datei latex-vorspann.tex mit gesetztem Schalter.

\newif\ifkorrekturansicht
\korrekturansichttrue

\input{../tex-inputs/latex-vorspann}


\renewcommand{\erwaehntePersonen}{Personen: Hermann Bahr, Richard Beer-Hofmann, Otto Brahm, Julius von Gans-Ludassy, Hugo von Hofmannsthal, Maria Charlotte Lamberg, Charlotte Pohl-Glas}
\renewcommand{\erwaehnteInstitutionen}{Institutionen: Deutsches Theater Berlin, Wiener Musik- und Theatergesellschaft}
\renewcommand{\erwaehnteOrte}{Orte: Hotel Oberpollinger, München, Volkstheater, Wien}
\renewcommand{\erwaehnteWerke}{Werke: Adele Sandrock, Die Zeit. Wiener Wochenschrift, Liebelei. Schauspiel in drei Akten}
\section[ Felix Salten an Arthur Schnitzler, 18. 2. 1895]{Felix Salten an Arthur Schnitzler, 18. 2. 1895}
\nopagebreak\mylabel{v}
\rehead{ }\normalsize\beginnumbering\briefempfaengerindex{Schnitzler, Arthur@\textsc{Schnitzler, Arthur}!zzzSalten, Felix@\emph{von Felix Salten}!1895-02-181@{18. 2. 1895}|(be}
\toendnotes[C]{\smallbreak\pagebreak[2]}\Standort{CUL, Schnitzler, B 89, A 1.}
\physDesc{Brief, 1 Blatt, 4 Seiten, 1280 Zeichen
\newline{}Handschrift: Bleistift, lateinische Kurrent
\newline{}Ordnung: mit Bleistift von unbekannter Hand nummeriert: »53« }
\buchAbdrucke{\weitereDrucke{Hermann Bahr, Arthur Schnitzler: \emph{Briefwechsel, Aufzeichnungen, Dokumente (1891–1931)}. Hg. Kurt Ifkovits und Martin Anton Müller. Göttingen: \emph{Wallstein} 2018, S. 97–98.} }\toendnotes[C]{\smallbreak}
\pstart
           \raggedleft{}{\pb}\textcolor{pink}{München}{}\ledrightnote{\textcolor{pink}{München}}\textcolor{gray}{,}{ }1\substVorne{}\textsuperscript{9}\substDazwischen{}8\substHinten{}./II. 95.\pend
           
\pstart
           Lieber Freund, ich habe zunächst eine grosse Bitte an
               Sie: da ich vorausssichtlich von \textcolor{pink}{hier}{}\ledrightnote{{$\rightarrow$}\textcolor{pink}{München}} nicht wegkomme, telegrafiren Sie mir gleich nach Erhalt dieses Briefes:
               »Salten \strikeout{Hotel}{ }\textcolor{pink}{München}{}\ledrightnote{\textcolor{pink}{München}}{ }\textcolor{pink}{Oberpollinger}{}\ledrightnote{\textcolor{pink}{Hotel Oberpollinger}}. Ihre Anwesenheit für Donnerstag erwünscht. Die Redaction.«\pend
           
\pstart
           Aus dieser Bitte entnehmen Sie ungefähr auch wie es mir geht. Ich \substVorne{}\textsuperscript{\textcolor{gray}{×}\-\textcolor{gray}{×}}\substDazwischen{}kä\substHinten{}me dann Donnerstag von der Bahn direkt in die
                  \textcolor{brown}{Musik {\kaufmannsund}
                  Theatergesellschaft}{}\ledrightnote{\textcolor{brown}{Wiener Musik- und Theatergesellschaft}}, wo wir uns \label{K_L03152-1v}\edtext{treffen können}{\lemma{\textnormal{\emph{treffen können}}}\Cendnote{\textnormal{Sie sahen sich erst am
                     Freitag, dem 22. 2. 1895.}}}\label{K_L03152-1h}.\pend
           
\pstart
           {\pb}Ich könnte jetzt sehr
               glücklich sein, wenn ich durch diese freundlichen Straßen mit einem Mädel ginge, das
               ich wirklich liebe. So aber ärgere ich mich ausschließlich, wenn ich mich nicht
               langweile. Morgen will ich ein paar Leute aufsuchen,
               da ich ja Heute schon ein \label{K_L03152-2v}\edtext{Zimmer für \textcolor{blue}{Lotte}{}\ledrightnote{\textcolor{blue}{Charlotte Pohl-Glas}}
                  aufgenommen}{\lemma{\textnormal{\emph{Zimmer … aufgenommen}}}\Cendnote{\textnormal{\textcolor{blue}{Charlotte Glas} war mit dem gemeinsamen \textcolor{blue}{Kind} schwanger. Eventuell
                  hätte sie es in \textcolor{pink}{München} gebären oder auch
                  nur die letzten Tage der Schwangerschaft dort verbringen sollen.}}}\label{K_L03152-2h} habe, mich
               also \uline{damit} nicht weiter aufzuhalten brauche.\pend
           
\pstart
           Ein Brief von Ihnen, der nicht schon unterwegs ist, träfe mich nicht mehr hier. Wenn
               etwas Wichtiges geschehen ist, dann telegrafiren Sie mir ja ohnedies noch separat.
               Sobald \label{K_L03152-3v}\edtext{\textcolor{blue}{Brahm}{}\ledrightnote{\textcolor{blue}{Otto Brahm}} Ihnen den Contract}{\lemma{\textnormal{\emph{Brahm Ihnen den Contract}}}\Cendnote{\textnormal{Gemeint war der Vertrag für das
                  Aufführungsrecht für \emph{\textcolor{green}{Liebelei}} am \emph{\textcolor{brown}{Deutschen Theater}}. Der Vertrag dürfte zu dem
                  Zeitpunkt bereits eingelangt sein (vgl. \emph{Bw}
                     Schnitzler/Brahm 4).}}}\label{K_L03152-3h} gesendet {\kaufmannsund} Sie
               diese Sache in die Zeitungen geben, vergessen Sie nicht, auch {\pb}\textcolor{blue}{Ludassy}{}\ledrightnote{\textcolor{blue}{Julius von Gans-Ludassy}} zu verständigen.\pend
           
\pstart
           Haben Sie \textcolor{blue}{Bahr}{}\ledrightnote{\textcolor{blue}{Hermann Bahr}}’s \label{K_L03152-4v}\edtext{Artikel \textcolor{green}{A. S.}{}\ledrightnote{\textcolor{green}{Adele Sandrock}}}{\lemma{\textnormal{\emph{Artikel A. S.}}}\Cendnote{\textnormal{\textcolor{blue}{Hermann Bahr}: \emph{\textcolor{green}{Adele Sandrock}}. In: \emph{\textcolor{green}{Die
                        Zeit}}, Bd. 2, Nr. 20, 16. 2. 1895,
                     S. 108–109.}}}\label{K_L03152-4h} gelesen? Ich habe ihn noch Samstag{ }Abend im \textcolor{pink}{Theater}{}\ledrightnote{{$\rightarrow$}\textcolor{pink}{Volkstheater}} gesprochen und er war wieder beängstigend freundlich.\pend
           
\pstart
           Leben Sie wol, und grüßen \textcolor{blue}{Beer Hofmann}{}\ledrightnote{\textcolor{blue}{Richard Beer-Hofmann}}{ }{\kaufmannsund}{ }\textcolor{blue}{Loris}{}\ledrightnote{\textcolor{blue}{Hugo von Hofmannsthal}}. Auf Wiedersehen\pend
           
\pstart
           Herzlichst Ihr {\\[\baselineskip]}\spacefill\mbox{Salten}\pend
           \leftskip=0em{}\endnumbering\briefempfaengerindex{Schnitzler, Arthur@\textsc{Schnitzler, Arthur}!zzzSalten, Felix@\emph{von Felix Salten}!1895-02-181@{18. 2. 1895}|)be}\mylabel{h}  \normalsize

\doendnotes{C}
\bigskip
\vfill

\clearpage

\footnotesize

\lohead{\textsc{register}}

% Definiere theindex-Environment komplett neu ohne reledmac
\makeatletter
\renewenvironment{theindex}{%
  \section*{\indexname}%
  \setlength{\parindent}{0pt}%
  \setlength{\parskip}{0pt plus 0.3pt}%
  \let\item\@idxitem
}{%
  \clearpage
}
\makeatother

\IfFileExists{\jobname-pw.ind}{\input{\jobname-pw.ind}}{}

\end{document}

      