%% latex-korrekturansicht-vorspann.tex
%% Vorspann für die Korrekturansicht.
%% Lädt die gemeinsame Datei latex-vorspann.tex mit gesetztem Schalter.

\newif\ifkorrekturansicht
\korrekturansichttrue

\input{../tex-inputs/latex-vorspann}


\renewcommand{\erwaehntePersonen}{Personen: Felix Salten, Ottilie Salten}
\renewcommand{\erwaehnteOrte}{Orte: Wien}
\renewcommand{\erwaehnteWerke}{}
\section[ Arthur Schnitzler an Felix Salten, 26. 8. 1903]{Arthur Schnitzler an Felix Salten, 26. 8. 1903}
\nopagebreak\mylabel{v}
\rehead{ }\normalsize\beginnumbering\briefempfaengerindex{Salten, Felix@\textsc{Salten, Felix}!zzzSchnitzler, Arthur@\emph{von Arthur Schnitzler}!1903-08-261@{26. 8. 1903}|(be}
\toendnotes[C]{\smallbreak\pagebreak[2]}\Standort{Wienbibliothek im Rathaus, ZPH 1681, 2.1.516.}
\physDesc{Brief, 1 Blatt, 2 Seiten, 281 Zeichen
\newline{}Handschrift: Bleistift, deutsche Kurrent
\newline{}Ordnung: mit Bleistift von unbekannter Hand Nummerierung der Blätter des Konvoluts:
                                    »53« }\toendnotes[C]{\smallbreak}
\pstart
           \raggedleft{}{\pb}26. 8. 903.\pend
           
\pstart
           lieber,{ }heute{ }Mittag hat die \label{K_L02983-1v}\edtext{Trauungsceremonie}{\lemma{\textnormal{\emph{Trauungsceremonie}}}\Cendnote{\textnormal{siehe A. S.: \emph{Tagebuch}, 26. 8. 1903}}}\label{K_L02983-1h} in ganz milder Form ſtattgefunden, was ich hiemit Ihnen und Frau \textcolor{blue}{Otti}{}\ledrightnote{\textcolor{blue}{Ottilie Salten}}, mit herzlichen Grüßen, mittheile.\pend
           
\pstart
           {\pb}Könnte man nicht \label{K_L02983-2v}\edtext{an einem der nächſten Abende zuſa{\geminationm}enſein}{\lemma{\textnormal{\emph{an … zuſammenſein}}}\Cendnote{\textnormal{Sie sahen sich am 28. 8. 1903.}}}\label{K_L02983-2h}?
                  Verſtändig{[}en{]} Sie mich in irgend einer Ihnen angenehmen
               Weiſe.\pend
           
\pstart
           Ihr {\\[\baselineskip]}\spacefill\mbox{A.}\pend
           \leftskip=0em{}\endnumbering\briefempfaengerindex{Salten, Felix@\textsc{Salten, Felix}!zzzSchnitzler, Arthur@\emph{von Arthur Schnitzler}!1903-08-261@{26. 8. 1903}|)be}\mylabel{h}  \normalsize

\doendnotes{C}
\bigskip
\vfill

\clearpage

\footnotesize

\lohead{\textsc{register}}

% Definiere theindex-Environment komplett neu ohne reledmac
\makeatletter
\renewenvironment{theindex}{%
  \section*{\indexname}%
  \setlength{\parindent}{0pt}%
  \setlength{\parskip}{0pt plus 0.3pt}%
  \let\item\@idxitem
}{%
  \clearpage
}
\makeatother

\IfFileExists{\jobname-pw.ind}{\input{\jobname-pw.ind}}{}

\end{document}

      