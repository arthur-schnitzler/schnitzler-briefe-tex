%% latex-korrekturansicht-vorspann.tex
%% Vorspann für die Korrekturansicht.
%% Lädt die gemeinsame Datei latex-vorspann.tex mit gesetztem Schalter.

\newif\ifkorrekturansicht
\korrekturansichttrue

\input{../tex-inputs/latex-vorspann}


\renewcommand{\erwaehntePersonen}{Personen:  ?? [Mediziner in Grado], Anna Katharina Rehmann, Felix Salten, Heinrich Schnitzler, Olga Schnitzler}
\renewcommand{\erwaehnteOrte}{Orte: Edmund-Weiß-Gasse 7, Grado, Höhlenstein, Villa Bauer, Wien}
\renewcommand{\erwaehnteWerke}{}
\section[ Felix Salten an Arthur Schnitzler, 12. 7. 1909]{Felix Salten an Arthur Schnitzler, 12. 7. 1909}
\nopagebreak\mylabel{v}
\rehead{ }\normalsize\beginnumbering\briefempfaengerindex{Schnitzler, Arthur@\textsc{Schnitzler, Arthur}!zzzSalten, Felix@\emph{von Felix Salten}!1909-07-123@{12. 7. 1909}|(be}
\toendnotes[C]{\smallbreak\pagebreak[2]}\Standort{CUL, Schnitzler, B 89, B 1.}
\physDesc{Postkarte, 561 Zeichen
\newline{}Handschrift: schwarze Tinte, lateinische Kurrent
\newline{}Versand: Stempel: »\nobreak{}\oindex{Grado@\textbf{Grado}, \emph{P.PPLA3}|pwk}{[}Gra{]}\textcolor{gray}{d}o a\nobreak{}«.  
\newline{}Schnitzler: mit Bleistift Vermerk: »\textsc{Salten}« 
\newline{}Ordnung: mit Bleistift von unbekannter Hand nummeriert: »252« }\toendnotes[C]{\smallbreak}\pstart{}{\pb}Salten, \textcolor{pink}{Grado}{}\ledrightnote{\textcolor{pink}{Grado}}\pend{}\pstart{}\textcolor{pink}{Villa Bauer}{}\ledrightnote{\textcolor{pink}{Villa Bauer}}.\pend{}
{\bigskip}\pstart{}Herrn\pend{}\pstart{}D\textsuperscript{r} Arthur Schnitzler\pend{}\pstart{}\textcolor{pink}{Wien}{}\ledrightnote{\textcolor{pink}{Wien}}\pend{}\pstart{}\textcolor{pink}{XVIII. Spöttelgaße 7}{}\ledrightnote{\textcolor{pink}{Edmund-Weiß-Gasse 7}}\pend{}
{\bigskip}
\pstart{}{\pb}Lieber,\pend
\pstart
           es tut uns herzlich leid, dass der arme \label{K_L03502-1v}\edtext{\textcolor{blue}{Heini}{}\ledrightnote{\textcolor{blue}{Heinrich Schnitzler}} von diesem bösen Husten geplagt}{\lemma{\textnormal{\emph{Heini … geplagt}}}\Cendnote{\textnormal{siehe A. S.: \emph{Tagebuch}, 1. 7. 1909}}}\label{K_L03502-1h} ist, und dass Sie wie Frau \textcolor{blue}{Olga}{}\ledrightnote{\textcolor{blue}{Olga Schnitzler}} nun
               auch diese Sorge haben. Wir wüßten sehr gerne, wie es \textcolor{blue}{Heini}{}\ledrightnote{\textcolor{blue}{Heinrich Schnitzler}} geht, und wären für eine Nachricht dankbar!\pend
           
\pstart
           \textcolor{blue}{Annerle}{}\ledrightnote{\textcolor{blue}{Anna Katharina Rehmann}} hat uns vor ein paar Tagen einen
               großen Schreck bereitet, indem sie über 40° Fieber bekam. Zweimal. Der \textcolor{blue}{Arzt}{}\ledrightnote{{$\rightarrow$}\textcolor{blue}{?? [Mediziner in Grado]}} glaubt, an Malaria, was
               sich heute entscheiden müßte.\pend
           
\pstart
           Wir reisen Donnerstag{ }früh und sind freitag in \textcolor{pink}{Landro}{}\ledrightnote{\textcolor{pink}{Höhlenstein}}!\pend
           
\pstart
           Alles herzliche von uns zu Ihnen {\\[\baselineskip]}Ihr {\\[\baselineskip]}\spacefill\mbox{Salten}\pend
           \leftskip=0em{}
\pstart
           \textcolor{pink}{Grado}{}\ledrightnote{\textcolor{pink}{Grado}}, 12. Juli 09\pend
           \endnumbering\briefempfaengerindex{Schnitzler, Arthur@\textsc{Schnitzler, Arthur}!zzzSalten, Felix@\emph{von Felix Salten}!1909-07-123@{12. 7. 1909}|)be}\mylabel{h}  \normalsize

\doendnotes{C}
\bigskip
\vfill

\clearpage

\footnotesize

\lohead{\textsc{register}}

% Definiere theindex-Environment komplett neu ohne reledmac
\makeatletter
\renewenvironment{theindex}{%
  \section*{\indexname}%
  \setlength{\parindent}{0pt}%
  \setlength{\parskip}{0pt plus 0.3pt}%
  \let\item\@idxitem
}{%
  \clearpage
}
\makeatother

\IfFileExists{\jobname-pw.ind}{\input{\jobname-pw.ind}}{}

\end{document}

      