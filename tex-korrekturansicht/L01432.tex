%% latex-korrekturansicht-vorspann.tex
%% Vorspann für die Korrekturansicht.
%% Lädt die gemeinsame Datei latex-vorspann.tex mit gesetztem Schalter.

\newif\ifkorrekturansicht
\korrekturansichttrue

\input{../tex-inputs/latex-vorspann}


               \section[Arthur Schnitzler an Hugo von Hofmannsthal, {[}24. 8. 1904{]}]{ Arthur Schnitzler an Hugo von Hofmannsthal, {[}24. 8. 1904{]}}\nopagebreak\mylabel{v}\rehead{ }\normalsize\beginnumbering\briefempfaengerindex{Hofmannsthal, Hugo von@\textsc{Hofmannsthal, Hugo von}!zzzSchnitzler, Arthur@\emph{von Arthur Schnitzler}!1904-08-241@{{[}24. 8. 1904{]}}|(be} \toendnotes[C]{\smallbreak\pagebreak[2]} \Standort{FDH, Hs-30885,113.}
\physDesc{Brief, 2 Blätter, 6 Seiten
\newline{}Handschrift: Bleistift, deutsche Kurrent\newline{}Ordnung: Beide Blätter von Schnitzler mit
                           Bleistift – mutmaßlich bei der Durchsicht der Briefe 1929 –
                           datiert: »24/8 904« respektive
                                 »24/8 04« und das zweite Blatt
                           auch mit »II« kenntlich gemacht }\buchAbdrucke{\weitereDrucke{Hugo von Hofmannsthal, Arthur Schnitzler: \emph{Briefwechsel}. Hg. Therese Nickl und Heinrich Schnitzler. Frankfurt am Main: \emph{S. Fischer} 1964, S. 200.} }\toendnotes[C]{\smallbreak}\pstart
           \noindent{}{\pb}lieber Hugo, we{\geminationn} es irgend möglich iſt, ſo werden wir am
                  3. bereit ſein – jedenfalls wird es \textcolor{blue}{\textsc{Gerty}}{}\ledrightnote{\textcolor{blue}{Gertrude von Hofmannsthal}} 3–4 Tage früher wiſſen. Wir wollen jedenfalls einige Zeit in \textcolor{pink}{Iſchl}{}\ledrightnote{\textcolor{pink}{Bad Ischl}} bleiben; ja unſre eigentliche Abſicht war, uns dort in
               Ruhe niederzulaſſen und von dort hie u da auszufliegen. Die Hotels an den \textcolor{pink}{Salzk.gut}{}\ledrightnote{\textcolor{pink}{Salzkammergut}}ſeen ſind mir ſoweit ich ſie kenne, zuwider,
               und ich denke, wir werden uns ev. auf \textcolor{pink}{Salz{\pb}burg}{}\ledrightnote{\textcolor{pink}{Salzburg}} einigen? Ich denke ja, \textcolor{blue}{\textsc{Gerty}}{}\ledrightnote{\textcolor{blue}{Gertrude von Hofmannsthal}} bleibt auch ein paar Tage bei ihrer \textcolor{blue}{Mama}{}\ledrightnote{→\textcolor{blue}{Franziska Schlesinger}} in \textcolor{pink}{Iſchl}{}\ledrightnote{\textcolor{pink}{Bad Ischl}}, und Sie
               holen ſie mindeſtens ab? Oder ſind in \textcolor{pink}{Iſchl}{}\ledrightnote{\textcolor{pink}{Bad Ischl}}, wenn
               ſie ankommt? Oder kommen aus \textcolor{pink}{Auſſee}{}\ledrightnote{\textcolor{pink}{Bad Aussee}} auf ein paar
               Stunden herüber, bei welcher Gelegenheit man weiteres beſprechen könnte? – Außer \textcolor{pink}{Iſchl}{}\ledrightnote{\textcolor{pink}{Bad Ischl}} hatten wir auch \textcolor{pink}{\textsc{Salegg}}{}\ledrightnote{\textcolor{pink}{Burg Salegg}} (bei \textcolor{pink}{Waidbruck}{}\ledrightnote{\textcolor{pink}{Ponte Gardena}}) in {\pb}Erwägung gezogen, wegen der, von \textcolor{blue}{Olga}{}\ledrightnote{\textcolor{blue}{Olga Schnitzler}} u mir ſehr erſehnten (mäßigen) Höhe und Stille. \textcolor{pink}{\textsc{Salegg}}{}\ledrightnote{\textcolor{pink}{Burg Salegg}} hätte dann auch den Vortheil, we{\geminationn} der
                  Herbſt mit Macht hereinbricht, daſs man \textcolor{pink}{Bozen}{}\ledrightnote{\textcolor{pink}{Bozen}}, \textcolor{pink}{Meran}{}\ledrightnote{\textcolor{pink}{Meran}} ganz nahe hat. –\pend
           \pstart
           Worauf ich einigermaßen rechne \substVorne{}\textsuperscript{ſind}\substDazwischen{}iſt\substHinten{} aber ganz beſonders irgend eine kleine Radtour, die wir, Sie und ich, machen
               könnten, ſo von 2–3 Tagen, oder 2 kleinere, {\pb}in welchem
               Betracht ich d\substVorne{}\textsuperscript{en}\substDazwischen{}ie\substHinten{}{ }\textsc{ego}- u \textcolor{blue}{\textsc{olga}}{}\ledrightnote{\textcolor{blue}{Olga Schnitzler}}iſtiſche Hoffnung nicht unterdrücken kann, daſs während dieſer Zeit \textcolor{blue}{Olga}{}\ledrightnote{\textcolor{blue}{Olga Schnitzler}} u \textcolor{blue}{\textsc{Gerty}}{}\ledrightnote{\textcolor{blue}{Gertrude von Hofmannsthal}} zuſa{\geminationm}en ſind oder uns gar auf hohem Einſpänner
               vorausraſen?\pend
           \pstart
           – Aber all dies eignet ſich zu mündlicher Verſtändigg; für heute möcht ich nur
               wiſſen, \uline{wann} ich Sie in \textcolor{pink}{Iſchl}{}\ledrightnote{\textcolor{pink}{Bad Ischl}}{ }ſprechen werde, den Fall geſetzt, daſs wir am
                  3.{ }\substVorne{}\textsuperscript{M}\substDazwischen{}Na\substHinten{}chmittag dortſelbſt eintreffen\pend
           \pstart
           {\pb}Noch eines; \textcolor{blue}{\textsc{Gerty}}{}\ledrightnote{\textcolor{blue}{Gertrude von Hofmannsthal}} wird ja wahrſcheinlich in \textcolor{pink}{Wien}{}\ledrightnote{\textcolor{pink}{Wien}} zu thun haben;
               es wäre ſehr hübſch von ihr, we{\geminationn}{ }sie, wann es ihr beliebt bei uns ſpeiſen wollte;
               wir bitten um eine vorherige telegr. Verſtändigung. –\pend
           \pstart
           Mir ginge es ganz gut, we{\geminationn} ich nicht einen etwas
               hartnäckigen Bronchialkatarrh hätte; der übrigens vielleicht noch in meinen
                  Septemberplänen eine kleine Rolle wird ſpielen müſſen. –\pend
           \pstart
           {\pb}Und \textcolor{blue}{Richard}{}\ledrightnote{\textcolor{blue}{Richard Beer-Hofmann}}? – Wird
               er zu bewegen ſein, nach \textcolor{pink}{Iſchl}{}\ledrightnote{\textcolor{pink}{Bad Ischl}}{ }\introOben{}oder \textcolor{pink}{Salzburg}{}\ledrightnote{\textcolor{pink}{Salzburg}}?\introOben{} zu ko{\geminationm}en? Jedenfalls möcht ich
               ihn ſehn – ſein \textcolor{green}{Stück}{}\ledrightnote{→\textcolor{green}{Der Graf von Charolais. Ein Trauerspiel}}
               hören. –\pend
           \pstart
           Herzliche Grüße.{\\[\baselineskip]}Ihr\spacefill\mbox{A.}\pend
           \leftskip=0em{}\endnumbering\briefempfaengerindex{Hofmannsthal, Hugo von@\textsc{Hofmannsthal, Hugo von}!zzzSchnitzler, Arthur@\emph{von Arthur Schnitzler}!1904-08-241@{{[}24. 8. 1904{]}}|)be}\mylabel{h}  \normalsize

\doendnotes{C}
\bigskip
\vfill

\clearpage

\footnotesize

\lohead{\textsc{register}}

% Definiere theindex-Environment komplett neu ohne reledmac
\makeatletter
\renewenvironment{theindex}{%
  \section*{\indexname}%
  \setlength{\parindent}{0pt}%
  \setlength{\parskip}{0pt plus 0.3pt}%
  \let\item\@idxitem
}{%
  \clearpage
}
\makeatother

\IfFileExists{\jobname-pw.ind}{\input{\jobname-pw.ind}}{}

\end{document}

      