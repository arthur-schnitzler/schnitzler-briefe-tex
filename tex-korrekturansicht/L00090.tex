%% latex-korrekturansicht-vorspann.tex
%% Vorspann für die Korrekturansicht.
%% Lädt die gemeinsame Datei latex-vorspann.tex mit gesetztem Schalter.

\newif\ifkorrekturansicht
\korrekturansichttrue

\input{../tex-inputs/latex-vorspann}


               \section[Arthur Schnitzler an Hugo von Hofmannsthal, 27. 3. 1892]{ Arthur Schnitzler an Hugo von Hofmannsthal, 27. 3. 1892}\nopagebreak\mylabel{v}\rehead{ }\normalsize\beginnumbering\briefempfaengerindex{Hofmannsthal, Hugo von@\textsc{Hofmannsthal, Hugo von}!zzzSchnitzler, Arthur@\emph{von Arthur Schnitzler}!1892-03-272@{27. 3. 1892}|(be} \toendnotes[C]{\smallbreak\pagebreak[2]} \Standort{FDH, Hs-30885,19.}
\physDesc{Brief, 1 Blatt, 4 Seiten
\newline{}Handschrift: Bleistift, deutsche Kurrent}\buchAbdrucke{\weitereDrucke{1) Hugo von Hofmannsthal, Arthur Schnitzler: \emph{Briefwechsel}. Hg. Therese Nickl und Heinrich Schnitzler. Frankfurt am Main: \emph{S. Fischer} 1964, S. 18–19.} \weitereDrucke{2) Hermann Bahr, Arthur Schnitzler: \emph{Briefwechsel, Aufzeichnungen, Dokumente (1891–1931)}. Hg. Kurt Ifkovits und Martin Anton Müller. Göttingen: \emph{Wallstein} 2018.} }\toendnotes[C]{\smallbreak}\pstart
           \raggedleft{}{\pb}27/3 92\pend
           \pstart{}Lieber Freund,\pend\pstart
           es war mir ſehr leid, daſs Sie heute nicht kamen. \textcolor{blue}{\textsc{Bölsche}}{}\ledrightnote{\textcolor{blue}{Wilhelm Bölsche}} hat auch mir geſchrieben – auf eine Anfrage, ob man Gedichte einſenden kann u
               was mit meinen »\textcolor{green}{Elixiren}{}\ledrightnote{\textcolor{green}{Die drei Elixire}}« los ſei. – Er will die
                  \textcolor{green}{Elixire}{}\ledrightnote{\textcolor{green}{Die drei Elixire}} bringen »ſobald es geht«, aber »offen
               geſtanden ſind ſie ihm nicht ſo lieb {\pb}wie die erſte \textcolor{green}{Novelle}{}\ledrightnote{→\textcolor{green}{Der Sohn. Aus den Papieren eines Arztes}}, ſie ſind lange nicht ſo
               aktuell.« – Sagt’ ich’s nicht? Auch \uline{die} Herren haben
               ſchon ihren Zopf. Wir brauchen ja doch »unſer« Blatt! – Ich will übrigens das »\textcolor{green}{Hi{\geminationm}elbett}{}\ledrightnote{\textcolor{green}{Das Himmelbett}}« an \textcolor{blue}{\textsc{Bölsche}}{}\ledrightnote{\textcolor{blue}{Wilhelm Bölsche}}{ }ſchicken. – Geſtern ſprach ich Herrn \textcolor{blue}{\textsc{Leo Geiringer}}{}\ledrightnote{\textcolor{blue}{Leopold Geiringer}}, den Dramaturgen des \textcolor{pink}{Dtſch Volksth.}{}\ledrightnote{\textcolor{pink}{Volkstheater}}, der mich
               um mein \textcolor{green}{Märchen}{}\ledrightnote{\textcolor{green}{Das Märchen. Schauspiel in drei Aufzügen}} gebeten hatte – ich ſandte es ihm
                  {\pb}als »Privatmann«. – Er ſagte: »Wirklich ein hübſches
               Talent, ich muſs nur bedauern, daß Sie ſich \uline{dieſer}
               Richtung zugewandt haben!{[}«{]}\pend
           \pstart
           \uline{Ich}{ }{\dotstwo}?{\dotsfour}! – ?\pend
           \pstart
           \uline{Er}. Nun ja, Sie werden doch zugeben, der Schluſs ist
                  unbefriedigend{\dots}\pend
           \pstart
           \uline{Ich}.{ }{\dotstwo}!{\dots}in den Charakteren{\dots}\pend
           \pstart
           \uline{Er}. Die Erfahrung lehrt nun einmal, daß unſer
               Publicum \textsc{etc etc}.\pend
           \pstart
           {\pb}\uline{Ich.}{ }{\dots}{ }\textcolor{green}{Wildente}{}\ledrightnote{\textcolor{green}{Die Wildente}}!!{\dotsfour}\pend
           \pstart
           \uline{Er}. Den Einfluſs merkt man auch deutlich {\dotstwo} ich will nicht gerade ſagen, daß Sie abgeſchrieben
                  haben{\dotsfour}\pend
           \pstart
           \label{T_L00090_1v}\edtext{!!.Ich.}{\lemma{\textnormal{\emph{!!.Ich.}}}\Cendnote{\textnormal{verkehrt zum Text}}}\label{T_L00090_1h}\pend
           \pstart
           Herzlichſt der Ihre, und ko{\geminationm}en Sie Dienſtag gef. zur \textcolor{blue}{\textsc{Bahr}}{}\ledrightnote{\textcolor{blue}{Hermann Bahr}}’ſchen \label{K_L00090_1v}\edtext{Myſtik}{\lemma{\textnormal{\emph{Myſtik}}}\Cendnote{\textnormal{Gemeint ist \textcolor{blue}{Bahrs} Vortrag über »Moderne Mystik«, den er am 29. 3. 1892 bei einer
                  Veranstaltung der \emph{\textcolor{brown}{Freien Bühne}} hielt.}}}\label{K_L00090_1h}!\pend
           \endnumbering\briefempfaengerindex{Hofmannsthal, Hugo von@\textsc{Hofmannsthal, Hugo von}!zzzSchnitzler, Arthur@\emph{von Arthur Schnitzler}!1892-03-272@{27. 3. 1892}|)be}\mylabel{h}  \normalsize

\doendnotes{C}
\bigskip
\vfill

\clearpage

\footnotesize

\lohead{\textsc{register}}

% Definiere theindex-Environment komplett neu ohne reledmac
\makeatletter
\renewenvironment{theindex}{%
  \section*{\indexname}%
  \setlength{\parindent}{0pt}%
  \setlength{\parskip}{0pt plus 0.3pt}%
  \let\item\@idxitem
}{%
  \clearpage
}
\makeatother

\IfFileExists{\jobname-pw.ind}{\input{\jobname-pw.ind}}{}

\end{document}

      