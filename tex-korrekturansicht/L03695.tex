%% latex-korrekturansicht-vorspann.tex
%% Vorspann für die Korrekturansicht.
%% Lädt die gemeinsame Datei latex-vorspann.tex mit gesetztem Schalter.

\newif\ifkorrekturansicht
\korrekturansichttrue

\input{../tex-inputs/latex-vorspann}


\renewcommand{\erwaehntePersonen}{Personen: Hermann Bahr, Elsa Plessner}
\renewcommand{\erwaehnteInstitutionen}{Institutionen: Bazar de la Charité, Leopold Weiss}
\renewcommand{\erwaehnteOrte}{Orte: Fröschelgasse 6, Leipzig, London, Paris, Sievering, Wien}
\renewcommand{\erwaehnteWerke}{Werke: Der gläserne Käfig. Eine Parabel, Der gläserne Käfig. Skizzen und Novellen, Meine Freundin Klothilde}
\section[Elsa Plessner an Arthur Schnitzler, 15. 5. 1897]{Elsa Plessner an Arthur Schnitzler, 15. 5. 1897}
\nopagebreak\mylabel{v}
\rehead{ }\normalsize\beginnumbering\briefempfaengerindex{Schnitzler, Arthur@\textsc{Schnitzler, Arthur}!zzzPlessner, Elsa@\emph{von Elsa Plessner}!1897-05-151@{15. 5. 1897}|(be}
\toendnotes[C]{\smallbreak\pagebreak[2]}\Standort{DLA, A:Schnitzler, 85.1.4198.}
\physDesc{Brief, 1 Blatt, 3 Seiten, 1941 Zeichen
\newline{}Handschrift: , lateinische Kurrent}\toendnotes[C]{\smallbreak}
\pstart
           {\pb}\textcolor{pink}{Sievring, Fröschelgasse 6}{}\ledrightnote{\textcolor{pink}{Fröschelgasse 6}}, den 15. V. 97.\pend
           
\pstart\center{}Verehrter Herr Doctor!\pend\vspace{0.5em}
\pstart
           Besten Dank für Ihre liebenswürdigen Zeilen aus \textcolor{pink}{Paris}{}\ledrightnote{\textcolor{pink}{Paris}}, von wo ich Sie wieder \label{K_L03695-1v}\edtext{zurückgekehrt}{\lemma{\textnormal{\emph{zurückgekehrt}}}\Cendnote{\textnormal{Sie irrt sich; \textcolor{blue}{Schnitzler} war noch in \textcolor{pink}{Paris}, reiste von dort nach \textcolor{pink}{London} und kehrte erst am 2. 6. 1897 von seinem Auslandsaufenthalt zurück.
                  Es ist also anzunehmen, dass ihm dieser Brief nachgeschickt wurde.}}}\label{} glaube.
               Da Ihre dortige Adresse mir – zu Ihrem Besten – unbekannt war, sparte ich mir den
               allerherzlichsten Dank für Ihre gütige Intervention bei \textcolor{blue}{H. Bahr}{}\ledrightnote{\textcolor{blue}{Hermann Bahr}} – bis jetzt auf. Es ist Ihnen sicherlich schon sehr
               langweilig, dass ich mich in jedem Brief an Sie bedanke – aber wenn Sie mir immer
               Grund dazu geben? – –\pend
           
\pstart
           Ich war in großer Angst und {\pb}Aufregung, als ich von der \textcolor{pink}{Pariser}{}\ledrightnote{\textcolor{pink}{Paris}}{ }\label{K_L03695-2v}\edtext{Unglücksgeschichte hörte, Sie in dem
               verbrannten Gebäude}{\lemma{\textnormal{\emph{Unglücksgeschichte … Gebäude}}}\Cendnote{\textnormal{Am 4. 5. 1897 brannte der \emph{\textcolor{brown}{Bazar de la Charité}}, eine Wohltätigkeitseinrichtung, ab. Dabei kamen über
                  120 Menschen ums Leben.}}}\label{} fürchtend – – – na – andere Leute, die mich
               interessieren kenne ich in \textcolor{pink}{Paris}{}\ledrightnote{\textcolor{pink}{Paris}} nicht.
               Hoffentlich sind Sie heil und wohl wieder hier eingetroffen – – – Ich sitze – – bei
                  \label{K_L03695-3v}\edtext{\textcolor{gray}{0}° R.}{\lemma{\textnormal{\emph{0° R.}}}\Cendnote{\textnormal{Wie die
                  Celsius-Skala setzt die Réaumur-Skala den Nullwert beim Taupunkt von
                  Wasser.}}}\label{} und unendlichem Regen in der »Sommerfrische« – – alle gerechten
               Menschen seien davor behütet!! Bedauern Sie mich, verehrter Herr Doctor! Ich bin
               einmal ein unglückliches Geschöpf. Schreiben thue ich \uline{jetzt}{ }\uline{gar nichts!!} – Kann nicht!! – Malheur oder
               Glück!?\pend
           
\pstart
           {\pb}Beiliegend ein kleiner Einfall! – Habe mir Mühe gegeben, nicht
               »schlampig« zu arbeiten. Ich hoffe auf Ihren Beifall! – Bin neugierig, wann und ob
               ich einmal eine Arbeit zu Ihrer rückhaltlosen Anerkennung bringen werde. »\label{K_L03695-4v}\edtext{\textcolor{green}{Meine Freundin Clotilde}{}\ledrightnote{\textcolor{green}{Meine Freundin Klothilde}}}{\lemma{\textnormal{\emph{Meine Freundin Clotilde}}}\Cendnote{\textnormal{Erstausgabe in: \emph{\textcolor{green}{Der gläserne Käfig. Skizzen
                        und Novellen}}. \textcolor{pink}{Wien}, \textcolor{pink}{Leipzig}: \emph{\textcolor{brown}{Leopold
                        Weiss}}{ }1901.}}}\label{}« vermeidet alle wissentliche Affectation – – – negativer Vorzug –
               Positiv? – Bilanz!? – – Ich bin jetzt furchtbar ängstlich in der Arbeit – darum
               geringe Lust dazu! Ist ja doch Stroh!! – Außer mir hat Keiner Freude davon und in
               fünfzig Jahren? – – –. Grau – grau – aber keine Theorie – leider die Praxis!– – –
               Doch Sie kommen aus \textcolor{pink}{Paris}{}\ledrightnote{\textcolor{pink}{Paris}}! und haben
               wahrscheinlich keine mitschwingende Saite für die Klage aus dem \textcolor{pink}{Sievringer}{}\ledrightnote{\textcolor{pink}{Sievering}} Wald. – Ich brauchte ein bisschen moralisches »\textcolor{pink}{Paris}{}\ledrightnote{\textcolor{pink}{Paris}}«–! d. h. um- und aufgekrempelt zu werden.
               Weinen Sie, wenn Sie wollen und lachen Sie, wenn Sie können über Ihre\pend
           \pstart \spacefill\mbox{ElsaPlessner}\pend{}
\pstart
           \noindent{}P. S. Causa \textcolor{blue}{H. Bahr}{}\ledrightnote{\textcolor{blue}{Hermann Bahr}} ist noch nicht
                  erledigt. Vielmehr »\textcolor{green}{gläserner Käfig}{}\ledrightnote{\textcolor{green}{Der gläserne Käfig. Eine Parabel}}«
                  hinzugekommen. Doppelt hält besser.\pend
           \endnumbering\briefempfaengerindex{Schnitzler, Arthur@\textsc{Schnitzler, Arthur}!zzzPlessner, Elsa@\emph{von Elsa Plessner}!1897-05-151@{15. 5. 1897}|)be}\mylabel{h}
\begin{anhang}
\end{anhang}\normalsize

\doendnotes{C}
\bigskip
\vfill

\clearpage

\footnotesize

\lohead{\textsc{register}}

% Definiere theindex-Environment komplett neu ohne reledmac
\makeatletter
\renewenvironment{theindex}{%
  \section*{\indexname}%
  \setlength{\parindent}{0pt}%
  \setlength{\parskip}{0pt plus 0.3pt}%
  \let\item\@idxitem
}{%
  \clearpage
}
\makeatother

\IfFileExists{\jobname-pw.ind}{\input{\jobname-pw.ind}}{}

\end{document}

      