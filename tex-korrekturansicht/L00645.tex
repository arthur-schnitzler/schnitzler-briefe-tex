%% latex-korrekturansicht-vorspann.tex
%% Vorspann für die Korrekturansicht.
%% Lädt die gemeinsame Datei latex-vorspann.tex mit gesetztem Schalter.

\newif\ifkorrekturansicht
\korrekturansichttrue

\input{../tex-inputs/latex-vorspann}


               \section[Arthur Schnitzler an Hugo von Hofmannsthal, {[}9. 2. 1897?{]}]{ Arthur Schnitzler an Hugo von Hofmannsthal, {[}9. 2. 1897?{]}}\nopagebreak\mylabel{v}\rehead{ }\normalsize\beginnumbering\briefempfaengerindex{Hofmannsthal, Hugo von@\textsc{Hofmannsthal, Hugo von}!zzzSchnitzler, Arthur@\emph{von Arthur Schnitzler}!1897-02-092@{{[}9. 2. 1897?{]}}|(be} \toendnotes[C]{\smallbreak\pagebreak[2]} \Standort{FDH, Hs-30885,54.}
\physDesc{Brief, 1 Blatt, 4 Seiten
\newline{}Handschrift: Bleistift, deutsche Kurrent\newline{}Ordnung: von Schnitzler mutmaßlich bei der Durchsicht der Korrespondenz 1929 mit Bleistift
                                    datiert: »Anf 97« }\buchAbdrucke{\weitereDrucke{Hugo von Hofmannsthal, Arthur Schnitzler: \emph{Briefwechsel}. Hg. Therese Nickl und Heinrich Schnitzler. Frankfurt am Main: \emph{S. Fischer} 1964, S. 78.} }\pstart
           \noindent{}{\pb}Lieber Hugo, ich habe der \textcolor{blue}{\textsc{Minnie}}{}\ledrightnote{\textcolor{blue}{Hermine von Schaffgotsch}}{ }\textsc{teleph.} wa{\geminationn} morgen Probe
                    ſei, ſie antwortete noch nicht besti{\geminationm}t,
                    wahrſcheinlich ½ 6; da{\geminationn} fragte ich, ob
                    ſie heute zu \textcolor{blue}{W.}{}\ledrightnote{\textcolor{blue}{August Wärndorfer}{\newline}\textcolor{blue}{Adrienne Wärndorfer}}s komme, {\pb}worauf ſie ſagte, ſie glaube nicht.\pend
           \pstart
           Damit war das Geſpräch (»Alſo auf Wiederſehen« (ich)) beendet.\pend
           \pstart
           Ich gehe alſo nicht zu \textcolor{blue}{W.}{}\ledrightnote{\textcolor{blue}{August Wärndorfer}{\newline}\textcolor{blue}{Adrienne Wärndorfer}}s. Die
                    Möglichkeit iſt zu bedenken, daſs ſie nur nicht will, dſs \uline{ich} heut hinaus komme. {\pb}Vielleicht
                    haben Sie \substVorne{}\textsuperscript{ke}\substDazwischen{}ir\substHinten{}gend eine Nachricht.\pend
           \pstart
           Wollen Sie noch was wiſſen, ſo können Sie mir wohl zu \textcolor{blue}{\textsc{Loebs}}{}\ledrightnote{\textcolor{blue}{Louis Loeb}{\newline}\textcolor{blue}{Regina Loeb}}{ }\textsc{teleph}. Ich
                    bleibe dort wohl bis ½ 5 oder 5, da{\geminationn} geh ich zu mir nach Haus. Spät Abds
                        (½ 11 denk ich) {\pb}bin ich im \textcolor{pink}{\textsc{Pucher}}{}\ledrightnote{\textcolor{pink}{Café Pucher}}. –\pend
           \pstart
           Herzlich der Ihre{\\[\baselineskip]}\spacefill\mbox{Arthur}\pend
           \leftskip=0em{}\endnumbering\briefempfaengerindex{Hofmannsthal, Hugo von@\textsc{Hofmannsthal, Hugo von}!zzzSchnitzler, Arthur@\emph{von Arthur Schnitzler}!1897-02-092@{{[}9. 2. 1897?{]}}|)be}\mylabel{h}  \normalsize

\doendnotes{C}
\bigskip
\vfill

\clearpage

\footnotesize

\lohead{\textsc{register}}

% Definiere theindex-Environment komplett neu ohne reledmac
\makeatletter
\renewenvironment{theindex}{%
  \section*{\indexname}%
  \setlength{\parindent}{0pt}%
  \setlength{\parskip}{0pt plus 0.3pt}%
  \let\item\@idxitem
}{%
  \clearpage
}
\makeatother

\IfFileExists{\jobname-pw.ind}{\input{\jobname-pw.ind}}{}

\end{document}

      