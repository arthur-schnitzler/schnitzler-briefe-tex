%% latex-korrekturansicht-vorspann.tex
%% Vorspann für die Korrekturansicht.
%% Lädt die gemeinsame Datei latex-vorspann.tex mit gesetztem Schalter.

\newif\ifkorrekturansicht
\korrekturansichttrue

\input{../tex-inputs/latex-vorspann}


\renewcommand{\erwaehnteOrte}{Orte: Wien}
\renewcommand{\erwaehnteWerke}{}
\section[ Felix Salten an Arthur Schnitzler, 19. 1. 1904]{Felix Salten an Arthur Schnitzler, 19. 1. 1904}
\nopagebreak\mylabel{v}
\rehead{ }\normalsize\beginnumbering\briefempfaengerindex{Schnitzler, Arthur@\textsc{Schnitzler, Arthur}!zzzSalten, Felix@\emph{von Felix Salten}!1904-01-191@{19. 1. 1904}|(be}
\toendnotes[C]{\smallbreak\pagebreak[2]}\Standort{CUL, Schnitzler, B 89, B 1.}
\physDesc{Brief, 1 Blatt, 1 Seite, 448 Zeichen
\newline{}Handschrift: Bleistift, lateinische Kurrent
\newline{}Ordnung: mit Bleistift von unbekannter Hand nummeriert: »183.« }\toendnotes[C]{\smallbreak}
\pstart
           \raggedleft{}{\pb}19. I. 04\pend
           
\pstart
           Lieber Arthur, natürlich kann in einer solchen \label{K_L03391-1v}\edtext{Nummer}{\lemma{\textnormal{\emph{Nummer}}}\Cendnote{\textnormal{nicht ermittelt, für welches Jubiläum \textcolor{blue}{Salten} Textspenden einsammelte}}}\label{K_L03391-1h} eine rein
               gesellschaftliche Gratulation nicht stehen. Ebenso natürlich, dass weiter nichts
               dabei ist, wenn Sie sich aus irgendwelchen Gründen mit einem anderen Beitrag nicht
               daran betheiligen können. Ich habe Ihnen lediglich das redactionelle Circular
               übersendet, das Sie deshalb, weil es meine Unterschrift trägt, um nichts mehr
               berücksichtigen brauchen, als ein andres.\pend
           
\pstart
           herzlich {\\[\baselineskip]}Ihr \spacefill\mbox{Salten}\pend
           \leftskip=0em{}\endnumbering\briefempfaengerindex{Schnitzler, Arthur@\textsc{Schnitzler, Arthur}!zzzSalten, Felix@\emph{von Felix Salten}!1904-01-191@{19. 1. 1904}|)be}\mylabel{h}  \normalsize

\doendnotes{C}
\bigskip
\vfill

\clearpage

\footnotesize

\lohead{\textsc{register}}

% Definiere theindex-Environment komplett neu ohne reledmac
\makeatletter
\renewenvironment{theindex}{%
  \section*{\indexname}%
  \setlength{\parindent}{0pt}%
  \setlength{\parskip}{0pt plus 0.3pt}%
  \let\item\@idxitem
}{%
  \clearpage
}
\makeatother

\IfFileExists{\jobname-pw.ind}{\input{\jobname-pw.ind}}{}

\end{document}

      