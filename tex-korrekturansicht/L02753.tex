%% latex-korrekturansicht-vorspann.tex
%% Vorspann für die Korrekturansicht.
%% Lädt die gemeinsame Datei latex-vorspann.tex mit gesetztem Schalter.

\newif\ifkorrekturansicht
\korrekturansichttrue

\input{../tex-inputs/latex-vorspann}


               \section[Paul Goldmann an Arthur Schnitzler, 15. 10. {[}1895{]}]{ Paul Goldmann an Arthur Schnitzler, 15. 10. {[}1895{]}}\nopagebreak\mylabel{v}\rehead{ }\normalsize\beginnumbering\briefempfaengerindex{Schnitzler, Arthur@\textsc{Schnitzler, Arthur}!zzzGoldmann, Paul@\emph{von Paul Goldmann}!1895-10-152@{15. 10. {[}1895{]}}|(be} \toendnotes[C]{\smallbreak\pagebreak[2]} \Standort{DLA, A:Schnitzler, HS.NZ85.1.3165.}
\physDesc{Brief, 4 Blätter, 16 Seiten
\newline{}Handschrift: blaue Tinte, deutsche Kurrent
\newline{}Schnitzler: 1) mit schwarzer Tinte das Jahr » 95« vermerkt 2) mit rotem Buntstift sechs Unterstreichungen}\toendnotes[C]{\smallbreak}\pstart
           \noindent{}{\pb}\textcolor{gray}{\textbf{\textbf{\textcolor{brown}{Frankfurter Zeitung}{}\ledrightnote{\textcolor{brown}{Frankfurter Zeitung}}}}}\pend
           \pstart
           \textcolor{gray}{\textbf{(\textcolor{brown}{\begin{otherlanguage}{french}Gazette de Francfort\end{otherlanguage}}{}\ledrightnote{\textcolor{brown}{Frankfurter Zeitung}})}}.\pend
           \pstart
           \textcolor{gray}{\textbf{\textbf{\begin{otherlanguage}{french}Fondateur M. \textcolor{blue}{L.
                                 Sonnemann}{}\ledrightnote{\textcolor{blue}{Leopold Sonnemann}}\end{otherlanguage}.}}}\hfill \textsc{\textcolor{pink}{Paris}{}\ledrightnote{\textcolor{pink}{Paris}}}, 15. October.\pend
           \pstart
           \begin{otherlanguage}{french}\textcolor{gray}{\textbf{\textcolor{green}{Journal}{}\ledrightnote{→\textcolor{green}{Frankfurter Zeitung}} politique,
                        financier,}}\end{otherlanguage}\pend
           \pstart
           \begin{otherlanguage}{french}\textcolor{gray}{\textbf{commercial et littéraire.}}\end{otherlanguage}\pend
           \pstart
           \begin{otherlanguage}{french}\textcolor{gray}{\textbf{\textbf{Paraissant trois fois par jour.}}}\end{otherlanguage}\pend
           \pstart
           \begin{otherlanguage}{french}\textcolor{gray}{\textbf{\textbf{Bureau à \textcolor{pink}{Paris}{}\ledrightnote{\textcolor{pink}{Paris}}:}}}\end{otherlanguage}\pend
           \pstart
           \begin{otherlanguage}{french}\textcolor{gray}{\textbf{\textbf{\textcolor{pink}{24. Rue Feydeau}{}\ledrightnote{\textcolor{pink}{rue Feydeau}}.}}}\end{otherlanguage}\pend
           \pstart\center{}Mein lieber Freund,\pend\pstart
           \textsc{\textcolor{blue}{Speidel}{}\ledrightnote{\textcolor{blue}{Ludwig Speidel}}s}{ }\label{K_L02753-1v}\edtext{\textcolor{green}{Feuilleton}{}\ledrightnote{→\textcolor{green}{Burgtheater. (»Liebelei«, Schauspiel in drei Aufzügen von Arthur Schnitzler. – »Rechte der Seele«, Schauspiel in einem Act von Giuseppe Giacosa, deutsch von Otto Eisenschitz.)}}}{\lemma{\textnormal{\emph{Feuilleton}}}\Cendnote{\textnormal{\textcolor{blue}{L. Sp.} [=\textcolor{blue}{Ludwig Speidel}]: \emph{\textcolor{green}{Burgtheater. (»Liebelei«, Schauspiel in drei Aufzügen von
                        Arthur Schnitzler. – »Rechte der Seele«, Schauspiel in einem Act von
                        Giuseppe Giacosa, deutsch von Otto Eisenschitz.)}}. In: \emph{\textcolor{green}{Neue Freie Presse}}, Nr. 11.184, 13. 10. 1895, Morgenblatt, S. 1–3.}}}\label{K_L02753-1h} habe ich geſtern geleſen, und es hat mich entzückt. Es iſt ſchön
               und einfach geſchrieben, und vor Allem freut es mich, daß er Deinem \uline{Character} ſo gerecht wird, daß er ſo wohl verſteht,
               wie der Werth Deiner Production \strikeout{\textcolor{gray}{ne}b} neben allem Talent auch im Moraliſchen liegt, i\substVorne{}\textsuperscript{m}\substDazwischen{}n\substHinten{} dem Muthe, in dem starken Streben, ganz einfach das Wahre zu ſagen, {\pb}unbekümmert um \strikeout{di\textcolor{gray}{e}} das Treiben und Reden der Anderen. Er iſt doch ein großer \textcolor{blue}{Kritiker}{}\ledrightnote{→\textcolor{blue}{Ludwig Speidel}} und z. B. \textsc{\textcolor{blue}{Herzl}{}\ledrightnote{\textcolor{blue}{Theodor Herzl}}} in ſeiner geſuchten und manierirten Art hätte das nie gefunden. Ob er Dich
               überſchätzt? Gewiß, er hätte Einiges tadeln können. Ich verſtehe vollſtändig, was Du
               meinſt. Ich begreife, daß es Dich in Verlegenheit ſetzt, ſo rückhaltslos gelobt zu
               werden. Vor Enttäuſchungen fürchte \uline{ich} mich zwar
               nicht. Aber ich kann es nachfühlen, daß Du, als ehrlich ſtrebender Menſch, Dich
               fortwährend unfertig {\pb}fühlſt und daß es Dir daher
               peinlich iſt, wenn man Dich als einen \strikeout{\textcolor{gray}{V}} Vollendeten hinſtellt. Ein \textsc{\textcolor{blue}{Herzl}{}\ledrightnote{\textcolor{blue}{Theodor Herzl}}}, \textsc{\textcolor{blue}{David}{}\ledrightnote{\textcolor{blue}{Jakob Julius David}}} oder \textsc{\textcolor{blue}{Nordau}{}\ledrightnote{\textcolor{blue}{Max Nordau}}} hätte \textsc{\textcolor{blue}{Speidel}{}\ledrightnote{\textcolor{blue}{Ludwig Speidel}}s}{ }\textcolor{green}{Feuilleton}{}\ledrightnote{→\textcolor{green}{Burgtheater. (»Liebelei«, Schauspiel in drei Aufzügen von Arthur Schnitzler. – »Rechte der Seele«, Schauspiel in einem Act von Giuseppe Giacosa, deutsch von Otto Eisenschitz.)}} einfach als den ihm
               gebührenden Tribut hingenommen. Du, in Deiner Beſcheidenheit und Grundehrlichkeit,
               mußteſt davon in Verlegenheit gebracht werden. Das ſtimmt Alles. Wenn aber Du ſagen
               mußt, \textsc{\textcolor{blue}{Speidel}{}\ledrightnote{\textcolor{blue}{Ludwig Speidel}}} habe \strikeout{ich} Dich überſchätzft, ſo darf \uline{ich} ſagen: Nein, er überſchätzt Dich \uline{nicht}. \strikeout{\textcolor{gray}{V}}{ }\strikeout{\textcolor{gray}{Verge}} Er ſagt von Dir gerade das, was Dir gebührt. Vergiß’ auch {\pb}nicht, mein lieber Freund, daß \textsc{\textcolor{blue}{Speidel}{}\ledrightnote{\textcolor{blue}{Ludwig Speidel}}} Dich in Deiner ganzen Art neu entdeckt – daß Deine ganze Perſönlichkeit ihm
               eine neue Erſcheinung iſt, \strikeout{\textcolor{gray}{×}} während wir dieſelbe längſt kennen – und daß er ſich mit dieſer bedeutenden
               Perſönlichkeit (entſchuldige die ſtarken Ausdrücke, aber ſie laſſen ſich nicht
               vermeiden) \strikeout{a\textcolor{gray}{b}} im Ganzen abzufinden hat, nicht blos bei deren letztem Ausſluß der, »\textcolor{green}{Liebelei}{}\ledrightnote{\textcolor{green}{Liebelei. Schauspiel in drei Akten}}«, deren kleine Mängel {\pb}er darum nicht ſieht, weil er das Geſammtbild in
               ſeinen großen Linien vor Augen hat. Das \textcolor{green}{Feuilleton}{}\ledrightnote{→\textcolor{green}{Burgtheater. (»Liebelei«, Schauspiel in drei Aufzügen von Arthur Schnitzler. – »Rechte der Seele«, Schauspiel in einem Act von Giuseppe Giacosa, deutsch von Otto Eisenschitz.)}} gilt auch mehr dem allgemeinen \textsc{Arthur Schnitzler}, als dem beſonderen \textcolor{green}{Drama}{}\ledrightnote{→\textcolor{green}{Liebelei. Schauspiel in drei Akten}}.\pend
           \pstart
           Daß der materielle Erfolg ſich nun auch einſtellt, habe ich gleichfalls
               vorausgeſehen. Ganz \textcolor{pink}{Wien}{}\ledrightnote{\textcolor{pink}{Wien}}{ }\strikeout{iſt} wird hineinlaufen, um dieſes \strikeout{ech} echt \textcolor{pink}{Wien}{}\ledrightnote{\textcolor{pink}{Wien}}er \textcolor{green}{Stück}{}\ledrightnote{→\textcolor{green}{Liebelei. Schauspiel in drei Akten}} zu \strikeout{ſehen} ſehen. {\pb}Ich bin
               wahrhaft glücklich, daß es ſo gut geht. Du ahnſt gar nicht, welch’ große \uline{materielle} Wirkung \textsc{\textcolor{blue}{Speidel}{}\ledrightnote{\textcolor{blue}{Ludwig Speidel}}s}{ }\textcolor{green}{Feuilleton}{}\ledrightnote{→\textcolor{green}{Burgtheater. (»Liebelei«, Schauspiel in drei Aufzügen von Arthur Schnitzler. – »Rechte der Seele«, Schauspiel in einem Act von Giuseppe Giacosa, deutsch von Otto Eisenschitz.)}} für Dich haben wird\substVorne{}\textsuperscript{;}\substDazwischen{}.\substHinten{} In jeder Beziehung biſt Du nun lancirt, – biſt aus der Menge der im Dunkeln
               Strebenden herausgehoben und ſtehſt auf der Höhe mit den Wenigen.\pend
           \pstart
           Um Dich dort zu erhalten, wirſt Du weiter thätig ſein, wie bisher. Und zwar muß ſich
               – das wird {\pb}ſich auch naturgemäß als
               Entwickelungs-Reſultat ergeben – Deine Kunſt erweitern und vertiefen. Sie muß, ſtatt
               wie bisher nur eine Seite des Lebens, allmälig das \uline{ganze}{ }\uline{Leben} umfaſſen. Concret \strikeout{\textcolor{gray}{l}eſ\textcolor{gray}{×}} geſprochen: Du darfſt höchſtens noch ein \label{K_L02753-55v}\edtext{Süßes-\substVorne{}\textsuperscript{M\textcolor{gray}{ä}del}\substDazwischen{}Mädel-\substHinten{} Stück}{\lemma{\textnormal{\emph{Süßes-MädelMädel- Stück}}}\Cendnote{\textnormal{vgl. Paul Goldmann an Arthur Schnitzler, 31. 12. [1894]}}}\label{K_L02753-55h} ſchreiben. Dann mußt Du hinaus ins große Ganze – immer weiter von Deines
               Herzens beſonderen Erlebniſſen weg – mußt aus dem Vollen {\pb}nehmen und geſtalten. In »\textcolor{green}{Märchen}{}\ledrightnote{\textcolor{green}{Das Märchen. Schauspiel in drei Aufzügen}}« und »\textcolor{green}{Liebelei}{}\ledrightnote{\textcolor{green}{Liebelei. Schauspiel in drei Akten}}«
               haſt Du Deine eigene Jugend poetiſch ausgeſtaltet; vielleicht wirſt Du das auch in
                  »\textcolor{green}{Freiwild}{}\ledrightnote{\textcolor{green}{Freiwild. Schauspiel in 3 Akten}}« thun; das macht nichts. Dann aber
               mußt Du zeigen, daß Du nicht nur Dein Leben, ſondern auch das Leben \strikeout{And} der Anderen zu geſtalten weißt, – das eigentliche,
               das große Leben. Wenn Du das kannſt, wirſt Du ein großer Dichter ſein\substVorne{}\textsuperscript{;}\substDazwischen{}.\substHinten{} Und ich bin überzeugt – \strikeout{auch} nach all’ dem
               Schönen, was dieſe {\pb}Tage gebracht haben, werden wir
               auch das noch \strikeout{er} erleben. Alle Zeichen deuten darauf
               hin.\pend
           \pstart
           Was Deine Umänderungs-Pläne betrifft, ſo halte ich Dein Gefühl für durchaus richtig.
               Gewiß, der alte \textsc{\textcolor{green}{Weiring}{}\ledrightnote{→\textcolor{green}{Liebelei. Schauspiel in drei Akten}}} müßte mehr hervortreten, müßte dramatiſcher werden. Die Art, wie Du ſeine
               dramatiſche \strikeout{B\textcolor{gray}{a}} Belebung Dir denkſt, finde ich {\pb}durchaus \strikeout{\textcolor{gray}{bi}ll} billigenswerth. Wenn Du Luſt und Stimmung dazu haſt,
               verſuchs immerhin. Der \label{K_L02753-2v}\edtext{zweite \textcolor{green}{Akt}{}\ledrightnote{→\textcolor{green}{Liebelei. Schauspiel in drei Akten}}}{\lemma{\textnormal{\emph{zweite Akt}}}\Cendnote{\textnormal{Am 11. 10. 1895 notierte \textcolor{blue}{Schnitzler} im \emph{\textcolor{green}{Tagebuch}}
                  die »Idee, die Schwester des alten \textcolor{green}{Weiring} in den 2. \textcolor{green}{Akt} zu bringen als Lebende«. \textcolor{blue}{Herzl} habe außerdem die Idee gehabt, »\textcolor{green}{Weir.} soll betonen, er
                     habe kein Recht, \textcolor{green}{Christine} zu halten, da er sein Leben verträumt etc.«. Ab dem
                     17. 10. 1895 arbeitete
                     \textcolor{blue}{Schnitzler} den zweiten \textcolor{green}{Akt} um, jedoch ohne je eine neue Fassung
                  fertigzustellen.}}}\label{K_L02753-2h} kann durch eine kräftige \strikeout{Sc\textcolor{gray}{e}} Scene dieſer Art nur gewinnen. Andrerſeits möchte ich Dir aber zu bedenken
               geben, daß es immerhin gewagt iſt, ein fertiges \textcolor{green}{Werk}{}\ledrightnote{→\textcolor{green}{Liebelei. Schauspiel in drei Akten}}, \uline{das auch bereits vor dem
                  Publicum ſeine Probe beſtanden hat}, nachträglich zu ändern. Werden die
               nachträglich {\pb}eingeſchobenen \textcolor{green}{Scenen}{}\ledrightnote{→\textcolor{green}{Liebelei. Schauspiel in drei Akten}} nicht einen anderen Ton anſchlagen
               und ſo den Geſammt-Ton des \textcolor{green}{Stück}{}\ledrightnote{→\textcolor{green}{Liebelei. Schauspiel in drei Akten}}es ſtören? Liegt nicht überhaupt die Gefahr \strikeout{f\textcolor{gray}{o}} vor, daß durch die nachträgliche Einſchiebung die ganze \strikeout{Ökon\textcolor{gray}{om}} Ökonomie des \textcolor{green}{Stück}{}\ledrightnote{→\textcolor{green}{Liebelei. Schauspiel in drei Akten}}es
                  \strikeout{geſ\textcolor{gray}{c}} geſchädigt wird? Das ſind Fragen, die nur Du allein beantworten kannſt. Im
               Allgemeinen bin ich, nach Erwägung aller Gründe und Gegengründe, eher {\pb}für die Änderung als dagegen. Du hältſt ſie für
               nöthig und haſt Luſt und Kraft dazu. Das iſt entſcheidend.\pend
           \pstart
           \textsc{\textcolor{blue}{Herzl}{}\ledrightnote{\textcolor{blue}{Theodor Herzl}}s} Vorſchlag gibt mir nur einen
               neuen Beweis von der Urtheilsloſigkeit des Mannes, und ich verſtehe nicht, wie Du
               ſeinen Rath als »klug« bezeichnen kannſt. Er will die Exiſtenzfrage hineinmiſchen.
               Aber, Du lieber Gott, das bringt ja ein {\pb}ganz neues
               und ganz fremdes Element in das \textcolor{green}{Stück}{}\ledrightnote{→\textcolor{green}{Liebelei. Schauspiel in drei Akten}} – das \uline{ſociale} Element, das Du, bewußt
               oder unbewußt, mit Feingefühl vermieden haſt!{\dotsfour}\pend
           \pstart
           \textsc{\textcolor{blue}{David}{}\ledrightnote{\textcolor{blue}{Jakob Julius David}}s} »\textcolor{green}{Regentag}{}\ledrightnote{\textcolor{green}{Ein Regentag}}« muß ein ſchöner Dreck ſein! Entzückend iſt die \label{K_L02753-3v}\edtext{»\textcolor{brown}{\textcolor{green}{Neue Fr. Pr.}{}\ledrightnote{→\textcolor{green}{Neue Freie Presse}}}{}\ledrightnote{\textcolor{brown}{Neue Freie Presse}}«}{\lemma{\textnormal{\emph{»Neue Fr. Pr.«}}}\Cendnote{\textnormal{O. V.: \emph{\textcolor{green}{Theater- und Kunstnachrichten.
                        [Deutsches Volkstheater.]}}. In: \emph{\textcolor{green}{Neue
                        Freie Presse}}, Nr. 1184, 13. 10. 1895,
                     Morgenblatt, S. 7.}}}\label{K_L02753-3h}, die dieſen Anlaß braucht, {\pb}um darzuthun, was für ein bedeutender Mann \textsc{\textcolor{blue}{David}{}\ledrightnote{\textcolor{blue}{Jakob Julius David}}} iſt.\pend
           \pstart
           Über \textsc{\textcolor{blue}{Bahr}{}\ledrightnote{\textcolor{blue}{Hermann Bahr}}} ſchrieb ich Dir bereits. Nochmals: ich erwarte von \textcolor{blue}{\textsc{Richard}}{}\ledrightnote{\textcolor{blue}{Richard Beer-Hofmann}} oder \textsc{\textcolor{blue}{Loris}{}\ledrightnote{\textcolor{blue}{Hugo von Hofmannsthal}}} auf das Beſtimmteſte, daß ſie dem \textcolor{blue}{Burſchen}{}\ledrightnote{→\textcolor{blue}{Hermann Bahr}} jene Zurechtweiſung zutheil werden laſſen, die
               infolge ſeiner perſönlichen Gemeinheiten unumgänglich nöthig geworden iſt, die Du ihm
               nicht ertheilen darfſt, und {\pb}die ich ihm leider,
                  \strikeout{nicht} fern von \textcolor{pink}{Wien}{}\ledrightnote{\textcolor{pink}{Wien}}, nicht ertheilen kann. Übrigens behalte ich mir doch noch ein
               Einſchreiten vor, falls die \textcolor{pink}{Wien}{}\ledrightnote{\textcolor{pink}{Wien}}er \textcolor{blue}{Freunde}{}\ledrightnote{→\textcolor{blue}{Hugo von Hofmannsthal}{\newline}→\textcolor{blue}{Richard Beer-Hofmann}} verſagen
               ſollten.\pend
           \pstart
           \label{K_L02753-4v}\edtext{\textsc{\textcolor{blue}{Granichstaedten}{}\ledrightnote{\textcolor{blue}{Emil Granichstaedten}}}}{\lemma{\textnormal{\emph{Granichstaedten}}}\Cendnote{\textnormal{Siehe \textcolor{blue}{Emil Granichstaedten}: \emph{\textcolor{green}{Deutsches Volkstheater. (»Ein Regentag«, Charakterbild von
                        J. J. David.)}}. In: \emph{\textcolor{green}{Die Presse}},
                     Jg. 48, Nr. 283, 15. 10. 1895, S. 1–2, hier:
                     S. 2. Siehe A. S.: \emph{Tagebuch}, 15. 10. 1895.}}}\label{K_L02753-4h}? Einen Dienſtmann engagiren, um ihm ins Geſicht zu \strikeout{ſpu\textcolor{gray}{ck}} ſpucken. Es lohnt nicht der Mühe, das ſelber zu thun. Aber im Sommer wart Ihr
               Beide ja ſehr verſöhnlich geſtimmt gegen den \textcolor{blue}{Herrn}{}\ledrightnote{→\textcolor{blue}{Emil Granichstaedten}}!{\dotssix}\pend
           \pstart
           {\pb}Stolz werden? Nein, nein, ich \strikeout{weiß} weiß! So meinte ich es auch nie. Ich dachte an
               etwas Anderes, das kommen wird, zwiſchen Dir und mir oder zwiſchen mir und Dir, –
               langſam, langſam, aber ich fürchte, es kommt. In dieſer Beziehung ſiehſt Du, glaube
               ich, \strikeout{nicht} nicht ſo klar, wie ſont in allen
               Dingen.\pend
           \pstart
           Viele treue Grüße, mein lieber, lieber Freund! Wie bin ich froh, Dich ſoweit zu
               haben!\pend
           \pstart Dein \spacefill\mbox{Paul Goldmann}\pend{}\endnumbering\briefempfaengerindex{Schnitzler, Arthur@\textsc{Schnitzler, Arthur}!zzzGoldmann, Paul@\emph{von Paul Goldmann}!1895-10-152@{15. 10. {[}1895{]}}|)be}\mylabel{h}\begin{anhang}\end{anhang}\normalsize

\doendnotes{C}
\bigskip
\vfill

\clearpage

\footnotesize

\lohead{\textsc{register}}

% Definiere theindex-Environment komplett neu ohne reledmac
\makeatletter
\renewenvironment{theindex}{%
  \section*{\indexname}%
  \setlength{\parindent}{0pt}%
  \setlength{\parskip}{0pt plus 0.3pt}%
  \let\item\@idxitem
}{%
  \clearpage
}
\makeatother

\IfFileExists{\jobname-pw.ind}{\input{\jobname-pw.ind}}{}

\end{document}

      