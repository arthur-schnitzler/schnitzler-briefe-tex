%% latex-korrekturansicht-vorspann.tex
%% Vorspann für die Korrekturansicht.
%% Lädt die gemeinsame Datei latex-vorspann.tex mit gesetztem Schalter.

\newif\ifkorrekturansicht
\korrekturansichttrue

\input{../tex-inputs/latex-vorspann}


               \section[Paul Goldmann an Arthur Schnitzler, 1. 10. 1890]{ Paul Goldmann an Arthur Schnitzler, 1. 10. 1890}\nopagebreak\mylabel{v}\rehead{ }\normalsize\beginnumbering\briefempfaengerindex{Schnitzler, Arthur@\textsc{Schnitzler, Arthur}!zzzGoldmann, Paul@\emph{von Paul Goldmann}!1890-10-011@{1. 10. 1890}|(be} \toendnotes[C]{\smallbreak\pagebreak[2]} \Standort{DLA, A:Schnitzler, HS.NZ85.1.3162.}
\physDesc{Brief, 3 Blätter, 11 Seiten
\newline{}Handschrift: blaue Tinte, deutsche Kurrent
\newline{}Schnitzler: mit rotem Buntstift eine Unterstreichung }\toendnotes[C]{\smallbreak}\pstart
           \noindent{}\centering{}{\pb}\textcolor{gray}{\textbf{\textbf{Adminiſtration: \textcolor{pink}{VII.
                           Seidengaſſe 7}{}\ledrightnote{\textcolor{pink}{Seidengasse}}} (\textcolor{brown}{Jos. Eberle {\kaufmannsund} Co.}{}\ledrightnote{\textcolor{brown}{Josef Eberle  Stein-, Buch und Musikaliendruckerei}})}}\pend
           \pstart
           \noindent{}\centering{}\textcolor{gray}{\textbf{\textcolor{brown}{An der Schönen Blauen Donau}{}\ledrightnote{\textcolor{brown}{An der schönen blauen Donau}}}}\pend
           \pstart
           \noindent{}\centering{}\textcolor{gray}{\textbf{Chef-Redacteur: Dr. \textcolor{blue}{F.
                        Mamroth}{}\ledrightnote{\textcolor{blue}{Fedor Mamroth}}. – Redaction: \textcolor{pink}{IX.,
                        Berggaſſe 31}{}\ledrightnote{\textcolor{pink}{Berggasse}}.}}\pend
           \pstart
           \raggedleft{}\textcolor{gray}{\textbf{\textcolor{pink}{Wien}{}\ledrightnote{\textcolor{pink}{Wien}}, den}}{ }1. October \textcolor{gray}{\textbf{18}}90.\pend
           \pstart\center{}Mein lieber Arthur!\pend\pstart
           Ich habe bei meiner Rückkehr eine wahnſinnige Arbeitslaſt vorgefunden und habe ſeit
                  geſtern{ }Morgen nicht einmal Zeit, »A« zu ſagen. Mit großer Kunſt habe ich mir
               jetzt, Abends um 10 Uhr, ein\strikeout{\textcolor{gray}{e}}{ }\strikeout{\textcolor{gray}{Pa}} Paar Minuten frei gemacht, um Dir wenigſtens zu ſagen, wie ſehr es mich zu
               einer Antwort auf Deinen letzten Brief drängt und wie ſchmerzlich ich es empfinde,
               daß ich in dieſen Tagen keine Zeit habe, all’ das Viele {\pb}Dir zu ſchreiben, das ich Dir zu ſchreiben
               hätte.\pend
           \pstart
           Nur das Allerweſentlichſte will ich rasch bemerken. Ich täusche mich gewiß nicht,
               wenn ich meine, daß wir in \label{K_L02651-1v}\edtext{\textcolor{pink}{Salzburg}{}\ledrightnote{\textcolor{pink}{Salzburg}}}{\lemma{\textnormal{\emph{Salzburg}}}\Cendnote{\textnormal{Am 27. 9. 1890, 28. 9. 1890 und 29. 9. 1890 verbrachten sie gemeinsam Zeit in
                     \textcolor{pink}{Salzburg}, wobei \textcolor{blue}{Schnitzler} sich noch immer da aufhielt und auch diesen
                  Brief am 2. 10. 1890
                  erhielt.}}}\label{K_L02651-1h} ein wenig verſtimmter, kühler und fremder geſchieden ſind, als
               dies früher zwiſchen uns Brauch war. Das heißt, Du biſt von mir ſo geſchieden, nicht
               ich von Dir. Und im Beſtreben, mir das zu motiviren, bin ich auf einen Grund
               gekommen, der mein Verhalten Dir gegenüber, das Du mir in Deinem Briefe zum Vorwurf
               machſt, ein wenig zu rechtfertigen ſcheint. Durch dieſen Deinen Brief verleitet, habe
               ich Dich nämlich rückhaltslos zum Vertrauten von einem Theile meines Leides gemacht
               und habe Dich ſogar perſönlich in dieſe unglückſeligen Vorgänge hineingezogen.
               Seitdem kann ich das Gefühl {\pb}nicht los werden – und
               Du haſt auch nichts gethan, um ſein Aufkommen zu verhindern, – daß Du geringer von
               mir denkſt und eine \textcolor{gray}{Nuance} von Widerwillen gegen mich haſt. Dieſe
               Leiden nämlich ſind ſo niedriger und gemeiner Natur, daß ſie den, der ſie tragen muß,
               nicht nur unglücklich machen, ſondern auch ſchänden. Ich ſpreche das deshalb ſo aus,
               weil ich in einem ähnlichen Fall gewiß Ähnliches empfinden würde. Das hat mit der
               Moral und \strikeout{L\textcolor{gray}{o}} Logik nichts zu thun. Wir – Du und ich – ſind eben ſo hyperſenſibel, daß uns
               alles Mißduftige und Gemeine verſtimmt, \strikeout{\textcolor{gray}{ſelbst}} felbſt wenn es ein unverſchuldetes Unglück iſt. \uline{Deine} Leiden, lieber Freund, ſind ritterlicher und cavaliermäßiger Natur,
               die meinen proletariſch und gemein. Und {\pb}die Furcht
               vor Deiner Hyperſenſibilität – ich betone nochmals, daß ich von \strikeout{D}{ }\uline{mir} auf Dich ſchließe, – iſt es hauptſächlich immer
               geweſen, was mich an vollem Vertrauen in dieſer Beziehung gehindert hat. Weniger der
               Zweifel an Deiner Theilnahme. Ich weiß, daß Du es gut und freundſchaftlich mit mir
               meinſt. Freilich glaube ich, daß in dieſer Beziehung die Rollen zwiſchen uns Beiden
               nicht ganz gleichmäßig vertheilt ſind. Ich glaube nicht, daß Du für mich jenes Gefühl
               inniger, eventuell bis zur Selbſtentäußerung gehender Zuneigung empfindeſt, das ich –
               keine Phraſe, mein Sohn! – für Dich empfinde. Erſtens weil ich mich nicht für den
               Mann halte, der imſtande iſt, bei einem Andern \strikeout{d\textcolor{gray}{e}} ein derartiges Gefühl hervorzurufen. {\pb}Und
               zweitens, weil Du doch nicht ſo durch die Schule des Lebens gegangen biſt wie ich und
               weil man eben nur in dieſer Schule – mag man von Natur mit noch ſoviel Herzensgüte
               begabt ſein – die Kunſt lernt, von ſich zu abſtrahiren und in Andern aufzugehen. Ich
               beklage mich durchaus nicht über dieſe Ungleicheit. Ich bin gewohnt, mit den
               gegebenen Verhältniſſen zu rechnen, verſtehe Deine Stellung zu mir und habe Dich
               deshalb auch nicht um ein\strikeout{\textcolor{gray}{en}} Gran weniger {\pb}gern. Hier und da nur thuſt Du
               mir weh. Und das iſt eben oft gerade in jenen Momenten, \strikeout{des} wo ich Dir von meine\substVorne{}\textsuperscript{m}\substDazwischen{}n\substHinten{} Schmerzen erzähle und wo ich nachher entweder immer das peinliche Gefühl
               habe, ich müſſe Dir dankbar dafür ſein, daß du mich angehört haſt, oder gar das
               Gefühl, daß du mich überhaupt nicht gehört haſt. Vielleicht daß ich Unrecht damit
               habe. Vielleicht, daß es richtig iſt, wenn Du ſagſt, ich litte am »Kleinheitswahn«
               und daß dann an dieſen Empfindungen ich ſchuld bin. Aber auf der andern Seite, wenn
               Du mich kennſt und meine abſcheuliche Empfindlichkeit auf dieſem Gebiete kennſt, ſo
               ſollteſt Du dieſe Empfindlichkeit nicht noch reizen, \strikeout{um
                  ſ} ſelbſt nicht durch kleine Äußerlichkeiten. Deine Zerſtreutheit {\pb}hier und da, ſagſt Du, iſt nur eine Äußerlichkeit.
               Gut! Umſo leichter müßte es Dir ſallen, ſie zu überwinden. Wenn Dir wirklich an
               meinem Vertrauen liegt, an meinem Vertrauen nämlich über \label{K_L02651-3v}\edtext{\textsc{res meae}}{\lemma{\textnormal{\emph{res meae}}}\Cendnote{\textnormal{lateinisch: meine
               Angelegenheiten}}}\label{K_L02651-3h}, ſo ſollte Dir das kleine Opfer der Rücksicht auf meine
               Empfindlichkeit kein zu hoher Preis dafür ſein.\pend
           \pstart
           Aber ich meine doch, es ginge auch ohne daß ich Dich in meine Leiden hineinziehe. Der
               Geſunde hat in der Stinkluft einer Krankenſtube nichts zu ſuchen, und Du biſt der
               Geſunde von uns zweien, ſo weh Dir auch gegenwärtig um’s Herz ſein mag. Verletzen
               darf Dich das aber nicht, das wäre kindiſch und Deiner nicht würdig. Wenn ich Dich
               mit meinen \label{K_L02651-4v}\edtext{Jeremiaden}{\lemma{\textnormal{\emph{Jeremiaden}}}\Cendnote{\textnormal{Klageliedern}}}\label{K_L02651-4h} verſchone und nur in
                  {\pb}Momenten damit herauskomme, wo mir das Herz gar
               zu voll iſt, – ſo thue ich das nicht aus Nichtachtung, ſondern aus \uline{Rücksicht} gegen Dich!{\dotsfive}\pend
           \pstart
           Vieles hätte ich Dir jetzt über das \textcolor{blue}{Mädel}{}\ledrightnote{→\textcolor{blue}{Marie Glümer}} zu ſchreiben. Der Eindruck, den ſie am \label{K_L02651-5v}\edtext{letzten{ }Abend}{\lemma{\textnormal{\emph{letzten Abend}}}\Cendnote{\textnormal{Am 29. 9. 1890 dinierten \textcolor{blue}{Goldmann}, \textcolor{blue}{Schnitzler}
                  und \textcolor{blue}{Marie Glümer} gemeinsam in \textcolor{pink}{Salzburg}.}}}\label{K_L02651-5h} auf mich gemacht, war nämlich
               ganz und gar nicht ſympathiſch, und ich habe mehr als je die Überzeugung, daß \strikeout{Du die D\textcolor{gray}{e}i\textcolor{gray}{ne}} ſich da Deine Phantaſie wieder ein \textcolor{blue}{Weſen}{}\ledrightnote{→\textcolor{blue}{Marie Glümer}} conſtruirt hat, das ſich von dem wirklichen ganz
               weſentlich unterſcheidet. Ich komme immer mehr zu der Anſicht, daß auch dieſe \textcolor{blue}{Geliebte}{}\ledrightnote{→\textcolor{blue}{Marie Glümer}} Deiner nicht würdig
               iſt. Ein liebes \textcolor{blue}{Mädel}{}\ledrightnote{→\textcolor{blue}{Marie Glümer}} ſchon,
               ein ſchönes \textcolor{blue}{Mädel}{}\ledrightnote{→\textcolor{blue}{Marie Glümer}} auch, aber
               weder ſo geſcheit, noch ſo künſtleriſch, noch auch ſo keuſch {\pb}und \textcolor{green}{grethchen}{}\ledrightnote{→\textcolor{green}{Faust}}haft als Du glaubſt. Ich kann Dir ſagen, daß mich,
               wie ich bei näherer Betrachtung herausgefunden, das Verhalten des \textcolor{blue}{Mädels}{}\ledrightnote{→\textcolor{blue}{Marie Glümer}} an dem letzten{ }Abend in manchen Beziehungen an die – \label{K_L02651-6v}\edtext{\textsc{\textcolor{blue}{Jeannette}{}\ledrightnote{\textcolor{blue}{Jeanette Heeger}}}}{\lemma{\textnormal{\emph{Jeannette}}}\Cendnote{\textnormal{\textcolor{blue}{Jeannette Heger}, \textcolor{blue}{Schnitzler}s zentrale Geliebte der letzten Jahre.}}}\label{K_L02651-6h}
               erinnert hat. Und, merkwürdig, heut war die \textsc{\label{K_L02651-44v}\edtext{\textcolor{blue}{Hildegard de St. Quentin}{}\ledrightnote{\textcolor{blue}{Hilda von Mitis}}}{\lemma{\textnormal{\emph{Hildegard de St. Quentin}}}\Cendnote{\textnormal{Es dürfte sich um ein Pseudonym von
                        \textcolor{blue}{Hildegard von Mitis} handeln. In der
                     von \textcolor{blue}{Goldmann} redaktionell betreuten
                     Zeitschrift \emph{\textcolor{green}{An der schönen blauen Donau}}
                     erschien im ersten Oktoberheft ein Text unter diesem Namen. (\emph{\textcolor{green}{Der Feiertag des Herzens. Ein Abriß}}.
                        In: \emph{\textcolor{green}{An der schönen blauen Donau}}, Jg. 5,
                        H. 20, 1. 10. 1890, S. 461–463.) Ein weiterer folgte
                        1892.}}}\label{K_L02651-44h}} wieder bei mir, – ich habe Dir einen ganzen Band über dieſes außergewöhnliche
               \textcolor{blue}{Weſen}{}\ledrightnote{→\textcolor{blue}{Hilda von Mitis}} zu erzählen – und
               da ſtellte es ſich heraus, daß {\pb}ſie im vorigen \substVorne{}\textsuperscript{Jahr}\substDazwischen{}Winter\substHinten{} das \textcolor{brown}{Conſervatorium}{}\ledrightnote{\textcolor{brown}{Konservatorium der Gesellschaft der Musikfreunde}} beſucht hat und auch
               die \textcolor{blue}{Kleine}{}\ledrightnote{→\textcolor{blue}{Marie Glümer}} kennt. »Die
               hübſche kleine \textsc{\textcolor{blue}{Chlum}{}\ledrightnote{\textcolor{blue}{Marie Glümer}}}«, ſagt ſie, »mit dem ewigen \label{K_L02651-7v}\edtext{Aſtrachankragen}{\lemma{\textnormal{\emph{Aſtrachankragen}}}\Cendnote{\textnormal{Pelzkragen}}}\label{K_L02651-7h}!«
               Und ſpricht ſich etwas ſehr von oben herab über das \textcolor{blue}{Mädel}{}\ledrightnote{→\textcolor{blue}{Marie Glümer}} aus, was im Munde dieſer \textcolor{blue}{Perſon}{}\ledrightnote{→\textcolor{blue}{Hilda von Mitis}} zweifellos weder Neid,
               noch Überholung, noch Böswilligkeit iſt.\pend
           \pstart
           Ich ſage Dir das Alles ſo brutal heraus, weil ich es für eine Medicin halte, um Dir
               den \label{K_L02651-8v}\edtext{Abſchied}{\lemma{\textnormal{\emph{Abſchied}}}\Cendnote{\textnormal{Erst 1893 flaute die Beziehung zwischen \textcolor{blue}{Schnitzler} und \textcolor{blue}{Marie Glümer} ab.}}}\label{K_L02651-8h} zu erleichtern. Du würdeſt \strikeout{mir} darum ein großes Unrecht an mir begehen, wenn Du
               mir darüber bös wäreſt.\pend
           \pstart
           Und nun, grüß’ Dich Gott, mein lieber Arthur! Alles gute Glück noch für den Reſt
               deines dortigen \label{K_L02651-9v}\edtext{Aufenthaltes}{\lemma{\textnormal{\emph{Aufenthaltes}}}\Cendnote{\textnormal{Schnitzler blieb noch bis 4. 10. 1890 in \textcolor{pink}{Salzburg}.}}}\label{K_L02651-9h} und {\pb}auf ſrohes Wiederſehen! {\\[\baselineskip]}Dein {\\[\baselineskip]}\spacefill\mbox{Paul Goldmann.}\pend
           \leftskip=0em{}\endnumbering\briefempfaengerindex{Schnitzler, Arthur@\textsc{Schnitzler, Arthur}!zzzGoldmann, Paul@\emph{von Paul Goldmann}!1890-10-011@{1. 10. 1890}|)be}\mylabel{h}  \normalsize

\doendnotes{C}
\bigskip
\vfill

\clearpage

\footnotesize

\lohead{\textsc{register}}

% Definiere theindex-Environment komplett neu ohne reledmac
\makeatletter
\renewenvironment{theindex}{%
  \section*{\indexname}%
  \setlength{\parindent}{0pt}%
  \setlength{\parskip}{0pt plus 0.3pt}%
  \let\item\@idxitem
}{%
  \clearpage
}
\makeatother

\IfFileExists{\jobname-pw.ind}{\input{\jobname-pw.ind}}{}

\end{document}

      