%% latex-korrekturansicht-vorspann.tex
%% Vorspann für die Korrekturansicht.
%% Lädt die gemeinsame Datei latex-vorspann.tex mit gesetztem Schalter.

\newif\ifkorrekturansicht
\korrekturansichttrue

\input{../tex-inputs/latex-vorspann}


\section[Arthur Schnitzler an Stefan Zweig, 25. 4. 1915]{L03761 Arthur Schnitzler an Stefan Zweig, 25. 4. 1915}
\nopagebreak\mylabel{L03761v}
\rehead{ }\normalsize\beginnumbering\briefempfaengerindex{, @\textsc{, }!zzz, @\emph{von  }!1915-04-251@{25. 4. 1915}|(be}
\toendnotes[C]{\smallbreak\pagebreak[2]}\Standort{Jerusalem, National Library of Israel, ARC. Ms. Var. 305 1 58 Stefan Zweig Collection.}
\physDesc{Briefkarte, 304 Zeichen
\newline{}Handschrift: schwarze Tinte, lateinische Kurrent}\toendnotes[C]{\smallbreak}
\pstart
           {\pb}\textcolor{gray}{\textbf{Dr. Arthur
                        Schnitzler}}\hfill 25. 4. 915\pend
           
\pstart
           \textcolor{gray}{\textbf{\textcolor{pink}{Wien XVIII.
                        Sternwartestrasse 71}\oindex{Wien@\textbf{Wien}!XVIII., Währing@\textbf{XVIII., Währing}!Sternwartestraße 71@\textbf{Sternwartestraße 71}, \emph{Wohngebäude}|pw}{}\ledrightnote{\textcolor{pink}{Sternwartestraße 71}}}}\pend
           
\pstart{}lieber Herr Doktor,\pend\vspace{0.5em}
\pstart
           für Ihre wunderbaren Worte zu \label{K_L03761-1v}\edtext{\textcolor{green}{\textcolor{blue}{Gustav Mahlers}\pwindex{Mahler, Gustav 7.\,7.\,1860 Kaliště – 18.\,5.\,1911 Wien@\textsc{Mahler, Gustav} (7.\,7.\,1860 Kaliště – 18.\,5.\,1911 Wien), \emph{Theaterleiter, Komponist, Dirigent}|pw}{}\ledrightnote{\textcolor{blue}{Gustav Mahler}}
                  Wiederkehr}\pwindex{Zweig, Stefan 28.\,11.\,1881 Wien – 23.\,2.\,1942 Petrópolis@\textsc{Zweig, Stefan} (28.\,11.\,1881 Wien – 23.\,2.\,1942 Petrópolis), \emph{Schriftsteller}!Gustav Mahlers Wiederkehr@\strich\emph{Gustav Mahlers Wiederkehr}|pw}{}\ledrightnote{\textcolor{green}{Gustav Mahlers Wiederkehr}}}{\lemma{\textnormal{\emph{Gustav … Wiederkehr}}}\Cendnote{\textnormal{\textcolor{blue}{Stefan Zweig}\pwindex{Zweig, Stefan 28.\,11.\,1881 Wien – 23.\,2.\,1942 Petrópolis@\textsc{Zweig, Stefan} (28.\,11.\,1881 Wien – 23.\,2.\,1942 Petrópolis), \emph{Schriftsteller}|pwk}: \emph{\textcolor{green}{Gustav Mahlers Wiederkehr}\pwindex{Zweig, Stefan 28.\,11.\,1881 Wien – 23.\,2.\,1942 Petrópolis@\textsc{Zweig, Stefan} (28.\,11.\,1881 Wien – 23.\,2.\,1942 Petrópolis), \emph{Schriftsteller}!Gustav Mahlers Wiederkehr@\strich\emph{Gustav Mahlers Wiederkehr}|pwk}}. In: \emph{\textcolor{green}{Neue Freie Presse}\pwindex{Neue Freie Presse@\emph{Neue Freie Presse}|pwk}}, Nr. 18.201,
                        25. 4. 1915, Morgenblatt, S. 1–4.}}}\label{K_L03761-1}
               laſſen Sie mich Ihnen von ganzem Herzen die Hand drücken. Sie haben mich im innerſten
               bewegt, und meiner \textcolor{blue}{Frau}\pwindex{Schnitzler, Olga 17.\,1.\,1882 Wien – 13.\,1.\,1970 Lugano@\textsc{Schnitzler, Olga} (17.\,1.\,1882 Wien – 13.\,1.\,1970 Lugano), \emph{Schauspielerin, Sängerin}|pwv}{}\ledrightnote{{$\rightarrow$}\emph{\textcolor{blue}{Olga Schnitzler}}} iſt
               es geradeſo ergangen. Wir danken Ihnen und hoffen Sie bald wiederzuſehen.\pend
           \pstart Wie immer der Ihrige, \spacefill\mbox{Arthur Schnitzer}\pend{}\selectlanguage{ngerman}\endnumbering\briefempfaengerindex{, @\textsc{, }!zzz, @\emph{von  }!1915-04-251@{25. 4. 1915}|)be}\mylabel{L03761h}  \normalsize

\doendnotes{C}
\bigskip
\vfill

\clearpage

\footnotesize

\lohead{\textsc{register}}

% Definiere theindex-Environment komplett neu ohne reledmac
\makeatletter
\renewenvironment{theindex}{%
  \section*{\indexname}%
  \setlength{\parindent}{0pt}%
  \setlength{\parskip}{0pt plus 0.3pt}%
  \let\item\@idxitem
}{%
  \clearpage
}
\makeatother

\IfFileExists{\jobname-pw.ind}{\input{\jobname-pw.ind}}{}

\end{document}

      