%% latex-korrekturansicht-vorspann.tex
%% Vorspann für die Korrekturansicht.
%% Lädt die gemeinsame Datei latex-vorspann.tex mit gesetztem Schalter.

\newif\ifkorrekturansicht
\korrekturansichttrue

\input{../tex-inputs/latex-vorspann}


\section[Elsa Plessner an Arthur Schnitzler, 17. 10. 1896]{L03705 Elsa Plessner an Arthur Schnitzler, 17. 10. 1896}
\nopagebreak\mylabel{L03705v}
\rehead{ }\normalsize\beginnumbering\briefempfaengerindex{Schnitzler, Arthur@\textsc{Schnitzler, Arthur}!zzzPlessner, Elsa@\emph{von Elsa Plessner}!1896-10-173@{17. 10. 1896}|(be}
\toendnotes[C]{\smallbreak\pagebreak[2]}
\correspDesc{Versand  durch Elsa Plessner am 17. 10. 1896 in Wien
\newline{}Erhalt  durch Arthur Schnitzler im Zeitraum [17. 10. 1896 – 22. 10. 1896?] in Wien}\toendnotes[C]{\smallbreak}
\Standort{DLA, A:Schnitzler, HS.1985.1.419.}
\physDesc{Brief, 1 Blatt, 3 Seiten, 1245 Zeichen
\newline{}Handschrift: schwarze Tinte, lateinische Kurrent
\newline{}Schnitzler: mit rotem Buntstift zwei Unterstreichungen }\toendnotes[C]{\smallbreak}
\pstart
           {\pb}\textcolor{pink}{I. Bäckerstraße N\textsuperscript{o} 1}\oindex{Wien@\textbf{Wien}!I., Innere Stadt@\textbf{I., Innere Stadt}!Bäckerstraße 1@\textbf{Bäckerstraße 1}, \emph{Wohngebäude}|pw}{}\ledrightnote{\textcolor{pink}{Bäckerstraße 1}}, den 17. 10. 96.\pend
           
\pstart{}Hochverehrter Herr Doctor!\pend\vspace{0.5em}
\pstart
           Gestern Abends beim \label{K_L03705-1v}\edtext{\textcolor{violet}{\textcolor{green}{\textcolor{blue}{Dörmann}\pwindex{Dörmann, Felix 29.\,5.\,1870 Wien – 26.\,10.\,1928 ebd.@\textsc{Dörmann, Felix} (29.\,5.\,1870 Wien – 26.\,10.\,1928 ebd.), \emph{Schriftsteller}|pw}{}\ledrightnote{\textcolor{blue}{Felix Dörmann}}-Premièren}\pwindex{Dörmann, Felix 29.\,5.\,1870 Wien – 26.\,10.\,1928 ebd.@\textsc{Dörmann, Felix} (29.\,5.\,1870 Wien – 26.\,10.\,1928 ebd.), \emph{Schriftsteller}!Sein Sohn. Schauspiel in vier Acten@\strich\emph{Sein Sohn. Schauspiel in vier Acten}|pwv}{}\ledrightnote{{$\rightarrow$}\emph{\textcolor{green}{Sein Sohn. Schauspiel in vier Acten}}}feste}\eventindex{Raimund-Theater@\textbf{Raimund-Theater}!Uraufführung von Sein Sohn, 16.10.1896@Uraufführung von Sein Sohn, 16.10.1896|pw}{}\ledrightnote{\textcolor{violet}{Uraufführung von Sein Sohn, 16.10.1896}}}{\lemma{\textnormal{\emph{Dörmann-Premièrenfeste}}}\Cendnote{\textnormal{Am 16. 10. 1896 hatte im \textcolor{pink}{Raimund-Theater}\oindex{Wien@\textbf{Wien}!VI., Mariahilf@\textbf{VI., Mariahilf}!Raimund-Theater@\textbf{Raimund-Theater}, \emph{Theater}|pwk} die Uraufführung von \textcolor{blue}{Felix
                     Dörmanns}\pwindex{Dörmann, Felix 29.\,5.\,1870 Wien – 26.\,10.\,1928 ebd.@\textsc{Dörmann, Felix} (29.\,5.\,1870 Wien – 26.\,10.\,1928 ebd.), \emph{Schriftsteller}|pwk} Drama \emph{\textcolor{green}{Sein Sohn}\pwindex{Dörmann, Felix 29.\,5.\,1870 Wien – 26.\,10.\,1928 ebd.@\textsc{Dörmann, Felix} (29.\,5.\,1870 Wien – 26.\,10.\,1928 ebd.), \emph{Schriftsteller}!Sein Sohn. Schauspiel in vier Acten@\strich\emph{Sein Sohn. Schauspiel in vier Acten}|pwk}}
                  stattgefunden, die auch \textcolor{blue}{Schnitzler} besucht
                  hatte, vgl. A. S.: \emph{Tagebuch}, 16. 10. 1896.}}}\label{K_L03705-1}
               erfuhr ich von Herrn D\textsuperscript{r}{ }\textcolor{blue}{Leo Hirschfeld}\pwindex{Feld, Leo 14.\,2.\,1869 Augsburg – 5.\,9.\,1924 Florenz@\textsc{Feld, Leo} (14.\,2.\,1869 Augsburg – 5.\,9.\,1924 Florenz), \emph{Schriftsteller, Übersetzer, Dirigent}|pw}{}\ledrightnote{\textcolor{blue}{Leo Feld}}, dass
               Director \textcolor{blue}{Brahm}\pwindex{Brahm, Otto 5.\,2.\,1856 Hamburg – 28.\,11.\,1912 Berlin@\textsc{Brahm, Otto} (5.\,2.\,1856 Hamburg – 28.\,11.\,1912 Berlin), \emph{Theaterleiter, Regisseur}|pw}{}\ledrightnote{\textcolor{blue}{Otto Brahm}} sich in \textcolor{pink}{Wien}\oindex{Wien@\textbf{Wien}, \emph{Verwaltungsgebiet}|pw}{}\ledrightnote{\textcolor{pink}{Wien}} befindet. Sie können sich denken, wie erstaunt und erfreut
               ich war, denn ich ventilirte mit \textcolor{blue}{Mama}\pwindex{Plessner, Clementine 7.\,12.\,1855 Wien – 27.\,2.\,1943 Konzentrationslager Theresienstadt@\textsc{Plessner, Clementine} (7.\,12.\,1855 Wien – 27.\,2.\,1943 Konzentrationslager Theresienstadt), \emph{Schauspielerin, Filmschauspielerin}|pwv}{}\ledrightnote{{$\rightarrow$}\emph{\textcolor{blue}{Clementine Plessner}}} bereits die Frage einer kurzen Reise nach {\pb}\textcolor{pink}{Berlin}\oindex{Berlin@\textbf{Berlin}, \emph{Hauptstadt}|pw}{}\ledrightnote{\textcolor{pink}{Berlin}} –. Da Sie mir einmal den keinen Finger
               gereicht haben, so bitte ich Sie, jetzt, falls Sie meine \textcolor{green}{Arbeit}\pwindex{Plessner, Elsa 22.\,8.\,1875 Wien – 7.\,5.\,1932 Alicante@\textsc{Plessner, Elsa} (22.\,8.\,1875 Wien – 7.\,5.\,1932 Alicante), \emph{Schriftstellerin}!Heimweh [dreiaktige Tragikomödie]@\strich\emph{Heimweh [dreiaktige Tragikomödie]}|pwv}{}\ledrightnote{{$\rightarrow$}\emph{\textcolor{green}{Heimweh [dreiaktige Tragikomödie]}}} dessen würdig erachten, Ihre ganze,
               vielvermögende Hand dabei ins Spiel zu bringen und mir mitzutheilen, ob und wie ich
               mit Herrn Director \textcolor{blue}{Brahm}\pwindex{Brahm, Otto 5.\,2.\,1856 Hamburg – 28.\,11.\,1912 Berlin@\textsc{Brahm, Otto} (5.\,2.\,1856 Hamburg – 28.\,11.\,1912 Berlin), \emph{Theaterleiter, Regisseur}|pw}{}\ledrightnote{\textcolor{blue}{Otto Brahm}} diesbezüglich \introOben{}(meiner \textcolor{green}{Arbeit}\pwindex{Plessner, Elsa 22.\,8.\,1875 Wien – 7.\,5.\,1932 Alicante@\textsc{Plessner, Elsa} (22.\,8.\,1875 Wien – 7.\,5.\,1932 Alicante), \emph{Schriftstellerin}!Heimweh [dreiaktige Tragikomödie]@\strich\emph{Heimweh [dreiaktige Tragikomödie]}|pwv}{}\ledrightnote{{$\rightarrow$}\emph{\textcolor{green}{Heimweh [dreiaktige Tragikomödie]}}})\introOben{} mich in \strikeout{D} directes
               Einvernehmen setzen soll. Sie sind doch einmal der gute Geist – {\pb}der
               liebe Herrgott muss sich noch viel mehr Bitten gefallen lassen! Von Dankbarkeit und
               s. w. will und kann ich Ihnen nicht reden, weil wir doch Beide wißen, was dran ist –
               aber wenn ich auch nicht rede – Sie werden sehen – – !! Wirklich! Verehrter, einziger
               Herr Doctor, wenn Sie mir den Herrn \textcolor{blue}{Director}\pwindex{Brahm, Otto 5.\,2.\,1856 Hamburg – 28.\,11.\,1912 Berlin@\textsc{Brahm, Otto} (5.\,2.\,1856 Hamburg – 28.\,11.\,1912 Berlin), \emph{Theaterleiter, Regisseur}|pwv}{}\ledrightnote{{$\rightarrow$}\emph{\textcolor{blue}{Otto Brahm}}} auf \strikeout{45} 1 Stunde
               festnageln könnten, daß ich ihm mein \textcolor{green}{Stück}\pwindex{Plessner, Elsa 22.\,8.\,1875 Wien – 7.\,5.\,1932 Alicante@\textsc{Plessner, Elsa} (22.\,8.\,1875 Wien – 7.\,5.\,1932 Alicante), \emph{Schriftstellerin}!Heimweh [dreiaktige Tragikomödie]@\strich\emph{Heimweh [dreiaktige Tragikomödie]}|pwv}{}\ledrightnote{{$\rightarrow$}\emph{\textcolor{green}{Heimweh [dreiaktige Tragikomödie]}}} vorlese – –  wenn Sie das thun würden!! Geht’s? – Sie
               haben doch so viel Einfluß!! – Bitte!\pend
           
\pstart
           N. B. Ohne Unbescheidenheit. Ich soll \uline{gut}
               vorlesen wie man sagt! – – Bitte um Nachricht! – \label{K_L03705-2v}\edtext{\begin{otherlanguage}{french}Sans phrase\end{otherlanguage}}{\lemma{\textnormal{\emph{Sans phrase}}}\Cendnote{\textnormal{französisch: ohne Umschweife}}}\label{K_L03705-2} in
               Ewigkeit ergeben\pend
           \pstart \spacefill\mbox{Elsa Plessner}\pend{}\selectlanguage{ngerman}\endnumbering\briefempfaengerindex{Schnitzler, Arthur@\textsc{Schnitzler, Arthur}!zzzPlessner, Elsa@\emph{von Elsa Plessner}!1896-10-173@{17. 10. 1896}|)be}\mylabel{L03705h}  \normalsize

\doendnotes{C}
\bigskip
\vfill

\clearpage

\footnotesize

\lohead{\textsc{register}}

% Definiere theindex-Environment komplett neu ohne reledmac
\makeatletter
\renewenvironment{theindex}{%
  \section*{\indexname}%
  \setlength{\parindent}{0pt}%
  \setlength{\parskip}{0pt plus 0.3pt}%
  \let\item\@idxitem
}{%
  \clearpage
}
\makeatother

\IfFileExists{\jobname-pw.ind}{\input{\jobname-pw.ind}}{}

\end{document}

      