%% latex-korrekturansicht-vorspann.tex
%% Vorspann für die Korrekturansicht.
%% Lädt die gemeinsame Datei latex-vorspann.tex mit gesetztem Schalter.

\newif\ifkorrekturansicht
\korrekturansichttrue

\input{../tex-inputs/latex-vorspann}


\renewcommand{\erwaehntePersonen}{Personen: Albert Bassermann, Richard Beer-Hofmann, Paula Beer-Hofmann, Mirjam Beer-Hofmann, Dora Erl, Julius von Gans-Ludassy, Olga von Gans-Ludassy, Theodor Herzl, Hugo von Hofmannsthal, Arthur Kaufmann, Alfred Kerr, Anna Loew, Johann Loew, Charlotte Pohl-Glas, Jacob Pollak, Emanuel Reicher, Felix Salten, Felix Speidel, Else Speidel-Haeberle}
\renewcommand{\erwaehnteInstitutionen}{Institutionen: S. Fischer Verlag}
\renewcommand{\erwaehnteOrte}{Orte: Berlin, Brüssel, Deutsches Theater Berlin, Dresden, Edmund-Weiß-Gasse 7, Griechenland, Heringsdorf, Hinterbrühl, Hotel Radetzky, Meissl {\kaufmannsund}  Schadn, Spanien, Theater an der Wien, Wartburg, Waterloo, Wien, Świnoujście}
\renewcommand{\erwaehnteWerke}{Werke: Altern, Der einsame Weg, Der einsame Weg. Schauspiel in fünf Akten, Die Historie von König David. Ein Zyklus, Die Zeit, Die fremde Stadt. Thema mit Variationen, Herr Wenzel auf Rehberg. Novelle}
\section[ Arthur Schnitzler an Felix Salten, 16. 5. 1906]{Arthur Schnitzler an Felix Salten, 16. 5. 1906}
\nopagebreak\mylabel{v}
\rehead{ }\normalsize\beginnumbering\briefempfaengerindex{Salten, Felix@\textsc{Salten, Felix}!zzzSchnitzler, Arthur@\emph{von Arthur Schnitzler}!1906-05-161@{16. 5. 1906}|(be}
\toendnotes[C]{\smallbreak\pagebreak[2]}\Standort{Wienbibliothek im Rathaus, ZPH 1681, 2.1.516.}
\physDesc{Brief, 2 Blätter, 8 Seiten, 4271 Zeichen
\newline{}Handschrift: schwarze Tinte, deutsche Kurrent
\newline{}Ordnung: mit Bleistift von unbekannter Hand Nummerierung der Doppelseiten des
                                 Konvoluts: »12«–»15« }\toendnotes[C]{\smallbreak}
\pstart
           \noindent{}{\pb}\textcolor{gray}{\textbf{Dr. Arthur Schnitzler}}\hfill 16. Mai 906\pend
           
\pstart
           \textcolor{gray}{\textbf{\textcolor{pink}{Wien XVIII. Spoettelgasse 7}{}\ledrightnote{\textcolor{pink}{Edmund-Weiß-Gasse 7}}.}}\pend
           
\pstart
           lieber, beim Nachhauſeko{\geminationm}en aus \label{K_L03005-1v}\edtext{\textcolor{pink}{Theater}{}\ledrightnote{{$\rightarrow$}\textcolor{pink}{Theater an der Wien}} und \textcolor{pink}{Hotel}{}\ledrightnote{{$\rightarrow$}\textcolor{pink}{Meissl {\kaufmannsund} Schadn}}}{\lemma{\textnormal{\emph{Theater und Hotel}}}\Cendnote{\textnormal{siehe A. S.: \emph{Tagebuch}, 15. 5. 1906}}}\label{K_L03005-1h} hab \label{T_L03005-1v}\edtext{ich}{\lemma{\textnormal{\emph{ich}}}\Cendnote{\textnormal{in der Vorlage steht: »ich ich«}}}\label{T_L03005-1h} Ihren
               kurzen aber klingenden \label{K_L03005-2v}\edtext{Brief}{\lemma{\textnormal{\emph{Brief}}}\Cendnote{\textnormal{Felix Salten an Arthur Schnitzler, 14. 5. 1906}}}\label{K_L03005-2h} vorgefunden und mich ſehr damit gefreut. Es mußte für mich freilich nicht
               gerade der \textcolor{green}{Einſ. Weg}{}\ledrightnote{\textcolor{green}{Der einsame Weg. Schauspiel in fünf Akten}} kommen, um mich Ihr
               Fernſein ſchmerzlich empfinden zu laſſen. Der Abend{ }geſtern iſt überraſchend gut ausgefallen: jedenfalls
               war er äußerlich der ſtärkſte Erfolg meiner Theaterlaufbahn. Völlige Stu{\geminationm}heit nach dem erſten \textcolor{green}{Akt}{}\ledrightnote{{$\rightarrow$}\textcolor{green}{Der einsame Weg. Schauspiel in fünf Akten}}, wahre »Stürme« nach \textcolor{green}{2.}{}\ledrightnote{{$\rightarrow$}\textcolor{green}{Der einsame Weg. Schauspiel in fünf Akten}}, \textcolor{green}{3.}{}\ledrightnote{{$\rightarrow$}\textcolor{green}{Der einsame Weg. Schauspiel in fünf Akten}}, gedämpft nach dem \textcolor{green}{4}{}\ledrightnote{{$\rightarrow$}\textcolor{green}{Der einsame Weg. Schauspiel in fünf Akten}},
               wieder ſehr ſtark {\pb}nach dem \textcolor{green}{5. Akt}{}\ledrightnote{{$\rightarrow$}\textcolor{green}{Der einsame Weg. Schauspiel in fünf Akten}}. \textcolor{blue}{Baſſermann}{}\ledrightnote{\textcolor{blue}{Albert Bassermann}} anfangs etwas bläßlich, am Schluſs unvergleichlich. \label{K_L03005-3v}\edtext{\textcolor{blue}{Reicher}{}\ledrightnote{\textcolor{blue}{Emanuel Reicher}} hat mich in gewiſſem Sinne angenehm
               enttäuſcht}{\lemma{\textnormal{\emph{Reicher … enttäuſcht}}}\Cendnote{\textnormal{vgl. Felix Salten u. a. an Arthur Schnitzler, 19. 4. 1906}}}\label{K_L03005-3h}. Im ganzen war er wohl unerträglich genug; aber die Leiſtung als ganzes
               war von einer gewiſſen Geſchloſſenheit, ſo daſs man einen mehr menſchlichen als
               künſtleriſchen Widerwillen gegen die \label{K_L03005-4v}\edtext{\textcolor{green}{Figur}{}\ledrightnote{{$\rightarrow$}\textcolor{green}{Der einsame Weg. Schauspiel in fünf Akten}}}{\lemma{\textnormal{\emph{Figur}}}\Cendnote{\textnormal{\textcolor{blue}{Albert Bassermann} spielte den \textcolor{green}{Stephan von Sala}.}}}\label{K_L03005-4h}
               kriegte. – Seltſam ſind doch Dramenſchicksale. Eine ſolche Aufnahme \label{K_L03005-5v}\edtext{in \textcolor{pink}{Berlin}{}\ledrightnote{\textcolor{pink}{Berlin}} vor 2 ½ Jahren}{\lemma{\textnormal{\emph{in Berlin vor 2 ½ Jahren}}}\Cendnote{\textnormal{Uraufführung von \emph{\textcolor{green}{Der einsame Weg}} am \textcolor{pink}{Deutschen Theater Berlin} am 13. 2. 1904}}}\label{K_L03005-5h} – und Ihre Profezeihung wäre erfüllt geweſen.\pend
           
\pstart
           – Den \label{K_L03005-6v}\edtext{\textcolor{green}{Rehberg}{}\ledrightnote{\textcolor{green}{Herr Wenzel auf Rehberg. Novelle}} hab ich in der \textcolor{pink}{Hinterbrühl}{}\ledrightnote{\textcolor{pink}{Hinterbrühl}}}{\lemma{\textnormal{\emph{Rehberg … Hinterbrühl}}}\Cendnote{\textnormal{siehe A. S.: \emph{Tagebuch}, 8. 5. 1906}}}\label{K_L03005-6h} geleſen, wo wir höchſt angenehme \label{K_L03005-7v}\edtext{acht Tage im Hotel \textcolor{pink}{Radetzky}{}\ledrightnote{\textcolor{pink}{Hotel Radetzky}}}{\lemma{\textnormal{\emph{acht … Radetzky}}}\Cendnote{\textnormal{von 7. 5. 1906 bis 14. 5. 1906}}}\label{K_L03005-7h}{ }{\pb}gewohnt und \textsc{Tennis}
               geſpielt haben (Einmal mit \label{K_L03005-8v}\edtext{\textcolor{blue}{Hugo}{}\ledrightnote{\textcolor{blue}{Hugo von Hofmannsthal}}, den ich im \textsc{\begin{otherlanguage}{english}single set\end{otherlanguage}} 6:4 ſchlug}{\lemma{\textnormal{\emph{Hugo, … set 6:4 ſchlug}}}\Cendnote{\textnormal{siehe A. S.: \emph{Tagebuch}, 11. 5. 1906}}}\label{K_L03005-8h}!) – Es iſt ein glänzendes \textcolor{green}{Ding}{}\ledrightnote{{$\rightarrow$}\textcolor{green}{Herr Wenzel auf Rehberg. Novelle}}, und es gibt vielleicht im ganzen darin nur 3–5 Stellen, bei denen mir
               im Stil irgend was wie ein falſcher Ton erſcheint. Doch möcht ichs, nach einem
               Zwiſchenraum von ein paar Wochen, noch einmal leſen, um mich ſelber nachzuprüfen.
               Hingegen ſage ich ſchon heute mit Entſchiedenheit,
               daſs ich den vorletzten \textcolor{green}{Abſatz}{}\ledrightnote{{$\rightarrow$}\textcolor{green}{Herr Wenzel auf Rehberg. Novelle}}
               fortwünſchte. Hier werden Zuſa{\geminationm}enhänge mit einer meinen
               Geſchmack ſtörenden Deutlichkeit aufgezeigt; \strikeout{die}
                  Zuſa{\geminationm}enhänge, die im {\pb}Gang der \textcolor{green}{Geſchichte}{}\ledrightnote{{$\rightarrow$}\textcolor{green}{Herr Wenzel auf Rehberg. Novelle}}{ }\strikeout{wirklich} für jeden erſichtlich werden, der in
               anſtändiger Weiſe zu leſen verſteht, und mir erſchien daher dieſer ganze \textcolor{green}{Abſatz}{}\ledrightnote{{$\rightarrow$}\textcolor{green}{Herr Wenzel auf Rehberg. Novelle}} wie eine
                  \textcolor{gray}{Referenz} vor den oberflächlichen, die ihnen nicht gebührt. Ich
               habe mich natürlich auch gefragt, ob dieſer Rückblick vielleicht als Ergänzung zum
               Charakterbild des \textcolor{green}{Erzähler}{}\ledrightnote{{$\rightarrow$}\textcolor{green}{Herr Wenzel auf Rehberg. Novelle}}s
               Ihnen unerläßlich ſcheinen mochte – doch find ich daſs die etwa neuen Züge höchſtens
               um Sinne philoſophiſcher Altersveränderungen zu deuten wären, die mit dem
               köſtlich-fertigen Chronik-\textcolor{green}{Rehberg}{}\ledrightnote{\textcolor{green}{Herr Wenzel auf Rehberg. Novelle}}, den Sie
               geſtalteten, nichts weiter zu thun haben. Auch wirkt {\pb}die \textcolor{green}{Stelle}{}\ledrightnote{{$\rightarrow$}\textcolor{green}{Herr Wenzel auf Rehberg. Novelle}}, wo \textcolor{green}{Rehberg}{}\ledrightnote{{$\rightarrow$}\textcolor{green}{Herr Wenzel auf Rehberg. Novelle}} zum Selbſtankläger wird »\textcolor{green}{Und da{\geminationn} hat mich dies Treiben ſo
                  weit von meinem Worte fortgeriſſen}{}\ledrightnote{{$\rightarrow$}\textcolor{green}{Herr Wenzel auf Rehberg. Novelle}}{ }\textsc{etc}« keineswegs bezwingend wahr. Weder ſubjectiv noch
               objektiv. – Ich würde daher \label{K_L03005-9v}\edtext{in der \textcolor{green}{Buchausgabe}{}\ledrightnote{{$\rightarrow$}\textcolor{green}{Herr Wenzel auf Rehberg. Novelle}} von dem \textcolor{green}{Abſatz}{}\ledrightnote{{$\rightarrow$}\textcolor{green}{Herr Wenzel auf Rehberg. Novelle}} nur die erſten Zeilen
               ſtehen laſſen bei »\textcolor{green}{als der Kaiſer
                  gegen ihn geweſen}{}\ledrightnote{{$\rightarrow$}\textcolor{green}{Herr Wenzel auf Rehberg. Novelle}}« – oder nicht einmal die}{\lemma{\textnormal{\emph{in … die}}}\Cendnote{\textnormal{\textcolor{blue}{Salten} übernahm \textcolor{blue}{Schnitzler}s Vorschläge für die 1907 bei \emph{\textcolor{brown}{S. Fischer}} erschienene
                  Buchausgabe von \emph{\textcolor{green}{Herr Wenzel auf Rehberg}}
                  nicht.}}}\label{K_L03005-9h} – und ruhig auf den letzten Abſatz übergehen. –\pend
           
\pstart
           Ihr \label{K_L03005-10v}\edtext{\textcolor{pink}{Berlin}{}\ledrightnote{\textcolor{pink}{Berlin}}er \textcolor{green}{Feu{[}i{]}lleton}{}\ledrightnote{{$\rightarrow$}\textcolor{green}{Die fremde Stadt. Thema mit Variationen}}}{\lemma{\textnormal{\emph{Berliner Feuilleton}}}\Cendnote{\textnormal{\textcolor{blue}{Felix Salten}: \emph{\textcolor{green}{Die fremde Stadt. Thema mit Variationen}}. In: \emph{\textcolor{green}{Die Zeit}}, Jg. 5, Nr. 1.304, 13. 5. 1906, Morgenblatt, S. 1–3.}}}\label{K_L03005-10h} in
               der \textcolor{green}{Zeit}{}\ledrightnote{\textcolor{green}{Die Zeit}} hab ich mit Ergriffenheit geleſen. Sind
                  {\pb}Sie nun ſchon an der \textsc{\textcolor{blue}{Herzl}{}\ledrightnote{\textcolor{blue}{Theodor Herzl}}}-Biographie? Und welches ſind die größten Sachen, die Sie componiren? – Die
                  \label{K_L03005-11v}\edtext{\textcolor{pink}{Wartburg}{}\ledrightnote{\textcolor{pink}{Wartburg}}erreiſe}{\lemma{\textnormal{\emph{Wartburgerreiſe}}}\Cendnote{\textnormal{siehe Felix Salten, Paul Lindau und Marie Barthel an Arthur
               Schnitzler, 9. 5. 1906}}}\label{K_L03005-11h} war ein Ausflug zum Vergnügen oder ſonſt was?– Wie ſtehts mit \label{K_L03005-12v}\edtext{\textcolor{pink}{Spanien}{}\ledrightnote{\textcolor{pink}{Spanien}}}{\lemma{\textnormal{\emph{Spanien}}}\Cendnote{\textnormal{siehe Felix Salten an Arthur Schnitzler, 1. 5. 1906}}}\label{K_L03005-12h}?– Unser Kinderarzt Dr \textsc{\textcolor{blue}{Pollak}{}\ledrightnote{\textcolor{blue}{Jacob Pollak}}} theilt mir mit, dſs \textcolor{pink}{Heringsdorf}{}\ledrightnote{\textcolor{pink}{Heringsdorf}} u
               beſonders \textsc{\textcolor{pink}{Swinemünde}{}\ledrightnote{\textcolor{pink}{Świnoujście}}} enorm gelſengeplagt ſind.\footnote{\noindent{}Er war in \textcolor{pink}{Sw.}} Erkundg Sie ſich doch gut, eh Sie miethen. –\pend
           
\pstart
           Eben bekam ich von \textcolor{blue}{Ludaſſy}{}\ledrightnote{\textcolor{blue}{Julius von Gans-Ludassy}} eine Gratul-Karte
               zum geſtrigen \textcolor{green}{Erfolg}{}\ledrightnote{{$\rightarrow$}\textcolor{green}{Der einsame Weg. Schauspiel in fünf Akten}}. Seine
                  \textcolor{blue}{Frau}{}\ledrightnote{{$\rightarrow$}\textcolor{blue}{Olga von Gans-Ludassy}} hat eben eine
               ſchwere Lungenentzündg durchgemacht, und ich muſs ſie \label{K_L03005-13v}\edtext{nächſtens beſuchen}{\lemma{\textnormal{\emph{nächſtens beſuchen}}}\Cendnote{\textnormal{siehe A. S.: \emph{Tagebuch}, 2. 6. 1906}}}\label{K_L03005-13h}. So wär es mir ſehr lieb, {\pb}we{\geminationn} Sie mir raſch nur mit 2 Worten \strikeout{mit} ſagten, wie nun eigentlich Ihre \label{K_L03005-14v}\edtext{Prozeſsſache}{\lemma{\textnormal{\emph{Prozeſsſache}}}\Cendnote{\textnormal{siehe Felix Salten an Arthur Schnitzler, 9. 3. 1906}}}\label{K_L03005-14h} ſteht?–\pend
           
\pstart
           Frl \textcolor{blue}{Erl}{}\ledrightnote{\textcolor{blue}{Dora Erl}} iſt ab nach \textcolor{pink}{Dresden}{}\ledrightnote{\textcolor{pink}{Dresden}} (vorläufg ohne beſti{\geminationm}tes
                  Engagement){[}.{]}{ }\textsc{Tennis} regelmäßig \textsc{\textcolor{blue}{Kaufma{\geminationn}}{}\ledrightnote{\textcolor{blue}{Arthur Kaufmann}}}, manchmal \textsc{\textcolor{blue}{Speidels}{}\ledrightnote{\textcolor{blue}{Felix Speidel}{\newline}\textcolor{blue}{Else Speidel-Haeberle}}} (\textcolor{blue}{er}{}\ledrightnote{{$\rightarrow$}\textcolor{blue}{Felix Speidel}} kam erſt jüngſt aus
                  \textcolor{pink}{Griechenland}{}\ledrightnote{\textcolor{pink}{Griechenland}} zurück). –\pend
           
\pstart
           – \label{K_L03005-15v}\edtext{\textcolor{blue}{Richard}{}\ledrightnote{\textcolor{blue}{Richard Beer-Hofmann}} war einmal bei uns in der \textcolor{pink}{Hinterbrühl}{}\ledrightnote{\textcolor{pink}{Hinterbrühl}}, mit \textcolor{blue}{Paula}{}\ledrightnote{\textcolor{blue}{Paula Beer-Hofmann}} u \textcolor{blue}{Mirjam}{}\ledrightnote{\textcolor{blue}{Mirjam Beer-Hofmann}}}{\lemma{\textnormal{\emph{Richard … Mirjam}}}\Cendnote{\textnormal{siehe A. S.: \emph{Tagebuch}, 12. 5. 1906}}}\label{K_L03005-15h}; ſehr erfüllt von ſeinem \label{K_L03005-16v}\edtext{\textcolor{green}{Fünfabend Stück}{}\ledrightnote{{$\rightarrow$}\textcolor{green}{Die Historie von König David. Ein Zyklus}}}{\lemma{\textnormal{\emph{Fünfabend Stück}}}\Cendnote{\textnormal{der Dramenzyklus \emph{\textcolor{green}{Die Historie von König David}}}}}\label{K_L03005-16h}. Erfülltſein iſt doch der neidenswertheſte Zuſtand von allen; – we{\geminationn} nicht die Verpflichtungsgefühle ſich einſtellen – die
               oft trügeriſch ſind, we{\geminationn} ſie ſich auf uns ſelbſt, und
               immer we{\geminationn} ſie ſich auf die Welt (ſowohl »Mit« als
               »Nach«) {\pb}beziehen. Dies iſt eine Wahrheit.
               Sollte es aber nicht wahrere Wahrheiten geben?\pend
           
\pstart
           – Wir haben ein neues Fräulein, angenehm jüdiſch, \textcolor{blue}{Anna Loew}{}\ledrightnote{\textcolor{blue}{Anna Loew}} betitelt, und wegen einer Halsentzündg in \textcolor{pink}{Hinterbrühl}{}\ledrightnote{\textcolor{pink}{Hinterbrühl}} zurückgeblieben. Sie hat einen Bruder, \textsc{\textcolor{blue}{Johann Loew}{}\ledrightnote{\textcolor{blue}{Johann Loew}}}, Arbeiterführer, und ſo bekam ich plötzlich aus \textcolor{pink}{Brüſſel}{}\ledrightnote{\textcolor{pink}{Brüssel}} eine, \textsc{resp.} zwei \textcolor{pink}{waterlo}{}\ledrightnote{\textcolor{pink}{Waterloo}}hende Karten, von \textsc{\textcolor{blue}{Johann Loew}{}\ledrightnote{\textcolor{blue}{Johann Loew}}} und \textsc{\textcolor{blue}{Lotte Pohl-Glas}{}\ledrightnote{\textcolor{blue}{Charlotte Pohl-Glas}}}. Wer die Zuſa{\geminationm}enhänge begreift, lebt ewig.\pend
           
\pstart
           Dies wünſcht Ihnen, nebſt vielen herzlichen Güßen für Sie und die Ihren von uns
               allen. {\\[\baselineskip]}Ihr {\\[\baselineskip]}\spacefill\mbox{Arthur}\pend
           \leftskip=0em{}
\pstart
           \noindent{}\textcolor{blue}{Richard}{}\ledrightnote{\textcolor{blue}{Richard Beer-Hofmann}} hat zwei ſchöne \textcolor{green}{Gedichte}{}\ledrightnote{\textcolor{green}{Der einsame Weg}{\newline}\textcolor{green}{Altern}} geſchrieben, eins »\textcolor{green}{Der einſame Weg}{}\ledrightnote{\textcolor{green}{Der einsame Weg}}« – ein andres »\textcolor{green}{Altern}{}\ledrightnote{\textcolor{green}{Altern}}«, 1 an mich, 1 an \textsc{\textcolor{blue}{Kerr}{}\ledrightnote{\textcolor{blue}{Alfred Kerr}}}.\pend
           \endnumbering\briefempfaengerindex{Salten, Felix@\textsc{Salten, Felix}!zzzSchnitzler, Arthur@\emph{von Arthur Schnitzler}!1906-05-161@{16. 5. 1906}|)be}\mylabel{h}  \normalsize

\doendnotes{C}
\bigskip
\vfill

\clearpage

\footnotesize

\lohead{\textsc{register}}

% Definiere theindex-Environment komplett neu ohne reledmac
\makeatletter
\renewenvironment{theindex}{%
  \section*{\indexname}%
  \setlength{\parindent}{0pt}%
  \setlength{\parskip}{0pt plus 0.3pt}%
  \let\item\@idxitem
}{%
  \clearpage
}
\makeatother

\IfFileExists{\jobname-pw.ind}{\input{\jobname-pw.ind}}{}

\end{document}

      