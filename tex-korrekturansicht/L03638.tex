%% latex-korrekturansicht-vorspann.tex
%% Vorspann für die Korrekturansicht.
%% Lädt die gemeinsame Datei latex-vorspann.tex mit gesetztem Schalter.

\newif\ifkorrekturansicht
\korrekturansichttrue

\input{../tex-inputs/latex-vorspann}


\section[Stefan Zweig an Arthur Schnitzler, {[}12. oder 13.?{]} 6. {[}1913?{]}]{L03638 Stefan Zweig an Arthur Schnitzler, {[}12. oder 13.?{]} 6. {[}1913?{]}}
\nopagebreak\mylabel{L03638v}
\rehead{ }\normalsize\beginnumbering\briefempfaengerindex{Schnitzler, Arthur@\textsc{Schnitzler, Arthur}!zzzZweig, Stefan@\emph{von Stefan Zweig}!1913-06-131@{{[}12. oder
                  13.?{]} 6. {[}1913?{]}}|(be}
\toendnotes[C]{\smallbreak\pagebreak[2]}\Standort{CUL, Schnitzler, B 118.}
\physDesc{Bildpostkarte, 314 Zeichen
\newline{}Handschrift: blaue Tinte, lateinische Kurrent
\newline{}Versand: Stempel: »\nobreak{}\oindex{VIII., Josefstadt@\textbf{VIII., Josefstadt}, \emph{A.ADM3}|pwk}8/ Wien, 1\textcolor{gray}{×}. VI. {[}1913{]}, 7\nobreak{}«.  }
\buchAbdrucke{\weitereDrucke{Stefan Zweig: \emph{Briefwechsel mit Hermann Bahr, Sigmund Freud, Rainer Maria
                        Rilke und Arthur Schnitzler}. Frankfurt am Main: \emph{S. Fischer} 1987, S. 374.} }\toendnotes[C]{\smallbreak}\pstart{}{\pb}D\textsuperscript{r} Artur
                  Schnitzler\pend{}\pstart{}\textcolor{pink}{Wien – Cottage}\oindex{Waehringer Cottage@\textbf{Währinger Cottage}, \emph{Teil eines besiedelten Ortes (A.BSOX)}|pw}{}\ledrightnote{\textcolor{pink}{Währinger Cottage}}\pend{}\pstart{}\textcolor{pink}{\label{K_L03638-1v}\edtext{Sternwartestrasse 72}{\lemma{\textnormal{\emph{Sternwartestrasse 72}}}\Cendnote{\textnormal{\textcolor{blue}{Zweig}\pwindex{Zweig, Stefan 28.11.1881 – 23.02.1942@\textsc{Zweig, Stefan} (28.11.1881 – 23.02.1942), \emph{Schriftsteller}|pwk} wechselt bei der Adressierung
                        seiner Schreiben an \textcolor{blue}{Schnitzler} immer
                        wieder zwischen der falschen Hausnummer »72« und der
                        richtigen »71«.}}}\label{K_L03638-1}}\oindex{Sternwartestrasse 71@\textbf{Sternwartestraße 71}, \emph{Wohngebäude (K.WHS)}|pw}{}\ledrightnote{\textcolor{pink}{Sternwartestraße 71}}\pend{}{\bigskip}
\pstart
           \noindent{}\centering{}{\pb}\textcolor{gray}{\textbf{\textcolor{pink}{Wien – Maximilianplatz}\oindex{Rooseveltplatz@\textbf{Rooseveltplatz}, \emph{Platz (K.PLT)}|pw}{}\ledrightnote{\textcolor{pink}{Rooseveltplatz}} u. \textcolor{pink}{Votivkirche}\oindex{Votivkirche@\textbf{Votivkirche}, \emph{Kirche (K.KRC)}|pw}{}\ledrightnote{\textcolor{pink}{Votivkirche}}}}\pend
           \vspace{1em}
\pstart
           \noindent{}{\pb}Verehrter Herr Doktor, ich höre eben von \label{K_L03638-2v}\edtext{\textcolor{blue}{Heinis}\pwindex{Schnitzler, Heinrich 09.08.1902 – 12.07.1982@\textsc{Schnitzler, Heinrich} (09.08.1902 – 12.07.1982), \emph{Regisseur, Schauspieler}|pw}{}\ledrightnote{\textcolor{blue}{Heinrich Schnitzler}} Erkrankung}{\lemma{\textnormal{\emph{Heinis Erkrankung}}}\Cendnote{\textnormal{
                  Am 9. 6. 1913 waren \textcolor{blue}{Olga}\pwindex{Schnitzler, Olga 17.01.1882 – 13.01.1970@\textsc{Schnitzler, Olga} (17.01.1882 – 13.01.1970), \emph{Schauspielerin, Sängerin}|pwk} und
                  \textcolor{blue}{Arthur Schnitzler} zu einer mehrwöchigen Reise in die \textcolor{pink}{Schweiz}\oindex{Schweiz@\textbf{Schweiz}, \emph{A.PCLI}|pwk} aufgebrochen.
                  Bereits am Folgetag erhielten sie ein Telegramm, demnach ihr Sohn \textcolor{blue}{Heinrich}\pwindex{Schnitzler, Heinrich 09.08.1902 – 12.07.1982@\textsc{Schnitzler, Heinrich} (09.08.1902 – 12.07.1982), \emph{Regisseur, Schauspieler}|pwk} an Scharlach erkrankt
                  war. Die \textcolor{blue}{Eltern}\pwindex{Schnitzler, Olga 17.01.1882 – 13.01.1970@\textsc{Schnitzler, Olga} (17.01.1882 – 13.01.1970), \emph{Schauspielerin, Sängerin}|pwkv} verließen das eben
                  erreichte \textcolor{pink}{Chur}\oindex{Chur@\textbf{Chur}, \emph{P.PPLA}|pwk} sofort wieder und kehrten
                  bereits am 11. 6. 1913 nach \textcolor{pink}{Wien}\oindex{Wien@\textbf{Wien}, \emph{A.ADM2}|pwk}
                  zurück.}}}\label{K_L03638-2} und Ihrer \label{K_L03638-3v}\edtext{jähen Rückkehr}{\lemma{\textnormal{\emph{jähen Rückkehr}}}\Cendnote{\textnormal{Die Karte ist nicht
                  datiert. Auf dem Poststempel läßt sich entziffern, dass sie im Juni gesendet
                  wurde und die Tagesziffer mit der Ziffer 1 beginnt und zweistellig ist. 
                  Von der zweiten Ziffer ist nur eine obere Rundung zu sehen, wie sie bei den Zahlen
                  ›2‹, ›3‹, ›8‹ und ›9‹ vorkommt. Die Karte muss also am
                     12. oder 13. 6. 1913 versandt sein, als die Aufregung über
                  die durchkreuzten Reisepläne von \textcolor{blue}{Olga}\pwindex{Schnitzler, Olga 17.01.1882 – 13.01.1970@\textsc{Schnitzler, Olga} (17.01.1882 – 13.01.1970), \emph{Schauspielerin, Sängerin}|pwk} und
                     \textcolor{blue}{Arthur Schnitzler} noch frisch
               war.}}}\label{K_L03638-3}. Hoffentlich geht alles gut und rasch vorbei, meine innigsten Wünsche
               sind mit Ihnen in all diesen erregten und hoffentlich bald beruhigten Stunden.\pend
           
\pstart
           Ihr aufrichtig getreuer{\\[\baselineskip]}\spacefill\mbox{Stefan Zweig}\pend
           \leftskip=0em{}\selectlanguage{ngerman}\endnumbering\briefempfaengerindex{Schnitzler, Arthur@\textsc{Schnitzler, Arthur}!zzzZweig, Stefan@\emph{von Stefan Zweig}!1913-06-121@{{[}12. oder
                  13.?{]} 6. {[}1913?{]}}|)be}\mylabel{L03638h}  \normalsize

\doendnotes{C}
\bigskip
\vfill

\clearpage

\footnotesize

\lohead{\textsc{register}}

% Definiere theindex-Environment komplett neu ohne reledmac
\makeatletter
\renewenvironment{theindex}{%
  \section*{\indexname}%
  \setlength{\parindent}{0pt}%
  \setlength{\parskip}{0pt plus 0.3pt}%
  \let\item\@idxitem
}{%
  \clearpage
}
\makeatother

\IfFileExists{\jobname-pw.ind}{\input{\jobname-pw.ind}}{}

\end{document}

      