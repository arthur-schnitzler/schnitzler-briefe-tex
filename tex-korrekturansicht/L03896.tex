%% latex-korrekturansicht-vorspann.tex
%% Vorspann für die Korrekturansicht.
%% Lädt die gemeinsame Datei latex-vorspann.tex mit gesetztem Schalter.

\newif\ifkorrekturansicht
\korrekturansichttrue

\input{../tex-inputs/latex-vorspann}


\section[Theodor Herzl an Arthur Schnitzler, 23. 6. 1895]{L03896 Theodor Herzl an Arthur Schnitzler, 23. 6. 1895}
\nopagebreak\mylabel{L03896v}
\rehead{ }\normalsize\beginnumbering\briefempfaengerindex{, @\textsc{, }!zzz, @\emph{von  }!1895-06-231@{23. 6. 1895}|(be}
\toendnotes[C]{\smallbreak\pagebreak[2]}\Standort{Wien, Österreichische Gesellschaft für Literatur, Abschrift Herzl.}
\physDesc{Brief, maschinenschriftliche Abschrift, 1 Blatt, 1 Seite, 2329 Zeichen
\newline{}maschinell}
\buchAbdrucke{\weitereDrucke{1) \emph{Die Geburt des Judenstaates.} In: \emph{Jüdische Nachrichten für die österreichischen Alpenländer}, Nr. 20, 3. 7. 1920, S. 4–5.} \weitereDrucke{2) \emph{Herzl-Briefe}. Herausgegeben und eingeleitet Manfred Georg. Berlin: \emph{Brandusche Verlagsbuchhandlung} [1935], S. 53–54.} }\toendnotes[C]{\smallbreak}
\pstart
           \raggedleft{}{\pb}23. VI. 95.\pend
           
\pstart{}Mein lieber Freund,\pend\vspace{0.5em}
\pstart
           Dank für Ihren Brief. Die Sache liegt in \textcolor{pink}{Prag}\oindex{Prag@\textbf{Prag}, \emph{Land}|pw}{}\ledrightnote{\textcolor{pink}{Prag}}, eine
               Entscheidung ist noch nicht da. Das Ganze ist jetzt in den Hintergrund meines
               Bewusstseins getreten.\pend
           
\pstart
           Aber Sie hatten damals Recht, als Sie mit Ihrem klugen Blick sahen, dass ich mit
               dieser einen Erruption mir die Sache nicht vom Herzen und nicht von der Seele geladen
               habe.\pend
           
\pstart
           In den Wochen, seit ich Ihnen nicht geschrieben, ist etwas Anderes, Neues, viel
               Grösseres in mir aufgeschossen, was mir jetzt wie ein Basaltberg vorkommt, vielleicht
               weil ich noch so erschüttert bin und das Entstandene noch so fürchterlich glüht. {\pb}Wochen der ungeheuerlichsten Poduktionsaufregung, in
               der ich manchmal fürchtete, verrückt zu werden.\pend
           
\pstart
           Es sind vorläufig nur die Planskizzen – sie sind schon ein ganzes \textcolor{green}{Buch}\pwindex{Herzl, Theodor 2.\,5.\,1860 Budapest – 3.\,7.\,1904 Edlach@\textsc{Herzl, Theodor} (2.\,5.\,1860 Budapest – 3.\,7.\,1904 Edlach), \emph{Schriftsteller, Journalist}!Judenstaat. Versuch einer modernen Lösung der Judenfrage@\strich\emph{Der Judenstaat. Versuch einer modernen Lösung der Judenfrage}|pwv}{}\ledrightnote{{$\rightarrow$}\emph{\textcolor{green}{Der Judenstaat. Versuch einer modernen Lösung der Judenfrage}}}.\pend
           
\pstart
           Wir werden, wenn wir im Sommer im \textcolor{pink}{Salzkammergut}\oindex{Salzkammergut@\textbf{Salzkammergut}, \emph{Region}|pw}{}\ledrightnote{\textcolor{pink}{Salzkammergut}}
               Zusammentreffen, darüber reden.\pend
           
\pstart
           Dieses \textcolor{green}{Werk}\pwindex{Herzl, Theodor 2.\,5.\,1860 Budapest – 3.\,7.\,1904 Edlach@\textsc{Herzl, Theodor} (2.\,5.\,1860 Budapest – 3.\,7.\,1904 Edlach), \emph{Schriftsteller, Journalist}!Judenstaat. Versuch einer modernen Lösung der Judenfrage@\strich\emph{Der Judenstaat. Versuch einer modernen Lösung der Judenfrage}|pwv}{}\ledrightnote{{$\rightarrow$}\emph{\textcolor{green}{Der Judenstaat. Versuch einer modernen Lösung der Judenfrage}}} ist jedenfalls für mich und
               mein ferneres Leben von der grössten Bedeutung – vielleicht auch für andere Menschen.
               Denn was mich annehmen lässt, dass ich etwas Wertvolles entworfen habe, ist die
               Tatsache, dass ich dabei keine Sekunde lang literatenhaft an mich gedacht habe,
               sondern immer an andere Menschen, welche schwer leiden.\pend
           
\pstart
           Noch ein paar Tage Arbeit, und die Sache ist so fertig, dass sie nicht mehr verloren
               gehen kann, auch wenn ich durch Umstände des Lebens an der munitiösen Ausführung
               verhindert werden sollte.\pend
           
\pstart
           Dann verlasse ich \textcolor{pink}{Paris}\oindex{Paris@\textbf{Paris}, \emph{Hauptstadt}|pw}{}\ledrightnote{\textcolor{pink}{Paris}} auf einige Tage, um mich
               zu erholen. Mein Urlaub ist das noch nicht; den trete ich erst Mitte oder Ende Juli
               an.\pend
           
\pstart
           Sie kennen das liebe Gedicht von \textcolor{blue}{Heyse}\pwindex{Heyse, Paul 15.\,3.\,1830 Berlin – 2.\,4.\,1914 München@\textsc{Heyse, Paul} (15.\,3.\,1830 Berlin – 2.\,4.\,1914 München), \emph{Schriftsteller}|pw}{}\ledrightnote{\textcolor{blue}{Paul Heyse}} »\label{K_L03896-1v}\edtext{\textcolor{green}{an den
                  Künstler}\pwindex{Heyse, Paul 15.\,3.\,1830 Berlin – 2.\,4.\,1914 München@\textsc{Heyse, Paul} (15.\,3.\,1830 Berlin – 2.\,4.\,1914 München), \emph{Schriftsteller}!Weihe der Kunst@\strich\emph{Weihe der Kunst}|pw}{}\ledrightnote{\textcolor{green}{Weihe der Kunst}}}{\lemma{\textnormal{\emph{an den
                  Künstler}}}\Cendnote{\textnormal{»Und bangſt, du möchteſt über Nacht{ / }Hinfahren, eh dies Werk vollbracht:«. \textcolor{blue}{Paul Heyse}\pwindex{Heyse, Paul 15.\,3.\,1830 Berlin – 2.\,4.\,1914 München@\textsc{Heyse, Paul} (15.\,3.\,1830 Berlin – 2.\,4.\,1914 München), \emph{Schriftsteller}|pwk}: \emph{\textcolor{green}{Weihe der
                           Kunst}\pwindex{Heyse, Paul 15.\,3.\,1830 Berlin – 2.\,4.\,1914 München@\textsc{Heyse, Paul} (15.\,3.\,1830 Berlin – 2.\,4.\,1914 München), \emph{Schriftsteller}!Weihe der Kunst@\strich\emph{Weihe der Kunst}|pwk}}. In: \emph{\textcolor{green}{Der Kunstwart}\pwindex{Kunstwart@\emph{Der Kunstwart}|pwk}}, Jg. 1, H. 1, 5. 10. 1887, S. 10.}}}\label{K_L03896-1}«, das ich oft citiere. Da heisst es\pend
           \stanza{}{\dots}{ }\textcolor{green}{Bangend, er könnte über Nacht}\pwindex{Heyse, Paul 15.\,3.\,1830 Berlin – 2.\,4.\,1914 München@\textsc{Heyse, Paul} (15.\,3.\,1830 Berlin – 2.\,4.\,1914 München), \emph{Schriftsteller}!Weihe der Kunst@\strich\emph{Weihe der Kunst}|pwv}{}\ledrightnote{{$\rightarrow$}\emph{\textcolor{green}{Weihe der Kunst}}}\newverse{}\textcolor{green}{Hinfahren ehe dies Werk
                  vollbracht.}\pwindex{Heyse, Paul 15.\,3.\,1830 Berlin – 2.\,4.\,1914 München@\textsc{Heyse, Paul} (15.\,3.\,1830 Berlin – 2.\,4.\,1914 München), \emph{Schriftsteller}!Weihe der Kunst@\strich\emph{Weihe der Kunst}|pwv}{}\ledrightnote{{$\rightarrow$}\emph{\textcolor{green}{Weihe der Kunst}}}\stanzaend{}
\pstart
           Das ist meine Stimmung.\pend
           
\pstart
           Ich habe den Stoss bisheriger Notizen im \textcolor{brown}{\begin{otherlanguage}{french}Comptoir d’Escompte\end{otherlanguage}}\orgindex{Comptoir d’Escompte@Comptoir d’Escompte|pw}{}\ledrightnote{\textcolor{brown}{Comptoir d’Escompte}} deponiert, in der Kasse Nr. 6, Fach
               Nr. 2. Um zu öffnen muss man jeden der drei Knöpfe siebenmal nach rechts
               drücken. Jemand muss das wissen, falls ich »\textcolor{green}{hinfahre über Nacht.}\pwindex{Heyse, Paul 15.\,3.\,1830 Berlin – 2.\,4.\,1914 München@\textsc{Heyse, Paul} (15.\,3.\,1830 Berlin – 2.\,4.\,1914 München), \emph{Schriftsteller}!Weihe der Kunst@\strich\emph{Weihe der Kunst}|pwv}{}\ledrightnote{{$\rightarrow$}\emph{\textcolor{green}{Weihe der Kunst}}}«\pend
           
\pstart
           Das sind jetzt Sie.\pend
           
\pstart
           Komme ich Ihnen aufgeregt vor? Ich bin es nicht. Ich war nie in einer so glücklichen
               hohen Stimmung. Ich denke nicht ans Sterben, sondern an ein Leben voll männlicher
               Taten, das alles Niedere, Wüste, Verworrene, das je in mir gewesen sein mag,
               auslöscht, aufhebt und alle mit mir versöhnt, so wie ich mich durch diese Arbeit mit
               allen versöhnt habe.\pend
           
\pstart
           Ich grüsse Sie herzlich{\\[\baselineskip]} Ihr Freund{\\[\baselineskip]}\spacefill\mbox{Herzl}\pend
           \leftskip=0em{}\selectlanguage{ngerman}\endnumbering\briefempfaengerindex{, @\textsc{, }!zzz, @\emph{von  }!1895-06-231@{23. 6. 1895}|)be}\mylabel{L03896h}
\begin{anhang}
\end{anhang}\normalsize

\doendnotes{C}
\bigskip
\vfill

\clearpage

\footnotesize

\lohead{\textsc{register}}

% Definiere theindex-Environment komplett neu ohne reledmac
\makeatletter
\renewenvironment{theindex}{%
  \section*{\indexname}%
  \setlength{\parindent}{0pt}%
  \setlength{\parskip}{0pt plus 0.3pt}%
  \let\item\@idxitem
}{%
  \clearpage
}
\makeatother

\IfFileExists{\jobname-pw.ind}{\input{\jobname-pw.ind}}{}

\end{document}

      