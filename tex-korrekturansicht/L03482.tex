%% latex-korrekturansicht-vorspann.tex
%% Vorspann für die Korrekturansicht.
%% Lädt die gemeinsame Datei latex-vorspann.tex mit gesetztem Schalter.

\newif\ifkorrekturansicht
\korrekturansichttrue

\input{../tex-inputs/latex-vorspann}


\renewcommand{\erwaehntePersonen}{Personen: Eva Marie Goldmann, Franziska Goldmann, Heinrich Schnitzler}
\renewcommand{\erwaehnteOrte}{Orte: Bad Ischl, Le Prese, Lindaustraße, Wien}
\renewcommand{\erwaehnteWerke}{}
\section[ Paul Goldmann an Arthur Schnitzler, 16. 8. 1930]{Paul Goldmann an Arthur Schnitzler, 16. 8. 1930}
\nopagebreak\mylabel{v}
\rehead{ }\normalsize\beginnumbering\briefempfaengerindex{Schnitzler, Arthur@\textsc{Schnitzler, Arthur}!zzzGoldmann, Paul@\emph{von Paul Goldmann}!1930-08-161@{16. 8. 1930}|(be}
\toendnotes[C]{\smallbreak\pagebreak[2]}\Standort{DLA, A:Schnitzler, HS.NZ85.1.3176.}
\physDesc{Brief, 1 Blatt, 3 Seiten, 1187 Zeichen
\newline{}Handschrift: schwarze Tinte, deutsche Kurrent
\newline{}Schnitzler: mit Bleistift die Datumszeile unterstrichen und drei weitere
                                 Unterstreichung }\toendnotes[C]{\smallbreak}
\pstart
           {\pb}\textcolor{pink}{Bad-Iſchl}{}\ledrightnote{\textcolor{pink}{Bad Ischl}}, \textcolor{pink}{Lindauſtraße 19}{}\ledrightnote{\textcolor{pink}{Lindaustraße}}\pend
           
\pstart
           \raggedleft{}16. 8. 30.
               \pend
           
\pstart{}Mein lieber Freund,\pend
\pstart
           Ich danke Dir für Deine Karte aus \label{K_L03482-1v}\edtext{\textsc{\textcolor{pink}{Le Prese}{}\ledrightnote{\textcolor{pink}{Le Prese}}}}{\lemma{\textnormal{\emph{Le Prese}}}\Cendnote{\textnormal{siehe A. S.: \emph{Tagebuch}, 6. 8. 1930}}}\label{K_L03482-1h}, u. ich habe mich ſehr gefreut, daß Du meiner gedacht haſt.\pend
           
\pstart
           Jugend – es geht mir gerade fortwährend im Kopfe herum. In wenigen Jahren, \strikeout{bin} wenn ich es erlebe, was nicht ſehr ſicher iſt, bin
               ich \label{K_L03482-2v}\edtext{ſiebzig}{\lemma{\textnormal{\emph{ſiebzig}}}\Cendnote{\textnormal{\textcolor{blue}{Goldmann} wurde am 31. 1. 1935 siebzig Jahre alt. Am 25. 9. 1935 starb er.}}}\label{K_L03482-2h}. Ich kann es gar nicht verſtehen. Denn das
               Ich, die eigentliche, die innere Perſönlichkeit, iſt dieſelbe geblieben, wie ſtets,
               iſt nicht gealtert, {\pb}iſt nicht über die Mitte
               der Sechzig hinaus u. wird nicht ſiebzig ſein. Der weißhaarige alte Herr, den mir die
               Spiegelſcheiben der Schaufenſter zeigen, dem die Mädchen auf der Trambahn ihren Platz
               anbieten, – das ſoll \uline{ich} ſein? Aber es iſt doch nicht
               möglich! Das Eigentliche iſt doch noch nicht gekommen, das, was getan werden ſollte,
               iſt noch nicht getan! Das Leben, das ich nicht gelebt habe, das ich ſo gern leben
               möchte, ſoll vorüber ſein? Ich kanns nicht begreifen{\dotsfive}\pend
           
\pstart
           {\pb}Nur \uline{ein} Gutes
               iſt: wenn das \strikeout{\textcolor{gray}{gramſte}} Nichtmehrwiſſen kommt, wird man auch nichts mehr von all’ dem Verfehlten u.
               Verſäumten wiſſen, wird man auch nicht mehr zu bereuen brauchen{\dotsfive}\pend
           
\pstart
           Herzliche Grüße an Dich (auch von \textcolor{blue}{Frau}{}\ledrightnote{{$\rightarrow$}\textcolor{blue}{Eva Marie Goldmann}} u. \textcolor{blue}{Tochter}{}\ledrightnote{{$\rightarrow$}\textcolor{blue}{Franziska Goldmann}})! Und Empfehlungen an Deinen \textcolor{blue}{Sohn}{}\ledrightnote{{$\rightarrow$}\textcolor{blue}{Heinrich Schnitzler}}! {\\[\baselineskip]}Dein {\\[\baselineskip]}\spacefill\mbox{Paul Goldmann.}\pend
           \leftskip=0em{}\endnumbering\briefempfaengerindex{Schnitzler, Arthur@\textsc{Schnitzler, Arthur}!zzzGoldmann, Paul@\emph{von Paul Goldmann}!1930-08-161@{16. 8. 1930}|)be}\mylabel{h}
\begin{anhang}
\end{anhang}\normalsize

\doendnotes{C}
\bigskip
\vfill

\clearpage

\footnotesize

\lohead{\textsc{register}}

% Definiere theindex-Environment komplett neu ohne reledmac
\makeatletter
\renewenvironment{theindex}{%
  \section*{\indexname}%
  \setlength{\parindent}{0pt}%
  \setlength{\parskip}{0pt plus 0.3pt}%
  \let\item\@idxitem
}{%
  \clearpage
}
\makeatother

\IfFileExists{\jobname-pw.ind}{\input{\jobname-pw.ind}}{}

\end{document}

      