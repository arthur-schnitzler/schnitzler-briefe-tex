%% latex-korrekturansicht-vorspann.tex
%% Vorspann für die Korrekturansicht.
%% Lädt die gemeinsame Datei latex-vorspann.tex mit gesetztem Schalter.

\newif\ifkorrekturansicht
\korrekturansichttrue

\input{../tex-inputs/latex-vorspann}


               \section[Arthur Schnitzler an Gerty von Hofmannsthal, 12. 8. 1929]{ Arthur Schnitzler an Gerty von Hofmannsthal, 12. 8. 1929}\nopagebreak\mylabel{v}\rehead{ }\normalsize\beginnumbering\briefempfaengerindex{Hofmannsthal, Gertrude von@\textsc{Hofmannsthal, Gertrude von}!zzzSchnitzler, Arthur@\emph{von Arthur Schnitzler}!1929-08-121@{12. 8. 1929}|(be} \toendnotes[C]{\smallbreak\pagebreak[2]} \Standort{FDH, Hs-31346,3.}
\physDesc{Brief, 1 Blatt (Briefpapier mit Trauerrand), 1 Seite
\newline{}Handschrift: schwarze Tinte, lateinische Kurrent}\pstart
           \raggedleft{}{\pb}\textcolor{pink}{Wien}{}\ledrightnote{\textcolor{pink}{Wien}}, 12/8 929\pend
           \pstart
           liebe Gerty, also Fräulein \textcolor{blue}{Pollak}{}\ledrightnote{\textcolor{blue}{Frieda Pollak}} besorgt Ihnen die Briefe, wie ich eben an \textcolor{blue}{Christiane}{}\ledrightnote{\textcolor{blue}{Christiane von Hofmannsthal}} schrieb. We{\geminationn} Sie eine
               Beihilfe zum Abschreiben der Briefe benötigen – es ist nicht viel, – bringe ich \textcolor{blue}{\uline{Magda Pollaczek}}{}\ledrightnote{\textcolor{blue}{Magda Pollaczek}} in Vorschlag;– sie hat in der letzten Zeit manches, auch schwer leserliches für
               mich abgeschrieben und macht das ausgezeichnet.\pend
           \pstart
           Ich bin noch immer hier, morgen ko{\geminationm}t \textcolor{blue}{Heini}{}\ledrightnote{\textcolor{blue}{Heinrich Schnitzler}}, etwa in 8 Tagen dürfte ich abreisen, – vermutlich in die
                  \textcolor{pink}{französ. Schweiz}{}\ledrightnote{\textcolor{pink}{Schweiz}}. \pend
           \pstart
           Alles herzliche, auf Wiedersehen,{\\[\baselineskip]}Ihr{\\[\baselineskip]}\spacefill\mbox{Arthur}\pend
           \leftskip=0em{}\endnumbering\briefempfaengerindex{Hofmannsthal, Gertrude von@\textsc{Hofmannsthal, Gertrude von}!zzzSchnitzler, Arthur@\emph{von Arthur Schnitzler}!1929-08-121@{12. 8. 1929}|)be}\mylabel{h}  \normalsize

\doendnotes{C}
\bigskip
\vfill

\clearpage

\footnotesize

\lohead{\textsc{register}}

% Definiere theindex-Environment komplett neu ohne reledmac
\makeatletter
\renewenvironment{theindex}{%
  \section*{\indexname}%
  \setlength{\parindent}{0pt}%
  \setlength{\parskip}{0pt plus 0.3pt}%
  \let\item\@idxitem
}{%
  \clearpage
}
\makeatother

\IfFileExists{\jobname-pw.ind}{\input{\jobname-pw.ind}}{}

\end{document}

      