%% latex-korrekturansicht-vorspann.tex
%% Vorspann für die Korrekturansicht.
%% Lädt die gemeinsame Datei latex-vorspann.tex mit gesetztem Schalter.

\newif\ifkorrekturansicht
\korrekturansichttrue

\input{../tex-inputs/latex-vorspann}


               \section[Arthur Schnitzler an Richard Beer-Hofmann, 31. 1. 1896]{ Arthur Schnitzler an Richard Beer-Hofmann,
               31. 1. 1896}\nopagebreak\mylabel{v}\rehead{ }\normalsize\beginnumbering\briefempfaengerindex{Beer-Hofmann, Richard@\textsc{Beer-Hofmann, Richard}!zzzSchnitzler, Arthur@\emph{von Arthur Schnitzler}!1896-01-311@{31. 1. 1896}|(be} \toendnotes[C]{\smallbreak\pagebreak[2]} \Standort{YCGL, MSS 31.}
\physDesc{Brief, 2 Blätter, 7 Seiten, Umschlag
\newline{}Handschrift: 1) Bleistift, deutsche Kurrent\hspace{1em}2) schwarze Tinte, deutsche Kurrent (\noindent{}Umschlag)\hspace{1em}\newline{}Versand: Stempel: »\nobreak{}\oindex{Berlin@\textbf{Berlin}, \emph{https://www.geonames.org/ontologyP.PPLC}|pwk}Berlin W., 31 1 96, 9–10N\nobreak{}«.  }\buchAbdrucke{\weitereDrucke{Arthur Schnitzler, Richard Beer-Hofmann: \emph{Briefwechsel 1891–1931}. Hg. Konstanze Fliedl. Wien, Zürich: \emph{Europaverlag} 1992, S. 89–90.} }\toendnotes[C]{\smallbreak}\pstart{}{\pb}\textsc{Dr. Arthur Schnitzler, \textcolor{pink}{Berlin}{}\ledrightnote{\textcolor{pink}{Berlin}}, \textcolor{pink}{Westminster Hotel}{}\ledrightnote{\textcolor{pink}{Hotel Westminster}}}.\pend{}{\bigskip}\pstart{}{\pb}Herrn \textsc{Dr. Richard
                     Beer-Hofmann}\pend{}\pstart{}\textcolor{pink}{Wien}{}\ledrightnote{\textcolor{pink}{Wien}}\pend{}\pstart{}\textsc{\textcolor{pink}{I. Wollzeile 15}{}\ledrightnote{\textcolor{pink}{Wollzeile}}}.\pend{}{\bigskip}\pstart{}{\pb}Lieber Richard,\pend\pstart
           Erſtens ist \textcolor{pink}{\textsc{Westminster Hotel}}{}\ledrightnote{\textcolor{pink}{Hotel Westminster}} ein Protzenhotel, wie mir von den verſchiedenſten Seiten verſichert wird. Aber
               ich wohne doch dort. –\pend
           \pstart
           Zweitens war ſelbſtverſtändlich der erſte Menſch, dem ich begegnete, »College« \textcolor{blue}{Stümke}{}\ledrightnote{\textcolor{blue}{Heinrich Stümcke}}, der zur Zeit \textcolor{pink}{Berlin}{}\ledrightnote{\textcolor{pink}{Berlin}} vielfach anſpuckt und mehr Unſinn redet, als (über den {\pb}Vergleich denk ich nächſtens nach). Er fragte
               gleich nach der \textsc{\textcolor{blue}{Brion}{}\ledrightnote{\textcolor{blue}{Lou Brion}}}.
               Ein Herr \textcolor{blue}{\textsc{Ehrenzweig}}{}\ledrightnote{\textcolor{blue}{Ehrenzweig}}, den ich vorher ke{\geminationn}en gelernt hatte (folglich war
                  \textcolor{blue}{Stümke}{}\ledrightnote{\textcolor{blue}{Heinrich Stümcke}} nicht der erſte Menſch \introOben{}\textsc{etc}\introOben{}) und ſich an meiner Seite
               befand, kannte die \textsc{\textcolor{blue}{Brion}{}\ledrightnote{\textcolor{blue}{Lou Brion}}}
               natürlich auch. Ich ahnte fürchterliches. Aber wir ſchweiften ab (Ich meine es nicht
               ſo.)\pend
           \pstart
           Geſtern war ich bei der \textcolor{green}{Jüdin von Toledo}{}\ledrightnote{\textcolor{green}{Die Jüdin von Toledo}} und
               verliebte mich in {\pb}die \textcolor{blue}{Sorma}{}\ledrightnote{\textcolor{blue}{Agnes Sorma}}; aber \textcolor{blue}{Kainz}{}\ledrightnote{\textcolor{blue}{Josef Kainz}} war
               ebenſo herrlich.\pend
           \pstart
           Mit \textcolor{blue}{Brahm}{}\ledrightnote{\textcolor{blue}{Otto Brahm}} hab ich mich ſofort gezankt, er hat das
               Kind der \textcolor{green}{\textsc{Katharina Binder}}{}\ledrightnote{→\textcolor{green}{Liebelei. Schauspiel in drei Akten}} gemordet – angeblich aus künſtleriſchen Gründen. Als ich dieſelben wi\strikeout{e}derlegte, ſtellte ſich heraus, daſs er überhaupt kein
               Kind zur Verfügung hatte. Ein paar Striche, die ganz überflüſſiger Weiſe geſchehn
               waren, machte ich wieder auf.\pend
           \pstart
           {\pb}Heute war Probe. Ich unterhielt mich ſehr gut.
               Sie wollen mehr wiſſen? Gelegentlich.\pend
           \pstart
           \textcolor{blue}{Stümke}{}\ledrightnote{\textcolor{blue}{Heinrich Stümcke}} möchte nicht in meiner Haut ſtecken
               (Gegenſeitig!) Nemlich weil die Sti{\geminationm}ung gegen \textcolor{blue}{Brahm}{}\ledrightnote{\textcolor{blue}{Otto Brahm}}{ }ſehr heftig iſt und bei den \textsc{Premièren} »jedenfalls« auf Hausſchlüſſeln gepfiffen wird.
               Ich ka{\geminationn} natürlich kein Auge zuthun. »Gehn S’, ſein S’
               feſch, {\pb}und ko{\geminationm}en S’
               her!« Glauben Sie, daſs Librettiſten auf Nachſchlüſſeln pfeifen? (Herrn \textcolor{blue}{\textsc{Julius Bauer}}{}\ledrightnote{\textcolor{blue}{Julius Bauer}} wohlgeboren)\pend
           \pstart
           – Wohin war mein erſter Gang? Zu dem Hauſe, das \textsc{ICH} vor
               8 Jahren bewohnt hatte. Jedes Poëtchen hat sein Pietätchen.\pend
           \pstart
           Schneit es in \textcolor{pink}{Wien}{}\ledrightnote{\textcolor{pink}{Wien}} noch ſo vehement, und wie geht es
                  \textcolor{blue}{Paula}{}\ledrightnote{\textcolor{blue}{Paula Beer-Hofmann}}? (\introOben{}Ja we{\geminationn}
                  Sie wüßten was ich urſprünglich in diese Kla{\geminationm}er ſchreiben wollte!\introOben{})\pend
           \pstart
           {\pb}\textcolor{blue}{\textsc{Jarno}}{}\ledrightnote{\textcolor{blue}{Josef Jarno}} läßt Sie grüßen; Sie waren ſeine erſte Frage. Die \textcolor{blue}{Staglé}{}\ledrightnote{\textcolor{blue}{Helene Staglé}} ist engagirt, ſpielt im »\textcolor{green}{zerbrochnen Krug}{}\ledrightnote{\textcolor{green}{Der zerbrochene Krug}}« mit, der zur \textcolor{green}{Liebelei}{}\ledrightnote{\textcolor{green}{Liebelei. Schauspiel in drei Akten}}
               dazu gegeben wird.\pend
           \pstart
           – Jetzt kleid ich mich um, gehe zum \textcolor{green}{\textsc{König Chilperich}}{}\ledrightnote{\textcolor{green}{König Chilperich}}. Da{\geminationn} bin ich \textcolor{blue}{eingeladen}{}\ledrightnote{→\textcolor{blue}{Augusta Burchardt}}. \label{K_L00531_1v}\edtext{\textsc{Si vous croyez, que c’est rigolo}!}{\lemma{\textnormal{\emph{Si … rigolo!}}}\Cendnote{\textnormal{französisch: Glauben Sie ja nicht, dass das unterhaltsam
                  ist!}}}\label{K_L00531_1h} – Womöglich als Zitat entnommen aus: \textcolor{blue}{Gyp}: \emph{\textcolor{green}{Le Mariage de Chiffon}}.
                  Paris: \emph{\textcolor{brown}{Calmann-Lévy}}{ }1894, S. 47.\pend
           \pstart
           Grüßen Sie \textcolor{blue}{Salten}{}\ledrightnote{\textcolor{blue}{Felix Salten}}, \textcolor{blue}{Hugo}{}\ledrightnote{\textcolor{blue}{Hugo von Hofmannsthal}} un\textcolor{gray}{d} manche andre. Schreiben {\pb}Sie
               mir.\pend
           \pstart
           Herzlich der Ihre{\\[\baselineskip]}\spacefill\mbox{Arth}\pend
           \leftskip=0em{}\endnumbering\briefempfaengerindex{Beer-Hofmann, Richard@\textsc{Beer-Hofmann, Richard}!zzzSchnitzler, Arthur@\emph{von Arthur Schnitzler}!1896-01-311@{31. 1. 1896}|)be}\mylabel{h}  \normalsize

\doendnotes{C}
\bigskip
\vfill

\clearpage

\footnotesize

\lohead{\textsc{register}}

% Definiere theindex-Environment komplett neu ohne reledmac
\makeatletter
\renewenvironment{theindex}{%
  \section*{\indexname}%
  \setlength{\parindent}{0pt}%
  \setlength{\parskip}{0pt plus 0.3pt}%
  \let\item\@idxitem
}{%
  \clearpage
}
\makeatother

\IfFileExists{\jobname-pw.ind}{\input{\jobname-pw.ind}}{}

\end{document}

      