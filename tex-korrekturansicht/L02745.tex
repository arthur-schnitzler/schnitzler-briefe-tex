%% latex-korrekturansicht-vorspann.tex
%% Vorspann für die Korrekturansicht.
%% Lädt die gemeinsame Datei latex-vorspann.tex mit gesetztem Schalter.

\newif\ifkorrekturansicht
\korrekturansichttrue

\input{../tex-inputs/latex-vorspann}


               \section[Paul Goldmann an Arthur Schnitzler, 19. 8. {[}1895{]}]{ Paul Goldmann an Arthur Schnitzler, 19. 8. {[}1895{]}}\nopagebreak\mylabel{v}\rehead{ }\normalsize\beginnumbering\briefempfaengerindex{Schnitzler, Arthur@\textsc{Schnitzler, Arthur}!zzzGoldmann, Paul@\emph{von Paul Goldmann}!1895-08-192@{19. 8. {[}1895{]}}|(be} \toendnotes[C]{\smallbreak\pagebreak[2]} \Standort{DLA, A:Schnitzler, HS.NZ85.1.3165.}
\physDesc{Brief, 1 Blatt, 4 Seiten
\newline{}Handschrift: schwarze Tinte, deutsche Kurrent
\newline{}Schnitzler: 1) mit Bleistift das Jahr »95« vermerkt 2) mit rotem Buntstift zwei Unterstreichungen}\toendnotes[C]{\smallbreak}\pstart
           \noindent{}{\pb}\textcolor{gray}{\textbf{\textbf{\textcolor{brown}{Frankfurter Zeitung}{}\ledrightnote{\textcolor{brown}{Frankfurter Zeitung}}}}}\pend
           \pstart
           \textcolor{gray}{\textbf{(\textcolor{brown}{\begin{otherlanguage}{french}Gazette de Francfort\end{otherlanguage}}{}\ledrightnote{\textcolor{brown}{Frankfurter Zeitung}}). }}\pend
           \pstart
           \textcolor{gray}{\textbf{\textbf{\begin{otherlanguage}{french}Fondateur M. \textcolor{blue}{L.
                                 Sonnemann}{}\ledrightnote{\textcolor{blue}{Leopold Sonnemann}}\end{otherlanguage}.}}}\hfill \textsc{\textcolor{pink}{Toelz}{}\ledrightnote{\textcolor{pink}{Bad Tölz}}}, 19. Auguſt.\pend
           \pstart
           \begin{otherlanguage}{french}\textcolor{gray}{\textbf{\textcolor{green}{Journal}{}\ledrightnote{→\textcolor{green}{Frankfurter Zeitung}} politique,
                        financier,}}\end{otherlanguage}\pend
           \pstart
           \begin{otherlanguage}{french}\textcolor{gray}{\textbf{commercial et littéraire.}}\end{otherlanguage}\pend
           \pstart
           \begin{otherlanguage}{french}\textcolor{gray}{\textbf{\textbf{Paraissant trois fois par jour.}}}\end{otherlanguage}\pend
           \pstart
           \begin{otherlanguage}{french}\textcolor{gray}{\textbf{\textbf{Bureau à \textcolor{pink}{Paris}{}\ledrightnote{\textcolor{pink}{Paris}}:}}}\end{otherlanguage}\pend
           \pstart
           \begin{otherlanguage}{french}\textcolor{gray}{\textbf{\textbf{\textcolor{pink}{24. Rue Feydeau}{}\ledrightnote{\textcolor{pink}{rue Feydeau}}.}}}\end{otherlanguage}\pend
           \pstart\center{}Mein lieber Freund,\pend\pstart
           Alſo von Herzen Glück auf den Weg – auf den guten Weg, der Dich zu mir führen ſoll.
               Ich freue mich auf unſer Wiederſehn und ich fürchte mich zugleich davor – aus
               Gründen, die Du gewiß verſtehſt, ohne daß ich ſie ſage{\dotsfive}\pend
           \pstart
           Ich wohne in \textsc{\textcolor{pink}{Krankenheil}{}\ledrightnote{\textcolor{pink}{Bad Krankenheil}}}, \textsc{\textcolor{pink}{Villa Carlo}{}\ledrightnote{\textcolor{pink}{Villa Carlo}}}. Aber Du telegraphirſt mir wohl am Tage vor Deiner Ankunft, {\pb}damit ich nur ja zu Hauſe bin.\pend
           \pstart
           Deine Fahrt wird ſchön ſein. Wenn ich nur wüßte, was man thun könnte, damit Du gutes
               Wetter haſt!\pend
           \pstart
           Wenn Du die Frau \textsc{\textcolor{blue}{Andreas}{}\ledrightnote{\textcolor{blue}{Lou Andreas-Salomé}}} ſiehſt, ſo grüße ſie von mir recht herzlich. Ich möchte ſie gern einmal
               wiederſehen, wüßte ich nur wie und wo?\pend
           \pstart
           \label{K_L02745-44v}\edtext{\textsc{\textcolor{blue}{Mamroth}{}\ledrightnote{\textcolor{blue}{Fedor Mamroth}}} iſt it noch bei der »\textcolor{brown}{Frankfurter
                  Zeitung}{}\ledrightnote{\textcolor{brown}{Frankfurter Zeitung}}}{\lemma{\textnormal{\emph{Mamroth … Zeitung}}}\Cendnote{\textnormal{siehe Hugo von Hofmannsthal an Arthur Schnitzler, 9. 8. [1895]}}}\label{K_L02745-44h}«, auch tritt er ſeinen großen Urlaub
               erſt nächſtens an{[}.{]}{ }{\pb}Hingegen war er in der letzten Zeit mehrmals vom
                  \textcolor{brown}{Büreau}{}\ledrightnote{→\textcolor{brown}{Frankfurter Zeitung}} abweſend, und ich
               müßte den präciſen Zeitpunkt wiſſen, um die Anfrage \strikeout{g\textcolor{gray}{enau}} genau beantworten zu können{\dotssix}\pend
           \pstart
           Ich bin heut ſo traurig u. hoffnungslos. Bin hier ganz
               allein u. habe Muße, über mich nachzudenken. Das iſt ſchrecklich. Ich ſchreibe Dir
               das nur, um Dich darauf vorzubereiten, daß Du mich nicht in jener guten Stimmung
               treffen wirſt, von der Dein lieber Brief ſpricht.\pend
           \pstart
           {\pb}Das ganze Jahr über habe \strikeout{i\textcolor{gray}{c}} ich mich auf das Wiederſehen mit Dir gefreut. Jetzt ſolls kaum mehr eine Woche
               dauern. Merkwürdig, wie die Begebenheiten langſam und geräuſchlos heranrücken! Es
               ſcheint Alles ſtill zu ſtehen, und nun auf einmal iſts nur noch eine Woche!{\dotsfive}\pend
           \pstart
           Grüß’ Dich Gott, mein lieber Freund!\pend
           \pstart
           Dein {\\[\baselineskip]}\spacefill\mbox{Paul Goldmann}\pend
           \leftskip=0em{}\pstart
           \noindent{}Grüße an Herrn \textsc{\textcolor{blue}{Salten}{}\ledrightnote{\textcolor{blue}{Felix Salten}}}!\pend
           \endnumbering\briefempfaengerindex{Schnitzler, Arthur@\textsc{Schnitzler, Arthur}!zzzGoldmann, Paul@\emph{von Paul Goldmann}!1895-08-192@{19. 8. {[}1895{]}}|)be}\mylabel{h}  \normalsize

\doendnotes{C}
\bigskip
\vfill

\clearpage

\footnotesize

\lohead{\textsc{register}}

% Definiere theindex-Environment komplett neu ohne reledmac
\makeatletter
\renewenvironment{theindex}{%
  \section*{\indexname}%
  \setlength{\parindent}{0pt}%
  \setlength{\parskip}{0pt plus 0.3pt}%
  \let\item\@idxitem
}{%
  \clearpage
}
\makeatother

\IfFileExists{\jobname-pw.ind}{\input{\jobname-pw.ind}}{}

\end{document}

      