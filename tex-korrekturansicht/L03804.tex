%% latex-korrekturansicht-vorspann.tex
%% Vorspann für die Korrekturansicht.
%% Lädt die gemeinsame Datei latex-vorspann.tex mit gesetztem Schalter.

\newif\ifkorrekturansicht
\korrekturansichttrue

\input{../tex-inputs/latex-vorspann}


\renewcommand{\erwaehntePersonen}{Personen: Stefan Zweig}
\renewcommand{\erwaehnteOrte}{Orte: Edmund-Weiß-Gasse, I., Innere Stadt, Rathausstraße 17, Wien, XVIII., Währing}
\renewcommand{\erwaehnteWerke}{Werke: Die Liebe der Erika Ewald. Novellen}
\section[Arthur Schnitzler an Stefan Zweig, 17. 10. 1904]{Arthur Schnitzler an Stefan Zweig, 17. 10. 1904}
\nopagebreak\mylabel{v}
\rehead{ }\normalsize\beginnumbering\briefempfaengerindex{Zweig, Stefan@\textsc{Zweig, Stefan}!zzzSchnitzler, Arthur@\emph{von Arthur Schnitzler}!1904-10-171@{17. 10. 1904}|(be}
\toendnotes[C]{\smallbreak\pagebreak[2]}\Standort{Jerusalem, National Library of Israel, ARC. Ms. Var. 305 1 58 Stefan Zweig Collection.}
\physDesc{Kartenbrief, 1 Blatt, 2 Seiten, 260 Zeichen
\newline{}Handschrift: 1) schwarze Tinte, deutsche Kurrent\hspace{1em}2) schwarze Tinte, lateinische Kurrent (\noindent{}Adresse)\hspace{1em}
\newline{}Versand: 1) Stempel: »\nobreak{}\oindex{XVIII., Waehring@\textbf{XVIII., Währing}, \emph{A.ADM3}|pwk}\textcolor{gray}{18/1 Wien}
                                       110, 18. X. 04, XI\nobreak{}«.   2) Stempel: »\nobreak{}18. 10. 0\textcolor{gray}{4}, 3–4 ½, Bestellt\nobreak{}«. }\toendnotes[C]{\smallbreak}\pstart{}{\pb}Herrn Dr Phil. Stefan Zweig\pend{}\pstart{}\textcolor{pink}{Wien I}{}\ledrightnote{\textcolor{pink}{I., Innere Stadt}}\pend{}\pstart{}\textcolor{pink}{Rathhaustraße 17}{}\ledrightnote{\textcolor{pink}{Rathausstraße 17}}.\pend{}
{\bigskip}\vspace{1em}
\pstart
           \raggedleft{}{\pb}\textsc{\textcolor{pink}{XVIII
                        Spoettelgasse 7}{}\ledrightnote{\textcolor{pink}{Edmund-Weiß-Gasse}}}\pend
           
\pstart
           \raggedleft{}\textcolor{pink}{Wien}{}\ledrightnote{\textcolor{pink}{Wien}},
                        17. 10 90\textcolor{gray}{4}\pend
           
\pstart{}verehrteſter Herr Doktor,\pend\vspace{0.5em}
\pstart
           ſchönſten \label{K_L03804-1v}\edtext{Dank}{\lemma{\textnormal{\emph{Dank}}}\Cendnote{\textnormal{26. 12. 1887.
               }}}\label{} für die liebenswürdg Überſendg Ihres \textcolor{green}{Novellenbuches}{}\ledrightnote{{$\rightarrow$}\textcolor{green}{Die Liebe der Erika Ewald. Novellen}}, auf deſſen Lecture ich mich
                  aufricht\textcolor{gray}{g} freue.\pend
           
\pstart
           Ihr Sie hochſchätzender{\\[\baselineskip]}\spacefill\mbox{Arthur Schnitzler.}\pend
           \leftskip=0em{}\endnumbering\briefempfaengerindex{Zweig, Stefan@\textsc{Zweig, Stefan}!zzzSchnitzler, Arthur@\emph{von Arthur Schnitzler}!1904-10-171@{17. 10. 1904}|)be}\mylabel{h}
\begin{anhang}
\end{anhang}\normalsize

\doendnotes{C}
\bigskip
\vfill

\clearpage

\footnotesize

\lohead{\textsc{register}}

% Definiere theindex-Environment komplett neu ohne reledmac
\makeatletter
\renewenvironment{theindex}{%
  \section*{\indexname}%
  \setlength{\parindent}{0pt}%
  \setlength{\parskip}{0pt plus 0.3pt}%
  \let\item\@idxitem
}{%
  \clearpage
}
\makeatother

\IfFileExists{\jobname-pw.ind}{\input{\jobname-pw.ind}}{}

\end{document}

      