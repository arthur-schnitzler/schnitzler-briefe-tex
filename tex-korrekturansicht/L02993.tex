%% latex-korrekturansicht-vorspann.tex
%% Vorspann für die Korrekturansicht.
%% Lädt die gemeinsame Datei latex-vorspann.tex mit gesetztem Schalter.

\newif\ifkorrekturansicht
\korrekturansichttrue

\input{../tex-inputs/latex-vorspann}


\renewcommand{\erwaehntePersonen}{Personen: Carl Hochsinger, Felix Salten, Ottilie Salten}
\renewcommand{\erwaehnteInstitutionen}{Institutionen: Erstes öffentliches Kinderkrankeninstitut}
\renewcommand{\erwaehnteOrte}{Orte: Carl-Theater, IX., Alsergrund, Porzellangasse, VIII., Josefstadt, Wien}
\renewcommand{\erwaehnteWerke}{Werke: Artur Schnitzler-Abend, Die Zeit}
\section[ Arthur Schnitzler an Felix Salten, 13. 12. 1904]{Arthur Schnitzler an Felix Salten, 13. 12. 1904}
\nopagebreak\mylabel{v}
\rehead{ }\normalsize\beginnumbering\briefempfaengerindex{Salten, Felix@\textsc{Salten, Felix}!zzzSchnitzler, Arthur@\emph{von Arthur Schnitzler}!1904-12-131@{13. 12. 1904}|(be}
\toendnotes[C]{\smallbreak\pagebreak[2]}\Standort{Wienbibliothek im Rathaus, ZPH 1681, 2.1.516.}
\physDesc{Kartenbrief, 411 Zeichen
\newline{}Handschrift: schwarze Tinte, deutsche Kurrent
\newline{}Versand: Stempel: »\nobreak{}\oindex{VIII., Josefstadt@\textbf{VIII., Josefstadt}, \emph{A.ADM3}|pwk}18/1 Wien 110 a, 14. X\textcolor{gray}{II}. 04, X\nobreak{}«.  
\newline{}Ordnung: mit Bleistift von unbekannter Hand Nummerierung der Blätter des Konvoluts:
                                    »31« }\toendnotes[C]{\smallbreak}\pstart{}{\pb}Herrn Felix Salten\pend{}\pstart{}\textcolor{pink}{Wien IX}{}\ledrightnote{\textcolor{pink}{IX., Alsergrund}}\pend{}\pstart{}\textsc{\textcolor{pink}{Porzellangasse 45}{}\ledrightnote{\textcolor{pink}{Porzellangasse}}.}\pend{}
{\bigskip}
\pstart
           \raggedleft{}{\pb}13. 12. 904\pend
           
\pstart
           lieber, könnten Sie am Samſtag (we{\geminationn} Ihre \textcolor{blue}{Frau}{}\ledrightnote{{$\rightarrow$}\textcolor{blue}{Ottilie Salten}} ſchon da iſt, natürlich Sie beide) bei uns
               nachtmahlen? Beſtimmen Sie ſelbſt die Stunde.\pend
           
\pstart
           Herzlichſt der Ihrige {\\[\baselineskip]}\spacefill\mbox{Arthur.}\pend
           \leftskip=0em{}
\pstart
           \noindent{}Über Ihren \label{K_L02993-1v}\edtext{\textcolor{green}{Artikel}{}\ledrightnote{{$\rightarrow$}\textcolor{green}{Artur Schnitzler-Abend}}}{\lemma{\textnormal{\emph{Artikel}}}\Cendnote{\textnormal{Am 12. 12. 1904
                     hatte ein »Arthur-Schnitzler-Abend« im \textcolor{pink}{Carl-Theater} stattgefunden. Dieser wurde für das seit 1787 bestehende \emph{\textcolor{brown}{Erste öffentliche Kinderkrankeninstitut}}
                     abgehalten, dessen Leitung \textcolor{blue}{Carl Hochsinger}
                     inne hatte. \textcolor{blue}{Salten} rezensierte ihn in: \textcolor{blue}{Felix Salten}: \emph{\textcolor{green}{Artur Schnitzler-Abend}}. In: \emph{\textcolor{green}{Die Zeit}}, Jg. 3, Nr. 796, Morgenblatt, 13. 12. 1904 , S. 3.}}}\label{K_L02993-1h} hab ich mich
                  wie Sie ſich denken können ſehr gefreut. Im allgemeinen hab ich allerdings diesmal
                  die Empfindung, als we{\geminationn} man mich in Schulden geſtürzt
                  hätte, die ich nicht bezahlen kann.\pend
           \endnumbering\briefempfaengerindex{Salten, Felix@\textsc{Salten, Felix}!zzzSchnitzler, Arthur@\emph{von Arthur Schnitzler}!1904-12-131@{13. 12. 1904}|)be}\mylabel{h}  \normalsize

\doendnotes{C}
\bigskip
\vfill

\clearpage

\footnotesize

\lohead{\textsc{register}}

% Definiere theindex-Environment komplett neu ohne reledmac
\makeatletter
\renewenvironment{theindex}{%
  \section*{\indexname}%
  \setlength{\parindent}{0pt}%
  \setlength{\parskip}{0pt plus 0.3pt}%
  \let\item\@idxitem
}{%
  \clearpage
}
\makeatother

\IfFileExists{\jobname-pw.ind}{\input{\jobname-pw.ind}}{}

\end{document}

      