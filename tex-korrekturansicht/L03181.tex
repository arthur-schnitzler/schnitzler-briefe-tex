%% latex-korrekturansicht-vorspann.tex
%% Vorspann für die Korrekturansicht.
%% Lädt die gemeinsame Datei latex-vorspann.tex mit gesetztem Schalter.

\newif\ifkorrekturansicht
\korrekturansichttrue

\input{../tex-inputs/latex-vorspann}


\renewcommand{\erwaehntePersonen}{Personen: Jenny Bally, Isidor S. Bally, Samuel Fischer, Friedrich Hebbel, Wilhelmine Mitterwurzer, Friedrich Mitterwurzer, Max Nordau, Marie Reinhard}
\renewcommand{\erwaehnteInstitutionen}{Institutionen: Wiener Allgemeine Zeitung, Wiener Verlag}
\renewcommand{\erwaehnteOrte}{Orte: Berlin, Burgtheater, Wien}
\renewcommand{\erwaehnteWerke}{Werke: Begräbnis, Cocotte und Kellner, Der Hinterbliebene, Der Hinterbliebene. Kurze Novellen, Der Hund, Die Hochzeit auf dem Lande, Die kleine Veronika, Don Karlos, Infant von Spanien, Einiges über Schiller’s »Don Carlos«, Flucht, Freiwild. Schauspiel in 3 Akten, Heldentod, Judith. Eine Tragödie in fünf Aufzügen, Neue Freie Presse}
\section[Felix Salten an Arthur Schnitzler, {[}30. 10. 1896{]}]{Felix Salten an Arthur Schnitzler, {[}30. 10. 1896{]}}
\nopagebreak\mylabel{v}
\rehead{ }\normalsize\beginnumbering\briefempfaengerindex{Schnitzler, Arthur@\textsc{Schnitzler, Arthur}!zzzSalten, Felix@\emph{von Felix Salten}!1896-10-303@{{[}30. 10. 1896{]}}|(be}
\toendnotes[C]{\smallbreak\pagebreak[2]}\Standort{CUL, Schnitzler, B 89, A 1.}
\physDesc{Brief, 1 Blatt, 4 Seiten, 2019 Zeichen
\newline{}Handschrift: Bleistift, lateinische Kurrent
\newline{}Schnitzler: mit Bleistift datiert: »Ende Oct 96« 
\newline{}Ordnung: mit Bleistift von unbekannter Hand nummeriert: »80« }\toendnotes[C]{\smallbreak}
\pstart
           \noindent{}{\pb}Lieber Arthur, ob die \label{K_L03181-1v}\edtext{Versti{\geminationm}ung über das \textcolor{green}{Stück}{}\ledrightnote{{$\rightarrow$}\textcolor{green}{Freiwild. Schauspiel in 3 Akten}}}{\lemma{\textnormal{\emph{Verstimmung … Stück}}}\Cendnote{\textnormal{vgl. A. S.: \emph{Tagebuch}, 28. 10. 1896}}}\label{K_L03181-1h} nicht jenes unangenehme Gefühl ist, das man i{\geminationm}er
               hat, wenn man fremde Leute zum ersten Mal eigene Worte aussprechen hört? Ich \label{K_L03181-2v}\edtext{fahre Montag{ }Abend von \textcolor{pink}{hier}{}\ledrightnote{{$\rightarrow$}\textcolor{pink}{Wien}} ab
               und bin also Dienstag{ }Mittag bei Ihnen}{\lemma{\textnormal{\emph{fahre … Ihnen}}}\Cendnote{\textnormal{siehe A. S.: \emph{Tagebuch}, 3. 11. 1896}}}\label{K_L03181-2h}. Wenn es Ihre sonstigen Umstände zulaßen, und Sie es leicht können, möchte
               ich Sie um etwas bitten. Sprechen Sie vielleicht mit dem Verleger \textcolor{blue}{Fischer}{}\ledrightnote{\textcolor{blue}{Samuel Fischer}} von mir. Ich will endlich mein \label{K_L03181-3v}\edtext{\textcolor{green}{Buch}{}\ledrightnote{{$\rightarrow$}\textcolor{green}{Der Hinterbliebene. Kurze Novellen}}}{\lemma{\textnormal{\emph{Buch}}}\Cendnote{\textnormal{Die Novellensammlung \emph{\textcolor{green}{Der Hinterbliebene}} erschien erst 1900 im \emph{\textcolor{brown}{Wiener Verlag}}. Aus der hier
                  projektierten Abmachung wurde also nichts.}}}\label{K_L03181-3h} herausgeben. {\pb}Sie wissen, dass mich
               nicht innerliche Gründe dazu bestimmen, denn in der Stimmung, in der ich jetzt seit
               längerer Zeit lebe, möchte ich am liebsten Alles verbrennen. Aber ganz äußerlich
               brauche ich dieses \textcolor{green}{Buch}{}\ledrightnote{{$\rightarrow$}\textcolor{green}{Der Hinterbliebene. Kurze Novellen}} gerade
               jetzt, aus vielen Gründen, vor mir selbst und vor den anderen. Ich habe meine \textcolor{green}{Novellen}{}\ledrightnote{{$\rightarrow$}\textcolor{green}{Der Hinterbliebene. Kurze Novellen}} fertig. \textcolor{green}{Heldentod}{}\ledrightnote{\textcolor{green}{Heldentod}}, \textcolor{green}{Hinterbliebener}{}\ledrightnote{\textcolor{green}{Der Hinterbliebene}} – \textcolor{green}{Flucht}{}\ledrightnote{\textcolor{green}{Flucht}} – \textcolor{green}{Cocotte u. Kellner}{}\ledrightnote{\textcolor{green}{Cocotte und Kellner}} – \textcolor{green}{Begräbnis}{}\ledrightnote{\textcolor{green}{Begräbnis}} – \textcolor{green}{Der Hund}{}\ledrightnote{\textcolor{green}{Der Hund}} –
                  \textcolor{green}{Die Hochzeit auf dem Lande}{}\ledrightnote{\textcolor{green}{Die Hochzeit auf dem Lande}} – \textcolor{green}{Die Confirmandin}{}\ledrightnote{\textcolor{green}{Die kleine Veronika}}. Wenn wir wieder in \textcolor{pink}{Wien}{}\ledrightnote{\textcolor{pink}{Wien}} sind, werde ich Ihnen, \label{K_L03181-4v}\edtext{was Sie noch nicht kennen}{\lemma{\textnormal{\emph{was … kennen}}}\Cendnote{\textnormal{Nachweislich hatte \textcolor{blue}{Salten}{ }\textcolor{blue}{Schnitzler} bereits \emph{\textcolor{green}{Begräbnis}} (18. 5. 1893), \emph{\textcolor{green}{Der
                     Hinterbliebene}} (19. 4. 1894) und \emph{\textcolor{green}{Heldentod}} (31. 7. 1894)
                  vorgelesen.}}}\label{K_L03181-4h}, vorlesen. Für {\pb}jetzt wäre es mir nur von
               Werth, wenn ich mit \textcolor{blue}{Fischer}{}\ledrightnote{\textcolor{blue}{Samuel Fischer}} principiell ins
               Reine komme, die \textcolor{green}{Manuscripte}{}\ledrightnote{{$\rightarrow$}\textcolor{green}{Der Hinterbliebene. Kurze Novellen}}
               schickte ich ihm dann von \textcolor{pink}{hier}{}\ledrightnote{{$\rightarrow$}\textcolor{pink}{Wien}}
               aus. Ich will nur, wenn ich einmal \textcolor{pink}{dort}{}\ledrightnote{{$\rightarrow$}\textcolor{pink}{Berlin}} bin, die Sache persönlich betreiben können.\pend
           
\pstart
           Wenn Sie glauben, dass ich recht habe, und wenn Sie soweit Sie sich meiner \textcolor{green}{Novellen}{}\ledrightnote{{$\rightarrow$}\textcolor{green}{Der Hinterbliebene. Kurze Novellen}} entsinnen, denken,
               dass ich es wagen kann, dann, bitte, sprechen Sie mit \textcolor{blue}{Fischer}{}\ledrightnote{\textcolor{blue}{Samuel Fischer}}, – natürlich nur, wenn es Ihnen sonst nicht unbequem
               ist, mit ihn zu reden. In der \textcolor{brown}{Allg. Zeitg}{}\ledrightnote{\textcolor{brown}{Wiener Allgemeine Zeitung}}
               scheinen sich {\pb}Veränderungen
               vorzubereiten, nach welchen es fraglich wird, ob ich meine \label{K_L03181-5v}\edtext{Stellung behalte}{\lemma{\textnormal{\emph{Stellung behalte}}}\Cendnote{\textnormal{\textcolor{blue}{Salten} blieb bis Ende Juni 1902 in der Redaktion der \emph{\textcolor{brown}{Wiener Allgemeinen Zeitung}}.}}}\label{K_L03181-5h}. Doch davon mündlich. Haben Sie heute{ }\textcolor{blue}{Max Nordau}{}\ledrightnote{\textcolor{blue}{Max Nordau}}{ }\label{K_L03181-6v}\edtext{\textcolor{green}{über den \textcolor{green}{Don Carlos}{}\ledrightnote{\textcolor{green}{Don Karlos, Infant von Spanien}}}{}\ledrightnote{{$\rightarrow$}\textcolor{green}{Einiges über Schiller’s »Don Carlos«}}}{\lemma{\textnormal{\emph{über den Don Carlos}}}\Cendnote{\textnormal{\textcolor{blue}{Max Nordau}: \emph{\textcolor{green}{Einiges über Schiller’s »Don Carlos«}}. In: \emph{\textcolor{green}{Neue Freie Presse}}, Nr. 11.561, 30. 10. 1896, Morgenblatt, S. 1–3. Durch
                  diesen Verweis ist der Brief datierbar.}}}\label{K_L03181-6h} gelesen? Er kommt sich riesig
               bahnbrechend vor. Frl. \label{K_L03181-7v}\edtext{\textcolor{blue}{M. II.}{}\ledrightnote{\textcolor{blue}{Marie Reinhard}}}{\lemma{\textnormal{\emph{M. II.}}}\Cendnote{\textnormal{\textcolor{blue}{Marie Reinhard}}}}\label{K_L03181-7h} saß neulich im \textcolor{pink}{Burgtheater}{}\ledrightnote{\textcolor{pink}{Burgtheater}} einige Reihen
               von mir, Mittelgang Ecke – fein! elejant! Und \textcolor{blue}{Jenny
                  Singer}{}\ledrightnote{\textcolor{blue}{Jenny Bally}} hat sich wieder einmal \label{K_L03181-8v}\edtext{\textcolor{blue}{verlobt}{}\ledrightnote{{$\rightarrow$}\textcolor{blue}{Isidor S. Bally}}}{\lemma{\textnormal{\emph{verlobt}}}\Cendnote{\textnormal{\textcolor{blue}{Jenny Singer} hatte sich mit \textcolor{blue}{Isidor S. Bally} verlobt. Sie heirateten am
                     27. 12. 1896.}}}\label{K_L03181-8h}\textcolor{gray}{.}\pend
           
\pstart
           \label{T_L03181-1v}\edtext{Geheim:}{\lemma{\textnormal{\emph{Geheim:}}}\Cendnote{\textnormal{ohne Doppelpunkt, dafür mit Markierung durch eine Klammer
                  seitlich am linken Rand des folgenden Absatzes}}}\label{T_L03181-1h}\pend
           
\pstart
           \textcolor{green}{Judith}{}\ledrightnote{\textcolor{green}{Judith. Eine Tragödie in fünf Aufzügen}} soll \label{K_L03181-9v}\edtext{nicht aufgeführt}{\lemma{\textnormal{\emph{nicht aufgeführt}}}\Cendnote{\textnormal{\textcolor{blue}{Friedrich Hebbel}s Fünfakter \emph{\textcolor{green}{Judith}} wurde trotzdem aufgeführt. \textcolor{blue}{Schnitzler} besuchte die Vorstellung am 13. 11. 1896.}}}\label{K_L03181-9h}
               werden, weil Frau \textcolor{blue}{Mittwz.}{}\ledrightnote{\textcolor{blue}{Wilhelmine Mitterwurzer}} fürchtet, der Erfolg
               wird nicht gross genug sein, und Herr \textcolor{blue}{Mitterwurzer}{}\ledrightnote{\textcolor{blue}{Friedrich Mitterwurzer}} trägt einen Revolver mit sich, mit dem er sich erschiessen
               will, weil er in seine \textcolor{blue}{Frau}{}\ledrightnote{{$\rightarrow$}\textcolor{blue}{Wilhelmine Mitterwurzer}}
               verliebt und auf den \label{K_L03181-10v}\edtext{Cadetten}{\lemma{\textnormal{\emph{Cadetten}}}\Cendnote{\textnormal{nicht ermittelt}}}\label{K_L03181-10h} eifersüchtig
               ist.\pend
           \pstart herzlichst \spacefill\mbox{Salten}\pend{}\endnumbering\briefempfaengerindex{Schnitzler, Arthur@\textsc{Schnitzler, Arthur}!zzzSalten, Felix@\emph{von Felix Salten}!1896-10-303@{{[}30. 10. 1896{]}}|)be}\mylabel{h}  \normalsize

\doendnotes{C}
\bigskip
\vfill

\clearpage

\footnotesize

\lohead{\textsc{register}}

% Definiere theindex-Environment komplett neu ohne reledmac
\makeatletter
\renewenvironment{theindex}{%
  \section*{\indexname}%
  \setlength{\parindent}{0pt}%
  \setlength{\parskip}{0pt plus 0.3pt}%
  \let\item\@idxitem
}{%
  \clearpage
}
\makeatother

\IfFileExists{\jobname-pw.ind}{\input{\jobname-pw.ind}}{}

\end{document}

      