%% latex-korrekturansicht-vorspann.tex
%% Vorspann für die Korrekturansicht.
%% Lädt die gemeinsame Datei latex-vorspann.tex mit gesetztem Schalter.

\newif\ifkorrekturansicht
\korrekturansichttrue

\input{../tex-inputs/latex-vorspann}


               \section[Paul Goldmann an Arthur Schnitzler, 4. 5. {[}1893{]}]{ Paul Goldmann an Arthur Schnitzler, 4. 5. {[}1893{]}}\nopagebreak\mylabel{v}\rehead{ }\normalsize\beginnumbering\briefempfaengerindex{Schnitzler, Arthur@\textsc{Schnitzler, Arthur}!zzzGoldmann, Paul@\emph{von Paul Goldmann}!1893-05-042@{4. 5. {[}1893{]}}|(be} \toendnotes[C]{\smallbreak\pagebreak[2]} \Standort{DLA, A:Schnitzler, HS.NZ85.1.3163.}
\physDesc{Brief, 1 Blatt, 1 Seite
\newline{}Handschrift: schwarze Tinte, deutsche Kurrent
\newline{}Schnitzler: mit Bleistift das Jahr »93« vermerkt }\toendnotes[C]{\smallbreak}\pstart
           \noindent{}{\pb}\textcolor{gray}{\textbf{\textbf{\textcolor{brown}{Frankfurter Zeitung}{}\ledrightnote{\textcolor{brown}{Frankfurter Zeitung}}.}}}\pend
           \pstart
           \textcolor{gray}{\textbf{\textbf{(\textcolor{brown}{\begin{otherlanguage}{french}Gazette de Francfort\end{otherlanguage}}{}\ledrightnote{\textcolor{brown}{Frankfurter Zeitung}}.)}}}\pend
           \pstart
           \textcolor{gray}{\textbf{\begin{otherlanguage}{french}\textcolor{blue}{Directeur}{}\ledrightnote{→\textcolor{blue}{Leopold Sonnemann}}\end{otherlanguage}{ }\textbf{M. \textcolor{blue}{L. Sonnemann}{}\ledrightnote{\textcolor{blue}{Leopold Sonnemann}}.}}}\hfill \textsc{\textcolor{pink}{Paris}{}\ledrightnote{\textcolor{pink}{Paris}}}, 4. Mai.\pend
           \pstart
           \begin{otherlanguage}{french}\textcolor{gray}{\textbf{\textcolor{green}{Journal}{}\ledrightnote{\textcolor{green}{Frankfurter Zeitung}} politique, financier,}}\end{otherlanguage}\pend
           \pstart
           \begin{otherlanguage}{french}\textcolor{gray}{\textbf{commercial et litteraire.}}\end{otherlanguage}\pend
           \pstart
           \begin{otherlanguage}{french}\textcolor{gray}{\textbf{\textbf{Paraissant trois fois par jour}}}\end{otherlanguage}\pend
           \pstart
           \begin{otherlanguage}{french}\textcolor{gray}{\textbf{\textbf{Bureaux à \textcolor{pink}{Paris}{}\ledrightnote{\textcolor{pink}{Paris}}:}}}\end{otherlanguage}\pend
           \pstart
           \begin{otherlanguage}{french}\textcolor{gray}{\textbf{\textbf{\textcolor{pink}{rue Richelieu 75}{}\ledrightnote{\textcolor{pink}{rue Richelieu}}.}}}\end{otherlanguage}\pend
           \pstart
           Mein theurer Freund!\pend
           \pstart
           Soeben erfahre ich von dem traurigen \label{K_L02706-v}\edtext{Ereigniß}{\lemma{\textnormal{\emph{Ereigniß}}}\Cendnote{\textnormal{Am 2. 5. 1893 verstarb \textcolor{blue}{Schnitzler}s \textcolor{blue}{Vater}{ }\textcolor{blue}{Johann Schnitzler} an einer
                  Blutvergiftung.}}}\label{K_L02706-h}. Ich drücke Dir mit tiefgefühlter Theilnahme die Hand, Dir
               und allen Deinen Angehörigen. Ich bin mit meinen Gedanken bei Dir in dieſen
               ſchmerzlichen Tagen. Wenn Du ein wenig zur Ruhe gekommen ſein, wirſt, bitte ich Dich
               dringend um ein Wort der Beruhigung über Deine Zukunft.\pend
           \pstart
           Ich begrüße Dich von Herzen {\\[\baselineskip]}Dein {\\[\baselineskip]}\spacefill\mbox{Paul Goldmann.}\pend
           \leftskip=0em{}\endnumbering\briefempfaengerindex{Schnitzler, Arthur@\textsc{Schnitzler, Arthur}!zzzGoldmann, Paul@\emph{von Paul Goldmann}!1893-05-042@{4. 5. {[}1893{]}}|)be}\mylabel{h}\begin{anhang}\end{anhang}\normalsize

\doendnotes{C}
\bigskip
\vfill

\clearpage

\footnotesize

\lohead{\textsc{register}}

% Definiere theindex-Environment komplett neu ohne reledmac
\makeatletter
\renewenvironment{theindex}{%
  \section*{\indexname}%
  \setlength{\parindent}{0pt}%
  \setlength{\parskip}{0pt plus 0.3pt}%
  \let\item\@idxitem
}{%
  \clearpage
}
\makeatother

\IfFileExists{\jobname-pw.ind}{\input{\jobname-pw.ind}}{}

\end{document}

      