%% latex-korrekturansicht-vorspann.tex
%% Vorspann für die Korrekturansicht.
%% Lädt die gemeinsame Datei latex-vorspann.tex mit gesetztem Schalter.

\newif\ifkorrekturansicht
\korrekturansichttrue

\input{../tex-inputs/latex-vorspann}


               \section[ Paul Goldmann an Arthur Schnitzler, 26. 2. {[}1899{]}]{Paul Goldmann an Arthur Schnitzler, 26. 2. {[}1899{]}}\nopagebreak\mylabel{v}\rehead{ }\normalsize\beginnumbering\briefempfaengerindex{Schnitzler, Arthur@\textsc{Schnitzler, Arthur}!zzzGoldmann, Paul@\emph{von Paul Goldmann}!1899-02-261@{26. 2. {[}1899{]}}|(be} \toendnotes[C]{\smallbreak\pagebreak[2]} \Standort{DLA, A:Schnitzler, HS.NZ85.1.3169.}
\physDesc{Brief, 1 Blatt, 1 Seite
\newline{}Handschrift: schwarze Tinte, deutsche Kurrent
\newline{}Schnitzler: mit Bleistift das Jahr »99« vermerkt }\toendnotes[C]{\smallbreak}\pstart
           \raggedleft{}{\pb}\textsc{\textcolor{pink}{Paris}{}\ledrightnote{\textcolor{pink}{Paris}}}, 26. Februar.\pend
           \pstart\center{}Mein lieber Freund,\pend\pstart
           Ich war acht Tage in \textsc{\textcolor{pink}{Paris}{}\ledrightnote{\textcolor{pink}{Paris}}} zur \label{K_L02867-1v}\edtext{Berichterſtattung{ }\strikeout{über} über den Congreß u. das Begräbniß \textsc{\textcolor{blue}{Faure}{}\ledrightnote{\textcolor{blue}{Félix Faure}}s}}{\lemma{\textnormal{\emph{Berichterſtattung … Faures}}}\Cendnote{\textnormal{\textcolor{blue}{Félix Faure}, der bis zu seinem Tod durch
                  einen Schlaganfall am 16. 2. 1899{ }\textcolor{pink}{Frankreich}s Präsident war, wurde am 23. 2. 1899 beerdigt. Am 18. 2. 1899 war der
                  \emph{\textcolor{brown}{Congrès}} zusammengetreten, um \textcolor{blue}{Émile Loubet} zu seinem
                  Nachfolger zu wählen.}}}\label{K_L02867-1h}. Nach \textcolor{pink}{Wien}{}\ledrightnote{\textcolor{pink}{Wien}} komme ich \uuline{nicht}. Wie ſich das Alles ergeben, theile ich Dir von \textcolor{pink}{Frankfurt}{}\ledrightnote{\textcolor{pink}{Frankfurt am Main}} aus ausführlich mit, ſobald ich einen freien
               Augenblick finde. Einſtweilen viele treue Grüße! 
            \pend
           \pstart
           Dein{\\[\baselineskip]}\spacefill\mbox{Paul Goldmann}\pend
           \leftskip=0em{}\endnumbering\briefempfaengerindex{Schnitzler, Arthur@\textsc{Schnitzler, Arthur}!zzzGoldmann, Paul@\emph{von Paul Goldmann}!1899-02-261@{26. 2. {[}1899{]}}|)be}\mylabel{h}  \normalsize

\doendnotes{C}
\bigskip
\vfill

\clearpage

\footnotesize

\lohead{\textsc{register}}

% Definiere theindex-Environment komplett neu ohne reledmac
\makeatletter
\renewenvironment{theindex}{%
  \section*{\indexname}%
  \setlength{\parindent}{0pt}%
  \setlength{\parskip}{0pt plus 0.3pt}%
  \let\item\@idxitem
}{%
  \clearpage
}
\makeatother

\IfFileExists{\jobname-pw.ind}{\input{\jobname-pw.ind}}{}

\end{document}

       ew