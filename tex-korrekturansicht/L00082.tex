%% latex-korrekturansicht-vorspann.tex
%% Vorspann für die Korrekturansicht.
%% Lädt die gemeinsame Datei latex-vorspann.tex mit gesetztem Schalter.

\newif\ifkorrekturansicht
\korrekturansichttrue

\input{../tex-inputs/latex-vorspann}


               \section[Arthur Schnitzler an Hugo von Hofmannsthal, 16. 3. 1892]{ Arthur Schnitzler an Hugo von Hofmannsthal, 16. 3. 1892}\nopagebreak\mylabel{v}\rehead{ }\normalsize\beginnumbering\briefempfaengerindex{Hofmannsthal, Hugo von@\textsc{Hofmannsthal, Hugo von}!zzzSchnitzler, Arthur@\emph{von Arthur Schnitzler}!1892-03-161@{16. 3. 1892}|(be} \toendnotes[C]{\smallbreak\pagebreak[2]} \Standort{FDH, Hs-30885,18.}
\physDesc{Brief, 1 Blatt, 2 Seiten
\newline{}Handschrift: schwarze Tinte, deutsche Kurrent\newline{}Ordnung: von Schnitzler mutmaßlich bei der Durchsicht der Briefe
                                    1929 mit Bleistift datiert: »16/\substVorne{}\textsuperscript{5}\substDazwischen{}3\substHinten{} 92«; eventuell die Korrektur der Monatsangabe
                                 von anderer Hand }\buchAbdrucke{\weitereDrucke{Hugo von Hofmannsthal, Arthur Schnitzler: \emph{Briefwechsel}. Hg. Therese Nickl und Heinrich Schnitzler. Frankfurt am Main: \emph{S. Fischer} 1964, S. 16–17.} }\toendnotes[C]{\smallbreak}\pstart{}{\pb}Lieber Freund,\pend\pstart
           die beiliegende Karte kam an mich. Geſtern ſtellte man von derſelben Seite die \strikeout{Bedin} Frage an mich, unter welchen Bedingungen ich ev.
               mein \textcolor{green}{Stück}{}\ledrightnote{→\textcolor{green}{Das Märchen. Schauspiel in drei Aufzügen}} zum Abdruck
               überlaſſen würde. –\pend
           \pstart
           \textcolor{blue}{Bèraton}{}\ledrightnote{\textcolor{blue}{Ferry Bératon}}{ }ſprach dieſer Tage mit mir über die materielle
               Seite des \label{K_L00082_1v}\edtext{\textcolor{blue}{\textsc{Maeterlinck}}{}\ledrightnote{\textcolor{blue}{Maurice Maeterlinck}}-Abends}{\lemma{\textnormal{\emph{Maeterlinck-Abends}}}\Cendnote{\textnormal{am 2. 5. 1892}}}\label{K_L00082_1h}. Vorläufig habe ich ihm
               zehn Gulden geſchickt. Ueber dieſen Abend wäre manches {\pb}zu
               ſprechen.\pend
           \pstart
           Möchten Sie mir die Adreſſe von \textcolor{blue}{\textsc{Schwarzkopf}}{}\ledrightnote{\textcolor{blue}{Gustav Schwarzkopf}} mittheilen? Ich möchte ihn um eine Empfehlung an \textcolor{brown}{\textsc{Bonz}}{}\ledrightnote{\textcolor{brown}{Adolf Bonz {\kaufmannsund} Comp.}} wegen meines \textcolor{green}{\textsc{Anatol-Cyclus}}{}\ledrightnote{\textcolor{green}{Anatol}} erſuchen. Was glauben Sie? –\pend
           \pstart
           Herzlichſt der Ihre{\\[\baselineskip]}\spacefill\mbox{Arth Sch}\pend
           \leftskip=0em{}\pstart
           16. März 92\hfill \textcolor{pink}{Wien}{}\ledrightnote{\textcolor{pink}{Wien}}.\pend
           \endnumbering\briefempfaengerindex{Hofmannsthal, Hugo von@\textsc{Hofmannsthal, Hugo von}!zzzSchnitzler, Arthur@\emph{von Arthur Schnitzler}!1892-03-161@{16. 3. 1892}|)be}\mylabel{h}  \normalsize

\doendnotes{C}
\bigskip
\vfill

\clearpage

\footnotesize

\lohead{\textsc{register}}

% Definiere theindex-Environment komplett neu ohne reledmac
\makeatletter
\renewenvironment{theindex}{%
  \section*{\indexname}%
  \setlength{\parindent}{0pt}%
  \setlength{\parskip}{0pt plus 0.3pt}%
  \let\item\@idxitem
}{%
  \clearpage
}
\makeatother

\IfFileExists{\jobname-pw.ind}{\input{\jobname-pw.ind}}{}

\end{document}

      