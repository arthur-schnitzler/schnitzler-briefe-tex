%% latex-korrekturansicht-vorspann.tex
%% Vorspann für die Korrekturansicht.
%% Lädt die gemeinsame Datei latex-vorspann.tex mit gesetztem Schalter.

\newif\ifkorrekturansicht
\korrekturansichttrue

\input{../tex-inputs/latex-vorspann}


\renewcommand{\erwaehntePersonen}{Personen: Richard Beer-Hofmann, Paul Goldmann, Alfred Kerr}
\renewcommand{\erwaehnteOrte}{Orte: Bad Aussee, Berghaus Moserboden, Berlin, Bludenz, Bormio, Hotel und Pension Rudolfshöhe (Leopold Petter), Innsbruck, Iseosee, Italien, Küblis, Pontresina, Salzburg, Schruns, Schweiz, Solda, Sulzfluh, Trafoi, Zell am See, Österreichischer Hof}
\renewcommand{\erwaehnteWerke}{}
\section[ Arthur Schnitzler an Paul Goldmann, 30. 7. 1900]{Arthur Schnitzler an Paul Goldmann, 30. 7. 1900}
\nopagebreak\mylabel{v}
\rehead{ }\normalsize\beginnumbering\briefempfaengerindex{Goldmann, Paul@\textsc{Goldmann, Paul}!zzzSchnitzler, Arthur@\emph{von Arthur Schnitzler}!1900-07-301@{30. 7. 1900}|(be}
\toendnotes[C]{\smallbreak\pagebreak[2]}\Standort{Berlin, Akademie der Künste, Alfred Kerr-Archiv, 2487.}
\physDesc{Brief, 1 Blatt, 4 Seiten, 927 Zeichen
\newline{}Handschrift: Bleistift, deutsche Kurrent
\newline{}Ordnung: mit Bleistift nummeriert: »69« }\toendnotes[C]{\smallbreak}
\pstart
           \raggedleft{}{\pb}\textcolor{pink}{Aussee}{}\ledrightnote{\textcolor{pink}{Bad Aussee}}{ }30/7 900.\pend
           
\pstart
           Mein lieber Paul, wir (\textcolor{blue}{Richard}{}\ledrightnote{\textcolor{blue}{Richard Beer-Hofmann}} u. ich) haben heute
                  folgend\textcolor{gray}{e} Punkte feſtgestellt:\pend
           
\pstart
           So{\geminationn}tag 12. Rendezvous \uline{\textcolor{pink}{Salzburg}{}\ledrightnote{\textcolor{pink}{Salzburg}}} (\textcolor{pink}{Oeſterr. Hof}{}\ledrightnote{\textcolor{pink}{Österreichischer Hof}}.)\pend
           
\pstart
           13.{ }\textcolor{pink}{Salzburg}{}\ledrightnote{\textcolor{pink}{Salzburg}}.\pend
           
\pstart
           14. Nach \textcolor{pink}{Zell am
                  See}{}\ledrightnote{\textcolor{pink}{Zell am See}}, \textcolor{pink}{Moſerboden}{}\ledrightnote{\textcolor{pink}{Berghaus Moserboden}}\pend
           
\pstart
           15. – \textcolor{pink}{Moſerboden}{}\ledrightnote{\textcolor{pink}{Berghaus Moserboden}}
               – \textcolor{pink}{Zell am See}{}\ledrightnote{\textcolor{pink}{Zell am See}} – \textcolor{pink}{\uuline{Innsbruck}}{}\ledrightnote{\textcolor{pink}{Innsbruck}}\pend
           
\pstart
           16.{ }\textcolor{pink}{Innsbruck}{}\ledrightnote{\textcolor{pink}{Innsbruck}} Bahn \textcolor{pink}{Bludenz}{}\ledrightnote{\textcolor{pink}{Bludenz}} Poſt \textcolor{pink}{Schruns}{}\ledrightnote{\textcolor{pink}{Schruns}}.\pend
           
\pstart
           {\pb}17., event. 18.{ }\textcolor{pink}{Schruns}{}\ledrightnote{\textcolor{pink}{Schruns}}.–\pend
           
\pstart
           19.{ }\textcolor{pink}{Schruns}{}\ledrightnote{\textcolor{pink}{Schruns}} – \textcolor{pink}{Sulzfluh}{}\ledrightnote{\textcolor{pink}{Sulzfluh}}\pend
           
\pstart
           20.\hspace*{4.5em}– \textcolor{pink}{Küblis}{}\ledrightnote{\textcolor{pink}{Küblis}}\pend
           
\pstart
           21.{ }22. auf einen noch nicht beſti{\geminationm}ten Weg nach \textcolor{pink}{Pontreſina}{}\ledrightnote{\textcolor{pink}{Pontresina}}.\pend
           
\pstart
           Von dort nach Übereinko{\geminationm}en \introOben{}1)\introOben{}
               entweder \textcolor{pink}{Bormio}{}\ledrightnote{\textcolor{pink}{Bormio}}, \introOben{}a)\introOben{}
               von \textcolor{pink}{Bormio}{}\ledrightnote{\textcolor{pink}{Bormio}} gegen \textcolor{pink}{Trafoi}{}\ledrightnote{\textcolor{pink}{Trafoi}}, \textcolor{pink}{Sulden}{}\ledrightnote{\textcolor{pink}{Solda}}{ }\textsc{etc} b) von \textcolor{pink}{Bormio}{}\ledrightnote{\textcolor{pink}{Bormio}}
               nach dem \textsc{\textcolor{pink}{Lago d’Iseo}{}\ledrightnote{\textcolor{pink}{Iseosee}}} oder 2) von \textcolor{pink}{Pontreſina}{}\ledrightnote{\textcolor{pink}{Pontresina}} weſtlich in die
                  \textcolor{pink}{Schweiz}{}\ledrightnote{\textcolor{pink}{Schweiz}}{ }{\pb}oder 3) von \textcolor{pink}{Pontreſina}{}\ledrightnote{\textcolor{pink}{Pontresina}} nach \textcolor{pink}{Italien}{}\ledrightnote{\textcolor{pink}{Italien}}. –\pend
           
\pstart
           Am ſchönſten wäre alſo, \label{K_L03176-1v}\edtext{we{\geminationn} Du am 12. od. 13. in \textcolor{pink}{Salzburg}{}\ledrightnote{\textcolor{pink}{Salzburg}}
                  einträfſt}{\lemma{\textnormal{\emph{wenn … einträfſt}}}\Cendnote{\textnormal{siehe Paul Goldmann an Arthur Schnitzler, 16. 6. [1900]}}}\label{K_L03176-1h}; we{\geminationn} Dir das nicht möglich am 15. in \textcolor{pink}{Innsbruck}{}\ledrightnote{\textcolor{pink}{Innsbruck}}.–\pend
           
\pstart
           Vielleicht ſendeſt Du dieſen Brief gleich an \textsc{\textcolor{blue}{Kerr}{}\ledrightnote{\textcolor{blue}{Alfred Kerr}}}, für den dasſelbe gilt. Ich weiſs nicht, wohin
               ich zu adreſſiren hätte.\pend
           
\pstart
           {\pb}Ich fahre morgen nach
                  \textcolor{pink}{\uline{Iſchl, Rudolfshöhe}}{}\ledrightnote{\textcolor{pink}{Hotel und Pension Rudolfshöhe (Leopold Petter)}}, wo mich \label{K_L03176-2v}\edtext{bis auf weiteres}{\lemma{\textnormal{\emph{bis auf weiteres}}}\Cendnote{\textnormal{\textcolor{blue}{Schnitzler} reiste am 10. 8. 1900 nach \textcolor{pink}{Salzburg} ab.}}}\label{K_L03176-2h} Nachrichten treffen.\pend
           
\pstart
           Bitte um recht baldige Verſtändg \textsc{resp} Einverſtändnis.\pend
           
\pstart
           Herzlichſtenſt {\\[\baselineskip]}Dein {\\[\baselineskip]}\spacefill\mbox{Arthur}\pend
           \leftskip=0em{}\endnumbering\briefempfaengerindex{Goldmann, Paul@\textsc{Goldmann, Paul}!zzzSchnitzler, Arthur@\emph{von Arthur Schnitzler}!1900-07-301@{30. 7. 1900}|)be}\mylabel{h}  \normalsize

\doendnotes{C}
\bigskip
\vfill

\clearpage

\footnotesize

\lohead{\textsc{register}}

% Definiere theindex-Environment komplett neu ohne reledmac
\makeatletter
\renewenvironment{theindex}{%
  \section*{\indexname}%
  \setlength{\parindent}{0pt}%
  \setlength{\parskip}{0pt plus 0.3pt}%
  \let\item\@idxitem
}{%
  \clearpage
}
\makeatother

\IfFileExists{\jobname-pw.ind}{\input{\jobname-pw.ind}}{}

\end{document}

      