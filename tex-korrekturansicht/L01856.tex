%% latex-korrekturansicht-vorspann.tex
%% Vorspann für die Korrekturansicht.
%% Lädt die gemeinsame Datei latex-vorspann.tex mit gesetztem Schalter.

\newif\ifkorrekturansicht
\korrekturansichttrue

\input{../tex-inputs/latex-vorspann}


               \section[Max Burckhard an Arthur Schnitzler, 12. 7. 1909]{ Max Burckhard an Arthur Schnitzler, 12. 7. 1909}\nopagebreak\mylabel{v}\rehead{ }\normalsize\beginnumbering\briefempfaengerindex{Schnitzler, Arthur@\textsc{Schnitzler, Arthur}!zzzBurckhard, Max Eugen@\emph{von Max Eugen Burckhard}!1909-07-122@{12. 7. 1909}|(be} \toendnotes[C]{\smallbreak\pagebreak[2]} \Standort{CUL, Schnitzler, B 20.}
\physDesc{Bildpostkarte
\newline{}Handschrift: schwarze Tinte, deutsche Kurrent\newline{}Versand: 1) Stempel: »\nobreak{}\oindex{Lueg am Wolfgangsee@\textbf{Lueg am Wolfgangsee}, \emph{Bezirk (A.BZK)}|pwk}Lueg (St. Gilgen)\nobreak{}«.  2) Stempel: »\nobreak{}\oindex{Salzburg@\textbf{Salzburg}, \emph{Besiedelter Ort (A.BSO)}|pwk}Salzburg, 12. 7. 09\nobreak{}«. 
\newline{}Schnitzler: mit Bleistift beschriftet: »B \textsc{Burckhard}« }\toendnotes[C]{\smallbreak}\pstart{}{\pb}\textsc{H. D\textsuperscript{r} Artur Schnitzler}\pend{}\pstart{}\textsc{\textcolor{pink}{Wien}{}\ledrightnote{\textcolor{pink}{Wien}}}\pend{}\pstart{}\textsc{\textcolor{pink}{XVIIII Spöttelgaße 7}{}\ledrightnote{\textcolor{pink}{Edmund-Weiß-Gasse}}}\pend{}{\bigskip}\pstart
           \noindent{}\centering{}{\pb}{[}Burckhards Haus auf der \textcolor{pink}{Franzosenschanze}{}\ledrightnote{\textcolor{pink}{Franzosenschanze}} in \textcolor{pink}{St. Gilgen}{}\ledrightnote{\textcolor{pink}{St. Gilgen}}{]}\pend
           \pstart{}{\pb}Lieber verehrter Herr
                        Doctor!\pend\pstart
           Leider muß ich ſagen: ſeien Sie froh, daſs Sie fort ſind, denn es gießt hier
                    ununterbrochen\pend
           \pstart
           Ich hoffe, daſs es Ihrem \textcolor{blue}{Kleinen}{}\ledrightnote{→\textcolor{blue}{Heinrich Schnitzler}}{ }ſo gut geht als es eben bei Huſten ſein kann,
                    Ihnen \textcolor{blue}{beiden}{}\ledrightnote{→\textcolor{blue}{Olga Schnitzler}} aber in jeder
                    Hinſicht glänzend.\pend
           \pstart Herzlich\spacefill\mbox{DrBurc\textcolor{gray}{khard}}\pend{}\endnumbering\briefempfaengerindex{Schnitzler, Arthur@\textsc{Schnitzler, Arthur}!zzzBurckhard, Max Eugen@\emph{von Max Eugen Burckhard}!1909-07-122@{12. 7. 1909}|)be}\mylabel{h}  \normalsize

\doendnotes{C}
\bigskip
\vfill

\clearpage

\footnotesize

\lohead{\textsc{register}}

% Definiere theindex-Environment komplett neu ohne reledmac
\makeatletter
\renewenvironment{theindex}{%
  \section*{\indexname}%
  \setlength{\parindent}{0pt}%
  \setlength{\parskip}{0pt plus 0.3pt}%
  \let\item\@idxitem
}{%
  \clearpage
}
\makeatother

\IfFileExists{\jobname-pw.ind}{\input{\jobname-pw.ind}}{}

\end{document}

      