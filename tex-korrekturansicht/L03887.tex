%% latex-korrekturansicht-vorspann.tex
%% Vorspann für die Korrekturansicht.
%% Lädt die gemeinsame Datei latex-vorspann.tex mit gesetztem Schalter.

\newif\ifkorrekturansicht
\korrekturansichttrue

\input{../tex-inputs/latex-vorspann}


\section[Sigmund Freud an Arthur Schnitzler, 14. 5. 1922]{L03887 Sigmund Freud an Arthur Schnitzler, 14. 5. 1922}
\nopagebreak\mylabel{L03887v}
\rehead{ }\normalsize\beginnumbering\briefempfaengerindex{, @\textsc{, }!zzz, @\emph{von  }!1922-05-141@{14. 5. 1922}|(be}
\toendnotes[C]{\smallbreak\pagebreak[2]}\Standort{Washington, DC, Library of Congress, Freud Archives, C41F8.}
\physDesc{Brief, Fotokopie, 2 Blätter, 2 Seiten, 2931 Zeichen
\newline{}Handschrift: schwarze Tinte, deutsche Kurrent
\newline{}Zusatz: Der Verbleib des Originals ist ungeklärt. Zum Zeitpunkt der
                                 ersten Edition 1955 befand es sich im Besitz von \textcolor{blue}{Heinrich Schnitzler}\pwindex{Schnitzler, Heinrich 9.\,8.\,1902 Hinterbrühl – 12.\,7.\,1982 Wien@\textsc{Schnitzler, Heinrich} (9.\,8.\,1902 Hinterbrühl – 12.\,7.\,1982 Wien), \emph{Regisseur, Schauspieler}|pw}. }
\buchAbdrucke{\weitereDrucke{1) Sigmund Freud: \emph{Briefe an Arthur Schnitzler.} Herausgegeben von Henry Schnitzler. In: \emph{Neue deutsche Rundschau}, Jg. 66 (Januar 1955) Nr. 1, S. 96–97.} \weitereDrucke{2) Sigmund Freud: \emph{Briefe 1873–1939}.  Ausgewählt und herausgegeben von Ernst L. Freud. Frankfurt am Main: \emph{S. Fischer} 1960, S. 338–340.} \weitereDrucke{3) Sigmund Freud: \emph{Sigmund Freud Edition. Digitale historisch-kritische
                        Gesamtausgabe}. Herausgegeben von Christine Diercks,  Arkadi Blatow und  Elisabeth Skale. (2014–2025) \url{https://www.freudedition.net/briefe/freud-sigmund/schnitzler-arthur/1922/05/14}.} }\toendnotes[C]{\smallbreak}
\pstart
           \raggedleft{}{\pb}14 Mai 1922\pend
           
\pstart
           \textcolor{gray}{\textbf{PROF. D\textsuperscript{R.} FREUD}}\hfill \textcolor{gray}{\textbf{\textcolor{pink}{WIEN IX., BERGGASSE 19}\oindex{Wien@\textbf{Wien}!IX., Alsergrund@\textbf{IX., Alsergrund}!Berggasse 19@\textbf{Berggasse 19}, \emph{Wohngebäude}|pw}{}\ledrightnote{\textcolor{pink}{Berggasse 19}}. }}\pend
           
\pstart\center{}Verehrter Herr Doktor\pend\vspace{0.5em}
\pstart
           Nun ſind auch Sie beim 60ſten Jahrestag angekommen, während ich, um 6 Jahre älter,
               der Lebensgrenze nah gerückt bin und erwarten darf, bald das Ende vom fünften Akt
               dieſer ziemlich unverſtändlichen und nicht immer amüſanten Komödie zu ſehen.\pend
           
\pstart
           Wenn ich noch einen Reſt von Glauben an die »\label{K_L03887-1v}\edtext{Allmacht der Gedanken}{\lemma{\textnormal{\emph{Allmacht der Gedanken}}}\Cendnote{\textnormal{\textcolor{blue}{Freud}\pwindex{Freud, Sigmund 6.\,5.\,1856 Pribor – 23.\,9.\,1939 London@\textsc{Freud, Sigmund} (6.\,5.\,1856 Pribor – 23.\,9.\,1939 London), \emph{Psychoanalytiker}|pwk} hatte den Begriff ein Jahrzehnt
                  vorher im Aufsatz \emph{\textcolor{green}{Animismus, Magie und Allmacht
                     der Gedanken}\pwindex{Freud, Sigmund 6.\,5.\,1856 Pribor – 23.\,9.\,1939 London@\textsc{Freud, Sigmund} (6.\,5.\,1856 Pribor – 23.\,9.\,1939 London), \emph{Psychoanalytiker}!Animismus, Magie und Allmacht der Gedanken@\strich\emph{Animismus, Magie und Allmacht der Gedanken}|pwk}} (1913) und in \emph{\textcolor{green}{Totem und Tabu}\pwindex{Freud, Sigmund 6.\,5.\,1856 Pribor – 23.\,9.\,1939 London@\textsc{Freud, Sigmund} (6.\,5.\,1856 Pribor – 23.\,9.\,1939 London), \emph{Psychoanalytiker}!Totem und Tabu@\strich\emph{Totem und Tabu}|pwk}} (1913) geprägt. Er bezeichnet damit den
                  Glauben, mit Hilfe von Gedanken Handlungen und Ereignisse der Außenwelt bewirken
                  zu können.}}}\label{K_L03887-1}« bewahrt hätte, würde ich jetzt nicht verſäumen, Ihnen die
               ſtärkſten und herzlichſten Glückwünſche für die zu erwartende Folge von Jahren
               zuzuſchicken. Ich überlaſſe dies thörichte Thun der unüberſehbaren Schaar von
               Zeitgenoſſen, die am 15\textsuperscript{t} Mai Ihrer gedenken
               wird.\pend
           
\pstart
           Ich will Ihnen aber ein Geſtändnis ablegen welches Sie gütigſt aus Rückſicht für mich
               für ſich behalten, mit keinem Freunde oder Fremden theilen wollen. Ich habe mich mit
               der Frage gequält warum ich eigentlich in all dieſen Jahren nie den Verſuch gemacht
               habe Ihren Verkehr aufzuſuchen und ein Geſpräch mit Ihnen zu führen. (Wobei natürlich
               nicht in Betracht gezogen wird, ob Sie ſelbſt eine ſolche Annäherung von mir gerne
               geſehen hätten).\pend
           
\pstart
           Die Antwort auf dieſe Frage enthält das mir zu intim erſcheinende Geständnis. Ich
               meine, ich habe Sie gemieden aus einer Art von \label{K_L03887-2v}\edtext{Doppelgängerſcheu}{\lemma{\textnormal{\emph{Doppelgängerſcheu}}}\Cendnote{\textnormal{Das ist der vermutlich am häufigsten wiederholte Ausdruck,
                  um eine verbindende Verwandtschaft zwischen \textcolor{blue}{Schnitzler} im Literarischen und \textcolor{blue}{Freud}\pwindex{Freud, Sigmund 6.\,5.\,1856 Pribor – 23.\,9.\,1939 London@\textsc{Freud, Sigmund} (6.\,5.\,1856 Pribor – 23.\,9.\,1939 London), \emph{Psychoanalytiker}|pwk} im Psychologischen zu begründen.}}}\label{K_L03887-2}. Nicht etwa, daß ich ſonſt
               ſo leicht geneigt {\pb}wäre, mich mit einem anderen zu
               identifiziren oder daß ich mich über die Differenz der Begabung hinwegſetzen wollte,
               die mich von Ihnen trennt, ſondern ich habe immer wieder, wenn ich mich in Ihre
               ſchönen Schöpfungen vertiefte, hinter deren poetiſchen Schein die nämlichen
               Vorausſetzungen, Intereſſen und Ergebniße zu finden geglaubt, die mir als die eigenen
               bekannt waren. Ihr Determinismus wie Ihre Skepsis – was die Leute Peſſimismus heißen
               –, Ihr Ergriffenſein von den Wahrheiten des Unbewußten, von der Triebnatur des
               Menſchen, Ihre Zerſetzung der kulturell-konventionellen Sicherheiten, das Haften
               Ihrer Gedanken an der Polarität von Lieben und Sterben, das alles berührte mich mit
               einer unheimlichen Vertrautheit. (In einer kleinen Schrift vom J 1920{ }\label{T_L03887-1v}\edtext{(\textcolor{green}{Jenſeits des Luſtprinzips}\pwindex{Freud, Sigmund 6.\,5.\,1856 Pribor – 23.\,9.\,1939 London@\textsc{Freud, Sigmund} (6.\,5.\,1856 Pribor – 23.\,9.\,1939 London), \emph{Psychoanalytiker}!Jenseits des Lustprinzips@\strich\emph{Jenseits des Lustprinzips}|pw}{}\ledrightnote{\textcolor{green}{Jenseits des Lustprinzips}})}{\lemma{\textnormal{\emph{(Jenſeits des Luſtprinzips)}}}\Cendnote{\textnormal{Er
                  verwendet eckige Klammern für die Klammern innerhalb der runden Klammer.}}}\label{T_L03887-1}
               habe ich verſucht, den Eros und den Todestrieb als die Urkräfte aufzuzeigen, deren
               Gegenſpiel alle Rätſel des Lebens beherrſcht.\substVorne{}\textsuperscript{]}\substDazwischen{})\substHinten{} So habe ich den Eindruck gewonnen, daß Sie durch Intuition – eigentlich aber
               in Folge feiner Selbſtwahrnehmung – alles das wiſſen, was ich in mühſeligher Arbeit
               an anderen Menſchen aufgedeckt habe. Ja ich glaube, im Grunde Ihres Wesens ſind Sie
               ein \label{K_L03887-3v}\edtext{pſychologiſcher Tiefenforſcher}{\lemma{\textnormal{\emph{pſychologiſcher Tiefenforſcher}}}\Cendnote{\textnormal{\textcolor{blue}{Heinrich Schnitzler}\pwindex{Schnitzler, Heinrich 9.\,8.\,1902 Hinterbrühl – 12.\,7.\,1982 Wien@\textsc{Schnitzler, Heinrich} (9.\,8.\,1902 Hinterbrühl – 12.\,7.\,1982 Wien), \emph{Regisseur, Schauspieler}|pwk} verfasste in seiner
                  Edition 1955 dazu folgenden Kommentar: »Es mag in diesem Zusammenhang angebracht sein,
                     auf die einige Jahre später veröffentlichte Schrift \textcolor{blue}{Arthur Schnitzlers} ›\textcolor{green}{Der Geist in
                     Wort und der Geist in der Tat; Vorläufige Bemerkungen zu zwei Diagrammen}\pwindex{Schnitzler, Arthur 15. 5. 1862 Wien – 21. 10. 1931 ebd.@\textsc{Schnitzler, Arthur} (15. 5. 1862 Wien – 21. 10. 1931 ebd.), \emph{Schriftsteller, Mediziner}!Geist im Wort und der Geist in der Tat@\strich\emph{Der Geist im Wort und der Geist in der Tat}|pw}‹
                     (\textcolor{pink}{Berlin}\oindex{Berlin@\textbf{Berlin}, \emph{Hauptstadt}|pw}, \textcolor{brown}{S. Fischer Verlag}\orgindex{S. Fischer Verlag@S. Fischer Verlag|pw}, 1927) hinzuweisen. Wie die ›Vorbemerkung‹ ausführt,
                     war dies ein Versuch, ›\textcolor{green}{das Gebiet des menschlichen Geistes, erstens insofern er sich
                     durch das Wort und zweitens durch die Tat kundzugeben vermag, insbesondere
                     die Beziehung zwischen den Urtypen des menschlichen Geistes, schematisch in
                     zwei Diagrammen darzustellen…}\pwindex{Schnitzler, Arthur 15. 5. 1862 Wien – 21. 10. 1931 ebd.@\textsc{Schnitzler, Arthur} (15. 5. 1862 Wien – 21. 10. 1931 ebd.), \emph{Schriftsteller, Mediziner}!Geist im Wort und der Geist in der Tat@\strich\emph{Der Geist im Wort und der Geist in der Tat}|pwv}‹. Den in dieser Schrift aufgestellten Typen zufolge
                     betrachtet sich \textcolor{blue}{Schnitzler} keineswegs als Dichter, sondern als Naturforscher –
                     eine von ihm auch im Gespräch wiederholt vertretene Ansicht. Auf S. 39 der eben
                     erwähnten Schrift heißt es: ›…es gibt auch dichterische Begabungen (besonders
                     solche mit vorwiegend psychologischer Einstellung), die der Geistesverfassung
                     nach dem Typ Naturforscher … angehören…«. In seinem \textcolor{green}{Tagebuch}\pwindex{Schnitzler, Arthur 15. 5. 1862 Wien – 21. 10. 1931 ebd.@\textsc{Schnitzler, Arthur} (15. 5. 1862 Wien – 21. 10. 1931 ebd.), \emph{Schriftsteller, Mediziner}!Tagebuch@\strich\emph{Tagebuch}|pw} erwähnt
                     \textcolor{blue}{Schnitzler} sowohl den Empfang von \textcolor{blue}{Freuds}\pwindex{Freud, Sigmund 6.\,5.\,1856 Pribor – 23.\,9.\,1939 London@\textsc{Freud, Sigmund} (6.\,5.\,1856 Pribor – 23.\,9.\,1939 London), \emph{Psychoanalytiker}|pw} Brief wie auch die Abfassung einer
                     Antwort. Notizen dieser Art finden sich in den \textcolor{green}{Tagebüchern}\pwindex{Schnitzler, Arthur 15. 5. 1862 Wien – 21. 10. 1931 ebd.@\textsc{Schnitzler, Arthur} (15. 5. 1862 Wien – 21. 10. 1931 ebd.), \emph{Schriftsteller, Mediziner}!Tagebuch@\strich\emph{Tagebuch}|pw} sehr selten und nur
                     in Fällen, in denen \textcolor{blue}{Schnitzler} den betreffenden Briefen besondere Bedeutung beimaß.«}}}\label{K_L03887-3}, ſo ehrlich unparteiiſch und unerſchrocken wie nur
               je einer war, und wenn Sie das nicht wären, hätten Ihre künstleriſchen Fähigkeiten,
               Ihre Sprachkunſt und Geſtaltungskraft, freies Spiel gehabt und Sie zu einem Dichter
               weit mehr nach dem Wunſch der Menge gemacht. Mir liegt es nahe, dem Forſcher den
               Vorrang zu geben, aber verzeihen Sie, daß ich in die Analyſe geraten bin, ich kann
               eben nichts anderes. Nur weiß ich, daß die \label{T_L03887-2v}\edtext{Analyſe kein Mittel iſt,}{\lemma{\textnormal{\emph{Analyſe kein Mittel iſt,}}}\Cendnote{\textnormal{Ab hier seitlich entlang des linken Blattrandes in zwei
                  Textblöcken geschrieben.}}}\label{T_L03887-2} ſich beliebt zu machen.\pend
           
\pstart
           In herzlicher Ergebenheit{\\[\baselineskip]} Ihr \spacefill\mbox{Freud}\pend
           \leftskip=0em{}\selectlanguage{ngerman}\endnumbering\briefempfaengerindex{, @\textsc{, }!zzz, @\emph{von  }!1922-05-141@{14. 5. 1922}|)be}\mylabel{L03887h}
\begin{anhang}
\end{anhang}\normalsize

\doendnotes{C}
\bigskip
\vfill

\clearpage

\footnotesize

\lohead{\textsc{register}}

% Definiere theindex-Environment komplett neu ohne reledmac
\makeatletter
\renewenvironment{theindex}{%
  \section*{\indexname}%
  \setlength{\parindent}{0pt}%
  \setlength{\parskip}{0pt plus 0.3pt}%
  \let\item\@idxitem
}{%
  \clearpage
}
\makeatother

\IfFileExists{\jobname-pw.ind}{\input{\jobname-pw.ind}}{}

\end{document}

      