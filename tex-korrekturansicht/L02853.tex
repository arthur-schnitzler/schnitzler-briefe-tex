%% latex-korrekturansicht-vorspann.tex
%% Vorspann für die Korrekturansicht.
%% Lädt die gemeinsame Datei latex-vorspann.tex mit gesetztem Schalter.

\newif\ifkorrekturansicht
\korrekturansichttrue

\input{../tex-inputs/latex-vorspann}


               \section[ Paul Goldmann an Arthur Schnitzler, 21. 7. {[}1898{]}]{Paul Goldmann an Arthur Schnitzler, 21. 7. {[}1898{]}}\nopagebreak\mylabel{v}\rehead{ }\normalsize\beginnumbering\briefempfaengerindex{Schnitzler, Arthur@\textsc{Schnitzler, Arthur}!zzzGoldmann, Paul@\emph{von Paul Goldmann}!1898-07-211@{21. 7. {[}1898{]}}|(be} \toendnotes[C]{\smallbreak\pagebreak[2]} \Standort{DLA, A:Schnitzler, HS.NZ85.1.3168.}
\physDesc{Brief, 1 Blatt, 3 Seiten
\newline{}Handschrift: blaue Tinte, deutsche Kurrent
\newline{}Schnitzler: mit rotem Buntstift eine Unterstreichung }\toendnotes[C]{\smallbreak}\pstart
           \raggedleft{}{\pb}\textsc{\textcolor{pink}{Shanghai}{}\ledrightnote{\textcolor{pink}{Shanghai}}}, 21. Juli.\pend
           \pstart\center{}Mein lieber Freund,\pend\pstart
           Dieſer Tage empfing ich Deine lieben \label{K_L02853-1v}\edtext{Karten aus 
               \textsc{\textcolor{pink}{Steiermark}{}\ledrightnote{\textcolor{pink}{Steiermark}}}}{\lemma{\textnormal{\emph{Karten aus 
               Steiermark}}}\Cendnote{\textnormal{Von 5. 6. 1898 bis
                  10. 6. 1898 machten \textcolor{blue}{Schnitzler}
                  und \textcolor{blue}{Leopold Kramer} eine gemeinsame Radpartie durch die \textcolor{pink}{Steiermark}
                  bis \textcolor{pink}{Kärnten}. Am 7. 6. 1898 sitegen
                  sie für eine Nacht in \textcolor{pink}{Steindorf am Ossiachersee} ab, wo \textcolor{blue}{Richard}
                  und \textcolor{blue}{Paula Beer-Hofmann} für den Sommer wohnten.}}}\label{K_L02853-1h}. Ich ſage Dir, \textsc{\textcolor{blue}{Richard}{}\ledrightnote{\textcolor{blue}{Richard Beer-Hofmann}}} u. ſeiner \textcolor{blue}{Frau}{}\ledrightnote{→\textcolor{blue}{Paula Beer-Hofmann}} vielen
               Dank, daß Ihr an mich gedacht habt. Auch dem Herrn \textsc{\textcolor{blue}{Kramer}{}\ledrightnote{\textcolor{blue}{Leopold Kramer}}} bitte ich, zu \label{K_L02853-2v}\edtext{danken}{\lemma{\textnormal{\emph{danken}}}\Cendnote{\textnormal{Bezug unklar}}}\label{K_L02853-2h}; wenn ich wieder einmal
               ein Familienblatt herausgebe, ſo werde ich alle Gedichte von ihm nehmen.\pend
           \pstart
           Ich leide hier ganz namenlos unter der fürchterlichen Hitze des tropiſchen \textcolor{pink}{chin}{}\ledrightnote{→\textcolor{pink}{China}}eſiſchen Sommers. Seit
               Wochen ſchlafe {\pb}ich keine Nacht mehr als zwei bis
               drei Stunden. Es iſt einfach zum Verrücktwerden; und da es im Norden dieſes
               verfluchten \textcolor{pink}{Land}{}\ledrightnote{→\textcolor{pink}{China}}es genau ſo
               heiß iſt, wie im Süden, gibt es keine Flucht vor der Hitze. Auch habe ich \textcolor{pink}{China}{}\ledrightnote{\textcolor{pink}{China}} ſatt bis oben hinauf. Letzte Woche kam ich
               in einen \textcolor{pink}{Chin}{}\ledrightnote{→\textcolor{pink}{China}}eſen-Aufruhr
               hinein und wäre beinahe todt geſchlagen worden. Den ſchlimmſten Theil der Reiſe habe
               ich leider noch vor mir. \textsc{\textcolor{pink}{Kiau-tschou}{}\ledrightnote{\textcolor{pink}{Kiautschou}}}, wo es noch kein \textcolor{pink}{europ}{}\ledrightnote{→\textcolor{pink}{Europa}}äiſches Haus gibt, und \textsc{\textcolor{pink}{Peking}{}\ledrightnote{\textcolor{pink}{Peking}}}, das gräßlichſte Schmutzneſt der {\pb}Welt, wo man
               die Pocken kriegen kann, wie nichts. Nächſten Montag
               fahre ich nach \textsc{\textcolor{pink}{Kiautschou}{}\ledrightnote{\textcolor{pink}{Kiautschou}}} (Meine Adreſſe bleibt \textsc{\textcolor{pink}{Shanghai}{}\ledrightnote{\textcolor{pink}{Shanghai}}}). Ich ſage Dir: vierzehn Tage in \textcolor{pink}{Florenz}{}\ledrightnote{\textcolor{pink}{Florenz}}
               ſind beſſer, als ſechs Monate in \textcolor{pink}{China}{}\ledrightnote{\textcolor{pink}{China}}. Das
               Heimweh plagt mich unabläſſig, und ich wünſchte, ich wäre ſchon wieder in \textcolor{pink}{Europa}{}\ledrightnote{\textcolor{pink}{Europa}}.\pend
           \pstart
           Hoffentlich höre ich bald wieder von Dir. Grüß’ mir Deine \textcolor{blue}{Freundin}{}\ledrightnote{→\textcolor{blue}{Marie Reinhard}} u. ſei Du ſelbſt von Herzen
               gegrüßt! Dein treuer \spacefill\mbox{Paul Goldmann.}\pend
           \endnumbering\briefempfaengerindex{Schnitzler, Arthur@\textsc{Schnitzler, Arthur}!zzzGoldmann, Paul@\emph{von Paul Goldmann}!1898-07-211@{21. 7. {[}1898{]}}|)be}\mylabel{h}  \normalsize

\doendnotes{C}
\bigskip
\vfill

\clearpage

\footnotesize

\lohead{\textsc{register}}

% Definiere theindex-Environment komplett neu ohne reledmac
\makeatletter
\renewenvironment{theindex}{%
  \section*{\indexname}%
  \setlength{\parindent}{0pt}%
  \setlength{\parskip}{0pt plus 0.3pt}%
  \let\item\@idxitem
}{%
  \clearpage
}
\makeatother

\IfFileExists{\jobname-pw.ind}{\input{\jobname-pw.ind}}{}

\end{document}

      