%% latex-korrekturansicht-vorspann.tex
%% Vorspann für die Korrekturansicht.
%% Lädt die gemeinsame Datei latex-vorspann.tex mit gesetztem Schalter.

\newif\ifkorrekturansicht
\korrekturansichttrue

\input{../tex-inputs/latex-vorspann}


\section[Arthur Schnitzler an Berta Zuckerkandl, 25. 3. 1915]{L03987 Arthur Schnitzler an Berta Zuckerkandl, 25. 3. 1915}
\nopagebreak\mylabel{L03987v}
\rehead{ }\normalsize\beginnumbering\briefempfaengerindex{Zuckerkandl, Berta@\textsc{Zuckerkandl, Berta}!zzzSchnitzler, Arthur@\emph{von Arthur Schnitzler}!1915-03-251@{25. 3. 1915}|(be}
\toendnotes[C]{\smallbreak\pagebreak[2]}
\correspDesc{Versand  durch Arthur Schnitzler am 25. 3. 1915 in Wien
\newline{}Erhalt  durch Berta Zuckerkandl im Zeitraum [25. 3. 1915 – 28. 3. 1915?] in Wien}\toendnotes[C]{\smallbreak}
\Standort{Wien, Österreichische Nationalbibliothek, 405/B78/2 LIT MAG.}
\physDesc{Briefkarte, 876 Zeichen
\newline{}Handschrift: Bleistift, lateinische Kurrent}\toendnotes[C]{\smallbreak}
\pstart
           {\pb}\textcolor{gray}{\textbf{D\textsuperscript{R} ARTHUR SCHNITZLER}}\hfill 25. 3. 1915\pend
           
\pstart
           \textcolor{gray}{\textbf{\textcolor{pink}{WIEN, XVIII.
                        STERNWARTESTRASSE 71}\oindex{Wien@\textbf{Wien}!XVIII., Währing@\textbf{XVIII., Währing}!Sternwartestraße 71@\textbf{Sternwartestraße 71}, \emph{Wohngebäude}|pw}{}\ledrightnote{\textcolor{pink}{Sternwartestraße 71}}.}}\pend
           \vspace{0.5em}
\pstart
           Verehrte gändige Frau,\textcolor{blue}{Olga}\pwindex{Schnitzler, Olga 17.\,1.\,1882 Wien – 13.\,1.\,1970 Lugano@\textsc{Schnitzler, Olga} (17.\,1.\,1882 Wien – 13.\,1.\,1970 Lugano), \emph{Schauspielerin, Sängerin}|pw}{}\ledrightnote{\textcolor{blue}{Olga Schnitzler}} speist bei \textcolor{blue}{Bachrachs}\pwindex{Bachrach, Eugenie 4.\,3.\,1857 Wien – 4.\,12.\,1937 Purkersdorf@\textsc{Bachrach, Eugenie} (4.\,3.\,1857 Wien – 4.\,12.\,1937 Purkersdorf)|pw}\pwindex{Zuckerkandl, Marianne 6.\,8.\,1882 Wien – 1964 Ascona@\textsc{Zuckerkandl, Marianne} (6.\,8.\,1882 Wien – 1964 Ascona), \emph{Übersetzerin}|pw}{}\ledrightnote{\textcolor{blue}{Eugenie Bachrach}{\newline}\textcolor{blue}{Marianne Zuckerkandl}} – ich ko{\geminationm}e eben vom \textcolor{pink}{Anninger}\oindex{Anninger@\textbf{Anninger}, \emph{Berg}|pw}{}\ledrightnote{\textcolor{pink}{Anninger}} – so war ich so kühn Brief zu eröffnen. Vor allem
               wünsch ich gute Besserung – ferner, da wir \textcolor{green}{Krieg u
                  Frieden}\pwindex{Tolstoi, Lew Nikolajewitsch 9.\,9.\,1828 Yasnaya Polyana – 20.\,11.\,1910 Lev Tolstoy@\textsc{Tolstoi, Lew Nikolajewitsch} (9.\,9.\,1828 Yasnaya Polyana – 20.\,11.\,1910 Lev Tolstoy), \emph{Schriftsteller}!Krieg und Frieden@\strich\emph{Krieg und Frieden}|pw}{}\ledrightnote{\textcolor{green}{Krieg und Frieden}} schmählicher Weise nicht besitzten, erlaube ich mir die herrlichen
               \textcolor{blue}{Tolstoi}\pwindex{Tolstoi, Lew Nikolajewitsch 9.\,9.\,1828 Yasnaya Polyana – 20.\,11.\,1910 Lev Tolstoy@\textsc{Tolstoi, Lew Nikolajewitsch} (9.\,9.\,1828 Yasnaya Polyana – 20.\,11.\,1910 Lev Tolstoy), \emph{Schriftsteller}|pw}{}\ledrightnote{\textcolor{blue}{Lew Nikolajewitsch Tolstoi}}ſchen Novellen, und – da wir schon in \textcolor{pink}{Rußland}\oindex{Russland@\textbf{Russland}|pw}{}\ledrightnote{\textcolor{pink}{Russland}} sind (nebbich) – die sehr schönen \textcolor{blue}{Tschechow}\pwindex{Čechov, Anton Pavlovič 17.\,1.\,1860 Taganrog – 15.\,7.\,1904 Badenweiler@\textsc{Čechov, Anton Pavlovič} (17.\,1.\,1860 Taganrog – 15.\,7.\,1904 Badenweiler), \emph{Schriftsteller}|pw}{}\ledrightnote{\textcolor{blue}{Anton Pavlovič Čechov}}ſchen zu übersenden. Auch den wundervollen \textcolor{green}{Ulenspiegel}\pwindex{Coster, Charles de 20.\,8.\,1827 München – 7.\,5.\,1879 Ixelles@\textsc{Coster, Charles de} (20.\,8.\,1827 München – 7.\,5.\,1879 Ixelles), \emph{Schriftsteller}!Tyll Ulenspiegel und Lamm Goedzak@\strich\emph{Tyll Ulenspiegel und Lamm Goedzak}|pw}{}\ledrightnote{\textcolor{green}{Tyll Ulenspiegel und Lamm Goedzak}}{ }{\pb}leg ich bei – de{\geminationn} mir ist als sagten Sie einmal, Sie hätten ihn noch
               nicht gelesen. Alles passt in unsre große Zeit. (Was werden wir nur anfangen, we{\geminationn} sie – noch größer wird? –) Furchtbar was Sie \label{K_L03986-1v}\edtext{von
               \textcolor{pink}{Prz}\oindex{Przemyśl@\textbf{Przemyśl}, \emph{Hauptstadt}|pw}{}\ledrightnote{\textcolor{pink}{Przemyśl}}}{\lemma{\textnormal{\emph{von
               Prz}}}\Cendnote{\textnormal{Am 22. 3. 1915 kapitulierte die in der Garnison \textcolor{pink}{Przemyśl}\oindex{Przemyśl@\textbf{Przemyśl}, \emph{Hauptstadt}|pwk} stationierte
                  \emph{\textcolor{brown}{österreichisch-ungarische Armee}\orgindex{Streitkräfte von Österreich-Ungarn@Streitkräfte von Österreich-Ungarn|pwk}} nach vier Monaten Belagerung. \textcolor{blue}{Hermann Kusmanek}\pwindex{Kusmanek, Hermann 16.\,9.\,1860 Sibiu – 7.\,8.\,1934 Wien@\textsc{Kusmanek, Hermann} (16.\,9.\,1860 Sibiu – 7.\,8.\,1934 Wien), \emph{General}|pwk} übergab die Festung an die \textcolor{pink}{russischen}\oindex{Russland@\textbf{Russland}|pwk} Belagerer und trat eine Kriegsgefangenschaft
                  an, die bis 1918 dauerte.}}}\label{K_L03986-1} schreiben! Ob von dem Schwert, das der tapfre \textcolor{blue}{Kusmanek}\pwindex{Kusmanek, Hermann 16.\,9.\,1860 Sibiu – 7.\,8.\,1934 Wien@\textsc{Kusmanek, Hermann} (16.\,9.\,1860 Sibiu – 7.\,8.\,1934 Wien), \emph{General}|pw}{}\ledrightnote{\textcolor{blue}{Hermann Kusmanek}} behalten dürfte, auch nur Einer wieder lebendig
               wird? – Und so ließe sich noch allerlei sagen – (sprach Frau Censur und strich
               auch das vorige.)\pend
           
\pstart
           Auf baldiges Wiedersehen und nochmals – gute Besserung{\\[\baselineskip]}herzliche
               Grüße{\\[\baselineskip]}Ihr \spacefill\mbox{A. S.}\pend
           \leftskip=0em{}\selectlanguage{ngerman}\endnumbering\briefempfaengerindex{Zuckerkandl, Berta@\textsc{Zuckerkandl, Berta}!zzzSchnitzler, Arthur@\emph{von Arthur Schnitzler}!1915-03-251@{25. 3. 1915}|)be}\mylabel{L03987h}
\begin{anhang}
\end{anhang}\normalsize

\doendnotes{C}
\bigskip
\vfill

\clearpage

\footnotesize

\lohead{\textsc{register}}

% Definiere theindex-Environment komplett neu ohne reledmac
\makeatletter
\renewenvironment{theindex}{%
  \section*{\indexname}%
  \setlength{\parindent}{0pt}%
  \setlength{\parskip}{0pt plus 0.3pt}%
  \let\item\@idxitem
}{%
  \clearpage
}
\makeatother

\IfFileExists{\jobname-pw.ind}{\input{\jobname-pw.ind}}{}

\end{document}

      