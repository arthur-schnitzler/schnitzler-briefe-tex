%% latex-korrekturansicht-vorspann.tex
%% Vorspann für die Korrekturansicht.
%% Lädt die gemeinsame Datei latex-vorspann.tex mit gesetztem Schalter.

\newif\ifkorrekturansicht
\korrekturansichttrue

\input{../tex-inputs/latex-vorspann}


               \section[Hermann Bahr an Arthur Schnitzler, 15. 10. 1902]{ Hermann Bahr an Arthur Schnitzler, 15. 10. 1902}\nopagebreak\mylabel{v}\rehead{ }\normalsize\beginnumbering\briefempfaengerindex{Schnitzler, Arthur@\textsc{Schnitzler, Arthur}!zzzBahr, Hermann@\emph{von Hermann Bahr}!1902-10-152@{15. 10. 1902}|(be} \toendnotes[C]{\smallbreak\pagebreak[2]} \Standort{CUL, Schnitzler, B 5b.}
\physDesc{Brief, 1 Blatt, 1 Seite
\newline{}Handschrift: schwarze Tinte, deutsche Kurrent\newline{}Ordnung: mit Bleistift von unbekannter Hand nummeriert:
                                    »91« }\buchAbdrucke{\weitereDrucke{Hermann Bahr, Arthur Schnitzler: \emph{Briefwechsel, Aufzeichnungen, Dokumente (1891–1931)}. Hg. Kurt Ifkovits und Martin Anton Müller. Göttingen: \emph{Wallstein} 2018, S. 244.} }\toendnotes[C]{\smallbreak}\pstart
           \noindent{}{\pb}\textcolor{gray}{\textbf{\textcolor{pink}{GRAND HÔTEL}{}\ledrightnote{\textcolor{pink}{Hotel de Rome}}}}\hfill \textcolor{gray}{\textbf{\textcolor{pink}{Berlin N. W.}{}\ledrightnote{\textcolor{pink}{Berlin}}, den .......... 190}}\pend
           \pstart
           \textcolor{gray}{\textbf{DE ROME U. DU NORD}}\hfill \textcolor{pink}{\textcolor{gray}{\textbf{Unter den Linden 39.}}}{}\ledrightnote{\textcolor{pink}{Unter den Linden}}\pend
           \pstart
           \textcolor{blue}{\textcolor{gray}{\textbf{A. MÜHLING}}}{}\ledrightnote{\textcolor{blue}{Adolph Mühling}}\hfill 15. 10\pend
           \pstart
           \textcolor{gray}{\textbf{Kgl. Hoflieferant}}\pend
           \pstart
           \textcolor{pink}{\textcolor{gray}{\textbf{BERLIN}}}{}\ledrightnote{\textcolor{pink}{Berlin}}\pend
           \pstart
           \textcolor{gray}{\textbf{Fernsprecher: Amt I, No. 4438.}}\pend
           \pstart\center{}Lieber Arthur!\pend\pstart
           Herzlichſten Dank! In einer Zeitung las ich: \textcolor{blue}{Halm}{}\ledrightnote{\textcolor{blue}{Alfred Halm}}
               hätte als D\textsuperscript{r}{ }\label{K_L01242_1v}\edtext{\textcolor{green}{Mohn}{}\ledrightnote{→\textcolor{green}{Wienerinnen}}}{\lemma{\textnormal{\emph{Mohn}}}\Cendnote{\textnormal{Figur aus \emph{\textcolor{green}{Wienerinnen}}}}}\label{K_L01242_1h}{ }\label{K_L01242_2v}\edtext{Deine Maske gehabt}{\lemma{\textnormal{\emph{Deine Maske gehabt}}}\Cendnote{\textnormal{nicht nachgewiesen; vielleicht eine
                  Fehlleistung \textcolor{blue}{Bahr}s zur \textcolor{green}{Rezension} von \textcolor{blue}{Karl Strecker}: »Herr \textcolor{blue}{Halm}, der
                     auch die Regie führte, gab einen modernen Ästheten mit gedrehter Stirnlocke,
                     einen eitlen Faiseur, seltsamerweise aber in der Maske von \textcolor{blue}{Hermann Bahr}.« (\emph{\textcolor{green}{Berliner Theater. Hermann Bahr: »Wienerinnen«.
                        (Eine nicht einwandfreie Kritik)}}. In: \emph{\textcolor{green}{Tägliche Rundschau}}, Jg. 20, Nr. 483, Morgenblatt, 1. Ausgabe,
                        15. 10. 1902, S. [2]). Vgl. A. S.: \emph{Tagebuch}, 18. 10. 1894}}}\label{K_L01242_2h}. Wahr iſt, daß er einen blonden Vollbart trug, aus lauter Angſt, in die Maske
                  \textcolor{blue}{Sudermanns}{}\ledrightnote{\textcolor{blue}{Hermann Sudermann}} zu gerathen. Daß es ganz albern
               wäre, einem ſpöttelnden Salon-Kritiker Deine Züge zu geben, brauche ich Dir ja nicht
               erſt zu ſagen. Die Leut ſind ſo blöd!\pend
           \pstart
           Herzlichſt{\\[\baselineskip]}Dein{\\[\baselineskip]}\spacefill\mbox{Hermann}\pend
           \leftskip=0em{}\endnumbering\briefempfaengerindex{Schnitzler, Arthur@\textsc{Schnitzler, Arthur}!zzzBahr, Hermann@\emph{von Hermann Bahr}!1902-10-152@{15. 10. 1902}|)be}\mylabel{h}  \normalsize

\doendnotes{C}
\bigskip
\vfill

\clearpage

\footnotesize

\lohead{\textsc{register}}

% Definiere theindex-Environment komplett neu ohne reledmac
\makeatletter
\renewenvironment{theindex}{%
  \section*{\indexname}%
  \setlength{\parindent}{0pt}%
  \setlength{\parskip}{0pt plus 0.3pt}%
  \let\item\@idxitem
}{%
  \clearpage
}
\makeatother

\IfFileExists{\jobname-pw.ind}{\input{\jobname-pw.ind}}{}

\end{document}

      