%% latex-korrekturansicht-vorspann.tex
%% Vorspann für die Korrekturansicht.
%% Lädt die gemeinsame Datei latex-vorspann.tex mit gesetztem Schalter.

\newif\ifkorrekturansicht
\korrekturansichttrue

\input{../tex-inputs/latex-vorspann}


\renewcommand{\erwaehntePersonen}{Personen: Eva Marie Goldmann}
\renewcommand{\erwaehnteOrte}{Orte: Berlin, Riva del Garda, Sternwartestraße, Venedig, Wien, XVIII., Währing}
\renewcommand{\erwaehnteWerke}{}
\section[ Paul Goldmann an Arthur Schnitzler, 22. 4. 1927]{Paul Goldmann an Arthur Schnitzler, 22. 4. 1927}
\nopagebreak\mylabel{v}
\rehead{ }\normalsize\beginnumbering\briefempfaengerindex{Schnitzler, Arthur@\textsc{Schnitzler, Arthur}!zzzGoldmann, Paul@\emph{von Paul Goldmann}!1927-04-221@{22. 4. 1927}|(be}
\toendnotes[C]{\smallbreak\pagebreak[2]}\Standort{DLA, A:Schnitzler, HS.NZ85.1.3176.}
\physDesc{Postkarte, 451 Zeichen
\newline{}Handschrift: 1) schwarze Tinte, deutsche Kurrent\hspace{1em}2) schwarze Tinte, lateinische Kurrent (\noindent{}Adresse)\hspace{1em}
\newline{}Versand: Stempel: »\nobreak{}Wien
                                          \textcolor{gray}{110}\nobreak{}«. Stempel: »\nobreak{}Wien 1, 22\textcolor{gray}{.} 4. 27, 9—10 N\nobreak{}«.  
\newline{}Schnitzler: mit rotem Buntstift eine Unterstreichung }\toendnotes[C]{\smallbreak}\pstart{}{\pb}Herrn\pend{}\pstart{}Dr. Arthur Schnitzler\pend{}\pstart{}\textcolor{pink}{Sternwartestraſse 71}{}\ledrightnote{\textcolor{pink}{Sternwartestraße}}\pend{}\pstart{}\textcolor{pink}{Wien XVIII.}{}\ledrightnote{\textcolor{pink}{XVIII., Währing}}\pend{}
{\bigskip}
\pstart
           \textcolor{pink}{Wien}{}\ledrightnote{\textcolor{pink}{Wien}}{ }22. \substVorne{}\textsuperscript{7}\substDazwischen{}4\substHinten{}. 27.\pend
           
\pstart{}Lieber Freund,\pend
\pstart
           Meine \textcolor{blue}{Frau}{}\ledrightnote{{$\rightarrow$}\textcolor{blue}{Eva Marie Goldmann}} u. ich ſind für
               einige Tage in \textcolor{pink}{Wien}{}\ledrightnote{\textcolor{pink}{Wien}}. Ich habe heut bei Dir angerufen, um Dich zu fragen, wann \textcolor{blue}{wir}{}\ledrightnote{{$\rightarrow$}\textcolor{blue}{Eva Marie Goldmann}} Dich beſuchen können. Zu
               meinem großen Bedauern erfahre ich, daß Du \label{K_L03481-1v}\edtext{verreiſt}{\lemma{\textnormal{\emph{verreiſt}}}\Cendnote{\textnormal{\textcolor{blue}{Schnitzler} war seit 1. 4. 1927 und noch
                  bis 2. 5. 1927 in
                     \textcolor{pink}{Venedig}.}}}\label{K_L03481-1h} biſt. {\pb}Ich ſende Dir alſo auf dieſem Wege meiner \textcolor{blue}{Frau}{}\ledrightnote{{$\rightarrow$}\textcolor{blue}{Eva Marie Goldmann}} u. meine herzlichſten
               Grüße. Wir hoffen auf ein \label{K_L03481-2v}\edtext{Wiederſehen
               in \textcolor{pink}{Berlin}{}\ledrightnote{\textcolor{pink}{Berlin}}}{\lemma{\textnormal{\emph{Wiederſehen
               in Berlin}}}\Cendnote{\textnormal{\textcolor{blue}{Schnitzler} und \textcolor{blue}{Goldmann} sahen sich am 12. 8. 1927 in \textcolor{pink}{Riva del Garda} wieder, dann am 7. 10. 1927 in \textcolor{pink}{Wien} und am 5. 12. 1927 in \textcolor{pink}{Berlin}.}}}\label{K_L03481-2h}, da
               wir ſo bald nicht wieder nach \textcolor{pink}{Wien}{}\ledrightnote{\textcolor{pink}{Wien}} kommen
               dürften.\pend
           
\pstart
           Dein {\\[\baselineskip]}\spacefill\mbox{Paul Goldmann.}\pend
           \leftskip=0em{}\endnumbering\briefempfaengerindex{Schnitzler, Arthur@\textsc{Schnitzler, Arthur}!zzzGoldmann, Paul@\emph{von Paul Goldmann}!1927-04-221@{22. 4. 1927}|)be}\mylabel{h}
\begin{anhang}
\end{anhang}\normalsize

\doendnotes{C}
\bigskip
\vfill

\clearpage

\footnotesize

\lohead{\textsc{register}}

% Definiere theindex-Environment komplett neu ohne reledmac
\makeatletter
\renewenvironment{theindex}{%
  \section*{\indexname}%
  \setlength{\parindent}{0pt}%
  \setlength{\parskip}{0pt plus 0.3pt}%
  \let\item\@idxitem
}{%
  \clearpage
}
\makeatother

\IfFileExists{\jobname-pw.ind}{\input{\jobname-pw.ind}}{}

\end{document}

      