%% latex-korrekturansicht-vorspann.tex
%% Vorspann für die Korrekturansicht.
%% Lädt die gemeinsame Datei latex-vorspann.tex mit gesetztem Schalter.

\newif\ifkorrekturansicht
\korrekturansichttrue

\input{../tex-inputs/latex-vorspann}


\renewcommand{\erwaehntePersonen}{Personen: Maximilian Harden, Felix Salten, Paul Schlenther, Olga Schnitzler, Heinrich Schnitzler}
\renewcommand{\erwaehnteOrte}{Orte: Berlin, Dessauer Straße, Italien, Südtirol, Wien}
\renewcommand{\erwaehnteWerke}{Werke: Berliner Theater. (»Der Schleier der Beatrice« von Arthur Schnitzler.), Der Schleier der Beatrice, Der Schleier der Beatrice. Schauspiel in fünf Akten, Die Zeit. Wiener Wochenschrift, Die Zukunft, Tagebuch, [Schlenther als Leiter des Burgtheaters]}
\section[ Paul Goldmann an Arthur Schnitzler, 27. 3. {[}1903{]}]{Paul Goldmann an Arthur Schnitzler, 27. 3. {[}1903{]}}
\nopagebreak\mylabel{v}
\rehead{ }\normalsize\beginnumbering\briefempfaengerindex{Schnitzler, Arthur@\textsc{Schnitzler, Arthur}!zzzGoldmann, Paul@\emph{von Paul Goldmann}!1903-03-272@{27. 3. {[}1903{]}}|(be}
\toendnotes[C]{\smallbreak\pagebreak[2]}\Standort{DLA, A:Schnitzler, HS.NZ85.1.3173.}
\physDesc{Brief, 1 Blatt, 3 Seiten
\newline{}Handschrift: blaue Tinte, deutsche Kurrent
\newline{}Schnitzler: 1) mit Bleistift das Jahr »{[}1{]}903.« vermerkt  2) mit rotem Buntstift zwei Unterstreichungen}\toendnotes[C]{\smallbreak}
\pstart
           \noindent{}\raggedleft{}{\pb}\textcolor{gray}{\textbf{\textcolor{pink}{DESSAUERSTRASSE 19}{}\ledrightnote{\textcolor{pink}{Dessauer Straße}}}}\pend
           
\pstart
           \textcolor{pink}{Berlin}{}\ledrightnote{\textcolor{pink}{Berlin}}, 27. März.\pend
           
\pstart\center{}Mein lieber Freund,\pend
\pstart
           Täglich will ich Dir ſchreiben, und immer verhindert mich die Arbeit daran. Arbeit
               und Verſtimmung: ich kann mich zu gar nichs mehr aufraffen. Dein lieber Brief war mir
               eine \label{K_L03370-1v}\edtext{große Freude und
                  Herzenserleichterung}{\lemma{\textnormal{\emph{große … Herzenserleichterung}}}\Cendnote{\textnormal{Bezug auf \textcolor{blue}{Goldmann}s kritisches \emph{\textcolor{green}{Beatrice}}-\textcolor{green}{Feuilleton}, siehe Paul Goldmann an Arthur Schnitzler, 17. 3. [1903]. \textcolor{blue}{Schnitzler} dürfte seine
                  Verärgerung über das \textcolor{green}{Feuilleton}{ }\textcolor{blue}{Goldmann} gegenüber noch nicht ausgedrückt
                  haben. Vgl. etwa das \emph{\textcolor{green}{Tagebuch}} ab dem 19. 3. 1903.}}}\label{K_L03370-1h}.
               Sachlich hätte ich noch mancherlei zu ſagen. Aber ich möchte über dieſes {\pb}unglückſelige \textcolor{green}{Feuilleton}{}\ledrightnote{{$\rightarrow$}\textcolor{green}{Berliner Theater. (»Der Schleier der Beatrice« von Arthur Schnitzler.)}}, das ich habe ſchreiben \uline{müſſen}, überhaupt nicht mehr reden.\pend
           
\pstart
           Heute{ }\textcolor{green}{tritt}{}\ledrightnote{{$\rightarrow$}\textcolor{green}{Der Schleier der Beatrice}}{ }\label{K_L03370-2v}\edtext{\textsc{\textcolor{blue}{Harden}{}\ledrightnote{\textcolor{blue}{Maximilian Harden}}}}{\lemma{\textnormal{\emph{Harden}}}\Cendnote{\textnormal{\textcolor{blue}{M. H.} [=\textcolor{blue}{Maximilian Harden}]: \emph{\textcolor{green}{Der Schleier der Beatrice}}. In: \emph{\textcolor{green}{Die Zukunft}}, Bd. 42, 28. 3. 1903, S. 517–530.}}}\label{K_L03370-2h} mit großer Wärme für die »\textsc{\textcolor{green}{Beatrice}{}\ledrightnote{\textcolor{green}{Der Schleier der Beatrice. Schauspiel in fünf Akten}}}« ein. Ich liebe zwar dieſe ſeine \label{K_L03370-3v}\edtext{»rhapſodiſchen«}{\lemma{\textnormal{\emph{»rhapſodiſchen«}}}\Cendnote{\textnormal{unzusammenhängend,
                  lückenhaft}}}\label{K_L03370-3h} Aufſätze nicht; aber ich freue mich des ſtarken \textcolor{blue}{Anhänger}{}\ledrightnote{{$\rightarrow$}\textcolor{blue}{Maximilian Harden}}s, der Dir und Deinem
                  \textcolor{green}{Werke}{}\ledrightnote{{$\rightarrow$}\textcolor{green}{Der Schleier der Beatrice. Schauspiel in fünf Akten}} erwächſt.\pend
           
\pstart
           \label{K_L03370-4v}\edtext{\textsc{\textcolor{green}{\textcolor{blue}{Salten}{}\ledrightnote{\textcolor{blue}{Felix Salten}}}{}\ledrightnote{{$\rightarrow$}\textcolor{green}{[Schlenther als Leiter des Burgtheaters]}}} über \textsc{\textcolor{blue}{Schlenther}{}\ledrightnote{\textcolor{blue}{Paul Schlenther}}}}{\lemma{\textnormal{\emph{Salten über Schlenther}}}\Cendnote{\textnormal{\textcolor{blue}{Felix Salten}: \emph{\textcolor{green}{[Schlenther als Leiter des Burgtheaters]}}. In: \emph{\textcolor{green}{Die Zeit. Wiener Wochenschrift}}, Jg. XXXX,
                     Nr. 484, März 1903, S. YYYY.}}}\label{K_L03370-4h}{ }{\pb}hat mir und hoffentlich auch Dir ſehr wohl
               gethan.\pend
           
\pstart
           Wie geht es Dir? \textsc{\textcolor{blue}{Olga}{}\ledrightnote{\textcolor{blue}{Olga Schnitzler}}}? Dem \textcolor{blue}{Sohn}{}\ledrightnote{{$\rightarrow$}\textcolor{blue}{Heinrich Schnitzler}}? Wirſt Du
                  \label{K_L03370-7v}\edtext{verreiſen}{\lemma{\textnormal{\emph{verreiſen}}}\Cendnote{\textnormal{Die nächste größere Reise war zwischen 28. 5. 1903 und 15. 6. 1903 nach \textcolor{pink}{Italien} und \textcolor{pink}{Südtirol}, gemeinsam mit \textcolor{blue}{Olga
                     Gussmann}.}}}\label{K_L03370-7h}? Wann? Wohin?\pend
           
\pstart
           Sei vielmals gegrüßt von Deinem getreuen {\\[\baselineskip]}\spacefill\mbox{Paul Goldmn}\pend
           \leftskip=0em{}\endnumbering\briefempfaengerindex{Schnitzler, Arthur@\textsc{Schnitzler, Arthur}!zzzGoldmann, Paul@\emph{von Paul Goldmann}!1903-03-272@{27. 3. {[}1903{]}}|)be}\mylabel{h}
\begin{anhang}
\end{anhang}\normalsize

\doendnotes{C}
\bigskip
\vfill

\clearpage

\footnotesize

\lohead{\textsc{register}}

% Definiere theindex-Environment komplett neu ohne reledmac
\makeatletter
\renewenvironment{theindex}{%
  \section*{\indexname}%
  \setlength{\parindent}{0pt}%
  \setlength{\parskip}{0pt plus 0.3pt}%
  \let\item\@idxitem
}{%
  \clearpage
}
\makeatother

\IfFileExists{\jobname-pw.ind}{\input{\jobname-pw.ind}}{}

\end{document}

      