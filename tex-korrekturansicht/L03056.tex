%% latex-korrekturansicht-vorspann.tex
%% Vorspann für die Korrekturansicht.
%% Lädt die gemeinsame Datei latex-vorspann.tex mit gesetztem Schalter.

\newif\ifkorrekturansicht
\korrekturansichttrue

\input{../tex-inputs/latex-vorspann}


\renewcommand{\erwaehntePersonen}{Personen:  ?? [behandelnder Arzt von Marie Glümer, Anfang 1901], Auguste Chlum, Marie Glümer, Markus Hajek, Arthur Kuttner, Louise Schnitzler}
\renewcommand{\erwaehnteOrte}{Orte: Berlin, Dessauer Straße, Wien}
\renewcommand{\erwaehnteWerke}{Werke: Lieutenant Gustl. Novelle}
\section[ Paul Goldmann an Arthur Schnitzler, 29. 1. {[}1901{]}]{Paul Goldmann an Arthur Schnitzler, 29. 1. {[}1901{]}}
\nopagebreak\mylabel{v}
\rehead{ }\normalsize\beginnumbering\briefempfaengerindex{Schnitzler, Arthur@\textsc{Schnitzler, Arthur}!zzzGoldmann, Paul@\emph{von Paul Goldmann}!1901-01-292@{29. 1. {[}1901{]}}|(be}
\toendnotes[C]{\smallbreak\pagebreak[2]}\Standort{DLA, A:Schnitzler, HS.NZ85.1.3171.}
\physDesc{Brief, 1 Blatt, 3 Seiten
\newline{}Handschrift Paul Goldmann: blaue Tinte, deutsche Kurrent
\newline{}Handschrift Auguste Chlum: Bleistift, deutsche Kurrent
\newline{}Beilage: handschriftlicher Brief, 1 Blatt, 3 Seiten, Bleistift,
                                 lateinische Kurrent 
\newline{}Schnitzler: 1) mit Bleistift das Jahr »{[}1{]}901« vermerkt  2) mit rotem Buntstift eine Unterstreichung}\toendnotes[C]{\smallbreak}
\pstart
           \noindent{}\raggedleft{}{\pb}\textcolor{pink}{\textcolor{gray}{\textbf{DESSAUERSTRASSE 19}}}{}\ledrightnote{\textcolor{pink}{Dessauer Straße}}\pend
           
\pstart
           \textcolor{pink}{Berlin}{}\ledrightnote{\textcolor{pink}{Berlin}}, 29. Januar.\pend
           
\pstart\center{}Mein lieber Freund,\pend
\pstart
           Auch ich war unruhig, aber es liegt kein Gund dazu vor, wie beifolgender Brief
               beweiſt. \substVorne{}\textsuperscript{\textcolor{gray}{W}}\substDazwischen{}D\substHinten{}a ich ein großes Mißtrauen gegen den behandelnden »\label{K_L03056-1v}\edtext{\textcolor{blue}{Wunderdoktor}{}\ledrightnote{{$\rightarrow$}\textcolor{blue}{?? [behandelnder Arzt von Marie Glümer, Anfang 1901]}}}{\lemma{\textnormal{\emph{Wunderdoktor}}}\Cendnote{\textnormal{nicht ermittelt}}}\label{K_L03056-1h}« hatte, ſandte
               ich das \label{K_L03056-4v}\edtext{\textcolor{blue}{Mädel}{}\ledrightnote{{$\rightarrow$}\textcolor{blue}{Marie Glümer}}}{\lemma{\textnormal{\emph{Mädel}}}\Cendnote{\textnormal{siehe Paul Goldmann an Arthur Schnitzler, 22. 1. [1901]}}}\label{K_L03056-4h} zu meinem
               Freunde \textsc{Dr. \textcolor{blue}{Kuttner}{}\ledrightnote{\textcolor{blue}{Arthur Kuttner}}} (den \textsc{Dr. \textcolor{blue}{Hajek}{}\ledrightnote{\textcolor{blue}{Markus Hajek}}} kennt u. ſchätzt). Die Viſite fand geſtern
               ſtatt. \textsc{Dr. \textcolor{blue}{K.}{}\ledrightnote{{$\rightarrow$}\textcolor{blue}{Arthur Kuttner}}} telephonirte mir: Beſſerung ſei bald zu erwarten. Er glaube, daß der
               behandelnde \textcolor{blue}{Arzt}{}\ledrightnote{{$\rightarrow$}\textcolor{blue}{?? [behandelnder Arzt von Marie Glümer, Anfang 1901]}} mit ſeinen
               Heilmitteln (\label{K_L03056-2v}\edtext{Arſenik}{\lemma{\textnormal{\emph{Arſenik}}}\Cendnote{\textnormal{psychoaktive Substanz zur Steigerung des
                  Appetits und des Wohlbefindens}}}\label{K_L03056-2h}) im Weſentlichen auf dem rechten Wege ſei,
               wünſche {\pb}auch, daß das \textcolor{blue}{Fräulein}{}\ledrightnote{{$\rightarrow$}\textcolor{blue}{Marie Glümer}} weiter bei dieſem \textcolor{blue}{Arzt}{}\ledrightnote{{$\rightarrow$}\textcolor{blue}{?? [behandelnder Arzt von Marie Glümer, Anfang 1901]}} in Behandlung bleibe,
               da er großen pſychiſchen Einfluß auf ſeine Patienten habe. Die Behandlung in der Naſe
               ſei allerdings eine »Gemeinheit«. Ob Malaria vorliege, könne man nicht wiſſen,
               ſolange keine Temperater-Meſſungen u. Blut-Unterſuchungen vorgenommen, woran der
               behandelnde \textcolor{blue}{Arzt}{}\ledrightnote{{$\rightarrow$}\textcolor{blue}{?? [behandelnder Arzt von Marie Glümer, Anfang 1901]}} nicht zu
               denken ſcheine{\dotsfour}\pend
           
\pstart
           Daß man Dich doch noch \label{K_L03056-3v}\edtext{ehrengerichtlich verfolgt}{\lemma{\textnormal{\emph{ehrengerichtlich verfolgt}}}\Cendnote{\textnormal{wegen des \emph{\textcolor{green}{Lieutenant Gustl}}, siehe Paul Goldmann an Arthur Schnitzler, 11. 1. [1901]}}}\label{K_L03056-3h}, iſt {\pb}empörend! Sei nur ja recht vorſichtig
               und thue keinen Schritt, ohne vorher mit Rechts- und Landeskundigen Dich berathen zu
               haben!\pend
           
\pstart
           In Eile!\pend
           
\pstart
           Dein {\\[\baselineskip]}\spacefill\mbox{P. G.}\pend
           \leftskip=0em{}
\pstart
           \noindent{}{\pb}{[}hs. Chlum:{]} Lieber Herr Doktor,\pend
           
\pstart
           Vor allem vielen Dank für Ihre Bemühungen. Wir ſind heute mit Beruhigung
               von \textsc{D\textsuperscript{r}}{ }\textcolor{blue}{Kuttner}{}\ledrightnote{\textcolor{blue}{Arthur Kuttner}} weggegangen. Auſführlicher werde ich
               Ihnen natürlich berichten. Die Krankheit, die {\pb}ſich
               plötzlich geſtern, So{\geminationn}tag{ }Nachm. brach, iſt tatſächlich am Verſchwinden und k\strikeout{l}ein Rückfall mehr zu befürchten. – Wir ſind Ihnen
               jedenfalls für dieſe Beruhigung ſehr dankbar, die \textcolor{gray}{wir}\textcolor{gray}{ums} ſelbſt zu verſtehen, {\pb}wahrſcheinlich noch nicht die Energie gehabt hätten. – Bitte gelegentlich um ein
               Stückchen Ihrer freien Zeit.\pend
           
\pstart
           Mit besten Empfehlungen für Ihre \textcolor{blue}{Frau Ma{\geminationm}a}{}\ledrightnote{{$\rightarrow$}\textcolor{blue}{Louise Schnitzler}}{ }{\\[\baselineskip]}Ihre ergebenen {\\[\baselineskip]}\spacefill\mbox{Marie + GustiGlümer}\pend
           \leftskip=0em{}\endnumbering\briefempfaengerindex{Schnitzler, Arthur@\textsc{Schnitzler, Arthur}!zzzGoldmann, Paul@\emph{von Paul Goldmann}!1901-01-292@{29. 1. {[}1901{]}}|)be}\mylabel{h}
\begin{anhang}
\end{anhang}\normalsize

\doendnotes{C}
\bigskip
\vfill

\clearpage

\footnotesize

\lohead{\textsc{register}}

% Definiere theindex-Environment komplett neu ohne reledmac
\makeatletter
\renewenvironment{theindex}{%
  \section*{\indexname}%
  \setlength{\parindent}{0pt}%
  \setlength{\parskip}{0pt plus 0.3pt}%
  \let\item\@idxitem
}{%
  \clearpage
}
\makeatother

\IfFileExists{\jobname-pw.ind}{\input{\jobname-pw.ind}}{}

\end{document}

      