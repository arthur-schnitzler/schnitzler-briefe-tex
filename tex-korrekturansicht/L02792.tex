%% latex-korrekturansicht-vorspann.tex
%% Vorspann für die Korrekturansicht.
%% Lädt die gemeinsame Datei latex-vorspann.tex mit gesetztem Schalter.

\newif\ifkorrekturansicht
\korrekturansichttrue

\input{../tex-inputs/latex-vorspann}


               \section[ Paul Goldmann an Arthur Schnitzler, 2. {[}1.? 1897{]}]{Paul Goldmann an Arthur Schnitzler, 2. {[}1.? 1897{]}}\nopagebreak\mylabel{v}\rehead{ }\normalsize\beginnumbering\briefempfaengerindex{Schnitzler, Arthur@\textsc{Schnitzler, Arthur}!zzzGoldmann, Paul@\emph{von Paul Goldmann}!1897-01-021@{2. {[}1.? 1897{]}}|(be} \toendnotes[C]{\smallbreak\pagebreak[2]} \Standort{DLA, A:Schnitzler, HS.NZ85.1.3166.}
\physDesc{Brief, 5 Blätter, 18 Seiten
\newline{}Handschrift: blaue Tinte, deutsche Kurrent
\newline{}Schnitzler: 1) mit Bleistift das Jahr »96« sowie die Tagesangabe des Datums unterstrichen und mit
                                 »?« kommentiert 2) mit rotem Buntstift acht Unterstreichungen}\toendnotes[C]{\smallbreak}\pstart
           \noindent{}{\pb}\textcolor{gray}{\textbf{\textbf{\textcolor{brown}{Frankfurter Zeitung}{}\ledrightnote{\textcolor{brown}{Frankfurter Zeitung}}}}}\pend
           \pstart
           \textcolor{gray}{\textbf{(\textcolor{brown}{\begin{otherlanguage}{french}Gazette de Francfort\end{otherlanguage}}{}\ledrightnote{\textcolor{brown}{Frankfurter Zeitung}}).}}\pend
           \pstart
           \textcolor{gray}{\textbf{\textbf{\begin{otherlanguage}{french}Fondateur M.\end{otherlanguage}{ }\textcolor{blue}{L. Sonnemann}{}\ledrightnote{\textcolor{blue}{Leopold Sonnemann}}.}}}\pend
           \pstart
           \begin{otherlanguage}{french}\textcolor{gray}{\textbf{\textcolor{green}{Journal}{}\ledrightnote{→\textcolor{green}{Frankfurter Zeitung}} politique,
                        financier,}}\end{otherlanguage}\pend
           \pstart
           \begin{otherlanguage}{french}\textcolor{gray}{\textbf{commercial et littéraire.}}\end{otherlanguage}\pend
           \pstart
           \begin{otherlanguage}{french}\textcolor{gray}{\textbf{\textbf{Paraissant trois fois par jour.}}}\end{otherlanguage}\hfill \textsc{\textcolor{pink}{Paris}{}\ledrightnote{\textcolor{pink}{Paris}}}, \label{K_L02792-48v}\edtext{2. December}{\lemma{\textnormal{\emph{2. December}}}\Cendnote{\textnormal{Es ist davon auszugehen, dass \textcolor{blue}{Goldmann} den Brief falsch datierte und
                        nicht am 2. 12. 1896, sondern am 2. 1. 1897 verfasste. Dafür spricht, dass er \textcolor{blue}{Schnitzler} eingangs ein frohes neues
                           Jahr wünscht.}}}\label{K_L02792-48h}.\pend
           \pstart
           \begin{otherlanguage}{french}\textcolor{gray}{\textbf{\textbf{Bureau à \textcolor{pink}{Paris}{}\ledrightnote{\textcolor{pink}{Paris}}}}}\end{otherlanguage}\pend
           \pstart
           \begin{otherlanguage}{french}\textcolor{gray}{\textbf{\textbf{\textcolor{pink}{24. Rue Feydeau}{}\ledrightnote{\textcolor{pink}{rue Feydeau}}.}}}\end{otherlanguage}\pend
           \pstart{}Mein lieber Freund,\pend\pstart
           Ich wünſche Dir von Herzen ein glückliches neues Jahr. Im
               alten Jahr waren die Tage, die ich mit Dir verlebt, für mich
               wohl das Beſte. Ich danke Dir \strikeout{\textcolor{gray}{×}\-\textcolor{gray}{×}\-\textcolor{gray}{×}\-\textcolor{gray}{×}} vielmals für alle Deine Treue und Güte{\dotsseven}\pend
           \pstart
           Sehr habe ich mich mit Deinem lieben ausführlichen Briefe gefreut. Er hätte gleich
               beantwortet werden ſollen. In jenen Tagen hatte ich keine Zeit dazu, und dann kam ein
               ſchrecklicher \strikeout{Zuſ\textcolor{gray}{×}\-\textcolor{gray}{×}\-\textcolor{gray}{×}} Zuſammenbruch: neue Erſcheinungen der gewiſſen \label{K_L02792-1v}\edtext{Krankheit}{\lemma{\textnormal{\emph{Krankheit}}}\Cendnote{\textnormal{vermutlich Syphilis}}}\label{K_L02792-1h}, Verſchlimmerung des Augenübels, eine vom Arzt
               conſtatirte unheilbare \label{K_L02792-2v}\edtext{\textsc{Mydriase}}{\lemma{\textnormal{\emph{Mydriase}}}\Cendnote{\textnormal{Pupillenerweiterung}}}\label{K_L02792-2h}, {\pb}mit Möglichkeit der Verſchlimmerung, vielleicht gar
               des Sehverluſtes. Was ſoll ich das Alles aufzählen? Seitdem habe ich nicht mehr die
               Kraft, irgend etwas zu thun. Ich gehe nirgends hin, weiſe alle Beſuche ab, bleibe bis
                  Mittag im Bett liegen und denke nur über das Sterben nach. In den
               Schmerz miſcht ſich die Reue, in die Todes- und Selbſtmord-Gedanken die Sehnſucht
               nach dem Leben, nach dem ich heißer begehre als je. Das ſind ſchlimme Tage, und Du
               begreifſt, daß \strikeout{\textcolor{gray}{×}h} dein Brief
               unbeantwortet bleiben mußte. Nun möchte ich Dir aber trotzdem ſagen, daß ich oft an
               Dich denke, und ſo raffe ich mich auf und ſchreibe Dir doch{\dotssix}\pend
           \pstart
           {\pb}Vor einiger Zeit war ich bei \textsc{\textcolor{blue}{Thorel}{}\ledrightnote{\textcolor{blue}{Jean Thorel}}}. Durch die \label{K_L02792-3v}\edtext{Directons-Kriſis im
                  »\textsc{\textcolor{brown}{Odéon}{}\ledrightnote{\textcolor{brown}{Odéon}}}«}{\lemma{\textnormal{\emph{Directons-Kriſis im »Odéon«}}}\Cendnote{\textnormal{Zwischen 14. 6. 1896 und 29. 10. 1896 waren \textcolor{blue}{Paul Ginisty} und \textcolor{blue}{André Antoine} gemeinsam die Direktoren des \emph{\textcolor{brown}{Odéon}}-Theaters. Danach hatte \textcolor{blue}{Ginisty} diese Funktion alleine inne.}}}\label{K_L02792-3h} und den
               Weggang \textsc{\textcolor{blue}{Antoine}{}\ledrightnote{\textcolor{blue}{André Antoine}}s} iſt \label{K_L02792-4v}\edtext{eine unſerer Combinationen}{\lemma{\textnormal{\emph{eine … Combinationen}}}\Cendnote{\textnormal{hier im Sinne von: Überlegungen, siehe Paul Goldmann an Arthur Schnitzler, 4. 6. [1896]}}}\label{K_L02792-4h} geſtört worden. \textsc{\textcolor{blue}{Thorel}{}\ledrightnote{\textcolor{blue}{Jean Thorel}}} hat dem übrigbleibenden Director \textsc{\textcolor{blue}{Ginisty}{}\ledrightnote{\textcolor{blue}{Paul Ginisty}}} zwar das \textcolor{green}{Stück}{}\ledrightnote{→\textcolor{green}{Amourette. Pièce en trois actes. Adaptée de Arthur Schnitzler}}
               überreicht; aber das iſt ein Flachkopf, und er wird es kaum acceptiren. Ein anderes
                  \label{K_L02792-84v}\edtext{\textcolor{green}{Manuſkript}{}\ledrightnote{→\textcolor{green}{Amourette. Pièce en trois actes. Adaptée de Arthur Schnitzler}}}{\lemma{\textnormal{\emph{Manuſkript}}}\Cendnote{\textnormal{\textcolor{blue}{Goldmann} meinte ein weiteres Exemplar von
                     \emph{\textcolor{green}{Amourette}}, der Übersetzung von \emph{\textcolor{green}{Liebelei}}. \textcolor{blue}{Albert Carré} lobte dieses einige Monate später (vgl. A. S.: \emph{Tagebuch}, 7. 5. 1897).}}}\label{K_L02792-84h} iſt zur
               Zeit bei \textsc{\textcolor{blue}{Carré}{}\ledrightnote{\textcolor{blue}{Albert Carré}}}, dem \textcolor{blue}{Director}{}\ledrightnote{→\textcolor{blue}{Albert Carré}} des »\textsc{\textcolor{brown}{Vaudeville}{}\ledrightnote{\textcolor{brown}{Théâtre du Vaudeville}}}«. \textsc{\textcolor{blue}{Thorel}{}\ledrightnote{\textcolor{blue}{Jean Thorel}}}{ }\strikeout{\textcolor{gray}{ſt}} wird auf dieſer Seite mit allen Mitteln arbeiten. Freunde \textsc{\textcolor{blue}{Carré}{}\ledrightnote{\textcolor{blue}{Albert Carré}}}s ſollen in Bewegung geſetzt werden, \textsc{\textcolor{blue}{Pierre Loti}{}\ledrightnote{\textcolor{blue}{Pierre Loti}}}, \textsc{\textcolor{blue}{Thorel}{}\ledrightnote{\textcolor{blue}{Jean Thorel}}s} intimer \textcolor{blue}{Freund}{}\ledrightnote{→\textcolor{blue}{Pierre Loti}}, ſoll auch ein Wort mitreden. In den
               nächſten Wochen werden wir Bericht über das Ergebniß erhalten.\pend
           \pstart
           Du findeſt in dieſem Briefe 1.) eine \textcolor{green}{Beſprechung}{}\ledrightnote{→\textcolor{green}{Het Tooneel. Groote Schouwburg. Minnespel. (Liebelei, van Arthur Schnitzler.)}} der »\textcolor{green}{Liebelei}{}\ledrightnote{\textcolor{green}{Liebelei. Schauspiel in drei Akten}}« {\pb}im \label{K_L02792-78v}\edtext{»\textsc{\textcolor{green}{Rotterdamsche Courant}{}\ledrightnote{\textcolor{green}{Nieuwe Rotterdamsche Courant}}}«}{\lemma{\textnormal{\emph{»Rotterdamsche Courant«}}}\Cendnote{\textnormal{[O. V.:] 
                     \emph{\textcolor{green}{Het Tooneel. Groote Schouwburg. Minnespel.
                        (Liebelei, van Arthur Schnitzler.)}}. In: \emph{\textcolor{green}{Nieuwe Rotterdamsche Courant}}, Jg. 53, Nr. 300, 15. 12. 1896, S. 1. \textcolor{blue}{Schnitzler} bewahrte diese \textcolor{green}{Besprechung} in seiner Zeitungsausschnittssammlung auf.
                  Die Premiere des Stückes (\emph{\textcolor{green}{Minne-spel}}) in der
                  Übersetzung von \textcolor{blue}{Frans Mijnssen} und
                  veranstaltet von \emph{\textcolor{brown}{Vereenigde Rotterdamsche
                     Tooneelisten}} fand am 11. 12. 1896 in der
                     \textcolor{pink}{Groote Schouwburg} statt.}}}\label{K_L02792-78h}, die mir
               der hieſige \label{K_L02792-67v}\edtext{\textcolor{blue}{Correſpondent}{}\ledrightnote{→\textcolor{blue}{Émile Wesly}}}{\lemma{\textnormal{\emph{Correſpondent}}}\Cendnote{\textnormal{möglicherweise der Komponist und
                  Journalist \textcolor{blue}{Émile Wesly}}}}\label{K_L02792-67h} des \textcolor{brown}{Blatt}{}\ledrightnote{→\textcolor{brown}{Nieuwe Rotterdamsche Courant}}es, ein guter
               Freund von mir, übergeben hat, um ſie an Dich zu befördern. 2.) Einen Brief von
                  \label{K_L02792-889v}\edtext{\textsc{\textcolor{blue}{Brandes}{}\ledrightnote{\textcolor{blue}{Georg Brandes}}} an mich 3.) Einen Brief von \textsc{\textcolor{blue}{Nansen}{}\ledrightnote{\textcolor{blue}{Peter Nansen}}}}{\lemma{\textnormal{\emph{Brandes … Nansen}}}\Cendnote{\textnormal{Beide Briefbeilagen sind nicht
                  überliefert und dürften \textcolor{blue}{Goldmann}
                  zurückgesandt worden sein.}}}\label{K_L02792-889h} an mich. Beide Briefe bitte ich Dich, mir \uline{zurückzuſenden}. Beide Briefe \strikeout{\textcolor{gray}{×}\-\textcolor{gray}{×}\-\textcolor{gray}{×}} hätte ich Dir ſchon längſt ſenden ſollen\textcolor{gray}{,} aber ich wollte
               ſie erſt beantworten. Beide Briefe geben auch Dir wohl \label{K_L02792-74v}\edtext{Anlaß zu einer Antwort an die \textcolor{blue}{Abſender}{}\ledrightnote{→\textcolor{blue}{Peter Nansen}{\newline}→\textcolor{blue}{Georg Brandes}}}{\lemma{\textnormal{\emph{Anlaß … Abſender}}}\Cendnote{\textnormal{Der nächste Brief \textcolor{blue}{Schnitzler}s an \textcolor{blue}{Brandes} (Arthur Schnitzler an Georg Brandes, 11. 1. 1897) enthält keinen
                  Hinweis, dass diese Aufforderung motivierend wirkte. Der nächste Brief der
                  überlieferten Korrespondenz \textcolor{blue}{Schnitzler}–\textcolor{blue}{Nansen} datiert vom
                  15. 3. 1897.}}}\label{K_L02792-74h}.\pend
           \pstart
           Die \label{K_L02792-88v}\edtext{\textcolor{green}{Kritik}{}\ledrightnote{→\textcolor{green}{Le livre à Paris. Francis de Pressensé: Le Cardinal Manning. – Arthur Schnitzler (traduction Gaspard Vallette): Mourir}} in »\textsc{\textcolor{green}{Cosmopolis}{}\ledrightnote{\textcolor{green}{Cosmopolis}}}}{\lemma{\textnormal{\emph{Kritik in »Cosmopolis}}}\Cendnote{\textnormal{\textcolor{blue}{Émile Faguet}: \emph{\textcolor{green}{Le livre à Paris. Francis de Pressensé: Le Cardinal
                        Manning. – Arthur Schnitzler (traduction Gaspard Vallette): Mourir}}. In:
                        \emph{\textcolor{green}{Cosmopolis}}, Jg. 4, H. 12,
                        Dezember 1896, S. 792–803.}}}\label{K_L02792-88h}« hat mich \substVorne{}\textsuperscript{rieſig}{\allowbreak}\substDazwischen{}rieſig\substHinten{} gefreut. \textsc{\textcolor{blue}{Faguet}{}\ledrightnote{\textcolor{blue}{Émile Faguet}}} iſt, wie Du wohl weißt, der \strikeout{Nachf}{ }\textcolor{blue}{Nachfolger}{}\ledrightnote{→\textcolor{blue}{Émile Faguet}} von \textsc{\textcolor{blue}{Jules Lemaître}{}\ledrightnote{\textcolor{blue}{Jules Lemaître}}} als Theater-Kritiker im »\textsc{\textcolor{brown}{Journal des Débats}{}\ledrightnote{\textcolor{brown}{Journal des débats}}}« und einer der größten Literatur-\substVorne{}\textsuperscript{Bo}\substDazwischen{}Bonzen\substHinten{} von \textsc{\textcolor{pink}{Paris}{}\ledrightnote{\textcolor{pink}{Paris}}}. \pend
           \pstart
           {\pb}Die \label{K_L02792-89v}\edtext{Aufnahme der \textcolor{green}{Lausbüberei}{}\ledrightnote{→\textcolor{green}{Die demolirte Literatur}} des
                  \textsc{\textcolor{blue}{Kraus}{}\ledrightnote{\textcolor{blue}{Karl Kraus}}} in die \textcolor{green}{Frankf. Zeit.}{}\ledrightnote{\textcolor{green}{Frankfurter Zeitung}}}{\lemma{\textnormal{\emph{Aufnahme … Zeit.}}}\Cendnote{\textnormal{Am 19. 12. 1896 wurde \emph{\textcolor{green}{Die demolirte Literatur}} in der \emph{\textcolor{green}{Frankfurter Zeitung}} nachgedruckt (Jg. 41, Nr. 352,
                     Abendausgabe, S. 1). vgl. A. S.: \emph{Tagebuch}, 19. 12. 1896}}}\label{K_L02792-89h} hat mich bitter gekränkt. Ich habe mich ſofort bei meinem \textcolor{blue}{Onkel}{}\ledrightnote{→\textcolor{blue}{Fedor Mamroth}} beſchwert. Dieſer iſt vollſtändig
                  \label{K_L02792-21v}\edtext{\textsc{bona fide}}{\lemma{\textnormal{\emph{bona fide}}}\Cendnote{\textnormal{lateinisch: guten Glaubens}}}\label{K_L02792-21h}, hat
               keine Ahnung gehabt, um wen es ſich handelt, und hat die Sache, wie er mir mittheilt,
               nur aufgenommen, weil er ſie »vorzüglich geſchrieben fand«. Ich vermuthe, daß meines
                  \textcolor{blue}{Onkel}{}\ledrightnote{→\textcolor{blue}{Fedor Mamroth}}s \textcolor{blue}{Frau}{}\ledrightnote{→\textcolor{blue}{Johanna Mamroth}} dahinterſteckt; ſie dürfte das neue
               Genie \textsc{\textcolor{blue}{Kraus}{}\ledrightnote{\textcolor{blue}{Karl Kraus}}} entdeckt haben, das ſieht ihr ſchon ähnlich; und mein \textcolor{blue}{Onkel}{}\ledrightnote{→\textcolor{blue}{Fedor Mamroth}} ſieht in dieſen \strikeout{Fälle\textcolor{gray}{m}} Fällen nur mit {\pb}ihren Augen. \strikeout{\textcolor{gray}{A}\textcolor{gray}{×}} Oder auch iſt die \textcolor{green}{Sache}{}\ledrightnote{→\textcolor{green}{Die demolirte Literatur}}
               von \textsc{\textcolor{blue}{Altenberg}{}\ledrightnote{\textcolor{blue}{Peter Altenberg}}} gekommen, mit welchem die große \label{K_L02792-123v}\edtext{\textcolor{blue}{Kritikerin}{}\ledrightnote{→\textcolor{blue}{Johanna Mamroth}} im
                  Briefwechſel}{\lemma{\textnormal{\emph{Kritikerin im Briefwechſel}}}\Cendnote{\textnormal{vgl. den Brief \textcolor{blue}{Peter Altenberg}s an \textcolor{blue}{Hermann Bahr}, Dezember 1898: »Frau \textcolor{blue}{Johanna Schwarz-Mamroth}, welche über mein
                        \textcolor{green}{2. Buch} in der \textcolor{green}{Frankfurter Zeitung} sehr lobend
                        {[}g{]}eschrieben hat, bittet mich von \textcolor{pink}{Florenz} aus {[}{\dots}{]}« (Hermann Bahr und Peter Altenberg: \emph{Korrespondenz
                        von Peter Altenberg an H. B. (1895-1913)}. Hgg. von Heinz Lunzer und
                     Victoria Lunzer-Talos. In: Jeanne Bennay und Alfred Pfabigan, Hgg.: \emph{Hermann Bahr – Für eine andere Moderne}. Bern: Peter Lang
                     2004, S. 249–262, hier: 258.) Nachgewiesen ist nur eine Rezension des
                  ersten Buches, nicht des zweiten (\emph{\textcolor{green}{Ashantee}}): \textcolor{blue}{J. S.}: \emph{\textcolor{green}{»Wie ich es sehe«}}. In: \emph{\textcolor{green}{Frankfurter
                        Zeitung}}, Jg. XXXX, Nr. YYYY, 8. 6. 1896, S YYYY.
               }}}\label{K_L02792-123h} ſteht, ſeit ſie ihn als Dichter gekrönt hat. Ich bin machtlos gegen ſolche
               Dinge, kann nur hinterher wüthend ſein und kann nicht einmal einer Wiederholung
                  vorbeugen{\dotsfour}\pend
           \pstart
           Mit großer Theilnahme habe ich die Skizze von Deinem Tagewerk geleſen, die Du mir
               entworfen haſt. Daß auch Du von körperlichen Leiden geplagt biſt, iſt recht garſtig.
               Soviel ich von Medicin verſtehe, will mir freilich ein \label{K_L02792-12v}\edtext{Ohren-Katarrh}{\lemma{\textnormal{\emph{Ohren-Katarrh}}}\Cendnote{\textnormal{\textcolor{blue}{Schnitzler} litt seit
                     Herbst 1896 an Otosklerose – einer Verknöcherung des Innenohrs mit
                  zunehmender Schwerhörigkeit.}}}\label{K_L02792-12h} nicht ſchlimm erſcheinen. Wer weiß, ob Du ihn
               überhaupt entdeckt hätteſt, {\pb}wenn Du nicht Arzt
               wäreſt? Wie gern möchte ich ihn noch zu alle dem dazu nehmen, was ich habe! Auf einen
               Ohren-Katarrh mehr oder weniger käme es mir, weiß Gott, nicht an, wenn ich Dich \strikeout{\textcolor{gray}{von}} um dieſen Preis davon befreien könnte! Aber ich meine, das Ganze iſt doch ſo
               unbedeutend, daß Du Unrecht hätteſt, Dir deßwegen auch nur eine Minute Deines Lebens
               zu verſtören.\pend
           \pstart
           Merkwürdig iſt, daß Du trotz all’ dem Schönen, was Du haſt, Deines Lebens nicht froh
               wirſt. Ich komme um vor Sehnſucht und Reue – und Du, der Du Vieles von dem haſt, was
               ich erſehne, und Vieles noch haſt von dem, deſſen Verluſt ich bereue, – Du biſt darum
               doch {\pb}anſcheinend nicht ruhiger noch zufriedener.
               Ich werde von der Angſt gequält, daß ich werde ſterben müſſen, ohne je gelebt zu
               haben, – und Du, Du lebſt und leideſt darunter, daß Du Dich nicht leben fühlſt. Was
               ſind das für Räthſel? Deine und meine und \strikeout{\textcolor{gray}{a}} wahrſcheinlich aller Menſchen Lebensthätigkeit kommt auf dieſe Weiſe darauf
               hinaus, daß wir, Jeder in ſeiner Art, unſer Leben vertrödeln und verlieren. Was Dich
               anlangt, ſo meine ich, Du grübelſt zuviel. Du haſt zuviel Raum vor Deinen Blicken.
                  \strikeout{I\textcolor{gray}{c}h}{ }\strikeout{\textcolor{gray}{×}} Du ſollteſt Dir ſelbſt Grenzen aufſtellen. Die Löſung aller dieſer Probleme
                  {\pb}liegt vielleicht darin, daß man ſich ein Bett im
               Gewöhnlichen graben und ruhig zwiſchen zwei Ufern hinfließen ſoll. Das iſt zu
               bildlich ausgedrückt. Für Dich heißt die reale Überſetzung vielleicht: Du ſollteſt
               doch heirathen. Heirathen und Kinder haben – das iſt vielleicht der einzige Weg, jene
               Übereinſtimmung mit dem dunklen Willen der Natur herzuſtellen, die ſich durch inneren
               Frieden belohnt. Die Freiheit? Was hat das zu ſagen? Sie iſt doch nur dazu gut, um
                  {\pb}\strikeout{\textcolor{gray}{e}} einmal Jemandem ein großes Geſchenk damit zu machen, und wir machen \strikeout{\textcolor{gray}{e}i} eigentlich nur fortwährend Verſuche, ſie dem oder
               Jenem oder vielmehr Dieſer oder Jener \strikeout{h} wegzugeben, –
               die Freiheit{\dotssix}\pend
           \pstart
           Arbeiteſt Du nun wieder? \strikeout{H\textcolor{gray}{ub}} Hübſch iſt die Idee, ein \label{K_L02792-45v}\edtext{\textcolor{green}{Schlußſtück}{}\ledrightnote{→\textcolor{green}{Anatols Größenwahn}} zum »\textsc{\textcolor{green}{Anatol}{}\ledrightnote{\textcolor{green}{Anatol}}}«}{\lemma{\textnormal{\emph{Schlußſtück zum »Anatol«}}}\Cendnote{\textnormal{Unzufrieden mit dem letzten
                  Einakter \emph{\textcolor{green}{Anatols Hochzeitsmorgen}}, wünschte
                  sich \textcolor{blue}{Mitterwurzer} »ein anderes
                     letztes Stück ›Anatols Tod‹: Warum soll so ein Lump nicht sterben?«.
                     \textcolor{blue}{Schnitzler} verfasste in Folge \emph{\textcolor{green}{Anatols Größenwahn}}, das aber weder \textcolor{blue}{Mitterwurzer} noch \textcolor{blue}{Schnitzler} gefiel und nicht in die Buchausgabe aufgenommen
                  wurde. (\emph{\textcolor{green}{Anatol. Historisch-kritische Ausgabe}}
                     18.)}}}\label{K_L02792-45h} zu ſchreiben. Auch ſoll \textsc{\textcolor{blue}{Mitterwurzer}{}\ledrightnote{\textcolor{blue}{Friedrich Mitterwurzer}}} ruhig den \label{K_L02792-878v}\edtext{\textcolor{green}{Cyclus}{}\ledrightnote{→\textcolor{green}{Anatol}} der kleinen Stücke}{\lemma{\textnormal{\emph{Cyclus … Stücke}}}\Cendnote{\textnormal{\emph{\textcolor{green}{Anatol}}, dessen Szenen noch nie gemeinsam
                  gespielt waren}}}\label{K_L02792-878h} ſpielen. Deine ganze Eigenart ſteckt doch darin, wenn ſie
               auch klein ſind. Die \label{K_L02792-56v}\edtext{Idee der »\textcolor{green}{Entrüſteten}{}\ledrightnote{→\textcolor{green}{Der Weg ins Freie. Roman}}«}{\lemma{\textnormal{\emph{Idee der »Entrüſteten«}}}\Cendnote{\textnormal{Stoff, der sich über ein Jahrzehnt
                  entwickelte und der zum Roman \emph{\textcolor{green}{Der Weg ins
                     Freie}} wurde. Die Idee (noch als Bühnenstück) notierte \textcolor{blue}{Schnitzler} am 24. 3. 1895 im \emph{\textcolor{green}{Tagebuch}}.}}}\label{K_L02792-56h} gefällt mir ſehr. Es ſollte {\pb}einmal \strikeout{\textcolor{gray}{×}} ſchlankweg ein Luſtſpiel werden. Dazu gehört ſreilich Ruhe und
               Seelen-Heiterkeit; aber Du wirſt ſie ſchon wieder finden. Könnteſt Du nicht auf ein
               paar Wochen nach dem Süden fahren? Der \label{K_L02792-987v}\edtext{\textcolor{green}{Theater-Roman}{}\ledrightnote{→\textcolor{green}{Theaterroman}}}{\lemma{\textnormal{\emph{Theater-Roman}}}\Cendnote{\textnormal{Romanidee, die er bis zu seinem Tod
                  weiterverfolgte, aber erst 1967 publiziert wurde.}}}\label{K_L02792-987h}
               muß wohl erſt \substVorne{}\textsuperscript{reifen}{\allowbreak}\substDazwischen{}reifen\substHinten{}. Laß’ den \label{K_L02792-8976v}\edtext{\textsc{\textcolor{blue}{Bahr}{}\ledrightnote{\textcolor{blue}{Hermann Bahr}}} nur ruhig \strikeout{\textcolor{gray}{vo}} vorangehen}{\lemma{\textnormal{\emph{Bahr … vorangehen}}}\Cendnote{\textnormal{Am
                     20. 3. 1897 erschien von \textcolor{blue}{Bahr} ein im Theatermilieu angesiedelter Text: \emph{\textcolor{green}{Theater. Ein Wiener Roman}} im \emph{\textcolor{brown}{S. Fischer-Verlag}}.}}}\label{K_L02792-8976h}! Was hat denn das für Belang,
               was der \strikeout{\textcolor{gray}{M}}{ }Hanswurſt ſchreibt? Du ſcheinſt übrigens wieder gut
               mit ihm \strikeout{\textcolor{gray}{z}} zu ſtehen? Die »\textcolor{green}{Zeit}{}\ledrightnote{\textcolor{green}{Die Zeit. Wiener Wochenschrift}}« iſt ſo zuckerſüß
               für Dich. Was der \label{K_L02792-24v}\edtext{\textsc{\textcolor{blue}{Servaes}{}\ledrightnote{\textcolor{blue}{Franz Servaes}}} dort über Dich geſchrieben}{\lemma{\textnormal{\emph{Servaes … geſchrieben}}}\Cendnote{\textnormal{\textcolor{blue}{Franz Servaes}: \emph{\textcolor{green}{Jung Wien. Berliner Eindrücke}}. In: \emph{\textcolor{green}{Die Zeit}}, Bd. 10, Nr. 118, 2. 1. 1897,
                     S. 6–8: »Der erste, der kam, war \so{Arthur Schnitzler}, und damit kam gleich ein echtes Stück vom guten, alten, nun
                     wieder jung gewordenen \textcolor{pink}{Wien}. Er ist nicht
                     gar zu schnell berühmt geworden, und das war sein Glück. So bewahrte er sich
                     umso länger seine Naivetät, die gerade bei ihm von unschätzbarem Juwelenglanz
                     ist. Er hat etwas \textcolor{blue}{Goeth}isches in seinem
                     Naturell, etwas vom frühen \textcolor{blue}{Goethe}, in
                     der Art, wie er im Volke wurzelt, wie er das Volk fühlt und liebt und wie er
                     doch wieder als der vornehme Herr und denkende Mensch zum Volke sich herab
                     lässt. Diese Innigkeit der Gemüthsverbindung macht seine Naivetät. Er hat so
                     schöne, schlichte Worte für seine ›süßen Mädln‹, und die süßen Mädln haben die
                     gleichen Worte für ihn. Trotzdem ist er ein neugieriger, wissbegieriger
                     Experimentator. Aber das ist der Unterschied gegen \textcolor{pink}{Berlin}: hier experimentiert man mit dem Verstande, \textcolor{blue}{Schnitzler} thut es mit dem Herzen; bei uns
                     experimentiert man an sorglich zubereiteten Präparaten, \textcolor{blue}{Schnitzler} thut es am lebenden Organismus. Und niemals
                     verwischt er beim Experimentieren den \so{Duft} des
                     Lebens. Er lässt es auf sich wirken in seiner Ganzheit, Unberührbarkeit, er
                     schlürft mit feiner prüfender Zunge seine Poesie. Ja, wenn man es recht nimmt,
                     experimentiert er eigentlich nur an sich selber. Das Draußen liegt heiter,
                     gelassen, nur wenig in Mitleidenschaft gezogen, schaukelt in seinen Bahnen
                     ruhig auf und nieder. Aber in ihm selber sitzt der Nerv, der feine,
                     empfindliche, der bei jeder Berührung zuckt, und der stets in der Wonne bereits
                     die Qual, in der Lust die Unlust spürt. Und dann wieder die Freude, solche
                     Schmerzen empfinden zu können, weil man soviel edler darum ist, soviel weiser.
                     Und die noch viel höhere Freude, den ganzen Complex von Schmerzen und
                     Seligkeiten, diesen wüsten durcheinandergeschlungenen Ballen
                     ineinanderverbissener Amphibien, den mit zarter fühlender Hand sachte
                     aufdröseln zu können, Worte dafür zu finden, malende Ausdrücke, spiegelnde
                     Verdichtungen! Die Sprache zu zwingen, dass sie den Erlebnissen unseres Inneren
                     folgt, die spröde, geizige, verschämte deutsche Sprache, die doch einen
                     Reichthum in sich birgt und ein fesselloses Jauchzen, eine Biegsamkeit und
                     herrische Uebergewalt wie – ja, das meine ich wirklich! – wie keine zweite
                     Sprache der Welt. Und \textcolor{blue}{Schnitzler} hat vor
                     allem die Wärme und die Anmuth unserer Sprache und ihre leise, singende
                     Wehmuth.«}}}\label{K_L02792-24h}, iſt {\pb}gewiß ſehr
               ſchön; aber der Unſinn ſonſt in dem \textcolor{green}{Artikel}{}\ledrightnote{→\textcolor{green}{Jung Wien. Berliner Eindrücke}}! Und \textsc{\textcolor{blue}{Bahr}{}\ledrightnote{\textcolor{blue}{Hermann Bahr}}} als der \textcolor{blue}{Entbinder}{}\ledrightnote{→\textcolor{blue}{Hermann Bahr}}, der
                  \label{K_L02792-87v}\edtext{\textsc{\textcolor{blue}{Georg Brandes}{}\ledrightnote{\textcolor{blue}{Georg Brandes}}} von \textsc{\textcolor{pink}{Wien}{}\ledrightnote{\textcolor{pink}{Wien}}}}{\lemma{\textnormal{\emph{Georg Brandes von Wien}}}\Cendnote{\textnormal{\textcolor{blue}{Hermann Bahr} wird von \textcolor{blue}{Franz Servaes} als der Erfinder von Jung-\textcolor{pink}{Wien} geschildert, als ihr Sprachrohr. Das war eine
                  historische Ungenauigkeit, zu der \textcolor{blue}{Bahr}
                  seinen Beitrag geleistet hat. Eine junge \textcolor{pink}{Wien}er
                  Literaturbewegung entwickelte sich tatsächlich noch bevor \textcolor{blue}{Bahr}1891 aus \textcolor{pink}{Berlin} nach \textcolor{pink}{Wien} übersiedelte. \textcolor{blue}{Bahr} war es aber, der die Literaturbewegung im
                  deutschsprachigen Feuilleton bewarb und bekannt machte – und insofern erst recht
                  wieder als ihr Erfinder gelten kann.}}}\label{K_L02792-87h}! Das kränkt mich immer bitter, weil
               ich ſehe, daß der \textcolor{blue}{Kerl}{}\ledrightnote{→\textcolor{blue}{Hermann Bahr}}{ }\label{K_L02792-98765v}\edtext{mir perſönlich etwas \strikeout{ſtie} ſtiehlt}{\lemma{\textnormal{\emph{mir … ſtiehlt}}}\Cendnote{\textnormal{\textcolor{blue}{Goldmann} konnte durch seine Tätigkeit als
                  Redakteur von \emph{\textcolor{brown}{An der schönen blauen Donau}} bis
                  zum Jahresende 1890 Anspruch darauf erheben, dem schriftstellerischen
                  Nachwuchs eine Publikationsmöglichkeit geschaffen zu haben. Zudem könnte er sich
                  auf eine geplante Vereinsbildung beziehen, von der am 2. 4. 1890 im \emph{\textcolor{green}{Tagebuch}} berichtet wird: »Ansätze zu
                     einem lit. Verein Jung \textcolor{pink}{Wien}: \textcolor{blue}{Poestion}, \textcolor{blue}{Lemmermayer}, \textcolor{blue}{Steiner}, \textcolor{blue}{List}, \textcolor{blue}{Wodiczka}, \textcolor{blue}{Ludaßy}, \textcolor{blue}{Klein}, \textcolor{blue}{Breitenstein}, \textcolor{blue}{Goldmann}, ich.« Spannend ist, dass bei diesem frühen
                  Zusammenschluss mit \textcolor{blue}{Guido von List} und \textcolor{blue}{Rudolf Steiner} deutschnationale und
                  antroposophische Mythenmetze beteiligt gewesen waren.}}}\label{K_L02792-98765h}. Die jungen \textcolor{pink}{Wien}{}\ledrightnote{\textcolor{pink}{Wien}}er haben keines Entbinders bedurft; aber wenn
               ſchon \strikeout{in} Einer da war, der ſie zuſammengeſucht hat,
               ſo war \uline{ich} es. Als \textsc{\textcolor{blue}{Bahr}{}\ledrightnote{\textcolor{blue}{Hermann Bahr}}} nach \textsc{\textcolor{pink}{Wien}{}\ledrightnote{\textcolor{pink}{Wien}}} kam, waren ſchon \strikeout{All} Alle da; und ſeine
               Wirkſamkeit hat ſich darauf beſchränkt, daß er Dich beſchimpft {\pb}und verkannt hat; daß er den \textsc{\textcolor{blue}{Loris}{}\ledrightnote{\textcolor{blue}{Hugo von Hofmannsthal}}} mißverſtanden und verdorben hat; und daß er als neues Genie den grotesken
               Zieraffen \textsc{\textcolor{blue}{Andrian}{}\ledrightnote{\textcolor{blue}{Leopold von Andrian-Werburg}}} gefunden hat. Und das läßt ſich \label{K_L02792-98v}\edtext{als \textcolor{blue}{Begründer}{}\ledrightnote{→\textcolor{blue}{Hermann Bahr}} der \textcolor{pink}{Wien}{}\ledrightnote{\textcolor{pink}{Wien}}er Bewegung preiſen}{\lemma{\textnormal{\emph{als … preiſen}}}\Cendnote{\textnormal{siehe Paul Goldmann an Arthur Schnitzler, 1. 6. [1894]}}}\label{K_L02792-98h}, deren gute Leiſtungen immer nur \uline{trotz}{ }\textsc{\textcolor{blue}{Bahr}{}\ledrightnote{\textcolor{blue}{Hermann Bahr}}} entſtanden ſind! {\dotsfour}\pend
           \pstart
           Dieſer Dr. \textsc{\textcolor{blue}{Graf}{}\ledrightnote{\textcolor{blue}{Max Graf}}}, den mir \textsc{\textcolor{blue}{Richard}{}\ledrightnote{\textcolor{blue}{Richard Beer-Hofmann}}} geſchickt hat, gefällt mir recht gut. Er hat eine angenehme Art, iſt aber wohl
               keine {\pb}ſtarke Perſönlichkeit und kein ſehr klarer
               Kopf. Er ſtreckt unſicher ſeine Fühlhörner ins \strikeout{Leben}
               Leben aus. \strikeout{W} Seine \textsc{\textcolor{blue}{Bahr}{}\ledrightnote{\textcolor{blue}{Hermann Bahr}}}-Bewunderung habe ich bereits ein wenig erſchüttert; aber es iſt nicht gut
               möglich, ihm auszureden, daß \textsc{\textcolor{blue}{Altenberg}{}\ledrightnote{\textcolor{blue}{Peter Altenberg}}} ein genialer Dichtergeiſt iſt. Wollen ſehen, was man aus ihm machen kann.
               Einſtweilen habe ich ihm kleine Arbeiten für unſer \textcolor{brown}{Blatt}{}\ledrightnote{→\textcolor{brown}{Frankfurter Zeitung}} verſchafft.\pend
           \pstart
           \strikeout{Di} Die Fragen, die Du an mich ſtellſt, \label{K_L02792-77v}\edtext{\textsc{\begin{otherlanguage}{french}me concernant\end{otherlanguage}}}{\lemma{\textnormal{\emph{me concernant}}}\Cendnote{\textnormal{französisch: mich betreffend}}}\label{K_L02792-77h},
               beantworten ſich von ſelbſt durch den Eingang dieſes Briefes {\pb}(zu deſſen Fertigftellung ich drei Tage gebraucht).
               Stimmung: verzweifelt (ich werde nie dazu kommen, den tiefen Riß in meinem Leben \strikeout{a} auszufüllen); Stellung: unerfreulich; Arbeit: null;
               Freunde: ein paar brave \label{K_L02792-34v}\edtext{Leute}{\lemma{\textnormal{\emph{Leute}}}\Cendnote{\textnormal{nicht identifiziert}}}\label{K_L02792-34h} auf \textsc{\textcolor{pink}{Montmartre}{}\ledrightnote{\textcolor{pink}{Paris 18 Buttes-Montmartre}}}, ehrliche und ſimple Menſchen, die mich in ihrer kühlen Weiſe gern haben und
               nicht verſtehen; Geliebte: ſchwere pſychiſche (?) Impotenz{\dotsfour}\pend
           \pstart
           Willſt Du mir einen Gefallen thun? Ich möchte gern den »\label{K_L02792-888v}\edtext{\textsc{\textcolor{green}{Lorenzaccio}{}\ledrightnote{\textcolor{green}{Lorenzaccio. Drame romantique en cinq actes}}}}{\lemma{\textnormal{\emph{Lorenzaccio}}}\Cendnote{\textnormal{\emph{\textcolor{green}{Lorenzaccio. Drame romantique en cinq actes}}
                  wurde postum am 3. 12. 1896 und damit zweiundsechzig Jahre nach der
                  Veröffentlichung am \emph{\textcolor{brown}{Théâtre de la Renaissance}}
                  uraufgeführt. Die Hauptrolle spielte \textcolor{blue}{Sarah
                     Bernhardt}.}}}\label{K_L02792-888h}« von \textsc{\textcolor{blue}{Musset}{}\ledrightnote{\textcolor{blue}{Alfred de Musset}}}{ }\label{K_L02792-677v}\edtext{für die deutſche {\pb}Bühne bearbeiten}{\lemma{\textnormal{\emph{für … bearbeiten}}}\Cendnote{\textnormal{Die Idee bestand jedenfalls seit
                     1894, vgl. A. S.: \emph{Tagebuch}, 8. 9. 1894 und vgl. Paul Goldmann an Arthur Schnitzler, 21. 9. [1894]. \textcolor{blue}{Schnitzler} fühlte bei \textcolor{blue}{Otto Brahm} vor, der ihm am 13. 5. 1897
                  antwortete: »Wegen einer \uline{\textcolor{green}{Lorenzaccio}}-Übersetzung bin ich Ihnen auch noch eine Antwort schuldig. Es ist
                     inzwischen eine bei uns eingelaufen und abgelehnt worden. Ist das die Ihres
                     Protegés? Ich glaube kaum, daß das Stück bei uns Chancen hätte; aber wenn die
                     Sache für Ihren Unbekannten noch nicht erledigt ist – einreichen kann er ja
                     immer, das ist Menschenrecht.« (Brahm/Schnitzler,
                  33)}}}\label{K_L02792-677h}. Ich ſende Dir anbei das \textcolor{green}{Feuilleton}{}\ledrightnote{→\textcolor{green}{?? [Feuilleton über Lorenzaccio von Musset]}}, das ich darüber geſchrieben. Könnte ich
               vielleicht vom »\textcolor{brown}{Burgtheater}{}\ledrightnote{\textcolor{brown}{Burgtheater}}« den Auftrag zu
               dieſer Bearbeitung bekommen? Könnteſt Du ein Wort mit \textsc{\textcolor{blue}{Burckhardt}{}\ledrightnote{\textcolor{blue}{Max Eugen Burckhard}}} oder mit \textsc{\textcolor{blue}{Uhl}{}\ledrightnote{\textcolor{blue}{Friedrich Uhl}}} reden? In meinem \textcolor{green}{Feuilleton}{}\ledrightnote{→\textcolor{green}{?? [Feuilleton über Lorenzaccio von Musset]}} finden ſie alle nöthigen ſachlichen Angaben über das \textcolor{green}{Stück}{}\ledrightnote{→\textcolor{green}{Lorenzaccio. Drame romantique en cinq actes}}. Das iſt ſo eine
               phantaſtiſche Idee, die ich habe; ausführbar wird ſie natürlich nicht ſein; und es
               lohnt nicht der Mühe, daß Du Dir deßwegen auch nur einen überflüßigen Weg machſt{\dotsfive}\pend
           \pstart
           {\pb}Wie gern würde ich Dich bald einmal wiederſehen\substVorne{}\textsuperscript{?}\substDazwischen{}!\substHinten{} Iſt gar keine Ausſicht, daß Du \label{K_L02792-54v}\edtext{nach \textsc{\textcolor{pink}{Paris}{}\ledrightnote{\textcolor{pink}{Paris}}}}{\lemma{\textnormal{\emph{nach Paris}}}\Cendnote{\textnormal{\textcolor{blue}{Schnitzler} und \textcolor{blue}{Marie Reinhard} kamen am 12. 4. 1897 nach \textcolor{pink}{Paris} und er blieb bis zum 24. 5. 1897, sie reiste einen Tag früher
                  ab.}}}\label{K_L02792-54h} kommſt?\pend
           \pstart
           Grüß’ mir den lieben \textsc{\textcolor{blue}{Richard}{}\ledrightnote{\textcolor{blue}{Richard Beer-Hofmann}}} und auch \textsc{\textcolor{blue}{Leo Vanjung}{}\ledrightnote{\textcolor{blue}{Leo Van-Jung}}}, wenn Du ihn \label{K_L02792-90v}\edtext{ſiehſt}{\lemma{\textnormal{\emph{ſiehſt}}}\Cendnote{\textnormal{Das nächste Mal trafen sich \textcolor{blue}{Schnitzler} und \textcolor{blue}{Leo Van-Jung} vermutlich am 12. 1. 1897.}}}\label{K_L02792-90h}!\pend
           \pstart
           Allen den Deinigen wünſche ich ein glückliches neues Jahr; empfiehl’ mich
               insbeſondere Deiner Frau \textcolor{blue}{Mutter}{}\ledrightnote{→\textcolor{blue}{Louise Schnitzler}} und grüße mir recht herzlich Deinen \textcolor{blue}{Bruder}{}\ledrightnote{→\textcolor{blue}{Julius Schnitzler}} und Deine \textcolor{blue}{Schwägerin}{}\ledrightnote{→\textcolor{blue}{Helene Schnitzler}}.\pend
           \pstart
           {\pb}Und ſei’ Du ſelbſt von Herzen gegrüßt!\pend
           \pstart
           In Treue {\\[\baselineskip]}Dein {\\[\baselineskip]}\spacefill\mbox{Paul Goldmann.}\pend
           \leftskip=0em{}\pstart
           \noindent{}Nicht wahr, Du ſchreibſt mir bald wieder eimmal?\pend
           \endnumbering\briefempfaengerindex{Schnitzler, Arthur@\textsc{Schnitzler, Arthur}!zzzGoldmann, Paul@\emph{von Paul Goldmann}!1897-01-021@{2. {[}1.? 1897{]}}|)be}\mylabel{h}  \normalsize

\doendnotes{C}
\bigskip
\vfill

\clearpage

\footnotesize

\lohead{\textsc{register}}

% Definiere theindex-Environment komplett neu ohne reledmac
\makeatletter
\renewenvironment{theindex}{%
  \section*{\indexname}%
  \setlength{\parindent}{0pt}%
  \setlength{\parskip}{0pt plus 0.3pt}%
  \let\item\@idxitem
}{%
  \clearpage
}
\makeatother

\IfFileExists{\jobname-pw.ind}{\input{\jobname-pw.ind}}{}

\end{document}

      