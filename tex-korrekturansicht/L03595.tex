%% latex-korrekturansicht-vorspann.tex
%% Vorspann für die Korrekturansicht.
%% Lädt die gemeinsame Datei latex-vorspann.tex mit gesetztem Schalter.

\newif\ifkorrekturansicht
\korrekturansichttrue

\input{../tex-inputs/latex-vorspann}


\renewcommand{\erwaehntePersonen}{Personen: Paul Claudel, Ellis O. Jones, Felix Salten}
\renewcommand{\erwaehnteInstitutionen}{Institutionen: Foreign Press Service}
\renewcommand{\erwaehnteOrte}{Orte: Berlin, Vereinigte Staaten von Amerika (USA), Wien}
\renewcommand{\erwaehnteWerke}{Werke: Der Tausch. Drama in drei Akten}
\section[ Felix Salten an Arthur Schnitzler, {[}22. 2. 1922?{]}]{Felix Salten an Arthur Schnitzler, {[}22. 2. 1922?{]}}
\nopagebreak\mylabel{v}
\rehead{ }\normalsize\beginnumbering\briefempfaengerindex{Schnitzler, Arthur@\textsc{Schnitzler, Arthur}!zzzSalten, Felix@\emph{von Felix Salten}!1922-02-221@{{[}22. 2. 1922?{]}}|(be}
\toendnotes[C]{\smallbreak\pagebreak[2]}\Standort{CUL, Schnitzler, B 89, B 2.}
\physDesc{Brief, 1 Blatt, 1 Seite, 639 Zeichen
\newline{}Handschrift: Bleistift, lateinische Kurrent
\newline{}Schnitzler: mit Bleistift womöglich Vermerk der Jahreszahl: »/22« 
\newline{}Ordnung: mit Bleistift von unbekannter Hand beschriftet: »?« }\toendnotes[C]{\smallbreak}
\pstart
           \raggedleft{}{\pb}\label{K_L03595-1v}\edtext{Dienstag}{\lemma{\textnormal{\emph{Dienstag}}}\Cendnote{\textnormal{Die Datierung des undatierten Korrespondenzstücks
                        gelingt durch Annäherung. Der eine überlieferte Brief von \textcolor{blue}{Ellis O. Jones} an \textcolor{blue}{Schnitzler} (\emph{DLA Marbach}, HS.1985.1.3581) ist datiert mit
                        »Nov 8 1921« und in \textcolor{pink}{Berlin} abgefasst. Aus ihm geht hervor, dass dieser
                        darauf hofft, \textcolor{blue}{Schnitzler} persönlich
                        kennenzulernen. Damit kann der Zeitraum des vorliegenden Korrespondenzstücks
                        nach vorne hin eingegrenzt werden. Die Verknüpfung von
                        »Generalprobe«, »Dienstag« und dem unsicher
                        gelesenen »Claudel« kann ferner als Hinweis auf die Generalprobe
                        des Stücks \emph{\textcolor{green}{Der Tausch}} am Mittwoch, dem
                        23. 2. 1921,
                        gelesen werden, die sowohl \textcolor{blue}{Schnitzler} als
                        auch \textcolor{blue}{Salten} besuchte. Ein Besuch \textcolor{blue}{Salten}s
                  am selben Abend ist nicht belegt.}}}\label{K_L03595-1h}.\pend
           
\pstart{}Lieber,\pend
\pstart
           ein M\textsuperscript{r}{ }\textcolor{blue}{Ellis O. Jones}{}\ledrightnote{\textcolor{blue}{Ellis O. Jones}}, \textcolor{pink}{amerika}{}\ledrightnote{\textcolor{pink}{Vereinigte Staaten von Amerika (USA)}}nischer Journalist, Vertreter des \textcolor{brown}{Foreign Press Service}{}\ledrightnote{\textcolor{brown}{Foreign Press Service}}, kommt heute{ }Nachmittag \uline{um 3h}, durch eine \label{K_L03595-2v}\edtext{Bekannte}{\lemma{\textnormal{\emph{Bekannte}}}\Cendnote{\textnormal{nicht ermittelt}}}\label{K_L03595-2h} eingeführt, um mich
               zu interviewen. Er will, \label{K_L03595-3v}\edtext{mit der
               gleichen Absicht, auch zu Ihnen}{\lemma{\textnormal{\emph{mit … Ihnen}}}\Cendnote{\textnormal{Weder
                  Besuch noch Interview können nachgewiesen werden.}}}\label{K_L03595-3h}. Ich weiß \uline{garnichts} von ihm, kann ihn weder empfehlen noch
               einführen, sondern habe es nur übernommen, die Anfrage an Sie weiterzugeben.
               Vielleicht laßen Sie zu mir her Bescheid sagen, ob Sie Herrn \textcolor{blue}{Jones}{}\ledrightnote{\textcolor{blue}{Ellis O. Jones}} überhaupt und ob Sie ihn dann, wenn er von mir
               fortgeht oder sonst wann empfangen wollen.\pend
           \pstart Herzlichst Ihr \spacefill\mbox{Salten}\pend{}
\pstart
           \noindent{}Mir interessantes:\pend
           
\pstart
           Sind Sie heute, etwa nach dem Nachtmahl, frei?\pend
           
\pstart
           Morgen in der Generalprobe von 
                  \textcolor{blue}{\textcolor{green}{\textcolor{gray}{Claudel}}{}\ledrightnote{{$\rightarrow$}\textcolor{green}{Der Tausch. Drama in drei Akten}}}{}\ledrightnote{\textcolor{blue}{Paul Claudel}}?\pend
           \endnumbering\briefempfaengerindex{Schnitzler, Arthur@\textsc{Schnitzler, Arthur}!zzzSalten, Felix@\emph{von Felix Salten}!1922-02-221@{{[}22. 2. 1922?{]}}|)be}\mylabel{h}  \normalsize

\doendnotes{C}
\bigskip
\vfill

\clearpage

\footnotesize

\lohead{\textsc{register}}

% Definiere theindex-Environment komplett neu ohne reledmac
\makeatletter
\renewenvironment{theindex}{%
  \section*{\indexname}%
  \setlength{\parindent}{0pt}%
  \setlength{\parskip}{0pt plus 0.3pt}%
  \let\item\@idxitem
}{%
  \clearpage
}
\makeatother

\IfFileExists{\jobname-pw.ind}{\input{\jobname-pw.ind}}{}

\end{document}

      