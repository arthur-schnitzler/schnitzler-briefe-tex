%% latex-korrekturansicht-vorspann.tex
%% Vorspann für die Korrekturansicht.
%% Lädt die gemeinsame Datei latex-vorspann.tex mit gesetztem Schalter.

\newif\ifkorrekturansicht
\korrekturansichttrue

\input{../tex-inputs/latex-vorspann}


\renewcommand{\erwaehntePersonen}{Personen: Marie Suin Beausacq, Eva Marie Goldmann, Maximilian Harden, Alfred Kerr, Heinrich Schnitzler, Olga Schnitzler, Hermann Sudermann}
\renewcommand{\erwaehnteInstitutionen}{Institutionen: Deutsches Theater Berlin, Helianthus}
\renewcommand{\erwaehnteOrte}{Orte: Berlin, Dessauer Straße, Paris, Wien}
\renewcommand{\erwaehnteWerke}{Werke: ?? [Die Roheit der Kritik], Berliner Theater. (»König Laurin« von Ernst v. Wildenbruch.), Der Schleier der Beatrice. Schauspiel in fünf Akten, Der Tag, Die Zukunft, Herr Sudermann, der D .. Di .. Dichter. Ein kritisches Vademecum, Livre d’or de la comtesse Diane, préface par Gaston Bergeret, Maximes de la vie. Préface par Sully Prud’homme, Neue Freie Presse, Theater [Erwiderung auf Sudermanns Verrohung in der Literaturkritik, II], Theater [Erwiderung auf Sudermanns Verrohung in der Literaturkritik], Verrohung in der Theaterkritik}
\section[ Paul Goldmann an Arthur Schnitzler, 24. 11. {[}1902{]}]{Paul Goldmann an Arthur Schnitzler, 24. 11. {[}1902{]}}
\nopagebreak\mylabel{v}
\rehead{ }\normalsize\beginnumbering\briefempfaengerindex{Schnitzler, Arthur@\textsc{Schnitzler, Arthur}!zzzGoldmann, Paul@\emph{von Paul Goldmann}!1902-11-242@{24. 11. {[}1902{]}}|(be}
\toendnotes[C]{\smallbreak\pagebreak[2]}\Standort{DLA, A:Schnitzler, HS.NZ85.1.3172.}
\physDesc{Brief, 1 Blatt, 4 Seiten
\newline{}Handschrift: blaue Tinte, deutsche Kurrent
\newline{}Schnitzler: 1) mit Bleistift das Jahr »{[}1{]}902« vermerkt  2) mit rotem Buntstift vier Unterstreichungen}\toendnotes[C]{\smallbreak}
\pstart
           \noindent{}\raggedleft{}{\pb}\textcolor{pink}{\textcolor{gray}{\textbf{DESSAUERSTRASSE 19}}}{}\ledrightnote{\textcolor{pink}{Dessauer Straße}}\pend
           
\pstart
           \textcolor{pink}{Berlin}{}\ledrightnote{\textcolor{pink}{Berlin}}, 24. November.\pend
           
\pstart{}Mein lieber Freund,\pend
\pstart
           Der Beifall, den Du in ſo gütigen Worten meinem \label{K_L03230-1v}\edtext{\textcolor{green}{Feuilleton}{}\ledrightnote{{$\rightarrow$}\textcolor{green}{Berliner Theater. (»König Laurin« von Ernst v. Wildenbruch.)}}}{\lemma{\textnormal{\emph{Feuilleton}}}\Cendnote{\textnormal{\textcolor{blue}{Paul Goldmann}: \emph{\textcolor{green}{Berliner Theater. (»König Laurin« von Ernst v.
                        Wildenbruch.)}}. In: \emph{\textcolor{green}{Neue Freie
                        Presse}}, Nr. 13.737, 22. 11. 1902,
                     Morgenblatt, S. 1–4. Die Reihenfolge, in der der Dank in diesem Brief
                  ausgesprochen wird, legt nahe, dass \textcolor{blue}{Schnitzler} seine Gratulation in einem separaten Schreiben, möglicherweise
                  einem Telegramm oder einer Karte ausdrückte.}}}\label{K_L03230-1h} ſpendeſt, hat mich innig
               erfreut, und ich Danke Dir von Herzen dafür.\pend
           
\pstart
           Dein lieber Brief, den ich \label{K_L03230-v}\edtext{Samſtag}{\lemma{\textnormal{\emph{Samſtag}}}\Cendnote{\textnormal{22. 11. 1902}}}\label{K_L03230-h} empfing, iſt nicht beſonders erfreulich. Warum ſo \label{K_L03230-2v}\edtext{mißgelaunt}{\lemma{\textnormal{\emph{mißgelaunt}}}\Cendnote{\textnormal{\textcolor{blue}{Schnitzler} plagten in dieser Zeit
                  Nervosität, Arbeitsunfähigkeit und Zukunftsängste, vgl. A. S.: \emph{Tagebuch}, 12. 11. 1902, 13. 11. 1902, 14. 11. 1902, 20. 11. 1902 und 23. 11. 1902.}}}\label{K_L03230-2h}? Wer wird ſich ſo vom Wetter
               abhängig machen? Und wenn es gegenwärtig mit dem Produziren nicht recht geht, ſo wird
               ſchon {\pb}wieder ein produktiver Zuſtand kommen. Der
               Geiſt ſammelt eben neue Kraft.\pend
           
\pstart
           Was iſt mit der \label{K_L03230-3v}\edtext{»\textsc{\textcolor{green}{Beatrice}{}\ledrightnote{\textcolor{green}{Der Schleier der Beatrice. Schauspiel in fünf Akten}}}« und dem »\textcolor{brown}{Deutſchen Theater}{}\ledrightnote{\textcolor{brown}{Deutsches Theater Berlin}}«}{\lemma{\textnormal{\emph{»Beatrice« … Theater«}}}\Cendnote{\textnormal{siehe Paul Goldmann an Arthur Schnitzler, 16. 6. [1902]}}}\label{K_L03230-3h}?\pend
           
\pstart
           Die \label{K_L03230-4v}\edtext{Bücher}{\lemma{\textnormal{\emph{Bücher}}}\Cendnote{\textnormal{nicht ermittelt}}}\label{K_L03230-4h}, die Du mir empfiehlſt, möchte ich
               gern leſen; nur wird die Erfüllung dieſes Wunſches an dem Umſtande ſcheitern, daß ich
               die Namen zumeiſt nicht leſen kann. Insbeſondere von Demjenigen, den Du mir ans Herz
               legſt, habe ich trotz eifriger Bemühung nicht mehr herausbekommen können, als daß er
               mit \textsc{L.}{ }{\pb}anfängt.\pend
           
\pstart
           Haſt Du Dir die »\textsc{\textcolor{green}{Maximes de la Vie}{}\ledrightnote{\textcolor{green}{Maximes de la vie. Préface par Sully Prud’homme}}}« \substVorne{}\textsuperscript{\textsc{der}}\substDazwischen{}der\substHinten{}{ }\label{K_L03230-7v}\edtext{\textsc{\textcolor{blue}{Comtesse Diane}{}\ledrightnote{\textcolor{blue}{Marie Suin Beausacq}}}}{\lemma{\textnormal{\emph{Comtesse Diane}}}\Cendnote{\textnormal{Zu \emph{\textcolor{green}{Maximes de la vie}}{ }siehe Paul Goldmann an Arthur Schnitzler, 2. [10. 1902]. Auch eine Lektüre
                  von \emph{\textcolor{green}{Livre d’or}} (\textcolor{pink}{Paris}{ }1886) ist nicht nachweisbar.}}}\label{K_L03230-7h} kommen laſſen? Noch
               ſchöner vielleicht iſt das \textsc{\textcolor{green}{Livre d’or}{}\ledrightnote{\textcolor{green}{Livre d’or de la comtesse Diane, préface par Gaston Bergeret}}} von derſelben, – ein entzückendes Spiel des Geiſtes und zugleich eine Quelle
               tiefer Lebensweisheit.\pend
           
\pstart
           Was \label{K_L03230-11v}\edtext{\textsc{\textcolor{blue}{\textcolor{green}{Sudermann}{}\ledrightnote{{$\rightarrow$}\textcolor{green}{Verrohung in der Theaterkritik}}}{}\ledrightnote{\textcolor{blue}{Hermann Sudermann}}}}{\lemma{\textnormal{\emph{Sudermann}}}\Cendnote{\textnormal{siehe Paul Goldmann an Arthur Schnitzler, 10. 11. [1902]}}}\label{K_L03230-11h} anlangt, bin ich durchaus Deiner Anſicht. Vielleicht ergreife ich in dem
               Streit noch das \label{K_L03230-12v}\edtext{Wort}{\lemma{\textnormal{\emph{Wort}}}\Cendnote{\textnormal{ein solches Feuilleton ist nicht
                  bekannt}}}\label{K_L03230-12h}, obwohl mir Andere gerade das, was ich ſagen möchte, weggeſchrieben
                  {\pb}haben. \label{K_L03230-13v}\edtext{\textsc{\textcolor{blue}{Kerr}{}\ledrightnote{\textcolor{blue}{Alfred Kerr}}s}{ }\textcolor{green}{Erwiderung}{}\ledrightnote{{$\rightarrow$}\textcolor{green}{?? [Die Roheit der Kritik]}}}{\lemma{\textnormal{\emph{Kerrs Erwiderung}}}\Cendnote{\textnormal{\textcolor{blue}{Alfred Kerr}: \emph{\textcolor{green}{Die Roheit der Kritik}}. In: \emph{\textcolor{green}{Der Tag}}, Jg. XXXX, Nr. YYYY, 21. 11. 1902,
                     S. YYYY. Weitgehend parallel dazu, wenngleich auf 1903
                  vordatiert, erschien dieser Text und gesammelte Kritiken \textcolor{blue}{Kerr}s zu \textcolor{blue}{Sudermann}s
                  Stücken als Broschüre: \textcolor{blue}{Alfred Kerr}: \emph{\textcolor{green}{Herr Sudermann, der D .. Di .. Dichter. Ein kritisches
                        Vademecum}}. \textcolor{pink}{Berlin}: \emph{\textcolor{brown}{Helianthus}}{ }1903. Die Vorbemerkung zur dritten Auflage – wohl zu lesen als 3. und 4.
                  Tausend – ist mit dem 6. 12. 1902 datiert. }}}\label{K_L03230-13h} war zum Theil
               hübſch in der Form, aber der Geſinnung nach lausbübiſch, wie überhaupt ein
               Lausbuben-Zug immer ſtärker bei ihm hervortritt. \textsc{\textcolor{blue}{Harden}{}\ledrightnote{\textcolor{blue}{Maximilian Harden}}} war, im erſten Theil ſeiner \label{K_L03230-32v}\edtext{\textcolor{green}{Erwiderung}{}\ledrightnote{{$\rightarrow$}\textcolor{green}{Theater [Erwiderung auf Sudermanns Verrohung in der Literaturkritik]}}}{\lemma{\textnormal{\emph{Erwiderung}}}\Cendnote{\textnormal{\textcolor{blue}{M. H.} [=\textcolor{blue}{Maximilian Harden}]: \emph{\textcolor{green}{Theater}}. In: \emph{\textcolor{green}{Die
                           Zukunft}}, Bd. 41, 22. 11. 1902, S. 311–326. (Der \textcolor{green}{zweite Teil} erschien in der Folgewoche, 29. 11. 1902, S. 356–370.}}}\label{K_L03230-32h}, viel bedeutender; im
               zweiten ſpricht er zu viel und zu eitel von ſich.\pend
           
\pstart
           Fräulein \label{K_L03230-223v}\edtext{\textsc{\textcolor{blue}{Eva F.}{}\ledrightnote{\textcolor{blue}{Eva Marie Goldmann}}}}{\lemma{\textnormal{\emph{Eva F.}}}\Cendnote{\textnormal{\textcolor{blue}{Eva Fränkel}, \textcolor{blue}{Goldmann}s spätere Ehefrau, die \textcolor{blue}{Schnitzler} bereits kannte}}}\label{K_L03230-223h} iſt hier. Ich habe ſie
               einmal geſehen und in den erſten fünf Minuten den Eindruck gehabt: »Es iſt
               unmöglich.« Es iſt beinahe eine phyſiſche Antipathie, die ich nicht werde überwinden
               können.\pend
           
\pstart
           Grüße \textsc{\textcolor{blue}{Heinrich}{}\ledrightnote{\textcolor{blue}{Heinrich Schnitzler}}} und ſeine \textcolor{blue}{Mutter}{}\ledrightnote{{$\rightarrow$}\textcolor{blue}{Olga Schnitzler}} und
               ſei Du ſelbſt vielmals gegrüßt {\\[\baselineskip]}von Deinem \spacefill\mbox{Paul Goldm}\pend
           \leftskip=0em{}\endnumbering\briefempfaengerindex{Schnitzler, Arthur@\textsc{Schnitzler, Arthur}!zzzGoldmann, Paul@\emph{von Paul Goldmann}!1902-11-242@{24. 11. {[}1902{]}}|)be}\mylabel{h}
\begin{anhang}
\end{anhang}\normalsize

\doendnotes{C}
\bigskip
\vfill

\clearpage

\footnotesize

\lohead{\textsc{register}}

% Definiere theindex-Environment komplett neu ohne reledmac
\makeatletter
\renewenvironment{theindex}{%
  \section*{\indexname}%
  \setlength{\parindent}{0pt}%
  \setlength{\parskip}{0pt plus 0.3pt}%
  \let\item\@idxitem
}{%
  \clearpage
}
\makeatother

\IfFileExists{\jobname-pw.ind}{\input{\jobname-pw.ind}}{}

\end{document}

      