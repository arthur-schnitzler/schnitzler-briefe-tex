%% latex-korrekturansicht-vorspann.tex
%% Vorspann für die Korrekturansicht.
%% Lädt die gemeinsame Datei latex-vorspann.tex mit gesetztem Schalter.

\newif\ifkorrekturansicht
\korrekturansichttrue

\input{../tex-inputs/latex-vorspann}


\section[Romain Rolland an Arthur Schnitzler, 19. 12. 1913]{L03882 Romain Rolland an Arthur Schnitzler, 19. 12. 1913}
\nopagebreak\mylabel{L03882v}
\rehead{ }\normalsize\beginnumbering\briefempfaengerindex{, @\textsc{, }!zzz, @\emph{von  }!1914-12-191@{19. 12. 1914}|(be}
\toendnotes[C]{\smallbreak\pagebreak[2]}\Standort{CUL, Schnitzler, B 86.}
\physDesc{Brief, 1 Blatt, 2 Seiten, 738 Zeichen
\newline{}Handschrift: blaue Tinte, lateinische Kurrent
\newline{}Schnitzler: mit Bleistift Vermerk: »\textsc{Rolland}« }\toendnotes[C]{\smallbreak}
\pstart
           {\pb}\textcolor{gray}{\textbf{\textbf{\textcolor{pink}{HOTEL BEAU-SÉJOUR}\oindex{Hôtel Beau-Séjour@\textbf{Hôtel Beau-Séjour}, \emph{Hotel}|pw}{}\ledrightnote{\textcolor{pink}{Hôtel Beau-Séjour}}}}}\hfill \label{K_L03882-1v}\edtext{Samedi 19 déc. 1914}{\lemma{\textnormal{\emph{Samedi 19 déc. 1914}}}\Cendnote{\textnormal{französisch: »Samstag, 19. Dezember 1914{ / }Sehr geehrter Herr Arthur Schnitzler,{ / }Die Übersetzung Ihres \textcolor{green}{Protestbriefes}\pwindex{Schnitzler, Arthur 15. 5. 1862 Wien – 21. 10. 1931 ebd.@\textsc{Schnitzler, Arthur} (15. 5. 1862 Wien – 21. 10. 1931 ebd.), \emph{Schriftsteller, Mediziner}!Une protestation d’Arthur Schnitzler@\strich\emph{Une protestation d’Arthur Schnitzler}|pwv}\pwindex{Rolland, Romain 29.\,1.\,1866 Clamecy – 30.\,12.\,1944 Vézelay@\textsc{Rolland, Romain} (29.\,1.\,1866 Clamecy – 30.\,12.\,1944 Vézelay), \emph{Schriftsteller}!Une protestation d’Arthur Schnitzler@\strich\emph{Une protestation d’Arthur Schnitzler}|pwv} wird morgen oder übermorgen
                              im \uline{\textcolor{green}{Journal de Genève}\pwindex{Journal de Genève@\emph{Journal de Genève}|pw}} erscheinen. Ich lasse Ihnen das betreffende Exemplar sofort
                              zuschicken. Ein zweites \textcolor{green}{Exemplar}\pwindex{Schnitzler, Arthur 15. 5. 1862 Wien – 21. 10. 1931 ebd.@\textsc{Schnitzler, Arthur} (15. 5. 1862 Wien – 21. 10. 1931 ebd.), \emph{Schriftsteller, Mediziner}!Brief Artur Schnitzlers@\strich\emph{Ein Brief Artur Schnitzlers}|pwv} habe ich an die \uline{\textcolor{brown}{Neue Zürcher Zeitung}\orgindex{Neue Zürcher Zeitung@Neue Zürcher Zeitung|pw}} gesandt.{ / }Ich freue mich, Ihnen behilflich sein zu können, eine Stimme der
                              Gerechtigkeit und der Vernunft inmitten dieses Deliriums des \textcolor{pink}{europäischen}\oindex{Europa@\textbf{Europa}|pw} Geistes hörbar zu
                              machen. Man sagt, dass selbst das Schlimmste seinen Nutzen habe: ich
                              will es glauben, da mir dieser abscheuliche Krieg die Gelegenheit
                              verschafft hat, mit Ihnen in freundschaftliche Beziehungen zu treten
                              und Ihnen meine aufrichtige künstlerische Sympathie für Ihr so
                              menschliches Werk — Dramen und Romane — auszusprechen.{ / }Glauben Sie, sehr geehrter Herr Dr. Schnitzler, an meine Ihnen ganz
                              ergebenen Gefühle{ / }Romain Rolland«.}}}\label{K_L03882-1}\pend
           
\pstart
           \begin{otherlanguage}{french}\textcolor{gray}{\textbf{\textcolor{blue}{R. SANTO}\pwindex{Santo, R. @\textsc{Santo, R.}, \emph{Hoteldirektor}|pw}{}\ledrightnote{\textcolor{blue}{R. Santo}}, DIR.}}\end{otherlanguage}\pend
           
\pstart
           \begin{otherlanguage}{french}\textcolor{gray}{\textbf{\textbf{\textcolor{pink}{GENÈVE-CHAMPEL}\oindex{Champel@\textbf{Champel}, \emph{Ehemaliger Ort}|pw}{}\ledrightnote{\textcolor{pink}{Champel}}}}}\end{otherlanguage}\pend
           
\pstart
           \begin{otherlanguage}{french}\textcolor{gray}{\textbf{Adr. Télégr.: \textcolor{pink}{Beauséjour, Genève-Champel}\oindex{Hôtel Beau-Séjour@\textbf{Hôtel Beau-Séjour}, \emph{Hotel}|pw}{}\ledrightnote{\textcolor{pink}{Hôtel Beau-Séjour}}}}\end{otherlanguage}\pend
           
\pstart
           \begin{otherlanguage}{french}\textcolor{gray}{\textbf{ÉTABLISSEMENT HYDROTHÉRAPIQUE}}\end{otherlanguage}\pend
           
\pstart
           \begin{otherlanguage}{french}\textcolor{gray}{\textbf{»\textcolor{pink}{CHAMPEL-LES-BAINS}\oindex{Champel@\textbf{Champel}, \emph{Ehemaliger Ort}|pw}{}\ledrightnote{\textcolor{pink}{Champel}}«}}\end{otherlanguage}\pend
           
\pstart
           \begin{otherlanguage}{french}\textcolor{gray}{\textbf{\textbf{\textcolor{pink}{GENÈVE}\oindex{Genf@\textbf{Genf}|pw}{}\ledrightnote{\textcolor{pink}{Genf}}}}}\end{otherlanguage}\pend
           
\pstart{}Cher Monsieur Arthur Schnitzler\pend\vspace{0.5em}
\pstart
           \begin{otherlanguage}{french}La traduction de votre \textcolor{green}{lettre
                     de protestation}\pwindex{Schnitzler, Arthur 15. 5. 1862 Wien – 21. 10. 1931 ebd.@\textsc{Schnitzler, Arthur} (15. 5. 1862 Wien – 21. 10. 1931 ebd.), \emph{Schriftsteller, Mediziner}!Une protestation d’Arthur Schnitzler@\strich\emph{Une protestation d’Arthur Schnitzler}|pw}\pwindex{Rolland, Romain 29.\,1.\,1866 Clamecy – 30.\,12.\,1944 Vézelay@\textsc{Rolland, Romain} (29.\,1.\,1866 Clamecy – 30.\,12.\,1944 Vézelay), \emph{Schriftsteller}!Une protestation d’Arthur Schnitzler@\strich\emph{Une protestation d’Arthur Schnitzler}|pw}{}\ledrightnote{\textcolor{green}{Une protestation d’Arthur Schnitzler}} paraîtra dans le \uline{\textcolor{green}{Journal de Genève}\pwindex{Journal de Genève@\emph{Journal de Genève}|pw}{}\ledrightnote{\textcolor{green}{Journal de Genève}}} de demain ou d’après-demain. Je vous ferai envoyer aussitôt le numéro. Je
                  l’ai adressé le second \textcolor{green}{exemplaire}\pwindex{Schnitzler, Arthur 15. 5. 1862 Wien – 21. 10. 1931 ebd.@\textsc{Schnitzler, Arthur} (15. 5. 1862 Wien – 21. 10. 1931 ebd.), \emph{Schriftsteller, Mediziner}!Brief Artur Schnitzlers@\strich\emph{Ein Brief Artur Schnitzlers}|pwv}{}\ledrightnote{{$\rightarrow$}\emph{\textcolor{green}{Ein Brief Artur Schnitzlers}}} à la \textcolor{brown}{\uline{Neue Zürcher Zeitung}}\orgindex{Neue Zürcher Zeitung@Neue Zürcher Zeitung|pw}{}\ledrightnote{\textcolor{brown}{Neue Zürcher Zeitung}}.\end{otherlanguage}\pend
           
\pstart
           \begin{otherlanguage}{french}Je suis heureux de pouvoir\end{otherlanguage}{ }\introOben{}\begin{otherlanguage}{french}vous\end{otherlanguage}\introOben{}{ }\begin{otherlanguage}{french}aider à faire entendre une voix
                  de justice et de raison, au milieu de ce délire de l’esprit \textcolor{pink}{européen}\oindex{Europa@\textbf{Europa}|pw}{}\ledrightnote{\textcolor{pink}{Europa}}. On dit que la pire chose a son utilité: je veux le
                  croire, en voyant que cette guerre abominable m’a procuré l’occasion d’entrer en
                  rapports amicaux avec {\pb}vous, et de
                  vous témoigner ma vive sympathie artistique pour votre œuvre si humaine, – théâtre
                  et romans\end{otherlanguage}\pend
           
\pstart
           \begin{otherlanguage}{french}Veuillez croire, cher Monsieur Dr. Schnitzler, à mes
                  sentiments tout dévoués\end{otherlanguage}{\\[\baselineskip]}\spacefill\mbox{Romain Rolland}\pend
           \leftskip=0em{}\selectlanguage{ngerman}\endnumbering\briefempfaengerindex{, @\textsc{, }!zzz, @\emph{von  }!1914-12-191@{19. 12. 1914}|)be}\mylabel{L03882h}
\begin{anhang}
\end{anhang}\normalsize

\doendnotes{C}
\bigskip
\vfill

\clearpage

\footnotesize

\lohead{\textsc{register}}

% Definiere theindex-Environment komplett neu ohne reledmac
\makeatletter
\renewenvironment{theindex}{%
  \section*{\indexname}%
  \setlength{\parindent}{0pt}%
  \setlength{\parskip}{0pt plus 0.3pt}%
  \let\item\@idxitem
}{%
  \clearpage
}
\makeatother

\IfFileExists{\jobname-pw.ind}{\input{\jobname-pw.ind}}{}

\end{document}

      