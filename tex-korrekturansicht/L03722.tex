%% latex-korrekturansicht-vorspann.tex
%% Vorspann für die Korrekturansicht.
%% Lädt die gemeinsame Datei latex-vorspann.tex mit gesetztem Schalter.

\newif\ifkorrekturansicht
\korrekturansichttrue

\input{../tex-inputs/latex-vorspann}


\section[Elsa Plessner an Arthur Schnitzler, 26. 1. 1899]{L03722 Elsa Plessner an Arthur Schnitzler, 26. 1. 1899}
\nopagebreak\mylabel{L03722v}
\rehead{ }\normalsize\beginnumbering\briefempfaengerindex{Schnitzler, Arthur@\textsc{Schnitzler, Arthur}!zzzPlessner, Elsa@\emph{von Elsa Plessner}!1899-01-261@{26. 1. 1899}|(be}
\toendnotes[C]{\smallbreak\pagebreak[2]}
\correspDesc{Versand  durch Elsa Plessner am 26. 1. 1899 in Wien
\newline{}Erhalt  durch Arthur Schnitzler im Zeitraum [26. 1. 1899
                  – 29. 1. 1899?] in Wien}\toendnotes[C]{\smallbreak}
\Standort{DLA, A:Schnitzler, HS.1985.1.419.}
\physDesc{Brief, 1 Blatt, 4 Seiten, 1904 Zeichen (Briefpapier mit Blumenmotiv (Iris) auf S. 1)
\newline{}Handschrift: schwarze Tinte, lateinische Kurrent}\toendnotes[C]{\smallbreak}
\pstart
           \centering{}{\pb}den 26. I. 99.\pend
           
\pstart{}Verehrter Herr Doctor!\pend\vspace{0.5em}
\pstart
           Ihre heutigen lieben \label{K_L03722-1v}\edtext{Zeilen}{\lemma{\textnormal{\emph{Zeilen}}}\Cendnote{\textnormal{nicht überliefert}}}\label{K_L03722-1} haben eine Scene
               verursacht, die ich ihrer Komik halber Ihnen schildern muss. Also stellen Sie sich
               vor – ich – im Bad, meine \textcolor{blue}{Schwester}\pwindex{Askonas, Johanna Leonie 20.\,11.\,1877 Wien – 30.\,7.\,1930 ebd.@\textsc{Askonas, Johanna Leonie} (20.\,11.\,1877 Wien – 30.\,7.\,1930 ebd.), \emph{Pensionsinhaberin}|pwv}{}\ledrightnote{{$\rightarrow$}\emph{\textcolor{blue}{Johanna Leonie Askonas}}} mit eiligem Schritt mir den \introOben{}(Ihren)\introOben{}{ }\label{K_L03722-2v}\edtext{Brief}{\lemma{\textnormal{\emph{Brief}}}\Cendnote{\textnormal{nicht überliefert}}}\label{K_L03722-2} überbringend. Ich – mit eiligst
               getrockneten \strikeout{l} aber noch immer feuchten Fingern das
               unheilvolle Couvert ergreifend und – na – sagen wir aufmachend – (die Stücke
               desselben habe ich nachher nicht mehr finden können) meine \textcolor{blue}{Schwester}\pwindex{Askonas, Johanna Leonie 20.\,11.\,1877 Wien – 30.\,7.\,1930 ebd.@\textsc{Askonas, Johanna Leonie} (20.\,11.\,1877 Wien – 30.\,7.\,1930 ebd.), \emph{Pensionsinhaberin}|pwv}{}\ledrightnote{{$\rightarrow$}\emph{\textcolor{blue}{Johanna Leonie Askonas}}} mir über die Schulter blickend
               – – und {\pb}und? – – – Und?! –\pend
           
\pstart
           Meine \textcolor{blue}{Schwester}\pwindex{Askonas, Johanna Leonie 20.\,11.\,1877 Wien – 30.\,7.\,1930 ebd.@\textsc{Askonas, Johanna Leonie} (20.\,11.\,1877 Wien – 30.\,7.\,1930 ebd.), \emph{Pensionsinhaberin}|pwv}{}\ledrightnote{{$\rightarrow$}\emph{\textcolor{blue}{Johanna Leonie Askonas}}} schreibt es
               der Wirkung – – d. h. dem Umstande zu, dass ich mich in der angeführten wässrigen
               Situation befand, dass ich nicht einen ordentlichen shoc davongetragen habe. – Sie
               hat mich hellauf ausgelacht (ich habe nämlich schändlich geheult) und mir zu bedenken
               gegeben, dass ich erspare, in die \textcolor{pink}{Donau}\oindex{Donau [Wien]@\textbf{Donau [Wien]}, \emph{Fluss}|pw}{}\ledrightnote{\textcolor{pink}{Donau [Wien]}} zu
               gehen, da ich mich ja ohnedies im Wasser befände. – – – –\pend
           
\pstart
           Nein! — Sie dürfen nicht glauben, dass ich schon so weit bin über meinen neuerlichen
               Missgriff lachen zu können! – Ich habe ja nicht {\pb}viel erwartet – aber so
               gar nichts? – Sie haben mir schon vor zwei Jahren klar gemacht, wie
               wenig an einem \label{K_L03722-3v}\edtext{verfehlten Stück}{\lemma{\textnormal{\emph{verfehlten Stück}}}\Cendnote{\textnormal{\textcolor{blue}{Elsa Plessner}\pwindex{Plessner, Elsa 22.\,8.\,1875 Wien – 7.\,5.\,1932 Alicante@\textsc{Plessner, Elsa} (22.\,8.\,1875 Wien – 7.\,5.\,1932 Alicante), \emph{Schriftstellerin}|pwk} legte \textcolor{blue}{Schnitzler} am 14. 3. 1896 das Schauspiel \emph{\textcolor{green}{Heimweh}\pwindex{Plessner, Elsa 22.\,8.\,1875 Wien – 7.\,5.\,1932 Alicante@\textsc{Plessner, Elsa} (22.\,8.\,1875 Wien – 7.\,5.\,1932 Alicante), \emph{Schriftstellerin}!Heimweh [dreiaktige Tragikomödie]@\strich\emph{Heimweh [dreiaktige Tragikomödie]}|pwk}} und am 29. 12. 1896{ }\emph{\textcolor{green}{Orchideen}\pwindex{Plessner, Elsa 22.\,8.\,1875 Wien – 7.\,5.\,1932 Alicante@\textsc{Plessner, Elsa} (22.\,8.\,1875 Wien – 7.\,5.\,1932 Alicante), \emph{Schriftstellerin}!Orchideen [Schauspiel in drei Akten]@\strich\emph{Orchideen [Schauspiel in drei Akten]}|pwk}} vor. Seine harte Kritik, besonders beim zweiten
                  Stück, löste Verzweiflung bei ihr aus, vgl. Elsa Plessner an Arthur Schnitzler, 13. 1. 1897.}}}\label{K_L03722-3} liegt! – Aber trotzdem! – Obzwar ich mit
               dem \textcolor{green}{Stück}\pwindex{Plessner, Elsa 22.\,8.\,1875 Wien – 7.\,5.\,1932 Alicante@\textsc{Plessner, Elsa} (22.\,8.\,1875 Wien – 7.\,5.\,1932 Alicante), \emph{Schriftstellerin}!Ehrlosen. Schauspiel in drei Acten@\strich\emph{Die Ehrlosen. Schauspiel in drei Acten}|pwv}{}\ledrightnote{{$\rightarrow$}\emph{\textcolor{green}{Die Ehrlosen. Schauspiel in drei Acten}}} nicht Literatur,
               sondern Geld machen wollte thut es mir doch so weh, wieder einmal etwas verhauen zu
               haben! – Es wundert mich aber, dass Sie gerade einen Satz herausgegriffen haben, der
               mir als Phrase nachträglich sehr missfallen hat. – – – Ja, ich habe immer Ideen und
               komme doch damit nicht weiter! – – – – Es ist wirk{\pb}lich schrecklich und
               fängt schon an, mich zu entmuthigen! – Wirklich!! Das soll keine Phrase sein! –\pend
           
\pstart
           Wenn ich nur wüsste, was ich da machen soll. Ich arbeite so intensiv ich kann (nicht
                  \uline{viel} – wie Sie glauben!) – (Im ganzen Jahr \uline{nur} den »\textcolor{green}{neuen
                     Lehrer}\pwindex{Plessner, Elsa 22.\,8.\,1875 Wien – 7.\,5.\,1932 Alicante@\textsc{Plessner, Elsa} (22.\,8.\,1875 Wien – 7.\,5.\,1932 Alicante), \emph{Schriftstellerin}!neue Lehrer. Novelle@\strich\emph{Der neue Lehrer. Novelle}|pw}{}\ledrightnote{\textcolor{green}{Der neue Lehrer. Novelle}}« und das \textcolor{green}{Stück}\pwindex{Plessner, Elsa 22.\,8.\,1875 Wien – 7.\,5.\,1932 Alicante@\textsc{Plessner, Elsa} (22.\,8.\,1875 Wien – 7.\,5.\,1932 Alicante), \emph{Schriftstellerin}!Ehrlosen. Schauspiel in drei Acten@\strich\emph{Die Ehrlosen. Schauspiel in drei Acten}|pwv}{}\ledrightnote{{$\rightarrow$}\emph{\textcolor{green}{Die Ehrlosen. Schauspiel in drei Acten}}}!) Ich sehe aber, das mir nichts nützt! Das Beste was ich kann ist doch
               nicht genug!\pend
           
\pstart
           Herzlichen Dank und herzlichen Gruß{\\[\baselineskip]}\spacefill\mbox{Elsa Plessner}\pend
           \leftskip=0em{}\selectlanguage{ngerman}\endnumbering\briefempfaengerindex{Schnitzler, Arthur@\textsc{Schnitzler, Arthur}!zzzPlessner, Elsa@\emph{von Elsa Plessner}!1899-01-261@{26. 1. 1899}|)be}\mylabel{L03722h}  \normalsize

\doendnotes{C}
\bigskip
\vfill

\clearpage

\footnotesize

\lohead{\textsc{register}}

% Definiere theindex-Environment komplett neu ohne reledmac
\makeatletter
\renewenvironment{theindex}{%
  \section*{\indexname}%
  \setlength{\parindent}{0pt}%
  \setlength{\parskip}{0pt plus 0.3pt}%
  \let\item\@idxitem
}{%
  \clearpage
}
\makeatother

\IfFileExists{\jobname-pw.ind}{\input{\jobname-pw.ind}}{}

\end{document}

      