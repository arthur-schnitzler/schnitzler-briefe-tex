%% latex-korrekturansicht-vorspann.tex
%% Vorspann für die Korrekturansicht.
%% Lädt die gemeinsame Datei latex-vorspann.tex mit gesetztem Schalter.

\newif\ifkorrekturansicht
\korrekturansichttrue

\input{../tex-inputs/latex-vorspann}


         
         \renewcommand{\erwaehntePersonen}{Personen: Erich Freund, Paul Schlenther}
         \renewcommand{\erwaehnteInstitutionen}{Institutionen: Burgtheater}
         \renewcommand{\erwaehnteOrte}{Orte: Berlin, Breslau, Lobe-Theater, Wien}
         \renewcommand{\erwaehnteWerke}{Werke: Der Schleier der Beatrice. Schauspiel in fünf Akten, Neue Freie Presse, Theater und Musik [Schleier der Beatrice], Theater- und Kunstnachrichten [Uraufführung von Schleier der Beatrice], Vossische Zeitung}
               \section[ Paul Goldmann an Arthur Schnitzler, 3. 12. {[}1900{]}]{Paul Goldmann an Arthur Schnitzler, 3. 12. {[}1900{]}}\nopagebreak\mylabel{v}\rehead{ }\normalsize\beginnumbering\briefempfaengerindex{Schnitzler, Arthur@\textsc{Schnitzler, Arthur}!zzzGoldmann, Paul@\emph{von Paul Goldmann}!1900-12-031@{3. 12. {[}1900{]}}|(be} \toendnotes[C]{\smallbreak\pagebreak[2]} \Standort{DLA, A:Schnitzler, HS.NZ85.1.3170.}
\physDesc{Brief, 1 Blatt, 1 Seite
\newline{}Handschrift: blaue Tinte, deutsche Kurrent\newline{}Beilage: ein Zeitungsartikel, beschnitten und aufgeklebt 
\newline{}Schnitzler: 1) mit Bleistift das Jahr »{[}1{]}900« vermerkt  2) mit rotem Buntstift eine seitliche Markierung neben der
                                 Begrüßungsformel}\toendnotes[C]{\smallbreak}\pstart
           \raggedleft{}{\pb}\textcolor{pink}{Berlin}{}\ledrightnote{\textcolor{pink}{Berlin}}, 3. December.\pend
           \pstart\center{}Mein lieber Freund,\pend\pstart
           Das \label{K_L02943-1v}\edtext{\textcolor{green}{Telegramm}{}\ledrightnote{{$\rightarrow$}\textcolor{green}{Theater- und Kunstnachrichten [Uraufführung von Schleier der Beatrice]}}}{\lemma{\textnormal{\emph{Telegramm}}}\Cendnote{\textnormal{[\textcolor{blue}{Erich Freund}]: \emph{\textcolor{green}{Theater- und Kunstnachrichten [Telegramm]}}. In: \emph{\textcolor{green}{Neue Freie Presse}}, Nr. 13031, 2. 12. 1900, S. 10. Siehe auch 9. 12. [1900].}}}\label{K_L02943-1h} des \textsc{Dr. \textcolor{blue}{Freund}{}\ledrightnote{\textcolor{blue}{Erich Freund}}} in der \textcolor{green}{N. Fr. Pr.}{}\ledrightnote{\textcolor{green}{Neue Freie Presse}} iſt blödſinng. Offenbar
               ſind auch Streichungen erfolgt. Beifolgender \label{K_L02943-2v}\edtext{\textcolor{green}{Ausſchnitt}{}\ledrightnote{{$\rightarrow$}\textcolor{green}{Theater und Musik [Schleier der Beatrice]}}}{\lemma{\textnormal{\emph{Ausſchnitt}}}\Cendnote{\textnormal{[O. V.]: \emph{\textcolor{green}{Theater und Musik}}. In: \emph{\textcolor{green}{Vossische Zeitung}}, Nr. 565, 3. 12. 1900, Abend-Ausgabe, S. [7].}}}\label{K_L02943-2h}
               iſt aus der \textcolor{green}{Voſſiſchen Zeitung}{}\ledrightnote{\textcolor{green}{Vossische Zeitung}}. Viele Grüße!\pend
           \pstart
           Dein {\\[\baselineskip]}\spacefill\mbox{Paul Goldmann.}\pend
           \leftskip=0em{}\pstart
           \textcolor{gray}{\textbf{– Das neue \textcolor{green}{Drama}{}\ledrightnote{{$\rightarrow$}\textcolor{green}{Der Schleier der Beatrice. Schauspiel in fünf Akten}} von \textbf{Arthur Schnitzler, »\textcolor{green}{Der Schleier der Beatrice}{}\ledrightnote{\textcolor{green}{Der Schleier der Beatrice. Schauspiel in fünf Akten}}«,} deſſen Einreichung
                  beim \textcolor{brown}{Hofburgtheater}{}\ledrightnote{\textcolor{brown}{Burgtheater}} im letzten Frühjahr zu
                  einem Konflikt des Dichters mit Direktor \textsc{Dr.}{ }\textcolor{blue}{Schlenther}{}\ledrightnote{\textcolor{blue}{Paul Schlenther}} Anlaß gegeben hat, wurde am
                     Sonnabend im \textcolor{pink}{\so{Lobe-Theater}}{}\ledrightnote{\textcolor{pink}{Lobe-Theater}} zu \textcolor{pink}{\so{Breslau}}{}\ledrightnote{\textcolor{pink}{Breslau}} zum erſten Male aufgeführt. Der äußere Erfolg des \textcolor{green}{Stück}{}\ledrightnote{{$\rightarrow$}\textcolor{green}{Der Schleier der Beatrice. Schauspiel in fünf Akten}}es wurde durch die wenig gute
                  Aufführung ſtark beeinträchtigt. Das \textcolor{green}{Stück}{}\ledrightnote{{$\rightarrow$}\textcolor{green}{Der Schleier der Beatrice. Schauspiel in fünf Akten}} ſelbſt erzielte bei ausverkauftem \textcolor{pink}{Hauſe}{}\ledrightnote{{$\rightarrow$}\textcolor{pink}{Lobe-Theater}} eine große Wirkung. Man meldet uns
                  darüber aus \textcolor{pink}{Breslau}{}\ledrightnote{\textcolor{pink}{Breslau}}: »Schnitzlers \textcolor{green}{Stück}{}\ledrightnote{{$\rightarrow$}\textcolor{green}{Der Schleier der Beatrice. Schauspiel in fünf Akten}} iſt ein
                  farbenglühendes Gemälde aus der Hochrenaiſſancezeit und faßt die Tragik zweier
                  hochgeſtimmter Charaktere in der unbewußten Tragödie einer Mädchenſeele zuſammen.
                  Das \textcolor{green}{Stück}{}\ledrightnote{{$\rightarrow$}\textcolor{green}{Der Schleier der Beatrice. Schauspiel in fünf Akten}} ſteigert ſich in
                  der dramatiſchen Wirkung von Akt zu Akt, und das ſichtlich lebhaft intereſſirte
                  Publikum bereitete dem anweſenden Dichter einen ſich ſtetig ſteigernden großen
                  Erfolg.«}}\pend
           \endnumbering\briefempfaengerindex{Schnitzler, Arthur@\textsc{Schnitzler, Arthur}!zzzGoldmann, Paul@\emph{von Paul Goldmann}!1900-12-031@{3. 12. {[}1900{]}}|)be}\mylabel{h}  \normalsize

\doendnotes{C}
\bigskip
\vfill

\clearpage

\footnotesize

\lohead{\textsc{register}}

% Definiere theindex-Environment komplett neu ohne reledmac
\makeatletter
\renewenvironment{theindex}{%
  \section*{\indexname}%
  \setlength{\parindent}{0pt}%
  \setlength{\parskip}{0pt plus 0.3pt}%
  \let\item\@idxitem
}{%
  \clearpage
}
\makeatother

\IfFileExists{\jobname-pw.ind}{\input{\jobname-pw.ind}}{}

\end{document}

      