%% latex-korrekturansicht-vorspann.tex
%% Vorspann für die Korrekturansicht.
%% Lädt die gemeinsame Datei latex-vorspann.tex mit gesetztem Schalter.

\newif\ifkorrekturansicht
\korrekturansichttrue

\input{../tex-inputs/latex-vorspann}


         
         \renewcommand{\erwaehntePersonen}{Personen: Georg Hirschfeld, Alfred Kerr}
         \renewcommand{\erwaehnteInstitutionen}{Institutionen: Berliner Theater, Schauspielhaus Berlin}
         \renewcommand{\erwaehnteOrte}{Orte: Berlin, China, Dessauer Straße, Innsbruck, Lago di Garda, Reichenau an der Rax, Riva del Garda}
         \renewcommand{\erwaehnteWerke}{Werke: Der Schleier der Beatrice. Schauspiel in fünf Akten}
               \section[ Paul Goldmann an Arthur Schnitzler, 18. 7. {[}1900{]}]{Paul Goldmann an Arthur Schnitzler, 18. 7. {[}1900{]}}\nopagebreak\mylabel{v}\rehead{ }\normalsize\beginnumbering\briefempfaengerindex{Schnitzler, Arthur@\textsc{Schnitzler, Arthur}!zzzGoldmann, Paul@\emph{von Paul Goldmann}!1900-07-182@{18. 7. {[}1900{]}}|(be} \toendnotes[C]{\smallbreak\pagebreak[2]} \Standort{DLA, A:Schnitzler, HS.NZ85.1.3170.}
\physDesc{Brief, 1 Blatt, 3 Seiten
\newline{}Handschrift: blaue Tinte, deutsche Kurrent
\newline{}Schnitzler: 1) mit Bleistift das Jahr »{[}1{]}90\textcolor{gray}{0}« vermerkt  2) mit rotem Buntstift zwei Unterstreichungen}\toendnotes[C]{\smallbreak}\pstart
           \noindent{}\raggedleft{}{\pb}\textcolor{pink}{\textcolor{gray}{\textbf{DESSAUERSTRASSE 19}}}{}\ledrightnote{\textcolor{pink}{Dessauer Straße}}\pend
           \pstart
           \textcolor{pink}{Berlin}{}\ledrightnote{\textcolor{pink}{Berlin}}, 18. Juli.\pend
           \pstart\center{}Mein lieber Freund,\pend\pstart
           Mit der \label{K_L02924-1v}\edtext{Fußparthie}{\lemma{\textnormal{\emph{Fußparthie}}}\Cendnote{\textnormal{siehe Paul Goldmann an Arthur Schnitzler, 16. 6. [1900]}}}\label{K_L02924-1h}, wie Du ſie entworfen haſt, und mit dem \label{K_L02924-2v}\edtext{Zuſammentreffen in \textsc{\textcolor{pink}{Innsbruck}{}\ledrightnote{\textcolor{pink}{Innsbruck}}}}{\lemma{\textnormal{\emph{Zuſammentreffen in Innsbruck}}}\Cendnote{\textnormal{siehe A. S.: \emph{Tagebuch}, 16. 8. 1900}}}\label{K_L02924-2h} bin ich einverſtanden, – vorausgeſetzt, daß ich überhaupt fortkomme, was
               durch die \label{K_L02924-3v}\edtext{\textcolor{pink}{chin}{}\ledrightnote{{$\rightarrow$}\textcolor{pink}{China}}eſiſchen Ereigniſſe}{\lemma{\textnormal{\emph{chineſiſchen Ereigniſſe}}}\Cendnote{\textnormal{siehe Paul Goldmann an Arthur Schnitzler, 5. 7. [1900]}}}\label{K_L02924-3h} immer fraglicher wird. Ich habe noch nicht einmal um Urlaub geſchrieben.
               Immerhin hoffe ich, zum 15. Auguſt fortzukommen. Laß’
               mich Deine Adreſſe wiſſen, damit ich Dir das Nähere telegraphiſch oder brieflich
               mittheilen kann.\pend
           \pstart
           Von \textsc{\textcolor{blue}{Kerr}{}\ledrightnote{\textcolor{blue}{Alfred Kerr}}} hatte ich heut eine Karte mit der Bitte, ihm
               nach \textsc{\textcolor{pink}{Riva}{}\ledrightnote{\textcolor{pink}{Riva del Garda}}} (\textcolor{pink}{Gardaſee}{}\ledrightnote{\textcolor{pink}{Lago di Garda}}) zu {\pb}ſchreiben. Er ſagt, er erwarte von Dir Nachricht,
               und wird jedenfalls pünktlich beim 
                  \textsc{Rendezvous}
                in \textsc{\textcolor{pink}{Innsbruck}{}\ledrightnote{\textcolor{pink}{Innsbruck}}} ſein. \strikeout{\textcolor{gray}{×}\-\textcolor{gray}{×}\-\textcolor{gray}{×}\-\textcolor{gray}{×}\-\textcolor{gray}{×}\-\textcolor{gray}{×}\-\textcolor{gray}{×}\-\textcolor{gray}{×}\-\textcolor{gray}{×}\-\textcolor{gray}{×}\-\textcolor{gray}{×}\-\textcolor{gray}{×}\-\textcolor{gray}{×}\-\textcolor{gray}{×}\-\textcolor{gray}{×}\-\textcolor{gray}{×}} Bitte, ſchreib’ ihm ſofort.\pend
           \pstart
           Daß \label{K_L02924-4v}\edtext{\textsc{\textcolor{blue}{Hirschfeld}{}\ledrightnote{\textcolor{blue}{Georg Hirschfeld}}} mitgeht}{\lemma{\textnormal{\emph{Hirschfeld mitgeht}}}\Cendnote{\textnormal{nicht geschehen}}}\label{K_L02924-4h}, iſt
               mir nicht ſympathiſch. Er ſoll doch lieber zu Hauſe bleiben und \label{K_L02924-5v}\edtext{»\textsc{Milieu}-Stücke«}{\lemma{\textnormal{\emph{»Milieu-Stücke«}}}\Cendnote{\textnormal{siehe Paul Goldmann an Arthur Schnitzler, 21. 6. [1900]}}}\label{K_L02924-5h} ſchreiben.\pend
           \pstart
           Wenn das \label{K_L02924-6v}\edtext{\textcolor{brown}{Schauſpielhaus}{}\ledrightnote{\textcolor{brown}{Schauspielhaus Berlin}} Dein \textcolor{green}{Stück}{}\ledrightnote{{$\rightarrow$}\textcolor{green}{Der Schleier der Beatrice. Schauspiel in fünf Akten}} refüſiren}{\lemma{\textnormal{\emph{Schauſpielhaus … refüſiren}}}\Cendnote{\textnormal{vgl. Paul Goldmann an Arthur Schnitzler, 5. 7. [1900]}}}\label{K_L02924-6h} ſollte, was noch gar nicht ausgemacht iſt, ſo verſuchen wir es beim \textcolor{brown}{Berliner Theater}{}\ledrightnote{\textcolor{brown}{Berliner Theater}}, wo ich die Annahme für ſicher
               halte.\pend
           \pstart
           {\pb}Für heut nur dieſes
               Wenige. Ich habe unmenſchlich viel zu thun.\pend
           \pstart
           Viele treue Grüße! {\\[\baselineskip]}Dein {\\[\baselineskip]}\spacefill\mbox{Paul Goldmann.}\pend
           \leftskip=0em{}\endnumbering\briefempfaengerindex{Schnitzler, Arthur@\textsc{Schnitzler, Arthur}!zzzGoldmann, Paul@\emph{von Paul Goldmann}!1900-07-182@{18. 7. {[}1900{]}}|)be}\mylabel{h}\begin{anhang}\end{anhang}\normalsize

\doendnotes{C}
\bigskip
\vfill

\clearpage

\footnotesize

\lohead{\textsc{register}}

% Definiere theindex-Environment komplett neu ohne reledmac
\makeatletter
\renewenvironment{theindex}{%
  \section*{\indexname}%
  \setlength{\parindent}{0pt}%
  \setlength{\parskip}{0pt plus 0.3pt}%
  \let\item\@idxitem
}{%
  \clearpage
}
\makeatother

\IfFileExists{\jobname-pw.ind}{\input{\jobname-pw.ind}}{}

\end{document}

      