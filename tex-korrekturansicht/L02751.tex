%% latex-korrekturansicht-vorspann.tex
%% Vorspann für die Korrekturansicht.
%% Lädt die gemeinsame Datei latex-vorspann.tex mit gesetztem Schalter.

\newif\ifkorrekturansicht
\korrekturansichttrue

\input{../tex-inputs/latex-vorspann}


               \section[Paul Goldmann an Arthur Schnitzler, 13. 10. {[}1895{]}]{ Paul Goldmann an Arthur Schnitzler, 13. 10. {[}1895{]}}\nopagebreak\mylabel{v}\rehead{ }\normalsize\beginnumbering\briefempfaengerindex{Schnitzler, Arthur@\textsc{Schnitzler, Arthur}!zzzGoldmann, Paul@\emph{von Paul Goldmann}!1895-10-131@{13. 10. {[}1895{]}}|(be} \toendnotes[C]{\smallbreak\pagebreak[2]} \Standort{DLA, A:Schnitzler, HS.NZ85.1.3165.}
\physDesc{Brief, 3 Blätter, 11 Seiten
\newline{}Handschrift: blaue Tinte, deutsche Kurrent
\newline{}Schnitzler: 1) mit Bleistift das Jahr »95« vermerkt 2) mit rotem Buntstift eine seitliche Markierung und sieben
                                 Unterstreichungen}\toendnotes[C]{\smallbreak}\pstart
           \noindent{}{\pb}\textcolor{gray}{\textbf{\textbf{\textcolor{brown}{Frankfurter Zeitung}{}\ledrightnote{\textcolor{brown}{Frankfurter Zeitung}}}}}\pend
           \pstart
           \textcolor{gray}{\textbf{(\textcolor{brown}{\begin{otherlanguage}{french}Gazette de Francfort\end{otherlanguage}}{}\ledrightnote{\textcolor{brown}{Frankfurter Zeitung}}). }}\pend
           \pstart
           \textcolor{gray}{\textbf{\textbf{\begin{otherlanguage}{french}Fondateur M. \textcolor{blue}{L.
                              Sonnemann}{}\ledrightnote{\textcolor{blue}{Leopold Sonnemann}}\end{otherlanguage}.}}}\pend
           \pstart
           \begin{otherlanguage}{french}\textcolor{gray}{\textbf{\textcolor{green}{Journal}{}\ledrightnote{→\textcolor{green}{Frankfurter Zeitung}} politique,
                        financier,}}\end{otherlanguage}\pend
           \pstart
           \begin{otherlanguage}{french}\textcolor{gray}{\textbf{commercial et littéraire.}}\end{otherlanguage}\hfill \textsc{\textcolor{pink}{Paris}{}\ledrightnote{\textcolor{pink}{Paris}}}, 13. October.\pend
           \pstart
           \begin{otherlanguage}{french}\textcolor{gray}{\textbf{\textbf{Paraissant trois fois par jour.}}}\end{otherlanguage}\pend
           \pstart
           \begin{otherlanguage}{french}\textcolor{gray}{\textbf{\textbf{Bureau à \textcolor{pink}{Paris}{}\ledrightnote{\textcolor{pink}{Paris}}:}}}\end{otherlanguage}\pend
           \pstart
           \begin{otherlanguage}{french}\textcolor{gray}{\textbf{\textbf{\textcolor{pink}{24. Rue Feydeau}{}\ledrightnote{\textcolor{pink}{rue Feydeau}}.}}}\end{otherlanguage}\pend
           \pstart\center{}Mein lieber Freund,\pend\pstart
           Nochmals innigen Glückwunſch!\pend
           \pstart
           Jetzt, nachdem ich einige Referate geleſen, ſehe ich \strikeout{\textcolor{gray}{w}} erſt, wie groß Dein Erfolg iſt, was aus Deiner Depeſche nicht klar genug
               hervorging. Wie ich die Sache anſehe, biſt Du jetzt \label{K_L02751-1v}\edtext{lancirt}{\lemma{\textnormal{\emph{lancirt}}}\Cendnote{\textnormal{im Sinne
                  von: in der Öffentlichkeit bekannt}}}\label{K_L02751-1h}. Nach dem \textcolor{pink}{Wien}{}\ledrightnote{\textcolor{pink}{Wien}}er Erfolge werden die \textcolor{pink}{Berlin}{}\ledrightnote{\textcolor{pink}{Berlin}}er bald
               mit dem \textcolor{green}{Stücke}{}\ledrightnote{→\textcolor{green}{Liebelei. Schauspiel in drei Akten}} herauskommen.
               Dort wird es einen nicht minder großen Erfolg haben und eine noch intelligentere
               Kritik finden (\label{K_L02751-2v}\edtext{\textsc{\textcolor{blue}{Mauthner}{}\ledrightnote{\textcolor{blue}{Fritz Mauthner}}} im »\textcolor{brown}{Tageblatt}{}\ledrightnote{\textcolor{brown}{Berliner Tageblatt}}«}{\lemma{\textnormal{\emph{Mauthner im »Tageblatt«}}}\Cendnote{\textnormal{\textcolor{blue}{Fr. M.} [=\textcolor{blue}{Fritz Mauthner}]: \emph{\textcolor{green}{Deutsches Theater}}. In: \emph{\textcolor{green}{Berliner Tageblatt}}, Jg. 25, Nr. 64, 5. 2. 1896, Morgen-Ausgabe, S. 2–3, \textcolor{blue}{Fritz Mauthner}: \emph{\textcolor{green}{Der zerbrochene Krug im Deutschen Theater}}. In: \emph{\textcolor{green}{Berliner Tageblatt}}, Jg. 25, Nr. 65, 5. 2. 1896, Abend-Ausgabe, S. 1–2.}}}\label{K_L02751-2h}).
               Dann wird {\pb}es über alle \textcolor{pink}{deutſch}{}\ledrightnote{→\textcolor{pink}{Deutschland}}en Bühnen gehen. Wenn Du ruhig ſo
               weiter arbeiteſt – und ich weiß, Du wirſt es thun – kann am Ende ein deutſcher \textsc{\textcolor{blue}{Emile Augier}{}\ledrightnote{\textcolor{blue}{Émile Augier}}} daraus werden. Der erſte entſcheidende Schritt auf dieſem Wege iſt gethan, und
               ich bin recht glücklich darüber, daß Dich gleich zu Anfang der Erfolg \strikeout{\textcolor{gray}{in die Hand}} an der Hand nimmt; das iſt ein guter Führer. Wenn ich übrigens »\textsc{\textcolor{blue}{Émile Augier}{}\ledrightnote{\textcolor{blue}{Émile Augier}}}« ſage, ſo gilt dies nur einſtweilen, und ich behalte mir vor, im Laufe der
               Zeit, je nachdem die Dinge ſich entwickeln, {\pb}noch
               viel unbeſcheidener zu werden. Immerhin bedenke nur: In ſo jungen Jahren am erſten
                  \textcolor{pink}{deutſch}{}\ledrightnote{→\textcolor{pink}{Deutschland}}en \textcolor{brown}{Theater}{}\ledrightnote{→\textcolor{brown}{Burgtheater}} mit dem zweiten \textcolor{green}{Stücke}{}\ledrightnote{→\textcolor{green}{Liebelei. Schauspiel in drei Akten}} ein von allen \strikeout{ernſtz\textcolor{gray}{u}} ernſtzunehmenden Leuten laut anerkannter Erfolg! Das iſt etwas, was Du in der
                  \textcolor{pink}{deutſch}{}\ledrightnote{→\textcolor{pink}{Deutschland}}en Bühnengeſchichte
               ſelten finden dürfteſt. Es ſcheint wirklich, daß Du zu ſchönen Hoffnungen für die
               Zukunft berechtigſt, wie \label{K_L02751-77v}\edtext{\textcolor{blue}{einer}{}\ledrightnote{→\textcolor{blue}{Max Kalbeck}} der weiſen Männer}{\lemma{\textnormal{\emph{einer der weiſen Männer}}}\Cendnote{\textnormal{Von \textcolor{blue}{Max
                     Kalbeck} erschien ein Feuilleton und eine Nachtkritik, wobei sich die
                  erwähnte Aussage in der Nachtkritik findet. \textcolor{blue}{Max Kalbeck}: \emph{\textcolor{green}{Burgtheater. »Liebelei«, Schauspiel in drei Acten von Arthur
                        Schnitzler. – »Rechte der Seele«, Schauspiel in einem Acte von Guiseppe
                        Giacosa; deutsch von Otto Eisenschitz}}. In: \emph{\textcolor{green}{Neues Wiener Tagblatt}}, Jg. 29, Nr. 279, 11. 10. 1895, S. 1–3. \textcolor{blue}{M. K.} [=\textcolor{blue}{Max Kalbeck}]: \emph{\textcolor{green}{Theater,
                        Kunst und Literatur. Burgtheater}}. In: \emph{\textcolor{green}{Neues Wiener Tagblatt}}, Jg. 29, Nr. 278, 10. 10. 1895, S. 7.}}}\label{K_L02751-77h} ſich ausdrückte, die über Dein \textcolor{green}{Stück}{}\ledrightnote{→\textcolor{green}{Liebelei. Schauspiel in drei Akten}} geſchrieben haben.\pend
           \pstart
           {\pb}Ich habe geleſen die Referate von: \label{K_L02751-888v}\edtext{\textsc{\textcolor{green}{\textcolor{blue}{Speidel}{}\ledrightnote{\textcolor{blue}{Ludwig Speidel}}}{}\ledrightnote{→\textcolor{green}{Theater- und Kunstnachrichten. [Burgtheater] [Liebelei, Rechte der Seele]}}}}{\lemma{\textnormal{\emph{Speidel}}}\Cendnote{\textnormal{[\textcolor{blue}{Ludwig Speidel}]: \emph{\textcolor{green}{Theater- und Kunstnachrichten. [Burgtheater]}}. In: \emph{\textcolor{green}{Neue Freie Presse}}, Nr. 11.181, 10. 10. 1895, S. 7. Ein weiteres Feuilleton
                  erschien am Tag dieses Briefes und war \textcolor{blue}{Goldmann} zu diesem Zeitpunkt noch unbekannt: \textcolor{blue}{L. Sp.} [=\textcolor{blue}{Ludwig Speidel}]: \emph{\textcolor{green}{Burgtheater. (»Liebelei«, Schauspiel in drei Aufzügen von
                        Arthur Schnitzler. – »Rechte der Seele«, Schauspiel in einem Act von
                        Giuseppe Giacosa, deutsch von Otto Eisenschitz.)}}. In: \emph{\textcolor{green}{Neue Freie Presse}}, Nr. 11.184, 13. 10. 1895, Morgenblatt, S. 1–3.}}}\label{K_L02751-888h} (prachtvoll), \textsc{\textcolor{green}{\textcolor{blue}{Kalbeck}{}\ledrightnote{\textcolor{blue}{Max Kalbeck}}}{}\ledrightnote{→\textcolor{green}{Theater, Kunst und Literatur. Burgtheater [Liebelei, Rechte der Seele]}{\newline}→\textcolor{green}{Burgtheater. »Liebelei«, Schauspiel in drei Acten von Arthur Schnitzler. – »Rechte der Seele«, Schauspiel in einem Acte von Guiseppe Giacosa; deutsch von Otto Eisenschitz}}} (die erſten sympathiſchen Zeilen, die ich von dem \textcolor{blue}{Manne}{}\ledrightnote{→\textcolor{blue}{Max Kalbeck}} leſe), \label{K_L02751-22v}\edtext{\textsc{\textcolor{green}{\textcolor{blue}{Schoenthan}{}\ledrightnote{\textcolor{blue}{Paul von Schönthan-Pernwald}}}{}\ledrightnote{→\textcolor{green}{Theater, Kunst und Literatur. (Burgtheater.) [Liebelei]}}}}{\lemma{\textnormal{\emph{Schoenthan}}}\Cendnote{\textnormal{\textcolor{blue}{p. v. s.} [=\textcolor{blue}{Paul von Schönthan-Pernwald}]: \emph{\textcolor{green}{Theater, Kunst und Literatur.
                        (Burgtheater.)}}. In: \emph{\textcolor{green}{Wiener
                        Tagblatt}}, Jg. XXXX, Nr. XXXX, 10. 10. 1895, S. XXXX.}}}\label{K_L02751-22h} (der vor Bühnendichter-Neid
               zerſpringt); ferner das \label{K_L02751-33v}\edtext{\textcolor{green}{Referat}{}\ledrightnote{→\textcolor{green}{Theater und Kunst. (Burgtheater.) [Liebelei, Rechte der Seele]}} des »\textcolor{green}{Wiener Journal}{}\ledrightnote{\textcolor{green}{Wiener Tagblatt}}}{\lemma{\textnormal{\emph{Referat … Journal}}}\Cendnote{\textnormal{\textcolor{blue}{–v–} [=\textcolor{blue}{Jakob Julius David}]: \emph{\textcolor{green}{Theater und Kunst. (Burgtheater.)}}. In: \emph{\textcolor{green}{Neues Wiener Journal}}, Jg. 3, Nr. 704, 10. 10. 1895, S. 5.}}}\label{K_L02751-33h}« (verſtändnißlos, aber mit
               Einzelheiten, die ausſöhnen), endlich \label{K_L02751-44v}\edtext{\textsc{\textcolor{green}{\textcolor{blue}{Granichstaedten}{}\ledrightnote{\textcolor{blue}{Emil Granichstaedten}}}{}\ledrightnote{→\textcolor{green}{Burgtheater. Zwei Schauspiele: »Rechte der Seele« von Giuseppe Giacosa. – »Liebelei« von Arthur Schnitzler}}}}{\lemma{\textnormal{\emph{Granichstaedten}}}\Cendnote{\textnormal{\textcolor{blue}{Emil Granichstaedten}: \emph{\textcolor{green}{Feuilleton. Burgtheater}}. In: \emph{\textcolor{green}{Die Presse}}, Jg. 48, Nr. 279, 11. 10. 1895, S. 1–2.}}}\label{K_L02751-44h}, das widerliche Thier
               (Ohrfeigen!!!). \label{K_L02751-66v}\edtext{\textsc{\textcolor{green}{\textcolor{blue}{Uhl}{}\ledrightnote{\textcolor{blue}{Friedrich Uhl}}}{}\ledrightnote{→\textcolor{green}{Wiener Brief. [Liebelei, Rechte der Seele]}}} in der »\textcolor{green}{Frankfurter Zeitung}{}\ledrightnote{\textcolor{green}{Frankfurter Zeitung}}}{\lemma{\textnormal{\emph{Uhl … Zeitung}}}\Cendnote{\textnormal{[\textcolor{blue}{Friedrich Uhl}]: \emph{\textcolor{green}{XXXX}}. In: \emph{\textcolor{green}{Frankfurter
                        Zeitung}}, Jg. 39, Nr. XXXX, XXXX, S. XXXX}}}\label{K_L02751-66h}« hätte wärmer und ausführlicher ſein können; ich vermuthe, daß es ihn {\pb}verſtimmt, weil die Officiellen (\textsc{\textcolor{blue}{Speidel}{}\ledrightnote{\textcolor{blue}{Ludwig Speidel}} etc.}) Dich loben. Auch iſt er
               wohl von Denen, die jemanden fördern, – bis er einen Erfolg hat, die aber ſofort von
               dem Erfolge ſelbſt unſympathiſch berührt werden. Eine echte Oppoſitions-Natur mit
               einem Worte. In \strikeout{B\textcolor{gray}{e}}{ }\textcolor{pink}{Berlin}{}\ledrightnote{\textcolor{pink}{Berlin}}er Blättern las ich das kurze, aber ſehr
               freundliche \label{K_L02751-777v}\edtext{\textcolor{green}{Telegramm}{}\ledrightnote{→\textcolor{green}{[Aus Wien, 9. Oktober] [Liebelei]}} des »\textcolor{green}{Tageblatt}{}\ledrightnote{\textcolor{green}{Berliner Tageblatt}}}{\lemma{\textnormal{\emph{Telegramm des »Tageblatt}}}\Cendnote{\textnormal{[O. V.]: \emph{\textcolor{green}{[Aus Wien, 9. Oktober]}}. In:
                        \emph{\textcolor{green}{Berliner Tageblatt}}, Jg. 24, Nr. 516,
                        10. 10. 1895, Abend-Ausgabe,
                  S. 3.}}}\label{K_L02751-777h}«, das ſehr warme \label{K_L02751-7777v}\edtext{\textcolor{green}{Telegramm}{}\ledrightnote{→\textcolor{green}{»Liebelei« [Telegramm zur Uraufführung]}} des »\textcolor{green}{Lokalanzeiger}{}\ledrightnote{\textcolor{green}{Berliner Lokal-Anzeiger}}}{\lemma{\textnormal{\emph{Telegramm des »Lokalanzeiger}}}\Cendnote{\textnormal{»›\textcolor{green}{Liebelei}‹, ein Drama eines jungen
                  \textcolor{pink}{Wien}er Schriftstellers, ist gestern (Mittwoch) Abend im \textcolor{brown}{Wiener Burgtheater} zum 
                  ersten Male aufgeführt worden; wir erhalten darüber folgendes \so{Privat-Telegramm}:{ / }\textcolor{pink}{\textbf{Wien}}, 9. October, 11 Uhr 50 Min. Abends (Von
                  unserem \textsc{\textcolor{gray}{n}.a.}-Correspondenten.){ / }Das bürgerliche Drama ›\textcolor{green}{Liebelei}‹ von \textcolor{blue}{Arthur Schnitzler}
                     hatte heute im \textcolor{brown}{Burgtheater} einen bedeutenden Erfolg. Der Verfasser
                  wurde nach jedem Akt wiederholt gerufen, obwohl in dem Stück sociale Verhältnisse behandelt
                  werden, die auf dem Hoftheater sonst Befremden erregen. Das Bürgermädchen, das an einer Liebelei zu
                  Grunde geht, wurde von der \textcolor{blue}{Sandrock} mit tragischem 
                     Nachdruck gespielt, ergreifend war auch \textcolor{blue}{Sonnenthal}
                  als ihr Vater.« (\emph{\textcolor{green}{Berliner Lokal-Anzeiger}}, Jg. 13, Nr. 475,
                        10. 10. 1895, Morgenblatt, 1. Ausgabe, S. 3.)}}}\label{K_L02751-7777h}« und {\pb}das blödſinnig-freche \label{K_L02751-88v}\edtext{\textcolor{green}{Telegramm}{}\ledrightnote{→\textcolor{green}{[Wien, 9. Oktober] [Liebelei]}} des »\textcolor{green}{Kleinen Journal}{}\ledrightnote{\textcolor{green}{Das Kleine Journal}}}{\lemma{\textnormal{\emph{Telegramm … Journal}}}\Cendnote{\textnormal{ [\textcolor{blue}{Julius Konried}]: \emph{\textcolor{green}{[Wien, 9. Oktober]}}. In: \emph{\textcolor{green}{Das Kleine Journal}}, Jg. 18, Nr. XXXX, 10. 10. 1895, S. XXXX.}}}\label{K_L02751-88h}« (\textcolor{blue}{Correſpondent}{}\ledrightnote{→\textcolor{blue}{Julius Konried}} Herr \textsc{\textcolor{blue}{Conried}{}\ledrightnote{\textcolor{blue}{Julius Konried}}} vom »\textcolor{green}{Neuen Wiener Tagblatt}{}\ledrightnote{\textcolor{green}{Neues Wiener Tagblatt}}«), das Dich
               einen Mann aus der \textsc{\textcolor{blue}{Hermann Bahr}{}\ledrightnote{\textcolor{blue}{Hermann Bahr}}schen} Schule nennt.\pend
           \pstart
           Den Abend der \textsc{Première} verbrachte ich mit \textsc{\textcolor{blue}{Th. Wolff}{}\ledrightnote{\textcolor{blue}{Theodor Wolff}}} (vom »\textcolor{green}{Berliner Tageblatt}{}\ledrightnote{\textcolor{green}{Berliner Tageblatt}}«) und ſah fleißig
               auf die Uhr. Um neun Uhr meinte ich, Dein Schickfal müſſe ſich wohl
               entſchieden haben, und da ſchlug \textsc{\textcolor{blue}{Wolff}{}\ledrightnote{\textcolor{blue}{Theodor Wolff}}} vor, auf Dein {\pb}Wohl anzuſtoßen, was
               geſchah.\pend
           \pstart
           Die Meinigen, mein \textcolor{blue}{Onkel}{}\ledrightnote{→\textcolor{blue}{Fedor Mamroth}},
               meine \textcolor{blue}{Mutter}{}\ledrightnote{→\textcolor{blue}{Clementine Goldmann}}, mein \textcolor{blue}{Schwager}{}\ledrightnote{→\textcolor{blue}{Josef Rosengart}}, ſind, wie mir heut meine \textcolor{blue}{Mutter}{}\ledrightnote{→\textcolor{blue}{Clementine Goldmann}} ſchreibt, hocherfreut über Deinen Erfolg und laſſen
               Dir von Herzen gratuliren.\pend
           \pstart
           Am Tag nach der \begin{otherlanguage}{french}\textsc{Première}\end{otherlanguage}, nachdem ich Dein Telegramm erhalten, fuhr ich zur »\textsc{\textcolor{brown}{Liberté}{}\ledrightnote{\textcolor{brown}{La Liberté}}}« und zu den »\textsc{\textcolor{brown}{Débats}{}\ledrightnote{\textcolor{brown}{Journal des débats}}}« und bat um eine \label{K_L02751-4v}\edtext{\textcolor{green}{Notiz}{}\ledrightnote{→\textcolor{green}{Thêatres. [Notre correspondant de Vienne]}{\newline}→\textcolor{green}{Courrier des Théatres [Liebelei]}}}{\lemma{\textnormal{\emph{Notiz}}}\Cendnote{\textnormal{[\textcolor{blue}{Georges Aubry}]: \emph{\textcolor{green}{Thêatres. [Notre correspondant de Vienne]}}. In: \emph{\textcolor{green}{La Liberté}}, Jg. 30, Nr. 11.289, 12. 10. 1895, S. 3. Siehe dazu auch Paul Goldmann an Arthur Schnitzler, 7. 10. [1895].  [\textcolor{blue}{Hippolyte Fierens-Gevaert}]: \emph{\textcolor{green}{Courrier
                        des Théatres}}. In: \emph{\textcolor{green}{Journal des débats
                        politiques et littéraires}}, Jg. 107, 12. 10. 1895, S. 3.}}}\label{K_L02751-4h}. Beide \textcolor{brown}{Blätter}{}\ledrightnote{→\textcolor{brown}{La Liberté}{\newline}→\textcolor{brown}{Journal des débats}} haben die Bitte mit
               großer {\pb}Liebenswürdigkeit erfüllt. Ich ſende ſie Dir
               anbei; ſtoße Dich nicht an die Unrichtigkeiten, die Du in den \textcolor{green}{Notizen}{}\ledrightnote{→\textcolor{green}{Thêatres. [Notre correspondant de Vienne]}{\newline}→\textcolor{green}{Courrier des Théatres [Liebelei]}} findeſt; ich habe ihnen
               die Geſchichte zwar genau erklärt, aber ſie haben doch geſchrieben, was ſie wollten;
               das iſt ſo \textcolor{pink}{Pariſ}{}\ledrightnote{\textcolor{pink}{Paris}}er Art. Jedenfalls aber mußt Du
               Dich bedanken; das iſt hier ſo Sitte. Zuerſt mußt Du \strikeout{\textcolor{gray}{ei}} Deine Viſitkarte mit der Aufſchrift: \label{K_L02751-5v}\edtext{\begin{otherlanguage}{french}\textsc{remercie bien vivement M. \textcolor{blue}{Fierens-Gevaert}{}\ledrightnote{\textcolor{blue}{Hippolyte Fierens-Gevaert}} de son amabilité}\end{otherlanguage}}{\lemma{\textnormal{\emph{remercie … amabilité}}}\Cendnote{\textnormal{französisch: dankt sehr herzlich Herrn
                     \textcolor{blue}{Fierens-Gevaert} für seine
                  Freundlichkeit}}}\label{K_L02751-5h}{ }{\pb}ſchicken an: \begin{otherlanguage}{french}\textsc{M. \textcolor{blue}{Fierens-Gevaert}{}\ledrightnote{\textcolor{blue}{Hippolyte Fierens-Gevaert}}, du
                        »\textcolor{brown}{Journal des Débats}{}\ledrightnote{\textcolor{brown}{Journal des débats}}«, \textcolor{pink}{Rue des Prêtres – St. Germain l’Auxerrois, Paris}{}\ledrightnote{\textcolor{pink}{Rue Des Prêtres Saint-Germain L'Auxerrois}}}\end{otherlanguage}. Eine zweite Karte ſendeſt Du an \textsc{\textcolor{blue}{M. Aubry}{}\ledrightnote{\textcolor{blue}{Georges Aubry}}\textcolor{gray}{,}{ }\begin{otherlanguage}{french}de la »\textcolor{brown}{Liberté}{}\ledrightnote{\textcolor{brown}{La Liberté}}«, \textcolor{pink}{10. Rue Camou, Paris}{}\ledrightnote{\textcolor{pink}{Rue du Général Camou}}\end{otherlanguage}}. Hier mußt Du ſchon etwas wärmer ſchreiben, da \textcolor{blue}{Aubry}{}\ledrightnote{\textcolor{blue}{Georges Aubry}} ein ſehr herzliches Intereſſe für Dich bezeigt, ſich
               eine mörderiſche Mühe {\pb}gegeben hat, um die von
               ſeiner \textcolor{blue}{Frau}{}\ledrightnote{→\textcolor{blue}{[MMe. Georges] Aubry}} überſetzte »\textcolor{green}{Kleine Komödie}{}\ledrightnote{\textcolor{green}{Die kleine Komödie}}« in gutes Franzöſiſch zu bringen
               (die \textcolor{green}{Überſetzung}{}\ledrightnote{→\textcolor{green}{La petite comédie. Mœurs viennois}} iſt
               infolgedeſſen vortrefflich) \textsc{et{[}c{]}}. Du ſchreibſt alſo vielleicht auf Deine Karte: \label{K_L02751-6v}\edtext{\begin{otherlanguage}{french}\textsc{remercie M. \textcolor{blue}{Aubry}{}\ledrightnote{\textcolor{blue}{Georges Aubry}} du
                        \strikeout{bel} très-bel \textcolor{green}{article}{}\ledrightnote{→\textcolor{green}{Thêatres. [Notre correspondant de Vienne]}} au sujet de la »\textcolor{green}{Liebelei}{}\ledrightnote{\textcolor{green}{Liebelei. Schauspiel in drei Akten}}«, le remercie en outre de toute la
                     peine, qu’il s’est donnée pour la \textcolor{green}{traduction}{}\ledrightnote{→\textcolor{green}{La petite comédie. Mœurs viennois}} de la »\textcolor{green}{Petite comédie}{}\ledrightnote{\textcolor{green}{Die kleine Komödie}}«, le remercie en un mot de toute son amabilité
                     charmante et espère {\pb}de lui serrer un jour la
                        \strikeout{main} main en ami, soit à \textcolor{pink}{Paris}{}\ledrightnote{\textcolor{pink}{Paris}}, soit à \textcolor{pink}{Vienne}{}\ledrightnote{\textcolor{pink}{Wien}}}\end{otherlanguage}}{\lemma{\textnormal{\emph{remercie … Vienne}}}\Cendnote{\textnormal{französisch: dankt Herrn \textcolor{blue}{Aubry} für den sehr schönen \textcolor{green}{Artikel} über die \emph{\textcolor{green}{Liebelei}}, dankt auch für all die Mühen, die er sich um die
                     \textcolor{green}{Übersetzung} der »\emph{\textcolor{green}{Kleinen Komödie}}« gemacht hat, dankt ihm mit
                  einem Wort für all seine liebenswürdige Freundlichkeit und hofft, ihm eines Tages
                  in \textcolor{pink}{Paris} oder in \textcolor{pink}{Wien} als \textcolor{blue}{Freund}
                  die Hand drücken zu dürfen}}}\label{K_L02751-6h}{\dotsfive}\pend
           \pstart
           So, da haſt Du wieder ein wenig Arbeit.\pend
           \pstart
           Nochmals, vielen Dank für Dein Telegramm! Danke auch \textsc{\textcolor{blue}{Richard}{}\ledrightnote{\textcolor{blue}{Richard Beer-Hofmann}}} für das ſeinige! Und ſei von Herzen gegrüßt!\pend
           \pstart
           Dein {\\[\baselineskip]}\spacefill\mbox{Paul Goldmann.}\pend
           \leftskip=0em{}\pstart
           \noindent{}Bitte, empfiehl mich Deiner Frau \textcolor{blue}{Mama}{}\ledrightnote{→\textcolor{blue}{Louise Schnitzler}} und ſag’ ihr, ich laſſe ihr zu ihrem Sohne
                  gratuliren.\pend
           \endnumbering\briefempfaengerindex{Schnitzler, Arthur@\textsc{Schnitzler, Arthur}!zzzGoldmann, Paul@\emph{von Paul Goldmann}!1895-10-131@{13. 10. {[}1895{]}}|)be}\mylabel{h}  \normalsize

\doendnotes{C}
\bigskip
\vfill

\clearpage

\footnotesize

\lohead{\textsc{register}}

% Definiere theindex-Environment komplett neu ohne reledmac
\makeatletter
\renewenvironment{theindex}{%
  \section*{\indexname}%
  \setlength{\parindent}{0pt}%
  \setlength{\parskip}{0pt plus 0.3pt}%
  \let\item\@idxitem
}{%
  \clearpage
}
\makeatother

\IfFileExists{\jobname-pw.ind}{\input{\jobname-pw.ind}}{}

\end{document}

      