%% latex-korrekturansicht-vorspann.tex
%% Vorspann für die Korrekturansicht.
%% Lädt die gemeinsame Datei latex-vorspann.tex mit gesetztem Schalter.

\newif\ifkorrekturansicht
\korrekturansichttrue

\input{../tex-inputs/latex-vorspann}


\renewcommand{\erwaehntePersonen}{Personen: Olga Schnitzler}
\renewcommand{\erwaehnteInstitutionen}{Institutionen: Neue Freie Presse}
\renewcommand{\erwaehnteOrte}{Orte: Dessauer Straße, Wien}
\renewcommand{\erwaehnteWerke}{}
\section[ Paul Goldmann an Arthur Schnitzler, {[}10. 8. 1904{]}]{Paul Goldmann an Arthur Schnitzler, {[}10. 8. 1904{]}}
\nopagebreak\mylabel{v}
\rehead{ }\normalsize\beginnumbering\briefempfaengerindex{Schnitzler, Arthur@\textsc{Schnitzler, Arthur}!zzzGoldmann, Paul@\emph{von Paul Goldmann}!1904-08-101@{{[}10. 8. 1904{]}}|(be}
\toendnotes[C]{\smallbreak\pagebreak[2]}\Standort{DLA, A:Schnitzler, HS.NZ85.1.3174.}
\physDesc{Visitenkarte, 321 Zeichen
\newline{}Handschrift: Bleistift, deutsche Kurrent
\newline{}Schnitzler: mit Bleistift das Datum »10/8 {[}1{]}904« vermerkt }\toendnotes[C]{\smallbreak}
\pstart
           \noindent{}\centering{}{\pb}\textcolor{gray}{\textbf{D\textsuperscript{r} Paul Goldmann}}\pend
           
\pstart
           \noindent{}\raggedleft{}\textcolor{gray}{\textbf{»\textcolor{brown}{Neue Freie Presse}{}\ledrightnote{\textcolor{brown}{Neue Freie Presse}}«}}\pend
           
\pstart
           \noindent{}\textcolor{gray}{\textbf{\textcolor{pink}{DESSAUERSTRASSE 19}{}\ledrightnote{\textcolor{pink}{Dessauer Straße}}. }}\pend
           
\pstart
           {\pb}Mein lieber Freund, Es thut mir unendlich
               leid, Dich verfehlt zu haben. Ich fahre \label{K_L03453-1v}\edtext{heut}{\lemma{\textnormal{\emph{heut}}}\Cendnote{\textnormal{\textcolor{blue}{Schnitzler} war nicht verreist, hatte aber die letzten Tage mit Ausflügen
                  angefüllt. \textcolor{blue}{Goldmann} beschloss, nachdem er
                  diese Karte hinterlegt hatte, seine Abreise um einen Tag zu verschieben, um \textcolor{blue}{Schnitzler} doch noch zu sehen (siehe Paul Goldmann an Arthur Schnitzler, 10. 8. 1904).}}}\label{K_L03453-1h}{ }9 Uhr 40 Abends weiter und muß alſo \textcolor{pink}{Wien}{}\ledrightnote{\textcolor{pink}{Wien}} verlaſſen, ohne Dich geſehen zu haben. Ich wäre Dir gern noch
               nachgekommen, aber Niemand weiß, wohin \textcolor{blue}{Ihr}{}\ledrightnote{{$\rightarrow$}\textcolor{blue}{Olga Schnitzler}} gegangen ſeid. Hoffentlich ſehen wir uns auf der
               Rückreiſe. Herzliche Grüße Dir und Deiner \textcolor{blue}{Frau}{}\ledrightnote{{$\rightarrow$}\textcolor{blue}{Olga Schnitzler}}!\pend
           \endnumbering\briefempfaengerindex{Schnitzler, Arthur@\textsc{Schnitzler, Arthur}!zzzGoldmann, Paul@\emph{von Paul Goldmann}!1904-08-101@{{[}10. 8. 1904{]}}|)be}\mylabel{h}  \normalsize

\doendnotes{C}
\bigskip
\vfill

\clearpage

\footnotesize

\lohead{\textsc{register}}

% Definiere theindex-Environment komplett neu ohne reledmac
\makeatletter
\renewenvironment{theindex}{%
  \section*{\indexname}%
  \setlength{\parindent}{0pt}%
  \setlength{\parskip}{0pt plus 0.3pt}%
  \let\item\@idxitem
}{%
  \clearpage
}
\makeatother

\IfFileExists{\jobname-pw.ind}{\input{\jobname-pw.ind}}{}

\end{document}

      