%% latex-korrekturansicht-vorspann.tex
%% Vorspann für die Korrekturansicht.
%% Lädt die gemeinsame Datei latex-vorspann.tex mit gesetztem Schalter.

\newif\ifkorrekturansicht
\korrekturansichttrue

\input{../tex-inputs/latex-vorspann}


               \section[Robert Adam an Arthur Schnitzler, 15. 11. 1918]{ Robert Adam an Arthur Schnitzler, 15. 11. 1918}\nopagebreak\mylabel{v}\rehead{ }\normalsize\beginnumbering\briefempfaengerindex{Schnitzler, Arthur@\textsc{Schnitzler, Arthur}!zzzAdam, Robert@\emph{von Robert Adam}!1918-11-151@{15. 11. 1918}|(be} \toendnotes[C]{\smallbreak\pagebreak[2]} \Standort{CUL, Schnitzler, B 1.}
\physDesc{Brief, 1 Blatt, 3 Seiten
\newline{}Handschrift: schwarze Tinte, deutsche Kurrent
\newline{}Schnitzler: 1) mit Bleistift beschriftet: »\textsc{Adam}« 2) mit rotem Buntstift drei Unterstreichungen\newline{}Ordnung: von unbekannter Hand nummeriert:
                                    »9« }\Standort{Wien, Österreichische Nationalbibliothek, Cod.ser. 52.269, 225.}
\physDesc{Brief, maschinelle Abschrift
\newline{}Schreibmaschine}\toendnotes[C]{\smallbreak}\pstart
           \raggedleft{}{\pb}\textcolor{pink}{Wien}{}\ledrightnote{\textcolor{pink}{Wien}}, am 15. November 1918\pend
           \pstart\center{}Hochverehrter Herr Doktor!\pend\pstart
           Ich habe geſtern, ſofort nach Erhalt Ihres Schreibens, beide Stücke – den »\textcolor{green}{Fremden}{}\ledrightnote{\textcolor{green}{Der Fremde}}« und »\textcolor{green}{Yppl}{}\ledrightnote{\textcolor{green}{Yppl. Idylle in fünf Akten}}« beim \textcolor{pink}{Deutſchen Volkstheater}{}\ledrightnote{\textcolor{pink}{Volkstheater}}
                    eingereicht, und zwar zu Händen des Dramaturgen D\textsuperscript{r}{ }\textcolor{blue}{Glücksmann}{}\ledrightnote{\textcolor{blue}{Heinrich Glücksmann}}, dem ich einen kurzen an die
                    Direktion gerichteten Brief mit Berufung auf Ihre mündliche Empfehlung übergab;
                    in dieſem Schreiben wies ich darauf hin, daß es mit dem Stil des »\textcolor{green}{Fremden}{}\ledrightnote{\textcolor{green}{Der Fremde}}« vereinbar wäre, wenn die Perſonen –
                    wie auf \textcolor{blue}{Uhde}{}\ledrightnote{\textcolor{blue}{Fritz von Uhde}}’ſchen Bildern – in modernen oder
                    halbmodernen Koſtümen er{\pb}ſcheinen, daß daher die
                    Koſtümfrage kaum Schwierigkeiten bereiten dürfte. Heute vormittags wollte ich
                    beim \textcolor{blue}{Direktor}{}\ledrightnote{\textcolor{blue}{Alfred Bernau}} vorſprechen, traf ihn aber
                    nicht an und hinterließ meine Karte, wobei ich den Sekretär erſuchte, darauf
                    aufmerkſam zu machen, daß die Stücke bereits eingereicht ſeien.\pend
           \pstart
           Nun muß ich die Dinge ihren Lauf gehen laſſen und ſehe der Entſcheidung mit oft
                    erprobtem Fatalismus entgegen. Hätte ich diesmal nicht wieder Pech, ſo wär’s ein
                    Wunder! –\pend
           \pstart
           Die letzten Tage, die uns die Republik und mir damit die Erfüllung langjähriger
                    Träume gebracht haben, habe ich in größter Erregung durchlebt, von der auch eine
                    ziemlich geſchmackloſe \label{K_L02311_1v}\edtext{Kundgebung}{\lemma{\textnormal{\emph{Kundgebung}}}\Cendnote{\textnormal{[O. V.:] \emph{\textcolor{green}{Ein Richter für die
                                Republik}}. In: \emph{\textcolor{green}{Wiener Allgemeine
                                Zeitung}}, Nr. 12169, 12. 11. 1918, 6 Uhr-Blatt,
                            S. 1: »An der Türe des Verhandlungssaales IV beim \textcolor{brown}{Bezirksgericht Josefstadt} war heute folgende \so{Kundmachung} auf einem halben Kanzleibogen
                                zu lesen:{ / }›Am Tage, da die demokratische Republik und der Anschluß an \textcolor{pink}{Deutschland} verkündigt wird, \so{will ich keine Strafurteile zu fällen haben. Die Strafverhandlungen werden daher nicht stattfinden. Es lebe die Republik!}{ / }12. November 1918.\hspace*{1.5em}
                                Landesgerichtsrat Dr. Pollak‹«.}}}\label{K_L02311_1h} zeugt, die ich am Tage der Proklamation verbrach und {\pb}die ihren Weg in die Blätter gefunden hat (wie ich
                    höre ſogar in’s \label{K_L02311_2v}\edtext{\textcolor{brown}{Prager Tagblatt}{}\ledrightnote{\textcolor{brown}{Prager Tagblatt}}}{\lemma{\textnormal{\emph{Prager Tagblatt}}}\Cendnote{\textnormal{[O. V.:] \emph{\textcolor{green}{Kein Strafurteil an dem ersten
                                Tag der Republik}}. In: \emph{\textcolor{green}{Prager
                                Tagblatt}}, Jg. 43, Nr. 264, 13. 11. 1918,
                            Morgen-Ausgabe, S. 3.}}}\label{K_L02311_2h}; dies iſt ſchließlich in Anbetracht
                    der Eigentümlichkeit der \textcolor{pink}{Prag}{}\ledrightnote{\textcolor{pink}{Prag}}er Pſyche nichts
                    Verwunderliches). Ich tröſte mich mit einem Spruch: »Begeiſterung macht Schmöcke
                    aus uns allen«. – Ich habe auch die furchtbare Panik vor dem Parlament miterlebt
                    und weiß jetzt, wie einem zumute iſt, wenn man wehrlos im Maſchinengewehrfeuer
                    zu ſtehen vermeint. Es waren ganz entſetzliche und ſehr intereſſante Minuten.
                    –\pend
           \pstart
           Ich danke Ihnen herzlich für Ihre liebenswürdige Verwendung und gebe in
                    Anbetracht derſelben, trotz allem Kleinmut, die Hoffnung nicht auf, diesmal doch
                    einen Durchbruch zu erzielen.\pend
           \pstart
           Mit beſten Grüßen Ihr ergebener\pend
           \pstart \spacefill\mbox{D\textsuperscript{r}RAdam}\pend{}\endnumbering\briefempfaengerindex{Schnitzler, Arthur@\textsc{Schnitzler, Arthur}!zzzAdam, Robert@\emph{von Robert Adam}!1918-11-151@{15. 11. 1918}|)be}\mylabel{h}  \normalsize

\doendnotes{C}
\bigskip
\vfill

\clearpage

\footnotesize

\lohead{\textsc{register}}

% Definiere theindex-Environment komplett neu ohne reledmac
\makeatletter
\renewenvironment{theindex}{%
  \section*{\indexname}%
  \setlength{\parindent}{0pt}%
  \setlength{\parskip}{0pt plus 0.3pt}%
  \let\item\@idxitem
}{%
  \clearpage
}
\makeatother

\IfFileExists{\jobname-pw.ind}{\input{\jobname-pw.ind}}{}

\end{document}

      