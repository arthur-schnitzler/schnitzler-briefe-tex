%% latex-korrekturansicht-vorspann.tex
%% Vorspann für die Korrekturansicht.
%% Lädt die gemeinsame Datei latex-vorspann.tex mit gesetztem Schalter.

\newif\ifkorrekturansicht
\korrekturansichttrue

\input{../tex-inputs/latex-vorspann}


               \section[Paul Goldmann an Arthur Schnitzler, 17. 8. 1903]{ Paul Goldmann an Arthur Schnitzler, 17. 8. 1903}\nopagebreak\mylabel{v}\rehead{ }\normalsize\beginnumbering\briefempfaengerindex{Schnitzler, Arthur@\textsc{Schnitzler, Arthur}!zzzGoldmann, Paul@\emph{von Paul Goldmann}!1903-08-171@{17. 8. 1903}|(be} \toendnotes[C]{\smallbreak\pagebreak[2]} \Standort{DLA, A:Schnitzler, HS.NZ85.1.3173.}
\physDesc{Telegramm
\newline{}Handschrift: schwarze Tinte, lateinische Kurrent\newline{}Versand: »\noindent{}\textcolor{gray}{\textbf{\textit{17 Aug 1903}}}{ }18
                                             N\textcolor{gray}{ach}\textcolor{gray}{\textbf{Mittag}}{ / }\textcolor{gray}{\textbf{Von}}{ }\textcolor{pink}{Riva}{ / }\textcolor{gray}{\textbf{Aufgabe-Nr.}} 368 \textcolor{gray}{\textbf{mit}} 20 \textcolor{gray}{\textbf{Taxworten}}« \newline{}Ordnung: beschnitten }\pstart{}{\pb}D\textsuperscript{r}
                  Schnitzler\pend{}\pstart{}postlagernd\pend{}{\bigskip}\pstart
           \noindent{}{\pb}\strikeout{Postlagernd.} Erbitte
               heute Drahtantwort \textcolor{pink}{Riva Hotel Lido}{}\ledrightnote{\textcolor{pink}{Palast Hotel Lido}} ob morgen
               übermorgen zusa{\geminationm}enkunft in \textcolor{pink}{Trient}{}\ledrightnote{\textcolor{pink}{Trient}} oder anderswo moeglich.\pend
           \pstart
           herzlichst{\\[\baselineskip]}\spacefill\mbox{Goldmann}\pend
           \leftskip=0em{}\endnumbering\briefempfaengerindex{Schnitzler, Arthur@\textsc{Schnitzler, Arthur}!zzzGoldmann, Paul@\emph{von Paul Goldmann}!1903-08-171@{17. 8. 1903}|)be}\mylabel{h}  \normalsize

\doendnotes{C}
\bigskip
\vfill

\clearpage

\footnotesize

\lohead{\textsc{register}}

% Definiere theindex-Environment komplett neu ohne reledmac
\makeatletter
\renewenvironment{theindex}{%
  \section*{\indexname}%
  \setlength{\parindent}{0pt}%
  \setlength{\parskip}{0pt plus 0.3pt}%
  \let\item\@idxitem
}{%
  \clearpage
}
\makeatother

\IfFileExists{\jobname-pw.ind}{\input{\jobname-pw.ind}}{}

\end{document}

      