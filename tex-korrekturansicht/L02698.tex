%% latex-korrekturansicht-vorspann.tex
%% Vorspann für die Korrekturansicht.
%% Lädt die gemeinsame Datei latex-vorspann.tex mit gesetztem Schalter.

\newif\ifkorrekturansicht
\korrekturansichttrue

\input{../tex-inputs/latex-vorspann}


               \section[Paul Goldmann an Arthur Schnitzler, 24. 6. {[}1892{]}]{ Paul Goldmann an Arthur Schnitzler, 24. 6. {[}1892{]}}\nopagebreak\mylabel{v}\rehead{ }\normalsize\beginnumbering\briefempfaengerindex{Schnitzler, Arthur@\textsc{Schnitzler, Arthur}!zzzGoldmann, Paul@\emph{von Paul Goldmann}!1892-06-241@{24. 6. {[}1892{]}}|(be} \toendnotes[C]{\smallbreak\pagebreak[2]} \Standort{DLA, A:Schnitzler, HS.NZ85.1.3163.}
\physDesc{Brief, 1 Blatt, 4 Seiten
\newline{}Handschrift: schwarze Tinte, deutsche Kurrent
\newline{}Schnitzler: 1) mit rotem Buntstift eine Unterstreichung 2) mit Bleistift das Jahr »92« vermerkt}\toendnotes[C]{\smallbreak}\pstart
           \noindent{}{\pb}\textcolor{gray}{\textbf{\textcolor{brown}{Frankfurter Zeitung}{}\ledrightnote{\textcolor{brown}{Frankfurter Zeitung}}.}}\pend
           \pstart
           \textcolor{gray}{\textbf{(\textcolor{brown}{Gazette de Francfort}{}\ledrightnote{\textcolor{brown}{Frankfurter Zeitung}}.)}}\pend
           \pstart
           \textcolor{gray}{\textbf{\begin{otherlanguage}{french}Directeur\end{otherlanguage}: \textbf{M. \textcolor{blue}{L. Sonnemann}{}\ledrightnote{\textcolor{blue}{Leopold Sonnemann}}}.}}\hfill \textsc{\textcolor{pink}{Paris}{}\ledrightnote{\textcolor{pink}{Paris}}}, 24. Juni.\pend
           \pstart
           \textcolor{gray}{\textbf{\begin{otherlanguage}{french}Journal politique, financier,\end{otherlanguage}}}\pend
           \pstart
           \textcolor{gray}{\textbf{\begin{otherlanguage}{french}commercial et litteraire.\end{otherlanguage}}}\pend
           \pstart
           \textcolor{gray}{\textbf{\begin{otherlanguage}{french}\textbf{Paraissant trois fois par jour}\end{otherlanguage}}}\pend
           \pstart
           \textcolor{gray}{\textbf{–}}\pend
           \pstart
           \textcolor{gray}{\textbf{\begin{otherlanguage}{french}\textbf{Bureaux à \textcolor{pink}{Paris}{}\ledrightnote{\textcolor{pink}{Paris}}:}\end{otherlanguage}}}\pend
           \pstart
           \textcolor{gray}{\textbf{\begin{otherlanguage}{french}\textbf{\textcolor{pink}{rue Richelieu 75.}{}\ledrightnote{\textcolor{pink}{rue Richelieu}}.}\end{otherlanguage}}}\pend
           \pstart
           \centering{}Mein lieber Arthur!\pend
           \pstart
           \noindent{}Ich habe \strikeout{heute}{ }\textsc{\textcolor{blue}{Herzl}{}\ledrightnote{\textcolor{blue}{Theodor Herzl}}}{ }\strikeout{h} dein \textcolor{green}{Märchen}{}\ledrightnote{\textcolor{green}{Das Märchen. Schauspiel in drei Aufzügen}}
               gegeben und war heute bei ihm. Derſelbe ſprach ſich
               darüber in Worten der \label{K_L02698-1v}\edtext{Begeiſterung}{\lemma{\textnormal{\emph{Begeiſterung}}}\Cendnote{\textnormal{Am 28. 6. 1892 notierte \textcolor{blue}{Schnitzler} in seinem \emph{\textcolor{green}{Tagebuch}}: »\textcolor{blue}{Herzl}’s begeistertes Urtheil übers \textcolor{green}{Märchen}, was mich lebhaft
                  freute.«}}}\label{K_L02698-1h} (wörtlich zu nehmen) aus. Er meinte, Du ſeieſt der einzige
               von uns allen Jungen – ihn inbegriffen – der ’was kann. Er meinte, du ſeieſt ein
               wahrer Dichter. Er meinte, das \textcolor{green}{Ding}{}\ledrightnote{→\textcolor{green}{Das Märchen. Schauspiel in drei Aufzügen}} habe ihn ſo gepackt, daß er es in einem Zuge ausgeleſen. Er meinte,
               meinte und meinte, ich weiß nicht, was noch Alles Wunderſchönes für Dich, weil es der
               von {\pb}ſich ſelbſt eingenommenſte \textcolor{blue}{Menſch}{}\ledrightnote{→\textcolor{blue}{Theodor Herzl}}{ }\textcolor{pink}{Europa}{}\ledrightnote{\textcolor{pink}{Europa}}s meint. Er ſagte ſchließlich, daß er Dir
               ſofort \label{K_L02698-2v}\edtext{geſchrieben}{\lemma{\textnormal{\emph{geſchrieben}}}\Cendnote{\textnormal{\textcolor{blue}{Theodor Herzl} schrieb erst am
                     29. 7. 1892 an Schnitzler (was dieser am 4. 8. 1892 im \emph{\textcolor{green}{Tagebuch}} festhielt). Siehe \textcolor{blue}{Theodor Herzl}: \emph{Briefe und Tagebücher}. Hg.
                     Alex Bein, Hermann Greive, Moshe Schaerf und Julius H. Schoeps. Bd. 1.:
                        \emph{Briefe und autobiographische Notizen. 1866–1895}.
                     Bearbeitet von Johannes Wachten. In Zusammenarbeit mit Chaya Harel, Daisy Tycho
                     und Manfred Winkler. Berlin,
                     Frankfurt am Main, Wien: \emph{\textcolor{brown}{Ullstein}}/\emph{\textcolor{brown}{Propyläen}}{ }1983,
                     S. 498–502.}}}\label{K_L02698-2h} hätte, wenn er nicht gefürchtet hätte – \textsc{pardon}, ich
               referire wörtlich – Du ſeieſt ein \textcolor{pink}{Wien}{}\ledrightnote{\textcolor{pink}{Wien}}er Jüdel und
               würdeſt Dir \label{K_L02698-3v}\edtext{\textsc{parchanische}}{\lemma{\textnormal{\emph{parchanische}}}\Cendnote{\textnormal{unklar; es könnte vom jiddischen Wort
                     »parve« herrühren, und »nicht koscher«
                  bedeuten; es könnte das jiddische oder tschechische Wort für »Bastard« gemeint
                  sein.}}}\label{K_L02698-3h} Gedanken darüber machen.\pend
           \pstart
           Ich gratulire Dir herzlich zu dieſem ſchönen Erfolge Deines Talentes.\pend
           \pstart
           Das iſt das einzige Dich intereſſirende, was ich ſeit langer Zeit zu berichten
               finde.\pend
           \pstart
           Über mich laß’ mich ſchweigen. Ich verfalle und verrohe, \textcolor{pink}{Paris}{}\ledrightnote{\textcolor{pink}{Paris}} iſt mir widerlich, meine Stellung entſetzlich, das
               Heimweh nach \textcolor{pink}{Wien}{}\ledrightnote{\textcolor{pink}{Wien}}, nach Dir und all’ {\pb}den lieben Menſchen verzehrt mich. Ich bin einſam,
               zertreten und lieblos. Die Freundſchaft habe ich auch verloren, wie Du weißt. Durch
               meine Schuld, jawohl. Ich kann mich nicht mehr dazu aufſchwingen, dir ſo zu
               ſchreiben, wie ich Dir es ſchuldig wäre. Ich bin ſchon zu tief. Und ich denke, es iſt
               beſſer; ich laſſe mich langſam in die Vergeſſenheit herunterſinken.\pend
           \pstart
           Ich grüße \textsc{\textcolor{blue}{Richard}{}\ledrightnote{\textcolor{blue}{Richard Beer-Hofmann}}} und \textcolor{blue}{\textsc{Loris}}{}\ledrightnote{\textcolor{blue}{Hugo von Hofmannsthal}} und
               umarme Dich von Herzen\pend
           \pstart
           Dein {\\[\baselineskip]}treuer {\\[\baselineskip]}\spacefill\mbox{Paul Goldmann.}\pend
           \leftskip=0em{}\pstart
           \noindent{}{\pb}Es ſei denn, daß Du ein Mittel wüßteſt, wie ich
                  Dich im Auguſt, wo ich wahrſcheinlich kurzen Urlaub
                  bekomme, \label{K_L02698-4v}\edtext{ſehen kann}{\lemma{\textnormal{\emph{ſehen kann}}}\Cendnote{\textnormal{Das nächste Wiedersehen fand am 17. 9. 1893
                     statt.}}}\label{K_L02698-4h}. Aber nach \textsc{\textcolor{pink}{Wien}{}\ledrightnote{\textcolor{pink}{Wien}}} komme ich nicht, weil ich nicht ein zweites Mal die Kraft fände, mich
                  loszureißen.\pend
           \pstart
           Meine einzige Freude iſt \textsc{\textcolor{blue}{Arthur Klein}{}\ledrightnote{\textcolor{blue}{Arthur Klein}}}. \textsc{\textcolor{blue}{Leopold Spitzer}{}\ledrightnote{\textcolor{blue}{Leopold Spitzer}}}, der eine widerlich gemeine \label{K_L02698-5v}\edtext{Ladenſchwung}{\lemma{\textnormal{\emph{Ladenſchwung}}}\Cendnote{\textnormal{abwertende
                     Bezeichnung für einen Ladendiener oder Ladenjungen}}}\label{K_L02698-5h}-Seele iſt, habe ich
                  vor 14 Tagen geohrfeigt, was mich um ein Haar um meine Stellung gebracht hätte und
                  vielleicht noch bringt.\pend
           \endnumbering\briefempfaengerindex{Schnitzler, Arthur@\textsc{Schnitzler, Arthur}!zzzGoldmann, Paul@\emph{von Paul Goldmann}!1892-06-241@{24. 6. {[}1892{]}}|)be}\mylabel{h}  \normalsize

\doendnotes{C}
\bigskip
\vfill

\clearpage

\footnotesize

\lohead{\textsc{register}}

% Definiere theindex-Environment komplett neu ohne reledmac
\makeatletter
\renewenvironment{theindex}{%
  \section*{\indexname}%
  \setlength{\parindent}{0pt}%
  \setlength{\parskip}{0pt plus 0.3pt}%
  \let\item\@idxitem
}{%
  \clearpage
}
\makeatother

\IfFileExists{\jobname-pw.ind}{\input{\jobname-pw.ind}}{}

\end{document}

      