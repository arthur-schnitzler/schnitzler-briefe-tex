%% latex-korrekturansicht-vorspann.tex
%% Vorspann für die Korrekturansicht.
%% Lädt die gemeinsame Datei latex-vorspann.tex mit gesetztem Schalter.

\newif\ifkorrekturansicht
\korrekturansichttrue

\input{../tex-inputs/latex-vorspann}


               \section[Hugo von Hofmannsthal an Arthur Schnitzler, 1. 2. {[}1893{]}]{ Hugo von Hofmannsthal an Arthur Schnitzler, 1. 2. {[}1893{]}}\nopagebreak\mylabel{v}\rehead{ }\normalsize\beginnumbering\briefempfaengerindex{Schnitzler, Arthur@\textsc{Schnitzler, Arthur}!zzzHofmannsthal, Hugo von@\emph{von Hugo von Hofmannsthal}!1893-02-013@{1. 2. {[}1893{]}}|(be} \toendnotes[C]{\smallbreak\pagebreak[2]} \Standort{CUL, Schnitzler, B 43.}
\physDesc{Brief, 1 Blatt, 2 Seiten
\newline{}Handschrift: schwarze Tinte, deutsche Kurrent
\newline{}Schnitzler: mit Bleistift nummeriert: »43« und umdatiert zu: »1. III.« }\buchAbdrucke{\weitereDrucke{1) Hugo von Hofmannsthal, Arthur Schnitzler: \emph{Briefwechsel}. Hg. Therese Nickl und Heinrich Schnitzler. Frankfurt am Main: \emph{S. Fischer} 1964, S. 35–36.} \weitereDrucke{2) Hermann Bahr, Arthur Schnitzler: \emph{Briefwechsel, Aufzeichnungen, Dokumente
                                (1891–1931)}. Hg. Kurt Ifkovits und Martin Anton Müller. Göttingen: \emph{Wallstein} 2018, S. 33.} }\pstart
           \raggedleft{}{\pb}1. II\pend
           \pstart{}lieber Arthur.\pend\pstart
           \textcolor{blue}{Bahr}{}\ledrightnote{\textcolor{blue}{Hermann Bahr}}{ }ſtellte mir zu meiner Freude folgenden
                    Antrag: er ſei im Stande und gern bereit, \textcolor{blue}{Fels}{}\ledrightnote{\textcolor{blue}{Friedrich Michael Fels}} von Anfang März an mit einem Gehalt von 100 fl in
                    der \textcolor{brown}{Deutſchen Zeitung}{}\ledrightnote{\textcolor{brown}{Deutsche Zeitung}} als Redacteur
                    unterzubringen. Es handelt ſich nur um Fähigkeit und Bereitwilligkeit. Dritten
                    Perſonen werden Sie es vorläufig ebenſowenig erzählen, wie ich.\pend
           \pstart
           Falls wir Sonntag bei Ihnen Zuſammenkommen, zu welchem {\pb}Zweck ich wenigſtens
                    vorläufig eine Einladung abgelehnt habe, ſeien Sie doch ſogut, \textcolor{blue}{Robert Ehrhardt}{}\ledrightnote{\textcolor{blue}{Robert Ehrhart von Ehrhartstein}} (\textsc{V. \textcolor{pink}{Siebenbrunng.}{}\ledrightnote{\textcolor{pink}{Siebenbrunnengasse}} 29}) ausdrücklich einzuladen. Er
                    geht der Trauer wegen faſt nicht in Geſellschaft und würde gewiſs gern
                    kommen.\pend
           \pstart
           Herzlichſt\hspace*{2.5em}Ihr{\\[\baselineskip]}\spacefill\mbox{Loris.}\pend
           \leftskip=0em{}\pstart
           \noindent{}P. S.{\\}Ich denke ſehr oft an die Novelle vom \textcolor{green}{Sterben}{}\ledrightnote{\textcolor{green}{Sterben. Novelle}} und möchte viel mehr davon reden, als geſchieht. Sie haben
                        was gegen die Geſchichte. Wenigſtens ſcheinen Sie ſie todtſchweigen zu
                        wollen.\pend
           \endnumbering\briefempfaengerindex{Schnitzler, Arthur@\textsc{Schnitzler, Arthur}!zzzHofmannsthal, Hugo von@\emph{von Hugo von Hofmannsthal}!1893-02-013@{1. 2. {[}1893{]}}|)be}\mylabel{h}  \normalsize

\doendnotes{C}
\bigskip
\vfill

\clearpage

\footnotesize

\lohead{\textsc{register}}

% Definiere theindex-Environment komplett neu ohne reledmac
\makeatletter
\renewenvironment{theindex}{%
  \section*{\indexname}%
  \setlength{\parindent}{0pt}%
  \setlength{\parskip}{0pt plus 0.3pt}%
  \let\item\@idxitem
}{%
  \clearpage
}
\makeatother

\IfFileExists{\jobname-pw.ind}{\input{\jobname-pw.ind}}{}

\end{document}

      