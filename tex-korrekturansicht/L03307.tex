%% latex-korrekturansicht-vorspann.tex
%% Vorspann für die Korrekturansicht.
%% Lädt die gemeinsame Datei latex-vorspann.tex mit gesetztem Schalter.

\newif\ifkorrekturansicht
\korrekturansichttrue

\input{../tex-inputs/latex-vorspann}


\renewcommand{\erwaehntePersonen}{Personen: Richard Beer-Hofmann, Paul Goldmann, Alfred Kerr, Ottilie Salten, Leo Van-Jung}
\renewcommand{\erwaehnteOrte}{Orte: Alpen, Bad Ischl, Karlsbad, Meran, Pressbaum, Schruns, Vorarlberg}
\renewcommand{\erwaehnteWerke}{}
\section[ Felix Salten an Arthur Schnitzler, 5. 8. 1900]{Felix Salten an Arthur Schnitzler, 5. 8. 1900}
\nopagebreak\mylabel{v}
\rehead{ }\normalsize\beginnumbering\briefempfaengerindex{Schnitzler, Arthur@\textsc{Schnitzler, Arthur}!zzzSalten, Felix@\emph{von Felix Salten}!1900-08-052@{5. 8. 1900}|(be}
\toendnotes[C]{\smallbreak\pagebreak[2]}\Standort{CUL, Schnitzler, B 89, A 2.}
\physDesc{Brief, 1 Blatt, 2 Seiten, 594 Zeichen
\newline{}Handschrift: Bleistift, lateinische Kurrent
\newline{}Ordnung: mit Bleistift von unbekannter Hand nummeriert: »131« }\toendnotes[C]{\smallbreak}
\pstart
           \raggedleft{}{\pb}\textcolor{pink}{Pressbaum}{}\ledrightnote{\textcolor{pink}{Pressbaum}}, 5./VIII. 00\pend
           
\pstart
           Lieber Freund, wahrscheinlich komme ich noch vor dem 10. nach \label{K_L03307-1v}\edtext{\textcolor{pink}{Ischl}{}\ledrightnote{\textcolor{pink}{Bad Ischl}}}{\lemma{\textnormal{\emph{Ischl}}}\Cendnote{\textnormal{Am 17. 8. 1900 startete \textcolor{blue}{Schnitzler} gemeinsam mit \textcolor{blue}{Richard Beer-Hofmann}, \textcolor{blue}{Paul Goldmann}, \textcolor{blue}{Alfred Kerr} und \textcolor{blue}{Leo
                           Van-Jung} eine \textcolor{pink}{Alpen}wanderung in \textcolor{pink}{Schruns} (\textcolor{pink}{Vorarlberg}). Am 28. 8. 1900 reiste \textcolor{blue}{Schnitzler} alleine weiter nach \textcolor{pink}{Meran}, wo er schließlich auf \textcolor{blue}{Salten} traf.}}}\label{K_L03307-1h}. Ungefähr Dienstag{ }Abend oder Mittwoch{ }früh. Aber ich werde eher den Schluß der Parthie mitmachen, als den
               Anfang. Ich kann am 12. noch nicht von \textcolor{pink}{Ischl}{}\ledrightnote{\textcolor{pink}{Bad Ischl}} fort, weil \textcolor{blue}{Otti}{}\ledrightnote{\textcolor{blue}{Ottilie Salten}} die \textcolor{pink}{Vorarlberg}{}\ledrightnote{\textcolor{pink}{Vorarlberg}}er Sache nicht
               mitmacht, sondern mich allein fahren läßt. So will ich doch bis 16. od. 17. bei ihr
               bleiben und dann direct nach {\pb}\textcolor{pink}{Schruns}{}\ledrightnote{\textcolor{pink}{Schruns}} fahren. Ich dachte nicht, dass die
               Parthie schon so bald losgeht. Übrigens machen wir wol mündlich noch alles nähere
               aus.\pend
           
\pstart
           Auf \label{K_L03307-2v}\edtext{Wiedersehen, vorraussichtlich in \textcolor{pink}{Ischl}{}\ledrightnote{\textcolor{pink}{Bad Ischl}}}{\lemma{\textnormal{\emph{Wiedersehen, … Ischl}}}\Cendnote{\textnormal{In \textcolor{pink}{Ischl} trafen sie sich nicht, weil \textcolor{blue}{Schnitzler} bei \textcolor{blue}{Salten}s Ankunft bereits abgereist war.}}}\label{K_L03307-2h}.\pend
           
\pstart
           Herzlichst Ihr {\\[\baselineskip]}\spacefill\mbox{Salten.}\pend
           \leftskip=0em{}
\pstart
           \noindent{}\textcolor{blue}{Otti}{}\ledrightnote{\textcolor{blue}{Ottilie Salten}} ist jetzt in \textcolor{pink}{Karlsbad}{}\ledrightnote{\textcolor{pink}{Karlsbad}}.\pend
           \endnumbering\briefempfaengerindex{Schnitzler, Arthur@\textsc{Schnitzler, Arthur}!zzzSalten, Felix@\emph{von Felix Salten}!1900-08-052@{5. 8. 1900}|)be}\mylabel{h}  \normalsize

\doendnotes{C}
\bigskip
\vfill

\clearpage

\footnotesize

\lohead{\textsc{register}}

% Definiere theindex-Environment komplett neu ohne reledmac
\makeatletter
\renewenvironment{theindex}{%
  \section*{\indexname}%
  \setlength{\parindent}{0pt}%
  \setlength{\parskip}{0pt plus 0.3pt}%
  \let\item\@idxitem
}{%
  \clearpage
}
\makeatother

\IfFileExists{\jobname-pw.ind}{\input{\jobname-pw.ind}}{}

\end{document}

      