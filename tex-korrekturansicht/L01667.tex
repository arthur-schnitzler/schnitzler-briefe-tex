%% latex-korrekturansicht-vorspann.tex
%% Vorspann für die Korrekturansicht.
%% Lädt die gemeinsame Datei latex-vorspann.tex mit gesetztem Schalter.

\newif\ifkorrekturansicht
\korrekturansichttrue

\input{../tex-inputs/latex-vorspann}


               \section[Hugo von Hofmannsthal an Arthur Schnitzler, 13. 4. 1907]{ Hugo von Hofmannsthal an Arthur Schnitzler, 13. 4. 1907}\nopagebreak\mylabel{v}\rehead{ }\normalsize\beginnumbering\briefempfaengerindex{Schnitzler, Arthur@\textsc{Schnitzler, Arthur}!zzzHofmannsthal, Hugo von@\emph{von Hugo von Hofmannsthal}!1907-04-131@{13. 4. 1907}|(be} \toendnotes[C]{\smallbreak\pagebreak[2]} \Standort{CUL, Schnitzler, B 43.}
\physDesc{Postkarte
\newline{}Handschrift: Bleistift, lateinische Kurrent\newline{}Versand: 1) Rohrpost 2) Stempel: »\nobreak{}\oindex{I., Innere Stadt@\textbf{I., Innere Stadt}, \emph{Bezirk (A.BZK)}|pwk}1/1 Wien 15, 13 IV 07, 1–N\nobreak{}«. 3) Stempel: »\nobreak{}\oindex{XVIII., Waehring@\textbf{XVIII., Währing}, \emph{Bezirk (A.BZK)}|pwk}18/1 Wien 111, 14 IV 07, 1\textsuperscript{50}\nobreak{}«. 
\newline{}Schnitzler: mit Bleistift datiert: »14/4 907« \newline{}Ordnung: 1) mit Bleistift von unbekannter Hand nummeriert: »270« 2) mit Bleistift von unbekannter Hand nummeriert: »273«}\buchAbdrucke{\weitereDrucke{Hugo von Hofmannsthal, Arthur Schnitzler: \emph{Briefwechsel}. Hg. Therese Nickl und Heinrich Schnitzler. Frankfurt am Main: \emph{S. Fischer} 1964, S. 227.} }\toendnotes[C]{\smallbreak}\pstart{}{\pb}Herrn D\textsuperscript{r} Arthur Schnitzler\pend{}\pstart{}\textcolor{pink}{Wien}{}\ledrightnote{\textcolor{pink}{Wien}}\pend{}\pstart{}\textcolor{pink}{XVII Spöttelgasse 7}{}\ledrightnote{\textcolor{pink}{Edmund-Weiß-Gasse}}\pend{}{\bigskip}\pstart
           \raggedleft{}{\pb}Samstag\pend
           \pstart
           Können wir morgen Sonntag nachmittg mit \textcolor{blue}{Christiane}{}\ledrightnote{\textcolor{blue}{Christiane von Hofmannsthal}} ko{\geminationm}en? (\uline{Nicht} zum Nachtmahl)\pend
           \pstart
           Bitte pneumatisch umgehend \textcolor{pink}{\uline{Elisabethstraße \introOben{}6\introOben{}}}{}\ledrightnote{\textcolor{pink}{Elisabethstraße}}{ }\introOben{}\textcolor{blue}{Schlesinger}{}\ledrightnote{\textcolor{blue}{Franziska Schlesinger}}\introOben{} oder \uline{telephonisch} 229 (vielleicht durch \textcolor{blue}{Beers}{}\ledrightnote{\textcolor{blue}{Richard Beer-Hofmann}{\newline}\textcolor{blue}{Paula Beer-Hofmann}}).\pend
           \pstart
           \label{T_L01667_1v}\edtext{(die Antwort
               trifft uns dort bis heute abends 8\textsuperscript{h})}{\lemma{\textnormal{\emph{(die … 8h)}}}\Cendnote{\textnormal{oberhalb
                  des Textes in der linken Ecke.}}}\label{T_L01667_1h}\pend
           \endnumbering\briefempfaengerindex{Schnitzler, Arthur@\textsc{Schnitzler, Arthur}!zzzHofmannsthal, Hugo von@\emph{von Hugo von Hofmannsthal}!1907-04-131@{13. 4. 1907}|)be}\mylabel{h}  \normalsize

\doendnotes{C}
\bigskip
\vfill

\clearpage

\footnotesize

\lohead{\textsc{register}}

% Definiere theindex-Environment komplett neu ohne reledmac
\makeatletter
\renewenvironment{theindex}{%
  \section*{\indexname}%
  \setlength{\parindent}{0pt}%
  \setlength{\parskip}{0pt plus 0.3pt}%
  \let\item\@idxitem
}{%
  \clearpage
}
\makeatother

\IfFileExists{\jobname-pw.ind}{\input{\jobname-pw.ind}}{}

\end{document}

      