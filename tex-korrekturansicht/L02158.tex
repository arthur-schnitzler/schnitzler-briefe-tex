%% latex-korrekturansicht-vorspann.tex
%% Vorspann für die Korrekturansicht.
%% Lädt die gemeinsame Datei latex-vorspann.tex mit gesetztem Schalter.

\newif\ifkorrekturansicht
\korrekturansichttrue

\input{../tex-inputs/latex-vorspann}


               \section[Stefan Großmann an Arthur Schnitzler, 30. 11. 1913]{ Stefan Großmann an Arthur Schnitzler, 30. 11. 1913}\nopagebreak\mylabel{v}\rehead{ }\normalsize\beginnumbering\briefempfaengerindex{Schnitzler, Arthur@\textsc{Schnitzler, Arthur}!zzzGrossmann, Stefan@\emph{von Stefan Großmann}!1913-11-301@{30. 11. 1913}|(be} \toendnotes[C]{\smallbreak\pagebreak[2]} \Standort{CUL, Schnitzler, B 34.}
\physDesc{Brief, 1 Blatt, 1 Seite
\newline{}Handschrift: schwarze Tinte, deutsche Kurrent\newline{}Ordnung: mit Bleistift von unbekannter Hand nummeriert: »13« }\pstart
           \noindent{}{\pb}\textcolor{gray}{\textbf{STEFAN GROSSMANN}}\hfill \textcolor{gray}{\textbf{\textcolor{pink}{WIEN,}{}\ledrightnote{\textcolor{pink}{Wien}}}}{ }30. \strikeout{Dece} Nov. 1913\pend
           \pstart
           \raggedleft{}\textcolor{pink}{I. \textsc{Dominikanerbastei}
                            5}{}\ledrightnote{\textcolor{pink}{Dominikanerbastei}}\pend
           \leftskip=3em{}\pstart
           \noindent{}Herrn D\textsuperscript{r} Arthur Schnitzler\pend
           \leftskip=0em{}\pstart
           \noindent{}\raggedleft{}\textcolor{pink}{\uline{Wien}}{}\ledrightnote{\textcolor{pink}{Wien}}\pend
           \pstart{}Sehr verehrter Herr.\pend\pstart
           Im Aufträge des Verlegers \uline{\textcolor{brown}{Ullſtein}{}\ledrightnote{\textcolor{brown}{Ullstein Verlag}}} in \textcolor{pink}{Berlin}{}\ledrightnote{\textcolor{pink}{Berlin}}, der die »\textsc{\textcolor{brown}{Vossische Zeitung}{}\ledrightnote{\textcolor{brown}{Vossische Zeitung}}}«, die »\textsc{\textcolor{brown}{B. Z. am Mittag}{}\ledrightnote{\textcolor{brown}{B.Z. am Mittag}}}« u die »\textcolor{brown}{\textsc{Morgenpost}}{}\ledrightnote{\textcolor{brown}{Morgenpost}}« herausgibt und in deſſen Dienſte ich getreten bin, habe ich an Sie die
                    höflichſte und dringende Bitte zu richten, ob Sie, verehrter Herr Doktor, dem
                    Verlage aus Anlaſs einer Jubiläumsnummer der \textcolor{brown}{\textsc{Morgenpost}}{}\ledrightnote{\textcolor{brown}{Morgenpost}}, die jetzt einen Abonnentenſtand von 400.000 Abonennten erreicht, einen
                    kleinen Beitrag überlaſſen wollten.\pend
           \pstart
           Der Verlag \textcolor{brown}{Ullſtein}{}\ledrightnote{\textcolor{brown}{Ullstein Verlag}} bittet Sie darum inſtändig.
                    Die beſten deutſchen Autoren ſind in dieser N\textsuperscript{o}
                    vertreten. Wollen Sie ſelbſt, wenn Sie Ihren Beitrag ankündigen, das Honorar
                    beſtimmen.\pend
           \pstart
           Mit außerordentlicher Hochschätzung{\\[\baselineskip]}sehr ergeben: \spacefill\mbox{Stefan
                        Großmann}\pend
           \leftskip=0em{}\endnumbering\briefempfaengerindex{Schnitzler, Arthur@\textsc{Schnitzler, Arthur}!zzzGrossmann, Stefan@\emph{von Stefan Großmann}!1913-11-301@{30. 11. 1913}|)be}\mylabel{h}  \normalsize

\doendnotes{C}
\bigskip
\vfill

\clearpage

\footnotesize

\lohead{\textsc{register}}

% Definiere theindex-Environment komplett neu ohne reledmac
\makeatletter
\renewenvironment{theindex}{%
  \section*{\indexname}%
  \setlength{\parindent}{0pt}%
  \setlength{\parskip}{0pt plus 0.3pt}%
  \let\item\@idxitem
}{%
  \clearpage
}
\makeatother

\IfFileExists{\jobname-pw.ind}{\input{\jobname-pw.ind}}{}

\end{document}

      