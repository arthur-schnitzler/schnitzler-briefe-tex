%% latex-korrekturansicht-vorspann.tex
%% Vorspann für die Korrekturansicht.
%% Lädt die gemeinsame Datei latex-vorspann.tex mit gesetztem Schalter.

\newif\ifkorrekturansicht
\korrekturansichttrue

\input{../tex-inputs/latex-vorspann}


               \section[Paul Goldmann an Arthur Schnitzler, 21. 4. {[}1894{]}]{ Paul Goldmann an Arthur Schnitzler, 21. 4. {[}1894{]}}\nopagebreak\mylabel{v}\rehead{ }\normalsize\beginnumbering\briefempfaengerindex{Schnitzler, Arthur@\textsc{Schnitzler, Arthur}!zzzGoldmann, Paul@\emph{von Paul Goldmann}!1894-04-212@{21. 4. {[}1894{]}}|(be} \toendnotes[C]{\smallbreak\pagebreak[2]} \Standort{DLA, A:Schnitzler, HS.NZ85.1.3164.}
\physDesc{Brief, 2 Blätter, 7 Seiten
\newline{}Handschrift: schwarze Tinte, deutsche Kurrent
\newline{}Schnitzler: 1) mit Bleistift auf dem ersten Blatt die Jahreszahl »94« vermerkt 2) mit rotem Buntstift zwei Unterstreichungen}\toendnotes[C]{\smallbreak}\pstart
           \noindent{}{\pb}\textcolor{gray}{\textbf{\textcolor{brown}{Frankfurter Zeitung}{}\ledrightnote{\textcolor{brown}{Frankfurter Zeitung}}.}}\hfill \textsc{\textcolor{pink}{Paris}{}\ledrightnote{\textcolor{pink}{Paris}}}, \textcolor{gray}{2}1. April.\pend
           \pstart
           \textcolor{gray}{\textbf{(\textcolor{brown}{Gazette de
                     Francfort}{}\ledrightnote{\textcolor{brown}{Frankfurter Zeitung}}.)}}\pend
           \pstart
           \textcolor{gray}{\textbf{Directeur \textbf{M. \textcolor{blue}{L. Sonnemann}{}\ledrightnote{\textcolor{blue}{Leopold Sonnemann}}}.}}\pend
           \pstart
           \textcolor{gray}{\textbf{\begin{otherlanguage}{french}Journal politique, financier,\end{otherlanguage}}}\pend
           \pstart
           \textcolor{gray}{\textbf{\begin{otherlanguage}{french}commercial et litteraire.\end{otherlanguage}}}\pend
           \pstart
           \textcolor{gray}{\textbf{\begin{otherlanguage}{french}\textbf{Paraissant trois fois par jour}\end{otherlanguage}}}\pend
           \pstart
           \textcolor{gray}{\textbf{–}}\pend
           \pstart
           \textcolor{gray}{\textbf{\begin{otherlanguage}{french}\textbf{Bureaux à \textcolor{pink}{Paris}{}\ledrightnote{\textcolor{pink}{Paris}}:}\end{otherlanguage}}}\pend
           \pstart
           \textcolor{gray}{\textbf{\begin{otherlanguage}{french}\textcolor{pink}{rue Richelieu 75}{}\ledrightnote{\textcolor{pink}{rue Richelieu}}.\end{otherlanguage}}}\pend
           \pstart\center{}Mein lieber Arthur,\pend\pstart
           Von morgen ab wechſele ich meine Adreſſe, die fortan
               lautet: \textsc{\textcolor{pink}{\uline{24. Rue Feydeau}}{}\ledrightnote{\textcolor{pink}{rue Feydeau}}}.\pend
           \pstart
           Ich verzichte darauf, Dir \strikeout{zu ſ\textcolor{gray}{a}} jedes mal zu ſagen, eine wie große Freude Du mir ſtets mit Deinen lieben
               Briefen machſt. Du ahnſt nicht, wie wohl mir Deine treue Freundſchaft thut. Ein
               Feſttag in meinem armen Leben. Und ich bin Dir ſo von Herzen dankbar.\pend
           \pstart
           Ich habe mich ſchon {\pb}gefreut, daß Du mir die
               Bekanntſchaft mit Fräulein \textsc{\textcolor{blue}{Sandrock}{}\ledrightnote{\textcolor{blue}{Adele Sandrock}}} vermittelt, und ich danke Dir ſehr für dieſe neue intereſſante Beziehung.\pend
           \pstart
           \textsc{\textcolor{blue}{Albert}{}\ledrightnote{\textcolor{blue}{Henri Albert}}} habe ich einige Tage lang nicht geſehen. Ich glaube, er wird ſich nun bald an
               Deine \textcolor{green}{Überſetzung}{}\ledrightnote{→\textcolor{green}{Les Emplettes de Noël}} machen. Auch
               die Frage der Aufführung an einem hieſigen Theater haben wir oft erörtert. Wir
               verkennen aber \textcolor{blue}{Beide}{}\ledrightnote{→\textcolor{blue}{Henri Albert}} nicht
               die Schwierigkeiten. Fremde Stücke führen hier überhaupt nur die freien Bühnen auf,
               alſo »\textsc{\textcolor{brown}{Théâtre Libre}{}\ledrightnote{\textcolor{brown}{Théâtre Libre}}}« und »\textsc{\textcolor{brown}{Oeuvre}{}\ledrightnote{\textcolor{brown}{Théâtre de l'Œuvre}}}«. Während Du alſo bei den übrigen Theatern kaum {\pb}ankommen könnteſt, weil Du ein deutſscher Dichter
               biſt, ſo ſteht Dir bei den beiden letz{[}t{]}genannten der Umſtand
               entgegen, daß Du in Geiſt und Sprache zu fein und zu franzöſiſch biſt. Die Freien
               Bühnen ſuchen in den deutſchen Stücken das für \textsc{\textcolor{pink}{Paris}{}\ledrightnote{\textcolor{pink}{Paris}}} Fremdartige: Myſticismus, Romantik, überhaupt die germaniſche Note. Der \textcolor{blue}{Director}{}\ledrightnote{→\textcolor{blue}{Aurélien-Marie Lugné-Poe}} des »\textsc{\textcolor{brown}{Oeuvre}{}\ledrightnote{\textcolor{brown}{Théâtre de l'Œuvre}}}« bereitet für die nächſte \textsc{Saison} zum Beiſpiel als
               beſondere Delikateſſe \textsc{\textcolor{blue}{Schillers}{}\ledrightnote{\textcolor{blue}{Friedrich von Schiller}}} »\textcolor{green}{Räuber}{}\ledrightnote{\textcolor{green}{Die Räuber}}« vor. Kurzum, die
               Aufführungs-Chancen ſtehen nicht gut für Dich. Ich habe mir bereits ebenſo redlich
               als vergeblich Mühe gegeben. Trotzdem gebe ichs nicht auf; eine {\pb}Möglichkeit kann ſich immer noch bieten. Vielleicht
               gelingt es, für die \label{K_L02617-1v}\edtext{»Wiener
                  Schule«}{\lemma{\textnormal{\emph{»Wiener
                  Schule«}}}\Cendnote{\textnormal{Das kann als Hinweis gelesen
                  werden, dass es noch keinen etablierten Begriff für die neuere Literaturströmung
                  gab, die dann später, mit propagandistischem Zutun von \textcolor{blue}{Hermann Bahr}, als »Jung-Wien« in die Literaturgeschichte
                  einging. (Der Begriff »Jung Wien« war zu dem Zeitpunkt bereits in Verwendung, vgl. Paul Goldmann an Arthur Schnitzler, 16. 5. 1891, vgl. A. S.: \emph{Tagebuch}, 17. 3. 1890 und den
                     gleichnamigen \textcolor{brown}{Verein}, der sich
                     zumindest zwischen 17. 3. 1891 und
                     5. 5. 1891 wöchentlich traf)}}}\label{K_L02617-1h}
               in den \textsc{Revuen} Skandal zu machen, ſo daß man dann auch nach
               ihrem Theater verlangt. Auch ein in \textcolor{pink}{Deutschland}{}\ledrightnote{\textcolor{pink}{Deutschland}}
               davongetragener großer Erfolg würde Dir ſehr für \textsc{\textcolor{pink}{Paris}{}\ledrightnote{\textcolor{pink}{Paris}}} zu Statten kommen \textsc{etc}. Alles Dich betreffende
               Literariſche will Dir übrigens \label{K_L02617-4v}\edtext{\textsc{\textcolor{blue}{Albert}{}\ledrightnote{\textcolor{blue}{Henri Albert}}} direct ſchreiben}{\lemma{\textnormal{\emph{Albert direct ſchreiben}}}\Cendnote{\textnormal{Das verzögerte
                  sich, \textcolor{blue}{Albert}s Brief ist mit 23. 5. 1894 datiert. Das Projekt einer Aufführung
                  wird in einem Satz abgehandelt: »Für das ›\textcolor{green}{Abschiedsouper}‹ denke
                     ich einen Versuch an einer hiesigen Freien Bühne zu machen«. (\emph{DLA}, HS.1985.1.2331,2.)}}}\label{K_L02617-4h}.\pend
           \pstart
           Deine große Productivität, über die \substVorne{}\textsuperscript{di\textcolor{gray}{r}}\substDazwischen{}mir\substHinten{} Deine Briefe berichten, freut mich von Herzen. Ich möchte gern bei
               Gelegenheit etwas von Deinen \label{K_L02617-6v}\edtext{neuen
                  Stücken}{\lemma{\textnormal{\emph{neuen
                  Stücken}}}\Cendnote{\textnormal{Am 29. 3. 1894 hatte \textcolor{blue}{Schnitzler} eine zweite Fassung des später \emph{\textcolor{green}{Liebelei}} genannten Stücks beendet. Am 14. 6. 1894 begann er eine
                  dritte Fassung. Ein nur als späteres Typoskript überlieferter Text ist zeitlich
                  dazwischen angesiedelt, was belegt, dass \textcolor{blue}{Schnitzler} weiter daran arbeitete. (A. S.: \emph{\textcolor{green}{Liebelei}}. Historisch-kritische Ausgabe. Hg. Peter Michael
                     Braunwarth, Gerhard Hubmann und Isabella Schwentner. Berlin, Boston:
                        \emph{de Gruyter}{ }2014 (Werke in historisch-kritischen Ausgaben, hg. Konstanze
                     Fliedl), S. 5.) Ansonsten beschäftigte sich \textcolor{blue}{Schnitzler} in diesen Tagen laut seinem \emph{\textcolor{green}{Tagebuch}} vor allem mit Prosawerken: \emph{\textcolor{green}{Sterben}}, \emph{\textcolor{green}{Geschichte vom
                     greisen Dichter}} (\emph{\textcolor{green}{Später Ruhm}}) und \emph{\textcolor{green}{Die kleine Komödie}}.}}}\label{K_L02617-6h} hören. Daß Du \strikeout{V\textcolor{gray}{e}} »verdichteſt«, iſt gewiß recht. Ich werde ein {\pb}immer überzeugterer Anhänger von Kürze und Einfachheit.\pend
           \pstart
           Was Du mir über \substVorne{}\textsuperscript{\textcolor{gray}{Deine}}\substDazwischen{}meine\substHinten{} letzte \label{K_L02617-2v}\edtext{\textcolor{green}{Arbeit}{}\ledrightnote{→\textcolor{green}{[?? Feuilleton über Charles Meunier]}}}{\lemma{\textnormal{\emph{Arbeit}}}\Cendnote{\textnormal{wohl \emph{\textcolor{green}{[??
                     Feuilleton über Charles Meunier XXXX]}}, siehe Paul Goldmann an Arthur Schnitzler, 3. 4. [1894]}}}\label{K_L02617-2h} ſchreibſt, iſt eitel Güte und Freundſchaft. Aber außer Dir und ſonſt noch ein
               Paar lieben Leuten habe ich kein Publikum. Meine Erfolge ſind rein moraliſcher Natur,
               – kein materielles Vorwärtskommen. Meine Laufbahn iſt auf ihrem Gipfel angelangt –
               der niedrig genug iſt – und jetzt gibt es nur ein hinunterſteigen.\pend
           \pstart
           {\pb}Mein \textcolor{blue}{Schwager}{}\ledrightnote{→\textcolor{blue}{Josef Rosengart}} meint, einer der Hauptgründe des mangelnden
               Heilerfolges ſei der Umſtand, daß mir die geiſtige Ruhe während der Kur gefehlt hat.
               Es iſt etwas Richtiges daran. Wenn ich nicht geſund werde und nimmer geſund werden
               kann, ſo liegt das auch an dem anſtregenden Berufe. Darum ſoll ich wenigſtens auf 4
               Wochen nach \textcolor{pink}{Frankfurt}{}\ledrightnote{\textcolor{pink}{Frankfurt am Main}}, um in Ruhe behandelt
               werden zu können. Freilich war es den ganzen Winter lang mein Traum, im Herbſt mit
               Dir zu reiſen. Nun muß ich darauf verzichten. Das thut mir in der Seele {\pb}weh. Aber es war ſo ſelbſtverſtändlich, daß ich auf
               dieſen Wunſch, weil er mir gar ſo lieb war, würde verzichten müſſen.\pend
           \pstart
           Grüß’ Dich Gott, mein lieber Freund! Sei recht froh! Und ſchreib’ mir bald!\pend
           \pstart
           In Treue {\\[\baselineskip]}Dein {\\[\baselineskip]}\spacefill\mbox{Paul Goldmann.}\pend
           \leftskip=0em{}\endnumbering\briefempfaengerindex{Schnitzler, Arthur@\textsc{Schnitzler, Arthur}!zzzGoldmann, Paul@\emph{von Paul Goldmann}!1894-04-212@{21. 4. {[}1894{]}}|)be}\mylabel{h}  \normalsize

\doendnotes{C}
\bigskip
\vfill

\clearpage

\footnotesize

\lohead{\textsc{register}}

% Definiere theindex-Environment komplett neu ohne reledmac
\makeatletter
\renewenvironment{theindex}{%
  \section*{\indexname}%
  \setlength{\parindent}{0pt}%
  \setlength{\parskip}{0pt plus 0.3pt}%
  \let\item\@idxitem
}{%
  \clearpage
}
\makeatother

\IfFileExists{\jobname-pw.ind}{\input{\jobname-pw.ind}}{}

\end{document}

      