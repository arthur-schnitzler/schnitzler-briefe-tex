%% latex-korrekturansicht-vorspann.tex
%% Vorspann für die Korrekturansicht.
%% Lädt die gemeinsame Datei latex-vorspann.tex mit gesetztem Schalter.

\newif\ifkorrekturansicht
\korrekturansichttrue

\input{../tex-inputs/latex-vorspann}


\renewcommand{\erwaehntePersonen}{Personen: Eva Marie Goldmann, Rudolf Lothar, Olga Schnitzler, Lili Schnitzler, Heinrich Schnitzler}
\renewcommand{\erwaehnteInstitutionen}{Institutionen: k. k. Post- und Telegraphenverwaltung}
\renewcommand{\erwaehnteOrte}{Orte: Bahnhof Payerbach-Reichenau, Berlin, Deutsches Theater Berlin, Edlach, Hotel Edlacherhof, Niederösterreich, Semmering, Wien}
\renewcommand{\erwaehnteWerke}{Werke: Faust bei Reinhardt, Pester Lloyd}
\section[ Paul Goldmann an Arthur Schnitzler, 26. 4. 1909]{Paul Goldmann an Arthur Schnitzler, 26. 4. 1909}
\nopagebreak\mylabel{v}
\rehead{ }\normalsize\beginnumbering\briefempfaengerindex{Schnitzler, Arthur@\textsc{Schnitzler, Arthur}!zzzGoldmann, Paul@\emph{von Paul Goldmann}!1909-04-261@{26. 4. 1909}|(be}
\toendnotes[C]{\smallbreak\pagebreak[2]}\Standort{DLA, A:Schnitzler, HS.NZ85.1.3175.}
\physDesc{Brief, 1 Blatt, 3 Seiten, 1022 Zeichen
\newline{}Handschrift: schwarze Tinte, deutsche Kurrent
\newline{}Schnitzler: mit Bleistift »\textcolor{blue}{Goldm{[}ann{]}}« vermerkt }\toendnotes[C]{\smallbreak}
\pstart
           \noindent{}{\pb}\textcolor{gray}{\textbf{\textcolor{pink}{HÔTEL EDLACHERHOF}{}\ledrightnote{\textcolor{pink}{Hotel Edlacherhof}}}}\pend
           
\pstart
           \raggedleft{}\textcolor{gray}{\textbf{IN \textcolor{pink}{EDLACH}{}\ledrightnote{\textcolor{pink}{Edlach}}, \textcolor{pink}{N.-Ö.}{}\ledrightnote{\textcolor{pink}{Niederösterreich}}}}\pend
           
\pstart
           \noindent{}\raggedleft{}\textcolor{gray}{\textbf{\textbf{\textcolor{pink}{Südbahnstation Payerbach-Reichenau}{}\ledrightnote{\textcolor{pink}{Bahnhof Payerbach-Reichenau}}}}}\pend
           
\pstart
           \noindent{}\textcolor{gray}{\textbf{Telegramm-Adresse:}}{\\}\textcolor{gray}{\textbf{\textbf{\textcolor{pink}{EDLACHERHOF}{}\ledrightnote{\textcolor{pink}{Hotel Edlacherhof}}, \textcolor{pink}{EDLACH}{}\ledrightnote{\textcolor{pink}{Edlach}}.}}}{\\}\textcolor{gray}{\textbf{INTERURBAN TELEPHON}}{\\}\textcolor{gray}{\textbf{\textbf{EDLACH Nr. 1.}}}\pend
           
\pstart
           \textcolor{gray}{\textbf{\textcolor{brown}{K. k. Post- und Telegraphen-Amt}{}\ledrightnote{\textcolor{brown}{k. k. Post- und Telegraphenverwaltung}}}}\hfill \textcolor{gray}{\textbf{\textcolor{pink}{Edlacherhof}{}\ledrightnote{\textcolor{pink}{Hotel Edlacherhof}},}}{ }26. 4. 09.\pend
           
\pstart
           \textcolor{gray}{\textbf{\textcolor{pink}{Edlach}{}\ledrightnote{\textcolor{pink}{Edlach}}.}}\pend
           
\pstart\center{}Lieber Freund,\pend
\pstart
           Beifolgendes \label{K_L03468-1v}\edtext{\textcolor{green}{Feuilleton}{}\ledrightnote{{$\rightarrow$}\textcolor{green}{Faust bei Reinhardt}}}{\lemma{\textnormal{\emph{Feuilleton}}}\Cendnote{\textnormal{Höchstwahrscheinlich Bezug auf \textcolor{blue}{Rudolf Lothar}: \emph{\textcolor{green}{Faust bei Reinhardt}}. In: \emph{\textcolor{green}{Pester Lloyd}}, Jg. 46, Nr. 95, 22. 4. 1909, Morgenblatt, S. 1–2. Das \textcolor{green}{Feuilleton} beginnt wie folgt: »\textcolor{green}{Fünfundzwanzig Jahre sind es
                        her, da nahmen zwei junge Leute, die Poeten werden wollten, Abschied von \textcolor{pink}{Wien}. Sie hatten die Absicht, die Welt zu
                        sehen und ihr erstes Ziel war \textcolor{pink}{Berlin}.
                        Der eine dieser beiden Wanderer war \textcolor{blue}{Arthur
                           Schnitzler}, der andere war \textcolor{blue}{ich}. Wir kamen mittags in \textcolor{pink}{Berlin} an und saßen abends schon im Theater. Im \textcolor{pink}{Deutschen Theater}.}« (S. 1) \textcolor{blue}{Lothar} hatte sich
                  vermutlich an die gemeinsame \textcolor{pink}{Berlin}-Reise im
                  Frühjahr 1888 erinnert. Am Tag der Ankunft waren sie
                  jedenfalls nicht im \textcolor{pink}{Deutschen Theater} gewesen
                     (vgl. A. S.: \emph{Tagebuch}, 5. 4. 1888).}}}\label{K_L03468-1h}
               von \textsc{\textcolor{blue}{Rudolf Lothar}{}\ledrightnote{\textcolor{blue}{Rudolf Lothar}}} wird Dich vielleicht ebenſo amüſiren, wie es mich amüſirt hat. \pend
           
\pstart
           \textcolor{blue}{Wir}{}\ledrightnote{{$\rightarrow$}\textcolor{blue}{Eva Marie Goldmann}} haben acht Tage der Ruhe
               in dem reizenden \textcolor{pink}{Edlach}{}\ledrightnote{\textcolor{pink}{Edlach}} verbracht, das ich Dir
               nicht dringend genug \label{K_L03468-2v}\edtext{empfehlen}{\lemma{\textnormal{\emph{empfehlen}}}\Cendnote{\textnormal{\textcolor{blue}{Schnitzler} hatte \textcolor{pink}{Edlach} bereits gekannt.}}}\label{K_L03468-2h} kann, wenn Du fern von allem
               mondianen Getriebe (wie es in den \textsc{Hotels} auf dem Gipfel des
                  \textcolor{pink}{Semmering}{}\ledrightnote{\textcolor{pink}{Semmering}} herrſcht) in erfriſchender Luft {\pb}Dich eine Zeit lang erholen willſt. Heut kehren wir nach \textcolor{pink}{Wien}{}\ledrightnote{\textcolor{pink}{Wien}} zurück, von wo aus wir in einigen Tagen die Rückreiſe nach \textcolor{pink}{Berlin}{}\ledrightnote{\textcolor{pink}{Berlin}} antreten.\pend
           
\pstart
           Aufſuchen konnte ich Dich vor meiner Abreiſe nach \textcolor{pink}{Edlach}{}\ledrightnote{\textcolor{pink}{Edlach}} nicht mehr, weil ich buchſtäblich keine Stunde frei hatte.\pend
           
\pstart
           Die \label{K_L03468-3v}\edtext{Spannung}{\lemma{\textnormal{\emph{Spannung}}}\Cendnote{\textnormal{Bezug unklar}}}\label{K_L03468-3h} zwiſchen unſeren beiderſeitigen \textcolor{blue}{Frauen}{}\ledrightnote{{$\rightarrow$}\textcolor{blue}{Eva Marie Goldmann}{\newline}{$\rightarrow$}\textcolor{blue}{Olga Schnitzler}} wird ſich
               hoffentlich beilegen laſſen. Jedenfalls aber wird zwiſchen uns Beiden hoffentlich
               Alles ſo bleiben, wie bisher.\pend
           
\pstart
           Ich wünſche Dir einen \label{K_L03468-4v}\edtext{zweiten
                  Sohn}{\lemma{\textnormal{\emph{zweiten
                  Sohn}}}\Cendnote{\textnormal{\textcolor{blue}{Olga Schnitzler} war mit \textcolor{blue}{Lili Schnitzler} schwanger. Sie wurde am 13. 9. 1909
                  geboren.}}}\label{K_L03468-4h}, der \strikeout{ſ\textcolor{gray}{o}} ein {\pb}ebenſo prächtiger Burſch ſein möge, wie
               der \textcolor{blue}{erſte}{}\ledrightnote{{$\rightarrow$}\textcolor{blue}{Heinrich Schnitzler}}, – oder, wenn Du
               Dir eine Tochter wünſcheſt, ſo bin ich auch mit einer Tochter einverſtanden, – u. bin
               mit herzlichen Grüßen (auch von meiner \textcolor{blue}{Frau}{}\ledrightnote{{$\rightarrow$}\textcolor{blue}{Eva Marie Goldmann}})\pend
           
\pstart
           Dein {\\[\baselineskip]}\spacefill\mbox{Paul Goldmann.}\pend
           \leftskip=0em{}\endnumbering\briefempfaengerindex{Schnitzler, Arthur@\textsc{Schnitzler, Arthur}!zzzGoldmann, Paul@\emph{von Paul Goldmann}!1909-04-261@{26. 4. 1909}|)be}\mylabel{h}  \normalsize

\doendnotes{C}
\bigskip
\vfill

\clearpage

\footnotesize

\lohead{\textsc{register}}

% Definiere theindex-Environment komplett neu ohne reledmac
\makeatletter
\renewenvironment{theindex}{%
  \section*{\indexname}%
  \setlength{\parindent}{0pt}%
  \setlength{\parskip}{0pt plus 0.3pt}%
  \let\item\@idxitem
}{%
  \clearpage
}
\makeatother

\IfFileExists{\jobname-pw.ind}{\input{\jobname-pw.ind}}{}

\end{document}

      