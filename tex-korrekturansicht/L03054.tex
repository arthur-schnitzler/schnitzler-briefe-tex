%% latex-korrekturansicht-vorspann.tex
%% Vorspann für die Korrekturansicht.
%% Lädt die gemeinsame Datei latex-vorspann.tex mit gesetztem Schalter.

\newif\ifkorrekturansicht
\korrekturansichttrue

\input{../tex-inputs/latex-vorspann}


\renewcommand{\erwaehnteInstitutionen}{Institutionen: k. u. k. Kriegsministerium}
\renewcommand{\erwaehnteOrte}{Orte: Berlin, Dessauer Straße, Wien}
\renewcommand{\erwaehnteWerke}{Werke: Berliner Börsen-Courier, Lieutenant Gustl. Novelle, Neue Freie Presse, [Telegramm zu den Maßregelungen der Militärbehörde resp. Lieutenant Gustl]}
\section[ Paul Goldmann an Arthur Schnitzler, 11. 1. {[}1901{]}]{Paul Goldmann an Arthur Schnitzler, 11. 1. {[}1901{]}}
\nopagebreak\mylabel{v}
\rehead{ }\normalsize\beginnumbering\briefempfaengerindex{Schnitzler, Arthur@\textsc{Schnitzler, Arthur}!zzzGoldmann, Paul@\emph{von Paul Goldmann}!1901-01-111@{11. 1. {[}1901{]}}|(be}
\toendnotes[C]{\smallbreak\pagebreak[2]}\Standort{DLA, A:Schnitzler, HS.NZ85.1.3171.}
\physDesc{Brief, 1 Blatt, 2 Seiten
\newline{}Handschrift: blaue Tinte, deutsche Kurrent
\newline{}Schnitzler: mit Bleistift das Jahr »{[}1{]}901« vermerkt }\toendnotes[C]{\smallbreak}
\pstart
           \noindent{}\raggedleft{}{\pb}\textcolor{pink}{\textcolor{gray}{\textbf{DESSAUERSTRASSE 19}}}{}\ledrightnote{\textcolor{pink}{Dessauer Straße}}\pend
           
\pstart
           \textcolor{pink}{Berlin}{}\ledrightnote{\textcolor{pink}{Berlin}}, 11. Januar.\pend
           
\pstart\center{}Mein lieber Freund,\pend
\pstart
           Im »\textcolor{green}{Börſencourier}{}\ledrightnote{\textcolor{green}{Berliner Börsen-Courier}}« finde ich ein \label{K_L03054-1v}\edtext{\textcolor{green}{Telegramm}{}\ledrightnote{{$\rightarrow$}\textcolor{green}{[Telegramm zu den Maßregelungen der Militärbehörde resp. Lieutenant Gustl]}}}{\lemma{\textnormal{\emph{Telegramm}}}\Cendnote{\textnormal{XXXX}}}\label{K_L03054-1h} über \label{K_L03054-2v}\edtext{Maßregelungen}{\lemma{\textnormal{\emph{Maßregelungen}}}\Cendnote{\textnormal{\emph{\textcolor{green}{Lieutenant Gustl}} stieß aufgrund seiner
                  offenen Kritik an Militär und Gesellschaft schnell auf Widerstand seitens Armee
                  und Regierung. \textcolor{blue}{Schnitzler} wurde infolge u.
                  a. seines Offiziersstandes enthoben.}}}\label{K_L03054-2h}, die Dir die \textcolor{brown}{Militärbehörde}{}\ledrightnote{{$\rightarrow$}\textcolor{brown}{k. u. k. Kriegsministerium}} wegen des »\textcolor{green}{Lieutnant Guſtl}{}\ledrightnote{\textcolor{green}{Lieutenant Gustl. Novelle}}« angedroht habe. Ich bin lebhaft beunruhigt
               und bitte, mir umgehend mitzutheilen, was vorgeht. Wäre es Dir möglich, mir ein
                  complet{[}t{]}es Exemplar der \textcolor{green}{Erzählung}{}\ledrightnote{{$\rightarrow$}\textcolor{green}{Lieutenant Gustl. Novelle}} zu überſenden? {\pb}Ich habe \textcolor{green}{ſie}{}\ledrightnote{{$\rightarrow$}\textcolor{green}{Lieutenant Gustl. Novelle}} bisher nicht geleſen, weil in der
                  \label{K_L03054-3v}\edtext{Nummer der \textcolor{green}{N. Fr. Pr.}{}\ledrightnote{\textcolor{green}{Neue Freie Presse}}}{\lemma{\textnormal{\emph{Nummer der N. Fr. Pr.}}}\Cendnote{\textnormal{\textcolor{blue}{Arthur Schnitzler}: \emph{\textcolor{green}{Lieutenant Gustl}}. In: \emph{\textcolor{green}{Neue Freie Presse}}, Nr. 13053, 25. 12. 1900, Morgenblatt, S. 34–41.}}}\label{K_L03054-3h}, die mir
               zugegangen iſt, der Schluß fehlt.\pend
           
\pstart
           Viele Grüße! {\\[\baselineskip]}Dein \spacefill\mbox{Paul Goldmann}\pend
           \leftskip=0em{}\endnumbering\briefempfaengerindex{Schnitzler, Arthur@\textsc{Schnitzler, Arthur}!zzzGoldmann, Paul@\emph{von Paul Goldmann}!1901-01-111@{11. 1. {[}1901{]}}|)be}\mylabel{h}
\begin{anhang}
\end{anhang}\normalsize

\doendnotes{C}
\bigskip
\vfill

\clearpage

\footnotesize

\lohead{\textsc{register}}

% Definiere theindex-Environment komplett neu ohne reledmac
\makeatletter
\renewenvironment{theindex}{%
  \section*{\indexname}%
  \setlength{\parindent}{0pt}%
  \setlength{\parskip}{0pt plus 0.3pt}%
  \let\item\@idxitem
}{%
  \clearpage
}
\makeatother

\IfFileExists{\jobname-pw.ind}{\input{\jobname-pw.ind}}{}

\end{document}

      