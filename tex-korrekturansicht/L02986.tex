%% latex-korrekturansicht-vorspann.tex
%% Vorspann für die Korrekturansicht.
%% Lädt die gemeinsame Datei latex-vorspann.tex mit gesetztem Schalter.

\newif\ifkorrekturansicht
\korrekturansichttrue

\input{../tex-inputs/latex-vorspann}


\renewcommand{\erwaehntePersonen}{Personen: Hugo von Hofmannsthal, Felix Salten, Ottilie Salten}
\renewcommand{\erwaehnteOrte}{Orte: Wien}
\renewcommand{\erwaehnteWerke}{Werke: Der Schrei der Liebe. Novelle}
\section[ Arthur Schnitzler an Felix Salten, {[}12. 10. 1903?{]}]{Arthur Schnitzler an Felix Salten, {[}12. 10. 1903?{]}}
\nopagebreak\mylabel{v}
\rehead{ }\normalsize\beginnumbering\briefempfaengerindex{Salten, Felix@\textsc{Salten, Felix}!zzzSchnitzler, Arthur@\emph{von Arthur Schnitzler}!1903-10-121@{{[}12. 10. 1903?{]}}|(be}
\toendnotes[C]{\smallbreak\pagebreak[2]}\Standort{Wienbibliothek im Rathaus, ZPH 1681, 2.1.516.}
\physDesc{Brief, 1 Blatt, 3 Seiten, 382 Zeichen
\newline{}Handschrift: Bleistift, deutsche Kurrent
\newline{}Ordnung: mit Bleistift von unbekannter Hand Nummerierung der Blätter des Konvoluts:
                                    »3«–»4« }\toendnotes[C]{\smallbreak}
\pstart
           \raggedleft{}{\pb}\uline{Montag.}\pend
           
\pstart
           lieber,{ }\textcolor{blue}{Hofmth.}{}\ledrightnote{\textcolor{blue}{Hugo von Hofmannsthal}} ſagte mir, dſs Sie \label{K_L02986-1v}\edtext{morgen Dinſtag den \textcolor{green}{Schrei}{}\ledrightnote{\textcolor{green}{Der Schrei der Liebe. Novelle}} vorleſen}{\lemma{\textnormal{\emph{morgen … vorleſen}}}\Cendnote{\textnormal{Da \textcolor{blue}{Salten}s Antwortschreiben (Felix Salten an Arthur Schnitzler, [12. 10. 1903]) von \textcolor{blue}{Schnitzler} datiert wurde, kann auch dieser Brief auf den
                     [12. 10. 1903?] datiert
                  werden.}}}\label{K_L02986-1h} werden – ich habe bisher von Ihnen keine Nachricht erhalten u denke
               an die Möglichkeit, dſs ein {\pb}Brief verloren
               gegangen wäre?\pend
           
\pstart
           Könnten Sie nicht an irgend einem Abend mit \textcolor{blue}{Otti}{}\ledrightnote{\textcolor{blue}{Ottilie Salten}} bei uns \label{K_L02986-2v}\edtext{nachtmahlen}{\lemma{\textnormal{\emph{nachtmahlen}}}\Cendnote{\textnormal{Nicht zum Abendessen,
                  aber nachmittags sahen sie sich kurz darauf, am 18. 10. 1903.}}}\label{K_L02986-2h}?
               Eſſen müſſen Sie ja doch irgendwo, und ich finde es mehr als aergerlich, {\pb}daſs man einander ſo entſchwindet.\pend
           
\pstart
           Herzlichſt Ihr {\\[\baselineskip]}\spacefill\mbox{A.}\pend
           \leftskip=0em{}\endnumbering\briefempfaengerindex{Salten, Felix@\textsc{Salten, Felix}!zzzSchnitzler, Arthur@\emph{von Arthur Schnitzler}!1903-10-121@{{[}12. 10. 1903?{]}}|)be}\mylabel{h}  \normalsize

\doendnotes{C}
\bigskip
\vfill

\clearpage

\footnotesize

\lohead{\textsc{register}}

% Definiere theindex-Environment komplett neu ohne reledmac
\makeatletter
\renewenvironment{theindex}{%
  \section*{\indexname}%
  \setlength{\parindent}{0pt}%
  \setlength{\parskip}{0pt plus 0.3pt}%
  \let\item\@idxitem
}{%
  \clearpage
}
\makeatother

\IfFileExists{\jobname-pw.ind}{\input{\jobname-pw.ind}}{}

\end{document}

      