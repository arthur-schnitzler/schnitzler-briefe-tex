%% latex-korrekturansicht-vorspann.tex
%% Vorspann für die Korrekturansicht.
%% Lädt die gemeinsame Datei latex-vorspann.tex mit gesetztem Schalter.

\newif\ifkorrekturansicht
\korrekturansichttrue

\input{../tex-inputs/latex-vorspann}


               \section[Olga, Arthur Schnitzler, Hugo und Gerty Hofmannsthal an Max Mell, {[}18.? 7. 1907{]}]{ Olga, Arthur Schnitzler, Hugo und Gerty Hofmannsthal an Max Mell,
                    {[}18.? 7. 1907{]}}\nopagebreak\mylabel{v}\rehead{ }\normalsize\beginnumbering\briefempfaengerindex{Mell, Max@\textsc{Mell, Max}!zzzHofmannsthal, Gertrude von@\emph{von Gertrude von Hofmannsthal}!1907-07-181@{{[}18.? 7. 1907{]}}|(be}\briefempfaengerindex{Mell, Max@\textsc{Mell, Max}!zzzHofmannsthal, Hugo von@\emph{von Hugo von Hofmannsthal}!1907-07-181@{{[}18.? 7. 1907{]}}|(be}\briefempfaengerindex{Mell, Max@\textsc{Mell, Max}!zzzSchnitzler, Olga@\emph{von Olga Schnitzler}!1907-07-181@{{[}18.? 7. 1907{]}}|(be}\briefempfaengerindex{Mell, Max@\textsc{Mell, Max}!zzzSchnitzler, Arthur@\emph{von Arthur Schnitzler}!1907-07-181@{{[}18.? 7. 1907{]}}|(be} \toendnotes[C]{\smallbreak\pagebreak[2]} \Standort{Wienbibliothek im Rathaus, H.I.N.-207636.}
\physDesc{Bildpostkarte
\newline{}Handschrift Arthur Schnitzler: Bleistift, deutsche Kurrent\newline{}Handschrift Olga Schnitzler: Bleistift, lateinische Kurrent\newline{}Handschrift Gertrude von Hofmannsthal: Bleistift, lateinische Kurrent\newline{}Handschrift Hugo von Hofmannsthal: Bleistift, deutsche Kurrent}\pstart{}{\pb}{[}hs. O. Schnitzler:{]} \textsc{Herrn Max Mell}\pend{}\pstart{}\textcolor{pink}{\textsc{Wien II/\textsubscript{2}}}{}\ledrightnote{\textcolor{pink}{II., Leopoldstadt}}\pend{}\pstart{}\textcolor{pink}{\textsc{Wittelsbachstrasse 5}}{}\ledrightnote{\textcolor{pink}{Wittelsbachstraße}}.\pend{}{\bigskip}\pstart
           \noindent{}\centering{}\textcolor{gray}{\textbf{{\pb}\textcolor{pink}{Bruneck, Pustertal (Tirol)}{}\ledrightnote{\textcolor{pink}{Bruneck}}}}\pend
           \pstart
           \centering{}{[}hs. Schnitzler:{]} Wir ſind in \textcolor{pink}{\textsc{Welsberg}-Waldbrunn}{}\ledrightnote{\textcolor{pink}{Wildbad Waldbrunn}}\pend
           \pstart
           \noindent{}{\pb}{[}hs. G. Hofmannsthal:{]} \textsc{Herzliche Grüsse}{ }\spacefill\mbox{Gerty Hofmannsthal}\pend
           \pstart
           {[}hs. Schnitzler:{]} Herzliche Grüße Ihnen und Fräulein \textcolor{blue}{\textsc{Mary}}{}\ledrightnote{\textcolor{blue}{Maria Mell}}! Ihr{ }\spacefill\mbox{ArthSchnitzler}\pend
           \pstart
           {[}hs. O. Schnitzler:{]} \textsc{Auch von mir!}{ }\spacefill\mbox{OlgaSchnitzler}\pend
           \pstart
           {[}hs. Hofmannsthal:{]} Herzlich Gruſs. Wir ſind in wenigen Tagen in \textcolor{pink}{Rodaun}{}\ledrightnote{\textcolor{pink}{Rodaun}}.{ }\spacefill\mbox{HvHofmannsthal}\pend
           \endnumbering\briefempfaengerindex{Mell, Max@\textsc{Mell, Max}!zzzHofmannsthal, Gertrude von@\emph{von Gertrude von Hofmannsthal}!1907-07-181@{{[}18.? 7. 1907{]}}|)be}\briefempfaengerindex{Mell, Max@\textsc{Mell, Max}!zzzHofmannsthal, Hugo von@\emph{von Hugo von Hofmannsthal}!1907-07-181@{{[}18.? 7. 1907{]}}|)be}\briefempfaengerindex{Mell, Max@\textsc{Mell, Max}!zzzSchnitzler, Olga@\emph{von Olga Schnitzler}!1907-07-181@{{[}18.? 7. 1907{]}}|)be}\briefempfaengerindex{Mell, Max@\textsc{Mell, Max}!zzzSchnitzler, Arthur@\emph{von Arthur Schnitzler}!1907-07-181@{{[}18.? 7. 1907{]}}|)be}\mylabel{h}  \normalsize

\doendnotes{C}
\bigskip
\vfill

\clearpage

\footnotesize

\lohead{\textsc{register}}

% Definiere theindex-Environment komplett neu ohne reledmac
\makeatletter
\renewenvironment{theindex}{%
  \section*{\indexname}%
  \setlength{\parindent}{0pt}%
  \setlength{\parskip}{0pt plus 0.3pt}%
  \let\item\@idxitem
}{%
  \clearpage
}
\makeatother

\IfFileExists{\jobname-pw.ind}{\input{\jobname-pw.ind}}{}

\end{document}

      