%% latex-korrekturansicht-vorspann.tex
%% Vorspann für die Korrekturansicht.
%% Lädt die gemeinsame Datei latex-vorspann.tex mit gesetztem Schalter.

\newif\ifkorrekturansicht
\korrekturansichttrue

\input{../tex-inputs/latex-vorspann}


\section[Arthur Schnitzler an Stefan Zweig, 4. 9. 1914]{L03777 Arthur Schnitzler an Stefan Zweig, 4. 9. 1914}
\nopagebreak\mylabel{L03777v}
\rehead{ }\normalsize\beginnumbering\briefempfaengerindex{Zweig, Stefan@\textsc{Zweig, Stefan}!zzzSchnitzler, Arthur@\emph{von Arthur Schnitzler}!1914-09-041@{4. 9. 1914}|(be}
\toendnotes[C]{\smallbreak\pagebreak[2]}\Standort{Jerusalem, National Library of Israel, ARC. Ms. Var. 305 1 58 Stefan Zweig Collection.}
\physDesc{Bildpostkarte, 543 Zeichen
\newline{}Handschrift: schwarze Tinte, deutsche Kurrent
\newline{}Versand: Stempel: »\nobreak{}\oindex{XVIII., Waehring@\textbf{XVIII., Währing}, \emph{A.ADM3}|pwk}18/\textsubscript{1} Wien
                                       110, 5. IX. 14, 9\nobreak{}«.  
\newline{}Zusatz: Postkartenmotiv mit \textcolor{blue}{Olga}\pwindex{Schnitzler, Olga 17.01.1882 – 13.01.1970@\textsc{Schnitzler, Olga} (17.01.1882 – 13.01.1970), \emph{Schauspielerin, Sängerin}|pw}
                                 und \textcolor{blue}{Heinrich}\pwindex{Schnitzler, Heinrich 09.08.1902 – 12.07.1982@\textsc{Schnitzler, Heinrich} (09.08.1902 – 12.07.1982), \emph{Regisseur, Schauspieler}|pw} links vor dem
                                 Haus und Schnitzler und \textcolor{blue}{Lili}\pwindex{Cappellini, Lili 13.09.1909 – 26.07.1928@\textsc{Cappellini, Lili} (13.09.1909 – 26.07.1928)|pw}
                                 auf dem Söller }\toendnotes[C]{\smallbreak}\pstart{}{\pb}Hrn \textsc{Dr. Stefan Zweig}\pend{}\pstart{}\textcolor{pink}{Wien VIII}\oindex{VIII., Josefstadt@\textbf{VIII., Josefstadt}, \emph{A.ADM3}|pw}{}\ledrightnote{\textcolor{pink}{VIII., Josefstadt}}\pend{}\pstart{}\textcolor{pink}{\textsc{Kochgasse 8}}\oindex{Kochgasse 8@\textbf{Kochgasse 8}, \emph{Wohngebäude (K.WHS)}|pw}{}\ledrightnote{\textcolor{pink}{Kochgasse 8}}.\pend{}{\bigskip}
\pstart
           \noindent{}\centering{}{\pb}\textcolor{green}{\textcolor{gray}{\textbf{\textcolor{pink}{Wien, XVIII, Sternwartestr. 71}\oindex{Sternwartestrasse 71@\textbf{Sternwartestraße 71}, \emph{Wohngebäude (K.WHS)}|pw}{}\ledrightnote{\textcolor{pink}{Sternwartestraße 71}}.}}}\pwindex{Wien, XVIII, Sternwartestr. 71.@\emph{Wien, XVIII, Sternwartestr. 71.}|pw}{}\ledrightnote{\textcolor{green}{Wien, XVIII, Sternwartestr. 71.}}\pend
           \vspace{1em}
\pstart
           \raggedleft{}{\pb}4. 9. 14\pend
           \vspace{0.5em}
\pstart
           lieber Herr Doctor Zweig, es iſt wohl anzunehmen, dſs Ihnen \textcolor{blue}{Unruh}\pwindex{Unruh, Fritz von 10.05.1885 – 28.11.1970@\textsc{Unruh, Fritz von} (10.05.1885 – 28.11.1970), \emph{Schriftsteller}|pw}{}\ledrightnote{\textcolor{blue}{Fritz von Unruh}} ſchon direct geſchrieben hat – jedenfalls
               richt ich Ihnen gerne einen herzlichen \label{K_L03777-1v}\edtext{Gruſs an Sie}{\lemma{\textnormal{\emph{Gruſs an Sie}}}\Cendnote{\textnormal{\textcolor{blue}{Fritz v. Unruh}\pwindex{Unruh, Fritz von 10.05.1885 – 28.11.1970@\textsc{Unruh, Fritz von} (10.05.1885 – 28.11.1970), \emph{Schriftsteller}|pwk} schrieb am
                     13. 8. 1914 an \textcolor{blue}{Schnitzler}: »In Eile, da ich auf Patrouille fort muss. Ich bitte
                     um herzliche Grüsse an \textcolor{blue}{Stefan Zweig}\pwindex{Zweig, Stefan 28.11.1881 – 23.02.1942@\textsc{Zweig, Stefan} (28.11.1881 – 23.02.1942), \emph{Schriftsteller}|pw} und
                     Dr. \textcolor{blue}{Rosenbaum}\pwindex{Rosenbaum, Richard 04.11.1867 – 25.06.1942@\textsc{Rosenbaum, Richard} (04.11.1867 – 25.06.1942), \emph{Dramaturg, Verleger}|pw}. Ich werde für die lieben
                     Bundesbrüder gern mein Leben geben.« (Ulrich K. Goldsmith: \emph{Der Briefwechsel Fritz von Unruhs mit Arthur Schnitzler}.
                        In: \emph{Modern Austrian Literature}, Jg. 10, 1977,
                     Nr. 3/4,  S. 95.)}}}\label{K_L03777-1} aus, der ſich in
               einer Karte an mich befand, die hier (wir kamen vorgeſtern an) für mich aufbewahrt
               lag – und füge ſchönſte Grüße von mir und auch von {\pb}meiner
                  \textcolor{blue}{Gattin}\pwindex{Schnitzler, Olga 17.01.1882 – 13.01.1970@\textsc{Schnitzler, Olga} (17.01.1882 – 13.01.1970), \emph{Schauspielerin, Sängerin}|pwv}{}\ledrightnote{{$\rightarrow$}\emph{\textcolor{blue}{Olga Schnitzler}}} bei. Hoffentlich
               ſehn wir Sie bald! Wollen Sie am \label{K_L03777-2v}\edtext{Montag mit uns}{\lemma{\textnormal{\emph{Montag mit uns}}}\Cendnote{\textnormal{Siehe A. S.: \emph{Tagebuch}, 7. 9. 1914.}}}\label{K_L03777-2} u \textcolor{blue}{Rosenbaum’s}\pwindex{Rosenbaum, Richard 04.11.1867 – 25.06.1942@\textsc{Rosenbaum, Richard} (04.11.1867 – 25.06.1942), \emph{Dramaturg, Verleger}|pw}\pwindex{Rosenbaum, Kory Elisabeth 26.06.1868 – 28.01.1930@\textsc{Rosenbaum, Kory Elisabeth} (26.06.1868 – 28.01.1930), \emph{Schriftstellerin}|pw}{}\ledrightnote{\textcolor{blue}{Richard Rosenbaum}{\newline}\textcolor{blue}{Kory Elisabeth Rosenbaum}} im Freien nachtmahlen? So
               erwarten \introOben{}wir\introOben{} Sie bei uns nach 6 Uhr\pend
           
\pstart
           Wir würden uns ſehr freuen\pend
           
\pstart
           Ihr{\\[\baselineskip]}\spacefill\mbox{Arthur Schnitzler}\pend
           \leftskip=0em{}\selectlanguage{ngerman}\endnumbering\briefempfaengerindex{Zweig, Stefan@\textsc{Zweig, Stefan}!zzzSchnitzler, Arthur@\emph{von Arthur Schnitzler}!1914-09-041@{4. 9. 1914}|)be}\mylabel{L03777h}  \normalsize

\doendnotes{C}
\bigskip
\vfill

\clearpage

\footnotesize

\lohead{\textsc{register}}

% Definiere theindex-Environment komplett neu ohne reledmac
\makeatletter
\renewenvironment{theindex}{%
  \section*{\indexname}%
  \setlength{\parindent}{0pt}%
  \setlength{\parskip}{0pt plus 0.3pt}%
  \let\item\@idxitem
}{%
  \clearpage
}
\makeatother

\IfFileExists{\jobname-pw.ind}{\input{\jobname-pw.ind}}{}

\end{document}

      