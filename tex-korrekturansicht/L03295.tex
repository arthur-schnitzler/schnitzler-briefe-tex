%% latex-korrekturansicht-vorspann.tex
%% Vorspann für die Korrekturansicht.
%% Lädt die gemeinsame Datei latex-vorspann.tex mit gesetztem Schalter.

\newif\ifkorrekturansicht
\korrekturansichttrue

\input{../tex-inputs/latex-vorspann}


\renewcommand{\erwaehntePersonen}{Personen: Richard Beer-Hofmann, Alfred Dreyfus, Leopold Geiringer, Paul Goldmann, Ottilie Salten, Moriz Szeps, Jakob Wassermann}
\renewcommand{\erwaehnteInstitutionen}{Institutionen: Wiener Allgemeine Montags-Zeitung}
\renewcommand{\erwaehnteOrte}{Orte: Frankreich, Niederdorf, Seeboden, Unterach am Attersee, Velden am Wörthersee, Villach, Wien}
\renewcommand{\erwaehnteWerke}{Werke: ?? [Feuilleton über Paul Goldmann], Ein Sommer in China. Reisebilder, Frankfurter Zeitung, Neue französische Humoristen, Wiener Allgemeine Montags-Zeitung, Wiener Allgemeine Rundschau, Wiener Allgemeine Zeitung}
\section[ Felix Salten an Arthur Schnitzler, 27. 7. 1899]{Felix Salten an Arthur Schnitzler, 27. 7. 1899}
\nopagebreak\mylabel{v}
\rehead{ }\normalsize\beginnumbering\briefempfaengerindex{Schnitzler, Arthur@\textsc{Schnitzler, Arthur}!zzzSalten, Felix@\emph{von Felix Salten}!1899-07-273@{27. 7. 1899}|(be}
\toendnotes[C]{\smallbreak\pagebreak[2]}\Standort{CUL, Schnitzler, B 89, A 2.}
\physDesc{Brief, 1 Blatt, 3 Seiten, 1184 Zeichen
\newline{}Handschrift: Bleistift, lateinische Kurrent
\newline{}Ordnung: mit Bleistift von unbekannter Hand nummeriert: »119« }\toendnotes[C]{\smallbreak}
\pstart
           \raggedleft{}{\pb}\textcolor{pink}{Wien}{}\ledrightnote{\textcolor{pink}{Wien}}, 27. Juli 99\pend
           
\pstart
           Lieber Freund, ich war jetzt ein paar Tage in \textcolor{pink}{Unterach}{}\ledrightnote{\textcolor{pink}{Unterach am Attersee}}, wo die \textcolor{blue}{Otti}{}\ledrightnote{\textcolor{blue}{Ottilie Salten}}
               wohnt. Nun bin ich wieder hier, und plage mich mit der \label{K_L03295-1v}\edtext{\textcolor{green}{W\textsuperscript{r} Allg Rundschau}{}\ledrightnote{\textcolor{green}{Wiener Allgemeine Rundschau}}}{\lemma{\textnormal{\emph{W\textsuperscript{r} Allg Rundschau}}}\Cendnote{\textnormal{siehe Felix Salten an Arthur Schnitzler, 21. 6. 1899}}}\label{K_L03295-1h}, die weder mir, noch dem D\textsuperscript{r}{ }\textcolor{blue}{Szeps}{}\ledrightnote{\textcolor{blue}{Moriz Szeps}} noch den Abonnenten Freude macht. Den
               Abonnenten nicht, weil sie literarisch ist, dem D\textsuperscript{r}{ }\textcolor{blue}{Szeps}{}\ledrightnote{\textcolor{blue}{Moriz Szeps}} nicht, weil die Abonnenten murren, und
               mir nicht, weil ich nun schon mit meinem Namen dabei bin, und es nicht gerne schlecht
               machen möchte. Mich verstimmt das einigermaßen, wie Sie wol denken können. Mit
                  \label{K_L03295-2v}\edtext{\textcolor{blue}{Geiringer}{}\ledrightnote{\textcolor{blue}{Leopold Geiringer}}}{\lemma{\textnormal{\emph{Geiringer}}}\Cendnote{\textnormal{\textcolor{blue}{Leopold Geiringer}?}}}\label{K_L03295-2h} ist es nichts. Es
               ist ganz wirr und nicht einen Menschen, der für \textcolor{blue}{Geirin{\pb}ger}{}\ledrightnote{\textcolor{blue}{Leopold Geiringer}}s Ideen Geld verlieren möchte. Deshalb sein \label{K_L03295-3v}\edtext{Plan mit \textcolor{blue}{BeerHofmann}{}\ledrightnote{\textcolor{blue}{Richard Beer-Hofmann}}}{\lemma{\textnormal{\emph{Plan mit BeerHofmann}}}\Cendnote{\textnormal{Womöglich sollte \textcolor{blue}{Beer-Hofmann} die neue \emph{\textcolor{green}{Wiener Allgemeine Montags-Zeitung}} finanzieren.}}}\label{K_L03295-3h}! Von mir verlangt
               er, ich solle ihm einen Capitalisten schaffen. Dann will er mir eine \textcolor{brown}{Redaction}{}\ledrightnote{{$\rightarrow$}\textcolor{brown}{Wiener Allgemeine Montags-Zeitung}}sstelle gegen – Gewinnstantheil –
               verleihen!!\pend
           
\pstart
           Ich arbeite wenig, denn die \textcolor{green}{Zeitung}{}\ledrightnote{{$\rightarrow$}\textcolor{green}{Wiener Allgemeine Montags-Zeitung}{\newline}{$\rightarrow$}\textcolor{green}{Wiener Allgemeine Zeitung}} macht mir viel Kopfzerbrechen und auch
               sonst kommt wieder einmal viel auf einmal zusammen. In ein paar Tagen fahre ich
               wieder nach \textcolor{pink}{Unterach}{}\ledrightnote{\textcolor{pink}{Unterach am Attersee}}, Schreiben Sie mir aber
                  imm\textcolor{gray}{er}hin nur hierher. Das \label{K_L03295-4v}\edtext{\textcolor{green}{Feuilleton}{}\ledrightnote{{$\rightarrow$}\textcolor{green}{?? [Feuilleton über Paul Goldmann]}} über \textcolor{blue}{Goldmann}{}\ledrightnote{\textcolor{blue}{Paul Goldmann}}}{\lemma{\textnormal{\emph{Feuilleton über Goldmann}}}\Cendnote{\textnormal{Ein Feuilleton über \textcolor{blue}{Goldmann} in der \emph{\textcolor{green}{Wiener
                     Allgemeinen Montags-Zeitung}} konnte nicht nachgewiesen werden. Im November und Dezember
                  erschienen zwei längere Auszüge aus \textcolor{blue}{Goldmann}s Reisebericht \emph{\textcolor{green}{Ein Sommer in
                     China}}, aber diese dürften hier nicht gemeint gewesen sein. Mutmaßlich
                  hatte \textcolor{blue}{Goldmann} sich auf eine
                  Vermittlungsposition beschränkt und das »über« ist als ›ein über
                  Vermittlung von \textcolor{blue}{Goldmann} erhaltenes
                  Feuilleton‹ zu lesen. Die Ausgabe vom 7. 8. 1899
                  behandelte etwa ausführlich den aktuellen Stand der \textcolor{blue}{Dreyfus}-Affäre, über die auch \textcolor{blue}{Goldmann} berichtete. Auch sind in dem \textcolor{green}{Blatt} in der kurzen Zeit seines
                  Bestehens mehrere Texte von \textcolor{pink}{fran}zösischen Autoren erschienen, mit denen \textcolor{blue}{Goldmann} bereits 1893/1894 in der \emph{\textcolor{green}{Frankfurter
                     Zeitung}} die Feuilletonreihe \emph{\textcolor{green}{Neue
                     französische Humoristen}} bestritten hatte (siehe Paul Goldmann an Arthur Schnitzler, 7. 9. [1896]).}}}\label{K_L03295-4h} erscheint in den nächsten Tagen. Ich
               sende {\pb}\textcolor{green}{es}{}\ledrightnote{{$\rightarrow$}\textcolor{green}{?? [Feuilleton über Paul Goldmann]}} Ihnen gleich.\pend
           
\pstart
           Auf Wiedersehen: hoffentlich bald. \label{K_L03295-5v}\edtext{Grüßen Sie \textcolor{blue}{Wassermann}{}\ledrightnote{\textcolor{blue}{Jakob Wassermann}} und den emsigen \textcolor{blue}{Richard}{}\ledrightnote{\textcolor{blue}{Richard Beer-Hofmann}}}{\lemma{\textnormal{\emph{Grüßen … Richard}}}\Cendnote{\textnormal{\textcolor{blue}{Jakob Wassermann} hielt sich gemeinsam mit
                     \textcolor{blue}{Schnitzler} in \textcolor{pink}{Velden am Wörthersee} auf. Am 28. 7. 1899 reisten sie weiter nach \textcolor{pink}{Villach}. \textcolor{blue}{Richard Beer-Hofmann} hielt sich im nahegelegenen \textcolor{pink}{Seeboden} auf und traf \textcolor{blue}{Schnitzler} in dieser Zeit ebenso. Am 5. 8. 1899 starteten \textcolor{blue}{Schnitzler}, \textcolor{blue}{Wassermann} und \textcolor{blue}{Beer-Hofmann} in \textcolor{pink}{Niederdorf} eine mehrtätige gemeinsame
                  Wanderung.}}}\label{K_L03295-5h}. Frl. \textcolor{blue}{Metzl}{}\ledrightnote{\textcolor{blue}{Ottilie Salten}} grüßt
               Sie.\pend
           
\pstart
           Herzlichst {\\[\baselineskip]}Ihr {\\[\baselineskip]}\spacefill\mbox{Salten}\pend
           \leftskip=0em{}\endnumbering\briefempfaengerindex{Schnitzler, Arthur@\textsc{Schnitzler, Arthur}!zzzSalten, Felix@\emph{von Felix Salten}!1899-07-273@{27. 7. 1899}|)be}\mylabel{h}  \normalsize

\doendnotes{C}
\bigskip
\vfill

\clearpage

\footnotesize

\lohead{\textsc{register}}

% Definiere theindex-Environment komplett neu ohne reledmac
\makeatletter
\renewenvironment{theindex}{%
  \section*{\indexname}%
  \setlength{\parindent}{0pt}%
  \setlength{\parskip}{0pt plus 0.3pt}%
  \let\item\@idxitem
}{%
  \clearpage
}
\makeatother

\IfFileExists{\jobname-pw.ind}{\input{\jobname-pw.ind}}{}

\end{document}

      