%% latex-korrekturansicht-vorspann.tex
%% Vorspann für die Korrekturansicht.
%% Lädt die gemeinsame Datei latex-vorspann.tex mit gesetztem Schalter.

\newif\ifkorrekturansicht
\korrekturansichttrue

\input{../tex-inputs/latex-vorspann}


               \section[Paul Goldmann an Arthur Schnitzler, 7. 10. {[}1895{]}]{ Paul Goldmann an Arthur Schnitzler, 7. 10. {[}1895{]}}\nopagebreak\mylabel{v}\rehead{ }\normalsize\beginnumbering\briefempfaengerindex{Schnitzler, Arthur@\textsc{Schnitzler, Arthur}!zzzGoldmann, Paul@\emph{von Paul Goldmann}!1895-10-071@{7. 10. {[}1895{]}}|(be} \toendnotes[C]{\smallbreak\pagebreak[2]} \Standort{DLA, A:Schnitzler, HS.NZ85.1.3165.}
\physDesc{Brief, 5 Blätter, 18 Seiten
\newline{}Handschrift Paul Goldmann: blaue Tinte, deutsche Kurrent\newline{}Handschrift M. O. Riese: schwarze Tinte, deutsche Kurrent\newline{}Beilage: handschriftlicher Brief, 1 Blatt, 1 Seite, deutsche Kurrent; im
                                 Deutschen Literaturarchiv Marbach unter der Signatur
                                 HS.NZ85.1.3166/9 eingeordnet und damit den Korrespondenzstücken des
                                 Jahres 1896 zugeordnet. Bleistiftvermerk
                                 von Schnitzler: »\textsc{\textcolor{brown}{Inst. Rud\textcolor{gray}{y}}}« 
\newline{}Schnitzler: mit rotem Buntstift vier Unterstreichungen und das Jahr »95« vermerkt }\toendnotes[C]{\smallbreak}\pstart
           \noindent{}{\pb}\textcolor{gray}{\textbf{\textbf{\textcolor{brown}{Frankfurter Zeitung}{}\ledrightnote{\textcolor{brown}{Frankfurter Zeitung}}}}}\pend
           \pstart
           \textcolor{gray}{\textbf{(\textcolor{brown}{\begin{otherlanguage}{french}Gazette de Francfort\end{otherlanguage}}{}\ledrightnote{\textcolor{brown}{Frankfurter Zeitung}}). }}\pend
           \pstart
           \textcolor{gray}{\textbf{\textbf{\begin{otherlanguage}{french}Fondateur M. \textcolor{blue}{L.
                              Sonnemann}{}\ledrightnote{\textcolor{blue}{Leopold Sonnemann}}\end{otherlanguage}.}}}\pend
           \pstart
           \begin{otherlanguage}{french}\textcolor{gray}{\textbf{\textcolor{green}{Journal}{}\ledrightnote{→\textcolor{green}{Frankfurter Zeitung}} politique,
                           financier,}}\end{otherlanguage}\hfill \textsc{\textcolor{pink}{Paris}{}\ledrightnote{\textcolor{pink}{Paris}}}, 7. Oktober.\pend
           \pstart
           \begin{otherlanguage}{french}\textcolor{gray}{\textbf{commercial et littéraire.}}\end{otherlanguage}\pend
           \pstart
           \begin{otherlanguage}{french}\textcolor{gray}{\textbf{\textbf{Paraissant trois fois par jour.}}}\end{otherlanguage}\pend
           \pstart
           \begin{otherlanguage}{french}\textcolor{gray}{\textbf{\textbf{Bureau à \textcolor{pink}{Paris}{}\ledrightnote{\textcolor{pink}{Paris}}:}}}\end{otherlanguage}\pend
           \pstart
           \begin{otherlanguage}{french}\textcolor{gray}{\textbf{\textbf{\textcolor{pink}{24. Rue Feydeau}{}\ledrightnote{\textcolor{pink}{rue Feydeau}}.}}}\end{otherlanguage}\pend
           \pstart\center{}Mein lieber Freund,\pend\pstart
           dieſer Brief trifft Dich alſo am Vorabend{ }\label{K_L02750-1v}\edtext{großer Ereigniſſe}{\lemma{\textnormal{\emph{großer Ereigniſſe}}}\Cendnote{\textnormal{Uraufführung der \emph{\textcolor{green}{Liebelei}} am 9. 10. 1895 im \textcolor{pink}{Burgtheater}}}}\label{K_L02750-1h}, oder hoffentlich ſchon am Ereignißtage
               ſelbſt. Du kannſt Dir denken, mit wie wachſendem Intereſſe ich Deine letzten lieben
               Briefe geleſen. Gern hätte ich ſie raſch beantwortet; aber bei mir iſt wieder der
               Trübſinn eingekehrt; und ich wollte nicht, daß mir allzuviel davon in die Feder
               flöſſe. Ich danke Dir \strikeout{\textcolor{gray}{×}} von Herzen, daß Du mir ſo treulich berichtet haſt. Gern \strikeout{h\textcolor{gray}{at}te} hätte ich all’ dieſe Zeit {\pb}mit Dir verlebt; aber durch Deine Briefe habe ich
               doch wenigſtens einen Wellenſchlag davon zu ſpüren bekommen. Am \strikeout{\textcolor{gray}{Schwerſten}} Schmerzlichſten iſt es mir, daß ich Mittwoch
               nicht da ſein kann. Erſtens, um raſcher zu wiſſen, wie es ausgegangen, und zweitens,
               um \strikeout{D\textcolor{gray}{ir}} mit Dir ein wenig die Zeit bis zum Abend zu verplaudern. Freilich
               hätteſt Du meiner wohl kaum bedurft. Mit großer Freude ſehe ich aus Deinen Briefen,
               wie ruhig Du biſt. Und wenn doch am Mittwoch{ }Nachmittag das Herzklopfen kommen ſollte – in jener Stunde beſonders, wo
               der Abend über den {\pb}\strikeout{\textcolor{gray}{×}\-\textcolor{gray}{×}\-\textcolor{gray}{×}\-\textcolor{gray}{×}\-\textcolor{gray}{×}\-\textcolor{gray}{×}\-\textcolor{gray}{×}}{ }\textcolor{pink}{Volksgarten}{}\ledrightnote{\textcolor{pink}{Volksgarten}} niederſinkt, eigens für Dich
               niederſinkt – ſo wirſt Du ſchon eine liebe Hand in Deiner Nähe haben, die bereit iſt,
               die Deinige zu drücken. Ich ſelbſt bin Deiner Sache ſicher. \strikeout{F\textcolor{gray}{ür}} Für mich kann es ſich nur um die Größe des Erfolges handeln; ein Mißerfolg iſt
               ausgeſchloſſen, \strikeout{d\textcolor{gray}{a}} aus dem einfachen Grunde, weil nicht das ganze \textcolor{pink}{Wien}{}\ledrightnote{\textcolor{pink}{Wien}}er Publicum plötzlich irrſinnig werden kann. Oh, ich glaube, es wird
               ſchön ſein. Vielleicht nicht allzu ſtürmiſch, aber ſchön. Und wenn ich denke, {\pb}daß Du dahin gekommen, ſtill und ehrlich, Dir \strikeout{ſ} ſelbſt getreu, und einfach Deines lieben Herzens
               Sprache redend, – ſo fühle ich, daß es ein hoher Ehrentag iſt für Dich, für den Poeten ſo ſehr wie für den Menſchen, und
               ein ſtarkes Beiſpiel für uns Alle. Ich habe das Bedürfniß, jeden dieſer Briefe mit
               Wünſchen zu füllen. Leider kann ich ja bei der ganzen Angelegenheit nichts thun, als
               Dir fortwährend »Glück!« und »Glück!« zurufen. Aber hier will ich es wenigſtens an
               den Meinigen nicht {\pb}fehlen laſſen. So kommt denn
               noch ein letzter herzinniger Wunſch, daß es gut werden möge. Damit umarme ich Dich
               und laſſe Dich Deinen Weg gehen........\pend
           \pstart
           Den Mittwoch{ }Abend werde ich mit meinen Gedanken in \textcolor{pink}{Wien}{}\ledrightnote{\textcolor{pink}{Wien}} ſein und werde verſuchen, die Zeit bis zum nächſten
                  Vormittag nicht lang zu finden. Denn, nicht wahr, Du \label{K_L02750-2v}\edtext{telegraphirſt}{\lemma{\textnormal{\emph{telegraphirſt}}}\Cendnote{\textnormal{\textcolor{blue}{Schnitzler} schickte tatsächlich ein
                  Telegramm, \textcolor{blue}{Goldmann}s Telegramm vom [10. 10. 1895?] reagiert
                  darauf.}}}\label{K_L02750-2h} mir ein paar Worte? Und dann ſchickſt Du mir auch wohl die
               Referate, ich ſende {\pb}ſie Dir umgehend zurück. Sehr
               lieb wäre es, wenn auch \textsc{\textcolor{blue}{Richard}{}\ledrightnote{\textcolor{blue}{Richard Beer-Hofmann}}} mir telegraphiren wollte; der könnte ſchon etwas ausführlicher berichten.\pend
           \pstart
           Dabei fällt mir ein, daß es am Ende vielleicht doch gut iſt, wenn ich nicht dabei
               bin. Ich hätte mich ausgenommen, wie die unverheirathete ältere Schweſter auf der
               Hochzeit der Jüngeren........\pend
           \pstart
           Dein letzter Brief war beſonders ſchön. So voll guter Stimmung, ſo zu Herzen gehend!
               Deinem \textcolor{green}{Stück}{}\ledrightnote{→\textcolor{green}{Liebelei. Schauspiel in drei Akten}} thuſt Du aber
               doch wohl Unrecht. Gar ſo {\pb}\strikeout{dün\textcolor{gray}{r}}{ }\strikeout{dün\textcolor{gray}{r}} dünn iſt es, weiß Gott, nicht. Du ſelbſt weißt, was Du hätteſt \strikeout{dazu} noch dazuthun können, der Zuſchauer aber nicht,
               und dieſem erſcheint es voll genug. Eines iſt \strikeout{r\textcolor{gray}{e}} richtig, daß die Figur des \textcolor{green}{Alten}{}\ledrightnote{→\textcolor{green}{Liebelei. Schauspiel in drei Akten}} hätte erweitert und vertieft werden können. Man hätte gern mit ihm
               nähere Bekanntſchaft gemacht. Aber den gibſt Du uns vielleicht in einem neuen Stücke.
               Und wer könnte auch den Reichthum des Lebens auf der Bühne verlangen, wie Du ſagſt?
                  \strikeout{\textcolor{gray}{×}} Das {\pb}Dramatiſche iſt ja gerade eine Auswahl
               aus der Fülle. Nur das Weſentliche gehört \strikeout{a} auf die
               Bühne; und Du weißt ſelbſt am Beſten, daß die dramatiſche Kunſt in der Aus\strikeout{\textcolor{gray}{×}} Ausſcheidung, Beſchränkung, Vereinfachung liegt. Für des Lebens Reichthum und
               Fülle \strikeout{hat das \textcolor{gray}{×}} iſt das Theater zu klein{\dotssix}\pend
           \pstart
           Es iſt ſchön, daß es mit den Proben ſo gut gegangen und daß die Leute ſo
               liebenswürdig zu Dir waren. \strikeout{Nach Allem} Nach den Namen
               der \textcolor{blue}{Schauſpieler}{}\ledrightnote{→\textcolor{blue}{Adolf von Sonnenthal}{\newline}→\textcolor{blue}{Adele Sandrock}{\newline}→\textcolor{blue}{Anna Kallina}{\newline}→\textcolor{blue}{Fanny Walbeck}{\newline}→\textcolor{blue}{Camilla Gerzhofer}{\newline}→\textcolor{blue}{Victor Kutschera}{\newline}→\textcolor{blue}{Carl von Zeska}{\newline}→\textcolor{blue}{Friedrich Mitterwurzer}{\newline}→\textcolor{blue}{Gustav Slanar}}{ }\strikeout{\textcolor{gray}{u}} und nach {\pb}dem, was Du ſchreibſt, zu
               ſchließen, wird die \textcolor{green}{Aufführung}{}\ledrightnote{→\textcolor{green}{Liebelei. Schauspiel in drei Akten}}
               eine vorzügliche ſein. Es iſt doch auch gut, wenn ein Director vor einem Stücke Angſt
               hat. So iſt er gezwungen, es zum Erfolg zu führen\strikeout{\textcolor{gray}{,}} und die beſten Kräſte ſeines Theaters dafür einzuſetzen. \textsc{\textcolor{blue}{Burckhardt}{}\ledrightnote{\textcolor{blue}{Max Eugen Burckhard}}s}{ }\strikeout{Z\textcolor{gray}{o}\textcolor{gray}{×}}{ }Haſenſüßerei, unter der Du ſoviel gelitten, kommt
               Dir hier doch am Ende zugute. So \substVorne{}\textsuperscript{läuft}\substDazwischen{}ſtellt\substHinten{} doch Alles am Ende wieder \strikeout{auf} Alles in den
               Dienſt {\pb}des Guten, ſelbſt das anfangs Hindernde. Die
               große \textcolor{blue}{Tragödin}{}\ledrightnote{→\textcolor{blue}{Adele Sandrock}} zum Beiſpiel!
               Dieſe verſtehe ich beſonders gut in der Sache. Sie hat geſehen, daß die \textcolor{green}{Rolle}{}\ledrightnote{→\textcolor{green}{Liebelei. Schauspiel in drei Akten}} vorzüglich iſt und daß
               ſie Erfolg haben wird. Das iſt doch \strikeout{\textcolor{gray}{w}} noch ein höherer Genuß, als der, \strikeout{I\textcolor{gray}{n}f} einem ehemaligen Geliebten Infamien anzuthun. So
               wird ſie \strikeout{ſü\textcolor{gray}{ß}} ſüß und zahm. Das läuft auf das heraus, was ich immer ſage: Man gebe ſich mit
               der Komödianten-Gemeinheit {\pb}nicht ab und ſchaffe
               ruhig weiter. Das unfehlbar beſte Mittel gegen \strikeout{Bühnen-} Theater-Intriguen iſt ein gutes \textcolor{green}{Stück}{}\ledrightnote{→\textcolor{green}{Liebelei. Schauspiel in drei Akten}}. Jawohl, mein Freund, der Sieg des Guten und Schönen.
               Es iſt gar nicht ſo gymnaſiaſtenhaft, daran zu glauben, wie Du ſchreibſt. Ich glaube
               immer mehr daran. Die Gemeinheit und alles Schlechte iſt ſehr ſtark hinieden; aber es
               gibt doch kaum etwas, das ſtärker iſt, als dieſe zwei Herkulaſſe: {\pb}Gut und Schön. Auch ahnſt Du gar nicht, wieviel
               gerade im Falle \textsc{Arthur Schnitzler} liegt, das Einen wieder
               mit dem Weltlauf auszuſöhnen vermag{\dotssix}\pend
           \pstart
           Reden wir ein wenig von Geſchäften. Anbei findeſt Du einen Brief, den ich nicht
               beantworten wollte, ohne Dich zu fragen. Ich rathe Dir ab, vorläufig das
               Überſetzungsrecht der »\textcolor{green}{Liebelei}{}\ledrightnote{\textcolor{green}{Liebelei. Schauspiel in drei Akten}}« zu vergeben.
               Warten wir erſt ab, wie die Dinge gehen. \textsc{Madame \textcolor{blue}{Aubry}{}\ledrightnote{\textcolor{blue}{[MMe. Georges] Aubry}}} iſt mit der \textcolor{green}{Überſetzung}{}\ledrightnote{→\textcolor{green}{La petite comédie. Mœurs viennois}}
               der {\pb}»\textcolor{green}{Kleinen
                  Komödie}{}\ledrightnote{\textcolor{green}{Die kleine Komödie}}« fertig. Ertheile ihr die Autoriſation in einem \uline{deutſchen} Briefe, den Du mir ſchicken magſt. \textsc{\textcolor{blue}{Aubry}{}\ledrightnote{\textcolor{blue}{Georges Aubry}}} hat mir verſprochen, einen kleinen \label{K_L02750-4v}\edtext{\textcolor{green}{Bericht}{}\ledrightnote{→\textcolor{green}{Thêatres. [Notre correspondant de Vienne]}}}{\lemma{\textnormal{\emph{Bericht}}}\Cendnote{\textnormal{[\textcolor{blue}{Georges Aubry}]: \emph{\textcolor{green}{Thêatres. [Notre correspondant de Vienne]}}. In: \emph{\textcolor{green}{La Liberté}}, Jg. 30, Nr. 11289, 12. 10. 1895, S. 3. Siehe dazu auch Paul Goldmann an Arthur Schnitzler, 13. 10. [1895].}}}\label{K_L02750-4h} über die Aufführung der »\textcolor{green}{Liebelei}{}\ledrightnote{\textcolor{green}{Liebelei. Schauspiel in drei Akten}}« in die »\textsc{\textcolor{green}{Liberté}{}\ledrightnote{\textcolor{green}{La Liberté}}}« zu bringen. Schon zu dieſem Zweck brauche ich das oben erbetene Telegramm. Dem
                  \textsc{\textcolor{blue}{Herzl}{}\ledrightnote{\textcolor{blue}{Theodor Herzl}}} ſollteſt Du \uline{doch} ein Feuilleton geben. Glaub’
               mir, Du kannſt es ſchreiben, es iſt Dir nur unbequem. {\pb}Du haſt doch auch ſchon kürzere Sachen gemacht, zum
               Teufel! Denk’ dir halt, daß Du es \uline{nicht} für die »\textcolor{brown}{Neue Freie Preſſe}{}\ledrightnote{\textcolor{brown}{Neue Freie Presse}}« ſchreibſt. Aber ich halte es
               für ſehr wichtig, daß Dein Name auch dort erſcheint. Daß »\textcolor{green}{Sterben}{}\ledrightnote{→\textcolor{green}{Mourir. Roman}}« bei, \textsc{\textcolor{brown}{Perrin}{}\ledrightnote{\textcolor{brown}{Éditions Perrin}}} erſcheint, iſt vortrefflich. Es iſt ein anſtändiger \textcolor{brown}{Verlag}{}\ledrightnote{→\textcolor{brown}{Éditions Perrin}}, der ſreilich wenig Verbindungen mit
               Zeitungen hat. Denn hier ſchreibt das Geſindel nur über {\pb}Bücher, wenn der Verleger dem Blatt ein Pauſchale
               zahlt. Aber laß’ gut ſein, ich \strikeout{ſchaff} ſchaff’ Dir
               ſchon eine oder die andere Beſprechung{\dotssix}\pend
           \pstart
           Was Du über »\textcolor{green}{Juliens Tagebuch}{}\ledrightnote{\textcolor{green}{Julies Tagebuch. Roman}}« ſchreibſt,
               überzeugt mich nicht. Inzwiſchen habe ich auch »\label{K_L02750-88v}\edtext{\textcolor{green}{Maria}{}\ledrightnote{\textcolor{green}{Maria. Ein Buch der Liebe}}}{\lemma{\textnormal{\emph{Maria}}}\Cendnote{\textnormal{\textcolor{blue}{Peter Nansen}: \emph{\textcolor{green}{Maria. Ein Buch der Liebe}}. Autorisierte Übersetzung aus
                     dem Dänischen von \textcolor{blue}{Mathilde Mann}. \textcolor{pink}{Berlin}: \emph{\textcolor{brown}{S.
                        Fischer}}{ }1895. (Originalausgabe: \emph{\textcolor{green}{Maria. En Bog om
                        Kjærlighed. Roman}}, 1894)}}}\label{K_L02750-88h}« geleſen. Das geſällt mir viel beſſer. Ich weiß nicht, ob es \strikeout{w\textcolor{gray}{a}} ein wahres Buch iſt; von dieſen Liebes-Dingen verſtehe ich wenig; aber es iſt
               poetiſch und ſtellenweiſe entzückend poetiſch. {\pb}In
                  »\textcolor{green}{Juliens Tagebuch}{}\ledrightnote{\textcolor{green}{Julies Tagebuch. Roman}}« mag ich vor Allem den \textcolor{green}{Mann}{}\ledrightnote{→\textcolor{green}{Julies Tagebuch. Roman}} nicht, dieſen
               Schwerenöther, dem alle Weiber zufliegen, der ſeine Syſteme mit ihnen hat, der \strikeout{J\textcolor{gray}{e}} auch in dem heißen Sturm mit \textcolor{green}{Julie}{}\ledrightnote{→\textcolor{green}{Julies Tagebuch. Roman}} ſtets den Kopf oben behält und der \textcolor{green}{Julien}{}\ledrightnote{→\textcolor{green}{Julies Tagebuch. Roman}}s Liebe in genau abgezählten Tropfen zu ſich nimmt:
               drei Eßlöffel voll und nicht mehr; das Übrige \strikeout{iſt
                     ſein\textcolor{gray}{er}} wäre ſeiner Geſundheit ſchädlich; und ſo hört er auf{[},{]}
               gerade, wo es nöthig iſt. Iſt das wirklich wahr? Du kennſt dieſe Seite des Lebens
               beſſer wie ich, {\pb}aber ich kanns nicht glauben, daß das
               wahr iſt. Gerade in dieſem \textcolor{green}{Buche}{}\ledrightnote{→\textcolor{green}{Julies Tagebuch. Roman}} fehlt mir \strikeout{des Lebensfülle} des Lebens
               Fülle. Gar ſo einfach liegen doch die Dinge nicht. Mir \strikeout{\textcolor{gray}{war} ſch} riecht \strikeout{das} das
                  \textcolor{green}{Buch}{}\ledrightnote{→\textcolor{green}{Julies Tagebuch. Roman}} zu ſehr nach \strikeout{Schreb} Schreibtiſch. In »\textcolor{green}{Maria}{}\ledrightnote{\textcolor{green}{Maria. Ein Buch der Liebe}}« iſt Wärme und Süßigkeit. Ich halte das für das erſte
               der beiden Bücher, und ich finde es unnöthig, daß \textsc{\textcolor{blue}{Nansen}{}\ledrightnote{\textcolor{blue}{Peter Nansen}}} nach der poetiſchen Liebesgeſchichte uns dieſelbe Geſchichte noch einmal »wahr«
               geſchrieben hat. Gibt es überhaupt {\pb}wahre
               Liebesgeſchichten? {\dotsfive} Das iſt vielleicht Alles ſehr \strikeout{du} dumm, was ich da ſage; aber mir fehlt etwas an dem
                  \textcolor{green}{Buche}{}\ledrightnote{→\textcolor{green}{Julies Tagebuch. Roman}}, und ich kann nicht
               recht ausdrücken, was mir fehlt{\dots}\pend
           \pstart
           Das wäre wohl Alles für heut. Bald, allerbaldigſt höre
               ich von Dir, nicht wahr?\pend
           \pstart
           Grüß’ Dich Gott, mein lieber Freund!\pend
           \pstart
           Dein treuer {\\[\baselineskip]}\spacefill\mbox{Paul Goldmann}\pend
           \leftskip=0em{}\pstart
           \noindent{}Viele Grüße an \textsc{\textcolor{blue}{Richard}{}\ledrightnote{\textcolor{blue}{Richard Beer-Hofmann}}}!\pend
           {\bigskip}\pstart
           \noindent{}{\pb}\textcolor{gray}{\textbf{\textcolor{brown}{INSTITUT RUDY}{}\ledrightnote{\textcolor{brown}{Institut Rudy}}}}\hfill \textcolor{gray}{\textbf{\textcolor{pink}{Paris}{}\ledrightnote{\textcolor{pink}{Paris}}, le}}{ }3 October \textcolor{gray}{\textbf{189}}5.\pend
           \pstart
           \textcolor{gray}{\textbf{\begin{otherlanguage}{french}FONDÉ EN\end{otherlanguage}}}{ }1860\pend
           \pstart
           \textcolor{gray}{\textbf{\begin{otherlanguage}{french}LANGUES, LETTRES, SCIENCES\end{otherlanguage}}}\pend
           \pstart
           \textcolor{gray}{\textbf{\begin{otherlanguage}{french}ARTS D’AGRÉMENT\end{otherlanguage}}}\pend
           \pstart
           \textcolor{gray}{\textbf{\textcolor{pink}{4, \begin{otherlanguage}{french}RUE CAUMARTIN\end{otherlanguage}, 4}{}\ledrightnote{\textcolor{pink}{Rue de Caumartin}}}}\pend
           \pstart
           \textcolor{gray}{\textbf{\emph{\begin{otherlanguage}{french}\textcolor{pink}{(BOULEVARD DES CAPUCINES)}{}\ledrightnote{\textcolor{pink}{Boulevard de Capucines}}\end{otherlanguage}}}}\pend
           \pstart
           \textcolor{gray}{\textbf{\begin{otherlanguage}{french}CI-DEVANT: \textcolor{pink}{7, RUE
                              ROYALE}{}\ledrightnote{\textcolor{pink}{Rue Royale}}\end{otherlanguage}}}\pend
           \pstart\center{}Sehr geehrter Herr Doctor!\pend\pstart
           Auf Empfehlung des Herrn D\textsuperscript{r}{ }\label{K_L02750-444v}\edtext{\textcolor{blue}{Gollmann}{}\ledrightnote{\textcolor{blue}{Wilhelm Gollmann}}}{\lemma{\textnormal{\emph{Gollmann}}}\Cendnote{\textnormal{\textcolor{blue}{Wilhelm Gollmann} war ein \textcolor{pink}{Wien}er Mediziner, der von \textcolor{blue}{Schnitzler} die Erlaubnis hatte, \emph{\textcolor{green}{Sterben}} ins
                  Englische zu übersetzen. Er delegierte die Aufgabe an \textcolor{blue}{Mary Hargrave}. Der Verleger \textcolor{blue}{William Heinemann} sagte aber ab, weil: »\begin{otherlanguage}{english}there has been so marked a reaction in this country of late against the
                        morbid and the horrible in fiction that I feel almost certain the book in
                        spite of its merits would be a failure here\end{otherlanguage}«. (Brief
                     von \textcolor{blue}{Wilhelm Gollmann} an \textcolor{blue}{Schnitzler},
                        21. 9. 1896, \emph{DLA},
                     85.1.3186)}}}\label{K_L02750-444h} erlaube ich mir Sie um die Adreſſe des Herrn \textsc{Schnitzler}, Schriftſteller in \textcolor{pink}{\textsc{Wien}}{}\ledrightnote{\textcolor{pink}{Wien}}, zu erſuchen, da ich mich beftreffs \label{K_L02750-23v}\edtext{Ueberſetzung}{\lemma{\textnormal{\emph{Ueberſetzung}}}\Cendnote{\textnormal{keine Übersetzung der \emph{\textcolor{green}{Liebelei}} von \textcolor{blue}{Riese} bekannt}}}\label{K_L02750-23h}{ }\introOben{}ins Franzöſische\introOben{} ſeines \textcolor{green}{Stück}{}\ledrightnote{→\textcolor{green}{Liebelei. Schauspiel in drei Akten}}es \textsc{\textcolor{green}{Liebelei}{}\ledrightnote{\textcolor{green}{Liebelei. Schauspiel in drei Akten}}} an ihn wenden möchte.\pend
           \pstart
           Ihnen im Voraus für Ihre freundliche Mühe beſtens dankend zeichne\pend
           \pstart
           Hochachtungsvoll {\\[\baselineskip]}\spacefill\mbox{\textcolor{blue}{\label{K_L02750-122v}\edtext{M O Riese}{\lemma{\textnormal{\emph{M O Riese}}}\Cendnote{\textnormal{Sprachlehrer für Deutsch und
                        Englisch in \textcolor{pink}{Paris}}}}\label{K_L02750-122h}}{}\ledrightnote{\textcolor{blue}{M. O. Riese}}}\pend
           \leftskip=0em{}\endnumbering\briefempfaengerindex{Schnitzler, Arthur@\textsc{Schnitzler, Arthur}!zzzGoldmann, Paul@\emph{von Paul Goldmann}!1895-10-071@{7. 10. {[}1895{]}}|)be}\mylabel{h}  \normalsize

\doendnotes{C}
\bigskip
\vfill

\clearpage

\footnotesize

\lohead{\textsc{register}}

% Definiere theindex-Environment komplett neu ohne reledmac
\makeatletter
\renewenvironment{theindex}{%
  \section*{\indexname}%
  \setlength{\parindent}{0pt}%
  \setlength{\parskip}{0pt plus 0.3pt}%
  \let\item\@idxitem
}{%
  \clearpage
}
\makeatother

\IfFileExists{\jobname-pw.ind}{\input{\jobname-pw.ind}}{}

\end{document}

      