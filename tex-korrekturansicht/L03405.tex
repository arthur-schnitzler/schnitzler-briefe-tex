%% latex-korrekturansicht-vorspann.tex
%% Vorspann für die Korrekturansicht.
%% Lädt die gemeinsame Datei latex-vorspann.tex mit gesetztem Schalter.

\newif\ifkorrekturansicht
\korrekturansichttrue

\input{../tex-inputs/latex-vorspann}


\renewcommand{\erwaehntePersonen}{Personen: Elvira Leontine Hervay von Kirchberg, Adele Sandrock}
\renewcommand{\erwaehnteOrte}{Orte: Wien}
\renewcommand{\erwaehnteWerke}{}
\section[ Felix Salten an Arthur Schnitzler, 11. 1. 1905]{Felix Salten an Arthur Schnitzler, 11. 1. 1905}
\nopagebreak\mylabel{v}
\rehead{ }\normalsize\beginnumbering\briefempfaengerindex{Schnitzler, Arthur@\textsc{Schnitzler, Arthur}!zzzSalten, Felix@\emph{von Felix Salten}!1905-01-111@{11. 1. 1905}|(be}
\toendnotes[C]{\smallbreak\pagebreak[2]}\Standort{CUL, Schnitzler, B 89, B 1.}
\physDesc{Brief, 1 Blatt, 1 Seite, 896 Zeichen
\newline{}Handschrift: blaue Tinte, lateinische Kurrent
\newline{}Schnitzler: mit Bleistift Vermerk: »\textsc{Salten}« 
\newline{}Ordnung: mit Bleistift von unbekannter Hand nummeriert: »198« }\toendnotes[C]{\smallbreak}
\pstart
           \raggedleft{}{\pb}\textcolor{pink}{Wien}{}\ledrightnote{\textcolor{pink}{Wien}}, 11. I. 05\pend
           
\pstart
           Lieber, jedenfalls will ich es versuchen, der \textcolor{blue}{Sandrock}{}\ledrightnote{\textcolor{blue}{Adele Sandrock}} Ihren \label{K_L03405-1v}\edtext{Brief}{\lemma{\textnormal{\emph{Brief}}}\Cendnote{\textnormal{Arthur Schnitzler an Felix Salten, 10. 1. 1905}}}\label{K_L03405-1h} begreiflich zu machen. Ich bin selbst nur Eingeladener, – was ich für nötig
               halte, zu betonen, da Frau \textcolor{blue}{v. Hervay}{}\ledrightnote{\textcolor{blue}{Elvira Leontine Hervay von Kirchberg}} sich heute bei mir, als bei dem »Veranstalter« der Sache
               bedankt hat, und ich deswegen vermuthe, die \textcolor{blue}{Sandrock}{}\ledrightnote{\textcolor{blue}{Adele Sandrock}} habe Ihnen dasselbe gesagt. Ich versprach – wenn die Sache zustande kommt, – zu lesen. Die \textcolor{blue}{Sandrock}{}\ledrightnote{\textcolor{blue}{Adele Sandrock}} wollte
               dann, dass ich auch Sie dazu anwerbe, – ich habe es aber abgelehnt, bei Ihnen zu
               interveniren. Einmal, weil es meine Sache nicht ist, den Entrepreneur zu machen, und
               dann, weil ich mir ungefähr alles das gedacht habe, was Sie mir heute schrieben.\pend
           
\pstart
           Characteristisch\footnote{\noindent{}Eben meldet sie es mir selbst. Echt \textcolor{blue}{Sandrock}!} ist nur, dass mir Frau \textcolor{blue}{v. Hervay}{}\ledrightnote{\textcolor{blue}{Elvira Leontine Hervay von Kirchberg}}{ }heute von der \textcolor{blue}{Sandrock}{}\ledrightnote{\textcolor{blue}{Adele Sandrock}} meldet, Sie hätten Ihre Mitwirkung absolut sicher zugesagt (!!) Ich
               will also versuchen, mit der \textcolor{blue}{Sandrock}{}\ledrightnote{\textcolor{blue}{Adele Sandrock}} zu
               sprechen, weiß aber im Voraus, – es ist umsonst.\pend
           \pstart Mit herzlichen Grüßen Ihr \spacefill\mbox{Salten}\pend{}\endnumbering\briefempfaengerindex{Schnitzler, Arthur@\textsc{Schnitzler, Arthur}!zzzSalten, Felix@\emph{von Felix Salten}!1905-01-111@{11. 1. 1905}|)be}\mylabel{h}  \normalsize

\doendnotes{C}
\bigskip
\vfill

\clearpage

\footnotesize

\lohead{\textsc{register}}

% Definiere theindex-Environment komplett neu ohne reledmac
\makeatletter
\renewenvironment{theindex}{%
  \section*{\indexname}%
  \setlength{\parindent}{0pt}%
  \setlength{\parskip}{0pt plus 0.3pt}%
  \let\item\@idxitem
}{%
  \clearpage
}
\makeatother

\IfFileExists{\jobname-pw.ind}{\input{\jobname-pw.ind}}{}

\end{document}

      