%% latex-korrekturansicht-vorspann.tex
%% Vorspann für die Korrekturansicht.
%% Lädt die gemeinsame Datei latex-vorspann.tex mit gesetztem Schalter.

\newif\ifkorrekturansicht
\korrekturansichttrue

\input{../tex-inputs/latex-vorspann}


               \section[Charlotte Ehrenstein an Arthur Schnitzler, {[}Mitte Februar 1906?{]}]{ Charlotte Ehrenstein an Arthur Schnitzler, {[}Mitte Februar
                    1906?{]}}\nopagebreak\mylabel{v}\rehead{ }\normalsize\beginnumbering\briefempfaengerindex{Schnitzler, Arthur@\textsc{Schnitzler, Arthur}!zzzEhrenstein, Charlotte@\emph{von Charlotte Ehrenstein}!1906-02-151@{{[}Mitte Februar 1906?{]}}|(be} \toendnotes[C]{\smallbreak\pagebreak[2]} \Standort{DLA, A:Schnitzler, HS.NZ85.1.2837,2.}
\physDesc{Brief, 1 Blatt, 2 Seiten
\newline{}Handschrift: Bleistift, deutsche Kurrent
\newline{}Schnitzler: mit Bleistift beschriftet: »\textsc{Ehrenstein}« }\toendnotes[C]{\smallbreak}\pstart
           \noindent{}{\pb}\textsc{Hochwohlgeb. Herrn Dr. Arthur Schnitzler}.\pend
           \pstart\center{}Sehr geehrter Herr Doctor!\pend\pstart
           Heute darf ich über das Befinden meines l. \textcolor{blue}{Albert}{}\ledrightnote{\textcolor{blue}{Albert Ehrenstein}}{ }ſchon recht Befriedigendes berichten. \label{K_L01584_1v}\edtext{Vor einigen Tagen}{\lemma{\textnormal{\emph{Vor einigen Tagen}}}\Cendnote{\textnormal{Das letzte mit Gewissheit zu datierende
                        Korrespondenzstück stammt vom 29. 1. 1906. Entsprechend des anzunehmenden
                        Krankheitsverlaufs dürfte dieses Schreiben wenige Wochen danach abgefasst
                        sein.}}}\label{K_L01584_1h} war Dr \textcolor{blue}{Kornfeld}{}\ledrightnote{\textcolor{blue}{Sigmund Kornfeld}} hier, u.
                    erlaubte ihm, da er Zuſtand und Ausſehen befriedigend fand, \textcolor{blue}{Albert}{}\ledrightnote{\textcolor{blue}{Albert Ehrenstein}} nahm während ſeiner Krankheit fünf Kilo an Gewicht
                    zu, täglich von 3–5 Nachmittags das Bett zu verlaſſen. Auch über ſein weiteres
                    Studium ſprach er mit ihm, er ſchlägt \textcolor{blue}{Albert}{}\ledrightnote{\textcolor{blue}{Albert Ehrenstein}}en das Mittelſchulprofeſſor-Studium vor, Geographie, Geſchichte und
                    Deutſch oder Naturgeſchichte, da er {\pb}meint, das
                    Doctorat in Medicin für \textcolor{blue}{Albert}{}\ledrightnote{\textcolor{blue}{Albert Ehrenstein}}{ }ſchwer zu erringen ſein würde. Und nun bitte
                    ich, mir zu verzeihen, wenn ich außer mit meinem Heutigem, noch mit der Bitte um
                    Ihre Meinung beläſtige, da ſie uns allen ſehr maßgebend iſt, vor allen aber,
                    Ihrer, Sie \pend
           \pstart
           verehrenden{\\[\baselineskip]}\spacefill\mbox{Charlotte Ehrenſtein}\pend
           \leftskip=0em{}\endnumbering\briefempfaengerindex{Schnitzler, Arthur@\textsc{Schnitzler, Arthur}!zzzEhrenstein, Charlotte@\emph{von Charlotte Ehrenstein}!1906-02-151@{{[}Mitte Februar 1906?{]}}|)be}\mylabel{h}  \normalsize

\doendnotes{C}
\bigskip
\vfill

\clearpage

\footnotesize

\lohead{\textsc{register}}

% Definiere theindex-Environment komplett neu ohne reledmac
\makeatletter
\renewenvironment{theindex}{%
  \section*{\indexname}%
  \setlength{\parindent}{0pt}%
  \setlength{\parskip}{0pt plus 0.3pt}%
  \let\item\@idxitem
}{%
  \clearpage
}
\makeatother

\IfFileExists{\jobname-pw.ind}{\input{\jobname-pw.ind}}{}

\end{document}

      