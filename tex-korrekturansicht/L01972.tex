%% latex-korrekturansicht-vorspann.tex
%% Vorspann für die Korrekturansicht.
%% Lädt die gemeinsame Datei latex-vorspann.tex mit gesetztem Schalter.

\newif\ifkorrekturansicht
\korrekturansichttrue

\input{../tex-inputs/latex-vorspann}


               \section[Robert Adam an Arthur Schnitzler, 29. 10. 1910]{ Robert Adam an Arthur Schnitzler, 29. 10. 1910}\nopagebreak\mylabel{v}\rehead{ }\normalsize\beginnumbering\briefempfaengerindex{Schnitzler, Arthur@\textsc{Schnitzler, Arthur}!zzzAdam, Robert@\emph{von Robert Adam}!1910-10-291@{29. 10. 1910}|(be} \toendnotes[C]{\smallbreak\pagebreak[2]} \Standort{DLA, A:Schnitzler, HS.NZ85.1.4230,2.}
\physDesc{Brief, 1 Blatt, 2 Seiten
\newline{}Handschrift: schwarze Tinte, deutsche Kurrent
\newline{}Schnitzler: 1) mit Bleistift beschriftet: »\textsc{Adam}« 2) mit rotem Buntstift eine Unterstreichung}\toendnotes[C]{\smallbreak}\pstart
           \raggedleft{}{\pb}\textcolor{pink}{Wien}{}\ledrightnote{\textcolor{pink}{Wien}}, am 29. Oktober 1910\pend
           \pstart{}Hochverehrter Herr Doktor!\pend\pstart
           Die wohlwollenden Zeilen, die Sie mir im vorigen Jahre anläßlich der Überſendung
                    meiner Komödie: »\textcolor{green}{Die Geſchichte Alî ibn Bekkârs mit
                        Schams an-Nahâr}{}\ledrightnote{\textcolor{green}{Die Geschichte des Alî ibn Bekkâr mit Schams an-Nahâr}}« ſandten, geben mir den Mut, mit einer Bitte an Sie
                    heranzutreten.\pend
           \pstart
           Ich habe eine neue Komödie zum Abſchluſſe gebracht, die den Titel \textcolor{green}{\textsc{Neidhard}}{}\ledrightnote{\textcolor{green}{Neidhard}} führt, und möchte gerne, bevor ich mit ihr in die Öffentlichkeit trete,
                    {\pb}Ihren Rat, hochverehrter Herr Doktor, einholen,
                    welcher Weg wohl einzuſchlagen wäre, um dieſer Komödie, an der ich ſehr lang mit
                    ganzem Herzen arbeitete und die ich ſelbſt für reifer und intereſſanter halte
                    als die Ihnen bekannte \textcolor{green}{arabiſche}{}\ledrightnote{→\textcolor{green}{Die Geschichte des Alî ibn Bekkâr mit Schams an-Nahâr}}, mehr Publizität zu ſichern, als jener zuteil geworden
                    iſt.\pend
           \pstart
           Sollten Sie die Güte haben, einem ratloſen Poeten freundlich beizuſtehen, ſo
                    bitte ich um kurze Nachricht, wann ich bei Ihnen vorſprechen könnte.\pend
           \pstart
           Seien Sie, hochverehrter Herr Doktor, meiner Dankbarkeit und unbegrenzten
                    Hochſchätzung gewiß!\pend
           \pstart
           Ihr ergebener{\\[\baselineskip]}\spacefill\mbox{Robert Adam}\pend
           \leftskip=0em{}\pstart
           \noindent{}\textcolor{pink}{Wien XII/\textsubscript{1} Meidlinger
                            Hauptſtr. 56}{}\ledrightnote{\textcolor{pink}{Meidlinger Hauptstraße}}\pend
           \endnumbering\briefempfaengerindex{Schnitzler, Arthur@\textsc{Schnitzler, Arthur}!zzzAdam, Robert@\emph{von Robert Adam}!1910-10-291@{29. 10. 1910}|)be}\mylabel{h}  \normalsize

\doendnotes{C}
\bigskip
\vfill

\clearpage

\footnotesize

\lohead{\textsc{register}}

% Definiere theindex-Environment komplett neu ohne reledmac
\makeatletter
\renewenvironment{theindex}{%
  \section*{\indexname}%
  \setlength{\parindent}{0pt}%
  \setlength{\parskip}{0pt plus 0.3pt}%
  \let\item\@idxitem
}{%
  \clearpage
}
\makeatother

\IfFileExists{\jobname-pw.ind}{\input{\jobname-pw.ind}}{}

\end{document}

      