%% latex-korrekturansicht-vorspann.tex
%% Vorspann für die Korrekturansicht.
%% Lädt die gemeinsame Datei latex-vorspann.tex mit gesetztem Schalter.

\newif\ifkorrekturansicht
\korrekturansichttrue

\input{../tex-inputs/latex-vorspann}


\renewcommand{\erwaehntePersonen}{Personen: Elisabeth Steinrück}
\renewcommand{\erwaehnteInstitutionen}{Institutionen: Reichstag}
\renewcommand{\erwaehnteOrte}{Orte: Berlin, Deutsches Theater Berlin, Palasthotel Berlin, Wien}
\renewcommand{\erwaehnteWerke}{Werke: Der Schleier der Beatrice. Schauspiel in fünf Akten, Tagebuch}
\section[ Paul Goldmann an Arthur Schnitzler, 6. 3. 1903]{Paul Goldmann an Arthur Schnitzler, 6. 3. 1903}
\nopagebreak\mylabel{v}
\rehead{ }\normalsize\beginnumbering\briefempfaengerindex{Schnitzler, Arthur@\textsc{Schnitzler, Arthur}!zzzGoldmann, Paul@\emph{von Paul Goldmann}!1903-03-061@{6. 3. 1903}|(be}
\toendnotes[C]{\smallbreak\pagebreak[2]}\Standort{DLA, A:Schnitzler, HS.NZ85.1.3173.}
\physDesc{Postkarte
\newline{}Handschrift: 1) blaue Tinte, deutsche Kurrent\hspace{1em}2) blaue Tinte, lateinische Kurrent (\noindent{}Adresse)\hspace{1em}
\newline{}Versand: Stempel: »\nobreak{}\oindex{Berlin@\textbf{Berlin}, \emph{https://www.geonames.org/ontologyP.PPLC}|pwk}Berlin, S.W. 11a, 6. 3. 03, 7—8 N.\nobreak{}«. Stempel: »\nobreak{}\oindex{Berlin@\textbf{Berlin}, \emph{https://www.geonames.org/ontologyP.PPLC}|pwk}{[}Berli{]}\textcolor{gray}{n}, 7/3. 03, Beste{[}llt{]} vom Postamte
                                          \textcolor{gray}{9}\nobreak{}«.  
\newline{}Schnitzler: mit Bleistift das Jahr »{[}1{]}903« vermerkt }\toendnotes[C]{\smallbreak}\pstart{}{\pb}Herrn\pend{}\pstart{}Dr. Arthur Schnitzler\pend{}\pstart{}in \strikeout{\textcolor{pink}{Wien}{}\ledrightnote{\textcolor{pink}{Wien}}}{ }\textcolor{pink}{Berlin W.}{}\ledrightnote{\textcolor{pink}{Berlin}}\pend{}\pstart{}\textcolor{pink}{Palasthôtel}{}\ledrightnote{\textcolor{pink}{Palasthotel Berlin}}\pend{}
{\bigskip}
\pstart
           {\pb}\textcolor{pink}{Berlin}{}\ledrightnote{\textcolor{pink}{Berlin}}, 6. März.\pend
           
\pstart
           Liebſter Freund, Es thut mir unendlich leid, Deinen
               lieben Beſuch verfehlt zu haben. HeutAbend habe ich mit einer großen Zuckerſteuerdebatte im \textcolor{brown}{Reichstag}{}\ledrightnote{\textcolor{brown}{Reichstag}} mindeſtens \label{K_L03366-1v}\edtext{bis zehn Uhr}{\lemma{\textnormal{\emph{bis zehn Uhr}}}\Cendnote{\textnormal{Danach dürfte \textcolor{blue}{Goldmann} bei \textcolor{blue}{Elisabeth
                     Gussmann} gewesen sein, wo sich auch \textcolor{blue}{Schnitzler} aufhielt.}}}\label{K_L03366-1h} zu thun. Morgen um ½ 2 komme ich ins \label{K_L03366-2v}\edtext{\textcolor{pink}{Palaſthotel}{}\ledrightnote{\textcolor{pink}{Palasthotel Berlin}}}{\lemma{\textnormal{\emph{Palaſthotel}}}\Cendnote{\textnormal{Während seines \textcolor{pink}{Berlin}-Aufenthalts zwischen 22. 2. 1903 und 9. 3. 1903 wohnte \textcolor{blue}{Schnitzler} im \textcolor{pink}{Palasthotel}. Dem \emph{\textcolor{green}{Tagebuch}} ist nicht
                  zu entnehmen, ob \textcolor{blue}{Goldmann} ihn dort am 7. 3. 1903, noch vor
                  der Premiere von \emph{\textcolor{green}{Der Schleier der Beatrice}} am
                     \textcolor{pink}{Deutschen Theater Berlin},
               besuchte.}}}\label{K_L03366-2h}.\pend
           
\pstart
           Herzlichſt {\\[\baselineskip]}Dein \spacefill\mbox{Paul Goldm}\pend
           \leftskip=0em{}\endnumbering\briefempfaengerindex{Schnitzler, Arthur@\textsc{Schnitzler, Arthur}!zzzGoldmann, Paul@\emph{von Paul Goldmann}!1903-03-061@{6. 3. 1903}|)be}\mylabel{h}
\begin{anhang}
\end{anhang}\normalsize

\doendnotes{C}
\bigskip
\vfill

\clearpage

\footnotesize

\lohead{\textsc{register}}

% Definiere theindex-Environment komplett neu ohne reledmac
\makeatletter
\renewenvironment{theindex}{%
  \section*{\indexname}%
  \setlength{\parindent}{0pt}%
  \setlength{\parskip}{0pt plus 0.3pt}%
  \let\item\@idxitem
}{%
  \clearpage
}
\makeatother

\IfFileExists{\jobname-pw.ind}{\input{\jobname-pw.ind}}{}

\end{document}

      