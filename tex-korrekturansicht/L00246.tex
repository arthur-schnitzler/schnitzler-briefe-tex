%% latex-korrekturansicht-vorspann.tex
%% Vorspann für die Korrekturansicht.
%% Lädt die gemeinsame Datei latex-vorspann.tex mit gesetztem Schalter.

\newif\ifkorrekturansicht
\korrekturansichttrue

\input{../tex-inputs/latex-vorspann}


               \section[Arthur Schnitzler an Richard Beer-Hofmann, 29. 7. 1893]{ Arthur Schnitzler an Richard Beer-Hofmann, 29. 7. 1893}\nopagebreak\mylabel{v}\rehead{ }\normalsize\beginnumbering\briefempfaengerindex{Beer-Hofmann, Richard@\textsc{Beer-Hofmann, Richard}!zzzSchnitzler, Arthur@\emph{von Arthur Schnitzler}!1893-07-291@{29. 7. 1893}|(be} \toendnotes[C]{\smallbreak\pagebreak[2]} \Standort{YCGL, MSS 31.}
\physDesc{Briefkarte mit Trauerrand, Umschlag mit Trauerrand
\newline{}Handschrift: Bleistift, deutsche Kurrent\newline{}Versand: 1) Stempel: »\nobreak{}Wien 1/1, 29. 7. 93, 2–3 N\nobreak{}«.  2) Stempel: »\nobreak{}\oindex{Bad Ischl@\textbf{Bad Ischl}, \emph{Besiedelter Ort (A.BSO)}|pwk}Ischl, 30 7 93, 7–F\nobreak{}«. \newline{}Ordnung: mit Rotstift von unbekannter Hand oberhalb des Textes mit einem
                                    »X« versehen }\buchAbdrucke{\weitereDrucke{Arthur Schnitzler, Richard Beer-Hofmann: \emph{Briefwechsel 1891–1931}. Hg. Konstanze Fliedl. Wien, Zürich: \emph{Europaverlag} 1992, S. 49.} }\toendnotes[C]{\smallbreak}\pstart{}{\pb}\textsc{Herrn Doctor}\pend{}\pstart{}\textsc{Richard Beer-Hof\damage{mann}}\pend{}\pstart{}\textsc{\textcolor{pink}{Isch}{}\ledrightnote{\textcolor{pink}{Bad Ischl}}\damage{l}}\pend{}\pstart{}\textcolor{pink}{\textsc{Schulgasse }}{}\ledrightnote{\textcolor{pink}{Schulgasse}}\damage{8}\pend{}{\bigskip}\pstart
           \noindent{}{\pb}Lieber Richard! – Der \textcolor{blue}{Abſchreiber}{}\ledrightnote{→\textcolor{blue}{?? [Schreibkraft für Arthur Schnitzler]}} bringt die \textcolor{green}{Novelle}{}\ledrightnote{→\textcolor{green}{Das Kind}}{ }Montag; – Dinſtag haben Sie ſie. – Neulich \textcolor{green}{ſtand}{}\ledrightnote{→\textcolor{green}{[Am Lessingtheater … Ohne Geläut … Das Märchen … zur Aufführung]}} im \textcolor{green}{Magazin}{}\ledrightnote{\textcolor{green}{Magazin für die Literatur des Auslandes}} (\textcolor{blue}{Kraus}{}\ledrightnote{\textcolor{blue}{Karl Kraus}}{ }ſchickt es mir) dſs noch dieſen So{\geminationm}er im \textcolor{pink}{Leſſ.th.}{}\ledrightnote{\textcolor{pink}{Lessing-Theater}} das \textcolor{green}{Märchen}{}\ledrightnote{\textcolor{green}{Das Märchen. Schauspiel in drei Aufzügen}} dranko{\geminationm}t. – Die »lustige« \textcolor{green}{Novelle}{}\ledrightnote{→\textcolor{green}{Die kleine Komödie}} beendet. – Aerztlich beſchäftigt, eine \textcolor{blue}{Cousine}{}\ledrightnote{→\textcolor{blue}{Adele von Suppé}}, 14j. Mädel, ſchwerer
               Typhus. – Habe noch keine {\pb}Einberufung. – \textcolor{green}{Notiz}{}\ledrightnote{→\textcolor{green}{[Man schreibt uns aus Ischl]}} im \textcolor{green}{B. B.}{}\ledrightnote{\textcolor{green}{Berliner Börsen-Courier}} geleſen; ſehr gut – aber natürlich »\textcolor{green}{naturaliſtiſcher Dichter}{}\ledrightnote{→\textcolor{green}{Berliner Börsen-Courier}}«. – Geſtern war ich
               angeblich im \textcolor{blue}{\textcolor{green}{\textsc{Szeps}}{}\ledrightnote{→\textcolor{green}{Wiener Tagblatt}}}{}\ledrightnote{\textcolor{blue}{Moriz Szeps}}{ }\label{K_L00246_1v}\edtext{\textcolor{green}{verschimpfirt}{}\ledrightnote{→\textcolor{green}{Die Saison in Ischl}}}{\lemma{\textnormal{\emph{verschimpfirt}}}\Cendnote{\textnormal{In dem Bericht ohne Autornennung heißt
                  es: »Das Theaterleben ist ein sehr bewegtes, Tag für Tag Vorstellung,
                     berühmte und unberühmte Gäste, ja sogar Novitäten, sogenannte Sommer-Einakter,
                     die freilich oft nur aus Courtoisie aufgeführt werden. Ein realistisches
                     Stückchen ›\textcolor{green}{Das Abschieds-Souper}‹, aus der
                     Feder eines jungen \textcolor{pink}{Wien}er \textcolor{blue}{Realisten} hat wenig Erfolg gehabt, um
                     nicht zu sagen, gar keinen«. (\emph{\textcolor{green}{Die Saison in Ischl}}. In: \emph{\textcolor{green}{Wiener Tagblatt}}, Jg. 43, Nr. 206, 28. 7. 1893,
                     S. 4.)}}}\label{K_L00246_1h} (las es nicht) – nachdem ich vor 3 Tagen \label{K_L00246_2v}\edtext{gelobt}{\lemma{\textnormal{\emph{gelobt}}}\Cendnote{\textnormal{nicht nachweisbar}}}\label{K_L00246_2h} war. Gute Redaction! – Was macht der
                  \textcolor{green}{Götterliebling}{}\ledrightnote{\textcolor{green}{Der Tod Georgs}}? – Ist \textcolor{blue}{Löbl}{}\ledrightnote{\textcolor{blue}{Emil Löbl}} noch in \textcolor{pink}{Iſchl}{}\ledrightnote{\textcolor{pink}{Bad Ischl}}? Wohin
               ſchreibt man ihm? Las übrigens die Nu{\geminationm}er noch gar
               nicht. – Schreibt \textcolor{blue}{Loris}{}\ledrightnote{\textcolor{blue}{Hugo von Hofmannsthal}}? – Grüßen Sie alles!
                  \label{T_L00246_1v}\edtext{Ich würde mehr ſchreiben, we{\geminationn} ich nicht auf}{\lemma{\textnormal{\emph{Ich … auf}}}\Cendnote{\textnormal{quer am rechten Rand weiter}}}\label{T_L00246_1h}{ }\label{T_L00246_2v}\edtext{dieſem blöden Karterl angefangen
                  hätte.}{\lemma{\textnormal{\emph{dieſem … hätte.}}}\Cendnote{\textnormal{am linken Rand der
                  Vorderseite}}}\label{T_L00246_2h}\pend
           \endnumbering\briefempfaengerindex{Beer-Hofmann, Richard@\textsc{Beer-Hofmann, Richard}!zzzSchnitzler, Arthur@\emph{von Arthur Schnitzler}!1893-07-291@{29. 7. 1893}|)be}\mylabel{h}  \normalsize

\doendnotes{C}
\bigskip
\vfill

\clearpage

\footnotesize

\lohead{\textsc{register}}

% Definiere theindex-Environment komplett neu ohne reledmac
\makeatletter
\renewenvironment{theindex}{%
  \section*{\indexname}%
  \setlength{\parindent}{0pt}%
  \setlength{\parskip}{0pt plus 0.3pt}%
  \let\item\@idxitem
}{%
  \clearpage
}
\makeatother

\IfFileExists{\jobname-pw.ind}{\input{\jobname-pw.ind}}{}

\end{document}

      