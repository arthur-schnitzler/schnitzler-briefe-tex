%% latex-korrekturansicht-vorspann.tex
%% Vorspann für die Korrekturansicht.
%% Lädt die gemeinsame Datei latex-vorspann.tex mit gesetztem Schalter.

\newif\ifkorrekturansicht
\korrekturansichttrue

\input{../tex-inputs/latex-vorspann}


\renewcommand{\erwaehntePersonen}{Personen:  ,  ,  ,  ,  ,  ,  ,  , Rosine Hutzler, Josef Kainz, Sara Kainz, Elsa Plessner, Louis Plessner, Jolantha Ramazetta, Lajos Stael-Dergy, Louis Stael-Dergy}
\renewcommand{\erwaehnteInstitutionen}{Institutionen: Deutsches Theater Berlin, Volkstheater}
\renewcommand{\erwaehnteOrte}{Orte: Berlin, Budapest, Kärntner Straße 10, Meran, Paris, Sankt Petersburg, Ungarn, Volkstheater, Wien}
\renewcommand{\erwaehnteWerke}{Werke: Am Wege, Anatol, Baby, Berliner Börsen-Courier, Das erste Kapitel. Schauspiel in drei Akten, Der Begräbnißtag, Der Herr Cassenchef, Der gläserne Käfig. Eine Parabel, Der gläserne Käfig. Skizzen und Novellen, Der neue Lehrer. Novelle, Die Ehrlosen. Schauspiel in drei Acten, Die Ehrlosen. Schauspiel in drei Acten, Die Leiter der Seele, Ein Brief, Im Feuer geprüft, Im Widerschein, Meine Freundin Clotilde, Neues Wiener Tagblatt, Reminiscenz, Theater, Kunst und Literatur. Deutsches Volkstheater, Warten, Warum}
\section[Elsa Plessner an Arthur Schnitzler, 12. 10. 1900]{Elsa Plessner an Arthur Schnitzler, 12. 10. 1900}
\nopagebreak\mylabel{v}
\rehead{ }\normalsize\beginnumbering\briefempfaengerindex{Schnitzler, Arthur@\textsc{Schnitzler, Arthur}!zzzPlessner, Elsa@\emph{von Elsa Plessner}!1900-10-122@{12. 10. 1900}|(be}
\toendnotes[C]{\smallbreak\pagebreak[2]}\Standort{DLA, A:Schnitzler, HS.1985.1.419.}
\physDesc{Brief,  Blätter, 8 Seiten, 5247 Zeichen
\newline{}Handschrift: , lateinische Kurrent}\toendnotes[C]{\smallbreak}
\pstart
           {\pb}\textcolor{pink}{Wien I. Kärnthnerstrasse N\textsuperscript{o} 10}{}\ledrightnote{\textcolor{pink}{Kärntner Straße 10}}\pend
           
\pstart
           \raggedleft{}den 12. Oktober 1900\pend
           
\pstart{}Verehrter Herr Doctor!\pend\vspace{0.5em}
\pstart
           \label{K_L03728-1v}\edtext{Sie wünschen}{\lemma{\textnormal{\emph{Sie wünschen}}}\Cendnote{\textnormal{\textcolor{blue}{Schnitzlers} Brief ist nicht überliefert.}}}\label{} die chronologische Reihenfolge der Arbeiten zu wissen und indem Sie mir
               eine Erwiderung auf meine Bemerkung über meine Entwickelung versprechen, kündigen
               Sie mir – deutlich genug für mein zartes Verständnis – einen neuerlichen Putzer
               an. – \pend
           
\pstart
           Zuerst entspreche ich beiliegend Ihrem Wunsche und dann muss ich – zu Ihrer
               Orientirung \uline{vor} dem Putzer – etwas weiter ausholen.
               Sie finden also zuerst die merkwürdig Thatsache dass »\textcolor{green}{Baby}{}\ledrightnote{\textcolor{green}{Baby}}« von allen die älteste Arbeit ist. Es ist mir sehr
               verständlich, dass »\textcolor{green}{Baby}{}\ledrightnote{\textcolor{green}{Baby}}« Ihnen nicht allzusehr
               missfällt, denn es beweist neuerdings, dass das Wesensverwandte jeden Menschen
               anzieht. Es ist ja auf 1000 Schritte sichtbar, dass diese \textcolor{green}{Geschich- }{}\ledrightnote{{$\rightarrow$}\textcolor{green}{Baby}}{\pb}unter Ihrem \strikeout{D}{ }\uline{directen} Einfluss entstanden ist und – bitte um
               Verzeihung für die Arroganz – Leo, der Held könnte ganz gut – \textcolor{green}{Anatol}{}\ledrightnote{{$\rightarrow$}\textcolor{green}{Anatol}} – heißen, was natürlich nichts \strikeout{zu} daran ändert, dass die \textcolor{green}{Geschichte}{}\ledrightnote{{$\rightarrow$}\textcolor{green}{Baby}} selbstverständlich nicht im
               entferntesten an Ihren »\textcolor{green}{Anatol}{}\ledrightnote{\textcolor{green}{Anatol}}« heranreicht.\pend
           
\pstart
           Das heißt mit anderen Worten: »Ich hatte mich so in Ihr \textcolor{green}{Buch}{}\ledrightnote{{$\rightarrow$}\textcolor{green}{Anatol}} hereingelesen, dass ich auf einmal
               Ihre Sprache sprach«. – Vielleicht interessiert Sie die Thatsache, dass »\label{K_L03728-2v}\edtext{\textcolor{green}{Baby}{}\ledrightnote{\textcolor{green}{Baby}}}{\lemma{\textnormal{\emph{Baby}}}\Cendnote{\textnormal{Die Geschichte handelt
                  davon, dass eine Frau ihrem Liebhaber das gemeinsame Kind vorstellt, das bislang
                  am Land aufgezogen wird. Sie möchte, dass der Vater das Kind als Anlass nimmt, sie
                  zu heiraten. Zwar empfindet dieser väterliche Gefühle, doch zu einer Eheschließung
                  kann er sich nicht durchringen. Die Frau erwähnt, mit einem »Baron«
                  in Kontakt zu stehen, der sie heiraten wolle und das Kind als seines aufzuziehen
                  bereit sei.}}}\label{}« zwei Tage nach dem \label{K_L03728-3v}\edtext{Tod meines \textcolor{blue}{Vaters}{}\ledrightnote{{$\rightarrow$}\textcolor{blue}{Louis Plessner}}}{\lemma{\textnormal{\emph{Tod meines Vaters}}}\Cendnote{\textnormal{\textcolor{blue}{Louis Plessner} starb am
                     19. 9. 1895 in \textcolor{pink}{Wien}.}}}\label{} entstanden ist. – Außerdem theile ich Ihnen im Vertrauen mit,
               dass der Held dieser Geschichte \label{K_L03728-4v}\edtext{eigentlich \uline{\textcolor{blue}{Kainz}{}\ledrightnote{\textcolor{blue}{Josef Kainz}}}}{\lemma{\textnormal{\emph{eigentlich Kainz}}}\Cendnote{\textnormal{Die Anekdote selbst
                  ist nicht verifizierbar, doch passen die hier gegebenen Hinweise mit historischen
                  Fakten zusammen. Die aus \textcolor{pink}{Ungarn} stammende
                  Schauspielerin \textcolor{blue}{Jolán Ramazetter}, die den
                  eingedeutschten Bühnennamen \textcolor{blue}{Jolantha
                     Ramazetta} verwendete, war um 1884 gemeinsam mit \textcolor{blue}{Kainz} am \emph{\textcolor{brown}{Deutschen Theater}} in \textcolor{pink}{Berlin}
                  engagiert. Am 28. 7. 1884 übernahm das \emph{\textcolor{green}{Neue Wiener Tagblatt}} eine Meldung des \emph{\textcolor{green}{Berliner Börsen-Couriers}}, dass \textcolor{blue}{Kainz} und \textcolor{blue}{Ramazetta}
                  verlobt seien (Nr. 207, S. 3). Am
                     14. 10. 1911 starb in \textcolor{pink}{Budapest} der Journalist und Sprachlehrer \textcolor{blue}{Lajos Staél-Dergy} an einer Schussverletzung. Er war zum
                  Zeitpunkt des Todes 26 Jahre alt und nach \textcolor{pink}{Paris} zuständig. Als Mutter wird \textcolor{blue}{Jolán
                     Ramazetter} genannt, als Vater \textcolor{blue}{Lajos
                     Staél-Dergy} (der Vorname dürfte dem französischen ›Louis‹ entsprechen).
                     \textcolor{blue}{Jolán Ramazetter} war zum mutmaßlichen
                  Zeitpunkt der Geburt mehrere Monate an einem Theater in \textcolor{pink}{Sankt Petersburg} engagiert. Danach nahm sie Engagements in
                     \textcolor{pink}{Paris} an.}}}\label{} ist, um dessen \textcolor{blue}{Kind}{}\ledrightnote{{$\rightarrow$}\textcolor{blue}{Lajos Stael-Dergy}} es sich vor fünf Jahren
               nach dem Tod seiner ersten \textcolor{blue}{Frau}{}\ledrightnote{{$\rightarrow$}\textcolor{blue}{Sara Kainz}} handelte. Ich habe die \textcolor{green}{Geschichte}{}\ledrightnote{{$\rightarrow$}\textcolor{green}{Baby}} – die natürlich anders sich abspielte – {\pb}direct von \textcolor{blue}{Rosie Hutzler}{}\ledrightnote{\textcolor{blue}{Rosine Hutzler}}, seiner
               Stieftochter erfahren. Die Mutter – ehemals Mitglied des »\textcolor{brown}{Deutschen Theaters}{}\ledrightnote{\textcolor{brown}{Deutsches Theater Berlin}}{[}«{]} – Fräulein \textcolor{blue}{Ramacetta}{}\ledrightnote{\textcolor{blue}{Jolantha Ramazetta}}
               ist in \textcolor{pink}{Paris}{}\ledrightnote{\textcolor{pink}{Paris}} an einen \textcolor{blue}{Baron}{}\ledrightnote{{$\rightarrow$}\textcolor{blue}{Louis Stael-Dergy}} verheirathet, der das etwa
               zwölfjährige \textcolor{blue}{Kind}{}\ledrightnote{{$\rightarrow$}\textcolor{blue}{Lajos Stael-Dergy}} adoptirt
               hat. – – –\pend
           
\pstart
           Nach dieser kleinen Abschweifung in die \begin{otherlanguage}{french}chronique
                  scandaleuse\end{otherlanguage}
                kehre ich zur Materie \introOben{}dieser
                  Epistel\introOben{} zurück. –\pend
           
\pstart
           Das ganze Buch »\textcolor{green}{D. g. Käfig}{}\ledrightnote{\textcolor{green}{Der gläserne Käfig. Skizzen und Novellen}}« hat keinen anderen
               Zweck als den meiner im Anfang December stattfindenden \label{K_L03728-5v}\edtext{Première}{}\ledrightnote{}}{\lemma{\textnormal{\emph{Première}}}\Cendnote{\textnormal{Die
                  Premiere} von \textcolor{blue}{Plessners} Schauspiel \emph{\textcolor{green}{Die
                     Ehrlosen}} fand erst am 16. 3. 1901 am Volkstheater
                  statt.}}}\label{} zu präludieren und ein paar Talentproben in die Welt der
               Premièrenbesucher zu schleudern, damit ich nicht ganz wie ein rother Hund behandelt
               werde, wenn man gar nichts von mir weiß und kennt. Glauben Sie ja nicht, dass ich
               mich irgend welchen Illusionen über den {\pb}Wert des \textcolor{green}{Buches}{}\ledrightnote{{$\rightarrow$}\textcolor{green}{Der gläserne Käfig. Skizzen und Novellen}} hingebe. Aber da ich meine
               novellistische »Thätigkeit« seit 2 Jahren abgeschlossen habe – (\label{K_L03728-6v}\edtext{»\textcolor{green}{Der neue
                  Lehrer}{}\ledrightnote{\textcolor{green}{Der neue Lehrer. Novelle}}« war das letzte}{\lemma{\textnormal{\emph{Première}}}\Cendnote{\textnormal{Am 22. 10 1898 erwähnte \textcolor{blue}{Plessner} ihren längsten Prosatext \emph{\textcolor{green}{Der neue Lehrer}} erstmals, ohne jedoch den
                  Titel zu nennen.}}}\label{}) hat es mir Spaß gemacht, die besseren Arbeiten dieser
               Sorte zu einem Debut zusammenzufassen. –\pend
           
\pstart
           Ich muss Sie bitten mir zu glauben, dass ich mein Vertrauen \strikeout{und meine} nicht so offen in der Hand zu jedermanns Belieben herumtrage.
               Aber da Sie sich kennen und Ihre Fähigkeit zu verstehen, werden Sie es begreiflich
               finden, dass ich gerade bei Ihnen Verständnis suchte und noch suche, denn einen
               Menschen muss man doch haben, bei dem man sich ausjammern kann, ohne dass er es
               anders deutet. Das heißt mit kurzen Worten: Ich bin seit mehr als einem Jahr an einem
               toten {\pb}Punkt meiner Entwickelung angelangt, den ich nicht überwinden
               kann. Seit dem »\textcolor{green}{ersten Capitel}{}\ledrightnote{\textcolor{green}{Das erste Kapitel. Schauspiel in drei Akten}}« habe ich außer
               zu Briefen nicht die Feder in die Hand genommen und nicht eine Zeile schreiben
               können. Ich würde mich wieder für »fertig« halten, wenn Sie mir das nicht seinerzeit
               nach \textcolor{pink}{Meran}{}\ledrightnote{\textcolor{pink}{Meran}} so nachdrücklich verwiesen hätten.
               Aber eine so fürchterliche Zeit absoluter Leere und Unfähigkeit wie dieses
                  Jahr habe ich noch nie durchgemacht und zu einer Zeit, wo mein brennender
               äußerer Ehrgeiz \strikeout{eigentlich} zu seinem Rechte zu kommen
               beginnt – bin ich eigentlich so sterbensunglücklich wie ein Mensch es nur sein
               kann!\pend
           
\pstart
           Vielleicht ist es das Warten auf die Première}{}\ledrightnote{},
               das mich so lähmt – aber was mache ich, wenn die »\textcolor{green}{Ehrlosen}{}\ledrightnote{\textcolor{green}{Die Ehrlosen. Schauspiel in drei Acten}}« {\pb}durchfallen, was doch immerhin möglich ist? Bei dem
               absoluten Versagen aller meiner innerlichen Lebensmöglichkeiten sehe ich nichts
               weiter vor mir, wenn auch mein äußerer Lebenszweck unerreichbar ist. Ich habe die
               schönsten und wertvollsten Jahre meines Lebens vergehen lassen, ohne nach rechts und
               links zu schauen wie andere Mädchen, habe mit Scheuklappen auf mein künstlerisches
               Ziel hingearbeitet und im Gefühle einer gewissen inneren Kraft auf Manches
               verzichtet, um mich nicht zu verzetteln und zu zersplittern – und wenn ich mir jetzt
               vorstelle, dass das Alles umsonst war, könnte ich weinen um jeden Ball, auf dem ich
               mir den Kopf zerbrochen habe um eine Arbeit, statt zu tanzen und mich – zu
               amüsieren. – – Ich habe auf {\pb}der ganzen Welt nichts, als meine
               Arbeit – ob gut oder schlecht ist eigentlich egal. Aber wenn ich nicht einmal mehr
               arbeiten kann – ?\pend
           
\pstart
           Also wenn Sie jetzt noch vo\substVorne{}\textsuperscript{m}\substDazwischen{}n\substHinten{} Entwickelung in Bezug auf mich sprechen wollen, so können sie nur von der
               Zeit sprechen, die weit hinter mir liegt! Zu dem Standpunkt der alten Arbeiten kann
               ich nicht zurück und vor mir liegt kein Weg mehr. Außer Sie sehen weiter und mehr als
               ich selbst.\pend
           
\pstart
           Das musste ich Ihnen noch vorher sagen und \strikeout{dass ich}
               Sie mit den Voraussetzungen bekannt machen \strikeout{musste},
               aus denen Sie Ihre Schlüsse ziehen {\pb}können. Ich bin neugierig wie
               dieselben ausfallen werden.\pend
           
\pstart
           Herzlich und stets verehrend{\\[\baselineskip]}Ihre{\\[\baselineskip]}\spacefill\mbox{Elsa Plessner}\pend
           \leftskip=0em{}
\pstart
           \noindent{}\textcolor{green}{Baby}{}\ledrightnote{\textcolor{green}{Baby}} (September 95)\pend
           
\pstart
           \textcolor{green}{Begräbnistag}{}\ledrightnote{\textcolor{green}{Der Begräbnißtag}} (95\pend
           
\pstart
           \textcolor{green}{Selbstmörder}{}\ledrightnote{\textcolor{green}{Die Leiter der Seele}}{ }96\pend
           
\pstart
           \textcolor{green}{Im Feuer geprüft}{}\ledrightnote{\textcolor{green}{Im Feuer geprüft}}{ }96\pend
           
\pstart
           \textcolor{green}{Widerschein}{}\ledrightnote{\textcolor{green}{Im Widerschein}}{ }96\pend
           
\pstart
           \textcolor{green}{Am Wege}{}\ledrightnote{\textcolor{green}{Am Wege}}{ }9\uline{6}\pend
           
\pstart
           \textcolor{green}{Cassenchef}{}\ledrightnote{\textcolor{green}{Der Herr Cassenchef}}{ }97\pend
           
\pstart
           \textcolor{green}{Ein Brief}{}\ledrightnote{\textcolor{green}{Ein Brief}} 97{ }\label{K_L03728-7v}\edtext{\textcolor{pink}{Meran}{}\ledrightnote{\textcolor{pink}{Meran}}}{\lemma{\textnormal{\emph{Meran}}}\Cendnote{\textnormal{Die
                     Angabe des Entstehungsortes \textcolor{pink}{Meran} wird
                     mit geschweifter Klammer auch auf die darüber und die darunter liegende Zeile
                     bezogen.}}}\label{}\pend
           
\pstart
           \textcolor{green}{Der gläserne Käfig}{}\ledrightnote{\textcolor{green}{Der gläserne Käfig. Eine Parabel}}{ }97\pend
           
\pstart
           \textcolor{green}{Meine Freundin Clotilde}{}\ledrightnote{\textcolor{green}{Meine Freundin Clotilde}}{ }97\pend
           
\pstart
           \textcolor{green}{Reminiscenz}{}\ledrightnote{\textcolor{green}{Reminiscenz}}{ }97\pend
           
\pstart
           \textcolor{green}{Warten}{}\ledrightnote{\textcolor{green}{Warten}}{ }98\pend
           
\pstart
           \textcolor{green}{Warum}{}\ledrightnote{\textcolor{green}{Warum}}{ }9\substVorne{}\textsuperscript{9}\substDazwischen{}8\substHinten{}\pend
           
\pstart
           \textcolor{green}{Der neue Lehrer}{}\ledrightnote{\textcolor{green}{Der neue Lehrer. Novelle}} (Juli 9\substVorne{}\textsuperscript{9}\substDazwischen{}8\substHinten{})\pend
           
\pstart
            (\textcolor{green}{Die Ehrlosen}{}\ledrightnote{\textcolor{green}{Die Ehrlosen. Schauspiel in drei Acten}}{ }November 98)\pend
           
\pstart
            (\textcolor{green}{Das erste Capitel}{}\ledrightnote{\textcolor{green}{Das erste Kapitel. Schauspiel in drei Akten}}{ }October 99{[}){]}\pend
           
\pstart
           und sonst keine Zeile.\pend
           \endnumbering\briefempfaengerindex{Schnitzler, Arthur@\textsc{Schnitzler, Arthur}!zzzPlessner, Elsa@\emph{von Elsa Plessner}!1900-10-122@{12. 10. 1900}|)be}\mylabel{h}
\begin{anhang}
\end{anhang}\normalsize

\doendnotes{C}
\bigskip
\vfill

\clearpage

\footnotesize

\lohead{\textsc{register}}

% Definiere theindex-Environment komplett neu ohne reledmac
\makeatletter
\renewenvironment{theindex}{%
  \section*{\indexname}%
  \setlength{\parindent}{0pt}%
  \setlength{\parskip}{0pt plus 0.3pt}%
  \let\item\@idxitem
}{%
  \clearpage
}
\makeatother

\IfFileExists{\jobname-pw.ind}{\input{\jobname-pw.ind}}{}

\end{document}

      