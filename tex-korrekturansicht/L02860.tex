%% latex-korrekturansicht-vorspann.tex
%% Vorspann für die Korrekturansicht.
%% Lädt die gemeinsame Datei latex-vorspann.tex mit gesetztem Schalter.

\newif\ifkorrekturansicht
\korrekturansichttrue

\input{../tex-inputs/latex-vorspann}


               \section[ Paul Goldmann an Arthur Schnitzler, 13. 10. 1898]{Paul Goldmann an Arthur Schnitzler, 13. 10. 1898}\nopagebreak\mylabel{v}\rehead{ }\normalsize\beginnumbering\briefempfaengerindex{Schnitzler, Arthur@\textsc{Schnitzler, Arthur}!zzzGoldmann, Paul@\emph{von Paul Goldmann}!1898-10-131@{13. 10. 1898}|(be} \toendnotes[C]{\smallbreak\pagebreak[2]} \Standort{DLA, A:Schnitzler, HS.NZ85.1.3168.}
\physDesc{Postkarte
\newline{}Handschrift: 1) blaue Tinte, deutsche Kurrent\hspace{1em}2) blaue Tinte, lateinische Kurrent (\noindent{}Adresse)\hspace{1em}\newline{}Versand: 1) Stempel: »\nobreak{}\oindex{Peking@\textbf{Peking}, \emph{https://www.geonames.org/ontologyP.PPLC}|pwk}Peking, 14 Oct 98\nobreak{}«.  2) Stempel: »\nobreak{}\oindex{Shanghai@\textbf{Shanghai}, \emph{http://www.geonames.org/ontologyP.PPLA}|pwk}Shan{[}ghai{]}, 18 Oct 98\nobreak{}«. 3) Stempel: »\nobreak{}Oc 20 98\nobreak{}«. 4) Stempel: »\nobreak{}\oindex{Hong Kong@\textbf{Hong Kong}, \emph{https://www.geonames.org/ontologyP.PPLC}|pwk}{[}Ho{]}\textcolor{gray}{ng Kon}{[}g{]}, Oc 27 98\nobreak{}«. 
\newline{}Schnitzler: mit Bleistift das Jahr »98« vermerkt }\toendnotes[C]{\smallbreak}\pstart{}{\pb}\begin{otherlanguage}{english}\textcolor{pink}{Austria}{}\ledrightnote{\textcolor{pink}{Österreich}}\end{otherlanguage}.\pend{}\pstart{}Herrn Dr. Arthur Schnitzler\pend{}\pstart{}\textcolor{pink}{Wien}{}\ledrightnote{\textcolor{pink}{Wien}}\pend{}\pstart{}\textcolor{pink}{IX. Frankgaße 1}{}\ledrightnote{\textcolor{pink}{Frankgasse}}.\pend{}{\bigskip}\pstart
           \raggedleft{}{\pb}\textsc{\textcolor{pink}{Peking}{}\ledrightnote{\textcolor{pink}{Peking}}}, 13. Oktober.\pend
           \pstart
           Einen ſchönen Gruß aus der \textcolor{pink}{chin}{}\ledrightnote{→\textcolor{pink}{China}}eſiſchen \textcolor{pink}{Hauptſtadt}{}\ledrightnote{→\textcolor{pink}{Peking}}!\pend
           \pstart
           Dein treuer {\\[\baselineskip]}\spacefill\mbox{P. G.}\pend
           \leftskip=0em{}\endnumbering\briefempfaengerindex{Schnitzler, Arthur@\textsc{Schnitzler, Arthur}!zzzGoldmann, Paul@\emph{von Paul Goldmann}!1898-10-131@{13. 10. 1898}|)be}\mylabel{h}  \normalsize

\doendnotes{C}
\bigskip
\vfill

\clearpage

\footnotesize

\lohead{\textsc{register}}

% Definiere theindex-Environment komplett neu ohne reledmac
\makeatletter
\renewenvironment{theindex}{%
  \section*{\indexname}%
  \setlength{\parindent}{0pt}%
  \setlength{\parskip}{0pt plus 0.3pt}%
  \let\item\@idxitem
}{%
  \clearpage
}
\makeatother

\IfFileExists{\jobname-pw.ind}{\input{\jobname-pw.ind}}{}

\end{document}

      