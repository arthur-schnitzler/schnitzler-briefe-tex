%% latex-korrekturansicht-vorspann.tex
%% Vorspann für die Korrekturansicht.
%% Lädt die gemeinsame Datei latex-vorspann.tex mit gesetztem Schalter.

\newif\ifkorrekturansicht
\korrekturansichttrue

\input{../tex-inputs/latex-vorspann}


\renewcommand{\erwaehntePersonen}{Personen: Rolf von Brockdorff, Eugen Gura,  Paulus, Rudolf Strauss}
\renewcommand{\erwaehnteInstitutionen}{Institutionen: Wiener Allgemeine Zeitung}
\renewcommand{\erwaehnteOrte}{Orte: Ronacher, Wien}
\renewcommand{\erwaehnteWerke}{Werke: Liebelei. Eine Wiener Zeitschrift, Wiener Allgemeine Zeitung}
\section[ Felix Salten an Arthur Schnitzler, {[}12?. 12. 1895{]}]{Felix Salten an Arthur Schnitzler, {[}12?. 12. 1895{]}}
\nopagebreak\mylabel{v}
\rehead{ }\normalsize\beginnumbering\briefempfaengerindex{Schnitzler, Arthur@\textsc{Schnitzler, Arthur}!zzzSalten, Felix@\emph{von Felix Salten}!1895-12-121@{{[}12?. 12. 1895{]}}|(be}
\toendnotes[C]{\smallbreak\pagebreak[2]}\Standort{CUL, Schnitzler, B 89, A 1.}
\physDesc{Brief, 1 Blatt, 1 Seite, 405 Zeichen
\newline{}Handschrift: Bleistift, lateinische Kurrent
\newline{}Schnitzler: mit Bleistift datiert: »11/12 95« 
\newline{}Ordnung: mit Bleistift von unbekannter Hand nummeriert: »67« }\toendnotes[C]{\smallbreak}
\pstart{}{\pb}Lieber \label{K_L03167-1v}\edtext{F.}{\lemma{\textnormal{\emph{F.}}}\Cendnote{\textnormal{Freund}}}\label{K_L03167-1h}\pend
\pstart
           Es soll bei \textcolor{green}{uns}{}\ledrightnote{{$\rightarrow$}\textcolor{green}{Wiener Allgemeine Zeitung}} eine scharfe
               Notiz gegen die \label{K_L03167-2v}\edtext{Zeitung »\textcolor{green}{Liebelei}{}\ledrightnote{\textcolor{green}{Liebelei. Eine Wiener Zeitschrift}}«}{\lemma{\textnormal{\emph{Zeitung »Liebelei«}}}\Cendnote{\textnormal{Ab 1. 1. 1896 erschien die von
                     \textcolor{blue}{Rolf von Brockdorff} und \textcolor{blue}{Rudolf Strauss} herausgegebene Zeitschrift
                     \emph{\textcolor{green}{Liebelei}}. Im Dezember 1895 findet sich keine Kritik daran in der \emph{\textcolor{green}{Wiener Allgemeinen Zeitung}}.}}}\label{K_L03167-2h} geschrieben werden.
               Soll ich das verhindern, oder begünstigen? Ich habe die Empfindung, als ob Sie jetzt
               ganz gut ein Wort gegen diese Sache sagen könnten. Aber es geht auch, wenn die »\textcolor{brown}{W\textsuperscript{r} Allgemeine}{}\ledrightnote{\textcolor{brown}{Wiener Allgemeine Zeitung}}«, quasi
               als Ihr Officiosus in dieser Notiz Ihre Stellung zu dem \textcolor{green}{Unternehmen}{}\ledrightnote{{$\rightarrow$}\textcolor{green}{Liebelei. Eine Wiener Zeitschrift}} erklärt.\pend
           
\pstart
           Wollen Sie \label{K_L03167-3v}\edtext{heute}{\lemma{\textnormal{\emph{heute}}}\Cendnote{\textnormal{\textcolor{blue}{Schnitzler} datierte den Brief mit »11/12 95«, das angesprochene Konzert von \textcolor{blue}{Eugen
                     Gura} fand jedoch am 12. 12. 1895 statt, weswegen sich \textcolor{blue}{Schnitzler} mit der Datumsangabe um einen Tag vertan haben dürfte.
                  Alternativ wäre es möglich, dass \textcolor{blue}{Salten} den
                  Brief am 11.{ }abends verfasste und also das »heute« vordatierte –
                  wissend, dass es erst am Folgetag in den Händen \textcolor{blue}{Schnitzler}s sein dürfte. Auffällig ist, dass sich auch für das vorhergehende
                  Korrespondenzstück eine ähnliche Argumentation rechtfertigen ließe, siehe Felix Salten an Arthur Schnitzler, [16. 11. 1895].}}}\label{K_L03167-3h} nach \label{K_L03167-4v}\edtext{\textcolor{blue}{Gura}{}\ledrightnote{\textcolor{blue}{Eugen Gura}} zum 
               \uline{\textcolor{blue}{Paulus}{}\ledrightnote{\textcolor{blue}{Paulus}}} (\textcolor{pink}{Ronacher}{}\ledrightnote{\textcolor{pink}{Ronacher}})}{\lemma{\textnormal{\emph{Gura … (Ronacher)}}}\Cendnote{\textnormal{\textcolor{blue}{Schnitzler} besuchte zuerst das Konzert von \textcolor{blue}{Eugen Gura}, dann ging er tatsächlich ins \textcolor{pink}{Ronacher}, siehe A. S.: \emph{Tagebuch}, 12. 12. 1895.}}}\label{K_L03167-4h}
               gehen?\pend
           
\pstart
           Ihr {\\[\baselineskip]}\spacefill\mbox{Salten}\pend
           \leftskip=0em{}\endnumbering\briefempfaengerindex{Schnitzler, Arthur@\textsc{Schnitzler, Arthur}!zzzSalten, Felix@\emph{von Felix Salten}!1895-12-121@{{[}12?. 12. 1895{]}}|)be}\mylabel{h}  \normalsize

\doendnotes{C}
\bigskip
\vfill

\clearpage

\footnotesize

\lohead{\textsc{register}}

% Definiere theindex-Environment komplett neu ohne reledmac
\makeatletter
\renewenvironment{theindex}{%
  \section*{\indexname}%
  \setlength{\parindent}{0pt}%
  \setlength{\parskip}{0pt plus 0.3pt}%
  \let\item\@idxitem
}{%
  \clearpage
}
\makeatother

\IfFileExists{\jobname-pw.ind}{\input{\jobname-pw.ind}}{}

\end{document}

      