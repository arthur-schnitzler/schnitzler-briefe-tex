%% latex-korrekturansicht-vorspann.tex
%% Vorspann für die Korrekturansicht.
%% Lädt die gemeinsame Datei latex-vorspann.tex mit gesetztem Schalter.

\newif\ifkorrekturansicht
\korrekturansichttrue

\input{../tex-inputs/latex-vorspann}


               \section[Paul Goldmann an Arthur Schnitzler, 3. 6. 1893]{ Paul Goldmann an Arthur Schnitzler, 3. 6. 1893}\nopagebreak\mylabel{v}\rehead{ }\normalsize\beginnumbering\briefempfaengerindex{Schnitzler, Arthur@\textsc{Schnitzler, Arthur}!zzzGoldmann, Paul@\emph{von Paul Goldmann}!1893-06-031@{3. 6. 1893}|(be} \toendnotes[C]{\smallbreak\pagebreak[2]} \Standort{DLA, A:Schnitzler, HS.NZ85.1.3163.}
\physDesc{Brief, 3 Blätter, 10 Seiten
\newline{}Handschrift: blaue Tinte, deutsche Kurrent
\newline{}Schnitzler: 1) mit Bleistift das erste Blatt mit »1.«
                                 nummeriert 2) mit rotem Buntstift eine Unterstreichung}\toendnotes[C]{\smallbreak}\pstart
           \noindent{}{\pb}\textcolor{brown}{\textcolor{gray}{\textbf{\textbf{Frankfurter Zeitung}}}}{}\ledrightnote{\textcolor{brown}{Frankfurter Zeitung}}\pend
           \pstart
           \textcolor{gray}{\textbf{und}}\pend
           \pstart
           \textcolor{gray}{\textbf{\textcolor{brown}{\textbf{Handelsblatt}}{}\ledrightnote{→\textcolor{brown}{Frankfurter Zeitung}}.}}\hfill \textcolor{pink}{\textcolor{gray}{\textbf{Frankfurt a. M.}}}{}\ledrightnote{\textcolor{pink}{Frankfurt am Main}}, 3. Juni \textcolor{gray}{\textbf{189}}3.
                  \pend
           \pstart
           \textcolor{gray}{\textbf{\textcolor{brown}{\textbf{Redaktion.}}{}\ledrightnote{→\textcolor{brown}{Frankfurter Zeitung}}\footnote{\noindent{}\textcolor{gray}{\textbf{Für die \textcolor{brown}{Redaktion} bestimmte Briefe und Sendungen wolle
                              man \so{nicht} an die Person eines Redakteurs,
                              sondern stets \textbf{an die \textcolor{brown}{Redaktion} der \textcolor{green}{Frankfurter Zeitung}} adressiren.}}}}}\pend
           \pstart
           \textcolor{gray}{\textbf{\textbf{Telegramm-Adresse:}}}\pend
           \pstart
           \textcolor{gray}{\textbf{\textbf{\textcolor{brown}{Zeitung}{}\ledrightnote{→\textcolor{brown}{Frankfurter Zeitung}}{ }\textcolor{pink}{Frankfurt Main}{}\ledrightnote{\textcolor{pink}{Frankfurt am Main}}.}}}\pend
           \pstart\center{}Mein lieber Arthur!\pend\pstart
           Ich bin für wenige Tage zum Beſuch in \textcolor{pink}{Frankfurt}{}\ledrightnote{\textcolor{pink}{Frankfurt am Main}},
               um der \label{K_L02709-1v}\edtext{Hochzeit meiner \textcolor{blue}{Schweſter}{}\ledrightnote{→\textcolor{blue}{Vally Rosengart}}}{\lemma{\textnormal{\emph{Hochzeit … Schweſter}}}\Cendnote{\textnormal{\textcolor{blue}{Vally Goldmann} heiratete den in \textcolor{pink}{Laupheim} geborenen \textcolor{blue}{Arzt}{ }\textcolor{blue}{Josef Rosengart}.}}}\label{K_L02709-1h} beizuwohnen. Mein
                  \textcolor{blue}{Onkel}{}\ledrightnote{→\textcolor{blue}{Fedor Mamroth}} ſpricht mir
               natürlich von Dir, erzählt mir mit wahrem Enthuſiasmus von Deinem \textcolor{green}{Roman}{}\ledrightnote{→\textcolor{green}{Sterben. Novelle}}, den er als ein bedeutendes \textcolor{green}{Werk}{}\ledrightnote{→\textcolor{green}{Sterben. Novelle}} bezeichnet, und zeigt mir
               ſchließlich Deinen \label{K_L02709-2v}\edtext{Brief}{\lemma{\textnormal{\emph{Brief}}}\Cendnote{\textnormal{nicht erhalten; in seinen Antwortbriefen
                  vom 4. 6. 1893 und 17. 11. 1892 lobte \textcolor{blue}{Fedor Mamroth} ausdrücklich \textcolor{blue}{Schnitzler}s \textcolor{green}{Novelle}{ }\emph{\textcolor{green}{Sterben}}, siehe Fedor Mamroth an Arthur Schnitzler, 5. 3. 1893. Gedruckt wurde \emph{\textcolor{green}{Sterben}}
                  zuerst von Oktober bis Dezember 1894 in den Heften 10 bis 12 der \emph{\textcolor{green}{Neuen Deutschen Rundschau}}.}}}\label{K_L02709-2h}, es tief beklagend, daß
                  \label{K_L02709-3v}\edtext{zwiſchen Dich und ihn etwas
                  getreten}{\lemma{\textnormal{\emph{zwiſchen … getreten}}}\Cendnote{\textnormal{Im Kern geht es, wie aus dem
                  Folgenden hervorgeht, um die ausbleibende Rezension des \emph{\textcolor{green}{Anatol}} in der \emph{\textcolor{brown}{Frankfurter
                     Zeitung}}. In einem größeren Zusammenhang könnte es auch eine Kränkung \textcolor{blue}{Schnitzler}s aufgrund der wiederholten
                  Ablehnungen \textcolor{blue}{Fedor Mamroth}s – zuletzt \emph{\textcolor{green}{Das Märchen}} und \emph{\textcolor{green}{Sterben}} – gegeben haben. Der Brief \textcolor{blue}{Mamroth}s an \textcolor{blue}{Schnitzler} vom 17. 11. 1892
                  legt Nahe, dass \textcolor{blue}{Schnitzler} den ausbleibenden
                  Kontakt nach der Ablehnung des \emph{\textcolor{green}{Märchen}}s als
                  unhöflich empfunden haben könnte.}}}\label{K_L02709-3h} iſt, das beſſer nicht da wäre. Dein
               Brief, mein lieber Freund, iſt {\pb}ebenſo an mich
               gerichtet, wie an meinen \textcolor{blue}{Onkel}{}\ledrightnote{→\textcolor{blue}{Fedor Mamroth}}. Vieles von dem, was Du zu ihm ſagſt, bezieht ſich auch auf mich. Und
               ich kann mich von der Schuld nicht freiſprechen, ein wenig die Bitterkeit
               mitveranlaßt zu haben, von der ich Dich erfüllt ſehe. Objectiv haſt Du vollſtändig
               Recht. Nun aber ſubjektiv: Gewiß, wenn ein Menſch auf der Welt verpflichtet war, über
                  »\textcolor{green}{Anatol}{}\ledrightnote{\textcolor{green}{Anatol}}« zu ſchreiben, ſo war ich es. Das \textcolor{green}{Buch}{}\ledrightnote{→\textcolor{green}{Anatol}} kam bei mir an in einer
               meiner ſchwerſten Arbeitszeiten – Arbeit, von deren Wucht und Depreſ{\pb}ſionsmacht Du keinerlei Ahnung haben kannſt. Ich
               mußte es zurücklegen für ſpäter. Und als dann das »ſpäter« kam, kam über mich das
                  \label{K_L02709-44v}\edtext{Unheil}{\lemma{\textnormal{\emph{Unheil}}}\Cendnote{\textnormal{die Erkrankung an einer Geschlechtskrankheit}}}\label{K_L02709-44h}, das Du
               kennſt, mit der Unmöglichkeit, auch nur ein wenig Spannkraft zu finden, um aus dem
               mechaniſchen Trott der täglichen Arbeit herauszugehen und \strikeout{\textcolor{gray}{×}} ein \textcolor{green}{Werk}{}\ledrightnote{→\textcolor{green}{Anatol}} von Dir in
               einer Deiner würdigen Weiſe zu bearbeiten. Eine kleine Reklamenotiz hätte ich als
               einen \textsc{Affront} für Dich empfunden. Es mußte etwas Hübſches
               und Feines {\pb}ſein. Das aber war ich außerſtande zu
               ſchaffen. Noch heut bin ich es nicht imſtande. Denn
               ich bin nicht geheilt, werde es wohl auch nie werden, und bin durch dieſen Schlag und
               durch gewiſſen ſchweren Familien- und Berufs-Kummer, durch die entſetzliche
               Zukunftsloſigkeit meiner \textsc{Carrière} zerbrochener als je. Um
               Dich nicht warten zu laſſen, ſandte mein \textcolor{blue}{Onkel}{}\ledrightnote{→\textcolor{blue}{Fedor Mamroth}} ſofort Dein \textcolor{green}{Buch}{}\ledrightnote{→\textcolor{green}{Anatol}} unſerem \label{K_L02709-5v}\edtext{\textcolor{pink}{Berlin}{}\ledrightnote{\textcolor{pink}{Berlin}}er \textcolor{blue}{Berichterſtatter}{}\ledrightnote{→\textcolor{blue}{August Stein}{\newline}→\textcolor{blue}{Kurt Eisner}}}{\lemma{\textnormal{\emph{Berliner Berichterſtatter}}}\Cendnote{\textnormal{Es könnte sich hierbei um \textcolor{blue}{August Stein} handeln, der seit 1883 das \textcolor{pink}{Berlin}er Büro
                  der \emph{\textcolor{brown}{Frankfurter Zeitung}} leitete, oder um \textcolor{blue}{Kurt Eisner}.}}}\label{K_L02709-5h}. Der \textcolor{blue}{Herr}{}\ledrightnote{→\textcolor{blue}{August Stein}{\newline}→\textcolor{blue}{Kurt Eisner}} hat einfach nicht
               darüber geſchrieben. Und wie {\pb}bei unſerem \textcolor{brown}{Blatte}{}\ledrightnote{→\textcolor{brown}{Frankfurter Zeitung}} die Verhältniſſe liegen,
               iſt mein \textcolor{blue}{Onkel}{}\ledrightnote{→\textcolor{blue}{Fedor Mamroth}} machtlos, ihn
               dazu zu zwingen. Mein \textcolor{blue}{Onkel}{}\ledrightnote{→\textcolor{blue}{Fedor Mamroth}}
               ſelbſt hat ſich dann längere Zeit mit dem Gedanken getragen, ſelber darüber zu
               ſchreiben. Aber es iſt eine Unproductivität über ihn gekommen, die auch ihm die Feder
               lähmt, ſoweit es ſich nicht um Arbeiten handelt, die der Dienſt von ihm erzwingt. Das
               Alles iſt {\pb}\strikeout{mündlich} ſchriftlich ſchwer auseinanderzuſetzen.
               Mündlich würde ich es Dir leicht begreiflich machen. Das praktiſche Reſultat: Ich
               gehe nach \textcolor{pink}{\textsc{Paris}}{}\ledrightnote{\textcolor{pink}{Paris}} zurück, mit dem feſten Vorſatz, doch über Dein \textcolor{green}{Werk}{}\ledrightnote{→\textcolor{green}{Anatol}} zu \label{K_L02709-7v}\edtext{ſchreiben}{\lemma{\textnormal{\emph{ſchreiben}}}\Cendnote{\textnormal{dazu kam es nicht}}}\label{K_L02709-7h}, kann aber bei meinem ſchwachen Character für nichts einſtehen.
               Das Geſcheiteſte, im Intereſſe einer raſchen Erledigung, wäre, wenn einer von den \textcolor{pink}{Wien}{}\ledrightnote{\textcolor{pink}{Wien}}er Freunden, \textsc{\textcolor{blue}{Richard}{}\ledrightnote{\textcolor{blue}{Richard Beer-Hofmann}}} oder \textsc{\textcolor{blue}{Loris}{}\ledrightnote{\textcolor{blue}{Hugo von Hofmannsthal}}}, uns ein kleines \introOben{}\label{K_L02709-8v}\edtext{Artikelchen}{\lemma{\textnormal{\emph{Artikelchen}}}\Cendnote{\textnormal{dazu kam es nicht}}}\label{K_L02709-8h}\introOben{}{ }\strikeout{\textcolor{gray}{×}\-\textcolor{gray}{×}\-\textcolor{gray}{×}\-\textcolor{gray}{×}\-\textcolor{gray}{×}\-\textcolor{gray}{×}} darüber machen wollte. Mein \textcolor{blue}{Onkel}{}\ledrightnote{→\textcolor{blue}{Fedor Mamroth}} verſpricht {\pb}ſofortigen Abdruck. Wenn
               nicht, ſo gewähre mir, liebſter Freund, noch eine Friſt, und ich will alle Kraft
               aufbieten, um zu thun, was ich Dir ſchulde und was ich auch gar ſo gern thun
               möchte.\pend
           \pstart
           Über den \textcolor{green}{Roman}{}\ledrightnote{→\textcolor{green}{Sterben. Novelle}} haben wir lange
               geſprochen, mein \textcolor{blue}{Onkel}{}\ledrightnote{→\textcolor{blue}{Fedor Mamroth}} und
               ich. Ein Abdruck in der \textcolor{brown}{Frkf.
                     Ztg.}{}\ledrightnote{\textcolor{brown}{Frankfurter Zeitung}} iſt unmöglich wegen der \label{K_L02709-4v}\edtext{Philiſtroſität}{\lemma{\textnormal{\emph{Philiſtroſität}}}\Cendnote{\textnormal{Spießbürgerlichkeit, Engstirnigkeit}}}\label{K_L02709-4h} des Publicums. Weder mein \textcolor{blue}{Onkel}{}\ledrightnote{→\textcolor{blue}{Fedor Mamroth}} noch ich ſind in keinen
               Beziehungen mit einem Verleger. {\pb}Das Einzige, was
               man für’s Erſte thun könnte, wäre ein Brief, den Du dann beifügſt, wenn Du das \textcolor{green}{Manuſkript}{}\ledrightnote{→\textcolor{green}{Sterben. Novelle}} einem \label{K_L02709-6v}\edtext{Verleger Deiner Wahl}{\lemma{\textnormal{\emph{Verleger Deiner Wahl}}}\Cendnote{\textnormal{In Buchform erschien \emph{\textcolor{green}{Sterben}} erstmals im November 1894 (vordatiert auf 1895) bei \emph{\textcolor{brown}{S. Fischer}}.}}}\label{K_L02709-6h} einſchickſt und der
               wenigſtens den Vortheil hat, Dir durch den Namen der \textcolor{brown}{Frankf. Ztg.}{}\ledrightnote{\textcolor{brown}{Frankfurter Zeitung}} jene Accredition zu geben, deren Du bei jenen urtheilsloſen
               Buch-Handwerkern noch bedarfſt. Dein Stolz wird ſich gegen dieſes Mittel wehren, Dein
               Verſtand wird Dir zeigen, daß es doch {\pb}nicht zu
               verſchmähen iſt. Biſt Du aber erſt ein mal mit einem Verleger in Beziehung und brauchſt Du meinen \textcolor{blue}{Onkel}{}\ledrightnote{→\textcolor{blue}{Fedor Mamroth}} oder mich zur weiteren
               Förderung der Angelegenheit, ſo wirſt Du uns auf dem Laufenden erhalten, und
               vielleicht ergibt ſich am Ende doch die Möglichkeit, etwas Poſitiveres und
               Specielleres zu erwirken.\pend
           \pstart
           Der Brief folgt anbei. {\pb}\strikeout{M}Nimm' dieſen Brief auch als Antwort meines \textcolor{blue}{Onkel}{}\ledrightnote{→\textcolor{blue}{Fedor Mamroth}}s, der Dich lieb hat und
               Dir gern das Blaue vom Himmel herunterholen würde, wenn er könnte. Aber Du haſt keine
               Ahnung, w\substVorne{}\textsuperscript{ie}\substDazwischen{}as\substHinten{} für arme, macht- und bedeutungsloſe
               Menſchen wir ſind, er und ich, wir zwei mit dem verfehlten Leben.\pend
           \pstart
           Grüß’ Dich Gott, mein theurer Freund! {\\[\baselineskip]}Dein {\\[\baselineskip]}\spacefill\mbox{Paul Goldmann.}\pend
           \leftskip=0em{}\endnumbering\briefempfaengerindex{Schnitzler, Arthur@\textsc{Schnitzler, Arthur}!zzzGoldmann, Paul@\emph{von Paul Goldmann}!1893-06-031@{3. 6. 1893}|)be}\mylabel{h}\begin{anhang}\end{anhang}\normalsize

\doendnotes{C}
\bigskip
\vfill

\clearpage

\footnotesize

\lohead{\textsc{register}}

% Definiere theindex-Environment komplett neu ohne reledmac
\makeatletter
\renewenvironment{theindex}{%
  \section*{\indexname}%
  \setlength{\parindent}{0pt}%
  \setlength{\parskip}{0pt plus 0.3pt}%
  \let\item\@idxitem
}{%
  \clearpage
}
\makeatother

\IfFileExists{\jobname-pw.ind}{\input{\jobname-pw.ind}}{}

\end{document}

      