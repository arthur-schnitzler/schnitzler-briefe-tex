%% latex-korrekturansicht-vorspann.tex
%% Vorspann für die Korrekturansicht.
%% Lädt die gemeinsame Datei latex-vorspann.tex mit gesetztem Schalter.

\newif\ifkorrekturansicht
\korrekturansichttrue

\input{../tex-inputs/latex-vorspann}


               \section[Arthur Schnitzler an Hermann Bahr, 8. 10. 1910]{ Arthur Schnitzler an Hermann Bahr, 8. 10. 1910}\nopagebreak\mylabel{v}\rehead{ }\normalsize\beginnumbering\briefempfaengerindex{Bahr, Hermann@\textsc{Bahr, Hermann}!zzzSchnitzler, Arthur@\emph{von Arthur Schnitzler}!1910-10-081@{8. 10. 1910}|(be} \toendnotes[C]{\smallbreak\pagebreak[2]} \Standort{TMW, HS AM 60144 Ba.}
\physDesc{Postkarte
\newline{}Schreibmaschine
\newline{}Handschrift: schwarze Tinte, lateinische Kurrent (\noindent{}Anschrift,
                              Unterschrift und Korrekturen)\newline{}Versand: 1) Stempel: »\nobreak{}8. X. 10, 3\nobreak{}«.  2) Stempel: »\nobreak{}\oindex{London@\textbf{London}, \emph{http://www.geonames.org/ontologyP.PPLC}|pwk}London\nobreak{}«. }\buchAbdrucke{\weitereDrucke{1) \emph{8. 10. 1910, Abschrift.} In: Arthur Schnitzler: \emph{The Letters of Arthur Schnitzler to Hermann Bahr}. Edited, annotated, and with an introduction, by Donald G.
                        Daviau. Chapel Hill: \emph{The University of North Carolina Press} 1978, S. 106 (University of North Carolina studies in the Germanic languages
                        and literatures, 89).} \weitereDrucke{2) Hermann Bahr, Arthur Schnitzler: \emph{Briefwechsel, Aufzeichnungen, Dokumente (1891–1931)}. Hg. Kurt Ifkovits und Martin Anton Müller. Göttingen: \emph{Wallstein} 2018, S. 439.} }\toendnotes[C]{\smallbreak}\pstart{}{\pb}\textcolor{gray}{\textbf{Dr. Arthur Schnitzler}}\pend{}\pstart{}\textcolor{gray}{\textbf{\textcolor{pink}{Wien XVIII. Sternwartestrasse 71}{}\ledrightnote{\textcolor{pink}{Sternwartestraße}}}}\pend{}{\bigskip}\pstart{}\textsc{Herrn Hermann Bahr}\pend{}\pstart{}\textsc{\textcolor{pink}{London}{}\ledrightnote{\textcolor{pink}{London}} E. C.}\pend{}\pstart{}\textsc{\textcolor{pink}{Victoria Embankment}{}\ledrightnote{\textcolor{pink}{Victoria Embankment}}}\pend{}\pstart{}\textsc{\textcolor{pink}{D\textsuperscript{r } Kaysers Hotel}{}\ledrightnote{\textcolor{pink}{De Keysers Royal Hotel}}}\pend{}\pstart{}\textsc{\textcolor{pink}{England}{}\ledrightnote{\textcolor{pink}{England}}.}\pend{}{\bigskip}\pstart
           \raggedleft{}{\pb}8. 10. 1910.\pend
           \pstart
           Lieber Hermann. Ein gewisser D\introOben{}r\introOben{}. \textcolor{blue}{Cesare Levi}{}\ledrightnote{\textcolor{blue}{Cesare Levi}} möchte Dein \textcolor{green}{Konzert}{}\ledrightnote{\textcolor{green}{Das Konzert. Lustspiel in drei Akten}} ins \textcolor{pink}{Italienische}{}\ledrightnote{\textcolor{pink}{Italien}}
               übersetzen. Zu seiner Empfehlung kann ich nur sagen, dass in seiner \label{K_L01963_1v}\edtext{Uebersetzung}{\lemma{\textnormal{\emph{Uebersetzung}}}\Cendnote{\textnormal{\emph{\textcolor{green}{Il matrimonio d’Anatolio (Anatols
                     Hochzeitsmorgen)}}, \emph{\textcolor{green}{Cena d’addio
                     (Abschiedssouper)}}, \emph{\textcolor{green}{Letteratura
                     (Literatur)}}, \emph{\textcolor{green}{Il burattinaio (Der
                     Puppenspieler)}} und \emph{\textcolor{green}{L’ultime maschere (Die
                     letzten Masken)}}.}}}\label{K_L01963_1h}{ }\textcolor{green}{einige
                  meiner Einakter}{}\ledrightnote{→\textcolor{green}{Anatols Hochzeitsmorgen}{\newline}→\textcolor{green}{Der Puppenspieler}{\newline}→\textcolor{green}{Abschiedssouper}{\newline}→\textcolor{green}{Die letzten Masken}{\newline}→\textcolor{green}{Literatur}} in \textcolor{pink}{Italien}{}\ledrightnote{\textcolor{pink}{Italien}} aufgeführt
               worden sind und seither eine wahre Flut von Lire auf mich niederströmt. \substVorne{}\textsuperscript{Neulich}{\allowbreak}\substDazwischen{}Im letzten Vierteljahr\substHinten{} waren es vierzehn.\pend
           \pstart
           Nächstens bekommst Du den \textcolor{green}{Medardus}{}\ledrightnote{\textcolor{green}{Der junge Medardus. Dramatische Historie in einem Vorspiel und fünf Aufzügen}}.\pend
           \pstart
           Herzlichst Dein{\\[\baselineskip]}\spacefill\mbox{{[}hs.:{]} Arthur.}\pend
           \leftskip=0em{}\endnumbering\briefempfaengerindex{Bahr, Hermann@\textsc{Bahr, Hermann}!zzzSchnitzler, Arthur@\emph{von Arthur Schnitzler}!1910-10-081@{8. 10. 1910}|)be}\mylabel{h}  \normalsize

\doendnotes{C}
\bigskip
\vfill

\clearpage

\footnotesize

\lohead{\textsc{register}}

% Definiere theindex-Environment komplett neu ohne reledmac
\makeatletter
\renewenvironment{theindex}{%
  \section*{\indexname}%
  \setlength{\parindent}{0pt}%
  \setlength{\parskip}{0pt plus 0.3pt}%
  \let\item\@idxitem
}{%
  \clearpage
}
\makeatother

\IfFileExists{\jobname-pw.ind}{\input{\jobname-pw.ind}}{}

\end{document}

      