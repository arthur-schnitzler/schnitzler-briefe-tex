%% latex-korrekturansicht-vorspann.tex
%% Vorspann für die Korrekturansicht.
%% Lädt die gemeinsame Datei latex-vorspann.tex mit gesetztem Schalter.

\newif\ifkorrekturansicht
\korrekturansichttrue

\input{../tex-inputs/latex-vorspann}


               \section[Paul Goldmann an Arthur Schnitzler, 28. 11. {[}1893{]}]{ Paul Goldmann an Arthur Schnitzler, 28. 11. {[}1893{]}}\nopagebreak\mylabel{v}\rehead{ }\normalsize\beginnumbering\briefempfaengerindex{Schnitzler, Arthur@\textsc{Schnitzler, Arthur}!zzzGoldmann, Paul@\emph{von Paul Goldmann}!1893-11-281@{28. 11. {[}1893{]}}|(be} \toendnotes[C]{\smallbreak\pagebreak[2]} \Standort{DLA, A:Schnitzler, HS.NZ85.1.3163.}
\physDesc{Brief, 1 Blatt, 3 Seiten
\newline{}Handschrift: schwarze Tinte, deutsche Kurrent
\newline{}Schnitzler: mit Bleistift das Jahr »93« vermerkt }\toendnotes[C]{\smallbreak}\pstart
           \raggedleft{}{\pb}\textsc{\textcolor{pink}{Paris}{}\ledrightnote{\textcolor{pink}{Paris}}}, 28. November.\pend
           \pstart\center{}Mein lieber Freund!\pend\pstart
           Ich \label{K_L02720-1v}\edtext{freue mich}{\lemma{\textnormal{\emph{freue mich}}}\Cendnote{\textnormal{\textcolor{blue}{Goldmann} dürfte sich hier auf den
                  Probenbeginn für die Uraufführung von \emph{\textcolor{green}{Märchen}}s beziehen,
               der am 24. 11. 1893
                  stattfand.}}}\label{K_L02720-1h} von Herzen und wünſche Dir ſo viel Glück, ſo viel Glück – ach, es
               iſt ſchwer zu ſagen, wieviel Glück ich Dir wünſche. Wir ſind mitten in einer \label{K_L02720-2v}\edtext{Miniſterkriſis}{\lemma{\textnormal{\emph{Miniſterkriſis}}}\Cendnote{\textnormal{Siehe dazu etwa N. N.: \emph{\textcolor{green}{Privat-Telegramme des »Neuen Wiener Journal«. Miniſterkriſe in
                        Frankreich}}. In: \emph{\textcolor{green}{Neues Wiener
                        Journal}}, Jg. 1, Nr. 35, 26. 11. 1893,
                     S. 5.}}}\label{K_L02720-2h}, und ich muß mir mit tauſend Liſten eine Minute ſtehlen, um
               Dir die Hand drücken zu können. {\pb}Ich kann Dir all’
               das nicht ſagen, was ich Dir ſagen möchte! Ich habe keine Zeit. Es iſt vielleicht
               auch beſſer ſo. Mit einem Worte: Es iſt erreicht, – und das iſt genug. Und \strikeout{\textcolor{gray}{×}\-\textcolor{gray}{×}\-\textcolor{gray}{×}}{ }\strikeout{\textcolor{gray}{×}\-\textcolor{gray}{×}\-\textcolor{gray}{×}\-\textcolor{gray}{×}\-\textcolor{gray}{×}} nun eine Bitte: Am Tage nach der \textcolor{green}{Aufführung}{}\ledrightnote{→\textcolor{green}{Das Märchen. Schauspiel in drei Aufzügen}}, ſo zeitig als Du
               kannſt, ſchickſt Du mir wohl ein Telegramm über \label{K_L02720-3v}\edtext{Aufnahme durch Publicum und Preſſe}{\lemma{\textnormal{\emph{Aufnahme … Preſſe}}}\Cendnote{\textnormal{Am 1. 12. 1893, dem Tag der Premiere des
                     \emph{\textcolor{green}{Märchen}}s im \emph{\textcolor{brown}{Volkstheater}}, notierte \textcolor{blue}{Schnitzler}
                  im \emph{\textcolor{green}{Tagebuch}}, dass das \textcolor{green}{Stück} bis auf den dritten \textcolor{green}{Akt} vom Publikum gut
                  aufgenommen worden sei. Zur Presse schrieb er am drauffolgenden Tag, dem 2. 12. 1893, dass die
                     »Kritiken nicht gar so übel {[}seien{]}; außer den antisemit.
                     Blättern«. siehe Paul Goldmann an Arthur Schnitzler, 6. 12. [1893]}}}\label{K_L02720-3h}?
               Und einen ausführlichen {\pb}Brief hinterdrein, nicht
               wahr?\pend
           \pstart
           Alſo glückauf!!!\pend
           \pstart
           Dein treuer {\\[\baselineskip]}\spacefill\mbox{Paul Goldm}\pend
           \leftskip=0em{}\endnumbering\briefempfaengerindex{Schnitzler, Arthur@\textsc{Schnitzler, Arthur}!zzzGoldmann, Paul@\emph{von Paul Goldmann}!1893-11-281@{28. 11. {[}1893{]}}|)be}\mylabel{h}  \normalsize

\doendnotes{C}
\bigskip
\vfill

\clearpage

\footnotesize

\lohead{\textsc{register}}

% Definiere theindex-Environment komplett neu ohne reledmac
\makeatletter
\renewenvironment{theindex}{%
  \section*{\indexname}%
  \setlength{\parindent}{0pt}%
  \setlength{\parskip}{0pt plus 0.3pt}%
  \let\item\@idxitem
}{%
  \clearpage
}
\makeatother

\IfFileExists{\jobname-pw.ind}{\input{\jobname-pw.ind}}{}

\end{document}

      