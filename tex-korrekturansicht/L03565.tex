%% latex-korrekturansicht-vorspann.tex
%% Vorspann für die Korrekturansicht.
%% Lädt die gemeinsame Datei latex-vorspann.tex mit gesetztem Schalter.

\newif\ifkorrekturansicht
\korrekturansichttrue

\input{../tex-inputs/latex-vorspann}


\renewcommand{\erwaehntePersonen}{Personen: Richard Beer-Hofmann, Paula Beer-Hofmann, Lili Cappellini, Felix Salten, Olga Schnitzler, Heinrich Schnitzler}
\renewcommand{\erwaehnteOrte}{Orte: Bad Ischl, Balkanhalbinsel, Berghof, Sarajevo, Schweiz, Serbien, Unterach am Attersee, Weißenbach am Attersee, Wien, Österreich}
\renewcommand{\erwaehnteWerke}{}
\section[ Felix Salten an Arthur Schnitzler, 10. 8. 1914]{Felix Salten an Arthur Schnitzler, 10. 8. 1914}
\nopagebreak\mylabel{v}
\rehead{ }\normalsize\beginnumbering\briefempfaengerindex{Schnitzler, Arthur@\textsc{Schnitzler, Arthur}!zzzSalten, Felix@\emph{von Felix Salten}!1914-08-102@{10. 8. 1914}|(be}
\toendnotes[C]{\smallbreak\pagebreak[2]}\Standort{CUL, Schnitzler, B 89, B 2.}
\physDesc{Briefkarte, 890 Zeichen
\newline{}Handschrift: schwarze Tinte, lateinische Kurrent
\newline{}Schnitzler: 1) mit Bleistift Vermerk: »\textsc{Salten}«  2) mit rotem Buntstift eine Unterstreichung
\newline{}Ordnung: mit Bleistift von unbekannter Hand nummeriert: »278« }\toendnotes[C]{\smallbreak}
\pstart
           \raggedleft{}{\pb}\textcolor{pink}{Berghof}{}\ledrightnote{\textcolor{pink}{Berghof}}, 10. 8. 14\pend
           
\pstart{}Lieber,\pend
\pstart
           Ihre Karte aus der \label{K_L03565-1v}\edtext{\textcolor{pink}{Schweiz}{}\ledrightnote{\textcolor{pink}{Schweiz}}}{\lemma{\textnormal{\emph{Schweiz}}}\Cendnote{\textnormal{\textcolor{blue}{Schnitzler} war am 18. 7. 1914 mit seiner \textcolor{blue}{Frau}
                  und den \textcolor{blue}{Kindern}
                  in der \textcolor{pink}{Schweiz} angekommen. Die Heimreise nach Kriegsausbruch
                  erwies sich als schwierig. Am 15. 8. 1914 reisten sie 
                  nach \textcolor{pink}{Österreich}, zuerst aber nach \textcolor{pink}{Bad Ischl}. Am 2. 9. 1914 waren sie wieder in
                     \textcolor{pink}{Wien}.}}}\label{K_L03565-1h} bekam ich vor zwei Tagen,
               nehme aber an, dass Sie jetzt wieder zu Hause sind. Wann ich nach \textcolor{pink}{Wien}{}\ledrightnote{\textcolor{pink}{Wien}} komme, weiß ich nicht, weiß nicht einmal, ob ich es soll.
                  \textcolor{pink}{Hier}{}\ledrightnote{{$\rightarrow$}\textcolor{pink}{Unterach am Attersee}} ist es so ganz still,
               ganz einsam und das beruhigt einigermaßen. Sonst – wenn man sich’s klar macht, was
               jetzt geschieht und warum es geschieht – könnte man verzweifeln. Wer dran glaubt,
                  \label{K_L03565-2v}\edtext{dies alles sei wegen \textcolor{pink}{Serbien}{}\ledrightnote{\textcolor{pink}{Serbien}}}{\lemma{\textnormal{\emph{dies … Serbien}}}\Cendnote{\textnormal{Das Attentat von \textcolor{pink}{Sarajevo} wurde in Zusammenhang mit Bestrebungen \textcolor{pink}{Serbien}s gesehen,
                  das eine politische Einigung am \textcolor{pink}{Balkan} unter seiner Führung anstrebte.}}}\label{K_L03565-2h}, ist eigentlich zu
               beneiden. Denn er hat doch etwas, um sein Rechtsgefühl damit zu füttern. Vielleicht
               ist es gut, dass dieser Krieg eben \label{K_L03565-3v}\edtext{jetzt ausgebrochen}{\lemma{\textnormal{\emph{jetzt ausgebrochen}}}\Cendnote{\textnormal{\textcolor{pink}{Österreich} hatte \textcolor{pink}{Serbien} am 28. 7. 1914 den
                  Krieg erklärt. Damit hatte der Erste Weltkrieg begonnen.}}}\label{K_L03565-3h} wird. Gut: für
               unsere Söhne. Das mag hässlich und egoistisch gedacht sein, aber ich denke es eben.
                  \textcolor{blue}{Beer-Hofmanns}{}\ledrightnote{\textcolor{blue}{Richard Beer-Hofmann}{\newline}\textcolor{blue}{Paula Beer-Hofmann}} sind hier in \textcolor{pink}{Weißenbach}{}\ledrightnote{\textcolor{pink}{Weißenbach am Attersee}}. Ich glaube, sie sind dort fast die
               einzigen. Wir sehen uns manchmal. Lassen Sie mich wißen, wie es bei Ihnen geht. Viele
               herzlichste Grüße von uns an Sie \textcolor{blue}{Beide}{}\ledrightnote{{$\rightarrow$}\textcolor{blue}{Olga Schnitzler}} und die \textcolor{blue}{Kinder}{}\ledrightnote{{$\rightarrow$}\textcolor{blue}{Heinrich Schnitzler}{\newline}{$\rightarrow$}\textcolor{blue}{Lili Cappellini}}!\pend
           \pstart Ihr \spacefill\mbox{Salten}\pend{}\endnumbering\briefempfaengerindex{Schnitzler, Arthur@\textsc{Schnitzler, Arthur}!zzzSalten, Felix@\emph{von Felix Salten}!1914-08-102@{10. 8. 1914}|)be}\mylabel{h}  \normalsize

\doendnotes{C}
\bigskip
\vfill

\clearpage

\footnotesize

\lohead{\textsc{register}}

% Definiere theindex-Environment komplett neu ohne reledmac
\makeatletter
\renewenvironment{theindex}{%
  \section*{\indexname}%
  \setlength{\parindent}{0pt}%
  \setlength{\parskip}{0pt plus 0.3pt}%
  \let\item\@idxitem
}{%
  \clearpage
}
\makeatother

\IfFileExists{\jobname-pw.ind}{\input{\jobname-pw.ind}}{}

\end{document}

      