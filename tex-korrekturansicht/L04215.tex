%% latex-korrekturansicht-vorspann.tex
%% Vorspann für die Korrekturansicht.
%% Lädt die gemeinsame Datei latex-vorspann.tex mit gesetztem Schalter.

\newif\ifkorrekturansicht
\korrekturansichttrue

\input{../tex-inputs/latex-vorspann}


\section[Arthur Schnitzler an Romain Rolland, 11. 2. 1915]{L04215 Arthur Schnitzler an Romain Rolland, 11. 2. 1915}
\nopagebreak\mylabel{L04215v}
\rehead{ }\normalsize\beginnumbering\briefempfaengerindex{Rolland, Romain@\textsc{Rolland, Romain}!zzzSchnitzler, Arthur@\emph{von Arthur Schnitzler}!1915-02-111@{11. 2. 1915}|(be}
\toendnotes[C]{\smallbreak\pagebreak[2]}
\correspDesc{Versand  durch Arthur Schnitzler am 11. 2. 1915 in Wien
\newline{}Erhalt  durch Romain Rolland im Zeitraum [12. 2. 1915 – 16. 2. 1915?] in Genf}\toendnotes[C]{\smallbreak}
\Standort{Paris, Bibliothèque Nationale de France, Fonds Romain Rolland, Cote NAF 28400.}
\physDesc{Briefkarte, 775 Zeichen
\newline{}Handschrift: schwarze Tinte, lateinische Kurrent
\newline{}Ordnung: mit Bleistift Blatt paginiert: »7« }\toendnotes[C]{\smallbreak}
\pstart
           {\pb}\textcolor{gray}{\textbf{Dr. Arthur Schnitzler}}\hfill 11. Feber 1915.\pend
           
\pstart
           \textcolor{gray}{\textbf{\textcolor{pink}{Wien XVIII. Sternwartestrasse 71}\oindex{Wien@\textbf{Wien}!XVIII., Währing@\textbf{XVIII., Währing}!Sternwartestraße 71@\textbf{Sternwartestraße 71}, \emph{Wohngebäude}|pw}{}\ledrightnote{\textcolor{pink}{Sternwartestraße 71}}}}\pend
           \vspace{0.5em}
\pstart
           verehrter Herr Rolland, durch \label{K_L03884-1v}\edtext{\textcolor{blue}{Stefan Zweig}\pwindex{Zweig, Stefan 28.\,11.\,1881 Wien – 23.\,2.\,1942 Petrópolis@\textsc{Zweig, Stefan} (28.\,11.\,1881 Wien – 23.\,2.\,1942 Petrópolis), \emph{Schriftsteller}|pw}{}\ledrightnote{\textcolor{blue}{Stefan Zweig}}
              empfang ich Ihre lieben Grüße}{\lemma{\textnormal{\emph{Stefan … Grüße}}}\Cendnote{\textnormal{Stefan Zweig an Arthur Schnitzler, [zwischen 7. und
               10. 2. 1915?].}}}\label{K_L03884-1}, die ich herzlichst erwidre.
              Sie haben also die Angriffe oder wenigstens von den
              Angriffen gelesen, die anläßlich meines von Ihnen so
              schön übersetzten Protestes gegen mich gerichtet worden
              sind. Übrigens standen sie nur in antisemiti¬
              schen Blättern – und von dieser Seite bin ich dergleichen
              seit Jahren, ja seit Jahrzehnten, bei jeder möglichen und
              {\pb}unmöglichen Gelegenheit so reichlich gewohnt, daß sie
              mich vollko{\geminationm}en kalt lassen. Warum sollte der
              Krieg gerade auf diese traurige Menschensorte eine
              »läuternde« Wirkung ausüben, da doch auch anderswo nicht eben viel davon zu merken ist.\pend
           
\pstart
           Auf bessere Zeiten denn, und einen freundschaftlichen Händedruck Ihres sehr ergebenen
              {\\[\baselineskip]}\spacefill\mbox{Arthur Schnitzler}\pend
           \leftskip=0em{}\selectlanguage{ngerman}\endnumbering\briefempfaengerindex{Rolland, Romain@\textsc{Rolland, Romain}!zzzSchnitzler, Arthur@\emph{von Arthur Schnitzler}!1915-02-111@{11. 2. 1915}|)be}\mylabel{L04215h}
\begin{anhang}
\end{anhang}\normalsize

\doendnotes{C}
\bigskip
\vfill

\clearpage

\footnotesize

\lohead{\textsc{register}}

% Definiere theindex-Environment komplett neu ohne reledmac
\makeatletter
\renewenvironment{theindex}{%
  \section*{\indexname}%
  \setlength{\parindent}{0pt}%
  \setlength{\parskip}{0pt plus 0.3pt}%
  \let\item\@idxitem
}{%
  \clearpage
}
\makeatother

\IfFileExists{\jobname-pw.ind}{\input{\jobname-pw.ind}}{}

\end{document}

      