%% latex-korrekturansicht-vorspann.tex
%% Vorspann für die Korrekturansicht.
%% Lädt die gemeinsame Datei latex-vorspann.tex mit gesetztem Schalter.

\newif\ifkorrekturansicht
\korrekturansichttrue

\input{../tex-inputs/latex-vorspann}


\renewcommand{\erwaehnteInstitutionen}{Institutionen: Volkstheater}
\renewcommand{\erwaehnteOrte}{Orte: Berlin, Charlottenstraße, Frankgasse, Lanzsch {\kaufmannsund}  Co., Schauspielhaus, Wien}
\renewcommand{\erwaehnteWerke}{Werke: Abschiedssouper, Anatol, Der Schleier der Beatrice. Schauspiel in fünf Akten, Weihnachts-Einkäufe}
\section[Paul Goldmann, Bertha und Rudolf Christians an Arthur Schnitzler, 24. 3. 1901]{Paul Goldmann, Bertha und Rudolf Christians an Arthur
               Schnitzler, 24. 3. 1901}
\nopagebreak\mylabel{v}
\rehead{ }\normalsize\beginnumbering\briefempfaengerindex{Schnitzler, Arthur@\textsc{Schnitzler, Arthur}!zzzKlein, Bertha@\emph{von Bertha Klein}!1901-03-241@{24. 3. 1901}|(be}\briefempfaengerindex{Schnitzler, Arthur@\textsc{Schnitzler, Arthur}!zzzChristians, Rudolf@\emph{von Rudolf Christians}!1901-03-241@{24. 3. 1901}|(be}\briefempfaengerindex{Schnitzler, Arthur@\textsc{Schnitzler, Arthur}!zzzGoldmann, Paul@\emph{von Paul Goldmann}!1901-03-241@{24. 3. 1901}|(be}
\toendnotes[C]{\smallbreak\pagebreak[2]}\Standort{DLA, A:Schnitzler, HS.NZ85.1.3171.}
\physDesc{Bildpostkarte
\newline{}Handschrift Paul Goldmann: 1) Bleistift, deutsche Kurrent\hspace{1em}2) Bleistift, lateinische Kurrent (\noindent{}Adresse)\hspace{1em}
\newline{}Handschrift Rudolf Christians: Bleistift, deutsche Kurrent
\newline{}Handschrift Bertha Klein: Bleistift, deutsche Kurrent
\newline{}Versand: 1) Stempel: »\nobreak{}\oindex{Berlin@\textbf{Berlin}, \emph{https://www.geonames.org/ontologyP.PPLC}|pwk}Berlin W, 24. 3. {[}19{]}01, \textcolor{gray}{9–3 V.} 8h\nobreak{}«.   2) Stempel: »\nobreak{}Wien 9/3 72, 25. 3. {[}19{]}01, 8. V, Bestellt\nobreak{}«. }\toendnotes[C]{\smallbreak}\pstart{}{\pb}Herrn\pend{}\pstart{}Dr. Arthur Schnitzler\pend{}\pstart{}\textcolor{pink}{Wien}{}\ledrightnote{\textcolor{pink}{Wien}}\pend{}\pstart{}\textcolor{pink}{IX. Frankgaße 1}{}\ledrightnote{\textcolor{pink}{Frankgasse}}.\pend{}
{\bigskip}
\pstart
           \noindent{}\centering{}{\pb}\textcolor{gray}{\textbf{\textbf{Restaurant ersten Ranges \textcolor{pink}{Lanzsch
                           {\kaufmannsund} Co.}{}\ledrightnote{\textcolor{pink}{Lanzsch {\kaufmannsund} Co.}}}}}\pend
           
\pstart
           \noindent{}\centering{}\textcolor{gray}{\textbf{\textbf{\textcolor{pink}{BERLIN}{}\ledrightnote{\textcolor{pink}{Berlin}}}, \textcolor{pink}{Charlotten-Strasse 56}{}\ledrightnote{\textcolor{pink}{Charlottenstraße}}}}\pend
           
\pstart
           \noindent{}\centering{}\textcolor{gray}{\textbf{\begin{otherlanguage}{french}vis à vis\end{otherlanguage}{ }\textcolor{pink}{Schauspielhaus}{}\ledrightnote{\textcolor{pink}{Schauspielhaus}}}}\pend
           
\pstart
           \noindent{}Lieber Freund, Gerade erzählt mir Herr \textsc{\textcolor{blue}{Christians}{}\ledrightnote{\textcolor{blue}{Rudolf Christians}}}, daß er der \label{K_L03061-1v}\edtext{erſte \textsc{\textcolor{green}{Anatol}{}\ledrightnote{{$\rightarrow$}\textcolor{green}{Anatol}}}}{\lemma{\textnormal{\emph{erſte Anatol}}}\Cendnote{\textnormal{\textcolor{blue}{Rudolf Christians} spielte 1898 den \textcolor{green}{Anatol}
                  in den \emph{\textcolor{green}{Weihnachts-Einkäufen}} und im \emph{\textcolor{green}{Abschiedssouper}} im \emph{\textcolor{brown}{Volkstheater}}.}}}\label{K_L03061-1h} war. Wir benutzen die Gelegenheit,
               Dir einen Gruß zu ſenden. Herzlichſt Dein \spacefill\mbox{Paul Goldmann.}\pend
           
\pstart\center{}{[}hs. Christians:{]} Mein ſehr verehrter, lieber Herr
                  Schnitzler!\pend
\pstart
           Ich freue mich richtig, Ihnen, verehrteſter Herr D\textsuperscript{r}, in
               Erinnerung an unſere »\textsc{\textcolor{green}{Weihnachtseinkäufe}{}\ledrightnote{\textcolor{green}{Weihnachts-Einkäufe}}}« die herzlichſten Grüße zu ſenden! Was macht »\textsc{\textcolor{green}{Schleier der Beatrice}{}\ledrightnote{\textcolor{green}{Der Schleier der Beatrice. Schauspiel in fünf Akten}}}«? Warum nicht ich?\pend
           \pstart Ihr \spacefill\mbox{Christians}\pend{}
\pstart
           \noindent{}{[}hs. Klein:{]} Höflichen Gruß \spacefill\mbox{Bertha Christians.}\pend
           \endnumbering\briefempfaengerindex{Schnitzler, Arthur@\textsc{Schnitzler, Arthur}!zzzKlein, Bertha@\emph{von Bertha Klein}!1901-03-241@{24. 3. 1901}|)be}\briefempfaengerindex{Schnitzler, Arthur@\textsc{Schnitzler, Arthur}!zzzChristians, Rudolf@\emph{von Rudolf Christians}!1901-03-241@{24. 3. 1901}|)be}\briefempfaengerindex{Schnitzler, Arthur@\textsc{Schnitzler, Arthur}!zzzGoldmann, Paul@\emph{von Paul Goldmann}!1901-03-241@{24. 3. 1901}|)be}\mylabel{h}
\begin{anhang}
\end{anhang}\normalsize

\doendnotes{C}
\bigskip
\vfill

\clearpage

\footnotesize

\lohead{\textsc{register}}

% Definiere theindex-Environment komplett neu ohne reledmac
\makeatletter
\renewenvironment{theindex}{%
  \section*{\indexname}%
  \setlength{\parindent}{0pt}%
  \setlength{\parskip}{0pt plus 0.3pt}%
  \let\item\@idxitem
}{%
  \clearpage
}
\makeatother

\IfFileExists{\jobname-pw.ind}{\input{\jobname-pw.ind}}{}

\end{document}

      