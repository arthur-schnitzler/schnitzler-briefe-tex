%% latex-korrekturansicht-vorspann.tex
%% Vorspann für die Korrekturansicht.
%% Lädt die gemeinsame Datei latex-vorspann.tex mit gesetztem Schalter.

\newif\ifkorrekturansicht
\korrekturansichttrue

\input{../tex-inputs/latex-vorspann}


         
         \renewcommand{\erwaehntePersonen}{Personen: Moriz Benedikt, Rudolf Lothar}
         \renewcommand{\erwaehnteInstitutionen}{Institutionen: Neue Freie Presse, Volkstheater}
         \renewcommand{\erwaehnteOrte}{Orte: Berlin, Dessauer Straße, Deutsches Theater Berlin, Europa, Wien}
         \renewcommand{\erwaehnteWerke}{Werke: König Harlekin. Maskenspiel in vier Aufzügen, Neue Freie Presse}
               \section[ Paul Goldmann an Arthur Schnitzler, 10. 5. {[}1900{]}]{Paul Goldmann an Arthur Schnitzler, 10. 5. {[}1900{]}}\nopagebreak\mylabel{v}\rehead{ }\normalsize\beginnumbering\briefempfaengerindex{Schnitzler, Arthur@\textsc{Schnitzler, Arthur}!zzzGoldmann, Paul@\emph{von Paul Goldmann}!1900-05-101@{10. 5. {[}1900{]}}|(be} \toendnotes[C]{\smallbreak\pagebreak[2]} \Standort{DLA, A:Schnitzler, HS.NZ85.1.3170.}
\physDesc{Brief, 1 Blatt, 3 Seiten
\newline{}Handschrift: blaue Tinte, deutsche Kurrent
\newline{}Schnitzler: 1) mit Bleistift das Jahr »{[}1{]}900« vermerkt  2) mit rotem Buntstift eine Unterstreichung}\toendnotes[C]{\smallbreak}\pstart
           \noindent{}{\pb}\textcolor{pink}{\textcolor{gray}{\textbf{DESSAUERSTRASSE 19}}}{}\ledrightnote{\textcolor{pink}{Dessauer Straße}}\pend
           \pstart
           \raggedleft{}\textcolor{pink}{Berlin}{}\ledrightnote{\textcolor{pink}{Berlin}}, 10. Mai.\pend
           \pstart\center{}Mein lieber Freund,\pend\pstart
           Als ich das letzte Mal in \textcolor{pink}{Wien}{}\ledrightnote{\textcolor{pink}{Wien}} war, ſprachen wir
               über \textsc{\textcolor{blue}{Rudolf Lothar}{}\ledrightnote{\textcolor{blue}{Rudolf Lothar}}}, und Du ſagteſt, er ſei ein anſtändiger Menſch. Laß’ Dir folgenden Beitrag zu
               ſeiner Anſtändigkeit liefern:\pend
           \pstart
           Heut bekomme ich einen Brief von der Redaktion der \textcolor{brown}{N. Fr. Pr.}{}\ledrightnote{\textcolor{brown}{Neue Freie Presse}}, welcher mich informirt, daß \textsc{\textcolor{blue}{Lothar}{}\ledrightnote{\textcolor{blue}{Rudolf Lothar}}} bei \textsc{\textcolor{blue}{Benedikt}{}\ledrightnote{\textcolor{blue}{Moriz Benedikt}}} war und erwirkt hat, daß ich über ſein \textcolor{green}{Stück}{}\ledrightnote{{$\rightarrow$}\textcolor{green}{König Harlekin. Maskenspiel in vier Aufzügen}}, welches das \label{K_L02915-1v}\edtext{\textcolor{brown}{Volkstheater}{}\ledrightnote{\textcolor{brown}{Volkstheater}}}{\lemma{\textnormal{\emph{Volkstheater}}}\Cendnote{\textnormal{Nachdem die für den 31. 3. 1900 geplante \textcolor{pink}{Wien}er Premiere von \textcolor{blue}{Rudolf Lothar}s
                  Satire \emph{\textcolor{green}{König Harlekin}} aus Zensurgründen
                  abgesagt worden war, kam es am 19. 5. 1900 als
                  Gastspiel des \textcolor{pink}{Wien}er \emph{\textcolor{brown}{Volkstheater}}s am \textcolor{pink}{Deutschen
                     Theater Berlin} zur Uraufführung. In \textcolor{pink}{Wien} fand die Premiere am 14. 9. 1901
                  statt. Das \textcolor{green}{Stück} wurde
                  aufgrund seiner politischen Tendenzen \textcolor{pink}{europa}weit zensiert.}}}\label{K_L02915-1h}{ }\textcolor{pink}{hier}{}\ledrightnote{{$\rightarrow$}\textcolor{pink}{Berlin}} zur Aufführung bringt, \uline{nicht} referire. Demgemäß erhalte ich die Weiſung, \strikeout{d\textcolor{gray}{em}} den »\textsc{\textcolor{green}{König Harlekin}{}\ledrightnote{\textcolor{green}{König Harlekin. Maskenspiel in vier Aufzügen}}}« aus meinem Referat auszuſchalten.\pend
           \pstart
           {\pb}Das heißt alſo: Dieſer \textcolor{blue}{Burſche}{}\ledrightnote{{$\rightarrow$}\textcolor{blue}{Rudolf Lothar}} weiß ſehr wohl, daß ich nicht lüge
               und daß ich, wenn ſein \textcolor{green}{Stück}{}\ledrightnote{{$\rightarrow$}\textcolor{green}{König Harlekin. Maskenspiel in vier Aufzügen}},
               wie vorauszuſehen, einen Mißerfolg haben wird, einen Mißerfolg conſtatiren werde.
               Darum benutzt er ſeinen Einfluß, um mich aus meinem Kritiker-Amt zu verdrängen und um
                  \introOben{}dann\introOben{} ſelbſt an die \textcolor{brown}{N. Fr.
                  Pr.}{}\ledrightnote{\textcolor{brown}{Neue Freie Presse}}{ }\strikeout{d} gefälſchte \label{K_L02915-3v}\edtext{Berichte}{\lemma{\textnormal{\emph{Berichte}}}\Cendnote{\textnormal{siehe Paul Goldmann an Arthur Schnitzler, 27. 5. [1900]}}}\label{K_L02915-3h} abzuſenden \textsc{resp.} ſie durch eine Kreatur abſenden zu
               laſſen.\pend
           \pstart
           Was \strikeout{\textcolor{gray}{×}} ich Dir da ſage, iſt Dienſtgeheimmiß, und ich muß Dich daher um ſtrengſte
               Diskretion bitten.\pend
           \pstart
           {\pb}Hingegen würdeſt Du mir einen großen Gefallen
               erweiſen, wenn Du allen Freunden und Bekannten mittheilen wollteſt, ich hätte Dir
               geſchrieben, daß ich über \textsc{\textcolor{blue}{Lothar}{}\ledrightnote{\textcolor{blue}{Rudolf Lothar}}s}{ }\textcolor{green}{Stück}{}\ledrightnote{{$\rightarrow$}\textcolor{green}{König Harlekin. Maskenspiel in vier Aufzügen}} weder im \textcolor{green}{Feuilleton}{}\ledrightnote{{$\rightarrow$}\textcolor{green}{Neue Freie Presse}} noch in der \textcolor{green}{Theaterrubrik}{}\ledrightnote{{$\rightarrow$}\textcolor{green}{Neue Freie Presse}} berichten würde.\pend
           \pstart
           Was treibſt Du ſonſt, mein lieber Freund? Mache mir bald wieder einmal die Freude
               eines Briefes.\pend
           \pstart
           Viele treue Grüße! {\\[\baselineskip]}Dein {\\[\baselineskip]}\spacefill\mbox{Paul Goldmann.}\pend
           \leftskip=0em{}\endnumbering\briefempfaengerindex{Schnitzler, Arthur@\textsc{Schnitzler, Arthur}!zzzGoldmann, Paul@\emph{von Paul Goldmann}!1900-05-101@{10. 5. {[}1900{]}}|)be}\mylabel{h}\begin{anhang}\end{anhang}\normalsize

\doendnotes{C}
\bigskip
\vfill

\clearpage

\footnotesize

\lohead{\textsc{register}}

% Definiere theindex-Environment komplett neu ohne reledmac
\makeatletter
\renewenvironment{theindex}{%
  \section*{\indexname}%
  \setlength{\parindent}{0pt}%
  \setlength{\parskip}{0pt plus 0.3pt}%
  \let\item\@idxitem
}{%
  \clearpage
}
\makeatother

\IfFileExists{\jobname-pw.ind}{\input{\jobname-pw.ind}}{}

\end{document}

      