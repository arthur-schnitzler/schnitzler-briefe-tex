%% latex-korrekturansicht-vorspann.tex
%% Vorspann für die Korrekturansicht.
%% Lädt die gemeinsame Datei latex-vorspann.tex mit gesetztem Schalter.

\newif\ifkorrekturansicht
\korrekturansichttrue

\input{../tex-inputs/latex-vorspann}


\renewcommand{\erwaehntePersonen}{Personen: Caroline Kotter, Felix Salten, Ottilie Salten}
\renewcommand{\erwaehnteInstitutionen}{Institutionen: Neue akademische Vereinigung}
\renewcommand{\erwaehnteOrte}{Orte: Brünn, Deutsches Haus, Wien}
\renewcommand{\erwaehnteWerke}{}
\section[ Arthur Schnitzler an Felix Salten, 15. 10. 1903]{Arthur Schnitzler an Felix Salten, 15. 10. 1903}
\nopagebreak\mylabel{v}
\rehead{ }\normalsize\beginnumbering\briefempfaengerindex{Salten, Felix@\textsc{Salten, Felix}!zzzSchnitzler, Arthur@\emph{von Arthur Schnitzler}!1903-10-151@{15. 10. 1903}|(be}
\toendnotes[C]{\smallbreak\pagebreak[2]}\Standort{Wienbibliothek im Rathaus, ZPH 1681, 2.1.516.}
\physDesc{Brief, 1 Blatt, 4 Seiten, 659 Zeichen
\newline{}Handschrift: Bleistift, deutsche Kurrent
\newline{}Ordnung: mit Bleistift von unbekannter Hand Nummerierung der Doppelseiten des
                                 Konvoluts: »51«–»52« }\toendnotes[C]{\smallbreak}
\pstart
           \raggedleft{}{\pb}15. 10. 903.\pend
           
\pstart
           lieber, gegen \label{K_L02985-1v}\edtext{Mittwoch nächſter Woche}{\lemma{\textnormal{\emph{Mittwoch nächſter Woche}}}\Cendnote{\textnormal{siehe A. S.: \emph{Tagebuch}, 21. 10. 1903}}}\label{K_L02985-1h} hab ich nichts einzuwenden. \textcolor{gray}{×}\-\textcolor{gray}{×}{ }\textcolor{gray}{×}\-\textcolor{gray}{×}\pend
           
\pstart
           Tagesausflug iſt mir kein verführeriſcher Gedanke. Hingegen ſchlag ich Ihnen vor, mit
                  \textcolor{blue}{Otti}{}\ledrightnote{\textcolor{blue}{Ottilie Salten}} und dem kleinen \textcolor{blue}{Fräulein}{}\ledrightnote{{$\rightarrow$}\textcolor{blue}{Caroline Kotter}}{ }\label{K_L02985-2v}\edtext{So{\geminationn}tag}{\lemma{\textnormal{\emph{Sonntag}}}\Cendnote{\textnormal{siehe A. S.: \emph{Tagebuch}, 18. 10. 1903}}}\label{K_L02985-2h} (um 1, we{\geminationn}s {\pb}Ihnen recht iſt) bei uns zu ſpeiſen – We{\geminationn} das Wetter ſchön iſt, iſt bei uns auch Land. Und dann
               können Sie noch immer in fernere Fernen. –\pend
           
\pstart
           Wenn nicht (was ſchade wäre) ſo wählen Sie bitte irgend einen Abend der {\pb}nächſten Woche, an dem wir das Vergnügen
               haben können, Sie bei uns zu ſehen – nur nicht Montag:
               da wartet mein der \label{K_L02985-3v}\edtext{Vorleſetiſch in dem
                  \textcolor{pink}{Tuchmacherſtädtchen}{}\ledrightnote{{$\rightarrow$}\textcolor{pink}{Brünn}}}{\lemma{\textnormal{\emph{Vorleſetiſch … Tuchmacherſtädtchen}}}\Cendnote{\textnormal{\textcolor{blue}{Schnitzler}
                  las am
                  19. 10. 1903
                  für die
                  \emph{\textcolor{brown}{Neue akademische Vereinigung}}
                  im kleinen Festsaal des
                  \textcolor{pink}{Deutschen Hauses}.}}}\label{K_L02985-3h}. —\pend
           
\pstart
           Herzlichſt {\\[\baselineskip]}Ihr {\\[\baselineskip]}\spacefill\mbox{A.}\pend
           \leftskip=0em{}
\pstart
           \noindent{}{\pb}Wollen Sie So{\geminationn}tag eine andere Stunde, ſo beſtimmen
                  Sie\pend
           
\pstart
           \strikeout{\textcolor{gray}{[2 Zeilen unleserlich{]} }}\pend
           
\pstart
           {[}Zeichnung einer Straßenbahn{]}\pend
           \endnumbering\briefempfaengerindex{Salten, Felix@\textsc{Salten, Felix}!zzzSchnitzler, Arthur@\emph{von Arthur Schnitzler}!1903-10-151@{15. 10. 1903}|)be}\mylabel{h}  \normalsize

\doendnotes{C}
\bigskip
\vfill

\clearpage

\footnotesize

\lohead{\textsc{register}}

% Definiere theindex-Environment komplett neu ohne reledmac
\makeatletter
\renewenvironment{theindex}{%
  \section*{\indexname}%
  \setlength{\parindent}{0pt}%
  \setlength{\parskip}{0pt plus 0.3pt}%
  \let\item\@idxitem
}{%
  \clearpage
}
\makeatother

\IfFileExists{\jobname-pw.ind}{\input{\jobname-pw.ind}}{}

\end{document}

      