%% latex-korrekturansicht-vorspann.tex
%% Vorspann für die Korrekturansicht.
%% Lädt die gemeinsame Datei latex-vorspann.tex mit gesetztem Schalter.

\newif\ifkorrekturansicht
\korrekturansichttrue

\input{../tex-inputs/latex-vorspann}


               \section[Arthur Schnitzler: Widmungsexemplar Ruf des Lebens für Hugo von Hofmannsthal, 21. 3. 1909]{ Arthur Schnitzler: Widmungsexemplar Ruf des Lebens für Hugo von
                    Hofmannsthal, 21. 3. 1909}\nopagebreak\mylabel{v}\rehead{ }\normalsize\beginnumbering\briefempfaengerindex{Hofmannsthal, Hugo von@\textsc{Hofmannsthal, Hugo von}!zzzSchnitzler, Arthur@\emph{von Arthur Schnitzler}!1909-03-211@{21. 3. 1909}|(be} \toendnotes[C]{\smallbreak\pagebreak[2]} \Standort{FDH, FDH 3233.}
\physDesc{Widmung am Vortitel
\newline{}Handschrift: schwarze Tinte, deutsche Kurrent\newline{}Ordnung: mit Bleistift von unbekannter Hand beschriftet: »HvH-S. CI, 54« }\buchAbdrucke{\weitereDrucke{Hugo von Hofmannsthal: \emph{Bibliothek}. Hg. Ellen Ritter † in Zusammenarbeit mit Dalia Bukauskaité und
                                Konrad Heumann. Frankfurt am Main: \emph{S. Fischer} 2011, S. 606 (Sämtliche Werke. Kritische Ausgabe, XL).} }\pstart \spacefill\mbox{{\pb}Arthur Schnitzler}\pend{}\pstart
           \textcolor{pink}{Wien}{}\ledrightnote{\textcolor{pink}{Wien}}{ }21. 3. 09.\pend
           {\bigskip}\pstart
           \noindent{}{\pb}\textcolor{gray}{\textbf{\textcolor{green}{Der Ruf des Lebens}{}\ledrightnote{\textcolor{green}{Der Ruf des Lebens. Schauspiel in drei Akten}}}}\pend
           \pstart
           \textcolor{gray}{\textbf{Schauſpiel in drei Akten von Arthur Schnitzler}}\pend
           {\bigskip}\pstart
           \noindent{}\textcolor{gray}{\textbf{Als Bühnen-Manuſkript gedruckt und vervielfältigt. Das
                        Recht der Aufführung ist nur von \textcolor{brown}{S. Fiſcher,
                            Verlag}{}\ledrightnote{\textcolor{brown}{S. Fischer Verlag}} (Theaterabteilung) in \textcolor{pink}{Berlin W.,
                            Bülowſtr. 91}{}\ledrightnote{\textcolor{pink}{Bülowstraße}} zu erwerben.}}\pend
           \endnumbering\briefempfaengerindex{Hofmannsthal, Hugo von@\textsc{Hofmannsthal, Hugo von}!zzzSchnitzler, Arthur@\emph{von Arthur Schnitzler}!1909-03-211@{21. 3. 1909}|)be}\mylabel{h}  \normalsize

\doendnotes{C}
\bigskip
\vfill

\clearpage

\footnotesize

\lohead{\textsc{register}}

% Definiere theindex-Environment komplett neu ohne reledmac
\makeatletter
\renewenvironment{theindex}{%
  \section*{\indexname}%
  \setlength{\parindent}{0pt}%
  \setlength{\parskip}{0pt plus 0.3pt}%
  \let\item\@idxitem
}{%
  \clearpage
}
\makeatother

\IfFileExists{\jobname-pw.ind}{\input{\jobname-pw.ind}}{}

\end{document}

      