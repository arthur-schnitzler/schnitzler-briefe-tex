%% latex-korrekturansicht-vorspann.tex
%% Vorspann für die Korrekturansicht.
%% Lädt die gemeinsame Datei latex-vorspann.tex mit gesetztem Schalter.

\newif\ifkorrekturansicht
\korrekturansichttrue

\input{../tex-inputs/latex-vorspann}


\section[Stefan Zweig an Arthur Schnitzler, 13. 5. 1923]{L03665 Stefan Zweig an Arthur Schnitzler, 13. 5. 1923}
\nopagebreak\mylabel{L03665v}
\rehead{ }\normalsize\beginnumbering\briefempfaengerindex{, @\textsc{, }!zzz, @\emph{von  }!1923-05-131@{13. 5. 1923}|(be}
\toendnotes[C]{\smallbreak\pagebreak[2]}\Standort{CUL, Schnitzler, B 118.}
\physDesc{Brief, 1 Blatt, 1 Seite, 638 Zeichen
\newline{}Handschrift: blaue Tinte, lateinische Kurrent
\newline{}Beilage: \emph{Deutschen Literaturarchiv Marbach},
                                    HS.NZ85.1.4978: Ausschnitt mit den Seiten 60 und 61 aus
                                 dem Antiquariatskatalog von \textcolor{brown}{Emil
                                    Hirsch}\orgindex{Antiquariat Emil Hirsch@Antiquariat Emil Hirsch|pw}, 1 Blatt, 2 Seiten. Die Angabe der Lot-Nummer und
                                 die Adresse des Antiquariats mit Bleistift unterstrichen. Der
                                 Hinweis auf Schnitzler in der Beschreibung mit blauem Buntstift
                                 (von Zweig?) unterstrichen. Auf der ersten Seite mit rotem
                                 Buntstift Vermerk von Schnitzler: »\textcolor{blue}{\textsc{Krell}}\pwindex{Krell, Max 24.\,9.\,1887 Hubertusburg – 11.\,6.\,1962 Florenz@\textsc{Krell, Max} (24.\,9.\,1887 Hubertusburg – 11.\,6.\,1962 Florenz), \emph{Schriftsteller, Verlagslektor}|pw}«. 
\newline{}Schnitzler: mit rotem Buntstift eine Unterstreichung }
\buchAbdrucke{\weitereDrucke{Stefan Zweig: \emph{Briefwechsel mit Hermann Bahr, Sigmund Freud, Rainer Maria
                        Rilke und Arthur Schnitzler}. Herausgegeben von Jeffrey B. Berlin, Hans-Ulrich Lindken und Donald A. Prater. Frankfurt am Main: \emph{S. Fischer} 1987, S. 414.} }\toendnotes[C]{\smallbreak}
\pstart
           {\pb}\textcolor{pink}{\uline{Salzburg}, Kapuzinerberg 5}\oindex{Paschinger Schlössl@\textbf{Paschinger Schlössl}, \emph{Wohngebäude}|pw}{}\ledrightnote{\textcolor{pink}{Paschinger Schlössl}}\hfill 13. Mai 1923\pend
           \vspace{0.5em}
\pstart
           Lieber verehrter Herr Doktor, in einem Versteigerungskatalog
               entdecke ich eben dieses \label{K_L03665-1v}\edtext{\textcolor{green}{Buch}\pwindex{Wassermann, Jakob 10.\,3.\,1873 Fürth – 1.\,1.\,1934 Altaussee@\textsc{Wassermann, Jakob} (10.\,3.\,1873 Fürth – 1.\,1.\,1934 Altaussee), \emph{Schriftsteller}!Gänsemännchen. Roman@\strich\emph{Das Gänsemännchen. Roman}|pwv}{}\ledrightnote{{$\rightarrow$}\emph{\textcolor{green}{Das Gänsemännchen. Roman}}}}{\lemma{\textnormal{\emph{Buch}}}\Cendnote{\textnormal{Am 29. 5. 1923 schrieb \textcolor{blue}{Schnitzler} an den Antiquar \textcolor{blue}{Emil Hirsch}\pwindex{Hirsch, Emil 14.\,3.\,1866 Bad Mergentheim – 27.\,7.\,1954 New York City@\textsc{Hirsch, Emil} (14.\,3.\,1866 Bad Mergentheim – 27.\,7.\,1954 New York City), \emph{Verleger, Antiquar}|pwk}: »29. 5. 1923.{ / }Sehr geehrter Herr.{ / }In ihrem Versteigerungskatalog, der mir von befreundeter Seite zugesandt
                        wird, finde ich auf Seite 60, Nr. 784, \textcolor{blue}{Wassermann Jakob}\pwindex{Wassermann, Jakob 10.\,3.\,1873 Fürth – 1.\,1.\,1934 Altaussee@\textsc{Wassermann, Jakob} (10.\,3.\,1873 Fürth – 1.\,1.\,1934 Altaussee), \emph{Schriftsteller}|pw}, \textcolor{green}{Das
                           Gänsemännchen}\pwindex{Wassermann, Jakob 10.\,3.\,1873 Fürth – 1.\,1.\,1934 Altaussee@\textsc{Wassermann, Jakob} (10.\,3.\,1873 Fürth – 1.\,1.\,1934 Altaussee), \emph{Schriftsteller}!Gänsemännchen. Roman@\strich\emph{Das Gänsemännchen. Roman}|pw} mit handschriftlicher Widmung an \textcolor{blue}{Arthur} und \textcolor{blue}{Olga
                           Schnitzler}\pwindex{Schnitzler, Olga 17.\,1.\,1882 Wien – 13.\,1.\,1970 Lugano@\textsc{Schnitzler, Olga} (17.\,1.\,1882 Wien – 13.\,1.\,1970 Lugano), \emph{Schauspielerin, Sängerin}|pw}. Ich ersuche hiemit die Versteigerung dieses Buches, das
                        auf eine mir vorläufig unbegreifliche Weise aus meinem Besitz verschwunden
                        ist, zu unterlassen und das mir gehörige Exemplar an meine Adresse
                        freundlichst rücksenden zu wollen.{ / }Mit vorzüglicher Hochachtung{ / }{[}Raum für Unterschrift{]}{ / }Herrn \textcolor{blue}{Emil Hirsch}\pwindex{Hirsch, Emil 14.\,3.\,1866 Bad Mergentheim – 27.\,7.\,1954 New York City@\textsc{Hirsch, Emil} (14.\,3.\,1866 Bad Mergentheim – 27.\,7.\,1954 New York City), \emph{Verleger, Antiquar}|pw}, Verleger,{ / }\textcolor{pink}{München}\oindex{München@\textbf{München}|pw}.« (\emph{DLA}, HS.1985.1.1016). Aus den zwei weiteren
                  Schreiben \textcolor{blue}{Schnitzlers} an \textcolor{blue}{Hirsch}\pwindex{Hirsch, Emil 14.\,3.\,1866 Bad Mergentheim – 27.\,7.\,1954 New York City@\textsc{Hirsch, Emil} (14.\,3.\,1866 Bad Mergentheim – 27.\,7.\,1954 New York City), \emph{Verleger, Antiquar}|pwk} geht hervor, dass das Exemplar von \textcolor{blue}{Max Krell}\pwindex{Krell, Max 24.\,9.\,1887 Hubertusburg – 11.\,6.\,1962 Florenz@\textsc{Krell, Max} (24.\,9.\,1887 Hubertusburg – 11.\,6.\,1962 Florenz), \emph{Schriftsteller, Verlagslektor}|pwk} zum Verkauf freigegeben wurde –
                  der es wiederum von \textcolor{blue}{Schnitzlers} Schwägerin
                     \textcolor{blue}{Elisabeth Steinrück}\pwindex{Steinrück, Elisabeth 19.\,11.\,1885 – 7.\,4.\,1920 Partenkirchen@\textsc{Steinrück, Elisabeth} (19.\,11.\,1885 – 7.\,4.\,1920 Partenkirchen)|pwk} bezogen hatte. Ob nun
                  diese oder \textcolor{blue}{Krell}\pwindex{Krell, Max 24.\,9.\,1887 Hubertusburg – 11.\,6.\,1962 Florenz@\textsc{Krell, Max} (24.\,9.\,1887 Hubertusburg – 11.\,6.\,1962 Florenz), \emph{Schriftsteller, Verlagslektor}|pwk} das Buch sich zu Unrecht
                  angeeignet hat, lässt sich nicht mehr bestimmen. Die erhaltene Korrespondenz
                  zwischen \textcolor{blue}{Schnitzler} und \textcolor{blue}{Krell}\pwindex{Krell, Max 24.\,9.\,1887 Hubertusburg – 11.\,6.\,1962 Florenz@\textsc{Krell, Max} (24.\,9.\,1887 Hubertusburg – 11.\,6.\,1962 Florenz), \emph{Schriftsteller, Verlagslektor}|pwk} ist im betreffenden Zeitraum ausgesetzt. Das Buch
                  dürfte letztlich an \textcolor{blue}{Schnitzler} retourniert
                  worden sein, wohingegen er dem Verleger ein von ihm gewidmetes Exemplar der
                  Erstausgabe von \emph{\textcolor{green}{Das Märchen}\pwindex{Schnitzler, Arthur 15. 5. 1862 Wien – 21. 10. 1931 ebd.@\textsc{Schnitzler, Arthur} (15. 5. 1862 Wien – 21. 10. 1931 ebd.), \emph{Schriftsteller, Mediziner}!Märchen. Schauspiel in drei Aufzügen@\strich\emph{Das Märchen. Schauspiel in drei Aufzügen}|pwk}} zukommen ließ.
               }}}\label{K_L03665-1}. Da ich nicht annehme, dass Sie die Exemplare Ihrer gewidmeten Bücher
               verkaufen (vielleicht werden wir bald so weit sein) so handelt es sich offenbar um
               ein entwendetes Exemplar und Sie haben wohl das Recht es zurückzufordern. Ich glaubte
               Sie aufmerksam machen zu müssen, weil ich selbst jüngst ähnlich einem entwendeten
                  \label{K_L03665-2v}\edtext{\textcolor{green}{Buch}\pwindex{?? [Widmungsexemplar eines unbekannten Buchs an Stefan Zweig, 1923]@\emph{?? [Widmungsexemplar eines unbekannten Buchs an Stefan Zweig, 1923]}|pwv}{}\ledrightnote{{$\rightarrow$}\emph{\textcolor{green}{?? [Widmungsexemplar eines unbekannten Buchs an Stefan Zweig, 1923]}}}}{\lemma{\textnormal{\emph{Buch}}}\Cendnote{\textnormal{nicht identifiziert}}}\label{K_L03665-2} auf die Spur
               kam – und dann freue ich mich jeder Gelegenheit, Ihnen meine herzliche Verehrung
               aussprechen zu können. \pend
           
\pstart
           Ihr getreuer{\\[\baselineskip]}\spacefill\mbox{Stefan Zweig}\pend
           \leftskip=0em{}\selectlanguage{ngerman}\vspace{1em}
\pstart
           \noindent{}{[}\ldots{]}\pend
           
\pstart
           {\pb}\textcolor{gray}{\textbf{784}}\hspace*{1.5em}\textcolor{gray}{\textbf{\textcolor{blue}{\textbf{WASSERMANN}, Jakob}\pwindex{Wassermann, Jakob 10.\,3.\,1873 Fürth – 1.\,1.\,1934 Altaussee@\textsc{Wassermann, Jakob} (10.\,3.\,1873 Fürth – 1.\,1.\,1934 Altaussee), \emph{Schriftsteller}|pw}{}\ledrightnote{\textcolor{blue}{Jakob Wassermann}}. \textcolor{green}{Das Gänsemännchen.
                     Roman}\pwindex{Wassermann, Jakob 10.\,3.\,1873 Fürth – 1.\,1.\,1934 Altaussee@\textsc{Wassermann, Jakob} (10.\,3.\,1873 Fürth – 1.\,1.\,1934 Altaussee), \emph{Schriftsteller}!Gänsemännchen. Roman@\strich\emph{Das Gänsemännchen. Roman}|pw}{}\ledrightnote{\textcolor{green}{Das Gänsemännchen. Roman}}. \textcolor{pink}{Berlin}\oindex{Berlin@\textbf{Berlin}, \emph{Hauptstadt}|pw}{}\ledrightnote{\textcolor{pink}{Berlin}}, \textcolor{brown}{S. Fischer}\orgindex{S. Fischer Verlag@S. Fischer Verlag|pw}{}\ledrightnote{\textcolor{brown}{S. Fischer Verlag}}, 1915. 8. Origlwd. }}\pend
           
\pstart
           \textcolor{gray}{\textbf{Erste Ausgabe.{ }}}{ }\so{Mit }{ }\so{handschriftl. }{ }\so{Widmung}{ }\textcolor{gray}{\textbf{{ }des Verf. an Arthur u. Olga Schnitzler.}}\pend
           
\pstart
           {[}\ldots{]}\pend
           
\pstart
           {\pb}{[}\ldots{]}\pend
           
\pstart
           \centering{}\textcolor{gray}{\textbf{\textbf{\textcolor{brown}{Emil Hirsch}\orgindex{Antiquariat Emil Hirsch@Antiquariat Emil Hirsch|pw}{}\ledrightnote{\textcolor{brown}{Antiquariat Emil Hirsch}}, \textcolor{pink}{Karlstr. 10, München}\oindex{Karlstraße 10@\textbf{Karlstraße 10}, \emph{Gebäude}|pw}{}\ledrightnote{\textcolor{pink}{Karlstraße 10}}.}}}\pend
           
\pstart
           \raggedleft{}Versteigerung 4. Juni\pend
           \selectlanguage{ngerman}\endnumbering\briefempfaengerindex{, @\textsc{, }!zzz, @\emph{von  }!1923-05-131@{13. 5. 1923}|)be}\mylabel{L03665h}
\begin{anhang}
\end{anhang}\normalsize

\doendnotes{C}
\bigskip
\vfill

\clearpage

\footnotesize

\lohead{\textsc{register}}

% Definiere theindex-Environment komplett neu ohne reledmac
\makeatletter
\renewenvironment{theindex}{%
  \section*{\indexname}%
  \setlength{\parindent}{0pt}%
  \setlength{\parskip}{0pt plus 0.3pt}%
  \let\item\@idxitem
}{%
  \clearpage
}
\makeatother

\IfFileExists{\jobname-pw.ind}{\input{\jobname-pw.ind}}{}

\end{document}

      