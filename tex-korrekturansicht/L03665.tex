%% latex-korrekturansicht-vorspann.tex
%% Vorspann für die Korrekturansicht.
%% Lädt die gemeinsame Datei latex-vorspann.tex mit gesetztem Schalter.

\newif\ifkorrekturansicht
\korrekturansichttrue

\input{../tex-inputs/latex-vorspann}


\renewcommand{\erwaehntePersonen}{Personen: Olga Schnitzler, Jakob Wassermann, Stefan Zweig}
\renewcommand{\erwaehnteInstitutionen}{Institutionen: Antiquariat Emil Hirsch}
\renewcommand{\erwaehnteOrte}{Orte: Paschinger Schlössl, Salzburg, Wien}
\renewcommand{\erwaehnteWerke}{Werke: ?? [Widmungsexemplar eines unbekannten Buchs an Stefan Zweig, 1923], Das Gänsemännchen. Roman}
\section[Stefan Zweig an Arthur Schnitzler, 13. 5. 1923]{Stefan Zweig an Arthur Schnitzler, 13. 5. 1923}
\nopagebreak\mylabel{v}
\rehead{ }\normalsize\beginnumbering\briefempfaengerindex{Schnitzler, Arthur@\textsc{Schnitzler, Arthur}!zzzZweig, Stefan@\emph{von Stefan Zweig}!1923-05-131@{13. 5. 1923}|(be}
\toendnotes[C]{\smallbreak\pagebreak[2]}\Standort{CUL, Schnitzler, B 118.}
\physDesc{Brief, 1 Blatt, 1 Seite, 590 Zeichen
\newline{}Handschrift: blaue Tinte, lateinische Kurrent}\toendnotes[C]{\smallbreak}
\pstart
           {\pb}\textcolor{pink}{\uline{Salzburg}, Kapuzinerberg 5}{}\ledrightnote{\textcolor{pink}{Paschinger Schlössl}}\hfill 13. Mai 1923\pend
           
\pstart
           Lieber verehrter Herr Doktor, in einem Versteigerungskatalog
               entdecke ich eben dieses \label{K_L03665-1v}\edtext{\textcolor{green}{Buch}{}\ledrightnote{{$\rightarrow$}\textcolor{green}{Das Gänsemännchen. Roman}}}{\lemma{\textnormal{\emph{Buch}}}\Cendnote{\textnormal{Die Herausgeber der ersten Edition der Korrespondenz \textcolor{blue}{Schnitzler}–\textcolor{blue}{Zweig} nennen als Titel den Roman \emph{\textcolor{green}{Das
                     Gänsemännchen}} (1915) von \textcolor{blue}{Jakob Wassermann}, der vom \emph{\textcolor{brown}{Antiquariat
                     Emil Hirsch}} mit Widmung an \textcolor{blue}{Olga} und
                     \textcolor{blue}{Arthur Schnitzler} angeboten wurde. (\textcolor{blue}{Stefan Zweig}: \emph{Briefwechsel mit Hermann Bahr, Sigmund Freud, Rainer Maria Rilke und Arthur
                        Schnitzler}, S. 480)
                  }}}\label{K_L03665-1h}. Da ich nicht annehme, dass Sie die Exemplare Ihrer gewidmeten Bücher
               verkaufen (vielleicht werden wir bald so weit sein) so handelt es sich offenbar um
               ein entwendetes Exemplar und Sie haben wohl das Recht es zurückzufordern. Ich glaubte
               Sie aufmerksam machen zu müssen, weil ich selbst jüngst ähnlich einem entwendeten
                  \label{K_L03665-2v}\edtext{\textcolor{green}{Buch}{}\ledrightnote{{$\rightarrow$}\textcolor{green}{?? [Widmungsexemplar eines unbekannten Buchs an Stefan Zweig, 1923]}}}{\lemma{\textnormal{\emph{Buch}}}\Cendnote{\textnormal{nicht
                  identifiziert}}}\label{K_L03665-2h} auf die Spur kam – und dann freue ich mich jeder Gelegenheit,
               Ihnen meine herzliche Verehrung aussprechen zu können. \pend
           
\pstart
           Ihr getreuer{\\[\baselineskip]}\spacefill\mbox{Stefan Zweig}\pend
           \leftskip=0em{}\endnumbering\briefempfaengerindex{Schnitzler, Arthur@\textsc{Schnitzler, Arthur}!zzzZweig, Stefan@\emph{von Stefan Zweig}!1923-05-131@{13. 5. 1923}|)be}\mylabel{h}
\begin{anhang}
\end{anhang}\normalsize

\doendnotes{C}
\bigskip
\vfill

\clearpage

\footnotesize

\lohead{\textsc{register}}

% Definiere theindex-Environment komplett neu ohne reledmac
\makeatletter
\renewenvironment{theindex}{%
  \section*{\indexname}%
  \setlength{\parindent}{0pt}%
  \setlength{\parskip}{0pt plus 0.3pt}%
  \let\item\@idxitem
}{%
  \clearpage
}
\makeatother

\IfFileExists{\jobname-pw.ind}{\input{\jobname-pw.ind}}{}

\end{document}

      