%% latex-korrekturansicht-vorspann.tex
%% Vorspann für die Korrekturansicht.
%% Lädt die gemeinsame Datei latex-vorspann.tex mit gesetztem Schalter.

\newif\ifkorrekturansicht
\korrekturansichttrue

\input{../tex-inputs/latex-vorspann}


               \section[Hugo von Hofmannsthal an Arthur Schnitzler, 9. 9. {[}1891{]}]{ Hugo von Hofmannsthal an Arthur Schnitzler,
                    9. 9. {[}1891{]}}\nopagebreak\mylabel{v}\rehead{ }\normalsize\beginnumbering\briefempfaengerindex{Schnitzler, Arthur@\textsc{Schnitzler, Arthur}!zzzHofmannsthal, Hugo von@\emph{von Hugo von Hofmannsthal}!1891-09-091@{9. 9. {[}1891{]}}|(be} \toendnotes[C]{\smallbreak\pagebreak[2]} \Standort{CUL, Schnitzler, B 43.}
\physDesc{Brief, 1 Blatt, 2 Seiten
\newline{}Handschrift: schwarze Tinte, deutsche Kurrent
\newline{}Schnitzler: mit Bleistift die Jahreszahl hinzugefügt: »91« \newline{}Ordnung: mit Bleistift von unbekannter Hand nummeriert:
                                    »7« }\buchAbdrucke{\weitereDrucke{Hugo von Hofmannsthal, Arthur Schnitzler: \emph{Briefwechsel}. Hg. Therese Nickl und Heinrich Schnitzler. Frankfurt am Main: \emph{S. Fischer} 1964, S. 13.} }\toendnotes[C]{\smallbreak}\pstart
           \noindent{}{\pb}Daſs Sie mich überhaupt noch
                    grüßen laſſen, iſt wirklich hübſch von Ihnen. Der \label{K_L00038_1v}\edtext{Anfang}{\lemma{\textnormal{\emph{Anfang}}}\Cendnote{\textnormal{\textcolor{blue}{Arthur Schnitzler}: \emph{\textcolor{green}{Reichtum}}. In: \emph{\textcolor{green}{Moderne
                                Rundschau}}, Bd. 3, H. 11, 1. 9. 1891, S. 385–391
                            (1. von 4 Teilen).}}}\label{K_L00038_1h} von »\textcolor{green}{Reichthum}{}\ledrightnote{\textcolor{green}{Reichtum. Erzählung}}« ſcheint mir mit ſeiner Märchenſtimmung und ſeinen
                    unwahrſcheinlichen Ariſtokratennamen etwas phantaſtiſches, \textcolor{blue}{arnimeskes}{}\ledrightnote{→\textcolor{blue}{Achim von Arnim}} zu verſprechen. Dann wäre es
                    mir doppelt ſympathiſch.\pend
           \pstart
           Aber – es wird doch nicht vielleicht eine ſociale Novelle werden wollen? Ich
                    hoffe, Sie und \textcolor{blue}{Hoffmann}{}\ledrightnote{\textcolor{blue}{Richard Beer-Hofmann}} werden mir über die
                    erſten 8 Tage in {\pb}\textcolor{pink}{Wien}{}\ledrightnote{\textcolor{pink}{Wien}} hinweghelfen; vorläufig kann ich mir das
                        \label{K_L00038_2v}\edtext{Aufhören}{\lemma{\textnormal{\emph{Aufhören}}}\Cendnote{\textnormal{Mitte September 1891 war Schulbeginn, \textcolor{blue}{Hofmannsthal}s abschließendes Schuljahr begann.}}}\label{K_L00038_2h}
                    oder das Ertragen des Aufhörens nicht vorſtellen.\pend
           \pstart
           Herzlichſt{\\[\baselineskip]}\spacefill\mbox{Loris.}\pend
           \leftskip=0em{}\pstart
           \textsc{9. IX. im Segelboot.}\pend
           \endnumbering\briefempfaengerindex{Schnitzler, Arthur@\textsc{Schnitzler, Arthur}!zzzHofmannsthal, Hugo von@\emph{von Hugo von Hofmannsthal}!1891-09-091@{9. 9. {[}1891{]}}|)be}\mylabel{h}  \normalsize

\doendnotes{C}
\bigskip
\vfill

\clearpage

\footnotesize

\lohead{\textsc{register}}

% Definiere theindex-Environment komplett neu ohne reledmac
\makeatletter
\renewenvironment{theindex}{%
  \section*{\indexname}%
  \setlength{\parindent}{0pt}%
  \setlength{\parskip}{0pt plus 0.3pt}%
  \let\item\@idxitem
}{%
  \clearpage
}
\makeatother

\IfFileExists{\jobname-pw.ind}{\input{\jobname-pw.ind}}{}

\end{document}

      