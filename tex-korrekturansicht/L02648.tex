%% latex-korrekturansicht-vorspann.tex
%% Vorspann für die Korrekturansicht.
%% Lädt die gemeinsame Datei latex-vorspann.tex mit gesetztem Schalter.

\newif\ifkorrekturansicht
\korrekturansichttrue

\input{../tex-inputs/latex-vorspann}


               \section[Paul Goldmann an Arthur Schnitzler, 11. 8. 1890]{ Paul Goldmann an Arthur Schnitzler, 11. 8. 1890}\nopagebreak\mylabel{v}\rehead{ }\normalsize\beginnumbering\briefempfaengerindex{Schnitzler, Arthur@\textsc{Schnitzler, Arthur}!zzzGoldmann, Paul@\emph{von Paul Goldmann}!1890-08-111@{11. 8. 1890}|(be} \toendnotes[C]{\smallbreak\pagebreak[2]} \Standort{DLA, A:Schnitzler, HS.NZ85.1.3162.}
\physDesc{Brief, 2 Blätter, 7 Seiten
\newline{}Handschrift: schwarze Tinte, deutsche Kurrent
\newline{}Schnitzler: mit rotem Buntstift eine Unterstreichung }\toendnotes[C]{\smallbreak}\pstart
           \noindent{}\centering{}{\pb}\textcolor{gray}{\textbf{\textbf{Adminiſtration: \textcolor{pink}{VII.
                           Seidengaſſe 7}{}\ledrightnote{\textcolor{pink}{Seidengasse}}} (\textcolor{brown}{Jos. Eberle {\kaufmannsund} Co.}{}\ledrightnote{\textcolor{brown}{Josef Eberle  Stein-, Buch und Musikaliendruckerei}})}}\pend
           \pstart
           \noindent{}\centering{}\textcolor{gray}{\textbf{\textcolor{brown}{An der Schönen Blauen Donau}{}\ledrightnote{\textcolor{brown}{An der schönen blauen Donau}}}}\pend
           \pstart
           \noindent{}\centering{}\textcolor{gray}{\textbf{Chef-Redacteur: Dr. \textcolor{blue}{F.
                        Mamroth}{}\ledrightnote{\textcolor{blue}{Fedor Mamroth}}. – Redaction: \textcolor{pink}{IX.,
                        Berggaſſe 31}{}\ledrightnote{\textcolor{pink}{Berggasse}}.}}\pend
           \pstart
           \raggedleft{}\textsc{\textcolor{pink}{Pörtschach}{}\ledrightnote{\textcolor{pink}{Pörtschach}}}{ }\textcolor{gray}{\textbf{\strikeout{\textcolor{pink}{Wien}{}\ledrightnote{\textcolor{pink}{Wien}}}, den}}{ }11. August \textcolor{gray}{\textbf{18}}90.\pend
           \pstart\center{}Lieber Arthur!\pend\pstart
           Du haſt Recht gehabt: ich bin von dieſer \label{K_L02648-11v}\edtext{\textcolor{blue}{Frau}{}\ledrightnote{→\textcolor{blue}{Olga Waissnix}}}{\lemma{\textnormal{\emph{Frau}}}\Cendnote{\textnormal{Mit \textcolor{blue}{Olga Waissnix} verband \textcolor{blue}{Schnitzler} in
                  den Jahren nach 1886 eine für ihn bedeutsame Liebesbeziehung. Sie war
                  die Wirtin des \textcolor{pink}{Thalhof}es in \textcolor{pink}{Reichenau}. Ihr Ehemann \textcolor{blue}{Carl Waissnix} wird zugleich als gutmütig und eifersüchtig beschrieben.
                     \textcolor{blue}{Schnitzler} und \textcolor{blue}{Goldmann} hatten sich am 7. 8. 1890 zuletzt gesehen, so dass der
                  zweitägige Besuch im \textcolor{pink}{Thalhof} auf dem Weg nach
                     \textcolor{pink}{Pörtschach} stattfand und zeitlich
                  weitgehend genau eingegrenzt werden kann.}}}\label{K_L02648-11h} mit einer Empfindung warmer und
               aufrichtiger Sympathie weggegangen. Viele Fehler wohl, \strikeout{\textcolor{gray}{aber}} die typischen Fehler der ſchönen Frau: eitel, \label{K_L02648-1v}\edtext{\begin{otherlanguage}{french}\textsc{poseure}\end{otherlanguage}}{\lemma{\textnormal{\emph{poseure}}}\Cendnote{\textnormal{französisch: wichtigtuerisch}}}\label{K_L02648-1h},
                  \begin{otherlanguage}{french}coquett\end{otherlanguage}; aber wenn man auf den Grund kommt,
               findet man einen Schatz von Ehrlichkeit und Natürlichkeit. Ich bin der \textcolor{blue}{Frau}{}\ledrightnote{→\textcolor{blue}{Olga Waissnix}} mit allen möglichen
               Vorurtheilen {\pb}entgegengekommen; aber
               als wir am letzten Tag allein im Walde ſaßen und die gewiſſen tieferen Sachen
               beſprachen, da kam ein ſo heißer Glückshunger, ein ſo rechtes Streben nach dem
               Beſſeren zutage, daß ich dabei etwas empfand, das ich nicht anders, als Rührung
               nennen kann. Ich bin der Frau \textsc{\textcolor{blue}{Olga}{}\ledrightnote{\textcolor{blue}{Olga Waissnix}}} ein wahrer Freund geworden; und in dieser Eigenschaft muß ich Dir Eines sagen:
               Du darfſt dieſe \textcolor{blue}{Frau}{}\ledrightnote{→\textcolor{blue}{Olga Waissnix}} unter
               keinen Umſtänden \label{K_L02648-2v}\edtext{betrügen}{\lemma{\textnormal{\emph{betrügen}}}\Cendnote{\textnormal{Die Beziehung zwischen \textcolor{blue}{Olga Waissnix} und \textcolor{blue}{Schnitzler} war weitgehend platonisch, doch wie dieser Brief, aber auch
                  die im \emph{\textcolor{green}{Tagebuch}} festgehaltenen Küsse
                  beweisen, waren sie sich zu diesem Zeitpunkt der Beziehung unsicher, ob das so
                  bleiben sollte.}}}\label{K_L02648-2h}. Sie iſt auf Alles vorbereitet: daß das Liebesglück, das
               ſie ſucht, kurz dauern, daß es mit Qualen verbunden sein und mit Enttäuſchungen enden
               kann. Aber in einer Beziehung glaubt ſie an Dich – meine Vermuthung; {\pb}Confidencen hat’s nicht gegeben –
               daß Du ſie nur dann zur \textcolor{blue}{Deinigen}{}\ledrightnote{→\textcolor{blue}{Olga Waissnix}} machen wirſt, wenn du ſie liebſt. Ich habe mit Erſtaunen geſehen,
               daß dieſe \textcolor{blue}{Frau}{}\ledrightnote{→\textcolor{blue}{Olga Waissnix}} wirklich und
               ehrlich kämpft und daß es \substVorne{}\textsuperscript{\textcolor{gray}{gl}}\substDazwischen{}ſie\substHinten{} einen großen Entschluß koſtet, über ſo und ſoviel Pflichten hinweg dahin zu
               gehen, wo ſie ihr Glück vermuthet. Aber eben darum hat ſie doppelt das Recht, nicht
               getäuſcht zu werden. Wenn ſie wieder zu Dir kommt – und ſie wird wieder kommen, ich
               glaube das iſt das Facit unſerer Geſpräche, ich habe mich bemüht ihr Muth zum Glück
               zu machen – ſo ſage ihr, wie es mit Dir ſteht. Will ſie dann immer noch, ſo brauchſt
               Du keine Scrupeln {\pb}mehr zu haben.
               Aber dieſe \textcolor{blue}{Frau}{}\ledrightnote{→\textcolor{blue}{Olga Waissnix}} aus bloßer
               Sinnenluſt zu genießen, mit einer Lüge auf der Zunge, wäre ein Verrath an Allem, was
               gut und edel iſt auf der Welt{\dotsfour}\pend
           \pstart
           Dies, \label{K_L02648-3v}\edtext{\textsc{ut animam meam salvarem}}{\lemma{\textnormal{\emph{ut animam meam salvarem}}}\Cendnote{\textnormal{lateinisch: um meine Seele zu
                  retten}}}\label{K_L02648-3h}. Im Übrigen haben wir, wie geſagt, viel von Dir geſprochen, direct
               und indirect, und ich habe es als meine Aufgabe betrachtet, die \textcolor{blue}{Frau}{}\ledrightnote{→\textcolor{blue}{Olga Waissnix}} in der Liebe zu Dir zu bestärken, um ſo
               mehr, als ich diese Liebe auch – trotz Allem und Allen – als ein großes Glück für
               Dich erkannt habe. Ich habe natürlich die größte Vorſicht vorgewendet, und ich glaube
               nicht, daß Frau \textsc{\textcolor{blue}{Olga}{}\ledrightnote{\textcolor{blue}{Olga Waissnix}}} eine Ahnung hat, daß ich Mit{\pb}wiſſer bin. In dieſem Punkte kannſt Du alſo vollauf beruhigt ſein. Im Übrigen hat
               ſie mir außerordentlich viel auch von den \label{K_L02648-5v}\edtext{\textsc{\textcolor{blue}{Pick}{}\ledrightnote{→\textcolor{blue}{Rudolf Pick}{\newline}→\textcolor{blue}{Gustav Pick}{\newline}→\textcolor{blue}{Alfred Pick}}}’s}{\lemma{\textnormal{\emph{Pick’s}}}\Cendnote{\textnormal{\textcolor{blue}{Schnitzler}s Verwandte \textcolor{blue}{Gustav Pick} und dessen Söhne \textcolor{blue}{Rudolf} und \textcolor{blue}{Alfred}.}}}\label{K_L02648-5h} erzählt, offenbar, damit ich es wiedererzähle, was ich mich
               hiermit thue. Ich ſelbſt bin größtentheils von einer neuen mentalen Blindheit
               geweſen. Und ich werde ſie stark enttäuſcht haben. Wenn Du mir einen großen
               Freundesdienſt thun willſt – ich bitte Dich recht ſehr darum – ſo ſchreib’ {\pb}mir, \label{K_L02648-8v}\edtext{was ſie Dir über mich}{\lemma{\textnormal{\emph{was ſie Dir über mich}}}\Cendnote{\textnormal{Sie schrieb \textcolor{blue}{Schnitzler}:
                     »Dr. \textcolor{blue}{Goldmann} ist schon
                     abgereist, er schrieb mir aus \textcolor{pink}{Pörtschach}.
                     Wir haben in den 2 Tagen viel mit einander geplaudert, vieles auch über Sie.
                     Ausgefragt hab’ ich ihn nicht, erstens weil es mir zu gemein schien u. zweitens
                     weil ich ja doch weiß, er sagt mir nichts. Übrigens, ich bin sage comme une
                     image u. will gar nichts wissen.« (\textcolor{blue}{Arthur Schnitzler}, \textcolor{blue}{Olga Waissnix}: \emph{Liebe, die starb vor der
                        Zeit. Ein Briefwechsel}. Mit einem Vorwort von Hans Weigel. Hg. von
                     Therese Nickl und Heinrich Schnitzler. Wien, München, Zürich: \emph{Fritz
                        Molden}{ }1970, S. 216.) }}}\label{K_L02648-8h} geſchrieben hat. Verliebt habe ich
               mich \uuline{\edtext{nicht}{\Cendnote{vierfach unterstrichen}}}; ſinnlich läßt mich die \textcolor{blue}{Frau}{}\ledrightnote{→\textcolor{blue}{Olga Waissnix}} kalt.\pend
           \pstart
           Thatſächliches von meinem Aufenthalte iſt, daß ich bei meiner Ankunft ein Zimmer
               reſervirt fand (das vom vorigem Jahr); daß \textcolor{blue}{\uline{er}}{}\ledrightnote{→\textcolor{blue}{Carl Waissnix}} um mich herum gegangen \strikeout{hat} iſt, als wollte \textcolor{blue}{er}{}\ledrightnote{→\textcolor{blue}{Carl Waissnix}} mich freſſen, zuletzt aber
               recht zuthunlich und geſprächig geworden; daß ich \textsc{\textcolor{blue}{Herzl}{}\ledrightnote{\textcolor{blue}{Theodor Herzl}}} und \textcolor{blue}{Frau}{}\ledrightnote{→\textcolor{blue}{Julie Herzl}} dort
               geſprochen und meine Antipathie gegen \textcolor{blue}{Beide}{}\ledrightnote{→\textcolor{blue}{Theodor Herzl}{\newline}→\textcolor{blue}{Julie Herzl}} recht grämlich verſtärkt habe; daß ich bei
               meiner Abreiſe, als ich die Zimmerrechnung verlangte, den Beſcheid erhielt: der
               gnädigen \textcolor{blue}{Frau}{}\ledrightnote{→\textcolor{blue}{Olga Waissnix}} war es ein
                  Vergnügen\strikeout{,} – was mir unendlich peinlich war; daß
               ſie mir, in Gegenwart von {\pb}Fremden
               beim Abschied ſagte: »Wenn Sie nach \textcolor{pink}{Wien}{}\ledrightnote{\textcolor{pink}{Wien}} Briefe
               ſenden, ſo ſagen Sie viele Grüße von mir.« Daß \textsc{\label{K_L02648-6v}\edtext{\textcolor{blue}{Rettinger}{}\ledrightnote{\textcolor{blue}{Franz Rettinger}}}{\lemma{\textnormal{\emph{Rettinger}}}\Cendnote{\textnormal{In \emph{\textcolor{green}{Jugend in Wien}} wird er von \textcolor{blue}{Schnitzler} folgendermaßen beschrieben: »Das war der Buchhalter,
                        Geschäftsführer, Vizedirektor des \textcolor{pink}{Thalhof}s; ein kleiner, dicker, beweglicher Mann in den Dreißigern,
                        meist städtisch gekleidet oder mit einem grünen Jagdrock angetan, aber
                        jederzeit ohne Kragen und Halsbinde. Er hatte eine spaßige, geschwinde Art
                        zu reden, war das Faktotum, der Vertraute und mehr oder weniger auch der
                        Spion des \textcolor{blue}{Gatten},
                        was ihn nicht hinderte oder vielleicht erst recht dazu veranlaßte, mit Frau
                           \textcolor{blue}{Olga} auf freundschaftlichem Fuß zu
                        stehen, die ihm keineswegs traute, aber eine gewisse Sympathie für ihn
                        hegte.« (S. 243)}}}\label{K_L02648-6h}} im Herbſt nach \textcolor{pink}{Wien}{}\ledrightnote{\textcolor{pink}{Wien}} kommt.\pend
           \pstart
           Alle Details mündlich.\pend
           \pstart
           Bitte, ſchreib’ mir genau, wie es Dir geht! Adreſſe: \textsc{\textcolor{pink}{Pörtschach}{}\ledrightnote{\textcolor{pink}{Pörtschach}}, Poste restante}.\pend
           \pstart
           Viele Grüße! {\\[\baselineskip]}Dein {\\[\baselineskip]}\spacefill\mbox{Paul Goldmann}\pend
           \leftskip=0em{}\pstart
           \noindent{}\label{K_L02648-111v}\edtext{Strombad}{\lemma{\textnormal{\emph{Strombad}}}\Cendnote{\textnormal{\textcolor{pink}{Wien} verfügte über
                     mehrere Badeschiffe, die sowohl am Ufer des \textcolor{pink}{Donaukanal}s wie der \textcolor{pink}{Donau} vor Anker
                     lagen. Geschwommen wurde nicht direkt im Fluss, sondern in Becken innerhalb des
                     Schiffes, die vom Fluss gespeist wurden.}}}\label{K_L02648-111h}?? Biſt Du viel mit \textsc{\textcolor{blue}{Hirschfeld}{}\ledrightnote{\textcolor{blue}{Robert Hirschfeld}}} zuſammen? Grüße an \textsc{\textcolor{blue}{Kapper}{}\ledrightnote{\textcolor{blue}{Friedrich Kapper}}}!\pend
           \endnumbering\briefempfaengerindex{Schnitzler, Arthur@\textsc{Schnitzler, Arthur}!zzzGoldmann, Paul@\emph{von Paul Goldmann}!1890-08-111@{11. 8. 1890}|)be}\mylabel{h}  \normalsize

\doendnotes{C}
\bigskip
\vfill

\clearpage

\footnotesize

\lohead{\textsc{register}}

% Definiere theindex-Environment komplett neu ohne reledmac
\makeatletter
\renewenvironment{theindex}{%
  \section*{\indexname}%
  \setlength{\parindent}{0pt}%
  \setlength{\parskip}{0pt plus 0.3pt}%
  \let\item\@idxitem
}{%
  \clearpage
}
\makeatother

\IfFileExists{\jobname-pw.ind}{\input{\jobname-pw.ind}}{}

\end{document}

      