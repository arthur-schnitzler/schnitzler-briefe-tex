%% latex-korrekturansicht-vorspann.tex
%% Vorspann für die Korrekturansicht.
%% Lädt die gemeinsame Datei latex-vorspann.tex mit gesetztem Schalter.

\newif\ifkorrekturansicht
\korrekturansichttrue

\input{../tex-inputs/latex-vorspann}


               \section[Arthur Schnitzler an Richard Beer-Hofmann, 14. 9. 1896]{ Arthur Schnitzler an Richard Beer-Hofmann, 14. 9. 1896}\nopagebreak\mylabel{v}\rehead{ }\normalsize\beginnumbering\briefempfaengerindex{Beer-Hofmann, Richard@\textsc{Beer-Hofmann, Richard}!zzzSchnitzler, Arthur@\emph{von Arthur Schnitzler}!1896-09-142@{14. 9. 1896}|(be} \toendnotes[C]{\smallbreak\pagebreak[2]} \Standort{YCGL, MSS 31.}
\physDesc{Brief, 1 Blatt, 1 Seite, Umschlag
\newline{}Handschrift: Bleistift, deutsche Kurrent\newline{}Versand: 1) Stempel: »\nobreak{}\oindex{I., Innere Stadt@\textbf{I., Innere Stadt}, \emph{Bezirk (A.BZK)}|pwk}Wien 1/1, 14. 9. 96, 9–10 N\nobreak{}«.  2) Stempel: »\nobreak{}\oindex{Baden bei Wien@\textbf{Baden bei Wien}, \emph{Besiedelter Ort (A.BSO)}|pwk}Baden 1, 15. 9. 96, 7–10 V, Bestellt\nobreak{}«. }\buchAbdrucke{\weitereDrucke{Arthur Schnitzler, Richard Beer-Hofmann: \emph{Briefwechsel 1891–1931}. Hg. Konstanze Fliedl. Wien, Zürich: \emph{Europaverlag} 1992, S. 96–97.} }\toendnotes[C]{\smallbreak}\pstart{}{\pb}Herrn \textsc{Dr. Rich.
                     Beer-Hofmann}\pend{}\pstart{}\textsc{\textcolor{pink}{Baden bei Wien}{}\ledrightnote{\textcolor{pink}{Baden bei Wien}}}\pend{}\pstart{}\textsc{\textcolor{pink}{Franzensgassse 54}{}\ledrightnote{\textcolor{pink}{Kaiser-Franz-Ring}}}, Thür 8.\pend{}{\bigskip}\pstart
           \raggedleft{}{\pb}14. 9. 96.\pend
           \pstart
           Das hab ich gewußt, mein lieber Richard! Ich habe ſogar ſcherzhaft
                  \introOben{}(\introOben{}in der beſti{\geminationm}ten Hoffnung,
               Sie ſchauen durch die Fensterritzen\substVorne{}\textsuperscript{, {\dots}}\substDazwischen{})\substHinten{} nach Ihrem unglaublich verſchloſſnen Fenſter hin gedroht und ernſthaft
               gelächelt. Zeuge: {\pb}der bereits geſtern erwähnte Doctor
                  \textcolor{blue}{Schwarzkopf}{}\ledrightnote{\textcolor{blue}{Gustav Schwarzkopf}}. – Aber was hätte mein Klopfen
               genützt? Ich hoffe, Sie wären nicht in der Lage geweſen, mir zu öffnen.\pend
           \pstart
           Ich komme wohl noch einmal vorm 24. nach \textcolor{pink}{Baden}{}\ledrightnote{\textcolor{pink}{Baden bei Wien}}, {\pb}aber da telegrafir ich vorher (ohne
               Bindung für Sie.)\pend
           \pstart Herzlich Ihr \spacefill\mbox{Arthur}\pend{}\pstart
           \noindent{}Sehr decorativ wirkte geſtern in Ihrem kleinen Garten die Zuſa{\geminationm}enſtellung: dicke Dame, Ihr \textcolor{blue}{Diener}{}\ledrightnote{→\textcolor{blue}{?? [Dienstbote]}} mit Ihrem Strohhut und \label{K_L00590_1v}\edtext{\textsc{Flirt}}{\lemma{\textnormal{\emph{Flirt}}}\Cendnote{\textnormal{\textcolor{blue}{Beer-Hofmann}s Hund}}}\label{K_L00590_1h}. – \pend
           \endnumbering\briefempfaengerindex{Beer-Hofmann, Richard@\textsc{Beer-Hofmann, Richard}!zzzSchnitzler, Arthur@\emph{von Arthur Schnitzler}!1896-09-142@{14. 9. 1896}|)be}\mylabel{h}  \normalsize

\doendnotes{C}
\bigskip
\vfill

\clearpage

\footnotesize

\lohead{\textsc{register}}

% Definiere theindex-Environment komplett neu ohne reledmac
\makeatletter
\renewenvironment{theindex}{%
  \section*{\indexname}%
  \setlength{\parindent}{0pt}%
  \setlength{\parskip}{0pt plus 0.3pt}%
  \let\item\@idxitem
}{%
  \clearpage
}
\makeatother

\IfFileExists{\jobname-pw.ind}{\input{\jobname-pw.ind}}{}

\end{document}

      