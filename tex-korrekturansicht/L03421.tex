%% latex-korrekturansicht-vorspann.tex
%% Vorspann für die Korrekturansicht.
%% Lädt die gemeinsame Datei latex-vorspann.tex mit gesetztem Schalter.

\newif\ifkorrekturansicht
\korrekturansichttrue

\input{../tex-inputs/latex-vorspann}


\renewcommand{\erwaehntePersonen}{Personen: Otto Brahm, Alphonse Daudet, Julius Elias, Paul Goldmann, Adalbert von Goldschmidt, Siegfried Jacobsohn, Emanuel Reicher, Rudolf Rittner, Ottilie Salten, Olga Schnitzler}
\renewcommand{\erwaehnteOrte}{Orte: Berlin, Lessing-Theater, Wien}
\renewcommand{\erwaehnteWerke}{Werke: Der Ruf des Lebens. Schauspiel in drei Akten, Der einsame Weg, Der einsame Weg. Schauspiel in fünf Akten, Die Schaubühne, Elga}
\section[ Felix Salten an Arthur Schnitzler, 22.–23. 4. 1906]{Felix Salten an Arthur Schnitzler, 22.–23. 4. 1906}
\nopagebreak\mylabel{v}
\rehead{ }\normalsize\beginnumbering\briefempfaengerindex{Schnitzler, Arthur@\textsc{Schnitzler, Arthur}!zzzSalten, Felix@\emph{von Felix Salten}!1@{22.–23. 4. 1906}|(be}
\toendnotes[C]{\smallbreak\pagebreak[2]}\Standort{CUL, Schnitzler, B 89, B 1.}
\physDesc{Brief, 1 Blatt, 2 Seiten, 3327 Zeichen
\newline{}Handschrift: schwarze Tinte, lateinische Kurrent
\newline{}Ordnung: mit Bleistift von unbekannter Hand nummeriert: »211« }\toendnotes[C]{\smallbreak}
\pstart
           \raggedleft{}{\pb}\textcolor{pink}{Berlin}{}\ledrightnote{\textcolor{pink}{Berlin}}, 22. IV. 06\pend
           
\pstart
           Lieber, eben, da ich mich hinsetzen will, um Ihnen zu schreiben,
               kommt Ihre zweite Depesche. Ich bin nun einigermaßen in Verlegenheit. Denn wie leicht
               kann \textcolor{blue}{Brahm}{}\ledrightnote{\textcolor{blue}{Otto Brahm}} meinen unverlangten \label{K_L03421-1v}\edtext{Rath}{\lemma{\textnormal{\emph{Rath}}}\Cendnote{\textnormal{siehe Felix Salten an Arthur Schnitzler, 21. 4. [1906]}}}\label{K_L03421-1h} ablehnen; kann ihn, was mir noch weniger lieb wäre, missdeuten, und als die
               Sucht, »dreinzureden« auffassen. Ganz abgesehen davon, dass ich ja garnicht weiss, ob
                  \textcolor{blue}{Brahm}{}\ledrightnote{\textcolor{blue}{Otto Brahm}} auf mein Urteil auch nur das Mindeste
               gibt. Und ausserdem habe ich, als wir \label{K_L03421-2v}\edtext{nach der \textcolor{green}{Vorstellung}{}\ledrightnote{\textcolor{green}{Der einsame Weg. Schauspiel in fünf Akten}}}{\lemma{\textnormal{\emph{nach der Vorstellung}}}\Cendnote{\textnormal{siehe Felix Salten u. a. an Arthur Schnitzler, 19. 4. 1906}}}\label{K_L03421-2h} beisammen waren, zu merken geglaubt, dass \textcolor{blue}{Brahm}{}\ledrightnote{\textcolor{blue}{Otto Brahm}} (vielleicht aus Theaterpolitik) \textcolor{blue}{Reicher}{}\ledrightnote{\textcolor{blue}{Emanuel Reicher}}s \textcolor{green}{Julian}{}\ledrightnote{{$\rightarrow$}\textcolor{green}{Der einsame Weg. Schauspiel in fünf Akten}} über
               den von \textcolor{blue}{Rittner}{}\ledrightnote{\textcolor{blue}{Rudolf Rittner}} zu stellen geneigt ist. Ich
               kann mich ja darin irren. Jedenfalls erleichtert es die Situation nicht, denn ich
               habe \textcolor{blue}{Rittner}{}\ledrightnote{\textcolor{blue}{Rudolf Rittner}} in dieser Rolle nicht gesehen.
               Wie immer er aber auch gewesen sein mag, er war sicherlich besser als \textcolor{blue}{Reicher}{}\ledrightnote{\textcolor{blue}{Emanuel Reicher}}. Einfach aus dem Grund, weil es
               unmöglich ist, schlechter zu sein als Herr \textcolor{blue}{Reicher}{}\ledrightnote{\textcolor{blue}{Emanuel Reicher}} war. (Dieser Satz könnte von \textcolor{blue}{Goldmann}{}\ledrightnote{\textcolor{blue}{Paul Goldmann}} sein; ist aber gleichwol richtig) Um Rittner ist doch stets ein
               Hauch von der Fülle der Erlebnisse. Auch ein leiser Hauch von Einsamkeit ist jetzt
               mehr und mehr um ihn. \textcolor{blue}{Rittner}{}\ledrightnote{\textcolor{blue}{Rudolf Rittner}} ist doch auf
               eine glaubhafte Art von Verliebtheit umgeben, von allerlei Karessen, und das Parfum
               vieler Frauen haftet gleichsam in seinen Kleidern. Wenn nun alle diese Dinge welk und
               herbstlich werden, dann haben Sie, wie es der \textcolor{green}{Julian}{}\ledrightnote{{$\rightarrow$}\textcolor{green}{Der einsame Weg. Schauspiel in fünf Akten}} braucht{[},{]} jene Melancholie,
               deren besondere Schattierung eben ein Goldton ist, ein verblaßender, vormals aber –
               das sieht man noch genau – üppiger und leuchtender Goldton. Von solchen {\pb}Dingen ist bei \textcolor{blue}{Reicher}{}\ledrightnote{\textcolor{blue}{Emanuel Reicher}} nichts zu spüren. Er ist ganz und gar bürgerlich. Hat
               leider den Moment versäumt, Kinder zu zeugen, mit denen er jetzt \label{K_L03421-3v}\edtext{Schabbes}{\lemma{\textnormal{\emph{Schabbes}}}\Cendnote{\textnormal{Sabbat}}}\label{K_L03421-3h} machen oder den \label{K_L03421-4v}\edtext{Seder-Abend}{\lemma{\textnormal{\emph{Seder-Abend}}}\Cendnote{\textnormal{Abendessen am Vorabend des Pessach-Festes}}}\label{K_L03421-4h} halten könnte. Mir wäre, wie ich
               gewiss nicht erst zu sagen brauche, auch der jüdische \textcolor{green}{Julian}{}\ledrightnote{{$\rightarrow$}\textcolor{green}{Der einsame Weg. Schauspiel in fünf Akten}} recht, wenn es nur eben ein \textcolor{green}{Julian}{}\ledrightnote{{$\rightarrow$}\textcolor{green}{Der einsame Weg. Schauspiel in fünf Akten}} wäre: etwa \textcolor{blue}{Adalbert Goldschmidt}{}\ledrightnote{\textcolor{blue}{Adalbert von Goldschmidt}}, der ja den jüdischen und zugleich einen
                  \textcolor{blue}{Daudet}{}\ledrightnote{\textcolor{blue}{Alphonse Daudet}}’schen Einschlag hat. Allein \textcolor{blue}{Reicher}{}\ledrightnote{\textcolor{blue}{Emanuel Reicher}} ist trocken, und erscheint höchstens
               als verkrachter Familienvater. – – –\pend
           
\pstart
           Montag.\pend
           
\pstart
           Gestern wurde ich durch Besuche (die Leute machen hier
               unaufhörlich Besuche) unterbrochen. Abends taf ich zufällig \textcolor{blue}{Rittner}{}\ledrightnote{\textcolor{blue}{Rudolf Rittner}}. Er ist nicht abgeneigt, den \textcolor{green}{Julian}{}\ledrightnote{{$\rightarrow$}\textcolor{green}{Der einsame Weg. Schauspiel in fünf Akten}} in \textcolor{pink}{Wien}{}\ledrightnote{\textcolor{pink}{Wien}} zu spielen. Oder genauer: »im Prinzip nicht
                  dagegen{[}«{]}. Als ich ihm sagte, \uline{Sie} hätten keineswegs darauf bestanden, dass er den \label{K_L03421-5v}\edtext{\textcolor{green}{Forstadjunkten}{}\ledrightnote{{$\rightarrow$}\textcolor{green}{Der Ruf des Lebens. Schauspiel in drei Akten}}}{\lemma{\textnormal{\emph{Forstadjunkten}}}\Cendnote{\textnormal{Bei der deutschsprachigen Uraufführung
                  von \emph{\textcolor{green}{Der Ruf des Lebens}} am 24. 2. 1906 im \textcolor{pink}{Lessing-Theater} gab \textcolor{blue}{Rittner} den Forstadjunkten \textcolor{green}{Eduard Rainer}.}}}\label{K_L03421-5h} gibt, und hätten
               ihm sein Versagen auch nicht übelgenommen, war er erfreut. Er meint nur, es wird für
                  \textcolor{blue}{Brahm}{}\ledrightnote{\textcolor{blue}{Otto Brahm}} schwer sein, \textcolor{blue}{Reicher}{}\ledrightnote{\textcolor{blue}{Emanuel Reicher}} die Rolle abzunehmen, und die für \textcolor{blue}{Rittner}{}\ledrightnote{\textcolor{blue}{Rudolf Rittner}} nötigen Proben abzuhalten. Außerdem wird \textcolor{blue}{Brahm}{}\ledrightnote{\textcolor{blue}{Otto Brahm}} es nicht gerne sehen, wenn \textcolor{blue}{Rittner}{}\ledrightnote{\textcolor{blue}{Rudolf Rittner}} über seine Garantie kommt. Die betragt
               für \textcolor{pink}{Wien}{}\ledrightnote{\textcolor{pink}{Wien}} 12 Abende, welche mit »\textcolor{green}{Elga}{}\ledrightnote{\textcolor{green}{Elga}}« gedeckt scheinen. Ist er im »\textcolor{green}{Einsamen Weg}{}\ledrightnote{\textcolor{green}{Der einsame Weg. Schauspiel in fünf Akten}}« tätig, muß dann \textcolor{blue}{Brahm}{}\ledrightnote{\textcolor{blue}{Otto Brahm}} das Plus zahlen, was er – wie Sie wissen – überhaupt,
               und im Fall \textcolor{blue}{Rittner}{}\ledrightnote{\textcolor{blue}{Rudolf Rittner}} erst recht lieber
               vermeidet.\pend
           
\pstart
           Was soll ich, nach Ihrer Meinung, tun? Dass ich mit Vergnügen zu allem bereit bin,
               brauche ich nicht erst zu sagen. Erwägen Sie, was ich Ihnen wegen mir u. \textcolor{blue}{Brahm}{}\ledrightnote{\textcolor{blue}{Otto Brahm}} sagte, und denken Sie nach, wie man es
               machen könnte, dass ich bei \textcolor{blue}{Brahm}{}\ledrightnote{\textcolor{blue}{Otto Brahm}} nicht eine
               Unannehmlichkeit erfahre. Soll ich vielleicht \textcolor{blue}{Elias}{}\ledrightnote{\textcolor{blue}{Julius Elias}} zu ihm schicken? Das will ich auf alle Fälle gleich tun.\pend
           
\pstart
           Eben kommt wieder Besuch. (Die Leute machen hier unaufhörlich Besuche) Ich will aber,
               dass der Brief heute abgeht.\pend
           
\pstart
           Also viele herzlichste Grüße von \textcolor{blue}{uns}{}\ledrightnote{{$\rightarrow$}\textcolor{blue}{Ottilie Salten}} an Sie \textcolor{blue}{Beide}{}\ledrightnote{{$\rightarrow$}\textcolor{blue}{Olga Schnitzler}}. {\\[\baselineskip]}Ihr \spacefill\mbox{Salten}\pend
           \leftskip=0em{}
\pstart
           \noindent{}\textcolor{gray}{NB}. \textcolor{blue}{Jacobsohn}{}\ledrightnote{\textcolor{blue}{Siegfried Jacobsohn}}{ }\label{K_L03421-6v}\edtext{\textcolor{green}{tobt}{}\ledrightnote{{$\rightarrow$}\textcolor{green}{Der einsame Weg}}}{\lemma{\textnormal{\emph{tobt}}}\Cendnote{\textnormal{[\textcolor{blue}{Siegfried Jacobsohn}]: \emph{\textcolor{green}{Der einsame Weg}}. In: \emph{\textcolor{green}{Die Schaubühne}}, Jg. 2, Nr. 17, 26. 4. 1906, S. 487–491.}}}\label{K_L03421-6h} ja auch
                  gegen \textcolor{blue}{Reicher}{}\ledrightnote{\textcolor{blue}{Emanuel Reicher}}!\pend
           \endnumbering\briefempfaengerindex{Schnitzler, Arthur@\textsc{Schnitzler, Arthur}!zzzSalten, Felix@\emph{von Felix Salten}!1@{22.–23. 4. 1906}|)be}\mylabel{h}  \normalsize

\doendnotes{C}
\bigskip
\vfill

\clearpage

\footnotesize

\lohead{\textsc{register}}

% Definiere theindex-Environment komplett neu ohne reledmac
\makeatletter
\renewenvironment{theindex}{%
  \section*{\indexname}%
  \setlength{\parindent}{0pt}%
  \setlength{\parskip}{0pt plus 0.3pt}%
  \let\item\@idxitem
}{%
  \clearpage
}
\makeatother

\IfFileExists{\jobname-pw.ind}{\input{\jobname-pw.ind}}{}

\end{document}

      