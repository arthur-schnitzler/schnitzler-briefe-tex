%% latex-korrekturansicht-vorspann.tex
%% Vorspann für die Korrekturansicht.
%% Lädt die gemeinsame Datei latex-vorspann.tex mit gesetztem Schalter.

\newif\ifkorrekturansicht
\korrekturansichttrue

\input{../tex-inputs/latex-vorspann}


               \section[Arthur Schnitzler an Richard Beer-Hofmann, 9. 10. 1894]{ Arthur Schnitzler an Richard Beer-Hofmann, 9. 10. 1894}\nopagebreak\mylabel{v}\rehead{ }\normalsize\beginnumbering\briefempfaengerindex{Beer-Hofmann, Richard@\textsc{Beer-Hofmann, Richard}!zzzSchnitzler, Arthur@\emph{von Arthur Schnitzler}!1894-10-091@{9. 10. 1894}|(be} \toendnotes[C]{\smallbreak\pagebreak[2]} \Standort{YCGL, MSS 31.}
\physDesc{Postkarte
\newline{}Handschrift: Bleistift, deutsche Kurrent\newline{}Versand: 1) Stempel: »\nobreak{}\oindex{IX., Alsergrund@\textbf{IX., Alsergrund}, \emph{Bezirk (A.BZK)}|pwk}Wien 9/3, 9. 10. 94, 5–6 N\nobreak{}«.  2) Stempel: »\nobreak{}\oindex{Rom@\textbf{Rom}, \emph{Besiedelter Ort (A.BSO)}|pwk}Roma, 11 10 {[}94{]}, 8 M\nobreak{}«. }\buchAbdrucke{\weitereDrucke{Hermann Bahr, Arthur Schnitzler: \emph{Briefwechsel, Aufzeichnungen, Dokumente
                                (1891–1931)}. Hg. Kurt Ifkovits und Martin Anton Müller. Göttingen: \emph{Wallstein} 2018.} }\toendnotes[C]{\smallbreak}\pstart{}{\pb}\textsc{\textcolor{pink}{Italien}{}\ledrightnote{\textcolor{pink}{Italien}}}\pend{}\pstart{}\textsc{Dr. Rich Beer-Hofmann}\pend{}\pstart{}\textsc{\textcolor{pink}{Rom}{}\ledrightnote{\textcolor{pink}{Rom}}}\pend{}\pstart{}\textsc{\textcolor{pink}{Hotel Quirinal}{}\ledrightnote{\textcolor{pink}{Hotel Quirinale}}}\pend{}{\bigskip}\pstart
           \noindent{}{\pb}\textcolor{pink}{\textcolor{gray}{Wien}}{}\ledrightnote{\textcolor{pink}{Wien}}\pend
           \pstart
           \raggedleft{}Dienstag, 9. 10. 94.\pend
           \pstart
           Lieber Richard, bitte theilen Sie mir mit, ob Sie meinen Brief
                        \textcolor{pink}{Rom}{}\ledrightnote{\textcolor{pink}{Rom}}{ }\textsc{a post. ferm} der »Lieber Bekannter« anfing, nicht
                    erhalten haben. Und die 2 Karten nach \textcolor{pink}{Pallanza}{}\ledrightnote{\textcolor{pink}{Pallanza}}? –\pend
           \pstart
           \textcolor{blue}{\textsc{Bahr}}{}\ledrightnote{\textcolor{blue}{Hermann Bahr}}: \textcolor{pink}{Wien, VIII \textsc{Lammgasse} 3}{}\ledrightnote{\textcolor{pink}{Lammgasse}}. Er hat ſich ſehr über Ihr Telegr.
                    gefreut. Erſte \textcolor{brown}{Nu{\geminationm}er}{}\ledrightnote{→\textcolor{brown}{Die Zeit. Wiener Wochenschrift}} wohlgelungen. \textsc{\label{K_L00380_1v}\edtext{\textcolor{green}{\textcolor{blue}{Helferich}{}\ledrightnote{\textcolor{blue}{Emil Heilbut}}}{}\ledrightnote{→\textcolor{green}{»Schöne Frauen«}}}{\lemma{\textnormal{\emph{Helferich}}}\Cendnote{\textnormal{\textcolor{blue}{Hermann Helferich}: \emph{\textcolor{green}{»Schöne Frauen«}}. In: \emph{\textcolor{green}{Die Zeit}}, Bd. 1, Nr. 1,
                                    6. 10. 1894, S. 7–8.}}}\label{K_L00380_1h}} famos; \label{K_L00380_2v}\edtext{\textcolor{blue}{\textsc{Bahr}}{}\ledrightnote{\textcolor{blue}{Hermann Bahr}}’s Sachen}{\lemma{\textnormal{\emph{Bahr’s Sachen}}}\Cendnote{\textnormal{\textcolor{blue}{Bahr} hat, unter verschiedenen Namen und Kürzeln,
                        zwei Aufsätze, zwei Buch- und drei Theaterbesprechungen im ersten
                        Heft.}}}\label{K_L00380_2h}, beſonders \label{K_L00380_3v}\edtext{\textcolor{green}{Burgtheater}{}\ledrightnote{\textcolor{green}{Burgtheater [Das fünfte Jahr]}}}{\lemma{\textnormal{\emph{Burgtheater}}}\Cendnote{\textnormal{\textcolor{blue}{Hermann Bahr}: \emph{\textcolor{green}{Burgtheater}}. In: \emph{\textcolor{green}{Die Zeit}}, Bd. 1, Nr. 1, 6. 10. 1894,
                            S. 9–10.}}}\label{K_L00380_3h} vorzüglich. –\pend
           \pstart
           \textcolor{green}{Schmetterlingsſchlacht}{}\ledrightnote{\textcolor{green}{Die Schmetterlingsschlacht}} noch nicht geſehen,
                    will Freitag gehen. – Schreiben Sie mehr, wann ko{\geminationm}en Sie?\pend
           \pstart Herzlichen Gruſs\hspace*{1.5em}Ihr
                        \spacefill\mbox{Arthur}\pend{}\endnumbering\briefempfaengerindex{Beer-Hofmann, Richard@\textsc{Beer-Hofmann, Richard}!zzzSchnitzler, Arthur@\emph{von Arthur Schnitzler}!1894-10-091@{9. 10. 1894}|)be}\mylabel{h}  \normalsize

\doendnotes{C}
\bigskip
\vfill

\clearpage

\footnotesize

\lohead{\textsc{register}}

% Definiere theindex-Environment komplett neu ohne reledmac
\makeatletter
\renewenvironment{theindex}{%
  \section*{\indexname}%
  \setlength{\parindent}{0pt}%
  \setlength{\parskip}{0pt plus 0.3pt}%
  \let\item\@idxitem
}{%
  \clearpage
}
\makeatother

\IfFileExists{\jobname-pw.ind}{\input{\jobname-pw.ind}}{}

\end{document}

      