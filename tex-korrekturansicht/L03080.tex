%% latex-korrekturansicht-vorspann.tex
%% Vorspann für die Korrekturansicht.
%% Lädt die gemeinsame Datei latex-vorspann.tex mit gesetztem Schalter.

\newif\ifkorrekturansicht
\korrekturansichttrue

\input{../tex-inputs/latex-vorspann}


\renewcommand{\erwaehntePersonen}{Personen: Louis Cima, Olga Schnitzler, Anton Pavlovič Čechov}
\renewcommand{\erwaehnteOrte}{Orte: Bad Ischl, Berlin, Engadin, Hôtel Metropole, Innsbruck, Russland, Sankt Moritz, Sankt Moritz-Bad, Schweiz, Wien}
\renewcommand{\erwaehnteWerke}{Werke: Ein Zweikampf}
\section[ Paul Goldmann an Arthur Schnitzler, 21. 8. {[}1905?{]}]{Paul Goldmann an Arthur Schnitzler, 21. 8. {[}1905?{]}}
\nopagebreak\mylabel{v}
\rehead{ }\normalsize\beginnumbering\briefempfaengerindex{Schnitzler, Arthur@\textsc{Schnitzler, Arthur}!zzzGoldmann, Paul@\emph{von Paul Goldmann}!1905-08-212@{21. 8. {[}1905?{]}}|(be}
\toendnotes[C]{\smallbreak\pagebreak[2]}\Standort{DLA, A:Schnitzler, HS.NZ85.1.3171.}
\physDesc{Brief, 1 Blatt, 4 Seiten
\newline{}Handschrift: schwarze Tinte, deutsche Kurrent
\newline{}Schnitzler: 1) mit Bleistift das Jahr »{[}1{]}901« vermerkt  2) mit rotem Buntstift eine Unterstreichung}\toendnotes[C]{\smallbreak}
\pstart
           \noindent{}{\pb}\textcolor{gray}{\textbf{\textbf{\begin{otherlanguage}{french}\textcolor{pink}{HÔTEL MÉTROPOLE}{}\ledrightnote{\textcolor{pink}{Hôtel Metropole}}{ }\textcolor{pink}{ST. MORITZ}{}\ledrightnote{\textcolor{pink}{Sankt Moritz-Bad}}\end{otherlanguage}}}}\pend
           
\pstart
           \textcolor{gray}{\textbf{\emph{\begin{otherlanguage}{french}Hôtel de 1\textsuperscript{er}
                           Ordre\end{otherlanguage}}}}\pend
           
\pstart
           \textcolor{gray}{\textbf{\begin{otherlanguage}{french}\textcolor{pink}{ENGADINE}{}\ledrightnote{\textcolor{pink}{Engadin}} · \textcolor{pink}{SUISSE}{}\ledrightnote{\textcolor{pink}{Schweiz}}\end{otherlanguage}}}\pend
           
\pstart
           \textcolor{gray}{\textbf{\begin{otherlanguage}{french}NOUVELLEMENT CONSTRUIT AVEC TOUS LES CONFORTS
                           MODERNES\end{otherlanguage}}}\pend
           
\pstart
           \textcolor{gray}{\textbf{\begin{otherlanguage}{french}120 CHAMBRES\end{otherlanguage}}}\pend
           
\pstart
           \textcolor{gray}{\textbf{\begin{otherlanguage}{french}SITUATION SPLENDIDE\end{otherlanguage}}}\pend
           
\pstart
           \textcolor{gray}{\textbf{\begin{otherlanguage}{french}ASCENSEUR ET LUMIÈRE ELECTRIQUE\end{otherlanguage}}}\pend
           
\pstart
           \textcolor{gray}{\textbf{\begin{otherlanguage}{french}RESTAURANT À LA CARTE ET ARRANGEMENTS POUR
                           FAMILLES\end{otherlanguage}}}\pend
           
\pstart
           \textcolor{gray}{\textbf{\textcolor{blue}{LOUIS CIMA}{}\ledrightnote{\textcolor{blue}{Louis Cima}},{ }\begin{otherlanguage}{french}\emph{PROPR.}\end{otherlanguage}}}\pend
           
\pstart
           \raggedleft{}\textcolor{gray}{\textbf{\textcolor{pink}{St. Moritz-Bad}{}\ledrightnote{\textcolor{pink}{Sankt Moritz-Bad}}, \begin{otherlanguage}{french}le\end{otherlanguage}}}{ }\label{K_L03080-1v}\edtext{21. Auguſt}{\lemma{\textnormal{\emph{21. Auguſt}}}\Cendnote{\textnormal{ 2 \textcolor{blue}{Schnitzler}s Datierung des Briefs auf den 21. 8. 1901 ist falsch. Er und \textcolor{blue}{Goldmann} waren zu dieser Zeit im Jahr 1901
                     gemeinsam auf Reisen (vgl. Paul Goldmann und Arthur Schnitzler an Georg Brandes,
               21. 8. 1901). 1905 ist \textcolor{blue}{Goldmann} nachweislich in \textcolor{pink}{Sankt
                        Moritz}, vgl. Paul Goldmann an Arthur Schnitzler, 26. 8. 1905.
                     Davor, am 31. 7. 1905 hatte er \textcolor{blue}{Schnitzler} und dessen \textcolor{blue}{Frau} in \textcolor{pink}{Wien} einen
                     Besuch abgestattet.}}}\label{K_L03080-1h}.\pend
           
\pstart\center{}Mein lieber Freund,\pend
\pstart
           Ich komme erſt heut dazu, Dir und Deiner \textcolor{blue}{Frau}{}\ledrightnote{{$\rightarrow$}\textcolor{blue}{Olga Schnitzler}} für die Freundſchaft zu
               danken, mit der Ihr in \textcolor{pink}{Wien}{}\ledrightnote{\textcolor{pink}{Wien}} mich aufgenommen
               habt.\pend
           
\pstart
           Die erſte Hälfte meines Urlaubs habe ich leider ſehr unzweckmäßig verbracht. Der
               Aufenthalt in \textsc{\textcolor{pink}{Ischl}{}\ledrightnote{\textcolor{pink}{Bad Ischl}}} hat mir gar keine Erholung gewährt, und ich bedaure {\pb}es ſehr, daß ich nicht die Energie gefunden habe, mich
               früher von dort loszureißen, obwohl doch eigentlich nichts mich hielt. Seit vorigem
                  Donnerſtag bin ich hier, und jetzt erſt beginne
               ich, mich zu kräftigen, und zu erfriſchen. Du kennſt ja den \textcolor{pink}{Ort}{}\ledrightnote{{$\rightarrow$}\textcolor{pink}{Sankt Moritz-Bad}} von unſerem \label{K_L03080-2v}\edtext{gemeinſamen Aufenthalt}{\lemma{\textnormal{\emph{gemeinſamen Aufenthalt}}}\Cendnote{\textnormal{A. S.: \emph{Tagebuch}, 21. 8. 1900, vgl. A. S.: \emph{Tagebuch}, 6. 8. 1930}}}\label{K_L03080-2h} her, an den \strikeout{\textcolor{gray}{ich}} mich \strikeout{h\textcolor{gray}{×}\-\textcolor{gray}{×}\-\textcolor{gray}{×}} hier Manches erinnert, aber in ſeiner ganzen Herrlichkeit entfaltet ſich das
                  \textcolor{pink}{Engadin}{}\ledrightnote{\textcolor{pink}{Engadin}} doch erſt bei längerem Aufenthalt.
               Mein Entſchluß iſt gefaßt: Ich werde fortan \uline{jeden}
               Urlaub im \textcolor{pink}{Engadin}{}\ledrightnote{\textcolor{pink}{Engadin}} verbringen. Nirgends wieder
               gibt es eine {\pb}ſolche Luft, das Athmen allein iſt ein
               Vergnügen, und für abgearbeitete Menſchen iſt hier und hier allein die rechte
               Erholung. Obwohl Du ja nicht abgearbeitet biſt, rate ich Dir auch dringend, nächſten
               Sommer hier einen längeren \label{K_L03080-3v}\edtext{Aufenthalt}{\lemma{\textnormal{\emph{Aufenthalt}}}\Cendnote{\textnormal{\textcolor{blue}{Schnitzler} kam erst am vgl. A. S.: \emph{Tagebuch}, 26. 8. 1913 wieder nach \textcolor{pink}{Sankt Moritz}.}}}\label{K_L03080-3h} zu nehmen. Da die Bahn
               jetzt bis \textsc{\textcolor{pink}{St. Moritz}{}\ledrightnote{\textcolor{pink}{Sankt Moritz}}} fährt, kommt man bequem hin (von \textcolor{pink}{Innsbruck}{}\ledrightnote{\textcolor{pink}{Innsbruck}}
               in 10 Stunden).\pend
           
\pstart
           Das \label{K_L03080-4v}\edtext{\textcolor{green}{Buch}{}\ledrightnote{{$\rightarrow$}\textcolor{green}{Ein Zweikampf}}}{\lemma{\textnormal{\emph{Buch}}}\Cendnote{\textnormal{Es dürfte sich um die Novelle \emph{\textcolor{green}{Ein Zweikampf}} (zumeist übersetzt als \emph{\textcolor{green}{Das Duell}}) handeln, dessen Lektüre durch \textcolor{blue}{Schnitzler} für den 7. 10. 1904 belegt
                  ist. Vgl. A. S.: \emph{»Das Zeitlose ist von kürzester Dauer«}, Tschechow, 18. 1. 1910M144. }}}\label{K_L03080-4h} von
                  \textsc{\textcolor{blue}{Tschechow}{}\ledrightnote{\textcolor{blue}{Anton Pavlovič Čechov}}} hat mich nicht begeiſtert. Es enthält manches Feine, im Übrigen habe ich es vor
               allen Dingen quälend gefunden, und Quälen iſt nicht Dichten. {\pb}Meine Anſicht, daß \textsc{\textcolor{blue}{Tschechow}{}\ledrightnote{\textcolor{blue}{Anton Pavlovič Čechov}}} ein feines Talent iſt, aber zu den bedeutenden und eigenartigen
               Perſönlichkeiten der \textcolor{pink}{ruſſ}{}\ledrightnote{{$\rightarrow$}\textcolor{pink}{Russland}}iſchen Literatur \uline{nicht} gehört, hat durch
               dieſes \textcolor{green}{Buch}{}\ledrightnote{{$\rightarrow$}\textcolor{green}{Ein Zweikampf}} eine
               Beſtärkung erfahren.\pend
           
\pstart
           Auf der Rückreiſe komme ich nicht über \textcolor{pink}{Wien}{}\ledrightnote{\textcolor{pink}{Wien}}, ich
               hoffe aber, Dich im \label{K_L03080-7v}\edtext{Winter in \textcolor{pink}{Berlin}{}\ledrightnote{\textcolor{pink}{Berlin}}}{\lemma{\textnormal{\emph{Winter in Berlin}}}\Cendnote{\textnormal{\textcolor{blue}{Schnitzler} und \textcolor{blue}{Goldmann} trafen sich jedenfalls am 21. 11. 1905 und 23. 11. 1905 in \textcolor{pink}{Berlin}.}}}\label{K_L03080-7h} wiederzuſehen.\pend
           
\pstart
           Mit vielen herzlichen Grüßen an Deine \textcolor{blue}{Frau}{}\ledrightnote{{$\rightarrow$}\textcolor{blue}{Olga Schnitzler}} und Dich bin ich {\\[\baselineskip]}Dein getreuer {\\[\baselineskip]}\spacefill\mbox{Paul Goldmann.}\pend
           \leftskip=0em{}\endnumbering\briefempfaengerindex{Schnitzler, Arthur@\textsc{Schnitzler, Arthur}!zzzGoldmann, Paul@\emph{von Paul Goldmann}!1905-08-212@{21. 8. {[}1905?{]}}|)be}\mylabel{h}
\begin{anhang}
\end{anhang}\normalsize

\doendnotes{C}
\bigskip
\vfill

\clearpage

\footnotesize

\lohead{\textsc{register}}

% Definiere theindex-Environment komplett neu ohne reledmac
\makeatletter
\renewenvironment{theindex}{%
  \section*{\indexname}%
  \setlength{\parindent}{0pt}%
  \setlength{\parskip}{0pt plus 0.3pt}%
  \let\item\@idxitem
}{%
  \clearpage
}
\makeatother

\IfFileExists{\jobname-pw.ind}{\input{\jobname-pw.ind}}{}

\end{document}

      