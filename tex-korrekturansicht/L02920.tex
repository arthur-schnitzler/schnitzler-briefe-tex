%% latex-korrekturansicht-vorspann.tex
%% Vorspann für die Korrekturansicht.
%% Lädt die gemeinsame Datei latex-vorspann.tex mit gesetztem Schalter.

\newif\ifkorrekturansicht
\korrekturansichttrue

\input{../tex-inputs/latex-vorspann}


         
         \renewcommand{\erwaehntePersonen}{Personen: Richard Beer-Hofmann, Alfred Kerr, Leopoldine Müller, Olga Schnitzler, Leo Van-Jung}
         \renewcommand{\erwaehnteOrte}{Orte: Alpen, Berlin, Dessauer Straße, Dolomiten, Innsbruck, Pontresina, Schruns, Schweiz, Tirol, Trafoi, Vorarlberg, Wien}
         \renewcommand{\erwaehnteWerke}{}
               \section[ Paul Goldmann an Arthur Schnitzler, 16. 6. {[}1900{]}]{Paul Goldmann an Arthur Schnitzler, 16. 6. {[}1900{]}}\nopagebreak\mylabel{v}\rehead{ }\normalsize\beginnumbering\briefempfaengerindex{Schnitzler, Arthur@\textsc{Schnitzler, Arthur}!zzzGoldmann, Paul@\emph{von Paul Goldmann}!1900-06-161@{16. 6. {[}1900{]}}|(be} \toendnotes[C]{\smallbreak\pagebreak[2]} \Standort{DLA, A:Schnitzler, HS.NZ85.1.3170.}
\physDesc{Brief, 1 Blatt, 4 Seiten
\newline{}Handschrift: blaue Tinte, deutsche Kurrent
\newline{}Schnitzler: 1) mit Bleistift das Jahr »1900« vermerkt  2) mit rotem Buntstift eine Unterstreichung}\toendnotes[C]{\smallbreak}\pstart
           \noindent{}{\pb}\textcolor{pink}{\textcolor{gray}{\textbf{DESSAUERSTRASSE 19}}}{}\ledrightnote{\textcolor{pink}{Dessauer Straße}}\pend
           \pstart
           \raggedleft{}\textcolor{pink}{Berlin}{}\ledrightnote{\textcolor{pink}{Berlin}}, 16. Juni.\pend
           \pstart\center{}Mein lieber Freund,\pend\pstart
           Ich danke Dir für die ausführliche Beantwortung meiner Briefe und freue mich ſehr,
               wieder einmal Direktes von Dir gehört zu haben. Nächſte Woche werde ich
               vorausſichtlich viel zu thun haben. Ich antworte Dir daher gleich, und zwar nur wegen
               der Sommerpläne.\pend
           \pstart
           Du ſcheinſt einen überwiegend \label{K_L02920-1v}\edtext{weiblichen Sommer}{\lemma{\textnormal{\emph{weiblichen Sommer}}}\Cendnote{\textnormal{Bezug auf \textcolor{blue}{Olga Gussmann} (später \textcolor{blue}{Schnitzler}) und \textcolor{blue}{Leopoldine Müller}}}}\label{K_L02920-1h} verleben zu wollen.\pend
           \pstart
           Ich ſinne noch über ein Mittel zur Löſung der Finanz-Schwierigkeiten {\pb}nach, die einer Urlaubsreiſe diesmal bei mir \strikeout{\textcolor{gray}{i}} entgegenſtehen. Habe ich es gefunden und bekomme ich Urlaub – zwei noch ſehr
               fragliche Dinge – ſo möchte ich \strikeout{End\textcolor{gray}{e}} Anfang Auguſt eine Fußwanderung in den \textcolor{pink}{Alpen}{}\ledrightnote{\textcolor{pink}{Alpen}}, in \textcolor{pink}{Tirol}{}\ledrightnote{\textcolor{pink}{Tirol}} womöglich, machen. Erſtens weil es ſchön iſt, zweitens aus
               Abmagerungs-Gründen. Denn ich werde dick wie ein Schwein. Ich frage Dich alſo:\pend
           \pstart
           1.) Möchteſt Du bei ſo etwas \label{K_L02920-2v}\edtext{mitmachen}{\lemma{\textnormal{\emph{mitmachen}}}\Cendnote{\textnormal{Tatsächlich unternahmen
                     \textcolor{blue}{Schnitzler} und \textcolor{blue}{Goldmann} in der zweiten Augusthälte 1900 gemeinsam mit \textcolor{blue}{Richard Beer-Hofmann}, \textcolor{blue}{Alfred Kerr} und \textcolor{blue}{Leo Van-Jung} eine
                     \textcolor{pink}{Alpen}wanderung. Am 16. 8. 1900 in \textcolor{pink}{Innsbruck} beginnend, kamen sie über \textcolor{pink}{Vorarlberg} und die \textcolor{pink}{Schweiz} am 27. 8. 1900 in \textcolor{pink}{Trafoi} an. \textcolor{blue}{Van-Jung} war nur bis
                     \textcolor{pink}{Schruns}, \textcolor{blue}{Beer-Hofmann} und \textcolor{blue}{Kerr} waren bis \textcolor{pink}{Pontresina} mit dabei. \textcolor{blue}{Beer-Hofmann} dokumentierte die Wanderung in
                  einer Fotoserie (vgl. Heinrich Schnitzler, Christian Brandstätter und
                     Reinhard Urbach (Hg.): \emph{Arthur Schnitzler. Sein Leben. Sein
                        Werk. Seine Zeit}.  Mit 324 Abbildungen. Frankfurt
                     am Main: \emph{S. Fischer}{ }1981, S. 79).}}}\label{K_L02920-2h}?\pend
           \pstart
           2.) Was könnte man unternehmen? \textcolor{pink}{Dolomiten}{}\ledrightnote{\textcolor{pink}{Dolomiten}}?\pend
           \pstart
           3.) Würden \textsc{\textcolor{blue}{Richard}{}\ledrightnote{\textcolor{blue}{Richard Beer-Hofmann}}} und {\pb}\textsc{\textcolor{blue}{Leo}{}\ledrightnote{\textcolor{blue}{Leo Van-Jung}}} mitkommen?\pend
           \pstart
           4.) Was macht \textsc{\textcolor{blue}{Richard}{}\ledrightnote{\textcolor{blue}{Richard Beer-Hofmann}}} überhaupt in dieſem Sommer?\pend
           \pstart
           5.) Wäre es Dir recht, wenn \textsc{\textcolor{blue}{Kerr}{}\ledrightnote{\textcolor{blue}{Alfred Kerr}}} mitkäme? Ich habe ihm von der Idee geſprochen und ihn zum Mitkommen animirt. Er
               thäte es ſehr gern, iſt aber Dir gegenüber etwas ſchüchtern und erwartet, daß Du ihn
               dazu aufforderſt. Bitte, \label{K_L02920-3v}\edtext{ſchreib’
                  ihm}{\lemma{\textnormal{\emph{ſchreib’
                  ihm}}}\Cendnote{\textnormal{vgl. \textcolor{blue}{Schnitzler}s Brief an \textcolor{blue}{Alfred Kerr} vom 21. 6. 1900:
                     Elgin Helmstaedt (Hg.): \emph{Alfred Kerr, Arthur Schnitzler.
                        »Es ist eine sehr seltsame Gefühlsmischung, die Sie erwecken«. Briefwechsel
                        1896–1925}. In: \emph{Sinn und Form}, Jg. 69, H. 5,  September/Oktober 2017, S. 581–618, hier: S. 602.}}}\label{K_L02920-3h}
               jedenfalls, daß ich Dir ſeine eventuelle Bereitwilligkeit mitgetheilt habe, und \strikeout{f\textcolor{gray}{ordere} ih\textcolor{gray}{n}}{ }\strikeout{\textcolor{gray}{a}uf} ſage ihm etwas Freundliches {\pb}darüber. Selbſt wenn ich nicht mitkäme, könnteſt Du
               ja mit \textcolor{blue}{ihm}{}\ledrightnote{{$\rightarrow$}\textcolor{blue}{Alfred Kerr}} immer etwas
               verabreden und hätteſt dann einen ſehr angenehmen Reiſebegleiter für die
               nicht-weiblichen Tage.\pend
           \pstart
           Kann ich die Reiſe aber nicht ermöglichen, ſo werde ich es wenigſtens einzurichten
               ſuchen, daß ich \label{K_L02920-4v}\edtext{Anfang September auf ein paar Tage nach \textcolor{pink}{Wien}{}\ledrightnote{\textcolor{pink}{Wien}}}{\lemma{\textnormal{\emph{Anfang … Wien}}}\Cendnote{\textnormal{\textcolor{blue}{Goldmann} hielt sich jedenfalls von 6. 9. 1900 bis 16. 9. [1900] in \textcolor{pink}{Wien} auf.}}}\label{K_L02920-4h} komme.\pend
           \pstart
           Bitte, antworte mir \uline{bald} auf meine Fragen und
               ſchreibe \uline{bald} an \textsc{\textcolor{blue}{Kerr}{}\ledrightnote{\textcolor{blue}{Alfred Kerr}}}!\pend
           \pstart
           Viele treue Grüße! {\\[\baselineskip]}Dein {\\[\baselineskip]}\spacefill\mbox{Paul Goldmann.}\pend
           \leftskip=0em{}\endnumbering\briefempfaengerindex{Schnitzler, Arthur@\textsc{Schnitzler, Arthur}!zzzGoldmann, Paul@\emph{von Paul Goldmann}!1900-06-161@{16. 6. {[}1900{]}}|)be}\mylabel{h}\begin{anhang}\end{anhang}\normalsize

\doendnotes{C}
\bigskip
\vfill

\clearpage

\footnotesize

\lohead{\textsc{register}}

% Definiere theindex-Environment komplett neu ohne reledmac
\makeatletter
\renewenvironment{theindex}{%
  \section*{\indexname}%
  \setlength{\parindent}{0pt}%
  \setlength{\parskip}{0pt plus 0.3pt}%
  \let\item\@idxitem
}{%
  \clearpage
}
\makeatother

\IfFileExists{\jobname-pw.ind}{\input{\jobname-pw.ind}}{}

\end{document}

      