%% latex-korrekturansicht-vorspann.tex
%% Vorspann für die Korrekturansicht.
%% Lädt die gemeinsame Datei latex-vorspann.tex mit gesetztem Schalter.

\newif\ifkorrekturansicht
\korrekturansichttrue

\input{../tex-inputs/latex-vorspann}


               \section[ Paul Goldmann an Arthur Schnitzler, {[}4. 5. 1897{]}]{Paul Goldmann an Arthur Schnitzler, {[}4. 5. 1897{]}}\nopagebreak\mylabel{v}\rehead{ }\normalsize\beginnumbering\briefempfaengerindex{Schnitzler, Arthur@\textsc{Schnitzler, Arthur}!zzzGoldmann, Paul@\emph{von Paul Goldmann}!1897-05-041@{{[}4. 5. 1897{]}}|(be} \toendnotes[C]{\smallbreak\pagebreak[2]} \Standort{DLA, A:Schnitzler, HS.NZ85.1.3167.}
\physDesc{Brief, 1 Blatt, 1 Seite
\newline{}Handschrift: schwarze Tinte, deutsche Kurrent
\newline{}Schnitzler: mit Bleistift das Datum »4/ 5 97« vermerkt }\toendnotes[C]{\smallbreak}\pstart\center{}{\pb}Liebſter Freund,\pend\pstart
           Furchtbare \label{K_L02811-1v}\edtext{Brand-Kataſtrophe}{\lemma{\textnormal{\emph{Brand-Kataſtrophe}}}\Cendnote{\textnormal{Am Abend des 4. 5. 1897 gab es
                  einen Brand im \textcolor{pink}{Bazar de la Charité} in der
                     \textcolor{pink}{Rue Jean Goujon}.}}}\label{K_L02811-1h}. Kann nicht
               kommen. Vielleicht biſt Du gegen 11 Uhr im \label{K_L02811-2v}\edtext{\textsc{Café}}{\lemma{\textnormal{\emph{Café}}}\Cendnote{\textnormal{nicht ermittelt}}}\label{K_L02811-2h} an der Ecke der
                  \textsc{\textcolor{pink}{R. Maubeuge}{}\ledrightnote{\textcolor{pink}{rue de Maubeuge}}}. Wenn ich kann, komm ich vorbei.\pend
           \pstart
           Herzlichſt {\\[\baselineskip]}\spacefill\mbox{P. G.}\pend
           \leftskip=0em{}\pstart
           \noindent{}\textsc{D\textsuperscript{r} Schnitzler}\pend
           \pstart
           \textsc{\textcolor{pink}{5. rue Maubeuge}{}\ledrightnote{\textcolor{pink}{rue de Maubeuge}}}\pend
           \pstart
           \textsc{chez{ }\textcolor{blue}{M\textsuperscript{me} Hauser}{}\ledrightnote{\textcolor{blue}{Hauser}}}.\pend
           \endnumbering\briefempfaengerindex{Schnitzler, Arthur@\textsc{Schnitzler, Arthur}!zzzGoldmann, Paul@\emph{von Paul Goldmann}!1897-05-041@{{[}4. 5. 1897{]}}|)be}\mylabel{h}\begin{anhang}\end{anhang}\normalsize

\doendnotes{C}
\bigskip
\vfill

\clearpage

\footnotesize

\lohead{\textsc{register}}

% Definiere theindex-Environment komplett neu ohne reledmac
\makeatletter
\renewenvironment{theindex}{%
  \section*{\indexname}%
  \setlength{\parindent}{0pt}%
  \setlength{\parskip}{0pt plus 0.3pt}%
  \let\item\@idxitem
}{%
  \clearpage
}
\makeatother

\IfFileExists{\jobname-pw.ind}{\input{\jobname-pw.ind}}{}

\end{document}

      