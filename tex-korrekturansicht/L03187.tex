%% latex-korrekturansicht-vorspann.tex
%% Vorspann für die Korrekturansicht.
%% Lädt die gemeinsame Datei latex-vorspann.tex mit gesetztem Schalter.

\newif\ifkorrekturansicht
\korrekturansichttrue

\input{../tex-inputs/latex-vorspann}


\renewcommand{\erwaehntePersonen}{Personen: Georg Hirschfeld}
\renewcommand{\erwaehnteInstitutionen}{Institutionen: Wiener Allgemeine Zeitung}
\renewcommand{\erwaehnteOrte}{Orte: Berlin, Schulerstraße, Universitätsstraße, Wien}
\renewcommand{\erwaehnteWerke}{Werke: Die Mütter. Schauspiel in vier Acten, Freiwild. Schauspiel in 3 Akten}
\section[ Felix Salten an Arthur Schnitzler, 6. 11. 1896]{Felix Salten an Arthur Schnitzler, 6. 11. 1896}
\nopagebreak\mylabel{v}
\rehead{ }\normalsize\beginnumbering\briefempfaengerindex{Schnitzler, Arthur@\textsc{Schnitzler, Arthur}!zzzSalten, Felix@\emph{von Felix Salten}!1896-11-061@{6. 11. 1896}|(be}
\toendnotes[C]{\smallbreak\pagebreak[2]}\Standort{CUL, Schnitzler, B 89, A 1.}
\physDesc{Brief, 1 Blatt, 1 Seite, 280 Zeichen
\newline{}Handschrift: schwarze Tinte, lateinische Kurrent
\newline{}Ordnung: mit Bleistift von unbekannter Hand nummeriert: »81« }\toendnotes[C]{\smallbreak}
\pstart
           \noindent{}{\pb}\textcolor{gray}{\textbf{\textbf{»\textcolor{brown}{Wiener Allgemeine
                        Zeitung}{}\ledrightnote{\textcolor{brown}{Wiener Allgemeine Zeitung}}«}}}\pend
           
\pstart
           \textcolor{gray}{\textbf{Redaction:}}\pend
           
\pstart
           \textcolor{gray}{\textbf{\textbf{\textcolor{pink}{IX/3, Univerſitätsſtraße Nr. 6}{}\ledrightnote{\textcolor{pink}{Universitätsstraße}}}}}\pend
           
\pstart
           \textcolor{gray}{\textbf{Adminiſtration:}}\hfill \textcolor{gray}{\textbf{\textcolor{pink}{Wien}{}\ledrightnote{\textcolor{pink}{Wien}}, am}}{ }6. Nov. \textcolor{gray}{\textbf{189}}6.\pend
           
\pstart
           \textcolor{gray}{\textbf{\textbf{\textcolor{pink}{I. Schulerſtraße Nr. 20}{}\ledrightnote{\textcolor{pink}{Schulerstraße}}. }}}\pend
           
\pstart
           \textcolor{gray}{\textbf{Telegramm-Adreſſe: »Allgemeine, \textcolor{pink}{Wien}{}\ledrightnote{\textcolor{pink}{Wien}}«.}}\pend
           
\pstart
           \textcolor{gray}{\textbf{Telephon der Redaction: Nr. 805 u. 2180.}}\pend
           
\pstart
           \textcolor{gray}{\textbf{\hspace*{1.5em}„\hspace*{1.5em}„\hspace*{1.5em} Adminiſtration: Nr. 1024.}}\pend
           
\pstart
           Lieber Freund, ich hab die neue Adreße \textcolor{blue}{Hirschfeld}{}\ledrightnote{\textcolor{blue}{Georg Hirschfeld}}s verlegt. Sie sind wol so freundl.
               und \label{K_L03187-1v}\edtext{laßen ihm die Zeitungen, die ich
               eben absandte, zugehen}{\lemma{\textnormal{\emph{laßen … zugehen}}}\Cendnote{\textnormal{Diese separat
                  versandte Beilage ist nicht erhalten. Es dürfte sich um \textcolor{pink}{Wien}er Besprechungen von \textcolor{blue}{Georg Hirschfeld}s Stück \emph{\textcolor{green}{Die Mütter}}
                  gehandelt haben, das am 17. 10. 1896 in \textcolor{pink}{Wien} Premiere
                  gehabt hatte.}}}\label{K_L03187-1h}.\hspace*{2em}Die \label{K_L03187-2v}\edtext{\textcolor{pink}{Wien}{}\ledrightnote{\textcolor{pink}{Wien}}er Blätter}{\lemma{\textnormal{\emph{Wiener Blätter}}}\Cendnote{\textnormal{zur Uraufführung von \emph{\textcolor{green}{Freiwild}} am 3. 11. 1896}}}\label{K_L03187-2h} werd ich Ihnen aufheben.\hspace*{2em}Hier haben die Leute sehr stark den Eindruck eines
               grossen Erfolges.\pend
           
\pstart
           Herzlich {\\[\baselineskip]}Ihr {\\[\baselineskip]}\spacefill\mbox{Salten}\pend
           \leftskip=0em{}\endnumbering\briefempfaengerindex{Schnitzler, Arthur@\textsc{Schnitzler, Arthur}!zzzSalten, Felix@\emph{von Felix Salten}!1896-11-061@{6. 11. 1896}|)be}\mylabel{h}  \normalsize

\doendnotes{C}
\bigskip
\vfill

\clearpage

\footnotesize

\lohead{\textsc{register}}

% Definiere theindex-Environment komplett neu ohne reledmac
\makeatletter
\renewenvironment{theindex}{%
  \section*{\indexname}%
  \setlength{\parindent}{0pt}%
  \setlength{\parskip}{0pt plus 0.3pt}%
  \let\item\@idxitem
}{%
  \clearpage
}
\makeatother

\IfFileExists{\jobname-pw.ind}{\input{\jobname-pw.ind}}{}

\end{document}

      