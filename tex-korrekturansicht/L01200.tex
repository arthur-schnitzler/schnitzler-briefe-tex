%% latex-korrekturansicht-vorspann.tex
%% Vorspann für die Korrekturansicht.
%% Lädt die gemeinsame Datei latex-vorspann.tex mit gesetztem Schalter.

\newif\ifkorrekturansicht
\korrekturansichttrue

\input{../tex-inputs/latex-vorspann}


               \section[Hugo von Hofmannsthal an Arthur Schnitzler, {[}5. 2. 1902?{]}]{ Hugo von Hofmannsthal an Arthur Schnitzler, {[}5. 2. 1902?{]}}\nopagebreak\mylabel{v}\rehead{ }\normalsize\beginnumbering\briefempfaengerindex{Schnitzler, Arthur@\textsc{Schnitzler, Arthur}!zzzHofmannsthal, Hugo von@\emph{von Hugo von Hofmannsthal}!1902-02-051@{{[}5. 2. 1902?{]}}|(be} \toendnotes[C]{\smallbreak\pagebreak[2]} \Standort{CUL, Schnitzler, B 43.}
\physDesc{Brief, 1 Blatt, 2 Seiten
\newline{}Handschrift: schwarze Tinte, deutsche Kurrent
\newline{}Schnitzler: mit Bleistift datiert: »\textsc{Anf Feber 902}« \newline{}Ordnung: 1) mit Bleistift von unbekannter Hand nummeriert:
                              »\strikeout{191}« 2) mit Bleistift von unbekannter Hand nummeriert: »184«}\buchAbdrucke{\weitereDrucke{Hugo von Hofmannsthal, Arthur Schnitzler: \emph{Briefwechsel}. Hg. Therese Nickl und Heinrich Schnitzler. Frankfurt am Main: \emph{S. Fischer} 1964, S. 153.} }\toendnotes[C]{\smallbreak}\pstart
           \raggedleft{}{\pb}Mittwoch{ }abends.\pend
           \pstart{}lieber Arthur\pend\pstart
           es wäre ſchön wenn man zusa{\geminationm}en ſpazieren gehen könnte!
               Wir waren heute über \textcolor{pink}{Liechtenſtein}{}\ledrightnote{\textcolor{pink}{Burg Liechtenstein}} bei Ihnen,
               leider vergeblich.\pend
           \pstart
           Es würde mir eine große Freude machen, wenn Sie Sonntag gegen
                  ½ 7 zu mir kommen und zum Nachtmahl bleiben würden. Es kommt \textcolor{blue}{\textsc{Zemlinsky}}{}\ledrightnote{\textcolor{blue}{Alexander von Zemlinsky}}, {\pb}der einiges aus dem \textcolor{green}{\textsc{Ballet}}{}\ledrightnote{→\textcolor{green}{Der Triumph der Zeit}}{ }ſpielen will, Herr \textcolor{blue}{\textsc{J. Wolff}}{}\ledrightnote{\textcolor{blue}{Erich J. Wolff}}, der die \textcolor{green}{\textsc{Pantomime}}{}\ledrightnote{→\textcolor{green}{Der Schüler. Pantomime in einem Aufzug}} auffallend hübſch componiert hat, eine \textcolor{blue}{Frau}{}\ledrightnote{→\textcolor{blue}{?? [Sängerin]}}, welche ſingt, ſonſt niemand.\pend
           \pstart Adieu. Von Herzen \spacefill\mbox{Hugo}\pend{}\pstart
           Samſtag bin ich nicht heraußen.\pend
           \pstart
           \numberlinefalse{}–\numberlinetrue{}\pend
           \pstart
           Sie haben Sonntag zur Rückfahrt Dampftramway um 9\textsuperscript{h}40.\pend
           \endnumbering\briefempfaengerindex{Schnitzler, Arthur@\textsc{Schnitzler, Arthur}!zzzHofmannsthal, Hugo von@\emph{von Hugo von Hofmannsthal}!1902-02-051@{{[}5. 2. 1902?{]}}|)be}\mylabel{h}  \normalsize

\doendnotes{C}
\bigskip
\vfill

\clearpage

\footnotesize

\lohead{\textsc{register}}

% Definiere theindex-Environment komplett neu ohne reledmac
\makeatletter
\renewenvironment{theindex}{%
  \section*{\indexname}%
  \setlength{\parindent}{0pt}%
  \setlength{\parskip}{0pt plus 0.3pt}%
  \let\item\@idxitem
}{%
  \clearpage
}
\makeatother

\IfFileExists{\jobname-pw.ind}{\input{\jobname-pw.ind}}{}

\end{document}

      