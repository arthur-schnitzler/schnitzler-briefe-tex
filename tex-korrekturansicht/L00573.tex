%% latex-korrekturansicht-vorspann.tex
%% Vorspann für die Korrekturansicht.
%% Lädt die gemeinsame Datei latex-vorspann.tex mit gesetztem Schalter.

\newif\ifkorrekturansicht
\korrekturansichttrue

\input{../tex-inputs/latex-vorspann}


               \section[Richard Beer-Hofmann an Arthur Schnitzler, 29. 7. 1896]{ Richard Beer-Hofmann an Arthur Schnitzler, 29. 7. 1896}\nopagebreak\mylabel{v}\rehead{ }\normalsize\beginnumbering\briefempfaengerindex{Schnitzler, Arthur@\textsc{Schnitzler, Arthur}!zzzBeer-Hofmann, Richard@\emph{von Richard Beer-Hofmann}!1896-07-293@{29. 7. 1896}|(be} \toendnotes[C]{\smallbreak\pagebreak[2]} \Standort{CUL, Schnitzler, B 8.}
\physDesc{Telegramm
\newline{}Handschrift einer Schreibkraft: blaue Tinte, lateinische Kurrent\newline{}Ordnung: mit Bleistift von unbekannter Hand nummeriert: »79« }\pstart{}{\pb}\textsc{Doktor Arthur}\pend{}\pstart{}\textsc{Schnitzler}\pend{}\pstart{}\textsc{poste restante \textcolor{pink}{Stcklm}{}\ledrightnote{\textcolor{pink}{Stockholm}}}\pend{}{\bigskip}\pstart
           {\pb}\textcolor{gray}{\textbf{Inlemnadt i}}{ }\textcolor{pink}{Köpenhamn}{}\ledrightnote{\textcolor{pink}{Kopenhagen}}{ }\textcolor{gray}{\textbf{Nr}} 44/2206{ }\textcolor{gray}{\textbf{Ord}} 18{ }\textcolor{gray}{\textbf{År}} 96{ }\textcolor{gray}{\textbf{Datum}}{ }29/7{ }\textcolor{gray}{\textbf{Kl.}} 2e\pend
           \pstart
           Wäre mir lieb, wenn sie schon 31 kämen bitte telegrafisch Antwort\pend
           \pstart \spacefill\mbox{Richard}\pend{}\endnumbering\briefempfaengerindex{Schnitzler, Arthur@\textsc{Schnitzler, Arthur}!zzzBeer-Hofmann, Richard@\emph{von Richard Beer-Hofmann}!1896-07-293@{29. 7. 1896}|)be}\mylabel{h}  \normalsize

\doendnotes{C}
\bigskip
\vfill

\clearpage

\footnotesize

\lohead{\textsc{register}}

% Definiere theindex-Environment komplett neu ohne reledmac
\makeatletter
\renewenvironment{theindex}{%
  \section*{\indexname}%
  \setlength{\parindent}{0pt}%
  \setlength{\parskip}{0pt plus 0.3pt}%
  \let\item\@idxitem
}{%
  \clearpage
}
\makeatother

\IfFileExists{\jobname-pw.ind}{\input{\jobname-pw.ind}}{}

\end{document}

      