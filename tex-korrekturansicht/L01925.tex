%% latex-korrekturansicht-vorspann.tex
%% Vorspann für die Korrekturansicht.
%% Lädt die gemeinsame Datei latex-vorspann.tex mit gesetztem Schalter.

\newif\ifkorrekturansicht
\korrekturansichttrue

\input{../tex-inputs/latex-vorspann}


               \section[Arthur Schnitzler an Hugo von Hofmannsthal, 21. 4. 1910]{ Arthur Schnitzler an Hugo von Hofmannsthal, 21. 4. 1910}\nopagebreak\mylabel{v}\rehead{ }\normalsize\beginnumbering\briefempfaengerindex{Hofmannsthal, Hugo von@\textsc{Hofmannsthal, Hugo von}!zzzSchnitzler, Arthur@\emph{von Arthur Schnitzler}!1910-04-211@{21. 4. 1910}|(be} \toendnotes[C]{\smallbreak\pagebreak[2]} \Standort{FDH, Hs-30885,136.}
\physDesc{Brief, 1 Blatt, 1 Seite, maschineller Durchschlag
\newline{}Schreibmaschine
\newline{}Handschrift: roter Buntstift, deutsche Kurrent (\noindent{}Beschriftung: »\textsc{Hofmannsthal}« und eine Unterstreichung)\newline{}Zusatz: Die Überlieferung im Nachlass Hofmannsthals deutet darauf hin, dass
                                 Schnitzler mit den eigenen Durchschlägen bei der Durchsicht seiner
                                 Briefe an Hofmannsthal 1929, Lücken ergänzte. }\buchAbdrucke{\weitereDrucke{Hugo von Hofmannsthal, Arthur Schnitzler: \emph{Briefwechsel}. Hg. Therese Nickl und Heinrich Schnitzler. Frankfurt am Main: \emph{S. Fischer} 1964, S. 249.} }\toendnotes[C]{\smallbreak}\pstart
           \raggedleft{}{\pb}21. 4. 1910.\pend
           \pstart\center{}Lieber Hugo. \pend\pstart
           Traf eben Dr. \textcolor{blue}{Anton
                  Bet{[}t{]}elheim}{}\ledrightnote{\textcolor{blue}{Anton Bettelheim}}; er wollte Ihnen schreiben. Es handelt sich
               um eine \label{K_L01925_1v}\edtext{\textcolor{brown}{Ebner-Eschenbach-Stiftung}{}\ledrightnote{\textcolor{brown}{Ebner-Eschenbach-Stiftung}}}{\lemma{\textnormal{\emph{Ebner-Eschenbach-Stiftung}}}\Cendnote{\textnormal{Obzwar im April 1910 ins
                  Leben gerufen, versandete das Unternehmen schnell. Ob tatsächlich Schulkindern zum
                  Schulabschluss Werke \textcolor{blue}{Ebner-Eschenbach}s
                  geschenkt wurden, ist nicht nachgewiesen.}}}\label{K_L01925_1h} zum 80. Geburtstag. Aufruf: \textcolor{blue}{Erich Schmiedt}{}\ledrightnote{\textcolor{blue}{Erich Schmidt}}{ }\textcolor{blue}{Lobmeyer}{}\ledrightnote{\textcolor{blue}{Ludwig Lobmeyr}}, \textcolor{blue}{Schönherr}{}\ledrightnote{\textcolor{blue}{Karl Schönherr}}, ich etc. Sie werden gebeten auch zu unterschreiben.
               Verpflichtungen sind damit keine übernommen, man zeichnet dann einen kleinen Betrag
               (ich zum Exempel etwa 20 K.). Bitte um eine Zeile, ob ich \textcolor{blue}{Bettelheim}{}\ledrightnote{\textcolor{blue}{Anton Bettelheim}} Ihre Zustimmung vermelden darf.\pend
           \endnumbering\briefempfaengerindex{Hofmannsthal, Hugo von@\textsc{Hofmannsthal, Hugo von}!zzzSchnitzler, Arthur@\emph{von Arthur Schnitzler}!1910-04-211@{21. 4. 1910}|)be}\mylabel{h}  \normalsize

\doendnotes{C}
\bigskip
\vfill

\clearpage

\footnotesize

\lohead{\textsc{register}}

% Definiere theindex-Environment komplett neu ohne reledmac
\makeatletter
\renewenvironment{theindex}{%
  \section*{\indexname}%
  \setlength{\parindent}{0pt}%
  \setlength{\parskip}{0pt plus 0.3pt}%
  \let\item\@idxitem
}{%
  \clearpage
}
\makeatother

\IfFileExists{\jobname-pw.ind}{\input{\jobname-pw.ind}}{}

\end{document}

      