%% latex-korrekturansicht-vorspann.tex
%% Vorspann für die Korrekturansicht.
%% Lädt die gemeinsame Datei latex-vorspann.tex mit gesetztem Schalter.

\newif\ifkorrekturansicht
\korrekturansichttrue

\input{../tex-inputs/latex-vorspann}


\renewcommand{\erwaehntePersonen}{Personen: Frieda Pollak, Felix Salten}
\renewcommand{\erwaehnteInstitutionen}{Institutionen: Wiener Literarische Anstalt}
\renewcommand{\erwaehnteOrte}{Orte: Berchtesgaden, Berghof, Leipzig, Unterach am Attersee, Wien}
\renewcommand{\erwaehnteWerke}{Werke: Das Burgtheater. Naturgeschichte eines alten Hauses}
\section[ Felix Salten an Arthur Schnitzler, 17. 8. 1922]{Felix Salten an Arthur Schnitzler, 17. 8. 1922}
\nopagebreak\mylabel{v}
\rehead{ }\normalsize\beginnumbering\briefempfaengerindex{Schnitzler, Arthur@\textsc{Schnitzler, Arthur}!zzzSalten, Felix@\emph{von Felix Salten}!1922-08-171@{17. 8. 1922}|(be}
\toendnotes[C]{\smallbreak\pagebreak[2]}\Standort{CUL, Schnitzler, B 89, B 2.}
\physDesc{Brief, 1 Blatt, 1 Seite, 388 Zeichen
\newline{}Handschrift: schwarze Tinte, lateinische Kurrent
\newline{}Ordnung: 1) mit Bleistift von \textcolor{blue}{Frieda Pollak} (?) mit
                                 dem Buchstaben »A« (Abgeschrieben/Abschrift)
                                 gekennzeichnet  2) mit Bleistift von unbekannter Hand nummeriert: »29\substVorne{}\textsuperscript{3}\substDazwischen{}2\substHinten{}.«}\toendnotes[C]{\smallbreak}
\pstart
           \raggedleft{}{\pb}\textcolor{pink}{Berghof}{}\ledrightnote{\textcolor{pink}{Berghof}}, 17. 8. 22.\pend
           
\pstart
           Lieber, vielen Dank für Ihre \label{K_L03582-1v}\edtext{Karte}{\lemma{\textnormal{\emph{Karte}}}\Cendnote{\textnormal{nicht
                  erhalten}}}\label{K_L03582-1h}. Es geht uns allen ganz gut. Ich bin seit drei Wochen \textcolor{pink}{da}{}\ledrightnote{{$\rightarrow$}\textcolor{pink}{Unterach am Attersee}} und faullenze. Lassen Sie sich
               das beiliegende kleine \label{K_L03582-2v}\edtext{\textcolor{green}{Buch}{}\ledrightnote{{$\rightarrow$}\textcolor{green}{Das Burgtheater. Naturgeschichte eines alten Hauses}}}{\lemma{\textnormal{\emph{Buch}}}\Cendnote{\textnormal{Beilage nicht erhalten; vermutlich war
                  es: \textcolor{blue}{Felix Salten}: \emph{\textcolor{green}{Das Burgtheater. Naturgeschichte eines alten Hauses}}.
                        \textcolor{pink}{Wien}/\textcolor{pink}{Leipzig}: \emph{\textcolor{brown}{WILA Wiener literarische
                        Anstalt}}{ }1922.}}}\label{K_L03582-2h} gefallen. Und – wenn es irgend geht, – aber es ginge gewiß! –
               \label{K_L03582-3v}\edtext{kommen Sie doch jetzt}{\lemma{\textnormal{\emph{kommen Sie doch jetzt}}}\Cendnote{\textnormal{Zu \textcolor{blue}{Schnitzler}s
                  Verhältnis zum \textcolor{pink}{Berghof}{ }siehe Felix Salten an Arthur Schnitzler, [25.? 8. 1892].}}}\label{K_L03582-3h}, da Sie so nahe
               sind, auf der Heimfahrt wenigstens für ein paar Tage zu uns. Wir würden uns alle so
               sehr mit Ihnen freuen!\pend
           
\pstart
           Herzlichst Ihr {\\[\baselineskip]}\spacefill\mbox{Salten}\pend
           \leftskip=0em{}\endnumbering\briefempfaengerindex{Schnitzler, Arthur@\textsc{Schnitzler, Arthur}!zzzSalten, Felix@\emph{von Felix Salten}!1922-08-171@{17. 8. 1922}|)be}\mylabel{h}  \normalsize

\doendnotes{C}
\bigskip
\vfill

\clearpage

\footnotesize

\lohead{\textsc{register}}

% Definiere theindex-Environment komplett neu ohne reledmac
\makeatletter
\renewenvironment{theindex}{%
  \section*{\indexname}%
  \setlength{\parindent}{0pt}%
  \setlength{\parskip}{0pt plus 0.3pt}%
  \let\item\@idxitem
}{%
  \clearpage
}
\makeatother

\IfFileExists{\jobname-pw.ind}{\input{\jobname-pw.ind}}{}

\end{document}

      