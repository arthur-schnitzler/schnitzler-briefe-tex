%% latex-korrekturansicht-vorspann.tex
%% Vorspann für die Korrekturansicht.
%% Lädt die gemeinsame Datei latex-vorspann.tex mit gesetztem Schalter.

\newif\ifkorrekturansicht
\korrekturansichttrue

\input{../tex-inputs/latex-vorspann}


               \section[Paul Goldmann an Arthur Schnitzler, 1. 4. {[}1896{]}]{ Paul Goldmann an Arthur Schnitzler, 1. 4. {[}1896{]}}\nopagebreak\mylabel{v}\rehead{ }\normalsize\beginnumbering\briefempfaengerindex{Schnitzler, Arthur@\textsc{Schnitzler, Arthur}!zzzGoldmann, Paul@\emph{von Paul Goldmann}!1896-04-011@{1. 4. {[}1896{]}}|(be} \toendnotes[C]{\smallbreak\pagebreak[2]} \Standort{DLA, A:Schnitzler, HS.NZ85.1.3166.}
\physDesc{Brief, 1 Blatt, 5 Seiten
\newline{}Handschrift: blaue Tinte, deutsche Kurrent
\newline{}Schnitzler: 1) mit Bleistift das Jahr »96« vermerkt 2) mit rotem Buntstift vier Unterstreichungen}\toendnotes[C]{\smallbreak}\pstart
           \noindent{}{\pb}\textcolor{gray}{\textbf{\textbf{\textcolor{brown}{Frankfurter Zeitung}{}\ledrightnote{\textcolor{brown}{Frankfurter Zeitung}}}}}\pend
           \pstart
           \textcolor{gray}{\textbf{(\textcolor{brown}{\begin{otherlanguage}{french}Gazette de Francfort\end{otherlanguage}}{}\ledrightnote{\textcolor{brown}{Frankfurter Zeitung}}).}}\pend
           \pstart
           \textcolor{gray}{\textbf{\textbf{\begin{otherlanguage}{french}Fondateur M.\end{otherlanguage}{ }\textcolor{blue}{L. Sonnemann}{}\ledrightnote{\textcolor{blue}{Leopold Sonnemann}}.}}}\pend
           \pstart
           \begin{otherlanguage}{french}\textcolor{gray}{\textbf{\textcolor{green}{Journal}{}\ledrightnote{→\textcolor{green}{Frankfurter Zeitung}} politique,
                        financier,}}\end{otherlanguage}\pend
           \pstart
           \begin{otherlanguage}{french}\textcolor{gray}{\textbf{commercial et littéraire.}}\end{otherlanguage}\pend
           \pstart
           \begin{otherlanguage}{french}\textcolor{gray}{\textbf{\textbf{Paraissant trois fois par jour.}}}\end{otherlanguage}\hfill \textsc{\textcolor{pink}{Paris}{}\ledrightnote{\textcolor{pink}{Paris}}}, 1. April.\pend
           \pstart
           \begin{otherlanguage}{french}\textcolor{gray}{\textbf{\textbf{Bureau à \textcolor{pink}{Paris}{}\ledrightnote{\textcolor{pink}{Paris}}}}}\end{otherlanguage}\pend
           \pstart
           \begin{otherlanguage}{french}\textcolor{gray}{\textbf{\textbf{\textcolor{pink}{24. Rue Feydeau}{}\ledrightnote{\textcolor{pink}{rue Feydeau}}.}}}\end{otherlanguage}\pend
           \pstart\center{}Mein lieber Freund,\pend\pstart
           Du ſiehſt wohl, was Alles in der \textcolor{pink}{franzöſiſch}{}\ledrightnote{→\textcolor{pink}{Frankreich}}en Politik vorgeht. Der Teufel iſt los, und ich komme noch immer
               nicht dazu, Dir zu ſchreiben. Ich will Dir nur in der Eile für Deinen letzten lieben
               Brief danken. Auch für Deine Photographie, die mich unendlich erfreut hat, habe ich
               Dir wohl noch nicht gedankt. \textsc{\textcolor{blue}{Richard Specht}{}\ledrightnote{\textcolor{blue}{Richard Specht}}} iſt hier und macht mir viel Vergnügen; er iſt ein lieber, ſanfter \textcolor{blue}{Menſch}{}\ledrightnote{→\textcolor{blue}{Richard Specht}} geworden; aber Talent
               hat er wohl nicht; er las uns ein \label{K_L02769-99v}\edtext{\textcolor{green}{Versdrama}{}\ledrightnote{→\textcolor{green}{Pierrot bossu. Eine Commedia dell’Arte zur Fastnacht in gar zierlichen Reimen}}}{\lemma{\textnormal{\emph{Versdrama}}}\Cendnote{\textnormal{\emph{\textcolor{green}{Pierrot bossu. Eine Commedia dell’Arte zur
                     Fastnacht in gar zierlichen Reimen}}. Verfertigt von \textcolor{blue}{Richard Specht}, war Mitte Februar 1896 bei \emph{\textcolor{brown}{E. Pierson}} erschienen.}}}\label{K_L02769-99h}; Verſe, aber
               keine Poeſie. Armer Burſch! Er möchte ſo gern!\pend
           \pstart
           {\pb}Was Du über die Judenfrage im Zuſammenhang mit \textsc{\textcolor{blue}{Herzl}{}\ledrightnote{\textcolor{blue}{Theodor Herzl}}s}{ }\textcolor{green}{Buch}{}\ledrightnote{→\textcolor{green}{Der Judenstaat. Versuch einer modernen Lösung der Judenfrage}} ſchreibſt, iſt prächtig
               und mir ganz aus der Seele geſprochen. Aber das \textcolor{green}{Buch}{}\ledrightnote{→\textcolor{green}{Der Judenstaat. Versuch einer modernen Lösung der Judenfrage}} iſt wirklich albern, – oberflächlich noch dazu und
               falſch ſentimental. Echte ſchlechte Feuilletoniſten-Literatur. Aber wie verbohrt, wie
               falſch beobachtend muß ein Menſch ſein, der heut noch behauptet, die Juden ſeien ein
               Volk. Du und ich, der Rabbi{ }\strikeout{\textcolor{gray}{×}\-\textcolor{gray}{×}\-\textcolor{gray}{×}\-\textcolor{gray}{×}\-\textcolor{gray}{×}{ }\textsc{\textcolor{gray}{Bloch}}}{ }\textsc{\label{K_L02769-88v}\edtext{\textcolor{blue}{Bloch}{}\ledrightnote{\textcolor{blue}{Joseph Samuel Bloch}}}{\lemma{\textnormal{\emph{Bloch}}}\Cendnote{\textnormal{\textcolor{blue}{Joseph Samuel Bloch} trat als
                     Abgeordneter im \emph{\textcolor{brown}{Reichsrat}} engagiert gegen
                     antisemitische Verleumdungen auf.}}}\label{K_L02769-88h}} und der Jud’, der unten »handel\textcolor{gray}{e}« ſchreit – ein Volk! Das
               iſt echt \textsc{\textcolor{blue}{Herzl}{}\ledrightnote{\textcolor{blue}{Theodor Herzl}}}. So hat er auch die \textcolor{pink}{franzöſiſch}{}\ledrightnote{→\textcolor{pink}{Frankreich}}en Dinge angeſchaut u. immer unrichtig geſehen. Für mich gibt es
               eben nur eine Löſung der Judenfrage: daß die Juden ſchließlich {\pb}Alle Chriſten werden. Jeſus iſt mir doch der
               ſympathiſcheſte Jude und ich will gern zu ſeinen Jüngern zählen. {\dotsfour}\pend
           \pstart
           Mein \textcolor{blue}{Onkel}{}\ledrightnote{→\textcolor{blue}{Fedor Mamroth}} hat nett über »\textsc{\textcolor{green}{Anatol}{}\ledrightnote{\textcolor{green}{Anatol}}}« \label{K_L02769-888v}\edtext{\textcolor{green}{geſchrieben}{}\ledrightnote{→\textcolor{green}{Schauspielhaus. [Untreu und Abschiedssouper]}}}{\lemma{\textnormal{\emph{geſchrieben}}}\Cendnote{\textnormal{\textcolor{blue}{m.} [=\textcolor{blue}{Fedor Mamroth}]: \emph{\textcolor{green}{Schauspielhaus}}.
                     In: \emph{\textcolor{green}{Frankfurter Zeitung}}, Jg. XXXX,
                     Nr. YYYY, 29. 3. 1896, S. YYYY. Er besprach die gemeinsame
                  Aufführung von \emph{\textcolor{green}{Untreu}} von \textcolor{blue}{Roberto Bracco} und \textcolor{blue}{Schnitzler}s \emph{\textcolor{green}{Abschiedssouper}} am \emph{\textcolor{brown}{Frankfurter Schauspielhaus}} am
                     26. 3. 1896.}}}\label{K_L02769-888h}. Meine \textcolor{blue}{Mutter}{}\ledrightnote{→\textcolor{blue}{Clementine Goldmann}} ſendet noch folgende Ergänzungs-Kritik:\pend
           {\bigskip}\pstart
           \noindent{}{[}hs. Clementine Goldmann:{]} \label{T_L02769-12v}\edtext{Das \textcolor{green}{»Abschieds« Souper}{}\ledrightnote{\textcolor{green}{Abschiedssouper}} von deinem Freunde hat uns ſehr gefallen – we{\geminationn} es auch für die ſtupiden \textcolor{pink}{Frankfurt}{}\ledrightnote{\textcolor{pink}{Frankfurt am Main}}er – viel zu fein war.}{\lemma{\textnormal{\emph{Das … war.}}}\Cendnote{\textnormal{Ausschnitt aus einem Brief von \textcolor{blue}{Clementine Goldmann} auf einem eingeklebten Zettel (blaue
                  Tinte, deutsche Kurrentschrift)}}}\label{T_L02769-12h}\pend
           {\bigskip}\pstart
           \noindent{}{[}hs. Paul Goldmann:{]} Oſtern möchte ich nach \textcolor{pink}{Frankfurt}{}\ledrightnote{\textcolor{pink}{Frankfurt am Main}} fahren, weiß aber noch nicht, woher ich das Geld nehmen werde.
               Aber ich bin todt gearbeitet und habe ein {\pb}heftiges
               Bedürfniß nach ein paar Ruhetagen. Mit meinen Augen geht es ſchlecht, ſie wollen
               nicht mehr mit, und ich habe große Sorgen.\pend
           \pstart
           Vielleicht ſchreibe ich Dir den langen Brief doch noch vor den Feiertagen. Wenn
               nicht: fröhliche Oſtern.\pend
           \pstart
           Grüß’ Dich Gott, mein lieber Freund{\\[\baselineskip]}Dein {\\[\baselineskip]}\spacefill\mbox{Paul Goldmann.}\pend
           \leftskip=0em{}\pstart
           \noindent{}Der \label{K_L02769-4v}\edtext{\textcolor{green}{Artikel}{}\ledrightnote{→\textcolor{green}{Gedichte von Stefan George}}}{\lemma{\textnormal{\emph{Artikel}}}\Cendnote{\textnormal{\textcolor{blue}{Hugo von Hofmannsthal}: \emph{\textcolor{green}{Gedichte von Stefan George}}. In: \emph{\textcolor{green}{Die Zeit}}, Bd. 6, Nr. 77, 21. 3. 1896, S. 189–191.}}}\label{K_L02769-4h} des
                  kleinen \textsc{\textcolor{blue}{Loris}{}\ledrightnote{\textcolor{blue}{Hugo von Hofmannsthal}}} in der »\textcolor{green}{Zeit}{}\ledrightnote{\textcolor{green}{Die Zeit. Wiener Wochenschrift}}« über \textsc{\textcolor{blue}{Stefan Georges}{}\ledrightnote{\textcolor{blue}{Stefan George}}} hat mich einfach empört. \textsc{\textcolor{blue}{Stefan Georges}{}\ledrightnote{\textcolor{blue}{Stefan George}}} iſt eine prätentiöſe Talentloſigkeit, und der \textcolor{green}{Artikel}{}\ledrightnote{→\textcolor{green}{Gedichte von Stefan George}}, abgeſehen von dem falſchen
                  Urtheil, iſt in einem unerhört ſchwülſtigen u. manierirten Styl geſchrieben. Ein
                  zweiter \textsc{\textcolor{blue}{Hermann Bahr}{}\ledrightnote{\textcolor{blue}{Hermann Bahr}}}!\pend
           \pstart
           {\pb}\label{T_L02769-1v}\edtext{\uline{Gruß an \textsc{\textcolor{blue}{Richard}{}\ledrightnote{\textcolor{blue}{Richard Beer-Hofmann}}}!}}{\lemma{\textnormal{\emph{Gruß an Richard!}}}\Cendnote{\textnormal{kopfüber am oberen Rand der ersten
                     Seite}}}\label{T_L02769-1h}\pend
           \endnumbering\briefempfaengerindex{Schnitzler, Arthur@\textsc{Schnitzler, Arthur}!zzzGoldmann, Paul@\emph{von Paul Goldmann}!1896-04-011@{1. 4. {[}1896{]}}|)be}\mylabel{h}  \normalsize

\doendnotes{C}
\bigskip
\vfill

\clearpage

\footnotesize

\lohead{\textsc{register}}

% Definiere theindex-Environment komplett neu ohne reledmac
\makeatletter
\renewenvironment{theindex}{%
  \section*{\indexname}%
  \setlength{\parindent}{0pt}%
  \setlength{\parskip}{0pt plus 0.3pt}%
  \let\item\@idxitem
}{%
  \clearpage
}
\makeatother

\IfFileExists{\jobname-pw.ind}{\input{\jobname-pw.ind}}{}

\end{document}

      