%% latex-korrekturansicht-vorspann.tex
%% Vorspann für die Korrekturansicht.
%% Lädt die gemeinsame Datei latex-vorspann.tex mit gesetztem Schalter.

\newif\ifkorrekturansicht
\korrekturansichttrue

\input{../tex-inputs/latex-vorspann}


\renewcommand{\erwaehntePersonen}{Personen: Sébastien Roch Nicolas Chamfort, Leo Ebermann, Eduard Grisebach, Gerhart Hauptmann, Fritz Mauthner, Olga Schnitzler, Arthur Schopenhauer, Elisabeth Steinrück, Irene Triesch}
\renewcommand{\erwaehnteInstitutionen}{Institutionen: Ernst Hofmann {\kaufmannsund}  Co.}
\renewcommand{\erwaehnteOrte}{Orte: Berlin, Dessauer Straße, Hamburg, München, Payerbach, Wien}
\renewcommand{\erwaehnteWerke}{Werke: Berliner Brief. [»Schluck und Jau« von Gerhart Hauptmann am Deutschen Theater], Berliner Tageblatt, Berliner Theater. »Der Rothe Hahn.«, Berliner Theater. »Einsame Menschen« im Deutschen Theater, Einsame Menschen. Drama, Hebbels »Maria Magdalena«. (Deutsches Theater.), Lebendige Stunden. Vier Einakter, Neue Freie Presse, Schopenhauer’s Gespräche und Selbstgespräche: Nach der Handschrift eis heauton, »Michael Kramer.«, Œuvres choisies de N. Chamfort, publiées avec préface, notes et tables}
\section[ Paul Goldmann an Arthur Schnitzler, 9. 11. {[}1901{]}]{Paul Goldmann an Arthur Schnitzler, 9. 11. {[}1901{]}}
\nopagebreak\mylabel{v}
\rehead{ }\normalsize\beginnumbering\briefempfaengerindex{Schnitzler, Arthur@\textsc{Schnitzler, Arthur}!zzzGoldmann, Paul@\emph{von Paul Goldmann}!1901-11-092@{9. 11. {[}1901{]}}|(be}
\toendnotes[C]{\smallbreak\pagebreak[2]}\Standort{DLA, A:Schnitzler, HS.NZ85.1.3171.}
\physDesc{Brief, 1 Blatt, 4 Seiten
\newline{}Handschrift: blaue Tinte, deutsche Kurrent
\newline{}Schnitzler: 1) mit Bleistift das Jahr »1901« vermerkt  2) mit rotem Buntstift drei Unterstreichungen}\toendnotes[C]{\smallbreak}
\pstart
           \noindent{}\raggedleft{}{\pb}\textcolor{pink}{\textcolor{gray}{\textbf{DESSAUERSTRASSE 19}}}{}\ledrightnote{\textcolor{pink}{Dessauer Straße}}\pend
           
\pstart
           \textcolor{pink}{Berlin}{}\ledrightnote{\textcolor{pink}{Berlin}}, 9. November.\pend
           
\pstart\center{}Mein lieber Freund,\pend
\pstart
           Ich habe \introOben{}mich\introOben{} ſehr gefreut, endlich wieder einmal etwas von
               Dir zu hören. Daß die Aufführung Deiner \textcolor{green}{Stücke}{}\ledrightnote{{$\rightarrow$}\textcolor{green}{Lebendige Stunden. Vier Einakter}} bis \label{K_L03090-1v}\edtext{Februar}{\lemma{\textnormal{\emph{Februar}}}\Cendnote{\textnormal{Grund war ein geplantes Gastspiel \textcolor{blue}{Irene Triesch}s, die in den weiblichen
                  Hauptrollen auftrat (vgl. \emph{Der Briefwechsel Arthur
                        Schnitzler — Otto Brahm}. Vollständige Ausgabe. Herausgegeben,
                     eingeleitet und erläutert von Oskar Seidlin. Tübingen:
                        \emph{Niemeyer}{ }1975, S. 102). Die Uraufführung konnte
                  schließlich noch vor \textcolor{blue}{Triesch}s geplanter
                  Abwesenheit (Mitte Januar bis Mitte Februar 1902), am 4. 1. 1902,
                  stattfinden.}}}\label{K_L03090-1h} verſchoben werden ſoll, iſt bedauerlich. Könnteſt Du nicht
               wenigſtens anderswo, in \textcolor{pink}{Hamburg}{}\ledrightnote{\textcolor{pink}{Hamburg}}, \textcolor{pink}{München}{}\ledrightnote{\textcolor{pink}{München}}, vielleicht gar in \textcolor{pink}{Wien}{}\ledrightnote{\textcolor{pink}{Wien}}, eine frühere Aufführung veranlaſſen \introOben{}damit Dir nicht
                  der Winter verloren geht\introOben{}? Die \textsc{\textcolor{blue}{Triesch}{}\ledrightnote{\textcolor{blue}{Irene Triesch}}} wird hier von der kunſtunverſtändigen Kritik ſo {\pb}\label{K_L03090-2v}\edtext{wenig begriffen}{\lemma{\textnormal{\emph{wenig begriffen}}}\Cendnote{\textnormal{Siehe etwa \textcolor{blue}{F. M.} [=\textcolor{blue}{Fritz Mauthner}]: \emph{\textcolor{green}{Hebbels »Maria Magdalena«. (Deutsches Theater.)}}. In:
                        \emph{\textcolor{green}{Berliner Tageblatt}}, Jg. 30, Nr. 565,
                        6. 11. 1901, S. [3].}}}\label{K_L03090-2h}, daß es
               beinahe eine Gefahr für Deine \textcolor{green}{Stücke}{}\ledrightnote{{$\rightarrow$}\textcolor{green}{Lebendige Stunden. Vier Einakter}} iſt, wenn ſie die \label{K_L03090-3v}\edtext{Hauptrolle}{\lemma{\textnormal{\emph{Hauptrolle}}}\Cendnote{\textnormal{siehe Paul Goldmann an Arthur Schnitzler, 23. 9. [1901]}}}\label{K_L03090-3h} ſpielt, die ſie natürlich herrlich ſpielen wird. Ich habe mit dieſer
               hyſteriſchen Jüdin, die mir unerträglich geworden iſt, alle Beziehungen
               abgebrochen.\pend
           
\pstart
           Daß \label{K_L03090-4v}\edtext{\textsc{\textcolor{blue}{Olga}{}\ledrightnote{\textcolor{blue}{Olga Schnitzler}}} krank}{\lemma{\textnormal{\emph{Olga krank}}}\Cendnote{\textnormal{\textcolor{blue}{Sie} hatte Angina (vgl. A. S.: \emph{Tagebuch}, 25. 10. 1901).}}}\label{K_L03090-4h} war,
               habe ich mit Bedauern vernommen. Was ihr gefehlt hat, habe ich, trotz langjähriger
               Kenntniß Deiner Handſchrift, nicht entziffern können. Immerhin freue ich mich, daß
               ſie wieder geſund iſt, und bitte Dich, ſie ſammt {\pb}der \textcolor{blue}{Schweſter}{}\ledrightnote{{$\rightarrow$}\textcolor{blue}{Elisabeth Steinrück}} zu
               grüßen.\pend
           
\pstart
           Was meine \label{K_L03090-5v}\edtext{\textcolor{green}{Feuilletons}{}\ledrightnote{{$\rightarrow$}\textcolor{green}{Einsame Menschen. Drama}{\newline}{$\rightarrow$}\textcolor{green}{Berliner Brief. [»Schluck und Jau« von Gerhart Hauptmann am Deutschen Theater]}{\newline}{$\rightarrow$}\textcolor{green}{»Michael Kramer.«}{\newline}{$\rightarrow$}\textcolor{green}{Berliner Theater. »Der Rothe Hahn.«}}}{\lemma{\textnormal{\emph{Feuilletons}}}\Cendnote{\textnormal{\textcolor{blue}{Paul Goldmann}: \emph{\textcolor{green}{Berliner Brief}}. In: \emph{\textcolor{green}{Neue Freie Presse}}, Nr. 12.735, 6. 2. 1900, Morgenblatt, S. 1–3. \textcolor{blue}{Paul Goldmann}: \emph{\textcolor{green}{»Michael Kramer.«}}. In: \emph{\textcolor{green}{Neue Freie Presse}}, Nr. 13.055, 28. 12. 1900, Morgenblatt, S. 1–3. \textcolor{blue}{Paul Goldmann}: \emph{\textcolor{green}{Berliner Theater. »Der Rothe Hahn.«}}. In: \emph{\textcolor{green}{Neue Freie Presse}}, Nr. 13.391, 4. 12. 1901, Morgenblatt, S. 1–3. \textcolor{blue}{Paul Goldmann}: \emph{\textcolor{green}{Berliner Theater. »Einsame Menschen« im Deutschen Theater}}.
                     In: \emph{\textcolor{green}{Neue Freie Presse}}, Nr. 13.345, 19. 10. 1901, Morgenblatt, S. 1–3. }}}\label{K_L03090-5h}
               über \textsc{\textcolor{blue}{Gerhart Hauptmann}{}\ledrightnote{\textcolor{blue}{Gerhart Hauptmann}}} anlangt, ſo ſtimmen mir noch andere Leute zu, als Herr \textsc{\textcolor{blue}{Ebermann}{}\ledrightnote{\textcolor{blue}{Leo Ebermann}}}. Im Übrigen wäre es mir ſehr gleichgiltig, auch wenn Niemand mir zuſtimmte, da
               ich weiß, daß ich Recht habe. Was Du über den \label{K_L03090-7v}\edtext{»Ton«}{\lemma{\textnormal{\emph{»Ton«}}}\Cendnote{\textnormal{siehe A. S.: \emph{Tagebuch}, 27. 11. 1901 und Paul Goldmann an Arthur Schnitzler, 6. 12. [1901]}}}\label{K_L03090-7h} ſchreibſt, verſtehe ich nicht. Das heißt, ich begreife nicht, wie Einer, der
               ſelbſt ſchreibt, dieſen Einwand erheben kann. Mein Ton bin nämlich ich ſelbſt. Aus
               dieſem Grunde wird es nicht leicht ſein, ihn zu ändern.\pend
           
\pstart
           Es thut mir unendlich leid, daß {\pb}durch den Aufſchub
               der Aufführung Deiner \textcolor{green}{Stücke}{}\ledrightnote{{$\rightarrow$}\textcolor{green}{Lebendige Stunden. Vier Einakter}}{ }\strikeout{D\textcolor{gray}{ei}} auch Deine \label{K_L03090-8v}\edtext{Reiſe nach \textcolor{pink}{Berlin}{}\ledrightnote{\textcolor{pink}{Berlin}} verſchoben}{\lemma{\textnormal{\emph{Reiſe … verſchoben}}}\Cendnote{\textnormal{\textcolor{blue}{Schnitzler} war letztendlich von 28. 12. 1901 bis 6. 1. 1902 in \textcolor{pink}{Berlin}.}}}\label{K_L03090-8h} iſt.\pend
           
\pstart
           Haſt Du den \label{K_L03090-9v}\edtext{\textsc{\textcolor{blue}{\textcolor{green}{Chamfort}{}\ledrightnote{{$\rightarrow$}\textcolor{green}{Œuvres choisies de N. Chamfort, publiées avec préface, notes et tables}}}{}\ledrightnote{\textcolor{blue}{Sébastien Roch Nicolas Chamfort}}}}{\lemma{\textnormal{\emph{Chamfort}}}\Cendnote{\textnormal{siehe Paul Goldmann an Arthur Schnitzler, 23. 9. [1901]}}}\label{K_L03090-9h} nun endlich erhalten? Und haſt Du ihn geleſen? Lies’ auch die eben von \textsc{\textcolor{blue}{Griesebach}{}\ledrightnote{\textcolor{blue}{Eduard Grisebach}}} herausgegebenen \label{K_L03090-11v}\edtext{\textcolor{green}{Geſpräche mit \textsc{\textcolor{blue}{Schopenhauer}{}\ledrightnote{\textcolor{blue}{Arthur Schopenhauer}}}}{}\ledrightnote{\textcolor{green}{Schopenhauer’s Gespräche und Selbstgespräche: Nach der Handschrift eis heauton}}}{\lemma{\textnormal{\emph{Geſpräche mit Schopenhauer}}}\Cendnote{\textnormal{\emph{\textcolor{green}{Schopenhauer’s Gespräche und Selbstgespräche:
                        Nach der Handschrift eis heauton}}. Hg. v. \textcolor{blue}{Eduard Grisebach}. \textcolor{pink}{Berlin}: \emph{\textcolor{brown}{Ernst Hofmann {\kaufmannsund} Co.}}{ }1898. Eine Lektüre durch \textcolor{blue}{Schnitzler} ist
                  nicht belegbar.}}}\label{K_L03090-11h}.\pend
           
\pstart
           Leb’ wohl für heut! Viele treue Grüße! {\\[\baselineskip]}Dein {\\[\baselineskip]}\spacefill\mbox{Paul Goldmann.}\pend
           \leftskip=0em{}\endnumbering\briefempfaengerindex{Schnitzler, Arthur@\textsc{Schnitzler, Arthur}!zzzGoldmann, Paul@\emph{von Paul Goldmann}!1901-11-092@{9. 11. {[}1901{]}}|)be}\mylabel{h}
\begin{anhang}
\end{anhang}\normalsize

\doendnotes{C}
\bigskip
\vfill

\clearpage

\footnotesize

\lohead{\textsc{register}}

% Definiere theindex-Environment komplett neu ohne reledmac
\makeatletter
\renewenvironment{theindex}{%
  \section*{\indexname}%
  \setlength{\parindent}{0pt}%
  \setlength{\parskip}{0pt plus 0.3pt}%
  \let\item\@idxitem
}{%
  \clearpage
}
\makeatother

\IfFileExists{\jobname-pw.ind}{\input{\jobname-pw.ind}}{}

\end{document}

      