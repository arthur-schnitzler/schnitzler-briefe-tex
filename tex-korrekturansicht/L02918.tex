%% latex-korrekturansicht-vorspann.tex
%% Vorspann für die Korrekturansicht.
%% Lädt die gemeinsame Datei latex-vorspann.tex mit gesetztem Schalter.

\newif\ifkorrekturansicht
\korrekturansichttrue

\input{../tex-inputs/latex-vorspann}


         
         \renewcommand{\erwaehntePersonen}{Personen: Auguste Chlum, Marie Glümer, Gilbert Otto Neumann-Hofer}
         \renewcommand{\erwaehnteInstitutionen}{Institutionen: Lessing-Theater}
         \renewcommand{\erwaehnteOrte}{Orte: Berlin, Wien}
         \renewcommand{\erwaehnteWerke}{}
               \section[ Paul Goldmann an Arthur Schnitzler, 31. 5. {[}1900{]}]{Paul Goldmann an Arthur Schnitzler, 31. 5. {[}1900{]}}\nopagebreak\mylabel{v}\rehead{ }\normalsize\beginnumbering\briefempfaengerindex{Schnitzler, Arthur@\textsc{Schnitzler, Arthur}!zzzGoldmann, Paul@\emph{von Paul Goldmann}!1900-05-311@{31. 5. {[}1900{]}}|(be} \toendnotes[C]{\smallbreak\pagebreak[2]} \Standort{DLA, A:Schnitzler, HS.NZ85.1.3170.}
\physDesc{Brief, 1 Blatt, 2 Seiten
\newline{}Handschrift: schwarze Tinte, deutsche Kurrent
\newline{}Schnitzler: 1) mit Bleistift das Jahr »{[}1{]}900« vermerkt  2) mit rotem Buntstift eine Unterstreichung}\toendnotes[C]{\smallbreak}\pstart
           \raggedleft{}{\pb}\textcolor{pink}{Berlin}{}\ledrightnote{\textcolor{pink}{Berlin}}, 31.
                  Mai.\pend
           \pstart\center{}Mein lieber Freund,\pend\pstart
           Der \textcolor{blue}{Direktor}{}\ledrightnote{{$\rightarrow$}\textcolor{blue}{Gilbert Otto Neumann-Hofer}} des \textcolor{brown}{\textsc{Lessing}-Theater}{}\ledrightnote{\textcolor{brown}{Lessing-Theater}}s hat eben dem \textsc{Frl. \textcolor{blue}{Glümer}{}\ledrightnote{\textcolor{blue}{Marie Glümer}}} ihre dreimonatliche Kündigung geſchickt. Das arme \textcolor{blue}{Mädel}{}\ledrightnote{{$\rightarrow$}\textcolor{blue}{Marie Glümer}}, die heut bereits nach \textcolor{pink}{Wien}{}\ledrightnote{\textcolor{pink}{Wien}} reiſen wollte,
               iſt ganz niedergeſchmettert. Wir ſitzen eben bei \textsc{\textcolor{blue}{Glümers}{}\ledrightnote{{$\rightarrow$}\textcolor{blue}{Marie Glümer}{\newline}{$\rightarrow$}\textcolor{blue}{Auguste Chlum}}} zuſammen und berathen. Das heißt \textsc{\textcolor{blue}{Gusti}{}\ledrightnote{\textcolor{blue}{Auguste Chlum}}} und ich. \textsc{\textcolor{blue}{Mizzi}{}\ledrightnote{\textcolor{blue}{Marie Glümer}}} iſt nach durchwachten und durchweinten Nächten endlich ein wenig eingeſchlafen.
               Ich ſage, das Nöthigſte ſei, Dir zu {\pb}ſchreiben.
               Vielleicht kannſt Du rathen oder helfen. So ſchreibe ich Dir alſo. Die \textcolor{blue}{Mädels}{}\ledrightnote{{$\rightarrow$}\textcolor{blue}{Marie Glümer}{\newline}{$\rightarrow$}\textcolor{blue}{Auguste Chlum}} hätten Dir
               ohnedies dieſer Tage Mittheilung gemacht.\pend
           \pstart
           Viele treue Grüße! {\\[\baselineskip]}Dein {\\[\baselineskip]}\spacefill\mbox{Paul Goldmann.}\pend
           \leftskip=0em{}\endnumbering\briefempfaengerindex{Schnitzler, Arthur@\textsc{Schnitzler, Arthur}!zzzGoldmann, Paul@\emph{von Paul Goldmann}!1900-05-311@{31. 5. {[}1900{]}}|)be}\mylabel{h}\begin{anhang}\end{anhang}\normalsize

\doendnotes{C}
\bigskip
\vfill

\clearpage

\footnotesize

\lohead{\textsc{register}}

% Definiere theindex-Environment komplett neu ohne reledmac
\makeatletter
\renewenvironment{theindex}{%
  \section*{\indexname}%
  \setlength{\parindent}{0pt}%
  \setlength{\parskip}{0pt plus 0.3pt}%
  \let\item\@idxitem
}{%
  \clearpage
}
\makeatother

\IfFileExists{\jobname-pw.ind}{\input{\jobname-pw.ind}}{}

\end{document}

      