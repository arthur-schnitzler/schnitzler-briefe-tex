%% latex-korrekturansicht-vorspann.tex
%% Vorspann für die Korrekturansicht.
%% Lädt die gemeinsame Datei latex-vorspann.tex mit gesetztem Schalter.

\newif\ifkorrekturansicht
\korrekturansichttrue

\input{../tex-inputs/latex-vorspann}


\renewcommand{\erwaehntePersonen}{Personen: Felix Salten}
\renewcommand{\erwaehnteOrte}{Orte: Edmund-Weiß-Gasse 7, Stratford-upon-Avon, Wien, Österreich}
\renewcommand{\erwaehnteWerke}{Werke: Shakespeare Monument}
\section[ Felix Salten an Arthur Schnitzler, 23. 6. 1906]{Felix Salten an Arthur Schnitzler, 23. 6. 1906}
\nopagebreak\mylabel{v}
\rehead{ }\normalsize\beginnumbering\briefempfaengerindex{Schnitzler, Arthur@\textsc{Schnitzler, Arthur}!zzzSalten, Felix@\emph{von Felix Salten}!1906-06-232@{23. 6. 1906}|(be}
\toendnotes[C]{\smallbreak\pagebreak[2]}\Standort{CUL, Schnitzler, B 89, B 1.}
\physDesc{Bildpostkarte, 150 Zeichen
\newline{}Handschrift: schwarze Tinte, lateinische Kurrent
\newline{}Versand: Stempel: »\nobreak{}\oindex{Stratford-upon-Avon@\textbf{Stratford-upon-Avon}, \emph{P.PPL}|pwk}Stratford\textcolor{gray}{-on-}Avon 2, 23 \textcolor{gray}{JU} 06, 17. PM\nobreak{}«.  
\newline{}Ordnung: mit Bleistift von unbekannter Hand nummeriert: »219« }\pstart{}{\pb}Herrn D\textsuperscript{r} Arthur Schnitzler\pend{}\pstart{}\begin{otherlanguage}{english}\textcolor{pink}{Vienna}{}\ledrightnote{\textcolor{pink}{Wien}}\end{otherlanguage}{ }\textcolor{pink}{Wien}{}\ledrightnote{\textcolor{pink}{Wien}}\pend{}\pstart{}\textcolor{pink}{XVIII. Spöttelgaße 7}{}\ledrightnote{\textcolor{pink}{Edmund-Weiß-Gasse 7}}\pend{}\pstart{}\begin{otherlanguage}{english}\textcolor{pink}{Austria}{}\ledrightnote{\textcolor{pink}{Österreich}}\end{otherlanguage}\pend{}
{\bigskip}
\pstart
           \noindent{}\centering{}{\pb}\textcolor{gray}{\textbf{\textcolor{green}{ALTO-RELIEVO}{}\ledrightnote{\textcolor{green}{Shakespeare Monument}}{ }\begin{otherlanguage}{english}IN NEW PLACE\end{otherlanguage}, \textcolor{pink}{STRATFORD-ON-AVON}{}\ledrightnote{\textcolor{pink}{Stratford-upon-Avon}}. H B {\kaufmannsund} S}}\pend
           
\pstart
           \centering{}{\pb}\textcolor{pink}{Stratford}{}\ledrightnote{\textcolor{pink}{Stratford-upon-Avon}}, 23. VI. 06.\pend
           
\pstart
           Hierher müßte man auf ein paar Tage \uline{allein} gehen.\pend
           
\pstart
           herzlichst {\\[\baselineskip]}Ihr {\\[\baselineskip]}\spacefill\mbox{Salten}\pend
           \leftskip=0em{}\endnumbering\briefempfaengerindex{Schnitzler, Arthur@\textsc{Schnitzler, Arthur}!zzzSalten, Felix@\emph{von Felix Salten}!1906-06-232@{23. 6. 1906}|)be}\mylabel{h}  \normalsize

\doendnotes{C}
\bigskip
\vfill

\clearpage

\footnotesize

\lohead{\textsc{register}}

% Definiere theindex-Environment komplett neu ohne reledmac
\makeatletter
\renewenvironment{theindex}{%
  \section*{\indexname}%
  \setlength{\parindent}{0pt}%
  \setlength{\parskip}{0pt plus 0.3pt}%
  \let\item\@idxitem
}{%
  \clearpage
}
\makeatother

\IfFileExists{\jobname-pw.ind}{\input{\jobname-pw.ind}}{}

\end{document}

      