%% latex-korrekturansicht-vorspann.tex
%% Vorspann für die Korrekturansicht.
%% Lädt die gemeinsame Datei latex-vorspann.tex mit gesetztem Schalter.

\newif\ifkorrekturansicht
\korrekturansichttrue

\input{../tex-inputs/latex-vorspann}


\renewcommand{\erwaehntePersonen}{Personen: Georg Brandes, Maurice Donnay, Theodor Herzl, Heinrich Kanner, Fedor Mamroth, Fritz Mauthner, Felix Philippi, Theodore Rottenberg, Paul Schlenther, Olga Schnitzler, William Shakespeare, Isidor Singer, Elisabeth Steinrück, Hermann Sudermann, Hugo Wittmann}
\renewcommand{\erwaehnteInstitutionen}{Institutionen: Berliner Tageblatt}
\renewcommand{\erwaehnteOrte}{Orte: Berlin, Dessauer Straße, Deutsches Theater Berlin, Deutschland, Frankfurt am Main, Kurhaus Mödling, Wien, Österreich}
\renewcommand{\erwaehnteWerke}{Werke: Berliner Theater. (»Lebendige Stunden« von Arthur Schnitzler.), Burgtheater. (Zum erstenmale: »Es lebe das Leben«, Drama in fünf Acten von Hermann Sudermann.), Es lebe das Leben, Lebendige Stunden. Vier Einakter, Neue Freie Presse, Theater- und Kunstnachrichten. [Burgtheater.] [Es lebe das Leben], William Shakespeare}
\section[ Paul Goldmann an Arthur Schnitzler, 2. 2. {[}1902{]}]{Paul Goldmann an Arthur Schnitzler, 2. 2. {[}1902{]}}
\nopagebreak\mylabel{v}
\rehead{ }\normalsize\beginnumbering\briefempfaengerindex{Schnitzler, Arthur@\textsc{Schnitzler, Arthur}!zzzGoldmann, Paul@\emph{von Paul Goldmann}!1902-02-021@{2. 2. {[}1902{]}}|(be}
\toendnotes[C]{\smallbreak\pagebreak[2]}\Standort{DLA, A:Schnitzler, HS.NZ85.1.3172.}
\physDesc{Brief, 1 Blatt, 4 Seiten
\newline{}Handschrift: blaue Tinte, deutsche Kurrent
\newline{}Beilage: ein handschriftlicher Brief, schwarze Tinte, deutsche
                                 Kurrentschrift, beschnitten und eingeklebt 
\newline{}Schnitzler: 1) mit Bleistift das Jahr »{[}1{]}902« vermerkt  2) mit rotem Buntstift fünf Unterstreichungen}\toendnotes[C]{\smallbreak}
\pstart
           \noindent{}\raggedleft{}{\pb}\textcolor{pink}{\textcolor{gray}{\textbf{DESSAUERSTRASSE 19}}}{}\ledrightnote{\textcolor{pink}{Dessauer Straße}}\pend
           
\pstart
           \textcolor{pink}{Berlin}{}\ledrightnote{\textcolor{pink}{Berlin}}, 2. Februar.\pend
           
\pstart{}Mein lieber Freund,\pend
\pstart
           Die Regelung der \label{K_L03196-1v}\edtext{Landaufenthalts-Frage}{\lemma{\textnormal{\emph{Landaufenthalts-Frage}}}\Cendnote{\textnormal{siehe Paul Goldmann an Arthur Schnitzler, 14. 1. [1902]}}}\label{K_L03196-1h} freut mich ſehr. »\textcolor{pink}{Kurhaus in Mödling}{}\ledrightnote{\textcolor{pink}{Kurhaus Mödling}}«
               klingt vielverſprechend. Ich wünſchte, ich könnte auch hin. Ich bin ſchwer
               überarbeitet und leide\strikeout{t} ſeit einer Woche
               ununterbrochen an Kopfſchmerzen.\pend
           
\pstart
           \substVorne{}\textsuperscript{\textcolor{gray}{V}}\substDazwischen{}D\substHinten{}ie Vorſtellungen von »\label{K_L03196-3v}\edtext{\textcolor{green}{Lebendige Stunden}{}\ledrightnote{\textcolor{green}{Lebendige Stunden. Vier Einakter}}}{\lemma{\textnormal{\emph{Lebendige Stunden}}}\Cendnote{\textnormal{im \textcolor{pink}{Deutschen Theater Berlin}}}}\label{K_L03196-3h}« ſollen ſtets ausverkauft ſein. Ich freue mich ſehr darüber, daß Dir Deine
               Arbeit auch Geld bringt. Du kannſt es brauchen. Wie hat ſich \textsc{\textcolor{blue}{Schlenther}{}\ledrightnote{\textcolor{blue}{Paul Schlenther}}} verhalten?\pend
           
\pstart
           \textsc{\textcolor{blue}{Sudermann}{}\ledrightnote{\textcolor{blue}{Hermann Sudermann}}s} neues \textcolor{green}{Stück}{}\ledrightnote{{$\rightarrow$}\textcolor{green}{Es lebe das Leben}} iſt elend. {\pb}\introOben{}In der Art von \textsc{\textcolor{blue}{Philippi}{}\ledrightnote{\textcolor{blue}{Felix Philippi}}}. Nur macht es \textsc{\textcolor{blue}{Philippi}{}\ledrightnote{\textcolor{blue}{Felix Philippi}}} beſſer.\introOben{} Ich konnte nur ganz kurz darüber \label{K_L03196-5v}\edtext{\textcolor{green}{telegraphiren}{}\ledrightnote{{$\rightarrow$}\textcolor{green}{Theater- und Kunstnachrichten. [Burgtheater.] [Es lebe das Leben]}}}{\lemma{\textnormal{\emph{telegraphiren}}}\Cendnote{\textnormal{[O. V.] [=\textcolor{blue}{Paul Goldmann}]: \emph{\textcolor{green}{Theater- und Kunstnachrichten.
                        [Burgtheater.]}}. In: \emph{\textcolor{green}{Neue Freie
                        Presse}}, Nr. 13.455, 8. 2. 1902,
                     Morgenblatt, S. 7.}}}\label{K_L03196-5h}, weil \strikeout{\textcolor{gray}{d}} die Vorſtellung erſt nach elf aus war, und ein Feuilleton darüber
               zu ſchreiben, wurde mir telegraphiſch unterſagt. Herrn \textsc{\textcolor{blue}{Wittmann}{}\ledrightnote{\textcolor{blue}{Hugo Wittmann}}}s kritiſcher \label{K_L03196-76v}\edtext{\textcolor{green}{Würdigung}{}\ledrightnote{{$\rightarrow$}\textcolor{green}{Burgtheater. (Zum erstenmale: »Es lebe das Leben«, Drama in fünf Acten von Hermann Sudermann.)}}}{\lemma{\textnormal{\emph{Würdigung}}}\Cendnote{\textnormal{\textcolor{blue}{W.} [=\textcolor{blue}{Hugo Wittmann}]: \emph{\textcolor{green}{Burgtheater. (Zum erstenmale: »Es lebe das Leben«, Drama in fünf Acten von
                        Hermann Sudermann.)}}. In: \emph{\textcolor{green}{Neue Freie
                        Presse}}, Nr. 13.456, 9. 2. 1902,
                     Morgenblatt, S. 1–3.}}}\label{K_L03196-76h} darf ein armer Reporter wie ich bin, nicht
               vorgreifen.\pend
           
\pstart
           Dank für die Bücherempfehlungen. Ich leſe nach wie vor mit Genuß die \textcolor{green}{\textsc{\textcolor{blue}{Shakespeare}{}\ledrightnote{\textcolor{blue}{William Shakespeare}}}-Biographie}{}\ledrightnote{\textcolor{green}{William Shakespeare}} von \textsc{\textcolor{blue}{Brandes}{}\ledrightnote{\textcolor{blue}{Georg Brandes}}}.\pend
           
\pstart
           \textsc{\textcolor{blue}{Brandes}{}\ledrightnote{\textcolor{blue}{Georg Brandes}}} iſt \textcolor{pink}{hier}{}\ledrightnote{{$\rightarrow$}\textcolor{pink}{Berlin}}, läßt ſich aber
               bei mir nicht ſehen. Übermorgen feiert \introOben{}er\introOben{} ſeinen 60. Geburtstag. Vergiß nicht, ihm zu \label{K_L03196-11v}\edtext{gratuliren}{\lemma{\textnormal{\emph{gratuliren}}}\Cendnote{\textnormal{kein entsprechendes Korrespondenzstück
               überliefert}}}\label{K_L03196-11h}.\pend
           
\pstart
           {\pb}Mit \textsc{\textcolor{blue}{Singer}{}\ledrightnote{\textcolor{blue}{Isidor Singer}}} ſprich’, bitte, einſtweilen nicht. \textsc{\textcolor{blue}{Kanner}{}\ledrightnote{\textcolor{blue}{Heinrich Kanner}}} ſoll bald wieder \textcolor{pink}{hier}{}\ledrightnote{{$\rightarrow$}\textcolor{pink}{Berlin}}herkommen, und ich werde verſuchen, ihn \label{K_L03196-13v}\edtext{zur Rede zu ſtellen}{\lemma{\textnormal{\emph{zur Rede zu ſtellen}}}\Cendnote{\textnormal{siehe Paul Goldmann an Arthur Schnitzler, 25. 1. [1902]}}}\label{K_L03196-13h}.\pend
           
\pstart
           An \textsc{\textcolor{blue}{Mauthner}{}\ledrightnote{\textcolor{blue}{Fritz Mauthner}}s} Stelle ſoll mein \label{K_L03196-55v}\edtext{\textcolor{blue}{Onkel}{}\ledrightnote{{$\rightarrow$}\textcolor{blue}{Fedor Mamroth}} zum \textcolor{brown}{Berliner Tageblatt}{}\ledrightnote{\textcolor{brown}{Berliner Tageblatt}}}{\lemma{\textnormal{\emph{Onkel … Tageblatt}}}\Cendnote{\textnormal{nicht belegbar}}}\label{K_L03196-55h} kommen. An mich
               denkt ſelbſtverſtändlich Niemand. Ich bin nicht literariſch.\pend
           
\pstart
           Anbei der \label{K_L03196-67v}\edtext{Brief von \textsc{\textcolor{blue}{Herzl}{}\ledrightnote{\textcolor{blue}{Theodor Herzl}}}}{\lemma{\textnormal{\emph{Brief von Herzl}}}\Cendnote{\textnormal{siehe Paul Goldmann an Arthur Schnitzler, 25. 1. [1902]}}}\label{K_L03196-67h}. Sende ihn mir, bitte, gelegentlich zurück.\pend
           
\pstart
           \strikeout{»\textcolor{blue}{Sie}{}\ledrightnote{\textcolor{blue}{Theodore Rottenberg}}}{ }\label{K_L03196-77v}\edtext{»\textcolor{blue}{Sie}{}\ledrightnote{{$\rightarrow$}\textcolor{blue}{Theodore Rottenberg}}«}{\lemma{\textnormal{\emph{»Sie«}}}\Cendnote{\textnormal{mit großer
                  Wahrscheinlichkeit \textcolor{blue}{Theodore Rottenberg}, mit
                  der \textcolor{blue}{Goldmann} seit 1899 intim war, siehe Paul Goldmann an Arthur Schnitzler, 8. 10. [1899]}}}\label{K_L03196-77h} (aus \textcolor{pink}{Frankfurt}{}\ledrightnote{\textcolor{pink}{Frankfurt am Main}}) ſchreibt Folgendes\substVorne{}\textsuperscript{,}\substDazwischen{}:\substHinten{}\pend
           {\bigskip}
\pstart
           \noindent{}{\pb}{[}hs. Rottenberg:{]} Dein \textcolor{green}{\textsc{Schnitzler}-Feuilleton}{}\ledrightnote{{$\rightarrow$}\textcolor{green}{Berliner Theater. (»Lebendige Stunden« von Arthur Schnitzler.)}}, womit er doch wohl
               einverſtanden ſein wird, iſt fein, fein, mein Liebſter. Nur die \label{K_L03196-132v}\edtext{Epiſoden-Sache}{\lemma{\textnormal{\emph{Epiſoden-Sache}}}\Cendnote{\textnormal{siehe Paul Goldmann an Arthur Schnitzler, 25. 1. [1902]}}}\label{K_L03196-132h} mißfällt mir. Es \uline{giebt} Männer {\kaufmannsund} viele tauſend Frauen, die von der Liebe leben. Bei \textsc{Schnitzler} wird Kunſt {\kaufmannsund} Liebe
               ſicherlich i{\geminationm}er eins bleiben; halb Frauenpoſe {\kaufmannsund} halb \textcolor{pink}{Öſterreich}{}\ledrightnote{\textcolor{pink}{Österreich}}er iſt er nun einmal. Die wahre, erhabene \label{K_L03196-87v}\edtext{{[}»{]}\textcolor{pink}{deutſch}{}\ledrightnote{{$\rightarrow$}\textcolor{pink}{Deutschland}}e Männlichkeit«}{\lemma{\textnormal{\emph{»deutſche Männlichkeit«}}}\Cendnote{\textnormal{Bezug auf die erwähnte
                     »Epiſoden-Sache«, denn \textcolor{blue}{Schnitzler} habe sich vom Thema der Liebe loszulösen und »\textcolor{green}{das starke Werk seiner Mannesjahre}« zu
                  schreiben (\textcolor{blue}{Paul Goldmann}: \emph{\textcolor{green}{Berliner Theater. (»Lebendige Stunden« von Arthur
                        Schnitzler.)}}. In: \emph{\textcolor{green}{Neue Freie
                        Presse}}, Nr. 13.438, 22. 1. 1902,
                     Morgenblatt, S. 1–4, hier: S. 4)}}}\label{K_L03196-87h} kann ich mir von ihm aber eben
               ſo wenig denken wie von \textcolor{blue}{M. \textsc{Donnay}}{}\ledrightnote{\textcolor{blue}{Maurice Donnay}} z. B.\pend
           {\bigskip}
\pstart
           {[}hs. Goldmann:{]} Viele treue Grüße, mein lieber Freund, Dir und den \textcolor{blue}{Mädels}{}\ledrightnote{{$\rightarrow$}\textcolor{blue}{Olga Schnitzler}{\newline}{$\rightarrow$}\textcolor{blue}{Elisabeth Steinrück}}. {\\[\baselineskip]}Dein {\\[\baselineskip]}\spacefill\mbox{Paul Goldmnn}\pend
           \leftskip=0em{}\endnumbering\briefempfaengerindex{Schnitzler, Arthur@\textsc{Schnitzler, Arthur}!zzzGoldmann, Paul@\emph{von Paul Goldmann}!1902-02-021@{2. 2. {[}1902{]}}|)be}\mylabel{h}
\begin{anhang}
\end{anhang}\normalsize

\doendnotes{C}
\bigskip
\vfill

\clearpage

\footnotesize

\lohead{\textsc{register}}

% Definiere theindex-Environment komplett neu ohne reledmac
\makeatletter
\renewenvironment{theindex}{%
  \section*{\indexname}%
  \setlength{\parindent}{0pt}%
  \setlength{\parskip}{0pt plus 0.3pt}%
  \let\item\@idxitem
}{%
  \clearpage
}
\makeatother

\IfFileExists{\jobname-pw.ind}{\input{\jobname-pw.ind}}{}

\end{document}

      