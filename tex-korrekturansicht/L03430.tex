%% latex-korrekturansicht-vorspann.tex
%% Vorspann für die Korrekturansicht.
%% Lädt die gemeinsame Datei latex-vorspann.tex mit gesetztem Schalter.

\newif\ifkorrekturansicht
\korrekturansichttrue

\input{../tex-inputs/latex-vorspann}


\renewcommand{\erwaehntePersonen}{Personen: Ludwig Bauer, Ernst Egon Friedegg, Julius von Gans-Ludassy, Max Liebermann, Anna Katharina Rehmann, Felix Salten, Ottilie Salten, Paul Salten, Ignaz Salzmann, Agathe Salzmann, Marie Salzmann, Katharina Salzmann, Philipp Salzmann, Olga Schnitzler, Heinrich Schnitzler, Richard Strauss}
\renewcommand{\erwaehnteInstitutionen}{Institutionen: Die Zeit, Morgen. Wochenschrift für deutsche Kultur}
\renewcommand{\erwaehnteOrte}{Orte: Bansin, Berlin, Cambridge, Cherbourg-Octeville, Deutschland, Dornbach, England, Kopenhagen, London, Marienlyst, Nordsee, Plymouth, Preußen, Seestraße, Southampton, Stratford-upon-Avon, Wien, Österreich}
\renewcommand{\erwaehnteWerke}{Werke: Morgen. Wochenschrift für deutsche Kultur}
\section[ Felix Salten an Arthur Schnitzler, 6. 7. 1906]{Felix Salten an Arthur Schnitzler, 6. 7. 1906}
\nopagebreak\mylabel{v}
\rehead{ }\normalsize\beginnumbering\briefempfaengerindex{Schnitzler, Arthur@\textsc{Schnitzler, Arthur}!zzzSalten, Felix@\emph{von Felix Salten}!1906-07-061@{6. 7. 1906}|(be}
\toendnotes[C]{\smallbreak\pagebreak[2]}\Standort{CUL, Schnitzler, B 89, B 1.}
\physDesc{Brief, 1 Blatt, 4 Seiten, 4140 Zeichen
\newline{}Handschrift: schwarze Tinte, lateinische Kurrent
\newline{}Schnitzler: mit rotem Buntstift fünf Unterstreichungen 
\newline{}Ordnung: mit Bleistift von unbekannter Hand nummeriert: »221« }\toendnotes[C]{\smallbreak}
\pstart
           \raggedleft{}{\pb}\textcolor{pink}{Berlin}{}\ledrightnote{\textcolor{pink}{Berlin}}, 6. 7. 06.\pend
           
\pstart
           Lieber, wie Schade, dass Sie gerade jetzt \label{K_L03430-1v}\edtext{durch \textcolor{pink}{Berlin}{}\ledrightnote{\textcolor{pink}{Berlin}} kamen}{\lemma{\textnormal{\emph{durch Berlin kamen}}}\Cendnote{\textnormal{auf dem Weg nach \textcolor{pink}{Marienlyst}, siehe A. S.: \emph{Tagebuch}, 26. 6. 1906}}}\label{K_L03430-1h}, während meiner Abwesenheit. Man hätte vielleicht doch eine Stunde gehabt, um
               sich auszusprechen. Schreiben ist in manchen Fällen so schwer. Was ich jetzt, in der
               nächsten Zeit, beginne, liegt noch im Halbdunkel; und was ich Ihnen davon mitteile,
               ist – einstweilen – nur für Sie. In \textcolor{pink}{Berlin}{}\ledrightnote{\textcolor{pink}{Berlin}} will
               ich nicht bleiben; kann es ehrlicherweise garnicht tun und spüre, dass ein Bruch in
               mein Leben käme, wollte ich versuchen\textcolor{gray}{,} mich zu zwingen. \label{K_L03430-2v}\edtext{»\textcolor{brown}{Die
                  Zeit}{}\ledrightnote{\textcolor{brown}{Die Zeit}}« will mich wieder haben}{\lemma{\textnormal{\emph{»Die … haben}}}\Cendnote{\textnormal{\textcolor{blue}{Salten} arbeitete ab Oktober 1906 wieder für \emph{\textcolor{brown}{Die
                  Zeit}}.}}}\label{K_L03430-2h}, und ich bin gerne geneigt, abzuschließen. Dabei bietet sich
               hier der Plan zu einer \label{K_L03430-3v}\edtext{\textcolor{brown}{\textcolor{green}{Wochenschrift}{}\ledrightnote{{$\rightarrow$}\textcolor{green}{Morgen. Wochenschrift für deutsche Kultur}}}{}\ledrightnote{\textcolor{brown}{Morgen. Wochenschrift für deutsche Kultur}}}{\lemma{\textnormal{\emph{Wochenschrift}}}\Cendnote{\textnormal{Der \emph{\textcolor{brown}{\emph{\textcolor{green}{Morgen}}}}?}}}\label{K_L03430-3h}, die ich mit \textcolor{blue}{Max Liebermann}{}\ledrightnote{\textcolor{blue}{Max Liebermann}}
               und \textcolor{blue}{Rich. Strauß}{}\ledrightnote{\textcolor{blue}{Richard Strauss}} zusammen herausgeben, und
               allein leiten soll. Ihr Bestand ist für drei Jahre garantirt. Honorarbudget, ohne
               meine Gage, nur für Mitarbeiter 1000 Mark pro Nummer. Sie soll das Blatt der
               »anständigen Leute« werden, der Besten, ganz einfach. Ein kleiner, exclusiver,
               ständiger Mitarbeiterkreis. Ich hätte ausser der Gage noch einen Besitzanteil. Jetzt
               überleg ich mir’s, ob ich die Sache nicht von \textcolor{pink}{Wien}{}\ledrightnote{\textcolor{pink}{Wien}}
               aus machen kann. Technisch gehts ganz gut. Die Schwierigkeiten, die sich freilich
               ergeben, würden reichlich durch manche Vorteile, {\pb}die sich dran knüpfen,
               aufgewogen. Ich könnte z. B. die \textcolor{pink}{Berlin}{}\ledrightnote{\textcolor{pink}{Berlin}}er u. \textcolor{pink}{Wien}{}\ledrightnote{\textcolor{pink}{Wien}}er Theater zusammen überschauen und besprechen.
               Würde bei allen wichtigen Aufführungen (an die Premiere bin ich ja nicht gebunden) in
                  \textcolor{pink}{Berlin}{}\ledrightnote{\textcolor{pink}{Berlin}} sein. Könnte \textcolor{pink}{deutsch}{}\ledrightnote{{$\rightarrow$}\textcolor{pink}{Deutschland}}e und \textcolor{pink}{österreich}{}\ledrightnote{{$\rightarrow$}\textcolor{pink}{Österreich}}ische Kultur- und
               Gesellschaftskritik zusammen treiben, was dem \textcolor{brown}{\textcolor{green}{Blatte}{}\ledrightnote{{$\rightarrow$}\textcolor{green}{Morgen. Wochenschrift für deutsche Kultur}}}{}\ledrightnote{\textcolor{brown}{Morgen. Wochenschrift für deutsche Kultur}} ebenso wie meiner Stellung etwas ganz Besonderes gäbe. Und wenn – binnen Kurzem
               – ein Thronwechsel in \textcolor{pink}{Österreich}{}\ledrightnote{\textcolor{pink}{Österreich}} alles
               Interesse erregt, wär’s für eine solche Wochenschrift eine ganz einzige Conjunctur.
               Ganz abgesehen davon, dass ich, als in \textcolor{pink}{Wien}{}\ledrightnote{\textcolor{pink}{Wien}}
               lebend, nicht mehr unter der Fuchtel der politischen Polizei in \textcolor{pink}{Preussen}{}\ledrightnote{\textcolor{pink}{Preußen}}, die ärger ist als man glaubt, und nicht mehr unter
               der Ausweisungsgefahr leben müßte.\pend
           
\pstart
           Glauben Sie, dass mein Wiedereintritt in die die »\textcolor{brown}{Zeit}{}\ledrightnote{\textcolor{brown}{Die Zeit}}« für mich gut wäre? Dass man mich dort braucht, sehe ich, und dass die
                  »\textcolor{brown}{Zeit}{}\ledrightnote{\textcolor{brown}{Die Zeit}}« jetzt ihre literarische Stimme
               eingebüßt hat, kann ich wol, ohne Ihrem Freund \label{K_L03430-4v}\edtext{\textcolor{blue}{Bauer}{}\ledrightnote{\textcolor{blue}{Ludwig Bauer}}}{\lemma{\textnormal{\emph{Bauer}}}\Cendnote{\textnormal{\textcolor{blue}{Salten}s Nachfolger, vgl. A. S.: \emph{Tagebuch}, 15. 2. 1906}}}\label{K_L03430-4h} allzu unrecht zu thun, sagen.\pend
           
\pstart
           Von sonstigen Dingen: dass Herr \textcolor{blue}{Friedegg}{}\ledrightnote{\textcolor{blue}{Ernst Egon Friedegg}}
               knapp vor der Verhandlung eine umfassende Ehrenerklärung abgegeben hat. Dass der
                  \label{K_L03430-5v}\edtext{\textcolor{blue}{Ludassy}{}\ledrightnote{\textcolor{blue}{Julius von Gans-Ludassy}}-Prozess}{\lemma{\textnormal{\emph{Ludassy-Prozess}}}\Cendnote{\textnormal{siehe Felix Salten an Arthur Schnitzler, 9. 3. 1906}}}\label{K_L03430-5h} vertagt ist. Dass mein \textcolor{blue}{Bruder}{}\ledrightnote{{$\rightarrow$}\textcolor{blue}{Ignaz Salzmann}} leider weit davon entfernt ist, ein Millionär {\pb}zu sein, dass er aber freilich,
               gottseidank, ein so ahnsehnliches Geld verdient hat, dass ich – hoffentlich – für
               alle Zukunft der Sorge um ihn und um meine Familie enthoben bin. \uline{Wie} viel er besitzt, weiß ich nicht, weiß nur, dass er mit seiner \textcolor{blue}{Frau}{}\ledrightnote{{$\rightarrow$}\textcolor{blue}{Agathe Salzmann}} sechs Wochen in \textcolor{pink}{England}{}\ledrightnote{\textcolor{pink}{England}} war, ihr um 20.000 Kronen Schmuck gekauft
               hat, für meine \textcolor{blue}{Mama}{}\ledrightnote{{$\rightarrow$}\textcolor{blue}{Marie Salzmann}} alles
               Erdenkliche tut, und meiner sel. \textcolor{blue}{Schwester}{}\ledrightnote{{$\rightarrow$}\textcolor{blue}{Katharina Salzmann}} wie meinem \textcolor{blue}{Papa}{}\ledrightnote{{$\rightarrow$}\textcolor{blue}{Philipp Salzmann}} ein kostbares Grabmonument hat errichten laßen, dass er bei alledem
               doch weit von einer Million entfernt, und bei alledem von seinem Glück geradezu
               melancholisch geworden ist, weil der \textcolor{blue}{Papa}{}\ledrightnote{{$\rightarrow$}\textcolor{blue}{Philipp Salzmann}} jahrelang darauf gewartet hat, und – genau \label{K_L03430-6v}\edtext{zwei Wochen zu früh starb}{\lemma{\textnormal{\emph{zwei … starb}}}\Cendnote{\textnormal{\textcolor{blue}{Philip Salzmann} war am 2. 4. 1905 verstorben.}}}\label{K_L03430-6h}.\pend
           
\pstart
           Ich hatte im \label{K_L03430-7v}\edtext{Mai eine heftige Nierenkolik}{\lemma{\textnormal{\emph{Mai … Nierenkolik}}}\Cendnote{\textnormal{siehe Felix Salten u. a. an Arthur Schnitzler, 4. 6. 1906}}}\label{K_L03430-7h}. Zweimal an zwei aufeinanderfolgenden Tagen. Bekam zweimal Morphium,
               beidemale mit einer unreinen Spritze oder mit einer mangelhaft gekochten Lösung.
               Musste dann fünf Tage lang rasende Schmerzen leiden, und am Ende froh sein, dass
               nicht Schlimmeres geschah. Dabei weiß ich trotz zweier Ärzte nicht, ob ich den
               Nierenstein habe, oder ob es nur eine akute Sache gewesen ist.\pend
           
\pstart
           \textcolor{blue}{Otti}{}\ledrightnote{\textcolor{blue}{Ottilie Salten}} und die \textcolor{blue}{Kinder}{}\ledrightnote{{$\rightarrow$}\textcolor{blue}{Anna Katharina Rehmann}{\newline}{$\rightarrow$}\textcolor{blue}{Paul Salten}} sind wol und frisch in \textcolor{pink}{Bansin}{}\ledrightnote{\textcolor{pink}{Bansin}}, dessen sonstige Gesellschaft mir als
               der Ausbund alles Grausenhaften geschildert wird. Ich gehe am 15. Juli zu ihnen. Dann wollen wir einmal, vielleicht sogar mit den \textcolor{blue}{Kindern}{}\ledrightnote{{$\rightarrow$}\textcolor{blue}{Anna Katharina Rehmann}{\newline}{$\rightarrow$}\textcolor{blue}{Paul Salten}}, per Schiff
                  \label{K_L03430-8v}\edtext{nach \textcolor{pink}{Kopenhagen}{}\ledrightnote{\textcolor{pink}{Kopenhagen}}, wo wir uns sehen {\pb}könnten}{\lemma{\textnormal{\emph{nach … könnten}}}\Cendnote{\textnormal{Sie sahen sich nicht in \textcolor{pink}{Kopenhagen}, aber am 2. 8. 1906 in \textcolor{pink}{Marienlyst}.}}}\label{K_L03430-8h}. An dem Ausflug an die \textcolor{pink}{Nordsee}{}\ledrightnote{\textcolor{pink}{Nordsee}} werd ich wol nicht teil nehmen. Ich will, wenn’s geht, in \textcolor{pink}{Bansin}{}\ledrightnote{\textcolor{pink}{Bansin}} noch arbeiten. Die vierzehn Tage \textcolor{pink}{London}{}\ledrightnote{\textcolor{pink}{London}} – \textcolor{pink}{Stratford}{}\ledrightnote{\textcolor{pink}{Stratford-upon-Avon}} – \textcolor{pink}{Cambridge}{}\ledrightnote{\textcolor{pink}{Cambridge}} waren sehr schön.
               Die Seefahrt – hin nach \textcolor{pink}{Southampton}{}\ledrightnote{\textcolor{pink}{Southampton}}, zurück von
                  \textcolor{pink}{Plymouth}{}\ledrightnote{\textcolor{pink}{Plymouth}} über \textcolor{pink}{Cherbourg}{}\ledrightnote{\textcolor{pink}{Cherbourg-Octeville}} – wundervoll. Die \textcolor{pink}{engl}{}\ledrightnote{{$\rightarrow$}\textcolor{pink}{England}}ische Landschaft ist beinahe überall so
               schön wie \textcolor{pink}{Dornbach}{}\ledrightnote{\textcolor{pink}{Dornbach}}.\pend
           
\pstart
           Schreiben Sie mir bis zum 14. nach \textcolor{pink}{Berlin}{}\ledrightnote{\textcolor{pink}{Berlin}}. Von da ab Seebad \textcolor{pink}{Bansin}{}\ledrightnote{\textcolor{pink}{Bansin}}, \textcolor{pink}{Seestraße 5}{}\ledrightnote{\textcolor{pink}{Seestraße}}.\pend
           
\pstart
           Viele herzliche Grüße Ihnen, Frau \textcolor{blue}{Olga}{}\ledrightnote{\textcolor{blue}{Olga Schnitzler}} und \textcolor{blue}{Heini}{}\ledrightnote{\textcolor{blue}{Heinrich Schnitzler}}.\pend
           \pstart Ihr \spacefill\mbox{Salten}\pend{}\endnumbering\briefempfaengerindex{Schnitzler, Arthur@\textsc{Schnitzler, Arthur}!zzzSalten, Felix@\emph{von Felix Salten}!1906-07-061@{6. 7. 1906}|)be}\mylabel{h}  \normalsize

\doendnotes{C}
\bigskip
\vfill

\clearpage

\footnotesize

\lohead{\textsc{register}}

% Definiere theindex-Environment komplett neu ohne reledmac
\makeatletter
\renewenvironment{theindex}{%
  \section*{\indexname}%
  \setlength{\parindent}{0pt}%
  \setlength{\parskip}{0pt plus 0.3pt}%
  \let\item\@idxitem
}{%
  \clearpage
}
\makeatother

\IfFileExists{\jobname-pw.ind}{\input{\jobname-pw.ind}}{}

\end{document}

      