%% latex-korrekturansicht-vorspann.tex
%% Vorspann für die Korrekturansicht.
%% Lädt die gemeinsame Datei latex-vorspann.tex mit gesetztem Schalter.

\newif\ifkorrekturansicht
\korrekturansichttrue

\input{../tex-inputs/latex-vorspann}


               \section[Paul Goldmann an Arthur Schnitzler, Paul Goldmann an Arthur Schnitzler, 4. 4. {[}1897{]}]{ Paul Goldmann an Arthur Schnitzler, 4. 4. {[}1897{]}}\nopagebreak\mylabel{v}\rehead{ }\normalsize\beginnumbering\briefempfaengerindex{Schnitzler, Arthur@\textsc{Schnitzler, Arthur}!zzzGoldmann, Paul@\emph{von Paul Goldmann}!1897-04-041@{4. 4. {[}1897{]}}|(be} \toendnotes[C]{\smallbreak\pagebreak[2]} \Standort{DLA, A:Schnitzler, HS.NZ85.1.3167.}
\physDesc{Brief, 1 Blatt, 4 Seiten
\newline{}Handschrift: blaue Tinte, deutsche Kurrent
\newline{}Schnitzler: mit Bleistift das Jahr »97« vermerkt }\toendnotes[C]{\smallbreak}\pstart
           \noindent{}{\pb}\textcolor{gray}{\textbf{\textbf{\textcolor{brown}{Frankfurter Zeitung}{}\ledrightnote{\textcolor{brown}{Frankfurter Zeitung}}}}}\pend
           \pstart
           \textcolor{gray}{\textbf{(\textcolor{brown}{\begin{otherlanguage}{french}Gazette de Francfort\end{otherlanguage}}{}\ledrightnote{\textcolor{brown}{Frankfurter Zeitung}}).}}\pend
           \pstart
           \textcolor{gray}{\textbf{\textbf{\begin{otherlanguage}{french}Fondateur M.\end{otherlanguage}{ }\textcolor{blue}{L. Sonnemann}{}\ledrightnote{\textcolor{blue}{Leopold Sonnemann}}.}}}\pend
           \pstart
           \begin{otherlanguage}{french}\textcolor{gray}{\textbf{Journal politique, financier,}}\end{otherlanguage}\pend
           \pstart
           \begin{otherlanguage}{french}\textcolor{gray}{\textbf{commercial et littéraire.}}\end{otherlanguage}\pend
           \pstart
           \begin{otherlanguage}{french}\textcolor{gray}{\textbf{\textbf{Paraissant trois fois par jour.}}}\end{otherlanguage}\hfill \textsc{\textcolor{pink}{Paris}{}\ledrightnote{\textcolor{pink}{Paris}}}, 4. April.\pend
           \pstart
           \begin{otherlanguage}{french}\textcolor{gray}{\textbf{\textbf{Bureau à \textcolor{pink}{Paris}{}\ledrightnote{\textcolor{pink}{Paris}}}}}\end{otherlanguage}\pend
           \pstart
           \begin{otherlanguage}{french}\textcolor{gray}{\textbf{\textbf{\textcolor{pink}{24. Rue Feydeau}{}\ledrightnote{\textcolor{pink}{rue Feydeau}}.}}}\end{otherlanguage}\pend
           \pstart\center{}Mein lieber Freund,\pend\pstart
           Hoffentlich erreichen dieſe Zeilen Dich noch. Sie ſollen Dir nichts ſagen, als daß
               ich Dir von ganzem Herzen glückliche \label{K_L02807-1v}\edtext{Reiſe}{\lemma{\textnormal{\emph{Reiſe}}}\Cendnote{\textnormal{\textcolor{blue}{Schnitzler} reiste am 7. 4. 1897 nach \textcolor{pink}{München} ab, am 10. 4. 1897 ging es
                  für ihn weiter nach \textcolor{pink}{Zürich}. Von 12. 4. 1897 bis 24. 5. 1897 war er in
                     \textcolor{pink}{Paris}, dann bis 1. 6. 1897 in \textcolor{pink}{London}. Am 2. 6. 1897 kam er wieder nach \textcolor{pink}{Wien}.}}}\label{K_L02807-1h} wünſche und daß ich mich unendlich auf das Wiederſehen mit Dir
               freue (obwohl es nicht nöthig iſt, das zu ſagen) {\dotsfour}\pend
           \pstart
           Ich denke ans \textsc{\textcolor{pink}{Hotel de l’Athénée}{}\ledrightnote{\textcolor{pink}{Hotel de l’Athénée}}}. Im Centrum der \textcolor{pink}{Stadt}{}\ledrightnote{→\textcolor{pink}{Paris}},
               hinter der \textcolor{pink}{Oper}{}\ledrightnote{→\textcolor{pink}{Opéra Garnier}} gelegen.
               Größtentheils \textcolor{pink}{engl}{}\ledrightnote{→\textcolor{pink}{England}}iſche und \textcolor{pink}{fran}{}\ledrightnote{→\textcolor{pink}{Frankreich}}zöſiſche Kundſchaft. Nie hat
               ſich noch ein {\pb}\strikeout{\textcolor{pink}{Englä}{}\ledrightnote{→\textcolor{pink}{England}}\textcolor{gray}{×}}{ }\textcolor{pink}{Öſterreich}{}\ledrightnote{\textcolor{pink}{Österreich}}er dorthin verirrt. Preis: ein Zimmer
               im vierten Stock (\label{K_L02807-3v}\edtext{\begin{otherlanguage}{french}\textsc{Ascenseur}\end{otherlanguage}}{\lemma{\textnormal{\emph{Ascenseur}}}\Cendnote{\textnormal{französisch: Aufzug}}}\label{K_L02807-3h}) 7 \textsc{Francs}, in einem niedrigen Stockwerk natürlich theurer.
               Gegenwärtig iſt das \textcolor{pink}{Haus}{}\ledrightnote{→\textcolor{pink}{Hotel de l’Athénée}}
               (welches als vortrefflich bekannt iſt) bis unters Dach gefüllt. Man hat mir aber
               verſprochen, daß, wenn ich drei Tage vorher Deine Ankunft melde, man mir zwei Zimmer
               reſerviren wird. Im \textcolor{pink}{Centrum}{}\ledrightnote{→\textcolor{pink}{Paris}}
               mußt Du wohnen, ich hab’ mir das überlegt: Du verlierſt ſonſt zuviel Zeit. Auch
               könnte \uline{ich} Dich ſonſt zu ſelten ſehen.\pend
           \pstart
           {\pb}Wenn ich das Reiſegeld habe (was zurſtunde mehr als
               fraglich iſt) und wenn im \label{K_L02807-23v}\edtext{Orient}{\lemma{\textnormal{\emph{Orient}}}\Cendnote{\textnormal{siehe Paul Goldmann an Arthur Schnitzler, 11. 3. [1897]}}}\label{K_L02807-23h} kein Krieg ausbricht, fahre ich nach \textcolor{pink}{Frankfurt}{}\ledrightnote{\textcolor{pink}{Frankfurt am Main}} um den 19. April herum und
               bleibe 10 bis 14 Tage.\pend
           \pstart
           Damenſtrohhüte? Wird \strikeout{de} das \textcolor{blue}{Fräulein}{}\ledrightnote{→\textcolor{blue}{Marie Reinhard}} im \textcolor{pink}{\textsc{Louvre}}{}\ledrightnote{\textcolor{pink}{Louvre}} oder \label{K_L02807-45v}\edtext{\textsc{\textcolor{pink}{Bon Marché}{}\ledrightnote{\textcolor{pink}{Le Bon Marché}}}}{\lemma{\textnormal{\emph{Bon Marché}}}\Cendnote{\textnormal{\textcolor{pink}{Kaufhaus im \textcolor{pink}{7. Arrondissement}}}}}\label{K_L02807-45h} kaufen. \strikeout{A} Außerdem kann ſie ſonſt zwiſchen
               tauſend und einigen Geſchäften wählen.\pend
           \pstart
           Cylinder? Den ſollſt Du gewiß mitbringen,
                wenn Du hier Beſuche machen willſt. Wenn {\pb}Du ihn nicht mitbringſt, ſo ſchadet es auch nichts.\pend
           \pstart
           Grüß’ Dich Gott, liebſter Freund, ſchreib’ mir ein Wort \strikeout{\textcolor{gray}{×}} von unterwegs und komme ſo bald als möglich!\pend
           \pstart
           Dein treuer {\\[\baselineskip]}\spacefill\mbox{Paul Goldmann.}\pend
           \leftskip=0em{}\pstart
           \noindent{}Habe natürlich keinem Menſchen eine Sylbe von Deiner bevorſtehenden Ankunft
                  geſagt.\pend
           \endnumbering\briefempfaengerindex{Schnitzler, Arthur@\textsc{Schnitzler, Arthur}!zzzGoldmann, Paul@\emph{von Paul Goldmann}!1897-04-041@{4. 4. {[}1897{]}}|)be}\mylabel{h}  \normalsize

\doendnotes{C}
\bigskip
\vfill

\clearpage

\footnotesize

\lohead{\textsc{register}}

% Definiere theindex-Environment komplett neu ohne reledmac
\makeatletter
\renewenvironment{theindex}{%
  \section*{\indexname}%
  \setlength{\parindent}{0pt}%
  \setlength{\parskip}{0pt plus 0.3pt}%
  \let\item\@idxitem
}{%
  \clearpage
}
\makeatother

\IfFileExists{\jobname-pw.ind}{\input{\jobname-pw.ind}}{}

\end{document}

      