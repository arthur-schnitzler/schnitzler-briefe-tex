%% latex-korrekturansicht-vorspann.tex
%% Vorspann für die Korrekturansicht.
%% Lädt die gemeinsame Datei latex-vorspann.tex mit gesetztem Schalter.

\newif\ifkorrekturansicht
\korrekturansichttrue

\input{../tex-inputs/latex-vorspann}


               \section[Arthur Schnitzler an Albert Ehrenstein, 9. 10. 1910]{ Arthur Schnitzler an Albert Ehrenstein, 9. 10. 1910}\nopagebreak\mylabel{v}\rehead{ }\normalsize\beginnumbering\briefempfaengerindex{Ehrenstein, Albert@\textsc{Ehrenstein, Albert}!zzzSchnitzler, Arthur@\emph{von Arthur Schnitzler}!1910-10-092@{9. 10. 1910}|(be} \toendnotes[C]{\smallbreak\pagebreak[2]} \Standort{Jerusalem, The National Library of Israel, ARC. Ms. Var. 306 1 118.}
\physDesc{Brief, 1 Blatt, 1 Seite
\newline{}Handschrift: schwarze Tinte, lateinische Kurrent}\toendnotes[C]{\smallbreak}\pstart
           {\pb}\textcolor{gray}{\textbf{Dr. Arthur Schnitzler}}\hfill 9. X. 910\pend
           \pstart
           \textcolor{gray}{\textbf{\textcolor{pink}{Wien XVIII.}{}\ledrightnote{\textcolor{pink}{XVIII., Währing}}}}{ }\substVorne{}\textsuperscript{\textcolor{gray}{\textbf{\textcolor{pink}{Spoettelgasse 7.}{}\ledrightnote{\textcolor{pink}{Edmund-Weiß-Gasse}}}}}{\allowbreak}\substDazwischen{}\textcolor{pink}{Sternwartestr 71}{}\ledrightnote{\textcolor{pink}{Sternwartestraße}}.\substHinten{}\pend
           \pstart{}lieber Herr Ehrenstein,\pend\pstart
           noch bin ich Ihnen Antwort auf ihren Sommerbrief und Rückgabe der \textcolor{green}{Manuscripte}{}\ledrightnote{→\textcolor{green}{Graf Cilli}}
               schuldig. Wollen Sie beides
                    persönlich in Empfang nehmen, so sind Sie Mittwoch Abend gegen 7
                    willkommen\pend
           \pstart
           Ihrem Sie bestens grüßenden{\\[\baselineskip]}\spacefill\mbox{A. S.}\pend
           \leftskip=0em{}\endnumbering\briefempfaengerindex{Ehrenstein, Albert@\textsc{Ehrenstein, Albert}!zzzSchnitzler, Arthur@\emph{von Arthur Schnitzler}!1910-10-092@{9. 10. 1910}|)be}\mylabel{h}  \normalsize

\doendnotes{C}
\bigskip
\vfill

\clearpage

\footnotesize

\lohead{\textsc{register}}

% Definiere theindex-Environment komplett neu ohne reledmac
\makeatletter
\renewenvironment{theindex}{%
  \section*{\indexname}%
  \setlength{\parindent}{0pt}%
  \setlength{\parskip}{0pt plus 0.3pt}%
  \let\item\@idxitem
}{%
  \clearpage
}
\makeatother

\IfFileExists{\jobname-pw.ind}{\input{\jobname-pw.ind}}{}

\end{document}

      