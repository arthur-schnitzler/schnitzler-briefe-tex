%% latex-korrekturansicht-vorspann.tex
%% Vorspann für die Korrekturansicht.
%% Lädt die gemeinsame Datei latex-vorspann.tex mit gesetztem Schalter.

\newif\ifkorrekturansicht
\korrekturansichttrue

\input{../tex-inputs/latex-vorspann}


               \section[Arthur Schnitzler an Richard Beer-Hofmann mit Beilage Alfred Gold an Schnitzler, 17. 8. 1899]{ Arthur Schnitzler an Richard Beer-Hofmann mit Beilage Alfred Gold an
               Schnitzler, 17. 8. 1899}\nopagebreak\mylabel{v}\rehead{ }\normalsize\beginnumbering\briefempfaengerindex{Beer-Hofmann, Richard@\textsc{Beer-Hofmann, Richard}!zzzSchnitzler, Arthur@\emph{von Arthur Schnitzler}!1899-08-171@{17. 8. 1899}|(be} \toendnotes[C]{\smallbreak\pagebreak[2]} \Standort{YCGL, MSS 31.}
\physDesc{Briefkarte, Umschlag, Fragment
\newline{}Handschrift: schwarze Tinte, deutsche Kurrent\newline{}Beilage: \textcolor{blue}{Alfred Gold}: Brief, 1 Blatt, 1 Seite, schwarze Tinte, Kurrentschrift. Diese wird
                                 in Beer-Hofmanns Nachlass unter den Briefen Schnitzlers aufbewahrt.
                                 Die Zuordnung als Beilage basiert darauf, dass das Brieffragment
                                 zeitlich mit der Übermittlung des \textcolor{blue}{Gold}-Briefes zusammenfällt \newline{}Versand: 1) Stempel: »\nobreak{}\oindex{Bad Ischl@\textbf{Bad Ischl}, \emph{Besiedelter Ort (A.BSO)}|pwk}Isch\textcolor{gray}{l}, 17. 8. \textcolor{gray}{99}, 12–1 N\nobreak{}«.  2) Stempel: »\nobreak{}\oindex{Seeboden@\textbf{Seeboden}, \emph{http://www.geonames.org/ontologyA.ADM3}|pwk}Seeboden, 17. 8. 99\nobreak{}«. \newline{}Ordnung: mit Bleistift von unbekannter Hand: »Anfang
                                    fehlt?« und datiert »17. 8. 1899« }\toendnotes[C]{\smallbreak}\pstart{}{\pb}\textcolor{pink}{\textsc{Kärnthen}}{}\ledrightnote{\textcolor{pink}{Kärnten}}.\pend{}\pstart{}Herrn \textsc{Dr. Richard Beer-Hofmann}\pend{}\pstart{}\textcolor{pink}{\textsc{Seeboden am Millstätter}see}{}\ledrightnote{\textcolor{pink}{Seeboden}}\pend{}\pstart{}\textcolor{pink}{\textsc{Villa Platzer}}{}\ledrightnote{\textcolor{pink}{Villa Platzer}}\pend{}{\bigskip}\pstart
           \noindent{}{\pb}hatte es ſchon auf dem Bahnhof für Sie mit – vergaſs
               natürlich es Ihnen zu geben.\pend
           \pstart Herzlichen Gruß! Ihr \spacefill\mbox{Arthur}\pend{}\pstart
           17/8\pend
           {\bigskip}\pstart
           \noindent{}{\pb}{[}hs. Gold:{]} \textcolor{gray}{\textbf{»\textcolor{brown}{Die Zeit}{}\ledrightnote{\textcolor{brown}{Die Zeit. Wiener Wochenschrift}}«}}\hfill \textcolor{gray}{\textbf{\textbf{\textcolor{pink}{Wien}{}\ledrightnote{\textcolor{pink}{Wien}}}, den}}{ }14. 8. \textcolor{gray}{\textbf{189}}9\pend
           \pstart
           \textcolor{gray}{\textbf{Wiener Wochenſchrift}}\hfill \textcolor{pink}{IX/3, Günthergasse 1}{}\ledrightnote{\textcolor{pink}{Günthergasse}}.\pend
           \pstart
           \textcolor{gray}{\textbf{\textbf{Herausgeber}:}}{\\}\textcolor{gray}{\textbf{Profeſſor Dr. \textcolor{blue}{I. Singer}{}\ledrightnote{\textcolor{blue}{Isidor Singer}},
                        \textcolor{blue}{Hermann Bahr}{}\ledrightnote{\textcolor{blue}{Hermann Bahr}}, Dr. \textcolor{blue}{Heinrich Kanner}{}\ledrightnote{\textcolor{blue}{Heinrich Kanner}}.}}\pend
           \pstart
           \textcolor{gray}{\textbf{Telephon Nr. 6415.}}\pend
           \pstart{}Verehrter D\textsuperscript{r} Schnitzler,\pend\pstart
           Es iſt ſo gut wie ſicher, daſs ich mit der \textcolor{green}{Novelle}{}\ledrightnote{→\textcolor{green}{Der Tod Georgs. Fragment}}{ }ſchon im October beginnen kann (in der
               Nr. vom 7.) Bitte mir aber, wenn irgend möglich, das \textcolor{green}{Mſcr.}{}\ledrightnote{→\textcolor{green}{Der Tod Georgs. Fragment}}{ }\uline{noch im Auguſt} – u. zw. mit den Abtheilungen des
               Verf.– zu ſchicken. Besten Dank für frdl. Vermittlung.\pend
           \pstart
           In Eile Ihr herzlich ergebener{\\[\baselineskip]}\spacefill\mbox{\textcolor{blue}{AlfGold}{}\ledrightnote{\textcolor{blue}{Alfred Gold}}}\pend
           \leftskip=0em{}\pstart
           \noindent{}Grüße an B.-H. u. \textcolor{blue}{Waſſermann}{}\ledrightnote{\textcolor{blue}{Jakob Wassermann}}.\pend
           \pstart
           Herrn D\textsuperscript{r} Alfred Schnitzler\pend
           \pstart
           \textcolor{pink}{\textsc{Ischl}}{}\ledrightnote{\textcolor{pink}{Bad Ischl}}\pend
           \pstart
           \textcolor{gray}{\textbf{\label{T_L00959_1v}\edtext{Alle für »\textcolor{brown}{Die Zeit}{}\ledrightnote{\textcolor{brown}{Die Zeit. Wiener Wochenschrift}}« beſtimmten Zuſchriften und Sendungen ſind an die
                  Redaction der »\textcolor{brown}{Zeit}{}\ledrightnote{\textcolor{brown}{Die Zeit. Wiener Wochenschrift}}« und \textbf{nicht} an die Perſon eines der Herausgeber zu richten.}{\lemma{\textnormal{\emph{Alle … richten.}}}\Cendnote{\textnormal{am unteren Rand der Seite}}}\label{T_L00959_1h}}}\pend
           \endnumbering\briefempfaengerindex{Beer-Hofmann, Richard@\textsc{Beer-Hofmann, Richard}!zzzSchnitzler, Arthur@\emph{von Arthur Schnitzler}!1899-08-171@{17. 8. 1899}|)be}\mylabel{h}  \normalsize

\doendnotes{C}
\bigskip
\vfill

\clearpage

\footnotesize

\lohead{\textsc{register}}

% Definiere theindex-Environment komplett neu ohne reledmac
\makeatletter
\renewenvironment{theindex}{%
  \section*{\indexname}%
  \setlength{\parindent}{0pt}%
  \setlength{\parskip}{0pt plus 0.3pt}%
  \let\item\@idxitem
}{%
  \clearpage
}
\makeatother

\IfFileExists{\jobname-pw.ind}{\input{\jobname-pw.ind}}{}

\end{document}

      