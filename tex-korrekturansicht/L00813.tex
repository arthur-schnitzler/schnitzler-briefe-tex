%% latex-korrekturansicht-vorspann.tex
%% Vorspann für die Korrekturansicht.
%% Lädt die gemeinsame Datei latex-vorspann.tex mit gesetztem Schalter.

\newif\ifkorrekturansicht
\korrekturansichttrue

\input{../tex-inputs/latex-vorspann}


               \section[Arthur Schnitzler an Hugo von Hofmannsthal, 5. 7. 1898]{ Arthur Schnitzler an Hugo von Hofmannsthal, 5. 7. 1898}\nopagebreak\mylabel{v}\rehead{ }\normalsize\beginnumbering\briefempfaengerindex{Hofmannsthal, Hugo von@\textsc{Hofmannsthal, Hugo von}!zzzSchnitzler, Arthur@\emph{von Arthur Schnitzler}!1898-07-051@{5. 7. 1898}|(be} \toendnotes[C]{\smallbreak\pagebreak[2]} \Standort{FDH, Hs-30885,68.}
\physDesc{Brief, 1 Blatt, 3 Seiten
\newline{}Handschrift: schwarze Tinte, deutsche Kurrent}\buchAbdrucke{\weitereDrucke{Hugo von Hofmannsthal, Arthur Schnitzler: \emph{Briefwechsel}. Hg. Therese Nickl und Heinrich Schnitzler. Frankfurt am Main: \emph{S. Fischer} 1964, S. 104.} }\toendnotes[C]{\smallbreak}\pstart
           \raggedleft{}{\pb}\textcolor{pink}{Wien}{}\ledrightnote{\textcolor{pink}{Wien}}, 5. Juli 98.\pend
           \pstart
           mein lieber Hugo, das ka{\geminationn} ich ganz
                    gut ſo einrichten, daſs wir uns etwa am 9. Auguſt treffen – ob \textcolor{pink}{Innsbruck}{}\ledrightnote{\textcolor{pink}{Innsbruck}} oder vielleicht \textcolor{pink}{München}{}\ledrightnote{\textcolor{pink}{München}}, das wollen wir noch ſehn; ich dürfte ja vom
                        1. bis 9. Auguſt unter ſolchen Umſtänden (we{\geminationn} nicht meine \textcolor{blue}{Mama}{}\ledrightnote{→\textcolor{blue}{Louise Schnitzler}} doch noch auf mich Anſprüche macht) in \textcolor{pink}{Tegernſee}{}\ledrightnote{\textcolor{pink}{Tegernsee}}{ }ſein. Hoffentlich wird Ihre Sti{\geminationm}ung {\pb}noch in \textcolor{pink}{Galizien}{}\ledrightnote{\textcolor{pink}{Galizien}} beſſer. Haben Sie viel zu thun?\pend
           \pstart
           Ich werde wahrſcheinlich Montag abreiſen; eine Reihe von Tagen in
                        \textcolor{pink}{Graz}{}\ledrightnote{\textcolor{pink}{Graz}} bleiben. Sie werden i{\geminationm}er wiſſen, wo ich bin. Wie wird das nur mit \textcolor{blue}{Richard}{}\ledrightnote{\textcolor{blue}{Richard Beer-Hofmann}}{ }ſein, we{\geminationn} unſer Rendezvous ſo weit hinaus
                    geſchoben iſt? Ich erwarte heute einen Brief von ihm, der telegrafiſch aviſirt
                    iſt.\pend
           \pstart
           Ich ſchreibe an dem Stück, das vorläufig »\textcolor{green}{\textsc{Shawl}}{}\ledrightnote{\textcolor{green}{Der Schleier der Beatrice. Schauspiel in fünf Akten}}« heißen ſoll; bin im 2. Akt, {\pb}der mir
                    aber bisher im Ton durchaus nicht gelingen will.\pend
           \pstart
           Im übrigen bin ich recht gequält. –\pend
           \pstart
           Schauen wir nur, daſs dieſes Zuſa{\geminationm}enſein im
                        Auguſt zuſtande kommt.\pend
           \pstart Von Herzen Ihr \spacefill\mbox{Arthur.}\pend{}\endnumbering\briefempfaengerindex{Hofmannsthal, Hugo von@\textsc{Hofmannsthal, Hugo von}!zzzSchnitzler, Arthur@\emph{von Arthur Schnitzler}!1898-07-051@{5. 7. 1898}|)be}\mylabel{h}  \normalsize

\doendnotes{C}
\bigskip
\vfill

\clearpage

\footnotesize

\lohead{\textsc{register}}

% Definiere theindex-Environment komplett neu ohne reledmac
\makeatletter
\renewenvironment{theindex}{%
  \section*{\indexname}%
  \setlength{\parindent}{0pt}%
  \setlength{\parskip}{0pt plus 0.3pt}%
  \let\item\@idxitem
}{%
  \clearpage
}
\makeatother

\IfFileExists{\jobname-pw.ind}{\input{\jobname-pw.ind}}{}

\end{document}

      