%% latex-korrekturansicht-vorspann.tex
%% Vorspann für die Korrekturansicht.
%% Lädt die gemeinsame Datei latex-vorspann.tex mit gesetztem Schalter.

\newif\ifkorrekturansicht
\korrekturansichttrue

\input{../tex-inputs/latex-vorspann}


\renewcommand{\erwaehnteOrte}{Orte: Wien}
\renewcommand{\erwaehnteWerke}{}
\section[ Felix Salten an Arthur Schnitzler, {[}13. 9. 1894{]}]{Felix Salten an Arthur Schnitzler, {[}13. 9. 1894{]}}
\nopagebreak\mylabel{v}
\rehead{ }\normalsize\beginnumbering\briefempfaengerindex{Schnitzler, Arthur@\textsc{Schnitzler, Arthur}!zzzSalten, Felix@\emph{von Felix Salten}!1894-09-131@{{[}13. 9. 1894{]}}|(be}
\toendnotes[C]{\smallbreak\pagebreak[2]}\Standort{CUL, Schnitzler, B 89, A 1.}
\physDesc{Brief, 1 Blatt, 2 Seiten, 262 Zeichen
\newline{}Handschrift: blauer Buntstift, lateinische Kurrent
\newline{}Schnitzler: mit Bleistift datiert: »13/9 94.« 
\newline{}Ordnung: mit Bleistift von unbekannter Hand nummeriert: »47« }\toendnotes[C]{\smallbreak}
\pstart{}{\pb}Lieber Freund!\pend
\pstart
           \label{K_L03146-1v}\edtext{Vormittag}{\lemma{\textnormal{\emph{Vormittag}}}\Cendnote{\textnormal{Bezug unklar}}}\label{K_L03146-1h} kann ich nicht, das
               ist einmal sicher. Wenns nur schon sich{[}er{]} wär’, dass ich
                  Nachmittag kann. Ich weiss das aber noch nicht bestimmt {\kaufmannsund} kann Ihnen nichts {\pb}Anderes mittheilen, dass ich, –
                  \uline{wenn irgend möglich} – Nachmittag
               hinaus komme\textcolor{gray}{.}\pend
           
\pstart
           Wiedersehen, {\\[\baselineskip]}\spacefill\mbox{Salten}\pend
           \leftskip=0em{}\endnumbering\briefempfaengerindex{Schnitzler, Arthur@\textsc{Schnitzler, Arthur}!zzzSalten, Felix@\emph{von Felix Salten}!1894-09-131@{{[}13. 9. 1894{]}}|)be}\mylabel{h}  \normalsize

\doendnotes{C}
\bigskip
\vfill

\clearpage

\footnotesize

\lohead{\textsc{register}}

% Definiere theindex-Environment komplett neu ohne reledmac
\makeatletter
\renewenvironment{theindex}{%
  \section*{\indexname}%
  \setlength{\parindent}{0pt}%
  \setlength{\parskip}{0pt plus 0.3pt}%
  \let\item\@idxitem
}{%
  \clearpage
}
\makeatother

\IfFileExists{\jobname-pw.ind}{\input{\jobname-pw.ind}}{}

\end{document}

      