%% latex-korrekturansicht-vorspann.tex
%% Vorspann für die Korrekturansicht.
%% Lädt die gemeinsame Datei latex-vorspann.tex mit gesetztem Schalter.

\newif\ifkorrekturansicht
\korrekturansichttrue

\input{../tex-inputs/latex-vorspann}


               \section[Paul Goldmann an Arthur Schnitzler, 3. 11. {[}1894{]}]{ Paul Goldmann an Arthur Schnitzler, 3. 11. {[}1894{]}}\nopagebreak\mylabel{v}\rehead{ }\normalsize\beginnumbering\briefempfaengerindex{Schnitzler, Arthur@\textsc{Schnitzler, Arthur}!zzzGoldmann, Paul@\emph{von Paul Goldmann}!1894-11-031@{3. 11. {[}1894{]}}|(be} \toendnotes[C]{\smallbreak\pagebreak[2]} \Standort{DLA, A:Schnitzler, HS.NZ85.1.3164.}
\physDesc{Brief, 1 Blatt, 4 Seiten
\newline{}Handschrift: schwarze Tinte, deutsche Kurrent
\newline{}Schnitzler: 1) mit Bleistift auf dem ersten Blatt die Jahreszahl »94« vermerkt 2) mit rotem Buntstift eine Unterstreichung}\toendnotes[C]{\smallbreak}\pstart
           \noindent{}{\pb}\textcolor{gray}{\textbf{\textcolor{brown}{Frankfurter Zeitung}{}\ledrightnote{\textcolor{brown}{Frankfurter Zeitung}}.}}\hfill \textsc{\textcolor{pink}{Paris}{}\ledrightnote{\textcolor{pink}{Paris}}}, 3. November.\pend
           \pstart
           \textcolor{gray}{\textbf{(\textcolor{brown}{Gazette de
                     Francfort}{}\ledrightnote{\textcolor{brown}{Frankfurter Zeitung}}).}}\pend
           \pstart
           \textcolor{gray}{\textbf{\begin{otherlanguage}{french}Fondateur\end{otherlanguage}{ }\textbf{M. \textcolor{blue}{L. Sonnemann}{}\ledrightnote{\textcolor{blue}{Leopold Sonnemann}}}.}}\pend
           \pstart
           \textcolor{gray}{\textbf{\begin{otherlanguage}{french}Journal politique, financier,\end{otherlanguage}}}\pend
           \pstart
           \textcolor{gray}{\textbf{\begin{otherlanguage}{french}commercial et littéraire.\end{otherlanguage}}}\pend
           \pstart
           \textcolor{gray}{\textbf{\begin{otherlanguage}{french}\textbf{Paraissant trois fois par jour}\end{otherlanguage}}}.\pend
           \pstart
           \textcolor{gray}{\textbf{\begin{otherlanguage}{french}\textbf{Bureaux à \textcolor{pink}{Paris}{}\ledrightnote{\textcolor{pink}{Paris}}:}\end{otherlanguage}}}\pend
           \pstart
           \textcolor{gray}{\textbf{\begin{otherlanguage}{french}\textcolor{pink}{\textbf{24. Rue Feydeau}}{}\ledrightnote{\textcolor{pink}{rue Feydeau}}.\end{otherlanguage}}}\pend
           \pstart\center{}Mein lieber Freund,\pend\pstart
           Wir ſind mitten im \label{K_L02618-1v}\edtext{\textcolor{pink}{Ruſſen}{}\ledrightnote{\textcolor{pink}{Russland}}fieber}{\lemma{\textnormal{\emph{Ruſſenfieber}}}\Cendnote{\textnormal{Die politische Annäherung zwischen \textcolor{pink}{Russland} und \textcolor{pink}{Frankreich} führte zu einer Begeisterungswelle, die durch öffentliche
                  »Freundschaftsfeste« weiter gefördert wurden.}}}\label{K_L02618-1h} und ich finde gerade Zeit,
               Dir raſch beide Hände zu drücken, mit einem innigen \label{K_L02618-2v}\edtext{Glückwunſch}{\lemma{\textnormal{\emph{Glückwunſch}}}\Cendnote{\textnormal{siehe Max Burckhard an Arthur Schnitzler, [31. 10. 1894]}}}\label{K_L02618-2h}. So ſcheint alſo der liebſte Wunſch, den ich für Dich gehegt, wahr werden zu
               wollen. Ich habe mir heut Früh’, als ich Deinen lieben \label{K_L02618-22v}\edtext{Brief}{\lemma{\textnormal{\emph{Brief}}}\Cendnote{\textnormal{vgl. A. S.: \emph{Tagebuch}, 31. 10. 1894}}}\label{K_L02618-22h} erhielt, die Zukunft ausgemalt und habe mich an all’ dem Licht und der Freude
               ergötzt, die ich darin für Dich fand. Ich bin ſicher: Du wirſt {\pb}aufgeführt werden; ich bin ſicher: Du wirſt Erfolg
               haben, – ſo ſicher, daß mir iſt, als ſei das Alles ſchon geſchehen. \textcolor{blue}{B.}{}\ledrightnote{\textcolor{blue}{Max Eugen Burckhard}}’s Telegramm bedeutet ſicher die Annahme, und
               der \textcolor{blue}{Director}{}\ledrightnote{→\textcolor{blue}{Max Eugen Burckhard}} gefällt mir
               ſehr, der in dieſer Form anzunehmen verſteht. Bitte, ſchreib’ mir ſofort, \strikeout{daß} wie die Unterredung mit \textcolor{blue}{B.}{}\ledrightnote{\textcolor{blue}{Max Eugen Burckhard}} ausgefallen. Im Übrigen will ich gar nicht länger darüber
               reden, aus Aberglauben – denn es iſt gar zu ſchön. Und den Namen des \textcolor{brown}{Theaters}{}\ledrightnote{→\textcolor{brown}{Burgtheater}} nenne ich erſt gar
               nicht, auch aus Aberglauben. Aber froh bin ich; und {\pb}ich fühle die glückliche Wendung und denke, daß Niemand in der Welt ſie mehr
               verdient hat, als Du, mein lieber Freund.\pend
           \pstart
           Ich \strikeout{\textcolor{gray}{be}} möchte gern das Alles beſſer ſagen. Aber es iſt ſo ſchwer, über die guten
               Dinge zu ſchreiben{[}.{]} Überdies empfing ich heut mein \label{K_L02618-78v}\edtext{\textcolor{green}{Feuilleton}{}\ledrightnote{→\textcolor{green}{»Gismonda«}}}{\lemma{\textnormal{\emph{Feuilleton}}}\Cendnote{\textnormal{\textcolor{blue}{G.} [=\textcolor{blue}{Paul Goldmann}]: \emph{\textcolor{green}{»Gismonda«}}. In: \emph{\textcolor{green}{Frankfurter
                        Zeitung}}, Jg. 39, Nr. 305, 3. 11. 1894, Erstes Morgenblatt,
                     S. 1–2.}}}\label{K_L02618-78h} über »\label{K_L02618-111v}\edtext{\textsc{\textcolor{green}{Gismonda}{}\ledrightnote{\textcolor{green}{Gismonda. Pièce en 4 actes et 5 tableaux}}}}{\lemma{\textnormal{\emph{Gismonda}}}\Cendnote{\textnormal{\emph{\textcolor{green}{Gismonda. Pièce en 4 actes et 5 tableaux}}, von
                     \textcolor{blue}{Victorien Sardou} für \textcolor{blue}{Sarah Bernhardt} geschrieben, erlebte seine Uraufführung am
                     31. 10. 1894 am \emph{\textcolor{brown}{Théâtre de la
                     Renaissance}}.}}}\label{K_L02618-111h}«, das mein \textcolor{blue}{Onkel}{}\ledrightnote{→\textcolor{blue}{Fedor Mamroth}} in einer irrſinnigen Weiſe zuſammengeſtrichen hat.
               Das iſt ein Lähmungsſchlag ins Gehirn.\pend
           \pstart
           Ich danke Dir von ganzem Herzen für den Freundſchafts-Beweis, den Du mir
               gegeben, indem Du mir ſofort die {\pb}Nachricht
               mitgetheilt; und ich begrüße Dich vielmals und in Treue {\\[\baselineskip]}Dein \spacefill\mbox{Paul
                  Goldmann}\pend
           \leftskip=0em{}\endnumbering\briefempfaengerindex{Schnitzler, Arthur@\textsc{Schnitzler, Arthur}!zzzGoldmann, Paul@\emph{von Paul Goldmann}!1894-11-031@{3. 11. {[}1894{]}}|)be}\mylabel{h}  \normalsize

\doendnotes{C}
\bigskip
\vfill

\clearpage

\footnotesize

\lohead{\textsc{register}}

% Definiere theindex-Environment komplett neu ohne reledmac
\makeatletter
\renewenvironment{theindex}{%
  \section*{\indexname}%
  \setlength{\parindent}{0pt}%
  \setlength{\parskip}{0pt plus 0.3pt}%
  \let\item\@idxitem
}{%
  \clearpage
}
\makeatother

\IfFileExists{\jobname-pw.ind}{\input{\jobname-pw.ind}}{}

\end{document}

      