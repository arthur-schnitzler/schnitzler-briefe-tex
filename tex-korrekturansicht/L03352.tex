%% latex-korrekturansicht-vorspann.tex
%% Vorspann für die Korrekturansicht.
%% Lädt die gemeinsame Datei latex-vorspann.tex mit gesetztem Schalter.

\newif\ifkorrekturansicht
\korrekturansichttrue

\input{../tex-inputs/latex-vorspann}


\renewcommand{\erwaehntePersonen}{Personen: Gustav Klimt, Bertold Löffler, Ottilie Salten, Paul Salten, Paul Schlenther}
\renewcommand{\erwaehnteInstitutionen}{Institutionen: Wiener Verlag}
\renewcommand{\erwaehnteOrte}{Orte: Kahlenberg, Leipzig, Semmering, Waidhofen an der Ybbs, Wien}
\renewcommand{\erwaehnteWerke}{Werke: Gustav Klimt. Gelegentliche Anmerkungen}
\section[ Felix Salten an Arthur Schnitzler, 27. 11. 1903]{Felix Salten an Arthur Schnitzler, 27. 11. 1903}
\nopagebreak\mylabel{v}
\rehead{ }\normalsize\beginnumbering\briefempfaengerindex{Schnitzler, Arthur@\textsc{Schnitzler, Arthur}!zzzSalten, Felix@\emph{von Felix Salten}!1903-11-271@{27. 11. 1903}|(be}
\toendnotes[C]{\smallbreak\pagebreak[2]}\Standort{CUL, Schnitzler, B 89, A 2.}
\physDesc{Brief, 1 Blatt, 1 Seite, 362 Zeichen
\newline{}Handschrift: Bleistift, lateinische Kurrent
\newline{}Ordnung: mit Bleistift von unbekannter Hand nummeriert: »178« }\toendnotes[C]{\smallbreak}
\pstart
           \raggedleft{}{\pb}\textcolor{pink}{Am Kahlenberg}{}\ledrightnote{\textcolor{pink}{Kahlenberg}}, 27 XI. 03\pend
           
\pstart
           Lieber, ich bin doch nicht nach \textcolor{pink}{Waidhofen}{}\ledrightnote{\textcolor{pink}{Waidhofen an der Ybbs}} sondern lieber hier herauf, wo es wunderschön und ganz still ist.
               Gedenke mir diesen \textcolor{pink}{Berg}{}\ledrightnote{{$\rightarrow$}\textcolor{pink}{Kahlenberg}} jetzt
               als meinen Privat-\textcolor{pink}{Semmering}{}\ledrightnote{\textcolor{pink}{Semmering}} anzuschaffen. Herzl.
               Dank für Ihre Wolmeinung über meinen \label{K_L03352-1v}\edtext{\textcolor{green}{\textcolor{blue}{Klimt}{}\ledrightnote{\textcolor{blue}{Gustav Klimt}}-Aufsatz}{}\ledrightnote{{$\rightarrow$}\textcolor{green}{Gustav Klimt. Gelegentliche Anmerkungen}}}{\lemma{\textnormal{\emph{Klimt-Aufsatz}}}\Cendnote{\textnormal{\textcolor{blue}{Felix Salten}: \emph{\textcolor{green}{Gustav Klimt. Gelegentliche Anmerkungen}}. Buchschmuck
                     von \textcolor{blue}{Bertold Löffler}. \textcolor{pink}{Wien}/\textcolor{pink}{Leipzig}: \emph{\textcolor{brown}{Wiener Verlag}}{ }1903.}}}\label{K_L03352-1h}. \label{K_L03352-2v}\edtext{Nächstens ziehe
               ich mich hierher mit \textcolor{blue}{Schlenther}{}\ledrightnote{\textcolor{blue}{Paul Schlenther}} zurück}{\lemma{\textnormal{\emph{Nächstens … zurück}}}\Cendnote{\textnormal{Das ist nicht nachweisbar, jedoch plante
                     \textcolor{blue}{Salten} mit seiner Frau \textcolor{blue}{Ottilie} und dem gemeinsamen Sohn \textcolor{blue}{Paul} über Ostern 1904 auf
                  den \textcolor{pink}{Kahlenberg} zu fahren, vgl. Felix Salten an Arthur Schnitzler, 30. 3. 1904.}}}\label{K_L03352-2h}, und hoffe Sie
               noch besser zu bedienen.\pend
           \pstart herzlichst Ihr \spacefill\mbox{S.}\pend{}\endnumbering\briefempfaengerindex{Schnitzler, Arthur@\textsc{Schnitzler, Arthur}!zzzSalten, Felix@\emph{von Felix Salten}!1903-11-271@{27. 11. 1903}|)be}\mylabel{h}  \normalsize

\doendnotes{C}
\bigskip
\vfill

\clearpage

\footnotesize

\lohead{\textsc{register}}

% Definiere theindex-Environment komplett neu ohne reledmac
\makeatletter
\renewenvironment{theindex}{%
  \section*{\indexname}%
  \setlength{\parindent}{0pt}%
  \setlength{\parskip}{0pt plus 0.3pt}%
  \let\item\@idxitem
}{%
  \clearpage
}
\makeatother

\IfFileExists{\jobname-pw.ind}{\input{\jobname-pw.ind}}{}

\end{document}

      