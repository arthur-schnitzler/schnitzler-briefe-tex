%% latex-korrekturansicht-vorspann.tex
%% Vorspann für die Korrekturansicht.
%% Lädt die gemeinsame Datei latex-vorspann.tex mit gesetztem Schalter.

\newif\ifkorrekturansicht
\korrekturansichttrue

\input{../tex-inputs/latex-vorspann}


\section[Stefan Zweig an Arthur Schnitzler, 25. 11. 1915]{L03652 Stefan Zweig an Arthur Schnitzler, 25. 11. 1915}
\nopagebreak\mylabel{L03652v}
\rehead{ }\normalsize\beginnumbering\briefempfaengerindex{, @\textsc{, }!zzz, @\emph{von  }!1915-11-251@{25. 11. 1915}|(be}
\toendnotes[C]{\smallbreak\pagebreak[2]}\Standort{CUL, Schnitzler, B 118.}
\physDesc{Bildpostkarte, 478 Zeichen
\newline{}Handschrift: schwarze Tinte, lateinische Kurrent
\newline{}Versand: Stempel: »\nobreak{}\oindex{VII., Neubau@\textbf{VII., Neubau}, \emph{Verwaltungsgebiet}|pwk}7 \textcolor{gray}{Wien}, 26. 11. 15, 4\nobreak{}«.  }
\buchAbdrucke{\weitereDrucke{Stefan Zweig: \emph{Briefwechsel mit Hermann Bahr, Sigmund Freud, Rainer Maria
                        Rilke und Arthur Schnitzler}. Herausgegeben von Jeffrey B. Berlin,  Hans-Ulrich Lindken und  Donald A. Prater. Frankfurt am Main: \emph{S. Fischer} 1987, S. 398.} }\toendnotes[C]{\smallbreak}\pstart{}{\pb}D\textsuperscript{r}
                  Arthur Schnitzler\pend{}\pstart{}\textcolor{pink}{Wien – Cottage}\oindex{Wien@\textbf{Wien}!XVIII., Währing@\textbf{XVIII., Währing}!Währinger Cottage@\textbf{Währinger Cottage}, \emph{Teil eines besiedelten Ortes}|pw}{}\ledrightnote{\textcolor{pink}{Währinger Cottage}}\pend{}\pstart{}\textcolor{pink}{\label{K_L03652-1v}\edtext{Sternwartestrasse 72}{\lemma{\textnormal{\emph{Sternwartestrasse 72}}}\Cendnote{\textnormal{\textcolor{blue}{Zweig}\pwindex{Zweig, Stefan 28.\,11.\,1881 Wien – 23.\,2.\,1942 Petrópolis@\textsc{Zweig, Stefan} (28.\,11.\,1881 Wien – 23.\,2.\,1942 Petrópolis), \emph{Schriftsteller}|pwk} wechselt
                        bei der Adressierung seiner Schreiben an \textcolor{blue}{Schnitzler} immer wieder zwischen der falschen Hausnummer
                           »72« und der richtigen
                  »71«.}}}\label{K_L03652-1}}\oindex{Wien@\textbf{Wien}!XVIII., Währing@\textbf{XVIII., Währing}!Sternwartestraße 71@\textbf{Sternwartestraße 71}, \emph{Wohngebäude}|pw}{}\ledrightnote{\textcolor{pink}{Sternwartestraße 71}}\pend{}{\bigskip}
\pstart
           \noindent{}\centering{}{\pb}\textcolor{gray}{\textbf{\textcolor{pink}{Wien – Schönbrunn, röm. Ruine}\oindex{Wien@\textbf{Wien}!XIII., Hietzing@\textbf{XIII., Hietzing}!Römische Ruine [Schlosspark Schönbrunn]@\textbf{Römische Ruine [Schlosspark Schönbrunn]}, \emph{Monument}|pw}{}\ledrightnote{\textcolor{pink}{Römische Ruine [Schlosspark Schönbrunn]}}}}\pend
           \vspace{1em}
\pstart
           \noindent{}{\pb}Lieber verehrter Herr
                  Doktor, am 29. Januar ist \textcolor{blue}{Romain Rollands}\pwindex{Rolland, Romain 29.\,1.\,1866 Clamecy – 30.\,12.\,1944 Vézelay@\textsc{Rolland, Romain} (29.\,1.\,1866 Clamecy – 30.\,12.\,1944 Vézelay), \emph{Schriftsteller}|pw}{}\ledrightnote{\textcolor{blue}{Romain Rolland}} fünfzigster Geburtstag. Seine Freunde und
               alle, die ihm für seine menschliche Haltung in dieser Zeit dankbar sind, wollen ihm
               zu diesem Tage ein Wort \label{K_L03652-2v}\edtext{telegrafieren}{\lemma{\textnormal{\emph{telegrafieren}}}\Cendnote{\textnormal{Vgl. Romain Rolland an Arthur Schnitzler, 6. 2. 1916.}}}\label{K_L03652-2}. Ist es auch Ihre Absicht, so sage ich Ihnen
               auf jeden Fall seine Adresse \textcolor{pink}{\uline{Genf–Champel}, Hotel Beau Sejour}\oindex{Hôtel Beau-Séjour@\textbf{Hôtel Beau-Séjour}, \emph{Hotel}|pw}{}\ledrightnote{\textcolor{pink}{Hôtel Beau-Séjour}}. \label{K_L03652-3v}\edtext{\textcolor{violet}{Gestern}\eventindex{Wiener Konzerthaus@\textbf{Wiener Konzerthaus}!1. Sonatenabend von Arnold Rosé und Bruno Walter, 24.11.1915@1. Sonatenabend von Arnold Rosé und Bruno Walter, 24.11.1915|pwv}{}\ledrightnote{{$\rightarrow$}\emph{\textcolor{violet}{1. Sonatenabend von Arnold Rosé und Bruno Walter, 24.11.1915}}} sah ich Sie von ferne bei \textcolor{blue}{Rosé}\pwindex{Rosé, Arnold 24.\,10.\,1863 Iași – 25.\,8.\,1946 London@\textsc{Rosé, Arnold} (24.\,10.\,1863 Iași – 25.\,8.\,1946 London), \emph{Violinist}|pw}{}\ledrightnote{\textcolor{blue}{Arnold Rosé}}}{\lemma{\textnormal{\emph{Gestern … Rosé}}}\Cendnote{\textnormal{Am 24. 11. 1915 besuchte \textcolor{blue}{Schnitzler}
                  einen \textcolor{violet}{Sonatenabend von \textcolor{blue}{Arnold Rosé}\pwindex{Rosé, Arnold 24.\,10.\,1863 Iași – 25.\,8.\,1946 London@\textsc{Rosé, Arnold} (24.\,10.\,1863 Iași – 25.\,8.\,1946 London), \emph{Violinist}|pwk} und \textcolor{blue}{Bruno Walter}\pwindex{Walter, Bruno 15.\,9.\,1876 Berlin – 17.\,2.\,1962 Beverly Hills@\textsc{Walter, Bruno} (15.\,9.\,1876 Berlin – 17.\,2.\,1962 Beverly Hills), \emph{Theaterleiter, Komponist, Dirigent}|pwk}}\eventindex{Wiener Konzerthaus@\textbf{Wiener Konzerthaus}!1. Sonatenabend von Arnold Rosé und Bruno Walter, 24.11.1915@1. Sonatenabend von Arnold Rosé und Bruno Walter, 24.11.1915|pwkv} im \textcolor{pink}{Wiener Konzerthaus}\oindex{Wien@\textbf{Wien}!III., Landstraße@\textbf{III., Landstraße}!Wiener Konzerthaus@\textbf{Wiener Konzerthaus}, \emph{Konzertsaal}|pwk}. }}}\label{K_L03652-3}. Es war herrlich über alle
               Maassen.\pend
           
\pstart
           Treulichst Ihr{\\[\baselineskip]}\spacefill\mbox{Stefan Zweig}\pend
           \leftskip=0em{}\selectlanguage{ngerman}\endnumbering\briefempfaengerindex{, @\textsc{, }!zzz, @\emph{von  }!1915-11-251@{25. 11. 1915}|)be}\mylabel{L03652h}  \normalsize

\doendnotes{C}
\bigskip
\vfill

\clearpage

\footnotesize

\lohead{\textsc{register}}

% Definiere theindex-Environment komplett neu ohne reledmac
\makeatletter
\renewenvironment{theindex}{%
  \section*{\indexname}%
  \setlength{\parindent}{0pt}%
  \setlength{\parskip}{0pt plus 0.3pt}%
  \let\item\@idxitem
}{%
  \clearpage
}
\makeatother

\IfFileExists{\jobname-pw.ind}{\input{\jobname-pw.ind}}{}

\end{document}

      