%% latex-korrekturansicht-vorspann.tex
%% Vorspann für die Korrekturansicht.
%% Lädt die gemeinsame Datei latex-vorspann.tex mit gesetztem Schalter.

\newif\ifkorrekturansicht
\korrekturansichttrue

\input{../tex-inputs/latex-vorspann}


\renewcommand{\erwaehntePersonen}{Personen: Romain Rolland, Arnold Rosé, Bruno Walter, Stefan Zweig}
\renewcommand{\erwaehnteOrte}{Orte: Hôtel Beau-Séjour, Römische Ruine [Schlosspark Schönbrunn], Sternwartestraße 71, VII., Neubau, Wien, Wiener Konzerthaus, Währinger Cottage}
\renewcommand{\erwaehnteWerke}{}
\section[Stefan Zweig an Arthur Schnitzler, 25. 11. 1915]{Stefan Zweig an Arthur Schnitzler, 25. 11. 1915}
\nopagebreak\mylabel{v}
\rehead{ }\normalsize\beginnumbering\briefempfaengerindex{Schnitzler, Arthur@\textsc{Schnitzler, Arthur}!zzzZweig, Stefan@\emph{von Stefan Zweig}!1915-11-251@{25. 11. 1915}|(be}
\toendnotes[C]{\smallbreak\pagebreak[2]}\Standort{CUL, Schnitzler, B 118.}
\physDesc{Bildpostkarte, 480 Zeichen
\newline{}Handschrift: schwarze Tinte, lateinische Kurrent
\newline{}Versand: Stempel: »\nobreak{}\oindex{VII., Neubau@\textbf{VII., Neubau}, \emph{A.ADM3}|pwk}7 \textcolor{gray}{Wien}, 25. 11. 15\nobreak{}«.  }
\buchAbdrucke{\weitereDrucke{Stefan Zweig: \emph{Briefwechsel mit Hermann Bahr, Sigmund Freud, Rainer Maria
                        Rilke und Arthur Schnitzler}. Hg. Jeffrey B. Berlin, Hans-Ulrich Lindken und Donald A. Prater. Frankfurt am Main: \emph{S. Fischer} 1987, S. 398.} }\toendnotes[C]{\smallbreak}\pstart{}{\pb}D\textsuperscript{r}
                  Arthur Schnitzler\pend{}\pstart{}\textcolor{pink}{Wien – Cottage}{}\ledrightnote{\textcolor{pink}{Währinger Cottage}}\pend{}\pstart{}\textcolor{pink}{\label{K_L03652-1v}\edtext{Sternwartestrasse 72}{\lemma{\textnormal{\emph{Sternwartestrasse 72}}}\Cendnote{\textnormal{\textcolor{blue}{Zweig} wechselt
                        bei der Adressierung seiner Schreiben an \textcolor{blue}{Schnitzler} immer wieder zwischen der falschen Hausnummer
                           »72« und der richtigen
                  »71«.}}}\label{K_L03652-1h}}{}\ledrightnote{\textcolor{pink}{Sternwartestraße 71}}\pend{}
{\bigskip}
\pstart
           \noindent{}{\pb}\textcolor{gray}{\textbf{\textcolor{pink}{Wien – Schönbrunn, röm. Ruine}{}\ledrightnote{\textcolor{pink}{Römische Ruine [Schlosspark Schönbrunn]}}}}\pend
           
\pstart
           \noindent{}{\pb}Lieber verehrter Herr
                  Doktor, am 29. Januar ist \textcolor{blue}{Romain Rollands}{}\ledrightnote{\textcolor{blue}{Romain Rolland}} fünfzigster Geburtstag. Seine Freunde und
               alle, die ihm für seine menschliche Haltung in dieser Zeit dankbar sind, wollen ihm
               zu diesem Tage ein Wort telegrafieren. Ist es auch Ihre Absicht, so sage ich Ihnen
               auf jeden Fall seine Adresse \textcolor{pink}{\uline{Genf–Champel}, Hotel Beau Sejour}{}\ledrightnote{\textcolor{pink}{Hôtel Beau-Séjour}}. \label{K_L03652-2v}\edtext{Gestern sah ich Sie von ferne bei \textcolor{blue}{Rosé}{}\ledrightnote{\textcolor{blue}{Arnold Rosé}}}{\lemma{\textnormal{\emph{Gestern … Rosé}}}\Cendnote{\textnormal{Am 24. 11. 1915 besuchte \textcolor{blue}{Schnitzler}
                  einen Sonatenabend von \textcolor{blue}{Arnold Rosé} und \textcolor{blue}{Bruno Walter} im \textcolor{pink}{Wiener Konzerthaus}. }}}\label{K_L03652-2h}. Es war herrlich über alle
               Maassen.\pend
           
\pstart
           Treulichst Ihr{\\[\baselineskip]}\spacefill\mbox{Stefan Zweig}\pend
           \leftskip=0em{}\endnumbering\briefempfaengerindex{Schnitzler, Arthur@\textsc{Schnitzler, Arthur}!zzzZweig, Stefan@\emph{von Stefan Zweig}!1915-11-251@{25. 11. 1915}|)be}\mylabel{h}
\begin{anhang}
\end{anhang}\normalsize

\doendnotes{C}
\bigskip
\vfill

\clearpage

\footnotesize

\lohead{\textsc{register}}

% Definiere theindex-Environment komplett neu ohne reledmac
\makeatletter
\renewenvironment{theindex}{%
  \section*{\indexname}%
  \setlength{\parindent}{0pt}%
  \setlength{\parskip}{0pt plus 0.3pt}%
  \let\item\@idxitem
}{%
  \clearpage
}
\makeatother

\IfFileExists{\jobname-pw.ind}{\input{\jobname-pw.ind}}{}

\end{document}

      