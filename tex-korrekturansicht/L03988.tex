%% latex-korrekturansicht-vorspann.tex
%% Vorspann für die Korrekturansicht.
%% Lädt die gemeinsame Datei latex-vorspann.tex mit gesetztem Schalter.

\newif\ifkorrekturansicht
\korrekturansichttrue

\input{../tex-inputs/latex-vorspann}


\section[Felix Salten: Widmungsexemplar von Gestalten und Erscheinungen an Olga Schnitzler, Oktober 1913]{L03988 Felix Salten: Widmungsexemplar von Gestalten und Erscheinungen an Olga
               Schnitzler, Oktober 1913}
\nopagebreak\mylabel{L03988v}
\rehead{ }\normalsize\beginnumbering\briefempfaengerindex{Schnitzler, Olga@\textsc{Schnitzler, Olga}!zzzSalten, Felix@\emph{von Felix Salten}!1913-01-301@{Oktober 1913}|(be}
\toendnotes[C]{\smallbreak\pagebreak[2]}
\correspDesc{Versand  durch Felix Salten im Zeitraum Oktober 1913 in Wien
\newline{}Erhalt  durch Olga Schnitzler im Zeitraum Oktober 1913 in Wien}\toendnotes[C]{\smallbreak}
\Standort{Wien, Österreichische Nationalbibliothek, Lit-1,918.010-B.}
\physDesc{Widmung am Titetei:lblatt, 100 Zeichen
\newline{}Handschrift: schwarze Tinte, lateinische Kurrent}\toendnotes[C]{\smallbreak}
\pstart
           \noindent{}\centering{}{\pb}\textcolor{gray}{\textbf{\textcolor{green}{GESTALTEN}\pwindex{Salten, Felix 6.\,9.\,1869 Budapest – 8.\,10.\,1945 Zürich@\textsc{Salten, Felix} (6.\,9.\,1869 Budapest – 8.\,10.\,1945 Zürich), \emph{Schriftsteller, Journalist, Chefredakteur}!Gestalten und Erscheinungen@\strich\emph{Gestalten und Erscheinungen}|pw}{}\ledrightnote{\textcolor{green}{Gestalten und Erscheinungen}}}}\pend
           
\pstart
           \centering{}\textcolor{gray}{\textbf{\textcolor{green}{UND}\pwindex{Salten, Felix 6.\,9.\,1869 Budapest – 8.\,10.\,1945 Zürich@\textsc{Salten, Felix} (6.\,9.\,1869 Budapest – 8.\,10.\,1945 Zürich), \emph{Schriftsteller, Journalist, Chefredakteur}!Gestalten und Erscheinungen@\strich\emph{Gestalten und Erscheinungen}|pw}{}\ledrightnote{\textcolor{green}{Gestalten und Erscheinungen}}}}\pend
           
\pstart
           \centering{}\textcolor{gray}{\textbf{\textcolor{green}{ERSCHEINUNGEN}\pwindex{Salten, Felix 6.\,9.\,1869 Budapest – 8.\,10.\,1945 Zürich@\textsc{Salten, Felix} (6.\,9.\,1869 Budapest – 8.\,10.\,1945 Zürich), \emph{Schriftsteller, Journalist, Chefredakteur}!Gestalten und Erscheinungen@\strich\emph{Gestalten und Erscheinungen}|pw}{}\ledrightnote{\textcolor{green}{Gestalten und Erscheinungen}}}}\pend
           {\vspace{1\baselineskip}}
\pstart
           \raggedleft{}\label{K_L03988-1v}\edtext{Olga Schnitzler}{\lemma{\textnormal{\emph{Olga Schnitzler}}}\Cendnote{\textnormal{Ein Grund dafür, dass \textcolor{blue}{Salten}\pwindex{Salten, Felix 6.\,9.\,1869 Budapest – 8.\,10.\,1945 Zürich@\textsc{Salten, Felix} (6.\,9.\,1869 Budapest – 8.\,10.\,1945 Zürich), \emph{Schriftsteller, Journalist, Chefredakteur}|pwk} das Buch nicht \textcolor{blue}{Schnitzler} selbst widmete, könnte daran liegen, dass es einen \textcolor{green}{Aufsatz}\pwindex{Salten, Felix 6.\,9.\,1869 Budapest – 8.\,10.\,1945 Zürich@\textsc{Salten, Felix} (6.\,9.\,1869 Budapest – 8.\,10.\,1945 Zürich), \emph{Schriftsteller, Journalist, Chefredakteur}!Artur Schnitzler. (Zum 50. Geburtstag)@\strich\emph{Artur Schnitzler. (Zum 50. Geburtstag)}|pwkv} über diesen enthält. Der Aufsatz
                  war zuerst anlässlich von \textcolor{blue}{Schnitzlers}
                  50. Geburtstag im \emph{\textcolor{green}{Fremden-Blatt}\pwindex{Fremden-Blatt@\emph{Fremden-Blatt}|pwk}}
                  erschienen.}}}\label{K_L03988-1}\pend
           
\pstart
           \raggedleft{}in aufrichtiger Verehrung und herzlicher Freundschaft\pend
           
\pstart
           \raggedleft{}Felix Salten\pend
           
\pstart
           \raggedleft{}\textcolor{pink}{Wien}\oindex{Wien@\textbf{Wien}, \emph{Verwaltungsgebiet}|pw}{}\ledrightnote{\textcolor{pink}{Wien}}, Oktober 1913\pend
           {\vspace{1\baselineskip}}
\pstart
           \centering{}\textcolor{gray}{\textbf{VON}}\pend
           
\pstart
           \centering{}\textcolor{gray}{\textbf{FELIX SALTEN}}\pend
           {\vspace{1\baselineskip}}
\pstart
           \centering{}\textcolor{gray}{\textbf{1913}}\pend
           
\pstart
           \centering{}\textcolor{gray}{\textbf{\textcolor{brown}{S. FISCHER ⋅ VERLAG}\orgindex{S. Fischer Verlag@S. Fischer Verlag|pw}{}\ledrightnote{\textcolor{brown}{S. Fischer Verlag}} ⋅ \textcolor{pink}{BERLIN}\oindex{Berlin@\textbf{Berlin}, \emph{Hauptstadt}|pw}{}\ledrightnote{\textcolor{pink}{Berlin}}}}\pend
           \selectlanguage{ngerman}\endnumbering\briefempfaengerindex{Schnitzler, Olga@\textsc{Schnitzler, Olga}!zzzSalten, Felix@\emph{von Felix Salten}!1913-01-011@{Oktober 1913}|)be}\mylabel{L03988h}
\begin{anhang}
\end{anhang}\normalsize

\doendnotes{C}
\bigskip
\vfill

\clearpage

\footnotesize

\lohead{\textsc{register}}

% Definiere theindex-Environment komplett neu ohne reledmac
\makeatletter
\renewenvironment{theindex}{%
  \section*{\indexname}%
  \setlength{\parindent}{0pt}%
  \setlength{\parskip}{0pt plus 0.3pt}%
  \let\item\@idxitem
}{%
  \clearpage
}
\makeatother

\IfFileExists{\jobname-pw.ind}{\input{\jobname-pw.ind}}{}

\end{document}

      