%% latex-korrekturansicht-vorspann.tex
%% Vorspann für die Korrekturansicht.
%% Lädt die gemeinsame Datei latex-vorspann.tex mit gesetztem Schalter.

\newif\ifkorrekturansicht
\korrekturansichttrue

\input{../tex-inputs/latex-vorspann}


               \section[Arthur Schnitzler an Georg Brandes, 14. 5. 1901]{ Arthur Schnitzler an Georg Brandes, 14. 5. 1901}\nopagebreak\mylabel{v}\rehead{ }\normalsize\beginnumbering\briefempfaengerindex{Brandes, Georg@\textsc{Brandes, Georg}!zzzSchnitzler, Arthur@\emph{von Arthur Schnitzler}!1901-05-141@{14. 5. 1901}|(be} \toendnotes[C]{\smallbreak\pagebreak[2]} \Standort{Kopenhagen, Det Kongelige Bibliotek, Georg Brandes Arkiv, box 125.}
\physDesc{Brief, 1 Blatt, 3 Seiten
\newline{}Handschrift: schwarze Tinte, deutsche Kurrent\newline{}Ordnung: mit Bleistift von 
                                    unbekannter Hand nummeriert: »23.«,
                                    datiert: »14. 5.01« und beschriftet mit: »Arth.
                                    Schnitzler« }\buchAbdrucke{\weitereDrucke{Georg Brandes, Arthur Schnitzler: \emph{Ein Briefwechsel}. Hg. Kurt Bergel. Bern: \emph{Francke} 1956, S. 86.} }\pstart{}{\pb}Liebſter Herr Brandes\pend\pstart
           da meine Wohnung etwa zwiſchen Ihren beiden Bahnhöfen liegt, iſt es am beſten,
                    Sie fahren mit Ihrem Gepäck zu mir (der Portier in unſerm Haus kann es
                    aufbewahren; er wird aviſirt ſein) wenn Sie es nicht vorziehen, das Gepäck vom
                        \textcolor{pink}{Nordbahnhof}{}\ledrightnote{\textcolor{pink}{Nordbahnhof}} direct zum \textcolor{pink}{Südbahnhof}{}\ledrightnote{\textcolor{pink}{Südbahnhof}}{ }ſchaffen zu laſſen. {\pb}Aber ich würde den Vortheil dieſer
                    letztern Anordnung nicht einſehen es wäre nicht einmal eine Erſparnis.\pend
           \pstart
           Unſer Eſſen werden wir ſo einrichten, daſs Sie bequem zu Ihrem Zug auf der \textcolor{pink}{Südbahn}{}\ledrightnote{\textcolor{pink}{Südbahnhof}}{ }ſind.\pend
           \pstart
           Somit hoff ich Sie am Donnerſtg kurz nach 4 bei mir zu
                    begrüßen. (Ich wohne jetzt 2 Treppen höher.) Natürlich würde ich Sie auch gerne
                    von der Bahn {\pb}abholen aber es gibt
                    Menſchen, denen das unangenehm iſt u ich weiſs nicht ob Sie am Ende zu dieſen
                    gehören.\pend
           \pstart
           Alſo auf Wiederſehen.\pend
           \pstart
           Mit den herzlichſten Grüßen.\pend
           \pstart
           Ihr treuer{\\[\baselineskip]}\spacefill\mbox{ArthSchnitzler}\pend
           \leftskip=0em{}\pstart
           \textcolor{pink}{Wien}{}\ledrightnote{\textcolor{pink}{Wien}},
                        14. 5. 901.\pend
           \endnumbering\briefempfaengerindex{Brandes, Georg@\textsc{Brandes, Georg}!zzzSchnitzler, Arthur@\emph{von Arthur Schnitzler}!1901-05-141@{14. 5. 1901}|)be}\mylabel{h}  \normalsize

\doendnotes{C}
\bigskip
\vfill

\clearpage

\footnotesize

\lohead{\textsc{register}}

% Definiere theindex-Environment komplett neu ohne reledmac
\makeatletter
\renewenvironment{theindex}{%
  \section*{\indexname}%
  \setlength{\parindent}{0pt}%
  \setlength{\parskip}{0pt plus 0.3pt}%
  \let\item\@idxitem
}{%
  \clearpage
}
\makeatother

\IfFileExists{\jobname-pw.ind}{\input{\jobname-pw.ind}}{}

\end{document}

      