%% latex-korrekturansicht-vorspann.tex
%% Vorspann für die Korrekturansicht.
%% Lädt die gemeinsame Datei latex-vorspann.tex mit gesetztem Schalter.

\newif\ifkorrekturansicht
\korrekturansichttrue

\input{../tex-inputs/latex-vorspann}


\section[Arthur Schnitzler an Stefan Zweig, 1{[}1{]}. 1{[}2{]}. 1911]{L03790 Arthur Schnitzler an Stefan Zweig, 1{[}1{]}. 1{[}2{]}. 1911}
\nopagebreak\mylabel{L03790v}
\rehead{ }\normalsize\beginnumbering\briefempfaengerindex{Zweig, Stefan@\textsc{Zweig, Stefan}!zzzSchnitzler, Arthur@\emph{von Arthur Schnitzler}!1911-12-111@{1{[}1{]}. 1{[}2{]}. 1911}|(be}
\toendnotes[C]{\smallbreak\pagebreak[2]}\Standort{Jerusalem, National Library of Israel, ARC. Ms. Var. 305 1 58 Stefan Zweig Collection.}
\physDesc{Brief, 1 Blatt, 1 Seite, 226 Zeichen
\newline{}Handschrift: schwarze Tinte, deutsche Kurrent}\toendnotes[C]{\smallbreak}
\pstart
           {\pb}\textcolor{gray}{\textbf{Dr. Arthur Schnitzler}}\hfill \textcolor{pink}{Wien}\oindex{Wien@\textbf{Wien}|pw}{}\ledrightnote{\textcolor{pink}{Wien}}{ }1\textcolor{gray}{1}. 1\textcolor{gray}{2}. 911.\pend
           
\pstart
           \textcolor{gray}{\textbf{\textcolor{pink}{Wien XVIII. Sternwartestrasse 71}\oindex{Sternwartestrasse 71@\textbf{Sternwartestraße 71}|pw}{}\ledrightnote{\textcolor{pink}{Sternwartestraße 71}}}}\pend
           
\pstart{}lieber Doctor Zweig,\pend\vspace{0.5em}
\pstart
           wollen Sie morgen \label{K_L03790-1v}\edtext{Dinſtag}{\lemma{\textnormal{\emph{Dinſtag}}}\Cendnote{\textnormal{Vgl. A. S.: \emph{Tagebuch}, 12. 12. 1911.}}}\label{K_L03790-1} bei uns
               nachtmahlen, (8 Uhr), ſo würd’s uns ſehr freuen.\pend
           
\pstart
           \textsc{\textcolor{blue}{Hagemann}\pwindex{Hagemann, Carl 22.09.1871 – 24.12.1945@\textsc{Hagemann, Carl} (22.09.1871 – 24.12.1945), \emph{Theaterleiter/Theaterleiterin, Dramaturg/Dramaturgin}|pw}{}\ledrightnote{\textcolor{blue}{Carl Hagemann}}}, ſowie einige \textcolor{pink}{Wiener}\oindex{Wien@\textbf{Wien}|pw}{}\ledrightnote{\textcolor{pink}{Wien}}{ }\textcolor{blue}{Freunde}\pwindex{Salten, Felix 06.09.1869 – 08.10.1945@\textsc{Salten, Felix} (06.09.1869 – 08.10.1945), \emph{Schriftsteller/Schriftstellerin, Journalist/Journalistin, Chefredakteur/Chefredakteurin}|pwuv}\pwindex{Wassermann, Jakob 10.03.1873 – 01.01.1934@\textsc{Wassermann, Jakob} (10.03.1873 – 01.01.1934), \emph{Schriftsteller/Schriftstellerin}|pwuv}\pwindex{Beer-Hofmann, Richard 1866-07-11 – 1945-09-26@\textsc{Beer-Hofmann, Richard} (1866-07-11 – 1945-09-26), \emph{Schriftsteller/Schriftstellerin}|pwuv}{}\ledrightnote{{$\rightarrow$}\emph{\textcolor{blue}{Felix Salten}}{\newline}{$\rightarrow$}\emph{\textcolor{blue}{Jakob Wassermann}}{\newline}{$\rightarrow$}\emph{\textcolor{blue}{Richard Beer-Hofmann}}}
                ſind bei uns. Um genaue
               Antwort wird gebeten.\pend
           
\pstart
           Herzlichſt grüßt Sie{\\[\baselineskip]}Ihr{\\[\baselineskip]}\spacefill\mbox{A. S.}\pend
           \leftskip=0em{}\selectlanguage{ngerman}\endnumbering\briefempfaengerindex{Zweig, Stefan@\textsc{Zweig, Stefan}!zzzSchnitzler, Arthur@\emph{von Arthur Schnitzler}!1911-12-111@{1{[}1{]}. 1{[}2{]}. 1911}|)be}\mylabel{L03790h}  \normalsize

\doendnotes{C}
\bigskip
\vfill

\clearpage

\footnotesize

\lohead{\textsc{register}}

% Definiere theindex-Environment komplett neu ohne reledmac
\makeatletter
\renewenvironment{theindex}{%
  \section*{\indexname}%
  \setlength{\parindent}{0pt}%
  \setlength{\parskip}{0pt plus 0.3pt}%
  \let\item\@idxitem
}{%
  \clearpage
}
\makeatother

\IfFileExists{\jobname-pw.ind}{\input{\jobname-pw.ind}}{}

\end{document}

      