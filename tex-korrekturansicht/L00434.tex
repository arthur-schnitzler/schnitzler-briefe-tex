%% latex-korrekturansicht-vorspann.tex
%% Vorspann für die Korrekturansicht.
%% Lädt die gemeinsame Datei latex-vorspann.tex mit gesetztem Schalter.

\newif\ifkorrekturansicht
\korrekturansichttrue

\input{../tex-inputs/latex-vorspann}


               \section[Hugo von Hofmannsthal an Arthur Schnitzler, {[}28. 4. 1895{]}]{ Hugo von Hofmannsthal an Arthur Schnitzler, {[}28. 4. 1895{]}}\nopagebreak\mylabel{v}\rehead{ }\normalsize\beginnumbering\briefempfaengerindex{Schnitzler, Arthur@\textsc{Schnitzler, Arthur}!zzzHofmannsthal, Hugo von@\emph{von Hugo von Hofmannsthal}!1895-04-282@{{[}28. 4. 1895{]}}|(be} \toendnotes[C]{\smallbreak\pagebreak[2]} \Standort{CUL, Schnitzler, B 43.}
\physDesc{Brief, 1 Blatt (Briefpapier mit aufgeprägtem Wappen), 3 Seiten
\newline{}Handschrift: schwarze Tinte, deutsche Kurrent
\newline{}Schnitzler: mit Bleistift datiert: »28/4 95« und nummeriert: »70« }\buchAbdrucke{\weitereDrucke{Hugo von Hofmannsthal, Arthur Schnitzler: \emph{Briefwechsel}. Hg. Therese Nickl und Heinrich Schnitzler. Frankfurt am Main: \emph{S. Fischer} 1964, S. 53.} }\pstart{}{\pb}mein lieber
                        Arthur,\pend\pstart
           ich mache die beſten Fortſchritte, fahre jeden Tag nach \textcolor{pink}{Schönbrunn}{}\ledrightnote{\textcolor{pink}{Schloß Schönbrunn}} oder \textcolor{pink}{Döbling}{}\ledrightnote{\textcolor{pink}{XIX., Döbling}}
                    und kann ſchon 1 ½ Stunden ohne Ermüdung gehen. Morgen bin ich durch Familie
                    occupiert. Übermorgen will ich ſchon in der Früh zur \textcolor{blue}{Tini}{}\ledrightnote{\textcolor{blue}{Christine Schönberger}} fahren, vielleicht {\pb}dort das \textcolor{green}{Märchen}{}\ledrightnote{\textcolor{green}{Das Märchen der 672. Nacht}} fertigſchreiben oder wenn das ſchon fertig wäre,
                    eine Geſchichte des \textcolor{green}{Actäon}{}\ledrightnote{\textcolor{green}{Der neue Actäon}} anfangen. Ich hab
                    dem \textcolor{blue}{Richard}{}\ledrightnote{\textcolor{blue}{Richard Beer-Hofmann}} geſchrieben, ob er mir nicht
                    nachfahren will. Es wär ſehr ſchön, wenn Sie mit ihm ſich über ſo etwas einigen
                    würden. Den Nachmittag könnten wir dann wo anders hin, von der \textcolor{pink}{Brühl}{}\ledrightnote{\textcolor{pink}{Brühl}} aus.\pend
           \pstart
           {\pb}Jedenfalls rechne ich
                    darauf, mit Ihnen in der nächſten Woche mindeſtens einen Abend zu
                    verbringen.\pend
           \pstart
           Herzlich{\\[\baselineskip]} Ihr{\\[\baselineskip]}\spacefill\mbox{Hugo.}\pend
           \leftskip=0em{}\endnumbering\briefempfaengerindex{Schnitzler, Arthur@\textsc{Schnitzler, Arthur}!zzzHofmannsthal, Hugo von@\emph{von Hugo von Hofmannsthal}!1895-04-282@{{[}28. 4. 1895{]}}|)be}\mylabel{h}  \normalsize

\doendnotes{C}
\bigskip
\vfill

\clearpage

\footnotesize

\lohead{\textsc{register}}

% Definiere theindex-Environment komplett neu ohne reledmac
\makeatletter
\renewenvironment{theindex}{%
  \section*{\indexname}%
  \setlength{\parindent}{0pt}%
  \setlength{\parskip}{0pt plus 0.3pt}%
  \let\item\@idxitem
}{%
  \clearpage
}
\makeatother

\IfFileExists{\jobname-pw.ind}{\input{\jobname-pw.ind}}{}

\end{document}

      