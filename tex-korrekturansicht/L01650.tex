%% latex-korrekturansicht-vorspann.tex
%% Vorspann für die Korrekturansicht.
%% Lädt die gemeinsame Datei latex-vorspann.tex mit gesetztem Schalter.

\newif\ifkorrekturansicht
\korrekturansichttrue

\input{../tex-inputs/latex-vorspann}


               \section[Hermann Bahr an Arthur Schnitzler, 11. 1. 1907]{ Hermann Bahr an Arthur Schnitzler, 11. 1. 1907}\nopagebreak\mylabel{v}\rehead{ }\normalsize\beginnumbering\briefempfaengerindex{Schnitzler, Arthur@\textsc{Schnitzler, Arthur}!zzzBahr, Hermann@\emph{von Hermann Bahr}!1907-01-111@{11. 1. 1907}|(be} \toendnotes[C]{\smallbreak\pagebreak[2]} \Standort{CUL, Schnitzler, B 5b.}
\physDesc{Brief, 1 Blatt, 3 Seiten
\newline{}Handschrift Lisa Clarus: blaue Tinte, lateinische Kurrent\newline{}Handschrift Hermann Bahr: blaue Tinte (\noindent{}Unterschrift)\newline{}Ordnung: mit Bleistift von unbekannter Hand nummeriert:
                                    »143« }\buchAbdrucke{\weitereDrucke{Hermann Bahr, Arthur Schnitzler: \emph{Briefwechsel, Aufzeichnungen, Dokumente (1891–1931)}. Hg. Kurt Ifkovits und Martin Anton Müller. Göttingen: \emph{Wallstein} 2018, S. 387.} }\toendnotes[C]{\smallbreak}\pstart
           \raggedleft{}{\pb}\textcolor{pink}{Wien XIII/\textsubscript{7}}{}\ledrightnote{\textcolor{pink}{Ober Sankt Veit}} den 11. 1. 07.\pend
           \pstart\center{}Lieber Arthur!\pend\pstart
           Ich war \label{K_L01650_1v}\edtext{vierzehn Tage}{\lemma{\textnormal{\emph{vierzehn Tage}}}\Cendnote{\textnormal{Nachweisbar war \textcolor{blue}{Bahr} am 2. und 4. 1. 1907 auf dem
                     \textcolor{pink}{Semmering}.}}}\label{K_L01650_1h} auf dem \textcolor{pink}{Semmering}{}\ledrightnote{\textcolor{pink}{Semmering}} und bin nun seit Dienstag hier, für etwa zwölf Tage,
               mit dem Vorsatze:\pend
           \pstart
           1. Das \label{K_L01650_2v}\edtext{Regiebuch von \textcolor{green}{Hedda Gabler}{}\ledrightnote{\textcolor{green}{Hedda Gabler}}}{\lemma{\textnormal{\emph{Regiebuch … Gabler}}}\Cendnote{\textnormal{in \textcolor{blue}{Bahrs} Nachlass (\emph{Theatermuseum Wien}, HS VM 3683 Ba), die Premiere
                  von \textcolor{blue}{Ibsen}s \textcolor{green}{Stück} am 11. 3. 1907}}}\label{K_L01650_2h} zu machen,
               deren Proben am 24. d. beginnen sollen.\pend
           \pstart
           \introOben{}2.\introOben{}{ }\substVorne{}\textsuperscript{z}\substDazwischen{}Z\substHinten{}u versuchen, ob mein neues \textcolor{green}{Stück}{}\ledrightnote{→\textcolor{green}{Die gelbe Nachtigall}} schon so weit ist, dass sich mir ungefähr ein Szenarium ergibt,
               welches dann im Sommer ausgearbeitet werden soll, und 3. ein{\pb}mal mit Dir, \textcolor{blue}{Richard}{}\ledrightnote{\textcolor{blue}{Richard Beer-Hofmann}} und \textcolor{blue}{Salten}{}\ledrightnote{\textcolor{blue}{Felix Salten}} zusammen zu sein,
               einmal mit \textcolor{blue}{Kainz}{}\ledrightnote{\textcolor{blue}{Margarethe Kainz}}, gelegentlich auch \textcolor{blue}{Fred}{}\ledrightnote{\textcolor{blue}{W. Fred}} und \textcolor{blue}{Handl}{}\ledrightnote{\textcolor{blue}{Willi Handl}}
               zu sehen, sonst aber mich zu verstecken. Dies ist es was ich »incognito« nenne. Meine
               Absicht war, Dir vorzuschlagen, ob ich nicht nächste Woche einmal von \substVorne{}\textsuperscript{e}\substDazwischen{}E\substHinten{}ilf bis Drei bei Dir sein und dort vielleicht auch gleich \textcolor{blue}{Salten}{}\ledrightnote{\textcolor{blue}{Felix Salten}} und \textcolor{blue}{Richard}{}\ledrightnote{\textcolor{blue}{Richard Beer-Hofmann}} treffen
               könnte. Dass Du nun aber Sonntag Vormittag zu mir kommen willst, ist mir sehr
               erwünscht, stört mich gar nicht, freut mich riesig (ich kann Dir nur nichts zu {\pb}essen geben, weil ich keine Köchin habe) und wir
               können dann alles Mögliche besprechen.\pend
           \pstart
           Mit den herzlichsten Grüssen an Deine liebe \textcolor{blue}{Frau}{}\ledrightnote{→\textcolor{blue}{Olga Schnitzler}}{\\[\baselineskip]}Dein alter{\\[\baselineskip]}\spacefill\mbox{{[}hs. Bahr:{]} Hermann}\pend
           \leftskip=0em{}\pstart
           \noindent{}{[}hs. Clarus:{]} PS.{\\}»\textcolor{green}{Ringelspiel}{}\ledrightnote{\textcolor{green}{Brief an Arthur Schnitzler}}« und »\textcolor{green}{Grotesken}{}\ledrightnote{\textcolor{green}{Grotesken}}« hast Du
                  hoffentlich richtig bekommen?\pend
           \endnumbering\briefempfaengerindex{Schnitzler, Arthur@\textsc{Schnitzler, Arthur}!zzzBahr, Hermann@\emph{von Hermann Bahr}!1907-01-111@{11. 1. 1907}|)be}\mylabel{h}  \normalsize

\doendnotes{C}
\bigskip
\vfill

\clearpage

\footnotesize

\lohead{\textsc{register}}

% Definiere theindex-Environment komplett neu ohne reledmac
\makeatletter
\renewenvironment{theindex}{%
  \section*{\indexname}%
  \setlength{\parindent}{0pt}%
  \setlength{\parskip}{0pt plus 0.3pt}%
  \let\item\@idxitem
}{%
  \clearpage
}
\makeatother

\IfFileExists{\jobname-pw.ind}{\input{\jobname-pw.ind}}{}

\end{document}

      