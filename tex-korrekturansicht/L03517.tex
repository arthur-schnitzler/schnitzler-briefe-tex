%% latex-korrekturansicht-vorspann.tex
%% Vorspann für die Korrekturansicht.
%% Lädt die gemeinsame Datei latex-vorspann.tex mit gesetztem Schalter.

\newif\ifkorrekturansicht
\korrekturansichttrue

\input{../tex-inputs/latex-vorspann}


\renewcommand{\erwaehnteInstitutionen}{Institutionen: S. Fischer Verlag}
\renewcommand{\erwaehnteOrte}{Orte: Bendlerstraße, Berlin, Semmering, Sternwartestraße, Wien}
\renewcommand{\erwaehnteWerke}{Werke: ?? [Roman mit erotischen Schilderungen], Im Spiel der Sommerlüfte. In drei Aufzügen}
\section[ Paul Goldmann an Arthur Schnitzler, 19. 5. 1931]{Paul Goldmann an Arthur Schnitzler, 19. 5. 1931}
\nopagebreak\mylabel{v}
\rehead{ }\normalsize\beginnumbering\briefempfaengerindex{Schnitzler, Arthur@\textsc{Schnitzler, Arthur}!zzzGoldmann, Paul@\emph{von Paul Goldmann}!1931-05-191@{19. 5. 1931}|(be}
\toendnotes[C]{\smallbreak\pagebreak[2]}\Standort{DLA, A:Schnitzler, HS.NZ85.1.3176.}
\physDesc{Postkarte, 603 Zeichen
\newline{}Schreibmaschine
\newline{}Handschrift: lila Tinte, lateinische Kurrent (\noindent{}ein Komma und Unterschrift)
\newline{}Versand: 1) Stempel: »\nobreak{}Luftpost. Befördert Briefe – Zeitungen –
                                       Pakete\nobreak{}«.   2) Stempel: »\nobreak{}\oindex{Berlin@\textbf{Berlin}, \emph{P.PPLC}|pwk}Berlin SW 11, 19. 5. 31, 14—15\nobreak{}«. 
\newline{}Schnitzler: mit rotem Buntstift drei Unterstreichungen }\toendnotes[C]{\smallbreak}\pstart{}\textcolor{gray}{\textbf{\textit{{\pb}Dr. PAUL GOLDMANN}}}\pend{}\pstart{}\textcolor{gray}{\textbf{\textit{\textcolor{pink}{BENDLERSTR. 36}{}\ledrightnote{\textcolor{pink}{Bendlerstraße}}}}}\pend{}\pstart{}\textcolor{gray}{\textbf{\textit{\textcolor{pink}{BERLIN W.}{}\ledrightnote{\textcolor{pink}{Berlin}}}}}\pend{}
{\bigskip}\pstart{}Herrn\pend{}\pstart{}Dr. Arthur Schnitzler\pend{}\pstart{}\textcolor{pink}{\so{Wien}}{}\ledrightnote{\textcolor{pink}{Wien}}\pend{}\pstart{}\textcolor{pink}{\label{T_L03517-1v}\edtext{XVIII.}{\lemma{\textnormal{\emph{XVIII.}}}\Cendnote{\textnormal{korrigiert aus »XV111.«}}}\label{T_L03517-1h}
                     Sternwartstrasse 71}{}\ledrightnote{\textcolor{pink}{Sternwartestraße}}\pend{}
{\bigskip}
\pstart
           \centering{}{\pb}\textcolor{pink}{Berlin}{}\ledrightnote{\textcolor{pink}{Berlin}}, den 19. Mai 1931\pend
           
\pstart{}Lieber Freund,\pend
\pstart
           Ich danke Dir herzlichst für die so überraschend schnelle Übersendung der beiden
                  \label{K_L03517-1v}\edtext{\textcolor{green}{Bücher}{}\ledrightnote{{$\rightarrow$}\textcolor{green}{?? [Roman mit erotischen Schilderungen]}{\newline}{$\rightarrow$}\textcolor{green}{Im Spiel der Sommerlüfte. In drei Aufzügen}}}{\lemma{\textnormal{\emph{Bücher}}}\Cendnote{\textnormal{Es handelt sich um einen nicht zu
                  identifizierenden \textcolor{green}{Roman} und ein Schauspiel von \textcolor{blue}{Schnitzler}. Bei letzterem könnte es sich um den Dreiakter \emph{\textcolor{green}{Im Spiel der Sommerlüfte}} handeln, der bereits am 21. 12. 1929 bei \emph{\textcolor{brown}{S.
                     Fischer}} in \textcolor{pink}{Berlin} erschienen
                  war.}}}\label{K_L03517-1h}. Den Roman, den ich zurücksenden muss, werde ich so rasch als möglich
               lesen. Immerhin könnten einige Wochen vergehen\introOben{},\introOben{} und ich
               bitte Dich, trotzdem ganz sicher zu sein, dass D\substVorne{}\textsuperscript{i}\substDazwischen{}u\substHinten{} Dein \textcolor{green}{Buch}{}\ledrightnote{{$\rightarrow$}\textcolor{green}{?? [Roman mit erotischen Schilderungen]}} zurückbekommst. Für die Widmung in dem Exemplar Deines \textcolor{green}{Schauspiel}{}\ledrightnote{{$\rightarrow$}\textcolor{green}{Im Spiel der Sommerlüfte. In drei Aufzügen}}s danke
               ich Dir noch ganz besonders. Ich wünsche Dir angenehme Tage auf dem \label{K_L03517-2v}\edtext{\textcolor{pink}{Semmering}{}\ledrightnote{\textcolor{pink}{Semmering}}}{\lemma{\textnormal{\emph{Semmering}}}\Cendnote{\textnormal{\textcolor{blue}{Schnitzler} verbrachte die Tage um
                   seinen 69. Geburtstages am \textcolor{pink}{Semmering}, 
                  vom 13. 5. 1931 bis zum 
                  16. 5. 1931.}}}\label{K_L03517-2h}
               und verbleibe mit herzlichen Grüssen {\\}Dein {\\}{[}hs.:{]} \spacefill\mbox{Paul Goldmann.}\pend
           \endnumbering\briefempfaengerindex{Schnitzler, Arthur@\textsc{Schnitzler, Arthur}!zzzGoldmann, Paul@\emph{von Paul Goldmann}!1931-05-191@{19. 5. 1931}|)be}\mylabel{h}  \normalsize

\doendnotes{C}
\bigskip
\vfill

\clearpage

\footnotesize

\lohead{\textsc{register}}

% Definiere theindex-Environment komplett neu ohne reledmac
\makeatletter
\renewenvironment{theindex}{%
  \section*{\indexname}%
  \setlength{\parindent}{0pt}%
  \setlength{\parskip}{0pt plus 0.3pt}%
  \let\item\@idxitem
}{%
  \clearpage
}
\makeatother

\IfFileExists{\jobname-pw.ind}{\input{\jobname-pw.ind}}{}

\end{document}

      