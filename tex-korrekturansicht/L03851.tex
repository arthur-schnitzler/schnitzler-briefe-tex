%% latex-korrekturansicht-vorspann.tex
%% Vorspann für die Korrekturansicht.
%% Lädt die gemeinsame Datei latex-vorspann.tex mit gesetztem Schalter.

\newif\ifkorrekturansicht
\korrekturansichttrue

\input{../tex-inputs/latex-vorspann}


\section[Theodor Herzl an Arthur Schnitzler, 20. 2. 1895]{L03851 Theodor Herzl an Arthur Schnitzler, 20. 2. 1895}
\nopagebreak\mylabel{L03851v}
\rehead{ }\normalsize\beginnumbering\briefempfaengerindex{, @\textsc{, }!zzz, @\emph{von  }!1895-02-202@{20. 2. 1895}|(be}
\toendnotes[C]{\smallbreak\pagebreak[2]}\Standort{CUL, Schnitzler, B 39.}
\physDesc{Brief, 1 Blatt, 4 Seiten, 2910 Zeichen
\newline{}Handschrift: schwarze Tinte, lateinische Kurrent
\newline{}Ordnung: mit Bleistift von unbekannter Hand nummeriert: »30« }
\buchAbdrucke{\weitereDrucke{Theodor Herzl: \emph{Briefe und
                        autobiographische Notizen 1866–1895}. Bearbeitet von Johannes Wachten in Zusammenarbeit mit Chaya Harel, Daisy Tycho und Manfred Winkler. Berlin, Frankfurt am Main, Wien: \emph{Propyläen} 1983, S. 574–576 (Briefe und Tagebücher. Herausgegeben von Alex Bein, Hermann Greive, Moshe Schaerf, Julius H. Schoeps und Johannes Wachten, 1).} }\toendnotes[C]{\smallbreak}
\pstart
           {\pb}\textcolor{pink}{\textcolor{gray}{\textbf{Grand Hôtel}}}\oindex{Grand Hôtel@\textbf{Grand Hôtel}, \emph{Hotel}|pw}{}\ledrightnote{\textcolor{pink}{Grand Hôtel}}\hfill \textcolor{gray}{\textbf{le}}{ }20 / II \textcolor{gray}{\textbf{189}}5\pend
           
\pstart
           \textcolor{pink}{\textcolor{gray}{\textbf{Boulevard des Capucines, 12}}}\oindex{Grand Hôtel@\textbf{Grand Hôtel}, \emph{Hotel}|pw}{}\ledrightnote{\textcolor{pink}{Grand Hôtel}}\pend
           
\pstart
           \textcolor{pink}{\textcolor{gray}{\textbf{Paris }}}\oindex{Paris@\textbf{Paris}, \emph{Hauptstadt}|pw}{}\ledrightnote{\textcolor{pink}{Paris}}\pend
           
\pstart{}Mein liebster Freund,\pend\vspace{0.5em}
\pstart
           auf einer Jagdpause nur zwei Zeilen. Für Ihren lieben, sehr lieben \label{K_L03851-1v}\edtext{Brief}{\lemma{\textnormal{\emph{Brief}}}\Cendnote{\textnormal{rexXXXX18.2.1895}}}\label{K_L03851-1} habe ich ihnen schon \label{K_L03851-2v}\edtext{telegraphisch gedankt}{\lemma{\textnormal{\emph{telegraphisch gedankt}}}\Cendnote{\textnormal{Theodor Herzl an Arthur Schnitzler, 20. [2. 1895?].}}}\label{K_L03851-2}. Ich folge Ihrem Rath,
               mich wenigstens dem Director zu nennen – da man es nun eben leider mit niederen
               Menschen zu thun hat u. ich das \textcolor{green}{Stück}\pwindex{Herzl, Theodor 2.\,5.\,1860 Budapest – 3.\,7.\,1904 Edlach@\textsc{Herzl, Theodor} (2.\,5.\,1860 Budapest – 3.\,7.\,1904 Edlach), \emph{Schriftsteller, Journalist}!neue Ghetto. Schauspiel in vier Acten@\strich\emph{Das neue Ghetto. Schauspiel in vier Acten}|pwv}{}\ledrightnote{{$\rightarrow$}\emph{\textcolor{green}{Das neue Ghetto. Schauspiel in vier Acten}}} ebensogut ungeschrieben hätte lassen können, wenn es nicht einmal
               gelesen wird. Aber ist es für uns nicht sehr demüthigend zu denken, dass zu dieser
               Zeit ein wirklicher Schnabel mit ungelesenen Manuscripten herumvagabundirt
               u. verzweifelt. Und vielleicht hat er mehr Talent als der Pseudoschnabel! \pend
           
\pstart
           Also nennen Sie mich in Gottes Namen dem \textcolor{blue}{Müller
                  Guttenbrunn}\pwindex{Müller-Guttenbrunn, Adam 22.\,10.\,1852 Zăbrani – 5.\,1.\,1923 Wien@\textsc{Müller-Guttenbrunn, Adam} (22.\,10.\,1852 Zăbrani – 5.\,1.\,1923 Wien), \emph{Schriftsteller, Theaterleiter, Beamter}|pw}{}\ledrightnote{\textcolor{blue}{Adam Müller-Guttenbrunn}}, wenn er sein Ehrenwort gibt, das Maul zu halten. \textcolor{blue}{Müller}\pwindex{Müller-Guttenbrunn, Adam 22.\,10.\,1852 Zăbrani – 5.\,1.\,1923 Wien@\textsc{Müller-Guttenbrunn, Adam} (22.\,10.\,1852 Zăbrani – 5.\,1.\,1923 Wien), \emph{Schriftsteller, Theaterleiter, Beamter}|pw}{}\ledrightnote{\textcolor{blue}{Adam Müller-Guttenbrunn}} ziehe ich \textcolor{blue}{Bukovics}\pwindex{Bukovics, Emerich von 28.\,2.\,1844 Wien – 4.\,7.\,1905 ebd.@\textsc{Bukovics, Emerich von} (28.\,2.\,1844 Wien – 4.\,7.\,1905 ebd.), \emph{Journalist, Theaterleiter}|pw}{}\ledrightnote{\textcolor{blue}{Emerich von Bukovics}} vor, weil ich glaube, dass er sein Wort hält. \textcolor{blue}{Bukovics}\pwindex{Bukovics, Emerich von 28.\,2.\,1844 Wien – 4.\,7.\,1905 ebd.@\textsc{Bukovics, Emerich von} (28.\,2.\,1844 Wien – 4.\,7.\,1905 ebd.), \emph{Journalist, Theaterleiter}|pw}{}\ledrightnote{\textcolor{blue}{Emerich von Bukovics}} hat mir einmal sein Wort gebrochen
               (Annahme von »\textcolor{green}{Was wird man sagen}\pwindex{Herzl, Theodor 2.\,5.\,1860 Budapest – 3.\,7.\,1904 Edlach@\textsc{Herzl, Theodor} (2.\,5.\,1860 Budapest – 3.\,7.\,1904 Edlach), \emph{Schriftsteller, Journalist}!Was wird man sagen?@\strich\emph{Was wird man sagen?}|pw}{}\ledrightnote{\textcolor{green}{Was wird man sagen?}}«) und darauf
               ist er gestrichen u. gelöscht. An so was rühr' ich nicht mehr an. {\pb}\textcolor{blue}{Müller}\pwindex{Müller-Guttenbrunn, Adam 22.\,10.\,1852 Zăbrani – 5.\,1.\,1923 Wien@\textsc{Müller-Guttenbrunn, Adam} (22.\,10.\,1852 Zăbrani – 5.\,1.\,1923 Wien), \emph{Schriftsteller, Theaterleiter, Beamter}|pw}{}\ledrightnote{\textcolor{blue}{Adam Müller-Guttenbrunn}} ist mir auch darum lieber weil ich
               weiss, dass ich ihm zuwider bin, u. meine \uline{Sehnsucht nach
                  ehrlichen Bitternissen} wird so ein wenig befriedigt. Die ganze Illusion, dass
               ich mir Aufführung u. event. Erfolg mit dem \textcolor{green}{Stück}\pwindex{Herzl, Theodor 2.\,5.\,1860 Budapest – 3.\,7.\,1904 Edlach@\textsc{Herzl, Theodor} (2.\,5.\,1860 Budapest – 3.\,7.\,1904 Edlach), \emph{Schriftsteller, Journalist}!neue Ghetto. Schauspiel in vier Acten@\strich\emph{Das neue Ghetto. Schauspiel in vier Acten}|pwv}{}\ledrightnote{{$\rightarrow$}\emph{\textcolor{green}{Das neue Ghetto. Schauspiel in vier Acten}}} selbst erworben habe, geht dabei in die Brüche, denn
               vielleicht ist \textcolor{blue}{Müller}\pwindex{Müller-Guttenbrunn, Adam 22.\,10.\,1852 Zăbrani – 5.\,1.\,1923 Wien@\textsc{Müller-Guttenbrunn, Adam} (22.\,10.\,1852 Zăbrani – 5.\,1.\,1923 Wien), \emph{Schriftsteller, Theaterleiter, Beamter}|pw}{}\ledrightnote{\textcolor{blue}{Adam Müller-Guttenbrunn}} ein Opportunist
               geworden, u. rechnet mit der \textcolor{brown}{N. Fr. Pr}\orgindex{Neue Freie Presse@Neue Freie Presse|pw}{}\ledrightnote{\textcolor{brown}{Neue Freie Presse}}.\pend
           
\pstart
           Ueberschätze ich meine \textcolor{brown}{Zeitung}\orgindex{Neue Freie Presse@Neue Freie Presse|pwv}{}\ledrightnote{{$\rightarrow$}\emph{\textcolor{brown}{Neue Freie Presse}}}?
               Vielleicht! Aber verstehen Sie doch, dass das mein Trost in der Handwerkerei ist!\pend
           
\pstart
           Uebrigens wird mich vielleicht \textcolor{blue}{Müller}\pwindex{Müller-Guttenbrunn, Adam 22.\,10.\,1852 Zăbrani – 5.\,1.\,1923 Wien@\textsc{Müller-Guttenbrunn, Adam} (22.\,10.\,1852 Zăbrani – 5.\,1.\,1923 Wien), \emph{Schriftsteller, Theaterleiter, Beamter}|pw}{}\ledrightnote{\textcolor{blue}{Adam Müller-Guttenbrunn}} vor \uline{Ihnen} rehabilitiren, indem er das \textcolor{green}{Stück}\pwindex{Herzl, Theodor 2.\,5.\,1860 Budapest – 3.\,7.\,1904 Edlach@\textsc{Herzl, Theodor} (2.\,5.\,1860 Budapest – 3.\,7.\,1904 Edlach), \emph{Schriftsteller, Journalist}!neue Ghetto. Schauspiel in vier Acten@\strich\emph{Das neue Ghetto. Schauspiel in vier Acten}|pwv}{}\ledrightnote{{$\rightarrow$}\emph{\textcolor{green}{Das neue Ghetto. Schauspiel in vier Acten}}} ablehnt. Dann habe ich wenigstens
               Ihnen gegenüber Recht gehabt.\pend
           
\pstart
           Dann kommt \textcolor{pink}{Prag}\oindex{Prag@\textbf{Prag}, \emph{Land}|pw}{}\ledrightnote{\textcolor{pink}{Prag}}, dann werde ichs unter meinem
               Namen drucken lassen, dann werde ich mit dem Revolver die Aufführung
               erpressen. – –\pend
           
\pstart
           Sie wissen, dass ich scherze. Nach {\pb}\textcolor{pink}{Prag}\oindex{Prag@\textbf{Prag}, \emph{Land}|pw}{}\ledrightnote{\textcolor{pink}{Prag}} ist's aus.\pend
           
\pstart
           Haben Sie noch so lange Geduld mit mir!\pend
           
\pstart
           Warum hat Ihnen meine \textcolor{green}{Heimatkritik}\pwindex{Herzl, Theodor 2.\,5.\,1860 Budapest – 3.\,7.\,1904 Edlach@\textsc{Herzl, Theodor} (2.\,5.\,1860 Budapest – 3.\,7.\,1904 Edlach), \emph{Schriftsteller, Journalist}!Heimat« von Sudermann@\strich\emph{»Heimat« von Sudermann}|pwv}{}\ledrightnote{{$\rightarrow$}\emph{\textcolor{green}{»Heimat« von Sudermann}}} nicht gefallen? Schreiben Sie mir das sofort! War ich \textcolor{blue}{Sudermann}\pwindex{Sudermann, Hermann 30.\,9.\,1857 Macikai – 21.\,11.\,1928 Berlin@\textsc{Sudermann, Hermann} (30.\,9.\,1857 Macikai – 21.\,11.\,1928 Berlin), \emph{Schriftsteller}|pw}{}\ledrightnote{\textcolor{blue}{Hermann Sudermann}} zu günstig? Es ist schwer. Ich kenne
               diesen lieben, diesen ehemals lieben \textcolor{blue}{Menschen}\pwindex{Sudermann, Hermann 30.\,9.\,1857 Macikai – 21.\,11.\,1928 Berlin@\textsc{Sudermann, Hermann} (30.\,9.\,1857 Macikai – 21.\,11.\,1928 Berlin), \emph{Schriftsteller}|pwv}{}\ledrightnote{{$\rightarrow$}\emph{\textcolor{blue}{Hermann Sudermann}}} seit 8 Jahren. Ich hatte u. habe ihn noch gern.
               Ich finde seinen Erfolg übertrieben, aber ursprünglich gerechtfertigt. Im Erfolg ist
                  \uline{nie} das rechte Mass. Nun klagt er mir über seine
               Feinde. Ist aus alledem, namentlich aus dem Wunsch, seinen Erfolg nicht \uline{literarisch} gegen ihn sprechen zu lassen, eine
               Ueberschätzung geworden?\pend
           
\pstart
           Sagen Sie mir das.\pend
           
\pstart
           Pudelnärrisch ist, dass ich mich wegen der hiesigen \textcolor{violet}{Aufführung}\eventindex{Théâtre de la Renaissance@\textbf{Théâtre de la Renaissance}!Premiere von Magda@Premiere von Magda|pwv}{}\ledrightnote{{$\rightarrow$}\emph{\textcolor{violet}{Premiere von Magda}}}, resp. wegen meiner \textcolor{green}{Berichte}\pwindex{Herzl, Theodor 2.\,5.\,1860 Budapest – 3.\,7.\,1904 Edlach@\textsc{Herzl, Theodor} (2.\,5.\,1860 Budapest – 3.\,7.\,1904 Edlach), \emph{Schriftsteller, Journalist}!Sudermann in Paris@\strich\emph{Sudermann in Paris}|pw}\pwindex{Herzl, Theodor 2.\,5.\,1860 Budapest – 3.\,7.\,1904 Edlach@\textsc{Herzl, Theodor} (2.\,5.\,1860 Budapest – 3.\,7.\,1904 Edlach), \emph{Schriftsteller, Journalist}!Heimat« von Sudermann@\strich\emph{»Heimat« von Sudermann}|pw}{}\ledrightnote{\textcolor{green}{Sudermann in Paris}{\newline}\textcolor{green}{»Heimat« von Sudermann}} mit ihm überworfen habe. Er war \label{K_L03851-3v}\edtext{mit meinen \textcolor{green}{Berichten}\pwindex{Herzl, Theodor 2.\,5.\,1860 Budapest – 3.\,7.\,1904 Edlach@\textsc{Herzl, Theodor} (2.\,5.\,1860 Budapest – 3.\,7.\,1904 Edlach), \emph{Schriftsteller, Journalist}!Sudermann in Paris@\strich\emph{Sudermann in Paris}|pwv}\pwindex{Herzl, Theodor 2.\,5.\,1860 Budapest – 3.\,7.\,1904 Edlach@\textsc{Herzl, Theodor} (2.\,5.\,1860 Budapest – 3.\,7.\,1904 Edlach), \emph{Schriftsteller, Journalist}!Heimat« von Sudermann@\strich\emph{»Heimat« von Sudermann}|pwv}{}\ledrightnote{{$\rightarrow$}\emph{\textcolor{green}{Sudermann in Paris}}{\newline}{$\rightarrow$}\emph{\textcolor{green}{»Heimat« von Sudermann}}}}{\lemma{\textnormal{\emph{mit meinen Berichten}}}\Cendnote{\textnormal{\textcolor{blue}{Herzl}\pwindex{Herzl, Theodor 2.\,5.\,1860 Budapest – 3.\,7.\,1904 Edlach@\textsc{Herzl, Theodor} (2.\,5.\,1860 Budapest – 3.\,7.\,1904 Edlach), \emph{Schriftsteller, Journalist}|pwk} referierte zunächst die \textcolor{pink}{Pariser}\oindex{Paris@\textbf{Paris}, \emph{Hauptstadt}|pwk} Berichterstattung zu der \textcolor{pink}{französischen}\oindex{Frankreich@\textbf{Frankreich}|pwk}{ }\textcolor{violet}{Erstaufführung}\eventindex{Théâtre de la Renaissance@\textbf{Théâtre de la Renaissance}!Premiere von Magda@Premiere von Magda|pwkv} von \textcolor{blue}{Sudermanns}\pwindex{Sudermann, Hermann 30.\,9.\,1857 Macikai – 21.\,11.\,1928 Berlin@\textsc{Sudermann, Hermann} (30.\,9.\,1857 Macikai – 21.\,11.\,1928 Berlin), \emph{Schriftsteller}|pwk}{ }\textcolor{green}{Stück}\pwindex{Sudermann, Hermann 30.\,9.\,1857 Macikai – 21.\,11.\,1928 Berlin@\textsc{Sudermann, Hermann} (30.\,9.\,1857 Macikai – 21.\,11.\,1928 Berlin), \emph{Schriftsteller}!Heimat@\strich\emph{Heimat}|pwkv} (\emph{\textcolor{green}{Sudermann in Paris}\pwindex{Herzl, Theodor 2.\,5.\,1860 Budapest – 3.\,7.\,1904 Edlach@\textsc{Herzl, Theodor} (2.\,5.\,1860 Budapest – 3.\,7.\,1904 Edlach), \emph{Schriftsteller, Journalist}!Sudermann in Paris@\strich\emph{Sudermann in Paris}|pwk}}. In: \emph{\textcolor{green}{Neue Freie Presse}\pwindex{Neue Freie Presse@\emph{Neue Freie Presse}|pwk}}, Nr. 10.947, 14.\,2.\,1895,
                     Abendblatt, S. 3), die nicht nur postitive Töne enthielt, und
                  publizierte dann eine eigene lange Theaterkritik (\textcolor{blue}{Theodor Herzl}\pwindex{Herzl, Theodor 2.\,5.\,1860 Budapest – 3.\,7.\,1904 Edlach@\textsc{Herzl, Theodor} (2.\,5.\,1860 Budapest – 3.\,7.\,1904 Edlach), \emph{Schriftsteller, Journalist}|pwk}: \emph{\textcolor{green}{Pariser Theater (»Heimat« von Sudermann)}\pwindex{Herzl, Theodor 2.\,5.\,1860 Budapest – 3.\,7.\,1904 Edlach@\textsc{Herzl, Theodor} (2.\,5.\,1860 Budapest – 3.\,7.\,1904 Edlach), \emph{Schriftsteller, Journalist}!Heimat« von Sudermann@\strich\emph{»Heimat« von Sudermann}|pwk}}. In: \emph{\textcolor{green}{Neue Freie Presse}\pwindex{Neue Freie Presse@\emph{Neue Freie Presse}|pwk}}, Nr. 10.949,
                        16. 2. 1895, Morgenblatt, S. 1–2).}}}\label{K_L03851-3} nicht
               zufrieden – darauf habe ich ihm einen {\pb}\label{K_L03851-4v}\edtext{Absagebrief}{\lemma{\textnormal{\emph{Absagebrief}}}\Cendnote{\textnormal{Der Absagebrief \textcolor{blue}{Herzls}\pwindex{Herzl, Theodor 2.\,5.\,1860 Budapest – 3.\,7.\,1904 Edlach@\textsc{Herzl, Theodor} (2.\,5.\,1860 Budapest – 3.\,7.\,1904 Edlach), \emph{Schriftsteller, Journalist}|pwk} ist nicht überliefert, aber aus seinem Brief vom
                     28. 2. 1895 an \textcolor{blue}{Sudermann}\pwindex{Sudermann, Hermann 30.\,9.\,1857 Macikai – 21.\,11.\,1928 Berlin@\textsc{Sudermann, Hermann} (30.\,9.\,1857 Macikai – 21.\,11.\,1928 Berlin), \emph{Schriftsteller}|pwk} lässt sich die Kontroverse rekonstruieren: \textcolor{blue}{Herzl}\pwindex{Herzl, Theodor 2.\,5.\,1860 Budapest – 3.\,7.\,1904 Edlach@\textsc{Herzl, Theodor} (2.\,5.\,1860 Budapest – 3.\,7.\,1904 Edlach), \emph{Schriftsteller, Journalist}|pwk} hatte direkt nach der \textcolor{violet}{Premiere}\eventindex{Théâtre de la Renaissance@\textbf{Théâtre de la Renaissance}!Premiere von Magda@Premiere von Magda|pwkv} einen ersten \textcolor{green}{Bericht}\pwindex{Herzl, Theodor 2.\,5.\,1860 Budapest – 3.\,7.\,1904 Edlach@\textsc{Herzl, Theodor} (2.\,5.\,1860 Budapest – 3.\,7.\,1904 Edlach), \emph{Schriftsteller, Journalist}!Sudermann in Paris@\strich\emph{Sudermann in Paris}|pwkv} per Telegramm an die \emph{\textcolor{brown}{Neue Freie Presse}\orgindex{Neue Freie Presse@Neue Freie Presse|pwk}} geschickt, der am
                  14. 2. 1895 erschien und \textcolor{blue}{Sudermann}\pwindex{Sudermann, Hermann 30.\,9.\,1857 Macikai – 21.\,11.\,1928 Berlin@\textsc{Sudermann, Hermann} (30.\,9.\,1857 Macikai – 21.\,11.\,1928 Berlin), \emph{Schriftsteller}|pwk} verstimmte aufgrund der aus den \textcolor{pink}{Pariser}\oindex{Paris@\textbf{Paris}, \emph{Hauptstadt}|pwk} Zeitungen widergegebenen, zum Teil kritischen
                  Töne. Die per Post abgesendete und daher erst am
                     20. 2. 1895 publizierte ausführliche \textcolor{green}{Theaterkritik}\pwindex{Herzl, Theodor 2.\,5.\,1860 Budapest – 3.\,7.\,1904 Edlach@\textsc{Herzl, Theodor} (2.\,5.\,1860 Budapest – 3.\,7.\,1904 Edlach), \emph{Schriftsteller, Journalist}!Heimat« von Sudermann@\strich\emph{»Heimat« von Sudermann}|pwkv}{ }\textcolor{blue}{Herzls}\pwindex{Herzl, Theodor 2.\,5.\,1860 Budapest – 3.\,7.\,1904 Edlach@\textsc{Herzl, Theodor} (2.\,5.\,1860 Budapest – 3.\,7.\,1904 Edlach), \emph{Schriftsteller, Journalist}|pwk} nahm \textcolor{blue}{Sudermann}\pwindex{Sudermann, Hermann 30.\,9.\,1857 Macikai – 21.\,11.\,1928 Berlin@\textsc{Sudermann, Hermann} (30.\,9.\,1857 Macikai – 21.\,11.\,1928 Berlin), \emph{Schriftsteller}|pwk} zunächst nicht zur Kenntnis, was
                  wiederum \textcolor{blue}{Herzl}\pwindex{Herzl, Theodor 2.\,5.\,1860 Budapest – 3.\,7.\,1904 Edlach@\textsc{Herzl, Theodor} (2.\,5.\,1860 Budapest – 3.\,7.\,1904 Edlach), \emph{Schriftsteller, Journalist}|pwk} erbitterte. Die Verstimmung
                     konnte jedoch bald beigelegt werden, vgl. \emph{Theodor Herzl an Hermann Sudermann, 28. 2. 1895}. In: \emph{Briefe und
                           autobiographische Notizen 1866–1895}, S. 574–575.}}}\label{K_L03851-4} geschrieben. Das alles unter uns.
               Wir plaudern noch darüber, bis ich mehr Zeit habe.\pend
           
\pstart
           Leben Sie wohl und haben Sie viel viel Glück und Freude mit Ihrem \textcolor{green}{Stück}\pwindex{Schnitzler, Arthur 15. 5. 1862 Wien – 21. 10. 1931 ebd.@\textsc{Schnitzler, Arthur} (15. 5. 1862 Wien – 21. 10. 1931 ebd.), \emph{Schriftsteller, Mediziner}!Liebelei. Schauspiel in drei Akten@\strich\emph{Liebelei. Schauspiel in drei Akten}|pwv}{}\ledrightnote{{$\rightarrow$}\emph{\textcolor{green}{Liebelei. Schauspiel in drei Akten}}}! Sie sind kindisch, auch nur zu
               erwähnen, dass sie dem \textcolor{brown}{Deutschen Theater}\orgindex{Deutsches Theater Berlin@Deutsches Theater Berlin|pw}{}\ledrightnote{\textcolor{brown}{Deutsches Theater Berlin}} die \textcolor{brown}{Burg}\orgindex{Burgtheater@Burgtheater|pw}{}\ledrightnote{\textcolor{brown}{Burgtheater}}annahme mittheilten. Das war so
               selbstverständlich u. correct wie nur möglich.\pend
           
\pstart
           Herzlich der Ihrige {\\[\baselineskip]}\spacefill\mbox{Th H.}\pend
           \leftskip=0em{}
\pstart
           \noindent{}Kann ich Ihr \textcolor{green}{Stück}\pwindex{Schnitzler, Arthur 15. 5. 1862 Wien – 21. 10. 1931 ebd.@\textsc{Schnitzler, Arthur} (15. 5. 1862 Wien – 21. 10. 1931 ebd.), \emph{Schriftsteller, Mediziner}!Liebelei. Schauspiel in drei Akten@\strich\emph{Liebelei. Schauspiel in drei Akten}|pwv}{}\ledrightnote{{$\rightarrow$}\emph{\textcolor{green}{Liebelei. Schauspiel in drei Akten}}} nicht
                  vor der Aufführung lesen? Ich glaube, gegenwärtig finden Sie keinen, der Ihnen mit
                  besserer Meinung rathschlagt als ich.\pend
           \selectlanguage{ngerman}\endnumbering\briefempfaengerindex{, @\textsc{, }!zzz, @\emph{von  }!1895-02-202@{20. 2. 1895}|)be}\mylabel{L03851h}
\begin{anhang}
\end{anhang}\normalsize

\doendnotes{C}
\bigskip
\vfill

\clearpage

\footnotesize

\lohead{\textsc{register}}

% Definiere theindex-Environment komplett neu ohne reledmac
\makeatletter
\renewenvironment{theindex}{%
  \section*{\indexname}%
  \setlength{\parindent}{0pt}%
  \setlength{\parskip}{0pt plus 0.3pt}%
  \let\item\@idxitem
}{%
  \clearpage
}
\makeatother

\IfFileExists{\jobname-pw.ind}{\input{\jobname-pw.ind}}{}

\end{document}

      