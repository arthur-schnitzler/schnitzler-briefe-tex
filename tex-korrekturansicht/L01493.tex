%% latex-korrekturansicht-vorspann.tex
%% Vorspann für die Korrekturansicht.
%% Lädt die gemeinsame Datei latex-vorspann.tex mit gesetztem Schalter.

\newif\ifkorrekturansicht
\korrekturansichttrue

\input{../tex-inputs/latex-vorspann}


               \section[Arthur Schnitzler an Hermann Bahr, 23. 1. 1905]{ Arthur Schnitzler an Hermann Bahr, 23. 1. 1905}\nopagebreak\mylabel{v}\rehead{ }\normalsize\beginnumbering\briefempfaengerindex{Bahr, Hermann@\textsc{Bahr, Hermann}!zzzSchnitzler, Arthur@\emph{von Arthur Schnitzler}!1905-01-231@{23. 1. 1905}|(be} \toendnotes[C]{\smallbreak\pagebreak[2]} \Standort{TMW, HS AM 23371 Ba.}
\physDesc{Brief, 1 Blatt, 1 Seite
\newline{}Handschrift: schwarze Tinte, deutsche Kurrent\newline{}Ordnung: Lochung }\buchAbdrucke{\weitereDrucke{1) \emph{23. 1. 1905.} In: Arthur Schnitzler: \emph{The Letters of Arthur Schnitzler to Hermann Bahr}. Edited, annotated, and with an introduction, by Donald G.
                        Daviau. Chapel Hill: \emph{The University of North Carolina Press} 1978, S. 88 (University of North Carolina studies in the Germanic languages
                        and literatures, 89).} \weitereDrucke{2) Hermann Bahr, Arthur Schnitzler: \emph{Briefwechsel, Aufzeichnungen, Dokumente (1891–1931)}. Hg. Kurt Ifkovits und Martin Anton Müller. Göttingen: \emph{Wallstein} 2018, S. 339.} }\toendnotes[C]{\smallbreak}\pstart
           \raggedleft{}{\pb}\textsc{\textcolor{pink}{Wien XVIII.}{}\ledrightnote{\textcolor{pink}{XVIII., Währing}}}{\\}am \introOben{}23/1 905\introOben{}\pend
           \pstart
           leider, mein lieber Hermann, hab ich gar nichts rechtes luſtiges,
               kurzes, ungedrucktes, zur Lectüre geeignetes – aber ſehen möcht ich dich je eher je
               lieber. Hoffentlich nächſte Woche. Und So{\geminationn}tag hörſt du
               dir wohl auch die \label{K_L01493_1v}\edtext{\textcolor{blue}{Mahler}{}\ledrightnote{\textcolor{blue}{Gustav Mahler}} Lieder}{\lemma{\textnormal{\emph{Mahler Lieder}}}\Cendnote{\textnormal{\emph{\textcolor{green}{Des Knaben Wunderhorn}} am
                     29. 1. 1905 im \textcolor{pink}{Bösendorfer-Saal}.}}}\label{K_L01493_1h} an? \textcolor{blue}{Wir}{}\ledrightnote{→\textcolor{blue}{Olga Schnitzler}} grüßen dich beide.\pend
           \pstart
           Herzlichſt dein{\\[\baselineskip]}\spacefill\mbox{Arthur}\pend
           \leftskip=0em{}\endnumbering\briefempfaengerindex{Bahr, Hermann@\textsc{Bahr, Hermann}!zzzSchnitzler, Arthur@\emph{von Arthur Schnitzler}!1905-01-231@{23. 1. 1905}|)be}\mylabel{h}  \normalsize

\doendnotes{C}
\bigskip
\vfill

\clearpage

\footnotesize

\lohead{\textsc{register}}

% Definiere theindex-Environment komplett neu ohne reledmac
\makeatletter
\renewenvironment{theindex}{%
  \section*{\indexname}%
  \setlength{\parindent}{0pt}%
  \setlength{\parskip}{0pt plus 0.3pt}%
  \let\item\@idxitem
}{%
  \clearpage
}
\makeatother

\IfFileExists{\jobname-pw.ind}{\input{\jobname-pw.ind}}{}

\end{document}

      