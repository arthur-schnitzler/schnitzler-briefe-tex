%% latex-korrekturansicht-vorspann.tex
%% Vorspann für die Korrekturansicht.
%% Lädt die gemeinsame Datei latex-vorspann.tex mit gesetztem Schalter.

\newif\ifkorrekturansicht
\korrekturansichttrue

\input{../tex-inputs/latex-vorspann}


\renewcommand{\erwaehntePersonen}{Personen: Samuel Fischer, Hans Jacob, Felix Salten, Franz Werfel, Paul Zsolnay}
\renewcommand{\erwaehnteInstitutionen}{Institutionen: Paul Zsolnay Verlag, S. Fischer Verlag}
\renewcommand{\erwaehnteOrte}{Orte: Berlin, Paris, Wien, XVIII., Währing}
\renewcommand{\erwaehnteWerke}{Werke: Verdi}
\section[ Arthur Schnitzler an Felix Salten, 10. 12. 1923]{Arthur Schnitzler an Felix Salten, 10. 12. 1923}
\nopagebreak\mylabel{v}
\rehead{ }\normalsize\beginnumbering\briefempfaengerindex{Salten, Felix@\textsc{Salten, Felix}!zzzSchnitzler, Arthur@\emph{von Arthur Schnitzler}!1923-12-101@{10. 12. 1923}|(be}
\toendnotes[C]{\smallbreak\pagebreak[2]}\Standort{DLA, A:Schnitzler, HS.NZ85.1.1751.}
\physDesc{Brief, Durchschlag, 1 Blatt, 1 Seite, 918 Zeichen
\newline{}maschinell
\newline{}Handschrift: roter Buntstift, lateinische Kurrent (\noindent{}drei Unterstreichungen und in der linken oberen Ecke Vermerk: »\textcolor{blue}{Salten}«)}\toendnotes[C]{\smallbreak}
\pstart
           \raggedleft{}{\pb}10. 12. 1923.\pend
           
\pstart{}Lieber,\pend
\pstart
           \label{K_L02948-1v}\edtext{gestern war \textcolor{blue}{Hans
                  Jacob}{}\ledrightnote{\textcolor{blue}{Hans Jacob}} bei mir}{\lemma{\textnormal{\emph{gestern … mir}}}\Cendnote{\textnormal{siehe A. S.: \emph{Tagebuch}, 9. 12. 1923}}}\label{K_L02948-1h}, von dem ich Ihnen neulich sprach und der mir in meinen Verhandlungen mit \textcolor{blue}{\textcolor{brown}{S. Fischer}{}\ledrightnote{\textcolor{brown}{S. Fischer Verlag}}}{}\ledrightnote{\textcolor{blue}{Samuel Fischer}} in der letzten Zeit ganz unschätzbare Dienste geleistet hat. Das Gespräch kam
               begreiflicherweise auch auf hiesige Verlagsgründungen, eine Frage, die mich momentan
               aus in Ihnen bekannten Gründen besonders interessiert, ist insbesondere die
               Angliederung eines Theatervertriebs an den \label{K_L02948-2v}\edtext{Buchverlag, den \textcolor{blue}{Zsolnay}{}\ledrightnote{\textcolor{blue}{Paul Zsolnay}}
               zu gründen gedenkt}{\lemma{\textnormal{\emph{Buchverlag, … gedenkt}}}\Cendnote{\textnormal{\textcolor{blue}{Paul Zsolnay}s Bemühungen um die Gründung
                  eines eigenen Verlags manifestierten sich in den kommenden Wochen. Im April 1924 erschien das erste Buch im \emph{\textcolor{brown}{Paul Zsolnay Verlag}}: \textcolor{blue}{Franz Werfel}s \emph{\textcolor{green}{Verdi}}.}}}\label{K_L02948-2h}. Aber
               auch allerlei anderes kam zur Sprache und \textcolor{blue}{Hans
                  Jacob}{}\ledrightnote{\textcolor{blue}{Hans Jacob}} berichtete mir viel, was, wie ich glaube, auch für \textcolor{blue}{Z.}{}\ledrightnote{\textcolor{blue}{Paul Zsolnay}} mancherlei Interesse haben könnte. Ich will Sie heute nur fragen, lieber, ob Sie einmal für \textcolor{blue}{Hans Jacob}{}\ledrightnote{\textcolor{blue}{Hans Jacob}} (der für einige, vielleicht längere
               Zeit aus \textcolor{pink}{Berlin}{}\ledrightnote{\textcolor{pink}{Berlin}} hier ist) eine halbe Stunde Zeit
               haben. Er würde besonderen Wert darauf legen Sie zu sprechen. Darf ich ihm eine
               günstige Botschaft bestellen?\pend
           \pstart Auf bald und sehr herzliche Grüsse\pend{}{\bigskip}
\pstart
           \noindent{}Herrn Felix Salten,\pend
           
\pstart
           \textcolor{pink}{Wien XVIII.}{}\ledrightnote{\textcolor{pink}{XVIII., Währing}}\pend
           \endnumbering\briefempfaengerindex{Salten, Felix@\textsc{Salten, Felix}!zzzSchnitzler, Arthur@\emph{von Arthur Schnitzler}!1923-12-101@{10. 12. 1923}|)be}\mylabel{h}  \normalsize

\doendnotes{C}
\bigskip
\vfill

\clearpage

\footnotesize

\lohead{\textsc{register}}

% Definiere theindex-Environment komplett neu ohne reledmac
\makeatletter
\renewenvironment{theindex}{%
  \section*{\indexname}%
  \setlength{\parindent}{0pt}%
  \setlength{\parskip}{0pt plus 0.3pt}%
  \let\item\@idxitem
}{%
  \clearpage
}
\makeatother

\IfFileExists{\jobname-pw.ind}{\input{\jobname-pw.ind}}{}

\end{document}

      