%% latex-korrekturansicht-vorspann.tex
%% Vorspann für die Korrekturansicht.
%% Lädt die gemeinsame Datei latex-vorspann.tex mit gesetztem Schalter.

\newif\ifkorrekturansicht
\korrekturansichttrue

\input{../tex-inputs/latex-vorspann}


               \section[Hugo von Hofmannsthal an Arthur Schnitzler, 22. 12. 1900]{ Hugo von Hofmannsthal an Arthur Schnitzler, 22. 12. 1900}\nopagebreak\mylabel{v}\rehead{ }\normalsize\beginnumbering\briefempfaengerindex{Schnitzler, Arthur@\textsc{Schnitzler, Arthur}!zzzHofmannsthal, Hugo von@\emph{von Hugo von Hofmannsthal}!1900-12-221@{22. 12. 1900}|(be} \toendnotes[C]{\smallbreak\pagebreak[2]} \Standort{CUL, Schnitzler, B 43.}
\physDesc{Postkarte
\newline{}Handschrift: schwarze Tinte, deutsche Kurrent\newline{}Versand: 1) Rohrpost 2) Stempel: »\nobreak{}\oindex{III., Landstrasse@\textbf{III., Landstraße}, \emph{Bezirk (A.BZK)}|pwk}Wien 3/3, 22 XII 00, 5 30N\nobreak{}«. 3) Stempel: »\nobreak{}\oindex{IX., Alsergrund@\textbf{IX., Alsergrund}, \emph{Bezirk (A.BZK)}|pwk}Wien 9/2, \textcolor{gray}{22} XII 00, 5 {[}40N{]}\nobreak{}«. 
\newline{}Schnitzler: mit Bleistift datiert: »25/12 900« \newline{}Ordnung: mit Bleistift von unbekannter Hand mehrfach nummeriert, diese
                                 gestrichen und zuletzt geändert zu: »170« }\buchAbdrucke{\weitereDrucke{Hugo von Hofmannsthal, Arthur Schnitzler: \emph{Briefwechsel}. Hg. Therese Nickl und Heinrich Schnitzler. Frankfurt am Main: \emph{S. Fischer} 1964, S. 145.} }\toendnotes[C]{\smallbreak}\pstart{}{\pb}\textsc{Herrn D\textsuperscript{r} Arthur Schnitzler}\pend{}\pstart{}\textsc{\textcolor{pink}{IX. Franckgasse 1.}{}\ledrightnote{\textcolor{pink}{Frankgasse}}}\pend{}\pstart{}\textsc{\textcolor{pink}{Wien}{}\ledrightnote{\textcolor{pink}{Wien}}}\pend{}{\bigskip}\pstart
           \noindent{}{\pb}lieber Arthur, ich bin auch morgen Sonntag wieder bei
                  \textcolor{blue}{Richard}{}\ledrightnote{\textcolor{blue}{Richard Beer-Hofmann}}, vielleicht daſs Sie gegen
                  ¾ 8 hinko{\geminationm}en, mich abzuholen oder
               gemeinſam dortzubleiben, das wäre ſehr ſchön.\pend
           \pstart
           Herzlich{\\[\baselineskip]}\spacefill\mbox{Hugo}\pend
           \leftskip=0em{}\pstart
           Samstag.\pend
           \pstart
           Man kann Sie nun ruhig den \label{K_L01088_1v}\edtext{\textcolor{blue}{\textsc{Kotzebue}}{}\ledrightnote{→\textcolor{blue}{August von Kotzebue}} der Novelle}{\lemma{\textnormal{\emph{Kotzebue der Novelle}}}\Cendnote{\textnormal{Anlässlich der
                     bevorstehenden Veröffentlichung von \emph{\textcolor{green}{Lieutenant
                        Gustl}} am 25. 12. 1900 eine scherzhafte Bemerkung, \textcolor{blue}{August von Kotzebue} hat ein sehr
                     umfangreiches Theaterwerk von über 200 Stücken hinterlassen.}}}\label{K_L01088_1h} nennen.\pend
           \endnumbering\briefempfaengerindex{Schnitzler, Arthur@\textsc{Schnitzler, Arthur}!zzzHofmannsthal, Hugo von@\emph{von Hugo von Hofmannsthal}!1900-12-221@{22. 12. 1900}|)be}\mylabel{h}  \normalsize

\doendnotes{C}
\bigskip
\vfill

\clearpage

\footnotesize

\lohead{\textsc{register}}

% Definiere theindex-Environment komplett neu ohne reledmac
\makeatletter
\renewenvironment{theindex}{%
  \section*{\indexname}%
  \setlength{\parindent}{0pt}%
  \setlength{\parskip}{0pt plus 0.3pt}%
  \let\item\@idxitem
}{%
  \clearpage
}
\makeatother

\IfFileExists{\jobname-pw.ind}{\input{\jobname-pw.ind}}{}

\end{document}

      