%% latex-korrekturansicht-vorspann.tex
%% Vorspann für die Korrekturansicht.
%% Lädt die gemeinsame Datei latex-vorspann.tex mit gesetztem Schalter.

\newif\ifkorrekturansicht
\korrekturansichttrue

\input{../tex-inputs/latex-vorspann}


               \section[Hugo von Hofmannsthal an Arthur Schnitzler, 7. 8. 1898]{ Hugo von Hofmannsthal an Arthur Schnitzler,
                    7. 8. 1898}\nopagebreak\mylabel{v}\rehead{ }\normalsize\beginnumbering\briefempfaengerindex{Schnitzler, Arthur@\textsc{Schnitzler, Arthur}!zzzHofmannsthal, Hugo von@\emph{von Hugo von Hofmannsthal}!1898-08-071@{7. 8. 1898}|(be} \toendnotes[C]{\smallbreak\pagebreak[2]} \Standort{CUL, Schnitzler, B 43.}
\physDesc{Telegramm
\newline{}Handschrift einer Schreibkraft: Bleistift, deutsche Kurrent\newline{}Versand: »\noindent{}\begin{center}\textcolor{pink}{\textcolor{gray}{\textbf{\textbf{K. B. Telegraphenſtation}}}{ }\textcolor{gray}{\textbf{\textit{\textcolor{pink}{TEGERNSEE}}}}}\end{center}{ / }\textcolor{gray}{\textbf{Nr.}}{ }310{ / }\textcolor{gray}{\textbf{Aufgegeben in}}{ }\textcolor{pink}{Hinterbrühl}\hfill \textcolor{gray}{\textbf{Abgefertigt}}{ }7/8{ }\textcolor{gray}{\textbf{189}}8{ }11\textcolor{gray}{\textbf{U}}10\textcolor{gray}{\textbf{M}}{ }V\textcolor{gray}{\textbf{Mttg.}}{ / }\textcolor{gray}{\textbf{Nr.}}{ }103 0 30\textcolor{gray}{\textbf{W.}}{ / }\textcolor{gray}{\textbf{den}}{ }7/8 \textcolor{gray}{\textbf{189}}8{ }9\textcolor{gray}{\textbf{U}}–\textcolor{gray}{\textbf{M}}{ }V\textcolor{gray}{\textbf{Mttg.}}« \newline{}Ordnung: mit Bleistift von unbekannter Hand nummeriert:
                                    »121« }\buchAbdrucke{\weitereDrucke{Hugo von Hofmannsthal, Arthur Schnitzler: \emph{Briefwechsel}. Hg. Therese Nickl und Heinrich Schnitzler. Frankfurt am Main: \emph{S. Fischer} 1964, S. 110.} }\pstart{}{\pb}Schnitzler\pend{}\pstart{}\textcolor{pink}{Hotel Post}{}\ledrightnote{\textcolor{pink}{Hotel Post}} über \textcolor{pink}{München}{}\ledrightnote{\textcolor{pink}{München}}\pend{}{\bigskip}\pstart
           \noindent{}\raggedleft{}{\pb}\textcolor{gray}{\textbf{\textit{\textcolor{pink}{TEGERNSEE}{}\ledrightnote{\textcolor{pink}{Tegernsee}}}}}\pend
           \pstart
           \noindent{}Bin aus vielen Gründen ſchon Mittwoch{ }Abend in \textcolor{pink}{Baſel}{}\ledrightnote{\textcolor{pink}{Basel}} bitte Drahtantwort
                        \textcolor{pink}{Hinterbrühl}{}\ledrightnote{\textcolor{pink}{Hinterbrühl}} ob ſie ſpäteſtens
                        Donnerſtag auch dort ſein können und welches Gaſthaus\pend
           \pstart \spacefill\mbox{Hugo}\pend{}\pstart
           \noindent{}{\pb}\textcolor{pink}{\textsc{Hotel National}}{}\ledrightnote{\textcolor{pink}{Hotel National}} b. d. Bahn\pend
           \endnumbering\briefempfaengerindex{Schnitzler, Arthur@\textsc{Schnitzler, Arthur}!zzzHofmannsthal, Hugo von@\emph{von Hugo von Hofmannsthal}!1898-08-071@{7. 8. 1898}|)be}\mylabel{h}  \normalsize

\doendnotes{C}
\bigskip
\vfill

\clearpage

\footnotesize

\lohead{\textsc{register}}

% Definiere theindex-Environment komplett neu ohne reledmac
\makeatletter
\renewenvironment{theindex}{%
  \section*{\indexname}%
  \setlength{\parindent}{0pt}%
  \setlength{\parskip}{0pt plus 0.3pt}%
  \let\item\@idxitem
}{%
  \clearpage
}
\makeatother

\IfFileExists{\jobname-pw.ind}{\input{\jobname-pw.ind}}{}

\end{document}

      