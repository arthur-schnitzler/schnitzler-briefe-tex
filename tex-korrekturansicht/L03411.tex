%% latex-korrekturansicht-vorspann.tex
%% Vorspann für die Korrekturansicht.
%% Lädt die gemeinsame Datei latex-vorspann.tex mit gesetztem Schalter.

\newif\ifkorrekturansicht
\korrekturansichttrue

\input{../tex-inputs/latex-vorspann}


\renewcommand{\erwaehnteOrte}{Orte: Edmund-Weiß-Gasse 7, Mendelgebirge, Monte Penegal, Südtirol, Wien, XVIII., Währing}
\renewcommand{\erwaehnteWerke}{Werke: Tagebuch}
\section[Felix Salten an Arthur Schnitzler, {[}zwischen 25. 8. und 3. 9. 1905?{]}]{Felix Salten an Arthur Schnitzler,
               {[}zwischen 25. 8. und 3. 9. 1905?{]}}
\nopagebreak\mylabel{v}
\rehead{ }\normalsize\beginnumbering\briefempfaengerindex{Schnitzler, Arthur@\textsc{Schnitzler, Arthur}!zzzSalten, Felix@\emph{von Felix Salten}!1@{{[}zwischen 25. 8. und 3. 9. 1905?{]}}|(be}
\toendnotes[C]{\smallbreak\pagebreak[2]}\Standort{CUL, Schnitzler, B 89, B 1.}
\physDesc{Bildpostkarte, 66 Zeichen
\newline{}Handschrift: Bleistift, lateinische Kurrent
\newline{}Ordnung: mit Bleistift von unbekannter Hand nummeriert: »203« }\toendnotes[C]{\smallbreak}\pstart{}{\pb}Herrn D\textsuperscript{r} Arthur Schnitzler\pend{}\pstart{}\textcolor{pink}{Wien XVIII.}{}\ledrightnote{\textcolor{pink}{XVIII., Währing}}\pend{}\pstart{}\textcolor{pink}{Spöttelgasse 7}{}\ledrightnote{\textcolor{pink}{Edmund-Weiß-Gasse 7}}\pend{}
{\bigskip}
\pstart
           \noindent{}\centering{}{\pb}\textcolor{gray}{\textbf{Auf dem \label{K_L03411-1v}\edtext{\textcolor{pink}{Penegal}{}\ledrightnote{\textcolor{pink}{Monte Penegal}}}{\lemma{\textnormal{\emph{Penegal}}}\Cendnote{\textnormal{Die Postkarte ist undatiert und
                        der Poststempel nicht zu entziffern, weswegen externe Faktoren für die
                        Datierung herangezogen werden müssen. Innerhalb der weitgehend
                        chronologischen Reihenfolge der überlieferten Korrespondenzstücke \textcolor{blue}{Salten}s an \textcolor{blue}{Schnitzler} liegt die Karte im
                           Sommer 1905. Für den 23. 8. 1905 erwähnt \textcolor{blue}{Schnitzler}s \emph{\textcolor{green}{Tagebuch}}, dass \textcolor{blue}{Salten} nach
                           \textcolor{pink}{Südtirol} fahre. Für den 4. 9. 1905 ist
                        die nächste Begegnung festgehalten, sodass die Karte im dazwischenliegenden
                        Zeitraum zu verorten sein dürfte.}}}\label{K_L03411-1h} (\textcolor{pink}{Mendel}{}\ledrightnote{\textcolor{pink}{Mendelgebirge}}).}}\pend
           
\pstart
           herzlichst Ihr \spacefill\mbox{S.}\pend
           \endnumbering\briefempfaengerindex{Schnitzler, Arthur@\textsc{Schnitzler, Arthur}!zzzSalten, Felix@\emph{von Felix Salten}!1905-08-251@{{[}zwischen 25. 8. und 3. 9. 1905?{]}}|)be}\mylabel{h}  \normalsize

\doendnotes{C}
\bigskip
\vfill

\clearpage

\footnotesize

\lohead{\textsc{register}}

% Definiere theindex-Environment komplett neu ohne reledmac
\makeatletter
\renewenvironment{theindex}{%
  \section*{\indexname}%
  \setlength{\parindent}{0pt}%
  \setlength{\parskip}{0pt plus 0.3pt}%
  \let\item\@idxitem
}{%
  \clearpage
}
\makeatother

\IfFileExists{\jobname-pw.ind}{\input{\jobname-pw.ind}}{}

\end{document}

      