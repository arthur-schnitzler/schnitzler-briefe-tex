%% latex-korrekturansicht-vorspann.tex
%% Vorspann für die Korrekturansicht.
%% Lädt die gemeinsame Datei latex-vorspann.tex mit gesetztem Schalter.

\newif\ifkorrekturansicht
\korrekturansichttrue

\input{../tex-inputs/latex-vorspann}


         
         \renewcommand{\erwaehntePersonen}{Personen: Samuel Fischer, Hugo von Hofmannsthal}
         \renewcommand{\erwaehnteInstitutionen}{Institutionen: Volkstheater}
         \renewcommand{\erwaehnteOrte}{Orte: Berlin, Frankgasse, IX., Alsergrund, Passauerstraße, Wien}
         \renewcommand{\erwaehnteWerke}{Werke: Anatol, Die Gouvernante, Theater, Kunst und Literatur [Gouvernante], Wiener Allgemeine Zeitung}
               \section[Paul Goldmann, Marie Glümer, Auguste Chlum und Moritz Coschell an Arthur Schnitzler, 11. 1. 1900]{Paul Goldmann, Marie Glümer, Auguste Chlum und Moritz Coschell an
               Arthur Schnitzler, 11. 1. 1900}\nopagebreak\mylabel{v}\rehead{ }\normalsize\beginnumbering\briefempfaengerindex{Schnitzler, Arthur@\textsc{Schnitzler, Arthur}!zzzCoschell, Moritz@\emph{von Moritz Coschell}!1900-01-111@{11. 1. 1900}|(be}\briefempfaengerindex{Schnitzler, Arthur@\textsc{Schnitzler, Arthur}!zzzChlum, Auguste@\emph{von Auguste Chlum}!1900-01-111@{11. 1. 1900}|(be}\briefempfaengerindex{Schnitzler, Arthur@\textsc{Schnitzler, Arthur}!zzzGluemer, Marie@\emph{von Marie Glümer}!1900-01-111@{11. 1. 1900}|(be}\briefempfaengerindex{Schnitzler, Arthur@\textsc{Schnitzler, Arthur}!zzzGoldmann, Paul@\emph{von Paul Goldmann}!1900-01-111@{11. 1. 1900}|(be} \toendnotes[C]{\smallbreak\pagebreak[2]} \Standort{DLA, A:Schnitzler, HS.NZ85.1.3170.}
\physDesc{Postkarte
\newline{}Handschrift Paul Goldmann: 1) schwarze Tinte, deutsche Kurrent (\noindent{}auch die Unterstreichung von »\textcolor{brown}{Volkstheater}« und das Fußnotenzeichen stammen von \textcolor{blue}{Goldmann})\hspace{1em}2) schwarze Tinte, lateinische Kurrent (\noindent{}Adresse)\hspace{1em}\newline{}Handschrift Marie Glümer: schwarze Tinte, deutsche Kurrent\newline{}Handschrift Auguste Chlum: schwarze Tinte, deutsche Kurrent\newline{}Handschrift Moritz Coschell: schwarze Tinte, lateinische Kurrent\newline{}Versand: Stempel: »\nobreak{}\oindex{Berlin@\textbf{Berlin}, \emph{https://www.geonames.org/ontologyP.PPLC}|pwk}Berlin, W., 13. 1. 00, 12\textcolor{gray}{–}1V.\nobreak{}«. Stempel: »\nobreak{}\oindex{IX., Alsergrund@\textbf{IX., Alsergrund}, \emph{Bezirk (A.BZK)}|pwk}Wien 9/3 72, 14. 1{[}.{]} 00, 9{[}.V{]}, \textcolor{gray}{Bestellt}\nobreak{}«.  }\toendnotes[C]{\smallbreak}\pstart{}{\pb}\textsc{Herrn}\pend{}\pstart{}\textsc{Dr. Arthur Schnitzler}\pend{}\pstart{}\textsc{\textcolor{pink}{Wien}{}\ledrightnote{\textcolor{pink}{Wien}}}\pend{}\pstart{}\textcolor{pink}{IX. Frankgaſse 1}{}\ledrightnote{\textcolor{pink}{Frankgasse}}.\pend{}{\bigskip}\pstart
           {\pb}\textcolor{pink}{Berlin}{}\ledrightnote{\textcolor{pink}{Berlin}} (leider)\pend
           \pstart
           \raggedleft{}Den 11. 1. 1900.\pend
           \pstart
           Lieber Freund, Ich ſende Dir einen herzlichen Gruß aus
               der \label{K_L02902-1v}\edtext{\textcolor{pink}{Paſſauerſtraße 37}{}\ledrightnote{\textcolor{pink}{Passauerstraße}}}{\lemma{\textnormal{\emph{Paſſauerſtraße 37}}}\Cendnote{\textnormal{Wohnort von \textcolor{blue}{Auguste Chlum} und \textcolor{blue}{Marie
                     Glümer}}}}\label{K_L02902-1h}. Wir haben von Dir geſprochen – und auch ein wenig von \textsc{\textcolor{blue}{Hoffmannsthal}{}\ledrightnote{\textcolor{blue}{Hugo von Hofmannsthal}}}. \strikeout{D\textcolor{gray}{ein} tr} Dein treuer
                  \spacefill\mbox{Paul Goldmann}\pend
           \pstart
           {[}hs. Glümer:{]} Ja, vom lieben \textcolor{blue}{Hofmannsthal}{}\ledrightnote{\textcolor{blue}{Hugo von Hofmannsthal}} haben wir geſprochen. – \label{K_L02902-66v}\edtext{G.}{\lemma{\textnormal{\emph{G.}}}\Cendnote{\textnormal{\textcolor{blue}{Goldmann}}}}\label{K_L02902-66h} iſt ſehr lieb, wir ſind wieder einmal ganz
               glücklich – Sonſt geht es elend. – Was iſt das für eine \label{K_L02902-4v}\edtext{\textcolor{green}{Gouvernante}{}\ledrightnote{\textcolor{green}{Die Gouvernante}} im \uline{\textcolor{brown}{Volkstheater}{}\ledrightnote{\textcolor{brown}{Volkstheater}}}}{\lemma{\textnormal{\emph{Gouvernante im Volkstheater}}}\Cendnote{\textnormal{Bezug auf eine
                  Zeitungsmeldung, dass \textcolor{blue}{Schnitzler} einen
                  Vierakter mit dem Titel \emph{\textcolor{green}{Die Gouvernante}}
                  abgeschlossen habe und dieser im \emph{\textcolor{brown}{Volkstheater}}
                  aufgeführt werden solle (vgl. [O. V.]: \emph{\textcolor{green}{Theater, Kunst und Literatur}}. In: \emph{\textcolor{green}{Wiener Allgemeine Zeitung}}, Nr. 6554, 12. 1. 1900, 6 Uhr-Blatt, S. 3)}}}\label{K_L02902-4h}\footnote{\noindent{}{[}hs. Goldmann:{]} \label{T_L02902-1v}\toendnotes[C]{\begin{minipage}[t]{4em}{\makebox[3.6em][r]{\tiny{Fußnote}}}\end{minipage}\begin{minipage}[t]{\dimexpr\linewidth-4em}\textit{Das unterſtrichene Wort ſoll »Volkstheater« heißen und nicht »Kohlrabi«.}\,{]} kopfüber am oberen Rand\end{minipage}\par}Das unterſtrichene Wort ſoll »\textcolor{brown}{Volkstheater}« heißen und nicht »Kohlrabi«.\label{T_L02902-1h}}? 
               G. weiß auch nichts. \uline{\textsc{Telefonirt}}?\pend
           \pstart
           {[}hs. Chlum:{]} Die \label{K_L02902-2v}\edtext{Herzogin}{\lemma{\textnormal{\emph{Herzogin}}}\Cendnote{\textnormal{unklare Anspielung}}}\label{K_L02902-2h}
               küſst Sie auf die \textsc{Dichterstirne}!\pend
           \pstart
           {[}hs. Coschell:{]} Heute beim \textcolor{blue}{Fischer}{}\ledrightnote{\textcolor{blue}{Samuel Fischer}} gewesen und über \textcolor{green}{Anatol}{}\ledrightnote{\textcolor{green}{Anatol}}
               conferiert. Mache noch einige Ze\textcolor{gray}{i}chnungen. Brief folgt.\pend
           \pstart
           1000 Grüße{\\[\baselineskip]}\spacefill\mbox{Coschell}\pend
           \leftskip=0em{}\endnumbering\briefempfaengerindex{Schnitzler, Arthur@\textsc{Schnitzler, Arthur}!zzzCoschell, Moritz@\emph{von Moritz Coschell}!1900-01-111@{11. 1. 1900}|)be}\briefempfaengerindex{Schnitzler, Arthur@\textsc{Schnitzler, Arthur}!zzzChlum, Auguste@\emph{von Auguste Chlum}!1900-01-111@{11. 1. 1900}|)be}\briefempfaengerindex{Schnitzler, Arthur@\textsc{Schnitzler, Arthur}!zzzGluemer, Marie@\emph{von Marie Glümer}!1900-01-111@{11. 1. 1900}|)be}\briefempfaengerindex{Schnitzler, Arthur@\textsc{Schnitzler, Arthur}!zzzGoldmann, Paul@\emph{von Paul Goldmann}!1900-01-111@{11. 1. 1900}|)be}\mylabel{h}\begin{anhang}\end{anhang}\normalsize

\doendnotes{C}
\bigskip
\vfill

\clearpage

\footnotesize

\lohead{\textsc{register}}

% Definiere theindex-Environment komplett neu ohne reledmac
\makeatletter
\renewenvironment{theindex}{%
  \section*{\indexname}%
  \setlength{\parindent}{0pt}%
  \setlength{\parskip}{0pt plus 0.3pt}%
  \let\item\@idxitem
}{%
  \clearpage
}
\makeatother

\IfFileExists{\jobname-pw.ind}{\input{\jobname-pw.ind}}{}

\end{document}

      