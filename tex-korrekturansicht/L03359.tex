%% latex-korrekturansicht-vorspann.tex
%% Vorspann für die Korrekturansicht.
%% Lädt die gemeinsame Datei latex-vorspann.tex mit gesetztem Schalter.

\newif\ifkorrekturansicht
\korrekturansichttrue

\input{../tex-inputs/latex-vorspann}


\renewcommand{\erwaehntePersonen}{Personen: Aura Hertwig, Felix Salten}
\renewcommand{\erwaehnteOrte}{Orte: Wien}
\renewcommand{\erwaehnteWerke}{Werke: Arthur Schnitzler [Halbprofil 1903]}
\section[ Arthur Schnitzler an Felix Salten, 12. 11. 1903]{Arthur Schnitzler an Felix Salten, 12. 11. 1903}
\nopagebreak\mylabel{v}
\rehead{ }\normalsize\beginnumbering\briefempfaengerindex{Salten, Felix@\textsc{Salten, Felix}!zzzSchnitzler, Arthur@\emph{von Arthur Schnitzler}!1903-11-122@{12. 11. 1903}|(be}
\toendnotes[C]{\smallbreak\pagebreak[2]}\Standort{Wienbibliothek im Rathaus, Nachlass Salten, ZPH 1681, 17.3.11.11.40.2.}
\physDesc{Fotografie, 51 Zeichen
\newline{}Handschrift: schwarze Tinte, deutsche Kurrent
\newline{}Editorischer Hinweis: mit Bleistift auf der Fotografie die handschriftliche Signatur
                                 »\textcolor{blue}{\textsc{A Hertwig}}« und »1903« }\toendnotes[C]{\smallbreak}\begin{figure}[H]\centering\includegraphics[width=10cm]{../tex-inputs/img/ZPH1681_Box_17_3_11_11_40_2_0001_1.jpg}\end{figure}
\pstart
           \noindent{}\textcolor{green}{{\pb}Meinem lieben Felix Salten}{}\ledrightnote{{$\rightarrow$}\textcolor{green}{Arthur Schnitzler [Halbprofil 1903]}}\pend
           \pstart \spacefill\mbox{Arth Sch}\pend{}
\pstart
           \textcolor{pink}{Wien}{}\ledrightnote{\textcolor{pink}{Wien}}{ }12. 11. 903.\pend
           \endnumbering\briefempfaengerindex{Salten, Felix@\textsc{Salten, Felix}!zzzSchnitzler, Arthur@\emph{von Arthur Schnitzler}!1903-11-122@{12. 11. 1903}|)be}\mylabel{h}  \normalsize

\doendnotes{C}
\bigskip
\vfill

\clearpage

\footnotesize

\lohead{\textsc{register}}

% Definiere theindex-Environment komplett neu ohne reledmac
\makeatletter
\renewenvironment{theindex}{%
  \section*{\indexname}%
  \setlength{\parindent}{0pt}%
  \setlength{\parskip}{0pt plus 0.3pt}%
  \let\item\@idxitem
}{%
  \clearpage
}
\makeatother

\IfFileExists{\jobname-pw.ind}{\input{\jobname-pw.ind}}{}

\end{document}

      