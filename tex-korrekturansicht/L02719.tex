%% latex-korrekturansicht-vorspann.tex
%% Vorspann für die Korrekturansicht.
%% Lädt die gemeinsame Datei latex-vorspann.tex mit gesetztem Schalter.

\newif\ifkorrekturansicht
\korrekturansichttrue

\input{../tex-inputs/latex-vorspann}


               \section[Paul Goldmann an Arthur Schnitzler, 4. 11. {[}1893{]}]{ Paul Goldmann an Arthur Schnitzler, 4. 11. {[}1893{]}}\nopagebreak\mylabel{v}\rehead{ }\normalsize\beginnumbering\briefempfaengerindex{Schnitzler, Arthur@\textsc{Schnitzler, Arthur}!zzzGoldmann, Paul@\emph{von Paul Goldmann}!1893-11-041@{4. 11. {[}1893{]}}|(be} \toendnotes[C]{\smallbreak\pagebreak[2]} \Standort{DLA, A:Schnitzler, HS.NZ85.1.3163.}
\physDesc{Brief, 5 Blätter, 13 Seiten
\newline{}Handschrift: schwarze Tinte, deutsche Kurrent
\newline{}Schnitzler: 1) mit schwarzer Tinte das Jahr »93« vermerkt 2) mit rotem Buntstift zwei Unterstreichungen und fünf vertikale
                                 Markierungen}\toendnotes[C]{\smallbreak}\pstart
           \raggedleft{}{\pb}\textsc{\textcolor{pink}{Paris}{}\ledrightnote{\textcolor{pink}{Paris}}}, 4. November.\pend
           \pstart\center{}Mein lieber Freund,\pend\pstart
           Du mußt mir nicht böſe ſein: Ich habe hier wenig Beziehungen zur ärztlichen Welt und
               da ich außerdem mit tauſend Dingen die Hände voll zu thun hatte, habe ich eine Woche
               gebraucht, ehe ich Dir das Gewünſchte verſchaffen gekonnt. Ich ſende Dir anbei das
                  \label{K_L02719-1v}\edtext{»\textsc{\textcolor{green}{Agenda médical}{}\ledrightnote{\textcolor{green}{Agenda médical}}}«}{\lemma{\textnormal{\emph{»Agenda médical«}}}\Cendnote{\textnormal{Die \emph{\textcolor{green}{Agenda médical}} erschien jährlich und listete unter anderem
                  französische Mediziner. \textcolor{blue}{Goldmann} sandte \textcolor{blue}{Schnitzler} vermutlich die neueste Ausgabe für
                  das Jahr 1894. Es ist unklar, wofür \textcolor{blue}{Schnitzler} die Namen der Professoren brauchte.}}}\label{K_L02719-1h}. Auf
               S. 381 findeſt Du die Namen derjenigen Profeſſoren unterſtrichen, die mir als die
               bedeutendſten bezeichnet worden; ihre Adreſſen ſind in dem S. 299 beginnenden {\pb}Verzeichniß enthalten. Wenn Du nun Weiteres brauchſt,
               für dieſe ſowie für alle zukünftigen Angelegenheiten – wenn Gänge zu machen oder
               Briefe auszutragen ſind \textsc{etc.} – ſo ſchreibe mir ſtets.
               Insbeſondere den mechaniſchen Theil eventueller journaliſtiſcher Maßnahmen kann ich
               Dir leicht beſtreiten helfen, da ich hier einen Büreaudiener habe. Aber auch ſonſt
               betrachte mich als Deinen \label{K_L02719-2v}\edtext{\textsc{\begin{otherlanguage}{french}ministre plénipotentiaire\end{otherlanguage}}}{\lemma{\textnormal{\emph{ministre plénipotentiaire}}}\Cendnote{\textnormal{französisch: Gesandter}}}\label{K_L02719-2h} und gib’
               mir etwas zu {\pb}arbeiten. Freilich verlange ich einen
               Gegendienſt. Das iſt gemein, aber ich kann nicht anders. Schon während unſeres
                  \label{K_L02719-3v}\edtext{letzten Beiſammenſeins}{\lemma{\textnormal{\emph{letzten Beiſammenſeins}}}\Cendnote{\textnormal{am 18. 9. 1893 in \textcolor{pink}{Salzburg}}}}\label{K_L02719-3h} hatte ich die Bitte auf der Zunge, aber es erſchien mir doch gar zu
               erbärmlich, Dir damit zu kommen. Alſo ſchriftlich: Wäre Dir möglich, wenigſtens ein
               paar Monate lang, meinem \textcolor{blue}{Schwager}{}\ledrightnote{→\textcolor{blue}{Josef Rosengart}} ein \label{K_L02719-4v}\edtext{\textcolor{green}{Freiexemplar}{}\ledrightnote{→\textcolor{green}{Internationale klinische Rundschau}}}{\lemma{\textnormal{\emph{Freiexemplar}}}\Cendnote{\textnormal{der \emph{\textcolor{green}{Internationalen Klinischen Rundschau}}, die bis September 1894 von \textcolor{blue}{Schnitzler}
                  herausgegeben wurde. vgl. Paul Goldmann an Arthur Schnitzler, 29. 5. [1894]}}}\label{K_L02719-4h} zu bewilligen. Seine Praxi\strikeout{ſ}s geht noch nicht
                  {\pb}gut genug, ihm ein Abonnement zu erlauben.
               Anderſeits möchte er gar zu gern\strikeout{,} das \textcolor{green}{Blatt}{}\ledrightnote{→\textcolor{green}{Internationale klinische Rundschau}} leſen. Und da durch einen glücklichen
                  Zufall{\dotsfour} Ich bitte Dich alſo um Gewährung meiner Bitte,
               indem ich zugleich gegen die von mir begangene ſchamloſe Ausbeutung proteſtire.
               Adreſſe: \textsc{\textcolor{blue}{Dr. Josef Rosengart}{}\ledrightnote{\textcolor{blue}{Josef Rosengart}}}, \textsc{\textcolor{pink}{Frankfurt \textsuperscript{a}/M, \strikeout{\textcolor{pink}{Rossmark}{}\ledrightnote{→\textcolor{pink}{Roßmarkt}}}}{}\ledrightnote{\textcolor{pink}{Frankfurt am Main}}{ }\textcolor{pink}{Rossmarkt 20}{}\ledrightnote{\textcolor{pink}{Roßmarkt}}}.\pend
           \pstart
           Es iſt viel Erfreuliches in Deinem lieben Briefe. Vor allen Dingen bin ich {\pb}von Herzen froh, daß es endlich mit der \textcolor{green}{Aufführung}{}\ledrightnote{→\textcolor{green}{Das Märchen. Schauspiel in drei Aufzügen}} ernſt wird. Da ich
               ſo gar nichts hörte, glaubte ich, es ſei wieder eine Verſchiebung eingetreten.
               Nochmals: ſobald die Aufführung feſtgeſetzt iſt, theile mir das \uline{umgehend} mit. Und reg’ Dich nicht auf wenn die Komödiantenbande, der
               Gewohnheit gemäß, Dich kränken ſollte. Ich hätte ſo gern genaue Details über die
               Proben gewußt, ich bin auch überzeugt, daß Du bei unſerem nächſten {\pb}Beiſammerſein behaupten wirſt, ſie mir geſchrieben
               zu haben. Damit werde ich mich wohl begnügen müſſen. \strikeout{Sehr} Laß’ mich wenigſtens bald etwas über den Fortgang der Affaire wiſſen,
               – ja? Und ſtärkt Dir das nicht richtig die Productionsluſt, dieſe endliche
               Verwirklichung des ſo lange Erhofften?\pend
           \pstart
           Ich habe den »\textsc{\textcolor{green}{Anatol}{}\ledrightnote{\textcolor{green}{Anatol}}}« und das »\textcolor{green}{Märchen}{}\ledrightnote{\textcolor{green}{Das Märchen. Schauspiel in drei Aufzügen}}« hier dem neu
               begründeten \textcolor{brown}{Freien Theater für
                  ausländiſche Kunſt}{}\ledrightnote{→\textcolor{brown}{Théâtre de l’Œuvre}}, dem »\textsc{\textcolor{brown}{Oeuvre}{}\ledrightnote{\textcolor{brown}{Théâtre de l’Œuvre}}}« eingereicht. {\pb}Die \label{K_L02719-8v}\edtext{\textcolor{blue}{Herren}{}\ledrightnote{→\textcolor{blue}{Aurélien-Marie Lugné-Poe}}}{\lemma{\textnormal{\emph{Herren}}}\Cendnote{\textnormal{Es ist nicht letztgültig zu klären, wen
                     \textcolor{blue}{Goldmann} hiermit meinte. Geleitet wurde
                  das \emph{\textcolor{brown}{Théâtre de l’Œuvre}} zu dieser Zeit
                  jedenfalls von \textcolor{blue}{Aurélien-Marie Lugné-Poe}.
                  Auch in späteren Jahren spielte das \emph{\textcolor{brown}{Théâtre de
                     l’Œuvre}} für \textcolor{blue}{Schnitzler} eine Rolle.
                  So empfahl etwa \textcolor{blue}{Marcel Schulz}{ }\textcolor{blue}{Lugné-Poe} den \emph{\textcolor{green}{Schleier der Beatrice}} (Vgl. A. S.: \emph{Tagebuch}, 29. 1. 1907) und auch \textcolor{blue}{Paul Zifferer} legte \textcolor{blue}{Schnitzler} das
                     \emph{\textcolor{brown}{Théâtre de l’Œuvre}} »wegen [s]einer
                     Stücke für \textcolor{pink}{Paris}« nahe (Vgl. A. S.: \emph{Tagebuch}, 6. 5. 1927). 1912 und 1922 inszenierte
                  das \emph{\textcolor{brown}{Théâtre de l’Œuvre}} den \emph{\textcolor{green}{Einakter}}{ }\emph{\textcolor{green}{Die letzten Masken}} (\textcolor{green}{Les Derniers masques}).}}}\label{K_L02719-8h} waren ſehr
               vergnügt über mein ihnen gewidmetes \label{K_L02719-7v}\edtext{\textcolor{green}{Feuilleton}{}\ledrightnote{\textcolor{green}{Pariser Theater}}}{\lemma{\textnormal{\emph{Feuilleton}}}\Cendnote{\textnormal{\textcolor{blue}{Paul Goldmann}: \emph{\textcolor{green}{Pariser Theater}}.
                     In: \emph{\textcolor{green}{Frankfurter Zeitung}}, Jg. 38, Nr. 282,
                     11.10.1893, Erstes Morgenblatt, S. 1–2.}}}\label{K_L02719-7h}, und da ich nicht gern
                  \strikeout{auf} die Gelegenheit zum Verlangen von
               Gegendienſten vorübergehen laſſe (ſiehe oben), ſo bat ich ſie, Deine \textcolor{green}{Stücke}{}\ledrightnote{→\textcolor{green}{Anatol}{\newline}→\textcolor{green}{Das Märchen. Schauspiel in drei Aufzügen}} zu leſen. Es ſind nämlich
               Leute darin, die deutſch können. Mach’ Dir aber keine allzu großen Hoffnungen. \strikeout{D}{ }\strikeout{\textcolor{gray}{S}ie} Sie frugen mich nämlich, ob die \textcolor{green}{Stücke}{}\ledrightnote{→\textcolor{green}{Anatol}{\newline}→\textcolor{green}{Das Märchen. Schauspiel in drei Aufzügen}} »myſtiſch« ſeien? Ich
               wußte nicht recht, was {\pb}ich ſagen ſollte: Bitte,
               ſind ſie myſtiſch?\pend
           \pstart
           Übrigens habe ich noch andere Eiſen für Dich hier im Feuer. Doch davon ſpäter.\pend
           \pstart
           Das Blühen in der lieben \textcolor{pink}{Wien}{}\ledrightnote{\textcolor{pink}{Wien}}er Künſtler-Laube –
               oh verdammt, welch’ ein Gleichniß! – beobachte ich mit wehmüthiger Freude. Gewiß, ich
               weiß, daß \textcolor{blue}{Eure}{}\ledrightnote{→\textcolor{blue}{Richard Beer-Hofmann}{\newline}→\textcolor{blue}{Hugo von Hofmannsthal}} drei Namen weit klingen werden, und in nicht langer Zeit. Ich ſehe, wie
               Ihr formt und ſchafft, und wünſche allen Segen {\pb}auf
               dieſes Schaffen herab. Und dann kehre ich in mich ein und habe das traurige Gefühl
               des Mannes, der einſam und ſchwach auf einem Stein ſitzen geblieben iſt und nur noch
               die fernen Stimmen der Begleiter hört, die durch den Wald hallen: aber ſie ſind weit
               und er wird ihnen nimmer nachkommen. Meine Arbeiten? Gewiß weiß ichs nicht, wenn ich
               etwas Gutes ſchreibe. Und wenn ich es wüßte: Hat das einen Werth, was ich thue? Geh’,
               das mußt {\pb}Du mir ſelbſt zugeben, daß ich in unſerem
               Kreiſe bereits immer deutlicher die bitterböſe Rolle übernehme »des Mannes, aus dem
               etwas hätte werden können«.\pend
           \pstart
           Ich bitte Dich inſtändig: veranlaſſe \textsc{\textcolor{blue}{Loris}{}\ledrightnote{\textcolor{blue}{Hugo von Hofmannsthal}}} und \textsc{\textcolor{blue}{Richard}{}\ledrightnote{\textcolor{blue}{Richard Beer-Hofmann}}}, daß ſie mir die erſchienen{[}en{]} oder zu erſcheinenden
                  \label{K_L02719-10v}\edtext{\textcolor{green}{Sachen}{}\ledrightnote{→\textcolor{green}{Novellen}}}{\lemma{\textnormal{\emph{Sachen}}}\Cendnote{\textnormal{Die einzige selbstständige
                  Veröffentlichung – \textcolor{blue}{Goldmann} bezieht sich
                  auf »\textcolor{green}{Bücher}« – aus dieser Zeit stellt eine Novellensammlung \textcolor{blue}{Richard Beer-Hofmann}s dar, doch erschien diese erst im
                     Dezember 1893. \textcolor{blue}{Richard Beer-Hofmann}: \emph{\textcolor{green}{Novellen}}. Berlin: \emph{Freund {\kaufmannsund} Jeckel}{ }1893.}}}\label{K_L02719-10h} ſchicken. Ohne Briefe: ich weiß, daß die Briefe nach ſo langer
               Zeit ſchwer zu ſchreiben ſind. Die gewiſſe Furcht vor der Einleitung. Ich {\pb}möchte deßwegen aber nicht um die \textcolor{green}{Bücher}{}\ledrightnote{→\textcolor{green}{Novellen}} kommen.\pend
           \pstart
           Wenn Du kannſt, ſo ſchick’ mir, bitte, gelegentlich noch einen »\textsc{\textcolor{green}{Anatol}{}\ledrightnote{\textcolor{green}{Anatol}}}« – zu Propaganda-Zwecken.\pend
           \pstart
           \textsc{\textcolor{blue}{Bahr}{}\ledrightnote{\textcolor{blue}{Hermann Bahr}}}: Du haſt eine ſo merkwürdige Art, gegen Leute gerecht ſein zu wollen, die ſich
               ſchurkiſch gegen Dich benehmen. Nein, – der \textcolor{blue}{Mann}{}\ledrightnote{→\textcolor{blue}{Hermann Bahr}} iſt für mich kein großes \textcolor{blue}{Talent}{}\ledrightnote{→\textcolor{blue}{Hermann Bahr}}, ſelbſt wenn er es ſein ſollte.
               Ungerechte {\pb}Beurheilung iſt bereits eine halbe
               Befriedigung des Haſſes. Und ſeit der hundsföttischen \textcolor{green}{Kritik}{}\ledrightnote{→\textcolor{green}{Das junge Österreich}} über Dich haſſe ich den \textcolor{blue}{Kerl}{}\ledrightnote{→\textcolor{blue}{Hermann Bahr}} mehr als je.\pend
           \pstart
           Der \label{K_L02719-12v}\edtext{Briefkaſten-Diebſtahl}{\lemma{\textnormal{\emph{Briefkaſten-Diebſtahl}}}\Cendnote{\textnormal{In \emph{\textcolor{green}{Ridicula}} versammelte \emph{\textcolor{green}{Theodor von
                     Sosnosky}} vermeintliche »literarische Lächerlichkeiten« (Breslau:{ }\emph{\textcolor{brown}{Trewendt}}{ }1894 [von 1893 vordatiert]). Im Kapitel
                  »Briefkastenpoesie« wurden – ohne Erlaubnis – 50 Seiten aus dem \textcolor{green}{Briefkasten} der \emph{\textcolor{green}{Schönen blauen Donau}} aufgenommen. Vgl. \textcolor{blue}{h. k.}: \emph{\textcolor{green}{Neue Bücher}}. In: \emph{\textcolor{green}{An der schönen blauen
                        Donau}}, Jg. 8, Nr. 23, 1. 12. 1893,
                     S. 552.}}}\label{K_L02719-12h} des \textsc{\textcolor{blue}{Sosnosky}{}\ledrightnote{\textcolor{blue}{Theodor von Sosnosky}}} iſt ſcheußlich. Ich habe mit meinem \textcolor{blue}{Onkel}{}\ledrightnote{→\textcolor{blue}{Fedor Mamroth}} berathen, aber glaube, wir können nichts machen,
               geſetzlich. Höchſtens eine Züchtigung im \textcolor{green}{Blatte}{}\ledrightnote{→\textcolor{green}{An der schönen blauen Donau}}, die aber auch eine Reklame für das \textcolor{green}{Buch}{}\ledrightnote{→\textcolor{green}{Ridicula}} des \textcolor{blue}{Gauner}{}\ledrightnote{→\textcolor{blue}{Theodor von Sosnosky}}s wäre.\pend
           \pstart
           {\pb}\textsc{\textcolor{blue}{Herzl}{}\ledrightnote{\textcolor{blue}{Theodor Herzl}}} iſt ſeit einigen Wochen ſehr \label{K_L02719-13v}\edtext{krank}{\lemma{\textnormal{\emph{krank}}}\Cendnote{\textnormal{Von seiner Malariainfektion
                  berichtete \textcolor{blue}{Theodor Herzl} am 8. 12. 1893 in einem Brief an \textcolor{blue}{Schnitzler}. Siehe \textcolor{blue}{Theodor Herzl}: \emph{Briefe und Tagebücher}. Hg.
                     Alex Bein, Hermann Greive, Moshe Schaerf und Julius H. Schoeps. Bd. 1.: \emph{Briefe und autobiographische Notizen. 1866–1895}.
                     Bearbeitet von Johannes Wachten. In Zusammenarbeit mit Chaya Harel, Daisy Tycho
                     und Manfred Winkler. Berlin, Frankfurt am
                        Main, Wien: \emph{\textcolor{brown}{Ullstein}}/\emph{\textcolor{brown}{Propyläen}}{ }1983, S. 545.}}}\label{K_L02719-13h}: \textsc{Malaria} oder ſo etwas.\pend
           \pstart
           Was Neues in \textcolor{pink}{Wien}{}\ledrightnote{\textcolor{pink}{Wien}}? Bitte ſchreibe bald.\pend
           \pstart
           Auch ein perſönliches Wort: Geſundheit, Production, materielle Fragen.\pend
           \pstart
           Mir geht es ſchlecht, oh ſo ſchlecht!\pend
           \pstart
           Viele treue Grüße!\pend
           \pstart
           Dein {\\[\baselineskip]}\spacefill\mbox{Paul Goldm}\pend
           \leftskip=0em{}\endnumbering\briefempfaengerindex{Schnitzler, Arthur@\textsc{Schnitzler, Arthur}!zzzGoldmann, Paul@\emph{von Paul Goldmann}!1893-11-041@{4. 11. {[}1893{]}}|)be}\mylabel{h}\begin{anhang}\end{anhang}\normalsize

\doendnotes{C}
\bigskip
\vfill

\clearpage

\footnotesize

\lohead{\textsc{register}}

% Definiere theindex-Environment komplett neu ohne reledmac
\makeatletter
\renewenvironment{theindex}{%
  \section*{\indexname}%
  \setlength{\parindent}{0pt}%
  \setlength{\parskip}{0pt plus 0.3pt}%
  \let\item\@idxitem
}{%
  \clearpage
}
\makeatother

\IfFileExists{\jobname-pw.ind}{\input{\jobname-pw.ind}}{}

\end{document}

      