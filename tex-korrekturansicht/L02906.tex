%% latex-korrekturansicht-vorspann.tex
%% Vorspann für die Korrekturansicht.
%% Lädt die gemeinsame Datei latex-vorspann.tex mit gesetztem Schalter.

\newif\ifkorrekturansicht
\korrekturansichttrue

\input{../tex-inputs/latex-vorspann}


         
         \renewcommand{\erwaehntePersonen}{Personen: D. W. Schröder}
         \renewcommand{\erwaehnteOrte}{Orte: Berlin, Dessauer Straße, Hotel Saxonia, Potsdamer Platz, Stresemannstraße, Tiergarten, Wien}
         \renewcommand{\erwaehnteWerke}{}
               \section[ Paul Goldmann an Arthur Schnitzler, 4. 3. 1900]{Paul Goldmann an Arthur Schnitzler, 4. 3. 1900}\nopagebreak\mylabel{v}\rehead{ }\normalsize\beginnumbering\briefempfaengerindex{Schnitzler, Arthur@\textsc{Schnitzler, Arthur}!zzzGoldmann, Paul@\emph{von Paul Goldmann}!1900-03-041@{4. 3. 1900}|(be} \toendnotes[C]{\smallbreak\pagebreak[2]} \Standort{DLA, A:Schnitzler, HS.NZ85.1.3170.}
\physDesc{Brief, 1 Blatt, 1 Seite
\newline{}Handschrift: schwarze Tinte, deutsche Kurrent}\pstart
           \noindent{}\centering{}{\pb}\textcolor{gray}{\textbf{\textbf{\textcolor{pink}{HOTEL SAXONIA}{}\ledrightnote{\textcolor{pink}{Hotel Saxonia}}}}}\pend
           \pstart
           \noindent{}\raggedleft{}\textcolor{gray}{\textbf{am \textcolor{pink}{Potsdamer Platz}{}\ledrightnote{\textcolor{pink}{Potsdamer Platz}} und
                        \textcolor{pink}{Thiergarten}{}\ledrightnote{\textcolor{pink}{Tiergarten}}}}\pend
           \pstart
           \noindent{}\centering{}\textcolor{gray}{\textbf{\textcolor{blue}{D. W. SCHRÖDER}{}\ledrightnote{\textcolor{blue}{D. W. Schröder}}.}}\pend
           \pstart
           \noindent{}\textcolor{gray}{\textbf{Fernsprecher:}}\pend
           \pstart
           \textcolor{gray}{\textbf{\textbf{Amt VI. No. 2838.}}}\pend
           \pstart
           \raggedleft{}\textcolor{gray}{\textbf{\textcolor{pink}{BERLIN W.}{}\ledrightnote{\textcolor{pink}{Berlin}}, den}}{ }4. März \textcolor{gray}{\textbf{1}}900. \pend
           \pstart
           \raggedleft{}\textcolor{gray}{\textbf{\textcolor{pink}{Königgrätzerstrasse 10}{}\ledrightnote{\textcolor{pink}{Stresemannstraße}}.}}\pend
           \pstart{}Mein lieber Freund,\pend\pstart
           Ich beziehe heut mein hieſige Wohnung, und meine
               Adreſſe lautet jetzt: \textcolor{pink}{Deſſauer Straße 19}{}\ledrightnote{\textcolor{pink}{Dessauer Straße}}.\pend
           \pstart
           Es iſt ſehr bedauerlich, daß Du ſo ſchreibfaul geworden biſt.\pend
           \pstart
           Viele treue Grüße! {\\[\baselineskip]}Dein {\\[\baselineskip]}\spacefill\mbox{Paul Goldmann}\pend
           \leftskip=0em{}\endnumbering\briefempfaengerindex{Schnitzler, Arthur@\textsc{Schnitzler, Arthur}!zzzGoldmann, Paul@\emph{von Paul Goldmann}!1900-03-041@{4. 3. 1900}|)be}\mylabel{h}\begin{anhang}\end{anhang}\normalsize

\doendnotes{C}
\bigskip
\vfill

\clearpage

\footnotesize

\lohead{\textsc{register}}

% Definiere theindex-Environment komplett neu ohne reledmac
\makeatletter
\renewenvironment{theindex}{%
  \section*{\indexname}%
  \setlength{\parindent}{0pt}%
  \setlength{\parskip}{0pt plus 0.3pt}%
  \let\item\@idxitem
}{%
  \clearpage
}
\makeatother

\IfFileExists{\jobname-pw.ind}{\input{\jobname-pw.ind}}{}

\end{document}

      