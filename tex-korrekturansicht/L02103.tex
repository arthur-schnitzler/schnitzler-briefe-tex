%% latex-korrekturansicht-vorspann.tex
%% Vorspann für die Korrekturansicht.
%% Lädt die gemeinsame Datei latex-vorspann.tex mit gesetztem Schalter.

\newif\ifkorrekturansicht
\korrekturansichttrue

\input{../tex-inputs/latex-vorspann}


               \section[Hugo von Hofmannsthal an Arthur Schnitzler, 21. 11. 1912]{ Hugo von Hofmannsthal an Arthur Schnitzler, 21. 11. 1912}\nopagebreak\mylabel{v}\rehead{ }\normalsize\beginnumbering\briefempfaengerindex{Schnitzler, Arthur@\textsc{Schnitzler, Arthur}!zzzHofmannsthal, Hugo von@\emph{von Hugo von Hofmannsthal}!1912-11-211@{21. 11. 1912}|(be} \toendnotes[C]{\smallbreak\pagebreak[2]} \Standort{CUL, Schnitzler, B 43.}
\physDesc{Postkarte
\newline{}Handschrift: schwarze Tinte, deutsche Kurrent\newline{}Versand: 1) Rohrpost 2) Stempel: »\nobreak{}\oindex{I., Innere Stadt@\textbf{I., Innere Stadt}, \emph{Bezirk (A.BZK)}|pwk}\textcolor{gray}{1/1} Wien 11, 21 XI 12, XII\nobreak{}«. 3) Stempel: »\nobreak{}\oindex{XVIII., Waehring@\textbf{XVIII., Währing}, \emph{Bezirk (A.BZK)}|pwk}18/1 Wien 111, 21 \textcolor{gray}{X}I 12, XII\textsuperscript{\textcolor{gray}{1}0}\nobreak{}«. \newline{}Ordnung: 1) mit Bleistift von unbekannter Hand nummeriert: »\strikeout{382}« 2) mit Bleistift von unbekannter Hand nummeriert:
                                    »343«}\buchAbdrucke{\weitereDrucke{Hugo von Hofmannsthal, Arthur Schnitzler: \emph{Briefwechsel}. Hg. Therese Nickl und Heinrich Schnitzler. Frankfurt am Main: \emph{S. Fischer} 1964, S. 270.} }\toendnotes[C]{\smallbreak}\pstart{}{\pb}Hofmannsthal\pend{}{\bigskip}\pstart{}\textsc{Herrn D\textsuperscript{r} Arthur Schnitzler}\pend{}\pstart{}\textcolor{pink}{\textsc{Wien}}{}\ledrightnote{\textcolor{pink}{Wien}}\pend{}\pstart{}\textcolor{pink}{XVIII Sternwartestrasse 71}{}\ledrightnote{\textcolor{pink}{Sternwartestraße}}\pend{}{\bigskip}\pstart
           \noindent{}{\pb}lieber, erwartete i{\geminationm}er ein Wort \strikeout{fin} von Ihnen!\hspace*{1.5em}Nun
                  \label{K_L02103_1v}\edtext{Freitag}{\lemma{\textnormal{\emph{Freitag}}}\Cendnote{\textnormal{\textcolor{blue}{Schnitzler} dürfte ihn zu einem Abend anlässlich
                  des Besuchs von \textcolor{blue}{Georg Brandes} geladen
                  haben.}}}\label{K_L02103_1h} gerade haben wir Plätze zu \textcolor{blue}{\textsc{Casals}}{}\ledrightnote{\textcolor{blue}{Pablo Casals}}. Das iſt eine Muſik die mir ſo viel Freude macht, daſs ich die Plätze wirklich
               nicht aufgeben möchte. Alſo dann auf Wiederſehen nach dem \label{K_L02103_2v}\edtext{12\textsuperscript{ten} December}{\lemma{\textnormal{\emph{12ten December}}}\Cendnote{\textnormal{\textcolor{blue}{Schnitzler} war vom 23. 11. 1912 bis zum 2. 12. 1912 in \textcolor{pink}{Berlin}, wo die Uraufführung von \emph{\textcolor{green}{Professor
                     Bernhardi}} stattfand. \textcolor{blue}{Hofmannsthal}
                  reiste am 30. 11. 1912 nach \textcolor{pink}{Auerbach
                     (Vogtland)} und in Folge an mehrere deutsche Orte. In \textcolor{pink}{Berlin} war er zwischen 6. 12. 1912 und
                     12. 12. 1912. Er kehrte am 15. 12. 1912 nach \textcolor{pink}{Rodaun} zurück.}}}\label{K_L02103_2h}! Es wird wohl die längſte
               Pauſe in unſerem bisherigen Verkehr geweſen sein! Vielleicht bin ich zur \textcolor{green}{Première}{}\ledrightnote{→\textcolor{green}{Professor Bernhardi. Komödie in fünf Akten}} in \textcolor{pink}{Berlin}{}\ledrightnote{\textcolor{pink}{Berlin}}! \pend
           \pstart Alles Gute an \textcolor{blue}{Olga}{}\ledrightnote{\textcolor{blue}{Olga Schnitzler}}. Ihr\spacefill\mbox{Hugo}\pend{}\endnumbering\briefempfaengerindex{Schnitzler, Arthur@\textsc{Schnitzler, Arthur}!zzzHofmannsthal, Hugo von@\emph{von Hugo von Hofmannsthal}!1912-11-211@{21. 11. 1912}|)be}\mylabel{h}  \normalsize

\doendnotes{C}
\bigskip
\vfill

\clearpage

\footnotesize

\lohead{\textsc{register}}

% Definiere theindex-Environment komplett neu ohne reledmac
\makeatletter
\renewenvironment{theindex}{%
  \section*{\indexname}%
  \setlength{\parindent}{0pt}%
  \setlength{\parskip}{0pt plus 0.3pt}%
  \let\item\@idxitem
}{%
  \clearpage
}
\makeatother

\IfFileExists{\jobname-pw.ind}{\input{\jobname-pw.ind}}{}

\end{document}

      