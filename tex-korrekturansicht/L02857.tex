%% latex-korrekturansicht-vorspann.tex
%% Vorspann für die Korrekturansicht.
%% Lädt die gemeinsame Datei latex-vorspann.tex mit gesetztem Schalter.

\newif\ifkorrekturansicht
\korrekturansichttrue

\input{../tex-inputs/latex-vorspann}


               \section[ Paul Goldmann an Arthur Schnitzler, 10. 9. 1898]{Paul Goldmann an Arthur Schnitzler, 10. 9. 1898}\nopagebreak\mylabel{v}\rehead{ }\normalsize\beginnumbering\briefempfaengerindex{Schnitzler, Arthur@\textsc{Schnitzler, Arthur}!zzzGoldmann, Paul@\emph{von Paul Goldmann}!1898-09-101@{10. 9. 1898}|(be} \toendnotes[C]{\smallbreak\pagebreak[2]} \Standort{DLA, A:Schnitzler, HS.NZ85.1.3168.}
\physDesc{Bildpostkarte
\newline{}Handschrift: 1) blaue Tinte, deutsche Kurrent\hspace{1em}2) blaue Tinte, lateinische Kurrent (\noindent{}Adresse)\hspace{1em}\newline{}Versand: 1) Stempel: »\nobreak{}\oindex{Tianjin@\textbf{Tianjin}, \emph{Besiedelter Ort (A.BSO)}|pwk}Tientsin, 10/9 98, \textcolor{brown}{Kaiserl. deutsche
                                          Postagentur}\nobreak{}«.  2) Stempel: »\nobreak{}Wien 9/3 72, 22. 10. 98, 1. N, Bestellt\nobreak{}«. 
\newline{}Schnitzler: mit Bleistift das Jahr »98« vermerkt }\toendnotes[C]{\smallbreak}\pstart{}{\pb}\begin{otherlanguage}{english}\textcolor{pink}{Austria}{}\ledrightnote{\textcolor{pink}{Österreich}}\end{otherlanguage}.\pend{}\pstart{}Herrn Dr. Arthur Schnitzler\pend{}\pstart{}\textcolor{pink}{Wien}{}\ledrightnote{\textcolor{pink}{Wien}}\pend{}\pstart{}\textcolor{pink}{IX. Frankgaße 1}{}\ledrightnote{\textcolor{pink}{Frankgasse}}.\pend{}{\bigskip}\pstart
           \raggedleft{}{\pb}\textcolor{gray}{\textbf{\textcolor{pink}{TIENTSIN}{}\ledrightnote{\textcolor{pink}{Tianjin}}}}, 10. September.\pend
           \pstart
           Viele Grüße und herzlichen Dank für die Karte aus \label{K_L02857-1v}\edtext{\textsc{\textcolor{pink}{Toelz}{}\ledrightnote{\textcolor{pink}{Bad Tölz}}}}{\lemma{\textnormal{\emph{Toelz}}}\Cendnote{\textnormal{\textcolor{blue}{Schnitzler} unternahm am 3. 8. 1898 mit \textcolor{blue}{Marie Reinhard} und deren Schwester \textcolor{blue}{Caroline Burger} einen Radausflug von \textcolor{pink}{Tegernsee} nach \textcolor{pink}{Bad Tölz} und wieder zurück. \textcolor{blue}{Goldmann} hielt sich im Sommer 1895 mehrere
                  Wochen für eine Kur in \textcolor{pink}{Bad Tölz}
               auf.}}}\label{K_L02857-1h}!\pend
           \pstart
           Dein {\\[\baselineskip]}\spacefill\mbox{Paul Goldmann.}\pend
           \leftskip=0em{}\endnumbering\briefempfaengerindex{Schnitzler, Arthur@\textsc{Schnitzler, Arthur}!zzzGoldmann, Paul@\emph{von Paul Goldmann}!1898-09-101@{10. 9. 1898}|)be}\mylabel{h}\begin{anhang}\end{anhang}\normalsize

\doendnotes{C}
\bigskip
\vfill

\clearpage

\footnotesize

\lohead{\textsc{register}}

% Definiere theindex-Environment komplett neu ohne reledmac
\makeatletter
\renewenvironment{theindex}{%
  \section*{\indexname}%
  \setlength{\parindent}{0pt}%
  \setlength{\parskip}{0pt plus 0.3pt}%
  \let\item\@idxitem
}{%
  \clearpage
}
\makeatother

\IfFileExists{\jobname-pw.ind}{\input{\jobname-pw.ind}}{}

\end{document}

      