%% latex-korrekturansicht-vorspann.tex
%% Vorspann für die Korrekturansicht.
%% Lädt die gemeinsame Datei latex-vorspann.tex mit gesetztem Schalter.

\newif\ifkorrekturansicht
\korrekturansichttrue

\input{../tex-inputs/latex-vorspann}


               \section[Michael Konstantin an Arthur Schnitzler, 22. 5. 1890]{ Michael Konstantin an Arthur Schnitzler, 22. 5. 1890}\nopagebreak\mylabel{v}\rehead{ }\normalsize\beginnumbering\briefempfaengerindex{Schnitzler, Arthur@\textsc{Schnitzler, Arthur}!zzzKonstantin, Michael@\emph{von Michael Konstantin}!1890-05-221@{22. 5. 1890}|(be} \toendnotes[C]{\smallbreak\pagebreak[2]} \Standort{DLA, A:Schnitzler, HS.NZ85.1.3750.}
\physDesc{Postkarte
\newline{}Handschrift: schwarze Tinte, deutsche Kurrent\newline{}Versand: 1) Stempel: »\nobreak{}\oindex{Bruenn@\textbf{Brünn}, \emph{Besiedelter Ort (A.BSO)}|pwk}Brünn Bahnhof Brno nádraží, 22 5 90\nobreak{}«.  2) Stempel: »\nobreak{}{[}Wi{]}en, 23 5 90, 8.F\nobreak{}«. 
\newline{}Schnitzler: mit rotem Buntstift zwei Unterstreichungen }\toendnotes[C]{\smallbreak}\pstart{}{\pb}Herrn \textsc{Arthur
                            Schnitzler}\pend{}\pstart{}\textsc{\textcolor{pink}{Wien}{}\ledrightnote{\textcolor{pink}{Wien}}}\pend{}\pstart{}\textcolor{pink}{I Giselastraße 11}{}\ledrightnote{\textcolor{pink}{Bösendorferstraße}}\pend{}{\bigskip}\pstart
           \noindent{}{\pb}\textcolor{gray}{\textbf{\textcolor{green}{Moderne Dichtung}{}\ledrightnote{\textcolor{green}{Moderne Dichtung. Monatsschrift für Literatur und Kritik}}.}}\hfill \textcolor{pink}{Brünn}{}\ledrightnote{\textcolor{pink}{Brünn}}{ }22/5 1890\pend
           \pstart
           \textcolor{gray}{\textbf{Monatsſchrift für Literatur und Kritik.}}\hfill Herrn \textsc{Arthur Schnitzler}\pend
           \pstart
           \textcolor{gray}{\textbf{\textcolor{brown}{Redaction}{}\ledrightnote{→\textcolor{brown}{Moderne Dichtung/Moderne Rundschau}}.}}\hfill \textsc{\uline{\textcolor{pink}{Wien}{}\ledrightnote{\textcolor{pink}{Wien}}}}\pend
           \pstart
           \textcolor{gray}{\textbf{\textcolor{pink}{Brünn, Schreibwaldſtraße 35}{}\ledrightnote{\textcolor{pink}{Výstaviště}}.}}\hfill \textcolor{pink}{I Giſelaſtraße 11}{}\ledrightnote{\textcolor{pink}{Bösendorferstraße}}\pend
           \pstart{}Geehrter Herr!\pend\pstart
           Die Handlungsweiſe des \textcolor{brown}{\textsc{B. Tgbtt.}}{}\ledrightnote{\textcolor{brown}{Budapester Tagblatt}} iſt einfach eine \label{K_L00003_1v}\edtext{Gemeinheit}{\lemma{\textnormal{\emph{Gemeinheit}}}\Cendnote{\textnormal{Es dürfte sich um
                        den unerlaubten und korrumpierten Nachdruck von \emph{\textcolor{green}{Die Frage an das Schicksal}} im \emph{\textcolor{green}{Budapester Tageblatt}} vom 13. 5. 1890 handeln. Er
                        basiert auf dem Erstdruck in der \emph{\textcolor{green}{Modernen
                            Dichtung}} vom 1. 5. 1890.}}}\label{K_L00003_1h}. Ich werde Gelegenheit
                    nehmen der Redaction derſelben meine Meinung zu ſagen.\pend
           \pstart
           Die Plauderei »\textsc{\textcolor{green}{Anatols Hochzeitsmorgen}{}\ledrightnote{\textcolor{green}{Anatols Hochzeitsmorgen}}}« ſenden Sie gefl. baldigſt ein; wenn verwendbar, würde ich dieſelbe gerne
                    im \label{K_L00003_2v}\edtext{\textcolor{green}{Juliheft}{}\ledrightnote{→\textcolor{green}{Moderne Dichtung. Monatsschrift für Literatur und Kritik}}}{\lemma{\textnormal{\emph{Juliheft}}}\Cendnote{\textnormal{Am 7. 4. 1890 hatte
                            \textcolor{blue}{Michael Konstantin} an \textcolor{blue}{Gerhart Hauptmann} geschrieben,
                            »daß wir es uns zur Ehre rechnen würden, Ihnen unser Heft 7
                            widmen zu dürfen.« Konstantin bat um die Einsendung eines Photos
                        und einer Novelle; \textcolor{blue}{Hauptmann}{ }schickte beides, und mit \emph{\textcolor{green}{Der Apostel}} begann dann auch das Heft (\textcolor{blue}{Gerhart Hauptmann}: \emph{Notiz-Kalender. 1889–1891.} Hg. von Martin Machatzke.
                            Frankfurt am Main 1982, S. 237). Auf den Seiten
                        431–442 findet sich Schnitzlers \emph{\textcolor{green}{Anatols
                            Hochzeitsmorgen}}.}}}\label{K_L00003_2h} bringen, in welchem vornehmlich \textcolor{pink}{Oesterreich}{}\ledrightnote{\textcolor{pink}{Österreich}}er das Wort führen werden. Ich
                    ſende vom \textcolor{green}{Maiheft}{}\ledrightnote{→\textcolor{green}{Moderne Dichtung. Monatsschrift für Literatur und Kritik}} 5 Exempl.
                    als Belegnu{\geminationm}ern an Ihre Adreſſe.\pend
           \pstart
           Hochachtungsvoll{\\[\baselineskip]}\spacefill\mbox{\textcolor{gray}{\textbf{\textit{»Moderne Dichtung«}}}}\spacefill\mbox{Michael Konstantin.}\pend
           \leftskip=0em{}\endnumbering\briefempfaengerindex{Schnitzler, Arthur@\textsc{Schnitzler, Arthur}!zzzKonstantin, Michael@\emph{von Michael Konstantin}!1890-05-221@{22. 5. 1890}|)be}\mylabel{h}  \normalsize

\doendnotes{C}
\bigskip
\vfill

\clearpage

\footnotesize

\lohead{\textsc{register}}

% Definiere theindex-Environment komplett neu ohne reledmac
\makeatletter
\renewenvironment{theindex}{%
  \section*{\indexname}%
  \setlength{\parindent}{0pt}%
  \setlength{\parskip}{0pt plus 0.3pt}%
  \let\item\@idxitem
}{%
  \clearpage
}
\makeatother

\IfFileExists{\jobname-pw.ind}{\input{\jobname-pw.ind}}{}

\end{document}

      