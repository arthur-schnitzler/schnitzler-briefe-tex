%% latex-korrekturansicht-vorspann.tex
%% Vorspann für die Korrekturansicht.
%% Lädt die gemeinsame Datei latex-vorspann.tex mit gesetztem Schalter.

\newif\ifkorrekturansicht
\korrekturansichttrue

\input{../tex-inputs/latex-vorspann}


               \section[Arthur Schnitzler an Hermann Bahr, 19. 4. 1901]{ Arthur Schnitzler an Hermann Bahr, 19. 4. 1901}\nopagebreak\mylabel{v}\rehead{ }\normalsize\beginnumbering\briefempfaengerindex{Bahr, Hermann@\textsc{Bahr, Hermann}!zzzSchnitzler, Arthur@\emph{von Arthur Schnitzler}!1901-04-191@{19. 4. 1901}|(be} \toendnotes[C]{\smallbreak\pagebreak[2]} \Standort{TMW, HS AM 23342 Ba.}
\physDesc{Brief, 1 Blatt, 2 Seiten
\newline{}Handschrift: schwarze Tinte, deutsche Kurrent\newline{}Ordnung: 1) Lochung 2) mit Bleistift von unbekannter Hand datiert: »19. 4. 01«}\buchAbdrucke{\weitereDrucke{1) \emph{19. 4. 1901.} In: Arthur Schnitzler: \emph{The Letters of Arthur Schnitzler to Hermann Bahr}. Edited, annotated, and with an introduction, by Donald G.
                        Daviau. Chapel Hill: \emph{The University of North Carolina Press} 1978, S. 68 (University of North Carolina studies in the Germanic languages
                        and literatures, 89).} \weitereDrucke{2) Hermann Bahr, Arthur Schnitzler: \emph{Briefwechsel, Aufzeichnungen, Dokumente (1891–1931)}. Hg. Kurt Ifkovits und Martin Anton Müller. Göttingen: \emph{Wallstein} 2018, S. 202.} }\toendnotes[C]{\smallbreak}\pstart{}{\pb}lieber
                  Hermann,\pend\pstart
           die Vorſtellung der \textcolor{pink}{Schauſpielſchule}{}\ledrightnote{\textcolor{pink}{Konservatorium der Gesellschaft der Musikfreunde}} von der ich dir
               neulich geſprochen findet So{\geminationn}tag den 28.
                  April{ }ſtatt; u. das Fräulein \textcolor{blue}{Guſsmann}{}\ledrightnote{\textcolor{blue}{Olga Schnitzler}} wird nicht die \label{K_L01110_1v}\edtext{\textcolor{green}{Rebecca}{}\ledrightnote{→\textcolor{green}{Rosmersholm}}}{\lemma{\textnormal{\emph{Rebecca}}}\Cendnote{\textnormal{Figur aus \emph{\textcolor{green}{Rosmersholm}} von \textcolor{blue}{Ibsen}}}}\label{K_L01110_1h}{ }ſondern die \label{K_L01110_2v}\edtext{\textcolor{green}{Maria Magdalena}{}\ledrightnote{→\textcolor{green}{Maria Magdalena}}}{\lemma{\textnormal{\emph{Maria Magdalena}}}\Cendnote{\textnormal{\textcolor{blue}{Olga
                     Gussmann} hatte ursprünglich die Rolle der Protagonistin aus \textcolor{blue}{Hebbel}s \emph{\textcolor{green}{Maria
                     Magdalena}} ausgesucht; zwischenzeitlich wurde ihr dies aber untersagt
                     (vgl. A. S. \emph{Briefe} I,402).}}}\label{K_L01110_2h}{ }ſpielen, was vielleicht noch intereſſanter ſein
               dürfte. We{\geminationn} du also Zeit {\pb}und Laune haſt, möcht
               ich dich bitten zu ko{\geminationm}en. Den Sitz erhältſt du
               jedenfalls zugeſandt.\pend
           \pstart
           Herzlich grüßend dein{\\[\baselineskip]}\spacefill\mbox{Arthur Schnitzler}\pend
           \leftskip=0em{}\pstart
           \textcolor{pink}{Wien}{}\ledrightnote{\textcolor{pink}{Wien}}, 19. 4. 901.\pend
           \endnumbering\briefempfaengerindex{Bahr, Hermann@\textsc{Bahr, Hermann}!zzzSchnitzler, Arthur@\emph{von Arthur Schnitzler}!1901-04-191@{19. 4. 1901}|)be}\mylabel{h}  \normalsize

\doendnotes{C}
\bigskip
\vfill

\clearpage

\footnotesize

\lohead{\textsc{register}}

% Definiere theindex-Environment komplett neu ohne reledmac
\makeatletter
\renewenvironment{theindex}{%
  \section*{\indexname}%
  \setlength{\parindent}{0pt}%
  \setlength{\parskip}{0pt plus 0.3pt}%
  \let\item\@idxitem
}{%
  \clearpage
}
\makeatother

\IfFileExists{\jobname-pw.ind}{\input{\jobname-pw.ind}}{}

\end{document}

      