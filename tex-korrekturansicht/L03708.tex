%% latex-korrekturansicht-vorspann.tex
%% Vorspann für die Korrekturansicht.
%% Lädt die gemeinsame Datei latex-vorspann.tex mit gesetztem Schalter.

\newif\ifkorrekturansicht
\korrekturansichttrue

\input{../tex-inputs/latex-vorspann}


\section[Elsa Plessner an Arthur Schnitzler, 1. 12. 1896]{L03708 Elsa Plessner an Arthur Schnitzler, 1. 12. 1896}
\nopagebreak\mylabel{L03708v}
\rehead{ }\normalsize\beginnumbering\briefempfaengerindex{Schnitzler, Arthur@\textsc{Schnitzler, Arthur}!zzzPlessner, Elsa@\emph{von Elsa Plessner}!1896-12-011@{1. 12. 1896}|(be}
\toendnotes[C]{\smallbreak\pagebreak[2]}
\correspDesc{Versand  durch Elsa Plessner am 1. 12. 1896 in Meran
\newline{}Erhalt  durch Arthur Schnitzler im Zeitraum [2. 12. 1896
                  – 6. 12. 1896?] in Wien}\toendnotes[C]{\smallbreak}
\Standort{DLA, A:Schnitzler, HS.1985.1.419.}
\physDesc{Brief, 1 Blatt, 3 Seiten, 2238 Zeichen
\newline{}Handschrift: schwarze Tinte, lateinische Kurrent}\toendnotes[C]{\smallbreak}
\pstart
           {\pb}\textcolor{pink}{Meran, Pension Wolf}\oindex{Hotel Meranerhof@\textbf{Hotel Meranerhof}, \emph{Hotel}|pw}{}\ledrightnote{\textcolor{pink}{Hotel Meranerhof}}, den
                     1. December 1896\pend
           
\pstart
           Motto: »Unverschämt! – Was? – « (?)\pend
           
\pstart\center{}Hochverehrter Herr Doctor!\pend\vspace{0.5em}
\pstart
           \label{K_L03708-1v}\edtext{Beifolgenden Brief}{\lemma{\textnormal{\emph{Beifolgenden Brief}}}\Cendnote{\textnormal{Der abschlägige Brief von \textcolor{blue}{Otto Brahm}\pwindex{Brahm, Otto 5.\,2.\,1856 Hamburg – 28.\,11.\,1912 Berlin@\textsc{Brahm, Otto} (5.\,2.\,1856 Hamburg – 28.\,11.\,1912 Berlin), \emph{Theaterleiter, Regisseur}|pwk}, dem Leiter des \emph{\textcolor{brown}{Deutschen Theaters}\orgindex{Deutsches Theater Berlin@Deutsches Theater Berlin|pwk}} in \textcolor{pink}{Berlin}\oindex{Berlin@\textbf{Berlin}, \emph{Hauptstadt}|pwk}, ist
                  nicht überliefert. \textcolor{blue}{Plessner}\pwindex{Plessner, Elsa 22.\,8.\,1875 Wien – 7.\,5.\,1932 Alicante@\textsc{Plessner, Elsa} (22.\,8.\,1875 Wien – 7.\,5.\,1932 Alicante), \emph{Schriftstellerin}|pwk} hatte ihm in
                  Absprache mit \textcolor{blue}{Schnitzler} ihr Theaterstück
                     \emph{\textcolor{green}{Heimweh}\pwindex{Plessner, Elsa 22.\,8.\,1875 Wien – 7.\,5.\,1932 Alicante@\textsc{Plessner, Elsa} (22.\,8.\,1875 Wien – 7.\,5.\,1932 Alicante), \emph{Schriftstellerin}!Heimweh [dreiaktige Tragikomödie]@\strich\emph{Heimweh [dreiaktige Tragikomödie]}|pwk}} zur Aufführung angeboten, vgl. Elsa Plessner an Arthur Schnitzler, 28. 9. 1896.}}}\label{K_L03708-1} erhielt ich –
                  gestern von meiner \textcolor{blue}{Mama}\pwindex{Plessner, Clementine 7.\,12.\,1855 Wien – 27.\,2.\,1943 Konzentrationslager Theresienstadt@\textsc{Plessner, Clementine} (7.\,12.\,1855 Wien – 27.\,2.\,1943 Konzentrationslager Theresienstadt), \emph{Schauspielerin, Filmschauspielerin}|pw}{}\ledrightnote{\textcolor{blue}{Clementine Plessner}}
               zugesendet, nachdem sie ihn vierzehn Tage lang aus »Rücksicht für meinen
               Gesundheitszustand« und zu dessen »Schonung« zurückbehalten hat und erst \strikeout{auf} die Erwähnung meinerseits dort – (\textcolor{blue}{B}\pwindex{Brahm, Otto 5.\,2.\,1856 Hamburg – 28.\,11.\,1912 Berlin@\textsc{Brahm, Otto} (5.\,2.\,1856 Hamburg – 28.\,11.\,1912 Berlin), \emph{Theaterleiter, Regisseur}|pw}{}\ledrightnote{\textcolor{blue}{Otto Brahm}}) nochmals angefragt zu haben hat sie veranlasst, ihn
               herauszugeben. Die »Schonung«, an und für sich überflüssig, ist in diesem Fall gar
               nicht angebracht, denn ich habe ja dieses Resultat täglich erwartet und das sage ich
               ganz ehrlich!! – Sie wissen ja! – Die Pille, so {\pb}liebenswürdig in einer verbindlichen Oblate, (medicinisch richtig! – was?) hat mich
               durchaus nicht niedergeschmettert –. \uline{Vergleich
                  ausgeschlossen.} Kam die »\textcolor{green}{Athenerin}\pwindex{Ebermann, Leo 16.\,7.\,1863 Draganovka – 9.\,10.\,1914 Wien@\textsc{Ebermann, Leo} (16.\,7.\,1863 Draganovka – 9.\,10.\,1914 Wien), \emph{Schriftsteller, Journalist, Rechtswissenschaftler}!Athenerin. Drama in drei Aufzügen@\strich\emph{Die Athenerin. Drama in drei Aufzügen}|pw}{}\ledrightnote{\textcolor{green}{Die Athenerin. Drama in drei Aufzügen}}« 4
               mal von dort zurück!! \introOben{}\uuline{\edtext{sagt}{\Cendnote{vierfach unterstrichen}}} man!!\introOben{} Ganz \begin{otherlanguage}{french}\label{K_L03708-2v}\edtext{entre nous}{\lemma{\textnormal{\emph{entre nous}}}\Cendnote{\textnormal{französisch: unter uns}}}\label{K_L03708-2}{ }\end{otherlanguage} gesagt, sah ich bei der letzten Lecture meines \textcolor{green}{Opus}\pwindex{Plessner, Elsa 22.\,8.\,1875 Wien – 7.\,5.\,1932 Alicante@\textsc{Plessner, Elsa} (22.\,8.\,1875 Wien – 7.\,5.\,1932 Alicante), \emph{Schriftstellerin}!Heimweh [dreiaktige Tragikomödie]@\strich\emph{Heimweh [dreiaktige Tragikomödie]}|pwv}{}\ledrightnote{{$\rightarrow$}\emph{\textcolor{green}{Heimweh [dreiaktige Tragikomödie]}}} Schwächen die ich früher nie gesehen
               habe! \begin{otherlanguage}{french}\label{K_L03708-3v}\edtext{Chose agreable}{\lemma{\textnormal{\emph{Chose agreable}}}\Cendnote{\textnormal{französisch: angenehme Sache}}}\label{K_L03708-3}{ }\end{otherlanguage} – d. h. ich bin drüber hinaus gewachsen. Um so angenehmer, da neues \textcolor{green}{Stück}\pwindex{Plessner, Elsa 22.\,8.\,1875 Wien – 7.\,5.\,1932 Alicante@\textsc{Plessner, Elsa} (22.\,8.\,1875 Wien – 7.\,5.\,1932 Alicante), \emph{Schriftstellerin}!Orchideen [Schauspiel in drei Akten]@\strich\emph{Orchideen [Schauspiel in drei Akten]}|pwv}{}\ledrightnote{{$\rightarrow$}\emph{\textcolor{green}{Orchideen [Schauspiel in drei Akten]}}} vor mir! – Hoffe gut! –
                  \begin{otherlanguage}{italian}\label{K_L03708-4v}\edtext{Vederemo}{\lemma{\textnormal{\emph{Vederemo}}}\Cendnote{\textnormal{italienisch vedremo: wir werden sehen}}}\label{K_L03708-4}\end{otherlanguage}! – – – – Hauptsache – was mache ich jetzt mit »\textcolor{green}{Heimweh}\pwindex{Plessner, Elsa 22.\,8.\,1875 Wien – 7.\,5.\,1932 Alicante@\textsc{Plessner, Elsa} (22.\,8.\,1875 Wien – 7.\,5.\,1932 Alicante), \emph{Schriftstellerin}!Heimweh [dreiaktige Tragikomödie]@\strich\emph{Heimweh [dreiaktige Tragikomödie]}|pw}{}\ledrightnote{\textcolor{green}{Heimweh [dreiaktige Tragikomödie]}}«.  – Bitte, bitte, guten Rath!! – Bühne? – Keine
               Lust {\pb}glaube auch aussichtslos. – Was nun? – Wenn Sie so
               gut sein wollten, mir einen guten Rath zu geben – – Verlag? – \textcolor{brown}{S. Fischer}\orgindex{S. Fischer Verlag@S. Fischer Verlag|pw}{}\ledrightnote{\textcolor{brown}{S. Fischer Verlag}}? –
               \substVorne{}\textsuperscript{{\dots}}\substDazwischen{}\textcolor{brown}{A.}\orgindex{Albert Langen@Albert Langen|pw}{}\ledrightnote{\textcolor{brown}{Albert Langen}}\substHinten{}{ }\textcolor{brown}{Langen}\orgindex{Albert Langen@Albert Langen|pw}{}\ledrightnote{\textcolor{brown}{Albert Langen}} hat es im Vorjahr der \textcolor{blue}{Marholm}\pwindex{Marholm, Laura 19.\,4.\,1854 Riga – 6.\,10.\,1928 Jūrmala@\textsc{Marholm, Laura} (19.\,4.\,1854 Riga – 6.\,10.\,1928 Jūrmala), \emph{Schriftstellerin}|pw}{}\ledrightnote{\textcolor{blue}{Laura Marholm}}{ }\label{K_L03708-5v}\edtext{refusirt}{\lemma{\textnormal{\emph{refusirt}}}\Cendnote{\textnormal{Tatsächlich war \textcolor{blue}{Laura
                     Marholms}\pwindex{Marholm, Laura 19.\,4.\,1854 Riga – 6.\,10.\,1928 Jūrmala@\textsc{Marholm, Laura} (19.\,4.\,1854 Riga – 6.\,10.\,1928 Jūrmala), \emph{Schriftstellerin}|pwk} Drama \emph{\textcolor{green}{Karla Bühring}\pwindex{Marholm, Laura 19.\,4.\,1854 Riga – 6.\,10.\,1928 Jūrmala@\textsc{Marholm, Laura} (19.\,4.\,1854 Riga – 6.\,10.\,1928 Jūrmala), \emph{Schriftstellerin}!Karla Bühring. Ein Frauendrama in vier Acten@\strich\emph{Karla Bühring. Ein Frauendrama in vier Acten}|pwk}}{ }1895 bei \emph{\textcolor{brown}{A. Langen}\orgindex{Albert Langen@Albert Langen|pwk}}
                  erschienen.}}}\label{K_L03708-5}!! Möchte doch \uline{so} gern hinaus! –
               Vielleicht kindisch – »ein Buch!« Wirklich und wahrhaftig ein gedrucktes Buch!!  –
               Alte Leidenschaft von mir! – Drum – aber wahr! – Lachen Sie so herzlich Sie wollen,
               verehrter Herr Doctor, ich lache auch mit – da liegt mir gar nichts dran – aber
               rathen Sie mir!! – – – – – – Richtig! – Nochmals herzlichsten Dank für Ihre gütige
               Intervention bei Dir. \textcolor{blue}{B.}\pwindex{Brahm, Otto 5.\,2.\,1856 Hamburg – 28.\,11.\,1912 Berlin@\textsc{Brahm, Otto} (5.\,2.\,1856 Hamburg – 28.\,11.\,1912 Berlin), \emph{Theaterleiter, Regisseur}|pw}{}\ledrightnote{\textcolor{blue}{Otto Brahm}}! – Wenn Sie jetzt, wo
               die schöne \textcolor{pink}{Wiener}\oindex{Wien@\textbf{Wien}, \emph{Verwaltungsgebiet}|pw}{}\ledrightnote{\textcolor{pink}{Wien}} Saison, aus der ich mich bis zum
                  Frühjahr selber verbannt habe, so prächtig im Gange ist, ein paar
               Augenblicke für mich Zeit finden, so packen Sie sie beim Schopf und senden ein paar
               Zeilen als Strahlen der Literatursonne an einer armen, bleichsüchtigen Blaustrumpf
               und die werden mir mehr Freude bereiten, als die \introOben{}der\introOben{}{ }\textcolor{pink}{Meraner}\oindex{Meran@\textbf{Meran}, \emph{Hauptstadt}|pw}{}\ledrightnote{\textcolor{pink}{Meran}} Sonne, die auf so viel krankes
               Menschenzeug herabstrahlen müssen. – – Bitte! – Ja? – Was mach ich also?! – \pend
           
\pstart
           Voraus Dank mit zwei Dutzend Ausrufungszeichen – ergebenste Grüße{\\[\baselineskip]}\spacefill\mbox{Elsa Plessner}\pend
           \leftskip=0em{}\selectlanguage{ngerman}\endnumbering\briefempfaengerindex{Schnitzler, Arthur@\textsc{Schnitzler, Arthur}!zzzPlessner, Elsa@\emph{von Elsa Plessner}!1896-12-011@{1. 12. 1896}|)be}\mylabel{L03708h}  \normalsize

\doendnotes{C}
\bigskip
\vfill

\clearpage

\footnotesize

\lohead{\textsc{register}}

% Definiere theindex-Environment komplett neu ohne reledmac
\makeatletter
\renewenvironment{theindex}{%
  \section*{\indexname}%
  \setlength{\parindent}{0pt}%
  \setlength{\parskip}{0pt plus 0.3pt}%
  \let\item\@idxitem
}{%
  \clearpage
}
\makeatother

\IfFileExists{\jobname-pw.ind}{\input{\jobname-pw.ind}}{}

\end{document}

      