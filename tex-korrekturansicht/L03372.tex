%% latex-korrekturansicht-vorspann.tex
%% Vorspann für die Korrekturansicht.
%% Lädt die gemeinsame Datei latex-vorspann.tex mit gesetztem Schalter.

\newif\ifkorrekturansicht
\korrekturansichttrue

\input{../tex-inputs/latex-vorspann}


\renewcommand{\erwaehntePersonen}{Personen: Otto Brahm, Felix Paul Greve, Alfred de Musset, Olga Schnitzler, Oscar Wilde}
\renewcommand{\erwaehnteInstitutionen}{Institutionen: Deutsches Theater Berlin, J. C. C. Bruns}
\renewcommand{\erwaehnteOrte}{Orte: Berlin, Dessauer Straße, Frankfurt am Main, Minden, Schauspielhaus Frankfurt am Main, Wien}
\renewcommand{\erwaehnteWerke}{Werke: Berliner Theater. (»Der Schleier der Beatrice« von Arthur Schnitzler.), Der Schleier der Beatrice. Schauspiel in fünf Akten, Fingerzeige, Man soll nichts verschwören. Komödie}
\section[ Paul Goldmann an Arthur Schnitzler, 13. 5. {[}1903{]}]{Paul Goldmann an Arthur Schnitzler, 13. 5. {[}1903{]}}
\nopagebreak\mylabel{v}
\rehead{ }\normalsize\beginnumbering\briefempfaengerindex{Schnitzler, Arthur@\textsc{Schnitzler, Arthur}!zzzGoldmann, Paul@\emph{von Paul Goldmann}!1903-05-131@{13. 5. {[}1903{]}}|(be}
\toendnotes[C]{\smallbreak\pagebreak[2]}\Standort{DLA, A:Schnitzler, HS.NZ85.1.3173.}
\physDesc{Brief, 1 Blatt, 4 Seiten
\newline{}Handschrift: blaue Tinte, deutsche Kurrent
\newline{}Schnitzler: 1) mit Bleistift das Jahr »{[}1{]}903« vermerkt  2) mit rotem Buntstift eine Unterstreichung}\toendnotes[C]{\smallbreak}
\pstart
           \noindent{}\raggedleft{}{\pb}\textcolor{gray}{\textbf{\textcolor{pink}{DESSAUERSTRASSE 19}{}\ledrightnote{\textcolor{pink}{Dessauer Straße}}}}\pend
           
\pstart
           \textcolor{pink}{Berlin}{}\ledrightnote{\textcolor{pink}{Berlin}}, 13. Mai.\pend
           
\pstart\center{}Mein lieber Freund,\pend
\pstart
           Ich ſende heut an \textsc{\textcolor{blue}{Olga}{}\ledrightnote{\textcolor{blue}{Olga Schnitzler}}} die verſprochene \label{K_L03372-1v}\edtext{Tiſchglocke}{\lemma{\textnormal{\emph{Tiſchglocke}}}\Cendnote{\textnormal{eine Glocke, die
                  geklingelt wird, wenn das Essen angerichtet ist}}}\label{K_L03372-1h} ab. Ich konnte ſie nicht
               früher ſenden, weil ich ſeit meiner \label{K_L03372-2v}\edtext{Rückkehr aus \textcolor{pink}{Wien}{}\ledrightnote{\textcolor{pink}{Wien}}}{\lemma{\textnormal{\emph{Rückkehr aus Wien}}}\Cendnote{\textnormal{\textcolor{blue}{Goldmann} war nachweislich zwischen 14. 4. 1903 und 21. 4. 1903 in \textcolor{pink}{Wien}. In dieser Zeit traf er \textcolor{blue}{Schnitzler}, von dem er ursprünglich gedacht
                  hatte, dass er auf Reisen sei (vgl. A. S.: \emph{Tagebuch}, 14. 4. 1903), mehrmals.}}}\label{K_L03372-2h} ohne Diener war, der \strikeout{\textcolor{gray}{×}\textcolor{gray}{it}} mir das Paket hätte machen und expediren können. Entſchuldige mich bei \textsc{\textcolor{blue}{Olga}{}\ledrightnote{\textcolor{blue}{Olga Schnitzler}}} wegen der Verſpätung und grüße ſie herzlichſt.\pend
           
\pstart
           {\pb}Ich habe \strikeout{die} in letzter Zeit \strikeout{\textcolor{gray}{vie}}{ }\textsc{\textcolor{blue}{Oscar Wilde}{}\ledrightnote{\textcolor{blue}{Oscar Wilde}}} geleſen und in ihm einen der glänzendſten modernen Geiſter gefunden. Lies’
                  \label{K_L03372-3v}\edtext{»\textcolor{green}{Fingerzeige}{}\ledrightnote{\textcolor{green}{Fingerzeige}}«, in der Überſetzung von \textsc{\textcolor{blue}{Greve}{}\ledrightnote{\textcolor{blue}{Felix Paul Greve}}}}{\lemma{\textnormal{\emph{»Fingerzeige«, … Greve}}}\Cendnote{\textnormal{\textcolor{blue}{Oscar Wilde}: \emph{\textcolor{green}{Fingerzeige}}. Übers. v. \textcolor{blue}{Felix Paul Greve}. \textcolor{pink}{Minden}: \emph{\textcolor{brown}{J. C. C. Bruns’ Verlag}}
                        [1903?]. Eine Lektüre durch \textcolor{blue}{Schnitzler} ist nicht bekannt.}}}\label{K_L03372-3h} (\textcolor{brown}{Verlag von \textsc{Bruns}}{}\ledrightnote{\textcolor{brown}{J. C. C. Bruns}} in \textsc{\textcolor{pink}{Minden}{}\ledrightnote{\textcolor{pink}{Minden}}}). Die beiden \textcolor{green}{Dialoge}{}\ledrightnote{{$\rightarrow$}\textcolor{green}{Fingerzeige}}
               über die Kritik als \uline{ſchaffende} Kunst geben wieder,
               was ich im Innerſten über die Kritik denke, – \substVorne{}\textsuperscript{i}\substDazwischen{}m\substHinten{}it den
               { }{\pb}
               Worten eines großen Dichters und ſprühenden Geiſtes allerdings, die
               ich nie im ſtande geweſen wäre zu finden.\pend
           
\pstart
           Meine \label{K_L03372-4v}\edtext{\textcolor{green}{\textsc{Musset}-Überſetzung}{}\ledrightnote{{$\rightarrow$}\textcolor{green}{Man soll nichts verschwören. Komödie}}}{\lemma{\textnormal{\emph{Musset-Überſetzung}}}\Cendnote{\textnormal{\emph{\textcolor{green}{Man soll nichts verschwören}} (1902) hatte am 9. 5. 1903 im
                     \textcolor{pink}{Frankfurter Schauspielhaus}
                  Premiere.}}}\label{K_L03372-4h} iſt in \textcolor{pink}{Frankfurt}{}\ledrightnote{\textcolor{pink}{Frankfurt am Main}}
               durchgefallen. \textsc{\textcolor{blue}{Musset}{}\ledrightnote{\textcolor{blue}{Alfred de Musset}}} ſcheint nicht mehr bühnenmöglich zu ſein; ich habe mich durch den glänzenden
               Dialog irreführen laſſen. Wahrſcheinlich ziehe ich das \textcolor{green}{Stück}{}\ledrightnote{{$\rightarrow$}\textcolor{green}{Man soll nichts verschwören. Komödie}} nun auch in {\pb}\label{K_L03372-7v}\edtext{\textcolor{pink}{Berlin}{}\ledrightnote{\textcolor{pink}{Berlin}}}{\lemma{\textnormal{\emph{Berlin}}}\Cendnote{\textnormal{vermutlich bei \textcolor{blue}{Otto Brahm} bzw. dem \emph{\textcolor{brown}{Deutschen Theater Berlin}}}}}\label{K_L03372-7h} zurück.\pend
           
\pstart
           Ich vermiſſe ſehr Deine lieben Nachrichten. Wie geht es Dir? Warum \label{K_L03372-9v}\edtext{schweigſt}{\lemma{\textnormal{\emph{schweigſt}}}\Cendnote{\textnormal{Gut möglich, dass \textcolor{blue}{Schnitzler} aufgrund von \textcolor{blue}{Goldmann}s
                     \textcolor{green}{Feuilleton} über \emph{\textcolor{green}{Der Schleier der Beatrice}} noch immer stark
                  gekränkt war und sich deswegen nicht meldete.}}}\label{K_L03372-9h} Du ſo sehr?\pend
           
\pstart
           Viele treue Grüße! {\\[\baselineskip]}Dein {\\[\baselineskip]}\spacefill\mbox{Paul Goldmn}\pend
           \leftskip=0em{}\endnumbering\briefempfaengerindex{Schnitzler, Arthur@\textsc{Schnitzler, Arthur}!zzzGoldmann, Paul@\emph{von Paul Goldmann}!1903-05-131@{13. 5. {[}1903{]}}|)be}\mylabel{h}
\begin{anhang}
\end{anhang}\normalsize

\doendnotes{C}
\bigskip
\vfill

\clearpage

\footnotesize

\lohead{\textsc{register}}

% Definiere theindex-Environment komplett neu ohne reledmac
\makeatletter
\renewenvironment{theindex}{%
  \section*{\indexname}%
  \setlength{\parindent}{0pt}%
  \setlength{\parskip}{0pt plus 0.3pt}%
  \let\item\@idxitem
}{%
  \clearpage
}
\makeatother

\IfFileExists{\jobname-pw.ind}{\input{\jobname-pw.ind}}{}

\end{document}

      