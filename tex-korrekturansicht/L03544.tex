%% latex-korrekturansicht-vorspann.tex
%% Vorspann für die Korrekturansicht.
%% Lädt die gemeinsame Datei latex-vorspann.tex mit gesetztem Schalter.

\newif\ifkorrekturansicht
\korrekturansichttrue

\input{../tex-inputs/latex-vorspann}


\renewcommand{\erwaehntePersonen}{Personen: Alfred von Berger, Ludwig Metzl, Felix Salten, Ottilie Salten}
\renewcommand{\erwaehnteInstitutionen}{Institutionen: Burgtheater}
\renewcommand{\erwaehnteOrte}{Orte: Berlin, Cottagegasse, Edmund-Weiß-Gasse 7, Wien, XVIII., Währing}
\renewcommand{\erwaehnteWerke}{Werke: Der junge Medardus. Dramatische Historie in einem Vorspiel und fünf Aufzügen}
\section[ Felix Salten an Arthur Schnitzler, 29. 1. 1910]{Felix Salten an Arthur Schnitzler, 29. 1. 1910}
\nopagebreak\mylabel{v}
\rehead{ }\normalsize\beginnumbering\briefempfaengerindex{Schnitzler, Arthur@\textsc{Schnitzler, Arthur}!zzzSalten, Felix@\emph{von Felix Salten}!1910-01-291@{29. 1. 1910}|(be}
\toendnotes[C]{\smallbreak\pagebreak[2]}\Standort{CUL, Schnitzler, B 89, B 2.}
\physDesc{Postkarte, 475 Zeichen
\newline{}Handschrift: schwarze Tinte, lateinische Kurrent
\newline{}Versand: Stempel: »\nobreak{}\oindex{XVIII., Waehring@\textbf{XVIII., Währing}, \emph{A.ADM3}|pwk}18/\textsubscript{1} Wien 111, 29. I. \textcolor{gray}{10}, 4\nobreak{}«.  
\newline{}Ordnung: mit Bleistift von unbekannter Hand nummeriert: »259« und
                                    »2« }\toendnotes[C]{\smallbreak}\pstart{}{\pb}\textcolor{gray}{\textbf{\textit{FELIX SALTEN}}}\pend{}\pstart{}\textcolor{pink}{\textcolor{gray}{\textbf{\textit{WIEN, XVIII.}}}}{}\ledrightnote{\textcolor{pink}{XVIII., Währing}}\pend{}\pstart{}\textcolor{pink}{\textcolor{gray}{\textbf{\textit{COTTAGEGASSE 37}}}}{}\ledrightnote{\textcolor{pink}{Cottagegasse}}\pend{}
{\bigskip}\pstart{}Herrn D\textsuperscript{r} Arthur Schnitzler\pend{}\pstart{}\textcolor{pink}{Wien}{}\ledrightnote{\textcolor{pink}{Wien}}\pend{}\pstart{}\textcolor{pink}{\label{T_L03544-1v}\edtext{X\substVorne{}\textsuperscript{IX}\substDazwischen{}VI\substHinten{}II}{\lemma{\textnormal{\emph{XVIII}}}\Cendnote{\textnormal{Zur Verdeutlichung wurde  von \textcolor{blue}{Salten} »XVIII« seitlich neuerlich wiederholt.}}}\label{T_L03544-1h}. Spöttelgaße 7}{}\ledrightnote{\textcolor{pink}{Edmund-Weiß-Gasse 7}}\pend{}
{\bigskip}
\pstart{}{\pb}Lieber,\pend
\pstart
           mein Schwager \textcolor{blue}{Ludwig}{}\ledrightnote{\textcolor{blue}{Ludwig Metzl}} ist unverhofft aus \textcolor{pink}{Berlin}{}\ledrightnote{\textcolor{pink}{Berlin}} angekommen und legt mich heute, wie auch morgen,
                  Sonntag, in Beschlag. Ich kann also leider nicht mit Ihnen spazieren gehen.
               Nächster Tage \label{K_L03544-1v}\edtext{Vormittag komme ich einmal zu Ihnen}{\lemma{\textnormal{\emph{Vormittag … Ihnen}}}\Cendnote{\textnormal{siehe A. S.: \emph{Tagebuch}, 2. 2. 1910; am 1. 2. 1910 besuchte
                     \textcolor{blue}{Schnitzler}{ }\textcolor{blue}{Salten}}}}\label{K_L03544-1h}. Muss Ihnen übrigens auch vom \label{K_L03544-2v}\edtext{Baron \textcolor{blue}{B.}{}\ledrightnote{\textcolor{blue}{Alfred von Berger}}}{\lemma{\textnormal{\emph{Baron B.}}}\Cendnote{\textnormal{\textcolor{blue}{Alfred von Berger}, der neue Direktor des
                     \emph{\textcolor{brown}{Burgtheater}}s}}}\label{K_L03544-2h} erzählen, der will den
                  \textcolor{green}{Medardus}{}\ledrightnote{\textcolor{green}{Der junge Medardus. Dramatische Historie in einem Vorspiel und fünf Aufzügen}}{ }\uline{mit} der Bastei spielen. Auf Montag oder Dienstag also!\pend
           
\pstart
           Alles Herzliche von \textcolor{blue}{uns}{}\ledrightnote{{$\rightarrow$}\textcolor{blue}{Ottilie Salten}}
               zu Ihnen{\\[\baselineskip]} Ihr{\\[\baselineskip]}\spacefill\mbox{Salten}\pend
           \leftskip=0em{}
\pstart
           28. I. 10\pend
           \endnumbering\briefempfaengerindex{Schnitzler, Arthur@\textsc{Schnitzler, Arthur}!zzzSalten, Felix@\emph{von Felix Salten}!1910-01-291@{29. 1. 1910}|)be}\mylabel{h}  \normalsize

\doendnotes{C}
\bigskip
\vfill

\clearpage

\footnotesize

\lohead{\textsc{register}}

% Definiere theindex-Environment komplett neu ohne reledmac
\makeatletter
\renewenvironment{theindex}{%
  \section*{\indexname}%
  \setlength{\parindent}{0pt}%
  \setlength{\parskip}{0pt plus 0.3pt}%
  \let\item\@idxitem
}{%
  \clearpage
}
\makeatother

\IfFileExists{\jobname-pw.ind}{\input{\jobname-pw.ind}}{}

\end{document}

      