%% latex-korrekturansicht-vorspann.tex
%% Vorspann für die Korrekturansicht.
%% Lädt die gemeinsame Datei latex-vorspann.tex mit gesetztem Schalter.

\newif\ifkorrekturansicht
\korrekturansichttrue

\input{../tex-inputs/latex-vorspann}


               \section[Arthur Schnitzler an Hermann Bahr, 18. 11. 1911]{ Arthur Schnitzler an Hermann Bahr, 18. 11. 1911}\nopagebreak\mylabel{v}\rehead{ }\normalsize\beginnumbering\briefempfaengerindex{Bahr, Hermann@\textsc{Bahr, Hermann}!zzzSchnitzler, Arthur@\emph{von Arthur Schnitzler}!1911-11-181@{18. 11. 1911}|(be} \toendnotes[C]{\smallbreak\pagebreak[2]} \Standort{TMW, HS AM 60142 Ba.}
\physDesc{Bildpostkarte
\newline{}Handschrift: schwarze Tinte, deutsche Kurrent\newline{}Versand: 1) Stempel: »\nobreak{}\oindex{XIII., Hietzing@\textbf{XIII., Hietzing}, \emph{Bezirk (A.BZK)}|pwk}Wien 13 7, 18. XI. 11\nobreak{}«.  2) mit Bleistift von unbekannter Hand Postrayon »/9« zu
                                    »/7« verbessert, um eine
                                 Verwechslung mit dem namensgleichen Privatbeamten \textcolor{blue}{Hermann Bahr} in der \textcolor{pink}{Töpfelgasse 7} zu korrigieren}\buchAbdrucke{\weitereDrucke{1) \emph{18. 11. 1911, Abschrift.} In: Arthur Schnitzler: \emph{The Letters of Arthur Schnitzler to Hermann Bahr}. Edited, annotated, and with an introduction, by Donald G.
                        Daviau. Chapel Hill: \emph{The University of North Carolina Press} 1978, S. 109 (University of North Carolina studies in the Germanic languages
                        and literatures, 89).} \weitereDrucke{2) Hermann Bahr, Arthur Schnitzler: \emph{Briefwechsel, Aufzeichnungen, Dokumente (1891–1931)}. Hg. Kurt Ifkovits und Martin Anton Müller. Göttingen: \emph{Wallstein} 2018, S. 461.} }\toendnotes[C]{\smallbreak}\pstart{}{\pb}Herrn Hermann
                  Bahr\pend{}\pstart{}\textcolor{pink}{Wien XIII}{}\ledrightnote{\textcolor{pink}{XIII., Hietzing}}\pend{}\pstart{}\textsc{\textcolor{pink}{St. Veit}{}\ledrightnote{\textcolor{pink}{Ober Sankt Veit}}}\pend{}\pstart{}\textsc{\textcolor{pink}{Veilissengasse}{}\ledrightnote{\textcolor{pink}{Veitlissengasse}}}\pend{}{\bigskip}\pstart
           \noindent{}\centering{}\textcolor{gray}{\textbf{{\pb}\textcolor{pink}{Türkenschanz-Park}{}\ledrightnote{\textcolor{pink}{Türkenschanzpark}}}}\pend
           \pstart
           {\pb}\textcolor{pink}{Wien}{}\ledrightnote{\textcolor{pink}{Wien}}, 18. 11. 911.\pend
           \pstart
           herzlichen Dank, lieber Hermann, für dein und deiner verehrten \textcolor{blue}{Gattin}{}\ledrightnote{→\textcolor{blue}{Anna Bahr-Mildenburg}}{ }\textcolor{green}{Bayreuth}{}\ledrightnote{\textcolor{green}{Bayreuth}} Buch, das ich von einer Reiſe heimkehrend
               vorfinde u auf deſſen Lecture ich mich ſehr freue. Immer Dein \spacefill\mbox{Arthur}\pend
           \endnumbering\briefempfaengerindex{Bahr, Hermann@\textsc{Bahr, Hermann}!zzzSchnitzler, Arthur@\emph{von Arthur Schnitzler}!1911-11-181@{18. 11. 1911}|)be}\mylabel{h}  \normalsize

\doendnotes{C}
\bigskip
\vfill

\clearpage

\footnotesize

\lohead{\textsc{register}}

% Definiere theindex-Environment komplett neu ohne reledmac
\makeatletter
\renewenvironment{theindex}{%
  \section*{\indexname}%
  \setlength{\parindent}{0pt}%
  \setlength{\parskip}{0pt plus 0.3pt}%
  \let\item\@idxitem
}{%
  \clearpage
}
\makeatother

\IfFileExists{\jobname-pw.ind}{\input{\jobname-pw.ind}}{}

\end{document}

      