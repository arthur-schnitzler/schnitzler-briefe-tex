%% latex-korrekturansicht-vorspann.tex
%% Vorspann für die Korrekturansicht.
%% Lädt die gemeinsame Datei latex-vorspann.tex mit gesetztem Schalter.

\newif\ifkorrekturansicht
\korrekturansichttrue

\input{../tex-inputs/latex-vorspann}


               \section[Paul Goldmann an Arthur Schnitzler, Paul Goldmann an Arthur Schnitzler, 22. 9. {[}1896{]}]{ Paul Goldmann an Arthur Schnitzler, 22. 9. {[}1896{]}}\nopagebreak\mylabel{v}\rehead{ }\normalsize\beginnumbering\briefempfaengerindex{Schnitzler, Arthur@\textsc{Schnitzler, Arthur}!zzzGoldmann, Paul@\emph{von Paul Goldmann}!1896-09-221@{22. 9. {[}1896{]}}|(be} \toendnotes[C]{\smallbreak\pagebreak[2]} \Standort{DLA, A:Schnitzler, HS.NZ85.1.3166.}
\physDesc{Brief, 2 Blätter, 4 Seiten
\newline{}Handschrift: blaue Tinte, deutsche Kurrent\newline{}Beilage: handschriftlicher Brief: 1 Blatt, 2 Seiten, lila (evtl.
                                 ursprünglich schwarze?) Tinte, lateinische Kurrent. Mit Bleistift
                                 Vermerk des Datums von Schnitzler mit »Sept{[}ember{]} 96« 
\newline{}Schnitzler: 1) mit Bleistift das Jahr »96« vermerkt 2) mit rotem Buntstift zwei Unterstreichungen}\toendnotes[C]{\smallbreak}\pstart
           \noindent{}{\pb}\textcolor{gray}{\textbf{\textbf{\textcolor{brown}{Frankfurter Zeitung}{}\ledrightnote{\textcolor{brown}{Frankfurter Zeitung}}}}}\pend
           \pstart
           \textcolor{gray}{\textbf{(\textcolor{brown}{\begin{otherlanguage}{french}Gazette de Francfort\end{otherlanguage}}{}\ledrightnote{\textcolor{brown}{Frankfurter Zeitung}}).}}\pend
           \pstart
           \textcolor{gray}{\textbf{\textbf{\begin{otherlanguage}{french}Fondateur M.\end{otherlanguage}{ }\textcolor{blue}{L. Sonnemann}{}\ledrightnote{\textcolor{blue}{Leopold Sonnemann}}.}}}\pend
           \pstart
           \begin{otherlanguage}{french}\textcolor{gray}{\textbf{\textcolor{green}{Journal}{}\ledrightnote{→\textcolor{green}{Frankfurter Zeitung}} politique,
                        financier,}}\end{otherlanguage}\pend
           \pstart
           \begin{otherlanguage}{french}\textcolor{gray}{\textbf{commercial et littéraire.}}\end{otherlanguage}\pend
           \pstart
           \begin{otherlanguage}{french}\textcolor{gray}{\textbf{\textbf{Paraissant trois fois par jour.}}}\end{otherlanguage}\hfill \textsc{\textcolor{pink}{Paris}{}\ledrightnote{\textcolor{pink}{Paris}}}, 22. September.\pend
           \pstart
           \begin{otherlanguage}{french}\textcolor{gray}{\textbf{\textbf{Bureau à \textcolor{pink}{Paris}{}\ledrightnote{\textcolor{pink}{Paris}}}}}\end{otherlanguage}\pend
           \pstart
           \begin{otherlanguage}{french}\textcolor{gray}{\textbf{\textbf{\textcolor{pink}{24. Rue Feydeau}{}\ledrightnote{\textcolor{pink}{rue Feydeau}}.}}}\end{otherlanguage}\pend
           \pstart{}Mein lieber Freund,\pend\pstart
           Ich habe in dieſen Tagen ungewöhnlich viel zu thun gehabt. Auch gab es allerlei
               Aufregungen. Man \label{K_L02785-1v}\edtext{beſchimpft mich in
               der hieſigen Preſſe und verlangt meine Ausweiſung}{\lemma{\textnormal{\emph{beſchimpft … Ausweiſung}}}\Cendnote{\textnormal{\textcolor{blue}{Goldmann}s Berichterstattung über die \textcolor{blue}{Dreyfus}-Affäre führte zu einem Pistolenduell
                  zwischen \textcolor{blue}{Goldmann} und dem antisemitischen
                  Chefredakteur \textcolor{blue}{Lucien Millevoye}, das am
                     21. 11. 1896 stattfand. (siehe Arthur Schnitzler an Paul Goldmann, 21. 11. 1896) Die publizierte Invektiven gegen \textcolor{blue}{Goldmann} aus dem September 1896
                  konnten bislang nicht belegt werden.}}}\label{K_L02785-1h}, weil ich \strikeout{\textcolor{gray}{von}} für die Unſchuld des \textsc{\textcolor{blue}{Dreyfus}{}\ledrightnote{\textcolor{blue}{Alfred Dreyfus}}} eingetreten bin, von der ich, nach den neueſten Enthüllungen, feſter als je
               überzeugt bin. Zudem geht in meiner Familie Alles dunter und drüber. Kurzum ich weiß
               nicht recht, wo mir {\pb}der Kopf ſteht.\pend
           \pstart
           Dies um mich zu entſchuldigen, daß ich \strikeout{d} beifolgenden
               Brief von \textsc{\textcolor{blue}{Thorel}{}\ledrightnote{\textcolor{blue}{Jean Thorel}}} ſolange liegen ließ. Heb’ ihn Dir gut auf, denn, wie Du aus ſeinem Inhalt
               erſiehſt, vertritt er die Stelle eines Contracts. Ich habe ihn unter irgend einem
               Vorwand von 6 auf 500 heruntergeſchraubt und habe mir ausdrücklich ausbedungen, daß
               dieſe Zahlung nur als Vorſchuß auf etwaige {\pb}\strikeout{\textcolor{gray}{×}}{ }\textsc{Tantièmen} oder Honorare zu betrachten iſt. Ich fürchte
               allerdings, daß letztere Clauſel platoniſch bleiben dürfte. Nun kannſt Du das Geld
               dieſer Tage an mich ſchicken, wenn Du willſt (aber nicht wieder \label{K_L02785-2v}\edtext{in Goldſtücken in einem recommandirten
                  Brief}{\lemma{\textnormal{\emph{in … Brief}}}\Cendnote{\textnormal{siehe Paul Goldmann an Arthur Schnitzler, 21. 12. [1895]}}}\label{K_L02785-2h}). Ich werde bei dieſem Geſchäft leider nichts verdienen können, aber Du
               brauchſt hoffentlich bald wieder ein \label{K_L02785-3v}\edtext{Opernglas}{\lemma{\textnormal{\emph{Opernglas}}}\Cendnote{\textnormal{siehe Paul Goldmann an Arthur Schnitzler, 11. 1. [1896]}}}\label{K_L02785-3h}.\pend
           \pstart
           Bei \textsc{\textcolor{blue}{Forain}{}\ledrightnote{\textcolor{blue}{Jean-Louis Forain}}} war ich auch, aber er iſt noch auf {\pb}dem
               Lande.\pend
           \pstart
           Was gibts \strikeout{es} Neues bei Dir? Leben und dichten?
                  \label{K_L02785-4v}\edtext{Was hörſt Du von \textcolor{pink}{Berlin}{}\ledrightnote{\textcolor{pink}{Berlin}} und wann gehſt Du hin?}{\lemma{\textnormal{\emph{Was … hin?}}}\Cendnote{\textnormal{\textcolor{blue}{Goldmann} bezieht sich auf die bevorstehende
                  Uraufführung des Dreiakters \emph{\textcolor{green}{Freiwild}} am 3. 11. 1896 am \textcolor{pink}{Deutschen Theater} in \textcolor{pink}{Berlin}. Siehe dazu vor allem \emph{Der Briefwechsel Arthur Schnitzler — Otto Brahm}.
                     Vollständige Ausgabe. Herausgegeben, eingeleitet und erläutert von Oskar
                     Seidlin. Tübingen: \emph{Niemeyer}{ }1975, S. 14–28.\textcolor{blue}{Schnitzler} war dafür zwischen 26. 10. 1896 und 9. 11. 1896 in \textcolor{pink}{Berlin}.}}}\label{K_L02785-4h}{ }\label{K_L02785-6v}\edtext{\textsc{\textcolor{blue}{\textcolor{green}{Ebermann}{}\ledrightnote{→\textcolor{green}{Die Athenerin}}}{}\ledrightnote{\textcolor{blue}{Leo Ebermann}}}}{\lemma{\textnormal{\emph{Ebermann}}}\Cendnote{\textnormal{\textcolor{blue}{Schnitzler} war nicht nur bei Proben von \textcolor{blue}{Leo Ebermann}s Stück \emph{\textcolor{green}{Die Athenerin}} anwesend, sondern besuchte am 19. 9. 1896 auch die
                  Uraufführung im \textcolor{pink}{Burgtheater}. Siehe zum Erfolg
                  des \textcolor{green}{Stück}s etwa auch Arthur Schnitzler an Richard Beer-Hofmann, 21. 9. 1896 und A. S.: \emph{Tagebuch}, 22. 9. 1896.}}}\label{K_L02785-6h} ſcheint ja wohl einen großen Erfolg gehabt zu
               haben?\pend
           \pstart
           Lies \label{K_L02785-7v}\edtext{\textsc{\textcolor{blue}{Karl Hillebrand}{}\ledrightnote{\textcolor{blue}{Karl Hillebrand}}}: \textcolor{green}{Frankreich und die Franzoſen}{}\ledrightnote{\textcolor{green}{Frankreich und die Franzosen in der zweiten Hälfte des XIX. Jahrhunderts: Eindrücke und Erfahrungen}}}{\lemma{\textnormal{\emph{Karl … Franzoſen}}}\Cendnote{\textnormal{Eine Lektüre durch \textcolor{blue}{Schnitzler} konnte bislang nicht belegt werden}}}\label{K_L02785-7h}. Der
               einzige Deutſche, der \textcolor{pink}{Frankreich}{}\ledrightnote{\textcolor{pink}{Frankreich}} kennt, – und
               eine \textcolor{blue}{Perſönlichkeit}{}\ledrightnote{→\textcolor{blue}{Karl Hillebrand}}. Ich
               leſe \textsc{\textcolor{blue}{Schiller}{}\ledrightnote{\textcolor{blue}{Friedrich von Schiller}}s} und \textsc{\textcolor{blue}{Goethe}{}\ledrightnote{\textcolor{blue}{Johann Wolfgang von Goethe}}s}{ }\textcolor{green}{Briefwechſel}{}\ledrightnote{→\textcolor{green}{Briefwechsel zwischen Schiller und Goethe}}. Bisher iſt er
               mir unſympathiſch, und \strikeout{\textcolor{gray}{ba}} beſonders der \textsc{\textcolor{blue}{Schiller}{}\ledrightnote{\textcolor{blue}{Friedrich von Schiller}}} langweilt mich mit ſeinem verfluchten Theoretiſiren.\pend
           \pstart \label{T_L02785-1v}\edtext{Grüß’ Dich Gott, liebſter Freund!
               Schreib’ bald! Dein \spacefill\mbox{P. G.}}{\lemma{\textnormal{\emph{Grüß’ … G.}}}\Cendnote{\textnormal{seitlich am linken Rand}}}\label{T_L02785-1h}\pend{}{\bigskip}\pstart
           \raggedleft{}{\pb}{[}hs. Thorel:{]} \begin{otherlanguage}{french}\label{K_L02785-44v}\edtext{Chez \textcolor{blue}{Francis Vielé-Griffin}{}\ledrightnote{\textcolor{blue}{Francis Vielé-Griffin}}}{\lemma{\textnormal{\emph{Chez … Vielé-Griffin}}}\Cendnote{\textnormal{französisch: Bei \textcolor{blue}{Francis Vielé-Griffin}}}}\label{K_L02785-44h}\end{otherlanguage}\pend
           \pstart
           \raggedleft{}\begin{otherlanguage}{french} au \textcolor{pink}{château de
                        Noyelles}{}\ledrightnote{\textcolor{pink}{Chateau De Noyelles}}\end{otherlanguage}\pend
           \pstart
           \noindent{}\raggedleft{}\begin{otherlanguage}{french}(\textcolor{pink}{Indre-et-Loire}{}\ledrightnote{\textcolor{pink}{Indre-et-Loire}})\end{otherlanguage}\pend
           \pstart{}\label{K_L02785-55v}\edtext{\begin{otherlanguage}{french}Cher Ami,\end{otherlanguage}}{\lemma{\textnormal{\emph{Cher Ami,}}}\Cendnote{\textnormal{Lieber Freund!}}}\label{K_L02785-55h}\pend\pstart
           \label{K_L02785-23v}\edtext{\begin{otherlanguage}{french}La chose est donc convenue, aux conditions que vous dites:
                  cinq cent francs que vous me verserez aux premiers jours d’octobre. Et moi, je vais me mettre tout de suite à l’œuvre, afin
                  d’arriver en temps utile pour profiter des chances de cette saison.\end{otherlanguage}}{\lemma{\textnormal{\emph{La … saison.}}}\Cendnote{\textnormal{französisch: Die Sache ist also
                  ausgemacht, zu den von Ihnen genannten Bedingungen: fünfhundert Francs, die Sie
                  mir in den ersten Oktobertagen auszahlen werden.
                  Und ich werde mich sofort an die Arbeit machen, damit die Gelegenheiten genützt
                  werden können, die die Saison bietet.}}}\label{K_L02785-23h}\pend
           \pstart
           \label{K_L02785-66v}\edtext{\begin{otherlanguage}{french}Pour acter et de préciser le côté affaire, et pour que vous
                  pourriez envoyer un engagement signé de moi à M. Schnitzler, si vous le désirez, –
                  je rappelle ici qu’il est bien entendu que cette somme de cinq cents francs n’est
                  qu’une avance sur le droits de toute nature que pourra produire la \textcolor{green}{traduction}{}\ledrightnote{→\textcolor{green}{Amourette. Pièce en trois actes}} de \uline{\textcolor{green}{Liebelei}{}\ledrightnote{\textcolor{green}{Liebelei. Schauspiel in drei Akten}}}, droit de représentation, ou de publication en revue ou en librairie; – Et
                  pour les droits, il va de soi qu’ils seront partagés par moitiés égals entre M.
                  Schnitzler et moi –\end{otherlanguage}}{\lemma{\textnormal{\emph{Pour … –}}}\Cendnote{\textnormal{Um vorwärtszukommen und die
                  geschäftliche Seite zu präzisieren, und damit Sie, wenn Sie dies wünschen, Herrn
                     \textcolor{blue}{Schnitzler} eine von mir unterschriebene
                  Verpflichtungserklärung schicken können, – halte ich sie hier fest, um zu
                  verdeutlichen, dass diese Summe von fünfhundert Francs nur ein Vorschuss auf die
                  Rechte jeglicher Art ist, die die \textcolor{green}{Übersetzung} der \uline{\emph{\textcolor{green}{Liebelei}}} mit sich bringt, wie Aufführungsrechte oder Veröffentlichungen in
                  Zeitschriften oder Buchhandlungen; – Und die Rechte werden selbstverständlich zu
                  gleichen Teilen zwischen Herrn \textcolor{blue}{Schnitzler}
                  und mir geteilt. }}}\label{K_L02785-66h}\pend
           \pstart
           \label{K_L02785-99v}\edtext{\begin{otherlanguage}{french}Je rentrerai à \textcolor{pink}{Paris}{}\ledrightnote{\textcolor{pink}{Paris}},
                  vers la fin de {\pb}septembre. Mon adreſse est: \textcolor{pink}{Noyelles}{}\ledrightnote{→\textcolor{pink}{Chateau De Noyelles}} jusqu’au 14;{\\}et à partir du 15
                  elle sera (et moi aussi) chez \textcolor{blue}{madame \textcolor{blue}{Paul Bert}{}\ledrightnote{\textcolor{blue}{Paul Bert}}}{}\ledrightnote{→\textcolor{blue}{Josephine Clayton}} à \textcolor{pink}{\uline{Auxerre}}{}\ledrightnote{\textcolor{pink}{Auxerre}} (\textcolor{pink}{\uline{Yonne}}{}\ledrightnote{\textcolor{pink}{Yonne}})\end{otherlanguage}}{\lemma{\textnormal{\emph{Je … (Yonne)}}}\Cendnote{\textnormal{französisch: Ich werde gegen Ende September nach \textcolor{pink}{Paris}
                  zurückkehren. Meine Adreſse ist: \textcolor{pink}{Noyelles} bis zum 14.; und
                  ab dem 15. ist sie (und ich auch) bei \textcolor{blue}{Madame \textcolor{blue}{Paul Bert}} in \textcolor{pink}{\uline{Auxerre}} (\textcolor{pink}{\uline{Yonne}}). Ihnen sehr ergeben \textcolor{blue}{Jean Thorel}}}}\label{K_L02785-99h}\pend
           \pstart
           \begin{otherlanguage}{french}Votre bien dévoué\end{otherlanguage}{ }{\\[\baselineskip]}\spacefill\mbox{\textcolor{blue}{Jean Thorel}{}\ledrightnote{\textcolor{blue}{Jean Thorel}}}\pend
           \leftskip=0em{}\endnumbering\briefempfaengerindex{Schnitzler, Arthur@\textsc{Schnitzler, Arthur}!zzzGoldmann, Paul@\emph{von Paul Goldmann}!1896-09-221@{22. 9. {[}1896{]}}|)be}\mylabel{h}\begin{anhang}\end{anhang}\normalsize

\doendnotes{C}
\bigskip
\vfill

\clearpage

\footnotesize

\lohead{\textsc{register}}

% Definiere theindex-Environment komplett neu ohne reledmac
\makeatletter
\renewenvironment{theindex}{%
  \section*{\indexname}%
  \setlength{\parindent}{0pt}%
  \setlength{\parskip}{0pt plus 0.3pt}%
  \let\item\@idxitem
}{%
  \clearpage
}
\makeatother

\IfFileExists{\jobname-pw.ind}{\input{\jobname-pw.ind}}{}

\end{document}

      