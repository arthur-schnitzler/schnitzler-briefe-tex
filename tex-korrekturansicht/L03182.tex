%% latex-korrekturansicht-vorspann.tex
%% Vorspann für die Korrekturansicht.
%% Lädt die gemeinsame Datei latex-vorspann.tex mit gesetztem Schalter.

\newif\ifkorrekturansicht
\korrekturansichttrue

\input{../tex-inputs/latex-vorspann}


\renewcommand{\erwaehntePersonen}{Personen: Michael Emil Salzmann}
\renewcommand{\erwaehnteOrte}{Orte: Café Griensteidl, Wien}
\renewcommand{\erwaehnteWerke}{}
\section[ Felix Salten an Arthur Schnitzler, {[}1895–21. 1. 1897?{]}]{Felix Salten an Arthur Schnitzler, {[}1895–21. 1. 1897?{]}}
\nopagebreak\mylabel{v}
\rehead{ }\normalsize\beginnumbering\briefempfaengerindex{Schnitzler, Arthur@\textsc{Schnitzler, Arthur}!zzzSalten, Felix@\emph{von Felix Salten}!1895-01-011@{{[}1895–21. 1. 1897?{]}}|(be}
\toendnotes[C]{\smallbreak\pagebreak[2]}\Standort{CUL, Schnitzler, B 89, A 1.}
\physDesc{Brief, 1 Blatt, 1 Seite, 293 Zeichen
\newline{}Handschrift: Bleistift, lateinische Kurrent
\newline{}Ordnung: mit Bleistift von unbekannter Hand nummeriert: »82« }\toendnotes[C]{\smallbreak}
\pstart
           \noindent{}{\pb}Lieber Freund, ich bitte Sie recht sehr, leihen Sie
               mir bis zum Abend zehn Gulden. ich benöthige es recht dringend, und mei\substVorne{}\textsuperscript{\textcolor{gray}{m}}\substDazwischen{}n\substHinten{}{ }\textcolor{blue}{Bruder}{}\ledrightnote{{$\rightarrow$}\textcolor{blue}{Michael Emil Salzmann}},
               welcher Geld von mir hat, ist nicht zu Hause.\pend
           
\pstart
           Hoffentlich trifft Sie dieser Brief noch an. Ich frage Abends gegen 9 im
                  \label{K_L03182-1v}\edtext{\textcolor{pink}{Griensteidl}{}\ledrightnote{\textcolor{pink}{Café Griensteidl}}}{\lemma{\textnormal{\emph{Griensteidl}}}\Cendnote{\textnormal{Das Korrespondenzstück ist undatiert und es
                  gibt nur einen zeitlichen Anhaltspunkt: Es muss vor dem 21. 1. 1897 verfasst sein, da an diesem Tag das \textcolor{pink}{Café Griensteidl} zum letzten Mal geöffnet war. Eingeordnet
                  ist es im Nachlass am Ende der Korrespondenz von 1896,
                  weswegen wir annehmen, dass es in etwa in dieser Zeit und somit frühestens 1895
                  übermittelt wurde.}}}\label{K_L03182-1h}, wo ich Sie finde.\pend
           
\pstart
           Herzlichst {\\[\baselineskip]}\spacefill\mbox{Salten.}\pend
           \leftskip=0em{}\endnumbering\briefempfaengerindex{Schnitzler, Arthur@\textsc{Schnitzler, Arthur}!zzzSalten, Felix@\emph{von Felix Salten}!1895-01-011@{{[}1895–21. 1. 1897?{]}}|)be}\mylabel{h}  \normalsize

\doendnotes{C}
\bigskip
\vfill

\clearpage

\footnotesize

\lohead{\textsc{register}}

% Definiere theindex-Environment komplett neu ohne reledmac
\makeatletter
\renewenvironment{theindex}{%
  \section*{\indexname}%
  \setlength{\parindent}{0pt}%
  \setlength{\parskip}{0pt plus 0.3pt}%
  \let\item\@idxitem
}{%
  \clearpage
}
\makeatother

\IfFileExists{\jobname-pw.ind}{\input{\jobname-pw.ind}}{}

\end{document}

      