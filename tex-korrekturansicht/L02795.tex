%% latex-korrekturansicht-vorspann.tex
%% Vorspann für die Korrekturansicht.
%% Lädt die gemeinsame Datei latex-vorspann.tex mit gesetztem Schalter.

\newif\ifkorrekturansicht
\korrekturansichttrue

\input{../tex-inputs/latex-vorspann}


               \section[Clementine Goldmann und Vally Rosengart an Arthur Schnitzler, Clementine Goldmann und Vally Rosengart an Arthur Schnitzler, {[}11. 1. 1896{]}]{ Clementine Goldmann und Vally Rosengart an Arthur
               Schnitzler, {[}11. 1. 1896{]}}\nopagebreak\mylabel{v}\rehead{ }\normalsize\beginnumbering\briefempfaengerindex{Schnitzler, Arthur@\textsc{Schnitzler, Arthur}!zzzRosengart, Vally@\emph{von Vally Rosengart}!1896-01-112@{{[}11. 1. 1896{]}}|(be}\briefempfaengerindex{Schnitzler, Arthur@\textsc{Schnitzler, Arthur}!zzzGoldmann, Clementine@\emph{von Clementine Goldmann}!1896-01-112@{{[}11. 1. 1896{]}}|(be} \toendnotes[C]{\smallbreak\pagebreak[2]} \Standort{DLA, A:Schnitzler, HS.NZ85.1.3159.}
\physDesc{Karte
\newline{}Handschrift Clementine Goldmann: blaue Tinte, deutsche Kurrent\newline{}Handschrift Vally Rosengart: blaue Tinte, deutsche Kurrent
\newline{}Schnitzler: mit Bleistift das Datum »11/1 96« vermerkt }\toendnotes[C]{\smallbreak}\pstart
           \raggedleft{}{\pb}\textsc{Samstag}{ }Abend\pend
           \pstart{}Sehr geehrter Herr Doctor!\pend\pstart
           Nehmen Sie wärmſten Glückwunſch zu Ihrem großen \label{K_L02795-1v}\edtext{\textcolor{green}{Erfolge}{}\ledrightnote{→\textcolor{green}{Liebelei. Schauspiel in drei Akten}}}{\lemma{\textnormal{\emph{Erfolge}}}\Cendnote{\textnormal{Diese Karte ist nach der Premiere von \emph{\textcolor{green}{Liebelei}}
                  am \emph{\textcolor{brown}{Frankfurter Schauspielhaus}} verfasst, zu der \textcolor{blue}{Schnitzler}
               angereist war.}}}\label{K_L02795-1h} ud. noch beſonderen Dank für den
               ſeltenen Genuß, den Sie mir mit Ihrem geiſtvollen, {\pb}intereſſanten \textcolor{green}{Stück}{}\ledrightnote{→\textcolor{green}{Liebelei. Schauspiel in drei Akten}} bereitet. Wer ein ſo
               feiner Beobachter des Lebens iſt – wie Sie – der wird noch vieles Bedeutendes ſchaffen.\pend
           \pstart
           Auf Wiederſehen bis morgen ud.
               herzliche Grüße{\\[\baselineskip]}von Ihrer{\\[\baselineskip]}\spacefill\mbox{Clementine Goldmann.}\pend
           \leftskip=0em{}\pstart
           \noindent{}{[}hs. Rosengart:{]} Sehr veerehrter Herr Dr. – ich ſchließe mich den
               Glückwünſchen meiner Mutter auf’s herzlichſte an. Mein \label{K_L02795-8v}\edtext{\textcolor{blue}{Mann}{}\ledrightnote{→\textcolor{blue}{Josef Rosengart}}
               wird morgen früh perſönlich bei Ihnen vorſprechen}{\lemma{\textnormal{\emph{Mann … vorſprechen}}}\Cendnote{\textnormal{siehe A. S.: \emph{Tagebuch}, 12. 1. 1896}}}\label{K_L02795-8h}.\pend
           \pstart
           Mit warmem Gruß{\\[\baselineskip]}Ihre{\\[\baselineskip]}\spacefill\mbox{Vally Rosengart.}\pend
           \leftskip=0em{}\endnumbering\briefempfaengerindex{Schnitzler, Arthur@\textsc{Schnitzler, Arthur}!zzzRosengart, Vally@\emph{von Vally Rosengart}!1896-01-112@{{[}11. 1. 1896{]}}|)be}\briefempfaengerindex{Schnitzler, Arthur@\textsc{Schnitzler, Arthur}!zzzGoldmann, Clementine@\emph{von Clementine Goldmann}!1896-01-112@{{[}11. 1. 1896{]}}|)be}\mylabel{h}\begin{anhang}\end{anhang}\normalsize

\doendnotes{C}
\bigskip
\vfill

\clearpage

\footnotesize

\lohead{\textsc{register}}

% Definiere theindex-Environment komplett neu ohne reledmac
\makeatletter
\renewenvironment{theindex}{%
  \section*{\indexname}%
  \setlength{\parindent}{0pt}%
  \setlength{\parskip}{0pt plus 0.3pt}%
  \let\item\@idxitem
}{%
  \clearpage
}
\makeatother

\IfFileExists{\jobname-pw.ind}{\input{\jobname-pw.ind}}{}

\end{document}

      