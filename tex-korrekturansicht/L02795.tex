%% latex-korrekturansicht-vorspann.tex
%% Vorspann für die Korrekturansicht.
%% Lädt die gemeinsame Datei latex-vorspann.tex mit gesetztem Schalter.

\newif\ifkorrekturansicht
\korrekturansichttrue

\input{../tex-inputs/latex-vorspann}


\renewcommand{\erwaehntePersonen}{Personen: Clementine Goldmann, Vally Rosengart, Josef Rosengart}
\renewcommand{\erwaehnteInstitutionen}{Institutionen: Frankfurter Städtisches Schauspielhaus}
\renewcommand{\erwaehnteOrte}{Orte: Frankfurt am Main, Wien}
\renewcommand{\erwaehnteWerke}{Werke: Liebelei. Schauspiel in drei Akten}
\section[ Clementine Goldmann und Vally Rosengart an Arthur Schnitzler, {[}11. 1. 1896{]}]{Clementine Goldmann und Vally Rosengart an Arthur
               Schnitzler, {[}11. 1. 1896{]}}
\nopagebreak\mylabel{v}
\rehead{ }\normalsize\beginnumbering\briefempfaengerindex{Schnitzler, Arthur@\textsc{Schnitzler, Arthur}!zzzRosengart, Vally@\emph{von Vally Rosengart}!1896-01-112@{{[}11. 1. 1896{]}}|(be}\briefempfaengerindex{Schnitzler, Arthur@\textsc{Schnitzler, Arthur}!zzzGoldmann, Clementine@\emph{von Clementine Goldmann}!1896-01-112@{{[}11. 1. 1896{]}}|(be}
\toendnotes[C]{\smallbreak\pagebreak[2]}\Standort{DLA, A:Schnitzler, HS.NZ85.1.3159.}
\physDesc{Briefkarte, 566 Zeichen
\newline{}Handschrift Clementine Goldmann: blaue Tinte, deutsche Kurrent
\newline{}Handschrift Vally Rosengart: blaue Tinte, deutsche Kurrent
\newline{}Schnitzler: mit Bleistift das Datum »11/1 96« vermerkt }\toendnotes[C]{\smallbreak}
\pstart
           \raggedleft{}{\pb}\textsc{Samstag}{ }Abend\pend
           
\pstart\center{}Sehr geehrter Herr \textsc{Doctor}!\pend
\pstart
           Nehmen Sie wärmſten Glückwunſch zu Ihrem großen \label{K_L02795-1v}\edtext{\textcolor{green}{Erfolge}{}\ledrightnote{{$\rightarrow$}\textcolor{green}{Liebelei. Schauspiel in drei Akten}}}{\lemma{\textnormal{\emph{Erfolge}}}\Cendnote{\textnormal{Diese Karte wurde nach der Premiere von
                     \emph{\textcolor{green}{Liebelei}} am \emph{\textcolor{brown}{Frankfurter Schauspielhaus}} verfasst. \textcolor{blue}{Schnitzler} war zu dieser angereist.}}}\label{K_L02795-1h} ud. noch beſonderen Dank für
               den ſeltenen Genuß, den Sie mir mit Ihrem geiſtvollen, {\pb}intereſſanten \textcolor{green}{Stück}{}\ledrightnote{{$\rightarrow$}\textcolor{green}{Liebelei. Schauspiel in drei Akten}}
               bereitet. Wer ein ſo feiner Beobachter des Lebens iſt – wie Sie – der wird noch
               vieles Bedeutende ſchaffen!\pend
           
\pstart
           Auf Wiederſehen bis morgen ud. herzliche Grüße{\\[\baselineskip]}von Ihrer{\\[\baselineskip]}\spacefill\mbox{Clementine Goldmann.}\pend
           \leftskip=0em{}
\pstart
           \noindent{}{[}hs. Rosengart:{]} Sehr verehrter Herr \textsc{Dr}. –
               ich ſchließe mich den Glückwünſchen meiner Mutter auf’s herzlichſte an. Mein \label{K_L02795-2v}\edtext{\textcolor{blue}{Mann}{}\ledrightnote{{$\rightarrow$}\textcolor{blue}{Josef Rosengart}} wird morgen früh
               perſönlich bei Ihnen vorſprechen}{\lemma{\textnormal{\emph{Mann … vorſprechen}}}\Cendnote{\textnormal{siehe A. S.: \emph{Tagebuch}, 12. 1. 1896}}}\label{K_L02795-2h}. Mit warmem Gruß{\\}Ihre{\\}\spacefill\mbox{Vally Rosengart.}\pend
           \endnumbering\briefempfaengerindex{Schnitzler, Arthur@\textsc{Schnitzler, Arthur}!zzzRosengart, Vally@\emph{von Vally Rosengart}!1896-01-112@{{[}11. 1. 1896{]}}|)be}\briefempfaengerindex{Schnitzler, Arthur@\textsc{Schnitzler, Arthur}!zzzGoldmann, Clementine@\emph{von Clementine Goldmann}!1896-01-112@{{[}11. 1. 1896{]}}|)be}\mylabel{h}  \normalsize

\doendnotes{C}
\bigskip
\vfill

\clearpage

\footnotesize

\lohead{\textsc{register}}

% Definiere theindex-Environment komplett neu ohne reledmac
\makeatletter
\renewenvironment{theindex}{%
  \section*{\indexname}%
  \setlength{\parindent}{0pt}%
  \setlength{\parskip}{0pt plus 0.3pt}%
  \let\item\@idxitem
}{%
  \clearpage
}
\makeatother

\IfFileExists{\jobname-pw.ind}{\input{\jobname-pw.ind}}{}

\end{document}

      