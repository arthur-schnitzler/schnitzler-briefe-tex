%% latex-korrekturansicht-vorspann.tex
%% Vorspann für die Korrekturansicht.
%% Lädt die gemeinsame Datei latex-vorspann.tex mit gesetztem Schalter.

\newif\ifkorrekturansicht
\korrekturansichttrue

\input{../tex-inputs/latex-vorspann}


\renewcommand{\erwaehntePersonen}{Personen: Hugo Haberfeld, Richard Klein, Ottilie Salten, Heinrich Schnitzler, Olga Schnitzler, Isidor Singer}
\renewcommand{\erwaehnteOrte}{Orte: Italien, Neapel, Palermo, Pompei, Rom, Taormina, Wien}
\renewcommand{\erwaehnteWerke}{}
\section[ Felix Salten an Arthur Schnitzler, {[}14. 4. 1904{]}]{Felix Salten an Arthur Schnitzler, {[}14. 4. 1904{]}}
\nopagebreak\mylabel{v}
\rehead{ }\normalsize\beginnumbering\briefempfaengerindex{Schnitzler, Arthur@\textsc{Schnitzler, Arthur}!zzzSalten, Felix@\emph{von Felix Salten}!1904-04-141@{{[}14. 4. 1904{]}}|(be}
\toendnotes[C]{\smallbreak\pagebreak[2]}\Standort{CUL, Schnitzler, B 89, B 1.}
\physDesc{Brief, 1 Blatt, 2 Seiten, 668 Zeichen
\newline{}Handschrift: Bleistift, lateinische Kurrent
\newline{}Schnitzler: mit Bleistift datiert: »1\substVorne{}\textsuperscript{5}\substDazwischen{}4\substHinten{}/4 904« 
\newline{}Ordnung: mit Bleistift von unbekannter Hand nummeriert: »187« }\toendnotes[C]{\smallbreak}
\pstart
           \raggedleft{}{\pb}Donnerstag\pend
           
\pstart
           Lieber Arthur,{ }gestern hörte ich durch einen Zufall, dass Ihr \textcolor{blue}{Bub}{}\ledrightnote{{$\rightarrow$}\textcolor{blue}{Heinrich Schnitzler}} Masern hat. Ihr \label{K_L03395-1v}\edtext{Brief heute}{\lemma{\textnormal{\emph{Brief heute}}}\Cendnote{\textnormal{Arthur Schnitzler an Felix Salten, 13. 4. 1904}}}\label{K_L03395-1h} läßt erfreulicherweise die Vermuthung zu, dass die Sache garnicht arg ist.
               Wollen es hoffen und herzlichst wünschen. Wird Ihre \label{K_L03395-2v}\edtext{Reise}{\lemma{\textnormal{\emph{Reise}}}\Cendnote{\textnormal{Zwischen
                     1. 5. 1904 und
                     29. 5. 1904
                  reisten \textcolor{blue}{Arthur} und \textcolor{blue}{Olga Schnitzler} nach \textcolor{pink}{Italien}. Die Hauptstationen bildeten \textcolor{pink}{Rom}, \textcolor{pink}{Neapel}, \textcolor{pink}{Pompei}, \textcolor{pink}{Palermo} und \textcolor{pink}{Taormina}.}}}\label{K_L03395-2h} dadurch wesentlich verschoben?
               Wenn es mit \textcolor{blue}{Heini}{}\ledrightnote{\textcolor{blue}{Heinrich Schnitzler}} soweit besser geworden,
               möchten wir Sie gerne noch \label{K_L03395-3v}\edtext{einen Abend
               bei uns sehen}{\lemma{\textnormal{\emph{einen … sehen}}}\Cendnote{\textnormal{Vor der Abreise sahen sich
                     \textcolor{blue}{Schnitzler} und \textcolor{blue}{Salten} nachweislich am 27. 4. 1904 im Kaffeehaus.}}}\label{K_L03395-3h}, ehe Sie
               abreisen.\pend
           
\pstart
           Über \textcolor{blue}{Klein}{}\ledrightnote{\textcolor{blue}{Richard Klein}} würde ich gerne schreiben. Leider
               gehts nicht. Und ich steh’ mit D\textsuperscript{r}{ }\textcolor{blue}{H.}{}\ledrightnote{\textcolor{blue}{Hugo Haberfeld}} nicht so, dass ich ihm was sagen \substVorne{}\textsuperscript{\textcolor{gray}{un}}\substDazwischen{}kö\substHinten{}nnte.
               Deshalb werde ich also versuchen, Ihre Bitte dem Professor \label{K_L03395-4v}\edtext{\textcolor{blue}{Singer}{}\ledrightnote{\textcolor{blue}{Isidor Singer}}}{\lemma{\textnormal{\emph{Singer}}}\Cendnote{\textnormal{kein Bericht nachweisbar}}}\label{K_L03395-4h} zu
               comuniziren.\pend
           
\pstart
           Bitte geben Sie bald Nachricht, {\pb}wie es bei Ihnen geht.\pend
           
\pstart
           Herzl. Grüße von \textcolor{blue}{Otti}{}\ledrightnote{\textcolor{blue}{Ottilie Salten}} und mir an Sie \textcolor{blue}{Beide}{}\ledrightnote{{$\rightarrow$}\textcolor{blue}{Olga Schnitzler}}.\pend
           
\pstart
           Ihr {\\[\baselineskip]}\spacefill\mbox{S.}\pend
           \leftskip=0em{}\endnumbering\briefempfaengerindex{Schnitzler, Arthur@\textsc{Schnitzler, Arthur}!zzzSalten, Felix@\emph{von Felix Salten}!1904-04-141@{{[}14. 4. 1904{]}}|)be}\mylabel{h}  \normalsize

\doendnotes{C}
\bigskip
\vfill

\clearpage

\footnotesize

\lohead{\textsc{register}}

% Definiere theindex-Environment komplett neu ohne reledmac
\makeatletter
\renewenvironment{theindex}{%
  \section*{\indexname}%
  \setlength{\parindent}{0pt}%
  \setlength{\parskip}{0pt plus 0.3pt}%
  \let\item\@idxitem
}{%
  \clearpage
}
\makeatother

\IfFileExists{\jobname-pw.ind}{\input{\jobname-pw.ind}}{}

\end{document}

      