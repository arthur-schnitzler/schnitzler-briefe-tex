%% latex-korrekturansicht-vorspann.tex
%% Vorspann für die Korrekturansicht.
%% Lädt die gemeinsame Datei latex-vorspann.tex mit gesetztem Schalter.

\newif\ifkorrekturansicht
\korrekturansichttrue

\input{../tex-inputs/latex-vorspann}


               \section[Arthur Schnitzler: Widmungsexemplar Reigen für Hermann Bahr, {[}2.?{]} 4. 1903]{ Arthur Schnitzler: Widmungsexemplar Reigen für Hermann Bahr,
               {[}2.?{]} 4. 1903}\nopagebreak\mylabel{v}\rehead{ }\normalsize\beginnumbering\briefempfaengerindex{Bahr, Hermann@\textsc{Bahr, Hermann}!zzzSchnitzler, Arthur@\emph{von Arthur Schnitzler}!1903-04-021@{{[}2.?{]} 4. 1903}|(be} \toendnotes[C]{\smallbreak\pagebreak[2]} \Standort{Wien, Antiquariat Georg Fritsch, Katalog, September 2015.}
\physDesc{Widmung am Vortitel
\newline{}Handschrift: schwarze Tinte, deutsche Kurrent\newline{}Zusatz: Als Empfänger ist Bahr anzunehmen, da die Verwendung des
                                 Vornamens mit den Formulierungen weiterer Widmungen übereinstimmt;
                                 Provenienz und Verbleib ungeklärt }\buchAbdrucke{\weitereDrucke{Hermann Bahr, Arthur Schnitzler: \emph{Briefwechsel, Aufzeichnungen, Dokumente (1891–1931)}. Hg. Kurt Ifkovits und Martin Anton Müller. Göttingen: \emph{Wallstein} 2018, S. 258.} }\pstart
           \noindent{}{\pb}\textsc{Meinem lieben Hermann}\pend
           \pstart
           \textcolor{pink}{Wien}{}\ledrightnote{\textcolor{pink}{Wien}}{ }April 903. \pend
           \pstart \spacefill\mbox{Arthur}\pend{}{\bigskip}\pstart
           \noindent{}\centering{}\textcolor{gray}{\textbf{\textcolor{green}{REIGEN}{}\ledrightnote{\textcolor{green}{Reigen. Zehn Dialoge}}}}\pend
           {\bigskip}\pstart
           \noindent{}\centering{}{\pb}\textcolor{gray}{\textbf{ARTHUR}}\pend
           \pstart
           \noindent{}\centering{}\textcolor{gray}{\textbf{SCHNITZLER}}\pend
           \pstart
           \noindent{}\centering{}\textcolor{gray}{\textbf{\textcolor{green}{REIGEN}{}\ledrightnote{\textcolor{green}{Reigen. Zehn Dialoge}}}}\pend
           \pstart
           \noindent{}\centering{}\textcolor{gray}{\textbf{ZEHN DIALOGE}}\pend
           \pstart
           \noindent{}\centering{}\textcolor{gray}{\textbf{GESCHRIEBEN Winter 1896–97}}\pend
           \pstart
           \noindent{}\centering{}\textcolor{gray}{\textbf{BUCHSCHMUCK VON \textcolor{blue}{BERTHOLD
                        LÖFFLER}{}\ledrightnote{\textcolor{blue}{Bertold Löffler}}}}\pend
           {\bigskip}\pstart
           \noindent{}\centering{}\textcolor{gray}{\textbf{\textcolor{brown}{WIENER VERLAG}{}\ledrightnote{\textcolor{brown}{Wiener Verlag}}}}\pend
           \pstart
           \noindent{}\centering{}\textcolor{gray}{\textbf{\textcolor{pink}{WIEN}{}\ledrightnote{\textcolor{pink}{Wien}} UND \textcolor{pink}{LEIPZIG}{}\ledrightnote{\textcolor{pink}{Leipzig}}}}\pend
           \pstart
           \noindent{}\centering{}\textcolor{gray}{\textbf{1903}}\pend
           \endnumbering\briefempfaengerindex{Bahr, Hermann@\textsc{Bahr, Hermann}!zzzSchnitzler, Arthur@\emph{von Arthur Schnitzler}!1903-04-021@{{[}2.?{]} 4. 1903}|)be}\mylabel{h}  \normalsize

\doendnotes{C}
\bigskip
\vfill

\clearpage

\footnotesize

\lohead{\textsc{register}}

% Definiere theindex-Environment komplett neu ohne reledmac
\makeatletter
\renewenvironment{theindex}{%
  \section*{\indexname}%
  \setlength{\parindent}{0pt}%
  \setlength{\parskip}{0pt plus 0.3pt}%
  \let\item\@idxitem
}{%
  \clearpage
}
\makeatother

\IfFileExists{\jobname-pw.ind}{\input{\jobname-pw.ind}}{}

\end{document}

      