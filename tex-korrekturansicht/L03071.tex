%% latex-korrekturansicht-vorspann.tex
%% Vorspann für die Korrekturansicht.
%% Lädt die gemeinsame Datei latex-vorspann.tex mit gesetztem Schalter.

\newif\ifkorrekturansicht
\korrekturansichttrue

\input{../tex-inputs/latex-vorspann}


\renewcommand{\erwaehntePersonen}{Personen: Alfred Kerr, Olga Schnitzler, Elisabeth Steinrück}
\renewcommand{\erwaehnteInstitutionen}{Institutionen: Südbahn-Gesellschaft}
\renewcommand{\erwaehnteOrte}{Orte: Berlin, Dessauer Straße, Kärnten, Schweiz, St. Anton am Arlberg, Südtirol, Tirol, Wörthersee, Österreich}
\renewcommand{\erwaehnteWerke}{}
\section[Paul Goldmann an Arthur Schnitzler, Olga und Elisabeth Gussmann, 3. 7. {[}1901{]}]{Paul Goldmann an Arthur Schnitzler, Olga und Elisabeth
               Gussmann, 3. 7. {[}1901{]}}
\nopagebreak\mylabel{v}
\rehead{ }\normalsize\beginnumbering\briefempfaengerindex{Steinrueck, Elisabeth@\textsc{Steinrück, Elisabeth}!zzzGoldmann, Paul@\emph{von Paul Goldmann}!1901-07-032@{3. 7. {[}1901{]}}|(be}\briefempfaengerindex{Schnitzler, Olga@\textsc{Schnitzler, Olga}!zzzGoldmann, Paul@\emph{von Paul Goldmann}!1901-07-032@{3. 7. {[}1901{]}}|(be}\briefempfaengerindex{Schnitzler, Arthur@\textsc{Schnitzler, Arthur}!zzzGoldmann, Paul@\emph{von Paul Goldmann}!1901-07-032@{3. 7. {[}1901{]}}|(be}
\toendnotes[C]{\smallbreak\pagebreak[2]}\Standort{DLA, A:Schnitzler, HS.NZ85.1.3171.}
\physDesc{Brief, 2 Blätter, 7 Seiten
\newline{}Handschrift: blaue Tinte, deutsche Kurrent
\newline{}Schnitzler: mit rotem Buntstift eine Unterstreichung }\toendnotes[C]{\smallbreak}
\pstart
           \noindent{}\raggedleft{}{\pb}\textcolor{pink}{\textcolor{gray}{\textbf{DESSAUERSTRASSE 19}}}{}\ledrightnote{\textcolor{pink}{Dessauer Straße}}\pend
           
\pstart
           \textcolor{pink}{Berlin}{}\ledrightnote{\textcolor{pink}{Berlin}}, 3. Juli.\pend
           
\pstart\center{}Mein lieber Freund,\pend
\pstart
           Ich habe mich ſehr mit Deinem und der kleinen \textsc{\textcolor{blue}{Liesl}{}\ledrightnote{\textcolor{blue}{Elisabeth Steinrück}}} Briefe gefreut.\pend
           
\pstart
           Du kannſt Dir denken, wie gern ich mit Euch Allen zuſammenſein würde. Aber Du machſt
               es mir gar zu \strikeout{ſch\textcolor{gray}{we}} ſchwer; und wenn Du nach der \label{K_L03071-1v}\edtext{\textcolor{pink}{Schweiz}{}\ledrightnote{\textcolor{pink}{Schweiz}}}{\lemma{\textnormal{\emph{Schweiz}}}\Cendnote{\textnormal{\textcolor{blue}{Schnitzler} war im Sommer 1901 nicht in der \textcolor{pink}{Schweiz}. Er und
                     \textcolor{blue}{Goldmann} konnten sich treffen, vgl. Paul Goldmann an Arthur Schnitzler, 26. 4. [1901].}}}\label{K_L03071-1h} gehſt, wird es
               ganz unmöglich ſein. Ich bekomme eine Freikarte auf der \textcolor{brown}{Südbahn}{}\ledrightnote{\textcolor{brown}{Südbahn-Gesellschaft}}. Danach muß ich mich richten, bei meinen beſchränkten
               Geldmitteln. Wenn Du alſo mit mir {\pb}zuſammen ſein
               willſt, ſo mußt Du mir entgegenkommen. Das heißt alſo: Gehſt Du nach \introOben{}\textcolor{pink}{Kärnthen}{}\ledrightnote{\textcolor{pink}{Kärnten}} oder\introOben{}{ }\textcolor{pink}{Tirol}{}\ledrightnote{\textcolor{pink}{Tirol}}, nach \textcolor{pink}{Südtirol}{}\ledrightnote{\textcolor{pink}{Südtirol}} womöglich, ſo werden wir uns ſehen. Wenn \strikeout{\textcolor{gray}{nic}h\textcolor{gray}{t}} nicht, ſo werde ich diesmal meinen Urlaub in \textcolor{pink}{Öſterreich}{}\ledrightnote{\textcolor{pink}{Österreich}} verbringen, ohne Dir die Hand drücken zu können, und das wird
               ſehr traurig ſein. Im Übrigen denke ich mir: Ihr \textcolor{blue}{Zwei}{}\ledrightnote{{$\rightarrow$}\textcolor{blue}{Olga Schnitzler}} ſeid glücklich miteinander. Gewiß, ich würde Euch
               nicht ſtören. Aber ſoll ich mir das anthun, ich Einſamer, {\pb}dem Alles verſagt iſt, in der Nähe eines ſo großen
               Glücks zu leben?\pend
           
\pstart
           Theile mir alſo \introOben{}(und zwar möglichſt raſch)\introOben{}{ }\strikeout{noch N\textcolor{gray}{ä}} Jedenfalls noch Näheres über Deine Reiſepläne mit! \textsc{\textcolor{blue}{Kerr}{}\ledrightnote{\textcolor{blue}{Alfred Kerr}}} möchte auch mit Dir und mir zuſammen ſein. Soll ich ihm ſagen, wo Du biſt? Und
               mit wem? Einſtweilen haben \textsc{\textcolor{blue}{Kerr}{}\ledrightnote{\textcolor{blue}{Alfred Kerr}}} und ich feſtgeſetzt, daß wir uns am \textcolor{pink}{Wörtherſee}{}\ledrightnote{\textcolor{pink}{Wörthersee}} treffen und vielleicht \label{K_L03071-7v}\edtext{zuſammen hingehn}{\lemma{\textnormal{\emph{zuſammen hingehn}}}\Cendnote{\textnormal{nicht geschehen}}}\label{K_L03071-7h}\substVorne{}\textsuperscript{?}\substDazwischen{}.\substHinten{}\pend
           
\pstart
           {\pb}Sei vielmals und von Herzen gegrüßt von {\\[\baselineskip]}Deinem
                  \spacefill\mbox{Paul Goldmann}\pend
           \leftskip=0em{}{\bigskip}
\pstart
           \noindent{}Liebes Fräulein \textsc{Olga}, Ich
               danke für Ihre lieben Zeilen und freue mich auf Ihren Brief. Könnten Sie nicht den
                  \textsc{Arthur} beſtimmen, daß er nach \textcolor{pink}{Tirol}{}\ledrightnote{\textcolor{pink}{Tirol}} oder \textcolor{pink}{Kärnthen}{}\ledrightnote{\textcolor{pink}{Kärnten}} geht
               ſtatt nach der \textcolor{pink}{Schweiz}{}\ledrightnote{\textcolor{pink}{Schweiz}}? Nach mir richtet er
               ſich nicht; das weiß ich aus Erfahrung. Aber wenn Sie es verlangen, richtet er ſich
               vielleicht nach Ihnen. Das Ganze kann ja ein Geheimniß bleiben zwiſchen uns \strikeout{Beiden.} Beiden.\pend
           
\pstart
           Herzlichſt Ihr {\\[\baselineskip]}\spacefill\mbox{Dr. Paul Goldmann.}\pend
           \leftskip=0em{}{\bigskip}
\pstart{}{\pb}Liebes Fräulein Liesl,\pend
\pstart
           Mir fällt ein, daß ich Ihnen auch gleich antworten möchte. Ich danke Ihnen für Ihr
               liebes Briefchen, und es thut mir unendlich leid, daß Sie ſoviel \label{K_L03071-12v}\edtext{Kummer}{\lemma{\textnormal{\emph{Kummer}}}\Cendnote{\textnormal{\textcolor{blue}{Elisabeth Gussmann} dürfte erkrankt sein,
                     siehe Paul Goldmann an Arthur Schnitzler und Olga
               Gussmann, 7. 7. [1901].}}}\label{K_L03071-12h} gehabt
               haben. Aber warten Sie nur, es wird ſchon beſſer kommen. Ich möchte Sie gern
               wiederſehen und ein Bischen mit Ihnen plaudern und Sie quietſchen hören (quietſchen
               Sie noch ſo gut?). Aber dieſer {\pb}Schurke, der \textsc{Arthur} (bitte, \strikeout{S\textcolor{gray}{a}} ſagen Sie es ihm \strikeout{\textcolor{gray}{×}\-\textcolor{gray}{×}\-\textcolor{gray}{×}} nicht, daß ich ihn Schurke genannt habe) will nach der \textcolor{pink}{Schweiz}{}\ledrightnote{\textcolor{pink}{Schweiz}} gehen. So macht er es mir unmöglich, mit Ihnen
               zuſammenzukommen. Ich glaube, er thut es abſichtlich. Er will beide \textcolor{blue}{Schweſtern}{}\ledrightnote{\textcolor{blue}{Olga Schnitzler}{\newline}\textcolor{blue}{Elisabeth Steinrück}} ganz für ſich haben und gönnt ſie Keinem.
               Er war immer ſo ein Intriguant.\pend
           
\pstart
           {\pb}Bitte, ſchreiben Sie mir bald wieder, und ſeien Sie
               herzlichſt gegrüßt von {\\[\baselineskip]}Ihrem {\\[\baselineskip]}\spacefill\mbox{Dr. Paul Goldmann.}\pend
           \leftskip=0em{}\endnumbering\briefempfaengerindex{Steinrueck, Elisabeth@\textsc{Steinrück, Elisabeth}!zzzGoldmann, Paul@\emph{von Paul Goldmann}!1901-07-032@{3. 7. {[}1901{]}}|)be}\briefempfaengerindex{Schnitzler, Olga@\textsc{Schnitzler, Olga}!zzzGoldmann, Paul@\emph{von Paul Goldmann}!1901-07-032@{3. 7. {[}1901{]}}|)be}\briefempfaengerindex{Schnitzler, Arthur@\textsc{Schnitzler, Arthur}!zzzGoldmann, Paul@\emph{von Paul Goldmann}!1901-07-032@{3. 7. {[}1901{]}}|)be}\mylabel{h}
\begin{anhang}
\end{anhang}\normalsize

\doendnotes{C}
\bigskip
\vfill

\clearpage

\footnotesize

\lohead{\textsc{register}}

% Definiere theindex-Environment komplett neu ohne reledmac
\makeatletter
\renewenvironment{theindex}{%
  \section*{\indexname}%
  \setlength{\parindent}{0pt}%
  \setlength{\parskip}{0pt plus 0.3pt}%
  \let\item\@idxitem
}{%
  \clearpage
}
\makeatother

\IfFileExists{\jobname-pw.ind}{\input{\jobname-pw.ind}}{}

\end{document}

      