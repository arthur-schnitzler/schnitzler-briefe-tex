%% latex-korrekturansicht-vorspann.tex
%% Vorspann für die Korrekturansicht.
%% Lädt die gemeinsame Datei latex-vorspann.tex mit gesetztem Schalter.

\newif\ifkorrekturansicht
\korrekturansichttrue

\input{../tex-inputs/latex-vorspann}


               \section[Paul Goldmann an Arthur Schnitzler, Paul Goldmann an Arthur Schnitzler, 9. {[}4.{]} 1896]{ Paul Goldmann an Arthur Schnitzler, 9. {[}4.{]} 1896}\nopagebreak\mylabel{v}\rehead{ }\normalsize\beginnumbering\briefempfaengerindex{Schnitzler, Arthur@\textsc{Schnitzler, Arthur}!zzzGoldmann, Paul@\emph{von Paul Goldmann}!1896-04-091@{9. {[}4.{]} 1896}|(be} \toendnotes[C]{\smallbreak\pagebreak[2]} \Standort{DLA, A:Schnitzler, HS.NZ85.1.3166.}
\physDesc{Brief, 2 Blätter, 8 Seiten
\newline{}Handschrift: blaue Tinte, deutsche Kurrent
\newline{}Schnitzler: 1) mit Bleistift \textcolor{blue}{Goldmann}s
                                 Datierung »März« durchgestrichen und darunter »April« vermerkt 2) mit rotem Buntstift zwei Unterstreichungen}\toendnotes[C]{\smallbreak}\pstart
           \noindent{}{\pb}\textcolor{gray}{\textbf{\textbf{\textcolor{brown}{Frankfurter Zeitung}{}\ledrightnote{\textcolor{brown}{Frankfurter Zeitung}}}}}\hfill \textcolor{gray}{\textbf{\textcolor{pink}{Frankfurt a. M.}{}\ledrightnote{\textcolor{pink}{Frankfurt am Main}}, }}9. März \textcolor{gray}{\textbf{189}}6.\pend
           \pstart
           \textcolor{gray}{\textbf{und}}\pend
           \pstart
           \textcolor{gray}{\textbf{\textcolor{brown}{Handelsblatt}{}\ledrightnote{→\textcolor{brown}{Frankfurter Zeitung}}}}\pend
           \pstart
           \textcolor{gray}{\textbf{\textbf{\textcolor{brown}{Redaktion}{}\ledrightnote{→\textcolor{brown}{Frankfurter Zeitung}}.\footnote{\noindent{}\textcolor{gray}{\textbf{Für die Redaktion beſtimmte Briefe und Sendungen
                                 wolle man \so{nicht} an die Perſon eines
                                 Redakteurs, ſondern ſtets \textbf{an die Redaktion der
                                    Frankfurter Zeitung} adreſſiren}}.}}}}\pend
           \pstart
           \textcolor{gray}{\textbf{Telegramm-Adresse:}}\pend
           \pstart
           \textcolor{gray}{\textbf{\textbf{\textcolor{brown}{Zeitung}{}\ledrightnote{→\textcolor{brown}{Frankfurter Zeitung}}{ }\textcolor{pink}{Frankfurt Main}{}\ledrightnote{\textcolor{pink}{Frankfurt am Main}}.}}}\pend
           \pstart\center{}Mein lieber Freund,\pend\pstart
           Ich bekam Deinen lieben Brief hierher nachgeſandt, kann Dir alſo den Brief, von dem
               Du ſprichſt, erſt nächſte Woche nach meiner Rückkehr zurückfenden.\pend
           \pstart
           Du ſollſt nur einen kurzen Gruß von unterwegs erhalten. Ich bin hier, müde und
               ruhebedürftig. Mein \strikeout{ A\textcolor{gray}{ug}} Auge iſt krank, und \strikeout{\textcolor{gray}{×}} auch die Ruhe will nicht mehr viel nutzen. \label{T_L02771-4v}\edtext{Hieſige}{\lemma{\textnormal{\emph{Hieſige}}}\Cendnote{\textnormal{in der
                  Vorlage steht: »Hiesiger«}}}\label{T_L02771-4h} Eindrücke wenig erfreulich. Meine
               Familie, die {\pb}friedliche, in \strikeout{z\textcolor{gray}{×}} Parteien geſpalten, – aufgelöſt durch das neu hinzugekommene \label{K_L02771-1v}\edtext{\textsc{dissolvant}}{\lemma{\textnormal{\emph{dissolvant}}}\Cendnote{\textnormal{französisch: Lösungsmittel; womöglich
                  ist \textcolor{blue}{Johanna Schwabacher} gemeint, deren
                  Heirat mit \textcolor{blue}{Fedor Mamroth} bevorstand.}}}\label{K_L02771-1h}.
               Schlimme Dinge, ſchlimmme Dinge! \pend
           \pstart
           Von Dir ſpricht alle Welt mit wärmſter Sympathie, und während Deines Aufenhalts in
                  \textcolor{pink}{Frankfurt}{}\ledrightnote{\textcolor{pink}{Frankfurt am Main}} haſt Du bei uns alle Herzen
               gewonnen. Freundlich grüßt mich Dein Name aus den Schaufenſtern der
               Buchhandlungen.\pend
           \pstart
           Was Du mir über Deine Stimmungen ſchreibſt, iſt gar ſeltſam. Daß auch Du dieſe Idee
               haſt, Dein Leben zu verlieren\textcolor{gray}{,}{ }{\pb}Du, deſſen Leben reich iſt, wie kein zweites, das
               ich kenne. So ſcheint es, daß \strikeout{\textcolor{gray}{×}} wir auf allen Stufen, bei allen Geſchicken, im Glück und Unglück das Gefühl
               haben, das Leben zu verlieren; und vielleicht verlieren wirs auch Alle wirklich.\pend
           \pstart
           Gern möchte ich Dich im Sommer wiederſehen, vorausgeſetzt, daß ich bis dahin noch in
               keinem Spital liege: \textcolor{pink}{Holland}{}\ledrightnote{\textcolor{pink}{Niederlande}}, \textcolor{pink}{Dänemark}{}\ledrightnote{\textcolor{pink}{Dänemark}}, wo Du willſt. Freilich wirſt Du bei unſerem
               Wiederſehen {\pb}merken, daß ſich Manches verändert
               hat.\pend
           \pstart
           Und warum kommſt Du nicht nach \textsc{\textcolor{pink}{Paris}{}\ledrightnote{\textcolor{pink}{Paris}}}? \pend
           \pstart
           Dem \textsc{\textcolor{blue}{Hugo}{}\ledrightnote{\textcolor{blue}{Hugo von Hofmannsthal}}} thue ich \uline{nicht} Unrecht. Ich ſoll den \label{K_L02771-66v}\edtext{\textcolor{green}{Artikel}{}\ledrightnote{→\textcolor{green}{Gedichte von Stefan George}}}{\lemma{\textnormal{\emph{Artikel}}}\Cendnote{\textnormal{\textcolor{blue}{Hugo von Hofmannsthal}: \emph{\textcolor{green}{Gedichte von Stefan George}}. In: \emph{\textcolor{green}{Die Zeit}}, Bd. 6, Nr. 77, 21. 3. 1896, S. 189–191.}}}\label{K_L02771-66h} leſen, als handle er nicht
               von \textsc{\textcolor{blue}{St. Georges}{}\ledrightnote{\textcolor{blue}{Stefan George}}}. Ja, er handelt aber davon. Ich kann Form und Inhalt nicht ſcheiden, beſonders
               nicht bei einer Kritik. Und wenn die Form gut iſt, das Urtheil aber falſch, ſo iſts
               eine ſchlechte Kritik. Auch iſt die Form nicht gut, – verfluchte Manier! {\pb}Hoffentlich nimmſt Du das \label{K_L02771-3v}\edtext{\textcolor{brown}{Burgtheater}{}\ledrightnote{\textcolor{brown}{Burgtheater}}-Referat in der »\textcolor{green}{Zeit}{}\ledrightnote{\textcolor{green}{Die Zeit. Wiener Wochenschrift}}«}{\lemma{\textnormal{\emph{Burgtheater-Referat … »Zeit«}}}\Cendnote{\textnormal{gemeint
                  ist, dass er alle Rezensionen der \emph{\textcolor{green}{Zeit}} über
                  dieses Theater verantworten würde; dazu kam es nicht}}}\label{K_L02771-3h} an. Du biſt der
               geborene Kritiker – wahrhaftig und unbeſtechlich, ich meine ſeeliſch unbeſtechlich,
               nicht einmal ein \label{K_L02771-4v}\edtext{\begin{otherlanguage}{french}\textsc{emballé}\end{otherlanguage}}{\lemma{\textnormal{\emph{emballé}}}\Cendnote{\textnormal{französisch: Mitgerissener}}}\label{K_L02771-4h}, wie
               ich. Und dann Du mit Deinem \strikeout{klug} klugen Urtheil und
               feinen Kunſtſinn! Nimms \strikeout{\textcolor{gray}{es}} an! \strikeout{Da} Daß Du nicht journaliſtiſch thätig ſein
               kannſt, {\pb}iſt eine Deiner Wahnideen, die am Beſten
               durch die Praxis widerlegt werden. Auch ſchafft Dir eine regelmäßige kritiſche
               Thätigkeit gewiſſe Lebensgrenzen, – \begin{otherlanguage}{french}Barrièren\end{otherlanguage},
               welche Deine Gedanken verhindern, im Unendlichen Unfug zu treiben. Wenn Du genöthigt
               biſt, \textsc{\textcolor{blue}{Rudolf Lothar}{}\ledrightnote{\textcolor{blue}{Rudolf Lothar}}} und \textsc{\textcolor{blue}{Davis}{}\ledrightnote{\textcolor{blue}{Gustav Davis}}} kritiſch zu behandeln, wirſt Du weniger an den Tod denken.\pend
           \pstart
           Wie wenn Du mir ein Wort hierher ſchriebeſt? (\textcolor{pink}{\textsc{Niddastraße 37}}{}\ledrightnote{\textcolor{pink}{Niddastraße}}.) Das wäre ſchön\textcolor{gray}{.}\pend
           \pstart
           {\pb}Iſt Dein \label{K_L02771-7v}\edtext{\textcolor{green}{Stück}{}\ledrightnote{→\textcolor{green}{Freiwild. Schauspiel in 3 Akten}} fertig}{\lemma{\textnormal{\emph{Stück fertig}}}\Cendnote{\textnormal{Es ging dem Ende zu. \textcolor{blue}{Schnitzler} begann eine neue Niederschrift von \emph{\textcolor{green}{Freiwild}} am 27. 4. 1896. Am 3. 5. 1896 las er es \textcolor{blue}{Felix Salten} vor, dessen positive Rückmeldung ihn
                  bestärkte. Am 5. 6. 1896 hatte \textcolor{blue}{Schnitzler}
                  das Stück »sozusagen beendet.«}}}\label{K_L02771-7h}? Kann man das \textcolor{green}{Manuſkript}{}\ledrightnote{→\textcolor{green}{Freiwild. Schauspiel in 3 Akten}} ſehen?\pend
           \pstart
           Bitte, ſchick’ mir nach \textsc{\textcolor{pink}{Paris}{}\ledrightnote{\textcolor{pink}{Paris}}} die im Buchhandel erſchienenen \textcolor{green}{\textsc{Anatol}-Sachen}{}\ledrightnote{\textcolor{green}{Anatol}}.\pend
           \pstart
           Grüß’ Dich Gott, mein lieber Freund!\pend
           \pstart
           Dein {\\[\baselineskip]}\spacefill\mbox{Paul Goldmann.}\pend
           \leftskip=0em{}\pstart
           \noindent{}Gruß an \textsc{\textcolor{blue}{Richard}{}\ledrightnote{\textcolor{blue}{Richard Beer-Hofmann}}}.\pend
           \pstart
           {\pb}Gefunden in einem alten deutſchen \textcolor{blue}{\textcolor{blue}{Myſtiker}{}\ledrightnote{→\textcolor{blue}{Angelus Silesius}}}{}\ledrightnote{→\textcolor{blue}{Angelus Silesius}}:\pend
           \stanza{} »\label{K_L02771-987v}\edtext{\textcolor{green}{Der Zufall muß
                        hinweg}{}\ledrightnote{→\textcolor{green}{Cherubinischer Wandersmann}}}{\lemma{\textnormal{\emph{Der Zufall muß
                        hinweg}}}\Cendnote{\textnormal{Epigramm 274 aus
                           \emph{\textcolor{green}{Geistreiche Sinn- und Schlussrime}}
                           (1657) von \textcolor{blue}{Angelus
                           Silesius}.}}}\label{K_L02771-987h}\newverse{}\textcolor{green}{und aller falſcher
                        Schein,}{}\ledrightnote{→\textcolor{green}{Cherubinischer Wandersmann}}\newverse{}\textcolor{green}{Du mußt ganz
                     weſentlich}{}\ledrightnote{→\textcolor{green}{Cherubinischer Wandersmann}}\newverse{}\textcolor{green}{und ungefärbet
                     ſein.}{}\ledrightnote{→\textcolor{green}{Cherubinischer Wandersmann}}«\stanzaend{}\pstart
           Und was ſagſt Du zu Frau \label{K_L02771-5v}\edtext{\textsc{\textcolor{blue}{Lou Andreas}{}\ledrightnote{\textcolor{blue}{Lou Andreas-Salomé}}}}{\lemma{\textnormal{\emph{Lou Andreas}}}\Cendnote{\textnormal{\emph{\textcolor{green}{Ruth}} hatte \textcolor{blue}{Schnitzler} bereits am 10. 1. 1896 gelesen. Zu \textcolor{blue}{Lou Andreas-Salomé} dürfte zu dieser Zeit kein näherer
                     Kontakt bestanden haben.}}}\label{K_L02771-5h}’ Buch »\textcolor{green}{Ruth}{}\ledrightnote{\textcolor{green}{Ruth. Erzählung}}«? Hörſt Du etwas von ihr?\pend
           \endnumbering\briefempfaengerindex{Schnitzler, Arthur@\textsc{Schnitzler, Arthur}!zzzGoldmann, Paul@\emph{von Paul Goldmann}!1896-04-091@{9. {[}4.{]} 1896}|)be}\mylabel{h}  \normalsize

\doendnotes{C}
\bigskip
\vfill

\clearpage

\footnotesize

\lohead{\textsc{register}}

% Definiere theindex-Environment komplett neu ohne reledmac
\makeatletter
\renewenvironment{theindex}{%
  \section*{\indexname}%
  \setlength{\parindent}{0pt}%
  \setlength{\parskip}{0pt plus 0.3pt}%
  \let\item\@idxitem
}{%
  \clearpage
}
\makeatother

\IfFileExists{\jobname-pw.ind}{\input{\jobname-pw.ind}}{}

\end{document}

      