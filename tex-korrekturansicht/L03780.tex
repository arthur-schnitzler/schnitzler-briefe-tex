%% latex-korrekturansicht-vorspann.tex
%% Vorspann für die Korrekturansicht.
%% Lädt die gemeinsame Datei latex-vorspann.tex mit gesetztem Schalter.

\newif\ifkorrekturansicht
\korrekturansichttrue

\input{../tex-inputs/latex-vorspann}


\section[Arthur Schnitzler an Stefan Zweig, 2[4?]. 5. 1913]{L03780 Arthur Schnitzler an Stefan Zweig,2[4?]. 5. 1913}
\nopagebreak\mylabel{L03780v}
\rehead{ }\normalsize\beginnumbering\briefempfaengerindex{Zweig, Stefan@\textsc{Zweig, Stefan}!zzzSchnitzler, Arthur@\emph{von Arthur Schnitzler}!1913-05-241@{2[4?]. 5. 1913}|(be}
\toendnotes[C]{\smallbreak\pagebreak[2]}\Standort{Jerusalem, National Library of Israel, ARC. Ms. Var. 305 1 58 Stefan Zweig Collection.}
\physDesc{Briefkarte, 1 Blatt, 2 Seiten, 1361 Zeichen
\newline{}Handschrift: schwarze Tinte, lateinische Kurrent}\toendnotes[C]{\smallbreak}
\pstart
           {\pb}\textcolor{gray}{\textbf{Dr. Arthur Schnitzler}}\hfill 2\textcolor{gray}{4}. 5. 913\pend
           
\pstart
           \textcolor{gray}{\textbf{\textcolor{pink}{Wien XVIII. Sternwartestrasse 71}\oindex{Sternwartestrasse 71@\textbf{Sternwartestraße 71}|pw}{}\ledrightnote{\textcolor{pink}{Sternwartestraße 71}}}}\pend
           \vspace{0.5em}
\pstart
           lieber Herr Doctor, Ihr ſchöner \label{K_L03780-1v}\edtext{Brief}{\lemma{\textnormal{\emph{Brief}}}\Cendnote{\textnormal{Stefan Zweig an Arthur Schnitzler, 23. 5. 1913. }}}\label{K_L03780-1} hat mir wahrhaft
               wohlgethan. So sicher ich bei dem Dichter des »\textcolor{green}{Kinderlands}SEXref\pwindex{Zweig, Stefan 28.\,11.\,1881 Wien – 23.\,2.\,1942 Petrópolis@\textsc{Zweig, Stefan} (28.\,11.\,1881 Wien – 23.\,2.\,1942 Petrópolis), \emph{Schriftsteller}!Erstes Erlebnis. Vier Geschichten aus Kinderland@\strich\emph{Erstes Erlebnis. Vier Geschichten aus Kinderland}|pw}{}\ledrightnote{\textcolor{green}{Erstes Erlebnis. Vier Geschichten aus Kinderland}}« auf vollko{\geminationm}enes Verständnis
               gefasst sein dürfte (Ihre Bedenken hinsichtlich der Schlusses theil ich sogar – seit
               einiger Zeit erst); die warme menschliche Antheilnahme die Sie an meinem Schaffen
               haben und deren ich i{\geminationm}er gewiss war, hat sich selten so
               lebhaft ausgedrückt als in Ihren letzten Worten, für die ich Ihnen freundschaftlichst
               die Hand drücke. –\pend
           
\pstart
           Ich danke auch für die Einladg zur \label{K_L03780-2v}\edtext{\textcolor{violet}{\textcolor{blue}{Bahr}\pwindex{Bahr, Hermann 19.\,7.\,1863 Linz – 15.\,1.\,1934 Muenchen@\textsc{Bahr, Hermann} (19.\,7.\,1863 Linz – 15.\,1.\,1934 München), \emph{Schriftsteller, Kritiker}|pw}{}\ledrightnote{\textcolor{blue}{Hermann Bahr}} Feier}\eventindex{Elektrotechnisches Institut der Technischen Universitaet@\textbf{Elektrotechnisches Institut der Technischen Universität}!Hermann-Bahr-Feier, 26.5.1913@Hermann-Bahr-Feier, 26.5.1913|pw}{}\ledrightnote{\textcolor{violet}{Hermann-Bahr-Feier, 26.5.1913}}}{\lemma{\textnormal{\emph{Bahr Feier}}}\Cendnote{\textnormal{Siehe Stefan Zweig an Arthur Schnitzler, 23. 5. 1913.}}}\label{K_L03780-2} u. bitte
               zugleich um Entschuldigg, daß ich nicht kommen werde. Sie wissen ja, daß ich mich
               (aus Gründen, die nicht ausschließlich \label{K_L03780-3v}\edtext{nervöser Natur}{\lemma{\textnormal{\emph{nervöser Natur}}}\Cendnote{\textnormal{\textcolor{blue}{Schnitzler} litt an Tinnitus.}}}\label{K_L03780-3} sind) von solchen {\pb}Veranstaltungen wie es nur irgend angeht fern halte (das
                  \label{K_L03780-4v}\edtext{\textcolor{violet}{\textcolor{blue}{Hauptmann}\pwindex{Hauptmann, Gerhart 15.\,11.\,1862 Szczawno-Zdrój – 6.\,6.\,1946 Jagniątków@\textsc{Hauptmann, Gerhart} (15.\,11.\,1862 Szczawno-Zdrój – 6.\,6.\,1946 Jagniątków), \emph{Schriftsteller}|pw}{}\ledrightnote{\textcolor{blue}{Gerhart Hauptmann}} Bankett}\eventindex{Oesterreichischer Ingenieur- und Architektenverein@\textbf{Österreichischer Ingenieur- und Architektenverein}!Hauptmann-Bankett der Concordia, 17.11.1912@Hauptmann-Bankett der Concordia, 17.11.1912|pw}{}\ledrightnote{\textcolor{violet}{Hauptmann-Bankett der Concordia, 17.11.1912}}}{\lemma{\textnormal{\emph{Hauptmann Bankett}}}\Cendnote{\textnormal{ Das \emph{\textcolor{violet}{Bankett zu Ehren von \textcolor{blue}{Gerhart Hauptmann}\pwindex{Hauptmann, Gerhart 15.\,11.\,1862 Szczawno-Zdrój – 6.\,6.\,1946 Jagniątków@\textsc{Hauptmann, Gerhart} (15.\,11.\,1862 Szczawno-Zdrój – 6.\,6.\,1946 Jagniątków), \emph{Schriftsteller}|pwk}}\eventindex{Oesterreichischer Ingenieur- und Architektenverein@\textbf{Österreichischer Ingenieur- und Architektenverein}!Hauptmann-Bankett der Concordia, 17.11.1912@Hauptmann-Bankett der Concordia, 17.11.1912|pwk}} wurde vom Journalisten- und Schriftstellerverein \emph{\textcolor{brown}{Concordia}\orgindex{Concordia. Journalisten- und Schriftstellerverein@Concordia. Journalisten- und Schriftstellerverein|pwk}} veranstaltet und fand am 17. 11. 1912 im \textcolor{pink}{Österreichischer
                     Ingenieur- und Architektenverein}\oindex{Oesterreichischer Ingenieur- und Architektenverein@\textbf{Österreichischer Ingenieur- und Architektenverein}|pwk} statt. }}}\label{K_L03780-4} war eine Ausnahme, weil
               ich, nach einem Misverständnis zwischen \textcolor{blue}{Hauptma{\geminationn}}\pwindex{Hauptmann, Gerhart 15.\,11.\,1862 Szczawno-Zdrój – 6.\,6.\,1946 Jagniątków@\textsc{Hauptmann, Gerhart} (15.\,11.\,1862 Szczawno-Zdrój – 6.\,6.\,1946 Jagniątków), \emph{Schriftsteller}|pw}{}\ledrightnote{\textcolor{blue}{Gerhart Hauptmann}} u. mir die Gelegenheit benutzen mußte ihm zu begegnen) – auch \textcolor{blue}{Bahr}\pwindex{Bahr, Hermann 19.\,7.\,1863 Linz – 15.\,1.\,1934 Muenchen@\textsc{Bahr, Hermann} (19.\,7.\,1863 Linz – 15.\,1.\,1934 München), \emph{Schriftsteller, Kritiker}|pw}{}\ledrightnote{\textcolor{blue}{Hermann Bahr}} (der übrigens glaub ich dasselbe thut) kennt diese meine
               Gepflogenheit und wird fern davon sein mir mein Ausbleiben übel zu nehmen. Sie aber,
               lieber Freund, bitt ich um das gleiche – und zugleich um Mittheilg wo Ihre \label{K_L03780-5v}\edtext{\textcolor{green}{Rede}SEXref\pwindex{Zweig, Stefan 28.\,11.\,1881 Wien – 23.\,2.\,1942 Petrópolis@\textsc{Zweig, Stefan} (28.\,11.\,1881 Wien – 23.\,2.\,1942 Petrópolis), \emph{Schriftsteller}!Hermann Bahr, der Fuenfzigjaehrige. (Eine Rede im Akademischen Verband fuer Literatur)@\strich\emph{Hermann Bahr, der Fünfzigjährige. (Eine Rede im Akademischen Verband für Literatur)}|pwv}{}\ledrightnote{{$\rightarrow$}\emph{\textcolor{green}{Hermann Bahr, der Fünfzigjährige. (Eine Rede im Akademischen Verband für Literatur)}}} ausführlich in Druck}{\lemma{\textnormal{\emph{Rede … Druck}}}\Cendnote{\textnormal{\textcolor{blue}{Stefan Zweig}\pwindex{Zweig, Stefan 28.\,11.\,1881 Wien – 23.\,2.\,1942 Petrópolis@\textsc{Zweig, Stefan} (28.\,11.\,1881 Wien – 23.\,2.\,1942 Petrópolis), \emph{Schriftsteller}|pwk}: \emph{\textcolor{green}{Hermann Bahr, der Fünfzigjährige. (Eine Rede im Akademischen
                        Verband für Literatur)}SEXref\pwindex{Zweig, Stefan 28.\,11.\,1881 Wien – 23.\,2.\,1942 Petrópolis@\textsc{Zweig, Stefan} (28.\,11.\,1881 Wien – 23.\,2.\,1942 Petrópolis), \emph{Schriftsteller}!Hermann Bahr, der Fuenfzigjaehrige. (Eine Rede im Akademischen Verband fuer Literatur)@\strich\emph{Hermann Bahr, der Fünfzigjährige. (Eine Rede im Akademischen Verband für Literatur)}|pwk}}. In: \emph{\textcolor{green}{Neue Freie
                        Presse}\pwindex{Neue Freie Presse@\emph{Neue Freie Presse}|pwk}}, Nr. 17.513, 13. 5. 1913, Morgenblatt,
                     S. 1–3.}}}\label{K_L03780-5} erscheinen wird. Wie sind Ihre So{\geminationm}erpläne? Wir wollen Anfang Juni einige Wochen fort
               sein, und da{\geminationn} bis gegen Ende Juli in \textcolor{pink}{Wien}\oindex{Wien@\textbf{Wien}|pw}{}\ledrightnote{\textcolor{pink}{Wien}} verbringen.\pend
           
\pstart
           Ein baldgs Wiedersehen erhoffend und mit herzlichen Grüßen{\\[\baselineskip]}Ihr aufrichtig ergebner{\\[\baselineskip]}\spacefill\mbox{Arthur Schnitzler}\pend
           \leftskip=0em{}\selectlanguage{ngerman}\endnumbering\briefempfaengerindex{Zweig, Stefan@\textsc{Zweig, Stefan}!zzzSchnitzler, Arthur@\emph{von Arthur Schnitzler}!1913-05-241@{2[4?]. 5. 1913}|)be}\mylabel{L03780h}  \normalsize

\doendnotes{C}
\bigskip
\vfill

\clearpage

\footnotesize

\lohead{\textsc{register}}

% Definiere theindex-Environment komplett neu ohne reledmac
\makeatletter
\renewenvironment{theindex}{%
  \section*{\indexname}%
  \setlength{\parindent}{0pt}%
  \setlength{\parskip}{0pt plus 0.3pt}%
  \let\item\@idxitem
}{%
  \clearpage
}
\makeatother

\IfFileExists{\jobname-pw.ind}{\input{\jobname-pw.ind}}{}

\end{document}

      