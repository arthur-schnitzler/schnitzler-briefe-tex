%% latex-korrekturansicht-vorspann.tex
%% Vorspann für die Korrekturansicht.
%% Lädt die gemeinsame Datei latex-vorspann.tex mit gesetztem Schalter.

\newif\ifkorrekturansicht
\korrekturansichttrue

\input{../tex-inputs/latex-vorspann}


               \section[Max Mell an Arthur Schnitzler, 11. 10. 1906]{ Max Mell an Arthur Schnitzler, 11. 10. 1906}\nopagebreak\mylabel{v}\rehead{ }\normalsize\beginnumbering\briefempfaengerindex{Schnitzler, Arthur@\textsc{Schnitzler, Arthur}!zzzMell, Max@\emph{von Max Mell}!1906-10-111@{11. 10. 1906}|(be} \toendnotes[C]{\smallbreak\pagebreak[2]} \Standort{CUL, Schnitzler, B 70.}
\physDesc{Brief, 1 Blatt, 1 Seite
\newline{}Handschrift: schwarze Tinte, deutsche Kurrent
\newline{}Schnitzler: 1) mit rotem Buntstift beschriftet: »\textsc{Mell}« 2) mit rotem Buntstift eine Unterstreichung}\toendnotes[C]{\smallbreak}\pstart
           \noindent{}\raggedleft{}{\pb}\textcolor{pink}{Wien II.}{}\ledrightnote{\textcolor{pink}{II., Leopoldstadt}}{ }\textcolor{pink}{Wittelsbacherſtr}{}\ledrightnote{\textcolor{pink}{Wittelsbachstraße}}. 5.\pend
           \pstart
           \raggedleft{}11. Oktober 1906.\pend
           \pstart{}Sehr verehrter Herr Doktor,\pend\pstart
           ich nehme mir die Freiheit, Ihnen mein \textcolor{green}{Stück}{}\ledrightnote{→\textcolor{green}{Die Komödianten}} zu überreichen, ermutigt durch Sie ſelbſt und in Ungeduld, denen
                    auch als Dramatiker bekannt zu werden, die ſich meiner Novellen erinnern. Mein
                    Ziel ist die Komödie; und hoffentlich werden Sie mir die Fähigkeit, es zu
                    erreichen, zuſprechen.\pend
           \pstart
           Darf ich auch einen kleinen \textcolor{green}{Aufſatz}{}\ledrightnote{→\textcolor{green}{Über die Briefe Beethovens}}
                    aus der \textcolor{brown}{Frankfurter Zeitung}{}\ledrightnote{\textcolor{brown}{Frankfurter Zeitung}} beilegen?\pend
           \pstart
           Vielleicht geben Sie das Manuſkript gelegentlich meiner \textcolor{blue}{Schweſter}{}\ledrightnote{→\textcolor{blue}{Maria Mell}} zurück, wenn sie Ihre Frau \textcolor{blue}{Gemahlin}{}\ledrightnote{→\textcolor{blue}{Olga Schnitzler}} beſucht, auch
                    werde ich mir erlauben, Ihnen meine \textcolor{pink}{Berlin}{}\ledrightnote{\textcolor{pink}{Berlin}}er
                    Adreſſe mitzuteilen. Ich hab das \textcolor{green}{Stück}{}\ledrightnote{→\textcolor{green}{Die Komödianten}} in \textcolor{pink}{Berlin}{}\ledrightnote{\textcolor{pink}{Berlin}} noch nirgends
                    eingereicht, aber es an \textcolor{blue}{Kainz}{}\ledrightnote{\textcolor{blue}{Josef Kainz}} geſchickt.\pend
           \pstart
           Es wäre mir ſehr erfreulich, wenn auch Ihre Frau \textcolor{blue}{Gemahlin}{}\ledrightnote{→\textcolor{blue}{Olga Schnitzler}}, der ich mich beſtens zu empfehlen bitte, es
                    leſen wollte.\pend
           \pstart
           Ich bin, in aufrichtiger Hochachtung{\\[\baselineskip]}Ihr ſehr ergebener{\\[\baselineskip]}\spacefill\mbox{Max Mell.}\pend
           \leftskip=0em{}\endnumbering\briefempfaengerindex{Schnitzler, Arthur@\textsc{Schnitzler, Arthur}!zzzMell, Max@\emph{von Max Mell}!1906-10-111@{11. 10. 1906}|)be}\mylabel{h}  \normalsize

\doendnotes{C}
\bigskip
\vfill

\clearpage

\footnotesize

\lohead{\textsc{register}}

% Definiere theindex-Environment komplett neu ohne reledmac
\makeatletter
\renewenvironment{theindex}{%
  \section*{\indexname}%
  \setlength{\parindent}{0pt}%
  \setlength{\parskip}{0pt plus 0.3pt}%
  \let\item\@idxitem
}{%
  \clearpage
}
\makeatother

\IfFileExists{\jobname-pw.ind}{\input{\jobname-pw.ind}}{}

\end{document}

      