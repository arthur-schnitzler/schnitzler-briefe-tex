%% latex-korrekturansicht-vorspann.tex
%% Vorspann für die Korrekturansicht.
%% Lädt die gemeinsame Datei latex-vorspann.tex mit gesetztem Schalter.

\newif\ifkorrekturansicht
\korrekturansichttrue

\input{../tex-inputs/latex-vorspann}


               \section[Arthur Schnitzler an Richard Beer-Hofmann, 21. 9. 1895]{ Arthur Schnitzler an Richard Beer-Hofmann, 21. 9. 1895}\nopagebreak\mylabel{v}\rehead{ }\normalsize\beginnumbering\briefempfaengerindex{Beer-Hofmann, Richard@\textsc{Beer-Hofmann, Richard}!zzzSchnitzler, Arthur@\emph{von Arthur Schnitzler}!1895-09-211@{21. 9. 1895}|(be} \toendnotes[C]{\smallbreak\pagebreak[2]} \Standort{YCGL, MSS 31.}
\physDesc{Brief, 2 Blätter, 8 Seiten, Umschlag
\newline{}Handschrift: 1) Bleistift, deutsche Kurrent\hspace{1em}2) schwarze Tinte, deutsche Kurrent (\noindent{}Umschlag)\hspace{1em}\newline{}Versand: 1) mit Tinte von unbekannter Hand nachgesandt nach »\textsc{\textcolor{pink}{Gardone}
                                          p{[}ost{]}. r{[}estante{]}}.« 2) Stempel: »\nobreak{}Wien, 21. 9. \textcolor{gray}{9}5\nobreak{}«. 3) Stempel: »\nobreak{}\oindex{Riva del Garda@\textbf{Riva del Garda}, \emph{Besiedelter Ort (A.BSO)}|pwk}{\pb}Riva, 22. 9. 95\nobreak{}«. 4) Stempel: »\nobreak{}\oindex{Gardone Riviera@\textbf{Gardone Riviera}, \emph{Besiedelter Ort (A.BSO)}|pwk}Gardone Riviera, 24 9 95\nobreak{}«. }\buchAbdrucke{\weitereDrucke{Arthur Schnitzler, Richard Beer-Hofmann: \emph{Briefwechsel 1891–1931}. Hg. Konstanze Fliedl. Wien, Zürich: \emph{Europaverlag} 1992, S. 83.} }\toendnotes[C]{\smallbreak}\pstart{}{\pb}Herrn \textsc{Dr. Richard
                     Beer-Hofmann}\pend{}\pstart{}\textcolor{pink}{\textsc{Riva am G}\damage{\textcolor{gray}{ardasee}}}{}\ledrightnote{\textcolor{pink}{Riva del Garda}}\pend{}\pstart{}\textsc{post restante}\pend{}{\bigskip}\pstart
           \raggedleft{}{\pb}21. 9. 95\pend
           \pstart
           Lieber Richard, meine Karte haben Sie wohl. In \textcolor{pink}{\textsc{Riva}}{}\ledrightnote{\textcolor{pink}{Riva del Garda}} iſt es \uline{mir} nemlich vor 3 Jahren paſſirt, daſs
               der Poſtbeamte mir die Briefe an mich nicht gab – ich verlangte damals die Einläufe
               durchzuſehen, da entdeckte ich meine Briefe. Und ich hatte nicht gepfiffen! –\pend
           \pstart
           {\pb}Die \label{K_L00489_1v}\edtext{\textcolor{green}{Leſeprobe}{}\ledrightnote{→\textcolor{green}{Liebelei. Schauspiel in drei Akten}}}{\lemma{\textnormal{\emph{Leſeprobe}}}\Cendnote{\textnormal{vgl. A. S.: \emph{Tagebuch}, 18. 9. 1895}}}\label{K_L00489_1h} fiel gut aus. Frl. \textcolor{blue}{S.}{}\ledrightnote{\textcolor{blue}{Adele Sandrock}} ignorirte mich,
               aber that ſehr ergriffen von dem \textcolor{green}{Stück}{}\ledrightnote{→\textcolor{green}{Liebelei. Schauspiel in drei Akten}}, Nachmittag telephonirte ſie \label{K_L00489_2v}\edtext{\textsc{en bon camerade}}{\lemma{\textnormal{\emph{en bon camerade}}}\Cendnote{\textnormal{französisch: kameradschaftlich.}}}\label{K_L00489_2h}.
                  \textcolor{blue}{So{\geminationn}enthal}{}\ledrightnote{\textcolor{blue}{Adolf von Sonnenthal}} hat
               »gute Hoffnung«. Beim \textcolor{green}{1. Akt}{}\ledrightnote{→\textcolor{green}{Liebelei. Schauspiel in drei Akten}}
               wurde viel gelacht. Vom \textcolor{green}{3.}{}\ledrightnote{→\textcolor{green}{Liebelei. Schauspiel in drei Akten}}
               verſpricht man ſich ſichre Wirkung. Dem \textcolor{green}{2.}{}\ledrightnote{→\textcolor{green}{Liebelei. Schauspiel in drei Akten}}{ }ſcheint man am wenigſtens zu vertrauen.
                  {\pb}\textcolor{blue}{\textsc{Mitterwurzer}}{}\ledrightnote{\textcolor{blue}{Friedrich Mitterwurzer}} war nicht anweſend; er ſpielt aber ſicher, ließ ſich officiell entſchuldigen.
               Die \textcolor{blue}{\textsc{Kallina}}{}\ledrightnote{\textcolor{blue}{Anna Kallina}} wird überraſchen. Dazu will \textcolor{blue}{\textsc{Burckhard}}{}\ledrightnote{\textcolor{blue}{Max Eugen Burckhard}} einen Einakter von \textcolor{blue}{\textsc{Giacosa}}{}\ledrightnote{\textcolor{blue}{Giuseppe Giacosa}}{ }\textcolor{green}{Rechte der Seele}{}\ledrightnote{\textcolor{green}{Rechte der Seele}} geben; während der Leſeprobe half
               er den \label{K_L00489_3v}\edtext{\textcolor{blue}{\textsc{Laube}}{}\ledrightnote{\textcolor{blue}{Heinrich Laube}} in \textcolor{pink}{Sprottau}{}\ledrightnote{\textcolor{pink}{Sprottau}}}{\lemma{\textnormal{\emph{Laube in Sprottau}}}\Cendnote{\textnormal{Die Enthüllung des Denkmals für \textcolor{blue}{Heinrich Laube} in dessen Geburtstadt fand
                  ebenfalls am 18. 9. 1895 statt.}}}\label{K_L00489_3h} ent{\pb}hüllen. Ich wünſchte ihm angenehme Enthüllung. Er
               ſagte, die Enthüllung des Fräulein \label{K_L00489_4v}\edtext{\textcolor{blue}{\textsc{Dandler}}{}\ledrightnote{\textcolor{blue}{Anna Dandler}}}{\lemma{\textnormal{\emph{Dandler}}}\Cendnote{\textnormal{Diese war zeitlebens für das \emph{\textcolor{brown}{Münchner Hoftheater}} tätig. Ob hier eine sexuelle
                  Zote (anzunehmen) oder der Wunsch ausgedrückt wird, sie ans \emph{\textcolor{brown}{Burgtheater}} zu holen (weniger wahrscheinlich), kann nicht
                  geklärt werden.}}}\label{K_L00489_4h} zöge er vor. –\pend
           \pstart
           \textcolor{blue}{\textsc{Fels}}{}\ledrightnote{\textcolor{blue}{Friedrich Michael Fels}}{ }ſchreibt mir \label{K_L00489_5v}\edtext{heute}{\lemma{\textnormal{\emph{heute}}}\Cendnote{\textnormal{Friedrich M. Fels an Arthur Schnitzler, 19. 9. 1895.}}}\label{K_L00489_5h}. Sie können ſich denken. Er appellirt an uns zuſa{\geminationm}en, die Summe iſt 25 fl. Ich hab ihm gleich 10 fl {\pb}geſchickt. Darf ich ihm auch für Sie was ſchicken? Auch an \textcolor{blue}{Hugo}{}\ledrightnote{\textcolor{blue}{Hugo von Hofmannsthal}} wandt ich mich bereits. –\pend
           \pstart
           Geſtern war ich beim »\label{K_L00489_6v}\edtext{\textcolor{green}{Pelikan}{}\ledrightnote{\textcolor{green}{Ein Pelikan. Schauspiel in fünf Aufzügen}}}{\lemma{\textnormal{\emph{Pelikan}}}\Cendnote{\textnormal{im \textcolor{pink}{Burgtheater}}}}\label{K_L00489_6h}«. Dieſes Blaßwerden guter Stücke iſt ſeltſam. –
                  \label{K_L00489_7v}\edtext{Heute}{\lemma{\textnormal{\emph{Heute}}}\Cendnote{\textnormal{Gegeben wurde zum ersten Mal \emph{\textcolor{green}{Die
                     Doppelhochzeit}} von \textcolor{blue}{Victor Léon} und \textcolor{blue}{Heinrich von Waldberg}, Musik von \textcolor{blue}{Josef Hellmesberger}.}}}\label{K_L00489_7h} geh ich zur
               Eröffnung der \textcolor{pink}{\textsc{Josefstadt}}{}\ledrightnote{\textcolor{pink}{Theater in der Josefstadt}}. – Gearbeitet hab ich noch i{\geminationm}er gar nichts; heute
                  {\pb}Nacht will ich anfangen. Glauben Sie? –\pend
           \pstart
           Das Datum der \textcolor{green}{L.}{}\ledrightnote{\textcolor{green}{Liebelei. Schauspiel in drei Akten}} iſt noch nicht feſtgeſtellt. –\pend
           \pstart
           Den \textcolor{blue}{Hugo}{}\ledrightnote{\textcolor{blue}{Hugo von Hofmannsthal}} hab ich geſtern begegnet, vorgeſtern iſt
               er zurückgeko{\geminationm}en. Er ſieht gut aus, »wettergebräunt«.
               Nach und nach wird man zu allen Worten Anführungszeichen {\pb}machen müſſen – das wird dann das Ende der Literatur
               sein.\pend
           \pstart
           Wie geht’s Ihnen? Nächſtens ſchreiben Sie mir einen Brief ſtatt einer Depeſche; da
               werde ich weniger erſchrecken und mich beſſer unterhalten. Ich wünſche Ihnen weiter
               gute Laune, {\pb}gutes Wetter, gute Sti{\geminationm}ung und lebhafte Empfindung Ihrer Freiheit und Ihres
               Lebens.\pend
           \pstart
           Herzliche Grüße Ihr{\\[\baselineskip]}\spacefill\mbox{Arthur}\pend
           \leftskip=0em{}\endnumbering\briefempfaengerindex{Beer-Hofmann, Richard@\textsc{Beer-Hofmann, Richard}!zzzSchnitzler, Arthur@\emph{von Arthur Schnitzler}!1895-09-211@{21. 9. 1895}|)be}\mylabel{h}  \normalsize

\doendnotes{C}
\bigskip
\vfill

\clearpage

\footnotesize

\lohead{\textsc{register}}

% Definiere theindex-Environment komplett neu ohne reledmac
\makeatletter
\renewenvironment{theindex}{%
  \section*{\indexname}%
  \setlength{\parindent}{0pt}%
  \setlength{\parskip}{0pt plus 0.3pt}%
  \let\item\@idxitem
}{%
  \clearpage
}
\makeatother

\IfFileExists{\jobname-pw.ind}{\input{\jobname-pw.ind}}{}

\end{document}

      