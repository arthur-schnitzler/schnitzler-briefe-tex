%% latex-korrekturansicht-vorspann.tex
%% Vorspann für die Korrekturansicht.
%% Lädt die gemeinsame Datei latex-vorspann.tex mit gesetztem Schalter.

\newif\ifkorrekturansicht
\korrekturansichttrue

\input{../tex-inputs/latex-vorspann}


               \section[Paul Goldmann an Arthur Schnitzler, 9. 8. {[}1895{]}]{ Paul Goldmann an Arthur Schnitzler, 9. 8. {[}1895{]}}\nopagebreak\mylabel{v}\rehead{ }\normalsize\beginnumbering\briefempfaengerindex{Schnitzler, Arthur@\textsc{Schnitzler, Arthur}!zzzGoldmann, Paul@\emph{von Paul Goldmann}!1895-08-093@{9. 8. {[}1895{]}}|(be} \toendnotes[C]{\smallbreak\pagebreak[2]} \Standort{DLA, A:Schnitzler, HS.NZ85.1.3165.}
\physDesc{Brief, 2 Blätter, 8 Seiten
\newline{}Handschrift: schwarze Tinte, deutsche Kurrent
\newline{}Schnitzler: 1) mit Bleistift das Jahr »95« vermerkt 2) mit rotem Buntstift drei Unterstreichungen}\toendnotes[C]{\smallbreak}\pstart
           \raggedleft{}{\pb}\textsc{\textcolor{pink}{Toelz}{}\ledrightnote{\textcolor{pink}{Bad Tölz}}}, 9. Auguſt.\pend
           \pstart\center{}Mein lieber Freund,\pend\pstart
           Ich bin erſt Donnerſtag von \textsc{\textcolor{pink}{Paris}{}\ledrightnote{\textcolor{pink}{Paris}}} abgefahren u. bin ſpäter nach \textsc{\textcolor{pink}{Muenchen}{}\ledrightnote{\textcolor{pink}{München}}} gekommen, als ich gedacht. Denn ich habe mich in \textcolor{pink}{Straßburg}{}\ledrightnote{\textcolor{pink}{Straßburg}} u. im \textcolor{pink}{Schwarzwald}{}\ledrightnote{\textcolor{pink}{Schwarzwald}}
               aufgehalten zuſammen mit \textsc{\textcolor{blue}{Henri Albert}{}\ledrightnote{\textcolor{blue}{Henri Albert}}} u. \textsc{\textcolor{blue}{Charles Simon}{}\ledrightnote{\textcolor{blue}{Charles Simon}}}, einem neuen Bekannten, einem Menſchen von {\pb}Werth u. Eigenart, von dem ich Dir mündlich erzählen werde.\pend
           \pstart
           In \textsc{\textcolor{pink}{Muenchen}{}\ledrightnote{\textcolor{pink}{München}}} fand ich Deine lieben Briefe vor, die mich innig erfreut haben. Ich wollte ſie
               gleich beantworten, kam aber nicht dazu. Denn meine Zeit wurde ausgefüllt von \textsc{\textcolor{blue}{Albert Langen}{}\ledrightnote{\textcolor{blue}{Albert Langen}}}, dem Verleger u. Lausbuben, mit dem ich ein ſchweres Ärgerniß hatte, und von
               einem \label{K_L02743-1v}\edtext{\textcolor{blue}{Kindheits-{\pb}Freunde}{}\ledrightnote{→\textcolor{blue}{?? [Kindheitsfreund von Paul Goldmann]}}}{\lemma{\textnormal{\emph{Kindheits-Freunde}}}\Cendnote{\textnormal{nicht identifiziert}}}\label{K_L02743-1h}, den ich
               zufällig dort traf. Seit geſtern bin ich in \textsc{\textcolor{pink}{Toelz}{}\ledrightnote{\textcolor{pink}{Bad Tölz}}} u. die erſte freie Minute benütze ich, um Dir zu ſchreiben.\pend
           \pstart
           Vielen, vielen Dank für Deine lieben Briefe. Es war ſoviel Tröſtliches u.
               Ermuthigendes darin! Das hat mich tief bewegt! {\dotsfive}\pend
           \pstart
           Mir iſt es leid, daß ich auf unſere Zuſammenkunft noch ſo lange warten {\pb}ſoll. Aber es geht ja leider nicht anders wegen
               dieſer verdammten Kur (die auch nicht nützen wird, wie alle früheren). Hier muß ich
               mindeſtens 3 Wochen bleiben, kann alſo vor 30.{ }\strikeout{Se}{ }Auguſt nicht fort. So muß ich Dich denn bitten: entweder
               tritt Deine \textsc{Bicycle}-Tour fünf Tage ſpäter an {\pb}oder komme auf ein paar Tage hierher. Letzteres wäre
               freilich eine \strikeout{Z} Zumuthung für Dich. Denn \label{K_L02743-2v}\edtext{\textsc{\textcolor{pink}{Toelz}{}\ledrightnote{\textcolor{pink}{Bad Tölz}}}}{\lemma{\textnormal{\emph{Toelz}}}\Cendnote{\textnormal{Auch \textcolor{blue}{Schnitzler} war von \textcolor{pink}{Bad Tölz} nicht
                  angetan. Am 26. 8. 1895
                  notierte er im \emph{\textcolor{green}{Tagebuch}}: »Schlechter
                     Eindruck von \textcolor{pink}{Tölz},
                  verstimmend.«}}}\label{K_L02743-2h} iſt das ſtumpfſinnigſte \textcolor{pink}{Neſt}{}\ledrightnote{→\textcolor{pink}{Bad Tölz}}, das ich kenne, u. \strikeout{bot} bietet gar nichts. Auch landſchaftlich iſt es recht
               mäßig. Jedenfalls werde ich nicht mit Dir nach dem Norden reiſen können. Zwiſchen
                  10. u. 15. September{ }{\pb}muß ich wieder in \textsc{\textcolor{pink}{Paris}{}\ledrightnote{\textcolor{pink}{Paris}}} ſein. Auch habe ich kein Geld. Die Kur koſtet Unſummen.\pend
           \pstart
           Was den Brief der \textsc{\textcolor{blue}{Candiani}{}\ledrightnote{\textcolor{blue}{Regina Candiani}}} betrifft, ſo kann ich Dir von hier aus nicht rathen. Ich hielt ſchon ſeinerzeit
               Umfrage, fand aber Niemanden, der die \textcolor{blue}{Dame}{}\ledrightnote{\textcolor{blue}{Regina Candiani}}
               kannte. Das Geſcheiteſte wäre, Du ſchriebeſt ihr: Herr \textsc{Goldmann}, der Mitte September{ }{\pb}nach \textsc{\textcolor{pink}{Paris}{}\ledrightnote{\textcolor{pink}{Paris}}} kommt, wird ſich mit Ihnen in Verbindung ſetzen. Ich würde dann hingehen u.
               verſuchen, mir \label{K_L02743-3v}\edtext{\begin{otherlanguage}{french}\textsc{de vive}\end{otherlanguage}}{\lemma{\textnormal{\emph{de vive}}}\Cendnote{\textnormal{französisch: aus dem Leben}}}\label{K_L02743-3h} ein
               Urtheil zu bilden. Die »\textsc{\textcolor{green}{Revue des jeunes filles}{}\ledrightnote{→\textcolor{green}{Revue pour les jeunes filles}}}«, von der ſie ſchreibt, iſt ein literariſch werthloſes, wenn ich nicht irre neu
               begründetes \textcolor{green}{Blatt}{}\ledrightnote{→\textcolor{green}{Revue pour les jeunes filles}} für höhere
               Töchter. Anbei der \label{K_L02743-4v}\edtext{Brief}{\lemma{\textnormal{\emph{Brief}}}\Cendnote{\textnormal{Beilage nicht erhalten}}}\label{K_L02743-4h}. {\pb}Daß Du den \label{K_L02743-5v}\edtext{erſten Akt von »\textcolor{green}{Freiwild}{}\ledrightnote{\textcolor{green}{Freiwild. Schauspiel in 3 Akten}}« beendet}{\lemma{\textnormal{\emph{erſten … beendet}}}\Cendnote{\textnormal{Siehe A. S.: \emph{Tagebuch}, 2. 8. 1895}}}\label{K_L02743-5h} haſt, iſt hoch erfreulich. Hoffentlich bringſt Du was zum Vorleſen mit.\pend
           \pstart
           Die Tinte, mit der ich ſchreibe, gibt Dir einen Begriff von \textsc{\textcolor{pink}{Toelz}{}\ledrightnote{\textcolor{pink}{Bad Tölz}}er} Comfort. Es ist die beſte im \textcolor{pink}{Ort}{}\ledrightnote{→\textcolor{pink}{Bad Tölz}}.\pend
           \pstart
           Schreib’ mir, bitte, eine Zeile: \textsc{\textcolor{pink}{Toelz}{}\ledrightnote{\textcolor{pink}{Bad Tölz}}}, \textsc{\textcolor{pink}{Baiern}{}\ledrightnote{\textcolor{pink}{Bayern}}}, \textsc{Poste restante}.\pend
           \pstart \label{T_L02743-1v}\edtext{Viele treue Grüße! Dein \spacefill\mbox{Paul
                  Goldmann}}{\lemma{\textnormal{\emph{Viele … Goldmann}}}\Cendnote{\textnormal{von oben nach unten entlang des linken Randes, normal zum Text}}}\label{T_L02743-1h}\pend{}\pstart
           \noindent{}\label{T_L02743-2v}\edtext{Die herzlichſten Grüße an \textsc{\textcolor{blue}{Richard}{}\ledrightnote{\textcolor{blue}{Richard Beer-Hofmann}}}!}{\lemma{\textnormal{\emph{Die … Richard!}}}\Cendnote{\textnormal{entlang des Mittelfalzes von unten nach oben, normal zum Text}}}\label{T_L02743-2h}\pend
           \endnumbering\briefempfaengerindex{Schnitzler, Arthur@\textsc{Schnitzler, Arthur}!zzzGoldmann, Paul@\emph{von Paul Goldmann}!1895-08-093@{9. 8. {[}1895{]}}|)be}\mylabel{h}  \normalsize

\doendnotes{C}
\bigskip
\vfill

\clearpage

\footnotesize

\lohead{\textsc{register}}

% Definiere theindex-Environment komplett neu ohne reledmac
\makeatletter
\renewenvironment{theindex}{%
  \section*{\indexname}%
  \setlength{\parindent}{0pt}%
  \setlength{\parskip}{0pt plus 0.3pt}%
  \let\item\@idxitem
}{%
  \clearpage
}
\makeatother

\IfFileExists{\jobname-pw.ind}{\input{\jobname-pw.ind}}{}

\end{document}

      