%% latex-korrekturansicht-vorspann.tex
%% Vorspann für die Korrekturansicht.
%% Lädt die gemeinsame Datei latex-vorspann.tex mit gesetztem Schalter.

\newif\ifkorrekturansicht
\korrekturansichttrue

\input{../tex-inputs/latex-vorspann}


\renewcommand{\erwaehntePersonen}{Personen: Johann Schnitzler}
\renewcommand{\erwaehnteOrte}{Orte: Wien}
\renewcommand{\erwaehnteWerke}{}
\section[Felix Salten an Arthur Schnitzler, {[}2. 5. 1893{]}]{Felix Salten an Arthur Schnitzler, {[}2. 5. 1893{]}}
\nopagebreak\mylabel{v}
\rehead{ }\normalsize\beginnumbering\briefempfaengerindex{Schnitzler, Arthur@\textsc{Schnitzler, Arthur}!zzzSalten, Felix@\emph{von Felix Salten}!1893-05-023@{{[}2. 5. 1893{]}}|(be}
\toendnotes[C]{\smallbreak\pagebreak[2]}\Standort{CUL, Schnitzler, B 89, A 1.}
\physDesc{Briefkarte, 417 Zeichen
\newline{}Handschrift: Bleistift, lateinische Kurrent
\newline{}Schnitzler: mit Bleistift datiert: »2/5 93« 
\newline{}Ordnung: mit Bleistift von unbekannter Hand nummeriert: »25« }\toendnotes[C]{\smallbreak}
\pstart
           \noindent{}{\pb}Theuerster Freund! Ich bin so furchtbar \label{K_L03122-1v}\edtext{erschüttert}{\lemma{\textnormal{\emph{erschüttert}}}\Cendnote{\textnormal{Am 2. 5. 1893 war \textcolor{blue}{Schnitzler}s
                  Vater \textcolor{blue}{Johann Schnitzler} verstorben.}}}\label{K_L03122-1h},
               dass ich nicht weiss, was ich Ihnen sagen, was ich denken soll, Ich habe nur einen
               Wunsch, u. das ist, Ihnen tragen helfen, was ja doch {\pb}zu schwer sein muss für Sie, zu
               schwer. Bitte, Sie wissen ja, wie sehr ich Sie liebe, laßen Sie mich, wenn es Ihnen
               Erleichterung ist an Ihrer Seite sein so oft Sie es immer wollen – \pend
           
\pstart
           Ich weine, es ist doch zu traurig alles\pend
           \pstart \uline{Ihr}{ }\spacefill\mbox{Salten}\pend{}\endnumbering\briefempfaengerindex{Schnitzler, Arthur@\textsc{Schnitzler, Arthur}!zzzSalten, Felix@\emph{von Felix Salten}!1893-05-023@{{[}2. 5. 1893{]}}|)be}\mylabel{h}  \normalsize

\doendnotes{C}
\bigskip
\vfill

\clearpage

\footnotesize

\lohead{\textsc{register}}

% Definiere theindex-Environment komplett neu ohne reledmac
\makeatletter
\renewenvironment{theindex}{%
  \section*{\indexname}%
  \setlength{\parindent}{0pt}%
  \setlength{\parskip}{0pt plus 0.3pt}%
  \let\item\@idxitem
}{%
  \clearpage
}
\makeatother

\IfFileExists{\jobname-pw.ind}{\input{\jobname-pw.ind}}{}

\end{document}

      