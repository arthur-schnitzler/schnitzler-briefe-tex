%% latex-korrekturansicht-vorspann.tex
%% Vorspann für die Korrekturansicht.
%% Lädt die gemeinsame Datei latex-vorspann.tex mit gesetztem Schalter.

\newif\ifkorrekturansicht
\korrekturansichttrue

\input{../tex-inputs/latex-vorspann}


\renewcommand{\erwaehntePersonen}{Personen: Georg Hirschfeld}
\renewcommand{\erwaehnteOrte}{Orte: Berlin, Dresden, Wien}
\renewcommand{\erwaehnteWerke}{}
\section[ Felix Salten an Arthur Schnitzler, 4. 5. 1899]{Felix Salten an Arthur Schnitzler, 4. 5. 1899}
\nopagebreak\mylabel{v}
\rehead{ }\normalsize\beginnumbering\briefempfaengerindex{Schnitzler, Arthur@\textsc{Schnitzler, Arthur}!zzzSalten, Felix@\emph{von Felix Salten}!1899-05-041@{4. 5. 1899}|(be}
\toendnotes[C]{\smallbreak\pagebreak[2]}\Standort{CUL, Schnitzler, B 89, A 2.}
\physDesc{Brief, 1 Blatt, 1 Seite, 370 Zeichen
\newline{}Handschrift: schwarze Tinte, lateinische Kurrent
\newline{}Ordnung: mit Bleistift von unbekannter Hand nummeriert: »114« }\toendnotes[C]{\smallbreak}
\pstart
           \raggedleft{}{\pb}\textcolor{pink}{Wien}{}\ledrightnote{\textcolor{pink}{Wien}}, 4. Mai 99\pend
           
\pstart
           Lieber Freund, von \textcolor{blue}{Hirschfeld}{}\ledrightnote{\textcolor{blue}{Georg Hirschfeld}} höre ich eben, dass Sie \textcolor{pink}{hier}{}\ledrightnote{{$\rightarrow$}\textcolor{pink}{Wien}} sind. Ich schrieb Ihnen nach \textcolor{pink}{Berlin}{}\ledrightnote{\textcolor{pink}{Berlin}}, – haben Sie meinen \label{K_L03290-1v}\edtext{Brief}{\lemma{\textnormal{\emph{Brief}}}\Cendnote{\textnormal{Felix Salten an Arthur Schnitzler, 28. 4. 1899}}}\label{K_L03290-1h} bekommen? Heute{ }Abend verreise ich auf ein paar Tage, nach
                  \textcolor{pink}{Dresden}{}\ledrightnote{\textcolor{pink}{Dresden}}. Ich sage Ihnen bald noch näheres
               darüber. Wenn ich wiederkomme, such ich Sie gleich auf. Inzwischen grüße ich Sie
               herzlichst\pend
           \pstart Ihr \spacefill\mbox{Salten}\pend{}
\pstart
           \noindent{}Ich bin sehr verstimmt und sehr, sehr nervös.\pend
           \endnumbering\briefempfaengerindex{Schnitzler, Arthur@\textsc{Schnitzler, Arthur}!zzzSalten, Felix@\emph{von Felix Salten}!1899-05-041@{4. 5. 1899}|)be}\mylabel{h}  \normalsize

\doendnotes{C}
\bigskip
\vfill

\clearpage

\footnotesize

\lohead{\textsc{register}}

% Definiere theindex-Environment komplett neu ohne reledmac
\makeatletter
\renewenvironment{theindex}{%
  \section*{\indexname}%
  \setlength{\parindent}{0pt}%
  \setlength{\parskip}{0pt plus 0.3pt}%
  \let\item\@idxitem
}{%
  \clearpage
}
\makeatother

\IfFileExists{\jobname-pw.ind}{\input{\jobname-pw.ind}}{}

\end{document}

      