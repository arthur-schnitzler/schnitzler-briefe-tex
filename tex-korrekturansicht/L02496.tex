%% latex-korrekturansicht-vorspann.tex
%% Vorspann für die Korrekturansicht.
%% Lädt die gemeinsame Datei latex-vorspann.tex mit gesetztem Schalter.

\newif\ifkorrekturansicht
\korrekturansichttrue

\input{../tex-inputs/latex-vorspann}


               \section[Hugo Hofmannsthal an Arthur Schnitzler, 29. 12. 1927]{ Hugo Hofmannsthal an Arthur Schnitzler, 29. 12. 1927}\nopagebreak\mylabel{v}\rehead{ }\normalsize\beginnumbering\briefempfaengerindex{Schnitzler, Arthur@\textsc{Schnitzler, Arthur}!zzzHofmannsthal, Hugo von@\emph{von Hugo von Hofmannsthal}!1927-12-291@{29. 12. 1927}|(be} \toendnotes[C]{\smallbreak\pagebreak[2]} \Standort{CUL, Schnitzler, B 43.}
\physDesc{Brief, 1 Blatt, 2 Seiten
\newline{}Handschrift: schwarze Tinte, lateinische Kurrent
\newline{}Schnitzler: mit rotem Buntstift beschriftet: »\textcolor{green}{\textsc{Aph}{[}orismen{]}}« und zahlreiche Unterstreichungen \newline{}Ordnung: mit Bleistift von unbekannter Hand nummeriert:
                                                »392« }\buchAbdrucke{\weitereDrucke{Hugo von Hofmannsthal, Arthur Schnitzler: \emph{Briefwechsel}. Hg. Therese Nickl und Heinrich Schnitzler. Frankfurt am Main: \emph{S. Fischer} 1964, S. 307–308.} }\toendnotes[C]{\smallbreak}\pstart
           {\pb}\textcolor{pink}{Rodaun}{}\ledrightnote{\textcolor{pink}{Rodaun}}. 29 XII. 27\pend
           \pstart{}mein lieber Arthur,\pend\pstart
           nicht leicht hätte mich etwas so bewegen können, wie dieses \textcolor{green}{Buch}{}\ledrightnote{\textcolor{green}{Buch der Sprüche und Bedenken}} mit einer Auswahl Ihrer Betrachtungen und Aphorismen.
                    Wenn ich eines Ihrer Stücke oder eine Ihrer unvergleichlichen Erzählungen
                    aufschlage (beides immer mehrmals im Jahr) so bin ich freilich vermöge der
                    Gegenwart dessen, der hinter den Gestalten steht, auch in Ihrer Gesellschaft.
                    Hier aber widerfährt mir dies in einer doch viel directeren Weise. Es sind nicht
                    die Resultate des Denkens, die bei mir vielfach andere wären, auch nicht einmal
                    die Gegenstände des Denkens (auch in denen tritt die individuelle
                    Verschiedenheit zu Tage, die zwischen uns fast so groß ist wie die
                    wechselseitige Sympathie) – {\pb}es ist etwas viel Intensiveres: der Rhyt{[}h{]}mus Ihres
                    Denkens rührt mich unmittelbar an, und damit das wahre unauflösliche Geheimnis
                    Ihrer Person – und bewegt mich tief. – Ich erinnere mich, dass wenige Tage nach
                    dem \label{K_L02496_1v}\edtext{Tod meiner \textcolor{blue}{Mutter}{}\ledrightnote{\textcolor{blue}{Anna von Hofmannsthal}}}{\lemma{\textnormal{\emph{Tod meiner Mutter}}}\Cendnote{\textnormal{Sie starb am
                            22. 3. 1904.}}}\label{K_L02496_1h} mich der Anblick \substVorne{}\textsuperscript{I}\substDazwischen{}i\substHinten{}hres Kastens tief erschütterte; da lagen in Fächern, nett in
                    Seidenpapier gewickelt, Schlüssel, Notizbücher, hundert kleine unansehnliche
                    Gegenstände, alle verknüpft mit den kleinen Bemühungen und Lasten eines
                    weiblichen Lebens, und aus dem allen brach das Gefühl dieses nun abgerissenen
                    Lebens mit einer das Herz zusa{\geminationm}endrückenden Gewalt
                    hervor, ganz anders als etwa aus hinterlassenen Briefen.\pend
           \pstart
           Ich weiß, Sie haben jetzt den Besuch Ihrer \textcolor{blue}{Tochter}{}\ledrightnote{→\textcolor{blue}{Lili Schnitzler}}. Wenn Sie mich später einmal sehen wollen,
                    schreiben Sie mir ein Wort.\pend
           \pstart
           In Freundschaft{\\[\baselineskip]}Ihr \spacefill\mbox{Hugo.}\pend
           \leftskip=0em{}\endnumbering\briefempfaengerindex{Schnitzler, Arthur@\textsc{Schnitzler, Arthur}!zzzHofmannsthal, Hugo von@\emph{von Hugo von Hofmannsthal}!1927-12-291@{29. 12. 1927}|)be}\mylabel{h}  \normalsize

\doendnotes{C}
\bigskip
\vfill

\clearpage

\footnotesize

\lohead{\textsc{register}}

% Definiere theindex-Environment komplett neu ohne reledmac
\makeatletter
\renewenvironment{theindex}{%
  \section*{\indexname}%
  \setlength{\parindent}{0pt}%
  \setlength{\parskip}{0pt plus 0.3pt}%
  \let\item\@idxitem
}{%
  \clearpage
}
\makeatother

\IfFileExists{\jobname-pw.ind}{\input{\jobname-pw.ind}}{}

\end{document}

      