%% latex-korrekturansicht-vorspann.tex
%% Vorspann für die Korrekturansicht.
%% Lädt die gemeinsame Datei latex-vorspann.tex mit gesetztem Schalter.

\newif\ifkorrekturansicht
\korrekturansichttrue

\input{../tex-inputs/latex-vorspann}


               \section[Hermann Bahr an Arthur Schnitzler, 20. 7. 1913]{ Hermann Bahr an Arthur Schnitzler, 20. 7. 1913}\nopagebreak\mylabel{v}\rehead{ }\normalsize\beginnumbering\briefempfaengerindex{Schnitzler, Arthur@\textsc{Schnitzler, Arthur}!zzzBahr, Hermann@\emph{von Hermann Bahr}!1913-07-201@{20. 7. 1913}|(be} \toendnotes[C]{\smallbreak\pagebreak[2]} \Standort{CUL, Schnitzler, B 5b.}
\physDesc{Bildpostkarte
\newline{}Handschrift: schwarze Tinte, deutsche Kurrent\newline{}Versand: ohne postalischen Übermittlungsvermerk \newline{}Ordnung: mit Bleistift von unbekannter Hand nummeriert:
                                    »178« }\buchAbdrucke{\weitereDrucke{Hermann Bahr, Arthur Schnitzler: \emph{Briefwechsel, Aufzeichnungen, Dokumente (1891–1931)}. Hg. Kurt Ifkovits und Martin Anton Müller. Göttingen: \emph{Wallstein} 2018, S. 490.} }\toendnotes[C]{\smallbreak}\pstart
           \noindent{}\centering{}\textcolor{gray}{\textbf{{\pb}\textcolor{pink}{Salzburg, Am Stein}{}\ledrightnote{\textcolor{pink}{Äußerer Stein}}}}\pend
           \pstart
           \raggedleft{}{\pb}20. 7. 13\pend
           \pstart{}Lieber Arthur!\pend\pstart
           Herzlichen Dank für Dein ſo liebes Geſchenk, das meine Wand ſchmückt, und für \textcolor{blue}{Euer}{}\ledrightnote{→\textcolor{blue}{Olga Schnitzler}} gutes Telegramm, das mein
               Herz erfreut!\pend
           \pstart
           Bleibt mir, was \textcolor{blue}{Ihr}{}\ledrightnote{→\textcolor{blue}{Olga Schnitzler}} mir ſeit ſo
               vielen Jahren ſeid! Ich will immer der \textcolor{blue}{Eure}{}\ledrightnote{→\textcolor{blue}{Olga Schnitzler}}{ }ſein!\pend
           \pstart \spacefill\mbox{HermannBahr}\pend{}\endnumbering\briefempfaengerindex{Schnitzler, Arthur@\textsc{Schnitzler, Arthur}!zzzBahr, Hermann@\emph{von Hermann Bahr}!1913-07-201@{20. 7. 1913}|)be}\mylabel{h}  \normalsize

\doendnotes{C}
\bigskip
\vfill

\clearpage

\footnotesize

\lohead{\textsc{register}}

% Definiere theindex-Environment komplett neu ohne reledmac
\makeatletter
\renewenvironment{theindex}{%
  \section*{\indexname}%
  \setlength{\parindent}{0pt}%
  \setlength{\parskip}{0pt plus 0.3pt}%
  \let\item\@idxitem
}{%
  \clearpage
}
\makeatother

\IfFileExists{\jobname-pw.ind}{\input{\jobname-pw.ind}}{}

\end{document}

      