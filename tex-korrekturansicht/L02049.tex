%% latex-korrekturansicht-vorspann.tex
%% Vorspann für die Korrekturansicht.
%% Lädt die gemeinsame Datei latex-vorspann.tex mit gesetztem Schalter.

\newif\ifkorrekturansicht
\korrekturansichttrue

\input{../tex-inputs/latex-vorspann}


               \section[Georg Brandes an Arthur Schnitzler, vor dem 2. 12. 1911]{ Georg Brandes an Arthur Schnitzler, vor dem 2. 12. 1911}\nopagebreak\mylabel{v}\rehead{ }\normalsize\beginnumbering\briefempfaengerindex{Schnitzler, Arthur@\textsc{Schnitzler, Arthur}!zzzBrandes, Georg@\emph{von Georg Brandes}!1911-12-021@{vor dem 2. 12. 1911}|(be} \toendnotes[C]{\smallbreak\pagebreak[2]} \Standort{DLA, A:Schnitzler, HS.NZ85.1.3446,3.}
\physDesc{Visitenkarte
\newline{}Handschrift: schwarze Tinte, lateinische Kurrent}\toendnotes[C]{\smallbreak}\pstart
           \noindent{}\centering{}{\pb}\textcolor{gray}{\textbf{Georg Brandes}}\pend
           \pstart
           \noindent{}Verehrter Freund, lieber Arthur Schnitzler. Frau \label{K_L02049_1v}\edtext{\textcolor{blue}{Helene Heyman}{}\ledrightnote{\textcolor{blue}{Helene Heyman}}}{\lemma{\textnormal{\emph{Helene Heyman}}}\Cendnote{\textnormal{Die undatierte Karte geht dem Brief von
                     \textcolor{blue}{Helene Heyman} voran (\emph{Deutsches Literaturarchiv}, A:Schnitzler, HS.NZ85.1.3446,1): »\textcolor{pink}{Kopenhagen}, \textcolor{pink}{\textcolor{gray}{\textbf{ROSBÆKSVEJ STRANDVEJ}}}{ }2./12 1911{ / }Sehr geehrter Herr Schnitzler!{ / }Ich bitte Sie zu entschuldigen, dass ich mich, ehedem ich Ihnen vollständig
                        fremd bin, an Sie wende mit der Vorfrage Ihr neuestes Werk »\textcolor{green}{Das weite Land}« auf \textcolor{pink}{dänisch} übersetzen zu dürfen, und es womöglich hier am \textcolor{pink}{königlichen Theater} zur Aufführung zu
                        bringen.{ / }Herr Professor \textcolor{blue}{Georg Brandes} hatte die
                        grosse Freundlichkeit mich bei Ihnen zu introducieren, da ich ja nicht
                        erwarten konnte, dass Sie einer Fremden Ihre Arbeit anvertrauen wollten.
                        {\pb}Ich darf wohl hinzufügen, dass ich eine grosse Verehrerin Ihrer Werke
                        bin, besonders »\textcolor{green}{Der Weg ins Freie}«, hat
                        mich aufs höchste interessiert und gefesselt. Alle Menschen, die in dem
                        Roman vorkommen, standen Einem geradezu nahe, man fühlte mit ihnen. Die
                        grösste Lust hatte ich schon damals an Sie zu schreiben und Ihnen zu
                        erzählen wie ganz anders das Verhältnis zwischen Christen und Juden hier
                        ist, es giebt wirklich nur wenig richtigen Antisemitismus hier, man merkt
                        ihn jedenfalls sehr selten. Aber die Juden hier sind auch sehr
                        ver{\pb}nünftig, gemischte Ehen gehören zur Tagesordnung, und sie sind nicht
                        auf pekuniären Vorteil taxiert. Im Verkehr mit Christen wird die
                        Religionsfrage so wenig wie möglich berührt, während gerade die \textcolor{pink}{deutschen} und \textcolor{pink}{österreichischen} Juden immer und ewig auf dies heikle Thema
                        zurückkommen, und dadurch die Kluft zwischen den Rassen nur erweitern.{ / }Ich selbst bin \textcolor{pink}{Rheinländerin} und kam als ganz junges Mädchen hierher, und bin nun
                        schon 17 Jahre hier verheiratet, also ich kann den Unterschied nur {\pb}zu
                        gut merken.{ / }Entschuldigen Sie, dass ich so frei war mich an Sie zu wenden, hoffentlich
                        wird Ihre Antwort eine günstige für mich sein.{ / }Mit vorzüglicher Hochachtung zeichnet{ / }Ihre ergebene{ / }Helene Heyman.«}}}\label{K_L02049_1h} wünscht von mir bei Ihnen introducirt zu werden. Sie können der
               Dame vollständig vertrauen. Deutsch geboren versteht und schreibt sie \textcolor{pink}{Dänisch}{}\ledrightnote{\textcolor{pink}{Dänemark}} mit vollkommenster Sicherheit.\pend
           \endnumbering\briefempfaengerindex{Schnitzler, Arthur@\textsc{Schnitzler, Arthur}!zzzBrandes, Georg@\emph{von Georg Brandes}!1911-12-021@{vor dem 2. 12. 1911}|)be}\mylabel{h}  \normalsize

\doendnotes{C}
\bigskip
\vfill

\clearpage

\footnotesize

\lohead{\textsc{register}}

% Definiere theindex-Environment komplett neu ohne reledmac
\makeatletter
\renewenvironment{theindex}{%
  \section*{\indexname}%
  \setlength{\parindent}{0pt}%
  \setlength{\parskip}{0pt plus 0.3pt}%
  \let\item\@idxitem
}{%
  \clearpage
}
\makeatother

\IfFileExists{\jobname-pw.ind}{\input{\jobname-pw.ind}}{}

\end{document}

      