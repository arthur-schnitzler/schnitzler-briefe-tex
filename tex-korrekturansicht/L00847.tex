%% latex-korrekturansicht-vorspann.tex
%% Vorspann für die Korrekturansicht.
%% Lädt die gemeinsame Datei latex-vorspann.tex mit gesetztem Schalter.

\newif\ifkorrekturansicht
\korrekturansichttrue

\input{../tex-inputs/latex-vorspann}


               \section[Stefan Großmann an Arthur Schnitzler, 16. 9. 1898]{ Stefan Großmann an Arthur Schnitzler, 16. 9. 1898}\nopagebreak\mylabel{v}\rehead{ }\normalsize\beginnumbering\briefempfaengerindex{Schnitzler, Arthur@\textsc{Schnitzler, Arthur}!zzzGrossmann, Stefan@\emph{von Stefan Großmann}!1898-09-162@{16. 9. 1898}|(be} \toendnotes[C]{\smallbreak\pagebreak[2]} \Standort{CUL, Schnitzler, B 34.}
\physDesc{Brief, 1 Blatt, 2 Seiten
\newline{}Handschrift: schwarze Tinte, deutsche Kurrent
\newline{}Schnitzler: mit rotem Buntstift drei Unterstreichungen \newline{}Ordnung: mit Bleistift von unbekannter Hand nummeriert: »1« }\toendnotes[C]{\smallbreak}\pstart
           \noindent{}{\pb}\textcolor{gray}{\textbf{\textcolor{brown}{WIENER RUNDSCHAU}{}\ledrightnote{\textcolor{brown}{Wiener Rundschau}}.}}\pend
           \pstart
           \textcolor{gray}{\textbf{HERAUSGEBER}}\pend
           \pstart
           \textcolor{gray}{\textbf{\textcolor{blue}{GUSTAV SCHOENAICH}{}\ledrightnote{\textcolor{blue}{Gustav Schönaich}}.}}\pend
           \pstart
           \textcolor{gray}{\textbf{\textcolor{blue}{FELIX RAPPAPORT}{}\ledrightnote{\textcolor{blue}{Felix Rappaport}}.}}\hfill \textcolor{gray}{\textbf{\textcolor{pink}{Wien}{}\ledrightnote{\textcolor{pink}{Wien}},}}{ }16. September \textcolor{gray}{\textbf{189}}8\pend
           \pstart
           \textcolor{gray}{\textbf{REDACTION UND ADMINISTRATION:}}\pend
           \pstart
           \textcolor{gray}{\textbf{\textcolor{pink}{WIEN}{}\ledrightnote{\textcolor{pink}{Wien}}}}\pend
           \pstart
           \textcolor{gray}{\textbf{\textcolor{pink}{I/1 SPIEGELGASSE 11}{}\ledrightnote{\textcolor{pink}{Spiegelgasse}}.}}\pend
           \pstart
           \textcolor{gray}{\textbf{TELEPHON NR. 2579.}}\pend
           \pstart\center{}Sehr geehrter Herr Doctor!\pend\pstart
           Ich leſe in den Zeitungen von \introOben{}Ihren\introOben{} drei neuen \textcolor{green}{Einactern}{}\ledrightnote{→\textcolor{green}{Der grüne Kakadu – Paracelsus – Die Gefährtin. Drei Einakter}}, die D\textsuperscript{r} \textsc{\textcolor{blue}{Brahm}{}\ledrightnote{\textcolor{blue}{Otto Brahm}}} im »\textcolor{brown}{Deutſchen Theater}{}\ledrightnote{\textcolor{brown}{Deutsches Theater Berlin}}« aufführen wird.\pend
           \pstart
           Darf ich Sie nochmals, aufrichtig und innigſt bitten, ob Sie mir einen von dieſen zum
               Abdruck in der »\textcolor{brown}{Rundſchau}{}\ledrightnote{\textcolor{brown}{Wiener Rundschau}}« überlaſſen möchten? Ich
               gebe Ihnen die Verſicherung, daſs ich glücklich wäre, wenn Sie meine Bitte erfüllen
               würden, daſs ich von Tag zu Tag \strikeout{\textcolor{gray}{×}\-\textcolor{gray}{×}} mehr einſehe, wie bornirt, leicht-fertig meine \strikeout{Radi} literariſchen Radicalismen von ſeinerzeit waren.
               Ich brauche nur an die \uline{nach} Ihnen Kommenden zu denken
               u bin beſchämt.\pend
           \pstart
           Überdies würden Sie \substVorne{}\textsuperscript{ſich}\substDazwischen{}mich\substHinten{} hiedurch beſonders verpflichten, weil mir Ihre Gabe eine moraliſche Unter{\pb}ſtützung wäre, gerade jetzt beſonders
               werthvoll, wo die literariſchen Schwarzkünſtler aller Art meinem \textcolor{blue}{Herausgeber}{}\ledrightnote{→\textcolor{blue}{Gustav Schönaich}{\newline}→\textcolor{blue}{Felix Rappaport}} in den
               Ohren liegen.\pend
           \pstart
           Verzeihen Sie, bitte, die Beläſtigung und erfüllen Sie – bitte – bald mein
               Anſuchen.\pend
           \pstart
           Ich bin{\\[\baselineskip]} Ihr \uline{sehr}{ }\uline{ergebener}{\\[\baselineskip]}\spacefill\mbox{Stefan Großmann}\pend
           \leftskip=0em{}\endnumbering\briefempfaengerindex{Schnitzler, Arthur@\textsc{Schnitzler, Arthur}!zzzGrossmann, Stefan@\emph{von Stefan Großmann}!1898-09-162@{16. 9. 1898}|)be}\mylabel{h}  \normalsize

\doendnotes{C}
\bigskip
\vfill

\clearpage

\footnotesize

\lohead{\textsc{register}}

% Definiere theindex-Environment komplett neu ohne reledmac
\makeatletter
\renewenvironment{theindex}{%
  \section*{\indexname}%
  \setlength{\parindent}{0pt}%
  \setlength{\parskip}{0pt plus 0.3pt}%
  \let\item\@idxitem
}{%
  \clearpage
}
\makeatother

\IfFileExists{\jobname-pw.ind}{\input{\jobname-pw.ind}}{}

\end{document}

      