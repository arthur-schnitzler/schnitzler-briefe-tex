%% latex-korrekturansicht-vorspann.tex
%% Vorspann für die Korrekturansicht.
%% Lädt die gemeinsame Datei latex-vorspann.tex mit gesetztem Schalter.

\newif\ifkorrekturansicht
\korrekturansichttrue

\input{../tex-inputs/latex-vorspann}


               \section[Arthur Schnitzler an Richard Beer-Hofmann, 11. 1. 1893]{ Arthur Schnitzler an Richard Beer-Hofmann, 11. 1. 1893}\nopagebreak\mylabel{v}\rehead{ }\normalsize\beginnumbering\briefempfaengerindex{Beer-Hofmann, Richard@\textsc{Beer-Hofmann, Richard}!zzzSchnitzler, Arthur@\emph{von Arthur Schnitzler}!1893-01-112@{11. 1. 1893}|(be} \toendnotes[C]{\smallbreak\pagebreak[2]} \Standort{YCGL, MSS 31.}
\physDesc{Kartenbrief
\newline{}Handschrift: schwarze Tinte, deutsche Kurrent\newline{}Versand: 1) Stempel: »\nobreak{}Wien 1/1, 11 1 93, 3–4 N\nobreak{}«.  2) Stempel: »\nobreak{}Wien 1/{[}1{]}, 11/1. 93, 6½–8 N, Bestellt\nobreak{}«. }\buchAbdrucke{\weitereDrucke{Arthur Schnitzler, Richard Beer-Hofmann: \emph{Briefwechsel 1891–1931}. Hg. Konstanze Fliedl. Wien, Zürich: \emph{Europaverlag} 1992, S. 41.} }\toendnotes[C]{\smallbreak}\pstart{}{\pb}\textsc{Herrn Doctor Richard}\pend{}\pstart{}\textsc{Beer Hofmann}\pend{}\pstart{}\textsc{\textcolor{pink}{Wien}{}\ledrightnote{\textcolor{pink}{Wien}}}\pend{}\pstart{}\textsc{\textcolor{pink}{I Wollzeile
                  15}{}\ledrightnote{\textcolor{pink}{Wollzeile}}}.\pend{}{\bigskip}\pstart{}{\pb}Lieber Richard, \pend\pstart
           der kleine \textcolor{blue}{Kraus}{}\ledrightnote{\textcolor{blue}{Karl Kraus}} wird Ihnen für
                  Samſtag{ }Abend einen Sitz zu den \textcolor{green}{Räubern}{}\ledrightnote{\textcolor{green}{Die Räuber}} in
                  \textcolor{pink}{\textsc{Rdlfsheim}}{}\ledrightnote{→\textcolor{pink}{Volkstheater in Rudolphsheim}} (\textcolor{green}{Franz Moor}{}\ledrightnote{→\textcolor{green}{Die Räuber}} – Herr \textcolor{blue}{Kraus}{}\ledrightnote{\textcolor{blue}{Karl Kraus}}) ſenden.\pend
           \pstart
           Bitte gehen Sie, wir gehen alle. Sollt ich Sie nicht früher ſehen, ſo wollen wir uns
               vielleicht im \textcolor{pink}{\textsc{Griensteidl}}{}\ledrightnote{\textcolor{pink}{Café Griensteidl}} um 6 Uhr Abds treffen.\pend
           \pstart
           Herzlich{\\[\baselineskip]}Ihr \spacefill\mbox{Arthur}\pend
           \leftskip=0em{}\endnumbering\briefempfaengerindex{Beer-Hofmann, Richard@\textsc{Beer-Hofmann, Richard}!zzzSchnitzler, Arthur@\emph{von Arthur Schnitzler}!1893-01-112@{11. 1. 1893}|)be}\mylabel{h}  \normalsize

\doendnotes{C}
\bigskip
\vfill

\clearpage

\footnotesize

\lohead{\textsc{register}}

% Definiere theindex-Environment komplett neu ohne reledmac
\makeatletter
\renewenvironment{theindex}{%
  \section*{\indexname}%
  \setlength{\parindent}{0pt}%
  \setlength{\parskip}{0pt plus 0.3pt}%
  \let\item\@idxitem
}{%
  \clearpage
}
\makeatother

\IfFileExists{\jobname-pw.ind}{\input{\jobname-pw.ind}}{}

\end{document}

      