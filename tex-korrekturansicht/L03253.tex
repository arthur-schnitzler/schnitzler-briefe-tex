%% latex-korrekturansicht-vorspann.tex
%% Vorspann für die Korrekturansicht.
%% Lädt die gemeinsame Datei latex-vorspann.tex mit gesetztem Schalter.

\newif\ifkorrekturansicht
\korrekturansichttrue

\input{../tex-inputs/latex-vorspann}


\renewcommand{\erwaehntePersonen}{Personen: Richard Beer-Hofmann, Olga Schnitzler, Leo Van-Jung}
\renewcommand{\erwaehnteOrte}{Orte: Edmund-Weiß-Gasse, Hotel Tegetthoff, Wien}
\renewcommand{\erwaehnteWerke}{}
\section[ Paul Goldmann an Arthur Schnitzler, 7. 10. 1907]{Paul Goldmann an Arthur Schnitzler, 7. 10. 1907}
\nopagebreak\mylabel{v}
\rehead{ }\normalsize\beginnumbering\briefempfaengerindex{Schnitzler, Arthur@\textsc{Schnitzler, Arthur}!zzzGoldmann, Paul@\emph{von Paul Goldmann}!1907-10-075@{7. 10. 1907}|(be}
\toendnotes[C]{\smallbreak\pagebreak[2]}\Standort{DLA, A:Schnitzler, HS.NZ85.1.3175.}
\physDesc{Postkarte
\newline{}Handschrift: 1) schwarze Tinte, deutsche Kurrent\hspace{1em}2) schwarze Tinte, lateinische Kurrent (\noindent{}Adresse)\hspace{1em}
\newline{}Versand: 1) Stempel: »\nobreak{}1/1 Wien 18, 7. X. 07, XII h\nobreak{}«.   2) Stempel: »\nobreak{}Wien 110, 7. X. 07, –3\nobreak{}«. 
\newline{}Schnitzler: mit Bleistift das Datum »7. 10. {[}19{]}07« vermerkt }\toendnotes[C]{\smallbreak}\pstart{}{\pb}Herrn\pend{}\pstart{}Dr. Arthur Schnitzler\pend{}\pstart{}\textcolor{pink}{Wien}{}\ledrightnote{\textcolor{pink}{Wien}}\pend{}\pstart{}\textcolor{pink}{XVIII. Spöttelgaſse 7}{}\ledrightnote{\textcolor{pink}{Edmund-Weiß-Gasse}}.\pend{}
{\bigskip}
\pstart
           \noindent{}{\pb}\textcolor{gray}{\textbf{TELEGRAMME}}\hfill Montag{ }früh.\pend
           
\pstart
           \textcolor{gray}{\textbf{\textcolor{pink}{TEGETTHOFFHOTEL}{}\ledrightnote{\textcolor{pink}{Hotel Tegetthoff}}, \textcolor{pink}{WIEN}{}\ledrightnote{\textcolor{pink}{Wien}}.}}\pend
           
\pstart
           Lieber Freund, Darf ich vielleicht morgen, Dienſtag, Abend zu \textcolor{blue}{Euch}{}\ledrightnote{{$\rightarrow$}\textcolor{blue}{Olga Schnitzler}} kommen? Oder wollen wir uns vielleicht
               in der \textcolor{pink}{Stadt}{}\ledrightnote{{$\rightarrow$}\textcolor{pink}{Wien}} treffen? \textsc{\textcolor{blue}{Beer-Hofmann}{}\ledrightnote{\textcolor{blue}{Richard Beer-Hofmann}}} u. \textsc{\textcolor{blue}{Leo Van-Jung}{}\ledrightnote{\textcolor{blue}{Leo Van-Jung}}} würden gern \label{K-L03253-1v}\edtext{mit dabei ſein}{\lemma{\textnormal{\emph{mit dabei ſein}}}\Cendnote{\textnormal{siehe A. S.: \emph{Tagebuch}, 8. 10. 1907}}}\label{K-L03253-1h}. Ich bitte Dich um Nachricht ins \textsc{\textcolor{pink}{Hotel Tegethoff}{}\ledrightnote{\textcolor{pink}{Hotel Tegetthoff}}} u. bin mit herzlichen Grüßen an Deine \textcolor{blue}{Frau}{}\ledrightnote{{$\rightarrow$}\textcolor{blue}{Olga Schnitzler}} u. Dich\pend
           
\pstart
           Dein {\\[\baselineskip]}\spacefill\mbox{Paul Goldmann.}\pend
           \leftskip=0em{}\endnumbering\briefempfaengerindex{Schnitzler, Arthur@\textsc{Schnitzler, Arthur}!zzzGoldmann, Paul@\emph{von Paul Goldmann}!1907-10-075@{7. 10. 1907}|)be}\mylabel{h}  \normalsize

\doendnotes{C}
\bigskip
\vfill

\clearpage

\footnotesize

\lohead{\textsc{register}}

% Definiere theindex-Environment komplett neu ohne reledmac
\makeatletter
\renewenvironment{theindex}{%
  \section*{\indexname}%
  \setlength{\parindent}{0pt}%
  \setlength{\parskip}{0pt plus 0.3pt}%
  \let\item\@idxitem
}{%
  \clearpage
}
\makeatother

\IfFileExists{\jobname-pw.ind}{\input{\jobname-pw.ind}}{}

\end{document}

      