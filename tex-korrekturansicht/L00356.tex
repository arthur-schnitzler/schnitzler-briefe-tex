%% latex-korrekturansicht-vorspann.tex
%% Vorspann für die Korrekturansicht.
%% Lädt die gemeinsame Datei latex-vorspann.tex mit gesetztem Schalter.

\newif\ifkorrekturansicht
\korrekturansichttrue

\input{../tex-inputs/latex-vorspann}


               \section[Arthur Schnitzler an Richard Beer-Hofmann, {[}18. 7. 1894?{]}]{ Arthur Schnitzler an Richard Beer-Hofmann, {[}18. 7. 1894?{]}}\nopagebreak\mylabel{v}\rehead{ }\normalsize\beginnumbering\briefempfaengerindex{Beer-Hofmann, Richard@\textsc{Beer-Hofmann, Richard}!zzzSchnitzler, Arthur@\emph{von Arthur Schnitzler}!1894-07-183@{{[}18. 7. 1894?{]}}|(be} \toendnotes[C]{\smallbreak\pagebreak[2]} \Standort{YCGL, MSS 31.}
\physDesc{Brief, 1 Blatt, 3 Seiten, Umschlag
\newline{}Handschrift: Bleistift, deutsche Kurrent\newline{}Versand: ohne postalischen Übermittlungsvermerk }\pstart{}{\pb}Herrn \textsc{Dr. Richard Beer
                     Hofmann}\pend{}\pstart{}\textsc{\textcolor{pink}{Ischl}{}\ledrightnote{\textcolor{pink}{Bad Ischl}}}\pend{}\pstart{}\textcolor{pink}{\textsc{Egelmoos 22}}{}\ledrightnote{\textcolor{pink}{Eglmoosgasse}}.\pend{}{\bigskip}\pstart
           \noindent{}{\pb}Lieber Richard! –Ich wüßt nicht, warum \textcolor{pink}{Salzburg}{}\ledrightnote{\textcolor{pink}{Salzburg}} ganz ins Waſſer fallen ſoll, weil \textcolor{blue}{Bahr}{}\ledrightnote{\textcolor{blue}{Hermann Bahr}} keine Zeit hat. Auch hat \textcolor{blue}{Hugo}{}\ledrightnote{\textcolor{blue}{Hugo von Hofmannsthal}}
               ziemlich ſicher zugeſagt. – Ich fahr jedenfalls über \textcolor{pink}{Salzburg}{}\ledrightnote{\textcolor{pink}{Salzburg}} zurück. – Ich antworte dem {\pb}\textcolor{blue}{Bahr}{}\ledrightnote{\textcolor{blue}{Hermann Bahr}} natürlich, daſs ich Samſtag
               noch hier bin. Ich werd wohl Sonntag wegfahren. –\pend
           \pstart
           Heut geh ich zwiſchen 5 u 6 zu \textcolor{blue}{Ornſtein}{}\ledrightnote{\textcolor{blue}{Sophie Ornstein}{\newline}\textcolor{blue}{Wilhelm Ornstein}}{ }\introOben{}(\textcolor{blue}{Gina Z.}{}\ledrightnote{\textcolor{blue}{Regine Zeisler}})\introOben{}. Ich glaube, dſs ich dann
               zwiſchen 7 u ½ 8 auf die \textcolor{pink}{\textsc{Esplan.}}{}\ledrightnote{\textcolor{pink}{Esplanade}} wi{\geminationm}eln werde. Nett {\pb}wärs we{\geminationn}
               Sie mit mir bei \textcolor{pink}{Leopold}{}\ledrightnote{\textcolor{pink}{Hotel und Pension Rudolfshöhe (Leopold Petter)}}{ }\introOben{}zu Nacht,\introOben{}{ }ſpeiſten.\pend
           \pstart Herzlich Ihr \spacefill\mbox{Arthur}\pend{}\endnumbering\briefempfaengerindex{Beer-Hofmann, Richard@\textsc{Beer-Hofmann, Richard}!zzzSchnitzler, Arthur@\emph{von Arthur Schnitzler}!1894-07-183@{{[}18. 7. 1894?{]}}|)be}\mylabel{h}  \normalsize

\doendnotes{C}
\bigskip
\vfill

\clearpage

\footnotesize

\lohead{\textsc{register}}

% Definiere theindex-Environment komplett neu ohne reledmac
\makeatletter
\renewenvironment{theindex}{%
  \section*{\indexname}%
  \setlength{\parindent}{0pt}%
  \setlength{\parskip}{0pt plus 0.3pt}%
  \let\item\@idxitem
}{%
  \clearpage
}
\makeatother

\IfFileExists{\jobname-pw.ind}{\input{\jobname-pw.ind}}{}

\end{document}

      