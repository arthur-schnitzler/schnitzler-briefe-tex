%% latex-korrekturansicht-vorspann.tex
%% Vorspann für die Korrekturansicht.
%% Lädt die gemeinsame Datei latex-vorspann.tex mit gesetztem Schalter.

\newif\ifkorrekturansicht
\korrekturansichttrue

\input{../tex-inputs/latex-vorspann}


\renewcommand{\erwaehntePersonen}{Personen: Paul Goldmann, Fedor Mamroth, A. Monney, Josef Rosengart, Heinrich Schnitzler, Olga Schnitzler}
\renewcommand{\erwaehnteOrte}{Orte: Berlin, Frankfurt am Main, Grand Hôtel Beau Séjour au Lac, Grand Hôtel Monney, Montreux, Reuterweg, Schweiz, Südtirol, Tirol, Wien}
\renewcommand{\erwaehnteWerke}{}
\section[ Paul Goldmann an Arthur Schnitzler, 1. 9. {[}1902{]}]{Paul Goldmann an Arthur Schnitzler, 1. 9. {[}1902{]}}
\nopagebreak\mylabel{v}
\rehead{ }\normalsize\beginnumbering\briefempfaengerindex{Schnitzler, Arthur@\textsc{Schnitzler, Arthur}!zzzGoldmann, Paul@\emph{von Paul Goldmann}!1902-09-011@{1. 9. {[}1902{]}}|(be}
\toendnotes[C]{\smallbreak\pagebreak[2]}\Standort{DLA, A:Schnitzler, HS.NZ85.1.3172.}
\physDesc{Brief, 1 Blatt, 2 Seiten, 802 Zeichen
\newline{}Handschrift: schwarze Tinte, deutsche Kurrent
\newline{}Schnitzler: mit rotem Buntstift eine Unterstreichung }\toendnotes[C]{\smallbreak}
\pstart
           \noindent{}\raggedleft{}{\pb}\textcolor{gray}{\textbf{\textsc{\label{K_L03222-1v}\edtext{G\textsuperscript{ds}{ }Hôtels }{\lemma{\textnormal{\emph{G\textsuperscript{ds} Hôtels}}}\Cendnote{\textnormal{\begin{otherlanguage}{french}Grands Hôtels\end{otherlanguage}}}}\label{K_L03222-1h}}}}\pend
           
\pstart
           \noindent{}\raggedleft{}\textcolor{gray}{\textbf{\textbf{\textcolor{pink}{MONNEY}{}\ledrightnote{\textcolor{pink}{Grand Hôtel Monney}}}}}\pend
           
\pstart
           \noindent{}\raggedleft{}\textcolor{gray}{\textbf{{\kaufmannsund}}}\pend
           
\pstart
           \noindent{}\textcolor{gray}{\textbf{\textcolor{blue}{A. MONNEY}{}\ledrightnote{\textcolor{blue}{A. Monney}}}}\hfill \textcolor{gray}{\textbf{\textbf{\textcolor{pink}{BEAU SÉJOUR AU LAC}{}\ledrightnote{\textcolor{pink}{Grand Hôtel Beau Séjour au Lac}}}}}\pend
           
\pstart
           \textcolor{gray}{\textbf{\label{K_L03222-2v}\edtext{\begin{otherlanguage}{french}PROPRIET\textsuperscript{RE}\end{otherlanguage}}{\lemma{\textnormal{\emph{Propriet\textsuperscript{RE}}}}\Cendnote{\textnormal{propriétaire, französisch:
                              Eigentümer/Eigentümerin}}}\label{K_L03222-2h}}}\hfill \textcolor{gray}{\textbf{\textcolor{pink}{MONTREUX}{}\ledrightnote{\textcolor{pink}{Montreux}} (\begin{otherlanguage}{french}\textcolor{pink}{SUISSE}{}\ledrightnote{\textcolor{pink}{Schweiz}}\end{otherlanguage})}}\pend
           
\pstart
           \centering{}\textcolor{gray}{\textbf{\label{K_L03222-3v}\edtext{\begin{otherlanguage}{french}Ascenseur hydraulique\end{otherlanguage}}{\lemma{\textnormal{\emph{Ascenseur hydraulique}}}\Cendnote{\textnormal{französisch: hydraulischer
                        Aufzug}}}\label{K_L03222-3h}}}\pend
           
\pstart
           \textsc{\textcolor{pink}{Montreux}{}\ledrightnote{\textcolor{pink}{Montreux}}}, 1. September.\pend
           
\pstart{}Mein lieber Freund,\pend
\pstart
           Mit jeder Poſt aus \textcolor{pink}{Berlin}{}\ledrightnote{\textcolor{pink}{Berlin}} habe ich Deine lieben
               Nachrichten erwartet. Nachdem ſie heut wieder nicht
               eingelangt ſind, bin ich wirklich in Unruhe. Ich werde Donnerſtag in \textcolor{pink}{Frankfurt}{}\ledrightnote{\textcolor{pink}{Frankfurt am Main}} ſein und bitte
               Dich ſehr, mir dorthin an die Adreſſe meines \textcolor{blue}{Schwager}{}\ledrightnote{{$\rightarrow$}\textcolor{blue}{Josef Rosengart}}s (\textsc{Dr. \textcolor{blue}{Rosengart}{}\ledrightnote{\textcolor{blue}{Josef Rosengart}}}, \textcolor{pink}{\textsc{Reuterweg 59}}{}\ledrightnote{\textcolor{pink}{Reuterweg}}) zu ſchreiben, wie es Dir, dem \textcolor{blue}{Kinde}{}\ledrightnote{{$\rightarrow$}\textcolor{blue}{Heinrich Schnitzler}} und der \textcolor{blue}{Mutter}{}\ledrightnote{{$\rightarrow$}\textcolor{blue}{Olga Schnitzler}} geht?\pend
           
\pstart
           {\pb}Nach \textcolor{pink}{\textsc{Wien}}{}\ledrightnote{\textcolor{pink}{Wien}} komme ich nicht. Die Zeit iſt um, das Geld iſt alle. Ich habe ein ſehr ſchönes
               Stück Welt geſehen. In der \textcolor{pink}{Schweiz}{}\ledrightnote{\textcolor{pink}{Schweiz}} ſind die
               großartigen landſchaftlichen Eindrücke gehäufter, als in \textcolor{pink}{Tirol}{}\ledrightnote{\textcolor{pink}{Tirol}{\newline}\textcolor{pink}{Südtirol}}, und leichter zu erreichen. \label{K_L03222-4v}\edtext{Nächſtes Jahr mußt Du hingehen}{\lemma{\textnormal{\emph{Nächſtes … hingehen}}}\Cendnote{\textnormal{nicht geschehen}}}\label{K_L03222-4h}. Ich war \strikeout{die ga} 14 Tage lang mit meinem \textcolor{blue}{Onkel}{}\ledrightnote{{$\rightarrow$}\textcolor{blue}{Fedor Mamroth}} zuſammen und habe in ihm einen überaus
               liebenswürdigen und anregenden Reiſekameraden gehabt.\pend
           
\pstart
           Viele treue Grüße! {\\[\baselineskip]}Dein {\\[\baselineskip]}\spacefill\mbox{Paul Goldmnn}\pend
           \leftskip=0em{}
\pstart
           \noindent{}Herzliche Grüße an \textsc{\textcolor{blue}{Olga}{}\ledrightnote{\textcolor{blue}{Olga Schnitzler}}}!\pend
           \endnumbering\briefempfaengerindex{Schnitzler, Arthur@\textsc{Schnitzler, Arthur}!zzzGoldmann, Paul@\emph{von Paul Goldmann}!1902-09-011@{1. 9. {[}1902{]}}|)be}\mylabel{h}  \normalsize

\doendnotes{C}
\bigskip
\vfill

\clearpage

\footnotesize

\lohead{\textsc{register}}

% Definiere theindex-Environment komplett neu ohne reledmac
\makeatletter
\renewenvironment{theindex}{%
  \section*{\indexname}%
  \setlength{\parindent}{0pt}%
  \setlength{\parskip}{0pt plus 0.3pt}%
  \let\item\@idxitem
}{%
  \clearpage
}
\makeatother

\IfFileExists{\jobname-pw.ind}{\input{\jobname-pw.ind}}{}

\end{document}

      