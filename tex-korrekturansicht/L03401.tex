%% latex-korrekturansicht-vorspann.tex
%% Vorspann für die Korrekturansicht.
%% Lädt die gemeinsame Datei latex-vorspann.tex mit gesetztem Schalter.

\newif\ifkorrekturansicht
\korrekturansichttrue

\input{../tex-inputs/latex-vorspann}


\renewcommand{\erwaehntePersonen}{Personen: Siegfried Jacobsohn, Ottilie Salten, Olga Schnitzler}
\renewcommand{\erwaehnteOrte}{Orte: Edmund-Weiß-Gasse 7, Riedhof, Wien, XVIII., Währing}
\renewcommand{\erwaehnteWerke}{Werke: Der Fall Jacobsohn}
\section[ Felix Salten an Arthur Schnitzler, 19. 12. 1904]{Felix Salten an Arthur Schnitzler, 19. 12. 1904}
\nopagebreak\mylabel{v}
\rehead{ }\normalsize\beginnumbering\briefempfaengerindex{Schnitzler, Arthur@\textsc{Schnitzler, Arthur}!zzzSalten, Felix@\emph{von Felix Salten}!1904-12-191@{19. 12. 1904}|(be}
\toendnotes[C]{\smallbreak\pagebreak[2]}\Standort{CUL, Schnitzler, B 89, B 1.}
\physDesc{Kartenbrief, 583 Zeichen
\newline{}Handschrift: schwarze Tinte, lateinische Kurrent
\newline{}Versand: 1) Stempel: »\nobreak{}Wien \textcolor{gray}{5/1 a 66}, 20 12. 04, 6–7 V\nobreak{}«.   2) Stempel: »\nobreak{}\oindex{XVIII., Waehring@\textbf{XVIII., Währing}, \emph{A.ADM3}|pwk}18/1 Wien 110, 20. 12. 04, 12. V, Bestellt\nobreak{}«. 
\newline{}Schnitzler: mit Bleistift datiert: »20/12 904« 
\newline{}Ordnung: mit Bleistift von unbekannter Hand nummeriert: »194« }\toendnotes[C]{\smallbreak}\pstart{}{\pb}Herrn D\textsuperscript{r} Arthur Schnitzler\pend{}\pstart{}\textcolor{pink}{Wien}{}\ledrightnote{\textcolor{pink}{Wien}}\pend{}\pstart{}\textcolor{pink}{XVIII. Spöttelgaße 7}{}\ledrightnote{\textcolor{pink}{Edmund-Weiß-Gasse 7}}\pend{}
{\bigskip}
\pstart
           \raggedleft{}{\pb}Montag.\pend
           
\pstart
           Lieber, wenn es Ihnen recht ist, treffen wir uns morgen (Dienstag) oder
                  Mittwoch{ }Abend (½ 9) im \label{K_L03401-1v}\edtext{\textcolor{pink}{Riedhof}{}\ledrightnote{\textcolor{pink}{Riedhof}}}{\lemma{\textnormal{\emph{Riedhof}}}\Cendnote{\textnormal{Das Treffen fand erst am 23. 12. 1904 statt,
                  nachdem man sich am Vorabend noch verfehlt hatte. An den vorgeschlagenen
                  Feiertagen sahen sie sich nicht.}}}\label{K_L03401-1h}. Da \label{K_L03401-2v}\edtext{\textcolor{blue}{Otti}{}\ledrightnote{\textcolor{blue}{Ottilie Salten}} nur auf 3 Stunden vom Haus fort kann}{\lemma{\textnormal{\emph{Otti … kann}}}\Cendnote{\textnormal{siehe Felix Salten an Arthur Schnitzler, [15. 12. 1904]}}}\label{K_L03401-2h} ist das ein Ausweg. Sonst müßen wirs bis nach den Feiertagen laßen, außer Sie
               könnten \textcolor{blue}{Beide}{}\ledrightnote{\textcolor{blue}{Olga Schnitzler}} am Sonntag od. Montag{ }Abend bei uns sein, was uns sehr freuen würde.\pend
           
\pstart
           Es wäre mir nicht unwichtig bald mit Ihnen zu sprechen, da ich über den \textcolor{green}{\label{K_L03401-3v}\edtext{Artikel}{\lemma{\textnormal{\emph{Artikel}}}\Cendnote{\textnormal{\textcolor{blue}{Arthur Schnitzler}: \emph{Der Fall Jacobsohn}. In: \emph{Die Zukunft}, Jg. 13, Bd. 49, Nr. 12, 17. 12. 1904, S. 401–404. (A. S.: \emph{»Das Zeitlose ist von kürzester Dauer«}, Der Fall Jacobsohn, 17. 12. 1904) }}}\label{K_L03401-3h}}{}\ledrightnote{{$\rightarrow$}\textcolor{green}{Der Fall Jacobsohn}}, den Sie Herrn \textcolor{blue}{Siegfried Jacobsohn}{}\ledrightnote{\textcolor{blue}{Siegfried Jacobsohn}}
               gewidmet haben, \label{K_L03401-4v}\edtext{manches wesentliche zu
                  bemerken}{\lemma{\textnormal{\emph{manches … bemerken}}}\Cendnote{\textnormal{Siehe A. S.: \emph{Tagebuch}, 20. 12. 1904: »Brief
                        \textcolor{blue}{Salten}s, mit Bemerkung, er hätte über
                     meinen Artikel \textcolor{green}{J.}
                     wesentliches zu bemerken, irritirte mich. (Bin zum Journalisten nicht
                     geschaffen!)«}}}\label{K_L03401-4h} hätte.\pend
           
\pstart
           Mit herzlichen Grüßen an Sie \textcolor{blue}{Beide}{}\ledrightnote{{$\rightarrow$}\textcolor{blue}{Olga Schnitzler}} von \textcolor{blue}{Otti}{}\ledrightnote{\textcolor{blue}{Ottilie Salten}} und mir\pend
           \pstart Ihr \spacefill\mbox{Salten}\pend{}
\pstart
           \raggedleft{}\strikeout{\textcolor{gray}{Felix Salten}}\pend
           \endnumbering\briefempfaengerindex{Schnitzler, Arthur@\textsc{Schnitzler, Arthur}!zzzSalten, Felix@\emph{von Felix Salten}!1904-12-191@{19. 12. 1904}|)be}\mylabel{h}  \normalsize

\doendnotes{C}
\bigskip
\vfill

\clearpage

\footnotesize

\lohead{\textsc{register}}

% Definiere theindex-Environment komplett neu ohne reledmac
\makeatletter
\renewenvironment{theindex}{%
  \section*{\indexname}%
  \setlength{\parindent}{0pt}%
  \setlength{\parskip}{0pt plus 0.3pt}%
  \let\item\@idxitem
}{%
  \clearpage
}
\makeatother

\IfFileExists{\jobname-pw.ind}{\input{\jobname-pw.ind}}{}

\end{document}

      