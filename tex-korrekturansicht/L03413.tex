%% latex-korrekturansicht-vorspann.tex
%% Vorspann für die Korrekturansicht.
%% Lädt die gemeinsame Datei latex-vorspann.tex mit gesetztem Schalter.

\newif\ifkorrekturansicht
\korrekturansichttrue

\input{../tex-inputs/latex-vorspann}


\renewcommand{\erwaehntePersonen}{Personen: Otto Brahm, Anna Katharina Rehmann, Ottilie Salten, Paul Salten, Olga Schnitzler, Heinrich Schnitzler}
\renewcommand{\erwaehnteInstitutionen}{Institutionen: Theater des Westens}
\renewcommand{\erwaehnteOrte}{Orte: Berlin, Charlottenburg, Edmund-Weiß-Gasse 7, Hotel Saxonia, Kantstraße, Potsdamer Platz, Wien, XVIII., Währing}
\renewcommand{\erwaehnteWerke}{}
\section[ Felix Salten an Arthur Schnitzler, 29. 1. 1906]{Felix Salten an Arthur Schnitzler, 29. 1. 1906}
\nopagebreak\mylabel{v}
\rehead{ }\normalsize\beginnumbering\briefempfaengerindex{Schnitzler, Arthur@\textsc{Schnitzler, Arthur}!zzzSalten, Felix@\emph{von Felix Salten}!1906-01-294@{29. 1. 1906}|(be}
\toendnotes[C]{\smallbreak\pagebreak[2]}\Standort{CUL, Schnitzler, B 89, B 1.}
\physDesc{Postkarte, 778 Zeichen
\newline{}Handschrift: schwarze Tinte, lateinische Kurrent
\newline{}Versand: Stempel: »\nobreak{}\oindex{Berlin@\textbf{Berlin}, \emph{P.PPLC}|pwk}Berlin, S. W. 68, 29. 1. 06, 2–3 N\nobreak{}«. Stempel: »\nobreak{}\oindex{XVIII., Waehring@\textbf{XVIII., Währing}, \emph{A.ADM3}|pwk}18/1 Wien 110, 30 I 06, X, Bestellt\nobreak{}«.  
\newline{}Ordnung: mit Bleistift von unbekannter Hand nummeriert: »204a« }\toendnotes[C]{\smallbreak}\pstart{}{\pb}Herrn D\textsuperscript{r} Arthur Schnitzler\pend{}\pstart{}\textcolor{pink}{Wien XVIII}{}\ledrightnote{\textcolor{pink}{XVIII., Währing}}\pend{}\pstart{}\textcolor{pink}{Spöttelgasse 7}{}\ledrightnote{\textcolor{pink}{Edmund-Weiß-Gasse 7}}\pend{}
{\bigskip}
\pstart
           \raggedleft{}{\pb}\textcolor{pink}{Berlin}{}\ledrightnote{\textcolor{pink}{Berlin}}, 29. I. 06\pend
           
\pstart
           Lieber, wir sind also \label{K_L03413-1v}\edtext{vorigen Dienstag}{\lemma{\textnormal{\emph{vorigen Dienstag}}}\Cendnote{\textnormal{\textcolor{blue}{Salten} dürfte sich auf den 16. 1. 1906 bezogen haben, denn für den 14. 1. 1906 hatte \textcolor{blue}{Schnitzler} den Abschied in \textcolor{pink}{Wien} festgehalten. Die Formulierung ist jedoch soweit offen,
                  dass es sich auch um den 23. 1. 1906 gehandelt
                  haben könnte.}}}\label{K_L03413-1h} hier angekommen, und schon am Donnerst\textcolor{gray}{a}g habe ich die Geschäfte übernommen. Da bin
               ich denn gleich so tief in Arbeit gerathen, dass ich weiter nichts von \textcolor{pink}{Berlin}{}\ledrightnote{\textcolor{pink}{Berlin}} bemerke. Wir wohnen im »\textcolor{pink}{Saxonia}{}\ledrightnote{\textcolor{pink}{Hotel Saxonia}}«, nahe am \textcolor{pink}{Potsdamer
                  Platz}{}\ledrightnote{\textcolor{pink}{Potsdamer Platz}}, schöne Zimmer aber elende Bedienung. Heute haben wir eine Wohnung gemiethet: \textcolor{pink}{Charlottenburg}{}\ledrightnote{\textcolor{pink}{Charlottenburg}}, \textcolor{pink}{Kantstraße 34}{}\ledrightnote{\textcolor{pink}{Kantstraße}},
               dieselbe Straße, in der das \textcolor{brown}{Theater d. Westens}{}\ledrightnote{\textcolor{brown}{Theater des Westens}}
               ist. Morgen sind wir schon drin. Die Freiwohnung, die
               mir angeboten war, wollte ich nicht beziehen, weil mir vor dem zweimaligen
               Übersiedeln graut. \textcolor{blue}{Otti}{}\ledrightnote{\textcolor{blue}{Ottilie Salten}} u. den \textcolor{blue}{Kindern}{}\ledrightnote{{$\rightarrow$}\textcolor{blue}{Anna Katharina Rehmann}{\newline}{$\rightarrow$}\textcolor{blue}{Paul Salten}} geht es gut.
                  \label{K_L03413-2v}\edtext{Wann kommen Sie}{\lemma{\textnormal{\emph{Wann kommen Sie}}}\Cendnote{\textnormal{\textcolor{blue}{Schnitzler} war zwischen 4. 2. 1906 und 7. 2. 1906 sowie
                  zwischen 18. 2. 1906
                  und 27. 2. 1906 in
                     \textcolor{pink}{Berlin}.}}}\label{K_L03413-2h}? Wir freuen uns schon
               darauf! Wissen Sie, dass \label{K_L03413-3v}\edtext{\textcolor{blue}{Brahm}{}\ledrightnote{\textcolor{blue}{Otto Brahm}} am 5. Feber 50 J. alt}{\lemma{\textnormal{\emph{Brahm … alt}}}\Cendnote{\textnormal{\textcolor{blue}{Schnitzler} war zur Geburtstagsfeier
                  eingeladen, vgl. A. S.: \emph{Tagebuch}, 5. 2. 1906.}}}\label{K_L03413-3h} wird?\pend
           
\pstart
           Viele herzlichste Grüße von uns an Sie \textcolor{blue}{Drei}{}\ledrightnote{\textcolor{blue}{Olga Schnitzler}{\newline}\textcolor{blue}{Heinrich Schnitzler}}{\\[\baselineskip]}Ihr \spacefill\mbox{S.}\pend
           \leftskip=0em{}\endnumbering\briefempfaengerindex{Schnitzler, Arthur@\textsc{Schnitzler, Arthur}!zzzSalten, Felix@\emph{von Felix Salten}!1906-01-294@{29. 1. 1906}|)be}\mylabel{h}  \normalsize

\doendnotes{C}
\bigskip
\vfill

\clearpage

\footnotesize

\lohead{\textsc{register}}

% Definiere theindex-Environment komplett neu ohne reledmac
\makeatletter
\renewenvironment{theindex}{%
  \section*{\indexname}%
  \setlength{\parindent}{0pt}%
  \setlength{\parskip}{0pt plus 0.3pt}%
  \let\item\@idxitem
}{%
  \clearpage
}
\makeatother

\IfFileExists{\jobname-pw.ind}{\input{\jobname-pw.ind}}{}

\end{document}

      