%% latex-korrekturansicht-vorspann.tex
%% Vorspann für die Korrekturansicht.
%% Lädt die gemeinsame Datei latex-vorspann.tex mit gesetztem Schalter.

\newif\ifkorrekturansicht
\korrekturansichttrue

\input{../tex-inputs/latex-vorspann}


\renewcommand{\erwaehntePersonen}{Personen: Richard Beer-Hofmann, Theodore Rottenberg, Olga Schnitzler}
\renewcommand{\erwaehnteInstitutionen}{Institutionen: Houghton Library}
\renewcommand{\erwaehnteOrte}{Orte: Berlin, Dessauer Straße, Frankfurt am Main, Prag, Wien}
\renewcommand{\erwaehnteWerke}{}
\section[ Paul Goldmann an Arthur Schnitzler, 20. 3. {[}1902{]}]{Paul Goldmann an Arthur Schnitzler, 20. 3. {[}1902{]}}
\nopagebreak\mylabel{v}
\rehead{ }\normalsize\beginnumbering\briefempfaengerindex{Schnitzler, Arthur@\textsc{Schnitzler, Arthur}!zzzGoldmann, Paul@\emph{von Paul Goldmann}!1902-03-201@{20. 3. {[}1902{]}}|(be}
\toendnotes[C]{\smallbreak\pagebreak[2]}\Standort{DLA, A:Schnitzler, HS.NZ85.1.3172.}
\physDesc{Brief, 1 Blatt, 2 Seiten
\newline{}Handschrift: blaue Tinte, deutsche Kurrent
\newline{}Schnitzler: mit Bleistift das Jahr »{[}1{]}902« vermerkt }\toendnotes[C]{\smallbreak}
\pstart
           \noindent{}\raggedleft{}{\pb}\textcolor{pink}{\textcolor{gray}{\textbf{DESSAUERSTRASSE 19}}}{}\ledrightnote{\textcolor{pink}{Dessauer Straße}}\pend
           
\pstart
           \textcolor{pink}{Berlin}{}\ledrightnote{\textcolor{pink}{Berlin}}, 20. März\pend
           
\pstart{}Mein lieber Freund,\pend
\pstart
           Deinen letzten, ſo ſehr lieben und intereſſanten Brief, der mich wahrhaft erfreut
               hat, beantworte ich demnächſt. Meine \textcolor{pink}{Frankfurt}{}\ledrightnote{\textcolor{pink}{Frankfurt am Main}}er
                  \textcolor{blue}{Freundin}{}\ledrightnote{\textcolor{blue}{Theodore Rottenberg}} iſt in \textcolor{pink}{Berlin}{}\ledrightnote{\textcolor{pink}{Berlin}} und nimmt alle meine freie Zeit in Anſpruch. Wir
               verleben frohe Tage, aber auch hier miſcht ſich mancherlei Bitterkeit ein.\pend
           
\pstart
           Für heut nur Folgendes: Zu Oſtern möchte ich (ohne
               Urlaub) auf zwei, drei Tage fortreiſen. Nach \textcolor{pink}{Wien}{}\ledrightnote{\textcolor{pink}{Wien}}
               kann ich nicht kommen, weil die Reiſe zu weit iſt und weil ich eben ohne Urlaub
               weggehen will. Aber ich würde, wenn Du Luſt hätteſt, Dich auf halbem Wege zwiſchen
                  \textcolor{pink}{Berlin}{}\ledrightnote{\textcolor{pink}{Berlin}}{ }{\pb}und \textcolor{pink}{Wien}{}\ledrightnote{\textcolor{pink}{Wien}} mit mir
               zu treffen, ſehr gern nach \label{K_L03200-1v}\edtext{\textsc{\textcolor{pink}{Prag}{}\ledrightnote{\textcolor{pink}{Prag}}}}{\lemma{\textnormal{\emph{Prag}}}\Cendnote{\textnormal{\textcolor{blue}{Goldmann} fuhr von Ende März bis Anfang April 1902 nach \textcolor{pink}{Prag}, es kam dabei jedoch zu keinem
                  Zusammentreffen mit \textcolor{blue}{Schnitzler}.}}}\label{K_L03200-1h}
               kommen, das ich noch nicht kenne und das eine intereſſante Stadt ſein ſoll. Ich würde
               mich unendlich freuen, wenn Du es möglich machen könnteſt, die Oſtertage mit mir zu
               verbringen. Bitte, antworte mir umgehend!\pend
           
\pstart
           Viele Grüße an \textsc{\textcolor{blue}{Olga}{}\ledrightnote{\textcolor{blue}{Olga Schnitzler}}} und an Dich! {\\[\baselineskip]}Von Herzen {\\[\baselineskip]}Dein {\\[\baselineskip]}\spacefill\mbox{Paul Goldm}\pend
           \leftskip=0em{}
\pstart
           \noindent{}Auch an \label{K_L03200-2v}\edtext{\textsc{\textcolor{blue}{Richard}{}\ledrightnote{\textcolor{blue}{Richard Beer-Hofmann}}}}{\lemma{\textnormal{\emph{Richard}}}\Cendnote{\textnormal{\textcolor{blue}{Goldmann} schrieb \textcolor{blue}{Beer-Hofmann} noch am selben Tag, vgl. \emph{\textcolor{brown}{Houghton Library}},
                        Harvard (Signatur 825.978)}}}\label{K_L03200-2h} ſchreibe ich.\pend
           \endnumbering\briefempfaengerindex{Schnitzler, Arthur@\textsc{Schnitzler, Arthur}!zzzGoldmann, Paul@\emph{von Paul Goldmann}!1902-03-201@{20. 3. {[}1902{]}}|)be}\mylabel{h}
\begin{anhang}
\end{anhang}\normalsize

\doendnotes{C}
\bigskip
\vfill

\clearpage

\footnotesize

\lohead{\textsc{register}}

% Definiere theindex-Environment komplett neu ohne reledmac
\makeatletter
\renewenvironment{theindex}{%
  \section*{\indexname}%
  \setlength{\parindent}{0pt}%
  \setlength{\parskip}{0pt plus 0.3pt}%
  \let\item\@idxitem
}{%
  \clearpage
}
\makeatother

\IfFileExists{\jobname-pw.ind}{\input{\jobname-pw.ind}}{}

\end{document}

      