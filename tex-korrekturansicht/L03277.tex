%% latex-korrekturansicht-vorspann.tex
%% Vorspann für die Korrekturansicht.
%% Lädt die gemeinsame Datei latex-vorspann.tex mit gesetztem Schalter.

\newif\ifkorrekturansicht
\korrekturansichttrue

\input{../tex-inputs/latex-vorspann}


\renewcommand{\erwaehntePersonen}{Personen: Clemens von Franckenstein}
\renewcommand{\erwaehnteOrte}{Orte: Frankgasse 1, Prater, Wien}
\renewcommand{\erwaehnteWerke}{Werke: Tagebuch}
\section[ Felix Salten an Arthur Schnitzler, 2. 4. 1898]{Felix Salten an Arthur Schnitzler, 2. 4. 1898}
\nopagebreak\mylabel{v}
\rehead{ }\normalsize\beginnumbering\briefempfaengerindex{Schnitzler, Arthur@\textsc{Schnitzler, Arthur}!zzzSalten, Felix@\emph{von Felix Salten}!1898-04-021@{2. 4. 1898}|(be}
\toendnotes[C]{\smallbreak\pagebreak[2]}\Standort{CUL, Schnitzler, B 89, A 2.}
\physDesc{Postkarte, 238 Zeichen
\newline{}Handschrift: Bleistift, lateinische Kurrent
\newline{}Versand: Stempel: »\nobreak{}1/1 Wien 1, 2. 4. 98, 7–8 V\nobreak{}«. Stempel: »\nobreak{}Wien 9/3 72, 2. 4. \textcolor{gray}{9}8, Bestellt\nobreak{}«.  
\newline{}Ordnung: mit Bleistift von unbekannter Hand nummeriert: »100« }\toendnotes[C]{\smallbreak}\pstart{}{\pb}Herrn D\textsuperscript{r} Arthur Schnitzler \pend{}\pstart{}\textcolor{pink}{Wien}{}\ledrightnote{\textcolor{pink}{Wien}}\pend{}\pstart{}\textcolor{pink}{IX. Frankgaße 1}{}\ledrightnote{\textcolor{pink}{Frankgasse 1}}\pend{}
{\bigskip}
\pstart
           \noindent{}{\pb}Nach diesem Regen ist wol nicht
               mehr viel zu sagen. Doch wenn es \label{K_L03277-1v}\edtext{morgen}{\lemma{\textnormal{\emph{morgen}}}\Cendnote{\textnormal{Vermutlich hatten sie einen Radausflug
                  geplant. Im \emph{\textcolor{green}{Tagebuch}} notierte \textcolor{blue}{Schnitzler} für den 3. 4. 1898:
                     »Vorm. Bic. \textcolor{pink}{Prater}.« Womöglich
                  wurde er von \textcolor{blue}{Salten} und \textcolor{blue}{Clemens von Franckenstein} begleitet.}}}\label{K_L03277-1h} nicht sehr
               schön wird, komme ich gegen 3 zu Ihnen, und wir verabreden das
               Nähere.\pend
           
\pstart
           Herzlichst {\\[\baselineskip]}\spacefill\mbox{Salten}\pend
           \leftskip=0em{}
\pstart
           \noindent{}\textcolor{blue}{Frankenstein}{}\ledrightnote{\textcolor{blue}{Clemens von Franckenstein}} fährt event. mit.\pend
           \endnumbering\briefempfaengerindex{Schnitzler, Arthur@\textsc{Schnitzler, Arthur}!zzzSalten, Felix@\emph{von Felix Salten}!1898-04-021@{2. 4. 1898}|)be}\mylabel{h}  \normalsize

\doendnotes{C}
\bigskip
\vfill

\clearpage

\footnotesize

\lohead{\textsc{register}}

% Definiere theindex-Environment komplett neu ohne reledmac
\makeatletter
\renewenvironment{theindex}{%
  \section*{\indexname}%
  \setlength{\parindent}{0pt}%
  \setlength{\parskip}{0pt plus 0.3pt}%
  \let\item\@idxitem
}{%
  \clearpage
}
\makeatother

\IfFileExists{\jobname-pw.ind}{\input{\jobname-pw.ind}}{}

\end{document}

      