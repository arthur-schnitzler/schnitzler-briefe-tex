%% latex-korrekturansicht-vorspann.tex
%% Vorspann für die Korrekturansicht.
%% Lädt die gemeinsame Datei latex-vorspann.tex mit gesetztem Schalter.

\newif\ifkorrekturansicht
\korrekturansichttrue

\input{../tex-inputs/latex-vorspann}


               \section[Richard Beer-Hofmann an Arthur Schnitzler, 23. 9. 1895]{ Richard Beer-Hofmann an Arthur Schnitzler,
               23. 9. 1895}\nopagebreak\mylabel{v}\rehead{ }\normalsize\beginnumbering\briefempfaengerindex{Schnitzler, Arthur@\textsc{Schnitzler, Arthur}!zzzBeer-Hofmann, Richard@\emph{von Richard Beer-Hofmann}!1895-09-232@{23. 9. 1895}|(be} \toendnotes[C]{\smallbreak\pagebreak[2]} \Standort{CUL, Schnitzler, B 8.}
\physDesc{Postkarte
\newline{}Handschrift: Bleistift, lateinische Kurrent\newline{}Versand: Stempel: »\nobreak{}\oindex{IX., Alsergrund@\textbf{IX., Alsergrund}, \emph{Bezirk (A.BZK)}|pwk}Wien 9/3, 26. 9. 95, 8.V, Bestellt\nobreak{}«.  
\newline{}Schnitzler: mit Bleistift nummeriert: »65« }\buchAbdrucke{\weitereDrucke{Arthur Schnitzler, Richard Beer-Hofmann: \emph{Briefwechsel 1891–1931}. Hg. Konstanze Fliedl. Wien, Zürich: \emph{Europaverlag} 1992, S. 84.} }\pstart{}{\pb}\textcolor{gray}{\textbf{A}}n\pend{}\pstart{}D\textsuperscript{r} Arthur Schnitzler\pend{}\pstart{}\textcolor{pink}{Wien}{}\ledrightnote{\textcolor{pink}{Wien}}\pend{}\pstart{}\textcolor{pink}{IX Frankgasse 1}{}\ledrightnote{\textcolor{pink}{Frankgasse}}\pend{}\pstart{}\textcolor{pink}{Austria}{}\ledrightnote{\textcolor{pink}{Österreich}}\pend{}{\bigskip}\pstart
           \raggedleft{}{\pb}\introOben{}Montag{ }abend\introOben{}{ }\textcolor{pink}{Hôtel Gardone, Gardasee}{}\ledrightnote{\textcolor{pink}{Grand Hotel Gardone}}\pend
           \pstart
           Lieber Arthur! Ich denke daß man mir morgen einen Brief von Ihnen
               hieher nachsenden wird. Bis mindestens Donnerstag bleibe ich hier, aber
               adressiren Sie ruhig auch weiterhin hieher es wird nachgeschickt. Wo werde ich bei
                  »\textcolor{green}{Liebelei}{}\ledrightnote{\textcolor{green}{Liebelei. Schauspiel in drei Akten}}« sitzen? Halten Sie mich doch auf dem
               Laufenden was geschieht\pend
           \pstart Herzlichst \spacefill\mbox{Richard}\pend{}\endnumbering\briefempfaengerindex{Schnitzler, Arthur@\textsc{Schnitzler, Arthur}!zzzBeer-Hofmann, Richard@\emph{von Richard Beer-Hofmann}!1895-09-232@{23. 9. 1895}|)be}\mylabel{h}  \normalsize

\doendnotes{C}
\bigskip
\vfill

\clearpage

\footnotesize

\lohead{\textsc{register}}

% Definiere theindex-Environment komplett neu ohne reledmac
\makeatletter
\renewenvironment{theindex}{%
  \section*{\indexname}%
  \setlength{\parindent}{0pt}%
  \setlength{\parskip}{0pt plus 0.3pt}%
  \let\item\@idxitem
}{%
  \clearpage
}
\makeatother

\IfFileExists{\jobname-pw.ind}{\input{\jobname-pw.ind}}{}

\end{document}

      