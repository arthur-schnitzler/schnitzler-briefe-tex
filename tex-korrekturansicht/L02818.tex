%% latex-korrekturansicht-vorspann.tex
%% Vorspann für die Korrekturansicht.
%% Lädt die gemeinsame Datei latex-vorspann.tex mit gesetztem Schalter.

\newif\ifkorrekturansicht
\korrekturansichttrue

\input{../tex-inputs/latex-vorspann}


               \section[ Paul Goldmann an Arthur Schnitzler, 18. 7. {[}1897{]}]{Paul Goldmann an Arthur Schnitzler, 18. 7. {[}1897{]}}\nopagebreak\mylabel{v}\rehead{ }\normalsize\beginnumbering\briefempfaengerindex{Schnitzler, Arthur@\textsc{Schnitzler, Arthur}!zzzGoldmann, Paul@\emph{von Paul Goldmann}!1897-07-182@{18. 7. {[}1897{]}}|(be} \toendnotes[C]{\smallbreak\pagebreak[2]} \Standort{DLA, A:Schnitzler, HS.NZ85.1.3167.}
\physDesc{Brief, 1 Blatt, 1 Seite
\newline{}Handschrift: blaue Tinte, deutsche Kurrent
\newline{}Schnitzler: 1) mit Bleistift das Jahr »97« vermerkt 2) mit rotem Buntstift eine Unterstreichung}\toendnotes[C]{\smallbreak}\pstart
           \noindent{}{\pb}\textcolor{gray}{\textbf{\textbf{\textcolor{brown}{Frankfurter Zeitung}{}\ledrightnote{\textcolor{brown}{Frankfurter Zeitung}}}}}\pend
           \pstart
           \textcolor{gray}{\textbf{(\textcolor{brown}{\begin{otherlanguage}{french}Gazette de Francfort\end{otherlanguage}}{}\ledrightnote{\textcolor{brown}{Frankfurter Zeitung}}).}}\pend
           \pstart
           \textcolor{gray}{\textbf{\textbf{\begin{otherlanguage}{french}Fondateur M.\end{otherlanguage}{ }\textcolor{blue}{L. Sonnemann}{}\ledrightnote{\textcolor{blue}{Leopold Sonnemann}}.}}}\pend
           \pstart
           \begin{otherlanguage}{french}\textcolor{gray}{\textbf{Journal politique, financier,}}\end{otherlanguage}\pend
           \pstart
           \begin{otherlanguage}{french}\textcolor{gray}{\textbf{commercial et littéraire.}}\end{otherlanguage}\pend
           \pstart
           \begin{otherlanguage}{french}\textcolor{gray}{\textbf{\textbf{Paraissant trois fois par jour.}}}\end{otherlanguage}\pend
           \pstart
           \begin{otherlanguage}{french}\textcolor{gray}{\textbf{\textbf{Bureau à \textcolor{pink}{Paris}{}\ledrightnote{\textcolor{pink}{Paris}}}}}\end{otherlanguage}\hfill \textsc{\textcolor{pink}{Paris}{}\ledrightnote{\textcolor{pink}{Paris}}}, 18. Juli.\pend
           \pstart
           \begin{otherlanguage}{french}\textcolor{gray}{\textbf{\textbf{\textcolor{pink}{10 Rue de la Bourse}{}\ledrightnote{\textcolor{pink}{rue de la Bourse}}.}}}\end{otherlanguage}\pend
           \pstart{}Mein lieber Freund,\pend\pstart
           Setzen wir alſo die Sache ſo feſt: Am 11. Auguſt
               muß ich in \textsc{\textcolor{pink}{Bayreuth}{}\ledrightnote{\textcolor{pink}{Bayreuth}}} ſein. Von da fahre ich nach \textsc{\textcolor{pink}{Muenchen}{}\ledrightnote{\textcolor{pink}{München}}} und komme ſo \label{K_L02818-1v}\edtext{zwiſchen 15. u. 20. Auguſt}{\lemma{\textnormal{\emph{zwiſchen … 20. Auguſt}}}\Cendnote{\textnormal{\textcolor{blue}{Goldmann} kam am 19. 8. 1897 in \textcolor{pink}{Bad Ischl} an und blieb bis 30. 8. 1897.}}}\label{K_L02818-1h}
               nach \textsc{\textcolor{pink}{Ischl}{}\ledrightnote{\textcolor{pink}{Bad Ischl}}}. Dort bleibe ich mit \textcolor{blue}{Euch}{}\ledrightnote{→\textcolor{blue}{Richard Beer-Hofmann}} zuſammen, ſolange es geht und ſahre dann über \textsc{\textcolor{pink}{Muenchen}{}\ledrightnote{\textcolor{pink}{München}}} nach \textsc{\textcolor{pink}{Paris}{}\ledrightnote{\textcolor{pink}{Paris}}} zurück. Bitte, laß’ mich umgehend wiſſen, ob Du mit dieſem Programm
               einverſtanden biſt?\pend
           \pstart
           Viele treue Grüße an Dich und \textsc{\textcolor{blue}{Richard}{}\ledrightnote{\textcolor{blue}{Richard Beer-Hofmann}}}!\pend
           \pstart
           Dein {\\[\baselineskip]}\spacefill\mbox{Paul Goldmann}\pend
           \leftskip=0em{}\pstart
           \noindent{}\textsc{\textcolor{blue}{Richard}{}\ledrightnote{\textcolor{blue}{Richard Beer-Hofmann}}} ſoll auch am 11. Auguſt nach \label{K_L02818-3v}\edtext{\textsc{\textcolor{pink}{Bayreuth}{}\ledrightnote{\textcolor{pink}{Bayreuth}}}}{\lemma{\textnormal{\emph{Bayreuth}}}\Cendnote{\textnormal{siehe Paul Goldmann an Arthur Schnitzler, 13. 7. [1897]}}}\label{K_L02818-3h} kommen u. dann mit mir über \textsc{\textcolor{pink}{Muenchen}{}\ledrightnote{\textcolor{pink}{München}}} nach \textsc{\textcolor{pink}{Ischl}{}\ledrightnote{\textcolor{pink}{Bad Ischl}}} zurückfahren.\pend
           \pstart
           \label{T_L02818-1v}\edtext{Muß ich fürchten, den \label{K_L02818-11v}\edtext{\textsc{\textcolor{blue}{Bahr}{}\ledrightnote{\textcolor{blue}{Hermann Bahr}}} in \textsc{\textcolor{pink}{Ischl}{}\ledrightnote{\textcolor{pink}{Bad Ischl}}}}{\lemma{\textnormal{\emph{Bahr in Ischl}}}\Cendnote{\textnormal{\textcolor{blue}{Hermann Bahr} verbrachte seine
                     Sommerfrische 1897 am \textcolor{pink}{Schliersee}. Dass \textcolor{blue}{Goldmann} seine
                     Abneigung vor einem möglichen Zusammentreffen mit \textcolor{blue}{Bahr} wichtig war, findet sich auch in der Nachschrift
                     eines Briefs an \textcolor{blue}{Beer-Hofmann} vom
                        24. 7. {[}1897{]}: »Sorg’ mir nur dafür, daß ich
                        in \textcolor{pink}{\textsc{Ischl}}
                        keinen \textcolor{blue}{Bahr} und keinen \textcolor{blue}{Graf} treffe. Ich will mir nicht meine
                        Ferien durch Beſtialität verderben laſſen.« \emph{Houghton Library}, Harvard
                        (Signatur 825.978)}}}\label{K_L02818-11h} zu treffen.}{\lemma{\textnormal{\emph{Muß … treffen.}}}\Cendnote{\textnormal{seitlich entlang
                     des Mittelfalzes}}}\label{T_L02818-1h}\pend
           \endnumbering\briefempfaengerindex{Schnitzler, Arthur@\textsc{Schnitzler, Arthur}!zzzGoldmann, Paul@\emph{von Paul Goldmann}!1897-07-182@{18. 7. {[}1897{]}}|)be}\mylabel{h}\begin{anhang}\end{anhang}\normalsize

\doendnotes{C}
\bigskip
\vfill

\clearpage

\footnotesize

\lohead{\textsc{register}}

% Definiere theindex-Environment komplett neu ohne reledmac
\makeatletter
\renewenvironment{theindex}{%
  \section*{\indexname}%
  \setlength{\parindent}{0pt}%
  \setlength{\parskip}{0pt plus 0.3pt}%
  \let\item\@idxitem
}{%
  \clearpage
}
\makeatother

\IfFileExists{\jobname-pw.ind}{\input{\jobname-pw.ind}}{}

\end{document}

      