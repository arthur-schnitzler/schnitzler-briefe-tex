%% latex-korrekturansicht-vorspann.tex
%% Vorspann für die Korrekturansicht.
%% Lädt die gemeinsame Datei latex-vorspann.tex mit gesetztem Schalter.

\newif\ifkorrekturansicht
\korrekturansichttrue

\input{../tex-inputs/latex-vorspann}


\renewcommand{\erwaehntePersonen}{Personen: Olga Schnitzler}
\renewcommand{\erwaehnteOrte}{Orte: Berlin, Dessauer Straße, Wien}
\renewcommand{\erwaehnteWerke}{}
\section[ Paul Goldmann an Arthur Schnitzler, 9. 6. {[}1902{]}]{Paul Goldmann an Arthur Schnitzler, 9. 6. {[}1902{]}}
\nopagebreak\mylabel{v}
\rehead{ }\normalsize\beginnumbering\briefempfaengerindex{Schnitzler, Arthur@\textsc{Schnitzler, Arthur}!zzzGoldmann, Paul@\emph{von Paul Goldmann}!1902-06-091@{9. 6. {[}1902{]}}|(be}
\toendnotes[C]{\smallbreak\pagebreak[2]}\Standort{DLA, A:Schnitzler, HS.NZ85.1.3172.}
\physDesc{Brief, 1 Blatt, 3 Seiten
\newline{}Handschrift: blaue Tinte, deutsche Kurrent
\newline{}Schnitzler: mit Bleistift das Jahr »{[}1{]}902« vermerkt }\toendnotes[C]{\smallbreak}
\pstart
           \noindent{}\raggedleft{}{\pb}\textcolor{pink}{\textcolor{gray}{\textbf{DESSAUERSTRASSE 19}}}{}\ledrightnote{\textcolor{pink}{Dessauer Straße}}\pend
           
\pstart
           \textcolor{pink}{Berlin}{}\ledrightnote{\textcolor{pink}{Berlin}}, 9. Juni.\pend
           
\pstart\center{}Mein lieber Freund,\pend
\pstart
           Seit ich \label{K_L03210-1v}\edtext{aus \textcolor{pink}{Wien}{}\ledrightnote{\textcolor{pink}{Wien}} zurück}{\lemma{\textnormal{\emph{aus Wien zurück}}}\Cendnote{\textnormal{\textcolor{blue}{Goldmann}s letzter nachgewiesener Tag in \textcolor{pink}{Wien}
               ist der 25. 5. 1902.}}}\label{K_L03210-1h} bin, will ich Dir
               ſchreiben. Es hätte ein großer Brief werden ſollen, aber aus Mangel an Zeit iſt es
               nicht einmal ein kleiner geworden, und da mir Deine lieben Nachrichten mangeln, ſo
               ſchreibe ich Dir heut nur, um Dich zu fragen, wie es
               Dir geht, was \textsc{\textcolor{blue}{Olga}{}\ledrightnote{\textcolor{blue}{Olga Schnitzler}}} macht, was {\pb}es ſonſt Neues gibt, wie es mit
               Deinen Sommerplänen ſteht, \textsc{etc}. \strikeout{?,}\pend
           
\pstart
           Von mir kann ich nichts mittheilen, als daß ich viel und ſchwer zu arbeiten habe, und
               daß ich mich danach ſehne, ein paar Monate in Ruhe, mehr körperlich als geiſtig
               beſchäftigt, zu leben, was natürlich unmöglich ſein wird. Inzwiſchen habe ich {\pb}nach wie vor die Abſicht, \label{K_L03210-2v}\edtext{zwiſchen 20. und 25. Juli nach \textcolor{pink}{Wien}{}\ledrightnote{\textcolor{pink}{Wien}}}{\lemma{\textnormal{\emph{zwiſchen … Wien}}}\Cendnote{\textnormal{nicht geschehen}}}\label{K_L03210-2h} zu kommen, wo ich
               Dich zu ſehen hoffe.\pend
           
\pstart
           Viele treue Grüße! {\\[\baselineskip]}Dein {\\[\baselineskip]}\spacefill\mbox{Paul Goldmnn}\pend
           \leftskip=0em{}
\pstart
           \noindent{}Viele Grüße an \textsc{\textcolor{blue}{Olga}{}\ledrightnote{\textcolor{blue}{Olga Schnitzler}}}!\pend
           \endnumbering\briefempfaengerindex{Schnitzler, Arthur@\textsc{Schnitzler, Arthur}!zzzGoldmann, Paul@\emph{von Paul Goldmann}!1902-06-091@{9. 6. {[}1902{]}}|)be}\mylabel{h}
\begin{anhang}
\end{anhang}\normalsize

\doendnotes{C}
\bigskip
\vfill

\clearpage

\footnotesize

\lohead{\textsc{register}}

% Definiere theindex-Environment komplett neu ohne reledmac
\makeatletter
\renewenvironment{theindex}{%
  \section*{\indexname}%
  \setlength{\parindent}{0pt}%
  \setlength{\parskip}{0pt plus 0.3pt}%
  \let\item\@idxitem
}{%
  \clearpage
}
\makeatother

\IfFileExists{\jobname-pw.ind}{\input{\jobname-pw.ind}}{}

\end{document}

      