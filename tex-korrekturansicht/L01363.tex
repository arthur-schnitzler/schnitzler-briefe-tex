%% latex-korrekturansicht-vorspann.tex
%% Vorspann für die Korrekturansicht.
%% Lädt die gemeinsame Datei latex-vorspann.tex mit gesetztem Schalter.

\newif\ifkorrekturansicht
\korrekturansichttrue

\input{../tex-inputs/latex-vorspann}


               \section[Arthur Schnitzler an Michael Georg Conrad, 24. 1. 1904]{ Arthur Schnitzler an Michael Georg Conrad, 24. 1. 1904}\nopagebreak\mylabel{v}\rehead{ }\normalsize\beginnumbering\briefempfaengerindex{Conrad, Michael Georg@\textsc{Conrad, Michael Georg}!zzzSchnitzler, Arthur@\emph{von Arthur Schnitzler}!1904-01-241@{24. 1. 1904}|(be} \toendnotes[C]{\smallbreak\pagebreak[2]} \Standort{München, Monacensia, Schnitzler, Arthur A I/2.}
\physDesc{Kartenbrief
\newline{}Handschrift: schwarze Tinte, deutsche Kurrent\newline{}Versand: 1) Stempel: »\nobreak{}Wien, 4–5 N\nobreak{}«.  2) Stempel: »\nobreak{}\oindex{Muenchen@\textbf{München}, \emph{https://www.geonames.org/ontologyP.PPLA}|pwk}München 2. B.Z., 25. Jan. 04, V. 7–8\nobreak{}«. 
\newline{}Conrad: mit Bleistift beschriftet: »Artur
                                 Schnitzler« }\toendnotes[C]{\smallbreak}\pstart{}{\pb}Herrn\pend{}\pstart{}\textsc{M. G. Conrad}\pend{}\pstart{}\textsc{\textcolor{pink}{München}{}\ledrightnote{\textcolor{pink}{München}}}\pend{}\pstart{}\textcolor{pink}{Steinsdorfſtraße 7}{}\ledrightnote{\textcolor{pink}{Steinsdorfstraße}}.\pend{}{\bigskip}\pstart
           \raggedleft{}{\pb}\textcolor{pink}{\textsc{XVIII. Spöttelgasse 7}}{}\ledrightnote{\textcolor{pink}{Edmund-Weiß-Gasse}}{\\}\textcolor{pink}{Wien}{}\ledrightnote{\textcolor{pink}{Wien}}{ }24./1 0\textcolor{gray}{4}\pend
           \pstart
           lieber Herr Conrad, ich habe das \textcolor{green}{Buch}{}\ledrightnote{→\textcolor{green}{Evoë! Ein Schritt zur Lichtung des Seelenlebens}} der Frau \textcolor{blue}{\textsc{Knorr Schmidt}}{}\ledrightnote{\textcolor{blue}{Marie Knorr-Schmidt}} erhalten u mancherlei Blicke hinein gethan – der Arzt in mir regt ſich und
               meint: man dürfe über dergleichen \textsc{Imanationen} nicht
               urtheilen, ehe man mehr über das betreffende Individium erfährt oder es wirklich als
               ganzes kennen lernt. Als »Fall« wäre die Sache vielleicht intereſſant.\pend
           \pstart
           Im übrigen: Können Sie ſich talentloſe \label{K_L01363-1v}\edtext{Geiſter}{\lemma{\textnormal{\emph{Geiſter}}}\Cendnote{\textnormal{Beim folgenden Brief, der
                  als »Aus der Korrespondenz mit unbekannten Autoren« veröffentlicht wurde, könnte
                  es sich um \textcolor{blue}{Schnitzler}s Schreiben an \textcolor{blue}{Marie Knorr-Schmidt} handeln: »Sehr
                        geehrte gnädige Frau!{ / }Ich bin keineswegs befugt, die Frage zu entscheiden, ob es Geister gibt oder
                        nicht. Nicht leugnen will ich indes, daß ich mich einer gewissen vorgefaßten
                        Meinung, wenn auch keiner unbegründeten, schuldig weiß. Sollte es aber
                        Geister geben, so flößen mir diejenigen, welche Ihnen Ihre Gedichte
                        diktieren, durch ihren auffallenden Mangel an Geschmack und Talent nicht
                        genügend Interesse ein, um der Erforschung des Problems vorläufig
                        näherzutreten.{ / }Mit vorzüglicher Hochachtung{ / }Arthur Schnitzler«. (\emph{Aus der
                        Korrespondenz mit unbekannten Autoren}. In: \emph{Literatur und Kritik}, Jg. 12, März 1967,
                     S. 87.)}}}\label{K_L01363-1h} vorſtellen? Oder Geſpenſter, die abgeſtandene Witze
               machen? We{\geminationn} ich mich entſchließen ſollte, an einen Geiſt
               zu glauben, ſo müßte er ſich ſchon die Mühe machen, ein Genie zu ſein. –\pend
           \pstart
           Herzliche Grüße.\hspace*{1.5em}Ihr{\\[\baselineskip]}\spacefill\mbox{ArthurSchnitzler}\pend
           \leftskip=0em{}\endnumbering\briefempfaengerindex{Conrad, Michael Georg@\textsc{Conrad, Michael Georg}!zzzSchnitzler, Arthur@\emph{von Arthur Schnitzler}!1904-01-241@{24. 1. 1904}|)be}\mylabel{h}  \normalsize

\doendnotes{C}
\bigskip
\vfill

\clearpage

\footnotesize

\lohead{\textsc{register}}

% Definiere theindex-Environment komplett neu ohne reledmac
\makeatletter
\renewenvironment{theindex}{%
  \section*{\indexname}%
  \setlength{\parindent}{0pt}%
  \setlength{\parskip}{0pt plus 0.3pt}%
  \let\item\@idxitem
}{%
  \clearpage
}
\makeatother

\IfFileExists{\jobname-pw.ind}{\input{\jobname-pw.ind}}{}

\end{document}

      