%% latex-korrekturansicht-vorspann.tex
%% Vorspann für die Korrekturansicht.
%% Lädt die gemeinsame Datei latex-vorspann.tex mit gesetztem Schalter.

\newif\ifkorrekturansicht
\korrekturansichttrue

\input{../tex-inputs/latex-vorspann}


               \section[Arthur Schnitzler an Richard Beer-Hofmann, {[}zwischen 5. 10. und 14. 11. 1893{]}]{ Arthur Schnitzler an Richard Beer-Hofmann, {[}zwischen 5. 10. und
               14. 11. 1893{]}}\nopagebreak\mylabel{v}\rehead{ }\normalsize\beginnumbering\briefempfaengerindex{Beer-Hofmann, Richard@\textsc{Beer-Hofmann, Richard}!zzzSchnitzler, Arthur@\emph{von Arthur Schnitzler}!1893-10-051@{{[}zwischen 5. 10. und
                  14. 11. 1893{]}}|(be} \toendnotes[C]{\smallbreak\pagebreak[2]} \Standort{YCGL, MSS 31.}
\physDesc{Brief, 1 Blatt (Briefpapier mit Trauerrand), 2 Seiten
\newline{}Handschrift: Bleistift, deutsche Kurrent}\toendnotes[C]{\smallbreak}\pstart{}{\pb}Lieber Richard,\pend\pstart
           bitte ſehr, ſenden Sie durch Ueberbringer dieſes den \label{K_L00268_1v}\edtext{\textcolor{blue}{\textsc{Rosé}}{}\ledrightnote{\textcolor{blue}{Arnold Rosé}}ſitz}{\lemma{\textnormal{\emph{Roséſitz}}}\Cendnote{\textnormal{\textcolor{blue}{Arnold Rosé} 
               war ein beliebter Violinist, dessen Aufführungen \textcolor{blue}{Schnitzler} gerne besuchte. Ein offensichtliches
                  Konzert bietet sich in dem Zeitraum aber nicht an, doch trat \textcolor{blue}{Rosé} mehrmals als Begleitmusiker
                  auf. Möglicherweise handelt es sich aber auch um bei \emph{\textcolor{brown}{Alexander Rosé Concertbureau}} besorgte
               Karten für eine musikalische Vorführung?}}}\label{K_L00268_1h}, den Sie wohl noch bei ſich haben,
                  {\pb}\textsc{\label{K_L00268_2v}\edtext{\textcolor{pink}{Burgring 1}{}\ledrightnote{\textcolor{pink}{Burgring}}}{\lemma{\textnormal{\emph{Burgring 1}}}\Cendnote{\textnormal{Das
                     undatierte Korrespondenzstück ist mit Trauerrand versehen und damit nach dem Tod des 
                        \textcolor{blue}{Vaters} anzusetzen. Da \textcolor{blue}{Schnitzler} für
                        fünf Monate nicht ins Theater ging und am 15. 11. 1893 übersiedelte,
                        lässt sich das mögliche Zeitfenster weiter eingrenzen.}}}\label{K_L00268_2h}}. – (an meinen
               Namen)\pend
           \pstart
           Herzlich{\\[\baselineskip]}Ihr \spacefill\mbox{Arthur.}\pend
           \leftskip=0em{}\pstart
           \noindent{}Seh ich Sie heut Abend? hoffentlich\pend
           \endnumbering\briefempfaengerindex{Beer-Hofmann, Richard@\textsc{Beer-Hofmann, Richard}!zzzSchnitzler, Arthur@\emph{von Arthur Schnitzler}!1893-10-051@{{[}zwischen 5. 10. und
                  14. 11. 1893{]}}|)be}\mylabel{h}  \normalsize

\doendnotes{C}
\bigskip
\vfill

\clearpage

\footnotesize

\lohead{\textsc{register}}

% Definiere theindex-Environment komplett neu ohne reledmac
\makeatletter
\renewenvironment{theindex}{%
  \section*{\indexname}%
  \setlength{\parindent}{0pt}%
  \setlength{\parskip}{0pt plus 0.3pt}%
  \let\item\@idxitem
}{%
  \clearpage
}
\makeatother

\IfFileExists{\jobname-pw.ind}{\input{\jobname-pw.ind}}{}

\end{document}

      