%% latex-korrekturansicht-vorspann.tex
%% Vorspann für die Korrekturansicht.
%% Lädt die gemeinsame Datei latex-vorspann.tex mit gesetztem Schalter.

\newif\ifkorrekturansicht
\korrekturansichttrue

\input{../tex-inputs/latex-vorspann}


\renewcommand{\erwaehntePersonen}{Personen: Paul Goldmann, Gerhart Hauptmann, Olga Schnitzler, Elisabeth Steinrück}
\renewcommand{\erwaehnteOrte}{Orte: Berlin, Dessauer Straße, Hotel Edlacherhof, Wien}
\renewcommand{\erwaehnteWerke}{Werke: Berliner Brief. [»Schluck und Jau« von Gerhart Hauptmann am Deutschen Theater], Berliner Theater. »Einsame Menschen« im Deutschen Theater, Einsame Menschen. Drama, Neue Freie Presse, »Michael Kramer.«}
\section[ Paul Goldmann an Olga Gussmann, 15. 11. {[}1901{]}]{Paul Goldmann an Olga Gussmann, 15. 11. {[}1901{]}}
\nopagebreak\mylabel{v}
\rehead{ }\normalsize\beginnumbering\briefempfaengerindex{Schnitzler, Olga@\textsc{Schnitzler, Olga}!zzzGoldmann, Paul@\emph{von Paul Goldmann}!1901-11-151@{15. 11. {[}1901{]}}|(be}
\toendnotes[C]{\smallbreak\pagebreak[2]}\Standort{DLA, A:Schnitzler, HS.NZ85.1.5247.}
\physDesc{Brief, 1 Blatt, 4 Seiten, 2212 Zeichen
\newline{}Handschrift: blaue Tinte, deutsche Kurrent}\toendnotes[C]{\smallbreak}
\pstart
           \noindent{}\raggedleft{}{\pb}\textcolor{gray}{\textbf{\textcolor{pink}{DESSAUERSTRASSE 19}{}\ledrightnote{\textcolor{pink}{Dessauer Straße}}}}\pend
           
\pstart
           \textcolor{pink}{Berlin}{}\ledrightnote{\textcolor{pink}{Berlin}}, 15. November.\pend
           
\pstart{}Liebes Fräulein \textsc{Olga},\pend
\pstart
           Ich danke Ihnen für Ihren lieben Brief und freue mich, daß Sie und \textsc{\textcolor{blue}{Arthur}{}\ledrightnote{}} ein paar \label{K_L03535-1v}\edtext{frohe und friedliche
                  Tage}{\lemma{\textnormal{\emph{frohe … Tage}}}\Cendnote{\textnormal{\textcolor{blue}{Schnitzler} und \textcolor{blue}{Olga Gussmann} waren erst am Vortag, dem 14. 11. 1901 aus \textcolor{pink}{Payerbach} nach \textcolor{pink}{Wien} zurückgekehrt, wo sie vier Tage verlebt hatten.}}}\label{K_L03535-1h} haben
               verleben können. Ihre Schilderungen ſind ſehr eindrucksvoll, und an Ihren Worten iſt
               ein Schimmer von Glück haften geblieben.\pend
           
\pstart
           Ihr Brief erfordert eine ausführliche Beantwortung, und ſie ſoll Ihnen werden, ſobald
               die Arbeit mir ein wenig Luft läßt.\pend
           
\pstart
           Eines aber muß ich mir gleich von der Seele ſchreiben. Ich danke Ihnen für {\pb}die Offenheit, mit der Sie zu mir über meine
               Feuilletons ſprechen, und werde Ihnen mit derſelben Offenheit antworten. Und da muß
               ich Ihnen ſagen, daß Ihre Äußerungen mich außerordentlich geſchmerzt, – daß ſie mich
               in einem Punkte getroffen haben, \strikeout{\textcolor{gray}{wo}} an dem ich überaus empfindlich bin. Oder, um es etwas weniger ſentimental
               auszudrücken: Ich bin \strikeout{\textcolor{gray}{×}\-\textcolor{gray}{×}\-\textcolor{gray}{×}\-\textcolor{gray}{×}\-\textcolor{gray}{×}} verblüfft, von Ihnen ſo ganz und gar nicht verſtanden zu werden. Ich bin
               verblüfft, daß Sie nicht begreifen, wieviel ehrliche Kunſtbegeiſterung, welch’ heißes
               Wahrheitsſtreben in meinen \label{K_L03535-2v}\edtext{\textcolor{green}{Kritiken}{}\ledrightnote{{$\rightarrow$}\textcolor{green}{Einsame Menschen. Drama}{\newline}{$\rightarrow$}\textcolor{green}{Berliner Brief. [»Schluck und Jau« von Gerhart Hauptmann am Deutschen Theater]}{\newline}{$\rightarrow$}\textcolor{green}{»Michael Kramer.«}}
               über \textsc{\textcolor{blue}{Hauptmann}{}\ledrightnote{\textcolor{blue}{Gerhart Hauptmann}}}}{\lemma{\textnormal{\emph{Kritiken
               über Hauptmann}}}\Cendnote{\textnormal{Der unmittelbare Auslöser der
                  Auseinandersetzung war diese Rezension: \textcolor{blue}{Paul Goldmann}: \emph{\textcolor{green}{Berliner Theater. »Einsame Menschen« im Deutschen Theater}}.
                     In: \emph{\textcolor{green}{Neue Freie Presse}}, Nr. 13.345, 19. 10. 1901, Morgenblatt, S. 1–3. Dabei
                  dürften auch frühere Feuilletons thematisiert worden sein: \textcolor{blue}{Paul Goldmann}: \emph{\textcolor{green}{Berliner Brief}}. In: \emph{\textcolor{green}{Neue Freie Presse}}, Nr. 12.735, 6. 2. 1900, Morgenblatt, S. 1–3. \textcolor{blue}{Paul Goldmann}: \emph{\textcolor{green}{»Michael Kramer.«}}. In: \emph{\textcolor{green}{Neue Freie Presse}}, Nr. 13.055, 28. 12. 1900, Morgenblatt, S. 1–3. Siehe auch Paul Goldmann an Arthur Schnitzler, 9. 11. [1901], 23. 11. [1901] und 29. 11. [1901].}}}\label{K_L03535-2h} ſich {\pb}ausdrückt. Ich bin verblüfft, daß Sie in einem
               Falle, wo Ihre und meine Meinung ſich gegenüberſtehen, nicht einen Augenblick \substVorne{}\textsuperscript{\textcolor{gray}{den} Fall}{\allowbreak}\substDazwischen{}die Frage\substHinten{} in Erwägung ziehen, ob nicht vielleicht Sie im Unrecht ſind, und daß Sie
               ohneweiters eine Auslegung ſich zurechtmachen, die mich (ich kann es nicht anders
               ſagen) in meiner \strikeout{\textcolor{gray}{kr}itiſch} Ehre als Kritiker trifft. Denn ich würde es für
               unehrenhaft halten, wenn ich, wie Sie meinen, in meinem Kampf gegen \textsc{\textcolor{blue}{Hauptmann}{}\ledrightnote{\textcolor{blue}{Gerhart Hauptmann}}} mich auch nur im Mindeſten durch perſönliche Motive leiten ließe. Wenn Sie
               meine Angriffe gegen \textsc{\textcolor{blue}{Hauptmann}{}\ledrightnote{\textcolor{blue}{Gerhart Hauptmann}}} perſönlich {\pb}finden, ſo wiſſen Sie wohl nicht,
               was perſönliche Angriffe ſind. Meine Einwendungen ſind einer abſolut ſachlichen Art;
               und wenn ſie im heftigen Tone vorgebracht werden, ſo kommt dieſer Ton von meinem
               Temperament, – ſo kommt er von der Erbitterung her, die mich erfüllt, einen ſo
               minderwerthigen Geiſt, wie \textsc{\textcolor{blue}{Gerhart Hauptmann}{}\ledrightnote{\textcolor{blue}{Gerhart Hauptmann}}}, zum großen Dichter erhoben zu ſehen. Und daß Sie mir dieſe Erbitterung nicht
               glauben wollen, daß Sie nach perſönlichen Motiven ſuchen, – Sie, eine Freundin, – das
               hat mich verblüfft, das hat mich ſchwer gekränkt{\dotsfive}\pend
           \pstart Grüßen Sie, bitte, \textsc{\textcolor{blue}{Liesl}{}\ledrightnote{\textcolor{blue}{Elisabeth Steinrück}}}; und ſeien Sie ſammt \textsc{\textcolor{blue}{Arthur}{}\ledrightnote{}} herzlichſt gegrüßt von Ihrem \spacefill\mbox{Paul Goldmann}\pend{}\endnumbering\briefempfaengerindex{Schnitzler, Olga@\textsc{Schnitzler, Olga}!zzzGoldmann, Paul@\emph{von Paul Goldmann}!1901-11-151@{15. 11. {[}1901{]}}|)be}\mylabel{h}  \normalsize

\doendnotes{C}
\bigskip
\vfill

\clearpage

\footnotesize

\lohead{\textsc{register}}

% Definiere theindex-Environment komplett neu ohne reledmac
\makeatletter
\renewenvironment{theindex}{%
  \section*{\indexname}%
  \setlength{\parindent}{0pt}%
  \setlength{\parskip}{0pt plus 0.3pt}%
  \let\item\@idxitem
}{%
  \clearpage
}
\makeatother

\IfFileExists{\jobname-pw.ind}{\input{\jobname-pw.ind}}{}

\end{document}

      