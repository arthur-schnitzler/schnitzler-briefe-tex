%% latex-korrekturansicht-vorspann.tex
%% Vorspann für die Korrekturansicht.
%% Lädt die gemeinsame Datei latex-vorspann.tex mit gesetztem Schalter.

\newif\ifkorrekturansicht
\korrekturansichttrue

\input{../tex-inputs/latex-vorspann}


\renewcommand{\erwaehntePersonen}{Personen: Else Berger, Margherita Schlesinger, Franziska Schlesinger, Alfred Schlesinger, Emil Schlesinger}
\renewcommand{\erwaehnteOrte}{Orte: Bad Ischl, Badehotel, Dänemark, Skodsborg}
\renewcommand{\erwaehnteWerke}{}
\section[ Felix Salten u. a. an Arthur Schnitzler, 6. 8. 1896]{Felix Salten u. a. an Arthur Schnitzler, 6. 8. 1896}
\nopagebreak\mylabel{v}
\rehead{ }\normalsize\beginnumbering\briefempfaengerindex{Schnitzler, Arthur@\textsc{Schnitzler, Arthur}!zzzSchlesinger, Alfred@\emph{von Alfred Schlesinger}!1896-08-061@{6. 8. 1896}|(be}\briefempfaengerindex{Schnitzler, Arthur@\textsc{Schnitzler, Arthur}!zzzBerger, Else@\emph{von Else Berger}!1896-08-061@{6. 8. 1896}|(be}\briefempfaengerindex{Schnitzler, Arthur@\textsc{Schnitzler, Arthur}!zzzSchlesinger, Franziska@\emph{von Franziska Schlesinger}!1896-08-061@{6. 8. 1896}|(be}\briefempfaengerindex{Schnitzler, Arthur@\textsc{Schnitzler, Arthur}!zzzSchlesinger, Margherita@\emph{von Margherita Schlesinger}!1896-08-061@{6. 8. 1896}|(be}\briefempfaengerindex{Schnitzler, Arthur@\textsc{Schnitzler, Arthur}!zzzLaurent, M.@\emph{von M. Laurent}!1896-08-061@{6. 8. 1896}|(be}\briefempfaengerindex{Schnitzler, Arthur@\textsc{Schnitzler, Arthur}!zzzEisler, Richard@\emph{von Richard Eisler}!1896-08-061@{6. 8. 1896}|(be}\briefempfaengerindex{Schnitzler, Arthur@\textsc{Schnitzler, Arthur}!zzzFanto, Leonhard@\emph{von Leonhard Fanto}!1896-08-061@{6. 8. 1896}|(be}\briefempfaengerindex{Schnitzler, Arthur@\textsc{Schnitzler, Arthur}!zzzSchlesinger, Julius@\emph{von Julius Schlesinger}!1896-08-061@{6. 8. 1896}|(be}\briefempfaengerindex{Schnitzler, Arthur@\textsc{Schnitzler, Arthur}!zzzSchlesinger, Therese@\emph{von Therese Schlesinger}!1896-08-061@{6. 8. 1896}|(be}\briefempfaengerindex{Schnitzler, Arthur@\textsc{Schnitzler, Arthur}!zzzSalten, Felix@\emph{von Felix Salten}!1896-08-061@{6. 8. 1896}|(be}
\toendnotes[C]{\smallbreak\pagebreak[2]}\Standort{CUL, Schnitzler, B 89, A 1.}
\physDesc{Postkarte, 858 Zeichen
\newline{}Handschrift Felix Salten: schwarze Tinte, lateinische Kurrent
\newline{}Handschrift Margherita Schlesinger: schwarze Tinte
\newline{}Handschrift Leonhard Fanto: schwarze Tinte
\newline{}Handschrift Richard Eisler: schwarze Tinte
\newline{}Handschrift Franziska Schlesinger: schwarze Tinte
\newline{}Handschrift Alfred Schlesinger: schwarze Tinte, lateinische Kurrent
\newline{}Handschrift Else Berger: schwarze Tinte, lateinische Kurrent
\newline{}Handschrift Julius Schlesinger: schwarze Tinte
\newline{}Handschrift M. Laurent: schwarze Tinte
\newline{}Versand: Stempel: »\nobreak{}\oindex{Bad Ischl@\textbf{Bad Ischl}, \emph{P.PPL}|pwk}Ischl, 7. 8. 96, 10–11 V\textcolor{gray}{.}\nobreak{}«.  
\newline{}Ordnung: mit Bleistift von unbekannter Hand nummeriert: »76« }\toendnotes[C]{\smallbreak}\pstart{}{\pb}Herrn D\textsuperscript{r} Arthur Schnitzler\pend{}\pstart{}\textcolor{pink}{Skodsborg}{}\ledrightnote{\textcolor{pink}{Skodsborg}}\pend{}\pstart{}\textcolor{pink}{Dänemark}{}\ledrightnote{\textcolor{pink}{Dänemark}}\pend{}\pstart{}\textcolor{pink}{Badehôtel}{}\ledrightnote{\textcolor{pink}{Badehotel}}\pend{}
{\bigskip}
\pstart
           \raggedleft{}{\pb}\textcolor{pink}{Ischl}{}\ledrightnote{\textcolor{pink}{Bad Ischl}}, \substVorne{}\textsuperscript{2}\substDazwischen{}6\substHinten{}. August 96.\pend
           
\pstart
           Man soupirt nämlich heute{ }Abend bei \textcolor{blue}{Schlesinger}{}\ledrightnote{\textcolor{blue}{Franziska Schlesinger}{\newline}\textcolor{blue}{Emil Schlesinger}}.
               Es war Kalbsbraten da, und über den Weg »Schnitzl« kam ein Toast auf Sie zu stande.
               Die Consequenz dieses lobenden Gefühlsausbruches ist »vorliegende« Karte, welche
               Ihnen Grüße von nachstehenden \label{K_L03177-1v}\edtext{Persönlichkeiten}{\lemma{\textnormal{\emph{Persönlichkeiten}}}\Cendnote{\textnormal{siehe dazu auch
                     Felix Salten an Arthur Schnitzler, 8. 8. 1896}}}\label{K_L03177-1h} übermittelt:\pend
           
\pstart
           \spacefill\mbox{{[}hs. Therese Schlesinger:{]} Therese Schlesinger}{ }\spacefill\mbox{{[}hs. Julius Schlesinger:{]} Julius Schlesinger}{\\}\spacefill\mbox{{[}hs. Fanto:{]} Fanto}{ }\spacefill\mbox{{[}hs. Eisler:{]} D\textsuperscript{r} REisler}{ }\spacefill\mbox{{[}hs. Laurent:{]} M. Laurent}{\\}\spacefill\mbox{{[}hs. Margherita Schlesinger:{]} Gretl Schlesinger,}{ }\spacefill\mbox{{[}hs. Franziska Schlesinger:{]} Fanny Schlesinger}\pend
           
\pstart
           {[}hs. Berger:{]} Trotzdem Herr Salten mir absolut nicht erlauben \introOben{}will\introOben{} mehr als meinen Namen zu schreiben, benütze ich die gute
               Gelegenheit Ihnen viele herzliche Grüße zu senden. Herzlich und freundschaftlich,
               Ihre \spacefill\mbox{Else.}\pend
           
\pstart
           {[}hs. Alfred Schlesinger:{]} Med. Dr. Alfred Schlesinger \introOben{}in
                  spe\introOben{} grüßt den zukünftigen Herrn Collegen bestens nachdem er seine Matura
               glücklich überstanden\pend
           
\pstart
           \noindent{}\label{T_L03177-1v}\edtext{{[}hs. Berger:{]} Fanny fragt warum »nachstehend« nicht unter
                  Anführungszeichen steht. Bitte, erklären \uuline{Sie} ihr
                  das!! \spacefill\mbox{\textcolor{gray}{E}}}{\lemma{\textnormal{\emph{Fanny … E}}}\Cendnote{\textnormal{am linken Rand, quer zum Text}}}\label{T_L03177-1h}\pend
           \endnumbering\briefempfaengerindex{Schnitzler, Arthur@\textsc{Schnitzler, Arthur}!zzzSchlesinger, Alfred@\emph{von Alfred Schlesinger}!1896-08-061@{6. 8. 1896}|)be}\briefempfaengerindex{Schnitzler, Arthur@\textsc{Schnitzler, Arthur}!zzzBerger, Else@\emph{von Else Berger}!1896-08-061@{6. 8. 1896}|)be}\briefempfaengerindex{Schnitzler, Arthur@\textsc{Schnitzler, Arthur}!zzzSchlesinger, Franziska@\emph{von Franziska Schlesinger}!1896-08-061@{6. 8. 1896}|)be}\briefempfaengerindex{Schnitzler, Arthur@\textsc{Schnitzler, Arthur}!zzzSchlesinger, Margherita@\emph{von Margherita Schlesinger}!1896-08-061@{6. 8. 1896}|)be}\briefempfaengerindex{Schnitzler, Arthur@\textsc{Schnitzler, Arthur}!zzzLaurent, M.@\emph{von M. Laurent}!1896-08-061@{6. 8. 1896}|)be}\briefempfaengerindex{Schnitzler, Arthur@\textsc{Schnitzler, Arthur}!zzzEisler, Richard@\emph{von Richard Eisler}!1896-08-061@{6. 8. 1896}|)be}\briefempfaengerindex{Schnitzler, Arthur@\textsc{Schnitzler, Arthur}!zzzFanto, Leonhard@\emph{von Leonhard Fanto}!1896-08-061@{6. 8. 1896}|)be}\briefempfaengerindex{Schnitzler, Arthur@\textsc{Schnitzler, Arthur}!zzzSchlesinger, Julius@\emph{von Julius Schlesinger}!1896-08-061@{6. 8. 1896}|)be}\briefempfaengerindex{Schnitzler, Arthur@\textsc{Schnitzler, Arthur}!zzzSchlesinger, Therese@\emph{von Therese Schlesinger}!1896-08-061@{6. 8. 1896}|)be}\briefempfaengerindex{Schnitzler, Arthur@\textsc{Schnitzler, Arthur}!zzzSalten, Felix@\emph{von Felix Salten}!1896-08-061@{6. 8. 1896}|)be}\mylabel{h}  \normalsize

\doendnotes{C}
\bigskip
\vfill

\clearpage

\footnotesize

\lohead{\textsc{register}}

% Definiere theindex-Environment komplett neu ohne reledmac
\makeatletter
\renewenvironment{theindex}{%
  \section*{\indexname}%
  \setlength{\parindent}{0pt}%
  \setlength{\parskip}{0pt plus 0.3pt}%
  \let\item\@idxitem
}{%
  \clearpage
}
\makeatother

\IfFileExists{\jobname-pw.ind}{\input{\jobname-pw.ind}}{}

\end{document}

      