%% latex-korrekturansicht-vorspann.tex
%% Vorspann für die Korrekturansicht.
%% Lädt die gemeinsame Datei latex-vorspann.tex mit gesetztem Schalter.

\newif\ifkorrekturansicht
\korrekturansichttrue

\input{../tex-inputs/latex-vorspann}


\renewcommand{\erwaehntePersonen}{Personen: Paul Blasel, Max Eugen Burckhard}
\renewcommand{\erwaehnteInstitutionen}{Institutionen: Grillparzer-Gesellschaft, Stadttheater (Teplitz)}
\renewcommand{\erwaehnteOrte}{Orte: Teplice, Wien, Österreichischer Ingenieur- und Architektenverein}
\renewcommand{\erwaehnteWerke}{Werke: Dulfein. Ein Liebesmärchen, In der Schule des Lebens}
\section[ Felix Salten an Arthur Schnitzler, 16. 1. 1897]{Felix Salten an Arthur Schnitzler, 16. 1. 1897}
\nopagebreak\mylabel{v}
\rehead{ }\normalsize\beginnumbering\briefempfaengerindex{Schnitzler, Arthur@\textsc{Schnitzler, Arthur}!zzzSalten, Felix@\emph{von Felix Salten}!1897-01-164@{16. 1. 1897}|(be}
\toendnotes[C]{\smallbreak\pagebreak[2]}\Standort{CUL, Schnitzler, B 89, A 2.}
\physDesc{Brief, 1 Blatt, 2 Seiten, 333 Zeichen
\newline{}Handschrift: Bleistift, lateinische Kurrent
\newline{}Ordnung: mit Bleistift von unbekannter Hand nummeriert: »85« }\toendnotes[C]{\smallbreak}
\pstart
           \raggedleft{}{\pb}\textcolor{pink}{Teplitz}{}\ledrightnote{\textcolor{pink}{Teplice}}, 16/I. 97\pend
           
\pstart
           Lieber Freund!{ }Heute habe ich alles \label{K_L03263-1v}\edtext{eingeleitet}{\lemma{\textnormal{\emph{eingeleitet}}}\Cendnote{\textnormal{\textcolor{blue}{Paul Blasel} hatte zum Jahreswechsel
                  bekanntgegeben, dass er nach zwei Spielzeiten die Leitung des \emph{\textcolor{brown}{Stadttheaters}} in \textcolor{pink}{Teplitz} mit Ablauf der Saison zurückgeben werde. \textcolor{blue}{Salten} bemühte sich um die Nachfolge. Siehe dazu auch Felix Salten an Arthur Schnitzler, 6. 5. 1899 und Felix Salten an Arthur Schnitzler, 1[3]. 5. 1899.}}}\label{K_L03263-1h}. Die Chancen sind meiner Ansicht nach nur gering,
               obwol man mir das Gegentheil zu sagen versucht. Schade, dass Sie sich \label{K_L03263-2v}\edtext{nicht entschließen}{\lemma{\textnormal{\emph{nicht entschließen}}}\Cendnote{\textnormal{Es gibt keine Hinweise, dass sich \textcolor{blue}{Schnitzler} ernsthaft überlegte, mit \textcolor{blue}{Salten} gemeinsam ein Theater zu führen. Überhaupt dürfte
                  sich \textcolor{blue}{Schnitzler} nie wirklich erwogen haben,
                  ein Theater zu leiten.}}}\label{K_L03263-2h} können. \uline{Das} wäre die
               absolute Sicherheit. Die {\pb}\textcolor{pink}{Stadt}{}\ledrightnote{{$\rightarrow$}\textcolor{pink}{Teplice}} ist reizend und billig.
               Das \textcolor{brown}{Theater}{}\ledrightnote{{$\rightarrow$}\textcolor{brown}{Stadttheater (Teplitz)}} prachtvoll.\pend
           
\pstart
           Auf Wiedersehen \label{K_L03263-3v}\edtext{Dienstag}{\lemma{\textnormal{\emph{Dienstag}}}\Cendnote{\textnormal{vermutlich bei der Lesung von \textcolor{blue}{Max Burckhard} im \textcolor{pink}{Österreichischen Ingenieur- und Architektenverein}. \textcolor{blue}{Burckhard} las für Mitglieder der \emph{\textcolor{brown}{Grillparzer-Gesellschaft}} zwei eigene
                  Erzählungen, \emph{\textcolor{green}{In der Schule des Lebens}} und
                     \emph{\textcolor{green}{Dulfein}}. Vgl. A. S.: \emph{Tagebuch}, 19. 1. 1897}}}\label{K_L03263-3h}. {\\[\baselineskip]}Herzlich {\\[\baselineskip]}Ihr {\\[\baselineskip]}\spacefill\mbox{Salten}\pend
           \leftskip=0em{}\endnumbering\briefempfaengerindex{Schnitzler, Arthur@\textsc{Schnitzler, Arthur}!zzzSalten, Felix@\emph{von Felix Salten}!1897-01-164@{16. 1. 1897}|)be}\mylabel{h}  \normalsize

\doendnotes{C}
\bigskip
\vfill

\clearpage

\footnotesize

\lohead{\textsc{register}}

% Definiere theindex-Environment komplett neu ohne reledmac
\makeatletter
\renewenvironment{theindex}{%
  \section*{\indexname}%
  \setlength{\parindent}{0pt}%
  \setlength{\parskip}{0pt plus 0.3pt}%
  \let\item\@idxitem
}{%
  \clearpage
}
\makeatother

\IfFileExists{\jobname-pw.ind}{\input{\jobname-pw.ind}}{}

\end{document}

      