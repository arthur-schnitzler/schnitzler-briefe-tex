%% latex-korrekturansicht-vorspann.tex
%% Vorspann für die Korrekturansicht.
%% Lädt die gemeinsame Datei latex-vorspann.tex mit gesetztem Schalter.

\newif\ifkorrekturansicht
\korrekturansichttrue

\input{../tex-inputs/latex-vorspann}


               \section[Paul Goldmann an Arthur Schnitzler, Paul Goldmann an Arthur Schnitzler, 24. 2. {[}1897{]}]{ Paul Goldmann an Arthur Schnitzler, 24. 2. {[}1897{]}}\nopagebreak\mylabel{v}\rehead{ }\normalsize\beginnumbering\briefempfaengerindex{Schnitzler, Arthur@\textsc{Schnitzler, Arthur}!zzzGoldmann, Paul@\emph{von Paul Goldmann}!1897-02-241@{24. 2. {[}1897{]}}|(be} \toendnotes[C]{\smallbreak\pagebreak[2]} \Standort{DLA, A:Schnitzler, HS.NZ85.1.3167.}
\physDesc{Brief, 1 Blatt, 4 Seiten
\newline{}Handschrift: blaue Tinte, deutsche Kurrent
\newline{}Schnitzler: mit Bleistift das Jahr »97« vermerkt }\toendnotes[C]{\smallbreak}\pstart
           \noindent{}{\pb}\textcolor{gray}{\textbf{\textbf{\textcolor{brown}{Frankfurter Zeitung}{}\ledrightnote{\textcolor{brown}{Frankfurter Zeitung}}}}}\pend
           \pstart
           \textcolor{gray}{\textbf{(\textcolor{brown}{\begin{otherlanguage}{french}Gazette de Francfort\end{otherlanguage}}{}\ledrightnote{\textcolor{brown}{Frankfurter Zeitung}}).}}\pend
           \pstart
           \textcolor{gray}{\textbf{\textbf{\begin{otherlanguage}{french}Fondateur M.\end{otherlanguage}{ }\textcolor{blue}{L. Sonnemann}{}\ledrightnote{\textcolor{blue}{Leopold Sonnemann}}.}}}\pend
           \pstart
           \begin{otherlanguage}{french}\textcolor{gray}{\textbf{Journal politique, financier,}}\end{otherlanguage}\hfill \textsc{\textcolor{pink}{Paris}{}\ledrightnote{\textcolor{pink}{Paris}}}, 24. Februar.\pend
           \pstart
           \begin{otherlanguage}{french}\textcolor{gray}{\textbf{commercial et littéraire.}}\end{otherlanguage}\pend
           \pstart
           \begin{otherlanguage}{french}\textcolor{gray}{\textbf{\textbf{Paraissant trois fois par jour.}}}\end{otherlanguage}\pend
           \pstart
           \begin{otherlanguage}{french}\textcolor{gray}{\textbf{\textbf{Bureau à \textcolor{pink}{Paris}{}\ledrightnote{\textcolor{pink}{Paris}}}}}\end{otherlanguage}\pend
           \pstart
           \begin{otherlanguage}{french}\textcolor{gray}{\textbf{\textbf{\textcolor{pink}{24. Rue Feydeau}{}\ledrightnote{\textcolor{pink}{rue Feydeau}}.}}}\end{otherlanguage}\pend
           \pstart\center{}Mein lieber Freund,\pend\pstart
           Du ſchreibſt mir wohl umgehend ein kurzes Wort über die Art, wie der \textcolor{blue}{Vater}{}\ledrightnote{→\textcolor{blue}{Carl Reinhard}}\strikeout{die} die \label{K_L02804-1v}\edtext{Sache}{\lemma{\textnormal{\emph{Sache}}}\Cendnote{\textnormal{Carl Reinhard wurde am 23. 2. 1897 über \textcolor{blue}{Mizi}s Schwangerschaft informiert. Laut \textcolor{blue}{Schnitzler}s \emph{\textcolor{green}{Tagebuch}} sei er »entsetzt« gewesen, doch machte \textcolor{blue}{Schnitzler} ihm zu seiner Beruhigung vor,
                  seine \textcolor{blue}{Tochter} so bald wie
                  möglich heiraten zu wollen.}}}\label{K_L02804-1h} aufgenommen hat. Hoffentlich bleibts bei der
                  \textcolor{pink}{Pariſ}{}\ledrightnote{\textcolor{pink}{Paris}}er Reiſe. Ich habe mich mit dem Gedanken,
               Dich einige Wochen hier zu haben, bereits ſo vertraut gemacht, daß es mir recht
               ſchmerzlich wäre, darauf zu verzichten. Daß das \textcolor{blue}{Mädel}{}\ledrightnote{→\textcolor{blue}{Marie Reinhard}} ſich ſo brav benimmt, freut mich ſehr; übrigens
               überraſcht mich nichts Günſtiges, \strikeout{\textcolor{gray}{d}} was ich von einer jungen \textcolor{blue}{Dame}{}\ledrightnote{→\textcolor{blue}{Marie Reinhard}} höre, welche zwei Jahre lang Dich geliebt hat und von Dir geliebt
               worden iſt. {\pb}Ich wünſchte nur Du wäreſt aus allen
               dieſen Aufregungen ſchon heraus.\pend
           \pstart
           Ein comfortables und ruhiges \textsc{Hotel} wird natürlich hier \strikeout{raſch} raſch gefunden ſein. Du brauchſt mir nur die
               ungefähre \strikeout{Pres} Preislage mitzutheilen und anzugeben,
               ob Du im Centrum der \textcolor{pink}{Stadt}{}\ledrightnote{→\textcolor{pink}{Paris}}
               wohnen willſt. Jedenfalls möchte ich, daß Du den \textsc{Hotel}-Aufenthalt möglichſt abkürzeſt; die \textcolor{pink}{Pariſ}{}\ledrightnote{\textcolor{pink}{Paris}}er Hotels ſind ungemüthlich, und ſelbſt die comfortablen mangeln des
               Comforts. Die Art, wie Du wohnen willſt, mußt Du Dir aber dann hier ſelbſt ausſuchen.
               Ich werde Dir einige Vorſchläge machen, wage aber nicht, für Dich {\pb}eine Wohnung aufzunehmen. Die Idee der Penſion bei
               einer gut bürgerlichen Familie iſt undurchführbar. Die gut bürgerlichen \textcolor{pink}{fran}{}\ledrightnote{→\textcolor{pink}{Frankreich}}zöſiſchen Familien geben
               keine Penſion. Die Fremden gehen hier in die \textsc{Hotels} mit
               Penſion, die im Style der \textcolor{pink}{engl}{}\ledrightnote{\textcolor{pink}{England}}iſchen \begin{otherlanguage}{english}\textsc{boarding-houses}\end{otherlanguage} ſind. Das möchte ich aber auch nicht rathen, wegen des Schlangenfraßes.
               Das Beſte wäre, daß Du ſowohl wie Deine \textcolor{blue}{Freundin}{}\ledrightnote{→\textcolor{blue}{Marie Reinhard}} je eine kleine möblirte Wohnung in einer der ſtillen
               Seitenſtraßen der \textsc{\textcolor{pink}{Champs Élysées}{}\ledrightnote{\textcolor{pink}{Champs-Élysées}}} nähmet. Eſſen im Reſtaurant, \strikeout{\textcolor{gray}{×}} Mittag vielleicht zu Haufe. {\pb}So ſeid Ihr
               ungeſtört. Die junge \textcolor{blue}{Dame}{}\ledrightnote{→\textcolor{blue}{Marie Reinhard}}
               wird allerdings ſehr allein ſein, aber das liegt vielleicht in ihren Wünſchen. Preis
               einer ſolchen Wohnung: 150 bis 200 \textsc{Francs} monatlich.\pend
           \pstart
           \strikeout{Anf\textcolor{gray}{a}} Ende März bin ich jedenfalls hier. Es iſt noch
               ganz unbeſtimmt, ob ich überhaupt fortgehe.\pend
           \pstart
           Schreib’ mir bald und ſei von Herzen gegrüßt!\pend
           \pstart
           Dein treuer {\\[\baselineskip]}\spacefill\mbox{Paul Goldmnn}\pend
           \leftskip=0em{}\endnumbering\briefempfaengerindex{Schnitzler, Arthur@\textsc{Schnitzler, Arthur}!zzzGoldmann, Paul@\emph{von Paul Goldmann}!1897-02-241@{24. 2. {[}1897{]}}|)be}\mylabel{h}\begin{anhang}\end{anhang}\normalsize

\doendnotes{C}
\bigskip
\vfill

\clearpage

\footnotesize

\lohead{\textsc{register}}

% Definiere theindex-Environment komplett neu ohne reledmac
\makeatletter
\renewenvironment{theindex}{%
  \section*{\indexname}%
  \setlength{\parindent}{0pt}%
  \setlength{\parskip}{0pt plus 0.3pt}%
  \let\item\@idxitem
}{%
  \clearpage
}
\makeatother

\IfFileExists{\jobname-pw.ind}{\input{\jobname-pw.ind}}{}

\end{document}

      