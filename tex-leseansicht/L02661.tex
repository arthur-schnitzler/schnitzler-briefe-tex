%% latex-leseansicht-vorspann.tex
%% Vorspann für die Leseansicht.
%% Lädt die gemeinsame Datei latex-vorspann.tex mit nicht gesetztem Schalter.

\newif\ifkorrekturansicht
\korrekturansichtfalse

\input{../tex-inputs/latex-vorspann}


\section[Paul Goldmann an Arthur Schnitzler, 27. 4. 1891]{L02661 Paul Goldmann an Arthur Schnitzler, 27. 4. 1891}
\nopagebreak\mylabel{L02661v}
\rehead{ }\normalsize\beginnumbering\briefempfaengerindex{Schnitzler, Arthur@\textsc{Schnitzler, Arthur}!zzzGoldmann, Paul@\emph{von Paul Goldmann}!1891-04-271@{27. 4. 1891}|(be}
\toendnotes[C]{\smallbreak\pagebreak[2]}
\correspDesc{Versand  durch Paul Goldmann am 27. 4. 1891 in Frankfurt am Main
\newline{}Erhalt  durch Arthur Schnitzler im Zeitraum [28. 4. 1891
                  – 2. 5. 1891?] in Wien}\toendnotes[C]{\smallbreak}
\Standort{DLA, A:Schnitzler, HS.NZ85.1.3162.}
\physDesc{Brief, 3 Blätter, 10 Seiten, 6883 Zeichen
\newline{}Handschrift: blaue Tinte, deutsche Kurrent
\newline{}Schnitzler: mit rotem Buntstift eine Unterstreichung }\toendnotes[C]{\smallbreak}
\pstart
           {\pb}\textcolor{gray}{\textbf{\textsc{Frankfurter Zeitung}}}\orgindex{Frankfurter Zeitung@Frankfurter Zeitung|pw}\pend
           
\pstart
           \textcolor{gray}{\textbf{\textsc{und}}}\pend
           
\pstart
           \textcolor{gray}{\textbf{\textsc{Handelsblatt.}}}\hfill \textcolor{gray}{\textbf{Frankfurt a. M.\oindex{Frankfurt am Main@\textbf{Frankfurt am Main}, \emph{Hauptstadt}|pw},}}{ }27. April \textcolor{gray}{\textbf{189}}1.\pend
           
\pstart
           \textcolor{gray}{\textbf{\textbf{\textsc{Redaction.}}}}\pend
           
\pstart
           \textcolor{gray}{\textbf{\textbf{\textsc{Telegramm-Adresse:}}}}\pend
           
\pstart
           \textcolor{gray}{\textbf{\textbf{\textsc{Zeitung Frankfurt
                              Main\oindex{Frankfurt am Main@\textbf{Frankfurt am Main}, \emph{Hauptstadt}|pw}.}}}}\pend
           
\pstart\center{}Lieber Freund!\pend\vspace{0.5em}
\pstart
           Die \label{K_L02661-1v}\edtext{Nummer\pwindex{Moderne Rundschau@\emph{Moderne Rundschau}|pwv}}{\lemma{\textnormal{\emph{Nummer}}}\Cendnote{\textnormal{Im zweiten Heft\pwindex{Moderne Rundschau@\emph{Moderne Rundschau}|pwkv} des dritten Bandes vom
                     15. 4. 1891 erschien auf S. 58 Schnitzlers Gedicht \emph{Tagebuchblatt}\pwindex{Schnitzler, Arthur 15.\,5.\,1862 Wien – 21.\,10.\,1931 ebd.@\textsc{Schnitzler, Arthur} (15.\,5.\,1862 Wien – 21.\,10.\,1931 ebd.), \emph{Schriftsteller, Mediziner}!Tagebuchblatt@\strich\emph{Tagebuchblatt}|pwk}.}}}\label{K_L02661-1} der »Modernen
                  Rundſchau\pwindex{Moderne Rundschau@\emph{Moderne Rundschau}|pw}«, die ich{ }ſoeben in die Hand bekomme, hat das Heimweh nach Wien\oindex{Wien@\textbf{Wien}, \emph{Verwaltungsgebiet}|pw} und nach Dir, das einige Tage lang{ }ſtill
               geweſen, mit einem mächtigen Stoß wieder aufgerüttelt. Und jetzt{ }ſitze ich
                  da\textcolor{gray}{,} und{ }ſchaue Dein Gedicht\pwindex{Schnitzler, Arthur 15.\,5.\,1862 Wien – 21.\,10.\,1931 ebd.@\textsc{Schnitzler, Arthur} (15.\,5.\,1862 Wien – 21.\,10.\,1931 ebd.), \emph{Schriftsteller, Mediziner}!Tagebuchblatt@\strich\emph{Tagebuchblatt}|pwv} an, und ich habe das Gefühl, als{ }ſäßen in meinen
               Herzen{ }ſiebenhundert Bohrwürmer.\pend
           
\pstart
           Im Übrigen habe ich in den letzten Tagen verſucht, mich – nach gewohntem Recept – an
               Arbeit zu betrinken. Mit Erfolg. Gelegenheit zur Thätigkeit iſt genug da. Und{ }ſo{ }ſitze ich denn von früh bis Abend im Büreau und komme \strikeout{g} gar nicht zu mir{ }ſelbſt. Politik, Feuilleton, Blätter- und
               Correcturen-Leſen, Briefe{ }ſchreiben und Notizen redigiren – {\pb}das{ }ſind Alles ausgezeichnete Mittel gegen das
               Heimweh. Man bekämpft das Unglück am Beſten, wenn man{ }ſich in die Lage{ }ſetzt, daß man
               keine Zeit hat, unglücklich zu{ }ſein. Anfang Mai{ }ſchon –
               alſo 4 Wochen früher, als anfänglich beſtimmt –{ }ſoll ich nach Brüſſel\oindex{Brüssel@\textbf{Brüssel}, \emph{Hauptstadt}|pw} gehen. Ich habe auf Herrn Sonnemann\pwindex{Sonnemann, Leopold 29.\,10.\,1831 Höchberg – 30.\,10.\,1909 Frankfurt am Main@\textsc{Sonnemann, Leopold} (29.\,10.\,1831 Höchberg – 30.\,10.\,1909 Frankfurt am Main), \emph{Journalist, Herausgeber}|pw}, unſeren Chefredacteur\pwindex{Sonnemann, Leopold 29.\,10.\,1831 Höchberg – 30.\,10.\,1909 Frankfurt am Main@\textsc{Sonnemann, Leopold} (29.\,10.\,1831 Höchberg – 30.\,10.\,1909 Frankfurt am Main), \emph{Journalist, Herausgeber}|pwv}, unerwarteter Weiſe einen nicht ungünſtigen
               Eindruck gemacht; was freilich wenig beſagen will, da dieſer hypernervöſe und
               -impreſſioniſtiſche Herr\pwindex{Sonnemann, Leopold 29.\,10.\,1831 Höchberg – 30.\,10.\,1909 Frankfurt am Main@\textsc{Sonnemann, Leopold} (29.\,10.\,1831 Höchberg – 30.\,10.\,1909 Frankfurt am Main), \emph{Journalist, Herausgeber}|pwv}{ }\textcolor{gray}{ſe}ine Eindrücke täglich ändert. Er hat mir zugeſagt, daß ich in{ }ſpäteſtens zwei Jahren nach Paris\oindex{Paris@\textbf{Paris}, \emph{Hauptstadt}|pw} gehen{ }ſoll,
               wenn ich mich dort (in Brüſſel\oindex{Brüssel@\textbf{Brüssel}, \emph{Hauptstadt}|pw})
                  bewähre\textcolor{gray}{.} Aber erſtens wird{ }ſo eine Zuſage heut gemacht und
               morgen vergeſſen; und dann zweifle ich mehr als je daran, daß ich mich in Brüſſel\oindex{Brüssel@\textbf{Brüssel}, \emph{Hauptstadt}|pw} bewähren {\pb}werde; die »Frankfurter Zeitung\orgindex{Frankfurter Zeitung@Frankfurter Zeitung|pw}« wird wirklich
               im größten Styl geführt und{ }ſtellt ungeheure Anforderungen an die Kunft jedes
               Einzelnen. Aber{ }ſelbſt wenn mir’s glückt, wartet meiner eine Zukunft ohne Hoffnung
               und Ausſicht. Ich habe hier\oindex{Frankfurt am Main@\textbf{Frankfurt am Main}, \emph{Hauptstadt}|pwv},
               wie ich Dir{ }ſchon angedeutet, meine Familienverhältniſſe in ziemlich kritiſchem
               Zuſtande angetroffen. Mein Breslau\oindex{Breslau@\textbf{Breslau}|pw}er Onkel\pwindex{Mamroth, Albert 1888 Breslau – nach 1954@\textsc{Mamroth, Albert} (1888 Breslau – nach 1954)|pwv}, der bisher einen Theil
               der Laſten für den Unterhalt meiner Familie getragen, gedenkt zu heirathen; mein
               hieſiger Onkel\pwindex{Mamroth, Fedor 21.\,2.\,1851 Breslau – 25.\,6.\,1907 Frankfurt am Main@\textsc{Mamroth, Fedor} (21.\,2.\,1851 Breslau – 25.\,6.\,1907 Frankfurt am Main), \emph{Journalist, Kritiker}|pwv}
               wartet auch mit Sehnſucht auf den Moment, wo er die für ihn kaum mehr erträgliche
               Bürde der Mitſorge für die Meinen ablegen kann; meine Mutter\pwindex{Goldmann, Clementine 15.\,5.\,1842 Breslau – 24.\,2.\,1924 Frankfurt am Main@\textsc{Goldmann, Clementine} (15.\,5.\,1842 Breslau – 24.\,2.\,1924 Frankfurt am Main)|pwv} und Schweſter\pwindex{Rosengart, Vally 29.\,12.\,1866 Breslau – nach 1926@\textsc{Rosengart, Vally} (29.\,12.\,1866 Breslau – nach 1926)|pwv}{ }ſehnen{ }ſich unausſprechlich
               danach, mit ihrem Sohn bez. Bruder, der ihre rechtmäßige Stütze iſt, endlich{ }ſich {\pb}zu vereinigen. Und{ }ſo wird mir binnen Kurzem allein
               die Pflicht zufallen, für die Meinen zu{ }ſorgen – womit natürlich das Einſargen aller
               individuellen Pläne und Wünſche für alle Zeit verbunden iſt. Dann heißt es: Geld
               verdienen um jeden Preis, und nichts als Geld verdienen. Alſo auch in dieſer
               Beziehung habe ich in Wien\oindex{Wien@\textbf{Wien}, \emph{Verwaltungsgebiet}|pw} eine Art Paradies
               verloren – jenen Ort\oindex{Wien@\textbf{Wien}, \emph{Verwaltungsgebiet}|pwv} nämlich, wo
               ich – trotz aller Sorgen – doch mein beſſeres Ich{ }ſein durfte. Nun werde ich
               unerbittlich auf die tiefere Stufe des bloßen Arbeitsthieres herabgedrückt{\dotsfive}\pend
           
\pstart
           Soviel von mir. Dein lieber Brief hat mich unendlich gefreut. Es iſt recht{ }ſehr
               freundſchaftlich von Dir, daß Du mich verſicherſt, ich ginge Dir ab; es iſt zwar
               jedenfalls nicht wahr; aber Du weißt, daß es mir wohlthut, und darum iſt es recht{ }ſehr freundſchaftlich, daß Du es mir{ }ſchreibſt.\pend
           
\pstart
           {\pb}\textsc{Pardon} für die Beſchmutzung des vorigen Bogens; ich wollte
               die Sache nicht noch einmal abſchreiben!\pend
           
\pstart
           Alſo weiter: die Geſchichte mit Deinem Dich-Allein-Fühlen verſtehe ich vollauf. Wie
               ich immer{ }ſagte: das Mädel\pwindex{Glümer, Marie 3.\,7.\,1867 Wien – 16.\,11.\,1925 München@\textsc{Glümer, Marie} (3.\,7.\,1867 Wien – 16.\,11.\,1925 München), \emph{Schauspielerin}|pwv} deckt{ }ſich nur mit einer Seite Deines Ich, und nicht
               mit Deiner beſten. Die letztere bleibt ewig unbefriedigt bei Allem; und dieſes
               Alleingefühl iſt nichts als ein Lebenszeichen Deines beſſeren Ich, ein Hunger
               desſelben nach Befriedigung. Thu’ ihm den Gefallen, lieber Arthur; nimm’ Dir eine
               große Aufgabe her und{ }ſtell’ Dich in deren Dienſt,{ }ſei{ }ſie künſtleriſch oder
               wiſſenſchaftlich. Ich habe erſt jetzt wieder den vollen Segen der großen Arbeit
               empfunden. Es iſt ein großer Trieb zur {\pb}Arbeit in
               uns Allen (bei Vielen unbewußt, wie z. B. bei Dir); und wer den \strikeout{\textcolor{gray}{ertödte}n} ertödten will, der hat dieſelben{ }ſchlimmen
               Rückwirkungen zu tragen, wie{ }ſie{ }ſich überhaupt einſtellen, wenn man eine Naturkraft
               in{ }ſich abtödten will. Glaub’ mir und{ }ſolge mir! So wird das Mädel\pwindex{Glümer, Marie 3.\,7.\,1867 Wien – 16.\,11.\,1925 München@\textsc{Glümer, Marie} (3.\,7.\,1867 Wien – 16.\,11.\,1925 München), \emph{Schauspielerin}|pwv} zu dem herabſinken, was{ }ſie in Deinem Leben einzig{ }ſein{ }ſoll und kann: zur \label{K_L02661-2v}\edtext{Epiſode\pwindex{Schnitzler, Arthur 15.\,5.\,1862 Wien – 21.\,10.\,1931 ebd.@\textsc{Schnitzler, Arthur} (15.\,5.\,1862 Wien – 21.\,10.\,1931 ebd.), \emph{Schriftsteller, Mediziner}!Episode@\strich\emph{Episode}|pwv}}{\lemma{\textnormal{\emph{Episode}}}\Cendnote{\textnormal{Hier wohl als eine Anspielung auf den
                  ersten veröffentlichten Einakter\pwindex{Schnitzler, Arthur 15.\,5.\,1862 Wien – 21.\,10.\,1931 ebd.@\textsc{Schnitzler, Arthur} (15.\,5.\,1862 Wien – 21.\,10.\,1931 ebd.), \emph{Schriftsteller, Mediziner}!Episode@\strich\emph{Episode}|pwkv} aus dem \emph{Anatol}\pwindex{Schnitzler, Arthur 15.\,5.\,1862 Wien – 21.\,10.\,1931 ebd.@\textsc{Schnitzler, Arthur} (15.\,5.\,1862 Wien – 21.\,10.\,1931 ebd.), \emph{Schriftsteller, Mediziner}!Anatol@\strich\emph{Anatol}|pwk}-Zyklus zu
                  verstehen. \emph{Episode}\pwindex{Schnitzler, Arthur 15.\,5.\,1862 Wien – 21.\,10.\,1931 ebd.@\textsc{Schnitzler, Arthur} (15.\,5.\,1862 Wien – 21.\,10.\,1931 ebd.), \emph{Schriftsteller, Mediziner}!Episode@\strich\emph{Episode}|pwk} war Mitte September 1889 in der von Goldmann\pwindex{Goldmann, Paul 31.\,1.\,1865 Breslau – 25.\,9.\,1935 Wien@\textsc{Goldmann, Paul} (31.\,1.\,1865 Breslau – 25.\,9.\,1935 Wien), \emph{Schriftsteller, Journalist}|pwk} redigierten Zeitschrift \emph{An der
                     schönen blauen Donau}\pwindex{der schönen blauen Donau@\emph{An der schönen blauen Donau}|pwk} erschienen. }}}\label{K_L02661-2}; und Du wirſt nicht von ihr verlangen, was{ }ſie nimmer gewähren kann: daß{ }ſie Dich als ganzen Menſchen befriedige! Das klingt wie
               Moral, iſt aber nur Vernunft{\dotsfive}\pend
           
\pstart
           Daß Du \label{K_L02661-3v}\edtext{aufgeführt\pwindex{Schnitzler, Arthur 15.\,5.\,1862 Wien – 21.\,10.\,1931 ebd.@\textsc{Schnitzler, Arthur} (15.\,5.\,1862 Wien – 21.\,10.\,1931 ebd.), \emph{Schriftsteller, Mediziner}!Abenteuer seines Lebens. Ein einaktiges Lustspiel@\strich\emph{Das Abenteuer seines Lebens. Ein einaktiges Lustspiel}|pwv}}{\lemma{\textnormal{\emph{aufgeführt}}}\Cendnote{\textnormal{Am 11. 4. 1891 wurde Schnitzlers Einakter \emph{Das Abenteuer seines Lebens}\pwindex{Schnitzler, Arthur 15.\,5.\,1862 Wien – 21.\,10.\,1931 ebd.@\textsc{Schnitzler, Arthur} (15.\,5.\,1862 Wien – 21.\,10.\,1931 ebd.), \emph{Schriftsteller, Mediziner}!Abenteuer seines Lebens. Ein einaktiges Lustspiel@\strich\emph{Das Abenteuer seines Lebens. Ein einaktiges Lustspiel}|pwk} im Volkstheater in Rudolphsheim\oindex{Wien@\textbf{Wien}!XV., Rudolfsheim-Fünfhaus@\textbf{XV., Rudolfsheim-Fünfhaus}!Volkstheater in Rudolfsheim@\textbf{Volkstheater in Rudolfsheim}, \emph{Theater}|pwk} erstmals aufgeführt. Es handelt
                  sich dabei um die erste Aufführung eines Stücks von Schnitzler.}}}\label{K_L02661-3} worden biſt, erfahre ich zum erſten Mal
               aus Deinem Briefe. Ich leſe die Wien\oindex{Wien@\textbf{Wien}, \emph{Verwaltungsgebiet}|pw}er Blätter
               nicht, weil mir die Lectüre zu weh thut. So iſt mir Alles entgangen. Alſo bitte{ }ſehr:{ }ſchreib’ mir Einiges {\pb}über Erfolg und Kritik; wenn
               möglich{ }ſchicke mir eine oder die andere Beſprechung; Du bekommſt{ }ſie bald zurück.
               Jedenfalls herzlichen Glückwunſch zum erſten Schritt vor die Rampe. Ich hätte
               freilich gewünſcht, daß Dich das Burgtheater\orgindex{Burgtheater@Burgtheater|pw} aus
               der Taufe gehoben hätte; immerhin freut es mich, daß man gerade das »Abenteuer{ }ſeines Lebens\pwindex{Schnitzler, Arthur 15.\,5.\,1862 Wien – 21.\,10.\,1931 ebd.@\textsc{Schnitzler, Arthur} (15.\,5.\,1862 Wien – 21.\,10.\,1931 ebd.), \emph{Schriftsteller, Mediziner}!Abenteuer seines Lebens. Ein einaktiges Lustspiel@\strich\emph{Das Abenteuer seines Lebens. Ein einaktiges Lustspiel}|pw}« gewählt hat, welches ich für das
               bühnenwirkſamſte Deiner Stücke halte. Lieber Gott, wie gern wäre ich dabei geweſen!
               Wie hat{ }ſich Dein \label{K_L02661-4v}\edtext{Vater\pwindex{Schnitzler, Johann 10.\,4.\,1835 Nagykanizsa – 2.\,5.\,1893 Wien@\textsc{Schnitzler, Johann} (10.\,4.\,1835 Nagykanizsa – 2.\,5.\,1893 Wien), \emph{Laryngologe}|pwv}}{\lemma{\textnormal{\emph{Vater}}}\Cendnote{\textnormal{Am 14. 5. 1891 notierte Schnitzler in seinem \emph{Tagebuch}\pwindex{Schnitzler, Arthur 15.\,5.\,1862 Wien – 21.\,10.\,1931 ebd.@\textsc{Schnitzler, Arthur} (15.\,5.\,1862 Wien – 21.\,10.\,1931 ebd.), \emph{Schriftsteller, Mediziner}!Tagebuch@\strich\emph{Tagebuch}|pwk}: »Mein Papa\pwindex{Schnitzler, Johann 10.\,4.\,1835 Nagykanizsa – 2.\,5.\,1893 Wien@\textsc{Schnitzler, Johann} (10.\,4.\,1835 Nagykanizsa – 2.\,5.\,1893 Wien), \emph{Laryngologe}|pwv} ist sehr erfreut über den Erfolg.«}}}\label{K_L02661-4}
               zu der Sache verhalten? Wie{ }ſteht’s mit Deinem großen Stück\pwindex{Schnitzler, Arthur 15.\,5.\,1862 Wien – 21.\,10.\,1931 ebd.@\textsc{Schnitzler, Arthur} (15.\,5.\,1862 Wien – 21.\,10.\,1931 ebd.), \emph{Schriftsteller, Mediziner}!Märchen. Schauspiel in drei Aufzügen@\strich\emph{Das Märchen. Schauspiel in drei Aufzügen}|pwv}? Haſt Du etwas
               Pſychologie hinausgeworfen und etwas Action hineingegeben? Und wann bekomme ich den
               dritten Act\pwindex{Schnitzler, Arthur 15.\,5.\,1862 Wien – 21.\,10.\,1931 ebd.@\textsc{Schnitzler, Arthur} (15.\,5.\,1862 Wien – 21.\,10.\,1931 ebd.), \emph{Schriftsteller, Mediziner}!Märchen. Schauspiel in drei Aufzügen@\strich\emph{Das Märchen. Schauspiel in drei Aufzügen}|pwv}? {\dotsfive}\pend
           
\pstart
           Und jetzt im Allgemeinen: wie lebſt Du? Mit wem verkehrſt Du? Kommſt Du in’s \textsc{Griensteidl\oindex{Wien@\textbf{Wien}!I., Innere Stadt@\textbf{I., Innere Stadt}!Café Griensteidl@\textbf{Café Griensteidl}, \emph{Kaffeehaus}|pw}}? Siehſt Du \textsc{Loris\pwindex{Hofmannsthal, Hugo von 1.\,2.\,1874 Wien – 15.\,7.\,1929 Rodaun@\textsc{Hofmannsthal, Hugo von} (1.\,2.\,1874 Wien – 15.\,7.\,1929 Rodaun), \emph{Schriftsteller}|pwv}}, {\pb}\textsc{Beer-Hoffmann\pwindex{Beer-Hofmann, Richard 11.\,7.\,1866 Wien – 26.\,9.\,1945 New York City@\textsc{Beer-Hofmann, Richard} (11.\,7.\,1866 Wien – 26.\,9.\,1945 New York City), \emph{Schriftsteller}|pw}}, die \textsc{Fanjung\pwindex{Van-Jung, Leo 15.\,10.\,1866 Odessa – 2.\,7.\,1939 Riga@\textsc{Van-Jung, Leo} (15.\,10.\,1866 Odessa – 2.\,7.\,1939 Riga), \emph{Gesangspädagoge, Mathematiker}|pwv}\pwindex{Van-Jung, Boris 15.\,10.\,1872 Odessa – 3.\,10.\,1899 Wien@\textsc{Van-Jung, Boris} (15.\,10.\,1872 Odessa – 3.\,10.\,1899 Wien), \emph{Mediziner}|pwv}}’s?\pend
           
\pstart
           Mir gefallen die jungen Naturaliſten ganz und gar nicht mehr. Es wird wieder einmal
               Ereigniß, was für Wien\oindex{Wien@\textbf{Wien}, \emph{Verwaltungsgebiet}|pw}{ }ſo \strikeout{\textcolor{gray}{t}\textcolor{gray}{×}} typiſch iſt: ein paar Streber bemächtigen{ }ſich einer Idee, um daran in die
               Höhe zu klettern. Dieſer \textsc{Joachim\pwindex{Joachim, Jaques 24.\,11.\,1866 Wien – 7.\,11.\,1925 ebd.@\textsc{Joachim, Jaques} (24.\,11.\,1866 Wien – 7.\,11.\,1925 ebd.), \emph{Rechtswissenschaftler, Rechtsanwalt, Herausgeber}|pw}} iſt – unter uns geſagt – nur ein gewöhnlicher \label{K_L02661-5v}\edtext{\textsc{\begin{otherlanguage}{french}Faiseur\end{otherlanguage}}}{\lemma{\textnormal{\emph{Faiseur}}}\Cendnote{\textnormal{französisch: Prahler}}}\label{K_L02661-5}; ich habe
               hier mancherlei gehört, was mir{ }ſehr den Geſchmack an ihm verdorben hat.\pend
           
\pstart
           \textsc{Hildegard\pwindex{Mitis, Hilda von 30.\,8.\,1876 Wien – 14.\,12.\,1894 Bratislava@\textsc{Mitis, Hilda von} (30.\,8.\,1876 Wien – 14.\,12.\,1894 Bratislava), \emph{Schriftstellerin, Telefonistin}|pwv}} hat mir zweimal geſchrieben – \strikeout{ſie ha} ich habe
               ihr keinmal geantwortet. Im zweiten Briefe kündigt{ }ſie mir noch einen dritten an –
               dann keinen mehr,{ }ſie{ }ſei gewohnt, nur dreimal zu bitten. Ich habe einen Haß gegen
               dieſes Weib\pwindex{Mitis, Hilda von 30.\,8.\,1876 Wien – 14.\,12.\,1894 Bratislava@\textsc{Mitis, Hilda von} (30.\,8.\,1876 Wien – 14.\,12.\,1894 Bratislava), \emph{Schriftstellerin, Telefonistin}|pwv} und einen
               unüberwindlichen Widerwillen (Fleißaufgabe für junge Pſychologen, das zu erklären).
                  {\pb}Vielleicht iſt es ihre Verlogenheit, ihre
               Empfindungsloſigkeit mir gegenüber, die{ }ſich hinter{ }ſchönen Briefen verbirgt. Ich
               haſſe{ }ſie{ }ſeit dem unverſchämt gut{ }ſ\textcolor{gray}{t}yliſirten Abſchiedsbrief, den{ }ſie mir geſchrieben. Vielleicht iſt es auch meine {\dotsfour} hm, hm
                  {\dotsfour} Kurzum,{ }ſie iſt mir zuwider, und ich werde{ }ſie
               wahrſcheinlich dreimal vergeblich bitten laſſen. Sie{ }ſchrieb auch davon, daß{ }ſie{ }ſich
               mit Dir in Verbindung{ }ſetzen wolle, wenn »die Sehnſucht nach \strikeout{Di\textcolor{gray}{r} g\textcolor{gray}{ar}} mir gar zu groß werde«. Du erinnerſt Dich wohl, was Du mir diesbezüglich
               verſprochen haſt? {\dotsfour}\pend
           
\pstart
           Und nun{ }ſei vielmals gegrüßt, mein Alter! Laß’ es Dir wohl{ }ſein im lieben, lieben,
               lieben Wien\oindex{Wien@\textbf{Wien}, \emph{Verwaltungsgebiet}|pw}! Quäl’ {\pb}Dich nicht{ }ſo{ }ſehr mit Deiner verfluchten
               Pſychologie und{ }ſei{ }ſubjectiv{ }ſo glücklich, als Du es objectiv biſt.\pend
           
\pstart
           Vor meiner Reiſe nach Brüſſe{[}l{]}\oindex{Brüssel@\textbf{Brüssel}, \emph{Hauptstadt}|pw} höre ich wohl noch etwas von Dir? Das müßte freilich bald{ }ſein.\pend
           
\pstart
           Dein treuer {\\[\baselineskip]}\spacefill\mbox{Paul Goldmann.}\pend
           \leftskip=0em{}
\pstart
           \noindent{}Empfiehl’ mich den Deinen, und grüße \textsc{Kapper\pwindex{Kapper, Friedrich 21.\,4.\,1861 Wien – 22.\,7.\,1939 ebd.@\textsc{Kapper, Friedrich} (21.\,4.\,1861 Wien – 22.\,7.\,1939 ebd.), \emph{Mediziner}|pw}} und \textsc{Loris\pwindex{Hofmannsthal, Hugo von 1.\,2.\,1874 Wien – 15.\,7.\,1929 Rodaun@\textsc{Hofmannsthal, Hugo von} (1.\,2.\,1874 Wien – 15.\,7.\,1929 Rodaun), \emph{Schriftsteller}|pw}}, aber \uline{nicht}{ }\textsc{Beer-Hoffmann\pwindex{Beer-Hofmann, Richard 11.\,7.\,1866 Wien – 26.\,9.\,1945 New York City@\textsc{Beer-Hofmann, Richard} (11.\,7.\,1866 Wien – 26.\,9.\,1945 New York City), \emph{Schriftsteller}|pw}}, weil mir der Schurke\pwindex{Beer-Hofmann, Richard 11.\,7.\,1866 Wien – 26.\,9.\,1945 New York City@\textsc{Beer-Hofmann, Richard} (11.\,7.\,1866 Wien – 26.\,9.\,1945 New York City), \emph{Schriftsteller}|pwv} nicht{ }ſchreibt. Wie macht{ }ſich \textsc{Hirschfeld\pwindex{Hirschfeld, Robert 17.\,9.\,1857 Žďár nad Sázavou – 2.\,4.\,1914 Salzburg@\textsc{Hirschfeld, Robert} (17.\,9.\,1857 Žďár nad Sázavou – 2.\,4.\,1914 Salzburg), \emph{Journalist, Musikkritiker}|pw}} in der \textcolor{gray}{»}Sonn- und Montagszeitung\orgindex{Wiener Sonn- und Montagszeitung@Wiener Sonn- und Montagszeitung|pw}{[}«{]}?\pend
           \selectlanguage{ngerman}\endnumbering\briefempfaengerindex{Schnitzler, Arthur@\textsc{Schnitzler, Arthur}!zzzGoldmann, Paul@\emph{von Paul Goldmann}!1891-04-271@{27. 4. 1891}|)be}\mylabel{L02661h}  \newcommand{\dateiname}{L02661}\newcommand{\titel}{Paul Goldmann an Arthur Schnitzler, 27. 4. 1891}\newcommand{\editorInnen}{Martin Anton Müller und Laura Untner}%% latex-leseansicht-abspann.tex
%% Abspann für die Leseansicht.
%% Der Schalter \ifkorrekturansicht ist bereits durch den Vorspann gesetzt.

%% latex-abspann.tex
%% Gemeinsamer Abspann für Korrekturansicht und Leseansicht.
%% Setzt den Schalter \ifkorrekturansicht voraus (gesetzt in den
%% einbindenden Dateien latex-korrekturansicht-abspann.tex bzw.
%% latex-leseansicht-abspann.tex).
%% ---------------------------------------------------------------

\normalsize

% Das esempio-Environment wird nur in der Leseansicht benötigt
\ifkorrekturansicht\else
\newenvironment{esempio}[3]%
{
    \vspace{1.5ex}
    \rlap{\underline{#1}}
    \par
    \setlength{\parindent}{0cm}
    \nopagebreak
    \leftskip=#2cm
    \rightskip=#3cm
}
{
    \par
}
\fi

\doendnotes{C}
\bigskip
\vfill

\clearpage

\footnotesize

\ifkorrekturansicht
  \lohead{\textsc{register}}
\fi

% theindex-Environment neu definieren ohne reledmac
\makeatletter
\renewenvironment{theindex}{%
  \ifkorrekturansicht
    \section*{\indexname}%
  \else
    \subsubsection*{Index der erwähnten Entitäten}%
  \fi
  \setlength{\parindent}{0pt}%
  \setlength{\parskip}{0pt plus 0.3pt}%
  \let\item\@idxitem
}{%
  \ifkorrekturansicht\clearpage\fi
}
\makeatother

\IfFileExists{\jobname-pw.ind}{\input{\jobname-pw.ind}}{}

% Quellenangabe nur in der Leseansicht
\ifkorrekturansicht\else
% Fallback-Definitionen, falls die .tex-Datei \titel etc. nicht gesetzt hat
\providecommand{\titel}{}
\providecommand{\editorInnen}{}
\providecommand{\dateiname}{\jobname}

\vspace{3cm}

\vfill

\footnotesize
\textsc{Quelle}: \titel. Herausgegeben von {\editorInnen}. In: \emph{Arthur Schnitzler: Briefwechsel mit Autorinnen und Autoren}.
 Digitale Edition, https://schnitzler-briefe.acdh.oeaw.ac.at/{\dateiname}.html (Stand \today)
\fi

\end{document}


