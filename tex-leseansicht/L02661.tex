\input{../tex-inputs/latex-pdf-vorspann}
\begin{center}
            \textcolor{red}{ENTWURF. ENTZIFFERUNG NOCH NICHT KORREKTURGELESEN}
                      \end{center}
            
               \section[Paul Goldmann an Arthur Schnitzler, 27. 4. 1891]{ Paul Goldmann an Arthur Schnitzler, 27. 4. 1891}\nopagebreak\mylabel{v}\rehead{ }\begin{ledgroupsized}[t]{13cm}\normalsize\beginnumbering\briefempfaengerindex{Schnitzler, Arthur@\textsc{Schnitzler, Arthur}!zzzGoldmann, Paul@\emph{von Paul Goldmann}!1891-04-271@{27. 4. 1891}|(be} \toendnotes[C]{\smallbreak\pagebreak[2]} \Standort{DLA, A:Schnitzler, HS.NZ85.1.3162.}
\physDesc{Brief, 3 Blätter, 10 Seiten
\newline{}Handschrift: blaue Tinte, deutsche Kurrent
\newline{}Schnitzler: mit rotem Buntstift eine Unterstreichung }\toendnotes[C]{\smallbreak}\pstart
           \noindent{}{\pb}\textcolor{gray}{\textbf{\textsc{Frankfurter Zeitung}}}\orgindex{Frankfurter Zeitung@Frankfurter Zeitung|pw}\pend
           \pstart
           \textcolor{gray}{\textbf{\textsc{und}}}\pend
           \pstart
           \textcolor{gray}{\textbf{\textsc{Handelsblatt.}}}\hfill \textcolor{gray}{\textbf{Frankfurt a. M.\oindex{Frankfurt am Main@\textbf{Frankfurt am Main}|pw}, }}27. April \textcolor{gray}{\textbf{189}}1.\pend
           \pstart
           \textcolor{gray}{\textbf{\textbf{\textsc{Redaction.}}}}\pend
           \pstart
           \textcolor{gray}{\textbf{\textbf{\textsc{Telegramm-Adresse:}}}}\pend
           \pstart
           \textcolor{gray}{\textbf{\textbf{\textsc{Zeitung Frankfurt
                              Main\oindex{Frankfurt am Main@\textbf{Frankfurt am Main}|pw}.}}}}\pend
           \pstart\center{}Lieber Freund!\pend\pstart
           Die \label{K_L02661-1v}\edtext{Nummer}{\lemma{\textnormal{\emph{Nummer}}}\Cendnote{\textnormal{Im zweiten Heft des dritten Bandes vom
                     15. 4. 1891 erschien auf S. 58 Schnitzler\pwindex{Schnitzler, Arthur 15.05.1862 – 21.10.1931@\textsc{Schnitzler, Arthur} (15.05.1862 – 21.10.1931), \emph{Schriftsteller, Mediziner}|pwk}s Gedicht \emph{Tagebuchblatt}\pwindex{Schnitzler, Arthur 15.05.1862 – 21.10.1931@\textsc{Schnitzler, Arthur} (15.05.1862 – 21.10.1931), \emph{Schriftsteller, Mediziner}!Tagebuchblatt15.04.1891 – 15.04.1891@\strich\emph{Tagebuchblatt} {[}15.04.1891 – 15.04.1891{]}|pwk}.}}}\label{K_L02661-1h} der »Modernen
                  Rundſchau\pwindex{Moderne Rundschau1.4.1891 – 31.12.1891@\emph{Moderne Rundschau}|pw}«, die ich ſoeben in die Hand bekomme, hat das Heimweh nach Wien\oindex{Wien@\textbf{Wien}|pw} und nach Dir, das einige Tage lang ſtill
               geweſen, mit einem mächtigen Stoß wieder aufgerüttelt. Und jetzt ſitze ich da und
               ſchaue Dein Gedicht\pwindex{Schnitzler, Arthur 15.05.1862 – 21.10.1931@\textsc{Schnitzler, Arthur} (15.05.1862 – 21.10.1931), \emph{Schriftsteller, Mediziner}!Tagebuchblatt15.04.1891 – 15.04.1891@\strich\emph{Tagebuchblatt} {[}15.04.1891 – 15.04.1891{]}|pwv} an, und
               ich habe das Gefühl, als ſäßen in meinen Herzen \label{T_L02661-1v}\edtext{ſieben\textcolor{gray}{h}undert}{\lemma{\textnormal{\emph{ſiebenhundert}}}\Cendnote{\textnormal{Goldmann\pwindex{Goldmann, Paul 31.01.1865 – 25.09.1935@\textsc{Goldmann, Paul} (31.01.1865 – 25.09.1935), \emph{Schriftsteller, Journalist}|pwk} machte die Unterlänge nicht
                  fertig, weswegen es sich auch um ein »f« handeln könnte.}}}\label{T_L02661-1h} Bohrwürmer.\pend
           \pstart
           Im Übrigen habe ich in den letzten Tagen verſucht, mich – nach gewohntem Recept – an
               Arbeit zu betrinken. Mit Erfolg. Gelegenheit zur Thätigkeit iſt genug da. Und ſo
               ſitze ich denn von früh bis Abend im Büreau und komme \strikeout{g} gar nicht zu mir ſelbſt. Politik, Feuilleton, Blätter- und
               Correcturen-Leſen, Briefe ſchreiben und Notizen redigiren – {\pb}das ſind Alles ausgezeichnete Mittel gegen das
               Heimweh. Man bekämpft das Unglück am Beſten, wenn man ſich in die Lage ſetzt, daß man
               keine Zeit hat, unglücklich zu ſein. Anfang Mai ſchon –
               alſo 4 Wochen früher, als anfänglich beſtimmt – ſoll ich nach Brüſſel\oindex{Bruessel@\textbf{Brüssel}|pw} gehen. Ich habe auf Herrn Sonnemann\pwindex{Sonnemann, Leopold 1831-10-29 – 1909-10-30@\textsc{Sonnemann, Leopold} (1831-10-29 – 1909-10-30), \emph{Journalist, Herausgeber}|pw}, unſeren Chefredacteur\pwindex{Sonnemann, Leopold 1831-10-29 – 1909-10-30@\textsc{Sonnemann, Leopold} (1831-10-29 – 1909-10-30), \emph{Journalist, Herausgeber}|pwv}, unerwarteter Weiſe einen nicht ungünſtigen
               Eindruck gemacht; was freilich wenig beſagen will, da dieſer hypernervöſe und
               -impreſſioniſtiſche Herr\pwindex{Sonnemann, Leopold 1831-10-29 – 1909-10-30@\textsc{Sonnemann, Leopold} (1831-10-29 – 1909-10-30), \emph{Journalist, Herausgeber}|pwv}
               ſeine Eindrücke täglich ändert. Er hat mir zugeſagt, daß ich in ſpäteſtens zwei
               Jahren nach Paris\oindex{Paris@\textbf{Paris}|pw} gehen ſoll, wenn ich mich dort
               (in Brüſſel\oindex{Bruessel@\textbf{Brüssel}|pw}) bewähre. Aber erſtens wird ſo eine
               Zuſage heut gemacht und morgen vergeſſen; und dann zweifle ich mehr als je daran, daß
               ich mich in Brüſſel\oindex{Bruessel@\textbf{Brüssel}|pw} bewähren {\pb}werde; die »Frankfurter Zeitung\orgindex{Frankfurter Zeitung@Frankfurter Zeitung|pw}« wird wirklich im größten Styl geführt und ſtellt
               ungeheure Anforderungen an die Kunft jedes Einzelnen. Aber ſelbſt wenn mir’s glückt,
               wartet meiner eine Zukunft ohne Hoffnung und Ausſicht. Ich habe hier, wie ich Dir
               ſchon angedeutet, meine Familienverhältniſſe in ziemlich kritiſchem Zuſtande
               angetroffen. Mein Breslau\oindex{Breslau@\textbf{Breslau}|pw}er Onkel\pwindex{Mamroth, Albert 1888 – nach 1954@\textsc{Mamroth, Albert} (1888 – nach 1954)|pwv}, der bisher einen Theil der Laſten
               für den Unterhalt meiner Familie getragen, gedenkt zu heirathen; mein hieſiger Onkel\pwindex{Mamroth, Fedor 21.02.1851 – 25.06.1907@\textsc{Mamroth, Fedor} (21.02.1851 – 25.06.1907), \emph{Journalist, Kritiker}|pwv} wartet
               auch mit Sehnſucht auf den Moment, wo er die für ihn kaum mehr erträgliche Bürde der
               Mitſorge für die Meinen ablegen kann; meine Mutter\pwindex{Goldmann, Clementine 1842-05-15 – 1924-02-24@\textsc{Goldmann, Clementine} (1842-05-15 – 1924-02-24)|pwv} und Schweſter\pwindex{Rosengart, Vally *~1866-12-29@\textsc{Rosengart, Vally} (*~1866-12-29)|pwv} ſehnen ſich unausſprechlich danach, mit ihrem Sohn
               bez. Bruder, der ihre rechtmäßige Stütze iſt, endlich ſich {\pb}zu vereinigen. Und ſo wird mir binnen Kurzem allein
               die Pflicht zufallen, für die Meinen zu ſorgen – womit natürlich das Einſargen aller
               individuellen Pläne und Wünſche für alle Zeit verbunden iſt. Dann heißt es: Geld
               verdienen um jeden Preis, und nichts als Geld verdienen. Alſo auch in dieſer
               Beziehung habe ich in Wien\oindex{Wien@\textbf{Wien}|pw} eine Art Paradies\oindex{Wien@\textbf{Wien}|pwv} verloren – jenen Ort\oindex{Wien@\textbf{Wien}|pwv} nämlich, wo ich – trotz aller
               Sorgen – doch mein beſſeres Ich ſein durfte. Nun werde ich unerbittlich auf die
               tiefere Stufe des bloßen Arbeitsthieres herabgedrückt{\dotsfive}\pend
           \pstart
           Soviel von mir. Dein lieber Brief hat mich unendlich gefreut. Es iſt echt ſehr
               freundſchaftlich von Dir, daß Du mich verſicherſt, ich ginge Dir ab; es iſt zwar
               jedenfalls nicht wahr; aber Du weißt, daß es mir wohlthut, und darum iſt es echt ſehr
               freundſchaftlich, daß Du es mir ſchreibſt.\pend
           \pstart
           {\pb}\textsc{Pardon} für die Beſchmutzung des vorigen Bogens; ich wollte
               die Sache nicht noch einmal abſchreiben!\pend
           \pstart
           Alſo weiter: die Geſchichte mit Deinem Dich-Allein-Fühlen verſtehe ich vollauf. Wie
               ich immer ſagte: das Mädel\pwindex{Gluemer, Marie 03.07.1867 – 16.11.1925@\textsc{Glümer, Marie} (03.07.1867 – 16.11.1925), \emph{Schauspielerin}|pwv} deckt ſich nur mit einer Seite Deines Ich, und nicht
               mit Deiner beſten. Die letztere bleibt ewig unbefriedigt bei Allem; und dieſes
               Alleingefühl iſt nichts als ein Lebenszeichen Deines beſſeren Ich, ein Hunger
               desſelben nach Befriedigung. Thu’ ihm den Gefallen, lieber Arthur; nimm’ Dir eine
               große Aufgabe her und ſtell’ Dich in deren Dienſt, ſei ſie künſtleriſch oder
               wiſſenſchaftlich. Ich habe erſt jetzt wieder den vollen Segen der großen Arbeit
               empfunden. Es iſt ein großer Trieb zur {\pb}Arbeit in
               uns Allen (bei Vielen unbewußt, wie z. B. bei Dir); und wer den \strikeout{ertödten} ertödten will, der hat dieſelben ſchlimmen
               Rückwirkungen zu tragen, wie ſie ſich überhaupt einſtellen, wenn man eine Naturkraft
               in ſich abtödten will. Glaub’ mir und ſolge mir! So wird das Mädel\pwindex{Gluemer, Marie 03.07.1867 – 16.11.1925@\textsc{Glümer, Marie} (03.07.1867 – 16.11.1925), \emph{Schauspielerin}|pwv} zu dem herabſinken, was
               ſie in Deinem Leben einzig ſein ſoll und kann: zur \label{K_L02661-111v}\edtext{Epiſode\pwindex{Schnitzler, Arthur 15.05.1862 – 21.10.1931@\textsc{Schnitzler, Arthur} (15.05.1862 – 21.10.1931), \emph{Schriftsteller, Mediziner}!Episode8. 09. 1889@\strich\emph{Episode} {[}8. 09. 1889{]}|pwv}}{\lemma{\textnormal{\emph{Epiſode}}}\Cendnote{\textnormal{Hier wohl als eine Anspielung auf den
                  ersten veröffentlichte Einakter aus dem \emph{Anatol}\pwindex{Schnitzler, Arthur 15.05.1862 – 21.10.1931@\textsc{Schnitzler, Arthur} (15.05.1862 – 21.10.1931), \emph{Schriftsteller, Mediziner}!Anatol1892-10-29 – 1892-10-29@\strich\emph{Anatol} {[}1892-10-29 – 1892-10-29{]}|pwk}-Zyklus zu verstehen. \emph{Episode}\pwindex{Schnitzler, Arthur 15.05.1862 – 21.10.1931@\textsc{Schnitzler, Arthur} (15.05.1862 – 21.10.1931), \emph{Schriftsteller, Mediziner}!Episode8. 09. 1889@\strich\emph{Episode} {[}8. 09. 1889{]}|pwk}
                  erschien Mitte September 1889 in der von Goldmann\pwindex{Goldmann, Paul 31.01.1865 – 25.09.1935@\textsc{Goldmann, Paul} (31.01.1865 – 25.09.1935), \emph{Schriftsteller, Journalist}|pwk} redigierten Zeitschrift \emph{An der schönen blauen Donau}\pwindex{der schoenen blauen Donau1886 – 1896@\emph{An der schönen blauen Donau}|pwk}. }}}\label{K_L02661-111h}; und Du wirſt nicht
               von ihr verlangen, was ſie nimmer gewähren kann: daß ſie Dich als ganzen Menſchen
               befriedige! Das klingt wie Moral, iſt aber nur Vernunft{\dotsfive}\pend
           \pstart
           Daß Du \label{K_L02661-2v}\edtext{aufgeführt}{\lemma{\textnormal{\emph{aufgeführt}}}\Cendnote{\textnormal{Am 11. 4. 1891 wurde Schnitzler\pwindex{Schnitzler, Arthur 15.05.1862 – 21.10.1931@\textsc{Schnitzler, Arthur} (15.05.1862 – 21.10.1931), \emph{Schriftsteller, Mediziner}|pwk}s Einakter \emph{Das Abenteuer seines Lebens}\pwindex{Schnitzler, Arthur 15.05.1862 – 21.10.1931@\textsc{Schnitzler, Arthur} (15.05.1862 – 21.10.1931), \emph{Schriftsteller, Mediziner}!Abenteuer seines Lebens1888@\strich\emph{Das Abenteuer seines Lebens} {[}1888{]}|pwk} im Volkstheater in Rudolphsheim\oindex{Volkstheater in Rudolphsheim@\textbf{Volkstheater in Rudolphsheim}|pwk} erstmals aufgeführt. Es handelte
                  sich dabei um die erste Aufführung eines Stück\pwindex{Schnitzler, Arthur 15.05.1862 – 21.10.1931@\textsc{Schnitzler, Arthur} (15.05.1862 – 21.10.1931), \emph{Schriftsteller, Mediziner}!Abenteuer seines Lebens1888@\strich\emph{Das Abenteuer seines Lebens} {[}1888{]}|pwkv}s von Schnitzler\pwindex{Schnitzler, Arthur 15.05.1862 – 21.10.1931@\textsc{Schnitzler, Arthur} (15.05.1862 – 21.10.1931), \emph{Schriftsteller, Mediziner}|pwk}.}}}\label{K_L02661-2h} worden biſt, erfahre ich zum erſten Mal aus Deinem
               Briefe. Ich leſe die Wien\oindex{Wien@\textbf{Wien}|pw}er Blätter nicht, weil
               mir die Lectüre zu weh thut. So iſt mir Alles entgangen. Alſo bitte ſehr: ſchreib’
               mir Einiges {\pb}über Erfolg und Kritik; wenn möglich
               ſchicke mir eine oder die andere Beſprechung; Du bekommſt ſie bald zurück. Jedenfalls
               herzlichen Glückwunſch zum erſten Schritt vor die Rampe. Ich hätte freilich
               gewünſcht, daß Dich das Burgtheater\orgindex{Burgtheater@Burgtheater|pw} aus der Taufe
               gehoben hätte; immerhin freut es mich, daß man gerade das »Abenteuer ſeines Lebens\pwindex{Schnitzler, Arthur 15.05.1862 – 21.10.1931@\textsc{Schnitzler, Arthur} (15.05.1862 – 21.10.1931), \emph{Schriftsteller, Mediziner}!Abenteuer seines Lebens1888@\strich\emph{Das Abenteuer seines Lebens} {[}1888{]}|pw}« gewählt hat, welches ich für das
               bühnenwirkſamſte Deiner Stücke halte. Lieber Gott, wie gern wäre ich dabei geweſen!
               Wie hat ſich Dein \label{K_L02661-4v}\edtext{Vater\pwindex{Schnitzler, Johann 10.04.1835 – 02.05.1893@\textsc{Schnitzler, Johann} (10.04.1835 – 02.05.1893), \emph{Laryngologe}|pwv}}{\lemma{\textnormal{\emph{Vater}}}\Cendnote{\textnormal{Am 14. 5. 1891 notierte Schnitzler\pwindex{Schnitzler, Arthur 15.05.1862 – 21.10.1931@\textsc{Schnitzler, Arthur} (15.05.1862 – 21.10.1931), \emph{Schriftsteller, Mediziner}|pwk} in seinem \emph{Tagebuch}\pwindex{Schnitzler, Arthur 15.05.1862 – 21.10.1931@\textsc{Schnitzler, Arthur} (15.05.1862 – 21.10.1931), \emph{Schriftsteller, Mediziner}!Tagebuch1981 – 2000@\strich\emph{Tagebuch} {[}1981 – 2000{]}|pwk}: »Mein Papa\pwindex{Schnitzler, Johann 10.04.1835 – 02.05.1893@\textsc{Schnitzler, Johann} (10.04.1835 – 02.05.1893), \emph{Laryngologe}|pwv} ist sehr erfreut über den Erfolg.«}}}\label{K_L02661-4h}
               zu der Sache verhalten? Wie ſteht’s mit Deinem großen Stück\pwindex{Schnitzler, Arthur 15.05.1862 – 21.10.1931@\textsc{Schnitzler, Arthur} (15.05.1862 – 21.10.1931), \emph{Schriftsteller, Mediziner}!Maerchen. Schauspiel in drei Aufzuegen1891 – 1891@\strich\emph{Das Märchen. Schauspiel in drei Aufzügen} {[}1891 – 1891{]}|pwv}? Haſt Du etwas
               Pſychologie hinausgeworfen und etwas Action hineingegeben? Und wann bekomme ich den
               dritten Act? {\dotsfive}\pend
           \pstart
           Und jetzt im Allgemeinen: wie lebſt Du? Mit wem verkehrſt Du? Kommſt Du in’s \textsc{Griensteidl\oindex{Cafe Griensteidl@\textbf{Café Griensteidl}|pw}}? Siehſt Du \textsc{Loris\pwindex{Hofmannsthal, Hugo von 01.02.1874 – 15.07.1929@\textsc{Hofmannsthal, Hugo von} (01.02.1874 – 15.07.1929), \emph{Schriftsteller}|pwv}}, {\pb}\textsc{Beer-Hoffmann\pwindex{Beer-Hofmann, Richard 11.07.1866 – 26.09.1945@\textsc{Beer-Hofmann, Richard} (11.07.1866 – 26.09.1945), \emph{Schriftsteller}|pw}}, die \textsc{Fanjung\pwindex{Van-Jung, Leo 15.10.1866 – 02.07.1939@\textsc{Van-Jung, Leo} (15.10.1866 – 02.07.1939), \emph{Gesangspädagoge, Mathematiker}|pwv}\pwindex{Van-Jung, Boris 15.10.1872 – 03.10.1899@\textsc{Van-Jung, Boris} (15.10.1872 – 03.10.1899), \emph{Mediziner}|pwv}}’s?\pend
           \pstart
           Mir gefallen die jungen Naturaliſten ganz und gar nicht mehr. Es wird wieder einmal
               Ereigniß, was für Wien\oindex{Wien@\textbf{Wien}|pw} ſo \strikeout{t\textcolor{gray}{×}} typiſch iſt: ein Paar Streber bemächtigen ſich einer Idee, um daran in die
               Höhe zu klettern. Dieſer \textsc{Joachim\pwindex{Joachim, Jaques 24.11.1866 – 07.11.1925@\textsc{Joachim, Jaques} (24.11.1866 – 07.11.1925), \emph{Rechtswissenschaftler, Rechtsanwalt, Herausgeber}|pw}} iſt – unter uns geſagt – nur ein gewöhnlicher \label{K_L02661-5v}\edtext{\textsc{\begin{otherlanguage}{french}Faiseur\end{otherlanguage}}}{\lemma{\textnormal{\emph{Faiseur}}}\Cendnote{\textnormal{französisch: Prahler}}}\label{K_L02661-5h}; ich habe
               hier mancherlei gehört, was mir ſehr den Geſchmack an ihm verdorben hat.\pend
           \pstart
           \textsc{Hildegard\pwindex{Mitis, Hilda von 1876-08-30 – 1894-12-14@\textsc{Mitis, Hilda von} (1876-08-30 – 1894-12-14), \emph{Schriftstellerin, Telefonistin}|pwv}} hat mir zweimal geſchrieben – \strikeout{ſie ha} ich habe
               ihr keinmal geantwortet. Im zweiten Briefe kündigt ſie mir noch einen dritten an –
               dann keinen mehr, ſie ſei gewohnt, nur dreimal zu bitten. Ich habe einen Haß gegen
               dieſes Weib\pwindex{Mitis, Hilda von 1876-08-30 – 1894-12-14@\textsc{Mitis, Hilda von} (1876-08-30 – 1894-12-14), \emph{Schriftstellerin, Telefonistin}|pwv} und einen
               unüberwindlichen Widerwillen (Fleißaufgabe für junge Pſychologen, das zu erklären).
                  {\pb}Vielleicht iſt es ihre Verlogenheit, ihre
               Empfindungsloſigkeit mir gegenüber, die ſich hinter ſchönen Briefen verbirgt. Ich
               haſſe ſie ſeit dem unverſchämt gut ſ\textcolor{gray}{t}yliſirten Abſchiedsbrief, den
               ſie mir geſchrieben. Vielleicht iſt es auch meine {\dotsfour} hm, hm
                  {\dotsfour} Kurzum, ſie iſt mir zuwider, und ich werde ſie
               wahrſcheinlich dreimal vergeblich bitten laſſen. Sie ſchrieb auch davon, daß ſie ſich
               mit Dir in Verbindung ſetzen wolle, wenn »die Sehnſucht nach \strikeout{Di\textcolor{gray}{r} gar} mir gar zu groß werde«. Du
               erinnerſt Dich wohl, was Du mir diesbezüglich verſprochen haſt? {\dotsfour}\pend
           \pstart
           Und nun ſei vielmals gegrüßt, mein Alter! Laß’ es Dir wohl ſein im lieben, lieben,
               lieben Wien\oindex{Wien@\textbf{Wien}|pw}! Quäl’ {\pb}Dich nicht ſo ſehr mit Deiner verfluchten
               Pſychologie und ſei ſubjectiv ſo glücklich, als Du es objectiv biſt.\pend
           \pstart
           Vor meiner Reiſe nach Brüſſe{[}l{]}\oindex{Bruessel@\textbf{Brüssel}|pw} höre ich wohl noch etwas von Dir? Das müßte freilich bald ſein.\pend
           \pstart
           Dein treuer {\\[\baselineskip]}\spacefill\mbox{Paul Goldmann.}\pend
           \leftskip=0em{}\pstart
           \noindent{}Empfiehl’ mich den Deinen, und grüße \textsc{Kapper\pwindex{Kapper, Friedrich 21.04.1861 – 22.07.1939@\textsc{Kapper, Friedrich} (21.04.1861 – 22.07.1939), \emph{Mediziner}|pw}} und \textsc{Loris\pwindex{Hofmannsthal, Hugo von 01.02.1874 – 15.07.1929@\textsc{Hofmannsthal, Hugo von} (01.02.1874 – 15.07.1929), \emph{Schriftsteller}|pw}}, aber \uline{nicht}{ }\textsc{Beer-Hoffmann\pwindex{Beer-Hofmann, Richard 11.07.1866 – 26.09.1945@\textsc{Beer-Hofmann, Richard} (11.07.1866 – 26.09.1945), \emph{Schriftsteller}|pw}}, weil mir der Schurke\pwindex{Beer-Hofmann, Richard 11.07.1866 – 26.09.1945@\textsc{Beer-Hofmann, Richard} (11.07.1866 – 26.09.1945), \emph{Schriftsteller}|pwv} nicht ſchreibt. Wie macht ſich \textsc{Hirschfeld\pwindex{Hirschfeld, Robert 17.09.1857 – 02.04.1914@\textsc{Hirschfeld, Robert} (17.09.1857 – 02.04.1914), \emph{Journalist, Musikkritiker}|pw}} in der Sonn- und Montagszeitung\orgindex{Wiener Sonn- und Montagszeitung@Wiener Sonn- und Montagszeitung|pw}?\pend
           \endnumbering\briefempfaengerindex{Schnitzler, Arthur@\textsc{Schnitzler, Arthur}!zzzGoldmann, Paul@\emph{von Paul Goldmann}!1891-04-271@{27. 4. 1891}|)be}\mylabel{h}\end{ledgroupsized}\begin{anhang}\end{anhang}\newcommand{\dateiname}{L02661}\newcommand{\titel}{Paul Goldmann an Arthur Schnitzler, 27. 4. 1891}\newcommand{\editorInnen}{Martin Anton Müller und Laura Untner}\input{../tex-inputs/latex-pdf-abspann}
      