%% latex-leseansicht-vorspann.tex
%% Vorspann für die Leseansicht.
%% Lädt die gemeinsame Datei latex-vorspann.tex mit nicht gesetztem Schalter.

\newif\ifkorrekturansicht
\korrekturansichtfalse

\input{../tex-inputs/latex-vorspann}


\section[Theodor Herzl an Arthur Schnitzler, {[}zwischen 28. 7.  und 5. 8. 1895?{]}]{L03894 Theodor Herzl an Arthur Schnitzler, {[}zwischen 28. 7.  und 5. 8. 1895?{]}}
\nopagebreak\mylabel{L03894v}
\rehead{ }\normalsize\beginnumbering\briefempfaengerindex{Schnitzler, Arthur@\textsc{Schnitzler, Arthur}!zzzHerzl, Theodor@\emph{von Theodor Herzl}!1895-08-053@{{[}zwischen 28. 7.  und 5. 8. 1895?{]}}|(be}
\toendnotes[C]{\smallbreak\pagebreak[2]}
\correspDesc{Versand  durch Theodor Herzl im Zeitraum [zwischen 28. 7.  und 5. 8. 1895?] in Bad Aussee
\newline{}Erhalt  durch Arthur Schnitzler im Zeitraum [Anfang August 1895] in Bad Ischl}\toendnotes[C]{\smallbreak}
\Standort{CUL, Schnitzler, B 39.}
\physDesc{Brief, 1 Blatt, 1 Seite, 413 Zeichen
\newline{}Handschrift: schwarze Tinte, lateinische Kurrent
\newline{}Schnitzler: mit Bleistift datiert: »Anf. Aug 95« 
\newline{}Ordnung: mit Bleistift von unbekannter Hand nummeriert: »43« }
\buchAbdrucke{\weitereDrucke{Theodor Herzl: \emph{Briefe Anfang Mai 1895 – Anfang Dezember 1898}. Bearbeitet von Barbara Schäfer in Zusammenarbeit mit Sofia Gelmann, Chaya Harel, Ines Rubin und Daisy Ticho. Berlin, Frankfurt am Main, Wien: \emph{Propyläen} 1990, S. 61 (Briefe und Tagebücher. Herausgegeben von Alex Bein, Hermann Greive, Moshe Schaerf, Julius H. Schoeps und Johannes Wachten, 4).} }\toendnotes[C]{\smallbreak}
\pstart
           \raggedleft{}{\pb}Aussee Villa Fuchs\oindex{Villa Fuchs@\textbf{Villa Fuchs}, \emph{Wohngebäude}|pw}\pend
           
\pstart{}Lieber Freund!\pend\vspace{0.5em}
\pstart
           \label{K_L03894-1v}\edtext{Hier bin ich}{\lemma{\textnormal{\emph{Hier bin ich}}}\Cendnote{\textnormal{Theodor Herzl reiste am Abend des 27. 7. 1895 aus Paris\oindex{Paris@\textbf{Paris}, \emph{Hauptstadt}|pwk} ab. Der Brief ist demnach frühestens am 28. 7. 1895 geschrieben worden und spätestens am 5. 8. 1895, wenn man die Frist berücksichtigt, die die Vereinbarung von Schnitzlers Besuch am 7. 8. 1895 in Bad Aussee\oindex{Bad Aussee@\textbf{Bad Aussee}, \emph{Hauptstadt}|pwk} in Anspruch genommen hat.}}}\label{K_L03894-1} und werde mich sehr freuen, Sie \label{K_L03894-2v}\edtext{bald zu sehen}{\lemma{\textnormal{\emph{bald zu sehen}}}\Cendnote{\textnormal{Die 
                  vorliegende Einladung dürfte den Besuch durch Schnitzler am 7. 8. 1895 angeregt haben}}}\label{K_L03894-2}.\pend
           
\pstart
           Zeigen Sie mir bitte einen Tag früher Ihre Ankunft an. Ich weiss noch nicht, ob ich
               in den nächsten Tagen nicht werde auf 24 Stunden wegfahren müssen. Sollten Sie sich
               gerade für \uline{den} Tag anmelden, so telegraphire ich
               Ihnen ab.\pend
           
\pstart
           Ich werde Ihnen, wenn ich Ihren Zug erfahre Ihnen zur Bahn\oindex{Bahnhof Bad Aussee@\textbf{Bahnhof Bad Aussee}, \emph{Bahnhofsgebäude}|pwv} entgegenkommen.\pend
           
\pstart
           Herzlich Ihr Freund{\\[\baselineskip]}\spacefill\mbox{Th. Herzl}\pend
           \leftskip=0em{}\selectlanguage{ngerman}\endnumbering\briefempfaengerindex{Schnitzler, Arthur@\textsc{Schnitzler, Arthur}!zzzHerzl, Theodor@\emph{von Theodor Herzl}!1895-07-283@{{[}zwischen 28. 7.  und 5. 8. 1895?{]}}|)be}\mylabel{L03894h}
\begin{anhang}
\end{anhang}\newcommand{\dateiname}{L03894}\newcommand{\titel}{Theodor Herzl an Arthur Schnitzler, [zwischen 28. 7.  und 5. 8. 1895?]}\newcommand{\editorInnen}{Selma Jahnke und Martin Anton Müller}%% latex-leseansicht-abspann.tex
%% Abspann für die Leseansicht.
%% Der Schalter \ifkorrekturansicht ist bereits durch den Vorspann gesetzt.

%% latex-abspann.tex
%% Gemeinsamer Abspann für Korrekturansicht und Leseansicht.
%% Setzt den Schalter \ifkorrekturansicht voraus (gesetzt in den
%% einbindenden Dateien latex-korrekturansicht-abspann.tex bzw.
%% latex-leseansicht-abspann.tex).
%% ---------------------------------------------------------------

\normalsize

% Das esempio-Environment wird nur in der Leseansicht benötigt
\ifkorrekturansicht\else
\newenvironment{esempio}[3]%
{
    \vspace{1.5ex}
    \rlap{\underline{#1}}
    \par
    \setlength{\parindent}{0cm}
    \nopagebreak
    \leftskip=#2cm
    \rightskip=#3cm
}
{
    \par
}
\fi

\doendnotes{C}
\bigskip
\vfill

\clearpage

\footnotesize

\ifkorrekturansicht
  \lohead{\textsc{register}}
\fi

% theindex-Environment neu definieren ohne reledmac
\makeatletter
\renewenvironment{theindex}{%
  \ifkorrekturansicht
    \section*{\indexname}%
  \else
    \subsubsection*{Index der erwähnten Entitäten}%
  \fi
  \setlength{\parindent}{0pt}%
  \setlength{\parskip}{0pt plus 0.3pt}%
  \let\item\@idxitem
}{%
  \ifkorrekturansicht\clearpage\fi
}
\makeatother

\IfFileExists{\jobname-pw.ind}{\input{\jobname-pw.ind}}{}

% Quellenangabe nur in der Leseansicht
\ifkorrekturansicht\else
% Fallback-Definitionen, falls die .tex-Datei \titel etc. nicht gesetzt hat
\providecommand{\titel}{}
\providecommand{\editorInnen}{}
\providecommand{\dateiname}{\jobname}

\vspace{3cm}

\vfill

\footnotesize
\textsc{Quelle}: \titel. Herausgegeben von {\editorInnen}. In: \emph{Arthur Schnitzler: Briefwechsel mit Autorinnen und Autoren}.
 Digitale Edition, https://schnitzler-briefe.acdh.oeaw.ac.at/{\dateiname}.html (Stand \today)
\fi

\end{document}


