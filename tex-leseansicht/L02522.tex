%% latex-leseansicht-vorspann.tex
%% Vorspann für die Leseansicht.
%% Lädt die gemeinsame Datei latex-vorspann.tex mit nicht gesetztem Schalter.

\newif\ifkorrekturansicht
\korrekturansichtfalse

\input{../tex-inputs/latex-vorspann}


\section[Christiane Hofmannsthal an Arthur Schnitzler, 3. 9. 1929]{L02522 Christiane Hofmannsthal an Arthur Schnitzler, 3. 9. 1929}
\nopagebreak\mylabel{L02522v}
\rehead{ }\normalsize\beginnumbering\briefempfaengerindex{Schnitzler, Arthur@\textsc{Schnitzler, Arthur}!zzzZimmer, Christiane@\emph{von Christiane Zimmer}!1929-09-032@{3. 9. 1929}|(be}
\toendnotes[C]{\smallbreak\pagebreak[2]}
\correspDesc{Versand  durch Christiane Hofmannsthal am 3. 9. 1929 in Bad Aussee
\newline{}Erhalt  durch Arthur Schnitzler im Zeitraum [4. 9. 1929
                  – 8. 9. 1929?] in Wien}\toendnotes[C]{\smallbreak}
\Standort{CUL, Schnitzler, B 43.}
\physDesc{Brief, 1 Blatt, 2 Seiten, 1769 Zeichen
\newline{}Schreibmaschine
\newline{}Handschrift: schwarze Tinte, lateinische Kurrent (\noindent{}Unterschrift, Nachschrift)
\newline{}Schnitzler: mit rotem Buntstift beschriftet mit »Hofm« und
                                 fünf Unterstreichungen }\toendnotes[C]{\smallbreak}
\pstart
           \raggedleft{}{\pb}Bad Aussee\oindex{Bad Aussee@\textbf{Bad Aussee}, \emph{Hauptstadt}|pw}, 3. September 1929\pend
           
\pstart{}Lieber Arthur,\pend\vspace{0.5em}
\pstart
           Du kennst den Wunsch der neuen Rundschau\orgindex{Neue Rundschau, Neue Deutsche Rundschau, Freie Bühne@Neue Rundschau, Neue Deutsche Rundschau, Freie Bühne|pw}, in dem
                  Sonderheft im November\pwindex{Hofmannsthal, Hugo von 1.\,2.\,1874 Wien – 15.\,7.\,1929 Rodaun@\textsc{Hofmannsthal, Hugo von} (1.\,2.\,1874 Wien – 15.\,7.\,1929 Rodaun), \emph{Schriftsteller}!Aus dem Nachlass@\strich\emph{Aus dem Nachlass}|pw} einige von Papas\pwindex{Hofmannsthal, Hugo von 1.\,2.\,1874 Wien – 15.\,7.\,1929 Rodaun@\textsc{Hofmannsthal, Hugo von} (1.\,2.\,1874 Wien – 15.\,7.\,1929 Rodaun), \emph{Schriftsteller}|pwv} Briefen abzudrucken.
               Ich glaube, dass man sich dieser Absicht nicht ganz verschliessen soll, da wir sonst
               aus dem Nachlass sehr wenig für diesen Zweck Geeignetes zur Verfügung stellen konnten
               und es uns daher lieb wäre, wenn der Raum, der mit Papas\pwindex{Hofmannsthal, Hugo von 1.\,2.\,1874 Wien – 15.\,7.\,1929 Rodaun@\textsc{Hofmannsthal, Hugo von} (1.\,2.\,1874 Wien – 15.\,7.\,1929 Rodaun), \emph{Schriftsteller}|pwv} eigenen Sachen erfüllt ist, etwas grösser würde als
               das Geschreibe über ihn.\pend
           
\pstart
           Die Briefe sind alle wunderschön, aber wie Du selbst gesehen haben wirst, doch sehr
               schwierig in ihrem ganzen Umfang zu veröffentlichen, sie sind zu intim im Ton,
               sprechen viel von Menschen und Zeitdingen, was vielleicht jetzt noch etwas frühe
               wäre, und ausserdem ohne Deine Gegenbriefe nicht sehr sinnvoll.\pend
           
\pstart
           Wir haben daher aus einigen Briefen Auszüge gemacht, die auf Papas\pwindex{Hofmannsthal, Hugo von 1.\,2.\,1874 Wien – 15.\,7.\,1929 Rodaun@\textsc{Hofmannsthal, Hugo von} (1.\,2.\,1874 Wien – 15.\,7.\,1929 Rodaun), \emph{Schriftsteller}|pwv} Schaffen und seine Werke Beziehung
               haben, und sende Dir dieselben zur Einsichtnahme ein. Ich lasse aber noch einige aus
               späteren Jahren, die wir noch nicht ganz gesichtet haben, nachfolgen. Ich würde es
               für richtig finden, in diesem Heft, als Vorläufigstes, nur solche, gleichsam
               biographisch erklärende Briefe aufzunehmen, denn alles andere bedürfte zu viel
               Commentares und wäre auch verfrüht.\pend
           
\pstart
           Falls es Dir in dieser Form geeignet scheint, lass es uns oder am Liebsten Dr. Kayser\pwindex{Kayser, Rudolf 28.\,11.\,1889 Parchim – 6.\,2.\,1964 New York City@\textsc{Kayser, Rudolf} (28.\,11.\,1889 Parchim – 6.\,2.\,1964 New York City), \emph{Schriftsteller, Verlagslektor}|pw}, der einen Durchschlag hat, wissen,
               ebenso auch falls Du im Einzelnen oder im Ganzen Einspruch erheben möchtest, was
               natürlich völlig Dir überlassen bleibt. Die Briefe ganz und ungekürzt zu nehmen
               erschiene mir heute noch nicht richtig, ich weiss natürlich nicht, wie Du darüber
               denkst.\pend
           
\pstart
           Mit innigen Grüssen von uns allen{\\[\baselineskip]}\spacefill\mbox{{[}hs.:{]} Christiane}\pend
           \leftskip=0em{}
\pstart
           \noindent{}{\pb}Von den schönen Briefen, \uline{Deine Werke} betreffend, halte ich den jetzigen
                  Zeitpunkt auch noch verfrüht zur Veröffentlichung, meinst Du nicht?\pend
           \selectlanguage{ngerman}\endnumbering\briefempfaengerindex{Schnitzler, Arthur@\textsc{Schnitzler, Arthur}!zzzZimmer, Christiane@\emph{von Christiane Zimmer}!1929-09-032@{3. 9. 1929}|)be}\mylabel{L02522h}  \newcommand{\dateiname}{L02522}\newcommand{\titel}{Christiane Hofmannsthal an Arthur Schnitzler, 3. 9. 1929}\newcommand{\editorInnen}{Martin Anton Müller und Gerd-Hermann Susen}%% latex-leseansicht-abspann.tex
%% Abspann für die Leseansicht.
%% Der Schalter \ifkorrekturansicht ist bereits durch den Vorspann gesetzt.

%% latex-abspann.tex
%% Gemeinsamer Abspann für Korrekturansicht und Leseansicht.
%% Setzt den Schalter \ifkorrekturansicht voraus (gesetzt in den
%% einbindenden Dateien latex-korrekturansicht-abspann.tex bzw.
%% latex-leseansicht-abspann.tex).
%% ---------------------------------------------------------------

\normalsize

% Das esempio-Environment wird nur in der Leseansicht benötigt
\ifkorrekturansicht\else
\newenvironment{esempio}[3]%
{
    \vspace{1.5ex}
    \rlap{\underline{#1}}
    \par
    \setlength{\parindent}{0cm}
    \nopagebreak
    \leftskip=#2cm
    \rightskip=#3cm
}
{
    \par
}
\fi

\doendnotes{C}
\bigskip
\vfill

\clearpage

\footnotesize

\ifkorrekturansicht
  \lohead{\textsc{register}}
\fi

% theindex-Environment neu definieren ohne reledmac
\makeatletter
\renewenvironment{theindex}{%
  \ifkorrekturansicht
    \section*{\indexname}%
  \else
    \subsubsection*{Index der erwähnten Entitäten}%
  \fi
  \setlength{\parindent}{0pt}%
  \setlength{\parskip}{0pt plus 0.3pt}%
  \let\item\@idxitem
}{%
  \ifkorrekturansicht\clearpage\fi
}
\makeatother

\IfFileExists{\jobname-pw.ind}{\input{\jobname-pw.ind}}{}

% Quellenangabe nur in der Leseansicht
\ifkorrekturansicht\else
% Fallback-Definitionen, falls die .tex-Datei \titel etc. nicht gesetzt hat
\providecommand{\titel}{}
\providecommand{\editorInnen}{}
\providecommand{\dateiname}{\jobname}

\vspace{3cm}

\vfill

\footnotesize
\textsc{Quelle}: \titel. Herausgegeben von {\editorInnen}. In: \emph{Arthur Schnitzler: Briefwechsel mit Autorinnen und Autoren}.
 Digitale Edition, https://schnitzler-briefe.acdh.oeaw.ac.at/{\dateiname}.html (Stand \today)
\fi

\end{document}


