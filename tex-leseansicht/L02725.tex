%% latex-leseansicht-vorspann.tex
%% Vorspann für die Leseansicht.
%% Lädt die gemeinsame Datei latex-vorspann.tex mit nicht gesetztem Schalter.

\newif\ifkorrekturansicht
\korrekturansichtfalse

\input{../tex-inputs/latex-vorspann}


\section[Paul Goldmann an Arthur Schnitzler, {[}28./29.?{]} 12. 1893]{L02725 Paul Goldmann an Arthur Schnitzler, [28./29.?] 12. 1893}
\nopagebreak\mylabel{L02725v}
\rehead{ }\normalsize\beginnumbering\briefempfaengerindex{Schnitzler, Arthur@\textsc{Schnitzler, Arthur}!zzzGoldmann, Paul@\emph{von Paul Goldmann}!1893-12-281@{[28./29.?] 12. 1893}|(be}
\toendnotes[C]{\smallbreak\pagebreak[2]}
\correspDesc{Versand  durch Paul Goldmann am [28./29.?] 12. 1893 in Paris
\newline{}Erhalt  durch Arthur Schnitzler am 31. 12. 1893 in Wien}\toendnotes[C]{\smallbreak}
\Standort{DLA, A:Schnitzler, HS.NZ85.1.3163.}
\physDesc{Postkarte, 195 Zeichen
\newline{}Handschrift: schwarze Tinte, deutsche Kurrent
\newline{}Versand: 1) Stempel: »\nobreak{}\oindex{place de la Bourse@\textbf{place de la Bourse}, \emph{Platz}|pwk}Par{[}is{]} Pl. de la
                                          Bour{[}se{]}, \textcolor{gray}{×}\-\textcolor{gray}{×}\begin{otherlanguage}{french}Dec\end{otherlanguage}{[}.{]} 93, 4\textsuperscript{E}\nobreak{}«.   2) Stempel: »\nobreak{}\oindex{IX., Alsergrund@\textbf{IX., Alsergrund}, \emph{Verwaltungsgebiet}|pwk}Wien 9/3 72, 31. 12. 93, 8. V, Bestellt\nobreak{}«. }\toendnotes[C]{\smallbreak}\pstart{}\textsc{{\pb}Autriche\oindex{Österreich@\textbf{Österreich}|pw}.}\pend{}\pstart{}\textsc{Herrn Dr. Arthur Schnitzler}\pend{}\pstart{}\textsc{IX. Frankgaſse 1\oindex{Wien@\textbf{Wien}!IX., Alsergrund@\textbf{IX., Alsergrund}!Frankgasse 1@\textbf{Frankgasse 1}, \emph{Wohngebäude}|pw}}\pend{}\pstart{}\textsc{Wien\oindex{Wien@\textbf{Wien}, \emph{Verwaltungsgebiet}|pw}.}\pend{}{\bigskip}\vspace{1em}
\pstart
           \noindent{}{\pb}1.) Leſen: \label{K_L02725-1v}\edtext{\textsc{Albrecht Dürers\pwindex{Dürer, Albrecht 21.\,5.\,1471 Nürnberg – 6.\,4.\,1528 ebd.@\textsc{Dürer, Albrecht} (21.\,5.\,1471 Nürnberg – 6.\,4.\,1528 ebd.), \emph{Maler}|pw}}{ }Briefe und Tagebücher\pwindex{Dürer, Albrecht 21.\,5.\,1471 Nürnberg – 6.\,4.\,1528 ebd.@\textsc{Dürer, Albrecht} (21.\,5.\,1471 Nürnberg – 6.\,4.\,1528 ebd.), \emph{Maler}!Dürers Briefe, Tagebücher und Reime@\strich\emph{Dürers Briefe, Tagebücher und Reime}|pw}}{\lemma{\textnormal{\emph{Albrecht … Tagebücher}}}\Cendnote{\textnormal{\emph{Dürers\pwindex{Dürer, Albrecht 21.\,5.\,1471 Nürnberg – 6.\,4.\,1528 ebd.@\textsc{Dürer, Albrecht} (21.\,5.\,1471 Nürnberg – 6.\,4.\,1528 ebd.), \emph{Maler}|pwk}s Briefe, Tagebücher und Reime
                        nebst einem Anhange von Zuschriften an und für Dürer}\pwindex{Dürer, Albrecht 21.\,5.\,1471 Nürnberg – 6.\,4.\,1528 ebd.@\textsc{Dürer, Albrecht} (21.\,5.\,1471 Nürnberg – 6.\,4.\,1528 ebd.), \emph{Maler}!Dürers Briefe, Tagebücher und Reime@\strich\emph{Dürers Briefe, Tagebücher und Reime}|pwk}. Übersetzt und mit
                     Einleitung, Anmerkungen, Personenverzeichniss und einer Reisekarte versehen von
                        Moritz Thausing\pwindex{Thausing, Moritz 3.\,6.\,1838 Čížkovice – 11.\,8.\,1884 Litoměřice@\textsc{Thausing, Moritz} (3.\,6.\,1838 Čížkovice – 11.\,8.\,1884 Litoměřice), \emph{Kunsthistoriker}|pwk}.
                     Wien: \emph{Braumüller}\orgindex{Verlag Wilhelm Braumüller@Verlag Wilhelm Braumüller|pwk}{ }1872. (Quellenschriften für Kunstgeschichte und Kunsttechnik des
                     Mittelalters und der Renaissance 3) – Eine Lektüre durch Schnitzler ist nicht belegt. }}}\label{K_L02725-1} (\textsc{Braumueller\orgindex{Verlag Wilhelm Braumüller@Verlag Wilhelm Braumüller|pw}, Wien\oindex{Wien@\textbf{Wien}, \emph{Verwaltungsgebiet}|pw}}, 1872).\pend
           
\pstart
           2.) Mir{ }ſchreiben.\pend
           
\pstart
           3.) Fröhliches Neujahr Dir und den Freunden.\pend
           \pstart \spacefill\mbox{P. G.}\pend{}\selectlanguage{ngerman}\endnumbering\briefempfaengerindex{Schnitzler, Arthur@\textsc{Schnitzler, Arthur}!zzzGoldmann, Paul@\emph{von Paul Goldmann}!1893-12-281@{[28./29.?] 12. 1893}|)be}\mylabel{L02725h}  \newcommand{\dateiname}{L02725}\newcommand{\titel}{Paul Goldmann an Arthur Schnitzler, [28./29.?] 12. 1893}\newcommand{\editorInnen}{Martin Anton Müller und Laura Untner}%% latex-leseansicht-abspann.tex
%% Abspann für die Leseansicht.
%% Der Schalter \ifkorrekturansicht ist bereits durch den Vorspann gesetzt.

%% latex-abspann.tex
%% Gemeinsamer Abspann für Korrekturansicht und Leseansicht.
%% Setzt den Schalter \ifkorrekturansicht voraus (gesetzt in den
%% einbindenden Dateien latex-korrekturansicht-abspann.tex bzw.
%% latex-leseansicht-abspann.tex).
%% ---------------------------------------------------------------

\normalsize

% Das esempio-Environment wird nur in der Leseansicht benötigt
\ifkorrekturansicht\else
\newenvironment{esempio}[3]%
{
    \vspace{1.5ex}
    \rlap{\underline{#1}}
    \par
    \setlength{\parindent}{0cm}
    \nopagebreak
    \leftskip=#2cm
    \rightskip=#3cm
}
{
    \par
}
\fi

\doendnotes{C}
\bigskip
\vfill

\clearpage

\footnotesize

\ifkorrekturansicht
  \lohead{\textsc{register}}
\fi

% theindex-Environment neu definieren ohne reledmac
\makeatletter
\renewenvironment{theindex}{%
  \ifkorrekturansicht
    \section*{\indexname}%
  \else
    \subsubsection*{Index der erwähnten Entitäten}%
  \fi
  \setlength{\parindent}{0pt}%
  \setlength{\parskip}{0pt plus 0.3pt}%
  \let\item\@idxitem
}{%
  \ifkorrekturansicht\clearpage\fi
}
\makeatother

\IfFileExists{\jobname-pw.ind}{\input{\jobname-pw.ind}}{}

% Quellenangabe nur in der Leseansicht
\ifkorrekturansicht\else
% Fallback-Definitionen, falls die .tex-Datei \titel etc. nicht gesetzt hat
\providecommand{\titel}{}
\providecommand{\editorInnen}{}
\providecommand{\dateiname}{\jobname}

\vspace{3cm}

\vfill

\footnotesize
\textsc{Quelle}: \titel. Herausgegeben von {\editorInnen}. In: \emph{Arthur Schnitzler: Briefwechsel mit Autorinnen und Autoren}.
 Digitale Edition, https://schnitzler-briefe.acdh.oeaw.ac.at/{\dateiname}.html (Stand \today)
\fi

\end{document}


