%% latex-korrekturansicht-vorspann.tex
%% Vorspann für die Korrekturansicht.
%% Lädt die gemeinsame Datei latex-vorspann.tex mit gesetztem Schalter.

\newif\ifkorrekturansicht
\korrekturansichttrue

\input{../tex-inputs/latex-vorspann}


\section[Paul Goldmann an Arthur Schnitzler, {[}28./29.?{]} 12. 1893]{L02725 Paul Goldmann an Arthur Schnitzler, {[}28./29.?{]} 12. 1893}
\nopagebreak\mylabel{L02725v}
\rehead{ }\normalsize\beginnumbering\briefempfaengerindex{Schnitzler, Arthur@\textsc{Schnitzler, Arthur}!zzzGoldmann, Paul@\emph{von Paul Goldmann}!1893-12-281@{{[}28./29.?{]} 12. 1893}|(be}
\toendnotes[C]{\smallbreak\pagebreak[2]}\Standort{DLA, A:Schnitzler, HS.NZ85.1.3163.}
\physDesc{Postkarte, 195 Zeichen
\newline{}Handschrift: 1) schwarze Tinte, deutsche Kurrent\hspace{1em}2) schwarze Tinte, lateinische Kurrent (\noindent{}Adresse)\hspace{1em}
\newline{}Versand: 1) Stempel: »\nobreak{}\oindex{place de la Bourse@\textbf{place de la Bourse}, \emph{Platz (K.PLT)}|pwk}Par{[}is{]} Pl. de la
                                          Bour{[}se{]}, \textcolor{gray}{×}\-\textcolor{gray}{×}\begin{otherlanguage}{french}Dec\end{otherlanguage}{[}.{]} 93, 4\textsuperscript{E}\nobreak{}«.   2) Stempel: »\nobreak{}\oindex{IX., Alsergrund@\textbf{IX., Alsergrund}, \emph{A.ADM3}|pwk}Wien 9/3 72, 31. 12. 93, 8. V, Bestellt\nobreak{}«. }\toendnotes[C]{\smallbreak}\pstart{}{\pb}Autriche\oindex{Oesterreich@\textbf{Österreich}, \emph{A.PCLI}|pw}.\pend{}\pstart{}Herrn Dr. Arthur Schnitzler\pend{}\pstart{}IX. Frankgaſse 1\oindex{Frankgasse 1@\textbf{Frankgasse 1}, \emph{Wohngebäude (K.WHS)}|pw}\pend{}\pstart{}Wien\oindex{Wien@\textbf{Wien}, \emph{A.ADM2}|pw}. \pend{}{\bigskip}\vspace{1em}
\pstart
           \noindent{}{\pb}1.) Leſen: \label{K_L02725-1v}\edtext{\textsc{Albrecht Dürers\pwindex{Duerer, Albrecht 21.05.1471 – 06.04.1528@\textsc{Dürer, Albrecht} (21.05.1471 – 06.04.1528), \emph{Maler/Malerin}|pw}}{ }Briefe und Tagebücher\pwindex{Duerers Briefe, Tagebuecher und Reime@\emph{Dürers Briefe, Tagebücher und Reime}|pw}}{\lemma{\textnormal{\emph{Albrecht … Tagebücher}}}\Cendnote{\textnormal{\emph{Dürers\pwindex{Duerer, Albrecht 21.05.1471 – 06.04.1528@\textsc{Dürer, Albrecht} (21.05.1471 – 06.04.1528), \emph{Maler/Malerin}|pwk}s Briefe, Tagebücher und Reime
                        nebst einem Anhange von Zuschriften an und für Dürer}\pwindex{Duerers Briefe, Tagebuecher und Reime@\emph{Dürers Briefe, Tagebücher und Reime}|pwk}. Übersetzt und mit
                     Einleitung, Anmerkungen, Personenverzeichniss und einer Reisekarte versehen von
                        Moritz Thausing\pwindex{Thausing, Moritz 1838-06-03 – 1884-08-11@\textsc{Thausing, Moritz} (1838-06-03 – 1884-08-11), \emph{Kunsthistoriker/Kunsthistorikerin}|pwk}.
                     Wien: \emph{Braumüller}\orgindex{Verlag Wilhelm Braumueller@Verlag Wilhelm Braumüller|pwk}{ }1872. (Quellenschriften für Kunstgeschichte und Kunsttechnik des
                     Mittelalters und der Renaissance 3) – Eine Lektüre durch Schnitzler ist nicht belegt. }}}\label{K_L02725-1} (\textsc{Braumueller\orgindex{Verlag Wilhelm Braumueller@Verlag Wilhelm Braumüller|pw}, Wien\oindex{Wien@\textbf{Wien}, \emph{A.ADM2}|pw}}, 1872).\pend
           
\pstart
           2.) Mir ſchreiben.\pend
           
\pstart
           3.) Fröhliches Neujahr Dir und den Freunden.\pend
           \pstart \spacefill\mbox{P. G.}\pend{}\selectlanguage{ngerman}\endnumbering\briefempfaengerindex{Schnitzler, Arthur@\textsc{Schnitzler, Arthur}!zzzGoldmann, Paul@\emph{von Paul Goldmann}!1893-12-281@{{[}28./29.?{]} 12. 1893}|)be}\mylabel{L02725h}  \normalsize

\doendnotes{C}
\bigskip
\vfill

\clearpage

\footnotesize

\lohead{\textsc{register}}

% Definiere theindex-Environment komplett neu ohne reledmac
\makeatletter
\renewenvironment{theindex}{%
  \section*{\indexname}%
  \setlength{\parindent}{0pt}%
  \setlength{\parskip}{0pt plus 0.3pt}%
  \let\item\@idxitem
}{%
  \clearpage
}
\makeatother

\IfFileExists{\jobname-pw.ind}{\input{\jobname-pw.ind}}{}

\end{document}

      