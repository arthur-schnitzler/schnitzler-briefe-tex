%% latex-leseansicht-vorspann.tex
%% Vorspann für die Leseansicht.
%% Lädt die gemeinsame Datei latex-vorspann.tex mit nicht gesetztem Schalter.

\newif\ifkorrekturansicht
\korrekturansichtfalse

\input{../tex-inputs/latex-vorspann}

\begin{center}
            \textcolor{red}{ENTWURF. ENTZIFFERUNG NOCH NICHT KORREKTURGELESEN}
                      \end{center}
            
               \section[Paul Goldmann an Arthur Schnitzler, {[}28./29.?{]} 12. 1893]{ Paul Goldmann an Arthur Schnitzler, {[}28./29.?{]} 12. 1893}\nopagebreak\mylabel{v}\rehead{ }\begin{ledgroupsized}[t]{13cm}\normalsize\beginnumbering\briefempfaengerindex{Schnitzler, Arthur@\textsc{Schnitzler, Arthur}!zzzGoldmann, Paul@\emph{von Paul Goldmann}!1893-12-281@{{[}28./29.?{]} 12. 1893}|(be} \toendnotes[C]{\smallbreak\pagebreak[2]} \Standort{DLA, A:Schnitzler, HS.NZ85.1.3163.}
\physDesc{Postkarte
\newline{}Handschrift: 1) schwarze Tinte, deutsche Kurrent\hspace{1em}2) schwarze Tinte, lateinische Kurrent (\noindent{}Adresse)\hspace{1em}\newline{}Versand: 1) Stempel: »\nobreak{}\oindex{Place de la Bourse@\textbf{Place de la Bourse}|pwk}Par{[}is{]} Pl. de la
                                          Bour{[}se{]}, \textcolor{gray}{×}\-\textcolor{gray}{×}\-\textcolor{gray}{×}\begin{otherlanguage}{french}Dec\end{otherlanguage}{[}.{]} 93, 4\textcolor{gray}{×}\-\textcolor{gray}{×}\nobreak{}«.  2) Stempel: »\nobreak{}Wien 9/3, 31. 12. 93, 8.V, 72 Bestellt\nobreak{}«. }\toendnotes[C]{\smallbreak}\pstart{}{\pb}\textsc{Autriche\oindex{Oesterreich@\textbf{Österreich}|pw}}.\pend{}\pstart{}\textsc{Herrn Dr. Arthur Schnitzler}\pend{}\pstart{}\textsc{IX. Frankgaße 1\oindex{Frankgasse@\textbf{Frankgasse}|pw}}\pend{}\pstart{}\textsc{Wien\oindex{Wien@\textbf{Wien}|pw}. }\pend{}{\bigskip}\pstart
           \noindent{}{\pb}1.) Leſen: \textsc{Albrecht Dürer\pwindex{Duerer, Albrecht 21.05.1471 – 06.04.1528@\textsc{Dürer, Albrecht} (21.05.1471 – 06.04.1528), \emph{Bildender Künstler}|pw}s} Briefe und
                  Tagebücher\pwindex{Duerer, Albrecht 21.05.1471 – 06.04.1528@\textsc{Dürer, Albrecht} (21.05.1471 – 06.04.1528), \emph{Bildender Künstler}!Duerers Briefe, Tagebuecher und Reime1872 – 1872@\strich\emph{Dürers Briefe, Tagebücher und Reime} {[}1872 – 1872{]}|pw} (\textsc{Braumueller\orgindex{Verlag Wilhelm Braumueller@Verlag Wilhelm Braumüller|pw}, Wien\oindex{Wien@\textbf{Wien}|pw}}, 1872).\pend
           \pstart
           2.) Mir ſchreiben.\pend
           \pstart
           3.) Fröhliches \label{K_L02725-1v}\edtext{Neujahr}{\lemma{\textnormal{\emph{Neujahr}}}\Cendnote{\textnormal{Der Empfangsstempel vom 31. 12. 1893 ermöglicht eine ungefähre Datierung der
                  Postkarte. Wenn davon ausgegangen wird, dass die Postkarte etwa zwei Tage nach Wien\oindex{Wien@\textbf{Wien}|pwk} unterwegs war, kann sie auf den 28. oder 29. 12. 1893
                  datiert werden.}}}\label{K_L02725-1h} Dir und den Freunden.\pend
           \pstart \spacefill\mbox{P. G.}\pend{}\endnumbering\briefempfaengerindex{Schnitzler, Arthur@\textsc{Schnitzler, Arthur}!zzzGoldmann, Paul@\emph{von Paul Goldmann}!1893-12-281@{{[}28./29.?{]} 12. 1893}|)be}\mylabel{h}\end{ledgroupsized}\begin{anhang}\end{anhang}\newcommand{\dateiname}{L02725}\newcommand{\titel}{Paul Goldmann an Arthur Schnitzler, [28./29.?] 12. 1893}\newcommand{\editorInnen}{Martin Anton Müller und Laura Untner}%% latex-leseansicht-abspann.tex
%% Abspann für die Leseansicht.
%% Der Schalter \ifkorrekturansicht ist bereits durch den Vorspann gesetzt.

%% latex-abspann.tex
%% Gemeinsamer Abspann für Korrekturansicht und Leseansicht.
%% Setzt den Schalter \ifkorrekturansicht voraus (gesetzt in den
%% einbindenden Dateien latex-korrekturansicht-abspann.tex bzw.
%% latex-leseansicht-abspann.tex).
%% ---------------------------------------------------------------

\normalsize

% Das esempio-Environment wird nur in der Leseansicht benötigt
\ifkorrekturansicht\else
\newenvironment{esempio}[3]%
{
    \vspace{1.5ex}
    \rlap{\underline{#1}}
    \par
    \setlength{\parindent}{0cm}
    \nopagebreak
    \leftskip=#2cm
    \rightskip=#3cm
}
{
    \par
}
\fi

\doendnotes{C}
\bigskip
\vfill

\clearpage

\footnotesize

\ifkorrekturansicht
  \lohead{\textsc{register}}
\fi

% theindex-Environment neu definieren ohne reledmac
\makeatletter
\renewenvironment{theindex}{%
  \ifkorrekturansicht
    \section*{\indexname}%
  \else
    \subsubsection*{Index der erwähnten Entitäten}%
  \fi
  \setlength{\parindent}{0pt}%
  \setlength{\parskip}{0pt plus 0.3pt}%
  \let\item\@idxitem
}{%
  \ifkorrekturansicht\clearpage\fi
}
\makeatother

\IfFileExists{\jobname-pw.ind}{\input{\jobname-pw.ind}}{}

% Quellenangabe nur in der Leseansicht
\ifkorrekturansicht\else
% Fallback-Definitionen, falls die .tex-Datei \titel etc. nicht gesetzt hat
\providecommand{\titel}{}
\providecommand{\editorInnen}{}
\providecommand{\dateiname}{\jobname}

\vspace{3cm}

\vfill

\footnotesize
\textsc{Quelle}: \titel. Herausgegeben von {\editorInnen}. In: \emph{Arthur Schnitzler: Briefwechsel mit Autorinnen und Autoren}.
 Digitale Edition, https://schnitzler-briefe.acdh.oeaw.ac.at/{\dateiname}.html (Stand \today)
\fi

\end{document}


      