%% latex-korrekturansicht-vorspann.tex
%% Vorspann für die Korrekturansicht.
%% Lädt die gemeinsame Datei latex-vorspann.tex mit gesetztem Schalter.

\newif\ifkorrekturansicht
\korrekturansichttrue

\input{../tex-inputs/latex-vorspann}


\section[ Paul Goldmann an Arthur Schnitzler, 26. 1. {[}1898{]}]{L02837 Paul Goldmann an Arthur Schnitzler, 26. 1. {[}1898{]}}
\nopagebreak\mylabel{L02837v}
\rehead{ }\normalsize\beginnumbering\briefempfaengerindex{Schnitzler, Arthur@\textsc{Schnitzler, Arthur}!zzzGoldmann, Paul@\emph{von Paul Goldmann}!1898-01-261@{26. 1. {[}1898{]}}|(be}
\toendnotes[C]{\smallbreak\pagebreak[2]}\Standort{DLA, A:Schnitzler, HS.NZ85.1.3168.}
\physDesc{Brief, 1 Blatt, 3 Seiten, 984 Zeichen
\newline{}Handschrift: blaue Tinte, deutsche Kurrent
\newline{}Schnitzler: 1) mit Bleistift das Jahr »98« vermerkt  2) mit rotem Buntstift drei Unterstreichungen}\toendnotes[C]{\smallbreak}
\pstart
           {\pb}\textcolor{gray}{\textbf{\textbf{Frankfurter Zeitung\orgindex{Frankfurter Zeitung@Frankfurter Zeitung|pw}}}}\pend
           
\pstart
           \textcolor{gray}{\textbf{(\begin{otherlanguage}{french}Gazette de Francfort\end{otherlanguage}\orgindex{Frankfurter Zeitung@Frankfurter Zeitung|pw}).}}\pend
           
\pstart
           \textcolor{gray}{\textbf{\textbf{\begin{otherlanguage}{french}Fondateur M.\end{otherlanguage}{ }L. Sonnemann\pwindex{Sonnemann, Leopold 1831-10-29 – 1909-10-30@\textsc{Sonnemann, Leopold} (1831-10-29 – 1909-10-30), \emph{Journalist/Journalistin, Herausgeber/Herausgeberin}|pw}.}}}\pend
           
\pstart
           \begin{otherlanguage}{french}\textcolor{gray}{\textbf{Journal politique, financier,}}\end{otherlanguage}\pend
           
\pstart
           \begin{otherlanguage}{french}\textcolor{gray}{\textbf{commercial et littéraire.}}\end{otherlanguage}\pend
           
\pstart
           \begin{otherlanguage}{french}\textcolor{gray}{\textbf{\textbf{Paraissant trois fois par jour.}}}\end{otherlanguage}\hfill \textsc{Paris\oindex{Paris@\textbf{Paris}, \emph{P.PPLC}|pw}}, 26. Januar. \pend
           
\pstart
           \begin{otherlanguage}{french}\textcolor{gray}{\textbf{\textbf{Bureau à Paris\oindex{Paris@\textbf{Paris}, \emph{P.PPLC}|pw}}}}\end{otherlanguage}\pend
           
\pstart
           \begin{otherlanguage}{french}\textcolor{gray}{\textbf{\textbf{10 \so{Rue de la Bourse}\oindex{rue de la Bourse@\textbf{rue de la Bourse}, \emph{Straße (K.STR)}|pw}.}}}\end{otherlanguage}\pend
           \vspace{0.5em}
\pstart
           Tauſend Dank, liebſter Freund, für Deinen \label{K_L02837-1v}\edtext{Schritt bei \textsc{Brahm\pwindex{Brahm, Otto 05.02.1856 – 28.11.1912@\textsc{Brahm, Otto} (05.02.1856 – 28.11.1912), \emph{Theaterleiter/Theaterleiterin, Regisseur/Regisseurin}|pw}}}{\lemma{\textnormal{\emph{Schritt bei Brahm}}}\Cendnote{\textnormal{Siehe Vally Rosengart an Arthur Schnitzler, [16. 1. 1898].
               }}}\label{K_L02837-1}. Natürlich iſt Alles vergeblich. Nie bekomme ich dieſe Stelle\orgindex{Vossische Zeitung@Vossische Zeitung|pwv}. Erſtens paſſe ich nicht in
                  dieſe{[}s{]} temperamentsloſe Spießbürger-Blatt\orgindex{Vossische Zeitung@Vossische Zeitung|pwv} hinein. Zweitens nehmen die Leute keinen
               Juden. Drittens: Wer bin ich? Wer kennt mich? Bin ich eine literariſche
               Perſönlichkeit? Ich bin ein »Journaliſt«! Frag’ nur Deinen Freund \textsc{Hugo\pwindex{Hofmannsthal, Hugo von 1874-02-01 – 1929-07-15@\textsc{Hofmannsthal, Hugo von} (1874-02-01 – 1929-07-15), \emph{Schriftsteller/Schriftstellerin}|pw}}!\pend
           
\pstart
           Aber tauſend Dank trotzdem! Es thut mir furchtbar leid, daß meine Leute Dich doch {\pb}mit der Angelegenheit beläſtigt haben.\pend
           
\pstart
           \label{K_L02837-2v}\edtext{\textsc{Bahrs\pwindex{Bahr, Hermann 19.07.1863 – 15.01.1934@\textsc{Bahr, Hermann} (19.07.1863 – 15.01.1934), \emph{Schriftsteller/Schriftstellerin, Kritiker/Kritikerin}|pw}}{ }Artikel\pwindex{Burgtheater. [Demission von Max Burckhard]@\emph{Burgtheater. [Demission von Max Burckhard]}|pwv} über die Burgtheater\orgindex{Burgtheater@Burgtheater|pw}-Kriſis}{\lemma{\textnormal{\emph{Bahrs … Burgtheater-Kriſis}}}\Cendnote{\textnormal{Hermann Bahr\pwindex{Bahr, Hermann 19.07.1863 – 15.01.1934@\textsc{Bahr, Hermann} (19.07.1863 – 15.01.1934), \emph{Schriftsteller/Schriftstellerin, Kritiker/Kritikerin}|pwk}: \emph{Burgtheater}\pwindex{Burgtheater. [Demission von Max Burckhard]@\emph{Burgtheater. [Demission von Max Burckhard]}|pwk}. In: \emph{Die
                        Zeit. Wiener Wochenschrift}\pwindex{Zeit. Wiener Wochenschrift@\emph{Die Zeit. Wiener Wochenschrift}|pwk}, Jg. 14, Nr. 173, 22. 1. 1898, S. 59–60.}}}\label{K_L02837-2} iſt glänzend. Wie ſchade, daß
               dieſes Schwein\pwindex{Bahr, Hermann 19.07.1863 – 15.01.1934@\textsc{Bahr, Hermann} (19.07.1863 – 15.01.1934), \emph{Schriftsteller/Schriftstellerin, Kritiker/Kritikerin}|pwv} Talent hat!
               Wenn man dem \textsc{Prof. Singer\pwindex{Singer, Isidor 16.01.1857 – 08.12.1927@\textsc{Singer, Isidor} (16.01.1857 – 08.12.1927), \emph{Journalist/Journalistin, Herausgeber/Herausgeberin, Soziologe/Soziologin}|pw}} die Meinung über \textsc{Bahr\pwindex{Bahr, Hermann 19.07.1863 – 15.01.1934@\textsc{Bahr, Hermann} (19.07.1863 – 15.01.1934), \emph{Schriftsteller/Schriftstellerin, Kritiker/Kritikerin}|pw}} ſagt, ſo wird er beleidigt. Oder er ſagt: »Schön; aber er wird geleſen!«
               Hübſche Äußerung für den Herausgeber\pwindex{Singer, Isidor 16.01.1857 – 08.12.1927@\textsc{Singer, Isidor} (16.01.1857 – 08.12.1927), \emph{Journalist/Journalistin, Herausgeber/Herausgeberin, Soziologe/Soziologin}|pwv} eines Blatt\pwindex{Zeit. Wiener Wochenschrift@\emph{Die Zeit. Wiener Wochenschrift}|pwv}es, das für Recht und Wahrheit kämpft.\pend
           
\pstart
           {\pb}Was macht Dein \label{K_L02837-3v}\edtext{Stück\pwindex{Vermaechtnis. Schauspiel in drei Akten@\emph{Das Vermächtnis. Schauspiel in drei Akten}|pwv}}{\lemma{\textnormal{\emph{Stück}}}\Cendnote{\textnormal{Siehe Paul Goldmann an Arthur Schnitzler, 19. 1. [1898].
               }}}\label{K_L02837-3}? Iſts fertig? Wann wirds geſpielt?\pend
           
\pstart
           Bitte, bitte, ſchreib’ mir bald! Ich fühle mich ſo einſam!\pend
           
\pstart
           Sei von Herzen gegrüßt!\pend
           
\pstart
           Dein treuer {\\[\baselineskip]}\spacefill\mbox{Paul Goldmn}\pend
           \leftskip=0em{}
\pstart
           \noindent{}Und was ſagſt \strikeout{z} Du zu \label{K_L02837-4v}\edtext{Frankreich\oindex{Frankreich@\textbf{Frankreich}, \emph{A.PCLI}|pw}}{\lemma{\textnormal{\emph{Frankreich}}}\Cendnote{\textnormal{vermutlich Bezug auf die Dreyfus\pwindex{Dreyfus, Alfred 1859-10-09 – 1935-07-12@\textsc{Dreyfus, Alfred} (1859-10-09 – 1935-07-12), \emph{Militär/Militärin}|pwk}-Affäre}}}\label{K_L02837-4}?\pend
           \selectlanguage{ngerman}\endnumbering\briefempfaengerindex{Schnitzler, Arthur@\textsc{Schnitzler, Arthur}!zzzGoldmann, Paul@\emph{von Paul Goldmann}!1898-01-261@{26. 1. {[}1898{]}}|)be}\mylabel{L02837h}  \normalsize

\doendnotes{C}
\bigskip
\vfill

\clearpage

\footnotesize

\lohead{\textsc{register}}

% Definiere theindex-Environment komplett neu ohne reledmac
\makeatletter
\renewenvironment{theindex}{%
  \section*{\indexname}%
  \setlength{\parindent}{0pt}%
  \setlength{\parskip}{0pt plus 0.3pt}%
  \let\item\@idxitem
}{%
  \clearpage
}
\makeatother

\IfFileExists{\jobname-pw.ind}{\input{\jobname-pw.ind}}{}

\end{document}

      