%% latex-korrekturansicht-vorspann.tex
%% Vorspann für die Korrekturansicht.
%% Lädt die gemeinsame Datei latex-vorspann.tex mit gesetztem Schalter.

\newif\ifkorrekturansicht
\korrekturansichttrue

\input{../tex-inputs/latex-vorspann}


\section[Elsa Plessner an Arthur Schnitzler, 14. 4. 1896]{L03700 Elsa Plessner an Arthur Schnitzler, 14. 4. 1896}
\nopagebreak\mylabel{L03700v}
\rehead{ }\normalsize\beginnumbering\briefempfaengerindex{Schnitzler, Arthur@\textsc{Schnitzler, Arthur}!zzzPlessner, Elsa@\emph{von Elsa Plessner}!1896-04-141@{14. 4. 1896}|(be}
\toendnotes[C]{\smallbreak\pagebreak[2]}\Standort{DLA, A:Schnitzler, HS.1985.1.419.}
\physDesc{Brief, 1 Blatt, 3 Seiten, 2061 Zeichen
\newline{}Handschrift: , lateinische Kurrent}\toendnotes[C]{\smallbreak}
\pstart
           {\pb}Wien\oindex{Wien@\textbf{Wien}, \emph{A.ADM2}|pw} , den 14. April 1896\hfill Bäckerstraße N\textsuperscript{o}
                        1\oindex{Baeckerstrasse 1@\textbf{Bäckerstraße 1}, \emph{Wohngebäude (K.WHS)}|pw}.\pend
           
\pstart{}Verehrter Herr Doctor!\pend\vspace{0.5em}
\pstart
           Durch andauernde Unpässlichkeit war ich lange verhindert, Ihnen meinen aufrichtigen
               Dank für die große, große Liebenswürdigkeit auszusprechen, die Sie mir in so reichem
               Maße zu Theil werden lassen. Nun haben Sie mich aber ein wenig verwöhnt und ich wage
               es, Ihrem Wohlwollen eine abermalige Belastungsprobe zuzumuthen. \label{K_L03700-1v}\edtext{Beiliegend}{\lemma{\textnormal{\emph{Beiliegend}}}\Cendnote{\textnormal{Die Beilage ist nicht überliefert. Es handelte sich um einen
                  Entwurf der Novelle \emph{Warten}\pwindex{Warten@\emph{Warten}|pwk}, wie aus den
                  wörtlichen Zitaten hervorgeht.}}}\label{K_L03700-1} übersende Ihnen das Manuscript einer Novelle\pwindex{Warten@\emph{Warten}|pwv}, d. h. blos das Gerüst
               und Gerippe zu einer solchen, indem ich Sie herzlichst bitte, diesen Blättern eine
               doppelt destillierte Aufmerksamkeit zu widmen. Ich glaube nämlich, damit einen etwas
               ungebahnten Weg betreten zu haben und möchte von Ihnen erfahren, ob der eventuelle
               literarische Wert die Kühnheit der Arbeit\pwindex{Warten@\emph{Warten}|pwv} rechtfertigen kann. –\pend
           
\pstart
           Kehren Sie sich bitte, nicht an das, stellenweise etwas tote Papierdeutsch, das {\pb}sich in diesem Entwurfe\pwindex{Warten@\emph{Warten}|pwv}, wie ich ja selbst genau weiß, noch vorfindet, sondern sehen Sie
               die Sache als Ganzes an. Es soll nämlich \label{K_L03700-2v}\edtext{eine gröstere Novelle}{\lemma{\textnormal{\emph{eine gröstere Novelle}}}\Cendnote{\textnormal{Am 15. 9. 1896 sendete Plessner\pwindex{Plessner, Elsa 22.08.1875 – 01.05.1932@\textsc{Plessner, Elsa} (22.08.1875 – 01.05.1932), \emph{Schriftsteller/Schriftstellerin}|pwk}{ }Schnitzler erneut eine überarbeitete Version von
                     \emph{Warten}\pwindex{Warten@\emph{Warten}|pwk} in einem Paket mit anderen Texten
                   und teilte mit, dass sie nicht mehr beabsichtigte, den Text weiter
                  auszuführen, sondern ihn vielmehr als Fragment und »quasi-croquois« zu publizieren
                  gedachte.}}}\label{K_L03700-2} werden, zu deren Ausführung ich mir vorliegende Disposition
               gemacht habe, um den Gang und die Stimmung festzuhalten und theilweise auch den Stil.
               Die Ausführung ist so gedacht, dass, wenn ich z. B. an einer Stelle von dem »\label{K_L03700-3v}\edtext{behäbigen Dutzendbengel}{\lemma{\textnormal{\emph{behäbigen Dutzendbengel}}}\Cendnote{\textnormal{Die Passage lautet im Druck: »O ja – gewiß
                  – mehr wie ein behäbiger Dutzendbengel, der überhaupt ›solche Lämmer‹ gut leiden
                  kann, findet mich ›nett‹ und tanzt mit mir auf dem Subscribtionsball« (Elsa Pessner\pwindex{Plessner, Elsa 22.08.1875 – 01.05.1932@\textsc{Plessner, Elsa} (22.08.1875 – 01.05.1932), \emph{Schriftsteller/Schriftstellerin}|pwk}: \emph{Warten}\pwindex{Warten@\emph{Warten}|pwk}. In: \emph{Der
                        Gläserne Käfig. Skizzen und Novellen}\pwindex{glaeserne Kaefig. Skizzen und Novellen@\emph{Der gläserne Käfig. Skizzen und Novellen}|pwk}. Wien\oindex{Wien@\textbf{Wien}, \emph{A.ADM2}|pwk}: \emph{Leopold Weiss}\orgindex{Leopold Weiss@Leopold Weiss|pwk}{ }1901, S. 39–56, hier S. 43).}}}\label{K_L03700-3}« spreche »der kleine
               Backfische ganz gut leiden mag«, ich dies nicht blos erzählen, sondern scenisch
               ausmalen will.\pend
           
\pstart
           Der »Ich«ton ist, wie ich glaube, der hier einzig mögliche, um die seelischen
               Feinheiten herauszubringen. Die Characterisirung der andern, der Männerfigur lässt
               sich durch die Heldin selbst ganz gut bewerkstelligen, denn sie notirt ja sein Reden
               und Verhalten und hauptsächlich ist es mir doch darum zu thun, die Wirkung {\pb}seiner Person auf sie zu zeigen – und das thut sie ja selbst in diesen
               Aufzeichnunggen! – Nun, Sie werden ja selbst sehen!\pend
           
\pstart
           Und somit danke ich Ihnen, meinem verehrten literarischen Beichtvater, für die
               Geduld, mit der Sie diese Zeilen durchlesen (falls Sie bis hierher kommen) und
               schließe mit nochmaligen Empfehlung dieser Blätter\pwindex{Warten@\emph{Warten}|pwv} an Ihre erwiesene Güte dankbar ergebenst\pend
           \pstart \spacefill\mbox{Elsa Plessner}\pend{}\selectlanguage{ngerman}\endnumbering\briefempfaengerindex{Schnitzler, Arthur@\textsc{Schnitzler, Arthur}!zzzPlessner, Elsa@\emph{von Elsa Plessner}!1896-04-141@{14. 4. 1896}|)be}\mylabel{L03700h}
\begin{anhang}
\end{anhang}\normalsize

\doendnotes{C}
\bigskip
\vfill

\clearpage

\footnotesize

\lohead{\textsc{register}}

% Definiere theindex-Environment komplett neu ohne reledmac
\makeatletter
\renewenvironment{theindex}{%
  \section*{\indexname}%
  \setlength{\parindent}{0pt}%
  \setlength{\parskip}{0pt plus 0.3pt}%
  \let\item\@idxitem
}{%
  \clearpage
}
\makeatother

\IfFileExists{\jobname-pw.ind}{\input{\jobname-pw.ind}}{}

\end{document}

      