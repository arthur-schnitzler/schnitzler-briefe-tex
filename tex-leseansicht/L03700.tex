%% latex-leseansicht-vorspann.tex
%% Vorspann für die Leseansicht.
%% Lädt die gemeinsame Datei latex-vorspann.tex mit nicht gesetztem Schalter.

\newif\ifkorrekturansicht
\korrekturansichtfalse

\input{../tex-inputs/latex-vorspann}


\section[Elsa Plessner an Arthur Schnitzler, 14. 4. 1896]{L03700 Elsa Plessner an Arthur Schnitzler, 14. 4. 1896}
\nopagebreak\mylabel{L03700v}
\rehead{ }\normalsize\beginnumbering\briefempfaengerindex{Schnitzler, Arthur@\textsc{Schnitzler, Arthur}!zzzPlessner, Elsa@\emph{von Elsa Plessner}!1896-04-141@{14. 4. 1896}|(be}
\toendnotes[C]{\smallbreak\pagebreak[2]}
\correspDesc{Versand  durch Elsa Plessner am 14. 4. 1896 in Wien
\newline{}Erhalt  durch Arthur Schnitzler im Zeitraum [14. 4. 1896
                  – 17. 4. 1896?] in Wien}\toendnotes[C]{\smallbreak}
\Standort{DLA, A:Schnitzler, HS.1985.1.419.}
\physDesc{Brief, 1 Blatt, 3 Seiten, 2060 Zeichen
\newline{}Handschrift: schwarze Tinte, lateinische Kurrent}\toendnotes[C]{\smallbreak}
\pstart
           {\pb}Wien\oindex{Wien@\textbf{Wien}, \emph{Verwaltungsgebiet}|pw}, den 14. April 1896\hfill Bäckerstraße N\textsuperscript{o}
                        1\oindex{Wien@\textbf{Wien}!I., Innere Stadt@\textbf{I., Innere Stadt}!Bäckerstraße 1@\textbf{Bäckerstraße 1}, \emph{Wohngebäude}|pw}.\pend
           
\pstart\center{}Verehrter Herr Doctor!\pend\vspace{0.5em}
\pstart
           Durch andauernde Unpässlichkeit war ich lange verhindert, Ihnen meinen aufrichtigen
               Dank für die große, große Liebenswürdigkeit auszusprechen, die Sie mir in so reichem
               Maße zu Theil werden lassen. Nun haben Sie mich aber ein wenig verwöhnt und ich wage
               es, Ihrem Wohlwollen eine abermalige Belastungsprobe zuzumuthen. – \label{K_L03700-1v}\edtext{Beiliegend}{\lemma{\textnormal{\emph{Beiliegend}}}\Cendnote{\textnormal{Die Beilage ist nicht überliefert. Es handelte sich um einen
                  Entwurf der Novelle \emph{Warten}\pwindex{Plessner, Elsa 22.\,8.\,1875 Wien – 7.\,5.\,1932 Alicante@\textsc{Plessner, Elsa} (22.\,8.\,1875 Wien – 7.\,5.\,1932 Alicante), \emph{Schriftstellerin}!Warten. Novelle@\strich\emph{Warten. Novelle}|pwk}, wie aus den
                  wörtlichen Zitaten hervorgeht.}}}\label{K_L03700-1} übersende {[}ich{]} Ihnen das Manuscript einer Novelle\pwindex{Plessner, Elsa 22.\,8.\,1875 Wien – 7.\,5.\,1932 Alicante@\textsc{Plessner, Elsa} (22.\,8.\,1875 Wien – 7.\,5.\,1932 Alicante), \emph{Schriftstellerin}!Warten. Novelle@\strich\emph{Warten. Novelle}|pwv}, d. h. blos das Gerüst
               und Gerippe zu einer solchen, indem ich Sie herzlichst bitte, diesen Blättern eine
               doppelt destillierte Aufmerksamkeit zu widmen. Ich glaube nämlich, damit einen etwas
               ungebahnten Weg betreten zu haben und möchte von Ihnen erfahren, ob der eventuelle
               literarische Wert die Kühnheit der Arbeit\pwindex{Plessner, Elsa 22.\,8.\,1875 Wien – 7.\,5.\,1932 Alicante@\textsc{Plessner, Elsa} (22.\,8.\,1875 Wien – 7.\,5.\,1932 Alicante), \emph{Schriftstellerin}!Warten. Novelle@\strich\emph{Warten. Novelle}|pwv} rechtfertigen kann. –\pend
           
\pstart
           Kehren Sie sich, bitte, nicht an das, stellenweise etwas tote Papierdeutsch, das {\pb}sich in diesem Entwurfe\pwindex{Plessner, Elsa 22.\,8.\,1875 Wien – 7.\,5.\,1932 Alicante@\textsc{Plessner, Elsa} (22.\,8.\,1875 Wien – 7.\,5.\,1932 Alicante), \emph{Schriftstellerin}!Warten. Novelle@\strich\emph{Warten. Novelle}|pwv}, wie ich ja selbst genau weiß, noch vorfindet,
               sondern sehen Sie die Sache als Ganzes an. Es soll nämlich \label{K_L03700-2v}\edtext{eine größere Novelle}{\lemma{\textnormal{\emph{eine größere Novelle}}}\Cendnote{\textnormal{Am XXXX Auszeichnungsfehler: Dokument L03702 nicht gefunden sandte Plessner\pwindex{Plessner, Elsa 22.\,8.\,1875 Wien – 7.\,5.\,1932 Alicante@\textsc{Plessner, Elsa} (22.\,8.\,1875 Wien – 7.\,5.\,1932 Alicante), \emph{Schriftstellerin}|pwk}{ }Schnitzler erneut eine überarbeitete Version
                  von \emph{Warten}\pwindex{Plessner, Elsa 22.\,8.\,1875 Wien – 7.\,5.\,1932 Alicante@\textsc{Plessner, Elsa} (22.\,8.\,1875 Wien – 7.\,5.\,1932 Alicante), \emph{Schriftstellerin}!Warten. Novelle@\strich\emph{Warten. Novelle}|pwk} in einem Paket mit anderen Texten
                  und teilte mit, dass sie nicht mehr beabsichtigte, den Text weiter auszuführen,
                  sondern ihn vielmehr als Fragment zu publizieren
                  dachte.}}}\label{K_L03700-2} werden, zu deren Ausführung ich mir vorliegende Disposition
               gemacht habe, um den Gang und die Stimmung festzuhalten und theilweise auch den Stil.
               Die Ausführung ist so gedacht, dass, wenn ich z. B. an einer Stelle von dem »\label{K_L03700-3v}\edtext{behäbigen Dutzendbengel}{\lemma{\textnormal{\emph{behäbigen Dutzendbengel}}}\Cendnote{\textnormal{Die Passage lautet im Erstdruck: »O ja – gewiß
                  – mehr wie ein behäbiger Dutzendbengel, der überhaupt ›solche Lämmer‹ gut leiden
                  kann, findet mich ›nett‹ und tanzt mit mir auf allen möglichen Bällen«.
                     (E. Pleßner\pwindex{Plessner, Elsa 22.\,8.\,1875 Wien – 7.\,5.\,1932 Alicante@\textsc{Plessner, Elsa} (22.\,8.\,1875 Wien – 7.\,5.\,1932 Alicante), \emph{Schriftstellerin}|pwk}: \emph{Warten. Novelle}\pwindex{Plessner, Elsa 22.\,8.\,1875 Wien – 7.\,5.\,1932 Alicante@\textsc{Plessner, Elsa} (22.\,8.\,1875 Wien – 7.\,5.\,1932 Alicante), \emph{Schriftstellerin}!Warten. Novelle@\strich\emph{Warten. Novelle}|pwk}. In: \emph{Magazin für Litteratur}\pwindex{Magazin für die Literatur des Auslandes@\emph{Magazin für die Literatur des Auslandes}|pwk}, Jg. 66, Nr. 29, 24. 7. 1897, Sp. 867–875, hier: 869.)
                     In der Erstausgabe wird »auf dem Subscribtionsball« getanzt.  (Elsa Pessner\pwindex{Plessner, Elsa 22.\,8.\,1875 Wien – 7.\,5.\,1932 Alicante@\textsc{Plessner, Elsa} (22.\,8.\,1875 Wien – 7.\,5.\,1932 Alicante), \emph{Schriftstellerin}|pwk}: \emph{Warten}\pwindex{Plessner, Elsa 22.\,8.\,1875 Wien – 7.\,5.\,1932 Alicante@\textsc{Plessner, Elsa} (22.\,8.\,1875 Wien – 7.\,5.\,1932 Alicante), \emph{Schriftstellerin}!Warten. Novelle@\strich\emph{Warten. Novelle}|pwk}. In: \emph{Der
                        Gläserne Käfig. Skizzen und Novellen}\pwindex{Plessner, Elsa 22.\,8.\,1875 Wien – 7.\,5.\,1932 Alicante@\textsc{Plessner, Elsa} (22.\,8.\,1875 Wien – 7.\,5.\,1932 Alicante), \emph{Schriftstellerin}!gläserne Käfig. Skizzen und Novellen@\strich\emph{Der gläserne Käfig. Skizzen und Novellen}|pwk}. Wien\oindex{Wien@\textbf{Wien}, \emph{Verwaltungsgebiet}|pwk}: \emph{Leopold Weiss}\orgindex{Leopold Weiss@Leopold Weiss|pwk}{ }1901, S. 39–56, hier S. 43.)}}}\label{K_L03700-3}« spreche »der kleine
               Backfische ganz gut leiden mag«, ich dies nicht blos erzählen, sondern scenisch
               ausmalen will.\pend
           
\pstart
           Der »Ich«ton ist, wie ich glaube, der hier einzig mögliche, um die seelischen
               Feinheiten herauszubringen. Die Characterisirung der andern, der Männerfigur lässt
               sich durch die Heldin selbst ganz gut bewerkstelligen, denn sie notirt ja sein Reden
               und Verhalten und hauptsächlich ist es mir doch darum zu thun, die Wirkung {\pb}seiner Person auf sie zu zeigen – und das thut sie ja
               selbst in diesen Aufzeichnungen! – Nun, Sie werden ja selbst sehen!\pend
           
\pstart
           Und somit danke ich Ihnen, meinem verehrten literarischen Beichtvater, für die
               Geduld, mit der Sie diese Zeilen durchlesen (falls Sie bis hierher kommen) und
               schließe mit nochmaliger Empfehlung dieser Blätter\pwindex{Plessner, Elsa 22.\,8.\,1875 Wien – 7.\,5.\,1932 Alicante@\textsc{Plessner, Elsa} (22.\,8.\,1875 Wien – 7.\,5.\,1932 Alicante), \emph{Schriftstellerin}!Warten. Novelle@\strich\emph{Warten. Novelle}|pwv} an Ihre erwiesene Güte dankbar ergebenst\pend
           \pstart \spacefill\mbox{Elsa Plessner.}\pend{}\selectlanguage{ngerman}\endnumbering\briefempfaengerindex{Schnitzler, Arthur@\textsc{Schnitzler, Arthur}!zzzPlessner, Elsa@\emph{von Elsa Plessner}!1896-04-141@{14. 4. 1896}|)be}\mylabel{L03700h}  \newcommand{\dateiname}{L03700}\newcommand{\titel}{Elsa Plessner an Arthur Schnitzler, 14. 4. 1896}\newcommand{\editorInnen}{Selma Jahnke und Martin Anton Müller}%% latex-leseansicht-abspann.tex
%% Abspann für die Leseansicht.
%% Der Schalter \ifkorrekturansicht ist bereits durch den Vorspann gesetzt.

%% latex-abspann.tex
%% Gemeinsamer Abspann für Korrekturansicht und Leseansicht.
%% Setzt den Schalter \ifkorrekturansicht voraus (gesetzt in den
%% einbindenden Dateien latex-korrekturansicht-abspann.tex bzw.
%% latex-leseansicht-abspann.tex).
%% ---------------------------------------------------------------

\normalsize

% Das esempio-Environment wird nur in der Leseansicht benötigt
\ifkorrekturansicht\else
\newenvironment{esempio}[3]%
{
    \vspace{1.5ex}
    \rlap{\underline{#1}}
    \par
    \setlength{\parindent}{0cm}
    \nopagebreak
    \leftskip=#2cm
    \rightskip=#3cm
}
{
    \par
}
\fi

\doendnotes{C}
\bigskip
\vfill

\clearpage

\footnotesize

\ifkorrekturansicht
  \lohead{\textsc{register}}
\fi

% theindex-Environment neu definieren ohne reledmac
\makeatletter
\renewenvironment{theindex}{%
  \ifkorrekturansicht
    \section*{\indexname}%
  \else
    \subsubsection*{Index der erwähnten Entitäten}%
  \fi
  \setlength{\parindent}{0pt}%
  \setlength{\parskip}{0pt plus 0.3pt}%
  \let\item\@idxitem
}{%
  \ifkorrekturansicht\clearpage\fi
}
\makeatother

\IfFileExists{\jobname-pw.ind}{\input{\jobname-pw.ind}}{}

% Quellenangabe nur in der Leseansicht
\ifkorrekturansicht\else
% Fallback-Definitionen, falls die .tex-Datei \titel etc. nicht gesetzt hat
\providecommand{\titel}{}
\providecommand{\editorInnen}{}
\providecommand{\dateiname}{\jobname}

\vspace{3cm}

\vfill

\footnotesize
\textsc{Quelle}: \titel. Herausgegeben von {\editorInnen}. In: \emph{Arthur Schnitzler: Briefwechsel mit Autorinnen und Autoren}.
 Digitale Edition, https://schnitzler-briefe.acdh.oeaw.ac.at/{\dateiname}.html (Stand \today)
\fi

\end{document}


