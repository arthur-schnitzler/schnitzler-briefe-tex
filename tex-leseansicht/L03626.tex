%% latex-korrekturansicht-vorspann.tex
%% Vorspann für die Korrekturansicht.
%% Lädt die gemeinsame Datei latex-vorspann.tex mit gesetztem Schalter.

\newif\ifkorrekturansicht
\korrekturansichttrue

\input{../tex-inputs/latex-vorspann}


\section[Stefan Zweig an Arthur Schnitzler, {[}5. oder 6. 10. 1910?{]}]{L03626 Stefan Zweig an Arthur Schnitzler, {[}5. oder 6. 10. 1910?{]}}
\nopagebreak\mylabel{L03626v}
\rehead{ }\normalsize\beginnumbering\briefempfaengerindex{Schnitzler, Arthur@\textsc{Schnitzler, Arthur}!zzzZweig, Stefan@\emph{von Stefan Zweig}!1910-10-061@{{[}5. oder
                  6. 10. 1910?{]}}|(be}
\toendnotes[C]{\smallbreak\pagebreak[2]}\Standort{CUL, Schnitzler, B 118.}
\physDesc{Brief, 1 Blatt, 2 Seiten, 929 Zeichen
\newline{}Handschrift: lila Tinte, lateinische Kurrent
\newline{}Schnitzler: mit Bleistift »\textsc{Zweig}« }
\buchAbdrucke{\weitereDrucke{Stefan Zweig: \emph{Briefwechsel mit Hermann Bahr, Sigmund Freud, Rainer Maria
                        Rilke und Arthur Schnitzler}. Frankfurt am Main: \emph{S. Fischer} 1987, S. 359–360.} }\toendnotes[C]{\smallbreak}
\pstart
           {\pb}\textcolor{gray}{\textbf{SZ}}\hfill \textcolor{gray}{\textbf{VIII. KOCHGASSE 8\oindex{Kochgasse 8@\textbf{Kochgasse 8}, \emph{Wohngebäude (K.WHS)}|pw}}}\pend
           
\pstart
           \raggedleft{}\textcolor{gray}{\textbf{WIEN\oindex{Wien@\textbf{Wien}, \emph{A.ADM2}|pw},}}\pend
           
\pstart{}Sehr verehrter Herr Doktor,\pend\vspace{0.5em}
\pstart
           als Sie Ihr schönes \label{K_L03626-1v}\edtext{Haus\oindex{Sternwartestrasse 71@\textbf{Sternwartestraße 71}, \emph{Wohngebäude (K.WHS)}|pwv} bezogen}{\lemma{\textnormal{\emph{Haus bezogen}}}\Cendnote{\textnormal{Am 16. 7. 1910 übersiedelte
                     Arthur Schnitzler mit seiner Frau Olga\pwindex{Schnitzler, Olga 17.01.1882 – 13.01.1970@\textsc{Schnitzler, Olga} (17.01.1882 – 13.01.1970), \emph{Schauspieler/Schauspielerin, Sänger/Sängerin}|pwk} und den Kindern Heinrich\pwindex{Schnitzler, Heinrich 09.08.1902 – 12.07.1982@\textsc{Schnitzler, Heinrich} (09.08.1902 – 12.07.1982), \emph{Regisseur/Regisseurin, Schauspieler/Schauspielerin}|pwk} und Lili\pwindex{Cappellini, Lili 13.09.1909 – 26.07.1928@\textsc{Cappellini, Lili} (13.09.1909 – 26.07.1928)|pwk} in
                  die selbst erworbene Villa in der Sternwartestraße 71\oindex{Sternwartestrasse 71@\textbf{Sternwartestraße 71}, \emph{Wohngebäude (K.WHS)}|pwk}.}}}\label{K_L03626-1} und ich es zum erstenmal \label{K_L03626-2v}\edtext{sehen durfte}{\lemma{\textnormal{\emph{sehen durfte}}}\Cendnote{\textnormal{Am 12. 9. 1910 dokumentiert Schnitzler einen Besuch Stefan
                     Zweigs\pwindex{Zweig, Stefan 28.11.1881 – 23.02.1942@\textsc{Zweig, Stefan} (28.11.1881 – 23.02.1942), \emph{Schriftsteller/Schriftstellerin}|pwk} im \emph{Tagebuch}\pwindex{Tagebuch@\emph{Tagebuch}|pwk}.}}}\label{K_L03626-2}, sagte ich
               Ihnen von dem kleinen Schmuckstück, das ich mir dafür ausgedacht hatte. Es sollte der
                  \label{K_L03626-3v}\edtext{\uline{Hausspruch}\pwindex{Hausspruch von Goethe]@\emph{[Hausspruch von Goethe]}|pwv} von \uline{Goethe}\pwindex{Goethe, Johann Wolfgang von 1749-08-28 – 1832-03-22@\textsc{Goethe, Johann Wolfgang von} (1749-08-28 – 1832-03-22), \emph{Schriftsteller/Schriftstellerin}|pw}}{\lemma{\textnormal{\emph{Hausspruch von Goethe}}}\Cendnote{\textnormal{Dieser Brief stellt das Begleitschreiben
                  zu einem von Johann Peter Eckermann\pwindex{Eckermann, Johann Peter 21.09.1792 – 03.12.1854@\textsc{Eckermann, Johann Peter} (21.09.1792 – 03.12.1854), \emph{Sekretär/Sekretärin}|pwk}
                  zertifizierten Goethe\pwindex{Goethe, Johann Wolfgang von 1749-08-28 – 1832-03-22@\textsc{Goethe, Johann Wolfgang von} (1749-08-28 – 1832-03-22), \emph{Schriftsteller/Schriftstellerin}|pwk}autograf\pwindex{Hausspruch von Goethe]@\emph{[Hausspruch von Goethe]}|pwkv} dar,
                  das derzeit (2025) verschollen ist und nicht autopsiert werden konnte. Auf Fotografien von Schnitzlers Arbeitszimmer hängt es – neben anderen Goethe\pwindex{Goethe, Johann Wolfgang von 1749-08-28 – 1832-03-22@\textsc{Goethe, Johann Wolfgang von} (1749-08-28 – 1832-03-22), \emph{Schriftsteller/Schriftstellerin}|pwk}-Memorabilien – an der Wand über dem
                  Stehpult. Die Herausgeber der ersten Edition der Korrespondenz Schnitzler–Zweig\pwindex{Zweig, Stefan 28.11.1881 – 23.02.1942@\textsc{Zweig, Stefan} (28.11.1881 – 23.02.1942), \emph{Schriftsteller/Schriftstellerin}|pwk}
                  zitierten den Inhalt des Autografs, gaben aber keine Auskunft über ihre Quelle.
                  Sie schrieben: »Vorderseite:\pwindex{Hausspruch von Goethe]@\emph{[Hausspruch von Goethe]}|pwv}{ / }Gott segne das Haus\pwindex{Hausspruch von Goethe]@\emph{[Hausspruch von Goethe]}|pwv}{ / }Zweymal rannt ich
                           heraus,\pwindex{Hausspruch von Goethe]@\emph{[Hausspruch von Goethe]}|pwv}{ / }Denn zweymal ist’s
                           abgebrannt,\pwindex{Hausspruch von Goethe]@\emph{[Hausspruch von Goethe]}|pwv}{ / }Komm ich zum drittenmal
                           gerannt,\pwindex{Hausspruch von Goethe]@\emph{[Hausspruch von Goethe]}|pwv}{ / }Da segne Gott meinen
                           Lauf,\pwindex{Hausspruch von Goethe]@\emph{[Hausspruch von Goethe]}|pwv}{ / }Ich bau’s warlich nicht
                           wieder auf.\pwindex{Hausspruch von Goethe]@\emph{[Hausspruch von Goethe]}|pwv}{ / }Was mehr ist als eine
                           Laus\pwindex{Hausspruch von Goethe]@\emph{[Hausspruch von Goethe]}|pwv}{ / }Trage du in’s Haus.\pwindex{Hausspruch von Goethe]@\emph{[Hausspruch von Goethe]}|pwv}{ / }Daß obige Zeilen von Goethes\pwindex{Goethe, Johann Wolfgang von 1749-08-28 – 1832-03-22@\textsc{Goethe, Johann Wolfgang von} (1749-08-28 – 1832-03-22), \emph{Schriftsteller/Schriftstellerin}|pw} eigener
                        Hand geschrieben sind, bezeuge ich hiemit. Weimar\oindex{Weimar@\textbf{Weimar}, \emph{A.ADM3}|pw} d. 16: April 1851. J. P. Eckermann\pwindex{Eckermann, Johann Peter 21.09.1792 – 03.12.1854@\textsc{Eckermann, Johann Peter} (21.09.1792 – 03.12.1854), \emph{Sekretär/Sekretärin}|pw}.{ / }Rückseite (von Schnitzler recherchiert
                        und aufgeklebt): Annalen oder Tages- u-
                           Jahreshefte\pwindex{Tag- und Jahreshefte@\emph{Tag- und Jahreshefte}|pw}{ / }1801.{ / }›In Pyrmont\oindex{Bad Pyrmont@\textbf{Bad Pyrmont}, \emph{P.PPL}|pw} bezog ich eine schöne ruhige gegen das
                           Ende des Orts liegende Wohnung bei dem Brunnencassierer {\dots}{[}‹{]} (folgen Bemerkungen über Brunnengäste,
                           Bekanntschaften, Wetterberichte et cet.) ›Der Flusspfad nach LUEDGE\oindex{Luegde@\textbf{Lügde}, \emph{A.ADM4}|pw} zwischen abgeschränkten Weidenplätzen her, ward öfters
                           zurückgelegt. In dem Oertchen, das einigemal abgebrannt war, erregte eine
                           desparate Hausinschrift unsere Aufmerksamkeit, die
                              lautete:{[}‹{]}\pwindex{Hausspruch von Goethe]@\emph{[Hausspruch von Goethe]}|pwv}« (Stefan Zweig\pwindex{Zweig, Stefan 28.11.1881 – 23.02.1942@\textsc{Zweig, Stefan} (28.11.1881 – 23.02.1942), \emph{Schriftsteller/Schriftstellerin}|pwk}: \emph{Briefwechsel mit Hermann Bahr, Sigmund Freud, Rainer Maria Rilke und Arthur
                        Schnitzler}, S. 455–456.)}}}\label{K_L03626-3} in seiner Handschrift sein, und wirklich glückte es mir, ihn zu
               erlangen.\pend
           
\pstart
           Nun ist \introOben{}er\introOben{} freilich nicht ein Edelspruch Goethes\pwindex{Goethe, Johann Wolfgang von 1749-08-28 – 1832-03-22@\textsc{Goethe, Johann Wolfgang von} (1749-08-28 – 1832-03-22), \emph{Schriftsteller/Schriftstellerin}|pw}, sondern eher einer wo sein Genius geschlafen hat:
               aber immerhin, nehmen Sie nur die erste Zeile davon als den Wunsch eines
               Erlauchtesten für Ihr Haus\oindex{Sternwartestrasse 71@\textbf{Sternwartestraße 71}, \emph{Wohngebäude (K.WHS)}|pwv}
               und {\pb}möge er sich – aber nur die erste
               Zeile! – erfüllen. Ich hätte ihn rahmen lassen, wüsste ich, wie und wo Sie ihn
               placieren wollten: \label{K_L03626-4v}\edtext{nehmen Sie ihn \substVorne{}\textsuperscript{aber}\substDazwischen{}nun\substHinten{}}{\lemma{\textnormal{\emph{nehmen Sie ihn nun}}}\Cendnote{\textnormal{Im \emph{Tagebuch}\pwindex{Tagebuch@\emph{Tagebuch}|pwk}-Eintrag zum 6. 10. 1910 notiert Schnitzler
                  den Erhalt des Autografen. Seinen Dank  spricht er »gleich« aus (Arthur Schnitzler an Stefan Zweig, 6. 10. 1910). Da für gewöhnlich
                  die Post innerhalb Wiens\oindex{Wien@\textbf{Wien}, \emph{A.ADM2}|pwk} noch am selben oder am Folgetag zugestellt wurde, ist anzunehmen,
                  dass das vorliegende, undatierte Korrespondenzstück auf den 5. oder
                     6. 10. 1910 zu datieren ist.}}}\label{K_L03626-4} so als einen Dank für Vieles, für das
               Schöne, \substVorne{}\textsuperscript{w}\substDazwischen{}d\substHinten{}as ich von Ihnen mit vielen andern aus Ihren Büchern, für das Schöne, \substVorne{}\textsuperscript{w}\substDazwischen{}d\substHinten{}as ich von Ihnen allein durch \label{K_L03626-5v}\edtext{Ihr Manuscript\pwindex{Ruf des Lebens. Schauspiel in drei Akten@\emph{Der Ruf des Lebens. Schauspiel in drei Akten}|pwv}}{\lemma{\textnormal{\emph{Ihr Manuscript}}}\Cendnote{\textnormal{Stefan Zweig\pwindex{Zweig, Stefan 28.11.1881 – 23.02.1942@\textsc{Zweig, Stefan} (28.11.1881 – 23.02.1942), \emph{Schriftsteller/Schriftstellerin}|pwk} sammelte Handschriften. Am
                     28. 12. 1909
                  vermerkte Schnitzler im \emph{Tagebuch}\pwindex{Tagebuch@\emph{Tagebuch}|pwk}, er habe dem »Sammler Zweig\pwindex{Zweig, Stefan 28.11.1881 – 23.02.1942@\textsc{Zweig, Stefan} (28.11.1881 – 23.02.1942), \emph{Schriftsteller/Schriftstellerin}|pw} Urform des ›Ruf
                        des Lebens\pwindex{Ruf des Lebens. Schauspiel in drei Akten@\emph{Der Ruf des Lebens. Schauspiel in drei Akten}|pw}‹ geschenkt«. Es wird heute im \emph{Theatermuseum} in Wien\oindex{Wien@\textbf{Wien}, \emph{A.ADM2}|pwk} verwahrt.}}}\label{K_L03626-5}, Ihr Bild, vor
               allem aber manches gute Gespräch und Ihre Güte empfangen habe. In herzlicher Liebe und Verehrung Ihr getreuer \pend
           \pstart \spacefill\mbox{Stefan Zweig}\pend{}\selectlanguage{ngerman}\endnumbering\briefempfaengerindex{Schnitzler, Arthur@\textsc{Schnitzler, Arthur}!zzzZweig, Stefan@\emph{von Stefan Zweig}!1910-10-051@{{[}5. oder
                  6. 10. 1910?{]}}|)be}\mylabel{L03626h}  \normalsize

\doendnotes{C}
\bigskip
\vfill

\clearpage

\footnotesize

\lohead{\textsc{register}}

% Definiere theindex-Environment komplett neu ohne reledmac
\makeatletter
\renewenvironment{theindex}{%
  \section*{\indexname}%
  \setlength{\parindent}{0pt}%
  \setlength{\parskip}{0pt plus 0.3pt}%
  \let\item\@idxitem
}{%
  \clearpage
}
\makeatother

\IfFileExists{\jobname-pw.ind}{\input{\jobname-pw.ind}}{}

\end{document}

      