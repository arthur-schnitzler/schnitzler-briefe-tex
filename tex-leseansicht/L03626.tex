%% latex-leseansicht-vorspann.tex
%% Vorspann für die Leseansicht.
%% Lädt die gemeinsame Datei latex-vorspann.tex mit nicht gesetztem Schalter.

\newif\ifkorrekturansicht
\korrekturansichtfalse

\input{../tex-inputs/latex-vorspann}


\section[Stefan Zweig an Arthur Schnitzler, {{[}}5. oder 6. 10. 1910?{{]}}]{L03626 Stefan Zweig an Arthur Schnitzler, {[}5. oder 6. 10. 1910?{]}}
\nopagebreak\mylabel{L03626v}
\rehead{ }\normalsize\beginnumbering\briefempfaengerindex{Schnitzler, Arthur@\textsc{Schnitzler, Arthur}!zzzZweig, Stefan@\emph{von Stefan Zweig}!1910-10-061@{{[}5. oder 6. 10. 1910?{]}}|(be}
\toendnotes[C]{\smallbreak\pagebreak[2]}
\correspDesc{Versand  durch Stefan Zweig im Zeitraum [5. oder
                  6. 10. 1910?] in Wien
\newline{}Erhalt  durch Arthur Schnitzler im Zeitraum [5. 10. 1910
                  – 8. 10. 1910?] in Wien}\toendnotes[C]{\smallbreak}
\Standort{CUL, Schnitzler, B 118.}
\physDesc{Brief, 1 Blatt, 2 Seiten, 929 Zeichen
\newline{}Handschrift: lila Tinte, lateinische Kurrent
\newline{}Schnitzler: mit Bleistift »\textsc{Zweig}« }
\buchAbdrucke{\weitereDrucke{Stefan Zweig: \emph{Briefwechsel mit Hermann Bahr, Sigmund Freud, Rainer Maria
                        Rilke und Arthur Schnitzler}. Herausgegeben von Jeffrey B. Berlin, Hans-Ulrich Lindken und Donald A. Prater. Frankfurt am Main: \emph{S. Fischer} 1987, S. 359–360.} }\toendnotes[C]{\smallbreak}
\pstart
           {\pb}\textcolor{gray}{\textbf{SZ}}\hfill \textcolor{gray}{\textbf{VIII. KOCHGASSE 8\oindex{Wien@\textbf{Wien}!VIII., Josefstadt@\textbf{VIII., Josefstadt}!Kochgasse 8@\textbf{Kochgasse 8}, \emph{Wohngebäude}|pw}}}\pend
           
\pstart
           \raggedleft{}\textcolor{gray}{\textbf{WIEN\oindex{Wien@\textbf{Wien}, \emph{Verwaltungsgebiet}|pw},}}\pend
           
\pstart{}Sehr verehrter Herr Doktor,\pend\vspace{0.5em}
\pstart
           als Sie Ihr schönes \label{K_L03626-1v}\edtext{Haus\oindex{Wien@\textbf{Wien}!XVIII., Währing@\textbf{XVIII., Währing}!Sternwartestraße 71@\textbf{Sternwartestraße 71}, \emph{Wohngebäude}|pwv} bezogen}{\lemma{\textnormal{\emph{Haus bezogen}}}\Cendnote{\textnormal{Am 16. 7. 1910 übersiedelte
                     Arthur Schnitzler mit seiner Frau Olga\pwindex{Schnitzler, Olga 17.\,1.\,1882 Wien – 13.\,1.\,1970 Lugano@\textsc{Schnitzler, Olga} (17.\,1.\,1882 Wien – 13.\,1.\,1970 Lugano), \emph{Schauspielerin, Sängerin}|pwk} und den Kindern Heinrich\pwindex{Schnitzler, Heinrich 9.\,8.\,1902 Hinterbrühl – 12.\,7.\,1982 Wien@\textsc{Schnitzler, Heinrich} (9.\,8.\,1902 Hinterbrühl – 12.\,7.\,1982 Wien), \emph{Regisseur, Schauspieler}|pwk} und Lili\pwindex{Cappellini, Lili 13.\,9.\,1909 Wien – 26.\,7.\,1928 Venedig@\textsc{Cappellini, Lili} (13.\,9.\,1909 Wien – 26.\,7.\,1928 Venedig)|pwk} in
                  die selbst erworbene Villa in der Sternwartestraße 71\oindex{Wien@\textbf{Wien}!XVIII., Währing@\textbf{XVIII., Währing}!Sternwartestraße 71@\textbf{Sternwartestraße 71}, \emph{Wohngebäude}|pwk}.}}}\label{K_L03626-1} und ich es zum erstenmal \label{K_L03626-2v}\edtext{sehen durfte}{\lemma{\textnormal{\emph{sehen durfte}}}\Cendnote{\textnormal{Am 12. 9. 1910 dokumentiert Schnitzler einen Besuch Stefan
                     Zweigs\pwindex{Zweig, Stefan 28.\,11.\,1881 Wien – 23.\,2.\,1942 Petrópolis@\textsc{Zweig, Stefan} (28.\,11.\,1881 Wien – 23.\,2.\,1942 Petrópolis), \emph{Schriftsteller}|pwk} im \emph{Tagebuch}\pwindex{Schnitzler, Arthur 15.\,5.\,1862 Wien – 21.\,10.\,1931 ebd.@\textsc{Schnitzler, Arthur} (15.\,5.\,1862 Wien – 21.\,10.\,1931 ebd.), \emph{Schriftsteller, Mediziner}!Tagebuch@\strich\emph{Tagebuch}|pwk}.}}}\label{K_L03626-2}, sagte ich
               Ihnen von dem kleinen Schmuckstück, das ich mir dafür ausgedacht hatte. Es sollte der
                  \label{K_L03626-3v}\edtext{\uline{Hausspruch}\pwindex{Goethe, Johann Wolfgang von 28.\,8.\,1749 Frankfurt am Main – 22.\,3.\,1832 Weimar@\textsc{Goethe, Johann Wolfgang von} (28.\,8.\,1749 Frankfurt am Main – 22.\,3.\,1832 Weimar), \emph{Schriftsteller}!Hausspruch von Goethe]@\strich\emph{[Hausspruch von Goethe]}|pwv} von \uline{Goethe}\pwindex{Goethe, Johann Wolfgang von 28.\,8.\,1749 Frankfurt am Main – 22.\,3.\,1832 Weimar@\textsc{Goethe, Johann Wolfgang von} (28.\,8.\,1749 Frankfurt am Main – 22.\,3.\,1832 Weimar), \emph{Schriftsteller}|pw}}{\lemma{\textnormal{\emph{Hausspruch von Goethe}}}\Cendnote{\textnormal{Dieser Brief stellt das Begleitschreiben
                  zu einem von Johann Peter Eckermann\pwindex{Eckermann, Johann Peter 21.\,9.\,1792 Winsen – 3.\,12.\,1854 Weimar@\textsc{Eckermann, Johann Peter} (21.\,9.\,1792 Winsen – 3.\,12.\,1854 Weimar), \emph{Sekretär}|pwk}
                  zertifizierten Goethe\pwindex{Goethe, Johann Wolfgang von 28.\,8.\,1749 Frankfurt am Main – 22.\,3.\,1832 Weimar@\textsc{Goethe, Johann Wolfgang von} (28.\,8.\,1749 Frankfurt am Main – 22.\,3.\,1832 Weimar), \emph{Schriftsteller}|pwk}autograf\pwindex{Goethe, Johann Wolfgang von 28.\,8.\,1749 Frankfurt am Main – 22.\,3.\,1832 Weimar@\textsc{Goethe, Johann Wolfgang von} (28.\,8.\,1749 Frankfurt am Main – 22.\,3.\,1832 Weimar), \emph{Schriftsteller}!Hausspruch von Goethe]@\strich\emph{[Hausspruch von Goethe]}|pwkv} dar, das derzeit
                     (2025) verschollen ist und nicht autopsiert werden konnte. Auf
                  Fotografien von Schnitzlers Arbeitszimmer
                  hängt es – neben anderen Goethe\pwindex{Goethe, Johann Wolfgang von 28.\,8.\,1749 Frankfurt am Main – 22.\,3.\,1832 Weimar@\textsc{Goethe, Johann Wolfgang von} (28.\,8.\,1749 Frankfurt am Main – 22.\,3.\,1832 Weimar), \emph{Schriftsteller}|pwk}-Memorabilien – an der Wand über dem Stehpult. Die Herausgeber der
                  ersten Edition der Korrespondenz Schnitzler–Zweig\pwindex{Zweig, Stefan 28.\,11.\,1881 Wien – 23.\,2.\,1942 Petrópolis@\textsc{Zweig, Stefan} (28.\,11.\,1881 Wien – 23.\,2.\,1942 Petrópolis), \emph{Schriftsteller}|pwk} zitierten den
                  Inhalt des Autografs, gaben aber keine Auskunft über ihre Quelle. Sie schrieben: »Vorderseite:\pwindex{Goethe, Johann Wolfgang von 28.\,8.\,1749 Frankfurt am Main – 22.\,3.\,1832 Weimar@\textsc{Goethe, Johann Wolfgang von} (28.\,8.\,1749 Frankfurt am Main – 22.\,3.\,1832 Weimar), \emph{Schriftsteller}!Hausspruch von Goethe]@\strich\emph{[Hausspruch von Goethe]}|pwv}{ / }Gott segne das Haus\pwindex{Goethe, Johann Wolfgang von 28.\,8.\,1749 Frankfurt am Main – 22.\,3.\,1832 Weimar@\textsc{Goethe, Johann Wolfgang von} (28.\,8.\,1749 Frankfurt am Main – 22.\,3.\,1832 Weimar), \emph{Schriftsteller}!Hausspruch von Goethe]@\strich\emph{[Hausspruch von Goethe]}|pwv}{ / }Zweymal rannt ich
                           heraus,\pwindex{Goethe, Johann Wolfgang von 28.\,8.\,1749 Frankfurt am Main – 22.\,3.\,1832 Weimar@\textsc{Goethe, Johann Wolfgang von} (28.\,8.\,1749 Frankfurt am Main – 22.\,3.\,1832 Weimar), \emph{Schriftsteller}!Hausspruch von Goethe]@\strich\emph{[Hausspruch von Goethe]}|pwv}{ / }Denn zweymal ist’s
                           abgebrannt,\pwindex{Goethe, Johann Wolfgang von 28.\,8.\,1749 Frankfurt am Main – 22.\,3.\,1832 Weimar@\textsc{Goethe, Johann Wolfgang von} (28.\,8.\,1749 Frankfurt am Main – 22.\,3.\,1832 Weimar), \emph{Schriftsteller}!Hausspruch von Goethe]@\strich\emph{[Hausspruch von Goethe]}|pwv}{ / }Komm ich zum drittenmal
                           gerannt,\pwindex{Goethe, Johann Wolfgang von 28.\,8.\,1749 Frankfurt am Main – 22.\,3.\,1832 Weimar@\textsc{Goethe, Johann Wolfgang von} (28.\,8.\,1749 Frankfurt am Main – 22.\,3.\,1832 Weimar), \emph{Schriftsteller}!Hausspruch von Goethe]@\strich\emph{[Hausspruch von Goethe]}|pwv}{ / }Da segne Gott meinen
                           Lauf,\pwindex{Goethe, Johann Wolfgang von 28.\,8.\,1749 Frankfurt am Main – 22.\,3.\,1832 Weimar@\textsc{Goethe, Johann Wolfgang von} (28.\,8.\,1749 Frankfurt am Main – 22.\,3.\,1832 Weimar), \emph{Schriftsteller}!Hausspruch von Goethe]@\strich\emph{[Hausspruch von Goethe]}|pwv}{ / }Ich bau’s warlich nicht
                           wieder auf.\pwindex{Goethe, Johann Wolfgang von 28.\,8.\,1749 Frankfurt am Main – 22.\,3.\,1832 Weimar@\textsc{Goethe, Johann Wolfgang von} (28.\,8.\,1749 Frankfurt am Main – 22.\,3.\,1832 Weimar), \emph{Schriftsteller}!Hausspruch von Goethe]@\strich\emph{[Hausspruch von Goethe]}|pwv}{ / }Was mehr ist als eine
                           Laus\pwindex{Goethe, Johann Wolfgang von 28.\,8.\,1749 Frankfurt am Main – 22.\,3.\,1832 Weimar@\textsc{Goethe, Johann Wolfgang von} (28.\,8.\,1749 Frankfurt am Main – 22.\,3.\,1832 Weimar), \emph{Schriftsteller}!Hausspruch von Goethe]@\strich\emph{[Hausspruch von Goethe]}|pwv}{ / }Trage du in’s Haus.\pwindex{Goethe, Johann Wolfgang von 28.\,8.\,1749 Frankfurt am Main – 22.\,3.\,1832 Weimar@\textsc{Goethe, Johann Wolfgang von} (28.\,8.\,1749 Frankfurt am Main – 22.\,3.\,1832 Weimar), \emph{Schriftsteller}!Hausspruch von Goethe]@\strich\emph{[Hausspruch von Goethe]}|pwv}{ / }Daß obige Zeilen von Goethes\pwindex{Goethe, Johann Wolfgang von 28.\,8.\,1749 Frankfurt am Main – 22.\,3.\,1832 Weimar@\textsc{Goethe, Johann Wolfgang von} (28.\,8.\,1749 Frankfurt am Main – 22.\,3.\,1832 Weimar), \emph{Schriftsteller}|pw} eigener
                        Hand geschrieben sind, bezeuge ich hiemit. Weimar\oindex{Weimar@\textbf{Weimar}, \emph{Verwaltungsgebiet}|pw} d. 16: April 1851. J. P. Eckermann\pwindex{Eckermann, Johann Peter 21.\,9.\,1792 Winsen – 3.\,12.\,1854 Weimar@\textsc{Eckermann, Johann Peter} (21.\,9.\,1792 Winsen – 3.\,12.\,1854 Weimar), \emph{Sekretär}|pw}.{ / }Rückseite (von Schnitzler recherchiert
                        und aufgeklebt): Annalen oder Tages- u-
                           Jahreshefte\pwindex{Goethe, Johann Wolfgang von 28.\,8.\,1749 Frankfurt am Main – 22.\,3.\,1832 Weimar@\textsc{Goethe, Johann Wolfgang von} (28.\,8.\,1749 Frankfurt am Main – 22.\,3.\,1832 Weimar), \emph{Schriftsteller}!Tag- und Jahreshefte@\strich\emph{Tag- und Jahreshefte}|pw}{ / }1801.{ / }›In Pyrmont\oindex{Bad Pyrmont@\textbf{Bad Pyrmont}|pw} bezog ich eine schöne ruhige gegen das
                           Ende des Orts liegende Wohnung bei dem Brunnencassierer {\dots}{[}‹{]} (folgen Bemerkungen über Brunnengäste,
                           Bekanntschaften, Wetterberichte et cet.) ›Der Flusspfad nach LUEDGE\oindex{Lügde@\textbf{Lügde}, \emph{Region}|pw} zwischen abgeschränkten Weidenplätzen her, ward öfters
                           zurückgelegt. In dem Oertchen, das einigemal abgebrannt war, erregte eine
                           desparate Hausinschrift unsere Aufmerksamkeit, die
                              lautete:{[}‹{]}\pwindex{Goethe, Johann Wolfgang von 28.\,8.\,1749 Frankfurt am Main – 22.\,3.\,1832 Weimar@\textsc{Goethe, Johann Wolfgang von} (28.\,8.\,1749 Frankfurt am Main – 22.\,3.\,1832 Weimar), \emph{Schriftsteller}!Tag- und Jahreshefte@\strich\emph{Tag- und Jahreshefte}|pwv}« (Stefan Zweig\pwindex{Zweig, Stefan 28.\,11.\,1881 Wien – 23.\,2.\,1942 Petrópolis@\textsc{Zweig, Stefan} (28.\,11.\,1881 Wien – 23.\,2.\,1942 Petrópolis), \emph{Schriftsteller}|pwk}: \emph{Briefwechsel mit Hermann Bahr, Sigmund Freud, Rainer Maria Rilke und Arthur
                        Schnitzler}, S. 455–456.)}}}\label{K_L03626-3} in seiner Handschrift sein,
               und wirklich glückte es mir, ihn zu erlangen.\pend
           
\pstart
           Nun ist \introOben{}er\introOben{} freilich nicht ein Edelspruch Goethes\pwindex{Goethe, Johann Wolfgang von 28.\,8.\,1749 Frankfurt am Main – 22.\,3.\,1832 Weimar@\textsc{Goethe, Johann Wolfgang von} (28.\,8.\,1749 Frankfurt am Main – 22.\,3.\,1832 Weimar), \emph{Schriftsteller}|pw}, sondern eher einer wo sein Genius geschlafen hat:
               aber immerhin, nehmen Sie nur die erste Zeile davon als den Wunsch eines
               Erlauchtesten für Ihr Haus\oindex{Wien@\textbf{Wien}!XVIII., Währing@\textbf{XVIII., Währing}!Sternwartestraße 71@\textbf{Sternwartestraße 71}, \emph{Wohngebäude}|pwv}
               und {\pb}möge er sich – aber nur die erste
               Zeile! – erfüllen. Ich hätte ihn rahmen lassen, wüsste ich, wie und wo Sie ihn
               placieren wollten: \label{K_L03626-4v}\edtext{nehmen Sie ihn \substVorne{}\textsuperscript{aber}\substDazwischen{}nun\substHinten{}}{\lemma{\textnormal{\emph{nehmen Sie ihn nun}}}\Cendnote{\textnormal{Im \emph{Tagebuch}\pwindex{Schnitzler, Arthur 15.\,5.\,1862 Wien – 21.\,10.\,1931 ebd.@\textsc{Schnitzler, Arthur} (15.\,5.\,1862 Wien – 21.\,10.\,1931 ebd.), \emph{Schriftsteller, Mediziner}!Tagebuch@\strich\emph{Tagebuch}|pwk}-Eintrag zum 6. 10. 1910 notiert Schnitzler
                  den Erhalt des Autografs. Seinen Dank spricht er »gleich« aus
                     (XXXX Auszeichnungsfehler: Dokument L03796 nicht gefunden). Da für gewöhnlich die Post
                  innerhalb Wiens\oindex{Wien@\textbf{Wien}, \emph{Verwaltungsgebiet}|pwk} noch am selben oder am Folgetag
                  zugestellt wurde, ist anzunehmen, dass das vorliegende, undatierte
                  Korrespondenzstück auf den 5. oder 6. 10. 1910 zu
                  datieren ist.}}}\label{K_L03626-4} so als einen Dank für Vieles, für das Schöne, \substVorne{}\textsuperscript{w}\substDazwischen{}d\substHinten{}as ich von Ihnen mit vielen andern aus Ihren Büchern, für das Schöne, \substVorne{}\textsuperscript{w}\substDazwischen{}d\substHinten{}as ich von Ihnen allein durch \label{K_L03626-5v}\edtext{Ihr Manuscript\pwindex{Schnitzler, Arthur 15.\,5.\,1862 Wien – 21.\,10.\,1931 ebd.@\textsc{Schnitzler, Arthur} (15.\,5.\,1862 Wien – 21.\,10.\,1931 ebd.), \emph{Schriftsteller, Mediziner}!Ruf des Lebens. Schauspiel in drei Akten@\strich\emph{Der Ruf des Lebens. Schauspiel in drei Akten}|pwv}}{\lemma{\textnormal{\emph{Ihr Manuscript}}}\Cendnote{\textnormal{Stefan Zweig\pwindex{Zweig, Stefan 28.\,11.\,1881 Wien – 23.\,2.\,1942 Petrópolis@\textsc{Zweig, Stefan} (28.\,11.\,1881 Wien – 23.\,2.\,1942 Petrópolis), \emph{Schriftsteller}|pwk} sammelte Handschriften. Am
                     28. 12. 1909
                  vermerkte Schnitzler im \emph{Tagebuch}\pwindex{Schnitzler, Arthur 15.\,5.\,1862 Wien – 21.\,10.\,1931 ebd.@\textsc{Schnitzler, Arthur} (15.\,5.\,1862 Wien – 21.\,10.\,1931 ebd.), \emph{Schriftsteller, Mediziner}!Tagebuch@\strich\emph{Tagebuch}|pwk}, er habe dem »Sammler Zweig\pwindex{Zweig, Stefan 28.\,11.\,1881 Wien – 23.\,2.\,1942 Petrópolis@\textsc{Zweig, Stefan} (28.\,11.\,1881 Wien – 23.\,2.\,1942 Petrópolis), \emph{Schriftsteller}|pw} Urform des ›Ruf
                        des Lebens\pwindex{Schnitzler, Arthur 15.\,5.\,1862 Wien – 21.\,10.\,1931 ebd.@\textsc{Schnitzler, Arthur} (15.\,5.\,1862 Wien – 21.\,10.\,1931 ebd.), \emph{Schriftsteller, Mediziner}!Ruf des Lebens. Schauspiel in drei Akten@\strich\emph{Der Ruf des Lebens. Schauspiel in drei Akten}|pw}‹ geschenkt«. Es wird heute im \emph{Theatermuseum} in Wien\oindex{Wien@\textbf{Wien}, \emph{Verwaltungsgebiet}|pwk} verwahrt.}}}\label{K_L03626-5}, Ihr Bild, vor
               allem aber manches gute Gespräch und Ihre Güte empfangen habe. In herzlicher Liebe
               und Verehrung Ihr getreuer\pend
           \pstart \spacefill\mbox{Stefan Zweig}\pend{}\selectlanguage{ngerman}\endnumbering\briefempfaengerindex{Schnitzler, Arthur@\textsc{Schnitzler, Arthur}!zzzZweig, Stefan@\emph{von Stefan Zweig}!1910-10-051@{{[}5. oder 6. 10. 1910?{]}}|)be}\mylabel{L03626h}  \newcommand{\dateiname}{L03626}\newcommand{\titel}{Stefan Zweig an Arthur Schnitzler, [5. oder 6. 10. 1910?]}\newcommand{\editorInnen}{Selma Jahnke und Martin Anton Müller}%% latex-leseansicht-abspann.tex
%% Abspann für die Leseansicht.
%% Der Schalter \ifkorrekturansicht ist bereits durch den Vorspann gesetzt.

%% latex-abspann.tex
%% Gemeinsamer Abspann für Korrekturansicht und Leseansicht.
%% Setzt den Schalter \ifkorrekturansicht voraus (gesetzt in den
%% einbindenden Dateien latex-korrekturansicht-abspann.tex bzw.
%% latex-leseansicht-abspann.tex).
%% ---------------------------------------------------------------

\normalsize

% Das esempio-Environment wird nur in der Leseansicht benötigt
\ifkorrekturansicht\else
\newenvironment{esempio}[3]%
{
    \vspace{1.5ex}
    \rlap{\underline{#1}}
    \par
    \setlength{\parindent}{0cm}
    \nopagebreak
    \leftskip=#2cm
    \rightskip=#3cm
}
{
    \par
}
\fi

\doendnotes{C}
\bigskip
\vfill

\clearpage

\footnotesize

\ifkorrekturansicht
  \lohead{\textsc{register}}
\fi

% theindex-Environment neu definieren ohne reledmac
\makeatletter
\renewenvironment{theindex}{%
  \ifkorrekturansicht
    \section*{\indexname}%
  \else
    \subsubsection*{Index der erwähnten Entitäten}%
  \fi
  \setlength{\parindent}{0pt}%
  \setlength{\parskip}{0pt plus 0.3pt}%
  \let\item\@idxitem
}{%
  \ifkorrekturansicht\clearpage\fi
}
\makeatother

\IfFileExists{\jobname-pw.ind}{\input{\jobname-pw.ind}}{}

% Quellenangabe nur in der Leseansicht
\ifkorrekturansicht\else
% Fallback-Definitionen, falls die .tex-Datei \titel etc. nicht gesetzt hat
\providecommand{\titel}{}
\providecommand{\editorInnen}{}
\providecommand{\dateiname}{\jobname}

\vspace{3cm}

\vfill

\footnotesize
\textsc{Quelle}: \titel. Herausgegeben von {\editorInnen}. In: \emph{Arthur Schnitzler: Briefwechsel mit Autorinnen und Autoren}.
 Digitale Edition, https://schnitzler-briefe.acdh.oeaw.ac.at/{\dateiname}.html (Stand \today)
\fi

\end{document}


