%% latex-leseansicht-vorspann.tex
%% Vorspann für die Leseansicht.
%% Lädt die gemeinsame Datei latex-vorspann.tex mit nicht gesetztem Schalter.

\newif\ifkorrekturansicht
\korrekturansichtfalse

\input{../tex-inputs/latex-vorspann}


               \section[Paul Goldmann an Arthur Schnitzler, 29. 11. 1891]{ Paul Goldmann an Arthur Schnitzler, 29. 11. 1891}\nopagebreak\mylabel{v}\rehead{ }\begin{ledgroupsized}[t]{13cm}\normalsize\beginnumbering\briefempfaengerindex{Schnitzler, Arthur@\textsc{Schnitzler, Arthur}!zzzGoldmann, Paul@\emph{von Paul Goldmann}!1891-11-291@{29. 11. 1891}|(be} \toendnotes[C]{\smallbreak\pagebreak[2]} \Standort{DLA, A:Schnitzler, HS.NZ85.1.3162.}
\physDesc{Postkarte
\newline{}Handschrift: 1) schwarze Tinte, deutsche Kurrent\hspace{1em}2) schwarze Tinte, lateinische Kurrent (\noindent{}Adresse)\hspace{1em}\newline{}Versand: 1) Stempel: »\nobreak{}\oindex{Amsterdam@\textbf{Amsterdam}|pwk}Amste\textcolor{gray}{rdam}, 30 Nov 91, 10–11V\nobreak{}«.  2) Stempel: »\nobreak{}Wien 1/1, 2/12. 91, 9½–11V., Bestellt\nobreak{}«. 
\newline{}Schnitzler: mit Bleistift das Datum »30/11 91« vermerkt }\pstart{}{\pb}\begin{otherlanguage}{french}Autriche\end{otherlanguage}\oindex{Oesterreich@\textbf{Österreich}|pw}! \pend{}\pstart{}Herrn \pend{}\pstart{}Dr. Arthur Schnitzler \pend{}\pstart{}Wien\oindex{Wien@\textbf{Wien}|pw}\pend{}\pstart{}I. Giselastraße 11\oindex{Boesendorferstrasse@\textbf{Bösendorferstraße}|pw}.\pend{}{\bigskip}\pstart
           \centering{}{\pb}Amſterdam\oindex{Amsterdam@\textbf{Amsterdam}|pw}, 29. November\pend
           \pstart
           Mein lieber Arthur! So ein Bildernarr bin ich
               geworden, daß ich noch im Fluge zwei Tage zufammengerafft habe, um in \textsc{Haarlem\oindex{Haarlem@\textbf{Haarlem}|pw}} die \textsc{Frans Hals\pwindex{Hals, Frans zwischen 1580 und 1585 – 1666-08-16@\textsc{Hals, Frans} (zwischen 1580 und 1585 – 1666-08-16), \emph{Bildender Künstler}|pw}} und in \textsc{Amsterdam\oindex{Amsterdam@\textbf{Amsterdam}|pw}} die \textsc{Rembrandt\pwindex{Rembrandt van Rijn 15.07.1606 – 04.10.1669@\textsc{Rembrandt van Rijn} (15.07.1606 – 04.10.1669), \emph{Bildender Künstler}|pw}} zu ſehen. Zwei herrliche Tage voll Schönheiten und Seltſamkeiten. Und daß ich
               über all’ dem Dein gedacht, ſollen Dir dieſe Zeilen ein Zeichen ſein. Schreib’ mir,
               bitte, ein Wort nach \textsc{Paris, Rue Vivienne 51\oindex{rue Vivienne@\textbf{rue Vivienne}|pw}}, »\textsc{\begin{otherlanguage}{french}Gazette de Francfort\end{otherlanguage}\orgindex{Frankfurter Zeitung@Frankfurter Zeitung|pw}}«\orgindex{Pariser Buero der Frankfurter Zeitung@Pariser Büro der Frankfurter Zeitung|pw}. Grüß’ Dich Gott! Dein \spacefill\mbox{Paul Goldmann}\pend
                     \endnumbering\briefempfaengerindex{Schnitzler, Arthur@\textsc{Schnitzler, Arthur}!zzzGoldmann, Paul@\emph{von Paul Goldmann}!1891-11-291@{29. 11. 1891}|)be}\mylabel{h}\end{ledgroupsized}  \newcommand{\dateiname}{L02672}\newcommand{\titel}{Paul Goldmann an Arthur Schnitzler, 29. 11. 1891}\newcommand{\editorInnen}{Martin Anton Müller und Laura Untner}%% latex-leseansicht-abspann.tex
%% Abspann für die Leseansicht.
%% Der Schalter \ifkorrekturansicht ist bereits durch den Vorspann gesetzt.

%% latex-abspann.tex
%% Gemeinsamer Abspann für Korrekturansicht und Leseansicht.
%% Setzt den Schalter \ifkorrekturansicht voraus (gesetzt in den
%% einbindenden Dateien latex-korrekturansicht-abspann.tex bzw.
%% latex-leseansicht-abspann.tex).
%% ---------------------------------------------------------------

\normalsize

% Das esempio-Environment wird nur in der Leseansicht benötigt
\ifkorrekturansicht\else
\newenvironment{esempio}[3]%
{
    \vspace{1.5ex}
    \rlap{\underline{#1}}
    \par
    \setlength{\parindent}{0cm}
    \nopagebreak
    \leftskip=#2cm
    \rightskip=#3cm
}
{
    \par
}
\fi

\doendnotes{C}
\bigskip
\vfill

\clearpage

\footnotesize

\ifkorrekturansicht
  \lohead{\textsc{register}}
\fi

% theindex-Environment neu definieren ohne reledmac
\makeatletter
\renewenvironment{theindex}{%
  \ifkorrekturansicht
    \section*{\indexname}%
  \else
    \subsubsection*{Index der erwähnten Entitäten}%
  \fi
  \setlength{\parindent}{0pt}%
  \setlength{\parskip}{0pt plus 0.3pt}%
  \let\item\@idxitem
}{%
  \ifkorrekturansicht\clearpage\fi
}
\makeatother

\IfFileExists{\jobname-pw.ind}{\input{\jobname-pw.ind}}{}

% Quellenangabe nur in der Leseansicht
\ifkorrekturansicht\else
% Fallback-Definitionen, falls die .tex-Datei \titel etc. nicht gesetzt hat
\providecommand{\titel}{}
\providecommand{\editorInnen}{}
\providecommand{\dateiname}{\jobname}

\vspace{3cm}

\vfill

\footnotesize
\textsc{Quelle}: \titel. Herausgegeben von {\editorInnen}. In: \emph{Arthur Schnitzler: Briefwechsel mit Autorinnen und Autoren}.
 Digitale Edition, https://schnitzler-briefe.acdh.oeaw.ac.at/{\dateiname}.html (Stand \today)
\fi

\end{document}


      