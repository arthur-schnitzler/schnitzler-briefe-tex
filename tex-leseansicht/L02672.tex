%% latex-korrekturansicht-vorspann.tex
%% Vorspann für die Korrekturansicht.
%% Lädt die gemeinsame Datei latex-vorspann.tex mit gesetztem Schalter.

\newif\ifkorrekturansicht
\korrekturansichttrue

\input{../tex-inputs/latex-vorspann}


\section[Paul Goldmann an Arthur Schnitzler, 29. 11. 1891]{L02672 Paul Goldmann an Arthur Schnitzler, 29. 11. 1891}
\nopagebreak\mylabel{L02672v}
\rehead{ }\normalsize\beginnumbering\briefempfaengerindex{Schnitzler, Arthur@\textsc{Schnitzler, Arthur}!zzzGoldmann, Paul@\emph{von Paul Goldmann}!1891-11-291@{29. 11. 1891}|(be}
\toendnotes[C]{\smallbreak\pagebreak[2]}\Standort{DLA, A:Schnitzler, HS.NZ85.1.3162.}
\physDesc{Postkarte, 506 Zeichen
\newline{}Handschrift: 1) schwarze Tinte, deutsche Kurrent\hspace{1em}2) schwarze Tinte, lateinische Kurrent (\noindent{}Adresse)\hspace{1em}
\newline{}Versand: 1) Stempel: »\nobreak{}\oindex{Amsterdam@\textbf{Amsterdam}, \emph{P.PPLC}|pwk}Amste\textcolor{gray}{rdam}, 30 Nov 91, 10–11V\nobreak{}«.   2) Stempel: »\nobreak{}Wien 1/1, 2/12. 91, 9½–11V., Bestellt\nobreak{}«. 
\newline{}Schnitzler: mit Bleistift das Datum »30/11 91« vermerkt }\pstart{}{\pb}\begin{otherlanguage}{french}Autriche\end{otherlanguage}\oindex{Oesterreich@\textbf{Österreich}, \emph{A.PCLI}|pw}! \pend{}\pstart{}Herrn \pend{}\pstart{}Dr. Arthur Schnitzler \pend{}\pstart{}Wien\oindex{Wien@\textbf{Wien}, \emph{A.ADM2}|pw}\pend{}\pstart{}I. Giselastraſse 11\oindex{Ordination Arthur Schnitzler [Boesendorferstrasse 11]@\textbf{Ordination Arthur Schnitzler [Bösendorferstraße 11]}, \emph{Ordination}|pw}.\pend{}{\bigskip}\vspace{1em}
\pstart
           \centering{}{\pb}Amſterdam\oindex{Amsterdam@\textbf{Amsterdam}, \emph{P.PPLC}|pw}, 29. November\pend
           \vspace{0.5em}
\pstart
           Mein lieber Arthur! So ein Bildernarr bin ich
               geworden, daß ich noch im Fluge zwei Tage zufammengerafft habe, um in \textsc{Haarlem\oindex{Haarlem@\textbf{Haarlem}, \emph{P.PPLA}|pw}} die \textsc{Frans Hals\pwindex{Hals, Frans zwischen 1580 und 1585 – 1666-08-16@\textsc{Hals, Frans} (zwischen 1580 und 1585 – 1666-08-16), \emph{Maler/Malerin}|pw}} und in \textsc{Amsterdam\oindex{Amsterdam@\textbf{Amsterdam}, \emph{P.PPLC}|pw}} die \textsc{Rembrandt\pwindex{Rembrandt van Rijn 15.07.1606 – 04.10.1669@\textsc{Rembrandt van Rijn} (15.07.1606 – 04.10.1669), \emph{Maler/Malerin}|pw}} zu ſehen. Zwei herrliche Tage voll Schönheiten und Seltſamkeiten. Und daß ich
               über all’ dem Dein gedacht, ſollen Dir dieſe Zeilen ein Zeichen ſein. Schreib’ mir,
               bitte, ein Wort nach \textsc{Paris, Rue Vivienne 51\oindex{rue Vivienne@\textbf{rue Vivienne}, \emph{Straße (K.STR)}|pw}}, »\textsc{\begin{otherlanguage}{french}Gazette de Francfort\end{otherlanguage}\orgindex{Frankfurter Zeitung@Frankfurter Zeitung|pw}}«\orgindex{Pariser Buero der Frankfurter Zeitung@Pariser Büro der Frankfurter Zeitung|pw}. Grüß’ Dich Gott! Dein \spacefill\mbox{Paul Goldmann}\pend
           \selectlanguage{ngerman}\endnumbering\briefempfaengerindex{Schnitzler, Arthur@\textsc{Schnitzler, Arthur}!zzzGoldmann, Paul@\emph{von Paul Goldmann}!1891-11-291@{29. 11. 1891}|)be}\mylabel{L02672h}  \normalsize

\doendnotes{C}
\bigskip
\vfill

\clearpage

\footnotesize

\lohead{\textsc{register}}

% Definiere theindex-Environment komplett neu ohne reledmac
\makeatletter
\renewenvironment{theindex}{%
  \section*{\indexname}%
  \setlength{\parindent}{0pt}%
  \setlength{\parskip}{0pt plus 0.3pt}%
  \let\item\@idxitem
}{%
  \clearpage
}
\makeatother

\IfFileExists{\jobname-pw.ind}{\input{\jobname-pw.ind}}{}

\end{document}

      