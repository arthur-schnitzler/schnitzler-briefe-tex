%% latex-leseansicht-vorspann.tex
%% Vorspann für die Leseansicht.
%% Lädt die gemeinsame Datei latex-vorspann.tex mit nicht gesetztem Schalter.

\newif\ifkorrekturansicht
\korrekturansichtfalse

\input{../tex-inputs/latex-vorspann}


\section[Paul Goldmann an Arthur Schnitzler, 29. 11. 1891]{L02672 Paul Goldmann an Arthur Schnitzler, 29. 11. 1891}
\nopagebreak\mylabel{L02672v}
\rehead{ }\normalsize\beginnumbering\briefempfaengerindex{Schnitzler, Arthur@\textsc{Schnitzler, Arthur}!zzzGoldmann, Paul@\emph{von Paul Goldmann}!1891-11-291@{29. 11. 1891}|(be}
\toendnotes[C]{\smallbreak\pagebreak[2]}
\correspDesc{Versand  durch Paul Goldmann am 29. 11. 1891 in Amsterdam
\newline{}Erhalt  durch Arthur Schnitzler am 2. 12. 1891 in Wien}\toendnotes[C]{\smallbreak}
\Standort{DLA, A:Schnitzler, HS.NZ85.1.3162.}
\physDesc{Postkarte, 506 Zeichen
\newline{}Handschrift: schwarze Tinte, deutsche Kurrent
\newline{}Versand: 1) Stempel: »\nobreak{}\oindex{Amsterdam@\textbf{Amsterdam}, \emph{Hauptstadt}|pwk}Amste\textcolor{gray}{rdam}, 30 Nov 91, 10–11V\nobreak{}«.   2) Stempel: »\nobreak{}\oindex{Wien@\textbf{Wien}, \emph{Verwaltungsgebiet}|pwk}Wien 1/1, 2/12. 91, 9½–11V., Bestellt\nobreak{}«. 
\newline{}Schnitzler: mit Bleistift das Datum »30/11 91« vermerkt }\pstart{}\textsc{{\pb}\begin{otherlanguage}{french}Autriche\end{otherlanguage}\oindex{Österreich@\textbf{Österreich}|pw}!}\pend{}\pstart{}\textsc{Herrn}\pend{}\pstart{}\textsc{Dr. Arthur Schnitzler}\pend{}\pstart{}\textsc{Wien\oindex{Wien@\textbf{Wien}, \emph{Verwaltungsgebiet}|pw}}\pend{}\pstart{}\textsc{I. Giselastraſse 11\oindex{Wien@\textbf{Wien}!I., Innere Stadt@\textbf{I., Innere Stadt}!Ordination Arthur Schnitzler [Bösendorferstraße 11]@\textbf{Ordination Arthur Schnitzler [Bösendorferstraße 11]}, \emph{Ordination}|pw}.}\pend{}{\bigskip}\vspace{1em}
\pstart
           \centering{}{\pb}Amſterdam\oindex{Amsterdam@\textbf{Amsterdam}, \emph{Hauptstadt}|pw}, 29. November\pend
           \vspace{0.5em}
\pstart
           Mein lieber Arthur! So ein Bildernarr bin ich
               geworden, daß ich noch im Fluge zwei Tage zufammengerafft habe, um in \textsc{Haarlem\oindex{Haarlem@\textbf{Haarlem}|pw}} die \textsc{Frans Hals\pwindex{Hals, Frans zwischen 1580 und 1585 Antwerpen – 16.\,8.\,1666 Haarlem@\textsc{Hals, Frans} (zwischen 1580 und 1585 Antwerpen – 16.\,8.\,1666 Haarlem), \emph{Maler}|pw}} und in \textsc{Amsterdam\oindex{Amsterdam@\textbf{Amsterdam}, \emph{Hauptstadt}|pw}} die \textsc{Rembrandt\pwindex{Rembrandt van Rijn 15.\,7.\,1606 Leiden – 4.\,10.\,1669 Amsterdam@\textsc{Rembrandt van Rijn} (15.\,7.\,1606 Leiden – 4.\,10.\,1669 Amsterdam), \emph{Maler}|pw}} zu{ }ſehen. Zwei herrliche Tage voll Schönheiten und Seltſamkeiten. Und daß ich
               über all’ dem Dein gedacht,{ }ſollen Dir dieſe Zeilen ein Zeichen{ }ſein. Schreib’ mir,
               bitte, ein Wort nach \textsc{Paris, Rue Vivienne 51\oindex{rue Vivienne@\textbf{rue Vivienne}, \emph{Straße}|pw}}, »\textsc{\begin{otherlanguage}{french}Gazette de Francfort\end{otherlanguage}\orgindex{Frankfurter Zeitung@Frankfurter Zeitung|pw}}«\orgindex{Pariser Büro der Frankfurter Zeitung@Pariser Büro der Frankfurter Zeitung|pw}. Grüß’ Dich Gott! Dein \spacefill\mbox{Paul Goldmann}\pend
           \selectlanguage{ngerman}\endnumbering\briefempfaengerindex{Schnitzler, Arthur@\textsc{Schnitzler, Arthur}!zzzGoldmann, Paul@\emph{von Paul Goldmann}!1891-11-291@{29. 11. 1891}|)be}\mylabel{L02672h}  \newcommand{\dateiname}{L02672}\newcommand{\titel}{Paul Goldmann an Arthur Schnitzler, 29. 11. 1891}\newcommand{\editorInnen}{Martin Anton Müller und Laura Untner}%% latex-leseansicht-abspann.tex
%% Abspann für die Leseansicht.
%% Der Schalter \ifkorrekturansicht ist bereits durch den Vorspann gesetzt.

%% latex-abspann.tex
%% Gemeinsamer Abspann für Korrekturansicht und Leseansicht.
%% Setzt den Schalter \ifkorrekturansicht voraus (gesetzt in den
%% einbindenden Dateien latex-korrekturansicht-abspann.tex bzw.
%% latex-leseansicht-abspann.tex).
%% ---------------------------------------------------------------

\normalsize

% Das esempio-Environment wird nur in der Leseansicht benötigt
\ifkorrekturansicht\else
\newenvironment{esempio}[3]%
{
    \vspace{1.5ex}
    \rlap{\underline{#1}}
    \par
    \setlength{\parindent}{0cm}
    \nopagebreak
    \leftskip=#2cm
    \rightskip=#3cm
}
{
    \par
}
\fi

\doendnotes{C}
\bigskip
\vfill

\clearpage

\footnotesize

\ifkorrekturansicht
  \lohead{\textsc{register}}
\fi

% theindex-Environment neu definieren ohne reledmac
\makeatletter
\renewenvironment{theindex}{%
  \ifkorrekturansicht
    \section*{\indexname}%
  \else
    \subsubsection*{Index der erwähnten Entitäten}%
  \fi
  \setlength{\parindent}{0pt}%
  \setlength{\parskip}{0pt plus 0.3pt}%
  \let\item\@idxitem
}{%
  \ifkorrekturansicht\clearpage\fi
}
\makeatother

\IfFileExists{\jobname-pw.ind}{\input{\jobname-pw.ind}}{}

% Quellenangabe nur in der Leseansicht
\ifkorrekturansicht\else
% Fallback-Definitionen, falls die .tex-Datei \titel etc. nicht gesetzt hat
\providecommand{\titel}{}
\providecommand{\editorInnen}{}
\providecommand{\dateiname}{\jobname}

\vspace{3cm}

\vfill

\footnotesize
\textsc{Quelle}: \titel. Herausgegeben von {\editorInnen}. In: \emph{Arthur Schnitzler: Briefwechsel mit Autorinnen und Autoren}.
 Digitale Edition, https://schnitzler-briefe.acdh.oeaw.ac.at/{\dateiname}.html (Stand \today)
\fi

\end{document}


