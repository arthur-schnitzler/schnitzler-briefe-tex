%% latex-korrekturansicht-vorspann.tex
%% Vorspann für die Korrekturansicht.
%% Lädt die gemeinsame Datei latex-vorspann.tex mit gesetztem Schalter.

\newif\ifkorrekturansicht
\korrekturansichttrue

\input{../tex-inputs/latex-vorspann}


\section[Karl Kraus an Arthur Schnitzler, 22. 3. 1894]{L00308 Karl Kraus an Arthur Schnitzler, 22. 3. 1894}
\nopagebreak\mylabel{L00308v}
\rehead{ }\normalsize\beginnumbering\briefempfaengerindex{Schnitzler, Arthur@\textsc{Schnitzler, Arthur}!zzzKraus, Karl@\emph{von Karl Kraus}!1894-03-221@{22. 3. 1894}|(be}
\toendnotes[C]{\smallbreak\pagebreak[2]}\Standort{CUL, Schnitzler, B 55.}
\physDesc{Kartenbrief, 198 Zeichen
\newline{}Handschrift: schwarze Tinte, deutsche Kurrent
\newline{}Versand: 1) Stempel: »\nobreak{}\oindex{I., Innere Stadt@\textbf{I., Innere Stadt}, \emph{A.ADM3}|pwk}Wien 1/1, 22 3 94, 5–6N\nobreak{}«.   2) Stempel: »\nobreak{}\oindex{IX., Alsergrund@\textbf{IX., Alsergrund}, \emph{A.ADM3}|pwk}Wien \textcolor{gray}{3/3}, 23. 3. 9\textcolor{gray}{4}, 8. \textcolor{gray}{V}, Bestellt\nobreak{}«. 
\newline{}Schnitzler: mit Bleistift datiert: »22/3 94« }
\buchAbdrucke{\weitereDrucke{\emph{Literatur und Kritik}, Bd. 49, Oktober 1970, S. 521.} }\toendnotes[C]{\smallbreak}\pstart{}{\pb}Herrn\pend{}\pstart{}D\textsuperscript{r} Arthur Schnitzler\pend{}\pstart{}Wien IX.\oindex{IX., Alsergrund@\textbf{IX., Alsergrund}, \emph{A.ADM3}|pw}\pend{}\pstart{}Frankgasse 1\oindex{Frankgasse 1@\textbf{Frankgasse 1}, \emph{Wohngebäude (K.WHS)}|pw}.\pend{}{\bigskip}\vspace{1em}
\pstart
           {\pb}Wien\oindex{Wien@\textbf{Wien}, \emph{A.ADM2}|pw}, Donnerſtag.\pend
           
\pstart{}L. Schn!\pend\vspace{0.5em}
\pstart
           Geſchieht es alſo mit Ihrer Erlaubnis, daſs am Samſtag{ }\strikeout{me}{ }Ihr Relief\pwindex{Arthur Schnitzler@\emph{Arthur Schnitzler}|pwv} zu mir und mein Relief\pwindex{Karl Kraus@\emph{Karl Kraus}|pwv} zu Ihnen gebracht wird?\pend
           
\pstart
           Hoffentlich{\\[\baselineskip]}Ihr \spacefill\mbox{Kraus}\pend
           \leftskip=0em{}\selectlanguage{ngerman}\endnumbering\briefempfaengerindex{Schnitzler, Arthur@\textsc{Schnitzler, Arthur}!zzzKraus, Karl@\emph{von Karl Kraus}!1894-03-221@{22. 3. 1894}|)be}\mylabel{L00308h}  \normalsize

\doendnotes{C}
\bigskip
\vfill

\clearpage

\footnotesize

\lohead{\textsc{register}}

% Definiere theindex-Environment komplett neu ohne reledmac
\makeatletter
\renewenvironment{theindex}{%
  \section*{\indexname}%
  \setlength{\parindent}{0pt}%
  \setlength{\parskip}{0pt plus 0.3pt}%
  \let\item\@idxitem
}{%
  \clearpage
}
\makeatother

\IfFileExists{\jobname-pw.ind}{\input{\jobname-pw.ind}}{}

\end{document}

      