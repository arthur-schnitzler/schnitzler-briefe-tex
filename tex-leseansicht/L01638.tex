%% latex-korrekturansicht-vorspann.tex
%% Vorspann für die Korrekturansicht.
%% Lädt die gemeinsame Datei latex-vorspann.tex mit gesetztem Schalter.

\newif\ifkorrekturansicht
\korrekturansichttrue

\input{../tex-inputs/latex-vorspann}


\section[Arthur Schnitzler an Hugo von Hofmannsthal, 27. 11. 1906]{L01638 Arthur Schnitzler an Hugo von Hofmannsthal, 27. 11. 1906}
\nopagebreak\mylabel{L01638v}
\rehead{ }\normalsize\beginnumbering\briefempfaengerindex{Hofmannsthal, Hugo von@\textsc{Hofmannsthal, Hugo von}!zzzSchnitzler, Arthur@\emph{von Arthur Schnitzler}!1906-11-271@{27. 11. 1906}|(be}
\toendnotes[C]{\smallbreak\pagebreak[2]}\Standort{FDH, Hs-30885,126.}
\physDesc{Brief, 1 Blatt, 4 Seiten, 1546 Zeichen
\newline{}Handschrift: schwarze Tinte, deutsche Kurrent}
\buchAbdrucke{\weitereDrucke{Hugo von Hofmannsthal, Arthur Schnitzler: \emph{Briefwechsel}. Frankfurt am Main: \emph{S. Fischer} 1964, S. 224.} }\toendnotes[C]{\smallbreak}
\pstart
           \raggedleft{}{\pb}Wien\oindex{Wien@\textbf{Wien}, \emph{A.ADM2}|pw}, 27. Nov 906\pend
           \vspace{0.5em}
\pstart
           lieber Hugo, ſchönen Dank für das \label{K_L01638-1v}\edtext{Buch}{\lemma{\textnormal{\emph{Buch}}}\Cendnote{\textnormal{unklar; die
                  kurze Erwähnung deutet auf kein bedeutenderes Werk hin. Zwar könnte es sich um den
                  ersten Band der zwölfbändigen Ausgabe von \emph{Tausendundeine Nacht}\pwindex{Tausendundeine Nacht@\emph{Tausendundeine Nacht}|pwk} in der Übersetzung von Felix Paul Greve\pwindex{Greve, Felix Paul 1879-02-14 – 1948-08-19@\textsc{Greve, Felix Paul} (1879-02-14 – 1948-08-19), \emph{Schriftsteller/Schriftstellerin, Übersetzer/Übersetzerin}|pwk} (\emph{Insel-Verlag}\orgindex{Insel Verlag@Insel Verlag|pwk}, Ausgabe ab November 1906) handeln, dessen Vorrede\pwindex{Vorrede@\emph{Vorrede}|pwkv} in Folge erwähnt
                  wird, doch ist diese auch unmittelbar vor dem Brief am 25. 11. 1906
                  in \emph{Der Tag}\pwindex{Tag@\emph{Der Tag}|pwk} erschienen.}}}\label{K_L01638-1}.
               Außerordentlich habe ich Ihre Vorrede\pwindex{Vorrede@\emph{Vorrede}|pwv} zu »Tauſend und eine Nacht\pwindex{Tausendundeine Nacht@\emph{Tausendundeine Nacht}|pw}«,
               dann Ihren Artikel\pwindex{unvergleichliche Taenzerin@\emph{Die unvergleichliche Tänzerin}|pwv} über die
               Tänzerin Ruth\pwindex{Saint Denis, Ruth 01.02.1877 – 21.07.1968@\textsc{Saint Denis, Ruth} (01.02.1877 – 21.07.1968), \emph{Tänzer/Tänzerin, Choreograf/Choreografin}|pw} gefunden. In früherer Zeit war
               in ſolchen Aufſätzen von Ihnen zuweilen ein oder das andere Wort enthalten, das ſich
               zu hoch davonſchwang, ſo daſs \substVorne{}\textsuperscript{zuweilen}\substDazwischen{}manchmal\substHinten{} gerade eine beſondere Schönheit mir den Rythmus des ganzen ein wenig ſtörte.
               Jetzt iſt Gleichmaß und {\pb}Flügelhaftigkeit auch dieſen
               Aufſätzen ſo vollkommen eigen, \strikeout{daſs man} und die
               Eigenart \strikeout{iſt} Ihres Proſaſtils iſt zugleich ſo gewahrt
               und ſo erhöht worden, daſs man für dieſe Produkte am liebſten einen eignen Namen
               erſinnen möchte. Sehr ſchön waren auch die Dialoge\pwindex{Unterhaltungen ueber ein neues Buch@\emph{Unterhaltungen über ein neues Buch}|pwv} über die »Schweſtern\pwindex{Schwestern. Drei Novellen@\emph{Die Schwestern. Drei Novellen}|pw}«, beſonders der zweite Artikel. Wunderbar iſt es Ihnen gelungen,
               den Widerſtreit der Empfindungen auszudrücken, mit dem man dem ganzen Problem {\pb}Waſſermann\pwindex{Wassermann, Jakob 10.03.1873 – 01.01.1934@\textsc{Wassermann, Jakob} (10.03.1873 – 01.01.1934), \emph{Schriftsteller/Schriftstellerin}|pw} gegenüberſteht, indem Sie, wohl
               auch zu eigner Beruhigung, Ihre Seele dialogiſch aufgelöſt und ſich dazu bekannt
               haben, daſs wir nicht nur der Welt, den Erlebniſſen, den Menſchen, ſondern auch jener
               einzigen Einheitlichkeit die wir Kunſtwerk nennen, durchaus nicht einheitlich,
               ſondern zugleich onkel- majors- mädchen- gutsbeſitzer- träumerhaft ins Auge ſchauen.
               Gewöhnlich ſchreibt über die Dinge Einer, der nur ein Onkel, {\pb}nur ein Träumer, nur ein Mädchen iſt. All dies ließe ſich
               richtiger ausdrücken, wozu mir die Sa{\geminationm}lung in dieſem
               Augenblicke fehlt.\pend
           
\pstart
           Hoffentlich ſieht man ſich wieder we{\geminationn} Sie \label{K_L01638-2v}\edtext{zurückkehren}{\lemma{\textnormal{\emph{zurückkehren}}}\Cendnote{\textnormal{Er ist von 28. 11. bis 16. 12. 1906 in
                     Deutschland\oindex{Deutschland@\textbf{Deutschland}, \emph{A.PCLI}|pwk} unterwegs.}}}\label{K_L01638-2}, aus München\oindex{Muenchen@\textbf{München}, \emph{P.PPLA}|pw}, Göttingen\oindex{Goettingen@\textbf{Göttingen}, \emph{P.PPLA3}|pw}, Berlin\oindex{Berlin@\textbf{Berlin}, \emph{P.PPLC}|pw}. Laſſen Sie
               gelegentlich was von ſich hören.\pend
           
\pstart
           Herzlichst{\\[\baselineskip]}Ihr{\\[\baselineskip]}\spacefill\mbox{Arthur.}\pend
           \leftskip=0em{}\selectlanguage{ngerman}\endnumbering\briefempfaengerindex{Hofmannsthal, Hugo von@\textsc{Hofmannsthal, Hugo von}!zzzSchnitzler, Arthur@\emph{von Arthur Schnitzler}!1906-11-271@{27. 11. 1906}|)be}\mylabel{L01638h}  \normalsize

\doendnotes{C}
\bigskip
\vfill

\clearpage

\footnotesize

\lohead{\textsc{register}}

% Definiere theindex-Environment komplett neu ohne reledmac
\makeatletter
\renewenvironment{theindex}{%
  \section*{\indexname}%
  \setlength{\parindent}{0pt}%
  \setlength{\parskip}{0pt plus 0.3pt}%
  \let\item\@idxitem
}{%
  \clearpage
}
\makeatother

\IfFileExists{\jobname-pw.ind}{\input{\jobname-pw.ind}}{}

\end{document}

      