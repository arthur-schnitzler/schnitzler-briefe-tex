%% latex-leseansicht-vorspann.tex
%% Vorspann für die Leseansicht.
%% Lädt die gemeinsame Datei latex-vorspann.tex mit nicht gesetztem Schalter.

\newif\ifkorrekturansicht
\korrekturansichtfalse

\input{../tex-inputs/latex-vorspann}


\section[Arthur Schnitzler an Hugo von Hofmannsthal, 27. 11. 1906]{L01638 Arthur Schnitzler an Hugo von Hofmannsthal, 27. 11. 1906}
\nopagebreak\mylabel{L01638v}
\rehead{ }\normalsize\beginnumbering\briefempfaengerindex{Hofmannsthal, Hugo von@\textsc{Hofmannsthal, Hugo von}!zzzSchnitzler, Arthur@\emph{von Arthur Schnitzler}!1906-11-271@{27. 11. 1906}|(be}
\toendnotes[C]{\smallbreak\pagebreak[2]}
\correspDesc{Versand  durch Arthur Schnitzler am 27. 11. 1906 in Wien
\newline{}Erhalt  durch Hugo von Hofmannsthal im Zeitraum [27. 11. 1906 – 1. 12. 1906?] in Wien}\toendnotes[C]{\smallbreak}
\Standort{FDH, Hs-30885,126.}
\physDesc{Brief, 1 Blatt, 4 Seiten, 1546 Zeichen
\newline{}Handschrift: schwarze Tinte, deutsche Kurrent}
\buchAbdrucke{\weitereDrucke{Hugo von Hofmannsthal, Arthur Schnitzler: \emph{Briefwechsel}. Herausgegeben von Therese Nickl und Heinrich Schnitzler. Frankfurt am Main: \emph{S. Fischer} 1964, S. 224.} }\toendnotes[C]{\smallbreak}
\pstart
           \raggedleft{}{\pb}Wien\oindex{Wien@\textbf{Wien}, \emph{Verwaltungsgebiet}|pw}, 27. Nov 906\pend
           \vspace{0.5em}
\pstart
           lieber Hugo,{ }ſchönen Dank für das \label{K_L01638-1v}\edtext{Buch}{\lemma{\textnormal{\emph{Buch}}}\Cendnote{\textnormal{unklar; die
                  kurze Erwähnung deutet auf kein bedeutenderes Werk hin. Zwar könnte es sich um den
                  ersten Band der zwölfbändigen Ausgabe von \emph{Tausendundeine Nacht}\pwindex{Tausendundeine Nacht@\emph{Tausendundeine Nacht}|pwk} in der Übersetzung von Felix Paul Greve\pwindex{Greve, Felix Paul 14.\,2.\,1879 Radomno – 19.\,8.\,1948 Simcoe@\textsc{Greve, Felix Paul} (14.\,2.\,1879 Radomno – 19.\,8.\,1948 Simcoe), \emph{Schriftsteller, Übersetzer}|pwk} (\emph{Insel-Verlag}\orgindex{Insel Verlag@Insel Verlag|pwk}, Ausgabe ab November 1906) handeln, dessen Vorrede\pwindex{Vorrede@\emph{Vorrede}|pwkv} in Folge erwähnt
                  wird, doch ist diese auch unmittelbar vor dem Brief am 25. 11. 1906
                  in \emph{Der Tag}\pwindex{Tag@\emph{Der Tag}|pwk} erschienen.}}}\label{K_L01638-1}.
               Außerordentlich habe ich Ihre Vorrede\pwindex{Vorrede@\emph{Vorrede}|pwv} zu »Tauſend und eine Nacht\pwindex{Tausendundeine Nacht@\emph{Tausendundeine Nacht}|pw}«,
               dann Ihren Artikel\pwindex{Hofmannsthal, Hugo von 1.\,2.\,1874 Wien – 15.\,7.\,1929 Rodaun@\textsc{Hofmannsthal, Hugo von} (1.\,2.\,1874 Wien – 15.\,7.\,1929 Rodaun), \emph{Schriftsteller}!unvergleichliche Tänzerin@\strich\emph{Die unvergleichliche Tänzerin}|pwv} über die
               Tänzerin Ruth\pwindex{Saint Denis, Ruth 1.\,2.\,1877 Newark – 21.\,7.\,1968 Hollywood@\textsc{Saint Denis, Ruth} (1.\,2.\,1877 Newark – 21.\,7.\,1968 Hollywood), \emph{Tänzerin, Choreografin}|pw} gefunden. In früherer Zeit war
               in{ }ſolchen Aufſätzen von Ihnen zuweilen ein oder das andere Wort enthalten, das{ }ſich
               zu hoch davonſchwang,{ }ſo daſs \substVorne{}\textsuperscript{zuweilen}\substDazwischen{}manchmal\substHinten{} gerade eine beſondere Schönheit mir den Rythmus des ganzen ein wenig{ }ſtörte.
               Jetzt iſt Gleichmaß und {\pb}Flügelhaftigkeit auch dieſen
               Aufſätzen{ }ſo vollkommen eigen, \strikeout{daſs man} und die
               Eigenart \strikeout{iſt} Ihres Proſaſtils iſt zugleich{ }ſo gewahrt
               und{ }ſo erhöht worden, daſs man für dieſe Produkte am liebſten einen eignen Namen
               erſinnen möchte. Sehr{ }ſchön waren auch die Dialoge\pwindex{Hofmannsthal, Hugo von 1.\,2.\,1874 Wien – 15.\,7.\,1929 Rodaun@\textsc{Hofmannsthal, Hugo von} (1.\,2.\,1874 Wien – 15.\,7.\,1929 Rodaun), \emph{Schriftsteller}!Unterhaltungen über ein neues Buch@\strich\emph{Unterhaltungen über ein neues Buch}|pwv} über die »Schweſtern\pwindex{Wassermann, Jakob 10.\,3.\,1873 Fürth – 1.\,1.\,1934 Altaussee@\textsc{Wassermann, Jakob} (10.\,3.\,1873 Fürth – 1.\,1.\,1934 Altaussee), \emph{Schriftsteller}!Schwestern. Drei Novellen@\strich\emph{Die Schwestern. Drei Novellen}|pw}«, beſonders der zweite Artikel. Wunderbar iſt es Ihnen gelungen,
               den Widerſtreit der Empfindungen auszudrücken, mit dem man dem ganzen Problem {\pb}Waſſermann\pwindex{Wassermann, Jakob 10.\,3.\,1873 Fürth – 1.\,1.\,1934 Altaussee@\textsc{Wassermann, Jakob} (10.\,3.\,1873 Fürth – 1.\,1.\,1934 Altaussee), \emph{Schriftsteller}|pw} gegenüberſteht, indem Sie, wohl
               auch zu eigner Beruhigung, Ihre Seele dialogiſch aufgelöſt und{ }ſich dazu bekannt
               haben, daſs wir nicht nur der Welt, den Erlebniſſen, den Menſchen,{ }ſondern auch jener
               einzigen Einheitlichkeit die wir Kunſtwerk nennen, durchaus nicht einheitlich,{ }ſondern zugleich onkel- majors- mädchen- gutsbeſitzer- träumerhaft ins Auge{ }ſchauen.
               Gewöhnlich{ }ſchreibt über die Dinge Einer, der nur ein Onkel, {\pb}nur ein Träumer, nur ein Mädchen iſt. All dies ließe{ }ſich
               richtiger ausdrücken, wozu mir die Sa{\geminationm}lung in dieſem
               Augenblicke fehlt.\pend
           
\pstart
           Hoffentlich{ }ſieht man{ }ſich wieder we{\geminationn} Sie \label{K_L01638-2v}\edtext{zurückkehren}{\lemma{\textnormal{\emph{zurückkehren}}}\Cendnote{\textnormal{Er ist von 28. 11. bis 16. 12. 1906 in
                     Deutschland\oindex{Deutschland@\textbf{Deutschland}|pwk} unterwegs.}}}\label{K_L01638-2}, aus München\oindex{München@\textbf{München}|pw}, Göttingen\oindex{Göttingen@\textbf{Göttingen}, \emph{Hauptstadt}|pw}, Berlin\oindex{Berlin@\textbf{Berlin}, \emph{Hauptstadt}|pw}. Laſſen Sie
               gelegentlich was von{ }ſich hören.\pend
           
\pstart
           Herzlichst{\\[\baselineskip]}Ihr{\\[\baselineskip]}\spacefill\mbox{Arthur.}\pend
           \leftskip=0em{}\selectlanguage{ngerman}\endnumbering\briefempfaengerindex{Hofmannsthal, Hugo von@\textsc{Hofmannsthal, Hugo von}!zzzSchnitzler, Arthur@\emph{von Arthur Schnitzler}!1906-11-271@{27. 11. 1906}|)be}\mylabel{L01638h}  \newcommand{\dateiname}{L01638}\newcommand{\titel}{Arthur Schnitzler an Hugo von Hofmannsthal, 27. 11. 1906}\newcommand{\editorInnen}{Martin Anton Müller und Gerd-Hermann Susen}%% latex-leseansicht-abspann.tex
%% Abspann für die Leseansicht.
%% Der Schalter \ifkorrekturansicht ist bereits durch den Vorspann gesetzt.

%% latex-abspann.tex
%% Gemeinsamer Abspann für Korrekturansicht und Leseansicht.
%% Setzt den Schalter \ifkorrekturansicht voraus (gesetzt in den
%% einbindenden Dateien latex-korrekturansicht-abspann.tex bzw.
%% latex-leseansicht-abspann.tex).
%% ---------------------------------------------------------------

\normalsize

% Das esempio-Environment wird nur in der Leseansicht benötigt
\ifkorrekturansicht\else
\newenvironment{esempio}[3]%
{
    \vspace{1.5ex}
    \rlap{\underline{#1}}
    \par
    \setlength{\parindent}{0cm}
    \nopagebreak
    \leftskip=#2cm
    \rightskip=#3cm
}
{
    \par
}
\fi

\doendnotes{C}
\bigskip
\vfill

\clearpage

\footnotesize

\ifkorrekturansicht
  \lohead{\textsc{register}}
\fi

% theindex-Environment neu definieren ohne reledmac
\makeatletter
\renewenvironment{theindex}{%
  \ifkorrekturansicht
    \section*{\indexname}%
  \else
    \subsubsection*{Index der erwähnten Entitäten}%
  \fi
  \setlength{\parindent}{0pt}%
  \setlength{\parskip}{0pt plus 0.3pt}%
  \let\item\@idxitem
}{%
  \ifkorrekturansicht\clearpage\fi
}
\makeatother

\IfFileExists{\jobname-pw.ind}{\input{\jobname-pw.ind}}{}

% Quellenangabe nur in der Leseansicht
\ifkorrekturansicht\else
% Fallback-Definitionen, falls die .tex-Datei \titel etc. nicht gesetzt hat
\providecommand{\titel}{}
\providecommand{\editorInnen}{}
\providecommand{\dateiname}{\jobname}

\vspace{3cm}

\vfill

\footnotesize
\textsc{Quelle}: \titel. Herausgegeben von {\editorInnen}. In: \emph{Arthur Schnitzler: Briefwechsel mit Autorinnen und Autoren}.
 Digitale Edition, https://schnitzler-briefe.acdh.oeaw.ac.at/{\dateiname}.html (Stand \today)
\fi

\end{document}


