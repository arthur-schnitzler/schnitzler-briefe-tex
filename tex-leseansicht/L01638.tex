%% latex-leseansicht-vorspann.tex
%% Vorspann für die Leseansicht.
%% Lädt die gemeinsame Datei latex-vorspann.tex mit nicht gesetztem Schalter.

\newif\ifkorrekturansicht
\korrekturansichtfalse

\input{../tex-inputs/latex-vorspann}


         
         \renewcommand{\erwaehntePersonen}{Personen: Felix Paul Greve, Hugo von Hofmannsthal, Ruth Saint Denis, Jakob Wassermann}
         \renewcommand{\erwaehnteInstitutionen}{Institutionen: Insel-Verlag}
         \renewcommand{\erwaehnteOrte}{Orte: Berlin, Deutschland, Göttingen, München, Wien}
         \renewcommand{\erwaehnteWerke}{Werke: Der Tag, Die Schwestern. Drei Novellen, Die unvergleichliche Tänzerin, Tausendundeine Nacht, Unterhaltungen über ein neues Buch, Vorrede}
               \section[Arthur Schnitzler an Hugo von Hofmannsthal, 27. 11. 1906]{ Arthur Schnitzler an Hugo von Hofmannsthal, 27. 11. 1906}\nopagebreak\mylabel{v}\rehead{ }\begin{ledgroupsized}[t]{13cm}\normalsize\beginnumbering \toendnotes[C]{\smallbreak\pagebreak[2]} \Standort{FDH, Hs-30885,126.}
\physDesc{Brief, 1 Blatt, 4 Seiten
\newline{}Handschrift: schwarze Tinte, deutsche Kurrent}\buchAbdrucke{\weitereDrucke{Hugo von Hofmannsthal, Arthur Schnitzler: \emph{Briefwechsel}. Hg. Therese Nickl und Heinrich Schnitzler. Frankfurt am Main: \emph{S. Fischer} 1964, S. 224.} }\toendnotes[C]{\smallbreak}\pstart
           \raggedleft{}{\pb}Wien\oindex{Wien@\textbf{Wien}|pw}, 27. Nov 906\pend
           \pstart
           lieber Hugo, ſchönen Dank für das \label{K_L01638_1v}\edtext{Buch}{\lemma{\textnormal{\emph{Buch}}}\Cendnote{\textnormal{unklar; die
                  kurze Erwähnung deutet auf kein bedeutenderes Werk hin. Zwar könnte es sich um den
                  ersten Band der zwölfbändigen Ausgabe von \emph{Tausendundeine Nacht}\pwindex{Tausendundeine NachtNone@\emph{Tausendundeine Nacht} {[}None{]}|pwk} in der Übersetzung von Felix Paul Greve\pwindex{Greve, Felix Paul 1879-02-14 – 1948-08-19@\textsc{Greve, Felix Paul} (1879-02-14 – 1948-08-19), \emph{Schriftsteller, Übersetzer}|pwk} (\emph{Insel-Verlag}\orgindex{Insel-Verlag@Insel-Verlag|pwk}, Ausgabe ab November 1906) handeln, dessen Vorrede\pwindex{?? Werk@Nicht ermittelte Verfasserinnen und Verfasser!Vorrede25.11.1906 – 25.11.906@\emph{Vorrede} {[}25.11.1906 – 25.11.906{]}|pwkv} in Folge erwähnt wird,
                  doch ist diese auch unmittelbar vor dem Brief am 25. 11. 1906 in \emph{Der Tag}\pwindex{?? Werk@Nicht ermittelte Verfasserinnen und Verfasser!Tag19.12.1900 – 1934@\emph{Der Tag} {[}19.12.1900 – 1934{]}|pwk} erschienen.}}}\label{K_L01638_1h}. Außerordentlich habe
               ich Ihre Vorrede\pwindex{?? Werk@Nicht ermittelte Verfasserinnen und Verfasser!Vorrede25.11.1906 – 25.11.906@\emph{Vorrede} {[}25.11.1906 – 25.11.906{]}|pwv} zu »Tauſend und eine Nacht\pwindex{Tausendundeine NachtNone@\emph{Tausendundeine Nacht} {[}None{]}|pw}«, dann Ihren Artikel\pwindex{Hofmannsthal, Hugo von 1874-02-01 – 1929-07-15@\textsc{Hofmannsthal, Hugo von} (1874-02-01 – 1929-07-15), \emph{Schriftsteller}!unvergleichliche Taenzerin25. 11. 1906@\strich\emph{Die unvergleichliche Tänzerin} {[}25. 11. 1906{]}|pwv} über die Tänzerin Ruth\pwindex{Saint Denis, Ruth 01.02.1877 – 21.07.1968@\textsc{Saint Denis, Ruth} (01.02.1877 – 21.07.1968), \emph{Tänzerin, Choreografin}|pw} gefunden. In früherer Zeit war in ſolchen Aufſätzen von
               Ihnen zuweilen ein oder das andere Wort enthalten, das ſich zu hoch davonſchwang, ſo
               daſs \substVorne{}\textsuperscript{zuweilen}{\allowbreak}\substDazwischen{}manchmal\substHinten{} gerade eine beſondere Schönheit mir den Rythmus des ganzen ein wenig ſtörte.
               Jetzt iſt Gleichmaß und {\pb}Flügelhaftigkeit auch dieſen
               Aufſätzen ſo vollkommen eigen, \strikeout{daſs man} und die
               Eigenart \strikeout{iſt} Ihres Proſaſtils iſt zugleich ſo gewahrt
               und ſo erhöht worden, daſs man für dieſe Produkte am liebſten einen eignen Namen
               erſinnen möchte. Sehr ſchön waren auch die Dialoge\pwindex{Hofmannsthal, Hugo von 1874-02-01 – 1929-07-15@\textsc{Hofmannsthal, Hugo von} (1874-02-01 – 1929-07-15), \emph{Schriftsteller}!Unterhaltungen ueber ein neues Buch1.11.1906 – 2.11.1906@\strich\emph{Unterhaltungen über ein neues Buch} {[}1.11.1906 – 2.11.1906{]}|pwv} über die »Schweſtern\pwindex{Wassermann, Jakob 10.03.1873 – 01.01.1934@\textsc{Wassermann, Jakob} (10.03.1873 – 01.01.1934), \emph{Schriftsteller}!Schwestern. Drei Novellen1906@\strich\emph{Die Schwestern. Drei Novellen} {[}1906{]}|pw}«, beſonders der zweite Artikel. Wunderbar iſt es Ihnen gelungen,
               den Widerſtreit der Empfindungen auszudrücken, mit dem man dem ganzen Problem {\pb}Waſſermann\pwindex{Wassermann, Jakob 10.03.1873 – 01.01.1934@\textsc{Wassermann, Jakob} (10.03.1873 – 01.01.1934), \emph{Schriftsteller}|pw} gegenüberſteht, indem Sie, wohl auch
               zu eigner Beruhigung, Ihre Seele dialogiſch aufgelöſt und ſich dazu bekannt haben,
               daſs wir nicht nur der Welt, den Erlebniſſen, den Menſchen, ſondern auch jener
               einzigen Einheitlichkeit die wir Kunſtwerk nennen, durchaus nicht einheitlich,
               ſondern zugleich onkel- majors- mädchen- gutsbeſitzer- träumerhaft ins Auge ſchauen.
               Gewöhnlich ſchreibt über die Dinge Einer, der nur ein Onkel, {\pb}nur ein Träumer, nur ein Mädchen iſt. All dies ließe ſich
               richtiger ausdrücken, wozu mir die Sa{\geminationm}lung in dieſem
               Augenblicke fehlt.\pend
           \pstart
           Hoffentlich ſieht man ſich wieder we{\geminationn} Sie \label{K_L01638_2v}\edtext{zurückkehren}{\lemma{\textnormal{\emph{zurückkehren}}}\Cendnote{\textnormal{Er ist von 28. 11. bis 16. 12. 1906 in
                     Deutschland\oindex{Deutschland@\textbf{Deutschland}|pwk} unterwegs.}}}\label{K_L01638_2h}, aus München\oindex{Muenchen@\textbf{München}|pw}, Göttingen\oindex{Goettingen@\textbf{Göttingen}|pw}, Berlin\oindex{Berlin@\textbf{Berlin}|pw}. Laſſen Sie gelegentlich
               was von ſich hören.\pend
           \pstart
           Herzlichst{\\[\baselineskip]}Ihr{\\[\baselineskip]}\spacefill\mbox{Arthur.}\pend
           \leftskip=0em{}
         
         \endnumbering\mylabel{h}\end{ledgroupsized}  \newcommand{\dateiname}{L01638}\newcommand{\titel}{Arthur Schnitzler an Hugo von Hofmannsthal, 27. 11. 1906}\newcommand{\editorInnen}{Martin Anton Müller und Gerd-Hermann Susen}%% latex-leseansicht-abspann.tex
%% Abspann für die Leseansicht.
%% Der Schalter \ifkorrekturansicht ist bereits durch den Vorspann gesetzt.

%% latex-abspann.tex
%% Gemeinsamer Abspann für Korrekturansicht und Leseansicht.
%% Setzt den Schalter \ifkorrekturansicht voraus (gesetzt in den
%% einbindenden Dateien latex-korrekturansicht-abspann.tex bzw.
%% latex-leseansicht-abspann.tex).
%% ---------------------------------------------------------------

\normalsize

% Das esempio-Environment wird nur in der Leseansicht benötigt
\ifkorrekturansicht\else
\newenvironment{esempio}[3]%
{
    \vspace{1.5ex}
    \rlap{\underline{#1}}
    \par
    \setlength{\parindent}{0cm}
    \nopagebreak
    \leftskip=#2cm
    \rightskip=#3cm
}
{
    \par
}
\fi

\doendnotes{C}
\bigskip
\vfill

\clearpage

\footnotesize

\ifkorrekturansicht
  \lohead{\textsc{register}}
\fi

% theindex-Environment neu definieren ohne reledmac
\makeatletter
\renewenvironment{theindex}{%
  \ifkorrekturansicht
    \section*{\indexname}%
  \else
    \subsubsection*{Index der erwähnten Entitäten}%
  \fi
  \setlength{\parindent}{0pt}%
  \setlength{\parskip}{0pt plus 0.3pt}%
  \let\item\@idxitem
}{%
  \ifkorrekturansicht\clearpage\fi
}
\makeatother

\IfFileExists{\jobname-pw.ind}{\input{\jobname-pw.ind}}{}

% Quellenangabe nur in der Leseansicht
\ifkorrekturansicht\else
% Fallback-Definitionen, falls die .tex-Datei \titel etc. nicht gesetzt hat
\providecommand{\titel}{}
\providecommand{\editorInnen}{}
\providecommand{\dateiname}{\jobname}

\vspace{3cm}

\vfill

\footnotesize
\textsc{Quelle}: \titel. Herausgegeben von {\editorInnen}. In: \emph{Arthur Schnitzler: Briefwechsel mit Autorinnen und Autoren}.
 Digitale Edition, https://schnitzler-briefe.acdh.oeaw.ac.at/{\dateiname}.html (Stand \today)
\fi

\end{document}


      