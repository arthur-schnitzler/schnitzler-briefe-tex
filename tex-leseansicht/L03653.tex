%% latex-leseansicht-vorspann.tex
%% Vorspann für die Leseansicht.
%% Lädt die gemeinsame Datei latex-vorspann.tex mit nicht gesetztem Schalter.

\newif\ifkorrekturansicht
\korrekturansichtfalse

\input{../tex-inputs/latex-vorspann}


\section[Stefan Zweig an Arthur Schnitzler, {[}zwischen 7. 4. 1915 und 9. 4. 1915?{]}]{L03653 Stefan Zweig an Arthur Schnitzler, {[}zwischen 7. 4. 1915 und 9. 4. 1915?{]}}
\nopagebreak\mylabel{L03653v}
\rehead{ }\normalsize\beginnumbering\briefempfaengerindex{Schnitzler, Arthur@\textsc{Schnitzler, Arthur}!zzzZweig, Stefan@\emph{von Stefan Zweig}!1915-04-091@{{[}zwischen 7. 4. 1915 und 9. 4. 1915?{]}}|(be}
\toendnotes[C]{\smallbreak\pagebreak[2]}
\correspDesc{Versand  durch Stefan Zweig im Zeitraum [zwischen
                  7. 4. 1915 und 9. 4. 1915?] in Wien
\newline{}Erhalt  durch Arthur Schnitzler im Zeitraum [zwischen
                  7. 4. 1915 und 9. 4. 1915?] in Wien}\toendnotes[C]{\smallbreak}
\Standort{CUL, Schnitzler, B 118.}
\physDesc{Brief, 1 Blatt, 2 Seiten, 1448 Zeichen
\newline{}Handschrift: lila Tinte, lateinische Kurrent
\newline{}Schnitzler: 1) mit Bleistift »\textsc{Zweig}«  2) mit rotem Buntstift eine Unterstreichung}
\buchAbdrucke{\weitereDrucke{Stefan Zweig: \emph{Briefwechsel mit Hermann Bahr, Sigmund Freud, Rainer Maria
                        Rilke und Arthur Schnitzler}. Herausgegeben von Jeffrey B. Berlin, Hans-Ulrich Lindken und Donald A. Prater. Frankfurt am Main: \emph{S. Fischer} 1987, S. 394–395.} }\toendnotes[C]{\smallbreak}
\pstart
           {\pb}\textcolor{gray}{\textbf{SZ}}\hfill \textcolor{gray}{\textbf{VIII. KOCHGASSE\oindex{Wien@\textbf{Wien}!VIII., Josefstadt@\textbf{VIII., Josefstadt}!Kochgasse 8@\textbf{Kochgasse 8}, \emph{Wohngebäude}|pw}}}\pend
           
\pstart
           \raggedleft{}\textcolor{gray}{\textbf{WIEN\oindex{Wien@\textbf{Wien}, \emph{Verwaltungsgebiet}|pw},}}\pend
           \vspace{0.5em}
\pstart
           Sehr verehrter lieber Herr Doktor, ich wäre sehr froh, wenn Sie
               nächstens einmal mir wieder eine \label{K_L03653-1v}\edtext{Stunde mit Ihnen}{\lemma{\textnormal{\emph{Stunde mit Ihnen}}}\Cendnote{\textnormal{Das Korrespondenzstück
                  ist undatiert. Schnitzlers Antwort dürfte
                  dessen Schreiben vom XXXX Auszeichnungsfehler: Dokument L03768 nicht gefunden darstellen, so dass eine zeitliche Begrenzung nach hinten
                  vorliegt. Das gewünschte Treffen wäre folglich jenes am 11. 4. 1915.}}}\label{K_L03653-1}
               verstatten wollten: ich hätte gerne mit Ihnen über die \label{K_L03653-2v}\edtext{Angelegenheit Unseres gemeinsamen Freundes Rosenbaum\pwindex{Rosenbaum, Richard 4.\,11.\,1867 Žikov – 25.\,6.\,1942 Konzentrationslager Theresienstadt@\textsc{Rosenbaum, Richard} (4.\,11.\,1867 Žikov – 25.\,6.\,1942 Konzentrationslager Theresienstadt), \emph{Dramaturg, Verleger}|pw}}{\lemma{\textnormal{\emph{Angelegenheit … Rosenbaum}}}\Cendnote{\textnormal{Richard Rosenbaum\pwindex{Rosenbaum, Richard 4.\,11.\,1867 Žikov – 25.\,6.\,1942 Konzentrationslager Theresienstadt@\textsc{Rosenbaum, Richard} (4.\,11.\,1867 Žikov – 25.\,6.\,1942 Konzentrationslager Theresienstadt), \emph{Dramaturg, Verleger}|pwk} war
                     »literarisch-artistischer Sekretär« des \emph{Burgtheaters}\orgindex{Burgtheater@Burgtheater|pwk} und seit Jahren ein zentraler
                  Verantwortungsträger dieses Theaters. Bei der Besetzung des Postens des Direktors
                  wurde er wegen seiner jüdischen Abstammung nicht in Betracht gezogen. Die
                  Konflikte mit dem seit 1912 mit der Leitung betreuten Hugo Thimig\pwindex{Thimig, Hugo 16.\,6.\,1854 Dresden – 24.\,9.\,1944 Wien@\textsc{Thimig, Hugo} (16.\,6.\,1854 Dresden – 24.\,9.\,1944 Wien), \emph{Theaterleiter, Schauspieler}|pwk} waren seither zunehmends
                  eskaliert, so dass dieser die Entlassung Rosenbaums\pwindex{Rosenbaum, Richard 4.\,11.\,1867 Žikov – 25.\,6.\,1942 Konzentrationslager Theresienstadt@\textsc{Rosenbaum, Richard} (4.\,11.\,1867 Žikov – 25.\,6.\,1942 Konzentrationslager Theresienstadt), \emph{Dramaturg, Verleger}|pwk} herbeiführte. Schnitzler
                  war seit dem 31. 3. 1915 über das Ultimatum von Hugo Thimig\pwindex{Thimig, Hugo 16.\,6.\,1854 Dresden – 24.\,9.\,1944 Wien@\textsc{Thimig, Hugo} (16.\,6.\,1854 Dresden – 24.\,9.\,1944 Wien), \emph{Theaterleiter, Schauspieler}|pwk} informiert, wonach Rosenbaum\pwindex{Rosenbaum, Richard 4.\,11.\,1867 Žikov – 25.\,6.\,1942 Konzentrationslager Theresienstadt@\textsc{Rosenbaum, Richard} (4.\,11.\,1867 Žikov – 25.\,6.\,1942 Konzentrationslager Theresienstadt), \emph{Dramaturg, Verleger}|pwk} entweder freiwillig kündigen könne – oder mit seiner
                  Entlassung rechnen müsse.}}}\label{K_L03653-2} gesprochen. Immerhin sind Wir – wenn auch
               machtlos gegen solche Entschliessungen — \substVorne{}\textsuperscript{ein}\substDazwischen{}der\substHinten{} wesentlichste Teil der Interessierten und es ist die Frage, ob wir Uns ganz
               unbeteiligt zu einer solchen brutalen Entscheidung stellen sollten. Bis zu einem
               gewissen Grade glaube ich die »Reichspost\orgindex{Reichspost@Reichspost|pw}« in
               dieser Sache zu spüren – inwieweit D\textsuperscript{r}{ }R.\pwindex{Rosenbaum, Richard 4.\,11.\,1867 Žikov – 25.\,6.\,1942 Konzentrationslager Theresienstadt@\textsc{Rosenbaum, Richard} (4.\,11.\,1867 Žikov – 25.\,6.\,1942 Konzentrationslager Theresienstadt), \emph{Dramaturg, Verleger}|pw} im seiner \label{K_L03653-3v}\edtext{Offenheit des Wortes}{\lemma{\textnormal{\emph{Offenheit des Wortes}}}\Cendnote{\textnormal{Vgl. A. S.: \emph{Tagebuch}, 17. 4. 1915. }}}\label{K_L03653-3} Etwas
               verschuldet hat, vermag ich nicht zu entscheiden – und vielleicht wäre eine Form der
               moralischen Satisfaction für diesen vortrefflichen {\pb}Menschen zu finden, der nach siebzehn
               Jahren Tätigkeit \label{K_L03653-4v}\edtext{cum infamia}{\lemma{\textnormal{\emph{cum infamia}}}\Cendnote{\textnormal{lateinisch: mit Schimpf und Schande}}}\label{K_L03653-4}
               weggejagt werden soll. Ich weiss nicht, wie Sie in dieser Sache denken, doch ich
               zweifle nicht, dass sie auch Sie seelisch beschäftigt hat: mir scheint sie nicht
               bloss ein Einzelfall, sondern das Symptom einer Gesinnung, die sich jetzt schon
               mitten im Kriege entfaltet, um dann nachher agitatorisch und aggressiv zu werden und
               der man vielleicht heute schon in Parade entgegentreten sollte.\pend
           
\pstart
           D\textsuperscript{r}{ }R.\pwindex{Rosenbaum, Richard 4.\,11.\,1867 Žikov – 25.\,6.\,1942 Konzentrationslager Theresienstadt@\textsc{Rosenbaum, Richard} (4.\,11.\,1867 Žikov – 25.\,6.\,1942 Konzentrationslager Theresienstadt), \emph{Dramaturg, Verleger}|pw} weiss selbstverständlich nichts von diesem
               Brief. Er tut mir sehr leid: das Burgtheater\orgindex{Burgtheater@Burgtheater|pw} war
               schon so sehr der Sinn seiner Existenz und seines Fühlens geworden, dass er sich kaum
               jemals wird wieder ganz finden können.\pend
           
\pstart
           In herzlicher Liebe und Verehrung Ihr getreuer{\\[\baselineskip]}\spacefill\mbox{Stefan Zweig}\pend
           \leftskip=0em{}\selectlanguage{ngerman}\endnumbering\briefempfaengerindex{Schnitzler, Arthur@\textsc{Schnitzler, Arthur}!zzzZweig, Stefan@\emph{von Stefan Zweig}!1915-04-071@{{[}zwischen 7. 4. 1915 und 9. 4. 1915?{]}}|)be}\mylabel{L03653h}  \newcommand{\dateiname}{L03653}\newcommand{\titel}{Stefan Zweig an Arthur Schnitzler, [zwischen 7. 4. 1915 und 9. 4. 1915?]}\newcommand{\editorInnen}{Selma Jahnke und Martin Anton Müller}%% latex-leseansicht-abspann.tex
%% Abspann für die Leseansicht.
%% Der Schalter \ifkorrekturansicht ist bereits durch den Vorspann gesetzt.

%% latex-abspann.tex
%% Gemeinsamer Abspann für Korrekturansicht und Leseansicht.
%% Setzt den Schalter \ifkorrekturansicht voraus (gesetzt in den
%% einbindenden Dateien latex-korrekturansicht-abspann.tex bzw.
%% latex-leseansicht-abspann.tex).
%% ---------------------------------------------------------------

\normalsize

% Das esempio-Environment wird nur in der Leseansicht benötigt
\ifkorrekturansicht\else
\newenvironment{esempio}[3]%
{
    \vspace{1.5ex}
    \rlap{\underline{#1}}
    \par
    \setlength{\parindent}{0cm}
    \nopagebreak
    \leftskip=#2cm
    \rightskip=#3cm
}
{
    \par
}
\fi

\doendnotes{C}
\bigskip
\vfill

\clearpage

\footnotesize

\ifkorrekturansicht
  \lohead{\textsc{register}}
\fi

% theindex-Environment neu definieren ohne reledmac
\makeatletter
\renewenvironment{theindex}{%
  \ifkorrekturansicht
    \section*{\indexname}%
  \else
    \subsubsection*{Index der erwähnten Entitäten}%
  \fi
  \setlength{\parindent}{0pt}%
  \setlength{\parskip}{0pt plus 0.3pt}%
  \let\item\@idxitem
}{%
  \ifkorrekturansicht\clearpage\fi
}
\makeatother

\IfFileExists{\jobname-pw.ind}{\input{\jobname-pw.ind}}{}

% Quellenangabe nur in der Leseansicht
\ifkorrekturansicht\else
% Fallback-Definitionen, falls die .tex-Datei \titel etc. nicht gesetzt hat
\providecommand{\titel}{}
\providecommand{\editorInnen}{}
\providecommand{\dateiname}{\jobname}

\vspace{3cm}

\vfill

\footnotesize
\textsc{Quelle}: \titel. Herausgegeben von {\editorInnen}. In: \emph{Arthur Schnitzler: Briefwechsel mit Autorinnen und Autoren}.
 Digitale Edition, https://schnitzler-briefe.acdh.oeaw.ac.at/{\dateiname}.html (Stand \today)
\fi

\end{document}


