%% latex-leseansicht-vorspann.tex
%% Vorspann für die Leseansicht.
%% Lädt die gemeinsame Datei latex-vorspann.tex mit nicht gesetztem Schalter.

\newif\ifkorrekturansicht
\korrekturansichtfalse

\input{../tex-inputs/latex-vorspann}


               \section[Arthur Schnitzler an Richard Beer-Hofmann, 26. 4. 1897]{ Arthur Schnitzler an Richard Beer-Hofmann,
               26. 4. 1897}\nopagebreak\mylabel{v}\rehead{ }\begin{ledgroupsized}[t]{13cm}\normalsize\beginnumbering\briefempfaengerindex{Beer-Hofmann, Richard@\textsc{Beer-Hofmann, Richard}!zzzSchnitzler, Arthur@\emph{von Arthur Schnitzler}!1897-04-261@{26. 4. 1897}|(be} \toendnotes[C]{\smallbreak\pagebreak[2]} \Standort{YCGL, MSS 31.}
\physDesc{Brief, 2 Blätter (Briefpapier mit Trauerrand), 8 Seiten, Umschlag
\newline{}Handschrift: schwarze Tinte, deutsche Kurrent\newline{}Versand: 1) Stempel: »\nobreak{}\oindex{rue La Fayette@\textbf{rue La Fayette}|pwk}Paris 51 R. Lafayette, 26 Avril 97, 8\textsuperscript{E}\nobreak{}«.  2) Stempel: »\nobreak{}\oindex{I., Innere Stadt@\textbf{I., Innere Stadt}|pwk}Wien 1/1, 28. 4. 97, 9–10½V., Bestellt\nobreak{}«. }\buchAbdrucke{\weitereDrucke{1) Arthur Schnitzler: \emph{Briefe 1875–1912}. Hg. Therese Nickl und Heinrich Schnitzler. Frankfurt am Main: \emph{S. Fischer} 1981, S. 317–318.} \weitereDrucke{2) Arthur Schnitzler, Richard Beer-Hofmann: \emph{Briefwechsel 1891–1931}. Hg. Konstanze Fliedl. Wien, Zürich: \emph{Europaverlag} 1992, S. 102–103.} }\toendnotes[C]{\smallbreak}\pstart{}{\pb}Herrn \textsc{Dr. Richard
                     Beer-Hofmann}\pend{}\pstart{}\textsc{Wien\oindex{Wien@\textbf{Wien}|pw}}\pend{}\pstart{}\textsc{I. Bezirk\oindex{I., Innere Stadt@\textbf{I., Innere Stadt}|pw}}\pend{}\pstart{}\textsc{Wollzeile 15}\oindex{Wollzeile@\textbf{Wollzeile}|pw}.\pend{}\pstart{}\textsc{Autriche}\oindex{Oesterreich@\textbf{Österreich}|pw}\pend{}{\bigskip}\pstart
           \raggedleft{}{\pb}26. 4. 97.\pend
           \pstart{}Lieber Richard,\pend\pstart
           allerdings würden Sie für Paris\oindex{Paris@\textbf{Paris}|pw} einige hundert Jahre
               brauchen!\pend
           \pstart
           Nur die \textsc{Bouquinerien}! – Und die \label{K_L00670_1v}\edtext{\textsc{Emaux}}{\lemma{\textnormal{\emph{Emaux}}}\Cendnote{\textnormal{französisch: Emailarbeiten}}}\label{K_L00670_1h} aus dem 16 u
               17. Jahrhundert im \textsc{Louvre}\oindex{Louvre@\textbf{Louvre}|pw} – \pend
           \pstart
           Ich ſchreibe ſo beiläufig her, wo\substVorne{}\textsuperscript{rin}\substDazwischen{}bei\substHinten{} ich am
               heftigſten an Sie gedacht – {\pb}– und die \textsc{Chinoiserien} im \textsc{Guimet}\oindex{Museum Guimet@\textbf{Museum Guimet}|pw} –\pend
           \pstart
           Wäre ich Altenberg\pwindex{Altenberg, Peter 09.03.1859 – 08.01.1919@\textsc{Altenberg, Peter} (09.03.1859 – 08.01.1919), \emph{Schriftsteller}|pw}{ }ſo würde ich sagen:\pend
           \pstart
           Paris\oindex{Paris@\textbf{Paris}|pw} iſt »die« Stadt {\dotsfive}{ }\textsc{La ville}{ }{\dotsseven}\pend
           \pstart
           Paris\oindex{Paris@\textbf{Paris}|pw} iſt \textsc{la grande ville}{ }{\dotsfour}\pend
           \pstart
           \numberlinefalse{}–\numberlinetrue{}\pend
           \pstart
           Im Ernſt geſprochen (im Gegenſatz zu Altenberg\pwindex{Altenberg, Peter 09.03.1859 – 08.01.1919@\textsc{Altenberg, Peter} (09.03.1859 – 08.01.1919), \emph{Schriftsteller}|pw}.):
               Die \uline{Form} für alles iſt da, \introOben{}das
                  iſt\introOben{} das weſentliche: die ganz {\pb}großen
                  \introOben{}ſchöpferiſchen\introOben{} Talente ſcheinen heute noch zu fehlen.
               Dagegen ſind die \textsc{reproducirenden} da; die ununterbrochen für
               die Form ſorgen. Auch die Decoration iſt für alles da; jederzeit können die großen
               Künſtler auftreten, ohne sich um etwas andres kü{\geminationm}e\textcolor{gray}{rn} zu
               müſſen als um ihr Genie. – Auch große Menſchen {\pb}jeder Art finden alles bereit; der \textsc{Concorde}-Platz\oindex{Place de la Concorde@\textbf{Place de la Concorde}|pw}{ }ſcheint
               eigentlich nur auf einen neuen Napoleon\pwindex{Bonaparte, Napoleon 15.08.1769 – 21.05.1821@\textsc{Bonaparte, Napoleon} (15.08.1769 – 21.05.1821), \emph{Kaiser}|pwv} zu warten.\pend
           \pstart
           – Aber dieſen Brief hab ich nur angefangen um mich bei Ihnen nach Ihnen zu
               erkundigen. Wie geht es Paula\pwindex{Beer-Hofmann, Paula 25.02.1879 – 30.10.1939@\textsc{Beer-Hofmann, Paula} (25.02.1879 – 30.10.1939)|pw}? Bei »uns\pwindex{Reinhard, Marie 13.03.1871 – 18.03.1899@\textsc{Reinhard, Marie} (13.03.1871 – 18.03.1899), \emph{Gesangspädagogin}|pwv}« – mit »Rieſen{\pb}ſchritten«.\pend
           \pstart
           Bleiben Sie in Wien\oindex{Wien@\textbf{Wien}|pw}? – \pend
           \pstart
           – Darüber ſein Sie ruhig: zu einem »wirklichen« Brief ko{\geminationm} ich hier nicht.\pend
           \pstart
           Graf\pwindex{Graf, Max 01.10.1873 – 24.06.1958@\textsc{Graf, Max} (01.10.1873 – 24.06.1958), \emph{Kritiker}|pw} iſt hier, Sie wiſſen ja, dem Sie eine
               zärtliche Empfehlung an Paul\pwindex{Goldmann, Paul 31.01.1865 – 25.09.1935@\textsc{Goldmann, Paul} (31.01.1865 – 25.09.1935), \emph{Schriftsteller, Journalist}|pw} gegeben. Den treff
               ich natürlich immer. {\pb}Alſo könnte der kleine Kraus\pwindex{Kraus, Karl 28.04.1874 – 12.06.1936@\textsc{Kraus, Karl} (28.04.1874 – 12.06.1936), \emph{Schriftsteller, Publizist}|pw} bald einen Artikel über die Flucht aus Wien\oindex{Wien@\textbf{Wien}|pw}{ }ſchreiben. –\pend
           \pstart
           Wie leben Sie? – \pend
           \pstart
           Ich: Vormittg \textsc{Louvre}\oindex{Louvre@\textbf{Louvre}|pw} oder \textsc{Luxemburg}\oindex{Jardin du Luxembourg@\textbf{Jardin du Luxembourg}|pw} oder so was; Abends immer im Theater. Entzückend die ganz kleinen. Es wi{\geminationm}elt von »Flohtheatern des Arthur Schnitzler«.\pend
           \pstart
           {\pb}Geſtern oder vorgeſtern Nachm in einem dieſer
               kleinen »\textsc{la Bodinière}« Aufführung von \introOben{}franzöſ.\oindex{Frankreich@\textbf{Frankreich}|pw}\introOben{} Muſik des 16. u 17. Jahrhunderts.\pend
           \pstart
           – In andern werden dieſe hübſchen Kleinigkeiten von \textsc{Lavedan}\pwindex{Lavedan, Henri Leon 09.04.1859 – 4.9.1940@\textsc{Lavedan, Henri Léon} (09.04.1859 – 4.9.1940), \emph{Schriftsteller, Journalist}|pw}, von \textsc{Courteline}\pwindex{Courteline, Georges 25.06.1858 – 25.06.1929@\textsc{Courteline, Georges} (25.06.1858 – 25.06.1929), \emph{Schriftsteller}|pw} aufgeführt. Oder, wie ich \label{K_L00670_2v}\edtext{neulich}{\lemma{\textnormal{\emph{neulich}}}\Cendnote{\textnormal{am
                     20. 4. 1897}}}\label{K_L00670_2h} in der »\textsc{Roulotte}\oindex{La Roulotte@\textbf{La Roulotte}|pw}« ſah, ein Volkslied von zwölf Zeilen wird einfach »aufgeführt«. Er und {\pb}Sie – kein lebendes Bild, was beka{\geminationn}tlich ſehr
               todt ist, ſondern ſie \uline{ſpielen} das Volkslied. –\pend
           \pstart
           Überhaupt »hier ka{\geminationn} man ſchon einmal alles haben«.\pend
           \pstart
           Schreiben Sie mir bald.\pend
           \pstart
           Adreſſe 5 \textsc{rue de Maubeuge}\oindex{rue de Maubeuge@\textbf{rue de Maubeuge}|pw}\pend
           \pstart
           Herzlichst Ihr{\\[\baselineskip]}\spacefill\mbox{Arthur.}\pend
           \leftskip=0em{}\pstart
           Paul\pwindex{Goldmann, Paul 31.01.1865 – 25.09.1935@\textsc{Goldmann, Paul} (31.01.1865 – 25.09.1935), \emph{Schriftsteller, Journalist}|pw}{ }ſchon 9 Tage in Frankfurt\oindex{Frankfurt am Main@\textbf{Frankfurt am Main}|pw}; ko{\geminationm}t bald. –\pend
                     \endnumbering\briefempfaengerindex{Beer-Hofmann, Richard@\textsc{Beer-Hofmann, Richard}!zzzSchnitzler, Arthur@\emph{von Arthur Schnitzler}!1897-04-261@{26. 4. 1897}|)be}\mylabel{h}\end{ledgroupsized}  \newcommand{\dateiname}{L00670}\newcommand{\titel}{Arthur Schnitzler an Richard Beer-Hofmann, 26. 4. 1897}\newcommand{\editorInnen}{Martin Anton Müller und Gerd-Hermann Susen}
            \footnotesize
\begin{ledgroupsized}[t]{11.5cm}
\doendnotes{C}
\end{ledgroupsized}
         %% latex-leseansicht-abspann.tex
%% Abspann für die Leseansicht.
%% Der Schalter \ifkorrekturansicht ist bereits durch den Vorspann gesetzt.

%% latex-abspann.tex
%% Gemeinsamer Abspann für Korrekturansicht und Leseansicht.
%% Setzt den Schalter \ifkorrekturansicht voraus (gesetzt in den
%% einbindenden Dateien latex-korrekturansicht-abspann.tex bzw.
%% latex-leseansicht-abspann.tex).
%% ---------------------------------------------------------------

\normalsize

% Das esempio-Environment wird nur in der Leseansicht benötigt
\ifkorrekturansicht\else
\newenvironment{esempio}[3]%
{
    \vspace{1.5ex}
    \rlap{\underline{#1}}
    \par
    \setlength{\parindent}{0cm}
    \nopagebreak
    \leftskip=#2cm
    \rightskip=#3cm
}
{
    \par
}
\fi

\doendnotes{C}
\bigskip
\vfill

\clearpage

\footnotesize

\ifkorrekturansicht
  \lohead{\textsc{register}}
\fi

% theindex-Environment neu definieren ohne reledmac
\makeatletter
\renewenvironment{theindex}{%
  \ifkorrekturansicht
    \section*{\indexname}%
  \else
    \subsubsection*{Index der erwähnten Entitäten}%
  \fi
  \setlength{\parindent}{0pt}%
  \setlength{\parskip}{0pt plus 0.3pt}%
  \let\item\@idxitem
}{%
  \ifkorrekturansicht\clearpage\fi
}
\makeatother

\IfFileExists{\jobname-pw.ind}{\input{\jobname-pw.ind}}{}

% Quellenangabe nur in der Leseansicht
\ifkorrekturansicht\else
% Fallback-Definitionen, falls die .tex-Datei \titel etc. nicht gesetzt hat
\providecommand{\titel}{}
\providecommand{\editorInnen}{}
\providecommand{\dateiname}{\jobname}

\vspace{3cm}

\vfill

\footnotesize
\textsc{Quelle}: \titel. Herausgegeben von {\editorInnen}. In: \emph{Arthur Schnitzler: Briefwechsel mit Autorinnen und Autoren}.
 Digitale Edition, https://schnitzler-briefe.acdh.oeaw.ac.at/{\dateiname}.html (Stand \today)
\fi

\end{document}


      