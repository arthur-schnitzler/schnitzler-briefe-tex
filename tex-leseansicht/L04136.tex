%% latex-leseansicht-vorspann.tex
%% Vorspann für die Leseansicht.
%% Lädt die gemeinsame Datei latex-vorspann.tex mit nicht gesetztem Schalter.

\newif\ifkorrekturansicht
\korrekturansichtfalse

\input{../tex-inputs/latex-vorspann}


\section[Arthur Schnitzler an Gustav Schwarzkopf, 16. 8. 1899]{L04136 Arthur Schnitzler an Gustav Schwarzkopf, 16. 8. 1899}
\nopagebreak\mylabel{L04136v}
\rehead{ }\normalsize\beginnumbering\briefempfaengerindex{Schwarzkopf, Gustav@\textsc{Schwarzkopf, Gustav}!zzzSchnitzler, Arthur@\emph{von Arthur Schnitzler}!1899-08-161@{16. 8. 1899}|(be}
\toendnotes[C]{\smallbreak\pagebreak[2]}
\correspDesc{Versand  durch Arthur Schnitzler am 16. 8. 1899 in Bad Ischl
\newline{}Erhalt  durch Gustav Schwarzkopf am 17. 8. 1899 in Wien}\toendnotes[C]{\smallbreak}
\Standort{CUL, Schnitzler, B 96.}
\physDesc{Postkarte, 339 Zeichen
\newline{}Handschrift: Bleistift, deutsche Kurrent
\newline{}Versand: 1) Stempel: »\nobreak{}\oindex{Bad Ischl@\textbf{Bad Ischl}|pwk}Ischl, 16. 8. 99, 12–1N\nobreak{}«.   2) Stempel: »\nobreak{}\oindex{I., Innere Stadt@\textbf{I., Innere Stadt}, \emph{Verwaltungsgebiet}|pwk}Wien 1/1 1, 17. 8. \textcolor{gray}{99}, 8–9½V.\nobreak{}«. }\toendnotes[C]{\smallbreak}\pstart{}{\pb}Herrn \textsc{Gustav Schwarzkopf}\pend{}\pstart{}\textsc{Wien}\oindex{Wien@\textbf{Wien}, \emph{Verwaltungsgebiet}|pw}\pend{}\pstart{}\textsc{I. Tiefer Graben 23}\oindex{Wien@\textbf{Wien}!I., Innere Stadt@\textbf{I., Innere Stadt}!Tiefer Graben 23@\textbf{Tiefer Graben 23}, \emph{Wohngebäude}|pw}.\pend{}{\bigskip}\vspace{1em}
\pstart
           \noindent{}{\pb}lieber Guſtav! Wann kommen Sie? – Laſſen Sie
       michs allzeit wiſſen, damit ich Ihnen beim
                 Petter\oindex{Hotel und Pension Rudolfshöhe (Leopold Petter)@\textbf{Hotel und Pension Rudolfshöhe (Leopold Petter)}, \emph{Hotel}|pw} ein Zi{\geminationm}er verſorge. \label{K_L04136-1v}\edtext{Kadelburg\pwindex{Kadelburg, Gustav 26.\,7.\,1851 Budapest – 11.\,9.\,1925 Berlin@\textsc{Kadelburg, Gustav} (26.\,7.\,1851 Budapest – 11.\,9.\,1925 Berlin), \emph{Schriftsteller, Schauspieler}|pw} reiſte
                 heute ab}{\lemma{\textnormal{\emph{Kadelburg … ab}}}\Cendnote{\textnormal{Die \emph{Ischler Badeliste}\pwindex{Ischler Cur-Liste@\emph{Ischler Cur-Liste}|pwk} führte ihn ab 10. 8. 1899 (\emph{Ischler Fremden-Liste}, Nr. 48, 10. 8. 1899) im
                     Hôtel und Pension Rudolfshöhe (Inh. Leopold Petter)\oindex{Hotel und Pension Rudolfshöhe (Leopold Petter)@\textbf{Hotel und Pension Rudolfshöhe (Leopold Petter)}, \emph{Hotel}|pwk}, wo auch Schnitzler
                    wohnte.}}}\label{K_L04136-1}, bedauernd, Guſtl nicht mehr{ }ſehen zu können. Ich bin geſtern angelangt,
      nach der wirklich schönen Reiſe, – in ſehr
      mäßiger Sti{\geminationm}ung\pend
           
\pstart
           Herzliche Grüße{\\[\baselineskip]}Ihr \spacefill\mbox{Arthur}\pend
           \leftskip=0em{}\selectlanguage{ngerman}\endnumbering\briefempfaengerindex{Schwarzkopf, Gustav@\textsc{Schwarzkopf, Gustav}!zzzSchnitzler, Arthur@\emph{von Arthur Schnitzler}!1899-08-161@{16. 8. 1899}|)be}\mylabel{L04136h}
\begin{anhang}
\end{anhang}\newcommand{\dateiname}{L04136}\newcommand{\titel}{Arthur Schnitzler an Gustav Schwarzkopf, 16. 8. 1899}\newcommand{\editorInnen}{Herausgegeben von Jahnke, SelmaMüller, Martin Anton}%% latex-leseansicht-abspann.tex
%% Abspann für die Leseansicht.
%% Der Schalter \ifkorrekturansicht ist bereits durch den Vorspann gesetzt.

%% latex-abspann.tex
%% Gemeinsamer Abspann für Korrekturansicht und Leseansicht.
%% Setzt den Schalter \ifkorrekturansicht voraus (gesetzt in den
%% einbindenden Dateien latex-korrekturansicht-abspann.tex bzw.
%% latex-leseansicht-abspann.tex).
%% ---------------------------------------------------------------

\normalsize

% Das esempio-Environment wird nur in der Leseansicht benötigt
\ifkorrekturansicht\else
\newenvironment{esempio}[3]%
{
    \vspace{1.5ex}
    \rlap{\underline{#1}}
    \par
    \setlength{\parindent}{0cm}
    \nopagebreak
    \leftskip=#2cm
    \rightskip=#3cm
}
{
    \par
}
\fi

\doendnotes{C}
\bigskip
\vfill

\clearpage

\footnotesize

\ifkorrekturansicht
  \lohead{\textsc{register}}
\fi

% theindex-Environment neu definieren ohne reledmac
\makeatletter
\renewenvironment{theindex}{%
  \ifkorrekturansicht
    \section*{\indexname}%
  \else
    \subsubsection*{Index der erwähnten Entitäten}%
  \fi
  \setlength{\parindent}{0pt}%
  \setlength{\parskip}{0pt plus 0.3pt}%
  \let\item\@idxitem
}{%
  \ifkorrekturansicht\clearpage\fi
}
\makeatother

\IfFileExists{\jobname-pw.ind}{\input{\jobname-pw.ind}}{}

% Quellenangabe nur in der Leseansicht
\ifkorrekturansicht\else
% Fallback-Definitionen, falls die .tex-Datei \titel etc. nicht gesetzt hat
\providecommand{\titel}{}
\providecommand{\editorInnen}{}
\providecommand{\dateiname}{\jobname}

\vspace{3cm}

\vfill

\footnotesize
\textsc{Quelle}: \titel. Herausgegeben von {\editorInnen}. In: \emph{Arthur Schnitzler: Briefwechsel mit Autorinnen und Autoren}.
 Digitale Edition, https://schnitzler-briefe.acdh.oeaw.ac.at/{\dateiname}.html (Stand \today)
\fi

\end{document}


