%% latex-leseansicht-vorspann.tex
%% Vorspann für die Leseansicht.
%% Lädt die gemeinsame Datei latex-vorspann.tex mit nicht gesetztem Schalter.

\newif\ifkorrekturansicht
\korrekturansichtfalse

\input{../tex-inputs/latex-vorspann}


\section[Arthur Schnitzler an Berta Zuckerkandl, 18. 4. 1911]{L03981 Arthur Schnitzler an Berta Zuckerkandl, 18. 4. 1911}
\nopagebreak\mylabel{L03981v}
\rehead{ }\normalsize\beginnumbering\briefempfaengerindex{Zuckerkandl, Berta@\textsc{Zuckerkandl, Berta}!zzzSchnitzler, Arthur@\emph{von Arthur Schnitzler}!1911-04-181@{18. 4. 1911}|(be}
\toendnotes[C]{\smallbreak\pagebreak[2]}
\correspDesc{Versand  durch Arthur Schnitzler am 18. 4. 1911 in Menton
\newline{}Erhalt  durch Berta Zuckerkandl im Zeitraum [19. 4. 1911
                  – 23. 4. 1911?] in Wien}\toendnotes[C]{\smallbreak}
\Standort{Wien, Österreichische Nationalbibliothek, 405/B78/1 LIT MAG.}
\physDesc{Brief, 1 Blatt, 4 Seiten, 1759 Zeichen
\newline{}Handschrift: schwarze Tinte, lateinische Kurrent}\toendnotes[C]{\smallbreak}
\pstart
           \raggedleft{}{\pb}\textcolor{gray}{\textbf{GRAND HOTEL NATIONAL}}\oindex{Grand Hôtel National [Menton]@\textbf{Grand Hôtel National [Menton]}, \emph{Hotel}|pw}\pend
           
\pstart
           \raggedleft{}\textcolor{gray}{\textbf{MENTON}}\oindex{Menton@\textbf{Menton}|pw}{ }18. 4. 1911.\pend
           
\pstart{}Verehrte gnädige Frau,\pend\vspace{0.5em}
\pstart
           ich danke Ihnen sehr für Ihren Brief. Medardus\pwindex{Schnitzler, Arthur 15.\,5.\,1862 Wien – 21.\,10.\,1931 ebd.@\textsc{Schnitzler, Arthur} (15.\,5.\,1862 Wien – 21.\,10.\,1931 ebd.), \emph{Schriftsteller, Mediziner}!junge Medardus. Dramatische Historie in einem Vorspiel und fünf Aufzügen@\strich\emph{Der junge Medardus. Dramatische Historie in einem Vorspiel und fünf Aufzügen}|pw}
               hab ich für Frankreich\oindex{Frankreich@\textbf{Frankreich}|pw} noch nicht vergeben – und
                  we{\geminationn} Sie glauben etwas damit {\kaufmannsund} dafür thun zu können, so werde ich höchst
               einverstanden sein. Nur theil ich Ihre Hoffnungen vorläufig gar nicht – wobei ich \substVorne{}\textsuperscript{mit}\substDazwischen{}von\substHinten{} meinen bisherigen Erfahrungen in Frankreich\oindex{Frankreich@\textbf{Frankreich}|pw} ganz absehen will. Speziell aber der Medardus\pwindex{Schnitzler, Arthur 15.\,5.\,1862 Wien – 21.\,10.\,1931 ebd.@\textsc{Schnitzler, Arthur} (15.\,5.\,1862 Wien – 21.\,10.\,1931 ebd.), \emph{Schriftsteller, Mediziner}!junge Medardus. Dramatische Historie in einem Vorspiel und fünf Aufzügen@\strich\emph{Der junge Medardus. Dramatische Historie in einem Vorspiel und fünf Aufzügen}|pw} – welcher französische\oindex{Frankreich@\textbf{Frankreich}|pw} Director wird sich dieser Mühe unterziehen? Antoine\pwindex{Antoine, André 31.\,1.\,1858 Limoges – 23.\,10.\,1943 Le Pouliguen@\textsc{Antoine, André} (31.\,1.\,1858 Limoges – 23.\,10.\,1943 Le Pouliguen), \emph{Theaterleiter, Schauspieler}|pw}?? Er hat von mir schon \label{K_L03981-1v}\edtext{zwei Einacter\pwindex{Schnitzler, Arthur 15.\,5.\,1862 Wien – 21.\,10.\,1931 ebd.@\textsc{Schnitzler, Arthur} (15.\,5.\,1862 Wien – 21.\,10.\,1931 ebd.), \emph{Schriftsteller, Mediziner}!grüne Kakadu. Groteske in einem Akt@\strich\emph{Der grüne Kakadu. Groteske in einem Akt}|pwv}\pwindex{Schnitzler, Arthur 15.\,5.\,1862 Wien – 21.\,10.\,1931 ebd.@\textsc{Schnitzler, Arthur} (15.\,5.\,1862 Wien – 21.\,10.\,1931 ebd.), \emph{Schriftsteller, Mediziner}!Gefährtin. Schauspiel in einem Akt@\strich\emph{Die Gefährtin. Schauspiel in einem Akt}|pwv} aufgeführt}{\lemma{\textnormal{\emph{zwei Einacter aufgeführt}}}\Cendnote{\textnormal{Die Übersetzung\pwindex{Schnitzler, Arthur 15.\,5.\,1862 Wien – 21.\,10.\,1931 ebd.@\textsc{Schnitzler, Arthur} (15.\,5.\,1862 Wien – 21.\,10.\,1931 ebd.), \emph{Schriftsteller, Mediziner}!Compagne. Comédie en une acte@\strich\emph{La Compagne. Comédie en une acte}|pwkv} von \emph{Die Gefährtin}\pwindex{Schnitzler, Arthur 15.\,5.\,1862 Wien – 21.\,10.\,1931 ebd.@\textsc{Schnitzler, Arthur} (15.\,5.\,1862 Wien – 21.\,10.\,1931 ebd.), \emph{Schriftsteller, Mediziner}!Gefährtin. Schauspiel in einem Akt@\strich\emph{Die Gefährtin. Schauspiel in einem Akt}|pwk} hatte
                  am 29. 4. 1902 am \emph{Théâtre
                     Antoine}\orgindex{Théâtre Antoine@Théâtre Antoine|pwk}{ }Premiere\eventindex{Théâtre Antoine-Simone Berriau@\textbf{Théâtre Antoine-Simone Berriau}!Premiere von La Compagne, 29.4.1902@Premiere von La Compagne, 29.4.1902|pwkv} und wurde dort insgesamt vier
                  mal aufgeführt, die Übersetzung\pwindex{Schnitzler, Arthur 15.\,5.\,1862 Wien – 21.\,10.\,1931 ebd.@\textsc{Schnitzler, Arthur} (15.\,5.\,1862 Wien – 21.\,10.\,1931 ebd.), \emph{Schriftsteller, Mediziner}!Au Perroquet Vert@\strich\emph{Au Perroquet Vert}|pwkv} von \emph{Der grüne Kakadu}\pwindex{Schnitzler, Arthur 15.\,5.\,1862 Wien – 21.\,10.\,1931 ebd.@\textsc{Schnitzler, Arthur} (15.\,5.\,1862 Wien – 21.\,10.\,1931 ebd.), \emph{Schriftsteller, Mediziner}!grüne Kakadu. Groteske in einem Akt@\strich\emph{Der grüne Kakadu. Groteske in einem Akt}|pwk}
                  feierte am 7. 11. 1903{ }ebendort\oindex{Théâtre Antoine-Simone Berriau@\textbf{Théâtre Antoine-Simone Berriau}, \emph{Theater}|pwkv}{ }Premiere\eventindex{Théâtre Antoine-Simone Berriau@\textbf{Théâtre Antoine-Simone Berriau}!Premiere von Au Perroquet Vert, 7.11.1903@Premiere von Au Perroquet Vert, 7.11.1903|pwkv}
                  und erlebte insgesamt zwölf Aufführungen.}}}\label{K_L03981-1}: Gefährtin\pwindex{Schnitzler, Arthur 15.\,5.\,1862 Wien – 21.\,10.\,1931 ebd.@\textsc{Schnitzler, Arthur} (15.\,5.\,1862 Wien – 21.\,10.\,1931 ebd.), \emph{Schriftsteller, Mediziner}!Gefährtin. Schauspiel in einem Akt@\strich\emph{Die Gefährtin. Schauspiel in einem Akt}|pw}{ }{\kaufmannsund}{ }Kakadu\pwindex{Schnitzler, Arthur 15.\,5.\,1862 Wien – 21.\,10.\,1931 ebd.@\textsc{Schnitzler, Arthur} (15.\,5.\,1862 Wien – 21.\,10.\,1931 ebd.), \emph{Schriftsteller, Mediziner}!grüne Kakadu. Groteske in einem Akt@\strich\emph{Der grüne Kakadu. Groteske in einem Akt}|pw}, {\pb}mit guten aber nicht dauernden
               Erfolge, der Cyclus \label{K_L03981-2v}\edtext{Lebendige Stunden\pwindex{Schnitzler, Arthur 15.\,5.\,1862 Wien – 21.\,10.\,1931 ebd.@\textsc{Schnitzler, Arthur} (15.\,5.\,1862 Wien – 21.\,10.\,1931 ebd.), \emph{Schriftsteller, Mediziner}!Lebendige Stunden. Vier Einakter@\strich\emph{Lebendige Stunden. Vier Einakter}|pw}}{\lemma{\textnormal{\emph{Lebendige Stunden}}}\Cendnote{\textnormal{Von diesen Übersetzungen wurden drei publiziert: \emph{La Femme au poignard}\pwindex{Schnitzler, Arthur 15.\,5.\,1862 Wien – 21.\,10.\,1931 ebd.@\textsc{Schnitzler, Arthur} (15.\,5.\,1862 Wien – 21.\,10.\,1931 ebd.), \emph{Schriftsteller, Mediziner}!femme au poignard@\strich\emph{La femme au poignard}|pwk}. In: \emph{Revue de Paris}\pwindex{Revue de Paris@\emph{La Revue de Paris}|pwk}, Jg. 19, Reihe 3 (Mai–Juni), 15. 5. 1912, S. 225–238, \emph{Les Derniers masques. Comédie en un act}\pwindex{Schnitzler, Arthur 15.\,5.\,1862 Wien – 21.\,10.\,1931 ebd.@\textsc{Schnitzler, Arthur} (15.\,5.\,1862 Wien – 21.\,10.\,1931 ebd.), \emph{Schriftsteller, Mediziner}!Derniers masques. Comédie en un act@\strich\emph{Les Derniers masques. Comédie en un act}|pwk}. In: \emph{Revue Politique et Littéraire. Revue bleue}\pwindex{Revue politique et littéraire@\emph{La Revue politique et littéraire}|pwk}, Jg. 50, 2. Semester, Nr. 20, 11. 11. 1912, S. 618–622; Nr. 21, 23. 11. 1912, S. 652–657 und \emph{Littérature. Comédi en en act}\pwindex{Schnitzler, Arthur 15.\,5.\,1862 Wien – 21.\,10.\,1931 ebd.@\textsc{Schnitzler, Arthur} (15.\,5.\,1862 Wien – 21.\,10.\,1931 ebd.), \emph{Schriftsteller, Mediziner}!Littérature. Comédie en en act@\strich\emph{Littérature. Comédie en en act}|pwk}. In: \emph{La Revue bleue. La Revue politique et littéraire}\pwindex{Revue politique et littéraire@\emph{La Revue politique et littéraire}|pwk}, Jahrgang 52, 1. Semester, Nr. 1, 3. 1. 1914, S. 11–16; Nr. 2, 10. 1. 1914, S. 44–50.}}}\label{K_L03981-2} (übersetzt
               von Rémon\pwindex{Rémon, Maurice 27.\,11.\,1861 Paris – 20.\,6.\,1945 Mérignac@\textsc{Rémon, Maurice} (27.\,11.\,1861 Paris – 20.\,6.\,1945 Mérignac), \emph{Übersetzer}|pw} u Mme Valentin\pwindex{Valentin, Noémi @\textsc{Valentin, Noémi}, \emph{Übersetzerin}|pw}) liegt seit etwa 6 Jahren angenommen
               bei ihm, aber er denkt nicht daran die Sachen\pwindex{Schnitzler, Arthur 15.\,5.\,1862 Wien – 21.\,10.\,1931 ebd.@\textsc{Schnitzler, Arthur} (15.\,5.\,1862 Wien – 21.\,10.\,1931 ebd.), \emph{Schriftsteller, Mediziner}!Lebendige Stunden. Vier Einakter@\strich\emph{Lebendige Stunden. Vier Einakter}|pwv} aufzuführen. Nach einem Brief von Lugne Poë\pwindex{Lugné-Poe, Aurélien-Marie 27.\,12.\,1869 Paris – 19.\,6.\,1940 Villeneuve-les-Avignon@\textsc{Lugné-Poe, Aurélien-Marie} (27.\,12.\,1869 Paris – 19.\,6.\,1940 Villeneuve-les-Avignon), \emph{Theaterleiter, Regisseur, Schauspieler}|pw} an mich (anläßlich Liebelei\pwindex{Schnitzler, Arthur 15.\,5.\,1862 Wien – 21.\,10.\,1931 ebd.@\textsc{Schnitzler, Arthur} (15.\,5.\,1862 Wien – 21.\,10.\,1931 ebd.), \emph{Schriftsteller, Mediziner}!Liebelei. Schauspiel in drei Akten@\strich\emph{Liebelei. Schauspiel in drei Akten}|pw} für die sich er und seine Frau\pwindex{Desprès, Suzanne 18.\,12.\,1875 Verdun – 29.\,6.\,1951 Paris@\textsc{Desprès, Suzanne} (18.\,12.\,1875 Verdun – 29.\,6.\,1951 Paris), \emph{Schauspielerin}|pwv} (derer Name mir in diesen Moment absolut nicht
               einfallen will) interessirt haben, stehen jetzt die Chancen für deutsche Dichter
               recht übel in Frankreich\oindex{Frankreich@\textbf{Frankreich}|pw}. – Nachdem ich Ihnen auf
               diese Weise, verehrte gnädige Frau, den zu einen solchen Unternehmen nöthigen Muth
               eingeflößt habe, ka{\geminationn} ich nur wiederholen –: we{\geminationn} Sie es wagen wollen – {\pb}Jedenfalls werde ich bitten, in Wien\oindex{Wien@\textbf{Wien}, \emph{Verwaltungsgebiet}|pw} (wo ich \label{K_L03981-3v}\edtext{Anfang Mai}{\lemma{\textnormal{\emph{Anfang Mai}}}\Cendnote{\textnormal{Schnitzler hatte Wien\oindex{Wien@\textbf{Wien}, \emph{Verwaltungsgebiet}|pwk}
                  am 10. 4. 1911
                  verlassen zu einer Reise über München\oindex{München@\textbf{München}|pwk} und Garmisch-Partenkirchen\oindex{Garmisch-Partenkirchen@\textbf{Garmisch-Partenkirchen}, \emph{Hauptstadt}|pwk} nach Ligurien\oindex{Ligurien@\textbf{Ligurien}|pwk} und Menton\oindex{Menton@\textbf{Menton}|pwk}
                  mit Abstechern nach Monte Carlo\oindex{Monte Carlo@\textbf{Monte Carlo}, \emph{Ehemaliger Ort}|pwk} und Nizza\oindex{Nizza@\textbf{Nizza}, \emph{Hauptstadt}|pwk}. Am 3. 5. 1911 kehrte er nach hause zurück, wo es laut \emph{Tagebuch}\pwindex{Schnitzler, Arthur 15.\,5.\,1862 Wien – 21.\,10.\,1931 ebd.@\textsc{Schnitzler, Arthur} (15.\,5.\,1862 Wien – 21.\,10.\,1931 ebd.), \emph{Schriftsteller, Mediziner}!Tagebuch@\strich\emph{Tagebuch}|pwk} am 10. 5. 1911 zur persönlichen Besprechung kam: »Bei der Hofr. Zuckerkandl\pwindex{Zuckerkandl, Berta 13.\,4.\,1864 Wien – 16.\,10.\,1945 Paris@\textsc{Zuckerkandl, Berta} (13.\,4.\,1864 Wien – 16.\,10.\,1945 Paris), \emph{Schriftstellerin, Journalistin, Übersetzerin}|pw} (die mir wegen Antoine\pwindex{Antoine, André 31.\,1.\,1858 Limoges – 23.\,10.\,1943 Le Pouliguen@\textsc{Antoine, André} (31.\,1.\,1858 Limoges – 23.\,10.\,1943 Le Pouliguen), \emph{Theaterleiter, Schauspieler}|pw} – Medardus\pwindex{Schnitzler, Arthur 15.\,5.\,1862 Wien – 21.\,10.\,1931 ebd.@\textsc{Schnitzler, Arthur} (15.\,5.\,1862 Wien – 21.\,10.\,1931 ebd.), \emph{Schriftsteller, Mediziner}!junge Medardus. Dramatische Historie in einem Vorspiel und fünf Aufzügen@\strich\emph{Der junge Medardus. Dramatische Historie in einem Vorspiel und fünf Aufzügen}|pw} geschrieben). Über meine bisherigen Erfahrungen und Chancen in Frankreich\oindex{Frankreich@\textbf{Frankreich}|pw}.«}}}\label{K_L03981-3} zu sein hoffe)
               persönlich über diese, und auch die Weite
               Land\pwindex{Schnitzler, Arthur 15.\,5.\,1862 Wien – 21.\,10.\,1931 ebd.@\textsc{Schnitzler, Arthur} (15.\,5.\,1862 Wien – 21.\,10.\,1931 ebd.), \emph{Schriftsteller, Mediziner}!weite Land. Tragikomödie in fünf Akten@\strich\emph{Das weite Land. Tragikomödie in fünf Akten}|pw}-Angelegenheit sprechen zu dürfen. Dieses Stück\pwindex{Schnitzler, Arthur 15.\,5.\,1862 Wien – 21.\,10.\,1931 ebd.@\textsc{Schnitzler, Arthur} (15.\,5.\,1862 Wien – 21.\,10.\,1931 ebd.), \emph{Schriftsteller, Mediziner}!weite Land. Tragikomödie in fünf Akten@\strich\emph{Das weite Land. Tragikomödie in fünf Akten}|pwv} scheint mir nach den internationalen Seite mehr zu
               versprechen als der Medardus\pwindex{Schnitzler, Arthur 15.\,5.\,1862 Wien – 21.\,10.\,1931 ebd.@\textsc{Schnitzler, Arthur} (15.\,5.\,1862 Wien – 21.\,10.\,1931 ebd.), \emph{Schriftsteller, Mediziner}!junge Medardus. Dramatische Historie in einem Vorspiel und fünf Aufzügen@\strich\emph{Der junge Medardus. Dramatische Historie in einem Vorspiel und fünf Aufzügen}|pw}. Man ist sowohl von
                  England\oindex{England@\textbf{England}, \emph{Land}|pw} als von Frankreich\oindex{Frankreich@\textbf{Frankreich}|pw} her (ohne \substVorne{}\textsuperscript{das Stück}\substDazwischen{}es\substHinten{} zu kennen) wegen dieses Stücks\pwindex{Schnitzler, Arthur 15.\,5.\,1862 Wien – 21.\,10.\,1931 ebd.@\textsc{Schnitzler, Arthur} (15.\,5.\,1862 Wien – 21.\,10.\,1931 ebd.), \emph{Schriftsteller, Mediziner}!weite Land. Tragikomödie in fünf Akten@\strich\emph{Das weite Land. Tragikomödie in fünf Akten}|pwv} an mich herangetreten, ich habe mich aber noch nicht gebunden. Ihr
               Interesse, verehrte gnädige Frau, für meine Arbeiten ist mir in jedem Falle sehr
                  erfreu{\pb}lich; ich darf wohl fernere
               Nachrichten von Ihnen erwarten.\pend
           
\pstart
           mit wiederholten Dank und der Versicherung meiner aufrichtigen
               Hochachtung{\\[\baselineskip]}Ihr sehr ergebener{\\[\baselineskip]}\spacefill\mbox{Arthur Schnitzler}\pend
           \leftskip=0em{}\selectlanguage{ngerman}\endnumbering\briefempfaengerindex{Zuckerkandl, Berta@\textsc{Zuckerkandl, Berta}!zzzSchnitzler, Arthur@\emph{von Arthur Schnitzler}!1911-04-181@{18. 4. 1911}|)be}\mylabel{L03981h}
\begin{anhang}
\end{anhang}\newcommand{\dateiname}{L03981}\newcommand{\titel}{Arthur Schnitzler an Berta Zuckerkandl, 18. 4. 1911}\newcommand{\editorInnen}{Herausgegeben von Jahnke, SelmaMüller, Martin Anton}%% latex-leseansicht-abspann.tex
%% Abspann für die Leseansicht.
%% Der Schalter \ifkorrekturansicht ist bereits durch den Vorspann gesetzt.

%% latex-abspann.tex
%% Gemeinsamer Abspann für Korrekturansicht und Leseansicht.
%% Setzt den Schalter \ifkorrekturansicht voraus (gesetzt in den
%% einbindenden Dateien latex-korrekturansicht-abspann.tex bzw.
%% latex-leseansicht-abspann.tex).
%% ---------------------------------------------------------------

\normalsize

% Das esempio-Environment wird nur in der Leseansicht benötigt
\ifkorrekturansicht\else
\newenvironment{esempio}[3]%
{
    \vspace{1.5ex}
    \rlap{\underline{#1}}
    \par
    \setlength{\parindent}{0cm}
    \nopagebreak
    \leftskip=#2cm
    \rightskip=#3cm
}
{
    \par
}
\fi

\doendnotes{C}
\bigskip
\vfill

\clearpage

\footnotesize

\ifkorrekturansicht
  \lohead{\textsc{register}}
\fi

% theindex-Environment neu definieren ohne reledmac
\makeatletter
\renewenvironment{theindex}{%
  \ifkorrekturansicht
    \section*{\indexname}%
  \else
    \subsubsection*{Index der erwähnten Entitäten}%
  \fi
  \setlength{\parindent}{0pt}%
  \setlength{\parskip}{0pt plus 0.3pt}%
  \let\item\@idxitem
}{%
  \ifkorrekturansicht\clearpage\fi
}
\makeatother

\IfFileExists{\jobname-pw.ind}{\input{\jobname-pw.ind}}{}

% Quellenangabe nur in der Leseansicht
\ifkorrekturansicht\else
% Fallback-Definitionen, falls die .tex-Datei \titel etc. nicht gesetzt hat
\providecommand{\titel}{}
\providecommand{\editorInnen}{}
\providecommand{\dateiname}{\jobname}

\vspace{3cm}

\vfill

\footnotesize
\textsc{Quelle}: \titel. Herausgegeben von {\editorInnen}. In: \emph{Arthur Schnitzler: Briefwechsel mit Autorinnen und Autoren}.
 Digitale Edition, https://schnitzler-briefe.acdh.oeaw.ac.at/{\dateiname}.html (Stand \today)
\fi

\end{document}


