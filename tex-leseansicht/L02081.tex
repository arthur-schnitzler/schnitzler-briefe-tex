%% latex-leseansicht-vorspann.tex
%% Vorspann für die Leseansicht.
%% Lädt die gemeinsame Datei latex-vorspann.tex mit nicht gesetztem Schalter.

\newif\ifkorrekturansicht
\korrekturansichtfalse

\input{../tex-inputs/latex-vorspann}


\section[Richard Beer-Hofmann an Arthur Schnitzler, 2. 8. 1912]{L02081 Richard Beer-Hofmann an Arthur Schnitzler, 2. 8. 1912}
\nopagebreak\mylabel{L02081v}
\rehead{ }\normalsize\beginnumbering\briefempfaengerindex{Schnitzler, Arthur@\textsc{Schnitzler, Arthur}!zzzBeer-Hofmann, Richard@\emph{von Richard Beer-Hofmann}!1912-08-021@{2. 8. 1912}|(be}
\toendnotes[C]{\smallbreak\pagebreak[2]}
\correspDesc{Versand  durch Richard Beer-Hofmann am 2. 8. 1912 in Sankt Moritz
\newline{}Erhalt  durch Arthur Schnitzler im Zeitraum [3. 8. 1912
                  – 7. 8. 1912?] in Brijuni}\toendnotes[C]{\smallbreak}
\Standort{CUL, Schnitzler, B 8.}
\physDesc{Bildpostkarte, 472 Zeichen
\newline{}Handschrift: schwarze Tinte, lateinische Kurrent
\newline{}Versand: Stempel: »\nobreak{}\oindex{St. Moritz@\textbf{St. Moritz}|pwk}St. Moritz-Dorf, 2. VIII. 12, 6\nobreak{}«.  
\newline{}Ordnung: mit Bleistift von unbekannter Hand nummeriert:
                                    »246« }\toendnotes[C]{\smallbreak}\pstart{}{\pb}S. H.\pend{}\pstart{}Herrn Arthur Schnitzler\pend{}\pstart{}Insel Brioni\oindex{Brijuni@\textbf{Brijuni}|pw}\pend{}\pstart{}i. d. Adria\oindex{Adriatisches Meer@\textbf{Adriatisches Meer}|pw}. \pend{}\pstart{}Oesterreich\oindex{Österreich@\textbf{Österreich}|pw}\pend{}{\bigskip}
\pstart
           \noindent{}\centering{}\textcolor{gray}{\textbf{St. Moritz\oindex{St. Moritz@\textbf{St. Moritz}|pw}.}}\pend
           \vspace{1em}
\pstart
           \noindent{}{\pb}Lieber Arthur! Es geht uns recht gut, wir sind fast (– \label{K_L02081-1v}\edtext{Dnieper\oindex{Dnepr@\textbf{Dnepr}, \emph{Fluss}|pw}los}{\lemma{\textnormal{\emph{Dnieperlos}}}\Cendnote{\textnormal{Die Anspielung ist unklar; eventuell steht sie in Zusammenhang mit dem Besuch Leo Van-Jungs\pwindex{Van-Jung, Leo 15.\,10.\,1866 Odessa – 2.\,7.\,1939 Riga@\textsc{Van-Jung, Leo} (15.\,10.\,1866 Odessa – 2.\,7.\,1939 Riga), \emph{Gesangspädagoge, Mathematiker}|pwk} und Isabella Vengerovas\pwindex{Vengerova, Isabella 1.\,3.\,1877 Minsk – 7.\,2.\,1956 New York City@\textsc{Vengerova, Isabella} (1.\,3.\,1877 Minsk – 7.\,2.\,1956 New York City), \emph{Musikpädagogin, Pianistin}|pwk}.}}}\label{K_L02081-1} –) zufrieden mit Luft,
               Wetter und Wohnung. Kaufmann\pwindex{Kaufmann, Arthur 4.\,4.\,1872 Iași – 25.\,7.\,1938 Wien@\textsc{Kaufmann, Arthur} (4.\,4.\,1872 Iași – 25.\,7.\,1938 Wien), \emph{Rechtswissenschaftler, Privatgelehrte, Privatier}|pw} ist seit
               vorgestern hier, und Leo\pwindex{Van-Jung, Leo 15.\,10.\,1866 Odessa – 2.\,7.\,1939 Riga@\textsc{Van-Jung, Leo} (15.\,10.\,1866 Odessa – 2.\,7.\,1939 Riga), \emph{Gesangspädagoge, Mathematiker}|pw} und Bella\pwindex{Vengerova, Isabella 1.\,3.\,1877 Minsk – 7.\,2.\,1956 New York City@\textsc{Vengerova, Isabella} (1.\,3.\,1877 Minsk – 7.\,2.\,1956 New York City), \emph{Musikpädagogin, Pianistin}|pw} sind in Celerina\oindex{Celerina@\textbf{Celerina}|pw} (30 Gehminuten entfernt) sesshaft. Am 11 gehen wir
               fort, und landen vermutlich in Ischl\oindex{Bad Ischl@\textbf{Bad Ischl}|pw}, – wohin die
                  Kinder\pwindex{Beer-Hofmann, Naëmah 20.\,12.\,1898 Wien – 10.\,11.\,1971 New York City@\textsc{Beer-Hofmann, Naëmah} (20.\,12.\,1898 Wien – 10.\,11.\,1971 New York City)|pwv}\pwindex{Beer-Hofmann, Gabriel 9.\,1.\,1901 Wien – 24.\,3.\,1971 St Albans@\textsc{Beer-Hofmann, Gabriel} (9.\,1.\,1901 Wien – 24.\,3.\,1971 St Albans), \emph{Schriftsteller, Filmagent}|pwv}\pwindex{Beer-Hofmann, Mirjam 4.\,9.\,1897 Wien – 24.\,12.\,1984 New York City@\textsc{Beer-Hofmann, Mirjam} (4.\,9.\,1897 Wien – 24.\,12.\,1984 New York City)|pwv}
               gerne möchten. Führt Ihr Rückweg nicht {\pb}vielleicht auch über Ischl\oindex{Bad Ischl@\textbf{Bad Ischl}|pw} – Lueg\oindex{Lueg@\textbf{Lueg}, \emph{Teil eines besiedelten Ortes}|pw} –
                  St Gilgen\oindex{St. Gilgen@\textbf{St. Gilgen}, \emph{Verwaltungsgebiet}|pw} – irgendeines? Herzliche Grüsse Ihnen
               Allen von uns.\pend
           \pstart \spacefill\mbox{Richard}\pend{}\selectlanguage{ngerman}\endnumbering\briefempfaengerindex{Schnitzler, Arthur@\textsc{Schnitzler, Arthur}!zzzBeer-Hofmann, Richard@\emph{von Richard Beer-Hofmann}!1912-08-021@{2. 8. 1912}|)be}\mylabel{L02081h}  \newcommand{\dateiname}{L02081}\newcommand{\titel}{Richard Beer-Hofmann an Arthur Schnitzler, 2. 8. 1912}\newcommand{\editorInnen}{Martin Anton Müller und Gerd-Hermann Susen}%% latex-leseansicht-abspann.tex
%% Abspann für die Leseansicht.
%% Der Schalter \ifkorrekturansicht ist bereits durch den Vorspann gesetzt.

%% latex-abspann.tex
%% Gemeinsamer Abspann für Korrekturansicht und Leseansicht.
%% Setzt den Schalter \ifkorrekturansicht voraus (gesetzt in den
%% einbindenden Dateien latex-korrekturansicht-abspann.tex bzw.
%% latex-leseansicht-abspann.tex).
%% ---------------------------------------------------------------

\normalsize

% Das esempio-Environment wird nur in der Leseansicht benötigt
\ifkorrekturansicht\else
\newenvironment{esempio}[3]%
{
    \vspace{1.5ex}
    \rlap{\underline{#1}}
    \par
    \setlength{\parindent}{0cm}
    \nopagebreak
    \leftskip=#2cm
    \rightskip=#3cm
}
{
    \par
}
\fi

\doendnotes{C}
\bigskip
\vfill

\clearpage

\footnotesize

\ifkorrekturansicht
  \lohead{\textsc{register}}
\fi

% theindex-Environment neu definieren ohne reledmac
\makeatletter
\renewenvironment{theindex}{%
  \ifkorrekturansicht
    \section*{\indexname}%
  \else
    \subsubsection*{Index der erwähnten Entitäten}%
  \fi
  \setlength{\parindent}{0pt}%
  \setlength{\parskip}{0pt plus 0.3pt}%
  \let\item\@idxitem
}{%
  \ifkorrekturansicht\clearpage\fi
}
\makeatother

\IfFileExists{\jobname-pw.ind}{\input{\jobname-pw.ind}}{}

% Quellenangabe nur in der Leseansicht
\ifkorrekturansicht\else
% Fallback-Definitionen, falls die .tex-Datei \titel etc. nicht gesetzt hat
\providecommand{\titel}{}
\providecommand{\editorInnen}{}
\providecommand{\dateiname}{\jobname}

\vspace{3cm}

\vfill

\footnotesize
\textsc{Quelle}: \titel. Herausgegeben von {\editorInnen}. In: \emph{Arthur Schnitzler: Briefwechsel mit Autorinnen und Autoren}.
 Digitale Edition, https://schnitzler-briefe.acdh.oeaw.ac.at/{\dateiname}.html (Stand \today)
\fi

\end{document}


