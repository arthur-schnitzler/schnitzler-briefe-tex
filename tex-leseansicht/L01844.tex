%% latex-korrekturansicht-vorspann.tex
%% Vorspann für die Korrekturansicht.
%% Lädt die gemeinsame Datei latex-vorspann.tex mit gesetztem Schalter.

\newif\ifkorrekturansicht
\korrekturansichttrue

\input{../tex-inputs/latex-vorspann}


\section[Hugo von Hofmannsthal an Arthur Schnitzler, {[}11. 6. 1909{]}]{L01844 Hugo von Hofmannsthal an Arthur Schnitzler, {[}11. 6. 1909{]}}
\nopagebreak\mylabel{L01844v}
\rehead{ }\normalsize\beginnumbering\briefempfaengerindex{Schnitzler, Arthur@\textsc{Schnitzler, Arthur}!zzzHofmannsthal, Hugo von@\emph{von Hugo von Hofmannsthal}!1909-06-112@{{[}11. 6. 1909{]}}|(be}
\toendnotes[C]{\smallbreak\pagebreak[2]}\Standort{CUL, Schnitzler, B 43.}
\physDesc{Brief, 1 Blatt, 2 Seiten, 262 Zeichen
\newline{}Handschrift: schwarze Tinte, deutsche Kurrent
\newline{}Schnitzler: mit Bleistift datiert: »11/6 09« und beschriftet: »Hofma{\geminationn}sthal« 
\newline{}Ordnung: 1) mit Bleistift von unbekannter Hand nummeriert: »\strikeout{30\textcolor{gray}{×}}«  2) mit Bleistift von unbekannter Hand nummeriert: »\strikeout{305}«}
\buchAbdrucke{\weitereDrucke{Hugo von Hofmannsthal, Arthur Schnitzler: \emph{Briefwechsel}. Frankfurt am Main: \emph{S. Fischer} 1964, S. 245.} }
\pstart
           \centering{}{\pb}\textcolor{gray}{\textbf{\textsc{Hotel}}}\pend
           
\pstart
           \centering{}\textcolor{gray}{\textbf{\textsc{Vier Jahreszeiten}\oindex{Hotel Vier Jahreszeiten@\textbf{Hotel Vier Jahreszeiten}, \emph{Hotel (K.HTL)}|pw}}}\pend
           
\pstart
           \centering{}\textcolor{gray}{\textbf{TELEGRAMM-ADRESSE: JAHRESZEITENTYP, MÜNCHEN\oindex{Muenchen@\textbf{München}, \emph{P.PPLA}|pw}.}}\pend
           
\pstart
           \centering{}\textcolor{gray}{\textbf{Lieber’s Code – International Hôtel-Code.}}\pend
           
\pstart
           \centering{}\textcolor{gray}{\textbf{Telefon 23073–23076.}}\pend
           
\pstart
           \raggedleft{}\textcolor{gray}{\textbf{\textsc{München}\oindex{Muenchen@\textbf{München}, \emph{P.PPLA}|pw},}}\pend
           
\pstart{}lieber\pend\vspace{0.5em}
\pstart
           wir treiben uns Automobilfahrend im ſüdlichen und weſtlichen Baiern\oindex{Bayern@\textbf{Bayern}, \emph{A.ADM1}|pw} herum (Lech\oindex{Lech@\textbf{Lech}, \emph{Besiedelter Ort (A.BSO)}|pw}, Augsburg\oindex{Augsburg@\textbf{Augsburg}, \emph{P.PPLA2}|pw}{ }\textsc{etc}.) ſchreibt doch ein Wort wann Ihr etwa nach München\oindex{Muenchen@\textbf{München}, \emph{P.PPLA}|pw} ko{\geminationm}t, an
                  {\pb}dieſe Adreſſe: \textsc{Villa Cantacuzene Starnberg\oindex{Villa Cantacuzene@\textbf{Villa Cantacuzène}, \emph{Gebäude (K.GBD)}|pw}}, es wird uns nachgeſchickt.\pend
           
\pstart
           Alles Liebe an Olga\pwindex{Schnitzler, Olga 17.01.1882 – 13.01.1970@\textsc{Schnitzler, Olga} (17.01.1882 – 13.01.1970), \emph{Schauspieler/Schauspielerin, Sänger/Sängerin}|pw}.\pend
           
\pstart
           Von Herzen Ihr{\\[\baselineskip]}\spacefill\mbox{Hugo.}\pend
           \leftskip=0em{}\selectlanguage{ngerman}\endnumbering\briefempfaengerindex{Schnitzler, Arthur@\textsc{Schnitzler, Arthur}!zzzHofmannsthal, Hugo von@\emph{von Hugo von Hofmannsthal}!1909-06-112@{{[}11. 6. 1909{]}}|)be}\mylabel{L01844h}  \normalsize

\doendnotes{C}
\bigskip
\vfill

\clearpage

\footnotesize

\lohead{\textsc{register}}

% Definiere theindex-Environment komplett neu ohne reledmac
\makeatletter
\renewenvironment{theindex}{%
  \section*{\indexname}%
  \setlength{\parindent}{0pt}%
  \setlength{\parskip}{0pt plus 0.3pt}%
  \let\item\@idxitem
}{%
  \clearpage
}
\makeatother

\IfFileExists{\jobname-pw.ind}{\input{\jobname-pw.ind}}{}

\end{document}

      