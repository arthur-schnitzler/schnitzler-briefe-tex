%% latex-leseansicht-vorspann.tex
%% Vorspann für die Leseansicht.
%% Lädt die gemeinsame Datei latex-vorspann.tex mit nicht gesetztem Schalter.

\newif\ifkorrekturansicht
\korrekturansichtfalse

\input{../tex-inputs/latex-vorspann}


\section[Hugo von Hofmannsthal an Arthur Schnitzler, {[}11. 6. 1909{]}]{L01844 Hugo von Hofmannsthal an Arthur Schnitzler, {[}11. 6. 1909{]}}
\nopagebreak\mylabel{L01844v}
\rehead{ }\normalsize\beginnumbering\briefempfaengerindex{Schnitzler, Arthur@\textsc{Schnitzler, Arthur}!zzzHofmannsthal, Hugo von@\emph{von Hugo von Hofmannsthal}!1909-06-112@{{[}11. 6. 1909{]}}|(be}
\toendnotes[C]{\smallbreak\pagebreak[2]}
\correspDesc{Versand  durch Hugo von Hofmannsthal am [11. 6. 1909] in München
\newline{}Erhalt  durch Arthur Schnitzler im Zeitraum [12. 6. 1909
                  – 16. 6. 1909?] in Wien}\toendnotes[C]{\smallbreak}
\Standort{CUL, Schnitzler, B 43.}
\physDesc{Brief, 1 Blatt, 2 Seiten, 262 Zeichen
\newline{}Handschrift: schwarze Tinte, deutsche Kurrent
\newline{}Schnitzler: mit Bleistift datiert: »11/6 09« und beschriftet: »Hofma{\geminationn}sthal« 
\newline{}Ordnung: 1) mit Bleistift von unbekannter Hand nummeriert: »\strikeout{30\textcolor{gray}{×}}«  2) mit Bleistift von unbekannter Hand nummeriert: »\strikeout{305}«}
\buchAbdrucke{\weitereDrucke{Hugo von Hofmannsthal, Arthur Schnitzler: \emph{Briefwechsel}. Herausgegeben von Therese Nickl und Heinrich Schnitzler. Frankfurt am Main: \emph{S. Fischer} 1964, S. 245.} }
\pstart
           \centering{}{\pb}\textcolor{gray}{\textbf{\textsc{Hotel}}}\pend
           
\pstart
           \centering{}\textcolor{gray}{\textbf{\textsc{Vier Jahreszeiten}\oindex{Hotel Vier Jahreszeiten@\textbf{Hotel Vier Jahreszeiten}, \emph{Hotel}|pw}}}\pend
           
\pstart
           \centering{}\textcolor{gray}{\textbf{TELEGRAMM-ADRESSE: JAHRESZEITENTYP, MÜNCHEN\oindex{München@\textbf{München}|pw}.}}\pend
           
\pstart
           \centering{}\textcolor{gray}{\textbf{Lieber’s Code – International Hôtel-Code.}}\pend
           
\pstart
           \centering{}\textcolor{gray}{\textbf{Telefon 23073–23076.}}\pend
           
\pstart
           \raggedleft{}\textcolor{gray}{\textbf{\textsc{München}\oindex{München@\textbf{München}|pw},}}\pend
           
\pstart{}lieber\pend\vspace{0.5em}
\pstart
           wir treiben uns Automobilfahrend im{ }ſüdlichen und weſtlichen Baiern\oindex{Bayern@\textbf{Bayern}, \emph{Land}|pw} herum (Lech\oindex{Lech@\textbf{Lech}|pw}, Augsburg\oindex{Augsburg@\textbf{Augsburg}, \emph{Hauptstadt}|pw}{ }\textsc{etc}.){ }ſchreibt doch ein Wort wann Ihr etwa nach München\oindex{München@\textbf{München}|pw} ko{\geminationm}t, an
                  {\pb}dieſe Adreſſe: \textsc{Villa Cantacuzene Starnberg\oindex{Villa Cantacuzène@\textbf{Villa Cantacuzène}, \emph{Gebäude}|pw}}, es wird uns nachgeſchickt.\pend
           
\pstart
           Alles Liebe an Olga\pwindex{Schnitzler, Olga 17.\,1.\,1882 Wien – 13.\,1.\,1970 Lugano@\textsc{Schnitzler, Olga} (17.\,1.\,1882 Wien – 13.\,1.\,1970 Lugano), \emph{Schauspielerin, Sängerin}|pw}.\pend
           
\pstart
           Von Herzen Ihr{\\[\baselineskip]}\spacefill\mbox{Hugo.}\pend
           \leftskip=0em{}\selectlanguage{ngerman}\endnumbering\briefempfaengerindex{Schnitzler, Arthur@\textsc{Schnitzler, Arthur}!zzzHofmannsthal, Hugo von@\emph{von Hugo von Hofmannsthal}!1909-06-112@{{[}11. 6. 1909{]}}|)be}\mylabel{L01844h}  \newcommand{\dateiname}{L01844}\newcommand{\titel}{Hugo von Hofmannsthal an Arthur Schnitzler, [11. 6. 1909]}\newcommand{\editorInnen}{Martin Anton Müller und Gerd-Hermann Susen}%% latex-leseansicht-abspann.tex
%% Abspann für die Leseansicht.
%% Der Schalter \ifkorrekturansicht ist bereits durch den Vorspann gesetzt.

%% latex-abspann.tex
%% Gemeinsamer Abspann für Korrekturansicht und Leseansicht.
%% Setzt den Schalter \ifkorrekturansicht voraus (gesetzt in den
%% einbindenden Dateien latex-korrekturansicht-abspann.tex bzw.
%% latex-leseansicht-abspann.tex).
%% ---------------------------------------------------------------

\normalsize

% Das esempio-Environment wird nur in der Leseansicht benötigt
\ifkorrekturansicht\else
\newenvironment{esempio}[3]%
{
    \vspace{1.5ex}
    \rlap{\underline{#1}}
    \par
    \setlength{\parindent}{0cm}
    \nopagebreak
    \leftskip=#2cm
    \rightskip=#3cm
}
{
    \par
}
\fi

\doendnotes{C}
\bigskip
\vfill

\clearpage

\footnotesize

\ifkorrekturansicht
  \lohead{\textsc{register}}
\fi

% theindex-Environment neu definieren ohne reledmac
\makeatletter
\renewenvironment{theindex}{%
  \ifkorrekturansicht
    \section*{\indexname}%
  \else
    \subsubsection*{Index der erwähnten Entitäten}%
  \fi
  \setlength{\parindent}{0pt}%
  \setlength{\parskip}{0pt plus 0.3pt}%
  \let\item\@idxitem
}{%
  \ifkorrekturansicht\clearpage\fi
}
\makeatother

\IfFileExists{\jobname-pw.ind}{\input{\jobname-pw.ind}}{}

% Quellenangabe nur in der Leseansicht
\ifkorrekturansicht\else
% Fallback-Definitionen, falls die .tex-Datei \titel etc. nicht gesetzt hat
\providecommand{\titel}{}
\providecommand{\editorInnen}{}
\providecommand{\dateiname}{\jobname}

\vspace{3cm}

\vfill

\footnotesize
\textsc{Quelle}: \titel. Herausgegeben von {\editorInnen}. In: \emph{Arthur Schnitzler: Briefwechsel mit Autorinnen und Autoren}.
 Digitale Edition, https://schnitzler-briefe.acdh.oeaw.ac.at/{\dateiname}.html (Stand \today)
\fi

\end{document}


