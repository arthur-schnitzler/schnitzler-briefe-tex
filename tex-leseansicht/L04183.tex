%% latex-leseansicht-vorspann.tex
%% Vorspann für die Leseansicht.
%% Lädt die gemeinsame Datei latex-vorspann.tex mit nicht gesetztem Schalter.

\newif\ifkorrekturansicht
\korrekturansichtfalse

\input{../tex-inputs/latex-vorspann}


\section[Arthur Schnitzler an Gustav Schwarzkopf, {{[}}9. 8. 1900?{{]}}]{L04183 Arthur Schnitzler an Gustav Schwarzkopf, {[}9. 8. 1900?{]}}
\nopagebreak\mylabel{L04183v}
\rehead{ }\normalsize\beginnumbering\briefempfaengerindex{Schwarzkopf, Gustav@\textsc{Schwarzkopf, Gustav}!zzzSchnitzler, Arthur@\emph{von Arthur Schnitzler}!1900-08-091@{{[}9. 8. 1900?{]}}|(be}
\toendnotes[C]{\smallbreak\pagebreak[2]}
\correspDesc{Versand  durch Arthur Schnitzler am [9. 8. 1900?] in Bad Ischl
\newline{}Erhalt  durch Gustav Schwarzkopf im Zeitraum [10. 8. 1900 – 14. 8. 1900?] in Wien}\toendnotes[C]{\smallbreak}
\Standort{CUL, Schnitzler, B 96.}
\physDesc{Brief, 2 Blätter, 5 Seiten, 1433 Zeichen
\newline{}Handschrift: Bleistift, deutsche Kurrent}\toendnotes[C]{\smallbreak}
\pstart
           \noindent{}{\pb}Mein lieber Guſtav, aus dem Tun wird es allmälig Ernſt, u es iſt
               wirklich ſchad, dſs Sie nicht einmal auf ein paar Tage da oder dort dabei ſein
               können. Ich verlaſſe Iſchl\oindex{Bad Ischl@\textbf{Bad Ischl}|pw} morgen und denke (auf
               einem \label{K_L04183-1v}\edtext{Umweg\oindex{Bischofshofen@\textbf{Bischofshofen}, \emph{Hauptstadt}|pwv}}{\lemma{\textnormal{\emph{Umweg}}}\Cendnote{\textnormal{Vgl. A. S.: \emph{Tagebuch}, 10. 8. 1900.}}}\label{K_L04183-1}) \label{K_L04183-2v}\edtext{So{\geminationn}tag in Salzburg\oindex{Salzburg@\textbf{Salzburg}, \emph{Verwaltungsgebiet}|pw}}{\lemma{\textnormal{\emph{Sonntag in Salzburg}}}\Cendnote{\textnormal{Der Plan
                  wurde verwirklicht, vgl. A. S.: \emph{Wiener Schnitzler}, 12. 8. 1900. }}}\label{K_L04183-2} zu ſein; Richard\pwindex{Beer-Hofmann, Richard 11.\,7.\,1866 Wien – 26.\,9.\,1945 New York City@\textsc{Beer-Hofmann, Richard} (11.\,7.\,1866 Wien – 26.\,9.\,1945 New York City), \emph{Schriftsteller}|pw} (auch Hugo\pwindex{Hofmannsthal, Hugo von 1.\,2.\,1874 Wien – 15.\,7.\,1929 Rodaun@\textsc{Hofmannsthal, Hugo von} (1.\,2.\,1874 Wien – 15.\,7.\,1929 Rodaun), \emph{Schriftsteller}|pw} und{ }ſo weiter), event. Leo Vanjung\pwindex{Van-Jung, Leo 15.\,10.\,1866 Odessa – 2.\,7.\,1939 Riga@\textsc{Van-Jung, Leo} (15.\,10.\,1866 Odessa – 2.\,7.\,1939 Riga), \emph{Gesangspädagoge, Mathematiker}|pw}{ }ſind dort. \label{K_L04183-3v}\edtext{Montag 13. verlaſſen wir Nachmittags {\pb}Salzburg\oindex{Salzburg@\textbf{Salzburg}, \emph{Verwaltungsgebiet}|pw}}{\lemma{\textnormal{\emph{Montag … Salzburg}}}\Cendnote{\textnormal{Vgl. A. S.: \emph{Wiener Schnitzler}, 13. 8. 1900.}}}\label{K_L04183-3}; in \label{K_L04183-4v}\edtext{Innsbruck\oindex{Innsbruck@\textbf{Innsbruck}, \emph{Verwaltungsgebiet}|pw} am 16.}{\lemma{\textnormal{\emph{Innsbruck am 16.}}}\Cendnote{\textnormal{Vgl. A. S.: \emph{Tagebuch}, 16. 8. 1900.}}}\label{K_L04183-4} stoßen \textsc{Kerr\pwindex{Kerr, Alfred 25.\,12.\,1867 Breslau – 12.\,10.\,1948 Hamburg@\textsc{Kerr, Alfred} (25.\,12.\,1867 Breslau – 12.\,10.\,1948 Hamburg), \emph{Schriftsteller, Kritiker}|pw}}{ }{\kaufmannsund}{ }\textsc{Goldm.\pwindex{Goldmann, Paul 31.\,1.\,1865 Breslau – 25.\,9.\,1935 Wien@\textsc{Goldmann, Paul} (31.\,1.\,1865 Breslau – 25.\,9.\,1935 Wien), \emph{Schriftsteller, Journalist}|pw}} uns. –\pend
           
\pstart
           Ich habe mich hier diesmal gar nicht wohl gefühlt, \textcolor{gray}{verſagen} meist
               recht tief verſtimmt, aus pſychiſchen u pſych. Gründen. Gearbeitet hier nahezu
               nichts, denken Sie, dſs Frl. \textsc{Marie E.\pwindex{Elsinger, Marie *~28.\,2.\,1874 St. Pölten@\textsc{Elsinger, Marie} (*~28.\,2.\,1874 St. Pölten), \emph{Schauspielerin}|pw}} hier 3 Tage war – zugleich mit mir – was ich aus {\pb}einem Brief erfuhr, den sie von Iſchl\oindex{Bad Ischl@\textbf{Bad Ischl}|pw} nach Wien\oindex{Wien@\textbf{Wien}, \emph{Verwaltungsgebiet}|pw}
               gerichtet und der mir hieher nachgeſchickt wurde, und daſs ich ſie nicht geſehen
               habe. Heute dürfte ſie in Wien\oindex{Wien@\textbf{Wien}, \emph{Verwaltungsgebiet}|pw} ſein. –\pend
           
\pstart
           Die Penſion\oindex{Hotel und Pension Rudolfshöhe (Leopold Petter)@\textbf{Hotel und Pension Rudolfshöhe (Leopold Petter)}, \emph{Hotel}|pwv} ist mir heuer recht
               zuwider; es überquillt von Familie, einzeln{ }ſind beinah alle nett, aber die
               Maſſe ſtört. Geſtern {\pb}erſchienen \textsc{Wasserma{\geminationn}\pwindex{Wassermann, Jakob 10.\,3.\,1873 Fürth – 1.\,1.\,1934 Altaussee@\textsc{Wassermann, Jakob} (10.\,3.\,1873 Fürth – 1.\,1.\,1934 Altaussee), \emph{Schriftsteller}|pw}} u \textsc{Wolff\pwindex{Wolff, Ludwig 7.\,3.\,1876 Bielsko-Biała – nach 1958 Vereinigte Staaten von Amerika [USA]@\textsc{Wolff, Ludwig} (7.\,3.\,1876 Bielsko-Biała – nach 1958 Vereinigte Staaten von Amerika [USA]), \emph{Schriftsteller, Dramaturg}|pw}}, die nach \textsc{\introOben{}Alt-\introOben{}Aussee\oindex{Bad Aussee@\textbf{Bad Aussee}, \emph{Hauptstadt}|pw}\oindex{Altaussee@\textbf{Altaussee}, \emph{Verwaltungsgebiet}|pw}} fahren. Ich war \label{K_L04183-5v}\edtext{geſtern}{\lemma{\textnormal{\emph{gestern}}}\Cendnote{\textnormal{Vgl. A. S.: \emph{Tagebuch}, 8. 8. 1900.}}}\label{K_L04183-5} dort und habe auch die \textsc{Speyerinnen\pwindex{Speyer, Nanette 5.\,1.\,1846 Iserlohn – 15.\,1.\,1925 Wien@\textsc{Speyer, Nanette} (5.\,1.\,1846 Iserlohn – 15.\,1.\,1925 Wien)|pw}\pwindex{Wassermann, Julie 5.\,12.\,1876 Wien – April 1963 Zürich@\textsc{Wassermann, Julie} (5.\,12.\,1876 Wien – April 1963 Zürich), \emph{Schriftstellerin}|pw}\pwindex{Schmidl, Paula 13.\,10.\,1874 Wien – 24.\,9.\,1966 Jerusalem@\textsc{Schmidl, Paula} (13.\,10.\,1874 Wien – 24.\,9.\,1966 Jerusalem)|pw}\pwindex{Ulmann, Agnes 23.\,12.\,1875 Wien – 1.\,4.\,1942 New York City@\textsc{Ulmann, Agnes} (23.\,12.\,1875 Wien – 1.\,4.\,1942 New York City), \emph{Malerin, Bildhauerin}|pw}\pwindex{Sgal, Emilie 7.\,5.\,1871 Wien – 3.\,12.\,1938 Den Haag@\textsc{Sgal, Emilie} (7.\,5.\,1871 Wien – 3.\,12.\,1938 Den Haag)|pw}\pwindex{Michaelis, Dora 23.\,5.\,1881 Wien – 22.\,1.\,1946 New York City@\textsc{Michaelis, Dora} (23.\,5.\,1881 Wien – 22.\,1.\,1946 New York City)|pw}\pwindex{Knepler, Sophie 13.\,5.\,1872 – 30.\,10.\,1908@\textsc{Knepler, Sophie} (13.\,5.\,1872 – 30.\,10.\,1908)|pw}} beſucht. Da ich mit Rad dort war, regnete es von ½ 2 an. –
               Haben Sie die \textsc{Renate\pwindex{Wassermann, Jakob 10.\,3.\,1873 Fürth – 1.\,1.\,1934 Altaussee@\textsc{Wassermann, Jakob} (10.\,3.\,1873 Fürth – 1.\,1.\,1934 Altaussee), \emph{Schriftsteller}!Geschichte der jungen Renate Fuchs@\strich\emph{Die Geschichte der jungen Renate Fuchs}|pw}} geleſen? Ich werde Sie von den Reiſe aus häufig grüßen. Wir wollen von \textsc{Schruns\oindex{Schruns@\textbf{Schruns}, \emph{Verwaltungsgebiet}|pw}} nach Pontreſina\oindex{Pontresina@\textbf{Pontresina}|pw}, dan Bormio\oindex{Bormio@\textbf{Bormio}, \emph{Hauptstadt}|pw}, Stilfſerjoch\oindex{Stilfser Joch@\textbf{Stilfser Joch}, \emph{Berg}|pw},
                  Meran\oindex{Meran@\textbf{Meran}, \emph{Hauptstadt}|pw}. Ehrlich: ich fürchte, wir{ }ſind zu
               viel zum wandern. Leben Sie wohl. Meine Mama\pwindex{Schnitzler, Louise 8.\,7.\,1840 Kőszeg – 9.\,9.\,1911 Wien@\textsc{Schnitzler, Louise} (8.\,7.\,1840 Kőszeg – 9.\,9.\,1911 Wien)|pwv} dankt {\pb}herzlich für Ihre Grüße. Es geht ihr recht gut. Sie reiſt wahrſcheinlich Giſela\pwindex{Hajek, Gisela 20.\,12.\,1867 Wien – 3.\,2.\,1953 Cambridge@\textsc{Hajek, Gisela} (20.\,12.\,1867 Wien – 3.\,2.\,1953 Cambridge)|pw} in die Schweiz\oindex{Schweiz@\textbf{Schweiz}|pw} entgegen.Richard\pwindex{Beer-Hofmann, Richard 11.\,7.\,1866 Wien – 26.\,9.\,1945 New York City@\textsc{Beer-Hofmann, Richard} (11.\,7.\,1866 Wien – 26.\,9.\,1945 New York City), \emph{Schriftsteller}|pw} hat den 1.
               Akt{ }ſeines Stücks\pwindex{Beer-Hofmann, Richard 11.\,7.\,1866 Wien – 26.\,9.\,1945 New York City@\textsc{Beer-Hofmann, Richard} (11.\,7.\,1866 Wien – 26.\,9.\,1945 New York City), \emph{Schriftsteller}!Graf von Charolais. Ein Trauerspiel@\strich\emph{Der Graf von Charolais. Ein Trauerspiel}|pwv}
               vollendet.\pend
           \pstart herzlichſt Ihr\spacefill\mbox{ArthurS}\pend{}\selectlanguage{ngerman}\endnumbering\briefempfaengerindex{Schwarzkopf, Gustav@\textsc{Schwarzkopf, Gustav}!zzzSchnitzler, Arthur@\emph{von Arthur Schnitzler}!1900-08-091@{{[}9. 8. 1900?{]}}|)be}\mylabel{L04183h}
\begin{anhang}
\end{anhang}\newcommand{\dateiname}{L04183}\newcommand{\titel}{Arthur Schnitzler an Gustav Schwarzkopf, [9. 8. 1900?]}\newcommand{\editorInnen}{Herausgegeben von Jahnke, SelmaMüller, Martin Anton}%% latex-leseansicht-abspann.tex
%% Abspann für die Leseansicht.
%% Der Schalter \ifkorrekturansicht ist bereits durch den Vorspann gesetzt.

%% latex-abspann.tex
%% Gemeinsamer Abspann für Korrekturansicht und Leseansicht.
%% Setzt den Schalter \ifkorrekturansicht voraus (gesetzt in den
%% einbindenden Dateien latex-korrekturansicht-abspann.tex bzw.
%% latex-leseansicht-abspann.tex).
%% ---------------------------------------------------------------

\normalsize

% Das esempio-Environment wird nur in der Leseansicht benötigt
\ifkorrekturansicht\else
\newenvironment{esempio}[3]%
{
    \vspace{1.5ex}
    \rlap{\underline{#1}}
    \par
    \setlength{\parindent}{0cm}
    \nopagebreak
    \leftskip=#2cm
    \rightskip=#3cm
}
{
    \par
}
\fi

\doendnotes{C}
\bigskip
\vfill

\clearpage

\footnotesize

\ifkorrekturansicht
  \lohead{\textsc{register}}
\fi

% theindex-Environment neu definieren ohne reledmac
\makeatletter
\renewenvironment{theindex}{%
  \ifkorrekturansicht
    \section*{\indexname}%
  \else
    \subsubsection*{Index der erwähnten Entitäten}%
  \fi
  \setlength{\parindent}{0pt}%
  \setlength{\parskip}{0pt plus 0.3pt}%
  \let\item\@idxitem
}{%
  \ifkorrekturansicht\clearpage\fi
}
\makeatother

\IfFileExists{\jobname-pw.ind}{\input{\jobname-pw.ind}}{}

% Quellenangabe nur in der Leseansicht
\ifkorrekturansicht\else
% Fallback-Definitionen, falls die .tex-Datei \titel etc. nicht gesetzt hat
\providecommand{\titel}{}
\providecommand{\editorInnen}{}
\providecommand{\dateiname}{\jobname}

\vspace{3cm}

\vfill

\footnotesize
\textsc{Quelle}: \titel. Herausgegeben von {\editorInnen}. In: \emph{Arthur Schnitzler: Briefwechsel mit Autorinnen und Autoren}.
 Digitale Edition, https://schnitzler-briefe.acdh.oeaw.ac.at/{\dateiname}.html (Stand \today)
\fi

\end{document}


