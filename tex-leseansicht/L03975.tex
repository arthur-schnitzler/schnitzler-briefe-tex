%% latex-leseansicht-vorspann.tex
%% Vorspann für die Leseansicht.
%% Lädt die gemeinsame Datei latex-vorspann.tex mit nicht gesetztem Schalter.

\newif\ifkorrekturansicht
\korrekturansichtfalse

\input{../tex-inputs/latex-vorspann}


\section[Arthur Schnitzler an Berta Zuckerkandl, 4. 3. 1929]{L03975 Arthur Schnitzler an Berta Zuckerkandl, 4. 3. 1929}
\nopagebreak\mylabel{L03975v}
\rehead{ }\normalsize\beginnumbering\briefempfaengerindex{Zuckerkandl, Berta@\textsc{Zuckerkandl, Berta}!zzzSchnitzler, Arthur@\emph{von Arthur Schnitzler}!1929-03-041@{4. 3. 1929}|(be}
\toendnotes[C]{\smallbreak\pagebreak[2]}
\correspDesc{Versand  durch Arthur Schnitzler am 4. 3. 1929 in Wien
\newline{}Erhalt  durch Berta Zuckerkandl im Zeitraum [4. 3. 1929
                  – 7. 3. 1929?] in Wien}\toendnotes[C]{\smallbreak}
\Standort{DLA, HS.1985.1.2282.}
\physDesc{Brief, Durchschlag, 2 Blätter, 3 Seiten, 4222 Zeichen
\newline{}Schreibmaschine
\newline{}Handschrift Arthur Schnitzler: roter Buntstift, lateinische Kurrent (\noindent{}beschriftet: »\uline{Zuckerkandl}« und »Frkr«, zwei
                                 Unterstreichungen)
\newline{}Handschrift Frieda Pollak: roter Buntstift, lateinische Kurrent (\noindent{}beschriftet: »K {[}opie ?{]}«)}
\buchAbdrucke{\weitereDrucke{Arthur Schnitzler: \emph{Briefe 1913–1931}. Herausgegeben von Peter Michael Braunwarth, Richard Miklin, Susanne Pertlik und Heinrich Schnitzler. Frankfurt am Main: \emph{S. Fischer} 1984, S. 594–596.} }\toendnotes[C]{\smallbreak}
\pstart
           \raggedleft{}{\pb}4. 3. 1929.\pend
           
\pstart{}Liebe und verehrte Freundin.\pend\vspace{0.5em}
\pstart
           Sie teilen mir mit, dass Antoine\pwindex{Antoine, André 31.\,1.\,1858 Limoges – 23.\,10.\,1943 Le Pouliguen@\textsc{Antoine, André} (31.\,1.\,1858 Limoges – 23.\,10.\,1943 Le Pouliguen), \emph{Theaterleiter, Schauspieler}|pw}{ }\label{K_L03975-1v}\edtext{die Absicht hegt}{\lemma{\textnormal{\emph{die Absicht hegt}}}\Cendnote{\textnormal{Bereits am 10. 12. 1928 wurde diese Frage laut \emph{Tagebuch}\pwindex{Schnitzler, Arthur 15. 5. 1862 Wien – 21. 10. 1931 ebd.@\textsc{Schnitzler, Arthur} (15. 5. 1862 Wien – 21. 10. 1931 ebd.), \emph{Schriftsteller, Mediziner}!Tagebuch@\strich\emph{Tagebuch}|pwk} mit Zuckerkandl\pwindex{Zuckerkandl, Berta 13.\,4.\,1864 Wien – 16.\,10.\,1945 Paris@\textsc{Zuckerkandl, Berta} (13.\,4.\,1864 Wien – 16.\,10.\,1945 Paris), \emph{Schriftstellerin, Journalistin, Übersetzerin}|pwk} besprochen.}}}\label{K_L03975-1}, meine Szenenreihe »Reigen\pwindex{Schnitzler, Arthur 15. 5. 1862 Wien – 21. 10. 1931 ebd.@\textsc{Schnitzler, Arthur} (15. 5. 1862 Wien – 21. 10. 1931 ebd.), \emph{Schriftsteller, Mediziner}!Reigen. Zehn Dialoge@\strich\emph{Reigen. Zehn Dialoge}|pw}« aufzuführen und die Frage der Inszenierung
               einigermassen schwierig findet. Mir persönlich scheint diese Frage keineswegs schwer
               lösbar. Ganz klar ist, dass man über die Gedankenstriche nicht anders hinwegkommen
               kann, als durch eine möglichst kurze Pause, ob diese nun durch Vorhang, Schleier oder
               Verdunkelung symbolisiert und zugleich verwirklicht wird. Verdunkelung, wenn
               vollkommen möglich, scheint mir immer noch das Beste. Man vergesse doch nicht, wie
               oft schon bei szenischen Aufführungen und selbst bei solchen von klassischen Stücken
               der psychologische oder besser psychophysiologische Moment, der dem Zuschauer
               natürlicherweise vorenthalten werden muss, durch Fallen des Vorhangs oder
               Verdunkelung gekennzeichnet wurde, ohne dass es den geringsten Anstoss erregte. Der
                  »Reigen\pwindex{Schnitzler, Arthur 15. 5. 1862 Wien – 21. 10. 1931 ebd.@\textsc{Schnitzler, Arthur} (15. 5. 1862 Wien – 21. 10. 1931 ebd.), \emph{Schriftsteller, Mediziner}!Reigen. Zehn Dialoge@\strich\emph{Reigen. Zehn Dialoge}|pw}« war in dieser Hinsicht keineswegs
               etwas Neues und die zehnfache Wiederkehr ändert nichts an dem Wesentlichen der Sache.
               Wir wissen ja Alle, dass der Widerstand gegen den »Reigen\pwindex{Schnitzler, Arthur 15. 5. 1862 Wien – 21. 10. 1931 ebd.@\textsc{Schnitzler, Arthur} (15. 5. 1862 Wien – 21. 10. 1931 ebd.), \emph{Schriftsteller, Mediziner}!Reigen. Zehn Dialoge@\strich\emph{Reigen. Zehn Dialoge}|pw}«, insbesondere der gegen die Aufführungen des »Reigen\pwindex{Schnitzler, Arthur 15. 5. 1862 Wien – 21. 10. 1931 ebd.@\textsc{Schnitzler, Arthur} (15. 5. 1862 Wien – 21. 10. 1931 ebd.), \emph{Schriftsteller, Mediziner}!Reigen. Zehn Dialoge@\strich\emph{Reigen. Zehn Dialoge}|pw}«, keineswegs aus reinen oder reinlichen Motiven
               erfolgt ist und dass die alberne, brutale und verlogene Hetze in hohem Maasse durch
               parteipolitische Motive bedingt war, die anderswo als in Deutschland\oindex{Deutschland@\textbf{Deutschland}|pw} und Oesterreich\oindex{Österreich@\textbf{Österreich}|pw}{ }\label{K_L03975-2v}\edtext{a priori}{\lemma{\textnormal{\emph{a priori}}}\Cendnote{\textnormal{von vorneherein}}}\label{K_L03975-2} wegfallen müssten.
               Wer sich über die Gründe, insbesondere aber über die Technik dieser »Reigen\pwindex{Schnitzler, Arthur 15. 5. 1862 Wien – 21. 10. 1931 ebd.@\textsc{Schnitzler, Arthur} (15. 5. 1862 Wien – 21. 10. 1931 ebd.), \emph{Schriftsteller, Mediziner}!Reigen. Zehn Dialoge@\strich\emph{Reigen. Zehn Dialoge}|pw}«-Hetze gründlich informieren, dabei ein Stück
               Kulturgeschichte kennen lernen und sich nebstbei vorzüglich amüsieren will, dem kann
               ich nur raten das Buch »Der Kampf um den Reigen\pwindex{Kampf um den Reigen. Vollständiger Bericht über die sechstägige Verhandlung gegen Direktion und Darsteller des Kleinen Schauspielhauses Berlin@\emph{Der Kampf um den Reigen. Vollständiger Bericht über die sechstägige Verhandlung gegen Direktion und Darsteller des Kleinen Schauspielhauses Berlin}|pw}«
               zu lesen, der das stenographische Protokoll eines im Jahre 1921
               stattgefundenen sechstätigen Prozesses gegen Direktion\pwindex{Eysoldt, Gertrud 30.\,11.\,1870 Pirna – 5.\,1.\,1955 Ohlstadt@\textsc{Eysoldt, Gertrud} (30.\,11.\,1870 Pirna – 5.\,1.\,1955 Ohlstadt), \emph{Theaterleiterin, Schauspielerin}|pwv}\pwindex{Sladek, Maximilian 30.\,5.\,1875 Ozimek – 9.\,11.\,1925 Berlin@\textsc{Sladek, Maximilian} (30.\,5.\,1875 Ozimek – 9.\,11.\,1925 Berlin), \emph{Theaterleiter, Schauspieler}|pwv} und Darsteller\pwindex{Reusch, Hubert 24.\,3.\,1862 Düsseldorf – 6.\,11.\,1925 Berlin@\textsc{Reusch, Hubert} (24.\,3.\,1862 Düsseldorf – 6.\,11.\,1925 Berlin), \emph{Theaterleiter, Regisseur, Schauspieler}|pwv}\pwindex{Forster-Larrinaga, Robert 1880 – 2.\,7.\,1932 Berlin@\textsc{Forster-Larrinaga, Robert} (1880 – 2.\,7.\,1932 Berlin), \emph{Schriftsteller, Regisseur, Schauspieler}|pwv}\pwindex{Schwanneke, Viktor 8.\,2.\,1880 Hedwigsburg – 7.\,6.\,1931 Berlin@\textsc{Schwanneke, Viktor} (8.\,2.\,1880 Hedwigsburg – 7.\,6.\,1931 Berlin), \emph{Schauspieler}|pwv}\pwindex{Kampers, Fritz 14.\,7.\,1891 München – 1.\,9.\,1950@\textsc{Kampers, Fritz} (14.\,7.\,1891 München – 1.\,9.\,1950), \emph{Schauspieler}|pwv}\pwindex{Skidelsky, Vera 4.\,7.\,1897 Tiflis – 1976-05 Chicago@\textsc{Skidelsky, Vera} (4.\,7.\,1897 Tiflis – 1976-05 Chicago), \emph{Schauspielerin}|pwv}\pwindex{Madeleine, Magda 30.\,10.\,1892 München – 11.\,2.\,1973 Hamburg@\textsc{Madeleine, Magda} (30.\,10.\,1892 München – 11.\,2.\,1973 Hamburg), \emph{Schauspielerin}|pwv}\pwindex{Copony, Marianne @\textsc{Copony, Marianne}, \emph{Schauspielerin}|pwv}\pwindex{Bach, Elvira @\textsc{Bach, Elvira}, \emph{Schauspielerin}|pwv}\pwindex{Tillo, Hans @\textsc{Tillo, Hans}, \emph{Schauspieler}|pwv}\pwindex{Sulzer, Hedwig Therese Anna @\textsc{Sulzer, Hedwig Therese Anna}, \emph{Schauspielerin}|pwv} des Kleinen
                  Schauspielhauses\orgindex{Kleines Schauspielhaus [Berlin]@Kleines Schauspielhaus [Berlin]|pw} in Berlin\oindex{Berlin@\textbf{Berlin}, \emph{Hauptstadt}|pw} enthält
               (herausgegeben vom Staats{\pb}minister a. D. Wolfgang Heine\pwindex{Heine, Wolfgang 3.\,5.\,1861 Poznan – 9.\,5.\,1944 Ascona@\textsc{Heine, Wolfgang} (3.\,5.\,1861 Poznan – 9.\,5.\,1944 Ascona), \emph{Notar, Politiker, Rechtsanwalt}|pw}, erschienen bei Rowohlt\orgindex{Ernst Rowohlt Verlag@Ernst Rowohlt Verlag|pw}) und nebstbei auch Gutachten führender Männer über
               das Werk bringt. Ich meine, dass sittliche Bedenken gegen eine Aufführung des Werkes\pwindex{Schnitzler, Arthur 15. 5. 1862 Wien – 21. 10. 1931 ebd.@\textsc{Schnitzler, Arthur} (15. 5. 1862 Wien – 21. 10. 1931 ebd.), \emph{Schriftsteller, Mediziner}!Reigen. Zehn Dialoge@\strich\emph{Reigen. Zehn Dialoge}|pwv} an einem künstlerisch
               geleiteten Theater heute überhaupt nicht mehr obwalten können und dass insbesondere
               im Laufe der letzten Jahre Dutzende von Theaterstücken aller Art auch an
               hervorragenden Bühnen in Szene gegangen sind, denen gegenüber der Vorwurf der
               Frivolität, Obszönität, Immoralität (wenn man sich überhaupt mit Vorwürfen dieser Art
               innerhalb der Kunst ernsthaft auseinandersetzen will) hundertmal mehr gerechtfertigt
               gewesen wären, als gerade gegenüber dem »Reigen\pwindex{Schnitzler, Arthur 15. 5. 1862 Wien – 21. 10. 1931 ebd.@\textsc{Schnitzler, Arthur} (15. 5. 1862 Wien – 21. 10. 1931 ebd.), \emph{Schriftsteller, Mediziner}!Reigen. Zehn Dialoge@\strich\emph{Reigen. Zehn Dialoge}|pw}«.\pend
           
\pstart
           Dies im allgemeinen. Selbstverständlich bietet der »Reigen\pwindex{Schnitzler, Arthur 15. 5. 1862 Wien – 21. 10. 1931 ebd.@\textsc{Schnitzler, Arthur} (15. 5. 1862 Wien – 21. 10. 1931 ebd.), \emph{Schriftsteller, Mediziner}!Reigen. Zehn Dialoge@\strich\emph{Reigen. Zehn Dialoge}|pw}«, wie am Ende jedes Stück, ausser der allgemeinen Frage der
               Bühnenmöglichkeit noch spezielle Probleme der Inszenierung, die aber meiner Ansicht
               nach nur in gemeinsamer Besprechung mit dem Regisseur und Dekorateur, keineswegs
               durch theoretische Erörterungen zu lösen wären. Schwierigkeiten sehe ich nirgends.
               Trotzdem aber möchte ich nicht verhehlen, dass mir persönlich eine Aufführung des
                  »Reigen\pwindex{Schnitzler, Arthur 15. 5. 1862 Wien – 21. 10. 1931 ebd.@\textsc{Schnitzler, Arthur} (15. 5. 1862 Wien – 21. 10. 1931 ebd.), \emph{Schriftsteller, Mediziner}!Reigen. Zehn Dialoge@\strich\emph{Reigen. Zehn Dialoge}|pw}« in Paris\oindex{Paris@\textbf{Paris}, \emph{Hauptstadt}|pw} erst dann recht willkommen wäre, wenn man vorher eines meiner anderen
               Dramen zur Aufführung gebracht hätte, das nicht von vornherein törichten oder
               böswilligen Missverständnissen ausgesetzt sein könnte. Dieser Gefahr läge ja meiner
               Ansicht nach überhaupt nicht vor, wenn nicht gerade dem »Reigen\pwindex{Schnitzler, Arthur 15. 5. 1862 Wien – 21. 10. 1931 ebd.@\textsc{Schnitzler, Arthur} (15. 5. 1862 Wien – 21. 10. 1931 ebd.), \emph{Schriftsteller, Mediziner}!Reigen. Zehn Dialoge@\strich\emph{Reigen. Zehn Dialoge}|pw}« jener ungerechtfertigte Ruf der Kühnheit oder
               Unsittlichkeit vorherginge und wenn ich in Frankreich\oindex{Frankreich@\textbf{Frankreich}|pw} heute schon bekannter, oder sagen wir populärer wäre, als ich es
               bin. Dies sind praktische Erwägungen, nichts weiter, die ich natürlich auch Antoine\pwindex{Antoine, André 31.\,1.\,1858 Limoges – 23.\,10.\,1943 Le Pouliguen@\textsc{Antoine, André} (31.\,1.\,1858 Limoges – 23.\,10.\,1943 Le Pouliguen), \emph{Theaterleiter, Schauspieler}|pw} gegenüber nicht verschweigen wollte,
               (dem ich schon für meine ersten, recht weit \label{K_L03975-3v}\edtext{zurückliegenden Erfolge}{\lemma{\textnormal{\emph{zurückliegenden Erfolge}}}\Cendnote{\textnormal{Durch das \emph{Theaterensembles}\orgindex{Théâtre Antoine@Théâtre Antoine|pwk}{ }Antoines\pwindex{Antoine, André 31.\,1.\,1858 Limoges – 23.\,10.\,1943 Le Pouliguen@\textsc{Antoine, André} (31.\,1.\,1858 Limoges – 23.\,10.\,1943 Le Pouliguen), \emph{Theaterleiter, Schauspieler}|pwk} waren \emph{Die Gefährtin}\pwindex{Schnitzler, Arthur 15. 5. 1862 Wien – 21. 10. 1931 ebd.@\textsc{Schnitzler, Arthur} (15. 5. 1862 Wien – 21. 10. 1931 ebd.), \emph{Schriftsteller, Mediziner}!Gefährtin. Schauspiel in einem Akt@\strich\emph{Die Gefährtin. Schauspiel in einem Akt}|pwk}\pwindex{Schnitzler, Arthur 15. 5. 1862 Wien – 21. 10. 1931 ebd.@\textsc{Schnitzler, Arthur} (15. 5. 1862 Wien – 21. 10. 1931 ebd.), \emph{Schriftsteller, Mediziner}!Compagne. Comédie en une acte@\strich\emph{La Compagne. Comédie en une acte}|pwk}, \emph{Der grüne Kakadu}\pwindex{Schnitzler, Arthur 15. 5. 1862 Wien – 21. 10. 1931 ebd.@\textsc{Schnitzler, Arthur} (15. 5. 1862 Wien – 21. 10. 1931 ebd.), \emph{Schriftsteller, Mediziner}!grüne Kakadu. Groteske in einem Akt@\strich\emph{Der grüne Kakadu. Groteske in einem Akt}|pwk}\pwindex{Schnitzler, Arthur 15. 5. 1862 Wien – 21. 10. 1931 ebd.@\textsc{Schnitzler, Arthur} (15. 5. 1862 Wien – 21. 10. 1931 ebd.), \emph{Schriftsteller, Mediziner}!Au Perroquet Vert@\strich\emph{Au Perroquet Vert}|pwk} und \emph{Die letzten Masken}\pwindex{Schnitzler, Arthur 15. 5. 1862 Wien – 21. 10. 1931 ebd.@\textsc{Schnitzler, Arthur} (15. 5. 1862 Wien – 21. 10. 1931 ebd.), \emph{Schriftsteller, Mediziner}!letzten Masken@\strich\emph{Die letzten Masken}|pwk}\pwindex{Schnitzler, Arthur 15. 5. 1862 Wien – 21. 10. 1931 ebd.@\textsc{Schnitzler, Arthur} (15. 5. 1862 Wien – 21. 10. 1931 ebd.), \emph{Schriftsteller, Mediziner}!Derniers masques. Comédie en un act@\strich\emph{Les Derniers masques. Comédie en un act}|pwk} in Frankreich\oindex{Frankreich@\textbf{Frankreich}|pwk}{ }erstaufgeführt\eventindex{Théâtre Antoine-Simone Berriau@\textbf{Théâtre Antoine-Simone Berriau}!Premiere von La Compagne, 29.4.1902@Premiere von La Compagne, 29.4.1902|pwkv}\eventindex{Théâtre Antoine-Simone Berriau@\textbf{Théâtre Antoine-Simone Berriau}!Premiere von Au Perroquet Vert, 7.11.1903@Premiere von Au Perroquet Vert, 7.11.1903|pwkv}\eventindex{Théâtre Antoine-Simone Berriau@\textbf{Théâtre Antoine-Simone Berriau}!Premiere von Les Derniers masques und Ariane blessée@Premiere von Les Derniers masques und Ariane blessée|pwkv} worden.}}}\label{K_L03975-3} in Paris\oindex{Berlin@\textbf{Berlin}, \emph{Hauptstadt}|pw} zu so
               herzlichem {\pb}Dank verpflichtet bin); dass ich im übrigen
               durchaus in der Lage bin jede Verantwortung für die Erlaubnis zu einer öffentlichen
               Aufführung des »Reigen\pwindex{Schnitzler, Arthur 15. 5. 1862 Wien – 21. 10. 1931 ebd.@\textsc{Schnitzler, Arthur} (15. 5. 1862 Wien – 21. 10. 1931 ebd.), \emph{Schriftsteller, Mediziner}!Reigen. Zehn Dialoge@\strich\emph{Reigen. Zehn Dialoge}|pw}« auf mich zu nehmen, wenn
               eine solche unter der Patronanz eines Theatermanns von Weltruf, wie Antoine\pwindex{Antoine, André 31.\,1.\,1858 Limoges – 23.\,10.\,1943 Le Pouliguen@\textsc{Antoine, André} (31.\,1.\,1858 Limoges – 23.\,10.\,1943 Le Pouliguen), \emph{Theaterleiter, Schauspieler}|pw}, stattfindet, brauche ich nicht erst zu
               versichern.\pend
           
\pstart
           Ich werde Ihnen dankbar sein, verehrteste Freundin, wenn Sie anlässlich einer
               neuerlichen Begegnung mit Antoine\pwindex{Antoine, André 31.\,1.\,1858 Limoges – 23.\,10.\,1943 Le Pouliguen@\textsc{Antoine, André} (31.\,1.\,1858 Limoges – 23.\,10.\,1943 Le Pouliguen), \emph{Theaterleiter, Schauspieler}|pw} ihn von
               diesem meinen Standpunkt in Kenntnis setzen und ihm zugleich meine herzlichen Grüssen
               bestellen wollten.\pend
           
\pstart
           Herzlichst{\\[\baselineskip]} Ihr aufrichtig ergebener\pend
           \leftskip=0em{}{\vspace{1\baselineskip}}
\pstart
           \noindent{}Frau Hofrätin Bertha Zuckerkandl,{\\}Wien\oindex{Wien@\textbf{Wien}, \emph{Verwaltungsgebiet}|pw}.\pend
           \selectlanguage{ngerman}\endnumbering\briefempfaengerindex{Zuckerkandl, Berta@\textsc{Zuckerkandl, Berta}!zzzSchnitzler, Arthur@\emph{von Arthur Schnitzler}!1929-03-041@{4. 3. 1929}|)be}\mylabel{L03975h}
\begin{anhang}
\end{anhang}\newcommand{\dateiname}{L03975}\newcommand{\titel}{Arthur Schnitzler an Berta Zuckerkandl, 4. 3. 1929}\newcommand{\editorInnen}{Herausgegeben von Jahnke, SelmaMüller, Martin Anton}%% latex-leseansicht-abspann.tex
%% Abspann für die Leseansicht.
%% Der Schalter \ifkorrekturansicht ist bereits durch den Vorspann gesetzt.

%% latex-abspann.tex
%% Gemeinsamer Abspann für Korrekturansicht und Leseansicht.
%% Setzt den Schalter \ifkorrekturansicht voraus (gesetzt in den
%% einbindenden Dateien latex-korrekturansicht-abspann.tex bzw.
%% latex-leseansicht-abspann.tex).
%% ---------------------------------------------------------------

\normalsize

% Das esempio-Environment wird nur in der Leseansicht benötigt
\ifkorrekturansicht\else
\newenvironment{esempio}[3]%
{
    \vspace{1.5ex}
    \rlap{\underline{#1}}
    \par
    \setlength{\parindent}{0cm}
    \nopagebreak
    \leftskip=#2cm
    \rightskip=#3cm
}
{
    \par
}
\fi

\doendnotes{C}
\bigskip
\vfill

\clearpage

\footnotesize

\ifkorrekturansicht
  \lohead{\textsc{register}}
\fi

% theindex-Environment neu definieren ohne reledmac
\makeatletter
\renewenvironment{theindex}{%
  \ifkorrekturansicht
    \section*{\indexname}%
  \else
    \subsubsection*{Index der erwähnten Entitäten}%
  \fi
  \setlength{\parindent}{0pt}%
  \setlength{\parskip}{0pt plus 0.3pt}%
  \let\item\@idxitem
}{%
  \ifkorrekturansicht\clearpage\fi
}
\makeatother

\IfFileExists{\jobname-pw.ind}{\input{\jobname-pw.ind}}{}

% Quellenangabe nur in der Leseansicht
\ifkorrekturansicht\else
% Fallback-Definitionen, falls die .tex-Datei \titel etc. nicht gesetzt hat
\providecommand{\titel}{}
\providecommand{\editorInnen}{}
\providecommand{\dateiname}{\jobname}

\vspace{3cm}

\vfill

\footnotesize
\textsc{Quelle}: \titel. Herausgegeben von {\editorInnen}. In: \emph{Arthur Schnitzler: Briefwechsel mit Autorinnen und Autoren}.
 Digitale Edition, https://schnitzler-briefe.acdh.oeaw.ac.at/{\dateiname}.html (Stand \today)
\fi

\end{document}


