%% latex-leseansicht-vorspann.tex
%% Vorspann für die Leseansicht.
%% Lädt die gemeinsame Datei latex-vorspann.tex mit nicht gesetztem Schalter.

\newif\ifkorrekturansicht
\korrekturansichtfalse

\input{../tex-inputs/latex-vorspann}


         
         \newcommand{\erwaehntePersonen}{Personen:  Molière}
         \newcommand{\erwaehnteInstitutionen}{}
         \newcommand{\erwaehnteOrte}{Orte: Andorf, Wien}
         \newcommand{\erwaehnteWerke}{Werke: Robert, Yppl. Idylle in fünf Akten}
               \section[Robert Adam an Arthur Schnitzler, 29. 7. 1918]{ Robert Adam an Arthur Schnitzler, 29. 7. 1918}\nopagebreak\mylabel{v}\rehead{ }\begin{ledgroupsized}[t]{13cm}\normalsize\beginnumbering \toendnotes[C]{\smallbreak\pagebreak[2]} \Standort{CUL, Schnitzler, B 1.}
\physDesc{Brief, 1 Blatt, 3 Seiten
\newline{}Handschrift: schwarze Tinte, deutsche Kurrent
\newline{}Schnitzler: 1) mit Bleistift beschriftet: »\textsc{Adam}«  2) mit rotem Buntstift zwei Unterstreichungen\newline{}Ordnung: von unbekannter Hand nummeriert: »5« }\Standort{Wien, Österreichische Nationalbibliothek, Cod.ser. 52.263, 209.}
\physDesc{Brief, maschinelle Abschrift
\newline{}Schreibmaschine}\toendnotes[C]{\smallbreak}\pstart
           \raggedleft{}{\pb}Wien\oindex{Wien@\textbf{Wien}|pw}, 29. Juli 1918.\pend
           \pstart\center{}Hochverehrter Herr Doktor!\pend\pstart
           Beſten Dank für Ihre Karte!\pend
           \pstart
           Ich bin ſeit geſtern – denn der Urlaub iſt zu Ende – wieder in Wien\oindex{Wien@\textbf{Wien}|pw} und habe heute früh den Dienſt wiederangetreten. Eine
                    Stellage und der Schreibtiſch voll unerledigter Akten laſſen mir die nächſten
                    Wochen wenig erfreulich erſcheinen; morgen iſt der erſte Verhandlungstag.\pend
           \pstart
           Den Urlaub habe ich, glaub ich, gut ausgenützt. Ich brachte von einem fünfaktigen
                        Stück\pwindex{Adam, Robert 20.04.1877 – 16.10.1961@\textsc{Adam, Robert} (20.04.1877 – 16.10.1961), \emph{Schriftsteller, Richter}!Yppl. Idylle in fuenf AktenNone@\strich\emph{Yppl. Idylle in fünf Akten} {[}None{]}|pwv} die erſten drei
                    Akte, die Hälfte des vierten und den fünften bis auf die Schlußſzene mit
                    nachhauſe: die Arbeit der letzten zehn Tage. Hoffentlich bringe ich ſie heut und
                    morgen gänzlich unter Dach; ſo lange wird wohl die {\pb}»Kraft« noch anhalten. Aber dies
                    Stück iſt keineswegs das fürchterliche Kriegsdrama\pwindex{Adam, Robert 20.04.1877 – 16.10.1961@\textsc{Adam, Robert} (20.04.1877 – 16.10.1961), \emph{Schriftsteller, Richter}!RobertNone@\strich\emph{Robert} {[}None{]}|pwv} geworden, das ich in Andorf\oindex{Andorf@\textbf{Andorf}|pw} vorerſt ſchreiben wollte: ich war viel zu weit weg
                    von Kriegsnot und Ärger, Hunger und Bitterkeit. Der heimkehrende Menſchenfreſſer
                    blieb liegen: vielleicht ſteht er im Winter wieder auf. Was entſtand iſt: Yppl\pwindex{Adam, Robert 20.04.1877 – 16.10.1961@\textsc{Adam, Robert} (20.04.1877 – 16.10.1961), \emph{Schriftsteller, Richter}!Yppl. Idylle in fuenf AktenNone@\strich\emph{Yppl. Idylle in fünf Akten} {[}None{]}|pw}, eine Idylle in 5 Akten aus der Zeit vor
                    dem neuen Mittelalter – eigentlich eine Provinzkomödie, die den Mangel ſtarker
                    Handlung durch die Bezeichnung Idylle beſchönigen will. Ich habe mit großer Luſt
                    und vielem Behagen dieſe vor ſehr vielen Jahren halb-ſelbſterlebten Szenen
                    niedergeſchrieben und bin ſehr begierig, ob ſie auch Ihnen Spaß machen. Ich
                    meine noch – denn ich bin ja noch nicht fertig –, daß man der Arbeit anſieht,
                    wie eifrig ich im letzten Jahr meinen Molière\pwindex{Moliere 14.01.1622 – 17.02.1673@\textsc{Molière} (14.01.1622 – 17.02.1673), \emph{Schriftsteller, Theaterleiter, Schauspieler}|pw}{ }ſtudiert habe.\pend
           \pstart
           {\pb}Wenn ich Sie vor Ihrer Abreiſe noch
                    ſehen könnte, wäre es mir \introOben{}eine\introOben{} außerordentliche Freude.
                    Ich habe ſelbſtverſtändlich immer Zeit.\pend
           \pstart
           Mit den besten Grüßen Ihr ſehr ergebener{\\[\baselineskip]}\spacefill\mbox{Robert Adam}\pend
           \leftskip=0em{}
         
         \endnumbering\mylabel{h}\end{ledgroupsized}  \newcommand{\dateiname}{L02291}\newcommand{\titel}{Robert Adam an Arthur Schnitzler, 29. 7. 1918}\newcommand{\editorInnen}{Martin Anton Müller und Gerd-Hermann Susen}%% latex-leseansicht-abspann.tex
%% Abspann für die Leseansicht.
%% Der Schalter \ifkorrekturansicht ist bereits durch den Vorspann gesetzt.

%% latex-abspann.tex
%% Gemeinsamer Abspann für Korrekturansicht und Leseansicht.
%% Setzt den Schalter \ifkorrekturansicht voraus (gesetzt in den
%% einbindenden Dateien latex-korrekturansicht-abspann.tex bzw.
%% latex-leseansicht-abspann.tex).
%% ---------------------------------------------------------------

\normalsize

% Das esempio-Environment wird nur in der Leseansicht benötigt
\ifkorrekturansicht\else
\newenvironment{esempio}[3]%
{
    \vspace{1.5ex}
    \rlap{\underline{#1}}
    \par
    \setlength{\parindent}{0cm}
    \nopagebreak
    \leftskip=#2cm
    \rightskip=#3cm
}
{
    \par
}
\fi

\doendnotes{C}
\bigskip
\vfill

\clearpage

\footnotesize

\ifkorrekturansicht
  \lohead{\textsc{register}}
\fi

% theindex-Environment neu definieren ohne reledmac
\makeatletter
\renewenvironment{theindex}{%
  \ifkorrekturansicht
    \section*{\indexname}%
  \else
    \subsubsection*{Index der erwähnten Entitäten}%
  \fi
  \setlength{\parindent}{0pt}%
  \setlength{\parskip}{0pt plus 0.3pt}%
  \let\item\@idxitem
}{%
  \ifkorrekturansicht\clearpage\fi
}
\makeatother

\IfFileExists{\jobname-pw.ind}{\input{\jobname-pw.ind}}{}

% Quellenangabe nur in der Leseansicht
\ifkorrekturansicht\else
% Fallback-Definitionen, falls die .tex-Datei \titel etc. nicht gesetzt hat
\providecommand{\titel}{}
\providecommand{\editorInnen}{}
\providecommand{\dateiname}{\jobname}

\vspace{3cm}

\vfill

\footnotesize
\textsc{Quelle}: \titel. Herausgegeben von {\editorInnen}. In: \emph{Arthur Schnitzler: Briefwechsel mit Autorinnen und Autoren}.
 Digitale Edition, https://schnitzler-briefe.acdh.oeaw.ac.at/{\dateiname}.html (Stand \today)
\fi

\end{document}


      