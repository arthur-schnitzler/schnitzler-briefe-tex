%% latex-korrekturansicht-vorspann.tex
%% Vorspann für die Korrekturansicht.
%% Lädt die gemeinsame Datei latex-vorspann.tex mit gesetztem Schalter.

\newif\ifkorrekturansicht
\korrekturansichttrue

\input{../tex-inputs/latex-vorspann}


\section[Robert Adam an Arthur Schnitzler, 29. 7. 1918]{L02291 Robert Adam an Arthur Schnitzler, 29. 7. 1918}
\nopagebreak\mylabel{L02291v}
\rehead{ }\normalsize\beginnumbering\briefempfaengerindex{Schnitzler, Arthur@\textsc{Schnitzler, Arthur}!zzzAdam, Robert@\emph{von Robert Adam}!1918-07-291@{29. 7. 1918}|(be}
\toendnotes[C]{\smallbreak\pagebreak[2]}\Standort{CUL, Schnitzler, B 1.}
\physDesc{Brief, 1 Blatt, 3 Seiten, 1636 Zeichen
\newline{}Handschrift: schwarze Tinte, deutsche Kurrent
\newline{}Schnitzler: 1) mit Bleistift beschriftet: »\textsc{Adam}«  2) mit rotem Buntstift zwei Unterstreichungen
\newline{}Ordnung: von unbekannter Hand nummeriert: »5« }\Standort{Wien, Österreichische Nationalbibliothek, Cod.ser. 52.263, 209.}
\physDesc{Brief, maschinenschriftliche Abschrift1 Blatt, 1 Seite, 1636 Zeichen
\newline{}Schreibmaschine}\toendnotes[C]{\smallbreak}
\pstart
           \raggedleft{}{\pb}Wien\oindex{Wien@\textbf{Wien}, \emph{A.ADM2}|pw}, 29. Juli 1918.\pend
           
\pstart\center{}Hochverehrter Herr Doktor!\pend\vspace{0.5em}
\pstart
           Beſten Dank für Ihre Karte!\pend
           
\pstart
           Ich bin ſeit geſtern – denn der Urlaub iſt zu Ende – wieder in Wien\oindex{Wien@\textbf{Wien}, \emph{A.ADM2}|pw} und habe heute früh den Dienſt wiederangetreten. Eine
               Stellage und der Schreibtiſch voll unerledigter Akten laſſen mir die nächſten Wochen
               wenig erfreulich erſcheinen; morgen iſt der erſte Verhandlungstag.\pend
           
\pstart
           Den Urlaub habe ich, glaub ich, gut ausgenützt. Ich brachte von einem fünfaktigen Stück\pwindex{Yppl. Idylle in fuenf Akten@\emph{Yppl. Idylle in fünf Akten}|pwv} die erſten drei Akte, die
               Hälfte des vierten und den fünften bis auf die Schlußſzene mit nachhauſe: die Arbeit
               der letzten zehn Tage. Hoffentlich bringe ich ſie heut und morgen gänzlich unter
               Dach; ſo lange wird wohl die {\pb}»Kraft« noch
               anhalten. Aber dies Stück iſt keineswegs das fürchterliche Kriegsdrama\pwindex{Robert@\emph{Robert}|pwv} geworden, das ich in Andorf\oindex{Andorf@\textbf{Andorf}, \emph{P.PPLA3}|pw} vorerſt ſchreiben wollte: ich war viel zu
               weit weg von Kriegsnot und Ärger, Hunger und Bitterkeit. Der heimkehrende
               Menſchenfreſſer blieb liegen: vielleicht ſteht er im Winter wieder auf. Was entſtand
               iſt: Yppl\pwindex{Yppl. Idylle in fuenf Akten@\emph{Yppl. Idylle in fünf Akten}|pw}, eine Idylle in 5 Akten aus der Zeit
               vor dem neuen Mittelalter – eigentlich eine Provinzkomödie, die den Mangel ſtarker
               Handlung durch die Bezeichnung Idylle beſchönigen will. Ich habe mit großer Luſt und
               vielem Behagen dieſe vor ſehr vielen Jahren halb-ſelbſterlebten Szenen
               niedergeſchrieben und bin ſehr begierig, ob ſie auch Ihnen Spaß machen. Ich meine
               noch – denn ich bin ja noch nicht fertig –, daß man der Arbeit anſieht, wie eifrig
               ich im letzten Jahr meinen Molière\pwindex{Moliere 14.01.1622 – 17.02.1673@\textsc{Molière} (14.01.1622 – 17.02.1673), \emph{Schriftsteller/Schriftstellerin, Theaterleiter/Theaterleiterin, Schauspieler/Schauspielerin}|pw}{ }ſtudiert habe.\pend
           
\pstart
           {\pb}Wenn ich Sie vor Ihrer Abreiſe noch ſehen
               könnte, wäre es mir \introOben{}eine\introOben{} außerordentliche Freude. Ich habe
               ſelbſtverſtändlich immer Zeit.\pend
           
\pstart
           Mit den besten Grüßen Ihr ſehr ergebener{\\[\baselineskip]}\spacefill\mbox{Robert Adam}\pend
           \leftskip=0em{}\selectlanguage{ngerman}\endnumbering\briefempfaengerindex{Schnitzler, Arthur@\textsc{Schnitzler, Arthur}!zzzAdam, Robert@\emph{von Robert Adam}!1918-07-291@{29. 7. 1918}|)be}\mylabel{L02291h}  \normalsize

\doendnotes{C}
\bigskip
\vfill

\clearpage

\footnotesize

\lohead{\textsc{register}}

% Definiere theindex-Environment komplett neu ohne reledmac
\makeatletter
\renewenvironment{theindex}{%
  \section*{\indexname}%
  \setlength{\parindent}{0pt}%
  \setlength{\parskip}{0pt plus 0.3pt}%
  \let\item\@idxitem
}{%
  \clearpage
}
\makeatother

\IfFileExists{\jobname-pw.ind}{\input{\jobname-pw.ind}}{}

\end{document}

      