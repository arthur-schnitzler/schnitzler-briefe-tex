%% latex-leseansicht-vorspann.tex
%% Vorspann für die Leseansicht.
%% Lädt die gemeinsame Datei latex-vorspann.tex mit nicht gesetztem Schalter.

\newif\ifkorrekturansicht
\korrekturansichtfalse

\input{../tex-inputs/latex-vorspann}


               \section[Arthur Schnitzler an Richard Beer-Hofmann, 20. 10. 1894]{ Arthur Schnitzler an Richard Beer-Hofmann, 20. 10. 1894}\nopagebreak\mylabel{v}\rehead{ }\begin{ledgroupsized}[t]{13cm}\normalsize\beginnumbering\briefempfaengerindex{Beer-Hofmann, Richard@\textsc{Beer-Hofmann, Richard}!zzzSchnitzler, Arthur@\emph{von Arthur Schnitzler}!1894-10-201@{20. 10. 1894}|(be} \toendnotes[C]{\smallbreak\pagebreak[2]} \Standort{YCGL, MSS 31.}
\physDesc{Brief, 3 Blätter, 12 Seiten, Umschlag
\newline{}Handschrift: Bleistift, deutsche Kurrent\newline{}Versand: 1) Stempel: »\nobreak{}\oindex{I., Innere Stadt@\textbf{I., Innere Stadt}|pwk}Wien 1/1, 20. 10. 94, \textcolor{gray}{7-8}N\nobreak{}«.  2) Stempel: »\nobreak{}\oindex{Neapel@\textbf{Neapel}|pwk}\textcolor{gray}{Nap}o\textcolor{gray}{l}i, \textcolor{gray}{23} 10-94, 3 S\nobreak{}«. }\buchAbdrucke{\weitereDrucke{1) Arthur Schnitzler: \emph{Briefe 1875–1912}. Hg. Therese Nickl und Heinrich Schnitzler. Frankfurt am Main: \emph{S. Fischer} 1981, S. 232–233.} \weitereDrucke{2) Arthur Schnitzler, Richard Beer-Hofmann: \emph{Briefwechsel 1891–1931}. Hg. Konstanze Fliedl. Wien, Zürich: \emph{Europaverlag} 1992, S. 66–67.} \weitereDrucke{3) Arthur Schnitzler: \emph{Briefe.} In: \emph{Die Neue Rundschau}, Bd. 68 (1957) Nr. 1, S. 88–89.} \weitereDrucke{4) Hermann Bahr, Arthur Schnitzler: \emph{Briefwechsel, Aufzeichnungen, Dokumente (1891–1931)}. Hg. Kurt Ifkovits und Martin Anton Müller. Göttingen: \emph{Wallstein} 2018.} }\toendnotes[C]{\smallbreak}\pstart{}{\pb}\textsc{Dr. Arthur Schnitzler}, Wien,
                     IX. Frankgaſſe 1.\oindex{Frankgasse@\textbf{Frankgasse}|pw}\pend{}{\bigskip}\pstart{}{\pb}\textsc{Italien\oindex{Italien@\textbf{Italien}|pw}}\pend{}\pstart{}\textsc{Dr. Richard Beer Hofmann}\pend{}\pstart{}\textsc{Neapel\oindex{Neapel@\textbf{Neapel}|pw}}\pend{}\pstart{}\textsc{Hotel Hassler\oindex{Hôtel Hassler@\textbf{Hôtel Hassler}|pw}}\pend{}{\bigskip}\pstart
           \raggedleft{}{\pb}20. 10. 94\pend
           \pstart{}Lieber Richard. –\pend\pstart
           Schmetterlingsſchlacht\pwindex{Sudermann, Hermann 30.09.1857 – 21.11.1928@\textsc{Sudermann, Hermann} (30.09.1857 – 21.11.1928), \emph{Schriftsteller}!Schmetterlingsschlacht1894@\strich\emph{Die Schmetterlingsschlacht} {[}1894{]}|pw}: Erſter Akt ſehr gut, voll
               glänzenden, nur zuweilen etwas abſichtlichen Details;– machte erwartungsvolle
               treffliche Sti{\geminationm}ung. Zweiter Akt läßt ſich nicht übel an;
               befremdet bereits durch einige Trivialitäten – enttäuſcht aber noch nicht recht. Der
               dritte Akt {\pb}ſchwach, ungeſchickt, ohne ſelbſt den
               ſtofflichen Inhalt, der in ihm ſteckt, auszuſchöpfen; verſti{\geminationm}end, mit einem affectirten, pſychologiſch falſchen,
               enervirenden Schluſs. Der letzte Akt kurzweg kläglich, geradezu erbitternd. – Suderma{\geminationn}\pwindex{Sudermann, Hermann 30.09.1857 – 21.11.1928@\textsc{Sudermann, Hermann} (30.09.1857 – 21.11.1928), \emph{Schriftsteller}|pw}{ }ſcheint doch nur der große Meiſter der erſten Akte
               zu ſein. – (Ehre\pwindex{Sudermann, Hermann 30.09.1857 – 21.11.1928@\textsc{Sudermann, Hermann} (30.09.1857 – 21.11.1928), \emph{Schriftsteller}!Ehre1889@\strich\emph{Die Ehre} {[}1889{]}|pw}, Sodom\pwindex{Sudermann, Hermann 30.09.1857 – 21.11.1928@\textsc{Sudermann, Hermann} (30.09.1857 – 21.11.1928), \emph{Schriftsteller}!Sodom s Ende1890@\strich\emph{Sodom’s Ende} {[}1890{]}|pw}, Heimath\pwindex{Sudermann, Hermann 30.09.1857 – 21.11.1928@\textsc{Sudermann, Hermann} (30.09.1857 – 21.11.1928), \emph{Schriftsteller}!Heimat1893@\strich\emph{Heimat} {[}1893{]}|pw} – {\pb}überall der erſte Akt am beſten.) – Einige Figuren der
                  Schmett.\pwindex{Sudermann, Hermann 30.09.1857 – 21.11.1928@\textsc{Sudermann, Hermann} (30.09.1857 – 21.11.1928), \emph{Schriftsteller}!Schmetterlingsschlacht1894@\strich\emph{Die Schmetterlingsschlacht} {[}1894{]}|pw} famos, andre unerlaubt läppiſch. Das
               ganze Stück nicht einer glücklichen Eingebung entſta{\geminationm}end, ſondern recht mühſelig und ohne Glück conſtruirt. Das ärgſte war zu vermeiden,
                  we{\geminationn} 3. u 4. Akt zu einem zusa{\geminationm}enge{\pb}zogen werden und
               die Rolle der naiven Roſi\pwindex{Sudermann, Hermann 30.09.1857 – 21.11.1928@\textsc{Sudermann, Hermann} (30.09.1857 – 21.11.1928), \emph{Schriftsteller}!Schmetterlingsschlacht1894@\strich\emph{Die Schmetterlingsschlacht} {[}1894{]}|pwv} aus der
               gemeinen Theaterſchablone ins menſchliche hinaufgehoben wird. Die Darſtellung ist
               großartig; ſie lügt geradezu Seelen in die Puppen. – Um die \textsc{Schm}.\pwindex{Sudermann, Hermann 30.09.1857 – 21.11.1928@\textsc{Sudermann, Hermann} (30.09.1857 – 21.11.1928), \emph{Schriftsteller}!Schmetterlingsschlacht1894@\strich\emph{Die Schmetterlingsschlacht} {[}1894{]}|pw} für Sud.’s\pwindex{Sudermann, Hermann 30.09.1857 – 21.11.1928@\textsc{Sudermann, Hermann} (30.09.1857 – 21.11.1928), \emph{Schriftsteller}|pw} beſtes Stück zu halten, muß man entweder nichts verſtehn – oder \textsc{Herma{\geminationn}{ }{\pb}Bahr}\pwindex{Bahr, Hermann 19.07.1863 – 15.01.1934@\textsc{Bahr, Hermann} (19.07.1863 – 15.01.1934), \emph{Schriftsteller, Kritiker}|pw}{ }ſein. Ueber ſeine \label{K_L00387_1v}\edtext{Kritik\pwindex{Bahr, Hermann 19.07.1863 – 15.01.1934@\textsc{Bahr, Hermann} (19.07.1863 – 15.01.1934), \emph{Schriftsteller, Kritiker}!Burgtheater (»Die Schmetterlingsschlacht«. Komoedie in vier Akten von Hermann Sudermann. Zum ersten Mal aufgefuehrt am 6. October 1894)13. 10. 1894@\strich\emph{Burgtheater (»Die Schmetterlingsschlacht«. Komödie in vier Akten von Hermann Sudermann. Zum ersten Mal aufgeführt am 6. October 1894)} {[}13. 10. 1894{]}|pwv}}{\lemma{\textnormal{\emph{Kritik}}}\Cendnote{\textnormal{Hermann Bahr\pwindex{Bahr, Hermann 19.07.1863 – 15.01.1934@\textsc{Bahr, Hermann} (19.07.1863 – 15.01.1934), \emph{Schriftsteller, Kritiker}|pwk}: \emph{Burgtheater (»Die Schmetterlingsschlacht«. Komödie in vier Akten von
                        Hermann Sudermann. Zum ersten Mal aufgeführt am 6. October
                        1894)}\pwindex{Bahr, Hermann 19.07.1863 – 15.01.1934@\textsc{Bahr, Hermann} (19.07.1863 – 15.01.1934), \emph{Schriftsteller, Kritiker}!Burgtheater (»Die Schmetterlingsschlacht«. Komoedie in vier Akten von Hermann Sudermann. Zum ersten Mal aufgefuehrt am 6. October 1894)13. 10. 1894@\strich\emph{Burgtheater (»Die Schmetterlingsschlacht«. Komödie in vier Akten von Hermann Sudermann. Zum ersten Mal aufgeführt am 6. October 1894)} {[}13. 10. 1894{]}|pwk}. In: \emph{Die Zeit}\pwindex{Zeit. Wiener Wochenschrift1894 – 1904@\emph{Die Zeit. Wiener Wochenschrift}|pwk}, Bd. 1,
                     H. 2, 13. 10. 1894, S. 26.}}}\label{K_L00387_1h} und noch vieles andre hab
               ich geſtern erſt zwei Stunden mit ihm geplauſcht. Ich zweifle gar nicht: er will
               immer intereſſant, i{\geminationm}er geiſtvoll, i{\geminationm}er bizarr ſein, und es gelingt ihm faſt i{\geminationm}er – aber we{\geminationn}\substVorne{}\textsuperscript{seine}\substDazwischen{}die\substHinten{} Originalität {\pb}und die Bizarrerie – ja ſagen
               wir zuweilen ſelbſt die Tiefe ſeiner künſtleriſchen Anſchauungen mit der Wahrheit
                  zuſa{\geminationm}enfällt, ſo iſt das gewiſs mehr Zufall als der
               ſchöne Drang nach kritiſcher Ehrlichkeit. Und was könnte dieſer Menſch nicht {\pb}leiſten, wenn er zu ſeinen außerordentlichen
               Eigenſchaften auch noch die der Verläßlichkeit hätte. Er iſt einer von den glänzenden
               – aber nicht einer von den Echten. –\pend
           \pstart
           Heut geh ich zur \textsc{Première} von den Komödianten\pwindex{\textcolor{red}{\textsuperscript{XXXX1 indx}}!Comoedianten1894@\strich\emph{Comödianten} {[}1894{]}|pw}. Haben Sie auch in \textsc{theatralibus} was {\pb}geſehen? Gehn Sie nach \textsc{Sicilien}\oindex{Sizilien@\textbf{Sizilien}|pw}? –\pend
           \pstart
           Heute holt der Abſchreiber\pwindex{?? [Schreibkraft fuer Arthur Schnitzler] @\textsc{?? [Schreibkraft für Arthur Schnitzler]}|pwv}
               meinen letzten Akt\pwindex{Schnitzler, Arthur 15.05.1862 – 21.10.1931@\textsc{Schnitzler, Arthur} (15.05.1862 – 21.10.1931), \emph{Schriftsteller, Mediziner}!Liebelei. Schauspiel in drei Akten9. 10. 1895@\strich\emph{Liebelei. Schauspiel in drei Akten} {[}9. 10. 1895{]}|pwv}. In acht Tagen
               hoff’ ichs einreichen zu können. – \strikeout{Auch}{ }\textsc{Hugo}\pwindex{Hofmannsthal, Hugo von 01.02.1874 – 15.07.1929@\textsc{Hofmannsthal, Hugo von} (01.02.1874 – 15.07.1929), \emph{Schriftsteller}|pw} und Salten\pwindex{Salten, Felix 06.09.1869 – 08.10.1945@\textsc{Salten, Felix} (06.09.1869 – 08.10.1945), \emph{Schriftsteller, Journalist}|pw} finden: Burgtheater\oindex{Burgtheater@\textbf{Burgtheater}|pw}. \textsc{Bahr}\pwindex{Bahr, Hermann 19.07.1863 – 15.01.1934@\textsc{Bahr, Hermann} (19.07.1863 – 15.01.1934), \emph{Schriftsteller, Kritiker}|pw} hat auch ſchon mit \textsc{Burckh}\pwindex{Burckhard, Max Eugen 14.07.1854 – 16.03.1912@\textsc{Burckhard, Max Eugen} (14.07.1854 – 16.03.1912), \emph{Schriftsteller, Rechtswissenschaftler, Theaterleiter}|pw}. geſprochen und Burckh\pwindex{Burckhard, Max Eugen 14.07.1854 – 16.03.1912@\textsc{Burckhard, Max Eugen} (14.07.1854 – 16.03.1912), \emph{Schriftsteller, Rechtswissenschaftler, Theaterleiter}|pw}. {\pb}»erwartet« das Stück\pwindex{Schnitzler, Arthur 15.05.1862 – 21.10.1931@\textsc{Schnitzler, Arthur} (15.05.1862 – 21.10.1931), \emph{Schriftsteller, Mediziner}!Liebelei. Schauspiel in drei Akten9. 10. 1895@\strich\emph{Liebelei. Schauspiel in drei Akten} {[}9. 10. 1895{]}|pwv}. Charakteriſtiſch übrigens, daſs Bahr\pwindex{Bahr, Hermann 19.07.1863 – 15.01.1934@\textsc{Bahr, Hermann} (19.07.1863 – 15.01.1934), \emph{Schriftsteller, Kritiker}|pw}, nachdem er mit \textsc{Burckh}\pwindex{Burckhard, Max Eugen 14.07.1854 – 16.03.1912@\textsc{Burckhard, Max Eugen} (14.07.1854 – 16.03.1912), \emph{Schriftsteller, Rechtswissenschaftler, Theaterleiter}|pw} geſprochen und nachdem er von dem Stück\pwindex{Schnitzler, Arthur 15.05.1862 – 21.10.1931@\textsc{Schnitzler, Arthur} (15.05.1862 – 21.10.1931), \emph{Schriftsteller, Mediziner}!Liebelei. Schauspiel in drei Akten9. 10. 1895@\strich\emph{Liebelei. Schauspiel in drei Akten} {[}9. 10. 1895{]}|pwv} nichts wußte als, was ihm Hugo\pwindex{Hofmannsthal, Hugo von 01.02.1874 – 15.07.1929@\textsc{Hofmannsthal, Hugo von} (01.02.1874 – 15.07.1929), \emph{Schriftsteller}|pw} geſagt, daſs es ſehr gut und »Burgtheater\oindex{Burgtheater@\textbf{Burgtheater}|pw}« ſei, mir gegenüber äußerte: {\pb}»Ich hab’ die Empfindung, daſs es ins Raimundtheater\oindex{Raimund-Theater@\textbf{Raimund-Theater}|pw} gehört.« – Man ka{\geminationn} übrigens
               weniger als je ans Raimundth\oindex{Raimund-Theater@\textbf{Raimund-Theater}|pw}. denken – es wird dort
               geſpielt wie an einem Provinztheater, wo die Leut eben zehn Proben haben, {\pb}ſtatt einer oder zwei. Aber dadurch kriegen die Herren
                  Heding\pwindex{Heding, Edmund 1867 – 1939@\textsc{Heding, Edmund} (1867 – 1939), \emph{Schauspieler/Schauspielerin}|pw} und Nerz\pwindex{Nerz, Ludwig 16.02.1866 – 20.01.1938@\textsc{Nerz, Ludwig} (16.02.1866 – 20.01.1938), \emph{Schauspieler/Schauspielerin, Filmproduzent/Filmproduzentin, Autor/Autorin >> Drehbuchautor/Drehbuchautorin}|pw} u. ſ. w. nicht mehr Talent als ſie haben. – Burgtheater\oindex{Burgtheater@\textbf{Burgtheater}|pw}verſuch muſs natürlich ſtrenges Geheimnis bleiben, da ich ja
               dann, we{\geminationn}{ }B.\pwindex{Burckhard, Max Eugen 14.07.1854 – 16.03.1912@\textsc{Burckhard, Max Eugen} (14.07.1854 – 16.03.1912), \emph{Schriftsteller, Rechtswissenschaftler, Theaterleiter}|pw} es reſusirt {\pb}beim Volkstheater\oindex{Volkstheater@\textbf{Volkstheater}|pw} einreichen will. – \pend
           \pstart
           Ich freue mich auf Ihre Rückkehr. – \pend
           \pstart
           Herzlichen Gruſs{\\[\baselineskip]}Ihr \spacefill\mbox{Arthur}\pend
           \leftskip=0em{}\endnumbering\briefempfaengerindex{Beer-Hofmann, Richard@\textsc{Beer-Hofmann, Richard}!zzzSchnitzler, Arthur@\emph{von Arthur Schnitzler}!1894-10-201@{20. 10. 1894}|)be}\mylabel{h}\end{ledgroupsized}  \newcommand{\dateiname}{L00387}\newcommand{\titel}{Arthur Schnitzler an Richard Beer-Hofmann, 20. 10. 1894}\newcommand{\editorInnen}{ Martin Anton Müller und Gerd-Hermann Susen}%% latex-leseansicht-abspann.tex
%% Abspann für die Leseansicht.
%% Der Schalter \ifkorrekturansicht ist bereits durch den Vorspann gesetzt.

%% latex-abspann.tex
%% Gemeinsamer Abspann für Korrekturansicht und Leseansicht.
%% Setzt den Schalter \ifkorrekturansicht voraus (gesetzt in den
%% einbindenden Dateien latex-korrekturansicht-abspann.tex bzw.
%% latex-leseansicht-abspann.tex).
%% ---------------------------------------------------------------

\normalsize

% Das esempio-Environment wird nur in der Leseansicht benötigt
\ifkorrekturansicht\else
\newenvironment{esempio}[3]%
{
    \vspace{1.5ex}
    \rlap{\underline{#1}}
    \par
    \setlength{\parindent}{0cm}
    \nopagebreak
    \leftskip=#2cm
    \rightskip=#3cm
}
{
    \par
}
\fi

\doendnotes{C}
\bigskip
\vfill

\clearpage

\footnotesize

\ifkorrekturansicht
  \lohead{\textsc{register}}
\fi

% theindex-Environment neu definieren ohne reledmac
\makeatletter
\renewenvironment{theindex}{%
  \ifkorrekturansicht
    \section*{\indexname}%
  \else
    \subsubsection*{Index der erwähnten Entitäten}%
  \fi
  \setlength{\parindent}{0pt}%
  \setlength{\parskip}{0pt plus 0.3pt}%
  \let\item\@idxitem
}{%
  \ifkorrekturansicht\clearpage\fi
}
\makeatother

\IfFileExists{\jobname-pw.ind}{\input{\jobname-pw.ind}}{}

% Quellenangabe nur in der Leseansicht
\ifkorrekturansicht\else
% Fallback-Definitionen, falls die .tex-Datei \titel etc. nicht gesetzt hat
\providecommand{\titel}{}
\providecommand{\editorInnen}{}
\providecommand{\dateiname}{\jobname}

\vspace{3cm}

\vfill

\footnotesize
\textsc{Quelle}: \titel. Herausgegeben von {\editorInnen}. In: \emph{Arthur Schnitzler: Briefwechsel mit Autorinnen und Autoren}.
 Digitale Edition, https://schnitzler-briefe.acdh.oeaw.ac.at/{\dateiname}.html (Stand \today)
\fi

\end{document}


      