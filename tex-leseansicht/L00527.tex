%% latex-leseansicht-vorspann.tex
%% Vorspann für die Leseansicht.
%% Lädt die gemeinsame Datei latex-vorspann.tex mit nicht gesetztem Schalter.

\newif\ifkorrekturansicht
\korrekturansichtfalse

\input{../tex-inputs/latex-vorspann}

\begin{center}
            \textcolor{red}{ENTWURF. ENTZIFFERUNG NOCH NICHT KORREKTURGELESEN}
                      \end{center}
            
               \section[Lou Andreas-Salomé an Arthur Schnitzler, {[}9. 1. 1896{]}]{ Lou Andreas-Salomé an Arthur Schnitzler, {[}9. 1. 1896{]}}\nopagebreak\mylabel{v}\rehead{ }\begin{ledgroupsized}[t]{13cm}\normalsize\beginnumbering\briefempfaengerindex{Schnitzler, Arthur@\textsc{Schnitzler, Arthur}!zzzAndreas-Salome, Lou@\emph{von Lou Andreas-Salomé}!1896-01-091@{{[}9. 1. 1896{]}}|(be} \toendnotes[C]{\smallbreak\pagebreak[2]} \Standort{CUL, Schnitzler, B 3.}
\physDesc{Kartenbrief
\newline{}Handschrift: schwarze Tinte, deutsche Kurrent\newline{}Versand: ohne postalischen Übermittlungsvermerk 
\newline{}Schnitzler: 1) mit Bleistift datiert: »9/1 96« 2) mit rotem Buntstift zwei Unterstreichungen\newline{}Ordnung: mit Bleistift von unbekannter Hand nummeriert:
                                    »15« }\toendnotes[C]{\smallbreak}\pstart{}{\pb}Herrn \textsc{D\textsuperscript{r}}\pend{}\pstart{}\textsc{Arthur Schnitzler.}\pend{}{\bigskip}\pstart
           \noindent{}{\pb}Lieber Herr \textsc{D\textsuperscript{r}}, glückliche \label{K_L00527_1v}\edtext{Reiſe}{\lemma{\textnormal{\emph{Reiſe}}}\Cendnote{\textnormal{Die Reise nach Frankfurt\oindex{Frankfurt am Main@\textbf{Frankfurt am Main}|pwk} fand von 10. 1. bis zum
                     15. 1. 1896 statt und führte auch nach Köln\oindex{Koeln@\textbf{Köln}|pwk}.}}}\label{K_L00527_1h} und heiteres Wiederſehn! Für den \textsc{Griensteidl}\oindex{Cafe Griensteidl@\textbf{Café Griensteidl}|pw} bin ich zu müde, ich ſchlafe ſo ſehr wenig und muß oft früh heraus. Ganz
               niedergeſchlagen hat mich in dieſen Tagen Hauptmann\pwindex{Hauptmann, Gerhart 15.11.1862 – 06.06.1946@\textsc{Hauptmann, Gerhart} (15.11.1862 – 06.06.1946), \emph{Schriftsteller}|pw}’s \label{K_L00527_2v}\edtext{Mißerfolg\pwindex{Hauptmann, Gerhart 15.11.1862 – 06.06.1946@\textsc{Hauptmann, Gerhart} (15.11.1862 – 06.06.1946), \emph{Schriftsteller}!Florian Geyer. Die Tragoedie des Bauernkrieges1896 – 1896@\strich\emph{Florian Geyer. Die Tragödie des Bauernkrieges} {[}1896 – 1896{]}|pwv}}{\lemma{\textnormal{\emph{Mißerfolg}}}\Cendnote{\textnormal{Die Uraufführung von \emph{Florian Geyer}\pwindex{Hauptmann, Gerhart 15.11.1862 – 06.06.1946@\textsc{Hauptmann, Gerhart} (15.11.1862 – 06.06.1946), \emph{Schriftsteller}!Florian Geyer. Die Tragoedie des Bauernkrieges1896 – 1896@\strich\emph{Florian Geyer. Die Tragödie des Bauernkrieges} {[}1896 – 1896{]}|pwk} fand am 4. 1. 1896 im Deutschen Theater in Berlin\oindex{Deutsches Theater Berlin@\textbf{Deutsches Theater Berlin}|pwk} statt.}}}\label{K_L00527_2h}, er ſelbſt iſt
               total herunter, nach den Berlin\oindex{Berlin@\textbf{Berlin}|pw}er Briefen zu
               urtheilen. Und gerade jetzt hatte er einen großen Sieg ſo nöthig. Da \label{K_L00527_3v}\edtext{\textsc{Halbe}\pwindex{Halbe, Max 04.10.1865 – 30.11.1944@\textsc{Halbe, Max} (04.10.1865 – 30.11.1944), \emph{Schriftsteller}|pw}\pwindex{Halbe, Max 04.10.1865 – 30.11.1944@\textsc{Halbe, Max} (04.10.1865 – 30.11.1944), \emph{Schriftsteller}!Lebenswende1896@\strich\emph{Lebenswende} {[}1896{]}|pwv}}{\lemma{\textnormal{\emph{Halbe}}}\Cendnote{\textnormal{\emph{Lebenswende}\pwindex{Halbe, Max 04.10.1865 – 30.11.1944@\textsc{Halbe, Max} (04.10.1865 – 30.11.1944), \emph{Schriftsteller}!Lebenswende1896@\strich\emph{Lebenswende} {[}1896{]}|pwk} hatte am 21. 1. 1896 im
                     Deutschen Theater\oindex{Deutsches Theater Berlin@\textbf{Deutsches Theater Berlin}|pwk} Uraufführung.}}}\label{K_L00527_3h} ihm
               zunächſt folgt, wird die \textsc{Liebelei}\pwindex{Schnitzler, Arthur 15.05.1862 – 21.10.1931@\textsc{Schnitzler, Arthur} (15.05.1862 – 21.10.1931), \emph{Schriftsteller, Mediziner}!Liebelei. Schauspiel in drei Akten9. 10. 1895@\strich\emph{Liebelei. Schauspiel in drei Akten} {[}9. 10. 1895{]}|pw} alſo in den \label{K_L00527_4v}\edtext{Februar}{\lemma{\textnormal{\emph{Februar}}}\Cendnote{\textnormal{Die Berlin\oindex{Berlin@\textbf{Berlin}|pwk}er Premiere fand am 4. 2. 1896 im Deutschen Theater\oindex{Deutsches Theater Berlin@\textbf{Deutsches Theater Berlin}|pwk} statt.}}}\label{K_L00527_4h} fallen, ſolange kann ich wohl
               nicht hier bleiben, obſchon ich gern bliebe.\pend
           \pstart
           Grüßen Sie in Frankfurt\oindex{Frankfurt am Main@\textbf{Frankfurt am Main}|pw}{ }\textsc{Goldmann}\pwindex{Goldmann, Paul 31.01.1865 – 25.09.1935@\textsc{Goldmann, Paul} (31.01.1865 – 25.09.1935), \emph{Schriftsteller, Journalist}|pw}’s \label{K_L00527_5v}\edtext{Schwager\pwindex{Mamroth, Fedor 21.02.1851 – 25.06.1907@\textsc{Mamroth, Fedor} (21.02.1851 – 25.06.1907), \emph{Journalist, Kritiker}|pwv}}{\lemma{\textnormal{\emph{Schwager}}}\Cendnote{\textnormal{Der Mediziner Josef Rosengart\pwindex{Rosengart, Josef 1860-02-08 – 1927-08-04@\textsc{Rosengart, Josef} (1860-02-08 – 1927-08-04), \emph{Arzt}|pwk}, der Mann der Schwester Vally\pwindex{Rosengart, Vally *~1866-12-29@\textsc{Rosengart, Vally} (*~1866-12-29)|pwk}}}}\label{K_L00527_5h}.\pend
           \pstart \spacefill\mbox{LouAS.}\pend{}\endnumbering\briefempfaengerindex{Schnitzler, Arthur@\textsc{Schnitzler, Arthur}!zzzAndreas-Salome, Lou@\emph{von Lou Andreas-Salomé}!1896-01-091@{{[}9. 1. 1896{]}}|)be}\mylabel{h}\end{ledgroupsized}  \newcommand{\dateiname}{L00527}\newcommand{\titel}{Lou Andreas-Salomé an Arthur Schnitzler, [9. 1. 1896]}\newcommand{\editorInnen}{Martin Anton Müller und Gerd-Hermann Susen}%% latex-leseansicht-abspann.tex
%% Abspann für die Leseansicht.
%% Der Schalter \ifkorrekturansicht ist bereits durch den Vorspann gesetzt.

%% latex-abspann.tex
%% Gemeinsamer Abspann für Korrekturansicht und Leseansicht.
%% Setzt den Schalter \ifkorrekturansicht voraus (gesetzt in den
%% einbindenden Dateien latex-korrekturansicht-abspann.tex bzw.
%% latex-leseansicht-abspann.tex).
%% ---------------------------------------------------------------

\normalsize

% Das esempio-Environment wird nur in der Leseansicht benötigt
\ifkorrekturansicht\else
\newenvironment{esempio}[3]%
{
    \vspace{1.5ex}
    \rlap{\underline{#1}}
    \par
    \setlength{\parindent}{0cm}
    \nopagebreak
    \leftskip=#2cm
    \rightskip=#3cm
}
{
    \par
}
\fi

\doendnotes{C}
\bigskip
\vfill

\clearpage

\footnotesize

\ifkorrekturansicht
  \lohead{\textsc{register}}
\fi

% theindex-Environment neu definieren ohne reledmac
\makeatletter
\renewenvironment{theindex}{%
  \ifkorrekturansicht
    \section*{\indexname}%
  \else
    \subsubsection*{Index der erwähnten Entitäten}%
  \fi
  \setlength{\parindent}{0pt}%
  \setlength{\parskip}{0pt plus 0.3pt}%
  \let\item\@idxitem
}{%
  \ifkorrekturansicht\clearpage\fi
}
\makeatother

\IfFileExists{\jobname-pw.ind}{\input{\jobname-pw.ind}}{}

% Quellenangabe nur in der Leseansicht
\ifkorrekturansicht\else
% Fallback-Definitionen, falls die .tex-Datei \titel etc. nicht gesetzt hat
\providecommand{\titel}{}
\providecommand{\editorInnen}{}
\providecommand{\dateiname}{\jobname}

\vspace{3cm}

\vfill

\footnotesize
\textsc{Quelle}: \titel. Herausgegeben von {\editorInnen}. In: \emph{Arthur Schnitzler: Briefwechsel mit Autorinnen und Autoren}.
 Digitale Edition, https://schnitzler-briefe.acdh.oeaw.ac.at/{\dateiname}.html (Stand \today)
\fi

\end{document}


      