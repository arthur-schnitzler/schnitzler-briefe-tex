%% latex-korrekturansicht-vorspann.tex
%% Vorspann für die Korrekturansicht.
%% Lädt die gemeinsame Datei latex-vorspann.tex mit gesetztem Schalter.

\newif\ifkorrekturansicht
\korrekturansichttrue

\input{../tex-inputs/latex-vorspann}


\section[ Felix Salten an Arthur Schnitzler, 8. 3. 1904]{L03393 Felix Salten an Arthur Schnitzler, 8. 3. 1904}
\nopagebreak\mylabel{L03393v}
\rehead{ }\normalsize\beginnumbering\briefempfaengerindex{Schnitzler, Arthur@\textsc{Schnitzler, Arthur}!zzzSalten, Felix@\emph{von Felix Salten}!1904-03-081@{8. 3. 1904}|(be}
\toendnotes[C]{\smallbreak\pagebreak[2]}\Standort{CUL, Schnitzler, B 89, B 1.}
\physDesc{Bildpostkarte, 104 Zeichen
\newline{}Handschrift: schwarze Tinte, lateinische Kurrent
\newline{}Versand: 1) Stempel: »\nobreak{}\oindex{Kairo@\textbf{Kairo}, \emph{P.PPLC}|pwk}Cairo\nobreak{}«.   2) Stempel: »\nobreak{}\oindex{XVIII., Waehring@\textbf{XVIII., Währing}, \emph{A.ADM3}|pwk}18/1 Wien 110, 1\textcolor{gray}{6}. 3. \textcolor{gray}{0}4, 8, Bestellt\nobreak{}«. 
\newline{}Ordnung: mit Bleistift von unbekannter Hand nummeriert: »185« }\pstart{}{\pb}M\textsuperscript{r} D\textsuperscript{r} Arthur Schnitzler\pend{}\pstart{}XVIII. Spöttelgaße 7\oindex{Edmund-Weiss-Gasse 7@\textbf{Edmund-Weiß-Gasse 7}, \emph{Wohngebäude (K.WHS)}|pw}\pend{}\pstart{}Wien\oindex{Wien@\textbf{Wien}, \emph{A.ADM2}|pw}\pend{}\pstart{}\begin{otherlanguage}{french}Autriche\oindex{Oesterreich@\textbf{Österreich}, \emph{A.PCLI}|pw}\end{otherlanguage}\pend{}\pstart{}Europe\oindex{Europa@\textbf{Europa}, \emph{Kontinent (A.KNT)}|pw}\pend{}{\bigskip}
\pstart
           {\pb}\textcolor{gray}{\textbf{LE CAIRE\oindex{Kairo@\textbf{Kairo}, \emph{P.PPLC}|pw}}}\hfill \textcolor{gray}{\textbf{La Citadelle\oindex{Zitadelle von Saladin@\textbf{Zitadelle von Saladin}, \emph{Monument (K.MON)}|pw}}}\pend
           \vspace{1em}
\pstart
           {\pb}Cairo\oindex{Kairo@\textbf{Kairo}, \emph{P.PPLC}|pw}, 8. III. 04.\pend
           \vspace{0.5em}
\pstart
           Wunderschön!! herzl. Ihr \spacefill\mbox{F S.}\pend
           \selectlanguage{ngerman}\endnumbering\briefempfaengerindex{Schnitzler, Arthur@\textsc{Schnitzler, Arthur}!zzzSalten, Felix@\emph{von Felix Salten}!1904-03-081@{8. 3. 1904}|)be}\mylabel{L03393h}  \normalsize

\doendnotes{C}
\bigskip
\vfill

\clearpage

\footnotesize

\lohead{\textsc{register}}

% Definiere theindex-Environment komplett neu ohne reledmac
\makeatletter
\renewenvironment{theindex}{%
  \section*{\indexname}%
  \setlength{\parindent}{0pt}%
  \setlength{\parskip}{0pt plus 0.3pt}%
  \let\item\@idxitem
}{%
  \clearpage
}
\makeatother

\IfFileExists{\jobname-pw.ind}{\input{\jobname-pw.ind}}{}

\end{document}

      