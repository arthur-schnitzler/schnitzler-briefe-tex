%% latex-korrekturansicht-vorspann.tex
%% Vorspann für die Korrekturansicht.
%% Lädt die gemeinsame Datei latex-vorspann.tex mit gesetztem Schalter.

\newif\ifkorrekturansicht
\korrekturansichttrue

\input{../tex-inputs/latex-vorspann}


\section[Hugo von Hofmannsthal an Arthur Schnitzler, {[}26. 6. 1911{]}]{L02024 Hugo von Hofmannsthal an Arthur Schnitzler, {[}26. 6. 1911{]}}
\nopagebreak\mylabel{L02024v}
\rehead{ }\normalsize\beginnumbering\briefempfaengerindex{Schnitzler, Arthur@\textsc{Schnitzler, Arthur}!zzzHofmannsthal, Hugo von@\emph{von Hugo von Hofmannsthal}!1911-06-261@{{[}26. 6. 1911{]}}|(be}
\toendnotes[C]{\smallbreak\pagebreak[2]}\Standort{CUL, Schnitzler, B 43.}
\physDesc{Briefkarte, 732 Zeichen
\newline{}Handschrift: schwarze Tinte, deutsche Kurrent
\newline{}Schnitzler: mit Bleistift datiert: »26/6 911« und beschriftet: »\textsc{Hugo}« 
\newline{}Ordnung: 1) mit Bleistift von unbekannter Hand nummeriert: »\strikeout{322}«  2) mit Bleistift von unbekannter Hand nummeriert:
                                    »331«}
\buchAbdrucke{\weitereDrucke{Hugo von Hofmannsthal, Arthur Schnitzler: \emph{Briefwechsel}. Frankfurt am Main: \emph{S. Fischer} 1964, S. 262.} }
\pstart
           \raggedleft{}{\pb}Montag\pend
           
\pstart{}mein lieber Arthur\pend\vspace{0.5em}
\pstart
           ich will unbedingt auf den Se{\geminationm}ering\oindex{Semmering@\textbf{Semmering}, \emph{A.ADM3}|pw} hinauf, dort 2 Tage mit Ihnen verbringen.
               Es iſt ein freundlicher Gebrauch, daſs man gleichzeitig auf der Welt iſt und man ſoll
               daran möglichſt feſthalten.\pend
           
\pstart
           Aber Schönherr\pwindex{Schoenherr, Karl 24.02.1867 – 15.03.1943@\textsc{Schönherr, Karl} (24.02.1867 – 15.03.1943), \emph{Schriftsteller/Schriftstellerin, Mediziner/Medizinerin}|pw} iſt mir ausgeſucht fatal, mit
               ihm näher bekannt werden, bei Mahlzeiten {\pb}zuſammenſitzen u. ſ. f. ein
                  wirklich\strikeout{er} kaum erträglicher Gedanke.\hspace*{1.5em}Überhaupt werden mir Litteraten immer
                  bedenklicher.\hspace*{1.5em}Aber er ko{\geminationm}t wohl auch nur für 1–2 Tage hinauf, ko{\geminationm}t vielleicht gar nicht. Bitte depeſchieren Sie mir
               darüber ſpäteſtens Mittwoch vormittag näheres. Eventuell können ſehr
               wohl Sie oder Brahm\pwindex{Brahm, Otto 05.02.1856 – 28.11.1912@\textsc{Brahm, Otto} (05.02.1856 – 28.11.1912), \emph{Theaterleiter/Theaterleiterin, Regisseur/Regisseurin}|pw} bei ihm telegrafiſch nach
               ſeinen Abſichten anfragen – »behufs Einteilung anderer Beſuche.« \pend
           
\pstart
           Alſo auf bald, hoffentlich. Ihr alter \spacefill\mbox{Hugo.}\pend
           \selectlanguage{ngerman}\endnumbering\briefempfaengerindex{Schnitzler, Arthur@\textsc{Schnitzler, Arthur}!zzzHofmannsthal, Hugo von@\emph{von Hugo von Hofmannsthal}!1911-06-261@{{[}26. 6. 1911{]}}|)be}\mylabel{L02024h}  \normalsize

\doendnotes{C}
\bigskip
\vfill

\clearpage

\footnotesize

\lohead{\textsc{register}}

% Definiere theindex-Environment komplett neu ohne reledmac
\makeatletter
\renewenvironment{theindex}{%
  \section*{\indexname}%
  \setlength{\parindent}{0pt}%
  \setlength{\parskip}{0pt plus 0.3pt}%
  \let\item\@idxitem
}{%
  \clearpage
}
\makeatother

\IfFileExists{\jobname-pw.ind}{\input{\jobname-pw.ind}}{}

\end{document}

      