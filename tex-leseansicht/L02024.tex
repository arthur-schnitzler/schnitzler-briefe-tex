%% latex-leseansicht-vorspann.tex
%% Vorspann für die Leseansicht.
%% Lädt die gemeinsame Datei latex-vorspann.tex mit nicht gesetztem Schalter.

\newif\ifkorrekturansicht
\korrekturansichtfalse

\input{../tex-inputs/latex-vorspann}


         
         \newcommand{\erwaehntePersonen}{Personen: Otto Brahm, Karl Schönherr}
         \newcommand{\erwaehnteInstitutionen}{}
         \newcommand{\erwaehnteOrte}{Orte: Semmering, Wien}
         \newcommand{\erwaehnteWerke}{
               \section[Hugo von Hofmannsthal an Arthur Schnitzler, {[}26. 6. 1911{]}]{ Hugo von Hofmannsthal an Arthur Schnitzler, {[}26. 6. 1911{]}}\nopagebreak\mylabel{v}\rehead{ }\begin{ledgroupsized}[t]{13cm}\normalsize\beginnumbering \toendnotes[C]{\smallbreak\pagebreak[2]} \Standort{CUL, Schnitzler, B 43.}
\physDesc{Briefkarte
\newline{}Handschrift: schwarze Tinte, deutsche Kurrent
\newline{}Schnitzler: mit Bleistift datiert: »26/6 911« und beschriftet: »\textsc{Hugo}« \newline{}Ordnung: 1) mit Bleistift von unbekannter Hand nummeriert: »\strikeout{322}«  2) mit Bleistift von unbekannter Hand nummeriert: »331«}\buchAbdrucke{\weitereDrucke{Hugo von Hofmannsthal, Arthur Schnitzler: \emph{Briefwechsel}. Hg. Therese Nickl und Heinrich Schnitzler. Frankfurt am Main: \emph{S. Fischer} 1964, S. 262.} }\pstart
           \raggedleft{}{\pb}Montag\pend
           \pstart{}mein lieber Arthur\pend\pstart
           ich will unbedingt auf den Se{\geminationm}ering\oindex{Semmering@\textbf{Semmering}|pw} hinauf, dort 2 Tage mit Ihnen verbringen. Es iſt ein
               freundlicher Gebrauch, daſs man gleichzeitig auf der Welt iſt und man ſoll daran
               möglichſt feſthalten.\pend
           \pstart
           Aber Schönherr\pwindex{Schoenherr, Karl 24.02.1867 – 15.03.1943@\textsc{Schönherr, Karl} (24.02.1867 – 15.03.1943), \emph{Schriftsteller, Mediziner}|pw} iſt mir ausgeſucht fatal, mit ihm
               näher bekannt werden, bei Mahlzeiten {\pb}zuſammenſitzen u. ſ. f. ein
                  wirklich\strikeout{er} kaum erträglicher Gedanke.\hspace*{1.5em}Überhaupt werden mir Litteraten immer
                  bedenklicher.\hspace*{1.5em}Aber er ko{\geminationm}t wohl auch nur für 1–2 Tage hinauf, ko{\geminationm}t vielleicht gar nicht. Bitte depeſchieren Sie mir
               darüber ſpäteſtens Mittwoch vormittag näheres. Eventuell können ſehr
               wohl Sie oder Brahm\pwindex{Brahm, Otto 05.02.1856 – 28.11.1912@\textsc{Brahm, Otto} (05.02.1856 – 28.11.1912), \emph{Theaterleiter, Regisseur}|pw} bei ihm telegrafiſch nach
               ſeinen Abſichten anfragen – »behufs Einteilung anderer Beſuche.« \pend
           \pstart
           Alſo auf bald, hoffentlich. Ihr alter \spacefill\mbox{Hugo.}\pend
           
         
         \endnumbering\mylabel{h}\end{ledgroupsized}  \newcommand{\dateiname}{L02024}\newcommand{\titel}{Hugo von Hofmannsthal an Arthur Schnitzler, [26. 6. 1911]}\newcommand{\editorInnen}{Martin Anton Müller und Gerd-Hermann Susen}%% latex-leseansicht-abspann.tex
%% Abspann für die Leseansicht.
%% Der Schalter \ifkorrekturansicht ist bereits durch den Vorspann gesetzt.

%% latex-abspann.tex
%% Gemeinsamer Abspann für Korrekturansicht und Leseansicht.
%% Setzt den Schalter \ifkorrekturansicht voraus (gesetzt in den
%% einbindenden Dateien latex-korrekturansicht-abspann.tex bzw.
%% latex-leseansicht-abspann.tex).
%% ---------------------------------------------------------------

\normalsize

% Das esempio-Environment wird nur in der Leseansicht benötigt
\ifkorrekturansicht\else
\newenvironment{esempio}[3]%
{
    \vspace{1.5ex}
    \rlap{\underline{#1}}
    \par
    \setlength{\parindent}{0cm}
    \nopagebreak
    \leftskip=#2cm
    \rightskip=#3cm
}
{
    \par
}
\fi

\doendnotes{C}
\bigskip
\vfill

\clearpage

\footnotesize

\ifkorrekturansicht
  \lohead{\textsc{register}}
\fi

% theindex-Environment neu definieren ohne reledmac
\makeatletter
\renewenvironment{theindex}{%
  \ifkorrekturansicht
    \section*{\indexname}%
  \else
    \subsubsection*{Index der erwähnten Entitäten}%
  \fi
  \setlength{\parindent}{0pt}%
  \setlength{\parskip}{0pt plus 0.3pt}%
  \let\item\@idxitem
}{%
  \ifkorrekturansicht\clearpage\fi
}
\makeatother

\IfFileExists{\jobname-pw.ind}{\input{\jobname-pw.ind}}{}

% Quellenangabe nur in der Leseansicht
\ifkorrekturansicht\else
% Fallback-Definitionen, falls die .tex-Datei \titel etc. nicht gesetzt hat
\providecommand{\titel}{}
\providecommand{\editorInnen}{}
\providecommand{\dateiname}{\jobname}

\vspace{3cm}

\vfill

\footnotesize
\textsc{Quelle}: \titel. Herausgegeben von {\editorInnen}. In: \emph{Arthur Schnitzler: Briefwechsel mit Autorinnen und Autoren}.
 Digitale Edition, https://schnitzler-briefe.acdh.oeaw.ac.at/{\dateiname}.html (Stand \today)
\fi

\end{document}


      