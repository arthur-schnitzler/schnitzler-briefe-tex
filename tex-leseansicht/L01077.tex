%% latex-korrekturansicht-vorspann.tex
%% Vorspann für die Korrekturansicht.
%% Lädt die gemeinsame Datei latex-vorspann.tex mit gesetztem Schalter.

\newif\ifkorrekturansicht
\korrekturansichttrue

\input{../tex-inputs/latex-vorspann}


\section[Hermann Bahr an Arthur Schnitzler, 12. 10. 1900]{L01077 Hermann Bahr an Arthur Schnitzler, 12. 10. 1900}
\nopagebreak\mylabel{L01077v}
\rehead{ }\normalsize\beginnumbering\briefempfaengerindex{Schnitzler, Arthur@\textsc{Schnitzler, Arthur}!zzzBahr, Hermann@\emph{von Hermann Bahr}!1900-10-121@{12. 10. 1900}|(be}
\toendnotes[C]{\smallbreak\pagebreak[2]}\Standort{CUL, Schnitzler, B 5b.}
\physDesc{Kartenbrief, 495 Zeichen
\newline{}Handschrift: schwarze Tinte, deutsche Kurrent
\newline{}Versand: 1) Stempel: »\nobreak{}\oindex{XIII., Hietzing@\textbf{XIII., Hietzing}, \emph{A.ADM3}|pwk}Wien 13/7, 12{[}.{]} 10. 00, 10–11 V\nobreak{}«.   2) Stempel: »\nobreak{}12. 10. 00, 3.N\nobreak{}«. 
\newline{}Schnitzler: mit Bleistift Jahreszahl ergänzt: »900« 
\newline{}Ordnung: mit Bleistift von unbekannter Hand nummeriert:
                                    »69« }
\buchAbdrucke{\weitereDrucke{Hermann Bahr, Arthur Schnitzler: \emph{Briefwechsel, Aufzeichnungen, Dokumente (1891–1931)}. Göttingen: \emph{Wallstein} 2018, S. 182.} }\pstart{}{\pb}Herrn \textsc{D\textsuperscript{r} Arthur Schnitzler}\pend{}\pstart{}\textsc{Wien IX}\oindex{IX., Alsergrund@\textbf{IX., Alsergrund}, \emph{A.ADM3}|pw}\pend{}\pstart{}Frankgaſſe 1\oindex{Frankgasse 1@\textbf{Frankgasse 1}, \emph{Wohngebäude (K.WHS)}|pw}\pend{}{\bigskip}\vspace{1em}
\pstart
           \raggedleft{}{\pb}12/10\pend
           
\pstart{}Lieber Arthur!\pend\vspace{0.5em}
\pstart
           Danke ſehr für Deine Zeilen. Natürlich habe ich eine große Freude, etwas Neues von
               Dir vorleſen zu können, und erwarte mit Ungeduld das \textsc{Manuscript}. Mit Dir nächstens einmal reden zu können verlangt mich ſehr, um
               Dir zu ſagen, wie menſchlich tief mich, bei manchen Bedenken des Theatermannes, Deine
                  Beatrice\pwindex{Schleier der Beatrice. Schauspiel in fuenf Akten@\emph{Der Schleier der Beatrice. Schauspiel in fünf Akten}|pw} berührt hat: ſie iſt mir weitaus das
               Liebſte, was \introOben{}Du\introOben{} noch geſchaffen, und hat mich völlig zu Dir
               hingeriſſen.\pend
           
\pstart
           Herzlichſt Dein{\\[\baselineskip]}\spacefill\mbox{Hermann}\pend
           \leftskip=0em{}\selectlanguage{ngerman}\endnumbering\briefempfaengerindex{Schnitzler, Arthur@\textsc{Schnitzler, Arthur}!zzzBahr, Hermann@\emph{von Hermann Bahr}!1900-10-121@{12. 10. 1900}|)be}\mylabel{L01077h}  \normalsize

\doendnotes{C}
\bigskip
\vfill

\clearpage

\footnotesize

\lohead{\textsc{register}}

% Definiere theindex-Environment komplett neu ohne reledmac
\makeatletter
\renewenvironment{theindex}{%
  \section*{\indexname}%
  \setlength{\parindent}{0pt}%
  \setlength{\parskip}{0pt plus 0.3pt}%
  \let\item\@idxitem
}{%
  \clearpage
}
\makeatother

\IfFileExists{\jobname-pw.ind}{\input{\jobname-pw.ind}}{}

\end{document}

      