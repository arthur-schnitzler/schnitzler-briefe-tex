%% latex-korrekturansicht-vorspann.tex
%% Vorspann für die Korrekturansicht.
%% Lädt die gemeinsame Datei latex-vorspann.tex mit gesetztem Schalter.

\newif\ifkorrekturansicht
\korrekturansichttrue

\input{../tex-inputs/latex-vorspann}


\section[Hermann Bahr an Arthur Schnitzler, 16. 3. 1907]{L01664 Hermann Bahr an Arthur Schnitzler, 16. 3. 1907}
\nopagebreak\mylabel{L01664v}
\rehead{ }\normalsize\beginnumbering\briefempfaengerindex{Schnitzler, Arthur@\textsc{Schnitzler, Arthur}!zzzBahr, Hermann@\emph{von Hermann Bahr}!1907-03-161@{16. 3. 1907}|(be}
\toendnotes[C]{\smallbreak\pagebreak[2]}\Standort{CUL, Schnitzler, B 5b.}
\physDesc{Kartenbrief, 569 Zeichen
\newline{}Handschrift: schwarze Tinte, deutsche Kurrent
\newline{}Versand: 1) Stempel: »\nobreak{}\oindex{Berlin@\textbf{Berlin}, \emph{P.PPLC}|pwk}Berlin. N.W., 16. 3. 07, 8–9N\nobreak{}«.   2) Stempel: »\nobreak{}Bestellt, \oindex{XVIII., Waehring@\textbf{XVIII., Währing}, \emph{A.ADM3}|pwk}18/1 Wien, 18. 3. 07, 9\nobreak{}«. 
\newline{}Ordnung: mit Bleistift von unbekannter Hand nummeriert:
                                    »145« }
\buchAbdrucke{\weitereDrucke{Hermann Bahr, Arthur Schnitzler: \emph{Briefwechsel, Aufzeichnungen, Dokumente (1891–1931)}. Göttingen: \emph{Wallstein} 2018, S. 390.} }\toendnotes[C]{\smallbreak}\pstart{}{\pb}Herrn \textsc{D\textsuperscript{r} Artur Schnitzler}\pend{}\pstart{}\textsc{Wien XVIII\oindex{XVIII., Waehring@\textbf{XVIII., Währing}, \emph{A.ADM3}|pw}}\pend{}\pstart{}\textsc{Spöttelgasse 7\oindex{Edmund-Weiss-Gasse 7@\textbf{Edmund-Weiß-Gasse 7}, \emph{Wohngebäude (K.WHS)}|pw}}\pend{}{\bigskip}\vspace{1em}
\pstart
           \raggedleft{}{\pb}16. 3. 07\pend
           
\pstart{}Lieber Artur!\pend\vspace{0.5em}
\pstart
           »Liebelei\pwindex{Liebelei. Schauspiel in drei Akten@\emph{Liebelei. Schauspiel in drei Akten}|pw}« ging im letzten Moment nicht, weil
               wir abſolut keine Mizzi
                  Schlager\pwindex{Liebelei. Schauspiel in drei Akten@\emph{Liebelei. Schauspiel in drei Akten}|pwv} hatten (da \textsc{Durieux}\pwindex{Durieux, Tilla 18.08.1880 – 20.02.1971@\textsc{Durieux, Tilla} (18.08.1880 – 20.02.1971), \emph{Schauspieler/Schauspielerin}|pw} gleichzeitg im Deutſchen\oindex{Deutsches Theater Berlin@\textbf{Deutsches Theater Berlin}, \emph{Theater (K.THE)}|pw} unentbehrlich).
               Dafür mache ich jetzt »\label{K_L01664-1v}\edtext{Comödie der Liebe\pwindex{Komoedie der Liebe@\emph{Komödie der Liebe}|pw}}{\lemma{\textnormal{\emph{Comödie der Liebe}}}\Cendnote{\textnormal{Die Premiere der \emph{Komödie der Liebe}\pwindex{Komoedie der Liebe@\emph{Komödie der Liebe}|pwk} von Ibsen\pwindex{Ibsen, Henrik 20.03.1828 – 23.05.1906@\textsc{Ibsen, Henrik} (20.03.1828 – 23.05.1906), \emph{Schriftsteller/Schriftstellerin}|pwk} am 25. 3. 1907 in den \emph{Kammerspielen des Deutschen Theaters}\orgindex{Kammerspiele Berlin@Kammerspiele Berlin|pwk}. Das Regiebuch findet sich in Bahrs\pwindex{Bahr, Hermann 19.07.1863 – 15.01.1934@\textsc{Bahr, Hermann} (19.07.1863 – 15.01.1934), \emph{Schriftsteller/Schriftstellerin, Kritiker/Kritikerin}|pwk} Nachlass (\emph{Theatermuseum Wien}, VM 3684 Ba).}}}\label{K_L01664-1}«.
               Hoffentlich kommts im Herbſt zur L.\pwindex{Liebelei. Schauspiel in drei Akten@\emph{Liebelei. Schauspiel in drei Akten}|pwv}, was ich ſchon wegen der Höflich\pwindex{Hoeflich, Lucie 20.02.1883 – 08.10.1956@\textsc{Höflich, Lucie} (20.02.1883 – 08.10.1956), \emph{Schauspieler/Schauspielerin}|pw}{ }ſehr möchte. Wegen »Märchen\pwindex{Maerchen. Schauspiel in drei Aufzuegen@\emph{Das Märchen. Schauspiel in drei Aufzügen}|pw}« ſprach ich mit Reinhardt\pwindex{Reinhardt, Max 09.09.1873 – 30.10.1943@\textsc{Reinhardt, Max} (09.09.1873 – 30.10.1943), \emph{Theaterleiter/Theaterleiterin, Regisseur/Regisseurin, Schauspieler/Schauspielerin}|pw},
               aber da wird man lang und viel bohren müſſen.\pend
           
\pstart
           Anfang April bin ich wieder in Wien\oindex{Wien@\textbf{Wien}, \emph{A.ADM2}|pw} und hab Euch\pwindex{Schnitzler, Olga 17.01.1882 – 13.01.1970@\textsc{Schnitzler, Olga} (17.01.1882 – 13.01.1970), \emph{Schauspieler/Schauspielerin, Sänger/Sängerin}|pwv} viel von hier zu erzälen,
               wo doch \label{LL306-1v}alles, faſt alles ganz famos\label{LL306-1h}
               ist.\pend
           
\pstart
           Herzlichſt{\\[\baselineskip]}mit vielen Grüßen an Deine Frau\pwindex{Schnitzler, Olga 17.01.1882 – 13.01.1970@\textsc{Schnitzler, Olga} (17.01.1882 – 13.01.1970), \emph{Schauspieler/Schauspielerin, Sänger/Sängerin}|pwv}{ }\spacefill\mbox{Hermann}\pend
           \leftskip=0em{}\selectlanguage{ngerman}\endnumbering\briefempfaengerindex{Schnitzler, Arthur@\textsc{Schnitzler, Arthur}!zzzBahr, Hermann@\emph{von Hermann Bahr}!1907-03-161@{16. 3. 1907}|)be}\mylabel{L01664h}  \normalsize

\doendnotes{C}
\bigskip
\vfill

\clearpage

\footnotesize

\lohead{\textsc{register}}

% Definiere theindex-Environment komplett neu ohne reledmac
\makeatletter
\renewenvironment{theindex}{%
  \section*{\indexname}%
  \setlength{\parindent}{0pt}%
  \setlength{\parskip}{0pt plus 0.3pt}%
  \let\item\@idxitem
}{%
  \clearpage
}
\makeatother

\IfFileExists{\jobname-pw.ind}{\input{\jobname-pw.ind}}{}

\end{document}

      