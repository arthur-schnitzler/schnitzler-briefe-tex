%% latex-leseansicht-vorspann.tex
%% Vorspann für die Leseansicht.
%% Lädt die gemeinsame Datei latex-vorspann.tex mit nicht gesetztem Schalter.

\newif\ifkorrekturansicht
\korrekturansichtfalse

\input{../tex-inputs/latex-vorspann}


\section[Hermann Bahr an Arthur Schnitzler, 16. 3. 1907]{L01664 Hermann Bahr an Arthur Schnitzler, 16. 3. 1907}
\nopagebreak\mylabel{L01664v}
\rehead{ }\normalsize\beginnumbering\briefempfaengerindex{Schnitzler, Arthur@\textsc{Schnitzler, Arthur}!zzzBahr, Hermann@\emph{von Hermann Bahr}!1907-03-161@{16. 3. 1907}|(be}
\toendnotes[C]{\smallbreak\pagebreak[2]}
\correspDesc{Versand  durch Hermann Bahr am 16. 3. 1907 in Berlin
\newline{}Erhalt  durch Arthur Schnitzler am 18. 3. 07 in Wien}\toendnotes[C]{\smallbreak}
\Standort{CUL, Schnitzler, B 5b.}
\physDesc{Kartenbrief, 569 Zeichen
\newline{}Handschrift: schwarze Tinte, deutsche Kurrent
\newline{}Versand: 1) Stempel: »\nobreak{}\oindex{Berlin@\textbf{Berlin}, \emph{Hauptstadt}|pwk}Berlin. N.W., 16. 3. 07, 8–9N\nobreak{}«.   2) Stempel: »\nobreak{}Bestellt, \oindex{XVIII., Währing@\textbf{XVIII., Währing}, \emph{Verwaltungsgebiet}|pwk}18/1 Wien, 18. 3. 07, 9\nobreak{}«. 
\newline{}Ordnung: mit Bleistift von unbekannter Hand nummeriert:
                                    »145« }
\buchAbdrucke{\weitereDrucke{Hermann Bahr, Arthur Schnitzler: \emph{Briefwechsel, Aufzeichnungen, Dokumente (1891–1931)}. Herausgegeben von Kurt Ifkovits und Martin Anton Müller. Göttingen: \emph{Wallstein} 2018, S. 390.} }\toendnotes[C]{\smallbreak}\pstart{}{\pb}Herrn \textsc{D\textsuperscript{r} Artur Schnitzler}\pend{}\pstart{}\textsc{Wien XVIII\oindex{XVIII., Währing@\textbf{XVIII., Währing}, \emph{Verwaltungsgebiet}|pw}}\pend{}\pstart{}\textsc{Spöttelgasse 7\oindex{Wien@\textbf{Wien}!XVIII., Währing@\textbf{XVIII., Währing}!Edmund-Weiß-Gasse 7@\textbf{Edmund-Weiß-Gasse 7}, \emph{Wohngebäude}|pw}}\pend{}{\bigskip}\vspace{1em}
\pstart
           \raggedleft{}{\pb}16. 3. 07\pend
           
\pstart{}Lieber Artur!\pend\vspace{0.5em}
\pstart
           »Liebelei\pwindex{Schnitzler, Arthur 15.\,5.\,1862 Wien – 21.\,10.\,1931 ebd.@\textsc{Schnitzler, Arthur} (15.\,5.\,1862 Wien – 21.\,10.\,1931 ebd.), \emph{Schriftsteller, Mediziner}!Liebelei. Schauspiel in drei Akten@\strich\emph{Liebelei. Schauspiel in drei Akten}|pw}« ging im letzten Moment nicht, weil
               wir abſolut keine Mizzi
                  Schlager\pwindex{Schnitzler, Arthur 15.\,5.\,1862 Wien – 21.\,10.\,1931 ebd.@\textsc{Schnitzler, Arthur} (15.\,5.\,1862 Wien – 21.\,10.\,1931 ebd.), \emph{Schriftsteller, Mediziner}!Liebelei. Schauspiel in drei Akten@\strich\emph{Liebelei. Schauspiel in drei Akten}|pwv} hatten (da \textsc{Durieux}\pwindex{Durieux, Tilla 18.\,8.\,1880 Wien – 20.\,2.\,1971 Berlin@\textsc{Durieux, Tilla} (18.\,8.\,1880 Wien – 20.\,2.\,1971 Berlin), \emph{Schauspielerin}|pw} gleichzeitg im Deutſchen\oindex{Deutsches Theater Berlin@\textbf{Deutsches Theater Berlin}, \emph{Theater}|pw} unentbehrlich).
               Dafür mache ich jetzt »\label{K_L01664-1v}\edtext{Comödie der Liebe\pwindex{Ibsen, Henrik 20.\,3.\,1828 Skien – 23.\,5.\,1906 Oslo@\textsc{Ibsen, Henrik} (20.\,3.\,1828 Skien – 23.\,5.\,1906 Oslo), \emph{Schriftsteller}!Komödie der Liebe@\strich\emph{Komödie der Liebe}|pw}}{\lemma{\textnormal{\emph{Comödie der Liebe}}}\Cendnote{\textnormal{Die Premiere der \emph{Komödie der Liebe}\pwindex{Ibsen, Henrik 20.\,3.\,1828 Skien – 23.\,5.\,1906 Oslo@\textsc{Ibsen, Henrik} (20.\,3.\,1828 Skien – 23.\,5.\,1906 Oslo), \emph{Schriftsteller}!Komödie der Liebe@\strich\emph{Komödie der Liebe}|pwk} von Ibsen\pwindex{Ibsen, Henrik 20.\,3.\,1828 Skien – 23.\,5.\,1906 Oslo@\textsc{Ibsen, Henrik} (20.\,3.\,1828 Skien – 23.\,5.\,1906 Oslo), \emph{Schriftsteller}|pwk} am 25. 3. 1907 in den \emph{Kammerspielen des Deutschen Theaters}\orgindex{Kammerspiele Berlin@Kammerspiele Berlin|pwk}. Das Regiebuch findet sich in Bahrs\pwindex{Bahr, Hermann 19.\,7.\,1863 Linz – 15.\,1.\,1934 München@\textsc{Bahr, Hermann} (19.\,7.\,1863 Linz – 15.\,1.\,1934 München), \emph{Schriftsteller, Kritiker}|pwk} Nachlass (\emph{Theatermuseum Wien}, VM 3684 Ba).}}}\label{K_L01664-1}«.
               Hoffentlich kommts im Herbſt zur L.\pwindex{Schnitzler, Arthur 15.\,5.\,1862 Wien – 21.\,10.\,1931 ebd.@\textsc{Schnitzler, Arthur} (15.\,5.\,1862 Wien – 21.\,10.\,1931 ebd.), \emph{Schriftsteller, Mediziner}!Liebelei. Schauspiel in drei Akten@\strich\emph{Liebelei. Schauspiel in drei Akten}|pwv}, was ich{ }ſchon wegen der Höflich\pwindex{Höflich, Lucie 20.\,2.\,1883 Hannover – 8.\,10.\,1956 Schmargendorf@\textsc{Höflich, Lucie} (20.\,2.\,1883 Hannover – 8.\,10.\,1956 Schmargendorf), \emph{Schauspielerin}|pw}{ }ſehr möchte. Wegen »Märchen\pwindex{Schnitzler, Arthur 15.\,5.\,1862 Wien – 21.\,10.\,1931 ebd.@\textsc{Schnitzler, Arthur} (15.\,5.\,1862 Wien – 21.\,10.\,1931 ebd.), \emph{Schriftsteller, Mediziner}!Märchen. Schauspiel in drei Aufzügen@\strich\emph{Das Märchen. Schauspiel in drei Aufzügen}|pw}«{ }ſprach ich mit Reinhardt\pwindex{Reinhardt, Max 9.\,9.\,1873 Baden bei Wien – 30.\,10.\,1943 New York City@\textsc{Reinhardt, Max} (9.\,9.\,1873 Baden bei Wien – 30.\,10.\,1943 New York City), \emph{Theaterleiter, Regisseur, Schauspieler}|pw},
               aber da wird man lang und viel bohren müſſen.\pend
           
\pstart
           Anfang April bin ich wieder in Wien\oindex{Wien@\textbf{Wien}, \emph{Verwaltungsgebiet}|pw} und hab Euch\pwindex{Schnitzler, Olga 17.\,1.\,1882 Wien – 13.\,1.\,1970 Lugano@\textsc{Schnitzler, Olga} (17.\,1.\,1882 Wien – 13.\,1.\,1970 Lugano), \emph{Schauspielerin, Sängerin}|pwv} viel von hier zu erzälen,
               wo doch \label{LL306-1v}alles, faſt alles ganz famos\label{LL306-1h}
               ist.\pend
           
\pstart
           Herzlichſt{\\[\baselineskip]}mit vielen Grüßen an Deine Frau\pwindex{Schnitzler, Olga 17.\,1.\,1882 Wien – 13.\,1.\,1970 Lugano@\textsc{Schnitzler, Olga} (17.\,1.\,1882 Wien – 13.\,1.\,1970 Lugano), \emph{Schauspielerin, Sängerin}|pwv}{ }\spacefill\mbox{Hermann}\pend
           \leftskip=0em{}\selectlanguage{ngerman}\endnumbering\briefempfaengerindex{Schnitzler, Arthur@\textsc{Schnitzler, Arthur}!zzzBahr, Hermann@\emph{von Hermann Bahr}!1907-03-161@{16. 3. 1907}|)be}\mylabel{L01664h}  \newcommand{\dateiname}{L01664}\newcommand{\titel}{Hermann Bahr an Arthur Schnitzler, 16. 3. 1907}\newcommand{\editorInnen}{Herausgegeben von Martin Anton Müller}%% latex-leseansicht-abspann.tex
%% Abspann für die Leseansicht.
%% Der Schalter \ifkorrekturansicht ist bereits durch den Vorspann gesetzt.

%% latex-abspann.tex
%% Gemeinsamer Abspann für Korrekturansicht und Leseansicht.
%% Setzt den Schalter \ifkorrekturansicht voraus (gesetzt in den
%% einbindenden Dateien latex-korrekturansicht-abspann.tex bzw.
%% latex-leseansicht-abspann.tex).
%% ---------------------------------------------------------------

\normalsize

% Das esempio-Environment wird nur in der Leseansicht benötigt
\ifkorrekturansicht\else
\newenvironment{esempio}[3]%
{
    \vspace{1.5ex}
    \rlap{\underline{#1}}
    \par
    \setlength{\parindent}{0cm}
    \nopagebreak
    \leftskip=#2cm
    \rightskip=#3cm
}
{
    \par
}
\fi

\doendnotes{C}
\bigskip
\vfill

\clearpage

\footnotesize

\ifkorrekturansicht
  \lohead{\textsc{register}}
\fi

% theindex-Environment neu definieren ohne reledmac
\makeatletter
\renewenvironment{theindex}{%
  \ifkorrekturansicht
    \section*{\indexname}%
  \else
    \subsubsection*{Index der erwähnten Entitäten}%
  \fi
  \setlength{\parindent}{0pt}%
  \setlength{\parskip}{0pt plus 0.3pt}%
  \let\item\@idxitem
}{%
  \ifkorrekturansicht\clearpage\fi
}
\makeatother

\IfFileExists{\jobname-pw.ind}{\input{\jobname-pw.ind}}{}

% Quellenangabe nur in der Leseansicht
\ifkorrekturansicht\else
% Fallback-Definitionen, falls die .tex-Datei \titel etc. nicht gesetzt hat
\providecommand{\titel}{}
\providecommand{\editorInnen}{}
\providecommand{\dateiname}{\jobname}

\vspace{3cm}

\vfill

\footnotesize
\textsc{Quelle}: \titel. Herausgegeben von {\editorInnen}. In: \emph{Arthur Schnitzler: Briefwechsel mit Autorinnen und Autoren}.
 Digitale Edition, https://schnitzler-briefe.acdh.oeaw.ac.at/{\dateiname}.html (Stand \today)
\fi

\end{document}


