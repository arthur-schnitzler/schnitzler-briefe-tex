%% latex-leseansicht-vorspann.tex
%% Vorspann für die Leseansicht.
%% Lädt die gemeinsame Datei latex-vorspann.tex mit nicht gesetztem Schalter.

\newif\ifkorrekturansicht
\korrekturansichtfalse

\input{../tex-inputs/latex-vorspann}


         
         \newcommand{\erwaehntePersonen}{Personen:  ?? [Staatsmann aus der Umgebung Renners], Georg Brandes, Gaius Iulius Caesar, Johann Wolfgang von Goethe, Markus Hajek, Gisela Hajek, Karl Renner, Olga Schnitzler, Lili Schnitzler, Heinrich Schnitzler,  Voltaire}
         \newcommand{\erwaehnteOrte}{Orte: Altaussee, Amerika, Amsterdam, Bad Gastein, Dänemark, Florenz, Kopenhagen, München, Niederlande, Russland, Salzkammergut, Schweden, Skandinavien, Sternwartestraße, Ungarn, Wien, Österreich}
         \newcommand{\erwaehnteWerke}{Werke: Gaius Julius Cæsar, Voltaire und sein Jahrhundert, Wolfgang Goethe}
               \section[Arthur Schnitzler an Georg Brandes, 16. 8. 1920]{ Arthur Schnitzler an Georg Brandes, 16. 8. 1920}\nopagebreak\mylabel{v}\rehead{ }\begin{ledgroupsized}[t]{13cm}\normalsize\beginnumbering \toendnotes[C]{\smallbreak\pagebreak[2]} \Standort{Kopenhagen, Det Kongelige Bibliotek, Georg Brandes Arkiv, box 125.}
\physDesc{Brief, 4 Blätter, 8 Seiten (Paginierung 1–8)
\newline{}Handschrift: Bleistift, lateinische Kurrent\newline{}Ordnung: mit Bleistift von unbekannter Hand nummeriert:
                                                »42.«, die neuen Blätter
                                            mit Datum versehen: »16/8 20« und beschriftet mit
                                 »Schnitzler« }\buchAbdrucke{\weitereDrucke{1) Georg Brandes, Arthur Schnitzler: \emph{Ein Briefwechsel}. Hg. Kurt Bergel. Bern: \emph{Francke} 1956, S. 127–130.} \weitereDrucke{2) Arthur Schnitzler: \emph{Briefe 1913–1931}. Hg. Peter Michael Braunwarth, Richard Miklin, Susanne Pertlik und Heinrich Schnitzler. Frankfurt am Main: \emph{S. Fischer} 1984, S. 212–215.} }\toendnotes[C]{\smallbreak}\pstart
           \noindent{}{\pb}XVIII Sternwartestr 71\oindex{Sternwartestrasse@\textbf{Sternwartestraße}|pw}\hfill Wien\oindex{Wien@\textbf{Wien}|pw}, 16. August 1920\pend
           \pstart
           lieber und verehrter Freund, mit Freude lese ich aus Ihrem
                    Brief, dass Sie arbeiten und sich wohl befinden. Wann aber werden wir, die nicht
                    daenisch verstehen, Ihre neuen Bücher kennen lernen? Goethe\pwindex{Goethe, Johann Wolfgang von 1749-08-28 – 1832-03-22@\textsc{Goethe, Johann Wolfgang von} (1749-08-28 – 1832-03-22), \emph{Schriftsteller}|pw}\pwindex{Brandes, Georg 04.02.1842 – 19.02.1927@\textsc{Brandes, Georg} (04.02.1842 – 19.02.1927)!Wolfgang Goethe1915@\strich\emph{Wolfgang Goethe} {[}1915{]}|pwv}, Voltaire\pwindex{Voltaire 21.11.1694 – 30.05.1778@\textsc{Voltaire} (21.11.1694 – 30.05.1778), \emph{Schriftsteller, Philosoph}|pw}\pwindex{Brandes, Georg 04.02.1842 – 19.02.1927@\textsc{Brandes, Georg} (04.02.1842 – 19.02.1927)!Voltaire und sein Jahrhundert1916 – 1917@\strich\emph{Voltaire und sein Jahrhundert} {[}1916 – 1917{]}|pwv}, Julius Caesar\pwindex{Caesar, Gaius Iulius 13.7.100? v. Chr. – 15.3.44 v. Chr.@\textsc{Caesar, Gaius Iulius} (13.7.100? v. Chr. – 15.3.44 v. Chr.), \emph{Politiker, Kaiser, Heerführer}|pw}\pwindex{Brandes, Georg 04.02.1842 – 19.02.1927@\textsc{Brandes, Georg} (04.02.1842 – 19.02.1927)!Gaius Julius Cæsar1918@\strich\emph{Gaius Julius Cæsar} {[}1918{]}|pwv} – keines von den dreien ist meines Wissens in deutscher Sprache erschienen
                    oder bisher nur angekündigt.\pend
           \pstart
           Verzeihen Sie mir daß ich mit Bleistift schreibe, – so wird es leserlicher als
                    mit der Feder (auch die sind während des Krieges hundertmal schlechter
                    geworden); – und seit einer ziemlich erheblichen Oberarmverletzung die ich im
                    Frühjahr durch einen Sturz über eine Baumwurzel erlitt und die mir durch ein
                    paar Wochen das Schreiben ganz unmöglich machte, scheint mir, dſs die Stahlfeder
                    meiner Schrift noch weniger entgegenkommt als früher. Die Sache ist übrigens
                    schon ganz gut. Auch sonst darf ich über mein Befinden (abgesehen von dem
                    vertrackten Ohr) nicht klagen. Wir alle bringen uns, materiell, körperlich,
                    seelisch, über diese Zeit des Grauens und der Schurkerei, ganz leidlich fort.
                    Alle {\dotstwo} d. h. die {\pb}Meinigen, nahe Verwandte und Freunde. Die Zustände in Oesterreich\oindex{Oesterreich@\textbf{Österreich}|pw}, in Wien\oindex{Wien@\textbf{Wien}|pw} vor
                    allem, sind schli{\geminationm} genug – aber in die Ferne
                    dringen doch alle Nachrichten so concentrirt, daß man notwendig ein
                        übertriebne\textcolor{gray}{s} Bild empfängt. Am übelesten \introOben{}dran\introOben{} ist natürlich der sog. Mittelstand, eine gewisse
                    Sorte von Beamten, ehemaligen Offizieren, Aerzten, Advokaten, Künstlern, –
                    Rentiers, die sich mit einer kleinen Rente ins Privatleben zurückgezogen haben
                    und nun, da alles, nach unserer Valuta 50–100mal theurer geworden ist, langsam
                    verhungern oder wenigstens proletarisiren. Dem sog. Proletariat, dem einstigen
                    (freilich gibt es auch hier Ausnahmen) geht es besser als je, und man darf nicht
                    behaupten, daß diese Schichte ethisch ihrem Aufstieg sich gewachsen zeigt. Aber
                        \introOben{}warum\introOben{} sollten unter den Kanalräumern,
                    Laternanzündern, Greißlern, Fabriksarbeitern, Locomotivführern u. s. w. die
                    Parvenus sich besser benehmen als sie es in andern Ständen zu thun pflegten? {\pb}An den sog. neuen Reichen und Schiebern
                    mangelt es in den neutralen Ländern, wie man weiß, so wenig als bei uns; – sie
                    machen sich vielfach unangenehm bemerkbar, – und viele Leute, Moralisten und
                    Vergnügungsreisende, beklagen sich und finden es furchtbar, daß in der selben
                    Stadt das schrecklichste Elend neben dem lächerlichsten Luxus und fabelhafter
                    Verschwendungssucht bestehen kann;– aber neulich sagte einer unsrer Staatsmä{\geminationn}er\pwindex{?? [Staatsmann aus der Umgebung Renners] 16.8.1920 – 16.8.1920@\textsc{?? [Staatsmann aus der Umgebung Renners]} (16.8.1920 – 16.8.1920)|pwv} (aus der nächsten Nähe Renner\pwindex{Renner, Karl 14.12.1870 – 31.12.1950@\textsc{Renner, Karl} (14.12.1870 – 31.12.1950), \emph{Politiker}|pw}s) zu mir, daß es vielleicht die
                    Schieber und Verschwender seien die uns retten oder wenigstens über Wasser
                    halten – was nationaloekonomisch vielleicht seine Richtigkeit hat. Das
                    entwertete Geld, das in Fluß ko{\geminationm}t, ist nie so
                    gefährlich als das aus dem Verkehr gezogene; – und ein großer Theil unsres
                    Unglückes liegt meiner Überzeugung nach in den Truhen der Bauern, in Gestalt von
                    Banknoten begraben. Hier ließe sich auch von dem unglückseligen Verhältnis
                    zwischen Stadt {\pb}und Land reden, das für den
                    Zustand Oesterreich\oindex{Oesterreich@\textbf{Österreich}|pw}s so charakteristisch ist
                    – aber das führte ins unendliche. Man glaubt vielfach, daß schon die Neuwahlen
                    im Herbst bei uns eine Niederlage der Sozialdemokraten oder mindestens
                    erhebliche Stimmenzunahme der Christlichsozialen bringen werden; – zu ganz russischen\oindex{Russland@\textbf{Russland}|pw} – oder zu ganz ungarischen\oindex{Ungarn@\textbf{Ungarn}|pw} Zuständen wird es bei uns nie kommen, de{\geminationn} bei uns bringt man \strikeout{\textcolor{gray}{×}\-\textcolor{gray}{×}} es nie zum Fanatismus, sondern nur \introOben{}bis\introOben{} zur
                    Lausbüberei (was aber in solchen Zeitläuften immerhin für kleinen rothen und
                    weißen Terror ausreichen mag.) Die schlimmsten Rollen spielen, wie jederzeit,
                    die Renegaten, – es hat seine geschichtlichen \introOben{}und
                        psycholog.\introOben{} Gründe, daß sich diese \introOben{}unerfreuliche und
                        gefährliche\introOben{} Spielart unter den Deutschen, den Juden und den Literaten
                    am häufigsten findet.\pend
           \pstart
           {\pb}Aber ich will Ihnen doch um Gotteswillen
                    keinen politischen Brief schreiben – schon darum weil es da{\geminationn} kein Brief sondern ein Buch würde, – mit
                    Parenthesen, Co{\geminationm}entaren, kleingedruckten
                    Anmerkungen; – de{\geminationn} welcher Satz, welche
                    Charakteristik dürfte ohne Einschränkung gelten?– Umso lebhafter hätt ich das
                    Bedürfnis wieder einmal mit Ihnen zu reden;– aber wa{\geminationn} ka{\geminationn} ich nach Daenemark\oindex{Daenemark@\textbf{Dänemark}|pw}, oder Sie nach Oesterreich\oindex{Oesterreich@\textbf{Österreich}|pw}? –\pend
           \pstart
           Übrigens ist \strikeout{e} diese verda{\geminationm}te Valuta, die ich daher doch nicht so ganz
                    verdammen kann, \introOben{}Schuld daran\introOben{}, daß ich mich in den
                    letzten zwei Jahren trotz der fürchterlichen Geldentwerthung mit den Meinigen
                    ohne eigentliche »Sorgen« weitergebracht habe: in Holland\oindex{Niederlande@\textbf{Niederlande}|pw}, Schweden\oindex{Schweden@\textbf{Schweden}|pw}, und auch bei
                    Ihnen wurde einiges von mir gespielt; auch Amerika\oindex{Amerika@\textbf{Amerika}|pw} fängt an sich zu melden;– und \strikeout{die} Beträge in nordischen\oindex{Skandinavien@\textbf{Skandinavien}|pwv} Kronen, oder holl\oindex{Niederlande@\textbf{Niederlande}|pw}.
                    Gulden, die früher gar nicht in Betracht geko{\geminationm}en
                        {\pb}wären, bedeuten für uns heruntergeko{\geminationm}ene Oesterreicher\oindex{Oesterreich@\textbf{Österreich}|pw} schon etwas. Daß keiner von uns auf dem gleichen Fuß wie
                    vor dem Krieg oder auch noch 1916, 17 leben ka{\geminationn}, ist selbstverständlich. ich habe neulich
                    berechnet, dſs ich, we{\geminationn} ich z. B. meine Existenz
                    nach \strikeout{dem} der von 1914 einrichten
                    wollte, – 1½–2 Millionen Kronen \introOben{}(als Jahresausgabe)\introOben{}
                    bräuchte – und wie ich es anstellen sollte, zu Schiff von Florenz\oindex{Florenz@\textbf{Florenz}|pw} nach Amsterdam\oindex{Amsterdam@\textbf{Amsterdam}|pw}
                    zu gelangen, (wie ich es im Mai 1914 gethan) – das wird mir \introOben{}auch we{\geminationn} ich noch eine halbe
                        Million zulegte\introOben{} keiner sagen kö{\geminationn}en. – Wir
                    wohnen \strikeout{selb} in unserer alten kleinen Villa, die
                    Sie kennen; – (für notwendige Reparaturen habe ich in diesem Jahr annähernd so
                    viel bezahlt, als das Haus 1910 gekostet hat);– ein solches Heim in
                    dieser Zeit \introOben{}zu\introOben{} haben, empfanden wir als besondre
                    Schicksalsgunst; – freilich \strikeout{fühlte} war man nicht
                    jederzeit sicher, daß man es sich unge{\pb}schmälert erhalten würde; – aber bisher \substVorne{}\textsuperscript{waren}\substDazwischen{}sind\substHinten{} wir von Zwangseinquartierungen, Anforderungen, – ja sogar (wir wollen
                    nichts verschreien) von Einbrüchen verschont geblieben;– und auch die
                    gelegentlich angedrohten Plünderungen haben in Wien\oindex{Wien@\textbf{Wien}|pw} im allgemeinen nicht stattgefunden. Bisher. Da die Weltgeschichte
                    ja leider ungehindert weitergeht, ist nicht abzusehen, was wir noch erleben
                    werden. Im übrigen lebt man ja doch weiter – als könnte nichts passiren. Meine
                        Frau\pwindex{Schnitzler, Olga 17.01.1882 – 13.01.1970@\textsc{Schnitzler, Olga} (17.01.1882 – 13.01.1970), \emph{Schauspielerin, Sängerin}|pwv} gebraucht eine Cur
                    in Gastein\oindex{Bad Gastein@\textbf{Bad Gastein}|pw}, meine kleine Tochter\pwindex{Schnitzler, Lili 13.09.1909 – 26.07.1928@\textsc{Schnitzler, Lili} (13.09.1909 – 26.07.1928)|pwv} ist bei meinem Schwager\pwindex{Hajek, Markus 25.11.1861 – 04.04.1941@\textsc{Hajek, Markus} (25.11.1861 – 04.04.1941), \emph{Mediziner, Laryngologe}|pwv} und meiner Schwester\pwindex{Hajek, Gisela 20.12.1867 – 03.02.1953@\textsc{Hajek, Gisela} (20.12.1867 – 03.02.1953)|pwv} in Altaussee\oindex{Altaussee@\textbf{Altaussee}|pw}, mein Sohn\pwindex{Schnitzler, Heinrich 09.08.1902 – 12.07.1982@\textsc{Schnitzler, Heinrich} (09.08.1902 – 12.07.1982), \emph{Regisseur, Schauspieler}|pwv}, achtzehn, (hat die Matura gemacht, muss aber
                    Mathematik wiederholen) – ist nach München\oindex{Muenchen@\textbf{München}|pw}
                    gereist, und auch ich verlasse in wenigen Tagen die Stadt, wahrschein{\pb}lich Salzka{\geminationm}ergut\oindex{Salzkammergut@\textbf{Salzkammergut}|pw}, – um Anfang September mit all den
                    Meinen in Altaussee\oindex{Altaussee@\textbf{Altaussee}|pw} zusa{\geminationm}enzutreffen.\pend
           \pstart
           Gearbeitet habe ich nicht viel in den letzten Jahren, allerlei angefangen;– ich
                    fühlte mich doch sehr bedrückt und verdüstert. Wäre man wenigstens freizügig wie
                    einst. Unsere schönen Reisen – wie offen lag die Welt! Jetzt ist es schon ein
                    kleines Problem, sich selbst und sein Gepäck zur Bahn zu schaffen – ein Billet
                    zu lösen u. s. w. –\pend
           \pstart
           Nun hab ich Ihnen sozusagen acht Seiten geschrieben; – es ist nichts. – Und Sie
                    Armer der sich trotzdem plagen musste es zu lesen!\pend
           \pstart
           – Denken Sie meiner weiter in Freundschaft; – ich halte an der Hoffnung fest, Sie
                    wiederzusehen, und bin von Herzen\pend
           \pstart
           Ihr getreuer{\\[\baselineskip]}\spacefill\mbox{Arthur Schnitzler}\pend
           \leftskip=0em{}
         
         \endnumbering\mylabel{h}\end{ledgroupsized}  \newcommand{\dateiname}{L02353}\newcommand{\titel}{Arthur Schnitzler an Georg Brandes, 16. 8. 1920}\newcommand{\editorInnen}{Martin Anton Müller und Gerd-Hermann Susen}%% latex-leseansicht-abspann.tex
%% Abspann für die Leseansicht.
%% Der Schalter \ifkorrekturansicht ist bereits durch den Vorspann gesetzt.

%% latex-abspann.tex
%% Gemeinsamer Abspann für Korrekturansicht und Leseansicht.
%% Setzt den Schalter \ifkorrekturansicht voraus (gesetzt in den
%% einbindenden Dateien latex-korrekturansicht-abspann.tex bzw.
%% latex-leseansicht-abspann.tex).
%% ---------------------------------------------------------------

\normalsize

% Das esempio-Environment wird nur in der Leseansicht benötigt
\ifkorrekturansicht\else
\newenvironment{esempio}[3]%
{
    \vspace{1.5ex}
    \rlap{\underline{#1}}
    \par
    \setlength{\parindent}{0cm}
    \nopagebreak
    \leftskip=#2cm
    \rightskip=#3cm
}
{
    \par
}
\fi

\doendnotes{C}
\bigskip
\vfill

\clearpage

\footnotesize

\ifkorrekturansicht
  \lohead{\textsc{register}}
\fi

% theindex-Environment neu definieren ohne reledmac
\makeatletter
\renewenvironment{theindex}{%
  \ifkorrekturansicht
    \section*{\indexname}%
  \else
    \subsubsection*{Index der erwähnten Entitäten}%
  \fi
  \setlength{\parindent}{0pt}%
  \setlength{\parskip}{0pt plus 0.3pt}%
  \let\item\@idxitem
}{%
  \ifkorrekturansicht\clearpage\fi
}
\makeatother

\IfFileExists{\jobname-pw.ind}{\input{\jobname-pw.ind}}{}

% Quellenangabe nur in der Leseansicht
\ifkorrekturansicht\else
% Fallback-Definitionen, falls die .tex-Datei \titel etc. nicht gesetzt hat
\providecommand{\titel}{}
\providecommand{\editorInnen}{}
\providecommand{\dateiname}{\jobname}

\vspace{3cm}

\vfill

\footnotesize
\textsc{Quelle}: \titel. Herausgegeben von {\editorInnen}. In: \emph{Arthur Schnitzler: Briefwechsel mit Autorinnen und Autoren}.
 Digitale Edition, https://schnitzler-briefe.acdh.oeaw.ac.at/{\dateiname}.html (Stand \today)
\fi

\end{document}


      