%% latex-korrekturansicht-vorspann.tex
%% Vorspann für die Korrekturansicht.
%% Lädt die gemeinsame Datei latex-vorspann.tex mit gesetztem Schalter.

\newif\ifkorrekturansicht
\korrekturansichttrue

\input{../tex-inputs/latex-vorspann}


\section[ Paul Goldmann an Arthur Schnitzler, 18. 10. {[}1901{]}]{L03089 Paul Goldmann an Arthur Schnitzler, 18. 10. {[}1901{]}}
\nopagebreak\mylabel{L03089v}
\rehead{ }\normalsize\beginnumbering\briefempfaengerindex{Schnitzler, Arthur@\textsc{Schnitzler, Arthur}!zzzGoldmann, Paul@\emph{von Paul Goldmann}!1901-10-182@{18. 10. {[}1901{]}}|(be}
\toendnotes[C]{\smallbreak\pagebreak[2]}\Standort{DLA, A:Schnitzler, HS.NZ85.1.3171.}
\physDesc{Brief, 1 Blatt, 3 Seiten, 642 Zeichen
\newline{}Handschrift: blaue Tinte, deutsche Kurrent
\newline{}Schnitzler: 1) mit Bleistift das Jahr »901« vermerkt  2) mit rotem Buntstift eine Unterstreichung}\toendnotes[C]{\smallbreak}
\pstart
           \raggedleft{}{\pb}\textcolor{gray}{\textbf{DESSAUERSTRASSE 19}}\oindex{Dessauer Strasse@\textbf{Dessauer Straße}, \emph{Straße (K.STR)}|pw}\pend
           
\pstart
           Berlin\oindex{Berlin@\textbf{Berlin}, \emph{P.PPLC}|pw}, 18. Oktober.\pend
           
\pstart\center{}Mein lieber Freund,\pend\vspace{0.5em}
\pstart
           Das \label{K_L03089-1v}\edtext{Telegramm\pwindex{Kleine Chronik. [Berliner Theater.]@\emph{Kleine Chronik. [Berliner Theater.]}|pwv}}{\lemma{\textnormal{\emph{Telegramm}}}\Cendnote{\textnormal{[Paul Goldmann\pwindex{Goldmann, Paul 31.01.1865 – 25.09.1935@\textsc{Goldmann, Paul} (31.01.1865 – 25.09.1935), \emph{Schriftsteller/Schriftstellerin, Journalist/Journalistin}|pwk}]: \emph{Kleine Chronik. [Berliner Theater]}\pwindex{Kleine Chronik. [Berliner Theater.]@\emph{Kleine Chronik. [Berliner Theater.]}|pwk}. In: \emph{Neue Freie Presse}\pwindex{Neue Freie Presse@\emph{Neue Freie Presse}|pwk}, Nr. 13.342, 16. 10. 1901, Abendblatt, S. 1. Darin wird
                  von der Annahme von \emph{Lebendige Stunden}\pwindex{Lebendige Stunden. Vier Einakter@\emph{Lebendige Stunden. Vier Einakter}|pwk} durch
                  das \emph{Deutsche Theater Berlin}\orgindex{Deutsches Theater Berlin@Deutsches Theater Berlin|pwk} berichtet. Otto Brahm\pwindex{Brahm, Otto 05.02.1856 – 28.11.1912@\textsc{Brahm, Otto} (05.02.1856 – 28.11.1912), \emph{Theaterleiter/Theaterleiterin, Regisseur/Regisseurin}|pwk} hatte keine Pressemitteilung
                  verfasst, vgl. seinen Brief an Schnitzler
                  vom 19. 10. 1901 (\emph{Der Briefwechsel Arthur Schnitzler – Otto Brahm}.
                     Vollständige Ausgabe. Herausgegeben, eingeleitet und erläutert von Oskar
                     Seidlin. Tübingen: \emph{Niemeyer}{ }1975, S. 100–101).}}}\label{K_L03089-1} kommt von mir. Die
               Nachricht iſt der \label{K_L03089-2v}\edtext{»Berliner Morgenpoſt\pwindex{Berliner Morgenpost@\emph{Berliner Morgenpost}|pw}\pwindex{Arthur Schnitzler [Lebendige Stunden am Deutschen Theater Berlin]@\emph{Arthur Schnitzler [Lebendige Stunden am Deutschen Theater Berlin]}|pwv}«}{\lemma{\textnormal{\emph{»Berliner Morgenpoſt«}}}\Cendnote{\textnormal{[O. V.]: \emph{Arthur Schnitzler}\pwindex{Arthur Schnitzler [Lebendige Stunden am Deutschen Theater Berlin]@\emph{Arthur Schnitzler [Lebendige Stunden am Deutschen Theater Berlin]}|pwk}. In: \emph{Berliner Morgenpost}\pwindex{Berliner Morgenpost@\emph{Berliner Morgenpost}|pwk}, Jg. 4, Nr. 243, 16. 10. 1901, S. 3. }}}\label{K_L03089-2} entnommen, einem
               in Theater-Angelegenheiten gut unterrichteten Blatte\orgindex{Berliner Morgenpost@Berliner Morgenpost|pwv}.\pend
           
\pstart
           \label{K_L03089-3v}\edtext{Brahm\pwindex{Brahm, Otto 05.02.1856 – 28.11.1912@\textsc{Brahm, Otto} (05.02.1856 – 28.11.1912), \emph{Theaterleiter/Theaterleiterin, Regisseur/Regisseurin}|pw} iſt blödſinnig}{\lemma{\textnormal{\emph{Brahm iſt blödſinnig}}}\Cendnote{\textnormal{Otto Brahm\pwindex{Brahm, Otto 05.02.1856 – 28.11.1912@\textsc{Brahm, Otto} (05.02.1856 – 28.11.1912), \emph{Theaterleiter/Theaterleiterin, Regisseur/Regisseurin}|pwk} hatte in seinem Brief vom
                     11. 10. 1901{ }Schnitzler gebeten, den Einakterzyklus \emph{Lebendige Stunden}\pwindex{Lebendige Stunden. Vier Einakter@\emph{Lebendige Stunden. Vier Einakter}|pwk} auf vier Stücke zu
                  reduzieren. In Folge wurde auf \emph{Der
                     Puppenspieler}\pwindex{Puppenspieler. Studie in einem Aufzuge@\emph{Der Puppenspieler. Studie in einem Aufzuge}|pwk} verzichtet. Vgl. \emph{Der Briefwechsel
                        Arthur Schnitzler – Otto Brahm}. Vollständige Ausgabe. Herausgegeben,
                     eingeleitet und erläutert von Oskar Seidlin. Tübingen:
                        \emph{Niemeyer}{ }1975, S. 99–101. Zur Uraufführung der Einakter\pwindex{Lebendige Stunden. Vier Einakter@\emph{Lebendige Stunden. Vier Einakter}|pwkv} kam es am 1. 4. 1902 am Deutschen Theater Berlin\oindex{Deutsches Theater Berlin@\textbf{Deutsches Theater Berlin}, \emph{Theater (K.THE)}|pwk}.}}}\label{K_L03089-3}. Ich wußte wohl,
               daß er ein unkünſtleriſcher Direktor iſt. Aber das hatte ich nicht
                  erwartet\textcolor{gray}{.}{ }{\pb}Wenn er bei ſeiner Weigerung bleibt, ſo ziehſt Du
               einfach ſämmtliche Stücke\pwindex{Lebendige Stunden. Vier Einakter@\emph{Lebendige Stunden. Vier Einakter}|pwv}
               zurück und gibſt ſie dem Leſſingtheater\orgindex{Lessing-Theater@Lessing-Theater|pw}. \strikeout{So} Das iſt ja wahrhaft ſkandalös!\pend
           
\pstart
           \strikeout{\textcolor{gray}{Mir} thut \textcolor{gray}{e}} Bitte, halte mich über den weiteren Verlauf der Angelegenheit auf dem
               Laufenden!\pend
           
\pstart
           Mir thut es leid, ſo ſelten und ſo wenig von {\pb}Dir zu
               hören.\pend
           
\pstart
           Viele Grüße an die beiden Mädchen\pwindex{Schnitzler, Olga 17.01.1882 – 13.01.1970@\textsc{Schnitzler, Olga} (17.01.1882 – 13.01.1970), \emph{Schauspieler/Schauspielerin, Sänger/Sängerin}|pwv}\pwindex{Steinrueck, Elisabeth 19.11.1885 – 07.04.1920@\textsc{Steinrück, Elisabeth} (19.11.1885 – 07.04.1920)|pwv} und an Dich! {\\[\baselineskip]}Dein {\\[\baselineskip]}\spacefill\mbox{Paul Goldmann.}\pend
           \leftskip=0em{}\selectlanguage{ngerman}\endnumbering\briefempfaengerindex{Schnitzler, Arthur@\textsc{Schnitzler, Arthur}!zzzGoldmann, Paul@\emph{von Paul Goldmann}!1901-10-182@{18. 10. {[}1901{]}}|)be}\mylabel{L03089h}  \normalsize

\doendnotes{C}
\bigskip
\vfill

\clearpage

\footnotesize

\lohead{\textsc{register}}

% Definiere theindex-Environment komplett neu ohne reledmac
\makeatletter
\renewenvironment{theindex}{%
  \section*{\indexname}%
  \setlength{\parindent}{0pt}%
  \setlength{\parskip}{0pt plus 0.3pt}%
  \let\item\@idxitem
}{%
  \clearpage
}
\makeatother

\IfFileExists{\jobname-pw.ind}{\input{\jobname-pw.ind}}{}

\end{document}

      