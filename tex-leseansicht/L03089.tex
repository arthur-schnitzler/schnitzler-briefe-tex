%% latex-leseansicht-vorspann.tex
%% Vorspann für die Leseansicht.
%% Lädt die gemeinsame Datei latex-vorspann.tex mit nicht gesetztem Schalter.

\newif\ifkorrekturansicht
\korrekturansichtfalse

\input{../tex-inputs/latex-vorspann}


         
         \renewcommand{\erwaehntePersonen}{Personen: Otto Brahm, Paul Goldmann, Olga Schnitzler, Elisabeth Steinrück}
         \renewcommand{\erwaehnteInstitutionen}{Institutionen: Berliner Morgenpost, Deutsches Theater Berlin, Lessing-Theater}
         \renewcommand{\erwaehnteOrte}{Orte: Berlin, Dessauer Straße, Deutsches Theater Berlin, Wien}
         \renewcommand{\erwaehnteWerke}{Werke: Arthur Schnitzler [Lebendige Stunden am Deutschen Theater Berlin], Berliner Morgenpost, Der Puppenspieler. Studie in einem Aufzuge, Kleine Chronik. [Berliner Theater.], Lebendige Stunden. Vier Einakter, Neue Freie Presse}
               \section[ Paul Goldmann an Arthur Schnitzler, 18. 10. {[}1901{]}]{ Paul Goldmann an Arthur Schnitzler, 18. 10. {[}1901{]}}\nopagebreak\mylabel{v}\rehead{ }\begin{ledgroupsized}[t]{13cm}\normalsize\beginnumbering\briefempfaengerindex{Schnitzler, Arthur@\textsc{Schnitzler, Arthur}!zzzGoldmann, Paul@\emph{von Paul Goldmann}!1901-10-182@{18. 10. {[}1901{]}}|(be} \toendnotes[C]{\smallbreak\pagebreak[2]} \Standort{DLA, A:Schnitzler, HS.NZ85.1.3171.}
\physDesc{Brief, 1 Blatt, 3 Seiten, 642 Zeichen
\newline{}Handschrift: blaue Tinte, deutsche Kurrent
\newline{}Schnitzler: 1) mit Bleistift das Jahr »901« vermerkt  2) mit rotem Buntstift eine Unterstreichung}\toendnotes[C]{\smallbreak}\pstart
           \noindent{}\raggedleft{}{\pb}\textcolor{gray}{\textbf{DESSAUERSTRASSE 19}}\oindex{Dessauer Strasse@\textbf{Dessauer Straße}|pw}\pend
           \pstart
           Berlin\oindex{Berlin@\textbf{Berlin}|pw}, 18. Oktober.\pend
           \pstart\center{}Mein lieber Freund,\pend\pstart
           Das \label{K_L03089-1v}\edtext{Telegramm\pwindex{Kleine Chronik. [Berliner Theater.]1901-10-16@\emph{Kleine Chronik. [Berliner Theater.]} {[}1901-10-16{]}|pwv}}{\lemma{\textnormal{\emph{Telegramm}}}\Cendnote{\textnormal{[Paul Goldmann\pwindex{Goldmann, Paul 31.01.1865 – 25.09.1935@\textsc{Goldmann, Paul} (31.01.1865 – 25.09.1935), \emph{Schriftsteller, Journalist}|pwk}]: \emph{Kleine Chronik. [Berliner Theater]}\pwindex{Kleine Chronik. [Berliner Theater.]1901-10-16@\emph{Kleine Chronik. [Berliner Theater.]} {[}1901-10-16{]}|pwk}. In: \emph{Neue Freie Presse}\pwindex{Neue Freie Presse1864 – 1939@\emph{Neue Freie Presse} {[}1864 – 1939{]}|pwk}, Nr. 13.342, 16. 10. 1901, Abendblatt, S. 1. Darin wird
                  von der Annahme von \emph{Lebendige Stunden}\pwindex{Schnitzler, Arthur 15.05.1862 – 21.10.1931@\textsc{Schnitzler, Arthur} (15.05.1862 – 21.10.1931), \emph{Schriftsteller, Mediziner}!Lebendige Stunden. Vier Einakter1901-12-23@\strich\emph{Lebendige Stunden. Vier Einakter} {[}1901-12-23{]}|pwk} durch
                  das \emph{Deutsche Theater Berlin}\orgindex{Deutsches Theater Berlin@Deutsches Theater Berlin|pwk} berichtet. Otto Brahm\pwindex{Brahm, Otto 05.02.1856 – 28.11.1912@\textsc{Brahm, Otto} (05.02.1856 – 28.11.1912), \emph{Theaterleiter, Regisseur}|pwk} hatte keine Pressemitteilung
                  verfasst, vgl. seinen Brief an Schnitzler\pwindex{Schnitzler, Arthur 15.05.1862 – 21.10.1931@\textsc{Schnitzler, Arthur} (15.05.1862 – 21.10.1931), \emph{Schriftsteller, Mediziner}|pwk}
                  vom 19. 10. 1901 (\emph{Der Briefwechsel Arthur Schnitzler – Otto Brahm}.
                     Vollständige Ausgabe. Herausgegeben, eingeleitet und erläutert von Oskar
                     Seidlin. Tübingen: \emph{Niemeyer}{ }1975, S. 100–101).}}}\label{K_L03089-1h} kommt von mir. Die
               Nachricht iſt der \label{K_L03089-2v}\edtext{»Berliner Morgenpoſt\pwindex{?? Werk@Nicht ermittelte Verfasserinnen und Verfasser!Berliner Morgenpost1898 – 1933@\emph{Berliner Morgenpost} {[}1898 – 1933{]}|pw}\pwindex{?? Werk@Nicht ermittelte Verfasserinnen und Verfasser!Arthur Schnitzler [Lebendige Stunden am Deutschen Theater Berlin]1901-10-16@\emph{Arthur Schnitzler [Lebendige Stunden am Deutschen Theater Berlin]} {[}1901-10-16{]}|pwv}«}{\lemma{\textnormal{\emph{»Berliner Morgenpoſt«}}}\Cendnote{\textnormal{[O. V.]: \emph{Arthur Schnitzler}\pwindex{?? Werk@Nicht ermittelte Verfasserinnen und Verfasser!Arthur Schnitzler [Lebendige Stunden am Deutschen Theater Berlin]1901-10-16@\emph{Arthur Schnitzler [Lebendige Stunden am Deutschen Theater Berlin]} {[}1901-10-16{]}|pwk}. In: \emph{Berliner Morgenpost}\pwindex{?? Werk@Nicht ermittelte Verfasserinnen und Verfasser!Berliner Morgenpost1898 – 1933@\emph{Berliner Morgenpost} {[}1898 – 1933{]}|pwk}, Jg. 4, Nr. 243, 16. 10. 1901, S. 3. }}}\label{K_L03089-2h} entnommen, einem
               in Theater-Angelegenheiten gut unterrichteten Blatte\orgindex{Berliner Morgenpost@Berliner Morgenpost|pwv}.\pend
           \pstart
           \label{K_L03089-3v}\edtext{Brahm\pwindex{Brahm, Otto 05.02.1856 – 28.11.1912@\textsc{Brahm, Otto} (05.02.1856 – 28.11.1912), \emph{Theaterleiter, Regisseur}|pw} iſt blödſinnig}{\lemma{\textnormal{\emph{Brahm iſt blödſinnig}}}\Cendnote{\textnormal{Otto Brahm\pwindex{Brahm, Otto 05.02.1856 – 28.11.1912@\textsc{Brahm, Otto} (05.02.1856 – 28.11.1912), \emph{Theaterleiter, Regisseur}|pwk} hatte in seinem Brief vom
                     11. 10. 1901{ }Schnitzler\pwindex{Schnitzler, Arthur 15.05.1862 – 21.10.1931@\textsc{Schnitzler, Arthur} (15.05.1862 – 21.10.1931), \emph{Schriftsteller, Mediziner}|pwk} gebeten, den Einakterzyklus \emph{Lebendige Stunden}\pwindex{Schnitzler, Arthur 15.05.1862 – 21.10.1931@\textsc{Schnitzler, Arthur} (15.05.1862 – 21.10.1931), \emph{Schriftsteller, Mediziner}!Lebendige Stunden. Vier Einakter1901-12-23@\strich\emph{Lebendige Stunden. Vier Einakter} {[}1901-12-23{]}|pwk} auf vier Stücke zu
                  reduzieren. In Folge wurde auf \emph{Der
                     Puppenspieler}\pwindex{Schnitzler, Arthur 15.05.1862 – 21.10.1931@\textsc{Schnitzler, Arthur} (15.05.1862 – 21.10.1931), \emph{Schriftsteller, Mediziner}!Puppenspieler. Studie in einem Aufzuge31. 05. 1903@\strich\emph{Der Puppenspieler. Studie in einem Aufzuge} {[}31. 05. 1903{]}|pwk} verzichtet. Vgl. \emph{Der Briefwechsel
                        Arthur Schnitzler – Otto Brahm}. Vollständige Ausgabe. Herausgegeben,
                     eingeleitet und erläutert von Oskar Seidlin. Tübingen:
                        \emph{Niemeyer}{ }1975, S. 99–101. Zur Uraufführung der Einakter\pwindex{Schnitzler, Arthur 15.05.1862 – 21.10.1931@\textsc{Schnitzler, Arthur} (15.05.1862 – 21.10.1931), \emph{Schriftsteller, Mediziner}!Lebendige Stunden. Vier Einakter1901-12-23@\strich\emph{Lebendige Stunden. Vier Einakter} {[}1901-12-23{]}|pwkv} kam es am 1. 4. 1902 am Deutschen Theater Berlin\oindex{Deutsches Theater Berlin@\textbf{Deutsches Theater Berlin}|pwk}.}}}\label{K_L03089-3h}. Ich wußte wohl,
               daß er ein unkünſtleriſcher Direktor iſt. Aber das hatte ich nicht
                  erwartet\textcolor{gray}{.}{ }{\pb}Wenn er bei ſeiner Weigerung bleibt, ſo ziehſt Du
               einfach ſämmtliche Stücke\pwindex{Schnitzler, Arthur 15.05.1862 – 21.10.1931@\textsc{Schnitzler, Arthur} (15.05.1862 – 21.10.1931), \emph{Schriftsteller, Mediziner}!Lebendige Stunden. Vier Einakter1901-12-23@\strich\emph{Lebendige Stunden. Vier Einakter} {[}1901-12-23{]}|pwv}
               zurück und gibſt ſie dem Leſſingtheater\orgindex{Lessing-Theater@Lessing-Theater|pw}. \strikeout{So} Das iſt ja wahrhaft ſkandalös!\pend
           \pstart
           \strikeout{\textcolor{gray}{Mir} thut \textcolor{gray}{e}} Bitte, halte mich über den weiteren Verlauf der Angelegenheit auf dem
               Laufenden!\pend
           \pstart
           Mir thut es leid, ſo ſelten und ſo wenig von {\pb}Dir zu
               hören.\pend
           \pstart
           Viele Grüße an die beiden Mädchen\pwindex{Schnitzler, Olga 17.01.1882 – 13.01.1970@\textsc{Schnitzler, Olga} (17.01.1882 – 13.01.1970), \emph{Schauspielerin, Sängerin}|pwv}\pwindex{Steinrueck, Elisabeth 19.11.1885 – 07.04.1920@\textsc{Steinrück, Elisabeth} (19.11.1885 – 07.04.1920)|pwv} und an Dich! {\\[\baselineskip]}Dein {\\[\baselineskip]}\spacefill\mbox{Paul Goldmann.}\pend
           \leftskip=0em{}
         
         \endnumbering\mylabel{h}\end{ledgroupsized}  \newcommand{\dateiname}{L03089}\newcommand{\titel}{Paul Goldmann an Arthur Schnitzler, 18. 10. [1901]}\newcommand{\editorInnen}{Martin Anton Müller und Laura Untner}%% latex-leseansicht-abspann.tex
%% Abspann für die Leseansicht.
%% Der Schalter \ifkorrekturansicht ist bereits durch den Vorspann gesetzt.

%% latex-abspann.tex
%% Gemeinsamer Abspann für Korrekturansicht und Leseansicht.
%% Setzt den Schalter \ifkorrekturansicht voraus (gesetzt in den
%% einbindenden Dateien latex-korrekturansicht-abspann.tex bzw.
%% latex-leseansicht-abspann.tex).
%% ---------------------------------------------------------------

\normalsize

% Das esempio-Environment wird nur in der Leseansicht benötigt
\ifkorrekturansicht\else
\newenvironment{esempio}[3]%
{
    \vspace{1.5ex}
    \rlap{\underline{#1}}
    \par
    \setlength{\parindent}{0cm}
    \nopagebreak
    \leftskip=#2cm
    \rightskip=#3cm
}
{
    \par
}
\fi

\doendnotes{C}
\bigskip
\vfill

\clearpage

\footnotesize

\ifkorrekturansicht
  \lohead{\textsc{register}}
\fi

% theindex-Environment neu definieren ohne reledmac
\makeatletter
\renewenvironment{theindex}{%
  \ifkorrekturansicht
    \section*{\indexname}%
  \else
    \subsubsection*{Index der erwähnten Entitäten}%
  \fi
  \setlength{\parindent}{0pt}%
  \setlength{\parskip}{0pt plus 0.3pt}%
  \let\item\@idxitem
}{%
  \ifkorrekturansicht\clearpage\fi
}
\makeatother

\IfFileExists{\jobname-pw.ind}{\input{\jobname-pw.ind}}{}

% Quellenangabe nur in der Leseansicht
\ifkorrekturansicht\else
% Fallback-Definitionen, falls die .tex-Datei \titel etc. nicht gesetzt hat
\providecommand{\titel}{}
\providecommand{\editorInnen}{}
\providecommand{\dateiname}{\jobname}

\vspace{3cm}

\vfill

\footnotesize
\textsc{Quelle}: \titel. Herausgegeben von {\editorInnen}. In: \emph{Arthur Schnitzler: Briefwechsel mit Autorinnen und Autoren}.
 Digitale Edition, https://schnitzler-briefe.acdh.oeaw.ac.at/{\dateiname}.html (Stand \today)
\fi

\end{document}


      