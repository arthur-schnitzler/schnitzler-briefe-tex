%% latex-leseansicht-vorspann.tex
%% Vorspann für die Leseansicht.
%% Lädt die gemeinsame Datei latex-vorspann.tex mit nicht gesetztem Schalter.

\newif\ifkorrekturansicht
\korrekturansichtfalse

\input{../tex-inputs/latex-vorspann}


\section[Arthur Schnitzler an Richard Beer-Hofmann, 11. 5. 1905]{L01516 Arthur Schnitzler an Richard Beer-Hofmann, 11. 5. 1905}
\nopagebreak\mylabel{L01516v}
\rehead{ }\normalsize\beginnumbering\briefempfaengerindex{Beer-Hofmann, Richard@\textsc{Beer-Hofmann, Richard}!zzzSchnitzler, Arthur@\emph{von Arthur Schnitzler}!1905-05-111@{11. 5. 1905}|(be}
\toendnotes[C]{\smallbreak\pagebreak[2]}
\correspDesc{Versand  durch Arthur Schnitzler am 11. 5. 1905 in Wien
\newline{}Erhalt  durch Richard Beer-Hofmann am 11. 5. 1905 in Rodaun}\toendnotes[C]{\smallbreak}
\Standort{YCGL, MSS 31.}
\physDesc{Brief, 1 Blatt, 2 Seiten, Kuvert, 337 Zeichen
\newline{}Handschrift: Bleistift, deutsche Kurrent
\newline{}Versand: 1) Stempel: »\nobreak{}\oindex{XVIII., Währing@\textbf{XVIII., Währing}, \emph{Verwaltungsgebiet}|pwk}Wien 1\textcolor{gray}{8}/1, 11. V. 05\nobreak{}«.   2) Stempel: »\nobreak{}\oindex{Wien@\textbf{Wien}!XXIII., Liesing@\textbf{XXIII., Liesing}!Rodaun@\textbf{Rodaun}, \emph{Region}|pwk}{\pb}Ro{[}da{]}un, 11. 5. 05, 12–2N\nobreak{}«. }\toendnotes[C]{\smallbreak}\pstart{}{\pb}\textcolor{gray}{\textbf{Dr. Arthur Schnitzler}}\pend{}\pstart{}\textcolor{gray}{\textbf{Wien, XVIII Spoettelgasse 7\oindex{Wien@\textbf{Wien}!XVIII., Währing@\textbf{XVIII., Währing}!Edmund-Weiß-Gasse 7@\textbf{Edmund-Weiß-Gasse 7}, \emph{Wohngebäude}|pw}.}}\pend{}{\bigskip}\pstart{}{\pb}\textsc{Herrn Dr Rich Beer-Hofma{\geminationn}}\pend{}\pstart{}Rodaun\oindex{Wien@\textbf{Wien}!XXIII., Liesing@\textbf{XXIII., Liesing}!Rodaun@\textbf{Rodaun}, \emph{Region}|pw}\pend{}\pstart{}\textsc{Liesingerstr 2}\oindex{Liesingerstraße@\textbf{Liesingerstraße}, \emph{Straße}|pw}\pend{}{\bigskip}\vspace{1em}
\pstart
           \raggedleft{}{\pb}11/5 905\pend
           
\pstart{}lieber Richard,\pend\vspace{0.5em}
\pstart
           ich erfahre eben von den wahnwitzigen \label{K_L01516-1v}\edtext{Preiſen bei \textsc{Reinhardt}\pwindex{Reinhardt, Max 9.\,9.\,1873 Baden bei Wien – 30.\,10.\,1943 New York City@\textsc{Reinhardt, Max} (9.\,9.\,1873 Baden bei Wien – 30.\,10.\,1943 New York City), \emph{Theaterleiter, Regisseur, Schauspieler}|pw}}{\lemma{\textnormal{\emph{Preisen bei Reinhardt}}}\Cendnote{\textnormal{Max Reinhardt\pwindex{Reinhardt, Max 9.\,9.\,1873 Baden bei Wien – 30.\,10.\,1943 New York City@\textsc{Reinhardt, Max} (9.\,9.\,1873 Baden bei Wien – 30.\,10.\,1943 New York City), \emph{Theaterleiter, Regisseur, Schauspieler}|pwk} kam mit seinen Bühnen \emph{Kleines
                     Theater}\orgindex{Kleines Theater@Kleines Theater|pwk} und \emph{Neues Theater}\orgindex{Neues Theater@Neues Theater|pwk} für ein Gastspiel 
                  an das Theater an der Wien\oindex{Wien@\textbf{Wien}!VI., Mariahilf@\textbf{VI., Mariahilf}!Theater an der Wien@\textbf{Theater an der Wien}, \emph{Theater}|pwk}. Am 15. 5. 1905
                  wurde \emph{Der Graf von Charolais}\pwindex{Beer-Hofmann, Richard 11.\,7.\,1866 Wien – 26.\,9.\,1945 New York City@\textsc{Beer-Hofmann, Richard} (11.\,7.\,1866 Wien – 26.\,9.\,1945 New York City), \emph{Schriftsteller}!Graf von Charolais. Ein Trauerspiel@\strich\emph{Der Graf von Charolais. Ein Trauerspiel}|pwk} gegeben. Schnitzler
                  saß im Publikum.}}}\label{K_L01516-1}. Alſo
               bitte (we{\geminationn} Sie{ }ſo gütig{ }ſind mir zu beſtellen) nicht
               1. Reihe Orcheſter{ }ſondern Parket vorn{ }ſehr vorn. Ecke unbedingt. Iſt die Beſtellung\pwindex{Beer-Hofmann, Richard 11.\,7.\,1866 Wien – 26.\,9.\,1945 New York City@\textsc{Beer-Hofmann, Richard} (11.\,7.\,1866 Wien – 26.\,9.\,1945 New York City), \emph{Schriftsteller}!Graf von Charolais. Ein Trauerspiel@\strich\emph{Der Graf von Charolais. Ein Trauerspiel}|pwv}{ }ſchon \substVorne{}\textsuperscript{verfügt}\substDazwischen{}erfolgt\substHinten{},{ }ſo bitte \uline{nichts}{ }{\pb}mehr zu verfügen. –\pend
           
\pstart
           Herzlichſt{\\[\baselineskip]}Ihr{\\[\baselineskip]}\spacefill\mbox{A.}\pend
           \leftskip=0em{}\selectlanguage{ngerman}\endnumbering\briefempfaengerindex{Beer-Hofmann, Richard@\textsc{Beer-Hofmann, Richard}!zzzSchnitzler, Arthur@\emph{von Arthur Schnitzler}!1905-05-111@{11. 5. 1905}|)be}\mylabel{L01516h}  \newcommand{\dateiname}{L01516}\newcommand{\titel}{Arthur Schnitzler an Richard Beer-Hofmann, 11. 5. 1905}\newcommand{\editorInnen}{Martin Anton Müller und Gerd-Hermann Susen}%% latex-leseansicht-abspann.tex
%% Abspann für die Leseansicht.
%% Der Schalter \ifkorrekturansicht ist bereits durch den Vorspann gesetzt.

%% latex-abspann.tex
%% Gemeinsamer Abspann für Korrekturansicht und Leseansicht.
%% Setzt den Schalter \ifkorrekturansicht voraus (gesetzt in den
%% einbindenden Dateien latex-korrekturansicht-abspann.tex bzw.
%% latex-leseansicht-abspann.tex).
%% ---------------------------------------------------------------

\normalsize

% Das esempio-Environment wird nur in der Leseansicht benötigt
\ifkorrekturansicht\else
\newenvironment{esempio}[3]%
{
    \vspace{1.5ex}
    \rlap{\underline{#1}}
    \par
    \setlength{\parindent}{0cm}
    \nopagebreak
    \leftskip=#2cm
    \rightskip=#3cm
}
{
    \par
}
\fi

\doendnotes{C}
\bigskip
\vfill

\clearpage

\footnotesize

\ifkorrekturansicht
  \lohead{\textsc{register}}
\fi

% theindex-Environment neu definieren ohne reledmac
\makeatletter
\renewenvironment{theindex}{%
  \ifkorrekturansicht
    \section*{\indexname}%
  \else
    \subsubsection*{Index der erwähnten Entitäten}%
  \fi
  \setlength{\parindent}{0pt}%
  \setlength{\parskip}{0pt plus 0.3pt}%
  \let\item\@idxitem
}{%
  \ifkorrekturansicht\clearpage\fi
}
\makeatother

\IfFileExists{\jobname-pw.ind}{\input{\jobname-pw.ind}}{}

% Quellenangabe nur in der Leseansicht
\ifkorrekturansicht\else
% Fallback-Definitionen, falls die .tex-Datei \titel etc. nicht gesetzt hat
\providecommand{\titel}{}
\providecommand{\editorInnen}{}
\providecommand{\dateiname}{\jobname}

\vspace{3cm}

\vfill

\footnotesize
\textsc{Quelle}: \titel. Herausgegeben von {\editorInnen}. In: \emph{Arthur Schnitzler: Briefwechsel mit Autorinnen und Autoren}.
 Digitale Edition, https://schnitzler-briefe.acdh.oeaw.ac.at/{\dateiname}.html (Stand \today)
\fi

\end{document}


