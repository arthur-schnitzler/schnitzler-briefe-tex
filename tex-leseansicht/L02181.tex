%% latex-korrekturansicht-vorspann.tex
%% Vorspann für die Korrekturansicht.
%% Lädt die gemeinsame Datei latex-vorspann.tex mit gesetztem Schalter.

\newif\ifkorrekturansicht
\korrekturansichttrue

\input{../tex-inputs/latex-vorspann}


\section[Arthur Schnitzler an Hermann Bahr, 12. 6. 1914]{L02181 Arthur Schnitzler an Hermann Bahr, 12. 6. 1914}
\nopagebreak\mylabel{L02181v}
\rehead{ }\normalsize\beginnumbering\briefempfaengerindex{Bahr, Hermann@\textsc{Bahr, Hermann}!zzzSchnitzler, Arthur@\emph{von Arthur Schnitzler}!1914-06-121@{12. 6. 1914}|(be}
\toendnotes[C]{\smallbreak\pagebreak[2]}\Standort{TMW, HS AM 23395 Ba.}
\physDesc{Brief, 1 Blatt, 2 Seiten, 996 Zeichen
\newline{}Schreibmaschine
\newline{}Handschrift: schwarze Tinte (\noindent{}Korrekturen, Unterschrift)}\Standort{DLA, A:Schnitzler, 85.1.294/5.}
\physDesc{Brief, Durchschlag2 Blätter, 2 Seiten, 996 Zeichen
\newline{}Schreibmaschine}
\buchAbdrucke{\weitereDrucke{1) Arthur Schnitzler: \emph{Briefe 1913–1931}. Frankfurt am Main: \emph{S. Fischer} 1984, S. 43.} \weitereDrucke{2) Arthur Schnitzler: \emph{The Letters of Arthur Schnitzler to Hermann Bahr}. Chapel Hill: \emph{The University of North Carolina Press} 1978, S. 113.} \weitereDrucke{3) Hermann Bahr, Arthur Schnitzler: \emph{Briefwechsel, Aufzeichnungen, Dokumente (1891–1931)}. Göttingen: \emph{Wallstein} 2018, S. 494.} }\toendnotes[C]{\smallbreak}
\pstart
           {\pb}\textcolor{gray}{\textbf{Dr. Arthur Schnitzler}}\hfill 12. 6. 1914. \pend
           
\pstart
           \textcolor{gray}{\textbf{Wien XVIII. Sternwartestrasse 71\oindex{Sternwartestrasse 71@\textbf{Sternwartestraße 71}, \emph{Wohngebäude (K.WHS)}|pw}}}\pend
           
\pstart{}Lieber Hermann.\pend\vspace{0.5em}
\pstart
           Wie Dir ja bekannt ist war der »Reigen\pwindex{Reigen. Zehn Dialoge@\emph{Reigen. Zehn Dialoge}|pw}« bisher
               in Deutschland\oindex{Deutschland@\textbf{Deutschland}, \emph{A.PCLI}|pw} ein verbotenes Buch. Nun soll von
               dem Verlag J. Singer {\kaufmannsund}
                  Co.\orgindex{J. Singer und Co.@J. Singer {\kaufmannsund}  Co.|pw}, Berlin\oindex{Berlin@\textbf{Berlin}, \emph{P.PPLC}|pw}, eine \label{K_L02181-1v}\edtext{Neuauflage}{\lemma{\textnormal{\emph{Neuauflage}}}\Cendnote{\textnormal{Es handelte sich um eine Titelauflage der Erstausgabe des \emph{Wiener Verlags}\orgindex{Wiener Verlag@Wiener Verlag|pwk}. Diese erschien ohne Jahresangabe. 
                     Für den Textblock wurden die Seiten des ursprünglichen Druckes verwendet, die mit neu gedruckten Titelseiten und
                     neuem
                  Umschlag versehen waren. Selbst die Verlagswerbung deutet auf das ursprüngliche
                  Erscheinen (»Im gleichen Verlag erscheint von Arthur Schnitzler«), ebenso die Hinweise auf die Auflage: »44. –46.
                  Tausend«.}}}\label{K_L02181-1} veröffentlicht werden, deren Beschlagnahme vorauszusehen
               ist, und es kommt dem Verlag darauf an bei einem eventuell bevorstehenden Prozess
               etliche \label{K_L02181-2v}\edtext{Gutachten}{\lemma{\textnormal{\emph{Gutachten}}}\Cendnote{\textnormal{Die Briefe der Genannten und ein weiterer
                  von Maximilian Harden\pwindex{Harden, Maximilian 20.10.1861 – 30.10.1927@\textsc{Harden, Maximilian} (20.10.1861 – 30.10.1927), \emph{Schriftsteller/Schriftstellerin, Publizist/Publizistin}|pwk} finden sich in der
                  Mappe B 128 in der \emph{Cambridge University Library}
                     (»Opinions on Reigen«).}}}\label{K_L02181-2} zur Verfügung zu haben. Solche von Liszt\pwindex{Liszt, Franz von 02.03.1851 – 21.06.1919@\textsc{Liszt, Franz von} (02.03.1851 – 21.06.1919), \emph{Rechtswissenschaftler/Rechtswissenschaftlerin}|pw}, Lilienthal\pwindex{Lilienthal, Karl von 31.08.1853 – 08.11.1927@\textsc{Lilienthal, Karl von} (31.08.1853 – 08.11.1927), \emph{Rechtswissenschaftler/Rechtswissenschaftlerin}|pw}, Eulenburg\pwindex{Eulenberg, Herbert 25.01.1876 – 04.09.1949@\textsc{Eulenberg, Herbert} (25.01.1876 – 04.09.1949), \emph{Schriftsteller/Schriftstellerin}|pw}, Simmel\pwindex{Simmel, Georg 01.03.1858 – 26.09.1918@\textsc{Simmel, Georg} (01.03.1858 – 26.09.1918), \emph{Philosoph/Philosophin, Soziologe/Soziologin}|pw}, Liebermann\pwindex{Liebermann, Max 20.07.1847 – 08.02.1935@\textsc{Liebermann, Max} (20.07.1847 – 08.02.1935), \emph{Maler/Malerin, Maler/Malerin, Maler/Malerin}|pw}, Fulda\pwindex{Fulda, Ludwig 15.07.1862 – 30.03.1939@\textsc{Fulda, Ludwig} (15.07.1862 – 30.03.1939), \emph{Schriftsteller/Schriftstellerin, Übersetzer/Übersetzerin}|pw} liegen schon vor
               (in zum Teil ganz überraschend günstigem Sinne, muss ich sagen); und da der Verlag
               doch gern auch aus Oesterreich\oindex{Oesterreich@\textbf{Österreich}, \emph{A.PCLI}|pw} etwas in der Art
               möchte vorweisen können, so fiel mir ein, dass vor Jahren, als dir einmal die
               öffentliche Vorlesung des »Reigen\pwindex{Reigen. Zehn Dialoge@\emph{Reigen. Zehn Dialoge}|pw}« untersagt
               wurde, Burckhardt\pwindex{Burckhard, Max Eugen 14.07.1854 – 16.03.1912@\textsc{Burckhard, Max Eugen} (14.07.1854 – 16.03.1912), \emph{Schriftsteller/Schriftstellerin, Rechtswissenschaftler/Rechtswissenschaftlerin, Theaterleiter/Theaterleiterin}|pw} einen Rekurs eingebracht
               hat, der sich vielleicht noch in Deinem Besitze finden mag. Ich frage Dich nun, ob Du
               dem Verlag J. Singer\orgindex{J. Singer und Co.@J. Singer {\kaufmannsund}  Co.|pw}, wenn er sich {\pb}mit entsprechender
               Bitte an Dich wenden sollte, jenes Schriftstück zu eventueller Benützung vor Gericht
               auszufolgen geneigt wärest? \pend
           
\pstart
           Mit herzlichem Gruss{\\[\baselineskip]}Dein{\\[\baselineskip]}\spacefill\mbox{{[}hs.:{]} Arthur}\pend
           \leftskip=0em{}\selectlanguage{ngerman}\endnumbering\briefempfaengerindex{Bahr, Hermann@\textsc{Bahr, Hermann}!zzzSchnitzler, Arthur@\emph{von Arthur Schnitzler}!1914-06-121@{12. 6. 1914}|)be}\mylabel{L02181h}  \normalsize

\doendnotes{C}
\bigskip
\vfill

\clearpage

\footnotesize

\lohead{\textsc{register}}

% Definiere theindex-Environment komplett neu ohne reledmac
\makeatletter
\renewenvironment{theindex}{%
  \section*{\indexname}%
  \setlength{\parindent}{0pt}%
  \setlength{\parskip}{0pt plus 0.3pt}%
  \let\item\@idxitem
}{%
  \clearpage
}
\makeatother

\IfFileExists{\jobname-pw.ind}{\input{\jobname-pw.ind}}{}

\end{document}

      