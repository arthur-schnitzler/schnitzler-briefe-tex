%% latex-leseansicht-vorspann.tex
%% Vorspann für die Leseansicht.
%% Lädt die gemeinsame Datei latex-vorspann.tex mit nicht gesetztem Schalter.

\newif\ifkorrekturansicht
\korrekturansichtfalse

\input{../tex-inputs/latex-vorspann}


         
         \renewcommand{\erwaehntePersonen}{Personen: Hermann Bahr, Max Eugen Burckhard, Herbert Eulenberg, Ludwig Fulda, Maximilian Harden, Max Liebermann, Karl von Lilienthal, Franz von Liszt, Georg Simmel}
         \renewcommand{\erwaehnteInstitutionen}{Institutionen: J. Singer & Co., Wiener Verlag}
         \renewcommand{\erwaehnteOrte}{Orte: Berlin, Deutschland, Sternwartestraße, Wien, Österreich}
         \renewcommand{\erwaehnteWerke}{Werke: Reigen. Zehn Dialoge}
               \section[Arthur Schnitzler an Hermann Bahr, 12. 6. 1914]{ Arthur Schnitzler an Hermann Bahr, 12. 6. 1914}\nopagebreak\mylabel{v}\rehead{ }\begin{ledgroupsized}[t]{13cm}\normalsize\beginnumbering \toendnotes[C]{\smallbreak\pagebreak[2]} \Standort{TMW, HS AM 23395 Ba.}
\physDesc{Brief, 1 Blatt, 2 Seiten
\newline{}Schreibmaschine
\newline{}Handschrift: schwarze Tinte (\noindent{}Korrekturen, Unterschrift)}\Standort{DLA, A:Schnitzler, 85.1.294/5.}
\physDesc{Brief, maschineller Durchschlag
\newline{}Schreibmaschine}\buchAbdrucke{\weitereDrucke{1) Arthur Schnitzler: \emph{Briefe 1913–1931}. Hg. Peter Michael Braunwarth, Richard Miklin, Susanne Pertlik und Heinrich Schnitzler. Frankfurt am Main: \emph{S. Fischer} 1984, S. 43.} \weitereDrucke{2) \emph{12. 6. 1914.} In: Arthur Schnitzler: \emph{The Letters of Arthur Schnitzler to Hermann Bahr}. Edited, annotated, and with an introduction, by Donald G.
                        Daviau. Chapel Hill: \emph{The University of North Carolina Press} 1978, S. 113 (University of North Carolina studies in the Germanic languages
                        and literatures, 89).} \weitereDrucke{3) Hermann Bahr, Arthur Schnitzler: \emph{Briefwechsel, Aufzeichnungen, Dokumente (1891–1931)}. Hg. Kurt Ifkovits und Martin Anton Müller. Göttingen: \emph{Wallstein} 2018, S. 494.} }\toendnotes[C]{\smallbreak}\pstart
           \noindent{}{\pb}\textcolor{gray}{\textbf{Dr. Arthur Schnitzler}}\hfill 12. 6. 1914. \pend
           \pstart
           \textcolor{gray}{\textbf{Wien XVIII. Sternwartestrasse 71\oindex{Sternwartestrasse@\textbf{Sternwartestraße}|pw}}}\pend
           \pstart{}Lieber Hermann.\pend\pstart
           Wie Dir ja bekannt ist war der »Reigen\pwindex{Schnitzler, Arthur 15.05.1862 – 21.10.1931@\textsc{Schnitzler, Arthur} (15.05.1862 – 21.10.1931), \emph{Schriftsteller, Mediziner}!Reigen. Zehn Dialoge1900@\strich\emph{Reigen. Zehn Dialoge} {[}1900{]}|pw}« bisher in
                  Deutschland\oindex{Deutschland@\textbf{Deutschland}|pw} ein verbotenes Buch. Nun soll von dem
               Verlag J. Singer {\kaufmannsund} Co.\orgindex{J. Singer und Co.@J. Singer {\kaufmannsund}  Co.|pw},
                  Berlin\oindex{Berlin@\textbf{Berlin}|pw}, eine \label{K_L02181_1v}\edtext{Neuauflage}{\lemma{\textnormal{\emph{Neuauflage}}}\Cendnote{\textnormal{Eine Titelauflage der Erstausgabe im \emph{Wiener Verlag}\orgindex{Wiener Verlag@Wiener Verlag|pwk},
                  erschienen ohne Jahresangabe. Das heißt, dass Seiten des ursprünglichen Druckes verwendet werden, aber mit einem
                  neuen Titel und Umschlag versehen sind.
                  Selbst die Verlagswerbung deutet auf das
                  ursprüngliche Erscheinen (»Im gleichen Verlag erscheint von Arthur Schnitzler\pwindex{Schnitzler, Arthur 15.05.1862 – 21.10.1931@\textsc{Schnitzler, Arthur} (15.05.1862 – 21.10.1931), \emph{Schriftsteller, Mediziner}|pw}«), ebenso die Hinweise auf die Auflage: »44.–46.
                  Tausend«.}}}\label{K_L02181_1h} veröffentlicht werden, deren Beschlagnahme vorauszusehen
               ist, und es kommt dem Verlag darauf an bei einem eventuell bevorstehenden Prozess
               etliche \label{K_L02181_2v}\edtext{Gutachten}{\lemma{\textnormal{\emph{Gutachten}}}\Cendnote{\textnormal{Die Briefe der Genannten und ein weiterer
                  von Maximilian Harden\pwindex{Harden, Maximilian 20.10.1861 – 30.10.1927@\textsc{Harden, Maximilian} (20.10.1861 – 30.10.1927), \emph{Schriftsteller, Publizist}|pwk} finden sich in der Mappe
                     B 128 in der \emph{Cambridge University Library}
                     (»Opinions on Reigen«).}}}\label{K_L02181_2h} zur Verfügung zu haben. Solche von Liszt\pwindex{Liszt, Franz von 02.03.1851 – 21.06.1919@\textsc{Liszt, Franz von} (02.03.1851 – 21.06.1919), \emph{Rechtswissenschaftler}|pw}, Lilienthal\pwindex{Lilienthal, Karl von 31.08.1853 – 08.11.1927@\textsc{Lilienthal, Karl von} (31.08.1853 – 08.11.1927), \emph{Rechtswissenschaftler}|pw}, Eulenburg\pwindex{Eulenberg, Herbert 25.01.1876 – 04.09.1949@\textsc{Eulenberg, Herbert} (25.01.1876 – 04.09.1949), \emph{Schriftsteller}|pw}, Simmel\pwindex{Simmel, Georg 01.03.1858 – 26.09.1918@\textsc{Simmel, Georg} (01.03.1858 – 26.09.1918), \emph{Philosoph, Soziologe}|pw}, Liebermann\pwindex{Liebermann, Max 20.07.1847 – 08.02.1935@\textsc{Liebermann, Max} (20.07.1847 – 08.02.1935), \emph{Maler}|pw}, Fulda\pwindex{Fulda, Ludwig 15.07.1862 – 30.03.1939@\textsc{Fulda, Ludwig} (15.07.1862 – 30.03.1939), \emph{Schriftsteller, Übersetzer}|pw} liegen schon vor (in zum Teil ganz
               überraschend günstigem Sinne, muss ich sagen); und da der Verlag doch gern auch aus
                  Oesterreich\oindex{Oesterreich@\textbf{Österreich}|pw} etwas in der Art möchte vorweisen
               können, so fiel mir ein, dass vor Jahren, als dir einmal die öffentliche Vorlesung
               des »Reigen\pwindex{Schnitzler, Arthur 15.05.1862 – 21.10.1931@\textsc{Schnitzler, Arthur} (15.05.1862 – 21.10.1931), \emph{Schriftsteller, Mediziner}!Reigen. Zehn Dialoge1900@\strich\emph{Reigen. Zehn Dialoge} {[}1900{]}|pw}« untersagt wurde, Burckhardt\pwindex{Burckhard, Max Eugen 14.07.1854 – 16.03.1912@\textsc{Burckhard, Max Eugen} (14.07.1854 – 16.03.1912), \emph{Schriftsteller, Rechtswissenschaftler, Theaterleiter}|pw} einen Rekurs eingebracht hat, der sich vielleicht
               noch in Deinem Besitze finden mag. Ich frage Dich nun, ob Du dem Verlag J. Singer\orgindex{J. Singer und Co.@J. Singer {\kaufmannsund}  Co.|pw}, wenn er sich {\pb}mit entsprechender
               Bitte an Dich wenden sollte, jenes Schriftstück zu eventueller Benützung vor Gericht
               auszufolgen geneigt wärest? \pend
           \pstart
           Mit herzlichem Gruss{\\[\baselineskip]}Dein{\\[\baselineskip]}\spacefill\mbox{{[}hs.:{]} Arthur}\pend
           \leftskip=0em{}
         
         \endnumbering\mylabel{h}\end{ledgroupsized}  \newcommand{\dateiname}{L02181}\newcommand{\titel}{Arthur Schnitzler an Hermann Bahr, 12. 6. 1914}\newcommand{\editorInnen}{ Kurt Ifkovits,  Martin Anton Müller}%% latex-leseansicht-abspann.tex
%% Abspann für die Leseansicht.
%% Der Schalter \ifkorrekturansicht ist bereits durch den Vorspann gesetzt.

%% latex-abspann.tex
%% Gemeinsamer Abspann für Korrekturansicht und Leseansicht.
%% Setzt den Schalter \ifkorrekturansicht voraus (gesetzt in den
%% einbindenden Dateien latex-korrekturansicht-abspann.tex bzw.
%% latex-leseansicht-abspann.tex).
%% ---------------------------------------------------------------

\normalsize

% Das esempio-Environment wird nur in der Leseansicht benötigt
\ifkorrekturansicht\else
\newenvironment{esempio}[3]%
{
    \vspace{1.5ex}
    \rlap{\underline{#1}}
    \par
    \setlength{\parindent}{0cm}
    \nopagebreak
    \leftskip=#2cm
    \rightskip=#3cm
}
{
    \par
}
\fi

\doendnotes{C}
\bigskip
\vfill

\clearpage

\footnotesize

\ifkorrekturansicht
  \lohead{\textsc{register}}
\fi

% theindex-Environment neu definieren ohne reledmac
\makeatletter
\renewenvironment{theindex}{%
  \ifkorrekturansicht
    \section*{\indexname}%
  \else
    \subsubsection*{Index der erwähnten Entitäten}%
  \fi
  \setlength{\parindent}{0pt}%
  \setlength{\parskip}{0pt plus 0.3pt}%
  \let\item\@idxitem
}{%
  \ifkorrekturansicht\clearpage\fi
}
\makeatother

\IfFileExists{\jobname-pw.ind}{\input{\jobname-pw.ind}}{}

% Quellenangabe nur in der Leseansicht
\ifkorrekturansicht\else
% Fallback-Definitionen, falls die .tex-Datei \titel etc. nicht gesetzt hat
\providecommand{\titel}{}
\providecommand{\editorInnen}{}
\providecommand{\dateiname}{\jobname}

\vspace{3cm}

\vfill

\footnotesize
\textsc{Quelle}: \titel. Herausgegeben von {\editorInnen}. In: \emph{Arthur Schnitzler: Briefwechsel mit Autorinnen und Autoren}.
 Digitale Edition, https://schnitzler-briefe.acdh.oeaw.ac.at/{\dateiname}.html (Stand \today)
\fi

\end{document}


      