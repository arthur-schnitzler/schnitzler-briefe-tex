%% latex-leseansicht-vorspann.tex
%% Vorspann für die Leseansicht.
%% Lädt die gemeinsame Datei latex-vorspann.tex mit nicht gesetztem Schalter.

\newif\ifkorrekturansicht
\korrekturansichtfalse

\input{../tex-inputs/latex-vorspann}


\section[Arthur Schnitzler an Hermann Bahr, 12. 6. 1914]{L02181 Arthur Schnitzler an Hermann Bahr, 12. 6. 1914}
\nopagebreak\mylabel{L02181v}
\rehead{ }\normalsize\beginnumbering\briefempfaengerindex{Bahr, Hermann@\textsc{Bahr, Hermann}!zzzSchnitzler, Arthur@\emph{von Arthur Schnitzler}!1914-06-121@{12. 6. 1914}|(be}
\toendnotes[C]{\smallbreak\pagebreak[2]}
\correspDesc{Versand  durch Arthur Schnitzler am 12. 6. 1914 in Wien
\newline{}Erhalt  durch Hermann Bahr im Zeitraum [12. 6. 1914
                  – 16. 6. 1914?] \textbf{Ort fehlend} }\toendnotes[C]{\smallbreak}
\Standort{TMW, HS AM 23395 Ba.}
\physDesc{Brief, 1 Blatt, 2 Seiten, 996 Zeichen
\newline{}Schreibmaschine
\newline{}Handschrift: schwarze Tinte (\noindent{}Korrekturen, Unterschrift)}\Standort{DLA, A:Schnitzler, 85.1.294/5.}
\physDesc{Brief, Durchschlag, 2 Blätter, 2 Seiten, 996 Zeichen
\newline{}Schreibmaschine}
\buchAbdrucke{\weitereDrucke{1) Arthur Schnitzler: \emph{Briefe 1913–1931}. Herausgegeben von Peter Michael Braunwarth, Richard Miklin, Susanne Pertlik und Heinrich Schnitzler. Frankfurt am Main: \emph{S. Fischer} 1984, S. 43.} \weitereDrucke{2) \emph{12. 6. 1914.} In: Arthur Schnitzler: \emph{The Letters of Arthur Schnitzler to Hermann Bahr}. Edited, annotated, and with an introduction, by Donald G. Daviau. Chapel Hill: \emph{The University of North Carolina Press} 1978, S. 113 (University of North Carolina studies in the Germanic languages
                        and literatures, 89).} \weitereDrucke{3) Hermann Bahr, Arthur Schnitzler: \emph{Briefwechsel, Aufzeichnungen, Dokumente (1891–1931)}. Herausgegeben von Kurt Ifkovits und Martin Anton Müller. Göttingen: \emph{Wallstein} 2018, S. 494.} }\toendnotes[C]{\smallbreak}
\pstart
           {\pb}\textcolor{gray}{\textbf{Dr. Arthur Schnitzler}}\hfill 12. 6. 1914.\pend
           
\pstart
           \textcolor{gray}{\textbf{Wien XVIII. Sternwartestrasse 71\oindex{Wien@\textbf{Wien}!XVIII., Währing@\textbf{XVIII., Währing}!Sternwartestraße 71@\textbf{Sternwartestraße 71}, \emph{Wohngebäude}|pw}}}\pend
           
\pstart{}Lieber Hermann.\pend\vspace{0.5em}
\pstart
           Wie Dir ja bekannt ist war der »Reigen\pwindex{Schnitzler, Arthur 15.\,5.\,1862 Wien – 21.\,10.\,1931 ebd.@\textsc{Schnitzler, Arthur} (15.\,5.\,1862 Wien – 21.\,10.\,1931 ebd.), \emph{Schriftsteller, Mediziner}!Reigen. Zehn Dialoge@\strich\emph{Reigen. Zehn Dialoge}|pw}« bisher
               in Deutschland\oindex{Deutschland@\textbf{Deutschland}|pw} ein verbotenes Buch. Nun soll von
               dem Verlag J. Singer {\kaufmannsund}
                  Co.\orgindex{J. Singer und Co.@J. Singer {\kaufmannsund}  Co.|pw}, Berlin\oindex{Berlin@\textbf{Berlin}, \emph{Hauptstadt}|pw}, eine \label{K_L02181-1v}\edtext{Neuauflage}{\lemma{\textnormal{\emph{Neuauflage}}}\Cendnote{\textnormal{Es handelte sich um eine Titelauflage der Erstausgabe des \emph{Wiener Verlags}\orgindex{Wiener Verlag@Wiener Verlag|pwk}. Diese erschien ohne Jahresangabe. 
                     Für den Textblock wurden die Seiten des ursprünglichen Druckes verwendet, die mit neu gedruckten Titelseiten und
                     neuem
                  Umschlag versehen waren. Selbst die Verlagswerbung deutet auf das ursprüngliche
                  Erscheinen (»Im gleichen Verlag erscheint von Arthur Schnitzler«), ebenso die Hinweise auf die Auflage: »44. –46.
                  Tausend«.}}}\label{K_L02181-1} veröffentlicht werden, deren Beschlagnahme vorauszusehen
               ist, und es kommt dem Verlag darauf an bei einem eventuell bevorstehenden Prozess
               etliche \label{K_L02181-2v}\edtext{Gutachten}{\lemma{\textnormal{\emph{Gutachten}}}\Cendnote{\textnormal{Die Briefe der Genannten und ein weiterer
                  von Maximilian Harden\pwindex{Harden, Maximilian 20.\,10.\,1861 Berlin – 30.\,10.\,1927 Montana@\textsc{Harden, Maximilian} (20.\,10.\,1861 Berlin – 30.\,10.\,1927 Montana), \emph{Schriftsteller, Publizist}|pwk} finden sich in der
                  Mappe B 128 in der \emph{Cambridge University Library}
                     (»Opinions on Reigen«).}}}\label{K_L02181-2} zur Verfügung zu haben. Solche von Liszt\pwindex{Liszt, Franz von 2.\,3.\,1851 Wien – 21.\,6.\,1919 Seeheim@\textsc{Liszt, Franz von} (2.\,3.\,1851 Wien – 21.\,6.\,1919 Seeheim), \emph{Rechtswissenschaftler}|pw}, Lilienthal\pwindex{Lilienthal, Karl von 31.\,8.\,1853 Elberfeld – 8.\,11.\,1927 Heidelberg@\textsc{Lilienthal, Karl von} (31.\,8.\,1853 Elberfeld – 8.\,11.\,1927 Heidelberg), \emph{Rechtswissenschaftler}|pw}, Eulenburg\pwindex{Eulenberg, Herbert 25.\,1.\,1876 Mülheim [Köln] – 4.\,9.\,1949 Düsseldorf@\textsc{Eulenberg, Herbert} (25.\,1.\,1876 Mülheim [Köln] – 4.\,9.\,1949 Düsseldorf), \emph{Schriftsteller}|pw}, Simmel\pwindex{Simmel, Georg 1.\,3.\,1858 Berlin – 26.\,9.\,1918 Straßburg@\textsc{Simmel, Georg} (1.\,3.\,1858 Berlin – 26.\,9.\,1918 Straßburg), \emph{Philosoph, Soziologe}|pw}, Liebermann\pwindex{Liebermann, Max 20.\,7.\,1847 Berlin – 8.\,2.\,1935 ebd.@\textsc{Liebermann, Max} (20.\,7.\,1847 Berlin – 8.\,2.\,1935 ebd.), \emph{Maler, Maler, Maler}|pw}, Fulda\pwindex{Fulda, Ludwig 15.\,7.\,1862 Frankfurt am Main – 30.\,3.\,1939 Berlin@\textsc{Fulda, Ludwig} (15.\,7.\,1862 Frankfurt am Main – 30.\,3.\,1939 Berlin), \emph{Schriftsteller, Übersetzer}|pw} liegen schon vor
               (in zum Teil ganz überraschend günstigem Sinne, muss ich sagen); und da der Verlag
               doch gern auch aus Oesterreich\oindex{Österreich@\textbf{Österreich}|pw} etwas in der Art
               möchte vorweisen können, so fiel mir ein, dass vor Jahren, als dir einmal die
               öffentliche Vorlesung des »Reigen\pwindex{Schnitzler, Arthur 15.\,5.\,1862 Wien – 21.\,10.\,1931 ebd.@\textsc{Schnitzler, Arthur} (15.\,5.\,1862 Wien – 21.\,10.\,1931 ebd.), \emph{Schriftsteller, Mediziner}!Reigen. Zehn Dialoge@\strich\emph{Reigen. Zehn Dialoge}|pw}« untersagt
               wurde, Burckhardt\pwindex{Burckhard, Max Eugen 14.\,7.\,1854 Korneuburg – 16.\,3.\,1912 Wien@\textsc{Burckhard, Max Eugen} (14.\,7.\,1854 Korneuburg – 16.\,3.\,1912 Wien), \emph{Schriftsteller, Rechtswissenschaftler, Theaterleiter}|pw} einen Rekurs eingebracht
               hat, der sich vielleicht noch in Deinem Besitze finden mag. Ich frage Dich nun, ob Du
               dem Verlag J. Singer\orgindex{J. Singer und Co.@J. Singer {\kaufmannsund}  Co.|pw}, wenn er sich {\pb}mit entsprechender
               Bitte an Dich wenden sollte, jenes Schriftstück zu eventueller Benützung vor Gericht
               auszufolgen geneigt wärest?\pend
           
\pstart
           Mit herzlichem Gruss{\\[\baselineskip]}Dein{\\[\baselineskip]}\spacefill\mbox{{[}hs.:{]} Arthur}\pend
           \leftskip=0em{}\selectlanguage{ngerman}\endnumbering\briefempfaengerindex{Bahr, Hermann@\textsc{Bahr, Hermann}!zzzSchnitzler, Arthur@\emph{von Arthur Schnitzler}!1914-06-121@{12. 6. 1914}|)be}\mylabel{L02181h}  \newcommand{\dateiname}{L02181}\newcommand{\titel}{Arthur Schnitzler an Hermann Bahr, 12. 6. 1914}\newcommand{\editorInnen}{Herausgegeben von Martin Anton Müller}%% latex-leseansicht-abspann.tex
%% Abspann für die Leseansicht.
%% Der Schalter \ifkorrekturansicht ist bereits durch den Vorspann gesetzt.

%% latex-abspann.tex
%% Gemeinsamer Abspann für Korrekturansicht und Leseansicht.
%% Setzt den Schalter \ifkorrekturansicht voraus (gesetzt in den
%% einbindenden Dateien latex-korrekturansicht-abspann.tex bzw.
%% latex-leseansicht-abspann.tex).
%% ---------------------------------------------------------------

\normalsize

% Das esempio-Environment wird nur in der Leseansicht benötigt
\ifkorrekturansicht\else
\newenvironment{esempio}[3]%
{
    \vspace{1.5ex}
    \rlap{\underline{#1}}
    \par
    \setlength{\parindent}{0cm}
    \nopagebreak
    \leftskip=#2cm
    \rightskip=#3cm
}
{
    \par
}
\fi

\doendnotes{C}
\bigskip
\vfill

\clearpage

\footnotesize

\ifkorrekturansicht
  \lohead{\textsc{register}}
\fi

% theindex-Environment neu definieren ohne reledmac
\makeatletter
\renewenvironment{theindex}{%
  \ifkorrekturansicht
    \section*{\indexname}%
  \else
    \subsubsection*{Index der erwähnten Entitäten}%
  \fi
  \setlength{\parindent}{0pt}%
  \setlength{\parskip}{0pt plus 0.3pt}%
  \let\item\@idxitem
}{%
  \ifkorrekturansicht\clearpage\fi
}
\makeatother

\IfFileExists{\jobname-pw.ind}{\input{\jobname-pw.ind}}{}

% Quellenangabe nur in der Leseansicht
\ifkorrekturansicht\else
% Fallback-Definitionen, falls die .tex-Datei \titel etc. nicht gesetzt hat
\providecommand{\titel}{}
\providecommand{\editorInnen}{}
\providecommand{\dateiname}{\jobname}

\vspace{3cm}

\vfill

\footnotesize
\textsc{Quelle}: \titel. Herausgegeben von {\editorInnen}. In: \emph{Arthur Schnitzler: Briefwechsel mit Autorinnen und Autoren}.
 Digitale Edition, https://schnitzler-briefe.acdh.oeaw.ac.at/{\dateiname}.html (Stand \today)
\fi

\end{document}


