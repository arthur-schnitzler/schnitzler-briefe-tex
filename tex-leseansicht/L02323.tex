%% latex-leseansicht-vorspann.tex
%% Vorspann für die Leseansicht.
%% Lädt die gemeinsame Datei latex-vorspann.tex mit nicht gesetztem Schalter.

\newif\ifkorrekturansicht
\korrekturansichtfalse

\input{../tex-inputs/latex-vorspann}


         
         \renewcommand{\erwaehntePersonen}{Personen: Frieda Pollak, Olga Schnitzler}
         \renewcommand{\erwaehnteInstitutionen}{Institutionen: Deutschösterreichischer Autorenverband}
         \renewcommand{\erwaehnteOrte}{Orte: Rodaun, Wien, Österreich}
         \renewcommand{\erwaehnteWerke}{}
               \section[Hugo Hofmannsthal an Arthur Schnitzler, 20. 4. 1919]{ Hugo Hofmannsthal an Arthur Schnitzler, 20. 4. 1919}\nopagebreak\mylabel{v}\rehead{ }\begin{ledgroupsized}[t]{13cm}\normalsize\beginnumbering \toendnotes[C]{\smallbreak\pagebreak[2]} \Standort{CUL, Schnitzler, B 43.}
\physDesc{Brief, 1 Blatt, 2 Seiten, 840 Zeichen
\newline{}Handschrift: schwarze Tinte, deutsche Kurrent
\newline{}Schnitzler: mit Bleistift beschriftet: »\textsc{Hugo}« 
\newline{}Ordnung: 1) mit Bleistift von Frieda
                                    Pollak\pwindex{Pollak, Frieda 08.12.1881 – 13.07.1937@\textsc{Pollak, Frieda} (08.12.1881 – 13.07.1937), \emph{Sekretärin}|pw} (?) mit dem Buchstaben »A«
                                 (Abgeschrieben/Abschrift) gekennzeichnet  2) mit Bleistift von unbekannter Hand nummeriert:
                                    »352«}\buchAbdrucke{\weitereDrucke{Hugo von Hofmannsthal, Arthur Schnitzler: \emph{Briefwechsel}. Hg. Therese Nickl und Heinrich Schnitzler. Frankfurt am Main: \emph{S. Fischer} 1964, S. 283.} }\toendnotes[C]{\smallbreak}\pstart
           \raggedleft{}{\pb}Rodaun\oindex{Rodaun@\textbf{Rodaun}|pw}, Osterſonntag 19.\pend
           \pstart{}mein lieber Arthur\pend\pstart
           grüß Sie Gott. Wie gehts Ihnen denn immer?\pend
           \pstart
           Ich bin ſchon ſeit 3 Wochen krank, muſs jetzt liegen wegen einer Rippenfellreizung.
               Sie waren ja auch in dieſem Winter \label{K_L02323-1v}\edtext{einmal recht krank}{\lemma{\textnormal{\emph{einmal recht krank}}}\Cendnote{\textnormal{siehe A. S.: \emph{Tagebuch}, 20. 1. 1919}}}\label{K_L02323-1h} u. ich hab es gar nicht gewuſst!\pend
           \pstart
           Ich bitte Sie Arthur, wegen dieſer Autorenorganiſation\orgindex{Deutschoesterreichischer Autorenverband@Deutschösterreichischer Autorenverband|pwv}, daſs Sie eventuell den Leuten von mir ſagen, daſs ich
               krank bin, und dann autoriſiere ich Sie, alles was Ihnen zu {\pb}beſchließen oder wozu zuzuſti{\geminationm}en Ihnen richtig erſcheint, dies auch in meinem Namen
               zu tun.\pend
           \pstart
           Ich wundere mich nur wie man eine ſpecielle Organiſation\orgindex{Deutschoesterreichischer Autorenverband@Deutschösterreichischer Autorenverband|pwv} in Oeſterreich\oindex{Oesterreich@\textbf{Österreich}|pw}{ }ſchaffen will, da wir doch alle an dem deutſchen
                  geſa{\geminationm}ten Bühnenweſen beteiligt ſind, – aber ſei dem
               wie immer.\pend
           \pstart
           In alter Liebe{\\[\baselineskip]}Ihr{\\[\baselineskip]}\spacefill\mbox{Hugo.}\pend
           \leftskip=0em{}\pstart
           \noindent{}\textsc{PS}. Alles Gute an Olga\pwindex{Schnitzler, Olga 17.01.1882 – 13.01.1970@\textsc{Schnitzler, Olga} (17.01.1882 – 13.01.1970), \emph{Schauspielerin, Sängerin}|pw}. Wie ſchön war man früher oft zuſa{\geminationm}en.
                  Im Bett liegend, genieße ich manches Freundliche in der Erinnerung.\pend
           
         
         \endnumbering\mylabel{h}\end{ledgroupsized}  \newcommand{\dateiname}{L02323}\newcommand{\titel}{Hugo Hofmannsthal an Arthur Schnitzler, 20. 4. 1919}\newcommand{\editorInnen}{Martin Anton Müller und Gerd-Hermann Susen}%% latex-leseansicht-abspann.tex
%% Abspann für die Leseansicht.
%% Der Schalter \ifkorrekturansicht ist bereits durch den Vorspann gesetzt.

%% latex-abspann.tex
%% Gemeinsamer Abspann für Korrekturansicht und Leseansicht.
%% Setzt den Schalter \ifkorrekturansicht voraus (gesetzt in den
%% einbindenden Dateien latex-korrekturansicht-abspann.tex bzw.
%% latex-leseansicht-abspann.tex).
%% ---------------------------------------------------------------

\normalsize

% Das esempio-Environment wird nur in der Leseansicht benötigt
\ifkorrekturansicht\else
\newenvironment{esempio}[3]%
{
    \vspace{1.5ex}
    \rlap{\underline{#1}}
    \par
    \setlength{\parindent}{0cm}
    \nopagebreak
    \leftskip=#2cm
    \rightskip=#3cm
}
{
    \par
}
\fi

\doendnotes{C}
\bigskip
\vfill

\clearpage

\footnotesize

\ifkorrekturansicht
  \lohead{\textsc{register}}
\fi

% theindex-Environment neu definieren ohne reledmac
\makeatletter
\renewenvironment{theindex}{%
  \ifkorrekturansicht
    \section*{\indexname}%
  \else
    \subsubsection*{Index der erwähnten Entitäten}%
  \fi
  \setlength{\parindent}{0pt}%
  \setlength{\parskip}{0pt plus 0.3pt}%
  \let\item\@idxitem
}{%
  \ifkorrekturansicht\clearpage\fi
}
\makeatother

\IfFileExists{\jobname-pw.ind}{\input{\jobname-pw.ind}}{}

% Quellenangabe nur in der Leseansicht
\ifkorrekturansicht\else
% Fallback-Definitionen, falls die .tex-Datei \titel etc. nicht gesetzt hat
\providecommand{\titel}{}
\providecommand{\editorInnen}{}
\providecommand{\dateiname}{\jobname}

\vspace{3cm}

\vfill

\footnotesize
\textsc{Quelle}: \titel. Herausgegeben von {\editorInnen}. In: \emph{Arthur Schnitzler: Briefwechsel mit Autorinnen und Autoren}.
 Digitale Edition, https://schnitzler-briefe.acdh.oeaw.ac.at/{\dateiname}.html (Stand \today)
\fi

\end{document}


      