%% latex-leseansicht-vorspann.tex
%% Vorspann für die Leseansicht.
%% Lädt die gemeinsame Datei latex-vorspann.tex mit nicht gesetztem Schalter.

\newif\ifkorrekturansicht
\korrekturansichtfalse

\input{../tex-inputs/latex-vorspann}


\section[Arthur Schnitzler an Hermann Bahr, 13. 12. 1903]{L01350 Arthur Schnitzler an Hermann Bahr, 13. 12. 1903}
\nopagebreak\mylabel{L01350v}
\rehead{ }\normalsize\beginnumbering\briefempfaengerindex{Bahr, Hermann@\textsc{Bahr, Hermann}!zzzSchnitzler, Arthur@\emph{von Arthur Schnitzler}!1903-12-131@{13. 12. 1903}|(be}
\toendnotes[C]{\smallbreak\pagebreak[2]}
\correspDesc{Versand  durch Arthur Schnitzler am 13. 12. 1903 in Wien
\newline{}Erhalt  durch Hermann Bahr am 13. 12. 1903 in Berlin}\toendnotes[C]{\smallbreak}
\Standort{TMW, HS AM 23362 Ba.}
\physDesc{Telegramm, 117 Zeichen
\newline{}maschinell
\newline{}Versand: 1) mit schwarzer Tinte von »Schott\pwindex{Schott @\textsc{Schott}, \emph{Briefträger/Briefträgerin}|pw}« signiert und mit weiterer Empfängeradresse versehen: »N.W.7 Hotel de Rom\oindex{Hotel de Rome@\textbf{Hotel de Rome}, \emph{Hotel}|pw} zu
                                    bestellen«  2) Stempel: »\nobreak{}\oindex{Berlin@\textbf{Berlin}, \emph{Hauptstadt}|pwk}Berlin N. W. 6, 13. 12. 03., 12\textsuperscript{20}\nobreak{}«.  3) Stempel: »\nobreak{}Ausgefertigt, 13 Dec. {[}1903{]}\nobreak{}«.  4) »\textcolor{gray}{\textbf{\textbf{Aufgenommen} von}} W \textcolor{gray}{\textbf{den}}{ }13\textcolor{gray}{\textbf{/}}12 um 11 \textcolor{gray}{\textbf{Uhr}} 57 \textcolor{gray}{\textbf{M.}}
                                       m{ }\textcolor{gray}{\textbf{durch}}{ }\textcolor{gray}{MW}«}
\buchAbdrucke{\weitereDrucke{1) \emph{13. 12. 1903.} In: Arthur Schnitzler: \emph{The Letters of Arthur Schnitzler to Hermann Bahr}. Edited, annotated, and with an introduction, by Donald G. Daviau. Chapel Hill: \emph{The University of North Carolina Press} 1978, S. 82 (University of North Carolina studies in the Germanic languages
                        and literatures, 89).} \weitereDrucke{2) Hermann Bahr, Arthur Schnitzler: \emph{Briefwechsel, Aufzeichnungen, Dokumente (1891–1931)}. Herausgegeben von Kurt Ifkovits und Martin Anton Müller. Göttingen: \emph{Wallstein} 2018, S. 284.} }\toendnotes[C]{\smallbreak}\pstart{}{\pb}hermann bahr berlin deutsches theater\oindex{Deutsches Theater Berlin@\textbf{Deutsches Theater Berlin}, \emph{Theater}|pw}\pend{}{\bigskip}\vspace{1em}
\pstart
           \noindent{}{\pb}\textcolor{gray}{\textbf{Telegramm}} fr wien\oindex{Wien@\textbf{Wien}, \emph{Verwaltungsgebiet}|pw} 110+
               466 12 13{ }11 m{ }\textcolor{gray}{\textbf{W.}}{ }\textcolor{gray}{\textbf{190}}3\pend
           
\pstart
           herzlichen \label{K_L01350-1v}\edtext{glueckwunsch}{\lemma{\textnormal{\emph{glueckwunsch}}}\Cendnote{\textnormal{Die Uraufführung\eventindex{Deutsches Theater Berlin@\textbf{Deutsches Theater Berlin}!Uraufführung von Der Meister, 12.12.1903@Uraufführung von Der Meister, 12.12.1903|pwkv} von \emph{Der Meister}\pwindex{Bahr, Hermann 19.\,7.\,1863 Linz – 15.\,1.\,1934 München@\textsc{Bahr, Hermann} (19.\,7.\,1863 Linz – 15.\,1.\,1934 München), \emph{Schriftsteller, Kritiker}!Meister. Komödie in drei Akten@\strich\emph{Der Meister. Komödie in drei Akten}|pwk}\pwindex{Bahr, Hermann 19.\,7.\,1863 Linz – 15.\,1.\,1934 München@\textsc{Bahr, Hermann} (19.\,7.\,1863 Linz – 15.\,1.\,1934 München), \emph{Schriftsteller, Kritiker}!Meister. Komödie in drei Akten@\strich\emph{Der Meister. Komödie in drei Akten}|pwk} fand am 12. 12. 1903 am \emph{Deutschen Theater}\orgindex{Deutsches Theater Berlin@Deutsches Theater Berlin|pwk} in Berlin\oindex{Berlin@\textbf{Berlin}, \emph{Hauptstadt}|pwk} statt.}}}\label{K_L01350-1} und gruss dein arthur schnitzler\pend
           \selectlanguage{ngerman}\endnumbering\briefempfaengerindex{Bahr, Hermann@\textsc{Bahr, Hermann}!zzzSchnitzler, Arthur@\emph{von Arthur Schnitzler}!1903-12-131@{13. 12. 1903}|)be}\mylabel{L01350h}  \newcommand{\dateiname}{L01350}\newcommand{\titel}{Arthur Schnitzler an Hermann Bahr, 13. 12. 1903}\newcommand{\editorInnen}{Herausgegeben von Martin Anton Müller}%% latex-leseansicht-abspann.tex
%% Abspann für die Leseansicht.
%% Der Schalter \ifkorrekturansicht ist bereits durch den Vorspann gesetzt.

%% latex-abspann.tex
%% Gemeinsamer Abspann für Korrekturansicht und Leseansicht.
%% Setzt den Schalter \ifkorrekturansicht voraus (gesetzt in den
%% einbindenden Dateien latex-korrekturansicht-abspann.tex bzw.
%% latex-leseansicht-abspann.tex).
%% ---------------------------------------------------------------

\normalsize

% Das esempio-Environment wird nur in der Leseansicht benötigt
\ifkorrekturansicht\else
\newenvironment{esempio}[3]%
{
    \vspace{1.5ex}
    \rlap{\underline{#1}}
    \par
    \setlength{\parindent}{0cm}
    \nopagebreak
    \leftskip=#2cm
    \rightskip=#3cm
}
{
    \par
}
\fi

\doendnotes{C}
\bigskip
\vfill

\clearpage

\footnotesize

\ifkorrekturansicht
  \lohead{\textsc{register}}
\fi

% theindex-Environment neu definieren ohne reledmac
\makeatletter
\renewenvironment{theindex}{%
  \ifkorrekturansicht
    \section*{\indexname}%
  \else
    \subsubsection*{Index der erwähnten Entitäten}%
  \fi
  \setlength{\parindent}{0pt}%
  \setlength{\parskip}{0pt plus 0.3pt}%
  \let\item\@idxitem
}{%
  \ifkorrekturansicht\clearpage\fi
}
\makeatother

\IfFileExists{\jobname-pw.ind}{\input{\jobname-pw.ind}}{}

% Quellenangabe nur in der Leseansicht
\ifkorrekturansicht\else
% Fallback-Definitionen, falls die .tex-Datei \titel etc. nicht gesetzt hat
\providecommand{\titel}{}
\providecommand{\editorInnen}{}
\providecommand{\dateiname}{\jobname}

\vspace{3cm}

\vfill

\footnotesize
\textsc{Quelle}: \titel. Herausgegeben von {\editorInnen}. In: \emph{Arthur Schnitzler: Briefwechsel mit Autorinnen und Autoren}.
 Digitale Edition, https://schnitzler-briefe.acdh.oeaw.ac.at/{\dateiname}.html (Stand \today)
\fi

\end{document}


