%% latex-leseansicht-vorspann.tex
%% Vorspann für die Leseansicht.
%% Lädt die gemeinsame Datei latex-vorspann.tex mit nicht gesetztem Schalter.

\newif\ifkorrekturansicht
\korrekturansichtfalse

\input{../tex-inputs/latex-vorspann}


\section[Arthur und Olga Schnitzler an Gustav Schwarzkopf, 9. 5. 1904]{L04071 Arthur und Olga Schnitzler an Gustav Schwarzkopf, 9. 5. 1904}
\nopagebreak\mylabel{L04071v}
\rehead{ }\normalsize\beginnumbering\briefempfaengerindex{Schwarzkopf, Gustav@\textsc{Schwarzkopf, Gustav}!zzzSchnitzler, Olga@\emph{von Olga Schnitzler}!1904-05-091@{9. 5. 1904}|(be}\briefempfaengerindex{Schwarzkopf, Gustav@\textsc{Schwarzkopf, Gustav}!zzzSchnitzler, Arthur@\emph{von Arthur Schnitzler}!1904-05-091@{9. 5. 1904}|(be}
\toendnotes[C]{\smallbreak\pagebreak[2]}
\correspDesc{Versand  durch Arthur Schnitzler, Olga Schnitzler am 9. 5. 1904 in Neapel
\newline{}Erhalt  durch Gustav Schwarzkopf im Zeitraum [10. 5. 1904
                  – 14. 5. 1904?] in Wien}\toendnotes[C]{\smallbreak}
\Standort{CUL, Schnitzler, B 96.}
\physDesc{Brief, 1 Blatt, 4 Seiten, 953 Zeichen
\newline{}Handschrift Arthur Schnitzler: Bleistift, deutsche Kurrent
\newline{}Handschrift Olga Schnitzler: Bleistift, lateinische Kurrent}\toendnotes[C]{\smallbreak}
\pstart
           \raggedleft{}{\pb}Neapel\oindex{Neapel@\textbf{Neapel}|pw}, 9. 5. 904\pend
           \vspace{0.5em}
\pstart
           lieber Guſtav, ich glaube, daſs der \label{K_L04071-55v}\edtext{reine
                  Thor\pwindex{Schwarzkopf, Max 12.\,6.\,1857 Wien – 14.\,4.\,1928 ebd.@\textsc{Schwarzkopf, Max} (12.\,6.\,1857 Wien – 14.\,4.\,1928 ebd.), \emph{Rechtsanwalt}!reine Tor. Gesellschaftsstück in vier Akten@\strich\emph{Der reine Tor. Gesellschaftsstück in vier Akten}|pwv} eingeſperrt iſt}{\lemma{\textnormal{\emph{reine … ist}}}\Cendnote{\textnormal{Vgl. XXXX Auszeichnungsfehler: Dokument L04145 nicht gefunden.}}}\label{K_L04071-55} (was ſchon
               manchm einem Thoren paſſirt iſt) – we{\geminationn} aber nicht, liegt
               er entweder im Rieſenkaſten, Abtheilung nächſt dem Fenſter, unterſtes Fach – oder in
               dem daran gerückten kleinen, direct am Fenſter ſtehenden Schrank {\pb}in einem der beiden \uline{offenen} Fächer. Bemühen Sie ſich bitte in die Spoettelgaſſe\oindex{Wien@\textbf{Wien}!XVIII., Währing@\textbf{XVIII., Währing}!Edmund-Weiß-Gasse 7@\textbf{Edmund-Weiß-Gasse 7}, \emph{Wohngebäude}|pw}, die Frau Tallian\pwindex{Tallian, Henriette 3.\,6.\,1837 Mannheim – 13.\,5.\,1923 Wien@\textsc{Tallian, Henriette} (3.\,6.\,1837 Mannheim – 13.\,5.\,1923 Wien), \emph{Kinderbetreuerin}|pw} aviſire ich unter einem,{ }ſo daſs jeder criminelle
               Verdacht bei Ihrem Einbruch ausgeſchloſſen bleibt. Wozu nur \label{K_L04071-1v}\edtext{\textsc{Angelo\pwindex{Neumann, Angelo 18.\,8.\,1838 Stupava – 20.\,12.\,1910 Prag@\textsc{Neumann, Angelo} (18.\,8.\,1838 Stupava – 20.\,12.\,1910 Prag), \emph{Theaterleiter, Sänger}|pw}} ein zweites Exemplar}{\lemma{\textnormal{\emph{Angelo … Exemplar}}}\Cendnote{\textnormal{Angelo Neumann\pwindex{Neumann, Angelo 18.\,8.\,1838 Stupava – 20.\,12.\,1910 Prag@\textsc{Neumann, Angelo} (18.\,8.\,1838 Stupava – 20.\,12.\,1910 Prag), \emph{Theaterleiter, Sänger}|pwk} hatte am
                     30. 8. 1903 das Stück \emph{Der reine
                     Thor}\pwindex{Schwarzkopf, Max 12.\,6.\,1857 Wien – 14.\,4.\,1928 ebd.@\textsc{Schwarzkopf, Max} (12.\,6.\,1857 Wien – 14.\,4.\,1928 ebd.), \emph{Rechtsanwalt}!reine Tor. Gesellschaftsstück in vier Akten@\strich\emph{Der reine Tor. Gesellschaftsstück in vier Akten}|pwk} von Max Schwarzkopf\pwindex{Schwarzkopf, Max 12.\,6.\,1857 Wien – 14.\,4.\,1928 ebd.@\textsc{Schwarzkopf, Max} (12.\,6.\,1857 Wien – 14.\,4.\,1928 ebd.), \emph{Rechtsanwalt}|pwk} als
                  Novität am \emph{Deutschen Landestheater}\orgindex{Ständetheater@Ständetheater|pwk}
                  angekündigt. (\emph{Prager Tagblatt}\pwindex{Prager Tagblatt@\emph{Prager Tagblatt}|pwk}, Jg. 27, Nr. 236,
                     Morgenausgabe, S. 9.) Die Uraufführung (und einzige Aufführung) fand am
                        17. 6. 1904\eventindex{Neues Deutsches Theater@\textbf{Neues Deutsches Theater}!Uraufführung von Der reine Tor, 17.6.1904@Uraufführung von Der reine Tor, 17.6.1904|pwkv} statt, wobei der Autor nicht mehr genannt wurde, sondern sich hinter dem
                  Pseudonym »Franz Bergleitner« verbarg (das von der Hauptfigur
                  entlehnt war).}}}\label{K_L04071-1} braucht –? ſollte {\pb}er am Ende ſchon an eine zweite
               Aufführung denken? –\pend
           
\pstart
           – Wir haben in \label{K_L04071-2v}\edtext{Rom\oindex{Rom@\textbf{Rom}, \emph{Hauptstadt}|pw} ein paar wunderſchöne Tage}{\lemma{\textnormal{\emph{Rom … Tage}}}\Cendnote{\textnormal{Zwischen 3. 5. 1904 und 8. 5. 1904 waren Arthur und Olga
                     Schnitzler\pwindex{Schnitzler, Olga 17.\,1.\,1882 Wien – 13.\,1.\,1970 Lugano@\textsc{Schnitzler, Olga} (17.\,1.\,1882 Wien – 13.\,1.\,1970 Lugano), \emph{Schauspielerin, Sängerin}|pwk} in Rom\oindex{Rom@\textbf{Rom}, \emph{Hauptstadt}|pwk}.}}}\label{K_L04071-2} verlebt, und
                  \label{K_L04071-3v}\edtext{geſtern ſind wir hier
                  angekommen}{\lemma{\textnormal{\emph{gestern sind … angekommen}}}\Cendnote{\textnormal{Siehe A. S.: \emph{Wiener Schnitzler}, 8. 5. 1904.}}}\label{K_L04071-3} und
               völlig entzückt. Von Hitze keine Spur; auch bei Tag ſelten ohne Überkleider. Erfreuen
               Sie mich vielleicht durch ein Wort {\pb}nach Palermo\oindex{Palermo@\textbf{Palermo}|pw}{ }\textsc{post restante}. Mit herzlichen Grüßen an Sie und Doctor Max\pwindex{Schwarzkopf, Max 12.\,6.\,1857 Wien – 14.\,4.\,1928 ebd.@\textsc{Schwarzkopf, Max} (12.\,6.\,1857 Wien – 14.\,4.\,1928 ebd.), \emph{Rechtsanwalt}|pw}\pend
           
\pstart
           Ihr{\\[\baselineskip]}\spacefill\mbox{A.}\pend
           \leftskip=0em{}\selectlanguage{ngerman}\vspace{1em}
\pstart
           \noindent{}{[}hs. Schnitzler:{]} Die Welt iſt \uline{doch} ſchön!\footnote{\noindent{}{[}hs. Schnitzler:{]} Haben Sie je daran gezweifelt? So werden von
                     kleinen Kindern rauhe Schalen misverſtanden. –}\pend
           
\pstart
           Herzliche Grüße, lieber{\\[\baselineskip]} Herr Schwarzkopf u. Bruder\pwindex{Schwarzkopf, Max 12.\,6.\,1857 Wien – 14.\,4.\,1928 ebd.@\textsc{Schwarzkopf, Max} (12.\,6.\,1857 Wien – 14.\,4.\,1928 ebd.), \emph{Rechtsanwalt}|pwv}.{\\[\baselineskip]}\spacefill\mbox{Olga S.}\pend
           \leftskip=0em{}\selectlanguage{ngerman}\endnumbering\briefempfaengerindex{Schwarzkopf, Gustav@\textsc{Schwarzkopf, Gustav}!zzzSchnitzler, Olga@\emph{von Olga Schnitzler}!1904-05-091@{9. 5. 1904}|)be}\briefempfaengerindex{Schwarzkopf, Gustav@\textsc{Schwarzkopf, Gustav}!zzzSchnitzler, Arthur@\emph{von Arthur Schnitzler}!1904-05-091@{9. 5. 1904}|)be}\mylabel{L04071h}
\begin{anhang}
\end{anhang}\newcommand{\dateiname}{L04071}\newcommand{\titel}{Arthur und Olga Schnitzler an Gustav Schwarzkopf, 9. 5. 1904}\newcommand{\editorInnen}{Herausgegeben von Jahnke, SelmaMüller, Martin Anton}%% latex-leseansicht-abspann.tex
%% Abspann für die Leseansicht.
%% Der Schalter \ifkorrekturansicht ist bereits durch den Vorspann gesetzt.

%% latex-abspann.tex
%% Gemeinsamer Abspann für Korrekturansicht und Leseansicht.
%% Setzt den Schalter \ifkorrekturansicht voraus (gesetzt in den
%% einbindenden Dateien latex-korrekturansicht-abspann.tex bzw.
%% latex-leseansicht-abspann.tex).
%% ---------------------------------------------------------------

\normalsize

% Das esempio-Environment wird nur in der Leseansicht benötigt
\ifkorrekturansicht\else
\newenvironment{esempio}[3]%
{
    \vspace{1.5ex}
    \rlap{\underline{#1}}
    \par
    \setlength{\parindent}{0cm}
    \nopagebreak
    \leftskip=#2cm
    \rightskip=#3cm
}
{
    \par
}
\fi

\doendnotes{C}
\bigskip
\vfill

\clearpage

\footnotesize

\ifkorrekturansicht
  \lohead{\textsc{register}}
\fi

% theindex-Environment neu definieren ohne reledmac
\makeatletter
\renewenvironment{theindex}{%
  \ifkorrekturansicht
    \section*{\indexname}%
  \else
    \subsubsection*{Index der erwähnten Entitäten}%
  \fi
  \setlength{\parindent}{0pt}%
  \setlength{\parskip}{0pt plus 0.3pt}%
  \let\item\@idxitem
}{%
  \ifkorrekturansicht\clearpage\fi
}
\makeatother

\IfFileExists{\jobname-pw.ind}{\input{\jobname-pw.ind}}{}

% Quellenangabe nur in der Leseansicht
\ifkorrekturansicht\else
% Fallback-Definitionen, falls die .tex-Datei \titel etc. nicht gesetzt hat
\providecommand{\titel}{}
\providecommand{\editorInnen}{}
\providecommand{\dateiname}{\jobname}

\vspace{3cm}

\vfill

\footnotesize
\textsc{Quelle}: \titel. Herausgegeben von {\editorInnen}. In: \emph{Arthur Schnitzler: Briefwechsel mit Autorinnen und Autoren}.
 Digitale Edition, https://schnitzler-briefe.acdh.oeaw.ac.at/{\dateiname}.html (Stand \today)
\fi

\end{document}


