%% latex-leseansicht-vorspann.tex
%% Vorspann für die Leseansicht.
%% Lädt die gemeinsame Datei latex-vorspann.tex mit nicht gesetztem Schalter.

\newif\ifkorrekturansicht
\korrekturansichtfalse

\input{../tex-inputs/latex-vorspann}


\section[Robert Adam an Arthur Schnitzler, 7. 6. 1929]{L02510 Robert Adam an Arthur Schnitzler, 7. 6. 1929}
\nopagebreak\mylabel{L02510v}
\rehead{ }\normalsize\beginnumbering\briefempfaengerindex{Schnitzler, Arthur@\textsc{Schnitzler, Arthur}!zzzAdam, Robert@\emph{von Robert Adam}!1929-06-071@{7. 6. 1929}|(be}
\toendnotes[C]{\smallbreak\pagebreak[2]}
\correspDesc{Versand  durch Robert Adam am 7. 6. 1929 in Wien
\newline{}Erhalt  durch Arthur Schnitzler im Zeitraum [7. 6. 1929
                  – 11. 6. 1929?] in Wien}\toendnotes[C]{\smallbreak}
\Standort{CUL, Schnitzler, B 1.}
\physDesc{Brief, 1 Blatt, 3 Seiten, 1256 Zeichen
\newline{}Handschrift: schwarze Tinte, deutsche Kurrent
\newline{}Schnitzler: mit rotem Buntstift beschriftet: »\textsc{Adam}« und »MdlHptstr 58\oindex{Wien@\textbf{Wien}!XII., Meidling@\textbf{XII., Meidling}!Meidlinger Hauptstraße@\textbf{Meidlinger Hauptstraße}, \emph{Straße}|pw}« sowie vereinzelte Unterstreichungen 
\newline{}Ordnung: mit Bleistift von unbekannter Hand nummeriert:
                                    »21« }\Standort{Wien, Österreichische Nationalbibliothek, Cod.ser. 52.269, 32–33.}
\physDesc{handschriftliche Abschrift. 1 Blatt, 2 Seiten, 1256 Zeichen
\newline{}Handschrift: schwarze Tinte, Gabelsberger Kurzschrift}\Standort{Wien, Österreichische Nationalbibliothek, Cod.ser. 52.269, 32–33.}
\physDesc{maschinenschriftliche Abschrift, 1 Blatt, 2 Seiten, 1256 Zeichen
\newline{}Schreibmaschine}\toendnotes[C]{\smallbreak}
\pstart
           \raggedleft{}{\pb}Wien\oindex{Wien@\textbf{Wien}, \emph{Verwaltungsgebiet}|pw}, am 7. Juni 1929\pend
           
\pstart{}Hochverehrter Herr Doktor!\pend\vspace{0.5em}
\pstart
           Frl. Frieda Pollak\pwindex{Pollak, Frieda 8.\,12.\,1881 Wien – 13.\,7.\,1937 ebd.@\textsc{Pollak, Frieda} (8.\,12.\,1881 Wien – 13.\,7.\,1937 ebd.), \emph{Sekretärin}|pw} hat mir mitgeteilt, daß Sie
               die große Güte hatten, meine jüngſte Arbeit zu leſen und{ }ſich für ihr Schickſal zu
               intereſſieren. Ich danke Ihnen, wie{ }ſchon{ }ſo oft, auf’s herzlichſte. Mit »Margot und das Jugendgericht\pwindex{Adam, Robert 20.\,4.\,1877 Wien – 16.\,10.\,1961 Baden bei Wien@\textsc{Adam, Robert} (20.\,4.\,1877 Wien – 16.\,10.\,1961 Baden bei Wien), \emph{Schriftsteller, Richter}!Margot und das Jugendgericht@\strich\emph{Margot und das Jugendgericht}|pw}« meine ich freilich
               nichts Schwerwiegendes und Hervorragendes geſchaffen zu haben, aber die frohe
               Befriedigung, die ich, trotz Alltags-Sorgen und -ärger, beim Schreiben {\pb}empfand, beſonders das eigene Vergnügen an
                  Margots\pwindex{Adam, Robert 20.\,4.\,1877 Wien – 16.\,10.\,1961 Baden bei Wien@\textsc{Adam, Robert} (20.\,4.\,1877 Wien – 16.\,10.\,1961 Baden bei Wien), \emph{Schriftsteller, Richter}!Margot und das Jugendgericht@\strich\emph{Margot und das Jugendgericht}|pwv} Erlebniſſen mit dem
               Heilpädagogen und in der Kaffeehausecke, gaben mir doch das{ }ſichere Gefühl, daß die
               Geſchichte meiner Heldin auch andern etwas Sympathie, deren{ }ſie{ }ſie so dringend
               bedarf, abgewinnen könne. Wenn es mir gelänge, mit dieſem leichten Stück endlich
               einmal den{ }ſo oft geſuchten Eingang zur Bühne zu finden, wäre es natürlich für mich
               von allergrößter Bedeutung. Nur haben mir die{ }ſtäten Enttäuſchungen früherer Jahre
               das Hoffen gründlichſt abgewöhnt.\pend
           
\pstart
           {\pb}Dürfte ich, hochverehrter Herr Doktor,
               nach langer Zeit wieder einmal perſönlich bei Ihnen vorſprechen? Jede Zeit wäre mir
               recht, und Frl. Pollak\pwindex{Pollak, Frieda 8.\,12.\,1881 Wien – 13.\,7.\,1937 ebd.@\textsc{Pollak, Frieda} (8.\,12.\,1881 Wien – 13.\,7.\,1937 ebd.), \emph{Sekretärin}|pw}, mit deren Bruder\pwindex{Pollak, Karl 7.\,10.\,1873 Šternberk – 29.\,5.\,1940 Wien@\textsc{Pollak, Karl} (7.\,10.\,1873 Šternberk – 29.\,5.\,1940 Wien), \emph{Richter}|pwv} ich in{ }ſtetem Kontakt
               bin, würde es gewiß übernehmen, mir die Ihnen genehme Stunde mitzuteilen.\pend
           
\pstart
           Mit ergebenſtem Gruß Ihr{\\[\baselineskip]}dankbarer{\\[\baselineskip]}\spacefill\mbox{D\textsuperscript{r}RAdam}\pend
           \leftskip=0em{}\selectlanguage{ngerman}\endnumbering\briefempfaengerindex{Schnitzler, Arthur@\textsc{Schnitzler, Arthur}!zzzAdam, Robert@\emph{von Robert Adam}!1929-06-071@{7. 6. 1929}|)be}\mylabel{L02510h}  \newcommand{\dateiname}{L02510}\newcommand{\titel}{Robert Adam an Arthur Schnitzler, 7. 6. 1929}\newcommand{\editorInnen}{Martin Anton Müller und Gerd-Hermann Susen}%% latex-leseansicht-abspann.tex
%% Abspann für die Leseansicht.
%% Der Schalter \ifkorrekturansicht ist bereits durch den Vorspann gesetzt.

%% latex-abspann.tex
%% Gemeinsamer Abspann für Korrekturansicht und Leseansicht.
%% Setzt den Schalter \ifkorrekturansicht voraus (gesetzt in den
%% einbindenden Dateien latex-korrekturansicht-abspann.tex bzw.
%% latex-leseansicht-abspann.tex).
%% ---------------------------------------------------------------

\normalsize

% Das esempio-Environment wird nur in der Leseansicht benötigt
\ifkorrekturansicht\else
\newenvironment{esempio}[3]%
{
    \vspace{1.5ex}
    \rlap{\underline{#1}}
    \par
    \setlength{\parindent}{0cm}
    \nopagebreak
    \leftskip=#2cm
    \rightskip=#3cm
}
{
    \par
}
\fi

\doendnotes{C}
\bigskip
\vfill

\clearpage

\footnotesize

\ifkorrekturansicht
  \lohead{\textsc{register}}
\fi

% theindex-Environment neu definieren ohne reledmac
\makeatletter
\renewenvironment{theindex}{%
  \ifkorrekturansicht
    \section*{\indexname}%
  \else
    \subsubsection*{Index der erwähnten Entitäten}%
  \fi
  \setlength{\parindent}{0pt}%
  \setlength{\parskip}{0pt plus 0.3pt}%
  \let\item\@idxitem
}{%
  \ifkorrekturansicht\clearpage\fi
}
\makeatother

\IfFileExists{\jobname-pw.ind}{\input{\jobname-pw.ind}}{}

% Quellenangabe nur in der Leseansicht
\ifkorrekturansicht\else
% Fallback-Definitionen, falls die .tex-Datei \titel etc. nicht gesetzt hat
\providecommand{\titel}{}
\providecommand{\editorInnen}{}
\providecommand{\dateiname}{\jobname}

\vspace{3cm}

\vfill

\footnotesize
\textsc{Quelle}: \titel. Herausgegeben von {\editorInnen}. In: \emph{Arthur Schnitzler: Briefwechsel mit Autorinnen und Autoren}.
 Digitale Edition, https://schnitzler-briefe.acdh.oeaw.ac.at/{\dateiname}.html (Stand \today)
\fi

\end{document}


