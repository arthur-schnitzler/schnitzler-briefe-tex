%% latex-leseansicht-vorspann.tex
%% Vorspann für die Leseansicht.
%% Lädt die gemeinsame Datei latex-vorspann.tex mit nicht gesetztem Schalter.

\newif\ifkorrekturansicht
\korrekturansichtfalse

\input{../tex-inputs/latex-vorspann}


         
         \renewcommand{\erwaehntePersonen}{Personen: Frieda Pollak, Karl Pollak}
         \renewcommand{\erwaehnteOrte}{Orte: Meidlinger Hauptstraße, Wien}
         \renewcommand{\erwaehnteWerke}{Werke: Margot und das Jugendgericht}
               \section[Robert Adam an Arthur Schnitzler, 7. 6. 1929]{ Robert Adam an Arthur Schnitzler, 7. 6. 1929}\nopagebreak\mylabel{v}\rehead{ }\begin{ledgroupsized}[t]{13cm}\normalsize\beginnumbering \toendnotes[C]{\smallbreak\pagebreak[2]} \Standort{CUL, Schnitzler, B 1.}
\physDesc{Brief, 1 Blatt, 3 Seiten
\newline{}Handschrift: schwarze Tinte, deutsche Kurrent
\newline{}Schnitzler: 1) mit rotem Buntstift beschriftet: »\textsc{Adam}« und  »MdlHptstr 58\oindex{Meidlinger Hauptstrasse@\textbf{Meidlinger Hauptstraße}|pw}«  2) mit rotem Buntstift vereinzelte Unterstreichungen\newline{}Ordnung: mit Bleistift von unbekannter Hand nummeriert:
                                    »21« }\Standort{Wien, Österreichische Nationalbibliothek, Cod.ser. 52.269, 32–33.}
\physDesc{handschriftliche Abschrift
\newline{}Handschrift: schwarze Tinte, Gabelsberger Kurzschrift}\Standort{Wien, Österreichische Nationalbibliothek, Cod.ser. 52.269, 32–33.}
\physDesc{maschinelle Abschrift
\newline{}Schreibmaschine}\toendnotes[C]{\smallbreak}\pstart
           \raggedleft{}{\pb}Wien\oindex{Wien@\textbf{Wien}|pw}, am 7. Juni 1929\pend
           \pstart{}Hochverehrter Herr Doktor!\pend\pstart
           Frl. Frieda Pollak\pwindex{Pollak, Frieda 08.12.1881 – 13.07.1937@\textsc{Pollak, Frieda} (08.12.1881 – 13.07.1937), \emph{Sekretärin}|pw} hat mir mitgeteilt, daß Sie
               die große Güte hatten, meine jüngſte Arbeit zu leſen und ſich für ihr Schickſal zu
               intereſſieren. Ich danke Ihnen, wie ſchon ſo oft, auf’s herzlichſte. Mit »Margot und das Jugendgericht\pwindex{Adam, Robert 20.04.1877 – 16.10.1961@\textsc{Adam, Robert} (20.04.1877 – 16.10.1961), \emph{Schriftsteller, Richter}!Margot und das Jugendgericht1931@\strich\emph{Margot und das Jugendgericht} {[}1931{]}|pw}« meine ich freilich
               nichts Schwerwiegendes und Hervorragendes geſchaffen zu haben, aber die frohe
               Befriedigung, die ich, trotz Alltags-Sorgen und -ärger, beim Schreiben {\pb}empfand, beſonders das eigene Vergnügen an
                  Margots\pwindex{Adam, Robert 20.04.1877 – 16.10.1961@\textsc{Adam, Robert} (20.04.1877 – 16.10.1961), \emph{Schriftsteller, Richter}!Margot und das Jugendgericht1931@\strich\emph{Margot und das Jugendgericht} {[}1931{]}|pwv} Erlebniſſen mit dem
               Heilpädagogen und in der Kaffeehausecke, gaben mir doch das ſichere Gefühl, daß die
               Geſchichte meiner Heldin auch andern etwas Sympathie, deren ſie ſie so dringend
               bedarf, abgewinnen könne. Wenn es mir gelänge, mit dieſem leichten Stück endlich
               einmal den ſo oft geſuchten Eingang zur Bühne zu finden, wäre es natürlich für mich
               von allergrößter Bedeutung. Nur haben mir die ſtäten Enttäuſchungen früherer Jahre
               das Hoffen gründlichſt abgewöhnt.\pend
           \pstart
           {\pb}Dürfte ich, hochverehrter Herr Doktor,
               nach langer Zeit wieder einmal perſönlich bei Ihnen vorſprechen? Jede Zeit wäre mir
               recht, und Frl. Pollak\pwindex{Pollak, Frieda 08.12.1881 – 13.07.1937@\textsc{Pollak, Frieda} (08.12.1881 – 13.07.1937), \emph{Sekretärin}|pw}, mit deren Bruder\pwindex{Pollak, Karl 07.10.1873 – 29.05.1940@\textsc{Pollak, Karl} (07.10.1873 – 29.05.1940), \emph{Richter}|pwv} ich in ſtetem Kontakt
               bin, würde es gewiß übernehmen, mir die Ihnen genehme Stunde mitzuteilen.\pend
           \pstart
           Mit ergebenſtem Gruß Ihr{\\[\baselineskip]}dankbarer{\\[\baselineskip]}\spacefill\mbox{D\textsuperscript{r}RAdam}\pend
           \leftskip=0em{}
         
         \endnumbering\mylabel{h}\end{ledgroupsized}  \newcommand{\dateiname}{L02510}\newcommand{\titel}{Robert Adam an Arthur Schnitzler, 7. 6. 1929}\newcommand{\editorInnen}{Martin Anton Müller und Gerd-Hermann Susen}%% latex-leseansicht-abspann.tex
%% Abspann für die Leseansicht.
%% Der Schalter \ifkorrekturansicht ist bereits durch den Vorspann gesetzt.

%% latex-abspann.tex
%% Gemeinsamer Abspann für Korrekturansicht und Leseansicht.
%% Setzt den Schalter \ifkorrekturansicht voraus (gesetzt in den
%% einbindenden Dateien latex-korrekturansicht-abspann.tex bzw.
%% latex-leseansicht-abspann.tex).
%% ---------------------------------------------------------------

\normalsize

% Das esempio-Environment wird nur in der Leseansicht benötigt
\ifkorrekturansicht\else
\newenvironment{esempio}[3]%
{
    \vspace{1.5ex}
    \rlap{\underline{#1}}
    \par
    \setlength{\parindent}{0cm}
    \nopagebreak
    \leftskip=#2cm
    \rightskip=#3cm
}
{
    \par
}
\fi

\doendnotes{C}
\bigskip
\vfill

\clearpage

\footnotesize

\ifkorrekturansicht
  \lohead{\textsc{register}}
\fi

% theindex-Environment neu definieren ohne reledmac
\makeatletter
\renewenvironment{theindex}{%
  \ifkorrekturansicht
    \section*{\indexname}%
  \else
    \subsubsection*{Index der erwähnten Entitäten}%
  \fi
  \setlength{\parindent}{0pt}%
  \setlength{\parskip}{0pt plus 0.3pt}%
  \let\item\@idxitem
}{%
  \ifkorrekturansicht\clearpage\fi
}
\makeatother

\IfFileExists{\jobname-pw.ind}{\input{\jobname-pw.ind}}{}

% Quellenangabe nur in der Leseansicht
\ifkorrekturansicht\else
% Fallback-Definitionen, falls die .tex-Datei \titel etc. nicht gesetzt hat
\providecommand{\titel}{}
\providecommand{\editorInnen}{}
\providecommand{\dateiname}{\jobname}

\vspace{3cm}

\vfill

\footnotesize
\textsc{Quelle}: \titel. Herausgegeben von {\editorInnen}. In: \emph{Arthur Schnitzler: Briefwechsel mit Autorinnen und Autoren}.
 Digitale Edition, https://schnitzler-briefe.acdh.oeaw.ac.at/{\dateiname}.html (Stand \today)
\fi

\end{document}


      