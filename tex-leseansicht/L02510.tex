%% latex-korrekturansicht-vorspann.tex
%% Vorspann für die Korrekturansicht.
%% Lädt die gemeinsame Datei latex-vorspann.tex mit gesetztem Schalter.

\newif\ifkorrekturansicht
\korrekturansichttrue

\input{../tex-inputs/latex-vorspann}


\section[Robert Adam an Arthur Schnitzler, 7. 6. 1929]{L02510 Robert Adam an Arthur Schnitzler, 7. 6. 1929}
\nopagebreak\mylabel{L02510v}
\rehead{ }\normalsize\beginnumbering\briefempfaengerindex{Schnitzler, Arthur@\textsc{Schnitzler, Arthur}!zzzAdam, Robert@\emph{von Robert Adam}!1929-06-071@{7. 6. 1929}|(be}
\toendnotes[C]{\smallbreak\pagebreak[2]}\Standort{CUL, Schnitzler, B 1.}
\physDesc{Brief, 1 Blatt, 3 Seiten, 1256 Zeichen
\newline{}Handschrift: schwarze Tinte, deutsche Kurrent
\newline{}Schnitzler: 1) mit rotem Buntstift beschriftet: »\textsc{Adam}« und »MdlHptstr 58\oindex{Meidlinger Hauptstrasse@\textbf{Meidlinger Hauptstraße}, \emph{Straße (K.STR)}|pw}«  2) mit rotem Buntstift vereinzelte Unterstreichungen
\newline{}Ordnung: mit Bleistift von unbekannter Hand nummeriert:
                                    »21« }\Standort{Wien, Österreichische Nationalbibliothek, Cod.ser. 52.269, 32–33.}
\physDesc{handschriftliche Abschrift1 Blatt, 2 Seiten, 1256 Zeichen
\newline{}Handschrift: schwarze Tinte, Gabelsberger Kurzschrift}\Standort{Wien, Österreichische Nationalbibliothek, Cod.ser. 52.269, 32–33.}
\physDesc{maschinenschriftliche Abschrift1 Blatt, 2 Seiten, 1256 Zeichen
\newline{}Schreibmaschine}\toendnotes[C]{\smallbreak}
\pstart
           \raggedleft{}{\pb}Wien\oindex{Wien@\textbf{Wien}, \emph{A.ADM2}|pw}, am 7. Juni 1929\pend
           
\pstart{}Hochverehrter Herr Doktor!\pend\vspace{0.5em}
\pstart
           Frl. Frieda Pollak\pwindex{Pollak, Frieda 08.12.1881 – 13.07.1937@\textsc{Pollak, Frieda} (08.12.1881 – 13.07.1937), \emph{Sekretär/Sekretärin}|pw} hat mir mitgeteilt, daß Sie
               die große Güte hatten, meine jüngſte Arbeit zu leſen und ſich für ihr Schickſal zu
               intereſſieren. Ich danke Ihnen, wie ſchon ſo oft, auf’s herzlichſte. Mit »Margot und das Jugendgericht\pwindex{Margot und das Jugendgericht@\emph{Margot und das Jugendgericht}|pw}« meine ich freilich
               nichts Schwerwiegendes und Hervorragendes geſchaffen zu haben, aber die frohe
               Befriedigung, die ich, trotz Alltags-Sorgen und -ärger, beim Schreiben {\pb}empfand, beſonders das eigene Vergnügen an
                  Margots\pwindex{Margot und das Jugendgericht@\emph{Margot und das Jugendgericht}|pwv} Erlebniſſen mit dem
               Heilpädagogen und in der Kaffeehausecke, gaben mir doch das ſichere Gefühl, daß die
               Geſchichte meiner Heldin auch andern etwas Sympathie, deren ſie ſie so dringend
               bedarf, abgewinnen könne. Wenn es mir gelänge, mit dieſem leichten Stück endlich
               einmal den ſo oft geſuchten Eingang zur Bühne zu finden, wäre es natürlich für mich
               von allergrößter Bedeutung. Nur haben mir die ſtäten Enttäuſchungen früherer Jahre
               das Hoffen gründlichſt abgewöhnt.\pend
           
\pstart
           {\pb}Dürfte ich, hochverehrter Herr Doktor,
               nach langer Zeit wieder einmal perſönlich bei Ihnen vorſprechen? Jede Zeit wäre mir
               recht, und Frl. Pollak\pwindex{Pollak, Frieda 08.12.1881 – 13.07.1937@\textsc{Pollak, Frieda} (08.12.1881 – 13.07.1937), \emph{Sekretär/Sekretärin}|pw}, mit deren Bruder\pwindex{Pollak, Karl 07.10.1873 – 29.05.1940@\textsc{Pollak, Karl} (07.10.1873 – 29.05.1940), \emph{Richter/Richterin}|pwv} ich in ſtetem Kontakt
               bin, würde es gewiß übernehmen, mir die Ihnen genehme Stunde mitzuteilen.\pend
           
\pstart
           Mit ergebenſtem Gruß Ihr{\\[\baselineskip]}dankbarer{\\[\baselineskip]}\spacefill\mbox{D\textsuperscript{r}RAdam}\pend
           \leftskip=0em{}\selectlanguage{ngerman}\endnumbering\briefempfaengerindex{Schnitzler, Arthur@\textsc{Schnitzler, Arthur}!zzzAdam, Robert@\emph{von Robert Adam}!1929-06-071@{7. 6. 1929}|)be}\mylabel{L02510h}  \normalsize

\doendnotes{C}
\bigskip
\vfill

\clearpage

\footnotesize

\lohead{\textsc{register}}

% Definiere theindex-Environment komplett neu ohne reledmac
\makeatletter
\renewenvironment{theindex}{%
  \section*{\indexname}%
  \setlength{\parindent}{0pt}%
  \setlength{\parskip}{0pt plus 0.3pt}%
  \let\item\@idxitem
}{%
  \clearpage
}
\makeatother

\IfFileExists{\jobname-pw.ind}{\input{\jobname-pw.ind}}{}

\end{document}

      