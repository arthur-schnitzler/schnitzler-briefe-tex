%% latex-leseansicht-vorspann.tex
%% Vorspann für die Leseansicht.
%% Lädt die gemeinsame Datei latex-vorspann.tex mit nicht gesetztem Schalter.

\newif\ifkorrekturansicht
\korrekturansichtfalse

\input{../tex-inputs/latex-vorspann}

\begin{center}
            \textcolor{red}{ENTWURF, NICHT FERTIG KORRIGIERT}
                      \end{center}
            
         
         \renewcommand{\erwaehntePersonen}{Personen: Siegfried Jacobsohn, Ottilie Salten, Olga Schnitzler}
         \renewcommand{\erwaehnteOrte}{Orte: Edmund-Weiß-Gasse, Riedhof, Wien, XVIII., Währing}
         \renewcommand{\erwaehnteWerke}{Werke: Der Fall Jacobsohn}
               \section[Felix Salten an Arthur Schnitzler, 19. 12. 1904]{ Felix Salten an Arthur Schnitzler, 19. 12. 1904}\nopagebreak\mylabel{v}\rehead{ }\begin{ledgroupsized}[t]{13cm}\normalsize\beginnumbering \toendnotes[C]{\smallbreak\pagebreak[2]} \Standort{CUL, Schnitzler, B 89, B 1.}
\physDesc{Kartenbrief, 591 Zeichen
\newline{}Handschrift: schwarze Tinte, lateinische Kurrent
\newline{}Versand: 1) Stempel: »\nobreak{}Wien \textcolor{gray}{66}, 20. 12. 04, 6–7V\nobreak{}«.   2) Stempel: »\nobreak{}\oindex{XVIII., Waehring@\textbf{XVIII., Währing}|pwk}18/1 Wien 110, 20. 12. 04, 12V, Bestellt\nobreak{}«. 
\newline{}Schnitzler: mit Bleistift datiert: »20/12 {[}1{]}904« 
\newline{}Ordnung: mit Bleistift von unbekannter Hand nummeriert:
                                    »194« }\toendnotes[C]{\smallbreak}\pstart{}{\pb}Herrn D\textsuperscript{r} Arthur Schnitzler\pend{}\pstart{}Wien\oindex{Wien@\textbf{Wien}|pw}\pend{}\pstart{}XVIII. Spöttelgaſse 7\oindex{Edmund-Weiss-Gasse@\textbf{Edmund-Weiß-Gasse}|pw}\pend{}{\bigskip}\pstart
           \raggedleft{}{\pb}Montag.\pend
           \pstart
           Lieber, wenn es Ihnen recht ist, treffen wir uns morgen
                  (Dienstag) oder Mittwoch Abend (½ 9) im \label{K_L03401-5v}\edtext{Riedhof\oindex{Riedhof@\textbf{Riedhof}|pw}}{\lemma{\textnormal{\emph{Riedhof}}}\Cendnote{\textnormal{Das Treffen fand erst am 23. 12. 1904 statt,
                  nachdem man sich am Vorabend noch verfehlt hatte.}}}\label{K_L03401-5h}. Da Otti\pwindex{Salten, Ottilie 07.03.1868 – 22.06.1942@\textsc{Salten, Ottilie} (07.03.1868 – 22.06.1942), \emph{Schauspielerin}|pw} nur auf 3 Stunden vom Haus fort kann ist das ein Ausweg.
               Sonst müßen wirs bis nach den Feiertagen laßen, außer Sie könnten Beide\pwindex{Schnitzler, Olga 17.01.1882 – 13.01.1970@\textsc{Schnitzler, Olga} (17.01.1882 – 13.01.1970), \emph{Schauspielerin, Sängerin}|pw} am Sonntag od. Montag Abend bei uns sein, was uns sehr
               freuen würde. \pend
           \pstart
           Es wäre mir nicht unwichtig bald mit Ihnen zu sprechen, da ich über den \label{K_L03401-1v}\edtext{Artikel}{\lemma{\textnormal{\emph{Artikel}}}\Cendnote{\textnormal{Arthur Schnitzler\pwindex{Schnitzler, Arthur 15.05.1862 – 21.10.1931@\textsc{Schnitzler, Arthur} (15.05.1862 – 21.10.1931), \emph{Schriftsteller, Mediziner}|pwk}: \emph{Der Fall Jacobsohn}. In: \emph{Die Zukunft}, Jg. 13, Bd. 49, Nr. 12, 17. 12. 1904, S. 401–404. A. S.: \emph{»Das Zeitlose ist von kürzester Dauer«}, Der Fall Jacobsohn, 17. 12. 1904}}}\label{K_L03401-1h}\pwindex{Schnitzler, Arthur 15.05.1862 – 21.10.1931@\textsc{Schnitzler, Arthur} (15.05.1862 – 21.10.1931), \emph{Schriftsteller, Mediziner}!Fall Jacobsohn17. 12. 1904@\strich\emph{Der Fall Jacobsohn} {[}17. 12. 1904{]}|pwv}, den Sie Herrn Siegfried Jacobsohn\pwindex{Jacobsohn, Siegfried 28.01.1881 – 03.12.1926@\textsc{Jacobsohn, Siegfried} (28.01.1881 – 03.12.1926), \emph{Journalist, Kritiker, Publizist}|pw}
               gewidmet haben, \label{K_L03401-3v}\edtext{manches wesentliche zu
                  bemerken}{\lemma{\textnormal{\emph{manches … bemerken}}}\Cendnote{\textnormal{vgl. A. S.: \emph{Tagebuch}, 20. 12. 1904: »Brief
                        Salten\pwindex{Salten, Felix 06.09.1869 – 08.10.1945@\textsc{Salten, Felix} (06.09.1869 – 08.10.1945), \emph{Schriftsteller, Journalist}|pw}s, mit Bemerkung, er hätte über
                     meinen Artikel J.\pwindex{Schnitzler, Arthur 15.05.1862 – 21.10.1931@\textsc{Schnitzler, Arthur} (15.05.1862 – 21.10.1931), \emph{Schriftsteller, Mediziner}!Fall Jacobsohn17. 12. 1904@\strich\emph{Der Fall Jacobsohn} {[}17. 12. 1904{]}|pwv}
                     wesentliches zu bemerken, irritirte mich. (Bin zum Journalisten nicht
                     geschaffen!)«}}}\label{K_L03401-3h} hätte. \pend
           \pstart
           Mit herzlichen Grüßen an Sie Beide\pwindex{Schnitzler, Olga 17.01.1882 – 13.01.1970@\textsc{Schnitzler, Olga} (17.01.1882 – 13.01.1970), \emph{Schauspielerin, Sängerin}|pwv} von Otti\pwindex{Salten, Ottilie 07.03.1868 – 22.06.1942@\textsc{Salten, Ottilie} (07.03.1868 – 22.06.1942), \emph{Schauspielerin}|pw} und mir\pend
           \pstart Ihr \spacefill\mbox{Salten}\pend{}\pstart
           \raggedleft{}\strikeout{\textcolor{gray}{Felix Salten }}\pend
           
         
         \endnumbering\mylabel{h}\end{ledgroupsized}\begin{anhang}\end{anhang}\newcommand{\dateiname}{L03401}\newcommand{\titel}{Felix Salten an Arthur Schnitzler, 19. 12. 1904}\newcommand{\editorInnen}{Martin Anton Müller und Laura Untner}%% latex-leseansicht-abspann.tex
%% Abspann für die Leseansicht.
%% Der Schalter \ifkorrekturansicht ist bereits durch den Vorspann gesetzt.

%% latex-abspann.tex
%% Gemeinsamer Abspann für Korrekturansicht und Leseansicht.
%% Setzt den Schalter \ifkorrekturansicht voraus (gesetzt in den
%% einbindenden Dateien latex-korrekturansicht-abspann.tex bzw.
%% latex-leseansicht-abspann.tex).
%% ---------------------------------------------------------------

\normalsize

% Das esempio-Environment wird nur in der Leseansicht benötigt
\ifkorrekturansicht\else
\newenvironment{esempio}[3]%
{
    \vspace{1.5ex}
    \rlap{\underline{#1}}
    \par
    \setlength{\parindent}{0cm}
    \nopagebreak
    \leftskip=#2cm
    \rightskip=#3cm
}
{
    \par
}
\fi

\doendnotes{C}
\bigskip
\vfill

\clearpage

\footnotesize

\ifkorrekturansicht
  \lohead{\textsc{register}}
\fi

% theindex-Environment neu definieren ohne reledmac
\makeatletter
\renewenvironment{theindex}{%
  \ifkorrekturansicht
    \section*{\indexname}%
  \else
    \subsubsection*{Index der erwähnten Entitäten}%
  \fi
  \setlength{\parindent}{0pt}%
  \setlength{\parskip}{0pt plus 0.3pt}%
  \let\item\@idxitem
}{%
  \ifkorrekturansicht\clearpage\fi
}
\makeatother

\IfFileExists{\jobname-pw.ind}{\input{\jobname-pw.ind}}{}

% Quellenangabe nur in der Leseansicht
\ifkorrekturansicht\else
% Fallback-Definitionen, falls die .tex-Datei \titel etc. nicht gesetzt hat
\providecommand{\titel}{}
\providecommand{\editorInnen}{}
\providecommand{\dateiname}{\jobname}

\vspace{3cm}

\vfill

\footnotesize
\textsc{Quelle}: \titel. Herausgegeben von {\editorInnen}. In: \emph{Arthur Schnitzler: Briefwechsel mit Autorinnen und Autoren}.
 Digitale Edition, https://schnitzler-briefe.acdh.oeaw.ac.at/{\dateiname}.html (Stand \today)
\fi

\end{document}


      