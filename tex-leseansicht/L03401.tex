%% latex-leseansicht-vorspann.tex
%% Vorspann für die Leseansicht.
%% Lädt die gemeinsame Datei latex-vorspann.tex mit nicht gesetztem Schalter.

\newif\ifkorrekturansicht
\korrekturansichtfalse

\input{../tex-inputs/latex-vorspann}


\section[ Felix Salten an Arthur Schnitzler, 19. 12. 1904]{L03401 Felix Salten an Arthur Schnitzler,  19. 12. 1904}
\nopagebreak\mylabel{L03401v}
\rehead{ }\normalsize\beginnumbering\briefempfaengerindex{Schnitzler, Arthur@\textsc{Schnitzler, Arthur}!zzzSalten, Felix@\emph{von Felix Salten}!1904-12-191@{19. 12. 1904}|(be}
\toendnotes[C]{\smallbreak\pagebreak[2]}
\correspDesc{Versand  durch Felix Salten am 19. 12. 1904 in Wien
\newline{}Übermittlung  am 20. 12. 1904 in Wien
\newline{}Erhalt  durch Arthur Schnitzler am 20. 12. 1904 in Wien}\toendnotes[C]{\smallbreak}
\Standort{CUL, Schnitzler, B 89, B 1.}
\physDesc{Kartenbrief, 583 Zeichen
\newline{}Handschrift: schwarze Tinte, lateinische Kurrent
\newline{}Versand: 1) Stempel: »\nobreak{}\oindex{Wien@\textbf{Wien}, \emph{Verwaltungsgebiet}|pwk}Wien \textcolor{gray}{5/1 66}, 20 12. 04, 6–7 V\nobreak{}«.   2) Stempel: »\nobreak{}\oindex{XVIII., Währing@\textbf{XVIII., Währing}, \emph{Verwaltungsgebiet}|pwk}18/1 Wien 110, 20. 12. 04, 12. V, Bestellt\nobreak{}«. 
\newline{}Schnitzler: mit Bleistift datiert: »20/12 904« 
\newline{}Ordnung: mit Bleistift von unbekannter Hand nummeriert: »194« }\toendnotes[C]{\smallbreak}\pstart{}{\pb}Herrn D\textsuperscript{r} Arthur Schnitzler\pend{}\pstart{}Wien\oindex{Wien@\textbf{Wien}, \emph{Verwaltungsgebiet}|pw}\pend{}\pstart{}XVIII. Spöttelgaße 7\oindex{Wien@\textbf{Wien}!XVIII., Währing@\textbf{XVIII., Währing}!Edmund-Weiß-Gasse 7@\textbf{Edmund-Weiß-Gasse 7}, \emph{Wohngebäude}|pw}\pend{}{\bigskip}\vspace{1em}
\pstart
           \raggedleft{}{\pb}Montag.\pend
           \vspace{0.5em}
\pstart
           Lieber, wenn es Ihnen recht ist, treffen wir uns morgen (Dienstag) oder
                  Mittwoch{ }Abend (½ 9) im \label{K_L03401-1v}\edtext{Riedhof\oindex{Wien@\textbf{Wien}!VIII., Josefstadt@\textbf{VIII., Josefstadt}!Riedhof@\textbf{Riedhof}, \emph{Lokal}|pw}}{\lemma{\textnormal{\emph{Riedhof}}}\Cendnote{\textnormal{Das Treffen fand erst am 23. 12. 1904 statt,
                  nachdem man sich am Vorabend noch verfehlt hatte. An den vorgeschlagenen
                  Feiertagen sahen sie sich nicht.}}}\label{K_L03401-1}. Da \label{K_L03401-2v}\edtext{Otti\pwindex{Salten, Ottilie 7.\,3.\,1868 Prag – 22.\,6.\,1942 Zürich@\textsc{Salten, Ottilie} (7.\,3.\,1868 Prag – 22.\,6.\,1942 Zürich), \emph{Schauspielerin}|pw} nur auf 3 Stunden vom Haus fort kann}{\lemma{\textnormal{\emph{Otti … kann}}}\Cendnote{\textnormal{Siehe XXXX Auszeichnungsfehler: Dokument L03400 nicht gefunden.
               }}}\label{K_L03401-2} ist das ein Ausweg. Sonst müßen wirs bis nach den Feiertagen laßen, außer Sie
               könnten Beide\pwindex{Schnitzler, Olga 17.\,1.\,1882 Wien – 13.\,1.\,1970 Lugano@\textsc{Schnitzler, Olga} (17.\,1.\,1882 Wien – 13.\,1.\,1970 Lugano), \emph{Schauspielerin, Sängerin}|pw} am Sonntag od. Montag{ }Abend bei uns sein, was uns sehr freuen würde.\pend
           
\pstart
           Es wäre mir nicht unwichtig bald mit Ihnen zu sprechen, da ich über den \label{K_L03401-3v}\edtext{Artikel}{\lemma{\textnormal{\emph{Artikel}}}\Cendnote{\textnormal{Siehe A. S.: \emph{»Das Zeitlose ist von kürzester Dauer«}, Der Fall Jacobsohn, 17. 12. 1904.
                  }}}\label{K_L03401-3}\pwindex{Schnitzler, Arthur 15.\,5.\,1862 Wien – 21.\,10.\,1931 ebd.@\textsc{Schnitzler, Arthur} (15.\,5.\,1862 Wien – 21.\,10.\,1931 ebd.), \emph{Schriftsteller, Mediziner}!Fall Jacobsohn@\strich\emph{Der Fall Jacobsohn}|pwv}, den Sie Herrn Siegfried Jacobsohn\pwindex{Jacobsohn, Siegfried 28.\,1.\,1881 Berlin – 3.\,12.\,1926 ebd.@\textsc{Jacobsohn, Siegfried} (28.\,1.\,1881 Berlin – 3.\,12.\,1926 ebd.), \emph{Journalist, Kritiker, Publizist}|pw}
               gewidmet haben, \label{K_L03401-4v}\edtext{manches wesentliche zu
                  bemerken}{\lemma{\textnormal{\emph{manches … bemerken}}}\Cendnote{\textnormal{Siehe A. S.: \emph{Tagebuch}, 20. 12. 1904: »Brief
                        Saltens\pwindex{Salten, Felix 6.\,9.\,1869 Budapest – 8.\,10.\,1945 Zürich@\textsc{Salten, Felix} (6.\,9.\,1869 Budapest – 8.\,10.\,1945 Zürich), \emph{Schriftsteller, Journalist, Chefredakteur}|pw}, mit Bemerkung, er hätte über
                     meinen Artikel J.\pwindex{Schnitzler, Arthur 15.\,5.\,1862 Wien – 21.\,10.\,1931 ebd.@\textsc{Schnitzler, Arthur} (15.\,5.\,1862 Wien – 21.\,10.\,1931 ebd.), \emph{Schriftsteller, Mediziner}!Fall Jacobsohn@\strich\emph{Der Fall Jacobsohn}|pwv}
                     wesentliches zu bemerken, irritirte mich. (Bin zum Journalisten nicht
                     geschaffen!)«}}}\label{K_L03401-4} hätte.\pend
           
\pstart
           Mit herzlichen Grüßen an Sie Beide\pwindex{Schnitzler, Olga 17.\,1.\,1882 Wien – 13.\,1.\,1970 Lugano@\textsc{Schnitzler, Olga} (17.\,1.\,1882 Wien – 13.\,1.\,1970 Lugano), \emph{Schauspielerin, Sängerin}|pwv} von Otti\pwindex{Salten, Ottilie 7.\,3.\,1868 Prag – 22.\,6.\,1942 Zürich@\textsc{Salten, Ottilie} (7.\,3.\,1868 Prag – 22.\,6.\,1942 Zürich), \emph{Schauspielerin}|pw} und mir\pend
           \pstart Ihr 
               \spacefill\mbox{\substVorne{}\textsuperscript{\textcolor{gray}{Felix Salten}}\substDazwischen{}Salten\substHinten{}}\pend{}\selectlanguage{ngerman}\endnumbering\briefempfaengerindex{Schnitzler, Arthur@\textsc{Schnitzler, Arthur}!zzzSalten, Felix@\emph{von Felix Salten}!1904-12-191@{19. 12. 1904}|)be}\mylabel{L03401h}  \newcommand{\dateiname}{L03401}\newcommand{\titel}{Felix Salten an Arthur Schnitzler, 19. 12. 1904}\newcommand{\editorInnen}{Martin Anton Müller und Laura Untner}%% latex-leseansicht-abspann.tex
%% Abspann für die Leseansicht.
%% Der Schalter \ifkorrekturansicht ist bereits durch den Vorspann gesetzt.

%% latex-abspann.tex
%% Gemeinsamer Abspann für Korrekturansicht und Leseansicht.
%% Setzt den Schalter \ifkorrekturansicht voraus (gesetzt in den
%% einbindenden Dateien latex-korrekturansicht-abspann.tex bzw.
%% latex-leseansicht-abspann.tex).
%% ---------------------------------------------------------------

\normalsize

% Das esempio-Environment wird nur in der Leseansicht benötigt
\ifkorrekturansicht\else
\newenvironment{esempio}[3]%
{
    \vspace{1.5ex}
    \rlap{\underline{#1}}
    \par
    \setlength{\parindent}{0cm}
    \nopagebreak
    \leftskip=#2cm
    \rightskip=#3cm
}
{
    \par
}
\fi

\doendnotes{C}
\bigskip
\vfill

\clearpage

\footnotesize

\ifkorrekturansicht
  \lohead{\textsc{register}}
\fi

% theindex-Environment neu definieren ohne reledmac
\makeatletter
\renewenvironment{theindex}{%
  \ifkorrekturansicht
    \section*{\indexname}%
  \else
    \subsubsection*{Index der erwähnten Entitäten}%
  \fi
  \setlength{\parindent}{0pt}%
  \setlength{\parskip}{0pt plus 0.3pt}%
  \let\item\@idxitem
}{%
  \ifkorrekturansicht\clearpage\fi
}
\makeatother

\IfFileExists{\jobname-pw.ind}{\input{\jobname-pw.ind}}{}

% Quellenangabe nur in der Leseansicht
\ifkorrekturansicht\else
% Fallback-Definitionen, falls die .tex-Datei \titel etc. nicht gesetzt hat
\providecommand{\titel}{}
\providecommand{\editorInnen}{}
\providecommand{\dateiname}{\jobname}

\vspace{3cm}

\vfill

\footnotesize
\textsc{Quelle}: \titel. Herausgegeben von {\editorInnen}. In: \emph{Arthur Schnitzler: Briefwechsel mit Autorinnen und Autoren}.
 Digitale Edition, https://schnitzler-briefe.acdh.oeaw.ac.at/{\dateiname}.html (Stand \today)
\fi

\end{document}


