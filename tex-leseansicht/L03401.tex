%% latex-korrekturansicht-vorspann.tex
%% Vorspann für die Korrekturansicht.
%% Lädt die gemeinsame Datei latex-vorspann.tex mit gesetztem Schalter.

\newif\ifkorrekturansicht
\korrekturansichttrue

\input{../tex-inputs/latex-vorspann}


\section[ Felix Salten an Arthur Schnitzler, 19. 12. 1904]{L03401 Felix Salten an Arthur Schnitzler, 19. 12. 1904}
\nopagebreak\mylabel{L03401v}
\rehead{ }\normalsize\beginnumbering\briefempfaengerindex{Schnitzler, Arthur@\textsc{Schnitzler, Arthur}!zzzSalten, Felix@\emph{von Felix Salten}!1904-12-191@{19. 12. 1904}|(be}
\toendnotes[C]{\smallbreak\pagebreak[2]}\Standort{CUL, Schnitzler, B 89, B 1.}
\physDesc{Kartenbrief, 583 Zeichen
\newline{}Handschrift: schwarze Tinte, lateinische Kurrent
\newline{}Versand: 1) Stempel: »\nobreak{}Wien \textcolor{gray}{5/1 66}, 20 12. 04, 6–7 V\nobreak{}«.   2) Stempel: »\nobreak{}\oindex{XVIII., Waehring@\textbf{XVIII., Währing}, \emph{A.ADM3}|pwk}18/1 Wien 110, 20. 12. 04, 12. V, Bestellt\nobreak{}«. 
\newline{}Schnitzler: mit Bleistift datiert: »20/12 904« 
\newline{}Ordnung: mit Bleistift von unbekannter Hand nummeriert: »194« }\toendnotes[C]{\smallbreak}\pstart{}{\pb}Herrn D\textsuperscript{r} Arthur Schnitzler\pend{}\pstart{}Wien\oindex{Wien@\textbf{Wien}, \emph{A.ADM2}|pw}\pend{}\pstart{}XVIII. Spöttelgaße 7\oindex{Edmund-Weiss-Gasse 7@\textbf{Edmund-Weiß-Gasse 7}, \emph{Wohngebäude (K.WHS)}|pw}\pend{}{\bigskip}\vspace{1em}
\pstart
           \raggedleft{}{\pb}Montag.\pend
           \vspace{0.5em}
\pstart
           Lieber, wenn es Ihnen recht ist, treffen wir uns morgen (Dienstag) oder
                  Mittwoch{ }Abend (½ 9) im \label{K_L03401-1v}\edtext{Riedhof\oindex{Riedhof@\textbf{Riedhof}, \emph{Lokal (K.LKL)}|pw}}{\lemma{\textnormal{\emph{Riedhof}}}\Cendnote{\textnormal{Das Treffen fand erst am 23. 12. 1904 statt,
                  nachdem man sich am Vorabend noch verfehlt hatte. An den vorgeschlagenen
                  Feiertagen sahen sie sich nicht.}}}\label{K_L03401-1}. Da \label{K_L03401-2v}\edtext{Otti\pwindex{Salten, Ottilie 07.03.1868 – 22.06.1942@\textsc{Salten, Ottilie} (07.03.1868 – 22.06.1942), \emph{Schauspieler/Schauspielerin}|pw} nur auf 3 Stunden vom Haus fort kann}{\lemma{\textnormal{\emph{Otti … kann}}}\Cendnote{\textnormal{Siehe Felix Salten an Arthur Schnitzler, [15. 12. 1904].
               }}}\label{K_L03401-2} ist das ein Ausweg. Sonst müßen wirs bis nach den Feiertagen laßen, außer Sie
               könnten Beide\pwindex{Schnitzler, Olga 17.01.1882 – 13.01.1970@\textsc{Schnitzler, Olga} (17.01.1882 – 13.01.1970), \emph{Schauspieler/Schauspielerin, Sänger/Sängerin}|pw} am Sonntag od. Montag{ }Abend bei uns sein, was uns sehr freuen würde.\pend
           
\pstart
           Es wäre mir nicht unwichtig bald mit Ihnen zu sprechen, da ich über den \label{K_L03401-3v}\edtext{Artikel}{\lemma{\textnormal{\emph{Artikel}}}\Cendnote{\textnormal{Siehe A. S.: \emph{»Das Zeitlose ist von kürzester Dauer«}, Der Fall Jacobsohn, 17. 12. 1904.
                  }}}\label{K_L03401-3}\pwindex{Fall Jacobsohn@\emph{Der Fall Jacobsohn}|pwv}, den Sie Herrn Siegfried Jacobsohn\pwindex{Jacobsohn, Siegfried 28.01.1881 – 03.12.1926@\textsc{Jacobsohn, Siegfried} (28.01.1881 – 03.12.1926), \emph{Journalist/Journalistin, Kritiker/Kritikerin, Publizist/Publizistin}|pw}
               gewidmet haben, \label{K_L03401-4v}\edtext{manches wesentliche zu
                  bemerken}{\lemma{\textnormal{\emph{manches … bemerken}}}\Cendnote{\textnormal{Siehe A. S.: \emph{Tagebuch}, 20. 12. 1904: »Brief
                        Saltens\pwindex{Salten, Felix 06.09.1869 – 08.10.1945@\textsc{Salten, Felix} (06.09.1869 – 08.10.1945), \emph{Schriftsteller/Schriftstellerin, Journalist/Journalistin, Chefredakteur/Chefredakteurin}|pw}, mit Bemerkung, er hätte über
                     meinen Artikel J.\pwindex{Fall Jacobsohn@\emph{Der Fall Jacobsohn}|pwv}
                     wesentliches zu bemerken, irritirte mich. (Bin zum Journalisten nicht
                     geschaffen!)«}}}\label{K_L03401-4} hätte.\pend
           
\pstart
           Mit herzlichen Grüßen an Sie Beide\pwindex{Schnitzler, Olga 17.01.1882 – 13.01.1970@\textsc{Schnitzler, Olga} (17.01.1882 – 13.01.1970), \emph{Schauspieler/Schauspielerin, Sänger/Sängerin}|pwv} von Otti\pwindex{Salten, Ottilie 07.03.1868 – 22.06.1942@\textsc{Salten, Ottilie} (07.03.1868 – 22.06.1942), \emph{Schauspieler/Schauspielerin}|pw} und mir\pend
           \pstart Ihr 
               \spacefill\mbox{\substVorne{}\textsuperscript{\textcolor{gray}{Felix Salten}}\substDazwischen{}Salten\substHinten{}}\pend{}\selectlanguage{ngerman}\endnumbering\briefempfaengerindex{Schnitzler, Arthur@\textsc{Schnitzler, Arthur}!zzzSalten, Felix@\emph{von Felix Salten}!1904-12-191@{19. 12. 1904}|)be}\mylabel{L03401h}  \normalsize

\doendnotes{C}
\bigskip
\vfill

\clearpage

\footnotesize

\lohead{\textsc{register}}

% Definiere theindex-Environment komplett neu ohne reledmac
\makeatletter
\renewenvironment{theindex}{%
  \section*{\indexname}%
  \setlength{\parindent}{0pt}%
  \setlength{\parskip}{0pt plus 0.3pt}%
  \let\item\@idxitem
}{%
  \clearpage
}
\makeatother

\IfFileExists{\jobname-pw.ind}{\input{\jobname-pw.ind}}{}

\end{document}

      