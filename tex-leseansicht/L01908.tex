%% latex-korrekturansicht-vorspann.tex
%% Vorspann für die Korrekturansicht.
%% Lädt die gemeinsame Datei latex-vorspann.tex mit gesetztem Schalter.

\newif\ifkorrekturansicht
\korrekturansichttrue

\input{../tex-inputs/latex-vorspann}


\section[Hermann Bahr an Arthur Schnitzler, 24. 12. 1909]{L01908 Hermann Bahr an Arthur Schnitzler, 24. 12. 1909}
\nopagebreak\mylabel{L01908v}
\rehead{ }\normalsize\beginnumbering\briefempfaengerindex{Schnitzler, Arthur@\textsc{Schnitzler, Arthur}!zzzBahr, Hermann@\emph{von Hermann Bahr}!1909-12-241@{24. 12. 1909}|(be}
\toendnotes[C]{\smallbreak\pagebreak[2]}\Standort{CUL, Schnitzler, B 5b.}
\physDesc{Brief, 1 Blatt, 1 Seite, 481 Zeichen
\newline{}Handschrift Lisa Clarus: schwarze Tinte, lateinische Kurrent
\newline{}Handschrift Hermann Bahr: schwarze Tinte (\noindent{}Unterschrift)
\newline{}Schnitzler: mit Bleistift ergänzt »Bahr« 
\newline{}Ordnung: mit Bleistift von unbekannter Hand nummeriert:
                                    »165« }
\buchAbdrucke{\weitereDrucke{Hermann Bahr, Arthur Schnitzler: \emph{Briefwechsel, Aufzeichnungen, Dokumente (1891–1931)}. Göttingen: \emph{Wallstein} 2018, S. 431.} }\toendnotes[C]{\smallbreak}
\pstart
           \raggedleft{}{\pb}24. 12. 09.\pend
           
\pstart
           \centering{}Wien XIII/\textsubscript{7}\oindex{Ober Sankt Veit@\textbf{Ober Sankt Veit}, \emph{P.PPLX}|pw}\pend
           
\pstart\center{}Lieber Artur!\pend\vspace{0.5em}
\pstart
           Ich freue mich sehr, dass Du Dienstag Vormittag kommen willst und man sich doch
               endlich, endlich wieder einmal aussprechen oder doch wenigstens gegenseitig anschauen
               kann, wonach mich längst stark verlangt! Ich will übrigens auch Deinen ärztlichen
               Rat, nicht für mich, aber für eine Figur meines neuen \label{K_L01908-1v}\edtext{Romans\pwindex{O Mensch@\emph{O Mensch{\rufezeichen}}|pwv}}{\lemma{\textnormal{\emph{Romans}}}\Cendnote{\textnormal{\emph{O Mensch!}\pwindex{O Mensch@\emph{O Mensch{\rufezeichen}}|pwk}}}}\label{K_L01908-1}.\pend
           
\pstart
           Und nun sei Dir, für Dich selbst, Deine verehrte Frau\pwindex{Schnitzler, Olga 17.01.1882 – 13.01.1970@\textsc{Schnitzler, Olga} (17.01.1882 – 13.01.1970), \emph{Schauspieler/Schauspielerin, Sänger/Sängerin}|pwv} und die Kinder\pwindex{Schnitzler, Heinrich 09.08.1902 – 12.07.1982@\textsc{Schnitzler, Heinrich} (09.08.1902 – 12.07.1982), \emph{Regisseur/Regisseurin, Schauspieler/Schauspielerin}|pwv}\pwindex{Cappellini, Lili 13.09.1909 – 26.07.1928@\textsc{Cappellini, Lili} (13.09.1909 – 26.07.1928)|pwv} weihnachtlich das allerschönste gewünscht!\pend
           
\pstart
           Herzlichst{\\[\baselineskip]}Dein alter{\\[\baselineskip]}\spacefill\mbox{{[}hs. :{]} HermannBahr}\pend
           \leftskip=0em{}\selectlanguage{ngerman}\endnumbering\briefempfaengerindex{Schnitzler, Arthur@\textsc{Schnitzler, Arthur}!zzzBahr, Hermann@\emph{von Hermann Bahr}!1909-12-241@{24. 12. 1909}|)be}\mylabel{L01908h}  \normalsize

\doendnotes{C}
\bigskip
\vfill

\clearpage

\footnotesize

\lohead{\textsc{register}}

% Definiere theindex-Environment komplett neu ohne reledmac
\makeatletter
\renewenvironment{theindex}{%
  \section*{\indexname}%
  \setlength{\parindent}{0pt}%
  \setlength{\parskip}{0pt plus 0.3pt}%
  \let\item\@idxitem
}{%
  \clearpage
}
\makeatother

\IfFileExists{\jobname-pw.ind}{\input{\jobname-pw.ind}}{}

\end{document}

      