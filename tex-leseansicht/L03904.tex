%% latex-leseansicht-vorspann.tex
%% Vorspann für die Leseansicht.
%% Lädt die gemeinsame Datei latex-vorspann.tex mit nicht gesetztem Schalter.

\newif\ifkorrekturansicht
\korrekturansichtfalse

\input{../tex-inputs/latex-vorspann}


\section[Arthur Schnitzler an Theodor Herzl, 4. 8. 1893]{L03904 Arthur Schnitzler an Theodor Herzl, 4. 8. 1893}
\nopagebreak\mylabel{L03904v}
\rehead{ }\normalsize\beginnumbering\briefempfaengerindex{Herzl, Theodor@\textsc{Herzl, Theodor}!zzzSchnitzler, Arthur@\emph{von Arthur Schnitzler}!1893-08-041@{4. 8. 1893}|(be}
\toendnotes[C]{\smallbreak\pagebreak[2]}
\correspDesc{Versand  durch Arthur Schnitzler am 4. 8. 1893 in Wien
\newline{}Erhalt  durch Theodor Herzl in Wien}\toendnotes[C]{\smallbreak}
\Standort{Jerusalem, Central Zionist Archives, H1:1924-9.}
\physDesc{,  Blätter,  Seiten
\newline{}Handschrift: , deutsche Kurrent}
\pstart
           {\pb}Wien\oindex{Wien@\textbf{Wien}, \emph{Verwaltungsgebiet}|pw}{ }4. 8. 93\pend
           
\pstart{}Verehrteſter Freund,\pend\vspace{0.5em}
\pstart
           haben Sie meinen Brief in die Schweiz\oindex{Schweiz@\textbf{Schweiz}|pw} beko{\geminationm}en?
               – Waren Sie in Oeſterreich\oindex{Österreich@\textbf{Österreich}|pw}? – Waren die in Wien\oindex{Wien@\textbf{Wien}, \emph{Verwaltungsgebiet}|pw}? Hab ich Sie verſäumt, während ich in Iſchl\oindex{Bad Ischl@\textbf{Bad Ischl}|pw} war? – Ich denke, es iſt irgend eine
               Nachricht verloren gegangen. Nun läßt mir das genaue {\pb}Studium der N. Fr. Pr.\pwindex{Neue Freie Presse@\emph{Neue Freie Presse}|pw} keinen Zweifel mehr
               übrig, dß Sie längſt wieder in Frankreich\oindex{Frankreich@\textbf{Frankreich}|pw} sind. –
               Ich gehe vielleicht am 20. Auguſt wieder auf 8–10 Tage weg.– Für alle
               Fälle hoff ich bald wieder mit 2 Worten zu erfahren, daß Sie u all
                  d\textcolor{gray}{ie} Ihren wohl{ }ſind.–\pend
           
\pstart
           Von meiner Sti{\geminationm}ung will ich lieber gar nicht reden; – der Strohhalm, mit dem ich
               mich an die Lebensfreude kla{\geminationm}ere, {\pb}iſt augenblicklich das
                  \textsc{Bicycle}! – Haben Sie{ }ſich auf Ihrer Reiſe nicht mit dem
               Drama eingelassen? – Sich nicht ausgeruht, indem Sie ein neues Lustspiel{ }ſchrieben?
               Refrain: Für alle Fälle hoffe ich bald wieder {\dots}{ }\textsc{etc.}{ }\textsc{etc.}–\pend
           
\pstart
           Herzlichen Gruß!{\\[\baselineskip]}Ganz der Ihre{\\[\baselineskip]}\spacefill\mbox{ArthurSchnitzler}\pend
           \leftskip=0em{}\selectlanguage{ngerman}\endnumbering\briefempfaengerindex{Herzl, Theodor@\textsc{Herzl, Theodor}!zzzSchnitzler, Arthur@\emph{von Arthur Schnitzler}!1893-08-041@{4. 8. 1893}|)be}\mylabel{L03904h}
\begin{anhang}
\end{anhang}\newcommand{\dateiname}{L03904}\newcommand{\titel}{Arthur Schnitzler an Theodor Herzl, 4. 8. 1893}\newcommand{\editorInnen}{Herausgegeben von Jahnke, SelmaMüller, Martin Anton}%% latex-leseansicht-abspann.tex
%% Abspann für die Leseansicht.
%% Der Schalter \ifkorrekturansicht ist bereits durch den Vorspann gesetzt.

%% latex-abspann.tex
%% Gemeinsamer Abspann für Korrekturansicht und Leseansicht.
%% Setzt den Schalter \ifkorrekturansicht voraus (gesetzt in den
%% einbindenden Dateien latex-korrekturansicht-abspann.tex bzw.
%% latex-leseansicht-abspann.tex).
%% ---------------------------------------------------------------

\normalsize

% Das esempio-Environment wird nur in der Leseansicht benötigt
\ifkorrekturansicht\else
\newenvironment{esempio}[3]%
{
    \vspace{1.5ex}
    \rlap{\underline{#1}}
    \par
    \setlength{\parindent}{0cm}
    \nopagebreak
    \leftskip=#2cm
    \rightskip=#3cm
}
{
    \par
}
\fi

\doendnotes{C}
\bigskip
\vfill

\clearpage

\footnotesize

\ifkorrekturansicht
  \lohead{\textsc{register}}
\fi

% theindex-Environment neu definieren ohne reledmac
\makeatletter
\renewenvironment{theindex}{%
  \ifkorrekturansicht
    \section*{\indexname}%
  \else
    \subsubsection*{Index der erwähnten Entitäten}%
  \fi
  \setlength{\parindent}{0pt}%
  \setlength{\parskip}{0pt plus 0.3pt}%
  \let\item\@idxitem
}{%
  \ifkorrekturansicht\clearpage\fi
}
\makeatother

\IfFileExists{\jobname-pw.ind}{\input{\jobname-pw.ind}}{}

% Quellenangabe nur in der Leseansicht
\ifkorrekturansicht\else
% Fallback-Definitionen, falls die .tex-Datei \titel etc. nicht gesetzt hat
\providecommand{\titel}{}
\providecommand{\editorInnen}{}
\providecommand{\dateiname}{\jobname}

\vspace{3cm}

\vfill

\footnotesize
\textsc{Quelle}: \titel. Herausgegeben von {\editorInnen}. In: \emph{Arthur Schnitzler: Briefwechsel mit Autorinnen und Autoren}.
 Digitale Edition, https://schnitzler-briefe.acdh.oeaw.ac.at/{\dateiname}.html (Stand \today)
\fi

\end{document}


