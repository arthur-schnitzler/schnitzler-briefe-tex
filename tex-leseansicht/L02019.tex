%% latex-korrekturansicht-vorspann.tex
%% Vorspann für die Korrekturansicht.
%% Lädt die gemeinsame Datei latex-vorspann.tex mit gesetztem Schalter.

\newif\ifkorrekturansicht
\korrekturansichttrue

\input{../tex-inputs/latex-vorspann}


\section[Arthur Schnitzler an Albert Ehrenstein, 6. 5. 1911]{L02019 Arthur Schnitzler an Albert Ehrenstein, 6. 5. 1911}
\nopagebreak\mylabel{L02019v}
\rehead{ }\normalsize\beginnumbering\briefempfaengerindex{Ehrenstein, Albert@\textsc{Ehrenstein, Albert}!zzzSchnitzler, Arthur@\emph{von Arthur Schnitzler}!1911-05-061@{6. 5. 1911}|(be}
\toendnotes[C]{\smallbreak\pagebreak[2]}\Standort{Jerusalem, The National Library of Israel, ARC. Ms. Var. 306 1 118.}
\physDesc{Brief, 6 Blätter, 6 Seiten, 5634 Zeichen
\newline{}Schreibmaschine\noindent{}Paginierung
\newline{}Handschrift: schwarze Tinte, deutsche Kurrent (\noindent{}Korrekturen und Unterschrift)}\Standort{DLA, A:Schnitzler, 85.1.642,1.}
\physDesc{Brief, Durchschlag6 Blätter, 6 Seiten, 5634 Zeichen
\newline{}Schreibmaschine
\newline{}Handschrift: roter Buntstift, lateinische Kurrent (\noindent{}Beschriftung: »\uline{Ehrenstein}« und Ergänzung: »auch«)}
\buchAbdrucke{\weitereDrucke{Arthur Schnitzler: \emph{Briefe 1875–1912}. Frankfurt am Main: \emph{S. Fischer} 1981, S. 657–660.} }\toendnotes[C]{\smallbreak}
\pstart
           \raggedleft{}{\pb}6. 5. 1911.\pend
           
\pstart
           \textcolor{gray}{\textbf{Dr. Arthur Schnitzler}}{\\}\textcolor{gray}{\textbf{Wien XVIII. Sternwartestrasse 71\oindex{Sternwartestrasse 71@\textbf{Sternwartestraße 71}, \emph{Wohngebäude (K.WHS)}|pw}}}\pend
           
\pstart{}Sehr geehrter Herr Doktor.\pend\vspace{0.5em}
\pstart
           Auch für mich war die Angelegenheit erledigt, woran mein letzter Brief an Sie einen
               Zweifel überhaupt nicht zuliess. Ich \substVorne{}\textsuperscript{hätte}\substDazwischen{}könnte\substHinten{} die Sache auch weiterhin auf sich beruhen lassen, umso mehr als Sie selbst
               durch eine ganze Reihe von Wochen sich zu einer Entschuldigung nicht gedrängt
               fühlten; doch Ihr Schreiben vom 27., das ich von einer Reise heimkehrend
               vorfinde, veranlasst mich zu folgender Erklärung und Abfertigung:\pend
           
\pstart
           Also: In jenem Gespräch zwischen Ihnen und mir war, wie von vielen Menschen und
               Dingen, im Anschluss an eine persönliche Erfahrung von Ihnen, die Sie glaubten mir
               erzählen zu müssen, auch von Herrn Stefan
                  Grossmann\pwindex{Grossmann, Stefan 19.05.1875 – 03.01.1935@\textsc{Großmann, Stefan} (19.05.1875 – 03.01.1935), \emph{Schriftsteller/Schriftstellerin, Journalist/Journalistin}|pw} die Rede und zwar von diesem mit dem aufrichtigsten Widerwillen
               sowohl Ihrer- als meinerseits. Dass er meine Gefühle für ihn kennt zweifle ich
               übrigens nicht; sollte es nicht der Fall sein, so habe ich jetzt jedenfalls den
               richtigen {\pb}Weg gewählt ihm diesen Umstand zur Kenntnis zu
               bringen. Ich zweifle auch nicht daran, dass er meine Gefühle erwidert. Niemals aber –
               ich wiederhole es – haben Sie mir gegenüber eine Aeusserung getan, die auch nur so
               hätte gedeutet werden können, als benütze Herr Grossmann\pwindex{Grossmann, Stefan 19.05.1875 – 03.01.1935@\textsc{Großmann, Stefan} (19.05.1875 – 03.01.1935), \emph{Schriftsteller/Schriftstellerin, Journalist/Journalistin}|pw} seine Stellung zur Erreichung erotischer Vorteile bei
               Schauspielerinnen. Dass Sie dergleichen zu mir geäussert hätten ist eine
               Erinnerungstäuschung von Ihnen, die nun freilich im Laufe der Wochen, während deren
               diese ganze Angelegenheit \introOben{}auch\introOben{}
                für Sie erledigt
               schien, Zeit genug hatte, in Ihnen unausrottbare Wurzeln zu fassen; und es ist eine
               Erinnerungstäuschung noch gröberer Art, dass ich Ihnen den Inhalt einer solchen
               Aeusserung auch nur mit einer Silbe bestätigt hätte. Ich erkläre hier nochmals auf
               das Allerdezidierteste, dass ich von der Existenz eines solchen Gerüchtes erst aus
               dem Brief des Herrn Grossmann\pwindex{Grossmann, Stefan 19.05.1875 – 03.01.1935@\textsc{Großmann, Stefan} (19.05.1875 – 03.01.1935), \emph{Schriftsteller/Schriftstellerin, Journalist/Journalistin}|pw} Kenntnis
               erhalten habe, in dem er mir die überraschende Mitteilung machte, dass Sie sich zu
               verschiedenen Leuten, unter denen er Herrn Kraus\pwindex{Kraus, Karl 28.04.1874 – 12.06.1936@\textsc{Kraus, Karl} (28.04.1874 – 12.06.1936), \emph{Schriftsteller/Schriftstellerin, Publizist/Publizistin, Schriftsteller/Schriftstellerin}|pw} nannte, geäussert hätten, von mir sei Ih{\pb}nen
               jenes Gerücht bestätigt worden. (Da nun Herr Grossmann\pwindex{Grossmann, Stefan 19.05.1875 – 03.01.1935@\textsc{Großmann, Stefan} (19.05.1875 – 03.01.1935), \emph{Schriftsteller/Schriftstellerin, Journalist/Journalistin}|pw} ausdrücklich Herrn Kraus\pwindex{Kraus, Karl 28.04.1874 – 12.06.1936@\textsc{Kraus, Karl} (28.04.1874 – 12.06.1936), \emph{Schriftsteller/Schriftstellerin, Publizist/Publizistin, Schriftsteller/Schriftstellerin}|pw}
               als denjenigen nannte, vor dem Sie mich fälschlicherweise als Bestätiger eines
               Tratsches angegeben haben, so war es natürlich nicht zu vermeiden in einem Brief, der
               Sie deswegen zur Rede stellte, den Namen des von Herrn Grossmann\pwindex{Grossmann, Stefan 19.05.1875 – 03.01.1935@\textsc{Großmann, Stefan} (19.05.1875 – 03.01.1935), \emph{Schriftsteller/Schriftstellerin, Journalist/Journalistin}|pw} geführten Zeugen zu \substVorne{}\textsuperscript{übergeben.}\substDazwischen{}\label{T_L02019-1v}\edtext{nennen.}{\lemma{\textnormal{\emph{nennen.}}}\Cendnote{\textnormal{eigentlich »zu« nochmals eingefügt, aber die Doppelung
                        analog zum richtig korrigierten Durchschlag behoben}}}\label{T_L02019-1}\substHinten{}\pend
           
\pstart
           Ihr kniffiger Versuch mich irgendwie dafür verantwortlich zu machen, dass Herr Kraus\pwindex{Kraus, Karl 28.04.1874 – 12.06.1936@\textsc{Kraus, Karl} (28.04.1874 – 12.06.1936), \emph{Schriftsteller/Schriftstellerin, Publizist/Publizistin, Schriftsteller/Schriftstellerin}|pw} in dieser Sache genannt werden musste,
               bedeutet am Ende nichts mehr als einen Strich mehr zu Ihrer Charakterphysiognomie,
               der nicht fehlen durfte.) Hätte ich vermuten können, dass Privatgespräche zwischen
               Ihnen und mir von Ihnen überhaupt weitergetragen werden, so hätte ich vielleicht,
               auch schon in früheren Fällen, den\strikeout{en} einen oder
               andern meiner Ausdrücke parlamentarischer gewählt; noch wahrscheinlicher ist
               freilich, dass ich auf das Vergnügen mich mit Ihnen zu unterhalten vollkommen
               verzichtet hätte. Dies eine aber steht fest, dass ich inhaltlich für alles, was ich
               sage, selbst wenn es {\pb}sich auf dem erbärmlichen Wege eines
               Klatsches weiterverbreitet, durchaus einzustehen in der Lage bin. Aber natürlich nur
               für das, was ich wirklich gesagt habe, nicht für das, was Misverstand, schlechtes
               Gedächtnis, Entstellung daraus zu machen belieben. Ich urteile stets nach eigenen
               Eindrücken und Erfahrungen; schon darum könnte es mir nie passieren irgend etwas
               nachzureden, was mir irgend ein Anderer hinterbracht hätte. Eine Bestätigung, wie Sie
               sie mir in den Mund legen wollen, könnte ich nie und nimmer ausgesprochen haben,
               schon weil \substVorne{}\textsuperscript{mich in}\substDazwischen{}nach\substHinten{} meiner Kenntnis jedes Substrat dafür fehlte; und nicht der Dümmste oder
               Gemeinste meiner Widersacher wird mir jemals zumuten, dass ich über einen Menschen,
               so geringe Sympathie ich für ihn auch hegte, irgend etwas erfinden sollte, was von
               manchen Menschen ob mit Recht oder Unrecht als ehrenrührig angesehen werden könnte.
               Ihre Bemerkung, dass Sie meine Behauptungen geradeso wie die Ihren mit Bedauern
               zurückziehen, weise ich als \substVorne{}\textsuperscript{deplaziert}\substDazwischen{}völlig unangebracht\substHinten{} zurück, und verbitte mir mit aller Ent{\pb}schiedenheit
               das, was Sie sich erlauben als Ihr Entgegenkommen zu bezeichnen; ebenso schüttle ich
               die Versicherung Ihrer Dankbarkeit von mir ab, auf die ich niemals irgend einen
               Anspruch erhoben habe. Nur aus Interesse für Ihr Talent habe ich die Manuscripte
               gelesen, die Sie mich baten mir vorlegen zu dürfen und \introOben{}habe\introOben{}
               versucht sie nach dem geringen Mass meines Einflusses schriftlich oder mündlich
               weiter zu empfehlen. Und wenn ich Ihnen die Empfehlung nicht zu geben vermochte, die
               Sie bei Ihrem letzten so reichhaltigen Besuche wünschten, eine Empfehlung für irgend
               ein Ministerium, so lag das nicht etwa daran, dass ich Sie für politische Dienste
                  \strikeout{für} unfähig hielte, sondern nur daran, dass mir
               die Verbindungen nach jener Richtung leider nicht zu Gebote stehen. Auch zu
               Gesprächen mit Ihnen habe ich mir gerne Zeit genommen und mich oft genug an manchen
               Ihrer kuriosen und boshaften Wendungen ergötzt. Aber absolut keine Zeit habe ich dazu
               mich um die abenteuerlichen Schicksale jener Gespräche in Ihrem Kopf und daraus
               entstehende Fol{\pb}gen zu kümmern. Und absolut keine Lust verspür
               ich mich auch nur eine Minute länger mit einem widerwärtigen Klatsch zu beschäftigen,
               in den, meines Wissens zum ersten Mal in meinem Leben, \introOben{}mir
                  durch\introOben{} Ihre,
               ausschliesslich Ihre Schuld, mein Name hineingezerrt wurde. In den Ekel, den ich
               dieser Tatsache gegenüber empfinde lassen Sie mich heute meinen endgiltigen Abschied
               von Ihnen den Ausdruck meines lebhaften Bedauerns hinzufügen, dass sich die Türe
               meiner Wohnung Ihnen jemals aufgetan hat.\pend
           
\pstart
           Hochachtungsvoll{\\[\baselineskip]}\spacefill\mbox{Dr Arthur Schnitzler}\pend
           \leftskip=0em{}
\pstart
           \noindent{}Herrn Dr. Albert Ehrenstein, Wien\oindex{Wien@\textbf{Wien}, \emph{A.ADM2}|pw}.\pend
           \selectlanguage{ngerman}\endnumbering\briefempfaengerindex{Ehrenstein, Albert@\textsc{Ehrenstein, Albert}!zzzSchnitzler, Arthur@\emph{von Arthur Schnitzler}!1911-05-061@{6. 5. 1911}|)be}\mylabel{L02019h}  \normalsize

\doendnotes{C}
\bigskip
\vfill

\clearpage

\footnotesize

\lohead{\textsc{register}}

% Definiere theindex-Environment komplett neu ohne reledmac
\makeatletter
\renewenvironment{theindex}{%
  \section*{\indexname}%
  \setlength{\parindent}{0pt}%
  \setlength{\parskip}{0pt plus 0.3pt}%
  \let\item\@idxitem
}{%
  \clearpage
}
\makeatother

\IfFileExists{\jobname-pw.ind}{\input{\jobname-pw.ind}}{}

\end{document}

      