%% latex-korrekturansicht-vorspann.tex
%% Vorspann für die Korrekturansicht.
%% Lädt die gemeinsame Datei latex-vorspann.tex mit gesetztem Schalter.

\newif\ifkorrekturansicht
\korrekturansichttrue

\input{../tex-inputs/latex-vorspann}


\section[Felix Salten an Arthur Schnitzler, 12. 9. 1891]{L03105 Felix Salten an Arthur Schnitzler, 12. 9. 1891}
\nopagebreak\mylabel{L03105v}
\rehead{ }\normalsize\beginnumbering\briefempfaengerindex{Schnitzler, Arthur@\textsc{Schnitzler, Arthur}!zzzSalten, Felix@\emph{von Felix Salten}!1891-09-121@{12. 9. 1891}|(be}
\toendnotes[C]{\smallbreak\pagebreak[2]}\Standort{CUL, Schnitzler, B 89, A 1.}
\physDesc{Brief, 1 Blatt, 2 Seiten, 2776 Zeichen
\newline{}Handschrift: schwarze Tinte, lateinische Kurrent
\newline{}Ordnung: mit Bleistift von unbekannter Hand nummeriert: »7« }\toendnotes[C]{\smallbreak}
\pstart
           \centering{}{\pb}Miskolcz\oindex{Miskolc@\textbf{Miskolc}, \emph{P.PPLA}|pw}{ }12. IX. 91.\pend
           \vspace{0.5em}
\pstart
           Lieber Freund! Herzlichen Dank für Ihre beiden \label{K_L03105-1v}\edtext{Briefe}{\lemma{\textnormal{\emph{Briefe}}}\Cendnote{\textnormal{Siehe Arthur Schnitzler an Felix Salten, [10.? 9. 1891].
               }}}\label{K_L03105-1} und verzeihen Sie, dass ich heftig wurde. Aber wenn man beinahe 100 Meilen
               weit von Wien\oindex{Wien@\textbf{Wien}, \emph{A.ADM2}|pw} ist fühlt man sich so
               ohnmächtig –. Ihr Brief, der erste nämlich ist verloren gegangen. Ich bin sehr froh,
               dass es Ihnen leidlich geht. Wann muss man zum \label{K_L03105-2v}\edtext{Engagement in Tr.\oindex{Opava@\textbf{Opava}, \emph{P.PPL}|pw}}{\lemma{\textnormal{\emph{Engagement in Tr.}}}\Cendnote{\textnormal{Marie Glümer\pwindex{Gluemer, Marie 03.07.1867 – 16.11.1925@\textsc{Glümer, Marie} (03.07.1867 – 16.11.1925), \emph{Schauspieler/Schauspielerin}|pwk} stand 
                  in Verhandlung wegen eines Engagements in Troppau\oindex{Opava@\textbf{Opava}, \emph{P.PPL}|pwk} (Opava\oindex{Opava@\textbf{Opava}, \emph{P.PPL}|pwk}), vgl. A. S.: \emph{Tagebuch}, 25. 8. 1891.
               }}}\label{K_L03105-2} eintreffen? Was das Arbeiten anlangt, geht es mir wie Ihnen. Ich habe keine
               Zeile geschrieben. Es war auch physisch unmöglich. Mein Rückfall\pwindex{Karlsburg, Bertha @\textsc{Karlsburg, Bertha}, \emph{Schauspieler/Schauspielerin}|pwuv} ist ziemlich
               unerklärlich, aber darum nicht weniger heftig. Was ich hier\oindex{Miskolc@\textbf{Miskolc}, \emph{P.PPLA}|pwv} leide, ist entsetzlich. Mein einziges
               Hülfsmittel ist das Kutschiren. Ich bin auch hier schon als rasender Fahrer bekannt,
               und mein Papa\pwindex{Salzmann, Philipp 1831-12-24 – 1905-04-02@\textsc{Salzmann, Philipp} (1831-12-24 – 1905-04-02), \emph{Bergbauunternehmer/Bergbauunternehmerin, Projektemacher/Projektemacherin}|pwv} fürchtet sich
               zu fahren, wenn ich kutschire. Es ist eine Wolthat, sage ich Ihnen, wenn man so
               gequält ist, dass man laut aufschreien möchte und man hat zwei Pferde und eine
               Peitsche in der Hand, die glatte Landstraße vor sich, und kann so sausen wie der
               Wind. Ich habe mich in meiner Verzweiflung erbötig gemacht, unseren neuen Bergdirektor\pwindex{?? [Bergdirektor] @\textsc{?? [Bergdirektor]}|pwv}\textcolor{gray}{,} sowie einen Ingenieur\pwindex{?? [Ingenieur] @\textsc{?? [Ingenieur]}|pwv} zu den Gruben nach Upony\oindex{Uppony@\textbf{Uppony}, \emph{P.PPL}|pw} zu
               fahren. Der erstere musste den Punkt suchen, wo der Einstich beginnen sollte, der
               zweite die Trace der Eisenbahn, welche gebaut werden soll, bestimmen. Wir fuhren um \uuline{4 Uhr} morgens aus – haben Sie gerne, was?,
               – und ich legte unter einem fürchterlichen Anfall von \label{K_L03105-3v}\edtext{\begin{otherlanguage}{french}image physi\textcolor{gray}{q}\end{otherlanguage}}{\lemma{\textnormal{\emph{image physiq}}}\Cendnote{\textnormal{französisch, eigentlich »image
                  physique«: physisches Erscheinungsbild; hier möglicherweise eine Anlehnung 
               an Diskurse aus der Hypnose, bei denen durch die Überlagerung eines Vorstellungsbildes
               eine Realität geschaffen wird.}}}\label{K_L03105-3} den Weg der sonst 8 Stunden dauert in 5 ½{ }zurück. Dazu kam, dass der
               junge Ingenieur\pwindex{?? [Ingenieur] @\textsc{?? [Ingenieur]}|pwv} (typisch
                  ungar\oindex{Ungarn@\textbf{Ungarn}, \emph{A.PCLI}|pwv}ischer Jude) sich bei
               mir angenehm machen wollte. Als wir durch den Uponyer Engpaß\oindex{Upponyi-szoros@\textbf{Upponyi-szoros}, \emph{Straße (K.STR)}|pw} fuhren, umringt von hohen Bergen, in denen mächtige
               Kohlenlager enthalten sind, be{\pb}gann der Mensch\pwindex{?? [Ingenieur] @\textsc{?? [Ingenieur]}|pwv} neben mir
               enthusiastisch zu werden, und mir von der »Mutter Natur« zu reden. Ich glaubte, ich
               müsse vom Wagen springen, und laut schreiend in’s Kafé Kremser\oindex{Cafe Kremser@\textbf{Café Kremser}, \emph{Kaffeehaus (K.KAF)}|pw} zu
               laufen, um mit Ihnen über die lächerliche Begeisterung des \uline{widerlichen} 1. Grades zu schimpfen. Das wird jedoch bald geschehen, und dann
               werde ich Ihnen das Milieu\oindex{Miskolc@\textbf{Miskolc}, \emph{P.PPLA}|pwv}
               schildern, in das ich hier gerathen bin. Schrecklich ist mir hier das Umworbenwerden,
               das Herandrängen der Familien u. das plumpe Angeln der Mütter u. Töchter. Mein Bruder
                  Emil\pwindex{Salzmann, Michael Emil 1858-01-19 – 1908-06-26@\textsc{Salzmann, Michael Emil} (1858-01-19 – 1908-06-26), \emph{Versicherungsbeamter/Versicherungsbeamtin}|pw} – »is scho hin \textcolor{gray}{×} is scho hin!«\pend
           
\pstart
           Mit \label{K_L03105-4v}\edtext{Italien\oindex{Italien@\textbf{Italien}, \emph{A.PCLI}|pw}}{\lemma{\textnormal{\emph{Italien}}}\Cendnote{\textnormal{Siehe Arthur Schnitzler an Felix Salten, [10.? 9. 1891].
               }}}\label{K_L03105-4} sieht's schlecht aus. Papa\pwindex{Salzmann, Philipp 1831-12-24 – 1905-04-02@\textsc{Salzmann, Philipp} (1831-12-24 – 1905-04-02), \emph{Bergbauunternehmer/Bergbauunternehmerin, Projektemacher/Projektemacherin}|pw} beginnt
               den \label{K_L03105-5v}\edtext{Betrieb}{\lemma{\textnormal{\emph{Betrieb}}}\Cendnote{\textnormal{Philipp Salzmann\pwindex{Salzmann, Philipp 1831-12-24 – 1905-04-02@\textsc{Salzmann, Philipp} (1831-12-24 – 1905-04-02), \emph{Bergbauunternehmer/Bergbauunternehmerin, Projektemacher/Projektemacherin}|pwk} war Unternehmer, der beruflich
                  mehrere Versuche unternahm, Erfolg zu haben, aber zumeist erfolglos blieb. In 
                  Miskolcz\oindex{Miskolc@\textbf{Miskolc}, \emph{P.PPLA}|pwk} versuchte er, einen Kohleabbau
                  aufzuziehen.}}}\label{K_L03105-5} und
               ich sehe, wie die Tausende nur so fliegen. Es wird schwer halten an ihn
               heranzutreten. Auf jeden Fall \label{K_L03105-6v}\edtext{sehe ich
                  Sie}{\lemma{\textnormal{\emph{sehe ich
                  Sie}}}\Cendnote{\textnormal{Siehe Felix Salten an Arthur Schnitzler, [28. 9. 1891?].
               }}}\label{K_L03105-6} im Verlaufe dieser Woche, und freue mich schon sehr darauf.\pend
           
\pstart
           Leben Sie recht wol, und berauschen Sie sich immerhin an der \label{K_L03105-7v}\edtext{Lüge, die nach Wahrheit duftet}{\lemma{\textnormal{\emph{Lüge, … duftet}}}\Cendnote{\textnormal{Siehe Arthur Schnitzler an Felix Salten, [10.? 9. 1891].
               }}}\label{K_L03105-7}, auch ich suche u. ersehne diesen Duft; – es ist ja unser Beider Schicksal,
               die wir nach der Wahrheit lechzen, dass wir uns am Duft der Lüge betäuben, und daher
               auch unser Hass gegen die Nüchternen.\pend
           
\pstart
           Grüßen Sie mir alle Herren die uns lieb sind, und senden Sie auch von mir die besten
               Wünsche mit nach Tropp\oindex{Opava@\textbf{Opava}, \emph{P.PPL}|pw}.\pend
           \pstart  Ihr herzlich ergebener \spacefill\mbox{Felix S.}\pend{}\selectlanguage{ngerman}\endnumbering\briefempfaengerindex{Schnitzler, Arthur@\textsc{Schnitzler, Arthur}!zzzSalten, Felix@\emph{von Felix Salten}!1891-09-121@{12. 9. 1891}|)be}\mylabel{L03105h}  \normalsize

\doendnotes{C}
\bigskip
\vfill

\clearpage

\footnotesize

\lohead{\textsc{register}}

% Definiere theindex-Environment komplett neu ohne reledmac
\makeatletter
\renewenvironment{theindex}{%
  \section*{\indexname}%
  \setlength{\parindent}{0pt}%
  \setlength{\parskip}{0pt plus 0.3pt}%
  \let\item\@idxitem
}{%
  \clearpage
}
\makeatother

\IfFileExists{\jobname-pw.ind}{\input{\jobname-pw.ind}}{}

\end{document}

      