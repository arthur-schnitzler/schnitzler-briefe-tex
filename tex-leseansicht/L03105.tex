%% latex-leseansicht-vorspann.tex
%% Vorspann für die Leseansicht.
%% Lädt die gemeinsame Datei latex-vorspann.tex mit nicht gesetztem Schalter.

\newif\ifkorrekturansicht
\korrekturansichtfalse

\input{../tex-inputs/latex-vorspann}

\begin{center}
            \textcolor{red}{ENTWURF, NICHT FERTIG KORRIGIERT}
                      \end{center}
            
         
         \renewcommand{\erwaehntePersonen}{Personen:  ?? [Bergdirektor],  ?? [Ingenieur], Philipp Salzmann, Michael Emil Salzmann}
         \renewcommand{\erwaehnteOrte}{Orte: Café Kremser, Italien, Miskolc, Opava, Uppony, Upponyi-szoros, Wien}
         \renewcommand{\erwaehnteWerke}{}
               \section[Felix Salten an Arthur Schnitzler, 12. 9. 1891]{ Felix Salten an Arthur Schnitzler, 12. 9. 1891}\nopagebreak\mylabel{v}\rehead{ }\begin{ledgroupsized}[t]{13cm}\normalsize\beginnumbering \toendnotes[C]{\smallbreak\pagebreak[2]} \Standort{CUL, Schnitzler, B 89, A 1.}
\physDesc{Brief, 1 Blatt, 2 Seiten, 2797 Zeichen
\newline{}Handschrift: schwarze Tinte, lateinische Kurrent
\newline{}Ordnung: mit Bleistift von unbekannter Hand nummeriert:
                                 »7« }\toendnotes[C]{\smallbreak}\pstart
           {\pb}Miskolcz\oindex{Miskolc@\textbf{Miskolc}|pw}, 12. IX. 91.
               \pend
           \pstart
           Lieber Freund! Herzlichen Dank für Ihre beiden Briefe und verzeihen
               Sie, dass ich heftig wurde. Aber wenn man beinahe 100 Meilen weit von Wien\oindex{Wien@\textbf{Wien}|pw} ist fühlt man sich so ohnmächtig. \pend
           \pstart
           – Ihr Brief, der erste nämlich ist verloren gegangen. Ich bin sehr froh, dass es
               ihnen leidlich geht. Wann muss man zum Engagement in Tr.\oindex{Opava@\textbf{Opava}|pw} eintreffen? Was das Arbeiten anlangt, geht es mir wie Ihnen. Ich habe
               keine Zeile geschrieben. Es war auch physisch unmöglich. Mein Rückfall ist ziemlich
               unerklärlich, aber darum nicht weniger heftig. Was ich hier leide, ist entsetzlich.
               Mein einziges Hülfsmittel ist das Kutschieren. Ich bin auch hier schon als rasender
               Fahrer bekannt, und mein Papa\pwindex{Salzmann, Philipp 1831-12-24 – 1905-04-02@\textsc{Salzmann, Philipp} (1831-12-24 – 1905-04-02), \emph{Unternehmer}|pwv} fürchtet sich zu fahren, wenn ich kutschiere. Es ist eine Wolthat, sage
               ich Ihnen, wenn man so gequält ist, dass man laut aufschreien möchte und man hat zwei
               Pferde und eine Peitsche in der Hand, die glatte Landstraße vor sich, und kann so
               sausen wie der Wind. Ich habe mich in meiner Verzweiflung erbötig gemacht, unseren
               neuen Bergdirektor\pwindex{?? [Bergdirektor] @\textsc{?? [Bergdirektor]}|pwv} sowie
               einen Ingenieur \pwindex{?? [Ingenieur] @\textsc{?? [Ingenieur]}|pwv} zu den
               Gruben nach Upony\oindex{Uppony@\textbf{Uppony}|pw} zu fahren. Der erstere
               musste den Punkt suchen, wo der Einstich beginnen sollte, der zweite die Trace der
               Eisenbahn, welche gebaut werden soll, bestimmen. Wir fuhren um \uuline{4 Uhr} morgens aus – haben Sie gerne, was?, – und ich legte unter einem
               fürchterlichen Anfall von \label{K_L03105-11v}\edtext{image
                  physic}{\lemma{\textnormal{\emph{image
                  physic}}}\Cendnote{\textnormal{XXXX}}}\label{K_L03105-11h} den Weg der sonst
               8 Stunden dauert in 5 ½ zurück. Dazu kam, dass der junge Ingenieur\pwindex{?? [Ingenieur] @\textsc{?? [Ingenieur]}|pwv} (typisch ungarischer Jude) sich
               bei mir angenehm machen wollte. Als wir durch den Uponyer Engpass\oindex{Upponyi-szoros@\textbf{Upponyi-szoros}|pw} fuhren, umringt von hohen Bergen, in denen mächtige
               Kohlenlager enthalten sind; be{\pb}gann der Mensch neben mir enthusiastisch zu werden, und mir von der »Mutter Natur«
               zu reden. Ich glaubte, ich müsse vom Wagen springen, um laut schreiend ins Kafé Kremser\oindex{Cafe Kremser@\textbf{Café Kremser}|pw} zu laufen, um mit Ihnen über die
               lächerliche Begeisterung des \uline{widerlichen} 1. Grades zu
               schimpfen. Das wird jedoch bald geschehen, und dann werde ich Ihnen das Milieu
               schildern, in das ich hier gerathen bin. Schrecklich ist mir hier das Umworbenwerden,
               das Herandrängen der Familien u. das plumpe Angeln der Mütter u. Töchter. Mein Bruder
                  Emil\pwindex{Salzmann, Michael Emil 1858-01-19 – 1908-06-26@\textsc{Salzmann, Michael Emil} (1858-01-19 – 1908-06-26), \emph{Versicherungsbeamter}|pw} – »ist scho hin i is scho hin!«\pend
           \pstart
           Mit Italien\oindex{Italien@\textbf{Italien}|pw} sieht's schlecht aus. Papa\pwindex{Salzmann, Philipp 1831-12-24 – 1905-04-02@\textsc{Salzmann, Philipp} (1831-12-24 – 1905-04-02), \emph{Unternehmer}|pw} beginnt den Betrieb und ich sehe, wie die
               Tausende nur so fliegen. Es wird schwer halten an ihn heranzutreten. Auf jeden Fall
               sehe ich Sie im Verlaufe dieser Woche, und freue mich schon sehr darauf. \pend
           \pstart
           Leben Sie recht wol, und berauschen Sie sich immerhin an der Lüge, die nach Wahrheit
               duftet, auch ich suche u. ersehne diesen Duft; – es ist ja unser Beider Schicksal,
               die wir nach der Wahrheit lechzen, dass wir uns am Duft der Lüge betäuben, und daher
               auch unser Hass gegen die Nüchternen. \pend
           \pstart
           Grüßen Sie mir alle Herren die uns lieb sind, und senden Sie auch von mir die besten
               Wünsche mit nach Tropp\oindex{Opava@\textbf{Opava}|pw}. \pend
           \pstart  Ihr herzlich ergebener \spacefill\mbox{Felix S.}\pend{}
         
         \endnumbering\mylabel{h}\end{ledgroupsized}\begin{anhang}\end{anhang}\newcommand{\dateiname}{L03105}\newcommand{\titel}{Felix Salten an Arthur Schnitzler, 12. 9. 1891}\newcommand{\editorInnen}{Martin Anton Müller und Laura Untner}%% latex-leseansicht-abspann.tex
%% Abspann für die Leseansicht.
%% Der Schalter \ifkorrekturansicht ist bereits durch den Vorspann gesetzt.

%% latex-abspann.tex
%% Gemeinsamer Abspann für Korrekturansicht und Leseansicht.
%% Setzt den Schalter \ifkorrekturansicht voraus (gesetzt in den
%% einbindenden Dateien latex-korrekturansicht-abspann.tex bzw.
%% latex-leseansicht-abspann.tex).
%% ---------------------------------------------------------------

\normalsize

% Das esempio-Environment wird nur in der Leseansicht benötigt
\ifkorrekturansicht\else
\newenvironment{esempio}[3]%
{
    \vspace{1.5ex}
    \rlap{\underline{#1}}
    \par
    \setlength{\parindent}{0cm}
    \nopagebreak
    \leftskip=#2cm
    \rightskip=#3cm
}
{
    \par
}
\fi

\doendnotes{C}
\bigskip
\vfill

\clearpage

\footnotesize

\ifkorrekturansicht
  \lohead{\textsc{register}}
\fi

% theindex-Environment neu definieren ohne reledmac
\makeatletter
\renewenvironment{theindex}{%
  \ifkorrekturansicht
    \section*{\indexname}%
  \else
    \subsubsection*{Index der erwähnten Entitäten}%
  \fi
  \setlength{\parindent}{0pt}%
  \setlength{\parskip}{0pt plus 0.3pt}%
  \let\item\@idxitem
}{%
  \ifkorrekturansicht\clearpage\fi
}
\makeatother

\IfFileExists{\jobname-pw.ind}{\input{\jobname-pw.ind}}{}

% Quellenangabe nur in der Leseansicht
\ifkorrekturansicht\else
% Fallback-Definitionen, falls die .tex-Datei \titel etc. nicht gesetzt hat
\providecommand{\titel}{}
\providecommand{\editorInnen}{}
\providecommand{\dateiname}{\jobname}

\vspace{3cm}

\vfill

\footnotesize
\textsc{Quelle}: \titel. Herausgegeben von {\editorInnen}. In: \emph{Arthur Schnitzler: Briefwechsel mit Autorinnen und Autoren}.
 Digitale Edition, https://schnitzler-briefe.acdh.oeaw.ac.at/{\dateiname}.html (Stand \today)
\fi

\end{document}


      