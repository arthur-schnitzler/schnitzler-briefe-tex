%% latex-korrekturansicht-vorspann.tex
%% Vorspann für die Korrekturansicht.
%% Lädt die gemeinsame Datei latex-vorspann.tex mit gesetztem Schalter.

\newif\ifkorrekturansicht
\korrekturansichttrue

\input{../tex-inputs/latex-vorspann}


\section[Stefan Zweig an Arthur Schnitzler, 12. 12. {[}1914{]}]{L03648 Stefan Zweig an Arthur Schnitzler, 12. 12. {[}1914{]}}
\nopagebreak\mylabel{L03648v}
\rehead{ }\normalsize\beginnumbering\briefempfaengerindex{Schnitzler, Arthur@\textsc{Schnitzler, Arthur}!zzzZweig, Stefan@\emph{von Stefan Zweig}!1914-12-121@{12. 12. {[}1914{]}}|(be}
\toendnotes[C]{\smallbreak\pagebreak[2]}\Standort{CUL, Schnitzler, B 118.}
\physDesc{Brief, 1 Blatt, 2 Seiten, 1050 Zeichen
\newline{}Handschrift: lila Tinte, lateinische Kurrent
\newline{}Schnitzler: mit rotem Buntstift eine Unterstreichung }
\buchAbdrucke{\weitereDrucke{Stefan Zweig: \emph{Briefwechsel mit Hermann Bahr, Sigmund Freud, Rainer Maria
                        Rilke und Arthur Schnitzler}. Frankfurt am Main: \emph{S. Fischer} 1987, S. 388–389.} }\toendnotes[C]{\smallbreak}
\pstart
           {\pb}\textcolor{gray}{\textbf{SZ}}\hfill \textcolor{gray}{\textbf{VIII. KOCHGASSE\oindex{Kochgasse 8@\textbf{Kochgasse 8}, \emph{Wohngebäude (K.WHS)}|pw}}}\pend
           
\pstart
           \raggedleft{}\textcolor{gray}{\textbf{WIEN\oindex{Wien@\textbf{Wien}, \emph{A.ADM2}|pw},{ }12, XII}}\pend
           \vspace{0.5em}
\pstart
           Verehrter Herr Doktor,{ }Romain Rolland\pwindex{Rolland, Romain 29.01.1866 – 30.12.1944@\textsc{Rolland, Romain} (29.01.1866 – 30.12.1944), \emph{Schriftsteller/Schriftstellerin}|pw} schreibt mir soeben »\label{K_L03648-1v}\edtext{Je recois le noble ècrit de Arthur
               Schnitzler. Je le traduirai avec plaisir et je prierai Seippel\pwindex{Seippel, Paul 1858-04-24 – 1926-03-13@\textsc{Seippel, Paul} (1858-04-24 – 1926-03-13), \emph{Herausgeber/Herausgeberin, Romanist/Romanistin}|pw} de le faire paraître dans le Journal de Genèvre\pwindex{Journal de Geneve@\emph{Journal de Genève}|pw}. (Envoyez moi un second exemplaire pour un
                  journal\pwindex{Neue Zuercher Zeitung@\emph{Neue Zürcher Zeitung}|pwv} de la Suisse Allemande\oindex{Schweiz@\textbf{Schweiz}, \emph{A.PCLI}|pw}.) Je crains seulement qu’on
               n’objecte que personne, ici ni en France\oindex{Frankreich@\textbf{Frankreich}, \emph{A.PCLI}|pw}, n’a
               entendu parler de ces mensonges; personne chez nons, n’a élevé, ni pensé a elever des
               accusations semblables contre A. S., ni contre aucun des principans écrivains
                  allemands.}{\lemma{\textnormal{\emph{Je … allemands.}}}\Cendnote{\textnormal{Romain Rolland\pwindex{Rolland, Romain 29.01.1866 – 30.12.1944@\textsc{Rolland, Romain} (29.01.1866 – 30.12.1944), \emph{Schriftsteller/Schriftstellerin}|pwk} an Stefan
                     Zweig\pwindex{Zweig, Stefan 28.11.1881 – 23.02.1942@\textsc{Zweig, Stefan} (28.11.1881 – 23.02.1942), \emph{Schriftsteller/Schriftstellerin}|pwk}, 9. 12. 1914: »Ich erhalte den hochherzigen Text
                     von Arthur Schnitzler. Ich werde ihn gern
                     übersetzen und Seippel\pwindex{Seippel, Paul 1858-04-24 – 1926-03-13@\textsc{Seippel, Paul} (1858-04-24 – 1926-03-13), \emph{Herausgeber/Herausgeberin, Romanist/Romanistin}|pw} bitten, dass er
                     ihn im ›Journal de Genèvre\pwindex{Journal de Geneve@\emph{Journal de Genève}|pw}‹
                     veröffentlicht. (Schicken Sie mir noch ein zweites Exemplar für eine Zeitung\pwindex{Neue Zuercher Zeitung@\emph{Neue Zürcher Zeitung}|pwv} in der deutschsprachigen Schweiz\oindex{Schweiz@\textbf{Schweiz}, \emph{A.PCLI}|pw}.) Nur glaube
                     ich, dass niemand von diesen Lügen etwas gehört hat, weder hier noch in Frankreich\oindex{Frankreich@\textbf{Frankreich}, \emph{A.PCLI}|pw}; es ist bei uns niemandem in den
                     Sinn gekommen, derartige Anschuldigungen gegen Arthur Schnitzler oder irgendeinen anderen großen deutschen
                     Schriftsteller zu erheben.«, zitiert nach: Romain Rolland\pwindex{Rolland, Romain 29.01.1866 – 30.12.1944@\textsc{Rolland, Romain} (29.01.1866 – 30.12.1944), \emph{Schriftsteller/Schriftstellerin}|pwk}, Stefan Zweig\pwindex{Zweig, Stefan 28.11.1881 – 23.02.1942@\textsc{Zweig, Stefan} (28.11.1881 – 23.02.1942), \emph{Schriftsteller/Schriftstellerin}|pwk}: \emph{Von Welt zu Welt. Briefe
                        einer Freundschaft 1914–1918}. Mit einem Begleitwort von Peter
                     Handke. Aus dem Französischen von Eva und Gerhard Schwewe (Briefe Rollands) und
                     Christel Gersch (Briefe Zweigs). Berlin: \emph{Aufbau
                        Verlag}{ }2014.}}}\label{K_L03648-1}«\pend
           
\pstart
           Ich freue mich für Sie, dass die Lügen also kurze Beine hatten und vorläufig nicht
               über Russland\oindex{Russland@\textbf{Russland}, \emph{A.PCLI}|pw} hinausgelaufen sind. Das Dementi
               kann aber doch nur {\pb}von Vorteil sein.
               Wenn Sie noch ein Exemplar\pwindex{Une protestation DArthur Schnitzler@\emph{Une protestation d’Arthur Schnitzler}|pwv}
               haben, so senden Sie es am besten direct an Romain
                  Rolland\pwindex{Rolland, Romain 29.01.1866 – 30.12.1944@\textsc{Rolland, Romain} (29.01.1866 – 30.12.1944), \emph{Schriftsteller/Schriftstellerin}|pw}{ }Genf, Hôtel Beau Sejour\oindex{Hôtel Beau-Sejour@\textbf{Hôtel Beau-Séjour}, \emph{Hotel (K.HTL)}|pw}.\pend
           
\pstart
           Das kleine \label{K_L03648-2v}\edtext{Gedicht\pwindex{?? [Gedicht, das Stefan Zweig fuer Olga Schnitzler uebersetzen soll]@\emph{?? [Gedicht, das Stefan Zweig für Olga Schnitzler übersetzen soll]}|pwv}}{\lemma{\textnormal{\emph{Gedicht}}}\Cendnote{\textnormal{nicht ermittelt. Die Details dürften beim letzten Treffen am
                     10. 12. 1914
                  mündlich besprochen worden sein.}}}\label{K_L03648-2} für das Lied Ihrer Frau Gemahlin\pwindex{Schnitzler, Olga 17.01.1882 – 13.01.1970@\textsc{Schnitzler, Olga} (17.01.1882 – 13.01.1970), \emph{Schauspieler/Schauspielerin, Sänger/Sängerin}|pwv} leistet der \uline{guten} Verdeutschung hartnäckigen Widerstand. Hier wie überall offenbart
               sich’s neuerlich, dass das Einfachste immer auch das Schwerste ist.\pend
           
\pstart
           Ich bleibe in treuer Ergebenheit und Verehrung Ihr{\\[\baselineskip]}\spacefill\mbox{Stefan Zweig}\pend
           \leftskip=0em{}\selectlanguage{ngerman}\endnumbering\briefempfaengerindex{Schnitzler, Arthur@\textsc{Schnitzler, Arthur}!zzzZweig, Stefan@\emph{von Stefan Zweig}!1914-12-121@{12. 12. {[}1914{]}}|)be}\mylabel{L03648h}
\begin{anhang}
\end{anhang}\normalsize

\doendnotes{C}
\bigskip
\vfill

\clearpage

\footnotesize

\lohead{\textsc{register}}

% Definiere theindex-Environment komplett neu ohne reledmac
\makeatletter
\renewenvironment{theindex}{%
  \section*{\indexname}%
  \setlength{\parindent}{0pt}%
  \setlength{\parskip}{0pt plus 0.3pt}%
  \let\item\@idxitem
}{%
  \clearpage
}
\makeatother

\IfFileExists{\jobname-pw.ind}{\input{\jobname-pw.ind}}{}

\end{document}

      