%% latex-leseansicht-vorspann.tex
%% Vorspann für die Leseansicht.
%% Lädt die gemeinsame Datei latex-vorspann.tex mit nicht gesetztem Schalter.

\newif\ifkorrekturansicht
\korrekturansichtfalse

\input{../tex-inputs/latex-vorspann}


\section[Stefan Zweig an Arthur Schnitzler, 12. 12. {[}1914?{]}]{L03648 Stefan Zweig an Arthur Schnitzler, 12. 12. [1914?]}
\nopagebreak\mylabel{L03648v}
\rehead{ }\normalsize\beginnumbering\briefempfaengerindex{Schnitzler, Arthur@\textsc{Schnitzler, Arthur}!zzzZweig, Stefan@\emph{von Stefan Zweig}!1914-12-121@{12. 12. [1914?]}|(be}
\toendnotes[C]{\smallbreak\pagebreak[2]}
\correspDesc{Versand  durch Stefan Zweig am 12. 12. [1914?] in Wien
\newline{}Erhalt  durch Arthur Schnitzler im Zeitraum [12. 12. 1914 – 16. 12. 1914?] in Wien}\toendnotes[C]{\smallbreak}
\Standort{CUL, Schnitzler, B 118.}
\physDesc{Brief, 1 Blatt, 2 Seiten, 1055 Zeichen
\newline{}Handschrift: blaue Tinte, lateinische Kurrent
\newline{}Schnitzler: 1) mit Bleistift Vermerk: »\textsc{Zweig}«  2) mit rotem Buntstift eine Unterstreichung}
\buchAbdrucke{\weitereDrucke{Stefan Zweig: \emph{Briefwechsel mit Hermann Bahr, Sigmund Freud, Rainer Maria
                        Rilke und Arthur Schnitzler}. Herausgegeben von Jeffrey B. Berlin, Hans-Ulrich Lindken und Donald A. Prater. Frankfurt am Main: \emph{S. Fischer} 1987, S. 388–389.} }\toendnotes[C]{\smallbreak}
\pstart
           {\pb}\textcolor{gray}{\textbf{SZ}}\hfill \textcolor{gray}{\textbf{VIII. KOCHGASSE\oindex{Wien@\textbf{Wien}!VIII., Josefstadt@\textbf{VIII., Josefstadt}!Kochgasse 8@\textbf{Kochgasse 8}, \emph{Wohngebäude}|pw}}}\pend
           
\pstart
           \raggedleft{}\textcolor{gray}{\textbf{WIEN\oindex{Wien@\textbf{Wien}, \emph{Verwaltungsgebiet}|pw},}}{ }12. XII\pend
           \vspace{0.5em}
\pstart
           Verehrter Herr Doktor,{ }Romain Rolland\pwindex{Rolland, Romain 29.\,1.\,1866 Clamecy – 30.\,12.\,1944 Vézelay@\textsc{Rolland, Romain} (29.\,1.\,1866 Clamecy – 30.\,12.\,1944 Vézelay), \emph{Schriftsteller}|pw} schreibt mir soeben »\label{K_L03648-1v}\edtext{\begin{otherlanguage}{french}Je recois le noble écrit\pwindex{Schnitzler, Arthur 15.\,5.\,1862 Wien – 21.\,10.\,1931 ebd.@\textsc{Schnitzler, Arthur} (15.\,5.\,1862 Wien – 21.\,10.\,1931 ebd.), \emph{Schriftsteller, Mediziner}!Une protestation d’Arthur Schnitzler@\strich\emph{Une protestation d’Arthur Schnitzler}|pwv} de Arthur
               Schnitzler. Je le traduirai avec plaisir et je prierai Seippel\pwindex{Seippel, Paul 24.\,4.\,1858 Gingins – 13.\,3.\,1926 Chêne-Bourg@\textsc{Seippel, Paul} (24.\,4.\,1858 Gingins – 13.\,3.\,1926 Chêne-Bourg), \emph{Herausgeber, Romanist}|pw} de le faire paraître dans le Journal de Genève\pwindex{Journal de Genève@\emph{Journal de Genève}|pw}. (Envoyez moi un second exemplaire pour un
                  journal\pwindex{Neue Zürcher Zeitung@\emph{Neue Zürcher Zeitung}|pwv} de la Suisse Allemande\oindex{Schweiz@\textbf{Schweiz}|pw}.) Je crains seulement qu’on
               n’objecte que personne, ici ni en France\oindex{Frankreich@\textbf{Frankreich}|pw}, n’a
               entendu parler de ces mensonges; personne chez nous, n’a élevé, ni pensé a elever des
               accusations semblables contre A. S., ni contre aucun des principaux écrivains
                  allemands.\end{otherlanguage}}{\lemma{\textnormal{\emph{Je … allemands.}}}\Cendnote{\textnormal{Romain Rolland\pwindex{Rolland, Romain 29.\,1.\,1866 Clamecy – 30.\,12.\,1944 Vézelay@\textsc{Rolland, Romain} (29.\,1.\,1866 Clamecy – 30.\,12.\,1944 Vézelay), \emph{Schriftsteller}|pwk} an Stefan
                     Zweig\pwindex{Zweig, Stefan 28.\,11.\,1881 Wien – 23.\,2.\,1942 Petrópolis@\textsc{Zweig, Stefan} (28.\,11.\,1881 Wien – 23.\,2.\,1942 Petrópolis), \emph{Schriftsteller}|pwk}, 9. 12. 1914: »Ich erhalte den hochherzigen Text\pwindex{Schnitzler, Arthur 15.\,5.\,1862 Wien – 21.\,10.\,1931 ebd.@\textsc{Schnitzler, Arthur} (15.\,5.\,1862 Wien – 21.\,10.\,1931 ebd.), \emph{Schriftsteller, Mediziner}!Une protestation d’Arthur Schnitzler@\strich\emph{Une protestation d’Arthur Schnitzler}|pwv}
                     von Arthur Schnitzler. Ich werde ihn gern
                     übersetzen und Seippel\pwindex{Seippel, Paul 24.\,4.\,1858 Gingins – 13.\,3.\,1926 Chêne-Bourg@\textsc{Seippel, Paul} (24.\,4.\,1858 Gingins – 13.\,3.\,1926 Chêne-Bourg), \emph{Herausgeber, Romanist}|pw} bitten, dass er
                     ihn im ›Journal de Genève\pwindex{Journal de Genève@\emph{Journal de Genève}|pw}‹
                     veröffentlicht. (Schicken Sie mir noch ein zweites Exemplar für eine Zeitung\pwindex{Neue Zürcher Zeitung@\emph{Neue Zürcher Zeitung}|pwv} in der deutschsprachigen Schweiz\oindex{Schweiz@\textbf{Schweiz}|pw}.) Nur glaube
                     ich, dass niemand von diesen Lügen etwas gehört hat, weder hier noch in Frankreich\oindex{Frankreich@\textbf{Frankreich}|pw}; es ist bei uns niemandem in den
                     Sinn gekommen, derartige Anschuldigungen gegen Arthur Schnitzler oder irgendeinen anderen großen deutschen
                     Schriftsteller zu erheben.«, zitiert nach: Romain Rolland\pwindex{Rolland, Romain 29.\,1.\,1866 Clamecy – 30.\,12.\,1944 Vézelay@\textsc{Rolland, Romain} (29.\,1.\,1866 Clamecy – 30.\,12.\,1944 Vézelay), \emph{Schriftsteller}|pwk}, Stefan Zweig\pwindex{Zweig, Stefan 28.\,11.\,1881 Wien – 23.\,2.\,1942 Petrópolis@\textsc{Zweig, Stefan} (28.\,11.\,1881 Wien – 23.\,2.\,1942 Petrópolis), \emph{Schriftsteller}|pwk}: \emph{Von Welt zu Welt. Briefe
                        einer Freundschaft 1914–1918}. Mit einem Begleitwort von Peter
                     Handke. Aus dem Französischen von Eva und Gerhard Schwewe (Briefe Rollands) und
                     Christel Gersch (Briefe Zweigs). Berlin: \emph{Aufbau
                        Verlag}{ }2014.}}}\label{K_L03648-1}«\pend
           
\pstart
           Ich freue mich für Sie, dass die Lügen also kurze Beine hatten und vorläufig nicht
               über Russland\oindex{Russland@\textbf{Russland}|pw} hinausgelaufen sind. Das Dementi
               kann aber doch nur {\pb}von Vorteil sein.
               Wenn Sie noch ein Exemplar\pwindex{Schnitzler, Arthur 15.\,5.\,1862 Wien – 21.\,10.\,1931 ebd.@\textsc{Schnitzler, Arthur} (15.\,5.\,1862 Wien – 21.\,10.\,1931 ebd.), \emph{Schriftsteller, Mediziner}!Une protestation d’Arthur Schnitzler@\strich\emph{Une protestation d’Arthur Schnitzler}|pwv}
               haben, so senden Sie es am besten direct an Romain
                  Rolland\pwindex{Rolland, Romain 29.\,1.\,1866 Clamecy – 30.\,12.\,1944 Vézelay@\textsc{Rolland, Romain} (29.\,1.\,1866 Clamecy – 30.\,12.\,1944 Vézelay), \emph{Schriftsteller}|pw}{ }Genf, Hôtel Beau Sejour\oindex{Hôtel Beau-Séjour@\textbf{Hôtel Beau-Séjour}, \emph{Hotel}|pw}.\pend
           
\pstart
           Das kleine \label{K_L03648-2v}\edtext{Gedicht\pwindex{?? [Gedicht, das Stefan Zweig für Olga Schnitzler übersetzen soll]@\emph{?? [Gedicht, das Stefan Zweig für Olga Schnitzler übersetzen soll]}|pwv}}{\lemma{\textnormal{\emph{Gedicht}}}\Cendnote{\textnormal{nicht ermittelt. Es dürfte sich um ein Lied 
                  handeln, das Olga Schnitzler\pwindex{Schnitzler, Olga 17.\,1.\,1882 Wien – 13.\,1.\,1970 Lugano@\textsc{Schnitzler, Olga} (17.\,1.\,1882 Wien – 13.\,1.\,1970 Lugano), \emph{Schauspielerin, Sängerin}|pwk} anlässlich des Liliencron\pwindex{Liliencron, Detlev von 3.\,6.\,1844 Kiel – 22.\,7.\,1909 Rahlstedt@\textsc{Liliencron, Detlev von} (3.\,6.\,1844 Kiel – 22.\,7.\,1909 Rahlstedt), \emph{Schriftsteller, Dichter, Dramatiker}|pwk}-Abends\eventindex{Volkshochschule Ottakring@\textbf{Volkshochschule Ottakring}!Dichterabend Detlev von Liliencron, 3.1.1915@Dichterabend Detlev von Liliencron, 3.1.1915|pwk}
                  am 3. 1. 1915 vortragen sollte. Die Details könnten beim letzten Treffen am
                     10. 12. 1914
                  mündlich besprochen worden sein.}}}\label{K_L03648-2} für das Lied Ihrer Frau Gemahlin\pwindex{Schnitzler, Olga 17.\,1.\,1882 Wien – 13.\,1.\,1970 Lugano@\textsc{Schnitzler, Olga} (17.\,1.\,1882 Wien – 13.\,1.\,1970 Lugano), \emph{Schauspielerin, Sängerin}|pwv} leistet der \uline{guten} Verdeutschung hartnäckigen Widerstand. Hier wie überall offenbart
               sich’s neuerlich, dass das Einfachste immer auch das Schwerste ist.\pend
           
\pstart
           Ich bleibe in treuer Ergebenheit und Verehrung Ihr{\\[\baselineskip]}\spacefill\mbox{Stefan Zweig}\pend
           \leftskip=0em{}\selectlanguage{ngerman}\endnumbering\briefempfaengerindex{Schnitzler, Arthur@\textsc{Schnitzler, Arthur}!zzzZweig, Stefan@\emph{von Stefan Zweig}!1914-12-121@{12. 12. [1914?]}|)be}\mylabel{L03648h}  \newcommand{\dateiname}{L03648}\newcommand{\titel}{Stefan Zweig an Arthur Schnitzler, 12. 12. [1914?]}\newcommand{\editorInnen}{Selma Jahnke und Martin Anton Müller}%% latex-leseansicht-abspann.tex
%% Abspann für die Leseansicht.
%% Der Schalter \ifkorrekturansicht ist bereits durch den Vorspann gesetzt.

%% latex-abspann.tex
%% Gemeinsamer Abspann für Korrekturansicht und Leseansicht.
%% Setzt den Schalter \ifkorrekturansicht voraus (gesetzt in den
%% einbindenden Dateien latex-korrekturansicht-abspann.tex bzw.
%% latex-leseansicht-abspann.tex).
%% ---------------------------------------------------------------

\normalsize

% Das esempio-Environment wird nur in der Leseansicht benötigt
\ifkorrekturansicht\else
\newenvironment{esempio}[3]%
{
    \vspace{1.5ex}
    \rlap{\underline{#1}}
    \par
    \setlength{\parindent}{0cm}
    \nopagebreak
    \leftskip=#2cm
    \rightskip=#3cm
}
{
    \par
}
\fi

\doendnotes{C}
\bigskip
\vfill

\clearpage

\footnotesize

\ifkorrekturansicht
  \lohead{\textsc{register}}
\fi

% theindex-Environment neu definieren ohne reledmac
\makeatletter
\renewenvironment{theindex}{%
  \ifkorrekturansicht
    \section*{\indexname}%
  \else
    \subsubsection*{Index der erwähnten Entitäten}%
  \fi
  \setlength{\parindent}{0pt}%
  \setlength{\parskip}{0pt plus 0.3pt}%
  \let\item\@idxitem
}{%
  \ifkorrekturansicht\clearpage\fi
}
\makeatother

\IfFileExists{\jobname-pw.ind}{\input{\jobname-pw.ind}}{}

% Quellenangabe nur in der Leseansicht
\ifkorrekturansicht\else
% Fallback-Definitionen, falls die .tex-Datei \titel etc. nicht gesetzt hat
\providecommand{\titel}{}
\providecommand{\editorInnen}{}
\providecommand{\dateiname}{\jobname}

\vspace{3cm}

\vfill

\footnotesize
\textsc{Quelle}: \titel. Herausgegeben von {\editorInnen}. In: \emph{Arthur Schnitzler: Briefwechsel mit Autorinnen und Autoren}.
 Digitale Edition, https://schnitzler-briefe.acdh.oeaw.ac.at/{\dateiname}.html (Stand \today)
\fi

\end{document}


