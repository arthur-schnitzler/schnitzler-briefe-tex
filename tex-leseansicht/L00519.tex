%% latex-korrekturansicht-vorspann.tex
%% Vorspann für die Korrekturansicht.
%% Lädt die gemeinsame Datei latex-vorspann.tex mit gesetztem Schalter.

\newif\ifkorrekturansicht
\korrekturansichttrue

\input{../tex-inputs/latex-vorspann}


\section[Lou Andreas-Salomé an Arthur Schnitzler, {[}1. 12. 1895{]}]{L00519 Lou Andreas-Salomé an Arthur Schnitzler, {[}1. 12. 1895{]}}
\nopagebreak\mylabel{L00519v}
\rehead{ }\normalsize\beginnumbering\briefempfaengerindex{Schnitzler, Arthur@\textsc{Schnitzler, Arthur}!zzzAndreas-Salome, Lou@\emph{von Lou Andreas-Salomé}!1895-12-011@{{[}1. 12. 1895{]}}|(be}
\toendnotes[C]{\smallbreak\pagebreak[2]}\Standort{CUL, Schnitzler, B 3.}
\physDesc{Brief, 1 Blatt, 1 Seite, 383 Zeichen
\newline{}Handschrift: schwarze Tinte, deutsche Kurrent
\newline{}Schnitzler: mit Bleistift datiert: »1/12 95« 
\newline{}Ordnung: mit rotem Buntstift von unbekannter Hand nummeriert:
                                    »11.« }
\pstart
           \raggedleft{}{\pb}Sonntag.\pend
           
\pstart{}Lieber Herr \textsc{D\textsuperscript{r}},\pend\vspace{0.5em}
\pstart
           am liebſten wäre es mir, wenn \introOben{}am Dienstag\introOben{} Jemand von Ihnen
                  \uline{nach} Ihrem Theaterbeſuch mich vom \textsc{Hôtel Royal}\oindex{Hotel Royal@\textbf{Hotel Royal}, \emph{Hotel (K.HTL)}|pw} zum Nachtmahl im \textsc{Griensteidl}\oindex{Cafe Griensteidl@\textbf{Café Griensteidl}, \emph{Kaffeehaus (K.KAF)}|pw} abholen könnte. Aber ich habe keine Ahnung ob das ein großer Umweg für Sie
               wäre, in dem Fall wage ich mich auch allein in’s \textsc{Griensteidl}\oindex{Cafe Griensteidl@\textbf{Café Griensteidl}, \emph{Kaffeehaus (K.KAF)}|pw}, \strikeout{f} wenn Sie mich wiſſen laſſen wollen um welche
               Zeit ich es thun ſoll.\pend
           
\pstart
           Mit herzlichen Grüßen Ihre{\\[\baselineskip]}\spacefill\mbox{LouAS.}\pend
           \leftskip=0em{}\selectlanguage{ngerman}\endnumbering\briefempfaengerindex{Schnitzler, Arthur@\textsc{Schnitzler, Arthur}!zzzAndreas-Salome, Lou@\emph{von Lou Andreas-Salomé}!1895-12-011@{{[}1. 12. 1895{]}}|)be}\mylabel{L00519h}  \normalsize

\doendnotes{C}
\bigskip
\vfill

\clearpage

\footnotesize

\lohead{\textsc{register}}

% Definiere theindex-Environment komplett neu ohne reledmac
\makeatletter
\renewenvironment{theindex}{%
  \section*{\indexname}%
  \setlength{\parindent}{0pt}%
  \setlength{\parskip}{0pt plus 0.3pt}%
  \let\item\@idxitem
}{%
  \clearpage
}
\makeatother

\IfFileExists{\jobname-pw.ind}{\input{\jobname-pw.ind}}{}

\end{document}

      