%% latex-leseansicht-vorspann.tex
%% Vorspann für die Leseansicht.
%% Lädt die gemeinsame Datei latex-vorspann.tex mit nicht gesetztem Schalter.

\newif\ifkorrekturansicht
\korrekturansichtfalse

\input{../tex-inputs/latex-vorspann}


         \newcommand{\erwaehnteOrte}{Orte: Wien, Österreich}
         \newcommand{\erwaehnteWerke}{Werke: Yppl. Idylle in fünf Akten}
               \section[Robert Adam an Arthur Schnitzler, 25. 10. 1918]{ Robert Adam an Arthur Schnitzler, 25. 10. 1918}\nopagebreak\mylabel{v}\rehead{ }\begin{ledgroupsized}[t]{13cm}\normalsize\beginnumbering \toendnotes[C]{\smallbreak\pagebreak[2]} \Standort{CUL, Schnitzler, B 1.}
\physDesc{Brief, 1 Blatt, 4 Seiten
\newline{}Handschrift: schwarze Tinte, deutsche Kurrent
\newline{}Schnitzler: 1) mit Bleistift beschriftet: »\textsc{Adam}«  2) mit rotem Buntstift eine Unterstreichung\newline{}Ordnung: von unbekannter Hand nummeriert: »8« }\Standort{Wien, Österreichische Nationalbibliothek, Cod.ser. 52.263, 223.}
\physDesc{Brief, maschinelle Abschrift
\newline{}Schreibmaschine}\toendnotes[C]{\smallbreak}\pstart
           \raggedleft{}{\pb}Wien\oindex{Wien@\textbf{Wien}|pw}, 25. Oktober 1918\pend
           \pstart\center{}Hochverehrter Herr Doktor!\pend\pstart
           Ich kann Ihnen vereinbarungsgemäß mitteilen, daß ich die »Yppl\pwindex{Adam, Robert 20.04.1877 – 16.10.1961@\textsc{Adam, Robert} (20.04.1877 – 16.10.1961), \emph{Schriftsteller, Richter}!Yppl. Idylle in fuenf AktenNone@\strich\emph{Yppl. Idylle in fünf Akten} {[}None{]}|pw}«-Komödie nunmehr abgeſchloſſen und heute zum Abſchreiben
               gegeben habe; im Laufe der nächſten Woche erhalte ich von der Schreibmaſchindame, die
               mit den vielen Dialektworten nicht einverſtanden und nahe daran war, ihretwegen das
               ohnehin horrende Honorar noch zu ſteigern, die Abſchriften ausgefolgt. An der Komödie
               habe ich bei der Überarbeitung nicht ſehr viel geändert; immerhin glaube ich durch
               Einfügungen den Charakter der \textsc{Steffi}\pwindex{Adam, Robert 20.04.1877 – 16.10.1961@\textsc{Adam, Robert} (20.04.1877 – 16.10.1961), \emph{Schriftsteller, Richter}!Yppl. Idylle in fuenf AktenNone@\strich\emph{Yppl. Idylle in fünf Akten} {[}None{]}|pwv} vertieft und auch vom D\textsuperscript{r}{ }\textsc{Greil}\pwindex{Adam, Robert 20.04.1877 – 16.10.1961@\textsc{Adam, Robert} (20.04.1877 – 16.10.1961), \emph{Schriftsteller, Richter}!Yppl. Idylle in fuenf AktenNone@\strich\emph{Yppl. Idylle in fünf Akten} {[}None{]}|pwv} – der allerdings nie und nimmer ein intereſſanter Menſch werden wird – ein
               klareres {\pb}Bild gegeben zu haben; dieſe
               Zuſätze betreffen faſt ausſchließlich den 3. Akt (Straße). Das geiſtreiche \label{K_L02308_1v}\edtext{Dilettantenſtück\pwindex{Adam, Robert 20.04.1877 – 16.10.1961@\textsc{Adam, Robert} (20.04.1877 – 16.10.1961), \emph{Schriftsteller, Richter}!Yppl. Idylle in fuenf AktenNone@\strich\emph{Yppl. Idylle in fünf Akten} {[}None{]}|pwv}}{\lemma{\textnormal{\emph{Dilettantenſtück}}}\Cendnote{\textnormal{kein eigenes Stück, sondern der vierte Akt
                  von \emph{Yppl}\pwindex{Adam, Robert 20.04.1877 – 16.10.1961@\textsc{Adam, Robert} (20.04.1877 – 16.10.1961), \emph{Schriftsteller, Richter}!Yppl. Idylle in fuenf AktenNone@\strich\emph{Yppl. Idylle in fünf Akten} {[}None{]}|pwk}}}}\label{K_L02308_1h} habe ich, ſoweit es anging,
               gekürzt. Ueber den 4. Akt müßten allerdings die Schauſpieler, deren Aufgabe,
               Dilettantenſchauſpieler zu imitieren, ſchließlich keine undankbare iſt,
               hinweghelfen.\pend
           \pstart
           Ich möchte nun anfragen, hochverehrter Herr Doktor, wie ich es mit der Verwertung
               meines Produkts am Beſten anfinge. Daß mir ſehr viel daran liegt, diesmal anzukommen,
               brauche ich nicht erſt zu ſchreiben; dazu kommt nun aber doch noch der Umſtand, daß
               es mir nun auch aus materiellen Gründen äußerſt erwünſcht wäre, mein Stück irgendwo
               akzeptiert zu wiſſen. Da es ganz unpolitiſch und nicht einmal gar ſo unmoraliſch iſt,
               wird ihm die Neugeſtaltung Öſterreichs\oindex{Oesterreich@\textbf{Österreich}|pw}, hoff’ ich,
               nicht hinderlich in den Weg treten.\pend
           \pstart
           Ich habe auch daran gedacht, ob es nicht vielleicht anginge, das {\pb}Stück vor allem einem Schauſpieler zu
               geben, der eine der dankbarſten Rollen zu ſpielen hätte, etwa den Präſidenten\pwindex{Adam, Robert 20.04.1877 – 16.10.1961@\textsc{Adam, Robert} (20.04.1877 – 16.10.1961), \emph{Schriftsteller, Richter}!Yppl. Idylle in fuenf AktenNone@\strich\emph{Yppl. Idylle in fünf Akten} {[}None{]}|pwv}? Sie werden mir jedenfalls den
               beſten Rat geben.\pend
           \pstart
           Verzeihen Sie, daß ich Sie mit ſo nichtigen Angelegenheiten, wie es das Geſchick
               meines Stücks\pwindex{Adam, Robert 20.04.1877 – 16.10.1961@\textsc{Adam, Robert} (20.04.1877 – 16.10.1961), \emph{Schriftsteller, Richter}!Yppl. Idylle in fuenf AktenNone@\strich\emph{Yppl. Idylle in fünf Akten} {[}None{]}|pwv} iſt, das fürwahr
               keine »große« Komödie darſtellt, zu einer Zeit beläſtige, die von Tag zu Tag größer
               und größer und, ſo fürchte ich, furchtbarer wird. Es kommt mir vor, als ob man ſich
               jetzt mütterlich und mühſam mit der Anfertigung von Kinderkleidchen abgebe und
               möglicherweiſe nach deren Fertigſtellung zu Tage kommen werde, daß die Kinder
               inzwiſchen \strikeout{ſo gewachſen seien,} den Kleidchen
               entwachſen ſeien. All dieſe neuen Staaten, die ſich konſtituieren, ſind doch
               eigentlich nur Form, und der Streit um Abgrenzungen und dergleichen ein Streit um
               Formen; welcher Inhalt dieſe Form füllen wird, davon iſt überhaupt noch nicht die
               Rede. Aber {\pb}die ſoziale Frage hat immer
               eine geſunde Lunge gehabt und wird ſchon demnächſt all die nationalen Schlagworte
               überbrüllen. –\pend
           \pstart
           Ich hoffe, daß Sie von der † † † Grippe verſchont geblieben ſind und bleiben; mir und
               den Meinen iſt dies bisher gelungen.\pend
           \pstart
           Mit beſtem Gruß Ihr{\\[\baselineskip]}ergebener{\\[\baselineskip]}\spacefill\mbox{D\textsuperscript{r}RAdam}\pend
           \leftskip=0em{}
         
         \endnumbering\mylabel{h}\end{ledgroupsized}  \newcommand{\dateiname}{L02308}\newcommand{\titel}{Robert Adam an Arthur Schnitzler, 25. 10. 1918}\newcommand{\editorInnen}{Martin Anton Müller und Gerd-Hermann Susen}%% latex-leseansicht-abspann.tex
%% Abspann für die Leseansicht.
%% Der Schalter \ifkorrekturansicht ist bereits durch den Vorspann gesetzt.

%% latex-abspann.tex
%% Gemeinsamer Abspann für Korrekturansicht und Leseansicht.
%% Setzt den Schalter \ifkorrekturansicht voraus (gesetzt in den
%% einbindenden Dateien latex-korrekturansicht-abspann.tex bzw.
%% latex-leseansicht-abspann.tex).
%% ---------------------------------------------------------------

\normalsize

% Das esempio-Environment wird nur in der Leseansicht benötigt
\ifkorrekturansicht\else
\newenvironment{esempio}[3]%
{
    \vspace{1.5ex}
    \rlap{\underline{#1}}
    \par
    \setlength{\parindent}{0cm}
    \nopagebreak
    \leftskip=#2cm
    \rightskip=#3cm
}
{
    \par
}
\fi

\doendnotes{C}
\bigskip
\vfill

\clearpage

\footnotesize

\ifkorrekturansicht
  \lohead{\textsc{register}}
\fi

% theindex-Environment neu definieren ohne reledmac
\makeatletter
\renewenvironment{theindex}{%
  \ifkorrekturansicht
    \section*{\indexname}%
  \else
    \subsubsection*{Index der erwähnten Entitäten}%
  \fi
  \setlength{\parindent}{0pt}%
  \setlength{\parskip}{0pt plus 0.3pt}%
  \let\item\@idxitem
}{%
  \ifkorrekturansicht\clearpage\fi
}
\makeatother

\IfFileExists{\jobname-pw.ind}{\input{\jobname-pw.ind}}{}

% Quellenangabe nur in der Leseansicht
\ifkorrekturansicht\else
% Fallback-Definitionen, falls die .tex-Datei \titel etc. nicht gesetzt hat
\providecommand{\titel}{}
\providecommand{\editorInnen}{}
\providecommand{\dateiname}{\jobname}

\vspace{3cm}

\vfill

\footnotesize
\textsc{Quelle}: \titel. Herausgegeben von {\editorInnen}. In: \emph{Arthur Schnitzler: Briefwechsel mit Autorinnen und Autoren}.
 Digitale Edition, https://schnitzler-briefe.acdh.oeaw.ac.at/{\dateiname}.html (Stand \today)
\fi

\end{document}


      