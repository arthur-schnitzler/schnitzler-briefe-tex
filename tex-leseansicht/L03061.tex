%% latex-korrekturansicht-vorspann.tex
%% Vorspann für die Korrekturansicht.
%% Lädt die gemeinsame Datei latex-vorspann.tex mit gesetztem Schalter.

\newif\ifkorrekturansicht
\korrekturansichttrue

\input{../tex-inputs/latex-vorspann}


\section[Paul Goldmann, Bertha und Rudolf Christians an Arthur Schnitzler, 24. 3. 1901]{L03061 Paul Goldmann, Bertha und Rudolf Christians an Arthur
               Schnitzler, 24. 3. 1901}
\nopagebreak\mylabel{L03061v}
\rehead{ }\normalsize\beginnumbering\briefempfaengerindex{Schnitzler, Arthur@\textsc{Schnitzler, Arthur}!zzzKlein, Bertha@\emph{von Bertha Klein}!1901-03-241@{24. 3. 1901}|(be}\briefempfaengerindex{Schnitzler, Arthur@\textsc{Schnitzler, Arthur}!zzzChristians, Rudolf@\emph{von Rudolf Christians}!1901-03-241@{24. 3. 1901}|(be}\briefempfaengerindex{Schnitzler, Arthur@\textsc{Schnitzler, Arthur}!zzzGoldmann, Paul@\emph{von Paul Goldmann}!1901-03-241@{24. 3. 1901}|(be}
\toendnotes[C]{\smallbreak\pagebreak[2]}\Standort{DLA, A:Schnitzler, HS.NZ85.1.3171.}
\physDesc{Bildpostkarte, 479 Zeichen
\newline{}Handschrift Paul Goldmann: 1) Bleistift, deutsche Kurrent\hspace{1em}2) Bleistift, lateinische Kurrent (\noindent{}Adresse)\hspace{1em}
\newline{}Handschrift Rudolf Christians: Bleistift, deutsche Kurrent
\newline{}Handschrift Bertha Klein: Bleistift, deutsche Kurrent
\newline{}Versand: 1) Stempel: »\nobreak{}\oindex{Berlin@\textbf{Berlin}, \emph{P.PPLC}|pwk}Berlin W, 24. 3. 01, \textcolor{gray}{9–3 V.} 8 h\nobreak{}«.   2) Stempel: »\nobreak{}\oindex{IX., Alsergrund@\textbf{IX., Alsergrund}, \emph{A.ADM3}|pwk}Wien 9/3 72, 25. 3. 01, 8. V, Bestellt\nobreak{}«. }\toendnotes[C]{\smallbreak}\pstart{}{\pb}Herrn\pend{}\pstart{}Dr. Arthur Schnitzler\pend{}\pstart{}Wien\oindex{Wien@\textbf{Wien}, \emph{A.ADM2}|pw}\pend{}\pstart{}IX. Frankgaſse 1\oindex{Frankgasse 1@\textbf{Frankgasse 1}, \emph{Wohngebäude (K.WHS)}|pw}.\pend{}{\bigskip}
\pstart
           \noindent{}\centering{}{\pb}\textcolor{gray}{\textbf{\textbf{Restaurant ersten Ranges Lanzsch
                        {\kaufmannsund} Co.\oindex{Lanzsch {\kaufmannsund} Co.@\textbf{Lanzsch {\kaufmannsund} Co.}, \emph{Gastgewerbegebäude (K.GGW)}|pw}}}}\pend
           
\pstart
           \centering{}\textcolor{gray}{\textbf{\textbf{BERLIN\oindex{Berlin@\textbf{Berlin}, \emph{P.PPLC}|pw},}Charlotten-Strasse 56\oindex{Charlottenstrasse@\textbf{Charlottenstraße}, \emph{Straße (K.STR)}|pw}}}\pend
           
\pstart
           \centering{}\textcolor{gray}{\textbf{\begin{otherlanguage}{french}vis à vis\end{otherlanguage}{ }Schauspielhaus\oindex{Schauspielhaus Berlin@\textbf{Schauspielhaus Berlin}, \emph{Theater (K.THE)}|pw}}}\pend
           \vspace{1em}
\pstart
           \noindent{}{\pb}Lieber Freund, Gerade erzählt mir Herr \textsc{Christians\pwindex{Christians, Rudolf 15.01.1869 – 07.02.1921@\textsc{Christians, Rudolf} (15.01.1869 – 07.02.1921), \emph{Schauspieler/Schauspielerin}|pw}}, daß er der \label{K_L03061-1v}\edtext{erſte \textsc{Anatol\pwindex{Anatol@\emph{Anatol}|pwv}}}{\lemma{\textnormal{\emph{erſte Anatol}}}\Cendnote{\textnormal{Am 16. 1. 1898 hatte Rudolf Christians\pwindex{Christians, Rudolf 15.01.1869 – 07.02.1921@\textsc{Christians, Rudolf} (15.01.1869 – 07.02.1921), \emph{Schauspieler/Schauspielerin}|pwk} bei der Uraufführung von \emph{Weihnachts-Einkäufe}\pwindex{Weihnachts-Einkaeufe@\emph{Weihnachts-Einkäufe}|pwk} und der Premiere von \emph{Abschiedssouper}\pwindex{Abschiedssouper@\emph{Abschiedssouper}|pwk} am \emph{Deutschen Volkstheater}\orgindex{Volkstheater@Volkstheater|pwk} die Figur des Anatol\pwindex{Anatol@\emph{Anatol}|pwkv} gespielt. Da einzelne Anatol\pwindex{Anatol@\emph{Anatol}|pwkv}-Stücke bereits früher aufgeführt
                  worden waren, stimmt die Behauptung, er wäre der erste gewesen, nicht.}}}\label{K_L03061-1} war.
               Wir benutzen die Gelegenheit, Dir einen Gruß zu ſenden. Herzlichſt Dein \spacefill\mbox{Paul
                     Goldmann\textcolor{gray}{.}}\pend
           \selectlanguage{ngerman}\vspace{1em}
\pstart\center{}{[}hs. :{]} Mein ſehr verehrter, lieber Herr
                  Schnitzler!\pend\vspace{0.5em}
\pstart
           Ich freue mich richtig, Ihnen, verehrteſter Herr D\textsuperscript{r}, in
               Erinnerung an unſere »\textsc{Weihnachtseinkäufe\pwindex{Weihnachts-Einkaeufe@\emph{Weihnachts-Einkäufe}|pw}}« die herzlichſten Grüße zu ſenden! Was macht »\textsc{Schleier der Beatrice\pwindex{Schleier der Beatrice. Schauspiel in fuenf Akten@\emph{Der Schleier der Beatrice. Schauspiel in fünf Akten}|pw}}«? Warum nicht ich?\pend
           \pstart Ihr \spacefill\mbox{Christians}\pend{}\selectlanguage{ngerman}\vspace{1em}
\pstart
           \noindent{}{[}hs. :{]} Höflichen Gruß \spacefill\mbox{Bertha Christians.}\pend
           \selectlanguage{ngerman}\endnumbering\briefempfaengerindex{Schnitzler, Arthur@\textsc{Schnitzler, Arthur}!zzzKlein, Bertha@\emph{von Bertha Klein}!1901-03-241@{24. 3. 1901}|)be}\briefempfaengerindex{Schnitzler, Arthur@\textsc{Schnitzler, Arthur}!zzzChristians, Rudolf@\emph{von Rudolf Christians}!1901-03-241@{24. 3. 1901}|)be}\briefempfaengerindex{Schnitzler, Arthur@\textsc{Schnitzler, Arthur}!zzzGoldmann, Paul@\emph{von Paul Goldmann}!1901-03-241@{24. 3. 1901}|)be}\mylabel{L03061h}  \normalsize

\doendnotes{C}
\bigskip
\vfill

\clearpage

\footnotesize

\lohead{\textsc{register}}

% Definiere theindex-Environment komplett neu ohne reledmac
\makeatletter
\renewenvironment{theindex}{%
  \section*{\indexname}%
  \setlength{\parindent}{0pt}%
  \setlength{\parskip}{0pt plus 0.3pt}%
  \let\item\@idxitem
}{%
  \clearpage
}
\makeatother

\IfFileExists{\jobname-pw.ind}{\input{\jobname-pw.ind}}{}

\end{document}

      