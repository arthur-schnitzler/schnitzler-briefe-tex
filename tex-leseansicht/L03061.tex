%% latex-leseansicht-vorspann.tex
%% Vorspann für die Leseansicht.
%% Lädt die gemeinsame Datei latex-vorspann.tex mit nicht gesetztem Schalter.

\newif\ifkorrekturansicht
\korrekturansichtfalse

\input{../tex-inputs/latex-vorspann}


\section[Paul Goldmann, Bertha und Rudolf Christians an Arthur Schnitzler, 24. 3. 1901]{L03061 Paul Goldmann, Bertha und Rudolf Christians an Arthur
               Schnitzler,  24. 3. 1901}
\nopagebreak\mylabel{L03061v}
\rehead{ }\normalsize\beginnumbering\briefempfaengerindex{Schnitzler, Arthur@\textsc{Schnitzler, Arthur}!zzzKlein, Bertha@\emph{von Bertha Klein}!1901-03-241@{24. 3. 1901}|(be}\briefempfaengerindex{Schnitzler, Arthur@\textsc{Schnitzler, Arthur}!zzzChristians, Rudolf@\emph{von Rudolf Christians}!1901-03-241@{24. 3. 1901}|(be}\briefempfaengerindex{Schnitzler, Arthur@\textsc{Schnitzler, Arthur}!zzzGoldmann, Paul@\emph{von Paul Goldmann}!1901-03-241@{24. 3. 1901}|(be}
\toendnotes[C]{\smallbreak\pagebreak[2]}
\correspDesc{Versand  durch Paul Goldmann, Rudolf Christians, Bertha Christians am 24. 3. 1901 in Berlin
\newline{}Erhalt  durch Arthur Schnitzler am 25. 3. 1901 in Wien}\toendnotes[C]{\smallbreak}
\Standort{DLA, A:Schnitzler, HS.NZ85.1.3171.}
\physDesc{Bildpostkarte, 479 Zeichen
\newline{}Handschrift Paul Goldmann: Bleistift, deutsche Kurrent
\newline{}Handschrift Rudolf Christians: Bleistift, deutsche Kurrent
\newline{}Handschrift Bertha Klein: Bleistift, deutsche Kurrent
\newline{}Versand: 1) Stempel: »\nobreak{}\oindex{Berlin@\textbf{Berlin}, \emph{Hauptstadt}|pwk}Berlin W, 24. 3. 01, \textcolor{gray}{9–3 V.} 8 h\nobreak{}«.   2) Stempel: »\nobreak{}\oindex{IX., Alsergrund@\textbf{IX., Alsergrund}, \emph{Verwaltungsgebiet}|pwk}Wien 9/3 72, 25. 3. 01, 8. V, Bestellt\nobreak{}«. }\toendnotes[C]{\smallbreak}\pstart{}\textsc{{\pb}Herrn}\pend{}\pstart{}\textsc{Dr. Arthur Schnitzler}\pend{}\pstart{}\textsc{Wien\oindex{Wien@\textbf{Wien}, \emph{Verwaltungsgebiet}|pw}}\pend{}\pstart{}\textsc{IX. Frankgaſse 1\oindex{Wien@\textbf{Wien}!IX., Alsergrund@\textbf{IX., Alsergrund}!Frankgasse 1@\textbf{Frankgasse 1}, \emph{Wohngebäude}|pw}.}\pend{}{\bigskip}
\pstart
           \noindent{}\centering{}{\pb}\textcolor{gray}{\textbf{\textbf{Restaurant ersten Ranges Lanzsch
                           {\kaufmannsund} Co.\oindex{Lanzsch {\kaufmannsund} Co.@\textbf{Lanzsch {\kaufmannsund} Co.}, \emph{Gastgewerbegebäude}|pw}}}}\pend
           
\pstart
           \centering{}\textcolor{gray}{\textbf{\textbf{BERLIN\oindex{Berlin@\textbf{Berlin}, \emph{Hauptstadt}|pw},}Charlotten-Strasse 56\oindex{Charlottenstraße@\textbf{Charlottenstraße}, \emph{Straße}|pw}}}\pend
           
\pstart
           \centering{}\textcolor{gray}{\textbf{\begin{otherlanguage}{french}vis à vis\end{otherlanguage}{ }Schauspielhaus\oindex{Schauspielhaus Berlin@\textbf{Schauspielhaus Berlin}, \emph{Theater}|pw}}}\pend
           \vspace{1em}
\pstart
           \noindent{}{\pb}Lieber Freund, Gerade erzählt mir Herr \textsc{Christians\pwindex{Christians, Rudolf 15.\,1.\,1869 Middoge – 7.\,2.\,1921 Pasadena@\textsc{Christians, Rudolf} (15.\,1.\,1869 Middoge – 7.\,2.\,1921 Pasadena), \emph{Schauspieler}|pw}}, daß er der \label{K_L03061-1v}\edtext{erſte \textsc{Anatol\pwindex{Schnitzler, Arthur 15.\,5.\,1862 Wien – 21.\,10.\,1931 ebd.@\textsc{Schnitzler, Arthur} (15.\,5.\,1862 Wien – 21.\,10.\,1931 ebd.), \emph{Schriftsteller, Mediziner}!Anatol@\strich\emph{Anatol}|pwv}}}{\lemma{\textnormal{\emph{erste Anatol}}}\Cendnote{\textnormal{Am 16. 1. 1898 hatte Rudolf Christians\pwindex{Christians, Rudolf 15.\,1.\,1869 Middoge – 7.\,2.\,1921 Pasadena@\textsc{Christians, Rudolf} (15.\,1.\,1869 Middoge – 7.\,2.\,1921 Pasadena), \emph{Schauspieler}|pwk} bei der Uraufführung von \emph{Weihnachts-Einkäufe}\pwindex{Schnitzler, Arthur 15.\,5.\,1862 Wien – 21.\,10.\,1931 ebd.@\textsc{Schnitzler, Arthur} (15.\,5.\,1862 Wien – 21.\,10.\,1931 ebd.), \emph{Schriftsteller, Mediziner}!Weihnachts-Einkäufe@\strich\emph{Weihnachts-Einkäufe}|pwk} und
                     der Premiere von \emph{Abschiedssouper}\pwindex{Schnitzler, Arthur 15.\,5.\,1862 Wien – 21.\,10.\,1931 ebd.@\textsc{Schnitzler, Arthur} (15.\,5.\,1862 Wien – 21.\,10.\,1931 ebd.), \emph{Schriftsteller, Mediziner}!Abschiedssouper@\strich\emph{Abschiedssouper}|pwk}\eventindex{Sofiensäle@\textbf{Sofiensäle}!Premiere von Abschiedssouper, Uraufführung Weihnachtseinkäufe, 16.1.1898@Premiere von Abschiedssouper, Uraufführung Weihnachtseinkäufe, 16.1.1898|pwk} am
                     \emph{Deutschen Volkstheater}\orgindex{Volkstheater@Volkstheater|pwk} die Figur des Anatol\pwindex{Schnitzler, Arthur 15.\,5.\,1862 Wien – 21.\,10.\,1931 ebd.@\textsc{Schnitzler, Arthur} (15.\,5.\,1862 Wien – 21.\,10.\,1931 ebd.), \emph{Schriftsteller, Mediziner}!Anatol@\strich\emph{Anatol}|pwkv} gespielt. Da einzelne
                     Anatol\pwindex{Schnitzler, Arthur 15.\,5.\,1862 Wien – 21.\,10.\,1931 ebd.@\textsc{Schnitzler, Arthur} (15.\,5.\,1862 Wien – 21.\,10.\,1931 ebd.), \emph{Schriftsteller, Mediziner}!Anatol@\strich\emph{Anatol}|pwkv}-Stücke bereits
                  früher aufgeführt worden waren, stimmt die Behauptung, er wäre der erste gewesen,
                  nicht.}}}\label{K_L03061-1} war. Wir benutzen die Gelegenheit, Dir einen Gruß zu{ }ſenden.
               Herzlichſt Dein \spacefill\mbox{Paul Goldmann\textcolor{gray}{.}}\pend
           \selectlanguage{ngerman}\vspace{1em}
\pstart\center{}{[}hs. Christians:{]} Mein{ }ſehr verehrter, lieber Herr
                  Schnitzler!\pend\vspace{0.5em}
\pstart
           Ich freue mich richtig, Ihnen, verehrteſter Herr D\textsuperscript{r}, in
               Erinnerung an unſere »\textsc{Weihnachtseinkäufe\eventindex{Sofiensäle@\textbf{Sofiensäle}!Premiere von Abschiedssouper, Uraufführung Weihnachtseinkäufe, 16.1.1898@Premiere von Abschiedssouper, Uraufführung Weihnachtseinkäufe, 16.1.1898|pwv}\pwindex{Schnitzler, Arthur 15.\,5.\,1862 Wien – 21.\,10.\,1931 ebd.@\textsc{Schnitzler, Arthur} (15.\,5.\,1862 Wien – 21.\,10.\,1931 ebd.), \emph{Schriftsteller, Mediziner}!Weihnachts-Einkäufe@\strich\emph{Weihnachts-Einkäufe}|pw}}« die herzlichſten Grüße zu{ }ſenden! Was macht »\textsc{Schleier der Beatrice\pwindex{Schnitzler, Arthur 15.\,5.\,1862 Wien – 21.\,10.\,1931 ebd.@\textsc{Schnitzler, Arthur} (15.\,5.\,1862 Wien – 21.\,10.\,1931 ebd.), \emph{Schriftsteller, Mediziner}!Schleier der Beatrice. Schauspiel in fünf Akten@\strich\emph{Der Schleier der Beatrice. Schauspiel in fünf Akten}|pw}}«? Warum nicht ich?\pend
           \pstart Ihr \spacefill\mbox{Christians}\pend{}\selectlanguage{ngerman}\vspace{1em}
\pstart
           \noindent{}{[}hs. Klein:{]} Höflichen Gruß \spacefill\mbox{Bertha Christians.}\pend
           \selectlanguage{ngerman}\endnumbering\briefempfaengerindex{Schnitzler, Arthur@\textsc{Schnitzler, Arthur}!zzzKlein, Bertha@\emph{von Bertha Klein}!1901-03-241@{24. 3. 1901}|)be}\briefempfaengerindex{Schnitzler, Arthur@\textsc{Schnitzler, Arthur}!zzzChristians, Rudolf@\emph{von Rudolf Christians}!1901-03-241@{24. 3. 1901}|)be}\briefempfaengerindex{Schnitzler, Arthur@\textsc{Schnitzler, Arthur}!zzzGoldmann, Paul@\emph{von Paul Goldmann}!1901-03-241@{24. 3. 1901}|)be}\mylabel{L03061h}  \newcommand{\dateiname}{L03061}\newcommand{\titel}{Paul Goldmann, Bertha und Rudolf Christians an Arthur Schnitzler, 24. 3. 1901}\newcommand{\editorInnen}{Martin Anton Müller und Laura Untner}%% latex-leseansicht-abspann.tex
%% Abspann für die Leseansicht.
%% Der Schalter \ifkorrekturansicht ist bereits durch den Vorspann gesetzt.

%% latex-abspann.tex
%% Gemeinsamer Abspann für Korrekturansicht und Leseansicht.
%% Setzt den Schalter \ifkorrekturansicht voraus (gesetzt in den
%% einbindenden Dateien latex-korrekturansicht-abspann.tex bzw.
%% latex-leseansicht-abspann.tex).
%% ---------------------------------------------------------------

\normalsize

% Das esempio-Environment wird nur in der Leseansicht benötigt
\ifkorrekturansicht\else
\newenvironment{esempio}[3]%
{
    \vspace{1.5ex}
    \rlap{\underline{#1}}
    \par
    \setlength{\parindent}{0cm}
    \nopagebreak
    \leftskip=#2cm
    \rightskip=#3cm
}
{
    \par
}
\fi

\doendnotes{C}
\bigskip
\vfill

\clearpage

\footnotesize

\ifkorrekturansicht
  \lohead{\textsc{register}}
\fi

% theindex-Environment neu definieren ohne reledmac
\makeatletter
\renewenvironment{theindex}{%
  \ifkorrekturansicht
    \section*{\indexname}%
  \else
    \subsubsection*{Index der erwähnten Entitäten}%
  \fi
  \setlength{\parindent}{0pt}%
  \setlength{\parskip}{0pt plus 0.3pt}%
  \let\item\@idxitem
}{%
  \ifkorrekturansicht\clearpage\fi
}
\makeatother

\IfFileExists{\jobname-pw.ind}{\input{\jobname-pw.ind}}{}

% Quellenangabe nur in der Leseansicht
\ifkorrekturansicht\else
% Fallback-Definitionen, falls die .tex-Datei \titel etc. nicht gesetzt hat
\providecommand{\titel}{}
\providecommand{\editorInnen}{}
\providecommand{\dateiname}{\jobname}

\vspace{3cm}

\vfill

\footnotesize
\textsc{Quelle}: \titel. Herausgegeben von {\editorInnen}. In: \emph{Arthur Schnitzler: Briefwechsel mit Autorinnen und Autoren}.
 Digitale Edition, https://schnitzler-briefe.acdh.oeaw.ac.at/{\dateiname}.html (Stand \today)
\fi

\end{document}


