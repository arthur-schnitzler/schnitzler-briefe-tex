%% latex-korrekturansicht-vorspann.tex
%% Vorspann für die Korrekturansicht.
%% Lädt die gemeinsame Datei latex-vorspann.tex mit gesetztem Schalter.

\newif\ifkorrekturansicht
\korrekturansichttrue

\input{../tex-inputs/latex-vorspann}


\section[Marie Herzfeld an Arthur Schnitzler, 30. 3. 1913]{L02594 Marie Herzfeld an Arthur Schnitzler, 30. 3. 1913}
\nopagebreak\mylabel{L02594v}
\rehead{ }\normalsize\beginnumbering\briefempfaengerindex{Schnitzler, Arthur@\textsc{Schnitzler, Arthur}!zzzHerzfeld, Marie@\emph{von Marie Herzfeld}!1913-03-301@{30. 3. 1913}|(be}
\toendnotes[C]{\smallbreak\pagebreak[2]}\Standort{DLA, A:Schnitzler, HS.1985.1.03436,5.}
\physDesc{Brief, 1 Blatt, 2 Seiten, 806 Zeichen
\newline{}Handschrift: schwarze Tinte, lateinische Kurrent
\newline{}Schnitzler: 1) mit Bleistift Vermerk »\textsc{\uline{Herzfel{[}d{]}.}}«  2) mit rotem Buntstift Vermerk »\textsc{Seybel\pwindex{Seybel, Georg von 1886 – 10.04.1924@\textsc{Seybel, Georg von} (1886 – 10.04.1924), \emph{Schriftsteller/Schriftstellerin}|pw}, Barbi\pwindex{Barbi, Alice 1.6.1858 – 4.9.1948@\textsc{Barbi, Alice} (1.6.1858 – 4.9.1948), \emph{Sänger/Sängerin, Komponist/Komponistin}|pw}}« und eine Unterstreichung}\toendnotes[C]{\smallbreak}
\pstart
           \raggedleft{}{\pb}Wien II/\textsubscript{1},
                     Lichtenauerg. 5\oindex{Lichtenauergasse@\textbf{Lichtenauergasse}, \emph{Straße (K.STR)}|pw}\pend
           
\pstart
           30/III 1913\pend
           
\pstart\center{}Geehrter Herr Doktor!\pend\vspace{0.5em}
\pstart
           D\textsuperscript{r}{ }Georg von Seybel\pwindex{Seybel, Georg von 1886 – 10.04.1924@\textsc{Seybel, Georg von} (1886 – 10.04.1924), \emph{Schriftsteller/Schriftstellerin}|pw} hat eine \label{K_L02594-1v}\edtext{Adresse an die Barbi\pwindex{Barbi, Alice 1.6.1858 – 4.9.1948@\textsc{Barbi, Alice} (1.6.1858 – 4.9.1948), \emph{Sänger/Sängerin, Komponist/Komponistin}|pw} verfasst, um sie zu bitten, dass sie nach Wien\oindex{Wien@\textbf{Wien}, \emph{A.ADM2}|pw} komme\strikeout{,} und als
               letzte im Bösendorfersal\oindex{Boesendorfer-Saal@\textbf{Bösendorfer-Saal}, \emph{Veranstaltungsgebäude (K.VSB)}|pw}{ }\strikeout{zu} singe\strikeout{n}.}{\lemma{\textnormal{\emph{Adresse … singe.}}}\Cendnote{\textnormal{Am 2. 5. 1913 wurde der Bösendorfer-Saal\oindex{Boesendorfer-Saal@\textbf{Bösendorfer-Saal}, \emph{Veranstaltungsgebäude (K.VSB)}|pwk} für immer geschlossen. Davor
                  sollten, nach Plan von Hugo Knepler\pwindex{Knepler, Hugo 10.08.1872 – 1944@\textsc{Knepler, Hugo} (10.08.1872 – 1944), \emph{Musikverleger/Musikverlegerin, Kunsthändler/Kunsthändlerin, Buchhändler/Buchhändlerin}|pwk}, dem
                  Inhaber der \emph{Konzertdirektion Gutmann}\orgindex{Concert-Direktion Albert Gutmann@Concert-Direktion Albert Gutmann|pwk}, vier
                     »Abschiedskonzerte« stattfinden ([O. V.]: \emph{Abschiedskonzerte im Bösendorfer-Saale}\pwindex{Abschiedskonzerte im Boesendorfer-Saale@\emph{Abschiedskonzerte im Bösendorfer-Saale}|pwk}. In: \emph{Fremden-Blatt}\pwindex{Abschiedskonzerte im Boesendorfer-Saale@\emph{Abschiedskonzerte im Bösendorfer-Saale}|pwk}, Jg. 67, Nr. 86,
                        30. 3. 1913, S. 10). Kurz vor der Schließung wurde von
                  der hier angesprochenen »Adresse« berichtet und dass die Sängerin
                     Alice Barbi\pwindex{Barbi, Alice 1.6.1858 – 4.9.1948@\textsc{Barbi, Alice} (1.6.1858 – 4.9.1948), \emph{Sänger/Sängerin, Komponist/Komponistin}|pwk} diese Einladung abgelehnt hatte
                     ([O. V.]: \emph{Abschiedskonzerte im
                        Bösendorfersaale}\pwindex{Einige der bekanntesten Persoenlichkeiten der Wiener Musikwelt]@\emph{[Einige der bekanntesten Persönlichkeiten der Wiener Musikwelt]}|pwk}. In: \emph{Neue Wiener
                        Tagblatt}\pwindex{Neues Wiener Tagblatt@\emph{Neues Wiener Tagblatt}|pwk}, Jg. 47, Nr. 104, 17. 4. 1913,
                  S. 16).}}}\label{K_L02594-1} Warum diese Sache als Geheimnis behandelt wird, weiß ich
               nicht; Faktum ist, dass nur »Auserwählte« unterzeichnen sollen – und dass alles mit
               feierlicher {\pb}Langsamkeit vor sich geht –, da der Verf.\pwindex{Seybel, Georg von 1886 – 10.04.1924@\textsc{Seybel, Georg von} (1886 – 10.04.1924), \emph{Schriftsteller/Schriftstellerin}|pwv} des Schriftstückes
               verreist. Von morgen an wird die Adresse, die bisher
                  \label{T_L02594-1v}\edtext{von Haus zu Haus}{\lemma{\textnormal{\emph{von Haus zu Haus}}}\Cendnote{\textnormal{In der Vorlage steht: »zu Haus zu
                     Haus«.}}}\label{T_L02594-1} getragen wurde, bei Gutmann\orgindex{Concert-Direktion Albert Gutmann@Concert-Direktion Albert Gutmann|pw} zur Unterzeichnung aufliegen und da ich weiß, wie hoch die Barbi\pwindex{Barbi, Alice 1.6.1858 – 4.9.1948@\textsc{Barbi, Alice} (1.6.1858 – 4.9.1948), \emph{Sänger/Sängerin, Komponist/Komponistin}|pw} ihre Arbeiten schätzt und umgekehrt weiß,
               wieviel Genuss Sie ihr danken, so hoffe ich, Sie setzen Ihren Namen auf die Blätter.
               Ob die Adresse im \label{K_L02594-2v}\edtext{Opernhaus\oindex{Oper@\textbf{Oper}, \emph{Oper (K.OPR)}|pw} oder in der Schellingg.\oindex{Schellinggasse@\textbf{Schellinggasse}, \emph{Straße (K.STR)}|pw}}{\lemma{\textnormal{\emph{Opernhaus … Schellingg.}}}\Cendnote{\textnormal{Die \emph{Konzertdirektion Gutmann}\orgindex{Concert-Direktion Albert Gutmann@Concert-Direktion Albert Gutmann|pwk} betrieb ein Kartenbüro in der Oper\oindex{Oper@\textbf{Oper}, \emph{Oper (K.OPR)}|pwk}, hatte aber ihren Haupsitz in der Schellinggasse\oindex{Schellinggasse@\textbf{Schellinggasse}, \emph{Straße (K.STR)}|pwk}.}}}\label{K_L02594-2} sein wird, lasse ich Ihnen morgen telephonieren.\pend
           
\pstart
           Wärmstens {\\[\baselineskip]}\spacefill\mbox{Marie Herzfeld}\pend
           \leftskip=0em{}\selectlanguage{ngerman}\endnumbering\briefempfaengerindex{Schnitzler, Arthur@\textsc{Schnitzler, Arthur}!zzzHerzfeld, Marie@\emph{von Marie Herzfeld}!1913-03-301@{30. 3. 1913}|)be}\mylabel{L02594h}  \normalsize

\doendnotes{C}
\bigskip
\vfill

\clearpage

\footnotesize

\lohead{\textsc{register}}

% Definiere theindex-Environment komplett neu ohne reledmac
\makeatletter
\renewenvironment{theindex}{%
  \section*{\indexname}%
  \setlength{\parindent}{0pt}%
  \setlength{\parskip}{0pt plus 0.3pt}%
  \let\item\@idxitem
}{%
  \clearpage
}
\makeatother

\IfFileExists{\jobname-pw.ind}{\input{\jobname-pw.ind}}{}

\end{document}

      