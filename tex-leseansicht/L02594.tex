%% latex-leseansicht-vorspann.tex
%% Vorspann für die Leseansicht.
%% Lädt die gemeinsame Datei latex-vorspann.tex mit nicht gesetztem Schalter.

\newif\ifkorrekturansicht
\korrekturansichtfalse

\input{../tex-inputs/latex-vorspann}


               \section[Marie Herzfeld an Arthur Schnitzler, 30. 3. 1913]{ Marie Herzfeld an Arthur Schnitzler, 30. 3. 1913}\nopagebreak\mylabel{v}\rehead{ }\begin{ledgroupsized}[t]{13cm}\normalsize\beginnumbering\briefempfaengerindex{Schnitzler, Arthur@\textsc{Schnitzler, Arthur}!zzzHerzfeld, Marie@\emph{von Marie Herzfeld}!1913-03-301@{30. 3. 1913}|(be} \toendnotes[C]{\smallbreak\pagebreak[2]} \Standort{DLA, A:Schnitzler, HS.1985.1.03436,5.}
\physDesc{Brief, 1 Blatt, 2 Seiten
\newline{}Handschrift: schwarze Tinte, lateinische Kurrent
\newline{}Schnitzler: 1) mit Bleistift Vermerk »\textsc{\uline{Herzfel{[}d{]}.}}« 2) mit rotem Buntstift Vermerk »\textsc{Seybel\pwindex{Seybel, Georg von 1886 – 10.04.1924@\textsc{Seybel, Georg von} (1886 – 10.04.1924), \emph{Schriftsteller}|pw}, Barbi\pwindex{Barbi, Alice 1.6.1858 – 4.9.1948@\textsc{Barbi, Alice} (1.6.1858 – 4.9.1948), \emph{Sängerin, Komponistin}|pw}}« und eine
                                 Unterstreichung}\toendnotes[C]{\smallbreak}\pstart
           \noindent{}\raggedleft{}{\pb}Wien II/\textsubscript{1}, Lichtenauerg. 5\oindex{Lichtenauergasse@\textbf{Lichtenauergasse}|pw}\pend
           \pstart
           30/III 1913\pend
           \pstart\center{}Geehrter Herr Doktor!\pend\pstart
           D\textsuperscript{r}{ }Georg von Seybel\pwindex{Seybel, Georg von 1886 – 10.04.1924@\textsc{Seybel, Georg von} (1886 – 10.04.1924), \emph{Schriftsteller}|pw} hat eine \label{K_L02594-1v}\edtext{Adresse an die Barbi\pwindex{Barbi, Alice 1.6.1858 – 4.9.1948@\textsc{Barbi, Alice} (1.6.1858 – 4.9.1948), \emph{Sängerin, Komponistin}|pw}
               verfasst, um sie zu bitten, dass sie nach Wien\oindex{Wien@\textbf{Wien}|pw}
                  komme\strikeout{,} und als letzte im Bösendorfersal\oindex{Boesendorfer-Saal@\textbf{Bösendorfer-Saal}|pw}{ }\strikeout{zu} singe\strikeout{n}.}{\lemma{\textnormal{\emph{Adresse … singen.}}}\Cendnote{\textnormal{Am 2. 5. 1913 wurde der Bösendorfer-Saal\oindex{Boesendorfer-Saal@\textbf{Bösendorfer-Saal}|pwk} für immer geschlossen. Davor
                  sollten, nach Plan von Hugo Knepler\pwindex{Knepler, Hugo 10.08.1872 – 1944@\textsc{Knepler, Hugo} (10.08.1872 – 1944), \emph{Musikverleger, Kunsthändler, Buchhändler}|pwk}, dem
                  Inhaber der \emph{Konzertdirektion Gutmann}\orgindex{Gutmann (Konzertdirektion)@Gutmann (Konzertdirektion)|pwk}, vier
                     »Abschiedskonzerte« stattfinden ([O. V.:] \emph{Abschiedskonzerte im Bösendorfer-Saale}\pwindex{?? Werk@Nicht ermittelte Verfasserinnen und Verfasser!Abschiedskonzerte im Boesendorfer-Saale1913-03-30 – 1913-03-30@\emph{Abschiedskonzerte im Bösendorfer-Saale} {[}1913-03-30 – 1913-03-30{]}|pwk}. In:
                        \emph{Fremden-Blatt}\pwindex{?? Werk@Nicht ermittelte Verfasserinnen und Verfasser!Abschiedskonzerte im Boesendorfer-Saale1913-03-30 – 1913-03-30@\emph{Abschiedskonzerte im Bösendorfer-Saale} {[}1913-03-30 – 1913-03-30{]}|pwk}, Jg. 67, Nr. 86,
                        30. 3. 1913, S. 10). Kurz vor der
                  Schließung wird von der hier angesprochenen »Adresse« berichtet und
                  dass die Sängerin Alice Barbi\pwindex{Barbi, Alice 1.6.1858 – 4.9.1948@\textsc{Barbi, Alice} (1.6.1858 – 4.9.1948), \emph{Sängerin, Komponistin}|pwk} diese Einladung
                  ablehnte ([O. V.:] \emph{Abschiedskonzerte im
                        Bösendorfersaale}\pwindex{?? Werk@Nicht ermittelte Verfasserinnen und Verfasser!Einige der bekanntesten Persoenlichkeiten der Wiener Musikwelt]1913-04-17 – 1913-04-17@\emph{[Einige der bekanntesten Persönlichkeiten der Wiener Musikwelt]} {[}1913-04-17 – 1913-04-17{]}|pwk}. In: \emph{Neue Wiener
                        Tagblatt}\pwindex{Neues Wiener Tagblatt1867 – 1945@\emph{Neues Wiener Tagblatt}|pwk}, Jg. 47, Nr. 104,
                     17. 4. 1913, S. 16).}}}\label{K_L02594-1h} Warum diese
               Sache als Geheimnis behandelt wird, weiß ich nicht; Faktum ist, dass nur
               »Auserwählte« unterzeichnen sollen – und dass alles mit feierlicher {\pb}Langsamkeit vor sich geht –, da der Verf.\pwindex{Seybel, Georg von 1886 – 10.04.1924@\textsc{Seybel, Georg von} (1886 – 10.04.1924), \emph{Schriftsteller}|pwv} des Schriftstückes verreist. Von morgen an wird die Adresse, die bisher \label{T_L02594-1v}\edtext{von Haus zu Haus}{\lemma{\textnormal{\emph{von Haus zu Haus}}}\Cendnote{\textnormal{sie schreibt: »zu Haus zu Haus«}}}\label{T_L02594-1h}
               getragen wurde, bei Gutmann\orgindex{Gutmann (Konzertdirektion)@Gutmann (Konzertdirektion)|pw} zur Unterzeichnung
               aufliegen und da ich weiß, wie hoch die Barbi\pwindex{Barbi, Alice 1.6.1858 – 4.9.1948@\textsc{Barbi, Alice} (1.6.1858 – 4.9.1948), \emph{Sängerin, Komponistin}|pw} ihre
               Arbeiten schätzt und umgekehrt weiß, wieviel Genuss Sie ihr danken, so hoffe ich, Sie
               setzen Ihren Namen auf die Blätter. Ob die Adresse im \label{K_L02594-11v}\edtext{Opernhaus\oindex{Oper@\textbf{Oper}|pw} oder in der Schellingg.\oindex{Schellinggasse@\textbf{Schellinggasse}|pw}}{\lemma{\textnormal{\emph{Opernhaus … Schellingg.}}}\Cendnote{\textnormal{Die \emph{Konzertdirektion Gutmann}\orgindex{Gutmann (Konzertdirektion)@Gutmann (Konzertdirektion)|pwk}
                        betrieb ein Kartenbüro in der Oper\oindex{Oper@\textbf{Oper}|pwk}, hatte aber ihren Haupsitz in der Schellinggasse\oindex{Schellinggasse@\textbf{Schellinggasse}|pwk}.}}}\label{K_L02594-11h} sein wird, lasse ich Ihnen morgen telephonieren.\pend
           \pstart
           Wärmstens {\\[\baselineskip]}\spacefill\mbox{Marie Herzfeld}\pend
           \leftskip=0em{}\endnumbering\briefempfaengerindex{Schnitzler, Arthur@\textsc{Schnitzler, Arthur}!zzzHerzfeld, Marie@\emph{von Marie Herzfeld}!1913-03-301@{30. 3. 1913}|)be}\mylabel{h}\end{ledgroupsized}  \newcommand{\dateiname}{L02594}\newcommand{\titel}{Marie Herzfeld an Arthur Schnitzler, 30. 3. 1913}\newcommand{\editorInnen}{Martin Anton Müller und Laura Untner}
            \footnotesize
\begin{ledgroupsized}[t]{11.5cm}
\doendnotes{C}
\end{ledgroupsized}
         %% latex-leseansicht-abspann.tex
%% Abspann für die Leseansicht.
%% Der Schalter \ifkorrekturansicht ist bereits durch den Vorspann gesetzt.

%% latex-abspann.tex
%% Gemeinsamer Abspann für Korrekturansicht und Leseansicht.
%% Setzt den Schalter \ifkorrekturansicht voraus (gesetzt in den
%% einbindenden Dateien latex-korrekturansicht-abspann.tex bzw.
%% latex-leseansicht-abspann.tex).
%% ---------------------------------------------------------------

\normalsize

% Das esempio-Environment wird nur in der Leseansicht benötigt
\ifkorrekturansicht\else
\newenvironment{esempio}[3]%
{
    \vspace{1.5ex}
    \rlap{\underline{#1}}
    \par
    \setlength{\parindent}{0cm}
    \nopagebreak
    \leftskip=#2cm
    \rightskip=#3cm
}
{
    \par
}
\fi

\doendnotes{C}
\bigskip
\vfill

\clearpage

\footnotesize

\ifkorrekturansicht
  \lohead{\textsc{register}}
\fi

% theindex-Environment neu definieren ohne reledmac
\makeatletter
\renewenvironment{theindex}{%
  \ifkorrekturansicht
    \section*{\indexname}%
  \else
    \subsubsection*{Index der erwähnten Entitäten}%
  \fi
  \setlength{\parindent}{0pt}%
  \setlength{\parskip}{0pt plus 0.3pt}%
  \let\item\@idxitem
}{%
  \ifkorrekturansicht\clearpage\fi
}
\makeatother

\IfFileExists{\jobname-pw.ind}{\input{\jobname-pw.ind}}{}

% Quellenangabe nur in der Leseansicht
\ifkorrekturansicht\else
% Fallback-Definitionen, falls die .tex-Datei \titel etc. nicht gesetzt hat
\providecommand{\titel}{}
\providecommand{\editorInnen}{}
\providecommand{\dateiname}{\jobname}

\vspace{3cm}

\vfill

\footnotesize
\textsc{Quelle}: \titel. Herausgegeben von {\editorInnen}. In: \emph{Arthur Schnitzler: Briefwechsel mit Autorinnen und Autoren}.
 Digitale Edition, https://schnitzler-briefe.acdh.oeaw.ac.at/{\dateiname}.html (Stand \today)
\fi

\end{document}


      