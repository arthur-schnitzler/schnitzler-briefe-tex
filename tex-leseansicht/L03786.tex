%% latex-leseansicht-vorspann.tex
%% Vorspann für die Leseansicht.
%% Lädt die gemeinsame Datei latex-vorspann.tex mit nicht gesetztem Schalter.

\newif\ifkorrekturansicht
\korrekturansichtfalse

\input{../tex-inputs/latex-vorspann}


\section[Arthur Schnitzler an Stefan Zweig, 18. 8. 1917]{L03786 Arthur Schnitzler an Stefan Zweig, 18. 8. 1917}
\nopagebreak\mylabel{L03786v}
\rehead{ }\normalsize\beginnumbering\briefempfaengerindex{Zweig, Stefan@\textsc{Zweig, Stefan}!zzzSchnitzler, Arthur@\emph{von Arthur Schnitzler}!1917-08-181@{18. 8. 1917}|(be}
\toendnotes[C]{\smallbreak\pagebreak[2]}
\correspDesc{Versand  durch Arthur Schnitzler am 18. 8. 1917 in Wien
\newline{}Erhalt  durch Stefan Zweig im Zeitraum [19. 8. 1917 – 23. 8. 1917?] in Wien}\toendnotes[C]{\smallbreak}
\Standort{Jerusalem, National Library of Israel, ARC. Ms. Var. 305 1 58 Stefan Zweig Collection.}
\physDesc{Briefkarte, 1394 Zeichen
\newline{}Handschrift: schwarze Tinte, lateinische Kurrent}\toendnotes[C]{\smallbreak}
\pstart
           {\pb}\textcolor{gray}{\textbf{Dr. Arthur Schnitzler}}\hfill 18. 8. 17\pend
           
\pstart
           \textcolor{gray}{\textbf{Wien XVIII. Sternwartestrasse 71\oindex{Wien@\textbf{Wien}!XVIII., Währing@\textbf{XVIII., Währing}!Sternwartestraße 71@\textbf{Sternwartestraße 71}, \emph{Wohngebäude}|pw}}}\pend
           \vspace{0.5em}
\pstart
           lieber Herr Doctor, Ihr Dank ko{\geminationm}t so beschämend rasch – noch bevor
               ich selbst auch nur ein einziges Exemplar meines neues Buches\pwindex{Schnitzler, Arthur 15. 5. 1862 Wien – 21. 10. 1931 ebd.@\textsc{Schnitzler, Arthur} (15. 5. 1862 Wien – 21. 10. 1931 ebd.), \emph{Schriftsteller, Mediziner}!Doktor Gräsler, Badearzt@\strich\emph{Doktor Gräsler, Badearzt}|pwv} in Händen habe, und ich freue mich wie immer, Ihrer
               Antheilnahme und der schönen Art, in der Sie sie kundzugeben wissen. Einwendungen,
               besonders we{\geminationn} sie von jemandem kommen, an deren Schätzung man nicht zweifeln kann,
               sind gewissermaßen immer berechtigt; und wird \introOben{}auch\introOben{} mein künstlerisches
               Gefühl gerade durch den Schluss durchaus befriedigt (we{\geminationn} man sich vielleicht auch
               einen Dr. Graesler\pwindex{Schnitzler, Arthur 15. 5. 1862 Wien – 21. 10. 1931 ebd.@\textsc{Schnitzler, Arthur} (15. 5. 1862 Wien – 21. 10. 1931 ebd.), \emph{Schriftsteller, Mediziner}!Doktor Gräsler, Badearzt@\strich\emph{Doktor Gräsler, Badearzt}|pw}, II. Theil denken könnte\substVorne{}\textsuperscript{)}\substDazwischen{},\substHinten{} der ihn als alternden Ehemann und Arzt in Lanzarote\oindex{Lanzarote@\textbf{Lanzarote}, \emph{Insel}|pw} zeichnete) {\pb}so halte
               ich es \introOben{}doch\introOben{} für sehr denkbar, daß irgend ein Mangel, der sich
                   andrer Stelle finden mag, wie das oft der Fall ist, erst
               am Ende herauskommt. Da jedes künstlerische Product eine Einheit vorstellt, handelt
               es sich hier nicht um einen Irrtum des Beurtheilers, sondern um etwas ähnliches wie
               bei der sog. »falschen Localisation« die dem Nervenarzt bekannt ist: Schmerzen werden
               an einer von der kranken \strikeout{Stelle} weit entfernten
               Stelle empfunden. Hier rühre ich vielleicht an ein aesthetisch kritisches Problem,
               das man näher betrachten könnte.\pend
           
\pstart
           Wie Sie unter den »tausend Tagen« leiden, vermag ich
               Ihnen wohl nachzufühlen\textcolor{gray}{:} möge Ihr Buch\pwindex{Zweig, Stefan 28.\,11.\,1881 Wien – 23.\,2.\,1942 Petrópolis@\textsc{Zweig, Stefan} (28.\,11.\,1881 Wien – 23.\,2.\,1942 Petrópolis), \emph{Schriftsteller}!Jeremias. Ein dramatische Dichtung in neun Bildern@\strich\emph{Jeremias. Ein dramatische Dichtung in neun Bildern}|pwv},
               dem ich mich entgegenfreue, Sie we{\geminationn} nicht befreit, doch wenigstens entschädigt
               haben. Seien Sie vielmals, auch von meiner Frau\pwindex{Schnitzler, Olga 17.\,1.\,1882 Wien – 13.\,1.\,1970 Lugano@\textsc{Schnitzler, Olga} (17.\,1.\,1882 Wien – 13.\,1.\,1970 Lugano), \emph{Schauspielerin, Sängerin}|pwv}, u herzlichst gegrüßt,\pend
           \pstart Ihr \spacefill\mbox{A. S.}\pend{}\selectlanguage{ngerman}\endnumbering\briefempfaengerindex{Zweig, Stefan@\textsc{Zweig, Stefan}!zzzSchnitzler, Arthur@\emph{von Arthur Schnitzler}!1917-08-181@{18. 8. 1917}|)be}\mylabel{L03786h}  \newcommand{\dateiname}{L03786}\newcommand{\titel}{Arthur Schnitzler an Stefan Zweig, 18. 8. 1917}\newcommand{\editorInnen}{Selma Jahnke und Martin Anton Müller}%% latex-leseansicht-abspann.tex
%% Abspann für die Leseansicht.
%% Der Schalter \ifkorrekturansicht ist bereits durch den Vorspann gesetzt.

%% latex-abspann.tex
%% Gemeinsamer Abspann für Korrekturansicht und Leseansicht.
%% Setzt den Schalter \ifkorrekturansicht voraus (gesetzt in den
%% einbindenden Dateien latex-korrekturansicht-abspann.tex bzw.
%% latex-leseansicht-abspann.tex).
%% ---------------------------------------------------------------

\normalsize

% Das esempio-Environment wird nur in der Leseansicht benötigt
\ifkorrekturansicht\else
\newenvironment{esempio}[3]%
{
    \vspace{1.5ex}
    \rlap{\underline{#1}}
    \par
    \setlength{\parindent}{0cm}
    \nopagebreak
    \leftskip=#2cm
    \rightskip=#3cm
}
{
    \par
}
\fi

\doendnotes{C}
\bigskip
\vfill

\clearpage

\footnotesize

\ifkorrekturansicht
  \lohead{\textsc{register}}
\fi

% theindex-Environment neu definieren ohne reledmac
\makeatletter
\renewenvironment{theindex}{%
  \ifkorrekturansicht
    \section*{\indexname}%
  \else
    \subsubsection*{Index der erwähnten Entitäten}%
  \fi
  \setlength{\parindent}{0pt}%
  \setlength{\parskip}{0pt plus 0.3pt}%
  \let\item\@idxitem
}{%
  \ifkorrekturansicht\clearpage\fi
}
\makeatother

\IfFileExists{\jobname-pw.ind}{\input{\jobname-pw.ind}}{}

% Quellenangabe nur in der Leseansicht
\ifkorrekturansicht\else
% Fallback-Definitionen, falls die .tex-Datei \titel etc. nicht gesetzt hat
\providecommand{\titel}{}
\providecommand{\editorInnen}{}
\providecommand{\dateiname}{\jobname}

\vspace{3cm}

\vfill

\footnotesize
\textsc{Quelle}: \titel. Herausgegeben von {\editorInnen}. In: \emph{Arthur Schnitzler: Briefwechsel mit Autorinnen und Autoren}.
 Digitale Edition, https://schnitzler-briefe.acdh.oeaw.ac.at/{\dateiname}.html (Stand \today)
\fi

\end{document}


