%% latex-korrekturansicht-vorspann.tex
%% Vorspann für die Korrekturansicht.
%% Lädt die gemeinsame Datei latex-vorspann.tex mit gesetztem Schalter.

\newif\ifkorrekturansicht
\korrekturansichttrue

\input{../tex-inputs/latex-vorspann}


\section[Arthur Schnitzler an Stefan Zweig, 18. 8. 1917]{L03786 Arthur Schnitzler an Stefan Zweig, 18. 8. 1917}
\nopagebreak\mylabel{L03786v}
\rehead{ }\normalsize\beginnumbering\briefempfaengerindex{Zweig, Stefan@\textsc{Zweig, Stefan}!zzzSchnitzler, Arthur@\emph{von Arthur Schnitzler}!1917-08-181@{18. 8. 1917}|(be}
\toendnotes[C]{\smallbreak\pagebreak[2]}\Standort{Jerusalem, National Library of Israel, ARC. Ms. Var. 305 1 58 Stefan Zweig Collection.}
\physDesc{Briefkarte, 1 Blatt, 2 Seiten, 1397 Zeichen
\newline{}Handschrift: schwarze Tinte, lateinische Kurrent}\toendnotes[C]{\smallbreak}
\pstart
           {\pb}\textcolor{gray}{\textbf{Dr. Arthur Schnitzler}}\hfill 18. 8. 17\pend
           
\pstart
           \textcolor{gray}{\textbf{Wien XVIII. Sternwartestrasse 71\oindex{Sternwartestrasse 71@\textbf{Sternwartestraße 71}, \emph{Wohngebäude (K.WHS)}|pw}}}\pend
           \vspace{0.5em}
\pstart
           lieber Herr Doctor, Ihr Dank ko{\geminationm}t so beschämend rasch – noch bevor
               ich selbst auch nur ein einziges Exemplar meines neues Buches\pwindex{Doktor Graesler, Badearzt@\emph{Doktor Gräsler, Badearzt}|pwv} in Händen habe, und ich freue mich wie immer, Ihrer
               Antheilnahme und der schönen Art, in der Sie sie kundzugeben wissen. Einwendungen,
               besonders we{\geminationn} sie von jemandem kommen, an deren Schätzung man nicht zweifeln kann,
               sind gewissermaßen immer berechtigt; und wird \introOben{}auch\introOben{} mein künstlerisches
               Gefühl gerade durch den Schluss durchaus befriedigt (we{\geminationn} man sich vielleicht auch
               einen Dr. Graesler\pwindex{Doktor Graesler, Badearzt@\emph{Doktor Gräsler, Badearzt}|pw}, II. Theil denken könnte\substVorne{}\textsuperscript{)}\substDazwischen{},\substHinten{} der ihn als alternden Ehemann und Arzt in Lanzarote\oindex{Lanzarote@\textbf{Lanzarote}, \emph{T.ISL}|pw} zeichnete) {\pb}so halte
               ich es \introOben{}doch\introOben{} für sehr denkbar, daß irgend ein Mangel, der sich
                  {[}an{]} andren Stelle finden mag, wie das oft der Fall ist, erst
               am Ende herauskommt. Da jedes künstlerische Product eine Einheit vorstellt, handelt
               es sich hier nicht um einen Irrtum des Beurtheilers, sondern um etwas ähnliches wie
               bei der sog. »falschen Localisation« die dem Nervenarzt bekannt ist: Schmerzen werden
               an einer von der kranken \strikeout{Stelle} weit entfernten
               Stelle empfunden. Hier rühre ich vielleicht an ein aesthetisch kritisches Problem,
               das man näher betrachten könnte. Wie Sie unter den »tausend Tagen« leiden, vermag ich
               Ihnen wohl nachzufühlen! möge Ihr Buch\pwindex{Jeremias. Ein dramatische Dichtung in neun Bildern@\emph{Jeremias. Ein dramatische Dichtung in neun Bildern}|pwv},
               dem ich mich entgegenfreue, Sie we{\geminationn} nicht befreit, doch wenigstens entschädigt
               haben.\pend
           \pstart Seien Sie vielmals, auch von meiner Frau\pwindex{Schnitzler, Olga 17.01.1882 – 13.01.1970@\textsc{Schnitzler, Olga} (17.01.1882 – 13.01.1970), \emph{Schauspieler/Schauspielerin, Sänger/Sängerin}|pwv}, u herzlichst gegrüßt, Ihr \spacefill\mbox{A. S.}\pend{}\selectlanguage{ngerman}\endnumbering\briefempfaengerindex{Zweig, Stefan@\textsc{Zweig, Stefan}!zzzSchnitzler, Arthur@\emph{von Arthur Schnitzler}!1917-08-181@{18. 8. 1917}|)be}\mylabel{L03786h}
\begin{anhang}
\end{anhang}\normalsize

\doendnotes{C}
\bigskip
\vfill

\clearpage

\footnotesize

\lohead{\textsc{register}}

% Definiere theindex-Environment komplett neu ohne reledmac
\makeatletter
\renewenvironment{theindex}{%
  \section*{\indexname}%
  \setlength{\parindent}{0pt}%
  \setlength{\parskip}{0pt plus 0.3pt}%
  \let\item\@idxitem
}{%
  \clearpage
}
\makeatother

\IfFileExists{\jobname-pw.ind}{\input{\jobname-pw.ind}}{}

\end{document}

      