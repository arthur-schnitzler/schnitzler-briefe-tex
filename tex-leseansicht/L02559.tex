%% latex-leseansicht-vorspann.tex
%% Vorspann für die Leseansicht.
%% Lädt die gemeinsame Datei latex-vorspann.tex mit nicht gesetztem Schalter.

\newif\ifkorrekturansicht
\korrekturansichtfalse

\input{../tex-inputs/latex-vorspann}


         
         \newcommand{\erwaehntePersonen}{Personen: Richard Beer-Hofmann, Paula Beer-Hofmann}
         \newcommand{\erwaehnteOrte}{Orte: Axenstein, Axenstraße, Flüelen, Hasenauerstraße, Hotel Axenstein, Morschach, Pilatus, Wien, XVIII., Währing}
         \newcommand{\erwaehnteWerke}{
               \section[Olga und Arthur Schnitzler an Richard und Paula Beer-Hofmann, 21. 5. 1910]{ Olga und Arthur Schnitzler an Richard und Paula Beer-Hofmann,
               21. 5. 1910}\nopagebreak\mylabel{v}\rehead{ }\begin{ledgroupsized}[t]{13cm}\normalsize\beginnumbering \toendnotes[C]{\smallbreak\pagebreak[2]} \Standort{YCGL, MSS 31.}
\physDesc{Bildpostkarte
\newline{}Handschrift Olga Schnitzler: Bleistift, lateinische Kurrent\newline{}Handschrift Arthur Schnitzler: Bleistift, deutsche Kurrent\newline{}Versand: Stempel: »\nobreak{}\oindex{Morschach@\textbf{Morschach}|pwk}Morschach, 21. V. 10\nobreak{}«.  }\toendnotes[C]{\smallbreak}\pstart{}{\pb}Herrn u. Frau\pwindex{Beer-Hofmann, Paula 25.02.1879 – 30.10.1939@\textsc{Beer-Hofmann, Paula} (25.02.1879 – 30.10.1939)|pwv}\pend{}\pstart{}D\textsuperscript{r} Richard Beer-Hofmann\pend{}\pstart{}Wien XVIII\oindex{XVIII., Waehring@\textbf{XVIII., Währing}|pw}\pend{}\pstart{}Hasenauerstrasse 59\oindex{Hasenauerstrasse@\textbf{Hasenauerstraße}|pw}\pend{}{\bigskip}\pstart
           \noindent{}\centering{}{\pb}\textcolor{gray}{\textbf{\textbf{Axenstein}\oindex{Axenstein@\textbf{Axenstein}|pw}, Ausblick gegen Pilatus\oindex{Pilatus@\textbf{Pilatus}|pw}}}\pend
           \pstart
           \noindent{}Die Karte ist lächerlich gegen die Wirklichkeit.\pend
           \pstart
           {\pb}\label{K_L02559-1v}\edtext{Hier}{\lemma{\textnormal{\emph{Hier}}}\Cendnote{\textnormal{Gemeint dürfte das Hotel
                     Axenstein\oindex{Hotel Axenstein@\textbf{Hotel Axenstein}|pwk} sein.}}}\label{K_L02559-1h} haben wir soeben zu Mittag gespeist, uns Zimmer
               angesehen, spazieren gegangen. Gehört schon zum Allerschönsten. Aber man kanns Euch
               nicht empfehlen weil Ihr dann sicher nie herkommt, und das wäre schade. Jetzt fahren
               wir gleich mit einem \label{K_L02559-2v}\edtext{Moto-wagerl}{\lemma{\textnormal{\emph{Moto-wagerl}}}\Cendnote{\textnormal{Automobil}}}\label{K_L02559-2h} nach Flüelen\oindex{Flueelen@\textbf{Flüelen}|pw}, Axenstrasse\oindex{Axenstrasse@\textbf{Axenstraße}|pw}!\pend
           \pstart Viele herzliche Grüsse!\spacefill\mbox{O.}\pend{}\pstart
           {\pb}Samstag, 21. Mai 1910.\pend
           \pstart {[}hs. Arthur Schnitzler:{]} Herzlichſt\spacefill\mbox{A.}\pend{}
         
         \endnumbering\mylabel{h}\end{ledgroupsized}  \newcommand{\dateiname}{L02559}\newcommand{\titel}{Olga und Arthur Schnitzler an Richard und Paula Beer-Hofmann, 21. 5. 1910}\newcommand{\editorInnen}{Martin Anton Müller und Gerd-Hermann Susen}%% latex-leseansicht-abspann.tex
%% Abspann für die Leseansicht.
%% Der Schalter \ifkorrekturansicht ist bereits durch den Vorspann gesetzt.

%% latex-abspann.tex
%% Gemeinsamer Abspann für Korrekturansicht und Leseansicht.
%% Setzt den Schalter \ifkorrekturansicht voraus (gesetzt in den
%% einbindenden Dateien latex-korrekturansicht-abspann.tex bzw.
%% latex-leseansicht-abspann.tex).
%% ---------------------------------------------------------------

\normalsize

% Das esempio-Environment wird nur in der Leseansicht benötigt
\ifkorrekturansicht\else
\newenvironment{esempio}[3]%
{
    \vspace{1.5ex}
    \rlap{\underline{#1}}
    \par
    \setlength{\parindent}{0cm}
    \nopagebreak
    \leftskip=#2cm
    \rightskip=#3cm
}
{
    \par
}
\fi

\doendnotes{C}
\bigskip
\vfill

\clearpage

\footnotesize

\ifkorrekturansicht
  \lohead{\textsc{register}}
\fi

% theindex-Environment neu definieren ohne reledmac
\makeatletter
\renewenvironment{theindex}{%
  \ifkorrekturansicht
    \section*{\indexname}%
  \else
    \subsubsection*{Index der erwähnten Entitäten}%
  \fi
  \setlength{\parindent}{0pt}%
  \setlength{\parskip}{0pt plus 0.3pt}%
  \let\item\@idxitem
}{%
  \ifkorrekturansicht\clearpage\fi
}
\makeatother

\IfFileExists{\jobname-pw.ind}{\input{\jobname-pw.ind}}{}

% Quellenangabe nur in der Leseansicht
\ifkorrekturansicht\else
% Fallback-Definitionen, falls die .tex-Datei \titel etc. nicht gesetzt hat
\providecommand{\titel}{}
\providecommand{\editorInnen}{}
\providecommand{\dateiname}{\jobname}

\vspace{3cm}

\vfill

\footnotesize
\textsc{Quelle}: \titel. Herausgegeben von {\editorInnen}. In: \emph{Arthur Schnitzler: Briefwechsel mit Autorinnen und Autoren}.
 Digitale Edition, https://schnitzler-briefe.acdh.oeaw.ac.at/{\dateiname}.html (Stand \today)
\fi

\end{document}


      