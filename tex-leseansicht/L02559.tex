%% latex-korrekturansicht-vorspann.tex
%% Vorspann für die Korrekturansicht.
%% Lädt die gemeinsame Datei latex-vorspann.tex mit gesetztem Schalter.

\newif\ifkorrekturansicht
\korrekturansichttrue

\input{../tex-inputs/latex-vorspann}


\section[Olga und Arthur Schnitzler an Richard und Paula Beer-Hofmann, 21. 5. 1910]{L02559 Olga und Arthur Schnitzler an Richard und Paula Beer-Hofmann,
               21. 5. 1910}
\nopagebreak\mylabel{L02559v}
\rehead{ }\normalsize\beginnumbering\briefempfaengerindex{Beer-Hofmann, Paula@\textsc{Beer-Hofmann, Paula}!zzzSchnitzler, Arthur@\emph{von Arthur Schnitzler}!1910-05-211@{21. 5. 1910}|(be}\briefempfaengerindex{Beer-Hofmann, Paula@\textsc{Beer-Hofmann, Paula}!zzzSchnitzler, Olga@\emph{von Olga Schnitzler}!1910-05-211@{21. 5. 1910}|(be}\briefempfaengerindex{Beer-Hofmann, Richard@\textsc{Beer-Hofmann, Richard}!zzzSchnitzler, Arthur@\emph{von Arthur Schnitzler}!1910-05-211@{21. 5. 1910}|(be}\briefempfaengerindex{Beer-Hofmann, Richard@\textsc{Beer-Hofmann, Richard}!zzzSchnitzler, Olga@\emph{von Olga Schnitzler}!1910-05-211@{21. 5. 1910}|(be}
\toendnotes[C]{\smallbreak\pagebreak[2]}\Standort{YCGL, MSS 31.}
\physDesc{Bildpostkarte, 446 Zeichen
\newline{}Handschrift Olga Schnitzler: Bleistift, lateinische Kurrent
\newline{}Handschrift Arthur Schnitzler: Bleistift, deutsche Kurrent
\newline{}Versand: Stempel: »\nobreak{}\oindex{Morschach@\textbf{Morschach}, \emph{A.ADM3}|pwk}Morschach, 21. V. 10\nobreak{}«.  }\toendnotes[C]{\smallbreak}\pstart{}{\pb}Herrn u. Frau\pwindex{Beer-Hofmann, Paula 25.02.1879 – 30.10.1939@\textsc{Beer-Hofmann, Paula} (25.02.1879 – 30.10.1939)|pwv}\pend{}\pstart{}D\textsuperscript{r} Richard Beer-Hofmann\pend{}\pstart{}Wien XVIII\oindex{XVIII., Waehring@\textbf{XVIII., Währing}, \emph{A.ADM3}|pw}\pend{}\pstart{}Hasenauerstrasse 59\oindex{Hasenauerstrasse 59@\textbf{Hasenauerstraße 59}, \emph{Wohngebäude (K.WHS)}|pw}\pend{}{\bigskip}
\pstart
           \noindent{}\centering{}{\pb}\textcolor{gray}{\textbf{\textbf{Axenstein}\oindex{Axenstein@\textbf{Axenstein}, \emph{Ausflugsziel}|pw}, Ausblick gegen Pilatus\oindex{Pilatus@\textbf{Pilatus}, \emph{Berg (N.BRG)}|pw}}}\pend
           \vspace{1em}
\pstart
           \noindent{}{\pb}\label{K_L02559-1v}\edtext{Hier}{\lemma{\textnormal{\emph{Hier}}}\Cendnote{\textnormal{Gemeint dürfte das Hotel
                     Axenstein\oindex{Hotel Axenstein@\textbf{Hotel Axenstein}, \emph{Hotel (K.HTL)}|pwk} sein.}}}\label{K_L02559-1} haben wir soeben zu Mittag gespeist, uns Zimmer
               angesehen, spazieren gegangen. Gehört schon zum Allerschönsten. Aber man kanns Euch
               nicht empfehlen weil Ihr dann sicher nie herkommt, und das wäre schade. Jetzt fahren
               wir gleich mit einem \label{K_L02559-2v}\edtext{Moto-wagerl}{\lemma{\textnormal{\emph{Moto-wagerl}}}\Cendnote{\textnormal{Automobil}}}\label{K_L02559-2} nach Flüelen\oindex{Flueelen@\textbf{Flüelen}, \emph{Besiedelter Ort (A.BSO)}|pw}, Axenstrasse\oindex{Axenstrasse@\textbf{Axenstraße}, \emph{Straße (K.STR)}|pw}!\pend
           \pstart Viele herzliche Grüsse!\spacefill\mbox{O.}\pend{}
\pstart
           {\pb}Samstag, 21. Mai 1910.\pend
           
\pstart
           Die Karte ist lächerlich gegen die Wirklichkeit.\pend
           \selectlanguage{ngerman}\vspace{1em}\pstart {[}hs. :{]} Herzlichſt\spacefill\mbox{A.}\pend{}\selectlanguage{ngerman}\endnumbering\briefempfaengerindex{Beer-Hofmann, Paula@\textsc{Beer-Hofmann, Paula}!zzzSchnitzler, Arthur@\emph{von Arthur Schnitzler}!1910-05-211@{21. 5. 1910}|)be}\briefempfaengerindex{Beer-Hofmann, Paula@\textsc{Beer-Hofmann, Paula}!zzzSchnitzler, Olga@\emph{von Olga Schnitzler}!1910-05-211@{21. 5. 1910}|)be}\briefempfaengerindex{Beer-Hofmann, Richard@\textsc{Beer-Hofmann, Richard}!zzzSchnitzler, Arthur@\emph{von Arthur Schnitzler}!1910-05-211@{21. 5. 1910}|)be}\briefempfaengerindex{Beer-Hofmann, Richard@\textsc{Beer-Hofmann, Richard}!zzzSchnitzler, Olga@\emph{von Olga Schnitzler}!1910-05-211@{21. 5. 1910}|)be}\mylabel{L02559h}  \normalsize

\doendnotes{C}
\bigskip
\vfill

\clearpage

\footnotesize

\lohead{\textsc{register}}

% Definiere theindex-Environment komplett neu ohne reledmac
\makeatletter
\renewenvironment{theindex}{%
  \section*{\indexname}%
  \setlength{\parindent}{0pt}%
  \setlength{\parskip}{0pt plus 0.3pt}%
  \let\item\@idxitem
}{%
  \clearpage
}
\makeatother

\IfFileExists{\jobname-pw.ind}{\input{\jobname-pw.ind}}{}

\end{document}

      