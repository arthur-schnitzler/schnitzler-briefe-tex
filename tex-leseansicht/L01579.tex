%% latex-leseansicht-vorspann.tex
%% Vorspann für die Leseansicht.
%% Lädt die gemeinsame Datei latex-vorspann.tex mit nicht gesetztem Schalter.

\newif\ifkorrekturansicht
\korrekturansichtfalse

\input{../tex-inputs/latex-vorspann}


               \section[Arthur Schnitzler an Charlotte Ehrenstein, 29. 1. 1906]{ Arthur Schnitzler an Charlotte Ehrenstein,
                    29. 1. 1906}\nopagebreak\mylabel{v}\rehead{ }\begin{ledgroupsized}[t]{13cm}\normalsize\beginnumbering\briefempfaengerindex{Ehrenstein, Charlotte@\textsc{Ehrenstein, Charlotte}!zzzSchnitzler, Arthur@\emph{von Arthur Schnitzler}!1906-01-293@{29. 1. 1906}|(be} \toendnotes[C]{\smallbreak\pagebreak[2]} \Standort{Jerusalem, The National Library of Israel, ARC. Ms. Var. 306 1 118.}
\physDesc{Briefkarte
\newline{}Handschrift: schwarze Tinte, deutsche Kurrent\newline{}Ordnung: mit Bleistift von unbekannter Hand nummeriert »3« }\toendnotes[C]{\smallbreak}\pstart
           \noindent{}{\pb}\textcolor{gray}{\textbf{Dr. Arthur Schnitzler}}{\\}\textcolor{gray}{\textbf{Wien, XVIII. Spoettelgasse 7\oindex{Edmund-Weiss-Gasse@\textbf{Edmund-Weiß-Gasse}|pw}.}}\hfill 29. 1. 906. \pend
           \pstart
           Sehr geehrte gnäd\textcolor{gray}{i}ge Frau, ich danke Ihnen u
                    Herrn Treibel\pwindex{Treibl, Adolf 02.03.1865 – 1935@\textsc{Treibl, Adolf} (02.03.1865 – 1935), \emph{Fabrikant}|pw} für die freundlichen
                    Nachrichten; hoffentlich erfahr’ ich bald vollko{\geminationm}en
                    günſtiges über das Befinden Ihres Sohn\textcolor{gray}{e}s\pwindex{Ehrenstein, Albert 23.12.1886 – 08.04.1950@\textsc{Ehrenstein, Albert} (23.12.1886 – 08.04.1950), \emph{Schriftsteller}|pwv}, den ich {\pb}beſtens zu grüßen bitte.\pend
           \pstart
           Mit Empfehlungen an Sie gnädige Frau und Ihren Gatten\pwindex{Ehrenstein, Alexander 29.03.1857 – 29.05.1925@\textsc{Ehrenstein, Alexander} (29.03.1857 – 29.05.1925), \emph{Kassier}|pwv}\pend
           \pstart
           ganz ergebens{\\[\baselineskip]}\spacefill\mbox{ArtSch.}\pend
           \leftskip=0em{}          \endnumbering\briefempfaengerindex{Ehrenstein, Charlotte@\textsc{Ehrenstein, Charlotte}!zzzSchnitzler, Arthur@\emph{von Arthur Schnitzler}!1906-01-293@{29. 1. 1906}|)be}\mylabel{h}\end{ledgroupsized}  \newcommand{\dateiname}{L01579}\newcommand{\titel}{Arthur Schnitzler an Charlotte Ehrenstein, 29. 1. 1906}\newcommand{\editorInnen}{Martin Anton Müller und Gerd-Hermann Susen}%% latex-leseansicht-abspann.tex
%% Abspann für die Leseansicht.
%% Der Schalter \ifkorrekturansicht ist bereits durch den Vorspann gesetzt.

%% latex-abspann.tex
%% Gemeinsamer Abspann für Korrekturansicht und Leseansicht.
%% Setzt den Schalter \ifkorrekturansicht voraus (gesetzt in den
%% einbindenden Dateien latex-korrekturansicht-abspann.tex bzw.
%% latex-leseansicht-abspann.tex).
%% ---------------------------------------------------------------

\normalsize

% Das esempio-Environment wird nur in der Leseansicht benötigt
\ifkorrekturansicht\else
\newenvironment{esempio}[3]%
{
    \vspace{1.5ex}
    \rlap{\underline{#1}}
    \par
    \setlength{\parindent}{0cm}
    \nopagebreak
    \leftskip=#2cm
    \rightskip=#3cm
}
{
    \par
}
\fi

\doendnotes{C}
\bigskip
\vfill

\clearpage

\footnotesize

\ifkorrekturansicht
  \lohead{\textsc{register}}
\fi

% theindex-Environment neu definieren ohne reledmac
\makeatletter
\renewenvironment{theindex}{%
  \ifkorrekturansicht
    \section*{\indexname}%
  \else
    \subsubsection*{Index der erwähnten Entitäten}%
  \fi
  \setlength{\parindent}{0pt}%
  \setlength{\parskip}{0pt plus 0.3pt}%
  \let\item\@idxitem
}{%
  \ifkorrekturansicht\clearpage\fi
}
\makeatother

\IfFileExists{\jobname-pw.ind}{\input{\jobname-pw.ind}}{}

% Quellenangabe nur in der Leseansicht
\ifkorrekturansicht\else
% Fallback-Definitionen, falls die .tex-Datei \titel etc. nicht gesetzt hat
\providecommand{\titel}{}
\providecommand{\editorInnen}{}
\providecommand{\dateiname}{\jobname}

\vspace{3cm}

\vfill

\footnotesize
\textsc{Quelle}: \titel. Herausgegeben von {\editorInnen}. In: \emph{Arthur Schnitzler: Briefwechsel mit Autorinnen und Autoren}.
 Digitale Edition, https://schnitzler-briefe.acdh.oeaw.ac.at/{\dateiname}.html (Stand \today)
\fi

\end{document}


      