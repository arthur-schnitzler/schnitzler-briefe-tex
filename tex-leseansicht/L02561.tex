%% latex-korrekturansicht-vorspann.tex
%% Vorspann für die Korrekturansicht.
%% Lädt die gemeinsame Datei latex-vorspann.tex mit gesetztem Schalter.

\newif\ifkorrekturansicht
\korrekturansichttrue

\input{../tex-inputs/latex-vorspann}


\section[Olga Schnitzler an Paula Beer-Hofmann, {[}19. 1. 1910{]}]{L02561 Olga Schnitzler an Paula Beer-Hofmann, {[}19. 1. 1910{]}}
\nopagebreak\mylabel{L02561v}
\rehead{ }\normalsize\beginnumbering\briefempfaengerindex{Beer-Hofmann, Paula@\textsc{Beer-Hofmann, Paula}!zzzSchnitzler, Olga@\emph{von Olga Schnitzler}!1910-01-191@{{[}19. 1. 1910{]}}|(be}
\toendnotes[C]{\smallbreak\pagebreak[2]}\Standort{YCGL, MSS 31.}
\physDesc{Briefkarte, , Umschlag, 98 Zeichen
\newline{}Handschrift: schwarze Tinte, lateinische Kurrent
\newline{}Versand: ohne postalischen Übermittlungsvermerk 
\newline{}Ordnung: auf der Umschlagrückseite von unbekannter Hand mit Bleistift
                                 datiert: »{\pb}19/I 1910« }\pstart{}{\pb}Frau Paula Beer-Hofmann\pend{}\pstart{}XIX Hasenauerg. 59\oindex{Hasenauerstrasse 59@\textbf{Hasenauerstraße 59}, \emph{Wohngebäude (K.WHS)}|pw}.\pend{}{\bigskip}\vspace{1em}
\pstart
           \noindent{}{\pb}\textcolor{gray}{\textbf{O.S.}}\pend
           
\pstart
           Dies, liebe Paula, ist Frl. Reiter\pwindex{Reiter, Anna @\textsc{Reiter, Anna}, \emph{Hausschneider/Hausschneiderin}|pw}!\pend
           \pstart Herzliche Grüsse!\spacefill\mbox{Olga.}\pend{}\selectlanguage{ngerman}\endnumbering\briefempfaengerindex{Beer-Hofmann, Paula@\textsc{Beer-Hofmann, Paula}!zzzSchnitzler, Olga@\emph{von Olga Schnitzler}!1910-01-191@{{[}19. 1. 1910{]}}|)be}\mylabel{L02561h}  \normalsize

\doendnotes{C}
\bigskip
\vfill

\clearpage

\footnotesize

\lohead{\textsc{register}}

% Definiere theindex-Environment komplett neu ohne reledmac
\makeatletter
\renewenvironment{theindex}{%
  \section*{\indexname}%
  \setlength{\parindent}{0pt}%
  \setlength{\parskip}{0pt plus 0.3pt}%
  \let\item\@idxitem
}{%
  \clearpage
}
\makeatother

\IfFileExists{\jobname-pw.ind}{\input{\jobname-pw.ind}}{}

\end{document}

      