%% latex-leseansicht-vorspann.tex
%% Vorspann für die Leseansicht.
%% Lädt die gemeinsame Datei latex-vorspann.tex mit nicht gesetztem Schalter.

\newif\ifkorrekturansicht
\korrekturansichtfalse

\input{../tex-inputs/latex-vorspann}


\section[Hugo von Hofmannsthal an Arthur Schnitzler, 31. 3. [1915]]{L02205 Hugo von Hofmannsthal an Arthur Schnitzler, 31. 3. [1915]}
\nopagebreak\mylabel{L02205v}
\rehead{ }\normalsize\beginnumbering\briefempfaengerindex{Schnitzler, Arthur@\textsc{Schnitzler, Arthur}!zzzHofmannsthal, Hugo von@\emph{von Hugo von Hofmannsthal}!1915-03-311@{31. 3. [1915]}|(be}
\toendnotes[C]{\smallbreak\pagebreak[2]}
\correspDesc{Versand  durch Hugo von Hofmannsthal am 31. 3. [1915] in Wien
\newline{}Erhalt  durch Arthur Schnitzler im Zeitraum [31. 3. 1915
                  – 4. 4. 1915?] in Wien}\toendnotes[C]{\smallbreak}
\Standort{CUL, Schnitzler, B 43.}
\physDesc{Brief, 1 Blatt, 2 Seiten, 870 Zeichen
\newline{}Handschrift: blaue Tinte, deutsche Kurrent
\newline{}Schnitzler: 1) mit Bleistift beschriftet: »\textsc{Hugo}« und eine Jahreszahl ergänzt: »19\textcolor{gray}{1}5«  2) mit rotem Buntstift eine Unterstreichung
\newline{}Ordnung: mit Bleistift von unbekannter Hand nummeriert:
                                    »389«, nachdem zwei weitere Nummern unleserlich
                                 gemacht wurden, und erneut mit einer Jahreszahl versehen: »1925?« }
\buchAbdrucke{\weitereDrucke{Hugo von Hofmannsthal, Arthur Schnitzler: \emph{Briefwechsel}. Herausgegeben von Therese Nickl und Heinrich Schnitzler. Frankfurt am Main: \emph{S. Fischer} 1964, S. 277.} }\toendnotes[C]{\smallbreak}
\pstart
           \raggedleft{}{\pb}31. III\pend
           
\pstart{}mein lieber Arthur\pend\vspace{0.5em}
\pstart
           ich bitte Sie,{ }ſagen Sie mir den Namen eines Ihres Erachtens guten Nervenarztes (\textsc{Psychiaters}) mit dem ich vertrauensvoll über meine wirklich
               abſurden Nerven{ }ſprechen könnte. – Zugleich müßte es aber jemand{ }ſein, der auch für’s
               Militär eine \uline{Autorität} wäre, womöglich{ }ſelbſt im
               Dienſte,{ }ſo daſs{ }ſein Gutachten eventuell \substVorne{}\textsuperscript{die}\substDazwischen{}zur\substHinten{} Anbahnung eines längeren Krankheitsurlaubes bei einer (ſehr wohlwollenden)
               Militärſtelle dienen könnte.\pend
           
\pstart
           Wenn es endlich jemand wäre mit dem Sie \strikeout{und} oder Julius\pwindex{Schnitzler, Julius 13.\,7.\,1865 Wien – 29.\,6.\,1939 ebd.@\textsc{Schnitzler, Julius} (13.\,7.\,1865 Wien – 29.\,6.\,1939 ebd.), \emph{Chirurg}|pw} in irgendwelcher Beziehung{ }ſind wäre es
               umſo beſſer, doch iſt dies minder wichtig. Bitte{ }ſprechen Sie allenfalls mit Julius\pwindex{Schnitzler, Julius 13.\,7.\,1865 Wien – 29.\,6.\,1939 ebd.@\textsc{Schnitzler, Julius} (13.\,7.\,1865 Wien – 29.\,6.\,1939 ebd.), \emph{Chirurg}|pw} und{ }ſchreiben mir den Namen möglichſt
               bald expreſs {\pb}nach Rodaun\oindex{Wien@\textbf{Wien}!XXIII., Liesing@\textbf{XXIII., Liesing}!Rodaun@\textbf{Rodaun}, \emph{Region}|pw}.\pend
           
\pstart
           Papa\pwindex{Hofmannsthal, Hugo August von 21.\,12.\,1841 Wien – 8.\,12.\,1915 ebd.@\textsc{Hofmannsthal, Hugo August von} (21.\,12.\,1841 Wien – 8.\,12.\,1915 ebd.), \emph{Bankdirektor}|pw} hat{ }ſich mit Ihrem \label{K_L02205-1v}\edtext{Beſuch}{\lemma{\textnormal{\emph{Besuch}}}\Cendnote{\textnormal{Am 16. 3. 1915; Schnitzler
                  wiederholte den Besuch am 1. 4. 1915, was als impliziter Hinweis genommen werden kann,
                  dass er diesen Brief zu dem Zeitpunkt bereits erhalten hatte.}}}\label{K_L02205-1}{ }ſo{ }ſehr gefreut. Vielleicht wiederholen Sie ihn
               noch einmal! Es wäre{ }ſehr lieb.\pend
           
\pstart
           Erwähnen Sie in dem Brief doch bitte auch ob Ihr über Oſtern{ }\uline{hier}{ }ſeid.\pend
           
\pstart
           Ihr{\\[\baselineskip]}\spacefill\mbox{Hugo.}\pend
           \leftskip=0em{}\selectlanguage{ngerman}\endnumbering\briefempfaengerindex{Schnitzler, Arthur@\textsc{Schnitzler, Arthur}!zzzHofmannsthal, Hugo von@\emph{von Hugo von Hofmannsthal}!1915-03-311@{31. 3. [1915]}|)be}\mylabel{L02205h}  \newcommand{\dateiname}{L02205}\newcommand{\titel}{Hugo von Hofmannsthal an Arthur Schnitzler, 31. 3. [1915]}\newcommand{\editorInnen}{Martin Anton Müller und Gerd-Hermann Susen}%% latex-leseansicht-abspann.tex
%% Abspann für die Leseansicht.
%% Der Schalter \ifkorrekturansicht ist bereits durch den Vorspann gesetzt.

%% latex-abspann.tex
%% Gemeinsamer Abspann für Korrekturansicht und Leseansicht.
%% Setzt den Schalter \ifkorrekturansicht voraus (gesetzt in den
%% einbindenden Dateien latex-korrekturansicht-abspann.tex bzw.
%% latex-leseansicht-abspann.tex).
%% ---------------------------------------------------------------

\normalsize

% Das esempio-Environment wird nur in der Leseansicht benötigt
\ifkorrekturansicht\else
\newenvironment{esempio}[3]%
{
    \vspace{1.5ex}
    \rlap{\underline{#1}}
    \par
    \setlength{\parindent}{0cm}
    \nopagebreak
    \leftskip=#2cm
    \rightskip=#3cm
}
{
    \par
}
\fi

\doendnotes{C}
\bigskip
\vfill

\clearpage

\footnotesize

\ifkorrekturansicht
  \lohead{\textsc{register}}
\fi

% theindex-Environment neu definieren ohne reledmac
\makeatletter
\renewenvironment{theindex}{%
  \ifkorrekturansicht
    \section*{\indexname}%
  \else
    \subsubsection*{Index der erwähnten Entitäten}%
  \fi
  \setlength{\parindent}{0pt}%
  \setlength{\parskip}{0pt plus 0.3pt}%
  \let\item\@idxitem
}{%
  \ifkorrekturansicht\clearpage\fi
}
\makeatother

\IfFileExists{\jobname-pw.ind}{\input{\jobname-pw.ind}}{}

% Quellenangabe nur in der Leseansicht
\ifkorrekturansicht\else
% Fallback-Definitionen, falls die .tex-Datei \titel etc. nicht gesetzt hat
\providecommand{\titel}{}
\providecommand{\editorInnen}{}
\providecommand{\dateiname}{\jobname}

\vspace{3cm}

\vfill

\footnotesize
\textsc{Quelle}: \titel. Herausgegeben von {\editorInnen}. In: \emph{Arthur Schnitzler: Briefwechsel mit Autorinnen und Autoren}.
 Digitale Edition, https://schnitzler-briefe.acdh.oeaw.ac.at/{\dateiname}.html (Stand \today)
\fi

\end{document}


