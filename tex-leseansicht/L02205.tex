%% latex-leseansicht-vorspann.tex
%% Vorspann für die Leseansicht.
%% Lädt die gemeinsame Datei latex-vorspann.tex mit nicht gesetztem Schalter.

\newif\ifkorrekturansicht
\korrekturansichtfalse

\input{../tex-inputs/latex-vorspann}


         
         \renewcommand{\erwaehntePersonen}{Personen: Hugo August von Hofmannsthal, Julius Schnitzler}
         \renewcommand{\erwaehnteOrte}{Orte: Rodaun, Wien}
         \renewcommand{\erwaehnteWerke}{}
               \section[Hugo von Hofmannsthal an Arthur Schnitzler, 31. 3. {[}1915{]}]{ Hugo von Hofmannsthal an Arthur Schnitzler, 31. 3. {[}1915{]}}\nopagebreak\mylabel{v}\rehead{ }\begin{ledgroupsized}[t]{13cm}\normalsize\beginnumbering \toendnotes[C]{\smallbreak\pagebreak[2]} \Standort{CUL, Schnitzler, B 43.}
\physDesc{Brief, 1 Blatt, 2 Seiten
\newline{}Handschrift: blaue Tinte, deutsche Kurrent
\newline{}Schnitzler: 1) mit Bleistift beschriftet: »\textsc{Hugo}« und eine Jahreszahl ergänzt: »19\textcolor{gray}{1}5«  2) mit rotem Buntstift eine Unterstreichung\newline{}Ordnung: mit Bleistift von unbekannter Hand nummeriert:
                                        »389«, nachdem zwei weitere Nummern
                                    unleserlich gemacht wurden, und erneut mit einer Jahreszahl
                                    versehen: »1925?« }\buchAbdrucke{\weitereDrucke{Hugo von Hofmannsthal, Arthur Schnitzler: \emph{Briefwechsel}. Hg. Therese Nickl und Heinrich Schnitzler. Frankfurt am Main: \emph{S. Fischer} 1964, S. 277.} }\toendnotes[C]{\smallbreak}\pstart
           \raggedleft{}{\pb}31. III\pend
           \pstart{}mein lieber Arthur\pend\pstart
           ich bitte Sie, ſagen Sie mir den Namen eines Ihres Erachtens guten Nervenarztes
                        (\textsc{Psychiaters}) mit dem ich vertrauensvoll über
                    meine wirklich abſurden Nerven ſprechen könnte. – Zugleich müßte es aber jemand
                    ſein, der auch für’s Militär eine \uline{Autorität}
                    wäre, womöglich ſelbſt im Dienſte, ſo daſs ſein Gutachten eventuell \substVorne{}\textsuperscript{die}\substDazwischen{}zur\substHinten{} Anbahnung eines längeren Krankheitsurlaubes bei einer (ſehr
                    wohlwollenden) Militärſtelle dienen könnte.\pend
           \pstart
           Wenn es endlich jemand wäre mit dem Sie \strikeout{und} oder
                        Julius\pwindex{Schnitzler, Julius 13.07.1865 – 29.06.1939@\textsc{Schnitzler, Julius} (13.07.1865 – 29.06.1939), \emph{Chirurg}|pw} in irgendwelcher Beziehung ſind
                    wäre es umſo beſſer, doch iſt dies minder wichtig. Bitte ſprechen Sie allenfalls
                    mit Julius\pwindex{Schnitzler, Julius 13.07.1865 – 29.06.1939@\textsc{Schnitzler, Julius} (13.07.1865 – 29.06.1939), \emph{Chirurg}|pw} und ſchreiben mir den Namen
                    möglichſt bald expreſs {\pb}nach
                        Rodaun\oindex{Rodaun@\textbf{Rodaun}|pw}.\pend
           \pstart
           Papa\pwindex{Hofmannsthal, Hugo August von 21.12.1841 – 08.12.1915@\textsc{Hofmannsthal, Hugo August von} (21.12.1841 – 08.12.1915), \emph{Bankdirektor}|pw} hat ſich mit Ihrem \label{K_L02205_1v}\edtext{Beſuch}{\lemma{\textnormal{\emph{Beſuch}}}\Cendnote{\textnormal{am 16. 3. 1915; Schnitzler\pwindex{Schnitzler, Arthur 15.05.1862 – 21.10.1931@\textsc{Schnitzler, Arthur} (15.05.1862 – 21.10.1931), \emph{Schriftsteller, Mediziner}|pwk} wiederholte ihn am
                            1. 4. 1915, was als impliziter Hinweis genommen werden kann,
                        dass er diesen Brief zu dem Zeitpunkt bereits erhalten hatte.}}}\label{K_L02205_1h}{ }ſo ſehr gefreut. Vielleicht wiederholen Sie
                    ihn noch einmal! Es wäre ſehr lieb.\pend
           \pstart
           Erwähnen Sie in dem Brief doch bitte auch ob Ihr über Oſtern{ }\uline{hier}{ }ſeid.\pend
           \pstart
           Ihr{\\[\baselineskip]}\spacefill\mbox{Hugo.}\pend
           \leftskip=0em{}
         
         \endnumbering\mylabel{h}\end{ledgroupsized}  \newcommand{\dateiname}{L02205}\newcommand{\titel}{Hugo von Hofmannsthal an Arthur Schnitzler, 31. 3. [1915]}\newcommand{\editorInnen}{Martin Anton Müller und Gerd-Hermann Susen}%% latex-leseansicht-abspann.tex
%% Abspann für die Leseansicht.
%% Der Schalter \ifkorrekturansicht ist bereits durch den Vorspann gesetzt.

%% latex-abspann.tex
%% Gemeinsamer Abspann für Korrekturansicht und Leseansicht.
%% Setzt den Schalter \ifkorrekturansicht voraus (gesetzt in den
%% einbindenden Dateien latex-korrekturansicht-abspann.tex bzw.
%% latex-leseansicht-abspann.tex).
%% ---------------------------------------------------------------

\normalsize

% Das esempio-Environment wird nur in der Leseansicht benötigt
\ifkorrekturansicht\else
\newenvironment{esempio}[3]%
{
    \vspace{1.5ex}
    \rlap{\underline{#1}}
    \par
    \setlength{\parindent}{0cm}
    \nopagebreak
    \leftskip=#2cm
    \rightskip=#3cm
}
{
    \par
}
\fi

\doendnotes{C}
\bigskip
\vfill

\clearpage

\footnotesize

\ifkorrekturansicht
  \lohead{\textsc{register}}
\fi

% theindex-Environment neu definieren ohne reledmac
\makeatletter
\renewenvironment{theindex}{%
  \ifkorrekturansicht
    \section*{\indexname}%
  \else
    \subsubsection*{Index der erwähnten Entitäten}%
  \fi
  \setlength{\parindent}{0pt}%
  \setlength{\parskip}{0pt plus 0.3pt}%
  \let\item\@idxitem
}{%
  \ifkorrekturansicht\clearpage\fi
}
\makeatother

\IfFileExists{\jobname-pw.ind}{\input{\jobname-pw.ind}}{}

% Quellenangabe nur in der Leseansicht
\ifkorrekturansicht\else
% Fallback-Definitionen, falls die .tex-Datei \titel etc. nicht gesetzt hat
\providecommand{\titel}{}
\providecommand{\editorInnen}{}
\providecommand{\dateiname}{\jobname}

\vspace{3cm}

\vfill

\footnotesize
\textsc{Quelle}: \titel. Herausgegeben von {\editorInnen}. In: \emph{Arthur Schnitzler: Briefwechsel mit Autorinnen und Autoren}.
 Digitale Edition, https://schnitzler-briefe.acdh.oeaw.ac.at/{\dateiname}.html (Stand \today)
\fi

\end{document}


      