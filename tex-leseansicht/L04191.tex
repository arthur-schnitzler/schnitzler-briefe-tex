%% latex-leseansicht-vorspann.tex
%% Vorspann für die Leseansicht.
%% Lädt die gemeinsame Datei latex-vorspann.tex mit nicht gesetztem Schalter.

\newif\ifkorrekturansicht
\korrekturansichtfalse

\input{../tex-inputs/latex-vorspann}


\section[Arthur Schnitzler an Gustav Schwarzkopf, 2. 4. 1899]{L04191 Arthur Schnitzler an Gustav Schwarzkopf, 2. 4. 1899}
\nopagebreak\mylabel{L04191v}
\rehead{ }\normalsize\beginnumbering\briefempfaengerindex{Schwarzkopf, Gustav@\textsc{Schwarzkopf, Gustav}!zzzSchnitzler, Arthur@\emph{von Arthur Schnitzler}!1899-04-021@{2. 4. 1899}|(be}
\toendnotes[C]{\smallbreak\pagebreak[2]}
\correspDesc{Versand  durch Arthur Schnitzler am 2. 4. 1899 in Mürzzuschlag
\newline{}Erhalt  durch Gustav Schwarzkopf am 2. 4. 1899 in Wien}\toendnotes[C]{\smallbreak}
\Standort{CUL, Schnitzler, B 96.}
\physDesc{Telegramm, 1 Blatt, 180 Zeichen
\newline{}Handschrift: lila Tinte, deutsche Kurrent
\newline{}Versand: 1) »\textcolor{gray}{\textbf{Aufgegeben am}}{ }2/4«  2) »\noindent{}\textcolor{gray}{\textbf{Eingelangt von ..........}}{ / }\textcolor{gray}{\textbf{Auf Leitung Nr.}} 738{ / }\textcolor{gray}{\textbf{am}}{ }2/4 \textcolor{gray}{\textbf{189}}9{ }\textcolor{gray}{\textbf{um {\dots} Uhr {\dots} Min. {\dots} Mittag}}{ / }\textcolor{gray}{\textbf{Aufgenommen durch}} Mg
                                          di\textcolor{gray}{y}« 3) Stempel: »\nobreak{}2 Apr {[}1899{]}, Ausgefertigt\nobreak{}«. }\toendnotes[C]{\smallbreak}\pstart{}{\pb}\textsc{Gustav Schwarzkopf}\pend{}\pstart{}\textsc{I. Tiefer Graben 23}\oindex{Wien@\textbf{Wien}!I., Innere Stadt@\textbf{I., Innere Stadt}!Tiefer Graben 23@\textbf{Tiefer Graben 23}, \emph{Wohngebäude}|pw}\pend{}\pstart{}Wien\oindex{Wien@\textbf{Wien}, \emph{Verwaltungsgebiet}|pw}\pend{}{\bigskip}\vspace{1em}
\pstart
           {\pb}\textcolor{gray}{\textbf{Von}}{ }Mürzzuſchlag\oindex{Mürzzuschlag@\textbf{Mürzzuschlag}, \emph{Verwaltungsgebiet}|pw}\pend
           
\pstart
           \textcolor{gray}{\textbf{Aufgabe-Nr.}} 37 \textcolor{gray}{\textbf{mit}} 28 \textcolor{gray}{\textbf{Taxworten}}\pend
           \vspace{0.5em}
\pstart
           Schauerliches Wetter fahre morgen früh direct Wien\oindex{Wien@\textbf{Wien}, \emph{Verwaltungsgebiet}|pw}
               Bitte \label{K_L04191-1v}\edtext{kommen Sie nachmittags}{\lemma{\textnormal{\emph{kommen Sie nachmittags}}}\Cendnote{\textnormal{Er kam, vgl. A. S.: \emph{Tagebuch}, 3. 4. 1899.}}}\label{K_L04191-1} zu mir wenn möglich recht früh
               herzlichst\pend
           \pstart \spacefill\mbox{Arthur}\pend{}\selectlanguage{ngerman}\endnumbering\briefempfaengerindex{Schwarzkopf, Gustav@\textsc{Schwarzkopf, Gustav}!zzzSchnitzler, Arthur@\emph{von Arthur Schnitzler}!1899-04-021@{2. 4. 1899}|)be}\mylabel{L04191h}
\begin{anhang}
\end{anhang}\newcommand{\dateiname}{L04191}\newcommand{\titel}{Arthur Schnitzler an Gustav Schwarzkopf, 2. 4. 1899}\newcommand{\editorInnen}{Herausgegeben von Jahnke, SelmaMüller, Martin Anton}%% latex-leseansicht-abspann.tex
%% Abspann für die Leseansicht.
%% Der Schalter \ifkorrekturansicht ist bereits durch den Vorspann gesetzt.

%% latex-abspann.tex
%% Gemeinsamer Abspann für Korrekturansicht und Leseansicht.
%% Setzt den Schalter \ifkorrekturansicht voraus (gesetzt in den
%% einbindenden Dateien latex-korrekturansicht-abspann.tex bzw.
%% latex-leseansicht-abspann.tex).
%% ---------------------------------------------------------------

\normalsize

% Das esempio-Environment wird nur in der Leseansicht benötigt
\ifkorrekturansicht\else
\newenvironment{esempio}[3]%
{
    \vspace{1.5ex}
    \rlap{\underline{#1}}
    \par
    \setlength{\parindent}{0cm}
    \nopagebreak
    \leftskip=#2cm
    \rightskip=#3cm
}
{
    \par
}
\fi

\doendnotes{C}
\bigskip
\vfill

\clearpage

\footnotesize

\ifkorrekturansicht
  \lohead{\textsc{register}}
\fi

% theindex-Environment neu definieren ohne reledmac
\makeatletter
\renewenvironment{theindex}{%
  \ifkorrekturansicht
    \section*{\indexname}%
  \else
    \subsubsection*{Index der erwähnten Entitäten}%
  \fi
  \setlength{\parindent}{0pt}%
  \setlength{\parskip}{0pt plus 0.3pt}%
  \let\item\@idxitem
}{%
  \ifkorrekturansicht\clearpage\fi
}
\makeatother

\IfFileExists{\jobname-pw.ind}{\input{\jobname-pw.ind}}{}

% Quellenangabe nur in der Leseansicht
\ifkorrekturansicht\else
% Fallback-Definitionen, falls die .tex-Datei \titel etc. nicht gesetzt hat
\providecommand{\titel}{}
\providecommand{\editorInnen}{}
\providecommand{\dateiname}{\jobname}

\vspace{3cm}

\vfill

\footnotesize
\textsc{Quelle}: \titel. Herausgegeben von {\editorInnen}. In: \emph{Arthur Schnitzler: Briefwechsel mit Autorinnen und Autoren}.
 Digitale Edition, https://schnitzler-briefe.acdh.oeaw.ac.at/{\dateiname}.html (Stand \today)
\fi

\end{document}


