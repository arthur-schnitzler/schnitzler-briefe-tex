%% latex-korrekturansicht-vorspann.tex
%% Vorspann für die Korrekturansicht.
%% Lädt die gemeinsame Datei latex-vorspann.tex mit gesetztem Schalter.

\newif\ifkorrekturansicht
\korrekturansichttrue

\input{../tex-inputs/latex-vorspann}


\section[Arthur und Olga Schnitzler an Hugo von Hofmannsthal, 28. 12. 1904]{L01485 Arthur und Olga Schnitzler an Hugo von Hofmannsthal,
               28. 12. 1904}
\nopagebreak\mylabel{L01485v}
\rehead{ }\normalsize\beginnumbering\briefempfaengerindex{Hofmannsthal, Hugo von@\textsc{Hofmannsthal, Hugo von}!zzzSchnitzler, Olga@\emph{von Olga Schnitzler}!1904-12-281@{28. 12. 1904}|(be}\briefempfaengerindex{Hofmannsthal, Hugo von@\textsc{Hofmannsthal, Hugo von}!zzzSchnitzler, Arthur@\emph{von Arthur Schnitzler}!1904-12-281@{28. 12. 1904}|(be}
\toendnotes[C]{\smallbreak\pagebreak[2]}\Standort{FDH, Hs-30885,118.}
\physDesc{Bildpostkarte, 171 Zeichen
\newline{}Handschrift Arthur Schnitzler: 1) Bleistift, deutsche Kurrent\hspace{1em}2) Bleistift, lateinische Kurrent (\noindent{}Adresse)\hspace{1em}
\newline{}Handschrift Olga Schnitzler: Bleistift
\newline{}Versand: 1) Stempel: »\nobreak{}\oindex{Salzburg@\textbf{Salzburg}, \emph{A.ADM2}|pwk}St{[}. Gilgen{]}\nobreak{}«.   2) Stempel: »\nobreak{}\oindex{Rodaun@\textbf{Rodaun}, \emph{A.ADM4}|pwk}Ro{[}dau{]}n, 29. 12. 04\nobreak{}«. }
\buchAbdrucke{\weitereDrucke{Hugo von Hofmannsthal, Arthur Schnitzler: \emph{Briefwechsel}. Frankfurt am Main: \emph{S. Fischer} 1964, S. 208.} }\toendnotes[C]{\smallbreak}\pstart{}{\pb}Herrn Hugo v. Hofmannsthal\pend{}\pstart{}Rodaun b (Wien)\oindex{Rodaun@\textbf{Rodaun}, \emph{A.ADM4}|pw}\pend{}\pstart{}Badgasse 5\oindex{Badgasse@\textbf{Badgasse}, \emph{Straße (K.STR)}|pw}\pend{}{\bigskip}
\pstart
           \noindent{}\centering{}{\pb}\textcolor{gray}{\textbf{St. Gilgen\oindex{St. Gilgen@\textbf{St. Gilgen}, \emph{A.ADM3}|pw}}}\pend
           
\pstart
           \centering{}\textcolor{gray}{\textbf{Salzburg\oindex{Salzburg [Land]@\textbf{Salzburg [Land]}, \emph{A.ADM1}|pw}}}\pend
           \vspace{1em}
\pstart
           \noindent{}{\pb}lieber Hugo, wir möchten Sie \label{K_L01485-1v}\edtext{Montag}{\lemma{\textnormal{\emph{Montag}}}\Cendnote{\textnormal{Siehe A. S.: \emph{Tagebuch}, 2. 1. 1905. }}}\label{K_L01485-1}{ }\introOben{}den 2.\introOben{}{ }Abend 8{ }Hietzing\oindex{XIII., Hietzing@\textbf{XIII., Hietzing}, \emph{A.ADM3}|pw}{ }ſehen.\pend
           
\pstart
           Antwort Wien\oindex{Wien@\textbf{Wien}, \emph{A.ADM2}|pw} erbeten.\pend
           \pstart Ihr \spacefill\mbox{A.}\pend{}
\pstart
           \noindent{}Ich \label{K_L01485-2v}\edtext{ſchreibe auch an Richard\pwindex{Beer-Hofmann, Richard 1866-07-11 – 1945-09-26@\textsc{Beer-Hofmann, Richard} (1866-07-11 – 1945-09-26), \emph{Schriftsteller/Schriftstellerin}|pw}}{\lemma{\textnormal{\emph{ſchreibe auch an Richard}}}\Cendnote{\textnormal{Das erlaubt, diese Karte vor jener an
                        Beer-Hofmann\pwindex{Beer-Hofmann, Richard 1866-07-11 – 1945-09-26@\textsc{Beer-Hofmann, Richard} (1866-07-11 – 1945-09-26), \emph{Schriftsteller/Schriftstellerin}|pwk} vom gleichen Tag
                     anzusetzen.}}}\label{K_L01485-2}.\pend
           \selectlanguage{ngerman}\vspace{1em}\pstart \spacefill\mbox{{[}hs. :{]} Olga.}\pend{}\selectlanguage{ngerman}\endnumbering\briefempfaengerindex{Hofmannsthal, Hugo von@\textsc{Hofmannsthal, Hugo von}!zzzSchnitzler, Olga@\emph{von Olga Schnitzler}!1904-12-281@{28. 12. 1904}|)be}\briefempfaengerindex{Hofmannsthal, Hugo von@\textsc{Hofmannsthal, Hugo von}!zzzSchnitzler, Arthur@\emph{von Arthur Schnitzler}!1904-12-281@{28. 12. 1904}|)be}\mylabel{L01485h}  \normalsize

\doendnotes{C}
\bigskip
\vfill

\clearpage

\footnotesize

\lohead{\textsc{register}}

% Definiere theindex-Environment komplett neu ohne reledmac
\makeatletter
\renewenvironment{theindex}{%
  \section*{\indexname}%
  \setlength{\parindent}{0pt}%
  \setlength{\parskip}{0pt plus 0.3pt}%
  \let\item\@idxitem
}{%
  \clearpage
}
\makeatother

\IfFileExists{\jobname-pw.ind}{\input{\jobname-pw.ind}}{}

\end{document}

      