%% latex-leseansicht-vorspann.tex
%% Vorspann für die Leseansicht.
%% Lädt die gemeinsame Datei latex-vorspann.tex mit nicht gesetztem Schalter.

\newif\ifkorrekturansicht
\korrekturansichtfalse

\input{../tex-inputs/latex-vorspann}

\begin{center}
            \textcolor{red}{ENTWURF. ENTZIFFERUNG NOCH NICHT KORREKTURGELESEN}
                      \end{center}
            
               \section[Arthur und Olga Schnitzler an Hugo von Hofmannsthal, 28. 12. 1904]{ Arthur und Olga Schnitzler an Hugo von Hofmannsthal,
               28. 12. 1904}\nopagebreak\mylabel{v}\rehead{ }\begin{ledgroupsized}[t]{13cm}\normalsize\beginnumbering\briefempfaengerindex{Hofmannsthal, Hugo von@\textsc{Hofmannsthal, Hugo von}!zzzSchnitzler, Olga@\emph{von Olga Schnitzler}!1904-12-281@{28. 12. 1904}|(be}\briefempfaengerindex{Hofmannsthal, Hugo von@\textsc{Hofmannsthal, Hugo von}!zzzSchnitzler, Arthur@\emph{von Arthur Schnitzler}!1904-12-281@{28. 12. 1904}|(be} \toendnotes[C]{\smallbreak\pagebreak[2]} \Standort{FDH, Hs-30885,118.}
\physDesc{Bildpostkarte
\newline{}Handschrift Arthur Schnitzler: Bleistift, deutsche Kurrent\newline{}Handschrift Olga Schnitzler: Bleistift\newline{}Versand: 1) Stempel: »\nobreak{}\oindex{Salzburg@\textbf{Salzburg}|pwk}St{[}. Gilgen{]}\nobreak{}«.  2) Stempel: »\nobreak{}\oindex{Rodaun@\textbf{Rodaun}|pwk}Ro{[}dau{]}n, 29. 12. 04\nobreak{}«. }\buchAbdrucke{\weitereDrucke{Hugo von Hofmannsthal, Arthur Schnitzler: \emph{Briefwechsel}. Hg. Therese Nickl und Heinrich Schnitzler. Frankfurt am Main: \emph{S. Fischer} 1964, S. 208.} }\toendnotes[C]{\smallbreak}\pstart{}{\pb}\textsc{Herrn Hugo v. Hofmannsthal}\pend{}\pstart{}\textsc{Rodaun b (Wien)\oindex{Rodaun@\textbf{Rodaun}|pw}}\pend{}\pstart{}\textsc{Badgasse 5}\oindex{Badgasse@\textbf{Badgasse}|pw}\pend{}{\bigskip}\pstart
           \noindent{}\centering{}\textcolor{gray}{\textbf{{\pb}St. Gilgen\oindex{St. Gilgen@\textbf{St. Gilgen}|pw}}}\pend
           \pstart
           \noindent{}\centering{}\textcolor{gray}{\textbf{Salzburg\oindex{Salzburg (Land)@\textbf{Salzburg (Land)}|pw}}}\pend
           \pstart
           {\pb}lieber Hugo, wir möchten Sie
               \label{K_L01485-1v}\edtext{Montag}{\lemma{\textnormal{\emph{Montag}}}\Cendnote{\textnormal{siehe A. S.: \emph{Tagebuch}, 2. 1. 1905}}}\label{K_L01485-1h}{ }\introOben{}den
                  2.\introOben{}{ }Abend 8{ }Hietzing\oindex{XIII., Hietzing@\textbf{XIII., Hietzing}|pw}{ }ſehen.\pend
           \pstart
           Antwort Wien\oindex{Wien@\textbf{Wien}|pw} erbeten.\pend
           \pstart Ihr \spacefill\mbox{A.}\pend{}\pstart
           \noindent{}Ich \label{K_L01485_1v}\edtext{ſchreibe auch an Richard\pwindex{Beer-Hofmann, Richard 11.07.1866 – 26.09.1945@\textsc{Beer-Hofmann, Richard} (11.07.1866 – 26.09.1945), \emph{Schriftsteller}|pw}}{\lemma{\textnormal{\emph{ſchreibe auch an Richard}}}\Cendnote{\textnormal{Das erlaubt, diese Karte
                  vor jener an Beer-Hofmann\pwindex{Beer-Hofmann, Richard 11.07.1866 – 26.09.1945@\textsc{Beer-Hofmann, Richard} (11.07.1866 – 26.09.1945), \emph{Schriftsteller}|pwk} vom gleichen Tag anzusetzen.}}}\label{K_L01485_1h}.\pend
           \pstart \spacefill\mbox{{[}hs. O. Schnitzler:{]} Olga.}\pend{}\endnumbering\briefempfaengerindex{Hofmannsthal, Hugo von@\textsc{Hofmannsthal, Hugo von}!zzzSchnitzler, Olga@\emph{von Olga Schnitzler}!1904-12-281@{28. 12. 1904}|)be}\briefempfaengerindex{Hofmannsthal, Hugo von@\textsc{Hofmannsthal, Hugo von}!zzzSchnitzler, Arthur@\emph{von Arthur Schnitzler}!1904-12-281@{28. 12. 1904}|)be}\mylabel{h}\end{ledgroupsized}  \newcommand{\dateiname}{L01485}\newcommand{\titel}{Arthur und Olga Schnitzler an Hugo von Hofmannsthal, 28. 12. 1904}\newcommand{\editorInnen}{Martin Anton Müller und Gerd-Hermann Susen}%% latex-leseansicht-abspann.tex
%% Abspann für die Leseansicht.
%% Der Schalter \ifkorrekturansicht ist bereits durch den Vorspann gesetzt.

%% latex-abspann.tex
%% Gemeinsamer Abspann für Korrekturansicht und Leseansicht.
%% Setzt den Schalter \ifkorrekturansicht voraus (gesetzt in den
%% einbindenden Dateien latex-korrekturansicht-abspann.tex bzw.
%% latex-leseansicht-abspann.tex).
%% ---------------------------------------------------------------

\normalsize

% Das esempio-Environment wird nur in der Leseansicht benötigt
\ifkorrekturansicht\else
\newenvironment{esempio}[3]%
{
    \vspace{1.5ex}
    \rlap{\underline{#1}}
    \par
    \setlength{\parindent}{0cm}
    \nopagebreak
    \leftskip=#2cm
    \rightskip=#3cm
}
{
    \par
}
\fi

\doendnotes{C}
\bigskip
\vfill

\clearpage

\footnotesize

\ifkorrekturansicht
  \lohead{\textsc{register}}
\fi

% theindex-Environment neu definieren ohne reledmac
\makeatletter
\renewenvironment{theindex}{%
  \ifkorrekturansicht
    \section*{\indexname}%
  \else
    \subsubsection*{Index der erwähnten Entitäten}%
  \fi
  \setlength{\parindent}{0pt}%
  \setlength{\parskip}{0pt plus 0.3pt}%
  \let\item\@idxitem
}{%
  \ifkorrekturansicht\clearpage\fi
}
\makeatother

\IfFileExists{\jobname-pw.ind}{\input{\jobname-pw.ind}}{}

% Quellenangabe nur in der Leseansicht
\ifkorrekturansicht\else
% Fallback-Definitionen, falls die .tex-Datei \titel etc. nicht gesetzt hat
\providecommand{\titel}{}
\providecommand{\editorInnen}{}
\providecommand{\dateiname}{\jobname}

\vspace{3cm}

\vfill

\footnotesize
\textsc{Quelle}: \titel. Herausgegeben von {\editorInnen}. In: \emph{Arthur Schnitzler: Briefwechsel mit Autorinnen und Autoren}.
 Digitale Edition, https://schnitzler-briefe.acdh.oeaw.ac.at/{\dateiname}.html (Stand \today)
\fi

\end{document}


      