%% latex-leseansicht-vorspann.tex
%% Vorspann für die Leseansicht.
%% Lädt die gemeinsame Datei latex-vorspann.tex mit nicht gesetztem Schalter.

\newif\ifkorrekturansicht
\korrekturansichtfalse

\input{../tex-inputs/latex-vorspann}


\section[Arthur Schnitzler an Stefan Zweig, 17. 1. 1911]{L03798 Arthur Schnitzler an Stefan Zweig, 17. 1. 1911}
\nopagebreak\mylabel{L03798v}
\rehead{ }\normalsize\beginnumbering\briefempfaengerindex{Zweig, Stefan@\textsc{Zweig, Stefan}!zzzSchnitzler, Arthur@\emph{von Arthur Schnitzler}!1911-01-171@{17. 1. 1911}|(be}
\toendnotes[C]{\smallbreak\pagebreak[2]}
\correspDesc{Versand  durch Arthur Schnitzler am 17. 1. 1911 in Wien
\newline{}Erhalt  durch Stefan Zweig im Zeitraum [17. 1. 1911 – 20. 1. 1911?] in Wien}\toendnotes[C]{\smallbreak}
\Standort{Jerusalem, National Library of Israel, ARC. Ms. Var. 305 1 58 Stefan Zweig Collection.}
\physDesc{Visitenkarte, 264 Zeichen
\newline{}Handschrift: schwarze Tinte, deutsche Kurrent}\toendnotes[C]{\smallbreak}
\pstart
           \noindent{}{\pb}lieber Herr Doctor, vielen Dank für die freundliche \label{K_L03798-1v}\edtext{Zuſendung}{\lemma{\textnormal{\emph{Zusendung}}}\Cendnote{\textnormal{Unklar. Eventuell handelte es sich um eine Besprechung der Uraufführung von \emph{Das weite Land}\pwindex{Schnitzler, Arthur 15.\,5.\,1862 Wien – 21.\,10.\,1931 ebd.@\textsc{Schnitzler, Arthur} (15.\,5.\,1862 Wien – 21.\,10.\,1931 ebd.), \emph{Schriftsteller, Mediziner}!weite Land. Tragikomödie in fünf Akten@\strich\emph{Das weite Land. Tragikomödie in fünf Akten}|pwk}\eventindex{Sankt Petersburg@\textbf{Sankt Petersburg}!Uraufführung von Das weite Land, 2.11.1910@Uraufführung von Das weite Land, 2.11.1910|pwk} durch das \emph{Novyj
                     dramatičeskij teatr}\orgindex{Komissarževskaja-Theater@Komissarževskaja-Theater|pwk} in Sankt
                     Petersburg\oindex{Sankt Petersburg@\textbf{Sankt Petersburg}|pwk} am 2. 11. 1910. Jedenfalls dürfte dadurch die in Folge angesprochene
                  Bezugnahme auf den »Mai« erklärbar werden. Jedenfalls war zu diesem Zeitpunkt für diesen Monat die deutschsprachige
                  Erstaufführung von \emph{Das weite Land}\pwindex{Schnitzler, Arthur 15.\,5.\,1862 Wien – 21.\,10.\,1931 ebd.@\textsc{Schnitzler, Arthur} (15.\,5.\,1862 Wien – 21.\,10.\,1931 ebd.), \emph{Schriftsteller, Mediziner}!weite Land. Tragikomödie in fünf Akten@\strich\emph{Das weite Land. Tragikomödie in fünf Akten}|pwk} noch geplant. Es dürfte Zweig\pwindex{Zweig, Stefan 28.\,11.\,1881 Wien – 23.\,2.\,1942 Petrópolis@\textsc{Zweig, Stefan} (28.\,11.\,1881 Wien – 23.\,2.\,1942 Petrópolis), \emph{Schriftsteller}|pwk} Anlass gegeben haben, um ein
                  Bühnenmanuskript von \emph{Das weite Land}\pwindex{Schnitzler, Arthur 15.\,5.\,1862 Wien – 21.\,10.\,1931 ebd.@\textsc{Schnitzler, Arthur} (15.\,5.\,1862 Wien – 21.\,10.\,1931 ebd.), \emph{Schriftsteller, Mediziner}!weite Land. Tragikomödie in fünf Akten@\strich\emph{Das weite Land. Tragikomödie in fünf Akten}|pwk} zu
                  erbitten (vgl. XXXX Auszeichnungsfehler: Dokument L03627 nicht gefunden), das Schnitzler womöglich schon mit dieser Karte Zweig\pwindex{Zweig, Stefan 28.\,11.\,1881 Wien – 23.\,2.\,1942 Petrópolis@\textsc{Zweig, Stefan} (28.\,11.\,1881 Wien – 23.\,2.\,1942 Petrópolis), \emph{Schriftsteller}|pwk} zukommen ließ.}}}\label{K_L03798-1}. Sehr{ }ſchön! Vom Wiederhall
               (insbeſondere{ }ſoweit es den Mai betrifft) bin ich weniger überzeugt.\pend
           
\pstart
           Laſſen Sie{ }ſich auch noch herzlichſt für Ihre liebe Theilnahme danken.\pend
           
\pstart
           {\pb}Auf bald\textcolor{gray}{i}gs Wiederſehn!{\\[\baselineskip]}Ihr{\\[\baselineskip]}\pend
           \leftskip=0em{}
\pstart
           \noindent{}\centering{}\strikeout{\textcolor{gray}{\textbf{D\textsuperscript{r}}}}{ }\textcolor{gray}{\textbf{Arthur Schnitzler}}\pend
           
\pstart
           17. 1. 1911\pend
           \selectlanguage{ngerman}\endnumbering\briefempfaengerindex{Zweig, Stefan@\textsc{Zweig, Stefan}!zzzSchnitzler, Arthur@\emph{von Arthur Schnitzler}!1911-01-171@{17. 1. 1911}|)be}\mylabel{L03798h}  \newcommand{\dateiname}{L03798}\newcommand{\titel}{Arthur Schnitzler an Stefan Zweig, 17. 1. 1911}\newcommand{\editorInnen}{Selma Jahnke und Martin Anton Müller}%% latex-leseansicht-abspann.tex
%% Abspann für die Leseansicht.
%% Der Schalter \ifkorrekturansicht ist bereits durch den Vorspann gesetzt.

%% latex-abspann.tex
%% Gemeinsamer Abspann für Korrekturansicht und Leseansicht.
%% Setzt den Schalter \ifkorrekturansicht voraus (gesetzt in den
%% einbindenden Dateien latex-korrekturansicht-abspann.tex bzw.
%% latex-leseansicht-abspann.tex).
%% ---------------------------------------------------------------

\normalsize

% Das esempio-Environment wird nur in der Leseansicht benötigt
\ifkorrekturansicht\else
\newenvironment{esempio}[3]%
{
    \vspace{1.5ex}
    \rlap{\underline{#1}}
    \par
    \setlength{\parindent}{0cm}
    \nopagebreak
    \leftskip=#2cm
    \rightskip=#3cm
}
{
    \par
}
\fi

\doendnotes{C}
\bigskip
\vfill

\clearpage

\footnotesize

\ifkorrekturansicht
  \lohead{\textsc{register}}
\fi

% theindex-Environment neu definieren ohne reledmac
\makeatletter
\renewenvironment{theindex}{%
  \ifkorrekturansicht
    \section*{\indexname}%
  \else
    \subsubsection*{Index der erwähnten Entitäten}%
  \fi
  \setlength{\parindent}{0pt}%
  \setlength{\parskip}{0pt plus 0.3pt}%
  \let\item\@idxitem
}{%
  \ifkorrekturansicht\clearpage\fi
}
\makeatother

\IfFileExists{\jobname-pw.ind}{\input{\jobname-pw.ind}}{}

% Quellenangabe nur in der Leseansicht
\ifkorrekturansicht\else
% Fallback-Definitionen, falls die .tex-Datei \titel etc. nicht gesetzt hat
\providecommand{\titel}{}
\providecommand{\editorInnen}{}
\providecommand{\dateiname}{\jobname}

\vspace{3cm}

\vfill

\footnotesize
\textsc{Quelle}: \titel. Herausgegeben von {\editorInnen}. In: \emph{Arthur Schnitzler: Briefwechsel mit Autorinnen und Autoren}.
 Digitale Edition, https://schnitzler-briefe.acdh.oeaw.ac.at/{\dateiname}.html (Stand \today)
\fi

\end{document}


