%% latex-korrekturansicht-vorspann.tex
%% Vorspann für die Korrekturansicht.
%% Lädt die gemeinsame Datei latex-vorspann.tex mit gesetztem Schalter.

\newif\ifkorrekturansicht
\korrekturansichttrue

\input{../tex-inputs/latex-vorspann}


\section[Arthur Schnitzler an Stefan Zweig, 17. 1. 1911]{L03798 Arthur Schnitzler an Stefan Zweig, 17. 1. 1911}
\nopagebreak\mylabel{L03798v}
\rehead{ }\normalsize\beginnumbering\briefempfaengerindex{Zweig, Stefan@\textsc{Zweig, Stefan}!zzzSchnitzler, Arthur@\emph{von Arthur Schnitzler}!1911-01-171@{17. 1. 1911}|(be}
\toendnotes[C]{\smallbreak\pagebreak[2]}\Standort{Jerusalem, National Library of Israel, ARC. Ms. Var. 305 1 58 Stefan Zweig Collection.}
\physDesc{Visitenkarte, 1 Blatt, 2 Seiten, 263 Zeichen
\newline{}Handschrift: schwarze Tinte, deutsche Kurrent}\toendnotes[C]{\smallbreak}
\pstart
           \noindent{}{\pb}lieber Herr Doctor, vielen Dank für die freundliche \label{K_L03798-1v}\edtext{Zuſendung}{\lemma{\textnormal{\emph{Zuſendung}}}\Cendnote{\textnormal{Unklar. Eventuell handelte es sich um eine Besprechung der \emph{Uraufführung von \emph{Das weite Land}\pwindex{weite Land. Tragikomoedie in fuenf Akten@\emph{Das weite Land. Tragikomödie in fünf Akten}|pwk}}\eventindex{Sankt Petersburg@\textbf{Sankt Petersburg}!Urauffuehrung von Das weite Land, 2.11.1910@Uraufführung von Das weite Land, 2.11.1910|pwk} durch das \emph{Novyj
                     dramatičeskij teatr}\orgindex{Komissarževskaja-Theater@Komissarževskaja-Theater|pwk} in Sankt
                     Petersburg\oindex{Sankt Petersburg@\textbf{Sankt Petersburg}, \emph{P.PPLA}|pwk} am 2. 11. 1910. Jedenfalls dürfte dadurch die in Folge angesprochene
                  Bezugnahme auf den »Mai« erklärbar werden. Jedenfalls war zu diesem Zeitpunkt für diesen Monat die deutschsprachige
                  Erstaufführung von \emph{Das weite Land}\pwindex{weite Land. Tragikomoedie in fuenf Akten@\emph{Das weite Land. Tragikomödie in fünf Akten}|pwk} noch geplant. Es dürfte Zweig\pwindex{Zweig, Stefan 28.11.1881 – 23.02.1942@\textsc{Zweig, Stefan} (28.11.1881 – 23.02.1942), \emph{Schriftsteller/Schriftstellerin}|pwk} Anlass gegeben haben, um ein
                  Bühnenmanuskript von \emph{Das weite Land}\pwindex{weite Land. Tragikomoedie in fuenf Akten@\emph{Das weite Land. Tragikomödie in fünf Akten}|pwk} zu
                  erbitten (vgl. Stefan Zweig an Arthur Schnitzler, 4. 2. 191[1]), das Schnitzler womöglich schon mit dieser Karte Zweig\pwindex{Zweig, Stefan 28.11.1881 – 23.02.1942@\textsc{Zweig, Stefan} (28.11.1881 – 23.02.1942), \emph{Schriftsteller/Schriftstellerin}|pwk} zukommen ließ.}}}\label{K_L03798-1}. Sehr ſchön! Vom Wiederhall
               (insbeſondere ſoweit es den Mai betrifft) bin ich weniger überzeugt.\pend
           
\pstart
           Laſſen Sie ſich auch noch herzlichſt für Ihre liebe Theilnahme danken.\pend
           
\pstart
           {\pb}Auf bald\textcolor{gray}{i}gs Wiederſehn!{\\[\baselineskip]}Ihr{\\[\baselineskip]}\pend
           \leftskip=0em{}
\pstart
           \noindent{}\centering{}\textcolor{gray}{\textbf{D\textsuperscript{r} Arthur
                     Schnitzler }}\pend
           
\pstart
           17. 1. 1911\pend
           \selectlanguage{ngerman}\endnumbering\briefempfaengerindex{Zweig, Stefan@\textsc{Zweig, Stefan}!zzzSchnitzler, Arthur@\emph{von Arthur Schnitzler}!1911-01-171@{17. 1. 1911}|)be}\mylabel{L03798h}  \normalsize

\doendnotes{C}
\bigskip
\vfill

\clearpage

\footnotesize

\lohead{\textsc{register}}

% Definiere theindex-Environment komplett neu ohne reledmac
\makeatletter
\renewenvironment{theindex}{%
  \section*{\indexname}%
  \setlength{\parindent}{0pt}%
  \setlength{\parskip}{0pt plus 0.3pt}%
  \let\item\@idxitem
}{%
  \clearpage
}
\makeatother

\IfFileExists{\jobname-pw.ind}{\input{\jobname-pw.ind}}{}

\end{document}

      