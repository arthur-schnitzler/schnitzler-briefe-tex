%% latex-leseansicht-vorspann.tex
%% Vorspann für die Leseansicht.
%% Lädt die gemeinsame Datei latex-vorspann.tex mit nicht gesetztem Schalter.

\newif\ifkorrekturansicht
\korrekturansichtfalse

\input{../tex-inputs/latex-vorspann}


\section[ Felix Salten an Arthur Schnitzler, {[}1895 – 21. 1. 1897?{]}]{L03182 Felix Salten an Arthur Schnitzler,  [1895–21. 1. 1897?]}
\nopagebreak\mylabel{L03182v}
\rehead{ }\normalsize\beginnumbering\briefempfaengerindex{Schnitzler, Arthur@\textsc{Schnitzler, Arthur}!zzzSalten, Felix@\emph{von Felix Salten}!1897-01-212@{[1895–21. 1. 1897?]}|(be}
\toendnotes[C]{\smallbreak\pagebreak[2]}
\correspDesc{Versand  durch Felix Salten im Zeitraum [1895 – 21. 1. 1897?] in Wien
\newline{}Erhalt  durch Arthur Schnitzler am [1895 – 21. 1. 1897?] in Wien}\toendnotes[C]{\smallbreak}
\Standort{CUL, Schnitzler, B 89, A 1.}
\physDesc{Brief, 1 Blatt, 1 Seite, 293 Zeichen
\newline{}Handschrift: Bleistift, lateinische Kurrent
\newline{}Ordnung: mit Bleistift von unbekannter Hand nummeriert: »82« }\toendnotes[C]{\smallbreak}
\pstart
           \noindent{}{\pb}Lieber Freund, ich bitte Sie recht sehr, leihen Sie
               mir bis zum Abend zehn Gulden. ich benöthige es recht dringend, und mei\substVorne{}\textsuperscript{\textcolor{gray}{m}}\substDazwischen{}n\substHinten{}{ }Bruder\pwindex{Salzmann, Michael Emil 19.\,1.\,1858 Szigetvár – 26.\,6.\,1908 Wien@\textsc{Salzmann, Michael Emil} (19.\,1.\,1858 Szigetvár – 26.\,6.\,1908 Wien), \emph{Versicherungsbeamter}|pwuv},
               welcher Geld von mir hat, ist nicht zu Hause.\pend
           
\pstart
           Hoffentlich trifft Sie dieser Brief noch an. Ich frage Abends gegen 9 im
                  \label{K_L03182-1v}\edtext{Griensteidl\oindex{Wien@\textbf{Wien}!I., Innere Stadt@\textbf{I., Innere Stadt}!Café Griensteidl@\textbf{Café Griensteidl}, \emph{Kaffeehaus}|pw}}{\lemma{\textnormal{\emph{Griensteidl}}}\Cendnote{\textnormal{Das Korrespondenzstück ist undatiert und
                  es gibt keinen Anhaltspunkt, außer dass es vor dem 21. 1. 1897 verfasst worden sein muss, da an diesem Tag das Café Griensteidl\oindex{Wien@\textbf{Wien}!I., Innere Stadt@\textbf{I., Innere Stadt}!Café Griensteidl@\textbf{Café Griensteidl}, \emph{Kaffeehaus}|pwk} zum letzten Mal geöffnet war. Eingeordnet
                  ist es im Nachlass am Ende der Korrespondenz von 1896,
                  weswegen angenommen werden kann, dass es frühestens 1895
                  übermittelt wurde.}}}\label{K_L03182-1}, wo ich Sie finde.\pend
           
\pstart
           Herzlichst {\\[\baselineskip]}\spacefill\mbox{Salten.}\pend
           \leftskip=0em{}\selectlanguage{ngerman}\endnumbering\briefempfaengerindex{Schnitzler, Arthur@\textsc{Schnitzler, Arthur}!zzzSalten, Felix@\emph{von Felix Salten}!1895-01-012@{[1895–21. 1. 1897?]}|)be}\mylabel{L03182h}  \newcommand{\dateiname}{L03182}\newcommand{\titel}{Felix Salten an Arthur Schnitzler, [1895 – 21. 1. 1897?]}\newcommand{\editorInnen}{Martin Anton Müller und Laura Untner}%% latex-leseansicht-abspann.tex
%% Abspann für die Leseansicht.
%% Der Schalter \ifkorrekturansicht ist bereits durch den Vorspann gesetzt.

%% latex-abspann.tex
%% Gemeinsamer Abspann für Korrekturansicht und Leseansicht.
%% Setzt den Schalter \ifkorrekturansicht voraus (gesetzt in den
%% einbindenden Dateien latex-korrekturansicht-abspann.tex bzw.
%% latex-leseansicht-abspann.tex).
%% ---------------------------------------------------------------

\normalsize

% Das esempio-Environment wird nur in der Leseansicht benötigt
\ifkorrekturansicht\else
\newenvironment{esempio}[3]%
{
    \vspace{1.5ex}
    \rlap{\underline{#1}}
    \par
    \setlength{\parindent}{0cm}
    \nopagebreak
    \leftskip=#2cm
    \rightskip=#3cm
}
{
    \par
}
\fi

\doendnotes{C}
\bigskip
\vfill

\clearpage

\footnotesize

\ifkorrekturansicht
  \lohead{\textsc{register}}
\fi

% theindex-Environment neu definieren ohne reledmac
\makeatletter
\renewenvironment{theindex}{%
  \ifkorrekturansicht
    \section*{\indexname}%
  \else
    \subsubsection*{Index der erwähnten Entitäten}%
  \fi
  \setlength{\parindent}{0pt}%
  \setlength{\parskip}{0pt plus 0.3pt}%
  \let\item\@idxitem
}{%
  \ifkorrekturansicht\clearpage\fi
}
\makeatother

\IfFileExists{\jobname-pw.ind}{\input{\jobname-pw.ind}}{}

% Quellenangabe nur in der Leseansicht
\ifkorrekturansicht\else
% Fallback-Definitionen, falls die .tex-Datei \titel etc. nicht gesetzt hat
\providecommand{\titel}{}
\providecommand{\editorInnen}{}
\providecommand{\dateiname}{\jobname}

\vspace{3cm}

\vfill

\footnotesize
\textsc{Quelle}: \titel. Herausgegeben von {\editorInnen}. In: \emph{Arthur Schnitzler: Briefwechsel mit Autorinnen und Autoren}.
 Digitale Edition, https://schnitzler-briefe.acdh.oeaw.ac.at/{\dateiname}.html (Stand \today)
\fi

\end{document}


