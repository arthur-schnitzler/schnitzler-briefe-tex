%% latex-leseansicht-vorspann.tex
%% Vorspann für die Leseansicht.
%% Lädt die gemeinsame Datei latex-vorspann.tex mit nicht gesetztem Schalter.

\newif\ifkorrekturansicht
\korrekturansichtfalse

\input{../tex-inputs/latex-vorspann}


\section[Arthur Schnitzler an Richard Beer-Hofmann, 4. 8. 1897]{L00712 Arthur Schnitzler an Richard Beer-Hofmann, 4. 8. 1897}
\nopagebreak\mylabel{L00712v}
\rehead{ }\normalsize\beginnumbering\briefempfaengerindex{Beer-Hofmann, Richard@\textsc{Beer-Hofmann, Richard}!zzzSchnitzler, Arthur@\emph{von Arthur Schnitzler}!1897-08-041@{4. 8. 1897}|(be}
\toendnotes[C]{\smallbreak\pagebreak[2]}
\correspDesc{Versand  durch Arthur Schnitzler am 4. 8. 1897 in Wien
\newline{}Erhalt  durch Richard Beer-Hofmann am 6. 8. 1897 in Bad Ischl}\toendnotes[C]{\smallbreak}
\Standort{YCGL, MSS 31.}
\physDesc{Brief, 1 Blatt, 3 Seiten, Kuvert, 624 Zeichen
\newline{}Handschrift: Bleistift, deutsche Kurrent
\newline{}Versand: 1) Stempel: »\nobreak{}\oindex{IX., Alsergrund@\textbf{IX., Alsergrund}, \emph{Verwaltungsgebiet}|pwk}Wien 9/3, 4. 8. 97, 5–6N\nobreak{}«.   2) Stempel: »\nobreak{}\oindex{Bad Ischl@\textbf{Bad Ischl}|pwk}Ischl, 6. 8. 97, 1–2N\nobreak{}«. }
\buchAbdrucke{\weitereDrucke{Arthur Schnitzler, Richard Beer-Hofmann: \emph{Briefwechsel 1891–1931}. Herausgegeben von Konstanze Fliedl. Wien, Zürich: \emph{Europaverlag} 1992, S. 112.} }\toendnotes[C]{\smallbreak}\pstart{}{\pb}Herrn \textsc{Dr. Richard
                     Beer-Hofmann}\pend{}\pstart{}\textsc{Ischl\oindex{Bad Ischl@\textbf{Bad Ischl}|pw}}\pend{}\pstart{}\textsc{Egelmoos 22}\oindex{Eglmoosgasse@\textbf{Eglmoosgasse}, \emph{Bezirk}|pw}.\pend{}{\bigskip}\vspace{1em}
\pstart{}{\pb}Lieber Richard.\pend\vspace{0.5em}
\pstart
           Thun Sie mir einen großen Gefallen.\pend
           
\pstart
           Frau F.\pwindex{Freudenthal, Rosa 1862 – 18.\,6.\,1905 Berlin@\textsc{Freudenthal, Rosa} (1862 – 18.\,6.\,1905 Berlin)|pw} iſt wieder in Iſchl\oindex{Bad Ischl@\textbf{Bad Ischl}|pw}; heute erhielt ich einen Brief von ihr, ich möge ihr \uline{durch Sie} Briefe u Bilder zurückſchicken, in Wien\oindex{Wien@\textbf{Wien}, \emph{Verwaltungsgebiet}|pw} erhalte ich die Erklärung. – Gehn Sie zu {\pb}Petter\oindex{Hotel und Pension Rudolfshöhe (Leopold Petter)@\textbf{Hotel und Pension Rudolfshöhe (Leopold Petter)}, \emph{Hotel}|pw},{ }ſie ist \label{K_L00712-1v}\edtext{\textsc{en fam.}}{\lemma{\textnormal{\emph{en fam.}}}\Cendnote{\textnormal{französisch en famille: mit ihrer
                  Familie}}}\label{K_L00712-1} dort, Sie werden{ }ſie aber leicht allein{ }ſprechen können. Sagen Sie
               ihr, ich käme bald{ }ſelbſt nach Iſchl\oindex{Bad Ischl@\textbf{Bad Ischl}|pw} und erfülle
               lieber perſönlich ihren Wunſch,{ }ſie kö{\geminationn}e{ }ſicher darauf
               rechnen. {\pb}Bringen Sie aber heraus was dahinter{ }ſteckt,
               ich ärgere mich mehr als die Geſchichte werth iſt. Antworten Sie mir gleich, am
               liebſten telegrafiſch.\pend
           
\pstart
           Herzlich Ihr{\\[\baselineskip]}\spacefill\mbox{Arthur}\pend
           \leftskip=0em{}\selectlanguage{ngerman}\endnumbering\briefempfaengerindex{Beer-Hofmann, Richard@\textsc{Beer-Hofmann, Richard}!zzzSchnitzler, Arthur@\emph{von Arthur Schnitzler}!1897-08-041@{4. 8. 1897}|)be}\mylabel{L00712h}  \newcommand{\dateiname}{L00712}\newcommand{\titel}{Arthur Schnitzler an Richard Beer-Hofmann, 4. 8. 1897}\newcommand{\editorInnen}{Martin Anton Müller und Gerd-Hermann Susen}%% latex-leseansicht-abspann.tex
%% Abspann für die Leseansicht.
%% Der Schalter \ifkorrekturansicht ist bereits durch den Vorspann gesetzt.

%% latex-abspann.tex
%% Gemeinsamer Abspann für Korrekturansicht und Leseansicht.
%% Setzt den Schalter \ifkorrekturansicht voraus (gesetzt in den
%% einbindenden Dateien latex-korrekturansicht-abspann.tex bzw.
%% latex-leseansicht-abspann.tex).
%% ---------------------------------------------------------------

\normalsize

% Das esempio-Environment wird nur in der Leseansicht benötigt
\ifkorrekturansicht\else
\newenvironment{esempio}[3]%
{
    \vspace{1.5ex}
    \rlap{\underline{#1}}
    \par
    \setlength{\parindent}{0cm}
    \nopagebreak
    \leftskip=#2cm
    \rightskip=#3cm
}
{
    \par
}
\fi

\doendnotes{C}
\bigskip
\vfill

\clearpage

\footnotesize

\ifkorrekturansicht
  \lohead{\textsc{register}}
\fi

% theindex-Environment neu definieren ohne reledmac
\makeatletter
\renewenvironment{theindex}{%
  \ifkorrekturansicht
    \section*{\indexname}%
  \else
    \subsubsection*{Index der erwähnten Entitäten}%
  \fi
  \setlength{\parindent}{0pt}%
  \setlength{\parskip}{0pt plus 0.3pt}%
  \let\item\@idxitem
}{%
  \ifkorrekturansicht\clearpage\fi
}
\makeatother

\IfFileExists{\jobname-pw.ind}{\input{\jobname-pw.ind}}{}

% Quellenangabe nur in der Leseansicht
\ifkorrekturansicht\else
% Fallback-Definitionen, falls die .tex-Datei \titel etc. nicht gesetzt hat
\providecommand{\titel}{}
\providecommand{\editorInnen}{}
\providecommand{\dateiname}{\jobname}

\vspace{3cm}

\vfill

\footnotesize
\textsc{Quelle}: \titel. Herausgegeben von {\editorInnen}. In: \emph{Arthur Schnitzler: Briefwechsel mit Autorinnen und Autoren}.
 Digitale Edition, https://schnitzler-briefe.acdh.oeaw.ac.at/{\dateiname}.html (Stand \today)
\fi

\end{document}


