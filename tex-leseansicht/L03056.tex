%% latex-korrekturansicht-vorspann.tex
%% Vorspann für die Korrekturansicht.
%% Lädt die gemeinsame Datei latex-vorspann.tex mit gesetztem Schalter.

\newif\ifkorrekturansicht
\korrekturansichttrue

\input{../tex-inputs/latex-vorspann}


\section[ Paul Goldmann an Arthur Schnitzler, 29. 1. {[}1901{]}]{L03056 Paul Goldmann an Arthur Schnitzler, 29. 1. {[}1901{]}}
\nopagebreak\mylabel{L03056v}
\rehead{ }\normalsize\beginnumbering\briefempfaengerindex{Schnitzler, Arthur@\textsc{Schnitzler, Arthur}!zzzGoldmann, Paul@\emph{von Paul Goldmann}!1901-01-292@{29. 1. {[}1901{]}}|(be}
\toendnotes[C]{\smallbreak\pagebreak[2]}\Standort{DLA, A:Schnitzler, HS.NZ85.1.3171.}
\physDesc{Brief, 1 Blatt, 3 Seiten, 1604 Zeichen
\newline{}Handschrift: blaue Tinte, deutsche Kurrent
\newline{}Beilage: handschriftlicher Brief Marie und Auguste
                                 Glümer, 1 Blatt, 3 Seiten, Handschrift Auguste Glümer, Bleistift,
                                 deutsche Kurrent 
\newline{}Schnitzler: 1) mit Bleistift das Jahr »901« vermerkt  2) mit rotem Buntstift eine Unterstreichung}\toendnotes[C]{\smallbreak}
\pstart
           \raggedleft{}{\pb}\textcolor{gray}{\textbf{DESSAUERSTRASSE 19}}\oindex{Dessauer Strasse@\textbf{Dessauer Straße}, \emph{Straße (K.STR)}|pw}\pend
           
\pstart
           Berlin\oindex{Berlin@\textbf{Berlin}, \emph{P.PPLC}|pw}, 29. Januar.\pend
           
\pstart\center{}Mein lieber Freund,\pend\vspace{0.5em}
\pstart
           Auch ich war unruhig, aber es liegt kein Gund dazu vor, wie beifolgender Brief
               beweiſt. \substVorne{}\textsuperscript{\textcolor{gray}{W}}\substDazwischen{}D\substHinten{}a ich ein großes Mißtrauen gegen den behandelnden »\label{K_L03056-1v}\edtext{Wunderdoktor\pwindex{?? [behandelnder Arzt von Marie Gluemer, Anfang 1901] @\textsc{?? [behandelnder Arzt von Marie Glümer, Anfang 1901]}|pwv}}{\lemma{\textnormal{\emph{Wunderdoktor}}}\Cendnote{\textnormal{nicht ermittelt}}}\label{K_L03056-1}« hatte, ſandte
               ich das \label{K_L03056-2v}\edtext{Mädel\pwindex{Gluemer, Marie 03.07.1867 – 16.11.1925@\textsc{Glümer, Marie} (03.07.1867 – 16.11.1925), \emph{Schauspieler/Schauspielerin}|pwv}}{\lemma{\textnormal{\emph{Mädel}}}\Cendnote{\textnormal{Siehe Paul Goldmann an Arthur Schnitzler, 22. 1. [1901]. }}}\label{K_L03056-2} zu meinem
               Freunde \textsc{Dr. Kuttner\pwindex{Kuttner, Arthur *~08.10.1862@\textsc{Kuttner, Arthur} (*~08.10.1862), \emph{Laryngologe/Laryngologin}|pw}} (den \textsc{Dr. Hajek\pwindex{Hajek, Markus 25.11.1861 – 04.04.1941@\textsc{Hajek, Markus} (25.11.1861 – 04.04.1941), \emph{Mediziner/Medizinerin, Laryngologe/Laryngologin}|pw}} kennt u. ſchätzt). Die Viſite fand geſtern
               ſtatt. \textsc{Dr. K.\pwindex{Kuttner, Arthur *~08.10.1862@\textsc{Kuttner, Arthur} (*~08.10.1862), \emph{Laryngologe/Laryngologin}|pwv}} telephonirte mir: Beſſerung ſei bald zu erwarten. Er glaube, daß der
               behandelnde Arzt\pwindex{?? [behandelnder Arzt von Marie Gluemer, Anfang 1901] @\textsc{?? [behandelnder Arzt von Marie Glümer, Anfang 1901]}|pwv} mit ſeinen
               Heilmitteln (\label{K_L03056-3v}\edtext{Arſenik}{\lemma{\textnormal{\emph{Arſenik}}}\Cendnote{\textnormal{Arsen}}}\label{K_L03056-3}) im Weſentlichen auf dem
               rechten Wege ſei, wünſche {\pb}auch, daß das Fräulein\pwindex{Gluemer, Marie 03.07.1867 – 16.11.1925@\textsc{Glümer, Marie} (03.07.1867 – 16.11.1925), \emph{Schauspieler/Schauspielerin}|pwv} weiter bei dieſem
                  Arzt\pwindex{?? [behandelnder Arzt von Marie Gluemer, Anfang 1901] @\textsc{?? [behandelnder Arzt von Marie Glümer, Anfang 1901]}|pwv} in Behandlung
               bleibe, da er großen pſychiſchen Einfluß auf ſeine Patienten habe. Die Behandlung in
               der Naſe ſei allerdings eine »Gemeinheit«. Ob Malaria vorliege, könne man nicht
               wiſſen, ſolange keine Temperatur-Meſſungen u. Blut-Unterſuchungen vorgenommen, woran
               der behandelnde Arzt\pwindex{?? [behandelnder Arzt von Marie Gluemer, Anfang 1901] @\textsc{?? [behandelnder Arzt von Marie Glümer, Anfang 1901]}|pwv} nicht
               zu denken ſcheine{\dotsfour}\pend
           
\pstart
           Daß man Dich doch noch \label{K_L03056-4v}\edtext{ehrengerichtlich verfolgt}{\lemma{\textnormal{\emph{ehrengerichtlich verfolgt}}}\Cendnote{\textnormal{Wegen \emph{Lieutenant Gustl}\pwindex{Lieutenant Gustl. Novelle@\emph{Lieutenant Gustl. Novelle}|pwk}, siehe Paul Goldmann an Arthur Schnitzler, 11. 1. [1901].}}}\label{K_L03056-4}, iſt {\pb}empörend! Sei nur ja recht vorſichtig und thue
               keinen Schritt, ohne vorher mit Rechts- und Landeskundigen Dich berathen zu
               haben!\pend
           
\pstart
           In Eile!\pend
           
\pstart
           Dein {\\[\baselineskip]}\spacefill\mbox{P. G.}\pend
           \leftskip=0em{}\selectlanguage{ngerman}\vspace{1em}{\vspace{1\baselineskip}}
\pstart
           {\pb}{[}hs. :{]} Lieber Herr Doktor,\pend
           
\pstart
           Vor allem vielen Dank für Ihre Bemühungen. Wir ſind heute mit Beruhigung
               von \textsc{D\textsuperscript{r}}{ }\textsc{Kuttner}\pwindex{Kuttner, Arthur *~08.10.1862@\textsc{Kuttner, Arthur} (*~08.10.1862), \emph{Laryngologe/Laryngologin}|pw} weggegangen. Ausführlicher werde ich Ihnen mündlich berichten. Die Krankheit,
               die {\pb}ſich plötzlich geſtern, So{\geminationn}tag{ }Nachm. brach, iſt tatſächlich im Verſchwinden und k\strikeout{l}ein Rückfall mehr zu befürchten. – Wir ſind Ihnen
               jedenfalls für dieſe Beruhigung ſehr dankbar, die wir uns ſelbſt zu verſchaffen, {\pb}wahrſcheinlich noch nicht die Energie gehabt hätten.
               – Bitte gelegentlich um ein Stückchen Ihrer freien Zeit.\pend
           
\pstart
           Mit beſten Empfehlungen für Ihre Frau Ma{\geminationm}a\pwindex{Schnitzler, Louise 1840-07-08 – 1911-09-09@\textsc{Schnitzler, Louise} (1840-07-08 – 1911-09-09)|pwv}{ }{\\[\baselineskip]}Ihre ergebenen {\\[\baselineskip]}\spacefill\mbox{Marie + GustiGlümer}\pend
           \leftskip=0em{}\selectlanguage{ngerman}\endnumbering\briefempfaengerindex{Schnitzler, Arthur@\textsc{Schnitzler, Arthur}!zzzGoldmann, Paul@\emph{von Paul Goldmann}!1901-01-292@{29. 1. {[}1901{]}}|)be}\mylabel{L03056h}  \normalsize

\doendnotes{C}
\bigskip
\vfill

\clearpage

\footnotesize

\lohead{\textsc{register}}

% Definiere theindex-Environment komplett neu ohne reledmac
\makeatletter
\renewenvironment{theindex}{%
  \section*{\indexname}%
  \setlength{\parindent}{0pt}%
  \setlength{\parskip}{0pt plus 0.3pt}%
  \let\item\@idxitem
}{%
  \clearpage
}
\makeatother

\IfFileExists{\jobname-pw.ind}{\input{\jobname-pw.ind}}{}

\end{document}

      