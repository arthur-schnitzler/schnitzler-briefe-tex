%% latex-leseansicht-vorspann.tex
%% Vorspann für die Leseansicht.
%% Lädt die gemeinsame Datei latex-vorspann.tex mit nicht gesetztem Schalter.

\newif\ifkorrekturansicht
\korrekturansichtfalse

\input{../tex-inputs/latex-vorspann}


         
         \renewcommand{\erwaehntePersonen}{Personen:  ?? [behandelnder Arzt von Marie Glümer, Anfang 1901], Marie Glümer, Auguste Glümer, Markus Hajek, Arthur Kuttner, Louise Schnitzler}
         \renewcommand{\erwaehnteOrte}{Orte: Berlin, Dessauer Straße, Wien}
         \renewcommand{\erwaehnteWerke}{Werke: Lieutenant Gustl. Novelle}
               \section[ Paul Goldmann an Arthur Schnitzler, 29. 1. {[}1901{]}]{ Paul Goldmann an Arthur Schnitzler, 29. 1. {[}1901{]}}\nopagebreak\mylabel{v}\rehead{ }\begin{ledgroupsized}[t]{13cm}\normalsize\beginnumbering \toendnotes[C]{\smallbreak\pagebreak[2]} \Standort{DLA, A:Schnitzler, HS.NZ85.1.3171.}
\physDesc{Brief, 1 Blatt, 3 Seiten, 1604 Zeichen
\newline{}Handschrift: blaue Tinte, deutsche Kurrent
\newline{}Beilage: handschriftlicher Brief Marie und Auguste
                                 Glümer, 1 Blatt, 3 Seiten, Handschrift Auguste Glümer, Bleistift,
                                 deutsche Kurrent 
\newline{}Schnitzler: 1) mit Bleistift das Jahr »901« vermerkt  2) mit rotem Buntstift eine Unterstreichung}\toendnotes[C]{\smallbreak}\pstart
           \noindent{}\raggedleft{}{\pb}\textcolor{gray}{\textbf{DESSAUERSTRASSE 19}}\oindex{Dessauer Strasse@\textbf{Dessauer Straße}|pw}\pend
           \pstart
           Berlin\oindex{Berlin@\textbf{Berlin}|pw}, 29. Januar.\pend
           \pstart\center{}Mein lieber Freund,\pend\pstart
           Auch ich war unruhig, aber es liegt kein Gund dazu vor, wie beifolgender Brief
               beweiſt. \substVorne{}\textsuperscript{\textcolor{gray}{W}}\substDazwischen{}D\substHinten{}a ich ein großes Mißtrauen gegen den behandelnden »\label{K_L03056-1v}\edtext{Wunderdoktor\pwindex{?? [behandelnder Arzt von Marie Gluemer, Anfang 1901] @\textsc{?? [behandelnder Arzt von Marie Glümer, Anfang 1901]}|pwv}}{\lemma{\textnormal{\emph{Wunderdoktor}}}\Cendnote{\textnormal{nicht ermittelt}}}\label{K_L03056-1h}« hatte, ſandte
               ich das \label{K_L03056-2v}\edtext{Mädel\pwindex{Gluemer, Marie 03.07.1867 – 16.11.1925@\textsc{Glümer, Marie} (03.07.1867 – 16.11.1925), \emph{Schauspielerin}|pwv}}{\lemma{\textnormal{\emph{Mädel}}}\Cendnote{\textnormal{siehe Paul Goldmann an Arthur Schnitzler, 22. 1. [1901]}}}\label{K_L03056-2h} zu meinem Freunde \textsc{Dr. Kuttner\pwindex{Kuttner, Arthur *~08.10.1862@\textsc{Kuttner, Arthur} (*~08.10.1862), \emph{Laryngologe}|pw}} (den \textsc{Dr. Hajek\pwindex{Hajek, Markus 25.11.1861 – 04.04.1941@\textsc{Hajek, Markus} (25.11.1861 – 04.04.1941), \emph{Mediziner, Laryngologe}|pw}} kennt u. ſchätzt). Die Viſite fand geſtern
               ſtatt. \textsc{Dr. K.\pwindex{Kuttner, Arthur *~08.10.1862@\textsc{Kuttner, Arthur} (*~08.10.1862), \emph{Laryngologe}|pwv}} telephonirte mir: Beſſerung ſei bald zu erwarten. Er glaube, daß der
               behandelnde Arzt\pwindex{?? [behandelnder Arzt von Marie Gluemer, Anfang 1901] @\textsc{?? [behandelnder Arzt von Marie Glümer, Anfang 1901]}|pwv} mit ſeinen
               Heilmitteln (\label{K_L03056-3v}\edtext{Arſenik}{\lemma{\textnormal{\emph{Arſenik}}}\Cendnote{\textnormal{Arsen}}}\label{K_L03056-3h}) im Weſentlichen auf dem
               rechten Wege ſei, wünſche {\pb}auch, daß das Fräulein\pwindex{Gluemer, Marie 03.07.1867 – 16.11.1925@\textsc{Glümer, Marie} (03.07.1867 – 16.11.1925), \emph{Schauspielerin}|pwv} weiter bei dieſem
                  Arzt\pwindex{?? [behandelnder Arzt von Marie Gluemer, Anfang 1901] @\textsc{?? [behandelnder Arzt von Marie Glümer, Anfang 1901]}|pwv} in Behandlung
               bleibe, da er großen pſychiſchen Einfluß auf ſeine Patienten habe. Die Behandlung in
               der Naſe ſei allerdings eine »Gemeinheit«. Ob Malaria vorliege, könne man nicht
               wiſſen, ſolange keine Temperatur-Meſſungen u. Blut-Unterſuchungen vorgenommen, woran
               der behandelnde Arzt\pwindex{?? [behandelnder Arzt von Marie Gluemer, Anfang 1901] @\textsc{?? [behandelnder Arzt von Marie Glümer, Anfang 1901]}|pwv} nicht
               zu denken ſcheine{\dotsfour}\pend
           \pstart
           Daß man Dich doch noch \label{K_L03056-4v}\edtext{ehrengerichtlich verfolgt}{\lemma{\textnormal{\emph{ehrengerichtlich verfolgt}}}\Cendnote{\textnormal{wegen des \emph{Lieutenant Gustl}\pwindex{Schnitzler, Arthur 15.05.1862 – 21.10.1931@\textsc{Schnitzler, Arthur} (15.05.1862 – 21.10.1931), \emph{Schriftsteller, Mediziner}!Lieutenant Gustl. Novelle1900-12-25@\strich\emph{Lieutenant Gustl. Novelle} {[}1900-12-25{]}|pwk}, siehe Paul Goldmann an Arthur Schnitzler, 11. 1. [1901]}}}\label{K_L03056-4h}, iſt {\pb}empörend! Sei nur ja recht vorſichtig
               und thue keinen Schritt, ohne vorher mit Rechts- und Landeskundigen Dich berathen zu
               haben!\pend
           \pstart
           In Eile!\pend
           \pstart
           Dein {\\[\baselineskip]}\spacefill\mbox{P. G.}\pend
           \leftskip=0em{}{\bigskip}\pstart
           \noindent{}{\pb}{[}hs. Glümer:{]} Lieber Herr Doktor,\pend
           \pstart
           Vor allem vielen Dank für Ihre Bemühungen. Wir ſind heute mit Beruhigung
               von \textsc{D\textsuperscript{r}}{ }\textsc{Kuttner}\pwindex{Kuttner, Arthur *~08.10.1862@\textsc{Kuttner, Arthur} (*~08.10.1862), \emph{Laryngologe}|pw} weggegangen. Ausführlicher werde ich Ihnen mündlich berichten. Die Krankheit,
               die {\pb}ſich plötzlich geſtern, So{\geminationn}tag{ }Nachm. brach, iſt tatſächlich im Verſchwinden und k\strikeout{l}ein Rückfall mehr zu befürchten. – Wir ſind Ihnen
               jedenfalls für dieſe Beruhigung ſehr dankbar, die wir uns ſelbſt zu verſchaffen, {\pb}wahrſcheinlich noch nicht die Energie gehabt hätten.
               – Bitte gelegentlich um ein Stückchen Ihrer freien Zeit.\pend
           \pstart
           Mit beſten Empfehlungen für Ihre Frau Ma{\geminationm}a\pwindex{Schnitzler, Louise 1840-07-08 – 1911-09-09@\textsc{Schnitzler, Louise} (1840-07-08 – 1911-09-09)|pwv}{ }{\\[\baselineskip]}Ihre ergebenen {\\[\baselineskip]}\spacefill\mbox{Marie + GustiGlümer}\pend
           \leftskip=0em{}
         
         \endnumbering\mylabel{h}\end{ledgroupsized}  \newcommand{\dateiname}{L03056}\newcommand{\titel}{Paul Goldmann an Arthur Schnitzler, 29. 1. [1901]}\newcommand{\editorInnen}{Martin Anton Müller und Laura Untner}%% latex-leseansicht-abspann.tex
%% Abspann für die Leseansicht.
%% Der Schalter \ifkorrekturansicht ist bereits durch den Vorspann gesetzt.

%% latex-abspann.tex
%% Gemeinsamer Abspann für Korrekturansicht und Leseansicht.
%% Setzt den Schalter \ifkorrekturansicht voraus (gesetzt in den
%% einbindenden Dateien latex-korrekturansicht-abspann.tex bzw.
%% latex-leseansicht-abspann.tex).
%% ---------------------------------------------------------------

\normalsize

% Das esempio-Environment wird nur in der Leseansicht benötigt
\ifkorrekturansicht\else
\newenvironment{esempio}[3]%
{
    \vspace{1.5ex}
    \rlap{\underline{#1}}
    \par
    \setlength{\parindent}{0cm}
    \nopagebreak
    \leftskip=#2cm
    \rightskip=#3cm
}
{
    \par
}
\fi

\doendnotes{C}
\bigskip
\vfill

\clearpage

\footnotesize

\ifkorrekturansicht
  \lohead{\textsc{register}}
\fi

% theindex-Environment neu definieren ohne reledmac
\makeatletter
\renewenvironment{theindex}{%
  \ifkorrekturansicht
    \section*{\indexname}%
  \else
    \subsubsection*{Index der erwähnten Entitäten}%
  \fi
  \setlength{\parindent}{0pt}%
  \setlength{\parskip}{0pt plus 0.3pt}%
  \let\item\@idxitem
}{%
  \ifkorrekturansicht\clearpage\fi
}
\makeatother

\IfFileExists{\jobname-pw.ind}{\input{\jobname-pw.ind}}{}

% Quellenangabe nur in der Leseansicht
\ifkorrekturansicht\else
% Fallback-Definitionen, falls die .tex-Datei \titel etc. nicht gesetzt hat
\providecommand{\titel}{}
\providecommand{\editorInnen}{}
\providecommand{\dateiname}{\jobname}

\vspace{3cm}

\vfill

\footnotesize
\textsc{Quelle}: \titel. Herausgegeben von {\editorInnen}. In: \emph{Arthur Schnitzler: Briefwechsel mit Autorinnen und Autoren}.
 Digitale Edition, https://schnitzler-briefe.acdh.oeaw.ac.at/{\dateiname}.html (Stand \today)
\fi

\end{document}


      