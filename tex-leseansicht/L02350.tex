%% latex-leseansicht-vorspann.tex
%% Vorspann für die Leseansicht.
%% Lädt die gemeinsame Datei latex-vorspann.tex mit nicht gesetztem Schalter.

\newif\ifkorrekturansicht
\korrekturansichtfalse

\input{../tex-inputs/latex-vorspann}


         
         \renewcommand{\erwaehntePersonen}{Personen: Richard Beer-Hofmann, Samuel Fischer, Hugo von Hofmannsthal, Lucy von Jacobi, Olga Schnitzler}
         \renewcommand{\erwaehnteOrte}{Orte: Abtenau, Bad Aussee, Bayern, Gartengasse, Kurhaus Abtenau-Bad, Salzburg, Steiermark, Sternwartestraße, Wien}
         \renewcommand{\erwaehnteWerke}{Werke: Gesammelte Werke}
               \section[Arthur Schnitzler an Richard Beer-Hofmann, 16. 7. 1920]{ Arthur Schnitzler an Richard Beer-Hofmann, 16. 7. 1920}\nopagebreak\mylabel{v}\rehead{ }\begin{ledgroupsized}[t]{13cm}\normalsize\beginnumbering \toendnotes[C]{\smallbreak\pagebreak[2]} \Standort{YCGL, MSS 31.}
\physDesc{Brief, 1 Blatt, 2 Seiten, Umschlag, 1112 Zeichen
\newline{}Handschrift: Bleistift, lateinische Kurrent
\newline{}Versand: Stempel: »\nobreak{}Wien 1\textcolor{gray}{1}0, 16. VII. 20, \textcolor{gray}{6}\nobreak{}«.  }\buchAbdrucke{\weitereDrucke{Arthur Schnitzler, Richard Beer-Hofmann: \emph{Briefwechsel 1891–1931}. Hg. Konstanze Fliedl. Wien, Zürich: \emph{Europaverlag} 1992, S. 227–228.} }\toendnotes[C]{\smallbreak}\pstart{}{\pb}A. S. Wien XVIII
                     Sternwartestr 71\oindex{Sternwartestrasse@\textbf{Sternwartestraße}|pw}.\pend{}{\bigskip}\pstart{}{\pb}Hrn Dr. Richard Beer Hofmann\pend{}\pstart{}Markt Aussee\oindex{Bad Aussee@\textbf{Bad Aussee}|pw}\pend{}\pstart{}Gartengasse\oindex{Gartengasse@\textbf{Gartengasse}|pw}\pend{}\pstart{}Steiermark\oindex{Steiermark@\textbf{Steiermark}|pw}\pend{}{\bigskip}\pstart
           \raggedleft{}{\pb}Wien\oindex{Wien@\textbf{Wien}|pw}{ }16. 7. 1920\pend
           \pstart{}lieber Richard,\pend\pstart
           über den Vorschlag Fischer\pwindex{Fischer, Samuel 24.12.1859 – 15.10.1934@\textsc{Fischer, Samuel} (24.12.1859 – 15.10.1934), \emph{Verleger}|pw} denk ich wie Sie,
               daß uns unter den augenblicklichen Verhältnissen kaum was übrig bleiben wird als
               anzunehmen, ist klar. Gegen all das wird man sich erst wehren können, wenn eine
               völlige in jeder Hinsicht gewährleistete und gesetzlich geschützte Solidarität der
               Schriftsteller bestehen wird – und ob nicht sogar da{\geminationn}
               die Unternehmersolidarität den Sieg davontragen wird, bleibt fraglich. Hugo\pwindex{Hofmannsthal, Hugo von 1874-02-01 – 1929-07-15@\textsc{Hofmannsthal, Hugo von} (1874-02-01 – 1929-07-15), \emph{Schriftsteller}|pw} war \label{KLL02350_Beer-Hofmann-1v}\edtext{gestern}{\lemma{\textnormal{\emph{gestern}}}\Cendnote{\textnormal{siehe A. S.: \emph{Tagebuch}, 15. 7. 1920}}}\label{KLL02350_Beer-Hofmann-1h} bei mir; er ist ungefähr der gleichen Ansicht. Ich bin eben wieder in einer
               »scharfen« Correspondenz mit Fischer\pwindex{Fischer, Samuel 24.12.1859 – 15.10.1934@\textsc{Fischer, Samuel} (24.12.1859 – 15.10.1934), \emph{Verleger}|pw}
               begriffen, wegen meiner »Gesa{\geminationm}elten\pwindex{Schnitzler, Arthur 15.05.1862 – 21.10.1931@\textsc{Schnitzler, Arthur} (15.05.1862 – 21.10.1931), \emph{Schriftsteller, Mediziner}!Gesammelte Werke1912 – 1922@\strich\emph{Gesammelte Werke} {[}1912 – 1922{]}|pw}«, ich »reagiere ab« aber sonst ko{\geminationm}t nicht viel {\pb}dabei
               heraus. –\pend
           \pstart
           Unsre Sommerpläne sind noch immer so vag als möglich. Frau Lucy von Jacoby\pwindex{Jacobi, Lucy von 08.09.1887 – 24.04.1956@\textsc{Jacobi, Lucy von} (08.09.1887 – 24.04.1956), \emph{Schriftstellerin, Dramaturgin, Übersetzerin}|pw} wohnt jetzt bei uns; wahrscheinlich wird Olga\pwindex{Schnitzler, Olga 17.01.1882 – 13.01.1970@\textsc{Schnitzler, Olga} (17.01.1882 – 13.01.1970), \emph{Schauspielerin, Sängerin}|pw} mit ihr nach Salzburg\oindex{Salzburg@\textbf{Salzburg}|pw} oder Bayern\oindex{Bayern@\textbf{Bayern}|pw} fahren, und es ist
               möglich, dſs man sich etwa am 15. August irgendwo trifft. Abtenau\oindex{Abtenau@\textbf{Abtenau}|pw} (Curh\substVorne{}\textsuperscript{ot}\substDazwischen{}aus\substHinten{}\oindex{Kurhaus Abtenau-Bad@\textbf{Kurhaus Abtenau-Bad}|pw}) wird in Erwägung gezogen.\pend
           \pstart
           Lassen Sie sichs wohl ergehen mein lieber Richard grüßen Sie die Ihren\pend
           \pstart
           Von Herzen Ihr{\\[\baselineskip]}\spacefill\mbox{Arthur}\pend
           \leftskip=0em{}
         
         \endnumbering\mylabel{h}\end{ledgroupsized}  \newcommand{\dateiname}{L02350}\newcommand{\titel}{Arthur Schnitzler an Richard Beer-Hofmann, 16. 7. 1920}\newcommand{\editorInnen}{Martin Anton Müller und Gerd-Hermann Susen}%% latex-leseansicht-abspann.tex
%% Abspann für die Leseansicht.
%% Der Schalter \ifkorrekturansicht ist bereits durch den Vorspann gesetzt.

%% latex-abspann.tex
%% Gemeinsamer Abspann für Korrekturansicht und Leseansicht.
%% Setzt den Schalter \ifkorrekturansicht voraus (gesetzt in den
%% einbindenden Dateien latex-korrekturansicht-abspann.tex bzw.
%% latex-leseansicht-abspann.tex).
%% ---------------------------------------------------------------

\normalsize

% Das esempio-Environment wird nur in der Leseansicht benötigt
\ifkorrekturansicht\else
\newenvironment{esempio}[3]%
{
    \vspace{1.5ex}
    \rlap{\underline{#1}}
    \par
    \setlength{\parindent}{0cm}
    \nopagebreak
    \leftskip=#2cm
    \rightskip=#3cm
}
{
    \par
}
\fi

\doendnotes{C}
\bigskip
\vfill

\clearpage

\footnotesize

\ifkorrekturansicht
  \lohead{\textsc{register}}
\fi

% theindex-Environment neu definieren ohne reledmac
\makeatletter
\renewenvironment{theindex}{%
  \ifkorrekturansicht
    \section*{\indexname}%
  \else
    \subsubsection*{Index der erwähnten Entitäten}%
  \fi
  \setlength{\parindent}{0pt}%
  \setlength{\parskip}{0pt plus 0.3pt}%
  \let\item\@idxitem
}{%
  \ifkorrekturansicht\clearpage\fi
}
\makeatother

\IfFileExists{\jobname-pw.ind}{\input{\jobname-pw.ind}}{}

% Quellenangabe nur in der Leseansicht
\ifkorrekturansicht\else
% Fallback-Definitionen, falls die .tex-Datei \titel etc. nicht gesetzt hat
\providecommand{\titel}{}
\providecommand{\editorInnen}{}
\providecommand{\dateiname}{\jobname}

\vspace{3cm}

\vfill

\footnotesize
\textsc{Quelle}: \titel. Herausgegeben von {\editorInnen}. In: \emph{Arthur Schnitzler: Briefwechsel mit Autorinnen und Autoren}.
 Digitale Edition, https://schnitzler-briefe.acdh.oeaw.ac.at/{\dateiname}.html (Stand \today)
\fi

\end{document}


      