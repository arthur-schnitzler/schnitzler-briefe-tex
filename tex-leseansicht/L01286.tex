%% latex-korrekturansicht-vorspann.tex
%% Vorspann für die Korrekturansicht.
%% Lädt die gemeinsame Datei latex-vorspann.tex mit gesetztem Schalter.

\newif\ifkorrekturansicht
\korrekturansichttrue

\input{../tex-inputs/latex-vorspann}


\section[Hermann Bahr an Arthur Schnitzler, 4. 4. {[}1903{]}]{L01286 Hermann Bahr an Arthur Schnitzler, 4. 4. {[}1903{]}}
\nopagebreak\mylabel{L01286v}
\rehead{ }\normalsize\beginnumbering\briefempfaengerindex{Schnitzler, Arthur@\textsc{Schnitzler, Arthur}!zzzBahr, Hermann@\emph{von Hermann Bahr}!1903-04-041@{4. 4. 1903}|(be}
\toendnotes[C]{\smallbreak\pagebreak[2]}\Standort{CUL, Schnitzler, B 5b.}
\physDesc{Brief, 2 Blätter, 3 Seiten, 1049 Zeichen
\newline{}Handschrift: schwarze Tinte, deutsche Kurrent
\newline{}Ordnung: mit Bleistift von unbekannter Hand nummeriert:
                                    »98« }
\buchAbdrucke{\weitereDrucke{Hermann Bahr, Arthur Schnitzler: \emph{Briefwechsel, Aufzeichnungen, Dokumente (1891–1931)}. Göttingen: \emph{Wallstein} 2018, S. 258–259.} }\toendnotes[C]{\smallbreak}
\pstart
           \raggedleft{}{\pb}4. 4.\pend
           
\pstart\center{}Lieber Arthur!\pend\vspace{0.5em}
\pstart
           Nächſtens erſcheint von mir bei Fiſcher\pwindex{Fischer, Samuel 24.12.1859 – 15.10.1934@\textsc{Fischer, Samuel} (24.12.1859 – 15.10.1934), \emph{Verleger/Verlegerin}|pw} ein
               Band »Rezenſionen\pwindex{Rezensionen. Wiener Theater 1901 bis 1903@\emph{Rezensionen. Wiener Theater 1901 bis 1903}|pw}«, Kritiken von 1901–1903. Mir
               wäre lieb, ihn Dir widmen zu dürfen. Macht Dir das aber keinen Spaß oder iſt es Dir
               aus irgend einem Grunde, den Du mir gar nicht zu nennen brauchſt, (vielleicht, weil
               man wieder Clique sagen wird), zuwider oder auch nur unbequem, kurz wenn Du irgend
               das Gefühl haſst: Lieber nicht, ſo werde ich weder beleidigt noch gekränkt noch
               verſchnupft noch irgend unangenehm berührt oder gegen Dich verändert ſein, ſo weit
               kennſt Du mich doch!{\pb}\pend
           
\pstart
           Im Neuen Wiener Journal\pwindex{Neues Wiener Journal@\emph{Neues Wiener Journal}|pw}{ }ſteht, daß Du \label{K_L01286-1v}\edtext{geheiratet haſt}{\lemma{\textnormal{\emph{geheiratet haſt}}}\Cendnote{\textnormal{\emph{Neues Wiener Journal}\pwindex{Neues Wiener Journal@\emph{Neues Wiener Journal}|pwk}, Jg. 11, Nr. 3389,
                        3. 4. 1903, S. 6: »Wie uns
                        mitgethei{[}l{]}t wird, hat sich Dr. Arthur \so{Schnitzler} dieser Tage in aller Stille \so{vermählt}. Seine Gattin\pwindex{Schnitzler, Olga 17.01.1882 – 13.01.1970@\textsc{Schnitzler, Olga} (17.01.1882 – 13.01.1970), \emph{Schauspieler/Schauspielerin, Sänger/Sängerin}|pwv} ist eine junge Dame, die noch vor Kurzem das
                        Conservatorium\oindex{Konservatorium der Gesellschaft der Musikfreunde@\textbf{Konservatorium der Gesellschaft der Musikfreunde}, \emph{Konservatorium (K.KON)}|pw} besucht hat.« Am
                  Folgetag stand auf S. 8: »Herr Dr. Arthur \so{Schnitzler} theilt uns mit, daß er noch immer
                     unvermählt ist.«}}}\label{K_L01286-1}. Vielleicht iſt es aber nicht wahr. Nach meinen
               Erfahrungen einer Ehe von acht Jahren kann man Dir in beiden Fällen herzlich
               gratulieren, was hiemit geſchieht.\pend
           
\pstart
           Damit Du aber ſiehſt, wie man in dieſer Inſtitution herabkommt, wiſſe, daß ich Deinem
                  \label{K_L01286-2v}\edtext{Bernhardiner}{\lemma{\textnormal{\emph{Bernhardiner}}}\Cendnote{\textnormal{Siehe Paul Goldmann an Arthur Schnitzler, 17. 4. [1902].
               }}}\label{K_L01286-2} leider entſagen muß, vorläufig wenigſtens, da meine Frau\pwindex{Bahr, Rosa 26.10.1871 – 17.02.1940@\textsc{Bahr, Rosa} (26.10.1871 – 17.02.1940), \emph{Schauspieler/Schauspielerin}|pwv} gerade wieder die Laune hat, alle
               Hunde zu haßen.\pend
           
\pstart
           Herzlichſt{\\[\baselineskip]}Dein{\\[\baselineskip]}\spacefill\mbox{Hermann}\pend
           \leftskip=0em{}
\pstart
           \noindent{}{\pb}Die Widmung soll lauten:\pend
           \begin{mdbar}
\pstart
           \noindent{}\centering{}Meinem lieben Arthur Schnitzler{\\}nach zwölf Jahren.\pend
           \end{mdbar}\selectlanguage{ngerman}\endnumbering\briefempfaengerindex{Schnitzler, Arthur@\textsc{Schnitzler, Arthur}!zzzBahr, Hermann@\emph{von Hermann Bahr}!1903-04-041@{4. 4. 1903}|)be}\mylabel{L01286h}  \normalsize

\doendnotes{C}
\bigskip
\vfill

\clearpage

\footnotesize

\lohead{\textsc{register}}

% Definiere theindex-Environment komplett neu ohne reledmac
\makeatletter
\renewenvironment{theindex}{%
  \section*{\indexname}%
  \setlength{\parindent}{0pt}%
  \setlength{\parskip}{0pt plus 0.3pt}%
  \let\item\@idxitem
}{%
  \clearpage
}
\makeatother

\IfFileExists{\jobname-pw.ind}{\input{\jobname-pw.ind}}{}

\end{document}

      