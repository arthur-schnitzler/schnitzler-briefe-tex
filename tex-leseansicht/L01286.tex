%% latex-leseansicht-vorspann.tex
%% Vorspann für die Leseansicht.
%% Lädt die gemeinsame Datei latex-vorspann.tex mit nicht gesetztem Schalter.

\newif\ifkorrekturansicht
\korrekturansichtfalse

\input{../tex-inputs/latex-vorspann}


\section[Hermann Bahr an Arthur Schnitzler, 4. 4. [1903]]{L01286 Hermann Bahr an Arthur Schnitzler, 4. 4. [1903]}
\nopagebreak\mylabel{L01286v}
\rehead{ }\normalsize\beginnumbering\briefempfaengerindex{Schnitzler, Arthur@\textsc{Schnitzler, Arthur}!zzzBahr, Hermann@\emph{von Hermann Bahr}!1903-04-041@{4. 4. 1903}|(be}
\toendnotes[C]{\smallbreak\pagebreak[2]}
\correspDesc{Versand  durch Hermann Bahr am 4. 4. 1903 in Wien
\newline{}Erhalt  durch Arthur Schnitzler im Zeitraum [4. 4. 1903
                  – 8. 4. 1903?] in Wien}\toendnotes[C]{\smallbreak}
\Standort{CUL, Schnitzler, B 5b.}
\physDesc{Brief, 2 Blätter, 3 Seiten, 1049 Zeichen
\newline{}Handschrift: schwarze Tinte, deutsche Kurrent
\newline{}Ordnung: mit Bleistift von unbekannter Hand nummeriert:
                                    »98« }
\buchAbdrucke{\weitereDrucke{Hermann Bahr, Arthur Schnitzler: \emph{Briefwechsel, Aufzeichnungen, Dokumente (1891–1931)}. Herausgegeben von Kurt Ifkovits und Martin Anton Müller. Göttingen: \emph{Wallstein} 2018, S. 258–259.} }\toendnotes[C]{\smallbreak}
\pstart
           \raggedleft{}{\pb}4. 4.\pend
           
\pstart\center{}Lieber Arthur!\pend\vspace{0.5em}
\pstart
           Nächſtens erſcheint von mir bei Fiſcher\pwindex{Fischer, Samuel 24.\,12.\,1859 Liptovský Mikuláš – 15.\,10.\,1934 Berlin@\textsc{Fischer, Samuel} (24.\,12.\,1859 Liptovský Mikuláš – 15.\,10.\,1934 Berlin), \emph{Verleger}|pw} ein
               Band »Rezenſionen\pwindex{Bahr, Hermann 19.\,7.\,1863 Linz – 15.\,1.\,1934 München@\textsc{Bahr, Hermann} (19.\,7.\,1863 Linz – 15.\,1.\,1934 München), \emph{Schriftsteller, Kritiker}!Rezensionen. Wiener Theater 1901 bis 1903@\strich\emph{Rezensionen. Wiener Theater 1901 bis 1903}|pw}«, Kritiken von 1901–1903. Mir
               wäre lieb, ihn Dir widmen zu dürfen. Macht Dir das aber keinen Spaß oder iſt es Dir
               aus irgend einem Grunde, den Du mir gar nicht zu nennen brauchſt, (vielleicht, weil
               man wieder Clique sagen wird), zuwider oder auch nur unbequem, kurz wenn Du irgend
               das Gefühl haſst: Lieber nicht,{ }ſo werde ich weder beleidigt noch gekränkt noch
               verſchnupft noch irgend unangenehm berührt oder gegen Dich verändert{ }ſein,{ }ſo weit
               kennſt Du mich doch!\pend
           
\pstart
           {\pb}Im Neuen Wiener Journal\pwindex{Neues Wiener Journal@\emph{Neues Wiener Journal}|pw}{ }ſteht, daß Du \label{K_L01286-1v}\edtext{geheiratet haſt}{\lemma{\textnormal{\emph{geheiratet hast}}}\Cendnote{\textnormal{\emph{Neues Wiener Journal}\pwindex{Neues Wiener Journal@\emph{Neues Wiener Journal}|pwk}, Jg. 11, Nr. 3389,
                        3. 4. 1903, S. 6: »Wie uns
                        mitgethei{[}l{]}t wird, hat sich Dr. Arthur \so{Schnitzler} dieser Tage in aller Stille \so{vermählt}. Seine Gattin\pwindex{Schnitzler, Olga 17.\,1.\,1882 Wien – 13.\,1.\,1970 Lugano@\textsc{Schnitzler, Olga} (17.\,1.\,1882 Wien – 13.\,1.\,1970 Lugano), \emph{Schauspielerin, Sängerin}|pwv} ist eine junge Dame, die noch vor Kurzem das
                        Conservatorium\oindex{Wien@\textbf{Wien}!I., Innere Stadt@\textbf{I., Innere Stadt}!Konservatorium der Gesellschaft der Musikfreunde@\textbf{Konservatorium der Gesellschaft der Musikfreunde}, \emph{Konservatorium}|pw} besucht hat.« Am
                  Folgetag stand auf S. 8: »Herr Dr. Arthur \so{Schnitzler} theilt uns mit, daß er noch immer
                     unvermählt ist.«}}}\label{K_L01286-1}. Vielleicht iſt es aber nicht wahr. Nach meinen
               Erfahrungen einer Ehe von acht Jahren kann man Dir in beiden Fällen herzlich
               gratulieren, was hiemit geſchieht.\pend
           
\pstart
           Damit Du aber{ }ſiehſt, wie man in dieſer Inſtitution herabkommt, wiſſe, daß ich Deinem
                  \label{K_L01286-2v}\edtext{Bernhardiner}{\lemma{\textnormal{\emph{Bernhardiner}}}\Cendnote{\textnormal{Siehe XXXX Auszeichnungsfehler: Dokument L03204 nicht gefunden.
               }}}\label{K_L01286-2} leider entſagen muß, vorläufig wenigſtens, da meine Frau\pwindex{Bahr, Rosa 26.\,10.\,1871 Prag – 17.\,2.\,1940 Berlin@\textsc{Bahr, Rosa} (26.\,10.\,1871 Prag – 17.\,2.\,1940 Berlin), \emph{Schauspielerin}|pwv} gerade wieder die Laune hat, alle
               Hunde zu haßen.\pend
           
\pstart
           Herzlichſt{\\[\baselineskip]}Dein{\\[\baselineskip]}\spacefill\mbox{Hermann}\pend
           \leftskip=0em{}
\pstart
           \noindent{}{\pb}Die Widmung soll lauten:\pend
           \begin{mdbar}
\pstart
           \noindent{}\centering{}Meinem lieben Arthur Schnitzler{\\}nach zwölf Jahren.\pend
           \end{mdbar}\selectlanguage{ngerman}\endnumbering\briefempfaengerindex{Schnitzler, Arthur@\textsc{Schnitzler, Arthur}!zzzBahr, Hermann@\emph{von Hermann Bahr}!1903-04-041@{4. 4. 1903}|)be}\mylabel{L01286h}  \newcommand{\dateiname}{L01286}\newcommand{\titel}{Hermann Bahr an Arthur Schnitzler, 4. 4. [1903]}\newcommand{\editorInnen}{Herausgegeben von Martin Anton Müller}%% latex-leseansicht-abspann.tex
%% Abspann für die Leseansicht.
%% Der Schalter \ifkorrekturansicht ist bereits durch den Vorspann gesetzt.

%% latex-abspann.tex
%% Gemeinsamer Abspann für Korrekturansicht und Leseansicht.
%% Setzt den Schalter \ifkorrekturansicht voraus (gesetzt in den
%% einbindenden Dateien latex-korrekturansicht-abspann.tex bzw.
%% latex-leseansicht-abspann.tex).
%% ---------------------------------------------------------------

\normalsize

% Das esempio-Environment wird nur in der Leseansicht benötigt
\ifkorrekturansicht\else
\newenvironment{esempio}[3]%
{
    \vspace{1.5ex}
    \rlap{\underline{#1}}
    \par
    \setlength{\parindent}{0cm}
    \nopagebreak
    \leftskip=#2cm
    \rightskip=#3cm
}
{
    \par
}
\fi

\doendnotes{C}
\bigskip
\vfill

\clearpage

\footnotesize

\ifkorrekturansicht
  \lohead{\textsc{register}}
\fi

% theindex-Environment neu definieren ohne reledmac
\makeatletter
\renewenvironment{theindex}{%
  \ifkorrekturansicht
    \section*{\indexname}%
  \else
    \subsubsection*{Index der erwähnten Entitäten}%
  \fi
  \setlength{\parindent}{0pt}%
  \setlength{\parskip}{0pt plus 0.3pt}%
  \let\item\@idxitem
}{%
  \ifkorrekturansicht\clearpage\fi
}
\makeatother

\IfFileExists{\jobname-pw.ind}{\input{\jobname-pw.ind}}{}

% Quellenangabe nur in der Leseansicht
\ifkorrekturansicht\else
% Fallback-Definitionen, falls die .tex-Datei \titel etc. nicht gesetzt hat
\providecommand{\titel}{}
\providecommand{\editorInnen}{}
\providecommand{\dateiname}{\jobname}

\vspace{3cm}

\vfill

\footnotesize
\textsc{Quelle}: \titel. Herausgegeben von {\editorInnen}. In: \emph{Arthur Schnitzler: Briefwechsel mit Autorinnen und Autoren}.
 Digitale Edition, https://schnitzler-briefe.acdh.oeaw.ac.at/{\dateiname}.html (Stand \today)
\fi

\end{document}


