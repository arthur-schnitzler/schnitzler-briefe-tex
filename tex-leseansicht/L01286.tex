%% latex-leseansicht-vorspann.tex
%% Vorspann für die Leseansicht.
%% Lädt die gemeinsame Datei latex-vorspann.tex mit nicht gesetztem Schalter.

\newif\ifkorrekturansicht
\korrekturansichtfalse

\input{../tex-inputs/latex-vorspann}


               \section[Hermann Bahr an Arthur Schnitzler, 4. 4. {[}1903{]}]{ Hermann Bahr an Arthur Schnitzler, 4. 4. {[}1903{]}}\nopagebreak\mylabel{v}\rehead{ }\begin{ledgroupsized}[t]{13cm}\normalsize\beginnumbering\briefempfaengerindex{Schnitzler, Arthur@\textsc{Schnitzler, Arthur}!zzzBahr, Hermann@\emph{von Hermann Bahr}!1903-04-041@{4. 4. 1903}|(be} \toendnotes[C]{\smallbreak\pagebreak[2]} \Standort{CUL, Schnitzler, B 5b.}
\physDesc{Brief, 2 Blätter, 3 Seiten
\newline{}Handschrift: schwarze Tinte, deutsche Kurrent\newline{}Ordnung: mit Bleistift von unbekannter Hand nummeriert:
                                    »98« }\buchAbdrucke{\weitereDrucke{Hermann Bahr, Arthur Schnitzler: \emph{Briefwechsel, Aufzeichnungen, Dokumente (1891–1931)}. Hg. Kurt Ifkovits und Martin Anton Müller. Göttingen: \emph{Wallstein} 2018, S. 258–259.} }\toendnotes[C]{\smallbreak}\pstart
           \raggedleft{}{\pb}4. 4.\pend
           \pstart\center{}Lieber Arthur!\pend\pstart
           Nächſtens erſcheint von mir bei Fiſcher\pwindex{Fischer, Samuel 24.12.1859 – 15.10.1934@\textsc{Fischer, Samuel} (24.12.1859 – 15.10.1934), \emph{Verleger}|pw} ein Band
                  »Rezenſionen\pwindex{Bahr, Hermann 19.07.1863 – 15.01.1934@\textsc{Bahr, Hermann} (19.07.1863 – 15.01.1934), \emph{Schriftsteller, Kritiker}!Rezensionen. Wiener Theater 1901 bis 19031903@\strich\emph{Rezensionen. Wiener Theater 1901 bis 1903} {[}1903{]}|pw}«, Kritiken von 1901–1903. Mir wäre
               lieb, ihn Dir widmen zu dürfen. Macht Dir das aber keinen Spaß oder iſt es Dir aus
               irgend einem Grunde, den Du mir gar nicht zu nennen brauchſt, (vielleicht, weil man
               wieder Clique sagen wird), zuwider oder auch nur unbequem, kurz wenn Du irgend das
               Gefühl haſst: Lieber nicht, ſo werde ich weder beleidigt noch gekränkt noch
               verſchnupft noch irgend unangenehm berührt oder gegen Dich verändert ſein, ſo weit
               kennſt Du mich doch!{\pb}\pend
           \pstart
           Im Neuen Wiener Journal\pwindex{Neues Wiener Journal1893 – 1939@\emph{Neues Wiener Journal}|pw}{ }ſteht, daß Du \label{K_L01286_1v}\edtext{geheiratet haſt}{\lemma{\textnormal{\emph{geheiratet haſt}}}\Cendnote{\textnormal{\emph{Neues Wiener Journal}\pwindex{Neues Wiener Journal1893 – 1939@\emph{Neues Wiener Journal}|pwk}, Jg. 11, Nr. 3389,
                        3. 4. 1903, S. 6: »Wie uns
                        mitgethei{[}l{]}t wird, hat sich Dr. Arthur \so{Schnitzler} dieser Tage in aller Stille \so{vermählt}. Seine Gattin\pwindex{Schnitzler, Olga 17.01.1882 – 13.01.1970@\textsc{Schnitzler, Olga} (17.01.1882 – 13.01.1970), \emph{Schauspielerin, Sängerin}|pwv} ist eine junge Dame, die noch vor Kurzem das
                        Conservatorium\oindex{Konservatorium der Gesellschaft der Musikfreunde@\textbf{Konservatorium der Gesellschaft der Musikfreunde}|pw} besucht hat.« Am
                  Folgetag stand auf S. 8: »Herr Dr. Arthur \so{Schnitzler} theilt uns mit, daß er noch immer
                     unvermählt ist.«}}}\label{K_L01286_1h}. Vielleicht iſt es aber nicht wahr. Nach meinen
               Erfahrungen einer Ehe von acht Jahren kann man Dir in beiden Fällen herzlich
               gratulieren, was hiemit geſchieht.\pend
           \pstart
           Damit Du aber ſiehſt, wie man in dieſer Inſtitution herabkommt, wiſſe, daß ich Deinem
                  \label{K_L01286_2v}\edtext{Bernhardiner}{\lemma{\textnormal{\emph{Bernhardiner}}}\Cendnote{\textnormal{Schnitzler\pwindex{Schnitzler, Arthur 15.05.1862 – 21.10.1931@\textsc{Schnitzler, Arthur} (15.05.1862 – 21.10.1931), \emph{Schriftsteller, Mediziner}|pwk} besaß für kurze Zeit, vermutlich ab
                  dem 23. 3. 1902, einen
                  Bernhardiner namens »Bern«. Im Oktober wurde er in dem im gleichen
                  Monat eröffneten Tierschutzhaus\oindex{Tierschutzhaus@\textbf{Tierschutzhaus}|pwk} des \emph{Wiener Tierschutz-Vereins}\orgindex{Wiener Tierschutz-Verein@Wiener Tierschutz-Verein|pwk} behandelt; Mitte
                     Dezember erneut. Ab Januar 1903 versucht er ihn zu
                  vermitteln, da wohnt er aber bereits nicht mehr bei ihnen (siehe Arthur Schnitzler an Richard Beer-Hofmann, 14. 1. 1903). In diesem Jahr finden sich noch drei
                  Erwähnungen im \emph{Tagebuch}\pwindex{Schnitzler, Arthur 15.05.1862 – 21.10.1931@\textsc{Schnitzler, Arthur} (15.05.1862 – 21.10.1931), \emph{Schriftsteller, Mediziner}!Tagebuch1981 – 2000@\strich\emph{Tagebuch} {[}1981 – 2000{]}|pwk} (23. 5. 1903, 18. 6. 1903 und 6. 8. 1903). Vgl.
                        \emph{Briefe} II,118.}}}\label{K_L01286_2h} leider entſagen muß,
               vorläufig wenigſtens, da meine Frau\pwindex{Bahr, Rosa 26.10.1871 – 17.02.1940@\textsc{Bahr, Rosa} (26.10.1871 – 17.02.1940), \emph{Schauspielerin}|pwv} gerade wieder die Laune hat, alle Hunde zu haßen.\pend
           \pstart
           Herzlichſt{\\[\baselineskip]}Dein{\\[\baselineskip]}\spacefill\mbox{Hermann }\pend
           \leftskip=0em{}\pstart
           \noindent{}{\pb}Die Widmung soll lauten:\pend
           \begin{mdbar}\pstart
           \noindent{}\centering{}Meinem lieben Arthur Schnitzler{\\}nach zwölf Jahren.\pend
           \end{mdbar}\endnumbering\briefempfaengerindex{Schnitzler, Arthur@\textsc{Schnitzler, Arthur}!zzzBahr, Hermann@\emph{von Hermann Bahr}!1903-04-041@{4. 4. 1903}|)be}\mylabel{h}\end{ledgroupsized}  \newcommand{\dateiname}{L01286}\newcommand{\titel}{Hermann Bahr an Arthur Schnitzler, 4. 4. [1903]}\newcommand{\editorInnen}{ Kurt Ifkovits,  Martin Anton Müller}
            \footnotesize
\begin{ledgroupsized}[t]{11.5cm}
\doendnotes{C}
\end{ledgroupsized}
         %% latex-leseansicht-abspann.tex
%% Abspann für die Leseansicht.
%% Der Schalter \ifkorrekturansicht ist bereits durch den Vorspann gesetzt.

%% latex-abspann.tex
%% Gemeinsamer Abspann für Korrekturansicht und Leseansicht.
%% Setzt den Schalter \ifkorrekturansicht voraus (gesetzt in den
%% einbindenden Dateien latex-korrekturansicht-abspann.tex bzw.
%% latex-leseansicht-abspann.tex).
%% ---------------------------------------------------------------

\normalsize

% Das esempio-Environment wird nur in der Leseansicht benötigt
\ifkorrekturansicht\else
\newenvironment{esempio}[3]%
{
    \vspace{1.5ex}
    \rlap{\underline{#1}}
    \par
    \setlength{\parindent}{0cm}
    \nopagebreak
    \leftskip=#2cm
    \rightskip=#3cm
}
{
    \par
}
\fi

\doendnotes{C}
\bigskip
\vfill

\clearpage

\footnotesize

\ifkorrekturansicht
  \lohead{\textsc{register}}
\fi

% theindex-Environment neu definieren ohne reledmac
\makeatletter
\renewenvironment{theindex}{%
  \ifkorrekturansicht
    \section*{\indexname}%
  \else
    \subsubsection*{Index der erwähnten Entitäten}%
  \fi
  \setlength{\parindent}{0pt}%
  \setlength{\parskip}{0pt plus 0.3pt}%
  \let\item\@idxitem
}{%
  \ifkorrekturansicht\clearpage\fi
}
\makeatother

\IfFileExists{\jobname-pw.ind}{\input{\jobname-pw.ind}}{}

% Quellenangabe nur in der Leseansicht
\ifkorrekturansicht\else
% Fallback-Definitionen, falls die .tex-Datei \titel etc. nicht gesetzt hat
\providecommand{\titel}{}
\providecommand{\editorInnen}{}
\providecommand{\dateiname}{\jobname}

\vspace{3cm}

\vfill

\footnotesize
\textsc{Quelle}: \titel. Herausgegeben von {\editorInnen}. In: \emph{Arthur Schnitzler: Briefwechsel mit Autorinnen und Autoren}.
 Digitale Edition, https://schnitzler-briefe.acdh.oeaw.ac.at/{\dateiname}.html (Stand \today)
\fi

\end{document}


      