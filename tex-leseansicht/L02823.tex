%% latex-leseansicht-vorspann.tex
%% Vorspann für die Leseansicht.
%% Lädt die gemeinsame Datei latex-vorspann.tex mit nicht gesetztem Schalter.

\newif\ifkorrekturansicht
\korrekturansichtfalse

\input{../tex-inputs/latex-vorspann}


         
         \renewcommand{\erwaehntePersonen}{Personen:  ?? [Totgeborener Sohn von Arthur Schnitzler und Marie Reinhard], Richard Beer-Hofmann, Mirjam Beer-Hofmann, Paula Beer-Hofmann, Rosa Freudenthal, Hermann Freudenthal, Paul Goldmann, Marie Reinhard, Josef Rosengart}
         \renewcommand{\erwaehnteInstitutionen}{Institutionen: Frankfurter Zeitung, Houghton Library}
         \renewcommand{\erwaehnteOrte}{Orte: Frankfurt am Main, Paris, Rossertstraße, Wien}
         \renewcommand{\erwaehnteWerke}{}
               \section[ Paul Goldmann an Arthur Schnitzler, 13. 9. 1897]{ Paul Goldmann an Arthur Schnitzler, 13. 9. 1897}\nopagebreak\mylabel{v}\rehead{ }\begin{ledgroupsized}[t]{13cm}\normalsize\beginnumbering\briefempfaengerindex{Schnitzler, Arthur@\textsc{Schnitzler, Arthur}!zzzGoldmann, Paul@\emph{von Paul Goldmann}!1897-09-131@{13. 9. 1897}|(be} \toendnotes[C]{\smallbreak\pagebreak[2]} \Standort{DLA, A:Schnitzler, HS.NZ85.1.3167.}
\physDesc{Brief, 1 Blatt, 4 Seiten, 2329 Zeichen
\newline{}Handschrift: blaue Tinte, deutsche Kurrent
\newline{}Beilage: eigenhändiger Brief, 1 Blatt, 2 Seiten, Handschrift: blaue
                                 Tinte, deutsche Kurrent; der Brief wurde von Schnitzler\pwindex{Schnitzler, Arthur 15.05.1862 – 21.10.1931@\textsc{Schnitzler, Arthur} (15.05.1862 – 21.10.1931), \emph{Schriftsteller, Mediziner}|pw} weitergereicht und findet sich
                                 heute in der \emph{Houghton Library}\orgindex{Houghton Library@Houghton Library|pwk},
                                    Harvard, Signatur 825.978 }\toendnotes[C]{\smallbreak}\pstart
           \noindent{}{\pb}\textcolor{gray}{\textbf{Frankfurter Zeitung}}\orgindex{Frankfurter Zeitung@Frankfurter Zeitung|pw}\hfill \textcolor{gray}{\textbf{Frankfurt a. M.\oindex{Frankfurt am Main@\textbf{Frankfurt am Main}|pw},}}{ }13. September \textcolor{gray}{\textbf{189}}7.\pend
           \pstart
           \textcolor{gray}{\textbf{und}}\pend
           \pstart
           \textcolor{gray}{\textbf{Handelsblatt.}}\pend
           \pstart
           \textcolor{gray}{\textbf{\textsc{Redaktion\orgindex{Frankfurter Zeitung@Frankfurter Zeitung|pwv}.\footnote{\noindent{}\textcolor{gray}{\textbf{Für die Redaktion\orgindex{Frankfurter Zeitung@Frankfurter Zeitung|pwv} beſtimmte Briefe und Sendungen wolle
                                 man \so{nicht} an die Perſon eines
                                 Redakteurs, ſondern ſtets \textbf{an die Redaktion der Frankfurter Zeitung\orgindex{Frankfurter Zeitung@Frankfurter Zeitung|pw}} adreſſiren.}}}}}}\pend
           \pstart
           \textcolor{gray}{\textbf{Telegramm-Adreſſe:}}\pend
           \pstart
           \textcolor{gray}{\textbf{\textsc{Zeitung\orgindex{Frankfurter Zeitung@Frankfurter Zeitung|pwv}{ }Frankfurt Main\oindex{Frankfurt am Main@\textbf{Frankfurt am Main}|pw}.}}}\pend
           \pstart\center{}Mein lieber Freund,\pend\pstart
           Erſt ſeit wenigen Stunden bin ich in Frankfurt\oindex{Frankfurt am Main@\textbf{Frankfurt am Main}|pw}.
               Ich habe den \label{K_L02823-1v}\edtext{Brief}{\lemma{\textnormal{\emph{Brief}}}\Cendnote{\textnormal{Bezug unklar}}}\label{K_L02823-1h} gleich nach \textsc{Paris\oindex{Paris@\textbf{Paris}|pw}} geſandt\strikeout{.} und hoffe, daß die Verzögerung, die
               durch meine verſpätete Ankunft in Frankfurt\oindex{Frankfurt am Main@\textbf{Frankfurt am Main}|pw}
               entſtanden iſt, keine ſtörenden Folgen hat.\pend
           \pstart
           Ich danke Dir für die lieben Mittheilungen Deines Briefes. Der \strikeout{\textcolor{gray}{×}\-\textcolor{gray}{×}\-\textcolor{gray}{×}h\textcolor{gray}{×}\-\textcolor{gray}{×}}{ }Gattin\pwindex{Freudenthal, Rosa 1862 – 18.06.1905@\textsc{Freudenthal, Rosa} (1862 – 18.06.1905)|pwv} des Rechtsgelehrten\pwindex{Freudenthal, Hermann 1852/1853 – 12.09.1925@\textsc{Freudenthal, Hermann} (1852/1853 – 12.09.1925), \emph{Rechtsanwalt}|pwv} geht es hoffentlich \label{K_L02823-2v}\edtext{beſſer}{\lemma{\textnormal{\emph{beſſer}}}\Cendnote{\textnormal{Siehe A. S.: \emph{Tagebuch}, 3. 9. 1897.
               }}}\label{K_L02823-2h}. Grüß’ ſie ſchön von mir.\pend
           \pstart
           Du ſelbſt wirſt \strikeout{hof} wohl bald die \strikeout{\textcolor{gray}{R}} Ruhe zur Arbeit {\pb}finden. Solche
               Übergangszeiten vom Sommer zum Winter ſind immer etwas unbehaglich und bei Dir drängt
               ſich gerade jetzt außergewöhnlich Vieles zuſammen. Wird ſich ſchon Alles lichten und
               klären.\pend
           \pstart
           Mein Schwager\pwindex{Rosengart, Josef 1860-02-08 – 1927-08-04@\textsc{Rosengart, Josef} (1860-02-08 – 1927-08-04), \emph{Arzt}|pwv} läßt Dich
               grüßen u. Dir ſagen, daß es lächerlich iſt, ſich über \label{K_L02823-3v}\edtext{Ohrenklingen}{\lemma{\textnormal{\emph{Ohrenklingen}}}\Cendnote{\textnormal{Schnitzler\pwindex{Schnitzler, Arthur 15.05.1862 – 21.10.1931@\textsc{Schnitzler, Arthur} (15.05.1862 – 21.10.1931), \emph{Schriftsteller, Mediziner}|pwk} litt seit
                     Herbst 1896 an Otosklerose – einer Verknöcherung des Innenohrs mit
                  zunehmender Schwerhörigkeit.}}}\label{K_L02823-3h} Sorgen zu machen. Nach ſeiner Erfahrung gibt
               es kaum einen Menſchen, deſſen Ohren ganz in Ordnung wären. Er hat mir geſagt: wenn
               ich darauf achtete, würde ich auch bald Ohrenklingen \strikeout{be\textcolor{gray}{i}} bei mir bemerken, und mir ſcheint in der That\strikeout{,}
               mehrmals am Tage, daß es auch bei mir klingt. {\pb}Wer
               wird ſich aber dabei aufhalten? Schade um jede Stunde Deines ſchönen Lebens, welche
               Du Dir dadurch verbitterſt.\pend
           \pstart
           Mein Fuß iſt geheilt. Ich bleibe wohl noch bis Ende der Woche hier\oindex{Frankfurt am Main@\textbf{Frankfurt am Main}|pwv} u. bitte Dich, mir hierher \strikeout{(\textsc{Rosse}} (\textsc{Rossertstraſse 15\oindex{Rossertstrasse@\textbf{Rossertstraße}|pw}}) zu ſchreiben, falls Du mir noch etwas zu ſagen haſt oder falls Dein \label{K_L02823-4v}\edtext{Sohn\pwindex{?? [Totgeborener Sohn von Arthur Schnitzler und Marie Reinhard] 1897-09-24 – 1897-09-24@\textsc{?? [Totgeborener Sohn von Arthur Schnitzler und Marie Reinhard]} (1897-09-24 – 1897-09-24)|pwv} ankommt}{\lemma{\textnormal{\emph{Sohn ankommt}}}\Cendnote{\textnormal{Der Sohn\pwindex{?? [Totgeborener Sohn von Arthur Schnitzler und Marie Reinhard] 1897-09-24 – 1897-09-24@\textsc{?? [Totgeborener Sohn von Arthur Schnitzler und Marie Reinhard]} (1897-09-24 – 1897-09-24)|pwkv} von Schnitzler\pwindex{Schnitzler, Arthur 15.05.1862 – 21.10.1931@\textsc{Schnitzler, Arthur} (15.05.1862 – 21.10.1931), \emph{Schriftsteller, Mediziner}|pwk}
                  und Marie Reinhard\pwindex{Reinhard, Marie 1871-03-13 – 1899-03-18@\textsc{Reinhard, Marie} (1871-03-13 – 1899-03-18), \emph{Gesangspädagogin}|pwk} wurde am 24. 9. 1897
                  totgeboren.}}}\label{K_L02823-4h}.\pend
           \pstart
           Deine Freundin\pwindex{Reinhard, Marie 1871-03-13 – 1899-03-18@\textsc{Reinhard, Marie} (1871-03-13 – 1899-03-18), \emph{Gesangspädagogin}|pwv} grüße recht
               herzlich von mir. Ich habe mich ſehr gefreut zu hören, daß es ihr gut geht.\pend
           \pstart
           Ich habe \textsc{Richards\pwindex{Beer-Hofmann, Richard 1866-07-11 – 1945-09-26@\textsc{Beer-Hofmann, Richard} (1866-07-11 – 1945-09-26), \emph{Schriftsteller}|pw}}{ }{\pb}Hausnummer vergeſſen. Du biſt wohl ſo gut, ihm den
               beifolgenden Brief zu übergeben.\pend
           \pstart
           Ich grüße Dich von Herzen {\\[\baselineskip]}Dein treuer {\\[\baselineskip]}\spacefill\mbox{Paul Goldm}\pend
           \leftskip=0em{}{\bigskip}\pstart
           \noindent{}{\pb}\textcolor{gray}{\textbf{\textsc{\textbf{Frankfurter Zeitung}}}}\orgindex{Frankfurter Zeitung@Frankfurter Zeitung|pw}\hfill \textcolor{gray}{\textbf{\textbf{Frankfurt a. M.\oindex{Frankfurt am Main@\textbf{Frankfurt am Main}|pw},}}}{ }13. September \textcolor{gray}{\textbf{189}}7.\pend
           \pstart
           \textsc{\textcolor{gray}{\textbf{und}}}\pend
           \pstart
           \textcolor{gray}{\textbf{\textsc{Handelsblatt.}}}\pend
           \pstart
           \textcolor{gray}{\textbf{\textsc{\textbf{Redaktion\orgindex{Frankfurter Zeitung@Frankfurter Zeitung|pwv}}.\footnote{\noindent{}\textcolor{gray}{\textbf{\textsc{Für die Redaktion\orgindex{Frankfurter Zeitung@Frankfurter Zeitung|pwv} beſtimmte Briefe und Sendungen
                                    wolle man \so{nicht} an die Perſon eines
                                    Redakteurs, ſondern ſtets \textbf{an die Redaktion der
                                          Frankfurter Zeitung\orgindex{Frankfurter Zeitung@Frankfurter Zeitung|pw}} adreſſiren}.}}}}}}\pend
           \pstart
           \textcolor{gray}{\textbf{\textsc{Telegramm-Adreſſe:}}}\pend
           \pstart
           \textcolor{gray}{\textbf{\textsc{\textbf{Zeitung\orgindex{Frankfurter Zeitung@Frankfurter Zeitung|pwv}{ }Frankfurt Main\oindex{Frankfurt am Main@\textbf{Frankfurt am Main}|pw}.}}}}\pend
           \pstart\center{}Mein lieber \textsc{Richard}\pwindex{Beer-Hofmann, Richard 1866-07-11 – 1945-09-26@\textsc{Beer-Hofmann, Richard} (1866-07-11 – 1945-09-26), \emph{Schriftsteller}|pw},\pend\pstart
           Erſt dieſer Tage haben meine Irrfahrten in Frankfurt\oindex{Frankfurt am Main@\textbf{Frankfurt am Main}|pw} geendet. Ich fand hier Deinen lieben Brief vor und \strikeout{ſa} erſah daraus mit inniger Freude, daß das große
                  \label{K_L02823-5v}\edtext{Ereigniß}{\lemma{\textnormal{\emph{Ereigniß}}}\Cendnote{\textnormal{Am 4. 9. 1897 war Mirjam Beer-Hofmann\pwindex{Beer-Hofmann, Mirjam 04.09.1897 – 24.12.1984@\textsc{Beer-Hofmann, Mirjam} (04.09.1897 – 24.12.1984)|pwk}, das erste Kind von Richard\pwindex{Beer-Hofmann, Richard 1866-07-11 – 1945-09-26@\textsc{Beer-Hofmann, Richard} (1866-07-11 – 1945-09-26), \emph{Schriftsteller}|pwk} und Paula Beer-Hofmann\pwindex{Beer-Hofmann, Paula 25.02.1879 – 30.10.1939@\textsc{Beer-Hofmann, Paula} (25.02.1879 – 30.10.1939)|pwk}, auf die Welt gekommen.}}}\label{K_L02823-5h} ſich
               vollzogen hat. Daß es \textsc{Mirjam\pwindex{Beer-Hofmann, Mirjam 04.09.1897 – 24.12.1984@\textsc{Beer-Hofmann, Mirjam} (04.09.1897 – 24.12.1984)|pw}} war und nicht \textsc{Jehoschuah}, überraſcht mich nicht. Es
               mußte ja \textsc{Mirjam\pwindex{Beer-Hofmann, Mirjam 04.09.1897 – 24.12.1984@\textsc{Beer-Hofmann, Mirjam} (04.09.1897 – 24.12.1984)|pw}} ſein.\pend
           \pstart
           Der alte jüdiſche Gott, auf den Du ſo große Stücke hältſt, \strikeout{ſoll} wird hoffentlich einmal an Deinem Kinde zeigen, was er kann. Er ſoll
               ein {\pb}liebes und frohes Menſchenkind daraus
               machen. Dir ſelbſt aber möge die kleine \textsc{Mirjam\pwindex{Beer-Hofmann, Mirjam 04.09.1897 – 24.12.1984@\textsc{Beer-Hofmann, Mirjam} (04.09.1897 – 24.12.1984)|pw}}{ }\strikeout{\textcolor{gray}{e}i\textcolor{gray}{ne}} nur Freuden bringen und Seelenfrieden in den düſteren Stunden des Grübel\substVorne{}\textsuperscript{s}\substDazwischen{}n\substHinten{}s und der Selbſtquälerei.\pend
           \pstart
           Ich \strikeout{\textcolor{gray}{×}\-\textcolor{gray}{×}\-\textcolor{gray}{×}\-\textcolor{gray}{×}} aber will ſie ſtets ſehr lieb haben.\pend
           \pstart
           Überbringe der Mutter\pwindex{Beer-Hofmann, Paula 25.02.1879 – 30.10.1939@\textsc{Beer-Hofmann, Paula} (25.02.1879 – 30.10.1939)|pwv} Deines
                  Kind\pwindex{Beer-Hofmann, Mirjam 04.09.1897 – 24.12.1984@\textsc{Beer-Hofmann, Mirjam} (04.09.1897 – 24.12.1984)|pwv}es meine herzlichſten
               Glückwünſche und Grüße und ſei ſelbſt von Herzen umarmt.\pend
           \pstart
           Dein treuer{\\[\baselineskip]}\spacefill\mbox{Paul Goldmann}\pend
           \leftskip=0em{}
         
         \endnumbering\mylabel{h}\end{ledgroupsized}  \newcommand{\dateiname}{L02823}\newcommand{\titel}{Paul Goldmann an Arthur Schnitzler, 13. 9. 1897}\newcommand{\editorInnen}{Martin Anton Müller und Laura Untner}%% latex-leseansicht-abspann.tex
%% Abspann für die Leseansicht.
%% Der Schalter \ifkorrekturansicht ist bereits durch den Vorspann gesetzt.

%% latex-abspann.tex
%% Gemeinsamer Abspann für Korrekturansicht und Leseansicht.
%% Setzt den Schalter \ifkorrekturansicht voraus (gesetzt in den
%% einbindenden Dateien latex-korrekturansicht-abspann.tex bzw.
%% latex-leseansicht-abspann.tex).
%% ---------------------------------------------------------------

\normalsize

% Das esempio-Environment wird nur in der Leseansicht benötigt
\ifkorrekturansicht\else
\newenvironment{esempio}[3]%
{
    \vspace{1.5ex}
    \rlap{\underline{#1}}
    \par
    \setlength{\parindent}{0cm}
    \nopagebreak
    \leftskip=#2cm
    \rightskip=#3cm
}
{
    \par
}
\fi

\doendnotes{C}
\bigskip
\vfill

\clearpage

\footnotesize

\ifkorrekturansicht
  \lohead{\textsc{register}}
\fi

% theindex-Environment neu definieren ohne reledmac
\makeatletter
\renewenvironment{theindex}{%
  \ifkorrekturansicht
    \section*{\indexname}%
  \else
    \subsubsection*{Index der erwähnten Entitäten}%
  \fi
  \setlength{\parindent}{0pt}%
  \setlength{\parskip}{0pt plus 0.3pt}%
  \let\item\@idxitem
}{%
  \ifkorrekturansicht\clearpage\fi
}
\makeatother

\IfFileExists{\jobname-pw.ind}{\input{\jobname-pw.ind}}{}

% Quellenangabe nur in der Leseansicht
\ifkorrekturansicht\else
% Fallback-Definitionen, falls die .tex-Datei \titel etc. nicht gesetzt hat
\providecommand{\titel}{}
\providecommand{\editorInnen}{}
\providecommand{\dateiname}{\jobname}

\vspace{3cm}

\vfill

\footnotesize
\textsc{Quelle}: \titel. Herausgegeben von {\editorInnen}. In: \emph{Arthur Schnitzler: Briefwechsel mit Autorinnen und Autoren}.
 Digitale Edition, https://schnitzler-briefe.acdh.oeaw.ac.at/{\dateiname}.html (Stand \today)
\fi

\end{document}


      