%% latex-leseansicht-vorspann.tex
%% Vorspann für die Leseansicht.
%% Lädt die gemeinsame Datei latex-vorspann.tex mit nicht gesetztem Schalter.

\newif\ifkorrekturansicht
\korrekturansichtfalse

\input{../tex-inputs/latex-vorspann}


\section[Arthur Schnitzler an Gustav Schwarzkopf, 31. 5. 1893]{L04101 Arthur Schnitzler an Gustav Schwarzkopf, 31. 5. 1893}
\nopagebreak\mylabel{L04101v}
\rehead{ }\normalsize\beginnumbering\briefempfaengerindex{Schwarzkopf, Gustav@\textsc{Schwarzkopf, Gustav}!zzzSchnitzler, Arthur@\emph{von Arthur Schnitzler}!1893-05-311@{31. 5. 1893}|(be}
\toendnotes[C]{\smallbreak\pagebreak[2]}
\correspDesc{Versand  durch Arthur Schnitzler am 31. 5. 1893 in Wien
\newline{}Erhalt  durch Gustav Schwarzkopf im Zeitraum [31. 5. 1893
                  – 3. 6. 1893?] in Wien}\toendnotes[C]{\smallbreak}
\Standort{CUL, Schnitzler, B 96.}
\physDesc{Kartenbrief, 456 Zeichen
\newline{}Handschrift: Bleistift, deutsche Kurrent
\newline{}Versand: Stempel: »\nobreak{}\oindex{VIII., Josefstadt@\textbf{VIII., Josefstadt}, \emph{Verwaltungsgebiet}|pwk}Wien 8/1 64, 31 5 1893, 10–11V\nobreak{}«.  }\pstart{}{\pb}Herrn \textsc{Gustav
                     Schwarzkopf}\pend{}\pstart{}Wien\oindex{Wien@\textbf{Wien}, \emph{Verwaltungsgebiet}|pw}\pend{}\pstart{}\textsc{I. Tiefer Graben 23}\oindex{Wien@\textbf{Wien}!I., Innere Stadt@\textbf{I., Innere Stadt}!Tiefer Graben 23@\textbf{Tiefer Graben 23}, \emph{Wohngebäude}|pw}\pend{}{\bigskip}\vspace{1em}
\pstart
           \noindent{}{\pb}Verehrteſter Freund, wir ſind jetzt Abends (u. zw. meiſt ſehr früh,)
               im \textsc{Café Auböck\oindex{Wien@\textbf{Wien}!I., Innere Stadt@\textbf{I., Innere Stadt}!Café Reichsrath (Inh. Karl Auböck)@\textbf{Café Reichsrath (Inh. Karl Auböck)}, \emph{Kaffeehaus}|pw}}, unter den Arkaden, \uline{hinter dem Parlament\oindex{Wien@\textbf{Wien}!I., Innere Stadt@\textbf{I., Innere Stadt}!Österreichisches Parlament@\textbf{Österreichisches Parlament}, \emph{Regierungsgebäude}|pw}}, (alſo nicht \textsc{Cafe} Wien\oindex{Wien@\textbf{Wien}!I., Innere Stadt@\textbf{I., Innere Stadt}!Café Arkaden@\textbf{Café Arkaden}, \emph{Kaffeehaus}|pw} oder \textsc{Arkadencafé\oindex{Wien@\textbf{Wien}!I., Innere Stadt@\textbf{I., Innere Stadt}!Café Arkaden@\textbf{Café Arkaden}, \emph{Kaffeehaus}|pw}}.) – Gute Luft! Es duftet nach Flieder und Abgeordneten! Man ſitzt in Freien,
                  ka{\geminationn} aber auch drin ſitzen. Sie ſehen, unzählige
               Vorzüge. – Ich bin faſt jeden Abend dort, auch \textsc{Salten\pwindex{Salten, Felix 6.\,9.\,1869 Budapest – 8.\,10.\,1945 Zürich@\textsc{Salten, Felix} (6.\,9.\,1869 Budapest – 8.\,10.\,1945 Zürich), \emph{Schriftsteller, Journalist, Chefredakteur}|pw}}, auch Richard\pwindex{Beer-Hofmann, Richard 11.\,7.\,1866 Wien – 26.\,9.\,1945 New York City@\textsc{Beer-Hofmann, Richard} (11.\,7.\,1866 Wien – 26.\,9.\,1945 New York City), \emph{Schriftsteller}|pw} u. \textsc{Loris\pwindex{Hofmannsthal, Hugo von 1.\,2.\,1874 Wien – 15.\,7.\,1929 Rodaun@\textsc{Hofmannsthal, Hugo von} (1.\,2.\,1874 Wien – 15.\,7.\,1929 Rodaun), \emph{Schriftsteller}|pw}} ko{\geminationm}en zuweilen, auch Fanjung\pwindex{Van-Jung, Leo 15.\,10.\,1866 Odessa – 2.\,7.\,1939 Riga@\textsc{Van-Jung, Leo} (15.\,10.\,1866 Odessa – 2.\,7.\,1939 Riga), \emph{Gesangspädagoge, Mathematiker}|pw} u. ſ. w. –\pend
           
\pstart
           Herzlich{\\[\baselineskip]} Ihr{\\[\baselineskip]}\spacefill\mbox{Artur Sch}\pend
           \leftskip=0em{}\selectlanguage{ngerman}\endnumbering\briefempfaengerindex{Schwarzkopf, Gustav@\textsc{Schwarzkopf, Gustav}!zzzSchnitzler, Arthur@\emph{von Arthur Schnitzler}!1893-05-311@{31. 5. 1893}|)be}\mylabel{L04101h}
\begin{anhang}
\end{anhang}\newcommand{\dateiname}{L04101}\newcommand{\titel}{Arthur Schnitzler an Gustav Schwarzkopf, 31. 5. 1893}\newcommand{\editorInnen}{Herausgegeben von Jahnke, SelmaMüller, Martin Anton}%% latex-leseansicht-abspann.tex
%% Abspann für die Leseansicht.
%% Der Schalter \ifkorrekturansicht ist bereits durch den Vorspann gesetzt.

%% latex-abspann.tex
%% Gemeinsamer Abspann für Korrekturansicht und Leseansicht.
%% Setzt den Schalter \ifkorrekturansicht voraus (gesetzt in den
%% einbindenden Dateien latex-korrekturansicht-abspann.tex bzw.
%% latex-leseansicht-abspann.tex).
%% ---------------------------------------------------------------

\normalsize

% Das esempio-Environment wird nur in der Leseansicht benötigt
\ifkorrekturansicht\else
\newenvironment{esempio}[3]%
{
    \vspace{1.5ex}
    \rlap{\underline{#1}}
    \par
    \setlength{\parindent}{0cm}
    \nopagebreak
    \leftskip=#2cm
    \rightskip=#3cm
}
{
    \par
}
\fi

\doendnotes{C}
\bigskip
\vfill

\clearpage

\footnotesize

\ifkorrekturansicht
  \lohead{\textsc{register}}
\fi

% theindex-Environment neu definieren ohne reledmac
\makeatletter
\renewenvironment{theindex}{%
  \ifkorrekturansicht
    \section*{\indexname}%
  \else
    \subsubsection*{Index der erwähnten Entitäten}%
  \fi
  \setlength{\parindent}{0pt}%
  \setlength{\parskip}{0pt plus 0.3pt}%
  \let\item\@idxitem
}{%
  \ifkorrekturansicht\clearpage\fi
}
\makeatother

\IfFileExists{\jobname-pw.ind}{\input{\jobname-pw.ind}}{}

% Quellenangabe nur in der Leseansicht
\ifkorrekturansicht\else
% Fallback-Definitionen, falls die .tex-Datei \titel etc. nicht gesetzt hat
\providecommand{\titel}{}
\providecommand{\editorInnen}{}
\providecommand{\dateiname}{\jobname}

\vspace{3cm}

\vfill

\footnotesize
\textsc{Quelle}: \titel. Herausgegeben von {\editorInnen}. In: \emph{Arthur Schnitzler: Briefwechsel mit Autorinnen und Autoren}.
 Digitale Edition, https://schnitzler-briefe.acdh.oeaw.ac.at/{\dateiname}.html (Stand \today)
\fi

\end{document}


