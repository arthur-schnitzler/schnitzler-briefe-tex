%% latex-leseansicht-vorspann.tex
%% Vorspann für die Leseansicht.
%% Lädt die gemeinsame Datei latex-vorspann.tex mit nicht gesetztem Schalter.

\newif\ifkorrekturansicht
\korrekturansichtfalse

\input{../tex-inputs/latex-vorspann}


\section[ Paul Goldmann an Arthur Schnitzler, 13. 10. 1898]{L02860 Paul Goldmann an Arthur Schnitzler,  13. 10. 1898}
\nopagebreak\mylabel{L02860v}
\rehead{ }\normalsize\beginnumbering\briefempfaengerindex{Schnitzler, Arthur@\textsc{Schnitzler, Arthur}!zzzGoldmann, Paul@\emph{von Paul Goldmann}!1898-10-131@{13. 10. 1898}|(be}
\toendnotes[C]{\smallbreak\pagebreak[2]}
\correspDesc{Versand  durch Paul Goldmann am 13. 10. 1898 in Peking
\newline{}Übermittlung  am 14. 10. 1898 in Peking
\newline{}Übermittlung  am 18. 10. 1898 in Shanghai
\newline{}Übermittlung  am 27. 10. 1898 in Hongkong
\newline{}Erhalt  durch Arthur Schnitzler im Zeitraum [15. 11. 1898 – 15. 12. 1898?] in Wien}\toendnotes[C]{\smallbreak}
\Standort{DLA, A:Schnitzler, HS.NZ85.1.3168.}
\physDesc{Postkarte, 140 Zeichen
\newline{}Handschrift: blaue Tinte, deutsche Kurrent
\newline{}Versand: 1) Stempel: »\nobreak{}\oindex{Peking@\textbf{Peking}, \emph{Hauptstadt}|pwk}Peking, 14 Oct 98\nobreak{}«.   2) Stempel: »\nobreak{}\oindex{Shanghai@\textbf{Shanghai}|pwk}Shan{[}ghai{]}, 18 Oct 98\nobreak{}«.  3) Stempel: »\nobreak{}\oindex{Shanghai@\textbf{Shanghai}|pwk}\textcolor{gray}{Shanghai}, Oc 20 98\nobreak{}«.  4) Stempel: »\nobreak{}\oindex{Hong Kong@\textbf{Hong Kong}, \emph{Hauptstadt}|pwk}{[}Ho{]}\textcolor{gray}{ng Kon}{[}g{]}, Oc 27 98\nobreak{}«. 
\newline{}Schnitzler: mit Bleistift das Jahr »98« vermerkt }\toendnotes[C]{\smallbreak}\pstart{}\textsc{{\pb}\begin{otherlanguage}{english}Austria\oindex{Österreich@\textbf{Österreich}|pw}\end{otherlanguage}.}\pend{}\pstart{}\textsc{Herrn Dr. Arthur Schnitzler}\pend{}\pstart{}\textsc{Wien\oindex{Wien@\textbf{Wien}, \emph{Verwaltungsgebiet}|pw}}\pend{}\pstart{}\textsc{IX. Frankgaſse 1\oindex{Wien@\textbf{Wien}!IX., Alsergrund@\textbf{IX., Alsergrund}!Frankgasse 1@\textbf{Frankgasse 1}, \emph{Wohngebäude}|pw}.}\pend{}{\bigskip}\vspace{1em}
\pstart
           \raggedleft{}{\pb}\textsc{Peking\oindex{Peking@\textbf{Peking}, \emph{Hauptstadt}|pw}}, 13. Oktober.\pend
           \vspace{0.5em}
\pstart
           Einen{ }ſchönen Gruß aus der chin\oindex{China@\textbf{China}|pwv}eſiſchen Hauptſtadt\oindex{Peking@\textbf{Peking}, \emph{Hauptstadt}|pwv}!\pend
           
\pstart
           Dein treuer {\\[\baselineskip]}\spacefill\mbox{P. G.}\pend
           \leftskip=0em{}\selectlanguage{ngerman}\endnumbering\briefempfaengerindex{Schnitzler, Arthur@\textsc{Schnitzler, Arthur}!zzzGoldmann, Paul@\emph{von Paul Goldmann}!1898-10-131@{13. 10. 1898}|)be}\mylabel{L02860h}  \newcommand{\dateiname}{L02860}\newcommand{\titel}{Paul Goldmann an Arthur Schnitzler, 13. 10. 1898}\newcommand{\editorInnen}{Martin Anton Müller und Laura Untner}%% latex-leseansicht-abspann.tex
%% Abspann für die Leseansicht.
%% Der Schalter \ifkorrekturansicht ist bereits durch den Vorspann gesetzt.

%% latex-abspann.tex
%% Gemeinsamer Abspann für Korrekturansicht und Leseansicht.
%% Setzt den Schalter \ifkorrekturansicht voraus (gesetzt in den
%% einbindenden Dateien latex-korrekturansicht-abspann.tex bzw.
%% latex-leseansicht-abspann.tex).
%% ---------------------------------------------------------------

\normalsize

% Das esempio-Environment wird nur in der Leseansicht benötigt
\ifkorrekturansicht\else
\newenvironment{esempio}[3]%
{
    \vspace{1.5ex}
    \rlap{\underline{#1}}
    \par
    \setlength{\parindent}{0cm}
    \nopagebreak
    \leftskip=#2cm
    \rightskip=#3cm
}
{
    \par
}
\fi

\doendnotes{C}
\bigskip
\vfill

\clearpage

\footnotesize

\ifkorrekturansicht
  \lohead{\textsc{register}}
\fi

% theindex-Environment neu definieren ohne reledmac
\makeatletter
\renewenvironment{theindex}{%
  \ifkorrekturansicht
    \section*{\indexname}%
  \else
    \subsubsection*{Index der erwähnten Entitäten}%
  \fi
  \setlength{\parindent}{0pt}%
  \setlength{\parskip}{0pt plus 0.3pt}%
  \let\item\@idxitem
}{%
  \ifkorrekturansicht\clearpage\fi
}
\makeatother

\IfFileExists{\jobname-pw.ind}{\input{\jobname-pw.ind}}{}

% Quellenangabe nur in der Leseansicht
\ifkorrekturansicht\else
% Fallback-Definitionen, falls die .tex-Datei \titel etc. nicht gesetzt hat
\providecommand{\titel}{}
\providecommand{\editorInnen}{}
\providecommand{\dateiname}{\jobname}

\vspace{3cm}

\vfill

\footnotesize
\textsc{Quelle}: \titel. Herausgegeben von {\editorInnen}. In: \emph{Arthur Schnitzler: Briefwechsel mit Autorinnen und Autoren}.
 Digitale Edition, https://schnitzler-briefe.acdh.oeaw.ac.at/{\dateiname}.html (Stand \today)
\fi

\end{document}


