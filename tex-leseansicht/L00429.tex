\input{../tex-inputs/latex-pdf-vorspann}
\begin{center}
            \textcolor{red}{ENTWURF. ENTZIFFERUNG NOCH NICHT KORREKTURGELESEN}
                      \end{center}
            
               \section[Laura Marholm an Arthur Schnitzler, 16. 4. 1895]{ Laura Marholm an Arthur Schnitzler, 16. 4. 1895}\nopagebreak\mylabel{v}\rehead{ }\begin{ledgroupsized}[t]{13cm}\normalsize\beginnumbering\briefempfaengerindex{Schnitzler, Arthur@\textsc{Schnitzler, Arthur}!zzzMarholm, Laura@\emph{von Laura Marholm}!1895-04-161@{16. 4. 1895}|(be} \toendnotes[C]{\smallbreak\pagebreak[2]} \Standort{CUL, Schnitzler, B 69.}
\physDesc{Brief, 1 Blatt, 2 Seiten
\newline{}Handschrift: schwarze Tinte, lateinische Kurrent
\newline{}Schnitzler: 1) mit Bleistift beschriftet: »\textsc{Marholm}« 2) mit rotem Buntstift eine Unterstreichung}\toendnotes[C]{\smallbreak}\pstart
           \noindent{}\raggedleft{}{\pb}Schliersee\oindex{Schliersee@\textbf{Schliersee}|pw}, Oberbaiern\oindex{Oberbayern@\textbf{Oberbayern}|pw}\pend
           \pstart
           \raggedleft{}16. April 95\pend
           \pstart\center{}Sehr geehrter Herr Professor\pend\pstart
           Ich erlaube mir Ihnen beifolgend mein »Buch der
                        Frauen\pwindex{Marholm, Laura 19.04.1854 – 06.10.1928@\textsc{Marholm, Laura} (19.04.1854 – 06.10.1928), \emph{Schriftstellerin}!Buch der Frauen1894@\strich\emph{Das Buch der Frauen} {[}1894{]}|pw}« zu übersenden, das in den Wien\oindex{Wien@\textbf{Wien}|pw}er Blättern viel besprochen worden ist und Ihnen daher
                    vielleicht nicht als ganz unbekannter Gast in die Hand kommt. Ich hätte \introOben{}dazu\introOben{} – obgleich ich weiß, das Sie das, was lebendig und
                    Lebensbeitrag in der Litteratur ist, mit aufmerksamen Blick verfolgen – doch
                    nicht den Muth \strikeout{dazu} gehabt, wenn mir nicht ein
                    gelehrter Herr in Straßburg\oindex{Strassburg@\textbf{Straßburg}|pw}, Dr. Kraft\pwindex{Kraft, Heinrich 05.07.1867 – 1944@\textsc{Kraft, Heinrich} (05.07.1867 – 1944), \emph{Mediziner}|pw} von der Frauenklinik\oindex{Universitaets-Frauenklinik@\textbf{Universitäts-Frauenklinik}|pw}, neulich geschrieben hätte, »Das Buch der Frauen\pwindex{Marholm, Laura 19.04.1854 – 06.10.1928@\textsc{Marholm, Laura} (19.04.1854 – 06.10.1928), \emph{Schriftstellerin}!Buch der Frauen1894@\strich\emph{Das Buch der Frauen} {[}1894{]}|pw}« sei ihm durch die
                    Übereinstimmung der intuitiv erfaßten Ausgangspunkte mit den anthropologischen,
                    psychologischen und physiologischen Ausgangspunkten in Havelock Ellis\pwindex{Ellis, Havelock 02.02.1859 – 08.07.1939@\textsc{Ellis, Havelock} (02.02.1859 – 08.07.1939), \emph{Schriftsteller, Sexualforscher}|pw} »Mann {\kaufmannsund} Weib\pwindex{Ellis, Havelock 02.02.1859 – 08.07.1939@\textsc{Ellis, Havelock} (02.02.1859 – 08.07.1939), \emph{Schriftsteller, Sexualforscher}!Mann und Weib1894@\strich\emph{Mann und Weib} {[}1894{]}|pw}« merkwürdig und verheißner
                    für die Sache {\pb}der Frauenkenntniß
                    selber und das Weitere, was ich zu sagen hätte. Und ich habe ja allerdings noch
                    kaum mit dem Heraussagen angefangen.\pend
           \pstart
           Ich bin ganz u. gar nicht eine gelehrte Frau und halte auch nichts davon für die
                    wirkliche Entwicklung des Weibes. Ich habe das Leben mitgelebt und einen Mann\pwindex{Hansson, Ola 12.11.1860 – 26.09.1925@\textsc{Hansson, Ola} (12.11.1860 – 26.09.1925), \emph{Schriftsteller}|pwv} gefunden, der alle meine Möglichkeiten
                    als Weib frei macht und zur Entwicklung treibt. Das ist alles und doch etwas
                    Seltenes. Und darum wage ich es, Ihnen dieses Buch\pwindex{Marholm, Laura 19.04.1854 – 06.10.1928@\textsc{Marholm, Laura} (19.04.1854 – 06.10.1928), \emph{Schriftstellerin}!Buch der Frauen1894@\strich\emph{Das Buch der Frauen} {[}1894{]}|pwv} zu übersenden mit der Bitte, es gelegentlich anzublättern. Das
                    ist immer alles, worauf es ankommt. Spricht ein Buch nicht zu einem beim ersten
                    Hineinblicken durch die Blutmale in seinem Satzbau, durch die Seelenschwingung
                    in seinem Stil – dann ist nichts rechtes dran.\pend
           \pstart
           Aber spricht es zu Ihnen, verehrter Herr Doktor, dann würden Sie mich durch ein
                    Zeichen der Mittheilung nicht nur sehr froh machen, sondern auch zu weiterer
                    Selbstmittheilung in anderen Büchern ermuthigen.\pend
           \pstart
           Mit ausgezeichneter Hochachtung{\\[\baselineskip]}\spacefill\mbox{Laura Hansson-Marholm}\pend
           \leftskip=0em{}\endnumbering\briefempfaengerindex{Schnitzler, Arthur@\textsc{Schnitzler, Arthur}!zzzMarholm, Laura@\emph{von Laura Marholm}!1895-04-161@{16. 4. 1895}|)be}\mylabel{h}\end{ledgroupsized}  \newcommand{\dateiname}{L00429}\newcommand{\titel}{Laura Marholm an Arthur Schnitzler, 16. 4. 1895}\newcommand{\editorInnen}{Martin Anton Müller und Gerd-Hermann Susen}\input{../tex-inputs/latex-pdf-abspann}
      