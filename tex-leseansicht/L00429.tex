%% latex-leseansicht-vorspann.tex
%% Vorspann für die Leseansicht.
%% Lädt die gemeinsame Datei latex-vorspann.tex mit nicht gesetztem Schalter.

\newif\ifkorrekturansicht
\korrekturansichtfalse

\input{../tex-inputs/latex-vorspann}


\section[Laura Marholm an Arthur Schnitzler, 16. 4. 1895]{L00429 Laura Marholm an Arthur Schnitzler, 16. 4. 1895}
\nopagebreak\mylabel{L00429v}
\rehead{ }\normalsize\beginnumbering\briefempfaengerindex{Schnitzler, Arthur@\textsc{Schnitzler, Arthur}!zzzMarholm, Laura@\emph{von Laura Marholm}!1895-04-162@{16. 4. 1895}|(be}
\toendnotes[C]{\smallbreak\pagebreak[2]}
\correspDesc{Versand  durch Laura Marholm am 16. 4. 1895 in Schliersee
\newline{}Erhalt  durch Arthur Schnitzler im Zeitraum [17. 4. 1895
                  – 21. 4. 1895?] in Wien}\toendnotes[C]{\smallbreak}
\Standort{CUL, Schnitzler, B 69.}
\physDesc{Brief, 1 Blatt, 2 Seiten, 1755 Zeichen
\newline{}Handschrift: schwarze Tinte, lateinische Kurrent
\newline{}Schnitzler: 1) mit Bleistift beschriftet: »\textsc{Marholm}«  2) mit rotem Buntstift eine Unterstreichung}\toendnotes[C]{\smallbreak}
\pstart
           \raggedleft{}{\pb}Schliersee\oindex{Schliersee@\textbf{Schliersee}|pw}, Oberbaiern\oindex{Oberbayern@\textbf{Oberbayern}, \emph{Verwaltungsgebiet}|pw}\pend
           
\pstart
           \raggedleft{}16. April 95\pend
           
\pstart\center{}Sehr geehrter Herr Professor\pend\vspace{0.5em}
\pstart
           Ich erlaube mir Ihnen beifolgend mein »Buch der
                  Frauen\pwindex{Marholm, Laura 19.\,4.\,1854 Riga – 6.\,10.\,1928 Jūrmala@\textsc{Marholm, Laura} (19.\,4.\,1854 Riga – 6.\,10.\,1928 Jūrmala), \emph{Schriftstellerin}!Buch der Frauen@\strich\emph{Das Buch der Frauen}|pw}« zu übersenden, das in den Wien\oindex{Wien@\textbf{Wien}, \emph{Verwaltungsgebiet}|pw}er
               Blättern viel besprochen worden ist und Ihnen daher vielleicht nicht als ganz
               unbekannter Gast in die Hand kommt. Ich hätte \introOben{}dazu\introOben{} –
               obgleich ich weiß, das Sie das, was lebendig und Lebensbeitrag in der Litteratur ist,
               mit aufmerksamen Blick verfolgen – doch nicht den Muth \strikeout{dazu} gehabt, wenn mir nicht ein gelehrter Herr in Straßburg\oindex{Straßburg@\textbf{Straßburg}|pw}, Dr. Kraft\pwindex{Kraft, Heinrich 5.\,7.\,1867 Ulm – 1944 ebd.@\textsc{Kraft, Heinrich} (5.\,7.\,1867 Ulm – 1944 ebd.), \emph{Mediziner}|pw} von
               der Frauenklinik\oindex{Universitäts-Frauenklinik@\textbf{Universitäts-Frauenklinik}, \emph{Krankenhaus}|pw}, neulich geschrieben hätte,
                  »Das Buch der Frauen\pwindex{Marholm, Laura 19.\,4.\,1854 Riga – 6.\,10.\,1928 Jūrmala@\textsc{Marholm, Laura} (19.\,4.\,1854 Riga – 6.\,10.\,1928 Jūrmala), \emph{Schriftstellerin}!Buch der Frauen@\strich\emph{Das Buch der Frauen}|pw}« sei ihm durch die
               Übereinstimmung der intuitiv erfaßten Ausgangspunkte mit den anthropologischen,
               psychologischen und physiologischen Ausgangspunkten in Havelock Ellis\pwindex{Ellis, Havelock 2.\,2.\,1859 Croydon – 8.\,7.\,1939 Hintlesham@\textsc{Ellis, Havelock} (2.\,2.\,1859 Croydon – 8.\,7.\,1939 Hintlesham), \emph{Schriftsteller, Sexualforscher}|pw} »Mann {\kaufmannsund} Weib\pwindex{Ellis, Havelock 2.\,2.\,1859 Croydon – 8.\,7.\,1939 Hintlesham@\textsc{Ellis, Havelock} (2.\,2.\,1859 Croydon – 8.\,7.\,1939 Hintlesham), \emph{Schriftsteller, Sexualforscher}!Mann und Weib@\strich\emph{Mann und Weib}|pw}« merkwürdig und verheißner für die Sache
                  {\pb}der Frauenkenntniß selber und das
               Weitere, was ich zu sagen hätte. Und ich habe ja allerdings noch kaum mit dem
               Heraussagen angefangen.\pend
           
\pstart
           Ich bin ganz u. gar nicht eine gelehrte Frau und halte auch nichts davon für die
               wirkliche Entwicklung des Weibes. Ich habe das Leben mitgelebt und einen Mann\pwindex{Hansson, Ola 12.\,11.\,1860 Hønsinge – 26.\,9.\,1925 Büyükdere@\textsc{Hansson, Ola} (12.\,11.\,1860 Hønsinge – 26.\,9.\,1925 Büyükdere), \emph{Schriftsteller}|pwv} gefunden, der alle meine
               Möglichkeiten als Weib frei macht und zur Entwicklung treibt. Das ist alles und doch
               etwas Seltenes. Und darum wage ich es, Ihnen dieses Buch\pwindex{Marholm, Laura 19.\,4.\,1854 Riga – 6.\,10.\,1928 Jūrmala@\textsc{Marholm, Laura} (19.\,4.\,1854 Riga – 6.\,10.\,1928 Jūrmala), \emph{Schriftstellerin}!Buch der Frauen@\strich\emph{Das Buch der Frauen}|pwv} zu übersenden mit der Bitte, es gelegentlich
               anzublättern. Das ist immer alles, worauf es ankommt. Spricht ein Buch nicht zu einem
               beim ersten Hineinblicken durch die Blutmale in seinem Satzbau, durch die
               Seelenschwingung in seinem Stil – dann ist nichts rechtes dran.\pend
           
\pstart
           Aber spricht es zu Ihnen, verehrter Herr Doktor, dann würden Sie mich durch ein
               Zeichen der Mittheilung nicht nur sehr froh machen, sondern auch zu weiterer
               Selbstmittheilung in anderen Büchern ermuthigen.\pend
           
\pstart
           Mit ausgezeichneter Hochachtung{\\[\baselineskip]}\spacefill\mbox{Laura Hansson-Marholm}\pend
           \leftskip=0em{}\selectlanguage{ngerman}\endnumbering\briefempfaengerindex{Schnitzler, Arthur@\textsc{Schnitzler, Arthur}!zzzMarholm, Laura@\emph{von Laura Marholm}!1895-04-162@{16. 4. 1895}|)be}\mylabel{L00429h}  \newcommand{\dateiname}{L00429}\newcommand{\titel}{Laura Marholm an Arthur Schnitzler, 16. 4. 1895}\newcommand{\editorInnen}{Martin Anton Müller und Gerd-Hermann Susen}%% latex-leseansicht-abspann.tex
%% Abspann für die Leseansicht.
%% Der Schalter \ifkorrekturansicht ist bereits durch den Vorspann gesetzt.

%% latex-abspann.tex
%% Gemeinsamer Abspann für Korrekturansicht und Leseansicht.
%% Setzt den Schalter \ifkorrekturansicht voraus (gesetzt in den
%% einbindenden Dateien latex-korrekturansicht-abspann.tex bzw.
%% latex-leseansicht-abspann.tex).
%% ---------------------------------------------------------------

\normalsize

% Das esempio-Environment wird nur in der Leseansicht benötigt
\ifkorrekturansicht\else
\newenvironment{esempio}[3]%
{
    \vspace{1.5ex}
    \rlap{\underline{#1}}
    \par
    \setlength{\parindent}{0cm}
    \nopagebreak
    \leftskip=#2cm
    \rightskip=#3cm
}
{
    \par
}
\fi

\doendnotes{C}
\bigskip
\vfill

\clearpage

\footnotesize

\ifkorrekturansicht
  \lohead{\textsc{register}}
\fi

% theindex-Environment neu definieren ohne reledmac
\makeatletter
\renewenvironment{theindex}{%
  \ifkorrekturansicht
    \section*{\indexname}%
  \else
    \subsubsection*{Index der erwähnten Entitäten}%
  \fi
  \setlength{\parindent}{0pt}%
  \setlength{\parskip}{0pt plus 0.3pt}%
  \let\item\@idxitem
}{%
  \ifkorrekturansicht\clearpage\fi
}
\makeatother

\IfFileExists{\jobname-pw.ind}{\input{\jobname-pw.ind}}{}

% Quellenangabe nur in der Leseansicht
\ifkorrekturansicht\else
% Fallback-Definitionen, falls die .tex-Datei \titel etc. nicht gesetzt hat
\providecommand{\titel}{}
\providecommand{\editorInnen}{}
\providecommand{\dateiname}{\jobname}

\vspace{3cm}

\vfill

\footnotesize
\textsc{Quelle}: \titel. Herausgegeben von {\editorInnen}. In: \emph{Arthur Schnitzler: Briefwechsel mit Autorinnen und Autoren}.
 Digitale Edition, https://schnitzler-briefe.acdh.oeaw.ac.at/{\dateiname}.html (Stand \today)
\fi

\end{document}


