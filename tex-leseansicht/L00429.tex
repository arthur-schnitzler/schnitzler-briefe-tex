%% latex-korrekturansicht-vorspann.tex
%% Vorspann für die Korrekturansicht.
%% Lädt die gemeinsame Datei latex-vorspann.tex mit gesetztem Schalter.

\newif\ifkorrekturansicht
\korrekturansichttrue

\input{../tex-inputs/latex-vorspann}


\section[Laura Marholm an Arthur Schnitzler, 16. 4. 1895]{L00429 Laura Marholm an Arthur Schnitzler, 16. 4. 1895}
\nopagebreak\mylabel{L00429v}
\rehead{ }\normalsize\beginnumbering\briefempfaengerindex{Schnitzler, Arthur@\textsc{Schnitzler, Arthur}!zzzMarholm, Laura@\emph{von Laura Marholm}!1895-04-161@{16. 4. 1895}|(be}
\toendnotes[C]{\smallbreak\pagebreak[2]}\Standort{CUL, Schnitzler, B 69.}
\physDesc{Brief, 1 Blatt, 2 Seiten, 1755 Zeichen
\newline{}Handschrift: schwarze Tinte, lateinische Kurrent
\newline{}Schnitzler: 1) mit Bleistift beschriftet: »\textsc{Marholm}«  2) mit rotem Buntstift eine Unterstreichung}\toendnotes[C]{\smallbreak}
\pstart
           \raggedleft{}{\pb}Schliersee\oindex{Schliersee@\textbf{Schliersee}, \emph{P.PPL}|pw}, Oberbaiern\oindex{Oberbayern@\textbf{Oberbayern}, \emph{A.ADM2}|pw}\pend
           
\pstart
           \raggedleft{}16. April 95\pend
           
\pstart\center{}Sehr geehrter Herr Professor\pend\vspace{0.5em}
\pstart
           Ich erlaube mir Ihnen beifolgend mein »Buch der
                  Frauen\pwindex{Buch der Frauen@\emph{Das Buch der Frauen}|pw}« zu übersenden, das in den Wien\oindex{Wien@\textbf{Wien}, \emph{A.ADM2}|pw}er
               Blättern viel besprochen worden ist und Ihnen daher vielleicht nicht als ganz
               unbekannter Gast in die Hand kommt. Ich hätte \introOben{}dazu\introOben{} –
               obgleich ich weiß, das Sie das, was lebendig und Lebensbeitrag in der Litteratur ist,
               mit aufmerksamen Blick verfolgen – doch nicht den Muth \strikeout{dazu} gehabt, wenn mir nicht ein gelehrter Herr in Straßburg\oindex{Strassburg@\textbf{Straßburg}, \emph{P.PPLA}|pw}, Dr. Kraft\pwindex{Kraft, Heinrich 1867-07-05 – 1944@\textsc{Kraft, Heinrich} (1867-07-05 – 1944), \emph{Mediziner/Medizinerin}|pw} von
               der Frauenklinik\oindex{Universitaets-Frauenklinik@\textbf{Universitäts-Frauenklinik}, \emph{Krankenhaus (K.KKH)}|pw}, neulich geschrieben hätte,
                  »Das Buch der Frauen\pwindex{Buch der Frauen@\emph{Das Buch der Frauen}|pw}« sei ihm durch die
               Übereinstimmung der intuitiv erfaßten Ausgangspunkte mit den anthropologischen,
               psychologischen und physiologischen Ausgangspunkten in Havelock Ellis\pwindex{Ellis, Havelock 02.02.1859 – 08.07.1939@\textsc{Ellis, Havelock} (02.02.1859 – 08.07.1939), \emph{Schriftsteller/Schriftstellerin, Sexualforscher/Sexualforscherin}|pw} »Mann {\kaufmannsund} Weib\pwindex{Mann und Weib@\emph{Mann und Weib}|pw}« merkwürdig und verheißner für die Sache
                  {\pb}der Frauenkenntniß selber und das
               Weitere, was ich zu sagen hätte. Und ich habe ja allerdings noch kaum mit dem
               Heraussagen angefangen.\pend
           
\pstart
           Ich bin ganz u. gar nicht eine gelehrte Frau und halte auch nichts davon für die
               wirkliche Entwicklung des Weibes. Ich habe das Leben mitgelebt und einen Mann\pwindex{Hansson, Ola 12.11.1860 – 26.09.1925@\textsc{Hansson, Ola} (12.11.1860 – 26.09.1925), \emph{Schriftsteller/Schriftstellerin}|pwv} gefunden, der alle meine
               Möglichkeiten als Weib frei macht und zur Entwicklung treibt. Das ist alles und doch
               etwas Seltenes. Und darum wage ich es, Ihnen dieses Buch\pwindex{Buch der Frauen@\emph{Das Buch der Frauen}|pwv} zu übersenden mit der Bitte, es gelegentlich
               anzublättern. Das ist immer alles, worauf es ankommt. Spricht ein Buch nicht zu einem
               beim ersten Hineinblicken durch die Blutmale in seinem Satzbau, durch die
               Seelenschwingung in seinem Stil – dann ist nichts rechtes dran.\pend
           
\pstart
           Aber spricht es zu Ihnen, verehrter Herr Doktor, dann würden Sie mich durch ein
               Zeichen der Mittheilung nicht nur sehr froh machen, sondern auch zu weiterer
               Selbstmittheilung in anderen Büchern ermuthigen.\pend
           
\pstart
           Mit ausgezeichneter Hochachtung{\\[\baselineskip]}\spacefill\mbox{Laura Hansson-Marholm}\pend
           \leftskip=0em{}\selectlanguage{ngerman}\endnumbering\briefempfaengerindex{Schnitzler, Arthur@\textsc{Schnitzler, Arthur}!zzzMarholm, Laura@\emph{von Laura Marholm}!1895-04-161@{16. 4. 1895}|)be}\mylabel{L00429h}  \normalsize

\doendnotes{C}
\bigskip
\vfill

\clearpage

\footnotesize

\lohead{\textsc{register}}

% Definiere theindex-Environment komplett neu ohne reledmac
\makeatletter
\renewenvironment{theindex}{%
  \section*{\indexname}%
  \setlength{\parindent}{0pt}%
  \setlength{\parskip}{0pt plus 0.3pt}%
  \let\item\@idxitem
}{%
  \clearpage
}
\makeatother

\IfFileExists{\jobname-pw.ind}{\input{\jobname-pw.ind}}{}

\end{document}

      