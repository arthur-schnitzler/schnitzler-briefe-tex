%% latex-korrekturansicht-vorspann.tex
%% Vorspann für die Korrekturansicht.
%% Lädt die gemeinsame Datei latex-vorspann.tex mit gesetztem Schalter.

\newif\ifkorrekturansicht
\korrekturansichttrue

\input{../tex-inputs/latex-vorspann}


\section[Richard Beer-Hofmann an Arthur Schnitzler, 28. 4. 1899]{L00913 Richard Beer-Hofmann an Arthur Schnitzler, 28. 4. 1899}
\nopagebreak\mylabel{L00913v}
\rehead{ }\normalsize\beginnumbering\briefempfaengerindex{Schnitzler, Arthur@\textsc{Schnitzler, Arthur}!zzzBeer-Hofmann, Richard@\emph{von Richard Beer-Hofmann}!1899-04-281@{28. 4. 1899}|(be}
\toendnotes[C]{\smallbreak\pagebreak[2]}\Standort{CUL, Schnitzler, B 8.}
\physDesc{Brief, 2 Blätter, 6 Seiten, 1496 Zeichen
\newline{}Handschrift: Bleistift, lateinische Kurrent
\newline{}Ordnung: mit Bleistift von unbekannter Hand nummeriert:
                                    »127« }
\buchAbdrucke{\weitereDrucke{Arthur Schnitzler, Richard Beer-Hofmann: \emph{Briefwechsel 1891–1931}. Wien, Zürich: \emph{Europaverlag} 1992, S. 127.} }\toendnotes[C]{\smallbreak}
\pstart
           \raggedleft{}{\pb}Spittal a. d. Drau\oindex{Spittal an der Drau@\textbf{Spittal an der Drau}, \emph{P.PPLA3}|pw}{\\}28/IV 99\pend
           \vspace{0.5em}
\pstart
           Lieber Arthur, ich bin hier um Wohnung zu suchen, und lese soeben
               daß eine junge Dame\pwindex{Dery, Juliane 1864-08-10 – 31.03.1899@\textsc{Déry, Juliane} (1864-08-10 – 31.03.1899), \emph{Schriftsteller/Schriftstellerin, Schauspieler/Schauspielerin}|pwv} zum
               Theil auch deshalb weil man ihr die Rolle der Christine\pwindex{Liebelei. Schauspiel in drei Akten@\emph{Liebelei. Schauspiel in drei Akten}|pwv} weggeno{\geminationm}en hat, sich
               vergiften wollte. Es steht das in einer \label{K_L00913-1v}\edtext{Kärntner Zeitung, in einer Skizze}{\lemma{\textnormal{\emph{Kärntner … Skizze}}}\Cendnote{\textnormal{Diese
                  konnte bislang nicht nachgewiesen werden. Inhaltliche Bedenken an der Angabe bestehen, wenn man zwei
                  Äußerungen der in Berlin\oindex{Berlin@\textbf{Berlin}, \emph{P.PPLC}|pwk} lebenden Meyer-Förster\pwindex{Meyer-Foerster, Elsbeth 05.01.1868 – 17.05.1902@\textsc{Meyer-Förster, Elsbeth} (05.01.1868 – 17.05.1902), \emph{Schriftsteller/Schriftstellerin}|pwk} über ihre Freundin Juliane Déry\pwindex{Dery, Juliane 1864-08-10 – 31.03.1899@\textsc{Déry, Juliane} (1864-08-10 – 31.03.1899), \emph{Schriftsteller/Schriftstellerin, Schauspieler/Schauspielerin}|pwk} als Orientierung nimmt. In
                  einem Leserbrief unmittelbar nach dem Suizid sprach sie deutlich von »\so{tieferem} menschlichem Leiden« als Motiv (\emph{Zu dem tragischen Hingang von Juliane
                     Dery}\pwindex{Zu dem tragischen Hingang von Juliane Dery@\emph{Zu dem tragischen Hingang von Juliane Dery}|pwk}. In: \emph{Berliner Tageblatt}\pwindex{Berliner Tageblatt@\emph{Berliner Tageblatt}|pwk}, Jg. 28,
                     Nr. 168, 2. 4. 1899, S. 3). In einem längeren Beitrag (\emph{Juliane Déry. Ein Nachruf}\pwindex{Juliane Dery. Ein Nachruf@\emph{Juliane Déry. Ein Nachruf}|pwk}. In: \emph{Wiener Rundschau}\pwindex{Wiener Rundschau@\emph{Wiener Rundschau}|pwk}, Jg. 3, Nr. 11,
                        15. 4. 1899, S. 265–267) erwähnte sie ebenfalls
                  neuerliche Theaterambitionen der Toten.}}}\label{K_L00913-1} von Elsbeth {\pb}Meyer-Förster\pwindex{Meyer-Foerster, Elsbeth 05.01.1868 – 17.05.1902@\textsc{Meyer-Förster, Elsbeth} (05.01.1868 – 17.05.1902), \emph{Schriftsteller/Schriftstellerin}|pw}. Sie werden also auch hier durch Litteratur in der Litteratur
               – man könnte dies mit dem Quadratzeichen ausdrücken – berühmt. \label{K_L00913-2v}\edtext{Morgen wenn man Ihre Stücke\pwindex{gruene Kakadu – Paracelsus – Die Gefaehrtin. Drei Einakter@\emph{Der grüne Kakadu – Paracelsus – Die Gefährtin. Drei Einakter}|pw}}{\lemma{\textnormal{\emph{Morgen … Stücke}}}\Cendnote{\textnormal{Am 29. 4. 1899 fand am \emph{Berliner Deutschen Theater}\orgindex{Deutsches Theater Berlin@Deutsches Theater Berlin|pwk}
                  die Premiere von \emph{Der grüne Kakadu – Paracelsus – Die Gefährtin}\pwindex{gruene Kakadu – Paracelsus – Die Gefaehrtin. Drei Einakter@\emph{Der grüne Kakadu – Paracelsus – Die Gefährtin. Drei Einakter}|pwk} statt.}}}\label{K_L00913-2}
               gibt, werde ich hier in der Wirtsstube sitzen und so wie heute die Glocken sieben {\pb}läuten hören. Wenn ich bis dahin
               nicht todt bin; man soll überhaupt nicht »ich werde« sagen, es ist i{\geminationm}er eine Provokation des Schicksals, und wenn ich morgen
               todt bin meint dann das du{\geminationm}e Schicksal es habe einen
               glänzenden Witz gemacht.\pend
           
\pstart
           {\pb}Ich wohne Zimmer Nr\textsuperscript{o} II. So steht über der Thür, das Schlüsselbrett und das
               Stubenmädchen haben mir verrathen daß II früher 13 hieß – Freitag ist auch noch
               gerade heute. Jetzt weiß ich nicht: Bleib ich auf Nr\textsuperscript{o} 13,
               so wird das vielleicht als Provocation aufgefasst; {\pb}wechsle ich das Zi{\geminationm}er, so heißt es: Damit entko{\geminationm}t man mir nicht. Auch daß ich das so niederschreibe,
               wird vielleicht als fauler Ausweg durchschaut. Finden Sie nicht, daß es schwer ist
               sich zu benehmen? Grüßen Sie mir Brahm\pwindex{Brahm, Otto 05.02.1856 – 28.11.1912@\textsc{Brahm, Otto} (05.02.1856 – 28.11.1912), \emph{Theaterleiter/Theaterleiterin, Regisseur/Regisseurin}|pw}, und
               wenn Sie ihn {\pb}sehen auch Kerr\pwindex{Kerr, Alfred 25.12.1867 – 12.10.1948@\textsc{Kerr, Alfred} (25.12.1867 – 12.10.1948), \emph{Schriftsteller/Schriftstellerin, Kritiker/Kritikerin}|pw}; den letzteren kenne ich zwar nur flüchtig
               aber ich laß ihn grüßen wegen des schönen \label{K_L00913-3v}\edtext{Artikels\pwindex{Hirschfeld, Halbe, Sudermann@\emph{Hirschfeld, Halbe, Sudermann}|pwv}}{\lemma{\textnormal{\emph{Artikels}}}\Cendnote{\textnormal{Alfred Kerr\pwindex{Kerr, Alfred 25.12.1867 – 12.10.1948@\textsc{Kerr, Alfred} (25.12.1867 – 12.10.1948), \emph{Schriftsteller/Schriftstellerin, Kritiker/Kritikerin}|pwk}: \emph{Hirschfeld, Halbe, Sudermann}\pwindex{Hirschfeld, Halbe, Sudermann@\emph{Hirschfeld, Halbe, Sudermann}|pwk}. In: \emph{Neue Deutsche Rundschau}\pwindex{Neue Deutsche Rundschau@\emph{Neue Deutsche Rundschau}|pwk}, Jg. 10, H. 4, April
                        1899, S. 439–446.}}}\label{K_L00913-3} über Sudermann\pwindex{Sudermann, Hermann 30.09.1857 – 21.11.1928@\textsc{Sudermann, Hermann} (30.09.1857 – 21.11.1928), \emph{Schriftsteller/Schriftstellerin}|pw} etc.\pend
           
\pstart
           Längstens Mittwoch bin ich wieder in Wien\oindex{Wien@\textbf{Wien}, \emph{A.ADM2}|pw}, – womit ich aber nichts unbescheidenes gesagt haben will –.\pend
           
\pstart
           Herzlichst {\\[\baselineskip]}Ihr \spacefill\mbox{Richard}\pend
           \leftskip=0em{}\selectlanguage{ngerman}\endnumbering\briefempfaengerindex{Schnitzler, Arthur@\textsc{Schnitzler, Arthur}!zzzBeer-Hofmann, Richard@\emph{von Richard Beer-Hofmann}!1899-04-281@{28. 4. 1899}|)be}\mylabel{L00913h}  \normalsize

\doendnotes{C}
\bigskip
\vfill

\clearpage

\footnotesize

\lohead{\textsc{register}}

% Definiere theindex-Environment komplett neu ohne reledmac
\makeatletter
\renewenvironment{theindex}{%
  \section*{\indexname}%
  \setlength{\parindent}{0pt}%
  \setlength{\parskip}{0pt plus 0.3pt}%
  \let\item\@idxitem
}{%
  \clearpage
}
\makeatother

\IfFileExists{\jobname-pw.ind}{\input{\jobname-pw.ind}}{}

\end{document}

      