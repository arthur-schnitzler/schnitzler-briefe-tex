%% latex-leseansicht-vorspann.tex
%% Vorspann für die Leseansicht.
%% Lädt die gemeinsame Datei latex-vorspann.tex mit nicht gesetztem Schalter.

\newif\ifkorrekturansicht
\korrekturansichtfalse

\input{../tex-inputs/latex-vorspann}


\section[Richard Beer-Hofmann an Arthur Schnitzler, 28. 4. 1899]{L00913 Richard Beer-Hofmann an Arthur Schnitzler, 28. 4. 1899}
\nopagebreak\mylabel{L00913v}
\rehead{ }\normalsize\beginnumbering\briefempfaengerindex{Schnitzler, Arthur@\textsc{Schnitzler, Arthur}!zzzBeer-Hofmann, Richard@\emph{von Richard Beer-Hofmann}!1899-04-282@{28. 4. 1899}|(be}
\toendnotes[C]{\smallbreak\pagebreak[2]}
\correspDesc{Versand  durch Richard Beer-Hofmann am 28. 4. 1899 in Spittal an der Drau
\newline{}Erhalt  durch Arthur Schnitzler im Zeitraum [29. 4. 1899
                  – 3. 5. 1899?] in Berlin}\toendnotes[C]{\smallbreak}
\Standort{CUL, Schnitzler, B 8.}
\physDesc{Brief, 2 Blätter, 6 Seiten, 1496 Zeichen
\newline{}Handschrift: Bleistift, lateinische Kurrent
\newline{}Ordnung: mit Bleistift von unbekannter Hand nummeriert:
                                    »127« }
\buchAbdrucke{\weitereDrucke{Arthur Schnitzler, Richard Beer-Hofmann: \emph{Briefwechsel 1891–1931}. Herausgegeben von Konstanze Fliedl. Wien, Zürich: \emph{Europaverlag} 1992, S. 127.} }\toendnotes[C]{\smallbreak}
\pstart
           \raggedleft{}{\pb}Spittal a. d. Drau\oindex{Spittal an der Drau@\textbf{Spittal an der Drau}, \emph{Hauptstadt}|pw}{\\}28/IV 99\pend
           \vspace{0.5em}
\pstart
           Lieber Arthur, ich bin hier um Wohnung zu suchen, und lese soeben
               daß eine junge Dame\pwindex{Déry, Juliane 10.\,8.\,1864 Baja – 31.\,3.\,1899 Berlin@\textsc{Déry, Juliane} (10.\,8.\,1864 Baja – 31.\,3.\,1899 Berlin), \emph{Schriftstellerin, Schauspielerin}|pwv} zum
               Theil auch deshalb weil man ihr die Rolle der Christine\pwindex{Schnitzler, Arthur 15.\,5.\,1862 Wien – 21.\,10.\,1931 ebd.@\textsc{Schnitzler, Arthur} (15.\,5.\,1862 Wien – 21.\,10.\,1931 ebd.), \emph{Schriftsteller, Mediziner}!Liebelei. Schauspiel in drei Akten@\strich\emph{Liebelei. Schauspiel in drei Akten}|pwv} weggeno{\geminationm}en hat, sich
               vergiften wollte. Es steht das in einer \label{K_L00913-1v}\edtext{Kärntner Zeitung, in einer Skizze}{\lemma{\textnormal{\emph{Kärntner … Skizze}}}\Cendnote{\textnormal{Diese
                  konnte bislang nicht nachgewiesen werden. Inhaltliche Bedenken an der Angabe bestehen, wenn man zwei
                  Äußerungen der in Berlin\oindex{Berlin@\textbf{Berlin}, \emph{Hauptstadt}|pwk} lebenden Meyer-Förster\pwindex{Meyer-Förster, Elsbeth 5.\,1.\,1868 Breslau – 17.\,5.\,1902 Bozen@\textsc{Meyer-Förster, Elsbeth} (5.\,1.\,1868 Breslau – 17.\,5.\,1902 Bozen), \emph{Schriftstellerin}|pwk} über ihre Freundin Juliane Déry\pwindex{Déry, Juliane 10.\,8.\,1864 Baja – 31.\,3.\,1899 Berlin@\textsc{Déry, Juliane} (10.\,8.\,1864 Baja – 31.\,3.\,1899 Berlin), \emph{Schriftstellerin, Schauspielerin}|pwk} als Orientierung nimmt. In
                  einem Leserbrief unmittelbar nach dem Suizid sprach sie deutlich von »\so{tieferem} menschlichem Leiden« als Motiv (\emph{Zu dem tragischen Hingang von Juliane
                     Dery}\pwindex{Meyer-Förster, Elsbeth 5.\,1.\,1868 Breslau – 17.\,5.\,1902 Bozen@\textsc{Meyer-Förster, Elsbeth} (5.\,1.\,1868 Breslau – 17.\,5.\,1902 Bozen), \emph{Schriftstellerin}!Zu dem tragischen Hingang von Juliane Dery@\strich\emph{Zu dem tragischen Hingang von Juliane Dery}|pwk}. In: \emph{Berliner Tageblatt}\pwindex{Berliner Tageblatt@\emph{Berliner Tageblatt}|pwk}, Jg. 28,
                     Nr. 168, 2. 4. 1899, S. 3). In einem längeren Beitrag (\emph{Juliane Déry. Ein Nachruf}\pwindex{Meyer-Förster, Elsbeth 5.\,1.\,1868 Breslau – 17.\,5.\,1902 Bozen@\textsc{Meyer-Förster, Elsbeth} (5.\,1.\,1868 Breslau – 17.\,5.\,1902 Bozen), \emph{Schriftstellerin}!Juliane Déry. Ein Nachruf@\strich\emph{Juliane Déry. Ein Nachruf}|pwk}. In: \emph{Wiener Rundschau}\pwindex{Wiener Rundschau@\emph{Wiener Rundschau}|pwk}, Jg. 3, Nr. 11,
                        15. 4. 1899, S. 265–267) erwähnte sie ebenfalls
                  neuerliche Theaterambitionen der Toten.}}}\label{K_L00913-1} von Elsbeth {\pb}Meyer-Förster\pwindex{Meyer-Förster, Elsbeth 5.\,1.\,1868 Breslau – 17.\,5.\,1902 Bozen@\textsc{Meyer-Förster, Elsbeth} (5.\,1.\,1868 Breslau – 17.\,5.\,1902 Bozen), \emph{Schriftstellerin}|pw}. Sie werden also auch hier durch Litteratur in der Litteratur
               – man könnte dies mit dem Quadratzeichen ausdrücken – berühmt. \label{K_L00913-2v}\edtext{Morgen wenn man Ihre Stücke\pwindex{Schnitzler, Arthur 15.\,5.\,1862 Wien – 21.\,10.\,1931 ebd.@\textsc{Schnitzler, Arthur} (15.\,5.\,1862 Wien – 21.\,10.\,1931 ebd.), \emph{Schriftsteller, Mediziner}!grüne Kakadu – Paracelsus – Die Gefährtin. Drei Einakter@\strich\emph{Der grüne Kakadu – Paracelsus – Die Gefährtin. Drei Einakter}|pw}}{\lemma{\textnormal{\emph{Morgen … Stücke}}}\Cendnote{\textnormal{Am 29. 4. 1899 fand am \emph{Berliner Deutschen Theater}\orgindex{Deutsches Theater Berlin@Deutsches Theater Berlin|pwk}
                  die Premiere von \emph{Der grüne Kakadu – Paracelsus – Die Gefährtin}\pwindex{Schnitzler, Arthur 15.\,5.\,1862 Wien – 21.\,10.\,1931 ebd.@\textsc{Schnitzler, Arthur} (15.\,5.\,1862 Wien – 21.\,10.\,1931 ebd.), \emph{Schriftsteller, Mediziner}!grüne Kakadu – Paracelsus – Die Gefährtin. Drei Einakter@\strich\emph{Der grüne Kakadu – Paracelsus – Die Gefährtin. Drei Einakter}|pwk} statt.}}}\label{K_L00913-2}
               gibt, werde ich hier in der Wirtsstube sitzen und so wie heute die Glocken sieben {\pb}läuten hören. Wenn ich bis dahin
               nicht todt bin; man soll überhaupt nicht »ich werde« sagen, es ist i{\geminationm}er eine Provokation des Schicksals, und wenn ich morgen
               todt bin meint dann das du{\geminationm}e Schicksal es habe einen
               glänzenden Witz gemacht.\pend
           
\pstart
           {\pb}Ich wohne Zimmer Nr\textsuperscript{o} II. So steht über der Thür, das Schlüsselbrett und das
               Stubenmädchen haben mir verrathen daß II früher 13 hieß – Freitag ist auch noch
               gerade heute. Jetzt weiß ich nicht: Bleib ich auf Nr\textsuperscript{o} 13,
               so wird das vielleicht als Provocation aufgefasst; {\pb}wechsle ich das Zi{\geminationm}er, so heißt es: Damit entko{\geminationm}t man mir nicht. Auch daß ich das so niederschreibe,
               wird vielleicht als fauler Ausweg durchschaut. Finden Sie nicht, daß es schwer ist
               sich zu benehmen? Grüßen Sie mir Brahm\pwindex{Brahm, Otto 5.\,2.\,1856 Hamburg – 28.\,11.\,1912 Berlin@\textsc{Brahm, Otto} (5.\,2.\,1856 Hamburg – 28.\,11.\,1912 Berlin), \emph{Theaterleiter, Regisseur}|pw}, und
               wenn Sie ihn {\pb}sehen auch Kerr\pwindex{Kerr, Alfred 25.\,12.\,1867 Breslau – 12.\,10.\,1948 Hamburg@\textsc{Kerr, Alfred} (25.\,12.\,1867 Breslau – 12.\,10.\,1948 Hamburg), \emph{Schriftsteller, Kritiker}|pw}; den letzteren kenne ich zwar nur flüchtig
               aber ich laß ihn grüßen wegen des schönen \label{K_L00913-3v}\edtext{Artikels\pwindex{Kerr, Alfred 25.\,12.\,1867 Breslau – 12.\,10.\,1948 Hamburg@\textsc{Kerr, Alfred} (25.\,12.\,1867 Breslau – 12.\,10.\,1948 Hamburg), \emph{Schriftsteller, Kritiker}!Hirschfeld, Halbe, Sudermann@\strich\emph{Hirschfeld, Halbe, Sudermann}|pwv}}{\lemma{\textnormal{\emph{Artikels}}}\Cendnote{\textnormal{Alfred Kerr\pwindex{Kerr, Alfred 25.\,12.\,1867 Breslau – 12.\,10.\,1948 Hamburg@\textsc{Kerr, Alfred} (25.\,12.\,1867 Breslau – 12.\,10.\,1948 Hamburg), \emph{Schriftsteller, Kritiker}|pwk}: \emph{Hirschfeld, Halbe, Sudermann}\pwindex{Kerr, Alfred 25.\,12.\,1867 Breslau – 12.\,10.\,1948 Hamburg@\textsc{Kerr, Alfred} (25.\,12.\,1867 Breslau – 12.\,10.\,1948 Hamburg), \emph{Schriftsteller, Kritiker}!Hirschfeld, Halbe, Sudermann@\strich\emph{Hirschfeld, Halbe, Sudermann}|pwk}. In: \emph{Neue Deutsche Rundschau}\pwindex{Neue Deutsche Rundschau@\emph{Neue Deutsche Rundschau}|pwk}, Jg. 10, H. 4, April 1899, S. 439–446.}}}\label{K_L00913-3} über Sudermann\pwindex{Sudermann, Hermann 30.\,9.\,1857 Macikai – 21.\,11.\,1928 Berlin@\textsc{Sudermann, Hermann} (30.\,9.\,1857 Macikai – 21.\,11.\,1928 Berlin), \emph{Schriftsteller}|pw} etc.\pend
           
\pstart
           Längstens Mittwoch bin ich wieder in Wien\oindex{Wien@\textbf{Wien}, \emph{Verwaltungsgebiet}|pw}, – womit ich aber nichts unbescheidenes gesagt haben will –.\pend
           
\pstart
           Herzlichst {\\[\baselineskip]}Ihr \spacefill\mbox{Richard}\pend
           \leftskip=0em{}\selectlanguage{ngerman}\endnumbering\briefempfaengerindex{Schnitzler, Arthur@\textsc{Schnitzler, Arthur}!zzzBeer-Hofmann, Richard@\emph{von Richard Beer-Hofmann}!1899-04-282@{28. 4. 1899}|)be}\mylabel{L00913h}  \newcommand{\dateiname}{L00913}\newcommand{\titel}{Richard Beer-Hofmann an Arthur Schnitzler, 28. 4. 1899}\newcommand{\editorInnen}{Martin Anton Müller und Gerd-Hermann Susen}%% latex-leseansicht-abspann.tex
%% Abspann für die Leseansicht.
%% Der Schalter \ifkorrekturansicht ist bereits durch den Vorspann gesetzt.

%% latex-abspann.tex
%% Gemeinsamer Abspann für Korrekturansicht und Leseansicht.
%% Setzt den Schalter \ifkorrekturansicht voraus (gesetzt in den
%% einbindenden Dateien latex-korrekturansicht-abspann.tex bzw.
%% latex-leseansicht-abspann.tex).
%% ---------------------------------------------------------------

\normalsize

% Das esempio-Environment wird nur in der Leseansicht benötigt
\ifkorrekturansicht\else
\newenvironment{esempio}[3]%
{
    \vspace{1.5ex}
    \rlap{\underline{#1}}
    \par
    \setlength{\parindent}{0cm}
    \nopagebreak
    \leftskip=#2cm
    \rightskip=#3cm
}
{
    \par
}
\fi

\doendnotes{C}
\bigskip
\vfill

\clearpage

\footnotesize

\ifkorrekturansicht
  \lohead{\textsc{register}}
\fi

% theindex-Environment neu definieren ohne reledmac
\makeatletter
\renewenvironment{theindex}{%
  \ifkorrekturansicht
    \section*{\indexname}%
  \else
    \subsubsection*{Index der erwähnten Entitäten}%
  \fi
  \setlength{\parindent}{0pt}%
  \setlength{\parskip}{0pt plus 0.3pt}%
  \let\item\@idxitem
}{%
  \ifkorrekturansicht\clearpage\fi
}
\makeatother

\IfFileExists{\jobname-pw.ind}{\input{\jobname-pw.ind}}{}

% Quellenangabe nur in der Leseansicht
\ifkorrekturansicht\else
% Fallback-Definitionen, falls die .tex-Datei \titel etc. nicht gesetzt hat
\providecommand{\titel}{}
\providecommand{\editorInnen}{}
\providecommand{\dateiname}{\jobname}

\vspace{3cm}

\vfill

\footnotesize
\textsc{Quelle}: \titel. Herausgegeben von {\editorInnen}. In: \emph{Arthur Schnitzler: Briefwechsel mit Autorinnen und Autoren}.
 Digitale Edition, https://schnitzler-briefe.acdh.oeaw.ac.at/{\dateiname}.html (Stand \today)
\fi

\end{document}


