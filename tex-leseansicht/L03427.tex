%% latex-korrekturansicht-vorspann.tex
%% Vorspann für die Korrekturansicht.
%% Lädt die gemeinsame Datei latex-vorspann.tex mit gesetztem Schalter.

\newif\ifkorrekturansicht
\korrekturansichttrue

\input{../tex-inputs/latex-vorspann}


\section[ Felix Salten an Arthur Schnitzler, 19. 6. 1906]{L03427 Felix Salten an Arthur Schnitzler, 19. 6. 1906}
\nopagebreak\mylabel{L03427v}
\rehead{ }\normalsize\beginnumbering\briefempfaengerindex{Schnitzler, Arthur@\textsc{Schnitzler, Arthur}!zzzSalten, Felix@\emph{von Felix Salten}!1906-06-191@{19. 6. 1906}|(be}
\toendnotes[C]{\smallbreak\pagebreak[2]}\Standort{CUL, Schnitzler, B 89, B 1.}
\physDesc{Bildpostkarte, 292 Zeichen
\newline{}Handschrift: schwarze Tinte, lateinische Kurrent
\newline{}Versand: Stempel: »\nobreak{}2\textcolor{gray}{0}. 6. 06, Deutsch-amerikanische Seepost
                                          Bremen–New York\orgindex{Deutsch-amerikanische Seepost@Deutsch-amerikanische Seepost|pw}\nobreak{}«.  
\newline{}Ordnung: mit Bleistift von unbekannter Hand nummeriert: »218« }\toendnotes[C]{\smallbreak}\pstart{}{\pb}Herrn D\textsuperscript{r} Arthur Schnitzler\pend{}\pstart{}Wien\oindex{Wien@\textbf{Wien}, \emph{A.ADM2}|pw}\pend{}\pstart{}XVIII. Spöttelgasse 7\oindex{Edmund-Weiss-Gasse 7@\textbf{Edmund-Weiß-Gasse 7}, \emph{Wohngebäude (K.WHS)}|pw}\pend{}\pstart{}Österreich\oindex{Oesterreich@\textbf{Österreich}, \emph{A.PCLI}|pw}.\pend{}{\bigskip}
\pstart
           {\pb}\textcolor{gray}{\textbf{Nordd. LLoyd\orgindex{Norddeutscher Lloyd@Norddeutscher Lloyd|pw}. »Kronprinz Wilhelm\orgindex{Kronprinz Wilhelm@Kronprinz Wilhelm|pw}«.}}\hfill \textcolor{gray}{\textbf{Rauchsalon I. Klasse.}}\pend
           \vspace{1em}
\pstart
           {\pb}ebenda.
                     19. VI. 06.\pend
           \vspace{0.5em}
\pstart
           Lieber,{ }\uline{so} sieht nun die \label{K_L03427-1v}\edtext{Radpartie}{\lemma{\textnormal{\emph{Radpartie}}}\Cendnote{\textnormal{Siehe Felix Salten an Arthur Schnitzler, 28. 3. 1906. }}}\label{K_L03427-1} und der Klopeiner See\oindex{Klopeiner See@\textbf{Klopeiner See}, \emph{H.LK}|pw} aus. Ich gehe auf \label{K_L03427-2v}\edtext{14 Tage nach England\oindex{England@\textbf{England}, \emph{A.ADM1}|pw}}{\lemma{\textnormal{\emph{14 Tage nach England}}}\Cendnote{\textnormal{Er war beruflich unterwegs. In seinen
                     \emph{Erinnerungen}\pwindex{Erinnerungen@\emph{Erinnerungen}|pwk} (\emph{Wienbibliothek im Rathaus}, Nachlass Salten\pwindex{Salten, Felix 06.09.1869 – 08.10.1945@\textsc{Salten, Felix} (06.09.1869 – 08.10.1945), \emph{Schriftsteller/Schriftstellerin, Journalist/Journalistin, Chefredakteur/Chefredakteurin}|pwk}, ZPH 1681/1 1.1.1.9.1, [S. 19–20])
                  schildert Salten\pwindex{Salten, Felix 06.09.1869 – 08.10.1945@\textsc{Salten, Felix} (06.09.1869 – 08.10.1945), \emph{Schriftsteller/Schriftstellerin, Journalist/Journalistin, Chefredakteur/Chefredakteurin}|pwk} eine offizielle
                     »Friedensreise« mit anderen Journalisten (Julius Ferdinand Wollf\pwindex{Wollf, Julius Ferdinand 22.05.1871 – 01.03.1942@\textsc{Wollf, Julius Ferdinand} (22.05.1871 – 01.03.1942), \emph{Journalist/Journalistin, Herausgeber/Herausgeberin, Verleger/Verlegerin}|pwk} und Max Meyerfeld\pwindex{Meyerfeld, Max 26.09.1875 – 1940-10-03@\textsc{Meyerfeld, Max} (26.09.1875 – 1940-10-03), \emph{Journalist/Journalistin, Anglist/Anglistin}|pwk}) über Bremerhaven\oindex{Bremerhaven@\textbf{Bremerhaven}, \emph{A.ADM4}|pwk} nach Southampton\oindex{Southampton@\textbf{Southampton}, \emph{P.PPLA2}|pwk} und
                  weiter nach London\oindex{London@\textbf{London}, \emph{P.PPLC}|pwk}. Dort will er mit Winston Churchill\pwindex{Churchill, Winston 1874-11-30 – 1965-01-24@\textsc{Churchill, Winston} (1874-11-30 – 1965-01-24), \emph{Politiker/Politikerin}|pwk}, David Lloyd George\pwindex{Lloyd George, David 17.01.1863 – 26.03.1945@\textsc{Lloyd George, David} (17.01.1863 – 26.03.1945), \emph{Politiker/Politikerin, Minister/Ministerin, Autor/Autorin}|pwk} und Richard Haldane\pwindex{Haldane, Richard 1856-07-30 – 1928-08-19@\textsc{Haldane, Richard} (1856-07-30 – 1928-08-19), \emph{Politiker/Politikerin}|pwk} gesprochen haben, die damals alle amtierende Minister
                  waren. Die weiteren Stationen (Stratford-upon-Avon\oindex{Stratford-upon-Avon@\textbf{Stratford-upon-Avon}, \emph{P.PPL}|pwk} und Cambridge\oindex{Cambridge@\textbf{Cambridge}, \emph{P.PPLA2}|pwk})
                  decken sich mit den Postkarten, die er am 23. 6. 1906 und am 27. 6. 1906 an Schnitzler geschrieben hat.}}}\label{K_L03427-2}. Otti\pwindex{Salten, Ottilie 07.03.1868 – 22.06.1942@\textsc{Salten, Ottilie} (07.03.1868 – 22.06.1942), \emph{Schauspieler/Schauspielerin}|pw} ist mit den Kinder\pwindex{Rehmann, Anna Katharina 18.08.1904 – 27.03.1977@\textsc{Rehmann, Anna Katharina} (18.08.1904 – 27.03.1977), \emph{Schauspieler/Schauspielerin, Übersetzer/Übersetzerin}|pwv}\pwindex{Salten, Paul 11.08.1903 – 08.05.1937@\textsc{Salten, Paul} (11.08.1903 – 08.05.1937), \emph{Filmcutter/Filmcutterin}|pwv}n in Bansin\oindex{Bansin@\textbf{Bansin}, \emph{P.PPL}|pw},
               bei Heringsdorf\oindex{Heringsdorf@\textbf{Heringsdorf}, \emph{P.PPLA4}|pw}. Vielleicht \label{K_L03427-3v}\edtext{sehen wir uns, wenn Sie nach Dänemark\oindex{Daenemark@\textbf{Dänemark}, \emph{A.PCLI}|pw} fahren}{\lemma{\textnormal{\emph{sehen … fahren}}}\Cendnote{\textnormal{Am Weg nach Dänemark\oindex{Daenemark@\textbf{Dänemark}, \emph{A.PCLI}|pwk}
                  (Ende Juni) sahen sie sich nicht, da Salten\pwindex{Salten, Felix 06.09.1869 – 08.10.1945@\textsc{Salten, Felix} (06.09.1869 – 08.10.1945), \emph{Schriftsteller/Schriftstellerin, Journalist/Journalistin, Chefredakteur/Chefredakteurin}|pwk} während Schnitzlers{ }Berlin\oindex{Berlin@\textbf{Berlin}, \emph{P.PPLC}|pwk}-Aufenthalt nicht vor Ort war (vgl. Felix Salten an Arthur Schnitzler, 6. 7. 1906). In Marienlyst\oindex{Marienlyst@\textbf{Marienlyst}, \emph{S.EST}|pwk} sahen sie sich am 2. 8. 1906.}}}\label{K_L03427-3}. Herzlichst Ihr {\\}\spacefill\mbox{Salten}\pend
           \selectlanguage{ngerman}\endnumbering\briefempfaengerindex{Schnitzler, Arthur@\textsc{Schnitzler, Arthur}!zzzSalten, Felix@\emph{von Felix Salten}!1906-06-191@{19. 6. 1906}|)be}\mylabel{L03427h}  \normalsize

\doendnotes{C}
\bigskip
\vfill

\clearpage

\footnotesize

\lohead{\textsc{register}}

% Definiere theindex-Environment komplett neu ohne reledmac
\makeatletter
\renewenvironment{theindex}{%
  \section*{\indexname}%
  \setlength{\parindent}{0pt}%
  \setlength{\parskip}{0pt plus 0.3pt}%
  \let\item\@idxitem
}{%
  \clearpage
}
\makeatother

\IfFileExists{\jobname-pw.ind}{\input{\jobname-pw.ind}}{}

\end{document}

      