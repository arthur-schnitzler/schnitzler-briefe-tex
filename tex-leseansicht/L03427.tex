%% latex-leseansicht-vorspann.tex
%% Vorspann für die Leseansicht.
%% Lädt die gemeinsame Datei latex-vorspann.tex mit nicht gesetztem Schalter.

\newif\ifkorrekturansicht
\korrekturansichtfalse

\input{../tex-inputs/latex-vorspann}


         
         \renewcommand{\erwaehntePersonen}{Personen: Winston Churchill, Richard Haldane, David Lloyd George, Max Meyerfeld, Anna Katharina Rehmann, Felix Salten, Ottilie Salten, Paul Salten, Julius Ferdinand Wollf}
         \renewcommand{\erwaehnteInstitutionen}{Institutionen: Deutsch-amerikanische Seepost, Kronprinz Wilhelm, Norddeutscher Lloyd}
         \renewcommand{\erwaehnteOrte}{Orte: Bansin, Berlin, Bremerhaven, Cambridge, Dänemark, Edmund-Weiß-Gasse 7, England, Heringsdorf, Klopeiner See, London, Marienlyst, Nordsee, Southampton, Stratford-upon-Avon, Wien, Österreich}
         \renewcommand{\erwaehnteWerke}{Werke: Erinnerungen}
               \section[ Felix Salten an Arthur Schnitzler, 19. 6. 1906]{ Felix Salten an Arthur Schnitzler, 19. 6. 1906}\nopagebreak\mylabel{v}\rehead{ }\begin{ledgroupsized}[t]{13cm}\normalsize\beginnumbering\briefempfaengerindex{Schnitzler, Arthur@\textsc{Schnitzler, Arthur}!zzzSalten, Felix@\emph{von Felix Salten}!1906-06-191@{19. 6. 1906}|(be} \toendnotes[C]{\smallbreak\pagebreak[2]} \Standort{CUL, Schnitzler, B 89, B 1.}
\physDesc{Bildpostkarte, 292 Zeichen
\newline{}Handschrift: schwarze Tinte, lateinische Kurrent
\newline{}Versand: Stempel: »\nobreak{}2\textcolor{gray}{0}. 6. 06, Deutsch-amerikanische Seepost
                                          Bremen–New York\orgindex{Deutsch-amerikanische Seepost@Deutsch-amerikanische Seepost|pw}\nobreak{}«.  
\newline{}Ordnung: mit Bleistift von unbekannter Hand nummeriert: »218« }\toendnotes[C]{\smallbreak}\pstart{}{\pb}Herrn D\textsuperscript{r} Arthur Schnitzler\pend{}\pstart{}Wien\oindex{Wien@\textbf{Wien}|pw}\pend{}\pstart{}XVIII. Spöttelgasse 7\oindex{Edmund-Weiss-Gasse 7@\textbf{Edmund-Weiß-Gasse 7}|pw}\pend{}\pstart{}Österreich\oindex{Oesterreich@\textbf{Österreich}|pw}.\pend{}{\bigskip}\pstart
           \noindent{}{\pb}\textcolor{gray}{\textbf{Nordd. LLoyd\orgindex{Norddeutscher Lloyd@Norddeutscher Lloyd|pw}. »Kronprinz Wilhelm\orgindex{Kronprinz Wilhelm@Kronprinz Wilhelm|pw}«.}}\hfill \textcolor{gray}{\textbf{Rauchsalon I. Klasse.}}\pend
           \pstart
           ebenda. 19. VI. 06.\pend
           \pstart
           Lieber,{ }\uline{so} sieht nun die \label{K_L03427-1v}\edtext{Radpartie}{\lemma{\textnormal{\emph{Radpartie}}}\Cendnote{\textnormal{siehe Felix Salten an Arthur Schnitzler, 28. 3. 1906}}}\label{K_L03427-1h} und der Klopeiner See\oindex{Klopeiner See@\textbf{Klopeiner See}|pw} aus. Ich gehe
                  auf \label{K_L03427-2v}\edtext{14 Tage nach England\oindex{England@\textbf{England}|pw}}{\lemma{\textnormal{\emph{14 Tage nach England}}}\Cendnote{\textnormal{Er war
                  beruflich unterwegs. In seinen \emph{Erinnerungen}\pwindex{Salten, Felix 06.09.1869 – 08.10.1945@\textsc{Salten, Felix} (06.09.1869 – 08.10.1945), \emph{Schriftsteller, Journalist}!Erinnerungen@\strich\emph{Erinnerungen}|pwk} (\emph{Wienbibliothek im Rathaus}, Nachlass Salten\pwindex{Salten, Felix 06.09.1869 – 08.10.1945@\textsc{Salten, Felix} (06.09.1869 – 08.10.1945), \emph{Schriftsteller, Journalist}|pwk}, ZPH 1681/1 1.1.1.9.1, [S. 19–20])
                  schildert Salten\pwindex{Salten, Felix 06.09.1869 – 08.10.1945@\textsc{Salten, Felix} (06.09.1869 – 08.10.1945), \emph{Schriftsteller, Journalist}|pwk} eine offizielle »Friedensreise« mit
                  anderen Journalisten (Julius Ferdinand
                     Wollf\pwindex{Wollf, Julius Ferdinand 22.05.1871 – 01.03.1942@\textsc{Wollf, Julius Ferdinand} (22.05.1871 – 01.03.1942), \emph{Journalist, Herausgeber, Verleger}|pwk} und Max Meyerfeld\pwindex{Meyerfeld, Max 26.09.1875 – 1940-10-03@\textsc{Meyerfeld, Max} (26.09.1875 – 1940-10-03), \emph{Journalist, Anglist}|pwk}) über Bremerhaven\oindex{Bremerhaven@\textbf{Bremerhaven}|pwk} nach Southampton\oindex{Southampton@\textbf{Southampton}|pwk} und weiter nach London\oindex{London@\textbf{London}|pwk}. Dort will er mit Winston
                     Churchill\pwindex{Churchill, Winston 1874-11-30 – 1965-01-24@\textsc{Churchill, Winston} (1874-11-30 – 1965-01-24), \emph{Politiker}|pwk}, David Lloyd George\pwindex{Lloyd George, David 17.01.1863 – 26.03.1945@\textsc{Lloyd George, David} (17.01.1863 – 26.03.1945), \emph{Politiker, Minister, Autor}|pwk} und
                     Richard Haldane\pwindex{Haldane, Richard 1856-07-30 – 1928-08-19@\textsc{Haldane, Richard} (1856-07-30 – 1928-08-19), \emph{Politiker}|pwk} gesprochen haben, die
                  damals alle amtierende Minister waren. Die weiteren Stationen (Stratford-upon-Avon\oindex{Stratford-upon-Avon@\textbf{Stratford-upon-Avon}|pwk} und Cambridge\oindex{Cambridge@\textbf{Cambridge}|pwk}) decken sich mit den Postkarten, die er am 23. 6. 1906 und am 27. 6. 1906 an Schnitzler\pwindex{Schnitzler, Arthur 15.05.1862 – 21.10.1931@\textsc{Schnitzler, Arthur} (15.05.1862 – 21.10.1931), \emph{Schriftsteller, Mediziner}|pwk} schreibt.}}}\label{K_L03427-2h}. Otti\pwindex{Salten, Ottilie 07.03.1868 – 22.06.1942@\textsc{Salten, Ottilie} (07.03.1868 – 22.06.1942), \emph{Schauspielerin}|pw} ist mit den Kinder\pwindex{Rehmann, Anna Katharina 18.08.1904 – 27.03.1977@\textsc{Rehmann, Anna Katharina} (18.08.1904 – 27.03.1977), \emph{Schauspielerin, Übersetzerin}|pwv}\pwindex{Salten, Paul 11.08.1903 – 08.05.1937@\textsc{Salten, Paul} (11.08.1903 – 08.05.1937), \emph{Filmcutter}|pwv}n in Bansin\oindex{Bansin@\textbf{Bansin}|pw}, bei Heringsdorf\oindex{Heringsdorf@\textbf{Heringsdorf}|pw}. Vielleicht \label{K_L03427-3v}\edtext{sehen
               wir uns, wenn Sie nach Dänemark\oindex{Daenemark@\textbf{Dänemark}|pw} fahren}{\lemma{\textnormal{\emph{sehen … fahren}}}\Cendnote{\textnormal{Am Weg nach Dänemark\oindex{Daenemark@\textbf{Dänemark}|pwk} (Ende Juni) sahen sie
                  sich nicht, da Salten\pwindex{Salten, Felix 06.09.1869 – 08.10.1945@\textsc{Salten, Felix} (06.09.1869 – 08.10.1945), \emph{Schriftsteller, Journalist}|pwk} während Schnitzler\pwindex{Schnitzler, Arthur 15.05.1862 – 21.10.1931@\textsc{Schnitzler, Arthur} (15.05.1862 – 21.10.1931), \emph{Schriftsteller, Mediziner}|pwk}s Berlin\oindex{Berlin@\textbf{Berlin}|pwk}-Aufenthalt nicht vor Ort war (vgl. Felix Salten an Arthur Schnitzler, 6. 7. 1906). In Marienlyst\oindex{Marienlyst@\textbf{Marienlyst}|pwk} sahen sie sich am 2. 8. 1906.}}}\label{K_L03427-3h}. Herzlichst Ihr {\\}\spacefill\mbox{Salten}\pend
           
         
         \endnumbering\mylabel{h}\end{ledgroupsized}  \newcommand{\dateiname}{L03427}\newcommand{\titel}{Felix Salten an Arthur Schnitzler, 19. 6. 1906}\newcommand{\editorInnen}{Martin Anton Müller und Laura Untner}%% latex-leseansicht-abspann.tex
%% Abspann für die Leseansicht.
%% Der Schalter \ifkorrekturansicht ist bereits durch den Vorspann gesetzt.

%% latex-abspann.tex
%% Gemeinsamer Abspann für Korrekturansicht und Leseansicht.
%% Setzt den Schalter \ifkorrekturansicht voraus (gesetzt in den
%% einbindenden Dateien latex-korrekturansicht-abspann.tex bzw.
%% latex-leseansicht-abspann.tex).
%% ---------------------------------------------------------------

\normalsize

% Das esempio-Environment wird nur in der Leseansicht benötigt
\ifkorrekturansicht\else
\newenvironment{esempio}[3]%
{
    \vspace{1.5ex}
    \rlap{\underline{#1}}
    \par
    \setlength{\parindent}{0cm}
    \nopagebreak
    \leftskip=#2cm
    \rightskip=#3cm
}
{
    \par
}
\fi

\doendnotes{C}
\bigskip
\vfill

\clearpage

\footnotesize

\ifkorrekturansicht
  \lohead{\textsc{register}}
\fi

% theindex-Environment neu definieren ohne reledmac
\makeatletter
\renewenvironment{theindex}{%
  \ifkorrekturansicht
    \section*{\indexname}%
  \else
    \subsubsection*{Index der erwähnten Entitäten}%
  \fi
  \setlength{\parindent}{0pt}%
  \setlength{\parskip}{0pt plus 0.3pt}%
  \let\item\@idxitem
}{%
  \ifkorrekturansicht\clearpage\fi
}
\makeatother

\IfFileExists{\jobname-pw.ind}{\input{\jobname-pw.ind}}{}

% Quellenangabe nur in der Leseansicht
\ifkorrekturansicht\else
% Fallback-Definitionen, falls die .tex-Datei \titel etc. nicht gesetzt hat
\providecommand{\titel}{}
\providecommand{\editorInnen}{}
\providecommand{\dateiname}{\jobname}

\vspace{3cm}

\vfill

\footnotesize
\textsc{Quelle}: \titel. Herausgegeben von {\editorInnen}. In: \emph{Arthur Schnitzler: Briefwechsel mit Autorinnen und Autoren}.
 Digitale Edition, https://schnitzler-briefe.acdh.oeaw.ac.at/{\dateiname}.html (Stand \today)
\fi

\end{document}


      