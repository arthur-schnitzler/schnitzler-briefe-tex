%% latex-leseansicht-vorspann.tex
%% Vorspann für die Leseansicht.
%% Lädt die gemeinsame Datei latex-vorspann.tex mit nicht gesetztem Schalter.

\newif\ifkorrekturansicht
\korrekturansichtfalse

\input{../tex-inputs/latex-vorspann}


\section[ Felix Salten an Arthur Schnitzler, 19. 6. 1906]{L03427 Felix Salten an Arthur Schnitzler,  19. 6. 1906}
\nopagebreak\mylabel{L03427v}
\rehead{ }\normalsize\beginnumbering\briefempfaengerindex{Schnitzler, Arthur@\textsc{Schnitzler, Arthur}!zzzSalten, Felix@\emph{von Felix Salten}!1906-06-191@{19. 6. 1906}|(be}
\toendnotes[C]{\smallbreak\pagebreak[2]}
\correspDesc{Versand  durch Felix Salten am 19. 6. 1906 in Nordsee
\newline{}Übermittlung  am 20. 6. 1906 in Nordsee
\newline{}Erhalt  durch Arthur Schnitzler im Zeitraum [21. 6. 1906
                  – 25. 6. 1906?] in Wien}\toendnotes[C]{\smallbreak}
\Standort{CUL, Schnitzler, B 89, B 1.}
\physDesc{Bildpostkarte, 292 Zeichen
\newline{}Handschrift: schwarze Tinte, lateinische Kurrent
\newline{}Versand: Stempel: »\nobreak{}2\textcolor{gray}{0}. 6. 06, Deutsch-amerikanische Seepost
                                          Bremen–New York\orgindex{Deutsch-amerikanische Seepost@Deutsch-amerikanische Seepost|pw}\nobreak{}«.  
\newline{}Ordnung: mit Bleistift von unbekannter Hand nummeriert: »218« }\toendnotes[C]{\smallbreak}\pstart{}{\pb}Herrn D\textsuperscript{r} Arthur Schnitzler\pend{}\pstart{}Wien\oindex{Wien@\textbf{Wien}, \emph{Verwaltungsgebiet}|pw}\pend{}\pstart{}XVIII. Spöttelgasse 7\oindex{Wien@\textbf{Wien}!XVIII., Währing@\textbf{XVIII., Währing}!Edmund-Weiß-Gasse 7@\textbf{Edmund-Weiß-Gasse 7}, \emph{Wohngebäude}|pw}\pend{}\pstart{}Österreich\oindex{Österreich@\textbf{Österreich}|pw}.\pend{}{\bigskip}
\pstart
           {\pb}\textcolor{gray}{\textbf{Nordd. LLoyd\orgindex{Norddeutscher Lloyd@Norddeutscher Lloyd|pw}. »Kronprinz Wilhelm\orgindex{Kronprinz Wilhelm@Kronprinz Wilhelm|pw}«.}}\hfill \textcolor{gray}{\textbf{Rauchsalon I. Klasse.}}\pend
           \vspace{1em}
\pstart
           {\pb}ebenda.
                     19. VI. 06.\pend
           \vspace{0.5em}
\pstart
           Lieber,{ }\uline{so} sieht nun die \label{K_L03427-1v}\edtext{Radpartie}{\lemma{\textnormal{\emph{Radpartie}}}\Cendnote{\textnormal{Siehe XXXX Auszeichnungsfehler: Dokument L03416 nicht gefunden. }}}\label{K_L03427-1} und der Klopeiner See\oindex{Klopeiner See@\textbf{Klopeiner See}, \emph{See}|pw} aus. Ich gehe auf \label{K_L03427-2v}\edtext{14 Tage nach England\oindex{England@\textbf{England}, \emph{Land}|pw}}{\lemma{\textnormal{\emph{14 Tage nach England}}}\Cendnote{\textnormal{Er war beruflich unterwegs. In seinen
                     \emph{Erinnerungen}\pwindex{Salten, Felix 6.\,9.\,1869 Budapest – 8.\,10.\,1945 Zürich@\textsc{Salten, Felix} (6.\,9.\,1869 Budapest – 8.\,10.\,1945 Zürich), \emph{Schriftsteller, Journalist, Chefredakteur}!Erinnerungen@\strich\emph{Erinnerungen}|pwk} (\emph{Wienbibliothek im Rathaus}, Nachlass Salten\pwindex{Salten, Felix 6.\,9.\,1869 Budapest – 8.\,10.\,1945 Zürich@\textsc{Salten, Felix} (6.\,9.\,1869 Budapest – 8.\,10.\,1945 Zürich), \emph{Schriftsteller, Journalist, Chefredakteur}|pwk}, ZPH 1681/1 1.1.1.9.1, [S. 19–20])
                  schildert Salten\pwindex{Salten, Felix 6.\,9.\,1869 Budapest – 8.\,10.\,1945 Zürich@\textsc{Salten, Felix} (6.\,9.\,1869 Budapest – 8.\,10.\,1945 Zürich), \emph{Schriftsteller, Journalist, Chefredakteur}|pwk} eine offizielle
                     »Friedensreise« mit anderen Journalisten (Julius Ferdinand Wollf\pwindex{Wollf, Julius Ferdinand 22.\,5.\,1871 Koblenz – 1.\,3.\,1942 Dresden@\textsc{Wollf, Julius Ferdinand} (22.\,5.\,1871 Koblenz – 1.\,3.\,1942 Dresden), \emph{Journalist, Herausgeber, Verleger}|pwk} und Max Meyerfeld\pwindex{Meyerfeld, Max 26.\,9.\,1875 Gießen – 3.\,10.\,1940 Berlin@\textsc{Meyerfeld, Max} (26.\,9.\,1875 Gießen – 3.\,10.\,1940 Berlin), \emph{Journalist, Anglist}|pwk}) über Bremerhaven\oindex{Bremerhaven@\textbf{Bremerhaven}, \emph{Region}|pwk} nach Southampton\oindex{Southampton@\textbf{Southampton}, \emph{Hauptstadt}|pwk} und
                  weiter nach London\oindex{London@\textbf{London}, \emph{Hauptstadt}|pwk}. Dort will er mit Winston Churchill\pwindex{Churchill, Winston 30.\,11.\,1874 Blenheim Palace – 24.\,1.\,1965 Hyde Park Gate@\textsc{Churchill, Winston} (30.\,11.\,1874 Blenheim Palace – 24.\,1.\,1965 Hyde Park Gate), \emph{Politiker}|pwk}, David Lloyd George\pwindex{Lloyd George, David 17.\,1.\,1863 Manchester – 26.\,3.\,1945 Llanystumdwy@\textsc{Lloyd George, David} (17.\,1.\,1863 Manchester – 26.\,3.\,1945 Llanystumdwy), \emph{Politiker, Minister, Autor}|pwk} und Richard Haldane\pwindex{Haldane, Richard 30.\,7.\,1856 Edinburgh – 19.\,8.\,1928 Auchterarder@\textsc{Haldane, Richard} (30.\,7.\,1856 Edinburgh – 19.\,8.\,1928 Auchterarder), \emph{Politiker}|pwk} gesprochen haben, die damals alle amtierende Minister
                  waren. Die weiteren Stationen (Stratford-upon-Avon\oindex{Stratford-upon-Avon@\textbf{Stratford-upon-Avon}|pwk} und Cambridge\oindex{Cambridge@\textbf{Cambridge}, \emph{Hauptstadt}|pwk})
                  decken sich mit den Postkarten, die er am XXXX Auszeichnungsfehler: Dokument L03428 nicht gefunden und am XXXX Auszeichnungsfehler: Dokument L03429 nicht gefunden an Schnitzler geschrieben hat.}}}\label{K_L03427-2}. Otti\pwindex{Salten, Ottilie 7.\,3.\,1868 Prag – 22.\,6.\,1942 Zürich@\textsc{Salten, Ottilie} (7.\,3.\,1868 Prag – 22.\,6.\,1942 Zürich), \emph{Schauspielerin}|pw} ist mit den Kinder\pwindex{Rehmann, Anna Katharina 18.\,8.\,1904 Wien – 27.\,3.\,1977 Zürich@\textsc{Rehmann, Anna Katharina} (18.\,8.\,1904 Wien – 27.\,3.\,1977 Zürich), \emph{Schauspielerin, Übersetzerin}|pwv}\pwindex{Salten, Paul 11.\,8.\,1903 Wien – 8.\,5.\,1937 ebd.@\textsc{Salten, Paul} (11.\,8.\,1903 Wien – 8.\,5.\,1937 ebd.), \emph{Filmcutter}|pwv}n in Bansin\oindex{Bansin@\textbf{Bansin}|pw},
               bei Heringsdorf\oindex{Heringsdorf@\textbf{Heringsdorf}, \emph{Hauptstadt}|pw}. Vielleicht \label{K_L03427-3v}\edtext{sehen wir uns, wenn Sie nach Dänemark\oindex{Dänemark@\textbf{Dänemark}|pw} fahren}{\lemma{\textnormal{\emph{sehen … fahren}}}\Cendnote{\textnormal{Am Weg nach Dänemark\oindex{Dänemark@\textbf{Dänemark}|pwk}
                  (Ende Juni) sahen sie sich nicht, da Salten\pwindex{Salten, Felix 6.\,9.\,1869 Budapest – 8.\,10.\,1945 Zürich@\textsc{Salten, Felix} (6.\,9.\,1869 Budapest – 8.\,10.\,1945 Zürich), \emph{Schriftsteller, Journalist, Chefredakteur}|pwk} während Schnitzlers{ }Berlin\oindex{Berlin@\textbf{Berlin}, \emph{Hauptstadt}|pwk}-Aufenthalt nicht vor Ort war (vgl. XXXX Auszeichnungsfehler: Dokument L03430 nicht gefunden). In Marienlyst\oindex{Marienlyst@\textbf{Marienlyst}, \emph{Gut}|pwk} sahen sie sich am 2. 8. 1906.}}}\label{K_L03427-3}. Herzlichst Ihr {\\}\spacefill\mbox{Salten}\pend
           \selectlanguage{ngerman}\endnumbering\briefempfaengerindex{Schnitzler, Arthur@\textsc{Schnitzler, Arthur}!zzzSalten, Felix@\emph{von Felix Salten}!1906-06-191@{19. 6. 1906}|)be}\mylabel{L03427h}  \newcommand{\dateiname}{L03427}\newcommand{\titel}{Felix Salten an Arthur Schnitzler, 19. 6. 1906}\newcommand{\editorInnen}{Martin Anton Müller und Laura Untner}%% latex-leseansicht-abspann.tex
%% Abspann für die Leseansicht.
%% Der Schalter \ifkorrekturansicht ist bereits durch den Vorspann gesetzt.

%% latex-abspann.tex
%% Gemeinsamer Abspann für Korrekturansicht und Leseansicht.
%% Setzt den Schalter \ifkorrekturansicht voraus (gesetzt in den
%% einbindenden Dateien latex-korrekturansicht-abspann.tex bzw.
%% latex-leseansicht-abspann.tex).
%% ---------------------------------------------------------------

\normalsize

% Das esempio-Environment wird nur in der Leseansicht benötigt
\ifkorrekturansicht\else
\newenvironment{esempio}[3]%
{
    \vspace{1.5ex}
    \rlap{\underline{#1}}
    \par
    \setlength{\parindent}{0cm}
    \nopagebreak
    \leftskip=#2cm
    \rightskip=#3cm
}
{
    \par
}
\fi

\doendnotes{C}
\bigskip
\vfill

\clearpage

\footnotesize

\ifkorrekturansicht
  \lohead{\textsc{register}}
\fi

% theindex-Environment neu definieren ohne reledmac
\makeatletter
\renewenvironment{theindex}{%
  \ifkorrekturansicht
    \section*{\indexname}%
  \else
    \subsubsection*{Index der erwähnten Entitäten}%
  \fi
  \setlength{\parindent}{0pt}%
  \setlength{\parskip}{0pt plus 0.3pt}%
  \let\item\@idxitem
}{%
  \ifkorrekturansicht\clearpage\fi
}
\makeatother

\IfFileExists{\jobname-pw.ind}{\input{\jobname-pw.ind}}{}

% Quellenangabe nur in der Leseansicht
\ifkorrekturansicht\else
% Fallback-Definitionen, falls die .tex-Datei \titel etc. nicht gesetzt hat
\providecommand{\titel}{}
\providecommand{\editorInnen}{}
\providecommand{\dateiname}{\jobname}

\vspace{3cm}

\vfill

\footnotesize
\textsc{Quelle}: \titel. Herausgegeben von {\editorInnen}. In: \emph{Arthur Schnitzler: Briefwechsel mit Autorinnen und Autoren}.
 Digitale Edition, https://schnitzler-briefe.acdh.oeaw.ac.at/{\dateiname}.html (Stand \today)
\fi

\end{document}


