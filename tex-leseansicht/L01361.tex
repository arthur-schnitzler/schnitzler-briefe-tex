%% latex-leseansicht-vorspann.tex
%% Vorspann für die Leseansicht.
%% Lädt die gemeinsame Datei latex-vorspann.tex mit nicht gesetztem Schalter.

\newif\ifkorrekturansicht
\korrekturansichtfalse

\input{../tex-inputs/latex-vorspann}

\begin{center}
            \textcolor{red}{ENTWURF. ENTZIFFERUNG NOCH NICHT KORREKTURGELESEN}
                      \end{center}
            
               \section[Hugo von Hofmannsthal an Arthur Schnitzler, {[}8.–9. 1. 1904{]}]{ Hugo von Hofmannsthal an Arthur Schnitzler, {[}8.–9. 1. 1904{]}}\nopagebreak\mylabel{v}\rehead{ }\begin{ledgroupsized}[t]{13cm}\normalsize\beginnumbering\briefempfaengerindex{Schnitzler, Arthur@\textsc{Schnitzler, Arthur}!zzzHofmannsthal, Hugo von@\emph{von Hugo von Hofmannsthal}!1904-01-083@{{[}8.–9. 1. 1904{]}}|(be} \toendnotes[C]{\smallbreak\pagebreak[2]} \Standort{CUL, Schnitzler, B 43b/1.}
\physDesc{Brief, 1 Blatt, 3 Seiten
\newline{}Handschrift Gertrude von Hofmannsthal: schwarze Tinte, lateinische Kurrent
\newline{}Schnitzler: mit Bleistift datiert: »Jänner 904« und beschriftet: »Hugo« \newline{}Ordnung: mit Bleistift von unbekannter Hand nummeriert:
                                                »251 213a« }\buchAbdrucke{\weitereDrucke{1) Hugo von Hofmannsthal, Arthur Schnitzler: \emph{Briefwechsel}. Hg. Therese Nickl und Heinrich Schnitzler. Frankfurt am Main: \emph{S. Fischer} 1964, S. 181–182.} \weitereDrucke{2) Hermann Bahr, Arthur Schnitzler: \emph{Briefwechsel, Aufzeichnungen, Dokumente
                                (1891–1931)}. Hg. Kurt Ifkovits und Martin Anton Müller. Göttingen: \emph{Wallstein} 2018, S. 288–289.} }\toendnotes[C]{\smallbreak}\pstart
           \noindent{}{\pb}Lieber Arthur, ich bin natürlich äusserst bestürzt über die
                    plötzlich so sehr ernsthaft gewordene Situation Bahrs\pwindex{Bahr, Hermann 19.07.1863 – 15.01.1934@\textsc{Bahr, Hermann} (19.07.1863 – 15.01.1934), \emph{Schriftsteller, Kritiker}|pw}. Die Diagnose Ortner’s\pwindex{Ortner-Rodenstaett, Norbert von 10.08.1865 – 01.03.1935@\textsc{Ortner-Rodenstätt, Norbert von} (10.08.1865 – 01.03.1935), \emph{Mediziner, Internist, Kardiologe}|pw}
                    lautete: schwere Erkrankung der Aorta und der Kranzarterien sowie Angina
                    pectoris. Der Frau Bahr\pwindex{Bahr, Rosa 26.10.1871 – 17.02.1940@\textsc{Bahr, Rosa} (26.10.1871 – 17.02.1940), \emph{Schauspielerin}|pw} scheint der Hausarzt\pwindex{Ortner-Rodenstaett, Norbert von 10.08.1865 – 01.03.1935@\textsc{Ortner-Rodenstätt, Norbert von} (10.08.1865 – 01.03.1935), \emph{Mediziner, Internist, Kardiologe}|pwv} den Zustand als
                    schwere Herzmuskelerkrankung {\pb}bezeichnet und wenig Hoffnung gegeben zu haben{[}.{]}\pend
           \pstart
           Bahr\pwindex{Bahr, Hermann 19.07.1863 – 15.01.1934@\textsc{Bahr, Hermann} (19.07.1863 – 15.01.1934), \emph{Schriftsteller, Kritiker}|pw} reist Mittwoch früh nach dem Sanatorium für Herzkranke in Marbach am
                        Bodensee\oindex{Sanatorium Schloss Marbach am Bodensee@\textbf{Sanatorium Schloss Marbach am Bodensee}|pw} für mindestens 3 Monate. Ich schwanke zwischen einer sehr
                    traurigen Auffassung und einer etwas hoffnungsvolleren, die darauf beruht, dass
                    doch Ihr Bruder\pwindex{Schnitzler, Julius 13.07.1865 – 29.06.1939@\textsc{Schnitzler, Julius} (13.07.1865 – 29.06.1939), \emph{Chirurg}|pwv} ihn {\pb}erst im April
                    untersucht hat ferner die Ärzte im Mai in Edlach\oindex{Edlach@\textbf{Edlach}|pw} und das so plötzliche Eintreten einer so äusserst
                    schweren Erkrankung in diesem Alter mir ganz räthselhaft erscheint.\pend
           \pstart
           Ich hin sehr bekümmert und wünsche mir sehr mit Ihnen darüber zu reden. Von
                    Herzen Ihr{\\[\baselineskip]}\spacefill\mbox{Hugo.}\pend
           \leftskip=0em{}\endnumbering\briefempfaengerindex{Schnitzler, Arthur@\textsc{Schnitzler, Arthur}!zzzHofmannsthal, Hugo von@\emph{von Hugo von Hofmannsthal}!1904-01-083@{{[}8.–9. 1. 1904{]}}|)be}\mylabel{h}\end{ledgroupsized}  \newcommand{\dateiname}{L01361}\newcommand{\titel}{Hugo von Hofmannsthal an Arthur Schnitzler, [8.–9. 1. 1904]}\newcommand{\editorInnen}{ Martin Anton Müller und Gerd-Hermann Susen}%% latex-leseansicht-abspann.tex
%% Abspann für die Leseansicht.
%% Der Schalter \ifkorrekturansicht ist bereits durch den Vorspann gesetzt.

%% latex-abspann.tex
%% Gemeinsamer Abspann für Korrekturansicht und Leseansicht.
%% Setzt den Schalter \ifkorrekturansicht voraus (gesetzt in den
%% einbindenden Dateien latex-korrekturansicht-abspann.tex bzw.
%% latex-leseansicht-abspann.tex).
%% ---------------------------------------------------------------

\normalsize

% Das esempio-Environment wird nur in der Leseansicht benötigt
\ifkorrekturansicht\else
\newenvironment{esempio}[3]%
{
    \vspace{1.5ex}
    \rlap{\underline{#1}}
    \par
    \setlength{\parindent}{0cm}
    \nopagebreak
    \leftskip=#2cm
    \rightskip=#3cm
}
{
    \par
}
\fi

\doendnotes{C}
\bigskip
\vfill

\clearpage

\footnotesize

\ifkorrekturansicht
  \lohead{\textsc{register}}
\fi

% theindex-Environment neu definieren ohne reledmac
\makeatletter
\renewenvironment{theindex}{%
  \ifkorrekturansicht
    \section*{\indexname}%
  \else
    \subsubsection*{Index der erwähnten Entitäten}%
  \fi
  \setlength{\parindent}{0pt}%
  \setlength{\parskip}{0pt plus 0.3pt}%
  \let\item\@idxitem
}{%
  \ifkorrekturansicht\clearpage\fi
}
\makeatother

\IfFileExists{\jobname-pw.ind}{\input{\jobname-pw.ind}}{}

% Quellenangabe nur in der Leseansicht
\ifkorrekturansicht\else
% Fallback-Definitionen, falls die .tex-Datei \titel etc. nicht gesetzt hat
\providecommand{\titel}{}
\providecommand{\editorInnen}{}
\providecommand{\dateiname}{\jobname}

\vspace{3cm}

\vfill

\footnotesize
\textsc{Quelle}: \titel. Herausgegeben von {\editorInnen}. In: \emph{Arthur Schnitzler: Briefwechsel mit Autorinnen und Autoren}.
 Digitale Edition, https://schnitzler-briefe.acdh.oeaw.ac.at/{\dateiname}.html (Stand \today)
\fi

\end{document}


      