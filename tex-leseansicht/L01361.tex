%% latex-korrekturansicht-vorspann.tex
%% Vorspann für die Korrekturansicht.
%% Lädt die gemeinsame Datei latex-vorspann.tex mit gesetztem Schalter.

\newif\ifkorrekturansicht
\korrekturansichttrue

\input{../tex-inputs/latex-vorspann}


\section[Hugo von Hofmannsthal an Arthur Schnitzler, {[}8. – 9. 1. 1904{]}]{L01361 Hugo von Hofmannsthal an Arthur Schnitzler, {[}8. – 9. 1. 1904{]}}
\nopagebreak\mylabel{L01361v}
\rehead{ }\normalsize\beginnumbering\briefempfaengerindex{Schnitzler, Arthur@\textsc{Schnitzler, Arthur}!zzzHofmannsthal, Hugo von@\emph{von Hugo von Hofmannsthal}!1904-01-093@{{[}8. – 9. 1. 1904{]}}|(be}
\toendnotes[C]{\smallbreak\pagebreak[2]}\Standort{CUL, Schnitzler, B 43b/1.}
\physDesc{Brief, 1 Blatt, 3 Seiten, 838 Zeichen
\newline{}Handschrift Gertrude von Hofmannsthal: schwarze Tinte, lateinische Kurrent
\newline{}Schnitzler: mit Bleistift datiert: »Jänner 904« und beschriftet: »Hugo« 
\newline{}Ordnung: mit Bleistift von unbekannter Hand nummeriert: »251
                                    213a« }
\buchAbdrucke{\weitereDrucke{1) Hugo von Hofmannsthal, Arthur Schnitzler: \emph{Briefwechsel}. Frankfurt am Main: \emph{S. Fischer} 1964, S. 181–182.} \weitereDrucke{2) Hermann Bahr, Arthur Schnitzler: \emph{Briefwechsel, Aufzeichnungen, Dokumente (1891–1931)}. Göttingen: \emph{Wallstein} 2018, S. 288–289.} }\toendnotes[C]{\smallbreak}
\pstart
           \noindent{}{\pb}Lieber Arthur, ich bin natürlich äusserst bestürzt über die
               plötzlich so sehr ernsthaft gewordene Situation Bahrs\pwindex{Bahr, Hermann 19.07.1863 – 15.01.1934@\textsc{Bahr, Hermann} (19.07.1863 – 15.01.1934), \emph{Schriftsteller/Schriftstellerin, Kritiker/Kritikerin}|pw}. Die Diagnose Ortner’s\pwindex{Ortner-Rodenstaett, Norbert von 10.08.1865 – 01.03.1935@\textsc{Ortner-Rodenstätt, Norbert von} (10.08.1865 – 01.03.1935), \emph{Mediziner/Medizinerin, Internist/Internistin, Kardiologe/Kardiologin}|pw} lautete:
               schwere Erkrankung der Aorta und der Kranzarterien sowie Angina pectoris. Der Frau
                  Bahr\pwindex{Bahr, Rosa 26.10.1871 – 17.02.1940@\textsc{Bahr, Rosa} (26.10.1871 – 17.02.1940), \emph{Schauspieler/Schauspielerin}|pw} scheint der Hausarzt\pwindex{Ortner-Rodenstaett, Norbert von 10.08.1865 – 01.03.1935@\textsc{Ortner-Rodenstätt, Norbert von} (10.08.1865 – 01.03.1935), \emph{Mediziner/Medizinerin, Internist/Internistin, Kardiologe/Kardiologin}|pwv} den Zustand als schwere
               Herzmuskelerkrankung {\pb}bezeichnet
               und wenig Hoffnung gegeben zu haben{[}.{]}\pend
           
\pstart
           Bahr\pwindex{Bahr, Hermann 19.07.1863 – 15.01.1934@\textsc{Bahr, Hermann} (19.07.1863 – 15.01.1934), \emph{Schriftsteller/Schriftstellerin, Kritiker/Kritikerin}|pw} reist Mittwoch früh nach dem Sanatorium für Herzkranke in Marbach am Bodensee\oindex{Sanatorium Schloss Marbach am Bodensee@\textbf{Sanatorium Schloss Marbach am Bodensee}, \emph{Sanatorium (K.SAN)}|pw}
               für mindestens 3 Monate. Ich schwanke zwischen einer sehr traurigen Auffassung und
               einer etwas hoffnungsvolleren, die darauf beruht, dass doch Ihr Bruder\pwindex{Schnitzler, Julius 13.07.1865 – 29.06.1939@\textsc{Schnitzler, Julius} (13.07.1865 – 29.06.1939), \emph{Chirurg/Chirurgin}|pwv} ihn {\pb}erst im April
               untersucht hat ferner die Ärzte im Mai in Edlach\oindex{Edlach@\textbf{Edlach}, \emph{P.PPL}|pw} und das so plötzliche Eintreten einer so äusserst schweren Erkrankung
               in diesem Alter mir ganz räthselhaft erscheint.\pend
           
\pstart
           Ich hin sehr bekümmert und wünsche mir sehr mit Ihnen darüber zu reden. Von
               Herzen Ihr{\\[\baselineskip]}\spacefill\mbox{Hugo.}\pend
           \leftskip=0em{}\selectlanguage{ngerman}\endnumbering\briefempfaengerindex{Schnitzler, Arthur@\textsc{Schnitzler, Arthur}!zzzHofmannsthal, Hugo von@\emph{von Hugo von Hofmannsthal}!3@{{[}8. – 9. 1. 1904{]}}|)be}\mylabel{L01361h}  \normalsize

\doendnotes{C}
\bigskip
\vfill

\clearpage

\footnotesize

\lohead{\textsc{register}}

% Definiere theindex-Environment komplett neu ohne reledmac
\makeatletter
\renewenvironment{theindex}{%
  \section*{\indexname}%
  \setlength{\parindent}{0pt}%
  \setlength{\parskip}{0pt plus 0.3pt}%
  \let\item\@idxitem
}{%
  \clearpage
}
\makeatother

\IfFileExists{\jobname-pw.ind}{\input{\jobname-pw.ind}}{}

\end{document}

      