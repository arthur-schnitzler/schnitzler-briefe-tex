%% latex-korrekturansicht-vorspann.tex
%% Vorspann für die Korrekturansicht.
%% Lädt die gemeinsame Datei latex-vorspann.tex mit gesetztem Schalter.

\newif\ifkorrekturansicht
\korrekturansichttrue

\input{../tex-inputs/latex-vorspann}


\section[Arthur Schnitzler an Hugo von Hofmannsthal, 4. 8. 1900]{L01065 Arthur Schnitzler an Hugo von Hofmannsthal, 4. 8. 1900}
\nopagebreak\mylabel{L01065v}
\rehead{ }\normalsize\beginnumbering\briefempfaengerindex{Hofmannsthal, Hugo von@\textsc{Hofmannsthal, Hugo von}!zzzSchnitzler, Arthur@\emph{von Arthur Schnitzler}!1900-08-041@{4. 8. 1900}|(be}
\toendnotes[C]{\smallbreak\pagebreak[2]}\Standort{FDH, Hs-30885,1.}
\physDesc{Brief, 1 Blatt, 3 Seiten, 776 Zeichen
\newline{}Handschrift: schwarze Tinte, deutsche Kurrent}
\buchAbdrucke{\weitereDrucke{Hugo von Hofmannsthal, Arthur Schnitzler: \emph{Briefwechsel}. Frankfurt am Main: \emph{S. Fischer} 1964, S. 144.} }
\pstart
           \raggedleft{}{\pb}Iſchl\oindex{Bad Ischl@\textbf{Bad Ischl}, \emph{P.PPL}|pw}, 4. 8. 900.\pend
           \vspace{0.5em}
\pstart
           Mein lieber Hugo, ich bin ein paar Tage in Auſſee\oindex{Bad Aussee@\textbf{Bad Aussee}, \emph{P.PPLA3}|pw} geweſen, jetzt in Iſchl\oindex{Bad Ischl@\textbf{Bad Ischl}, \emph{P.PPL}|pw}, \textsc{Pension Petter}\oindex{Hotel und Pension Rudolfshoehe (Leopold Petter)@\textbf{Hotel und Pension Rudolfshöhe (Leopold Petter)}, \emph{Hotel (K.HTL)}|pw}, habe vor meinem Fenſter, auch jetzt, während ich ſchreibe, den ſchmalen Weg,
               auf dem wir im vorigen Jahr nach dem Eſſen immer ſpazieren gegangen ſind und über Schleier\pwindex{Schleier der Beatrice. Schauspiel in fuenf Akten@\emph{Der Schleier der Beatrice. Schauspiel in fünf Akten}|pw} und Bergwerk\pwindex{Bergwerk zu Falun@\emph{Das Bergwerk zu Falun}|pw} geſpro{\pb}chen haben. Heuer geht es mir hier
               nicht ſo gut. Am 10. wahrſcheinlich fahr ich weg, am 12.
               dürft ich in Salzburg\oindex{Salzburg@\textbf{Salzburg}, \emph{A.ADM2}|pw}{ }ſein und freue mich ſehr Sie dort noch anzutreffen
               u. Ihnen mündlich ſagen zu können, wie ſehr von Herzen ich Ihnen Glück wünſche. Aber
               bevor ich Iſchl\oindex{Bad Ischl@\textbf{Bad Ischl}, \emph{P.PPL}|pw} verlaſſe, ſchreib ich Ihnen noch
               ein Wort und höre vielleicht auch noch von Ihnen. Sie wiſſen ja, {\pb}dſs Richard\pwindex{Beer-Hofmann, Richard 1866-07-11 – 1945-09-26@\textsc{Beer-Hofmann, Richard} (1866-07-11 – 1945-09-26), \emph{Schriftsteller/Schriftstellerin}|pw} auch
               nach S.\oindex{Salzburg@\textbf{Salzburg}, \emph{A.ADM2}|pw} ko{\geminationm}t,
               vielleicht auch Goldmann\pwindex{Goldmann, Paul 31.01.1865 – 25.09.1935@\textsc{Goldmann, Paul} (31.01.1865 – 25.09.1935), \emph{Schriftsteller/Schriftstellerin, Journalist/Journalistin}|pw}.\pend
           
\pstart
           Am 13.{ }Nachmittag dürften wir aufbrechen; ſpäteſtens am 14.\hspace*{2.5em}Auf Wiederſehen! Ihr \spacefill\mbox{Arthur.}\pend
           \selectlanguage{ngerman}\endnumbering\briefempfaengerindex{Hofmannsthal, Hugo von@\textsc{Hofmannsthal, Hugo von}!zzzSchnitzler, Arthur@\emph{von Arthur Schnitzler}!1900-08-041@{4. 8. 1900}|)be}\mylabel{L01065h}  \normalsize

\doendnotes{C}
\bigskip
\vfill

\clearpage

\footnotesize

\lohead{\textsc{register}}

% Definiere theindex-Environment komplett neu ohne reledmac
\makeatletter
\renewenvironment{theindex}{%
  \section*{\indexname}%
  \setlength{\parindent}{0pt}%
  \setlength{\parskip}{0pt plus 0.3pt}%
  \let\item\@idxitem
}{%
  \clearpage
}
\makeatother

\IfFileExists{\jobname-pw.ind}{\input{\jobname-pw.ind}}{}

\end{document}

      