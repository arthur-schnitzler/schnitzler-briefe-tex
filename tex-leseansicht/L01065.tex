\input{../tex-inputs/latex-pdf-vorspann}
\begin{center}
            \textcolor{red}{ENTWURF. ENTZIFFERUNG NOCH NICHT KORREKTURGELESEN}
                      \end{center}
            
               \section[Arthur Schnitzler an Hugo von Hofmannsthal, 4. 8. 1900]{ Arthur Schnitzler an Hugo von Hofmannsthal, 4. 8. 1900}\nopagebreak\mylabel{v}\rehead{ }\begin{ledgroupsized}[t]{13cm}\normalsize\beginnumbering\briefempfaengerindex{Hofmannsthal, Hugo von@\textsc{Hofmannsthal, Hugo von}!zzzSchnitzler, Arthur@\emph{von Arthur Schnitzler}!1900-08-041@{4. 8. 1900}|(be} \toendnotes[C]{\smallbreak\pagebreak[2]} \Standort{FDH, Hs-30885,1.}
\physDesc{Brief, 1 Blatt, 3 Seiten
\newline{}Handschrift: schwarze Tinte, deutsche Kurrent}\buchAbdrucke{\weitereDrucke{Hugo von Hofmannsthal, Arthur Schnitzler: \emph{Briefwechsel}. Hg. Therese Nickl und Heinrich Schnitzler. Frankfurt am Main: \emph{S. Fischer} 1964, S. 144.} }\pstart
           \raggedleft{}{\pb}Iſchl\oindex{Bad Ischl@\textbf{Bad Ischl}|pw}, 4. 8. 900.\pend
           \pstart
           Mein lieber Hugo, ich bin ein paar Tage in Auſſee\oindex{Bad Aussee@\textbf{Bad Aussee}|pw} geweſen, jetzt in Iſchl\oindex{Bad Ischl@\textbf{Bad Ischl}|pw}, \textsc{Pension Petter}\oindex{Hotel und Pension Rudolfshoehe (Leopold Petter)@\textbf{Hotel und Pension Rudolfshöhe (Leopold Petter)}|pw}, habe vor meinem Fenſter, auch jetzt, während ich ſchreibe, den ſchmalen
                    Weg, auf dem wir im vorigen Jahr nach dem Eſſen immer ſpazieren gegangen ſind
                    und über Schleier\pwindex{Schnitzler, Arthur 15.05.1862 – 21.10.1931@\textsc{Schnitzler, Arthur} (15.05.1862 – 21.10.1931), \emph{Schriftsteller, Mediziner}!Schleier der Beatrice. Schauspiel in fuenf Akten1900-12-01 – 1900-12-01@\strich\emph{Der Schleier der Beatrice. Schauspiel in fünf Akten} {[}1900-12-01 – 1900-12-01{]}|pw} und Bergwerk\pwindex{Hofmannsthal, Hugo von 01.02.1874 – 15.07.1929@\textsc{Hofmannsthal, Hugo von} (01.02.1874 – 15.07.1929), \emph{Schriftsteller}!Bergwerk zu Falun1900 – 1933@\strich\emph{Das Bergwerk zu Falun} {[}1900 – 1933{]}|pw} geſpro{\pb}chen haben. Heuer
                    geht es mir hier nicht ſo gut. Am 10. wahrſcheinlich fahr ich weg,
                    am 12. dürft ich in Salzburg\oindex{Salzburg@\textbf{Salzburg}|pw}{ }ſein und freue mich ſehr Sie dort noch
                    anzutreffen u. Ihnen mündlich ſagen zu können, wie ſehr von Herzen ich Ihnen
                    Glück wünſche. Aber bevor ich Iſchl\oindex{Bad Ischl@\textbf{Bad Ischl}|pw} verlaſſe,
                    ſchreib ich Ihnen noch ein Wort und höre vielleicht auch noch von Ihnen. Sie
                    wiſſen ja, {\pb}dſs Richard\pwindex{Beer-Hofmann, Richard 11.07.1866 – 26.09.1945@\textsc{Beer-Hofmann, Richard} (11.07.1866 – 26.09.1945), \emph{Schriftsteller}|pw} auch nach S.\oindex{Salzburg@\textbf{Salzburg}|pw} ko{\geminationm}t, vielleicht auch Goldmann\pwindex{Goldmann, Paul 31.01.1865 – 25.09.1935@\textsc{Goldmann, Paul} (31.01.1865 – 25.09.1935), \emph{Schriftsteller, Journalist}|pw}.\pend
           \pstart
           Am 13.{ }Nachmittag dürften wir aufbrechen; ſpäteſtens am
                        14.\hspace*{2.5em}Auf Wiederſehen! Ihr
                        \spacefill\mbox{Arthur.}\pend
           \endnumbering\briefempfaengerindex{Hofmannsthal, Hugo von@\textsc{Hofmannsthal, Hugo von}!zzzSchnitzler, Arthur@\emph{von Arthur Schnitzler}!1900-08-041@{4. 8. 1900}|)be}\mylabel{h}\end{ledgroupsized}  \newcommand{\dateiname}{L01065}\newcommand{\titel}{Arthur Schnitzler an Hugo von Hofmannsthal, 4. 8. 1900}\newcommand{\editorInnen}{Martin Anton Müller und Gerd-Hermann Susen}\input{../tex-inputs/latex-pdf-abspann}
      