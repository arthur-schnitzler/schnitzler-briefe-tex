%% latex-leseansicht-vorspann.tex
%% Vorspann für die Leseansicht.
%% Lädt die gemeinsame Datei latex-vorspann.tex mit nicht gesetztem Schalter.

\newif\ifkorrekturansicht
\korrekturansichtfalse

\input{../tex-inputs/latex-vorspann}

\begin{center}
            \textcolor{red}{ENTWURF. ENTZIFFERUNG NOCH NICHT KORREKTURGELESEN}
                      \end{center}
            
               \section[Arthur Schnitzler an Georg Brandes, 19. 5. 1899]{ Arthur Schnitzler an Georg Brandes, 19. 5. 1899}\nopagebreak\mylabel{v}\rehead{ }\begin{ledgroupsized}[t]{13cm}\normalsize\beginnumbering\briefempfaengerindex{Brandes, Georg@\textsc{Brandes, Georg}!zzzSchnitzler, Arthur@\emph{von Arthur Schnitzler}!1899-05-191@{19. 5. 1899}|(be} \toendnotes[C]{\smallbreak\pagebreak[2]} \Standort{Kopenhagen, Det Kongelige Bibliotek, Georg Brandes Arkiv, box 125.}
\physDesc{Brief, 2 Blätter, 6 Seiten
\newline{}Handschrift: schwarze Tinte, deutsche Kurrent\newline{}Ordnung: mit Bleistift von unbekannter Hand nummeriert und datiert:
                                    »16. Schnitzler 19/5 99« und auf der sechsten Seite:
                                 »Schnitzler« }\buchAbdrucke{\weitereDrucke{Georg Brandes, Arthur Schnitzler: \emph{Ein Briefwechsel}. Hg. Kurt Bergel. Bern: \emph{Francke} 1956, S. 77.} }\toendnotes[C]{\smallbreak}\pstart{}{\pb}Lieber und verehrter Herr Brandes,\pend\pstart
           innigen Dank für Ihre herzlichen Worte. Es iſt etwas erquickendes in der Art, wie Sie
               einem Worte ſagen, die von einem andern ausgeſprochen, eben nichts als Worte wären.
               Ich bin jung, ſagen Sie? Nun, wenn es ſelbſt ſo wäre – unter gewiſſen Umſtänden ſind
               Jugend, Frühling, Sonne ſo traurige Dinge, daſs man in ihrem Bewußtſein zuſa{\geminationm}enſchauert ſtatt ſich zu {\pb}freun. Dieſe Abende, die ich jetzt manchmal auf
               dem Land draußen verbringe, die Orte wo ich hinkomme, alles das dampft von
               Erinnerungen; – ahnt man denn, wie tief manche Gräber ſind! –\pend
           \pstart
           Verzeihen Sie daſs ich ſchon wieder davon rede; während Sie ſelbſt ohnedies nicht in
               der glücklichſten Sti{\geminationm}ung ſind. Ich wußte abſolut nicht,
               dſs Sie noch immer bettläge{\pb}rig \substVorne{}\textsuperscript{ſind}\substDazwischen{}waren\substHinten{}; wie gern möcht ich endlich hören, dſs Sie ganz geneſen ſind. Dabei iſt doch
               ſehr erfreulich, dſs die Sache völlig unbedenklich iſt und daſs Sie dabei arbeiten
               und ſich über den Zuſa{\geminationm}enfluſs von Büchern und Briefen
               auf Ihre\substVorne{}\textsuperscript{m}\substDazwischen{}r\substHinten{} Bettdecke freuen. Der Erfolg Ihrer Geſa{\geminationm}tausgabe\pwindex{Brandes, Georg 04.02.1842 – 19.02.1927@\textsc{Brandes, Georg} (04.02.1842 – 19.02.1927)!Samlede Skrifter [Gesammelte Werke]1899 – 1910@\strich\emph{Samlede Skrifter [Gesammelte Werke]} {[}1899 – 1910{]}|pwv} iſt ja
               ſelbſtverſtändlich. Ludwig Fulda\pwindex{Fulda, Ludwig 15.07.1862 – 30.03.1939@\textsc{Fulda, Ludwig} (15.07.1862 – 30.03.1939), \emph{Schriftsteller, Übersetzer}|pw}, auf deſſen
               Schreibtiſch ich vor ein paar Wochen {\pb}Ihre Gedichte\pwindex{Brandes, Georg 04.02.1842 – 19.02.1927@\textsc{Brandes, Georg} (04.02.1842 – 19.02.1927)!Ungdomsvers [Jugendgedichte]1898@\strich\emph{Ungdomsvers [Jugendgedichte]} {[}1898{]}|pwv} liegen ſah, hab ich ein
               wenig um ſein däniſch\oindex{Daenemark@\textbf{Dänemark}|pw} können beneidet. Die Zukunft\orgindex{Zukunft@Die Zukunft|pw}snu{\geminationm}er vom
                  7. April hab ich noch nicht geſehen, laſſe ſie mir durch meine
               Buchhandlung kommen.\pend
           \pstart
           Ich will in dieſem Frühjahr noch einige kleine Touren (mit dem Rade zumeiſt) in der
               Umgegend von Wien\oindex{Wien@\textbf{Wien}|pw} machen; immer neues entdeckt man in
               dieſem wunderſchönen aber vertrottelten Niederoeſterreich\oindex{Niederoesterreich@\textbf{Niederösterreich}|pw}.\pend
           \pstart
           {\pb}Leben Sie wohl, mein verehrter Herr Brandes und
               ſeien vielmals gegrüßt.\pend
           \pstart Ihr \spacefill\mbox{ArthurSchnitzler}\pend{}\pstart
           19. 5. 99.\pend
           \endnumbering\briefempfaengerindex{Brandes, Georg@\textsc{Brandes, Georg}!zzzSchnitzler, Arthur@\emph{von Arthur Schnitzler}!1899-05-191@{19. 5. 1899}|)be}\mylabel{h}\end{ledgroupsized}  \newcommand{\dateiname}{L00917}\newcommand{\titel}{Arthur Schnitzler an Georg Brandes, 19. 5. 1899}\newcommand{\editorInnen}{Martin Anton Müller und Gerd-Hermann Susen}%% latex-leseansicht-abspann.tex
%% Abspann für die Leseansicht.
%% Der Schalter \ifkorrekturansicht ist bereits durch den Vorspann gesetzt.

%% latex-abspann.tex
%% Gemeinsamer Abspann für Korrekturansicht und Leseansicht.
%% Setzt den Schalter \ifkorrekturansicht voraus (gesetzt in den
%% einbindenden Dateien latex-korrekturansicht-abspann.tex bzw.
%% latex-leseansicht-abspann.tex).
%% ---------------------------------------------------------------

\normalsize

% Das esempio-Environment wird nur in der Leseansicht benötigt
\ifkorrekturansicht\else
\newenvironment{esempio}[3]%
{
    \vspace{1.5ex}
    \rlap{\underline{#1}}
    \par
    \setlength{\parindent}{0cm}
    \nopagebreak
    \leftskip=#2cm
    \rightskip=#3cm
}
{
    \par
}
\fi

\doendnotes{C}
\bigskip
\vfill

\clearpage

\footnotesize

\ifkorrekturansicht
  \lohead{\textsc{register}}
\fi

% theindex-Environment neu definieren ohne reledmac
\makeatletter
\renewenvironment{theindex}{%
  \ifkorrekturansicht
    \section*{\indexname}%
  \else
    \subsubsection*{Index der erwähnten Entitäten}%
  \fi
  \setlength{\parindent}{0pt}%
  \setlength{\parskip}{0pt plus 0.3pt}%
  \let\item\@idxitem
}{%
  \ifkorrekturansicht\clearpage\fi
}
\makeatother

\IfFileExists{\jobname-pw.ind}{\input{\jobname-pw.ind}}{}

% Quellenangabe nur in der Leseansicht
\ifkorrekturansicht\else
% Fallback-Definitionen, falls die .tex-Datei \titel etc. nicht gesetzt hat
\providecommand{\titel}{}
\providecommand{\editorInnen}{}
\providecommand{\dateiname}{\jobname}

\vspace{3cm}

\vfill

\footnotesize
\textsc{Quelle}: \titel. Herausgegeben von {\editorInnen}. In: \emph{Arthur Schnitzler: Briefwechsel mit Autorinnen und Autoren}.
 Digitale Edition, https://schnitzler-briefe.acdh.oeaw.ac.at/{\dateiname}.html (Stand \today)
\fi

\end{document}


      