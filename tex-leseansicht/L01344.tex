%% latex-leseansicht-vorspann.tex
%% Vorspann für die Leseansicht.
%% Lädt die gemeinsame Datei latex-vorspann.tex mit nicht gesetztem Schalter.

\newif\ifkorrekturansicht
\korrekturansichtfalse

\input{../tex-inputs/latex-vorspann}


\section[Hugo von Hofmannsthal an Arthur Schnitzler, 13. 11. 1903]{L01344 Hugo von Hofmannsthal an Arthur Schnitzler, 13. 11. 1903}
\nopagebreak\mylabel{L01344v}
\rehead{ }\normalsize\beginnumbering\briefempfaengerindex{Schnitzler, Arthur@\textsc{Schnitzler, Arthur}!zzzHofmannsthal, Hugo von@\emph{von Hugo von Hofmannsthal}!1903-11-133@{13. 11. 1903}|(be}
\toendnotes[C]{\smallbreak\pagebreak[2]}
\correspDesc{Versand  durch Hugo von Hofmannsthal am 13. 11. 1903 in Rodaun
\newline{}Erhalt  durch Arthur Schnitzler im Zeitraum [14. 11. 1903 – 18. 11. 1903?] in Wien}\toendnotes[C]{\smallbreak}
\buchAlsQuelle{Hugo von Hofmannsthal, Arthur Schnitzler: \emph{Briefwechsel}. Herausgegeben von Therese Nickl und Heinrich Schnitzler. Frankfurt am Main: \emph{S. Fischer} 1964, S. 177.}\toendnotes[C]{\smallbreak}
\pstart
           \raggedleft{}{\pb}Rodaun\oindex{Wien@\textbf{Wien}!XXIII., Liesing@\textbf{XXIII., Liesing}!Rodaun@\textbf{Rodaun}, \emph{Region}|pw}, 13. 11. 1903\pend
           
\pstart{}Mein lieber Arthur,\pend\vspace{0.5em}
\pstart
           der »\so{einsame Weg}\pwindex{Schnitzler, Arthur 15.\,5.\,1862 Wien – 21.\,10.\,1931 ebd.@\textsc{Schnitzler, Arthur} (15.\,5.\,1862 Wien – 21.\,10.\,1931 ebd.), \emph{Schriftsteller, Mediziner}!einsame Weg. Schauspiel in fünf Akten@\strich\emph{Der einsame Weg. Schauspiel in fünf Akten}|pw}« ist ein schönes großes Theaterstück, dessen mit nichts zu vergleichende
               geistig-gespenstische und doch wieder reale Gestalten einen mit unglaublicher Kraft
               halten und halten, und nach einer ziemlich unruhigen Nacht, die sie verschuldet
               haben, am Morgen noch lebendiger, saugender in einem und um einen da sind. Der Ton,
               in dem da in einer geheimnisvoll verdünnten Luft ganze Existenzen miteinander ringen,
               miteinander abrechnen, Vergangenheit und Gegenwart ineinander wechselweise aufheben
               und sich ineinander verwinden, die geheimnisvollsten Verschuldungen ihre intimste
               feinste Bestrafung finden, – diesen Ton werde ich nie ganz vergessen und nie die
               Stunde, wo ich ihn zum ersten Mal gehört habe. Er war mir vielleicht um desto
               ergreifender, dieser Ton, weil er noch nicht ganz erobert, nicht ganz gesichert war
               und weil so, für den erregten Zuhörer, zu den überreichen Vorgängen des Dramas noch
               ein andres, Mitschwingendes dazukam: zu fühlen, wie Sie, in den bewegten Schatten
               dieses Dramas, für Monate Ihr ganzes Dasein, Ihr menschliches-künstlerisches,
               einziges Dasein, in einer Weise besessen haben, wie nie zuvor – besessen bis zum
               Erschaudern. – Ich bin sehr glücklich, lieber Arthur, daß Sie etwas so Schönes,
               Tiefes, mit nichts Vergleichbares machen konnten.\pend
           
\pstart
           Von Herzen{\\[\baselineskip]}Ihr\spacefill\mbox{Hugo}\pend
           \leftskip=0em{}
\pstart
           \noindent{}P. S. Felix\pwindex{Schnitzler, Arthur 15.\,5.\,1862 Wien – 21.\,10.\,1931 ebd.@\textsc{Schnitzler, Arthur} (15.\,5.\,1862 Wien – 21.\,10.\,1931 ebd.), \emph{Schriftsteller, Mediziner}!einsame Weg. Schauspiel in fünf Akten@\strich\emph{Der einsame Weg. Schauspiel in fünf Akten}|pwv}’ { }\so{erste} Worte: »\label{K_L01344-1v}\edtext{Die Begeisterung scheint nicht gerade groß zu sein}{\lemma{\textnormal{\emph{Die … sein}}}\Cendnote{\textnormal{Im gedruckten Text sagt Felix:
                        »Es ist nicht auf lang. –«. (1. Akt,
                     1. Szene.)}}}\label{K_L01344-1}« (oder so ähnlich), haben einen so saloppen, anatol\pwindex{Schnitzler, Arthur 15.\,5.\,1862 Wien – 21.\,10.\,1931 ebd.@\textsc{Schnitzler, Arthur} (15.\,5.\,1862 Wien – 21.\,10.\,1931 ebd.), \emph{Schriftsteller, Mediziner}!Anatol@\strich\emph{Anatol}|pw}-mäßigen \label{K_L01344-2v}\edtext{jour-Ton}{\lemma{\textnormal{\emph{jour-Ton}}}\Cendnote{\textnormal{französisch \begin{otherlanguage}{french}jour\end{otherlanguage}: Tag; hier wird der Ausdruck in der speziellen Bedeutung eines festgesetzten
                     Wochentages verwendet, an dem Besuch empfangen wird.}}}\label{K_L01344-2}, daß sie einem die Figur für
                  5 Minuten ganz falsch hinstellen. Warum soll dieser Mensch zu seiner rechten Schwester\pwindex{Schnitzler, Arthur 15.\,5.\,1862 Wien – 21.\,10.\,1931 ebd.@\textsc{Schnitzler, Arthur} (15.\,5.\,1862 Wien – 21.\,10.\,1931 ebd.), \emph{Schriftsteller, Mediziner}!einsame Weg. Schauspiel in fünf Akten@\strich\emph{Der einsame Weg. Schauspiel in fünf Akten}|pwv} nicht einfach
                  sagen: »Nun, deine Freude über meine Ankunft scheint mir nicht gerade groß«{\dots} oder so ähnlich. Dieses Wort »Begeisterung«: nämlich
                  ein großes Wort wählen, um es dann durch ironischen Ton sogleich zu drücken, also
                     \label{K_L01344-3v}\edtext{hausse und {\pb}baisse}{\lemma{\textnormal{\emph{hausse und baisse}}}\Cendnote{\textnormal{französisch: Zunahme und Abnahme}}}\label{K_L01344-3} in einem Satz
                  veranstalten, ist direct jüdisch-wien\oindex{Wien@\textbf{Wien}, \emph{Verwaltungsgebiet}|pw}erischer
                  Jargon und Felix\pwindex{Schnitzler, Arthur 15.\,5.\,1862 Wien – 21.\,10.\,1931 ebd.@\textsc{Schnitzler, Arthur} (15.\,5.\,1862 Wien – 21.\,10.\,1931 ebd.), \emph{Schriftsteller, Mediziner}!einsame Weg. Schauspiel in fünf Akten@\strich\emph{Der einsame Weg. Schauspiel in fünf Akten}|pwv} würde das
                  gewiß nicht in den Mund nehmen.\pend
           \selectlanguage{ngerman}\endnumbering\briefempfaengerindex{Schnitzler, Arthur@\textsc{Schnitzler, Arthur}!zzzHofmannsthal, Hugo von@\emph{von Hugo von Hofmannsthal}!1903-11-133@{13. 11. 1903}|)be}\mylabel{L01344h}  \newcommand{\dateiname}{L01344}\newcommand{\titel}{Hugo von Hofmannsthal an Arthur Schnitzler, 13. 11. 1903}\newcommand{\editorInnen}{Martin Anton Müller und Gerd-Hermann Susen}%% latex-leseansicht-abspann.tex
%% Abspann für die Leseansicht.
%% Der Schalter \ifkorrekturansicht ist bereits durch den Vorspann gesetzt.

%% latex-abspann.tex
%% Gemeinsamer Abspann für Korrekturansicht und Leseansicht.
%% Setzt den Schalter \ifkorrekturansicht voraus (gesetzt in den
%% einbindenden Dateien latex-korrekturansicht-abspann.tex bzw.
%% latex-leseansicht-abspann.tex).
%% ---------------------------------------------------------------

\normalsize

% Das esempio-Environment wird nur in der Leseansicht benötigt
\ifkorrekturansicht\else
\newenvironment{esempio}[3]%
{
    \vspace{1.5ex}
    \rlap{\underline{#1}}
    \par
    \setlength{\parindent}{0cm}
    \nopagebreak
    \leftskip=#2cm
    \rightskip=#3cm
}
{
    \par
}
\fi

\doendnotes{C}
\bigskip
\vfill

\clearpage

\footnotesize

\ifkorrekturansicht
  \lohead{\textsc{register}}
\fi

% theindex-Environment neu definieren ohne reledmac
\makeatletter
\renewenvironment{theindex}{%
  \ifkorrekturansicht
    \section*{\indexname}%
  \else
    \subsubsection*{Index der erwähnten Entitäten}%
  \fi
  \setlength{\parindent}{0pt}%
  \setlength{\parskip}{0pt plus 0.3pt}%
  \let\item\@idxitem
}{%
  \ifkorrekturansicht\clearpage\fi
}
\makeatother

\IfFileExists{\jobname-pw.ind}{\input{\jobname-pw.ind}}{}

% Quellenangabe nur in der Leseansicht
\ifkorrekturansicht\else
% Fallback-Definitionen, falls die .tex-Datei \titel etc. nicht gesetzt hat
\providecommand{\titel}{}
\providecommand{\editorInnen}{}
\providecommand{\dateiname}{\jobname}

\vspace{3cm}

\vfill

\footnotesize
\textsc{Quelle}: \titel. Herausgegeben von {\editorInnen}. In: \emph{Arthur Schnitzler: Briefwechsel mit Autorinnen und Autoren}.
 Digitale Edition, https://schnitzler-briefe.acdh.oeaw.ac.at/{\dateiname}.html (Stand \today)
\fi

\end{document}


