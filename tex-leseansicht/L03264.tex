%% latex-leseansicht-vorspann.tex
%% Vorspann für die Leseansicht.
%% Lädt die gemeinsame Datei latex-vorspann.tex mit nicht gesetztem Schalter.

\newif\ifkorrekturansicht
\korrekturansichtfalse

\input{../tex-inputs/latex-vorspann}


\section[ Felix Salten an Arthur Schnitzler, 5. 5. 1897]{L03264 Felix Salten an Arthur Schnitzler,  5. 5. 1897}
\nopagebreak\mylabel{L03264v}
\rehead{ }\normalsize\beginnumbering\briefempfaengerindex{Schnitzler, Arthur@\textsc{Schnitzler, Arthur}!zzzSalten, Felix@\emph{von Felix Salten}!1897-05-051@{5. 5. 1897}|(be}
\toendnotes[C]{\smallbreak\pagebreak[2]}
\correspDesc{Versand  durch Felix Salten am 5. 5. 1897 in Wien
\newline{}Erhalt  durch Arthur Schnitzler im Zeitraum [5. 5. 1897
                  – 9. 5. 1897?] in Wien}\toendnotes[C]{\smallbreak}
\Standort{CUL, Schnitzler, B 89, A 2.}
\physDesc{Brief, 1 Blatt, 3 Seiten, 2588 Zeichen
\newline{}Handschrift: schwarze Tinte, lateinische Kurrent
\newline{}Ordnung: mit Bleistift von unbekannter Hand nummeriert: »87« }\toendnotes[C]{\smallbreak}
\pstart
           \raggedleft{}{\pb}Wien\oindex{Wien@\textbf{Wien}, \emph{Verwaltungsgebiet}|pw}, am 5. Mai 97.\pend
           \vspace{0.5em}
\pstart
           Lieber Arthur, seit ein paar Tagen bin ich wieder in Wien\oindex{Wien@\textbf{Wien}, \emph{Verwaltungsgebiet}|pw}. Ich war in Riva\oindex{Riva del Garda@\textbf{Riva del Garda}, \emph{Hauptstadt}|pw} – sehr schön. Aber es hätte viel schöner sein können, wenn die Mitterwurzer\pwindex{Mitterwurzer, Wilhelmine 27.\,3.\,1848 Freiburg im Breisgau – 3.\,8.\,1909 Wien@\textsc{Mitterwurzer, Wilhelmine} (27.\,3.\,1848 Freiburg im Breisgau – 3.\,8.\,1909 Wien), \emph{Schauspielerin}|pw} nicht dabei gewesen wäre. Hier\oindex{Wien@\textbf{Wien}, \emph{Verwaltungsgebiet}|pwv} lebe ich in einer
               merkwürdigen Sorglosigkeit. Eigentlich begreife ich es selbst nicht, warum ich mich
               so völlig unbekümmert hintreiben laße. Manchmal sage ich mir, dass irgend eine
               günstige Wendung bevorsteht, dass ich sie in allen Gliedern spüre und dass ich
               deshalb so frei bin. Dabei fällt mir immer ein, was Sie mir gelegentlich sagten: Dass
               man sich bei mir immer eines Glückfalles versieht. Für Ihren Brief dank ich Ihnen
               sehr. Es war ja nicht viel, aber etwas, und ich bin jetzt – Frl. M.\pwindex{Salten, Ottilie 7.\,3.\,1868 Prag – 22.\,6.\,1942 Zürich@\textsc{Salten, Ottilie} (7.\,3.\,1868 Prag – 22.\,6.\,1942 Zürich), \emph{Schauspielerin}|pw} ausgenommen – sehr einsam. Ich arbeite ordentlich dabei
                  {\pb}und nach allen Seiten. Es
               geht kein Tag hin, an dem mir nicht etwas Erfreuliches einfiele. Dabei bin ich jetzt
               an Büchern und Menschen und an den Erinnerungen vorbei auf einem ziemlich directen
               Weg zu mir selbst. Es ist eine eigenthümlich aufregende Zeit. Frühling kann man nicht
               sagen, – denn es ist etwas \substVorne{}\textsuperscript{z}\substDazwischen{}Z\substHinten{}weites, alles ist dezidirter, kühler und alle Formen sind ohne den
               ahnungsvollen Nebel, und klarer. Es gibt keinen Menschen, kein Buch, nichts in meinem
               Leben, zu dem ich nicht eine \substVorne{}\textsuperscript{v}\substDazwischen{}t\substHinten{}otal veränderte Beziehung hätte, als vorher. Das ist natürlich nicht erst in
               acht Tagen geworden, aber erst auf meiner Reise, und dann jetzt hier, habe ich ein
               wenig Ordnung mit diesen Dingen gemacht, und meine Interessen gesäubert.\pend
           
\pstart
           Von äußeren Umständen weiß ich Ihnen {\pb}nichts Neues zu sagen.
               Vielleicht finde ich eine Stellung – Ludaßy\pwindex{Gans-Ludassy, Julius von 13.\,4.\,1858 Wien – 30.\,9.\,1922 ebd.@\textsc{Gans-Ludassy, Julius von} (13.\,4.\,1858 Wien – 30.\,9.\,1922 ebd.), \emph{Schriftsteller, Journalist, Herausgeber}|pw}
               behauptet, \strikeout{\textcolor{gray}{d}} er habe große Dinge vor, – vielleicht wird etwas mit einer \label{K_L03264-1v}\edtext{Direction}{\lemma{\textnormal{\emph{Direction}}}\Cendnote{\textnormal{Vgl. XXXX Auszeichnungsfehler: Dokument L03262 nicht gefunden.
               }}}\label{K_L03264-1}; vielleicht schreibe ich von den Stoffen, mit denen ich mich jetzt
               beschäftige einen zu Ende, – das letztere ist das wahrscheinlichste. \strikeout{\textcolor{gray}{×}\-\textcolor{gray}{×}\-\textcolor{gray}{×}} Das Bicycle und Frl. M.\pwindex{Salten, Ottilie 7.\,3.\,1868 Prag – 22.\,6.\,1942 Zürich@\textsc{Salten, Ottilie} (7.\,3.\,1868 Prag – 22.\,6.\,1942 Zürich), \emph{Schauspielerin}|pw} füllen meine
               übrige Zeit aus. Seit der Reise ist auch hier eine entscheidende Wendung eingetreten.
               Das macht mich auch besser und ruhiger und gibt meinem Leben wieder einen vollen
               Duft, denn ich habe lange Niemanden lieb gehabt. Sonst leb ich mit keinem Menschen
               und habe Keinen, mit dem ich sprechen möchte.\pend
           
\pstart
           Bei den übrigen ist, glaub ich, alles beim Alten, oder doch nichts wesentliches
               geschehen. Hugo\pwindex{Hofmannsthal, Hugo von 1.\,2.\,1874 Wien – 15.\,7.\,1929 Rodaun@\textsc{Hofmannsthal, Hugo von} (1.\,2.\,1874 Wien – 15.\,7.\,1929 Rodaun), \emph{Schriftsteller}|pw} sehe ich selten, und wenn,
               dann reden wir vom Bicycle. Mein Verkehr mit Beer-Hofmann\pwindex{Beer-Hofmann, Richard 11.\,7.\,1866 Wien – 26.\,9.\,1945 New York City@\textsc{Beer-Hofmann, Richard} (11.\,7.\,1866 Wien – 26.\,9.\,1945 New York City), \emph{Schriftsteller}|pw} beschränkt sich aufs Pokerspielen und mit Schwarzkopf\pwindex{Schwarzkopf, Gustav 7.\,11.\,1853 Wien – 13.\,11.\,1939 ebd.@\textsc{Schwarzkopf, Gustav} (7.\,11.\,1853 Wien – 13.\,11.\,1939 ebd.), \emph{Schriftsteller}|pw} kann ich garnichts sprechen. G. Hirschfeld\pwindex{Hirschfeld, Georg 11.\,2.\,1873 Berlin – 17.\,1.\,1942 München@\textsc{Hirschfeld, Georg} (11.\,2.\,1873 Berlin – 17.\,1.\,1942 München), \emph{Schriftsteller}|pw} sehe ich manchmal. Er will mir sein Stück\pwindex{Hirschfeld, Georg 11.\,2.\,1873 Berlin – 17.\,1.\,1942 München@\textsc{Hirschfeld, Georg} (11.\,2.\,1873 Berlin – 17.\,1.\,1942 München), \emph{Schriftsteller}!Agnes Jordan. Schauspiel in fünf Akten@\strich\emph{Agnes Jordan. Schauspiel in fünf Akten}|pwv} vorlesen. Brahm\pwindex{Brahm, Otto 5.\,2.\,1856 Hamburg – 28.\,11.\,1912 Berlin@\textsc{Brahm, Otto} (5.\,2.\,1856 Hamburg – 28.\,11.\,1912 Berlin), \emph{Theaterleiter, Regisseur}|pw} ist hier\oindex{Wien@\textbf{Wien}, \emph{Verwaltungsgebiet}|pwv}, glaube ich, – ich habe ihn aber noch nicht gesehen.\pend
           
\pstart
           Sie \label{K_L03264-2v}\edtext{kommen ja wol bald}{\lemma{\textnormal{\emph{kommen ja wol bald}}}\Cendnote{\textnormal{An den Aufenthalt in Paris\oindex{Paris@\textbf{Paris}, \emph{Hauptstadt}|pwk} (siehe XXXX Auszeichnungsfehler: Dokument L02963 nicht gefunden) hängte Schnitzler eine
                  Reise nach London\oindex{London@\textbf{London}, \emph{Hauptstadt}|pwk}. Am 2. 6. 1897 kam er
                  wieder in Wien\oindex{Wien@\textbf{Wien}, \emph{Verwaltungsgebiet}|pwk} an.}}}\label{K_L03264-2}? Bis dahin höre ich
               doch noch öfter, wie es Ihnen geht.\pend
           
\pstart
           Herzlich Ihr {\\[\baselineskip]}\spacefill\mbox{Salten}\pend
           \leftskip=0em{}\selectlanguage{ngerman}\endnumbering\briefempfaengerindex{Schnitzler, Arthur@\textsc{Schnitzler, Arthur}!zzzSalten, Felix@\emph{von Felix Salten}!1897-05-051@{5. 5. 1897}|)be}\mylabel{L03264h}  \newcommand{\dateiname}{L03264}\newcommand{\titel}{Felix Salten an Arthur Schnitzler, 5. 5. 1897}\newcommand{\editorInnen}{Martin Anton Müller und Laura Untner}%% latex-leseansicht-abspann.tex
%% Abspann für die Leseansicht.
%% Der Schalter \ifkorrekturansicht ist bereits durch den Vorspann gesetzt.

%% latex-abspann.tex
%% Gemeinsamer Abspann für Korrekturansicht und Leseansicht.
%% Setzt den Schalter \ifkorrekturansicht voraus (gesetzt in den
%% einbindenden Dateien latex-korrekturansicht-abspann.tex bzw.
%% latex-leseansicht-abspann.tex).
%% ---------------------------------------------------------------

\normalsize

% Das esempio-Environment wird nur in der Leseansicht benötigt
\ifkorrekturansicht\else
\newenvironment{esempio}[3]%
{
    \vspace{1.5ex}
    \rlap{\underline{#1}}
    \par
    \setlength{\parindent}{0cm}
    \nopagebreak
    \leftskip=#2cm
    \rightskip=#3cm
}
{
    \par
}
\fi

\doendnotes{C}
\bigskip
\vfill

\clearpage

\footnotesize

\ifkorrekturansicht
  \lohead{\textsc{register}}
\fi

% theindex-Environment neu definieren ohne reledmac
\makeatletter
\renewenvironment{theindex}{%
  \ifkorrekturansicht
    \section*{\indexname}%
  \else
    \subsubsection*{Index der erwähnten Entitäten}%
  \fi
  \setlength{\parindent}{0pt}%
  \setlength{\parskip}{0pt plus 0.3pt}%
  \let\item\@idxitem
}{%
  \ifkorrekturansicht\clearpage\fi
}
\makeatother

\IfFileExists{\jobname-pw.ind}{\input{\jobname-pw.ind}}{}

% Quellenangabe nur in der Leseansicht
\ifkorrekturansicht\else
% Fallback-Definitionen, falls die .tex-Datei \titel etc. nicht gesetzt hat
\providecommand{\titel}{}
\providecommand{\editorInnen}{}
\providecommand{\dateiname}{\jobname}

\vspace{3cm}

\vfill

\footnotesize
\textsc{Quelle}: \titel. Herausgegeben von {\editorInnen}. In: \emph{Arthur Schnitzler: Briefwechsel mit Autorinnen und Autoren}.
 Digitale Edition, https://schnitzler-briefe.acdh.oeaw.ac.at/{\dateiname}.html (Stand \today)
\fi

\end{document}


