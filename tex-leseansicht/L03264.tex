%% latex-leseansicht-vorspann.tex
%% Vorspann für die Leseansicht.
%% Lädt die gemeinsame Datei latex-vorspann.tex mit nicht gesetztem Schalter.

\newif\ifkorrekturansicht
\korrekturansichtfalse

\input{../tex-inputs/latex-vorspann}

\begin{center}
            \textcolor{red}{ENTWURF, NICHT FERTIG KORRIGIERT}
                      \end{center}
            
         
         \renewcommand{\erwaehntePersonen}{Personen: Richard Beer-Hofmann, Otto Brahm, Julius von Gans-Ludassy, Georg Hirschfeld, Hugo von Hofmannsthal, Wilhelmine Mitterwurzer, Ottilie Salten, Gustav Schwarzkopf}
         \renewcommand{\erwaehnteOrte}{Orte: Riva del Garda, Wien}
         \renewcommand{\erwaehnteWerke}{Werke: Agnes Jordan. Schauspiel in fünf Akten}
               \section[Felix Salten an Arthur Schnitzler, 5. 5. 1897]{ Felix Salten an Arthur Schnitzler, 5. 5. 1897}\nopagebreak\mylabel{v}\rehead{ }\begin{ledgroupsized}[t]{13cm}\normalsize\beginnumbering \toendnotes[C]{\smallbreak\pagebreak[2]} \Standort{CUL, Schnitzler, B 89, A 2.}
\physDesc{Brief, 1 Blatt, 3 Seiten
\newline{}Handschrift: schwarze Tinte, lateinische Kurrent\newline{}Ordnung: mit Bleistift von unbekannter Hand nummeriert:
                                    »87« }\toendnotes[C]{\smallbreak}\pstart
           \raggedleft{}{\pb} Wien\oindex{Wien@\textbf{Wien}|pw}, am 5. Mai 97. \pend
           \pstart
           Lieber Arthur, seit ein paar Tagen bin ich wieder in Wien\oindex{Wien@\textbf{Wien}|pw}. Ich war in Riva\oindex{Riva del Garda@\textbf{Riva del Garda}|pw} – sehr schön. Aber es hätte viel schöner sein können, wenn die Mitterwurzer\pwindex{Mitterwurzer, Wilhelmine 27.03.1848 – 03.08.1909@\textsc{Mitterwurzer, Wilhelmine} (27.03.1848 – 03.08.1909), \emph{Schauspielerin}|pw} nicht dabei gewesen wäre. Hier
               lebe ich in einer merkwürdigen Sorglosigkeit. Eigentlich begreife ich es selbst
               nicht, warum ich mich so völlig unbekümmert hintreiben laße. Manchmal sage ich mir,
               dass irgend eine günstige Wendung bevorsteht, dass ich sie in allen Geradem spüre und
               dass ich deshalb so frei bin. Dabei fällt mir immer ein, was Sie mir gelegentlich
               sagten: Dass man sich bei mir immer eines Glückfalles versieht. Für Ihren Brief danke
               ich Ihnen sehr. Es war ja nicht viel, aber etwas, und ich bin jetzt - Frl. M.\pwindex{Salten, Ottilie 07.03.1868 – 22.06.1942@\textsc{Salten, Ottilie} (07.03.1868 – 22.06.1942), \emph{Schauspielerin}|pw} ausgenommen - sehr einsam. Ich arbeite
               ordentlich dabei {\pb}und nach
               allen Seiten. Es geht kein Tag hin, an dem mir nicht etwas Erfreuliches einfiele.
               Dabei bin ich jetzt an Büchern und Menschen und an den Erinnerungen vorbei auf einem
               ziemlich directen Weg zu mir selbst. Es ist eine eigenthümlich aufregende Zeit.
               Frühling kann man nicht sagen, denn es ist etwas Zweites, alles ist dezidirter,
               kühler und alle Formen sind ohne den ahnungsvollen Nebel und klarer. Es gibt keinen
               Menschen, kein Buch, nichts in meinem Leben, zu dem ich nicht eine total veränderte
               Beziehung hätte, als vorher. Das ist natürlich nicht erst in acht Tagen geworden,
               aber erst auf meiner Reise, und dann jetzt hier, habe ich ein wenig Ordnung mit
               diesen Dingen gemacht, und meine Interessen gesäubert. \pend
           \pstart
           Von äußeren Umständen weiß ich Ihnen {\pb}nichts Neues zu sagen.
               Vielleicht finde ich eine Stellung – Ludaßy\pwindex{Gans-Ludassy, Julius von 13.04.1858 – 30.09.1922@\textsc{Gans-Ludassy, Julius von} (13.04.1858 – 30.09.1922), \emph{Schriftsteller, Journalist, Herausgeber}|pw}
               behauptet, er habe große Dinge vor, – vielleicht wird etwas mit einer Direction,
               vielleicht schreibe ich von den Stoffen, mit denen ich mich jetzt beschäftige einen
               zu Ende, – das letztere ist das wahrscheinlichste. Das Bicycle und Frl.M.\pwindex{Salten, Ottilie 07.03.1868 – 22.06.1942@\textsc{Salten, Ottilie} (07.03.1868 – 22.06.1942), \emph{Schauspielerin}|pw} füllen meine übrige Zeit aus. Seit der Reise
               ist auch hier eine entscheidende Wendung eingetreten. Das macht mich auch besser und
               ruhiger und gibt meinem Leben wieder einen vollen Duft, denn ich habe lange Niemanden
               lieb gehabt. Sonst leb ich mit keinem Menschen und habe keinen, mit dem ich sprechen
               möchte. \pend
           \pstart
           Bei den übrigen ist, glaub ich, alles beim Alten, oder doch nichts wesentliches
               geschehen. Hugo\pwindex{Hofmannsthal, Hugo von 1874-02-01 – 1929-07-15@\textsc{Hofmannsthal, Hugo von} (1874-02-01 – 1929-07-15), \emph{Schriftsteller}|pw} sehe ich selten, und wenn,
               dann reden wir vom Bicycle. Mein Verkehr mit Beer-Hofmann\pwindex{Beer-Hofmann, Richard 1866-07-11 – 1945-09-26@\textsc{Beer-Hofmann, Richard} (1866-07-11 – 1945-09-26), \emph{Schriftsteller}|pw} beschränkt sich auf Pokerspielen und mit Schwarzkopf\pwindex{Schwarzkopf, Gustav 07.11.1853 – 13.11.1939@\textsc{Schwarzkopf, Gustav} (07.11.1853 – 13.11.1939), \emph{Schriftsteller}|pw} kann ich garnicht sprechen. G. Hirschfeld\pwindex{Hirschfeld, Georg 11.02.1873 – 17.01.1942@\textsc{Hirschfeld, Georg} (11.02.1873 – 17.01.1942), \emph{Schriftsteller}|pw} sehe ich manchmal. Er will mir sein Stück\pwindex{Hirschfeld, Georg 11.02.1873 – 17.01.1942@\textsc{Hirschfeld, Georg} (11.02.1873 – 17.01.1942), \emph{Schriftsteller}!Agnes Jordan. Schauspiel in fuenf Akten1897@\strich\emph{Agnes Jordan. Schauspiel in fünf Akten} {[}1897{]}|pwv} vorlesen. Brahm\pwindex{Brahm, Otto 05.02.1856 – 28.11.1912@\textsc{Brahm, Otto} (05.02.1856 – 28.11.1912), \emph{Theaterleiter, Regisseur}|pw} ist hier, glaube ich, – ich habe ihn aber noch nicht
               gesehen. Sie kommen ja wol bald? Bis dahin höre ich doch noch öfter, wie es Ihnen
               geht. \pend
           \pstart
           Herzlich Ihr {\\[\baselineskip]}\spacefill\mbox{Salten}\pend
           \leftskip=0em{}
         
         \endnumbering\mylabel{h}\end{ledgroupsized}\begin{anhang}\end{anhang}\newcommand{\dateiname}{L03264}\newcommand{\titel}{Felix Salten an Arthur Schnitzler, 5. 5. 1897}\newcommand{\editorInnen}{Martin Anton Müller und Laura Untner}%% latex-leseansicht-abspann.tex
%% Abspann für die Leseansicht.
%% Der Schalter \ifkorrekturansicht ist bereits durch den Vorspann gesetzt.

%% latex-abspann.tex
%% Gemeinsamer Abspann für Korrekturansicht und Leseansicht.
%% Setzt den Schalter \ifkorrekturansicht voraus (gesetzt in den
%% einbindenden Dateien latex-korrekturansicht-abspann.tex bzw.
%% latex-leseansicht-abspann.tex).
%% ---------------------------------------------------------------

\normalsize

% Das esempio-Environment wird nur in der Leseansicht benötigt
\ifkorrekturansicht\else
\newenvironment{esempio}[3]%
{
    \vspace{1.5ex}
    \rlap{\underline{#1}}
    \par
    \setlength{\parindent}{0cm}
    \nopagebreak
    \leftskip=#2cm
    \rightskip=#3cm
}
{
    \par
}
\fi

\doendnotes{C}
\bigskip
\vfill

\clearpage

\footnotesize

\ifkorrekturansicht
  \lohead{\textsc{register}}
\fi

% theindex-Environment neu definieren ohne reledmac
\makeatletter
\renewenvironment{theindex}{%
  \ifkorrekturansicht
    \section*{\indexname}%
  \else
    \subsubsection*{Index der erwähnten Entitäten}%
  \fi
  \setlength{\parindent}{0pt}%
  \setlength{\parskip}{0pt plus 0.3pt}%
  \let\item\@idxitem
}{%
  \ifkorrekturansicht\clearpage\fi
}
\makeatother

\IfFileExists{\jobname-pw.ind}{\input{\jobname-pw.ind}}{}

% Quellenangabe nur in der Leseansicht
\ifkorrekturansicht\else
% Fallback-Definitionen, falls die .tex-Datei \titel etc. nicht gesetzt hat
\providecommand{\titel}{}
\providecommand{\editorInnen}{}
\providecommand{\dateiname}{\jobname}

\vspace{3cm}

\vfill

\footnotesize
\textsc{Quelle}: \titel. Herausgegeben von {\editorInnen}. In: \emph{Arthur Schnitzler: Briefwechsel mit Autorinnen und Autoren}.
 Digitale Edition, https://schnitzler-briefe.acdh.oeaw.ac.at/{\dateiname}.html (Stand \today)
\fi

\end{document}


      