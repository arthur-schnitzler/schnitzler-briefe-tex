%% latex-leseansicht-vorspann.tex
%% Vorspann für die Leseansicht.
%% Lädt die gemeinsame Datei latex-vorspann.tex mit nicht gesetztem Schalter.

\newif\ifkorrekturansicht
\korrekturansichtfalse

\input{../tex-inputs/latex-vorspann}

\begin{center}
            \textcolor{red}{ENTWURF, NICHT FERTIG KORRIGIERT}
                      \end{center}
            
         
         \renewcommand{\erwaehntePersonen}{Personen: Herbert Ginsberg, Ottilie Salten, Olga Schnitzler}
         \renewcommand{\erwaehnteInstitutionen}{Institutionen: B.Z. am Mittag}
         \renewcommand{\erwaehnteOrte}{Orte: Berlin, Griechenland, Kairo, Kochstraße, Wien}
         \renewcommand{\erwaehnteWerke}{}
               \section[Felix Salten an Arthur Schnitzler, 8. 4. 1906]{ Felix Salten an Arthur Schnitzler, 8. 4. 1906}\nopagebreak\mylabel{v}\rehead{ }\begin{ledgroupsized}[t]{13cm}\normalsize\beginnumbering \toendnotes[C]{\smallbreak\pagebreak[2]} \Standort{CUL, Schnitzler, B 89, B 1.}
\physDesc{Brief, 1 Blatt, 1 Seite
\newline{}Handschrift: schwarze Tinte, lateinische Kurrent\newline{}Ordnung: mit Bleistift von unbekannter Hand nummeriert:
                                    »208« }\toendnotes[C]{\smallbreak}\pstart
           \noindent{}{\pb}\textcolor{gray}{\textbf{B. Z. am Mittag}}\orgindex{B.Z. am Mittag@B.Z. am Mittag|pw}\hfill \textcolor{gray}{\textbf{BERLIN SW\oindex{Berlin@\textbf{Berlin}|pw},}}{ }8. IV. 06\pend
           \pstart
           \textcolor{gray}{\textbf{Chefredaktion}}\hfill \textcolor{gray}{\textbf{Kochstr. 23–25}}\oindex{Kochstrasse@\textbf{Kochstraße}|pw}\pend
           \pstart
           Lieber, erlauben Sie, dass ich Ihnen Herrn D\textsuperscript{r}  Herbert Ginsberg\pwindex{Ginsberg, Herbert 1881-09-27 – 1962-11-05@\textsc{Ginsberg, Herbert} (1881-09-27 – 1962-11-05), \emph{Industrieller, Kunstsammler, Bankier}|pw} vorstelle, den
               ich gerne bei Ihnen einführen möchte. Er kommt – studienhalber – für ein paar Monate
               nach Wien\oindex{Wien@\textbf{Wien}|pw}. Wenn Sie ihn freundlich aufnehmen
               wollen, werden Sie mich sehr verbinden und – gewiss – die lebhafte Sympathie, die ich
               für ihn habe, sehr bald teilen. Eine nähere Personalbeschreibung kann ich mir wol
               sparen. Aber unter manchen anderen Anknüpfungspunkten ist vielleicht der zu erwähnen,
               dass Herr D\textsuperscript{r} Ginsberg\pwindex{Ginsberg, Herbert 1881-09-27 – 1962-11-05@\textsc{Ginsberg, Herbert} (1881-09-27 – 1962-11-05), \emph{Industrieller, Kunstsammler, Bankier}|pw} viel gereist ist, (ich lernte ihn bei meinem Ausflug nach \label{K_L03417-1v}\edtext{Kairo\oindex{Kairo@\textbf{Kairo}|pw}}{\lemma{\textnormal{\emph{Kairo}}}\Cendnote{\textnormal{vgl. Felix Salten an Arthur Schnitzler, 8. 3. 1904. Das Journal »Reisen
                  der Jahre 1893« von Ginsberg\pwindex{Ginsberg, Herbert 1881-09-27 – 1962-11-05@\textsc{Ginsberg, Herbert} (1881-09-27 – 1962-11-05), \emph{Industrieller, Kunstsammler, Bankier}|pwk} ist online
                  einzusehen. Darin finden sich sowohl für den Aufenthalt in Kairo\oindex{Kairo@\textbf{Kairo}|pwk} wie auch für die beiden Begegnungen mit Schnitzler\pwindex{Schnitzler, Arthur 15.05.1862 – 21.10.1931@\textsc{Schnitzler, Arthur} (15.05.1862 – 21.10.1931), \emph{Schriftsteller, Mediziner}|pwk} (13. 4. 1906, S. 98 und 12. 6. 1906,
                  S. 112),
                     https://archive.org/details/gilbertfamily01reel05/page/n443.}}}\label{K_L03417-1h}
               kennen) und Ihnen gewiss über einige Gegenden, die Sie interessiren, z. B. Griechenland\oindex{Griechenland@\textbf{Griechenland}|pw}, interessante Aufschlüße zu geben
               weiß. \pend
           \pstart
           Herzlichste Grüße von Otti\pwindex{Salten, Ottilie 07.03.1868 – 22.06.1942@\textsc{Salten, Ottilie} (07.03.1868 – 22.06.1942), \emph{Schauspielerin}|pw} und mir an Sie Beide\pwindex{Schnitzler, Olga 17.01.1882 – 13.01.1970@\textsc{Schnitzler, Olga} (17.01.1882 – 13.01.1970), \emph{Schauspielerin, Sängerin}|pwv}.\pend
           \pstart Ihr \spacefill\mbox{Salten}\pend{}
         
         \endnumbering\mylabel{h}\end{ledgroupsized}\begin{anhang}\end{anhang}\newcommand{\dateiname}{L03417}\newcommand{\titel}{Felix Salten an Arthur Schnitzler, 8. 4. 1906}\newcommand{\editorInnen}{Martin Anton Müller und Laura Untner}%% latex-leseansicht-abspann.tex
%% Abspann für die Leseansicht.
%% Der Schalter \ifkorrekturansicht ist bereits durch den Vorspann gesetzt.

%% latex-abspann.tex
%% Gemeinsamer Abspann für Korrekturansicht und Leseansicht.
%% Setzt den Schalter \ifkorrekturansicht voraus (gesetzt in den
%% einbindenden Dateien latex-korrekturansicht-abspann.tex bzw.
%% latex-leseansicht-abspann.tex).
%% ---------------------------------------------------------------

\normalsize

% Das esempio-Environment wird nur in der Leseansicht benötigt
\ifkorrekturansicht\else
\newenvironment{esempio}[3]%
{
    \vspace{1.5ex}
    \rlap{\underline{#1}}
    \par
    \setlength{\parindent}{0cm}
    \nopagebreak
    \leftskip=#2cm
    \rightskip=#3cm
}
{
    \par
}
\fi

\doendnotes{C}
\bigskip
\vfill

\clearpage

\footnotesize

\ifkorrekturansicht
  \lohead{\textsc{register}}
\fi

% theindex-Environment neu definieren ohne reledmac
\makeatletter
\renewenvironment{theindex}{%
  \ifkorrekturansicht
    \section*{\indexname}%
  \else
    \subsubsection*{Index der erwähnten Entitäten}%
  \fi
  \setlength{\parindent}{0pt}%
  \setlength{\parskip}{0pt plus 0.3pt}%
  \let\item\@idxitem
}{%
  \ifkorrekturansicht\clearpage\fi
}
\makeatother

\IfFileExists{\jobname-pw.ind}{\input{\jobname-pw.ind}}{}

% Quellenangabe nur in der Leseansicht
\ifkorrekturansicht\else
% Fallback-Definitionen, falls die .tex-Datei \titel etc. nicht gesetzt hat
\providecommand{\titel}{}
\providecommand{\editorInnen}{}
\providecommand{\dateiname}{\jobname}

\vspace{3cm}

\vfill

\footnotesize
\textsc{Quelle}: \titel. Herausgegeben von {\editorInnen}. In: \emph{Arthur Schnitzler: Briefwechsel mit Autorinnen und Autoren}.
 Digitale Edition, https://schnitzler-briefe.acdh.oeaw.ac.at/{\dateiname}.html (Stand \today)
\fi

\end{document}


      