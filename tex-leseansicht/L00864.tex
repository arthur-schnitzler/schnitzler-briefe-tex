%% latex-leseansicht-vorspann.tex
%% Vorspann für die Leseansicht.
%% Lädt die gemeinsame Datei latex-vorspann.tex mit nicht gesetztem Schalter.

\newif\ifkorrekturansicht
\korrekturansichtfalse

\input{../tex-inputs/latex-vorspann}


         
         \renewcommand{\erwaehntePersonen}{Personen: Hermann Bahr}
         \renewcommand{\erwaehnteInstitutionen}{Institutionen: Berliner Tageblatt, Cosmopolis, Freie Bühne}
         \renewcommand{\erwaehnteOrte}{Orte: Berlin, Burgtheater, Volkstheater, Wien}
         \renewcommand{\erwaehnteWerke}{Werke: Das Vermächtnis. Schauspiel in drei Akten, Der Star. Ein Wiener Stück in vier Akten, Der grüne Kakadu – Paracelsus – Die Gefährtin. Drei Einakter, Der grüne Kakadu. Groteske in einem Akt, Die Gefährtin. Schauspiel in einem Akt, Neue Deutsche Rundschau, Paracelsus. Versspiel in einem Akt}
               \section[Arthur Schnitzler an Hermann Bahr, 1. 12. 1898]{ Arthur Schnitzler an Hermann Bahr, 1. 12. 1898}\nopagebreak\mylabel{v}\rehead{ }\begin{ledgroupsized}[t]{13cm}\normalsize\beginnumbering \toendnotes[C]{\smallbreak\pagebreak[2]} \Standort{TMW, HS AM 60159 Ba.}
\physDesc{Briefkarte
\newline{}Handschrift: schwarze Tinte, deutsche Kurrent\newline{}Ordnung: Lochung }\buchAbdrucke{\weitereDrucke{1) \emph{9. 12. 1898, Abschrift.} In: Arthur Schnitzler: \emph{The Letters of Arthur Schnitzler to Hermann Bahr}. Edited, annotated, and with an introduction, by Donald G.
                        Daviau. Chapel Hill: \emph{The University of North Carolina Press} 1978, S. 64 (University of North Carolina studies in the Germanic languages
                        and literatures, 89).} \weitereDrucke{2) Hermann Bahr, Arthur Schnitzler: \emph{Briefwechsel, Aufzeichnungen, Dokumente (1891–1931)}. Hg. Kurt Ifkovits und Martin Anton Müller. Göttingen: \emph{Wallstein} 2018, S. 165.} }\toendnotes[C]{\smallbreak}\pstart
           \noindent{}{\pb}Lieber Hermann, ich danke dir herzlich für deine freundlichen Glückw\damage{ün}ſche\pwindex{Schnitzler, Arthur 15.05.1862 – 21.10.1931@\textsc{Schnitzler, Arthur} (15.05.1862 – 21.10.1931), \emph{Schriftsteller, Mediziner}!Vermaechtnis. Schauspiel in drei Akten1898@\strich\emph{Das Vermächtnis. Schauspiel in drei Akten} {[}1898{]}|pwv}. Den »Kakadu\pwindex{Schnitzler, Arthur 15.05.1862 – 21.10.1931@\textsc{Schnitzler, Arthur} (15.05.1862 – 21.10.1931), \emph{Schriftsteller, Mediziner}!gruene Kakadu. Groteske in einem Akt1. 3. 1899@\strich\emph{Der grüne Kakadu. Groteske in einem Akt} {[}1. 3. 1899{]}|pw}« hat die F\damage{rei}e Bühne\pwindex{Neue Deutsche Rundschau1894-01-01 – 1903-12-31@\emph{Neue Deutsche Rundschau} {[}1894-01-01 – 1903-12-31{]}|pw}{ }ſchon (»Die \textsc{Neue Deutsche Rundschau}\pwindex{Neue Deutsche Rundschau1894-01-01 – 1903-12-31@\emph{Neue Deutsche Rundschau} {[}1894-01-01 – 1903-12-31{]}|pw}« mein’ ich); er ſoll, während der Recurs wegen der \label{K_L00864_1v}\edtext{Freigabe}{\lemma{\textnormal{\emph{Freigabe}}}\Cendnote{\textnormal{Nachdem
                  das Stück am Burgtheater\oindex{Burgtheater@\textbf{Burgtheater}|pwk} am
                     1. 3. 1899 zum ersten Mal gegeben worden war, wurde es in der Wien\oindex{Wien@\textbf{Wien}|pwk}er Einrichtung (Umbenennung einer Figur,
                  Kürzung von Freiheitsrufen) in Berlin\oindex{Berlin@\textbf{Berlin}|pwk} erneut der
                  Zensur eingereicht und diese »hat soeben das Stück in dieser Form zur
                     Aufführung freigegeben« (\emph{Berliner Tageblatt}\orgindex{Berliner Tageblatt@Berliner Tageblatt|pwk}, Jg. 28, Nr. 136,
                        15. 3. 1899, Morgen-Ausgabe, S. 3).}}}\label{K_L00864_1h} im Gang iſt,
               an der »Freien Bühne\orgindex{Freie Buehne@Freie Bühne|pw}« in Berlin\oindex{Berlin@\textbf{Berlin}|pw} aufgeführt werden. Jedenfalls iſt nun mein ganzer Einakter Abend\pwindex{Schnitzler, Arthur 15.05.1862 – 21.10.1931@\textsc{Schnitzler, Arthur} (15.05.1862 – 21.10.1931), \emph{Schriftsteller, Mediziner}!gruene Kakadu – Paracelsus – Die Gefaehrtin. Drei Einakter1898 – 1899@\strich\emph{Der grüne Kakadu – Paracelsus – Die Gefährtin. Drei Einakter} {[}1898 – 1899{]}|pwv} hinausgeſchoben. So iſt es
               vorläufig noch verfrüht, dir von der »Gefährtin\pwindex{Schnitzler, Arthur 15.05.1862 – 21.10.1931@\textsc{Schnitzler, Arthur} (15.05.1862 – 21.10.1931), \emph{Schriftsteller, Mediziner}!Gefaehrtin. Schauspiel in einem Akt1899-03-01@\strich\emph{Die Gefährtin. Schauspiel in einem Akt} {[}1899-03-01{]}|pw}«,
               einem dieſer Einakter, zu reden, den ich {\pb}keineswegs \label{LL023-1v}\uline{vor}\label{LL023-1h} der Aufführg erſcheinen laſſen möchte, den ich aber bi\damage{sh}er noch nicht vergeben habe. – Du \damage{hof}fſt meine \textsc{Kosmopolis}-Honorarforderungen\orgindex{Cosmopolis@Cosmopolis|pw} durchzuſetzen – das
               wäre ſehr ſchön – denn die \textsc{Kosmopolis}\orgindex{Cosmopolis@Cosmopolis|pw} iſt \label{K_L00864_2v}\edtext{verkracht und ſchuldet mir
               ungezählte Mark}{\lemma{\textnormal{\emph{verkracht … Mark}}}\Cendnote{\textnormal{\emph{Cosmopolis}\orgindex{Cosmopolis@Cosmopolis|pwk} erschien mehrsprachig und monatlich,
                  zum ersten Mal im Januar 1896, zum letzten Mal im November
                     1898. Zum finalen Heft hat Schnitzler\pwindex{Schnitzler, Arthur 15.05.1862 – 21.10.1931@\textsc{Schnitzler, Arthur} (15.05.1862 – 21.10.1931), \emph{Schriftsteller, Mediziner}|pwk}{ }\emph{Paracelsus}\pwindex{Schnitzler, Arthur 15.05.1862 – 21.10.1931@\textsc{Schnitzler, Arthur} (15.05.1862 – 21.10.1931), \emph{Schriftsteller, Mediziner}!Paracelsus. Versspiel in einem Akt01. 11. 1898@\strich\emph{Paracelsus. Versspiel in einem Akt} {[}01. 11. 1898{]}|pwk} (Bd. 12, H. 35,
                     S. 489–527) beigesteuert.}}}\label{K_L00864_2h}. Also verſuch’s\substVorne{}\textsuperscript{, –}\substDazwischen{}.\substHinten{}\pend
           \pstart
           – Auf baldige \label{K_L00864_3v}\edtext{Gratulationsrevanche}{\lemma{\textnormal{\emph{Gratulationsrevanche}}}\Cendnote{\textnormal{Premiere der
                  ersten Wiener\oindex{Wien@\textbf{Wien}|pwk} Inszenierung von \emph{Der Star}\pwindex{Bahr, Hermann 19.07.1863 – 15.01.1934@\textsc{Bahr, Hermann} (19.07.1863 – 15.01.1934), \emph{Schriftsteller, Kritiker}!Star. Ein Wiener Stueck in vier Akten10. 12. 1898@\strich\emph{Der Star. Ein Wiener Stück in vier Akten} {[}10. 12. 1898{]}|pwk} am 10. 12. 1897}}}\label{K_L00864_3h} im Volkstheater\oindex{Volkstheater@\textbf{Volkstheater}|pw}.\pend
           \pstart Herzlichen Gruſs. Dein \spacefill\mbox{Arthur Sch.}\pend{}\pstart
           Wien\oindex{Wien@\textbf{Wien}|pw}{ }1. 12. 98\pend
           
         
         \endnumbering\mylabel{h}\end{ledgroupsized}  \newcommand{\dateiname}{L00864}\newcommand{\titel}{Arthur Schnitzler an Hermann Bahr, 1. 12. 1898}\newcommand{\editorInnen}{ Kurt Ifkovits,  Martin Anton Müller}%% latex-leseansicht-abspann.tex
%% Abspann für die Leseansicht.
%% Der Schalter \ifkorrekturansicht ist bereits durch den Vorspann gesetzt.

%% latex-abspann.tex
%% Gemeinsamer Abspann für Korrekturansicht und Leseansicht.
%% Setzt den Schalter \ifkorrekturansicht voraus (gesetzt in den
%% einbindenden Dateien latex-korrekturansicht-abspann.tex bzw.
%% latex-leseansicht-abspann.tex).
%% ---------------------------------------------------------------

\normalsize

% Das esempio-Environment wird nur in der Leseansicht benötigt
\ifkorrekturansicht\else
\newenvironment{esempio}[3]%
{
    \vspace{1.5ex}
    \rlap{\underline{#1}}
    \par
    \setlength{\parindent}{0cm}
    \nopagebreak
    \leftskip=#2cm
    \rightskip=#3cm
}
{
    \par
}
\fi

\doendnotes{C}
\bigskip
\vfill

\clearpage

\footnotesize

\ifkorrekturansicht
  \lohead{\textsc{register}}
\fi

% theindex-Environment neu definieren ohne reledmac
\makeatletter
\renewenvironment{theindex}{%
  \ifkorrekturansicht
    \section*{\indexname}%
  \else
    \subsubsection*{Index der erwähnten Entitäten}%
  \fi
  \setlength{\parindent}{0pt}%
  \setlength{\parskip}{0pt plus 0.3pt}%
  \let\item\@idxitem
}{%
  \ifkorrekturansicht\clearpage\fi
}
\makeatother

\IfFileExists{\jobname-pw.ind}{\input{\jobname-pw.ind}}{}

% Quellenangabe nur in der Leseansicht
\ifkorrekturansicht\else
% Fallback-Definitionen, falls die .tex-Datei \titel etc. nicht gesetzt hat
\providecommand{\titel}{}
\providecommand{\editorInnen}{}
\providecommand{\dateiname}{\jobname}

\vspace{3cm}

\vfill

\footnotesize
\textsc{Quelle}: \titel. Herausgegeben von {\editorInnen}. In: \emph{Arthur Schnitzler: Briefwechsel mit Autorinnen und Autoren}.
 Digitale Edition, https://schnitzler-briefe.acdh.oeaw.ac.at/{\dateiname}.html (Stand \today)
\fi

\end{document}


      