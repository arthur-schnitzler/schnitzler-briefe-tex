%% latex-korrekturansicht-vorspann.tex
%% Vorspann für die Korrekturansicht.
%% Lädt die gemeinsame Datei latex-vorspann.tex mit gesetztem Schalter.

\newif\ifkorrekturansicht
\korrekturansichttrue

\input{../tex-inputs/latex-vorspann}


\section[Arthur Schnitzler an Hermann Bahr, 1. 12. 1898]{L00864 Arthur Schnitzler an Hermann Bahr, 1. 12. 1898}
\nopagebreak\mylabel{L00864v}
\rehead{ }\normalsize\beginnumbering\briefempfaengerindex{Bahr, Hermann@\textsc{Bahr, Hermann}!zzzSchnitzler, Arthur@\emph{von Arthur Schnitzler}!1898-12-012@{1. 12. 1898}|(be}
\toendnotes[C]{\smallbreak\pagebreak[2]}\Standort{TMW, HS AM 60159 Ba.}
\physDesc{Briefkarte, 781 Zeichen
\newline{}Handschrift: schwarze Tinte, deutsche Kurrent
\newline{}Ordnung: Lochung }
\buchAbdrucke{\weitereDrucke{1) Arthur Schnitzler: \emph{The Letters of Arthur Schnitzler to Hermann Bahr}. Chapel Hill: \emph{The University of North Carolina Press} 1978, S. 64.} \weitereDrucke{2) Hermann Bahr, Arthur Schnitzler: \emph{Briefwechsel, Aufzeichnungen, Dokumente (1891–1931)}. Göttingen: \emph{Wallstein} 2018, S. 165.} }\toendnotes[C]{\smallbreak}
\pstart
           \noindent{}{\pb}Lieber Hermann, ich danke dir herzlich für deine freundlichen Glückw\damage{ün}ſche\pwindex{Vermaechtnis. Schauspiel in drei Akten@\emph{Das Vermächtnis. Schauspiel in drei Akten}|pwv}. Den »Kakadu\pwindex{gruene Kakadu. Groteske in einem Akt@\emph{Der grüne Kakadu. Groteske in einem Akt}|pw}« hat die F\damage{rei}e Bühne\pwindex{Neue Deutsche Rundschau@\emph{Neue Deutsche Rundschau}|pw}{ }ſchon (»Die \textsc{Neue Deutsche Rundschau}\pwindex{Neue Deutsche Rundschau@\emph{Neue Deutsche Rundschau}|pw}« mein’ ich); er ſoll, während der Recurs wegen der \label{K_L00864-1v}\edtext{Freigabe}{\lemma{\textnormal{\emph{Freigabe}}}\Cendnote{\textnormal{Nachdem
                  das Stück am \emph{Burgtheater}\orgindex{Burgtheater@Burgtheater|pwk} am
                     1. 3. 1899 zum ersten Mal gegeben worden war, wurde es in der Wien\oindex{Wien@\textbf{Wien}, \emph{A.ADM2}|pwk}er Einrichtung (Umbenennung einer Figur,
                  Kürzung von Freiheitsrufen) in Berlin\oindex{Berlin@\textbf{Berlin}, \emph{P.PPLC}|pwk} erneut
                  der Zensur eingereicht und diese »hat soeben das Stück in dieser Form zur
                     Aufführung freigegeben« (\emph{Berliner Tageblatt}\orgindex{Berliner Tageblatt@Berliner Tageblatt|pwk}, Jg. 28, Nr. 136,
                        15. 3. 1899, Morgen-Ausgabe, S. 3).}}}\label{K_L00864-1} im Gang iſt,
               an der »Freien Bühne\orgindex{Freie Buehne@Freie Bühne|pw}« in Berlin\oindex{Berlin@\textbf{Berlin}, \emph{P.PPLC}|pw} aufgeführt werden. Jedenfalls iſt nun mein ganzer Einakter Abend\pwindex{gruene Kakadu – Paracelsus – Die Gefaehrtin. Drei Einakter@\emph{Der grüne Kakadu – Paracelsus – Die Gefährtin. Drei Einakter}|pwv} hinausgeſchoben.
               So iſt es vorläufig noch verfrüht, dir von der »Gefährtin\pwindex{Gefaehrtin. Schauspiel in einem Akt@\emph{Die Gefährtin. Schauspiel in einem Akt}|pw}«, einem dieſer Einakter, zu reden, den ich {\pb}keineswegs \label{LL023-1v}\uline{vor}\label{LL023-1h} der Aufführg erſcheinen laſſen möchte, den ich aber bi\damage{sh}er noch nicht vergeben habe. – Du \damage{hof}fſt meine \textsc{Kosmopolis}-Honorarforderungen\orgindex{Cosmopolis@Cosmopolis|pw} durchzuſetzen – das
               wäre ſehr ſchön – denn die \textsc{Kosmopolis}\orgindex{Cosmopolis@Cosmopolis|pw} iſt \label{K_L00864-2v}\edtext{verkracht und ſchuldet mir
               ungezählte Mark}{\lemma{\textnormal{\emph{verkracht … Mark}}}\Cendnote{\textnormal{\emph{Cosmopolis}\orgindex{Cosmopolis@Cosmopolis|pwk} erschien mehrsprachig und
                  monatlich, zum ersten Mal im Januar 1896, zum letzten Mal im
                     November 1898. Zum finalen Heft hat Schnitzler{ }\emph{Paracelsus}\pwindex{Paracelsus. Versspiel in einem Akt@\emph{Paracelsus. Versspiel in einem Akt}|pwk} (Bd. 12, H. 35,
                     S. 489–527) beigesteuert.}}}\label{K_L00864-2}. Also verſuch’s\substVorne{}\textsuperscript{, –}\substDazwischen{}.\substHinten{}\pend
           
\pstart
           – Auf baldige \label{K_L00864-3v}\edtext{Gratulationsrevanche}{\lemma{\textnormal{\emph{Gratulationsrevanche}}}\Cendnote{\textnormal{Premiere der
                  ersten Wiener\oindex{Wien@\textbf{Wien}, \emph{A.ADM2}|pwk} Inszenierung von \emph{Der Star}\pwindex{Star. Ein Wiener Stueck in vier Akten@\emph{Der Star. Ein Wiener Stück in vier Akten}|pwk} am 10. 12. 1897}}}\label{K_L00864-3} im Volkstheater\oindex{Volkstheater@\textbf{Volkstheater}, \emph{Theater (K.THE)}|pw}.\pend
           \pstart Herzlichen Gruſs. Dein \spacefill\mbox{Arthur Sch.}\pend{}
\pstart
           Wien\oindex{Wien@\textbf{Wien}, \emph{A.ADM2}|pw}{ }1. 12. 98\pend
           \selectlanguage{ngerman}\endnumbering\briefempfaengerindex{Bahr, Hermann@\textsc{Bahr, Hermann}!zzzSchnitzler, Arthur@\emph{von Arthur Schnitzler}!1898-12-012@{1. 12. 1898}|)be}\mylabel{L00864h}  \normalsize

\doendnotes{C}
\bigskip
\vfill

\clearpage

\footnotesize

\lohead{\textsc{register}}

% Definiere theindex-Environment komplett neu ohne reledmac
\makeatletter
\renewenvironment{theindex}{%
  \section*{\indexname}%
  \setlength{\parindent}{0pt}%
  \setlength{\parskip}{0pt plus 0.3pt}%
  \let\item\@idxitem
}{%
  \clearpage
}
\makeatother

\IfFileExists{\jobname-pw.ind}{\input{\jobname-pw.ind}}{}

\end{document}

      