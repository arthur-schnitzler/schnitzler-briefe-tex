%% latex-leseansicht-vorspann.tex
%% Vorspann für die Leseansicht.
%% Lädt die gemeinsame Datei latex-vorspann.tex mit nicht gesetztem Schalter.

\newif\ifkorrekturansicht
\korrekturansichtfalse

\input{../tex-inputs/latex-vorspann}


         
         \renewcommand{\erwaehntePersonen}{Personen: Hermann Bahr, Alois Bahr, Wilhelmine Bahr}
         \renewcommand{\erwaehnteOrte}{Orte: Edmund-Weiß-Gasse 7, Salzburg, Wien, XIII., Hietzing, XVIII., Währing}
         \renewcommand{\erwaehnteWerke}{Werke: Reigen. Zehn Dialoge}
               \section[Hermann Bahr an Arthur Schnitzler, 13. 11. 1903]{ Hermann Bahr an Arthur Schnitzler, 13. 11. 1903}\nopagebreak\mylabel{v}\rehead{ }\begin{ledgroupsized}[t]{13cm}\normalsize\beginnumbering \toendnotes[C]{\smallbreak\pagebreak[2]} \Standort{CUL, Schnitzler, B 5b.}
\physDesc{Postkarte, 278 Zeichen
\newline{}Handschrift: schwarze Tinte, deutsche Kurrent
\newline{}Versand: 1) Stempel: »\nobreak{}\oindex{XIII., Hietzing@\textbf{XIII., Hietzing}|pwk}Wien 13/7, 13. 11. 03, 2–3N\nobreak{}«.   2) Stempel: »\nobreak{}\oindex{XVIII., Waehring@\textbf{XVIII., Währing}|pwk}18/1 Wien, 13. 11. 03, 7.N, Bestellt\nobreak{}«. 
\newline{}Ordnung: mit Bleistift von unbekannter Hand nummeriert:
                                    »103« }\buchAbdrucke{\weitereDrucke{Hermann Bahr, Arthur Schnitzler: \emph{Briefwechsel, Aufzeichnungen, Dokumente (1891–1931)}. Hg. Kurt Ifkovits und Martin Anton Müller. Göttingen: \emph{Wallstein} 2018, S. 281.} }\toendnotes[C]{\smallbreak}\pstart{}{\pb}\textsc{Herrn D\textsuperscript{r} Arthur Schnitzler}\pend{}\pstart{}\textsc{Wien XVIII}\oindex{XVIII., Waehring@\textbf{XVIII., Währing}|pw}\pend{}\pstart{}Spöttelgaſſe 7\oindex{Edmund-Weiss-Gasse 7@\textbf{Edmund-Weiß-Gasse 7}|pw}\pend{}{\bigskip}\pstart
           \raggedleft{}{\pb}13. 11. 03\pend
           \pstart{}Lieber Arthur!\pend\pstart
           Danke ſehr. Ich freue mich ſehr, wenn Du wieder einmal heraus ko{\geminationm}ſt – nur bitte: dieſen Sonntag und
                  Montag nicht, weil ich \label{K_L01342-1v}\edtext{nicht hier}{\lemma{\textnormal{\emph{nicht hier}}}\Cendnote{\textnormal{Am Sonntag,
                     15. 11., besuchte er in Salzburg\oindex{Salzburg@\textbf{Salzburg}|pwk} das
                  Grab seiner Eltern\pwindex{Bahr, Alois 11.04.1834 – 05.09.1898@\textsc{Bahr, Alois} (11.04.1834 – 05.09.1898), \emph{Notar, Politiker}|pwkv}\pwindex{Bahr, Wilhelmine 06.06.1835 – 16.05.1902@\textsc{Bahr, Wilhelmine} (06.06.1835 – 16.05.1902)|pwkv}.}}}\label{K_L01342-1h} bin. Und bitte: ſchick mir den Rekurs\pwindex{Schnitzler, Arthur 15.05.1862 – 21.10.1931@\textsc{Schnitzler, Arthur} (15.05.1862 – 21.10.1931), \emph{Schriftsteller, Mediziner}!Reigen. Zehn Dialoge1900@\strich\emph{Reigen. Zehn Dialoge} {[}1900{]}|pwv} gelegentlich zurück.\pend
           \pstart
           Herzlichſt{\\[\baselineskip]}Dein{\\[\baselineskip]}\spacefill\mbox{Hermann}\pend
           \leftskip=0em{}
         
         \endnumbering\mylabel{h}\end{ledgroupsized}  \newcommand{\dateiname}{L01342}\newcommand{\titel}{Hermann Bahr an Arthur Schnitzler, 13. 11. 1903}\newcommand{\editorInnen}{ Kurt Ifkovits,  Martin Anton Müller}%% latex-leseansicht-abspann.tex
%% Abspann für die Leseansicht.
%% Der Schalter \ifkorrekturansicht ist bereits durch den Vorspann gesetzt.

%% latex-abspann.tex
%% Gemeinsamer Abspann für Korrekturansicht und Leseansicht.
%% Setzt den Schalter \ifkorrekturansicht voraus (gesetzt in den
%% einbindenden Dateien latex-korrekturansicht-abspann.tex bzw.
%% latex-leseansicht-abspann.tex).
%% ---------------------------------------------------------------

\normalsize

% Das esempio-Environment wird nur in der Leseansicht benötigt
\ifkorrekturansicht\else
\newenvironment{esempio}[3]%
{
    \vspace{1.5ex}
    \rlap{\underline{#1}}
    \par
    \setlength{\parindent}{0cm}
    \nopagebreak
    \leftskip=#2cm
    \rightskip=#3cm
}
{
    \par
}
\fi

\doendnotes{C}
\bigskip
\vfill

\clearpage

\footnotesize

\ifkorrekturansicht
  \lohead{\textsc{register}}
\fi

% theindex-Environment neu definieren ohne reledmac
\makeatletter
\renewenvironment{theindex}{%
  \ifkorrekturansicht
    \section*{\indexname}%
  \else
    \subsubsection*{Index der erwähnten Entitäten}%
  \fi
  \setlength{\parindent}{0pt}%
  \setlength{\parskip}{0pt plus 0.3pt}%
  \let\item\@idxitem
}{%
  \ifkorrekturansicht\clearpage\fi
}
\makeatother

\IfFileExists{\jobname-pw.ind}{\input{\jobname-pw.ind}}{}

% Quellenangabe nur in der Leseansicht
\ifkorrekturansicht\else
% Fallback-Definitionen, falls die .tex-Datei \titel etc. nicht gesetzt hat
\providecommand{\titel}{}
\providecommand{\editorInnen}{}
\providecommand{\dateiname}{\jobname}

\vspace{3cm}

\vfill

\footnotesize
\textsc{Quelle}: \titel. Herausgegeben von {\editorInnen}. In: \emph{Arthur Schnitzler: Briefwechsel mit Autorinnen und Autoren}.
 Digitale Edition, https://schnitzler-briefe.acdh.oeaw.ac.at/{\dateiname}.html (Stand \today)
\fi

\end{document}


      