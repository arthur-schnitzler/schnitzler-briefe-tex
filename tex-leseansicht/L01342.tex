%% latex-korrekturansicht-vorspann.tex
%% Vorspann für die Korrekturansicht.
%% Lädt die gemeinsame Datei latex-vorspann.tex mit gesetztem Schalter.

\newif\ifkorrekturansicht
\korrekturansichttrue

\input{../tex-inputs/latex-vorspann}


\section[Hermann Bahr an Arthur Schnitzler, 13. 11. 1903]{L01342 Hermann Bahr an Arthur Schnitzler, 13. 11. 1903}
\nopagebreak\mylabel{L01342v}
\rehead{ }\normalsize\beginnumbering\briefempfaengerindex{Schnitzler, Arthur@\textsc{Schnitzler, Arthur}!zzzBahr, Hermann@\emph{von Hermann Bahr}!1903-11-131@{13. 11. 1903}|(be}
\toendnotes[C]{\smallbreak\pagebreak[2]}\Standort{CUL, Schnitzler, B 5b.}
\physDesc{Postkarte, 278 Zeichen
\newline{}Handschrift: schwarze Tinte, deutsche Kurrent
\newline{}Versand: 1) Stempel: »\nobreak{}\oindex{XIII., Hietzing@\textbf{XIII., Hietzing}, \emph{A.ADM3}|pwk}Wien 13/7, 13. 11. 03, 2–3N\nobreak{}«.   2) Stempel: »\nobreak{}\oindex{XVIII., Waehring@\textbf{XVIII., Währing}, \emph{A.ADM3}|pwk}18/1 Wien, 13. 11. 03, 7.N, Bestellt\nobreak{}«. 
\newline{}Ordnung: mit Bleistift von unbekannter Hand nummeriert:
                                    »103« }
\buchAbdrucke{\weitereDrucke{Hermann Bahr, Arthur Schnitzler: \emph{Briefwechsel, Aufzeichnungen, Dokumente (1891–1931)}. Göttingen: \emph{Wallstein} 2018, S. 281.} }\toendnotes[C]{\smallbreak}\pstart{}{\pb}\textsc{Herrn D\textsuperscript{r} Arthur Schnitzler}\pend{}\pstart{}\textsc{Wien XVIII}\oindex{XVIII., Waehring@\textbf{XVIII., Währing}, \emph{A.ADM3}|pw}\pend{}\pstart{}Spöttelgaſſe 7\oindex{Edmund-Weiss-Gasse 7@\textbf{Edmund-Weiß-Gasse 7}, \emph{Wohngebäude (K.WHS)}|pw}\pend{}{\bigskip}\vspace{1em}
\pstart
           \raggedleft{}{\pb}13. 11. 03\pend
           
\pstart{}Lieber Arthur!\pend\vspace{0.5em}
\pstart
           Danke ſehr. Ich freue mich ſehr, wenn Du wieder einmal heraus ko{\geminationm}ſt – nur bitte: dieſen Sonntag und
                  Montag nicht, weil ich \label{K_L01342-1v}\edtext{nicht hier}{\lemma{\textnormal{\emph{nicht hier}}}\Cendnote{\textnormal{Am Sonntag,
                     15. 11., besuchte er in Salzburg\oindex{Salzburg@\textbf{Salzburg}, \emph{A.ADM2}|pwk} das
                  Grab seiner Eltern\pwindex{Bahr, Alois 11.04.1834 – 05.09.1898@\textsc{Bahr, Alois} (11.04.1834 – 05.09.1898), \emph{Notar/Notarin, Politiker/Politikerin}|pwkv}\pwindex{Bahr, Wilhelmine 06.06.1835 – 16.05.1902@\textsc{Bahr, Wilhelmine} (06.06.1835 – 16.05.1902)|pwkv}.}}}\label{K_L01342-1} bin. Und bitte: ſchick mir den Rekurs\pwindex{Reigen. Zehn Dialoge@\emph{Reigen. Zehn Dialoge}|pwv} gelegentlich zurück.\pend
           
\pstart
           Herzlichſt{\\[\baselineskip]}Dein{\\[\baselineskip]}\spacefill\mbox{Hermann}\pend
           \leftskip=0em{}\selectlanguage{ngerman}\endnumbering\briefempfaengerindex{Schnitzler, Arthur@\textsc{Schnitzler, Arthur}!zzzBahr, Hermann@\emph{von Hermann Bahr}!1903-11-131@{13. 11. 1903}|)be}\mylabel{L01342h}  \normalsize

\doendnotes{C}
\bigskip
\vfill

\clearpage

\footnotesize

\lohead{\textsc{register}}

% Definiere theindex-Environment komplett neu ohne reledmac
\makeatletter
\renewenvironment{theindex}{%
  \section*{\indexname}%
  \setlength{\parindent}{0pt}%
  \setlength{\parskip}{0pt plus 0.3pt}%
  \let\item\@idxitem
}{%
  \clearpage
}
\makeatother

\IfFileExists{\jobname-pw.ind}{\input{\jobname-pw.ind}}{}

\end{document}

      