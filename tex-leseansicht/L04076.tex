%% latex-leseansicht-vorspann.tex
%% Vorspann für die Leseansicht.
%% Lädt die gemeinsame Datei latex-vorspann.tex mit nicht gesetztem Schalter.

\newif\ifkorrekturansicht
\korrekturansichtfalse

\input{../tex-inputs/latex-vorspann}


\section[Arthur Schnitzler und Olga und Elisabeth Gussmann an Gustav Schwarzkopf, 3. 7. 1901]{L04076 Arthur Schnitzler und Olga und Elisabeth Gussmann an Gustav
               Schwarzkopf, 3. 7. 1901}
\nopagebreak\mylabel{L04076v}
\rehead{ }\normalsize\beginnumbering\briefempfaengerindex{Schwarzkopf, Gustav@\textsc{Schwarzkopf, Gustav}!zzzSteinrück, Elisabeth@\emph{von Elisabeth Steinrück}!1901-07-032@{3. 7. 1901}|(be}\briefempfaengerindex{Schwarzkopf, Gustav@\textsc{Schwarzkopf, Gustav}!zzzSchnitzler, Olga@\emph{von Olga Schnitzler}!1901-07-032@{3. 7. 1901}|(be}\briefempfaengerindex{Schwarzkopf, Gustav@\textsc{Schwarzkopf, Gustav}!zzzSchnitzler, Arthur@\emph{von Arthur Schnitzler}!1901-07-032@{3. 7. 1901}|(be}
\toendnotes[C]{\smallbreak\pagebreak[2]}
\correspDesc{Versand  durch Arthur Schnitzler, Olga Gussmann, Elisabeth Gussmann am 3. 7. 1901 in St. Anton am Arlberg
\newline{}Erhalt  durch Gustav Schwarzkopf im Zeitraum [4. 7. 1901
                  – 8. 7. 1901?] in Wien}\toendnotes[C]{\smallbreak}
\Standort{CUL, Schnitzler, B 96.}
\physDesc{Brief, 2 Blätter, 6 Seiten, 2942 Zeichen
\newline{}Handschrift Arthur Schnitzler: Bleistift, deutsche Kurrent
\newline{}Handschrift Olga Schnitzler: Bleistift, deutsche Kurrent
\newline{}Handschrift Elisabeth Steinrück: Bleistift, deutsche Kurrent}\toendnotes[C]{\smallbreak}
\pstart
           {\pb}St. Anton Arlberg\oindex{St. Anton am Arlberg@\textbf{St. Anton am Arlberg}, \emph{Verwaltungsgebiet}|pw},
                  3/7 901.\pend
           \vspace{0.5em}
\pstart
           Mein lieber Guſtav, hier ſind wir gelandet (auf wie lang?) In Salzburg\oindex{Salzburg@\textbf{Salzburg}, \emph{Verwaltungsgebiet}|pw} waren wir\pwindex{Schnitzler, Olga 17.\,1.\,1882 Wien – 13.\,1.\,1970 Lugano@\textsc{Schnitzler, Olga} (17.\,1.\,1882 Wien – 13.\,1.\,1970 Lugano), \emph{Schauspielerin, Sängerin}|pw} 14 Tage, lebten behaglich, ich arbeitete nicht wenig; da{\geminationn} gingen wir auf die Wanderſchaft: Innsbruck\oindex{Innsbruck@\textbf{Innsbruck}, \emph{Verwaltungsgebiet}|pw} (Stubaithal\oindex{Stubaital@\textbf{Stubaital}, \emph{Tal}|pw}, Igls\oindex{Igls@\textbf{Igls}, \emph{Teil eines Landes}|pw}) 2 Tage, dann Landeck\oindex{Landeck@\textbf{Landeck}, \emph{Hauptstadt}|pw}, woſelbſt vor 3 Tagen \textsc{Liesl\pwindex{Steinrück, Elisabeth 19.\,11.\,1885 Wien – 7.\,4.\,1920 Partenkirchen@\textsc{Steinrück, Elisabeth} (19.\,11.\,1885 Wien – 7.\,4.\,1920 Partenkirchen)|pw}} zu uns stieſs. Da{\geminationn} bin ich den Arlberg\oindex{Arlberg@\textbf{Arlberg}, \emph{Berg}|pw} abgeradelt (hinauf immer \textsc{per} Bahn), bis Bludenz\oindex{Bludenz@\textbf{Bludenz}, \emph{Hauptstadt}|pw}, und nun haben
               wir uns zu dieſem Ort entſchloſſen, deſſen Vorzüge ſich zweifellos noch mehr
               erſchließen werden, we{\geminationn} das Wetter conſtant ſein wird
               und ich nicht jede Nacht irgend eine Art von Ge{\pb}birgsaſthma haben werde. Es iſt
               vorläufg noch ziemlich leer – wir wohnen in einem netten Privathaus\oindex{?? [Unterkunft von Schnitzler und Gussmann in St. Anton. 1901]@\textbf{?? [Unterkunft von Schnitzler und Gussmann in St. Anton. 1901]}, \emph{Wohngebäude}|pwv} – ich zahle für mein hübſches
               angenehmes Zimmer 60 Kreuzer (zwei Fenſter), \strikeout{\textcolor{gray}{X}} außer uns dreien wohnt noch niemand in dem Haus, vor 15. Juli ko{\geminationm}t wohl niemand. Um dieſe Zeit ſteigen wir wohl wieder
               zu Thal. Es ſei denn dſs Sie ſich doch entſchlöſſen hieher zu kommen. Da{\geminationn} bleiben wir, ſolang es Ihnen beliebt. Schreiben Sie
               ein Wort – und ein Zimmer zu 60 Kronen iſt für Sie bereit. Denken Sie, daſs ich ein
               gemeiner Egoiſt bin und Sie nie zu etwas auffordern würde, was \uline{mir}{ }{\pb}nicht{ }ſehr angenehm wäre. So darf ich
               mir alles weitere erſparen. –\pend
           
\pstart
           Ich ſchreibe am 2.
                  Akt\pwindex{Schnitzler, Arthur 15. 5. 1862 Wien – 21. 10. 1931 ebd.@\textsc{Schnitzler, Arthur} (15. 5. 1862 Wien – 21. 10. 1931 ebd.), \emph{Schriftsteller, Mediziner}!einsame Weg. Schauspiel in fünf Akten@\strich\emph{Der einsame Weg. Schauspiel in fünf Akten}|pwv}\pwindex{Schnitzler, Arthur 15. 5. 1862 Wien – 21. 10. 1931 ebd.@\textsc{Schnitzler, Arthur} (15. 5. 1862 Wien – 21. 10. 1931 ebd.), \emph{Schriftsteller, Mediziner}!Professor Bernhardi. Komödie in fünf Akten@\strich\emph{Professor Bernhardi. Komödie in fünf Akten}|pwv} meines neuen Stückes\pwindex{Schnitzler, Arthur 15. 5. 1862 Wien – 21. 10. 1931 ebd.@\textsc{Schnitzler, Arthur} (15. 5. 1862 Wien – 21. 10. 1931 ebd.), \emph{Schriftsteller, Mediziner}!einsame Weg. Schauspiel in fünf Akten@\strich\emph{Der einsame Weg. Schauspiel in fünf Akten}|pwv}\pwindex{Schnitzler, Arthur 15. 5. 1862 Wien – 21. 10. 1931 ebd.@\textsc{Schnitzler, Arthur} (15. 5. 1862 Wien – 21. 10. 1931 ebd.), \emph{Schriftsteller, Mediziner}!Professor Bernhardi. Komödie in fünf Akten@\strich\emph{Professor Bernhardi. Komödie in fünf Akten}|pwv}. In dieſem 2. Akt\pwindex{Schnitzler, Arthur 15. 5. 1862 Wien – 21. 10. 1931 ebd.@\textsc{Schnitzler, Arthur} (15. 5. 1862 Wien – 21. 10. 1931 ebd.), \emph{Schriftsteller, Mediziner}!einsame Weg. Schauspiel in fünf Akten@\strich\emph{Der einsame Weg. Schauspiel in fünf Akten}|pw}\pwindex{Schnitzler, Arthur 15. 5. 1862 Wien – 21. 10. 1931 ebd.@\textsc{Schnitzler, Arthur} (15. 5. 1862 Wien – 21. 10. 1931 ebd.), \emph{Schriftsteller, Mediziner}!Professor Bernhardi. Komödie in fünf Akten@\strich\emph{Professor Bernhardi. Komödie in fünf Akten}|pw}
                  ko{\geminationm}t überhaupt kein weibliches Weſen vor, was
               mich ſehr ſtolz macht. – \label{K_L04076-1v}\edtext{Hofmannsthal\pwindex{Hofmannsthal, Hugo von 1.\,2.\,1874 Wien – 15.\,7.\,1929 Rodaun@\textsc{Hofmannsthal, Hugo von} (1.\,2.\,1874 Wien – 15.\,7.\,1929 Rodaun), \emph{Schriftsteller}|pw} und Frau\pwindex{Hofmannsthal, Gertrude von 16.\,3.\,1880 Wien – 9.\,11.\,1959 Paddington@\textsc{Hofmannsthal, Gertrude von} (16.\,3.\,1880 Wien – 9.\,11.\,1959 Paddington)|pwv}}{\lemma{\textnormal{\emph{Hofmannsthal und Frau}}}\Cendnote{\textnormal{Wie sehr Schnitzler über die Begegnung irritiert war, ergibt sich
                  aus der mehrfachen Erwähnung: A. S.: \emph{Tagebuch}, 27. 6. 1901; XXXX Auszeichnungsfehler: Dokument L01140 nicht gefunden, XXXX Auszeichnungsfehler: Dokument L01142 nicht gefunden. }}}\label{K_L04076-1}{ }ſahen wir in I{\geminationn}sbruck\oindex{Innsbruck@\textbf{Innsbruck}, \emph{Verwaltungsgebiet}|pw} – wir fuhren eben im Wagen – sie
               wandelten zu Fuſs. Von \label{K_L04076-2v}\edtext{Richard\pwindex{Beer-Hofmann, Richard 11.\,7.\,1866 Wien – 26.\,9.\,1945 New York City@\textsc{Beer-Hofmann, Richard} (11.\,7.\,1866 Wien – 26.\,9.\,1945 New York City), \emph{Schriftsteller}|pw} bekam ich geſtern einen Brief}{\lemma{\textnormal{\emph{Richard … Brief}}}\Cendnote{\textnormal{XXXX Auszeichnungsfehler: Dokument L01137 nicht gefunden.}}}\label{K_L04076-2}; er ſcheint nicht ſehr
               guter Laune, war ein paar Tage in Venedig\oindex{Venedig@\textbf{Venedig}|pw}. – We{\geminationn} es Ihnen nicht unbequem iſt, bitte kaufen Sie
               gelegentlich 2 Nummern des \label{K_L04076-3v}\edtext{Kikeriki\pwindex{Kikeriki@\emph{Kikeriki}|pw} vom 27. Juni}{\lemma{\textnormal{\emph{Kikeriki vom 27. Juni}}}\Cendnote{\textnormal{In dem bezeichneten Heft der
                  ›Satire‹zeitschrift findet sich diese, nicht namentlich gekennzeichnete
                  antisemitische Auslassung: [O.V.]: \emph{Das
                        Unrecht an Aaron Schnitzler}\pwindex{Unrecht an Aaron Schnitzler@\emph{Das Unrecht an Aaron Schnitzler}|pwk}. In: \emph{Kikeriki}\pwindex{Kikeriki@\emph{Kikeriki}|pwk}, Jg. 41, Nr. 51, 27. 6. 1901,
                  S. 2.}}}\label{K_L04076-3} u bewahren \label{T_L04076-1v}\edtext{sie}{\lemma{\textnormal{\emph{sie}}}\Cendnote{\textnormal{Er schreibt:
                  »Sie«.}}}\label{T_L04076-1} mir auf. Ich nehme an, Sie haben einiges über die Guſtelei\pwindex{Schnitzler, Arthur 15. 5. 1862 Wien – 21. 10. 1931 ebd.@\textsc{Schnitzler, Arthur} (15. 5. 1862 Wien – 21. 10. 1931 ebd.), \emph{Schriftsteller, Mediziner}!Lieutenant Gustl. Novelle@\strich\emph{Lieutenant Gustl. Novelle}|pwv} geleſen. Im Ausland
               war man allgemein faſt liebens{\pb}würdg
               gegen mich. In Wien\oindex{Wien@\textbf{Wien}, \emph{Verwaltungsgebiet}|pw} fehlte es natürlich nicht an
               gemeinen Lügen, und \label{K_L04076-4v}\edtext{Karl Kraus\pwindex{Kraus, Karl 28.\,4.\,1874 Jičín – 12.\,6.\,1936 Wien@\textsc{Kraus, Karl} (28.\,4.\,1874 Jičín – 12.\,6.\,1936 Wien), \emph{Schriftsteller, Publizist, Schriftsteller}|pw} hat die Neuigkeit erfunden\pwindex{Kraus, Karl 28.\,4.\,1874 Jičín – 12.\,6.\,1936 Wien@\textsc{Kraus, Karl} (28.\,4.\,1874 Jičín – 12.\,6.\,1936 Wien), \emph{Schriftsteller, Publizist, Schriftsteller}!(Wieder ein Märtyrer.)@\strich\emph{(Wieder ein Märtyrer.)}|pwv}, daſs ich ein Geſuch eingereicht
               habe, um Landwehrarzt}{\lemma{\textnormal{\emph{Karl … Landwehrarzt}}}\Cendnote{\textnormal{Karl Kraus\pwindex{Kraus, Karl 28.\,4.\,1874 Jičín – 12.\,6.\,1936 Wien@\textsc{Kraus, Karl} (28.\,4.\,1874 Jičín – 12.\,6.\,1936 Wien), \emph{Schriftsteller, Publizist, Schriftsteller}|pwk}: \emph{(Wieder ein Märtyrer.)}\pwindex{Kraus, Karl 28.\,4.\,1874 Jičín – 12.\,6.\,1936 Wien@\textsc{Kraus, Karl} (28.\,4.\,1874 Jičín – 12.\,6.\,1936 Wien), \emph{Schriftsteller, Publizist, Schriftsteller}!(Wieder ein Märtyrer.)@\strich\emph{(Wieder ein Märtyrer.)}|pwk} In: \emph{Die Fackel}\pwindex{Fackel@\emph{Die Fackel}|pwk}, Jg. 3, H. 80, Mitte Juni 1901,
                     S. 20–24, hier S. 22–23: »Herr Schnitzler hatte, als seine Landwehrpflicht abgelaufen war, die
                     schönste Gelegenheit, einem Stande Valet zu sagen, dessen Anschauungen den
                     seinen offenbar zuwider laufen, dessen Empfindlichkeit mindestens den
                     schrankenlos Schaffenden beengen musste. Aber er scheint darauf Wert gelegt zu
                     haben – ein ausdrückliches Gesuch nur konnte solchen Ehrgeiz verwirklichen —,
                     dem Armeeverbande auch weiterhin, als Oberarzt in der Evidenz der Landwehr,
                     anzugehören.«}}}\label{K_L04076-4} bleiben zu dürfen.\pend
           
\pstart
           Die \label{K_L04076-5v}\edtext{Reichswehr\pwindex{Reichswehr@\emph{Reichswehr}|pw}{ }erzählte\pwindex{Lieutenant Gustl«@\emph{»Lieutenant Gustl«}|pwv}}{\lemma{\textnormal{\emph{Reichswehr erzählte}}}\Cendnote{\textnormal{[Gustav Davis\pwindex{Davis, Gustav 3.\,3.\,1856 Bratislava – 21.\,8.\,1951 Hollenstein an der Ybbs@\textsc{Davis, Gustav} (3.\,3.\,1856 Bratislava – 21.\,8.\,1951 Hollenstein an der Ybbs), \emph{Journalist, Herausgeber}|pwk}]: \emph{»Lieutenant Gustl«}\pwindex{Lieutenant Gustl«@\emph{»Lieutenant Gustl«}|pwk}, \emph{Reichswehr}\pwindex{Reichswehr@\emph{Reichswehr}|pwk}, Jg. 14, Nr. 2645, 22. 6. 1901, Morgenblatt,
                     S. 1–2. Vgl. XXXX Auszeichnungsfehler: Dokument L01134 nicht gefunden.}}}\label{K_L04076-5}, wie gern ich mit Stürmer und Säbel herum{ }ſtolzirt.{ }ſei. – Daſs \introOben{}ich\introOben{} die 2. Beſchuldigung, ich habe gegen die Reichs\textcolor{gray}{wehr}\orgindex{Reichswehr@Reichswehr|pw} keine Schritte unternommen, erſt aus dem Urtheil erfuhr, können Sie{ }ſich
               denken. Ich ſchrieb es der N. Fr. Pr.\orgindex{Neue Freie Presse@Neue Freie Presse|pw} aber die
               war wohl froh, daſs ſie die Sache hinter ſich hatte. Leider hab ich eine Du{\geminationm}heit begangen. Zu meiner erſten Überraſchung über {\pb}den \label{K_L04076-6v}\edtext{Leitartikel\pwindex{Wien, 20. Juni@\emph{Wien, 20. Juni}|pwv}}{\lemma{\textnormal{\emph{Leitartikel}}}\Cendnote{\textnormal{[Moriz Benedikt\pwindex{Benedikt, Moriz 27.\,5.\,1849 Kvačice – 18.\,3.\,1920 Wien@\textsc{Benedikt, Moriz} (27.\,5.\,1849 Kvačice – 18.\,3.\,1920 Wien), \emph{Journalist, Herausgeber}|pwk}]: \emph{Wien, 20. Juni}\pwindex{Wien, 20. Juni@\emph{Wien, 20. Juni}|pwk}. In: \emph{Neue Freie Presse}\pwindex{Neue Freie Presse@\emph{Neue Freie Presse}|pwk}, Nr. 13.226, 21. 6. 1901, Morgenblatt, S. 1–2. }}}\label{K_L04076-6} der N. F. Pr.\pwindex{Neue Freie Presse@\emph{Neue Freie Presse}|pw} – aus dem ich überhaupt erſt erfuhr, dſs die Sache
               publik war – und da ich{ }ſicher gedacht hatte, gerade die N. Fr. Pr.\orgindex{Neue Freie Presse@Neue Freie Presse|pw} würde die Sache ganz todt ſchweigen – u nach
               einer ſehr flüchtigen Lecture hab ich mich bei der N.
                  Fr. Pr.\orgindex{Neue Freie Presse@Neue Freie Presse|pw}, allerdings ſehr kühl, bedankt. Trotzdem ärgert mich das heute. Denn
               ich finde \introOben{}heute\introOben{}, daſs der Leitartikel\pwindex{Wien, 20. Juni@\emph{Wien, 20. Juni}|pwv} über dieſe Sache ebenſo du{\geminationm} als feig war. Nun genug davon. –\pend
           
\pstart
           Laſſen Sie bald hören. Im übrigen verweiſe ich nochmals auf \label{K_L04076-7v}\edtext{Seite 3, Zeile 1 u. 2.\pwindex{Unrecht an Aaron Schnitzler@\emph{Das Unrecht an Aaron Schnitzler}|pwv}}{\lemma{\textnormal{\emph{Seite 3, Zeile 1 u. 2.}}}\Cendnote{\textnormal{\emph{Das Unrecht an Aaron Schnitzler}\pwindex{Unrecht an Aaron Schnitzler@\emph{Das Unrecht an Aaron Schnitzler}|pwk} beginnt mit:
                     »Armer Aaron, glänzendſter Dichterſtern im Augias-Muſenſtall
                     ›Jung-Israels‹m \textsc{recte} ›Jung-Wiens\oindex{Wien@\textbf{Wien}, \emph{Verwaltungsgebiet}|pw}‹, welch’ bitteres Unrecht iſt Dir geſchehen, als man
                     Dir das goldene \textsc{Porte d’epée} abnahm.« (Die
                  zweite Zeile endet mit »welch’«.) Die Erwähnung des »Porte
                     d’epée« stellt eine Beziehung zum Artikel\pwindex{Lieutenant Gustl«@\emph{»Lieutenant Gustl«}|pwkv} in der \emph{Reichswehr}\pwindex{Reichswehr@\emph{Reichswehr}|pwk} her.}}}\label{K_L04076-7} \pend
           
\pstart
           Von Herzen{\\[\baselineskip]} Ihr \spacefill\mbox{Arth Sch}\pend
           \leftskip=0em{}\selectlanguage{ngerman}\vspace{1em}
\pstart
           \noindent{}{\pb}{[}hs. Schnitzler:{]} Lieber Herr Schwarzkopf, wir grüßen Sie beide herzlich.\pend
           \pstart \spacefill\mbox{OlgaG.}\pend{}\selectlanguage{ngerman}\vspace{1em}
\pstart
           \noindent{}{[}hs. Steinrück:{]} Ich hoffe, Sie kommen gleich. Mehr kann ich heut’ nicht{ }ſagen, denn ich habe Halsweh!\pend
           
\pstart
           Ihre{\\[\baselineskip]}\spacefill\mbox{Liesl.}\pend
           \leftskip=0em{}\selectlanguage{ngerman}\endnumbering\briefempfaengerindex{Schwarzkopf, Gustav@\textsc{Schwarzkopf, Gustav}!zzzSteinrück, Elisabeth@\emph{von Elisabeth Steinrück}!1901-07-032@{3. 7. 1901}|)be}\briefempfaengerindex{Schwarzkopf, Gustav@\textsc{Schwarzkopf, Gustav}!zzzSchnitzler, Olga@\emph{von Olga Schnitzler}!1901-07-032@{3. 7. 1901}|)be}\briefempfaengerindex{Schwarzkopf, Gustav@\textsc{Schwarzkopf, Gustav}!zzzSchnitzler, Arthur@\emph{von Arthur Schnitzler}!1901-07-032@{3. 7. 1901}|)be}\mylabel{L04076h}
\begin{anhang}
\end{anhang}\newcommand{\dateiname}{L04076}\newcommand{\titel}{Arthur Schnitzler und Olga und Elisabeth Gussmann an Gustav Schwarzkopf, 3. 7. 1901}\newcommand{\editorInnen}{Herausgegeben von Jahnke, SelmaMüller, Martin Anton}%% latex-leseansicht-abspann.tex
%% Abspann für die Leseansicht.
%% Der Schalter \ifkorrekturansicht ist bereits durch den Vorspann gesetzt.

%% latex-abspann.tex
%% Gemeinsamer Abspann für Korrekturansicht und Leseansicht.
%% Setzt den Schalter \ifkorrekturansicht voraus (gesetzt in den
%% einbindenden Dateien latex-korrekturansicht-abspann.tex bzw.
%% latex-leseansicht-abspann.tex).
%% ---------------------------------------------------------------

\normalsize

% Das esempio-Environment wird nur in der Leseansicht benötigt
\ifkorrekturansicht\else
\newenvironment{esempio}[3]%
{
    \vspace{1.5ex}
    \rlap{\underline{#1}}
    \par
    \setlength{\parindent}{0cm}
    \nopagebreak
    \leftskip=#2cm
    \rightskip=#3cm
}
{
    \par
}
\fi

\doendnotes{C}
\bigskip
\vfill

\clearpage

\footnotesize

\ifkorrekturansicht
  \lohead{\textsc{register}}
\fi

% theindex-Environment neu definieren ohne reledmac
\makeatletter
\renewenvironment{theindex}{%
  \ifkorrekturansicht
    \section*{\indexname}%
  \else
    \subsubsection*{Index der erwähnten Entitäten}%
  \fi
  \setlength{\parindent}{0pt}%
  \setlength{\parskip}{0pt plus 0.3pt}%
  \let\item\@idxitem
}{%
  \ifkorrekturansicht\clearpage\fi
}
\makeatother

\IfFileExists{\jobname-pw.ind}{\input{\jobname-pw.ind}}{}

% Quellenangabe nur in der Leseansicht
\ifkorrekturansicht\else
% Fallback-Definitionen, falls die .tex-Datei \titel etc. nicht gesetzt hat
\providecommand{\titel}{}
\providecommand{\editorInnen}{}
\providecommand{\dateiname}{\jobname}

\vspace{3cm}

\vfill

\footnotesize
\textsc{Quelle}: \titel. Herausgegeben von {\editorInnen}. In: \emph{Arthur Schnitzler: Briefwechsel mit Autorinnen und Autoren}.
 Digitale Edition, https://schnitzler-briefe.acdh.oeaw.ac.at/{\dateiname}.html (Stand \today)
\fi

\end{document}


