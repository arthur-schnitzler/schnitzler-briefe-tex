%% latex-korrekturansicht-vorspann.tex
%% Vorspann für die Korrekturansicht.
%% Lädt die gemeinsame Datei latex-vorspann.tex mit gesetztem Schalter.

\newif\ifkorrekturansicht
\korrekturansichttrue

\input{../tex-inputs/latex-vorspann}


\section[Richard Beer-Hofmann an Arthur Schnitzler, 2{[}6?{]}. 9. 1894]{L00373 Richard Beer-Hofmann an Arthur Schnitzler, 2{[}6?{]}. 9. 1894}
\nopagebreak\mylabel{L00373v}
\rehead{ }\normalsize\beginnumbering\briefempfaengerindex{Schnitzler, Arthur@\textsc{Schnitzler, Arthur}!zzzBeer-Hofmann, Richard@\emph{von Richard Beer-Hofmann}!1894-09-261@{2{[}6?{]}. 9. 189}|(be}
\toendnotes[C]{\smallbreak\pagebreak[2]}\Standort{CUL, Schnitzler, B 8.}
\physDesc{Postkarte, 292 Zeichen (aufgedrucktes Hotelwappen )
\newline{}Handschrift: Bleistift, lateinische Kurrent
\newline{}Versand: Stempel: »\nobreak{}\oindex{Mailand@\textbf{Mailand}, \emph{P.PPLA}|pwk}Milano, 2\textcolor{gray}{×} 9-{[}94{]}\nobreak{}«.  
\newline{}Schnitzler: mit Bleistift nummeriert: »39« }
\buchAbdrucke{\weitereDrucke{Arthur Schnitzler, Richard Beer-Hofmann: \emph{Briefwechsel 1891–1931}. Wien, Zürich: \emph{Europaverlag} 1992, S. 60.} }\toendnotes[C]{\smallbreak}\pstart{}{\pb}\textcolor{gray}{\textbf{A}}n Herrn D\textsuperscript{r} Arthur
                  Schnitzler \pend{}\pstart{}Wien\oindex{Wien@\textbf{Wien}, \emph{A.ADM2}|pw}\pend{}\pstart{}IX Frankgasse 1\oindex{Frankgasse 1@\textbf{Frankgasse 1}, \emph{Wohngebäude (K.WHS)}|pw}\pend{}\pstart{}Austria\oindex{Oesterreich@\textbf{Österreich}, \emph{A.PCLI}|pw}\pend{}{\bigskip}\vspace{1em}
\pstart
           \noindent{}{\pb}\label{K_L00373-1v}\edtext{Mau}{\lemma{\textnormal{\emph{Mau}}}\Cendnote{\textnormal{Die Datierung dieses Korrespondenzstücks auf einen bestimmten
                  Tag ist problematisch. Der Poststempel gibt den sicheren Hinweis
                  »2«, doch war Beer-Hofmann\pwindex{Beer-Hofmann, Richard 1866-07-11 – 1945-09-26@\textsc{Beer-Hofmann, Richard} (1866-07-11 – 1945-09-26), \emph{Schriftsteller/Schriftstellerin}|pwk}
                  Anfang des Monats noch nicht auf seiner Reise. Nachdem Schnitzler am 29. 9. 1894 das »Mau« aufnimmt, ist es auf die
                  Woche davor anzusetzen.}}}\label{K_L00373-1}! hätt’ ich wenigstens gesagt wenn ich schon zum
               Schreiben zu faul bin.\pend
           
\pstart
           Bitte senden (lassen Sie) Sie mir die »Zeit\orgindex{Zeit. Wiener Wochenschrift@Die Zeit. Wiener Wochenschrift|pw}« a
                  \uline{posta ferma}{ }Florenz\oindex{Florenz@\textbf{Florenz}, \emph{P.PPLA}|pw} wo ich bis incl. 3\textsuperscript{ten} bin. Vielleicht ist dort auch eine Karte von Ihnen an mich.\pend
           \pstart  Herzlichst Ihr\spacefill\mbox{Richard}\pend{}\selectlanguage{ngerman}\endnumbering\briefempfaengerindex{Schnitzler, Arthur@\textsc{Schnitzler, Arthur}!zzzBeer-Hofmann, Richard@\emph{von Richard Beer-Hofmann}!1894-09-261@{2{[}6?{]}. 9. 189}|)be}\mylabel{L00373h}  \normalsize

\doendnotes{C}
\bigskip
\vfill

\clearpage

\footnotesize

\lohead{\textsc{register}}

% Definiere theindex-Environment komplett neu ohne reledmac
\makeatletter
\renewenvironment{theindex}{%
  \section*{\indexname}%
  \setlength{\parindent}{0pt}%
  \setlength{\parskip}{0pt plus 0.3pt}%
  \let\item\@idxitem
}{%
  \clearpage
}
\makeatother

\IfFileExists{\jobname-pw.ind}{\input{\jobname-pw.ind}}{}

\end{document}

      