%% latex-leseansicht-vorspann.tex
%% Vorspann für die Leseansicht.
%% Lädt die gemeinsame Datei latex-vorspann.tex mit nicht gesetztem Schalter.

\newif\ifkorrekturansicht
\korrekturansichtfalse

\input{../tex-inputs/latex-vorspann}


\section[Richard Beer-Hofmann an Arthur Schnitzler, 2{[}6?{]}. 9. 1894]{L00373 Richard Beer-Hofmann an Arthur Schnitzler, 2[6?]. 9. 1894}
\nopagebreak\mylabel{L00373v}
\rehead{ }\normalsize\beginnumbering\briefempfaengerindex{Schnitzler, Arthur@\textsc{Schnitzler, Arthur}!zzzBeer-Hofmann, Richard@\emph{von Richard Beer-Hofmann}!1894-09-261@{2[6?]. 9. 189}|(be}
\toendnotes[C]{\smallbreak\pagebreak[2]}
\correspDesc{Versand  durch Richard Beer-Hofmann am 2[6?]. 9. 189 in Mailand
\newline{}Erhalt  durch Arthur Schnitzler im Zeitraum [27. 9. 1894
                  – 1. 10. 1894?] in Wien}\toendnotes[C]{\smallbreak}
\Standort{CUL, Schnitzler, B 8.}
\physDesc{Postkarte, 292 Zeichen (aufgedrucktes Hotelwappen )
\newline{}Handschrift: Bleistift, lateinische Kurrent
\newline{}Versand: Stempel: »\nobreak{}\oindex{Mailand@\textbf{Mailand}|pwk}Milano, 2\textcolor{gray}{×} 9-{[}94{]}\nobreak{}«.  
\newline{}Schnitzler: mit Bleistift nummeriert: »39« }
\buchAbdrucke{\weitereDrucke{Arthur Schnitzler, Richard Beer-Hofmann: \emph{Briefwechsel 1891–1931}. Herausgegeben von Konstanze Fliedl. Wien, Zürich: \emph{Europaverlag} 1992, S. 60.} }\toendnotes[C]{\smallbreak}\pstart{}{\pb}\textcolor{gray}{\textbf{A}}n Herrn D\textsuperscript{r} Arthur
                  Schnitzler \pend{}\pstart{}Wien\oindex{Wien@\textbf{Wien}, \emph{Verwaltungsgebiet}|pw}\pend{}\pstart{}IX Frankgasse 1\oindex{Wien@\textbf{Wien}!IX., Alsergrund@\textbf{IX., Alsergrund}!Frankgasse 1@\textbf{Frankgasse 1}, \emph{Wohngebäude}|pw}\pend{}\pstart{}Austria\oindex{Österreich@\textbf{Österreich}|pw}\pend{}{\bigskip}\vspace{1em}
\pstart
           \noindent{}{\pb}\label{K_L00373-1v}\edtext{Mau}{\lemma{\textnormal{\emph{Mau}}}\Cendnote{\textnormal{Die Datierung dieses Korrespondenzstücks auf einen bestimmten
                  Tag ist problematisch. Der Poststempel gibt den sicheren Hinweis
                  »2«, doch war Beer-Hofmann\pwindex{Beer-Hofmann, Richard 11.\,7.\,1866 Wien – 26.\,9.\,1945 New York City@\textsc{Beer-Hofmann, Richard} (11.\,7.\,1866 Wien – 26.\,9.\,1945 New York City), \emph{Schriftsteller}|pwk}
                  Anfang des Monats noch nicht auf seiner Reise. Nachdem Schnitzler am XXXX Auszeichnungsfehler: Dokument L00374 nicht gefunden das »Mau« aufnimmt, ist es auf die
                  Woche davor anzusetzen.}}}\label{K_L00373-1}! hätt’ ich wenigstens gesagt wenn ich schon zum
               Schreiben zu faul bin.\pend
           
\pstart
           Bitte senden (lassen Sie) Sie mir die »Zeit\orgindex{Zeit. Wiener Wochenschrift@Die Zeit. Wiener Wochenschrift|pw}« a
                  \uline{posta ferma}{ }Florenz\oindex{Florenz@\textbf{Florenz}|pw} wo ich bis incl. 3\textsuperscript{ten} bin. Vielleicht ist dort auch eine Karte von Ihnen an mich.\pend
           \pstart Herzlichst Ihr\spacefill\mbox{Richard}\pend{}\selectlanguage{ngerman}\endnumbering\briefempfaengerindex{Schnitzler, Arthur@\textsc{Schnitzler, Arthur}!zzzBeer-Hofmann, Richard@\emph{von Richard Beer-Hofmann}!1894-09-261@{2[6?]. 9. 189}|)be}\mylabel{L00373h}  \newcommand{\dateiname}{L00373}\newcommand{\titel}{Richard Beer-Hofmann an Arthur Schnitzler, 2[6?]. 9. 1894}\newcommand{\editorInnen}{Martin Anton Müller und Gerd-Hermann Susen}%% latex-leseansicht-abspann.tex
%% Abspann für die Leseansicht.
%% Der Schalter \ifkorrekturansicht ist bereits durch den Vorspann gesetzt.

%% latex-abspann.tex
%% Gemeinsamer Abspann für Korrekturansicht und Leseansicht.
%% Setzt den Schalter \ifkorrekturansicht voraus (gesetzt in den
%% einbindenden Dateien latex-korrekturansicht-abspann.tex bzw.
%% latex-leseansicht-abspann.tex).
%% ---------------------------------------------------------------

\normalsize

% Das esempio-Environment wird nur in der Leseansicht benötigt
\ifkorrekturansicht\else
\newenvironment{esempio}[3]%
{
    \vspace{1.5ex}
    \rlap{\underline{#1}}
    \par
    \setlength{\parindent}{0cm}
    \nopagebreak
    \leftskip=#2cm
    \rightskip=#3cm
}
{
    \par
}
\fi

\doendnotes{C}
\bigskip
\vfill

\clearpage

\footnotesize

\ifkorrekturansicht
  \lohead{\textsc{register}}
\fi

% theindex-Environment neu definieren ohne reledmac
\makeatletter
\renewenvironment{theindex}{%
  \ifkorrekturansicht
    \section*{\indexname}%
  \else
    \subsubsection*{Index der erwähnten Entitäten}%
  \fi
  \setlength{\parindent}{0pt}%
  \setlength{\parskip}{0pt plus 0.3pt}%
  \let\item\@idxitem
}{%
  \ifkorrekturansicht\clearpage\fi
}
\makeatother

\IfFileExists{\jobname-pw.ind}{\input{\jobname-pw.ind}}{}

% Quellenangabe nur in der Leseansicht
\ifkorrekturansicht\else
% Fallback-Definitionen, falls die .tex-Datei \titel etc. nicht gesetzt hat
\providecommand{\titel}{}
\providecommand{\editorInnen}{}
\providecommand{\dateiname}{\jobname}

\vspace{3cm}

\vfill

\footnotesize
\textsc{Quelle}: \titel. Herausgegeben von {\editorInnen}. In: \emph{Arthur Schnitzler: Briefwechsel mit Autorinnen und Autoren}.
 Digitale Edition, https://schnitzler-briefe.acdh.oeaw.ac.at/{\dateiname}.html (Stand \today)
\fi

\end{document}


