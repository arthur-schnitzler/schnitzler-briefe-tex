\input{../tex-inputs/latex-pdf-vorspann}
\begin{center}
            \textcolor{red}{ENTWURF. ENTZIFFERUNG NOCH NICHT KORREKTURGELESEN}
                      \end{center}
            
               \section[Richard Beer-Hofmann an Arthur Schnitzler, 2{[}6?{]}. 9. 1894]{ Richard Beer-Hofmann an Arthur Schnitzler, 2{[}6?{]}. 9. 1894}\nopagebreak\mylabel{v}\rehead{ }\begin{ledgroupsized}[t]{13cm}\normalsize\beginnumbering\briefempfaengerindex{Schnitzler, Arthur@\textsc{Schnitzler, Arthur}!zzzBeer-Hofmann, Richard@\emph{von Richard Beer-Hofmann}!1894-09-261@{2{[}6?{]}. 9. 189}|(be} \toendnotes[C]{\smallbreak\pagebreak[2]} \Standort{CUL, Schnitzler, B 8.}
\physDesc{Postkarte (mit aufgedrucktem Hotelwappen)
\newline{}Handschrift: Bleistift, lateinische Kurrent\newline{}Versand: Stempel: »\nobreak{}\oindex{Mailand@\textbf{Mailand}|pwk}Milano, 2\textcolor{gray}{×} 9-{[}94{]}\nobreak{}«.  
\newline{}Schnitzler: mit Bleistift nummeriert: »39« }\buchAbdrucke{\weitereDrucke{Arthur Schnitzler, Richard Beer-Hofmann: \emph{Briefwechsel 1891–1931}. Hg. Konstanze Fliedl. Wien, Zürich: \emph{Europaverlag} 1992, S. 60.} }\toendnotes[C]{\smallbreak}\pstart{}{\pb}\textcolor{gray}{\textbf{A}}n Herrn D\textsuperscript{r} Arthur
                  Schnitzler \pend{}\pstart{}Wien\oindex{Wien@\textbf{Wien}|pw}\pend{}\pstart{}IX Frankgasse 1\oindex{Frankgasse@\textbf{Frankgasse}|pw}\pend{}\pstart{}Austria\oindex{Oesterreich@\textbf{Österreich}|pw}\pend{}{\bigskip}\pstart
           \noindent{}{\pb}\label{K_L00373_1v}\edtext{Mau}{\lemma{\textnormal{\emph{Mau}}}\Cendnote{\textnormal{Die Datierung dieses Korrespondenzstücks auf einen bestimmten Tag ist
                  problematisch. Der Poststempel gibt den sicheren Hinweis »2«,
                  doch war Beer-Hofmann\pwindex{Beer-Hofmann, Richard 11.07.1866 – 26.09.1945@\textsc{Beer-Hofmann, Richard} (11.07.1866 – 26.09.1945), \emph{Schriftsteller}|pwk} Anfang des Monats noch
                  nicht auf seiner Reise. Nachdem Schnitzler\pwindex{Schnitzler, Arthur 15.05.1862 – 21.10.1931@\textsc{Schnitzler, Arthur} (15.05.1862 – 21.10.1931), \emph{Schriftsteller, Mediziner}|pwk} am
                     29. 9. 1894 das »Mau« aufnimmt, ist es auf die
                  Woche davor anzusetzen.}}}\label{K_L00373_1h}! hätt’ ich wenigstens gesagt wenn ich schon zum
               Schreiben zu faul bin.\pend
           \pstart
           Bitte senden (lassen Sie) Sie mir die »Zeit\orgindex{Zeit. Wiener Wochenschrift@Die Zeit. Wiener Wochenschrift|pw}« a \uline{posta ferma}{ }Florenz\oindex{Florenz@\textbf{Florenz}|pw} wo ich bis incl. 3\textsuperscript{ten} bin. Vielleicht ist dort auch eine Karte von Ihnen an mich.\pend
           \pstart  Herzlichst Ihr\spacefill\mbox{Richard}\pend{}\endnumbering\briefempfaengerindex{Schnitzler, Arthur@\textsc{Schnitzler, Arthur}!zzzBeer-Hofmann, Richard@\emph{von Richard Beer-Hofmann}!1894-09-261@{2{[}6?{]}. 9. 189}|)be}\mylabel{h}\end{ledgroupsized}  \newcommand{\dateiname}{L00373}\newcommand{\titel}{Richard Beer-Hofmann an Arthur Schnitzler, 2[6?]. 9. 1894}\newcommand{\editorInnen}{Martin Anton Müller und Gerd-Hermann Susen}\input{../tex-inputs/latex-pdf-abspann}
      