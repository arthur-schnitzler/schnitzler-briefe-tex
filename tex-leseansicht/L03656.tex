%% latex-leseansicht-vorspann.tex
%% Vorspann für die Leseansicht.
%% Lädt die gemeinsame Datei latex-vorspann.tex mit nicht gesetztem Schalter.

\newif\ifkorrekturansicht
\korrekturansichtfalse

\input{../tex-inputs/latex-vorspann}


\section[Stefan Zweig an Arthur Schnitzler, {{[}}17. 7. 1915?{{]}}]{L03656 Stefan Zweig an Arthur Schnitzler, {[}17. 7. 1915?{]}}
\nopagebreak\mylabel{L03656v}
\rehead{ }\normalsize\beginnumbering\briefempfaengerindex{Schnitzler, Arthur@\textsc{Schnitzler, Arthur}!zzzZweig, Stefan@\emph{von Stefan Zweig}!1915-07-171@{{[}17. 7. 1915?{]}}|(be}
\toendnotes[C]{\smallbreak\pagebreak[2]}
\correspDesc{Versand  durch Stefan Zweig am [17. 7. 1915?] in Przemyśl
\newline{}Erhalt  durch Arthur Schnitzler im Zeitraum [18. 7. 1914 – 22. 7. 1915?] in Wien}\toendnotes[C]{\smallbreak}
\Standort{CUL, Schnitzler, B 118.}
\physDesc{Bildpostkarte, 494 Zeichen
\newline{}Handschrift: lila Tinte, lateinische Kurrent
\newline{}Versand: Stempel: »\nobreak{}K. u. K. MILITÄRZENSUR\nobreak{}«.  }
\buchAbdrucke{\weitereDrucke{Stefan Zweig: \emph{Briefwechsel mit Hermann Bahr, Sigmund Freud, Rainer Maria
                        Rilke und Arthur Schnitzler}. Herausgegeben von Jeffrey B. Berlin, Hans-Ulrich Lindken und Donald A. Prater. Frankfurt am Main: \emph{S. Fischer} 1987, S. 395.} }\toendnotes[C]{\smallbreak}\pstart{}Stefan Zweig, Einj. Freiw. Feldwebel\pend{}\pstart{}zugeteilt dem K. u. K Kriegsarchiv\orgindex{Kriegsarchiv@Kriegsarchiv|pw}\pend{}\pstart{}auf Dienstreise derzeit Przemysl\oindex{Przemyśl@\textbf{Przemyśl}, \emph{Hauptstadt}|pw}\pend{}{\bigskip}\pstart{}{\pb}\textcolor{gray}{\textbf{Pocztówka –}}{ }\substVorne{}\textsuperscript{\textcolor{gray}{\textbf{Post}}}\substDazwischen{}Feldpost\substHinten{}\textcolor{gray}{\textbf{karte}}\pend{}\pstart{}D\textsuperscript{r} Artur Schnitzler\pend{}\pstart{}Wien – Cottage\oindex{Wien@\textbf{Wien}!XVIII., Währing@\textbf{XVIII., Währing}!Währinger Cottage@\textbf{Währinger Cottage}, \emph{Teil eines besiedelten Ortes}|pw}\pend{}\pstart{}\label{K_L03656-1v}\edtext{Sternwartestrasse 72}{\lemma{\textnormal{\emph{Sternwartestrasse 72}}}\Cendnote{\textnormal{Zweig\pwindex{Zweig, Stefan 28.\,11.\,1881 Wien – 23.\,2.\,1942 Petrópolis@\textsc{Zweig, Stefan} (28.\,11.\,1881 Wien – 23.\,2.\,1942 Petrópolis), \emph{Schriftsteller}|pwk} wechselt bei der Adressierung
                        seiner Schreiben an Schnitzler immer
                        wieder zwischen der falschen Hausnummer »72« und der
                        richtigen »71«.}}}\label{K_L03656-1}\oindex{Wien@\textbf{Wien}!XVIII., Währing@\textbf{XVIII., Währing}!Sternwartestraße 71@\textbf{Sternwartestraße 71}, \emph{Wohngebäude}|pw}\pend{}{\bigskip}
\pstart
           \noindent{}\centering{}{\pb}\textcolor{gray}{\textbf{Przemyśl, ul. Mickiewicza\oindex{Adama Mickiewicza@\textbf{Adama Mickiewicza}, \emph{Straße}|pw} – Przemyśl, Mickiewiczstrasse\oindex{Adama Mickiewicza@\textbf{Adama Mickiewicza}, \emph{Straße}|pw}.}}\pend
           \vspace{1em}
\pstart
           \noindent{}{\pb}Lieber verehrter Herr Doktor, ich habe in diesen \label{K_L03656-2v}\edtext{galizischen\oindex{Galizien@\textbf{Galizien}|pw} Tagen}{\lemma{\textnormal{\emph{galizischen Tagen}}}\Cendnote{\textnormal{Zweig\pwindex{Zweig, Stefan 28.\,11.\,1881 Wien – 23.\,2.\,1942 Petrópolis@\textsc{Zweig, Stefan} (28.\,11.\,1881 Wien – 23.\,2.\,1942 Petrópolis), \emph{Schriftsteller}|pwk} war vom 13. 7. 1915 bis
                  zum 26. 7. 1915 im Kriegsgebiet von Galizien\oindex{Galizien@\textbf{Galizien}|pwk}, kurz nachdem die russische\oindex{Russland@\textbf{Russland}|pwk} Armee zurückgedrängt worden war. Zweigs\pwindex{Zweig, Stefan 28.\,11.\,1881 Wien – 23.\,2.\,1942 Petrópolis@\textsc{Zweig, Stefan} (28.\,11.\,1881 Wien – 23.\,2.\,1942 Petrópolis), \emph{Schriftsteller}|pwk} Reise begann in Krakau\oindex{Krakau@\textbf{Krakau}, \emph{Verwaltungsgebiet}|pwk} und endete in Budapest\oindex{Budapest@\textbf{Budapest}, \emph{Hauptstadt}|pwk}.
                        (\emph{Tagebuch aus dem Kriegsjahr 1915.
                        Zweiter Band}\pwindex{Zweig, Stefan 28.\,11.\,1881 Wien – 23.\,2.\,1942 Petrópolis@\textsc{Zweig, Stefan} (28.\,11.\,1881 Wien – 23.\,2.\,1942 Petrópolis), \emph{Schriftsteller}!Tagebuch aus dem Kriegsjahr 1915. Zweiter Band@\strich\emph{Tagebuch aus dem Kriegsjahr 1915. Zweiter Band}|pwk}, SZ-AAP/L3. SZ-AAP/L3) Ein Tag vor Reisebeginn
                  nannte er in einem Brief an Franz Karl Ginzkey\pwindex{Ginzkey, Franz Karl 8.\,9.\,1871 Pula – 11.\,4.\,1963 Wien@\textsc{Ginzkey, Franz Karl} (8.\,9.\,1871 Pula – 11.\,4.\,1963 Wien), \emph{Schriftsteller}|pwk} die
                  Route: »Tarnow\oindex{Tarnów@\textbf{Tarnów}, \emph{Hauptstadt}|pw}, Przemysl\oindex{Przemyśl@\textbf{Przemyśl}, \emph{Hauptstadt}|pw}, Lemberg\oindex{Lviv@\textbf{Lviv}|pw}, Stryi\oindex{Stryj@\textbf{Stryj}|pw}, Drohobycz\oindex{Drohobych@\textbf{Drohobych}, \emph{Hauptstadt}|pw}, Ungvar\oindex{Uzhhorod@\textbf{Uzhhorod}|pw}«.
                  (Stefan Zweig\pwindex{Zweig, Stefan 28.\,11.\,1881 Wien – 23.\,2.\,1942 Petrópolis@\textsc{Zweig, Stefan} (28.\,11.\,1881 Wien – 23.\,2.\,1942 Petrópolis), \emph{Schriftsteller}|pwk}: \emph{Briefe. Bd. II: 1914–1919}. Herausgegeben von Knut Beck, Jeffrey B. Berlin und Natascha
                     Weschenbach-Feggeler. Frankfurt am Main: \emph{S. Fischer}{ }1998, S. 76.)
                  Von den Herausgebern auf den Folgetag umdatiert ist ein von Zweig\pwindex{Zweig, Stefan 28.\,11.\,1881 Wien – 23.\,2.\,1942 Petrópolis@\textsc{Zweig, Stefan} (28.\,11.\,1881 Wien – 23.\,2.\,1942 Petrópolis), \emph{Schriftsteller}|pwk}
                  mit »16. Juli 1915« datiertes Schreiben aus Przemyśl\oindex{Przemyśl@\textbf{Przemyśl}, \emph{Hauptstadt}|pwk}
                  an Raoul Auernheimer\pwindex{Auernheimer, Raoul 15.\,4.\,1876 Wien – 6.\,1.\,1948 Oakland@\textsc{Auernheimer, Raoul} (15.\,4.\,1876 Wien – 6.\,1.\,1948 Oakland), \emph{Schriftsteller, Journalist, Kritiker}|pwk}, das in der Eröffnung 
                  ebenfalls die »galizischen\oindex{Galizien@\textbf{Galizien}|pw} Tage[]« erwähnt: »Lieber Freund, ich habe jetzt heiße und herrliche Tage hier in Galizien\oindex{Galizien@\textbf{Galizien}|pw}«
                   (ebd., S. 77). Die dichte Reiseroute spricht dafür, dass sich Zweig\pwindex{Zweig, Stefan 28.\,11.\,1881 Wien – 23.\,2.\,1942 Petrópolis@\textsc{Zweig, Stefan} (28.\,11.\,1881 Wien – 23.\,2.\,1942 Petrópolis), \emph{Schriftsteller}|pwk} nur
                  kurz in Przemyśl\oindex{Przemyśl@\textbf{Przemyśl}, \emph{Hauptstadt}|pwk} aufhielt und diese Karte am selben Tag
                  wie das Korrespondenzstück an Auernheimer\pwindex{Auernheimer, Raoul 15.\,4.\,1876 Wien – 6.\,1.\,1948 Oakland@\textsc{Auernheimer, Raoul} (15.\,4.\,1876 Wien – 6.\,1.\,1948 Oakland), \emph{Schriftsteller, Journalist, Kritiker}|pwk} verfasst wurde.}}}\label{K_L03656-2}
               Unendliches gesehn: den ungeheuren Gang dieser gewaltigen Centripetalmaschine, die
               alle Kraft eines Reiches mit Wucht nach aussen schleudert und dann eine tragische
               aber doch schöne Welt: Galizien\oindex{Galizien@\textbf{Galizien}|pw}. Ich werde Ihnen
               viel zu erzählen haben.\pend
           
\pstart
           In Verehrung getreu Ihr{\\[\baselineskip]}\spacefill\mbox{Stefan Zweig}\pend
           \leftskip=0em{}\selectlanguage{ngerman}\endnumbering\briefempfaengerindex{Schnitzler, Arthur@\textsc{Schnitzler, Arthur}!zzzZweig, Stefan@\emph{von Stefan Zweig}!1915-07-171@{{[}17. 7. 1915?{]}}|)be}\mylabel{L03656h}  \newcommand{\dateiname}{L03656}\newcommand{\titel}{Stefan Zweig an Arthur Schnitzler, [17. 7. 1915?]}\newcommand{\editorInnen}{Selma Jahnke und Martin Anton Müller}%% latex-leseansicht-abspann.tex
%% Abspann für die Leseansicht.
%% Der Schalter \ifkorrekturansicht ist bereits durch den Vorspann gesetzt.

%% latex-abspann.tex
%% Gemeinsamer Abspann für Korrekturansicht und Leseansicht.
%% Setzt den Schalter \ifkorrekturansicht voraus (gesetzt in den
%% einbindenden Dateien latex-korrekturansicht-abspann.tex bzw.
%% latex-leseansicht-abspann.tex).
%% ---------------------------------------------------------------

\normalsize

% Das esempio-Environment wird nur in der Leseansicht benötigt
\ifkorrekturansicht\else
\newenvironment{esempio}[3]%
{
    \vspace{1.5ex}
    \rlap{\underline{#1}}
    \par
    \setlength{\parindent}{0cm}
    \nopagebreak
    \leftskip=#2cm
    \rightskip=#3cm
}
{
    \par
}
\fi

\doendnotes{C}
\bigskip
\vfill

\clearpage

\footnotesize

\ifkorrekturansicht
  \lohead{\textsc{register}}
\fi

% theindex-Environment neu definieren ohne reledmac
\makeatletter
\renewenvironment{theindex}{%
  \ifkorrekturansicht
    \section*{\indexname}%
  \else
    \subsubsection*{Index der erwähnten Entitäten}%
  \fi
  \setlength{\parindent}{0pt}%
  \setlength{\parskip}{0pt plus 0.3pt}%
  \let\item\@idxitem
}{%
  \ifkorrekturansicht\clearpage\fi
}
\makeatother

\IfFileExists{\jobname-pw.ind}{\input{\jobname-pw.ind}}{}

% Quellenangabe nur in der Leseansicht
\ifkorrekturansicht\else
% Fallback-Definitionen, falls die .tex-Datei \titel etc. nicht gesetzt hat
\providecommand{\titel}{}
\providecommand{\editorInnen}{}
\providecommand{\dateiname}{\jobname}

\vspace{3cm}

\vfill

\footnotesize
\textsc{Quelle}: \titel. Herausgegeben von {\editorInnen}. In: \emph{Arthur Schnitzler: Briefwechsel mit Autorinnen und Autoren}.
 Digitale Edition, https://schnitzler-briefe.acdh.oeaw.ac.at/{\dateiname}.html (Stand \today)
\fi

\end{document}


