%% latex-korrekturansicht-vorspann.tex
%% Vorspann für die Korrekturansicht.
%% Lädt die gemeinsame Datei latex-vorspann.tex mit gesetztem Schalter.

\newif\ifkorrekturansicht
\korrekturansichttrue

\input{../tex-inputs/latex-vorspann}


\section[Stefan Zweig an Arthur Schnitzler, {[}17. 7. 1915?{]}]{L03656 Stefan Zweig an Arthur Schnitzler, {[}17. 7. 1915?{]}}
\nopagebreak\mylabel{L03656v}
\rehead{ }\normalsize\beginnumbering\briefempfaengerindex{Schnitzler, Arthur@\textsc{Schnitzler, Arthur}!zzzZweig, Stefan@\emph{von Stefan Zweig}!1914-07-171@{{[}17. 7. 1915?{]}}|(be}
\toendnotes[C]{\smallbreak\pagebreak[2]}\Standort{CUL, Schnitzler, B 118.}
\physDesc{, 494 Zeichen
\newline{}Handschrift: blaue Tinte, lateinische Kurrent
\newline{}Versand: Stempel: »\nobreak{}K. u. K. MILITÄRZENSUR\nobreak{}«.  }
\buchAbdrucke{\weitereDrucke{Stefan Zweig: \emph{Briefwechsel mit Hermann Bahr, Sigmund Freud, Rainer Maria
                        Rilke und Arthur Schnitzler}. Frankfurt am Main: \emph{S. Fischer} 1987, S. 395.} }\toendnotes[C]{\smallbreak}\pstart{}Stefan Zweig, Einj. Freiw. Feldwebel\pend{}\pstart{}zugeteilt dem K. u. K Kriegsarchiv\orgindex{Kriegsarchiv@Kriegsarchiv|pw}\pend{}\pstart{}auf Dienstreise derzeit Przemysl\oindex{Przemyśl@\textbf{Przemyśl}, \emph{P.PPLA2}|pw}\pend{}{\bigskip}\pstart{}{\pb}\substVorne{}\textsuperscript{\textcolor{gray}{\textbf{Postkarte}}}\substDazwischen{}Feldpost\substHinten{}\pend{}\pstart{}D\textsuperscript{r} Artur Schnitzler\pend{}\pstart{}Wien – Cottage\oindex{Waehringer Cottage@\textbf{Währinger Cottage}, \emph{Teil eines besiedelten Ortes (A.BSOX)}|pw}\pend{}\pstart{}\label{K_L03656-1v}\edtext{Sternwartestrasse 72}{\lemma{\textnormal{\emph{Sternwartestrasse 72}}}\Cendnote{\textnormal{Zweig\pwindex{Zweig, Stefan 28.11.1881 – 23.02.1942@\textsc{Zweig, Stefan} (28.11.1881 – 23.02.1942), \emph{Schriftsteller/Schriftstellerin}|pwk} wechselt bei der Adressierung
                        seiner Schreiben an Schnitzler immer
                        wieder zwischen der falschen Hausnummer »72« und der
                        richtigen »71«.}}}\label{K_L03656-1}\oindex{Sternwartestrasse 71@\textbf{Sternwartestraße 71}, \emph{Wohngebäude (K.WHS)}|pw}\pend{}{\bigskip}
\pstart
           \noindent{}\centering{}{\pb}\textcolor{gray}{\textbf{Przemyśl, ul. Mickiewicza\oindex{Adama Mickiewicza@\textbf{Adama Mickiewicza}, \emph{Straße (K.STR)}|pw} – Przemyśl, Mickiewiczstrasse\oindex{Adama Mickiewicza@\textbf{Adama Mickiewicza}, \emph{Straße (K.STR)}|pw}.}}\pend
           \vspace{1em}
\pstart
           \noindent{}{\pb}Lieber verehrter Herr Doktor, ich habe in diesen \label{K_L03656-2v}\edtext{galizischen\oindex{Galizien@\textbf{Galizien}, \emph{Region}|pw} Tagen}{\lemma{\textnormal{\emph{galizischen Tagen}}}\Cendnote{\textnormal{Zweig\pwindex{Zweig, Stefan 28.11.1881 – 23.02.1942@\textsc{Zweig, Stefan} (28.11.1881 – 23.02.1942), \emph{Schriftsteller/Schriftstellerin}|pwk} war vom 13. 7. 1915 bis
                  zum 26. 7. 1915 im Kriegsgebiet von Galizien\oindex{Galizien@\textbf{Galizien}, \emph{Region}|pwk}, kurz nachdem die russische\oindex{Russland@\textbf{Russland}, \emph{A.PCLI}|pwk} Armee zurückgedrängt worden war. Zweigs\pwindex{Zweig, Stefan 28.11.1881 – 23.02.1942@\textsc{Zweig, Stefan} (28.11.1881 – 23.02.1942), \emph{Schriftsteller/Schriftstellerin}|pwk} Reise begann in Krakau\oindex{Krakau@\textbf{Krakau}, \emph{A.ADM3}|pwk} und endete in Budapest\oindex{Budapest@\textbf{Budapest}, \emph{P.PPLC}|pwk}.
                        (\emph{Tagebuch aus dem Kriegsjahr 1915.
                        Zweiter Band}\pwindex{Tagebuch aus dem Kriegsjahr 1915. Zweiter Band@\emph{Tagebuch aus dem Kriegsjahr 1915. Zweiter Band}|pwk}, SZ-AAP/L3. SZ-AAP/L3) Ein Tag vor Reisebeginn
                  nannte er in einem Brief an Franz Karl Ginzkey\pwindex{Ginzkey, Franz Karl 08.09.1871 – 11.04.1963@\textsc{Ginzkey, Franz Karl} (08.09.1871 – 11.04.1963), \emph{Schriftsteller/Schriftstellerin}|pwk} die
                  Route: »Tarnow\oindex{Tarnów@\textbf{Tarnów}, \emph{P.PPLA2}|pw}, Przemysl\oindex{Przemyśl@\textbf{Przemyśl}, \emph{P.PPLA2}|pw}, Lemberg\oindex{Lviv@\textbf{Lviv}, \emph{P.PPLA}|pw}, Stryi\oindex{Stryj@\textbf{Stryj}, \emph{Besiedelter Ort (A.BSO)}|pw}, Drohobycz\oindex{Drohobych@\textbf{Drohobych}, \emph{P.PPLA2}|pw}, Ungvar\oindex{Uzhhorod@\textbf{Uzhhorod}, \emph{P.PPLA}|pw}«.
                  (Stefan Zweig\pwindex{Zweig, Stefan 28.11.1881 – 23.02.1942@\textsc{Zweig, Stefan} (28.11.1881 – 23.02.1942), \emph{Schriftsteller/Schriftstellerin}|pwk}: \emph{Briefe. Bd. II: 1914–1919}. Herausgegeben von Knut Beck, Jeffrey B. Berlin und Natascha
                     Weschenbach-Feggeler. Frankfurt am Main: \emph{S. Fischer}{ }1998, S. 76.)
                  Von den Herausgebern auf den Folgetag umdatiert ist ein von Zweig\pwindex{Zweig, Stefan 28.11.1881 – 23.02.1942@\textsc{Zweig, Stefan} (28.11.1881 – 23.02.1942), \emph{Schriftsteller/Schriftstellerin}|pwk}
                  mit »16. Juli 1915« datiertes Schreiben aus Przemyśl\oindex{Przemyśl@\textbf{Przemyśl}, \emph{P.PPLA2}|pwk}
                  an Raoul Auernheimer\pwindex{Auernheimer, Raoul 15.04.1876 – 06.01.1948@\textsc{Auernheimer, Raoul} (15.04.1876 – 06.01.1948), \emph{Schriftsteller/Schriftstellerin, Journalist/Journalistin, Kritiker/Kritikerin}|pwk}, das in der Eröffnung 
                  ebenfalls die »galizische[]\oindex{Galizien@\textbf{Galizien}, \emph{Region}|pw} Tagen« erwähnt: »Lieber Freund, ich habe jetzt heiße und herrliche Tage hier in Galizien\oindex{Galizien@\textbf{Galizien}, \emph{Region}|pw}«
                  erwähnt (ebd., S. 77). Die dichte Reiseroute spricht dafür, dass sich Zweig\pwindex{Zweig, Stefan 28.11.1881 – 23.02.1942@\textsc{Zweig, Stefan} (28.11.1881 – 23.02.1942), \emph{Schriftsteller/Schriftstellerin}|pwk} nur
                  kurz in Przemyśl\oindex{Przemyśl@\textbf{Przemyśl}, \emph{P.PPLA2}|pwk} aufhielt und diese Karte am selben Tag
                  wie der Brief an Auernheimer\pwindex{Auernheimer, Raoul 15.04.1876 – 06.01.1948@\textsc{Auernheimer, Raoul} (15.04.1876 – 06.01.1948), \emph{Schriftsteller/Schriftstellerin, Journalist/Journalistin, Kritiker/Kritikerin}|pwk} verfasst wurde.}}}\label{K_L03656-2}
               Unendliches gesehn: den ungeheuren Gang dieser gewaltigen Centripetalmaschine, die
               alle Kraft eines Reiches mit Wucht nach aussen schleudert und dann eine tragische
               aber doch schöne Welt: Galizien\oindex{Galizien@\textbf{Galizien}, \emph{Region}|pw}. Ich werde Ihnen
               viel zu erzählen haben. \pend
           
\pstart
           In Verehrung getreu Ihr{\\[\baselineskip]}\spacefill\mbox{Stefan Zweig}\pend
           \leftskip=0em{}\selectlanguage{ngerman}\endnumbering\briefempfaengerindex{Schnitzler, Arthur@\textsc{Schnitzler, Arthur}!zzzZweig, Stefan@\emph{von Stefan Zweig}!1914-07-171@{{[}17. 7. 1915?{]}}|)be}\mylabel{L03656h}
\begin{anhang}
\end{anhang}\normalsize

\doendnotes{C}
\bigskip
\vfill

\clearpage

\footnotesize

\lohead{\textsc{register}}

% Definiere theindex-Environment komplett neu ohne reledmac
\makeatletter
\renewenvironment{theindex}{%
  \section*{\indexname}%
  \setlength{\parindent}{0pt}%
  \setlength{\parskip}{0pt plus 0.3pt}%
  \let\item\@idxitem
}{%
  \clearpage
}
\makeatother

\IfFileExists{\jobname-pw.ind}{\input{\jobname-pw.ind}}{}

\end{document}

      