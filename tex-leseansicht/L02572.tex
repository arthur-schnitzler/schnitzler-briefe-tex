%% latex-korrekturansicht-vorspann.tex
%% Vorspann für die Korrekturansicht.
%% Lädt die gemeinsame Datei latex-vorspann.tex mit gesetztem Schalter.

\newif\ifkorrekturansicht
\korrekturansichttrue

\input{../tex-inputs/latex-vorspann}


\section[Therese Rie-Andro an Arthur Schnitzler, 3. 5. 1923]{L02572 Therese Rie-Andro an Arthur Schnitzler, 3. 5. 1923}
\nopagebreak\mylabel{L02572v}
\rehead{ }\normalsize\beginnumbering\briefempfaengerindex{Schnitzler, Arthur@\textsc{Schnitzler, Arthur}!zzzRie, Therese@\emph{von Therese Rie}!1923-05-031@{3. 5. 1923}|(be}
\toendnotes[C]{\smallbreak\pagebreak[2]}\Standort{DLA, A:Schnitzler, 85.1.4310.}
\physDesc{Brief, 2 Blätter, 4 Seiten, 4026 Zeichen (das zweite Blatt mit »II.« paginiert)
\newline{}Handschrift: blaue Tinte, lateinische Kurrent
\newline{}Schnitzler: 1) mit Bleistift beschriftet: »\textsc{Andro}«  2) mit rotem Buntstift vier Unterstreichungen
\newline{}Ordnung: mit Bleistift von unbekannter Hand den Artikel »eine« vor
                                    »Schokoladebonbon« durch Streichung des Schluss-e
                                 angepasst }\toendnotes[C]{\smallbreak}
\pstart
           \raggedleft{}{\pb}Wien\oindex{Wien@\textbf{Wien}, \emph{A.ADM2}|pw}, 3. Mai 1923.\pend
           
\pstart
           \raggedleft{}IV, Schönburgſtr. 48\oindex{Schoenburgstrasse@\textbf{Schönburgstraße}, \emph{Straße (K.STR)}|pw}.\pend
           
\pstart{}Verehrter Herr Doktor,\pend\vspace{0.5em}
\pstart
           wie sehr mich Ihre guten und lieben \label{K_L02572-1v}\edtext{Worte}{\lemma{\textnormal{\emph{Worte}}}\Cendnote{\textnormal{Sie reagiert hier auf eine
                  nicht überlieferte Karte Schnitzlers, in der
                  dieser ihr zu einer Arbeit gratuliert haben dürfte. Es dürfte sich um ihren Roman
                    \emph{Der Klimenole}\textcolor{red}{\textsuperscript{XXXX indx}} handeln, der eben in der \emph{Deutschen Verlags-Anstalt}XXXX ORGangabe fehlt
                    erschienen war. Vgl. Arthur Schnitzler an Stefan Zweig, 29. 5. 1923.}}}\label{K_L02572-1} erfreut haben, kann ich Ihnen
               schwer schildern; denn Sie sind es ja gewesen, der mich und meine ganze Generation
               künstlerisch gesäugt hat – die Kühnheit dieses Bildes bedrückt Sie hoffentlich nicht!
               – und es iſt kaum vorstellbar, was aus uns geworden wäre, wenn wir Sie, Gustav Mahler\pwindex{Mahler, Gustav 07.07.1860 – 18.05.1911@\textsc{Mahler, Gustav} (07.07.1860 – 18.05.1911), \emph{Theaterleiter/Theaterleiterin, Komponist/Komponistin, Dirigent/Dirigentin}|pw} und Hugo Wolf\pwindex{Wolf, Hugo 13.03.1860 – 22.02.1903@\textsc{Wolf, Hugo} (13.03.1860 – 22.02.1903), \emph{Komponist/Komponistin}|pw} nicht gehabt hätten, zu denen ich als
               Reſpondizierenden auch noch Kainz\pwindex{Kainz, Josef 02.01.1858 – 20.09.1910@\textsc{Kainz, Josef} (02.01.1858 – 20.09.1910), \emph{Schauspieler/Schauspielerin}|pw} rechnen
               möchte. Ich bin mein ganzes Leben lang \strikeout{\textcolor{gray}{viel}} mit Ihren Gestalten umgeben gewesen und namentlich Herr v. Sala\pwindex{einsame Weg. Schauspiel in fuenf Akten@\emph{Der einsame Weg. Schauspiel in fünf Akten}|pwv} war es, der mich oft und oft
               auf meinen Wienerwald\oindex{Wienerwald@\textbf{Wienerwald}, \emph{Ausflugsziel}|pw}-Spaziergängen begleitet
               hat. Es gibt kaum eine Frage meines Lebens, die ich nicht mit ihm durchgesprochen
               habe und oft habe ich mich auch über ihn ärgern müſſen, weil er gar nicht meiner
               Ansicht war und sich zuweilen in der nichtsnutzigsten Art über mich luſtig gemacht
               hat. Aber das war heilsam. Und das meiſtzitierte Werk in meinem Hause iſt jedenfalls
                  »Literatur\pwindex{Literatur@\emph{Literatur}|pw}« gewesen, das mich, so hoffe ich
               wenigſtens, vor mancher kleinen Geschmacksentgleisung bewahrt hat. So haben Sie also
               auch noch ungemein pädagogiſch gewirkt!\pend
           
\pstart
           {\pb}Manches Jahr habe ich mir gewünscht, Ihnen das einmal
               persönlich zu sagen, dann aber davon absehen gelernt. Denn es wäre nur auf Grund
               gemeinsamer gesellschaftlicher Beziehungen möglich gewesen und davon halte ich nicht
               sehr viel. Es ko{\geminationm}t dabei kaum jemals etwas Menschliches
               heraus und wird schließlich nur zu einer Serie von Verlegenheiten. Und am Ende iſt es
               einem Künſtler wol lieber, wenn die Saat, die er in andern gesät hat, zu einer, wenn
               auch noch so bescheidenen Frucht reift, als wenn ihm \uline{noch} eine Dame versichert, wie sehr sie seine Werke bewundere! – –\pend
           
\pstart
           Nur der freundliche Passus in Ihrer Karte: Sie wollten auch meine andern Arbeiten
               kennnen lernen, veranlaßt mich, Ihnen mein kleines Buch »Die Komödiantin Dora X.\pwindex{Komoediantin Dora X. Roman@\emph{Die Komödiantin Dora X. Roman}|pw}« zu schicken; sonſt bin ich nicht so,
               daß ich die Menschen mit meiner Literatur überschütte. Das Büchlein bitte ich Sie\strikeout{,} aber nur als \label{K_L02572-2v}\edtext{Eisenbahnlektüre}{\lemma{\textnormal{\emph{Eisenbahnlektüre}}}\Cendnote{\textnormal{Siehe A. S.: \emph{Tagebuch}, 7. 5. 1923.
               }}}\label{K_L02572-2} zu verwenden; zu viel mehr taugt es nicht. Es iſt ein nicht sehr tiefes
               Problem, nicht sehr tief gefaßt und für mich höchstens dadurch bemerkenswert, daß es
               Jahre später in meiner Umgebung ziemlich wahr geworden iſt. Wie es denn offenbar den
               meiſten Schreibenden, den Kleinen wie den Großen, so ergeht, {\pb}daß sie meinen, das Leben abzuschreiben, während es schließlich das Leben iſt, daß
                  \uline{sie} ganz munter plagiiert. – –\pend
           
\pstart
           Wenn ich aber vorhin von gemeinsamen Beziehungen sprach, die ich nicht für so wichtig
               halte, so möchte ich doch einer gedenken, die mir lieb und teuer iſt und an die ich
               denken muſs, so oft ich Ihren Namen höre: der Erinnerung an Ihre Eltern\pwindex{Schnitzler, Louise 1840-07-08 – 1911-09-09@\textsc{Schnitzler, Louise} (1840-07-08 – 1911-09-09)|pwv}\pwindex{Schnitzler, Johann 10.04.1835 – 02.05.1893@\textsc{Schnitzler, Johann} (10.04.1835 – 02.05.1893), \emph{Laryngologe/Laryngologin}|pwv}, die ich beide noch
               gekannt habe und namentlich an Ihren Vater\pwindex{Schnitzler, Johann 10.04.1835 – 02.05.1893@\textsc{Schnitzler, Johann} (10.04.1835 – 02.05.1893), \emph{Laryngologe/Laryngologin}|pwv}, der meine früheſte Kindheitserinnerung bildet. Man
               sagte mir, daſs er mich als 3jähriges Kind von einer schweren Diphteritis errettet
               habe und es iſt meine erſte Erinnerung überhaupt, wie er mir i{\geminationm}er eine Schokolodebonbon auf einen Löffel Chinin tat,
               daſs ich das bittere Zeug nehmen sollte. Wieviel iſt seither vorbeigegangen und
               vergessen worden, aber das Bild iſt mir geblieben! – – Im Nachlaß meiner Eltern\pwindex{Herz, Maximilian 1837-07-05 – 1890-07-13@\textsc{Herz, Maximilian} (1837-07-05 – 1890-07-13), \emph{Mediziner/Medizinerin}|pwv}\pwindex{Herz, Marie @\textsc{Herz, Marie}|pwv} fand ich
               später ein Tagebuch meines Vaters\pwindex{Herz, Maximilian 1837-07-05 – 1890-07-13@\textsc{Herz, Maximilian} (1837-07-05 – 1890-07-13), \emph{Mediziner/Medizinerin}|pwv} aus dem Jahre 1863, in welchem viel von \introOben{}einem
                  Briefwechsel mit\introOben{} dem Ihren die Rede iſt – sie waren ja Kollegen, wie ich
               weiß, schon vom \label{K_L02572-3v}\edtext{Schottengymnasium\orgindex{Schottengymnasium@Schottengymnasium|pw}}{\lemma{\textnormal{\emph{Schottengymnasium}}}\Cendnote{\textnormal{Johann Schnitzler\pwindex{Schnitzler, Johann 10.04.1835 – 02.05.1893@\textsc{Schnitzler, Johann} (10.04.1835 – 02.05.1893), \emph{Laryngologe/Laryngologin}|pwk} kam erst zum Studium nach
                     Wien\oindex{Wien@\textbf{Wien}, \emph{A.ADM2}|pwk}.}}}\label{K_L02572-3} her oder mindeſtens vom erſten
               Jahre Medizin. Ich habe oft nach Briefen gesucht, aber nichts gefunden – nur diese
               Karte fand ich einmal und schicke sie Ihnen. Trotz {\pb}des
               belanglosen Inhalts grüßt Sie vielleicht eine liebe und vertraute Schrift! –\pend
           
\pstart
           Bitte, lächeln Sie nicht über diesen langen Brief als Antwort auf Ihre Karte – Herr v. Sala\pwindex{einsame Weg. Schauspiel in fuenf Akten@\emph{Der einsame Weg. Schauspiel in fünf Akten}|pwv} täte es, sein
               Schöpfer iſt hoffentlich milder – aber ich habe ihn jahrelang »verdrängt«, um mich
               ganz modern auszudrücken, und einmal mußte er doch geschrieben werden. Ihre
               freundlichen Worte sind ein Anlaſs dazu. Möchte Ihnen das silberschi{\geminationm}ernde Dänemark\oindex{Daenemark@\textbf{Dänemark}, \emph{A.PCLI}|pw}
               viel Liebes und Freundliches geben! Seien Sie nochmals bedankt und begrüßt von
               Ihrer\pend
           \pstart \spacefill\mbox{Therese Rie.}\pend{}\selectlanguage{ngerman}\endnumbering\briefempfaengerindex{Schnitzler, Arthur@\textsc{Schnitzler, Arthur}!zzzRie, Therese@\emph{von Therese Rie}!1923-05-031@{3. 5. 1923}|)be}\mylabel{L02572h}  \normalsize

\doendnotes{C}
\bigskip
\vfill

\clearpage

\footnotesize

\lohead{\textsc{register}}

% Definiere theindex-Environment komplett neu ohne reledmac
\makeatletter
\renewenvironment{theindex}{%
  \section*{\indexname}%
  \setlength{\parindent}{0pt}%
  \setlength{\parskip}{0pt plus 0.3pt}%
  \let\item\@idxitem
}{%
  \clearpage
}
\makeatother

\IfFileExists{\jobname-pw.ind}{\input{\jobname-pw.ind}}{}

\end{document}

      