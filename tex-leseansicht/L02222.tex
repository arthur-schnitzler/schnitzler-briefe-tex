%% latex-leseansicht-vorspann.tex
%% Vorspann für die Leseansicht.
%% Lädt die gemeinsame Datei latex-vorspann.tex mit nicht gesetztem Schalter.

\newif\ifkorrekturansicht
\korrekturansichtfalse

\input{../tex-inputs/latex-vorspann}


               \section[Arthur Schnitzler an Georg Brandes, 9. 12. 1915]{ Arthur Schnitzler an Georg Brandes, 9. 12. 1915}\nopagebreak\mylabel{v}\rehead{ }\begin{ledgroupsized}[t]{13cm}\normalsize\beginnumbering\briefempfaengerindex{Brandes, Georg@\textsc{Brandes, Georg}!zzzSchnitzler, Arthur@\emph{von Arthur Schnitzler}!1915-12-091@{9. 12. 1915}|(be} \toendnotes[C]{\smallbreak\pagebreak[2]} \Standort{Kopenhagen, Det Kongelige Bibliotek, Georg Brandes Arkiv, box 125.}
\physDesc{Brief, 4 Blätter (mit Schreibmaschine paginiert »3«, »5« respektive »7«), 7 Seiten
\newline{}Schreibmaschine
\newline{}Handschrift: schwarze Tinte, lateinische Kurrent (\noindent{}Korrekturen, Schlussformel, Unterschrift)\newline{}Ordnung: mit Bleistift von unbekannter Hand auf dem ersten Blatt
                                 nummeriert: »38.«, die weiteren Blätter datiert mit »9/12 15« und auf der Rückseite des letzten Blattes notiert:
                                    »A. Schnitzler« \newline{}Editorischer Hinweis: Die Sofortkorrekturen mit Bleistift sind in der Wiedergabe nicht
                                 ausgewiesen, sehr wohl die Überarbeitung in Lateinschrift mit Tinte
                                 durch Schnitzler. }\buchAbdrucke{\weitereDrucke{1) Georg Brandes, Arthur Schnitzler: \emph{Ein Briefwechsel}. Hg. Kurt Bergel. Bern: \emph{Francke} 1956, S. 116–118.} \weitereDrucke{2) Arthur Schnitzler: \emph{Briefe 1913–1931}. Hg. Peter Michael Braunwarth, Richard Miklin, Susanne Pertlik und Heinrich Schnitzler. Frankfurt am Main: \emph{S. Fischer} 1984, S. 99–102.} }\toendnotes[C]{\smallbreak}\pstart
           \noindent{}{\pb}\textcolor{gray}{\textbf{Dr. Arthur Schnitzler}}{\\}\textcolor{gray}{\textbf{Wien XVIII. Sternwartestrasse 71\oindex{Sternwartestrasse@\textbf{Sternwartestraße}|pw}}}\pend
           \pstart
           \raggedleft{}9. 12. 1915\pend
           \pstart\center{}Lieber und verehrter Freund.\pend\pstart
           Ihr Brief vom 4. d. ist heute schon eingetroffen, was in dieser Zeit
               eine recht geschwinde Reise ist. Ich beeile mich Ihnen den Empfang zu bestätigen und
               Ihnen herzlichst zu danken. Es betrübt mich, dass Sie von Ihrem alten Leiden wieder
               heimgesucht sind, das aber doch wie es scheint, immer milder auftritt, und immer
               weniger die Macht besitzt Sie in Ihrer ausserordentlichen Tätigkeit zu behindern.
               Dass Sie ein Goethe\pwindex{Goethe, Johann Wolfgang von 28.08.1749 – 22.03.1832@\textsc{Goethe, Johann Wolfgang von} (28.08.1749 – 22.03.1832), \emph{Schriftsteller}|pw}-Buch\pwindex{Brandes, Georg 04.02.1842 – 19.02.1927@\textsc{Brandes, Georg} (04.02.1842 – 19.02.1927)!Wolfgang Goethe1915@\strich\emph{Wolfgang Goethe} {[}1915{]}|pw} geschrieben haben, das geht
               hier längst durch alle Blätter, und man wünschte nur, recht bald eine deutsche
               Ausgabe zu besitzen. Wird man lange darauf zu warten haben?\pend
           \pstart
           Auf Ihre Bemerkungen den »Bernhardi\pwindex{Schnitzler, Arthur 15.05.1862 – 21.10.1931@\textsc{Schnitzler, Arthur} (15.05.1862 – 21.10.1931), \emph{Schriftsteller, Mediziner}!Professor Bernhardi. Komoedie in fuenf Akten1912@\strich\emph{Professor Bernhardi. Komödie in fünf Akten} {[}1912{]}|pw}« betreffend,
               müssen Sie mir erlauben mit ein paar Worten zu erwidern, umso mehr als das Stück
               Ihrem Herzen doch ziemlich nahe steht. Meiner Ansicht nach ist es keineswegs
               geschaffen in {\pb}dem Sinne entmutigend zu wirken,
               wie Sie es in Ihrem Briefe ausdrücken. Was Sie sagen kann sich überhaupt nur auf die
               Schlussszene des Stücks beziehen und da weise ich vor allem darauf hin, dass der
               Autor in keiner Weise für die Aussprüche des Hofrats verantwortlich gemacht zu werden
               wünscht. Ich bin mit dem Hofrat nicht identisch, ja, mit einem leichten Paradox
               könnte man behaupten, dass der Hofrat es nicht einmal mit sich selber ist. Sie
               erinnern sich ja, dass Bernhardi\pwindex{Schnitzler, Arthur 15.05.1862 – 21.10.1931@\textsc{Schnitzler, Arthur} (15.05.1862 – 21.10.1931), \emph{Schriftsteller, Mediziner}!Professor Bernhardi. Komoedie in fuenf Akten1912@\strich\emph{Professor Bernhardi. Komödie in fünf Akten} {[}1912{]}|pwv}
               dem Hofrat auf seine, wenn Sie wollen, skeptisch-ironischen Vorhalte erwidert \introOben{}»\introOben{}\label{K_L02222_1v}\edtext{Sie hätten an meiner Statt
                  gerad{[}e{]}so gehandelt wie ich}{\lemma{\textnormal{\emph{Sie … ich}}}\Cendnote{\textnormal{Kein wörtliches Zitat des vorletzten Satzes des Stücks. Dieser
                  lautet in der gedruckten Fassung: »Sie in meinem Fall hätten genauso
                     gehandelt.« (\emph{Professor Bernhardi}\pwindex{Schnitzler, Arthur 15.05.1862 – 21.10.1931@\textsc{Schnitzler, Arthur} (15.05.1862 – 21.10.1931), \emph{Schriftsteller, Mediziner}!Professor Bernhardi. Komoedie in fuenf Akten1912@\strich\emph{Professor Bernhardi. Komödie in fünf Akten} {[}1912{]}|pwk}. Komödie in fünf Akten
                     von Arthur Schnitzler\pwindex{Schnitzler, Arthur 15.05.1862 – 21.10.1931@\textsc{Schnitzler, Arthur} (15.05.1862 – 21.10.1931), \emph{Schriftsteller, Mediziner}|pwk}. Berlin: \emph{S. Fischer}\orgindex{S. Fischer Verlag@S. Fischer Verlag|pwk} 1912,
                  S. 255.)}}}\label{K_L02222_1h}\introOben{}«\introOben{}; worauf der Hofrat zur Antwort gibt: »\label{K_L02222_2v}\edtext{Da wär ich halt grad so ein Viech gewesen
               wie Sie.}{\lemma{\textnormal{\emph{Da … Sie.}}}\Cendnote{\textnormal{In der Buchausgabe:
                     »Möglich. – Da wär ich halt – entschuldigen schon, Herr Professor, –
                     grad’ so ein Viech gewesen wie Sie.«}}}\label{K_L02222_2h}« Aber er hätte so gehandelt\substVorne{}\textsuperscript{.}\substDazwischen{}!\substHinten{}{ }Bernhardi\pwindex{Schnitzler, Arthur 15.05.1862 – 21.10.1931@\textsc{Schnitzler, Arthur} (15.05.1862 – 21.10.1931), \emph{Schriftsteller, Mediziner}!Professor Bernhardi. Komoedie in fuenf Akten1912@\strich\emph{Professor Bernhardi. Komödie in fünf Akten} {[}1912{]}|pwv} hätte in einem zweiten
               solchen Falle auch wieder so gehandelt. Und beide hätten sich nicht im Geringsten
               darum gekümmert, dass Andere oder sie selber sie für Viecher gehalten hätten. Und ich
               glaube, dass die \introOben{}Angelegenheiten der\introOben{} Welt von den Bernhardis\pwindex{Schnitzler, Arthur 15.05.1862 – 21.10.1931@\textsc{Schnitzler, Arthur} (15.05.1862 – 21.10.1931), \emph{Schriftsteller, Mediziner}!Professor Bernhardi. Komoedie in fuenf Akten1912@\strich\emph{Professor Bernhardi. Komödie in fünf Akten} {[}1912{]}|pwv},{\pb} ja sogar von den Hofräten in der Art dieses Hofrat Winkler\pwindex{Schnitzler, Arthur 15.05.1862 – 21.10.1931@\textsc{Schnitzler, Arthur} (15.05.1862 – 21.10.1931), \emph{Schriftsteller, Mediziner}!Professor Bernhardi. Komoedie in fuenf Akten1912@\strich\emph{Professor Bernhardi. Komödie in fünf Akten} {[}1912{]}|pwv} erheblicher
               gefördert werden, als von den Pflugfelders\pwindex{Schnitzler, Arthur 15.05.1862 – 21.10.1931@\textsc{Schnitzler, Arthur} (15.05.1862 – 21.10.1931), \emph{Schriftsteller, Mediziner}!Professor Bernhardi. Komoedie in fuenf Akten1912@\strich\emph{Professor Bernhardi. Komödie in fünf Akten} {[}1912{]}|pwv}, von \introOben{}den\introOben{} Gerechten mehr als von
                  {[}den{]} Rechthaberischen, von den Zweiflern mehr als von den
               Dogmatikern aller Parteien\introOben{};\introOben{} und je älter ich werde, umso
               vernehmlicher pfeife ich auf diejenigen Leute\strikeout{n}, die a
               priori mit sich selber einverstanden sind; und wenn mich nicht alles trügt, so blasen
               auch Sie, mein verehrter Freund, nicht ungern \substVorne{}\textsuperscript{diese}\substDazwischen{}die gleiche\substHinten{} Melodie \introOben{}mit mir\introOben{}. – Im übrigen ist ja der »Bernhardi\pwindex{Schnitzler, Arthur 15.05.1862 – 21.10.1931@\textsc{Schnitzler, Arthur} (15.05.1862 – 21.10.1931), \emph{Schriftsteller, Mediziner}!Professor Bernhardi. Komoedie in fuenf Akten1912@\strich\emph{Professor Bernhardi. Komödie in fünf Akten} {[}1912{]}|pw}« kein Tendenzstück und will es nicht sein,
               weder im Besonderen noch im Allgemeinen; – soll überhaupt kategorisiert werden, so
               möchte ich ihn am liebsten als Charakterkomödie angesehen wissen, und dass gerade
               dieses Stück auch in Ländern \substVorne{}\textsuperscript{die}\substDazwischen{}seine\substHinten{} Wirkung nicht versagt hat, wo von vornherein für spezifisch österreichische\oindex{Oesterreich@\textbf{Österreich}|pw} Verhältnisse kein besonderes Interesse regsam
               sein dürfte, scheint mir dafür zu sprechen, dass die Gestalten an sich das Publikum
               zu interessieren vermochten.\pend
           \pstart
           {\pb}Dass Ihnen die »Komödie der Worte\pwindex{Schnitzler, Arthur 15.05.1862 – 21.10.1931@\textsc{Schnitzler, Arthur} (15.05.1862 – 21.10.1931), \emph{Schriftsteller, Mediziner}!Komoedie der Worte. Drei Einakter1915@\strich\emph{Komödie der Worte. Drei Einakter} {[}1915{]}|pw}« einiges Vergnügen bereitet hat, freut mich sehr. Die
               Einakter werden viel gespielt und haben einen ansehnlichen Bühnenerfolg gehabt.
               Dagegen werde ich von einem gewissen Teil der Kritik in einer selbst nach meinen
               nicht unbedeutenden Erfahrungen auf diesem Gebiet fast emphatisch zu nennenden Weise
               angegriffen. Man hat nämlich bei uns (in Deutschland\oindex{Deutschland@\textbf{Deutschland}|pw}
               und Oesterreich\oindex{Oesterreich@\textbf{Österreich}|pw}) ein neues kritisches Mass für
               Kunstwerke entdeckt, \strikeout{nämlich} den Weltkrieg. Und wie
               es den Herren gerade passt, wird man dafür zur Rechenschaft gezogen, dass das
               betreffende Werk irgendwie an den Krieg erinnert oder dass es das nicht tut.
               Anlässlich des »Medardus\pwindex{Schnitzler, Arthur 15.05.1862 – 21.10.1931@\textsc{Schnitzler, Arthur} (15.05.1862 – 21.10.1931), \emph{Schriftsteller, Mediziner}!junge Medardus. Dramatische Historie in einem Vorspiel und fuenf Aufzuegen1910-10-26@\strich\emph{Der junge Medardus. Dramatische Historie in einem Vorspiel und fünf Aufzügen} {[}1910-10-26{]}|pw}«, der im vorigen Herbst in
                  Berlin\oindex{Berlin@\textbf{Berlin}|pw} aufgeführt wurde, wurde es mir sehr
               verübelt, dass mein Held sich nicht sofort seinem ursprünglichen Entschluss gemäss,
               aufmacht, um den Napoleon\pwindex{Bonaparte, Napoleon 15.08.1769 – 21.05.1821@\textsc{Bonaparte, Napoleon} (15.08.1769 – 21.05.1821), \emph{Kaiser}|pw} umzubringen, und sich
               statt dessen fünf Akte lang durch allerhand Privaterlebnisse, die für Kritiker
               selbstverständlich {\pb}nicht existieren, von der
               Ausführung seiner vaterländischen Absicht abhalten lässt. Die »Komödie der Worte\pwindex{Schnitzler, Arthur 15.05.1862 – 21.10.1931@\textsc{Schnitzler, Arthur} (15.05.1862 – 21.10.1931), \emph{Schriftsteller, Mediziner}!Komoedie der Worte. Drei Einakter1915@\strich\emph{Komödie der Worte. Drei Einakter} {[}1915{]}|pw}« hinwiederum hat das Sittlichkeitsgefühl dieser
               Herren aufs Tiefste beleidigt. Dass unter Sittlichkeit nach wie vor nicht etwa
               Wahrheit oder sonst etwas Vernünftiges oder Positives, sondern ausschliesslich
               Unterdrückung des Geschlechtstrieb\introOben{}e\introOben{}s verstanden wird,
               brauche ich Ihnen nicht erst zu erzählen. Und dass ich in dieser grossen Zeit, wo
               sämmtliche Männer für das Vaterland fechten, (ausser denen, die zuhause sitzen und
               Theaterreferate schreiben) und sämmtliche Frauen trauern oder klagen, nicht nur an
               Opfermut, sondern auch an Treue das Ungeheuerste leisten, (abgesehen von denen, die
               es nicht tun) »so erbärmliche Wichte« auf die Bühne zu stellen wage, das hat
               besonders gesinnungstüchtige Leute (in der Kölnischen
                  Zeitung\orgindex{Koelnische Zeitung@Kölnische Zeitung|pw}, und viele andere Zeitungen haben es gerne nachgedruckt) zu der
               kühnen Frage veranlasst: »Ob nicht
                  gerade jene letzten Dokumente eines Wien\oindex{Wien@\textbf{Wien}|pw}er
                  Literatentums (Schön{\pb}herrs\pwindex{Schoenherr, Karl 24.02.1867 – 15.03.1943@\textsc{Schönherr, Karl} (24.02.1867 – 15.03.1943), \emph{Schriftsteller, Mediziner}|pw} ›Weibsteufel\pwindex{Schoenherr, Karl 24.02.1867 – 15.03.1943@\textsc{Schönherr, Karl} (24.02.1867 – 15.03.1943), \emph{Schriftsteller, Mediziner}!Weibsteufel1914@\strich\emph{Der Weibsteufel} {[}1914{]}|pw}‹ und Bahrs\pwindex{Bahr, Hermann 19.07.1863 – 15.01.1934@\textsc{Bahr, Hermann} (19.07.1863 – 15.01.1934), \emph{Schriftsteller, Kritiker}|pw} ›Querulant\pwindex{Bahr, Hermann 19.07.1863 – 15.01.1934@\textsc{Bahr, Hermann} (19.07.1863 – 15.01.1934), \emph{Schriftsteller, Kritiker}!Querulant1914@\strich\emph{Der Querulant} {[}1914{]}|pw}‹ waren nämlich miteinbezogen) Beweis
                  dafür seien, dass unser trefflicher Bundesbruder in diesem Weltkrieg auch einer
                  inneren Reformation an Haupt und Gliedern bedarf, um fortan in einer neuen
                  deutschen Weltkultur bestehen zu können.\pwindex{Hermann Bahrs Querulant17. 11. 1915@\emph{Hermann Bahrs Querulant} {[}17. 11. 1915{]}|pwv}«\pend
           \pstart
           Aber auch abgesehen von diesen kleinen und etwas lächerlichen Erfahrungen kann man
               vielleicht finden, dass die Zeit nun eben gross genug geworden ist, und ein weiteres
               Wachstum von Uebel wäre. Ueber die militärischen und politischen Verhältnisse sind
               Sie ja wohl in Dänemark\oindex{Daenemark@\textbf{Dänemark}|pw} heute besser orientiert,
               als Sie es zu Anfang des Krieges gewesen \introOben{}sein\introOben{} dürften.
               Zusammengefasst kann man freilich nur sagen, dass die gemeinsame Sache der
               Zentralmächte so gut steht als möglich und dass ein Ende doch noch nicht abzusehen
               ist. Ihrem Schwiegersohn\pwindex{Philipp, Reinhold 15.08.1883 – 1968@\textsc{Philipp, Reinhold} (15.08.1883 – 1968), \emph{Fabrikant}|pwv} geht es
               hoffentlich weiterhin gut. Auch von uns stehen Verwandte und Freunde im Feld oder
               sind anderweitig durch die Kriegsverhältnisse in Mitleidenschaft gezogen; auch den
               Tod manches lieben {\pb}Bekannten haben wir zu
               beklagen. Im Einzelnen über all dies weiter zu reden müsste ins Grenzenlose führen.
               Ist es schon in ruhigeren Zeiten etwas verwegen, im Dezember vom nächsten Sommer zu
               sprechen, so erscheint es jetzt beinahe verrückt. Trotzdem möchte ich diesen Brief
               nicht gerne schliessen, ohne der Hoffnung einer baldigen Wiederbegegnung mit Ihnen
               Ausdruck zu geben, und jedenfalls wäre es sehr liebenswürdig von Ihnen uns ab und zu
               durch eine Zeile von Ihrem Befinden, von Ihrem Wohlbefinden zu benachrichtigen.
               Wollen Sie in meinem Namen auch Peter Nansen\pwindex{Nansen, Peter 20.01.1861 – 31.07.1918@\textsc{Nansen, Peter} (20.01.1861 – 31.07.1918), \emph{Schriftsteller, Journalist, Verleger}|pw} die
               besten Wünsche für seine baldige Genesung bestellen; seine neue Novelle\pwindex{Nansen, Peter 20.01.1861 – 31.07.1918@\textsc{Nansen, Peter} (20.01.1861 – 31.07.1918), \emph{Schriftsteller, Journalist, Verleger}!Brueder Menthe1915@\strich\emph{Die Brüder Menthe} {[}1915{]}|pwv} wird man wohl auch bald in deutscher
               Sprache zu lesen bekommen. In den vielen Jahren, da er leider schwieg, hat man ihn
               hier keineswegs vergessen\strikeout{,} und wird sich seiner neu
               erwachenden Produktionskraft aufrichtig freuen.\pend
           \pstart
           Und nun leben Sie wohl, und seien Sie, auch im Namen meiner Frau\pwindex{Schnitzler, Olga 17.01.1882 – 13.01.1970@\textsc{Schnitzler, Olga} (17.01.1882 – 13.01.1970), \emph{Schauspielerin, Sängerin}|pwv}, aufs Allerherzlichste grüsst.\pend
           \pstart
           {[}hs.:{]} Ihr treu ergebner{\\[\baselineskip]}\spacefill\mbox{Arthur Schnitzler}\pend
           \leftskip=0em{}          \endnumbering\briefempfaengerindex{Brandes, Georg@\textsc{Brandes, Georg}!zzzSchnitzler, Arthur@\emph{von Arthur Schnitzler}!1915-12-091@{9. 12. 1915}|)be}\mylabel{h}\end{ledgroupsized}  \newcommand{\dateiname}{L02222}\newcommand{\titel}{Arthur Schnitzler an Georg Brandes, 9. 12. 1915}\newcommand{\editorInnen}{ Martin Anton Müller und Gerd-Hermann Susen}
            \footnotesize
\begin{ledgroupsized}[t]{11.5cm}
\doendnotes{C}
\end{ledgroupsized}
         %% latex-leseansicht-abspann.tex
%% Abspann für die Leseansicht.
%% Der Schalter \ifkorrekturansicht ist bereits durch den Vorspann gesetzt.

%% latex-abspann.tex
%% Gemeinsamer Abspann für Korrekturansicht und Leseansicht.
%% Setzt den Schalter \ifkorrekturansicht voraus (gesetzt in den
%% einbindenden Dateien latex-korrekturansicht-abspann.tex bzw.
%% latex-leseansicht-abspann.tex).
%% ---------------------------------------------------------------

\normalsize

% Das esempio-Environment wird nur in der Leseansicht benötigt
\ifkorrekturansicht\else
\newenvironment{esempio}[3]%
{
    \vspace{1.5ex}
    \rlap{\underline{#1}}
    \par
    \setlength{\parindent}{0cm}
    \nopagebreak
    \leftskip=#2cm
    \rightskip=#3cm
}
{
    \par
}
\fi

\doendnotes{C}
\bigskip
\vfill

\clearpage

\footnotesize

\ifkorrekturansicht
  \lohead{\textsc{register}}
\fi

% theindex-Environment neu definieren ohne reledmac
\makeatletter
\renewenvironment{theindex}{%
  \ifkorrekturansicht
    \section*{\indexname}%
  \else
    \subsubsection*{Index der erwähnten Entitäten}%
  \fi
  \setlength{\parindent}{0pt}%
  \setlength{\parskip}{0pt plus 0.3pt}%
  \let\item\@idxitem
}{%
  \ifkorrekturansicht\clearpage\fi
}
\makeatother

\IfFileExists{\jobname-pw.ind}{\input{\jobname-pw.ind}}{}

% Quellenangabe nur in der Leseansicht
\ifkorrekturansicht\else
% Fallback-Definitionen, falls die .tex-Datei \titel etc. nicht gesetzt hat
\providecommand{\titel}{}
\providecommand{\editorInnen}{}
\providecommand{\dateiname}{\jobname}

\vspace{3cm}

\vfill

\footnotesize
\textsc{Quelle}: \titel. Herausgegeben von {\editorInnen}. In: \emph{Arthur Schnitzler: Briefwechsel mit Autorinnen und Autoren}.
 Digitale Edition, https://schnitzler-briefe.acdh.oeaw.ac.at/{\dateiname}.html (Stand \today)
\fi

\end{document}


      