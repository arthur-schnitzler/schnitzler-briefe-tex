%% latex-korrekturansicht-vorspann.tex
%% Vorspann für die Korrekturansicht.
%% Lädt die gemeinsame Datei latex-vorspann.tex mit gesetztem Schalter.

\newif\ifkorrekturansicht
\korrekturansichttrue

\input{../tex-inputs/latex-vorspann}


\section[Arthur Schnitzler an Richard Beer-Hofmann, 13. 8. 1898]{L00834 Arthur Schnitzler an Richard Beer-Hofmann, 13. 8. 1898}
\nopagebreak\mylabel{L00834v}
\rehead{ }\normalsize\beginnumbering\briefempfaengerindex{Beer-Hofmann, Richard@\textsc{Beer-Hofmann, Richard}!zzzSchnitzler, Arthur@\emph{von Arthur Schnitzler}!1898-08-131@{13. 8. 1898}|(be}
\toendnotes[C]{\smallbreak\pagebreak[2]}\Standort{YCGL, MSS 31.}
\physDesc{Postkarte, 381 Zeichen
\newline{}Handschrift: Bleistift, deutsche Kurrent
\newline{}Versand: 1) Stempel: »\nobreak{}\oindex{Biel@\textbf{Biel}, \emph{P.PPLA2}|pwk}Bienne, 13. VIII. 98, 5\nobreak{}«.   2) Stempel: »\nobreak{}\oindex{Steindorf am Ossiacher See@\textbf{Steindorf am Ossiacher See}, \emph{A.ADM3}|pwk}Steindorf am Ossiacher
                                       See, 15 8 98\nobreak{}«. 
\newline{}Ordnung: mit Bleistift von unbekannter Hand datiert: »13. 8.« }
\buchAbdrucke{\weitereDrucke{Arthur Schnitzler, Richard Beer-Hofmann: \emph{Briefwechsel 1891–1931}. Wien, Zürich: \emph{Europaverlag} 1992, S. 124.} }\toendnotes[C]{\smallbreak}\pstart{}{\pb}\textsc{Dr. Richard Beer-Hofmann}\pend{}\pstart{}\textsc{Steindorf\oindex{Steindorf am Ossiacher See@\textbf{Steindorf am Ossiacher See}, \emph{A.ADM3}|pw}}\pend{}\pstart{}\textsc{am}{ }\textsc{Ossiacher}ſee\oindex{Ossiacher See@\textbf{Ossiacher See}, \emph{See (N.SEE)}|pw}.\pend{}\pstart{}\textsc{Kärnthen}\oindex{Kaernten@\textbf{Kärnten}, \emph{A.ADM1}|pw}.\pend{}{\bigskip}\vspace{1em}
\pstart
           \noindent{}{\pb}Unſer lieber Richard, wir denken (ſagt Hugo\pwindex{Hofmannsthal, Hugo von 1874-02-01 – 1929-07-15@\textsc{Hofmannsthal, Hugo von} (1874-02-01 – 1929-07-15), \emph{Schriftsteller/Schriftstellerin}|pw}) oft an Sie (ſage ich) – ſchreiben Sie uns gleich (ſage
               ich) \textsc{Genf}\oindex{Genf@\textbf{Genf}, \emph{P.PPLA}|pw}{ }\textsc{post rest} (ſagt Hugo\pwindex{Hofmannsthal, Hugo von 1874-02-01 – 1929-07-15@\textsc{Hofmannsthal, Hugo von} (1874-02-01 – 1929-07-15), \emph{Schriftsteller/Schriftstellerin}|pw}), wo wir \label{K_L00834-1v}\edtext{Mittwoch}{\lemma{\textnormal{\emph{Mittwoch}}}\Cendnote{\textnormal{Vgl. A. S.: \emph{Tagebuch}, 17. 8. 1898.
               }}}\label{K_L00834-1} ſind. Ich möchte irgendwo am Genferſee\oindex{Genfer See@\textbf{Genfer See}, \emph{H.LK}|pw}
               bleiben, Hugo\pwindex{Hofmannsthal, Hugo von 1874-02-01 – 1929-07-15@\textsc{Hofmannsthal, Hugo von} (1874-02-01 – 1929-07-15), \emph{Schriftsteller/Schriftstellerin}|pw} geht wahrſcheinlich nach Lugano\oindex{Lugano@\textbf{Lugano}, \emph{P.PPLA2}|pw}, doch ist es möglich,
                  {[}d{]}ſs wir beide \introOben{}eine Zeit lang\introOben{} zuſa{\geminationm}en bleiben, hier oder dort. Von Herzen Ihr
                  \spacefill\mbox{Arthur}\pend
           \selectlanguage{ngerman}\endnumbering\briefempfaengerindex{Beer-Hofmann, Richard@\textsc{Beer-Hofmann, Richard}!zzzSchnitzler, Arthur@\emph{von Arthur Schnitzler}!1898-08-131@{13. 8. 1898}|)be}\mylabel{L00834h}  \normalsize

\doendnotes{C}
\bigskip
\vfill

\clearpage

\footnotesize

\lohead{\textsc{register}}

% Definiere theindex-Environment komplett neu ohne reledmac
\makeatletter
\renewenvironment{theindex}{%
  \section*{\indexname}%
  \setlength{\parindent}{0pt}%
  \setlength{\parskip}{0pt plus 0.3pt}%
  \let\item\@idxitem
}{%
  \clearpage
}
\makeatother

\IfFileExists{\jobname-pw.ind}{\input{\jobname-pw.ind}}{}

\end{document}

      