%% latex-korrekturansicht-vorspann.tex
%% Vorspann für die Korrekturansicht.
%% Lädt die gemeinsame Datei latex-vorspann.tex mit gesetztem Schalter.

\newif\ifkorrekturansicht
\korrekturansichttrue

\input{../tex-inputs/latex-vorspann}


\section[Arthur Schnitzler an Richard Beer-Hofmann, 10. 8. 1901]{L01158 Arthur Schnitzler an Richard Beer-Hofmann, 10. 8. 1901}
\nopagebreak\mylabel{L01158v}
\rehead{ }\normalsize\beginnumbering\briefempfaengerindex{Beer-Hofmann, Richard@\textsc{Beer-Hofmann, Richard}!zzzSchnitzler, Arthur@\emph{von Arthur Schnitzler}!1901-08-101@{10. 8. 1901}|(be}
\toendnotes[C]{\smallbreak\pagebreak[2]}\Standort{YCGL, MSS 31.}
\physDesc{Brief, 1 Blatt, 3 Seiten, Umschlag, 760 Zeichen
\newline{}Handschrift: 1) Bleistift, deutsche Kurrent\hspace{1em}2) schwarze Tinte, deutsche Kurrent (\noindent{}Umschlag)\hspace{1em}
\newline{}Versand: 1) Stempel: »\nobreak{}\oindex{Vahrn@\textbf{Vahrn}, \emph{P.PPLA3}|pwk}Vahr\textcolor{gray}{n}, 10. 8. \textcolor{gray}{01}\nobreak{}«.   2) Stempel: »\nobreak{}\oindex{Poertschach am Woerthersee@\textbf{Pörtschach am Wörthersee}, \emph{P.PPL}|pwk}{\pb}Pörtschach am See, 11 8 01\nobreak{}«. 
\newline{}Ordnung: mit Bleistift von unbekannter Hand datiert: »10. 8.« }
\buchAbdrucke{\weitereDrucke{Arthur Schnitzler, Richard Beer-Hofmann: \emph{Briefwechsel 1891–1931}. Wien, Zürich: \emph{Europaverlag} 1992, S. 154.} }\toendnotes[C]{\smallbreak}\pstart{}{\pb}Herrn \textsc{Dr. Richard
                     Beer-Hofmann}\pend{}\pstart{}\textsc{Pörtschach\oindex{Poertschach am Woerthersee@\textbf{Pörtschach am Wörthersee}, \emph{P.PPL}|pw}}\pend{}\pstart{}am \textsc{Wörther}ſee\oindex{Woerthersee@\textbf{Wörthersee}, \emph{H.LK}|pw}\pend{}\pstart{}\textsc{Villa Arnstein\oindex{Villa Arnstein@\textbf{Villa Arnstein}, \emph{Wohngebäude (K.WHS)}|pw}.}\pend{}{\bigskip}\vspace{1em}
\pstart
           \noindent{}{\pb}mein lieber Richard, am Montag fahren wir
               nach Bozen\oindex{Bozen@\textbf{Bozen}, \emph{P.PPLA2}|pw}, wo Rendez vous mit Paul, den ich
               neulich in \textsc{Welsberg\oindex{Welsberg-Taisten@\textbf{Welsberg-Taisten}, \emph{A.ADM3}|pw}} ſprach. Da{\geminationn}{ }Trient\oindex{Trient@\textbf{Trient}, \emph{P.PPLA}|pw}. Freitag wollen wir in \textsc{Welsberg\oindex{Welsberg-Taisten@\textbf{Welsberg-Taisten}, \emph{A.ADM3}|pw}} (Puſterthal\oindex{Pustertal@\textbf{Pustertal}, \emph{T.VAL}|pw}) ſein, haben dort Zimmer
               genommen (\textsc{Pens. Waldbrunn\oindex{Wildbad Waldbrunn@\textbf{Wildbad Waldbrunn}, \emph{S.SPA}|pw}}, entzückend gelegen, 1150 \textsc{meter}, 20 Minuten von der
               Bahn) wollen dort die letzten 14 Tage {\dots} ich mei{\pb}ne die letzten 14 Tage vor Wien\oindex{Wien@\textbf{Wien}, \emph{A.ADM2}|pw} (kümmert man ſich dort oben um meine »Meinung?«)
               verbringen. Es wäre wirklich hübſch von Ihnen, we{\geminationn} Sie
               auch dorthin kämen. Ich will auch arbeiten. Nebſtbei haben Sie’s ſo nah. – \pend
           
\pstart
           We{\geminationn} Sie mir \uline{gleich}
               antworten, bitte Trient \textsc{postrest} (thun Sie das!).\pend
           
\pstart
           {\pb}Leben Sie wohl und grüßen Sie Ihre Frau\pwindex{Beer-Hofmann, Paula 25.02.1879 – 30.10.1939@\textsc{Beer-Hofmann, Paula} (25.02.1879 – 30.10.1939)|pwv} und Ihre Kinder\pwindex{Beer-Hofmann, Gabriel 09.01.1901 – 24.03.1971@\textsc{Beer-Hofmann, Gabriel} (09.01.1901 – 24.03.1971), \emph{Schriftsteller/Schriftstellerin, Filmagent/Filmagentin}|pwv}\pwindex{Beer-Hofmann, Gabriel 09.01.1901 – 24.03.1971@\textsc{Beer-Hofmann, Gabriel} (09.01.1901 – 24.03.1971), \emph{Schriftsteller/Schriftstellerin, Filmagent/Filmagentin}|pwv}\pwindex{Beer-Hofmann, Mirjam 04.09.1897 – 24.12.1984@\textsc{Beer-Hofmann, Mirjam} (04.09.1897 – 24.12.1984)|pwv}, ſoweit sie es
               verſtehen.\pend
           
\pstart
           Herzlichſt Ihr{\\[\baselineskip]}\spacefill\mbox{Arth.}\pend
           \leftskip=0em{}
\pstart
           \textsc{Vahrn}\oindex{Vahrn@\textbf{Vahrn}, \emph{P.PPLA3}|pw}, 10. 8. 901.\pend
           \selectlanguage{ngerman}\endnumbering\briefempfaengerindex{Beer-Hofmann, Richard@\textsc{Beer-Hofmann, Richard}!zzzSchnitzler, Arthur@\emph{von Arthur Schnitzler}!1901-08-101@{10. 8. 1901}|)be}\mylabel{L01158h}  \normalsize

\doendnotes{C}
\bigskip
\vfill

\clearpage

\footnotesize

\lohead{\textsc{register}}

% Definiere theindex-Environment komplett neu ohne reledmac
\makeatletter
\renewenvironment{theindex}{%
  \section*{\indexname}%
  \setlength{\parindent}{0pt}%
  \setlength{\parskip}{0pt plus 0.3pt}%
  \let\item\@idxitem
}{%
  \clearpage
}
\makeatother

\IfFileExists{\jobname-pw.ind}{\input{\jobname-pw.ind}}{}

\end{document}

      