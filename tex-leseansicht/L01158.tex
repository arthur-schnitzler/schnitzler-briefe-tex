%% latex-leseansicht-vorspann.tex
%% Vorspann für die Leseansicht.
%% Lädt die gemeinsame Datei latex-vorspann.tex mit nicht gesetztem Schalter.

\newif\ifkorrekturansicht
\korrekturansichtfalse

\input{../tex-inputs/latex-vorspann}


         
         \newcommand{\erwaehntePersonen}{Personen: Richard Beer-Hofmann, Paula Beer-Hofmann, Gabriel Beer-Hofmann, Mirjam Beer-Hofmann}
         \newcommand{\erwaehnteInstitutionen}{}
         \newcommand{\erwaehnteOrte}{Orte: Bozen, Pustertal, Pörtschach, Trient, Vahrn, Villa Arnstein, Welsberg-Taisten, Wien, Wildbad Waldbrunn, Wörthersee}
         \newcommand{\erwaehnteWerke}{
               \section[Arthur Schnitzler an Richard Beer-Hofmann, 10. 8. 1901]{ Arthur Schnitzler an Richard Beer-Hofmann, 10. 8. 1901}\nopagebreak\mylabel{v}\rehead{ }\begin{ledgroupsized}[t]{13cm}\normalsize\beginnumbering \toendnotes[C]{\smallbreak\pagebreak[2]} \Standort{YCGL, MSS 31.}
\physDesc{Brief, 1 Blatt, 3 Seiten, Umschlag
\newline{}Handschrift: 1) Bleistift, deutsche Kurrent\hspace{1em}2) schwarze Tinte, deutsche Kurrent (\noindent{}Umschlag)\hspace{1em}\newline{}Versand: 1) Stempel: »\nobreak{}\oindex{Vahrn@\textbf{Vahrn}|pwk}Vahr\textcolor{gray}{n}, 10. 8. \textcolor{gray}{01}\nobreak{}«.   2) Stempel: »\nobreak{}\oindex{Poertschach@\textbf{Pörtschach}|pwk}{\pb}Pörtschach am
                              See, 11 8 01\nobreak{}«. \newline{}Ordnung: mit Bleistift von unbekannter Hand datiert:
                                 »10. 8.« }\buchAbdrucke{\weitereDrucke{Arthur Schnitzler, Richard Beer-Hofmann: \emph{Briefwechsel 1891–1931}. Hg. Konstanze Fliedl. Wien, Zürich: \emph{Europaverlag} 1992, S. 154.} }\toendnotes[C]{\smallbreak}\pstart{}{\pb}Herrn \textsc{Dr. Richard
                     Beer-Hofmann}\pend{}\pstart{}\textsc{Pörtschach\oindex{Poertschach@\textbf{Pörtschach}|pw}}\pend{}\pstart{}am \textsc{Wörther}ſee\oindex{Woerthersee@\textbf{Wörthersee}|pw}\pend{}\pstart{}\textsc{Villa Arnstein\oindex{Villa Arnstein@\textbf{Villa Arnstein}|pw}.}\pend{}{\bigskip}\pstart
           \noindent{}{\pb}mein lieber Richard, am Montag fahren wir
               nach Bozen\oindex{Bozen@\textbf{Bozen}|pw}, wo Rendez vous mit Paul, den ich
               neulich in \textsc{Welsberg\oindex{Welsberg-Taisten@\textbf{Welsberg-Taisten}|pw}} ſprach. Da{\geminationn}{ }Trient\oindex{Trient@\textbf{Trient}|pw}. Freitag wollen wir in \textsc{Welsberg\oindex{Welsberg-Taisten@\textbf{Welsberg-Taisten}|pw}} (Puſterthal\oindex{Pustertal@\textbf{Pustertal}|pw}) ſein, haben dort Zimmer genommen
                  (\textsc{Pens. Waldbrunn\oindex{Wildbad Waldbrunn@\textbf{Wildbad Waldbrunn}|pw}}, entzückend gelegen, 1150 \textsc{meter}, 20 Minuten von der
               Bahn) wollen dort die letzten 14 Tage {\dots} ich mei{\pb}ne die letzten 14 Tage vor Wien\oindex{Wien@\textbf{Wien}|pw} (kümmert man ſich dort oben um meine »Meinung?«) verbringen. Es wäre
               wirklich hübſch von Ihnen, we{\geminationn} Sie auch dorthin kämen.
               Ich will auch arbeiten. Nebſtbei haben Sie’s ſo nah. – \pend
           \pstart
           We{\geminationn} Sie mir \uline{gleich}
               antworten, bitte Trient \textsc{postrest} (thun Sie das!).\pend
           \pstart
           {\pb}Leben Sie wohl und grüßen Sie Ihre Frau\pwindex{Beer-Hofmann, Paula 25.02.1879 – 30.10.1939@\textsc{Beer-Hofmann, Paula} (25.02.1879 – 30.10.1939)|pwv} und Ihre
               Kinder\pwindex{Beer-Hofmann, Gabriel 09.01.1901 – 24.03.1971@\textsc{Beer-Hofmann, Gabriel} (09.01.1901 – 24.03.1971), \emph{Schriftsteller, Filmagent}|pwv}\pwindex{Beer-Hofmann, Gabriel 09.01.1901 – 24.03.1971@\textsc{Beer-Hofmann, Gabriel} (09.01.1901 – 24.03.1971), \emph{Schriftsteller, Filmagent}|pwv}\pwindex{Beer-Hofmann, Mirjam 04.09.1897 – 24.12.1984@\textsc{Beer-Hofmann, Mirjam} (04.09.1897 – 24.12.1984)|pwv}, ſoweit sie es verſtehen.\pend
           \pstart
           Herzlichſt Ihr{\\[\baselineskip]}\spacefill\mbox{Arth.}\pend
           \leftskip=0em{}\pstart
           \textsc{Vahrn}\oindex{Vahrn@\textbf{Vahrn}|pw}, 10. 8. 901.\pend
           
         
         \endnumbering\mylabel{h}\end{ledgroupsized}  \newcommand{\dateiname}{L01158}\newcommand{\titel}{Arthur Schnitzler an Richard Beer-Hofmann, 10. 8. 1901}\newcommand{\editorInnen}{Martin Anton Müller und Gerd-Hermann Susen}%% latex-leseansicht-abspann.tex
%% Abspann für die Leseansicht.
%% Der Schalter \ifkorrekturansicht ist bereits durch den Vorspann gesetzt.

%% latex-abspann.tex
%% Gemeinsamer Abspann für Korrekturansicht und Leseansicht.
%% Setzt den Schalter \ifkorrekturansicht voraus (gesetzt in den
%% einbindenden Dateien latex-korrekturansicht-abspann.tex bzw.
%% latex-leseansicht-abspann.tex).
%% ---------------------------------------------------------------

\normalsize

% Das esempio-Environment wird nur in der Leseansicht benötigt
\ifkorrekturansicht\else
\newenvironment{esempio}[3]%
{
    \vspace{1.5ex}
    \rlap{\underline{#1}}
    \par
    \setlength{\parindent}{0cm}
    \nopagebreak
    \leftskip=#2cm
    \rightskip=#3cm
}
{
    \par
}
\fi

\doendnotes{C}
\bigskip
\vfill

\clearpage

\footnotesize

\ifkorrekturansicht
  \lohead{\textsc{register}}
\fi

% theindex-Environment neu definieren ohne reledmac
\makeatletter
\renewenvironment{theindex}{%
  \ifkorrekturansicht
    \section*{\indexname}%
  \else
    \subsubsection*{Index der erwähnten Entitäten}%
  \fi
  \setlength{\parindent}{0pt}%
  \setlength{\parskip}{0pt plus 0.3pt}%
  \let\item\@idxitem
}{%
  \ifkorrekturansicht\clearpage\fi
}
\makeatother

\IfFileExists{\jobname-pw.ind}{\input{\jobname-pw.ind}}{}

% Quellenangabe nur in der Leseansicht
\ifkorrekturansicht\else
% Fallback-Definitionen, falls die .tex-Datei \titel etc. nicht gesetzt hat
\providecommand{\titel}{}
\providecommand{\editorInnen}{}
\providecommand{\dateiname}{\jobname}

\vspace{3cm}

\vfill

\footnotesize
\textsc{Quelle}: \titel. Herausgegeben von {\editorInnen}. In: \emph{Arthur Schnitzler: Briefwechsel mit Autorinnen und Autoren}.
 Digitale Edition, https://schnitzler-briefe.acdh.oeaw.ac.at/{\dateiname}.html (Stand \today)
\fi

\end{document}


      