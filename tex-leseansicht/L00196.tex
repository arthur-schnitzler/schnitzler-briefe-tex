%% latex-korrekturansicht-vorspann.tex
%% Vorspann für die Korrekturansicht.
%% Lädt die gemeinsame Datei latex-vorspann.tex mit gesetztem Schalter.

\newif\ifkorrekturansicht
\korrekturansichttrue

\input{../tex-inputs/latex-vorspann}


\section[Arthur Schnitzler an Wilhelm Bölsche, 10. 4. 1893]{L00196 Arthur Schnitzler an Wilhelm Bölsche, 10. 4. 1893}
\nopagebreak\mylabel{L00196v}
\rehead{ }\normalsize\beginnumbering\briefempfaengerindex{Boelsche, Wilhelm@\textsc{Bölsche, Wilhelm}!zzzSchnitzler, Arthur@\emph{von Arthur Schnitzler}!1893-04-101@{10. 4. 1893}|(be}
\toendnotes[C]{\smallbreak\pagebreak[2]}\Standort{Wrocław, Biblioteka Uniwersytecka, Böl.Pis 1766.}
\physDesc{Brief, 1 Blatt, 2 Seiten, 452 Zeichen
\newline{}Handschrift: schwarze Tinte, deutsche Kurrent
\newline{}Bölsche: 
                                 als »
                                 
                                    Erl
                                    {[}edigt{]}
                                 « gezeichnet
                               }
\buchAbdrucke{\weitereDrucke{1) \emph{Germanica Wratislaviensia} (1987) Nr. 77, S. 461.} \weitereDrucke{2) Wilhelm Bölsche: \emph{Briefwechsel. Mit Autoren der Freien Bühne}. Berlin: \emph{Weidler} 2010, S. 683.} }\toendnotes[C]{\smallbreak}
\pstart\center{}{\pb}
                  Sehr geehrter Herr,
               \pend\vspace{0.5em}
\pstart
           
               anbei eine 
               Studie\pwindex{Braut@\emph{Die Braut}|pwv}
                für Ihr erg.
                  
               Blatt\pwindex{Freie Buehne fuer den Entwickelungskampf der Zeit@\emph{Freie Bühne für den Entwickelungskampf der Zeit}|pwv}
               . Falls Sie dieſelbe
               drucken wollen, ſo erſuche ich 
               \uline{
                  beſti
                  {\geminationm}
                  t
               }
                um Correcturbogen. – Jedenfalls würden Sie
               mich durch 
               \uuline{baldige}
                Verſtändigung ſehr
               verbinden. –
            \pend
           
\pstart
           
               Ich habe mir erlaubt, der 
               Fr. B.\orgindex{Neue Rundschau, Neue Deutsche Rundschau, Freie Buehne@Neue Rundschau, Neue Deutsche Rundschau, Freie Bühne|pw}
                mein Buch »
               Anatol\pwindex{Anatol@\emph{Anatol}|pw}
               « zu ſenden. Vielleicht wäre es möglich, in
               Ihrer Zeitung ein paar Zeilen 
               {\pb}
               darüber zu bringen? –
            \pend
           
\pstart
           
               Ich bin in beſonderer Hochachtung
               {\\[\baselineskip]}
               Ihr ergebner
               {\\[\baselineskip]}\spacefill\mbox{Dr Arthur Schnitzler}\pend
           \leftskip=0em{}
\pstart
           \noindent{}
                     Wien I. 
                     \textsc{Grillparzerstraße 7}
                     .
                  \oindex{Grillparzerstrasse@\textbf{Grillparzerstraße}, \emph{R.ST}|pw}\pend
           
\pstart
           \textsc{
                     Am 
                     10. April 93}
                  . –
               \pend
           \selectlanguage{ngerman}\endnumbering\briefempfaengerindex{Boelsche, Wilhelm@\textsc{Bölsche, Wilhelm}!zzzSchnitzler, Arthur@\emph{von Arthur Schnitzler}!1893-04-101@{10. 4. 1893}|)be}\mylabel{L00196h}  \normalsize

\doendnotes{C}
\bigskip
\vfill

\clearpage

\footnotesize

\lohead{\textsc{register}}

% Definiere theindex-Environment komplett neu ohne reledmac
\makeatletter
\renewenvironment{theindex}{%
  \section*{\indexname}%
  \setlength{\parindent}{0pt}%
  \setlength{\parskip}{0pt plus 0.3pt}%
  \let\item\@idxitem
}{%
  \clearpage
}
\makeatother

\IfFileExists{\jobname-pw.ind}{\input{\jobname-pw.ind}}{}

\end{document}

      