%% latex-leseansicht-vorspann.tex
%% Vorspann für die Leseansicht.
%% Lädt die gemeinsame Datei latex-vorspann.tex mit nicht gesetztem Schalter.

\newif\ifkorrekturansicht
\korrekturansichtfalse

\input{../tex-inputs/latex-vorspann}


\section[ Felix Salten an Arthur Schnitzler, {[}27. 2. 1897{]}]{L03260 Felix Salten an Arthur Schnitzler,  [27. 2. 1897]}
\nopagebreak\mylabel{L03260v}
\rehead{ }\normalsize\beginnumbering\briefempfaengerindex{Schnitzler, Arthur@\textsc{Schnitzler, Arthur}!zzzSalten, Felix@\emph{von Felix Salten}!1897-02-271@{{[}27. 2. 1897{]}}|(be}
\toendnotes[C]{\smallbreak\pagebreak[2]}
\correspDesc{Versand  durch Felix Salten am [27. 2. 1897] in Wien
\newline{}Erhalt  durch Arthur Schnitzler im Zeitraum [27. 2. 1897
                  – 1. 3. 1897?] in Wien}\toendnotes[C]{\smallbreak}
\Standort{CUL, Schnitzler, B 89, A 2.}
\physDesc{Brief, 1 Blatt, 2 Seiten, 451 Zeichen
\newline{}Handschrift: Bleistift, lateinische Kurrent
\newline{}Beilage: vermutlich von Lotte Glas, 1 Blatt, 1 Seite, schwarze Tinte, lateinische Kurrent. Mit Bleistift von unbekannter
                              Hand nummeriert: »86a« 
\newline{}Schnitzler: mit Bleistift datiert: »27/2 97« 
\newline{}Ordnung: mit Bleistift von unbekannter Hand nummeriert: »86« }\toendnotes[C]{\smallbreak}
\pstart
           \noindent{}{\pb}Lieber Arthur,{ }\label{K_L03260-1v}\edtext{L.\pwindex{Pohl-Glas, Charlotte 1.\,1.\,1873 Wien – 15.\,2.\,1944 Zürich@\textsc{Pohl-Glas, Charlotte} (1.\,1.\,1873 Wien – 15.\,2.\,1944 Zürich), \emph{Schriftstellerin, Politikerin, Sozialistin}|pwu}}{\lemma{\textnormal{\emph{L.}}}\Cendnote{\textnormal{Schnitzlers{ }\emph{Tagebuch}\pwindex{Schnitzler, Arthur 15.\,5.\,1862 Wien – 21.\,10.\,1931 ebd.@\textsc{Schnitzler, Arthur} (15.\,5.\,1862 Wien – 21.\,10.\,1931 ebd.), \emph{Schriftsteller, Mediziner}!Tagebuch@\strich\emph{Tagebuch}|pwk} erwähnt zum 13. 11. 1896, dass er einem Treffen von Charlotte Glas\pwindex{Pohl-Glas, Charlotte 1.\,1.\,1873 Wien – 15.\,2.\,1944 Zürich@\textsc{Pohl-Glas, Charlotte} (1.\,1.\,1873 Wien – 15.\,2.\,1944 Zürich), \emph{Schriftstellerin, Politikerin, Sozialistin}|pwk} und Salten\pwindex{Salten, Felix 6.\,9.\,1869 Budapest – 8.\,10.\,1945 Zürich@\textsc{Salten, Felix} (6.\,9.\,1869 Budapest – 8.\,10.\,1945 Zürich), \emph{Schriftsteller, Journalist, Chefredakteur}|pwk} beigewohnt habe, das der Nachbereitung der Beziehung diente.
                  Womöglich kam es zu einem neuerlichen Kontakt.}}}\label{K_L03260-1} schreibt mir eben wieder.
               Die \label{K_L03260-2v}\edtext{Sache}{\lemma{\textnormal{\emph{Sache}}}\Cendnote{\textnormal{unklar}}}\label{K_L03260-2} ist noch nicht beendet und Sie drängt
               fürchterlich. Ich bitte Sie können Sie mir bis Donnerstag{ }Nachmittag{ }10f {\pb}leihen?
               Sie bekommen Sie \uline{gewiss} zurück, Donnerstag{ }Nachmittag.\pend
           
\pstart
           Herzl {\\[\baselineskip]}\spacefill\mbox{Salten}\pend
           \leftskip=0em{}\selectlanguage{ngerman}\vspace{1em}{\vspace{1\baselineskip}}
\pstart
           \noindent{}{\pb}{[}hs. Pohl-Glas:{]} \label{K_L03260-3v}\edtext{Noch Eines}{\lemma{\textnormal{\emph{Noch Eines}}}\Cendnote{\textnormal{Die Zuordnung der undatierten Beilage zum Brief wird durch 
               die inhaltliche Übereinstimmung (»10fl«) und die archivalische Nummerierung mit »86a« gestützt.}}}\label{K_L03260-3}: ich muß auch der Dame\pwindex{?? [Frau, die Lotte Glas Geld leiht] @\textsc{?? [Frau, die Lotte Glas Geld leiht]}|pwv}, die mir die 10fl. gegeben hat, das
               Geld geben. Sie sagte, es ist ihr Wochengeld, sie müße es haben. Es ist doch sehr
               nett von ihr u. ich würde nicht wagen, ihr unter die Augen zu treten.\pend
           \selectlanguage{ngerman}\endnumbering\briefempfaengerindex{Schnitzler, Arthur@\textsc{Schnitzler, Arthur}!zzzSalten, Felix@\emph{von Felix Salten}!1897-02-271@{{[}27. 2. 1897{]}}|)be}\mylabel{L03260h}  \newcommand{\dateiname}{L03260}\newcommand{\titel}{Felix Salten an Arthur Schnitzler, [27. 2. 1897]}\newcommand{\editorInnen}{Martin Anton Müller und Laura Untner}%% latex-leseansicht-abspann.tex
%% Abspann für die Leseansicht.
%% Der Schalter \ifkorrekturansicht ist bereits durch den Vorspann gesetzt.

%% latex-abspann.tex
%% Gemeinsamer Abspann für Korrekturansicht und Leseansicht.
%% Setzt den Schalter \ifkorrekturansicht voraus (gesetzt in den
%% einbindenden Dateien latex-korrekturansicht-abspann.tex bzw.
%% latex-leseansicht-abspann.tex).
%% ---------------------------------------------------------------

\normalsize

% Das esempio-Environment wird nur in der Leseansicht benötigt
\ifkorrekturansicht\else
\newenvironment{esempio}[3]%
{
    \vspace{1.5ex}
    \rlap{\underline{#1}}
    \par
    \setlength{\parindent}{0cm}
    \nopagebreak
    \leftskip=#2cm
    \rightskip=#3cm
}
{
    \par
}
\fi

\doendnotes{C}
\bigskip
\vfill

\clearpage

\footnotesize

\ifkorrekturansicht
  \lohead{\textsc{register}}
\fi

% theindex-Environment neu definieren ohne reledmac
\makeatletter
\renewenvironment{theindex}{%
  \ifkorrekturansicht
    \section*{\indexname}%
  \else
    \subsubsection*{Index der erwähnten Entitäten}%
  \fi
  \setlength{\parindent}{0pt}%
  \setlength{\parskip}{0pt plus 0.3pt}%
  \let\item\@idxitem
}{%
  \ifkorrekturansicht\clearpage\fi
}
\makeatother

\IfFileExists{\jobname-pw.ind}{\input{\jobname-pw.ind}}{}

% Quellenangabe nur in der Leseansicht
\ifkorrekturansicht\else
% Fallback-Definitionen, falls die .tex-Datei \titel etc. nicht gesetzt hat
\providecommand{\titel}{}
\providecommand{\editorInnen}{}
\providecommand{\dateiname}{\jobname}

\vspace{3cm}

\vfill

\footnotesize
\textsc{Quelle}: \titel. Herausgegeben von {\editorInnen}. In: \emph{Arthur Schnitzler: Briefwechsel mit Autorinnen und Autoren}.
 Digitale Edition, https://schnitzler-briefe.acdh.oeaw.ac.at/{\dateiname}.html (Stand \today)
\fi

\end{document}


