%% latex-korrekturansicht-vorspann.tex
%% Vorspann für die Korrekturansicht.
%% Lädt die gemeinsame Datei latex-vorspann.tex mit gesetztem Schalter.

\newif\ifkorrekturansicht
\korrekturansichttrue

\input{../tex-inputs/latex-vorspann}


\section[Laura Marholm an Arthur Schnitzler, 15. 5. 1895]{L00441 Laura Marholm an Arthur Schnitzler, 15. 5. 1895}
\nopagebreak\mylabel{L00441v}
\rehead{ }\normalsize\beginnumbering\briefempfaengerindex{Schnitzler, Arthur@\textsc{Schnitzler, Arthur}!zzzMarholm, Laura@\emph{von Laura Marholm}!1895-05-151@{15. 5. 1895}|(be}
\toendnotes[C]{\smallbreak\pagebreak[2]}\Standort{CUL, Schnitzler, B 69.}
\physDesc{Brief, 1 Blatt, 1 Seite, 1019 Zeichen
\newline{}Handschrift: schwarze Tinte, lateinische Kurrent}\toendnotes[C]{\smallbreak}
\pstart
           \raggedleft{}{\pb}Schliersee\oindex{Schliersee@\textbf{Schliersee}, \emph{P.PPL}|pw}, Oberbaiern\oindex{Oberbayern@\textbf{Oberbayern}, \emph{A.ADM2}|pw}\pend
           
\pstart
           \raggedleft{}15. Mai 95.\pend
           
\pstart{}Sehr geehrter Herr Doktor.\pend\vspace{0.5em}
\pstart
           Den Musenalmanach von 94\pwindex{Moderner Musen-Almanach auf das Jahr 1894. Ein Jahrbuch deutscher Kunst@\emph{Moderner Musen-Almanach auf das Jahr 1894. Ein Jahrbuch deutscher Kunst}|pw} hab ich noch nicht
               finden können, aber ich muß ihn haben und finde ihn schon. Das, was ich meine, ist
               vielleicht nur ein Erzeugniß der Einsamkeit, wo das Leben Einem zu dicht und stark an
               den Ohren klopft. Es ist sehr merkwürdig, daß ich es grade am stärksten in
               Glücksmomenten empfinde.\pend
           
\pstart
           Ich freue mich auf ihre weiteren Bücher!\pend
           
\pstart
           Heute nur eine Bitte: haben Sie nicht bemerkt, ob in der letzten Zeit von mir das
               eine oder andere Feuilleton: »Der Dichter des
                  Weibmysteriums\pwindex{Dichter des Weibmysteriums@\emph{Der Dichter des Weibmysteriums}|pw}« oder »Weisse Fläche\pwindex{Weisse Flaeche@\emph{Weiße Fläche}|pw}« in
               der N. freien Presse\pwindex{Neue Freie Presse@\emph{Neue Freie Presse}|pw} sichtbar gewesen ist? Man
               erfährt niemals was direct von daher. Und ich habe Niemanden in Wien\oindex{Wien@\textbf{Wien}, \emph{A.ADM2}|pw}, der mir darüber Auskunft gäbe. Sie sind doch Leser der N. fr. Presse\pwindex{Neue Freie Presse@\emph{Neue Freie Presse}|pw} und ich wäre Ihnen sehr dankbar für
               die Nachricht, ob das eine oder andere schon erschienen ist, oder bis Ende
                  Mai erscheint, da ich das erstere Feuilleton\pwindex{Dichter des Weibmysteriums@\emph{Der Dichter des Weibmysteriums}|pwv} bald in ein \label{K_L00441-1v}\edtext{Buch\pwindex{Wir Frauen und unsere Dichter@\emph{Wir Frauen und unsere Dichter}|pwv}}{\lemma{\textnormal{\emph{Buch}}}\Cendnote{\textnormal{Da der Text über Barbey d’Aurevilly\pwindex{Barbey DAurevilly, Jules-Amedee 02.11.1808 – 23.04.1889@\textsc{Barbey d’Aurevilly, Jules-Amédée} (02.11.1808 – 23.04.1889), \emph{Schriftsteller/Schriftstellerin}|pwk} erst am 2. 11. 1895 in der
                     \emph{Zukunft}\pwindex{Zukunft@\emph{Die Zukunft}|pwk} (Bd. 13, S. 219–226)
                  erschien, fehlte er in der 1. Auflage von \emph{Wir
                     Frauen und unsere Dichter}\pwindex{Wir Frauen und unsere Dichter@\emph{Wir Frauen und unsere Dichter}|pwk} (Wien, Leipzig: \emph{Verlag der
                        Wiener Mode}{ }1895), wurde aber in die »Zweite umgearbeitete und wesentlich vermehrte
                     Ausgabe mit 8 Portraits« aufgenommen (Berlin: \emph{Carl
                        Duncker}{ }{[}1900{]}, S. 271–289).}}}\label{K_L00441-1} aufnehmen will.\pend
           
\pstart
           \label{T_L00441-1v}\edtext{Also}{\lemma{\textnormal{\emph{Also}}}\Cendnote{\textnormal{weiter quer am linken Rand}}}\label{T_L00441-1} beste Grüße für diesmal. Kommt
               bald was von Ihnen?\pend
           \pstart Ihre ergeb.\hspace*{1.5em}\spacefill\mbox{Laura Marholm.}\pend{}\selectlanguage{ngerman}\endnumbering\briefempfaengerindex{Schnitzler, Arthur@\textsc{Schnitzler, Arthur}!zzzMarholm, Laura@\emph{von Laura Marholm}!1895-05-151@{15. 5. 1895}|)be}\mylabel{L00441h}  \normalsize

\doendnotes{C}
\bigskip
\vfill

\clearpage

\footnotesize

\lohead{\textsc{register}}

% Definiere theindex-Environment komplett neu ohne reledmac
\makeatletter
\renewenvironment{theindex}{%
  \section*{\indexname}%
  \setlength{\parindent}{0pt}%
  \setlength{\parskip}{0pt plus 0.3pt}%
  \let\item\@idxitem
}{%
  \clearpage
}
\makeatother

\IfFileExists{\jobname-pw.ind}{\input{\jobname-pw.ind}}{}

\end{document}

      