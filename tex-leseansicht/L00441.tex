%% latex-leseansicht-vorspann.tex
%% Vorspann für die Leseansicht.
%% Lädt die gemeinsame Datei latex-vorspann.tex mit nicht gesetztem Schalter.

\newif\ifkorrekturansicht
\korrekturansichtfalse

\input{../tex-inputs/latex-vorspann}


         
         \renewcommand{\erwaehntePersonen}{Personen: Jules-Amédée Barbey d’Aurevilly, Laura Marholm}
         \renewcommand{\erwaehnteOrte}{Orte: Oberbayern, Schliersee, Wien}
         \renewcommand{\erwaehnteWerke}{Werke: Der Dichter des Weibmysteriums, Die Zukunft, Moderner Musen-Almanach auf das Jahr 1894. Ein Jahrbuch deutscher Kunst, Neue Freie Presse, Weiße Fläche, Wir Frauen und unsere Dichter}
               \section[Laura Marholm an Arthur Schnitzler, 15. 5. 1895]{ Laura Marholm an Arthur Schnitzler, 15. 5. 1895}\nopagebreak\mylabel{v}\rehead{ }\begin{ledgroupsized}[t]{13cm}\normalsize\beginnumbering \toendnotes[C]{\smallbreak\pagebreak[2]} \Standort{CUL, Schnitzler, B 69.}
\physDesc{Brief, 1 Blatt, 1 Seite, 1019 Zeichen
\newline{}Handschrift: schwarze Tinte, lateinische Kurrent}\toendnotes[C]{\smallbreak}\pstart
           \noindent{}\raggedleft{}{\pb}Schliersee\oindex{Schliersee@\textbf{Schliersee}|pw}, Oberbaiern\oindex{Oberbayern@\textbf{Oberbayern}|pw}\pend
           \pstart
           \raggedleft{}15. Mai 95.\pend
           \pstart{}Sehr geehrter Herr Doktor.\pend\pstart
           Den Musenalmanach von 94\pwindex{Moderner Musen-Almanach auf das Jahr 1894. Ein Jahrbuch deutscher Kunst1893@\emph{Moderner Musen-Almanach auf das Jahr 1894. Ein Jahrbuch deutscher Kunst} {[}1893{]}|pw} hab ich noch nicht
               finden können, aber ich muß ihn haben und finde ihn schon. Das, was ich meine, ist
               vielleicht nur ein Erzeugniß der Einsamkeit, wo das Leben Einem zu dicht und stark an
               den Ohren klopft. Es ist sehr merkwürdig, daß ich es grade am stärksten in
               Glücksmomenten empfinde.\pend
           \pstart
           Ich freue mich auf ihre weiteren Bücher!\pend
           \pstart
           Heute nur eine Bitte: haben Sie nicht bemerkt, ob in der letzten Zeit von mir das
               eine oder andere Feuilleton: »Der Dichter des
                  Weibmysteriums\pwindex{Marholm, Laura 19.04.1854 – 06.10.1928@\textsc{Marholm, Laura} (19.04.1854 – 06.10.1928), \emph{Schriftstellerin}!Dichter des Weibmysteriums02. 11. 1895@\strich\emph{Der Dichter des Weibmysteriums} {[}02. 11. 1895{]}|pw}« oder »Weisse Fläche\pwindex{Marholm, Laura 19.04.1854 – 06.10.1928@\textsc{Marholm, Laura} (19.04.1854 – 06.10.1928), \emph{Schriftstellerin}!Weisse Flaeche18. 08. 1895@\strich\emph{Weiße Fläche} {[}18. 08. 1895{]}|pw}« in
               der N. freien Presse\pwindex{Neue Freie Presse1864 – 1939@\emph{Neue Freie Presse} {[}1864 – 1939{]}|pw} sichtbar gewesen ist? Man
               erfährt niemals was direct von daher. Und ich habe Niemanden in Wien\oindex{Wien@\textbf{Wien}|pw}, der mir darüber Auskunft gäbe. Sie sind doch Leser der N. fr. Presse\pwindex{Neue Freie Presse1864 – 1939@\emph{Neue Freie Presse} {[}1864 – 1939{]}|pw} und ich wäre Ihnen sehr dankbar für
               die Nachricht, ob das eine oder andere schon erschienen ist, oder bis Ende
                  Mai erscheint, da ich das erstere Feuilleton\pwindex{Marholm, Laura 19.04.1854 – 06.10.1928@\textsc{Marholm, Laura} (19.04.1854 – 06.10.1928), \emph{Schriftstellerin}!Dichter des Weibmysteriums02. 11. 1895@\strich\emph{Der Dichter des Weibmysteriums} {[}02. 11. 1895{]}|pwv} bald in ein \label{K_L00441-1v}\edtext{Buch\pwindex{Marholm, Laura 19.04.1854 – 06.10.1928@\textsc{Marholm, Laura} (19.04.1854 – 06.10.1928), \emph{Schriftstellerin}!Wir Frauen und unsere Dichter1895@\strich\emph{Wir Frauen und unsere Dichter} {[}1895{]}|pwv}}{\lemma{\textnormal{\emph{Buch}}}\Cendnote{\textnormal{Da der Text über Barbey d’Aurevilly\pwindex{Barbey DAurevilly, Jules-Amedee 02.11.1808 – 23.04.1889@\textsc{Barbey d’Aurevilly, Jules-Amédée} (02.11.1808 – 23.04.1889), \emph{Schriftsteller}|pwk} erst am 2. 11. 1895 in der
                     \emph{Zukunft}\pwindex{Zukunft1892 – 1922@\emph{Die Zukunft} {[}1892 – 1922{]}|pwk} (Bd. 13, S. 219–226)
                  erschien, fehlt er in der 1. Auflage von \emph{Wir
                     Frauen und unsere Dichter}\pwindex{Marholm, Laura 19.04.1854 – 06.10.1928@\textsc{Marholm, Laura} (19.04.1854 – 06.10.1928), \emph{Schriftstellerin}!Wir Frauen und unsere Dichter1895@\strich\emph{Wir Frauen und unsere Dichter} {[}1895{]}|pwk} (Wien, Leipzig: \emph{Verlag der
                        Wiener Mode}{ }1895), wurde aber in die »Zweite umgearbeitete und wesentlich vermehrte
                     Ausgabe mit 8 Portraits« aufgenommen (Berlin: \emph{Carl
                        Duncker}{ }{[}1900{]}, S. 271–289).}}}\label{K_L00441-1h} aufnehmen will.\pend
           \pstart
           \label{T_L00441-1v}\edtext{Also}{\lemma{\textnormal{\emph{Also}}}\Cendnote{\textnormal{weiter quer am linken Rand}}}\label{T_L00441-1h} beste Grüße für diesmal. Kommt
               bald was von Ihnen?\pend
           \pstart Ihre ergeb.\hspace*{1.5em}\spacefill\mbox{Laura Marholm.}\pend{}
         
         \endnumbering\mylabel{h}\end{ledgroupsized}  \newcommand{\dateiname}{L00441}\newcommand{\titel}{Laura Marholm an Arthur Schnitzler, 15. 5. 1895}\newcommand{\editorInnen}{Martin Anton Müller und Gerd-Hermann Susen}%% latex-leseansicht-abspann.tex
%% Abspann für die Leseansicht.
%% Der Schalter \ifkorrekturansicht ist bereits durch den Vorspann gesetzt.

%% latex-abspann.tex
%% Gemeinsamer Abspann für Korrekturansicht und Leseansicht.
%% Setzt den Schalter \ifkorrekturansicht voraus (gesetzt in den
%% einbindenden Dateien latex-korrekturansicht-abspann.tex bzw.
%% latex-leseansicht-abspann.tex).
%% ---------------------------------------------------------------

\normalsize

% Das esempio-Environment wird nur in der Leseansicht benötigt
\ifkorrekturansicht\else
\newenvironment{esempio}[3]%
{
    \vspace{1.5ex}
    \rlap{\underline{#1}}
    \par
    \setlength{\parindent}{0cm}
    \nopagebreak
    \leftskip=#2cm
    \rightskip=#3cm
}
{
    \par
}
\fi

\doendnotes{C}
\bigskip
\vfill

\clearpage

\footnotesize

\ifkorrekturansicht
  \lohead{\textsc{register}}
\fi

% theindex-Environment neu definieren ohne reledmac
\makeatletter
\renewenvironment{theindex}{%
  \ifkorrekturansicht
    \section*{\indexname}%
  \else
    \subsubsection*{Index der erwähnten Entitäten}%
  \fi
  \setlength{\parindent}{0pt}%
  \setlength{\parskip}{0pt plus 0.3pt}%
  \let\item\@idxitem
}{%
  \ifkorrekturansicht\clearpage\fi
}
\makeatother

\IfFileExists{\jobname-pw.ind}{\input{\jobname-pw.ind}}{}

% Quellenangabe nur in der Leseansicht
\ifkorrekturansicht\else
% Fallback-Definitionen, falls die .tex-Datei \titel etc. nicht gesetzt hat
\providecommand{\titel}{}
\providecommand{\editorInnen}{}
\providecommand{\dateiname}{\jobname}

\vspace{3cm}

\vfill

\footnotesize
\textsc{Quelle}: \titel. Herausgegeben von {\editorInnen}. In: \emph{Arthur Schnitzler: Briefwechsel mit Autorinnen und Autoren}.
 Digitale Edition, https://schnitzler-briefe.acdh.oeaw.ac.at/{\dateiname}.html (Stand \today)
\fi

\end{document}


      