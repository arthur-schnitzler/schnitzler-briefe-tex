%% latex-leseansicht-vorspann.tex
%% Vorspann für die Leseansicht.
%% Lädt die gemeinsame Datei latex-vorspann.tex mit nicht gesetztem Schalter.

\newif\ifkorrekturansicht
\korrekturansichtfalse

\input{../tex-inputs/latex-vorspann}


\section[Paul Goldmann an Arthur Schnitzler, 15. 11. 1891]{L02670 Paul Goldmann an Arthur Schnitzler, 15. 11. 1891}
\nopagebreak\mylabel{L02670v}
\rehead{ }\normalsize\beginnumbering\briefempfaengerindex{Schnitzler, Arthur@\textsc{Schnitzler, Arthur}!zzzGoldmann, Paul@\emph{von Paul Goldmann}!1891-11-151@{15. 11. 1891}|(be}
\toendnotes[C]{\smallbreak\pagebreak[2]}
\correspDesc{Versand  durch Paul Goldmann am 15. 11. 1891 in Brüssel
\newline{}Erhalt  durch Arthur Schnitzler im Zeitraum [16. 11. 1891 – 20. 11. 1891?] in Wien}\toendnotes[C]{\smallbreak}
\Standort{DLA, A:Schnitzler, HS.NZ85.1.3162.}
\physDesc{Brief, 2 Blätter, 8 Seiten, 5549 Zeichen
\newline{}Handschrift: blaue Tinte, deutsche Kurrent
\newline{}Schnitzler: mit rotem Buntstift eine Unterstreichung und eine seitliche Markierung }\toendnotes[C]{\smallbreak}
\pstart
           \centering{}{\pb}\textcolor{gray}{\textbf{Dr. jur. Paul Goldmann}}\pend
           
\pstart
           \centering{}\textcolor{gray}{\textbf{\begin{otherlanguage}{french}Correspondant de la »Gazette de Francfort\orgindex{Frankfurter Zeitung@Frankfurter Zeitung|pw}«\end{otherlanguage}}}\pend
           
\pstart
           \centering{}\textcolor{gray}{\textbf{\begin{otherlanguage}{french}Bruxelles, 21, rue des Plantes\end{otherlanguage}\oindex{rue des Plantes@\textbf{rue des Plantes}, \emph{Straße}|pw}.}}\pend
           
\pstart
           \raggedleft{}Brüſſel\oindex{Brüssel@\textbf{Brüssel}, \emph{Hauptstadt}|pw},
                  15. November 1891.\pend
           
\pstart\center{}Mein lieber Arthur!\pend\vspace{0.5em}
\pstart
           Der Dank für Deine lieben Briefe, die mich unendlich erfreut haben, brennt mir{ }ſchon
               lange auf dem Herzen. Aber eine große Affaire, die{ }ſeit ein paar Wochen im Zuge iſt,
               hat mir bisher die Hände gebunden. Heut iſt es entſchieden: in 14 Tagen gehe ich nach
                  Paris\oindex{Paris@\textbf{Paris}, \emph{Hauptstadt}|pw} als politiſcher und literariſcher
               Correſpondent der »Frankfurter Zeitung\orgindex{Frankfurter Zeitung@Frankfurter Zeitung|pw}«.
               Äußerlich recht ehrenvoll. Innerlich, unter uns, nur ein Verſuch{ }ſeitens des Blatt\orgindex{Frankfurter Zeitung@Frankfurter Zeitung|pwv}es, eine billige junge
               Kraft in zehnfachem Maße auszubeuten als bisher. Die Arbeit in Paris\oindex{Paris@\textbf{Paris}, \emph{Hauptstadt}|pw} wächſt in’s Unendliche, desgleichen die
               Verantwortlichkeit; keiner der früheren Correſpondenten {\pb}hat{ }ſich noch länger als drei Jahre halten können.
               In Bezug auf den Gehalt werde ich wahrſcheinlich betrogen werden; die kleine Erhöhung
               gegen bisher wird durch die theuren Lebensverhältniſſe aufgewogen; von meinem
               einzigen Ziel, zur Selbſtändigkeit zu \strikeout{g\textcolor{gray}{l}} gelangen, bin ich alſo ferner als je. Und bei meinem Ekel vor der Politik, der{ }ſich hier noch \strikeout{ac} accentuirt hat, bei meiner Ignoranz
               in der franzöſiſchen Sprache, bei meinem Hang zur ruhigen, \strikeout{\textcolor{gray}{ſt}} friedlichen, langſamen Arbeit habe ich alle Ausſichten, mich nicht zu bewähren
               und nicht zum Wohlbehagen zu gelangen. Ich gehe morgen
               von hier fort. Die Stadt\oindex{Brüssel@\textbf{Brüssel}, \emph{Hauptstadt}|pwv} iſt
               mir in den letzten Wochen lieb geworden; ich war im Begriff, mein \textsc{Milieu} zu finden. Und im Augenblick, wo ich mich hübſch
               behaglich in eine warme Ecke drücken will, {\pb}\strikeout{reißt} reißt das Leben die Thür auf, zwingt mir wieder
               den Wanderſtab \strikeout{heraus} in die Hand und{ }ſtößt mich in
               den Sturm und Regen der Landſtraße hinaus. Gott weiß allein, was er mit mir
               vorhat.\pend
           
\pstart
           Vielleicht finde ich vor meiner Abreiſe von hier noch Zeit, Dir ausführlich zu{ }ſchreiben. Einſtweilen laß’ Dir mit einem innigen Dankwort genügen für den
               Wärmeſtrom, den Du mit Deinen lieben Briefen in mein Herz geleitet. Was mich im
               Beſonderen für Dich erfreut, das iſt ein gewiſſer Hauch von Arbeitsfreude, der daraus
               hervorweht. Wenn das keine vorübergehende Stimmung,{ }ſondern ein bleibender
               Seelenzuſtand iſt,{ }ſo gibt es kein noch{ }ſo hohes Ziel, deſſen Erreichung ich für Dich
               nicht erhoffe. Einer Sorge möchte ich gleich hier Ausdruck verleihen: \strikeout{ich} die \label{K_L02670-1v}\edtext{Bedenken}{\lemma{\textnormal{\emph{Bedenken}}}\Cendnote{\textnormal{Siehe XXXX Auszeichnungsfehler: Dokument L02668 nicht gefunden. Am 28. 10. 1891 hatte der
                  erste (und letzte) »gesellige Abend«, wie er genannt wurde, stattgefunden. Bei
                  diesem hatte Max Devrient\pwindex{Devrient, Max 12.\,12.\,1857 Hannover – 13.\,6.\,1929 Chur@\textsc{Devrient, Max} (12.\,12.\,1857 Hannover – 13.\,6.\,1929 Chur), \emph{Regisseur, Schauspieler}|pwk} von Schnitzler zwei Gedichte\pwindex{Schnitzler, Arthur 15.\,5.\,1862 Wien – 21.\,10.\,1931 ebd.@\textsc{Schnitzler, Arthur} (15.\,5.\,1862 Wien – 21.\,10.\,1931 ebd.), \emph{Schriftsteller, Mediziner}!Am Flügel@\strich\emph{Am Flügel}|pwkv}\pwindex{Schnitzler, Arthur 15.\,5.\,1862 Wien – 21.\,10.\,1931 ebd.@\textsc{Schnitzler, Arthur} (15.\,5.\,1862 Wien – 21.\,10.\,1931 ebd.), \emph{Schriftsteller, Mediziner}!An die Alten@\strich\emph{An die Alten}|pwkv} rezitiert: \emph{Am Flügel}\pwindex{Schnitzler, Arthur 15.\,5.\,1862 Wien – 21.\,10.\,1931 ebd.@\textsc{Schnitzler, Arthur} (15.\,5.\,1862 Wien – 21.\,10.\,1931 ebd.), \emph{Schriftsteller, Mediziner}!Am Flügel@\strich\emph{Am Flügel}|pwk} und \emph{An die
                     Alten}\pwindex{Schnitzler, Arthur 15.\,5.\,1862 Wien – 21.\,10.\,1931 ebd.@\textsc{Schnitzler, Arthur} (15.\,5.\,1862 Wien – 21.\,10.\,1931 ebd.), \emph{Schriftsteller, Mediziner}!An die Alten@\strich\emph{An die Alten}|pwk}. Schnitzler dürfte Goldmann\pwindex{Goldmann, Paul 31.\,1.\,1865 Breslau – 25.\,9.\,1935 Wien@\textsc{Goldmann, Paul} (31.\,1.\,1865 Breslau – 25.\,9.\,1935 Wien), \emph{Schriftsteller, Journalist}|pwk} davon in einem Brief berichtet
                  haben.}}}\label{K_L02670-1}, welche {\pb}ich gegen das Bodenfaſſen
               der »Freien-Bühne\orgindex{»Freie Bühne« Verein für moderne Literatur@»Freie Bühne« Verein für moderne Literatur|pw}«-Bewegung gehabt,{ }ſind jetzt in
               mir faſt zur negativen Gewißheit erwachſen. Die \label{K_L02670-2v}\edtext{Macher der Bewegung\orgindex{»Freie Bühne« Verein für moderne Literatur@»Freie Bühne« Verein für moderne Literatur|pwv}}{\lemma{\textnormal{\emph{Macher der Bewegung}}}\Cendnote{\textnormal{Am 7. 7. 1891 hatte die Gründungssitzung der
                     \emph{Freien Bühne}\orgindex{»Freie Bühne« Verein für moderne Literatur@»Freie Bühne« Verein für moderne Literatur|pwk} stattgefunden, einem »Verein für moderne
                        Literatur\orgindex{»Freie Bühne« Verein für moderne Literatur@»Freie Bühne« Verein für moderne Literatur|pwv}«. Zum Obmann\pwindex{Fels, Friedrich Michael *~1864 Bad Dürkheim@\textsc{Fels, Friedrich Michael} (*~1864 Bad Dürkheim), \emph{Journalist}|pwkv} war Friedrich Michael Fels\pwindex{Fels, Friedrich Michael *~1864 Bad Dürkheim@\textsc{Fels, Friedrich Michael} (*~1864 Bad Dürkheim), \emph{Journalist}|pwk}
                  gewählt worden, Stellvertreter\pwindex{Wengraf, Edmund 9.\,1.\,1860 Mikulov – 8.\,12.\,1933 Wien@\textsc{Wengraf, Edmund} (9.\,1.\,1860 Mikulov – 8.\,12.\,1933 Wien), \emph{Schriftsteller, Journalist, Kaufmann}|pwkv}
                  wurden Edmund Wengraf\pwindex{Wengraf, Edmund 9.\,1.\,1860 Mikulov – 8.\,12.\,1933 Wien@\textsc{Wengraf, Edmund} (9.\,1.\,1860 Mikulov – 8.\,12.\,1933 Wien), \emph{Schriftsteller, Journalist, Kaufmann}|pwk}\pwindex{Fürst, Hermann 21.\,7.\,1849 Althart – 17.\,1.\,1895 Wien@\textsc{Fürst, Hermann} (21.\,7.\,1849 Althart – 17.\,1.\,1895 Wien), \emph{Schriftsteller}|pwk} und Hermann Fürst\pwindex{Fürst, Hermann 21.\,7.\,1849 Althart – 17.\,1.\,1895 Wien@\textsc{Fürst, Hermann} (21.\,7.\,1849 Althart – 17.\,1.\,1895 Wien), \emph{Schriftsteller}|pwk}. Schnitzler war Ausschussmitglied des Vereins\orgindex{»Freie Bühne« Verein für moderne Literatur@»Freie Bühne« Verein für moderne Literatur|pwkv}. Siehe XXXX Auszeichnungsfehler: Dokument L02668 nicht gefunden.
               }}}\label{K_L02670-2}{ }ſind \strikeout{zu} theils zu wenig erfahren, theils zu
               wenig begabt, theils zu wenig ehrlich; und der blöde Widerſtand des Publicums wie{ }ſeiner Lakaien, der »Kritiker«, iſt auf dieſe Weiſe nicht zu brechen. Die \textsc{Wengrafs\pwindex{Wengraf, Edmund 9.\,1.\,1860 Mikulov – 8.\,12.\,1933 Wien@\textsc{Wengraf, Edmund} (9.\,1.\,1860 Mikulov – 8.\,12.\,1933 Wien), \emph{Schriftsteller, Journalist, Kaufmann}|pw} etc.}{ }ſind die Schlauen,
               welche Wind \strikeout{h} davon haben und beizeiten ihren Einſatz
               aus dem Spiele ziehen. Denen werden wahrſcheinlich noch Andere folgen. Nun möchte ich
               um Alles in der Welt nicht, daß Du das Opfer Deiner makelloſen Ehrlichkeit wirſt und
               Deinen guten Namen an eine Sache hefteſt, die ihn bei ihrem Zuſammenbruch{ }ſchwer
               compromittiren könnte. Ein Martyrium für die gute Sache – {\pb}meinetwegen! Aber die Sache iſt nicht gut – dieſe
               Sache der \textsc{Joachims\pwindex{Joachim, Jaques 24.\,11.\,1866 Wien – 7.\,11.\,1925 ebd.@\textsc{Joachim, Jaques} (24.\,11.\,1866 Wien – 7.\,11.\,1925 ebd.), \emph{Rechtswissenschaftler, Rechtsanwalt, Herausgeber}|pw}}, \textsc{Kafkas\pwindex{Kafka, Eduard Michael 11.\,3.\,1869 Wien – 6.\,8.\,1893 Brünn@\textsc{Kafka, Eduard Michael} (11.\,3.\,1869 Wien – 6.\,8.\,1893 Brünn), \emph{Redakteur}|pw}{ }}\textsc{etc.} Und darum meine ich: wenn die Unternehmung\orgindex{»Freie Bühne« Verein für moderne Literatur@»Freie Bühne« Verein für moderne Literatur|pwv} nicht unbedingte Ausſicht auf
                  \label{K_L02670-3v}\edtext{Gedeihen}{\lemma{\textnormal{\emph{Gedeihen}}}\Cendnote{\textnormal{Tatsächlich kriselte es in der \emph{Freien Bühne}\orgindex{»Freie Bühne« Verein für moderne Literatur@»Freie Bühne« Verein für moderne Literatur|pwk} bereits wenige Wochen nach der Gründung. In einem Theaterbrief\pwindex{Fels, Friedrich Michael *~1864 Bad Dürkheim@\textsc{Fels, Friedrich Michael} (*~1864 Bad Dürkheim), \emph{Journalist}!Wiener Brief@\strich\emph{Wiener Brief}|pwkv} begründete Friedrich Michael Fels\pwindex{Fels, Friedrich Michael *~1864 Bad Dürkheim@\textsc{Fels, Friedrich Michael} (*~1864 Bad Dürkheim), \emph{Journalist}|pwk} das Scheitern des Vereins\orgindex{»Freie Bühne« Verein für moderne Literatur@»Freie Bühne« Verein für moderne Literatur|pwkv} damit, dass zu wenig
                  der geplanten Vorhaben umgesetzt wurden und außer dem einen »geselligen Abend«
                  nichts zustande kam. Vgl. Friedrich Michael Fels\pwindex{Fels, Friedrich Michael *~1864 Bad Dürkheim@\textsc{Fels, Friedrich Michael} (*~1864 Bad Dürkheim), \emph{Journalist}|pwk}: \emph{Wiener Brief}\pwindex{Fels, Friedrich Michael *~1864 Bad Dürkheim@\textsc{Fels, Friedrich Michael} (*~1864 Bad Dürkheim), \emph{Journalist}!Wiener Brief@\strich\emph{Wiener Brief}|pwk}. In: \emph{Freie
                        Bühne für den Entwickelungskampf der Zeit}\pwindex{Freie Bühne für den Entwickelungskampf der Zeit@\emph{Freie Bühne für den Entwickelungskampf der Zeit}|pwk}, Jg. 3, H. 1, Februar 1892, S. 197–201.}}}\label{K_L02670-3} bietet;
               wenn Du nicht{ }ſelbſt unumſchränkt leiten kannſt –{ }ſo zieh’ auch Du Dich ein wenig
               zurück. Du brauchſt, weiß Gott, keine Partei\orgindex{»Freie Bühne« Verein für moderne Literatur@»Freie Bühne« Verein für moderne Literatur|pwv} und biſt{ }ſtark genug, deine eigenen Wege zu gehen.
               Eine \label{K_L02670-4v}\edtext{Aufführung des »Märchen\pwindex{Schnitzler, Arthur 15.\,5.\,1862 Wien – 21.\,10.\,1931 ebd.@\textsc{Schnitzler, Arthur} (15.\,5.\,1862 Wien – 21.\,10.\,1931 ebd.), \emph{Schriftsteller, Mediziner}!Märchen. Schauspiel in drei Aufzügen@\strich\emph{Das Märchen. Schauspiel in drei Aufzügen}|pw}«}{\lemma{\textnormal{\emph{Aufführung des »Märchen«}}}\Cendnote{\textnormal{\emph{Das Märchen}\pwindex{Schnitzler, Arthur 15.\,5.\,1862 Wien – 21.\,10.\,1931 ebd.@\textsc{Schnitzler, Arthur} (15.\,5.\,1862 Wien – 21.\,10.\,1931 ebd.), \emph{Schriftsteller, Mediziner}!Märchen. Schauspiel in drei Aufzügen@\strich\emph{Das Märchen. Schauspiel in drei Aufzügen}|pwk} wurde eine Zeit lang – und
                  offenbar bis zur Gegenwart dieses Briefes – als Inszenierung der \emph{Freien Bühne}\orgindex{»Freie Bühne« Verein für moderne Literatur@»Freie Bühne« Verein für moderne Literatur|pwk} erwogen (vgl. A. S.: \emph{Tagebuch}, 13. 7. 1891). Schnitzler selbst lehnte dies jedoch ab und wollte das Drama\pwindex{Schnitzler, Arthur 15.\,5.\,1862 Wien – 21.\,10.\,1931 ebd.@\textsc{Schnitzler, Arthur} (15.\,5.\,1862 Wien – 21.\,10.\,1931 ebd.), \emph{Schriftsteller, Mediziner}!Märchen. Schauspiel in drei Aufzügen@\strich\emph{Das Märchen. Schauspiel in drei Aufzügen}|pwkv} am \emph{Burgtheater}\orgindex{Burgtheater@Burgtheater|pwk} aufgeführt wissen.}}}\label{K_L02670-4} durch die »Freie Bühne\orgindex{»Freie Bühne« Verein für moderne Literatur@»Freie Bühne« Verein für moderne Literatur|pw}«, wenn nicht ganz vorzügliche{ }ſchauſpieleriſche
               Kräfte geſichert{ }ſind, hielte ich für eine große Gefahr. Das Publicum iſt zu dumm, um
               das Stück\pwindex{Schnitzler, Arthur 15.\,5.\,1862 Wien – 21.\,10.\,1931 ebd.@\textsc{Schnitzler, Arthur} (15.\,5.\,1862 Wien – 21.\,10.\,1931 ebd.), \emph{Schriftsteller, Mediziner}!Märchen. Schauspiel in drei Aufzügen@\strich\emph{Das Märchen. Schauspiel in drei Aufzügen}|pwv} zu begreifen; und
               auf der andern Seite mangelt der »Freien Bühne\orgindex{»Freie Bühne« Verein für moderne Literatur@»Freie Bühne« Verein für moderne Literatur|pw}«
                  {\pb}in Wien\oindex{Wien@\textbf{Wien}, \emph{Verwaltungsgebiet}|pw} die
               Autorität, welche, als Surrogat des Verſtändniſſes, das dumme Volk zum Beifall
               zwingt. Nach dem von den »führenden Geiſtern« der Preſſe ausgehenden Loſungswort wird
               jeder Lausbub{ }ſich berechtigt glauben, Kritik zu üben; und die Zeitungen werden Dich
               zerreißen oder mit, \strikeout{\textcolor{gray}{g}} vernichtendem Wohlwollen behandeln. (\label{K_L02670-5v}\edtext{\textsc{N. B.}}{\lemma{\textnormal{\emph{N. B.}}}\Cendnote{\textnormal{nota bene, lateinisch: merke
                  wohl}}}\label{K_L02670-5}{ }\label{K_L02670-6v}\edtext{\textsc{Hugo Kleins\pwindex{Klein, Hugo 21.\,7.\,1853 Szeged – 29.\,6.\,1915 Karlsbad@\textsc{Klein, Hugo} (21.\,7.\,1853 Szeged – 29.\,6.\,1915 Karlsbad), \emph{Schriftsteller, Journalist, Kritiker}|pw}}{ }Artikel\pwindex{Klein, Hugo 21.\,7.\,1853 Szeged – 29.\,6.\,1915 Karlsbad@\textsc{Klein, Hugo} (21.\,7.\,1853 Szeged – 29.\,6.\,1915 Karlsbad), \emph{Schriftsteller, Journalist, Kritiker}!Freie Bühne«@\strich\emph{»Freie Bühne«}|pw}}{\lemma{\textnormal{\emph{Hugo Kleins Artikel}}}\Cendnote{\textnormal{h. k.\pwindex{Klein, Hugo 21.\,7.\,1853 Szeged – 29.\,6.\,1915 Karlsbad@\textsc{Klein, Hugo} (21.\,7.\,1853 Szeged – 29.\,6.\,1915 Karlsbad), \emph{Schriftsteller, Journalist, Kritiker}|pwkv} [ = Hugo Klein\pwindex{Klein, Hugo 21.\,7.\,1853 Szeged – 29.\,6.\,1915 Karlsbad@\textsc{Klein, Hugo} (21.\,7.\,1853 Szeged – 29.\,6.\,1915 Karlsbad), \emph{Schriftsteller, Journalist, Kritiker}|pwk}]: \emph{»Freie Bühne«}\pwindex{Klein, Hugo 21.\,7.\,1853 Szeged – 29.\,6.\,1915 Karlsbad@\textsc{Klein, Hugo} (21.\,7.\,1853 Szeged – 29.\,6.\,1915 Karlsbad), \emph{Schriftsteller, Journalist, Kritiker}!Freie Bühne«@\strich\emph{»Freie Bühne«}|pwk}. In: \emph{Die
                        Presse}\pwindex{Presse@\emph{Die Presse}|pwk}, Jg. 44, Nr. 298, 30. 10. 1891,
                     S. 9. Klein\pwindex{Klein, Hugo 21.\,7.\,1853 Szeged – 29.\,6.\,1915 Karlsbad@\textsc{Klein, Hugo} (21.\,7.\,1853 Szeged – 29.\,6.\,1915 Karlsbad), \emph{Schriftsteller, Journalist, Kritiker}|pwk} äußerte sich darin
                  satirisch-kritisch über den ersten Vortragsabend der \emph{Freien Bühne}\orgindex{»Freie Bühne« Verein für moderne Literatur@»Freie Bühne« Verein für moderne Literatur|pwk} am 28. 10. 1891. Schnitzler erwähnte er folgendermaßen: »zwei
                        Gedichte\pwindex{Schnitzler, Arthur 15.\,5.\,1862 Wien – 21.\,10.\,1931 ebd.@\textsc{Schnitzler, Arthur} (15.\,5.\,1862 Wien – 21.\,10.\,1931 ebd.), \emph{Schriftsteller, Mediziner}!Am Flügel@\strich\emph{Am Flügel}|pwv}\pwindex{Schnitzler, Arthur 15.\,5.\,1862 Wien – 21.\,10.\,1931 ebd.@\textsc{Schnitzler, Arthur} (15.\,5.\,1862 Wien – 21.\,10.\,1931 ebd.), \emph{Schriftsteller, Mediziner}!An die Alten@\strich\emph{An die Alten}|pwv}
                     von Arthur Schnitzler, von welchen
                     besonders das eine: ›Am Flügel\pwindex{Schnitzler, Arthur 15.\,5.\,1862 Wien – 21.\,10.\,1931 ebd.@\textsc{Schnitzler, Arthur} (15.\,5.\,1862 Wien – 21.\,10.\,1931 ebd.), \emph{Schriftsteller, Mediziner}!Am Flügel@\strich\emph{Am Flügel}|pw}‹,
                     unverkennbar den Einfluß Baumbach\pwindex{Baumbach, Rudolf 28.\,9.\,1840 Kranichfeld – 21.\,9.\,1905 Meiningen@\textsc{Baumbach, Rudolf} (28.\,9.\,1840 Kranichfeld – 21.\,9.\,1905 Meiningen), \emph{Schriftsteller}|pw}’s
                     widerspiegelt«. Siehe A. S.: \emph{Tagebuch}, 30. 10. 1891.
               }}}\label{K_L02670-6} habe ich geleſen; wäre ich in Wien\oindex{Wien@\textbf{Wien}, \emph{Verwaltungsgebiet}|pw}
               geweſen, ich hätte den Burſchen\pwindex{Klein, Hugo 21.\,7.\,1853 Szeged – 29.\,6.\,1915 Karlsbad@\textsc{Klein, Hugo} (21.\,7.\,1853 Szeged – 29.\,6.\,1915 Karlsbad), \emph{Schriftsteller, Journalist, Kritiker}|pwv} geohrfeigt, allein wegen der Stelle über Dich!). Etwas Anderes wäre
               die Aufführung in Berlin\oindex{Berlin@\textbf{Berlin}, \emph{Hauptstadt}|pw}. Kein{ }ſicherer Erfolg
               freilich; aber dort wirſt Du wenigſtens von Einigen{ }ſo ernſt genommen werden, als Du
               es verdienſt. Ich halte es für das Beſte, die \strikeout{\textcolor{gray}{Aufführu}}{ }\label{K_L02670-7v}\edtext{Antwort \textsc{Blumenthals\pwindex{Blumenthal, Oskar 13.\,3.\,1852 Berlin – 24.\,4.\,1917 ebd.@\textsc{Blumenthal, Oskar} (13.\,3.\,1852 Berlin – 24.\,4.\,1917 ebd.), \emph{Schriftsteller, Journalist, Theaterleiter}|pw}}}{\lemma{\textnormal{\emph{Antwort Blumenthals}}}\Cendnote{\textnormal{Siehe XXXX Auszeichnungsfehler: Dokument L00052 nicht gefunden.
               }}}\label{K_L02670-7} abzuwarten und {\pb}vorher in Wien\oindex{Wien@\textbf{Wien}, \emph{Verwaltungsgebiet}|pw} nicht einen Schritt zu thun. In \textsc{Burckhards\pwindex{Burckhard, Max Eugen 14.\,7.\,1854 Korneuburg – 16.\,3.\,1912 Wien@\textsc{Burckhard, Max Eugen} (14.\,7.\,1854 Korneuburg – 16.\,3.\,1912 Wien), \emph{Schriftsteller, Rechtswissenschaftler, Theaterleiter}|pw}}{ }\label{K_L02670-8v}\edtext{Antwort}{\lemma{\textnormal{\emph{Antwort}}}\Cendnote{\textnormal{Schnitzler hatte die Nachricht, dass Max Burckhard\pwindex{Burckhard, Max Eugen 14.\,7.\,1854 Korneuburg – 16.\,3.\,1912 Wien@\textsc{Burckhard, Max Eugen} (14.\,7.\,1854 Korneuburg – 16.\,3.\,1912 Wien), \emph{Schriftsteller, Rechtswissenschaftler, Theaterleiter}|pwk}{ }\emph{Das Märchen}\pwindex{Schnitzler, Arthur 15.\,5.\,1862 Wien – 21.\,10.\,1931 ebd.@\textsc{Schnitzler, Arthur} (15.\,5.\,1862 Wien – 21.\,10.\,1931 ebd.), \emph{Schriftsteller, Mediziner}!Märchen. Schauspiel in drei Aufzügen@\strich\emph{Das Märchen. Schauspiel in drei Aufzügen}|pwk} nicht am \emph{Burgtheater}\orgindex{Burgtheater@Burgtheater|pwk} inszenieren werde, am 28. 10. 1891 erhalten. Sie dürfte eher
                  mündlich als schriftlich mitgeteilt worden sein. Jedenfalls hat sich kein
                  entsprechendes Korrespondenzstück erhalten. Als Begründung notierte Schnitzler im \emph{Tagebuch}\pwindex{Schnitzler, Arthur 15.\,5.\,1862 Wien – 21.\,10.\,1931 ebd.@\textsc{Schnitzler, Arthur} (15.\,5.\,1862 Wien – 21.\,10.\,1931 ebd.), \emph{Schriftsteller, Mediziner}!Tagebuch@\strich\emph{Tagebuch}|pwk}: »zu viel Rede, zu wenig Handlung«.}}}\label{K_L02670-8} liegt,
               trotz der \label{K_L02670-9v}\edtext{literariſch-ungebildeten
                  Form}{\lemma{\textnormal{\emph{literarisch-ungebildeten Form}}}\Cendnote{\textnormal{Es handelt sich um eine Anspielung darauf, dass Burckhard\pwindex{Burckhard, Max Eugen 14.\,7.\,1854 Korneuburg – 16.\,3.\,1912 Wien@\textsc{Burckhard, Max Eugen} (14.\,7.\,1854 Korneuburg – 16.\,3.\,1912 Wien), \emph{Schriftsteller, Rechtswissenschaftler, Theaterleiter}|pwk}{ }Jurist\pwindex{Burckhard, Max Eugen 14.\,7.\,1854 Korneuburg – 16.\,3.\,1912 Wien@\textsc{Burckhard, Max Eugen} (14.\,7.\,1854 Korneuburg – 16.\,3.\,1912 Wien), \emph{Schriftsteller, Rechtswissenschaftler, Theaterleiter}|pwkv} war und ohne
                     künstlerisch-artistische Vorerfahrung die Leitung des \emph{Burgtheaters}\orgindex{Burgtheater@Burgtheater|pwk} überantwortet bekommen hatte.}}}\label{K_L02670-9},
               vielleicht ein geſunder Inſtinct. Du hätteſt ihm unter allen Umſtänden \label{K_L02670-10v}\edtext{zuerſt den \textsc{Alkandi\pwindex{Schnitzler, Arthur 15.\,5.\,1862 Wien – 21.\,10.\,1931 ebd.@\textsc{Schnitzler, Arthur} (15.\,5.\,1862 Wien – 21.\,10.\,1931 ebd.), \emph{Schriftsteller, Mediziner}!Alkandi’s Lied@\strich\emph{Alkandi’s Lied}|pw}}}{\lemma{\textnormal{\emph{zuerst den Alkandi}}}\Cendnote{\textnormal{Diesen Einakter\pwindex{Schnitzler, Arthur 15.\,5.\,1862 Wien – 21.\,10.\,1931 ebd.@\textsc{Schnitzler, Arthur} (15.\,5.\,1862 Wien – 21.\,10.\,1931 ebd.), \emph{Schriftsteller, Mediziner}!Alkandi’s Lied@\strich\emph{Alkandi’s Lied}|pwkv} hatte Max Burckhard\pwindex{Burckhard, Max Eugen 14.\,7.\,1854 Korneuburg – 16.\,3.\,1912 Wien@\textsc{Burckhard, Max Eugen} (14.\,7.\,1854 Korneuburg – 16.\,3.\,1912 Wien), \emph{Schriftsteller, Rechtswissenschaftler, Theaterleiter}|pwk} bereits am XXXX Auszeichnungsfehler: Dokument L00024 nicht gefunden abgelehnt (vgl. XXXX Auszeichnungsfehler: Dokument L00024 nicht gefunden).}}}\label{K_L02670-10} geben{ }ſollen; und ich rathe Dir
               entſchieden, es auch jetzt noch zu thun. Bringt er das Stück\pwindex{Schnitzler, Arthur 15.\,5.\,1862 Wien – 21.\,10.\,1931 ebd.@\textsc{Schnitzler, Arthur} (15.\,5.\,1862 Wien – 21.\,10.\,1931 ebd.), \emph{Schriftsteller, Mediziner}!Alkandi’s Lied@\strich\emph{Alkandi’s Lied}|pwv} und gefällt es,{ }ſo wäre es gar nicht
               unmöglich, daß er noch auf das »Märchen\pwindex{Schnitzler, Arthur 15.\,5.\,1862 Wien – 21.\,10.\,1931 ebd.@\textsc{Schnitzler, Arthur} (15.\,5.\,1862 Wien – 21.\,10.\,1931 ebd.), \emph{Schriftsteller, Mediziner}!Märchen. Schauspiel in drei Aufzügen@\strich\emph{Das Märchen. Schauspiel in drei Aufzügen}|pw}«
               zurückkäme. Im Übrigen behalte ich mir alle näheren Urtheile bis nach der Lectüre
               vor, die ich aufrichtigſt herbeiwünſche.\pend
           
\pstart
           Dies für heut. Tauſend Dank noch für die Beantwortung
               meiner Fragen, die ausführlichen Mittheilungen über die Lieben in Wien\oindex{Wien@\textbf{Wien}, \emph{Verwaltungsgebiet}|pw}, und all’ das Gütige und Freundſchaftliche, das Deine {\pb}Briefe{ }ſonſt noch enthalten haben. Sie waren mir
               eine Art Feſtgeſchenk. Ehe ich von hier{ }ſcheide (ich fahre etwa am 30. November) höre ich wohl noch ein Wort von Dir? Viele,
               viele Grüße an die Wien\oindex{Wien@\textbf{Wien}, \emph{Verwaltungsgebiet}|pw}er Freunde, vor Allem \textsc{Richard\pwindex{Beer-Hofmann, Richard 11.\,7.\,1866 Wien – 26.\,9.\,1945 New York City@\textsc{Beer-Hofmann, Richard} (11.\,7.\,1866 Wien – 26.\,9.\,1945 New York City), \emph{Schriftsteller}|pw}} und \textsc{Loris\pwindex{Hofmannsthal, Hugo von 1.\,2.\,1874 Wien – 15.\,7.\,1929 Rodaun@\textsc{Hofmannsthal, Hugo von} (1.\,2.\,1874 Wien – 15.\,7.\,1929 Rodaun), \emph{Schriftsteller}|pw}} und \textsc{Kapper\pwindex{Kapper, Friedrich 21.\,4.\,1861 Wien – 22.\,7.\,1939 ebd.@\textsc{Kapper, Friedrich} (21.\,4.\,1861 Wien – 22.\,7.\,1939 ebd.), \emph{Mediziner}|pw}}. Einen herzlichen Händedruck an \textsc{Salten\pwindex{Salten, Felix 6.\,9.\,1869 Budapest – 8.\,10.\,1945 Zürich@\textsc{Salten, Felix} (6.\,9.\,1869 Budapest – 8.\,10.\,1945 Zürich), \emph{Schriftsteller, Journalist, Chefredakteur}|pw}}, der mein{ }ſeeliger \label{K_L02670-11v}\edtext{Erbe\pwindex{Karlsburg, Bertha @\textsc{Karlsburg, Bertha}, \emph{Schauspielerin}|pwv} auf dem gewiſſen mit
               Kiſſen}{\lemma{\textnormal{\emph{Erbe … Kissen}}}\Cendnote{\textnormal{vermutlich Bezug auf Bertha Karlsburg\pwindex{Karlsburg, Bertha @\textsc{Karlsburg, Bertha}, \emph{Schauspielerin}|pwk}, mit der Salten\pwindex{Salten, Felix 6.\,9.\,1869 Budapest – 8.\,10.\,1945 Zürich@\textsc{Salten, Felix} (6.\,9.\,1869 Budapest – 8.\,10.\,1945 Zürich), \emph{Schriftsteller, Journalist, Chefredakteur}|pwk} ein Verhältnis hatte}}}\label{K_L02670-11} weich drapirten Sopha
               geworden zu{ }ſein{ }ſcheint. Ergebene Empfehlungen an die Deinen. Vielen Dank und Gruß
               an »\label{K_L02670-12v}\edtext{es\pwindex{Glümer, Marie 3.\,7.\,1867 Wien – 16.\,11.\,1925 München@\textsc{Glümer, Marie} (3.\,7.\,1867 Wien – 16.\,11.\,1925 München), \emph{Schauspielerin}|pwv}}{\lemma{\textnormal{\emph{es}}}\Cendnote{\textnormal{das »süße Mädel«, Marie Glümer\pwindex{Glümer, Marie 3.\,7.\,1867 Wien – 16.\,11.\,1925 München@\textsc{Glümer, Marie} (3.\,7.\,1867 Wien – 16.\,11.\,1925 München), \emph{Schauspielerin}|pwk}}}}\label{K_L02670-12}«, das meiner{ }ſo treulich gedenkt. Und, um im Austheilen der Gnaden
               fortzufahren, Dir, mein lieber Alter, das goldene Vließ meines Erbhauſes: eine
               herzliche Umarmung!\pend
           
\pstart
           Dein {\\[\baselineskip]}treuer {\\[\baselineskip]}\spacefill\mbox{Paul Goldmann.}\pend
           \leftskip=0em{}
\pstart
           \noindent{}\textsc{À propos}: Kennſt Du wen in Paris\oindex{Paris@\textbf{Paris}, \emph{Hauptstadt}|pw}, an den Du mich empfehlen könnteſt?\pend
           \selectlanguage{ngerman}\endnumbering\briefempfaengerindex{Schnitzler, Arthur@\textsc{Schnitzler, Arthur}!zzzGoldmann, Paul@\emph{von Paul Goldmann}!1891-11-151@{15. 11. 1891}|)be}\mylabel{L02670h}  \newcommand{\dateiname}{L02670}\newcommand{\titel}{Paul Goldmann an Arthur Schnitzler, 15. 11. 1891}\newcommand{\editorInnen}{Martin Anton Müller und Laura Untner}%% latex-leseansicht-abspann.tex
%% Abspann für die Leseansicht.
%% Der Schalter \ifkorrekturansicht ist bereits durch den Vorspann gesetzt.

%% latex-abspann.tex
%% Gemeinsamer Abspann für Korrekturansicht und Leseansicht.
%% Setzt den Schalter \ifkorrekturansicht voraus (gesetzt in den
%% einbindenden Dateien latex-korrekturansicht-abspann.tex bzw.
%% latex-leseansicht-abspann.tex).
%% ---------------------------------------------------------------

\normalsize

% Das esempio-Environment wird nur in der Leseansicht benötigt
\ifkorrekturansicht\else
\newenvironment{esempio}[3]%
{
    \vspace{1.5ex}
    \rlap{\underline{#1}}
    \par
    \setlength{\parindent}{0cm}
    \nopagebreak
    \leftskip=#2cm
    \rightskip=#3cm
}
{
    \par
}
\fi

\doendnotes{C}
\bigskip
\vfill

\clearpage

\footnotesize

\ifkorrekturansicht
  \lohead{\textsc{register}}
\fi

% theindex-Environment neu definieren ohne reledmac
\makeatletter
\renewenvironment{theindex}{%
  \ifkorrekturansicht
    \section*{\indexname}%
  \else
    \subsubsection*{Index der erwähnten Entitäten}%
  \fi
  \setlength{\parindent}{0pt}%
  \setlength{\parskip}{0pt plus 0.3pt}%
  \let\item\@idxitem
}{%
  \ifkorrekturansicht\clearpage\fi
}
\makeatother

\IfFileExists{\jobname-pw.ind}{\input{\jobname-pw.ind}}{}

% Quellenangabe nur in der Leseansicht
\ifkorrekturansicht\else
% Fallback-Definitionen, falls die .tex-Datei \titel etc. nicht gesetzt hat
\providecommand{\titel}{}
\providecommand{\editorInnen}{}
\providecommand{\dateiname}{\jobname}

\vspace{3cm}

\vfill

\footnotesize
\textsc{Quelle}: \titel. Herausgegeben von {\editorInnen}. In: \emph{Arthur Schnitzler: Briefwechsel mit Autorinnen und Autoren}.
 Digitale Edition, https://schnitzler-briefe.acdh.oeaw.ac.at/{\dateiname}.html (Stand \today)
\fi

\end{document}


