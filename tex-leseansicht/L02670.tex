%% latex-leseansicht-vorspann.tex
%% Vorspann für die Leseansicht.
%% Lädt die gemeinsame Datei latex-vorspann.tex mit nicht gesetztem Schalter.

\newif\ifkorrekturansicht
\korrekturansichtfalse

\input{../tex-inputs/latex-vorspann}


         
         \renewcommand{\erwaehntePersonen}{Personen: Rudolf Baumbach, Richard Beer-Hofmann, Oskar Blumenthal, Max Eugen Burckhard, Max Devrient, Friedrich Michael Fels, Hermann Fürst, Marie Glümer, Paul Goldmann, Hugo von Hofmannsthal, Jaques Joachim, Eduard Michael Kafka, Friedrich Kapper, Bertha Karlsburg, Hugo Klein, Felix Salten, Edmund Wengraf}
         \renewcommand{\erwaehnteInstitutionen}{Institutionen: Burgtheater, Frankfurter Zeitung, »Freie Bühne« Verein für moderne Literatur}
         \renewcommand{\erwaehnteOrte}{Orte: Berlin, Brüssel, Paris, Wien, rue des Plantes}
         \renewcommand{\erwaehnteWerke}{Werke: Alkandi’s Lied, Am Flügel, An die Alten, Das Märchen. Schauspiel in drei Aufzügen, Die Presse, Freie Bühne für den Entwickelungskampf der Zeit, Tagebuch, Wiener Brief, »Freie Bühne«}
               \section[Paul Goldmann an Arthur Schnitzler, 15. 11. 1891]{ Paul Goldmann an Arthur Schnitzler, 15. 11. 1891}\nopagebreak\mylabel{v}\rehead{ }\begin{ledgroupsized}[t]{13cm}\normalsize\beginnumbering \toendnotes[C]{\smallbreak\pagebreak[2]} \Standort{DLA, A:Schnitzler, HS.NZ85.1.3162.}
\physDesc{Brief, 2 Blätter, 8 Seiten, 5549 Zeichen
\newline{}Handschrift: blaue Tinte, deutsche Kurrent
\newline{}Schnitzler: mit rotem Buntstift eine Unterstreichung und eine seitliche Markierung }\toendnotes[C]{\smallbreak}\pstart
           \noindent{}\centering{}{\pb}\textcolor{gray}{\textbf{Dr. jur. Paul Goldmann}}\pend
           \pstart
           \noindent{}\centering{}\textcolor{gray}{\textbf{\begin{otherlanguage}{french}Correspondant de la »Gazette de Francfort\orgindex{Frankfurter Zeitung@Frankfurter Zeitung|pw}«\end{otherlanguage}}}\pend
           \pstart
           \noindent{}\centering{}\textcolor{gray}{\textbf{\begin{otherlanguage}{french}Bruxelles, 21, rue des Plantes\end{otherlanguage}\oindex{rue des Plantes@\textbf{rue des Plantes}|pw}.}}\pend
           \pstart
           \raggedleft{}Brüſſel\oindex{Bruessel@\textbf{Brüssel}|pw},
                  15. November 1891.\pend
           \pstart\center{}Mein lieber Arthur!\pend\pstart
           Der Dank für Deine lieben Briefe, die mich unendlich erfreut haben, brennt mir ſchon
               lange auf dem Herzen. Aber eine große Affaire, die ſeit ein paar Wochen im Zuge iſt,
               hat mir bisher die Hände gebunden. Heut iſt es entſchieden: in 14 Tagen gehe ich nach
                  Paris\oindex{Paris@\textbf{Paris}|pw} als politiſcher und literariſcher
               Correſpondent der »Frankfurter Zeitung\orgindex{Frankfurter Zeitung@Frankfurter Zeitung|pw}«.
               Äußerlich recht ehrenvoll. Innerlich, unter uns, nur ein Verſuch ſeitens des Blatt\orgindex{Frankfurter Zeitung@Frankfurter Zeitung|pwv}es, eine billige junge
               Kraft in zehnfachem Maße auszubeuten als bisher. Die Arbeit in Paris\oindex{Paris@\textbf{Paris}|pw} wächſt in’s Unendliche, desgleichen die
               Verantwortlichkeit; keiner der früheren Correſpondenten {\pb}hat ſich noch länger als drei Jahre halten können.
               In Bezug auf den Gehalt werde ich wahrſcheinlich betrogen werden; die kleine Erhöhung
               gegen bisher wird durch die theuren Lebensverhältniſſe aufgewogen; von meinem
               einzigen Ziel, zur Selbſtändigkeit zu \strikeout{g\textcolor{gray}{l}} gelangen, bin ich alſo ferner als je. Und bei meinem Ekel vor der Politik, der
               ſich hier noch \strikeout{ac} accentuirt hat, bei meiner Ignoranz
               in der franzöſiſchen Sprache, bei meinem Hang zur ruhigen, \strikeout{\textcolor{gray}{ſt}} friedlichen, langſamen Arbeit habe ich alle Ausſichten, mich nicht zu bewähren
               und nicht zum Wohlbehagen zu gelangen. Ich gehe morgen
               von hier fort. Die Stadt\oindex{Bruessel@\textbf{Brüssel}|pwv} iſt
               mir in den letzten Wochen lieb geworden; ich war im Begriff, mein \textsc{Milieu} zu finden. Und im Augenblick, wo ich mich hübſch
               behaglich in eine warme Ecke drücken will, {\pb}\strikeout{reißt} reißt das Leben die Thür auf, zwingt mir wieder
               den Wanderſtab \strikeout{heraus} in die Hand und ſtößt mich in
               den Sturm und Regen der Landſtraße hinaus. Gott weiß allein, was er mit mir
               vorhat.\pend
           \pstart
           Vielleicht finde ich vor meiner Abreiſe von hier noch Zeit, Dir ausführlich zu
               ſchreiben. Einſtweilen laß’ Dir mit einem innigen Dankwort genügen für den
               Wärmeſtrom, den Du mit Deinen lieben Briefen in mein Herz geleitet. Was mich im
               Beſonderen für Dich erfreut, das iſt ein gewiſſer Hauch von Arbeitsfreude, der daraus
               hervorweht. Wenn das keine vorübergehende Stimmung, ſondern ein bleibender
               Seelenzuſtand iſt, ſo gibt es kein noch ſo hohes Ziel, deſſen Erreichung ich für Dich
               nicht erhoffe. Einer Sorge möchte ich gleich hier Ausdruck verleihen: \strikeout{ich} die \label{K_L02670-1v}\edtext{Bedenken}{\lemma{\textnormal{\emph{Bedenken}}}\Cendnote{\textnormal{Siehe Paul Goldmann an Arthur Schnitzler, 4. 8. 1891. Am 28. 10. 1891 hatte der
                  erste (und letzte) »gesellige Abend«, wie er genannt wurde, stattgefunden. Bei
                  diesem hatte Max Devrient\pwindex{Devrient, Max 12.12.1857 – 13.06.1929@\textsc{Devrient, Max} (12.12.1857 – 13.06.1929), \emph{Regisseur, Schauspieler}|pwk} von Schnitzler\pwindex{Schnitzler, Arthur 15.05.1862 – 21.10.1931@\textsc{Schnitzler, Arthur} (15.05.1862 – 21.10.1931), \emph{Schriftsteller, Mediziner}|pwk} zwei Gedichte\pwindex{Schnitzler, Arthur 15.05.1862 – 21.10.1931@\textsc{Schnitzler, Arthur} (15.05.1862 – 21.10.1931), \emph{Schriftsteller, Mediziner}!Am Fluegel1. 1. 1889@\strich\emph{Am Flügel} {[}1. 1. 1889{]}|pwkv}\pwindex{An die Alten1890@\emph{An die Alten} {[}1890{]}|pwkv} rezitiert: \emph{Am Flügel}\pwindex{Schnitzler, Arthur 15.05.1862 – 21.10.1931@\textsc{Schnitzler, Arthur} (15.05.1862 – 21.10.1931), \emph{Schriftsteller, Mediziner}!Am Fluegel1. 1. 1889@\strich\emph{Am Flügel} {[}1. 1. 1889{]}|pwk} und \emph{An die
                     Alten}\pwindex{An die Alten1890@\emph{An die Alten} {[}1890{]}|pwk}. Schnitzler\pwindex{Schnitzler, Arthur 15.05.1862 – 21.10.1931@\textsc{Schnitzler, Arthur} (15.05.1862 – 21.10.1931), \emph{Schriftsteller, Mediziner}|pwk} dürfte Goldmann\pwindex{Goldmann, Paul 31.01.1865 – 25.09.1935@\textsc{Goldmann, Paul} (31.01.1865 – 25.09.1935), \emph{Schriftsteller, Journalist}|pwk} davon in einem Brief berichtet
                  haben.}}}\label{K_L02670-1h}, welche {\pb}ich gegen das Bodenfaſſen
               der »Freien-Bühne\orgindex{»Freie Buehne« Verein fuer moderne Literatur@»Freie Bühne« Verein für moderne Literatur|pw}«-Bewegung gehabt, ſind jetzt in
               mir faſt zur negativen Gewißheit erwachſen. Die \label{K_L02670-2v}\edtext{Macher der Bewegung\orgindex{»Freie Buehne« Verein fuer moderne Literatur@»Freie Bühne« Verein für moderne Literatur|pwv}}{\lemma{\textnormal{\emph{Macher der Bewegung}}}\Cendnote{\textnormal{Am 7. 7. 1891 hatte die Gründungssitzung der
                     \emph{Freien Bühne}\orgindex{»Freie Buehne« Verein fuer moderne Literatur@»Freie Bühne« Verein für moderne Literatur|pwk} stattgefunden, einem »Verein für moderne
                        Literatur\orgindex{»Freie Buehne« Verein fuer moderne Literatur@»Freie Bühne« Verein für moderne Literatur|pwv}«. Zum Obmann\pwindex{Fels, Friedrich Michael *~1864@\textsc{Fels, Friedrich Michael} (*~1864), \emph{Journalist}|pwkv} war Friedrich Michael Fels\pwindex{Fels, Friedrich Michael *~1864@\textsc{Fels, Friedrich Michael} (*~1864), \emph{Journalist}|pwk}
                  gewählt worden, Stellvertreter\pwindex{Wengraf, Edmund 09.01.1860 – 08.12.1933@\textsc{Wengraf, Edmund} (09.01.1860 – 08.12.1933), \emph{Journalist}|pwkv}
                  wurden Edmund Wengraf\pwindex{Wengraf, Edmund 09.01.1860 – 08.12.1933@\textsc{Wengraf, Edmund} (09.01.1860 – 08.12.1933), \emph{Journalist}|pwk}\pwindex{Fuerst, Hermann 1849-07-21 – 1895-01-17@\textsc{Fürst, Hermann} (1849-07-21 – 1895-01-17), \emph{Schriftsteller}|pwk} und Hermann Fürst\pwindex{Fuerst, Hermann 1849-07-21 – 1895-01-17@\textsc{Fürst, Hermann} (1849-07-21 – 1895-01-17), \emph{Schriftsteller}|pwk}. Schnitzler\pwindex{Schnitzler, Arthur 15.05.1862 – 21.10.1931@\textsc{Schnitzler, Arthur} (15.05.1862 – 21.10.1931), \emph{Schriftsteller, Mediziner}|pwk} war Ausschussmitglied\pwindex{Schnitzler, Arthur 15.05.1862 – 21.10.1931@\textsc{Schnitzler, Arthur} (15.05.1862 – 21.10.1931), \emph{Schriftsteller, Mediziner}|pwkv} des Verein\orgindex{»Freie Buehne« Verein fuer moderne Literatur@»Freie Bühne« Verein für moderne Literatur|pwkv}s. Siehe Paul Goldmann an Arthur Schnitzler, 4. 8. 1891.
               }}}\label{K_L02670-2h} ſind \strikeout{zu} theils zu wenig erfahren, theils zu
               wenig begabt, theils zu wenig ehrlich; und der blöde Widerſtand des Publicums wie
               ſeiner Lakaien, der »Kritiker«, iſt auf dieſe Weiſe nicht zu brechen. Die \textsc{Wengraf\pwindex{Wengraf, Edmund 09.01.1860 – 08.12.1933@\textsc{Wengraf, Edmund} (09.01.1860 – 08.12.1933), \emph{Journalist}|pw}s etc.} ſind die Schlauen,
               welche Wind \strikeout{h} davon haben und beizeiten ihren Einſatz
               aus dem Spiele ziehen. Denen werden wahrſcheinlich noch Andere folgen. Nun möchte ich
               um Alles in der Welt nicht, daß Du das Opfer Deiner makelloſen Ehrlichkeit wirſt und
               Deinen guten Namen an eine Sache hefteſt, die ihn bei ihrem Zuſammenbruch ſchwer
               compromittiren könnte. Ein Martyrium für die gute Sache – {\pb}meinetwegen! Aber die Sache iſt nicht gut – dieſe
               Sache der \textsc{Joachim\pwindex{Joachim, Jaques 24.11.1866 – 07.11.1925@\textsc{Joachim, Jaques} (24.11.1866 – 07.11.1925), \emph{Wissenschaftler, Rechtsanwalt, Herausgeber}|pw}s}, \textsc{Kafka\pwindex{Kafka, Eduard Michael 11.03.1869 – 06.08.1893@\textsc{Kafka, Eduard Michael} (11.03.1869 – 06.08.1893), \emph{Redakteur}|pw}s }\textsc{etc.} Und darum meine ich: wenn die Unternehmung\orgindex{»Freie Buehne« Verein fuer moderne Literatur@»Freie Bühne« Verein für moderne Literatur|pwv} nicht unbedingte Ausſicht auf
                  \label{K_L02670-3v}\edtext{Gedeihen}{\lemma{\textnormal{\emph{Gedeihen}}}\Cendnote{\textnormal{Tatsächlich kriselte es in der \emph{Freien Bühne}\orgindex{»Freie Buehne« Verein fuer moderne Literatur@»Freie Bühne« Verein für moderne Literatur|pwk} bereits wenige Wochen nach der Gründung. In einem Theaterbrief\pwindex{Fels, Friedrich Michael *~1864@\textsc{Fels, Friedrich Michael} (*~1864), \emph{Journalist}!Wiener Brief1892-02-01@\strich\emph{Wiener Brief} {[}1892-02-01{]}|pwkv} begründete Friedrich Michael Fels\pwindex{Fels, Friedrich Michael *~1864@\textsc{Fels, Friedrich Michael} (*~1864), \emph{Journalist}|pwk} das Scheitern des Verein\orgindex{»Freie Buehne« Verein fuer moderne Literatur@»Freie Bühne« Verein für moderne Literatur|pwkv}s damit, dass zu wenig
                  der geplanten Vorhaben umgesetzt wurden und außer dem einen »geselligen Abend«
                  nichts zustande kam. Vgl. Friedrich Michael Fels\pwindex{Fels, Friedrich Michael *~1864@\textsc{Fels, Friedrich Michael} (*~1864), \emph{Journalist}|pwk}: \emph{Wiener Brief}\pwindex{Fels, Friedrich Michael *~1864@\textsc{Fels, Friedrich Michael} (*~1864), \emph{Journalist}!Wiener Brief1892-02-01@\strich\emph{Wiener Brief} {[}1892-02-01{]}|pwk}. In: \emph{Freie
                        Bühne für den Entwickelungskampf der Zeit}\pwindex{Freie Buehne fuer den Entwickelungskampf der Zeit1892 – 1893@\emph{Freie Bühne für den Entwickelungskampf der Zeit} {[}1892 – 1893{]}|pwk}, Jg. 3, H. 1, Februar 1892, S. 197–201.}}}\label{K_L02670-3h} bietet;
               wenn Du nicht ſelbſt unumſchränkt leiten kannſt – ſo zieh’ auch Du Dich ein wenig
               zurück. Du brauchſt, weiß Gott, keine Partei\orgindex{»Freie Buehne« Verein fuer moderne Literatur@»Freie Bühne« Verein für moderne Literatur|pwv} und biſt ſtark genug, deine eigenen Wege zu gehen.
               Eine \label{K_L02670-4v}\edtext{Aufführung des »Märchen\pwindex{Schnitzler, Arthur 15.05.1862 – 21.10.1931@\textsc{Schnitzler, Arthur} (15.05.1862 – 21.10.1931), \emph{Schriftsteller, Mediziner}!Maerchen. Schauspiel in drei Aufzuegen1893-12-01@\strich\emph{Das Märchen. Schauspiel in drei Aufzügen} {[}1893-12-01{]}|pw}«}{\lemma{\textnormal{\emph{Aufführung des »Märchen«}}}\Cendnote{\textnormal{\emph{Das Märchen}\pwindex{Schnitzler, Arthur 15.05.1862 – 21.10.1931@\textsc{Schnitzler, Arthur} (15.05.1862 – 21.10.1931), \emph{Schriftsteller, Mediziner}!Maerchen. Schauspiel in drei Aufzuegen1893-12-01@\strich\emph{Das Märchen. Schauspiel in drei Aufzügen} {[}1893-12-01{]}|pwk} wurde eine Zeit lang – und
                  offenbar bis zur Gegenwart dieses Briefes – als Inszenierung der \emph{Freien Bühne}\orgindex{»Freie Buehne« Verein fuer moderne Literatur@»Freie Bühne« Verein für moderne Literatur|pwk} erwogen (vgl. A. S.: \emph{Tagebuch}, 13. 7. 1891). Schnitzler\pwindex{Schnitzler, Arthur 15.05.1862 – 21.10.1931@\textsc{Schnitzler, Arthur} (15.05.1862 – 21.10.1931), \emph{Schriftsteller, Mediziner}|pwk} selbst lehnte dies jedoch ab und wollte das Drama\pwindex{Schnitzler, Arthur 15.05.1862 – 21.10.1931@\textsc{Schnitzler, Arthur} (15.05.1862 – 21.10.1931), \emph{Schriftsteller, Mediziner}!Maerchen. Schauspiel in drei Aufzuegen1893-12-01@\strich\emph{Das Märchen. Schauspiel in drei Aufzügen} {[}1893-12-01{]}|pwkv} am \emph{Burgtheater}\orgindex{Burgtheater@Burgtheater|pwk} aufgeführt wissen.}}}\label{K_L02670-4h} durch die »Freie Bühne\orgindex{»Freie Buehne« Verein fuer moderne Literatur@»Freie Bühne« Verein für moderne Literatur|pw}«, wenn nicht ganz vorzügliche ſchauſpieleriſche
               Kräfte geſichert ſind, hielte ich für eine große Gefahr. Das Publicum iſt zu dumm, um
               das Stück\pwindex{Schnitzler, Arthur 15.05.1862 – 21.10.1931@\textsc{Schnitzler, Arthur} (15.05.1862 – 21.10.1931), \emph{Schriftsteller, Mediziner}!Maerchen. Schauspiel in drei Aufzuegen1893-12-01@\strich\emph{Das Märchen. Schauspiel in drei Aufzügen} {[}1893-12-01{]}|pwv} zu begreifen; und
               auf der andern Seite mangelt der »Freien Bühne\orgindex{»Freie Buehne« Verein fuer moderne Literatur@»Freie Bühne« Verein für moderne Literatur|pw}«
                  {\pb}in Wien\oindex{Wien@\textbf{Wien}|pw} die
               Autorität, welche, als Surrogat des Verſtändniſſes, das dumme Volk zum Beifall
               zwingt. Nach dem von den »führenden Geiſtern« der Preſſe ausgehenden Loſungswort wird
               jeder Lausbub ſich berechtigt glauben, Kritik zu üben; und die Zeitungen werden Dich
               zerreißen oder mit, \strikeout{\textcolor{gray}{g}} vernichtendem Wohlwollen behandeln. (\label{K_L02670-5v}\edtext{\textsc{N. B.}}{\lemma{\textnormal{\emph{N. B.}}}\Cendnote{\textnormal{nota bene, lateinisch: merke
                  wohl}}}\label{K_L02670-5h}{ }\label{K_L02670-6v}\edtext{\textsc{Hugo Klein\pwindex{Klein, Hugo 21.07.1853 – 29.06.1915@\textsc{Klein, Hugo} (21.07.1853 – 29.06.1915), \emph{Schriftsteller, Kritiker, Journalist}|pw}s}{ }Artikel\pwindex{Freie Buehne«1891-10-30@\emph{»Freie Bühne«} {[}1891-10-30{]}|pw}}{\lemma{\textnormal{\emph{Hugo Kleins Artikel}}}\Cendnote{\textnormal{h. k.\pwindex{Klein, Hugo 21.07.1853 – 29.06.1915@\textsc{Klein, Hugo} (21.07.1853 – 29.06.1915), \emph{Schriftsteller, Kritiker, Journalist}|pwkv} [ = Hugo Klein\pwindex{Klein, Hugo 21.07.1853 – 29.06.1915@\textsc{Klein, Hugo} (21.07.1853 – 29.06.1915), \emph{Schriftsteller, Kritiker, Journalist}|pwk}]: \emph{»Freie Bühne«}\pwindex{Freie Buehne«1891-10-30@\emph{»Freie Bühne«} {[}1891-10-30{]}|pwk}. In: \emph{Die
                        Presse}\pwindex{?? Werk@Nicht ermittelte Verfasserinnen und Verfasser!Presse1848-07-03@\emph{Die Presse} {[}1848-07-03{]}|pwk}, Jg. 44, Nr. 298, 30. 10. 1891,
                     S. 9. Klein\pwindex{Klein, Hugo 21.07.1853 – 29.06.1915@\textsc{Klein, Hugo} (21.07.1853 – 29.06.1915), \emph{Schriftsteller, Kritiker, Journalist}|pwk} äußerte sich darin
                  satirisch-kritisch über den ersten Vortragsabend der \emph{Freien Bühne}\orgindex{»Freie Buehne« Verein fuer moderne Literatur@»Freie Bühne« Verein für moderne Literatur|pwk} am 28. 10. 1891. Schnitzler\pwindex{Schnitzler, Arthur 15.05.1862 – 21.10.1931@\textsc{Schnitzler, Arthur} (15.05.1862 – 21.10.1931), \emph{Schriftsteller, Mediziner}|pwk} erwähnte er folgendermaßen: »zwei
                        Gedichte\pwindex{Schnitzler, Arthur 15.05.1862 – 21.10.1931@\textsc{Schnitzler, Arthur} (15.05.1862 – 21.10.1931), \emph{Schriftsteller, Mediziner}!Am Fluegel1. 1. 1889@\strich\emph{Am Flügel} {[}1. 1. 1889{]}|pwv}\pwindex{An die Alten1890@\emph{An die Alten} {[}1890{]}|pwv}
                     von Arthur Schnitzler\pwindex{Schnitzler, Arthur 15.05.1862 – 21.10.1931@\textsc{Schnitzler, Arthur} (15.05.1862 – 21.10.1931), \emph{Schriftsteller, Mediziner}|pw}, von welchen
                     besonders das eine: ›Am Flügel\pwindex{Schnitzler, Arthur 15.05.1862 – 21.10.1931@\textsc{Schnitzler, Arthur} (15.05.1862 – 21.10.1931), \emph{Schriftsteller, Mediziner}!Am Fluegel1. 1. 1889@\strich\emph{Am Flügel} {[}1. 1. 1889{]}|pw}‹,
                     unverkennbar den Einfluß Baumbach\pwindex{Baumbach, Rudolf 28.09.1840 – 21.09.1905@\textsc{Baumbach, Rudolf} (28.09.1840 – 21.09.1905), \emph{Schriftsteller}|pw}’s
                     widerspiegelt«. Siehe A. S.: \emph{Tagebuch}, 30. 10. 1891.
               }}}\label{K_L02670-6h} habe ich geleſen; wäre ich in Wien\oindex{Wien@\textbf{Wien}|pw}
               geweſen, ich hätte den Burſchen\pwindex{Klein, Hugo 21.07.1853 – 29.06.1915@\textsc{Klein, Hugo} (21.07.1853 – 29.06.1915), \emph{Schriftsteller, Kritiker, Journalist}|pwv} geohrfeigt, allein wegen der Stelle über Dich!). Etwas Anderes wäre
               die Aufführung in Berlin\oindex{Berlin@\textbf{Berlin}|pw}. Kein ſicherer Erfolg
               freilich; aber dort wirſt Du wenigſtens von Einigen ſo ernſt genommen werden, als Du
               es verdienſt. Ich halte es für das Beſte, die \strikeout{\textcolor{gray}{Aufführu}}{ }\label{K_L02670-7v}\edtext{Antwort \textsc{Blumenthal\pwindex{Blumenthal, Oskar 13.03.1852 – 24.04.1917@\textsc{Blumenthal, Oskar} (13.03.1852 – 24.04.1917), \emph{Schriftsteller, Journalist, Theaterleiter}|pw}s}}{\lemma{\textnormal{\emph{Antwort Blumenthals}}}\Cendnote{\textnormal{siehe Oscar Blumenthal an Arthur Schnitzler, 15. 12. 1891}}}\label{K_L02670-7h} abzuwarten und {\pb}vorher in Wien\oindex{Wien@\textbf{Wien}|pw} nicht einen Schritt zu thun. In \textsc{Burckhard\pwindex{Burckhard, Max Eugen 14.07.1854 – 16.03.1912@\textsc{Burckhard, Max Eugen} (14.07.1854 – 16.03.1912), \emph{Schriftsteller, Wissenschaftler, Theaterleiter}|pw}s}{ }\label{K_L02670-8v}\edtext{Antwort}{\lemma{\textnormal{\emph{Antwort}}}\Cendnote{\textnormal{Schnitzler\pwindex{Schnitzler, Arthur 15.05.1862 – 21.10.1931@\textsc{Schnitzler, Arthur} (15.05.1862 – 21.10.1931), \emph{Schriftsteller, Mediziner}|pwk} hatte die Nachricht, dass Max Burckhard\pwindex{Burckhard, Max Eugen 14.07.1854 – 16.03.1912@\textsc{Burckhard, Max Eugen} (14.07.1854 – 16.03.1912), \emph{Schriftsteller, Wissenschaftler, Theaterleiter}|pwk}{ }\emph{Das Märchen}\pwindex{Schnitzler, Arthur 15.05.1862 – 21.10.1931@\textsc{Schnitzler, Arthur} (15.05.1862 – 21.10.1931), \emph{Schriftsteller, Mediziner}!Maerchen. Schauspiel in drei Aufzuegen1893-12-01@\strich\emph{Das Märchen. Schauspiel in drei Aufzügen} {[}1893-12-01{]}|pwk} nicht am \emph{Burgtheater}\orgindex{Burgtheater@Burgtheater|pwk} inszenieren werde, am 28. 10. 1891 erhalten. Sie dürfte eher
                  mündlich als schriftlich mitgeteilt worden sein. Jedenfalls hat sich kein
                  entsprechendes Korrespondenzstück erhalten. Als Begründung notierte sich Schnitzler\pwindex{Schnitzler, Arthur 15.05.1862 – 21.10.1931@\textsc{Schnitzler, Arthur} (15.05.1862 – 21.10.1931), \emph{Schriftsteller, Mediziner}|pwk} im \emph{Tagebuch}\pwindex{Schnitzler, Arthur 15.05.1862 – 21.10.1931@\textsc{Schnitzler, Arthur} (15.05.1862 – 21.10.1931), \emph{Schriftsteller, Mediziner}!Tagebuch1981 – 2000@\strich\emph{Tagebuch} {[}1981 – 2000{]}|pwk}: »zu viel Rede, zu wenig Handlung«.}}}\label{K_L02670-8h} liegt,
               trotz der \label{K_L02670-9v}\edtext{literariſch-ungebildeten
                  Form}{\lemma{\textnormal{\emph{literariſch-ungebildeten Form}}}\Cendnote{\textnormal{Anspielung darauf, dass Burckhard\pwindex{Burckhard, Max Eugen 14.07.1854 – 16.03.1912@\textsc{Burckhard, Max Eugen} (14.07.1854 – 16.03.1912), \emph{Schriftsteller, Wissenschaftler, Theaterleiter}|pwk}{ }Jurist\pwindex{Burckhard, Max Eugen 14.07.1854 – 16.03.1912@\textsc{Burckhard, Max Eugen} (14.07.1854 – 16.03.1912), \emph{Schriftsteller, Wissenschaftler, Theaterleiter}|pwkv} war und ohne
                  künstlerisch-artistische Vorerfahrung die Leitung des \emph{Burgtheater}\orgindex{Burgtheater@Burgtheater|pwk}s überantwortet bekommen hatte.}}}\label{K_L02670-9h},
               vielleicht ein geſunder Inſtinct. Du hätteſt ihm unter allen Umſtänden \label{K_L02670-10v}\edtext{zuerſt den \textsc{Alkandi\pwindex{Schnitzler, Arthur 15.05.1862 – 21.10.1931@\textsc{Schnitzler, Arthur} (15.05.1862 – 21.10.1931), \emph{Schriftsteller, Mediziner}!Alkandi s Lied15.8.1890 – 1.9.1890@\strich\emph{Alkandi’s Lied} {[}15.8.1890 – 1.9.1890{]}|pw}}}{\lemma{\textnormal{\emph{zuerſt den Alkandi}}}\Cendnote{\textnormal{Diesen Einakter\pwindex{Schnitzler, Arthur 15.05.1862 – 21.10.1931@\textsc{Schnitzler, Arthur} (15.05.1862 – 21.10.1931), \emph{Schriftsteller, Mediziner}!Alkandi s Lied15.8.1890 – 1.9.1890@\strich\emph{Alkandi’s Lied} {[}15.8.1890 – 1.9.1890{]}|pwkv} hatte Max Burckhard\pwindex{Burckhard, Max Eugen 14.07.1854 – 16.03.1912@\textsc{Burckhard, Max Eugen} (14.07.1854 – 16.03.1912), \emph{Schriftsteller, Wissenschaftler, Theaterleiter}|pwk} bereits am 14. 7. 1891 abgelehnt (vgl. Max Burckhard an Arthur Schnitzler, 14. 7. 1891).}}}\label{K_L02670-10h} geben ſollen; und ich rathe Dir
               entſchieden, es auch jetzt noch zu thun. Bringt er das Stück\pwindex{Schnitzler, Arthur 15.05.1862 – 21.10.1931@\textsc{Schnitzler, Arthur} (15.05.1862 – 21.10.1931), \emph{Schriftsteller, Mediziner}!Alkandi s Lied15.8.1890 – 1.9.1890@\strich\emph{Alkandi’s Lied} {[}15.8.1890 – 1.9.1890{]}|pwv} und gefällt es, ſo wäre es gar nicht
               unmöglich, daß er noch auf das »Märchen\pwindex{Schnitzler, Arthur 15.05.1862 – 21.10.1931@\textsc{Schnitzler, Arthur} (15.05.1862 – 21.10.1931), \emph{Schriftsteller, Mediziner}!Maerchen. Schauspiel in drei Aufzuegen1893-12-01@\strich\emph{Das Märchen. Schauspiel in drei Aufzügen} {[}1893-12-01{]}|pw}«
               zurückkäme. Im Übrigen behalte ich mir alle näheren Urtheile bis nach der Lectüre
               vor, die ich aufrichtigſt herbeiwünſche.\pend
           \pstart
           Dies für heut. Tauſend Dank noch für die Beantwortung
               meiner Fragen, die ausführlichen Mittheilungen über die Lieben in Wien\oindex{Wien@\textbf{Wien}|pw}, und all’ das Gütige und Freundſchaftliche, das Deine {\pb}Briefe ſonſt noch enthalten haben. Sie waren mir
               eine Art Feſtgeſchenk. Ehe ich von hier ſcheide (ich fahre etwa am 30. November) höre ich wohl noch ein Wort von Dir? Viele,
               viele Grüße an die Wien\oindex{Wien@\textbf{Wien}|pw}er Freunde, vor Allem \textsc{Richard\pwindex{Beer-Hofmann, Richard 1866-07-11 – 1945-09-26@\textsc{Beer-Hofmann, Richard} (1866-07-11 – 1945-09-26), \emph{Schriftsteller}|pw}} und \textsc{Loris\pwindex{Hofmannsthal, Hugo von 1874-02-01 – 1929-07-15@\textsc{Hofmannsthal, Hugo von} (1874-02-01 – 1929-07-15), \emph{Schriftsteller}|pw}} und \textsc{Kapper\pwindex{Kapper, Friedrich 21.04.1861 – 22.07.1939@\textsc{Kapper, Friedrich} (21.04.1861 – 22.07.1939), \emph{Mediziner}|pw}}. Einen herzlichen Händedruck an \textsc{Salten\pwindex{Salten, Felix 06.09.1869 – 08.10.1945@\textsc{Salten, Felix} (06.09.1869 – 08.10.1945), \emph{Schriftsteller, Journalist}|pw}}, der mein ſeeliger \label{K_L02670-11v}\edtext{Erbe\pwindex{Karlsburg, Bertha @\textsc{Karlsburg, Bertha}, \emph{Schauspielerin}|pwv} auf dem gewiſſen mit
               Kiſſen }{\lemma{\textnormal{\emph{Erbe … Kiſſen}}}\Cendnote{\textnormal{vermutlich Bezug auf Bertha Karlsburg\pwindex{Karlsburg, Bertha @\textsc{Karlsburg, Bertha}, \emph{Schauspielerin}|pwk}, mit der Salten\pwindex{Salten, Felix 06.09.1869 – 08.10.1945@\textsc{Salten, Felix} (06.09.1869 – 08.10.1945), \emph{Schriftsteller, Journalist}|pwk} ein Verhältnis hatte}}}\label{K_L02670-11h} weich drapirten Sopha
               geworden zu ſein ſcheint. Ergebene Empfehlungen an die Deinen. Vielen Dank und Gruß
               an »\label{K_L02670-12v}\edtext{es\pwindex{Gluemer, Marie 03.07.1867 – 16.11.1925@\textsc{Glümer, Marie} (03.07.1867 – 16.11.1925), \emph{Schauspielerin}|pwv}}{\lemma{\textnormal{\emph{es}}}\Cendnote{\textnormal{das »süße Mädel«, Marie Glümer\pwindex{Gluemer, Marie 03.07.1867 – 16.11.1925@\textsc{Glümer, Marie} (03.07.1867 – 16.11.1925), \emph{Schauspielerin}|pwk}}}}\label{K_L02670-12h}«, das meiner ſo treulich gedenkt. Und, um im Austheilen der Gnaden
               fortzufahren, Dir, mein lieber Alter, das goldene Vließ meines Erbhauſes: eine
               herzliche Umarmung!\pend
           \pstart
           Dein {\\[\baselineskip]}treuer {\\[\baselineskip]}\spacefill\mbox{Paul Goldmann.}\pend
           \leftskip=0em{}\pstart
           \noindent{}\textsc{À propos}: Kennſt Du wen in Paris\oindex{Paris@\textbf{Paris}|pw}, an den Du mich empfehlen könnteſt?\pend
           
         
         \endnumbering\mylabel{h}\end{ledgroupsized}  \newcommand{\dateiname}{L02670}\newcommand{\titel}{Paul Goldmann an Arthur Schnitzler, 15. 11. 1891}\newcommand{\editorInnen}{Martin Anton Müller und Laura Untner}%% latex-leseansicht-abspann.tex
%% Abspann für die Leseansicht.
%% Der Schalter \ifkorrekturansicht ist bereits durch den Vorspann gesetzt.

%% latex-abspann.tex
%% Gemeinsamer Abspann für Korrekturansicht und Leseansicht.
%% Setzt den Schalter \ifkorrekturansicht voraus (gesetzt in den
%% einbindenden Dateien latex-korrekturansicht-abspann.tex bzw.
%% latex-leseansicht-abspann.tex).
%% ---------------------------------------------------------------

\normalsize

% Das esempio-Environment wird nur in der Leseansicht benötigt
\ifkorrekturansicht\else
\newenvironment{esempio}[3]%
{
    \vspace{1.5ex}
    \rlap{\underline{#1}}
    \par
    \setlength{\parindent}{0cm}
    \nopagebreak
    \leftskip=#2cm
    \rightskip=#3cm
}
{
    \par
}
\fi

\doendnotes{C}
\bigskip
\vfill

\clearpage

\footnotesize

\ifkorrekturansicht
  \lohead{\textsc{register}}
\fi

% theindex-Environment neu definieren ohne reledmac
\makeatletter
\renewenvironment{theindex}{%
  \ifkorrekturansicht
    \section*{\indexname}%
  \else
    \subsubsection*{Index der erwähnten Entitäten}%
  \fi
  \setlength{\parindent}{0pt}%
  \setlength{\parskip}{0pt plus 0.3pt}%
  \let\item\@idxitem
}{%
  \ifkorrekturansicht\clearpage\fi
}
\makeatother

\IfFileExists{\jobname-pw.ind}{\input{\jobname-pw.ind}}{}

% Quellenangabe nur in der Leseansicht
\ifkorrekturansicht\else
% Fallback-Definitionen, falls die .tex-Datei \titel etc. nicht gesetzt hat
\providecommand{\titel}{}
\providecommand{\editorInnen}{}
\providecommand{\dateiname}{\jobname}

\vspace{3cm}

\vfill

\footnotesize
\textsc{Quelle}: \titel. Herausgegeben von {\editorInnen}. In: \emph{Arthur Schnitzler: Briefwechsel mit Autorinnen und Autoren}.
 Digitale Edition, https://schnitzler-briefe.acdh.oeaw.ac.at/{\dateiname}.html (Stand \today)
\fi

\end{document}


      