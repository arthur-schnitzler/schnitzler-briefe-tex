%% latex-korrekturansicht-vorspann.tex
%% Vorspann für die Korrekturansicht.
%% Lädt die gemeinsame Datei latex-vorspann.tex mit gesetztem Schalter.

\newif\ifkorrekturansicht
\korrekturansichttrue

\input{../tex-inputs/latex-vorspann}


\section[ Auguste Hauschner an Arthur Schnitzler, 16. 1. 1909]{L02587 Auguste Hauschner an Arthur Schnitzler, 16. 1. 1909}
\nopagebreak\mylabel{L02587v}
\rehead{ }\normalsize\beginnumbering\briefempfaengerindex{Schnitzler, Arthur@\textsc{Schnitzler, Arthur}!zzzHauschner, Auguste@\emph{von Auguste Hauschner}!1909-01-162@{16. 1. 1909}|(be}
\toendnotes[C]{\smallbreak\pagebreak[2]}\Standort{DLA, A:Schnitzler, HS1985.1.3363.}
\physDesc{Brief, 1 Blatt, 2 Seiten, 685 Zeichen
\newline{}Handschrift: schwarze Tinte, lateinische Kurrent
\newline{}Schnitzler: 1) mit Bleistift Vermerk »\textsc{Hauschner\pwindex{Hauschner, Auguste 12.02.1850 – 10.04.1924@\textsc{Hauschner, Auguste} (12.02.1850 – 10.04.1924), \emph{Schriftsteller/Schriftstellerin}|pw}}«  2) mit rotem Buntstift eine Unterstreichung}\toendnotes[C]{\smallbreak}
\pstart
           \raggedleft{}{\pb}Berlin\oindex{Berlin@\textbf{Berlin}, \emph{P.PPLC}|pw} d. 16. 1. 09\pend
           \vspace{0.5em}
\pstart
           Sehr geehrter Herr Doctor – die »Hilfe\orgindex{Hilfe. Zeitschrift fuer Politik, Wirtschaft und geistige Bewegung@Die Hilfe. Zeitschrift für Politik, Wirtschaft und geistige Bewegung|pw}« hat meinen Beitrag\pwindex{Weg ins Freie@\emph{Der Weg ins Freie}|pwv} lange nicht gebracht, weil sie eigentlich so
               umfangreiche Buchbesprechungen sonst nicht annimmt. Sie wünschten meine Arbeit\pwindex{Weg ins Freie@\emph{Der Weg ins Freie}|pwv} zu lesen, \label{K_L02587-1v}\edtext{ich schicke sie daher}{\lemma{\textnormal{\emph{ich schicke sie daher}}}\Cendnote{\textnormal{Das 3. Heft\pwindex{Hilfe. Zeitschrift fuer Politik, Wirtschaft und geistige Bewegung@\emph{Die Hilfe. Zeitschrift für Politik, Wirtschaft und geistige Bewegung}|pwkv} des Jahres 1909, in dem die
                     \emph{Rezension}\pwindex{Weg ins Freie@\emph{Der Weg ins Freie}|pwk} abgedruckt ist, ist mit 17. 1. 1909 datiert.}}}\label{K_L02587-1}, obgleich, wie ich nun im
               Druck »sehe«, \strikeout{dass} mir der Schluss misslungen ist.
               Was mir das innerste Wesen Ihrer bedeutendsten Gestalten zu sein scheint, {\pb}der Trieb zur Vereinsamung und die Fremdheit zum Menschthum,
               habe ich, durch ein Paar untreffende Ausdrücke, zu schwer an ein einzelnes, im Grunde
               leichtlebiges, Individuum gehängt.\pend
           
\pstart
           Trotzdem werden Sie vielleicht meine innere Bewegtheit aus meinen Worten lesen
               können.\pend
           
\pstart
           Mit besondrer Hochschätzung{\\[\baselineskip]}\spacefill\mbox{Auguste Hauschner}\pend
           \leftskip=0em{}\selectlanguage{ngerman}\endnumbering\briefempfaengerindex{Schnitzler, Arthur@\textsc{Schnitzler, Arthur}!zzzHauschner, Auguste@\emph{von Auguste Hauschner}!1909-01-162@{16. 1. 1909}|)be}\mylabel{L02587h}  \normalsize

\doendnotes{C}
\bigskip
\vfill

\clearpage

\footnotesize

\lohead{\textsc{register}}

% Definiere theindex-Environment komplett neu ohne reledmac
\makeatletter
\renewenvironment{theindex}{%
  \section*{\indexname}%
  \setlength{\parindent}{0pt}%
  \setlength{\parskip}{0pt plus 0.3pt}%
  \let\item\@idxitem
}{%
  \clearpage
}
\makeatother

\IfFileExists{\jobname-pw.ind}{\input{\jobname-pw.ind}}{}

\end{document}

      