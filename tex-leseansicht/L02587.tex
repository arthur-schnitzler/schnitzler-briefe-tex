%% latex-leseansicht-vorspann.tex
%% Vorspann für die Leseansicht.
%% Lädt die gemeinsame Datei latex-vorspann.tex mit nicht gesetztem Schalter.

\newif\ifkorrekturansicht
\korrekturansichtfalse

\input{../tex-inputs/latex-vorspann}


\section[ Auguste Hauschner an Arthur Schnitzler, 16. 1. 1909]{L02587 Auguste Hauschner an Arthur Schnitzler,  16. 1. 1909}
\nopagebreak\mylabel{L02587v}
\rehead{ }\normalsize\beginnumbering\briefempfaengerindex{Schnitzler, Arthur@\textsc{Schnitzler, Arthur}!zzzHauschner, Auguste@\emph{von Auguste Hauschner}!1909-01-162@{16. 1. 1909}|(be}
\toendnotes[C]{\smallbreak\pagebreak[2]}
\correspDesc{Versand  durch Auguste Hauschner am 16. 1. 1909 in Berlin
\newline{}Erhalt  durch Arthur Schnitzler im Zeitraum [17. 1. 1909
                  – 21. 1. 1909?] in Wien}\toendnotes[C]{\smallbreak}
\Standort{DLA, A:Schnitzler, HS1985.1.3363.}
\physDesc{Brief, 1 Blatt, 2 Seiten, 685 Zeichen
\newline{}Handschrift: schwarze Tinte, lateinische Kurrent
\newline{}Schnitzler: 1) mit Bleistift Vermerk »\textsc{Hauschner\pwindex{Hauschner, Auguste 12.\,2.\,1850 Prag – 10.\,4.\,1924 Berlin@\textsc{Hauschner, Auguste} (12.\,2.\,1850 Prag – 10.\,4.\,1924 Berlin), \emph{Schriftstellerin}|pw}}«  2) mit rotem Buntstift eine Unterstreichung}\toendnotes[C]{\smallbreak}
\pstart
           \raggedleft{}{\pb}Berlin\oindex{Berlin@\textbf{Berlin}, \emph{Hauptstadt}|pw} d. 16. 1. 09\pend
           \vspace{0.5em}
\pstart
           Sehr geehrter Herr Doctor – die »Hilfe\orgindex{Hilfe. Zeitschrift für Politik, Wirtschaft und geistige Bewegung@Die Hilfe. Zeitschrift für Politik, Wirtschaft und geistige Bewegung|pw}« hat meinen Beitrag\pwindex{Hauschner, Auguste 12.\,2.\,1850 Prag – 10.\,4.\,1924 Berlin@\textsc{Hauschner, Auguste} (12.\,2.\,1850 Prag – 10.\,4.\,1924 Berlin), \emph{Schriftstellerin}!Weg ins Freie@\strich\emph{Der Weg ins Freie}|pwv} lange nicht gebracht, weil sie eigentlich so
               umfangreiche Buchbesprechungen sonst nicht annimmt. Sie wünschten meine Arbeit\pwindex{Hauschner, Auguste 12.\,2.\,1850 Prag – 10.\,4.\,1924 Berlin@\textsc{Hauschner, Auguste} (12.\,2.\,1850 Prag – 10.\,4.\,1924 Berlin), \emph{Schriftstellerin}!Weg ins Freie@\strich\emph{Der Weg ins Freie}|pwv} zu lesen, \label{K_L02587-1v}\edtext{ich schicke sie daher}{\lemma{\textnormal{\emph{ich schicke sie daher}}}\Cendnote{\textnormal{Das 3. Heft\pwindex{Hilfe. Zeitschrift für Politik, Wirtschaft und geistige Bewegung@\emph{Die Hilfe. Zeitschrift für Politik, Wirtschaft und geistige Bewegung}|pwkv} des Jahres 1909, in dem die
                     \emph{Rezension}\pwindex{Hauschner, Auguste 12.\,2.\,1850 Prag – 10.\,4.\,1924 Berlin@\textsc{Hauschner, Auguste} (12.\,2.\,1850 Prag – 10.\,4.\,1924 Berlin), \emph{Schriftstellerin}!Weg ins Freie@\strich\emph{Der Weg ins Freie}|pwk} abgedruckt ist, ist mit 17. 1. 1909 datiert.}}}\label{K_L02587-1}, obgleich, wie ich nun im
               Druck »sehe«, \strikeout{dass} mir der Schluss misslungen ist.
               Was mir das innerste Wesen Ihrer bedeutendsten Gestalten zu sein scheint, {\pb}der Trieb zur Vereinsamung und die Fremdheit zum Menschthum,
               habe ich, durch ein Paar untreffende Ausdrücke, zu schwer an ein einzelnes, im Grunde
               leichtlebiges, Individuum gehängt.\pend
           
\pstart
           Trotzdem werden Sie vielleicht meine innere Bewegtheit aus meinen Worten lesen
               können.\pend
           
\pstart
           Mit besondrer Hochschätzung{\\[\baselineskip]}\spacefill\mbox{Auguste Hauschner}\pend
           \leftskip=0em{}\selectlanguage{ngerman}\endnumbering\briefempfaengerindex{Schnitzler, Arthur@\textsc{Schnitzler, Arthur}!zzzHauschner, Auguste@\emph{von Auguste Hauschner}!1909-01-162@{16. 1. 1909}|)be}\mylabel{L02587h}  \newcommand{\dateiname}{L02587}\newcommand{\titel}{Auguste Hauschner an Arthur Schnitzler, 16. 1. 1909}\newcommand{\editorInnen}{Martin Anton Müller und Laura Untner}%% latex-leseansicht-abspann.tex
%% Abspann für die Leseansicht.
%% Der Schalter \ifkorrekturansicht ist bereits durch den Vorspann gesetzt.

%% latex-abspann.tex
%% Gemeinsamer Abspann für Korrekturansicht und Leseansicht.
%% Setzt den Schalter \ifkorrekturansicht voraus (gesetzt in den
%% einbindenden Dateien latex-korrekturansicht-abspann.tex bzw.
%% latex-leseansicht-abspann.tex).
%% ---------------------------------------------------------------

\normalsize

% Das esempio-Environment wird nur in der Leseansicht benötigt
\ifkorrekturansicht\else
\newenvironment{esempio}[3]%
{
    \vspace{1.5ex}
    \rlap{\underline{#1}}
    \par
    \setlength{\parindent}{0cm}
    \nopagebreak
    \leftskip=#2cm
    \rightskip=#3cm
}
{
    \par
}
\fi

\doendnotes{C}
\bigskip
\vfill

\clearpage

\footnotesize

\ifkorrekturansicht
  \lohead{\textsc{register}}
\fi

% theindex-Environment neu definieren ohne reledmac
\makeatletter
\renewenvironment{theindex}{%
  \ifkorrekturansicht
    \section*{\indexname}%
  \else
    \subsubsection*{Index der erwähnten Entitäten}%
  \fi
  \setlength{\parindent}{0pt}%
  \setlength{\parskip}{0pt plus 0.3pt}%
  \let\item\@idxitem
}{%
  \ifkorrekturansicht\clearpage\fi
}
\makeatother

\IfFileExists{\jobname-pw.ind}{\input{\jobname-pw.ind}}{}

% Quellenangabe nur in der Leseansicht
\ifkorrekturansicht\else
% Fallback-Definitionen, falls die .tex-Datei \titel etc. nicht gesetzt hat
\providecommand{\titel}{}
\providecommand{\editorInnen}{}
\providecommand{\dateiname}{\jobname}

\vspace{3cm}

\vfill

\footnotesize
\textsc{Quelle}: \titel. Herausgegeben von {\editorInnen}. In: \emph{Arthur Schnitzler: Briefwechsel mit Autorinnen und Autoren}.
 Digitale Edition, https://schnitzler-briefe.acdh.oeaw.ac.at/{\dateiname}.html (Stand \today)
\fi

\end{document}


