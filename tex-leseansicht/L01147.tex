%% latex-leseansicht-vorspann.tex
%% Vorspann für die Leseansicht.
%% Lädt die gemeinsame Datei latex-vorspann.tex mit nicht gesetztem Schalter.

\newif\ifkorrekturansicht
\korrekturansichtfalse

\input{../tex-inputs/latex-vorspann}


\section[Edith Brandes an Arthur Schnitzler, 15. 7. 1901]{L01147 Edith Brandes an Arthur Schnitzler, 15. 7. 1901}
\nopagebreak\mylabel{L01147v}
\rehead{ }\normalsize\beginnumbering\briefempfaengerindex{Schnitzler, Arthur@\textsc{Schnitzler, Arthur}!zzzPhilipp, Edith@\emph{von Edith Philipp}!1901-07-151@{15. 7. 1901}|(be}
\toendnotes[C]{\smallbreak\pagebreak[2]}
\correspDesc{Versand  durch Edith Brandes am 15. 7. 1901 in Skodsborg
\newline{}Weiterleitung  in Wien
\newline{}Erhalt  durch Arthur Schnitzler im Zeitraum [16. 7. 1901
                  – 20. 7. 1901?] in Vahrn}\toendnotes[C]{\smallbreak}
\Standort{CUL, Schnitzler, B 17.}
\physDesc{Brief, 1 Blatt, 3 Seiten, 700 Zeichen (Briefpapier mit aufgedruckten Tauben)
\newline{}Handschrift: schwarze Tinte, lateinische Kurrent
\newline{}Ordnung: mit Bleistift von unbekannter Hand nummeriert:
                                    »27« }\Standort{DLA, A:Schnitzler, HS.NZ85.1.2595.}
\physDesc{maschinenschriftliche Abschrift, 1 Blatt, 1 Seite, 700 Zeichen
\newline{}Schreibmaschine}
\buchAbdrucke{\weitereDrucke{Georg Brandes, Arthur Schnitzler: \emph{Ein Briefwechsel}. Herausgegeben von Kurt Bergel. Bern: \emph{Francke} 1956, S. 90.} }\toendnotes[C]{\smallbreak}
\pstart
           \raggedleft{}{\pb}Hotel Øresund. Skodsborg\oindex{Hotel Øresund@\textbf{Hotel Øresund}, \emph{Hotel}|pw}{\\}15-7-1901\pend
           
\pstart\center{}Verehrter Herr Schnitzler!\pend\vspace{0.5em}
\pstart
           Mit unendlicher Mühe habe ich Ihre freundlichen Zeilen dechiffrirt. Ich schäme mich
               ein bischen mich so als Stammbuchsdame Ihnen präsentirt zu haben; aber Sie nehmen die
               Aufgabe {\pb}zu feierlich. Sie
               brauchen nicht Ihre Bücher zu verschreiben, auch nicht geistreicher zu sein als wie
               Sie jeden Tag ohne Anstrengung sind. In meinem Album finden sich so spirituelle
               Sachen, wie »Willkommen noch einmal«! und ähnliches. Für eine beliebige Zeile bin ich
               dankbar. Es würde mir schwer fallen Ihnen zu sagen, welches von Ihren Büchern mir am
               besten gefällt\substVorne{}\textsuperscript{,}\substDazwischen{}. –\substHinten{}{ }\substVorne{}\textsuperscript{i}\substDazwischen{}I\substHinten{}n jedem findet sich so viel Schönes.\pend
           
\pstart
           {\pb}Mit besten Grüssen von meinem Papa\pwindex{Brandes, Georg 4.\,2.\,1842 Kopenhagen – 19.\,2.\,1927 ebd.@\textsc{Brandes, Georg} (4.\,2.\,1842 Kopenhagen – 19.\,2.\,1927 ebd.)|pwv} und mir{\\[\baselineskip]}\spacefill\mbox{Edith Brandes.}\pend
           \leftskip=0em{}\selectlanguage{ngerman}\endnumbering\briefempfaengerindex{Schnitzler, Arthur@\textsc{Schnitzler, Arthur}!zzzPhilipp, Edith@\emph{von Edith Philipp}!1901-07-151@{15. 7. 1901}|)be}\mylabel{L01147h}  \newcommand{\dateiname}{L01147}\newcommand{\titel}{Edith Brandes an Arthur Schnitzler, 15. 7. 1901}\newcommand{\editorInnen}{Martin Anton Müller und Gerd-Hermann Susen}%% latex-leseansicht-abspann.tex
%% Abspann für die Leseansicht.
%% Der Schalter \ifkorrekturansicht ist bereits durch den Vorspann gesetzt.

%% latex-abspann.tex
%% Gemeinsamer Abspann für Korrekturansicht und Leseansicht.
%% Setzt den Schalter \ifkorrekturansicht voraus (gesetzt in den
%% einbindenden Dateien latex-korrekturansicht-abspann.tex bzw.
%% latex-leseansicht-abspann.tex).
%% ---------------------------------------------------------------

\normalsize

% Das esempio-Environment wird nur in der Leseansicht benötigt
\ifkorrekturansicht\else
\newenvironment{esempio}[3]%
{
    \vspace{1.5ex}
    \rlap{\underline{#1}}
    \par
    \setlength{\parindent}{0cm}
    \nopagebreak
    \leftskip=#2cm
    \rightskip=#3cm
}
{
    \par
}
\fi

\doendnotes{C}
\bigskip
\vfill

\clearpage

\footnotesize

\ifkorrekturansicht
  \lohead{\textsc{register}}
\fi

% theindex-Environment neu definieren ohne reledmac
\makeatletter
\renewenvironment{theindex}{%
  \ifkorrekturansicht
    \section*{\indexname}%
  \else
    \subsubsection*{Index der erwähnten Entitäten}%
  \fi
  \setlength{\parindent}{0pt}%
  \setlength{\parskip}{0pt plus 0.3pt}%
  \let\item\@idxitem
}{%
  \ifkorrekturansicht\clearpage\fi
}
\makeatother

\IfFileExists{\jobname-pw.ind}{\input{\jobname-pw.ind}}{}

% Quellenangabe nur in der Leseansicht
\ifkorrekturansicht\else
% Fallback-Definitionen, falls die .tex-Datei \titel etc. nicht gesetzt hat
\providecommand{\titel}{}
\providecommand{\editorInnen}{}
\providecommand{\dateiname}{\jobname}

\vspace{3cm}

\vfill

\footnotesize
\textsc{Quelle}: \titel. Herausgegeben von {\editorInnen}. In: \emph{Arthur Schnitzler: Briefwechsel mit Autorinnen und Autoren}.
 Digitale Edition, https://schnitzler-briefe.acdh.oeaw.ac.at/{\dateiname}.html (Stand \today)
\fi

\end{document}


