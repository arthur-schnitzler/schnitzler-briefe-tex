%% latex-korrekturansicht-vorspann.tex
%% Vorspann für die Korrekturansicht.
%% Lädt die gemeinsame Datei latex-vorspann.tex mit gesetztem Schalter.

\newif\ifkorrekturansicht
\korrekturansichttrue

\input{../tex-inputs/latex-vorspann}


\section[Olga Schnitzler an Paula Beer-Hofmann, {[}30. 11. 1908?{]}]{L01815 Olga Schnitzler an Paula Beer-Hofmann, {[}30. 11. 1908?{]}}
\nopagebreak\mylabel{L01815v}
\rehead{ }\normalsize\beginnumbering\briefempfaengerindex{Beer-Hofmann, Paula@\textsc{Beer-Hofmann, Paula}!zzzSchnitzler, Olga@\emph{von Olga Schnitzler}!1908-11-301@{{[}30. 11. 1908?{]}}|(be}
\toendnotes[C]{\smallbreak\pagebreak[2]}\Standort{YCGL, MSS 31.}
\physDesc{Brief, 1 Blatt, 2 Seiten, Umschlag, 383 Zeichen
\newline{}Handschrift: schwarze Tinte, lateinische Kurrent
\newline{}Versand: ohne postalischen Übermittlungsvermerk }\toendnotes[C]{\smallbreak}\pstart{}{\pb}\textcolor{gray}{\textbf{Dr. Arthur Schnitzler}}\pend{}\pstart{}\textcolor{gray}{\textbf{Wien XVIII. Spoettelgasse 7\oindex{Edmund-Weiss-Gasse 7@\textbf{Edmund-Weiß-Gasse 7}, \emph{Wohngebäude (K.WHS)}|pw}.}}\pend{}{\bigskip}\pstart{}{\pb}Frau Paula Beer-Hofmann\pend{}\pstart{}XVIII\oindex{XVIII., Waehring@\textbf{XVIII., Währing}, \emph{A.ADM3}|pw}\pend{}\pstart{}Hasenauerstrasse 59\oindex{Hasenauerstrasse 59@\textbf{Hasenauerstraße 59}, \emph{Wohngebäude (K.WHS)}|pw}.\pend{}{\bigskip}\vspace{1em}
\pstart
           {\pb}\textcolor{gray}{\textbf{Dr. Arthur Schnitzler}}\pend
           
\pstart
           \textcolor{gray}{\textbf{Wien XVIII. Spoettelgasse 7\oindex{Edmund-Weiss-Gasse 7@\textbf{Edmund-Weiß-Gasse 7}, \emph{Wohngebäude (K.WHS)}|pw}.}}\pend
           \vspace{0.5em}
\pstart
           Liebe Paula, ich habe eine wirtschaftliche Bitte: lassen Sie mir Ihr
               heutiges Menü sagen, damit ich den Herren\pwindex{Beer-Hofmann, Richard 1866-07-11 – 1945-09-26@\textsc{Beer-Hofmann, Richard} (1866-07-11 – 1945-09-26), \emph{Schriftsteller/Schriftstellerin}|pw}\pwindex{Kerr, Alfred 25.12.1867 – 12.10.1948@\textsc{Kerr, Alfred} (25.12.1867 – 12.10.1948), \emph{Schriftsteller/Schriftstellerin, Kritiker/Kritikerin}|pw}{ }\label{K_L01815-1v}\edtext{morgen nicht dieselben Speisen}{\lemma{\textnormal{\emph{morgen … Speisen}}}\Cendnote{\textnormal{Das erlaubt die Datierung, da seit dem
						Einzug in die Hasenauerstrasse\oindex{Hasenauerstrasse 59@\textbf{Hasenauerstraße 59}, \emph{Wohngebäude (K.WHS)}|pwk} im
                     November 1906 nur ein Abendessen unter den hier
                  beschriebenen Bedingungen (Montag ist Paula
                     Beer-Hofmann\pwindex{Beer-Hofmann, Paula 25.02.1879 – 30.10.1939@\textsc{Beer-Hofmann, Paula} (25.02.1879 – 30.10.1939)|pwk}, Dienstag Olga
                     Schnitzler\pwindex{Schnitzler, Olga 17.01.1882 – 13.01.1970@\textsc{Schnitzler, Olga} (17.01.1882 – 13.01.1970), \emph{Schauspieler/Schauspielerin, Sänger/Sängerin}|pwk} Gastgeberin) belegt ist.}}}\label{K_L01815-1} vorsetze, was sich ja ereignen
               könnte. Unsere \label{K_L01815-2v}\edtext{Hedwig\pwindex{Knappe, Hedwig @\textsc{Knappe, Hedwig}, \emph{Haushaltshilfe/Haushaltshilfe}|pw}}{\lemma{\textnormal{\emph{Hedwig}}}\Cendnote{\textnormal{Es dürfte sich um Hedwig Knappe\pwindex{Knappe, Hedwig @\textsc{Knappe, Hedwig}, \emph{Haushaltshilfe/Haushaltshilfe}|pwk} handeln, ungeachtet dessen, dass das \emph{Tagebuch}\pwindex{Tagebuch@\emph{Tagebuch}|pwk} ihren Abschied für den 1. 11. 1907
                  vermerkt.}}}\label{K_L01815-2}{ }{\pb}sehe ich heute nicht mehr wenn ich nach Hause komme,
               und sie muss zeitlich früh einkaufen. Auf Wiedersehen, Dank und einen Kuss.\pend
           
\pstart
           Ihre{\\[\baselineskip]}\spacefill\mbox{Olga.}\pend
           \leftskip=0em{}
\pstart
           Montag.\pend
           \selectlanguage{ngerman}\endnumbering\briefempfaengerindex{Beer-Hofmann, Paula@\textsc{Beer-Hofmann, Paula}!zzzSchnitzler, Olga@\emph{von Olga Schnitzler}!1908-11-301@{{[}30. 11. 1908?{]}}|)be}\mylabel{L01815h}  \normalsize

\doendnotes{C}
\bigskip
\vfill

\clearpage

\footnotesize

\lohead{\textsc{register}}

% Definiere theindex-Environment komplett neu ohne reledmac
\makeatletter
\renewenvironment{theindex}{%
  \section*{\indexname}%
  \setlength{\parindent}{0pt}%
  \setlength{\parskip}{0pt plus 0.3pt}%
  \let\item\@idxitem
}{%
  \clearpage
}
\makeatother

\IfFileExists{\jobname-pw.ind}{\input{\jobname-pw.ind}}{}

\end{document}

      