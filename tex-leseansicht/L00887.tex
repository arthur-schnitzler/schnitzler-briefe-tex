%% latex-korrekturansicht-vorspann.tex
%% Vorspann für die Korrekturansicht.
%% Lädt die gemeinsame Datei latex-vorspann.tex mit gesetztem Schalter.

\newif\ifkorrekturansicht
\korrekturansichttrue

\input{../tex-inputs/latex-vorspann}


\section[Hugo von Hofmannsthal an Arthur Schnitzler, 8. 2. 1899]{L00887 Hugo von Hofmannsthal an Arthur Schnitzler, 8. 2. 1899}
\nopagebreak\mylabel{L00887v}
\rehead{ }\normalsize\beginnumbering\briefempfaengerindex{Schnitzler, Arthur@\textsc{Schnitzler, Arthur}!zzzHofmannsthal, Hugo von@\emph{von Hugo von Hofmannsthal}!1899-02-081@{8. 2. 1899}|(be}
\toendnotes[C]{\smallbreak\pagebreak[2]}\Standort{CUL, Schnitzler, B 43.}
\physDesc{Postkarte, 300 Zeichen
\newline{}Handschrift: 1) schwarze Tinte, deutsche Kurrent\hspace{1em}2) schwarze Tinte, lateinische Kurrent (\noindent{}Adresse)\hspace{1em}
\newline{}Versand: 1) Rohrpost  2) Stempel: »\nobreak{}\oindex{III., Landstrasse@\textbf{III., Landstraße}, \emph{A.ADM3}|pwk}Wien 3/3, 8 II 99, 3 10N\nobreak{}«.  3) Stempel: »\nobreak{}8 {[}II{]} 99, 3 50N\nobreak{}«. 
\newline{}Schnitzler: mit Bleistift datiert: »8/2 99« 
\newline{}Ordnung: mit Bleistift von unbekannter Hand nummeriert:
                                    »135« }
\buchAbdrucke{\weitereDrucke{Hugo von Hofmannsthal, Arthur Schnitzler: \emph{Briefwechsel}. Frankfurt am Main: \emph{S. Fischer} 1964, S. 118.} }\toendnotes[C]{\smallbreak}\pstart{}{\pb}Herrn D\textsuperscript{r} Arthur Schnitzler\pend{}\pstart{}Wien\oindex{Wien@\textbf{Wien}, \emph{A.ADM2}|pw}\pend{}\pstart{}IX Franckgasse 1\oindex{Frankgasse 1@\textbf{Frankgasse 1}, \emph{Wohngebäude (K.WHS)}|pw}\pend{}{\bigskip}\vspace{1em}
\pstart
           \noindent{}{\pb}Ich werde ſo frei ſein, heute
                  abend als Mittel gegen Ihre \label{K_L00887-1v}\edtext{Zahnſchmerzen}{\lemma{\textnormal{\emph{Zahnſchmerzen}}}\Cendnote{\textnormal{Vgl. A. S.: \emph{Tagebuch}, 3. 2. 1899.
               }}}\label{K_L00887-1} und gegen den dämoniſchen Fulda\pwindex{Fulda, Ludwig 15.07.1862 – 30.03.1939@\textsc{Fulda, Ludwig} (15.07.1862 – 30.03.1939), \emph{Schriftsteller/Schriftstellerin, Übersetzer/Übersetzerin}|pw} den
               ſehr luſtigen und angenehmen \textsc{Josi Schönborn}\pwindex{Schoenborn, Joseph von 15.11.1866 – 17.05.1913@\textsc{Schönborn, Joseph von} (15.11.1866 – 17.05.1913)|pw} mitzubringen; er wird entweder nach dem Nachtmahl oder (wenn er ſich freimachen
               kann) ſchon um ½ 9 ko{\geminationm}en.\pend
           \pstart Ihr \spacefill\mbox{Hugo.}\pend{}\selectlanguage{ngerman}\endnumbering\briefempfaengerindex{Schnitzler, Arthur@\textsc{Schnitzler, Arthur}!zzzHofmannsthal, Hugo von@\emph{von Hugo von Hofmannsthal}!1899-02-081@{8. 2. 1899}|)be}\mylabel{L00887h}  \normalsize

\doendnotes{C}
\bigskip
\vfill

\clearpage

\footnotesize

\lohead{\textsc{register}}

% Definiere theindex-Environment komplett neu ohne reledmac
\makeatletter
\renewenvironment{theindex}{%
  \section*{\indexname}%
  \setlength{\parindent}{0pt}%
  \setlength{\parskip}{0pt plus 0.3pt}%
  \let\item\@idxitem
}{%
  \clearpage
}
\makeatother

\IfFileExists{\jobname-pw.ind}{\input{\jobname-pw.ind}}{}

\end{document}

      