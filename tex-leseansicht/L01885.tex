%% latex-leseansicht-vorspann.tex
%% Vorspann für die Leseansicht.
%% Lädt die gemeinsame Datei latex-vorspann.tex mit nicht gesetztem Schalter.

\newif\ifkorrekturansicht
\korrekturansichtfalse

\input{../tex-inputs/latex-vorspann}

\begin{center}
            \textcolor{red}{ENTWURF. ENTZIFFERUNG NOCH NICHT KORREKTURGELESEN}
                      \end{center}
            
               \section[Hugo von Hofmannsthal an Arthur Schnitzler, 7. 11. 1909]{ Hugo von Hofmannsthal an Arthur Schnitzler, 7. 11. 1909}\nopagebreak\mylabel{v}\rehead{ }\begin{ledgroupsized}[t]{13cm}\normalsize\beginnumbering\briefempfaengerindex{Schnitzler, Arthur@\textsc{Schnitzler, Arthur}!zzzHofmannsthal, Hugo von@\emph{von Hugo von Hofmannsthal}!1909-11-071@{7. 11. 1909}|(be} \toendnotes[C]{\smallbreak\pagebreak[2]} \Standort{CUL, Schnitzler, B 43.}
\physDesc{Brief, 1 Blatt, 4 Seiten
\newline{}Handschrift: schwarze Tinte, deutsche Kurrent
\newline{}Schnitzler: mit Bleistift beschriftet: »Hugo« \newline{}Ordnung: 1) mit Bleistift von unbekannter Hand nummeriert: »\strikeout{301}« 2) mit Bleistift von unbekannter Hand nummeriert:
                                    »311«}\buchAbdrucke{\weitereDrucke{Hugo von Hofmannsthal, Arthur Schnitzler: \emph{Briefwechsel}. Hg. Therese Nickl und Heinrich Schnitzler. Frankfurt am Main: \emph{S. Fischer} 1964, S. 247.} }\toendnotes[C]{\smallbreak}\pstart
           \raggedleft{}{\pb}Sonntag 7/11 09.\pend
           \pstart{}mein lieber Arthur\pend\pstart
           wir waren \label{K_L01885_1v}\edtext{neulich}{\lemma{\textnormal{\emph{neulich}}}\Cendnote{\textnormal{vgl. A. S.: \emph{Tagebuch}, 1. 11. 1909}}}\label{K_L01885_1h} ſo eifrig mit mehr und minder energiſchen dramaturgiſchen Vorſchlägen, daſs
               vielleicht nicht ganz deutlich \strikeout{f} geworden iſt, wie
               ſehr man unter dem \textsc{charme} der eigentlichen Haupthandlung
               des Stückes\pwindex{Schnitzler, Arthur 15.05.1862 – 21.10.1931@\textsc{Schnitzler, Arthur} (15.05.1862 – 21.10.1931), \emph{Schriftsteller, Mediziner}!junge Medardus. Dramatische Historie in einem Vorspiel und fuenf Aufzuegen1910-10-26@\strich\emph{Der junge Medardus. Dramatische Historie in einem Vorspiel und fünf Aufzügen} {[}1910-10-26{]}|pwv} war. Es iſt eine
               außerordentliche Woltat, einmal durch ſprungweiſe Viſionen vorwärts gebracht zu
               werden {\pb}und nicht, wie man es
               gewöhnt iſt, bloß durch Entwicklung der Charaktere.\pend
           \pstart
           Aber ich glaube, wenn dieſe Kette von bildhaften Momenten, die zugleich Ballungen des
               Seeliſchen ſind, richtig von einem Publicum ſoll genoſſen werden, ſo müſſen Sie mit
               aller Härte hineinſchneiden, bis (ungefähr) ein \uline{normaler}{ }{\pb}Theaterabend herausko{\geminationm}t. Die Handlung, deren Trägerin Helene\pwindex{Schnitzler, Arthur 15.05.1862 – 21.10.1931@\textsc{Schnitzler, Arthur} (15.05.1862 – 21.10.1931), \emph{Schriftsteller, Mediziner}!junge Medardus. Dramatische Historie in einem Vorspiel und fuenf Aufzuegen1910-10-26@\strich\emph{Der junge Medardus. Dramatische Historie in einem Vorspiel und fünf Aufzügen} {[}1910-10-26{]}|pwv} (mit \textsc{Medardus}\pwindex{Schnitzler, Arthur 15.05.1862 – 21.10.1931@\textsc{Schnitzler, Arthur} (15.05.1862 – 21.10.1931), \emph{Schriftsteller, Mediziner}!junge Medardus. Dramatische Historie in einem Vorspiel und fuenf Aufzuegen1910-10-26@\strich\emph{Der junge Medardus. Dramatische Historie in einem Vorspiel und fünf Aufzügen} {[}1910-10-26{]}|pwv}) iſt, iſt ſtark genug um die Orcheſtrierung mit Vorgängen von 1809
               faſt entbehren zu können. Es wäre zu erwägen ob man nicht viel gewänne, wenn man mit
               roher Hand die Eſchenbacher-Tragödie\pwindex{Schnitzler, Arthur 15.05.1862 – 21.10.1931@\textsc{Schnitzler, Arthur} (15.05.1862 – 21.10.1931), \emph{Schriftsteller, Mediziner}!junge Medardus. Dramatische Historie in einem Vorspiel und fuenf Aufzuegen1910-10-26@\strich\emph{Der junge Medardus. Dramatische Historie in einem Vorspiel und fünf Aufzügen} {[}1910-10-26{]}|pwv} ganz wegſchnitte. Gewiſs, ſie gibt einiges ſchwer
               entbehrliche (contraſtmäßig); aber ſie koſtet unendlich viel Zeit, Nerven, {\pb}Aufnahmskraft. Für mich lebt das
               Stück Medardus\pwindex{Schnitzler, Arthur 15.05.1862 – 21.10.1931@\textsc{Schnitzler, Arthur} (15.05.1862 – 21.10.1931), \emph{Schriftsteller, Mediziner}!junge Medardus. Dramatische Historie in einem Vorspiel und fuenf Aufzuegen1910-10-26@\strich\emph{Der junge Medardus. Dramatische Historie in einem Vorspiel und fünf Aufzügen} {[}1910-10-26{]}|pwv} – Helene\pwindex{Schnitzler, Arthur 15.05.1862 – 21.10.1931@\textsc{Schnitzler, Arthur} (15.05.1862 – 21.10.1931), \emph{Schriftsteller, Mediziner}!junge Medardus. Dramatische Historie in einem Vorspiel und fuenf Aufzuegen1910-10-26@\strich\emph{Der junge Medardus. Dramatische Historie in einem Vorspiel und fünf Aufzügen} {[}1910-10-26{]}|pwv} a. von ſich ſelbſt, b von der höchſt
               geiſtreich verwendeten, occulten Nachbarſchaft der dämoniſchen Napoléon\pwindex{Bonaparte, Napoleon 15.08.1769 – 21.05.1821@\textsc{Bonaparte, Napoleon} (15.08.1769 – 21.05.1821), \emph{Kaiser}|pwv}-Figur – und c – aber dies c ko{\geminationm}t ſehr ſpät – von dem übrigen Beiwerk.\pend
           \pstart
           Es müſste ſich mit dem Stück ein \uline{ſtarker} Theaterſieg
               gewinnen laſſen, aber mit Opferung des \label{K_L01885_2v}\edtext{Bagage-train}{\lemma{\textnormal{\emph{Bagage-train}}}\Cendnote{\textnormal{Versorgungszug}}}\label{K_L01885_2h}s.\pend
           \pstart
           Ich bin fleißig\pwindex{Hofmannsthal, Hugo von 01.02.1874 – 15.07.1929@\textsc{Hofmannsthal, Hugo von} (01.02.1874 – 15.07.1929), \emph{Schriftsteller}!Cristinas Heimreise. Komoedie11.2.1910 – 11.2.1910@\strich\emph{Cristinas Heimreise. Komödie} {[}11.2.1910 – 11.2.1910{]}|pwv} und nähere mich
               dem Ende.\pend
           \pstart
           Ihr{\\[\baselineskip]}\spacefill\mbox{Hugo.}\pend
           \leftskip=0em{}\endnumbering\briefempfaengerindex{Schnitzler, Arthur@\textsc{Schnitzler, Arthur}!zzzHofmannsthal, Hugo von@\emph{von Hugo von Hofmannsthal}!1909-11-071@{7. 11. 1909}|)be}\mylabel{h}\end{ledgroupsized}  \newcommand{\dateiname}{L01885}\newcommand{\titel}{Hugo von Hofmannsthal an Arthur Schnitzler, 7. 11. 1909}\newcommand{\editorInnen}{Martin Anton Müller und Gerd-Hermann Susen}%% latex-leseansicht-abspann.tex
%% Abspann für die Leseansicht.
%% Der Schalter \ifkorrekturansicht ist bereits durch den Vorspann gesetzt.

%% latex-abspann.tex
%% Gemeinsamer Abspann für Korrekturansicht und Leseansicht.
%% Setzt den Schalter \ifkorrekturansicht voraus (gesetzt in den
%% einbindenden Dateien latex-korrekturansicht-abspann.tex bzw.
%% latex-leseansicht-abspann.tex).
%% ---------------------------------------------------------------

\normalsize

% Das esempio-Environment wird nur in der Leseansicht benötigt
\ifkorrekturansicht\else
\newenvironment{esempio}[3]%
{
    \vspace{1.5ex}
    \rlap{\underline{#1}}
    \par
    \setlength{\parindent}{0cm}
    \nopagebreak
    \leftskip=#2cm
    \rightskip=#3cm
}
{
    \par
}
\fi

\doendnotes{C}
\bigskip
\vfill

\clearpage

\footnotesize

\ifkorrekturansicht
  \lohead{\textsc{register}}
\fi

% theindex-Environment neu definieren ohne reledmac
\makeatletter
\renewenvironment{theindex}{%
  \ifkorrekturansicht
    \section*{\indexname}%
  \else
    \subsubsection*{Index der erwähnten Entitäten}%
  \fi
  \setlength{\parindent}{0pt}%
  \setlength{\parskip}{0pt plus 0.3pt}%
  \let\item\@idxitem
}{%
  \ifkorrekturansicht\clearpage\fi
}
\makeatother

\IfFileExists{\jobname-pw.ind}{\input{\jobname-pw.ind}}{}

% Quellenangabe nur in der Leseansicht
\ifkorrekturansicht\else
% Fallback-Definitionen, falls die .tex-Datei \titel etc. nicht gesetzt hat
\providecommand{\titel}{}
\providecommand{\editorInnen}{}
\providecommand{\dateiname}{\jobname}

\vspace{3cm}

\vfill

\footnotesize
\textsc{Quelle}: \titel. Herausgegeben von {\editorInnen}. In: \emph{Arthur Schnitzler: Briefwechsel mit Autorinnen und Autoren}.
 Digitale Edition, https://schnitzler-briefe.acdh.oeaw.ac.at/{\dateiname}.html (Stand \today)
\fi

\end{document}


      