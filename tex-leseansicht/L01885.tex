%% latex-leseansicht-vorspann.tex
%% Vorspann für die Leseansicht.
%% Lädt die gemeinsame Datei latex-vorspann.tex mit nicht gesetztem Schalter.

\newif\ifkorrekturansicht
\korrekturansichtfalse

\input{../tex-inputs/latex-vorspann}


\section[Hugo von Hofmannsthal an Arthur Schnitzler, 7.\,11.\,1909]{L01885 Hugo von Hofmannsthal an Arthur Schnitzler, 7.\,11.\,1909}
\nopagebreak\mylabel{L01885v}
\rehead{ }\normalsize\beginnumbering\briefempfaengerindex{Schnitzler, Arthur@\textsc{Schnitzler, Arthur}!zzzHofmannsthal, Hugo von@\emph{von Hugo von Hofmannsthal}!1909-11-071@{7. 11. 1909}|(be}
\toendnotes[C]{\smallbreak\pagebreak[2]}
\correspDesc{Versand  durch Hugo von Hofmannsthal am 7. 11. 1909 in Wien
\newline{}Erhalt  durch Arthur Schnitzler im Zeitraum [7. 11. 1909
                  – 11. 11. 1909?] in Wien}\toendnotes[C]{\smallbreak}
\Standort{CUL, Schnitzler, B 43.}
\physDesc{Brief, 1 Blatt, 4 Seiten, 1403 Zeichen
\newline{}Handschrift: schwarze Tinte, deutsche Kurrent
\newline{}Schnitzler: mit Bleistift beschriftet: »Hugo« 
\newline{}Ordnung: 1) mit Bleistift von unbekannter Hand nummeriert: »\strikeout{301}«  2) mit Bleistift von unbekannter Hand nummeriert:
                                    »311«}
\buchAbdrucke{\weitereDrucke{Hugo von Hofmannsthal, Arthur Schnitzler: \emph{Briefwechsel}. Herausgegeben von Therese Nickl und Heinrich Schnitzler. Frankfurt am Main: \emph{S. Fischer} 1964, S. 247.} }\toendnotes[C]{\smallbreak}
\pstart
           \raggedleft{}{\pb}Sonntag 7/11 09.\pend
           
\pstart{}mein lieber Arthur\pend\vspace{0.5em}
\pstart
           wir waren \label{K_L01885-1v}\edtext{neulich}{\lemma{\textnormal{\emph{neulich}}}\Cendnote{\textnormal{Vgl. A. S.: \emph{Tagebuch}, 1. 11. 1909.
               }}}\label{K_L01885-1}{ }ſo eifrig mit mehr und minder energiſchen dramaturgiſchen Vorſchlägen, daſs
               vielleicht nicht ganz deutlich \strikeout{f} geworden iſt, wie{ }ſehr man unter dem \textsc{charme} der eigentlichen Haupthandlung
               des Stückes\pwindex{Schnitzler, Arthur 15.\,5.\,1862 Wien – 21.\,10.\,1931 ebd.@\textsc{Schnitzler, Arthur} (15.\,5.\,1862 Wien – 21.\,10.\,1931 ebd.), \emph{Schriftsteller, Mediziner}!junge Medardus. Dramatische Historie in einem Vorspiel und fünf Aufzügen@\strich\emph{Der junge Medardus. Dramatische Historie in einem Vorspiel und fünf Aufzügen}|pwv} war. Es iſt eine
               außerordentliche Woltat, einmal durch{ }ſprungweiſe Viſionen vorwärts gebracht zu
               werden {\pb}und nicht, wie man es
               gewöhnt iſt, bloß durch Entwicklung der Charaktere.\pend
           
\pstart
           Aber ich glaube, wenn dieſe Kette von bildhaften Momenten, die zugleich Ballungen des
               Seeliſchen{ }ſind, richtig von einem Publicum{ }ſoll genoſſen werden,{ }ſo müſſen Sie mit
               aller Härte hineinſchneiden, bis (ungefähr) ein \uline{normaler}{ }{\pb}Theaterabend herausko{\geminationm}t. Die Handlung, deren Trägerin Helene\pwindex{Schnitzler, Arthur 15.\,5.\,1862 Wien – 21.\,10.\,1931 ebd.@\textsc{Schnitzler, Arthur} (15.\,5.\,1862 Wien – 21.\,10.\,1931 ebd.), \emph{Schriftsteller, Mediziner}!junge Medardus. Dramatische Historie in einem Vorspiel und fünf Aufzügen@\strich\emph{Der junge Medardus. Dramatische Historie in einem Vorspiel und fünf Aufzügen}|pwv} (mit \textsc{Medardus}\pwindex{Schnitzler, Arthur 15.\,5.\,1862 Wien – 21.\,10.\,1931 ebd.@\textsc{Schnitzler, Arthur} (15.\,5.\,1862 Wien – 21.\,10.\,1931 ebd.), \emph{Schriftsteller, Mediziner}!junge Medardus. Dramatische Historie in einem Vorspiel und fünf Aufzügen@\strich\emph{Der junge Medardus. Dramatische Historie in einem Vorspiel und fünf Aufzügen}|pwv}) iſt, iſt{ }ſtark genug um die Orcheſtrierung mit Vorgängen von 1809
               faſt entbehren zu können. Es wäre zu erwägen ob man nicht viel gewänne, wenn man mit
               roher Hand die Eſchenbacher-Tragödie\pwindex{Schnitzler, Arthur 15.\,5.\,1862 Wien – 21.\,10.\,1931 ebd.@\textsc{Schnitzler, Arthur} (15.\,5.\,1862 Wien – 21.\,10.\,1931 ebd.), \emph{Schriftsteller, Mediziner}!junge Medardus. Dramatische Historie in einem Vorspiel und fünf Aufzügen@\strich\emph{Der junge Medardus. Dramatische Historie in einem Vorspiel und fünf Aufzügen}|pwv} ganz wegſchnitte. Gewiſs,{ }ſie gibt einiges{ }ſchwer
               entbehrliche (contraſtmäßig); aber{ }ſie koſtet unendlich viel Zeit, Nerven, {\pb}Aufnahmskraft. Für mich lebt das
               Stück Medardus\pwindex{Schnitzler, Arthur 15.\,5.\,1862 Wien – 21.\,10.\,1931 ebd.@\textsc{Schnitzler, Arthur} (15.\,5.\,1862 Wien – 21.\,10.\,1931 ebd.), \emph{Schriftsteller, Mediziner}!junge Medardus. Dramatische Historie in einem Vorspiel und fünf Aufzügen@\strich\emph{Der junge Medardus. Dramatische Historie in einem Vorspiel und fünf Aufzügen}|pwv} – Helene\pwindex{Schnitzler, Arthur 15.\,5.\,1862 Wien – 21.\,10.\,1931 ebd.@\textsc{Schnitzler, Arthur} (15.\,5.\,1862 Wien – 21.\,10.\,1931 ebd.), \emph{Schriftsteller, Mediziner}!junge Medardus. Dramatische Historie in einem Vorspiel und fünf Aufzügen@\strich\emph{Der junge Medardus. Dramatische Historie in einem Vorspiel und fünf Aufzügen}|pwv} a. von{ }ſich{ }ſelbſt,
               b von der höchſt geiſtreich verwendeten, occulten Nachbarſchaft der dämoniſchen Napoléon\pwindex{Bonaparte, Napoleon 15.\,8.\,1769 Ajaccio – 21.\,5.\,1821 Longwood@\textsc{Bonaparte, Napoleon} (15.\,8.\,1769 Ajaccio – 21.\,5.\,1821 Longwood), \emph{Kaiser, Musiker}|pwv}-Figur – und c – aber
               dies c ko{\geminationm}t{ }ſehr{ }ſpät – von dem übrigen Beiwerk.\pend
           
\pstart
           Es müſste{ }ſich mit dem Stück ein \uline{ſtarker} Theaterſieg
               gewinnen laſſen, aber mit Opferung des \label{K_L01885-2v}\edtext{Bagage-trains}{\lemma{\textnormal{\emph{Bagage-trains}}}\Cendnote{\textnormal{Bagage-train: Versorgungszug}}}\label{K_L01885-2}.\pend
           
\pstart
           Ich bin fleißig\pwindex{Hofmannsthal, Hugo von 1.\,2.\,1874 Wien – 15.\,7.\,1929 Rodaun@\textsc{Hofmannsthal, Hugo von} (1.\,2.\,1874 Wien – 15.\,7.\,1929 Rodaun), \emph{Schriftsteller}!Cristinas Heimreise. Komödie@\strich\emph{Cristinas Heimreise. Komödie}|pwv} und nähere
               mich dem Ende.\pend
           
\pstart
           Ihr{\\[\baselineskip]}\spacefill\mbox{Hugo.}\pend
           \leftskip=0em{}\selectlanguage{ngerman}\endnumbering\briefempfaengerindex{Schnitzler, Arthur@\textsc{Schnitzler, Arthur}!zzzHofmannsthal, Hugo von@\emph{von Hugo von Hofmannsthal}!1909-11-071@{7. 11. 1909}|)be}\mylabel{L01885h}  \newcommand{\dateiname}{L01885}\newcommand{\titel}{Hugo von Hofmannsthal an Arthur Schnitzler, 7. 11. 1909}\newcommand{\editorInnen}{Martin Anton Müller und Gerd-Hermann Susen}%% latex-leseansicht-abspann.tex
%% Abspann für die Leseansicht.
%% Der Schalter \ifkorrekturansicht ist bereits durch den Vorspann gesetzt.

%% latex-abspann.tex
%% Gemeinsamer Abspann für Korrekturansicht und Leseansicht.
%% Setzt den Schalter \ifkorrekturansicht voraus (gesetzt in den
%% einbindenden Dateien latex-korrekturansicht-abspann.tex bzw.
%% latex-leseansicht-abspann.tex).
%% ---------------------------------------------------------------

\normalsize

% Das esempio-Environment wird nur in der Leseansicht benötigt
\ifkorrekturansicht\else
\newenvironment{esempio}[3]%
{
    \vspace{1.5ex}
    \rlap{\underline{#1}}
    \par
    \setlength{\parindent}{0cm}
    \nopagebreak
    \leftskip=#2cm
    \rightskip=#3cm
}
{
    \par
}
\fi

\doendnotes{C}
\bigskip
\vfill

\clearpage

\footnotesize

\ifkorrekturansicht
  \lohead{\textsc{register}}
\fi

% theindex-Environment neu definieren ohne reledmac
\makeatletter
\renewenvironment{theindex}{%
  \ifkorrekturansicht
    \section*{\indexname}%
  \else
    \subsubsection*{Index der erwähnten Entitäten}%
  \fi
  \setlength{\parindent}{0pt}%
  \setlength{\parskip}{0pt plus 0.3pt}%
  \let\item\@idxitem
}{%
  \ifkorrekturansicht\clearpage\fi
}
\makeatother

\IfFileExists{\jobname-pw.ind}{\input{\jobname-pw.ind}}{}

% Quellenangabe nur in der Leseansicht
\ifkorrekturansicht\else
% Fallback-Definitionen, falls die .tex-Datei \titel etc. nicht gesetzt hat
\providecommand{\titel}{}
\providecommand{\editorInnen}{}
\providecommand{\dateiname}{\jobname}

\vspace{3cm}

\vfill

\footnotesize
\textsc{Quelle}: \titel. Herausgegeben von {\editorInnen}. In: \emph{Arthur Schnitzler: Briefwechsel mit Autorinnen und Autoren}.
 Digitale Edition, https://schnitzler-briefe.acdh.oeaw.ac.at/{\dateiname}.html (Stand \today)
\fi

\end{document}


