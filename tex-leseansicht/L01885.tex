%% latex-korrekturansicht-vorspann.tex
%% Vorspann für die Korrekturansicht.
%% Lädt die gemeinsame Datei latex-vorspann.tex mit gesetztem Schalter.

\newif\ifkorrekturansicht
\korrekturansichttrue

\input{../tex-inputs/latex-vorspann}


\section[Hugo von Hofmannsthal an Arthur Schnitzler, 7. 11. 1909]{L01885 Hugo von Hofmannsthal an Arthur Schnitzler, 7. 11. 1909}
\nopagebreak\mylabel{L01885v}
\rehead{ }\normalsize\beginnumbering\briefempfaengerindex{Schnitzler, Arthur@\textsc{Schnitzler, Arthur}!zzzHofmannsthal, Hugo von@\emph{von Hugo von Hofmannsthal}!1909-11-071@{7. 11. 1909}|(be}
\toendnotes[C]{\smallbreak\pagebreak[2]}\Standort{CUL, Schnitzler, B 43.}
\physDesc{Brief, 1 Blatt, 4 Seiten, 1403 Zeichen
\newline{}Handschrift: schwarze Tinte, deutsche Kurrent
\newline{}Schnitzler: mit Bleistift beschriftet: »Hugo« 
\newline{}Ordnung: 1) mit Bleistift von unbekannter Hand nummeriert: »\strikeout{301}«  2) mit Bleistift von unbekannter Hand nummeriert:
                                    »311«}
\buchAbdrucke{\weitereDrucke{Hugo von Hofmannsthal, Arthur Schnitzler: \emph{Briefwechsel}. Frankfurt am Main: \emph{S. Fischer} 1964, S. 247.} }\toendnotes[C]{\smallbreak}
\pstart
           \raggedleft{}{\pb}Sonntag 7/11 09.\pend
           
\pstart{}mein lieber Arthur\pend\vspace{0.5em}
\pstart
           wir waren \label{K_L01885-1v}\edtext{neulich}{\lemma{\textnormal{\emph{neulich}}}\Cendnote{\textnormal{Vgl. A. S.: \emph{Tagebuch}, 1. 11. 1909.
               }}}\label{K_L01885-1} ſo eifrig mit mehr und minder energiſchen dramaturgiſchen Vorſchlägen, daſs
               vielleicht nicht ganz deutlich \strikeout{f} geworden iſt, wie
               ſehr man unter dem \textsc{charme} der eigentlichen Haupthandlung
               des Stückes\pwindex{junge Medardus. Dramatische Historie in einem Vorspiel und fuenf Aufzuegen@\emph{Der junge Medardus. Dramatische Historie in einem Vorspiel und fünf Aufzügen}|pwv} war. Es iſt eine
               außerordentliche Woltat, einmal durch ſprungweiſe Viſionen vorwärts gebracht zu
               werden {\pb}und nicht, wie man es
               gewöhnt iſt, bloß durch Entwicklung der Charaktere.\pend
           
\pstart
           Aber ich glaube, wenn dieſe Kette von bildhaften Momenten, die zugleich Ballungen des
               Seeliſchen ſind, richtig von einem Publicum ſoll genoſſen werden, ſo müſſen Sie mit
               aller Härte hineinſchneiden, bis (ungefähr) ein \uline{normaler}{ }{\pb}Theaterabend herausko{\geminationm}t. Die Handlung, deren Trägerin Helene\pwindex{junge Medardus. Dramatische Historie in einem Vorspiel und fuenf Aufzuegen@\emph{Der junge Medardus. Dramatische Historie in einem Vorspiel und fünf Aufzügen}|pwv} (mit \textsc{Medardus}\pwindex{junge Medardus. Dramatische Historie in einem Vorspiel und fuenf Aufzuegen@\emph{Der junge Medardus. Dramatische Historie in einem Vorspiel und fünf Aufzügen}|pwv}) iſt, iſt ſtark genug um die Orcheſtrierung mit Vorgängen von 1809
               faſt entbehren zu können. Es wäre zu erwägen ob man nicht viel gewänne, wenn man mit
               roher Hand die Eſchenbacher-Tragödie\pwindex{junge Medardus. Dramatische Historie in einem Vorspiel und fuenf Aufzuegen@\emph{Der junge Medardus. Dramatische Historie in einem Vorspiel und fünf Aufzügen}|pwv} ganz wegſchnitte. Gewiſs, ſie gibt einiges ſchwer
               entbehrliche (contraſtmäßig); aber ſie koſtet unendlich viel Zeit, Nerven, {\pb}Aufnahmskraft. Für mich lebt das
               Stück Medardus\pwindex{junge Medardus. Dramatische Historie in einem Vorspiel und fuenf Aufzuegen@\emph{Der junge Medardus. Dramatische Historie in einem Vorspiel und fünf Aufzügen}|pwv} – Helene\pwindex{junge Medardus. Dramatische Historie in einem Vorspiel und fuenf Aufzuegen@\emph{Der junge Medardus. Dramatische Historie in einem Vorspiel und fünf Aufzügen}|pwv} a. von ſich ſelbſt,
               b von der höchſt geiſtreich verwendeten, occulten Nachbarſchaft der dämoniſchen Napoléon\pwindex{Bonaparte, Napoleon 15.08.1769 – 21.05.1821@\textsc{Bonaparte, Napoleon} (15.08.1769 – 21.05.1821), \emph{Kaiser/Kaiserin, Musiker/Musikerin}|pwv}-Figur – und c – aber
               dies c ko{\geminationm}t ſehr ſpät – von dem übrigen Beiwerk.\pend
           
\pstart
           Es müſste ſich mit dem Stück ein \uline{ſtarker} Theaterſieg
               gewinnen laſſen, aber mit Opferung des \label{K_L01885-2v}\edtext{Bagage-trains}{\lemma{\textnormal{\emph{Bagage-trains}}}\Cendnote{\textnormal{Bagage-train: Versorgungszug}}}\label{K_L01885-2}.\pend
           
\pstart
           Ich bin fleißig\pwindex{Cristinas Heimreise. Komoedie@\emph{Cristinas Heimreise. Komödie}|pwv} und nähere
               mich dem Ende.\pend
           
\pstart
           Ihr{\\[\baselineskip]}\spacefill\mbox{Hugo.}\pend
           \leftskip=0em{}\selectlanguage{ngerman}\endnumbering\briefempfaengerindex{Schnitzler, Arthur@\textsc{Schnitzler, Arthur}!zzzHofmannsthal, Hugo von@\emph{von Hugo von Hofmannsthal}!1909-11-071@{7. 11. 1909}|)be}\mylabel{L01885h}  \normalsize

\doendnotes{C}
\bigskip
\vfill

\clearpage

\footnotesize

\lohead{\textsc{register}}

% Definiere theindex-Environment komplett neu ohne reledmac
\makeatletter
\renewenvironment{theindex}{%
  \section*{\indexname}%
  \setlength{\parindent}{0pt}%
  \setlength{\parskip}{0pt plus 0.3pt}%
  \let\item\@idxitem
}{%
  \clearpage
}
\makeatother

\IfFileExists{\jobname-pw.ind}{\input{\jobname-pw.ind}}{}

\end{document}

      