%% latex-korrekturansicht-vorspann.tex
%% Vorspann für die Korrekturansicht.
%% Lädt die gemeinsame Datei latex-vorspann.tex mit gesetztem Schalter.

\newif\ifkorrekturansicht
\korrekturansichttrue

\input{../tex-inputs/latex-vorspann}


\section[Max Mell an Arthur Schnitzler, 8. 12. 1909]{L01894 Max Mell an Arthur Schnitzler, 8. 12. 1909}
\nopagebreak\mylabel{L01894v}
\rehead{ }\normalsize\beginnumbering\briefempfaengerindex{Schnitzler, Arthur@\textsc{Schnitzler, Arthur}!zzzMell, Max@\emph{von Max Mell}!1909-12-081@{8. 12. 1909}|(be}
\toendnotes[C]{\smallbreak\pagebreak[2]}\Standort{DLA, A:Schnitzler, HS.NZ85.1.4055, S. [7].}
\physDesc{Brief, maschinenschriftliche Abschrift1 Blatt, 1 Seite, 605 Zeichen
\newline{}Schreibmaschine}\toendnotes[C]{\smallbreak}
\pstart
           \raggedleft{}{\pb}8. Dez. 1909.\pend
           
\pstart{}Sehr verehrter Herr Doktor,\pend\vspace{0.5em}
\pstart
           Kann ich Ihnen ohne allzu unbescheiden zu sein, mit einer Bitte kommen? Ich habe,
               obwohl ich von Schlenther\pwindex{Schlenther, Paul 20.08.1854 – 30.04.1916@\textsc{Schlenther, Paul} (20.08.1854 – 30.04.1916), \emph{Schriftsteller/Schriftstellerin, Kritiker/Kritikerin, Theaterleiter/Theaterleiterin}|pw} natürlich noch keine
               Entscheidung habe, mein Stück\pwindex{Kinder des Hauses@\emph{Die Kinder des Hauses}|pwv}
               jetzt an das Deutsche Volkstheater\orgindex{Volkstheater@Volkstheater|pw} geschickt –
               würden Sie die Güte haben mit einem Wort bei der Direktion nur dahin zu wirken, dass
               es überhaupt angesehen wird und nicht in dem notwendig \label{T_L01894-1v}\edtext{ungelesenen}{\lemma{\textnormal{\emph{ungelesenen}}}\Cendnote{\textnormal{die
                  Abschrift hat: »ungelesenem«}}}\label{T_L01894-1} Wust des Einlaufs verschwindet?
               Es handelt sich mir nur darum überhaupt eine Erledigung zu bekommen und Sie würden
               mich sehr verpflichten, wenn Sie mir dazu verhelfen wollten.\pend
           
\pstart
           Mit den besten Empfehlungen{\\[\baselineskip]}Ihres{\\[\baselineskip]}\spacefill\mbox{Max Mell}\pend
           \leftskip=0em{}\selectlanguage{ngerman}\endnumbering\briefempfaengerindex{Schnitzler, Arthur@\textsc{Schnitzler, Arthur}!zzzMell, Max@\emph{von Max Mell}!1909-12-081@{8. 12. 1909}|)be}\mylabel{L01894h}  \normalsize

\doendnotes{C}
\bigskip
\vfill

\clearpage

\footnotesize

\lohead{\textsc{register}}

% Definiere theindex-Environment komplett neu ohne reledmac
\makeatletter
\renewenvironment{theindex}{%
  \section*{\indexname}%
  \setlength{\parindent}{0pt}%
  \setlength{\parskip}{0pt plus 0.3pt}%
  \let\item\@idxitem
}{%
  \clearpage
}
\makeatother

\IfFileExists{\jobname-pw.ind}{\input{\jobname-pw.ind}}{}

\end{document}

      