%% latex-leseansicht-vorspann.tex
%% Vorspann für die Leseansicht.
%% Lädt die gemeinsame Datei latex-vorspann.tex mit nicht gesetztem Schalter.

\newif\ifkorrekturansicht
\korrekturansichtfalse

\input{../tex-inputs/latex-vorspann}


\section[Max Mell an Arthur Schnitzler, 8. 12. 1909]{L01894 Max Mell an Arthur Schnitzler, 8. 12. 1909}
\nopagebreak\mylabel{L01894v}
\rehead{ }\normalsize\beginnumbering\briefempfaengerindex{Schnitzler, Arthur@\textsc{Schnitzler, Arthur}!zzzMell, Max@\emph{von Max Mell}!1909-12-081@{8. 12. 1909}|(be}
\toendnotes[C]{\smallbreak\pagebreak[2]}
\correspDesc{Versand  durch Max Mell am 8. 12. 1909 in Wien
\newline{}Erhalt  durch Arthur Schnitzler im Zeitraum [8. 12. 1909
                  – 12. 12. 1909?] in Wien}\toendnotes[C]{\smallbreak}
\Standort{DLA, A:Schnitzler, HS.NZ85.1.4055, S. [7].}
\physDesc{Brief, maschinenschriftliche Abschrift, 1 Blatt, 1 Seite, 605 Zeichen
\newline{}Schreibmaschine}\toendnotes[C]{\smallbreak}
\pstart
           \raggedleft{}{\pb}8. Dez. 1909.\pend
           
\pstart{}Sehr verehrter Herr Doktor,\pend\vspace{0.5em}
\pstart
           Kann ich Ihnen ohne allzu unbescheiden zu sein, mit einer Bitte kommen? Ich habe,
               obwohl ich von Schlenther\pwindex{Schlenther, Paul 20.\,8.\,1854 Chernyakhovsk – 30.\,4.\,1916 Berlin@\textsc{Schlenther, Paul} (20.\,8.\,1854 Chernyakhovsk – 30.\,4.\,1916 Berlin), \emph{Schriftsteller, Kritiker, Theaterleiter}|pw} natürlich noch keine
               Entscheidung habe, mein Stück\pwindex{Mell, Max 10.\,11.\,1882 Maribor – 13.\,12.\,1971 Wien@\textsc{Mell, Max} (10.\,11.\,1882 Maribor – 13.\,12.\,1971 Wien), \emph{Schriftsteller}!Kinder des Hauses@\strich\emph{Die Kinder des Hauses}|pwv}
               jetzt an das Deutsche Volkstheater\orgindex{Volkstheater@Volkstheater|pw} geschickt –
               würden Sie die Güte haben mit einem Wort bei der Direktion nur dahin zu wirken, dass
               es überhaupt angesehen wird und nicht in dem notwendig \label{T_L01894-1v}\edtext{ungelesenen}{\lemma{\textnormal{\emph{ungelesenen}}}\Cendnote{\textnormal{die
                  Abschrift hat: »ungelesenem«}}}\label{T_L01894-1} Wust des Einlaufs verschwindet?
               Es handelt sich mir nur darum überhaupt eine Erledigung zu bekommen und Sie würden
               mich sehr verpflichten, wenn Sie mir dazu verhelfen wollten.\pend
           
\pstart
           Mit den besten Empfehlungen{\\[\baselineskip]}Ihres{\\[\baselineskip]}\spacefill\mbox{Max Mell}\pend
           \leftskip=0em{}\selectlanguage{ngerman}\endnumbering\briefempfaengerindex{Schnitzler, Arthur@\textsc{Schnitzler, Arthur}!zzzMell, Max@\emph{von Max Mell}!1909-12-081@{8. 12. 1909}|)be}\mylabel{L01894h}  \newcommand{\dateiname}{L01894}\newcommand{\titel}{Max Mell an Arthur Schnitzler, 8. 12. 1909}\newcommand{\editorInnen}{Martin Anton Müller und Gerd-Hermann Susen}%% latex-leseansicht-abspann.tex
%% Abspann für die Leseansicht.
%% Der Schalter \ifkorrekturansicht ist bereits durch den Vorspann gesetzt.

%% latex-abspann.tex
%% Gemeinsamer Abspann für Korrekturansicht und Leseansicht.
%% Setzt den Schalter \ifkorrekturansicht voraus (gesetzt in den
%% einbindenden Dateien latex-korrekturansicht-abspann.tex bzw.
%% latex-leseansicht-abspann.tex).
%% ---------------------------------------------------------------

\normalsize

% Das esempio-Environment wird nur in der Leseansicht benötigt
\ifkorrekturansicht\else
\newenvironment{esempio}[3]%
{
    \vspace{1.5ex}
    \rlap{\underline{#1}}
    \par
    \setlength{\parindent}{0cm}
    \nopagebreak
    \leftskip=#2cm
    \rightskip=#3cm
}
{
    \par
}
\fi

\doendnotes{C}
\bigskip
\vfill

\clearpage

\footnotesize

\ifkorrekturansicht
  \lohead{\textsc{register}}
\fi

% theindex-Environment neu definieren ohne reledmac
\makeatletter
\renewenvironment{theindex}{%
  \ifkorrekturansicht
    \section*{\indexname}%
  \else
    \subsubsection*{Index der erwähnten Entitäten}%
  \fi
  \setlength{\parindent}{0pt}%
  \setlength{\parskip}{0pt plus 0.3pt}%
  \let\item\@idxitem
}{%
  \ifkorrekturansicht\clearpage\fi
}
\makeatother

\IfFileExists{\jobname-pw.ind}{\input{\jobname-pw.ind}}{}

% Quellenangabe nur in der Leseansicht
\ifkorrekturansicht\else
% Fallback-Definitionen, falls die .tex-Datei \titel etc. nicht gesetzt hat
\providecommand{\titel}{}
\providecommand{\editorInnen}{}
\providecommand{\dateiname}{\jobname}

\vspace{3cm}

\vfill

\footnotesize
\textsc{Quelle}: \titel. Herausgegeben von {\editorInnen}. In: \emph{Arthur Schnitzler: Briefwechsel mit Autorinnen und Autoren}.
 Digitale Edition, https://schnitzler-briefe.acdh.oeaw.ac.at/{\dateiname}.html (Stand \today)
\fi

\end{document}


