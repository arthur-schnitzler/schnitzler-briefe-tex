%% latex-leseansicht-vorspann.tex
%% Vorspann für die Leseansicht.
%% Lädt die gemeinsame Datei latex-vorspann.tex mit nicht gesetztem Schalter.

\newif\ifkorrekturansicht
\korrekturansichtfalse

\input{../tex-inputs/latex-vorspann}


\section[Hugo von Hofmannsthal an Arthur Schnitzler, {{[}}20. 9. 1894{{]}}]{L00371 Hugo von Hofmannsthal an Arthur Schnitzler, {[}20. 9. 1894{]}}
\nopagebreak\mylabel{L00371v}
\rehead{ }\normalsize\beginnumbering\briefempfaengerindex{Schnitzler, Arthur@\textsc{Schnitzler, Arthur}!zzzHofmannsthal, Hugo von@\emph{von Hugo von Hofmannsthal}!1894-09-201@{{[}20. 9. 1894{]}}|(be}
\toendnotes[C]{\smallbreak\pagebreak[2]}
\correspDesc{Versand  durch Hugo von Hofmannsthal am [20. 9. 1894] in Wien
\newline{}Erhalt  durch Arthur Schnitzler im Zeitraum [20. 9. 1894
                  – 24. 9. 1894?] in Wien}\toendnotes[C]{\smallbreak}
\Standort{CUL, Schnitzler, B 43.}
\physDesc{Briefkarte, 154 Zeichen (aufgeprägtes Wappen, floraler Jugendstil-Karton )
\newline{}Handschrift: schwarze Tinte, deutsche Kurrent
\newline{}Schnitzler: mit Bleistift datiert: »20/9 94« und nummeriert: »67« }
\buchAbdrucke{\weitereDrucke{Hugo von Hofmannsthal, Arthur Schnitzler: \emph{Briefwechsel}. Herausgegeben von Therese Nickl und Heinrich Schnitzler. Frankfurt am Main: \emph{S. Fischer} 1964, S. 52.} }\toendnotes[C]{\smallbreak}
\pstart{}{\pb}lieber,\pend\vspace{0.5em}
\pstart
           Sterben\pwindex{Schnitzler, Arthur 15.\,5.\,1862 Wien – 21.\,10.\,1931 ebd.@\textsc{Schnitzler, Arthur} (15.\,5.\,1862 Wien – 21.\,10.\,1931 ebd.), \emph{Schriftsteller, Mediziner}!Sterben. Novelle@\strich\emph{Sterben. Novelle}|pw}. \uline{Abſolut}
               keine Punkte. Beſſer Novelle als Erzählung, am beſten einfach »von A. S.«\pend
           
\pstart
           Bitte hat Ihnen Stern\pwindex{Stern, Julius 11.\,5.\,1858 Wien – 29.\,11.\,1911 Bad Aussee@\textsc{Stern, Julius} (11.\,5.\,1858 Wien – 29.\,11.\,1911 Bad Aussee), \emph{Komponist, Dirigent}|pw} wegen \label{K_L00371-1v}\edtext{Generalprobe\pwindex{Stern, Julius 11.\,5.\,1858 Wien – 29.\,11.\,1911 Bad Aussee@\textsc{Stern, Julius} (11.\,5.\,1858 Wien – 29.\,11.\,1911 Bad Aussee), \emph{Komponist, Dirigent}!Fürst Malachoff@\strich\emph{Fürst Malachoff}|pwv}\pwindex{\textcolor{red}{\textsuperscript{XXXX indx1}}!Fürst Malachoff@\strich\emph{Fürst Malachoff}|pwv}}{\lemma{\textnormal{\emph{Generalprobe}}}\Cendnote{\textnormal{Zumindest Schnitzler besuchte die Uraufführung\eventindex{Carl-Theater@\textbf{Carl-Theater}!Premiere von Fürst Malachoff, 22.9.1894@Premiere von Fürst Malachoff, 22.9.1894|pwkv} am
                     22. 9. 1894 im Carl-Theater\oindex{Wien@\textbf{Wien}!II., Leopoldstadt@\textbf{II., Leopoldstadt}!Carl-Theater@\textbf{Carl-Theater}, \emph{Theater}|pwk}.}}}\label{K_L00371-1} was{ }ſagen laſſen?\pend
           \pstart \spacefill\mbox{Hugo.}\pend{}\selectlanguage{ngerman}\endnumbering\briefempfaengerindex{Schnitzler, Arthur@\textsc{Schnitzler, Arthur}!zzzHofmannsthal, Hugo von@\emph{von Hugo von Hofmannsthal}!1894-09-201@{{[}20. 9. 1894{]}}|)be}\mylabel{L00371h}  \newcommand{\dateiname}{L00371}\newcommand{\titel}{Hugo von Hofmannsthal an Arthur Schnitzler, [20. 9. 1894]}\newcommand{\editorInnen}{Martin Anton Müller und Gerd-Hermann Susen}%% latex-leseansicht-abspann.tex
%% Abspann für die Leseansicht.
%% Der Schalter \ifkorrekturansicht ist bereits durch den Vorspann gesetzt.

%% latex-abspann.tex
%% Gemeinsamer Abspann für Korrekturansicht und Leseansicht.
%% Setzt den Schalter \ifkorrekturansicht voraus (gesetzt in den
%% einbindenden Dateien latex-korrekturansicht-abspann.tex bzw.
%% latex-leseansicht-abspann.tex).
%% ---------------------------------------------------------------

\normalsize

% Das esempio-Environment wird nur in der Leseansicht benötigt
\ifkorrekturansicht\else
\newenvironment{esempio}[3]%
{
    \vspace{1.5ex}
    \rlap{\underline{#1}}
    \par
    \setlength{\parindent}{0cm}
    \nopagebreak
    \leftskip=#2cm
    \rightskip=#3cm
}
{
    \par
}
\fi

\doendnotes{C}
\bigskip
\vfill

\clearpage

\footnotesize

\ifkorrekturansicht
  \lohead{\textsc{register}}
\fi

% theindex-Environment neu definieren ohne reledmac
\makeatletter
\renewenvironment{theindex}{%
  \ifkorrekturansicht
    \section*{\indexname}%
  \else
    \subsubsection*{Index der erwähnten Entitäten}%
  \fi
  \setlength{\parindent}{0pt}%
  \setlength{\parskip}{0pt plus 0.3pt}%
  \let\item\@idxitem
}{%
  \ifkorrekturansicht\clearpage\fi
}
\makeatother

\IfFileExists{\jobname-pw.ind}{\input{\jobname-pw.ind}}{}

% Quellenangabe nur in der Leseansicht
\ifkorrekturansicht\else
% Fallback-Definitionen, falls die .tex-Datei \titel etc. nicht gesetzt hat
\providecommand{\titel}{}
\providecommand{\editorInnen}{}
\providecommand{\dateiname}{\jobname}

\vspace{3cm}

\vfill

\footnotesize
\textsc{Quelle}: \titel. Herausgegeben von {\editorInnen}. In: \emph{Arthur Schnitzler: Briefwechsel mit Autorinnen und Autoren}.
 Digitale Edition, https://schnitzler-briefe.acdh.oeaw.ac.at/{\dateiname}.html (Stand \today)
\fi

\end{document}


