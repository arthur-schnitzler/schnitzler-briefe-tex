%% latex-leseansicht-vorspann.tex
%% Vorspann für die Leseansicht.
%% Lädt die gemeinsame Datei latex-vorspann.tex mit nicht gesetztem Schalter.

\newif\ifkorrekturansicht
\korrekturansichtfalse

\input{../tex-inputs/latex-vorspann}

\begin{center}
            \textcolor{red}{ENTWURF. ENTZIFFERUNG NOCH NICHT KORREKTURGELESEN}
                      \end{center}
            
               \section[Hugo von Hofmannsthal an Arthur Schnitzler, {[}20. 9. 1894{]}]{ Hugo von Hofmannsthal an Arthur Schnitzler, {[}20. 9. 1894{]}}\nopagebreak\mylabel{v}\rehead{ }\begin{ledgroupsized}[t]{13cm}\normalsize\beginnumbering\briefempfaengerindex{Schnitzler, Arthur@\textsc{Schnitzler, Arthur}!zzzHofmannsthal, Hugo von@\emph{von Hugo von Hofmannsthal}!1894-09-201@{{[}20. 9. 1894{]}}|(be} \toendnotes[C]{\smallbreak\pagebreak[2]} \Standort{CUL, Schnitzler, B 43.}
\physDesc{Briefkarte (aufgeprägtes Wappen, floraler Jugendstil-Karton)
\newline{}Handschrift: schwarze Tinte, deutsche Kurrent
\newline{}Schnitzler: mit Bleistift datiert: »20/9 94« und nummeriert: »67« }\buchAbdrucke{\weitereDrucke{Hugo von Hofmannsthal, Arthur Schnitzler: \emph{Briefwechsel}. Hg. Therese Nickl und Heinrich Schnitzler. Frankfurt am Main: \emph{S. Fischer} 1964, S. 52.} }\toendnotes[C]{\smallbreak}\pstart{}{\pb}lieber,\pend\pstart
           Sterben\pwindex{Schnitzler, Arthur 15.05.1862 – 21.10.1931@\textsc{Schnitzler, Arthur} (15.05.1862 – 21.10.1931), \emph{Schriftsteller, Mediziner}!Sterben. Novelle1.10.1894 – 1.12.1894@\strich\emph{Sterben. Novelle} {[}1.10.1894 – 1.12.1894{]}|pw}. \uline{Abſolut} keine Punkte. Beſſer Novelle als Erzählung, am beſten einfach
                    »von A. S.«\pend
           \pstart
           Bitte hat Ihnen Stern\pwindex{Stern, Julius 11.05.1858 – 29.11.1911@\textsc{Stern, Julius} (11.05.1858 – 29.11.1911), \emph{Komponist, Dirigent}|pw} wegen \label{K_L00371_1v}\edtext{Generalprobe\pwindex{Stern, Julius 11.05.1858 – 29.11.1911@\textsc{Stern, Julius} (11.05.1858 – 29.11.1911), \emph{Komponist, Dirigent}!Fuerst Malachoff22. 9. 1894@\strich\emph{Fürst Malachoff} {[}22. 9. 1894{]}|pwv}\pwindex{\textcolor{red}{\textsuperscript{XXXX1 indx}}!Fuerst Malachoff22. 9. 1894@\strich\emph{Fürst Malachoff} {[}22. 9. 1894{]}|pwv}}{\lemma{\textnormal{\emph{Generalprobe}}}\Cendnote{\textnormal{Zumindest Schnitzler\pwindex{Schnitzler, Arthur 15.05.1862 – 21.10.1931@\textsc{Schnitzler, Arthur} (15.05.1862 – 21.10.1931), \emph{Schriftsteller, Mediziner}|pwk} besuchte die Uraufführung am
                            22. 9. 1894 im Carl-Theater\oindex{Carl-Theater@\textbf{Carl-Theater}|pwk}.}}}\label{K_L00371_1h} was ſagen laſſen? \pend
           \pstart \spacefill\mbox{Hugo.}\pend{}\endnumbering\briefempfaengerindex{Schnitzler, Arthur@\textsc{Schnitzler, Arthur}!zzzHofmannsthal, Hugo von@\emph{von Hugo von Hofmannsthal}!1894-09-201@{{[}20. 9. 1894{]}}|)be}\mylabel{h}\end{ledgroupsized}  \newcommand{\dateiname}{L00371}\newcommand{\titel}{Hugo von Hofmannsthal an Arthur Schnitzler, [20. 9. 1894]}\newcommand{\editorInnen}{Martin Anton Müller und Gerd-Hermann Susen}%% latex-leseansicht-abspann.tex
%% Abspann für die Leseansicht.
%% Der Schalter \ifkorrekturansicht ist bereits durch den Vorspann gesetzt.

%% latex-abspann.tex
%% Gemeinsamer Abspann für Korrekturansicht und Leseansicht.
%% Setzt den Schalter \ifkorrekturansicht voraus (gesetzt in den
%% einbindenden Dateien latex-korrekturansicht-abspann.tex bzw.
%% latex-leseansicht-abspann.tex).
%% ---------------------------------------------------------------

\normalsize

% Das esempio-Environment wird nur in der Leseansicht benötigt
\ifkorrekturansicht\else
\newenvironment{esempio}[3]%
{
    \vspace{1.5ex}
    \rlap{\underline{#1}}
    \par
    \setlength{\parindent}{0cm}
    \nopagebreak
    \leftskip=#2cm
    \rightskip=#3cm
}
{
    \par
}
\fi

\doendnotes{C}
\bigskip
\vfill

\clearpage

\footnotesize

\ifkorrekturansicht
  \lohead{\textsc{register}}
\fi

% theindex-Environment neu definieren ohne reledmac
\makeatletter
\renewenvironment{theindex}{%
  \ifkorrekturansicht
    \section*{\indexname}%
  \else
    \subsubsection*{Index der erwähnten Entitäten}%
  \fi
  \setlength{\parindent}{0pt}%
  \setlength{\parskip}{0pt plus 0.3pt}%
  \let\item\@idxitem
}{%
  \ifkorrekturansicht\clearpage\fi
}
\makeatother

\IfFileExists{\jobname-pw.ind}{\input{\jobname-pw.ind}}{}

% Quellenangabe nur in der Leseansicht
\ifkorrekturansicht\else
% Fallback-Definitionen, falls die .tex-Datei \titel etc. nicht gesetzt hat
\providecommand{\titel}{}
\providecommand{\editorInnen}{}
\providecommand{\dateiname}{\jobname}

\vspace{3cm}

\vfill

\footnotesize
\textsc{Quelle}: \titel. Herausgegeben von {\editorInnen}. In: \emph{Arthur Schnitzler: Briefwechsel mit Autorinnen und Autoren}.
 Digitale Edition, https://schnitzler-briefe.acdh.oeaw.ac.at/{\dateiname}.html (Stand \today)
\fi

\end{document}


      