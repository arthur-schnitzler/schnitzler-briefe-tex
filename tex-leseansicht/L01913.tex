%% latex-korrekturansicht-vorspann.tex
%% Vorspann für die Korrekturansicht.
%% Lädt die gemeinsame Datei latex-vorspann.tex mit gesetztem Schalter.

\newif\ifkorrekturansicht
\korrekturansichttrue

\input{../tex-inputs/latex-vorspann}


\section[Arthur Schnitzler an Albert Ehrenstein, 12. 2. 1910]{L01913 Arthur Schnitzler an Albert Ehrenstein, 12. 2. 1910}
\nopagebreak\mylabel{L01913v}
\rehead{ }\normalsize\beginnumbering\briefempfaengerindex{Ehrenstein, Albert@\textsc{Ehrenstein, Albert}!zzzSchnitzler, Arthur@\emph{von Arthur Schnitzler}!1910-02-121@{12. 2. 1910}|(be}
\toendnotes[C]{\smallbreak\pagebreak[2]}\Standort{Jerusalem, The National Library of Israel, ARC. Ms. Var. 306 1 118.}
\physDesc{Brief, 1 Blatt, 1 Seite, 731 Zeichen
\newline{}Schreibmaschine
\newline{}Handschrift: schwarze Tinte (\noindent{}Unterschrift)}\Standort{DLA, A:Schnitzler, 85.1.642,3.}
\physDesc{Brief, Durchschlag1 Blatt, 1 Seite, 731 Zeichen
\newline{}Schreibmaschine
\newline{}Handschrift: roter Buntstift, lateinische Kurrent (\noindent{}Beschriftung: »Ehrenstein«)}\toendnotes[C]{\smallbreak}
\pstart
           {\pb}\textcolor{gray}{\textbf{Dr. Arthur Schnitzler}}\hfill 12. 2. 1910.\pend
           
\pstart
           \textcolor{gray}{\textbf{Wien XVIII. Spoettelgasse 7\oindex{Edmund-Weiss-Gasse 7@\textbf{Edmund-Weiß-Gasse 7}, \emph{Wohngebäude (K.WHS)}|pw}.}}\pend
           
\pstart{}Lieber Herr Ehrenstein!\pend\vspace{0.5em}
\pstart
           Aus dem Brief von Bie\pwindex{Bie, Oskar 09.02.1864 – 21.04.1938@\textsc{Bie, Oskar} (09.02.1864 – 21.04.1938), \emph{Schriftsteller/Schriftstellerin, Journalist/Journalistin, Redakteur/Redakteurin}|pw} an Sie ist zu entnehmen,
               dass er »Tubutsch\pwindex{Tubutsch@\emph{Tubutsch}|pw}« nicht veröffentlichen will,
               dass aber für Ihre nächsten Einsendungen aufrichtiges Interesse und daher auch
               Druckchancen vorhanden sind. Das mit dem Wien\oindex{Wien@\textbf{Wien}, \emph{A.ADM2}|pw}er
               Leben müssen Sie nicht so wörtlich nehmen. Was die Schröder\pwindex{Schroeder, Rudolf Alexander 26.01.1878 – 22.08.1962@\textsc{Schröder, Rudolf Alexander} (26.01.1878 – 22.08.1962), \emph{Schriftsteller/Schriftstellerin}|pw}’sche Homer\pwindex{Homer @\textsc{Homer}, \emph{Schriftsteller/Schriftstellerin}|pw}überſetzung\pwindex{Odyssee@\emph{Odyssee}|pwv} anbelangt, so bringen
               Sie diesen Wunsch vielleicht Dr. Bie\pwindex{Bie, Oskar 09.02.1864 – 21.04.1938@\textsc{Bie, Oskar} (09.02.1864 – 21.04.1938), \emph{Schriftsteller/Schriftstellerin, Journalist/Journalistin, Redakteur/Redakteurin}|pw} direkt
               schriftlich zur Kenntnis.\pend
           
\pstart
           Medardus\pwindex{junge Medardus. Dramatische Historie in einem Vorspiel und fuenf Aufzuegen@\emph{Der junge Medardus. Dramatische Historie in einem Vorspiel und fünf Aufzügen}|pw} hätte am Tage der Erstaufführung im
               Buchhandel erscheinen sollen, zurückgezogen wurde er nie, vielmehr ist er gerade in
               den letzten Tagen angenommen worden und soll im Herbst gespielt werden, bei welcher
               Gelegenheit auch das Buch herauskommen wird.\pend
           
\pstart
           Auf Wiedersehen und besten Gruss!{\\[\baselineskip]}Ihr{\\[\baselineskip]}\spacefill\mbox{{[}hs. :{]} Arthur Schnitzler}\pend
           \leftskip=0em{}\selectlanguage{ngerman}\endnumbering\briefempfaengerindex{Ehrenstein, Albert@\textsc{Ehrenstein, Albert}!zzzSchnitzler, Arthur@\emph{von Arthur Schnitzler}!1910-02-121@{12. 2. 1910}|)be}\mylabel{L01913h}  \normalsize

\doendnotes{C}
\bigskip
\vfill

\clearpage

\footnotesize

\lohead{\textsc{register}}

% Definiere theindex-Environment komplett neu ohne reledmac
\makeatletter
\renewenvironment{theindex}{%
  \section*{\indexname}%
  \setlength{\parindent}{0pt}%
  \setlength{\parskip}{0pt plus 0.3pt}%
  \let\item\@idxitem
}{%
  \clearpage
}
\makeatother

\IfFileExists{\jobname-pw.ind}{\input{\jobname-pw.ind}}{}

\end{document}

      