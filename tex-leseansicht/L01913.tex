%% latex-leseansicht-vorspann.tex
%% Vorspann für die Leseansicht.
%% Lädt die gemeinsame Datei latex-vorspann.tex mit nicht gesetztem Schalter.

\newif\ifkorrekturansicht
\korrekturansichtfalse

\input{../tex-inputs/latex-vorspann}


         
         \renewcommand{\erwaehntePersonen}{Personen: Oskar Bie, Albert Ehrenstein,  Homer, Rudolf Alexander Schröder}
         \renewcommand{\erwaehnteOrte}{Orte: Edmund-Weiß-Gasse 7, Wien}
         \renewcommand{\erwaehnteWerke}{Werke: Der junge Medardus. Dramatische Historie in einem Vorspiel und fünf Aufzügen, Odyssee, Tubutsch}
               \section[Arthur Schnitzler an Albert Ehrenstein, 12. 2. 1910]{ Arthur Schnitzler an Albert Ehrenstein, 12. 2. 1910}\nopagebreak\mylabel{v}\rehead{ }\begin{ledgroupsized}[t]{13cm}\normalsize\beginnumbering\briefempfaengerindex{Ehrenstein, Albert@\textsc{Ehrenstein, Albert}!zzzSchnitzler, Arthur@\emph{von Arthur Schnitzler}!1910-02-121@{12. 2. 1910}|(be} \toendnotes[C]{\smallbreak\pagebreak[2]} \Standort{Jerusalem, The National Library of Israel, ARC. Ms. Var. 306 1 118.}
\physDesc{Brief, 1 Blatt, 1 Seite, 731 Zeichen
\newline{}Schreibmaschine
\newline{}Handschrift: schwarze Tinte (\noindent{}Unterschrift)}\Standort{DLA, A:Schnitzler, 85.1.642,3.}
\physDesc{Brief, Durchschlag, 1 Blatt, 1 Seite, 731 Zeichen
\newline{}Schreibmaschine
\newline{}Handschrift: roter Buntstift, lateinische Kurrent (\noindent{}Beschriftung: »Ehrenstein«)}\toendnotes[C]{\smallbreak}\pstart
           \noindent{}{\pb}\textcolor{gray}{\textbf{Dr. Arthur Schnitzler}}\hfill 12. 2. 1910.\pend
           \pstart
           \textcolor{gray}{\textbf{Wien XVIII. Spoettelgasse 7\oindex{Edmund-Weiss-Gasse 7@\textbf{Edmund-Weiß-Gasse 7}|pw}.}}\pend
           \pstart{}Lieber Herr Ehrenstein!\pend\pstart
           Aus dem Brief von Bie\pwindex{Bie, Oskar 09.02.1864 – 21.04.1938@\textsc{Bie, Oskar} (09.02.1864 – 21.04.1938), \emph{Schriftsteller, Journalist, Redakteur}|pw} an Sie ist zu entnehmen,
               dass er »Tubutsch\pwindex{Ehrenstein, Albert 23.12.1886 – 08.04.1950@\textsc{Ehrenstein, Albert} (23.12.1886 – 08.04.1950), \emph{Schriftsteller}!Tubutsch1911@\strich\emph{Tubutsch} {[}1911{]}|pw}« nicht veröffentlichen will,
               dass aber für Ihre nächsten Einsendungen aufrichtiges Interesse und daher auch
               Druckchancen vorhanden sind. Das mit dem Wien\oindex{Wien@\textbf{Wien}|pw}er
               Leben müssen Sie nicht so wörtlich nehmen. Was die Schröder\pwindex{Schroeder, Rudolf Alexander 26.01.1878 – 22.08.1962@\textsc{Schröder, Rudolf Alexander} (26.01.1878 – 22.08.1962), \emph{Schriftsteller}|pw}’sche Homer\pwindex{Homer @\textsc{Homer}, \emph{Schriftsteller}|pw}überſetzung\pwindex{Homer @\textsc{Homer}, \emph{Schriftsteller}!Odyssee@\strich\emph{Odyssee}|pwv} anbelangt, so bringen
               Sie diesen Wunsch vielleicht Dr. Bie\pwindex{Bie, Oskar 09.02.1864 – 21.04.1938@\textsc{Bie, Oskar} (09.02.1864 – 21.04.1938), \emph{Schriftsteller, Journalist, Redakteur}|pw} direkt
               schriftlich zur Kenntnis.\pend
           \pstart
           Medardus\pwindex{Schnitzler, Arthur 15.05.1862 – 21.10.1931@\textsc{Schnitzler, Arthur} (15.05.1862 – 21.10.1931), \emph{Schriftsteller, Mediziner}!junge Medardus. Dramatische Historie in einem Vorspiel und fuenf Aufzuegen1910-10-26@\strich\emph{Der junge Medardus. Dramatische Historie in einem Vorspiel und fünf Aufzügen} {[}1910-10-26{]}|pw} hätte am Tage der Erstaufführung im
               Buchhandel erscheinen sollen, zurückgezogen wurde er nie, vielmehr ist er gerade in
               den letzten Tagen angenommen worden und soll im Herbst gespielt werden, bei welcher
               Gelegenheit auch das Buch herauskommen wird.\pend
           \pstart
           Auf Wiedersehen und besten Gruss!{\\[\baselineskip]}Ihr{\\[\baselineskip]}\spacefill\mbox{{[}hs. Schnitzler:{]} Arthur Schnitzler}\pend
           \leftskip=0em{}
         
         \endnumbering\mylabel{h}\end{ledgroupsized}  \newcommand{\dateiname}{L01913}\newcommand{\titel}{Arthur Schnitzler an Albert Ehrenstein, 12. 2. 1910}\newcommand{\editorInnen}{Martin Anton Müller und Gerd-Hermann Susen}%% latex-leseansicht-abspann.tex
%% Abspann für die Leseansicht.
%% Der Schalter \ifkorrekturansicht ist bereits durch den Vorspann gesetzt.

%% latex-abspann.tex
%% Gemeinsamer Abspann für Korrekturansicht und Leseansicht.
%% Setzt den Schalter \ifkorrekturansicht voraus (gesetzt in den
%% einbindenden Dateien latex-korrekturansicht-abspann.tex bzw.
%% latex-leseansicht-abspann.tex).
%% ---------------------------------------------------------------

\normalsize

% Das esempio-Environment wird nur in der Leseansicht benötigt
\ifkorrekturansicht\else
\newenvironment{esempio}[3]%
{
    \vspace{1.5ex}
    \rlap{\underline{#1}}
    \par
    \setlength{\parindent}{0cm}
    \nopagebreak
    \leftskip=#2cm
    \rightskip=#3cm
}
{
    \par
}
\fi

\doendnotes{C}
\bigskip
\vfill

\clearpage

\footnotesize

\ifkorrekturansicht
  \lohead{\textsc{register}}
\fi

% theindex-Environment neu definieren ohne reledmac
\makeatletter
\renewenvironment{theindex}{%
  \ifkorrekturansicht
    \section*{\indexname}%
  \else
    \subsubsection*{Index der erwähnten Entitäten}%
  \fi
  \setlength{\parindent}{0pt}%
  \setlength{\parskip}{0pt plus 0.3pt}%
  \let\item\@idxitem
}{%
  \ifkorrekturansicht\clearpage\fi
}
\makeatother

\IfFileExists{\jobname-pw.ind}{\input{\jobname-pw.ind}}{}

% Quellenangabe nur in der Leseansicht
\ifkorrekturansicht\else
% Fallback-Definitionen, falls die .tex-Datei \titel etc. nicht gesetzt hat
\providecommand{\titel}{}
\providecommand{\editorInnen}{}
\providecommand{\dateiname}{\jobname}

\vspace{3cm}

\vfill

\footnotesize
\textsc{Quelle}: \titel. Herausgegeben von {\editorInnen}. In: \emph{Arthur Schnitzler: Briefwechsel mit Autorinnen und Autoren}.
 Digitale Edition, https://schnitzler-briefe.acdh.oeaw.ac.at/{\dateiname}.html (Stand \today)
\fi

\end{document}


      