%% latex-korrekturansicht-vorspann.tex
%% Vorspann für die Korrekturansicht.
%% Lädt die gemeinsame Datei latex-vorspann.tex mit gesetztem Schalter.

\newif\ifkorrekturansicht
\korrekturansichttrue

\input{../tex-inputs/latex-vorspann}


\section[Thomas Mann an Arthur Schnitzler, 9. 1. 1925]{L02430 Thomas Mann an Arthur Schnitzler, 9. 1. 1925}
\nopagebreak\mylabel{L02430v}
\rehead{ }\normalsize\beginnumbering\briefempfaengerindex{Schnitzler, Arthur@\textsc{Schnitzler, Arthur}!zzzMann, Thomas@\emph{von Thomas Mann}!1925-01-091@{9. 1. 1925}|(be}
\toendnotes[C]{\smallbreak\pagebreak[2]}\Standort{CUL, Schnitzler, B 67.}
\physDesc{Briefkarte, 665 Zeichen
\newline{}Handschrift: schwarze Tinte, deutsche Kurrent
\newline{}Schnitzler: mit rotem Buntstift drei Unterstreichungen }
\buchAbdrucke{\weitereDrucke{1) \emph{Modern Austrian Literature}, Jg. 7 (1974) Nr. 1/2, S. 24.} \weitereDrucke{2) Hans-Ulrich Lindken: \emph{Arthur Schnitzler. Aspekte und Akzente. Materialien zu Leben
                        und Werk}. Frankfurt am Main, Bern, Göttingen: \emph{Peter Lang} 1984, S. 199.} }\toendnotes[C]{\smallbreak}
\pstart
           {\pb}\textcolor{gray}{\textbf{\textsc{Dr. Thomas Mann}}}\hfill \textcolor{gray}{\textbf{MÜNCHEN\oindex{Muenchen@\textbf{München}, \emph{P.PPLA}|pw}}}{ }9. I. 25.\pend
           
\pstart
           \raggedleft{}\textcolor{gray}{\textbf{POSCHINGERSTR. 1\oindex{Poschingerstrasse@\textbf{Poschingerstraße}, \emph{Straße (K.STR)}|pw}}}\pend
           
\pstart{}Lieber und verehrter Herr Dr. Schnitzler,\pend\vspace{0.5em}
\pstart
           Dank für Ihr gütiges Eingehen auf den »Zauberberg\pwindex{Zauberberg. Roman@\emph{Der Zauberberg. Roman}|pw}«! Es freut mich beſonders, daß Sie an dem guten Joachim\pwindex{Zauberberg. Roman@\emph{Der Zauberberg. Roman}|pwv}{ }ſo teilnehmen, der ja gewiß der Beſte iſt von dem
               ganzen Gelichter. Ich war aufrichtig traurig an dem Tage, wo ich ihn zur Ruhe
               gebracht hatte. – Und Humor des Todes! Ja, das Buch\pwindex{Zauberberg. Roman@\emph{Der Zauberberg. Roman}|pwv} will eine Verſpot{\pb}tung des Todes ſein, eine antiromantiſche
               Desilluſionierung und ein europäiſcher\oindex{Europa@\textbf{Europa}, \emph{Kontinent (A.KNT)}|pw} Ruf zum
               Leben. Es wird vielfach falſch geleſen.\pend
           
\pstart
           Wie gern ſpräche ich einmal mit Ihnen darüber! Ob mich mein Weg dieſen Winter noch
               oder im Frühjahr nach Wien\oindex{Wien@\textbf{Wien}, \emph{A.ADM2}|pw} führt? Es iſt nicht
               ganz ausgeſchloſſen.\pend
           
\pstart
           In herzlicher Ergebenheit{\\[\baselineskip]}Ihr{\\[\baselineskip]}\spacefill\mbox{Thomas Mann.}\pend
           \leftskip=0em{}\selectlanguage{ngerman}\endnumbering\briefempfaengerindex{Schnitzler, Arthur@\textsc{Schnitzler, Arthur}!zzzMann, Thomas@\emph{von Thomas Mann}!1925-01-091@{9. 1. 1925}|)be}\mylabel{L02430h}  \normalsize

\doendnotes{C}
\bigskip
\vfill

\clearpage

\footnotesize

\lohead{\textsc{register}}

% Definiere theindex-Environment komplett neu ohne reledmac
\makeatletter
\renewenvironment{theindex}{%
  \section*{\indexname}%
  \setlength{\parindent}{0pt}%
  \setlength{\parskip}{0pt plus 0.3pt}%
  \let\item\@idxitem
}{%
  \clearpage
}
\makeatother

\IfFileExists{\jobname-pw.ind}{\input{\jobname-pw.ind}}{}

\end{document}

      