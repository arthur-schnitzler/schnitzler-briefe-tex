%% latex-korrekturansicht-vorspann.tex
%% Vorspann für die Korrekturansicht.
%% Lädt die gemeinsame Datei latex-vorspann.tex mit gesetztem Schalter.

\newif\ifkorrekturansicht
\korrekturansichttrue

\input{../tex-inputs/latex-vorspann}


\section[Hugo von Hofmannsthal an Arthur Schnitzler, 10. 4. {[}1912{]}]{L02059 Hugo von Hofmannsthal an Arthur Schnitzler, 10. 4. {[}1912{]}}
\nopagebreak\mylabel{L02059v}
\rehead{ }\normalsize\beginnumbering\briefempfaengerindex{Schnitzler, Arthur@\textsc{Schnitzler, Arthur}!zzzHofmannsthal, Hugo von@\emph{von Hugo von Hofmannsthal}!1912-04-101@{10. 4. {[}1912{]}}|(be}
\toendnotes[C]{\smallbreak\pagebreak[2]}\Standort{CUL, Schnitzler, B 43.}
\physDesc{Briefkarte, 165 Zeichen
\newline{}Handschrift: schwarze Tinte, deutsche Kurrent
\newline{}Schnitzler: mit Bleistift die Jahreszahl ergänzt: »912« und beschriftet: »\textsc{Hugo}« 
\newline{}Ordnung: mit Bleistift von unbekannter Hand doppelt nummeriert:
                                    »336« }
\buchAbdrucke{\weitereDrucke{Hugo von Hofmannsthal, Arthur Schnitzler: \emph{Briefwechsel}. Frankfurt am Main: \emph{S. Fischer} 1964, S. 265.} }\toendnotes[C]{\smallbreak}
\pstart
           \raggedleft{}{\pb}10 IV\pend
           
\pstart{}mein lieber Arthur\pend\vspace{0.5em}
\pstart
           nun iſt es wieder endlos daſs man einander nicht geſehen hat. Würdet Ihr’s uns nicht
               für einen Abend Anfang \label{K_L02059-1v}\edtext{nächſter
                  Woche}{\lemma{\textnormal{\emph{nächſter
                  Woche}}}\Cendnote{\textnormal{Siehe A. S.: \emph{Tagebuch}, 17. 4. 1912.
               }}}\label{K_L02059-1} wiſſen lassen? \pend
           
\pstart
           Ihr{\\[\baselineskip]}\spacefill\mbox{Hugo}\pend
           \leftskip=0em{}\selectlanguage{ngerman}\endnumbering\briefempfaengerindex{Schnitzler, Arthur@\textsc{Schnitzler, Arthur}!zzzHofmannsthal, Hugo von@\emph{von Hugo von Hofmannsthal}!1912-04-101@{10. 4. {[}1912{]}}|)be}\mylabel{L02059h}  \normalsize

\doendnotes{C}
\bigskip
\vfill

\clearpage

\footnotesize

\lohead{\textsc{register}}

% Definiere theindex-Environment komplett neu ohne reledmac
\makeatletter
\renewenvironment{theindex}{%
  \section*{\indexname}%
  \setlength{\parindent}{0pt}%
  \setlength{\parskip}{0pt plus 0.3pt}%
  \let\item\@idxitem
}{%
  \clearpage
}
\makeatother

\IfFileExists{\jobname-pw.ind}{\input{\jobname-pw.ind}}{}

\end{document}

      