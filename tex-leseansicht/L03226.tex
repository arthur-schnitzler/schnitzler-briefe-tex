%% latex-leseansicht-vorspann.tex
%% Vorspann für die Leseansicht.
%% Lädt die gemeinsame Datei latex-vorspann.tex mit nicht gesetztem Schalter.

\newif\ifkorrekturansicht
\korrekturansichtfalse

\input{../tex-inputs/latex-vorspann}

\begin{center}
            \textcolor{red}{ENTWURF, NICHT FERTIG KORRIGIERT}
                      \end{center}
            
         \renewcommand{\erwaehnteOrte}{Orte: Berlin, Dessauer Straße}
         \renewcommand{\erwaehnteWerke}{}
               \section[ Paul Goldmann an Arthur Schnitzler, 12. 10. {[}1902{]}]{ Paul Goldmann an Arthur Schnitzler, 12. 10. {[}1902{]}}\nopagebreak\mylabel{v}\rehead{ }\begin{ledgroupsized}[t]{13cm}\normalsize\beginnumbering \toendnotes[C]{\smallbreak\pagebreak[2]} \Standort{DLA, A:Schnitzler, HS.NZ85.1.3172.}
\physDesc{Brief, 1 Blatt, 2 Seiten
\newline{}Handschrift: blaue Tinte, deutsche Kurrent
\newline{}Schnitzler: mit Bleistift das Jahr »{[}1{]}902«
                                            vermerkt }\pstart
           \noindent{}\raggedleft{}{\pb}\textcolor{gray}{\textbf{DESSAUERSTRASSE 19}}\oindex{Dessauer Strasse@\textbf{Dessauer Straße}|pw}\pend
           \pstart
           Berlin\oindex{Berlin@\textbf{Berlin}|pw}, 1\textcolor{gray}{2}.
                            Okt.\pend
           \pstart\center{}Mein lieber Freund,\pend\pstart
           Sei herzlichſt willkommen! Ich freue mich unendlich, daß Du da\textcolor{red}{\textsuperscript{\textbf{KEY}}} biſt! \pend
           \pstart
           Ich habe wahnſinnig zu thun, daß es mir unmöglich iſt, während des Tages zu
                    Dir zu kommen. Komm’ auch nicht zu mir, denn ich habe keine freie Viertelſtunde.
                    Am Beſten iſt es wohl, wir treffen uns Abends in der \textsc{Première} von »Schall und Rauch\textcolor{red}{\textsuperscript{\textbf{KEY}}}«. Ein \strikeout{\textcolor{gray}{Stü}} Drama »Rauſch\textcolor{red}{\textsuperscript{\textbf{KEY}}}« von \textsc{Strindberg\textcolor{red}{\textsuperscript{\textbf{KEY}}}}{\pb} wird geſpielt. Es ſoll ein intereſſanter
                    Abend werden. Ich lege ein Billet bei; und wenn Du ganz lieb ſein willſt, ſo
                    kommſt Du gegen 7 Uhr zu mir, mich ins Theater\textcolor{red}{\textsuperscript{\textbf{KEY}}} abholen. {\\[\baselineskip]}Von Herzen\pend
           \leftskip=0em{}\pstart
           {\\[\baselineskip]}Dein\pend
           \leftskip=0em{}\pstart
           {\\[\baselineskip]}\spacefill\mbox{Paul Goldmn }\pend
           \leftskip=0em{}
         
         \endnumbering\mylabel{h}\end{ledgroupsized}\begin{anhang}\end{anhang}\newcommand{\dateiname}{L03226}\newcommand{\titel}{Paul Goldmann an Arthur Schnitzler, 12. 10. [1902]}\newcommand{\editorInnen}{Martin Anton Müller und Laura Untner}%% latex-leseansicht-abspann.tex
%% Abspann für die Leseansicht.
%% Der Schalter \ifkorrekturansicht ist bereits durch den Vorspann gesetzt.

%% latex-abspann.tex
%% Gemeinsamer Abspann für Korrekturansicht und Leseansicht.
%% Setzt den Schalter \ifkorrekturansicht voraus (gesetzt in den
%% einbindenden Dateien latex-korrekturansicht-abspann.tex bzw.
%% latex-leseansicht-abspann.tex).
%% ---------------------------------------------------------------

\normalsize

% Das esempio-Environment wird nur in der Leseansicht benötigt
\ifkorrekturansicht\else
\newenvironment{esempio}[3]%
{
    \vspace{1.5ex}
    \rlap{\underline{#1}}
    \par
    \setlength{\parindent}{0cm}
    \nopagebreak
    \leftskip=#2cm
    \rightskip=#3cm
}
{
    \par
}
\fi

\doendnotes{C}
\bigskip
\vfill

\clearpage

\footnotesize

\ifkorrekturansicht
  \lohead{\textsc{register}}
\fi

% theindex-Environment neu definieren ohne reledmac
\makeatletter
\renewenvironment{theindex}{%
  \ifkorrekturansicht
    \section*{\indexname}%
  \else
    \subsubsection*{Index der erwähnten Entitäten}%
  \fi
  \setlength{\parindent}{0pt}%
  \setlength{\parskip}{0pt plus 0.3pt}%
  \let\item\@idxitem
}{%
  \ifkorrekturansicht\clearpage\fi
}
\makeatother

\IfFileExists{\jobname-pw.ind}{\input{\jobname-pw.ind}}{}

% Quellenangabe nur in der Leseansicht
\ifkorrekturansicht\else
% Fallback-Definitionen, falls die .tex-Datei \titel etc. nicht gesetzt hat
\providecommand{\titel}{}
\providecommand{\editorInnen}{}
\providecommand{\dateiname}{\jobname}

\vspace{3cm}

\vfill

\footnotesize
\textsc{Quelle}: \titel. Herausgegeben von {\editorInnen}. In: \emph{Arthur Schnitzler: Briefwechsel mit Autorinnen und Autoren}.
 Digitale Edition, https://schnitzler-briefe.acdh.oeaw.ac.at/{\dateiname}.html (Stand \today)
\fi

\end{document}


      