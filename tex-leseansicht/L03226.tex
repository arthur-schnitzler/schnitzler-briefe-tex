%% latex-leseansicht-vorspann.tex
%% Vorspann für die Leseansicht.
%% Lädt die gemeinsame Datei latex-vorspann.tex mit nicht gesetztem Schalter.

\newif\ifkorrekturansicht
\korrekturansichtfalse

\input{../tex-inputs/latex-vorspann}


\section[ Paul Goldmann an Arthur Schnitzler, 12. 10. [1902]]{L03226 Paul Goldmann an Arthur Schnitzler,  12. 10. [1902]}
\nopagebreak\mylabel{L03226v}
\rehead{ }\normalsize\beginnumbering\briefempfaengerindex{Schnitzler, Arthur@\textsc{Schnitzler, Arthur}!zzzGoldmann, Paul@\emph{von Paul Goldmann}!1902-10-121@{12. 10. [1902]}|(be}
\toendnotes[C]{\smallbreak\pagebreak[2]}
\correspDesc{Versand  durch Paul Goldmann am 12. 10. [1902] in Berlin
\newline{}Erhalt  durch Arthur Schnitzler am 13. 10. [1902] in Berlin}\toendnotes[C]{\smallbreak}
\Standort{DLA, A:Schnitzler, HS.NZ85.1.3172.}
\physDesc{Brief, 1 Blatt, 2 Seiten, 566 Zeichen
\newline{}Handschrift: blaue Tinte, deutsche Kurrent
\newline{}Schnitzler: mit Bleistift das Jahr »902« vermerkt }\toendnotes[C]{\smallbreak}
\pstart
           \raggedleft{}{\pb}\textcolor{gray}{\textbf{DESSAUERSTRASSE 19}}\oindex{Dessauer Straße@\textbf{Dessauer Straße}, \emph{Straße}|pw}\pend
           
\pstart
           Berlin\oindex{Berlin@\textbf{Berlin}, \emph{Hauptstadt}|pw}, 12. Okt.\pend
           
\pstart\center{}Mein lieber Freund,\pend\vspace{0.5em}
\pstart
           Sei herzlichſt willkommen! Ich freue mich unendlich, daß Du \label{K_L03226-1v}\edtext{da\oindex{Berlin@\textbf{Berlin}, \emph{Hauptstadt}|pwv}}{\lemma{\textnormal{\emph{da}}}\Cendnote{\textnormal{Schnitzler reiste am 12. 10. 1902 in Wien\oindex{Wien@\textbf{Wien}, \emph{Verwaltungsgebiet}|pwk} ab und kam am nächsten Tag in Berlin\oindex{Berlin@\textbf{Berlin}, \emph{Hauptstadt}|pwk} an, wo er bis 18. 10. 1902 blieb.
                  Danach reiste er weiter nach Breslau\oindex{Breslau@\textbf{Breslau}|pwk}.}}}\label{K_L03226-1}
               biſt!\pend
           
\pstart
           Ich habe wahnſinnig zu thun, daß es mir unmöglich iſt, während des Tages zu Dir zu
               kommen. Komm’ auch nicht zu mir; denn ich habe keine freie Viertelſtunde. Am Beſten
               iſt es wohl, wir \label{K_L03226-2v}\edtext{treffen uns
                  Abends}{\lemma{\textnormal{\emph{treffen uns
                  Abends}}}\Cendnote{\textnormal{Siehe A. S.: \emph{Tagebuch}, 13. 10. 1902.
               }}}\label{K_L03226-2} in der \textsc{\begin{otherlanguage}{french}Première\end{otherlanguage}} von »Schall und Rauch\orgindex{Schall und Rauch@Schall und Rauch|pw}«. Ein \strikeout{\textcolor{gray}{Stü}} Drama »Rauſch\pwindex{Strindberg, August 22.\,1.\,1849 Stockholm – 14.\,5.\,1912 ebd.@\textsc{Strindberg, August} (22.\,1.\,1849 Stockholm – 14.\,5.\,1912 ebd.), \emph{Schriftsteller}!Rausch@\strich\emph{Rausch}|pw}« von \textsc{Strindberg\pwindex{Strindberg, August 22.\,1.\,1849 Stockholm – 14.\,5.\,1912 ebd.@\textsc{Strindberg, August} (22.\,1.\,1849 Stockholm – 14.\,5.\,1912 ebd.), \emph{Schriftsteller}|pw}}{ }{\pb}wird geſpielt. Es{ }ſoll ein intereſſanter Abend
               werden. Ich lege ein Billet bei; und wenn Du ganz lieb{ }ſein willſt,{ }ſo kommſt Du
               gegen 7 Uhr zu mir, mich ins Theater\oindex{Schall und Rauch@\textbf{Schall und Rauch}, \emph{Kabarett}|pwv} abholen.\pend
           
\pstart
           Von Herzen {\\[\baselineskip]}Dein {\\[\baselineskip]}\spacefill\mbox{Paul Goldm}\pend
           \leftskip=0em{}\selectlanguage{ngerman}\endnumbering\briefempfaengerindex{Schnitzler, Arthur@\textsc{Schnitzler, Arthur}!zzzGoldmann, Paul@\emph{von Paul Goldmann}!1902-10-121@{12. 10. [1902]}|)be}\mylabel{L03226h}  \newcommand{\dateiname}{L03226}\newcommand{\titel}{Paul Goldmann an Arthur Schnitzler, 12. 10. [1902]}\newcommand{\editorInnen}{Martin Anton Müller und Laura Untner}%% latex-leseansicht-abspann.tex
%% Abspann für die Leseansicht.
%% Der Schalter \ifkorrekturansicht ist bereits durch den Vorspann gesetzt.

%% latex-abspann.tex
%% Gemeinsamer Abspann für Korrekturansicht und Leseansicht.
%% Setzt den Schalter \ifkorrekturansicht voraus (gesetzt in den
%% einbindenden Dateien latex-korrekturansicht-abspann.tex bzw.
%% latex-leseansicht-abspann.tex).
%% ---------------------------------------------------------------

\normalsize

% Das esempio-Environment wird nur in der Leseansicht benötigt
\ifkorrekturansicht\else
\newenvironment{esempio}[3]%
{
    \vspace{1.5ex}
    \rlap{\underline{#1}}
    \par
    \setlength{\parindent}{0cm}
    \nopagebreak
    \leftskip=#2cm
    \rightskip=#3cm
}
{
    \par
}
\fi

\doendnotes{C}
\bigskip
\vfill

\clearpage

\footnotesize

\ifkorrekturansicht
  \lohead{\textsc{register}}
\fi

% theindex-Environment neu definieren ohne reledmac
\makeatletter
\renewenvironment{theindex}{%
  \ifkorrekturansicht
    \section*{\indexname}%
  \else
    \subsubsection*{Index der erwähnten Entitäten}%
  \fi
  \setlength{\parindent}{0pt}%
  \setlength{\parskip}{0pt plus 0.3pt}%
  \let\item\@idxitem
}{%
  \ifkorrekturansicht\clearpage\fi
}
\makeatother

\IfFileExists{\jobname-pw.ind}{\input{\jobname-pw.ind}}{}

% Quellenangabe nur in der Leseansicht
\ifkorrekturansicht\else
% Fallback-Definitionen, falls die .tex-Datei \titel etc. nicht gesetzt hat
\providecommand{\titel}{}
\providecommand{\editorInnen}{}
\providecommand{\dateiname}{\jobname}

\vspace{3cm}

\vfill

\footnotesize
\textsc{Quelle}: \titel. Herausgegeben von {\editorInnen}. In: \emph{Arthur Schnitzler: Briefwechsel mit Autorinnen und Autoren}.
 Digitale Edition, https://schnitzler-briefe.acdh.oeaw.ac.at/{\dateiname}.html (Stand \today)
\fi

\end{document}


