%% latex-leseansicht-vorspann.tex
%% Vorspann für die Leseansicht.
%% Lädt die gemeinsame Datei latex-vorspann.tex mit nicht gesetztem Schalter.

\newif\ifkorrekturansicht
\korrekturansichtfalse

\input{../tex-inputs/latex-vorspann}


         
         \renewcommand{\erwaehntePersonen}{Personen: Hermann Bahr, Anna Bahr-Mildenburg, Olga Schnitzler}
         \renewcommand{\erwaehnteOrte}{Orte: London, Wien}
         \renewcommand{\erwaehnteWerke}{Werke: Der junge Medardus. Dramatische Historie in einem Vorspiel und fünf Aufzügen}
               \section[Hermann Bahr an Arthur Schnitzler, 22. 10. 1910]{ Hermann Bahr an Arthur Schnitzler, 22. 10. 1910}\nopagebreak\mylabel{v}\rehead{ }\begin{ledgroupsized}[t]{13cm}\normalsize\beginnumbering\briefempfaengerindex{Schnitzler, Arthur@\textsc{Schnitzler, Arthur}!zzzBahr, Hermann@\emph{von Hermann Bahr}!1910-10-221@{22. 10. 1910}|(be} \toendnotes[C]{\smallbreak\pagebreak[2]} \Standort{CUL, Schnitzler, B 5b.}
\physDesc{Brief, 1 Blatt, 1 Seite, 414 Zeichen
\newline{}Handschrift: schwarze Tinte, deutsche Kurrent
\newline{}Schnitzler: mit Bleistift ergänzt »Bahr« 
\newline{}Ordnung: mit Bleistift von unbekannter Hand nummeriert:
                                    »168« }\buchAbdrucke{\weitereDrucke{Hermann Bahr, Arthur Schnitzler: \emph{Briefwechsel, Aufzeichnungen, Dokumente (1891–1931)}. Hg. Kurt Ifkovits und Martin Anton Müller. Göttingen: \emph{Wallstein} 2018, S. 441.} }\toendnotes[C]{\smallbreak}\pstart
           \raggedleft{}{\pb}London\oindex{London@\textbf{London}|pw}{ }22. 10. 10\pend
           \pstart{}Lieber Arthur!\pend\pstart
           Herzlichſten Dank für Deinen Brief. Ich freue mich ſehr auf das \label{LL329-1v}Buch\pwindex{Schnitzler, Arthur 15.05.1862 – 21.10.1931@\textsc{Schnitzler, Arthur} (15.05.1862 – 21.10.1931), \emph{Schriftsteller, Mediziner}!junge Medardus. Dramatische Historie in einem Vorspiel und fuenf
                  Aufzuegen1910-10-26@\strich\emph{Der junge Medardus. Dramatische Historie in einem Vorspiel und fünf Aufzügen} {[}1910-10-26{]}|pwv}. Wenn das Stück
                  wirklich erſt am 19. November iſt, kann ich leider nicht bei der
                  Première ſein, ich muß am 17. wieder auf eine der leidigen \label{K_L01969-1v}\edtext{Tourneen}{\lemma{\textnormal{\emph{Tourneen}}}\Cendnote{\textnormal{vgl. Hermann Bahr an Arthur Schnitzler, 26. 9. 1910}}}\label{K_L01969-1h}\label{LL329-1h}, mit denen der Menſch Geld verdient.\pend
           \pstart
           Grüße Deine liebe Frau\pwindex{Schnitzler, Olga 17.01.1882 – 13.01.1970@\textsc{Schnitzler, Olga} (17.01.1882 – 13.01.1970), \emph{Schauspielerin, Sängerin}|pwv}
               herzlichſt und ſei ſelbſt in alter Freundſchaft gegrüßt von\pend
           \pstart
           Deinem{\\[\baselineskip]}\spacefill\mbox{Hermann}\pend
           \leftskip=0em{}\pstart
           \noindent{}Auch meine Frau\pwindex{Bahr-Mildenburg, Anna 29.11.1872 – 27.01.1947@\textsc{Bahr-Mildenburg, Anna} (29.11.1872 – 27.01.1947), \emph{Sängerin}|pwv} läßt Dich
                  ſchönſtens grüßen.\pend
           
         
         \endnumbering\mylabel{h}\end{ledgroupsized}  \newcommand{\dateiname}{L01969}\newcommand{\titel}{Hermann Bahr an Arthur Schnitzler, 22. 10. 1910}\newcommand{\editorInnen}{ Kurt Ifkovits,  Martin Anton Müller}%% latex-leseansicht-abspann.tex
%% Abspann für die Leseansicht.
%% Der Schalter \ifkorrekturansicht ist bereits durch den Vorspann gesetzt.

%% latex-abspann.tex
%% Gemeinsamer Abspann für Korrekturansicht und Leseansicht.
%% Setzt den Schalter \ifkorrekturansicht voraus (gesetzt in den
%% einbindenden Dateien latex-korrekturansicht-abspann.tex bzw.
%% latex-leseansicht-abspann.tex).
%% ---------------------------------------------------------------

\normalsize

% Das esempio-Environment wird nur in der Leseansicht benötigt
\ifkorrekturansicht\else
\newenvironment{esempio}[3]%
{
    \vspace{1.5ex}
    \rlap{\underline{#1}}
    \par
    \setlength{\parindent}{0cm}
    \nopagebreak
    \leftskip=#2cm
    \rightskip=#3cm
}
{
    \par
}
\fi

\doendnotes{C}
\bigskip
\vfill

\clearpage

\footnotesize

\ifkorrekturansicht
  \lohead{\textsc{register}}
\fi

% theindex-Environment neu definieren ohne reledmac
\makeatletter
\renewenvironment{theindex}{%
  \ifkorrekturansicht
    \section*{\indexname}%
  \else
    \subsubsection*{Index der erwähnten Entitäten}%
  \fi
  \setlength{\parindent}{0pt}%
  \setlength{\parskip}{0pt plus 0.3pt}%
  \let\item\@idxitem
}{%
  \ifkorrekturansicht\clearpage\fi
}
\makeatother

\IfFileExists{\jobname-pw.ind}{\input{\jobname-pw.ind}}{}

% Quellenangabe nur in der Leseansicht
\ifkorrekturansicht\else
% Fallback-Definitionen, falls die .tex-Datei \titel etc. nicht gesetzt hat
\providecommand{\titel}{}
\providecommand{\editorInnen}{}
\providecommand{\dateiname}{\jobname}

\vspace{3cm}

\vfill

\footnotesize
\textsc{Quelle}: \titel. Herausgegeben von {\editorInnen}. In: \emph{Arthur Schnitzler: Briefwechsel mit Autorinnen und Autoren}.
 Digitale Edition, https://schnitzler-briefe.acdh.oeaw.ac.at/{\dateiname}.html (Stand \today)
\fi

\end{document}


      