%% latex-leseansicht-vorspann.tex
%% Vorspann für die Leseansicht.
%% Lädt die gemeinsame Datei latex-vorspann.tex mit nicht gesetztem Schalter.

\newif\ifkorrekturansicht
\korrekturansichtfalse

\input{../tex-inputs/latex-vorspann}


         
         \renewcommand{\erwaehntePersonen}{Personen: Richard Beer-Hofmann, Clemens von Franckenstein, Jakob Wassermann}
         \renewcommand{\erwaehnteOrte}{Orte: Altaussee, Gasthaus Brunnthaler, San Martino di Castrozza, Tirol}
         \renewcommand{\erwaehnteWerke}{}
               \section[Hugo von Hofmannsthal an Arthur Schnitzler, 6. 8. 1899]{ Hugo von Hofmannsthal an Arthur Schnitzler, 6. 8. 1899}\nopagebreak\mylabel{v}\rehead{ }\begin{ledgroupsized}[t]{13cm}\normalsize\beginnumbering \toendnotes[C]{\smallbreak\pagebreak[2]} \Standort{CUL, Schnitzler, B 43.}
\physDesc{Postkarte, 358 Zeichen
\newline{}Handschrift: 1) schwarze Tinte, deutsche Kurrent\hspace{1em}2) schwarze Tinte, lateinische Kurrent (\noindent{}Adresse)\hspace{1em}
\newline{}Versand: 1) Stempel: »\nobreak{}\oindex{Altaussee@\textbf{Altaussee}|pwk}Alt-Aussee, 6 8 99\nobreak{}«.   2) Stempel: »\nobreak{}\oindex{San Martino di Castrozza@\textbf{San Martino di Castrozza}|pwk}San Martin{[}o di
                                          Castrozza{]}, 8{[}. 8. 99{]}\nobreak{}«. 
\newline{}Schnitzler: mit Bleistift datiert: »6/8 99« 
\newline{}Ordnung: mit Bleistift von unbekannter Hand nummeriert:
                                    »154« }\buchAbdrucke{\weitereDrucke{Hugo von Hofmannsthal, Arthur Schnitzler: \emph{Briefwechsel}. Hg. Therese Nickl und Heinrich Schnitzler. Frankfurt am Main: \emph{S. Fischer} 1964, S. 129.} }\toendnotes[C]{\smallbreak}\pstart{}{\pb}Herrn D\textsuperscript{r} Arthur Schnitzler\pend{}\pstart{}\label{K_L00957-1v}\edtext{San Martino di Castrozzo\oindex{San Martino di Castrozza@\textbf{San Martino di Castrozza}|pw}}{\lemma{\textnormal{\emph{San Martino di Castrozzo}}}\Cendnote{\textnormal{Schnitzler\pwindex{Schnitzler, Arthur 15.05.1862 – 21.10.1931@\textsc{Schnitzler, Arthur} (15.05.1862 – 21.10.1931), \emph{Schriftsteller, Mediziner}|pwk} kam am 10. 8. 1899 in San Martino di Castrozza\oindex{San Martino di Castrozza@\textbf{San Martino di Castrozza}|pwk} an und dürfte die
                     Karte vorgefunden haben.}}}\label{K_L00957-1h}\pend{}\pstart{}Tirol\oindex{Tirol@\textbf{Tirol}|pw}\pend{}{\bigskip}\pstart
           \noindent{}{\pb}Ich freu mich, zu denken daſs Ihr
               alle beiſa{\geminationm}en ſeid und dieſe ſchönen Tage und
               Sternennächte genießt. \textsc{Frankenſtein}\pwindex{Franckenstein, Clemens von 14.07.1875 – 19.08.1942@\textsc{Franckenstein, Clemens von} (14.07.1875 – 19.08.1942), \emph{Theaterleiter, Komponist, Dirigent}|pw} freut ſich ſehr auf Waſſermann\pwindex{Wassermann, Jakob 10.03.1873 – 01.01.1934@\textsc{Wassermann, Jakob} (10.03.1873 – 01.01.1934), \emph{Schriftsteller}|pw}. Ich
               erbitte von Richard\pwindex{Beer-Hofmann, Richard 1866-07-11 – 1945-09-26@\textsc{Beer-Hofmann, Richard} (1866-07-11 – 1945-09-26), \emph{Schriftsteller}|pw} noch nähere Nachrichten wo
               er 13\textsuperscript{ten} bis 16\textsuperscript{ten} iſt, ebenſo von Ihnen\pend
           \pstart
           \textsc{Alt-Aussee Gasthaus Bru{\geminationn}thaler\oindex{Gasthaus Brunnthaler@\textbf{Gasthaus Brunnthaler}|pw}}.\pend
           \pstart
           Bin ſehr erholt und wohl.\pend
           \pstart
           Herzlich Euer{\\[\baselineskip]}\spacefill\mbox{Hugo.}\pend
           \leftskip=0em{}
         
         \endnumbering\mylabel{h}\end{ledgroupsized}  \newcommand{\dateiname}{L00957}\newcommand{\titel}{Hugo von Hofmannsthal an Arthur Schnitzler, 6. 8. 1899}\newcommand{\editorInnen}{Martin Anton Müller und Gerd-Hermann Susen}%% latex-leseansicht-abspann.tex
%% Abspann für die Leseansicht.
%% Der Schalter \ifkorrekturansicht ist bereits durch den Vorspann gesetzt.

%% latex-abspann.tex
%% Gemeinsamer Abspann für Korrekturansicht und Leseansicht.
%% Setzt den Schalter \ifkorrekturansicht voraus (gesetzt in den
%% einbindenden Dateien latex-korrekturansicht-abspann.tex bzw.
%% latex-leseansicht-abspann.tex).
%% ---------------------------------------------------------------

\normalsize

% Das esempio-Environment wird nur in der Leseansicht benötigt
\ifkorrekturansicht\else
\newenvironment{esempio}[3]%
{
    \vspace{1.5ex}
    \rlap{\underline{#1}}
    \par
    \setlength{\parindent}{0cm}
    \nopagebreak
    \leftskip=#2cm
    \rightskip=#3cm
}
{
    \par
}
\fi

\doendnotes{C}
\bigskip
\vfill

\clearpage

\footnotesize

\ifkorrekturansicht
  \lohead{\textsc{register}}
\fi

% theindex-Environment neu definieren ohne reledmac
\makeatletter
\renewenvironment{theindex}{%
  \ifkorrekturansicht
    \section*{\indexname}%
  \else
    \subsubsection*{Index der erwähnten Entitäten}%
  \fi
  \setlength{\parindent}{0pt}%
  \setlength{\parskip}{0pt plus 0.3pt}%
  \let\item\@idxitem
}{%
  \ifkorrekturansicht\clearpage\fi
}
\makeatother

\IfFileExists{\jobname-pw.ind}{\input{\jobname-pw.ind}}{}

% Quellenangabe nur in der Leseansicht
\ifkorrekturansicht\else
% Fallback-Definitionen, falls die .tex-Datei \titel etc. nicht gesetzt hat
\providecommand{\titel}{}
\providecommand{\editorInnen}{}
\providecommand{\dateiname}{\jobname}

\vspace{3cm}

\vfill

\footnotesize
\textsc{Quelle}: \titel. Herausgegeben von {\editorInnen}. In: \emph{Arthur Schnitzler: Briefwechsel mit Autorinnen und Autoren}.
 Digitale Edition, https://schnitzler-briefe.acdh.oeaw.ac.at/{\dateiname}.html (Stand \today)
\fi

\end{document}


      