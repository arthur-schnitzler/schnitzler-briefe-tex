%% latex-korrekturansicht-vorspann.tex
%% Vorspann für die Korrekturansicht.
%% Lädt die gemeinsame Datei latex-vorspann.tex mit gesetztem Schalter.

\newif\ifkorrekturansicht
\korrekturansichttrue

\input{../tex-inputs/latex-vorspann}


\section[Hugo von Hofmannsthal an Arthur Schnitzler, 6. 8. 1899]{L00957 Hugo von Hofmannsthal an Arthur Schnitzler, 6. 8. 1899}
\nopagebreak\mylabel{L00957v}
\rehead{ }\normalsize\beginnumbering\briefempfaengerindex{Schnitzler, Arthur@\textsc{Schnitzler, Arthur}!zzzHofmannsthal, Hugo von@\emph{von Hugo von Hofmannsthal}!1899-08-061@{6. 8. 1899}|(be}
\toendnotes[C]{\smallbreak\pagebreak[2]}\Standort{CUL, Schnitzler, B 43.}
\physDesc{Postkarte, 358 Zeichen
\newline{}Handschrift: 1) schwarze Tinte, deutsche Kurrent\hspace{1em}2) schwarze Tinte, lateinische Kurrent (\noindent{}Adresse)\hspace{1em}
\newline{}Versand: 1) Stempel: »\nobreak{}\oindex{Altaussee@\textbf{Altaussee}, \emph{A.ADM3}|pwk}Alt-Aussee, 6 8 99\nobreak{}«.   2) Stempel: »\nobreak{}\oindex{San Martino di Castrozza@\textbf{San Martino di Castrozza}, \emph{P.PPL}|pwk}San Martin{[}o di
                                          Castrozza{]}, 8{[}. 8. 99{]}\nobreak{}«. 
\newline{}Schnitzler: mit Bleistift datiert: »6/8 99« 
\newline{}Ordnung: mit Bleistift von unbekannter Hand nummeriert:
                                    »154« }
\buchAbdrucke{\weitereDrucke{Hugo von Hofmannsthal, Arthur Schnitzler: \emph{Briefwechsel}. Frankfurt am Main: \emph{S. Fischer} 1964, S. 129.} }\toendnotes[C]{\smallbreak}\pstart{}{\pb}Herrn D\textsuperscript{r} Arthur Schnitzler\pend{}\pstart{}\label{K_L00957-1v}\edtext{San Martino di Castrozzo\oindex{San Martino di Castrozza@\textbf{San Martino di Castrozza}, \emph{P.PPL}|pw}}{\lemma{\textnormal{\emph{San Martino di Castrozzo}}}\Cendnote{\textnormal{Schnitzler kam am 10. 8. 1899 in San Martino di Castrozza\oindex{San Martino di Castrozza@\textbf{San Martino di Castrozza}, \emph{P.PPL}|pwk} an und dürfte die
                     Karte vorgefunden haben.}}}\label{K_L00957-1}\pend{}\pstart{}Tirol\oindex{Tirol@\textbf{Tirol}, \emph{A.ADM1}|pw}\pend{}{\bigskip}\vspace{1em}
\pstart
           \noindent{}{\pb}Ich freu mich, zu denken daſs Ihr
               alle beiſa{\geminationm}en ſeid und dieſe ſchönen Tage und
               Sternennächte genießt. \textsc{Frankenſtein}\pwindex{Franckenstein, Clemens von 14.07.1875 – 19.08.1942@\textsc{Franckenstein, Clemens von} (14.07.1875 – 19.08.1942), \emph{Theaterleiter/Theaterleiterin, Komponist/Komponistin, Dirigent/Dirigentin}|pw} freut ſich ſehr auf Waſſermann\pwindex{Wassermann, Jakob 10.03.1873 – 01.01.1934@\textsc{Wassermann, Jakob} (10.03.1873 – 01.01.1934), \emph{Schriftsteller/Schriftstellerin}|pw}. Ich
               erbitte von Richard\pwindex{Beer-Hofmann, Richard 1866-07-11 – 1945-09-26@\textsc{Beer-Hofmann, Richard} (1866-07-11 – 1945-09-26), \emph{Schriftsteller/Schriftstellerin}|pw} noch nähere Nachrichten wo
               er 13\textsuperscript{ten} bis 16\textsuperscript{ten} iſt, ebenſo von Ihnen\pend
           
\pstart
           \textsc{Alt-Aussee Gasthaus Bru{\geminationn}thaler\oindex{Gasthaus Brunnthaler@\textbf{Gasthaus Brunnthaler}, \emph{Hotel (K.HTL)}|pw}}.\pend
           
\pstart
           Bin ſehr erholt und wohl.\pend
           
\pstart
           Herzlich Euer{\\[\baselineskip]}\spacefill\mbox{Hugo.}\pend
           \leftskip=0em{}\selectlanguage{ngerman}\endnumbering\briefempfaengerindex{Schnitzler, Arthur@\textsc{Schnitzler, Arthur}!zzzHofmannsthal, Hugo von@\emph{von Hugo von Hofmannsthal}!1899-08-061@{6. 8. 1899}|)be}\mylabel{L00957h}  \normalsize

\doendnotes{C}
\bigskip
\vfill

\clearpage

\footnotesize

\lohead{\textsc{register}}

% Definiere theindex-Environment komplett neu ohne reledmac
\makeatletter
\renewenvironment{theindex}{%
  \section*{\indexname}%
  \setlength{\parindent}{0pt}%
  \setlength{\parskip}{0pt plus 0.3pt}%
  \let\item\@idxitem
}{%
  \clearpage
}
\makeatother

\IfFileExists{\jobname-pw.ind}{\input{\jobname-pw.ind}}{}

\end{document}

      