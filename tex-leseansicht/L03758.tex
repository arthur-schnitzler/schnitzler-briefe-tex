%% latex-korrekturansicht-vorspann.tex
%% Vorspann für die Korrekturansicht.
%% Lädt die gemeinsame Datei latex-vorspann.tex mit gesetztem Schalter.

\newif\ifkorrekturansicht
\korrekturansichttrue

\input{../tex-inputs/latex-vorspann}


\section[Arthur Schnitzler an Stefan Zweig, 29. 5. 1923]{L03758 Arthur Schnitzler an Stefan Zweig, 29. 5. 1923}
\nopagebreak\mylabel{L03758v}
\rehead{ }\normalsize\beginnumbering\briefempfaengerindex{Zweig, Stefan@\textsc{Zweig, Stefan}!zzzSchnitzler, Arthur@\emph{von Arthur Schnitzler}!1923-05-291@{29. 5. 1923}|(be}
\toendnotes[C]{\smallbreak\pagebreak[2]}\Standort{Jerusalem, National Library of Israel, ARC. Ms. Var. 305 1 58 Stefan Zweig Collection.}
\physDesc{Postkarte, 1 Blatt, 2 Seiten, 804 Zeichen
\newline{}Schreibmaschine
\newline{}Handschrift: Bleistift (\noindent{}Unterschrift)
\newline{}Versand: Stempel: »\nobreak{}\oindex{IX., Alsergrund@\textbf{IX., Alsergrund}, \emph{A.ADM3}|pwk}9/\textcolor{gray}{×} Wien 72, 30. V. 23, VIII\nobreak{}«.  }\toendnotes[C]{\smallbreak}\pstart{}{\pb}\label{T_L03758-1v}\edtext{\textcolor{gray}{\textbf{A. S.}}}{\lemma{\textnormal{\emph{A. S.}}}\Cendnote{\textnormal{ovaler Absenderkleber}}}\label{T_L03758-1}\pend{}\pstart{}\textcolor{gray}{\textbf{WIEN, XVIII.}}\oindex{XVIII., Waehring@\textbf{XVIII., Währing}, \emph{A.ADM3}|pw}\pend{}\pstart{}\textcolor{gray}{\textbf{STERNWARTESTR. 71}}\oindex{Sternwartestrasse 71@\textbf{Sternwartestraße 71}, \emph{Wohngebäude (K.WHS)}|pw}\pend{}{\bigskip}\pstart{}Herrn\pend{}\pstart{}Dr. Stefan Zweig\pend{}\pstart{}Salzburg\oindex{Salzburg@\textbf{Salzburg}, \emph{A.ADM2}|pw}.\pend{}\pstart{}Kapuzinerberg 5\oindex{Paschinger Schloessl@\textbf{Paschinger Schlössl}, \emph{Wohngebäude (K.WHS)}|pw}.\pend{}{\bigskip}\vspace{1em}
\pstart
           \raggedleft{}{\pb}29. 5. 1923. \pend
           
\pstart{}Lieber Herr Doktor.\pend\vspace{0.5em}
\pstart
           Vielen Dank, dass Sie mich auf diese bevorstehende Versteigerung aufmerksam gemacht
               haben. Das Buch\pwindex{Gaensemaennchen. Roman@\emph{Das Gänsemännchen. Roman}|pwv} ist mir
               offenbar gestohlen worden. Ich habe gleich an das betreffende Antiquariat\orgindex{Antiquariat Emil Hirsch@Antiquariat Emil Hirsch|pwv} geschrieben und das Buch\pwindex{Gaensemaennchen. Roman@\emph{Das Gänsemännchen. Roman}|pwv} zurückgefordert.\pend
           
\pstart
           \label{K_L03758-1v}\edtext{Eben}{\lemma{\textnormal{\emph{Eben}}}\Cendnote{\textnormal{Er kam am 27. 5. 1923 wieder in Wien\oindex{Wien@\textbf{Wien}, \emph{A.ADM2}|pwk}
                  an.}}}\label{K_L03758-1} komme ich von einer sehr schönen Reise nach Dänemark\oindex{Daenemark@\textbf{Dänemark}, \emph{A.PCLI}|pw} und Schweden\oindex{Schweden@\textbf{Schweden}, \emph{A.PCLI}|pw}
               zurück und habe nun erst Ihren lieben Brief vorgefunden. Seien Sie herzlichst
               gegrüsst und lassen Sie mich hoffen, dass ich bald wieder das Vergnügen habe Sie
               persönlich wiederzusehen und ausführlicher {\pb}mit Ihnen zu
               reden. Sehr gefreut habe ich mich unter manchem anderm, was ich in der letzten Zeit
               von Ihnen las, an Ihrem warmen \label{K_L03758-2v}\edtext{Worten\pwindex{L. Andro (Therese Rie), »Der Klimenole«, Deutsche Verlagsanstalt, Stuttgart]@\emph{[L. Andro (Therese Rie), »Der Klimenole«, Deutsche Verlagsanstalt, Stuttgart]}|pwv}}{\lemma{\textnormal{\emph{Worten}}}\Cendnote{\textnormal{st. z.\pwindex{Zweig, Stefan 28.11.1881 – 23.02.1942@\textsc{Zweig, Stefan} (28.11.1881 – 23.02.1942), \emph{Schriftsteller/Schriftstellerin}|pwk} [= Stefan Zweig\pwindex{Zweig, Stefan 28.11.1881 – 23.02.1942@\textsc{Zweig, Stefan} (28.11.1881 – 23.02.1942), \emph{Schriftsteller/Schriftstellerin}|pwk}]: \emph{[L. Andro (Therese Rie), »Der Klimenole«, Deutsche
                        Verlagsanstalt, Stuttgart]}\pwindex{L. Andro (Therese Rie), »Der Klimenole«, Deutsche Verlagsanstalt, Stuttgart]@\emph{[L. Andro (Therese Rie), »Der Klimenole«, Deutsche Verlagsanstalt, Stuttgart]}|pwk}. In: \emph{Neue
                        Freie Presse}\pwindex{Neue Freie Presse@\emph{Neue Freie Presse}|pwk}, Nr. 21.088, 27. 5. 1923, Morgenblatt,
                     S. 33. }}}\label{K_L03758-2} über das neue Buch\pwindex{Klimenole. Roman@\emph{Der Klimenole. Roman}|pwv} von L. Andro\pwindex{Rie, Therese 1878-01-01 – 1934-07-23@\textsc{Rie, Therese} (1878-01-01 – 1934-07-23), \emph{Schriftsteller/Schriftstellerin}|pw},
               der ich auch \label{K_L03758-3v}\edtext{neulich schrieb}{\lemma{\textnormal{\emph{neulich schrieb}}}\Cendnote{\textnormal{Das Korrespondenzstück ist nicht erhalten,
                     vgl. Therese Rie-Andro an Arthur Schnitzler, 3. 5. 1923.}}}\label{K_L03758-3}, ohne sie
               persönlich zu kennen. Ihr\pend
           \pstart \spacefill\mbox{{[}hs.:{]} Arthur Schnitzler}\pend{}\selectlanguage{ngerman}\endnumbering\briefempfaengerindex{Zweig, Stefan@\textsc{Zweig, Stefan}!zzzSchnitzler, Arthur@\emph{von Arthur Schnitzler}!1923-05-291@{29. 5. 1923}|)be}\mylabel{L03758h}
\begin{anhang}
\end{anhang}\normalsize

\doendnotes{C}
\bigskip
\vfill

\clearpage

\footnotesize

\lohead{\textsc{register}}

% Definiere theindex-Environment komplett neu ohne reledmac
\makeatletter
\renewenvironment{theindex}{%
  \section*{\indexname}%
  \setlength{\parindent}{0pt}%
  \setlength{\parskip}{0pt plus 0.3pt}%
  \let\item\@idxitem
}{%
  \clearpage
}
\makeatother

\IfFileExists{\jobname-pw.ind}{\input{\jobname-pw.ind}}{}

\end{document}

      