%% latex-leseansicht-vorspann.tex
%% Vorspann für die Leseansicht.
%% Lädt die gemeinsame Datei latex-vorspann.tex mit nicht gesetztem Schalter.

\newif\ifkorrekturansicht
\korrekturansichtfalse

\input{../tex-inputs/latex-vorspann}


\section[Arthur Schnitzler an Stefan Zweig, 29. 5. 1923]{L03758 Arthur Schnitzler an Stefan Zweig, 29. 5. 1923}
\nopagebreak\mylabel{L03758v}
\rehead{ }\normalsize\beginnumbering\briefempfaengerindex{Zweig, Stefan@\textsc{Zweig, Stefan}!zzzSchnitzler, Arthur@\emph{von Arthur Schnitzler}!1923-05-291@{29. 5. 1923}|(be}
\toendnotes[C]{\smallbreak\pagebreak[2]}
\correspDesc{Versand  durch Arthur Schnitzler am 29. 5. 1923 in Wien
\newline{}Übermittlung  am 30. 5. 1923 in Wien
\newline{}Erhalt  durch Stefan Zweig im Zeitraum [30. 5. 1923 – 3. 6. 1923?] in Salzburg}\toendnotes[C]{\smallbreak}
\Standort{Jerusalem, National Library of Israel, ARC. Ms. Var. 305 1 58 Stefan Zweig Collection.}
\physDesc{Postkarte, 803 Zeichen
\newline{}Schreibmaschine
\newline{}Handschrift: Bleistift (\noindent{}Unterschrift)
\newline{}Versand: Stempel: »\nobreak{}\oindex{IX., Alsergrund@\textbf{IX., Alsergrund}, \emph{Verwaltungsgebiet}|pwk}9/\textcolor{gray}{×} Wien 72, 30. V. 23, VIII\nobreak{}«.  }\toendnotes[C]{\smallbreak}\pstart{}{\pb}\label{T_L03758-1v}\edtext{\textcolor{gray}{\textbf{A. S.}}}{\lemma{\textnormal{\emph{A. S.}}}\Cendnote{\textnormal{ovaler Absenderkleber}}}\label{T_L03758-1}\pend{}\pstart{}\textcolor{gray}{\textbf{WIEN, XVIII.}}\oindex{XVIII., Währing@\textbf{XVIII., Währing}, \emph{Verwaltungsgebiet}|pw}\pend{}\pstart{}\textcolor{gray}{\textbf{STERNWARTESTR. 71}}\oindex{Wien@\textbf{Wien}!XVIII., Währing@\textbf{XVIII., Währing}!Sternwartestraße 71@\textbf{Sternwartestraße 71}, \emph{Wohngebäude}|pw}\pend{}{\bigskip}\pstart{}Herrn\pend{}\pstart{}Dr. Stefan Zweig\pend{}\pstart{}Salzburg\oindex{Salzburg@\textbf{Salzburg}, \emph{Verwaltungsgebiet}|pw}.\pend{}\pstart{}Kapuzinerberg 5\oindex{Paschinger Schlössl@\textbf{Paschinger Schlössl}, \emph{Wohngebäude}|pw}.\pend{}{\bigskip}\vspace{1em}
\pstart
           \raggedleft{}{\pb}29. 5. 1923.\pend
           
\pstart{}Lieber Herr Doktor.\pend\vspace{0.5em}
\pstart
           Vielen Dank, dass Sie mich auf diese bevorstehende Versteigerung aufmerksam gemacht
               haben. Das Buch\pwindex{\textcolor{red}{\textsuperscript{XXXX indx1}}!Gänsemännchen. Roman@\strich\emph{Das Gänsemännchen. Roman}|pwv} ist mir
               offenbar gestohlen worden. Ich habe gleich an das betreffende Antiquariat\orgindex{Antiquariat Emil Hirsch@Antiquariat Emil Hirsch|pwv} geschrieben und das Buch\pwindex{\textcolor{red}{\textsuperscript{XXXX indx1}}!Gänsemännchen. Roman@\strich\emph{Das Gänsemännchen. Roman}|pwv} zurückgefordert.\pend
           
\pstart
           \label{K_L03758-1v}\edtext{Eben}{\lemma{\textnormal{\emph{Eben}}}\Cendnote{\textnormal{Er kam am 27. 5. 1923 wieder in Wien\oindex{Wien@\textbf{Wien}, \emph{Verwaltungsgebiet}|pwk}
                  an.}}}\label{K_L03758-1} komme ich von einer sehr schönen Reise nach Dänemark\oindex{Dänemark@\textbf{Dänemark}|pw} und Schweden\oindex{Schweden@\textbf{Schweden}|pw}
               zurück und habe nun erst Ihren lieben Brief vorgefunden. Seien Sie herzlichst
               gegrüsst und lassen Sie mich hoffen, dass ich bald wieder das Vergnügen habe Sie
               persönlich wiederzusehen und ausführlicher {\pb}mit Ihnen zu
               reden.\pend
           
\pstart
           Sehr gefreut habe ich mich unter manchem andern, was ich in der letzten Zeit
               von Ihnen las, an Ihrem warmen \label{K_L03758-2v}\edtext{Worten\pwindex{Zweig, Stefan 28.\,11.\,1881 Wien – 23.\,2.\,1942 Petrópolis@\textsc{Zweig, Stefan} (28.\,11.\,1881 Wien – 23.\,2.\,1942 Petrópolis), \emph{Schriftsteller}!L. Andro (Therese Rie), »Der Klimenole«, Deutsche Verlagsanstalt, Stuttgart]@\strich\emph{[L. Andro (Therese Rie), »Der Klimenole«, Deutsche Verlagsanstalt, Stuttgart]}|pwv}}{\lemma{\textnormal{\emph{Worten}}}\Cendnote{\textnormal{st. z.\pwindex{Zweig, Stefan 28.\,11.\,1881 Wien – 23.\,2.\,1942 Petrópolis@\textsc{Zweig, Stefan} (28.\,11.\,1881 Wien – 23.\,2.\,1942 Petrópolis), \emph{Schriftsteller}|pwk} [= Stefan Zweig\pwindex{Zweig, Stefan 28.\,11.\,1881 Wien – 23.\,2.\,1942 Petrópolis@\textsc{Zweig, Stefan} (28.\,11.\,1881 Wien – 23.\,2.\,1942 Petrópolis), \emph{Schriftsteller}|pwk}]: \emph{[L. Andro (Therese Rie), »Der Klimenole«, Deutsche
                        Verlagsanstalt, Stuttgart]}\pwindex{Zweig, Stefan 28.\,11.\,1881 Wien – 23.\,2.\,1942 Petrópolis@\textsc{Zweig, Stefan} (28.\,11.\,1881 Wien – 23.\,2.\,1942 Petrópolis), \emph{Schriftsteller}!L. Andro (Therese Rie), »Der Klimenole«, Deutsche Verlagsanstalt, Stuttgart]@\strich\emph{[L. Andro (Therese Rie), »Der Klimenole«, Deutsche Verlagsanstalt, Stuttgart]}|pwk}. In: \emph{Neue
                        Freie Presse}\pwindex{Neue Freie Presse@\emph{Neue Freie Presse}|pwk}, Nr. 21.088, 27. 5. 1923, Morgenblatt,
                     S. 33. }}}\label{K_L03758-2} über das neue Buch\pwindex{Rie, Therese 1.\,1.\,1878 Wien – 23.\,7.\,1934 ebd.@\textsc{Rie, Therese} (1.\,1.\,1878 Wien – 23.\,7.\,1934 ebd.), \emph{Schriftstellerin}!Klimenole. Roman@\strich\emph{Der Klimenole. Roman}|pwv} von L. Andro\pwindex{Rie, Therese 1.\,1.\,1878 Wien – 23.\,7.\,1934 ebd.@\textsc{Rie, Therese} (1.\,1.\,1878 Wien – 23.\,7.\,1934 ebd.), \emph{Schriftstellerin}|pw},
               der ich auch \label{K_L03758-3v}\edtext{neulich schrieb}{\lemma{\textnormal{\emph{neulich schrieb}}}\Cendnote{\textnormal{Das Korrespondenzstück ist nicht erhalten,
                     vgl. XXXX Auszeichnungsfehler: Dokument L02572 nicht gefunden.}}}\label{K_L03758-3}, ohne sie
               persönlich zu kennen. Ihr\pend
           \pstart \spacefill\mbox{{[}hs.:{]} Arthur Schnitzler}\pend{}\selectlanguage{ngerman}\endnumbering\briefempfaengerindex{Zweig, Stefan@\textsc{Zweig, Stefan}!zzzSchnitzler, Arthur@\emph{von Arthur Schnitzler}!1923-05-291@{29. 5. 1923}|)be}\mylabel{L03758h}  \newcommand{\dateiname}{L03758}\newcommand{\titel}{Arthur Schnitzler an Stefan Zweig, 29. 5. 1923}\newcommand{\editorInnen}{Selma Jahnke und Martin Anton Müller}%% latex-leseansicht-abspann.tex
%% Abspann für die Leseansicht.
%% Der Schalter \ifkorrekturansicht ist bereits durch den Vorspann gesetzt.

%% latex-abspann.tex
%% Gemeinsamer Abspann für Korrekturansicht und Leseansicht.
%% Setzt den Schalter \ifkorrekturansicht voraus (gesetzt in den
%% einbindenden Dateien latex-korrekturansicht-abspann.tex bzw.
%% latex-leseansicht-abspann.tex).
%% ---------------------------------------------------------------

\normalsize

% Das esempio-Environment wird nur in der Leseansicht benötigt
\ifkorrekturansicht\else
\newenvironment{esempio}[3]%
{
    \vspace{1.5ex}
    \rlap{\underline{#1}}
    \par
    \setlength{\parindent}{0cm}
    \nopagebreak
    \leftskip=#2cm
    \rightskip=#3cm
}
{
    \par
}
\fi

\doendnotes{C}
\bigskip
\vfill

\clearpage

\footnotesize

\ifkorrekturansicht
  \lohead{\textsc{register}}
\fi

% theindex-Environment neu definieren ohne reledmac
\makeatletter
\renewenvironment{theindex}{%
  \ifkorrekturansicht
    \section*{\indexname}%
  \else
    \subsubsection*{Index der erwähnten Entitäten}%
  \fi
  \setlength{\parindent}{0pt}%
  \setlength{\parskip}{0pt plus 0.3pt}%
  \let\item\@idxitem
}{%
  \ifkorrekturansicht\clearpage\fi
}
\makeatother

\IfFileExists{\jobname-pw.ind}{\input{\jobname-pw.ind}}{}

% Quellenangabe nur in der Leseansicht
\ifkorrekturansicht\else
% Fallback-Definitionen, falls die .tex-Datei \titel etc. nicht gesetzt hat
\providecommand{\titel}{}
\providecommand{\editorInnen}{}
\providecommand{\dateiname}{\jobname}

\vspace{3cm}

\vfill

\footnotesize
\textsc{Quelle}: \titel. Herausgegeben von {\editorInnen}. In: \emph{Arthur Schnitzler: Briefwechsel mit Autorinnen und Autoren}.
 Digitale Edition, https://schnitzler-briefe.acdh.oeaw.ac.at/{\dateiname}.html (Stand \today)
\fi

\end{document}


