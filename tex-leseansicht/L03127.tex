%% latex-korrekturansicht-vorspann.tex
%% Vorspann für die Korrekturansicht.
%% Lädt die gemeinsame Datei latex-vorspann.tex mit gesetztem Schalter.

\newif\ifkorrekturansicht
\korrekturansichttrue

\input{../tex-inputs/latex-vorspann}


\section[Felix Salten an Arthur Schnitzler, 14. 8. 1893]{L03127 Felix Salten an Arthur Schnitzler, 14. 8. 1893}
\nopagebreak\mylabel{L03127v}
\rehead{ }\normalsize\beginnumbering\briefempfaengerindex{Schnitzler, Arthur@\textsc{Schnitzler, Arthur}!zzzSalten, Felix@\emph{von Felix Salten}!1893-08-142@{14. 8. 1893}|(be}
\toendnotes[C]{\smallbreak\pagebreak[2]}\Standort{CUL, Schnitzler, B 89, A 1.}
\physDesc{Bildpostkarte, 565 Zeichen
\newline{}Handschrift: Bleistift, lateinische Kurrent
\newline{}Versand: 1) Stempel: »\nobreak{}\oindex{Cortina DAmpezzo@\textbf{Cortina d’Ampezzo}, \emph{P.PPLA3}|pwk}Co{[}rti{]}na, 15/8 9\textcolor{gray}{3}\nobreak{}«.   2) Stempel: »\nobreak{}\oindex{I., Innere Stadt@\textbf{I., Innere Stadt}, \emph{A.ADM3}|pwk}Wien 1/1 1, 17/8. 93, 8–9½ V., Bestellt\nobreak{}«. 
\newline{}Ordnung: mit Bleistift von unbekannter Hand nummeriert: »30.« }\toendnotes[C]{\smallbreak}\pstart{}{\pb}Herrn D\textsuperscript{r} Arthur Schnitzler.\pend{}\pstart{}Wien\oindex{Wien@\textbf{Wien}, \emph{A.ADM2}|pw}\pend{}\pstart{}I. Grillparzerstraße 7\oindex{Grillparzerstrasse@\textbf{Grillparzerstraße}, \emph{R.ST}|pw}.\pend{}{\bigskip}
\pstart
           \noindent{}\centering{}{\pb}\textcolor{gray}{\textbf{\textbf{Cortina d’Ampezzo\oindex{Cortina DAmpezzo@\textbf{Cortina d’Ampezzo}, \emph{P.PPLA3}|pw}.}}}\pend
           \vspace{1em}
\pstart
           \raggedleft{}{\pb}14. 8. 93.\pend
           \vspace{0.5em}
\pstart
           Lieber Freund! Die Fahrt hieher\oindex{Cortina DAmpezzo@\textbf{Cortina d’Ampezzo}, \emph{P.PPLA3}|pwv} einfach das Herrlichste, was es
               gibt, die Straße von unerhörter Glätte. \label{K_L03127-1v}\edtext{Wenn Sie kommen}{\lemma{\textnormal{\emph{Wenn Sie kommen}}}\Cendnote{\textnormal{Schnitzler kam am 23. 8. 1893 in Dölsach\oindex{Doelsach@\textbf{Dölsach}, \emph{A.ADM3}|pwk} an, noch am selben Tag ging es für ihn weiter nach Toblach\oindex{Toblach@\textbf{Toblach}, \emph{A.ADM3}|pwk}. Von dort aus unternahm er Radausflüge, etwa am 24. 8. 1893 nach Pieve di Cadore\oindex{Pieve di Cadore@\textbf{Pieve di Cadore}, \emph{A.ADM3}|pwk}. Später fuhr er weiter nach
                     Kärnten\oindex{Kaernten@\textbf{Kärnten}, \emph{A.ADM1}|pwk} und in die Steiermark\oindex{Steiermark@\textbf{Steiermark}, \emph{A.ADM1}|pwk}. Für den 31. 8. 1893 verzeichnete Schnitzler seine Rückkehr nach Wien\oindex{Wien@\textbf{Wien}, \emph{A.ADM2}|pwk}. Bei welchen Touren Salten\pwindex{Salten, Felix 06.09.1869 – 08.10.1945@\textsc{Salten, Felix} (06.09.1869 – 08.10.1945), \emph{Schriftsteller/Schriftstellerin, Journalist/Journalistin, Chefredakteur/Chefredakteurin}|pwk}
                  mitmachte, ist unklar. (Vgl. Felix Salten an Arthur Schnitzler, 18. 8. 1893.)}}}\label{K_L03127-1}, fahren wir nach Piève di Cadore\oindex{Pieve di Cadore@\textbf{Pieve di Cadore}, \emph{A.ADM3}|pw},
               ja? Es soll gleichfalls herrlich sein. Ich habe die 35 Km. in 1 ½ Stunden gemacht,
               ungerechnet den Aufenthalt in Landro\oindex{Hoehlenstein@\textbf{Höhlenstein}, \emph{P.PPLQ}|pw}. Dieses
               Bergabfahren von Landro\oindex{Hoehlenstein@\textbf{Höhlenstein}, \emph{P.PPLQ}|pw} an, na, Sie werden sehen.
               Ich habe, nach Cortina\oindex{Cortina DAmpezzo@\textbf{Cortina d’Ampezzo}, \emph{P.PPLA3}|pw}\textcolor{gray}{dauernd die}{ }\textcolor{gray}{×}\-\textcolor{gray}{×}\-\textcolor{gray}{×}\-\textcolor{gray}{×}eratur verachtet, u. als ich ankam, war ich
                  \textcolor{gray}{r}ein \substVorne{}\textsuperscript{\textcolor{gray}{×}}\substDazwischen{}e\substHinten{}rstgradig, was ich jetzt eher nicht
               mehr ganz bin. Ich schreibe nochmals genau. 
            \pend
           \pstart Herzlich Ihr \spacefill\mbox{Salten}\pend{}\selectlanguage{ngerman}\endnumbering\briefempfaengerindex{Schnitzler, Arthur@\textsc{Schnitzler, Arthur}!zzzSalten, Felix@\emph{von Felix Salten}!1893-08-142@{14. 8. 1893}|)be}\mylabel{L03127h}  \normalsize

\doendnotes{C}
\bigskip
\vfill

\clearpage

\footnotesize

\lohead{\textsc{register}}

% Definiere theindex-Environment komplett neu ohne reledmac
\makeatletter
\renewenvironment{theindex}{%
  \section*{\indexname}%
  \setlength{\parindent}{0pt}%
  \setlength{\parskip}{0pt plus 0.3pt}%
  \let\item\@idxitem
}{%
  \clearpage
}
\makeatother

\IfFileExists{\jobname-pw.ind}{\input{\jobname-pw.ind}}{}

\end{document}

      