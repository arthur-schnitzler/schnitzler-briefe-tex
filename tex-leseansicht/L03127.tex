%% latex-leseansicht-vorspann.tex
%% Vorspann für die Leseansicht.
%% Lädt die gemeinsame Datei latex-vorspann.tex mit nicht gesetztem Schalter.

\newif\ifkorrekturansicht
\korrekturansichtfalse

\input{../tex-inputs/latex-vorspann}


\section[Felix Salten an Arthur Schnitzler, 14. 8. 1893]{L03127 Felix Salten an Arthur Schnitzler, 14. 8. 1893}
\nopagebreak\mylabel{L03127v}
\rehead{ }\normalsize\beginnumbering\briefempfaengerindex{Schnitzler, Arthur@\textsc{Schnitzler, Arthur}!zzzSalten, Felix@\emph{von Felix Salten}!1893-08-142@{14. 8. 1893}|(be}
\toendnotes[C]{\smallbreak\pagebreak[2]}
\correspDesc{Versand  durch Felix Salten am 14. 8. 1893 in Cortina d’Ampezzo
\newline{}Übermittlung  am 15. 8. 1893 in Cortina d’Ampezzo
\newline{}Erhalt  durch Arthur Schnitzler am 17. 8. 1893 in Wien}\toendnotes[C]{\smallbreak}
\Standort{CUL, Schnitzler, B 89, A 1.}
\physDesc{Bildpostkarte, 565 Zeichen
\newline{}Handschrift: Bleistift, lateinische Kurrent
\newline{}Versand: 1) Stempel: »\nobreak{}\oindex{Cortina d’Ampezzo@\textbf{Cortina d’Ampezzo}, \emph{Hauptstadt}|pwk}Co{[}rti{]}na, 15/8 9\textcolor{gray}{3}\nobreak{}«.   2) Stempel: »\nobreak{}\oindex{I., Innere Stadt@\textbf{I., Innere Stadt}, \emph{Verwaltungsgebiet}|pwk}Wien 1/1 1, 17/8. 93, 8–9½ V., Bestellt\nobreak{}«. 
\newline{}Ordnung: mit Bleistift von unbekannter Hand nummeriert: »30.« }\toendnotes[C]{\smallbreak}\pstart{}{\pb}Herrn D\textsuperscript{r} Arthur Schnitzler.\pend{}\pstart{}Wien\oindex{Wien@\textbf{Wien}, \emph{Verwaltungsgebiet}|pw}\pend{}\pstart{}I. Grillparzerstraße 7\oindex{Wien@\textbf{Wien}!I., Innere Stadt@\textbf{I., Innere Stadt}!Grillparzerstraße@\textbf{Grillparzerstraße}, \emph{Straße}|pw}.\pend{}{\bigskip}
\pstart
           \noindent{}\centering{}{\pb}\textcolor{gray}{\textbf{\textbf{Cortina d’Ampezzo\oindex{Cortina d’Ampezzo@\textbf{Cortina d’Ampezzo}, \emph{Hauptstadt}|pw}.}}}\pend
           \vspace{1em}
\pstart
           \raggedleft{}{\pb}14. 8. 93.\pend
           \vspace{0.5em}
\pstart
           Lieber Freund! Die Fahrt hieher\oindex{Cortina d’Ampezzo@\textbf{Cortina d’Ampezzo}, \emph{Hauptstadt}|pwv} einfach das Herrlichste, was es
               gibt, die Straße von unerhörter Glätte. \label{K_L03127-1v}\edtext{Wenn Sie kommen}{\lemma{\textnormal{\emph{Wenn Sie kommen}}}\Cendnote{\textnormal{Schnitzler kam am 23. 8. 1893 in Dölsach\oindex{Dölsach@\textbf{Dölsach}, \emph{Verwaltungsgebiet}|pwk} an, noch am selben Tag ging es für ihn weiter nach Toblach\oindex{Toblach@\textbf{Toblach}, \emph{Verwaltungsgebiet}|pwk}. Von dort aus unternahm er Radausflüge, etwa am 24. 8. 1893 nach Pieve di Cadore\oindex{Pieve di Cadore@\textbf{Pieve di Cadore}, \emph{Verwaltungsgebiet}|pwk}. Später fuhr er weiter nach
                     Kärnten\oindex{Kärnten@\textbf{Kärnten}, \emph{Land}|pwk} und in die Steiermark\oindex{Steiermark@\textbf{Steiermark}, \emph{Land}|pwk}. Für den 31. 8. 1893 verzeichnete Schnitzler seine Rückkehr nach Wien\oindex{Wien@\textbf{Wien}, \emph{Verwaltungsgebiet}|pwk}. Bei welchen Touren Salten\pwindex{Salten, Felix 6.\,9.\,1869 Budapest – 8.\,10.\,1945 Zürich@\textsc{Salten, Felix} (6.\,9.\,1869 Budapest – 8.\,10.\,1945 Zürich), \emph{Schriftsteller, Journalist, Chefredakteur}|pwk}
                  mitmachte, ist unklar. (Vgl. XXXX Auszeichnungsfehler: Dokument L03128 nicht gefunden.)}}}\label{K_L03127-1}, fahren wir nach Piève di Cadore\oindex{Pieve di Cadore@\textbf{Pieve di Cadore}, \emph{Verwaltungsgebiet}|pw},
               ja? Es soll gleichfalls herrlich sein. Ich habe die 35 Km. in 1 ½ Stunden gemacht,
               ungerechnet den Aufenthalt in Landro\oindex{Höhlenstein@\textbf{Höhlenstein}|pw}. Dieses
               Bergabfahren von Landro\oindex{Höhlenstein@\textbf{Höhlenstein}|pw} an, na, Sie werden sehen.
               Ich habe, nach Cortina\oindex{Cortina d’Ampezzo@\textbf{Cortina d’Ampezzo}, \emph{Hauptstadt}|pw}{ }\textcolor{gray}{dauernd die}{ }\textcolor{gray}{×}\-\textcolor{gray}{×}\-\textcolor{gray}{×}\-\textcolor{gray}{×}eratur verachtet, u. als ich ankam, war ich
                  \textcolor{gray}{r}ein \substVorne{}\textsuperscript{\textcolor{gray}{×}}\substDazwischen{}e\substHinten{}rstgradig, was ich jetzt eher nicht
               mehr ganz bin. Ich schreibe nochmals genau.\pend
           \pstart Herzlich Ihr \spacefill\mbox{Salten}\pend{}\selectlanguage{ngerman}\endnumbering\briefempfaengerindex{Schnitzler, Arthur@\textsc{Schnitzler, Arthur}!zzzSalten, Felix@\emph{von Felix Salten}!1893-08-142@{14. 8. 1893}|)be}\mylabel{L03127h}  \newcommand{\dateiname}{L03127}\newcommand{\titel}{Felix Salten an Arthur Schnitzler, 14. 8. 1893}\newcommand{\editorInnen}{Martin Anton Müller und Laura Untner}%% latex-leseansicht-abspann.tex
%% Abspann für die Leseansicht.
%% Der Schalter \ifkorrekturansicht ist bereits durch den Vorspann gesetzt.

%% latex-abspann.tex
%% Gemeinsamer Abspann für Korrekturansicht und Leseansicht.
%% Setzt den Schalter \ifkorrekturansicht voraus (gesetzt in den
%% einbindenden Dateien latex-korrekturansicht-abspann.tex bzw.
%% latex-leseansicht-abspann.tex).
%% ---------------------------------------------------------------

\normalsize

% Das esempio-Environment wird nur in der Leseansicht benötigt
\ifkorrekturansicht\else
\newenvironment{esempio}[3]%
{
    \vspace{1.5ex}
    \rlap{\underline{#1}}
    \par
    \setlength{\parindent}{0cm}
    \nopagebreak
    \leftskip=#2cm
    \rightskip=#3cm
}
{
    \par
}
\fi

\doendnotes{C}
\bigskip
\vfill

\clearpage

\footnotesize

\ifkorrekturansicht
  \lohead{\textsc{register}}
\fi

% theindex-Environment neu definieren ohne reledmac
\makeatletter
\renewenvironment{theindex}{%
  \ifkorrekturansicht
    \section*{\indexname}%
  \else
    \subsubsection*{Index der erwähnten Entitäten}%
  \fi
  \setlength{\parindent}{0pt}%
  \setlength{\parskip}{0pt plus 0.3pt}%
  \let\item\@idxitem
}{%
  \ifkorrekturansicht\clearpage\fi
}
\makeatother

\IfFileExists{\jobname-pw.ind}{\input{\jobname-pw.ind}}{}

% Quellenangabe nur in der Leseansicht
\ifkorrekturansicht\else
% Fallback-Definitionen, falls die .tex-Datei \titel etc. nicht gesetzt hat
\providecommand{\titel}{}
\providecommand{\editorInnen}{}
\providecommand{\dateiname}{\jobname}

\vspace{3cm}

\vfill

\footnotesize
\textsc{Quelle}: \titel. Herausgegeben von {\editorInnen}. In: \emph{Arthur Schnitzler: Briefwechsel mit Autorinnen und Autoren}.
 Digitale Edition, https://schnitzler-briefe.acdh.oeaw.ac.at/{\dateiname}.html (Stand \today)
\fi

\end{document}


