%% latex-leseansicht-vorspann.tex
%% Vorspann für die Leseansicht.
%% Lädt die gemeinsame Datei latex-vorspann.tex mit nicht gesetztem Schalter.

\newif\ifkorrekturansicht
\korrekturansichtfalse

\input{../tex-inputs/latex-vorspann}


         
         \renewcommand{\erwaehntePersonen}{Personen: Paul Goldmann, August Niemann, Richard Salinger, Adolf Silberstein}
         \renewcommand{\erwaehnteInstitutionen}{Institutionen: Philosophisch-historischer Verlag Dr. R. Salinger}
         \renewcommand{\erwaehnteOrte}{Orte: Berlin, Frankgasse, IX., Alsergrund, Paris, Place de la Bourse, Wien, Österreich}
         \renewcommand{\erwaehnteWerke}{Werke: Der Junggesell. Humoreske, Heirathen oder nicht?, Pester Lloyd}
               \section[Paul Goldmann an Arthur Schnitzler, 23. 1. 1894]{ Paul Goldmann an Arthur Schnitzler, 23. 1. 1894}\nopagebreak\mylabel{v}\rehead{ }\begin{ledgroupsized}[t]{13cm}\normalsize\beginnumbering\briefempfaengerindex{Schnitzler, Arthur@\textsc{Schnitzler, Arthur}!zzzGoldmann, Paul@\emph{von Paul Goldmann}!1894-01-232@{23. 1. 1894}|(be} \toendnotes[C]{\smallbreak\pagebreak[2]} \Standort{DLA, A:Schnitzler, HS.NZ85.1.3164.}
\physDesc{Postkarte, 227 Zeichen
\newline{}Handschrift Arthur Schnitzler: 1) schwarze Tinte, deutsche Kurrent\hspace{1em}2) schwarze Tinte, lateinische Kurrent (\noindent{}Adresse)\hspace{1em}
\newline{}Versand: 1) Stempel: »\nobreak{}\oindex{Place de la Bourse@\textbf{Place de la Bourse}|pwk}{[}Paris{]} Pl. de la Bourse, {[}23{]} Janv. 94\nobreak{}«.   2) Stempel: »\nobreak{}\oindex{IX., Alsergrund@\textbf{IX., Alsergrund}|pwk}Wien 9/3 \textcolor{gray}{7}2, 25. 1. 94, 9.V, B{[}est{]}e{[}llt{]}\nobreak{}«. }\toendnotes[C]{\smallbreak}\pstart{}{\pb}\begin{otherlanguage}{french}Autriche\end{otherlanguage}\oindex{Oesterreich@\textbf{Österreich}|pw} .\pend{}\pstart{}Herrn\pend{}\pstart{}Dr. Arthur Schnitzler\pend{}\pstart{}IX. Frankgaſse 1\oindex{XXXX Ortsangabe fehlt|pw}\pend{}\pstart{}Wien\oindex{Wien@\textbf{Wien}|pw}. \pend{}{\bigskip}\pstart
           \centering{}{\pb}\textsc{Paris\oindex{Paris@\textbf{Paris}|pw}}{ }23. 1. 94\pend
           \pstart
           Sofort – ehe es \label{K_L02607-1v}\edtext{verboten}{\lemma{\textnormal{\emph{verboten}}}\Cendnote{\textnormal{Zu Kritik an Niemann\pwindex{Niemann, August 1839-06-27 – 1919-09-17@\textsc{Niemann, August} (1839-06-27 – 1919-09-17), \emph{Schriftsteller, Schauspieler, Redakteur}|pwk}s \emph{Der
                     Junggesell}\pwindex{Niemann, August 1839-06-27 – 1919-09-17@\textsc{Niemann, August} (1839-06-27 – 1919-09-17), \emph{Schriftsteller, Schauspieler, Redakteur}!Junggesell. Humoreske1893@\strich\emph{Der Junggesell. Humoreske} {[}1893{]}|pwk} siehe etwa Adolf Silberstein\pwindex{Silberstein, Adolf 1845-07-01 – 1899-02-12@\textsc{Silberstein, Adolf} (1845-07-01 – 1899-02-12), \emph{Schriftsteller, Journalist}|pwk}: \emph{Heirathen oder nicht?}\pwindex{Silberstein, Adolf 1845-07-01 – 1899-02-12@\textsc{Silberstein, Adolf} (1845-07-01 – 1899-02-12), \emph{Schriftsteller, Journalist}!Heirathen oder nicht?1894-06-23@\strich\emph{Heirathen oder nicht?} {[}1894-06-23{]}|pwk} In: \emph{Pester Lloyd}\pwindex{?? Werk@Nicht ermittelte Verfasserinnen und Verfasser!Pester LloydNone@\emph{Pester Lloyd} {[}None{]}|pwk}, Nr. 151, 23. 6. 1894,
                     S. [3–4].}}}\label{K_L02607-1h} wird – kommen laſſen: \textsc{August Niemann\pwindex{Niemann, August 1839-06-27 – 1919-09-17@\textsc{Niemann, August} (1839-06-27 – 1919-09-17), \emph{Schriftsteller, Schauspieler, Redakteur}|pw}}: Der Junggeſell\pwindex{Niemann, August 1839-06-27 – 1919-09-17@\textsc{Niemann, August} (1839-06-27 – 1919-09-17), \emph{Schriftsteller, Schauspieler, Redakteur}!Junggesell. Humoreske1893@\strich\emph{Der Junggesell. Humoreske} {[}1893{]}|pw}. Berlin\oindex{Berlin@\textbf{Berlin}|pw}, Philoſophiſch-Hiſtoriſcher Verlag\orgindex{Philosophisch-historischer Verlag Dr. R. Salinger@Philosophisch-historischer Verlag Dr. R. Salinger|pw}, \textsc{Dr. R. Salinger\pwindex{Salinger, Richard 1859-03-01 – 1926-07-28@\textsc{Salinger, Richard} (1859-03-01 – 1926-07-28), \emph{Verleger, Redakteur}|pw}}, 1894.\pend
           \pstart
           Grüße,{\\[\baselineskip]}\spacefill\mbox{Paul Goldmann.}\pend
           \leftskip=0em{}
         
         \endnumbering\mylabel{h}\end{ledgroupsized}  \newcommand{\dateiname}{L02607}\newcommand{\titel}{Paul Goldmann an Arthur Schnitzler, 23. 1. 1894}\newcommand{\editorInnen}{Martin Anton Müller und Laura Untner}%% latex-leseansicht-abspann.tex
%% Abspann für die Leseansicht.
%% Der Schalter \ifkorrekturansicht ist bereits durch den Vorspann gesetzt.

%% latex-abspann.tex
%% Gemeinsamer Abspann für Korrekturansicht und Leseansicht.
%% Setzt den Schalter \ifkorrekturansicht voraus (gesetzt in den
%% einbindenden Dateien latex-korrekturansicht-abspann.tex bzw.
%% latex-leseansicht-abspann.tex).
%% ---------------------------------------------------------------

\normalsize

% Das esempio-Environment wird nur in der Leseansicht benötigt
\ifkorrekturansicht\else
\newenvironment{esempio}[3]%
{
    \vspace{1.5ex}
    \rlap{\underline{#1}}
    \par
    \setlength{\parindent}{0cm}
    \nopagebreak
    \leftskip=#2cm
    \rightskip=#3cm
}
{
    \par
}
\fi

\doendnotes{C}
\bigskip
\vfill

\clearpage

\footnotesize

\ifkorrekturansicht
  \lohead{\textsc{register}}
\fi

% theindex-Environment neu definieren ohne reledmac
\makeatletter
\renewenvironment{theindex}{%
  \ifkorrekturansicht
    \section*{\indexname}%
  \else
    \subsubsection*{Index der erwähnten Entitäten}%
  \fi
  \setlength{\parindent}{0pt}%
  \setlength{\parskip}{0pt plus 0.3pt}%
  \let\item\@idxitem
}{%
  \ifkorrekturansicht\clearpage\fi
}
\makeatother

\IfFileExists{\jobname-pw.ind}{\input{\jobname-pw.ind}}{}

% Quellenangabe nur in der Leseansicht
\ifkorrekturansicht\else
% Fallback-Definitionen, falls die .tex-Datei \titel etc. nicht gesetzt hat
\providecommand{\titel}{}
\providecommand{\editorInnen}{}
\providecommand{\dateiname}{\jobname}

\vspace{3cm}

\vfill

\footnotesize
\textsc{Quelle}: \titel. Herausgegeben von {\editorInnen}. In: \emph{Arthur Schnitzler: Briefwechsel mit Autorinnen und Autoren}.
 Digitale Edition, https://schnitzler-briefe.acdh.oeaw.ac.at/{\dateiname}.html (Stand \today)
\fi

\end{document}


      