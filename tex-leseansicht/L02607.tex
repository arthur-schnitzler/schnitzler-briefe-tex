%% latex-leseansicht-vorspann.tex
%% Vorspann für die Leseansicht.
%% Lädt die gemeinsame Datei latex-vorspann.tex mit nicht gesetztem Schalter.

\newif\ifkorrekturansicht
\korrekturansichtfalse

\input{../tex-inputs/latex-vorspann}


\section[Paul Goldmann an Arthur Schnitzler, 23. 1. 1894]{L02607 Paul Goldmann an Arthur Schnitzler, 23. 1. 1894}
\nopagebreak\mylabel{L02607v}
\rehead{ }\normalsize\beginnumbering\briefempfaengerindex{Schnitzler, Arthur@\textsc{Schnitzler, Arthur}!zzzGoldmann, Paul@\emph{von Paul Goldmann}!1894-01-232@{23. 1. 1894}|(be}
\toendnotes[C]{\smallbreak\pagebreak[2]}
\correspDesc{Versand  durch Paul Goldmann am 23. 1. 1894 in Paris
\newline{}Erhalt  durch Arthur Schnitzler am 23. 1. 1894 in Wien}\toendnotes[C]{\smallbreak}
\Standort{DLA, A:Schnitzler, HS.NZ85.1.3164.}
\physDesc{Postkarte, 227 Zeichen
\newline{}HandschriftX2  : schwarze Tinte, deutsche Kurrent
\newline{}Versand: 1) Stempel: »\nobreak{}\oindex{place de la Bourse@\textbf{place de la Bourse}, \emph{Platz}|pwk}{[}Paris{]} Pl. de la Bourse, {[}23{]} Janv. 94\nobreak{}«.   2) Stempel: »\nobreak{}\oindex{IX., Alsergrund@\textbf{IX., Alsergrund}, \emph{Verwaltungsgebiet}|pwk}Wien 9/3 \textcolor{gray}{7}2, 25. 1. 94, 9.V, B{[}est{]}e{[}llt{]}\nobreak{}«. }\toendnotes[C]{\smallbreak}\pstart{}\textsc{{\pb}\begin{otherlanguage}{french}Autriche\end{otherlanguage}\oindex{Österreich@\textbf{Österreich}|pw}.}\pend{}\pstart{}\textsc{Herrn}\pend{}\pstart{}\textsc{Dr. Arthur Schnitzler}\pend{}\pstart{}\textsc{IX. Frankgaſse 1\oindex{Wien@\textbf{Wien}!IX., Alsergrund@\textbf{IX., Alsergrund}!Frankgasse 1@\textbf{Frankgasse 1}, \emph{Wohngebäude}|pw}}\pend{}\pstart{}\textsc{Wien\oindex{Wien@\textbf{Wien}, \emph{Verwaltungsgebiet}|pw}.}\pend{}{\bigskip}\vspace{1em}
\pstart
           \centering{}{\pb}\textsc{Paris\oindex{Paris@\textbf{Paris}, \emph{Hauptstadt}|pw}}{ }23. 1. 94\pend
           \vspace{0.5em}
\pstart
           Sofort – ehe es \label{K_L02607-1v}\edtext{verboten}{\lemma{\textnormal{\emph{verboten}}}\Cendnote{\textnormal{Zu Kritik an Niemanns\pwindex{Niemann, August 27.\,6.\,1839 – 17.\,9.\,1919@\textsc{Niemann, August} (27.\,6.\,1839 – 17.\,9.\,1919), \emph{Schriftsteller, Schauspieler, Redakteur}|pwk}{ }\emph{Der
                     Junggesell}\pwindex{Niemann, August 27.\,6.\,1839 – 17.\,9.\,1919@\textsc{Niemann, August} (27.\,6.\,1839 – 17.\,9.\,1919), \emph{Schriftsteller, Schauspieler, Redakteur}!Junggesell. Humoreske@\strich\emph{Der Junggesell. Humoreske}|pwk} siehe etwa Adolf Silberstein\pwindex{Silberstein, Adolf 1.\,7.\,1845 Budapest – 12.\,2.\,1899 ebd.@\textsc{Silberstein, Adolf} (1.\,7.\,1845 Budapest – 12.\,2.\,1899 ebd.), \emph{Schriftsteller, Journalist}|pwk}: \emph{Heirathen oder nicht?}\pwindex{Silberstein, Adolf 1.\,7.\,1845 Budapest – 12.\,2.\,1899 ebd.@\textsc{Silberstein, Adolf} (1.\,7.\,1845 Budapest – 12.\,2.\,1899 ebd.), \emph{Schriftsteller, Journalist}!Heirathen oder nicht?@\strich\emph{Heirathen oder nicht?}|pwk} In: \emph{Pester Lloyd}\pwindex{Pester Lloyd@\emph{Pester Lloyd}|pwk}, Nr. 151, 23. 6. 1894,
                     S. [3–4].}}}\label{K_L02607-1} wird – kommen laſſen: \textsc{August Niemann\pwindex{Niemann, August 27.\,6.\,1839 – 17.\,9.\,1919@\textsc{Niemann, August} (27.\,6.\,1839 – 17.\,9.\,1919), \emph{Schriftsteller, Schauspieler, Redakteur}|pw}}: Der Junggeſell\pwindex{Niemann, August 27.\,6.\,1839 – 17.\,9.\,1919@\textsc{Niemann, August} (27.\,6.\,1839 – 17.\,9.\,1919), \emph{Schriftsteller, Schauspieler, Redakteur}!Junggesell. Humoreske@\strich\emph{Der Junggesell. Humoreske}|pw}. Berlin\oindex{Berlin@\textbf{Berlin}, \emph{Hauptstadt}|pw}, Philoſophiſch-Hiſtoriſcher Verlag\orgindex{Philosophisch-historischer Verlag Dr. R. Salinger@Philosophisch-historischer Verlag Dr. R. Salinger|pw}, \textsc{Dr. R. Salinger\pwindex{Salinger, Richard 1.\,3.\,1859 Berlin – 28.\,7.\,1926@\textsc{Salinger, Richard} (1.\,3.\,1859 Berlin – 28.\,7.\,1926), \emph{Verleger, Redakteur}|pw}}, 1894.\pend
           
\pstart
           Grüße,{\\[\baselineskip]}\spacefill\mbox{Paul Goldmann.}\pend
           \leftskip=0em{}\selectlanguage{ngerman}\endnumbering\briefempfaengerindex{Schnitzler, Arthur@\textsc{Schnitzler, Arthur}!zzzGoldmann, Paul@\emph{von Paul Goldmann}!1894-01-232@{23. 1. 1894}|)be}\mylabel{L02607h}  \newcommand{\dateiname}{L02607}\newcommand{\titel}{Paul Goldmann an Arthur Schnitzler, 23. 1. 1894}\newcommand{\editorInnen}{Martin Anton Müller und Laura Untner}%% latex-leseansicht-abspann.tex
%% Abspann für die Leseansicht.
%% Der Schalter \ifkorrekturansicht ist bereits durch den Vorspann gesetzt.

%% latex-abspann.tex
%% Gemeinsamer Abspann für Korrekturansicht und Leseansicht.
%% Setzt den Schalter \ifkorrekturansicht voraus (gesetzt in den
%% einbindenden Dateien latex-korrekturansicht-abspann.tex bzw.
%% latex-leseansicht-abspann.tex).
%% ---------------------------------------------------------------

\normalsize

% Das esempio-Environment wird nur in der Leseansicht benötigt
\ifkorrekturansicht\else
\newenvironment{esempio}[3]%
{
    \vspace{1.5ex}
    \rlap{\underline{#1}}
    \par
    \setlength{\parindent}{0cm}
    \nopagebreak
    \leftskip=#2cm
    \rightskip=#3cm
}
{
    \par
}
\fi

\doendnotes{C}
\bigskip
\vfill

\clearpage

\footnotesize

\ifkorrekturansicht
  \lohead{\textsc{register}}
\fi

% theindex-Environment neu definieren ohne reledmac
\makeatletter
\renewenvironment{theindex}{%
  \ifkorrekturansicht
    \section*{\indexname}%
  \else
    \subsubsection*{Index der erwähnten Entitäten}%
  \fi
  \setlength{\parindent}{0pt}%
  \setlength{\parskip}{0pt plus 0.3pt}%
  \let\item\@idxitem
}{%
  \ifkorrekturansicht\clearpage\fi
}
\makeatother

\IfFileExists{\jobname-pw.ind}{\input{\jobname-pw.ind}}{}

% Quellenangabe nur in der Leseansicht
\ifkorrekturansicht\else
% Fallback-Definitionen, falls die .tex-Datei \titel etc. nicht gesetzt hat
\providecommand{\titel}{}
\providecommand{\editorInnen}{}
\providecommand{\dateiname}{\jobname}

\vspace{3cm}

\vfill

\footnotesize
\textsc{Quelle}: \titel. Herausgegeben von {\editorInnen}. In: \emph{Arthur Schnitzler: Briefwechsel mit Autorinnen und Autoren}.
 Digitale Edition, https://schnitzler-briefe.acdh.oeaw.ac.at/{\dateiname}.html (Stand \today)
\fi

\end{document}


