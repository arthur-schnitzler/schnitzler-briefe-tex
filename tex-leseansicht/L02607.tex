%% latex-korrekturansicht-vorspann.tex
%% Vorspann für die Korrekturansicht.
%% Lädt die gemeinsame Datei latex-vorspann.tex mit gesetztem Schalter.

\newif\ifkorrekturansicht
\korrekturansichttrue

\input{../tex-inputs/latex-vorspann}


\section[Paul Goldmann an Arthur Schnitzler, 23. 1. 1894]{L02607 Paul Goldmann an Arthur Schnitzler, 23. 1. 1894}
\nopagebreak\mylabel{L02607v}
\rehead{ }\normalsize\beginnumbering\briefempfaengerindex{Schnitzler, Arthur@\textsc{Schnitzler, Arthur}!zzzGoldmann, Paul@\emph{von Paul Goldmann}!1894-01-232@{23. 1. 1894}|(be}
\toendnotes[C]{\smallbreak\pagebreak[2]}\Standort{DLA, A:Schnitzler, HS.NZ85.1.3164.}
\physDesc{Postkarte, 227 Zeichen
\newline{}Handschrift Arthur Schnitzler: 1) schwarze Tinte, deutsche Kurrent\hspace{1em}2) schwarze Tinte, lateinische Kurrent (\noindent{}Adresse)\hspace{1em}
\newline{}Versand: 1) Stempel: »\nobreak{}\oindex{place de la Bourse@\textbf{place de la Bourse}, \emph{Platz (K.PLT)}|pwk}{[}Paris{]} Pl. de la Bourse, {[}23{]} Janv. 94\nobreak{}«.   2) Stempel: »\nobreak{}\oindex{IX., Alsergrund@\textbf{IX., Alsergrund}, \emph{A.ADM3}|pwk}Wien 9/3 \textcolor{gray}{7}2, 25. 1. 94, 9.V, B{[}est{]}e{[}llt{]}\nobreak{}«. }\toendnotes[C]{\smallbreak}\pstart{}{\pb}\begin{otherlanguage}{french}Autriche\end{otherlanguage}\oindex{Oesterreich@\textbf{Österreich}, \emph{A.PCLI}|pw} .\pend{}\pstart{}Herrn\pend{}\pstart{}Dr. Arthur Schnitzler\pend{}\pstart{}IX. Frankgaſse 1\oindex{Frankgasse 1@\textbf{Frankgasse 1}, \emph{Wohngebäude (K.WHS)}|pw}\pend{}\pstart{}Wien\oindex{Wien@\textbf{Wien}, \emph{A.ADM2}|pw}. \pend{}{\bigskip}\vspace{1em}
\pstart
           \centering{}{\pb}\textsc{Paris\oindex{Paris@\textbf{Paris}, \emph{P.PPLC}|pw}}{ }23. 1. 94\pend
           \vspace{0.5em}
\pstart
           Sofort – ehe es \label{K_L02607-1v}\edtext{verboten}{\lemma{\textnormal{\emph{verboten}}}\Cendnote{\textnormal{Zu Kritik an Niemanns\pwindex{Niemann, August 1839-06-27 – 1919-09-17@\textsc{Niemann, August} (1839-06-27 – 1919-09-17), \emph{Schriftsteller/Schriftstellerin, Schauspieler/Schauspielerin, Redakteur/Redakteurin}|pwk}{ }\emph{Der
                     Junggesell}\pwindex{Junggesell. Humoreske@\emph{Der Junggesell. Humoreske}|pwk} siehe etwa Adolf Silberstein\pwindex{Silberstein, Adolf 1845-07-01 – 1899-02-12@\textsc{Silberstein, Adolf} (1845-07-01 – 1899-02-12), \emph{Schriftsteller/Schriftstellerin, Journalist/Journalistin}|pwk}: \emph{Heirathen oder nicht?}\pwindex{Heirathen oder nicht?@\emph{Heirathen oder nicht?}|pwk} In: \emph{Pester Lloyd}\pwindex{Pester Lloyd@\emph{Pester Lloyd}|pwk}, Nr. 151, 23. 6. 1894,
                     S. [3–4].}}}\label{K_L02607-1} wird – kommen laſſen: \textsc{August Niemann\pwindex{Niemann, August 1839-06-27 – 1919-09-17@\textsc{Niemann, August} (1839-06-27 – 1919-09-17), \emph{Schriftsteller/Schriftstellerin, Schauspieler/Schauspielerin, Redakteur/Redakteurin}|pw}}: Der Junggeſell\pwindex{Junggesell. Humoreske@\emph{Der Junggesell. Humoreske}|pw}. Berlin\oindex{Berlin@\textbf{Berlin}, \emph{P.PPLC}|pw}, Philoſophiſch-Hiſtoriſcher Verlag\orgindex{Philosophisch-historischer Verlag Dr. R. Salinger@Philosophisch-historischer Verlag Dr. R. Salinger|pw}, \textsc{Dr. R. Salinger\pwindex{Salinger, Richard 1859-03-01 – 1926-07-28@\textsc{Salinger, Richard} (1859-03-01 – 1926-07-28), \emph{Verleger/Verlegerin, Redakteur/Redakteurin}|pw}}, 1894.\pend
           
\pstart
           Grüße,{\\[\baselineskip]}\spacefill\mbox{Paul Goldmann.}\pend
           \leftskip=0em{}\selectlanguage{ngerman}\endnumbering\briefempfaengerindex{Schnitzler, Arthur@\textsc{Schnitzler, Arthur}!zzzGoldmann, Paul@\emph{von Paul Goldmann}!1894-01-232@{23. 1. 1894}|)be}\mylabel{L02607h}  \normalsize

\doendnotes{C}
\bigskip
\vfill

\clearpage

\footnotesize

\lohead{\textsc{register}}

% Definiere theindex-Environment komplett neu ohne reledmac
\makeatletter
\renewenvironment{theindex}{%
  \section*{\indexname}%
  \setlength{\parindent}{0pt}%
  \setlength{\parskip}{0pt plus 0.3pt}%
  \let\item\@idxitem
}{%
  \clearpage
}
\makeatother

\IfFileExists{\jobname-pw.ind}{\input{\jobname-pw.ind}}{}

\end{document}

      