%% latex-korrekturansicht-vorspann.tex
%% Vorspann für die Korrekturansicht.
%% Lädt die gemeinsame Datei latex-vorspann.tex mit gesetztem Schalter.

\newif\ifkorrekturansicht
\korrekturansichttrue

\input{../tex-inputs/latex-vorspann}


\section[ Paul Goldmann an Arthur Schnitzler, 25. 9. {[}1897{]}]{L02826 Paul Goldmann an Arthur Schnitzler, 25. 9. {[}1897{]}}
\nopagebreak\mylabel{L02826v}
\rehead{ }\normalsize\beginnumbering\briefempfaengerindex{Schnitzler, Arthur@\textsc{Schnitzler, Arthur}!zzzGoldmann, Paul@\emph{von Paul Goldmann}!1897-09-253@{25. 9. {[}1897{]}}|(be}
\toendnotes[C]{\smallbreak\pagebreak[2]}\Standort{DLA, A:Schnitzler, HS.NZ85.1.3167.}
\physDesc{Brief, 1 Blatt, 4 Seiten, 1657 Zeichen
\newline{}Handschrift: blaue Tinte, deutsche Kurrent
\newline{}Schnitzler: mit Bleistift das Jahr »97« vermerkt }\toendnotes[C]{\smallbreak}
\pstart
           {\pb}\textcolor{gray}{\textbf{\textbf{Frankfurter Zeitung\orgindex{Frankfurter Zeitung@Frankfurter Zeitung|pw}}}}\pend
           
\pstart
           \textcolor{gray}{\textbf{(\begin{otherlanguage}{french}Gazette de Francfort\end{otherlanguage}\orgindex{Frankfurter Zeitung@Frankfurter Zeitung|pw}).}}\pend
           
\pstart
           \textcolor{gray}{\textbf{\textbf{\begin{otherlanguage}{french}Fondateur M.\end{otherlanguage}{ }L. Sonnemann\pwindex{Sonnemann, Leopold 1831-10-29 – 1909-10-30@\textsc{Sonnemann, Leopold} (1831-10-29 – 1909-10-30), \emph{Journalist/Journalistin, Herausgeber/Herausgeberin}|pw}.}}}\pend
           
\pstart
           \begin{otherlanguage}{french}\textcolor{gray}{\textbf{Journal politique, financier,}}\end{otherlanguage}\pend
           
\pstart
           \begin{otherlanguage}{french}\textcolor{gray}{\textbf{commercial et littéraire.}}\end{otherlanguage}\pend
           
\pstart
           \begin{otherlanguage}{french}\textcolor{gray}{\textbf{\textbf{Paraissant trois fois par jour.}}}\end{otherlanguage}\pend
           
\pstart
           \begin{otherlanguage}{french}\textcolor{gray}{\textbf{\textbf{Bureau à Paris\oindex{Paris@\textbf{Paris}, \emph{P.PPLC}|pw}}}}\end{otherlanguage}\hfill \textsc{Paris\oindex{Paris@\textbf{Paris}, \emph{P.PPLC}|pw}}, \substVorne{}\textsuperscript{\textcolor{gray}{3}}\substDazwischen{}2\substHinten{}5. September.\pend
           
\pstart
           \begin{otherlanguage}{french}\textcolor{gray}{\textbf{\textbf{10 \so{Rue de la Bourse}\oindex{rue de la Bourse@\textbf{rue de la Bourse}, \emph{Straße (K.STR)}|pw}.}}}\end{otherlanguage}\pend
           
\pstart\center{}Mein lieber Freund,\pend\vspace{0.5em}
\pstart
           Es iſt ſehr, ſehr \label{K_L02826-1v}\edtext{traurig}{\lemma{\textnormal{\emph{traurig}}}\Cendnote{\textnormal{Bezug auf die Totgeburt des Sohns\pwindex{?? [Totgeborener Sohn von Arthur Schnitzler und Marie Reinhard] 1897-09-24 – 1897-09-24@\textsc{?? [Totgeborener Sohn von Arthur Schnitzler und Marie Reinhard]} (1897-09-24 – 1897-09-24)|pwkv} von Schnitzler und Marie Reinhard\pwindex{Reinhard, Marie 1871-03-13 – 1899-03-18@\textsc{Reinhard, Marie} (1871-03-13 – 1899-03-18), \emph{Gesangspädagoge/Gesangspädagogin}|pwk} am 24. 9. 1897. Schnitzler gab sich selbst Schuld am Tod des Kind\pwindex{?? [Totgeborener Sohn von Arthur Schnitzler und Marie Reinhard] 1897-09-24 – 1897-09-24@\textsc{?? [Totgeborener Sohn von Arthur Schnitzler und Marie Reinhard]} (1897-09-24 – 1897-09-24)|pwk}es (vgl. A. S.: \emph{Tagebuch}, 30. 9. 1897).}}}\label{K_L02826-1}, und mich hat es tief ergriffen.
               Eines muß Dich tröſten: Du haſt keine Schuld. Alles, was Du thun konnteſt, haſt Du
               gethan. Das Schickſal hat es ſo gewollt, und \strikeout{d\textcolor{gray}{×}} da ſtand es nicht mehr in Deiner Macht, zu hindern. Warum das gerade Dich
               treffen mußte? Man muß ſich eben abgewöhnen, nach Gründen zu fragen; es gibt
               keine.\pend
           
\pstart
           Das arme Kind\pwindex{?? [Totgeborener Sohn von Arthur Schnitzler und Marie Reinhard] 1897-09-24 – 1897-09-24@\textsc{?? [Totgeborener Sohn von Arthur Schnitzler und Marie Reinhard]} (1897-09-24 – 1897-09-24)|pwv} wollen wir
               nicht beklagen. Es iſt ihm eben nur das Leben erſpart geblieben. Es iſt nach kurzer
                  {\pb}Reiſe an das Ziel gelangt, dem wir alle zugehen
               auf dieſem langen, ſchweren Wege. All’ die Thränen braucht es nicht zu weinen, und
               das Bischen Süßigkeit wird es nicht vermiſſen, weil es ſie nie gekannt hat{\dotssix}\pend
           
\pstart
           Was für bittere Stunden Du durchgemacht haben mußt, armer Freund! \strikeout{\textcolor{gray}{×}\-\textcolor{gray}{×}\-\textcolor{gray}{×}\-\textcolor{gray}{×}\-\textcolor{gray}{×}\-\textcolor{gray}{×}\-\textcolor{gray}{×}\-\textcolor{gray}{×}\-\textcolor{gray}{×}\-\textcolor{gray}{×}\-\textcolor{gray}{×}\-\textcolor{gray}{×}\-\textcolor{gray}{×}\-\textcolor{gray}{×}\-\textcolor{gray}{×}\-\textcolor{gray}{×}\-\textcolor{gray}{×}\-\textcolor{gray}{×}\-\textcolor{gray}{×}\-\textcolor{gray}{×}\-\textcolor{gray}{×}\-\textcolor{gray}{×}}{ }\strikeout{\textcolor{gray}{×}\-\textcolor{gray}{×}\-\textcolor{gray}{×}\-\textcolor{gray}{×}\-\textcolor{gray}{×}\-\textcolor{gray}{×}\-\textcolor{gray}{×}\-\textcolor{gray}{×}\-\textcolor{gray}{×}\-\textcolor{gray}{×}\-\textcolor{gray}{×}\-\textcolor{gray}{×}\-\textcolor{gray}{×}\-\textcolor{gray}{×}\-\textcolor{gray}{×}} Könnte ich nur wenigſtens einen Tag bei Dir ſein! Ich würde Dir immerfort
               ſagen: »\label{K_L02826-2v}\edtext{Du biſt jung, und nichts iſt
                  verloren.\pwindex{Erzaehlungen des Kuesters von Dandery@\emph{Erzählungen des Küsters von Dandery}|pwv}}{\lemma{\textnormal{\emph{Du … verloren.}}}\Cendnote{\textnormal{Möglicherweise ein nahezu wörtliches
                  Zitat (S. 100) aus August
                     Blanches\pwindex{Blanche, August Theodor 1811-09-17 – 1868-11-30@\textsc{Blanche, August Theodor} (1811-09-17 – 1868-11-30), \emph{Schriftsteller/Schriftstellerin}|pwk}{ }\emph{Erzählungen des Küsters von
                     Dandery}\pwindex{Erzaehlungen des Kuesters von Dandery@\emph{Erzählungen des Küsters von Dandery}|pwk} (deutsche Übersetzung 1876; das dänische\oindex{Daenemark@\textbf{Dänemark}, \emph{A.PCLI}|pwkv} Original von
                     1856 trägt den Titel \emph{Berättelser
                     af Klockaren i Danderyd}\pwindex{Erzaehlungen des Kuesters von Dandery@\emph{Erzählungen des Küsters von Dandery}|pwk}).}}}\label{K_L02826-2}«\pend
           
\pstart
           Am Meiſten aber dauert mich die arme Frau\pwindex{Reinhard, Marie 1871-03-13 – 1899-03-18@\textsc{Reinhard, Marie} (1871-03-13 – 1899-03-18), \emph{Gesangspädagoge/Gesangspädagogin}|pwv}. Du biſt {\pb}einfach um
               eine ſchöne Hoffnung ärmer (und auch das nur für den Augenblick). Sie\pwindex{Reinhard, Marie 1871-03-13 – 1899-03-18@\textsc{Reinhard, Marie} (1871-03-13 – 1899-03-18), \emph{Gesangspädagoge/Gesangspädagogin}|pwv} muß es aber als einen wahren \label{K_L02826-3v}\edtext{Zuſammenbruch}{\lemma{\textnormal{\emph{Zuſammenbruch}}}\Cendnote{\textnormal{Marie Reinhard\pwindex{Reinhard, Marie 1871-03-13 – 1899-03-18@\textsc{Reinhard, Marie} (1871-03-13 – 1899-03-18), \emph{Gesangspädagoge/Gesangspädagogin}|pwk} war zumindest Schnitzlers{ }\emph{Tagebuch}\pwindex{Tagebuch@\emph{Tagebuch}|pwk} zufolge »gefasst und brav« (A. S.: \emph{Tagebuch}, 25. 9. 1897).}}}\label{K_L02826-3} empfinden. Sei nur
               recht gut und lieb zu ihr. In der Erfüllung dieſer Pflicht wirſt Du auch für Dich den
               beſten Troſt finden. Und ſag’ ihr, daß ich ihr von ganzem Herzen die Hand drücke.\pend
           
\pstart
           Bitte, bitte: ſchreib’ mir bald, und wenn es auch nur ein paar Zeilen ſind.\pend
           
\pstart
           Du ſollteſt jetzt ſo bald als möglich eine \label{K_L02826-4v}\edtext{Reiſe machen}{\lemma{\textnormal{\emph{Reiſe machen}}}\Cendnote{\textnormal{Schnitzler verreiste erst im November 1897 wieder – nach Prag\oindex{Prag@\textbf{Prag}, \emph{A.ADM1}|pwk}, wo am 27. 11. 1897 die Premiere von \emph{Freiwild}\pwindex{Freiwild. Schauspiel in 3 Akten@\emph{Freiwild. Schauspiel in 3 Akten}|pwk} im Neuen Deutschen
                     Theater\oindex{Neues Deutsches Theater@\textbf{Neues Deutsches Theater}, \emph{Theater (K.THE)}|pwk} stattfand.}}}\label{K_L02826-4}. Komm zu mir nach \textsc{Paris}\oindex{Paris@\textbf{Paris}, \emph{P.PPLC}|pw}! {\dots}\pend
           
\pstart
           Armer Freund! Es thut mir innig leid, daß Du, gerade Du dieſen Schmerz {\pb}haben mußteſt! Es iſt auch für mich ein recht
               trauriger Tag.\pend
           
\pstart
           Ich umarme Dich von Herzen und in Treue {\\[\baselineskip]}Dein {\\[\baselineskip]}\spacefill\mbox{Paul Goldmann}\pend
           \leftskip=0em{}
\pstart
           \noindent{}Die \label{K_L02826-5v}\edtext{Briefe}{\lemma{\textnormal{\emph{Briefe}}}\Cendnote{\textnormal{Naheliegend wäre ein Bezug zu der von Jean Thorel\pwindex{Thorel, Jean 1859-09-11 – 1916-08-20@\textsc{Thorel, Jean} (1859-09-11 – 1916-08-20), \emph{Übersetzer/Übersetzerin, Dramatiker/Dramatikerin}|pwk} erstellten Übersetzung\pwindex{Amourette. Piece en trois actes. Adaptee de Arthur Schnitzler@\emph{Amourette. Pièce en trois actes. Adaptée de Arthur Schnitzler}|pwkv} von \emph{Liebelei}\pwindex{Liebelei. Schauspiel in drei Akten@\emph{Liebelei. Schauspiel in drei Akten}|pwk}, die noch immer nicht von einem Theater akzeptiert worden
                     war. Siehe Paul Goldmann an Arthur Schnitzler, 27. 10. [1897].}}}\label{K_L02826-5} ſind
                  alle beſorgt. Auf Deinen Brief antworte ich Dir nächſtens.\pend
           \selectlanguage{ngerman}\endnumbering\briefempfaengerindex{Schnitzler, Arthur@\textsc{Schnitzler, Arthur}!zzzGoldmann, Paul@\emph{von Paul Goldmann}!1897-09-253@{25. 9. {[}1897{]}}|)be}\mylabel{L02826h}  \normalsize

\doendnotes{C}
\bigskip
\vfill

\clearpage

\footnotesize

\lohead{\textsc{register}}

% Definiere theindex-Environment komplett neu ohne reledmac
\makeatletter
\renewenvironment{theindex}{%
  \section*{\indexname}%
  \setlength{\parindent}{0pt}%
  \setlength{\parskip}{0pt plus 0.3pt}%
  \let\item\@idxitem
}{%
  \clearpage
}
\makeatother

\IfFileExists{\jobname-pw.ind}{\input{\jobname-pw.ind}}{}

\end{document}

      