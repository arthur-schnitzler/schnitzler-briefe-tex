%% latex-leseansicht-vorspann.tex
%% Vorspann für die Leseansicht.
%% Lädt die gemeinsame Datei latex-vorspann.tex mit nicht gesetztem Schalter.

\newif\ifkorrekturansicht
\korrekturansichtfalse

\input{../tex-inputs/latex-vorspann}


\section[ Paul Goldmann an Arthur Schnitzler, 25. 9. {[}1897{]}]{L02826 Paul Goldmann an Arthur Schnitzler,  25. 9. [1897]}
\nopagebreak\mylabel{L02826v}
\rehead{ }\normalsize\beginnumbering\briefempfaengerindex{Schnitzler, Arthur@\textsc{Schnitzler, Arthur}!zzzGoldmann, Paul@\emph{von Paul Goldmann}!1897-09-253@{25. 9. [1897]}|(be}
\toendnotes[C]{\smallbreak\pagebreak[2]}
\correspDesc{Versand  durch Paul Goldmann am 25. 9. [1897] in Paris
\newline{}Erhalt  durch Arthur Schnitzler im Zeitraum [26. 9. 1897
                  – 30. 9. 1897?] in Wien}\toendnotes[C]{\smallbreak}
\Standort{DLA, A:Schnitzler, HS.NZ85.1.3167.}
\physDesc{Brief, 1 Blatt, 4 Seiten, 1657 Zeichen
\newline{}Handschrift: blaue Tinte, deutsche Kurrent
\newline{}Schnitzler: mit Bleistift das Jahr »97« vermerkt }\toendnotes[C]{\smallbreak}
\pstart
           {\pb}\textcolor{gray}{\textbf{\textbf{Frankfurter Zeitung\orgindex{Frankfurter Zeitung@Frankfurter Zeitung|pw}}}}\pend
           
\pstart
           \textcolor{gray}{\textbf{(\begin{otherlanguage}{french}Gazette de Francfort\end{otherlanguage}\orgindex{Frankfurter Zeitung@Frankfurter Zeitung|pw}).}}\pend
           
\pstart
           \textcolor{gray}{\textbf{\textbf{\begin{otherlanguage}{french}Fondateur M.\end{otherlanguage}{ }L. Sonnemann\pwindex{Sonnemann, Leopold 29.\,10.\,1831 Höchberg – 30.\,10.\,1909 Frankfurt am Main@\textsc{Sonnemann, Leopold} (29.\,10.\,1831 Höchberg – 30.\,10.\,1909 Frankfurt am Main), \emph{Journalist, Herausgeber}|pw}.}}}\pend
           
\pstart
           \begin{otherlanguage}{french}\textcolor{gray}{\textbf{Journal politique, financier,}}\end{otherlanguage}\pend
           
\pstart
           \begin{otherlanguage}{french}\textcolor{gray}{\textbf{commercial et littéraire.}}\end{otherlanguage}\pend
           
\pstart
           \begin{otherlanguage}{french}\textcolor{gray}{\textbf{\textbf{Paraissant trois fois par jour.}}}\end{otherlanguage}\pend
           
\pstart
           \begin{otherlanguage}{french}\textcolor{gray}{\textbf{\textbf{Bureau à Paris\oindex{Paris@\textbf{Paris}, \emph{Hauptstadt}|pw}}}}\end{otherlanguage}\hfill \textsc{Paris\oindex{Paris@\textbf{Paris}, \emph{Hauptstadt}|pw}}, \substVorne{}\textsuperscript{\textcolor{gray}{3}}\substDazwischen{}2\substHinten{}5. September.\pend
           
\pstart
           \begin{otherlanguage}{french}\textcolor{gray}{\textbf{\textbf{10 \so{Rue de la Bourse}\oindex{rue de la Bourse@\textbf{rue de la Bourse}, \emph{Straße}|pw}.}}}\end{otherlanguage}\pend
           
\pstart\center{}Mein lieber Freund,\pend\vspace{0.5em}
\pstart
           Es iſt{ }ſehr,{ }ſehr \label{K_L02826-1v}\edtext{traurig}{\lemma{\textnormal{\emph{traurig}}}\Cendnote{\textnormal{Bezug auf die Totgeburt des Sohns\pwindex{?? [Totgeborener Sohn von Arthur Schnitzler und Marie Reinhard] 24.\,9.\,1897 Endresstraße 68 – 24.\,9.\,1897 ebd.@\textsc{?? [Totgeborener Sohn von Arthur Schnitzler und Marie Reinhard]} (24.\,9.\,1897 Endresstraße 68 – 24.\,9.\,1897 ebd.)|pwkv} von Schnitzler und Marie Reinhard\pwindex{Reinhard, Marie 13.\,3.\,1871 Wien – 18.\,3.\,1899 ebd.@\textsc{Reinhard, Marie} (13.\,3.\,1871 Wien – 18.\,3.\,1899 ebd.), \emph{Gesangspädagogin}|pwk} am 24. 9. 1897. Schnitzler gab sich selbst Schuld am Tod des Kind\pwindex{?? [Totgeborener Sohn von Arthur Schnitzler und Marie Reinhard] 24.\,9.\,1897 Endresstraße 68 – 24.\,9.\,1897 ebd.@\textsc{?? [Totgeborener Sohn von Arthur Schnitzler und Marie Reinhard]} (24.\,9.\,1897 Endresstraße 68 – 24.\,9.\,1897 ebd.)|pwk}es (vgl. A. S.: \emph{Tagebuch}, 30. 9. 1897).}}}\label{K_L02826-1}, und mich hat es tief ergriffen.
               Eines muß Dich tröſten: Du haſt keine Schuld. Alles, was Du thun konnteſt, haſt Du
               gethan. Das Schickſal hat es{ }ſo gewollt, und \strikeout{d\textcolor{gray}{×}} da{ }ſtand es nicht mehr in Deiner Macht, zu hindern. Warum das gerade Dich
               treffen mußte? Man muß{ }ſich eben abgewöhnen, nach Gründen zu fragen; es gibt
               keine.\pend
           
\pstart
           Das arme Kind\pwindex{?? [Totgeborener Sohn von Arthur Schnitzler und Marie Reinhard] 24.\,9.\,1897 Endresstraße 68 – 24.\,9.\,1897 ebd.@\textsc{?? [Totgeborener Sohn von Arthur Schnitzler und Marie Reinhard]} (24.\,9.\,1897 Endresstraße 68 – 24.\,9.\,1897 ebd.)|pwv} wollen wir
               nicht beklagen. Es iſt ihm eben nur das Leben erſpart geblieben. Es iſt nach kurzer
                  {\pb}Reiſe an das Ziel gelangt, dem wir alle zugehen
               auf dieſem langen,{ }ſchweren Wege. All’ die Thränen braucht es nicht zu weinen, und
               das Bischen Süßigkeit wird es nicht vermiſſen, weil es{ }ſie nie gekannt hat{\dotssix}\pend
           
\pstart
           Was für bittere Stunden Du durchgemacht haben mußt, armer Freund! \strikeout{\textcolor{gray}{×}\-\textcolor{gray}{×}\-\textcolor{gray}{×}\-\textcolor{gray}{×}\-\textcolor{gray}{×}\-\textcolor{gray}{×}\-\textcolor{gray}{×}\-\textcolor{gray}{×}\-\textcolor{gray}{×}\-\textcolor{gray}{×}\-\textcolor{gray}{×}\-\textcolor{gray}{×}\-\textcolor{gray}{×}\-\textcolor{gray}{×}\-\textcolor{gray}{×}\-\textcolor{gray}{×}\-\textcolor{gray}{×}\-\textcolor{gray}{×}\-\textcolor{gray}{×}\-\textcolor{gray}{×}\-\textcolor{gray}{×}\-\textcolor{gray}{×}}{ }\strikeout{\textcolor{gray}{×}\-\textcolor{gray}{×}\-\textcolor{gray}{×}\-\textcolor{gray}{×}\-\textcolor{gray}{×}\-\textcolor{gray}{×}\-\textcolor{gray}{×}\-\textcolor{gray}{×}\-\textcolor{gray}{×}\-\textcolor{gray}{×}\-\textcolor{gray}{×}\-\textcolor{gray}{×}\-\textcolor{gray}{×}\-\textcolor{gray}{×}\-\textcolor{gray}{×}} Könnte ich nur wenigſtens einen Tag bei Dir{ }ſein! Ich würde Dir immerfort{ }ſagen: »\label{K_L02826-2v}\edtext{Du biſt jung, und nichts iſt
                  verloren.\pwindex{Blanche, August Theodor 17.\,9.\,1811 Stockholm – 30.\,11.\,1868 ebd.@\textsc{Blanche, August Theodor} (17.\,9.\,1811 Stockholm – 30.\,11.\,1868 ebd.), \emph{Schriftsteller}!Erzählungen des Küsters von Dandery@\strich\emph{Erzählungen des Küsters von Dandery}|pwv}}{\lemma{\textnormal{\emph{Du … verloren.}}}\Cendnote{\textnormal{Möglicherweise ein nahezu wörtliches
                  Zitat (S. 100) aus August
                     Blanches\pwindex{Blanche, August Theodor 17.\,9.\,1811 Stockholm – 30.\,11.\,1868 ebd.@\textsc{Blanche, August Theodor} (17.\,9.\,1811 Stockholm – 30.\,11.\,1868 ebd.), \emph{Schriftsteller}|pwk}{ }\emph{Erzählungen des Küsters von
                     Dandery}\pwindex{Blanche, August Theodor 17.\,9.\,1811 Stockholm – 30.\,11.\,1868 ebd.@\textsc{Blanche, August Theodor} (17.\,9.\,1811 Stockholm – 30.\,11.\,1868 ebd.), \emph{Schriftsteller}!Erzählungen des Küsters von Dandery@\strich\emph{Erzählungen des Küsters von Dandery}|pwk} (deutsche Übersetzung 1876; das dänische\oindex{Dänemark@\textbf{Dänemark}|pwkv} Original von
                     1856 trägt den Titel \emph{Berättelser
                     af Klockaren i Danderyd}\pwindex{Blanche, August Theodor 17.\,9.\,1811 Stockholm – 30.\,11.\,1868 ebd.@\textsc{Blanche, August Theodor} (17.\,9.\,1811 Stockholm – 30.\,11.\,1868 ebd.), \emph{Schriftsteller}!Erzählungen des Küsters von Dandery@\strich\emph{Erzählungen des Küsters von Dandery}|pwk}).}}}\label{K_L02826-2}«\pend
           
\pstart
           Am Meiſten aber dauert mich die arme Frau\pwindex{Reinhard, Marie 13.\,3.\,1871 Wien – 18.\,3.\,1899 ebd.@\textsc{Reinhard, Marie} (13.\,3.\,1871 Wien – 18.\,3.\,1899 ebd.), \emph{Gesangspädagogin}|pwv}. Du biſt {\pb}einfach um
               eine{ }ſchöne Hoffnung ärmer (und auch das nur für den Augenblick). Sie\pwindex{Reinhard, Marie 13.\,3.\,1871 Wien – 18.\,3.\,1899 ebd.@\textsc{Reinhard, Marie} (13.\,3.\,1871 Wien – 18.\,3.\,1899 ebd.), \emph{Gesangspädagogin}|pwv} muß es aber als einen wahren \label{K_L02826-3v}\edtext{Zuſammenbruch}{\lemma{\textnormal{\emph{Zusammenbruch}}}\Cendnote{\textnormal{Marie Reinhard\pwindex{Reinhard, Marie 13.\,3.\,1871 Wien – 18.\,3.\,1899 ebd.@\textsc{Reinhard, Marie} (13.\,3.\,1871 Wien – 18.\,3.\,1899 ebd.), \emph{Gesangspädagogin}|pwk} war zumindest Schnitzlers{ }\emph{Tagebuch}\pwindex{Schnitzler, Arthur 15.\,5.\,1862 Wien – 21.\,10.\,1931 ebd.@\textsc{Schnitzler, Arthur} (15.\,5.\,1862 Wien – 21.\,10.\,1931 ebd.), \emph{Schriftsteller, Mediziner}!Tagebuch@\strich\emph{Tagebuch}|pwk} zufolge »gefasst und brav« (A. S.: \emph{Tagebuch}, 25. 9. 1897).}}}\label{K_L02826-3} empfinden. Sei nur
               recht gut und lieb zu ihr. In der Erfüllung dieſer Pflicht wirſt Du auch für Dich den
               beſten Troſt finden. Und{ }ſag’ ihr, daß ich ihr von ganzem Herzen die Hand drücke.\pend
           
\pstart
           Bitte, bitte:{ }ſchreib’ mir bald, und wenn es auch nur ein paar Zeilen{ }ſind.\pend
           
\pstart
           Du{ }ſollteſt jetzt{ }ſo bald als möglich eine \label{K_L02826-4v}\edtext{Reiſe machen}{\lemma{\textnormal{\emph{Reise machen}}}\Cendnote{\textnormal{Schnitzler verreiste erst im November 1897 wieder – nach Prag\oindex{Prag@\textbf{Prag}, \emph{Land}|pwk}, wo am 27. 11. 1897 die Premiere von \emph{Freiwild}\pwindex{Schnitzler, Arthur 15.\,5.\,1862 Wien – 21.\,10.\,1931 ebd.@\textsc{Schnitzler, Arthur} (15.\,5.\,1862 Wien – 21.\,10.\,1931 ebd.), \emph{Schriftsteller, Mediziner}!Freiwild. Schauspiel in 3 Akten@\strich\emph{Freiwild. Schauspiel in 3 Akten}|pwk} im Neuen Deutschen
                     Theater\oindex{Neues Deutsches Theater@\textbf{Neues Deutsches Theater}, \emph{Theater}|pwk} stattfand.}}}\label{K_L02826-4}. Komm zu mir nach \textsc{Paris}\oindex{Paris@\textbf{Paris}, \emph{Hauptstadt}|pw}! {\dots}\pend
           
\pstart
           Armer Freund! Es thut mir innig leid, daß Du, gerade Du dieſen Schmerz {\pb}haben mußteſt! Es iſt auch für mich ein recht
               trauriger Tag.\pend
           
\pstart
           Ich umarme Dich von Herzen und in Treue {\\[\baselineskip]}Dein {\\[\baselineskip]}\spacefill\mbox{Paul Goldmann}\pend
           \leftskip=0em{}
\pstart
           \noindent{}Die \label{K_L02826-5v}\edtext{Briefe}{\lemma{\textnormal{\emph{Briefe}}}\Cendnote{\textnormal{Naheliegend wäre ein Bezug zu der von Jean Thorel\pwindex{Thorel, Jean 11.\,9.\,1859 Éragny – 20.\,8.\,1916 Enghien-les-Bains@\textsc{Thorel, Jean} (11.\,9.\,1859 Éragny – 20.\,8.\,1916 Enghien-les-Bains), \emph{Übersetzer, Dramatiker}|pwk} erstellten Übersetzung\pwindex{Schnitzler, Arthur 15.\,5.\,1862 Wien – 21.\,10.\,1931 ebd.@\textsc{Schnitzler, Arthur} (15.\,5.\,1862 Wien – 21.\,10.\,1931 ebd.), \emph{Schriftsteller, Mediziner}!Amourette. Pièce en trois actes. Adaptée de Arthur Schnitzler@\strich\emph{Amourette. Pièce en trois actes. Adaptée de Arthur Schnitzler}|pwkv} von \emph{Liebelei}\pwindex{Schnitzler, Arthur 15.\,5.\,1862 Wien – 21.\,10.\,1931 ebd.@\textsc{Schnitzler, Arthur} (15.\,5.\,1862 Wien – 21.\,10.\,1931 ebd.), \emph{Schriftsteller, Mediziner}!Liebelei. Schauspiel in drei Akten@\strich\emph{Liebelei. Schauspiel in drei Akten}|pwk}, die noch immer nicht von einem Theater akzeptiert worden
                     war. Siehe XXXX Auszeichnungsfehler: Dokument L02830 nicht gefunden.}}}\label{K_L02826-5}{ }ſind
                  alle beſorgt. Auf Deinen Brief antworte ich Dir nächſtens.\pend
           \selectlanguage{ngerman}\endnumbering\briefempfaengerindex{Schnitzler, Arthur@\textsc{Schnitzler, Arthur}!zzzGoldmann, Paul@\emph{von Paul Goldmann}!1897-09-253@{25. 9. [1897]}|)be}\mylabel{L02826h}  \newcommand{\dateiname}{L02826}\newcommand{\titel}{Paul Goldmann an Arthur Schnitzler, 25. 9. [1897]}\newcommand{\editorInnen}{Martin Anton Müller und Laura Untner}%% latex-leseansicht-abspann.tex
%% Abspann für die Leseansicht.
%% Der Schalter \ifkorrekturansicht ist bereits durch den Vorspann gesetzt.

%% latex-abspann.tex
%% Gemeinsamer Abspann für Korrekturansicht und Leseansicht.
%% Setzt den Schalter \ifkorrekturansicht voraus (gesetzt in den
%% einbindenden Dateien latex-korrekturansicht-abspann.tex bzw.
%% latex-leseansicht-abspann.tex).
%% ---------------------------------------------------------------

\normalsize

% Das esempio-Environment wird nur in der Leseansicht benötigt
\ifkorrekturansicht\else
\newenvironment{esempio}[3]%
{
    \vspace{1.5ex}
    \rlap{\underline{#1}}
    \par
    \setlength{\parindent}{0cm}
    \nopagebreak
    \leftskip=#2cm
    \rightskip=#3cm
}
{
    \par
}
\fi

\doendnotes{C}
\bigskip
\vfill

\clearpage

\footnotesize

\ifkorrekturansicht
  \lohead{\textsc{register}}
\fi

% theindex-Environment neu definieren ohne reledmac
\makeatletter
\renewenvironment{theindex}{%
  \ifkorrekturansicht
    \section*{\indexname}%
  \else
    \subsubsection*{Index der erwähnten Entitäten}%
  \fi
  \setlength{\parindent}{0pt}%
  \setlength{\parskip}{0pt plus 0.3pt}%
  \let\item\@idxitem
}{%
  \ifkorrekturansicht\clearpage\fi
}
\makeatother

\IfFileExists{\jobname-pw.ind}{\input{\jobname-pw.ind}}{}

% Quellenangabe nur in der Leseansicht
\ifkorrekturansicht\else
% Fallback-Definitionen, falls die .tex-Datei \titel etc. nicht gesetzt hat
\providecommand{\titel}{}
\providecommand{\editorInnen}{}
\providecommand{\dateiname}{\jobname}

\vspace{3cm}

\vfill

\footnotesize
\textsc{Quelle}: \titel. Herausgegeben von {\editorInnen}. In: \emph{Arthur Schnitzler: Briefwechsel mit Autorinnen und Autoren}.
 Digitale Edition, https://schnitzler-briefe.acdh.oeaw.ac.at/{\dateiname}.html (Stand \today)
\fi

\end{document}


