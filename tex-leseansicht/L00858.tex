%% latex-korrekturansicht-vorspann.tex
%% Vorspann für die Korrekturansicht.
%% Lädt die gemeinsame Datei latex-vorspann.tex mit gesetztem Schalter.

\newif\ifkorrekturansicht
\korrekturansichttrue

\input{../tex-inputs/latex-vorspann}


\section[Arthur Schnitzler an Richard Beer-Hofmann, {[}18. 11. 1898?{]}]{L00858 Arthur Schnitzler an Richard Beer-Hofmann, {[}18. 11. 1898?{]}}
\nopagebreak\mylabel{L00858v}
\rehead{ }\normalsize\beginnumbering\briefempfaengerindex{Beer-Hofmann, Richard@\textsc{Beer-Hofmann, Richard}!zzzSchnitzler, Arthur@\emph{von Arthur Schnitzler}!1898-11-181@{{[}18. 11. 1898?{]}}|(be}
\toendnotes[C]{\smallbreak\pagebreak[2]}\Standort{YCGL, MSS 31.}
\physDesc{Brief, Umschlag, 244 Zeichen
\newline{}Handschrift: Bleistift, deutsche Kurrent
\newline{}Versand: ohne postalischen Übermittlungsvermerk }\toendnotes[C]{\smallbreak}\pstart{}{\pb}\textsc{Herrn Dr. Richard Beer-Hofmann}\pend{}\pstart{}Wien\oindex{Wien@\textbf{Wien}, \emph{A.ADM2}|pw}\pend{}\pstart{}\textsc{I. Wollzeile 15\oindex{Wollzeile@\textbf{Wollzeile}, \emph{Straße (K.STR)}|pw}}.\pend{}{\bigskip}\vspace{1em}
\pstart
           \noindent{}{\pb}Lieber Richard, Sie erwieſen mir einen Gefallen, we{\geminationn}{ }Sie heut mit mir auf dieſen Sitz im \label{K_L00858-1v}\edtext{2. Stock}{\lemma{\textnormal{\emph{2. Stock}}}\Cendnote{\textnormal{Die archivalische Überlieferung bewahrt dieses undatierte
                  Korrespondenzstück in der Mappe für das Jahr 1898 auf. 
                     Sofern das 
                  verlässlich ist, ergibt sich mit dem \emph{Tagebuch}\pwindex{Tagebuch@\emph{Tagebuch}|pwk}
                  eine genauere Bestimmung. In diesem Jahr besuchte Schnitzler viermal Aufführungen im Raimundtheater\oindex{Raimund-Theater@\textbf{Raimund-Theater}, \emph{Theater (K.THE)}|pwk}. Nur an einem Abend, an dem \emph{Juana}\pwindex{Juana. Drama@\emph{Juana. Drama}|pwk} von Hermann Bahr\pwindex{Bahr, Hermann 19.07.1863 – 15.01.1934@\textsc{Bahr, Hermann} (19.07.1863 – 15.01.1934), \emph{Schriftsteller/Schriftstellerin, Kritiker/Kritikerin}|pwk} und Schnitzlers{ }\emph{Abschiedssouper}\pwindex{Abschiedssouper@\emph{Abschiedssouper}|pwk} gemeinsam gegeben wurden, lässt sich die Anwesenheit von
                     Beer-Hofmann\pwindex{Beer-Hofmann, Richard 1866-07-11 – 1945-09-26@\textsc{Beer-Hofmann, Richard} (1866-07-11 – 1945-09-26), \emph{Schriftsteller/Schriftstellerin}|pwk} belegen, siehe A. S.: \emph{Tagebuch}, 18. 11. 1898.}}}\label{K_L00858-1}, Rmdtheater\oindex{Raimund-Theater@\textbf{Raimund-Theater}, \emph{Theater (K.THE)}|pw} kämen. We{\geminationn}{ }Sie nicht wollen, ſenden Sie mir ihn raſch zurück,
               bitte.\pend
           \pstart Herzlichſt Ihr \spacefill\mbox{Arthur}\pend{}\selectlanguage{ngerman}\endnumbering\briefempfaengerindex{Beer-Hofmann, Richard@\textsc{Beer-Hofmann, Richard}!zzzSchnitzler, Arthur@\emph{von Arthur Schnitzler}!1898-11-181@{{[}18. 11. 1898?{]}}|)be}\mylabel{L00858h}  \normalsize

\doendnotes{C}
\bigskip
\vfill

\clearpage

\footnotesize

\lohead{\textsc{register}}

% Definiere theindex-Environment komplett neu ohne reledmac
\makeatletter
\renewenvironment{theindex}{%
  \section*{\indexname}%
  \setlength{\parindent}{0pt}%
  \setlength{\parskip}{0pt plus 0.3pt}%
  \let\item\@idxitem
}{%
  \clearpage
}
\makeatother

\IfFileExists{\jobname-pw.ind}{\input{\jobname-pw.ind}}{}

\end{document}

      