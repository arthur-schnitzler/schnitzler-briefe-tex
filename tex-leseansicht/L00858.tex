\input{../tex-inputs/latex-pdf-vorspann}
\begin{center}
            \textcolor{red}{ENTWURF. ENTZIFFERUNG NOCH NICHT KORREKTURGELESEN}
                      \end{center}
            
               \section[Arthur Schnitzler an Richard Beer-Hofmann, {[}18. 11. 1898?{]}]{ Arthur Schnitzler an Richard Beer-Hofmann, {[}18. 11. 1898?{]}}\nopagebreak\mylabel{v}\rehead{ }\begin{ledgroupsized}[t]{13cm}\normalsize\beginnumbering\briefempfaengerindex{Beer-Hofmann, Richard@\textsc{Beer-Hofmann, Richard}!zzzSchnitzler, Arthur@\emph{von Arthur Schnitzler}!1898-11-181@{{[}18. 11. 1898?{]}}|(be} \toendnotes[C]{\smallbreak\pagebreak[2]} \Standort{YCGL, MSS 31.}
\physDesc{Brief, Umschlag
\newline{}Handschrift: Bleistift, deutsche Kurrent\newline{}Versand: ohne postalischen Übermittlungsvermerk }\toendnotes[C]{\smallbreak}\pstart{}{\pb}\textsc{Herrn Dr. Richard Beer-Hofmann }\pend{}\pstart{}Wien\oindex{Wien@\textbf{Wien}|pw}\pend{}\pstart{}\textsc{I. Wollzeile 15\oindex{Wollzeile@\textbf{Wollzeile}|pw}}.\pend{}{\bigskip}\pstart
           \noindent{}{\pb}Lieber Richard, Sie erwieſen mir einen Gefallen, we{\geminationn}{ }Sie heut mit mir auf dieſen Sitz im \label{K_L00858_1v}\edtext{2. Stock}{\lemma{\textnormal{\emph{2. Stock}}}\Cendnote{\textnormal{Sofern die archivalische Überlieferung, die dieses undatierte
                  Korrespondenzstück in der Mappe für das Jahr 1898 überliefert,
                  verlässlich ist, ergibt sich mit dem \emph{Tagebuch}\pwindex{Schnitzler, Arthur 15.05.1862 – 21.10.1931@\textsc{Schnitzler, Arthur} (15.05.1862 – 21.10.1931), \emph{Schriftsteller, Mediziner}!Tagebuch1981 – 2000@\strich\emph{Tagebuch} {[}1981 – 2000{]}|pwk}
                  eine mögliche genauere Bestimmung. In diesem Jahr besuchte Schnitzler\pwindex{Schnitzler, Arthur 15.05.1862 – 21.10.1931@\textsc{Schnitzler, Arthur} (15.05.1862 – 21.10.1931), \emph{Schriftsteller, Mediziner}|pwk} viermal Aufführungen im Raimundtheater\oindex{Raimund-Theater@\textbf{Raimund-Theater}|pwk}. Nur an einem Abend, bei der \emph{Juana}\pwindex{Juana1896@\emph{Juana} {[}1896{]}|pwk} von Hermann Bahr\pwindex{Bahr, Hermann 19.07.1863 – 15.01.1934@\textsc{Bahr, Hermann} (19.07.1863 – 15.01.1934), \emph{Schriftsteller, Kritiker}|pwk} und
                  sein eigenes \emph{Abschiedssouper}\pwindex{Schnitzler, Arthur 15.05.1862 – 21.10.1931@\textsc{Schnitzler, Arthur} (15.05.1862 – 21.10.1931), \emph{Schriftsteller, Mediziner}!Abschiedssouper1892@\strich\emph{Abschiedssouper} {[}1892{]}|pwk} gemeinsam gegeben
                  wurden, lässt sich die Anwesenheit von Beer-Hofmann\pwindex{Beer-Hofmann, Richard 11.07.1866 – 26.09.1945@\textsc{Beer-Hofmann, Richard} (11.07.1866 – 26.09.1945), \emph{Schriftsteller}|pwk} belegen, siehe A. S.: \emph{Tagebuch}, 18. 11. 1898.}}}\label{K_L00858_1h}, Rmdtheater\oindex{Raimund-Theater@\textbf{Raimund-Theater}|pw} kämen. We{\geminationn}{ }Sie nicht wollen, ſenden Sie mir ihn raſch zurück,
               bitte.\pend
           \pstart Herzlichſt Ihr \spacefill\mbox{Arthur}\pend{}\endnumbering\briefempfaengerindex{Beer-Hofmann, Richard@\textsc{Beer-Hofmann, Richard}!zzzSchnitzler, Arthur@\emph{von Arthur Schnitzler}!1898-11-181@{{[}18. 11. 1898?{]}}|)be}\mylabel{h}\end{ledgroupsized}  \newcommand{\dateiname}{L00858}\newcommand{\titel}{Arthur Schnitzler an Richard Beer-Hofmann, [18. 11. 1898?]}\newcommand{\editorInnen}{Martin Anton Müller und Gerd-Hermann Susen}\input{../tex-inputs/latex-pdf-abspann}
      