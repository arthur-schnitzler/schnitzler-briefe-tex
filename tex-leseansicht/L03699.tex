%% latex-korrekturansicht-vorspann.tex
%% Vorspann für die Korrekturansicht.
%% Lädt die gemeinsame Datei latex-vorspann.tex mit gesetztem Schalter.

\newif\ifkorrekturansicht
\korrekturansichttrue

\input{../tex-inputs/latex-vorspann}


\section[Elsa Plessner an Arthur Schnitzler, 18. 3. 1896]{L03699 Elsa Plessner an Arthur Schnitzler, 18. 3. 1896}
\nopagebreak\mylabel{L03699v}
\rehead{ }\normalsize\beginnumbering\briefempfaengerindex{Schnitzler, Arthur@\textsc{Schnitzler, Arthur}!zzzPlessner, Elsa@\emph{von Elsa Plessner}!1896-03-181@{18. 3. 1896}|(be}
\toendnotes[C]{\smallbreak\pagebreak[2]}\Standort{DLA, A:Schnitzler, HS.1985.1.419.}
\physDesc{Brief, 1 Blatt, 2 Seiten, 1113 Zeichen
\newline{}Handschrift: , lateinische Kurrent
\newline{}Schnitzler: eine Unterstreichung }\toendnotes[C]{\smallbreak}
\pstart
           {\pb} Bäckerstrasse N\textsuperscript{o}
                     1\oindex{Baeckerstrasse 1@\textbf{Bäckerstraße 1}, \emph{Wohngebäude (K.WHS)}|pw}, den 18. III. 96.\pend
           
\pstart{}Verehrter Herr Doctor!\pend\vspace{0.5em}
\pstart
           Und es herrschte Freude und eitel Sonnenschein und siehe, eine unpässlich zu Bett
               liegende junge Dame wurde vor lauter Vergnügen plötzlich gesund. Das hat Ihr
               liebenswürdiger \label{K_L03699-1v}\edtext{Brief}{\lemma{\textnormal{\emph{Brief}}}\Cendnote{\textnormal{nicht überliefert}}}\label{K_L03699-1} verursacht, für
               den, sowie für die beispiellose bewundernswürdige Schnelligkeit, mit der Sie meine
               Bitte erfüllt haben, ich Ihnen auf das Herzlichste danke. – \pend
           
\pstart
           Wenn Ihre Spannung auf meine ferneren Arbeiten wohl kaum den Grad je erreichen
               dürfte, wie die meine auf Ihr Urtheil war, so will ich doch Gleiches mit Gleichem
               vergelten und Ihnen als Dank ungesäumt drei andere Arbeiten zur gütigen Durchsicht
               übersenden. N\textsuperscript{o} 1. »Pierettes Tagebuch\pwindex{Pierettes Tagebuch [19 unveroeffentlichte Gedichte]@\emph{Pierettes Tagebuch [19 unveröffentlichte Gedichte]}|pw}«, 19 Nummern {\pb}Lyrik, in einer Novelle\pwindex{Pierettes Tagebuch@\emph{Pierettes Tagebuch}|pwv} verstreut gewesene Gedichte\pwindex{Pierettes Tagebuch [19 unveroeffentlichte Gedichte]@\emph{Pierettes Tagebuch [19 unveröffentlichte Gedichte]}|pwv}, die nun für sich
               allein stehen sollen, da die Novelle\pwindex{Pierettes Tagebuch@\emph{Pierettes Tagebuch}|pwv} unbrauchbar war.\pend
           
\pstart
           \label{K_L03699-2v}\edtext{N\textsuperscript{o} 2 und
               3}{\lemma{\textnormal{\emph{N\textsuperscript{o} 2 und 3}}}\Cendnote{\textnormal{Die Beilagen sind nicht überliefert. Die lyrische Zusammenstellung \emph{Pierettes Tagebuch}\pwindex{Pierettes Tagebuch [19 unveroeffentlichte Gedichte]@\emph{Pierettes Tagebuch [19 unveröffentlichte Gedichte]}|pwk} wurde nie publiziert und ist verschollen. Um welche kleinen Prosatexte es sich darüber hinaus handelte, ist
                  nicht bekannt, vermutlich frühe Versionen zweier Texte aus \emph{Der
                     gläserne Käfig}\pwindex{glaeserne Kaefig. Skizzen und Novellen@\emph{Der gläserne Käfig. Skizzen und Novellen}|pwk}.}}}\label{K_L03699-2} kleine Skizzen, Federspritzer, wie ich sie
               sehr gern schreibe. Wenn das kritische Verfahren wieder nur annähernd so kurze Zeit
               in Anspruch nimmt, wie das erstemal, so bauen Sie sich eine weitere Staffel ins
               Himmelreich und einen Dankaltar in meinem Herzen. –\pend
           
\pstart
           Mit vorzüglicher Hochachtung{\\[\baselineskip]}\spacefill\mbox{Elsa Plessner}. \pend
           \leftskip=0em{}\selectlanguage{ngerman}\endnumbering\briefempfaengerindex{Schnitzler, Arthur@\textsc{Schnitzler, Arthur}!zzzPlessner, Elsa@\emph{von Elsa Plessner}!1896-03-181@{18. 3. 1896}|)be}\mylabel{L03699h}
\begin{anhang}
\end{anhang}\normalsize

\doendnotes{C}
\bigskip
\vfill

\clearpage

\footnotesize

\lohead{\textsc{register}}

% Definiere theindex-Environment komplett neu ohne reledmac
\makeatletter
\renewenvironment{theindex}{%
  \section*{\indexname}%
  \setlength{\parindent}{0pt}%
  \setlength{\parskip}{0pt plus 0.3pt}%
  \let\item\@idxitem
}{%
  \clearpage
}
\makeatother

\IfFileExists{\jobname-pw.ind}{\input{\jobname-pw.ind}}{}

\end{document}

      