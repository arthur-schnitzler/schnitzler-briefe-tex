%% latex-leseansicht-vorspann.tex
%% Vorspann für die Leseansicht.
%% Lädt die gemeinsame Datei latex-vorspann.tex mit nicht gesetztem Schalter.

\newif\ifkorrekturansicht
\korrekturansichtfalse

\input{../tex-inputs/latex-vorspann}


\section[Oscar Blumenthal an Arthur Schnitzler, {[}nach dem 13. 3. 1912{]}]{L02111 Oscar Blumenthal an Arthur Schnitzler, {[}nach dem 13. 3. 1912{]}}
\nopagebreak\mylabel{L02111v}
\rehead{ }\normalsize\beginnumbering\briefempfaengerindex{Schnitzler, Arthur@\textsc{Schnitzler, Arthur}!zzzBlumenthal, Oskar@\emph{von Oskar Blumenthal}!1913-04-301@{{[}nach dem 13. 3. 1912{]}}|(be}
\toendnotes[C]{\smallbreak\pagebreak[2]}
\correspDesc{Versand  durch Oscar Blumenthal im Zeitraum [nach dem 13. 3. 1912] in Berlin
\newline{}Erhalt  durch Arthur Schnitzler im Zeitraum [nach dem 13. 3. 1912] in Wien}\toendnotes[C]{\smallbreak}
\Standort{CUL, Schnitzler, B 15.}
\physDesc{Briefkarte, 1 Blatt, 2 Seiten, , 313 Zeichen (Klappkarte )
\newline{}\noindent{}Faksimilierte eigenhändige Danksagung (\noindent{}Faksimilierte eigenhändige Danksagung)
\newline{}Schnitzler: auf der ersten Seite mit Bleistift beschriftet: »{\pb}\textsc{Blumenthal}« 
\newline{}Ordnung: mit Bleistift von unbekannter Hand in eckiger Klammer datiert: »1912« }\toendnotes[C]{\smallbreak}
\pstart
           \centering{}{\pb}{[}Fotografie Blumenthals von Erwin Raupp\pwindex{Raupp, Erwin 31.\,1.\,1863 Karlsruhe – 10.\,10.\,1931 Darmstadt@\textsc{Raupp, Erwin} (31.\,1.\,1863 Karlsruhe – 10.\,10.\,1931 Darmstadt), \emph{Fotograf, Künstler}|pw}{]}\pend
           \vspace{0.5em}
\pstart
           {\pb}Für alle aufrichtenden Worte und
               tröſtenden Zurufe zu meinem \label{K_L02111-1v}\edtext{sechzigſten Geburtstag}{\lemma{\textnormal{\emph{sechzigsten Geburtstag}}}\Cendnote{\textnormal{am
                     13. 3. 1912}}}\label{K_L02111-1}{ }ſpricht der nebenſtehende ältere Herr seinen
               innigſten Dank aus. Denn wenn man sein Alter nicht mehr verbergen kann, so muß man
               damit coquettieren!{\dots} Mit einem warmen Händedruck\pend
           \pstart \spacefill\mbox{Osc. Blumenthal.}\pend{}\selectlanguage{ngerman}\endnumbering\briefempfaengerindex{Schnitzler, Arthur@\textsc{Schnitzler, Arthur}!zzzBlumenthal, Oskar@\emph{von Oskar Blumenthal}!1912-03-141@{{[}nach dem 13. 3. 1912{]}}|)be}\mylabel{L02111h}  \newcommand{\dateiname}{L02111}\newcommand{\titel}{Oscar Blumenthal an Arthur Schnitzler, [nach dem 13. 3. 1912]}\newcommand{\editorInnen}{Martin Anton Müller und Gerd-Hermann Susen}%% latex-leseansicht-abspann.tex
%% Abspann für die Leseansicht.
%% Der Schalter \ifkorrekturansicht ist bereits durch den Vorspann gesetzt.

%% latex-abspann.tex
%% Gemeinsamer Abspann für Korrekturansicht und Leseansicht.
%% Setzt den Schalter \ifkorrekturansicht voraus (gesetzt in den
%% einbindenden Dateien latex-korrekturansicht-abspann.tex bzw.
%% latex-leseansicht-abspann.tex).
%% ---------------------------------------------------------------

\normalsize

% Das esempio-Environment wird nur in der Leseansicht benötigt
\ifkorrekturansicht\else
\newenvironment{esempio}[3]%
{
    \vspace{1.5ex}
    \rlap{\underline{#1}}
    \par
    \setlength{\parindent}{0cm}
    \nopagebreak
    \leftskip=#2cm
    \rightskip=#3cm
}
{
    \par
}
\fi

\doendnotes{C}
\bigskip
\vfill

\clearpage

\footnotesize

\ifkorrekturansicht
  \lohead{\textsc{register}}
\fi

% theindex-Environment neu definieren ohne reledmac
\makeatletter
\renewenvironment{theindex}{%
  \ifkorrekturansicht
    \section*{\indexname}%
  \else
    \subsubsection*{Index der erwähnten Entitäten}%
  \fi
  \setlength{\parindent}{0pt}%
  \setlength{\parskip}{0pt plus 0.3pt}%
  \let\item\@idxitem
}{%
  \ifkorrekturansicht\clearpage\fi
}
\makeatother

\IfFileExists{\jobname-pw.ind}{\input{\jobname-pw.ind}}{}

% Quellenangabe nur in der Leseansicht
\ifkorrekturansicht\else
% Fallback-Definitionen, falls die .tex-Datei \titel etc. nicht gesetzt hat
\providecommand{\titel}{}
\providecommand{\editorInnen}{}
\providecommand{\dateiname}{\jobname}

\vspace{3cm}

\vfill

\footnotesize
\textsc{Quelle}: \titel. Herausgegeben von {\editorInnen}. In: \emph{Arthur Schnitzler: Briefwechsel mit Autorinnen und Autoren}.
 Digitale Edition, https://schnitzler-briefe.acdh.oeaw.ac.at/{\dateiname}.html (Stand \today)
\fi

\end{document}


