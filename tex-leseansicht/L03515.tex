%% latex-korrekturansicht-vorspann.tex
%% Vorspann für die Korrekturansicht.
%% Lädt die gemeinsame Datei latex-vorspann.tex mit gesetztem Schalter.

\newif\ifkorrekturansicht
\korrekturansichttrue

\input{../tex-inputs/latex-vorspann}


\section[ Paul Goldmann an Arthur Schnitzler, 23. 12. 1927]{L03515 Paul Goldmann an Arthur Schnitzler, 23. 12. 1927}
\nopagebreak\mylabel{L03515v}
\rehead{ }\normalsize\beginnumbering\briefempfaengerindex{Schnitzler, Arthur@\textsc{Schnitzler, Arthur}!zzzGoldmann, Paul@\emph{von Paul Goldmann}!1927-12-231@{23. 12. 1927}|(be}
\toendnotes[C]{\smallbreak\pagebreak[2]}\Standort{DLA, A:Schnitzler, HS.NZ85.1.3176.}
\physDesc{Brief, 1 Blatt, 1 Seite, 886 Zeichen
\newline{}Schreibmaschine
\newline{}Handschrift: lila Tinte, lateinische Kurrent (\noindent{}drei Korrekturen, Schlussformel und Unterschrift)
\newline{}Schnitzler: mit rotem Buntstift »Aph{[}orismen{]}\pwindex{Buch der Sprueche und Bedenken@\emph{Buch der Sprüche und Bedenken}|pwv}« vermerkt
                                 und vier Unterstreichungen }\toendnotes[C]{\smallbreak}
\pstart
           {\pb}\textcolor{gray}{\textbf{Dr. Paul Goldmann}}\hfill \textcolor{gray}{\textbf{Berlin W. 10\oindex{Berlin@\textbf{Berlin}, \emph{P.PPLC}|pw}}}\pend
           
\pstart
           \textcolor{gray}{\textbf{Vertreter der »Neuen Freien
                           Presse\orgindex{Neue Freie Presse@Neue Freie Presse|pw}«}}\hfill \textcolor{gray}{\textbf{Bendlerſtraße 36\oindex{Stauffenbergstrasse@\textbf{Stauffenbergstraße}, \emph{Straße (K.STR)}|pw}.}}\pend
           
\pstart
           \raggedleft{}\textcolor{gray}{\textbf{Tel.: Lützow 9142}}\pend
           
\pstart
           \raggedleft{}23. 12. 27.\pend
           
\pstart\center{}Lieber Arthur,\pend\vspace{0.5em}
\pstart
           In unser aller Namen danke ich Dir herzlichst für Dein neues \label{K_L03515-1v}\edtext{Buch\pwindex{Buch der Sprueche und Bedenken@\emph{Buch der Sprüche und Bedenken}|pwv}}{\lemma{\textnormal{\emph{Buch}}}\Cendnote{\textnormal{Die Aphorismensammlung \emph{Buch der Sprüche und Bedenken}\pwindex{Buch der Sprueche und Bedenken@\emph{Buch der Sprüche und Bedenken}|pwk} war am 17. 10. 1927 im Wien\oindex{Wien@\textbf{Wien}, \emph{A.ADM2}|pwk}er \emph{Phaidon-Verlag}\orgindex{Phaidon-Verlag@Phaidon-Verlag|pwk}
                  erschienen.}}}\label{K_L03515-1}. Einiges von \substVorne{}\textsuperscript{D}\substDazwischen{}s\substHinten{}einem Inhalt kenne ich bereits aus Zeitungen und Zeitschriften, das übrige
               freue ich mich, im Buche\pwindex{Buch der Sprueche und Bedenken@\emph{Buch der Sprüche und Bedenken}|pwv} zu
               lesen. Meine Tochter\pwindex{Goldmann, Franziska 1911-05-29 – 1963-08-19@\textsc{Goldmann, Franziska} (1911-05-29 – 1963-08-19), \emph{Schauspieler/Schauspielerin}|pwv} ist
               bereits in Deine Spruchweisheit vertieft, – während der Feiertage werde ich \substVorne{}\textsuperscript{I}\substDazwischen{}i\substHinten{}hr das Buch\pwindex{Buch der Sprueche und Bedenken@\emph{Buch der Sprüche und Bedenken}|pwv}
               entreissen. Es war sehr lieb von Dir, dass Du unser gedacht hast.\pend
           
\pstart
           Infolge der \label{K_L03515-2v}\edtext{Verschiebung der Première\pwindex{Es ist mein Wille Eine unwahrscheinliche Begebenheit aus dem 18. Jahrhundert in einem Akt@\emph{Es ist mein Wille{\rufezeichen} Eine unwahrscheinliche Begebenheit aus dem 18. Jahrhundert in einem Akt}|pwv} im Akademietheater\oindex{Akademietheater@\textbf{Akademietheater}, \emph{Theater (K.THE)}|pw}}{\lemma{\textnormal{\emph{Verschiebung … Akademietheater}}}\Cendnote{\textnormal{Die ursprünglich für Mitte Dezember 1927 angesetzte Uraufführung von Goldmanns\pwindex{Goldmann, Paul 31.01.1865 – 25.09.1935@\textsc{Goldmann, Paul} (31.01.1865 – 25.09.1935), \emph{Schriftsteller/Schriftstellerin, Journalist/Journalistin}|pwk} Einakter \emph{Es ist mein Wille! Eine unwahrscheinliche Begebenheit aus
                     dem 18. Jahrhundert in einem Akt}\pwindex{Es ist mein Wille Eine unwahrscheinliche Begebenheit aus dem 18. Jahrhundert in einem Akt@\emph{Es ist mein Wille{\rufezeichen} Eine unwahrscheinliche Begebenheit aus dem 18. Jahrhundert in einem Akt}|pwk} fand am 5. 1. 1928 im Wien\oindex{Wien@\textbf{Wien}, \emph{A.ADM2}|pwk}er Akademietheater\oindex{Akademietheater@\textbf{Akademietheater}, \emph{Theater (K.THE)}|pwk} statt. Bereits 1924 war das Stück\pwindex{Es ist mein Wille Eine unwahrscheinliche Begebenheit aus dem 18. Jahrhundert in einem Akt@\emph{Es ist mein Wille{\rufezeichen} Eine unwahrscheinliche Begebenheit aus dem 18. Jahrhundert in einem Akt}|pwkv} als Sonderdruck der \emph{Neuen Freien Presse}\orgindex{Neue Freie Presse@Neue Freie Presse|pwk} in der \emph{Österreichischen Journal A. G.}\orgindex{Oesterreichische Journal A.G.@Österreichische Journal A.G.|pwk} erschienen.}}}\label{K_L03515-2} hat sich auch meine
               Reise nach Wien\oindex{Wien@\textbf{Wien}, \emph{A.ADM2}|pw} verschoben. Das Stück\pwindex{Es ist mein Wille Eine unwahrscheinliche Begebenheit aus dem 18. Jahrhundert in einem Akt@\emph{Es ist mein Wille{\rufezeichen} Eine unwahrscheinliche Begebenheit aus dem 18. Jahrhundert in einem Akt}|pwv} soll angeblich Anfang Januar herauskommen, – ob ich dann werde meinen Berlin\oindex{Berlin@\textbf{Berlin}, \emph{P.PPLC}|pw}er Posten verlassen können, ist noch
               ungewiss. Wenn ich nach Wien\oindex{Wien@\textbf{Wien}, \emph{A.ADM2}|pw} komme und wenn mein
               Aufenthalt nicht allzu kurz bemessen ist, werde ich Dich natürlich dort \label{K_L03515-3v}\edtext{wiedersehen}{\lemma{\textnormal{\emph{wiedersehen}}}\Cendnote{\textnormal{Schnitzler besuchte die Aufführung von \emph{Es ist mein Wille!}\pwindex{Es ist mein Wille Eine unwahrscheinliche Begebenheit aus dem 18. Jahrhundert in einem Akt@\emph{Es ist mein Wille{\rufezeichen} Eine unwahrscheinliche Begebenheit aus dem 18. Jahrhundert in einem Akt}|pwk} am 8. 1. 1928. Goldmann\pwindex{Goldmann, Paul 31.01.1865 – 25.09.1935@\textsc{Goldmann, Paul} (31.01.1865 – 25.09.1935), \emph{Schriftsteller/Schriftstellerin, Journalist/Journalistin}|pwk} traf er am 10. 1. 1928. Dort
                  teilte er ihm mit, dass ihm das Stück\pwindex{Es ist mein Wille Eine unwahrscheinliche Begebenheit aus dem 18. Jahrhundert in einem Akt@\emph{Es ist mein Wille{\rufezeichen} Eine unwahrscheinliche Begebenheit aus dem 18. Jahrhundert in einem Akt}|pwkv} nicht gefallen hatte.}}}\label{K_L03515-3}. Inzwischen wünsche ich Dir, auch im
               Namen von Frau\pwindex{Goldmann, Eva Marie 27.10.1877 – 02.11.1937@\textsc{Goldmann, Eva Marie} (27.10.1877 – 02.11.1937)|pwv} und Tochter\pwindex{Goldmann, Franziska 1911-05-29 – 1963-08-19@\textsc{Goldmann, Franziska} (1911-05-29 – 1963-08-19), \emph{Schauspieler/Schauspielerin}|pwv}, frohe Feiertage und
               ein glückliches neues Jahr. Wir alle grüssen Dich
               herzlichst.\pend
           
\pstart
           {[}hs.:{]} Dein {\\[\baselineskip]}\spacefill\mbox{Paul Goldmann.}\pend
           \leftskip=0em{}\selectlanguage{ngerman}\endnumbering\briefempfaengerindex{Schnitzler, Arthur@\textsc{Schnitzler, Arthur}!zzzGoldmann, Paul@\emph{von Paul Goldmann}!1927-12-231@{23. 12. 1927}|)be}\mylabel{L03515h}  \normalsize

\doendnotes{C}
\bigskip
\vfill

\clearpage

\footnotesize

\lohead{\textsc{register}}

% Definiere theindex-Environment komplett neu ohne reledmac
\makeatletter
\renewenvironment{theindex}{%
  \section*{\indexname}%
  \setlength{\parindent}{0pt}%
  \setlength{\parskip}{0pt plus 0.3pt}%
  \let\item\@idxitem
}{%
  \clearpage
}
\makeatother

\IfFileExists{\jobname-pw.ind}{\input{\jobname-pw.ind}}{}

\end{document}

      