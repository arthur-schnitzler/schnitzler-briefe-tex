%% latex-leseansicht-vorspann.tex
%% Vorspann für die Leseansicht.
%% Lädt die gemeinsame Datei latex-vorspann.tex mit nicht gesetztem Schalter.

\newif\ifkorrekturansicht
\korrekturansichtfalse

\input{../tex-inputs/latex-vorspann}

\begin{center}
            \textcolor{red}{ENTWURF, NICHT FERTIG KORRIGIERT}
                      \end{center}
            
         
         \renewcommand{\erwaehntePersonen}{Personen: Paul Goldmann, Franziska Goldmann, Eva Marie Goldmann}
         \renewcommand{\erwaehnteInstitutionen}{Institutionen: Neue Freie Presse, Phaidon-Verlag, Österreichische Journal A.G.}
         \renewcommand{\erwaehnteOrte}{Orte: Akademietheater, Bendlerstraße, Berlin, Wien}
         \renewcommand{\erwaehnteWerke}{Werke: Buch der Sprüche und Bedenken, Es ist mein Wille{\rufezeichen} Eine unwahrscheinliche Begebenheit aus dem 18. Jahrhundert in einem Akt}
               \section[ Paul Goldmann an Arthur Schnitzler, 23. 12. 1927]{ Paul Goldmann an Arthur Schnitzler, 23. 12. 1927}\nopagebreak\mylabel{v}\rehead{ }\begin{ledgroupsized}[t]{13cm}\normalsize\beginnumbering \toendnotes[C]{\smallbreak\pagebreak[2]} \Standort{DLA, A:Schnitzler, HS.NZ85.1.3176.}
\physDesc{Brief, 1 Blatt, 1 Seite, 886 Zeichen
\newline{}Schreibmaschine
\newline{}Handschrift: lila Tinte, lateinische Kurrent (\noindent{}drei Korrekturen, Schlussformel und Unterschrift)
\newline{}Schnitzler: mit rotem Buntstift »Aph{[}orismen{]}« vermerkt und vier
                                 Unterstreichungen }\toendnotes[C]{\smallbreak}\pstart
           \noindent{}{\pb}\textcolor{gray}{\textbf{Dr. Paul Goldmann}}\hfill \textcolor{gray}{\textbf{Berlin W. 10\oindex{Berlin@\textbf{Berlin}|pw}}}\pend
           \pstart
           \textcolor{gray}{\textbf{Vertreter der »Neuen Freien
                           Presse\orgindex{Neue Freie Presse@Neue Freie Presse|pw}«}}\hfill \textcolor{gray}{\textbf{Bendlerſtraße 36\oindex{Bendlerstrasse@\textbf{Bendlerstraße}|pw}.}}\pend
           \pstart
           \raggedleft{}\textcolor{gray}{\textbf{Tel. Lützow 9142}}\pend
           \pstart
           \raggedleft{}23. 12. 27.\pend
           \pstart\center{}Lieber Arthur,\pend\pstart
           In unser aller Namen danke ich Dir herzlichst für Dein neues \label{K_L03515-1v}\edtext{Buch\pwindex{Schnitzler, Arthur 15.05.1862 – 21.10.1931@\textsc{Schnitzler, Arthur} (15.05.1862 – 21.10.1931), \emph{Schriftsteller, Mediziner}!Buch der Sprueche und Bedenken1927-10-17@\strich\emph{Buch der Sprüche und Bedenken} {[}1927-10-17{]}|pwv}}{\lemma{\textnormal{\emph{Buch}}}\Cendnote{\textnormal{Die Aphorismensammlung \emph{Buch der Sprüche und Bedenken}\pwindex{Schnitzler, Arthur 15.05.1862 – 21.10.1931@\textsc{Schnitzler, Arthur} (15.05.1862 – 21.10.1931), \emph{Schriftsteller, Mediziner}!Buch der Sprueche und Bedenken1927-10-17@\strich\emph{Buch der Sprüche und Bedenken} {[}1927-10-17{]}|pwk} war am 17. 10. 1927 im Wien\oindex{Wien@\textbf{Wien}|pwk}er \emph{Phaidon-Verlag}\orgindex{Phaidon-Verlag@Phaidon-Verlag|pwk}
                  erschienen.}}}\label{K_L03515-1h}. Einiges von \substVorne{}\textsuperscript{D}\substDazwischen{}s\substHinten{}einem Inhalt kenne ich bereits aus Zeitungen und Zeitschriften, das übrige
               freue ich mich, im Buche\pwindex{Schnitzler, Arthur 15.05.1862 – 21.10.1931@\textsc{Schnitzler, Arthur} (15.05.1862 – 21.10.1931), \emph{Schriftsteller, Mediziner}!Buch der Sprueche und Bedenken1927-10-17@\strich\emph{Buch der Sprüche und Bedenken} {[}1927-10-17{]}|pwv} zu
               lesen. Meine Tochter\pwindex{Goldmann, Franziska 1911-05-29 – 1963-08-19@\textsc{Goldmann, Franziska} (1911-05-29 – 1963-08-19), \emph{Schauspielerin}|pwv} ist
               bereits in Deine Spruchweisheit vertieft, – während der Feiertage werde ich \substVorne{}\textsuperscript{I}\substDazwischen{}i\substHinten{}hr das Buch\pwindex{Schnitzler, Arthur 15.05.1862 – 21.10.1931@\textsc{Schnitzler, Arthur} (15.05.1862 – 21.10.1931), \emph{Schriftsteller, Mediziner}!Buch der Sprueche und Bedenken1927-10-17@\strich\emph{Buch der Sprüche und Bedenken} {[}1927-10-17{]}|pwv}
               entreissen. Es war sehr lieb von Dir, dass Du unser gedacht hast.\pend
           \pstart
           Infolge der \label{K_L03515-2v}\edtext{Verschiebung der Première\pwindex{Goldmann, Paul 31.01.1865 – 25.09.1935@\textsc{Goldmann, Paul} (31.01.1865 – 25.09.1935), \emph{Schriftsteller, Journalist}!Es ist mein Wille Eine unwahrscheinliche Begebenheit aus dem 18. Jahrhundert in einem Akt1924@\strich\emph{Es ist mein Wille{\rufezeichen} Eine unwahrscheinliche Begebenheit aus dem 18. Jahrhundert in einem Akt} {[}1924{]}|pwv} im Akademietheater\oindex{Akademietheater@\textbf{Akademietheater}|pw}}{\lemma{\textnormal{\emph{Verschiebung … Akademietheater}}}\Cendnote{\textnormal{Die ursprünglich für Mitte Dezember 1927 angesetzte Uraufführung von Goldmann\pwindex{Goldmann, Paul 31.01.1865 – 25.09.1935@\textsc{Goldmann, Paul} (31.01.1865 – 25.09.1935), \emph{Schriftsteller, Journalist}|pwk}s Einakter \emph{Es ist mein Wille! Eine unwahrscheinliche Begebenheit aus
                     dem 18. Jahrhundert in einem Akt}\pwindex{Goldmann, Paul 31.01.1865 – 25.09.1935@\textsc{Goldmann, Paul} (31.01.1865 – 25.09.1935), \emph{Schriftsteller, Journalist}!Es ist mein Wille Eine unwahrscheinliche Begebenheit aus dem 18. Jahrhundert in einem Akt1924@\strich\emph{Es ist mein Wille{\rufezeichen} Eine unwahrscheinliche Begebenheit aus dem 18. Jahrhundert in einem Akt} {[}1924{]}|pwk} fand am 5. 1. 1928 im Wien\oindex{Wien@\textbf{Wien}|pwk}er Akademietheater\oindex{Akademietheater@\textbf{Akademietheater}|pwk} statt. Bereits 1924 war das Stück\pwindex{Goldmann, Paul 31.01.1865 – 25.09.1935@\textsc{Goldmann, Paul} (31.01.1865 – 25.09.1935), \emph{Schriftsteller, Journalist}!Es ist mein Wille Eine unwahrscheinliche Begebenheit aus dem 18. Jahrhundert in einem Akt1924@\strich\emph{Es ist mein Wille{\rufezeichen} Eine unwahrscheinliche Begebenheit aus dem 18. Jahrhundert in einem Akt} {[}1924{]}|pwkv} als Sonderdruck der \emph{Neuen Freien Presse}\orgindex{Neue Freie Presse@Neue Freie Presse|pwk} in der \emph{Österreichischen Journal A. G.}\orgindex{Oesterreichische Journal A.G.@Österreichische Journal A.G.|pwk} erschienen.}}}\label{K_L03515-2h} hat sich auch meine
               Reise nach Wien\oindex{Wien@\textbf{Wien}|pw} verschoben. Das Stück\pwindex{Goldmann, Paul 31.01.1865 – 25.09.1935@\textsc{Goldmann, Paul} (31.01.1865 – 25.09.1935), \emph{Schriftsteller, Journalist}!Es ist mein Wille Eine unwahrscheinliche Begebenheit aus dem 18. Jahrhundert in einem Akt1924@\strich\emph{Es ist mein Wille{\rufezeichen} Eine unwahrscheinliche Begebenheit aus dem 18. Jahrhundert in einem Akt} {[}1924{]}|pwv} soll angeblich Anfang Januar herauskommen, – ob ich dann werde meinen Berlin\oindex{Berlin@\textbf{Berlin}|pw}er Posten verlassen können, ist noch
               ungewiss. Wenn ich nach Wien\oindex{Wien@\textbf{Wien}|pw} komme und wenn mein
               Aufenthalt nicht allzu kurz bemessen ist, werde ich Dich natürlich dort \label{K_L03515-3v}\edtext{wiedersehen}{\lemma{\textnormal{\emph{wiedersehen}}}\Cendnote{\textnormal{Schnitzler\pwindex{Schnitzler, Arthur 15.05.1862 – 21.10.1931@\textsc{Schnitzler, Arthur} (15.05.1862 – 21.10.1931), \emph{Schriftsteller, Mediziner}|pwk} besuchte die Aufführung von \emph{Es ist mein Wille!}\pwindex{Goldmann, Paul 31.01.1865 – 25.09.1935@\textsc{Goldmann, Paul} (31.01.1865 – 25.09.1935), \emph{Schriftsteller, Journalist}!Es ist mein Wille Eine unwahrscheinliche Begebenheit aus dem 18. Jahrhundert in einem Akt1924@\strich\emph{Es ist mein Wille{\rufezeichen} Eine unwahrscheinliche Begebenheit aus dem 18. Jahrhundert in einem Akt} {[}1924{]}|pwk} am 8. 1. 1928. Er traf
                     Goldmann\pwindex{Goldmann, Paul 31.01.1865 – 25.09.1935@\textsc{Goldmann, Paul} (31.01.1865 – 25.09.1935), \emph{Schriftsteller, Journalist}|pwk} auch am 10. 1. 1928, wo er ihm
                  mitteilte, dass ihm das Stück\pwindex{Goldmann, Paul 31.01.1865 – 25.09.1935@\textsc{Goldmann, Paul} (31.01.1865 – 25.09.1935), \emph{Schriftsteller, Journalist}!Es ist mein Wille Eine unwahrscheinliche Begebenheit aus dem 18. Jahrhundert in einem Akt1924@\strich\emph{Es ist mein Wille{\rufezeichen} Eine unwahrscheinliche Begebenheit aus dem 18. Jahrhundert in einem Akt} {[}1924{]}|pwkv} nicht gefallen hatte.}}}\label{K_L03515-3h}. Inzwischen wünsche ich Dir, auch im
               Namen von Frau\pwindex{Goldmann, Eva Marie 27.10.1877 – 02.11.1937@\textsc{Goldmann, Eva Marie} (27.10.1877 – 02.11.1937)|pwv} und Tochter\pwindex{Goldmann, Franziska 1911-05-29 – 1963-08-19@\textsc{Goldmann, Franziska} (1911-05-29 – 1963-08-19), \emph{Schauspielerin}|pwv}, frohe Feiertage und
               ein glückliches neues Jahr. Wir alle grüssen Dich
               herzlichst.\pend
           \pstart
           {[}hs.:{]} Dein {\\[\baselineskip]}\spacefill\mbox{Paul Goldmann.}\pend
           \leftskip=0em{}
         
         \endnumbering\mylabel{h}\end{ledgroupsized}  \newcommand{\dateiname}{L03515}\newcommand{\titel}{Paul Goldmann an Arthur Schnitzler, 23. 12. 1927}\newcommand{\editorInnen}{Martin Anton Müller und Laura Untner}%% latex-leseansicht-abspann.tex
%% Abspann für die Leseansicht.
%% Der Schalter \ifkorrekturansicht ist bereits durch den Vorspann gesetzt.

%% latex-abspann.tex
%% Gemeinsamer Abspann für Korrekturansicht und Leseansicht.
%% Setzt den Schalter \ifkorrekturansicht voraus (gesetzt in den
%% einbindenden Dateien latex-korrekturansicht-abspann.tex bzw.
%% latex-leseansicht-abspann.tex).
%% ---------------------------------------------------------------

\normalsize

% Das esempio-Environment wird nur in der Leseansicht benötigt
\ifkorrekturansicht\else
\newenvironment{esempio}[3]%
{
    \vspace{1.5ex}
    \rlap{\underline{#1}}
    \par
    \setlength{\parindent}{0cm}
    \nopagebreak
    \leftskip=#2cm
    \rightskip=#3cm
}
{
    \par
}
\fi

\doendnotes{C}
\bigskip
\vfill

\clearpage

\footnotesize

\ifkorrekturansicht
  \lohead{\textsc{register}}
\fi

% theindex-Environment neu definieren ohne reledmac
\makeatletter
\renewenvironment{theindex}{%
  \ifkorrekturansicht
    \section*{\indexname}%
  \else
    \subsubsection*{Index der erwähnten Entitäten}%
  \fi
  \setlength{\parindent}{0pt}%
  \setlength{\parskip}{0pt plus 0.3pt}%
  \let\item\@idxitem
}{%
  \ifkorrekturansicht\clearpage\fi
}
\makeatother

\IfFileExists{\jobname-pw.ind}{\input{\jobname-pw.ind}}{}

% Quellenangabe nur in der Leseansicht
\ifkorrekturansicht\else
% Fallback-Definitionen, falls die .tex-Datei \titel etc. nicht gesetzt hat
\providecommand{\titel}{}
\providecommand{\editorInnen}{}
\providecommand{\dateiname}{\jobname}

\vspace{3cm}

\vfill

\footnotesize
\textsc{Quelle}: \titel. Herausgegeben von {\editorInnen}. In: \emph{Arthur Schnitzler: Briefwechsel mit Autorinnen und Autoren}.
 Digitale Edition, https://schnitzler-briefe.acdh.oeaw.ac.at/{\dateiname}.html (Stand \today)
\fi

\end{document}


      