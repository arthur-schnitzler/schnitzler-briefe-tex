%% latex-leseansicht-vorspann.tex
%% Vorspann für die Leseansicht.
%% Lädt die gemeinsame Datei latex-vorspann.tex mit nicht gesetztem Schalter.

\newif\ifkorrekturansicht
\korrekturansichtfalse

\input{../tex-inputs/latex-vorspann}


\section[Arthur Schnitzler an Richard Beer-Hofmann, 19. 8. 1895]{L00475 Arthur Schnitzler an Richard Beer-Hofmann, 19. 8. 1895}
\nopagebreak\mylabel{L00475v}
\rehead{ }\normalsize\beginnumbering\briefempfaengerindex{Beer-Hofmann, Richard@\textsc{Beer-Hofmann, Richard}!zzzSchnitzler, Arthur@\emph{von Arthur Schnitzler}!1895-08-191@{19. 8. 1895}|(be}
\toendnotes[C]{\smallbreak\pagebreak[2]}
\correspDesc{Versand  durch Arthur Schnitzler am 19. 8. 1895 in Salzburg
\newline{}Erhalt  durch Richard Beer-Hofmann am 19. 8. 1895 in Bad Ischl}\toendnotes[C]{\smallbreak}
\Standort{YCGL, MSS 31.}
\physDesc{Telegramm, 138 Zeichen
\newline{}HandschriftX2 einer Schreibkraft: blaue Tinte, deutsche Kurrent
\newline{}Versand: »\noindent{}\textcolor{gray}{\textbf{Aufgegeben am}}{ }19/8 \textcolor{gray}{\textbf{18}}95{ }\textcolor{gray}{\textbf{um}}{ }3 \textcolor{gray}{\textbf{Uhr}} 20 \textcolor{gray}{\textbf{Min.}} N \textcolor{gray}{\textbf{Mittag}}{ / }\textcolor{gray}{\textbf{Eingelangt von}} S \textcolor{gray}{\textbf{auf Leitung Nr.}} 587 \textcolor{gray}{\textbf{am}}{ }19/8\textcolor{gray}{\textbf{189}}5{ }\textcolor{gray}{\textbf{um}}{ }4 \textcolor{gray}{\textbf{Uhr}} 20 \textcolor{gray}{\textbf{Min. {\dots} Mittag}}{ / }\textcolor{gray}{\textbf{Aufgenommen durch}}{ }\textcolor{gray}{K}{ / }\textcolor{gray}{\textbf{Von}}{ }Salzburg\oindex{Salzburg@\textbf{Salzburg}, \emph{Verwaltungsgebiet}|pw}{ }\textcolor{gray}{\textbf{mit}} 1099\textcolor{gray}{P}{ }\textcolor{gray}{\textbf{Taxworten (}}21 \textcolor{gray}{\textbf{Worten {\dots} Chiffern)}}« }\pstart{}{\pb}Richard Beer-Hoffmann\pend{}\pstart{}Eglmosgasse 22\oindex{Eglmoosgasse@\textbf{Eglmoosgasse}, \emph{Bezirk}|pw}\pend{}\pstart{}\textcolor{gray}{\textbf{\textit{Ischl}}}\oindex{Bad Ischl@\textbf{Bad Ischl}|pw}\pend{}{\bigskip}\vspace{1em}
\pstart
           \noindent{}{\pb}Kommen Sie doch morgen abend damit wir wenigſtens
               ganzen Mittwoch Alle zuſa{\geminationm}en{ }ſind.\pend
           
\pstart
           Herzlichſt{\\[\baselineskip]}\spacefill\mbox{Arthur}\pend
           \leftskip=0em{}\selectlanguage{ngerman}\endnumbering\briefempfaengerindex{Beer-Hofmann, Richard@\textsc{Beer-Hofmann, Richard}!zzzSchnitzler, Arthur@\emph{von Arthur Schnitzler}!1895-08-191@{19. 8. 1895}|)be}\mylabel{L00475h}  \newcommand{\dateiname}{L00475}\newcommand{\titel}{Arthur Schnitzler an Richard Beer-Hofmann, 19. 8. 1895}\newcommand{\editorInnen}{Martin Anton Müller und Gerd-Hermann Susen}%% latex-leseansicht-abspann.tex
%% Abspann für die Leseansicht.
%% Der Schalter \ifkorrekturansicht ist bereits durch den Vorspann gesetzt.

%% latex-abspann.tex
%% Gemeinsamer Abspann für Korrekturansicht und Leseansicht.
%% Setzt den Schalter \ifkorrekturansicht voraus (gesetzt in den
%% einbindenden Dateien latex-korrekturansicht-abspann.tex bzw.
%% latex-leseansicht-abspann.tex).
%% ---------------------------------------------------------------

\normalsize

% Das esempio-Environment wird nur in der Leseansicht benötigt
\ifkorrekturansicht\else
\newenvironment{esempio}[3]%
{
    \vspace{1.5ex}
    \rlap{\underline{#1}}
    \par
    \setlength{\parindent}{0cm}
    \nopagebreak
    \leftskip=#2cm
    \rightskip=#3cm
}
{
    \par
}
\fi

\doendnotes{C}
\bigskip
\vfill

\clearpage

\footnotesize

\ifkorrekturansicht
  \lohead{\textsc{register}}
\fi

% theindex-Environment neu definieren ohne reledmac
\makeatletter
\renewenvironment{theindex}{%
  \ifkorrekturansicht
    \section*{\indexname}%
  \else
    \subsubsection*{Index der erwähnten Entitäten}%
  \fi
  \setlength{\parindent}{0pt}%
  \setlength{\parskip}{0pt plus 0.3pt}%
  \let\item\@idxitem
}{%
  \ifkorrekturansicht\clearpage\fi
}
\makeatother

\IfFileExists{\jobname-pw.ind}{\input{\jobname-pw.ind}}{}

% Quellenangabe nur in der Leseansicht
\ifkorrekturansicht\else
% Fallback-Definitionen, falls die .tex-Datei \titel etc. nicht gesetzt hat
\providecommand{\titel}{}
\providecommand{\editorInnen}{}
\providecommand{\dateiname}{\jobname}

\vspace{3cm}

\vfill

\footnotesize
\textsc{Quelle}: \titel. Herausgegeben von {\editorInnen}. In: \emph{Arthur Schnitzler: Briefwechsel mit Autorinnen und Autoren}.
 Digitale Edition, https://schnitzler-briefe.acdh.oeaw.ac.at/{\dateiname}.html (Stand \today)
\fi

\end{document}


