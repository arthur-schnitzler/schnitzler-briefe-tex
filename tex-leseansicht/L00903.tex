%% latex-korrekturansicht-vorspann.tex
%% Vorspann für die Korrekturansicht.
%% Lädt die gemeinsame Datei latex-vorspann.tex mit gesetztem Schalter.

\newif\ifkorrekturansicht
\korrekturansichttrue

\input{../tex-inputs/latex-vorspann}


\section[Arthur Schnitzler an Hermann Bahr, 9. 3. 1899]{L00903 Arthur Schnitzler an Hermann Bahr, 9. 3. 1899}
\nopagebreak\mylabel{L00903v}
\rehead{ }\normalsize\beginnumbering\briefempfaengerindex{Bahr, Hermann@\textsc{Bahr, Hermann}!zzzSchnitzler, Arthur@\emph{von Arthur Schnitzler}!1899-03-091@{9. 3. 1899}|(be}
\toendnotes[C]{\smallbreak\pagebreak[2]}\Standort{TMW, HS AM 23336 Ba.}
\physDesc{Brief, 1 Blatt, 3 Seiten, 627 Zeichen
\newline{}Handschrift: schwarze Tinte, deutsche Kurrent
\newline{}Ordnung: 1) Lochung  2) mit Bleistift von unbekannter Hand datiert: »9. 3. 99«}
\buchAbdrucke{\weitereDrucke{1) Arthur Schnitzler: \emph{The Letters of Arthur Schnitzler to Hermann Bahr}. Chapel Hill: \emph{The University of North Carolina Press} 1978, S. 65.} \weitereDrucke{2) Hermann Bahr, Arthur Schnitzler: \emph{Briefwechsel, Aufzeichnungen, Dokumente (1891–1931)}. Göttingen: \emph{Wallstein} 2018, S. 169.} }\toendnotes[C]{\smallbreak}
\pstart
           \noindent{}{\pb}Lieber Bahr, die Sache ſtimmt nicht. Ich habe dir von \label{K_L00903-1v}\edtext{Anfang an ſowohl geſchrieben}{\lemma{\textnormal{\emph{Anfang … geſchrieben}}}\Cendnote{\textnormal{Hier ist Schnitzler ungenau: Er hatte es nicht »vor« der Aufführung angeboten,
                     vgl. Arthur Schnitzler an Hermann Bahr, 1. 12. 1898.}}}\label{K_L00903-1} als
               geſagt, dſs ich dir das Stück\pwindex{Gefaehrtin. Schauspiel in einem Akt@\emph{Die Gefährtin. Schauspiel in einem Akt}|pwv}
               erſt \uline{nach} der Première geben kann und will; ja, vor
               etwa 3 Wochen, als ich dich in der Landesgerichtsſtraße\oindex{Landesgerichtsstrasse@\textbf{Landesgerichtsstraße}, \emph{Straße (K.STR)}|pw} begegnete und der \label{K_L00903-2v}\edtext{Aufführgs{\pb}termin}{\lemma{\textnormal{\emph{Aufführgstermin}}}\Cendnote{\textnormal{Dieser war bereits am 1. 3. 1899
                  gewesen.}}}\label{K_L00903-2} bereits feſt\damage{ſtan}d, ſagteſt du ſelbſt, daſs du es erſt im Mai (alſo eine beträchtliche Zeit
               nach der Aufführg) abdrucken wollteſt.\pend
           
\pstart
           Wozu alſo läßt du dich in die von mir von vornherein abgelehnte Discuſſion ein. Es
               war halt eine, na ſagen wir, eine Schlamperei von {\pb}dir; \damage{m}eine Verwunderung iſt ſo gering als mein Gram, und damit Schluſs.\pend
           
\pstart
           Ich grüß dich beſtens.{\\[\baselineskip]}Dein \spacefill\mbox{Arth Sch}\pend
           \leftskip=0em{}
\pstart
           Wien\oindex{Wien@\textbf{Wien}, \emph{A.ADM2}|pw}{ }9. 3. 99.\pend
           \selectlanguage{ngerman}\endnumbering\briefempfaengerindex{Bahr, Hermann@\textsc{Bahr, Hermann}!zzzSchnitzler, Arthur@\emph{von Arthur Schnitzler}!1899-03-091@{9. 3. 1899}|)be}\mylabel{L00903h}  \normalsize

\doendnotes{C}
\bigskip
\vfill

\clearpage

\footnotesize

\lohead{\textsc{register}}

% Definiere theindex-Environment komplett neu ohne reledmac
\makeatletter
\renewenvironment{theindex}{%
  \section*{\indexname}%
  \setlength{\parindent}{0pt}%
  \setlength{\parskip}{0pt plus 0.3pt}%
  \let\item\@idxitem
}{%
  \clearpage
}
\makeatother

\IfFileExists{\jobname-pw.ind}{\input{\jobname-pw.ind}}{}

\end{document}

      