%% latex-leseansicht-vorspann.tex
%% Vorspann für die Leseansicht.
%% Lädt die gemeinsame Datei latex-vorspann.tex mit nicht gesetztem Schalter.

\newif\ifkorrekturansicht
\korrekturansichtfalse

\input{../tex-inputs/latex-vorspann}


\section[Arthur Schnitzler an Hermann Bahr, 9. 3. 1899]{L00903 Arthur Schnitzler an Hermann Bahr, 9. 3. 1899}
\nopagebreak\mylabel{L00903v}
\rehead{ }\normalsize\beginnumbering\briefempfaengerindex{Bahr, Hermann@\textsc{Bahr, Hermann}!zzzSchnitzler, Arthur@\emph{von Arthur Schnitzler}!1899-03-091@{9. 3. 1899}|(be}
\toendnotes[C]{\smallbreak\pagebreak[2]}
\correspDesc{Versand  durch Arthur Schnitzler am 9. 3. 1899 in Wien
\newline{}Erhalt  durch Hermann Bahr im Zeitraum [9. 3. 1899
                  – 13. 3. 1899?] in Wien}\toendnotes[C]{\smallbreak}
\Standort{TMW, HS AM 23336 Ba.}
\physDesc{Brief, 1 Blatt, 3 Seiten, 627 Zeichen
\newline{}Handschrift: schwarze Tinte, deutsche Kurrent
\newline{}Ordnung: 1) Lochung  2) mit Bleistift von unbekannter Hand datiert: »9. 3. 99«}
\buchAbdrucke{\weitereDrucke{1) \emph{9. 3. 1899.} In: Arthur Schnitzler: \emph{The Letters of Arthur Schnitzler to Hermann Bahr}. Edited, annotated, and with an introduction, by Donald G. Daviau. Chapel Hill: \emph{The University of North Carolina Press} 1978, S. 65 (University of North Carolina studies in the Germanic languages
                        and literatures, 89).} \weitereDrucke{2) Hermann Bahr, Arthur Schnitzler: \emph{Briefwechsel, Aufzeichnungen, Dokumente (1891–1931)}. Herausgegeben von Kurt Ifkovits und Martin Anton Müller. Göttingen: \emph{Wallstein} 2018, S. 169.} }\toendnotes[C]{\smallbreak}
\pstart
           \noindent{}{\pb}Lieber Bahr, die Sache{ }ſtimmt nicht. Ich habe dir von \label{K_L00903-1v}\edtext{Anfang an{ }ſowohl geſchrieben}{\lemma{\textnormal{\emph{Anfang … geschrieben}}}\Cendnote{\textnormal{Hier ist Schnitzler ungenau: Er hatte es nicht »vor« der Aufführung angeboten,
                     vgl. XXXX Auszeichnungsfehler: Dokument L00864 nicht gefunden.}}}\label{K_L00903-1} als
               geſagt, dſs ich dir das Stück\pwindex{Schnitzler, Arthur 15.\,5.\,1862 Wien – 21.\,10.\,1931 ebd.@\textsc{Schnitzler, Arthur} (15.\,5.\,1862 Wien – 21.\,10.\,1931 ebd.), \emph{Schriftsteller, Mediziner}!Gefährtin. Schauspiel in einem Akt@\strich\emph{Die Gefährtin. Schauspiel in einem Akt}|pwv}
               erſt \uline{nach} der Première geben kann und will; ja, vor
               etwa 3 Wochen, als ich dich in der Landesgerichtsſtraße\oindex{Wien@\textbf{Wien}!I., Innere Stadt@\textbf{I., Innere Stadt}!Landesgerichtsstraße@\textbf{Landesgerichtsstraße}, \emph{Straße}|pw} begegnete und der \label{K_L00903-2v}\edtext{Aufführgs{\pb}termin}{\lemma{\textnormal{\emph{Aufführgstermin}}}\Cendnote{\textnormal{Dieser war bereits am 1. 3. 1899
                  gewesen.}}}\label{K_L00903-2} bereits feſt\damage{ſtan}d,{ }ſagteſt du{ }ſelbſt, daſs du es erſt im Mai (alſo eine beträchtliche Zeit
               nach der Aufführg) abdrucken wollteſt.\pend
           
\pstart
           Wozu alſo läßt du dich in die von mir von vornherein abgelehnte Discuſſion ein. Es
               war halt eine, na{ }ſagen wir, eine Schlamperei von {\pb}dir; \damage{m}eine Verwunderung iſt{ }ſo gering als mein Gram, und damit Schluſs.\pend
           
\pstart
           Ich grüß dich beſtens.{\\[\baselineskip]}Dein \spacefill\mbox{Arth Sch}\pend
           \leftskip=0em{}
\pstart
           Wien\oindex{Wien@\textbf{Wien}, \emph{Verwaltungsgebiet}|pw}{ }9. 3. 99.\pend
           \selectlanguage{ngerman}\endnumbering\briefempfaengerindex{Bahr, Hermann@\textsc{Bahr, Hermann}!zzzSchnitzler, Arthur@\emph{von Arthur Schnitzler}!1899-03-091@{9. 3. 1899}|)be}\mylabel{L00903h}  \newcommand{\dateiname}{L00903}\newcommand{\titel}{Arthur Schnitzler an Hermann Bahr, 9. 3. 1899}\newcommand{\editorInnen}{Herausgegeben von Martin Anton Müller}%% latex-leseansicht-abspann.tex
%% Abspann für die Leseansicht.
%% Der Schalter \ifkorrekturansicht ist bereits durch den Vorspann gesetzt.

%% latex-abspann.tex
%% Gemeinsamer Abspann für Korrekturansicht und Leseansicht.
%% Setzt den Schalter \ifkorrekturansicht voraus (gesetzt in den
%% einbindenden Dateien latex-korrekturansicht-abspann.tex bzw.
%% latex-leseansicht-abspann.tex).
%% ---------------------------------------------------------------

\normalsize

% Das esempio-Environment wird nur in der Leseansicht benötigt
\ifkorrekturansicht\else
\newenvironment{esempio}[3]%
{
    \vspace{1.5ex}
    \rlap{\underline{#1}}
    \par
    \setlength{\parindent}{0cm}
    \nopagebreak
    \leftskip=#2cm
    \rightskip=#3cm
}
{
    \par
}
\fi

\doendnotes{C}
\bigskip
\vfill

\clearpage

\footnotesize

\ifkorrekturansicht
  \lohead{\textsc{register}}
\fi

% theindex-Environment neu definieren ohne reledmac
\makeatletter
\renewenvironment{theindex}{%
  \ifkorrekturansicht
    \section*{\indexname}%
  \else
    \subsubsection*{Index der erwähnten Entitäten}%
  \fi
  \setlength{\parindent}{0pt}%
  \setlength{\parskip}{0pt plus 0.3pt}%
  \let\item\@idxitem
}{%
  \ifkorrekturansicht\clearpage\fi
}
\makeatother

\IfFileExists{\jobname-pw.ind}{\input{\jobname-pw.ind}}{}

% Quellenangabe nur in der Leseansicht
\ifkorrekturansicht\else
% Fallback-Definitionen, falls die .tex-Datei \titel etc. nicht gesetzt hat
\providecommand{\titel}{}
\providecommand{\editorInnen}{}
\providecommand{\dateiname}{\jobname}

\vspace{3cm}

\vfill

\footnotesize
\textsc{Quelle}: \titel. Herausgegeben von {\editorInnen}. In: \emph{Arthur Schnitzler: Briefwechsel mit Autorinnen und Autoren}.
 Digitale Edition, https://schnitzler-briefe.acdh.oeaw.ac.at/{\dateiname}.html (Stand \today)
\fi

\end{document}


