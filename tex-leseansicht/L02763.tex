%% latex-korrekturansicht-vorspann.tex
%% Vorspann für die Korrekturansicht.
%% Lädt die gemeinsame Datei latex-vorspann.tex mit gesetztem Schalter.

\newif\ifkorrekturansicht
\korrekturansichttrue

\input{../tex-inputs/latex-vorspann}


\section[ Paul Goldmann an Arthur Schnitzler, 13. 1. {[}1896{]}]{L02763 Paul Goldmann an Arthur Schnitzler, 13. 1. {[}1896{]}}
\nopagebreak\mylabel{L02763v}
\rehead{ }\normalsize\beginnumbering\briefempfaengerindex{Schnitzler, Arthur@\textsc{Schnitzler, Arthur}!zzzGoldmann, Paul@\emph{von Paul Goldmann}!1896-01-131@{13. 1. {[}1896{]}}|(be}
\toendnotes[C]{\smallbreak\pagebreak[2]}\Standort{DLA, A:Schnitzler, HS.NZ85.1.3166.}
\physDesc{Brief, 1 Blatt, 1 Seite, 379 Zeichen
\newline{}Handschrift: blaue Tinte, deutsche Kurrent
\newline{}Schnitzler: 1) mit Bleistift das Jahr »96« vermerkt  2) mit rotem Buntstift eine Unterstreichung}\toendnotes[C]{\smallbreak}
\pstart
           {\pb}\textcolor{gray}{\textbf{\textbf{Frankfurter Zeitung\orgindex{Frankfurter Zeitung@Frankfurter Zeitung|pw}}}}\pend
           
\pstart
           \textcolor{gray}{\textbf{(\begin{otherlanguage}{french}Gazette de Francfort\end{otherlanguage}\orgindex{Frankfurter Zeitung@Frankfurter Zeitung|pw}).}}\pend
           
\pstart
           \textcolor{gray}{\textbf{\textbf{\begin{otherlanguage}{french}Fondateur M.\end{otherlanguage}{ }L. Sonnemann\pwindex{Sonnemann, Leopold 1831-10-29 – 1909-10-30@\textsc{Sonnemann, Leopold} (1831-10-29 – 1909-10-30), \emph{Journalist/Journalistin, Herausgeber/Herausgeberin}|pw}.}}}\pend
           
\pstart
           \begin{otherlanguage}{french}\textcolor{gray}{\textbf{Journal\pwindex{Frankfurter Zeitung@\emph{Frankfurter Zeitung}|pwv} politique,
                        financier,}}\end{otherlanguage}\pend
           
\pstart
           \begin{otherlanguage}{french}\textcolor{gray}{\textbf{commercial et littéraire.}}\end{otherlanguage}\pend
           
\pstart
           \begin{otherlanguage}{french}\textcolor{gray}{\textbf{\textbf{Paraissant trois fois par jour.}}}\end{otherlanguage}\pend
           
\pstart
           \begin{otherlanguage}{french}\textcolor{gray}{\textbf{\textbf{Bureau à Paris\oindex{Paris@\textbf{Paris}, \emph{P.PPLC}|pw}:}}}\end{otherlanguage}\hfill \textsc{Paris\oindex{Paris@\textbf{Paris}, \emph{P.PPLC}|pw}}, 13. Januar.\pend
           
\pstart
           \begin{otherlanguage}{french}\textcolor{gray}{\textbf{\textbf{24. Rue Feydeau\oindex{rue Feydeau@\textbf{rue Feydeau}, \emph{Straße (K.STR)}|pw}.}}}\end{otherlanguage}\pend
           
\pstart\center{}Mein lieber Freund,\pend\vspace{0.5em}
\pstart
           Ich leſe eben das \label{K_L02763-1v}\edtext{Referat\pwindex{Schauspielhaus. [Premiere von Liebelei]@\emph{Schauspielhaus. [Premiere von Liebelei]}|pwv}}{\lemma{\textnormal{\emph{Referat}}}\Cendnote{\textnormal{m.\pwindex{Mamroth, Fedor 21.02.1851 – 25.06.1907@\textsc{Mamroth, Fedor} (21.02.1851 – 25.06.1907), \emph{Journalist/Journalistin, Kritiker/Kritikerin}|pwkv} [ = Fedor Mamroth\pwindex{Mamroth, Fedor 21.02.1851 – 25.06.1907@\textsc{Mamroth, Fedor} (21.02.1851 – 25.06.1907), \emph{Journalist/Journalistin, Kritiker/Kritikerin}|pwk}]: \emph{Schauspielhaus}\pwindex{Schauspielhaus. [Premiere von Liebelei]@\emph{Schauspielhaus. [Premiere von Liebelei]}|pwk}. In: \emph{Frankfurter
                        Zeitung}\pwindex{Frankfurter Zeitung@\emph{Frankfurter Zeitung}|pwk}, Jg. 40, Nr. 12, 12. 1. 1896,
                     Zweites Morgenblatt, S. 1. Rezension\pwindex{Schauspielhaus. [Premiere von Liebelei]@\emph{Schauspielhaus. [Premiere von Liebelei]}|pwkv} der Premiere von \emph{Liebelei}\pwindex{Liebelei. Schauspiel in drei Akten@\emph{Liebelei. Schauspiel in drei Akten}|pwk} am \emph{Frankfurter Städtischen
                     Schauspielhaus}XXXX ORGangabe fehlt; siehe A. S.: \emph{Tagebuch}, 11. 1. 1896. }}}\label{K_L02763-1} meines Onkels\pwindex{Mamroth, Fedor 21.02.1851 – 25.06.1907@\textsc{Mamroth, Fedor} (21.02.1851 – 25.06.1907), \emph{Journalist/Journalistin, Kritiker/Kritikerin}|pwv} und finde es wunderſchön. Zwiſchen Dir und ihm ſind
               jetzt hoffentlich alle \label{K_L02763-2v}\edtext{Mißverſtändniſſe}{\lemma{\textnormal{\emph{Mißverſtändniſſe}}}\Cendnote{\textnormal{wohl aufgrund
                  wiederholter Ablehnungen Mamroths\pwindex{Mamroth, Fedor 21.02.1851 – 25.06.1907@\textsc{Mamroth, Fedor} (21.02.1851 – 25.06.1907), \emph{Journalist/Journalistin, Kritiker/Kritikerin}|pwk}, Werke\pwindex{Sohn. Aus den Papieren eines Arztes@\emph{Der Sohn. Aus den Papieren eines Arztes}|pwkv}\pwindex{drei Elixire@\emph{Die drei Elixire}|pwkv}\pwindex{Sterben. Novelle@\emph{Sterben. Novelle}|pwkv}\pwindex{Blumen@\emph{Blumen}|pwkv} von Schnitzler
                  abzudrucken}}}\label{K_L02763-2} beſeitigt. Eine beſſere Einführung in Deutſchland\oindex{Deutschland@\textbf{Deutschland}, \emph{A.PCLI}|pw} konnte man für Dein Stück\pwindex{Liebelei. Schauspiel in drei Akten@\emph{Liebelei. Schauspiel in drei Akten}|pwv} kaum erträumen. Ich beglückwünſche
               Dich innig zu Deinem neuen Erfolge und danke Dir für Deine lieben Grüße aus Frankfurt\oindex{Frankfurt am Main@\textbf{Frankfurt am Main}, \emph{P.PPLA3}|pw}.\pend
           
\pstart
           In Treue {\\[\baselineskip]}Dein {\\[\baselineskip]}\spacefill\mbox{Paul Goldmnn}\pend
           \leftskip=0em{}\selectlanguage{ngerman}\endnumbering\briefempfaengerindex{Schnitzler, Arthur@\textsc{Schnitzler, Arthur}!zzzGoldmann, Paul@\emph{von Paul Goldmann}!1896-01-131@{13. 1. {[}1896{]}}|)be}\mylabel{L02763h}  \normalsize

\doendnotes{C}
\bigskip
\vfill

\clearpage

\footnotesize

\lohead{\textsc{register}}

% Definiere theindex-Environment komplett neu ohne reledmac
\makeatletter
\renewenvironment{theindex}{%
  \section*{\indexname}%
  \setlength{\parindent}{0pt}%
  \setlength{\parskip}{0pt plus 0.3pt}%
  \let\item\@idxitem
}{%
  \clearpage
}
\makeatother

\IfFileExists{\jobname-pw.ind}{\input{\jobname-pw.ind}}{}

\end{document}

      