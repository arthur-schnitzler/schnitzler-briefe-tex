%% latex-leseansicht-vorspann.tex
%% Vorspann für die Leseansicht.
%% Lädt die gemeinsame Datei latex-vorspann.tex mit nicht gesetztem Schalter.

\newif\ifkorrekturansicht
\korrekturansichtfalse

\input{../tex-inputs/latex-vorspann}


\section[Arthur Schnitzler an Albert Ehrenstein, 7. 7. 1909]{L01854 Arthur Schnitzler an Albert Ehrenstein, 7. 7. 1909}
\nopagebreak\mylabel{L01854v}
\rehead{ }\normalsize\beginnumbering\briefempfaengerindex{Ehrenstein, Albert@\textsc{Ehrenstein, Albert}!zzzSchnitzler, Arthur@\emph{von Arthur Schnitzler}!1909-07-071@{7. 7. 1909}|(be}
\toendnotes[C]{\smallbreak\pagebreak[2]}
\correspDesc{Versand  durch Arthur Schnitzler am 7. 7. 1909 in Edlach
\newline{}Erhalt  durch Albert Ehrenstein im Zeitraum [7. 7. 1909
                  – 11. 7. 1909?] \textbf{Ort fehlend} }\toendnotes[C]{\smallbreak}
\Standort{Jerusalem, The National Library of Israel, ARC. Ms. Var. 306 1 118.}
\physDesc{Brief, 1 Blatt, 4 Seiten, 874 Zeichen
\newline{}Handschrift: Bleistift, lateinische Kurrent}
\pstart
           {\pb}\textcolor{gray}{\textbf{Dr. Arthur Schnitzler}}\hfill Edlach\oindex{Edlach@\textbf{Edlach}|pw}{ }7/7 09\pend
           
\pstart
           \textcolor{gray}{\textbf{Wien XVIII. Spoettelgasse 7\oindex{Wien@\textbf{Wien}!XVIII., Währing@\textbf{XVIII., Währing}!Edmund-Weiß-Gasse 7@\textbf{Edmund-Weiß-Gasse 7}, \emph{Wohngebäude}|pw}.}}\hfill Edlacher Hof\oindex{Hotel Edlacherhof@\textbf{Hotel Edlacherhof}, \emph{Hotel}|pw}\pend
           
\pstart{}Lieber Herr Ehrenstein,\pend\vspace{0.5em}
\pstart
           die Manuscripte liegen in meiner Wohnung zum Abholen für Sie (unter Ihrem Namen)
               bereit.\pend
           
\pstart
           Im Herbst sprechen wir über die Sachen, we{\geminationn}s Ihnen recht
               ist. Für heute nur so viel, {\pb}dass ich einen äußern Erfolg gerade
               dieser letzten Sachen, d. h. insbesondere eine Annahme bei Zeit\pwindex{Zeit@\emph{Die Zeit}|pw} oder Presse\orgindex{Neue Freie Presse@Neue Freie Presse|pw} für nicht
               wahrscheinlich halte. Mit Auernh.\pwindex{Auernheimer, Raoul 15.\,4.\,1876 Wien – 6.\,1.\,1948 Oakland@\textsc{Auernheimer, Raoul} (15.\,4.\,1876 Wien – 6.\,1.\,1948 Oakland), \emph{Schriftsteller, Journalist, Kritiker}|pw}, der jetzt
               hier ist, will ich übrigens im allgemeinen über Sie reden, we{\geminationn} sie nichts dagegen haben. Auf dieser Bahn scheint mir
               ja nun {\pb}allerdings Ihre Zukunft nicht zu liegen (ich meine die
                  Zeit\pwindex{Zeit@\emph{Die Zeit}|pw} und Presse\orgindex{Neue Freie Presse@Neue Freie Presse|pw}-Bahn) Ihre Auffassung, dass \introOben{}selbst\introOben{} die
               Veröffentlichung einer oder der andern Arbeit in einer dieser Blätter Ihre Position
               bei den Professoren zu Gunsten der Prüfung beeinflussen könnte, theil ich nicht. Sie
               werden Ihre {\pb}Examen sicher bestehen, auch so.\pend
           
\pstart
           – Auf Wiedersehen und beste Grüße. Ihr ergebener{\\[\baselineskip]}\spacefill\mbox{A. S.}\pend
           \leftskip=0em{}\selectlanguage{ngerman}\endnumbering\briefempfaengerindex{Ehrenstein, Albert@\textsc{Ehrenstein, Albert}!zzzSchnitzler, Arthur@\emph{von Arthur Schnitzler}!1909-07-071@{7. 7. 1909}|)be}\mylabel{L01854h}  \newcommand{\dateiname}{L01854}\newcommand{\titel}{Arthur Schnitzler an Albert Ehrenstein, 7. 7. 1909}\newcommand{\editorInnen}{Martin Anton Müller und Gerd-Hermann Susen}%% latex-leseansicht-abspann.tex
%% Abspann für die Leseansicht.
%% Der Schalter \ifkorrekturansicht ist bereits durch den Vorspann gesetzt.

%% latex-abspann.tex
%% Gemeinsamer Abspann für Korrekturansicht und Leseansicht.
%% Setzt den Schalter \ifkorrekturansicht voraus (gesetzt in den
%% einbindenden Dateien latex-korrekturansicht-abspann.tex bzw.
%% latex-leseansicht-abspann.tex).
%% ---------------------------------------------------------------

\normalsize

% Das esempio-Environment wird nur in der Leseansicht benötigt
\ifkorrekturansicht\else
\newenvironment{esempio}[3]%
{
    \vspace{1.5ex}
    \rlap{\underline{#1}}
    \par
    \setlength{\parindent}{0cm}
    \nopagebreak
    \leftskip=#2cm
    \rightskip=#3cm
}
{
    \par
}
\fi

\doendnotes{C}
\bigskip
\vfill

\clearpage

\footnotesize

\ifkorrekturansicht
  \lohead{\textsc{register}}
\fi

% theindex-Environment neu definieren ohne reledmac
\makeatletter
\renewenvironment{theindex}{%
  \ifkorrekturansicht
    \section*{\indexname}%
  \else
    \subsubsection*{Index der erwähnten Entitäten}%
  \fi
  \setlength{\parindent}{0pt}%
  \setlength{\parskip}{0pt plus 0.3pt}%
  \let\item\@idxitem
}{%
  \ifkorrekturansicht\clearpage\fi
}
\makeatother

\IfFileExists{\jobname-pw.ind}{\input{\jobname-pw.ind}}{}

% Quellenangabe nur in der Leseansicht
\ifkorrekturansicht\else
% Fallback-Definitionen, falls die .tex-Datei \titel etc. nicht gesetzt hat
\providecommand{\titel}{}
\providecommand{\editorInnen}{}
\providecommand{\dateiname}{\jobname}

\vspace{3cm}

\vfill

\footnotesize
\textsc{Quelle}: \titel. Herausgegeben von {\editorInnen}. In: \emph{Arthur Schnitzler: Briefwechsel mit Autorinnen und Autoren}.
 Digitale Edition, https://schnitzler-briefe.acdh.oeaw.ac.at/{\dateiname}.html (Stand \today)
\fi

\end{document}


