%% latex-leseansicht-vorspann.tex
%% Vorspann für die Leseansicht.
%% Lädt die gemeinsame Datei latex-vorspann.tex mit nicht gesetztem Schalter.

\newif\ifkorrekturansicht
\korrekturansichtfalse

\input{../tex-inputs/latex-vorspann}


\section[Richard Beer-Hofmann an Arthur Schnitzler, {[}30. 7. 1894{]}]{L00360 Richard Beer-Hofmann an Arthur Schnitzler, {[}30. 7. 1894{]}}
\nopagebreak\mylabel{L00360v}
\rehead{ }\normalsize\beginnumbering\briefempfaengerindex{Schnitzler, Arthur@\textsc{Schnitzler, Arthur}!zzzBeer-Hofmann, Richard@\emph{von Richard Beer-Hofmann}!1894-07-302@{{[}30. 7. 1894{]}}|(be}
\toendnotes[C]{\smallbreak\pagebreak[2]}
\correspDesc{Versand  durch Richard Beer-Hofmann am [30. 7. 1894] in Bad Ischl
\newline{}Erhalt  durch Arthur Schnitzler am [30. 7. 1894] in Wien}\toendnotes[C]{\smallbreak}
\Standort{CUL, Schnitzler, B 8.}
\physDesc{Telegramm, 166 Zeichen
\newline{}maschinell
\newline{}Versand: Stempel des Telegrafenbeamten: »Edmund Winter\pwindex{Winter, Edmund @\textsc{Winter, Edmund}, \emph{Postbeamter}|pw}« 
\newline{}Schnitzler: mit Bleistift datiert: »30{[}/7 94{]}« und nummeriert: »30« 
\newline{}Ordnung: beschnitten }
\buchAbdrucke{\weitereDrucke{Arthur Schnitzler, Richard Beer-Hofmann: \emph{Briefwechsel 1891–1931}. Herausgegeben von Konstanze Fliedl. Wien, Zürich: \emph{Europaverlag} 1992, S. 58.} }\toendnotes[C]{\smallbreak}
\pstart
           {\pb}wien\oindex{Wien@\textbf{Wien}, \emph{Verwaltungsgebiet}|pw} fr \label{T_L00360-1v}\edtext{ischl\oindex{Bad Ischl@\textbf{Bad Ischl}|pw}}{\lemma{\textnormal{\emph{ischl}}}\Cendnote{\textnormal{Das »i« wurde von unbekannter
                     Hand mit Bleistift ergänzt.}}}\label{T_L00360-1} 5806 28 12 20 n\pend
           \vspace{0.5em}
\pstart
           wir sind am zwejten august{ }vormittag in salzburg oesterreichischer
                  hof\oindex{Österreichischer Hof@\textbf{Österreichischer Hof}, \emph{Hotel}|pw} bitte es dem \label{K_L00360-1v}\edtext{suendentraum}{\lemma{\textnormal{\emph{suendentraum}}}\Cendnote{\textnormal{unklare Anspielung;
                  eventuell auf Richard Specht\pwindex{Specht, Richard 7.\,12.\,1870 Wien – 18.\,3.\,1932 ebd.@\textsc{Specht, Richard} (7.\,12.\,1870 Wien – 18.\,3.\,1932 ebd.), \emph{Schriftsteller, Journalist, Kritiker}|pwk}, dessen
                  dramatische Dichtung \emph{Sündentraum}\pwindex{Specht, Richard 7.\,12.\,1870 Wien – 18.\,3.\,1932 ebd.@\textsc{Specht, Richard} (7.\,12.\,1870 Wien – 18.\,3.\,1932 ebd.), \emph{Schriftsteller, Journalist, Kritiker}!Sündentraum@\strich\emph{Sündentraum}|pwk}{ }1892 erschienen war, oder auf Adele
                     Sandrock\pwindex{Sandrock, Adele 19.\,8.\,1863 Rotterdam – 30.\,8.\,1937 Berlin@\textsc{Sandrock, Adele} (19.\,8.\,1863 Rotterdam – 30.\,8.\,1937 Berlin), \emph{Schauspielerin}|pwk}?}}}\label{K_L00360-1} der in wien\oindex{Wien@\textbf{Wien}, \emph{Verwaltungsgebiet}|pw} ist nicht
               zu sagen herzlichst \spacefill\mbox{= richard +}\pend
           \selectlanguage{ngerman}\endnumbering\briefempfaengerindex{Schnitzler, Arthur@\textsc{Schnitzler, Arthur}!zzzBeer-Hofmann, Richard@\emph{von Richard Beer-Hofmann}!1894-07-302@{{[}30. 7. 1894{]}}|)be}\mylabel{L00360h}  \newcommand{\dateiname}{L00360}\newcommand{\titel}{Richard Beer-Hofmann an Arthur Schnitzler, [30. 7. 1894]}\newcommand{\editorInnen}{Martin Anton Müller und Gerd-Hermann Susen}%% latex-leseansicht-abspann.tex
%% Abspann für die Leseansicht.
%% Der Schalter \ifkorrekturansicht ist bereits durch den Vorspann gesetzt.

%% latex-abspann.tex
%% Gemeinsamer Abspann für Korrekturansicht und Leseansicht.
%% Setzt den Schalter \ifkorrekturansicht voraus (gesetzt in den
%% einbindenden Dateien latex-korrekturansicht-abspann.tex bzw.
%% latex-leseansicht-abspann.tex).
%% ---------------------------------------------------------------

\normalsize

% Das esempio-Environment wird nur in der Leseansicht benötigt
\ifkorrekturansicht\else
\newenvironment{esempio}[3]%
{
    \vspace{1.5ex}
    \rlap{\underline{#1}}
    \par
    \setlength{\parindent}{0cm}
    \nopagebreak
    \leftskip=#2cm
    \rightskip=#3cm
}
{
    \par
}
\fi

\doendnotes{C}
\bigskip
\vfill

\clearpage

\footnotesize

\ifkorrekturansicht
  \lohead{\textsc{register}}
\fi

% theindex-Environment neu definieren ohne reledmac
\makeatletter
\renewenvironment{theindex}{%
  \ifkorrekturansicht
    \section*{\indexname}%
  \else
    \subsubsection*{Index der erwähnten Entitäten}%
  \fi
  \setlength{\parindent}{0pt}%
  \setlength{\parskip}{0pt plus 0.3pt}%
  \let\item\@idxitem
}{%
  \ifkorrekturansicht\clearpage\fi
}
\makeatother

\IfFileExists{\jobname-pw.ind}{\input{\jobname-pw.ind}}{}

% Quellenangabe nur in der Leseansicht
\ifkorrekturansicht\else
% Fallback-Definitionen, falls die .tex-Datei \titel etc. nicht gesetzt hat
\providecommand{\titel}{}
\providecommand{\editorInnen}{}
\providecommand{\dateiname}{\jobname}

\vspace{3cm}

\vfill

\footnotesize
\textsc{Quelle}: \titel. Herausgegeben von {\editorInnen}. In: \emph{Arthur Schnitzler: Briefwechsel mit Autorinnen und Autoren}.
 Digitale Edition, https://schnitzler-briefe.acdh.oeaw.ac.at/{\dateiname}.html (Stand \today)
\fi

\end{document}


