%% latex-leseansicht-vorspann.tex
%% Vorspann für die Leseansicht.
%% Lädt die gemeinsame Datei latex-vorspann.tex mit nicht gesetztem Schalter.

\newif\ifkorrekturansicht
\korrekturansichtfalse

\input{../tex-inputs/latex-vorspann}


         
         \newcommand{\erwaehntePersonen}{Personen: Hermann Bahr, Anna Bahr-Mildenburg, Richard Beer-Hofmann, Olga Schnitzler}
         \newcommand{\erwaehnteInstitutionen}{}
         \newcommand{\erwaehnteOrte}{Orte: Bayern, Wien}
         \newcommand{\erwaehnteWerke}{
               \section[Hugo von Hofmannsthal an Arthur Schnitzler, {[}7. 8. 1905{]}]{ Hugo von Hofmannsthal an Arthur Schnitzler, {[}7. 8. 1905{]}}\nopagebreak\mylabel{v}\rehead{ }\begin{ledgroupsized}[t]{13cm}\normalsize\beginnumbering \toendnotes[C]{\smallbreak\pagebreak[2]} \Standort{CUL, Schnitzler, B 43b/1.}
\physDesc{Brief, 1 Blatt, 4 Seiten
\newline{}Handschrift: schwarze Tinte, deutsche Kurrent
\newline{}Schnitzler: mit Bleistift datiert: »7/8 905« \newline{}Ordnung: mit Bleistift von unbekannter Hand nummeriert:
                                        »257 257a« }\buchAbdrucke{\weitereDrucke{1) Hugo von Hofmannsthal, Arthur Schnitzler: \emph{Briefwechsel}. Hg. Therese Nickl und Heinrich Schnitzler. Frankfurt am Main: \emph{S. Fischer} 1964, S. 212.} \weitereDrucke{2) Hermann Bahr, Arthur Schnitzler: \emph{Briefwechsel, Aufzeichnungen, Dokumente
                                (1891–1931)}. Hg. Kurt Ifkovits und Martin Anton Müller. Göttingen: \emph{Wallstein} 2018, S. 349.} }\toendnotes[C]{\smallbreak}\pstart
           \raggedleft{}{\pb}Montag früh\pend
           \pstart{}mein lieber Arthur,\pend\pstart
           wir freuen uns ja ſo ſehr, Euch\pwindex{Schnitzler, Olga 17.01.1882 – 13.01.1970@\textsc{Schnitzler, Olga} (17.01.1882 – 13.01.1970), \emph{Schauspielerin, Sängerin}|pwv}{ }Freitag hier zu ſehen, aber ich will Ihnen doch ſagen – um es durch
                    Ausſprechen loszuwerden, daſs mich dies Hinausſchieben um eine Woche heftig,
                    vielleicht unverhältnismäßig heftig verſtimmt hat. \label{OL450-1v}\label{OL450-1h}Sie können allerdings nicht wiſſen, {\pb}daſs ich aus gewiſſen
                    Arbeitsgründen Sonntag{ }ſchon wieder abreiſe und man ſich daher knapp
                    einmal ſehen wird, in Monaten – aber davon abgeſehen, ganz an und für ſich
                    betrachtet: man ſitzt auf der elenden Waffenübung, freut ſich ſo ſehr auf die
                    paar Menſchen {\pb}die man dann
                    wiederſehen kann – Richard\pwindex{Beer-Hofmann, Richard 1866-07-11 – 1945-09-26@\textsc{Beer-Hofmann, Richard} (1866-07-11 – 1945-09-26), \emph{Schriftsteller}|pw} kann ich nicht
                    rechnen, bis er wieder normaler und geſünder wird, Bahr\pwindex{Bahr, Hermann 19.07.1863 – 15.01.1934@\textsc{Bahr, Hermann} (19.07.1863 – 15.01.1934), \emph{Schriftsteller, Kritiker}|pw} iſt \label{K_L01539_1v}\edtext{verſchollen}{\lemma{\textnormal{\emph{verſchollen}}}\Cendnote{\textnormal{Er urlaubte mit Anna von Mildenburg\pwindex{Bahr-Mildenburg, Anna 29.11.1872 – 27.01.1947@\textsc{Bahr-Mildenburg, Anna} (29.11.1872 – 27.01.1947), \emph{Sängerin}|pwk} in Bayern\oindex{Bayern@\textbf{Bayern}|pwk}.}}}\label{K_L01539_1h} – kommt dann zurück, ſehnt ſich ſehr,
                    in andere Dinge wieder hineinzuko{\geminationm}en (Sie ahnen
                    nicht, wie einem ſolche vier Wochen den Kopf verderben können), telegrafirt {\pb}in der erſten halben Stunde,
                    hofft doch ein bischen, daſs der Andere auch irgend etwas von dieſer Ungeduld
                    hat, hofft in dieſem Fall, es wird heißen: übermorgen kommen wir zu Euch und
                    dann müſſen Sie zu mir kommen ich leſe Ihnen was vor {\dots} und dann beko{\geminationm}t man eine Antwort, aus der man ſo
                    ſehr ſpürt, daſs der andere ſich nicht will aus seiner »Einteilung« bringen
                    laſſen. Ich bin etwas \label{T_L01539_1v}\edtext{traurig
                    darüber. Wahrſcheinlich iſt das ganz dumm, aber es iſt vielleicht das Reſultat
                    von 200 kleinen Dingen.}{\lemma{\textnormal{\emph{traurig … Dingen.}}}\Cendnote{\textnormal{bis zum Schluss
                        in zwei Zeilen entlang des Mittelfalzes auf der vierten und ersten Seite}}}\label{T_L01539_1h}\pend
           \pstart Ihr \spacefill\mbox{Hugo.}\pend{}
         
         \endnumbering\mylabel{h}\end{ledgroupsized}  \newcommand{\dateiname}{L01539}\newcommand{\titel}{Hugo von Hofmannsthal an Arthur Schnitzler, [7. 8. 1905]}\newcommand{\editorInnen}{ Martin Anton Müller und Gerd-Hermann Susen}%% latex-leseansicht-abspann.tex
%% Abspann für die Leseansicht.
%% Der Schalter \ifkorrekturansicht ist bereits durch den Vorspann gesetzt.

%% latex-abspann.tex
%% Gemeinsamer Abspann für Korrekturansicht und Leseansicht.
%% Setzt den Schalter \ifkorrekturansicht voraus (gesetzt in den
%% einbindenden Dateien latex-korrekturansicht-abspann.tex bzw.
%% latex-leseansicht-abspann.tex).
%% ---------------------------------------------------------------

\normalsize

% Das esempio-Environment wird nur in der Leseansicht benötigt
\ifkorrekturansicht\else
\newenvironment{esempio}[3]%
{
    \vspace{1.5ex}
    \rlap{\underline{#1}}
    \par
    \setlength{\parindent}{0cm}
    \nopagebreak
    \leftskip=#2cm
    \rightskip=#3cm
}
{
    \par
}
\fi

\doendnotes{C}
\bigskip
\vfill

\clearpage

\footnotesize

\ifkorrekturansicht
  \lohead{\textsc{register}}
\fi

% theindex-Environment neu definieren ohne reledmac
\makeatletter
\renewenvironment{theindex}{%
  \ifkorrekturansicht
    \section*{\indexname}%
  \else
    \subsubsection*{Index der erwähnten Entitäten}%
  \fi
  \setlength{\parindent}{0pt}%
  \setlength{\parskip}{0pt plus 0.3pt}%
  \let\item\@idxitem
}{%
  \ifkorrekturansicht\clearpage\fi
}
\makeatother

\IfFileExists{\jobname-pw.ind}{\input{\jobname-pw.ind}}{}

% Quellenangabe nur in der Leseansicht
\ifkorrekturansicht\else
% Fallback-Definitionen, falls die .tex-Datei \titel etc. nicht gesetzt hat
\providecommand{\titel}{}
\providecommand{\editorInnen}{}
\providecommand{\dateiname}{\jobname}

\vspace{3cm}

\vfill

\footnotesize
\textsc{Quelle}: \titel. Herausgegeben von {\editorInnen}. In: \emph{Arthur Schnitzler: Briefwechsel mit Autorinnen und Autoren}.
 Digitale Edition, https://schnitzler-briefe.acdh.oeaw.ac.at/{\dateiname}.html (Stand \today)
\fi

\end{document}


      