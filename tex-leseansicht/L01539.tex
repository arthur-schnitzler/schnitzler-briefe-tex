%% latex-leseansicht-vorspann.tex
%% Vorspann für die Leseansicht.
%% Lädt die gemeinsame Datei latex-vorspann.tex mit nicht gesetztem Schalter.

\newif\ifkorrekturansicht
\korrekturansichtfalse

\input{../tex-inputs/latex-vorspann}


\section[Hugo von Hofmannsthal an Arthur Schnitzler, {{[}}7. 8. 1905{{]}}]{L01539 Hugo von Hofmannsthal an Arthur Schnitzler, {[}7. 8. 1905{]}}
\nopagebreak\mylabel{L01539v}
\rehead{ }\normalsize\beginnumbering\briefempfaengerindex{Schnitzler, Arthur@\textsc{Schnitzler, Arthur}!zzzHofmannsthal, Hugo von@\emph{von Hugo von Hofmannsthal}!1905-08-071@{{[}7. 8. 1905{]}}|(be}
\toendnotes[C]{\smallbreak\pagebreak[2]}
\correspDesc{Versand  durch Hugo von Hofmannsthal am [7. 8. 1905] in Wien
\newline{}Erhalt  durch Arthur Schnitzler im Zeitraum [7. 8. 1905
                  – 11. 8. 1905?] in Wien}\toendnotes[C]{\smallbreak}
\Standort{CUL, Schnitzler, B 43b/1.}
\physDesc{Brief, 1 Blatt, 4 Seiten, 1330 Zeichen
\newline{}Handschrift: schwarze Tinte, deutsche Kurrent
\newline{}Schnitzler: mit Bleistift datiert: »7/8 905« 
\newline{}Ordnung: mit Bleistift von unbekannter Hand nummeriert: »257
                                    257a« }
\buchAbdrucke{\weitereDrucke{1) Hugo von Hofmannsthal, Arthur Schnitzler: \emph{Briefwechsel}. Herausgegeben von Therese Nickl und Heinrich Schnitzler. Frankfurt am Main: \emph{S. Fischer} 1964, S. 212.} \weitereDrucke{2) Hermann Bahr, Arthur Schnitzler: \emph{Briefwechsel, Aufzeichnungen, Dokumente (1891–1931)}. Herausgegeben von Kurt Ifkovits und Martin Anton Müller. Göttingen: \emph{Wallstein} 2018, S. 349.} }\toendnotes[C]{\smallbreak}
\pstart
           \raggedleft{}{\pb}Montag früh\pend
           
\pstart{}mein lieber Arthur,\pend\vspace{0.5em}
\pstart
           wir freuen uns ja{ }ſo{ }ſehr, Euch\pwindex{Schnitzler, Olga 17.\,1.\,1882 Wien – 13.\,1.\,1970 Lugano@\textsc{Schnitzler, Olga} (17.\,1.\,1882 Wien – 13.\,1.\,1970 Lugano), \emph{Schauspielerin, Sängerin}|pwv}{ }Freitag hier zu{ }ſehen, aber ich will Ihnen doch{ }ſagen – um es durch
               Ausſprechen loszuwerden, daſs mich dies Hinausſchieben um eine Woche heftig,
               vielleicht unverhältnismäßig heftig verſtimmt hat. \label{OL450-1v}\label{OL450-1h}Sie können allerdings nicht wiſſen, {\pb}daſs ich aus gewiſſen
               Arbeitsgründen Sonntag{ }ſchon wieder abreiſe und man{ }ſich daher knapp
               einmal{ }ſehen wird, in Monaten – aber davon abgeſehen, ganz an und für{ }ſich
               betrachtet: man{ }ſitzt auf der elenden Waffenübung, freut{ }ſich{ }ſo{ }ſehr auf die paar
               Menſchen {\pb}die man dann wiederſehen
               kann – Richard\pwindex{Beer-Hofmann, Richard 11.\,7.\,1866 Wien – 26.\,9.\,1945 New York City@\textsc{Beer-Hofmann, Richard} (11.\,7.\,1866 Wien – 26.\,9.\,1945 New York City), \emph{Schriftsteller}|pw} kann ich nicht rechnen, bis er
               wieder normaler und geſünder wird, Bahr\pwindex{Bahr, Hermann 19.\,7.\,1863 Linz – 15.\,1.\,1934 München@\textsc{Bahr, Hermann} (19.\,7.\,1863 Linz – 15.\,1.\,1934 München), \emph{Schriftsteller, Kritiker}|pw} iſt
               \label{K_L01539-1v}\edtext{verſchollen}{\lemma{\textnormal{\emph{verschollen}}}\Cendnote{\textnormal{Bahr\pwindex{Bahr, Hermann 19.\,7.\,1863 Linz – 15.\,1.\,1934 München@\textsc{Bahr, Hermann} (19.\,7.\,1863 Linz – 15.\,1.\,1934 München), \emph{Schriftsteller, Kritiker}|pwk} urlaubte mit Anna von Mildenburg\pwindex{Bahr-Mildenburg, Anna 29.\,11.\,1872 Wien – 27.\,1.\,1947 ebd.@\textsc{Bahr-Mildenburg, Anna} (29.\,11.\,1872 Wien – 27.\,1.\,1947 ebd.), \emph{Sängerin}|pwk} in Bayern\oindex{Bayern@\textbf{Bayern}, \emph{Land}|pwk}.}}}\label{K_L01539-1} – kommt dann zurück,{ }ſehnt{ }ſich{ }ſehr, in andere Dinge
               wieder hineinzuko{\geminationm}en (Sie ahnen nicht, wie einem{ }ſolche
               vier Wochen den Kopf verderben können), telegrafirt {\pb}in der erſten halben Stunde, hofft
               doch ein bischen, daſs der Andere auch irgend etwas von dieſer Ungeduld hat, hofft in
               dieſem Fall, es wird heißen: übermorgen kommen wir zu Euch und dann müſſen Sie zu mir
               kommen ich leſe Ihnen was vor {\dots} und dann beko{\geminationm}t man eine Antwort, aus der man{ }ſo{ }ſehr{ }ſpürt, daſs der
               andere{ }ſich nicht will aus seiner »Einteilung« bringen laſſen. Ich bin etwas \label{T_L01539-1v}\edtext{traurig darüber. Wahrſcheinlich iſt das ganz
               dumm, aber es iſt vielleicht das Reſultat von 200 kleinen Dingen.}{\lemma{\textnormal{\emph{traurig … Dingen.}}}\Cendnote{\textnormal{bis zum Schluss in zwei Zeilen entlang des
                  Mittelfalzes auf der vierten und ersten Seite}}}\label{T_L01539-1}\pend
           \pstart Ihr \spacefill\mbox{Hugo.}\pend{}\selectlanguage{ngerman}\endnumbering\briefempfaengerindex{Schnitzler, Arthur@\textsc{Schnitzler, Arthur}!zzzHofmannsthal, Hugo von@\emph{von Hugo von Hofmannsthal}!1905-08-071@{{[}7. 8. 1905{]}}|)be}\mylabel{L01539h}  \newcommand{\dateiname}{L01539}\newcommand{\titel}{Hugo von Hofmannsthal an Arthur Schnitzler, [7. 8. 1905]}\newcommand{\editorInnen}{Herausgegeben von Martin Anton Müller}%% latex-leseansicht-abspann.tex
%% Abspann für die Leseansicht.
%% Der Schalter \ifkorrekturansicht ist bereits durch den Vorspann gesetzt.

%% latex-abspann.tex
%% Gemeinsamer Abspann für Korrekturansicht und Leseansicht.
%% Setzt den Schalter \ifkorrekturansicht voraus (gesetzt in den
%% einbindenden Dateien latex-korrekturansicht-abspann.tex bzw.
%% latex-leseansicht-abspann.tex).
%% ---------------------------------------------------------------

\normalsize

% Das esempio-Environment wird nur in der Leseansicht benötigt
\ifkorrekturansicht\else
\newenvironment{esempio}[3]%
{
    \vspace{1.5ex}
    \rlap{\underline{#1}}
    \par
    \setlength{\parindent}{0cm}
    \nopagebreak
    \leftskip=#2cm
    \rightskip=#3cm
}
{
    \par
}
\fi

\doendnotes{C}
\bigskip
\vfill

\clearpage

\footnotesize

\ifkorrekturansicht
  \lohead{\textsc{register}}
\fi

% theindex-Environment neu definieren ohne reledmac
\makeatletter
\renewenvironment{theindex}{%
  \ifkorrekturansicht
    \section*{\indexname}%
  \else
    \subsubsection*{Index der erwähnten Entitäten}%
  \fi
  \setlength{\parindent}{0pt}%
  \setlength{\parskip}{0pt plus 0.3pt}%
  \let\item\@idxitem
}{%
  \ifkorrekturansicht\clearpage\fi
}
\makeatother

\IfFileExists{\jobname-pw.ind}{\input{\jobname-pw.ind}}{}

% Quellenangabe nur in der Leseansicht
\ifkorrekturansicht\else
% Fallback-Definitionen, falls die .tex-Datei \titel etc. nicht gesetzt hat
\providecommand{\titel}{}
\providecommand{\editorInnen}{}
\providecommand{\dateiname}{\jobname}

\vspace{3cm}

\vfill

\footnotesize
\textsc{Quelle}: \titel. Herausgegeben von {\editorInnen}. In: \emph{Arthur Schnitzler: Briefwechsel mit Autorinnen und Autoren}.
 Digitale Edition, https://schnitzler-briefe.acdh.oeaw.ac.at/{\dateiname}.html (Stand \today)
\fi

\end{document}


