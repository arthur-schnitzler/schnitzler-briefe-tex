\input{../tex-inputs/latex-pdf-vorspann}
\begin{center}
            \textcolor{red}{ENTWURF. ENTZIFFERUNG NOCH NICHT KORREKTURGELESEN}
                      \end{center}
            
               \section[Arthur Schnitzler an Richard Beer-Hofmann, 17. 7. 1896]{ Arthur Schnitzler an Richard Beer-Hofmann,
               17. 7. 1896}\nopagebreak\mylabel{v}\rehead{ }\begin{ledgroupsized}[t]{13cm}\normalsize\beginnumbering\briefempfaengerindex{Beer-Hofmann, Richard@\textsc{Beer-Hofmann, Richard}!zzzSchnitzler, Arthur@\emph{von Arthur Schnitzler}!1896-07-181@{18. 7. 1896}|(be} \toendnotes[C]{\smallbreak\pagebreak[2]} \Standort{YCGL, MSS 31.}
\physDesc{Postkarte
\newline{}Handschrift: Bleistift, deutsche Kurrent\newline{}Versand: 1) Stempel: »\nobreak{}\oindex{Tromsø@\textbf{Tromsø}|pwk}Tromsø, 18. VII. 96\nobreak{}«.  2) Stempel: »\nobreak{}\oindex{Kopenhagen@\textbf{Kopenhagen}|pwk}Kjobenhavn, 23. 7. 96, 50M8\nobreak{}«. }\pstart{}{\pb}\textsc{Dr. Richard
                     Beer-Hofmann}\pend{}\pstart{}\textsc{Kopenhagen}\oindex{Kopenhagen@\textbf{Kopenhagen}|pw}\pend{}\pstart{}\textsc{post rest}\pend{}\pstart{}\textsc{Dänemark\oindex{Daenemark@\textbf{Dänemark}|pw}}\pend{}{\bigskip}\pstart
           \noindent{}{\pb}Mein lieber Richard, ich
               nehme an Sie beko{\geminationm}en dieſe Karte am 24. Da
               ſchreiben Sie mir \uline{gleich} nach \textsc{Stockholm}\oindex{Stockholm@\textbf{Stockholm}|pw}. (\textsc{post rest} natürlich) Ich werde wahrſcheinlich
                  24. 25. 26. in \textsc{Krist.}\oindex{Oslo@\textbf{Oslo}|pw}{ }ſein, dann bis etwa
                  30 od 31{ }\textsc{Stockholm}\oindex{Stockholm@\textbf{Stockholm}|pw}. Es wäre
               wunderſchön, wenn Sie doch wenigſtens nach \textsc{Stockh}\oindex{Stockholm@\textbf{Stockholm}|pw}. hinüberko{\geminationm}en \substVorne{}\textsuperscript{w}\substDazwischen{}mö\substHinten{}chten. Oder nach \textsc{Goetheborg}\oindex{Goeteborg@\textbf{Göteborg}|pw} mir entgegen. Überlegen Sie ſichs. Bitte laſſen
               Sie mich nicht ohne Nachricht. –\pend
           \pstart
           Herzlich der Ihre{\\[\baselineskip]}\spacefill\mbox{ArthSch}\pend
           \leftskip=0em{}\pstart
           \noindent{}an Bord der \textsc{Sig Jarl}{ }17/7 96.\pend
           \endnumbering\briefempfaengerindex{Beer-Hofmann, Richard@\textsc{Beer-Hofmann, Richard}!zzzSchnitzler, Arthur@\emph{von Arthur Schnitzler}!1896-07-181@{18. 7. 1896}|)be}\mylabel{h}\end{ledgroupsized}  \newcommand{\dateiname}{L00565}\newcommand{\titel}{Arthur Schnitzler an Richard Beer-Hofmann, 17. 7. 1896}\newcommand{\editorInnen}{Martin Anton Müller und Gerd-Hermann Susen}\input{../tex-inputs/latex-pdf-abspann}
      