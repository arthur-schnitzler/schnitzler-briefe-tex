%% latex-leseansicht-vorspann.tex
%% Vorspann für die Leseansicht.
%% Lädt die gemeinsame Datei latex-vorspann.tex mit nicht gesetztem Schalter.

\newif\ifkorrekturansicht
\korrekturansichtfalse

\input{../tex-inputs/latex-vorspann}


\section[Arthur Schnitzler an Richard Beer-Hofmann, 17. 7. 1896]{L00565 Arthur Schnitzler an Richard Beer-Hofmann, 17. 7. 1896}
\nopagebreak\mylabel{L00565v}
\rehead{ }\normalsize\beginnumbering\briefempfaengerindex{Beer-Hofmann, Richard@\textsc{Beer-Hofmann, Richard}!zzzSchnitzler, Arthur@\emph{von Arthur Schnitzler}!1896-07-171@{17. 7. 1896}|(be}
\toendnotes[C]{\smallbreak\pagebreak[2]}
\correspDesc{Versand  durch Arthur Schnitzler am 17. 7. 1896 \textbf{Ort fehlend} 
\newline{}Übermittlung  am 18. 7. 1896 in Tromsø
\newline{}Übermittlung  am 23. 7. 1896 in Kopenhagen
\newline{}Erhalt  durch Richard Beer-Hofmann am 26. 7. 1896 in Kopenhagen}\toendnotes[C]{\smallbreak}
\Standort{YCGL, MSS 31.}
\physDesc{Postkarte, 487 Zeichen
\newline{}Handschrift: Bleistift, deutsche Kurrent
\newline{}Versand: 1) Stempel: »\nobreak{}\oindex{Tromsø@\textbf{Tromsø}|pwk}Tromsø, 18. VII. 96\nobreak{}«.   2) Stempel: »\nobreak{}\oindex{Kopenhagen@\textbf{Kopenhagen}, \emph{Hauptstadt}|pwk}Kjobenhavn, 23. 7. 96, 50M8\nobreak{}«. }\pstart{}\textsc{{\pb}Dr. Richard Beer-Hofmann}\pend{}\pstart{}\textsc{Kopenhagen\oindex{Kopenhagen@\textbf{Kopenhagen}, \emph{Hauptstadt}|pw}}\pend{}\pstart{}\textsc{post rest}\pend{}\pstart{}\textsc{Dänemark\oindex{Dänemark@\textbf{Dänemark}|pw}}\pend{}{\bigskip}\vspace{1em}
\pstart
           \noindent{}{\pb}Mein lieber Richard, ich nehme an Sie
                  beko{\geminationm}en dieſe Karte am 24. Da{ }ſchreiben
               Sie mir \uline{gleich} nach \textsc{Stockholm}\oindex{Stockholm@\textbf{Stockholm}, \emph{Hauptstadt}|pw}. (\textsc{post rest} natürlich) Ich werde wahrſcheinlich
                  24.{ }25.{ }26. in \textsc{Krist.}\oindex{Oslo@\textbf{Oslo}, \emph{Hauptstadt}|pw}{ }ſein, dann bis etwa 30 od
                  31{ }\textsc{Stockholm}\oindex{Stockholm@\textbf{Stockholm}, \emph{Hauptstadt}|pw}. Es wäre wunderſchön, wenn Sie doch wenigſtens nach \textsc{Stockh}\oindex{Stockholm@\textbf{Stockholm}, \emph{Hauptstadt}|pw}. hinüberko{\geminationm}en \substVorne{}\textsuperscript{w}\substDazwischen{}mö\substHinten{}chten. Oder nach \textsc{Goetheborg}\oindex{Göteborg@\textbf{Göteborg}|pw} mir entgegen. Überlegen Sie{ }ſichs. Bitte laſſen Sie mich nicht ohne
               Nachricht. –\pend
           
\pstart
           Herzlich der Ihre{\\[\baselineskip]}\spacefill\mbox{ArthSch}\pend
           \leftskip=0em{}
\pstart
           \noindent{}an Bord der \textsc{Sig Jarl\orgindex{Sig Jarl@Sig Jarl|pw}}{ }17/7 96.\pend
           \selectlanguage{ngerman}\endnumbering\briefempfaengerindex{Beer-Hofmann, Richard@\textsc{Beer-Hofmann, Richard}!zzzSchnitzler, Arthur@\emph{von Arthur Schnitzler}!1896-07-171@{17. 7. 1896}|)be}\mylabel{L00565h}  \newcommand{\dateiname}{L00565}\newcommand{\titel}{Arthur Schnitzler an Richard Beer-Hofmann, 17. 7. 1896}\newcommand{\editorInnen}{Martin Anton Müller und Gerd-Hermann Susen}%% latex-leseansicht-abspann.tex
%% Abspann für die Leseansicht.
%% Der Schalter \ifkorrekturansicht ist bereits durch den Vorspann gesetzt.

%% latex-abspann.tex
%% Gemeinsamer Abspann für Korrekturansicht und Leseansicht.
%% Setzt den Schalter \ifkorrekturansicht voraus (gesetzt in den
%% einbindenden Dateien latex-korrekturansicht-abspann.tex bzw.
%% latex-leseansicht-abspann.tex).
%% ---------------------------------------------------------------

\normalsize

% Das esempio-Environment wird nur in der Leseansicht benötigt
\ifkorrekturansicht\else
\newenvironment{esempio}[3]%
{
    \vspace{1.5ex}
    \rlap{\underline{#1}}
    \par
    \setlength{\parindent}{0cm}
    \nopagebreak
    \leftskip=#2cm
    \rightskip=#3cm
}
{
    \par
}
\fi

\doendnotes{C}
\bigskip
\vfill

\clearpage

\footnotesize

\ifkorrekturansicht
  \lohead{\textsc{register}}
\fi

% theindex-Environment neu definieren ohne reledmac
\makeatletter
\renewenvironment{theindex}{%
  \ifkorrekturansicht
    \section*{\indexname}%
  \else
    \subsubsection*{Index der erwähnten Entitäten}%
  \fi
  \setlength{\parindent}{0pt}%
  \setlength{\parskip}{0pt plus 0.3pt}%
  \let\item\@idxitem
}{%
  \ifkorrekturansicht\clearpage\fi
}
\makeatother

\IfFileExists{\jobname-pw.ind}{\input{\jobname-pw.ind}}{}

% Quellenangabe nur in der Leseansicht
\ifkorrekturansicht\else
% Fallback-Definitionen, falls die .tex-Datei \titel etc. nicht gesetzt hat
\providecommand{\titel}{}
\providecommand{\editorInnen}{}
\providecommand{\dateiname}{\jobname}

\vspace{3cm}

\vfill

\footnotesize
\textsc{Quelle}: \titel. Herausgegeben von {\editorInnen}. In: \emph{Arthur Schnitzler: Briefwechsel mit Autorinnen und Autoren}.
 Digitale Edition, https://schnitzler-briefe.acdh.oeaw.ac.at/{\dateiname}.html (Stand \today)
\fi

\end{document}


