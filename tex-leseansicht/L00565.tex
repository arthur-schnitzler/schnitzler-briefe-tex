%% latex-korrekturansicht-vorspann.tex
%% Vorspann für die Korrekturansicht.
%% Lädt die gemeinsame Datei latex-vorspann.tex mit gesetztem Schalter.

\newif\ifkorrekturansicht
\korrekturansichttrue

\input{../tex-inputs/latex-vorspann}


\section[Arthur Schnitzler an Richard Beer-Hofmann, 17. 7. 1896]{L00565 Arthur Schnitzler an Richard Beer-Hofmann, 17. 7. 1896}
\nopagebreak\mylabel{L00565v}
\rehead{ }\normalsize\beginnumbering\briefempfaengerindex{Beer-Hofmann, Richard@\textsc{Beer-Hofmann, Richard}!zzzSchnitzler, Arthur@\emph{von Arthur Schnitzler}!1896-07-181@{18. 7. 1896}|(be}
\toendnotes[C]{\smallbreak\pagebreak[2]}\Standort{YCGL, MSS 31.}
\physDesc{Postkarte, 487 Zeichen
\newline{}Handschrift: 1) Bleistift, deutsche Kurrent\hspace{1em}2) Bleistift, lateinische Kurrent (\noindent{}Adresse)\hspace{1em}
\newline{}Versand: 1) Stempel: »\nobreak{}\oindex{Tromsø@\textbf{Tromsø}, \emph{P.PPLA}|pwk}Tromsø, 18. VII. 96\nobreak{}«.   2) Stempel: »\nobreak{}\oindex{Kopenhagen@\textbf{Kopenhagen}, \emph{P.PPLC}|pwk}Kjobenhavn, 23. 7. 96, 50M8\nobreak{}«. }\pstart{}{\pb}Dr. Richard Beer-Hofmann\pend{}\pstart{}Kopenhagen\oindex{Kopenhagen@\textbf{Kopenhagen}, \emph{P.PPLC}|pw}\pend{}\pstart{}post rest\pend{}\pstart{}Dänemark\oindex{Daenemark@\textbf{Dänemark}, \emph{A.PCLI}|pw}\pend{}{\bigskip}\vspace{1em}
\pstart
           \noindent{}{\pb}Mein lieber Richard, ich nehme an Sie
                  beko{\geminationm}en dieſe Karte am 24. Da ſchreiben
               Sie mir \uline{gleich} nach \textsc{Stockholm}\oindex{Stockholm@\textbf{Stockholm}, \emph{P.PPLC}|pw}. (\textsc{post rest} natürlich) Ich werde wahrſcheinlich
                  24.25.26. in \textsc{Krist.}\oindex{Oslo@\textbf{Oslo}, \emph{P.PPLC}|pw}{ }ſein, dann bis etwa 30 od
                  31{ }\textsc{Stockholm}\oindex{Stockholm@\textbf{Stockholm}, \emph{P.PPLC}|pw}. Es wäre wunderſchön, wenn Sie doch wenigſtens nach \textsc{Stockh}\oindex{Stockholm@\textbf{Stockholm}, \emph{P.PPLC}|pw}. hinüberko{\geminationm}en \substVorne{}\textsuperscript{w}\substDazwischen{}mö\substHinten{}chten. Oder nach \textsc{Goetheborg}\oindex{Goeteborg@\textbf{Göteborg}, \emph{P.PPLA}|pw} mir entgegen. Überlegen Sie ſichs. Bitte laſſen Sie mich nicht ohne
               Nachricht. –\pend
           
\pstart
           Herzlich der Ihre{\\[\baselineskip]}\spacefill\mbox{ArthSch}\pend
           \leftskip=0em{}
\pstart
           \noindent{}an Bord der \textsc{Sig Jarl}{ }17/7 96.\pend
           \selectlanguage{ngerman}\endnumbering\briefempfaengerindex{Beer-Hofmann, Richard@\textsc{Beer-Hofmann, Richard}!zzzSchnitzler, Arthur@\emph{von Arthur Schnitzler}!1896-07-181@{18. 7. 1896}|)be}\mylabel{L00565h}  \normalsize

\doendnotes{C}
\bigskip
\vfill

\clearpage

\footnotesize

\lohead{\textsc{register}}

% Definiere theindex-Environment komplett neu ohne reledmac
\makeatletter
\renewenvironment{theindex}{%
  \section*{\indexname}%
  \setlength{\parindent}{0pt}%
  \setlength{\parskip}{0pt plus 0.3pt}%
  \let\item\@idxitem
}{%
  \clearpage
}
\makeatother

\IfFileExists{\jobname-pw.ind}{\input{\jobname-pw.ind}}{}

\end{document}

      