%% latex-korrekturansicht-vorspann.tex
%% Vorspann für die Korrekturansicht.
%% Lädt die gemeinsame Datei latex-vorspann.tex mit gesetztem Schalter.

\newif\ifkorrekturansicht
\korrekturansichttrue

\input{../tex-inputs/latex-vorspann}


\section[ Felix Salten an Arthur Schnitzler, 14. 5. 1906]{L03474 Felix Salten an Arthur Schnitzler, 14. 5. 1906}
\nopagebreak\mylabel{L03474v}
\rehead{ }\normalsize\beginnumbering\briefempfaengerindex{Schnitzler, Arthur@\textsc{Schnitzler, Arthur}!zzzSalten, Felix@\emph{von Felix Salten}!1906-05-141@{14. 5. 1906}|(be}
\toendnotes[C]{\smallbreak\pagebreak[2]}\Standort{CUL, Schnitzler, B 89, B 1.}
\physDesc{Brief, 1 Blatt, 1 Seite, 271 Zeichen
\newline{}Handschrift: schwarze Tinte, lateinische Kurrent
\newline{}Ordnung: mit Bleistift von unbekannter Hand nummeriert: »215« }\toendnotes[C]{\smallbreak}
\pstart
           \raggedleft{}{\pb}Berlin\oindex{Berlin@\textbf{Berlin}, \emph{P.PPLC}|pw}, 14. V. 06.\pend
           
\pstart{}Lieber Freund,\pend\vspace{0.5em}
\pstart
           \label{K_L03474-1v}\edtext{morgen spielen sie in Wien\oindex{Wien@\textbf{Wien}, \emph{A.ADM2}|pw} Ihren »Einsamen\pwindex{einsame Weg. Schauspiel in fuenf Akten@\emph{Der einsame Weg. Schauspiel in fünf Akten}|pw}{ }\damage{Weg\pwindex{einsame Weg. Schauspiel in fuenf Akten@\emph{Der einsame Weg. Schauspiel in fünf Akten}|pw}«.}}{\lemma{\textnormal{\emph{morgen … Weg«.}}}\Cendnote{\textnormal{Das Gastspiel\pwindex{einsame Weg. Schauspiel in fuenf Akten@\emph{Der einsame Weg. Schauspiel in fünf Akten}|pwkv} des \emph{Lessing-Theaters}\orgindex{Lessing-Theater@Lessing-Theater|pwk} fand im Theater an der
                     Wien\oindex{Theater an der Wien@\textbf{Theater an der Wien}, \emph{Theater (K.THE)}|pwk} statt. Siehe A. S.: \emph{Tagebuch}, 15. 5. 1906.}}}\label{K_L03474-1} Irgendwie habe ich dabei das Gefühl, dass ich mir
               selbst (und viellei\damage{cht} auch Ihnen ein wenig) dort \label{K_L03474-2v}\edtext{fehle}{\lemma{\textnormal{\emph{fehle}}}\Cendnote{\textnormal{Salten\pwindex{Salten, Felix 06.09.1869 – 08.10.1945@\textsc{Salten, Felix} (06.09.1869 – 08.10.1945), \emph{Schriftsteller/Schriftstellerin, Journalist/Journalistin, Chefredakteur/Chefredakteurin}|pwk} fühlte
               sich womöglich auch deswegen involviert, weil er im Voraus Schnitzler empfohlen hatte, eine 
                  Umbesetzung von Emanuel
                     Reicher\pwindex{Reicher, Emanuel 18.06.1849 – 15.05.1924@\textsc{Reicher, Emanuel} (18.06.1849 – 15.05.1924), \emph{Schauspieler/Schauspielerin}|pwk} zu Rudolf Rittner\pwindex{Rittner, Rudolf 30.06.1869 – 04.02.1943@\textsc{Rittner, Rudolf} (30.06.1869 – 04.02.1943), \emph{Theaterleiter/Theaterleiterin, Schauspieler/Schauspielerin}|pwk} zu erwirken, vgl. Felix Salten u. a. an Arthur Schnitzler, 19. 4. 1906; 
                  Felix Salten an Arthur Schnitzler, 21. 4. [1906].
               }}}\label{K_L03474-2}.
               Jedenfalls möchte ich, dass Sie an diesem Tag einen Gruß von mir haben.\pend
           
\pstart
           herzlichst {\\[\baselineskip]}Ihr \spacefill\mbox{Salten}\pend
           \leftskip=0em{}\selectlanguage{ngerman}\endnumbering\briefempfaengerindex{Schnitzler, Arthur@\textsc{Schnitzler, Arthur}!zzzSalten, Felix@\emph{von Felix Salten}!1906-05-141@{14. 5. 1906}|)be}\mylabel{L03474h}  \normalsize

\doendnotes{C}
\bigskip
\vfill

\clearpage

\footnotesize

\lohead{\textsc{register}}

% Definiere theindex-Environment komplett neu ohne reledmac
\makeatletter
\renewenvironment{theindex}{%
  \section*{\indexname}%
  \setlength{\parindent}{0pt}%
  \setlength{\parskip}{0pt plus 0.3pt}%
  \let\item\@idxitem
}{%
  \clearpage
}
\makeatother

\IfFileExists{\jobname-pw.ind}{\input{\jobname-pw.ind}}{}

\end{document}

      