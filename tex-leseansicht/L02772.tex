%% latex-leseansicht-vorspann.tex
%% Vorspann für die Leseansicht.
%% Lädt die gemeinsame Datei latex-vorspann.tex mit nicht gesetztem Schalter.

\newif\ifkorrekturansicht
\korrekturansichtfalse

\input{../tex-inputs/latex-vorspann}

\begin{center}
            \textcolor{red}{ENTWURF, NICHT FERTIG KORRIGIERT}
                      \end{center}
            
         
         \renewcommand{\erwaehntePersonen}{Personen: Peter Altenberg,  Beck, Richard Beer-Hofmann, Paula Beer-Hofmann, Alfred von Berger, Léon Bourgeois, Adele Doré, Marie Glümer, Clementine Goldmann, Maximilian Harden, Heinrich Kanner, Alfred Kerr, Pierre Lalo, Albert Langen, Rudolf Leyrer, Hermann Mamroth, Heinz Monnard, Henri de Riaz, Christian Schefer, Leopold Sonnemann, Théodore de Wyzewa}
         \renewcommand{\erwaehnteInstitutionen}{Institutionen: Die Zeit. Wiener Wochenschrift, Frankfurter Zeitung, Simplicissimus}
         \renewcommand{\erwaehnteOrte}{Orte: Berlin, Deutschland, Dänemark, Frankfurt am Main, Köln, Paris, Rue Coëtlogon, Skodsborg, Wien, rue Feydeau}
         \renewcommand{\erwaehnteWerke}{Werke: Burgtheater [Rechte der Seele, Liebelei], Die Zukunft, Die überspannte Person, Frankfurter Zeitung, Freiwild. Schauspiel in 3 Akten, Liebelei. Schauspiel in drei Akten, Montags-Revue. Wochenschrift für Politik, Finanzen, Kunst und Literatur, Mourir. Roman, Simplicissimus, Theaternotizen [Liebelei], [Man schreibt uns aus Köln]}
               \section[ Paul Goldmann an Arthur Schnitzler, 29. 4. {[}1896{]}]{ Paul Goldmann an Arthur Schnitzler, 29. 4. {[}1896{]}}\nopagebreak\mylabel{v}\rehead{ }\begin{ledgroupsized}[t]{13cm}\normalsize\beginnumbering \toendnotes[C]{\smallbreak\pagebreak[2]} \Standort{DLA, A:Schnitzler, HS.NZ85.1.3166.}
\physDesc{Brief, 2 Blätter, 8 Seiten
\newline{}Handschrift: blaue Tinte, deutsche Kurrent
\newline{}Schnitzler: 1) mit Bleistift das Jahr »96« vermerkt sowie »\noindent{}\textsc{Kerr}\pwindex{Kerr, Alfred 25.12.1867 – 12.10.1948@\textsc{Kerr, Alfred} (25.12.1867 – 12.10.1948), \emph{Schriftsteller, Kritiker}|pw}?{ / }\textsc{Altenb\pwindex{Altenberg, Peter 09.03.1859 – 08.01.1919@\textsc{Altenberg, Peter} (09.03.1859 – 08.01.1919), \emph{Schriftsteller}|pw}?}{ / }\textcolor{gray}{Brief}« vermerkt  2) mit rotem Buntstift drei Unterstreichungen}\toendnotes[C]{\smallbreak}\pstart
           \noindent{}{\pb}\textcolor{gray}{\textbf{\textbf{Frankfurter Zeitung\orgindex{Frankfurter Zeitung@Frankfurter Zeitung|pw}}}}\pend
           \pstart
           \textcolor{gray}{\textbf{(\begin{otherlanguage}{french}Gazette de Francfort\end{otherlanguage}\orgindex{Frankfurter Zeitung@Frankfurter Zeitung|pw}).}}\pend
           \pstart
           \textcolor{gray}{\textbf{\textbf{\begin{otherlanguage}{french}Fondateur M.\end{otherlanguage}{ }L. Sonnemann\pwindex{Sonnemann, Leopold 1831-10-29 – 1909-10-30@\textsc{Sonnemann, Leopold} (1831-10-29 – 1909-10-30), \emph{Journalist, Herausgeber}|pw}.}}}\pend
           \pstart
           \begin{otherlanguage}{french}\textcolor{gray}{\textbf{Journal\pwindex{?? Werk@Nicht ermittelte Verfasserinnen und Verfasser!Frankfurter Zeitung1856 – 1943@\emph{Frankfurter Zeitung} {[}1856 – 1943{]}|pwv} politique,
                        financier,}}\end{otherlanguage}\pend
           \pstart
           \begin{otherlanguage}{french}\textcolor{gray}{\textbf{commercial et littéraire.}}\end{otherlanguage}\pend
           \pstart
           \begin{otherlanguage}{french}\textcolor{gray}{\textbf{\textbf{Paraissant trois fois par jour.}}}\end{otherlanguage}\pend
           \pstart
           \begin{otherlanguage}{french}\textcolor{gray}{\textbf{\textbf{Bureau à Paris\oindex{Paris@\textbf{Paris}|pw}:}}}\end{otherlanguage}\pend
           \pstart
           \begin{otherlanguage}{french}\textcolor{gray}{\textbf{\textbf{24. Rue Feydeau\oindex{rue Feydeau@\textbf{rue Feydeau}|pw}.}}}\end{otherlanguage}\hfill \textsc{Paris\oindex{Paris@\textbf{Paris}|pw}}, 29. April.\pend
           \pstart\center{}Mein lieber Freund,\pend\pstart
           Ich war 14 Tage in Frankfurt\oindex{Frankfurt am Main@\textbf{Frankfurt am Main}|pw}, habe geruht und
               neue Kräfte zu gewinnen geſtrebt. Nöthig wars. Zur Feier meiner Rückkunft fand eine
               feſtliche \label{K_L02772-1v}\edtext{Miniſter\pwindex{Bourgeois, Leon 1851-05-29 – 1925-09-29@\textsc{Bourgeois, Léon} (1851-05-29 – 1925-09-29), \emph{Politiker, Minister, Nobelpreisträger}|pwv}kriſis}{\lemma{\textnormal{\emph{Miniſterkriſis}}}\Cendnote{\textnormal{Mit dem 29. 4. 1896 endete das Ministerium von Léon Bourgeois\pwindex{Bourgeois, Leon 1851-05-29 – 1925-09-29@\textsc{Bourgeois, Léon} (1851-05-29 – 1925-09-29), \emph{Politiker, Minister, Nobelpreisträger}|pwk}.}}}\label{K_L02772-1h} ſtatt. Ich ſtecke bis über die
               Ohren in Arbeit, und ſo komme ich erſt heut dazu, Dir
               für Deinen ſo überaus lieben Brief zu danken, den ich noch in Frankfurt\oindex{Frankfurt am Main@\textbf{Frankfurt am Main}|pw} empfing. Als ich in Frankfurt\oindex{Frankfurt am Main@\textbf{Frankfurt am Main}|pw} war, wurde gerade dein Stück\pwindex{Schnitzler, Arthur 15.05.1862 – 21.10.1931@\textsc{Schnitzler, Arthur} (15.05.1862 – 21.10.1931), \emph{Schriftsteller, Mediziner}!Liebelei. Schauspiel in drei Akten1895-10-09@\strich\emph{Liebelei. Schauspiel in drei Akten} {[}1895-10-09{]}|pwv} in Köln\oindex{Koeln@\textbf{Köln}|pw} aufgeführt,
               und in der {\pb}Frankf. Zeit.\pwindex{?? Werk@Nicht ermittelte Verfasserinnen und Verfasser!Frankfurter Zeitung1856 – 1943@\emph{Frankfurter Zeitung} {[}1856 – 1943{]}|pw} erſchien eine kleine \label{K_L02772-2v}\edtext{Beſprechung\pwindex{?? Werk@Nicht ermittelte Verfasserinnen und Verfasser!Man schreibt uns aus Koeln]1896-04-13@\emph{[Man schreibt uns aus Köln]} {[}1896-04-13{]}|pwv}}{\lemma{\textnormal{\emph{Beſprechung}}}\Cendnote{\textnormal{\emph{[Man schreibt uns aus Köln]}\pwindex{?? Werk@Nicht ermittelte Verfasserinnen und Verfasser!Man schreibt uns aus Koeln]1896-04-13@\emph{[Man schreibt uns aus Köln]} {[}1896-04-13{]}|pwk}. In: \emph{Frankfurter Zeitung}\pwindex{?? Werk@Nicht ermittelte Verfasserinnen und Verfasser!Frankfurter Zeitung1856 – 1943@\emph{Frankfurter Zeitung} {[}1856 – 1943{]}|pwk}, Jg. 40, Nr. 103,
                        13. 4. 1896, Abendblatt, S. 2.}}}\label{K_L02772-2h}, die ich hier
               einfüge, da Du ſie vielleicht überſehen haſt.\pend
           {\bigskip}\pstart
           \noindent{}\textcolor{gray}{\textbf{Man ſchreibt uns aus \so{Köln}\oindex{Koeln@\textbf{Köln}|pw}, 11. April: Schnitzler’s Schauſpiel\pwindex{Schnitzler, Arthur 15.05.1862 – 21.10.1931@\textsc{Schnitzler, Arthur} (15.05.1862 – 21.10.1931), \emph{Schriftsteller, Mediziner}!Liebelei. Schauspiel in drei Akten1895-10-09@\strich\emph{Liebelei. Schauspiel in drei Akten} {[}1895-10-09{]}|pwv} »\so{Liebelei}\pwindex{Schnitzler, Arthur 15.05.1862 – 21.10.1931@\textsc{Schnitzler, Arthur} (15.05.1862 – 21.10.1931), \emph{Schriftsteller, Mediziner}!Liebelei. Schauspiel in drei Akten1895-10-09@\strich\emph{Liebelei. Schauspiel in drei Akten} {[}1895-10-09{]}|pw}« ging geſtern zum erſten Mal in Szene und erzielte einen ſehr
                  ſtarken Erfolg. Die Mitwirkenden wurden nach dem letzten Akt\pwindex{Schnitzler, Arthur 15.05.1862 – 21.10.1931@\textsc{Schnitzler, Arthur} (15.05.1862 – 21.10.1931), \emph{Schriftsteller, Mediziner}!Liebelei. Schauspiel in drei Akten1895-10-09@\strich\emph{Liebelei. Schauspiel in drei Akten} {[}1895-10-09{]}|pwv} fünfmal gerufen. Die Darſtellung war
                  im Ganzen recht befriedigend. Die Chriſtine\pwindex{Schnitzler, Arthur 15.05.1862 – 21.10.1931@\textsc{Schnitzler, Arthur} (15.05.1862 – 21.10.1931), \emph{Schriftsteller, Mediziner}!Liebelei. Schauspiel in drei Akten1895-10-09@\strich\emph{Liebelei. Schauspiel in drei Akten} {[}1895-10-09{]}|pwv} wußte Frau Doré\pwindex{Dore, Adele 1869-04-09 – 1918-02-09@\textsc{Doré, Adele} (1869-04-09 – 1918-02-09), \emph{Schauspielerin}|pw} in ergreifender Weise zu geſtalten. In der Mizi\pwindex{Schnitzler, Arthur 15.05.1862 – 21.10.1931@\textsc{Schnitzler, Arthur} (15.05.1862 – 21.10.1931), \emph{Schriftsteller, Mediziner}!Liebelei. Schauspiel in drei Akten1895-10-09@\strich\emph{Liebelei. Schauspiel in drei Akten} {[}1895-10-09{]}|pwv} des Frl. \so{Glümer}\pwindex{Gluemer, Marie 03.07.1867 – 16.11.1925@\textsc{Glümer, Marie} (03.07.1867 – 16.11.1925), \emph{Schauspielerin}|pw} und in dem Theodor\pwindex{Schnitzler, Arthur 15.05.1862 – 21.10.1931@\textsc{Schnitzler, Arthur} (15.05.1862 – 21.10.1931), \emph{Schriftsteller, Mediziner}!Liebelei. Schauspiel in drei Akten1895-10-09@\strich\emph{Liebelei. Schauspiel in drei Akten} {[}1895-10-09{]}|pwv}
                  des Hrn. \so{Leyrer}\pwindex{Leyrer, Rudolf 1857-08-19 – 1939-12-26@\textsc{Leyrer, Rudolf} (1857-08-19 – 1939-12-26), \emph{Schauspieler}|pw} fand die Wien\oindex{Wien@\textbf{Wien}|pw}er Leichtlebigkeit ihre
                  angemeſſene Vertretung. Fein und discret gab Herr \so{Beck}\pwindex{Beck @\textsc{Beck}, \emph{Schauspieler}|pw} den alten Musiker\pwindex{Schnitzler, Arthur 15.05.1862 – 21.10.1931@\textsc{Schnitzler, Arthur} (15.05.1862 – 21.10.1931), \emph{Schriftsteller, Mediziner}!Liebelei. Schauspiel in drei Akten1895-10-09@\strich\emph{Liebelei. Schauspiel in drei Akten} {[}1895-10-09{]}|pwv};
                  auch der Fritz\pwindex{Schnitzler, Arthur 15.05.1862 – 21.10.1931@\textsc{Schnitzler, Arthur} (15.05.1862 – 21.10.1931), \emph{Schriftsteller, Mediziner}!Liebelei. Schauspiel in drei Akten1895-10-09@\strich\emph{Liebelei. Schauspiel in drei Akten} {[}1895-10-09{]}|pwv} des Hrn. \so{Monnard}\pwindex{Monnard, Heinz 31.12.1873 – 11.07.1912@\textsc{Monnard, Heinz} (31.12.1873 – 11.07.1912), \emph{Schauspieler}|pw} war nicht ohne tiefere Wirkung. –}}\pend
           {\bigskip}\pstart
           \noindent{}auch lege ich einen \label{K_L02772-3v}\edtext{Brief}{\lemma{\textnormal{\emph{Brief}}}\Cendnote{\textnormal{Goldmann\pwindex{Goldmann, Paul 31.01.1865 – 25.09.1935@\textsc{Goldmann, Paul} (31.01.1865 – 25.09.1935), \emph{Schriftsteller, Journalist}|pwk} vergaß, ihn beizulegen (vgl. Paul Goldmann an Arthur Schnitzler, 3. 4. [1895]).}}}\label{K_L02772-3h} des Herrn \textsc{Christian Schefer\pwindex{Schefer, Christian 1866-07-14 – Februar 1944@\textsc{Schefer, Christian} (1866-07-14 – Februar 1944), \emph{Journalist, Lehrer}|pw}} bei, den ich noch in Frankfurt\oindex{Frankfurt am Main@\textbf{Frankfurt am Main}|pw} erhielt.
               Schicke ihm ein Exemplar von »\textsc{Mourir\pwindex{Schnitzler, Arthur 15.05.1862 – 21.10.1931@\textsc{Schnitzler, Arthur} (15.05.1862 – 21.10.1931), \emph{Schriftsteller, Mediziner}!Mourir. Roman1895-04-27 – 1895-06-01@\strich\emph{Mourir. Roman} {[}1895-04-27 – 1895-06-01{]}|pw}}«, ebenſo eines an \textsc{Lalo\pwindex{Lalo, Pierre 1866-09-06 – 1943-06-09@\textsc{Lalo, Pierre} (1866-09-06 – 1943-06-09), \emph{Kritiker}|pw}}, ein drittes an \textsc{M. de
                     Wyzewa\pwindex{Wyzewa, Theodore de 1862-09-12 – 1917-04-07@\textsc{Wyzewa, Théodore de} (1862-09-12 – 1917-04-07), \emph{Schriftsteller, Journalist}|pw}, 9. Rue Coëtlogon\oindex{Rue Coetlogon@\textbf{Rue Coëtlogon}|pw}}. Auch ſchicke mir noch zwei oder {\pb}drei \substVorne{}\textsuperscript{B\textcolor{gray}{üch}e}\substDazwischen{}Exemplare\pwindex{Schnitzler, Arthur 15.05.1862 – 21.10.1931@\textsc{Schnitzler, Arthur} (15.05.1862 – 21.10.1931), \emph{Schriftsteller, Mediziner}!Mourir. Roman1895-04-27 – 1895-06-01@\strich\emph{Mourir. Roman} {[}1895-04-27 – 1895-06-01{]}|pwv}\substHinten{} zur Propaganda. Das Buch\pwindex{Schnitzler, Arthur 15.05.1862 – 21.10.1931@\textsc{Schnitzler, Arthur} (15.05.1862 – 21.10.1931), \emph{Schriftsteller, Mediziner}!Mourir. Roman1895-04-27 – 1895-06-01@\strich\emph{Mourir. Roman} {[}1895-04-27 – 1895-06-01{]}|pwv} iſt ſehr gut ausgeſtattet und ſieht recht vornehm aus. Ferner ſende ich
               Dir die Briefe des Herrn \textsc{de Riaz\pwindex{Riaz, Henri de 1871 – 1951@\textsc{Riaz, Henri de} (1871 – 1951), \emph{Dichter}|pw}} zurück. Laß’ die \label{K_L02772-99v}\edtext{Überſetzung\pwindex{Schnitzler, Arthur 15.05.1862 – 21.10.1931@\textsc{Schnitzler, Arthur} (15.05.1862 – 21.10.1931), \emph{Schriftsteller, Mediziner}!Liebelei. Schauspiel in drei Akten1895-10-09@\strich\emph{Liebelei. Schauspiel in drei Akten} {[}1895-10-09{]}|pwv}s-Angelegenheit}{\lemma{\textnormal{\emph{Überſetzungs-Angelegenheit}}}\Cendnote{\textnormal{siehe Paul Goldmann an Arthur Schnitzler, 5. 12. [1895]}}}\label{K_L02772-99h} noch ruhn und antworte aufſchiebend. Endlich finde ich noch in meinen
               Papieren die \label{K_L02772-4v}\edtext{Kritik\pwindex{Berger, Alfred von 30.04.1853 – 24.08.1912@\textsc{Berger, Alfred von} (30.04.1853 – 24.08.1912), \emph{Schriftsteller, Journalist, Theaterleiter}!Burgtheater [Rechte der Seele, Liebelei]1895-10-14@\strich\emph{Burgtheater [Rechte der Seele, Liebelei]} {[}1895-10-14{]}|pwuv}}{\lemma{\textnormal{\emph{Kritik}}}\Cendnote{\textnormal{Alfred Freiherr von Berger\pwindex{Berger, Alfred von 30.04.1853 – 24.08.1912@\textsc{Berger, Alfred von} (30.04.1853 – 24.08.1912), \emph{Schriftsteller, Journalist, Theaterleiter}|pwk}: \emph{Burgtheater}\pwindex{Berger, Alfred von 30.04.1853 – 24.08.1912@\textsc{Berger, Alfred von} (30.04.1853 – 24.08.1912), \emph{Schriftsteller, Journalist, Theaterleiter}!Burgtheater [Rechte der Seele, Liebelei]1895-10-14@\strich\emph{Burgtheater [Rechte der Seele, Liebelei]} {[}1895-10-14{]}|pwk}. In: \emph{Montags-Revue}\pwindex{?? Werk@Nicht ermittelte Verfasserinnen und Verfasser!Montags-Revue. Wochenschrift fuer Politik, Finanzen, Kunst und Literatur1874 – 1915@\emph{Montags-Revue. Wochenschrift für Politik, Finanzen, Kunst und Literatur} {[}1874 – 1915{]}|pwk}, Jg. XXXX, Nr. XXXX, 14. 10. 1895, S. XXXX.}}}\label{K_L02772-4h} des Baron \textsc{Berger}\pwindex{Berger, Alfred von 30.04.1853 – 24.08.1912@\textsc{Berger, Alfred von} (30.04.1853 – 24.08.1912), \emph{Schriftsteller, Journalist, Theaterleiter}|pw}, die ich Dir gleichfalls zurückſende.\pend
           \pstart
           Zu erzählen habe ich Dir nichts. Mein Leben iſt vollſtändig unintereſſant. Es gibt
               nichts Neues und wird nie etwas Neues geben, {\pb}außer
               irgend einem definitiven Unglück. Intereſſant iſt nur Dein Leben, und ich möchte ſehr
               viel darüber wiſſen. Haſt Du alſo \label{K_L02772-5v}\edtext{zum
               dritten Mal angeſangen, das Stück\pwindex{Schnitzler, Arthur 15.05.1862 – 21.10.1931@\textsc{Schnitzler, Arthur} (15.05.1862 – 21.10.1931), \emph{Schriftsteller, Mediziner}!Freiwild. Schauspiel in 3 Akten1896@\strich\emph{Freiwild. Schauspiel in 3 Akten} {[}1896{]}|pwv} zu ſchreiben}{\lemma{\textnormal{\emph{zum … ſchreiben}}}\Cendnote{\textnormal{siehe A. S.: \emph{Tagebuch}, 27. 4. 1896}}}\label{K_L02772-5h}? Könnte man nicht doch das Manuſcript\pwindex{Schnitzler, Arthur 15.05.1862 – 21.10.1931@\textsc{Schnitzler, Arthur} (15.05.1862 – 21.10.1931), \emph{Schriftsteller, Mediziner}!Freiwild. Schauspiel in 3 Akten1896@\strich\emph{Freiwild. Schauspiel in 3 Akten} {[}1896{]}|pwv} ſehen? Wirſt Du \label{K_L02772-6v}\edtext{in
               die »Zeit\orgindex{Zeit. Wiener Wochenschrift@Die Zeit. Wiener Wochenschrift|pw}« eintreten}{\lemma{\textnormal{\emph{in
               die »Zeit« eintreten}}}\Cendnote{\textnormal{nicht geschehen}}}\label{K_L02772-6h}, jetzt nach \textsc{Kanner\pwindex{Kanner, Heinrich 09.11.1864 – 15.02.1930@\textsc{Kanner, Heinrich} (09.11.1864 – 15.02.1930), \emph{Herausgeber, Publizist}|pw}s} Rückkehr? Und wie iſt ſonſt
               Daſeinsführung und Stimmung?\pend
           \pstart
           Recht geärgert habe ich mich, als ich Deinen {\pb}\label{K_L02772-7v}\edtext{Namen im »\textsc{Simplicissimus}\pwindex{?? Werk@Nicht ermittelte Verfasserinnen und Verfasser!Simplicissimus1896-04-04 – 1944-09-13@\emph{Simplicissimus} {[}1896-04-04 – 1944-09-13{]}|pw}«}{\lemma{\textnormal{\emph{Namen im »Simplicissimus«}}}\Cendnote{\textnormal{Arthur Schnitzler\pwindex{Schnitzler, Arthur 15.05.1862 – 21.10.1931@\textsc{Schnitzler, Arthur} (15.05.1862 – 21.10.1931), \emph{Schriftsteller, Mediziner}|pwk}: \emph{Die überspannte Person}\pwindex{Schnitzler, Arthur 15.05.1862 – 21.10.1931@\textsc{Schnitzler, Arthur} (15.05.1862 – 21.10.1931), \emph{Schriftsteller, Mediziner}!ueberspannte Person1896-04-18@\strich\emph{Die überspannte Person} {[}1896-04-18{]}|pwk}. In: \emph{Simplicissimus}\pwindex{?? Werk@Nicht ermittelte Verfasserinnen und Verfasser!Simplicissimus1896-04-04 – 1944-09-13@\emph{Simplicissimus} {[}1896-04-04 – 1944-09-13{]}|pwk}, Jg. 1, H. 3, 18. 4. 1896, S. 3 u. 6.}}}\label{K_L02772-7h}{ }fand\pwindex{Schnitzler, Arthur 15.05.1862 – 21.10.1931@\textsc{Schnitzler, Arthur} (15.05.1862 – 21.10.1931), \emph{Schriftsteller, Mediziner}!ueberspannte Person1896-04-18@\strich\emph{Die überspannte Person} {[}1896-04-18{]}|pwv}. Dieſer
               Lausbub’ \textsc{Langen\pwindex{Langen, Albert 1869-07-08 – 1909-04-30@\textsc{Langen, Albert} (1869-07-08 – 1909-04-30), \emph{Verleger}|pw}}, der mir i\substVorne{}\textsuperscript{m}\substDazwischen{}n\substHinten{}{ }\textsc{Paris\oindex{Paris@\textbf{Paris}|pw}}, wenn ich ihn dazu drängte, Deine Bücher in Verlag zu nehmen, ſtets antwortete:
               Du könnteſt \label{K_L02772-88v}\edtext{nicht deutſch
                  ſchreiben}{\lemma{\textnormal{\emph{nicht deutſch
                  ſchreiben}}}\Cendnote{\textnormal{eventuell auf die
                  Verwendung von Austriazismen gemünzt}}}\label{K_L02772-88h}, – iſt jetzt in der Lage, ſein neues
                  Unternehmen\orgindex{Simplicissimus@Simplicissimus|pwv} mit Deinem
               jungen \textsc{Rénommée} aufzuputzen. Das hat er wahrlich nicht
               verdient. Warum haſt {\pb}Du ihm den Beitrag\pwindex{Schnitzler, Arthur 15.05.1862 – 21.10.1931@\textsc{Schnitzler, Arthur} (15.05.1862 – 21.10.1931), \emph{Schriftsteller, Mediziner}!ueberspannte Person1896-04-18@\strich\emph{Die überspannte Person} {[}1896-04-18{]}|pwv} gegeben\damage{?} Ich bekam in Deutſchland\oindex{Deutschland@\textbf{Deutschland}|pw} durch Zufall
               das Heft\pwindex{Zukunft1892 – 1922@\emph{Die Zukunft} {[}1892 – 1922{]}|pwv} der »Zukunft\pwindex{Zukunft1892 – 1922@\emph{Die Zukunft} {[}1892 – 1922{]}|pw}« in die Hand, das \label{K_L02772-8v}\edtext{\textsc{Harden\pwindex{Harden, Maximilian 20.10.1861 – 30.10.1927@\textsc{Harden, Maximilian} (20.10.1861 – 30.10.1927), \emph{Schriftsteller, Publizist}|pw}s}{ }Kritik\pwindex{Harden, Maximilian 20.10.1861 – 30.10.1927@\textsc{Harden, Maximilian} (20.10.1861 – 30.10.1927), \emph{Schriftsteller, Publizist}!Theaternotizen [Liebelei]1896-03-14@\strich\emph{Theaternotizen [Liebelei]} {[}1896-03-14{]}|pwv} über »Liebelei\pwindex{Schnitzler, Arthur 15.05.1862 – 21.10.1931@\textsc{Schnitzler, Arthur} (15.05.1862 – 21.10.1931), \emph{Schriftsteller, Mediziner}!Liebelei. Schauspiel in drei Akten1895-10-09@\strich\emph{Liebelei. Schauspiel in drei Akten} {[}1895-10-09{]}|pw}«}{\lemma{\textnormal{\emph{Hardens … »Liebelei«}}}\Cendnote{\textnormal{Maximilian Harden\pwindex{Harden, Maximilian 20.10.1861 – 30.10.1927@\textsc{Harden, Maximilian} (20.10.1861 – 30.10.1927), \emph{Schriftsteller, Publizist}|pwk}: \emph{Theaternotizen}\pwindex{Harden, Maximilian 20.10.1861 – 30.10.1927@\textsc{Harden, Maximilian} (20.10.1861 – 30.10.1927), \emph{Schriftsteller, Publizist}!Theaternotizen [Liebelei]1896-03-14@\strich\emph{Theaternotizen [Liebelei]} {[}1896-03-14{]}|pwk}. In: \emph{Die
                        Zukunft}\pwindex{Zukunft1892 – 1922@\emph{Die Zukunft} {[}1892 – 1922{]}|pwk}, Jg. 5, Bd. 14, 14. 3. 1896,
                     S. 527–528.}}}\label{K_L02772-8h} enthält. Das iſt doch eine recht unverſtändige Kritik\pwindex{Harden, Maximilian 20.10.1861 – 30.10.1927@\textsc{Harden, Maximilian} (20.10.1861 – 30.10.1927), \emph{Schriftsteller, Publizist}!Theaternotizen [Liebelei]1896-03-14@\strich\emph{Theaternotizen [Liebelei]} {[}1896-03-14{]}|pw}, die Dich völlig unterſchätzt. Biſt Du
               trotzdem bei Deiner großen Meinung über \textsc{Harden\pwindex{Harden, Maximilian 20.10.1861 – 30.10.1927@\textsc{Harden, Maximilian} (20.10.1861 – 30.10.1927), \emph{Schriftsteller, Publizist}|pw}} geblieben?\pend
           \pstart
           Aber ich will nicht fragen, und Du ſollſt den \strikeout{Ih}
               Inhalt des nächſten Briefes nach {\pb}\damage{freier} Wahl zuſammen\damage{th}un. Schreib’ mir nur recht viel über Dich.\pend
           \pstart
           Und wie gehts dem \textsc{Richard\pwindex{Beer-Hofmann, Richard 1866-07-11 – 1945-09-26@\textsc{Beer-Hofmann, Richard} (1866-07-11 – 1945-09-26), \emph{Schriftsteller}|pw}}? Er bringts wirklich fertig, mir keine Zeile zu ſchreiben. Erwartet hab’ ichs,
               aber es erſtaunt mich doch. Es iſt immerhin der ſchönſte Fall von Faulheit, der mir
               in meinem Leben vorgekommen iſt.\pend
           \pstart
           Gern ginge ich mit früh im August{ }{\pb}nach \label{K_L02772-9v}\edtext{Dänemark\oindex{Daenemark@\textbf{Dänemark}|pw}}{\lemma{\textnormal{\emph{Dänemark}}}\Cendnote{\textnormal{Von 5. 8. 1896 bis 21. 8. 1896 waren Schnitzler\pwindex{Schnitzler, Arthur 15.05.1862 – 21.10.1931@\textsc{Schnitzler, Arthur} (15.05.1862 – 21.10.1931), \emph{Schriftsteller, Mediziner}|pwk},
                     Goldmann\pwindex{Goldmann, Paul 31.01.1865 – 25.09.1935@\textsc{Goldmann, Paul} (31.01.1865 – 25.09.1935), \emph{Schriftsteller, Journalist}|pwk}, Richard\pwindex{Beer-Hofmann, Richard 1866-07-11 – 1945-09-26@\textsc{Beer-Hofmann, Richard} (1866-07-11 – 1945-09-26), \emph{Schriftsteller}|pwk} und Paula
                     Beer-Hofmann\pwindex{Beer-Hofmann, Paula 25.02.1879 – 30.10.1939@\textsc{Beer-Hofmann, Paula} (25.02.1879 – 30.10.1939)|pwk} gemeinsam in Skodsborg\oindex{Skodsborg@\textbf{Skodsborg}|pwk}.
               }}}\label{K_L02772-9h}, w\damage{enn} ich Geld hätte, w\damage{as} noch zweifelhaft iſt. Ich würde dann \label{K_L02772-11v}\edtext{über Berlin\oindex{Berlin@\textbf{Berlin}|pw}
                  zurückreiſen}{\lemma{\textnormal{\emph{über Berlin
                  zurückreiſen}}}\Cendnote{\textnormal{siehe A. S.: \emph{Tagebuch}, 26. 8. 1896}}}\label{K_L02772-11h}, wo mich meine Mutter\pwindex{Goldmann, Clementine 1842-05-15 – 1924-02-24@\textsc{Goldmann, Clementine} (1842-05-15 – 1924-02-24)|pwv} und mein Onkel\pwindex{Mamroth, Hermann @\textsc{Mamroth, Hermann}|pwuv} erwarten.\pend
           \pstart
           Grüß’ Dich Gott, mein lieber Freund, und ſchreib’ mir bald!\pend
           \pstart
           Dein treuer {\\[\baselineskip]}\spacefill\mbox{Paul Goldmann}\pend
           \leftskip=0em{}
         
         \endnumbering\mylabel{h}\end{ledgroupsized}  \newcommand{\dateiname}{L02772}\newcommand{\titel}{Paul Goldmann an Arthur Schnitzler, 29. 4. [1896]}\newcommand{\editorInnen}{Martin Anton Müller und Laura Untner}%% latex-leseansicht-abspann.tex
%% Abspann für die Leseansicht.
%% Der Schalter \ifkorrekturansicht ist bereits durch den Vorspann gesetzt.

%% latex-abspann.tex
%% Gemeinsamer Abspann für Korrekturansicht und Leseansicht.
%% Setzt den Schalter \ifkorrekturansicht voraus (gesetzt in den
%% einbindenden Dateien latex-korrekturansicht-abspann.tex bzw.
%% latex-leseansicht-abspann.tex).
%% ---------------------------------------------------------------

\normalsize

% Das esempio-Environment wird nur in der Leseansicht benötigt
\ifkorrekturansicht\else
\newenvironment{esempio}[3]%
{
    \vspace{1.5ex}
    \rlap{\underline{#1}}
    \par
    \setlength{\parindent}{0cm}
    \nopagebreak
    \leftskip=#2cm
    \rightskip=#3cm
}
{
    \par
}
\fi

\doendnotes{C}
\bigskip
\vfill

\clearpage

\footnotesize

\ifkorrekturansicht
  \lohead{\textsc{register}}
\fi

% theindex-Environment neu definieren ohne reledmac
\makeatletter
\renewenvironment{theindex}{%
  \ifkorrekturansicht
    \section*{\indexname}%
  \else
    \subsubsection*{Index der erwähnten Entitäten}%
  \fi
  \setlength{\parindent}{0pt}%
  \setlength{\parskip}{0pt plus 0.3pt}%
  \let\item\@idxitem
}{%
  \ifkorrekturansicht\clearpage\fi
}
\makeatother

\IfFileExists{\jobname-pw.ind}{\input{\jobname-pw.ind}}{}

% Quellenangabe nur in der Leseansicht
\ifkorrekturansicht\else
% Fallback-Definitionen, falls die .tex-Datei \titel etc. nicht gesetzt hat
\providecommand{\titel}{}
\providecommand{\editorInnen}{}
\providecommand{\dateiname}{\jobname}

\vspace{3cm}

\vfill

\footnotesize
\textsc{Quelle}: \titel. Herausgegeben von {\editorInnen}. In: \emph{Arthur Schnitzler: Briefwechsel mit Autorinnen und Autoren}.
 Digitale Edition, https://schnitzler-briefe.acdh.oeaw.ac.at/{\dateiname}.html (Stand \today)
\fi

\end{document}


      