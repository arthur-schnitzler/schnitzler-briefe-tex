%% latex-korrekturansicht-vorspann.tex
%% Vorspann für die Korrekturansicht.
%% Lädt die gemeinsame Datei latex-vorspann.tex mit gesetztem Schalter.

\newif\ifkorrekturansicht
\korrekturansichttrue

\input{../tex-inputs/latex-vorspann}


\section[ Paul Goldmann an Arthur Schnitzler, 29. 4. {[}1896{]}]{L02772 Paul Goldmann an Arthur Schnitzler, 29. 4. {[}1896{]}}
\nopagebreak\mylabel{L02772v}
\rehead{ }\normalsize\beginnumbering\briefempfaengerindex{Schnitzler, Arthur@\textsc{Schnitzler, Arthur}!zzzGoldmann, Paul@\emph{von Paul Goldmann}!1896-04-291@{29. 4. {[}1896{]}}|(be}
\toendnotes[C]{\smallbreak\pagebreak[2]}\Standort{DLA, A:Schnitzler, HS.NZ85.1.3166.}
\physDesc{Brief, 2 Blätter, 8 Seiten, 2782 Zeichen
\newline{}Handschrift: blaue Tinte, deutsche Kurrent
\newline{}Schnitzler: 1) mit Bleistift das Jahr »96« vermerkt sowie »\noindent{}\textsc{Kerr}\pwindex{Kerr, Alfred 25.12.1867 – 12.10.1948@\textsc{Kerr, Alfred} (25.12.1867 – 12.10.1948), \emph{Schriftsteller/Schriftstellerin, Kritiker/Kritikerin}|pw}?{ / }\textsc{Altenb\pwindex{Altenberg, Peter 09.03.1859 – 08.01.1919@\textsc{Altenberg, Peter} (09.03.1859 – 08.01.1919), \emph{Schriftsteller/Schriftstellerin}|pw}?}{ / }\textcolor{gray}{Brief}« vermerkt  2) mit rotem Buntstift drei Unterstreichungen}\toendnotes[C]{\smallbreak}
\pstart
           {\pb}\textcolor{gray}{\textbf{\textbf{Frankfurter Zeitung\orgindex{Frankfurter Zeitung@Frankfurter Zeitung|pw}}}}\pend
           
\pstart
           \textcolor{gray}{\textbf{(\begin{otherlanguage}{french}Gazette de Francfort\end{otherlanguage}\orgindex{Frankfurter Zeitung@Frankfurter Zeitung|pw}).}}\pend
           
\pstart
           \textcolor{gray}{\textbf{\textbf{\begin{otherlanguage}{french}Fondateur M.\end{otherlanguage}{ }L. Sonnemann\pwindex{Sonnemann, Leopold 1831-10-29 – 1909-10-30@\textsc{Sonnemann, Leopold} (1831-10-29 – 1909-10-30), \emph{Journalist/Journalistin, Herausgeber/Herausgeberin}|pw}.}}}\pend
           
\pstart
           \begin{otherlanguage}{french}\textcolor{gray}{\textbf{Journal\pwindex{Frankfurter Zeitung@\emph{Frankfurter Zeitung}|pwv} politique,
                        financier,}}\end{otherlanguage}\pend
           
\pstart
           \begin{otherlanguage}{french}\textcolor{gray}{\textbf{commercial et littéraire.}}\end{otherlanguage}\pend
           
\pstart
           \begin{otherlanguage}{french}\textcolor{gray}{\textbf{\textbf{Paraissant trois fois par jour.}}}\end{otherlanguage}\pend
           
\pstart
           \begin{otherlanguage}{french}\textcolor{gray}{\textbf{\textbf{Bureau à Paris\oindex{Paris@\textbf{Paris}, \emph{P.PPLC}|pw}:}}}\end{otherlanguage}\pend
           
\pstart
           \begin{otherlanguage}{french}\textcolor{gray}{\textbf{\textbf{24. Rue Feydeau\oindex{rue Feydeau@\textbf{rue Feydeau}, \emph{Straße (K.STR)}|pw}.}}}\end{otherlanguage}\hfill \textsc{Paris\oindex{Paris@\textbf{Paris}, \emph{P.PPLC}|pw}}, 29. April.\pend
           
\pstart\center{}Mein lieber Freund,\pend\vspace{0.5em}
\pstart
           Ich war 14 Tage in Frankfurt\oindex{Frankfurt am Main@\textbf{Frankfurt am Main}, \emph{P.PPLA3}|pw}, habe geruht und
               neue Kräfte zu gewinnen geſtrebt. Nöthig wars. Zur Feier meiner Rückkunft fand eine
               feſtliche \label{K_L02772-1v}\edtext{Miniſter\pwindex{Bourgeois, Leon 1851-05-29 – 1925-09-29@\textsc{Bourgeois, Léon} (1851-05-29 – 1925-09-29), \emph{Politiker/Politikerin, Minister/Ministerin, Nobelpreisträger/Nobelpreisträgerin}|pwv}kriſis}{\lemma{\textnormal{\emph{Miniſterkriſis}}}\Cendnote{\textnormal{Mit dem 29. 4. 1896 endete das Ministerium von Léon Bourgeois\pwindex{Bourgeois, Leon 1851-05-29 – 1925-09-29@\textsc{Bourgeois, Léon} (1851-05-29 – 1925-09-29), \emph{Politiker/Politikerin, Minister/Ministerin, Nobelpreisträger/Nobelpreisträgerin}|pwk}.}}}\label{K_L02772-1} ſtatt. Ich ſtecke bis über die
               Ohren in Arbeit, und ſo komme ich erſt heut dazu, Dir
               für Deinen ſo überaus lieben Brief zu danken, den ich noch in Frankfurt\oindex{Frankfurt am Main@\textbf{Frankfurt am Main}, \emph{P.PPLA3}|pw} empfing. Als ich in Frankfurt\oindex{Frankfurt am Main@\textbf{Frankfurt am Main}, \emph{P.PPLA3}|pw} war, wurde gerade dein Stück\pwindex{Liebelei. Schauspiel in drei Akten@\emph{Liebelei. Schauspiel in drei Akten}|pwv} in Köln\oindex{Koeln@\textbf{Köln}, \emph{P.PPLA2}|pw} aufgeführt,
               und in der {\pb}Frankf. Zeit.\pwindex{Frankfurter Zeitung@\emph{Frankfurter Zeitung}|pw} erſchien eine kleine \label{K_L02772-2v}\edtext{Beſprechung\pwindex{Man schreibt uns aus Koeln]@\emph{[Man schreibt uns aus Köln]}|pwv}}{\lemma{\textnormal{\emph{Beſprechung}}}\Cendnote{\textnormal{\emph{[Man schreibt uns aus Köln]}\pwindex{Man schreibt uns aus Koeln]@\emph{[Man schreibt uns aus Köln]}|pwk}. In: \emph{Frankfurter Zeitung}\pwindex{Frankfurter Zeitung@\emph{Frankfurter Zeitung}|pwk}, Jg. 40, Nr. 103,
                        13. 4. 1896, Abendblatt, S. 2.}}}\label{K_L02772-2}, die ich hier
               einfüge, da Du ſie vielleicht überſehen haſt.\pend
           {\vspace{1\baselineskip}}
\pstart
           \textcolor{gray}{\textbf{Man ſchreibt uns aus \so{Köln}\oindex{Koeln@\textbf{Köln}, \emph{P.PPLA2}|pw}, 11. April: Schnitzler’s Schauſpiel\pwindex{Liebelei. Schauspiel in drei Akten@\emph{Liebelei. Schauspiel in drei Akten}|pwv} »\so{Liebelei}\pwindex{Liebelei. Schauspiel in drei Akten@\emph{Liebelei. Schauspiel in drei Akten}|pw}« ging geſtern zum erſten Mal in Szene und erzielte einen ſehr
                  ſtarken Erfolg. Die Mitwirkenden wurden nach dem letzten Akt\pwindex{Liebelei. Schauspiel in drei Akten@\emph{Liebelei. Schauspiel in drei Akten}|pwv} fünfmal gerufen. Die Darſtellung war
                  im Ganzen recht befriedigend. Die Chriſtine\pwindex{Liebelei. Schauspiel in drei Akten@\emph{Liebelei. Schauspiel in drei Akten}|pwv} wußte Frau Doré\pwindex{Dore, Adele 1869-04-09 – 1918-02-09@\textsc{Doré, Adele} (1869-04-09 – 1918-02-09), \emph{Schauspieler/Schauspielerin}|pw} in ergreifender Weise zu geſtalten. In der Mizi\pwindex{Liebelei. Schauspiel in drei Akten@\emph{Liebelei. Schauspiel in drei Akten}|pwv} des Frl. \so{Glümer}\pwindex{Gluemer, Marie 03.07.1867 – 16.11.1925@\textsc{Glümer, Marie} (03.07.1867 – 16.11.1925), \emph{Schauspieler/Schauspielerin}|pw} und in dem Theodor\pwindex{Liebelei. Schauspiel in drei Akten@\emph{Liebelei. Schauspiel in drei Akten}|pwv}
                  des Hrn. \so{Leyrer}\pwindex{Leyrer, Rudolf 1857-08-19 – 1939-12-26@\textsc{Leyrer, Rudolf} (1857-08-19 – 1939-12-26), \emph{Schauspieler/Schauspielerin}|pw} fand die Wien\oindex{Wien@\textbf{Wien}, \emph{A.ADM2}|pw}er Leichtlebigkeit ihre
                  angemeſſene Vertretung. Fein und discret gab Herr \so{Beck}\pwindex{Beck @\textsc{Beck}, \emph{Schauspieler/Schauspielerin}|pw} den alten Musiker\pwindex{Liebelei. Schauspiel in drei Akten@\emph{Liebelei. Schauspiel in drei Akten}|pwv};
                  auch der Fritz\pwindex{Liebelei. Schauspiel in drei Akten@\emph{Liebelei. Schauspiel in drei Akten}|pwv} des Hrn. \so{Monnard}\pwindex{Monnard, Heinz 31.12.1873 – 11.07.1912@\textsc{Monnard, Heinz} (31.12.1873 – 11.07.1912), \emph{Schauspieler/Schauspielerin}|pw} war nicht ohne tiefere Wirkung. –}}\pend
           {\vspace{1\baselineskip}}
\pstart
           auch lege ich einen \label{K_L02772-3v}\edtext{Brief}{\lemma{\textnormal{\emph{Brief}}}\Cendnote{\textnormal{Goldmann\pwindex{Goldmann, Paul 31.01.1865 – 25.09.1935@\textsc{Goldmann, Paul} (31.01.1865 – 25.09.1935), \emph{Schriftsteller/Schriftstellerin, Journalist/Journalistin}|pwk} hatte vergessen, ihn beizulegen (vgl. Paul Goldmann an Arthur Schnitzler, 3. 4. [1895]).}}}\label{K_L02772-3} des Herrn \textsc{Christian Schefer\pwindex{Schefer, Christian 1866-07-14 – Februar 1944@\textsc{Schefer, Christian} (1866-07-14 – Februar 1944), \emph{Journalist/Journalistin, Lehrer/Lehrerin}|pw}} bei, den ich noch in Frankfurt\oindex{Frankfurt am Main@\textbf{Frankfurt am Main}, \emph{P.PPLA3}|pw} erhielt.
               Schicke ihm ein Exemplar von »\textsc{Mourir\pwindex{Mourir. Roman@\emph{Mourir. Roman}|pw}}«, ebenſo eines an \textsc{Lalo\pwindex{Lalo, Pierre 1866-09-06 – 1943-06-09@\textsc{Lalo, Pierre} (1866-09-06 – 1943-06-09), \emph{Kritiker/Kritikerin}|pw}}, ein drittes an \textsc{M. de
                     Wyzewa\pwindex{Wyzewa, Theodore de 1862-09-12 – 1917-04-07@\textsc{Wyzewa, Théodore de} (1862-09-12 – 1917-04-07), \emph{Schriftsteller/Schriftstellerin, Journalist/Journalistin}|pw}, 9. Rue Coëtlogon\oindex{Rue Coetlogon@\textbf{Rue Coëtlogon}, \emph{Straße (K.STR)}|pw}}. Auch ſchicke mir noch zwei oder {\pb}drei \substVorne{}\textsuperscript{B\textcolor{gray}{üch}e}\substDazwischen{}Exemplare\pwindex{Mourir. Roman@\emph{Mourir. Roman}|pwv}\substHinten{} zur Propaganda. Das Buch\pwindex{Mourir. Roman@\emph{Mourir. Roman}|pwv} iſt ſehr gut ausgeſtattet und ſieht recht vornehm aus. Ferner ſende ich
               Dir die Briefe des Herrn \textsc{de Riaz\pwindex{Riaz, Henri de 1871 – 1951@\textsc{Riaz, Henri de} (1871 – 1951), \emph{Dichter/Dichterin}|pw}} zurück. Laß’ die \label{K_L02772-4v}\edtext{Überſetzungs\pwindex{Liebelei. Schauspiel in drei Akten@\emph{Liebelei. Schauspiel in drei Akten}|pwv}-Angelegenheit}{\lemma{\textnormal{\emph{Überſetzungs-Angelegenheit}}}\Cendnote{\textnormal{Siehe Paul Goldmann an Arthur Schnitzler, 5. 12. [1895].
               }}}\label{K_L02772-4} noch ruhn und antworte aufſchiebend. Endlich finde ich noch in meinen
               Papieren die \label{K_L02772-5v}\edtext{Kritik\pwindex{Burgtheater [Rechte der Seele, Liebelei]@\emph{Burgtheater [Rechte der Seele, Liebelei]}|pwuv}}{\lemma{\textnormal{\emph{Kritik}}}\Cendnote{\textnormal{Alfred Freiherr von Berger\pwindex{Berger, Alfred von 30.04.1853 – 24.08.1912@\textsc{Berger, Alfred von} (30.04.1853 – 24.08.1912), \emph{Schriftsteller/Schriftstellerin, Journalist/Journalistin, Theaterleiter/Theaterleiterin}|pwk}: \emph{Burgtheater}\pwindex{Burgtheater [Rechte der Seele, Liebelei]@\emph{Burgtheater [Rechte der Seele, Liebelei]}|pwk}. In: \emph{Montags-Revue}\pwindex{Montags-Revue. Wochenschrift fuer Politik, Finanzen, Kunst und Literatur@\emph{Montags-Revue. Wochenschrift für Politik, Finanzen, Kunst und Literatur}|pwk}, Jg. 26, Nr. 41, 14. 10. 1895, S. 1–4.}}}\label{K_L02772-5} des Baron \textsc{Berger}\pwindex{Berger, Alfred von 30.04.1853 – 24.08.1912@\textsc{Berger, Alfred von} (30.04.1853 – 24.08.1912), \emph{Schriftsteller/Schriftstellerin, Journalist/Journalistin, Theaterleiter/Theaterleiterin}|pw}, die ich Dir gleichfalls zurückſende.\pend
           
\pstart
           Zu erzählen habe ich Dir nichts. Mein Leben iſt vollſtändig unintereſſant. Es gibt
               nichts Neues und wird nie etwas Neues geben, {\pb}außer
               irgend einem definitiven Unglück. Intereſſant iſt nur Dein Leben, und ich möchte ſehr
               viel darüber wiſſen. Haſt Du alſo \label{K_L02772-6v}\edtext{zum
               dritten Mal angeſangen, das Stück\pwindex{Freiwild. Schauspiel in 3 Akten@\emph{Freiwild. Schauspiel in 3 Akten}|pwv} zu ſchreiben}{\lemma{\textnormal{\emph{zum … ſchreiben}}}\Cendnote{\textnormal{Siehe A. S.: \emph{Tagebuch}, 27. 4. 1896.
               }}}\label{K_L02772-6}? Könnte man nicht doch das Manuſcript\pwindex{Freiwild. Schauspiel in 3 Akten@\emph{Freiwild. Schauspiel in 3 Akten}|pwv} ſehen? Wirſt Du \label{K_L02772-7v}\edtext{in
               die »Zeit\orgindex{Zeit. Wiener Wochenschrift@Die Zeit. Wiener Wochenschrift|pw}« eintreten}{\lemma{\textnormal{\emph{in
               die »Zeit« eintreten}}}\Cendnote{\textnormal{Dazu kam es nicht.}}}\label{K_L02772-7}, jetzt nach \textsc{Kanners\pwindex{Kanner, Heinrich 09.11.1864 – 15.02.1930@\textsc{Kanner, Heinrich} (09.11.1864 – 15.02.1930), \emph{Herausgeber/Herausgeberin, Publizist/Publizistin}|pw}} Rückkehr? Und wie iſt ſonſt
               Daſeinsführung und Stimmung?\pend
           
\pstart
           Recht geärgert habe ich mich, als ich Deinen {\pb}\label{K_L02772-8v}\edtext{Namen im »\textsc{Simplicissimus}\pwindex{Simplicissimus@\emph{Simplicissimus}|pw}«}{\lemma{\textnormal{\emph{Namen im »Simplicissimus«}}}\Cendnote{\textnormal{Arthur Schnitzler: \emph{Die überspannte Person}\pwindex{ueberspannte Person@\emph{Die überspannte Person}|pwk}. In: \emph{Simplicissimus}\pwindex{Simplicissimus@\emph{Simplicissimus}|pwk}, Jg. 1, H. 3, 18. 4. 1896, S. 3 u. 6.}}}\label{K_L02772-8}{ }fand\pwindex{ueberspannte Person@\emph{Die überspannte Person}|pwv}. Dieſer Lausbub’ \textsc{Langen\pwindex{Langen, Albert 1869-07-08 – 1909-04-30@\textsc{Langen, Albert} (1869-07-08 – 1909-04-30), \emph{Verleger/Verlegerin}|pw}}, der mir i\substVorne{}\textsuperscript{m}\substDazwischen{}n\substHinten{}{ }\textsc{Paris\oindex{Paris@\textbf{Paris}, \emph{P.PPLC}|pw}}, wenn ich ihn dazu drängte, Deine Bücher in Verlag zu nehmen, ſtets antwortete:
               Du könnteſt \label{K_L02772-9v}\edtext{nicht deutſch
                  ſchreiben}{\lemma{\textnormal{\emph{nicht deutſch
                  ſchreiben}}}\Cendnote{\textnormal{eventuell auf die
                  Verwendung von Austriazismen gemünzt}}}\label{K_L02772-9}, – iſt jetzt in der Lage, ſein neues
                  Unternehmen\orgindex{Simplicissimus@Simplicissimus|pwv} mit Deinem
               jungen \textsc{Rénommée} aufzuputzen. Das hat er wahrlich nicht
               verdient. Warum haſt {\pb}Du ihm den Beitrag\pwindex{ueberspannte Person@\emph{Die überspannte Person}|pwv} gegeben\damage{?} Ich bekam in Deutſchland\oindex{Deutschland@\textbf{Deutschland}, \emph{A.PCLI}|pw} durch Zufall
               das Heft\pwindex{Zukunft@\emph{Die Zukunft}|pwv} der »Zukunft\pwindex{Zukunft@\emph{Die Zukunft}|pw}« in die Hand, das \label{K_L02772-10v}\edtext{\textsc{Hardens\pwindex{Harden, Maximilian 20.10.1861 – 30.10.1927@\textsc{Harden, Maximilian} (20.10.1861 – 30.10.1927), \emph{Schriftsteller/Schriftstellerin, Publizist/Publizistin}|pw}}{ }Kritik\pwindex{Theaternotizen [Liebelei]@\emph{Theaternotizen [Liebelei]}|pwv} über »Liebelei\pwindex{Liebelei. Schauspiel in drei Akten@\emph{Liebelei. Schauspiel in drei Akten}|pw}«}{\lemma{\textnormal{\emph{Hardens … »Liebelei«}}}\Cendnote{\textnormal{Maximilian Harden\pwindex{Harden, Maximilian 20.10.1861 – 30.10.1927@\textsc{Harden, Maximilian} (20.10.1861 – 30.10.1927), \emph{Schriftsteller/Schriftstellerin, Publizist/Publizistin}|pwk}: \emph{Theaternotizen}\pwindex{Theaternotizen [Liebelei]@\emph{Theaternotizen [Liebelei]}|pwk}. In: \emph{Die
                        Zukunft}\pwindex{Zukunft@\emph{Die Zukunft}|pwk}, Jg. 5, Bd. 14, 14. 3. 1896,
                     S. 527–528.}}}\label{K_L02772-10} enthält. Das iſt doch eine recht unverſtändige Kritik\pwindex{Theaternotizen [Liebelei]@\emph{Theaternotizen [Liebelei]}|pw}, die Dich völlig unterſchätzt. Biſt Du
               trotzdem bei Deiner großen Meinung über \textsc{Harden\pwindex{Harden, Maximilian 20.10.1861 – 30.10.1927@\textsc{Harden, Maximilian} (20.10.1861 – 30.10.1927), \emph{Schriftsteller/Schriftstellerin, Publizist/Publizistin}|pw}} geblieben?\pend
           
\pstart
           Aber ich will nicht fragen, und Du ſollſt den \strikeout{Ih}
               Inhalt des nächſten Briefes nach {\pb}\damage{freier} Wahl zuſammen\damage{th}un. Schreib’ mir nur recht viel über Dich.\pend
           
\pstart
           Und wie gehts dem \textsc{Richard\pwindex{Beer-Hofmann, Richard 1866-07-11 – 1945-09-26@\textsc{Beer-Hofmann, Richard} (1866-07-11 – 1945-09-26), \emph{Schriftsteller/Schriftstellerin}|pw}}? Er bringts wirklich fertig, mir keine Zeile zu ſchreiben. Erwartet hab’ ichs,
               aber es erſtaunt mich doch. Es iſt immerhin der ſchönſte Fall von Faulheit, der mir
               in meinem Leben vorgekommen iſt.\pend
           
\pstart
           Gern ginge ich mit früh im August{ }{\pb}nach \label{K_L02772-11v}\edtext{Dänemark\oindex{Daenemark@\textbf{Dänemark}, \emph{A.PCLI}|pw}}{\lemma{\textnormal{\emph{Dänemark}}}\Cendnote{\textnormal{Vom 5. 8. 1896 bis zum 21. 8. 1896 waren Schnitzler,
                     Goldmann\pwindex{Goldmann, Paul 31.01.1865 – 25.09.1935@\textsc{Goldmann, Paul} (31.01.1865 – 25.09.1935), \emph{Schriftsteller/Schriftstellerin, Journalist/Journalistin}|pwk}, Richard\pwindex{Beer-Hofmann, Richard 1866-07-11 – 1945-09-26@\textsc{Beer-Hofmann, Richard} (1866-07-11 – 1945-09-26), \emph{Schriftsteller/Schriftstellerin}|pwk} und Paula
                     Beer-Hofmann\pwindex{Beer-Hofmann, Paula 25.02.1879 – 30.10.1939@\textsc{Beer-Hofmann, Paula} (25.02.1879 – 30.10.1939)|pwk} gemeinsam in Skodsborg\oindex{Skodsborg@\textbf{Skodsborg}, \emph{P.PPL}|pwk}.
               }}}\label{K_L02772-11}, w\damage{enn} ich Geld hätte, w\damage{as} noch zweifelhaft iſt. Ich würde dann \label{K_L02772-12v}\edtext{über Berlin\oindex{Berlin@\textbf{Berlin}, \emph{P.PPLC}|pw}
                  zurückreiſen}{\lemma{\textnormal{\emph{über Berlin
                  zurückreiſen}}}\Cendnote{\textnormal{Siehe A. S.: \emph{Tagebuch}, 26. 8. 1896.
               }}}\label{K_L02772-12}, wo mich meine Mutter\pwindex{Goldmann, Clementine 1842-05-15 – 1924-02-24@\textsc{Goldmann, Clementine} (1842-05-15 – 1924-02-24)|pwv} und mein Onkel\pwindex{Mamroth, Hermann @\textsc{Mamroth, Hermann}|pwuv} erwarten.\pend
           
\pstart
           Grüß’ Dich Gott, mein lieber Freund, und ſchreib’ mir bald!\pend
           
\pstart
           Dein treuer {\\[\baselineskip]}\spacefill\mbox{Paul Goldmann}\pend
           \leftskip=0em{}\selectlanguage{ngerman}\endnumbering\briefempfaengerindex{Schnitzler, Arthur@\textsc{Schnitzler, Arthur}!zzzGoldmann, Paul@\emph{von Paul Goldmann}!1896-04-291@{29. 4. {[}1896{]}}|)be}\mylabel{L02772h}  \normalsize

\doendnotes{C}
\bigskip
\vfill

\clearpage

\footnotesize

\lohead{\textsc{register}}

% Definiere theindex-Environment komplett neu ohne reledmac
\makeatletter
\renewenvironment{theindex}{%
  \section*{\indexname}%
  \setlength{\parindent}{0pt}%
  \setlength{\parskip}{0pt plus 0.3pt}%
  \let\item\@idxitem
}{%
  \clearpage
}
\makeatother

\IfFileExists{\jobname-pw.ind}{\input{\jobname-pw.ind}}{}

\end{document}

      