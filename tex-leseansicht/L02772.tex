%% latex-leseansicht-vorspann.tex
%% Vorspann für die Leseansicht.
%% Lädt die gemeinsame Datei latex-vorspann.tex mit nicht gesetztem Schalter.

\newif\ifkorrekturansicht
\korrekturansichtfalse

\input{../tex-inputs/latex-vorspann}


\section[ Paul Goldmann an Arthur Schnitzler, 29. 4. {[}1896{]}]{L02772 Paul Goldmann an Arthur Schnitzler,  29. 4. [1896]}
\nopagebreak\mylabel{L02772v}
\rehead{ }\normalsize\beginnumbering\briefempfaengerindex{Schnitzler, Arthur@\textsc{Schnitzler, Arthur}!zzzGoldmann, Paul@\emph{von Paul Goldmann}!1896-04-291@{29. 4. [1896]}|(be}
\toendnotes[C]{\smallbreak\pagebreak[2]}
\correspDesc{Versand  durch Paul Goldmann am 29. 4. [1896] in Paris
\newline{}Erhalt  durch Arthur Schnitzler im Zeitraum [30. 4. 1896
                  – 4. 5. 1896?] in Wien}\toendnotes[C]{\smallbreak}
\Standort{DLA, A:Schnitzler, HS.NZ85.1.3166.}
\physDesc{Brief, 2 Blätter, 8 Seiten, 2782 Zeichen
\newline{}Handschrift: blaue Tinte, deutsche Kurrent
\newline{}Schnitzler: 1) mit Bleistift das Jahr »96« vermerkt sowie »\noindent{}\textsc{Kerr}\pwindex{Kerr, Alfred 25.\,12.\,1867 Breslau – 12.\,10.\,1948 Hamburg@\textsc{Kerr, Alfred} (25.\,12.\,1867 Breslau – 12.\,10.\,1948 Hamburg), \emph{Schriftsteller, Kritiker}|pw}?{ / }\textsc{Altenb\pwindex{Altenberg, Peter 9.\,3.\,1859 Wien – 8.\,1.\,1919 ebd.@\textsc{Altenberg, Peter} (9.\,3.\,1859 Wien – 8.\,1.\,1919 ebd.), \emph{Schriftsteller}|pw}?}{ / }\textcolor{gray}{Brief}« vermerkt  2) mit rotem Buntstift drei Unterstreichungen}\toendnotes[C]{\smallbreak}
\pstart
           {\pb}\textcolor{gray}{\textbf{\textbf{Frankfurter Zeitung\orgindex{Frankfurter Zeitung@Frankfurter Zeitung|pw}}}}\pend
           
\pstart
           \textcolor{gray}{\textbf{(\begin{otherlanguage}{french}Gazette de Francfort\end{otherlanguage}\orgindex{Frankfurter Zeitung@Frankfurter Zeitung|pw}).}}\pend
           
\pstart
           \textcolor{gray}{\textbf{\textbf{\begin{otherlanguage}{french}Fondateur M.\end{otherlanguage}{ }L. Sonnemann\pwindex{Sonnemann, Leopold 29.\,10.\,1831 Höchberg – 30.\,10.\,1909 Frankfurt am Main@\textsc{Sonnemann, Leopold} (29.\,10.\,1831 Höchberg – 30.\,10.\,1909 Frankfurt am Main), \emph{Journalist, Herausgeber}|pw}.}}}\pend
           
\pstart
           \begin{otherlanguage}{french}\textcolor{gray}{\textbf{Journal\pwindex{Frankfurter Zeitung@\emph{Frankfurter Zeitung}|pwv} politique,
                        financier,}}\end{otherlanguage}\pend
           
\pstart
           \begin{otherlanguage}{french}\textcolor{gray}{\textbf{commercial et littéraire.}}\end{otherlanguage}\pend
           
\pstart
           \begin{otherlanguage}{french}\textcolor{gray}{\textbf{\textbf{Paraissant trois fois par jour.}}}\end{otherlanguage}\pend
           
\pstart
           \begin{otherlanguage}{french}\textcolor{gray}{\textbf{\textbf{Bureau à Paris\oindex{Paris@\textbf{Paris}, \emph{Hauptstadt}|pw}:}}}\end{otherlanguage}\pend
           
\pstart
           \begin{otherlanguage}{french}\textcolor{gray}{\textbf{\textbf{24. Rue Feydeau\oindex{rue Feydeau@\textbf{rue Feydeau}, \emph{Straße}|pw}.}}}\end{otherlanguage}\hfill \textsc{Paris\oindex{Paris@\textbf{Paris}, \emph{Hauptstadt}|pw}}, 29. April.\pend
           
\pstart\center{}Mein lieber Freund,\pend\vspace{0.5em}
\pstart
           Ich war 14 Tage in Frankfurt\oindex{Frankfurt am Main@\textbf{Frankfurt am Main}, \emph{Hauptstadt}|pw}, habe geruht und
               neue Kräfte zu gewinnen geſtrebt. Nöthig wars. Zur Feier meiner Rückkunft fand eine
               feſtliche \label{K_L02772-1v}\edtext{Miniſter\pwindex{Bourgeois, Léon 29.\,5.\,1851 Paris – 29.\,9.\,1925 Épernay@\textsc{Bourgeois, Léon} (29.\,5.\,1851 Paris – 29.\,9.\,1925 Épernay), \emph{Politiker, Minister, Nobelpreisträger}|pwv}kriſis}{\lemma{\textnormal{\emph{Ministerkrisis}}}\Cendnote{\textnormal{Mit dem 29. 4. 1896 endete das Ministerium von Léon Bourgeois\pwindex{Bourgeois, Léon 29.\,5.\,1851 Paris – 29.\,9.\,1925 Épernay@\textsc{Bourgeois, Léon} (29.\,5.\,1851 Paris – 29.\,9.\,1925 Épernay), \emph{Politiker, Minister, Nobelpreisträger}|pwk}.}}}\label{K_L02772-1}{ }ſtatt. Ich{ }ſtecke bis über die
               Ohren in Arbeit, und{ }ſo komme ich erſt heut dazu, Dir
               für Deinen{ }ſo überaus lieben Brief zu danken, den ich noch in Frankfurt\oindex{Frankfurt am Main@\textbf{Frankfurt am Main}, \emph{Hauptstadt}|pw} empfing. Als ich in Frankfurt\oindex{Frankfurt am Main@\textbf{Frankfurt am Main}, \emph{Hauptstadt}|pw} war, wurde gerade dein Stück\pwindex{Schnitzler, Arthur 15.\,5.\,1862 Wien – 21.\,10.\,1931 ebd.@\textsc{Schnitzler, Arthur} (15.\,5.\,1862 Wien – 21.\,10.\,1931 ebd.), \emph{Schriftsteller, Mediziner}!Liebelei. Schauspiel in drei Akten@\strich\emph{Liebelei. Schauspiel in drei Akten}|pwv} in Köln\oindex{Köln@\textbf{Köln}, \emph{Hauptstadt}|pw} aufgeführt,
               und in der {\pb}Frankf. Zeit.\pwindex{Frankfurter Zeitung@\emph{Frankfurter Zeitung}|pw} erſchien eine kleine \label{K_L02772-2v}\edtext{Beſprechung\pwindex{Man schreibt uns aus Köln]@\emph{[Man schreibt uns aus Köln]}|pwv}}{\lemma{\textnormal{\emph{Besprechung}}}\Cendnote{\textnormal{\emph{[Man schreibt uns aus Köln]}\pwindex{Man schreibt uns aus Köln]@\emph{[Man schreibt uns aus Köln]}|pwk}. In: \emph{Frankfurter Zeitung}\pwindex{Frankfurter Zeitung@\emph{Frankfurter Zeitung}|pwk}, Jg. 40, Nr. 103,
                        13. 4. 1896, Abendblatt, S. 2.}}}\label{K_L02772-2}, die ich hier
               einfüge, da Du{ }ſie vielleicht überſehen haſt.\pend
           {\vspace{1\baselineskip}}
\pstart
           \textcolor{gray}{\textbf{Man{ }ſchreibt uns aus \so{Köln}\oindex{Köln@\textbf{Köln}, \emph{Hauptstadt}|pw}, 11. April: Schnitzler’s Schauſpiel\pwindex{Schnitzler, Arthur 15.\,5.\,1862 Wien – 21.\,10.\,1931 ebd.@\textsc{Schnitzler, Arthur} (15.\,5.\,1862 Wien – 21.\,10.\,1931 ebd.), \emph{Schriftsteller, Mediziner}!Liebelei. Schauspiel in drei Akten@\strich\emph{Liebelei. Schauspiel in drei Akten}|pwv} »\so{Liebelei}\pwindex{Schnitzler, Arthur 15.\,5.\,1862 Wien – 21.\,10.\,1931 ebd.@\textsc{Schnitzler, Arthur} (15.\,5.\,1862 Wien – 21.\,10.\,1931 ebd.), \emph{Schriftsteller, Mediziner}!Liebelei. Schauspiel in drei Akten@\strich\emph{Liebelei. Schauspiel in drei Akten}|pw}« ging geſtern zum erſten Mal in Szene und erzielte einen{ }ſehr{ }ſtarken Erfolg. Die Mitwirkenden wurden nach dem letzten Akt\pwindex{Schnitzler, Arthur 15.\,5.\,1862 Wien – 21.\,10.\,1931 ebd.@\textsc{Schnitzler, Arthur} (15.\,5.\,1862 Wien – 21.\,10.\,1931 ebd.), \emph{Schriftsteller, Mediziner}!Liebelei. Schauspiel in drei Akten@\strich\emph{Liebelei. Schauspiel in drei Akten}|pwv} fünfmal gerufen. Die Darſtellung war
                  im Ganzen recht befriedigend. Die Chriſtine\pwindex{Schnitzler, Arthur 15.\,5.\,1862 Wien – 21.\,10.\,1931 ebd.@\textsc{Schnitzler, Arthur} (15.\,5.\,1862 Wien – 21.\,10.\,1931 ebd.), \emph{Schriftsteller, Mediziner}!Liebelei. Schauspiel in drei Akten@\strich\emph{Liebelei. Schauspiel in drei Akten}|pwv} wußte Frau Doré\pwindex{Doré, Adele 9.\,4.\,1869 Wien – 9.\,2.\,1918 Berlin@\textsc{Doré, Adele} (9.\,4.\,1869 Wien – 9.\,2.\,1918 Berlin), \emph{Schauspielerin}|pw} in ergreifender Weise zu geſtalten. In der Mizi\pwindex{Schnitzler, Arthur 15.\,5.\,1862 Wien – 21.\,10.\,1931 ebd.@\textsc{Schnitzler, Arthur} (15.\,5.\,1862 Wien – 21.\,10.\,1931 ebd.), \emph{Schriftsteller, Mediziner}!Liebelei. Schauspiel in drei Akten@\strich\emph{Liebelei. Schauspiel in drei Akten}|pwv} des Frl. \so{Glümer}\pwindex{Glümer, Marie 3.\,7.\,1867 Wien – 16.\,11.\,1925 München@\textsc{Glümer, Marie} (3.\,7.\,1867 Wien – 16.\,11.\,1925 München), \emph{Schauspielerin}|pw} und in dem Theodor\pwindex{Schnitzler, Arthur 15.\,5.\,1862 Wien – 21.\,10.\,1931 ebd.@\textsc{Schnitzler, Arthur} (15.\,5.\,1862 Wien – 21.\,10.\,1931 ebd.), \emph{Schriftsteller, Mediziner}!Liebelei. Schauspiel in drei Akten@\strich\emph{Liebelei. Schauspiel in drei Akten}|pwv}
                  des Hrn. \so{Leyrer}\pwindex{Leyrer, Rudolf 19.\,8.\,1857 Wien – 26.\,12.\,1939 ebd.@\textsc{Leyrer, Rudolf} (19.\,8.\,1857 Wien – 26.\,12.\,1939 ebd.), \emph{Schauspieler}|pw} fand die Wien\oindex{Wien@\textbf{Wien}, \emph{Verwaltungsgebiet}|pw}er Leichtlebigkeit ihre
                  angemeſſene Vertretung. Fein und discret gab Herr \so{Beck}\pwindex{Beck @\textsc{Beck}, \emph{Schauspieler}|pw} den alten Musiker\pwindex{Schnitzler, Arthur 15.\,5.\,1862 Wien – 21.\,10.\,1931 ebd.@\textsc{Schnitzler, Arthur} (15.\,5.\,1862 Wien – 21.\,10.\,1931 ebd.), \emph{Schriftsteller, Mediziner}!Liebelei. Schauspiel in drei Akten@\strich\emph{Liebelei. Schauspiel in drei Akten}|pwv};
                  auch der Fritz\pwindex{Schnitzler, Arthur 15.\,5.\,1862 Wien – 21.\,10.\,1931 ebd.@\textsc{Schnitzler, Arthur} (15.\,5.\,1862 Wien – 21.\,10.\,1931 ebd.), \emph{Schriftsteller, Mediziner}!Liebelei. Schauspiel in drei Akten@\strich\emph{Liebelei. Schauspiel in drei Akten}|pwv} des Hrn. \so{Monnard}\pwindex{Monnard, Heinz 31.\,12.\,1873 Frankfurt am Main – 11.\,7.\,1912 Berlin@\textsc{Monnard, Heinz} (31.\,12.\,1873 Frankfurt am Main – 11.\,7.\,1912 Berlin), \emph{Schauspieler}|pw} war nicht ohne tiefere Wirkung. –}}\pend
           {\vspace{1\baselineskip}}
\pstart
           auch lege ich einen \label{K_L02772-3v}\edtext{Brief}{\lemma{\textnormal{\emph{Brief}}}\Cendnote{\textnormal{Goldmann\pwindex{Goldmann, Paul 31.\,1.\,1865 Breslau – 25.\,9.\,1935 Wien@\textsc{Goldmann, Paul} (31.\,1.\,1865 Breslau – 25.\,9.\,1935 Wien), \emph{Schriftsteller, Journalist}|pwk} hatte vergessen, ihn beizulegen (vgl. XXXX Auszeichnungsfehler: Dokument L02733 nicht gefunden).}}}\label{K_L02772-3} des Herrn \textsc{Christian Schefer\pwindex{Schefer, Christian 14.\,7.\,1866 Paris – Februar 1944 Marokko@\textsc{Schefer, Christian} (14.\,7.\,1866 Paris – Februar 1944 Marokko), \emph{Journalist, Lehrer}|pw}} bei, den ich noch in Frankfurt\oindex{Frankfurt am Main@\textbf{Frankfurt am Main}, \emph{Hauptstadt}|pw} erhielt.
               Schicke ihm ein Exemplar von »\textsc{Mourir\pwindex{Schnitzler, Arthur 15.\,5.\,1862 Wien – 21.\,10.\,1931 ebd.@\textsc{Schnitzler, Arthur} (15.\,5.\,1862 Wien – 21.\,10.\,1931 ebd.), \emph{Schriftsteller, Mediziner}!Mourir. Roman@\strich\emph{Mourir. Roman}|pw}}«, ebenſo eines an \textsc{Lalo\pwindex{Lalo, Pierre 6.\,9.\,1866 Puteaux – 9.\,6.\,1943 Paris@\textsc{Lalo, Pierre} (6.\,9.\,1866 Puteaux – 9.\,6.\,1943 Paris), \emph{Kritiker}|pw}}, ein drittes an \textsc{M. de
                     Wyzewa\pwindex{Wyzewa, Théodore de 12.\,9.\,1862 Kalush – 7.\,4.\,1917 Paris@\textsc{Wyzewa, Théodore de} (12.\,9.\,1862 Kalush – 7.\,4.\,1917 Paris), \emph{Schriftsteller, Journalist}|pw}, 9. Rue Coëtlogon\oindex{Rue Coëtlogon@\textbf{Rue Coëtlogon}, \emph{Straße}|pw}}. Auch{ }ſchicke mir noch zwei oder {\pb}drei \substVorne{}\textsuperscript{B\textcolor{gray}{üch}e}\substDazwischen{}Exemplare\pwindex{Schnitzler, Arthur 15.\,5.\,1862 Wien – 21.\,10.\,1931 ebd.@\textsc{Schnitzler, Arthur} (15.\,5.\,1862 Wien – 21.\,10.\,1931 ebd.), \emph{Schriftsteller, Mediziner}!Mourir. Roman@\strich\emph{Mourir. Roman}|pwv}\substHinten{} zur Propaganda. Das Buch\pwindex{Schnitzler, Arthur 15.\,5.\,1862 Wien – 21.\,10.\,1931 ebd.@\textsc{Schnitzler, Arthur} (15.\,5.\,1862 Wien – 21.\,10.\,1931 ebd.), \emph{Schriftsteller, Mediziner}!Mourir. Roman@\strich\emph{Mourir. Roman}|pwv} iſt{ }ſehr gut ausgeſtattet und{ }ſieht recht vornehm aus. Ferner{ }ſende ich
               Dir die Briefe des Herrn \textsc{de Riaz\pwindex{Riaz, Henri de 1871 Lyon – 1951 Lausanne@\textsc{Riaz, Henri de} (1871 Lyon – 1951 Lausanne), \emph{Dichter}|pw}} zurück. Laß’ die \label{K_L02772-4v}\edtext{Überſetzungs\pwindex{Schnitzler, Arthur 15.\,5.\,1862 Wien – 21.\,10.\,1931 ebd.@\textsc{Schnitzler, Arthur} (15.\,5.\,1862 Wien – 21.\,10.\,1931 ebd.), \emph{Schriftsteller, Mediziner}!Liebelei. Schauspiel in drei Akten@\strich\emph{Liebelei. Schauspiel in drei Akten}|pwv}-Angelegenheit}{\lemma{\textnormal{\emph{Übersetzungs-Angelegenheit}}}\Cendnote{\textnormal{Siehe XXXX Auszeichnungsfehler: Dokument L02758 nicht gefunden.
               }}}\label{K_L02772-4} noch ruhn und antworte aufſchiebend. Endlich finde ich noch in meinen
               Papieren die \label{K_L02772-5v}\edtext{Kritik\pwindex{Berger, Alfred von 30.\,4.\,1853 Wien – 24.\,8.\,1912 ebd.@\textsc{Berger, Alfred von} (30.\,4.\,1853 Wien – 24.\,8.\,1912 ebd.), \emph{Schriftsteller, Journalist, Theaterleiter}!Burgtheater [Rechte der Seele, Liebelei]@\strich\emph{Burgtheater [Rechte der Seele, Liebelei]}|pwuv}}{\lemma{\textnormal{\emph{Kritik}}}\Cendnote{\textnormal{Alfred Freiherr von Berger\pwindex{Berger, Alfred von 30.\,4.\,1853 Wien – 24.\,8.\,1912 ebd.@\textsc{Berger, Alfred von} (30.\,4.\,1853 Wien – 24.\,8.\,1912 ebd.), \emph{Schriftsteller, Journalist, Theaterleiter}|pwk}: \emph{Burgtheater}\pwindex{Berger, Alfred von 30.\,4.\,1853 Wien – 24.\,8.\,1912 ebd.@\textsc{Berger, Alfred von} (30.\,4.\,1853 Wien – 24.\,8.\,1912 ebd.), \emph{Schriftsteller, Journalist, Theaterleiter}!Burgtheater [Rechte der Seele, Liebelei]@\strich\emph{Burgtheater [Rechte der Seele, Liebelei]}|pwk}. In: \emph{Montags-Revue}\pwindex{Montags-Revue. Wochenschrift für Politik, Finanzen, Kunst und Literatur@\emph{Montags-Revue. Wochenschrift für Politik, Finanzen, Kunst und Literatur}|pwk}, Jg. 26, Nr. 41, 14. 10. 1895, S. 1–4.}}}\label{K_L02772-5} des Baron \textsc{Berger}\pwindex{Berger, Alfred von 30.\,4.\,1853 Wien – 24.\,8.\,1912 ebd.@\textsc{Berger, Alfred von} (30.\,4.\,1853 Wien – 24.\,8.\,1912 ebd.), \emph{Schriftsteller, Journalist, Theaterleiter}|pw}, die ich Dir gleichfalls zurückſende.\pend
           
\pstart
           Zu erzählen habe ich Dir nichts. Mein Leben iſt vollſtändig unintereſſant. Es gibt
               nichts Neues und wird nie etwas Neues geben, {\pb}außer
               irgend einem definitiven Unglück. Intereſſant iſt nur Dein Leben, und ich möchte{ }ſehr
               viel darüber wiſſen. Haſt Du alſo \label{K_L02772-6v}\edtext{zum
               dritten Mal angeſangen, das Stück\pwindex{Schnitzler, Arthur 15.\,5.\,1862 Wien – 21.\,10.\,1931 ebd.@\textsc{Schnitzler, Arthur} (15.\,5.\,1862 Wien – 21.\,10.\,1931 ebd.), \emph{Schriftsteller, Mediziner}!Freiwild. Schauspiel in 3 Akten@\strich\emph{Freiwild. Schauspiel in 3 Akten}|pwv} zu{ }ſchreiben}{\lemma{\textnormal{\emph{zum … schreiben}}}\Cendnote{\textnormal{Siehe A. S.: \emph{Tagebuch}, 27. 4. 1896.
               }}}\label{K_L02772-6}? Könnte man nicht doch das Manuſcript\pwindex{Schnitzler, Arthur 15.\,5.\,1862 Wien – 21.\,10.\,1931 ebd.@\textsc{Schnitzler, Arthur} (15.\,5.\,1862 Wien – 21.\,10.\,1931 ebd.), \emph{Schriftsteller, Mediziner}!Freiwild. Schauspiel in 3 Akten@\strich\emph{Freiwild. Schauspiel in 3 Akten}|pwv}{ }ſehen? Wirſt Du \label{K_L02772-7v}\edtext{in
               die »Zeit\orgindex{Zeit. Wiener Wochenschrift@Die Zeit. Wiener Wochenschrift|pw}« eintreten}{\lemma{\textnormal{\emph{in
               die »Zeit« eintreten}}}\Cendnote{\textnormal{Dazu kam es nicht.}}}\label{K_L02772-7}, jetzt nach \textsc{Kanners\pwindex{Kanner, Heinrich 9.\,11.\,1864 Galați – 15.\,2.\,1930 Wien@\textsc{Kanner, Heinrich} (9.\,11.\,1864 Galați – 15.\,2.\,1930 Wien), \emph{Herausgeber, Publizist}|pw}} Rückkehr? Und wie iſt{ }ſonſt
               Daſeinsführung und Stimmung?\pend
           
\pstart
           Recht geärgert habe ich mich, als ich Deinen {\pb}\label{K_L02772-8v}\edtext{Namen im »\textsc{Simplicissimus}\pwindex{Simplicissimus@\emph{Simplicissimus}|pw}«}{\lemma{\textnormal{\emph{Namen im »Simplicissimus«}}}\Cendnote{\textnormal{Arthur Schnitzler: \emph{Die überspannte Person}\pwindex{Schnitzler, Arthur 15.\,5.\,1862 Wien – 21.\,10.\,1931 ebd.@\textsc{Schnitzler, Arthur} (15.\,5.\,1862 Wien – 21.\,10.\,1931 ebd.), \emph{Schriftsteller, Mediziner}!überspannte Person@\strich\emph{Die überspannte Person}|pwk}. In: \emph{Simplicissimus}\pwindex{Simplicissimus@\emph{Simplicissimus}|pwk}, Jg. 1, H. 3, 18. 4. 1896, S. 3 u. 6.}}}\label{K_L02772-8}{ }fand\pwindex{Schnitzler, Arthur 15.\,5.\,1862 Wien – 21.\,10.\,1931 ebd.@\textsc{Schnitzler, Arthur} (15.\,5.\,1862 Wien – 21.\,10.\,1931 ebd.), \emph{Schriftsteller, Mediziner}!überspannte Person@\strich\emph{Die überspannte Person}|pwv}. Dieſer Lausbub’ \textsc{Langen\pwindex{Langen, Albert 8.\,7.\,1869 Antwerpen – 30.\,4.\,1909 München@\textsc{Langen, Albert} (8.\,7.\,1869 Antwerpen – 30.\,4.\,1909 München), \emph{Verleger}|pw}}, der mir i\substVorne{}\textsuperscript{m}\substDazwischen{}n\substHinten{}{ }\textsc{Paris\oindex{Paris@\textbf{Paris}, \emph{Hauptstadt}|pw}}, wenn ich ihn dazu drängte, Deine Bücher in Verlag zu nehmen,{ }ſtets antwortete:
               Du könnteſt \label{K_L02772-9v}\edtext{nicht deutſch{ }ſchreiben}{\lemma{\textnormal{\emph{nicht deutsch schreiben}}}\Cendnote{\textnormal{eventuell auf die
                  Verwendung von Austriazismen gemünzt}}}\label{K_L02772-9}, – iſt jetzt in der Lage,{ }ſein neues
                  Unternehmen\orgindex{Simplicissimus@Simplicissimus|pwv} mit Deinem
               jungen \textsc{Rénommée} aufzuputzen. Das hat er wahrlich nicht
               verdient. Warum haſt {\pb}Du ihm den Beitrag\pwindex{Schnitzler, Arthur 15.\,5.\,1862 Wien – 21.\,10.\,1931 ebd.@\textsc{Schnitzler, Arthur} (15.\,5.\,1862 Wien – 21.\,10.\,1931 ebd.), \emph{Schriftsteller, Mediziner}!überspannte Person@\strich\emph{Die überspannte Person}|pwv} gegeben\damage{?} Ich bekam in Deutſchland\oindex{Deutschland@\textbf{Deutschland}|pw} durch Zufall
               das Heft\pwindex{Zukunft@\emph{Die Zukunft}|pwv} der »Zukunft\pwindex{Zukunft@\emph{Die Zukunft}|pw}« in die Hand, das \label{K_L02772-10v}\edtext{\textsc{Hardens\pwindex{Harden, Maximilian 20.\,10.\,1861 Berlin – 30.\,10.\,1927 Montana@\textsc{Harden, Maximilian} (20.\,10.\,1861 Berlin – 30.\,10.\,1927 Montana), \emph{Schriftsteller, Publizist}|pw}}{ }Kritik\pwindex{Harden, Maximilian 20.\,10.\,1861 Berlin – 30.\,10.\,1927 Montana@\textsc{Harden, Maximilian} (20.\,10.\,1861 Berlin – 30.\,10.\,1927 Montana), \emph{Schriftsteller, Publizist}!Theaternotizen [Liebelei]@\strich\emph{Theaternotizen [Liebelei]}|pwv} über »Liebelei\pwindex{Schnitzler, Arthur 15.\,5.\,1862 Wien – 21.\,10.\,1931 ebd.@\textsc{Schnitzler, Arthur} (15.\,5.\,1862 Wien – 21.\,10.\,1931 ebd.), \emph{Schriftsteller, Mediziner}!Liebelei. Schauspiel in drei Akten@\strich\emph{Liebelei. Schauspiel in drei Akten}|pw}«}{\lemma{\textnormal{\emph{Hardens … »Liebelei«}}}\Cendnote{\textnormal{Maximilian Harden\pwindex{Harden, Maximilian 20.\,10.\,1861 Berlin – 30.\,10.\,1927 Montana@\textsc{Harden, Maximilian} (20.\,10.\,1861 Berlin – 30.\,10.\,1927 Montana), \emph{Schriftsteller, Publizist}|pwk}: \emph{Theaternotizen}\pwindex{Harden, Maximilian 20.\,10.\,1861 Berlin – 30.\,10.\,1927 Montana@\textsc{Harden, Maximilian} (20.\,10.\,1861 Berlin – 30.\,10.\,1927 Montana), \emph{Schriftsteller, Publizist}!Theaternotizen [Liebelei]@\strich\emph{Theaternotizen [Liebelei]}|pwk}. In: \emph{Die
                        Zukunft}\pwindex{Zukunft@\emph{Die Zukunft}|pwk}, Jg. 5, Bd. 14, 14. 3. 1896,
                     S. 527–528.}}}\label{K_L02772-10} enthält. Das iſt doch eine recht unverſtändige Kritik\pwindex{Harden, Maximilian 20.\,10.\,1861 Berlin – 30.\,10.\,1927 Montana@\textsc{Harden, Maximilian} (20.\,10.\,1861 Berlin – 30.\,10.\,1927 Montana), \emph{Schriftsteller, Publizist}!Theaternotizen [Liebelei]@\strich\emph{Theaternotizen [Liebelei]}|pw}, die Dich völlig unterſchätzt. Biſt Du
               trotzdem bei Deiner großen Meinung über \textsc{Harden\pwindex{Harden, Maximilian 20.\,10.\,1861 Berlin – 30.\,10.\,1927 Montana@\textsc{Harden, Maximilian} (20.\,10.\,1861 Berlin – 30.\,10.\,1927 Montana), \emph{Schriftsteller, Publizist}|pw}} geblieben?\pend
           
\pstart
           Aber ich will nicht fragen, und Du{ }ſollſt den \strikeout{Ih}
               Inhalt des nächſten Briefes nach {\pb}\damage{freier} Wahl zuſammen\damage{th}un. Schreib’ mir nur recht viel über Dich.\pend
           
\pstart
           Und wie gehts dem \textsc{Richard\pwindex{Beer-Hofmann, Richard 11.\,7.\,1866 Wien – 26.\,9.\,1945 New York City@\textsc{Beer-Hofmann, Richard} (11.\,7.\,1866 Wien – 26.\,9.\,1945 New York City), \emph{Schriftsteller}|pw}}? Er bringts wirklich fertig, mir keine Zeile zu{ }ſchreiben. Erwartet hab’ ichs,
               aber es erſtaunt mich doch. Es iſt immerhin der{ }ſchönſte Fall von Faulheit, der mir
               in meinem Leben vorgekommen iſt.\pend
           
\pstart
           Gern ginge ich mit früh im August{ }{\pb}nach \label{K_L02772-11v}\edtext{Dänemark\oindex{Dänemark@\textbf{Dänemark}|pw}}{\lemma{\textnormal{\emph{Dänemark}}}\Cendnote{\textnormal{Vom 5. 8. 1896 bis zum 21. 8. 1896 waren Schnitzler,
                     Goldmann\pwindex{Goldmann, Paul 31.\,1.\,1865 Breslau – 25.\,9.\,1935 Wien@\textsc{Goldmann, Paul} (31.\,1.\,1865 Breslau – 25.\,9.\,1935 Wien), \emph{Schriftsteller, Journalist}|pwk}, Richard\pwindex{Beer-Hofmann, Richard 11.\,7.\,1866 Wien – 26.\,9.\,1945 New York City@\textsc{Beer-Hofmann, Richard} (11.\,7.\,1866 Wien – 26.\,9.\,1945 New York City), \emph{Schriftsteller}|pwk} und Paula
                     Beer-Hofmann\pwindex{Beer-Hofmann, Paula 25.\,2.\,1879 Wien – 30.\,10.\,1939 Zürich@\textsc{Beer-Hofmann, Paula} (25.\,2.\,1879 Wien – 30.\,10.\,1939 Zürich)|pwk} gemeinsam in Skodsborg\oindex{Skodsborg@\textbf{Skodsborg}|pwk}.
               }}}\label{K_L02772-11}, w\damage{enn} ich Geld hätte, w\damage{as} noch zweifelhaft iſt. Ich würde dann \label{K_L02772-12v}\edtext{über Berlin\oindex{Berlin@\textbf{Berlin}, \emph{Hauptstadt}|pw}
                  zurückreiſen}{\lemma{\textnormal{\emph{über Berlin
                  zurückreisen}}}\Cendnote{\textnormal{Siehe A. S.: \emph{Tagebuch}, 26. 8. 1896.
               }}}\label{K_L02772-12}, wo mich meine Mutter\pwindex{Goldmann, Clementine 15.\,5.\,1842 Breslau – 24.\,2.\,1924 Frankfurt am Main@\textsc{Goldmann, Clementine} (15.\,5.\,1842 Breslau – 24.\,2.\,1924 Frankfurt am Main)|pwv} und mein Onkel\pwindex{Mamroth, Hermann @\textsc{Mamroth, Hermann}|pwuv} erwarten.\pend
           
\pstart
           Grüß’ Dich Gott, mein lieber Freund, und{ }ſchreib’ mir bald!\pend
           
\pstart
           Dein treuer {\\[\baselineskip]}\spacefill\mbox{Paul Goldmann}\pend
           \leftskip=0em{}\selectlanguage{ngerman}\endnumbering\briefempfaengerindex{Schnitzler, Arthur@\textsc{Schnitzler, Arthur}!zzzGoldmann, Paul@\emph{von Paul Goldmann}!1896-04-291@{29. 4. [1896]}|)be}\mylabel{L02772h}  \newcommand{\dateiname}{L02772}\newcommand{\titel}{Paul Goldmann an Arthur Schnitzler, 29. 4. [1896]}\newcommand{\editorInnen}{Martin Anton Müller und Laura Untner}%% latex-leseansicht-abspann.tex
%% Abspann für die Leseansicht.
%% Der Schalter \ifkorrekturansicht ist bereits durch den Vorspann gesetzt.

%% latex-abspann.tex
%% Gemeinsamer Abspann für Korrekturansicht und Leseansicht.
%% Setzt den Schalter \ifkorrekturansicht voraus (gesetzt in den
%% einbindenden Dateien latex-korrekturansicht-abspann.tex bzw.
%% latex-leseansicht-abspann.tex).
%% ---------------------------------------------------------------

\normalsize

% Das esempio-Environment wird nur in der Leseansicht benötigt
\ifkorrekturansicht\else
\newenvironment{esempio}[3]%
{
    \vspace{1.5ex}
    \rlap{\underline{#1}}
    \par
    \setlength{\parindent}{0cm}
    \nopagebreak
    \leftskip=#2cm
    \rightskip=#3cm
}
{
    \par
}
\fi

\doendnotes{C}
\bigskip
\vfill

\clearpage

\footnotesize

\ifkorrekturansicht
  \lohead{\textsc{register}}
\fi

% theindex-Environment neu definieren ohne reledmac
\makeatletter
\renewenvironment{theindex}{%
  \ifkorrekturansicht
    \section*{\indexname}%
  \else
    \subsubsection*{Index der erwähnten Entitäten}%
  \fi
  \setlength{\parindent}{0pt}%
  \setlength{\parskip}{0pt plus 0.3pt}%
  \let\item\@idxitem
}{%
  \ifkorrekturansicht\clearpage\fi
}
\makeatother

\IfFileExists{\jobname-pw.ind}{\input{\jobname-pw.ind}}{}

% Quellenangabe nur in der Leseansicht
\ifkorrekturansicht\else
% Fallback-Definitionen, falls die .tex-Datei \titel etc. nicht gesetzt hat
\providecommand{\titel}{}
\providecommand{\editorInnen}{}
\providecommand{\dateiname}{\jobname}

\vspace{3cm}

\vfill

\footnotesize
\textsc{Quelle}: \titel. Herausgegeben von {\editorInnen}. In: \emph{Arthur Schnitzler: Briefwechsel mit Autorinnen und Autoren}.
 Digitale Edition, https://schnitzler-briefe.acdh.oeaw.ac.at/{\dateiname}.html (Stand \today)
\fi

\end{document}


