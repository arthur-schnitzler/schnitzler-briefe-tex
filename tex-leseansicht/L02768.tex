%% latex-leseansicht-vorspann.tex
%% Vorspann für die Leseansicht.
%% Lädt die gemeinsame Datei latex-vorspann.tex mit nicht gesetztem Schalter.

\newif\ifkorrekturansicht
\korrekturansichtfalse

\input{../tex-inputs/latex-vorspann}


         
         \renewcommand{\erwaehntePersonen}{Personen: Richard Beer-Hofmann, Theodor Herzl, Leopold Sonnemann, Jean Thorel}
         \renewcommand{\erwaehnteInstitutionen}{Institutionen: Frankfurter Zeitung}
         \renewcommand{\erwaehnteOrte}{Orte: Paris, Wien, rue Feydeau}
         \renewcommand{\erwaehnteWerke}{Werke: Amourette. Pièce en trois actes. Adaptée de Arthur Schnitzler, Der Judenstaat. Versuch einer modernen Lösung der Judenfrage, Frankfurter Zeitung, Freiwild. Schauspiel in 3 Akten}
               \section[Paul Goldmann an Arthur Schnitzler, 22. 3. {[}1896{]}]{ Paul Goldmann an Arthur Schnitzler, 22. 3. {[}1896{]}}\nopagebreak\mylabel{v}\rehead{ }\begin{ledgroupsized}[t]{13cm}\normalsize\beginnumbering \toendnotes[C]{\smallbreak\pagebreak[2]} \Standort{DLA, A:Schnitzler, HS.NZ85.1.3166.}
\physDesc{Brief, 1 Blatt, 3 Seiten, 1070 Zeichen
\newline{}Handschrift: blaue Tinte, deutsche Kurrent
\newline{}Schnitzler: 1) mit Bleistift das Jahr »96« vermerkt  2) mit rotem Buntstift eine Unterstreichung}\toendnotes[C]{\smallbreak}\pstart
           \noindent{}{\pb}\textcolor{gray}{\textbf{\textbf{Frankfurter Zeitung\orgindex{Frankfurter Zeitung@Frankfurter Zeitung|pw}}}}\pend
           \pstart
           \textcolor{gray}{\textbf{(\begin{otherlanguage}{french}Gazette de Francfort\end{otherlanguage}\orgindex{Frankfurter Zeitung@Frankfurter Zeitung|pw}).}}\pend
           \pstart
           \textcolor{gray}{\textbf{\textbf{\begin{otherlanguage}{french}Fondateur M.\end{otherlanguage}{ }L. Sonnemann\pwindex{Sonnemann, Leopold 1831-10-29 – 1909-10-30@\textsc{Sonnemann, Leopold} (1831-10-29 – 1909-10-30), \emph{Journalist, Herausgeber}|pw}.}}}\pend
           \pstart
           \begin{otherlanguage}{french}\textcolor{gray}{\textbf{Journal\pwindex{?? Werk@Nicht ermittelte Verfasserinnen und Verfasser!Frankfurter Zeitung1856 – 1943@\emph{Frankfurter Zeitung} {[}1856 – 1943{]}|pwv} politique,
                        financier,}}\end{otherlanguage}\pend
           \pstart
           \begin{otherlanguage}{french}\textcolor{gray}{\textbf{commercial et littéraire.}}\end{otherlanguage}\pend
           \pstart
           \begin{otherlanguage}{french}\textcolor{gray}{\textbf{\textbf{Paraissant trois fois par jour.}}}\end{otherlanguage}\hfill \textsc{Paris\oindex{Paris@\textbf{Paris}|pw}}, 22. März.\pend
           \pstart
           \begin{otherlanguage}{french}\textcolor{gray}{\textbf{\textbf{Bureau à Paris\oindex{Paris@\textbf{Paris}|pw}:}}}\end{otherlanguage}\pend
           \pstart
           \begin{otherlanguage}{french}\textcolor{gray}{\textbf{\textbf{24. Rue Feydeau\oindex{rue Feydeau@\textbf{rue Feydeau}|pw}.}}}\end{otherlanguage}\pend
           \pstart\center{}Mein lieber Freund,\pend\pstart
           Hab’ Geduld mit mir; Du haſt ſie, und ich bin Dir von Herzen dankbar dafür. Das iſt
               ein toller Arbeits-Monat. Es regnet Arbeit, alle Winde
               wehen Arbeit einher. Ich ſchreibe Artikel jeder Art über Gott und die Welt und
               Sonſtiges. Sonſt komme ich zu nichts. Jede Woche beginne ich mit dem Vorſatz: Nun
               werde ich ihm ſchreiben. Ihm biſt natürlich Du. Und die Woche geht vorüber, und ich
               habe nicht geſchrieben. {\pb}Auch bin ich krank. Mein
               Augenleiden wird ernſt. Die Ärzte ſagen, ich ſolle ausruhen. Haha! Und bei alledem
               denke ich faſt jeden Tag an Dich, mit Beſorgniß, und frage mich: Wie wird er das
               aufnehmen, daß ich ihm nicht ſchreibe? Nun weiß ichs und bin beruhigt. Diese Woche
               denke ich kann ich Dir doch den ausführlichen Brief ſchreiben. Neues weiß ich
               übrigens nicht. Die Überſetzung\pwindex{Thorel, Jean 1859-09-11 – 1916-08-20@\textsc{Thorel, Jean} (1859-09-11 – 1916-08-20), \emph{Übersetzer, Schriftsteller}!Amourette. Piece en trois actes. Adaptee de Arthur Schnitzler1897@\strich\emph{Amourette. Pièce en trois actes. Adaptée de Arthur Schnitzler} {[}Übersetzung, 1897{]}|pwv}s-Angelegenheit ſtockt. \textsc{Thorel\pwindex{Thorel, Jean 1859-09-11 – 1916-08-20@\textsc{Thorel, Jean} (1859-09-11 – 1916-08-20), \emph{Übersetzer, Schriftsteller}|pw}} und ich laufen uns nach und können {\pb}uns nicht
               treffen.\pend
           \pstart
           Dank’ für das \label{K_L02768-1v}\edtext{Bulletin}{\lemma{\textnormal{\emph{Bulletin}}}\Cendnote{\textnormal{möglicherweise die
                  »Depesche« des letzten Briefs, vgl. Paul Goldmann an Arthur Schnitzler, 22. 3. [1896]}}}\label{K_L02768-1h}. Was macht das neue \label{K_L02768-2v}\edtext{Stück\pwindex{Schnitzler, Arthur 15.05.1862 – 21.10.1931@\textsc{Schnitzler, Arthur} (15.05.1862 – 21.10.1931), \emph{Schriftsteller, Mediziner}!Freiwild. Schauspiel in 3 Akten1896@\strich\emph{Freiwild. Schauspiel in 3 Akten} {[}1896{]}|pwv}}{\lemma{\textnormal{\emph{Stück}}}\Cendnote{\textnormal{Am 23. 2. 1896 begann Schnitzler\pwindex{Schnitzler, Arthur 15.05.1862 – 21.10.1931@\textsc{Schnitzler, Arthur} (15.05.1862 – 21.10.1931), \emph{Schriftsteller, Mediziner}|pwk} ein weiteres Mal, \emph{Freiwild}\pwindex{Schnitzler, Arthur 15.05.1862 – 21.10.1931@\textsc{Schnitzler, Arthur} (15.05.1862 – 21.10.1931), \emph{Schriftsteller, Mediziner}!Freiwild. Schauspiel in 3 Akten1896@\strich\emph{Freiwild. Schauspiel in 3 Akten} {[}1896{]}|pwk} neu zu schreiben. Er war mit dem Stück\pwindex{Schnitzler, Arthur 15.05.1862 – 21.10.1931@\textsc{Schnitzler, Arthur} (15.05.1862 – 21.10.1931), \emph{Schriftsteller, Mediziner}!Freiwild. Schauspiel in 3 Akten1896@\strich\emph{Freiwild. Schauspiel in 3 Akten} {[}1896{]}|pwkv} noch immer nicht
               zufrieden.}}}\label{K_L02768-2h}? Was ſagſt Du zu \textsc{Herzl\pwindex{Herzl, Theodor 1860-05-02 – 1904-07-03@\textsc{Herzl, Theodor} (1860-05-02 – 1904-07-03), \emph{Schriftsteller, Journalist}|pw}s}{ }\strikeout{al\textcolor{gray}{b}} albernem \label{K_L02768-5v}\edtext{Buche\pwindex{Herzl, Theodor 1860-05-02 – 1904-07-03@\textsc{Herzl, Theodor} (1860-05-02 – 1904-07-03), \emph{Schriftsteller, Journalist}!Judenstaat. Versuch einer modernen Loesung der Judenfrage1896-02-20@\strich\emph{Der Judenstaat. Versuch einer modernen Lösung der Judenfrage} {[}1896-02-20{]}|pwv}}{\lemma{\textnormal{\emph{Buche}}}\Cendnote{\textnormal{\emph{Der Judenstaat. Versuch einer modernen Lösung der
                     Judenfrage}\pwindex{Herzl, Theodor 1860-05-02 – 1904-07-03@\textsc{Herzl, Theodor} (1860-05-02 – 1904-07-03), \emph{Schriftsteller, Journalist}!Judenstaat. Versuch einer modernen Loesung der Judenfrage1896-02-20@\strich\emph{Der Judenstaat. Versuch einer modernen Lösung der Judenfrage} {[}1896-02-20{]}|pwk} wurde Mitte Februar 1896 ausgeliefert. Schnitzler\pwindex{Schnitzler, Arthur 15.05.1862 – 21.10.1931@\textsc{Schnitzler, Arthur} (15.05.1862 – 21.10.1931), \emph{Schriftsteller, Mediziner}|pwk} sprach am 8. 3. 1896 mit Herzl\pwindex{Herzl, Theodor 1860-05-02 – 1904-07-03@\textsc{Herzl, Theodor} (1860-05-02 – 1904-07-03), \emph{Schriftsteller, Journalist}|pwk} über das Buch\pwindex{Herzl, Theodor 1860-05-02 – 1904-07-03@\textsc{Herzl, Theodor} (1860-05-02 – 1904-07-03), \emph{Schriftsteller, Journalist}!Judenstaat. Versuch einer modernen Loesung der Judenfrage1896-02-20@\strich\emph{Der Judenstaat. Versuch einer modernen Lösung der Judenfrage} {[}1896-02-20{]}|pwkv}.}}}\label{K_L02768-5h}? Was macht \textsc{Richard\pwindex{Beer-Hofmann, Richard 1866-07-11 – 1945-09-26@\textsc{Beer-Hofmann, Richard} (1866-07-11 – 1945-09-26), \emph{Schriftsteller}|pw}}?\pend
           \pstart
           Grüß’ Dich Gott, mein lieber Freund!\pend
           \pstart
           Von Herzen {\\[\baselineskip]}Dein {\\[\baselineskip]}\spacefill\mbox{Paul Goldmann}\pend
           \leftskip=0em{}
         
         \endnumbering\mylabel{h}\end{ledgroupsized}  \newcommand{\dateiname}{L02768}\newcommand{\titel}{Paul Goldmann an Arthur Schnitzler, 22. 3. [1896]}\newcommand{\editorInnen}{Martin Anton Müller und Laura Untner}%% latex-leseansicht-abspann.tex
%% Abspann für die Leseansicht.
%% Der Schalter \ifkorrekturansicht ist bereits durch den Vorspann gesetzt.

%% latex-abspann.tex
%% Gemeinsamer Abspann für Korrekturansicht und Leseansicht.
%% Setzt den Schalter \ifkorrekturansicht voraus (gesetzt in den
%% einbindenden Dateien latex-korrekturansicht-abspann.tex bzw.
%% latex-leseansicht-abspann.tex).
%% ---------------------------------------------------------------

\normalsize

% Das esempio-Environment wird nur in der Leseansicht benötigt
\ifkorrekturansicht\else
\newenvironment{esempio}[3]%
{
    \vspace{1.5ex}
    \rlap{\underline{#1}}
    \par
    \setlength{\parindent}{0cm}
    \nopagebreak
    \leftskip=#2cm
    \rightskip=#3cm
}
{
    \par
}
\fi

\doendnotes{C}
\bigskip
\vfill

\clearpage

\footnotesize

\ifkorrekturansicht
  \lohead{\textsc{register}}
\fi

% theindex-Environment neu definieren ohne reledmac
\makeatletter
\renewenvironment{theindex}{%
  \ifkorrekturansicht
    \section*{\indexname}%
  \else
    \subsubsection*{Index der erwähnten Entitäten}%
  \fi
  \setlength{\parindent}{0pt}%
  \setlength{\parskip}{0pt plus 0.3pt}%
  \let\item\@idxitem
}{%
  \ifkorrekturansicht\clearpage\fi
}
\makeatother

\IfFileExists{\jobname-pw.ind}{\input{\jobname-pw.ind}}{}

% Quellenangabe nur in der Leseansicht
\ifkorrekturansicht\else
% Fallback-Definitionen, falls die .tex-Datei \titel etc. nicht gesetzt hat
\providecommand{\titel}{}
\providecommand{\editorInnen}{}
\providecommand{\dateiname}{\jobname}

\vspace{3cm}

\vfill

\footnotesize
\textsc{Quelle}: \titel. Herausgegeben von {\editorInnen}. In: \emph{Arthur Schnitzler: Briefwechsel mit Autorinnen und Autoren}.
 Digitale Edition, https://schnitzler-briefe.acdh.oeaw.ac.at/{\dateiname}.html (Stand \today)
\fi

\end{document}


      