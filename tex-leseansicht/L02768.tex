%% latex-korrekturansicht-vorspann.tex
%% Vorspann für die Korrekturansicht.
%% Lädt die gemeinsame Datei latex-vorspann.tex mit gesetztem Schalter.

\newif\ifkorrekturansicht
\korrekturansichttrue

\input{../tex-inputs/latex-vorspann}


\section[Paul Goldmann an Arthur Schnitzler, 22. 3. {[}1896{]}]{L02768 Paul Goldmann an Arthur Schnitzler, 22. 3. {[}1896{]}}
\nopagebreak\mylabel{L02768v}
\rehead{ }\normalsize\beginnumbering\briefempfaengerindex{Schnitzler, Arthur@\textsc{Schnitzler, Arthur}!zzzGoldmann, Paul@\emph{von Paul Goldmann}!1896-03-221@{22. 3. {[}1896{]}}|(be}
\toendnotes[C]{\smallbreak\pagebreak[2]}\Standort{DLA, A:Schnitzler, HS.NZ85.1.3166.}
\physDesc{Brief, 1 Blatt, 3 Seiten, 1070 Zeichen
\newline{}Handschrift: blaue Tinte, deutsche Kurrent
\newline{}Schnitzler: 1) mit Bleistift das Jahr »96« vermerkt  2) mit rotem Buntstift eine Unterstreichung}\toendnotes[C]{\smallbreak}
\pstart
           {\pb}\textcolor{gray}{\textbf{\textbf{Frankfurter Zeitung\orgindex{Frankfurter Zeitung@Frankfurter Zeitung|pw}}}}\pend
           
\pstart
           \textcolor{gray}{\textbf{(\begin{otherlanguage}{french}Gazette de Francfort\end{otherlanguage}\orgindex{Frankfurter Zeitung@Frankfurter Zeitung|pw}).}}\pend
           
\pstart
           \textcolor{gray}{\textbf{\textbf{\begin{otherlanguage}{french}Fondateur M.\end{otherlanguage}{ }L. Sonnemann\pwindex{Sonnemann, Leopold 1831-10-29 – 1909-10-30@\textsc{Sonnemann, Leopold} (1831-10-29 – 1909-10-30), \emph{Journalist/Journalistin, Herausgeber/Herausgeberin}|pw}.}}}\pend
           
\pstart
           \begin{otherlanguage}{french}\textcolor{gray}{\textbf{Journal\pwindex{Frankfurter Zeitung@\emph{Frankfurter Zeitung}|pwv} politique,
                        financier,}}\end{otherlanguage}\pend
           
\pstart
           \begin{otherlanguage}{french}\textcolor{gray}{\textbf{commercial et littéraire.}}\end{otherlanguage}\pend
           
\pstart
           \begin{otherlanguage}{french}\textcolor{gray}{\textbf{\textbf{Paraissant trois fois par jour.}}}\end{otherlanguage}\hfill \textsc{Paris\oindex{Paris@\textbf{Paris}, \emph{P.PPLC}|pw}}, 22. März.\pend
           
\pstart
           \begin{otherlanguage}{french}\textcolor{gray}{\textbf{\textbf{Bureau à Paris\oindex{Paris@\textbf{Paris}, \emph{P.PPLC}|pw}:}}}\end{otherlanguage}\pend
           
\pstart
           \begin{otherlanguage}{french}\textcolor{gray}{\textbf{\textbf{24. Rue Feydeau\oindex{rue Feydeau@\textbf{rue Feydeau}, \emph{Straße (K.STR)}|pw}.}}}\end{otherlanguage}\pend
           
\pstart\center{}Mein lieber Freund,\pend\vspace{0.5em}
\pstart
           Hab’ Geduld mit mir; Du haſt ſie, und ich bin Dir von Herzen dankbar dafür. Das iſt
               ein toller Arbeits-Monat. Es regnet Arbeit, alle Winde
               wehen Arbeit einher. Ich ſchreibe Artikel jeder Art über Gott und die Welt und
               Sonſtiges. Sonſt komme ich zu nichts. Jede Woche beginne ich mit dem Vorſatz: Nun
               werde ich ihm ſchreiben. Ihm biſt natürlich Du. Und die Woche geht vorüber, und ich
               habe nicht geſchrieben. {\pb}Auch bin ich krank. Mein
               Augenleiden wird ernſt. Die Ärzte ſagen, ich ſolle ausruhen. Haha! Und bei alledem
               denke ich faſt jeden Tag an Dich, mit Beſorgniß, und frage mich: Wie wird er das
               aufnehmen, daß ich ihm nicht ſchreibe? Nun weiß ichs und bin beruhigt. Diese Woche
               denke ich kann ich Dir doch den ausführlichen Brief ſchreiben. Neues weiß ich
               übrigens nicht. Die Überſetzung\pwindex{Amourette. Piece en trois actes. Adaptee de Arthur Schnitzler@\emph{Amourette. Pièce en trois actes. Adaptée de Arthur Schnitzler}|pwv}s-Angelegenheit ſtockt. \textsc{Thorel\pwindex{Thorel, Jean 1859-09-11 – 1916-08-20@\textsc{Thorel, Jean} (1859-09-11 – 1916-08-20), \emph{Übersetzer/Übersetzerin, Dramatiker/Dramatikerin}|pw}} und ich laufen uns nach und können {\pb}uns nicht
               treffen.\pend
           
\pstart
           Dank’ für das \label{K_L02768-1v}\edtext{Bulletin}{\lemma{\textnormal{\emph{Bulletin}}}\Cendnote{\textnormal{möglicherweise die
                  »Depesche« des letzten Briefs, vgl. Paul Goldmann an Arthur Schnitzler, 22. 3. [1896].}}}\label{K_L02768-1}. Was macht das neue \label{K_L02768-2v}\edtext{Stück\pwindex{Freiwild. Schauspiel in 3 Akten@\emph{Freiwild. Schauspiel in 3 Akten}|pwv}}{\lemma{\textnormal{\emph{Stück}}}\Cendnote{\textnormal{Am 23. 2. 1896 begann Schnitzler ein weiteres Mal, \emph{Freiwild}\pwindex{Freiwild. Schauspiel in 3 Akten@\emph{Freiwild. Schauspiel in 3 Akten}|pwk} neu zu schreiben. Er war mit dem Stück\pwindex{Freiwild. Schauspiel in 3 Akten@\emph{Freiwild. Schauspiel in 3 Akten}|pwkv} noch immer nicht
               zufrieden.}}}\label{K_L02768-2}? Was ſagſt Du zu \textsc{Herzls\pwindex{Herzl, Theodor 1860-05-02 – 1904-07-03@\textsc{Herzl, Theodor} (1860-05-02 – 1904-07-03), \emph{Schriftsteller/Schriftstellerin, Journalist/Journalistin}|pw}}{ }\strikeout{al\textcolor{gray}{b}} albernem \label{K_L02768-3v}\edtext{Buche\pwindex{Judenstaat. Versuch einer modernen Loesung der Judenfrage@\emph{Der Judenstaat. Versuch einer modernen Lösung der Judenfrage}|pwv}}{\lemma{\textnormal{\emph{Buche}}}\Cendnote{\textnormal{\emph{Der Judenstaat. Versuch einer modernen Lösung der
                     Judenfrage}\pwindex{Judenstaat. Versuch einer modernen Loesung der Judenfrage@\emph{Der Judenstaat. Versuch einer modernen Lösung der Judenfrage}|pwk} wurde Mitte Februar 1896 ausgeliefert. Schnitzler hatte am 8. 3. 1896 mit Herzl\pwindex{Herzl, Theodor 1860-05-02 – 1904-07-03@\textsc{Herzl, Theodor} (1860-05-02 – 1904-07-03), \emph{Schriftsteller/Schriftstellerin, Journalist/Journalistin}|pwk} über das Buch\pwindex{Judenstaat. Versuch einer modernen Loesung der Judenfrage@\emph{Der Judenstaat. Versuch einer modernen Lösung der Judenfrage}|pwkv} gesprochen.}}}\label{K_L02768-3}? Was macht \textsc{Richard\pwindex{Beer-Hofmann, Richard 1866-07-11 – 1945-09-26@\textsc{Beer-Hofmann, Richard} (1866-07-11 – 1945-09-26), \emph{Schriftsteller/Schriftstellerin}|pw}}?\pend
           
\pstart
           Grüß’ Dich Gott, mein lieber Freund!\pend
           
\pstart
           Von Herzen {\\[\baselineskip]}Dein {\\[\baselineskip]}\spacefill\mbox{Paul Goldmann}\pend
           \leftskip=0em{}\selectlanguage{ngerman}\endnumbering\briefempfaengerindex{Schnitzler, Arthur@\textsc{Schnitzler, Arthur}!zzzGoldmann, Paul@\emph{von Paul Goldmann}!1896-03-221@{22. 3. {[}1896{]}}|)be}\mylabel{L02768h}  \normalsize

\doendnotes{C}
\bigskip
\vfill

\clearpage

\footnotesize

\lohead{\textsc{register}}

% Definiere theindex-Environment komplett neu ohne reledmac
\makeatletter
\renewenvironment{theindex}{%
  \section*{\indexname}%
  \setlength{\parindent}{0pt}%
  \setlength{\parskip}{0pt plus 0.3pt}%
  \let\item\@idxitem
}{%
  \clearpage
}
\makeatother

\IfFileExists{\jobname-pw.ind}{\input{\jobname-pw.ind}}{}

\end{document}

      