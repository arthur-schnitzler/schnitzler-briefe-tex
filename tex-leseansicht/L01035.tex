%% latex-leseansicht-vorspann.tex
%% Vorspann für die Leseansicht.
%% Lädt die gemeinsame Datei latex-vorspann.tex mit nicht gesetztem Schalter.

\newif\ifkorrekturansicht
\korrekturansichtfalse

\input{../tex-inputs/latex-vorspann}


         
         \renewcommand{\erwaehntePersonen}{Personen: Richard Beer-Hofmann, Paul Goldmann, Hugo August von Hofmannsthal, Anna von Hofmannsthal}
         \renewcommand{\erwaehnteInstitutionen}{Institutionen: Neue Freie Presse}
         \renewcommand{\erwaehnteOrte}{Orte: Brighton, Deutschland, London, Niederösterreich, Wien, Österreich}
         \renewcommand{\erwaehnteWerke}{Werke: Berliner Theater. (»Der König von Rom.«)}
               \section[Hugo von Hofmannsthal an Arthur Schnitzler, 6. 5. {[}1900{]}]{ Hugo von Hofmannsthal an Arthur Schnitzler, 6. 5. {[}1900{]}}\nopagebreak\mylabel{v}\rehead{ }\begin{ledgroupsized}[t]{13cm}\normalsize\beginnumbering \toendnotes[C]{\smallbreak\pagebreak[2]} \Standort{CUL, Schnitzler, B 43.}
\physDesc{Brief, 1 Blatt, 4 Seiten, 1014 Zeichen
\newline{}Handschrift: schwarze Tinte, deutsche Kurrent
\newline{}Schnitzler: mit Bleistift die Jahreszahl ergänzt: »900« 
\newline{}Ordnung: mit Bleistift von unbekannter Hand nummeriert:
                                    »161« }\buchAbdrucke{\weitereDrucke{Hugo von Hofmannsthal, Arthur Schnitzler: \emph{Briefwechsel}. Hg. Therese Nickl und Heinrich Schnitzler. Frankfurt am Main: \emph{S. Fischer} 1964, S. 138–139.} }\toendnotes[C]{\smallbreak}\pstart
           \raggedleft{}{\pb}\textsc{Brighton}\oindex{Brighton@\textbf{Brighton}|pw}, 6 V.\pend
           \pstart{}mein lieber Arthur\pend\pstart
           ich war ſehr froh darüber daſs Sie in der Zeit von Papa\pwindex{Hofmannsthal, Hugo August von 21.12.1841 – 08.12.1915@\textsc{Hofmannsthal, Hugo August von} (21.12.1841 – 08.12.1915), \emph{Bankdirektor}|pwv}s Krankheit meine Eltern\pwindex{Hofmannsthal, Anna von 27.01.1849 – 22.03.1904@\textsc{Hofmannsthal, Anna von} (27.01.1849 – 22.03.1904)|pwv}\pwindex{Hofmannsthal, Hugo August von 21.12.1841 – 08.12.1915@\textsc{Hofmannsthal, Hugo August von} (21.12.1841 – 08.12.1915), \emph{Bankdirektor}|pwv} oft beſucht und \introOben{}mir\introOben{}{ }ſo gut und beruhigend darüber geſchrieben
               haben.\pend
           \pstart
           Ein Zufall hat mich veranlaſst, für kurze Zeit hierher zu gehen und ſo werde ich auch
               noch mit einer etwas traumhaften {\pb}Flüchtigkeit London\oindex{London@\textbf{London}|pw}{ }ſehen.\pend
           \pstart
           Wenn ich auch nicht gar ſo viel Fertiges mitbringe, ſo dafür um ſo mehr angefangenes
               und entworfenes.\pend
           \pstart
           Hier iſt mir nach einer langen Zeit zuerſt die N. Fr.
                  Preſſe\orgindex{Neue Freie Presse@Neue Freie Presse|pw} wieder in die Hände gekommen. Das ſtrömt eine kleinliche, ordinäre,
               herabgekommene Atmoſphäre {\pb}aus, in
               welcher man \uline{niemals wirklich} zu leben trachten
               muſs.\pend
           \pstart
           Warum ſchreibt\pwindex{Goldmann, Paul 31.01.1865 – 25.09.1935@\textsc{Goldmann, Paul} (31.01.1865 – 25.09.1935), \emph{Schriftsteller, Journalist}!Berliner Theater. (»Der Koenig von Rom.«)3. 5. 1900@\strich\emph{Berliner Theater. (»Der König von Rom.«)} {[}3. 5. 1900{]}|pwv} ein anſtändiger
               Menſch wie Goldmann\pwindex{Goldmann, Paul 31.01.1865 – 25.09.1935@\textsc{Goldmann, Paul} (31.01.1865 – 25.09.1935), \emph{Schriftsteller, Journalist}|pw} 6 Spalten voll mit Nichts,
               dieſes Nichts in dem unbeſchreiblich widerwärtigen witzelnden jüdiſchen Ton, der
               nirgends auf der Welt exiſtiert als im Feuilleton deutſcher\oindex{Deutschland@\textbf{Deutschland}|pw} u. oeſterr.\oindex{Oesterreich@\textbf{Österreich}|pw} Zeitungen? \pend
           \pstart
           {\pb}Ungefähr den 18\textsuperscript{ten} werde ich in Wien\oindex{Wien@\textbf{Wien}|pw}{ }ſein und freue mich ſehr auf Sie und Richard\pwindex{Beer-Hofmann, Richard 1866-07-11 – 1945-09-26@\textsc{Beer-Hofmann, Richard} (1866-07-11 – 1945-09-26), \emph{Schriftsteller}|pw}, auf den Frühling in Niederöſterreich\oindex{Niederoesterreich@\textbf{Niederösterreich}|pw} und aufs Radfahren.\pend
           \pstart Von Herzen Ihr \spacefill\mbox{Hugo.}\pend{}
         
         \endnumbering\mylabel{h}\end{ledgroupsized}  \newcommand{\dateiname}{L01035}\newcommand{\titel}{Hugo von Hofmannsthal an Arthur Schnitzler, 6. 5. [1900]}\newcommand{\editorInnen}{Martin Anton Müller und Gerd-Hermann Susen}%% latex-leseansicht-abspann.tex
%% Abspann für die Leseansicht.
%% Der Schalter \ifkorrekturansicht ist bereits durch den Vorspann gesetzt.

%% latex-abspann.tex
%% Gemeinsamer Abspann für Korrekturansicht und Leseansicht.
%% Setzt den Schalter \ifkorrekturansicht voraus (gesetzt in den
%% einbindenden Dateien latex-korrekturansicht-abspann.tex bzw.
%% latex-leseansicht-abspann.tex).
%% ---------------------------------------------------------------

\normalsize

% Das esempio-Environment wird nur in der Leseansicht benötigt
\ifkorrekturansicht\else
\newenvironment{esempio}[3]%
{
    \vspace{1.5ex}
    \rlap{\underline{#1}}
    \par
    \setlength{\parindent}{0cm}
    \nopagebreak
    \leftskip=#2cm
    \rightskip=#3cm
}
{
    \par
}
\fi

\doendnotes{C}
\bigskip
\vfill

\clearpage

\footnotesize

\ifkorrekturansicht
  \lohead{\textsc{register}}
\fi

% theindex-Environment neu definieren ohne reledmac
\makeatletter
\renewenvironment{theindex}{%
  \ifkorrekturansicht
    \section*{\indexname}%
  \else
    \subsubsection*{Index der erwähnten Entitäten}%
  \fi
  \setlength{\parindent}{0pt}%
  \setlength{\parskip}{0pt plus 0.3pt}%
  \let\item\@idxitem
}{%
  \ifkorrekturansicht\clearpage\fi
}
\makeatother

\IfFileExists{\jobname-pw.ind}{\input{\jobname-pw.ind}}{}

% Quellenangabe nur in der Leseansicht
\ifkorrekturansicht\else
% Fallback-Definitionen, falls die .tex-Datei \titel etc. nicht gesetzt hat
\providecommand{\titel}{}
\providecommand{\editorInnen}{}
\providecommand{\dateiname}{\jobname}

\vspace{3cm}

\vfill

\footnotesize
\textsc{Quelle}: \titel. Herausgegeben von {\editorInnen}. In: \emph{Arthur Schnitzler: Briefwechsel mit Autorinnen und Autoren}.
 Digitale Edition, https://schnitzler-briefe.acdh.oeaw.ac.at/{\dateiname}.html (Stand \today)
\fi

\end{document}


      