%% latex-leseansicht-vorspann.tex
%% Vorspann für die Leseansicht.
%% Lädt die gemeinsame Datei latex-vorspann.tex mit nicht gesetztem Schalter.

\newif\ifkorrekturansicht
\korrekturansichtfalse

\input{../tex-inputs/latex-vorspann}


\section[Hugo von Hofmannsthal an Arthur Schnitzler, 15. 7. 1900]{L01056 Hugo von Hofmannsthal an Arthur Schnitzler, 15. 7. 1900}
\nopagebreak\mylabel{L01056v}
\rehead{ }\normalsize\beginnumbering\briefempfaengerindex{Schnitzler, Arthur@\textsc{Schnitzler, Arthur}!zzzHofmannsthal, Hugo von@\emph{von Hugo von Hofmannsthal}!1900-07-152@{15. 7. 1900}|(be}
\toendnotes[C]{\smallbreak\pagebreak[2]}
\correspDesc{Versand  durch Hugo von Hofmannsthal am 15. 7. 1900 in Bad Fusch
\newline{}Erhalt  durch Arthur Schnitzler im Zeitraum [16. 7. 1900
                  – 20. 7. 1900?] in Wien}\toendnotes[C]{\smallbreak}
\Standort{CUL, Schnitzler, B 43.}
\physDesc{Brief, 1 Blatt, 4 Seiten, 1196 Zeichen
\newline{}Handschrift: schwarze Tinte, deutsche Kurrent
\newline{}Schnitzler: mit Bleistift das Datum vervollständigt:
                                 »/7 900« 
\newline{}Ordnung: 1) mit Bleistift von unbekannter Hand nummeriert: »\strikeout{164}«  2) mit Bleistift von unbekannter Hand nummeriert:
                                    »163«}
\buchAbdrucke{\weitereDrucke{Hugo von Hofmannsthal, Arthur Schnitzler: \emph{Briefwechsel}. Herausgegeben von Therese Nickl und Heinrich Schnitzler. Frankfurt am Main: \emph{S. Fischer} 1964, S. 140.} }\toendnotes[C]{\smallbreak}
\pstart
           \raggedleft{}{\pb}Bad Fuſch\oindex{Bad Fusch@\textbf{Bad Fusch}|pw}{ }15\textsuperscript{ten}\pend
           
\pstart{}mein guter lieber Arthur\pend\vspace{0.5em}
\pstart
           wie die Dinge einmal eigenſinnig und unbegreiflich{ }ſind, finde ich hier, in
               vollkommener Ruhe, bei unverſtörten äußern Umſtänden{ }ſeit 14 Tagen nicht nur nicht
               die leiſeſte Möglichkeit des Arbeitens,{ }ſondern ich verſinke auch in eine{ }ſolche
               Verdroſſenheit,{ }ſolche Gelähmtheit aller inneren Sinne, {\pb}daſs ich ein Buch nach dem andern
               aus der Hand lege und weder am Morgen noch am Abend die geringſte Freude habe. Nun
               iſt mir vor 2 Stunden eingefallen, es mit einem Ausflug zu verſuchen. Wie{ }ſchön, wenn
               man in{ }ſolchen Momenten nicht{ }ſo weit auseinander wäre! Auch mein Rad iſt in der {\pb}Brühl\oindex{Brühl@\textbf{Brühl}, \emph{Tal}|pw}, ich will nicht abwarten, bis es herkäme,{ }ſondern fahre gleich nach \textsc{Saalfelden}\oindex{Saalfelden am Steinernen Meer@\textbf{Saalfelden am Steinernen Meer}, \emph{Verwaltungsgebiet}|pw}, von dort mit der Post an den \textsc{Hintersee}\oindex{Hintersee@\textbf{Hintersee}, \emph{See}|pw}, wo es{ }ſehr{ }ſchön{ }ſein{ }ſoll und von da entweder über \textsc{Salzburg}\oindex{Salzburg@\textbf{Salzburg}, \emph{Verwaltungsgebiet}|pw} oder \textsc{Golling}\oindex{Golling an der Salzach@\textbf{Golling an der Salzach}, \emph{Hauptstadt}|pw} oder{ }ſonſt zurück. Ich{ }ſehne mich unendlich nach Dörfern, die ich noch nicht
               geſehen habe, nach kleinen Häuſern am {\pb}Waldrand, Mühlen in einem tiefen
               Grund, Brücken, Alleen und andern{ }ſolchen Dingen. Von Richard\pwindex{Beer-Hofmann, Richard 11.\,7.\,1866 Wien – 26.\,9.\,1945 New York City@\textsc{Beer-Hofmann, Richard} (11.\,7.\,1866 Wien – 26.\,9.\,1945 New York City), \emph{Schriftsteller}|pw} bin ich ohne irgend eine Nachricht{ }ſeit Wien\oindex{Wien@\textbf{Wien}, \emph{Verwaltungsgebiet}|pw}.\pend
           
\pstart
           Papa\pwindex{Hofmannsthal, Hugo August von 21.\,12.\,1841 Wien – 8.\,12.\,1915 ebd.@\textsc{Hofmannsthal, Hugo August von} (21.\,12.\,1841 Wien – 8.\,12.\,1915 ebd.), \emph{Bankdirektor}|pwv} iſt gottlob wohl, meine
                  Eltern\pwindex{Hofmannsthal, Anna von 27.\,1.\,1849 Wien – 22.\,3.\,1904 Sanatorium Fürth@\textsc{Hofmannsthal, Anna von} (27.\,1.\,1849 Wien – 22.\,3.\,1904 Sanatorium Fürth)|pwv}\pwindex{Hofmannsthal, Hugo August von 21.\,12.\,1841 Wien – 8.\,12.\,1915 ebd.@\textsc{Hofmannsthal, Hugo August von} (21.\,12.\,1841 Wien – 8.\,12.\,1915 ebd.), \emph{Bankdirektor}|pwv} grüßen
               Sie vielmals; bitte{ }ſchreiben Sie mir bald, in 3 Tagen bin ich wieder hier.\pend
           
\pstart
           Von Herzen Ihr{\\[\baselineskip]}\spacefill\mbox{Hugo.}\pend
           \leftskip=0em{}\selectlanguage{ngerman}\endnumbering\briefempfaengerindex{Schnitzler, Arthur@\textsc{Schnitzler, Arthur}!zzzHofmannsthal, Hugo von@\emph{von Hugo von Hofmannsthal}!1900-07-152@{15. 7. 1900}|)be}\mylabel{L01056h}  \newcommand{\dateiname}{L01056}\newcommand{\titel}{Hugo von Hofmannsthal an Arthur Schnitzler, 15. 7. 1900}\newcommand{\editorInnen}{Martin Anton Müller und Gerd-Hermann Susen}%% latex-leseansicht-abspann.tex
%% Abspann für die Leseansicht.
%% Der Schalter \ifkorrekturansicht ist bereits durch den Vorspann gesetzt.

%% latex-abspann.tex
%% Gemeinsamer Abspann für Korrekturansicht und Leseansicht.
%% Setzt den Schalter \ifkorrekturansicht voraus (gesetzt in den
%% einbindenden Dateien latex-korrekturansicht-abspann.tex bzw.
%% latex-leseansicht-abspann.tex).
%% ---------------------------------------------------------------

\normalsize

% Das esempio-Environment wird nur in der Leseansicht benötigt
\ifkorrekturansicht\else
\newenvironment{esempio}[3]%
{
    \vspace{1.5ex}
    \rlap{\underline{#1}}
    \par
    \setlength{\parindent}{0cm}
    \nopagebreak
    \leftskip=#2cm
    \rightskip=#3cm
}
{
    \par
}
\fi

\doendnotes{C}
\bigskip
\vfill

\clearpage

\footnotesize

\ifkorrekturansicht
  \lohead{\textsc{register}}
\fi

% theindex-Environment neu definieren ohne reledmac
\makeatletter
\renewenvironment{theindex}{%
  \ifkorrekturansicht
    \section*{\indexname}%
  \else
    \subsubsection*{Index der erwähnten Entitäten}%
  \fi
  \setlength{\parindent}{0pt}%
  \setlength{\parskip}{0pt plus 0.3pt}%
  \let\item\@idxitem
}{%
  \ifkorrekturansicht\clearpage\fi
}
\makeatother

\IfFileExists{\jobname-pw.ind}{\input{\jobname-pw.ind}}{}

% Quellenangabe nur in der Leseansicht
\ifkorrekturansicht\else
% Fallback-Definitionen, falls die .tex-Datei \titel etc. nicht gesetzt hat
\providecommand{\titel}{}
\providecommand{\editorInnen}{}
\providecommand{\dateiname}{\jobname}

\vspace{3cm}

\vfill

\footnotesize
\textsc{Quelle}: \titel. Herausgegeben von {\editorInnen}. In: \emph{Arthur Schnitzler: Briefwechsel mit Autorinnen und Autoren}.
 Digitale Edition, https://schnitzler-briefe.acdh.oeaw.ac.at/{\dateiname}.html (Stand \today)
\fi

\end{document}


