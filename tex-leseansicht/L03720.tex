%% latex-leseansicht-vorspann.tex
%% Vorspann für die Leseansicht.
%% Lädt die gemeinsame Datei latex-vorspann.tex mit nicht gesetztem Schalter.

\newif\ifkorrekturansicht
\korrekturansichtfalse

\input{../tex-inputs/latex-vorspann}


\section[Elsa Plessner an Arthur Schnitzler, 19. 1. 1899]{L03720 Elsa Plessner an Arthur Schnitzler, 19. 1. 1899}
\nopagebreak\mylabel{L03720v}
\rehead{ }\normalsize\beginnumbering\briefempfaengerindex{Schnitzler, Arthur@\textsc{Schnitzler, Arthur}!zzzPlessner, Elsa@\emph{von Elsa Plessner}!1899-01-191@{19. 1. 1899}|(be}
\toendnotes[C]{\smallbreak\pagebreak[2]}
\correspDesc{Versand  durch Elsa Plessner am 19. 1. 1899 in Wien
\newline{}Erhalt  durch Arthur Schnitzler im Zeitraum [19. 1. 1899
                  – 22. 1. 1899?] in Wien}\toendnotes[C]{\smallbreak}
\Standort{DLA, A:Schnitzler, HS.1985.1.419.}
\physDesc{Brief, 1 Blatt, 4 Seiten, 2055 Zeichen (Briefpapier mit Blumenmotiv (Veilchen) auf S. 1)
\newline{}Handschrift: schwarze Tinte, lateinische Kurrent}\toendnotes[C]{\smallbreak}
\pstart
           \raggedleft{}{\pb}Wien I. Spiegelgasse 2\oindex{Wien@\textbf{Wien}!I., Innere Stadt@\textbf{I., Innere Stadt}!Spiegelgasse 2@\textbf{Spiegelgasse 2}, \emph{Wohngebäude}|pw}, den
                     19. 1. 99.\pend
           
\pstart
           \raggedleft{}Telef. 7819.\pend
           
\pstart{}Hochverehrter Herr Doctor!\pend\vspace{0.5em}
\pstart
           Ich liege in einem furchtbaren Kampf mit mir selbst. Wenn ich nur genau wüsste, wie
               Sie in verschwiegener Ruhe Ihres hübschen Arbeitszimmers meine Brief- und
               Manuscript-bombardements aufnehmen! Angeborene und anerzogene Zurückhaltung sollten
               mich überhaupt etwas wirksamer bändigen aber – !!! –\pend
           
\pstart
           Aber der ewige Wunsch gerade Ihr Urtheil über alle mein Arbeiten zu wissen!! – Sie
                  {\pb}haben mir einmal geschrieben, dass Sie mir wie
               einem Schüler Aufgaben geben wollten! – – Das ermuthigt mich andrerseits wieder,
               Ihnen wie einem Professor meine Arbeiten zur Correctur zu zeigen! –\pend
           
\pstart
           Also kurz – – ich habe im Herbst ein Stück\pwindex{Plessner, Elsa 22.\,8.\,1875 Wien – 7.\,5.\,1932 Alicante@\textsc{Plessner, Elsa} (22.\,8.\,1875 Wien – 7.\,5.\,1932 Alicante), \emph{Schriftstellerin}!Ehrlosen. Schauspiel in drei Acten@\strich\emph{Die Ehrlosen. Schauspiel in drei Acten}|pwv} geschrieben! 3 Acte Schauspiel. Es liegt jetzt über
               2 Monate im Schreibtisch – und hat unter dem Einfluss Ihres »Vermächtnis\pwindex{Schnitzler, Arthur 15. 5. 1862 Wien – 21. 10. 1931 ebd.@\textsc{Schnitzler, Arthur} (15. 5. 1862 Wien – 21. 10. 1931 ebd.), \emph{Schriftsteller, Mediziner}!Vermächtnis. Schauspiel in drei Akten@\strich\emph{Das Vermächtnis. Schauspiel in drei Akten}|pw}« eine Änderung erfahren. Meine Heldin hieß – – –
                  Toni\pwindex{Schnitzler, Arthur 15. 5. 1862 Wien – 21. 10. 1931 ebd.@\textsc{Schnitzler, Arthur} (15. 5. 1862 Wien – 21. 10. 1931 ebd.), \emph{Schriftsteller, Mediziner}!Vermächtnis. Schauspiel in drei Akten@\strich\emph{Das Vermächtnis. Schauspiel in drei Akten}|pwv}!! – Folglich heißt sie
               jetzt anders! –\pend
           
\pstart
           {\pb}– (Wenn ich mir erlauben darf, eine Meiung zu äußern,
               so meine ich, dass die rührendste Figur Ihres Stückes\pwindex{Schnitzler, Arthur 15. 5. 1862 Wien – 21. 10. 1931 ebd.@\textsc{Schnitzler, Arthur} (15. 5. 1862 Wien – 21. 10. 1931 ebd.), \emph{Schriftsteller, Mediziner}!Vermächtnis. Schauspiel in drei Akten@\strich\emph{Das Vermächtnis. Schauspiel in drei Akten}|pwv} – von einer Tragik, von eine\substVorne{}\textsuperscript{m}\substDazwischen{}r\substHinten{} geradezu erschütternden Schicksalsschwere die Figur der Agnes\pwindex{Schnitzler, Arthur 15. 5. 1862 Wien – 21. 10. 1931 ebd.@\textsc{Schnitzler, Arthur} (15. 5. 1862 Wien – 21. 10. 1931 ebd.), \emph{Schriftsteller, Mediziner}!Vermächtnis. Schauspiel in drei Akten@\strich\emph{Das Vermächtnis. Schauspiel in drei Akten}|pwv} ist – die ja etwas im Schatten steht!
               – Ich weiß nicht, ob blos für mich. Aber die Toni\pwindex{Schnitzler, Arthur 15. 5. 1862 Wien – 21. 10. 1931 ebd.@\textsc{Schnitzler, Arthur} (15. 5. 1862 Wien – 21. 10. 1931 ebd.), \emph{Schriftsteller, Mediziner}!Vermächtnis. Schauspiel in drei Akten@\strich\emph{Das Vermächtnis. Schauspiel in drei Akten}|pwv} hat ihr Leben hinter sich, hat etwas genossen und ist
               mir deshalb nicht so leid! – Die kleine Agnes\pwindex{Schnitzler, Arthur 15. 5. 1862 Wien – 21. 10. 1931 ebd.@\textsc{Schnitzler, Arthur} (15. 5. 1862 Wien – 21. 10. 1931 ebd.), \emph{Schriftsteller, Mediziner}!Vermächtnis. Schauspiel in drei Akten@\strich\emph{Das Vermächtnis. Schauspiel in drei Akten}|pwv} hätte \substVorne{}\textsuperscript{I}\substDazwischen{}i\substHinten{}hr Leben vor sich, könnte ihr Glück bauen – und ihr werden die Bausteine aus
               der Hand geschlagen! Sie stirbt nicht dran – aber was in ihr stirbt – – das ist das
                  \uline{beste}, was so ein junges Ding hat.) – – –
               Pardon für {\pb}diese Abschweifung! – – – – –\pend
           
\pstart
           Also – lieber guter, einziger Herr Doctor! Sein Sie so gut – sagen Sie mir \label{K_L03720-1v}\edtext{\begin{otherlanguage}{french}sans-gène\end{otherlanguage}}{\lemma{\textnormal{\emph{sans-gène}}}\Cendnote{\textnormal{französisch: ungeniert}}}\label{K_L03720-1}
               (vielleicht telefonisch) ob ich Ihrer Güte noch diese Belastungsprobe zumuthen darf
               – – ob Sie mein Stück\pwindex{Plessner, Elsa 22.\,8.\,1875 Wien – 7.\,5.\,1932 Alicante@\textsc{Plessner, Elsa} (22.\,8.\,1875 Wien – 7.\,5.\,1932 Alicante), \emph{Schriftstellerin}!Ehrlosen. Schauspiel in drei Acten@\strich\emph{Die Ehrlosen. Schauspiel in drei Acten}|pwv} lesen
               wollen? – – – Dann haben Sie’s aber gleich!!  – – – Meine Familie\pwindex{Plessner, Clementine 7.\,12.\,1855 Wien – 27.\,2.\,1943 Konzentrationslager Theresienstadt@\textsc{Plessner, Clementine} (7.\,12.\,1855 Wien – 27.\,2.\,1943 Konzentrationslager Theresienstadt), \emph{Schauspielerin, Filmschauspielerin}|pwv}\pwindex{Askonas, Johanna Leonie 20.\,11.\,1877 Wien – 30.\,7.\,1930 ebd.@\textsc{Askonas, Johanna Leonie} (20.\,11.\,1877 Wien – 30.\,7.\,1930 ebd.), \emph{Pensionsinhaberin}|pwv} will mich partout
               »berühmt«! Die »neuen Lehrer\pwindex{Plessner, Elsa 22.\,8.\,1875 Wien – 7.\,5.\,1932 Alicante@\textsc{Plessner, Elsa} (22.\,8.\,1875 Wien – 7.\,5.\,1932 Alicante), \emph{Schriftstellerin}!neue Lehrer. Novelle@\strich\emph{Der neue Lehrer. Novelle}|pw}« u. s. w. sind
               »gar nichts«, – – »was hab ich von Novellen«?!! Also – – \label{K_L03720-2v}\edtext{der Bien muss}{\lemma{\textnormal{\emph{der Bien muss}}}\Cendnote{\textnormal{Redewendung: ob sinnvoll oder nicht, man ist gezwungen, etwas zu tun}}}\label{K_L03720-2}! –
               Aber ganz schlecht scheint es doch nicht! Ich habe wirklich so etwas \uline{in} mir entdeckt, was »Stücke
                  schreibt\textcolor{gray}{«}!\pend
           
\pstart
           Verehrungsvoll{\\[\baselineskip]}\spacefill\mbox{Elsa Plessner.}\pend
           \leftskip=0em{}\selectlanguage{ngerman}\endnumbering\briefempfaengerindex{Schnitzler, Arthur@\textsc{Schnitzler, Arthur}!zzzPlessner, Elsa@\emph{von Elsa Plessner}!1899-01-191@{19. 1. 1899}|)be}\mylabel{L03720h}  \newcommand{\dateiname}{L03720}\newcommand{\titel}{Elsa Plessner an Arthur Schnitzler, 19. 1. 1899}\newcommand{\editorInnen}{Selma Jahnke und Martin Anton Müller}%% latex-leseansicht-abspann.tex
%% Abspann für die Leseansicht.
%% Der Schalter \ifkorrekturansicht ist bereits durch den Vorspann gesetzt.

%% latex-abspann.tex
%% Gemeinsamer Abspann für Korrekturansicht und Leseansicht.
%% Setzt den Schalter \ifkorrekturansicht voraus (gesetzt in den
%% einbindenden Dateien latex-korrekturansicht-abspann.tex bzw.
%% latex-leseansicht-abspann.tex).
%% ---------------------------------------------------------------

\normalsize

% Das esempio-Environment wird nur in der Leseansicht benötigt
\ifkorrekturansicht\else
\newenvironment{esempio}[3]%
{
    \vspace{1.5ex}
    \rlap{\underline{#1}}
    \par
    \setlength{\parindent}{0cm}
    \nopagebreak
    \leftskip=#2cm
    \rightskip=#3cm
}
{
    \par
}
\fi

\doendnotes{C}
\bigskip
\vfill

\clearpage

\footnotesize

\ifkorrekturansicht
  \lohead{\textsc{register}}
\fi

% theindex-Environment neu definieren ohne reledmac
\makeatletter
\renewenvironment{theindex}{%
  \ifkorrekturansicht
    \section*{\indexname}%
  \else
    \subsubsection*{Index der erwähnten Entitäten}%
  \fi
  \setlength{\parindent}{0pt}%
  \setlength{\parskip}{0pt plus 0.3pt}%
  \let\item\@idxitem
}{%
  \ifkorrekturansicht\clearpage\fi
}
\makeatother

\IfFileExists{\jobname-pw.ind}{\input{\jobname-pw.ind}}{}

% Quellenangabe nur in der Leseansicht
\ifkorrekturansicht\else
% Fallback-Definitionen, falls die .tex-Datei \titel etc. nicht gesetzt hat
\providecommand{\titel}{}
\providecommand{\editorInnen}{}
\providecommand{\dateiname}{\jobname}

\vspace{3cm}

\vfill

\footnotesize
\textsc{Quelle}: \titel. Herausgegeben von {\editorInnen}. In: \emph{Arthur Schnitzler: Briefwechsel mit Autorinnen und Autoren}.
 Digitale Edition, https://schnitzler-briefe.acdh.oeaw.ac.at/{\dateiname}.html (Stand \today)
\fi

\end{document}


