%% latex-korrekturansicht-vorspann.tex
%% Vorspann für die Korrekturansicht.
%% Lädt die gemeinsame Datei latex-vorspann.tex mit gesetztem Schalter.

\newif\ifkorrekturansicht
\korrekturansichttrue

\input{../tex-inputs/latex-vorspann}


\section[Elsa Plessner an Arthur Schnitzler, 19. 1. 1899]{L03720 Elsa Plessner an Arthur Schnitzler, 19. 1. 1899}
\nopagebreak\mylabel{L03720v}
\rehead{ }\normalsize\beginnumbering\briefempfaengerindex{Schnitzler, Arthur@\textsc{Schnitzler, Arthur}!zzzPlessner, Elsa@\emph{von Elsa Plessner}!1899-01-191@{19. 1. 1899}|(be}
\toendnotes[C]{\smallbreak\pagebreak[2]}\Standort{DLA, A:Schnitzler, HS.1985.1.419.}
\physDesc{Brief, 1 Blatt, 4 Seiten, 2056 Zeichen (Briefpapier mit Blumenmotiv (Vergissmeinnicht) auf S.
                                 1)
\newline{}Handschrift: , lateinische Kurrent}\toendnotes[C]{\smallbreak}
\pstart
           \raggedleft{}{\pb}Wien I. Spiegelgasse 2\oindex{Spiegelgasse 2@\textbf{Spiegelgasse 2}, \emph{Wohngebäude (K.WHS)}|pw}, den
                     19. 1. 99. \pend
           
\pstart
           \raggedleft{}Telef. 7819.\pend
           
\pstart{}Hochverehrter Herr Doctor!\pend\vspace{0.5em}
\pstart
           Ich liege in einem furchtbaren Kampf mit mir selbst. Wenn ich nur genau wüsste, wie
               Sie in verschwiegener Ruhe Ihres hübschen Arbeitszimmers meine Brief- und
               Manuscript-bombardements aufnehmen! Angeborene und anerzogene Zurückhaltung sollten
               mich überhaupt etwas wirksamer bändigen aber – !!! – \pend
           
\pstart
           Aber der ewige Wunsch gerade Ihr Urtheil über alle mein Arbeiten zu wissen!! – Sie
                  {\pb}haben mir einmal geschrieben, dass Sie mir wie einem Schüler Aufgaben
               geben wollten! – – Das ermuthigt mich andrerseits wieder, Ihnen wie einem Professor
               meine Arbeiten zur Correctur zu zeigen! –\pend
           
\pstart
           Also kurz – – ich habe im Herbst ein Stück\pwindex{Ehrlosen. Schauspiel in drei Acten@\emph{Die Ehrlosen. Schauspiel in drei Acten}|pwv} geschrieben! 3 Acte Schauspiel. Es liegt jetzt
               über 2 Monate im Schreibtisch – und hat unter dem Einfluss Ihres »Vermächtnis\pwindex{Vermaechtnis. Schauspiel in drei Akten@\emph{Das Vermächtnis. Schauspiel in drei Akten}|pw}« eine Änderung erfahren. Meine Heldin
               hieß – – – Toni!! – Folglich heißt sie jetzt anders! –\pend
           
\pstart
           {\pb}– (Wenn ich mir erlauben darf, eine Meiung zu äußern, so meine ich,
               dass die rührendste Figur Ihres Stückes\pwindex{Vermaechtnis. Schauspiel in drei Akten@\emph{Das Vermächtnis. Schauspiel in drei Akten}|pwv} – von einer Tragik, von eine\substVorne{}\textsuperscript{m}\substDazwischen{}r\substHinten{} geradezu erschütternden Schicksalsschwere die Figur der Agnes ist – die ja
               etwas im Schatten steht! – Ich weiß nicht, ob blos für mich. Aber die Toni hat ihr
               Leben hinter sich hat etwas genossen und ist mir deshalb nicht so leid! – Die kleine
               Agnes hätte \substVorne{}\textsuperscript{I }\substDazwischen{}i\substHinten{}hr Leben vor sich, könnte ihr Glück bauen – und ihr werden die Bausteine aus
               der Hand geschlagen! Sie stirbt nicht dran – aber was in ihr stirbt – – das ist das
                  \uline{beste}, was so ein junges Ding hat.) – –
               Pardon für {\pb}diese Abschweifung! – – – – –\pend
           
\pstart
           Also – lieber guter einziger Herr Doctor! Sein Sie so gut – sagen Sie nur \label{K_L03720-1v}\edtext{\begin{otherlanguage}{french}sans-gène\end{otherlanguage}}{\lemma{\textnormal{\emph{sans-gène}}}\Cendnote{\textnormal{französisch: ohne
                  Scham}}}\label{K_L03720-1} (vielleicht telefonisch) ob ich Ihrer Güte noch diese Belastungsprobe
               zumuthen darf – – ob Sie mein Stück\pwindex{Ehrlosen. Schauspiel in drei Acten@\emph{Die Ehrlosen. Schauspiel in drei Acten}|pwv} lesen wollen. – – – Dann haben Sie's aber gleich!! 
               – – – Meine Familie\pwindex{Plessner, Clementine 1855-12-07 – 1943-02-27@\textsc{Plessner, Clementine} (1855-12-07 – 1943-02-27), \emph{Schauspieler/Schauspielerin, Filmschauspieler/Filmschauspielerin}|pwv}\pwindex{Askonas, Johanna Leonie 1877-11-20 – 1930-07-30@\textsc{Askonas, Johanna Leonie} (1877-11-20 – 1930-07-30), \emph{Pensionsinhaber/Pensionsinhaberin}|pwv} will
               mich partout »berühmt«! Die »neuen Lehrer\pwindex{neue Lehrer. Novelle@\emph{Der neue Lehrer. Novelle}|pw}«
               u. s. w. sind »gar nichts« – – »was hab ich von Novellen«?!! Also
               – – der Bien muss! – Aber ganz schlecht scheint es doch nicht! Ich habe wirklich
               so etwas \uline{in} mir entdeckt, was Stücke schreibt!!\pend
           
\pstart
           Verehrungsvoll{\\[\baselineskip]}\spacefill\mbox{Elsa Plessner}\pend
           \leftskip=0em{}\selectlanguage{ngerman}\endnumbering\briefempfaengerindex{Schnitzler, Arthur@\textsc{Schnitzler, Arthur}!zzzPlessner, Elsa@\emph{von Elsa Plessner}!1899-01-191@{19. 1. 1899}|)be}\mylabel{L03720h}
\begin{anhang}
\end{anhang}\normalsize

\doendnotes{C}
\bigskip
\vfill

\clearpage

\footnotesize

\lohead{\textsc{register}}

% Definiere theindex-Environment komplett neu ohne reledmac
\makeatletter
\renewenvironment{theindex}{%
  \section*{\indexname}%
  \setlength{\parindent}{0pt}%
  \setlength{\parskip}{0pt plus 0.3pt}%
  \let\item\@idxitem
}{%
  \clearpage
}
\makeatother

\IfFileExists{\jobname-pw.ind}{\input{\jobname-pw.ind}}{}

\end{document}

      