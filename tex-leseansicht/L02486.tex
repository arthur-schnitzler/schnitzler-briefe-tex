%% latex-korrekturansicht-vorspann.tex
%% Vorspann für die Korrekturansicht.
%% Lädt die gemeinsame Datei latex-vorspann.tex mit gesetztem Schalter.

\newif\ifkorrekturansicht
\korrekturansichttrue

\input{../tex-inputs/latex-vorspann}


\section[Felix Braun an Arthur Schnitzler, 10. 5. 1927]{L02486 Felix Braun an Arthur Schnitzler, 10. 5. 1927}
\nopagebreak\mylabel{L02486v}
\rehead{ }\normalsize\beginnumbering\briefempfaengerindex{Schnitzler, Arthur@\textsc{Schnitzler, Arthur}!zzzBraun, Felix@\emph{von Felix Braun}!1927-05-101@{10. 5. 1927}|(be}
\toendnotes[C]{\smallbreak\pagebreak[2]}\Standort{TMW, HS Schn 1/72/1.}
\physDesc{Brief, 1 Blatt, 4 Seiten, 1776 Zeichen
\newline{}Handschrift: schwarze Tinte, deutsche Kurrent
\newline{}Schnitzler: 1) mit Bleistift beschriftet: »\textsc{Fel Braun}«  2) mit rotem Buntstift im Text ergänzt: »\textsc{Sp\pwindex{Spiel im Morgengrauen. Novelle@\emph{Spiel im Morgengrauen. Novelle}|pwv}}« und mehrere Unterstreichungen}\toendnotes[C]{\smallbreak}
\pstart
           \centering{}{\pb}Wien\oindex{Wien@\textbf{Wien}, \emph{A.ADM2}|pw}, den 10. V. 27.\pend
           
\pstart{}Verehrter Herr Doktor!\pend\vspace{0.5em}
\pstart
           Längſt ſchulde ich Ihnen Dank für die Gabe Ihres neuen Buches\pwindex{Spiel im Morgengrauen. Novelle@\emph{Spiel im Morgengrauen. Novelle}|pwv}, das ich ja auch längſt geleſen
               habe. Denn – es bekommen, aufſchlagen, beginnen und nicht ſogleich weiterleſen,
               angeſpannt, atemlos bis ans Ende – ich weiß nicht, welche dringende Beſchäftigung
               mich davon abzuhalten vermocht hätte. Das Buch\pwindex{Spiel im Morgengrauen. Novelle@\emph{Spiel im Morgengrauen. Novelle}|pwv} iſt die Frucht vollkommener Meiſterſchaft der
               Geſtaltenbildung ſowohl wie auch der Erzählungskunſt; ſprachlich und anſchaulich, der
                  Hand{\pb}lung wie der
               Begründung nach eine reine Freude des Leſens.\pend
           
\pstart
           Jemand\pwindex{?? [Leser von Spiel im Morgengrauen 1927] @\textsc{?? [Leser von Spiel im Morgengrauen 1927]}|pwv}, der gleich mir die
                  Novelle\pwindex{Spiel im Morgengrauen. Novelle@\emph{Spiel im Morgengrauen. Novelle}|pwv} geſpannt geleſen
               hatte, ein philoſophiſcher, tiefblickender Geiſt, wandte ein, daß der Schluß nicht
               befriedige, und auch ich empfinde das. Es hätte notgetan, ſagte der Betreffende\pwindex{?? [Leser von Spiel im Morgengrauen 1927] @\textsc{?? [Leser von Spiel im Morgengrauen 1927]}|pwv}, daß dem Tod des
               Leutnants etwas vorausgegangen wäre, davon er ſelbſt erhöht hätte werden müſſen: etwa
               die Annahme des Geldes, das die Frau ihm vielleicht hätte mitbringen ſollen, und die
               Scham darüber wäre dann ein triftigerer Grund zur Selbſtjuſtiz geweſen als bloß die
               Flucht. Ich {\pb}mußte
               dieſen Gedanken als einleuchtend anerkennen. Was mir fehlt, iſt Transſzendenz –
               vielleicht wäre ſie durch eine ſo geführte Linie der Motivierung ermöglicht worden.
               Nicht wahr, Sie ſind mir nicht böſe, Herr Doktor, wenn ich aufrichtig meine
               Empfindung ſchreibe?\pend
           
\pstart
           In einer Zeit der Anarchie iſt das Erſcheinen des geſchloſſenen Kunſtwerks, des
               gekonnten, gemeiſterten Formgebildes eine ſolche Seltenheit, daß ſich nur Verehrung
               und Dankbarkeit geziemen. Laſſen Sie mich dieſe ſchönen Gefühle nicht zurückhalten.
               Ich freue mich Ihrer ſtetig ſich harmoni{\pb}ſierenden produktiven
               Kräfte, die Werk auf Werk hervorgeſtalten. Seit dem »Gang zum Weiher\pwindex{Gang zum Weiher. Dramatische Dichtung@\emph{Der Gang zum Weiher. Dramatische Dichtung}|pw}« war mir keine Ihrer Dichtungen ſo nahe wie dieſe Novelle\pwindex{Spiel im Morgengrauen. Novelle@\emph{Spiel im Morgengrauen. Novelle}|pwv}.\pend
           
\pstart
           In verehrender Geſinnung ergeben{\\[\baselineskip]}\spacefill\mbox{Felix Braun.}\pend
           \leftskip=0em{}\selectlanguage{ngerman}\endnumbering\briefempfaengerindex{Schnitzler, Arthur@\textsc{Schnitzler, Arthur}!zzzBraun, Felix@\emph{von Felix Braun}!1927-05-101@{10. 5. 1927}|)be}\mylabel{L02486h}  \normalsize

\doendnotes{C}
\bigskip
\vfill

\clearpage

\footnotesize

\lohead{\textsc{register}}

% Definiere theindex-Environment komplett neu ohne reledmac
\makeatletter
\renewenvironment{theindex}{%
  \section*{\indexname}%
  \setlength{\parindent}{0pt}%
  \setlength{\parskip}{0pt plus 0.3pt}%
  \let\item\@idxitem
}{%
  \clearpage
}
\makeatother

\IfFileExists{\jobname-pw.ind}{\input{\jobname-pw.ind}}{}

\end{document}

      