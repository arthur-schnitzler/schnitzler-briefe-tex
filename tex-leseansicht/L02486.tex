%% latex-leseansicht-vorspann.tex
%% Vorspann für die Leseansicht.
%% Lädt die gemeinsame Datei latex-vorspann.tex mit nicht gesetztem Schalter.

\newif\ifkorrekturansicht
\korrekturansichtfalse

\input{../tex-inputs/latex-vorspann}


               \section[Felix Braun an Arthur Schnitzler, 10. 5. 1927]{ Felix Braun an Arthur Schnitzler, 10. 5. 1927}\nopagebreak\mylabel{v}\rehead{ }\begin{ledgroupsized}[t]{13cm}\normalsize\beginnumbering\briefempfaengerindex{Schnitzler, Arthur@\textsc{Schnitzler, Arthur}!zzzBraun, Felix@\emph{von Felix Braun}!1927-05-101@{10. 5. 1927}|(be} \toendnotes[C]{\smallbreak\pagebreak[2]} \Standort{TMW, HS Schn 1/72/1.}
\physDesc{Brief, 1 Blatt, 4 Seiten
\newline{}Handschrift: schwarze Tinte, deutsche Kurrent
\newline{}Schnitzler: 1) mit Bleistift beschriftet: »\textsc{Fel Braun}« 2) mit rotem Buntstift im Text ergänzt: »\textsc{Sp\pwindex{Schnitzler, Arthur 15.05.1862 – 21.10.1931@\textsc{Schnitzler, Arthur} (15.05.1862 – 21.10.1931), \emph{Schriftsteller, Mediziner}!Spiel im Morgengrauen. Novelle5.12.1926 – 9.1.1927@\strich\emph{Spiel im Morgengrauen. Novelle} {[}5.12.1926 – 9.1.1927{]}|pwv}}« und mehrere Unterstreichungen}\toendnotes[C]{\smallbreak}\pstart
           \centering{}{\pb}Wien\oindex{Wien@\textbf{Wien}|pw}, den 10. V. 27.\pend
           \pstart{}Verehrter Herr Doktor!\pend\pstart
           Längſt ſchulde ich Ihnen Dank für die Gabe Ihres neuen Buches\pwindex{Schnitzler, Arthur 15.05.1862 – 21.10.1931@\textsc{Schnitzler, Arthur} (15.05.1862 – 21.10.1931), \emph{Schriftsteller, Mediziner}!Spiel im Morgengrauen. Novelle5.12.1926 – 9.1.1927@\strich\emph{Spiel im Morgengrauen. Novelle} {[}5.12.1926 – 9.1.1927{]}|pwv}, das ich ja auch längſt geleſen
                    habe. Denn – es bekommen, aufſchlagen, beginnen und nicht ſogleich weiterleſen,
                    angeſpannt, atemlos bis ans Ende – ich weiß nicht, welche dringende
                    Beſchäftigung mich davon abzuhalten vermocht hätte. Das Buch\pwindex{Schnitzler, Arthur 15.05.1862 – 21.10.1931@\textsc{Schnitzler, Arthur} (15.05.1862 – 21.10.1931), \emph{Schriftsteller, Mediziner}!Spiel im Morgengrauen. Novelle5.12.1926 – 9.1.1927@\strich\emph{Spiel im Morgengrauen. Novelle} {[}5.12.1926 – 9.1.1927{]}|pwv} iſt die Frucht vollkommener
                    Meiſterſchaft der Geſtaltenbildung ſowohl wie auch der Erzählungskunſt;
                    ſprachlich und anſchaulich, der Hand{\pb}lung wie der
                    Begründung nach eine reine Freude des Leſens.\pend
           \pstart
           Jemand\pwindex{?? [Leser von Spiel im Morgengrauen 1927] @\textsc{?? [Leser von Spiel im Morgengrauen 1927]}|pwv}, der gleich mir die
                        Novelle\pwindex{Schnitzler, Arthur 15.05.1862 – 21.10.1931@\textsc{Schnitzler, Arthur} (15.05.1862 – 21.10.1931), \emph{Schriftsteller, Mediziner}!Spiel im Morgengrauen. Novelle5.12.1926 – 9.1.1927@\strich\emph{Spiel im Morgengrauen. Novelle} {[}5.12.1926 – 9.1.1927{]}|pwv} geſpannt geleſen
                    hatte, ein philoſophiſcher, tiefblickender Geiſt, wandte ein, daß der Schluß
                    nicht befriedige, und auch ich empfinde das. Es hätte notgetan, ſagte der Betreffende\pwindex{?? [Leser von Spiel im Morgengrauen 1927] @\textsc{?? [Leser von Spiel im Morgengrauen 1927]}|pwv}, daß dem Tod
                    des Leutnants etwas vorausgegangen wäre, davon er ſelbſt erhöht hätte werden
                    müſſen: etwa die Annahme des Geldes, das die Frau ihm vielleicht hätte
                    mitbringen ſollen, und die Scham darüber wäre dann ein triftigerer Grund zur
                    Selbſtjuſtiz geweſen als bloß die Flucht. Ich {\pb}mußte dieſen
                    Gedanken als einleuchtend anerkennen. Was mir fehlt, iſt Transſzendenz –
                    vielleicht wäre ſie durch eine ſo geführte Linie der Motivierung ermöglicht
                    worden. Nicht wahr, Sie ſind mir nicht böſe, Herr Doktor, wenn ich aufrichtig
                    meine Empfindung ſchreibe?\pend
           \pstart
           In einer Zeit der Anarchie iſt das Erſcheinen des geſchloſſenen Kunſtwerks, des
                    gekonnten, gemeiſterten Formgebildes eine ſolche Seltenheit, daß ſich nur
                    Verehrung und Dankbarkeit geziemen. Laſſen Sie mich dieſe ſchönen Gefühle nicht
                    zurückhalten. Ich freue mich Ihrer ſtetig ſich harmoni{\pb}ſierenden
                    produktiven Kräfte, die Werk auf Werk hervorgeſtalten. Seit dem »Gang zum Weiher\pwindex{Schnitzler, Arthur 15.05.1862 – 21.10.1931@\textsc{Schnitzler, Arthur} (15.05.1862 – 21.10.1931), \emph{Schriftsteller, Mediziner}!Gang zum Weiher. Dramatische Dichtung1926@\strich\emph{Der Gang zum Weiher. Dramatische Dichtung} {[}1926{]}|pw}« war mir keine Ihrer
                    Dichtungen ſo nahe wie dieſe Novelle\pwindex{Schnitzler, Arthur 15.05.1862 – 21.10.1931@\textsc{Schnitzler, Arthur} (15.05.1862 – 21.10.1931), \emph{Schriftsteller, Mediziner}!Spiel im Morgengrauen. Novelle5.12.1926 – 9.1.1927@\strich\emph{Spiel im Morgengrauen. Novelle} {[}5.12.1926 – 9.1.1927{]}|pwv}.\pend
           \pstart
           In verehrender Geſinnung ergeben{\\[\baselineskip]}\spacefill\mbox{Felix Braun.}\pend
           \leftskip=0em{}          \endnumbering\briefempfaengerindex{Schnitzler, Arthur@\textsc{Schnitzler, Arthur}!zzzBraun, Felix@\emph{von Felix Braun}!1927-05-101@{10. 5. 1927}|)be}\mylabel{h}\end{ledgroupsized}  \newcommand{\dateiname}{L02486}\newcommand{\titel}{Felix Braun an Arthur Schnitzler, 10. 5. 1927}\newcommand{\editorInnen}{Martin Anton Müller und Gerd-Hermann Susen}%% latex-leseansicht-abspann.tex
%% Abspann für die Leseansicht.
%% Der Schalter \ifkorrekturansicht ist bereits durch den Vorspann gesetzt.

%% latex-abspann.tex
%% Gemeinsamer Abspann für Korrekturansicht und Leseansicht.
%% Setzt den Schalter \ifkorrekturansicht voraus (gesetzt in den
%% einbindenden Dateien latex-korrekturansicht-abspann.tex bzw.
%% latex-leseansicht-abspann.tex).
%% ---------------------------------------------------------------

\normalsize

% Das esempio-Environment wird nur in der Leseansicht benötigt
\ifkorrekturansicht\else
\newenvironment{esempio}[3]%
{
    \vspace{1.5ex}
    \rlap{\underline{#1}}
    \par
    \setlength{\parindent}{0cm}
    \nopagebreak
    \leftskip=#2cm
    \rightskip=#3cm
}
{
    \par
}
\fi

\doendnotes{C}
\bigskip
\vfill

\clearpage

\footnotesize

\ifkorrekturansicht
  \lohead{\textsc{register}}
\fi

% theindex-Environment neu definieren ohne reledmac
\makeatletter
\renewenvironment{theindex}{%
  \ifkorrekturansicht
    \section*{\indexname}%
  \else
    \subsubsection*{Index der erwähnten Entitäten}%
  \fi
  \setlength{\parindent}{0pt}%
  \setlength{\parskip}{0pt plus 0.3pt}%
  \let\item\@idxitem
}{%
  \ifkorrekturansicht\clearpage\fi
}
\makeatother

\IfFileExists{\jobname-pw.ind}{\input{\jobname-pw.ind}}{}

% Quellenangabe nur in der Leseansicht
\ifkorrekturansicht\else
% Fallback-Definitionen, falls die .tex-Datei \titel etc. nicht gesetzt hat
\providecommand{\titel}{}
\providecommand{\editorInnen}{}
\providecommand{\dateiname}{\jobname}

\vspace{3cm}

\vfill

\footnotesize
\textsc{Quelle}: \titel. Herausgegeben von {\editorInnen}. In: \emph{Arthur Schnitzler: Briefwechsel mit Autorinnen und Autoren}.
 Digitale Edition, https://schnitzler-briefe.acdh.oeaw.ac.at/{\dateiname}.html (Stand \today)
\fi

\end{document}


      