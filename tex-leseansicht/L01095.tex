\input{../tex-inputs/latex-pdf-vorspann}
\begin{center}
            \textcolor{red}{ENTWURF. ENTZIFFERUNG NOCH NICHT KORREKTURGELESEN}
                      \end{center}
            
               \section[Arthur Schnitzler an Richard Beer-Hofmann, 26. 1. 1901]{ Arthur Schnitzler an Richard Beer-Hofmann,
                    26. 1. 1901}\nopagebreak\mylabel{v}\rehead{ }\begin{ledgroupsized}[t]{13cm}\normalsize\beginnumbering\briefempfaengerindex{Beer-Hofmann, Richard@\textsc{Beer-Hofmann, Richard}!zzzSchnitzler, Arthur@\emph{von Arthur Schnitzler}!1901-01-261@{26. 1. 1901}|(be} \toendnotes[C]{\smallbreak\pagebreak[2]} \Standort{YCGL, MSS 31.}
\physDesc{Postkarte
\newline{}Handschrift: Bleistift, deutsche Kurrent\newline{}Versand: 1) Stempel: »\nobreak{}\oindex{IX., Alsergrund@\textbf{IX., Alsergrund}|pwk}Wien 9/\textcolor{gray}{3}, 26 1. 01, 10–11V\nobreak{}«.  2) Stempel: »\nobreak{}Wien 1, 26. 1. {[}1901{]}, Bestellt\nobreak{}«. \newline{}Ordnung: mit Bleistift von unbekannter Hand
                                    datiert »21. 1.« }\toendnotes[C]{\smallbreak}\pstart{}{\pb}Herrn Dr \textsc{Rich
                            Beer-Hofmann}\pend{}\pstart{}Wien\oindex{Wien@\textbf{Wien}|pw}\pend{}\pstart{}\textsc{I. Wollzeile 15}.\oindex{Wollzeile@\textbf{Wollzeile}|pw}\pend{}{\bigskip}\pstart
           \noindent{}{\pb}lieber, die morgige Vorleſg entfällt. Auf Wiederſehen heute
                    Abend Club\orgindex{Wiener Schachclub@Wiener Schachclub|pwv}\pend
           \pstart
           Herzlichſt Ihr{\\[\baselineskip]}\spacefill\mbox{A.}\pend
           \leftskip=0em{}\endnumbering\briefempfaengerindex{Beer-Hofmann, Richard@\textsc{Beer-Hofmann, Richard}!zzzSchnitzler, Arthur@\emph{von Arthur Schnitzler}!1901-01-261@{26. 1. 1901}|)be}\mylabel{h}\end{ledgroupsized}  \newcommand{\dateiname}{L01095}\newcommand{\titel}{Arthur Schnitzler an Richard Beer-Hofmann, 26. 1. 1901}\newcommand{\editorInnen}{Martin Anton Müller und Gerd-Hermann Susen}\input{../tex-inputs/latex-pdf-abspann}
      