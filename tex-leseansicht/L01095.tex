%% latex-korrekturansicht-vorspann.tex
%% Vorspann für die Korrekturansicht.
%% Lädt die gemeinsame Datei latex-vorspann.tex mit gesetztem Schalter.

\newif\ifkorrekturansicht
\korrekturansichttrue

\input{../tex-inputs/latex-vorspann}


\section[Arthur Schnitzler an Richard Beer-Hofmann, 26. 1. 1901]{L01095 Arthur Schnitzler an Richard Beer-Hofmann, 26. 1. 1901}
\nopagebreak\mylabel{L01095v}
\rehead{ }\normalsize\beginnumbering\briefempfaengerindex{Beer-Hofmann, Richard@\textsc{Beer-Hofmann, Richard}!zzzSchnitzler, Arthur@\emph{von Arthur Schnitzler}!1901-01-261@{26. 1. 1901}|(be}
\toendnotes[C]{\smallbreak\pagebreak[2]}\Standort{YCGL, MSS 31.}
\physDesc{Postkarte, 129 Zeichen
\newline{}Handschrift: Bleistift, deutsche Kurrent
\newline{}Versand: 1) Stempel: »\nobreak{}\oindex{IX., Alsergrund@\textbf{IX., Alsergrund}, \emph{A.ADM3}|pwk}Wien 9/\textcolor{gray}{3}, 26 1. 01, 10–11V\nobreak{}«.   2) Stempel: »\nobreak{}Wien 1, 26. 1. {[}1901{]}, Bestellt\nobreak{}«. 
\newline{}Ordnung: mit Bleistift von unbekannter Hand datiert »21. 1.« }\toendnotes[C]{\smallbreak}\pstart{}{\pb}Herrn Dr \textsc{Rich
                     Beer-Hofmann}\pend{}\pstart{}Wien\oindex{Wien@\textbf{Wien}, \emph{A.ADM2}|pw}\pend{}\pstart{}\textsc{I. Wollzeile 15}.\oindex{Wollzeile@\textbf{Wollzeile}, \emph{Straße (K.STR)}|pw}\pend{}{\bigskip}\vspace{1em}
\pstart
           \noindent{}{\pb}lieber, die morgige Vorleſg entfällt. Auf Wiederſehen heute Abend
                  Club\orgindex{Wiener Schachclub@Wiener Schachclub|pwv}\pend
           
\pstart
           Herzlichſt Ihr{\\[\baselineskip]}\spacefill\mbox{A.}\pend
           \leftskip=0em{}\selectlanguage{ngerman}\endnumbering\briefempfaengerindex{Beer-Hofmann, Richard@\textsc{Beer-Hofmann, Richard}!zzzSchnitzler, Arthur@\emph{von Arthur Schnitzler}!1901-01-261@{26. 1. 1901}|)be}\mylabel{L01095h}  \normalsize

\doendnotes{C}
\bigskip
\vfill

\clearpage

\footnotesize

\lohead{\textsc{register}}

% Definiere theindex-Environment komplett neu ohne reledmac
\makeatletter
\renewenvironment{theindex}{%
  \section*{\indexname}%
  \setlength{\parindent}{0pt}%
  \setlength{\parskip}{0pt plus 0.3pt}%
  \let\item\@idxitem
}{%
  \clearpage
}
\makeatother

\IfFileExists{\jobname-pw.ind}{\input{\jobname-pw.ind}}{}

\end{document}

      