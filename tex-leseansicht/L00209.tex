%% latex-leseansicht-vorspann.tex
%% Vorspann für die Leseansicht.
%% Lädt die gemeinsame Datei latex-vorspann.tex mit nicht gesetztem Schalter.

\newif\ifkorrekturansicht
\korrekturansichtfalse

\input{../tex-inputs/latex-vorspann}


         
         \renewcommand{\erwaehntePersonen}{Personen: Johann Schnitzler, Bertha von Suttner}
         \renewcommand{\erwaehnteOrte}{Orte: Eggenburg, Harmannsdorf, Schloss Harmannsdorf, Wien}
         \renewcommand{\erwaehnteWerke}{}
               \section[Arthur von Suttner an Arthur Schnitzler, 3. 5. 1893]{ Arthur von Suttner an Arthur Schnitzler, 3. 5. 1893}\nopagebreak\mylabel{v}\rehead{ }\begin{ledgroupsized}[t]{13cm}\normalsize\beginnumbering \toendnotes[C]{\smallbreak\pagebreak[2]} \Standort{CUL, Schnitzler, B 104.}
\physDesc{Brief, 1 Blatt, 2 Seiten, 609 Zeichen
\newline{}Handschrift: schwarze Tinte, deutsche Kurrent
\newline{}Schnitzler: mit Bleistift beschriftet: »\textsc{Suttner}« }\Standort{DLA, A:Schnitzler, HS.NZ85.1.4773.}
\physDesc{Brief, Maschinenschriftliche Abschrift, 1 Blatt, 1 Seite, 609 Zeichen
\newline{}Schreibmaschine}\toendnotes[C]{\smallbreak}\pstart
           \noindent{}{\pb}\textsc{Schloss Harmannsdorf}\oindex{Schloss Harmannsdorf@\textbf{Schloss Harmannsdorf}|pw}\hfill \textcolor{gray}{\textbf{am}}{ }3/V \textcolor{gray}{\textbf{189}}3\pend
           \pstart
           \textsc{b/Eggenburg\oindex{Eggenburg@\textbf{Eggenburg}|pw}.}\pend
           \pstart{}Sehr geehrter Herr,\pend\pstart
           Geſtatten Sie einem Ihnen perſönlich Unbekannten, Ihnen ſein warmes Beileid zu dem
               ſchweren Verluſte auszudrücken. Nicht allein Sie, – die Wiſſenſchaft, – die
               Menſchheit hat viel verloren. Ich habe den trefflichen Mann\pwindex{Schnitzler, Johann 10.04.1835 – 02.05.1893@\textsc{Schnitzler, Johann} (10.04.1835 – 02.05.1893), \emph{Mediziner}|pwv} gekannt, der in ſeiner ganzen
               Vollkraft den \uline{wahren} Heldentod geſtorben iſt, auf dem
                  \uline{wahren} Felde der Ehre – zur Rettung eines
               Mitmenſchen.\pend
           \pstart
           Meine Frau\pwindex{Suttner, Bertha von 09.06.1843 – 21.06.1914@\textsc{Suttner, Bertha von} (09.06.1843 – 21.06.1914), \emph{Schriftstellerin, Pazifistin}|pwv} ſchließt ſich mir
               an, und ich bitte, die Verſicherung unſerer wärmſten, unſerer herzlichſten {\pb}Teilnahme für ſich und Ihre Familie in
               Empfang zu nehmen.\pend
           \pstart
           In vorzüglicher Hochachtung{\\[\baselineskip]}\spacefill\mbox{A. G. v. Suttner}\pend
           \leftskip=0em{}
         
         \endnumbering\mylabel{h}\end{ledgroupsized}  \newcommand{\dateiname}{L00209}\newcommand{\titel}{Arthur von Suttner an Arthur Schnitzler, 3. 5. 1893}\newcommand{\editorInnen}{Martin Anton Müller und Gerd-Hermann Susen}%% latex-leseansicht-abspann.tex
%% Abspann für die Leseansicht.
%% Der Schalter \ifkorrekturansicht ist bereits durch den Vorspann gesetzt.

%% latex-abspann.tex
%% Gemeinsamer Abspann für Korrekturansicht und Leseansicht.
%% Setzt den Schalter \ifkorrekturansicht voraus (gesetzt in den
%% einbindenden Dateien latex-korrekturansicht-abspann.tex bzw.
%% latex-leseansicht-abspann.tex).
%% ---------------------------------------------------------------

\normalsize

% Das esempio-Environment wird nur in der Leseansicht benötigt
\ifkorrekturansicht\else
\newenvironment{esempio}[3]%
{
    \vspace{1.5ex}
    \rlap{\underline{#1}}
    \par
    \setlength{\parindent}{0cm}
    \nopagebreak
    \leftskip=#2cm
    \rightskip=#3cm
}
{
    \par
}
\fi

\doendnotes{C}
\bigskip
\vfill

\clearpage

\footnotesize

\ifkorrekturansicht
  \lohead{\textsc{register}}
\fi

% theindex-Environment neu definieren ohne reledmac
\makeatletter
\renewenvironment{theindex}{%
  \ifkorrekturansicht
    \section*{\indexname}%
  \else
    \subsubsection*{Index der erwähnten Entitäten}%
  \fi
  \setlength{\parindent}{0pt}%
  \setlength{\parskip}{0pt plus 0.3pt}%
  \let\item\@idxitem
}{%
  \ifkorrekturansicht\clearpage\fi
}
\makeatother

\IfFileExists{\jobname-pw.ind}{\input{\jobname-pw.ind}}{}

% Quellenangabe nur in der Leseansicht
\ifkorrekturansicht\else
% Fallback-Definitionen, falls die .tex-Datei \titel etc. nicht gesetzt hat
\providecommand{\titel}{}
\providecommand{\editorInnen}{}
\providecommand{\dateiname}{\jobname}

\vspace{3cm}

\vfill

\footnotesize
\textsc{Quelle}: \titel. Herausgegeben von {\editorInnen}. In: \emph{Arthur Schnitzler: Briefwechsel mit Autorinnen und Autoren}.
 Digitale Edition, https://schnitzler-briefe.acdh.oeaw.ac.at/{\dateiname}.html (Stand \today)
\fi

\end{document}


      