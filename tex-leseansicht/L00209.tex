%% latex-leseansicht-vorspann.tex
%% Vorspann für die Leseansicht.
%% Lädt die gemeinsame Datei latex-vorspann.tex mit nicht gesetztem Schalter.

\newif\ifkorrekturansicht
\korrekturansichtfalse

\input{../tex-inputs/latex-vorspann}


\section[Arthur von Suttner an Arthur Schnitzler, 3. 5. 1893]{L00209 Arthur von Suttner an Arthur Schnitzler, 3. 5. 1893}
\nopagebreak\mylabel{L00209v}
\rehead{ }\normalsize\beginnumbering\briefempfaengerindex{Schnitzler, Arthur@\textsc{Schnitzler, Arthur}!zzzSuttner, Arthur Gundaccar von@\emph{von Arthur Gundaccar von Suttner}!1893-05-032@{3. 5. 1893}|(be}
\toendnotes[C]{\smallbreak\pagebreak[2]}
\correspDesc{Versand  durch Arthur von Suttner am 3. 5. 1893 in Harmannsdorf
\newline{}Erhalt  durch Arthur Schnitzler im Zeitraum [4. 5. 1893
                  – 8. 5. 1893?] in Wien}\toendnotes[C]{\smallbreak}
\Standort{CUL, Schnitzler, B 104.}
\physDesc{Brief, 1 Blatt, 2 Seiten, 609 Zeichen
\newline{}Handschrift: schwarze Tinte, deutsche Kurrent
\newline{}Schnitzler: mit Bleistift beschriftet: »\textsc{Suttner}« }\Standort{DLA, A:Schnitzler, HS.NZ85.1.4773.}
\physDesc{Brief, maschinenschriftliche Abschrift, 1 Blatt, 1 Seite, 609 Zeichen
\newline{}Schreibmaschine}\toendnotes[C]{\smallbreak}
\pstart
           {\pb}\textsc{Schloss Harmannsdorf}\oindex{Schloss Harmannsdorf@\textbf{Schloss Harmannsdorf}, \emph{Schloss}|pw}\hfill \textcolor{gray}{\textbf{am}}{ }3/V \textcolor{gray}{\textbf{189}}3\pend
           
\pstart
           \textsc{b/Eggenburg\oindex{Eggenburg@\textbf{Eggenburg}, \emph{Hauptstadt}|pw}.}\pend
           
\pstart{}Sehr geehrter Herr,\pend\vspace{0.5em}
\pstart
           Geſtatten Sie einem Ihnen perſönlich Unbekannten, Ihnen{ }ſein warmes Beileid zu dem{ }ſchweren Verluſte auszudrücken. Nicht allein Sie, – die Wiſſenſchaft, – die
               Menſchheit hat viel verloren. Ich habe den trefflichen Mann\pwindex{Schnitzler, Johann 10.\,4.\,1835 Nagykanizsa – 2.\,5.\,1893 Wien@\textsc{Schnitzler, Johann} (10.\,4.\,1835 Nagykanizsa – 2.\,5.\,1893 Wien), \emph{Laryngologe}|pwv} gekannt, der in{ }ſeiner ganzen
               Vollkraft den \uline{wahren} Heldentod geſtorben iſt, auf dem
                  \uline{wahren} Felde der Ehre – zur Rettung eines
               Mitmenſchen.\pend
           
\pstart
           Meine Frau\pwindex{Suttner, Bertha von 9.\,6.\,1843 Prag – 21.\,6.\,1914 Wien@\textsc{Suttner, Bertha von} (9.\,6.\,1843 Prag – 21.\,6.\,1914 Wien), \emph{Schriftstellerin, Pazifistin, Schriftstellerin}|pwv}{ }ſchließt{ }ſich mir
               an, und ich bitte, die Verſicherung unſerer wärmſten, unſerer herzlichſten {\pb}Teilnahme für{ }ſich und Ihre Familie in
               Empfang zu nehmen.\pend
           
\pstart
           In vorzüglicher Hochachtung{\\[\baselineskip]}\spacefill\mbox{A. G. v. Suttner}\pend
           \leftskip=0em{}\selectlanguage{ngerman}\endnumbering\briefempfaengerindex{Schnitzler, Arthur@\textsc{Schnitzler, Arthur}!zzzSuttner, Arthur Gundaccar von@\emph{von Arthur Gundaccar von Suttner}!1893-05-032@{3. 5. 1893}|)be}\mylabel{L00209h}  \newcommand{\dateiname}{L00209}\newcommand{\titel}{Arthur von Suttner an Arthur Schnitzler, 3. 5. 1893}\newcommand{\editorInnen}{Martin Anton Müller und Gerd-Hermann Susen}%% latex-leseansicht-abspann.tex
%% Abspann für die Leseansicht.
%% Der Schalter \ifkorrekturansicht ist bereits durch den Vorspann gesetzt.

%% latex-abspann.tex
%% Gemeinsamer Abspann für Korrekturansicht und Leseansicht.
%% Setzt den Schalter \ifkorrekturansicht voraus (gesetzt in den
%% einbindenden Dateien latex-korrekturansicht-abspann.tex bzw.
%% latex-leseansicht-abspann.tex).
%% ---------------------------------------------------------------

\normalsize

% Das esempio-Environment wird nur in der Leseansicht benötigt
\ifkorrekturansicht\else
\newenvironment{esempio}[3]%
{
    \vspace{1.5ex}
    \rlap{\underline{#1}}
    \par
    \setlength{\parindent}{0cm}
    \nopagebreak
    \leftskip=#2cm
    \rightskip=#3cm
}
{
    \par
}
\fi

\doendnotes{C}
\bigskip
\vfill

\clearpage

\footnotesize

\ifkorrekturansicht
  \lohead{\textsc{register}}
\fi

% theindex-Environment neu definieren ohne reledmac
\makeatletter
\renewenvironment{theindex}{%
  \ifkorrekturansicht
    \section*{\indexname}%
  \else
    \subsubsection*{Index der erwähnten Entitäten}%
  \fi
  \setlength{\parindent}{0pt}%
  \setlength{\parskip}{0pt plus 0.3pt}%
  \let\item\@idxitem
}{%
  \ifkorrekturansicht\clearpage\fi
}
\makeatother

\IfFileExists{\jobname-pw.ind}{\input{\jobname-pw.ind}}{}

% Quellenangabe nur in der Leseansicht
\ifkorrekturansicht\else
% Fallback-Definitionen, falls die .tex-Datei \titel etc. nicht gesetzt hat
\providecommand{\titel}{}
\providecommand{\editorInnen}{}
\providecommand{\dateiname}{\jobname}

\vspace{3cm}

\vfill

\footnotesize
\textsc{Quelle}: \titel. Herausgegeben von {\editorInnen}. In: \emph{Arthur Schnitzler: Briefwechsel mit Autorinnen und Autoren}.
 Digitale Edition, https://schnitzler-briefe.acdh.oeaw.ac.at/{\dateiname}.html (Stand \today)
\fi

\end{document}


