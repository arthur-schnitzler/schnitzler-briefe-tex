%% latex-leseansicht-vorspann.tex
%% Vorspann für die Leseansicht.
%% Lädt die gemeinsame Datei latex-vorspann.tex mit nicht gesetztem Schalter.

\newif\ifkorrekturansicht
\korrekturansichtfalse

\input{../tex-inputs/latex-vorspann}


\section[Arthur Schnitzler: Widmungsexemplar von Fräulein Else an Berta Zuckerkandl, {{[}}28. 11. 1924?{{]}}]{L03994 Arthur Schnitzler: Widmungsexemplar von Fräulein Else an Berta
               Zuckerkandl, {[}28. 11. 1924?{]}}
\nopagebreak\mylabel{L03994v}
\rehead{ }\normalsize\beginnumbering\briefempfaengerindex{Zuckerkandl, Berta@\textsc{Zuckerkandl, Berta}!zzzSchnitzler, Arthur@\emph{von Arthur Schnitzler}!1924-11-281@{{[}28. 11. 1924?{]}}|(be}
\toendnotes[C]{\smallbreak\pagebreak[2]}
\correspDesc{Versand  durch Arthur Schnitzler am [28. 11. 1924?] in Wien
\newline{}Erhalt  durch Berta Zuckerkandl im Zeitraum [29. 11. 1924 – 3. 12. 1924?] in Paris}\toendnotes[C]{\smallbreak}
\Standort{Wien, Österreichische Nationalbibliothek, ZUC.5.1.SchFin LIT MAG.}
\physDesc{Widmung am Schmutztitel, 77 Zeichen
\newline{}Handschrift: schwarze Tinte, lateinische Kurrent}\toendnotes[C]{\smallbreak}
\pstart
           \noindent{}{\pb}Frau Hofrätin Berta Zuckerkandl\pend
           
\pstart
           \centering{}\textcolor{gray}{\textbf{ARTHUR
               SCHNITZLER}}\pend
           
\pstart
           \centering{}\textcolor{gray}{\textbf{FRÄULEIN ELSE\pwindex{Schnitzler, Arthur 15.\,5.\,1862 Wien – 21.\,10.\,1931 ebd.@\textsc{Schnitzler, Arthur} (15.\,5.\,1862 Wien – 21.\,10.\,1931 ebd.), \emph{Schriftsteller, Mediziner}!Fräulein Else@\strich\emph{Fräulein Else}|pw}}}\pend
           
\pstart
           freundſchaftlich herzlichſt {\\[\baselineskip]}\spacefill\mbox{Arthur Schnitzler}\pend
           \leftskip=0em{}\selectlanguage{ngerman}\vspace{1em}{\vspace{1\baselineskip}}
\pstart
           \centering{}{\pb}\textcolor{gray}{\textbf{\so{ARTHUR SCHNITZLER}}}\pend
           
\pstart
           \centering{}\textcolor{gray}{\textbf{FRÄULEIN ELSE\pwindex{Schnitzler, Arthur 15.\,5.\,1862 Wien – 21.\,10.\,1931 ebd.@\textsc{Schnitzler, Arthur} (15.\,5.\,1862 Wien – 21.\,10.\,1931 ebd.), \emph{Schriftsteller, Mediziner}!Fräulein Else@\strich\emph{Fräulein Else}|pw}}}\pend
           
\pstart
           \centering{}\textcolor{gray}{\textbf{NOVELLE}}\pend
           {\vspace{1\baselineskip}}
\pstart
           \centering{}\textcolor{gray}{\textbf{\label{K_L03994-1v}\edtext{1924}{\lemma{\textnormal{\emph{1924}}}\Cendnote{\textnormal{Schnitzlers Widmung
                  ist undatiert, dürfte aber zusammen mit den anderen verfasst sein, die er am 28. 11. 1924 im Verlagshaus\oindex{Wien@\textbf{Wien}!IV., Wieden@\textbf{IV., Wieden}!Paul Zsolnay Verlag@\textbf{Paul Zsolnay Verlag}|pwkv} eintrug.}}}\label{K_L03994-1}}}\pend
           
\pstart
           \centering{}\textcolor{gray}{\textbf{PAUL ZSOLNAY VERLAG\orgindex{Paul Zsolnay Verlag@Paul Zsolnay Verlag|pw}}}\pend
           
\pstart
           \centering{}\textcolor{gray}{\textbf{BERLIN\oindex{Berlin@\textbf{Berlin}, \emph{Hauptstadt}|pw} ⋅ WIEN\oindex{Wien@\textbf{Wien}, \emph{Verwaltungsgebiet}|pw} ⋅ LEIPZIG\oindex{Leipzig@\textbf{Leipzig}, \emph{Hauptstadt}|pw}}}\pend
           \selectlanguage{ngerman}\endnumbering\briefempfaengerindex{Zuckerkandl, Berta@\textsc{Zuckerkandl, Berta}!zzzSchnitzler, Arthur@\emph{von Arthur Schnitzler}!1924-11-281@{{[}28. 11. 1924?{]}}|)be}\mylabel{L03994h}
\begin{anhang}
\end{anhang}\newcommand{\dateiname}{L03994}\newcommand{\titel}{Arthur Schnitzler: Widmungsexemplar von Fräulein Else an Berta Zuckerkandl, [28. 11. 1924?]}\newcommand{\editorInnen}{Herausgegeben von Jahnke, SelmaMüller, Martin Anton}%% latex-leseansicht-abspann.tex
%% Abspann für die Leseansicht.
%% Der Schalter \ifkorrekturansicht ist bereits durch den Vorspann gesetzt.

%% latex-abspann.tex
%% Gemeinsamer Abspann für Korrekturansicht und Leseansicht.
%% Setzt den Schalter \ifkorrekturansicht voraus (gesetzt in den
%% einbindenden Dateien latex-korrekturansicht-abspann.tex bzw.
%% latex-leseansicht-abspann.tex).
%% ---------------------------------------------------------------

\normalsize

% Das esempio-Environment wird nur in der Leseansicht benötigt
\ifkorrekturansicht\else
\newenvironment{esempio}[3]%
{
    \vspace{1.5ex}
    \rlap{\underline{#1}}
    \par
    \setlength{\parindent}{0cm}
    \nopagebreak
    \leftskip=#2cm
    \rightskip=#3cm
}
{
    \par
}
\fi

\doendnotes{C}
\bigskip
\vfill

\clearpage

\footnotesize

\ifkorrekturansicht
  \lohead{\textsc{register}}
\fi

% theindex-Environment neu definieren ohne reledmac
\makeatletter
\renewenvironment{theindex}{%
  \ifkorrekturansicht
    \section*{\indexname}%
  \else
    \subsubsection*{Index der erwähnten Entitäten}%
  \fi
  \setlength{\parindent}{0pt}%
  \setlength{\parskip}{0pt plus 0.3pt}%
  \let\item\@idxitem
}{%
  \ifkorrekturansicht\clearpage\fi
}
\makeatother

\IfFileExists{\jobname-pw.ind}{\input{\jobname-pw.ind}}{}

% Quellenangabe nur in der Leseansicht
\ifkorrekturansicht\else
% Fallback-Definitionen, falls die .tex-Datei \titel etc. nicht gesetzt hat
\providecommand{\titel}{}
\providecommand{\editorInnen}{}
\providecommand{\dateiname}{\jobname}

\vspace{3cm}

\vfill

\footnotesize
\textsc{Quelle}: \titel. Herausgegeben von {\editorInnen}. In: \emph{Arthur Schnitzler: Briefwechsel mit Autorinnen und Autoren}.
 Digitale Edition, https://schnitzler-briefe.acdh.oeaw.ac.at/{\dateiname}.html (Stand \today)
\fi

\end{document}


