%% latex-leseansicht-vorspann.tex
%% Vorspann für die Leseansicht.
%% Lädt die gemeinsame Datei latex-vorspann.tex mit nicht gesetztem Schalter.

\newif\ifkorrekturansicht
\korrekturansichtfalse

\input{../tex-inputs/latex-vorspann}


         
         \renewcommand{\erwaehntePersonen}{Personen: Charles de Coster, Alexandre père Dumas,  G. Lenotre, Ricarda Huch, Viktor Franz Patzner}
         \renewcommand{\erwaehnteOrte}{Orte: Wien}
         \renewcommand{\erwaehnteWerke}{Werke: Bleus, Blancs et Rouges, Der große Krieg in Deutschland, Jean Christophe, Le mariage de Monsieur de Bréchard, Meine Memoiren, Märchenkomödie, Tyll Ulenspiegel und Lamm Goedzak}
               \section[Robert Adam an Arthur Schnitzler, 24. 9. 1916]{ Robert Adam an Arthur Schnitzler, 24. 9. 1916}\nopagebreak\mylabel{v}\rehead{ }\begin{ledgroupsized}[t]{13cm}\normalsize\beginnumbering \toendnotes[C]{\smallbreak\pagebreak[2]} \Standort{DLA, A:Schnitzler, HS.NZ85.1.4230,14.}
\physDesc{Brief, 1 Blatt, 4 Seiten, 2885 Zeichen
\newline{}Handschrift: schwarze Tinte, deutsche Kurrent
\newline{}Schnitzler: 1) mit Bleistift beschriftet: »\textsc{Adam}«  2) mit rotem Buntstift mehrere Unterstreichungen}\Standort{Wien, Österreichische Nationalbibliothek, Cod.ser. 52.263, 177.}
\physDesc{Brief, Maschinenschriftliche Abschrift, 1 Blatt, 1 Seite, 2885 Zeichen
\newline{}Schreibmaschine}\toendnotes[C]{\smallbreak}\pstart
           \raggedleft{}{\pb}Wien\oindex{Wien@\textbf{Wien}|pw}, am 24. September 1916\pend
           \pstart{}Hochverehrter Herr Doktor!\pend\pstart
           Ich vermute Sie, nach einem ſchönen und erholungsreichen Sommer, ſchon wieder nach
                  Wien\oindex{Wien@\textbf{Wien}|pw} zurückgekehrt und bin, Ihrer
               liebenswürdigen Erlaubnis eingedenk, auch schon unbeſcheiden genug, anzufragen, ob
               ich Sie einmal durch einen Beſuch ſtören darf?\pend
           \pstart
           Mir iſt die Zeit ſeit Ende meines Urlaubs unter unausgeſetzter und ſehr anſtrengender
               Amtsarbeit vergangen, und wenn Sie mich fragen ſollten, was ich in dieſen Monaten
               Dichteriſches geleiſtet, ſo müßte ich ſehr kleinlaut werden. Ich habe allerdings an
               einer {\pb}ſonderbaren Märchenkomödie\pwindex{Adam, Robert 20.04.1877 – 16.10.1961@\textsc{Adam, Robert} (20.04.1877 – 16.10.1961), \emph{Schriftsteller, Richter}!MaerchenkomoedieNone@\strich\emph{Märchenkomödie} {[}None{]}|pwv} zu ſchreiben begonnen, aber kraft- und
               zuglos, gewiſſermaßen \strikeout{unter de} im drückenden
               Bewußtſein der Unterernährtheit, nur an freien Sonntagnachmittagen: und daß dabei
               nichts Erſprießliches herausſchauen konnte, iſt gewiß klar.\pend
           \pstart
           (Dafür habe ich in den letzten Tagen ein leibliches Kind\pwindex{Patzner, Viktor Franz 13.09.1916 – 21.12.1982@\textsc{Patzner, Viktor Franz} (13.09.1916 – 21.12.1982), \emph{Rechtsanwalt}|pwv} gekriegt, einen Buben, der anſcheinend gut gedeiht,
               und damit darf ich mich tröſten).\pend
           \pstart
           Ich bin Ihnen für viele Bücher, die Sie mir anrieten, großen Dank ſchuldig: vor allem
               für den \textsc{Coster}\pwindex{Coster, Charles de 20.08.1827 – 07.05.1879@\textsc{Coster, Charles de} (20.08.1827 – 07.05.1879), \emph{Schriftsteller}|pw}’ſchen \textsc{Uhlenspiegel}\pwindex{Coster, Charles de 20.08.1827 – 07.05.1879@\textsc{Coster, Charles de} (20.08.1827 – 07.05.1879), \emph{Schriftsteller}!Tyll Ulenspiegel und Lamm Goedzak1867@\strich\emph{Tyll Ulenspiegel und Lamm Goedzak} {[}1867{]}|pw} und den \textsc{Jean-Christophe}\pwindex{\textcolor{red}{\textsuperscript{XXXX1 indx}}!Jean Christophe1904 – 1912@\strich\emph{Jean Christophe} {[}1904 – 1912{]}|pw} (ich halte ſchon beim erſten Bande). Auch den »Deutſchen Krieg\pwindex{Huch, Ricarda 18.07.1864 – 17.11.1947@\textsc{Huch, Ricarda} (18.07.1864 – 17.11.1947), \emph{Schriftstellerin}!grosse Krieg in Deutschland1912/1914@\strich\emph{Der große Krieg in Deutschland} {[}1912/1914{]}|pw}« der \textsc{Ricarda Huch}\pwindex{Huch, Ricarda 18.07.1864 – 17.11.1947@\textsc{Huch, Ricarda} (18.07.1864 – 17.11.1947), \emph{Schriftstellerin}|pw} habe ich zu zwei Dritteln geleſen, mit großer Hochachtung für den
               phantaſievollen Geiſt, der den Canvas der pragmatiſchen Geſchichtsſchreibung mit {\pb}farbigen Bildern gediegenſter Ausführung beſtickt hat;
               aber ich kann mir halt nicht helfen, ich komme über den Eindruck einer – gewiß
               vorzüglichen und nie geſchmackloſen – Handarbeit nicht \strikeout{hinaus} hinweg, allerdings der umfangreichſten und mühevollſten Handarbeit,
               die ich noch je geleſen habe; ich muß hinzufügen: auch der originellſten.\pend
           \pstart
           Eines der Bücher von \textsc{Lenotre}\pwindex{G. Lenotre 1855-10-07 – 1935-02-07@\textsc{G. Lenotre} (1855-10-07 – 1935-02-07), \emph{Schriftsteller, Historiker}|pw} (deſſen Bekanntſchaft ich auch Ihnen verdanke) leſe ich gerade: \textsc{Bleus, Blancs + Rouges}\pwindex{G. Lenotre 1855-10-07 – 1935-02-07@\textsc{G. Lenotre} (1855-10-07 – 1935-02-07), \emph{Schriftsteller, Historiker}!Bleus, Blancs et Rouges1912@\strich\emph{Bleus, Blancs et Rouges} {[}1912{]}|pw} und werde gewiß auch die andern leſen; in dem Zitierten iſt ein wunderſchöner
               Komödienſtoff zu finden (\textsc{Le mariage de Monsieur de Bréchard}\pwindex{G. Lenotre 1855-10-07 – 1935-02-07@\textsc{G. Lenotre} (1855-10-07 – 1935-02-07), \emph{Schriftsteller, Historiker}!Le mariage de Monsieur de Brechard1912@\strich\emph{Le mariage de Monsieur de Bréchard} {[}1912{]}|pw}). Unangenehm berührt mich nur die prononzierte Parteinahme des Autors, der ein
               erzkatholiſcher Royaliſt ſein muß, für jeden Antirevolutionär und gegen jeden
               Terroriſten: die zur Folge hat, daß ſeine hiſtoriſchen Novellen nur Engel und Teu{\pb}fel zu Helden haben.\pend
           \pstart
           Wegen der Memoiren\pwindex{Dumas, Alexandre pere 24.07.1802 – 05.12.1870@\textsc{Dumas, Alexandre père} (24.07.1802 – 05.12.1870), \emph{Schriftsteller}!Meine Memoiren1852 – 1856@\strich\emph{Meine Memoiren} {[}1852 – 1856{]}|pwv} von \textsc{Alexandre Dumas Père}\pwindex{Dumas, Alexandre pere 24.07.1802 – 05.12.1870@\textsc{Dumas, Alexandre père} (24.07.1802 – 05.12.1870), \emph{Schriftsteller}|pw} habe ich vergeblich die Wien\oindex{Wien@\textbf{Wien}|pw}er
               Buchhandlungen beſucht; ich weiß ſicher, daß ich ein Exemplar bei Sommerbeginn in
               einer Auslage ſah; es muß ſeither verkauft worden ſein. Selbſtverſtändlich ſteht
               Ihnen, hochverehrter Herr Doktor, mein Exemplar jederzeit zur Verfügung. Darf ich es
               Ihnen ſchicken?\pend
           \pstart
           Ich freue mich ſchon ungemein darauf, Sie wiederzuſehen: ohne Ihre Teilnahme, das
               fühle ich, wäre ich ſchon längſt entmutigt von allen Dichterplänen abgekommen und zum
               einfachen Wien\oindex{Wien@\textbf{Wien}|pw}er Bezirksrichter mit einigen
               Gelehrſamkeitsaſpirationen geworden. Und vielleicht bringe ich, wenn nur erſt dieſer
               Krieg vorüber iſt, doch noch etwas Anſtändiges zuwege.\pend
           \pstart
           Mit den freundlichſten Grüßen Ihr ergebener\pend
           \pstart \spacefill\mbox{Robert Adam}\pend{}
         
         \endnumbering\mylabel{h}\end{ledgroupsized}  \newcommand{\dateiname}{L02241}\newcommand{\titel}{Robert Adam an Arthur Schnitzler, 24. 9. 1916}\newcommand{\editorInnen}{Martin Anton Müller und Gerd-Hermann Susen}%% latex-leseansicht-abspann.tex
%% Abspann für die Leseansicht.
%% Der Schalter \ifkorrekturansicht ist bereits durch den Vorspann gesetzt.

%% latex-abspann.tex
%% Gemeinsamer Abspann für Korrekturansicht und Leseansicht.
%% Setzt den Schalter \ifkorrekturansicht voraus (gesetzt in den
%% einbindenden Dateien latex-korrekturansicht-abspann.tex bzw.
%% latex-leseansicht-abspann.tex).
%% ---------------------------------------------------------------

\normalsize

% Das esempio-Environment wird nur in der Leseansicht benötigt
\ifkorrekturansicht\else
\newenvironment{esempio}[3]%
{
    \vspace{1.5ex}
    \rlap{\underline{#1}}
    \par
    \setlength{\parindent}{0cm}
    \nopagebreak
    \leftskip=#2cm
    \rightskip=#3cm
}
{
    \par
}
\fi

\doendnotes{C}
\bigskip
\vfill

\clearpage

\footnotesize

\ifkorrekturansicht
  \lohead{\textsc{register}}
\fi

% theindex-Environment neu definieren ohne reledmac
\makeatletter
\renewenvironment{theindex}{%
  \ifkorrekturansicht
    \section*{\indexname}%
  \else
    \subsubsection*{Index der erwähnten Entitäten}%
  \fi
  \setlength{\parindent}{0pt}%
  \setlength{\parskip}{0pt plus 0.3pt}%
  \let\item\@idxitem
}{%
  \ifkorrekturansicht\clearpage\fi
}
\makeatother

\IfFileExists{\jobname-pw.ind}{\input{\jobname-pw.ind}}{}

% Quellenangabe nur in der Leseansicht
\ifkorrekturansicht\else
% Fallback-Definitionen, falls die .tex-Datei \titel etc. nicht gesetzt hat
\providecommand{\titel}{}
\providecommand{\editorInnen}{}
\providecommand{\dateiname}{\jobname}

\vspace{3cm}

\vfill

\footnotesize
\textsc{Quelle}: \titel. Herausgegeben von {\editorInnen}. In: \emph{Arthur Schnitzler: Briefwechsel mit Autorinnen und Autoren}.
 Digitale Edition, https://schnitzler-briefe.acdh.oeaw.ac.at/{\dateiname}.html (Stand \today)
\fi

\end{document}


      