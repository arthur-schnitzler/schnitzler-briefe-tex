%% latex-korrekturansicht-vorspann.tex
%% Vorspann für die Korrekturansicht.
%% Lädt die gemeinsame Datei latex-vorspann.tex mit gesetztem Schalter.

\newif\ifkorrekturansicht
\korrekturansichttrue

\input{../tex-inputs/latex-vorspann}


\section[Robert Adam an Arthur Schnitzler, 24. 9. 1916]{L02241 Robert Adam an Arthur Schnitzler, 24. 9. 1916}
\nopagebreak\mylabel{L02241v}
\rehead{ }\normalsize\beginnumbering\briefempfaengerindex{Schnitzler, Arthur@\textsc{Schnitzler, Arthur}!zzzAdam, Robert@\emph{von Robert Adam}!1916-09-241@{24. 9. 1916}|(be}
\toendnotes[C]{\smallbreak\pagebreak[2]}\Standort{DLA, A:Schnitzler, HS.NZ85.1.4230,14.}
\physDesc{Brief, 1 Blatt, 4 Seiten, 2885 Zeichen
\newline{}Handschrift: schwarze Tinte, deutsche Kurrent
\newline{}Schnitzler: 1) mit Bleistift beschriftet: »\textsc{Adam}«  2) mit rotem Buntstift mehrere Unterstreichungen}\Standort{Wien, Österreichische Nationalbibliothek, Cod.ser. 52.263, 177.}
\physDesc{Brief, maschinenschriftliche Abschrift1 Blatt, 1 Seite, 2885 Zeichen
\newline{}Schreibmaschine}\toendnotes[C]{\smallbreak}
\pstart
           \raggedleft{}{\pb}Wien\oindex{Wien@\textbf{Wien}, \emph{A.ADM2}|pw}, am 24. September 1916\pend
           
\pstart{}Hochverehrter Herr Doktor!\pend\vspace{0.5em}
\pstart
           Ich vermute Sie, nach einem ſchönen und erholungsreichen Sommer, ſchon wieder nach
                  Wien\oindex{Wien@\textbf{Wien}, \emph{A.ADM2}|pw} zurückgekehrt und bin, Ihrer
               liebenswürdigen Erlaubnis eingedenk, auch schon unbeſcheiden genug, anzufragen, ob
               ich Sie einmal durch einen Beſuch ſtören darf?\pend
           
\pstart
           Mir iſt die Zeit ſeit Ende meines Urlaubs unter unausgeſetzter und ſehr anſtrengender
               Amtsarbeit vergangen, und wenn Sie mich fragen ſollten, was ich in dieſen Monaten
               Dichteriſches geleiſtet, ſo müßte ich ſehr kleinlaut werden. Ich habe allerdings an
               einer {\pb}ſonderbaren Märchenkomödie\pwindex{Maerchenkomoedie@\emph{Märchenkomödie}|pwv} zu ſchreiben begonnen, aber kraft- und
               zuglos, gewiſſermaßen \strikeout{unter de} im drückenden
               Bewußtſein der Unterernährtheit, nur an freien Sonntagnachmittagen: und daß dabei
               nichts Erſprießliches herausſchauen konnte, iſt gewiß klar.\pend
           
\pstart
           (Dafür habe ich in den letzten Tagen ein leibliches Kind\pwindex{Patzner, Viktor Franz 13.09.1916 – 21.12.1982@\textsc{Patzner, Viktor Franz} (13.09.1916 – 21.12.1982), \emph{Rechtsanwalt/Rechtsanwältin}|pwv} gekriegt, einen Buben, der anſcheinend gut gedeiht,
               und damit darf ich mich tröſten).\pend
           
\pstart
           Ich bin Ihnen für viele Bücher, die Sie mir anrieten, großen Dank ſchuldig: vor allem
               für den \textsc{Coster}\pwindex{Coster, Charles de 20.08.1827 – 07.05.1879@\textsc{Coster, Charles de} (20.08.1827 – 07.05.1879), \emph{Schriftsteller/Schriftstellerin}|pw}’ſchen \textsc{Uhlenspiegel}\pwindex{Tyll Ulenspiegel und Lamm Goedzak@\emph{Tyll Ulenspiegel und Lamm Goedzak}|pw} und den \textsc{Jean-Christophe}\pwindex{Jean Christophe@\emph{Jean Christophe}|pw} (ich halte ſchon beim erſten Bande). Auch den »Deutſchen Krieg\pwindex{grosse Krieg in Deutschland@\emph{Der große Krieg in Deutschland}|pw}« der \textsc{Ricarda Huch}\pwindex{Huch, Ricarda 18.07.1864 – 17.11.1947@\textsc{Huch, Ricarda} (18.07.1864 – 17.11.1947), \emph{Schriftsteller/Schriftstellerin}|pw} habe ich zu zwei Dritteln geleſen, mit großer Hochachtung für den
               phantaſievollen Geiſt, der den Canvas der pragmatiſchen Geſchichtsſchreibung mit {\pb}farbigen Bildern gediegenſter Ausführung beſtickt hat;
               aber ich kann mir halt nicht helfen, ich komme über den Eindruck einer – gewiß
               vorzüglichen und nie geſchmackloſen – Handarbeit nicht \strikeout{hinaus} hinweg, allerdings der umfangreichſten und mühevollſten Handarbeit,
               die ich noch je geleſen habe; ich muß hinzufügen: auch der originellſten.\pend
           
\pstart
           Eines der Bücher von \textsc{Lenotre}\pwindex{G. Lenotre 1855-10-07 – 1935-02-07@\textsc{G. Lenotre} (1855-10-07 – 1935-02-07), \emph{Schriftsteller/Schriftstellerin, Historiker/Historikerin}|pw} (deſſen Bekanntſchaft ich auch Ihnen verdanke) leſe ich gerade: \textsc{Bleus, Blancs + Rouges}\pwindex{Bleus, Blancs et Rouges@\emph{Bleus, Blancs et Rouges}|pw} und werde gewiß auch die andern leſen; in dem Zitierten iſt ein wunderſchöner
               Komödienſtoff zu finden (\textsc{Le mariage de Monsieur de Bréchard}\pwindex{Le mariage de Monsieur de Brechard@\emph{Le mariage de Monsieur de Bréchard}|pw}). Unangenehm berührt mich nur die prononzierte Parteinahme des Autors, der ein
               erzkatholiſcher Royaliſt ſein muß, für jeden Antirevolutionär und gegen jeden
               Terroriſten: die zur Folge hat, daß ſeine hiſtoriſchen Novellen nur Engel und Teu{\pb}fel zu Helden haben.\pend
           
\pstart
           Wegen der Memoiren\pwindex{Meine Memoiren@\emph{Meine Memoiren}|pwv} von \textsc{Alexandre Dumas Père}\pwindex{Dumas, Alexandre pere 24.07.1802 – 05.12.1870@\textsc{Dumas, Alexandre père} (24.07.1802 – 05.12.1870), \emph{Schriftsteller/Schriftstellerin}|pw} habe ich vergeblich die Wien\oindex{Wien@\textbf{Wien}, \emph{A.ADM2}|pw}er
               Buchhandlungen beſucht; ich weiß ſicher, daß ich ein Exemplar bei Sommerbeginn in
               einer Auslage ſah; es muß ſeither verkauft worden ſein. Selbſtverſtändlich ſteht
               Ihnen, hochverehrter Herr Doktor, mein Exemplar jederzeit zur Verfügung. Darf ich es
               Ihnen ſchicken?\pend
           
\pstart
           Ich freue mich ſchon ungemein darauf, Sie wiederzuſehen: ohne Ihre Teilnahme, das
               fühle ich, wäre ich ſchon längſt entmutigt von allen Dichterplänen abgekommen und zum
               einfachen Wien\oindex{Wien@\textbf{Wien}, \emph{A.ADM2}|pw}er Bezirksrichter mit einigen
               Gelehrſamkeitsaſpirationen geworden. Und vielleicht bringe ich, wenn nur erſt dieſer
               Krieg vorüber iſt, doch noch etwas Anſtändiges zuwege.\pend
           
\pstart
           Mit den freundlichſten Grüßen Ihr ergebener\pend
           \pstart \spacefill\mbox{Robert Adam}\pend{}\selectlanguage{ngerman}\endnumbering\briefempfaengerindex{Schnitzler, Arthur@\textsc{Schnitzler, Arthur}!zzzAdam, Robert@\emph{von Robert Adam}!1916-09-241@{24. 9. 1916}|)be}\mylabel{L02241h}  \normalsize

\doendnotes{C}
\bigskip
\vfill

\clearpage

\footnotesize

\lohead{\textsc{register}}

% Definiere theindex-Environment komplett neu ohne reledmac
\makeatletter
\renewenvironment{theindex}{%
  \section*{\indexname}%
  \setlength{\parindent}{0pt}%
  \setlength{\parskip}{0pt plus 0.3pt}%
  \let\item\@idxitem
}{%
  \clearpage
}
\makeatother

\IfFileExists{\jobname-pw.ind}{\input{\jobname-pw.ind}}{}

\end{document}

      