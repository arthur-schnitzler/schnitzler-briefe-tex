%% latex-leseansicht-vorspann.tex
%% Vorspann für die Leseansicht.
%% Lädt die gemeinsame Datei latex-vorspann.tex mit nicht gesetztem Schalter.

\newif\ifkorrekturansicht
\korrekturansichtfalse

\input{../tex-inputs/latex-vorspann}


\section[Robert Adam an Arthur Schnitzler, 24. 9. 1916]{L02241 Robert Adam an Arthur Schnitzler, 24. 9. 1916}
\nopagebreak\mylabel{L02241v}
\rehead{ }\normalsize\beginnumbering\briefempfaengerindex{Schnitzler, Arthur@\textsc{Schnitzler, Arthur}!zzzAdam, Robert@\emph{von Robert Adam}!1916-09-241@{24. 9. 1916}|(be}
\toendnotes[C]{\smallbreak\pagebreak[2]}
\correspDesc{Versand  durch Robert Adam am 24. 9. 1916 in Wien
\newline{}Erhalt  durch Arthur Schnitzler im Zeitraum [24. 9. 1916
                  – 28. 9. 1916?] in Wien}\toendnotes[C]{\smallbreak}
\Standort{DLA, A:Schnitzler, HS.NZ85.1.4230,14.}
\physDesc{Brief, 1 Blatt, 4 Seiten, 2885 Zeichen
\newline{}Handschrift: schwarze Tinte, deutsche Kurrent
\newline{}Schnitzler: 1) mit Bleistift beschriftet: »\textsc{Adam}«  2) mit rotem Buntstift mehrere Unterstreichungen}\Standort{Wien, Österreichische Nationalbibliothek, Cod.ser. 52.263, 177.}
\physDesc{Brief, maschinenschriftliche Abschrift, 1 Blatt, 1 Seite, 2885 Zeichen
\newline{}Schreibmaschine}\toendnotes[C]{\smallbreak}
\pstart
           \raggedleft{}{\pb}Wien\oindex{Wien@\textbf{Wien}, \emph{Verwaltungsgebiet}|pw}, am 24. September 1916\pend
           
\pstart{}Hochverehrter Herr Doktor!\pend\vspace{0.5em}
\pstart
           Ich vermute Sie, nach einem{ }ſchönen und erholungsreichen Sommer,{ }ſchon wieder nach
                  Wien\oindex{Wien@\textbf{Wien}, \emph{Verwaltungsgebiet}|pw} zurückgekehrt und bin, Ihrer
               liebenswürdigen Erlaubnis eingedenk, auch schon unbeſcheiden genug, anzufragen, ob
               ich Sie einmal durch einen Beſuch{ }ſtören darf?\pend
           
\pstart
           Mir iſt die Zeit{ }ſeit Ende meines Urlaubs unter unausgeſetzter und{ }ſehr anſtrengender
               Amtsarbeit vergangen, und wenn Sie mich fragen{ }ſollten, was ich in dieſen Monaten
               Dichteriſches geleiſtet,{ }ſo müßte ich{ }ſehr kleinlaut werden. Ich habe allerdings an
               einer {\pb}ſonderbaren Märchenkomödie\pwindex{Adam, Robert 20.\,4.\,1877 Wien – 16.\,10.\,1961 Baden bei Wien@\textsc{Adam, Robert} (20.\,4.\,1877 Wien – 16.\,10.\,1961 Baden bei Wien), \emph{Schriftsteller, Richter}!Märchenkomödie@\strich\emph{Märchenkomödie}|pwv} zu{ }ſchreiben begonnen, aber kraft- und
               zuglos, gewiſſermaßen \strikeout{unter de} im drückenden
               Bewußtſein der Unterernährtheit, nur an freien Sonntagnachmittagen: und daß dabei
               nichts Erſprießliches herausſchauen konnte, iſt gewiß klar.\pend
           
\pstart
           (Dafür habe ich in den letzten Tagen ein leibliches Kind\pwindex{Patzner, Viktor Franz 13.\,9.\,1916 Wien – 21.\,12.\,1982 ebd.@\textsc{Patzner, Viktor Franz} (13.\,9.\,1916 Wien – 21.\,12.\,1982 ebd.), \emph{Rechtsanwalt}|pwv} gekriegt, einen Buben, der anſcheinend gut gedeiht,
               und damit darf ich mich tröſten).\pend
           
\pstart
           Ich bin Ihnen für viele Bücher, die Sie mir anrieten, großen Dank{ }ſchuldig: vor allem
               für den \textsc{Coster}\pwindex{Coster, Charles de 20.\,8.\,1827 München – 7.\,5.\,1879 Ixelles@\textsc{Coster, Charles de} (20.\,8.\,1827 München – 7.\,5.\,1879 Ixelles), \emph{Schriftsteller}|pw}’ſchen \textsc{Uhlenspiegel}\pwindex{Coster, Charles de 20.\,8.\,1827 München – 7.\,5.\,1879 Ixelles@\textsc{Coster, Charles de} (20.\,8.\,1827 München – 7.\,5.\,1879 Ixelles), \emph{Schriftsteller}!Tyll Ulenspiegel und Lamm Goedzak@\strich\emph{Tyll Ulenspiegel und Lamm Goedzak}|pw} und den \textsc{Jean-Christophe}\pwindex{\textcolor{red}{\textsuperscript{XXXX indx1}}!Jean-Christophe@\strich\emph{Jean-Christophe}|pw} (ich halte{ }ſchon beim erſten Bande). Auch den »Deutſchen Krieg\pwindex{Huch, Ricarda 18.\,7.\,1864 Braunschweig – 17.\,11.\,1947 Schönberg@\textsc{Huch, Ricarda} (18.\,7.\,1864 Braunschweig – 17.\,11.\,1947 Schönberg), \emph{Schriftstellerin}!große Krieg in Deutschland@\strich\emph{Der große Krieg in Deutschland}|pw}« der \textsc{Ricarda Huch}\pwindex{Huch, Ricarda 18.\,7.\,1864 Braunschweig – 17.\,11.\,1947 Schönberg@\textsc{Huch, Ricarda} (18.\,7.\,1864 Braunschweig – 17.\,11.\,1947 Schönberg), \emph{Schriftstellerin}|pw} habe ich zu zwei Dritteln geleſen, mit großer Hochachtung für den
               phantaſievollen Geiſt, der den Canvas der pragmatiſchen Geſchichtsſchreibung mit {\pb}farbigen Bildern gediegenſter Ausführung beſtickt hat;
               aber ich kann mir halt nicht helfen, ich komme über den Eindruck einer – gewiß
               vorzüglichen und nie geſchmackloſen – Handarbeit nicht \strikeout{hinaus} hinweg, allerdings der umfangreichſten und mühevollſten Handarbeit,
               die ich noch je geleſen habe; ich muß hinzufügen: auch der originellſten.\pend
           
\pstart
           Eines der Bücher von \textsc{Lenotre}\pwindex{G. Lenotre 7.\,10.\,1855 Richemont – 7.\,2.\,1935 Paris@\textsc{G. Lenotre} (7.\,10.\,1855 Richemont – 7.\,2.\,1935 Paris), \emph{Schriftsteller, Historiker}|pw} (deſſen Bekanntſchaft ich auch Ihnen verdanke) leſe ich gerade: \textsc{Bleus, Blancs + Rouges}\pwindex{G. Lenotre 7.\,10.\,1855 Richemont – 7.\,2.\,1935 Paris@\textsc{G. Lenotre} (7.\,10.\,1855 Richemont – 7.\,2.\,1935 Paris), \emph{Schriftsteller, Historiker}!Bleus, Blancs et Rouges@\strich\emph{Bleus, Blancs et Rouges}|pw} und werde gewiß auch die andern leſen; in dem Zitierten iſt ein wunderſchöner
               Komödienſtoff zu finden (\textsc{Le mariage de Monsieur de Bréchard}\pwindex{G. Lenotre 7.\,10.\,1855 Richemont – 7.\,2.\,1935 Paris@\textsc{G. Lenotre} (7.\,10.\,1855 Richemont – 7.\,2.\,1935 Paris), \emph{Schriftsteller, Historiker}!Le mariage de Monsieur de Bréchard@\strich\emph{Le mariage de Monsieur de Bréchard}|pw}). Unangenehm berührt mich nur die prononzierte Parteinahme des Autors, der ein
               erzkatholiſcher Royaliſt{ }ſein muß, für jeden Antirevolutionär und gegen jeden
               Terroriſten: die zur Folge hat, daß{ }ſeine hiſtoriſchen Novellen nur Engel und Teu{\pb}fel zu Helden haben.\pend
           
\pstart
           Wegen der Memoiren\pwindex{Dumas, Alexandre père 24.\,7.\,1802 Villers-Cotterêts – 5.\,12.\,1870 Puys@\textsc{Dumas, Alexandre père} (24.\,7.\,1802 Villers-Cotterêts – 5.\,12.\,1870 Puys), \emph{Schriftsteller}!Meine Memoiren@\strich\emph{Meine Memoiren}|pwv} von \textsc{Alexandre Dumas Père}\pwindex{Dumas, Alexandre père 24.\,7.\,1802 Villers-Cotterêts – 5.\,12.\,1870 Puys@\textsc{Dumas, Alexandre père} (24.\,7.\,1802 Villers-Cotterêts – 5.\,12.\,1870 Puys), \emph{Schriftsteller}|pw} habe ich vergeblich die Wien\oindex{Wien@\textbf{Wien}, \emph{Verwaltungsgebiet}|pw}er
               Buchhandlungen beſucht; ich weiß{ }ſicher, daß ich ein Exemplar bei Sommerbeginn in
               einer Auslage{ }ſah; es muß{ }ſeither verkauft worden{ }ſein. Selbſtverſtändlich{ }ſteht
               Ihnen, hochverehrter Herr Doktor, mein Exemplar jederzeit zur Verfügung. Darf ich es
               Ihnen{ }ſchicken?\pend
           
\pstart
           Ich freue mich{ }ſchon ungemein darauf, Sie wiederzuſehen: ohne Ihre Teilnahme, das
               fühle ich, wäre ich{ }ſchon längſt entmutigt von allen Dichterplänen abgekommen und zum
               einfachen Wien\oindex{Wien@\textbf{Wien}, \emph{Verwaltungsgebiet}|pw}er Bezirksrichter mit einigen
               Gelehrſamkeitsaſpirationen geworden. Und vielleicht bringe ich, wenn nur erſt dieſer
               Krieg vorüber iſt, doch noch etwas Anſtändiges zuwege.\pend
           
\pstart
           Mit den freundlichſten Grüßen Ihr ergebener\pend
           \pstart \spacefill\mbox{Robert Adam}\pend{}\selectlanguage{ngerman}\endnumbering\briefempfaengerindex{Schnitzler, Arthur@\textsc{Schnitzler, Arthur}!zzzAdam, Robert@\emph{von Robert Adam}!1916-09-241@{24. 9. 1916}|)be}\mylabel{L02241h}  \newcommand{\dateiname}{L02241}\newcommand{\titel}{Robert Adam an Arthur Schnitzler, 24. 9. 1916}\newcommand{\editorInnen}{Martin Anton Müller und Gerd-Hermann Susen}%% latex-leseansicht-abspann.tex
%% Abspann für die Leseansicht.
%% Der Schalter \ifkorrekturansicht ist bereits durch den Vorspann gesetzt.

%% latex-abspann.tex
%% Gemeinsamer Abspann für Korrekturansicht und Leseansicht.
%% Setzt den Schalter \ifkorrekturansicht voraus (gesetzt in den
%% einbindenden Dateien latex-korrekturansicht-abspann.tex bzw.
%% latex-leseansicht-abspann.tex).
%% ---------------------------------------------------------------

\normalsize

% Das esempio-Environment wird nur in der Leseansicht benötigt
\ifkorrekturansicht\else
\newenvironment{esempio}[3]%
{
    \vspace{1.5ex}
    \rlap{\underline{#1}}
    \par
    \setlength{\parindent}{0cm}
    \nopagebreak
    \leftskip=#2cm
    \rightskip=#3cm
}
{
    \par
}
\fi

\doendnotes{C}
\bigskip
\vfill

\clearpage

\footnotesize

\ifkorrekturansicht
  \lohead{\textsc{register}}
\fi

% theindex-Environment neu definieren ohne reledmac
\makeatletter
\renewenvironment{theindex}{%
  \ifkorrekturansicht
    \section*{\indexname}%
  \else
    \subsubsection*{Index der erwähnten Entitäten}%
  \fi
  \setlength{\parindent}{0pt}%
  \setlength{\parskip}{0pt plus 0.3pt}%
  \let\item\@idxitem
}{%
  \ifkorrekturansicht\clearpage\fi
}
\makeatother

\IfFileExists{\jobname-pw.ind}{\input{\jobname-pw.ind}}{}

% Quellenangabe nur in der Leseansicht
\ifkorrekturansicht\else
% Fallback-Definitionen, falls die .tex-Datei \titel etc. nicht gesetzt hat
\providecommand{\titel}{}
\providecommand{\editorInnen}{}
\providecommand{\dateiname}{\jobname}

\vspace{3cm}

\vfill

\footnotesize
\textsc{Quelle}: \titel. Herausgegeben von {\editorInnen}. In: \emph{Arthur Schnitzler: Briefwechsel mit Autorinnen und Autoren}.
 Digitale Edition, https://schnitzler-briefe.acdh.oeaw.ac.at/{\dateiname}.html (Stand \today)
\fi

\end{document}


