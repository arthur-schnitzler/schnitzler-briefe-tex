%% latex-leseansicht-vorspann.tex
%% Vorspann für die Leseansicht.
%% Lädt die gemeinsame Datei latex-vorspann.tex mit nicht gesetztem Schalter.

\newif\ifkorrekturansicht
\korrekturansichtfalse

\input{../tex-inputs/latex-vorspann}

\begin{center}
            \textcolor{red}{ENTWURF. ENTZIFFERUNG NOCH NICHT KORREKTURGELESEN}
                      \end{center}
            
               \section[Arthur Schnitzler an Richard Beer-Hofmann, 1. 3. 1895]{ Arthur Schnitzler an Richard Beer-Hofmann, 1. 3. 1895}\nopagebreak\mylabel{v}\rehead{ }\begin{ledgroupsized}[t]{13cm}\normalsize\beginnumbering\briefempfaengerindex{Beer-Hofmann, Richard@\textsc{Beer-Hofmann, Richard}!zzzSchnitzler, Arthur@\emph{von Arthur Schnitzler}!1895-03-011@{1. 3. 1895}|(be} \toendnotes[C]{\smallbreak\pagebreak[2]} \Standort{YCGL, MSS 31.}
\physDesc{Postkarte
\newline{}Handschrift: Bleistift, deutsche Kurrent\newline{}Versand: 1) Rohrpost 2) Stempel: »\nobreak{}\oindex{I., Innere Stadt@\textbf{I., Innere Stadt}|pwk}Wien 1/1, 1 III 95, 11 10 V\nobreak{}«. 3) Stempel: »\nobreak{}\oindex{I., Innere Stadt@\textbf{I., Innere Stadt}|pwk}Wien 1/1, 1 III 95, 11 20 V\nobreak{}«. }\toendnotes[C]{\smallbreak}\pstart{}{\pb}Herrn \textsc{Dr. Rich
                     Beer Hofmann}\pend{}\pstart{}Wien\oindex{Wien@\textbf{Wien}|pw}\pend{}\pstart{}\textsc{I. Wollzeile 15\oindex{Wollzeile@\textbf{Wollzeile}|pw}}\pend{}{\bigskip}\pstart{}{\pb}L. R.\pend\pstart
           Bitte Nachricht, ob Sie morgen Samſtag mit mir \label{K_L00423_1v}\edtext{ins Theater}{\lemma{\textnormal{\emph{ins Theater}}}\Cendnote{\textnormal{Schnitzler\pwindex{Schnitzler, Arthur 15.05.1862 – 21.10.1931@\textsc{Schnitzler, Arthur} (15.05.1862 – 21.10.1931), \emph{Schriftsteller, Mediziner}|pwk} ging
                  stattdessen in die Inszenierung der Operette \emph{Olympia}\pwindex{\textcolor{red}{\textsuperscript{XXXX1 indx}}!Olympia. Operette in 3 Acten9.2.1894 – 9.2.1894@\strich\emph{Olympia. Operette in 3 Acten} {[}9.2.1894 – 9.2.1894{]}|pwk}\pwindex{\textcolor{red}{\textsuperscript{XXXX1 indx}}!Olympia. Operette in 3 Acten9.2.1894 – 9.2.1894@\strich\emph{Olympia. Operette in 3 Acten} {[}9.2.1894 – 9.2.1894{]}|pwk}.}}}\label{K_L00423_1h} wollen (zu »\textsc{Chansonette\pwindex{\textcolor{red}{\textsuperscript{XXXX1 indx}}!Chansonette. Operette in 3 Acten1895 – 1895@\strich\emph{Die Chansonette. Operette in 3 Acten} {[}1895 – 1895{]}|pw}\pwindex{\textcolor{red}{\textsuperscript{XXXX1 indx}}!Chansonette. Operette in 3 Acten1895 – 1895@\strich\emph{Die Chansonette. Operette in 3 Acten} {[}1895 – 1895{]}|pw}}«) u ob Sie aufn \label{K_L00423_2v}\edtext{Gſchnas}{\lemma{\textnormal{\emph{Gſchnas}}}\Cendnote{\textnormal{wienerisch: Kostümfest}}}\label{K_L00423_2h} gehn?\pend
           \pstart Herzlich grüßt \spacefill\mbox{Arthur}\pend{}\pstart
           \noindent{}Bin heute Abend nach 10 beſti{\geminationm}t \textsc{Grst\oindex{Cafe Griensteidl@\textbf{Café Griensteidl}|pw}}\pend
           \endnumbering\briefempfaengerindex{Beer-Hofmann, Richard@\textsc{Beer-Hofmann, Richard}!zzzSchnitzler, Arthur@\emph{von Arthur Schnitzler}!1895-03-011@{1. 3. 1895}|)be}\mylabel{h}\end{ledgroupsized}  \newcommand{\dateiname}{L00423}\newcommand{\titel}{Arthur Schnitzler an Richard Beer-Hofmann, 1. 3. 1895}\newcommand{\editorInnen}{Martin Anton Müller und Gerd-Hermann Susen}%% latex-leseansicht-abspann.tex
%% Abspann für die Leseansicht.
%% Der Schalter \ifkorrekturansicht ist bereits durch den Vorspann gesetzt.

%% latex-abspann.tex
%% Gemeinsamer Abspann für Korrekturansicht und Leseansicht.
%% Setzt den Schalter \ifkorrekturansicht voraus (gesetzt in den
%% einbindenden Dateien latex-korrekturansicht-abspann.tex bzw.
%% latex-leseansicht-abspann.tex).
%% ---------------------------------------------------------------

\normalsize

% Das esempio-Environment wird nur in der Leseansicht benötigt
\ifkorrekturansicht\else
\newenvironment{esempio}[3]%
{
    \vspace{1.5ex}
    \rlap{\underline{#1}}
    \par
    \setlength{\parindent}{0cm}
    \nopagebreak
    \leftskip=#2cm
    \rightskip=#3cm
}
{
    \par
}
\fi

\doendnotes{C}
\bigskip
\vfill

\clearpage

\footnotesize

\ifkorrekturansicht
  \lohead{\textsc{register}}
\fi

% theindex-Environment neu definieren ohne reledmac
\makeatletter
\renewenvironment{theindex}{%
  \ifkorrekturansicht
    \section*{\indexname}%
  \else
    \subsubsection*{Index der erwähnten Entitäten}%
  \fi
  \setlength{\parindent}{0pt}%
  \setlength{\parskip}{0pt plus 0.3pt}%
  \let\item\@idxitem
}{%
  \ifkorrekturansicht\clearpage\fi
}
\makeatother

\IfFileExists{\jobname-pw.ind}{\input{\jobname-pw.ind}}{}

% Quellenangabe nur in der Leseansicht
\ifkorrekturansicht\else
% Fallback-Definitionen, falls die .tex-Datei \titel etc. nicht gesetzt hat
\providecommand{\titel}{}
\providecommand{\editorInnen}{}
\providecommand{\dateiname}{\jobname}

\vspace{3cm}

\vfill

\footnotesize
\textsc{Quelle}: \titel. Herausgegeben von {\editorInnen}. In: \emph{Arthur Schnitzler: Briefwechsel mit Autorinnen und Autoren}.
 Digitale Edition, https://schnitzler-briefe.acdh.oeaw.ac.at/{\dateiname}.html (Stand \today)
\fi

\end{document}


      