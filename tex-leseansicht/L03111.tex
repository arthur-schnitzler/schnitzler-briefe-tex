%% latex-korrekturansicht-vorspann.tex
%% Vorspann für die Korrekturansicht.
%% Lädt die gemeinsame Datei latex-vorspann.tex mit gesetztem Schalter.

\newif\ifkorrekturansicht
\korrekturansichttrue

\input{../tex-inputs/latex-vorspann}


\section[Felix Salten an Arthur Schnitzler, 8. 8. 1892]{L03111 Felix Salten an Arthur Schnitzler, 8. 8. 1892}
\nopagebreak\mylabel{L03111v}
\rehead{ }\normalsize\beginnumbering\briefempfaengerindex{Schnitzler, Arthur@\textsc{Schnitzler, Arthur}!zzzSalten, Felix@\emph{von Felix Salten}!1892-08-081@{8. 8. 1892}|(be}
\toendnotes[C]{\smallbreak\pagebreak[2]}\Standort{CUL, Schnitzler, B 89, A 1.}
\physDesc{Brief, 1 Blatt, 4 Seiten, 858 Zeichen
\newline{}Handschrift: Bleistift, lateinische Kurrent
\newline{}Ordnung: mit Bleistift von unbekannter Hand nummeriert: »14« }
\buchAbdrucke{\weitereDrucke{Hermann Bahr, Arthur Schnitzler: \emph{Briefwechsel, Aufzeichnungen, Dokumente (1891–1931)}. Göttingen: \emph{Wallstein} 2018, S. 80.} }\toendnotes[C]{\smallbreak}
\pstart
           \raggedleft{}{\pb}Unterach\oindex{Unterach am Attersee@\textbf{Unterach am Attersee}, \emph{P.PPL}|pw}{ }8. VIII. 92.\pend
           \vspace{0.5em}
\pstart
           Lieber Freund!{ }Samstag{ }Abend wollte ich ins Kremser\oindex{Cafe Kremser@\textbf{Café Kremser}, \emph{Kaffeehaus (K.KAF)}|pw} kommen u
               ihnen Adieu sagen, da ich erst Sonntag zu reisen
               gedachte. Allein um 8 Uhr Abd. erhielt ich meine Kleider und so fuhr ich
               also zur selbigen Stunde. Seien Sie also nicht böse. Hier ist’s wunderschön, u ich
               denke oft an Sie u. an Ihre Arbeiten. Schreiben Sie mir, bitte, bald was Sie
               treiben.\pend
           
\pstart
           {\pb}Ich hoffe hier\oindex{Unterach am Attersee@\textbf{Unterach am Attersee}, \emph{P.PPL}|pwv} einiges arbeiten zu können,
               da man ganz ungezwungen lebt u tagelang allein sein kann. Nächste Woche will ich zu
                  Richard\pwindex{Beer-Hofmann, Richard 1866-07-11 – 1945-09-26@\textsc{Beer-Hofmann, Richard} (1866-07-11 – 1945-09-26), \emph{Schriftsteller/Schriftstellerin}|pw} nach Ischl\oindex{Bad Ischl@\textbf{Bad Ischl}, \emph{P.PPL}|pw} hinüber, und werde auch Loris\pwindex{Hofmannsthal, Hugo von 1874-02-01 – 1929-07-15@\textsc{Hofmannsthal, Hugo von} (1874-02-01 – 1929-07-15), \emph{Schriftsteller/Schriftstellerin}|pw}
               davon verständigen. Paul{ }{\pb}Horn\pwindex{Horn, Paul 13.02.1867 – 18.01.1936@\textsc{Horn, Paul} (13.02.1867 – 18.01.1936), \emph{Fabrikant/Fabrikantin}|pw} soll heute{ }Nachmittag ankommen. Leben Sie wol u. schreiben Sie bald, auch wie es
               mit jenem \label{K_L03111-1v}\edtext{Engagement nach Deutschld\oindex{Deutschland@\textbf{Deutschland}, \emph{A.PCLI}|pw}}{\lemma{\textnormal{\emph{Engagement nach Deutschld}}}\Cendnote{\textnormal{für Marie Glümer\pwindex{Gluemer, Marie 03.07.1867 – 16.11.1925@\textsc{Glümer, Marie} (03.07.1867 – 16.11.1925), \emph{Schauspieler/Schauspielerin}|pwk}}}}\label{K_L03111-1} steht.\pend
           
\pstart
           Ich werde übrigens auch bald wieder schreiben, sobald ich Ihnen künstlerisch ei{\pb}niges Neue zu sagen habe. Grüßen
               Sie Schwarzkopf\pwindex{Schwarzkopf, Gustav 07.11.1853 – 13.11.1939@\textsc{Schwarzkopf, Gustav} (07.11.1853 – 13.11.1939), \emph{Schriftsteller/Schriftstellerin}|pw} u. Bahr\pwindex{Bahr, Hermann 19.07.1863 – 15.01.1934@\textsc{Bahr, Hermann} (19.07.1863 – 15.01.1934), \emph{Schriftsteller/Schriftstellerin, Kritiker/Kritikerin}|pw}.\pend
           
\pstart
           Herzlichst Ihr {\\[\baselineskip]}treuester {\\[\baselineskip]}\spacefill\mbox{Salten}\pend
           \leftskip=0em{}
\pstart
           \noindent{}Unterach\oindex{Unterach am Attersee@\textbf{Unterach am Attersee}, \emph{P.PPL}|pw}, Berghof\oindex{Berghof@\textbf{Berghof}, \emph{Wohngebäude (K.WHS)}|pw}.\pend
           \selectlanguage{ngerman}\endnumbering\briefempfaengerindex{Schnitzler, Arthur@\textsc{Schnitzler, Arthur}!zzzSalten, Felix@\emph{von Felix Salten}!1892-08-081@{8. 8. 1892}|)be}\mylabel{L03111h}  \normalsize

\doendnotes{C}
\bigskip
\vfill

\clearpage

\footnotesize

\lohead{\textsc{register}}

% Definiere theindex-Environment komplett neu ohne reledmac
\makeatletter
\renewenvironment{theindex}{%
  \section*{\indexname}%
  \setlength{\parindent}{0pt}%
  \setlength{\parskip}{0pt plus 0.3pt}%
  \let\item\@idxitem
}{%
  \clearpage
}
\makeatother

\IfFileExists{\jobname-pw.ind}{\input{\jobname-pw.ind}}{}

\end{document}

      