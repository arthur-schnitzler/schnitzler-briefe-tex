%% latex-korrekturansicht-vorspann.tex
%% Vorspann für die Korrekturansicht.
%% Lädt die gemeinsame Datei latex-vorspann.tex mit gesetztem Schalter.

\newif\ifkorrekturansicht
\korrekturansichttrue

\input{../tex-inputs/latex-vorspann}


\section[ Felix Salten an Arthur Schnitzler, 30. 3. 1904]{L03394 Felix Salten an Arthur Schnitzler, 30. 3. 1904}
\nopagebreak\mylabel{L03394v}
\rehead{ }\normalsize\beginnumbering\briefempfaengerindex{Schnitzler, Arthur@\textsc{Schnitzler, Arthur}!zzzSalten, Felix@\emph{von Felix Salten}!1904-03-301@{30. 3. 1904}|(be}
\toendnotes[C]{\smallbreak\pagebreak[2]}\Standort{CUL, Schnitzler, B 89, B 1.}
\physDesc{Kartenbrief, 1068 Zeichen
\newline{}Handschrift: schwarze Tinte, lateinische Kurrent
\newline{}Versand: 1) Stempel: »\nobreak{}\oindex{I., Innere Stadt@\textbf{I., Innere Stadt}, \emph{A.ADM3}|pwk}Wien 1/1, 30 III 04, 2 30N\nobreak{}«.   2) Stempel: »\nobreak{}\oindex{XVIII., Waehring@\textbf{XVIII., Währing}, \emph{A.ADM3}|pwk}Wien 18/1 11\textcolor{gray}{1}, 30 \textcolor{gray}{III 04}, 3 10N\nobreak{}«. 
\newline{}Schnitzler: mit Bleistift datiert: »30. 3. 904.–« und Vermerk: »S{[}alten{]}« 
\newline{}Ordnung: mit Bleistift von unbekannter Hand nummeriert: »186« }\toendnotes[C]{\smallbreak}\pstart{}{\pb}Herrn D\textsuperscript{r} Arthur Schnitzler\pend{}\pstart{}Wien XVIII.\oindex{XVIII., Waehring@\textbf{XVIII., Währing}, \emph{A.ADM3}|pw}\pend{}\pstart{}Spöttelgaße 7\oindex{Edmund-Weiss-Gasse 7@\textbf{Edmund-Weiß-Gasse 7}, \emph{Wohngebäude (K.WHS)}|pw}.
               \pend{}{\bigskip}\vspace{1em}
\pstart
           \raggedleft{}{\pb}Mittwoch\pend
           \vspace{0.5em}
\pstart
           Lieber Freund, vielen Dank für Ihren Brief, über den ich mich sehr
               gefreut habe. Es geht ja oft wunderlich mit diesen kleinen Arbeiten: \label{K_L03394-1v}\edtext{diese\pwindex{Mattachich@\emph{Mattachich}|pwv}}{\lemma{\textnormal{\emph{diese}}}\Cendnote{\textnormal{Felix Salten\pwindex{Salten, Felix 06.09.1869 – 08.10.1945@\textsc{Salten, Felix} (06.09.1869 – 08.10.1945), \emph{Schriftsteller/Schriftstellerin, Journalist/Journalistin, Chefredakteur/Chefredakteurin}|pwk}: \emph{Mattachich}\pwindex{Mattachich@\emph{Mattachich}|pwk}. In: \emph{Die
                        Zeit}\pwindex{Zeit@\emph{Die Zeit}|pwk}, Jg. 3, Nr. 538, 27. 3. 1904,
                     Morgenblatt, S. 1–3.}}}\label{K_L03394-1} letzte mußte ich, schläfrig, müd und eilig,
               in drei Stunden fertigmachen, und wenn wirklich was dran zu loben ist, dann war es
               eben doch wol der »Schmiß« (kann – falls das Wort zu minder erscheint, etwa durch
               »Elan« ersetzt werden). Nicht wenig bin ich über \label{K_L03394-2v}\edtext{P. A.\pwindex{Altenberg, Peter 09.03.1859 – 08.01.1919@\textsc{Altenberg, Peter} (09.03.1859 – 08.01.1919), \emph{Schriftsteller/Schriftstellerin}|pw}}{\lemma{\textnormal{\emph{P. A.}}}\Cendnote{\textnormal{Peter Altenberg\pwindex{Altenberg, Peter 09.03.1859 – 08.01.1919@\textsc{Altenberg, Peter} (09.03.1859 – 08.01.1919), \emph{Schriftsteller/Schriftstellerin}|pwk}. Dieser lebte in finanziellen und gesundheitlich prekären Umständen, verursacht nicht zuletzt durch
                  übermäßigen Alkoholkonsum, vgl. A. S.: \emph{Tagebuch}, 7. 8. 1904.
               }}}\label{K_L03394-2} erschrocken. Habe gleich überall nach ihm gesucht, aber nichts gefunden. Wo
               denn? Dass ich manchmal in Satzmelodien falle, die mir lieb sind, weiß ich, und
               glaube, das hängt mit meiner musikalischer Empfänglichkeit zusammen. Aber A.\pwindex{Altenberg, Peter 09.03.1859 – 08.01.1919@\textsc{Altenberg, Peter} (09.03.1859 – 08.01.1919), \emph{Schriftsteller/Schriftstellerin}|pw}’s Sätze waren mir nie angenehm, haben nichts
               in mir dauernd berührt, und ich könnte es mir also nicht erklären.\pend
           
\pstart
           Otti\pwindex{Salten, Ottilie 07.03.1868 – 22.06.1942@\textsc{Salten, Ottilie} (07.03.1868 – 22.06.1942), \emph{Schauspieler/Schauspielerin}|pw}, Paul\pwindex{Salten, Paul 11.08.1903 – 08.05.1937@\textsc{Salten, Paul} (11.08.1903 – 08.05.1937), \emph{Filmcutter/Filmcutterin}|pw} und ich wollen Samstag früh über Ostern
               auf den Kahlenberg\oindex{Kahlenberg@\textbf{Kahlenberg}, \emph{T.MT}|pw}. (\label{K_L03394-3v}\edtext{Privat-Semmering\oindex{Semmering@\textbf{Semmering}, \emph{A.ADM3}|pwv}}{\lemma{\textnormal{\emph{Privat-Semmering}}}\Cendnote{\textnormal{Siehe Felix Salten an Arthur Schnitzler, 27. 11. 1903.
               }}}\label{K_L03394-3}) Wenn es Ihnen recht ist, \label{K_L03394-4v}\edtext{kommen wir morgen Donnerstag oder übermorgen Freitag um ½ 7–7 zu Ihnen}{\lemma{\textnormal{\emph{kommen … Ihnen}}}\Cendnote{\textnormal{Siehe A. S.: \emph{Tagebuch}, 1. 4. 1904.
               }}}\label{K_L03394-4}. Ich schlage vor, dass wir dann im Riedhof\oindex{Riedhof@\textbf{Riedhof}, \emph{Lokal (K.LKL)}|pw}
               nachtmahlen.\pend
           
\pstart
           herzlichste Grüße an Olga\pwindex{Schnitzler, Olga 17.01.1882 – 13.01.1970@\textsc{Schnitzler, Olga} (17.01.1882 – 13.01.1970), \emph{Schauspieler/Schauspielerin, Sänger/Sängerin}|pw} u. Sie {\\[\baselineskip]}Ihr {\\[\baselineskip]}\spacefill\mbox{Salten}\pend
           \leftskip=0em{}\selectlanguage{ngerman}\endnumbering\briefempfaengerindex{Schnitzler, Arthur@\textsc{Schnitzler, Arthur}!zzzSalten, Felix@\emph{von Felix Salten}!1904-03-301@{30. 3. 1904}|)be}\mylabel{L03394h}  \normalsize

\doendnotes{C}
\bigskip
\vfill

\clearpage

\footnotesize

\lohead{\textsc{register}}

% Definiere theindex-Environment komplett neu ohne reledmac
\makeatletter
\renewenvironment{theindex}{%
  \section*{\indexname}%
  \setlength{\parindent}{0pt}%
  \setlength{\parskip}{0pt plus 0.3pt}%
  \let\item\@idxitem
}{%
  \clearpage
}
\makeatother

\IfFileExists{\jobname-pw.ind}{\input{\jobname-pw.ind}}{}

\end{document}

      