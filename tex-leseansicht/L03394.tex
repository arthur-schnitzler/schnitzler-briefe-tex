%% latex-leseansicht-vorspann.tex
%% Vorspann für die Leseansicht.
%% Lädt die gemeinsame Datei latex-vorspann.tex mit nicht gesetztem Schalter.

\newif\ifkorrekturansicht
\korrekturansichtfalse

\input{../tex-inputs/latex-vorspann}


         
         \renewcommand{\erwaehntePersonen}{Personen: Peter Altenberg, Felix Salten, Ottilie Salten, Paul Salten, Olga Schnitzler}
         \renewcommand{\erwaehnteOrte}{Orte: Edmund-Weiß-Gasse 7, I., Innere Stadt, Kahlenberg, Riedhof, Semmering, Wien, XVIII., Währing}
         \renewcommand{\erwaehnteWerke}{Werke: Die Zeit, Mattachich}
               \section[ Felix Salten an Arthur Schnitzler, 30. 3. 1904]{ Felix Salten an Arthur Schnitzler, 30. 3. 1904}\nopagebreak\mylabel{v}\rehead{ }\begin{ledgroupsized}[t]{13cm}\normalsize\beginnumbering\briefempfaengerindex{Schnitzler, Arthur@\textsc{Schnitzler, Arthur}!zzzSalten, Felix@\emph{von Felix Salten}!1904-03-301@{30. 3. 1904}|(be} \toendnotes[C]{\smallbreak\pagebreak[2]} \Standort{CUL, Schnitzler, B 89, B 1.}
\physDesc{Kartenbrief, 1067 Zeichen
\newline{}Handschrift: schwarze Tinte, lateinische Kurrent
\newline{}Versand: 1) Stempel: »\nobreak{}\oindex{I., Innere Stadt@\textbf{I., Innere Stadt}|pwk}Wien 1/1, 30 III 04, 2 30N\nobreak{}«.   2) Stempel: »\nobreak{}\oindex{XVIII., Waehring@\textbf{XVIII., Währing}|pwk}Wien 18/1 11\textcolor{gray}{1}, 30 \textcolor{gray}{III 04}, 3 10N\nobreak{}«. 
\newline{}Schnitzler: mit Bleistift datiert: »30. 3. 904.–« und Vermerk: »S{[}alten{]}« 
\newline{}Ordnung: mit Bleistift von unbekannter Hand nummeriert: »186« }\toendnotes[C]{\smallbreak}\pstart{}{\pb}Herrn D\textsuperscript{r} Arthur Schnitzler\pend{}\pstart{}Wien XVIII.\oindex{XVIII., Waehring@\textbf{XVIII., Währing}|pw}\pend{}\pstart{}Spöttelgaße 7\oindex{Edmund-Weiss-Gasse 7@\textbf{Edmund-Weiß-Gasse 7}|pw}.
               \pend{}{\bigskip}\pstart
           \raggedleft{}{\pb}Mittwoch\pend
           \pstart
           Lieber Freund, vielen Dank für Ihren Brief, über den ich mich sehr
               gefreut habe. Es geht ja oft wunderlich mit diesen kleinen Arbeiten: \label{K_L03394-1v}\edtext{diese\pwindex{Salten, Felix 06.09.1869 – 08.10.1945@\textsc{Salten, Felix} (06.09.1869 – 08.10.1945), \emph{Schriftsteller, Journalist}!Mattachich1904-03-27@\strich\emph{Mattachich} {[}1904-03-27{]}|pwv}}{\lemma{\textnormal{\emph{diese}}}\Cendnote{\textnormal{Felix Salten\pwindex{Salten, Felix 06.09.1869 – 08.10.1945@\textsc{Salten, Felix} (06.09.1869 – 08.10.1945), \emph{Schriftsteller, Journalist}|pwk}: \emph{Mattachich}\pwindex{Salten, Felix 06.09.1869 – 08.10.1945@\textsc{Salten, Felix} (06.09.1869 – 08.10.1945), \emph{Schriftsteller, Journalist}!Mattachich1904-03-27@\strich\emph{Mattachich} {[}1904-03-27{]}|pwk}. In: \emph{Die
                        Zeit}\pwindex{Zeit1902-09-27 – 1919@\emph{Die Zeit} {[}1902-09-27 – 1919{]}|pwk}, Jg. 3, Nr. 538, 27. 3. 1904,
                     Morgenblatt, S. 1–3.}}}\label{K_L03394-1h} letzte mußte ich, schläfrig, müd und eilig,
               in drei Stunden fertigmachen, und wenn wirklich was dran zu loben ist, dann war es
               eben doch wol der »Schmiß« (kann – falls das Wort zu minder erscheint, etwa durch
               »Elan« ersetzt werden). Nicht wenig bin ich über \label{K_L03394-2v}\edtext{P. A.\pwindex{Altenberg, Peter 09.03.1859 – 08.01.1919@\textsc{Altenberg, Peter} (09.03.1859 – 08.01.1919), \emph{Schriftsteller}|pw}}{\lemma{\textnormal{\emph{P. A.}}}\Cendnote{\textnormal{Peter Altenberg\pwindex{Altenberg, Peter 09.03.1859 – 08.01.1919@\textsc{Altenberg, Peter} (09.03.1859 – 08.01.1919), \emph{Schriftsteller}|pwk}. Dieser lebte finanziellen und gesundheitlich prekären Umständen, verursacht nicht zuletzt durch
                  übermäßigen Alkoholkonsum, vgl. A. S.: \emph{Tagebuch}, 7. 8. 1904.
               }}}\label{K_L03394-2h} erschrocken. Habe gleich überall nach ihm gesucht, aber nichts gefunden. Wo
               denn? Dass ich manchmal in Satzmelodien falle, die mir lieb sind, weiß ich, und
               glaube, das hängt mit meiner musikalischer Empfänglichkeit zusammen. Aber A.\pwindex{Altenberg, Peter 09.03.1859 – 08.01.1919@\textsc{Altenberg, Peter} (09.03.1859 – 08.01.1919), \emph{Schriftsteller}|pw}’s Sätze waren mir nie angenehm, haben nichts
               in mir dauernd berührt, und ich könnte es mir also nicht erklären.\pend
           \pstart
           Otti\pwindex{Salten, Ottilie 07.03.1868 – 22.06.1942@\textsc{Salten, Ottilie} (07.03.1868 – 22.06.1942), \emph{Schauspielerin}|pw}, Paul\pwindex{Salten, Paul 11.08.1903 – 08.05.1937@\textsc{Salten, Paul} (11.08.1903 – 08.05.1937), \emph{Filmcutter}|pw} und ich wollen Samstag früh über Ostern
               auf den Kahlenberg\oindex{Kahlenberg@\textbf{Kahlenberg}|pw}. (\label{K_L03394-3v}\edtext{Privat-Semmering\oindex{Semmering@\textbf{Semmering}|pwv}}{\lemma{\textnormal{\emph{Privat-Semmering}}}\Cendnote{\textnormal{siehe Felix Salten an Arthur Schnitzler, 27. 11. 1903}}}\label{K_L03394-3h}) Wenn es Ihnen recht ist, \label{K_L03394-4v}\edtext{kommen wir morgen Donnerstag oder übermorgen Freitag um ½ 7–7 zu Ihnen}{\lemma{\textnormal{\emph{kommen … Ihnen}}}\Cendnote{\textnormal{siehe A. S.: \emph{Tagebuch}, 1. 4. 1904}}}\label{K_L03394-4h}. Ich schlage vor, dass wir dann im Riedhof\oindex{Riedhof@\textbf{Riedhof}|pw}
               nachtmahlen.\pend
           \pstart
           herzlichste Grüße an Olga\pwindex{Schnitzler, Olga 17.01.1882 – 13.01.1970@\textsc{Schnitzler, Olga} (17.01.1882 – 13.01.1970), \emph{Schauspielerin, Sängerin}|pw} u. Sie {\\[\baselineskip]}Ihr {\\[\baselineskip]}\spacefill\mbox{Salten}\pend
           \leftskip=0em{}
         
         \endnumbering\mylabel{h}\end{ledgroupsized}  \newcommand{\dateiname}{L03394}\newcommand{\titel}{Felix Salten an Arthur Schnitzler, 30. 3. 1904}\newcommand{\editorInnen}{Martin Anton Müller und Laura Untner}%% latex-leseansicht-abspann.tex
%% Abspann für die Leseansicht.
%% Der Schalter \ifkorrekturansicht ist bereits durch den Vorspann gesetzt.

%% latex-abspann.tex
%% Gemeinsamer Abspann für Korrekturansicht und Leseansicht.
%% Setzt den Schalter \ifkorrekturansicht voraus (gesetzt in den
%% einbindenden Dateien latex-korrekturansicht-abspann.tex bzw.
%% latex-leseansicht-abspann.tex).
%% ---------------------------------------------------------------

\normalsize

% Das esempio-Environment wird nur in der Leseansicht benötigt
\ifkorrekturansicht\else
\newenvironment{esempio}[3]%
{
    \vspace{1.5ex}
    \rlap{\underline{#1}}
    \par
    \setlength{\parindent}{0cm}
    \nopagebreak
    \leftskip=#2cm
    \rightskip=#3cm
}
{
    \par
}
\fi

\doendnotes{C}
\bigskip
\vfill

\clearpage

\footnotesize

\ifkorrekturansicht
  \lohead{\textsc{register}}
\fi

% theindex-Environment neu definieren ohne reledmac
\makeatletter
\renewenvironment{theindex}{%
  \ifkorrekturansicht
    \section*{\indexname}%
  \else
    \subsubsection*{Index der erwähnten Entitäten}%
  \fi
  \setlength{\parindent}{0pt}%
  \setlength{\parskip}{0pt plus 0.3pt}%
  \let\item\@idxitem
}{%
  \ifkorrekturansicht\clearpage\fi
}
\makeatother

\IfFileExists{\jobname-pw.ind}{\input{\jobname-pw.ind}}{}

% Quellenangabe nur in der Leseansicht
\ifkorrekturansicht\else
% Fallback-Definitionen, falls die .tex-Datei \titel etc. nicht gesetzt hat
\providecommand{\titel}{}
\providecommand{\editorInnen}{}
\providecommand{\dateiname}{\jobname}

\vspace{3cm}

\vfill

\footnotesize
\textsc{Quelle}: \titel. Herausgegeben von {\editorInnen}. In: \emph{Arthur Schnitzler: Briefwechsel mit Autorinnen und Autoren}.
 Digitale Edition, https://schnitzler-briefe.acdh.oeaw.ac.at/{\dateiname}.html (Stand \today)
\fi

\end{document}


      