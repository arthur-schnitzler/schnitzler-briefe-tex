%% latex-korrekturansicht-vorspann.tex
%% Vorspann für die Korrekturansicht.
%% Lädt die gemeinsame Datei latex-vorspann.tex mit gesetztem Schalter.

\newif\ifkorrekturansicht
\korrekturansichttrue

\input{../tex-inputs/latex-vorspann}


\section[Hugo von Hofmannsthal an Arthur Schnitzler, 11. 10. 1905]{L01560 Hugo von Hofmannsthal an Arthur Schnitzler, 11. 10. 1905}
\nopagebreak\mylabel{L01560v}
\rehead{ }\normalsize\beginnumbering\briefempfaengerindex{Schnitzler, Arthur@\textsc{Schnitzler, Arthur}!zzzHofmannsthal, Hugo von@\emph{von Hugo von Hofmannsthal}!1905-10-111@{11. 10. 1905}|(be}
\toendnotes[C]{\smallbreak\pagebreak[2]}\Standort{CUL, Schnitzler, B 43.}
\physDesc{Postkarte, 393 Zeichen
\newline{}Handschrift: schwarze Tinte, deutsche Kurrent
\newline{}Versand: 1) Stempel: »\nobreak{}\oindex{Rodaun@\textbf{Rodaun}, \emph{A.ADM4}|pwk}R{[}odaun{]}, 11. 10. {[}05{]}, 4\nobreak{}«.   2) Stempel: »\nobreak{}\oindex{XVIII., Waehring@\textbf{XVIII., Währing}, \emph{A.ADM3}|pwk}18/1 Wien, 11. X. 0{[}5{]}, Bestellt\nobreak{}«. 
\newline{}Schnitzler: mit Bleistift datiert: »1\textcolor{gray}{3}. 10 905« 
\newline{}Ordnung: 1) mit Bleistift von unbekannter Hand nummeriert:
                                    »254«  2) mit Bleistift von unbekannter Hand nummeriert:
                                    »258c«}
\buchAbdrucke{\weitereDrucke{Hugo von Hofmannsthal, Arthur Schnitzler: \emph{Briefwechsel}. Frankfurt am Main: \emph{S. Fischer} 1964, S. 217.} }\toendnotes[C]{\smallbreak}\pstart{}{\pb}\textsc{Herrn D\textsuperscript{r} Arthur Schnitzler}\pend{}\pstart{}\textsc{Wien}\oindex{Wien@\textbf{Wien}, \emph{A.ADM2}|pw}\pend{}\pstart{}\textsc{XVIII. Spöttelgasse 7}.\oindex{Edmund-Weiss-Gasse 7@\textbf{Edmund-Weiß-Gasse 7}, \emph{Wohngebäude (K.WHS)}|pw}\pend{}{\bigskip}\vspace{1em}
\pstart
           \noindent{}{\pb}Lieber, ich höre eben
               von Ida\pwindex{Gruenwald, Ida 28.06.1873 – Mai 1908@\textsc{Grünwald, Ida} (28.06.1873 – Mai 1908), \emph{Stenotypistin/Stenotypistin}|pw}, daß Sie nach der Première\pwindex{Zwischenspiel. Komoedie in drei Akten@\emph{Zwischenspiel. Komödie in drei Akten}|pwv} paar Tage weg wollen. Nun ich habe
               größte Luſt und Bedürfnis ebenfalls ab Freitag oder Samstag paar Tage \introOben{}weg\introOben{}zugehn. Se{\geminationm}ering\oindex{Semmering@\textbf{Semmering}, \emph{A.ADM3}|pw} oder ſonſt, jedenfalls nicht weit aber
               gute stärkende Luft. Wie ſchön wäre es endlich wieder zuſa{\geminationm}en zu ſein! Schreiben Sie mir gleich hoffentlich \strikeout{ſt} gehts zuſa{\geminationm}en.\pend
           \pstart \spacefill\mbox{Hugo.}\pend{}\selectlanguage{ngerman}\endnumbering\briefempfaengerindex{Schnitzler, Arthur@\textsc{Schnitzler, Arthur}!zzzHofmannsthal, Hugo von@\emph{von Hugo von Hofmannsthal}!1905-10-111@{11. 10. 1905}|)be}\mylabel{L01560h}  \normalsize

\doendnotes{C}
\bigskip
\vfill

\clearpage

\footnotesize

\lohead{\textsc{register}}

% Definiere theindex-Environment komplett neu ohne reledmac
\makeatletter
\renewenvironment{theindex}{%
  \section*{\indexname}%
  \setlength{\parindent}{0pt}%
  \setlength{\parskip}{0pt plus 0.3pt}%
  \let\item\@idxitem
}{%
  \clearpage
}
\makeatother

\IfFileExists{\jobname-pw.ind}{\input{\jobname-pw.ind}}{}

\end{document}

      