%% latex-leseansicht-vorspann.tex
%% Vorspann für die Leseansicht.
%% Lädt die gemeinsame Datei latex-vorspann.tex mit nicht gesetztem Schalter.

\newif\ifkorrekturansicht
\korrekturansichtfalse

\input{../tex-inputs/latex-vorspann}


\section[Gerty und Hugo von Hofmannsthal an Arthur und Olga Schnitzler, 5. 5. [1916?]]{L02225 Gerty und Hugo von Hofmannsthal an Arthur und Olga Schnitzler, 5. 5. [1916?]}
\nopagebreak\mylabel{L02225v}
\rehead{ }\normalsize\beginnumbering\briefempfaengerindex{Schnitzler, Olga@\textsc{Schnitzler, Olga}!zzzHofmannsthal, Hugo von@\emph{von Hugo von Hofmannsthal}!1916-05-051@{5. 5. [1916?]}|(be}\briefempfaengerindex{Schnitzler, Olga@\textsc{Schnitzler, Olga}!zzzHofmannsthal, Gertrude von@\emph{von Gertrude von Hofmannsthal}!1916-05-051@{5. 5. [1916?]}|(be}\briefempfaengerindex{Schnitzler, Arthur@\textsc{Schnitzler, Arthur}!zzzHofmannsthal, Hugo von@\emph{von Hugo von Hofmannsthal}!1916-05-051@{5. 5. [1916?]}|(be}\briefempfaengerindex{Schnitzler, Arthur@\textsc{Schnitzler, Arthur}!zzzHofmannsthal, Gertrude von@\emph{von Gertrude von Hofmannsthal}!1916-05-051@{5. 5. [1916?]}|(be}
\toendnotes[C]{\smallbreak\pagebreak[2]}
\correspDesc{Versand  durch Gerty von Hofmannsthal, Hugo von Hofmannsthal am 5. 5. [1916?] in Dürnstein
\newline{}Erhalt  durch Arthur Schnitzler, Olga Schnitzler im Zeitraum [6. 5. 1916
                  – 10. 5. 1916?] in Wien}\toendnotes[C]{\smallbreak}
\Standort{CUL, Schnitzler, B 43.}
\physDesc{Bildpostkarte, 320 Zeichen
\newline{}Handschrift Hugo von Hofmannsthal: schwarze Tinte, lateinische Kurrent
\newline{}Handschrift Gertrude von Hofmannsthal: schwarze Tinte, lateinische Kurrent
\newline{}Versand: Stempel: »\nobreak{}\oindex{Dürnstein@\textbf{Dürnstein}|pwk}Dürnstein, 5. V. {[}1{]}\textcolor{gray}{6}\nobreak{}«.  
\newline{}Schnitzler: mit Bleistift die falsche Jahreszahl ergänzt: »19 
\newline{}Ordnung: 1) mit Bleistift von Frieda
                                    Pollak\pwindex{Pollak, Frieda 8.\,12.\,1881 Wien – 13.\,7.\,1937 ebd.@\textsc{Pollak, Frieda} (8.\,12.\,1881 Wien – 13.\,7.\,1937 ebd.), \emph{Sekretärin}|pw} (?) mit dem Buchstaben »A«
                                 (Abgeschrieben/Abschrift) gekennzeichnet  2) mit Bleistift von unbekannter Hand nummeriert: »\strikeout{289}« 3) mit Bleistift von unbekannter Hand nummeriert:
                                    »360«}
\buchAbdrucke{\weitereDrucke{Hugo von Hofmannsthal, Arthur Schnitzler: \emph{Briefwechsel}. Herausgegeben von Therese Nickl und Heinrich Schnitzler. Frankfurt am Main: \emph{S. Fischer} 1964, S. 283.} }\toendnotes[C]{\smallbreak}\pstart{}{\pb}Herrn u Frau\pend{}\pstart{}D\textsuperscript{r} Arthur Schnitzler\pend{}\pstart{}Wien\oindex{Wien@\textbf{Wien}, \emph{Verwaltungsgebiet}|pw}\pend{}\pstart{}XVIII Sternwartestr. 71\oindex{Wien@\textbf{Wien}!XVIII., Währing@\textbf{XVIII., Währing}!Sternwartestraße 71@\textbf{Sternwartestraße 71}, \emph{Wohngebäude}|pw}\pend{}{\bigskip}
\pstart
           \noindent{}\centering{}{\pb}\textcolor{gray}{\textbf{Partie a. d. Kirche in DÜRNSTEIN\oindex{Stift Dürnstein@\textbf{Stift Dürnstein}, \emph{Kirche}|pw}}}\pend
           \vspace{1em}
\pstart
           \raggedleft{}{\pb}\label{K_L02225-1v}\edtext{6. V.}{\lemma{\textnormal{\emph{6. V.}}}\Cendnote{\textnormal{Bei der Angabe des Tages unterlief
                     der Verfasserin ein Irrtum, wie aus dem Poststempel ersichtlich ist. Der
                     Poststempel lässt die Zuordnung zu einem bestimmten Jahr nur unsicher zu. Die
                     verwendete 5-Heller-Marke stellt sicher, dass die Karte vor Oktober 1916 versandt wurde, zu welchem Zeitpunkt eine Tarifreform in Kraft
                     trat. Andere infrage kommende Jahre lassen sich dadurch ausschließen, dass die
                     Verfasser sich nicht in Dürnstein\oindex{Dürnstein@\textbf{Dürnstein}|pwk}
                     befunden haben können.}}}\label{K_L02225-1}\pend
           \vspace{0.5em}
\pstart
           Viele herzliche Grüsse von einem kleinen Ausflug den wir bei dem herrlichen Wetter
               sehr geniessen!\pend
           
\pstart
           Hoffentlich auf baldiges Wiedersehen in Wien\oindex{Wien@\textbf{Wien}, \emph{Verwaltungsgebiet}|pw}.\pend
           \pstart Herzlichst \spacefill\mbox{Gerty}\pend{}\selectlanguage{ngerman}\vspace{1em}
\pstart
           \noindent{}{[}hs. Hofmannsthal:{]} Ich hatte nach meiner Rückkehr eine physisch sehr
               schlechte Zeit. Nun ists besser.\pend
           \pstart Auf bald.\hspace*{1.5em}Ihr \spacefill\mbox{Hugo.}\pend{}\selectlanguage{ngerman}\endnumbering\briefempfaengerindex{Schnitzler, Olga@\textsc{Schnitzler, Olga}!zzzHofmannsthal, Hugo von@\emph{von Hugo von Hofmannsthal}!1916-05-051@{5. 5. [1916?]}|)be}\briefempfaengerindex{Schnitzler, Olga@\textsc{Schnitzler, Olga}!zzzHofmannsthal, Gertrude von@\emph{von Gertrude von Hofmannsthal}!1916-05-051@{5. 5. [1916?]}|)be}\briefempfaengerindex{Schnitzler, Arthur@\textsc{Schnitzler, Arthur}!zzzHofmannsthal, Hugo von@\emph{von Hugo von Hofmannsthal}!1916-05-051@{5. 5. [1916?]}|)be}\briefempfaengerindex{Schnitzler, Arthur@\textsc{Schnitzler, Arthur}!zzzHofmannsthal, Gertrude von@\emph{von Gertrude von Hofmannsthal}!1916-05-051@{5. 5. [1916?]}|)be}\mylabel{L02225h}  \newcommand{\dateiname}{L02225}\newcommand{\titel}{Gerty und Hugo von Hofmannsthal an Arthur und Olga Schnitzler, 5. 5. [1916?]}\newcommand{\editorInnen}{Martin Anton Müller und Gerd-Hermann Susen}%% latex-leseansicht-abspann.tex
%% Abspann für die Leseansicht.
%% Der Schalter \ifkorrekturansicht ist bereits durch den Vorspann gesetzt.

%% latex-abspann.tex
%% Gemeinsamer Abspann für Korrekturansicht und Leseansicht.
%% Setzt den Schalter \ifkorrekturansicht voraus (gesetzt in den
%% einbindenden Dateien latex-korrekturansicht-abspann.tex bzw.
%% latex-leseansicht-abspann.tex).
%% ---------------------------------------------------------------

\normalsize

% Das esempio-Environment wird nur in der Leseansicht benötigt
\ifkorrekturansicht\else
\newenvironment{esempio}[3]%
{
    \vspace{1.5ex}
    \rlap{\underline{#1}}
    \par
    \setlength{\parindent}{0cm}
    \nopagebreak
    \leftskip=#2cm
    \rightskip=#3cm
}
{
    \par
}
\fi

\doendnotes{C}
\bigskip
\vfill

\clearpage

\footnotesize

\ifkorrekturansicht
  \lohead{\textsc{register}}
\fi

% theindex-Environment neu definieren ohne reledmac
\makeatletter
\renewenvironment{theindex}{%
  \ifkorrekturansicht
    \section*{\indexname}%
  \else
    \subsubsection*{Index der erwähnten Entitäten}%
  \fi
  \setlength{\parindent}{0pt}%
  \setlength{\parskip}{0pt plus 0.3pt}%
  \let\item\@idxitem
}{%
  \ifkorrekturansicht\clearpage\fi
}
\makeatother

\IfFileExists{\jobname-pw.ind}{\input{\jobname-pw.ind}}{}

% Quellenangabe nur in der Leseansicht
\ifkorrekturansicht\else
% Fallback-Definitionen, falls die .tex-Datei \titel etc. nicht gesetzt hat
\providecommand{\titel}{}
\providecommand{\editorInnen}{}
\providecommand{\dateiname}{\jobname}

\vspace{3cm}

\vfill

\footnotesize
\textsc{Quelle}: \titel. Herausgegeben von {\editorInnen}. In: \emph{Arthur Schnitzler: Briefwechsel mit Autorinnen und Autoren}.
 Digitale Edition, https://schnitzler-briefe.acdh.oeaw.ac.at/{\dateiname}.html (Stand \today)
\fi

\end{document}


