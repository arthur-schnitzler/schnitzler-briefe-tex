%% latex-korrekturansicht-vorspann.tex
%% Vorspann für die Korrekturansicht.
%% Lädt die gemeinsame Datei latex-vorspann.tex mit gesetztem Schalter.

\newif\ifkorrekturansicht
\korrekturansichttrue

\input{../tex-inputs/latex-vorspann}


\section[Gerty und Hugo von Hofmannsthal an Arthur und Olga Schnitzler, 5. 5. {[}1916?{]}]{L02225 Gerty und Hugo von Hofmannsthal an Arthur und Olga Schnitzler,
               5. 5. {[}1916?{]}}
\nopagebreak\mylabel{L02225v}
\rehead{ }\normalsize\beginnumbering\briefempfaengerindex{Schnitzler, Olga@\textsc{Schnitzler, Olga}!zzzHofmannsthal, Hugo von@\emph{von Hugo von Hofmannsthal}!1916-05-051@{5. 5. {[}1916?{]}}|(be}\briefempfaengerindex{Schnitzler, Olga@\textsc{Schnitzler, Olga}!zzzHofmannsthal, Gertrude von@\emph{von Gertrude von Hofmannsthal}!1916-05-051@{5. 5. {[}1916?{]}}|(be}\briefempfaengerindex{Schnitzler, Arthur@\textsc{Schnitzler, Arthur}!zzzHofmannsthal, Hugo von@\emph{von Hugo von Hofmannsthal}!1916-05-051@{5. 5. {[}1916?{]}}|(be}\briefempfaengerindex{Schnitzler, Arthur@\textsc{Schnitzler, Arthur}!zzzHofmannsthal, Gertrude von@\emph{von Gertrude von Hofmannsthal}!1916-05-051@{5. 5. {[}1916?{]}}|(be}
\toendnotes[C]{\smallbreak\pagebreak[2]}\Standort{CUL, Schnitzler, B 43.}
\physDesc{Bildpostkarte, 320 Zeichen
\newline{}Handschrift Hugo von Hofmannsthal: schwarze Tinte, lateinische Kurrent
\newline{}Handschrift Gertrude von Hofmannsthal: schwarze Tinte, lateinische Kurrent
\newline{}Versand: Stempel: »\nobreak{}\oindex{Duernstein@\textbf{Dürnstein}, \emph{Besiedelter Ort (A.BSO)}|pwk}Dürnstein, 5. V. {[}1{]}\textcolor{gray}{6}\nobreak{}«.  
\newline{}Schnitzler: mit Bleistift die falsche Jahreszahl ergänzt: »19 
\newline{}Ordnung: 1) mit Bleistift von Frieda
                                    Pollak\pwindex{Pollak, Frieda 08.12.1881 – 13.07.1937@\textsc{Pollak, Frieda} (08.12.1881 – 13.07.1937), \emph{Sekretär/Sekretärin}|pw} (?) mit dem Buchstaben »A«
                                 (Abgeschrieben/Abschrift) gekennzeichnet  2) mit Bleistift von unbekannter Hand nummeriert: »\strikeout{289}« 3) mit Bleistift von unbekannter Hand nummeriert:
                                    »360«}
\buchAbdrucke{\weitereDrucke{Hugo von Hofmannsthal, Arthur Schnitzler: \emph{Briefwechsel}. Frankfurt am Main: \emph{S. Fischer} 1964, S. 283.} }\toendnotes[C]{\smallbreak}\pstart{}{\pb}Herrn u Frau\pend{}\pstart{}D\textsuperscript{r} Arthur Schnitzler\pend{}\pstart{}Wien\oindex{Wien@\textbf{Wien}, \emph{A.ADM2}|pw}\pend{}\pstart{}XVIII Sternwartestr. 71\oindex{Sternwartestrasse 71@\textbf{Sternwartestraße 71}, \emph{Wohngebäude (K.WHS)}|pw}\pend{}{\bigskip}
\pstart
           \noindent{}\centering{}{\pb}\textcolor{gray}{\textbf{Partie a. d. Kirche in DÜRNSTEIN\oindex{Stift Duernstein@\textbf{Stift Dürnstein}, \emph{Kirche (K.KRC)}|pw}}}\pend
           \vspace{1em}
\pstart
           \raggedleft{}{\pb}\label{K_L02225-1v}\edtext{6. V.}{\lemma{\textnormal{\emph{6. V.}}}\Cendnote{\textnormal{Bei der Angabe des Tages unterlief
                     der Verfasserin ein Irrtum, wie aus dem Poststempel ersichtlich ist. Der
                     Poststempel lässt die Zuordnung zu einem bestimmten Jahr nur unsicher zu. Die
                     verwendete 5-Heller-Marke stellt sicher, dass die Karte vor Oktober
                        1916 versandt wurde, zu welchem Zeitpunkt eine Tarifreform in Kraft
                     trat. Andere infrage kommende Jahre lassen sich dadurch ausschließen, dass die
                     Verfasser sich nicht in Dürnstein\oindex{Duernstein@\textbf{Dürnstein}, \emph{Besiedelter Ort (A.BSO)}|pwk}
                     befunden haben können.}}}\label{K_L02225-1}\pend
           \vspace{0.5em}
\pstart
           Viele herzliche Grüsse von einem kleinen Ausflug den wir bei dem herrlichen Wetter
               sehr geniessen!\pend
           
\pstart
           Hoffentlich auf baldiges Wiedersehen in Wien\oindex{Wien@\textbf{Wien}, \emph{A.ADM2}|pw}.\pend
           \pstart Herzlichst \spacefill\mbox{Gerty}\pend{}\selectlanguage{ngerman}\vspace{1em}
\pstart
           \noindent{}{[}hs. :{]} Ich hatte nach meiner Rückkehr eine physisch sehr
               schlechte Zeit. Nun ists besser.\pend
           \pstart Auf bald.\hspace*{1.5em}Ihr \spacefill\mbox{Hugo.}\pend{}\selectlanguage{ngerman}\endnumbering\briefempfaengerindex{Schnitzler, Olga@\textsc{Schnitzler, Olga}!zzzHofmannsthal, Hugo von@\emph{von Hugo von Hofmannsthal}!1916-05-051@{5. 5. {[}1916?{]}}|)be}\briefempfaengerindex{Schnitzler, Olga@\textsc{Schnitzler, Olga}!zzzHofmannsthal, Gertrude von@\emph{von Gertrude von Hofmannsthal}!1916-05-051@{5. 5. {[}1916?{]}}|)be}\briefempfaengerindex{Schnitzler, Arthur@\textsc{Schnitzler, Arthur}!zzzHofmannsthal, Hugo von@\emph{von Hugo von Hofmannsthal}!1916-05-051@{5. 5. {[}1916?{]}}|)be}\briefempfaengerindex{Schnitzler, Arthur@\textsc{Schnitzler, Arthur}!zzzHofmannsthal, Gertrude von@\emph{von Gertrude von Hofmannsthal}!1916-05-051@{5. 5. {[}1916?{]}}|)be}\mylabel{L02225h}  \normalsize

\doendnotes{C}
\bigskip
\vfill

\clearpage

\footnotesize

\lohead{\textsc{register}}

% Definiere theindex-Environment komplett neu ohne reledmac
\makeatletter
\renewenvironment{theindex}{%
  \section*{\indexname}%
  \setlength{\parindent}{0pt}%
  \setlength{\parskip}{0pt plus 0.3pt}%
  \let\item\@idxitem
}{%
  \clearpage
}
\makeatother

\IfFileExists{\jobname-pw.ind}{\input{\jobname-pw.ind}}{}

\end{document}

      