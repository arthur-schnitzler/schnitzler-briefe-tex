%% latex-korrekturansicht-vorspann.tex
%% Vorspann für die Korrekturansicht.
%% Lädt die gemeinsame Datei latex-vorspann.tex mit gesetztem Schalter.

\newif\ifkorrekturansicht
\korrekturansichttrue

\input{../tex-inputs/latex-vorspann}


\section[ Paul Goldmann an Arthur Schnitzler, 20. 8. 1899]{L02884 Paul Goldmann an Arthur Schnitzler, 20. 8. 1899}
\nopagebreak\mylabel{L02884v}
\rehead{ }\normalsize\beginnumbering\briefempfaengerindex{Schnitzler, Arthur@\textsc{Schnitzler, Arthur}!zzzGoldmann, Paul@\emph{von Paul Goldmann}!1899-08-201@{20. 8. 1899}|(be}
\toendnotes[C]{\smallbreak\pagebreak[2]}\Standort{DLA, A:Schnitzler, HS.NZ85.1.3169.}
\physDesc{Bildpostkarte, 410 Zeichen
\newline{}Handschrift: 1) schwarze Tinte, deutsche Kurrent\hspace{1em}2) schwarze Tinte, lateinische Kurrent (\noindent{}Adresse)\hspace{1em}
\newline{}Versand: Stempel: »\nobreak{}\oindex{Le Mont-Saint-Michel@\textbf{Le Mont-Saint-Michel}, \emph{Besiedelter Ort (A.BSO)}|pwk}Le Mont-S\textcolor{gray}{\textsuperscript{t}-}Michel
                                       Manche, 20 AOÛT 99, \textcolor{gray}{2 \textsuperscript{e}}\nobreak{}«. Stempel: »\nobreak{}\oindex{Bad Ischl@\textbf{Bad Ischl}, \emph{P.PPL}|pwk}Ischl, 23. 8. 99, 6–7 V\nobreak{}«.  
\newline{}Schnitzler: mit Bleistift das Jahr »99« vermerkt }\toendnotes[C]{\smallbreak}\pstart{}{\pb}\begin{otherlanguage}{french}Autriche\oindex{Oesterreich@\textbf{Österreich}, \emph{A.PCLI}|pw}\end{otherlanguage}!\pend{}\pstart{}\begin{otherlanguage}{french}\textcolor{gray}{\textbf{M}}onsieur le Dr. Arthur Schnitzler \end{otherlanguage}\pend{}\pstart{}Pension Petter\oindex{Hotel und Pension Rudolfshoehe (Leopold Petter)@\textbf{Hotel und Pension Rudolfshöhe (Leopold Petter)}, \emph{Hotel (K.HTL)}|pw}\pend{}\pstart{}Ischl\oindex{Bad Ischl@\textbf{Bad Ischl}, \emph{P.PPL}|pw}\pend{}{\bigskip}
\pstart
           \noindent{}\centering{}{\pb}\textcolor{gray}{\textbf{Auberge de la Truie qui File\oindex{La Truie qui File@\textbf{La Truie qui File}, \emph{Gebäude (K.GBD)}|pw} XIII\textsuperscript{e} Siècle}}\pend
           
\pstart
           \centering{}\textcolor{gray}{\textbf{La Tour du Guet\oindex{La Tour du Guet@\textbf{La Tour du Guet}, \emph{Gebäude (K.GBD)}|pw}}}\pend
           
\pstart
           \centering{}\textcolor{gray}{\textbf{L’Hôtel du Soleil Royal\oindex{L Hôtel du Soleil Royal@\textbf{L’Hôtel du Soleil Royal}, \emph{Hotel (K.HTL)}|pw}.}}\pend
           
\pstart
           \centering{}\textcolor{gray}{\textbf{La Porte du Roi\oindex{La Porte du Roy@\textbf{La Porte du Roy}, \emph{Gebäude (K.GBD)}|pw}.}}\pend
           \vspace{1em}
\pstart
           \raggedleft{}{\pb}\textcolor{gray}{\textbf{Mont Saint Michel\oindex{Le Mont-Saint-Michel@\textbf{Le Mont-Saint-Michel}, \emph{Besiedelter Ort (A.BSO)}|pw}, le}}{ }20. \textsc{août}\pend
           \vspace{0.5em}
\pstart
           Ein Sonntagsausflug von \textsc{Rennes\oindex{Rennes@\textbf{Rennes}, \emph{P.PPLA}|pw}} nach dem \textsc{Mont St. Michel\oindex{Le Mont-Saint-Michel@\textbf{Le Mont-Saint-Michel}, \emph{Besiedelter Ort (A.BSO)}|pw}}. Einer der \strikeout{\textcolor{gray}{mer}k} merkwürdigſten Orte der Welt: Felſen, Meer und
               Gothik. In \textsc{Rennes\oindex{Rennes@\textbf{Rennes}, \emph{P.PPLA}|pw}} arbeiten wir uns todt, und noch iſt \label{K_L02884-1v}\edtext{kein Ende}{\lemma{\textnormal{\emph{kein Ende}}}\Cendnote{\textnormal{Bezug
                  auf den seit 8. 8. 1899 andauernden neuen
                  Kriegsgerichtsprozess in der Affäre Dreyfus\pwindex{Dreyfus, Alfred 1859-10-09 – 1935-07-12@\textsc{Dreyfus, Alfred} (1859-10-09 – 1935-07-12), \emph{Militär/Militärin}|pwk},
                  der sich bis in den September ausdehnte (Schuldspruch am
                     9. 9. 1899, Begnadigung am 19. 9. 1899).}}}\label{K_L02884-1} abzuſehen\textcolor{gray}{.}
               Bisher ſcheint nur Verurtheilung\pwindex{Dreyfus, Alfred 1859-10-09 – 1935-07-12@\textsc{Dreyfus, Alfred} (1859-10-09 – 1935-07-12), \emph{Militär/Militärin}|pwv} oder wenigſtens ſehr ſchäbiger Freiſpruch zu fürchten.
               Schreib’ mir bald u. grüße Alle, die nach mir fragen. Viele Grüße! \spacefill\mbox{P.
                  G.}\pend
           \selectlanguage{ngerman}\endnumbering\briefempfaengerindex{Schnitzler, Arthur@\textsc{Schnitzler, Arthur}!zzzGoldmann, Paul@\emph{von Paul Goldmann}!1899-08-201@{20. 8. 1899}|)be}\mylabel{L02884h}  \normalsize

\doendnotes{C}
\bigskip
\vfill

\clearpage

\footnotesize

\lohead{\textsc{register}}

% Definiere theindex-Environment komplett neu ohne reledmac
\makeatletter
\renewenvironment{theindex}{%
  \section*{\indexname}%
  \setlength{\parindent}{0pt}%
  \setlength{\parskip}{0pt plus 0.3pt}%
  \let\item\@idxitem
}{%
  \clearpage
}
\makeatother

\IfFileExists{\jobname-pw.ind}{\input{\jobname-pw.ind}}{}

\end{document}

      