%% latex-leseansicht-vorspann.tex
%% Vorspann für die Leseansicht.
%% Lädt die gemeinsame Datei latex-vorspann.tex mit nicht gesetztem Schalter.

\newif\ifkorrekturansicht
\korrekturansichtfalse

\input{../tex-inputs/latex-vorspann}

\begin{center}
            \textcolor{red}{ENTWURF, NICHT FERTIG KORRIGIERT}
                      \end{center}
            
         
         \renewcommand{\erwaehntePersonen}{Personen: Alfred Dreyfus}
         \renewcommand{\erwaehnteOrte}{Orte: Bad Ischl, Hotel und Pension Rudolfshöhe (Leopold Petter), La Porte du Roy, La Tour du Guet, La Truie qui File, Le Mont-Saint-Michel, L’Hôtel du Soleil Royal, Rennes, Österreich}
         \renewcommand{\erwaehnteWerke}{
               \section[ Paul Goldmann an Arthur Schnitzler, 20. 8. 1899]{ Paul Goldmann an Arthur Schnitzler, 20. 8. 1899}\nopagebreak\mylabel{v}\rehead{ }\begin{ledgroupsized}[t]{13cm}\normalsize\beginnumbering \toendnotes[C]{\smallbreak\pagebreak[2]} \Standort{DLA, A:Schnitzler, HS.NZ85.1.3169.}
\physDesc{Bildpostkarte
\newline{}Handschrift: 1) schwarze Tinte, deutsche Kurrent\hspace{1em}2) schwarze Tinte, lateinische Kurrent (\noindent{}Adresse)\hspace{1em}\newline{}Versand: Stempel: »\nobreak{}\oindex{Le Mont-Saint-Michel@\textbf{Le Mont-Saint-Michel}|pwk}Le Mont-S\textcolor{gray}{\textsuperscript{t}-}Michel Manche, 20 AOÛT 99\nobreak{}«. Stempel: »\nobreak{}\oindex{Bad Ischl@\textbf{Bad Ischl}|pwk}Ischl, 23. 8. 99, 6–7 V\nobreak{}«.  
\newline{}Schnitzler: mit Bleistift das Jahr »99« vermerkt }\toendnotes[C]{\smallbreak}\pstart{}{\pb}\begin{otherlanguage}{french}Autriche\oindex{Oesterreich@\textbf{Österreich}|pw}\end{otherlanguage}!\pend{}\pstart{}\begin{otherlanguage}{french}\textcolor{gray}{\textbf{M}}onsieur le Dr. Arthur Schnitzler
                  \end{otherlanguage}\pend{}\pstart{}Pension Petter\oindex{Hotel und Pension Rudolfshoehe (Leopold Petter)@\textbf{Hotel und Pension Rudolfshöhe (Leopold Petter)}|pw}\pend{}\pstart{}Ischl\oindex{Bad Ischl@\textbf{Bad Ischl}|pw}\pend{}{\bigskip}\pstart
           \noindent{}\centering{}{\pb}\textcolor{gray}{\textbf{Auberge de la Truie qui File\oindex{La Truie qui File@\textbf{La Truie qui File}|pw}
                   XIII\textsuperscript{e} Siècle}}\pend
           \pstart
           \noindent{}\centering{}\textcolor{gray}{\textbf{La Tour du Guet\oindex{La Tour du Guet@\textbf{La Tour du Guet}|pw}}}\pend
           \pstart
           \noindent{}\centering{}\textcolor{gray}{\textbf{L’Hôtel du Soleil Royal\oindex{L Hôtel du Soleil Royal@\textbf{L’Hôtel du Soleil Royal}|pw}.}}\pend
           \pstart
           \noindent{}\centering{}\textcolor{gray}{\textbf{La Porte du Roi\oindex{La Porte du Roy@\textbf{La Porte du Roy}|pw}.}}\pend
           \pstart
           \raggedleft{}\textcolor{gray}{\textbf{Mont Saint Michel\oindex{Le Mont-Saint-Michel@\textbf{Le Mont-Saint-Michel}|pw}, le}}{ }20. \textsc{août}\pend
           \pstart
           Ein Sonntagsausflug von \textsc{Rennes\oindex{Rennes@\textbf{Rennes}|pw}} nach dem \textsc{Mont St. Michel\oindex{Le Mont-Saint-Michel@\textbf{Le Mont-Saint-Michel}|pw}}. Einer der \strikeout{\textcolor{gray}{mer}k} merkwürdigſten Orte der Welt: Felſen, Meer und
               Gothik. In \textsc{Rennes\oindex{Rennes@\textbf{Rennes}|pw}} arbeiten wir uns todt, und noch iſt \label{K_L02884-1v}\edtext{kein Ende}{\lemma{\textnormal{\emph{kein Ende}}}\Cendnote{\textnormal{Bezug
                  auf den seit 8. 8. 1899 andauerden neuen
                  Kriegsgerichtsprozess in der Affäre Dreyfus\pwindex{Dreyfus, Alfred 1859-10-09 – 1935-07-12@\textsc{Dreyfus, Alfred} (1859-10-09 – 1935-07-12), \emph{Militär}|pwk},
                  der bis in den September fortdauerte
                  (Schuldspruch am 9. 9. 1899, Begnadigung am
                     19. 9. 1899)}}}\label{K_L02884-1h}
                  abzuſehen\textcolor{gray}{.} Bisher ſcheint nur Verurtheilung oder
                  wenigſtens ſehr ſchäbiger Freiſpruch zu fürchten. Schreib’ mir
               bald u. grüße Alle, die nach mir fragen. Viele Grüße! \spacefill\mbox{P. G.}\pend
           
         
         \endnumbering\mylabel{h}\end{ledgroupsized}  \newcommand{\dateiname}{L02884}\newcommand{\titel}{Paul Goldmann an Arthur Schnitzler, 20. 8. 1899}\newcommand{\editorInnen}{Martin Anton Müller und Laura Untner}%% latex-leseansicht-abspann.tex
%% Abspann für die Leseansicht.
%% Der Schalter \ifkorrekturansicht ist bereits durch den Vorspann gesetzt.

%% latex-abspann.tex
%% Gemeinsamer Abspann für Korrekturansicht und Leseansicht.
%% Setzt den Schalter \ifkorrekturansicht voraus (gesetzt in den
%% einbindenden Dateien latex-korrekturansicht-abspann.tex bzw.
%% latex-leseansicht-abspann.tex).
%% ---------------------------------------------------------------

\normalsize

% Das esempio-Environment wird nur in der Leseansicht benötigt
\ifkorrekturansicht\else
\newenvironment{esempio}[3]%
{
    \vspace{1.5ex}
    \rlap{\underline{#1}}
    \par
    \setlength{\parindent}{0cm}
    \nopagebreak
    \leftskip=#2cm
    \rightskip=#3cm
}
{
    \par
}
\fi

\doendnotes{C}
\bigskip
\vfill

\clearpage

\footnotesize

\ifkorrekturansicht
  \lohead{\textsc{register}}
\fi

% theindex-Environment neu definieren ohne reledmac
\makeatletter
\renewenvironment{theindex}{%
  \ifkorrekturansicht
    \section*{\indexname}%
  \else
    \subsubsection*{Index der erwähnten Entitäten}%
  \fi
  \setlength{\parindent}{0pt}%
  \setlength{\parskip}{0pt plus 0.3pt}%
  \let\item\@idxitem
}{%
  \ifkorrekturansicht\clearpage\fi
}
\makeatother

\IfFileExists{\jobname-pw.ind}{\input{\jobname-pw.ind}}{}

% Quellenangabe nur in der Leseansicht
\ifkorrekturansicht\else
% Fallback-Definitionen, falls die .tex-Datei \titel etc. nicht gesetzt hat
\providecommand{\titel}{}
\providecommand{\editorInnen}{}
\providecommand{\dateiname}{\jobname}

\vspace{3cm}

\vfill

\footnotesize
\textsc{Quelle}: \titel. Herausgegeben von {\editorInnen}. In: \emph{Arthur Schnitzler: Briefwechsel mit Autorinnen und Autoren}.
 Digitale Edition, https://schnitzler-briefe.acdh.oeaw.ac.at/{\dateiname}.html (Stand \today)
\fi

\end{document}


      