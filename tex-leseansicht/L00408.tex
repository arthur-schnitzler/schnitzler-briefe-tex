\input{../tex-inputs/latex-pdf-vorspann}
\begin{center}
            \textcolor{red}{ENTWURF. ENTZIFFERUNG NOCH NICHT KORREKTURGELESEN}
                      \end{center}
            
               \section[Arthur Schnitzler an Richard Beer-Hofmann, 28. 11. 1894]{ Arthur Schnitzler an Richard Beer-Hofmann, 28. 11. 1894}\nopagebreak\mylabel{v}\rehead{ }\begin{ledgroupsized}[t]{13cm}\normalsize\beginnumbering\briefempfaengerindex{Beer-Hofmann, Richard@\textsc{Beer-Hofmann, Richard}!zzzSchnitzler, Arthur@\emph{von Arthur Schnitzler}!1894-11-281@{28. 11. 1894}|(be} \toendnotes[C]{\smallbreak\pagebreak[2]} \Standort{YCGL, MSS 31.}
\physDesc{Postkarte
\newline{}Handschrift: Bleistift, deutsche Kurrent\newline{}Versand: 1) Rohrpost 2) Stempel: »\nobreak{}\oindex{IV., Wieden@\textbf{IV., Wieden}|pwk}Wien 4/1, 28 XI 94, 11 50 V\nobreak{}«. 3) Stempel: »\nobreak{}\oindex{I., Innere Stadt@\textbf{I., Innere Stadt}|pwk}Wien 1/1, 28 XI 94, 12 10 M\nobreak{}«. }\pstart{}{\pb}Herrn \textsc{Dr.}\pend{}\pstart{}\textsc{Richard Beer-Hofmann}\pend{}\pstart{}Wien\oindex{Wien@\textbf{Wien}|pw}\pend{}\pstart{}\textsc{I Wollzeile 15\oindex{Wollzeile@\textbf{Wollzeile}|pw}}\pend{}{\bigskip}\pstart
           \noindent{}{\pb}lieber Richard, waren nur 10.
                    Rei\textcolor{gray}{he} da – die hab ich nicht geno{\geminationm}en. –\pend
           \pstart
           Sind Sie vielleicht heut nach dem Souper im Kfh?\pend
           \pstart
           Herzl Gruſs{\\[\baselineskip]}\spacefill\mbox{Arth}\pend
           \leftskip=0em{}\endnumbering\briefempfaengerindex{Beer-Hofmann, Richard@\textsc{Beer-Hofmann, Richard}!zzzSchnitzler, Arthur@\emph{von Arthur Schnitzler}!1894-11-281@{28. 11. 1894}|)be}\mylabel{h}\end{ledgroupsized}  \newcommand{\dateiname}{L00408}\newcommand{\titel}{Arthur Schnitzler an Richard Beer-Hofmann, 28. 11. 1894}\newcommand{\editorInnen}{ Martin Anton Müller und Gerd-Hermann Susen}\input{../tex-inputs/latex-pdf-abspann}
      