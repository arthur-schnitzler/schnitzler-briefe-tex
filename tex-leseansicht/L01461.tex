%% latex-korrekturansicht-vorspann.tex
%% Vorspann für die Korrekturansicht.
%% Lädt die gemeinsame Datei latex-vorspann.tex mit gesetztem Schalter.

\newif\ifkorrekturansicht
\korrekturansichttrue

\input{../tex-inputs/latex-vorspann}


\section[Arthur Schnitzler an Richard Beer-Hofmann, 28. 10. 1904]{L01461 Arthur Schnitzler an Richard Beer-Hofmann, 28. 10. 1904}
\nopagebreak\mylabel{L01461v}
\rehead{ }\normalsize\beginnumbering\briefempfaengerindex{Beer-Hofmann, Richard@\textsc{Beer-Hofmann, Richard}!zzzSchnitzler, Arthur@\emph{von Arthur Schnitzler}!1904-10-281@{28. 10. 1904}|(be}
\toendnotes[C]{\smallbreak\pagebreak[2]}\Standort{YCGL, MSS 31.}
\physDesc{Kartenbrief, 323 Zeichen
\newline{}Handschrift: schwarze Tinte, deutsche Kurrent
\newline{}Versand: 1) Stempel: »\nobreak{}2\textcolor{gray}{8}. X. 04, XI\nobreak{}«.   2) Stempel: »\nobreak{}\oindex{Rodaun@\textbf{Rodaun}, \emph{A.ADM4}|pwk}Ro{[}dau{]}n, 28 {[}10.{]} 0\textcolor{gray}{4}\nobreak{}«. 
\newline{}Beer-Hofmann: mit blauem Buntstift beschriftet: »30/X beantw.« }\toendnotes[C]{\smallbreak}\pstart{}{\pb}Herrn \textsc{Dr. Richard Beer
                     Hofmann}\pend{}\pstart{}\textsc{Rodaun} bei Wien\oindex{Rodaun@\textbf{Rodaun}, \emph{A.ADM4}|pw}\pend{}\pstart{}\textsc{Liesinger Straße 2\oindex{Liesingerstrasse@\textbf{Liesingerstraße}, \emph{Straße (K.STR)}|pw}.}\pend{}{\bigskip}\vspace{1em}
\pstart
           \raggedleft{}{\pb}Wien\oindex{Wien@\textbf{Wien}, \emph{A.ADM2}|pw}, 28. X. 904\pend
           \vspace{0.5em}
\pstart
           lieber Richard, ſind Sie ſchon da? Könnte man nicht nächſte Woche,
               an einem der erſten \label{K_L01461-1v}\edtext{Abende}{\lemma{\textnormal{\emph{Abende}}}\Cendnote{\textnormal{Siehe A. S.: \emph{Tagebuch}, 31. 10. 1904.
               }}}\label{K_L01461-1} in Hietzing\oindex{XIII., Hietzing@\textbf{XIII., Hietzing}, \emph{A.ADM3}|pw} miteinander nachtmahlen?
               Schreiben Sie mir ein oder noch lieber mehrere Worte. Auch wann Sie wieder nach Berlin\oindex{Berlin@\textbf{Berlin}, \emph{P.PPLC}|pw} fahren u. ſ. w.\pend
           
\pstart
           Herzlichen Gruß{\\[\baselineskip]}Ihr{\\[\baselineskip]}\spacefill\mbox{A.}\pend
           \leftskip=0em{}\selectlanguage{ngerman}\endnumbering\briefempfaengerindex{Beer-Hofmann, Richard@\textsc{Beer-Hofmann, Richard}!zzzSchnitzler, Arthur@\emph{von Arthur Schnitzler}!1904-10-281@{28. 10. 1904}|)be}\mylabel{L01461h}  \normalsize

\doendnotes{C}
\bigskip
\vfill

\clearpage

\footnotesize

\lohead{\textsc{register}}

% Definiere theindex-Environment komplett neu ohne reledmac
\makeatletter
\renewenvironment{theindex}{%
  \section*{\indexname}%
  \setlength{\parindent}{0pt}%
  \setlength{\parskip}{0pt plus 0.3pt}%
  \let\item\@idxitem
}{%
  \clearpage
}
\makeatother

\IfFileExists{\jobname-pw.ind}{\input{\jobname-pw.ind}}{}

\end{document}

      