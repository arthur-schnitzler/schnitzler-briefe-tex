%% latex-korrekturansicht-vorspann.tex
%% Vorspann für die Korrekturansicht.
%% Lädt die gemeinsame Datei latex-vorspann.tex mit gesetztem Schalter.

\newif\ifkorrekturansicht
\korrekturansichttrue

\input{../tex-inputs/latex-vorspann}


\section[Arthur Schnitzler an Hermann Bahr, 1{[}4?{]}. 11. 1897]{L00740 Arthur Schnitzler an Hermann Bahr, 1{[}4?{]}. 11. 1897}
\nopagebreak\mylabel{L00740v}
\rehead{ }\normalsize\beginnumbering\briefempfaengerindex{Bahr, Hermann@\textsc{Bahr, Hermann}!zzzSchnitzler, Arthur@\emph{von Arthur Schnitzler}!1897-11-141@{1{[}4?{]}. 11. 1897}|(be}
\toendnotes[C]{\smallbreak\pagebreak[2]}\Standort{TMW, HS AM 23326 Ba.}
\physDesc{Brief, 2 Blätter, 5 Seiten, 1965 Zeichen
\newline{}Handschrift: schwarze Tinte, deutsche Kurrent
\newline{}Ordnung: 1) Lochung  2) von unbekannter Hand das erste Blatt mit Bleistift datiert: »18. 11. 97« und beide Blätter nummeriert mit: »I« bzw.
                                    »II«}
\buchAbdrucke{\weitereDrucke{1) Arthur Schnitzler: \emph{The Letters of Arthur Schnitzler to Hermann Bahr}. Chapel Hill: \emph{The University of North Carolina Press} 1978, S. 62–63.} \weitereDrucke{2) Arthur Schnitzler: \emph{Briefe 1875–1912}. Frankfurt am Main: \emph{S. Fischer} 1981, S. 343–344.} \weitereDrucke{3) Hermann Bahr, Arthur Schnitzler: \emph{Briefwechsel, Aufzeichnungen, Dokumente (1891–1931)}. Göttingen: \emph{Wallstein} 2018, S. 156–157.} }\toendnotes[C]{\smallbreak}
\pstart
           \noindent{}{\pb}Lieber Hermann, deine Anſicht betreffs dieſer weitgehenden Rechte
               des Regiſſeurs und des Vorleſers – nach Belieben zu ſtreichen u zu ändern! – theile
               ich durchaus nicht. In Hinſicht auf »Regiſſeur« und auf »ſtreichen« könnte man \introOben{}ja\introOben{} manches zugeben; beim Theater handelt es ſich nicht nur um
                  \uline{einen} Abend und das Mislingen des erſten ka{\geminationn} natürlich die ſchwerſten Folgen haben. Auch verſteht
                  {\pb}der Regiſſeur
               manchmal beſſer als der Autor, was des letztern Vortheil iſt. Der Vorleſer hat dieſe
               Entſchuldigungen nicht für ſich. Er hat einfach die Pflicht, die Dinge ſo zu leſen
               wie ſie geſchrieben ſind. Ich will ihm noch etwas zugeſtehn: findet er das
               betreffende Werk zu lang und iſt der Autor unerreichbar für ihn – z. B. dadurch daſs
               er geſtorben iſt oder irgend einen andern Ausflug in {\pb}beſondere Fernen
               gemacht hat, – ſo mag er kürzen. Ka{\geminationn} er aber den Autor
               finden, ſo überlaſſe er \uline{ihm} die Kürzungen oder lege
               ihm mindeſtens die ſeinigen (die des Vorleſers) vor. Aenderungen ſind \uline{abſolut} unſtatthaft, we{\geminationn}{ }ſie nicht vom Autor ſelbſt oder mit Zuſti{\geminationm}ung des Autors gemacht ſind, wobei noch zu bedenken
               iſt, dſs auch gewiſſe Streichungen in ihrem Effekt nur dem {\pb}Sinne nach als
                  Aenderun\damage{gen} zu gelten haben. Würdeſt du beiſpielſweise, um etwas naheliegendes zu
               citiren, den Schluſs von »Die Todten ſchweigen\pwindex{Toten schweigen@\emph{Die Toten schweigen}|pw}«
               ſtreichen, ſo würdest du auch aendern. – Wohin käme man \introOben{}alſo\introOben{}, we{\geminationn} deine Idee über die Souveränität des
               Vorleſers zu Recht beſtände!\pend
           
\pstart
           – In meiner Nov. die du vorleſen willſt, bitte ich dich \label{K_L00740-1v}\edtext{zwei \textsc{Lapsus’} zu corrigiren}{\lemma{\textnormal{\emph{zwei … corrigiren}}}\Cendnote{\textnormal{Beide Fehler sind in der Erstausgabe \emph{Die Frau des Weisen}\pwindex{Frau des Weisen. Novelletten@\emph{Die Frau des Weisen. Novelletten}|pwk} (1898) behoben.}}}\label{K_L00740-1}:
               Auf der vierten Seite, Zeile 22 iſt der Satz zu ſtreichen: »Die Scheiben klirren nur
               ſo ſtark, weil der Sturm –« (der Wagen ist nemlich \uline{offen}, hat keine {\pb}Scheiben, die aus einer
                  \label{K_L00740-2v}\edtext{früheren \substVorne{}\textsuperscript{f}\substDazwischen{}F\substHinten{}aſſung}{\lemma{\textnormal{\emph{früheren Faſſung}}}\Cendnote{\textnormal{Diese Fassung findet sich
                  in Arthur Schnitzler: \emph{Die Toten schweigen}\pwindex{Toten schweigen@\emph{Die Toten schweigen}|pwk}.
                     Historisch-kritische Ausgabe. Herausgegeben von  Martin Anton Müller, Mitarbeit von Ingo
                     Börner, Anna Lindner und Isabella Schwentner. Berlin, Boston: \emph{de
                        Gruyter}{ }2015 (Werke in historisch-kritischen Ausgaben, herausgegeben von Konstanze
                     Fliedl), H 24,5–6 und H 100,4.}}}\label{K_L00740-2}{ }ſtehen geblieben ſind.) Auf der 16. Seite,
               Zeile 14, ſteht einmal Wohnzi{\geminationm}erthür ſtatt
               »Wohnungsthür«. –\pend
           
\pstart
           – Daſs ich nicht dabei ſein kann, wenn Du die Geſchichte\pwindex{Toten schweigen@\emph{Die Toten schweigen}|pwv} lieſt, bedaure ich wirklich. Du wirſt ſie gewiſs
               zu ſtarker Wirkung bringen.\pend
           
\pstart
           Herzlichen Gruſs, dein{\\[\baselineskip]}\spacefill\mbox{ArthSch}\pend
           \leftskip=0em{}
\pstart
           Wien\oindex{Wien@\textbf{Wien}, \emph{A.ADM2}|pw}, \label{T_L00740-1v}\edtext{1\textcolor{gray}{4}. 11.}{\lemma{\textnormal{\emph{14. 11.}}}\Cendnote{\textnormal{Bislang wurde der Brief auf den
                        18. 11. 1897 datiert. Das diesbezügliche Zeichen setzt sich aus
                     einem geschwungenen Teil, bei dem die Tinte zerronnen ist, und einem leicht
                     schrägen Strich zusammen. Mehrere inhaltliche Gründe sprechen gegen die Lesart
                     »18«, vor allem die (nicht thematisierte) lange Dauer der Antwort, obwohl Schnitzler sich – ohne besondere
                     Vorkommnisse – in Wien\oindex{Wien@\textbf{Wien}, \emph{A.ADM2}|pwk} aufhielt, und dass Bahrs\pwindex{Bahr, Hermann 19.07.1863 – 15.01.1934@\textsc{Bahr, Hermann} (19.07.1863 – 15.01.1934), \emph{Schriftsteller/Schriftstellerin, Kritiker/Kritikerin}|pwk} Schreiben vom 16. 11. 1897 übergangen wird.}}}\label{T_L00740-1} 97
               \pend
           \selectlanguage{ngerman}\endnumbering\briefempfaengerindex{Bahr, Hermann@\textsc{Bahr, Hermann}!zzzSchnitzler, Arthur@\emph{von Arthur Schnitzler}!1897-11-141@{1{[}4?{]}. 11. 1897}|)be}\mylabel{L00740h}  \normalsize

\doendnotes{C}
\bigskip
\vfill

\clearpage

\footnotesize

\lohead{\textsc{register}}

% Definiere theindex-Environment komplett neu ohne reledmac
\makeatletter
\renewenvironment{theindex}{%
  \section*{\indexname}%
  \setlength{\parindent}{0pt}%
  \setlength{\parskip}{0pt plus 0.3pt}%
  \let\item\@idxitem
}{%
  \clearpage
}
\makeatother

\IfFileExists{\jobname-pw.ind}{\input{\jobname-pw.ind}}{}

\end{document}

      