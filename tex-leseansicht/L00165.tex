%% latex-korrekturansicht-vorspann.tex
%% Vorspann für die Korrekturansicht.
%% Lädt die gemeinsame Datei latex-vorspann.tex mit gesetztem Schalter.

\newif\ifkorrekturansicht
\korrekturansichttrue

\input{../tex-inputs/latex-vorspann}


\section[Hugo von Hofmannsthal an Arthur Schnitzler, {[}30.? 1. 1893{]}]{L00165 Hugo von Hofmannsthal an Arthur Schnitzler, {[}30.? 1. 1893{]}}
\nopagebreak\mylabel{L00165v}
\rehead{ }\normalsize\beginnumbering\briefempfaengerindex{Schnitzler, Arthur@\textsc{Schnitzler, Arthur}!zzzHofmannsthal, Hugo von@\emph{von Hugo von Hofmannsthal}!1893-01-302@{{[}30.? 1. 1893{]}}|(be}
\toendnotes[C]{\smallbreak\pagebreak[2]}\Standort{CUL, Schnitzler, B 43.}
\physDesc{Briefkarte, 852 Zeichen (aufgeprägtes Wappen )
\newline{}Handschrift: schwarze Tinte, deutsche Kurrent
\newline{}Schnitzler: mit Bleistift nummeriert: »37« }
\buchAbdrucke{\weitereDrucke{Hugo von Hofmannsthal, Arthur Schnitzler: \emph{Briefwechsel}. Frankfurt am Main: \emph{S. Fischer} 1964, S. 33–34.} }\toendnotes[C]{\smallbreak}
\pstart
           \raggedleft{}{\pb}\label{K_L00165-1v}\edtext{Montag}{\lemma{\textnormal{\emph{Montag}}}\Cendnote{\textnormal{Der 30. 1. 1893 war ein
                     Montag. Die Einordnung erfolgt anhand des Antwortschreibens, Arthur Schnitzler an Hugo von Hofmannsthal, [1. 2. 1893].}}}\label{K_L00165-1}.\pend
           
\pstart{}lieber Arthur.\pend\vspace{0.5em}
\pstart
           Die Empfehlung Engländers\pwindex{Altenberg, Peter 09.03.1859 – 08.01.1919@\textsc{Altenberg, Peter} (09.03.1859 – 08.01.1919), \emph{Schriftsteller/Schriftstellerin}|pw} ſehr gern beim
               nächſten Zuſammentreffen mit Berger\pwindex{Berger, Alfred von 30.04.1853 – 24.08.1912@\textsc{Berger, Alfred von} (30.04.1853 – 24.08.1912), \emph{Schriftsteller/Schriftstellerin, Journalist/Journalistin, Theaterleiter/Theaterleiterin}|pw}, was für
               eine Arbeit iſt es denn?\pend
           
\pstart
           Über Fels\pwindex{Fels, Friedrich Michael *~1864@\textsc{Fels, Friedrich Michael} (*~1864), \emph{Journalist/Journalistin}|pw} höre ich unbeſtimmt erſchreckendes;
               ich werde Ihnen in den nächſten Tagen etwas ſchicken, eventuell ein paar Freunde ohne
               Namennennung um Mithilfe bitten; ſagen Sie mir doch, was wahr iſt. »\uline{Familie}\pwindex{Familie@\emph{Familie}|pw}«?!!\pend
           
\pstart
           Ein herausgegriffenes Kapitel aus dem »Kind\pwindex{Age of Innocence@\emph{Age of Innocence}|pw}« hat
               mir einen ſtarken Eindruck gemacht; ich freue mich ſehr auf die Vollendung.\pend
           
\pstart
           Das Exemplar\pwindex{Anatol@\emph{Anatol}|pwv} für die akademiſche Vereinigung\orgindex{Wiener Akademische Vereinigung@Wiener Akademische Vereinigung|pw}{ }ſchicken Sie am tactvollſten in das Hôtel Wandel\oindex{Hotel Wandl@\textbf{Hotel Wandl}, \emph{Hotel (K.HTL)}|pw}{ }{\pb}mit der Weiſung, es am
                  Samstagabend dem Präſidenten\pwindex{?? [Praesident der Akademischen Vereinigung] *~1893@\textsc{?? [Präsident der Akademischen Vereinigung]} (*~1893)|pw}
               zu übergeben.\pend
           
\pstart
           Der kleine \textsc{Teltſch}\pwindex{Telcs, Ede 1872-05-12 – 1948@\textsc{Telcs, Ede} (1872-05-12 – 1948), \emph{Bildhauer/Bildhauerin}|pw} möchte auch gern eins haben. Vor einer Woche hat mir eine \label{K_L00165-2v}\edtext{Ruſſin\oindex{Russland@\textbf{Russland}, \emph{A.PCLI}|pw}\pwindex{?? [Russin] @\textsc{?? [Russin]}|pwv}}{\lemma{\textnormal{\emph{Ruſſin}}}\Cendnote{\textnormal{Vgl. »Sonntag 22.{ / }Die beiden Russinnen.« (Hofmannsthal\pwindex{Hofmannsthal, Hugo von 1874-02-01 – 1929-07-15@\textsc{Hofmannsthal, Hugo von} (1874-02-01 – 1929-07-15), \emph{Schriftsteller/Schriftstellerin}|pwk}: \emph{Aufzeichnungen}, S. 204).}}}\label{K_L00165-2}, meine \textsc{Souper}nachbarin, ſehr von den »\textsc{proverbes de ce
                  monsieur, qui est en même temps médecin}«, \strikeout{gerſch} geſchwärmt.\pend
           
\pstart
           Wann ſoll denn Salten\pwindex{Salten, Felix 06.09.1869 – 08.10.1945@\textsc{Salten, Felix} (06.09.1869 – 08.10.1945), \emph{Schriftsteller/Schriftstellerin, Journalist/Journalistin, Chefredakteur/Chefredakteurin}|pw} fortkommen?\pend
           
\pstart
           Herzlichſt{\\[\baselineskip]}\spacefill\mbox{Loris.}\pend
           \leftskip=0em{}\selectlanguage{ngerman}\endnumbering\briefempfaengerindex{Schnitzler, Arthur@\textsc{Schnitzler, Arthur}!zzzHofmannsthal, Hugo von@\emph{von Hugo von Hofmannsthal}!1893-01-302@{{[}30.? 1. 1893{]}}|)be}\mylabel{L00165h}  \normalsize

\doendnotes{C}
\bigskip
\vfill

\clearpage

\footnotesize

\lohead{\textsc{register}}

% Definiere theindex-Environment komplett neu ohne reledmac
\makeatletter
\renewenvironment{theindex}{%
  \section*{\indexname}%
  \setlength{\parindent}{0pt}%
  \setlength{\parskip}{0pt plus 0.3pt}%
  \let\item\@idxitem
}{%
  \clearpage
}
\makeatother

\IfFileExists{\jobname-pw.ind}{\input{\jobname-pw.ind}}{}

\end{document}

      