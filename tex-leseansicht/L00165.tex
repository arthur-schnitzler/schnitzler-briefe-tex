%% latex-leseansicht-vorspann.tex
%% Vorspann für die Leseansicht.
%% Lädt die gemeinsame Datei latex-vorspann.tex mit nicht gesetztem Schalter.

\newif\ifkorrekturansicht
\korrekturansichtfalse

\input{../tex-inputs/latex-vorspann}


\section[Hugo von Hofmannsthal an Arthur Schnitzler, {[}30.? 1. 1893{]}]{L00165 Hugo von Hofmannsthal an Arthur Schnitzler, {[}30.? 1. 1893{]}}
\nopagebreak\mylabel{L00165v}
\rehead{ }\normalsize\beginnumbering\briefempfaengerindex{Schnitzler, Arthur@\textsc{Schnitzler, Arthur}!zzzHofmannsthal, Hugo von@\emph{von Hugo von Hofmannsthal}!1893-01-302@{{[}30.? 1. 1893{]}}|(be}
\toendnotes[C]{\smallbreak\pagebreak[2]}
\correspDesc{Versand  durch Hugo von Hofmannsthal am [30.? 1. 1893] in Wien
\newline{}Erhalt  durch Arthur Schnitzler im Zeitraum [30. 1. 1893
                  – 3. 2. 1893?] in Wien}\toendnotes[C]{\smallbreak}
\Standort{CUL, Schnitzler, B 43.}
\physDesc{Briefkarte, 852 Zeichen (aufgeprägtes Wappen )
\newline{}Handschrift: schwarze Tinte, deutsche Kurrent
\newline{}Schnitzler: mit Bleistift nummeriert: »37« }
\buchAbdrucke{\weitereDrucke{Hugo von Hofmannsthal, Arthur Schnitzler: \emph{Briefwechsel}. Herausgegeben von Therese Nickl und Heinrich Schnitzler. Frankfurt am Main: \emph{S. Fischer} 1964, S. 33–34.} }\toendnotes[C]{\smallbreak}
\pstart
           \raggedleft{}{\pb}\label{K_L00165-1v}\edtext{Montag}{\lemma{\textnormal{\emph{Montag}}}\Cendnote{\textnormal{Der 30. 1. 1893 war ein
                     Montag. Die Einordnung erfolgt anhand des Antwortschreibens, XXXX Auszeichnungsfehler: Dokument L00170 nicht gefunden.}}}\label{K_L00165-1}.\pend
           
\pstart{}lieber Arthur.\pend\vspace{0.5em}
\pstart
           Die Empfehlung Engländers\pwindex{Altenberg, Peter 9.\,3.\,1859 Wien – 8.\,1.\,1919 ebd.@\textsc{Altenberg, Peter} (9.\,3.\,1859 Wien – 8.\,1.\,1919 ebd.), \emph{Schriftsteller}|pw}{ }ſehr gern beim
               nächſten Zuſammentreffen mit Berger\pwindex{Berger, Alfred von 30.\,4.\,1853 Wien – 24.\,8.\,1912 ebd.@\textsc{Berger, Alfred von} (30.\,4.\,1853 Wien – 24.\,8.\,1912 ebd.), \emph{Schriftsteller, Journalist, Theaterleiter}|pw}, was für
               eine Arbeit iſt es denn?\pend
           
\pstart
           Über Fels\pwindex{Fels, Friedrich Michael *~1864 Bad Dürkheim@\textsc{Fels, Friedrich Michael} (*~1864 Bad Dürkheim), \emph{Journalist}|pw} höre ich unbeſtimmt erſchreckendes;
               ich werde Ihnen in den nächſten Tagen etwas{ }ſchicken, eventuell ein paar Freunde ohne
               Namennennung um Mithilfe bitten;{ }ſagen Sie mir doch, was wahr iſt. »\uline{Familie}\pwindex{Schnitzler, Arthur 15.\,5.\,1862 Wien – 21.\,10.\,1931 ebd.@\textsc{Schnitzler, Arthur} (15.\,5.\,1862 Wien – 21.\,10.\,1931 ebd.), \emph{Schriftsteller, Mediziner}!Familie@\strich\emph{Familie}|pw}«?!!\pend
           
\pstart
           Ein herausgegriffenes Kapitel aus dem »Kind\pwindex{Hofmannsthal, Hugo von 1.\,2.\,1874 Wien – 15.\,7.\,1929 Rodaun@\textsc{Hofmannsthal, Hugo von} (1.\,2.\,1874 Wien – 15.\,7.\,1929 Rodaun), \emph{Schriftsteller}!Age of Innocence@\strich\emph{Age of Innocence}|pw}« hat
               mir einen{ }ſtarken Eindruck gemacht; ich freue mich{ }ſehr auf die Vollendung.\pend
           
\pstart
           Das Exemplar\pwindex{Schnitzler, Arthur 15.\,5.\,1862 Wien – 21.\,10.\,1931 ebd.@\textsc{Schnitzler, Arthur} (15.\,5.\,1862 Wien – 21.\,10.\,1931 ebd.), \emph{Schriftsteller, Mediziner}!Anatol@\strich\emph{Anatol}|pwv} für die akademiſche Vereinigung\orgindex{Wiener Akademische Vereinigung@Wiener Akademische Vereinigung|pw}{ }ſchicken Sie am tactvollſten in das Hôtel Wandel\oindex{Wien@\textbf{Wien}!I., Innere Stadt@\textbf{I., Innere Stadt}!Hotel Wandl@\textbf{Hotel Wandl}, \emph{Hotel}|pw}{ }{\pb}mit der Weiſung, es am
                  Samstagabend dem Präſidenten\pwindex{?? [Präsident der Akademischen Vereinigung] *~1893@\textsc{?? [Präsident der Akademischen Vereinigung]} (*~1893)|pw}
               zu übergeben.\pend
           
\pstart
           Der kleine \textsc{Teltſch}\pwindex{Telcs, Ede 12.\,5.\,1872 Baja – 1948 Budapest@\textsc{Telcs, Ede} (12.\,5.\,1872 Baja – 1948 Budapest), \emph{Bildhauer}|pw} möchte auch gern eins haben. Vor einer Woche hat mir eine \label{K_L00165-2v}\edtext{Ruſſin\oindex{Russland@\textbf{Russland}|pw}\pwindex{?? [Russin] @\textsc{?? [Russin]}|pwv}}{\lemma{\textnormal{\emph{Russin}}}\Cendnote{\textnormal{Vgl. »Sonntag 22.{ / }Die beiden Russinnen.« (Hofmannsthal\pwindex{Hofmannsthal, Hugo von 1.\,2.\,1874 Wien – 15.\,7.\,1929 Rodaun@\textsc{Hofmannsthal, Hugo von} (1.\,2.\,1874 Wien – 15.\,7.\,1929 Rodaun), \emph{Schriftsteller}|pwk}: \emph{Aufzeichnungen}, S. 204).}}}\label{K_L00165-2}, meine \textsc{Souper}nachbarin,{ }ſehr von den »\textsc{proverbes de ce
                  monsieur, qui est en même temps médecin}«, \strikeout{gerſch} geſchwärmt.\pend
           
\pstart
           Wann{ }ſoll denn Salten\pwindex{Salten, Felix 6.\,9.\,1869 Budapest – 8.\,10.\,1945 Zürich@\textsc{Salten, Felix} (6.\,9.\,1869 Budapest – 8.\,10.\,1945 Zürich), \emph{Schriftsteller, Journalist, Chefredakteur}|pw} fortkommen?\pend
           
\pstart
           Herzlichſt{\\[\baselineskip]}\spacefill\mbox{Loris.}\pend
           \leftskip=0em{}\selectlanguage{ngerman}\endnumbering\briefempfaengerindex{Schnitzler, Arthur@\textsc{Schnitzler, Arthur}!zzzHofmannsthal, Hugo von@\emph{von Hugo von Hofmannsthal}!1893-01-302@{{[}30.? 1. 1893{]}}|)be}\mylabel{L00165h}  \newcommand{\dateiname}{L00165}\newcommand{\titel}{Hugo von Hofmannsthal an Arthur Schnitzler, [30.? 1. 1893]}\newcommand{\editorInnen}{Martin Anton Müller und Gerd-Hermann Susen}%% latex-leseansicht-abspann.tex
%% Abspann für die Leseansicht.
%% Der Schalter \ifkorrekturansicht ist bereits durch den Vorspann gesetzt.

%% latex-abspann.tex
%% Gemeinsamer Abspann für Korrekturansicht und Leseansicht.
%% Setzt den Schalter \ifkorrekturansicht voraus (gesetzt in den
%% einbindenden Dateien latex-korrekturansicht-abspann.tex bzw.
%% latex-leseansicht-abspann.tex).
%% ---------------------------------------------------------------

\normalsize

% Das esempio-Environment wird nur in der Leseansicht benötigt
\ifkorrekturansicht\else
\newenvironment{esempio}[3]%
{
    \vspace{1.5ex}
    \rlap{\underline{#1}}
    \par
    \setlength{\parindent}{0cm}
    \nopagebreak
    \leftskip=#2cm
    \rightskip=#3cm
}
{
    \par
}
\fi

\doendnotes{C}
\bigskip
\vfill

\clearpage

\footnotesize

\ifkorrekturansicht
  \lohead{\textsc{register}}
\fi

% theindex-Environment neu definieren ohne reledmac
\makeatletter
\renewenvironment{theindex}{%
  \ifkorrekturansicht
    \section*{\indexname}%
  \else
    \subsubsection*{Index der erwähnten Entitäten}%
  \fi
  \setlength{\parindent}{0pt}%
  \setlength{\parskip}{0pt plus 0.3pt}%
  \let\item\@idxitem
}{%
  \ifkorrekturansicht\clearpage\fi
}
\makeatother

\IfFileExists{\jobname-pw.ind}{\input{\jobname-pw.ind}}{}

% Quellenangabe nur in der Leseansicht
\ifkorrekturansicht\else
% Fallback-Definitionen, falls die .tex-Datei \titel etc. nicht gesetzt hat
\providecommand{\titel}{}
\providecommand{\editorInnen}{}
\providecommand{\dateiname}{\jobname}

\vspace{3cm}

\vfill

\footnotesize
\textsc{Quelle}: \titel. Herausgegeben von {\editorInnen}. In: \emph{Arthur Schnitzler: Briefwechsel mit Autorinnen und Autoren}.
 Digitale Edition, https://schnitzler-briefe.acdh.oeaw.ac.at/{\dateiname}.html (Stand \today)
\fi

\end{document}


