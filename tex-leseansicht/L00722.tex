%% latex-leseansicht-vorspann.tex
%% Vorspann für die Leseansicht.
%% Lädt die gemeinsame Datei latex-vorspann.tex mit nicht gesetztem Schalter.

\newif\ifkorrekturansicht
\korrekturansichtfalse

\input{../tex-inputs/latex-vorspann}


         
         \renewcommand{\erwaehntePersonen}{Personen: Richard Beer-Hofmann, Rosa Freudenthal, Louise Schnitzler}
         \renewcommand{\erwaehnteOrte}{Orte: Frankgasse, Wien, Willergasse, Wollzeile}
         \renewcommand{\erwaehnteWerke}{Werke: Die Meistersinger von Nürnberg, Tagebuch}
               \section[Arthur Schnitzler an Richard Beer-Hofmann, {[}17. 9. 1897?{]}]{ Arthur Schnitzler an Richard Beer-Hofmann, {[}17. 9. 1897?{]}}\nopagebreak\mylabel{v}\rehead{ }\begin{ledgroupsized}[t]{13cm}\normalsize\beginnumbering \toendnotes[C]{\smallbreak\pagebreak[2]} \Standort{YCGL, MSS 31.}
\physDesc{Brief, 1 Blatt, 2 Seiten, Umschlag, 241 Zeichen (Briefpapier mit Trauerrand)
\newline{}Handschrift: Bleistift, deutsche Kurrent
\newline{}Versand: ohne postalischen Übermittlungsvermerk }\toendnotes[C]{\smallbreak}\pstart{}{\pb}\textsc{Dr. Arth Schnitzler IX
                        Frankg 1\oindex{Frankgasse@\textbf{Frankgasse}|pw}}.\pend{}{\bigskip}\pstart{}{\pb}\textsc{Herrn Dr. Rich. Beer-Hofmann}\pend{}\pstart{}Wien\oindex{Wien@\textbf{Wien}|pw}\pend{}\pstart{}\textsc{I. Wollzeile 15\oindex{Wollzeile@\textbf{Wollzeile}|pw}}\pend{}{\bigskip}\pstart{}{\pb}Lieber Richard,\pend\pstart
           wir ſind nur 3 in der Loge u meine \label{K_L00722_1v}\edtext{Mama\pwindex{Schnitzler, Louise 1840-07-08 – 1911-09-09@\textsc{Schnitzler, Louise} (1840-07-08 – 1911-09-09)|pwv}}{\lemma{\textnormal{\emph{Mama}}}\Cendnote{\textnormal{Das Korrespondenzstück ist undatiert.
                  Zeitlich setzen die Adressen Grenzen: Am 15. 11. 1893 zog Schnitzler\pwindex{Schnitzler, Arthur 15.05.1862 – 21.10.1931@\textsc{Schnitzler, Arthur} (15.05.1862 – 21.10.1931), \emph{Schriftsteller, Mediziner}|pwk} in die Frankgasse\oindex{Frankgasse@\textbf{Frankgasse}|pwk}, ab 1. 5. 1901 wohnte Beer-Hofmann\pwindex{Beer-Hofmann, Richard 1866-07-11 – 1945-09-26@\textsc{Beer-Hofmann, Richard} (1866-07-11 – 1945-09-26), \emph{Schriftsteller}|pwk} in der Willergasse\oindex{Willergasse@\textbf{Willergasse}|pwk}. Das \emph{Tagebuch}\pwindex{Schnitzler, Arthur 15.05.1862 – 21.10.1931@\textsc{Schnitzler, Arthur} (15.05.1862 – 21.10.1931), \emph{Schriftsteller, Mediziner}!Tagebuch1981 – 2000@\strich\emph{Tagebuch} {[}1981 – 2000{]}|pwk} erwähnt
                  nur einen Theaterbesuch mit Louise
                     Schnitzler\pwindex{Schnitzler, Louise 1840-07-08 – 1911-09-09@\textsc{Schnitzler, Louise} (1840-07-08 – 1911-09-09)|pwk}, die Aufführung von \emph{Die
                     Meistersinger von Nürnberg}\pwindex{\textcolor{red}{\textsuperscript{XXXX1 indx}}!Meistersinger von Nuernberg1868@\strich\emph{Die Meistersinger von Nürnberg} {[}1868{]}|pwk}, gemeinsam mit Rosa Freudenthal\pwindex{Freudenthal, Rosa 1862 – 18.06.1905@\textsc{Freudenthal, Rosa} (1862 – 18.06.1905)|pwk} am 17. 9. 1897.}}}\label{K_L00722_1h} lädt Sie »dringend«
                  {\pb}zu uns ein, alſo bitte ko{\geminationm}en Sie!\pend
           \pstart
           Herzlichſt Ihr{\\[\baselineskip]}\spacefill\mbox{Arthur}\pend
           \leftskip=0em{}\pstart
           \noindent{}\introOben{}2. Stock.\introOben{}\pend
           \pstart
           Nr 2, links\pend
           \pstart
           \label{T_L00722_1v}\edtext{\uline{\label{K_L00722-1v}\edtext{Dſtm}{\lemma{\textnormal{\emph{Dſtm}}}\Cendnote{\textnormal{Dienstmann}}}\label{K_L00722-1h}. iſt bezahlt.}}{\lemma{\textnormal{\emph{Dſtm. iſt bezahlt.}}}\Cendnote{\textnormal{auf dem Umschlag neben der
                     Adresse}}}\label{T_L00722_1h}\pend
           
         
         \endnumbering\mylabel{h}\end{ledgroupsized}  \newcommand{\dateiname}{L00722}\newcommand{\titel}{Arthur Schnitzler an Richard Beer-Hofmann, [17. 9. 1897?]}\newcommand{\editorInnen}{Martin Anton Müller und Gerd-Hermann Susen}%% latex-leseansicht-abspann.tex
%% Abspann für die Leseansicht.
%% Der Schalter \ifkorrekturansicht ist bereits durch den Vorspann gesetzt.

%% latex-abspann.tex
%% Gemeinsamer Abspann für Korrekturansicht und Leseansicht.
%% Setzt den Schalter \ifkorrekturansicht voraus (gesetzt in den
%% einbindenden Dateien latex-korrekturansicht-abspann.tex bzw.
%% latex-leseansicht-abspann.tex).
%% ---------------------------------------------------------------

\normalsize

% Das esempio-Environment wird nur in der Leseansicht benötigt
\ifkorrekturansicht\else
\newenvironment{esempio}[3]%
{
    \vspace{1.5ex}
    \rlap{\underline{#1}}
    \par
    \setlength{\parindent}{0cm}
    \nopagebreak
    \leftskip=#2cm
    \rightskip=#3cm
}
{
    \par
}
\fi

\doendnotes{C}
\bigskip
\vfill

\clearpage

\footnotesize

\ifkorrekturansicht
  \lohead{\textsc{register}}
\fi

% theindex-Environment neu definieren ohne reledmac
\makeatletter
\renewenvironment{theindex}{%
  \ifkorrekturansicht
    \section*{\indexname}%
  \else
    \subsubsection*{Index der erwähnten Entitäten}%
  \fi
  \setlength{\parindent}{0pt}%
  \setlength{\parskip}{0pt plus 0.3pt}%
  \let\item\@idxitem
}{%
  \ifkorrekturansicht\clearpage\fi
}
\makeatother

\IfFileExists{\jobname-pw.ind}{\input{\jobname-pw.ind}}{}

% Quellenangabe nur in der Leseansicht
\ifkorrekturansicht\else
% Fallback-Definitionen, falls die .tex-Datei \titel etc. nicht gesetzt hat
\providecommand{\titel}{}
\providecommand{\editorInnen}{}
\providecommand{\dateiname}{\jobname}

\vspace{3cm}

\vfill

\footnotesize
\textsc{Quelle}: \titel. Herausgegeben von {\editorInnen}. In: \emph{Arthur Schnitzler: Briefwechsel mit Autorinnen und Autoren}.
 Digitale Edition, https://schnitzler-briefe.acdh.oeaw.ac.at/{\dateiname}.html (Stand \today)
\fi

\end{document}


      