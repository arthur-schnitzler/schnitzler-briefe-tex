%% latex-korrekturansicht-vorspann.tex
%% Vorspann für die Korrekturansicht.
%% Lädt die gemeinsame Datei latex-vorspann.tex mit gesetztem Schalter.

\newif\ifkorrekturansicht
\korrekturansichttrue

\input{../tex-inputs/latex-vorspann}


\section[Arthur Schnitzler an Richard Beer-Hofmann, {[}17. 9. 1897?{]}]{L00722 Arthur Schnitzler an Richard Beer-Hofmann, {[}17. 9. 1897?{]}}
\nopagebreak\mylabel{L00722v}
\rehead{ }\normalsize\beginnumbering\briefempfaengerindex{Beer-Hofmann, Richard@\textsc{Beer-Hofmann, Richard}!zzzSchnitzler, Arthur@\emph{von Arthur Schnitzler}!1897-09-171@{{[}17. 9. 1897?{]}}|(be}
\toendnotes[C]{\smallbreak\pagebreak[2]}\Standort{YCGL, MSS 31.}
\physDesc{Brief, 1 Blatt, 2 Seiten, Umschlag, 240 Zeichen
\newline{}Handschrift: Bleistift, deutsche Kurrent
\newline{}Versand: ohne postalischen Übermittlungsvermerk }\toendnotes[C]{\smallbreak}\pstart{}{\pb}\textsc{Dr. Arth Schnitzler IX
                        Frankg 1\oindex{Frankgasse 1@\textbf{Frankgasse 1}, \emph{Wohngebäude (K.WHS)}|pw}}.\pend{}{\bigskip}\pstart{}{\pb}\textsc{Herrn Dr. Rich. Beer-Hofmann}\pend{}\pstart{}Wien\oindex{Wien@\textbf{Wien}, \emph{A.ADM2}|pw}\pend{}\pstart{}\textsc{I. Wollzeile 15\oindex{Wollzeile@\textbf{Wollzeile}, \emph{Straße (K.STR)}|pw}}\pend{}{\bigskip}\vspace{1em}
\pstart{}{\pb}Lieber Richard,\pend\vspace{0.5em}
\pstart
           wir ſind nur 3 in der Loge u meine \label{K_L00722-1v}\edtext{Mama\pwindex{Schnitzler, Louise 1840-07-08 – 1911-09-09@\textsc{Schnitzler, Louise} (1840-07-08 – 1911-09-09)|pwv}}{\lemma{\textnormal{\emph{Mama}}}\Cendnote{\textnormal{Das Korrespondenzstück ist undatiert.
                  Zeitlich setzen die Adressen Grenzen: Am 15. 11. 1893 zog Schnitzler in die Frankgasse\oindex{Frankgasse 1@\textbf{Frankgasse 1}, \emph{Wohngebäude (K.WHS)}|pwk}, ab 1. 5. 1901 wohnte Beer-Hofmann\pwindex{Beer-Hofmann, Richard 1866-07-11 – 1945-09-26@\textsc{Beer-Hofmann, Richard} (1866-07-11 – 1945-09-26), \emph{Schriftsteller/Schriftstellerin}|pwk} in der Willergasse\oindex{Willergasse@\textbf{Willergasse}, \emph{Straße (K.STR)}|pwk}. Das \emph{Tagebuch}\pwindex{Tagebuch@\emph{Tagebuch}|pwk} erwähnt
                  nur einen Theaterbesuch mit Louise
                     Schnitzler\pwindex{Schnitzler, Louise 1840-07-08 – 1911-09-09@\textsc{Schnitzler, Louise} (1840-07-08 – 1911-09-09)|pwk}, die Aufführung von \emph{Die
                     Meistersinger von Nürnberg}\pwindex{Meistersinger von Nuernberg@\emph{Die Meistersinger von Nürnberg}|pwk}, gemeinsam mit Rosa Freudenthal\pwindex{Freudenthal, Rosa 1862 – 18.06.1905@\textsc{Freudenthal, Rosa} (1862 – 18.06.1905)|pwk} am 17. 9. 1897.}}}\label{K_L00722-1} lädt Sie »dringend«
                  {\pb}zu uns ein, alſo bitte ko{\geminationm}en Sie!\pend
           
\pstart
           Herzlichſt Ihr{\\[\baselineskip]}\spacefill\mbox{Arthur}\pend
           \leftskip=0em{}
\pstart
           \noindent{}\introOben{}2. Stock.\introOben{}\pend
           
\pstart
           Nr 2, links\pend
           
\pstart
           {\pb}\label{T_L00722-1v}\edtext{\uline{\label{K_L00722-2v}\edtext{Dſtm.}{\lemma{\textnormal{\emph{Dſtm.}}}\Cendnote{\textnormal{Dienstmann}}}\label{K_L00722-2} iſt bezahlt.}}{\lemma{\textnormal{\emph{Dſtm. iſt bezahlt.}}}\Cendnote{\textnormal{auf dem Umschlag neben der
                     Adresse}}}\label{T_L00722-1}\pend
           \selectlanguage{ngerman}\endnumbering\briefempfaengerindex{Beer-Hofmann, Richard@\textsc{Beer-Hofmann, Richard}!zzzSchnitzler, Arthur@\emph{von Arthur Schnitzler}!1897-09-171@{{[}17. 9. 1897?{]}}|)be}\mylabel{L00722h}  \normalsize

\doendnotes{C}
\bigskip
\vfill

\clearpage

\footnotesize

\lohead{\textsc{register}}

% Definiere theindex-Environment komplett neu ohne reledmac
\makeatletter
\renewenvironment{theindex}{%
  \section*{\indexname}%
  \setlength{\parindent}{0pt}%
  \setlength{\parskip}{0pt plus 0.3pt}%
  \let\item\@idxitem
}{%
  \clearpage
}
\makeatother

\IfFileExists{\jobname-pw.ind}{\input{\jobname-pw.ind}}{}

\end{document}

      