%% latex-korrekturansicht-vorspann.tex
%% Vorspann für die Korrekturansicht.
%% Lädt die gemeinsame Datei latex-vorspann.tex mit gesetztem Schalter.

\newif\ifkorrekturansicht
\korrekturansichttrue

\input{../tex-inputs/latex-vorspann}


\section[Richard Beer-Hofmann an Arthur Schnitzler, 12. 11. 1904]{L01470 Richard Beer-Hofmann an Arthur Schnitzler, 12. 11. 1904}
\nopagebreak\mylabel{L01470v}
\rehead{ }\normalsize\beginnumbering\briefempfaengerindex{Schnitzler, Arthur@\textsc{Schnitzler, Arthur}!zzzBeer-Hofmann, Richard@\emph{von Richard Beer-Hofmann}!1904-11-121@{12. 11. 1904}|(be}
\toendnotes[C]{\smallbreak\pagebreak[2]}\Standort{CUL, Schnitzler, B 8.}
\physDesc{Brief, 1 Blatt, 2 Seiten, 771 Zeichen
\newline{}Handschrift: rote Tinte, lateinische Kurrent
\newline{}Ordnung: mit Bleistift von unbekannter Hand nummeriert:
                                    »196« }
\buchAbdrucke{\weitereDrucke{Arthur Schnitzler, Richard Beer-Hofmann: \emph{Briefwechsel 1891–1931}. Wien, Zürich: \emph{Europaverlag} 1992, S. 169–170.} }\toendnotes[C]{\smallbreak}
\pstart
           \raggedleft{}{\pb}Rodaun\oindex{Rodaun@\textbf{Rodaun}, \emph{A.ADM4}|pw}{ }12/XI 04\pend
           \vspace{0.5em}
\pstart
           Lieber Arthur! Nach einer Berlin\oindex{Berlin@\textbf{Berlin}, \emph{P.PPLC}|pw}er Zeitungsnotiz ist die Première von Rüderer\pwindex{Ruederer, Josef 15.10.1861 – 20.10.1915@\textsc{Ruederer, Josef} (15.10.1861 – 20.10.1915), \emph{Schriftsteller/Schriftstellerin}|pw}\pwindex{Morgenroethe. Komoedie aus dem Jahre 1848@\emph{Die Morgenröthe. Komödie aus dem Jahre 1848}|pwv} am 15 Nov. – dann ko{\geminationm}e ich daran. Reinhardt\pwindex{Reinhardt, Max 09.09.1873 – 30.10.1943@\textsc{Reinhardt, Max} (09.09.1873 – 30.10.1943), \emph{Theaterleiter/Theaterleiterin, Regisseur/Regisseurin, Schauspieler/Schauspielerin}|pw} grüssen Sie von mir und sagen Sie ihm
               daß ich ein Telegra{\geminationm} von ihm erwarte – es kann auch ein
               Brief sein – um abzureisen. Vielleicht auch die Nachricht ob ich »Bristol\oindex{Hotel Bristol Berlin@\textbf{Hotel Bristol Berlin}, \emph{Hotel (K.HTL)}|pw}« oder »Carleton\oindex{Carlton Hotel [Berlin]@\textbf{Carlton Hotel [Berlin]}, \emph{Hotel (K.HTL)}|pw}«
               (schreibt man das so?) wohnen soll. »Carleton\oindex{Carlton Hotel [Berlin]@\textbf{Carlton Hotel [Berlin]}, \emph{Hotel (K.HTL)}|pw}«
               soll ganz neu, sehr gut, u. noch näher v. Theater gelegen sein, u. Reinhardt\pwindex{Reinhardt, Max 09.09.1873 – 30.10.1943@\textsc{Reinhardt, Max} (09.09.1873 – 30.10.1943), \emph{Theaterleiter/Theaterleiterin, Regisseur/Regisseurin, Schauspieler/Schauspielerin}|pw} sagte er würde es dieser Tage mit
                  »Carleton\oindex{Carlton Hotel [Berlin]@\textbf{Carlton Hotel [Berlin]}, \emph{Hotel (K.HTL)}|pw}« versuchen. \uline{Moissi}\pwindex{Moissi, Alexander 02.04.1879 – 22.03.1935@\textsc{Moissi, Alexander} (02.04.1879 – 22.03.1935), \emph{Schauspieler/Schauspielerin}|pw} behandeln Sie möglichst streng, arbeiten Sie persönlich – mit ihm – was Sie
               Ihrem »Henri\pwindex{gruene Kakadu. Groteske in einem Akt@\emph{Der grüne Kakadu. Groteske in einem Akt}|pwv}« tun, tun Sie
               meinem »Philipp\pwindex{Graf von Charolais. Ein Trauerspiel@\emph{Der Graf von Charolais. Ein Trauerspiel}|pwv}«. Kerr\pwindex{Kerr, Alfred 25.12.1867 – 12.10.1948@\textsc{Kerr, Alfred} (25.12.1867 – 12.10.1948), \emph{Schriftsteller/Schriftstellerin, Kritiker/Kritikerin}|pw}, Bie\pwindex{Bie, Oskar 09.02.1864 – 21.04.1938@\textsc{Bie, Oskar} (09.02.1864 – 21.04.1938), \emph{Schriftsteller/Schriftstellerin, Journalist/Journalistin, Redakteur/Redakteurin}|pw}, Heimann\pwindex{Heimann, Moritz 19.07.1868 – 22.09.1925@\textsc{Heimann, Moritz} (19.07.1868 – 22.09.1925), \emph{Schriftsteller/Schriftstellerin, Verlagslektor/Verlagslektorin}|pw} – ausdrückliche Grüße –
               außerdem Grüsse {\pb}à discretion – zum
               verteilen. Und schreiben Sie – zwei Zeilen – 2 – aus Berlin\oindex{Berlin@\textbf{Berlin}, \emph{P.PPLC}|pw}.\pend
           
\pstart
           Herzlichst Ihr{\\[\baselineskip]}\spacefill\mbox{Richard}\pend
           \leftskip=0em{}\selectlanguage{ngerman}\endnumbering\briefempfaengerindex{Schnitzler, Arthur@\textsc{Schnitzler, Arthur}!zzzBeer-Hofmann, Richard@\emph{von Richard Beer-Hofmann}!1904-11-121@{12. 11. 1904}|)be}\mylabel{L01470h}  \normalsize

\doendnotes{C}
\bigskip
\vfill

\clearpage

\footnotesize

\lohead{\textsc{register}}

% Definiere theindex-Environment komplett neu ohne reledmac
\makeatletter
\renewenvironment{theindex}{%
  \section*{\indexname}%
  \setlength{\parindent}{0pt}%
  \setlength{\parskip}{0pt plus 0.3pt}%
  \let\item\@idxitem
}{%
  \clearpage
}
\makeatother

\IfFileExists{\jobname-pw.ind}{\input{\jobname-pw.ind}}{}

\end{document}

      