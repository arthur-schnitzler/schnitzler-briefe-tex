%% latex-leseansicht-vorspann.tex
%% Vorspann für die Leseansicht.
%% Lädt die gemeinsame Datei latex-vorspann.tex mit nicht gesetztem Schalter.

\newif\ifkorrekturansicht
\korrekturansichtfalse

\input{../tex-inputs/latex-vorspann}


\section[Richard Beer-Hofmann an Arthur Schnitzler, 12. 11. 1904]{L01470 Richard Beer-Hofmann an Arthur Schnitzler, 12. 11. 1904}
\nopagebreak\mylabel{L01470v}
\rehead{ }\normalsize\beginnumbering\briefempfaengerindex{Schnitzler, Arthur@\textsc{Schnitzler, Arthur}!zzzBeer-Hofmann, Richard@\emph{von Richard Beer-Hofmann}!1904-11-121@{12. 11. 1904}|(be}
\toendnotes[C]{\smallbreak\pagebreak[2]}
\correspDesc{Versand  durch Richard Beer-Hofmann am 12. 11. 1904 in Rodaun
\newline{}Erhalt  durch Arthur Schnitzler im Zeitraum [13. 11. 1904 – 17. 11. 1904?] in Wien}\toendnotes[C]{\smallbreak}
\Standort{CUL, Schnitzler, B 8.}
\physDesc{Brief, 1 Blatt, 2 Seiten, 771 Zeichen
\newline{}Handschrift: rote Tinte, lateinische Kurrent
\newline{}Ordnung: mit Bleistift von unbekannter Hand nummeriert:
                                    »196« }
\buchAbdrucke{\weitereDrucke{Arthur Schnitzler, Richard Beer-Hofmann: \emph{Briefwechsel 1891–1931}. Herausgegeben von Konstanze Fliedl. Wien, Zürich: \emph{Europaverlag} 1992, S. 169–170.} }\toendnotes[C]{\smallbreak}
\pstart
           \raggedleft{}{\pb}Rodaun\oindex{Wien@\textbf{Wien}!XXIII., Liesing@\textbf{XXIII., Liesing}!Rodaun@\textbf{Rodaun}, \emph{Region}|pw}{ }12/XI 04\pend
           \vspace{0.5em}
\pstart
           Lieber Arthur! Nach einer Berlin\oindex{Berlin@\textbf{Berlin}, \emph{Hauptstadt}|pw}er Zeitungsnotiz ist die Première von Rüderer\pwindex{Ruederer, Josef 15.\,10.\,1861 München – 20.\,10.\,1915 ebd.@\textsc{Ruederer, Josef} (15.\,10.\,1861 München – 20.\,10.\,1915 ebd.), \emph{Schriftsteller}|pw}\pwindex{Ruederer, Josef 15.\,10.\,1861 München – 20.\,10.\,1915 ebd.@\textsc{Ruederer, Josef} (15.\,10.\,1861 München – 20.\,10.\,1915 ebd.), \emph{Schriftsteller}!Morgenröthe. Komödie aus dem Jahre 1848@\strich\emph{Die Morgenröthe. Komödie aus dem Jahre 1848}|pwv} am 15 Nov. – dann ko{\geminationm}e ich daran. Reinhardt\pwindex{Reinhardt, Max 9.\,9.\,1873 Baden bei Wien – 30.\,10.\,1943 New York City@\textsc{Reinhardt, Max} (9.\,9.\,1873 Baden bei Wien – 30.\,10.\,1943 New York City), \emph{Theaterleiter, Regisseur, Schauspieler}|pw} grüssen Sie von mir und sagen Sie ihm
               daß ich ein Telegra{\geminationm} von ihm erwarte – es kann auch ein
               Brief sein – um abzureisen. Vielleicht auch die Nachricht ob ich »Bristol\oindex{Hotel Bristol Berlin@\textbf{Hotel Bristol Berlin}, \emph{Hotel}|pw}« oder »Carleton\oindex{Carlton Hotel [Berlin]@\textbf{Carlton Hotel [Berlin]}, \emph{Hotel}|pw}«
               (schreibt man das so?) wohnen soll. »Carleton\oindex{Carlton Hotel [Berlin]@\textbf{Carlton Hotel [Berlin]}, \emph{Hotel}|pw}«
               soll ganz neu, sehr gut, u. noch näher v. Theater gelegen sein, u. Reinhardt\pwindex{Reinhardt, Max 9.\,9.\,1873 Baden bei Wien – 30.\,10.\,1943 New York City@\textsc{Reinhardt, Max} (9.\,9.\,1873 Baden bei Wien – 30.\,10.\,1943 New York City), \emph{Theaterleiter, Regisseur, Schauspieler}|pw} sagte er würde es dieser Tage mit
                  »Carleton\oindex{Carlton Hotel [Berlin]@\textbf{Carlton Hotel [Berlin]}, \emph{Hotel}|pw}« versuchen. \uline{Moissi}\pwindex{Moissi, Alexander 2.\,4.\,1879 Triest – 22.\,3.\,1935 Wien@\textsc{Moissi, Alexander} (2.\,4.\,1879 Triest – 22.\,3.\,1935 Wien), \emph{Schauspieler}|pw} behandeln Sie möglichst streng, arbeiten Sie persönlich – mit ihm – was Sie
               Ihrem »Henri\pwindex{Schnitzler, Arthur 15.\,5.\,1862 Wien – 21.\,10.\,1931 ebd.@\textsc{Schnitzler, Arthur} (15.\,5.\,1862 Wien – 21.\,10.\,1931 ebd.), \emph{Schriftsteller, Mediziner}!grüne Kakadu. Groteske in einem Akt@\strich\emph{Der grüne Kakadu. Groteske in einem Akt}|pwv}« tun, tun Sie
               meinem »Philipp\pwindex{Beer-Hofmann, Richard 11.\,7.\,1866 Wien – 26.\,9.\,1945 New York City@\textsc{Beer-Hofmann, Richard} (11.\,7.\,1866 Wien – 26.\,9.\,1945 New York City), \emph{Schriftsteller}!Graf von Charolais. Ein Trauerspiel@\strich\emph{Der Graf von Charolais. Ein Trauerspiel}|pwv}«. Kerr\pwindex{Kerr, Alfred 25.\,12.\,1867 Breslau – 12.\,10.\,1948 Hamburg@\textsc{Kerr, Alfred} (25.\,12.\,1867 Breslau – 12.\,10.\,1948 Hamburg), \emph{Schriftsteller, Kritiker}|pw}, Bie\pwindex{Bie, Oskar 9.\,2.\,1864 Breslau – 21.\,4.\,1938 Berlin@\textsc{Bie, Oskar} (9.\,2.\,1864 Breslau – 21.\,4.\,1938 Berlin), \emph{Schriftsteller, Journalist, Redakteur}|pw}, Heimann\pwindex{Heimann, Moritz 19.\,7.\,1868 Werder – 22.\,9.\,1925 Berlin@\textsc{Heimann, Moritz} (19.\,7.\,1868 Werder – 22.\,9.\,1925 Berlin), \emph{Schriftsteller, Verlagslektor}|pw} – ausdrückliche Grüße –
               außerdem Grüsse {\pb}à discretion – zum
               verteilen. Und schreiben Sie – zwei Zeilen – 2 – aus Berlin\oindex{Berlin@\textbf{Berlin}, \emph{Hauptstadt}|pw}.\pend
           
\pstart
           Herzlichst Ihr{\\[\baselineskip]}\spacefill\mbox{Richard}\pend
           \leftskip=0em{}\selectlanguage{ngerman}\endnumbering\briefempfaengerindex{Schnitzler, Arthur@\textsc{Schnitzler, Arthur}!zzzBeer-Hofmann, Richard@\emph{von Richard Beer-Hofmann}!1904-11-121@{12. 11. 1904}|)be}\mylabel{L01470h}  \newcommand{\dateiname}{L01470}\newcommand{\titel}{Richard Beer-Hofmann an Arthur Schnitzler, 12. 11. 1904}\newcommand{\editorInnen}{Martin Anton Müller und Gerd-Hermann Susen}%% latex-leseansicht-abspann.tex
%% Abspann für die Leseansicht.
%% Der Schalter \ifkorrekturansicht ist bereits durch den Vorspann gesetzt.

%% latex-abspann.tex
%% Gemeinsamer Abspann für Korrekturansicht und Leseansicht.
%% Setzt den Schalter \ifkorrekturansicht voraus (gesetzt in den
%% einbindenden Dateien latex-korrekturansicht-abspann.tex bzw.
%% latex-leseansicht-abspann.tex).
%% ---------------------------------------------------------------

\normalsize

% Das esempio-Environment wird nur in der Leseansicht benötigt
\ifkorrekturansicht\else
\newenvironment{esempio}[3]%
{
    \vspace{1.5ex}
    \rlap{\underline{#1}}
    \par
    \setlength{\parindent}{0cm}
    \nopagebreak
    \leftskip=#2cm
    \rightskip=#3cm
}
{
    \par
}
\fi

\doendnotes{C}
\bigskip
\vfill

\clearpage

\footnotesize

\ifkorrekturansicht
  \lohead{\textsc{register}}
\fi

% theindex-Environment neu definieren ohne reledmac
\makeatletter
\renewenvironment{theindex}{%
  \ifkorrekturansicht
    \section*{\indexname}%
  \else
    \subsubsection*{Index der erwähnten Entitäten}%
  \fi
  \setlength{\parindent}{0pt}%
  \setlength{\parskip}{0pt plus 0.3pt}%
  \let\item\@idxitem
}{%
  \ifkorrekturansicht\clearpage\fi
}
\makeatother

\IfFileExists{\jobname-pw.ind}{\input{\jobname-pw.ind}}{}

% Quellenangabe nur in der Leseansicht
\ifkorrekturansicht\else
% Fallback-Definitionen, falls die .tex-Datei \titel etc. nicht gesetzt hat
\providecommand{\titel}{}
\providecommand{\editorInnen}{}
\providecommand{\dateiname}{\jobname}

\vspace{3cm}

\vfill

\footnotesize
\textsc{Quelle}: \titel. Herausgegeben von {\editorInnen}. In: \emph{Arthur Schnitzler: Briefwechsel mit Autorinnen und Autoren}.
 Digitale Edition, https://schnitzler-briefe.acdh.oeaw.ac.at/{\dateiname}.html (Stand \today)
\fi

\end{document}


