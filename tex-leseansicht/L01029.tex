%% latex-korrekturansicht-vorspann.tex
%% Vorspann für die Korrekturansicht.
%% Lädt die gemeinsame Datei latex-vorspann.tex mit gesetztem Schalter.

\newif\ifkorrekturansicht
\korrekturansichttrue

\input{../tex-inputs/latex-vorspann}


\section[Georg Brandes an Arthur Schnitzler, 3. 4. {[}1900{]}]{L01029 Georg Brandes an Arthur Schnitzler, 3. 4. {[}1900{]}}
\nopagebreak\mylabel{L01029v}
\rehead{ }\normalsize\beginnumbering\briefempfaengerindex{Schnitzler, Arthur@\textsc{Schnitzler, Arthur}!zzzBrandes, Georg@\emph{von Georg Brandes}!1900-04-034@{3. 4. {[}1900{]}}|(be}
\toendnotes[C]{\smallbreak\pagebreak[2]}\Standort{CUL, Schnitzler, B 17.}
\physDesc{Brief, 1 Blatt, 1 Seite, 579 Zeichen
\newline{}Handschrift: schwarze Tinte, lateinische Kurrent
\newline{}Ordnung: von Schnitzler mit Bleistift die Jahreszahl hinzugefügt: »900«, von unbekannter Hand mit Bleistift nummeriert:
                                    »19« }
\buchAbdrucke{\weitereDrucke{Georg Brandes, Arthur Schnitzler: \emph{Ein Briefwechsel}. Bern: \emph{Francke} 1956, S. 80.} }\toendnotes[C]{\smallbreak}
\pstart
           \raggedleft{}{\pb}Budapest Hotel Royal\oindex{Hotel Royal@\textbf{Hotel Royal}, \emph{Hotel (K.HTL)}|pw}{\\}3 April\pend
           \vspace{0.5em}
\pstart
           Liebster Freund Schnitzler\hspace*{2.5em}So gern ich Sie auf der Reise treffen möchte, es
               wird mir nicht möglich. Ich habe nach längerem Sträuben die Einladung der hiesigen
               Minister (Handels-\pwindex{Hegedues, Sandór 22.04.1847 – 28.12.1906@\textsc{Hegedüs, Sandór} (22.04.1847 – 28.12.1906), \emph{Politiker/Politikerin}|pwv} und Ackerbau-Minister\pwindex{Darányi, Ignác 15.01.1849 – 27.04.1927@\textsc{Darányi, Ignác} (15.01.1849 – 27.04.1927), \emph{Politiker/Politikerin}|pwv}) angenommen,
               Donnerstag bis Sonntag auf Staatskosten Ungarn\oindex{Ungarn@\textbf{Ungarn}, \emph{A.PCLI}|pw}
               zu bereisen und mir die Provinzen zeigen zu lassen. Ob das amusant wird, weiss ich
               nicht, zweifle, aber ich kann mir die Gelegenheit nicht entgehen \strikeout{z} lassen, etwas zu lernen, das sich mir sonst nicht
               darbietet.\pend
           
\pstart
           Wir sehen uns vielleicht noch auf meiner Rückreise durch Wien\oindex{Wien@\textbf{Wien}, \emph{A.ADM2}|pw}.\pend
           \pstart Ihr ergebener Freund \spacefill\mbox{Georg B}\pend{}\selectlanguage{ngerman}\endnumbering\briefempfaengerindex{Schnitzler, Arthur@\textsc{Schnitzler, Arthur}!zzzBrandes, Georg@\emph{von Georg Brandes}!1900-04-034@{3. 4. {[}1900{]}}|)be}\mylabel{L01029h}  \normalsize

\doendnotes{C}
\bigskip
\vfill

\clearpage

\footnotesize

\lohead{\textsc{register}}

% Definiere theindex-Environment komplett neu ohne reledmac
\makeatletter
\renewenvironment{theindex}{%
  \section*{\indexname}%
  \setlength{\parindent}{0pt}%
  \setlength{\parskip}{0pt plus 0.3pt}%
  \let\item\@idxitem
}{%
  \clearpage
}
\makeatother

\IfFileExists{\jobname-pw.ind}{\input{\jobname-pw.ind}}{}

\end{document}

      