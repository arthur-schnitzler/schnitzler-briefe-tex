%% latex-leseansicht-vorspann.tex
%% Vorspann für die Leseansicht.
%% Lädt die gemeinsame Datei latex-vorspann.tex mit nicht gesetztem Schalter.

\newif\ifkorrekturansicht
\korrekturansichtfalse

\input{../tex-inputs/latex-vorspann}


\section[Georg Brandes an Arthur Schnitzler, 3. 4. [1900]]{L01029 Georg Brandes an Arthur Schnitzler, 3. 4. [1900]}
\nopagebreak\mylabel{L01029v}
\rehead{ }\normalsize\beginnumbering\briefempfaengerindex{Schnitzler, Arthur@\textsc{Schnitzler, Arthur}!zzzBrandes, Georg@\emph{von Georg Brandes}!1900-04-034@{3. 4. [1900]}|(be}
\toendnotes[C]{\smallbreak\pagebreak[2]}
\correspDesc{Versand  durch Georg Brandes am 3. 4. [1900] in Budapest
\newline{}Erhalt  durch Arthur Schnitzler im Zeitraum [3. 4. 1900
                  – 7. 4. 1900?] \textbf{Ort fehlend} }\toendnotes[C]{\smallbreak}
\Standort{CUL, Schnitzler, B 17.}
\physDesc{Brief, 1 Blatt, 1 Seite, 579 Zeichen
\newline{}Handschrift: schwarze Tinte, lateinische Kurrent
\newline{}Ordnung: von Schnitzler mit Bleistift die Jahreszahl hinzugefügt: »900«, von unbekannter Hand mit Bleistift nummeriert:
                                    »19« }
\buchAbdrucke{\weitereDrucke{Georg Brandes, Arthur Schnitzler: \emph{Ein Briefwechsel}. Herausgegeben von Kurt Bergel. Bern: \emph{Francke} 1956, S. 80.} }\toendnotes[C]{\smallbreak}
\pstart
           \raggedleft{}{\pb}Budapest Hotel Royal\oindex{Hotel Royal@\textbf{Hotel Royal}, \emph{Hotel}|pw}{\\}3 April\pend
           \vspace{0.5em}
\pstart
           Liebster Freund Schnitzler\hspace*{2.5em}So gern ich Sie auf der Reise treffen möchte, es
               wird mir nicht möglich. Ich habe nach längerem Sträuben die Einladung der hiesigen
               Minister (Handels-\pwindex{Hegedüs, Sandór 22.\,4.\,1847 Cluj-Napoca – 28.\,12.\,1906 Budapest@\textsc{Hegedüs, Sandór} (22.\,4.\,1847 Cluj-Napoca – 28.\,12.\,1906 Budapest), \emph{Politiker}|pwv} und Ackerbau-Minister\pwindex{Darányi, Ignác 15.\,1.\,1849 Budapest – 27.\,4.\,1927 ebd.@\textsc{Darányi, Ignác} (15.\,1.\,1849 Budapest – 27.\,4.\,1927 ebd.), \emph{Politiker}|pwv}) angenommen,
               Donnerstag bis Sonntag auf Staatskosten Ungarn\oindex{Ungarn@\textbf{Ungarn}|pw}
               zu bereisen und mir die Provinzen zeigen zu lassen. Ob das amusant wird, weiss ich
               nicht, zweifle, aber ich kann mir die Gelegenheit nicht entgehen \strikeout{z} lassen, etwas zu lernen, das sich mir sonst nicht
               darbietet.\pend
           
\pstart
           Wir sehen uns vielleicht noch auf meiner Rückreise durch Wien\oindex{Wien@\textbf{Wien}, \emph{Verwaltungsgebiet}|pw}.\pend
           \pstart Ihr ergebener Freund \spacefill\mbox{Georg B}\pend{}\selectlanguage{ngerman}\endnumbering\briefempfaengerindex{Schnitzler, Arthur@\textsc{Schnitzler, Arthur}!zzzBrandes, Georg@\emph{von Georg Brandes}!1900-04-034@{3. 4. [1900]}|)be}\mylabel{L01029h}  \newcommand{\dateiname}{L01029}\newcommand{\titel}{Georg Brandes an Arthur Schnitzler, 3. 4. [1900]}\newcommand{\editorInnen}{Martin Anton Müller und Gerd-Hermann Susen}%% latex-leseansicht-abspann.tex
%% Abspann für die Leseansicht.
%% Der Schalter \ifkorrekturansicht ist bereits durch den Vorspann gesetzt.

%% latex-abspann.tex
%% Gemeinsamer Abspann für Korrekturansicht und Leseansicht.
%% Setzt den Schalter \ifkorrekturansicht voraus (gesetzt in den
%% einbindenden Dateien latex-korrekturansicht-abspann.tex bzw.
%% latex-leseansicht-abspann.tex).
%% ---------------------------------------------------------------

\normalsize

% Das esempio-Environment wird nur in der Leseansicht benötigt
\ifkorrekturansicht\else
\newenvironment{esempio}[3]%
{
    \vspace{1.5ex}
    \rlap{\underline{#1}}
    \par
    \setlength{\parindent}{0cm}
    \nopagebreak
    \leftskip=#2cm
    \rightskip=#3cm
}
{
    \par
}
\fi

\doendnotes{C}
\bigskip
\vfill

\clearpage

\footnotesize

\ifkorrekturansicht
  \lohead{\textsc{register}}
\fi

% theindex-Environment neu definieren ohne reledmac
\makeatletter
\renewenvironment{theindex}{%
  \ifkorrekturansicht
    \section*{\indexname}%
  \else
    \subsubsection*{Index der erwähnten Entitäten}%
  \fi
  \setlength{\parindent}{0pt}%
  \setlength{\parskip}{0pt plus 0.3pt}%
  \let\item\@idxitem
}{%
  \ifkorrekturansicht\clearpage\fi
}
\makeatother

\IfFileExists{\jobname-pw.ind}{\input{\jobname-pw.ind}}{}

% Quellenangabe nur in der Leseansicht
\ifkorrekturansicht\else
% Fallback-Definitionen, falls die .tex-Datei \titel etc. nicht gesetzt hat
\providecommand{\titel}{}
\providecommand{\editorInnen}{}
\providecommand{\dateiname}{\jobname}

\vspace{3cm}

\vfill

\footnotesize
\textsc{Quelle}: \titel. Herausgegeben von {\editorInnen}. In: \emph{Arthur Schnitzler: Briefwechsel mit Autorinnen und Autoren}.
 Digitale Edition, https://schnitzler-briefe.acdh.oeaw.ac.at/{\dateiname}.html (Stand \today)
\fi

\end{document}


