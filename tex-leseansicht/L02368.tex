%% latex-korrekturansicht-vorspann.tex
%% Vorspann für die Korrekturansicht.
%% Lädt die gemeinsame Datei latex-vorspann.tex mit gesetztem Schalter.

\newif\ifkorrekturansicht
\korrekturansichttrue

\input{../tex-inputs/latex-vorspann}


\section[Arthur Schnitzler an Robert Adam, 25. 5. 1921]{L02368 Arthur Schnitzler an Robert Adam, 25. 5. 1921}
\nopagebreak\mylabel{L02368v}
\rehead{ }\normalsize\beginnumbering\briefempfaengerindex{Adam, Robert@\textsc{Adam, Robert}!zzzSchnitzler, Arthur@\emph{von Arthur Schnitzler}!1921-05-251@{25. 5. 1921}|(be}
\toendnotes[C]{\smallbreak\pagebreak[2]}\Standort{DLA, 96.34.2/26.}
\physDesc{Postkarte, 377 Zeichen
\newline{}Handschrift: schwarze Tinte, deutsche Kurrent
\newline{}Versand: 1) Stempel: »\nobreak{}\oindex{I., Innere Stadt@\textbf{I., Innere Stadt}, \emph{A.ADM3}|pwk}1/1 Wien 8, 25. V. 21, 5\nobreak{}«.   2) zusätzlicher Stempel: »\noindent{}›HELFT ÖSTERREICHS\oindex{Oesterreich@\textbf{Österreich}, \emph{A.PCLI}|pw} KINDERN!‹{ / }AMERIK.
                                             KINDERHILFSAKTION\orgindex{Amerikanische Kinderhilfsaktion@Amerikanische Kinderhilfsaktion|pw}{ / }WIEN I.\oindex{I., Innere Stadt@\textbf{I., Innere Stadt}, \emph{A.ADM3}|pw}{ / }ELISABETHSTR. 9\oindex{Elisabethstrasse [Wien]@\textbf{Elisabethstraße [Wien]}, \emph{Straße (K.STR)}|pw}«, dieser auch in englischer Sprache gestempelt, aber nur
                                 bruchstückhaft entzifferbar 3) die falsche Bezirksangabe in der Empfängeradresse wurde von
                                 unbekannter Hand mit Bleistift zu »XII«
                                 korrigiert.}\pstart{}{\pb}A. S. Wien XVIII. \textsc{Sternwartestr} 71\oindex{Sternwartestrasse 71@\textbf{Sternwartestraße 71}, \emph{Wohngebäude (K.WHS)}|pw}\pend{}{\bigskip}\pstart{}Herrn Ob. Landesgerichtsrat\pend{}\pstart{}\textsc{Dr. Robert Adam Pollak}\pend{}\pstart{}\textsc{Wien XIII}\oindex{XIII., Hietzing@\textbf{XIII., Hietzing}, \emph{A.ADM3}|pw}\pend{}\pstart{}58 \textsc{Meidlinger Hauptstr}
                     58\oindex{Meidlinger Hauptstrasse@\textbf{Meidlinger Hauptstraße}, \emph{Straße (K.STR)}|pw}\pend{}{\bigskip}\vspace{1em}
\pstart
           \raggedleft{}{\pb}25. 5. 1921\pend
           
\pstart{}Verehrtester Herr Doktor,\pend\vspace{0.5em}
\pstart
           wenn Sie denn am Dienſtag (31. 5.) gegen Abend nach 6 für
               mich Zeit hätten, wäre mir Ihr lieber Beſuch ſehr willko{\geminationm}en. In der Zwiſchenzeit war ich auch verreiſt, München\oindex{Muenchen@\textbf{München}, \emph{P.PPLA}|pw} u. Salzburg\oindex{Salzburg@\textbf{Salzburg}, \emph{A.ADM2}|pw}.\pend
           
\pstart
           Herzlichſt grüßend Ihr{\\[\baselineskip]}ſehr ergebener{\\[\baselineskip]}\spacefill\mbox{Arthur Schnitzler}\pend
           \leftskip=0em{}\selectlanguage{ngerman}\endnumbering\briefempfaengerindex{Adam, Robert@\textsc{Adam, Robert}!zzzSchnitzler, Arthur@\emph{von Arthur Schnitzler}!1921-05-251@{25. 5. 1921}|)be}\mylabel{L02368h}  \normalsize

\doendnotes{C}
\bigskip
\vfill

\clearpage

\footnotesize

\lohead{\textsc{register}}

% Definiere theindex-Environment komplett neu ohne reledmac
\makeatletter
\renewenvironment{theindex}{%
  \section*{\indexname}%
  \setlength{\parindent}{0pt}%
  \setlength{\parskip}{0pt plus 0.3pt}%
  \let\item\@idxitem
}{%
  \clearpage
}
\makeatother

\IfFileExists{\jobname-pw.ind}{\input{\jobname-pw.ind}}{}

\end{document}

      