\input{../tex-inputs/latex-pdf-vorspann}
\begin{center}
            \textcolor{red}{ENTWURF. ENTZIFFERUNG NOCH NICHT KORREKTURGELESEN}
                      \end{center}
            
               \section[Arthur Schnitzler an Robert Adam, 25. 5. 1921]{ Arthur Schnitzler an Robert Adam, 25. 5. 1921}\nopagebreak\mylabel{v}\rehead{ }\begin{ledgroupsized}[t]{13cm}\normalsize\beginnumbering\briefempfaengerindex{Adam, Robert@\textsc{Adam, Robert}!zzzSchnitzler, Arthur@\emph{von Arthur Schnitzler}!1921-05-251@{25. 5. 1921}|(be} \toendnotes[C]{\smallbreak\pagebreak[2]} \Standort{DLA, 96.34.2/26.}
\physDesc{Postkarte
\newline{}Handschrift: schwarze Tinte, deutsche Kurrent\newline{}Versand: 1) Stempel: »\nobreak{}\oindex{I., Innere Stadt@\textbf{I., Innere Stadt}|pwk}1/1 Wien 8, 25. V. 21, 5\nobreak{}«.  2) zusätzlicher Stempel: »\noindent{}›HELFT ÖSTERREICHS\oindex{Oesterreich@\textbf{Österreich}|pw} KINDERN!‹{ / }AMERIK. KINDERHILFSAKTION\orgindex{Amerikanische Kinderhilfsaktion@Amerikanische Kinderhilfsaktion|pw}{ / }WIEN I.\oindex{I., Innere Stadt@\textbf{I., Innere Stadt}|pw}{ / }ELISABETHSTR. 9\oindex{Elisabethstrasse@\textbf{Elisabethstraße}|pw}«, dieser auch in englischer Sprache gestempelt, aber nur
                                 bruchstückhaft entzifferbar3) die falsche Bezirksangabe in der Empfängeradresse wurde von
                                 unbekannter Hand mit Bleistift zu »XII«
                                 korrigiert.}\pstart{}{\pb}A. S. Wien XVIII. \textsc{Sternwartestr} 71\oindex{Sternwartestrasse@\textbf{Sternwartestraße}|pw}\pend{}{\bigskip}\pstart{}Herrn Ob. Landesgerichtsrat\pend{}\pstart{}\textsc{Dr. Robert Adam Pollak}\pend{}\pstart{}\textsc{Wien XIII}\oindex{XIII., Hietzing@\textbf{XIII., Hietzing}|pw}\pend{}\pstart{}58 \textsc{Meidlinger Hauptstr}
                     58\oindex{Meidlinger Hauptstrasse@\textbf{Meidlinger Hauptstraße}|pw}\pend{}{\bigskip}\pstart
           \raggedleft{}{\pb}25. 5. 1921\pend
           \pstart{}Verehrtester Herr Doktor,\pend\pstart
           wenn Sie denn am Dienſtag (31. 5.) gegen Abend nach 6 für
               mich Zeit hätten, wäre mir Ihr lieber Beſuch ſehr willko{\geminationm}en. In der Zwiſchenzeit war ich auch verreiſt, München\oindex{Muenchen@\textbf{München}|pw} u. Salzburg\oindex{Salzburg@\textbf{Salzburg}|pw}.\pend
           \pstart
           Herzlichſt grüßend Ihr{\\[\baselineskip]}ſehr ergebener{\\[\baselineskip]}\spacefill\mbox{Arthur Schnitzler}\pend
           \leftskip=0em{}\endnumbering\briefempfaengerindex{Adam, Robert@\textsc{Adam, Robert}!zzzSchnitzler, Arthur@\emph{von Arthur Schnitzler}!1921-05-251@{25. 5. 1921}|)be}\mylabel{h}\end{ledgroupsized}  \newcommand{\dateiname}{L02368}\newcommand{\titel}{Arthur Schnitzler an Robert Adam, 25. 5. 1921}\newcommand{\editorInnen}{Martin Anton Müller und Gerd-Hermann Susen}\input{../tex-inputs/latex-pdf-abspann}
      