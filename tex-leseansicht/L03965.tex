%% latex-leseansicht-vorspann.tex
%% Vorspann für die Leseansicht.
%% Lädt die gemeinsame Datei latex-vorspann.tex mit nicht gesetztem Schalter.

\newif\ifkorrekturansicht
\korrekturansichtfalse

\input{../tex-inputs/latex-vorspann}


\section[Arthur Schnitzler an Berta Zuckerkandl, 25. 1. 1926]{L03965 Arthur Schnitzler an Berta Zuckerkandl, 25. 1. 1926}
\nopagebreak\mylabel{L03965v}
\rehead{ }\normalsize\beginnumbering\briefempfaengerindex{Zuckerkandl, Berta@\textsc{Zuckerkandl, Berta}!zzzSchnitzler, Arthur@\emph{von Arthur Schnitzler}!1926-01-251@{25. 1. 1926}|(be}
\toendnotes[C]{\smallbreak\pagebreak[2]}
\correspDesc{Versand  durch Arthur Schnitzler am 25. 1. 1926 in Wien
\newline{}Erhalt  durch Berta Zuckerkandl im Zeitraum [26. 1. 1926
                  – 30. 1. 1926?] in Paris}\toendnotes[C]{\smallbreak}
\Standort{DLA, HS.1985.1.2282.}
\physDesc{Brief, Durchschlag, 2 Blätter, 4 Seiten, 4641 Zeichen
\newline{}Schreibmaschine
\newline{}Handschrift Arthur Schnitzler: 1) roter Buntstift, lateinische Kurrent (\noindent{}beschriftet: »\uline{Zuckerkandl}« und »\uline{Paris}«, vierundzwanzig Unterstreichungen)\hspace{1em}2) Bleistift, lateinische Kurrent (\noindent{}beschriftet S. 3: »\uline{Zuckerkandl}«, Korrekturen)\hspace{1em}
\newline{}Handschrift Frieda Pollak: Bleistift (\noindent{}beschriftet S. 3: »25/>1. 1926«)}\toendnotes[C]{\smallbreak}
\pstart
           \raggedleft{}{\pb}25. 1. 1926.\pend
           
\pstart{}Liebe und verehrte Frau Hofrätin.\pend\vspace{0.5em}
\pstart
           Vielen Dank für Ihre freundliche \label{K_L03965-1v}\edtext{Nachricht}{\lemma{\textnormal{\emph{Nachricht}}}\Cendnote{\textnormal{nicht überliefert}}}\label{K_L03965-1}.
               Bei meinen \label{K_L03965-2v}\edtext{Gesprächen mit Gemier\pwindex{Gémier, Firmin 21.\,2.\,1865 Aubervilliers – 26.\,11.\,1933 Paris@\textsc{Gémier, Firmin} (21.\,2.\,1865 Aubervilliers – 26.\,11.\,1933 Paris), \emph{Theaterleiter, Schauspieler, Drehbuchautor}|pw}}{\lemma{\textnormal{\emph{Gesprächen mit Gemier}}}\Cendnote{\textnormal{Vgl. A. S.: \emph{Tagebuch}, 14. 12. 1925 und 15. 12. 1925.}}}\label{K_L03965-2} hatte ich gleich den Eindruck, dass er das »Weite Land\pwindex{Schnitzler, Arthur 15. 5. 1862 Wien – 21. 10. 1931 ebd.@\textsc{Schnitzler, Arthur} (15. 5. 1862 Wien – 21. 10. 1931 ebd.), \emph{Schriftsteller, Mediziner}!weite Land. Tragikomödie in fünf Akten@\strich\emph{Das weite Land. Tragikomödie in fünf Akten}|pw}« eigentlich noch nicht kennt. Schade,
               dass er es nun doch \label{K_L03965-3v}\edtext{vor der Aufführung
               }{\lemma{\textnormal{\emph{vor der Aufführung
               }}}\Cendnote{\textnormal{Zu einer
                  Aufführung von \emph{Das weite Land}\pwindex{Schnitzler, Arthur 15. 5. 1862 Wien – 21. 10. 1931 ebd.@\textsc{Schnitzler, Arthur} (15. 5. 1862 Wien – 21. 10. 1931 ebd.), \emph{Schriftsteller, Mediziner}!weite Land. Tragikomödie in fünf Akten@\strich\emph{Das weite Land. Tragikomödie in fünf Akten}|pwk} in Paris\oindex{Paris@\textbf{Paris}, \emph{Hauptstadt}|pwk} durch Gémier\pwindex{Gémier, Firmin 21.\,2.\,1865 Aubervilliers – 26.\,11.\,1933 Paris@\textsc{Gémier, Firmin} (21.\,2.\,1865 Aubervilliers – 26.\,11.\,1933 Paris), \emph{Theaterleiter, Schauspieler, Drehbuchautor}|pwk} kam es nicht.}}}\label{K_L03965-3} an seinem Theater\oindex{Odéon@\textbf{Odéon}, \emph{Theater}|pwv} gelesen hat. Es ist ja erfreulich, dass ihm das Stück\pwindex{Schnitzler, Arthur 15. 5. 1862 Wien – 21. 10. 1931 ebd.@\textsc{Schnitzler, Arthur} (15. 5. 1862 Wien – 21. 10. 1931 ebd.), \emph{Schriftsteller, Mediziner}!weite Land. Tragikomödie in fünf Akten@\strich\emph{Das weite Land. Tragikomödie in fünf Akten}|pwv} so gut gefällt – ob er
               der richtige Darsteller für den Hofreiter\pwindex{Schnitzler, Arthur 15. 5. 1862 Wien – 21. 10. 1931 ebd.@\textsc{Schnitzler, Arthur} (15. 5. 1862 Wien – 21. 10. 1931 ebd.), \emph{Schriftsteller, Mediziner}!weite Land. Tragikomödie in fünf Akten@\strich\emph{Das weite Land. Tragikomödie in fünf Akten}|pwv} ist kann ich nicht beurteilen. Sein Aeusseres spricht ja nicht
               dafür. Wir werden ja sehen wie sich die Sache weiter entwickelt, eventuell kann ja
               auch Lenormand\pwindex{Lenormand, Henri-René 3.\,5.\,1882 Paris – 16.\,2.\,1951 ebd.@\textsc{Lenormand, Henri-René} (3.\,5.\,1882 Paris – 16.\,2.\,1951 ebd.), \emph{Schriftsteller}|pw} einen Einfluss auf die Besetzung
               nehmen – glauben Sie nicht? Und ich nehme an, dass sich die Sache noch einige Zeit
               hinausschieben wird.\pend
           
\pstart
           Ueber die \label{K_L03965-4v}\edtext{Aufführung des »\label{T_L03965-1v}\edtext{Tapferen}{\lemma{\textnormal{\emph{Tapferen}}}\Cendnote{\textnormal{In der Vorlage steht:
                        »Taüferen«.}}}\label{T_L03965-1} Cassian\pwindex{Schnitzler, Arthur 15. 5. 1862 Wien – 21. 10. 1931 ebd.@\textsc{Schnitzler, Arthur} (15. 5. 1862 Wien – 21. 10. 1931 ebd.), \emph{Schriftsteller, Mediziner}!tapfere Cassian. Puppenspiel in einem Akt@\strich\emph{Der tapfere Cassian. Puppenspiel in einem Akt}|pw}«}{\lemma{\textnormal{\emph{Aufführung … Cassian«}}}\Cendnote{\textnormal{Es fand keine Aufführung des des »\emph{Tapferen Cassian}\pwindex{Schnitzler, Arthur 15. 5. 1862 Wien – 21. 10. 1931 ebd.@\textsc{Schnitzler, Arthur} (15. 5. 1862 Wien – 21. 10. 1931 ebd.), \emph{Schriftsteller, Mediziner}!tapfere Cassian. Puppenspiel in einem Akt@\strich\emph{Der tapfere Cassian. Puppenspiel in einem Akt}|pwk}« in Paris\oindex{Paris@\textbf{Paris}, \emph{Hauptstadt}|pwk} im Jahr 1926 statt.}}}\label{K_L03965-4} werden Sie mir vielleicht
               doch noch persönlich berichten können, da ja Ihr letzter Brief noch nichts über den
               Termin Ihrer Rückreise aussagt.\pend
           
\pstart
           Was den »Reigen\pwindex{Schnitzler, Arthur 15. 5. 1862 Wien – 21. 10. 1931 ebd.@\textsc{Schnitzler, Arthur} (15. 5. 1862 Wien – 21. 10. 1931 ebd.), \emph{Schriftsteller, Mediziner}!Reigen. Zehn Dialoge@\strich\emph{Reigen. Zehn Dialoge}|pw}« anbelangt, so habe ich schon
                  im vorigen Jahre (anlässlich der einmaligen unberechtigten Aufführung\eventindex{Paris@\textbf{Paris}!?? [unautorisierte Aufführung von Reigen]@?? [unautorisierte Aufführung von Reigen]|pwv}) ganz ausdrücklich
               meinen \label{K_L03965-5v}\edtext{Wunsch kundgegeben}{\lemma{\textnormal{\emph{Wunsch kundgegeben}}}\Cendnote{\textnormal{Arthur
                     Schnitzler an Robert de Flers\pwindex{Flers, Robert de 25.\,11.\,1872 Pont-l'Évêque – 30.\,7.\,1927 Vittel@\textsc{Flers, Robert de} (25.\,11.\,1872 Pont-l'Évêque – 30.\,7.\,1927 Vittel), \emph{Schriftsteller}|pwk},
                     3. 2. 1922, \emph{Deutsches Literaturarchiv
                        Marbach}, HS.1985.1.734,1.}}}\label{K_L03965-5}, dass das Stück\pwindex{Schnitzler, Arthur 15. 5. 1862 Wien – 21. 10. 1931 ebd.@\textsc{Schnitzler, Arthur} (15. 5. 1862 Wien – 21. 10. 1931 ebd.), \emph{Schriftsteller, Mediziner}!Reigen. Zehn Dialoge@\strich\emph{Reigen. Zehn Dialoge}|pwv} vorläufig nicht auf der
                  französischen\oindex{Frankreich@\textbf{Frankreich}|pw} Bühne erscheine. Erinnere ich
               mich recht, so habe ich das damals \label{K_L03965-6v}\edtext{an
                  Robert des Flers\pwindex{Flers, Robert de 25.\,11.\,1872 Pont-l'Évêque – 30.\,7.\,1927 Vittel@\textsc{Flers, Robert de} (25.\,11.\,1872 Pont-l'Évêque – 30.\,7.\,1927 Vittel), \emph{Schriftsteller}|pw}}{\lemma{\textnormal{\emph{an
                  Robert des Flers}}}\Cendnote{\textnormal{Der Schriftsteller Robert
                     des Flers\pwindex{Flers, Robert de 25.\,11.\,1872 Pont-l'Évêque – 30.\,7.\,1927 Vittel@\textsc{Flers, Robert de} (25.\,11.\,1872 Pont-l'Évêque – 30.\,7.\,1927 Vittel), \emph{Schriftsteller}|pwk} war von 1920 bis 1923 Präsident der
                     \emph{Société des auteurs et compositeurs
                     dramatiques}\orgindex{Société des Auteurs et Compositeurs Dramatiques@Société des Auteurs et Compositeurs Dramatiques|pwk}.}}}\label{K_L03965-6} sozusagen offiziös geschrieben. Den »Reigen\pwindex{Schnitzler, Arthur 15. 5. 1862 Wien – 21. 10. 1931 ebd.@\textsc{Schnitzler, Arthur} (15. 5. 1862 Wien – 21. 10. 1931 ebd.), \emph{Schriftsteller, Mediziner}!Reigen. Zehn Dialoge@\strich\emph{Reigen. Zehn Dialoge}|pw}« darf man in Paris\oindex{Paris@\textbf{Paris}, \emph{Hauptstadt}|pw}
               erst öffentlich aufführen, wenn ich mit einem oder ein paar anderen {\pb}Stücken Erfolg gehabt habe. Bisher hat sich niemand vom
                  Theater Albert Ier\oindex{Théâtre Albert Ier@\textbf{Théâtre Albert Ier}, \emph{Theater}|pw} bei mir gemeldet und ich
               protestiere schon heute gegen die Absicht der Direktion den »Reigen\pwindex{Schnitzler, Arthur 15. 5. 1862 Wien – 21. 10. 1931 ebd.@\textsc{Schnitzler, Arthur} (15. 5. 1862 Wien – 21. 10. 1931 ebd.), \emph{Schriftsteller, Mediziner}!Reigen. Zehn Dialoge@\strich\emph{Reigen. Zehn Dialoge}|pw}« auf die Bühne zu bringen. (»Reigen\pwindex{Schnitzler, Arthur 15. 5. 1862 Wien – 21. 10. 1931 ebd.@\textsc{Schnitzler, Arthur} (15. 5. 1862 Wien – 21. 10. 1931 ebd.), \emph{Schriftsteller, Mediziner}!Reigen. Zehn Dialoge@\strich\emph{Reigen. Zehn Dialoge}|pw}« ist übrigens \label{K_L03965-7v}\edtext{bei Stock\orgindex{Éditions Stock@Éditions Stock|pw}}{\lemma{\textnormal{\emph{bei Stock}}}\Cendnote{\textnormal{Arthur Schnitzler: \emph{La ronde. dix
                        scènes dialoguées}\pwindex{Schnitzler, Arthur 15. 5. 1862 Wien – 21. 10. 1931 ebd.@\textsc{Schnitzler, Arthur} (15. 5. 1862 Wien – 21. 10. 1931 ebd.), \emph{Schriftsteller, Mediziner}!ronde. Dix scènes dialoguées@\strich\emph{La ronde. Dix scènes dialoguées}|pwk}, traduction de Maurice Rémon\pwindex{Rémon, Maurice 27.\,11.\,1861 Paris – 20.\,6.\,1945 Mérignac@\textsc{Rémon, Maurice} (27.\,11.\,1861 Paris – 20.\,6.\,1945 Mérignac), \emph{Übersetzer}|pwk} et Wilhelm Bauer\pwindex{Bauer, Wilhelm 27.\,11.\,1854 Zollingen – 11.\,9.\,1923 Paris@\textsc{Bauer, Wilhelm} (27.\,11.\,1854 Zollingen – 11.\,9.\,1923 Paris)|pwk},
                     Paris: \emph{Stock}\orgindex{Éditions Stock@Éditions Stock|pwk}{ }1912.}}}\label{K_L03965-7} erschienen in
               einer recht schlechten Uebersetzung\pwindex{Schnitzler, Arthur 15. 5. 1862 Wien – 21. 10. 1931 ebd.@\textsc{Schnitzler, Arthur} (15. 5. 1862 Wien – 21. 10. 1931 ebd.), \emph{Schriftsteller, Mediziner}!ronde. Dix scènes dialoguées@\strich\emph{La ronde. Dix scènes dialoguées}|pwv}).\pend
           
\pstart
           Nun zu »Fräulein
                  Else\pwindex{Schnitzler, Arthur 15. 5. 1862 Wien – 21. 10. 1931 ebd.@\textsc{Schnitzler, Arthur} (15. 5. 1862 Wien – 21. 10. 1931 ebd.), \emph{Schriftsteller, Mediziner}!Fräulein Else@\strich\emph{Fräulein Else}|pw}\pwindex{Schnitzler, Arthur 15. 5. 1862 Wien – 21. 10. 1931 ebd.@\textsc{Schnitzler, Arthur} (15. 5. 1862 Wien – 21. 10. 1931 ebd.), \emph{Schriftsteller, Mediziner}!Madmoiselle Else@\strich\emph{Madmoiselle Else}|pw}«{[}.{]} Vor allem bin ich absolut dagegen, dass die Novelle\pwindex{Schnitzler, Arthur 15. 5. 1862 Wien – 21. 10. 1931 ebd.@\textsc{Schnitzler, Arthur} (15. 5. 1862 Wien – 21. 10. 1931 ebd.), \emph{Schriftsteller, Mediziner}!Fräulein Else@\strich\emph{Fräulein Else}|pwv}\pwindex{Schnitzler, Arthur 15. 5. 1862 Wien – 21. 10. 1931 ebd.@\textsc{Schnitzler, Arthur} (15. 5. 1862 Wien – 21. 10. 1931 ebd.), \emph{Schriftsteller, Mediziner}!Madmoiselle Else@\strich\emph{Madmoiselle Else}|pwv} in einem
               Band zusammen mit anderen Novellen von mir erscheint. Es existieren bereits
               Uebersetzungen\pwindex{Schnitzler, Arthur 15. 5. 1862 Wien – 21. 10. 1931 ebd.@\textsc{Schnitzler, Arthur} (15. 5. 1862 Wien – 21. 10. 1931 ebd.), \emph{Schriftsteller, Mediziner}!Fräulein Else. A Novel@\strich\emph{Fräulein Else. A Novel}|pw}\pwindex{Schnitzler, Arthur 15. 5. 1862 Wien – 21. 10. 1931 ebd.@\textsc{Schnitzler, Arthur} (15. 5. 1862 Wien – 21. 10. 1931 ebd.), \emph{Schriftsteller, Mediziner}!Fräulein Else [englische Übersetzung]@\strich\emph{Fräulein Else [englische Übersetzung]}|pw} in England\oindex{England@\textbf{England}, \emph{Land}|pw} und Amerika\oindex{Vereinigte Staaten von Amerika [USA]@\textbf{Vereinigte Staaten von Amerika [USA]}|pw}, auch in Ungarn\oindex{Ungarn@\textbf{Ungarn}|pw}; in Holland\oindex{Niederlande@\textbf{Niederlande}|pw}, in der Czechoslowakei\oindex{Tschechoslowakei@\textbf{Tschechoslowakei}|pw} etc. werden welche vorbereitet.
               Ueberall erscheint die Novelle\pwindex{Schnitzler, Arthur 15. 5. 1862 Wien – 21. 10. 1931 ebd.@\textsc{Schnitzler, Arthur} (15. 5. 1862 Wien – 21. 10. 1931 ebd.), \emph{Schriftsteller, Mediziner}!Fräulein Else@\strich\emph{Fräulein Else}|pw} als Buch für
               sich. Und es gibt manche französische\oindex{Frankreich@\textbf{Frankreich}|pw} Bücher,
               sogar Romane, die keinen dickeren Band ausmachen, als die »Else\pwindex{Schnitzler, Arthur 15. 5. 1862 Wien – 21. 10. 1931 ebd.@\textsc{Schnitzler, Arthur} (15. 5. 1862 Wien – 21. 10. 1931 ebd.), \emph{Schriftsteller, Mediziner}!Fräulein Else@\strich\emph{Fräulein Else}|pw}\pwindex{Schnitzler, Arthur 15. 5. 1862 Wien – 21. 10. 1931 ebd.@\textsc{Schnitzler, Arthur} (15. 5. 1862 Wien – 21. 10. 1931 ebd.), \emph{Schriftsteller, Mediziner}!Madmoiselle Else@\strich\emph{Madmoiselle Else}|pw}« ausmachen würde. Uebrigens wird ja Herr Delamain\pwindex{Delamain, Maurice 28.\,4.\,1883 Jarnac – 2.\,5.\,1974 Paris@\textsc{Delamain, Maurice} (28.\,4.\,1883 Jarnac – 2.\,5.\,1974 Paris), \emph{Kritiker, Rechtsanwalt, Verleger}|pw} sich erst ein Urteil bilden können,
               wenn er die Novelle\pwindex{Schnitzler, Arthur 15. 5. 1862 Wien – 21. 10. 1931 ebd.@\textsc{Schnitzler, Arthur} (15. 5. 1862 Wien – 21. 10. 1931 ebd.), \emph{Schriftsteller, Mediziner}!Fräulein Else@\strich\emph{Fräulein Else}|pwv}\pwindex{Schnitzler, Arthur 15. 5. 1862 Wien – 21. 10. 1931 ebd.@\textsc{Schnitzler, Arthur} (15. 5. 1862 Wien – 21. 10. 1931 ebd.), \emph{Schriftsteller, Mediziner}!Madmoiselle Else@\strich\emph{Madmoiselle Else}|pwv}
               gelesen hat. Ich sende ihm heute das Buch\pwindex{Schnitzler, Arthur 15. 5. 1862 Wien – 21. 10. 1931 ebd.@\textsc{Schnitzler, Arthur} (15. 5. 1862 Wien – 21. 10. 1931 ebd.), \emph{Schriftsteller, Mediziner}!Fräulein Else@\strich\emph{Fräulein Else}|pwv} zu. Die Uebersetzung\pwindex{Schnitzler, Arthur 15. 5. 1862 Wien – 21. 10. 1931 ebd.@\textsc{Schnitzler, Arthur} (15. 5. 1862 Wien – 21. 10. 1931 ebd.), \emph{Schriftsteller, Mediziner}!Madmoiselle Else@\strich\emph{Madmoiselle Else}|pwv} der Frau Clara Pollaczek\pwindex{Pollaczek, Clara Katharina 15.\,1.\,1875 Wien – 22.\,7.\,1951 ebd.@\textsc{Pollaczek, Clara Katharina} (15.\,1.\,1875 Wien – 22.\,7.\,1951 ebd.), \emph{Schriftstellerin}|pw} sende ich aber erst an Sie, verehrte Freundin. Ich glaube,
               sie ist vorzüglich und viel höher zu werten als eine sogenannte Rohübersetzung. Ich
               halte es sogar für möglich, dass die Uebersetzung\pwindex{Schnitzler, Arthur 15. 5. 1862 Wien – 21. 10. 1931 ebd.@\textsc{Schnitzler, Arthur} (15. 5. 1862 Wien – 21. 10. 1931 ebd.), \emph{Schriftsteller, Mediziner}!Madmoiselle Else@\strich\emph{Madmoiselle Else}|pwv} so wie sie ist, wenn auch vielleicht mit
               einigen Retouchen, ohneweiters veröffentlicht werden könnte. Nun aber wird
               selbstverständlich bei jedem Menschen, der erfährt, dass die Uebersetzung\pwindex{Schnitzler, Arthur 15. 5. 1862 Wien – 21. 10. 1931 ebd.@\textsc{Schnitzler, Arthur} (15. 5. 1862 Wien – 21. 10. 1931 ebd.), \emph{Schriftsteller, Mediziner}!Madmoiselle Else@\strich\emph{Madmoiselle Else}|pwv} von jemandem stammt, dessen
               Muttersprache nicht das Französische\oindex{Frankreich@\textbf{Frankreich}|pw} ist, ein
               Vorurteil zu überwinden sein und ich frage daher an, ob es nicht möglich wäre Herrn
                  Delamain\pwindex{Delamain, Maurice 28.\,4.\,1883 Jarnac – 2.\,5.\,1974 Paris@\textsc{Delamain, Maurice} (28.\,4.\,1883 Jarnac – 2.\,5.\,1974 Paris), \emph{Kritiker, Rechtsanwalt, Verleger}|pw} die Uebersetzung\pwindex{Schnitzler, Arthur 15. 5. 1862 Wien – 21. 10. 1931 ebd.@\textsc{Schnitzler, Arthur} (15. 5. 1862 Wien – 21. 10. 1931 ebd.), \emph{Schriftsteller, Mediziner}!Madmoiselle Else@\strich\emph{Madmoiselle Else}|pwv} der Frau P.\pwindex{Pollaczek, Clara Katharina 15.\,1.\,1875 Wien – 22.\,7.\,1951 ebd.@\textsc{Pollaczek, Clara Katharina} (15.\,1.\,1875 Wien – 22.\,7.\,1951 ebd.), \emph{Schriftstellerin}|pw} vorläufig ohne Namensnennung, als eine ganz ernst und
               definitiv gemeinte zu übergeben. Wie wäre es, wenn Sie die Uebersetzung\pwindex{Schnitzler, Arthur 15. 5. 1862 Wien – 21. 10. 1931 ebd.@\textsc{Schnitzler, Arthur} (15. 5. 1862 Wien – 21. 10. 1931 ebd.), \emph{Schriftsteller, Mediziner}!Madmoiselle Else@\strich\emph{Madmoiselle Else}|pwv} zuerst einmal einem
               vollkommen objektiven Beurteiler, z. E. Geraldy\pwindex{Géraldy, Paul 6.\,3.\,1885 Paris – 9.\,3.\,1983 Neuilly-sur-Seine@\textsc{Géraldy, Paul} (6.\,3.\,1885 Paris – 9.\,3.\,1983 Neuilly-sur-Seine), \emph{Schriftsteller}|pw}
               oder {\pb}Lenormand\pwindex{Lenormand, Henri-René 3.\,5.\,1882 Paris – 16.\,2.\,1951 ebd.@\textsc{Lenormand, Henri-René} (3.\,5.\,1882 Paris – 16.\,2.\,1951 ebd.), \emph{Schriftsteller}|pw} lesen liessen, ohne ihnen zu sagen
               dass eine Oesterreicherin\oindex{Österreich@\textbf{Österreich}|pw} und keine gebürtige
                  Französin\oindex{Frankreich@\textbf{Frankreich}|pw} diese Uebersetzung verfasst hat.
               Natürlich dürfte man ihnen auch nicht unter dem Siegel der Verschwiegenheit verraten,
               wie sich die Sache wirklich verhält. Bitte sagen sie mir, wie sie darüber denken. In
               jedem Falle ist uns, da nun einmal diese »Rohübersetzung\pwindex{Schnitzler, Arthur 15. 5. 1862 Wien – 21. 10. 1931 ebd.@\textsc{Schnitzler, Arthur} (15. 5. 1862 Wien – 21. 10. 1931 ebd.), \emph{Schriftsteller, Mediziner}!Madmoiselle Else@\strich\emph{Madmoiselle Else}|pwv}« vorliegt, jede Art von Verhandlung sehr
               erleichtert und wir können ausser Delamain\pwindex{Delamain, Maurice 28.\,4.\,1883 Jarnac – 2.\,5.\,1974 Paris@\textsc{Delamain, Maurice} (28.\,4.\,1883 Jarnac – 2.\,5.\,1974 Paris), \emph{Kritiker, Rechtsanwalt, Verleger}|pw}
               immerhin auch Grasset\pwindex{Grasset, Bernard 6.\,3.\,1881 Chambéry – 20.\,10.\,1955 Paris@\textsc{Grasset, Bernard} (6.\,3.\,1881 Chambéry – 20.\,10.\,1955 Paris), \emph{Verleger}|pw} und andere Verleger in
               Betracht ziehen{[}.{]} Glauben sie nicht? Keineswegs wollen wir aber
               das Verlagsrecht ohne ein anständiges \label{K_L03965-8v}\edtext{\substVorne{}\textsuperscript{\begin{otherlanguage}{french}a\end{otherlanguage}}\substDazwischen{}\begin{otherlanguage}{french}à\end{otherlanguage}\substHinten{}{ }\begin{otherlanguage}{french}valoir\end{otherlanguage}}{\lemma{\textnormal{\emph{à valoir}}}\Cendnote{\textnormal{französisch:
                  Vorschuss}}}\label{K_L03965-8} aus der Hand geben.\pend
           
\pstart
           Wollen Sie nicht auch gelegentlich Delamain\pwindex{Delamain, Maurice 28.\,4.\,1883 Jarnac – 2.\,5.\,1974 Paris@\textsc{Delamain, Maurice} (28.\,4.\,1883 Jarnac – 2.\,5.\,1974 Paris), \emph{Kritiker, Rechtsanwalt, Verleger}|pw}
               fragen, ob noch Exemplare von »Anatol\pwindex{Schnitzler, Arthur 15. 5. 1862 Wien – 21. 10. 1931 ebd.@\textsc{Schnitzler, Arthur} (15. 5. 1862 Wien – 21. 10. 1931 ebd.), \emph{Schriftsteller, Mediziner}!Anatole. Suivi de La Compagne@\strich\emph{Anatole. Suivi de La Compagne}|pw}« und von
                  »La Ronde\pwindex{Schnitzler, Arthur 15. 5. 1862 Wien – 21. 10. 1931 ebd.@\textsc{Schnitzler, Arthur} (15. 5. 1862 Wien – 21. 10. 1931 ebd.), \emph{Schriftsteller, Mediziner}!ronde. Dix scènes dialoguées@\strich\emph{La ronde. Dix scènes dialoguées}|pw}« vorhanden sind? Es müsste nun
               wohl auch offiziell zu konstatieren sein, dass für neue Auflagen urheberrechtlich
               auch das Recht für Frankreich\oindex{Frankreich@\textbf{Frankreich}|pw}{ }heute wieder in meiner Hand ist und dass unbedingt diese alten Uebersetzungen\pwindex{Schnitzler, Arthur 15. 5. 1862 Wien – 21. 10. 1931 ebd.@\textsc{Schnitzler, Arthur} (15. 5. 1862 Wien – 21. 10. 1931 ebd.), \emph{Schriftsteller, Mediziner}!Anatole. Suivi de La Compagne@\strich\emph{Anatole. Suivi de La Compagne}|pw}\pwindex{Schnitzler, Arthur 15. 5. 1862 Wien – 21. 10. 1931 ebd.@\textsc{Schnitzler, Arthur} (15. 5. 1862 Wien – 21. 10. 1931 ebd.), \emph{Schriftsteller, Mediziner}!ronde. Dix scènes dialoguées@\strich\emph{La ronde. Dix scènes dialoguées}|pw} nicht mehr neu
               aufgelegt werden dürfen. Von Stock\orgindex{Éditions Stock@Éditions Stock|pw} hatte ich für
               jedes der beiden Bücher – 200 Francs erhalten. Es ist kaum denkbar, dass ein Verlag
               sich einbilden könnte dadurch auf ewige Zeiten ein Recht erworben zu haben. Natürlich
               würde nichts dagegen sprechen mit Delamain\pwindex{Delamain, Maurice 28.\,4.\,1883 Jarnac – 2.\,5.\,1974 Paris@\textsc{Delamain, Maurice} (28.\,4.\,1883 Jarnac – 2.\,5.\,1974 Paris), \emph{Kritiker, Rechtsanwalt, Verleger}|pw}
               auch über \introOben{}»\introOben{}Reigen\pwindex{Schnitzler, Arthur 15. 5. 1862 Wien – 21. 10. 1931 ebd.@\textsc{Schnitzler, Arthur} (15. 5. 1862 Wien – 21. 10. 1931 ebd.), \emph{Schriftsteller, Mediziner}!Reigen. Zehn Dialoge@\strich\emph{Reigen. Zehn Dialoge}|pw}\introOben{}«\introOben{} und »Anatole\pwindex{Schnitzler, Arthur 15. 5. 1862 Wien – 21. 10. 1931 ebd.@\textsc{Schnitzler, Arthur} (15. 5. 1862 Wien – 21. 10. 1931 ebd.), \emph{Schriftsteller, Mediziner}!Anatol@\strich\emph{Anatol}|pw}« von
               neuem abzuschliessen.\pend
           
\pstart
           Ich schreibe an Delamain\pwindex{Delamain, Maurice 28.\,4.\,1883 Jarnac – 2.\,5.\,1974 Paris@\textsc{Delamain, Maurice} (28.\,4.\,1883 Jarnac – 2.\,5.\,1974 Paris), \emph{Kritiker, Rechtsanwalt, Verleger}|pw}{ }heute nur ganz kurz, dass ich prinzipiell gegen die Vereinigung von »Fräulein Else\pwindex{Schnitzler, Arthur 15. 5. 1862 Wien – 21. 10. 1931 ebd.@\textsc{Schnitzler, Arthur} (15. 5. 1862 Wien – 21. 10. 1931 ebd.), \emph{Schriftsteller, Mediziner}!Fräulein Else@\strich\emph{Fräulein Else}|pw}« mit anderen Novellen in einem Band
               bin und dass Sie, verehrte Frau Hofrätin, binnen kurzem die französische\oindex{Frankreich@\textbf{Frankreich}|pw}{ }Uebersetzung\pwindex{Schnitzler, Arthur 15. 5. 1862 Wien – 21. 10. 1931 ebd.@\textsc{Schnitzler, Arthur} (15. 5. 1862 Wien – 21. 10. 1931 ebd.), \emph{Schriftsteller, Mediziner}!Madmoiselle Else@\strich\emph{Madmoiselle Else}|pwv} ihm übermitteln {\pb}werden. Ich bin noch bis ca. 4. Februar in
                  Wien\oindex{Wien@\textbf{Wien}, \emph{Verwaltungsgebiet}|pw} und hoffe am 15. wieder \label{K_L03965-9v}\edtext{von
               Berlin\oindex{Berlin@\textbf{Berlin}, \emph{Hauptstadt}|pw}}{\lemma{\textnormal{\emph{von
               Berlin}}}\Cendnote{\textnormal{Schnitzler hielt sich vom 6. 2. 1926 bis zum 12. 2. 1926 in Berlin\oindex{Berlin@\textbf{Berlin}, \emph{Hauptstadt}|pwk} auf.}}}\label{K_L03965-9} aus zurück zu sein.\pend
           
\pstart
           Mit den herzlichsten Grüssen, Ihre lieben Nachrichten mit Spannung
               erwartend,{\\[\baselineskip]} Ihr freundschaftlich ergebener\pend
           \leftskip=0em{}
\pstart
           \noindent{}Frau Hofrätin Bertha Zuckerkandl,{\\}Paris\oindex{Paris@\textbf{Paris}, \emph{Hauptstadt}|pw}.\pend
           \selectlanguage{ngerman}\endnumbering\briefempfaengerindex{Zuckerkandl, Berta@\textsc{Zuckerkandl, Berta}!zzzSchnitzler, Arthur@\emph{von Arthur Schnitzler}!1926-01-251@{25. 1. 1926}|)be}\mylabel{L03965h}
\begin{anhang}
\end{anhang}\newcommand{\dateiname}{L03965}\newcommand{\titel}{Arthur Schnitzler an Berta Zuckerkandl, 25. 1. 1926}\newcommand{\editorInnen}{Herausgegeben von Jahnke, SelmaMüller, Martin Anton}%% latex-leseansicht-abspann.tex
%% Abspann für die Leseansicht.
%% Der Schalter \ifkorrekturansicht ist bereits durch den Vorspann gesetzt.

%% latex-abspann.tex
%% Gemeinsamer Abspann für Korrekturansicht und Leseansicht.
%% Setzt den Schalter \ifkorrekturansicht voraus (gesetzt in den
%% einbindenden Dateien latex-korrekturansicht-abspann.tex bzw.
%% latex-leseansicht-abspann.tex).
%% ---------------------------------------------------------------

\normalsize

% Das esempio-Environment wird nur in der Leseansicht benötigt
\ifkorrekturansicht\else
\newenvironment{esempio}[3]%
{
    \vspace{1.5ex}
    \rlap{\underline{#1}}
    \par
    \setlength{\parindent}{0cm}
    \nopagebreak
    \leftskip=#2cm
    \rightskip=#3cm
}
{
    \par
}
\fi

\doendnotes{C}
\bigskip
\vfill

\clearpage

\footnotesize

\ifkorrekturansicht
  \lohead{\textsc{register}}
\fi

% theindex-Environment neu definieren ohne reledmac
\makeatletter
\renewenvironment{theindex}{%
  \ifkorrekturansicht
    \section*{\indexname}%
  \else
    \subsubsection*{Index der erwähnten Entitäten}%
  \fi
  \setlength{\parindent}{0pt}%
  \setlength{\parskip}{0pt plus 0.3pt}%
  \let\item\@idxitem
}{%
  \ifkorrekturansicht\clearpage\fi
}
\makeatother

\IfFileExists{\jobname-pw.ind}{\input{\jobname-pw.ind}}{}

% Quellenangabe nur in der Leseansicht
\ifkorrekturansicht\else
% Fallback-Definitionen, falls die .tex-Datei \titel etc. nicht gesetzt hat
\providecommand{\titel}{}
\providecommand{\editorInnen}{}
\providecommand{\dateiname}{\jobname}

\vspace{3cm}

\vfill

\footnotesize
\textsc{Quelle}: \titel. Herausgegeben von {\editorInnen}. In: \emph{Arthur Schnitzler: Briefwechsel mit Autorinnen und Autoren}.
 Digitale Edition, https://schnitzler-briefe.acdh.oeaw.ac.at/{\dateiname}.html (Stand \today)
\fi

\end{document}


