%% latex-leseansicht-vorspann.tex
%% Vorspann für die Leseansicht.
%% Lädt die gemeinsame Datei latex-vorspann.tex mit nicht gesetztem Schalter.

\newif\ifkorrekturansicht
\korrekturansichtfalse

\input{../tex-inputs/latex-vorspann}


         
         \renewcommand{\erwaehntePersonen}{Personen: Paul Goldmann, Clementine Goldmann, Olga Schnitzler}
         \renewcommand{\erwaehnteOrte}{Orte: Berlin, Tirol, Welsberg-Taisten, Wildbad Waldbrunn}
         \renewcommand{\erwaehnteWerke}{}
               \section[ Paul Goldmann an Arthur Schnitzler, 16. 8. 1907]{ Paul Goldmann an Arthur Schnitzler, 16. 8. 1907}\nopagebreak\mylabel{v}\rehead{ }\begin{ledgroupsized}[t]{13cm}\normalsize\beginnumbering \toendnotes[C]{\smallbreak\pagebreak[2]} \Standort{DLA, A:Schnitzler, HS.NZ85.1.3175.}
\physDesc{Postkarte, 543 Zeichen
\newline{}Handschrift: blaue Tinte, lateinische Kurrent
\newline{}Versand: 1) Stempel: »\nobreak{}\oindex{Berlin@\textbf{Berlin}|pwk}Berlin,
                                       W\textcolor{gray}{.} 9, 16. 8. 07, 11–12V\nobreak{}«.   2) Stempel: »\nobreak{}\oindex{Welsberg-Taisten@\textbf{Welsberg-Taisten}|pwk}Wels{[}berg{]}, 1\textcolor{gray}{×}. 8. \textcolor{gray}{0}\textcolor{gray}{7}\nobreak{}«. 
\newline{}Schnitzler: mit Bleistift das Datum »16. 8. {[}19{]}07« vermerkt }\toendnotes[C]{\smallbreak}\pstart{}{\pb}Herrn Dr. Arthur Schnitzler\pend{}\pstart{}Welsberg im Pustertal\oindex{Welsberg-Taisten@\textbf{Welsberg-Taisten}|pw}\pend{}\pstart{}Wildbad Waldbrunn\oindex{Wildbad Waldbrunn@\textbf{Wildbad Waldbrunn}|pw}.\pend{}\pstart{}Tirol\oindex{Tirol@\textbf{Tirol}|pw}.\pend{}{\bigskip}\pstart
           \noindent{}{\pb}Lieber Freund, Ich komme vielleicht nächste Woche mit
               meiner Mutter\pwindex{Goldmann, Clementine 1842-05-15 – 1924-02-24@\textsc{Goldmann, Clementine} (1842-05-15 – 1924-02-24)|pwv} nach \label{K_L03256-1v}\edtext{Welsberg\oindex{Welsberg-Taisten@\textbf{Welsberg-Taisten}|pw}}{\lemma{\textnormal{\emph{Welsberg}}}\Cendnote{\textnormal{siehe Paul Goldmann an Arthur Schnitzler, 18. 8. 1907}}}\label{K_L03256-1h}, kann Dich aber natürlich nicht bitten, mich \label{K_L03256-2v}\edtext{abzuwarten}{\lemma{\textnormal{\emph{abzuwarten}}}\Cendnote{\textnormal{Schnitzler\pwindex{Schnitzler, Arthur 15.05.1862 – 21.10.1931@\textsc{Schnitzler, Arthur} (15.05.1862 – 21.10.1931), \emph{Schriftsteller, Mediziner}|pwk} blieb bis zum 26. 8. 1907 in Welsberg\oindex{Welsberg-Taisten@\textbf{Welsberg-Taisten}|pwk}.}}}\label{K_L03256-2h}, da der Tag meines Eintreffens
               noch unbesti{\geminationm}t ist; hingegen bitte ich Dich sehr, für
               meine Mutter\pwindex{Goldmann, Clementine 1842-05-15 – 1924-02-24@\textsc{Goldmann, Clementine} (1842-05-15 – 1924-02-24)|pwv} und mich\strikeout{,} je ein ruhiges und nicht teueres Zimmer, etwa von
                  Donnerstag ab, reservieren zu lassen. Ich hoffe
               sicher, Dir \label{K_L03256-3v}\edtext{im Laufe meiner
               Urlaubsreise die Hand drücken}{\lemma{\textnormal{\emph{im … drücken}}}\Cendnote{\textnormal{nicht
                  geschehen}}}\label{K_L03256-3h} zu können und bin mit herzlichen Grüßen an Dich und Deine Frau\pwindex{Schnitzler, Olga 17.01.1882 – 13.01.1970@\textsc{Schnitzler, Olga} (17.01.1882 – 13.01.1970), \emph{Schauspielerin, Sängerin}|pwv}{ }\label{T_L03256-1v}\edtext{Dein}{\lemma{\textnormal{\emph{Dein}}}\Cendnote{\textnormal{in deutscher Kurrentschrift}}}\label{T_L03256-1h}{ }\spacefill\mbox{Paul Goldmann.}\pend
           
         
         \endnumbering\mylabel{h}\end{ledgroupsized}  \newcommand{\dateiname}{L03256}\newcommand{\titel}{Paul Goldmann an Arthur Schnitzler, 16. 8. 1907}\newcommand{\editorInnen}{Martin Anton Müller und Laura Untner}%% latex-leseansicht-abspann.tex
%% Abspann für die Leseansicht.
%% Der Schalter \ifkorrekturansicht ist bereits durch den Vorspann gesetzt.

%% latex-abspann.tex
%% Gemeinsamer Abspann für Korrekturansicht und Leseansicht.
%% Setzt den Schalter \ifkorrekturansicht voraus (gesetzt in den
%% einbindenden Dateien latex-korrekturansicht-abspann.tex bzw.
%% latex-leseansicht-abspann.tex).
%% ---------------------------------------------------------------

\normalsize

% Das esempio-Environment wird nur in der Leseansicht benötigt
\ifkorrekturansicht\else
\newenvironment{esempio}[3]%
{
    \vspace{1.5ex}
    \rlap{\underline{#1}}
    \par
    \setlength{\parindent}{0cm}
    \nopagebreak
    \leftskip=#2cm
    \rightskip=#3cm
}
{
    \par
}
\fi

\doendnotes{C}
\bigskip
\vfill

\clearpage

\footnotesize

\ifkorrekturansicht
  \lohead{\textsc{register}}
\fi

% theindex-Environment neu definieren ohne reledmac
\makeatletter
\renewenvironment{theindex}{%
  \ifkorrekturansicht
    \section*{\indexname}%
  \else
    \subsubsection*{Index der erwähnten Entitäten}%
  \fi
  \setlength{\parindent}{0pt}%
  \setlength{\parskip}{0pt plus 0.3pt}%
  \let\item\@idxitem
}{%
  \ifkorrekturansicht\clearpage\fi
}
\makeatother

\IfFileExists{\jobname-pw.ind}{\input{\jobname-pw.ind}}{}

% Quellenangabe nur in der Leseansicht
\ifkorrekturansicht\else
% Fallback-Definitionen, falls die .tex-Datei \titel etc. nicht gesetzt hat
\providecommand{\titel}{}
\providecommand{\editorInnen}{}
\providecommand{\dateiname}{\jobname}

\vspace{3cm}

\vfill

\footnotesize
\textsc{Quelle}: \titel. Herausgegeben von {\editorInnen}. In: \emph{Arthur Schnitzler: Briefwechsel mit Autorinnen und Autoren}.
 Digitale Edition, https://schnitzler-briefe.acdh.oeaw.ac.at/{\dateiname}.html (Stand \today)
\fi

\end{document}


      