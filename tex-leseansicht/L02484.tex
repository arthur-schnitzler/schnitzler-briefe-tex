%% latex-korrekturansicht-vorspann.tex
%% Vorspann für die Korrekturansicht.
%% Lädt die gemeinsame Datei latex-vorspann.tex mit gesetztem Schalter.

\newif\ifkorrekturansicht
\korrekturansichttrue

\input{../tex-inputs/latex-vorspann}


\section[Arthur Schnitzler an Robert Adam, 9. 4. 1927]{L02484 Arthur Schnitzler an Robert Adam, 9. 4. 1927}
\nopagebreak\mylabel{L02484v}
\rehead{ }\normalsize\beginnumbering\briefempfaengerindex{Adam, Robert@\textsc{Adam, Robert}!zzzSchnitzler, Arthur@\emph{von Arthur Schnitzler}!1927-04-091@{9. 4. 1927}|(be}
\toendnotes[C]{\smallbreak\pagebreak[2]}\Standort{DLA, 96.34.2/29.}
\physDesc{Postkarte, 940 Zeichen
\newline{}Handschrift: schwarze Tinte, lateinische Kurrent
\newline{}Versand: Stempel: »\nobreak{}9. IV. \textcolor{gray}{27}\nobreak{}«.  }\toendnotes[C]{\smallbreak}\pstart{}{\pb}\label{T_L02484-1v}\edtext{\textcolor{gray}{\textbf{A. S.}}}{\lemma{\textnormal{\emph{A. S.}}}\Cendnote{\textnormal{ovaler Absenderkleber}}}\label{T_L02484-1}\pend{}\pstart{}\textcolor{gray}{\textbf{WIEN, XVIII.}}\oindex{XVIII., Waehring@\textbf{XVIII., Währing}, \emph{A.ADM3}|pw}\pend{}\pstart{}\textcolor{gray}{\textbf{STERNWARTESTR. 71}}\oindex{Sternwartestrasse 71@\textbf{Sternwartestraße 71}, \emph{Wohngebäude (K.WHS)}|pw}\pend{}{\bigskip}\pstart{}H. Dr. Robert Adam Pollak\pend{}\pstart{}Ob.-Landesger-Rath\pend{}\pstart{}XII Wien Meidling\oindex{XII., Meidling@\textbf{XII., Meidling}, \emph{A.ADM3}|pw}\pend{}\pstart{}Meidlinger Hptstr 54\oindex{Meidlinger Hauptstrasse@\textbf{Meidlinger Hauptstraße}, \emph{Straße (K.STR)}|pw}.\pend{}{\bigskip}\vspace{1em}
\pstart
           \raggedleft{}{\pb}Wien\oindex{Wien@\textbf{Wien}, \emph{A.ADM2}|pw}, 9. 4. 927\pend
           \vspace{0.5em}
\pstart
           lieber und verehrter Herr Doctor, entschuldigen Sie dſs ich erst
               heute, u überdies auch mit ein paar flachligen Worten nur den Empfang Ihres
               interessanten u liebenswürdigen Briefes bestätige, der mit seinen Bedenken, wie nicht
               anders zu erwarten, gleich das Zentrum meiner kleinen Arbeit\pwindex{Geist im Wort und der Geist in der Tat@\emph{Der Geist im Wort und der Geist in der Tat}|pwv} trifft. Sie haben gewiſs recht, daſs es sich nie um
               eine \uline{Idee} handelt – aber ob nicht zugleich um etwas,
               das mit \textcolor{gray}{Recht} persönlicher \uline{Erfahrung} schon nah verwandt ist, wäre vielleicht zu erwägen. Ohne Erfahrung
               – gäbe es da{\geminationn} überhaupt eine Idee? – Doch das läßt {\pb}sich nicht auf dem Correspondenzwege (und überhaupt nicht
               endgiltig) erläutern. Vielleicht haben Sie, bei schönem Wetter, im späten Frühjahr
               einmal ein Stündchen Zeit für mich, ich denke an unsere Gespräche und an Sie selbst
               verehrter Herr Doktor in herzlicher Sympathie zurück.\pend
           \pstart Viele Grüſſe Ihr \spacefill\mbox{ArthSchnitzler}\pend{}\selectlanguage{ngerman}\endnumbering\briefempfaengerindex{Adam, Robert@\textsc{Adam, Robert}!zzzSchnitzler, Arthur@\emph{von Arthur Schnitzler}!1927-04-091@{9. 4. 1927}|)be}\mylabel{L02484h}  \normalsize

\doendnotes{C}
\bigskip
\vfill

\clearpage

\footnotesize

\lohead{\textsc{register}}

% Definiere theindex-Environment komplett neu ohne reledmac
\makeatletter
\renewenvironment{theindex}{%
  \section*{\indexname}%
  \setlength{\parindent}{0pt}%
  \setlength{\parskip}{0pt plus 0.3pt}%
  \let\item\@idxitem
}{%
  \clearpage
}
\makeatother

\IfFileExists{\jobname-pw.ind}{\input{\jobname-pw.ind}}{}

\end{document}

      