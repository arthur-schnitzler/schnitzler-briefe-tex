%% latex-leseansicht-vorspann.tex
%% Vorspann für die Leseansicht.
%% Lädt die gemeinsame Datei latex-vorspann.tex mit nicht gesetztem Schalter.

\newif\ifkorrekturansicht
\korrekturansichtfalse

\input{../tex-inputs/latex-vorspann}


         
         \renewcommand{\erwaehntePersonen}{Personen: Robert Adam}
         \renewcommand{\erwaehnteOrte}{Orte: Meidlinger Hauptstraße, Sternwartestraße, Wien, XII., Meidling, XVIII., Währing}
         \renewcommand{\erwaehnteWerke}{Werke: Der Geist im Wort und der Geist in der Tat}
               \section[Arthur Schnitzler an Robert Adam, 9. 4. 1927]{ Arthur Schnitzler an Robert Adam, 9. 4. 1927}\nopagebreak\mylabel{v}\rehead{ }\begin{ledgroupsized}[t]{13cm}\normalsize\beginnumbering \toendnotes[C]{\smallbreak\pagebreak[2]} \Standort{DLA, 96.34.2/29.}
\physDesc{Postkarte
\newline{}Handschrift: schwarze Tinte, lateinische Kurrent\newline{}Versand: Stempel: »\nobreak{}9. IV. \textcolor{gray}{27}\nobreak{}«.  }\toendnotes[C]{\smallbreak}\pstart{}{\pb}\label{T_L02484-1v}\edtext{\textcolor{gray}{\textbf{A. S.}}}{\lemma{\textnormal{\emph{A. S.}}}\Cendnote{\textnormal{ovaler Absenderkleber}}}\label{T_L02484-1h}\pend{}\pstart{}\textcolor{gray}{\textbf{WIEN, XVIII.}}\oindex{XVIII., Waehring@\textbf{XVIII., Währing}|pw}\pend{}\pstart{}\textcolor{gray}{\textbf{STERNWARTESTR. 71}}\oindex{Sternwartestrasse@\textbf{Sternwartestraße}|pw}\pend{}{\bigskip}\pstart{}H. Dr. Robert Adam Pollak\pend{}\pstart{}Ob.-Landesger-Rath\pend{}\pstart{}XII Wien Meidling\oindex{XII., Meidling@\textbf{XII., Meidling}|pw}\pend{}\pstart{}Meidlinger Hptstr 54\oindex{Meidlinger Hauptstrasse@\textbf{Meidlinger Hauptstraße}|pw}.\pend{}{\bigskip}\pstart
           \raggedleft{}{\pb}Wien\oindex{Wien@\textbf{Wien}|pw}, 9. 4. 927\pend
           \pstart
           lieber und verehrter Herr Doctor, entschuldigen Sie dſs ich
                    erst heute, u überdies auch mit ein paar flachligen Worten nur den Empfang Ihres
                    interessanten u liebenswürdigen Briefes bestätige, der mit seinen Bedenken, wie
                    nicht anders zu erwarten, gleich das Zentrum meiner kleinen Arbeit\pwindex{Schnitzler, Arthur 15.05.1862 – 21.10.1931@\textsc{Schnitzler, Arthur} (15.05.1862 – 21.10.1931), \emph{Schriftsteller, Mediziner}!Geist im Wort und der Geist in der Tat1927@\strich\emph{Der Geist im Wort und der Geist in der Tat} {[}1927{]}|pwv} trifft. Sie haben gewiſs recht,
                    daſs es sich nie um eine \uline{Idee} handelt – aber ob
                    nicht zugleich um etwas, das mit \textcolor{gray}{Recht} persönlicher \uline{Erfahrung} schon nah verwandt ist, wäre
                    vielleicht zu erwägen. Ohne Erfahrung – gäbe es da{\geminationn}
                    überhaupt eine Idee? – Doch das läßt {\pb}sich nicht auf dem
                    Correspondenzwege (und überhaupt nicht endgiltig) erläutern. Vielleicht haben
                    Sie, bei schönem Wetter, im späten Frühjahr einmal ein Stündchen Zeit für mich,
                    ich denke an unsere Gespräche und an Sie selbst verehrter Herr Doktor in
                    herzlicher Sympathie zurück.\pend
           \pstart Viele Grüſſe Ihr \spacefill\mbox{ArthSchnitzler}\pend{}
         
         \endnumbering\mylabel{h}\end{ledgroupsized}  \newcommand{\dateiname}{L02484}\newcommand{\titel}{Arthur Schnitzler an Robert Adam, 9. 4. 1927}\newcommand{\editorInnen}{Martin Anton Müller und Gerd-Hermann Susen}%% latex-leseansicht-abspann.tex
%% Abspann für die Leseansicht.
%% Der Schalter \ifkorrekturansicht ist bereits durch den Vorspann gesetzt.

%% latex-abspann.tex
%% Gemeinsamer Abspann für Korrekturansicht und Leseansicht.
%% Setzt den Schalter \ifkorrekturansicht voraus (gesetzt in den
%% einbindenden Dateien latex-korrekturansicht-abspann.tex bzw.
%% latex-leseansicht-abspann.tex).
%% ---------------------------------------------------------------

\normalsize

% Das esempio-Environment wird nur in der Leseansicht benötigt
\ifkorrekturansicht\else
\newenvironment{esempio}[3]%
{
    \vspace{1.5ex}
    \rlap{\underline{#1}}
    \par
    \setlength{\parindent}{0cm}
    \nopagebreak
    \leftskip=#2cm
    \rightskip=#3cm
}
{
    \par
}
\fi

\doendnotes{C}
\bigskip
\vfill

\clearpage

\footnotesize

\ifkorrekturansicht
  \lohead{\textsc{register}}
\fi

% theindex-Environment neu definieren ohne reledmac
\makeatletter
\renewenvironment{theindex}{%
  \ifkorrekturansicht
    \section*{\indexname}%
  \else
    \subsubsection*{Index der erwähnten Entitäten}%
  \fi
  \setlength{\parindent}{0pt}%
  \setlength{\parskip}{0pt plus 0.3pt}%
  \let\item\@idxitem
}{%
  \ifkorrekturansicht\clearpage\fi
}
\makeatother

\IfFileExists{\jobname-pw.ind}{\input{\jobname-pw.ind}}{}

% Quellenangabe nur in der Leseansicht
\ifkorrekturansicht\else
% Fallback-Definitionen, falls die .tex-Datei \titel etc. nicht gesetzt hat
\providecommand{\titel}{}
\providecommand{\editorInnen}{}
\providecommand{\dateiname}{\jobname}

\vspace{3cm}

\vfill

\footnotesize
\textsc{Quelle}: \titel. Herausgegeben von {\editorInnen}. In: \emph{Arthur Schnitzler: Briefwechsel mit Autorinnen und Autoren}.
 Digitale Edition, https://schnitzler-briefe.acdh.oeaw.ac.at/{\dateiname}.html (Stand \today)
\fi

\end{document}


      