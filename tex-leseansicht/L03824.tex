%% latex-leseansicht-vorspann.tex
%% Vorspann für die Leseansicht.
%% Lädt die gemeinsame Datei latex-vorspann.tex mit nicht gesetztem Schalter.

\newif\ifkorrekturansicht
\korrekturansichtfalse

\input{../tex-inputs/latex-vorspann}


\section[Theodor Herzl an Arthur Schnitzler, 16. 11. 1892]{L03824 Theodor Herzl an Arthur Schnitzler, 16. 11. 1892}
\nopagebreak\mylabel{L03824v}
\rehead{ }\normalsize\beginnumbering\briefempfaengerindex{Schnitzler, Arthur@\textsc{Schnitzler, Arthur}!zzzHerzl, Theodor@\emph{von Theodor Herzl}!1892-11-161@{16. 11. 1892}|(be}
\toendnotes[C]{\smallbreak\pagebreak[2]}
\correspDesc{Versand  durch Theodor Herzl am 16. 11. 1892 in Paris
\newline{}Erhalt  durch Arthur Schnitzler im Zeitraum [17. 11. 1892 – 21. 11. 1892?] in Wien}\toendnotes[C]{\smallbreak}
\Standort{CUL, Schnitzler, B 39.}
\physDesc{Brief, 1 Blatt, 3 Seiten, 1668 Zeichen
\newline{}Handschrift: schwarze Tinte, lateinische Kurrent
\newline{}Ordnung: 1) mit Bleistift von unbekannter Hand nummeriert: »5«  2) mit blauem Buntstift von Leon Kellner\pwindex{Kellner, Leon 17.\,4.\,1859 Tarnów – 5.\,12.\,1928 Wien@\textsc{Kellner, Leon} (17.\,4.\,1859 Tarnów – 5.\,12.\,1928 Wien), \emph{Zionist, Literaturhistoriker, Anglist}|pw} Markierung von Stellen für
                                 die Publikation}
\buchAbdrucke{\weitereDrucke{1) \pwindex{Kellner, Leon 17.\,4.\,1859 Tarnów – 5.\,12.\,1928 Wien@\textsc{Kellner, Leon} (17.\,4.\,1859 Tarnów – 5.\,12.\,1928 Wien), \emph{Zionist, Literaturhistoriker, Anglist}!Theodor Herzls Lehrjahre (1860–1895). Nach den handschriftlichen Quellen@\strich\emph{Theodor Herzls Lehrjahre (1860–1895). Nach den handschriftlichen Quellen}|pwk}\emph{[Auszug].} In: Leon Kellner: \emph{Theodor Herzls Lehrjahre (1860–1895). Nach den handschriftlichen Quellen}. Wien, Berlin: \emph{R. Löwit-Verlag} 1920, S. 112.} \weitereDrucke{2) M. [=Hermann Menkes]: \emph{Theodor Herzl’s Abkehr vom Theater. Interessante Briefe an Artur Schnitzler.} In: \emph{Neues Wiener Journal}, Jg. 28, Nr. 9543, 1. 6. 1920, S. 4.} \weitereDrucke{3) \emph{Herzl-Briefe}. Herausgegeben und eingeleitet Manfred Georg. Berlin: \emph{Brandusche Verlagsbuchhandlung} [1935], S. 30–31.} \weitereDrucke{4) Theodor Herzl: \emph{Briefe und
                        autobiographische Notizen 1866–1895}. Bearbeitet von Johannes Wachten in Zusammenarbeit mit Chaya Harel, Daisy Tycho und Manfred Winkler. Berlin, Frankfurt am Main, Wien: \emph{Propyläen} 1983, S. 503 (Briefe und Tagebücher. Herausgegeben von Alex Bein, Hermann Greive, Moshe Schaerf, Julius H. Schoeps und Johannes Wachten, 1).} }\toendnotes[C]{\smallbreak}
\pstart
           {\pb}\textcolor{gray}{\textbf{NOUVELLE PRESSE LIBRE}}\orgindex{Neue Freie Presse@Neue Freie Presse|pw}\hfill \textcolor{gray}{\textbf{8, Rue de Monceau}}\oindex{8, rue de Monceau@\textbf{8, rue de Monceau}, \emph{Wohngebäude}|pw}\pend
           
\pstart
           \textcolor{gray}{\textbf{D\textsuperscript{r}{ }TH. HERZL}}\hfill 16. XI. 92\pend
           
\pstart{}Verehrtester Freund,\pend\vspace{0.5em}
\pstart
           was aus Ihrem Briefe spricht,
               ist das Wiener\oindex{Wien@\textbf{Wien}, \emph{Verwaltungsgebiet}|pw} Découragement.
      Kenn ich. Es wird durch Ortsveränderung
      geheilt.\pend
           
\pstart
           Ein anderes ist das meinige.
      Ich bin von mir abgekommen.
      Das ist der Grund, warum ich
      Ihrem so freundlichen Wunsch,
      Ihnen etwas von mir zu schicken,
      nicht entspreche. Ich kann diesen
      Wunsch eben nur für eine
      Freundlichkeit halten, und dass
      ich bei einer gewissen Einsicht,
      zu der ich herangealtert bin,
      doch nicht so frei von Eitelkeit
      bin, um das Gelesenwerden nur
                  {\pb}der Reciprocität verdanken zu
      wollen, werden Sie begreiflich
      finden.\pend
           
\pstart
           Ja, mein lieber Schnitzler, es gibt
      schon Leute, die um 10 oder
      zwölf oder gar 15 Jahre jünger
      sind als wir, und fertige
      Künstler sind. Ich weiss ganz
      wol, dass darin einige Melancholie
      liegt. Aber wir wollen uns nur
      freuen. Sie speciell sind wie die
      jungen Mädchen, die erst spät
      in die Gesellschaften gekommen sind.
      Man sieht Ihnen Ihre 30 Jahre
      nicht an – verstehen Sie es im
      guten Sinn.\pend
           
\pstart
           Wenn Sie mich, wie Sie im Sommer
      schrieben, immer ein Stück Wegs
      vor sich sahen – der Vorsprung
      ist mit Müdigkeit bezahlt gewesen,
      u. heute wie gesagt sitze {\pb}ich schon auf einem Stein
      der Landstrasse u. lasse die
      Anderen an mir vorüberkommen.\pend
           
\pstart
           Leben Sie wohl! Schicken Sie
      mir immer, was von Ihnen
      herauskommt; ich interessere mich
      aufrichtig dafür und habe mir
      hier keine Gelegenheit, die neuen
      deutschen Erscheinungen \strikeout{zu} ohne
      Beihilfe zu verfolgen, sonst würde
      ich nicht warten, bis Sie mir
      Ihre Werke schicken.\pend
           
\pstart
           Sehr gern würde ich die Sachen
               von Hoffmannsthal\pwindex{Hofmannsthal, Hugo von 1.\,2.\,1874 Wien – 15.\,7.\,1929 Rodaun@\textsc{Hofmannsthal, Hugo von} (1.\,2.\,1874 Wien – 15.\,7.\,1929 Rodaun), \emph{Schriftsteller}|pw} kennen. Könnten
               Sie mir sie nicht verschaffen? Ich habe nach dem Gedicht\pwindex{Hofmannsthal, Hugo von 1.\,2.\,1874 Wien – 15.\,7.\,1929 Rodaun@\textsc{Hofmannsthal, Hugo von} (1.\,2.\,1874 Wien – 15.\,7.\,1929 Rodaun), \emph{Schriftsteller}!Prolog [zum Anatol]@\strich\emph{Prolog [zum Anatol]}|pwv} von
               Anatol\pwindex{Schnitzler, Arthur 15.\,5.\,1862 Wien – 21.\,10.\,1931 ebd.@\textsc{Schnitzler, Arthur} (15.\,5.\,1862 Wien – 21.\,10.\,1931 ebd.), \emph{Schriftsteller, Mediziner}!Anatol@\strich\emph{Anatol}|pw} den Eindruck: er lässt
               sich blühen.\pend
           
\pstart
           Herzliche Grüsse von Ihrem ergebenen{\\[\baselineskip]}\spacefill\mbox{Th. H.}\pend
           \leftskip=0em{}\selectlanguage{ngerman}\endnumbering\briefempfaengerindex{Schnitzler, Arthur@\textsc{Schnitzler, Arthur}!zzzHerzl, Theodor@\emph{von Theodor Herzl}!1892-11-161@{16. 11. 1892}|)be}\mylabel{L03824h}
\begin{anhang}
\end{anhang}\newcommand{\dateiname}{L03824}\newcommand{\titel}{Theodor Herzl an Arthur Schnitzler, 16. 11. 1892}\newcommand{\editorInnen}{Selma Jahnke und Martin Anton Müller}%% latex-leseansicht-abspann.tex
%% Abspann für die Leseansicht.
%% Der Schalter \ifkorrekturansicht ist bereits durch den Vorspann gesetzt.

%% latex-abspann.tex
%% Gemeinsamer Abspann für Korrekturansicht und Leseansicht.
%% Setzt den Schalter \ifkorrekturansicht voraus (gesetzt in den
%% einbindenden Dateien latex-korrekturansicht-abspann.tex bzw.
%% latex-leseansicht-abspann.tex).
%% ---------------------------------------------------------------

\normalsize

% Das esempio-Environment wird nur in der Leseansicht benötigt
\ifkorrekturansicht\else
\newenvironment{esempio}[3]%
{
    \vspace{1.5ex}
    \rlap{\underline{#1}}
    \par
    \setlength{\parindent}{0cm}
    \nopagebreak
    \leftskip=#2cm
    \rightskip=#3cm
}
{
    \par
}
\fi

\doendnotes{C}
\bigskip
\vfill

\clearpage

\footnotesize

\ifkorrekturansicht
  \lohead{\textsc{register}}
\fi

% theindex-Environment neu definieren ohne reledmac
\makeatletter
\renewenvironment{theindex}{%
  \ifkorrekturansicht
    \section*{\indexname}%
  \else
    \subsubsection*{Index der erwähnten Entitäten}%
  \fi
  \setlength{\parindent}{0pt}%
  \setlength{\parskip}{0pt plus 0.3pt}%
  \let\item\@idxitem
}{%
  \ifkorrekturansicht\clearpage\fi
}
\makeatother

\IfFileExists{\jobname-pw.ind}{\input{\jobname-pw.ind}}{}

% Quellenangabe nur in der Leseansicht
\ifkorrekturansicht\else
% Fallback-Definitionen, falls die .tex-Datei \titel etc. nicht gesetzt hat
\providecommand{\titel}{}
\providecommand{\editorInnen}{}
\providecommand{\dateiname}{\jobname}

\vspace{3cm}

\vfill

\footnotesize
\textsc{Quelle}: \titel. Herausgegeben von {\editorInnen}. In: \emph{Arthur Schnitzler: Briefwechsel mit Autorinnen und Autoren}.
 Digitale Edition, https://schnitzler-briefe.acdh.oeaw.ac.at/{\dateiname}.html (Stand \today)
\fi

\end{document}


