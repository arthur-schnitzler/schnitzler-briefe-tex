%% latex-korrekturansicht-vorspann.tex
%% Vorspann für die Korrekturansicht.
%% Lädt die gemeinsame Datei latex-vorspann.tex mit gesetztem Schalter.

\newif\ifkorrekturansicht
\korrekturansichttrue

\input{../tex-inputs/latex-vorspann}


\section[Theodor Herzl an Arthur Schnitzler, 16. 11. 1892]{L03824 Theodor Herzl an Arthur Schnitzler, 16. 11. 1892}
\nopagebreak\mylabel{L03824v}
\rehead{ }\normalsize\beginnumbering\briefempfaengerindex{Schnitzler, Arthur@\textsc{Schnitzler, Arthur}!zzzHerzl, Theodor@\emph{von Theodor Herzl}!1892-11-161@{16. 11. 1892}|(be}
\toendnotes[C]{\smallbreak\pagebreak[2]}\Standort{CUL, Schnitzler, B 39.}
\physDesc{Brief, 1 Blatt, 3 Seiten, 1668 Zeichen
\newline{}Handschrift: schwarze Tinte, lateinische Kurrent
\newline{}Ordnung: 1) mit Bleistift von unbekannter Hand nummeriert: »1«  2) mit blauem Buntstift von Leon Kellner\pwindex{Kellner, Leon 1859-04-17 – 1928-12-05@\textsc{Kellner, Leon} (1859-04-17 – 1928-12-05), \emph{Zionist/Zionistin, Literaturhistoriker/Literaturhistorikerin, Anglist/Anglistin}|pw} Markierung von Stellen für
                                 die Publikation}\toendnotes[C]{\smallbreak}
\pstart
           {\pb}\textcolor{gray}{\textbf{NOUVELLE PRESSE LIBRE}}\orgindex{Neue Freie Presse@Neue Freie Presse|pw}\hfill \textcolor{gray}{\textbf{8, Rue de Monceau
                  }}\oindex{8, Rue de Monceau@\textbf{8, Rue de Monceau}, \emph{Wohngebäude (K.WHS)}|pw}\pend
           
\pstart
           \textcolor{gray}{\textbf{D\textsuperscript{R} TH. HERZL}}\hfill 16. XI. 92\pend
           
\pstart{}Verehrtester Freund,\pend\vspace{0.5em}
\pstart
           was aus Ihrem Briefe spricht,
               ist das Wiener\oindex{Wien@\textbf{Wien}, \emph{A.ADM2}|pw} Découragement.
      Kenn ich. Es wird durch Ortsveränderung
      geheilt.
      \pend
           
\pstart
           Ein anderes ist das meinige.
      Ich bin von mir abgekommen.
      Das ist der Grund, warum ich
      Ihrem so freundlichen Wunsch,
      Ihnen etwas von mir zu schicken,
      nicht entspreche. Ich kann diesen
      Wunsch eben nur für eine
      Freundlichkeit halten, und dass
      ich bei einer gewissen Einsicht,
      zu der ich herangealtert bin,
      doch nicht so frei von Eitelkeit
      bin, um das Gelesenwerden nur
                  {\pb}der Reciprocität verdanken zu
      wollen, werden Sie begreiflich
      finden.\pend
           
\pstart
           Ja, mein lieber Schnitzler, es gibt
      schon Leute, die um 10 oder
      zwölf oder gar 15 Jahre jünger
      sind als wir, und fertige
      Künstler sind. Ich weiss ganz
      wol, dass darin einige Melancholie
      liegt. Aber wir wollen uns nur
      freuen. Sie speciell sind wie die
      jungen Mädchen, die erst spät
      in die Gesellschaften gekommen sind.
      Man sieht Ihnen Ihre 30 Jahre
      nicht an – verstehen Sie es im
      guten Sinn.\pend
           
\pstart
           Wenn Sie mich, wie Sie im Sommer
      schrieben, immer ein Stück Wegs
      vor sich sahen – der Vorsprung
      ist mit Müdigkeit bezahlt gewesen,
      u. heute wie gesagt sitze {\pb}ich schon auf einem Stein
      der Landstrasse u. lasse die
      Anderen an mir vorüberkommen.\pend
           
\pstart
           Leben Sie wohl! Schicken Sie
      mir immer, was von Ihnen
      herauskommt; ich interessere mich
      aufrichtig dafür und habe mir
      hier keine Gelegenheit, die neuen
      deutschen Erscheinungen \strikeout{zu} ohne
      Beihilfe zu verfolgen, sonst würde
      ich nicht warten, bis Sie mir
      Ihre Werke schicken.\pend
           
\pstart
           Sehr gern würde ich die Sachen
               von Hoffmannsthal\pwindex{Hofmannsthal, Hugo von 1874-02-01 – 1929-07-15@\textsc{Hofmannsthal, Hugo von} (1874-02-01 – 1929-07-15), \emph{Schriftsteller/Schriftstellerin}|pw} kennen. Könnten
               Sie mir sie nicht verschaffen? Ich habe nach dem Gedicht\pwindex{Prolog [zum Anatol]@\emph{Prolog [zum Anatol]}|pwv} von
               Anatol\pwindex{Anatol@\emph{Anatol}|pw} den Eindruck: er lässt
               sich blühen.\pend
           
\pstart
           Herzliche Grüsse von Ihrem ergebenen{\\[\baselineskip]}\spacefill\mbox{Th. H.}\pend
           \leftskip=0em{}\selectlanguage{ngerman}\endnumbering\briefempfaengerindex{Schnitzler, Arthur@\textsc{Schnitzler, Arthur}!zzzHerzl, Theodor@\emph{von Theodor Herzl}!1892-11-161@{16. 11. 1892}|)be}\mylabel{L03824h}
\begin{anhang}
\end{anhang}\normalsize

\doendnotes{C}
\bigskip
\vfill

\clearpage

\footnotesize

\lohead{\textsc{register}}

% Definiere theindex-Environment komplett neu ohne reledmac
\makeatletter
\renewenvironment{theindex}{%
  \section*{\indexname}%
  \setlength{\parindent}{0pt}%
  \setlength{\parskip}{0pt plus 0.3pt}%
  \let\item\@idxitem
}{%
  \clearpage
}
\makeatother

\IfFileExists{\jobname-pw.ind}{\input{\jobname-pw.ind}}{}

\end{document}

      