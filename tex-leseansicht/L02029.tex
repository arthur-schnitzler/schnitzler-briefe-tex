%% latex-leseansicht-vorspann.tex
%% Vorspann für die Leseansicht.
%% Lädt die gemeinsame Datei latex-vorspann.tex mit nicht gesetztem Schalter.

\newif\ifkorrekturansicht
\korrekturansichtfalse

\input{../tex-inputs/latex-vorspann}


         
         \renewcommand{\erwaehntePersonen}{Personen: Hermann Bahr, Max Eugen Burckhard, Lisa Clarus, Olga Schnitzler}
         \renewcommand{\erwaehnteOrte}{Orte: Ober Sankt Veit, Wien}
         \renewcommand{\erwaehnteWerke}{Werke: Das weite Land. Tragikomödie in fünf Akten}
               \section[Hermann Bahr an Arthur Schnitzler, 17. 9. 1911]{ Hermann Bahr an Arthur Schnitzler, 17. 9. 1911}\nopagebreak\mylabel{v}\rehead{ }\begin{ledgroupsized}[t]{13cm}\normalsize\beginnumbering \toendnotes[C]{\smallbreak\pagebreak[2]} \Standort{CUL, Schnitzler, B 5b.}
\physDesc{Brief, 1 Blatt, 1 Seite, 327 Zeichen
\newline{}Handschrift Lisa Clarus: schwarze Tinte, lateinische Kurrent\newline{}Handschrift Hermann Bahr: schwarze Tinte (\noindent{}Unterschrift)
\newline{}Ordnung: mit Bleistift von unbekannter Hand nummeriert:
                                    »171« }\buchAbdrucke{\weitereDrucke{Hermann Bahr, Arthur Schnitzler: \emph{Briefwechsel, Aufzeichnungen, Dokumente (1891–1931)}. Hg. Kurt Ifkovits und Martin Anton Müller. Göttingen: \emph{Wallstein} 2018, S. 452.} }\toendnotes[C]{\smallbreak}\pstart
           \raggedleft{}{\pb}Wien XIII/\textsubscript{7}\oindex{Ober Sankt Veit@\textbf{Ober Sankt Veit}|pw}\hspace*{2.5em}17. 9. 11.\pend
           \pstart\center{}Lieber Freund!\pend\pstart
           Möchtest Du so lieb sein, mir Dein neues Stück\pwindex{Schnitzler, Arthur 15.05.1862 – 21.10.1931@\textsc{Schnitzler, Arthur} (15.05.1862 – 21.10.1931), \emph{Schriftsteller, Mediziner}!weite Land. Tragikomoedie in fuenf Akten1910-10-20@\strich\emph{Das weite Land. Tragikomödie in fünf Akten} {[}1910-10-20{]}|pwv} zu schicken? Burckhard\pwindex{Burckhard, Max Eugen 14.07.1854 – 16.03.1912@\textsc{Burckhard, Max Eugen} (14.07.1854 – 16.03.1912), \emph{Schriftsteller, Wissenschaftler, Theaterleiter}|pw}
               hat mir neulich so viel davon so schön erzählt, dass ich noch neugieriger geworden
               bin. Und ich könnte es jetzt halbwegs in Ruhe lesen!\pend
           \pstart
           Mit den herzlichsten Grüssen von uns beiden, auch an Deine liebe Frau\pwindex{Schnitzler, Olga 17.01.1882 – 13.01.1970@\textsc{Schnitzler, Olga} (17.01.1882 – 13.01.1970), \emph{Schauspielerin, Sängerin}|pwv},\pend
           \pstart
           Dein alter{\\[\baselineskip]}\spacefill\mbox{{[}hs. Bahr:{]} HermannBahr}\pend
           \leftskip=0em{}
         
         \endnumbering\mylabel{h}\end{ledgroupsized}  \newcommand{\dateiname}{L02029}\newcommand{\titel}{Hermann Bahr an Arthur Schnitzler, 17. 9. 1911}\newcommand{\editorInnen}{ Kurt Ifkovits,  Martin Anton Müller}%% latex-leseansicht-abspann.tex
%% Abspann für die Leseansicht.
%% Der Schalter \ifkorrekturansicht ist bereits durch den Vorspann gesetzt.

%% latex-abspann.tex
%% Gemeinsamer Abspann für Korrekturansicht und Leseansicht.
%% Setzt den Schalter \ifkorrekturansicht voraus (gesetzt in den
%% einbindenden Dateien latex-korrekturansicht-abspann.tex bzw.
%% latex-leseansicht-abspann.tex).
%% ---------------------------------------------------------------

\normalsize

% Das esempio-Environment wird nur in der Leseansicht benötigt
\ifkorrekturansicht\else
\newenvironment{esempio}[3]%
{
    \vspace{1.5ex}
    \rlap{\underline{#1}}
    \par
    \setlength{\parindent}{0cm}
    \nopagebreak
    \leftskip=#2cm
    \rightskip=#3cm
}
{
    \par
}
\fi

\doendnotes{C}
\bigskip
\vfill

\clearpage

\footnotesize

\ifkorrekturansicht
  \lohead{\textsc{register}}
\fi

% theindex-Environment neu definieren ohne reledmac
\makeatletter
\renewenvironment{theindex}{%
  \ifkorrekturansicht
    \section*{\indexname}%
  \else
    \subsubsection*{Index der erwähnten Entitäten}%
  \fi
  \setlength{\parindent}{0pt}%
  \setlength{\parskip}{0pt plus 0.3pt}%
  \let\item\@idxitem
}{%
  \ifkorrekturansicht\clearpage\fi
}
\makeatother

\IfFileExists{\jobname-pw.ind}{\input{\jobname-pw.ind}}{}

% Quellenangabe nur in der Leseansicht
\ifkorrekturansicht\else
% Fallback-Definitionen, falls die .tex-Datei \titel etc. nicht gesetzt hat
\providecommand{\titel}{}
\providecommand{\editorInnen}{}
\providecommand{\dateiname}{\jobname}

\vspace{3cm}

\vfill

\footnotesize
\textsc{Quelle}: \titel. Herausgegeben von {\editorInnen}. In: \emph{Arthur Schnitzler: Briefwechsel mit Autorinnen und Autoren}.
 Digitale Edition, https://schnitzler-briefe.acdh.oeaw.ac.at/{\dateiname}.html (Stand \today)
\fi

\end{document}


      