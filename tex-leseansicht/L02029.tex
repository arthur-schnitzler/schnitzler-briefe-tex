%% latex-korrekturansicht-vorspann.tex
%% Vorspann für die Korrekturansicht.
%% Lädt die gemeinsame Datei latex-vorspann.tex mit gesetztem Schalter.

\newif\ifkorrekturansicht
\korrekturansichttrue

\input{../tex-inputs/latex-vorspann}


\section[Hermann Bahr an Arthur Schnitzler, 17. 9. 1911]{L02029 Hermann Bahr an Arthur Schnitzler, 17. 9. 1911}
\nopagebreak\mylabel{L02029v}
\rehead{ }\normalsize\beginnumbering\briefempfaengerindex{Schnitzler, Arthur@\textsc{Schnitzler, Arthur}!zzzBahr, Hermann@\emph{von Hermann Bahr}!1911-09-171@{17. 9. 1911}|(be}
\toendnotes[C]{\smallbreak\pagebreak[2]}\Standort{CUL, Schnitzler, B 5b.}
\physDesc{Brief, 1 Blatt, 1 Seite, 327 Zeichen
\newline{}Handschrift Lisa Clarus: schwarze Tinte, lateinische Kurrent
\newline{}Handschrift Hermann Bahr: schwarze Tinte (\noindent{}Unterschrift)
\newline{}Ordnung: mit Bleistift von unbekannter Hand nummeriert:
                                    »171« }
\buchAbdrucke{\weitereDrucke{Hermann Bahr, Arthur Schnitzler: \emph{Briefwechsel, Aufzeichnungen, Dokumente (1891–1931)}. Göttingen: \emph{Wallstein} 2018, S. 452.} }\toendnotes[C]{\smallbreak}
\pstart
           \raggedleft{}{\pb}Wien XIII/\textsubscript{7}\oindex{Ober Sankt Veit@\textbf{Ober Sankt Veit}, \emph{P.PPLX}|pw}\hspace*{2.5em}17. 9. 11.\pend
           
\pstart\center{}Lieber Freund!\pend\vspace{0.5em}
\pstart
           Möchtest Du so lieb sein, mir Dein neues Stück\pwindex{weite Land. Tragikomoedie in fuenf Akten@\emph{Das weite Land. Tragikomödie in fünf Akten}|pwv} zu schicken? Burckhard\pwindex{Burckhard, Max Eugen 14.07.1854 – 16.03.1912@\textsc{Burckhard, Max Eugen} (14.07.1854 – 16.03.1912), \emph{Schriftsteller/Schriftstellerin, Rechtswissenschaftler/Rechtswissenschaftlerin, Theaterleiter/Theaterleiterin}|pw}
               hat mir neulich so viel davon so schön erzählt, dass ich noch neugieriger geworden
               bin. Und ich könnte es jetzt halbwegs in Ruhe lesen!\pend
           
\pstart
           Mit den herzlichsten Grüssen von uns beiden, auch an Deine liebe Frau\pwindex{Schnitzler, Olga 17.01.1882 – 13.01.1970@\textsc{Schnitzler, Olga} (17.01.1882 – 13.01.1970), \emph{Schauspieler/Schauspielerin, Sänger/Sängerin}|pwv},\pend
           
\pstart
           Dein alter{\\[\baselineskip]}\spacefill\mbox{{[}hs. :{]} HermannBahr}\pend
           \leftskip=0em{}\selectlanguage{ngerman}\endnumbering\briefempfaengerindex{Schnitzler, Arthur@\textsc{Schnitzler, Arthur}!zzzBahr, Hermann@\emph{von Hermann Bahr}!1911-09-171@{17. 9. 1911}|)be}\mylabel{L02029h}  \normalsize

\doendnotes{C}
\bigskip
\vfill

\clearpage

\footnotesize

\lohead{\textsc{register}}

% Definiere theindex-Environment komplett neu ohne reledmac
\makeatletter
\renewenvironment{theindex}{%
  \section*{\indexname}%
  \setlength{\parindent}{0pt}%
  \setlength{\parskip}{0pt plus 0.3pt}%
  \let\item\@idxitem
}{%
  \clearpage
}
\makeatother

\IfFileExists{\jobname-pw.ind}{\input{\jobname-pw.ind}}{}

\end{document}

      