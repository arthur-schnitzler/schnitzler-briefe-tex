%% latex-korrekturansicht-vorspann.tex
%% Vorspann für die Korrekturansicht.
%% Lädt die gemeinsame Datei latex-vorspann.tex mit gesetztem Schalter.

\newif\ifkorrekturansicht
\korrekturansichttrue

\input{../tex-inputs/latex-vorspann}


\section[Richard Beer-Hofmann an Arthur Schnitzler, 1. 7. 1904]{L01413 Richard Beer-Hofmann an Arthur Schnitzler, 1. 7. 1904}
\nopagebreak\mylabel{L01413v}
\rehead{ }\normalsize\beginnumbering\briefempfaengerindex{Schnitzler, Arthur@\textsc{Schnitzler, Arthur}!zzzBeer-Hofmann, Richard@\emph{von Richard Beer-Hofmann}!1904-07-011@{1. 7. 1904}|(be}
\toendnotes[C]{\smallbreak\pagebreak[2]}\Standort{CUL, Schnitzler, B 8.}
\physDesc{Bildpostkarte, 176 Zeichen
\newline{}Handschrift: schwarze Tinte, lateinische Kurrent
\newline{}Versand: 1) Stempel: »\nobreak{}\oindex{Bad Aussee@\textbf{Bad Aussee}, \emph{P.PPLA3}|pwk}Auss\textcolor{gray}{ee} in
                                       Steiermark, 1 7 {[}04{]}\nobreak{}«.   2) Stempel: »\nobreak{}\oindex{XVIII., Waehring@\textbf{XVIII., Währing}, \emph{A.ADM3}|pwk}18/1 Wien 110, 2. 7. 04, 10.V, Bestellt\nobreak{}«. 
\newline{}Ordnung: mit Bleistift von unbekannter Hand nummeriert: »183« }
\buchAbdrucke{\weitereDrucke{Arthur Schnitzler, Richard Beer-Hofmann: \emph{Briefwechsel 1891–1931}. Wien, Zürich: \emph{Europaverlag} 1992, S. 164.} }\toendnotes[C]{\smallbreak}\pstart{}{\pb}Herrn\pend{}\pstart{}Arthur Schnitzler\pend{}\pstart{}Wien\oindex{Wien@\textbf{Wien}, \emph{A.ADM2}|pw}\pend{}\pstart{}XVIII. Spöttelgasse 7\oindex{Edmund-Weiss-Gasse 7@\textbf{Edmund-Weiß-Gasse 7}, \emph{Wohngebäude (K.WHS)}|pw}.\pend{}{\bigskip}
\pstart
           \noindent{}\centering{}{\pb}\textcolor{gray}{\textbf{Aussee\oindex{Bad Aussee@\textbf{Bad Aussee}, \emph{P.PPLA3}|pw} von Sixleithen\oindex{Sixleitengasse@\textbf{Sixleitengasse}, \emph{Straße (K.STR)}|pw}.}}\pend
           \vspace{1em}
\pstart
           \raggedleft{}{\pb}1/VII 04\pend
           \vspace{0.5em}
\pstart
           Herzliche Grüße! Der arme Baron L.\pwindex{Schicksal des Freiherrn von Leisenbohg. Novellette@\emph{Das Schicksal des Freiherrn von Leisenbohg. Novellette}|pwv}! Sigurd\pwindex{Schicksal des Freiherrn von Leisenbohg. Novellette@\emph{Das Schicksal des Freiherrn von Leisenbohg. Novellette}|pwv} hat auf
                  »\label{K_L01413-1v}\edtext{Schlag treffen}{\lemma{\textnormal{\emph{Schlag treffen}}}\Cendnote{\textnormal{Der Erstdruck von \emph{Das Schicksal des Freiherrn von Leisenbohg}\pwindex{Schicksal des Freiherrn von Leisenbohg. Novellette@\emph{Das Schicksal des Freiherrn von Leisenbohg. Novellette}|pwk} erschien im
                     Juli-Heft von \emph{Die neue
                     Rundschau}\pwindex{neue Rundschau@\emph{Die neue Rundschau}|pwk} (Jg. 15, H. 7, S. 829–842.), das
                  damit nachweislich bereits ausgeliefert war. Ein Bekenntnis Sigurds\pwindex{Schicksal des Freiherrn von Leisenbohg. Novellette@\emph{Das Schicksal des Freiherrn von Leisenbohg. Novellette}|pwkv} bewirkt in der Novelle\pwindex{Schicksal des Freiherrn von Leisenbohg. Novellette@\emph{Das Schicksal des Freiherrn von Leisenbohg. Novellette}|pwkv}, dass sein Konkurrent Leisenbohg\pwindex{Schicksal des Freiherrn von Leisenbohg. Novellette@\emph{Das Schicksal des Freiherrn von Leisenbohg. Novellette}|pwkv} einen Herzinfarkt
                  erleidet. Beer-Hofmann\pwindex{Beer-Hofmann, Richard 1866-07-11 – 1945-09-26@\textsc{Beer-Hofmann, Richard} (1866-07-11 – 1945-09-26), \emph{Schriftsteller/Schriftstellerin}|pwk} erklärt seine
                  Auffassung, dass der Protagonist\pwindex{Schicksal des Freiherrn von Leisenbohg. Novellette@\emph{Das Schicksal des Freiherrn von Leisenbohg. Novellette}|pwkv} dies absichtlich tat.}}}\label{K_L01413-1} gespielt«! Und werden Sie
               gesund.\pend
           \pstart Ihr \spacefill\mbox{Richard}\pend{}
\pstart
           \noindent{}\label{T_L01413-1v}\edtext{unsere Wohnung}{\lemma{\textnormal{\emph{unsere Wohnung}}}\Cendnote{\textnormal{Verweis auf Markierung im Bild}}}\label{T_L01413-1}\pend
           \selectlanguage{ngerman}\endnumbering\briefempfaengerindex{Schnitzler, Arthur@\textsc{Schnitzler, Arthur}!zzzBeer-Hofmann, Richard@\emph{von Richard Beer-Hofmann}!1904-07-011@{1. 7. 1904}|)be}\mylabel{L01413h}  \normalsize

\doendnotes{C}
\bigskip
\vfill

\clearpage

\footnotesize

\lohead{\textsc{register}}

% Definiere theindex-Environment komplett neu ohne reledmac
\makeatletter
\renewenvironment{theindex}{%
  \section*{\indexname}%
  \setlength{\parindent}{0pt}%
  \setlength{\parskip}{0pt plus 0.3pt}%
  \let\item\@idxitem
}{%
  \clearpage
}
\makeatother

\IfFileExists{\jobname-pw.ind}{\input{\jobname-pw.ind}}{}

\end{document}

      