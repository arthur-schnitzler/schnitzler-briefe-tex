%% latex-leseansicht-vorspann.tex
%% Vorspann für die Leseansicht.
%% Lädt die gemeinsame Datei latex-vorspann.tex mit nicht gesetztem Schalter.

\newif\ifkorrekturansicht
\korrekturansichtfalse

\input{../tex-inputs/latex-vorspann}


         \renewcommand{\erwaehnteWerke}{}
               \section[Richard Beer-Hofmann an Arthur Schnitzler, 1. 7. 1904]{ Richard Beer-Hofmann an Arthur Schnitzler, 1. 7. 1904}\nopagebreak\mylabel{v}\rehead{ }\begin{ledgroupsized}[t]{13cm}\normalsize\beginnumbering \toendnotes[C]{\smallbreak\pagebreak[2]} \Standort{CUL, Schnitzler, B 8.}
\physDesc{Bildpostkarte
\newline{}Handschrift: schwarze Tinte, lateinische Kurrent\newline{}Versand: 1) Stempel: »\nobreak{}\oindex{XXXX Ortsangabe fehlt|pwk}Auss\textcolor{gray}{ee} in
                                       Steiermark, 1 7 {[}04{]}\nobreak{}«.   2) Stempel: »\nobreak{}\oindex{XXXX Ortsangabe fehlt|pwk}18/1 Wien 110, 2. 7. 04, 10.V, Bestellt\nobreak{}«. \newline{}Ordnung: mit Bleistift von unbekannter Hand nummeriert:
                                    »183« }\buchAbdrucke{\weitereDrucke{Arthur Schnitzler, Richard Beer-Hofmann: \emph{Briefwechsel 1891–1931}. Hg. Konstanze Fliedl. Wien, Zürich: \emph{Europaverlag} 1992, S. 164.} }\toendnotes[C]{\smallbreak}\pstart{}{\pb}Herrn\pend{}\pstart{}Arthur Schnitzler\pend{}\pstart{}Wien\oindex{XXXX Ortsangabe fehlt|pw}\pend{}\pstart{}XVIII. Spöttelgasse 7\oindex{XXXX Ortsangabe fehlt|pw}.\pend{}{\bigskip}\pstart
           \noindent{}\centering{}{\pb}\textcolor{gray}{\textbf{Aussee\oindex{XXXX Ortsangabe fehlt|pw} von Sixleithen\oindex{XXXX Ortsangabe fehlt|pw}.}}\pend
           \pstart
           \raggedleft{}1/VII 04\pend
           \pstart
           Herzliche Grüße! Der arme Baron L.\textcolor{red}{\textsuperscript{XXXX indx}}! Sigurd\textcolor{red}{\textsuperscript{XXXX indx}} hat auf
                  »\label{K_L01413_1v}\edtext{Schlag treffen}{\lemma{\textnormal{\emph{Schlag treffen}}}\Cendnote{\textnormal{Eine Eröffnung Sigurd\textcolor{red}{\textsuperscript{XXXX indx}}s bewirkt in \emph{Das
                     Schicksal des Freiherrn von Leisenbohg}\textcolor{red}{\textsuperscript{XXXX indx}}, dass sein Konkurrent
                     Leisenbohg\textcolor{red}{\textsuperscript{XXXX indx}} einen Herzinfarkt erleidet. Beer-Hofmann\pwindex{\textcolor{red}{\textsuperscript{XXXX1 indx}}|pwk} sagt, dass dies Sigurd\textcolor{red}{\textsuperscript{XXXX indx}} mit Absicht tat.}}}\label{K_L01413_1h} gespielt«! Und werden Sie
               gesund.\pend
           \pstart Ihr \spacefill\mbox{Richard}\pend{}\pstart
           \noindent{}\label{T_L01413_1v}\edtext{unsere Wohnung}{\lemma{\textnormal{\emph{unsere Wohnung}}}\Cendnote{\textnormal{Verweis auf Markierung im Bild}}}\label{T_L01413_1h}\pend
           
         
         \endnumbering\mylabel{h}\end{ledgroupsized}  \newcommand{\dateiname}{L01413}\newcommand{\titel}{Richard Beer-Hofmann an Arthur Schnitzler, 1. 7. 1904}\newcommand{\editorInnen}{Martin Anton Müller und Gerd-Hermann Susen}%% latex-leseansicht-abspann.tex
%% Abspann für die Leseansicht.
%% Der Schalter \ifkorrekturansicht ist bereits durch den Vorspann gesetzt.

%% latex-abspann.tex
%% Gemeinsamer Abspann für Korrekturansicht und Leseansicht.
%% Setzt den Schalter \ifkorrekturansicht voraus (gesetzt in den
%% einbindenden Dateien latex-korrekturansicht-abspann.tex bzw.
%% latex-leseansicht-abspann.tex).
%% ---------------------------------------------------------------

\normalsize

% Das esempio-Environment wird nur in der Leseansicht benötigt
\ifkorrekturansicht\else
\newenvironment{esempio}[3]%
{
    \vspace{1.5ex}
    \rlap{\underline{#1}}
    \par
    \setlength{\parindent}{0cm}
    \nopagebreak
    \leftskip=#2cm
    \rightskip=#3cm
}
{
    \par
}
\fi

\doendnotes{C}
\bigskip
\vfill

\clearpage

\footnotesize

\ifkorrekturansicht
  \lohead{\textsc{register}}
\fi

% theindex-Environment neu definieren ohne reledmac
\makeatletter
\renewenvironment{theindex}{%
  \ifkorrekturansicht
    \section*{\indexname}%
  \else
    \subsubsection*{Index der erwähnten Entitäten}%
  \fi
  \setlength{\parindent}{0pt}%
  \setlength{\parskip}{0pt plus 0.3pt}%
  \let\item\@idxitem
}{%
  \ifkorrekturansicht\clearpage\fi
}
\makeatother

\IfFileExists{\jobname-pw.ind}{\input{\jobname-pw.ind}}{}

% Quellenangabe nur in der Leseansicht
\ifkorrekturansicht\else
% Fallback-Definitionen, falls die .tex-Datei \titel etc. nicht gesetzt hat
\providecommand{\titel}{}
\providecommand{\editorInnen}{}
\providecommand{\dateiname}{\jobname}

\vspace{3cm}

\vfill

\footnotesize
\textsc{Quelle}: \titel. Herausgegeben von {\editorInnen}. In: \emph{Arthur Schnitzler: Briefwechsel mit Autorinnen und Autoren}.
 Digitale Edition, https://schnitzler-briefe.acdh.oeaw.ac.at/{\dateiname}.html (Stand \today)
\fi

\end{document}


      