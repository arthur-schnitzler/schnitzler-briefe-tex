%% latex-leseansicht-vorspann.tex
%% Vorspann für die Leseansicht.
%% Lädt die gemeinsame Datei latex-vorspann.tex mit nicht gesetztem Schalter.

\newif\ifkorrekturansicht
\korrekturansichtfalse

\input{../tex-inputs/latex-vorspann}


\section[Hugo von Hofmannsthal an Arthur Schnitzler, {[}7. 2. 1912{]}]{L02053 Hugo von Hofmannsthal an Arthur Schnitzler, {[}7. 2. 1912{]}}
\nopagebreak\mylabel{L02053v}
\rehead{ }\normalsize\beginnumbering\briefempfaengerindex{Schnitzler, Arthur@\textsc{Schnitzler, Arthur}!zzzHofmannsthal, Hugo von@\emph{von Hugo von Hofmannsthal}!1912-02-071@{{[}7. 2. 1912{]}}|(be}
\toendnotes[C]{\smallbreak\pagebreak[2]}
\correspDesc{Versand  durch Hugo von Hofmannsthal am [7. 2. 1912] in Semmering
\newline{}Erhalt  durch Arthur Schnitzler im Zeitraum [8. 2. 1912
                  – 12. 2. 1912?] in Wien}\toendnotes[C]{\smallbreak}
\Standort{CUL, Schnitzler, B 43.}
\physDesc{Brief, 1 Blatt, 2 Seiten, 475 Zeichen
\newline{}Handschrift: schwarze Tinte, deutsche Kurrent
\newline{}Schnitzler: mit Bleistift datiert: »7/2 912« und beschriftet: »\textsc{Hugo}« 
\newline{}Ordnung: 1) mit Bleistift von unbekannter Hand nummeriert: »\strikeout{325}«  2) mit Bleistift von unbekannter Hand nummeriert: »335«}
\buchAbdrucke{\weitereDrucke{Hugo von Hofmannsthal, Arthur Schnitzler: \emph{Briefwechsel}. Herausgegeben von Therese Nickl und Heinrich Schnitzler. Frankfurt am Main: \emph{S. Fischer} 1964, S. 264.} }\toendnotes[C]{\smallbreak}
\pstart
           \centering{}{\pb}\textcolor{gray}{\textbf{SÜDBAHN-HOTEL SEMMERING BEI WIEN\oindex{Südbahnhotel [Semmering]@\textbf{Südbahnhotel [Semmering]}, \emph{Hotel}|pw}}}\pend
           
\pstart
           \textcolor{gray}{\textbf{ERSTES HOTEL M. 350 ZIMMERN, GESCHÜTZTE, SCHÖNSTE U.
                        KLIMATISCH GÜNSTIGSTE LAGE AM SEMMERING\oindex{Semmering@\textbf{Semmering}, \emph{Verwaltungsgebiet}|pw}
                        MIT AUSSICHT AUF RAX\oindex{Rax@\textbf{Rax}, \emph{Berg}|pw}, SCHNEEBERG\oindex{Schneeberg@\textbf{Schneeberg}, \emph{Berg}|pw}, EISENBAHNLINIE\orgindex{Südbahnstrecke@Südbahnstrecke|pwv} ETC. K.K. HAUPTPOST, TELEGRAPHEN-
                        U. TELEPHONAMT IM HOTEL}}\pend
           
\pstart
           \textcolor{gray}{\textbf{1000 M ÜBER DEM MEERE. SOMMER- UND WINTERKURORT ERSTEN
                        RANGES}}{[}.{]}{ }\textcolor{gray}{\textbf{GRÖSSTER UND VORNEHMSTER WINTERSPORTPLATZ ÖSTERREICHS\oindex{Österreich@\textbf{Österreich}|pw}. 2 STUNDEN EISENBAHNFAHRT VON WIEN\oindex{Wien@\textbf{Wien}, \emph{Verwaltungsgebiet}|pw} UND GRAZ\oindex{Graz@\textbf{Graz}, \emph{Verwaltungsgebiet}|pw}.}}\pend
           
\pstart
           \centering{}\textcolor{gray}{\textbf{TELEGR.- U BRIEF-ADR: SÜDBAHNHOTEL SEMMERING\oindex{Südbahnhotel [Semmering]@\textbf{Südbahnhotel [Semmering]}, \emph{Hotel}|pw}, TELEPHON SEMMERING 5\oindex{Semmering@\textbf{Semmering}, \emph{Verwaltungsgebiet}|pw}.}}\pend
           
\pstart
           \raggedleft{}\textcolor{gray}{\textbf{Semmering\oindex{Semmering@\textbf{Semmering}, \emph{Verwaltungsgebiet}|pw}, am ..........}}\pend
           
\pstart{}mein lieber Arthur\pend\vspace{0.5em}
\pstart
           Ihre Zeilen waren lieb und woltuend wie immer, ich danke Ihnen{ }ſehr.\pend
           
\pstart
           Leſe in der Zeitung daſs es die »\label{K_L02053-1v}\edtext{Marionetten\pwindex{Schnitzler, Arthur 15.\,5.\,1862 Wien – 21.\,10.\,1931 ebd.@\textsc{Schnitzler, Arthur} (15.\,5.\,1862 Wien – 21.\,10.\,1931 ebd.), \emph{Schriftsteller, Mediziner}!Marionetten. Drei Einakter@\strich\emph{Marionetten. Drei Einakter}|pw}}{\lemma{\textnormal{\emph{Marionetten}}}\Cendnote{\textnormal{am 10. 2. 1912 im Deutschen Volkstheater\oindex{Wien@\textbf{Wien}!VII., Neubau@\textbf{VII., Neubau}!Volkstheater@\textbf{Volkstheater}, \emph{Theater}|pwk}}}}\label{K_L02053-1}«{ }ſind, die man{ }ſpielt, würde ich wohl für den 1\textsuperscript{ten}{ }{\pb}oder 2\textsuperscript{ten} Abend \uline{2 Balconsitze} durch Sie beko{\geminationm}en können? würde mich{ }ſehr freuen; vielleicht iſt es am
               einfachſten, Sie bezahlen{ }ſie für mich und{ }ſchicken mir{ }ſie an die Adreſſe Eliſabethſtraße 6\oindex{Wien@\textbf{Wien}!I., Innere Stadt@\textbf{I., Innere Stadt}!Elisabethstraße [Wien]@\textbf{Elisabethstraße [Wien]}, \emph{Straße}|pw}.\pend
           
\pstart
           Vielleicht ko{\geminationm}t die Bitte{ }ſchon zu{ }ſpät, dann gehe ich
               halt in eine{ }ſpätere Vorstellung.\pend
           
\pstart
           Herzlich{\\[\baselineskip]}\spacefill\mbox{Hugo.}\pend
           \leftskip=0em{}\selectlanguage{ngerman}\endnumbering\briefempfaengerindex{Schnitzler, Arthur@\textsc{Schnitzler, Arthur}!zzzHofmannsthal, Hugo von@\emph{von Hugo von Hofmannsthal}!1912-02-071@{{[}7. 2. 1912{]}}|)be}\mylabel{L02053h}  \newcommand{\dateiname}{L02053}\newcommand{\titel}{Hugo von Hofmannsthal an Arthur Schnitzler, [7. 2. 1912]}\newcommand{\editorInnen}{Martin Anton Müller und Gerd-Hermann Susen}%% latex-leseansicht-abspann.tex
%% Abspann für die Leseansicht.
%% Der Schalter \ifkorrekturansicht ist bereits durch den Vorspann gesetzt.

%% latex-abspann.tex
%% Gemeinsamer Abspann für Korrekturansicht und Leseansicht.
%% Setzt den Schalter \ifkorrekturansicht voraus (gesetzt in den
%% einbindenden Dateien latex-korrekturansicht-abspann.tex bzw.
%% latex-leseansicht-abspann.tex).
%% ---------------------------------------------------------------

\normalsize

% Das esempio-Environment wird nur in der Leseansicht benötigt
\ifkorrekturansicht\else
\newenvironment{esempio}[3]%
{
    \vspace{1.5ex}
    \rlap{\underline{#1}}
    \par
    \setlength{\parindent}{0cm}
    \nopagebreak
    \leftskip=#2cm
    \rightskip=#3cm
}
{
    \par
}
\fi

\doendnotes{C}
\bigskip
\vfill

\clearpage

\footnotesize

\ifkorrekturansicht
  \lohead{\textsc{register}}
\fi

% theindex-Environment neu definieren ohne reledmac
\makeatletter
\renewenvironment{theindex}{%
  \ifkorrekturansicht
    \section*{\indexname}%
  \else
    \subsubsection*{Index der erwähnten Entitäten}%
  \fi
  \setlength{\parindent}{0pt}%
  \setlength{\parskip}{0pt plus 0.3pt}%
  \let\item\@idxitem
}{%
  \ifkorrekturansicht\clearpage\fi
}
\makeatother

\IfFileExists{\jobname-pw.ind}{\input{\jobname-pw.ind}}{}

% Quellenangabe nur in der Leseansicht
\ifkorrekturansicht\else
% Fallback-Definitionen, falls die .tex-Datei \titel etc. nicht gesetzt hat
\providecommand{\titel}{}
\providecommand{\editorInnen}{}
\providecommand{\dateiname}{\jobname}

\vspace{3cm}

\vfill

\footnotesize
\textsc{Quelle}: \titel. Herausgegeben von {\editorInnen}. In: \emph{Arthur Schnitzler: Briefwechsel mit Autorinnen und Autoren}.
 Digitale Edition, https://schnitzler-briefe.acdh.oeaw.ac.at/{\dateiname}.html (Stand \today)
\fi

\end{document}


