\input{../tex-inputs/latex-pdf-vorspann}
\begin{center}
            \textcolor{red}{ENTWURF. ENTZIFFERUNG NOCH NICHT KORREKTURGELESEN}
                      \end{center}
            
               \section[Hugo von Hofmannsthal an Arthur Schnitzler, {[}7. 2. 1912{]}]{ Hugo von Hofmannsthal an Arthur Schnitzler, {[}7. 2. 1912{]}}\nopagebreak\mylabel{v}\rehead{ }\begin{ledgroupsized}[t]{13cm}\normalsize\beginnumbering\briefempfaengerindex{Schnitzler, Arthur@\textsc{Schnitzler, Arthur}!zzzHofmannsthal, Hugo von@\emph{von Hugo von Hofmannsthal}!1912-02-071@{{[}7. 2. 1912{]}}|(be} \toendnotes[C]{\smallbreak\pagebreak[2]} \Standort{CUL, Schnitzler, B 43.}
\physDesc{Brief, 1 Blatt, 2 Seiten
\newline{}Handschrift: schwarze Tinte, deutsche Kurrent
\newline{}Schnitzler: mit Bleistift datiert: »7/2 912« und beschriftet: »\textsc{Hugo}« \newline{}Ordnung: 1) mit Bleistift von unbekannter Hand nummeriert: »\strikeout{325}« 2) mit Bleistift von unbekannter Hand nummeriert:
                                    »335«}\buchAbdrucke{\weitereDrucke{Hugo von Hofmannsthal, Arthur Schnitzler: \emph{Briefwechsel}. Hg. Therese Nickl und Heinrich Schnitzler. Frankfurt am Main: \emph{S. Fischer} 1964, S. 264.} }\toendnotes[C]{\smallbreak}\pstart
           \noindent{}\centering{}{\pb}\textcolor{gray}{\textbf{SÜDBAHN-HOTEL SEMMERING BEI WIEN\oindex{Suedbahnhotel@\textbf{Südbahnhotel}|pw}}}\pend
           \pstart
           \noindent{}\textcolor{gray}{\textbf{ERSTES HOTEL M. 350 ZIMMERN, GESCHÜTZTE, SCHÖNSTE U.
                        KLIMATISCH GÜNSTIGSTE LAGE AM SEMMERING\oindex{Semmering@\textbf{Semmering}|pw} MIT
                        AUSSICHT AUF RAX\oindex{Rax@\textbf{Rax}|pw}, SCHNEEBERG\oindex{Schneeberg@\textbf{Schneeberg}|pw}, EISENBAHNLINIE\orgindex{Suedbahnstrecke@Südbahnstrecke|pwv} ETC. K.K. HAUPTPOST, TELEGRAPHEN-
                        U. TELEPHONAMT IM HOTEL}}\pend
           \pstart
           \textcolor{gray}{\textbf{1000 M ÜBER DEM MEERE. SOMMER- UND WINTERKURORT ERSTEN
                        RANGES}}{[}.{]}{ }\textcolor{gray}{\textbf{GRÖSSTER UND VORNEHMSTER WINTERSPORTPLATZ ÖSTERREICHS\oindex{Oesterreich@\textbf{Österreich}|pw}. 2 STUNDEN EISENBAHNFAHRT VON WIEN\oindex{Wien@\textbf{Wien}|pw} UND GRAZ\oindex{Graz@\textbf{Graz}|pw}.}}\pend
           \pstart
           \centering{}\textcolor{gray}{\textbf{TELEGR.- U BRIEF-ADR: SÜDBAHNHOTEL SEMMERING\oindex{Suedbahnhotel@\textbf{Südbahnhotel}|pw}, TELEPHON SEMMERING 5\oindex{Semmering@\textbf{Semmering}|pw}.}}\pend
           \pstart
           \noindent{}\raggedleft{}\textcolor{gray}{\textbf{Semmering\oindex{Semmering@\textbf{Semmering}|pw}, am ..........}}\pend
           \pstart{}mein lieber Arthur \pend\pstart
           Ihre Zeilen waren lieb und woltuend wie immer, ich danke Ihnen ſehr.\pend
           \pstart
           Leſe in der Zeitung daſs es die »\label{K_L02053_1v}\edtext{Marionetten\pwindex{Schnitzler, Arthur 15.05.1862 – 21.10.1931@\textsc{Schnitzler, Arthur} (15.05.1862 – 21.10.1931), \emph{Schriftsteller, Mediziner}!Marionetten. Drei Einakter1906@\strich\emph{Marionetten. Drei Einakter} {[}1906{]}|pw}}{\lemma{\textnormal{\emph{Marionetten}}}\Cendnote{\textnormal{am 10. 2. 1912 im Deutschen Volkstheater\oindex{Volkstheater@\textbf{Volkstheater}|pwk}}}}\label{K_L02053_1h}« ſind, die man ſpielt, würde ich wohl für
               den 1\textsuperscript{ten}{ }{\pb}oder 2\textsuperscript{ten} Abend \uline{2 Balconsitze} durch Sie beko{\geminationm}en können? würde mich ſehr freuen; vielleicht iſt es am
               einfachſten, Sie bezahlen ſie für mich und ſchicken mir ſie an die Adreſſe Eliſabethſtraße 6\oindex{Elisabethstrasse@\textbf{Elisabethstraße}|pw}.\pend
           \pstart
           Vielleicht ko{\geminationm}t die Bitte ſchon zu ſpät, dann gehe ich
               halt in eine ſpätere Vorstellung.\pend
           \pstart
           Herzlich{\\[\baselineskip]}\spacefill\mbox{Hugo.}\pend
           \leftskip=0em{}\endnumbering\briefempfaengerindex{Schnitzler, Arthur@\textsc{Schnitzler, Arthur}!zzzHofmannsthal, Hugo von@\emph{von Hugo von Hofmannsthal}!1912-02-071@{{[}7. 2. 1912{]}}|)be}\mylabel{h}\end{ledgroupsized}  \newcommand{\dateiname}{L02053}\newcommand{\titel}{Hugo von Hofmannsthal an Arthur Schnitzler, [7. 2. 1912]}\newcommand{\editorInnen}{Martin Anton Müller und Gerd-Hermann Susen}\input{../tex-inputs/latex-pdf-abspann}
      