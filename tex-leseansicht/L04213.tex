%% latex-leseansicht-vorspann.tex
%% Vorspann für die Leseansicht.
%% Lädt die gemeinsame Datei latex-vorspann.tex mit nicht gesetztem Schalter.

\newif\ifkorrekturansicht
\korrekturansichtfalse

\input{../tex-inputs/latex-vorspann}


\section[Arthur Schnitzler an Romain Rolland, 11. 2. 1915]{L04213 Arthur Schnitzler an Romain Rolland, 11. 2. 1915}
\nopagebreak\mylabel{L04213v}
\rehead{ }\normalsize\beginnumbering\briefempfaengerindex{Rolland, Romain@\textsc{Rolland, Romain}!zzzSchnitzler, Arthur@\emph{von Arthur Schnitzler}!1915-02-111@{11. 2. 1915}|(be}
\toendnotes[C]{\smallbreak\pagebreak[2]}
\correspDesc{Versand  durch Arthur Schnitzler am 11. 2. 1915 in Wien
\newline{}Übermittlung  am 12. 2. 1915 in Wien
\newline{}Erhalt  durch Romain Rolland am 15. 2. 1915 in Genf}\toendnotes[C]{\smallbreak}
\Standort{Paris, Bibliothèque Nationale de France, Fonds Romain Rolland, Cote NAF 28400.}
\physDesc{Briefkarte, , Kuvert, 813 Zeichen
\newline{}Handschrift: schwarze Tinte, lateinische Kurrent
\newline{}Versand: 1) Stempel: »\nobreak{}\oindex{IX., Alsergrund@\textbf{IX., Alsergrund}, \emph{Verwaltungsgebiet}|pwk}9/1 Wien 65, 12 II 15, 9\nobreak{}«.   2) Stempel: »\nobreak{}\oindex{Genf@\textbf{Genf}|pwk}Genève, 15 II 15, VII\nobreak{}«. 
\newline{}Rolland: mit schwarzer Tinte Datierung: »11/2/1915« und Vermerk: »\uline{ARL}« 
\newline{}Ordnung: 1) mit Bleistift Kuvert nummeriert: »3«  2) mit Bleistift Blätter (einschliesslich des Kuverts) paginiert:
                                    »7« – »8«}\toendnotes[C]{\smallbreak}\pstart{}{\pb}\textcolor{gray}{\textbf{Dr. Arthur Schnitzler}}\pend{}\pstart{}\textcolor{gray}{\textbf{Wien XVIII. Sternwartestrasse 71\oindex{Wien@\textbf{Wien}!XVIII., Währing@\textbf{XVIII., Währing}!Sternwartestraße 71@\textbf{Sternwartestraße 71}, \emph{Wohngebäude}|pw}}}\pend{}{\bigskip}\pstart{}{\pb}M Romain Rolland,\pend{}\pstart{}Genf\oindex{Genf@\textbf{Genf}|pw}\pend{}\pstart{}Hotel Beau-Séjour\oindex{Hôtel Beau-Séjour@\textbf{Hôtel Beau-Séjour}, \emph{Hotel}|pw}\pend{}{\bigskip}\vspace{1em}
\pstart
           {\pb}\textcolor{gray}{\textbf{Dr. Arthur Schnitzler}}\hfill 11. Feber 1915.\pend
           
\pstart
           \textcolor{gray}{\textbf{Wien XVIII. Sternwartestrasse 71\oindex{Wien@\textbf{Wien}!XVIII., Währing@\textbf{XVIII., Währing}!Sternwartestraße 71@\textbf{Sternwartestraße 71}, \emph{Wohngebäude}|pw}}}\pend
           \vspace{0.5em}
\pstart
           verehrter Herr Rolland, durch \label{K_L04215-1v}\edtext{Stefan Zweig\pwindex{Zweig, Stefan 28.\,11.\,1881 Wien – 23.\,2.\,1942 Petrópolis@\textsc{Zweig, Stefan} (28.\,11.\,1881 Wien – 23.\,2.\,1942 Petrópolis), \emph{Schriftsteller}|pw} empfang ich Ihre lieben
                  Grüße}{\lemma{\textnormal{\emph{Stefan … Grüße}}}\Cendnote{\textnormal{XXXX Auszeichnungsfehler: Dokument L03651 nicht gefunden.}}}\label{K_L04215-1}, die ich herzlichst
               erwidre. Sie haben also die Angriffe oder wenigstens von den Angriffen gelesen, die
               anläßlich meines von Ihnen so schön übersetzten Protestes gegen mich gerichtet worden
               sind. Übrigens standen sie nur in antisemitischen Blättern – und von dieser Seite
               bin ich dergleichen seit Jahren, ja seit Jahrzehnten, bei jeder möglichen und {\pb}unmöglichen Gelegenheit so reichlich gewohnt, daß sie mich vollko{\geminationm}en kalt lassen. Warum sollte der Krieg gerade auf diese
               traurige Menschensorte eine »läuternde« Wirkung ausüben, da doch auch anderswo nicht
               eben viel davon zu merken ist.\pend
           
\pstart
           Auf bessere Zeiten denn, und einen freundschaftlichen Händedruck Ihres sehr
               ergebenen {\\[\baselineskip]}\spacefill\mbox{Arthur Schnitzler}\pend
           \leftskip=0em{}\selectlanguage{ngerman}\endnumbering\briefempfaengerindex{Rolland, Romain@\textsc{Rolland, Romain}!zzzSchnitzler, Arthur@\emph{von Arthur Schnitzler}!1915-02-111@{11. 2. 1915}|)be}\mylabel{L04213h}
\begin{anhang}
\end{anhang}\newcommand{\dateiname}{L04213}\newcommand{\titel}{Arthur Schnitzler an Romain Rolland, 11. 2. 1915}\newcommand{\editorInnen}{Selma Jahnke und Martin Anton Müller}%% latex-leseansicht-abspann.tex
%% Abspann für die Leseansicht.
%% Der Schalter \ifkorrekturansicht ist bereits durch den Vorspann gesetzt.

%% latex-abspann.tex
%% Gemeinsamer Abspann für Korrekturansicht und Leseansicht.
%% Setzt den Schalter \ifkorrekturansicht voraus (gesetzt in den
%% einbindenden Dateien latex-korrekturansicht-abspann.tex bzw.
%% latex-leseansicht-abspann.tex).
%% ---------------------------------------------------------------

\normalsize

% Das esempio-Environment wird nur in der Leseansicht benötigt
\ifkorrekturansicht\else
\newenvironment{esempio}[3]%
{
    \vspace{1.5ex}
    \rlap{\underline{#1}}
    \par
    \setlength{\parindent}{0cm}
    \nopagebreak
    \leftskip=#2cm
    \rightskip=#3cm
}
{
    \par
}
\fi

\doendnotes{C}
\bigskip
\vfill

\clearpage

\footnotesize

\ifkorrekturansicht
  \lohead{\textsc{register}}
\fi

% theindex-Environment neu definieren ohne reledmac
\makeatletter
\renewenvironment{theindex}{%
  \ifkorrekturansicht
    \section*{\indexname}%
  \else
    \subsubsection*{Index der erwähnten Entitäten}%
  \fi
  \setlength{\parindent}{0pt}%
  \setlength{\parskip}{0pt plus 0.3pt}%
  \let\item\@idxitem
}{%
  \ifkorrekturansicht\clearpage\fi
}
\makeatother

\IfFileExists{\jobname-pw.ind}{\input{\jobname-pw.ind}}{}

% Quellenangabe nur in der Leseansicht
\ifkorrekturansicht\else
% Fallback-Definitionen, falls die .tex-Datei \titel etc. nicht gesetzt hat
\providecommand{\titel}{}
\providecommand{\editorInnen}{}
\providecommand{\dateiname}{\jobname}

\vspace{3cm}

\vfill

\footnotesize
\textsc{Quelle}: \titel. Herausgegeben von {\editorInnen}. In: \emph{Arthur Schnitzler: Briefwechsel mit Autorinnen und Autoren}.
 Digitale Edition, https://schnitzler-briefe.acdh.oeaw.ac.at/{\dateiname}.html (Stand \today)
\fi

\end{document}


