%% latex-korrekturansicht-vorspann.tex
%% Vorspann für die Korrekturansicht.
%% Lädt die gemeinsame Datei latex-vorspann.tex mit gesetztem Schalter.

\newif\ifkorrekturansicht
\korrekturansichttrue

\input{../tex-inputs/latex-vorspann}


\section[ Paul Goldmann an Arthur Schnitzler, 29. 5. {[}1900{]}]{L02917 Paul Goldmann an Arthur Schnitzler, 29. 5. {[}1900{]}}
\nopagebreak\mylabel{L02917v}
\rehead{ }\normalsize\beginnumbering\briefempfaengerindex{Schnitzler, Arthur@\textsc{Schnitzler, Arthur}!zzzGoldmann, Paul@\emph{von Paul Goldmann}!1900-05-291@{29. 5. {[}1900{]}}|(be}
\toendnotes[C]{\smallbreak\pagebreak[2]}\Standort{DLA, A:Schnitzler, HS.NZ85.1.3170.}
\physDesc{Brief, 1 Blatt, 4 Seiten, 1862 Zeichen
\newline{}Handschrift: blaue Tinte, deutsche Kurrent
\newline{}Schnitzler: 1) mit Bleistift das Jahr »900« vermerkt  2) mit rotem Buntstift drei Unterstreichungen}\toendnotes[C]{\smallbreak}
\pstart
           
\pstart
           {\pb}\textcolor{gray}{\textbf{DESSAUERSTRASSE 19}}\oindex{Dessauer Strasse@\textbf{Dessauer Straße}, \emph{Straße (K.STR)}|pw}\pend
           
\pstart
           \raggedleft{}Berlin\oindex{Berlin@\textbf{Berlin}, \emph{P.PPLC}|pw}, 29. Mai.\pend
           \pend
           
\pstart\center{}Mein lieber Freund,\pend\vspace{0.5em}
\pstart
           Unſere Briefe haben ſich wieder einmal gekreuzt. Es iſt ſchön, daß Du in den Bergen\oindex{Alpen@\textbf{Alpen}, \emph{kein passender Code gefunden}|pwv} biſt, in guter Luft und
               in Ruhe. Wie der \label{K_L02917-1v}\edtext{Ort\oindex{Puchberg am Schneeberg@\textbf{Puchberg am Schneeberg}, \emph{P.PPLA3}|pwv} am Fuße des Schneebergs\oindex{Schneeberg@\textbf{Schneeberg}, \emph{Berg (N.BRG)}|pw}}{\lemma{\textnormal{\emph{Ort … Schneebergs}}}\Cendnote{\textnormal{Es handelte sich wohl um Puchberg am Schneeberg\oindex{Puchberg am Schneeberg@\textbf{Puchberg am Schneeberg}, \emph{P.PPLA3}|pwk}. Schnitzler
                  hielt sich dort vom 24. 5. 1900 bis zum 27. 5. 1900 auf.}}}\label{K_L02917-1} heißt, habe ich nicht enziffern können. Über
                  \label{K_L02917-2v}\edtext{\textsc{Schlenther\pwindex{Schlenther, Paul 20.08.1854 – 30.04.1916@\textsc{Schlenther, Paul} (20.08.1854 – 30.04.1916), \emph{Schriftsteller/Schriftstellerin, Kritiker/Kritikerin, Theaterleiter/Theaterleiterin}|pw}}}{\lemma{\textnormal{\emph{Schlenther}}}\Cendnote{\textnormal{Schnitzler hatte geglaubt, durch die Erfüllung aller Forderungen
                  von Schlenther\pwindex{Schlenther, Paul 20.08.1854 – 30.04.1916@\textsc{Schlenther, Paul} (20.08.1854 – 30.04.1916), \emph{Schriftsteller/Schriftstellerin, Kritiker/Kritikerin, Theaterleiter/Theaterleiterin}|pwk} die Annahme von \emph{Der Schleier der Beatrice}\pwindex{Schleier der Beatrice. Schauspiel in fuenf Akten@\emph{Der Schleier der Beatrice. Schauspiel in fünf Akten}|pwk}
                  erwirkt zu haben. Schlenther\pwindex{Schlenther, Paul 20.08.1854 – 30.04.1916@\textsc{Schlenther, Paul} (20.08.1854 – 30.04.1916), \emph{Schriftsteller/Schriftstellerin, Kritiker/Kritikerin, Theaterleiter/Theaterleiterin}|pwk} zögerte, und von Schnitzler
                  zunehmend unter Druck gesetzt, gab er das Stück im September 1900 zurück. Vgl. Hermann Bahr, Arthur Schnitzler: \emph{Briefwechsel, Aufzeichnungen, Dokumente (1891–1931)}, Hermann Bahr, Julius Bauer, J. J. David, Robert Hirschfeld, Felix Salten, Ludwig Speidel: Erklärung, 14. 9. 1900.}}}\label{K_L02917-2} ärgere Dich nicht. Aufführen muß er Dich\pwindex{Schleier der Beatrice. Schauspiel in fuenf Akten@\emph{Der Schleier der Beatrice. Schauspiel in fünf Akten}|pwv} ja doch, ob er will oder nicht. \strikeout{Üb} Im Übrigen iſt er ein erbärmlicher Kerl und wird
                  \label{K_L02917-3v}\edtext{nicht mehr lange das Burgtheater\orgindex{Burgtheater@Burgtheater|pw} dirigiren}{\lemma{\textnormal{\emph{nicht … dirigiren}}}\Cendnote{\textnormal{Paul Schlenther\pwindex{Schlenther, Paul 20.08.1854 – 30.04.1916@\textsc{Schlenther, Paul} (20.08.1854 – 30.04.1916), \emph{Schriftsteller/Schriftstellerin, Kritiker/Kritikerin, Theaterleiter/Theaterleiterin}|pwk} blieb bis 1910 Direktor des \emph{Burgtheaters}\orgindex{Burgtheater@Burgtheater|pwk}.}}}\label{K_L02917-3}. Daß \textsc{Brahm\pwindex{Brahm, Otto 05.02.1856 – 28.11.1912@\textsc{Brahm, Otto} (05.02.1856 – 28.11.1912), \emph{Theaterleiter/Theaterleiterin, Regisseur/Regisseurin}|pw}}{ }Dich\pwindex{Schleier der Beatrice. Schauspiel in fuenf Akten@\emph{Der Schleier der Beatrice. Schauspiel in fünf Akten}|pwv} bisher
               nicht aufgeführt hat, iſt begreiflich. Er iſt ein Geſchäftsmann und will zuerſt ſeine
               neuen Stücke bringen, die beſſere {\pb}Einnahmen
               verſprechen, als die ſchon bekannten.\pend
           
\pstart
           Ich habe jetzt wieder eine Zeit relativer Ruhe, könnte für mich arbeiten, zermartere
               mir den Kopf und bringe nicht \uline{einen} Gedanken heraus.
               Das verſtimmt mich tief. Ich bin eben offenbar doch nur ein Journaliſt\strikeout{,} und habe kein Recht zu höheren Prätentionen.\pend
           
\pstart
           Der Leiter\pwindex{Freund, Erich 1866-08-13 – 1940@\textsc{Freund, Erich} (1866-08-13 – 1940), \emph{Kritiker/Kritikerin, Musikjournalist/Musikjournalistin}|pwv} der Breslau\oindex{Breslau@\textbf{Breslau}, \emph{P.PPLA}|pw}er Freien
                  Literariſchen Vereinigung\orgindex{Freie literarische Vereinigung zu Breslau@Freie literarische Vereinigung zu Breslau|pw}, \textsc{Dr. Erich Freund\pwindex{Freund, Erich 1866-08-13 – 1940@\textsc{Freund, Erich} (1866-08-13 – 1940), \emph{Kritiker/Kritikerin, Musikjournalist/Musikjournalistin}|pw}}, der, wie Du weißt, ein Jugendfreund von mir iſt, weilt gegenwärtig in Berlin\oindex{Berlin@\textbf{Berlin}, \emph{P.PPLC}|pw} und hat mich gebeten, Dich {\pb}zu fragen, ob Du nicht in dieſem Winter einmal
                  \label{K_L02917-4v}\edtext{in Breslau\oindex{Breslau@\textbf{Breslau}, \emph{P.PPLA}|pw} leſen}{\lemma{\textnormal{\emph{in Breslau leſen}}}\Cendnote{\textnormal{In dieser Saison
                  wurde das nicht umgesetzt. Am 31. 12. 1905 trafen sich jedoch Freund\pwindex{Freund, Erich 1866-08-13 – 1940@\textsc{Freund, Erich} (1866-08-13 – 1940), \emph{Kritiker/Kritikerin, Musikjournalist/Musikjournalistin}|pwk} und Schnitzler, um über eine
                  solche Lesung zu sprechen, die dann am 22. 1. 1906 stattfand.}}}\label{K_L02917-4} möchteſt? Die Leute\orgindex{Freie literarische Vereinigung zu Breslau@Freie literarische Vereinigung zu Breslau|pwv} haben ein ſehr vornehmes
               Vortrags-Programm, zahlen von 150 \textsc{MK} aufwärts und wären
               ſehr glücklich, Dich einmal zu haben.\pend
           
\pstart
           Sommerpläne? Wie ich Dir ſchon geſchrieben habe: Ich wüßte mir natürlich nichts
               Beſſeres, als mit Dir und \textsc{Richard\pwindex{Beer-Hofmann, Richard 1866-07-11 – 1945-09-26@\textsc{Beer-Hofmann, Richard} (1866-07-11 – 1945-09-26), \emph{Schriftsteller/Schriftstellerin}|pw}} zuſammen zu ſein, aber ich werde kein Geld haben. Meine Haushalt-Ausgaben
               laufen fort, ob ich hier bin oder nicht, meine Mutter\pwindex{Goldmann, Clementine 1842-05-15 – 1924-02-24@\textsc{Goldmann, Clementine} (1842-05-15 – 1924-02-24)|pwv} muß aufs Land, endlich muß ich, wenn ich hier\oindex{Berlin@\textbf{Berlin}, \emph{P.PPLC}|pwv}{ }{\pb}weggehe, mir einen Vertreter zahlen. Es iſt ſehr
               lieb von Dir, daß Du mir etwas borgen willſt. Aber ich ſehe keine Möglichkeit, wie
               ich Dir das wiedergeben ſoll, und überdies ſchulde ich Dir noch 100 \textsc{Kronen} von \label{K_L02917-5v}\edtext{Kopenhagen\oindex{Kopenhagen@\textbf{Kopenhagen}, \emph{P.PPLC}|pw}}{\lemma{\textnormal{\emph{Kopenhagen}}}\Cendnote{\textnormal{Bezug auf die gemeinsame Dänemark\oindex{Daenemark@\textbf{Dänemark}, \emph{A.PCLI}|pwk}-Reise im Sommer
                     1896, siehe Paul Goldmann an Arthur Schnitzler, 7. 9. [1896].}}}\label{K_L02917-5} her. Wenn alſo bis zum Auguſt nicht ein Wunder
               geſchieht, werde ich in Berlin\oindex{Berlin@\textbf{Berlin}, \emph{P.PPLC}|pw} bleiben
               müſſen.\pend
           
\pstart
           Schreib’ mir bald und ſei von Herzen gegrüßt!\pend
           
\pstart
           Dein treuer {\\[\baselineskip]}\spacefill\mbox{Paul Goldmann.}\pend
           \leftskip=0em{}\selectlanguage{ngerman}\endnumbering\briefempfaengerindex{Schnitzler, Arthur@\textsc{Schnitzler, Arthur}!zzzGoldmann, Paul@\emph{von Paul Goldmann}!1900-05-291@{29. 5. {[}1900{]}}|)be}\mylabel{L02917h}  \normalsize

\doendnotes{C}
\bigskip
\vfill

\clearpage

\footnotesize

\lohead{\textsc{register}}

% Definiere theindex-Environment komplett neu ohne reledmac
\makeatletter
\renewenvironment{theindex}{%
  \section*{\indexname}%
  \setlength{\parindent}{0pt}%
  \setlength{\parskip}{0pt plus 0.3pt}%
  \let\item\@idxitem
}{%
  \clearpage
}
\makeatother

\IfFileExists{\jobname-pw.ind}{\input{\jobname-pw.ind}}{}

\end{document}

      