%% latex-leseansicht-vorspann.tex
%% Vorspann für die Leseansicht.
%% Lädt die gemeinsame Datei latex-vorspann.tex mit nicht gesetztem Schalter.

\newif\ifkorrekturansicht
\korrekturansichtfalse

\input{../tex-inputs/latex-vorspann}


\section[ Paul Goldmann an Arthur Schnitzler, 29. 5. [1900]]{L02917 Paul Goldmann an Arthur Schnitzler,  29. 5. [1900]}
\nopagebreak\mylabel{L02917v}
\rehead{ }\normalsize\beginnumbering\briefempfaengerindex{Schnitzler, Arthur@\textsc{Schnitzler, Arthur}!zzzGoldmann, Paul@\emph{von Paul Goldmann}!1900-05-291@{29. 5. [1900]}|(be}
\toendnotes[C]{\smallbreak\pagebreak[2]}
\correspDesc{Versand  durch Paul Goldmann am 29. 5. [1900] in Berlin
\newline{}Erhalt  durch Arthur Schnitzler am [1. 6. 1900?] in Wien}\toendnotes[C]{\smallbreak}
\Standort{DLA, A:Schnitzler, HS.NZ85.1.3170.}
\physDesc{Brief, 1 Blatt, 4 Seiten, 1862 Zeichen
\newline{}Handschrift: blaue Tinte, deutsche Kurrent
\newline{}Schnitzler: 1) mit Bleistift das Jahr »900« vermerkt  2) mit rotem Buntstift drei Unterstreichungen}\toendnotes[C]{\smallbreak}
\pstart
           
\pstart
           {\pb}\textcolor{gray}{\textbf{DESSAUERSTRASSE 19}}\oindex{Dessauer Straße@\textbf{Dessauer Straße}, \emph{Straße}|pw}\pend
           
\pstart
           \raggedleft{}Berlin\oindex{Berlin@\textbf{Berlin}, \emph{Hauptstadt}|pw}, 29. Mai.\pend
           \pend
           
\pstart\center{}Mein lieber Freund,\pend\vspace{0.5em}
\pstart
           Unſere Briefe haben{ }ſich wieder einmal gekreuzt. Es iſt{ }ſchön, daß Du in den Bergen\oindex{Alpen@\textbf{Alpen}|pwv} biſt, in guter Luft und
               in Ruhe. Wie der \label{K_L02917-1v}\edtext{Ort\oindex{Puchberg am Schneeberg@\textbf{Puchberg am Schneeberg}, \emph{Hauptstadt}|pwv} am Fuße des Schneebergs\oindex{Schneeberg@\textbf{Schneeberg}, \emph{Berg}|pw}}{\lemma{\textnormal{\emph{Ort … Schneebergs}}}\Cendnote{\textnormal{Es handelte sich wohl um Puchberg am Schneeberg\oindex{Puchberg am Schneeberg@\textbf{Puchberg am Schneeberg}, \emph{Hauptstadt}|pwk}. Schnitzler
                  hielt sich dort vom 24. 5. 1900 bis zum 27. 5. 1900 auf.}}}\label{K_L02917-1} heißt, habe ich nicht enziffern können. Über
                  \label{K_L02917-2v}\edtext{\textsc{Schlenther\pwindex{Schlenther, Paul 20.\,8.\,1854 Chernyakhovsk – 30.\,4.\,1916 Berlin@\textsc{Schlenther, Paul} (20.\,8.\,1854 Chernyakhovsk – 30.\,4.\,1916 Berlin), \emph{Schriftsteller, Kritiker, Theaterleiter}|pw}}}{\lemma{\textnormal{\emph{Schlenther}}}\Cendnote{\textnormal{Schnitzler hatte geglaubt, durch die Erfüllung aller Forderungen
                  von Schlenther\pwindex{Schlenther, Paul 20.\,8.\,1854 Chernyakhovsk – 30.\,4.\,1916 Berlin@\textsc{Schlenther, Paul} (20.\,8.\,1854 Chernyakhovsk – 30.\,4.\,1916 Berlin), \emph{Schriftsteller, Kritiker, Theaterleiter}|pwk} die Annahme von \emph{Der Schleier der Beatrice}\pwindex{Schnitzler, Arthur 15.\,5.\,1862 Wien – 21.\,10.\,1931 ebd.@\textsc{Schnitzler, Arthur} (15.\,5.\,1862 Wien – 21.\,10.\,1931 ebd.), \emph{Schriftsteller, Mediziner}!Schleier der Beatrice. Schauspiel in fünf Akten@\strich\emph{Der Schleier der Beatrice. Schauspiel in fünf Akten}|pwk}
                  erwirkt zu haben. Schlenther\pwindex{Schlenther, Paul 20.\,8.\,1854 Chernyakhovsk – 30.\,4.\,1916 Berlin@\textsc{Schlenther, Paul} (20.\,8.\,1854 Chernyakhovsk – 30.\,4.\,1916 Berlin), \emph{Schriftsteller, Kritiker, Theaterleiter}|pwk} zögerte, und von Schnitzler
                  zunehmend unter Druck gesetzt, gab er das Stück im September 1900 zurück. Vgl. Hermann Bahr, Arthur Schnitzler: \emph{Briefwechsel, Aufzeichnungen, Dokumente (1891–1931)}, Hermann Bahr, Julius Bauer, J. J. David, Robert Hirschfeld, Felix Salten, Ludwig Speidel: Erklärung, 14. 9. 1900.}}}\label{K_L02917-2} ärgere Dich nicht. Aufführen muß er Dich\pwindex{Schnitzler, Arthur 15.\,5.\,1862 Wien – 21.\,10.\,1931 ebd.@\textsc{Schnitzler, Arthur} (15.\,5.\,1862 Wien – 21.\,10.\,1931 ebd.), \emph{Schriftsteller, Mediziner}!Schleier der Beatrice. Schauspiel in fünf Akten@\strich\emph{Der Schleier der Beatrice. Schauspiel in fünf Akten}|pwv} ja doch, ob er will oder nicht. \strikeout{Üb} Im Übrigen iſt er ein erbärmlicher Kerl und wird
                  \label{K_L02917-3v}\edtext{nicht mehr lange das Burgtheater\orgindex{Burgtheater@Burgtheater|pw} dirigiren}{\lemma{\textnormal{\emph{nicht … dirigiren}}}\Cendnote{\textnormal{Paul Schlenther\pwindex{Schlenther, Paul 20.\,8.\,1854 Chernyakhovsk – 30.\,4.\,1916 Berlin@\textsc{Schlenther, Paul} (20.\,8.\,1854 Chernyakhovsk – 30.\,4.\,1916 Berlin), \emph{Schriftsteller, Kritiker, Theaterleiter}|pwk} blieb bis 1910 Direktor des \emph{Burgtheaters}\orgindex{Burgtheater@Burgtheater|pwk}.}}}\label{K_L02917-3}. Daß \textsc{Brahm\pwindex{Brahm, Otto 5.\,2.\,1856 Hamburg – 28.\,11.\,1912 Berlin@\textsc{Brahm, Otto} (5.\,2.\,1856 Hamburg – 28.\,11.\,1912 Berlin), \emph{Theaterleiter, Regisseur}|pw}}{ }Dich\pwindex{Schnitzler, Arthur 15.\,5.\,1862 Wien – 21.\,10.\,1931 ebd.@\textsc{Schnitzler, Arthur} (15.\,5.\,1862 Wien – 21.\,10.\,1931 ebd.), \emph{Schriftsteller, Mediziner}!Schleier der Beatrice. Schauspiel in fünf Akten@\strich\emph{Der Schleier der Beatrice. Schauspiel in fünf Akten}|pwv} bisher
               nicht aufgeführt hat, iſt begreiflich. Er iſt ein Geſchäftsmann und will zuerſt{ }ſeine
               neuen Stücke bringen, die beſſere {\pb}Einnahmen
               verſprechen, als die{ }ſchon bekannten.\pend
           
\pstart
           Ich habe jetzt wieder eine Zeit relativer Ruhe, könnte für mich arbeiten, zermartere
               mir den Kopf und bringe nicht \uline{einen} Gedanken heraus.
               Das verſtimmt mich tief. Ich bin eben offenbar doch nur ein Journaliſt\strikeout{,} und habe kein Recht zu höheren Prätentionen.\pend
           
\pstart
           Der Leiter\pwindex{Freund, Erich 13.\,8.\,1866 Breslau – 1940 Berlin@\textsc{Freund, Erich} (13.\,8.\,1866 Breslau – 1940 Berlin), \emph{Kritiker, Musikjournalist}|pwv} der Breslau\oindex{Breslau@\textbf{Breslau}|pw}er Freien
                  Literariſchen Vereinigung\orgindex{Freie literarische Vereinigung zu Breslau@Freie literarische Vereinigung zu Breslau|pw}, \textsc{Dr. Erich Freund\pwindex{Freund, Erich 13.\,8.\,1866 Breslau – 1940 Berlin@\textsc{Freund, Erich} (13.\,8.\,1866 Breslau – 1940 Berlin), \emph{Kritiker, Musikjournalist}|pw}}, der, wie Du weißt, ein Jugendfreund von mir iſt, weilt gegenwärtig in Berlin\oindex{Berlin@\textbf{Berlin}, \emph{Hauptstadt}|pw} und hat mich gebeten, Dich {\pb}zu fragen, ob Du nicht in dieſem Winter einmal
                  \label{K_L02917-4v}\edtext{in Breslau\oindex{Breslau@\textbf{Breslau}|pw} leſen}{\lemma{\textnormal{\emph{in Breslau lesen}}}\Cendnote{\textnormal{In dieser Saison
                  wurde das nicht umgesetzt. Am 31. 12. 1905 trafen sich jedoch Freund\pwindex{Freund, Erich 13.\,8.\,1866 Breslau – 1940 Berlin@\textsc{Freund, Erich} (13.\,8.\,1866 Breslau – 1940 Berlin), \emph{Kritiker, Musikjournalist}|pwk} und Schnitzler, um über eine
                  solche Lesung zu sprechen, die dann am 22. 1. 1906 stattfand.}}}\label{K_L02917-4} möchteſt? Die Leute\orgindex{Freie literarische Vereinigung zu Breslau@Freie literarische Vereinigung zu Breslau|pwv} haben ein{ }ſehr vornehmes
               Vortrags-Programm, zahlen von 150 \textsc{MK} aufwärts und wären{ }ſehr glücklich, Dich einmal zu haben.\pend
           
\pstart
           Sommerpläne? Wie ich Dir{ }ſchon geſchrieben habe: Ich wüßte mir natürlich nichts
               Beſſeres, als mit Dir und \textsc{Richard\pwindex{Beer-Hofmann, Richard 11.\,7.\,1866 Wien – 26.\,9.\,1945 New York City@\textsc{Beer-Hofmann, Richard} (11.\,7.\,1866 Wien – 26.\,9.\,1945 New York City), \emph{Schriftsteller}|pw}} zuſammen zu{ }ſein, aber ich werde kein Geld haben. Meine Haushalt-Ausgaben
               laufen fort, ob ich hier bin oder nicht, meine Mutter\pwindex{Goldmann, Clementine 15.\,5.\,1842 Breslau – 24.\,2.\,1924 Frankfurt am Main@\textsc{Goldmann, Clementine} (15.\,5.\,1842 Breslau – 24.\,2.\,1924 Frankfurt am Main)|pwv} muß aufs Land, endlich muß ich, wenn ich hier\oindex{Berlin@\textbf{Berlin}, \emph{Hauptstadt}|pwv}{ }{\pb}weggehe, mir einen Vertreter zahlen. Es iſt{ }ſehr
               lieb von Dir, daß Du mir etwas borgen willſt. Aber ich{ }ſehe keine Möglichkeit, wie
               ich Dir das wiedergeben{ }ſoll, und überdies{ }ſchulde ich Dir noch 100 \textsc{Kronen} von \label{K_L02917-5v}\edtext{Kopenhagen\oindex{Kopenhagen@\textbf{Kopenhagen}, \emph{Hauptstadt}|pw}}{\lemma{\textnormal{\emph{Kopenhagen}}}\Cendnote{\textnormal{Bezug auf die gemeinsame Dänemark\oindex{Dänemark@\textbf{Dänemark}|pwk}-Reise im Sommer 1896, siehe XXXX Auszeichnungsfehler: Dokument L02784 nicht gefunden.}}}\label{K_L02917-5} her. Wenn alſo bis zum Auguſt nicht ein Wunder
               geſchieht, werde ich in Berlin\oindex{Berlin@\textbf{Berlin}, \emph{Hauptstadt}|pw} bleiben
               müſſen.\pend
           
\pstart
           Schreib’ mir bald und{ }ſei von Herzen gegrüßt!\pend
           
\pstart
           Dein treuer {\\[\baselineskip]}\spacefill\mbox{Paul Goldmann.}\pend
           \leftskip=0em{}\selectlanguage{ngerman}\endnumbering\briefempfaengerindex{Schnitzler, Arthur@\textsc{Schnitzler, Arthur}!zzzGoldmann, Paul@\emph{von Paul Goldmann}!1900-05-291@{29. 5. [1900]}|)be}\mylabel{L02917h}  \newcommand{\dateiname}{L02917}\newcommand{\titel}{Paul Goldmann an Arthur Schnitzler, 29. 5. [1900]}\newcommand{\editorInnen}{Martin Anton Müller und Laura Untner}%% latex-leseansicht-abspann.tex
%% Abspann für die Leseansicht.
%% Der Schalter \ifkorrekturansicht ist bereits durch den Vorspann gesetzt.

%% latex-abspann.tex
%% Gemeinsamer Abspann für Korrekturansicht und Leseansicht.
%% Setzt den Schalter \ifkorrekturansicht voraus (gesetzt in den
%% einbindenden Dateien latex-korrekturansicht-abspann.tex bzw.
%% latex-leseansicht-abspann.tex).
%% ---------------------------------------------------------------

\normalsize

% Das esempio-Environment wird nur in der Leseansicht benötigt
\ifkorrekturansicht\else
\newenvironment{esempio}[3]%
{
    \vspace{1.5ex}
    \rlap{\underline{#1}}
    \par
    \setlength{\parindent}{0cm}
    \nopagebreak
    \leftskip=#2cm
    \rightskip=#3cm
}
{
    \par
}
\fi

\doendnotes{C}
\bigskip
\vfill

\clearpage

\footnotesize

\ifkorrekturansicht
  \lohead{\textsc{register}}
\fi

% theindex-Environment neu definieren ohne reledmac
\makeatletter
\renewenvironment{theindex}{%
  \ifkorrekturansicht
    \section*{\indexname}%
  \else
    \subsubsection*{Index der erwähnten Entitäten}%
  \fi
  \setlength{\parindent}{0pt}%
  \setlength{\parskip}{0pt plus 0.3pt}%
  \let\item\@idxitem
}{%
  \ifkorrekturansicht\clearpage\fi
}
\makeatother

\IfFileExists{\jobname-pw.ind}{\input{\jobname-pw.ind}}{}

% Quellenangabe nur in der Leseansicht
\ifkorrekturansicht\else
% Fallback-Definitionen, falls die .tex-Datei \titel etc. nicht gesetzt hat
\providecommand{\titel}{}
\providecommand{\editorInnen}{}
\providecommand{\dateiname}{\jobname}

\vspace{3cm}

\vfill

\footnotesize
\textsc{Quelle}: \titel. Herausgegeben von {\editorInnen}. In: \emph{Arthur Schnitzler: Briefwechsel mit Autorinnen und Autoren}.
 Digitale Edition, https://schnitzler-briefe.acdh.oeaw.ac.at/{\dateiname}.html (Stand \today)
\fi

\end{document}


