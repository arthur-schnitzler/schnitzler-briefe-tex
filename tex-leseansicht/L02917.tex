%% latex-leseansicht-vorspann.tex
%% Vorspann für die Leseansicht.
%% Lädt die gemeinsame Datei latex-vorspann.tex mit nicht gesetztem Schalter.

\newif\ifkorrekturansicht
\korrekturansichtfalse

\input{../tex-inputs/latex-vorspann}

\begin{center}
            \textcolor{red}{ENTWURF, NICHT FERTIG KORRIGIERT}
                      \end{center}
            
         
         \renewcommand{\erwaehntePersonen}{Personen: Richard Beer-Hofmann, Otto Brahm, Erich Freund, Clementine Goldmann, Paul Schlenther}
         \renewcommand{\erwaehnteInstitutionen}{Institutionen: Burgtheater, Freie literarische Vereinigung zu Breslau}
         \renewcommand{\erwaehnteOrte}{Orte: Alpen, Aussichtsturm Merkur, Berlin, Breslau, Dessauer Straße, Dänemark, Kopenhagen, Puchberg am Schneeberg, Schneeberg}
         \renewcommand{\erwaehnteWerke}{Werke: Der Schleier der Beatrice. Schauspiel in fünf Akten}
               \section[ Paul Goldmann an Arthur Schnitzler, 29. 5. {[}1900{]}]{ Paul Goldmann an Arthur Schnitzler, 29. 5. {[}1900{]}}\nopagebreak\mylabel{v}\rehead{ }\begin{ledgroupsized}[t]{13cm}\normalsize\beginnumbering \toendnotes[C]{\smallbreak\pagebreak[2]} \Standort{DLA, A:Schnitzler, HS.NZ85.1.3170.}
\physDesc{Brief, 1 Blatt, 4 Seiten
\newline{}Handschrift: blaue Tinte, deutsche Kurrent
\newline{}Schnitzler: 1) mit Bleistift das Jahr »{[}1{]}900« vermerkt  2) mit rotem Buntstift drei Unterstreichungen}\toendnotes[C]{\smallbreak}\pstart
           \raggedleft{}{\pb}Berlin\oindex{Berlin@\textbf{Berlin}|pw}, 29. Mai.\pend
           \pstart
           \textcolor{gray}{\textbf{DESSAUERSTRASSE 19}}\oindex{Dessauer Strasse@\textbf{Dessauer Straße}|pw}\pend
           \pstart\center{}Mein lieber Freund,\pend\pstart
           Unſere Briefe haben ſich wieder einmal gekreuzt. Es iſt ſchön, daß Du in den Bergen\oindex{Alpen@\textbf{Alpen}|pwv} biſt, in guter Luft und
               in Ruhe. Wie der \label{K_L02917-2v}\edtext{Ort\oindex{Puchberg am Schneeberg@\textbf{Puchberg am Schneeberg}|pwv} am Fuße des Schneeberg\oindex{Schneeberg@\textbf{Schneeberg}|pw}s}{\lemma{\textnormal{\emph{Ort … Schneebergs}}}\Cendnote{\textnormal{Es handelte sich wohl um Puchberg am Schneeberg\oindex{Puchberg am Schneeberg@\textbf{Puchberg am Schneeberg}|pwk}. Schnitzler\pwindex{Schnitzler, Arthur 15.05.1862 – 21.10.1931@\textsc{Schnitzler, Arthur} (15.05.1862 – 21.10.1931), \emph{Schriftsteller, Mediziner}|pwk}
                  hielt sich dort von 24. 5. 1900 bis 27. 5. 1900 auf.}}}\label{K_L02917-2h} heißt, habe ich nicht enziffern können. Über
                  \label{K_L02917-1v}\edtext{\textsc{Schlenther\pwindex{Schlenther, Paul 20.08.1854 – 30.04.1916@\textsc{Schlenther, Paul} (20.08.1854 – 30.04.1916), \emph{Schriftsteller, Kritiker, Theaterleiter}|pw}}}{\lemma{\textnormal{\emph{Schlenther}}}\Cendnote{\textnormal{höchstwahrscheinlich Bezug auf Schlenther\pwindex{Schlenther, Paul 20.08.1854 – 30.04.1916@\textsc{Schlenther, Paul} (20.08.1854 – 30.04.1916), \emph{Schriftsteller, Kritiker, Theaterleiter}|pwk}s leeres Versprechen zur
                  Uraufführung von \emph{Der Schleier der Beatrice}\pwindex{Schnitzler, Arthur 15.05.1862 – 21.10.1931@\textsc{Schnitzler, Arthur} (15.05.1862 – 21.10.1931), \emph{Schriftsteller, Mediziner}!Schleier der Beatrice. Schauspiel in fuenf Akten1900-12-01@\strich\emph{Der Schleier der Beatrice. Schauspiel in fünf Akten} {[}1900-12-01{]}|pwk}
                     (siehe Paul Goldmann an Arthur Schnitzler, 12. 11. [1899])}}}\label{K_L02917-1h} ärgere
               Dich nicht. Aufführen muß er Dich\pwindex{Schnitzler, Arthur 15.05.1862 – 21.10.1931@\textsc{Schnitzler, Arthur} (15.05.1862 – 21.10.1931), \emph{Schriftsteller, Mediziner}!Schleier der Beatrice. Schauspiel in fuenf Akten1900-12-01@\strich\emph{Der Schleier der Beatrice. Schauspiel in fünf Akten} {[}1900-12-01{]}|pwv} ja doch, ob er will oder nicht. \strikeout{Üb} Im Übrigen iſt er ein erbärmlicher Kerl und wird
                  \label{K_L02917-3v}\edtext{nicht mehr lange das Burgtheater\orgindex{Burgtheater@Burgtheater|pw} dirigiren}{\lemma{\textnormal{\emph{nicht … dirigiren}}}\Cendnote{\textnormal{Paul Schlenther\pwindex{Schlenther, Paul 20.08.1854 – 30.04.1916@\textsc{Schlenther, Paul} (20.08.1854 – 30.04.1916), \emph{Schriftsteller, Kritiker, Theaterleiter}|pwk} blieb bis 1910 Direktor des \emph{Burgtheater}\orgindex{Burgtheater@Burgtheater|pwk}s.}}}\label{K_L02917-3h}. Daß \textsc{Brahm\pwindex{Brahm, Otto 05.02.1856 – 28.11.1912@\textsc{Brahm, Otto} (05.02.1856 – 28.11.1912), \emph{Theaterleiter, Regisseur}|pw}}{ }Dich\pwindex{Schnitzler, Arthur 15.05.1862 – 21.10.1931@\textsc{Schnitzler, Arthur} (15.05.1862 – 21.10.1931), \emph{Schriftsteller, Mediziner}!Schleier der Beatrice. Schauspiel in fuenf Akten1900-12-01@\strich\emph{Der Schleier der Beatrice. Schauspiel in fünf Akten} {[}1900-12-01{]}|pwv} bisher
               nicht aufgeführt hat, iſt begreiflich! Er iſt ein Geſchäftsmann und will zuerſt ſeine
               neuen Stücke bringen, die beſſere {\pb}Einnahmen
               verſprechen, als die ſchon bekannten.\pend
           \pstart
           Ich habe jetzt wieder eine Zeit relativer Ruhe, konnte für mich arbeiten, zermartere
               mir den Kopf und bringe nicht \uline{einen} Gedanken heraus.
               Das verſtimmt mich tief. Ich bin eben offenbar doch nur ein Journaliſt\strikeout{,} und habe kein Recht zu höheren Prätentionen.\pend
           \pstart
           Der Leiter\pwindex{Freund, Erich 1866-08-13 – 1940@\textsc{Freund, Erich} (1866-08-13 – 1940), \emph{Kritiker, Musikjournalist}|pwv} der Breslau\oindex{Breslau@\textbf{Breslau}|pw}er Freien
                  Literariſchen Vereinigung\orgindex{Freie literarische Vereinigung zu Breslau@Freie literarische Vereinigung zu Breslau|pw}, \textsc{Dr. Erich Freund\pwindex{Freund, Erich 1866-08-13 – 1940@\textsc{Freund, Erich} (1866-08-13 – 1940), \emph{Kritiker, Musikjournalist}|pw}}, der, wie Du weißt, ein Jugendfreund von mir iſt, weilt gegenwärtig in Berlin\oindex{Berlin@\textbf{Berlin}|pw} und hat mich gebeten, Dich {\pb}zu fragen, ob Du nicht in dieſem Winter einmal
                  \label{K_L02917-4v}\edtext{in Breslau\oindex{Breslau@\textbf{Breslau}|pw} leſen}{\lemma{\textnormal{\emph{in Breslau leſen}}}\Cendnote{\textnormal{nicht
                  geschehen}}}\label{K_L02917-4h} möchteſt? Die Leute\orgindex{Freie literarische Vereinigung zu Breslau@Freie literarische Vereinigung zu Breslau|pwv} haben ein ſehr vornehmes Vortrags-Programm, zahlen von 150 \textsc{MK} aufwärts und wären ſehr glücklich, Dich einmal zu
               haben.\pend
           \pstart
           Sommerpläne? Wie ich Dir ſchon geſchrieben habe: Ich wüßte mir natürlich nichts
               Beſſeres, als mit Dir und \textsc{Richard\pwindex{Beer-Hofmann, Richard 1866-07-11 – 1945-09-26@\textsc{Beer-Hofmann, Richard} (1866-07-11 – 1945-09-26), \emph{Schriftsteller}|pw}} zuſammen zu ſein, aber ich werde kein Geld haben. Meine Haushalt-Ausgaben
               laufen fort, ob ich hier bin oder nicht, meine Mutter\pwindex{Goldmann, Clementine 1842-05-15 – 1924-02-24@\textsc{Goldmann, Clementine} (1842-05-15 – 1924-02-24)|pwv} muß aufs Land, endlich muß ich, wenn ich hier\oindex{Berlin@\textbf{Berlin}|pwv}{ }{\pb}weggehe, mir einen Vertreter zahlen. Es iſt ſehr
               lieb von Dir, daß Du mir etwas borgen willſt. Aber ich ſehe keine Möglichkeit, wie
               ich Dir das wiedergeben ſoll, und überdies ſchulde ich Dir noch 100 \textsc{Kronen} von \label{K_L02917-5v}\edtext{\textsc{Kopenhagen\oindex{Kopenhagen@\textbf{Kopenhagen}|pw}}}{\lemma{\textnormal{\emph{Kopenhagen}}}\Cendnote{\textnormal{Bezug auf die gemeinsame Dänemark\oindex{Daenemark@\textbf{Dänemark}|pwk}-Reise 1896,
                     siehe Paul Goldmann an Arthur Schnitzler, 7. 9. [1896]}}}\label{K_L02917-5h} her. Wenn alſo bis zum \textsc{Auguſt} nicht ein Wunder geſchieht, werde ich in Berlin\oindex{Berlin@\textbf{Berlin}|pw} bleiben müſſen.\pend
           \pstart
           Schreib’ mir bald und ſei von Herzen gegrüßt!\pend
           \pstart
           Dein treuer {\\[\baselineskip]}\spacefill\mbox{Paul Goldmann.}\pend
           \leftskip=0em{}
         
         \endnumbering\mylabel{h}\end{ledgroupsized}\begin{anhang}\end{anhang}\newcommand{\dateiname}{L02917}\newcommand{\titel}{Paul Goldmann an Arthur Schnitzler, 29. 5. [1900]}\newcommand{\editorInnen}{Martin Anton Müller und Laura Untner}%% latex-leseansicht-abspann.tex
%% Abspann für die Leseansicht.
%% Der Schalter \ifkorrekturansicht ist bereits durch den Vorspann gesetzt.

%% latex-abspann.tex
%% Gemeinsamer Abspann für Korrekturansicht und Leseansicht.
%% Setzt den Schalter \ifkorrekturansicht voraus (gesetzt in den
%% einbindenden Dateien latex-korrekturansicht-abspann.tex bzw.
%% latex-leseansicht-abspann.tex).
%% ---------------------------------------------------------------

\normalsize

% Das esempio-Environment wird nur in der Leseansicht benötigt
\ifkorrekturansicht\else
\newenvironment{esempio}[3]%
{
    \vspace{1.5ex}
    \rlap{\underline{#1}}
    \par
    \setlength{\parindent}{0cm}
    \nopagebreak
    \leftskip=#2cm
    \rightskip=#3cm
}
{
    \par
}
\fi

\doendnotes{C}
\bigskip
\vfill

\clearpage

\footnotesize

\ifkorrekturansicht
  \lohead{\textsc{register}}
\fi

% theindex-Environment neu definieren ohne reledmac
\makeatletter
\renewenvironment{theindex}{%
  \ifkorrekturansicht
    \section*{\indexname}%
  \else
    \subsubsection*{Index der erwähnten Entitäten}%
  \fi
  \setlength{\parindent}{0pt}%
  \setlength{\parskip}{0pt plus 0.3pt}%
  \let\item\@idxitem
}{%
  \ifkorrekturansicht\clearpage\fi
}
\makeatother

\IfFileExists{\jobname-pw.ind}{\input{\jobname-pw.ind}}{}

% Quellenangabe nur in der Leseansicht
\ifkorrekturansicht\else
% Fallback-Definitionen, falls die .tex-Datei \titel etc. nicht gesetzt hat
\providecommand{\titel}{}
\providecommand{\editorInnen}{}
\providecommand{\dateiname}{\jobname}

\vspace{3cm}

\vfill

\footnotesize
\textsc{Quelle}: \titel. Herausgegeben von {\editorInnen}. In: \emph{Arthur Schnitzler: Briefwechsel mit Autorinnen und Autoren}.
 Digitale Edition, https://schnitzler-briefe.acdh.oeaw.ac.at/{\dateiname}.html (Stand \today)
\fi

\end{document}


      