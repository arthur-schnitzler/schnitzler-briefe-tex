%% latex-leseansicht-vorspann.tex
%% Vorspann für die Leseansicht.
%% Lädt die gemeinsame Datei latex-vorspann.tex mit nicht gesetztem Schalter.

\newif\ifkorrekturansicht
\korrekturansichtfalse

\input{../tex-inputs/latex-vorspann}


\section[Michael Georg Conrad an Arthur Schnitzler, 21. 6. 1893]{L00224 Michael Georg Conrad an Arthur Schnitzler, 21. 6. 1893}
\nopagebreak\mylabel{L00224v}
\rehead{ }\normalsize\beginnumbering\briefempfaengerindex{Schnitzler, Arthur@\textsc{Schnitzler, Arthur}!zzzConrad, Michael Georg@\emph{von Michael Georg Conrad}!1893-06-212@{21. 6. 1893}|(be}
\toendnotes[C]{\smallbreak\pagebreak[2]}
\correspDesc{Versand  durch Michael Georg Conrad am 21. 6. 1893 in München
\newline{}Erhalt  durch Arthur Schnitzler am 22. 6. 1893 in Wien}\toendnotes[C]{\smallbreak}
\Standort{TMW, HS Schn 1/83/1.}
\physDesc{Postkarte, 464 Zeichen
\newline{}Handschrift: schwarze Tinte, deutsche Kurrent
\newline{}Versand: 1) Stempel: »\nobreak{}\oindex{München@\textbf{München}|pwk}Muenchen L., 21. \textcolor{gray}{JU}N{[}I 1893{]}, 4–\textcolor{gray}{5 N}\nobreak{}«.   2) Stempel: »\nobreak{}\oindex{Wien@\textbf{Wien}, \emph{Verwaltungsgebiet}|pwk}Wien, 22 6 93, 9–10½V\nobreak{}«. 
\newline{}Ordnung: 1) mit Bleistift von unbekannter Hand nummeriert:
                                 »2«  2) mit rotem Buntstift von unbekannter Hand nummeriert:
                                    »3«}\toendnotes[C]{\smallbreak}\pstart{}{\pb}Herrn \textsc{D\textsuperscript{r}} Arthur Schnitzler\pend{}\pstart{}Wien I.\oindex{I., Innere Stadt@\textbf{I., Innere Stadt}, \emph{Verwaltungsgebiet}|pw}\pend{}\pstart{}Grillparzerſtr. 7\oindex{Wien@\textbf{Wien}!I., Innere Stadt@\textbf{I., Innere Stadt}!Grillparzerstraße@\textbf{Grillparzerstraße}, \emph{Straße}|pw}.\pend{}{\bigskip}\vspace{1em}
\pstart
           {\pb}München\oindex{München@\textbf{München}|pw}{ }21. 6. 93.\pend
           \vspace{0.5em}
\pstart
           Lieber Herr Doktor, eben von einer Wahlreiſe heimgekehrt, finde ich
               Ihren werten Brief. Hier in Eile die Antwort: Ihre wunderſchönen Gedichte\pwindex{Schnitzler, Arthur 15.\,5.\,1862 Wien – 21.\,10.\,1931 ebd.@\textsc{Schnitzler, Arthur} (15.\,5.\,1862 Wien – 21.\,10.\,1931 ebd.), \emph{Schriftsteller, Mediziner}!gute Irrtum@\strich\emph{Der gute Irrtum}|pwv}\pwindex{Schnitzler, Arthur 15.\,5.\,1862 Wien – 21.\,10.\,1931 ebd.@\textsc{Schnitzler, Arthur} (15.\,5.\,1862 Wien – 21.\,10.\,1931 ebd.), \emph{Schriftsteller, Mediziner}!Ohnmacht@\strich\emph{Ohnmacht}|pwv} habe ich mit beſten
               Empfehlungen an Hans Merian\pwindex{Merian, Hans 18.\,2.\,1857 Basel – 29.\,5.\,1902 Leipzig@\textsc{Merian, Hans} (18.\,2.\,1857 Basel – 29.\,5.\,1902 Leipzig), \emph{Schriftsteller, Redakteur}|pw} zur Aufnahme in die
                  »Geſellſch.\pwindex{Gesellschaft. Monatsschrift für Litteratur, Kunst und Sozialpolitik@\emph{Die Gesellschaft. Monatsschrift für Litteratur, Kunst und Sozialpolitik}|pw}« übergeben. Ich bin überzeugt,
               daß \uline{nur} redaktionell-techniſche Gründe imſtande{ }ſein
               können, den Abdruck{ }ſo vortrefflicher Beiträge zu verzögern. Mit Dank und Gruß\pend
           \pstart Ihr ergebener \spacefill\mbox{Conrad.}\pend{}\selectlanguage{ngerman}\endnumbering\briefempfaengerindex{Schnitzler, Arthur@\textsc{Schnitzler, Arthur}!zzzConrad, Michael Georg@\emph{von Michael Georg Conrad}!1893-06-212@{21. 6. 1893}|)be}\mylabel{L00224h}  \newcommand{\dateiname}{L00224}\newcommand{\titel}{Michael Georg Conrad an Arthur Schnitzler, 21. 6. 1893}\newcommand{\editorInnen}{Martin Anton Müller und Gerd-Hermann Susen}%% latex-leseansicht-abspann.tex
%% Abspann für die Leseansicht.
%% Der Schalter \ifkorrekturansicht ist bereits durch den Vorspann gesetzt.

%% latex-abspann.tex
%% Gemeinsamer Abspann für Korrekturansicht und Leseansicht.
%% Setzt den Schalter \ifkorrekturansicht voraus (gesetzt in den
%% einbindenden Dateien latex-korrekturansicht-abspann.tex bzw.
%% latex-leseansicht-abspann.tex).
%% ---------------------------------------------------------------

\normalsize

% Das esempio-Environment wird nur in der Leseansicht benötigt
\ifkorrekturansicht\else
\newenvironment{esempio}[3]%
{
    \vspace{1.5ex}
    \rlap{\underline{#1}}
    \par
    \setlength{\parindent}{0cm}
    \nopagebreak
    \leftskip=#2cm
    \rightskip=#3cm
}
{
    \par
}
\fi

\doendnotes{C}
\bigskip
\vfill

\clearpage

\footnotesize

\ifkorrekturansicht
  \lohead{\textsc{register}}
\fi

% theindex-Environment neu definieren ohne reledmac
\makeatletter
\renewenvironment{theindex}{%
  \ifkorrekturansicht
    \section*{\indexname}%
  \else
    \subsubsection*{Index der erwähnten Entitäten}%
  \fi
  \setlength{\parindent}{0pt}%
  \setlength{\parskip}{0pt plus 0.3pt}%
  \let\item\@idxitem
}{%
  \ifkorrekturansicht\clearpage\fi
}
\makeatother

\IfFileExists{\jobname-pw.ind}{\input{\jobname-pw.ind}}{}

% Quellenangabe nur in der Leseansicht
\ifkorrekturansicht\else
% Fallback-Definitionen, falls die .tex-Datei \titel etc. nicht gesetzt hat
\providecommand{\titel}{}
\providecommand{\editorInnen}{}
\providecommand{\dateiname}{\jobname}

\vspace{3cm}

\vfill

\footnotesize
\textsc{Quelle}: \titel. Herausgegeben von {\editorInnen}. In: \emph{Arthur Schnitzler: Briefwechsel mit Autorinnen und Autoren}.
 Digitale Edition, https://schnitzler-briefe.acdh.oeaw.ac.at/{\dateiname}.html (Stand \today)
\fi

\end{document}


