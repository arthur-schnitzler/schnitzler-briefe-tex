%% latex-korrekturansicht-vorspann.tex
%% Vorspann für die Korrekturansicht.
%% Lädt die gemeinsame Datei latex-vorspann.tex mit gesetztem Schalter.

\newif\ifkorrekturansicht
\korrekturansichttrue

\input{../tex-inputs/latex-vorspann}


\section[Michael Georg Conrad an Arthur Schnitzler, 21. 6. 1893]{L00224 Michael Georg Conrad an Arthur Schnitzler, 21. 6. 1893}
\nopagebreak\mylabel{L00224v}
\rehead{ }\normalsize\beginnumbering\briefempfaengerindex{Schnitzler, Arthur@\textsc{Schnitzler, Arthur}!zzzConrad, Michael Georg@\emph{von Michael Georg Conrad}!1893-06-211@{21. 6. 1893}|(be}
\toendnotes[C]{\smallbreak\pagebreak[2]}\Standort{TMW, HS Schn 1/83/1.}
\physDesc{Postkarte, 464 Zeichen
\newline{}Handschrift: schwarze Tinte, deutsche Kurrent
\newline{}Versand: 1) Stempel: »\nobreak{}\oindex{Muenchen@\textbf{München}, \emph{P.PPLA}|pwk}Muenchen L., 21. \textcolor{gray}{JU}N{[}I 1893{]}, 4–\textcolor{gray}{5 N}\nobreak{}«.   2) Stempel: »\nobreak{}Wien, 22 6 93, 9–10½V\nobreak{}«. 
\newline{}Ordnung: 1) mit Bleistift von unbekannter Hand nummeriert:
                                 »2«  2) mit rotem Buntstift von unbekannter Hand nummeriert:
                                    »3«}\toendnotes[C]{\smallbreak}\pstart{}{\pb}Herrn \textsc{D\textsuperscript{r}} Arthur Schnitzler\pend{}\pstart{}Wien I.\oindex{I., Innere Stadt@\textbf{I., Innere Stadt}, \emph{A.ADM3}|pw}\pend{}\pstart{}Grillparzerſtr. 7\oindex{Grillparzerstrasse@\textbf{Grillparzerstraße}, \emph{R.ST}|pw}.\pend{}{\bigskip}\vspace{1em}
\pstart
           {\pb}München\oindex{Muenchen@\textbf{München}, \emph{P.PPLA}|pw}{ }21. 6. 93.\pend
           \vspace{0.5em}
\pstart
           Lieber Herr Doktor, eben von einer Wahlreiſe heimgekehrt, finde ich
               Ihren werten Brief. Hier in Eile die Antwort: Ihre wunderſchönen Gedichte\pwindex{gute Irrtum@\emph{Der gute Irrtum}|pwv}\pwindex{Ohnmacht@\emph{Ohnmacht}|pwv} habe ich mit beſten
               Empfehlungen an Hans Merian\pwindex{Merian, Hans 18.02.1857 – 29.05.1902@\textsc{Merian, Hans} (18.02.1857 – 29.05.1902), \emph{Schriftsteller/Schriftstellerin, Redakteur/Redakteurin}|pw} zur Aufnahme in die
                  »Geſellſch.\pwindex{Gesellschaft. Monatsschrift fuer Litteratur, Kunst und Sozialpolitik@\emph{Die Gesellschaft. Monatsschrift für Litteratur, Kunst und Sozialpolitik}|pw}« übergeben. Ich bin überzeugt,
               daß \uline{nur} redaktionell-techniſche Gründe imſtande ſein
               können, den Abdruck ſo vortrefflicher Beiträge zu verzögern. Mit Dank und Gruß\pend
           \pstart Ihr ergebener \spacefill\mbox{Conrad.}\pend{}\selectlanguage{ngerman}\endnumbering\briefempfaengerindex{Schnitzler, Arthur@\textsc{Schnitzler, Arthur}!zzzConrad, Michael Georg@\emph{von Michael Georg Conrad}!1893-06-211@{21. 6. 1893}|)be}\mylabel{L00224h}  \normalsize

\doendnotes{C}
\bigskip
\vfill

\clearpage

\footnotesize

\lohead{\textsc{register}}

% Definiere theindex-Environment komplett neu ohne reledmac
\makeatletter
\renewenvironment{theindex}{%
  \section*{\indexname}%
  \setlength{\parindent}{0pt}%
  \setlength{\parskip}{0pt plus 0.3pt}%
  \let\item\@idxitem
}{%
  \clearpage
}
\makeatother

\IfFileExists{\jobname-pw.ind}{\input{\jobname-pw.ind}}{}

\end{document}

      