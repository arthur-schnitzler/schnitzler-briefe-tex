%% latex-korrekturansicht-vorspann.tex
%% Vorspann für die Korrekturansicht.
%% Lädt die gemeinsame Datei latex-vorspann.tex mit gesetztem Schalter.

\newif\ifkorrekturansicht
\korrekturansichttrue

\input{../tex-inputs/latex-vorspann}


\section[ Paul Goldmann an Arthur Schnitzler, 7. 8. 1907]{L03255 Paul Goldmann an Arthur Schnitzler, 7. 8. 1907}
\nopagebreak\mylabel{L03255v}
\rehead{ }\normalsize\beginnumbering\briefempfaengerindex{Schnitzler, Arthur@\textsc{Schnitzler, Arthur}!zzzGoldmann, Paul@\emph{von Paul Goldmann}!1907-08-071@{7. 8. 1907}|(be}
\toendnotes[C]{\smallbreak\pagebreak[2]}\Standort{DLA, A:Schnitzler, HS.NZ85.1.3175.}
\physDesc{Bildpostkarte, 269 Zeichen
\newline{}Handschrift: 1) schwarze Tinte, deutsche Kurrent\hspace{1em}2) schwarze Tinte, lateinische Kurrent (\noindent{}Adresse)\hspace{1em}
\newline{}Versand: Stempel: »\nobreak{}\oindex{Swinoujście@\textbf{Świnoujście}|pwk}Swinemünde, 7. 8. 07, 10–11\nobreak{}«.  }\toendnotes[C]{\smallbreak}\pstart{}{\pb}Österreich\oindex{Oesterreich-Ungarn@\textbf{Österreich-Ungarn}|pw}.\pend{}\pstart{}Herrn\pend{}\pstart{}Dr. Arthur Schnitzler\pend{}\pstart{}Welsberg im Pustertal\oindex{Welsberg-Taisten@\textbf{Welsberg-Taisten}|pw}\pend{}\pstart{}Tirol\oindex{Tirol@\textbf{Tirol}|pw}.\pend{}{\bigskip}
\pstart
           {\pb}\textcolor{gray}{\textbf{\textbf{Swinemünde\oindex{Swinoujście@\textbf{Świnoujście}|pw}}}}\hfill \textcolor{gray}{\textbf{\textbf{Strand\oindex{Warszow plaża@\textbf{Warszòw plaża}|pw}}}}\pend
           \vspace{1em}
\pstart
           \noindent{}{\pb}7. 8. 07. Ich war hier\oindex{Swinoujście@\textbf{Świnoujście}|pwv}, um über die \label{K_L03255-1v}\edtext{Krisen-\textsc{Entrevue}}{\lemma{\textnormal{\emph{Krisen-Entrevue}}}\Cendnote{\textnormal{Gemeint war das Treffen zwischen Kaiser Wilhelm II.\pwindex{Wilhelm II. von Preussen 27.1.1859 – 4.6.1941@\textsc{Wilhelm II. von Preußen} (27.1.1859 – 4.6.1941), \emph{Kaiser/Kaiserin}|pwk} und Zar Nikolaus II.\pwindex{Nikolaus II. von Russland 1868-05-06 – 1918-07-17@\textsc{Nikolaus II. von Russland} (1868-05-06 – 1918-07-17), \emph{Zar/Zarin}|pwk} am 4. 8. 1907 in Swinemünde\oindex{Swinoujście@\textbf{Świnoujście}|pwk}.
               }}}\label{K_L03255-1} zu berichten. Herzliche Grüße Dir, lieber Freund, u. Deiner
                  Frau\pwindex{Schnitzler, Olga 17.01.1882 – 13.01.1970@\textsc{Schnitzler, Olga} (17.01.1882 – 13.01.1970), \emph{Schauspieler/Schauspielerin, Sänger/Sängerin}|pwv}. Nächste Woche gehe
               ich auf Urlaub, aber ich weiß noch immer nicht, wohin. Dein \spacefill\mbox{Paul
                  Goldmann.}\pend
           \selectlanguage{ngerman}\endnumbering\briefempfaengerindex{Schnitzler, Arthur@\textsc{Schnitzler, Arthur}!zzzGoldmann, Paul@\emph{von Paul Goldmann}!1907-08-071@{7. 8. 1907}|)be}\mylabel{L03255h}  \normalsize

\doendnotes{C}
\bigskip
\vfill

\clearpage

\footnotesize

\lohead{\textsc{register}}

% Definiere theindex-Environment komplett neu ohne reledmac
\makeatletter
\renewenvironment{theindex}{%
  \section*{\indexname}%
  \setlength{\parindent}{0pt}%
  \setlength{\parskip}{0pt plus 0.3pt}%
  \let\item\@idxitem
}{%
  \clearpage
}
\makeatother

\IfFileExists{\jobname-pw.ind}{\input{\jobname-pw.ind}}{}

\end{document}

      