%% latex-leseansicht-vorspann.tex
%% Vorspann für die Leseansicht.
%% Lädt die gemeinsame Datei latex-vorspann.tex mit nicht gesetztem Schalter.

\newif\ifkorrekturansicht
\korrekturansichtfalse

\input{../tex-inputs/latex-vorspann}


\section[ Paul Goldmann an Arthur Schnitzler, 7. 8. 1907]{L03255 Paul Goldmann an Arthur Schnitzler,  7. 8. 1907}
\nopagebreak\mylabel{L03255v}
\rehead{ }\normalsize\beginnumbering\briefempfaengerindex{Schnitzler, Arthur@\textsc{Schnitzler, Arthur}!zzzGoldmann, Paul@\emph{von Paul Goldmann}!1907-08-071@{7. 8. 1907}|(be}
\toendnotes[C]{\smallbreak\pagebreak[2]}
\correspDesc{Versand  durch Paul Goldmann am 7. 8. 1907 in Świnoujście
\newline{}Erhalt  durch Arthur Schnitzler im Zeitraum [8. 8. 1907
                  – 12. 8. 1907?] in Welsberg-Taisten}\toendnotes[C]{\smallbreak}
\Standort{DLA, A:Schnitzler, HS.NZ85.1.3175.}
\physDesc{Bildpostkarte, 269 Zeichen
\newline{}Handschrift: schwarze Tinte, deutsche Kurrent
\newline{}Versand: Stempel: »\nobreak{}\oindex{Świnoujście@\textbf{Świnoujście}, \emph{Hauptstadt}|pwk}Swinemünde, 7. 8. 07, 10–11\nobreak{}«.  }\toendnotes[C]{\smallbreak}\pstart{}\textsc{{\pb}Österreich\oindex{Österreich-Ungarn@\textbf{Österreich-Ungarn}|pw}.}\pend{}\pstart{}\textsc{Herrn}\pend{}\pstart{}\textsc{Dr. Arthur Schnitzler}\pend{}\pstart{}\textsc{Welsberg im Pustertal\oindex{Welsberg-Taisten@\textbf{Welsberg-Taisten}, \emph{Verwaltungsgebiet}|pw}}\pend{}\pstart{}\textsc{Tirol\oindex{Tirol@\textbf{Tirol}, \emph{Land}|pw}.}\pend{}{\bigskip}
\pstart
           {\pb}\textcolor{gray}{\textbf{\textbf{Swinemünde\oindex{Świnoujście@\textbf{Świnoujście}, \emph{Hauptstadt}|pw}}}}\hfill \textcolor{gray}{\textbf{\textbf{Strand\oindex{Warszòw plaża@\textbf{Warszòw plaża}|pw}}}}\pend
           \vspace{1em}
\pstart
           \noindent{}{\pb}7. 8. 07. Ich war hier\oindex{Świnoujście@\textbf{Świnoujście}, \emph{Hauptstadt}|pwv}, um über die \label{K_L03255-1v}\edtext{Kriſen-\textsc{Entrevue}}{\lemma{\textnormal{\emph{Krisen-Entrevue}}}\Cendnote{\textnormal{Gemeint war das Treffen zwischen Kaiser Wilhelm II.\pwindex{Wilhelm II. von Preußen 27.\,1.\,1859 Potsdam – 4.\,6.\,1941 Gemeente Utrechtse Heuvelrug@\textsc{Wilhelm II. von Preußen} (27.\,1.\,1859 Potsdam – 4.\,6.\,1941 Gemeente Utrechtse Heuvelrug), \emph{Kaiser}|pwk} und Zar Nikolaus II.\pwindex{Nikolaus II. von Russland 6.\,5.\,1868 Pushkin – 17.\,7.\,1918 Jekaterinburg@\textsc{Nikolaus II. von Russland} (6.\,5.\,1868 Pushkin – 17.\,7.\,1918 Jekaterinburg), \emph{Zar}|pwk} am 4. 8. 1907 in Swinemünde\oindex{Świnoujście@\textbf{Świnoujście}, \emph{Hauptstadt}|pwk}.
               }}}\label{K_L03255-1} zu berichten. Herzliche Grüße Dir, lieber Freund, u. Deiner
                  Frau\pwindex{Schnitzler, Olga 17.\,1.\,1882 Wien – 13.\,1.\,1970 Lugano@\textsc{Schnitzler, Olga} (17.\,1.\,1882 Wien – 13.\,1.\,1970 Lugano), \emph{Schauspielerin, Sängerin}|pwv}. Nächſte Woche gehe
               ich auf Urlaub, aber ich weiß noch immer nicht, wohin. Dein \spacefill\mbox{Paul
                  Goldmann.}\pend
           \selectlanguage{ngerman}\endnumbering\briefempfaengerindex{Schnitzler, Arthur@\textsc{Schnitzler, Arthur}!zzzGoldmann, Paul@\emph{von Paul Goldmann}!1907-08-071@{7. 8. 1907}|)be}\mylabel{L03255h}  \newcommand{\dateiname}{L03255}\newcommand{\titel}{Paul Goldmann an Arthur Schnitzler, 7. 8. 1907}\newcommand{\editorInnen}{Martin Anton Müller und Laura Untner}%% latex-leseansicht-abspann.tex
%% Abspann für die Leseansicht.
%% Der Schalter \ifkorrekturansicht ist bereits durch den Vorspann gesetzt.

%% latex-abspann.tex
%% Gemeinsamer Abspann für Korrekturansicht und Leseansicht.
%% Setzt den Schalter \ifkorrekturansicht voraus (gesetzt in den
%% einbindenden Dateien latex-korrekturansicht-abspann.tex bzw.
%% latex-leseansicht-abspann.tex).
%% ---------------------------------------------------------------

\normalsize

% Das esempio-Environment wird nur in der Leseansicht benötigt
\ifkorrekturansicht\else
\newenvironment{esempio}[3]%
{
    \vspace{1.5ex}
    \rlap{\underline{#1}}
    \par
    \setlength{\parindent}{0cm}
    \nopagebreak
    \leftskip=#2cm
    \rightskip=#3cm
}
{
    \par
}
\fi

\doendnotes{C}
\bigskip
\vfill

\clearpage

\footnotesize

\ifkorrekturansicht
  \lohead{\textsc{register}}
\fi

% theindex-Environment neu definieren ohne reledmac
\makeatletter
\renewenvironment{theindex}{%
  \ifkorrekturansicht
    \section*{\indexname}%
  \else
    \subsubsection*{Index der erwähnten Entitäten}%
  \fi
  \setlength{\parindent}{0pt}%
  \setlength{\parskip}{0pt plus 0.3pt}%
  \let\item\@idxitem
}{%
  \ifkorrekturansicht\clearpage\fi
}
\makeatother

\IfFileExists{\jobname-pw.ind}{\input{\jobname-pw.ind}}{}

% Quellenangabe nur in der Leseansicht
\ifkorrekturansicht\else
% Fallback-Definitionen, falls die .tex-Datei \titel etc. nicht gesetzt hat
\providecommand{\titel}{}
\providecommand{\editorInnen}{}
\providecommand{\dateiname}{\jobname}

\vspace{3cm}

\vfill

\footnotesize
\textsc{Quelle}: \titel. Herausgegeben von {\editorInnen}. In: \emph{Arthur Schnitzler: Briefwechsel mit Autorinnen und Autoren}.
 Digitale Edition, https://schnitzler-briefe.acdh.oeaw.ac.at/{\dateiname}.html (Stand \today)
\fi

\end{document}


