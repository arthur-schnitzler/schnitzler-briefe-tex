%% latex-leseansicht-vorspann.tex
%% Vorspann für die Leseansicht.
%% Lädt die gemeinsame Datei latex-vorspann.tex mit nicht gesetztem Schalter.

\newif\ifkorrekturansicht
\korrekturansichtfalse

\input{../tex-inputs/latex-vorspann}


\section[Theodor Herzl an Arthur Schnitzler, 9. 1. 1895]{L03845 Theodor Herzl an Arthur Schnitzler, 9. 1. 1895}
\nopagebreak\mylabel{L03845v}
\rehead{ }\normalsize\beginnumbering\briefempfaengerindex{Schnitzler, Arthur@\textsc{Schnitzler, Arthur}!zzzHerzl, Theodor@\emph{von Theodor Herzl}!1895-01-091@{9. 1. 1895}|(be}
\toendnotes[C]{\smallbreak\pagebreak[2]}
\correspDesc{Versand  durch Theodor Herzl am 9. 1. 1895 in Paris
\newline{}Erhalt  durch Arthur Schnitzler im Zeitraum [10. 1. 1895 – 14. 1. 1895?] in Wien}\toendnotes[C]{\smallbreak}
\Standort{CUL, Schnitzler, B 39.}
\physDesc{Brief, 1 Blatt, 3 Seiten, 2856 Zeichen
\newline{}Handschrift: schwarze Tinte, lateinische Kurrent
\newline{}Ordnung: 1) mit Bleistift von unbekannter Hand nummeriert: »24«  2) mit blauem Buntstift von Leon Kellner\pwindex{Kellner, Leon 17.\,4.\,1859 Tarnów – 5.\,12.\,1928 Wien@\textsc{Kellner, Leon} (17.\,4.\,1859 Tarnów – 5.\,12.\,1928 Wien), \emph{Zionist, Literaturhistoriker, Anglist}|pw} Markierung von Stellen für
                                 die Publikation 3) mit rotem Buntstift eine Anstreichung}
\buchAbdrucke{\weitereDrucke{1) \pwindex{Kellner, Leon 17.\,4.\,1859 Tarnów – 5.\,12.\,1928 Wien@\textsc{Kellner, Leon} (17.\,4.\,1859 Tarnów – 5.\,12.\,1928 Wien), \emph{Zionist, Literaturhistoriker, Anglist}!Theodor Herzls Lehrjahre (1860–1895). Nach den handschriftlichen Quellen@\strich\emph{Theodor Herzls Lehrjahre (1860–1895). Nach den handschriftlichen Quellen}|pwk}\emph{[Auszug].} In: Leon Kellner: \emph{Theodor Herzls Lehrjahre (1860–1895). Nach den handschriftlichen Quellen}. Wien, Berlin: \emph{R. Löwit-Verlag} 1920, S. 148–149.} \weitereDrucke{2) Theodor Herzl: \emph{Briefe und
                        autobiographische Notizen 1866–1895}. Bearbeitet von Johannes Wachten in Zusammenarbeit mit Chaya Harel, Daisy Tycho und Manfred Winkler. Berlin, Frankfurt am Main, Wien: \emph{Propyläen} 1983, S. 569–570 (Briefe und Tagebücher. Herausgegeben von Alex Bein, Hermann Greive, Moshe Schaerf, Julius H. Schoeps und Johannes Wachten, 1).} }\toendnotes[C]{\smallbreak}
\pstart
           \raggedleft{}{\pb}9. I. 95\pend
           
\pstart{}Mein lieber Freund!\pend\vspace{0.5em}
\pstart
           Herzlichen Dank für alles!\pend
           
\pstart
           Stärker noch als in Ihren früheren
      Briefen habe ich bei \label{K_L03845-1v}\edtext{}{\lemma{\textnormal{\emph{XXXX Lemmafehler}}}\Cendnote{\textnormal{XXXX7.1.1895}}}\label{K_L03845-1} das Gefühl der Wärme.\pend
           
\pstart
           Glauben Sie, dass ich diese Sympathie
      ganz erwiedere! Ich freue mich, dass
      ich Sie zu meinem Vertrauten gemacht
      habe. Sie sind der Einzige. Niemand,
      Niemand, Niemand hat auch nur eine
      Ahnung davon. Zweimal war die Versuchung stark, mich hier mitzutheilen.
      Zuerst beim Bildhauer Beer\pwindex{Beer, Friedrich 1.\,9.\,1846 Brünn – 18.\,10.\,1912 Florenz@\textsc{Beer, Friedrich} (1.\,9.\,1846 Brünn – 18.\,10.\,1912 Florenz), \emph{Bildhauer}|pw}, der meine
               Büste\pwindex{Beer, Friedrich 1.\,9.\,1846 Brünn – 18.\,10.\,1912 Florenz@\textsc{Beer, Friedrich} (1.\,9.\,1846 Brünn – 18.\,10.\,1912 Florenz), \emph{Bildhauer}!Portaitbüste von Theodor Herzl@\strich\emph{Portaitbüste von Theodor Herzl}|pw} machte. Bei dem ist nämlich das
               Stück\pwindex{Herzl, Theodor 2.\,5.\,1860 Budapest – 3.\,7.\,1904 Edlach@\textsc{Herzl, Theodor} (2.\,5.\,1860 Budapest – 3.\,7.\,1904 Edlach), \emph{Schriftsteller, Journalist}!neue Ghetto. Schauspiel in vier Acten@\strich\emph{Das neue Ghetto. Schauspiel in vier Acten}|pwv} entstanden. Einmal während einer
      Sitzung ereiferte ich mich, als ich ihm
      die Judenfrage in Oestreich\oindex{Österreich-Ungarn@\textbf{Österreich-Ungarn}|pw} auseinander
      setzte. Das schwang in mir stark fort
      als ich wegging. Auf dem Heimweg fiel
      mir das ganze Stück\pwindex{Herzl, Theodor 2.\,5.\,1860 Budapest – 3.\,7.\,1904 Edlach@\textsc{Herzl, Theodor} (2.\,5.\,1860 Budapest – 3.\,7.\,1904 Edlach), \emph{Schriftsteller, Journalist}!neue Ghetto. Schauspiel in vier Acten@\strich\emph{Das neue Ghetto. Schauspiel in vier Acten}|pwv} ein. Am nächsten
               Tag sagte ich ihm: Beer\pwindex{Beer, Friedrich 1.\,9.\,1846 Brünn – 18.\,10.\,1912 Florenz@\textsc{Beer, Friedrich} (1.\,9.\,1846 Brünn – 18.\,10.\,1912 Florenz), \emph{Bildhauer}|pw}, wenn ich jetzt
      nicht ein Taglöhner wäre, sondern mich
      vierzehn Tage nach Ravello\oindex{Ravello@\textbf{Ravello}, \emph{Hauptstadt}|pw} oberhalb von
               Amalfi\oindex{Amalfi@\textbf{Amalfi}, \emph{Hauptstadt}|pw} setzen könnte, schriebe ich ein {\pb}Stück.\pend
           
\pstart
           Er machte ein Gesicht, das mir ungläubig
      schien. Am dritten Tag blieb ich von
      der Sitzung aus u. blieb aus bis es
      fertig war. Als ich wiederkam, reizte
      es mich ungemein, es ihm zu sagen
      und es ihm vorzulesen. Ich widerstand
      aber und erklärte meine abwesenheit
      mit Zeitungssachen.\pend
           
\pstart
           Die zweite Versuchung war Nordau\pwindex{Nordau, Max 29.\,7.\,1849 Budapest – 22.\,1.\,1923 Paris@\textsc{Nordau, Max} (29.\,7.\,1849 Budapest – 22.\,1.\,1923 Paris), \emph{Schriftsteller, Mediziner}|pw}, der
      mir ein sehr guter Freund ist und mir
      in seiner rüden Wahrheitsliebe gewiss
      alle Fehler – die er hätte wahrnehmen
      können – gesagt hätte. Als er mir nun
      kürzlich sein \label{K_L03845-2v}\edtext{neuestes Stück\pwindex{Nordau, Max 29.\,7.\,1849 Budapest – 22.\,1.\,1923 Paris@\textsc{Nordau, Max} (29.\,7.\,1849 Budapest – 22.\,1.\,1923 Paris), \emph{Schriftsteller, Mediziner}!Kugel@\strich\emph{Die Kugel}|pwv}}{\lemma{\textnormal{\emph{neuestes Stück}}}\Cendnote{\textnormal{Max Nordaus\pwindex{Nordau, Max 29.\,7.\,1849 Budapest – 22.\,1.\,1923 Paris@\textsc{Nordau, Max} (29.\,7.\,1849 Budapest – 22.\,1.\,1923 Paris), \emph{Schriftsteller, Mediziner}|pwk} Schauspiel in fünf Aufzügen \emph{Die Kugel}\pwindex{Nordau, Max 29.\,7.\,1849 Budapest – 22.\,1.\,1923 Paris@\textsc{Nordau, Max} (29.\,7.\,1849 Budapest – 22.\,1.\,1923 Paris), \emph{Schriftsteller, Mediziner}!Kugel@\strich\emph{Die Kugel}|pwk} hatte am 31. 10. 1894{ }Uraufführung\eventindex{Lessing-Theater@\textbf{Lessing-Theater}!Uraufführung von Die Kugel, 31.10.1894@Uraufführung von Die Kugel, 31.10.1894|pwk} am \emph{Lessing-Theater}\orgindex{Lessing-Theater@Lessing-Theater|pwk} in Berlin\oindex{Berlin@\textbf{Berlin}, \emph{Hauptstadt}|pwk}.}}}\label{K_L03845-2} vorlas,
      riss es mich wieder. Aber der Vorsatz
      war gefasst. Ein wirkliches Geheimniss
      darf höchstens auf vier Augen stehen.
      Dabei bleibt es.\pend
           
\pstart
           Das Manuscript\pwindex{Herzl, Theodor 2.\,5.\,1860 Budapest – 3.\,7.\,1904 Edlach@\textsc{Herzl, Theodor} (2.\,5.\,1860 Budapest – 3.\,7.\,1904 Edlach), \emph{Schriftsteller, Journalist}!neue Ghetto. Schauspiel in vier Acten@\strich\emph{Das neue Ghetto. Schauspiel in vier Acten}|pwv}, das bei Ihnen ist, biete
      ich Ihnen als Geschenk an. Hat das
               Stück\pwindex{Herzl, Theodor 2.\,5.\,1860 Budapest – 3.\,7.\,1904 Edlach@\textsc{Herzl, Theodor} (2.\,5.\,1860 Budapest – 3.\,7.\,1904 Edlach), \emph{Schriftsteller, Journalist}!neue Ghetto. Schauspiel in vier Acten@\strich\emph{Das neue Ghetto. Schauspiel in vier Acten}|pwv} den Erfolg, den ich nicht für unmöglich halte, wird’s Ihnen in 10–20
      Jahren eine hübsche Erinnerung sein.
      Und verkracht es, gleich meinen früheren Versuchen, so möge es Sie späterhin an
      den Beginn unserer Freundschaft erinnern,
      die hoffentlich dauern wird.\pend
           
\pstart
           {\pb}In der Beilage finden Sie 13 fl. Die Sechserln
      kann ich Ihnen von hier nicht schicken.
      Geben Sie also die paar Knöpfe dem
      armen Teufel\pwindex{?? [Schreibkraft für Arthur Schnitzler] @\textsc{?? [Schreibkraft für Arthur Schnitzler]}|pwu}, dessen überflüssige Quittung
      Sie mir zum Spass einschickten.\pend
           
\pstart
           Das Lob meiner Feuilletons schmeckt mir
      von Ihnen, mein sehr Lieber, gut.
      Aber ich kann gerade das letzte\pwindex{Herzl, Theodor 2.\,5.\,1860 Budapest – 3.\,7.\,1904 Edlach@\textsc{Herzl, Theodor} (2.\,5.\,1860 Budapest – 3.\,7.\,1904 Edlach), \emph{Schriftsteller, Journalist}!Palais Bourbon. V. »Sprechen wir über Politik«@\strich\emph{Das Palais Bourbon. V. »Sprechen wir über Politik{\rufezeichen}«}|pw}, das
      Sie erwähnen, nicht gut finden.\pend
           
\pstart
           Ich will jetzt noch fünf solche \label{K_L03845-3v}\edtext{Palais Bourbon\oindex{Palais Bourbon@\textbf{Palais Bourbon}, \emph{Regierungsgebäude}|pw}-Sachen}{\lemma{\textnormal{\emph{Palais Bourbon-Sachen}}}\Cendnote{\textnormal{Herzl\pwindex{Herzl, Theodor 2.\,5.\,1860 Budapest – 3.\,7.\,1904 Edlach@\textsc{Herzl, Theodor} (2.\,5.\,1860 Budapest – 3.\,7.\,1904 Edlach), \emph{Schriftsteller, Journalist}|pwk} schrieb insgesamt elf Feuilletontexte unter diesem Obertitel für die \emph{Neue Freie Presse}\pwindex{Neue Freie Presse@\emph{Neue Freie Presse}|pwk}, die er danach noch einmal in einem Essayband publizierte, Theodor Herzl\pwindex{Herzl, Theodor 2.\,5.\,1860 Budapest – 3.\,7.\,1904 Edlach@\textsc{Herzl, Theodor} (2.\,5.\,1860 Budapest – 3.\,7.\,1904 Edlach), \emph{Schriftsteller, Journalist}|pwk}: \emph{Das Parlais Bourbon. Bilder aus dem französischen Parlamentsleben}\pwindex{Herzl, Theodor 2.\,5.\,1860 Budapest – 3.\,7.\,1904 Edlach@\textsc{Herzl, Theodor} (2.\,5.\,1860 Budapest – 3.\,7.\,1904 Edlach), \emph{Schriftsteller, Journalist}!Palais Bourbon. Bilder aus dem französischen Parlamentsleben@\strich\emph{Das Palais Bourbon. Bilder aus dem französischen Parlamentsleben}|pwk}, Leipzig: \emph{Duncker {\kaufmannsund} Humblot}{ }1895.}}}\label{K_L03845-3} schreiben. Das letzte
         wird den Sinn des Ganzen zusammenfassen und »die Schule des Journalisten\pwindex{Herzl, Theodor 2.\,5.\,1860 Budapest – 3.\,7.\,1904 Edlach@\textsc{Herzl, Theodor} (2.\,5.\,1860 Budapest – 3.\,7.\,1904 Edlach), \emph{Schriftsteller, Journalist}!Palais Bourbon. XI. »Die Schule des Journalisten«@\strich\emph{Das Palais Bourbon. XI. »Die Schule des Journalisten«}|pw}«
      heissen. So müssen wir ja unsere Poesie
      – von \label{K_L03845-4v}\edtext{\griechisch{ποιειν}}{\lemma{\textnormal{\emph{\griechisch{ποιειν}}}}\Cendnote{\textnormal{altgriechisch: setzen, stellen, legen, machen}}}\label{K_L03845-4}, hineinlegen – aus unserem
      Leben saugen. Denn einen andern Zweck
      hat ja unser Leben nicht. Ich finde
      diesen übrigens schön.\pend
           
\pstart
           Dass Ihnen mein Stück\pwindex{Herzl, Theodor 2.\,5.\,1860 Budapest – 3.\,7.\,1904 Edlach@\textsc{Herzl, Theodor} (2.\,5.\,1860 Budapest – 3.\,7.\,1904 Edlach), \emph{Schriftsteller, Journalist}!neue Ghetto. Schauspiel in vier Acten@\strich\emph{Das neue Ghetto. Schauspiel in vier Acten}|pwv} auch beim zweiten
      Lesen noch gefiel, hat mir wie gesagt
      ganz warm gemacht.\pend
           
\pstart
           Werden ja sehen, was die Directoren\pwindex{Blumenthal, Oskar 13.\,3.\,1852 Berlin – 24.\,4.\,1917 ebd.@\textsc{Blumenthal, Oskar} (13.\,3.\,1852 Berlin – 24.\,4.\,1917 ebd.), \emph{Schriftsteller, Journalist, Theaterleiter}|pwv}\pwindex{Brahm, Otto 5.\,2.\,1856 Hamburg – 28.\,11.\,1912 Berlin@\textsc{Brahm, Otto} (5.\,2.\,1856 Hamburg – 28.\,11.\,1912 Berlin), \emph{Theaterleiter, Regisseur}|pwv}\pwindex{Schlenther, Paul 20.\,8.\,1854 Chernyakhovsk – 30.\,4.\,1916 Berlin@\textsc{Schlenther, Paul} (20.\,8.\,1854 Chernyakhovsk – 30.\,4.\,1916 Berlin), \emph{Schriftsteller, Kritiker, Theaterleiter}|pwv} sagen;
      Sie, mein Freund, sind doch jedenfalls
      günstig voreingenommen. Vergessen Sie
      nicht, mir das erste Ergebnis zu telegraphiren u. gleich ausführlich unter der
      verabredeten Adresse zu schreiben.\pend
           
\pstart
           Mit herzlichem Grus. Ihr getreuer{\\[\baselineskip]}\spacefill\mbox{Th. H.}\pend
           \leftskip=0em{}\selectlanguage{ngerman}\endnumbering\briefempfaengerindex{Schnitzler, Arthur@\textsc{Schnitzler, Arthur}!zzzHerzl, Theodor@\emph{von Theodor Herzl}!1895-01-091@{9. 1. 1895}|)be}\mylabel{L03845h}
\begin{anhang}
\end{anhang}\newcommand{\dateiname}{L03845}\newcommand{\titel}{Theodor Herzl an Arthur Schnitzler, 9. 1. 1895}\newcommand{\editorInnen}{Selma Jahnke und Martin Anton Müller}%% latex-leseansicht-abspann.tex
%% Abspann für die Leseansicht.
%% Der Schalter \ifkorrekturansicht ist bereits durch den Vorspann gesetzt.

%% latex-abspann.tex
%% Gemeinsamer Abspann für Korrekturansicht und Leseansicht.
%% Setzt den Schalter \ifkorrekturansicht voraus (gesetzt in den
%% einbindenden Dateien latex-korrekturansicht-abspann.tex bzw.
%% latex-leseansicht-abspann.tex).
%% ---------------------------------------------------------------

\normalsize

% Das esempio-Environment wird nur in der Leseansicht benötigt
\ifkorrekturansicht\else
\newenvironment{esempio}[3]%
{
    \vspace{1.5ex}
    \rlap{\underline{#1}}
    \par
    \setlength{\parindent}{0cm}
    \nopagebreak
    \leftskip=#2cm
    \rightskip=#3cm
}
{
    \par
}
\fi

\doendnotes{C}
\bigskip
\vfill

\clearpage

\footnotesize

\ifkorrekturansicht
  \lohead{\textsc{register}}
\fi

% theindex-Environment neu definieren ohne reledmac
\makeatletter
\renewenvironment{theindex}{%
  \ifkorrekturansicht
    \section*{\indexname}%
  \else
    \subsubsection*{Index der erwähnten Entitäten}%
  \fi
  \setlength{\parindent}{0pt}%
  \setlength{\parskip}{0pt plus 0.3pt}%
  \let\item\@idxitem
}{%
  \ifkorrekturansicht\clearpage\fi
}
\makeatother

\IfFileExists{\jobname-pw.ind}{\input{\jobname-pw.ind}}{}

% Quellenangabe nur in der Leseansicht
\ifkorrekturansicht\else
% Fallback-Definitionen, falls die .tex-Datei \titel etc. nicht gesetzt hat
\providecommand{\titel}{}
\providecommand{\editorInnen}{}
\providecommand{\dateiname}{\jobname}

\vspace{3cm}

\vfill

\footnotesize
\textsc{Quelle}: \titel. Herausgegeben von {\editorInnen}. In: \emph{Arthur Schnitzler: Briefwechsel mit Autorinnen und Autoren}.
 Digitale Edition, https://schnitzler-briefe.acdh.oeaw.ac.at/{\dateiname}.html (Stand \today)
\fi

\end{document}


