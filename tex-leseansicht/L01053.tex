%% latex-leseansicht-vorspann.tex
%% Vorspann für die Leseansicht.
%% Lädt die gemeinsame Datei latex-vorspann.tex mit nicht gesetztem Schalter.

\newif\ifkorrekturansicht
\korrekturansichtfalse

\input{../tex-inputs/latex-vorspann}


\section[Richard Beer-Hofmann an Arthur Schnitzler, 13. 7. 1900]{L01053 Richard Beer-Hofmann an Arthur Schnitzler, 13. 7. 1900}
\nopagebreak\mylabel{L01053v}
\rehead{ }\normalsize\beginnumbering\briefempfaengerindex{Schnitzler, Arthur@\textsc{Schnitzler, Arthur}!zzzBeer-Hofmann, Richard@\emph{von Richard Beer-Hofmann}!1900-07-131@{13. 7. 1900}|(be}
\toendnotes[C]{\smallbreak\pagebreak[2]}
\correspDesc{Versand  durch Richard Beer-Hofmann am 13. 7. 1900 in Altaussee
\newline{}Erhalt  durch Arthur Schnitzler am 15. 7. 1900 in Reichenau an der Rax}\toendnotes[C]{\smallbreak}
\Standort{CUL, Schnitzler, B 8.}
\physDesc{Brief, 2 Blätter, 3 Seiten, 2129 Zeichen
\newline{}Handschrift: schwarze Tinte, lateinische Kurrent
\newline{}Ordnung: mit Bleistift von unbekannter Hand nummeriert:
                                    »155« }
\buchAbdrucke{\weitereDrucke{Arthur Schnitzler, Richard Beer-Hofmann: \emph{Briefwechsel 1891–1931}. Herausgegeben von Konstanze Fliedl. Wien, Zürich: \emph{Europaverlag} 1992, S. 147–148.} }\toendnotes[C]{\smallbreak}
\pstart
           \raggedleft{}{\pb}Alt-Aussee\oindex{Altaussee@\textbf{Altaussee}, \emph{Verwaltungsgebiet}|pw}{ }13/VII. 1900\pend
           \vspace{0.5em}
\pstart
           Lieber Arthur! Meiner Frau\pwindex{Beer-Hofmann, Paula 25.\,2.\,1879 Wien – 30.\,10.\,1939 Zürich@\textsc{Beer-Hofmann, Paula} (25.\,2.\,1879 Wien – 30.\,10.\,1939 Zürich)|pwv} geht es augenblicklich etwas besser. Seit 8 Tagen ko{\geminationm}t täglich der hiesige Doktor\pwindex{Engelhardt, Ludwig *~21.\,7.\,1867 Wien@\textsc{Engelhardt, Ludwig} (*~21.\,7.\,1867 Wien), \emph{Mediziner}|pwv}. Ob ein causaler Zusa{\geminationm}enhang zwischen beiden Sätzen besteht? Von Hugo\pwindex{Hofmannsthal, Hugo von 1.\,2.\,1874 Wien – 15.\,7.\,1929 Rodaun@\textsc{Hofmannsthal, Hugo von} (1.\,2.\,1874 Wien – 15.\,7.\,1929 Rodaun), \emph{Schriftsteller}|pw} ein Brief aus Bad-Fusch\oindex{Bad Fusch@\textbf{Bad Fusch}|pw}; er will
               Ihre Adresse. Von Goldmann\pwindex{Goldmann, Paul 31.\,1.\,1865 Breslau – 25.\,9.\,1935 Wien@\textsc{Goldmann, Paul} (31.\,1.\,1865 Breslau – 25.\,9.\,1935 Wien), \emph{Schriftsteller, Journalist}|pw} ein Brief wegen
               Fußtour. Wir fixiren also endgiltig (Schicksalsclauseln inbegriffen) den
                  15. August in Innsbruck\oindex{Innsbruck@\textbf{Innsbruck}, \emph{Verwaltungsgebiet}|pw}. Für den
               Zeitungsausschnitt Dank. Zur Beruhigung meines Papa\pwindex{Beer, Hermann 10.\,8.\,1835 Radiměř – 3.\,10.\,1902 Wien@\textsc{Beer, Hermann} (10.\,8.\,1835 Radiměř – 3.\,10.\,1902 Wien), \emph{Rechtsanwalt}|pwv}’s ganz gut. Meyer\pwindex{Mayer, Oskar 1876 – 15.\,5.\,1915 München@\textsc{Mayer, Oskar} (1876 – 15.\,5.\,1915 München), \emph{Schriftsteller, Beamter}|pw} war zu Besuch von Ischl\oindex{Bad Ischl@\textbf{Bad Ischl}|pw} hier, er
               will die Tour mitmachen. Er hat eine Unvorsichtigkeit begangen. »Die Hochzeit der Beatrice\pwindex{Schnitzler, Arthur 15.\,5.\,1862 Wien – 21.\,10.\,1931 ebd.@\textsc{Schnitzler, Arthur} (15.\,5.\,1862 Wien – 21.\,10.\,1931 ebd.), \emph{Schriftsteller, Mediziner}!Schleier der Beatrice. Schauspiel in fünf Akten@\strich\emph{Der Schleier der Beatrice. Schauspiel in fünf Akten}|pw}« hab ich ihm – wogegen Sie nichts
               hatten – geborgt. Nun setzt sich der Unglückliche in Marienbad\oindex{Marienbad@\textbf{Marienbad}|pw} auf {\pb}eine Bank,
               liest \introOben{}in\introOben{} dem Buch. Es erscheint: Minnie B.\pwindex{Schaffgotsch, Hermine von 25.\,11.\,1871 Wien – 25.\,11.\,1928 Purgstall@\textsc{Schaffgotsch, Hermine von} (25.\,11.\,1871 Wien – 25.\,11.\,1928 Purgstall)|pw} spricht M.\pwindex{Mayer, Oskar 1876 – 15.\,5.\,1915 München@\textsc{Mayer, Oskar} (1876 – 15.\,5.\,1915 München), \emph{Schriftsteller, Beamter}|pw} an
               erinnert ihn daß er \strikeout{S} sie eigentlich von einem Jour
               her kennen sollte, borgt sich das Buch aus; M.\pwindex{Mayer, Oskar 1876 – 15.\,5.\,1915 München@\textsc{Mayer, Oskar} (1876 – 15.\,5.\,1915 München), \emph{Schriftsteller, Beamter}|pw}
               wird zweimal zum Speisen geladen. Weiter: Minnie\pwindex{Schaffgotsch, Hermine von 25.\,11.\,1871 Wien – 25.\,11.\,1928 Purgstall@\textsc{Schaffgotsch, Hermine von} (25.\,11.\,1871 Wien – 25.\,11.\,1928 Purgstall)|pw} hat aber – (verdächtig) das Buch bei ihrer Abreise nach Levico\oindex{Levico Terme@\textbf{Levico Terme}|pw} noch nicht zu Ende gelesen, und erhält
               von M.\pwindex{Mayer, Oskar 1876 – 15.\,5.\,1915 München@\textsc{Mayer, Oskar} (1876 – 15.\,5.\,1915 München), \emph{Schriftsteller, Beamter}|pw} den Auftrag es nach Lesung mir zu
               schicken was sie noch nicht getan hat. M.\pwindex{Mayer, Oskar 1876 – 15.\,5.\,1915 München@\textsc{Mayer, Oskar} (1876 – 15.\,5.\,1915 München), \emph{Schriftsteller, Beamter}|pw} wird
               nun in meinem Namen urgieren damit ich das Buch \strikeout{kom}
                  beko{\geminationm}e. Hoffentlich haben bis dahin noch nicht die
                  versa{\geminationm}elten irgendwie nennenswerthen Curgäste in Levico\oindex{Levico Terme@\textbf{Levico Terme}|pw} bemerkt daß Sie Ihre unveröffentlichten
               Stücke Minnie\pwindex{Schaffgotsch, Hermine von 25.\,11.\,1871 Wien – 25.\,11.\,1928 Purgstall@\textsc{Schaffgotsch, Hermine von} (25.\,11.\,1871 Wien – 25.\,11.\,1928 Purgstall)|pw} anvertrauen. O
               Nachtkastelmotive. Bei alledem ärgert mich M.\pwindex{Mayer, Oskar 1876 – 15.\,5.\,1915 München@\textsc{Mayer, Oskar} (1876 – 15.\,5.\,1915 München), \emph{Schriftsteller, Beamter}|pw}’s Dummheit in dieser Sache. Er argumentirt: Da Sie mit Minnie\pwindex{Schaffgotsch, Hermine von 25.\,11.\,1871 Wien – 25.\,11.\,1928 Purgstall@\textsc{Schaffgotsch, Hermine von} (25.\,11.\,1871 Wien – 25.\,11.\,1928 Purgstall)|pw} gut {\pb}bekannt sind macht es nichts. Richtig muß es heißen: Da Sie gut bekannt sind und es
               ihr nicht geben, so wollen Sie eben nicht daß sie es hat. Außerdem ärgert mich: M.\pwindex{Mayer, Oskar 1876 – 15.\,5.\,1915 München@\textsc{Mayer, Oskar} (1876 – 15.\,5.\,1915 München), \emph{Schriftsteller, Beamter}|pw} auf dessen Verstand, Takt, und
               Geschicklichkeit ich einige Hoffnung setzte enttäuscht mich. Ob es denn mir einfiele
               ein als Manuscript gedrucktes Ding jungen Mädchen\pwindex{Schaffgotsch, Hermine von 25.\,11.\,1871 Wien – 25.\,11.\,1928 Purgstall@\textsc{Schaffgotsch, Hermine von} (25.\,11.\,1871 Wien – 25.\,11.\,1928 Purgstall)|pwv} in die Hand zu geben die – nach meiner Taxirung –
               gar kein wirkliches – außer persönliches – Interesse daran haben, und nur eine
               Primeurprotzerei damit anstellen wollen. Im Übrigen ist es wahrscheinlich nicht so
               wichtig.\pend
           
\pstart
           Wenn Sie Minnie\pwindex{Schaffgotsch, Hermine von 25.\,11.\,1871 Wien – 25.\,11.\,1928 Purgstall@\textsc{Schaffgotsch, Hermine von} (25.\,11.\,1871 Wien – 25.\,11.\,1928 Purgstall)|pw} einmal – damit die Leut Recht
               behalten – doch heirathen sollten wird dieser Brief mich nicht beliebt machen.\pend
           
\pstart
           Ich arbeite. Man überschätzt wie Sie sehen i{\geminationm}er noch die
               Menschen. Herzlich Ihr \spacefill\mbox{R.}\pend
           \selectlanguage{ngerman}\endnumbering\briefempfaengerindex{Schnitzler, Arthur@\textsc{Schnitzler, Arthur}!zzzBeer-Hofmann, Richard@\emph{von Richard Beer-Hofmann}!1900-07-131@{13. 7. 1900}|)be}\mylabel{L01053h}  \newcommand{\dateiname}{L01053}\newcommand{\titel}{Richard Beer-Hofmann an Arthur Schnitzler, 13. 7. 1900}\newcommand{\editorInnen}{Martin Anton Müller und Gerd-Hermann Susen}%% latex-leseansicht-abspann.tex
%% Abspann für die Leseansicht.
%% Der Schalter \ifkorrekturansicht ist bereits durch den Vorspann gesetzt.

%% latex-abspann.tex
%% Gemeinsamer Abspann für Korrekturansicht und Leseansicht.
%% Setzt den Schalter \ifkorrekturansicht voraus (gesetzt in den
%% einbindenden Dateien latex-korrekturansicht-abspann.tex bzw.
%% latex-leseansicht-abspann.tex).
%% ---------------------------------------------------------------

\normalsize

% Das esempio-Environment wird nur in der Leseansicht benötigt
\ifkorrekturansicht\else
\newenvironment{esempio}[3]%
{
    \vspace{1.5ex}
    \rlap{\underline{#1}}
    \par
    \setlength{\parindent}{0cm}
    \nopagebreak
    \leftskip=#2cm
    \rightskip=#3cm
}
{
    \par
}
\fi

\doendnotes{C}
\bigskip
\vfill

\clearpage

\footnotesize

\ifkorrekturansicht
  \lohead{\textsc{register}}
\fi

% theindex-Environment neu definieren ohne reledmac
\makeatletter
\renewenvironment{theindex}{%
  \ifkorrekturansicht
    \section*{\indexname}%
  \else
    \subsubsection*{Index der erwähnten Entitäten}%
  \fi
  \setlength{\parindent}{0pt}%
  \setlength{\parskip}{0pt plus 0.3pt}%
  \let\item\@idxitem
}{%
  \ifkorrekturansicht\clearpage\fi
}
\makeatother

\IfFileExists{\jobname-pw.ind}{\input{\jobname-pw.ind}}{}

% Quellenangabe nur in der Leseansicht
\ifkorrekturansicht\else
% Fallback-Definitionen, falls die .tex-Datei \titel etc. nicht gesetzt hat
\providecommand{\titel}{}
\providecommand{\editorInnen}{}
\providecommand{\dateiname}{\jobname}

\vspace{3cm}

\vfill

\footnotesize
\textsc{Quelle}: \titel. Herausgegeben von {\editorInnen}. In: \emph{Arthur Schnitzler: Briefwechsel mit Autorinnen und Autoren}.
 Digitale Edition, https://schnitzler-briefe.acdh.oeaw.ac.at/{\dateiname}.html (Stand \today)
\fi

\end{document}


