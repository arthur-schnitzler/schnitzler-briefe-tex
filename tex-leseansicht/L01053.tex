%% latex-leseansicht-vorspann.tex
%% Vorspann für die Leseansicht.
%% Lädt die gemeinsame Datei latex-vorspann.tex mit nicht gesetztem Schalter.

\newif\ifkorrekturansicht
\korrekturansichtfalse

\input{../tex-inputs/latex-vorspann}


         
         \renewcommand{\erwaehntePersonen}{Personen: Hermann Beer, Paula Beer-Hofmann, Ludwig Engelhardt, Paul Goldmann, Hugo von Hofmannsthal, Oskar Mayer, Hermine von Schaffgotsch}
         \renewcommand{\erwaehnteOrte}{Orte: Altaussee, Bad Fusch, Bad Ischl, Innsbruck, Levico Terme, Marienbad, Reichenau an der Rax}
         \renewcommand{\erwaehnteWerke}{Werke: Der Schleier der Beatrice. Schauspiel in fünf Akten}
               \section[Richard Beer-Hofmann an Arthur Schnitzler, 13. 7. 1900]{ Richard Beer-Hofmann an Arthur Schnitzler, 13. 7. 1900}\nopagebreak\mylabel{v}\rehead{ }\begin{ledgroupsized}[t]{13cm}\normalsize\beginnumbering \toendnotes[C]{\smallbreak\pagebreak[2]} \Standort{CUL, Schnitzler, B 8.}
\physDesc{Brief, 2 Blätter, 3 Seiten
\newline{}Handschrift: schwarze Tinte, lateinische Kurrent\newline{}Ordnung: mit Bleistift von unbekannter Hand nummeriert: »155« }\buchAbdrucke{\weitereDrucke{Arthur Schnitzler, Richard Beer-Hofmann: \emph{Briefwechsel 1891–1931}. Hg. Konstanze Fliedl. Wien, Zürich: \emph{Europaverlag} 1992, S. 147–148.} }\toendnotes[C]{\smallbreak}\pstart
           \raggedleft{}{\pb}Alt-Aussee\oindex{Altaussee@\textbf{Altaussee}|pw}{ }13/VII. 1900\pend
           \pstart
           Lieber Arthur! Meiner Frau\pwindex{Beer-Hofmann, Paula 25.02.1879 – 30.10.1939@\textsc{Beer-Hofmann, Paula} (25.02.1879 – 30.10.1939)|pwv} geht es augenblicklich etwas besser. Seit 8 Tagen ko{\geminationm}t täglich der hiesige Doktor\pwindex{Engelhardt, Ludwig *~1867-07-21@\textsc{Engelhardt, Ludwig} (*~1867-07-21), \emph{Mediziner}|pwv}. Ob ein causaler Zusa{\geminationm}enhang zwischen beiden Sätzen besteht? Von Hugo\pwindex{Hofmannsthal, Hugo von 1874-02-01 – 1929-07-15@\textsc{Hofmannsthal, Hugo von} (1874-02-01 – 1929-07-15), \emph{Schriftsteller}|pw}
               ein Brief aus Bad-Fusch\oindex{Bad Fusch@\textbf{Bad Fusch}|pw}; er will Ihre Adresse. Von
                  Goldmann\pwindex{Goldmann, Paul 31.01.1865 – 25.09.1935@\textsc{Goldmann, Paul} (31.01.1865 – 25.09.1935), \emph{Schriftsteller, Journalist}|pw} ein Brief wegen Fußtour. Wir fixiren
               also endgiltig (Schicksalsclauseln inbegriffen) den 15. August in Innsbruck\oindex{Innsbruck@\textbf{Innsbruck}|pw}. Für den Zeitungsausschnitt Dank. Zur
               Beruhigung meines Papa\pwindex{Beer, Hermann 10.8.1835 – 03.10.1902@\textsc{Beer, Hermann} (10.8.1835 – 03.10.1902), \emph{Rechtsanwalt}|pwv}’s ganz
               gut. Meyer\pwindex{Mayer, Oskar 1876 – 15.05.1915@\textsc{Mayer, Oskar} (1876 – 15.05.1915), \emph{Schriftsteller, Beamter}|pw} war zu Besuch von Ischl\oindex{Bad Ischl@\textbf{Bad Ischl}|pw} hier, er will die Tour mitmachen. Er hat eine
               Unvorsichtigkeit begangen. »Die Hochzeit der
                  Beatrice\pwindex{Schnitzler, Arthur 15.05.1862 – 21.10.1931@\textsc{Schnitzler, Arthur} (15.05.1862 – 21.10.1931), \emph{Schriftsteller, Mediziner}!Schleier der Beatrice. Schauspiel in fuenf Akten1900-12-01@\strich\emph{Der Schleier der Beatrice. Schauspiel in fünf Akten} {[}1900-12-01{]}|pw}« hab ich ihm – wogegen Sie nichts hatten – geborgt. Nun setzt sich
               der Unglückliche in Marienbad\oindex{Marienbad@\textbf{Marienbad}|pw} auf {\pb}eine Bank, liest \introOben{}in\introOben{} dem Buch. Es erscheint: Minnie B.\pwindex{Schaffgotsch, Hermine von 25.11.1871 – 25.11.1928@\textsc{Schaffgotsch, Hermine von} (25.11.1871 – 25.11.1928)|pw}
               spricht M.\pwindex{Mayer, Oskar 1876 – 15.05.1915@\textsc{Mayer, Oskar} (1876 – 15.05.1915), \emph{Schriftsteller, Beamter}|pw} an erinnert ihn daß er \strikeout{S} sie eigentlich von einem Jour her kennen sollte,
               borgt sich das Buch aus; M.\pwindex{Mayer, Oskar 1876 – 15.05.1915@\textsc{Mayer, Oskar} (1876 – 15.05.1915), \emph{Schriftsteller, Beamter}|pw} wird zweimal zum
               Speisen geladen. Weiter: Minnie\pwindex{Schaffgotsch, Hermine von 25.11.1871 – 25.11.1928@\textsc{Schaffgotsch, Hermine von} (25.11.1871 – 25.11.1928)|pw} hat aber –
               (verdächtig) das Buch bei ihrer Abreise nach Levico\oindex{Levico Terme@\textbf{Levico Terme}|pw} noch nicht zu Ende gelesen, und erhält von M.\pwindex{Mayer, Oskar 1876 – 15.05.1915@\textsc{Mayer, Oskar} (1876 – 15.05.1915), \emph{Schriftsteller, Beamter}|pw} den Auftrag es nach Lesung mir zu schicken was sie noch
               nicht getan hat. M.\pwindex{Mayer, Oskar 1876 – 15.05.1915@\textsc{Mayer, Oskar} (1876 – 15.05.1915), \emph{Schriftsteller, Beamter}|pw} wird nun in meinem Namen
               urgieren damit ich das Buch \strikeout{kom} beko{\geminationm}e. Hoffentlich
               haben bis dahin noch nicht die versa{\geminationm}elten irgendwie
               nennenswerthen Curgäste in Levico\oindex{Levico Terme@\textbf{Levico Terme}|pw} bemerkt daß Sie
               Ihre unveröffentlichten Stücke Minnie\pwindex{Schaffgotsch, Hermine von 25.11.1871 – 25.11.1928@\textsc{Schaffgotsch, Hermine von} (25.11.1871 – 25.11.1928)|pw}
               anvertrauen. O Nachtkastelmotive. Bei alledem ärgert mich M.\pwindex{Mayer, Oskar 1876 – 15.05.1915@\textsc{Mayer, Oskar} (1876 – 15.05.1915), \emph{Schriftsteller, Beamter}|pw}’s Dummheit in dieser Sache. Er argumentirt: Da Sie mit Minnie\pwindex{Schaffgotsch, Hermine von 25.11.1871 – 25.11.1928@\textsc{Schaffgotsch, Hermine von} (25.11.1871 – 25.11.1928)|pw} gut {\pb}bekannt sind macht es nichts.
               Richtig muß es heißen: Da Sie gut bekannt sind und es ihr nicht geben, so wollen Sie
               eben nicht daß sie es hat. Außerdem ärgert mich: M.\pwindex{Mayer, Oskar 1876 – 15.05.1915@\textsc{Mayer, Oskar} (1876 – 15.05.1915), \emph{Schriftsteller, Beamter}|pw} auf dessen Verstand, Takt, und Geschicklichkeit ich einige Hoffnung
               setzte enttäuscht mich. Ob es denn mir einfiele ein als Manuscript gedrucktes Ding
               jungen Mädchen\pwindex{Schaffgotsch, Hermine von 25.11.1871 – 25.11.1928@\textsc{Schaffgotsch, Hermine von} (25.11.1871 – 25.11.1928)|pwv} in die Hand zu
               geben die – nach meiner Taxirung – gar kein wirkliches – außer persönliches –
               Interesse daran haben, und nur eine Primeurprotzerei damit anstellen wollen. Im
               Übrigen ist es wahrscheinlich nicht so wichtig.\pend
           \pstart
           Wenn Sie Minnie\pwindex{Schaffgotsch, Hermine von 25.11.1871 – 25.11.1928@\textsc{Schaffgotsch, Hermine von} (25.11.1871 – 25.11.1928)|pw} einmal – damit die Leut Recht
               behalten – doch heirathen sollten wird dieser Brief mich nicht beliebt machen.\pend
           \pstart
           Ich arbeite. Man überschätzt wie Sie sehen i{\geminationm}er noch die
               Menschen. Herzlich Ihr \spacefill\mbox{R.}\pend
           
         
         \endnumbering\mylabel{h}\end{ledgroupsized}  \newcommand{\dateiname}{L01053}\newcommand{\titel}{Richard Beer-Hofmann an Arthur Schnitzler, 13. 7. 1900}\newcommand{\editorInnen}{Martin Anton Müller und Gerd-Hermann Susen}%% latex-leseansicht-abspann.tex
%% Abspann für die Leseansicht.
%% Der Schalter \ifkorrekturansicht ist bereits durch den Vorspann gesetzt.

%% latex-abspann.tex
%% Gemeinsamer Abspann für Korrekturansicht und Leseansicht.
%% Setzt den Schalter \ifkorrekturansicht voraus (gesetzt in den
%% einbindenden Dateien latex-korrekturansicht-abspann.tex bzw.
%% latex-leseansicht-abspann.tex).
%% ---------------------------------------------------------------

\normalsize

% Das esempio-Environment wird nur in der Leseansicht benötigt
\ifkorrekturansicht\else
\newenvironment{esempio}[3]%
{
    \vspace{1.5ex}
    \rlap{\underline{#1}}
    \par
    \setlength{\parindent}{0cm}
    \nopagebreak
    \leftskip=#2cm
    \rightskip=#3cm
}
{
    \par
}
\fi

\doendnotes{C}
\bigskip
\vfill

\clearpage

\footnotesize

\ifkorrekturansicht
  \lohead{\textsc{register}}
\fi

% theindex-Environment neu definieren ohne reledmac
\makeatletter
\renewenvironment{theindex}{%
  \ifkorrekturansicht
    \section*{\indexname}%
  \else
    \subsubsection*{Index der erwähnten Entitäten}%
  \fi
  \setlength{\parindent}{0pt}%
  \setlength{\parskip}{0pt plus 0.3pt}%
  \let\item\@idxitem
}{%
  \ifkorrekturansicht\clearpage\fi
}
\makeatother

\IfFileExists{\jobname-pw.ind}{\input{\jobname-pw.ind}}{}

% Quellenangabe nur in der Leseansicht
\ifkorrekturansicht\else
% Fallback-Definitionen, falls die .tex-Datei \titel etc. nicht gesetzt hat
\providecommand{\titel}{}
\providecommand{\editorInnen}{}
\providecommand{\dateiname}{\jobname}

\vspace{3cm}

\vfill

\footnotesize
\textsc{Quelle}: \titel. Herausgegeben von {\editorInnen}. In: \emph{Arthur Schnitzler: Briefwechsel mit Autorinnen und Autoren}.
 Digitale Edition, https://schnitzler-briefe.acdh.oeaw.ac.at/{\dateiname}.html (Stand \today)
\fi

\end{document}


      