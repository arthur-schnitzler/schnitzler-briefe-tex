%% latex-korrekturansicht-vorspann.tex
%% Vorspann für die Korrekturansicht.
%% Lädt die gemeinsame Datei latex-vorspann.tex mit gesetztem Schalter.

\newif\ifkorrekturansicht
\korrekturansichttrue

\input{../tex-inputs/latex-vorspann}


\section[Hermann Bahr an Arthur Schnitzler, 27. 4. 1904]{L01396 Hermann Bahr an Arthur Schnitzler, 27. 4. 1904}
\nopagebreak\mylabel{L01396v}
\rehead{ }\normalsize\beginnumbering\briefempfaengerindex{Schnitzler, Arthur@\textsc{Schnitzler, Arthur}!zzzBahr, Hermann@\emph{von Hermann Bahr}!1904-04-271@{27. 4. 1904}|(be}
\toendnotes[C]{\smallbreak\pagebreak[2]}\Standort{CUL, Schnitzler, B 5b.}
\physDesc{Kartenbrief, 452 Zeichen
\newline{}Handschrift: schwarze Tinte, deutsche Kurrent
\newline{}Versand: 1) Rohrpost  2) Stempel: »\nobreak{}\oindex{XIII., Hietzing@\textbf{XIII., Hietzing}, \emph{A.ADM3}|pwk}Wien 13/5, 27{[}.{]} IV. 04, XII\nobreak{}«.  3) Stempel: »\nobreak{}\oindex{XII., Meidling@\textbf{XII., Meidling}, \emph{A.ADM3}|pwk}Wien 12/1, 27 IV 04, 1 N\nobreak{}«.  4) Stempel: »\nobreak{}\oindex{XII., Meidling@\textbf{XII., Meidling}, \emph{A.ADM3}|pwk}Wien 12/1, 27 IV {[}04{]}, 2.30N\nobreak{}«. 
\newline{}Schnitzler: mit Bleistift die Jahreszahl zum Datum ergänzt: »904« 
\newline{}Ordnung: mit Bleistift von unbekannter Hand nummeriert: »117« }
\buchAbdrucke{\weitereDrucke{Hermann Bahr, Arthur Schnitzler: \emph{Briefwechsel, Aufzeichnungen, Dokumente (1891–1931)}. Göttingen: \emph{Wallstein} 2018, S. 306.} }\toendnotes[C]{\smallbreak}\pstart{}{\pb}Pneumatisch\pend{}\pstart{}Herrn \textsc{D\textsuperscript{r} Arthur
                     Schnitzler}\pend{}\pstart{}\textsc{Wien XVIII}\oindex{XVIII., Waehring@\textbf{XVIII., Währing}, \emph{A.ADM3}|pw}\pend{}\pstart{}\textsc{Spöttelgasse 7}\oindex{Edmund-Weiss-Gasse 7@\textbf{Edmund-Weiß-Gasse 7}, \emph{Wohngebäude (K.WHS)}|pw}\pend{}{\bigskip}\vspace{1em}
\pstart
           \raggedleft{}{\pb}27. 4.\pend
           
\pstart{}Lieber Arthur!\pend\vspace{0.5em}
\pstart
           Herzlichſten Dank für Deinen \label{LL279-1v}Brief\label{LL279-1h},
               der ſich mit meinem an Dich gekreuzt hat. Ich wollte nun heute abends nach Hietzing\oindex{XIII., Hietzing@\textbf{XIII., Hietzing}, \emph{A.ADM3}|pw} kommen. Da mir nun aber \label{K_L01396-1v}\edtext{Gerty\pwindex{Hofmannsthal, Gertrude von 16.03.1880 – 09.11.1959@\textsc{Hofmannsthal, Gertrude von} (16.03.1880 – 09.11.1959)|pw}{ }ſchreibt}{\lemma{\textnormal{\emph{Gerty ſchreibt}}}\Cendnote{\textnormal{Dieser Brief ist nicht überliefert und nicht in: Hugo und
                     Gerty von Hofmannsthal – Hermann Bahr: \emph{Briefwechsel
                        1891–1934}. Herausgegeben und kommentiert von Elsbeth
                     Dangel-Pelloquin. Göttingen:
                        \emph{Wallstein}{ }2013 abgedruckt. }}}\label{K_L01396-1}, Hugo\pwindex{Hofmannsthal, Hugo von 1874-02-01 – 1929-07-15@\textsc{Hofmannsthal, Hugo von} (1874-02-01 – 1929-07-15), \emph{Schriftsteller/Schriftstellerin}|pw}{ }ſei auf dem Semmering\oindex{Semmering@\textbf{Semmering}, \emph{A.ADM3}|pw}, denke ich, daß Du wol auch nicht kommen wirſt, und bitte um ein
               anderes Rendezvous, da ich Dich ſehr gern vor Deiner \label{K_L01396-2v}\edtext{Abreiſe}{\lemma{\textnormal{\emph{Abreiſe}}}\Cendnote{\textnormal{Am 30. 4. 1904 trat Schnitzler eine mehrwöchige Italien\oindex{Italien@\textbf{Italien}, \emph{A.PCLI}|pwk}reise an.}}}\label{K_L01396-2} noch ſehen
               möchte.\pend
           
\pstart
           Mit den beſten Grüßen an Deine Frau\pwindex{Schnitzler, Olga 17.01.1882 – 13.01.1970@\textsc{Schnitzler, Olga} (17.01.1882 – 13.01.1970), \emph{Schauspieler/Schauspielerin, Sänger/Sängerin}|pwv}{\\[\baselineskip]}herzlichſt{\\[\baselineskip]}\spacefill\mbox{HermB.}\pend
           \leftskip=0em{}\selectlanguage{ngerman}\endnumbering\briefempfaengerindex{Schnitzler, Arthur@\textsc{Schnitzler, Arthur}!zzzBahr, Hermann@\emph{von Hermann Bahr}!1904-04-271@{27. 4. 1904}|)be}\mylabel{L01396h}  \normalsize

\doendnotes{C}
\bigskip
\vfill

\clearpage

\footnotesize

\lohead{\textsc{register}}

% Definiere theindex-Environment komplett neu ohne reledmac
\makeatletter
\renewenvironment{theindex}{%
  \section*{\indexname}%
  \setlength{\parindent}{0pt}%
  \setlength{\parskip}{0pt plus 0.3pt}%
  \let\item\@idxitem
}{%
  \clearpage
}
\makeatother

\IfFileExists{\jobname-pw.ind}{\input{\jobname-pw.ind}}{}

\end{document}

      