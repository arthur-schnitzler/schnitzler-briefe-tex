%% latex-korrekturansicht-vorspann.tex
%% Vorspann für die Korrekturansicht.
%% Lädt die gemeinsame Datei latex-vorspann.tex mit gesetztem Schalter.

\newif\ifkorrekturansicht
\korrekturansichttrue

\input{../tex-inputs/latex-vorspann}


\section[Felix Salten an Arthur Schnitzler, {[}3. 7. 1894{]}]{L03140 Felix Salten an Arthur Schnitzler, {[}3. 7. 1894{]}}
\nopagebreak\mylabel{L03140v}
\rehead{ }\normalsize\beginnumbering\briefempfaengerindex{Schnitzler, Arthur@\textsc{Schnitzler, Arthur}!zzzSalten, Felix@\emph{von Felix Salten}!1894-07-032@{{[}3. 7. 1894{]}}|(be}
\toendnotes[C]{\smallbreak\pagebreak[2]}\Standort{CUL, Schnitzler, B 89, A 1.}
\physDesc{Brief, 1 Blatt, 1 Seite, 194 Zeichen
\newline{}Handschrift: Bleistift, lateinische Kurrent
\newline{}Schnitzler: mit Bleistift datiert: »3/7 94.« 
\newline{}Ordnung: mit Bleistift von unbekannter Hand nummeriert: »41« }\toendnotes[C]{\smallbreak}
\pstart
           \noindent{}{\pb}Lieber Freund, ich muss diesen Nachmittag
               u. Abd dem \label{K_L03140-1v}\edtext{Abschied}{\lemma{\textnormal{\emph{Abschied}}}\Cendnote{\textnormal{Es dürfte sich nicht
                  um eine Reise Saltens\pwindex{Salten, Felix 06.09.1869 – 08.10.1945@\textsc{Salten, Felix} (06.09.1869 – 08.10.1945), \emph{Schriftsteller/Schriftstellerin, Journalist/Journalistin, Chefredakteur/Chefredakteurin}|pwk} handeln, da er 8. 7. 1894 in Wien\oindex{Wien@\textbf{Wien}, \emph{A.ADM2}|pwk}
               war. Eventuell übersiedelte er oder wechselte den Arbeitsplatz.}}}\label{K_L03140-1} widmen. Werde aber
               vermuthlich gegen 10 Uhr ins Arcadencafé\oindex{Cafe Arkaden@\textbf{Café Arkaden}, \emph{Kaffeehaus (K.KAF)}|pw} kommen.\pend
           
\pstart
           Hoffentlich bringen wir das Versäumte noch reichlich ein\textcolor{gray}{.}\pend
           
\pstart
           Herzlichst {\\[\baselineskip]}Ihr {\\[\baselineskip]}\spacefill\mbox{Salten}\pend
           \leftskip=0em{}\selectlanguage{ngerman}\endnumbering\briefempfaengerindex{Schnitzler, Arthur@\textsc{Schnitzler, Arthur}!zzzSalten, Felix@\emph{von Felix Salten}!1894-07-032@{{[}3. 7. 1894{]}}|)be}\mylabel{L03140h}  \normalsize

\doendnotes{C}
\bigskip
\vfill

\clearpage

\footnotesize

\lohead{\textsc{register}}

% Definiere theindex-Environment komplett neu ohne reledmac
\makeatletter
\renewenvironment{theindex}{%
  \section*{\indexname}%
  \setlength{\parindent}{0pt}%
  \setlength{\parskip}{0pt plus 0.3pt}%
  \let\item\@idxitem
}{%
  \clearpage
}
\makeatother

\IfFileExists{\jobname-pw.ind}{\input{\jobname-pw.ind}}{}

\end{document}

      