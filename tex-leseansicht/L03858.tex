%% latex-leseansicht-vorspann.tex
%% Vorspann für die Leseansicht.
%% Lädt die gemeinsame Datei latex-vorspann.tex mit nicht gesetztem Schalter.

\newif\ifkorrekturansicht
\korrekturansichtfalse

\input{../tex-inputs/latex-vorspann}


\section[Theodor Herzl an Arthur Schnitzler, 16. 4. 1895]{L03858 Theodor Herzl an Arthur Schnitzler, 16. 4. 1895}
\nopagebreak\mylabel{L03858v}
\rehead{ }\normalsize\beginnumbering\briefempfaengerindex{Schnitzler, Arthur@\textsc{Schnitzler, Arthur}!zzzHerzl, Theodor@\emph{von Theodor Herzl}!1895-04-161@{16. 4. 1895}|(be}
\toendnotes[C]{\smallbreak\pagebreak[2]}
\correspDesc{Versand  durch Theodor Herzl am 16. 4. 1895 in Paris
\newline{}Erhalt  durch Arthur Schnitzler im Zeitraum [17. 4. 1895 – 21. 4. 1895?] in Wien}\toendnotes[C]{\smallbreak}
\Standort{CUL, Schnitzler, B 39.}
\physDesc{Brief, 1 Blatt, 1 Seite, 423 Zeichen
\newline{}Handschrift: schwarze Tinte, lateinische Kurrent
\newline{}Ordnung: mit Bleistift von unbekannter Hand nummeriert: »37« }
\buchAbdrucke{\weitereDrucke{Theodor Herzl: \emph{Briefe und autobiographische Notizen 1866–1895}. Bearbeitet von Johannes Wachten in Zusammenarbeit mit Chaya Harel, Daisy Tycho und Manfred Winkler. Berlin, Frankfurt am Main, Wien: \emph{Propyläen} 1983, S. 583 (Briefe und Tagebücher. Herausgegeben von Alex Bein, Hermann Greive, Moshe Schaerf, Julius H. Schoeps und Johannes Wachten, 1).} }\toendnotes[C]{\smallbreak}
\pstart
           {\pb}\textcolor{gray}{\textbf{NOUVELLE PRESSE LIBRE\orgindex{Neue Freie Presse@Neue Freie Presse|pw}}}\hfill \textcolor{gray}{\textbf{8, RUE DE MONCEAU\oindex{8, rue de Monceau@\textbf{8, rue de Monceau}, \emph{Wohngebäude}|pw}}}\pend
           
\pstart
           \textcolor{gray}{\textbf{
                        D\textsuperscript{r}{ }TH. HERZL}}\hfill 16. IV. 95\pend
           
\pstart{}Lieber Freund,\pend\vspace{0.5em}
\pstart
           zwischen Kisten u. Koffern diese zwei Worte.\pend
           
\pstart
           Ich wohne fortab\pend
           
\pstart
           \centering{}\uline{Hotel de Castille rue
                     Cambon\oindex{37, Rue Cambon@\textbf{37, Rue Cambon}, \emph{Wohngebäude}|pw}}\pend
           
\pstart
           für klare Briefe. \label{K_L03858-1v}\edtext{Schnabels
                  Adresse}{\lemma{\textnormal{\emph{Schnabels
                  Adresse}}}\Cendnote{\textnormal{Zu Herzls\pwindex{Herzl, Theodor 2.\,5.\,1860 Budapest – 3.\,7.\,1904 Edlach@\textsc{Herzl, Theodor} (2.\,5.\,1860 Budapest – 3.\,7.\,1904 Edlach), \emph{Schriftsteller, Journalist}|pwk} Vorgaben für die klandestine Kommunikation über
                  sein Stück\pwindex{Herzl, Theodor 2.\,5.\,1860 Budapest – 3.\,7.\,1904 Edlach@\textsc{Herzl, Theodor} (2.\,5.\,1860 Budapest – 3.\,7.\,1904 Edlach), \emph{Schriftsteller, Journalist}!neue Ghetto. Schauspiel in vier Acten@\strich\emph{Das neue Ghetto. Schauspiel in vier Acten}|pwkv}, das er unter
                  dem Pseudonym Albert Schnabel geschrieben hatte, vgl. XXXX Auszeichnungsfehler: Dokument L03836 nicht gefunden.}}}\label{K_L03858-1} fortab: \label{K_L03858-2v}\edtext{\begin{otherlanguage}{french}poste restante\end{otherlanguage}}{\lemma{\textnormal{\emph{poste restante}}}\Cendnote{\textnormal{französisch: postlagernd}}}\label{K_L03858-2}{ }bureau de la Madeleine\oindex{place de la Madeleine@\textbf{place de la Madeleine}, \emph{Platz}|pw}. Die \label{K_L03858-3v}\edtext{Nummer des Bureaus\oindex{place de la Madeleine@\textbf{place de la Madeleine}, \emph{Platz}|pwv} vergass ich bisher nachzusehen, schreibe sie
                  nächstens}{\lemma{\textnormal{\emph{Nummer … nächstens}}}\Cendnote{\textnormal{Siehe XXXX Auszeichnungsfehler: Dokument L03859 nicht gefunden.}}}\label{K_L03858-3}.\pend
           
\pstart
           Sie vergessen doch nicht, lieber Freund, dass alles was ich Ihnen über mich u. meine
               Zukunftspläne sagte strenges Geheimniss bleiben muss.\pend
           
\pstart
           Herzlich Ihr{\\[\baselineskip]}\spacefill\mbox{Th. H.}\pend
           \leftskip=0em{}\selectlanguage{ngerman}\endnumbering\briefempfaengerindex{Schnitzler, Arthur@\textsc{Schnitzler, Arthur}!zzzHerzl, Theodor@\emph{von Theodor Herzl}!1895-04-161@{16. 4. 1895}|)be}\mylabel{L03858h}
\begin{anhang}
\end{anhang}\newcommand{\dateiname}{L03858}\newcommand{\titel}{Theodor Herzl an Arthur Schnitzler, 16. 4. 1895}\newcommand{\editorInnen}{Selma Jahnke und Martin Anton Müller}%% latex-leseansicht-abspann.tex
%% Abspann für die Leseansicht.
%% Der Schalter \ifkorrekturansicht ist bereits durch den Vorspann gesetzt.

%% latex-abspann.tex
%% Gemeinsamer Abspann für Korrekturansicht und Leseansicht.
%% Setzt den Schalter \ifkorrekturansicht voraus (gesetzt in den
%% einbindenden Dateien latex-korrekturansicht-abspann.tex bzw.
%% latex-leseansicht-abspann.tex).
%% ---------------------------------------------------------------

\normalsize

% Das esempio-Environment wird nur in der Leseansicht benötigt
\ifkorrekturansicht\else
\newenvironment{esempio}[3]%
{
    \vspace{1.5ex}
    \rlap{\underline{#1}}
    \par
    \setlength{\parindent}{0cm}
    \nopagebreak
    \leftskip=#2cm
    \rightskip=#3cm
}
{
    \par
}
\fi

\doendnotes{C}
\bigskip
\vfill

\clearpage

\footnotesize

\ifkorrekturansicht
  \lohead{\textsc{register}}
\fi

% theindex-Environment neu definieren ohne reledmac
\makeatletter
\renewenvironment{theindex}{%
  \ifkorrekturansicht
    \section*{\indexname}%
  \else
    \subsubsection*{Index der erwähnten Entitäten}%
  \fi
  \setlength{\parindent}{0pt}%
  \setlength{\parskip}{0pt plus 0.3pt}%
  \let\item\@idxitem
}{%
  \ifkorrekturansicht\clearpage\fi
}
\makeatother

\IfFileExists{\jobname-pw.ind}{\input{\jobname-pw.ind}}{}

% Quellenangabe nur in der Leseansicht
\ifkorrekturansicht\else
% Fallback-Definitionen, falls die .tex-Datei \titel etc. nicht gesetzt hat
\providecommand{\titel}{}
\providecommand{\editorInnen}{}
\providecommand{\dateiname}{\jobname}

\vspace{3cm}

\vfill

\footnotesize
\textsc{Quelle}: \titel. Herausgegeben von {\editorInnen}. In: \emph{Arthur Schnitzler: Briefwechsel mit Autorinnen und Autoren}.
 Digitale Edition, https://schnitzler-briefe.acdh.oeaw.ac.at/{\dateiname}.html (Stand \today)
\fi

\end{document}


