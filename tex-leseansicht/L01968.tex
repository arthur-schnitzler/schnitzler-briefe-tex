%% latex-korrekturansicht-vorspann.tex
%% Vorspann für die Korrekturansicht.
%% Lädt die gemeinsame Datei latex-vorspann.tex mit gesetztem Schalter.

\newif\ifkorrekturansicht
\korrekturansichttrue

\input{../tex-inputs/latex-vorspann}


\section[Hugo von Hofmannsthal an Arthur Schnitzler, 20. 10. {[}1910{]}]{L01968 Hugo von Hofmannsthal an Arthur Schnitzler, 20. 10. {[}1910{]}}
\nopagebreak\mylabel{L01968v}
\rehead{ }\normalsize\beginnumbering\briefempfaengerindex{Schnitzler, Arthur@\textsc{Schnitzler, Arthur}!zzzHofmannsthal, Hugo von@\emph{von Hugo von Hofmannsthal}!1910-10-202@{20. 10. 1910}|(be}
\toendnotes[C]{\smallbreak\pagebreak[2]}\Standort{CUL, Schnitzler, B 43.}
\physDesc{Brief, 1 Blatt, 4 Seiten, 1680 Zeichen
\newline{}Handschrift: schwarze Tinte, deutsche Kurrent
\newline{}Schnitzler: mit Bleistift die Jahreszahl ergänzt: »910« und beschriftet: »\textsc{Hofmannsthal}« 
\newline{}Ordnung: 1) mit Bleistift von unbekannter Hand nummeriert: »\strikeout{318}«  2) mit Bleistift von unbekannter Hand nummeriert:
                                    »323«}
\buchAbdrucke{\weitereDrucke{Hugo von Hofmannsthal, Arthur Schnitzler: \emph{Briefwechsel}. Frankfurt am Main: \emph{S. Fischer} 1964, S. 254.} }\toendnotes[C]{\smallbreak}
\pstart
           \raggedleft{}{\pb}Rod.\oindex{Rodaun@\textbf{Rodaun}, \emph{A.ADM4}|pw}{ }20. X\pend
           \vspace{0.5em}
\pstart
           mein guter Arthur,\hspace*{1.5em}vielmals danke ich Ihnen für Ihren Brief und Ihre
               Depeſche nach Neubeuern\oindex{Neubeuern@\textbf{Neubeuern}, \emph{P.PPL}|pw} (wo wir 2
               unvergleichlich ſchöne und wirklich ſehr glückerfüllte \label{K_L01968-1v}\edtext{Herbſtwochen}{\lemma{\textnormal{\emph{Herbſtwochen}}}\Cendnote{\textnormal{vom 4. 10. 1910 bis zum
                     16. 10. 1910}}}\label{K_L01968-1} zubrachten) für Ihre Hilfe in der Beſetzungsſache und vor allem für die
               schönen Stunden, die mir Ihr neues {\pb}Stück\pwindex{weite Land. Tragikomoedie in fuenf Akten@\emph{Das weite Land. Tragikomödie in fünf Akten}|pwv} geſchenkt hat. Ich
               glaube, dieſes »weite Land\pwindex{weite Land. Tragikomoedie in fuenf Akten@\emph{Das weite Land. Tragikomödie in fünf Akten}|pw}« iſt wirklich die
               allerbeſte Arbeit Ihrer an guten Arbeiten ſo reichen zweiten Lebens- oder
               Arbeitsperiode.\pend
           
\pstart
           Das Stück gehört ſo ganz Ihnen, und iſt dabei ſo äußerſt kräftig, ſo wunderſchön
               zuſammengehalten. Alle Ihre nicht leicht in einem Athem aufzuzählenden Vorzüge: das
               ſo ganz perſönliche Lebensgefühl, die höchſt beſondere Scala der Wertungen, {\pb}die zarte und ſichere Geſtaltung,
               die leichte Hand für die Scenenführung, die Melancholie und der Witz, der höchſt
               nötige \textsc{bon sens}, normaler (aber ſeltener) Menſchenverſtand,
               und dazu das tiefere poetiſch-philoſophiſche Zuſammenſehen und Nebeneinanderſehen,
               die Güte, die Erfahrung und zugleich ein entzückender Mangel an Routine, ein
               Friſches, Blühendes, Geſpanntes überall – dies alles ko{\geminationm}t zuſammen, um ein {\pb}Werk
               herzuſtellen, das ſich in unvergleichlicher Weiſe im Gleichgewicht hält, weltlich und
               tief, theatermäßig und philoſophiſch, amüſant und bedeutend iſt.\hspace*{1.5em}Ich freue mich ſehr, es auch noch auf der Bühne zu ſehen – doch hab
               ich es auf der inneren Bühne tadellos beſetzt und ſehr ſchön mir aufgeführt.\pend
           
\pstart
           Ko{\geminationm}en Sie vielleicht Samstag zur
               Generalprobe der \label{K_L01968-2v}\edtext{Trauerfeier\pwindex{Thor und der Tod@\emph{Der Thor und der Tod}|pwv}\pwindex{Saul. Ein Tragoedienfragment@\emph{Saul. Ein Tragödienfragment}|pwv}}{\lemma{\textnormal{\emph{Trauerfeier}}}\Cendnote{\textnormal{In Erinnerung an Josef Kainz\pwindex{Kainz, Josef 02.01.1858 – 20.09.1910@\textsc{Kainz, Josef} (02.01.1858 – 20.09.1910), \emph{Schauspieler/Schauspielerin}|pwk} am \emph{Burgtheater}\orgindex{Burgtheater@Burgtheater|pwk}. Schnitzler war sowohl
                  am 22. 10. 1910 bei
                  der Generalprobe als auch am 23. 10. 1910 bei der Veranstaltung.}}}\label{K_L01968-2}? Das wäre mir ſehr lieb.
               Ich fahre dann noch für ein paar Tage \label{K_L01968-3v}\edtext{nach Grätz\oindex{Hradec nad Moravicí@\textbf{Hradec nad Moravicí}, \emph{A.ADM3}|pw}}{\lemma{\textnormal{\emph{nach Grätz}}}\Cendnote{\textnormal{Hofmannsthal\pwindex{Hofmannsthal, Hugo von 1874-02-01 – 1929-07-15@\textsc{Hofmannsthal, Hugo von} (1874-02-01 – 1929-07-15), \emph{Schriftsteller/Schriftstellerin}|pwk} war vom 25. 10. 1910 bis zum
                  30. 10. 1910 in Grätz\oindex{Hradec nad Moravicí@\textbf{Hradec nad Moravicí}, \emph{A.ADM3}|pwk} (Hradec nad Moravicí\oindex{Hradec nad Moravicí@\textbf{Hradec nad Moravicí}, \emph{A.ADM3}|pwk}).}}}\label{K_L01968-3} (zu Lichnowskys\pwindex{Lichnowsky, Karl Max 08.03.1860 – 27.02.1928@\textsc{Lichnowsky, Karl Max} (08.03.1860 – 27.02.1928), \emph{Diplomat/Diplomatin}|pw}\pwindex{Lichnowsky, Mechtilde 08.03.1879 – 04.06.1958@\textsc{Lichnowsky, Mechtilde} (08.03.1879 – 04.06.1958), \emph{Schriftsteller/Schriftstellerin, Schriftsteller/Schriftstellerin}|pw}) dann bin ich ganz hier und leſe Euch die Spieloper\pwindex{Rosenkavalier@\emph{Der Rosenkavalier}|pwv} bei Ihnen, ja?\pend
           \pstart Ihr \spacefill\mbox{Hugo}\pend{}
\pstart
           \label{T_L01968-1v}\edtext{\textsc{P.S.} Hab in Neubeuern\oindex{Neubeuern@\textbf{Neubeuern}, \emph{P.PPL}|pw}
               die »Weisſagung\pwindex{Weissagung@\emph{Die Weissagung}|pw}« vorgeleſen. Sie lieſt ſich
                  wunderſchön.}{\lemma{\textnormal{\emph{P.S. … wunderſchön.}}}\Cendnote{\textnormal{quer am linken Rand der
                  dritten Seite}}}\label{T_L01968-1}\pend
           \selectlanguage{ngerman}\endnumbering\briefempfaengerindex{Schnitzler, Arthur@\textsc{Schnitzler, Arthur}!zzzHofmannsthal, Hugo von@\emph{von Hugo von Hofmannsthal}!1910-10-202@{20. 10. 1910}|)be}\mylabel{L01968h}  \normalsize

\doendnotes{C}
\bigskip
\vfill

\clearpage

\footnotesize

\lohead{\textsc{register}}

% Definiere theindex-Environment komplett neu ohne reledmac
\makeatletter
\renewenvironment{theindex}{%
  \section*{\indexname}%
  \setlength{\parindent}{0pt}%
  \setlength{\parskip}{0pt plus 0.3pt}%
  \let\item\@idxitem
}{%
  \clearpage
}
\makeatother

\IfFileExists{\jobname-pw.ind}{\input{\jobname-pw.ind}}{}

\end{document}

      