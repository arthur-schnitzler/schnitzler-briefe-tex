%% latex-leseansicht-vorspann.tex
%% Vorspann für die Leseansicht.
%% Lädt die gemeinsame Datei latex-vorspann.tex mit nicht gesetztem Schalter.

\newif\ifkorrekturansicht
\korrekturansichtfalse

\input{../tex-inputs/latex-vorspann}


               \section[Hugo von Hofmannsthal an Arthur Schnitzler, 20. 10. {[}1910{]}]{ Hugo von Hofmannsthal an Arthur Schnitzler, 20. 10. {[}1910{]}}\nopagebreak\mylabel{v}\rehead{ }\begin{ledgroupsized}[t]{13cm}\normalsize\beginnumbering\briefempfaengerindex{Schnitzler, Arthur@\textsc{Schnitzler, Arthur}!zzzHofmannsthal, Hugo von@\emph{von Hugo von Hofmannsthal}!1910-10-202@{20. 10. 1910}|(be} \toendnotes[C]{\smallbreak\pagebreak[2]} \Standort{CUL, Schnitzler, B 43.}
\physDesc{Brief, 1 Blatt, 4 Seiten
\newline{}Handschrift: schwarze Tinte, deutsche Kurrent
\newline{}Schnitzler: mit Bleistift die Jahreszahl ergänzt: »910« und beschriftet: »\textsc{Hofmannsthal}« \newline{}Ordnung: 1) mit Bleistift von unbekannter Hand nummeriert: »\strikeout{318}« 2) mit Bleistift von unbekannter Hand nummeriert:
                                    »323«}\buchAbdrucke{\weitereDrucke{Hugo von Hofmannsthal, Arthur Schnitzler: \emph{Briefwechsel}. Hg. Therese Nickl und Heinrich Schnitzler. Frankfurt am Main: \emph{S. Fischer} 1964, S. 254.} }\toendnotes[C]{\smallbreak}\pstart
           \raggedleft{}{\pb}Rod.\oindex{Rodaun@\textbf{Rodaun}|pw}{ }20. X\pend
           \pstart
           mein guter Arthur, \hspace*{1.5em}vielmals danke ich Ihnen für Ihren Brief und Ihre
               Depeſche nach Neubeuern\oindex{Neubeuern@\textbf{Neubeuern}|pw} (wo wir 2 unvergleichlich
               ſchöne und wirklich ſehr glückerfüllte \label{K_L01968_1v}\edtext{Herbſtwochen}{\lemma{\textnormal{\emph{Herbſtwochen}}}\Cendnote{\textnormal{vom 4. 10. 1910 bis zum
                     16. 10. 1910}}}\label{K_L01968_1h} zubrachten) für Ihre Hilfe in der
               Beſetzungsſache und vor allem für die schönen Stunden, die mir Ihr neues {\pb}Stück\pwindex{Schnitzler, Arthur 15.05.1862 – 21.10.1931@\textsc{Schnitzler, Arthur} (15.05.1862 – 21.10.1931), \emph{Schriftsteller, Mediziner}!weite Land. Tragikomoedie in fuenf Akten1910-10-20@\strich\emph{Das weite Land. Tragikomödie in fünf Akten} {[}1910-10-20{]}|pwv} geſchenkt hat. Ich glaube,
               dieſes »weite Land\pwindex{Schnitzler, Arthur 15.05.1862 – 21.10.1931@\textsc{Schnitzler, Arthur} (15.05.1862 – 21.10.1931), \emph{Schriftsteller, Mediziner}!weite Land. Tragikomoedie in fuenf Akten1910-10-20@\strich\emph{Das weite Land. Tragikomödie in fünf Akten} {[}1910-10-20{]}|pw}« iſt wirklich die allerbeſte
               Arbeit Ihrer an guten Arbeiten ſo reichen zweiten Lebens- oder Arbeitsperiode.\pend
           \pstart
           Das Stück gehört ſo ganz Ihnen, und iſt dabei ſo äußerſt kräftig, ſo wunderſchön
               zuſammengehalten. Alle Ihre nicht leicht in einem Athem aufzuzählenden Vorzüge: das
               ſo ganz perſönliche Lebensgefühl, die höchſt beſondere Scala der Wertungen, {\pb}die zarte und ſichere Geſtaltung,
               die leichte Hand für die Scenenführung, die Melancholie und der Witz, der höchſt
               nötige \textsc{bon sens}, normaler (aber ſeltener) Menſchenverſtand,
               und dazu das tiefere poetiſch-philoſophiſche Zuſammenſehen und Nebeneinanderſehen,
               die Güte, die Erfahrung und zugleich ein entzückender Mangel an Routine, ein
               Friſches, Blühendes, Geſpanntes überall – dies alles ko{\geminationm}t zuſammen, um ein {\pb}Werk
               herzuſtellen, das ſich in unvergleichlicher Weiſe im Gleichgewicht hält, weltlich und
               tief, theatermäßig und philoſophiſch, amüſant und bedeutend iſt.\hspace*{1.5em}Ich freue mich ſehr, es auch noch auf der Bühne zu ſehen – doch hab
               ich es auf der inneren Bühne tadellos beſetzt und ſehr ſchön mir aufgeführt.\pend
           \pstart
           Ko{\geminationm}en Sie vielleicht Samstag zur
               Generalprobe der \label{K_L01968_2v}\edtext{Trauerfeier\pwindex{Hofmannsthal, Hugo von 01.02.1874 – 15.07.1929@\textsc{Hofmannsthal, Hugo von} (01.02.1874 – 15.07.1929), \emph{Schriftsteller}!Thor und der Tod1893@\strich\emph{Der Thor und der Tod} {[}1893{]}|pwv}\pwindex{Kainz, Josef 02.01.1858 – 20.09.1910@\textsc{Kainz, Josef} (02.01.1858 – 20.09.1910), \emph{Schauspieler}!Saul. Ein Tragoedienfragment23.10.1910 – 23.10.1910@\strich\emph{Saul. Ein Tragödienfragment} {[}23.10.1910 – 23.10.1910{]}|pwv}}{\lemma{\textnormal{\emph{Trauerfeier}}}\Cendnote{\textnormal{In Erinnerung an Josef Kainz\pwindex{Kainz, Josef 02.01.1858 – 20.09.1910@\textsc{Kainz, Josef} (02.01.1858 – 20.09.1910), \emph{Schauspieler}|pwk} am \emph{Burgtheater}\orgindex{Burgtheater@Burgtheater|pwk}.
                     Schnitzler\pwindex{Schnitzler, Arthur 15.05.1862 – 21.10.1931@\textsc{Schnitzler, Arthur} (15.05.1862 – 21.10.1931), \emph{Schriftsteller, Mediziner}|pwk} war sowohl am 22. 10. 1910 bei der
                  Generalprobe, als auch am 23. 10. 1910 bei der Veranstaltung.}}}\label{K_L01968_2h}? Das wäre mir ſehr lieb.
               Ich fahre dann noch für ein paar Tage \label{K_L01968_3v}\edtext{nach Grätz\oindex{Hradec nad Moravicí@\textbf{Hradec nad Moravicí}|pw}}{\lemma{\textnormal{\emph{nach Grätz}}}\Cendnote{\textnormal{vom 25. 10. 1910 bis zum
                     30. 10. 1910.}}}\label{K_L01968_3h} (zu Lichnowskys\pwindex{Lichnowsky, Karl Max 08.03.1860 – 27.02.1928@\textsc{Lichnowsky, Karl Max} (08.03.1860 – 27.02.1928), \emph{Diplomat}|pw}\pwindex{Lichnowsky, Mechtilde 08.03.1879 – 04.06.1958@\textsc{Lichnowsky, Mechtilde} (08.03.1879 – 04.06.1958), \emph{Schriftstellerin}|pw}) dann bin ich ganz hier und leſe Euch die Spieloper\pwindex{Hofmannsthal, Hugo von 01.02.1874 – 15.07.1929@\textsc{Hofmannsthal, Hugo von} (01.02.1874 – 15.07.1929), \emph{Schriftsteller}!Rosenkavalier1911@\strich\emph{Der Rosenkavalier} {[}1911{]}|pwv} bei Ihnen, ja?\pend
           \pstart Ihr \spacefill\mbox{Hugo}\pend{}\pstart
           \label{T_L01968_1v}\edtext{\textsc{P.S.} Hab in Neubeuern\oindex{Neubeuern@\textbf{Neubeuern}|pw}
               die »Weisſagung\pwindex{Schnitzler, Arthur 15.05.1862 – 21.10.1931@\textsc{Schnitzler, Arthur} (15.05.1862 – 21.10.1931), \emph{Schriftsteller, Mediziner}!Weissagung24. 12. 1905@\strich\emph{Die Weissagung} {[}24. 12. 1905{]}|pw}« vorgeleſen. Sie lieſt ſich
                  wunderſchön.}{\lemma{\textnormal{\emph{P.S. … wunderſchön.}}}\Cendnote{\textnormal{quer am linken Rand der
                  dritten Seite}}}\label{T_L01968_1h}\pend
           \endnumbering\briefempfaengerindex{Schnitzler, Arthur@\textsc{Schnitzler, Arthur}!zzzHofmannsthal, Hugo von@\emph{von Hugo von Hofmannsthal}!1910-10-202@{20. 10. 1910}|)be}\mylabel{h}\end{ledgroupsized}  \newcommand{\dateiname}{L01968}\newcommand{\titel}{Hugo von Hofmannsthal an Arthur Schnitzler, 20. 10. [1910]}\newcommand{\editorInnen}{Martin Anton Müller und Gerd-Hermann Susen}
            \footnotesize
\begin{ledgroupsized}[t]{11.5cm}
\doendnotes{C}
\end{ledgroupsized}
         %% latex-leseansicht-abspann.tex
%% Abspann für die Leseansicht.
%% Der Schalter \ifkorrekturansicht ist bereits durch den Vorspann gesetzt.

%% latex-abspann.tex
%% Gemeinsamer Abspann für Korrekturansicht und Leseansicht.
%% Setzt den Schalter \ifkorrekturansicht voraus (gesetzt in den
%% einbindenden Dateien latex-korrekturansicht-abspann.tex bzw.
%% latex-leseansicht-abspann.tex).
%% ---------------------------------------------------------------

\normalsize

% Das esempio-Environment wird nur in der Leseansicht benötigt
\ifkorrekturansicht\else
\newenvironment{esempio}[3]%
{
    \vspace{1.5ex}
    \rlap{\underline{#1}}
    \par
    \setlength{\parindent}{0cm}
    \nopagebreak
    \leftskip=#2cm
    \rightskip=#3cm
}
{
    \par
}
\fi

\doendnotes{C}
\bigskip
\vfill

\clearpage

\footnotesize

\ifkorrekturansicht
  \lohead{\textsc{register}}
\fi

% theindex-Environment neu definieren ohne reledmac
\makeatletter
\renewenvironment{theindex}{%
  \ifkorrekturansicht
    \section*{\indexname}%
  \else
    \subsubsection*{Index der erwähnten Entitäten}%
  \fi
  \setlength{\parindent}{0pt}%
  \setlength{\parskip}{0pt plus 0.3pt}%
  \let\item\@idxitem
}{%
  \ifkorrekturansicht\clearpage\fi
}
\makeatother

\IfFileExists{\jobname-pw.ind}{\input{\jobname-pw.ind}}{}

% Quellenangabe nur in der Leseansicht
\ifkorrekturansicht\else
% Fallback-Definitionen, falls die .tex-Datei \titel etc. nicht gesetzt hat
\providecommand{\titel}{}
\providecommand{\editorInnen}{}
\providecommand{\dateiname}{\jobname}

\vspace{3cm}

\vfill

\footnotesize
\textsc{Quelle}: \titel. Herausgegeben von {\editorInnen}. In: \emph{Arthur Schnitzler: Briefwechsel mit Autorinnen und Autoren}.
 Digitale Edition, https://schnitzler-briefe.acdh.oeaw.ac.at/{\dateiname}.html (Stand \today)
\fi

\end{document}


      