%% latex-korrekturansicht-vorspann.tex
%% Vorspann für die Korrekturansicht.
%% Lädt die gemeinsame Datei latex-vorspann.tex mit gesetztem Schalter.

\newif\ifkorrekturansicht
\korrekturansichttrue

\input{../tex-inputs/latex-vorspann}


\section[Arthur Schnitzler an Richard Beer-Hofmann, 13. 9. 1892]{L00123 Arthur Schnitzler an Richard Beer-Hofmann, 13. 9. 1892}
\nopagebreak\mylabel{L00123v}
\rehead{ }\normalsize\beginnumbering\briefempfaengerindex{Beer-Hofmann, Richard@\textsc{Beer-Hofmann, Richard}!zzzSchnitzler, Arthur@\emph{von Arthur Schnitzler}!1892-09-131@{13. 9. 1892}|(be}
\toendnotes[C]{\smallbreak\pagebreak[2]}\Standort{YCGL, MSS 31.}
\physDesc{Kartenbrief, 895 Zeichen
\newline{}Handschrift: schwarze Tinte, deutsche Kurrent
\newline{}Versand: 1) Stempel: »\nobreak{}\oindex{Riva del Garda@\textbf{Riva del Garda}, \emph{P.PPLA3}|pwk}Riva, 13 9 92, 5.N\nobreak{}«.   2) Stempel: »\nobreak{}\oindex{Bad Ischl@\textbf{Bad Ischl}, \emph{P.PPL}|pwk}Ischl, 14 9 {[}92{]}\nobreak{}«. }
\buchAbdrucke{\weitereDrucke{1) Arthur Schnitzler: \emph{Briefe 1875–1912}. Frankfurt am Main: \emph{S. Fischer} 1981, S. 129.} \weitereDrucke{2) Arthur Schnitzler, Richard Beer-Hofmann: \emph{Briefwechsel 1891–1931}. Wien, Zürich: \emph{Europaverlag} 1992, S. 38.} }\toendnotes[C]{\smallbreak}\pstart{}{\pb}Dr. \textsc{Richard Beer Hofmann}\pend{}\pstart{}\textsc{Ischl\oindex{Bad Ischl@\textbf{Bad Ischl}, \emph{P.PPL}|pw}.}\pend{}\pstart{}\textsc{Grazerstraße 4\oindex{Grazer Strasse [Bad Ischl]@\textbf{Grazer Straße [Bad Ischl]}, \emph{Straße (K.STR)}|pw}}.\pend{}\pstart{}\textsc{Ober-Oesterreich}\pend{}{\bigskip}\vspace{1em}
\pstart
           \raggedleft{}{\pb}Riva\oindex{Riva del Garda@\textbf{Riva del Garda}, \emph{P.PPLA3}|pw}{ }13. 9. 92\pend
           \vspace{0.5em}
\pstart
           Lieber Richard – es iſt ſo ſchwer Ihnen zu ſchreiben! Sie wiſſen ja
               alles. – Der tiefblaue See\oindex{Lago di Garda@\textbf{Lago di Garda}, \emph{See (N.SEE)}|pwv}!
               Der italien\oindex{Italien@\textbf{Italien}, \emph{A.PCLI}|pw}iſche Himmel. Die Einwohner, die
               nichts zu thun haben. Kinder, die in der Kirche ſpielen. Ein kleines Mädel mit
               lächerlich ſchwarzem Haar, die, wie ich vor einem verhüllten Altarbild ſtehe,
               plötzlich mittelſt eines herabhängenden Stricks die Hülle fallen läßt – und da iſt
               nun die brave unbefleckte Maria dahinter, was ja nicht einmal eine Überraſchung
               iſt. – Ein Balkon, auf dem die Sonne liegt, und unten der Park, und weiter, nun
               natürlich, der See\oindex{Lago di Garda@\textbf{Lago di Garda}, \emph{See (N.SEE)}|pwv}, der See,
               der tiefblaue See\oindex{Lago di Garda@\textbf{Lago di Garda}, \emph{See (N.SEE)}|pwv}. Uns
               gegenüber Berge. – Das Hotel deutſch, poſirt nur ein wenig das italien\oindex{Italien@\textbf{Italien}, \emph{A.PCLI}|pw}iſche durch Fliegen und zarte Unreinlichkeit. Schön, ſehr
               ſchön. – Und ich verſti{\geminationm}t. We{\geminationn} ich mich nicht ſchämte, würd ich ſagen: traurig. –\pend
           
\pstart
           Viele herzliche Grüße{\\[\baselineskip]}\spacefill\mbox{Arthur}\pend
           \leftskip=0em{}\selectlanguage{ngerman}\endnumbering\briefempfaengerindex{Beer-Hofmann, Richard@\textsc{Beer-Hofmann, Richard}!zzzSchnitzler, Arthur@\emph{von Arthur Schnitzler}!1892-09-131@{13. 9. 1892}|)be}\mylabel{L00123h}  \normalsize

\doendnotes{C}
\bigskip
\vfill

\clearpage

\footnotesize

\lohead{\textsc{register}}

% Definiere theindex-Environment komplett neu ohne reledmac
\makeatletter
\renewenvironment{theindex}{%
  \section*{\indexname}%
  \setlength{\parindent}{0pt}%
  \setlength{\parskip}{0pt plus 0.3pt}%
  \let\item\@idxitem
}{%
  \clearpage
}
\makeatother

\IfFileExists{\jobname-pw.ind}{\input{\jobname-pw.ind}}{}

\end{document}

      