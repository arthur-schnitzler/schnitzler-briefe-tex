%% latex-leseansicht-vorspann.tex
%% Vorspann für die Leseansicht.
%% Lädt die gemeinsame Datei latex-vorspann.tex mit nicht gesetztem Schalter.

\newif\ifkorrekturansicht
\korrekturansichtfalse

\input{../tex-inputs/latex-vorspann}


         
         \newcommand{\erwaehntePersonen}{Personen: Richard Beer-Hofmann}
         \newcommand{\erwaehnteInstitutionen}{}
         \newcommand{\erwaehnteOrte}{Orte: Bad Ischl, Grazer Straße, Italien, Lago di Garda, Riva del Garda}
         \newcommand{\erwaehnteWerke}{
               \section[Arthur Schnitzler an Richard Beer-Hofmann, 13. 9. 1892]{ Arthur Schnitzler an Richard Beer-Hofmann, 13. 9. 1892}\nopagebreak\mylabel{v}\rehead{ }\begin{ledgroupsized}[t]{13cm}\normalsize\beginnumbering \toendnotes[C]{\smallbreak\pagebreak[2]} \Standort{YCGL, MSS 31.}
\physDesc{Kartenbrief
\newline{}Handschrift: schwarze Tinte, deutsche Kurrent\newline{}Versand: 1) Stempel: »\nobreak{}\oindex{Riva del Garda@\textbf{Riva del Garda}|pwk}Riva, 13 9 92, 5.N\nobreak{}«.   2) Stempel: »\nobreak{}\oindex{Bad Ischl@\textbf{Bad Ischl}|pwk}Ischl, 14 9 {[}92{]}\nobreak{}«. }\buchAbdrucke{\weitereDrucke{1) Arthur Schnitzler: \emph{Briefe 1875–1912}. Hg. Therese Nickl und Heinrich Schnitzler. Frankfurt am Main: \emph{S. Fischer} 1981, S. 129.} \weitereDrucke{2) Arthur Schnitzler, Richard Beer-Hofmann: \emph{Briefwechsel 1891–1931}. Hg. Konstanze Fliedl. Wien, Zürich: \emph{Europaverlag} 1992, S. 38.} }\toendnotes[C]{\smallbreak}\pstart{}{\pb}Dr. \textsc{Richard Beer Hofmann}\pend{}\pstart{}\textsc{Ischl\oindex{Bad Ischl@\textbf{Bad Ischl}|pw}.}\pend{}\pstart{}\textsc{Grazerstraße 4\oindex{Grazer Strasse@\textbf{Grazer Straße}|pw}}.\pend{}\pstart{}\textsc{Ober-Oesterreich}\pend{}{\bigskip}\pstart
           \raggedleft{}{\pb}Riva\oindex{Riva del Garda@\textbf{Riva del Garda}|pw}{ }13. 9. 92\pend
           \pstart
           Lieber Richard – es iſt ſo ſchwer Ihnen zu ſchreiben! Sie wiſſen ja
               alles. – Der tiefblaue See\oindex{Lago di Garda@\textbf{Lago di Garda}|pwv}! Der italien\oindex{Italien@\textbf{Italien}|pw}iſche
               Himmel. Die Einwohner, die nichts zu thun haben. Kinder, die in der Kirche ſpielen.
               Ein kleines Mädel mit lächerlich ſchwarzem Haar, die, wie ich vor einem verhüllten
               Altarbild ſtehe, plötzlich mittelſt eines herabhängenden Stricks die Hülle fallen
               läßt – und da iſt nun die brave unbefleckte Maria dahinter, was ja nicht einmal eine
               Überraſchung iſt. – Ein Balkon, auf dem die Sonne liegt, und unten der Park, und
               weiter, nun natürlich, der See\oindex{Lago di Garda@\textbf{Lago di Garda}|pwv}, der See, der tiefblaue See\oindex{Lago di Garda@\textbf{Lago di Garda}|pwv}. Uns gegenüber Berge. – Das Hotel deutſch, poſirt nur ein
               wenig das italien\oindex{Italien@\textbf{Italien}|pw}iſche durch Fliegen und zarte
               Unreinlichkeit. Schön, ſehr ſchön. – Und ich verſti{\geminationm}t.
                  We{\geminationn} ich mich nicht ſchämte, würd ich ſagen:
               traurig. –\pend
           \pstart
           Viele herzliche Grüße{\\[\baselineskip]}\spacefill\mbox{Arthur}\pend
           \leftskip=0em{}
         
         \endnumbering\mylabel{h}\end{ledgroupsized}  \newcommand{\dateiname}{L00123}\newcommand{\titel}{Arthur Schnitzler an Richard Beer-Hofmann, 13. 9. 1892}\newcommand{\editorInnen}{Martin Anton Müller und Gerd-Hermann Susen}%% latex-leseansicht-abspann.tex
%% Abspann für die Leseansicht.
%% Der Schalter \ifkorrekturansicht ist bereits durch den Vorspann gesetzt.

%% latex-abspann.tex
%% Gemeinsamer Abspann für Korrekturansicht und Leseansicht.
%% Setzt den Schalter \ifkorrekturansicht voraus (gesetzt in den
%% einbindenden Dateien latex-korrekturansicht-abspann.tex bzw.
%% latex-leseansicht-abspann.tex).
%% ---------------------------------------------------------------

\normalsize

% Das esempio-Environment wird nur in der Leseansicht benötigt
\ifkorrekturansicht\else
\newenvironment{esempio}[3]%
{
    \vspace{1.5ex}
    \rlap{\underline{#1}}
    \par
    \setlength{\parindent}{0cm}
    \nopagebreak
    \leftskip=#2cm
    \rightskip=#3cm
}
{
    \par
}
\fi

\doendnotes{C}
\bigskip
\vfill

\clearpage

\footnotesize

\ifkorrekturansicht
  \lohead{\textsc{register}}
\fi

% theindex-Environment neu definieren ohne reledmac
\makeatletter
\renewenvironment{theindex}{%
  \ifkorrekturansicht
    \section*{\indexname}%
  \else
    \subsubsection*{Index der erwähnten Entitäten}%
  \fi
  \setlength{\parindent}{0pt}%
  \setlength{\parskip}{0pt plus 0.3pt}%
  \let\item\@idxitem
}{%
  \ifkorrekturansicht\clearpage\fi
}
\makeatother

\IfFileExists{\jobname-pw.ind}{\input{\jobname-pw.ind}}{}

% Quellenangabe nur in der Leseansicht
\ifkorrekturansicht\else
% Fallback-Definitionen, falls die .tex-Datei \titel etc. nicht gesetzt hat
\providecommand{\titel}{}
\providecommand{\editorInnen}{}
\providecommand{\dateiname}{\jobname}

\vspace{3cm}

\vfill

\footnotesize
\textsc{Quelle}: \titel. Herausgegeben von {\editorInnen}. In: \emph{Arthur Schnitzler: Briefwechsel mit Autorinnen und Autoren}.
 Digitale Edition, https://schnitzler-briefe.acdh.oeaw.ac.at/{\dateiname}.html (Stand \today)
\fi

\end{document}


      