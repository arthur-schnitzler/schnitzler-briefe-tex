%% latex-korrekturansicht-vorspann.tex
%% Vorspann für die Korrekturansicht.
%% Lädt die gemeinsame Datei latex-vorspann.tex mit gesetztem Schalter.

\newif\ifkorrekturansicht
\korrekturansichttrue

\input{../tex-inputs/latex-vorspann}


\section[Richard Beer-Hofmann an Arthur Schnitzler, 19. 10. 1907]{L01724 Richard Beer-Hofmann an Arthur Schnitzler, 19. 10. 1907}
\nopagebreak\mylabel{L01724v}
\rehead{ }\normalsize\beginnumbering\briefempfaengerindex{Schnitzler, Arthur@\textsc{Schnitzler, Arthur}!zzzBeer-Hofmann, Richard@\emph{von Richard Beer-Hofmann}!1907-10-191@{19. 10. 1907}|(be}
\toendnotes[C]{\smallbreak\pagebreak[2]}\Standort{CUL, Schnitzler, B 8.}
\physDesc{Brief, 1 Blatt, 1 Seite, 233 Zeichen (Briefpapier mit Trauerrand)
\newline{}Handschrift: Bleistift, lateinische Kurrent
\newline{}Ordnung: mit Bleistift von unbekannter Hand nummeriert:
                                    »214« }
\buchAbdrucke{\weitereDrucke{Arthur Schnitzler, Richard Beer-Hofmann: \emph{Briefwechsel 1891–1931}. Wien, Zürich: \emph{Europaverlag} 1992, S. 186.} }\toendnotes[C]{\smallbreak}
\pstart
           \raggedleft{}{\pb}19/X 07\pend
           \vspace{0.5em}
\pstart
           Lieber Arthur! Wollen Sie heute Abends – anstatt \label{K_L01724-1v}\edtext{{[}Zeichnung einer schwarz-weiß-gekachelten Fledermaus\orgindex{Cabaret Fledermaus@Cabaret Fledermaus|pw}{]}}{\lemma{\textnormal{\emph{Zeichnung … Fledermaus}}}\Cendnote{\textnormal{Am 19. 10. 1907
                  eröffnete das \emph{Cabaret Fledermaus}\orgindex{Cabaret Fledermaus@Cabaret Fledermaus|pwk} mit einer
                  Innenausstattung, die mit ihren schwarz-weiß gekachelten Fliesen Aufsehen erregte.
                     Altenberg\pwindex{Altenberg, Peter 09.03.1859 – 08.01.1919@\textsc{Altenberg, Peter} (09.03.1859 – 08.01.1919), \emph{Schriftsteller/Schriftstellerin}|pwk} war ein Unterstützer des neuen
                  Etablissements und trug selbst gerne karierte Kleidung.}}}\label{K_L01724-1} und {[}Zeichung von Altenberg\pwindex{Altenberg, Peter 09.03.1859 – 08.01.1919@\textsc{Altenberg, Peter} (09.03.1859 – 08.01.1919), \emph{Schriftsteller/Schriftstellerin}|pw} mit Sprechblase: {]} das höchste!!
                  {[}–{]} bei uns essen?\pend
           
\pstart
           (Leo\pwindex{Van-Jung, Leo 15.10.1866 – 02.07.1939@\textsc{Van-Jung, Leo} (15.10.1866 – 02.07.1939), \emph{Gesangspädagoge/Gesangspädagogin, Mathematiker/Mathematikerin}|pw} ko{\geminationm}t auch)
               um halbacht?\pend
           
\pstart
           Es wäre sehr schön.\pend
           
\pstart
           Herzlich{\\[\baselineskip]}\spacefill\mbox{Richard}\pend
           \leftskip=0em{}\selectlanguage{ngerman}\endnumbering\briefempfaengerindex{Schnitzler, Arthur@\textsc{Schnitzler, Arthur}!zzzBeer-Hofmann, Richard@\emph{von Richard Beer-Hofmann}!1907-10-191@{19. 10. 1907}|)be}\mylabel{L01724h}  \normalsize

\doendnotes{C}
\bigskip
\vfill

\clearpage

\footnotesize

\lohead{\textsc{register}}

% Definiere theindex-Environment komplett neu ohne reledmac
\makeatletter
\renewenvironment{theindex}{%
  \section*{\indexname}%
  \setlength{\parindent}{0pt}%
  \setlength{\parskip}{0pt plus 0.3pt}%
  \let\item\@idxitem
}{%
  \clearpage
}
\makeatother

\IfFileExists{\jobname-pw.ind}{\input{\jobname-pw.ind}}{}

\end{document}

      