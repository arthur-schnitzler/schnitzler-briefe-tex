%% latex-korrekturansicht-vorspann.tex
%% Vorspann für die Korrekturansicht.
%% Lädt die gemeinsame Datei latex-vorspann.tex mit gesetztem Schalter.

\newif\ifkorrekturansicht
\korrekturansichttrue

\input{../tex-inputs/latex-vorspann}


\section[Richard Beer-Hofmann an Arthur Schnitzler, 13. 9. 1895]{L00482 Richard Beer-Hofmann an Arthur Schnitzler, 13. 9. 1895}
\nopagebreak\mylabel{L00482v}
\rehead{ }\normalsize\beginnumbering\briefempfaengerindex{Schnitzler, Arthur@\textsc{Schnitzler, Arthur}!zzzBeer-Hofmann, Richard@\emph{von Richard Beer-Hofmann}!1895-09-131@{13. 9. 1895}|(be}
\toendnotes[C]{\smallbreak\pagebreak[2]}\Standort{CUL, Schnitzler, B 8.}
\physDesc{Briefkarte, 649 Zeichen
\newline{}Handschrift: Bleistift, lateinische Kurrent
\newline{}Schnitzler: mit Bleistift nummeriert: »69« }
\buchAbdrucke{\weitereDrucke{Arthur Schnitzler, Richard Beer-Hofmann: \emph{Briefwechsel 1891–1931}. Wien, Zürich: \emph{Europaverlag} 1992, S. 80.} }\toendnotes[C]{\smallbreak}
\pstart
           \raggedleft{}{\pb}Schönberg\oindex{Schoenberg im Stubaital@\textbf{Schönberg im Stubaital}, \emph{P.PPLA3}|pw}{ }13 Sept 95\pend
           \vspace{0.5em}
\pstart
           Lieber Arthur! Bitte um den ausführlichen Brief. Frau Lou\pwindex{Andreas-Salome, Lou 12.02.1861 – 05.02.1937@\textsc{Andreas-Salomé, Lou} (12.02.1861 – 05.02.1937), \emph{Schriftsteller/Schriftstellerin}|pw} erwidert Grüße etc. Von morgen früh an bin
               ich allein!!! Ich bleibe hier solange es schön ist – ich arbeite hier sehr gut – dann
               gehe ich etwas südlicher. Bozen\oindex{Bozen@\textbf{Bozen}, \emph{P.PPLA2}|pw} oder Riva\oindex{Riva del Garda@\textbf{Riva del Garda}, \emph{P.PPLA3}|pw}. Sie haben mich falsch verstanden; nicht
                  Ende Oktober, Ende Sept. will ich in Wien\oindex{Wien@\textbf{Wien}, \emph{A.ADM2}|pw} sein\pend
           
\pstart
           {\pb}Was macht Hugo\pwindex{Hofmannsthal, Hugo von 1874-02-01 – 1929-07-15@\textsc{Hofmannsthal, Hugo von} (1874-02-01 – 1929-07-15), \emph{Schriftsteller/Schriftstellerin}|pw}? Grüßen Sie Salten\pwindex{Salten, Felix 06.09.1869 – 08.10.1945@\textsc{Salten, Felix} (06.09.1869 – 08.10.1945), \emph{Schriftsteller/Schriftstellerin, Journalist/Journalistin, Chefredakteur/Chefredakteurin}|pw}{ }Schwarzkopf\pwindex{Schwarzkopf, Gustav 07.11.1853 – 13.11.1939@\textsc{Schwarzkopf, Gustav} (07.11.1853 – 13.11.1939), \emph{Schriftsteller/Schriftstellerin}|pw}, Sokal\pwindex{Sokal, Clemens *~21.01.1867@\textsc{Sokal, Clemens} (*~21.01.1867), \emph{Journalist/Journalistin, Rechtsanwalt/Rechtsanwältin}|pw} – genug. Momentan ist es kalt aber schön. Im übrigen teile ich Ihnen
               mit daß es am schönsten ist \uline{allein} zu reisen. Uns
               Zwei \introOben{}(Mich und Sie!)\introOben{} und Hugo\pwindex{Hofmannsthal, Hugo von 1874-02-01 – 1929-07-15@\textsc{Hofmannsthal, Hugo von} (1874-02-01 – 1929-07-15), \emph{Schriftsteller/Schriftstellerin}|pw} ausgeno{\geminationm}en. Paul\pwindex{Goldmann, Paul 31.01.1865 – 25.09.1935@\textsc{Goldmann, Paul} (31.01.1865 – 25.09.1935), \emph{Schriftsteller/Schriftstellerin, Journalist/Journalistin}|pw} leidet zuviel an Familie. Mein Papa\pwindex{Beer, Hermann 10.8.1835 – 03.10.1902@\textsc{Beer, Hermann} (10.8.1835 – 03.10.1902), \emph{Rechtsanwalt/Rechtsanwältin}|pwv} hat einen herrlichen Brief
               geschrieben. Ich zeig ihn Ihnen in Wien\oindex{Wien@\textbf{Wien}, \emph{A.ADM2}|pw}.
               Herzlichst Ihr\pend
           \pstart \spacefill\mbox{R.}\pend{}\selectlanguage{ngerman}\endnumbering\briefempfaengerindex{Schnitzler, Arthur@\textsc{Schnitzler, Arthur}!zzzBeer-Hofmann, Richard@\emph{von Richard Beer-Hofmann}!1895-09-131@{13. 9. 1895}|)be}\mylabel{L00482h}  \normalsize

\doendnotes{C}
\bigskip
\vfill

\clearpage

\footnotesize

\lohead{\textsc{register}}

% Definiere theindex-Environment komplett neu ohne reledmac
\makeatletter
\renewenvironment{theindex}{%
  \section*{\indexname}%
  \setlength{\parindent}{0pt}%
  \setlength{\parskip}{0pt plus 0.3pt}%
  \let\item\@idxitem
}{%
  \clearpage
}
\makeatother

\IfFileExists{\jobname-pw.ind}{\input{\jobname-pw.ind}}{}

\end{document}

      