%% latex-leseansicht-vorspann.tex
%% Vorspann für die Leseansicht.
%% Lädt die gemeinsame Datei latex-vorspann.tex mit nicht gesetztem Schalter.

\newif\ifkorrekturansicht
\korrekturansichtfalse

\input{../tex-inputs/latex-vorspann}


\section[ Felix Salten an Arthur Schnitzler, 15. 4. 1906]{L03418 Felix Salten an Arthur Schnitzler,  15. 4. 1906}
\nopagebreak\mylabel{L03418v}
\rehead{ }\normalsize\beginnumbering\briefempfaengerindex{Schnitzler, Arthur@\textsc{Schnitzler, Arthur}!zzzSalten, Felix@\emph{von Felix Salten}!1906-04-151@{15. 4. 1906}|(be}
\toendnotes[C]{\smallbreak\pagebreak[2]}
\correspDesc{Versand  durch Felix Salten am 15. 4. 1906 in Berlin
\newline{}Übermittlung  am 16. 4. 1906 in Berlin
\newline{}Erhalt  durch Arthur Schnitzler im Zeitraum [17. 4. 1906
                  – 21. 4. 1906?] in Wien}\toendnotes[C]{\smallbreak}
\Standort{CUL, Schnitzler, B 89, B 1.}
\physDesc{Bildpostkarte, 126 Zeichen
\newline{}Handschrift: schwarze Tinte, lateinische Kurrent
\newline{}Versand: Stempel: »\nobreak{}\oindex{Berlin@\textbf{Berlin}, \emph{Hauptstadt}|pwk}Berlin\textcolor{gray}{,}
                                       N. W., 16. 4. 06, 3–4 N.\nobreak{}«.  
\newline{}Ordnung: mit Bleistift von unbekannter Hand nummeriert: »209« }\pstart{}{\pb}Herrn D\textsuperscript{r} Arthur Schnitzler\pend{}\pstart{}Wien XVIII.\oindex{Berlin@\textbf{Berlin}, \emph{Hauptstadt}|pw}\pend{}\pstart{}Spöttelgasse 7\oindex{Wien@\textbf{Wien}!XVIII., Währing@\textbf{XVIII., Währing}!Edmund-Weiß-Gasse 7@\textbf{Edmund-Weiß-Gasse 7}, \emph{Wohngebäude}|pw}\pend{}{\bigskip}
\pstart
           {\pb}\textcolor{gray}{\textbf{Gruss aus Sanssouci-Potsdam\oindex{Schloss Sanssouci@\textbf{Schloss Sanssouci}, \emph{Burg}|pw}}}\hfill \textcolor{gray}{\textbf{Sicilianischer Garten\oindex{Garten von Sanssouci@\textbf{Garten von Sanssouci}, \emph{Park}|pw} mit Bogenschützen\pwindex{Geyger, Ernst Moritz 9.\,11.\,1861 Rixdorf – 29.\,12.\,1941 Florenz@\textsc{Geyger, Ernst Moritz} (9.\,11.\,1861 Rixdorf – 29.\,12.\,1941 Florenz), \emph{Radierer, Bildhauer}!Bogenschütze@\strich\emph{Bogenschütze}|pw} v. Prof. Geiger\pwindex{Geyger, Ernst Moritz 9.\,11.\,1861 Rixdorf – 29.\,12.\,1941 Florenz@\textsc{Geyger, Ernst Moritz} (9.\,11.\,1861 Rixdorf – 29.\,12.\,1941 Florenz), \emph{Radierer, Bildhauer}|pw}}}\pend
           \vspace{1em}
\pstart
           \noindent{}{\pb}Auf einer schönen
               Automobiltour\pend
           
\pstart
           herzliche Grüße{\\[\baselineskip]}\spacefill\mbox{Salten}\pend
           \leftskip=0em{}
\pstart
           {\pb}Ostersonntag 15. IV. 06.\pend
           \selectlanguage{ngerman}\endnumbering\briefempfaengerindex{Schnitzler, Arthur@\textsc{Schnitzler, Arthur}!zzzSalten, Felix@\emph{von Felix Salten}!1906-04-151@{15. 4. 1906}|)be}\mylabel{L03418h}  \newcommand{\dateiname}{L03418}\newcommand{\titel}{Felix Salten an Arthur Schnitzler, 15. 4. 1906}\newcommand{\editorInnen}{Martin Anton Müller und Laura Untner}%% latex-leseansicht-abspann.tex
%% Abspann für die Leseansicht.
%% Der Schalter \ifkorrekturansicht ist bereits durch den Vorspann gesetzt.

%% latex-abspann.tex
%% Gemeinsamer Abspann für Korrekturansicht und Leseansicht.
%% Setzt den Schalter \ifkorrekturansicht voraus (gesetzt in den
%% einbindenden Dateien latex-korrekturansicht-abspann.tex bzw.
%% latex-leseansicht-abspann.tex).
%% ---------------------------------------------------------------

\normalsize

% Das esempio-Environment wird nur in der Leseansicht benötigt
\ifkorrekturansicht\else
\newenvironment{esempio}[3]%
{
    \vspace{1.5ex}
    \rlap{\underline{#1}}
    \par
    \setlength{\parindent}{0cm}
    \nopagebreak
    \leftskip=#2cm
    \rightskip=#3cm
}
{
    \par
}
\fi

\doendnotes{C}
\bigskip
\vfill

\clearpage

\footnotesize

\ifkorrekturansicht
  \lohead{\textsc{register}}
\fi

% theindex-Environment neu definieren ohne reledmac
\makeatletter
\renewenvironment{theindex}{%
  \ifkorrekturansicht
    \section*{\indexname}%
  \else
    \subsubsection*{Index der erwähnten Entitäten}%
  \fi
  \setlength{\parindent}{0pt}%
  \setlength{\parskip}{0pt plus 0.3pt}%
  \let\item\@idxitem
}{%
  \ifkorrekturansicht\clearpage\fi
}
\makeatother

\IfFileExists{\jobname-pw.ind}{\input{\jobname-pw.ind}}{}

% Quellenangabe nur in der Leseansicht
\ifkorrekturansicht\else
% Fallback-Definitionen, falls die .tex-Datei \titel etc. nicht gesetzt hat
\providecommand{\titel}{}
\providecommand{\editorInnen}{}
\providecommand{\dateiname}{\jobname}

\vspace{3cm}

\vfill

\footnotesize
\textsc{Quelle}: \titel. Herausgegeben von {\editorInnen}. In: \emph{Arthur Schnitzler: Briefwechsel mit Autorinnen und Autoren}.
 Digitale Edition, https://schnitzler-briefe.acdh.oeaw.ac.at/{\dateiname}.html (Stand \today)
\fi

\end{document}


