%% latex-korrekturansicht-vorspann.tex
%% Vorspann für die Korrekturansicht.
%% Lädt die gemeinsame Datei latex-vorspann.tex mit gesetztem Schalter.

\newif\ifkorrekturansicht
\korrekturansichttrue

\input{../tex-inputs/latex-vorspann}


\section[ Felix Salten an Arthur Schnitzler, 15. 4. 1906]{L03418 Felix Salten an Arthur Schnitzler, 15. 4. 1906}
\nopagebreak\mylabel{L03418v}
\rehead{ }\normalsize\beginnumbering\briefempfaengerindex{Schnitzler, Arthur@\textsc{Schnitzler, Arthur}!zzzSalten, Felix@\emph{von Felix Salten}!1906-04-151@{15. 4. 1906}|(be}
\toendnotes[C]{\smallbreak\pagebreak[2]}\Standort{CUL, Schnitzler, B 89, B 1.}
\physDesc{Bildpostkarte, 126 Zeichen
\newline{}Handschrift: schwarze Tinte, lateinische Kurrent
\newline{}Versand: Stempel: »\nobreak{}\oindex{Berlin@\textbf{Berlin}|pwk}Berlin\textcolor{gray}{,}
                                       N. W., 16. 4. 06, 3–4 N.\nobreak{}«.  
\newline{}Ordnung: mit Bleistift von unbekannter Hand nummeriert: »209« }\pstart{}{\pb}Herrn D\textsuperscript{r} Arthur Schnitzler\pend{}\pstart{}Wien XVIII.\oindex{Berlin@\textbf{Berlin}|pw}\pend{}\pstart{}Spöttelgasse 7\oindex{Edmund-Weiss-Gasse 7@\textbf{Edmund-Weiß-Gasse 7}|pw}\pend{}{\bigskip}
\pstart
           {\pb}\textcolor{gray}{\textbf{Gruss aus Sanssouci-Potsdam\oindex{Schloss Sanssouci@\textbf{Schloss Sanssouci}|pw}}}\hfill \textcolor{gray}{\textbf{Sicilianischer Garten\oindex{Garten von Sanssouci@\textbf{Garten von Sanssouci}|pw} mit Bogenschützen\pwindex{Bogenschuetze@\emph{Bogenschütze}|pw} v. Prof. Geiger\pwindex{Geyger, Ernst Moritz 1861-11-09 – 1941-12-29@\textsc{Geyger, Ernst Moritz} (1861-11-09 – 1941-12-29), \emph{Radierer/Radiererin, Bildhauer/Bildhauerin}|pw}}}\pend
           \vspace{1em}
\pstart
           \noindent{}{\pb}Auf einer schönen
               Automobiltour\pend
           
\pstart
           herzliche Grüße{\\[\baselineskip]}\spacefill\mbox{Salten}\pend
           \leftskip=0em{}
\pstart
           {\pb}Ostersonntag 15. IV. 06.\pend
           \selectlanguage{ngerman}\endnumbering\briefempfaengerindex{Schnitzler, Arthur@\textsc{Schnitzler, Arthur}!zzzSalten, Felix@\emph{von Felix Salten}!1906-04-151@{15. 4. 1906}|)be}\mylabel{L03418h}  \normalsize

\doendnotes{C}
\bigskip
\vfill

\clearpage

\footnotesize

\lohead{\textsc{register}}

% Definiere theindex-Environment komplett neu ohne reledmac
\makeatletter
\renewenvironment{theindex}{%
  \section*{\indexname}%
  \setlength{\parindent}{0pt}%
  \setlength{\parskip}{0pt plus 0.3pt}%
  \let\item\@idxitem
}{%
  \clearpage
}
\makeatother

\IfFileExists{\jobname-pw.ind}{\input{\jobname-pw.ind}}{}

\end{document}

      