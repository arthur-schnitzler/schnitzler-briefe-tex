%% latex-korrekturansicht-vorspann.tex
%% Vorspann für die Korrekturansicht.
%% Lädt die gemeinsame Datei latex-vorspann.tex mit gesetztem Schalter.

\newif\ifkorrekturansicht
\korrekturansichttrue

\input{../tex-inputs/latex-vorspann}


\section[Thomas Mann an Arthur Schnitzler, 1. 2. 1929]{L02508 Thomas Mann an Arthur Schnitzler, 1. 2. 1929}
\nopagebreak\mylabel{L02508v}
\rehead{ }\normalsize\beginnumbering\briefempfaengerindex{Schnitzler, Arthur@\textsc{Schnitzler, Arthur}!zzzMann, Thomas@\emph{von Thomas Mann}!1929-02-011@{1. 2. 1929}|(be}
\toendnotes[C]{\smallbreak\pagebreak[2]}\Standort{CUL, Schnitzler, B 67.}
\physDesc{Brief, 1 Blatt, 2 Seiten, 1435 Zeichen
\newline{}Schreibmaschine
\newline{}Handschrift: schwarze Tinte (\noindent{}Unterschrift)
\newline{}Schnitzler: 1) mit rotem Buntstift mit Datum einer nicht erhaltenen Antwort 
                                 versehen oder eine Wiederholung von Monat und Jahr mit
                                 vorangestellter Jahreszahl: »29 II–« und beschrieben: »\textsc{Amerika\oindex{Amerika@\textbf{Amerika}, \emph{kein passender Code gefunden}|pw}, Liste}«  2) mit rotem Buntstift mehrere Unterstreichungen}
\buchAbdrucke{\weitereDrucke{\emph{Modern Austrian Literature}, Jg. 7 (1974) Nr. 1/2, S. 26.} }\toendnotes[C]{\smallbreak}
\pstart
           {\pb}\textcolor{gray}{\textbf{DR. THOMAS MANN}}\hfill \textcolor{gray}{\textbf{MÜNCHEN\oindex{Muenchen@\textbf{München}, \emph{P.PPLA}|pw} den}}{ }1. II. 29.\pend
           
\pstart
           \raggedleft{}\textcolor{gray}{\textbf{POSCHINGERSTR. 1\oindex{Poschingerstrasse@\textbf{Poschingerstraße}, \emph{Straße (K.STR)}|pw}}}\pend
           
\pstart{}Lieber, verehrter Herr Doktor Schnitzler,\pend\vspace{0.5em}
\pstart
           Haben Sie Dank für Ihre Zeilen! Ich war glücklich, sie zu erhalten, denn ich hatte
               ein schlechtes Gewissen, weil ich in Wien\oindex{Wien@\textbf{Wien}, \emph{A.ADM2}|pw} nicht
               versucht habe, mich mit Ihnen in Verbindung zu setzen, Sie zu sehen, zu sprechen. Ich
               brauche kaum zu sagen, warum es nicht geschah. Es war Scheu vor dem \label{K_L02508-1v}\edtext{schrecklichen Kummer}{\lemma{\textnormal{\emph{schrecklichen Kummer}}}\Cendnote{\textnormal{Am 26. 7. 1928
                  war Schnitzlers Tochter Lili\pwindex{Cappellini, Lili 13.09.1909 – 26.07.1928@\textsc{Cappellini, Lili} (13.09.1909 – 26.07.1928)|pwk} durch ein Unglück 
                  an einer selbst zugefügten Schussverletzung gestorben.
               }}}\label{K_L02508-1}, den das Schicksal Ihnen kürzlich zugefügt hat, und von dem wir alle mit
               Ihnen so tief erschüttert wurden. Ich wusste nicht, ob Sie aufgelegt seien, meinen
               Besuch oder irgend welchen anderen zu empfangen. Aber Ihre Zeilen lassen mich hoffen,
               dass ich bald wieder einmal die Freude haben werde, Ihnen die Hand zu drücken.\pend
           
\pstart
           Nun zu Ihrer Anfrage. Ich habe die Aufforderung des »Book of the Month Club\orgindex{Book of The Month Club@Book of The Month Club|pw}« erhalten und zustimmend beantwortet. Die Leute
               stellen sich ein präsentables Komitee zusammen, und da sie notorisch viel Geld haben,
               finde ich nichts Böses darin, mir meinen Namen honorieren zu lassen, zumal es ja in
               der Tat nicht ganz ausschliesslich der Name ist, sondern ich durchaus gesonnen bin,
               ihnen von Zeit zu Zeit einen Brief zu schreiben und sie auf deutsche Bücher
               hinzuweisen, die für ihre Veröffentlichungen in Betracht kommen. Das ist eine
               geistige Teilnahme, die sie belohnen {\pb}dürfen. Natürlich würde ich mich freuen, wenn Sie diesen Gesichtspunkt anerkennen
               würden und auch dabei wären.\pend
           
\pstart
           Seien Sie recht herzlich und verehrungsvoll begrüsst{\\[\baselineskip]} von Ihrem ergebenen{\\[\baselineskip]}\spacefill\mbox{{[}hs.:{]} Thomas Mann}\pend
           \leftskip=0em{}\selectlanguage{ngerman}\endnumbering\briefempfaengerindex{Schnitzler, Arthur@\textsc{Schnitzler, Arthur}!zzzMann, Thomas@\emph{von Thomas Mann}!1929-02-011@{1. 2. 1929}|)be}\mylabel{L02508h}  \normalsize

\doendnotes{C}
\bigskip
\vfill

\clearpage

\footnotesize

\lohead{\textsc{register}}

% Definiere theindex-Environment komplett neu ohne reledmac
\makeatletter
\renewenvironment{theindex}{%
  \section*{\indexname}%
  \setlength{\parindent}{0pt}%
  \setlength{\parskip}{0pt plus 0.3pt}%
  \let\item\@idxitem
}{%
  \clearpage
}
\makeatother

\IfFileExists{\jobname-pw.ind}{\input{\jobname-pw.ind}}{}

\end{document}

      