%% latex-leseansicht-vorspann.tex
%% Vorspann für die Leseansicht.
%% Lädt die gemeinsame Datei latex-vorspann.tex mit nicht gesetztem Schalter.

\newif\ifkorrekturansicht
\korrekturansichtfalse

\input{../tex-inputs/latex-vorspann}


\section[Paul Goldmann an Arthur Schnitzler, 7. 1. 1891]{L02659 Paul Goldmann an Arthur Schnitzler, 7. 1. 1891}
\nopagebreak\mylabel{L02659v}
\rehead{ }\normalsize\beginnumbering\briefempfaengerindex{Schnitzler, Arthur@\textsc{Schnitzler, Arthur}!zzzGoldmann, Paul@\emph{von Paul Goldmann}!1891-01-072@{7. 1. 1891}|(be}
\toendnotes[C]{\smallbreak\pagebreak[2]}
\correspDesc{Versand  durch Paul Goldmann am 7. 1. 1891 in Wien
\newline{}Übermittlung  am 8. 1. 1891 in Wien
\newline{}Erhalt  durch Arthur Schnitzler am 8. 1. 1891 in Wien}\toendnotes[C]{\smallbreak}
\Standort{DLA, A:Schnitzler, HS.NZ85.1.3162.}
\physDesc{Postkarte, 461 Zeichen
\newline{}Handschrift: blaue Tinte, deutsche Kurrent
\newline{}Versand: 1) Stempel: »\nobreak{}\oindex{Wien@\textbf{Wien}, \emph{Verwaltungsgebiet}|pwk}Wien
                                       {[}T{]}elegrafen-Centrale, 8-\textcolor{gray}{1-9}1, 12 V\nobreak{}«.   2) Stempel: »\nobreak{}\oindex{Wien@\textbf{Wien}, \emph{Verwaltungsgebiet}|pwk}Wien Kärntnerring, 8/1 91, 12–1 N\nobreak{}«. 
\newline{}Schnitzler: mit Bleistift das Datum »8/1 91« vermerkt }\toendnotes[C]{\smallbreak}\pstart{}\textsc{{\pb}Herrn}\pend{}\pstart{}\textsc{Dr. Arthur Schnitzler}\pend{}\pstart{}\textsc{I. Giselastraſse 11\textsubscript{III}\oindex{Wien@\textbf{Wien}!I., Innere Stadt@\textbf{I., Innere Stadt}!Ordination Arthur Schnitzler [Bösendorferstraße 11]@\textbf{Ordination Arthur Schnitzler [Bösendorferstraße 11]}, \emph{Ordination}|pw}.}\pend{}\pstart{}\textsc{Wien\oindex{Wien@\textbf{Wien}, \emph{Verwaltungsgebiet}|pw}}\pend{}{\bigskip}\vspace{1em}
\pstart
           {\pb}Wien\oindex{Wien@\textbf{Wien}, \emph{Verwaltungsgebiet}|pw} den \textsuperscript{7}/\textsubscript{1} 1891.\pend
           \vspace{0.5em}
\pstart
           Lieber Arthur! Herzlichſten Dank für Deine
               Liebenswürdigkeit! Das Referat\pwindex{(Burgtheater.) [Rezension des Gastspiels von Anna Hochenburger]@\emph{(Burgtheater.) [Rezension des Gastspiels von Anna Hochenburger]}|pwv}{ }ſchreib’{ }ſo groß wie Du willſt, 30, 40, 50 Zeilen; nur – nochmals – darf Niemand
               erfahren, daß Du es geſchrieben. Wenn du \label{K_L02659-1v}\edtext{heut{ }Abend}{\lemma{\textnormal{\emph{heut Abend}}}\Cendnote{\textnormal{Goldmann\pwindex{Goldmann, Paul 31.\,1.\,1865 Breslau – 25.\,9.\,1935 Wien@\textsc{Goldmann, Paul} (31.\,1.\,1865 Breslau – 25.\,9.\,1935 Wien), \emph{Schriftsteller, Journalist}|pwk} datierte die Postkarte auf den
                     7. 1. 1891, während der Poststempel den 8. 1. 1891 ausweist, was sich auch durch die erwähnte Theateraufführung
                  belegen lässt. Erklärbar wäre das damit, dass die Karte zwar tatsächlich am 7. verfasst wurde, aber zu einer so späten Uhrzeit,
                  dass nicht mehr die Theateraufführungen des gleichen Tages, sondern nur die vom Folgetag gemeint sein konnten.}}}\label{K_L02659-1} Zeit haſt, würde ich mich{ }ſehr
               freuen, Dich im »\label{K_L02659-2v}\edtext{Theater an der Wien\orgindex{Theater an der Wien@Theater an der Wien|pw}}{\lemma{\textnormal{\emph{Theater an der Wien}}}\Cendnote{\textnormal{Tatsächlich sahen sich beide am 8. 1. 1891 die
                  Operette \emph{Boccaccio}\pwindex{\textcolor{red}{\textsuperscript{XXXX indx1}}!Boccaccio. Komische Operette in 3 Acten@\strich\emph{Boccaccio. Komische Operette in 3 Acten}|pwk}\pwindex{Suppè, Franz von 18.\,4.\,1819 Split – 21.\,5.\,1895 Wien@\textsc{Suppè, Franz von} (18.\,4.\,1819 Split – 21.\,5.\,1895 Wien), \emph{Komponist}!Boccaccio. Komische Operette in 3 Acten@\strich\emph{Boccaccio. Komische Operette in 3 Acten}|pwk}\pwindex{\textcolor{red}{\textsuperscript{XXXX indx1}}!Boccaccio. Komische Operette in 3 Acten@\strich\emph{Boccaccio. Komische Operette in 3 Acten}|pwk} von Franz von Suppè\pwindex{Suppè, Franz von 18.\,4.\,1819 Split – 21.\,5.\,1895 Wien@\textsc{Suppè, Franz von} (18.\,4.\,1819 Split – 21.\,5.\,1895 Wien), \emph{Komponist}|pwk} an.}}}\label{K_L02659-2}« Loge N\textsuperscript{o} 6, 1. Stock, zu{ }ſehen \introOben{}Karte brauchſt Du keine.\introOben{}
                  (\textsc{Boccaccio\pwindex{\textcolor{red}{\textsuperscript{XXXX indx1}}!Boccaccio. Komische Operette in 3 Acten@\strich\emph{Boccaccio. Komische Operette in 3 Acten}|pw}\pwindex{Suppè, Franz von 18.\,4.\,1819 Split – 21.\,5.\,1895 Wien@\textsc{Suppè, Franz von} (18.\,4.\,1819 Split – 21.\,5.\,1895 Wien), \emph{Komponist}!Boccaccio. Komische Operette in 3 Acten@\strich\emph{Boccaccio. Komische Operette in 3 Acten}|pw}\pwindex{\textcolor{red}{\textsuperscript{XXXX indx1}}!Boccaccio. Komische Operette in 3 Acten@\strich\emph{Boccaccio. Komische Operette in 3 Acten}|pw}}). Schreib’ mir, ob Du kommen kannſt.\pend
           \pstart Herzl. Gruß Dein \spacefill\mbox{P. G.}\pend{}\selectlanguage{ngerman}\endnumbering\briefempfaengerindex{Schnitzler, Arthur@\textsc{Schnitzler, Arthur}!zzzGoldmann, Paul@\emph{von Paul Goldmann}!1891-01-072@{7. 1. 1891}|)be}\mylabel{L02659h}  \newcommand{\dateiname}{L02659}\newcommand{\titel}{Paul Goldmann an Arthur Schnitzler, 7. 1. 1891}\newcommand{\editorInnen}{Martin Anton Müller und Laura Untner}%% latex-leseansicht-abspann.tex
%% Abspann für die Leseansicht.
%% Der Schalter \ifkorrekturansicht ist bereits durch den Vorspann gesetzt.

%% latex-abspann.tex
%% Gemeinsamer Abspann für Korrekturansicht und Leseansicht.
%% Setzt den Schalter \ifkorrekturansicht voraus (gesetzt in den
%% einbindenden Dateien latex-korrekturansicht-abspann.tex bzw.
%% latex-leseansicht-abspann.tex).
%% ---------------------------------------------------------------

\normalsize

% Das esempio-Environment wird nur in der Leseansicht benötigt
\ifkorrekturansicht\else
\newenvironment{esempio}[3]%
{
    \vspace{1.5ex}
    \rlap{\underline{#1}}
    \par
    \setlength{\parindent}{0cm}
    \nopagebreak
    \leftskip=#2cm
    \rightskip=#3cm
}
{
    \par
}
\fi

\doendnotes{C}
\bigskip
\vfill

\clearpage

\footnotesize

\ifkorrekturansicht
  \lohead{\textsc{register}}
\fi

% theindex-Environment neu definieren ohne reledmac
\makeatletter
\renewenvironment{theindex}{%
  \ifkorrekturansicht
    \section*{\indexname}%
  \else
    \subsubsection*{Index der erwähnten Entitäten}%
  \fi
  \setlength{\parindent}{0pt}%
  \setlength{\parskip}{0pt plus 0.3pt}%
  \let\item\@idxitem
}{%
  \ifkorrekturansicht\clearpage\fi
}
\makeatother

\IfFileExists{\jobname-pw.ind}{\input{\jobname-pw.ind}}{}

% Quellenangabe nur in der Leseansicht
\ifkorrekturansicht\else
% Fallback-Definitionen, falls die .tex-Datei \titel etc. nicht gesetzt hat
\providecommand{\titel}{}
\providecommand{\editorInnen}{}
\providecommand{\dateiname}{\jobname}

\vspace{3cm}

\vfill

\footnotesize
\textsc{Quelle}: \titel. Herausgegeben von {\editorInnen}. In: \emph{Arthur Schnitzler: Briefwechsel mit Autorinnen und Autoren}.
 Digitale Edition, https://schnitzler-briefe.acdh.oeaw.ac.at/{\dateiname}.html (Stand \today)
\fi

\end{document}


