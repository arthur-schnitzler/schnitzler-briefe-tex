%% latex-leseansicht-vorspann.tex
%% Vorspann für die Leseansicht.
%% Lädt die gemeinsame Datei latex-vorspann.tex mit nicht gesetztem Schalter.

\newif\ifkorrekturansicht
\korrekturansichtfalse

\input{../tex-inputs/latex-vorspann}


\section[ Paul Goldmann an Arthur Schnitzler, 27. 10. {[}1896{]}]{L02788 Paul Goldmann an Arthur Schnitzler,  27. 10. [1896]}
\nopagebreak\mylabel{L02788v}
\rehead{ }\normalsize\beginnumbering\briefempfaengerindex{Schnitzler, Arthur@\textsc{Schnitzler, Arthur}!zzzGoldmann, Paul@\emph{von Paul Goldmann}!1896-10-272@{27. 10. [1896]}|(be}
\toendnotes[C]{\smallbreak\pagebreak[2]}
\correspDesc{Versand  durch Paul Goldmann am 27. 10. [1896] in Paris
\newline{}Erhalt  durch Arthur Schnitzler im Zeitraum [28. 10. 1896 – 1. 11. 1896?] in Berlin}\toendnotes[C]{\smallbreak}
\Standort{DLA, A:Schnitzler, HS.NZ85.1.3166.}
\physDesc{Brief, 1 Blatt, 4 Seiten, 1346 Zeichen
\newline{}Handschrift: blaue Tinte, deutsche Kurrent
\newline{}Schnitzler: mit Bleistift das Jahr »96« vermerkt }\toendnotes[C]{\smallbreak}
\pstart
           {\pb}\textcolor{gray}{\textbf{\textbf{Frankfurter Zeitung\orgindex{Frankfurter Zeitung@Frankfurter Zeitung|pw}}}}\pend
           
\pstart
           \textcolor{gray}{\textbf{(\begin{otherlanguage}{french}Gazette de Francfort\end{otherlanguage}\orgindex{Frankfurter Zeitung@Frankfurter Zeitung|pw}).}}\pend
           
\pstart
           \textcolor{gray}{\textbf{\textbf{\begin{otherlanguage}{french}Fondateur M.\end{otherlanguage}{ }L. Sonnemann\pwindex{Sonnemann, Leopold 29.\,10.\,1831 Höchberg – 30.\,10.\,1909 Frankfurt am Main@\textsc{Sonnemann, Leopold} (29.\,10.\,1831 Höchberg – 30.\,10.\,1909 Frankfurt am Main), \emph{Journalist, Herausgeber}|pw}.}}}\pend
           
\pstart
           \begin{otherlanguage}{french}\textcolor{gray}{\textbf{Journal\pwindex{Frankfurter Zeitung@\emph{Frankfurter Zeitung}|pwv} politique,
                        financier,}}\end{otherlanguage}\pend
           
\pstart
           \begin{otherlanguage}{french}\textcolor{gray}{\textbf{commercial et littéraire.}}\end{otherlanguage}\pend
           
\pstart
           \begin{otherlanguage}{french}\textcolor{gray}{\textbf{\textbf{Paraissant trois fois par jour.}}}\end{otherlanguage}\hfill \textsc{Paris\oindex{Paris@\textbf{Paris}, \emph{Hauptstadt}|pw}}, 27. October.\pend
           
\pstart
           \begin{otherlanguage}{french}\textcolor{gray}{\textbf{\textbf{Bureau à Paris\oindex{Paris@\textbf{Paris}, \emph{Hauptstadt}|pw}}}}\end{otherlanguage}\pend
           
\pstart
           \begin{otherlanguage}{french}\textcolor{gray}{\textbf{\textbf{24. Rue Feydeau\oindex{rue Feydeau@\textbf{rue Feydeau}, \emph{Straße}|pw}.}}}\end{otherlanguage}\pend
           
\pstart{}Mein lieber Freund,\pend\vspace{0.5em}
\pstart
           Deine lieben Briefe treffen mich in einer Zeit größter Arbeit. Ich kann Dir
               einſtweilen nur mit flüchtigen Worten{ }ſagen, wie{ }ſehr ich mich freue, daß \label{K_L02788-1v}\edtext{der große Tag}{\lemma{\textnormal{\emph{der große Tag}}}\Cendnote{\textnormal{die Uraufführung\eventindex{Deutsches Theater Berlin@\textbf{Deutsches Theater Berlin}!Uraufführung von Freiwild, 3.11.1896@Uraufführung von Freiwild, 3.11.1896|pwkv} von \emph{Freiwild}\pwindex{Schnitzler, Arthur 15.\,5.\,1862 Wien – 21.\,10.\,1931 ebd.@\textsc{Schnitzler, Arthur} (15.\,5.\,1862 Wien – 21.\,10.\,1931 ebd.), \emph{Schriftsteller, Mediziner}!Freiwild. Schauspiel in 3 Akten@\strich\emph{Freiwild. Schauspiel in 3 Akten}|pwk} am 3. 11. 1896 am Deutschen Theater\oindex{Deutsches Theater Berlin@\textbf{Deutsches Theater Berlin}, \emph{Theater}|pwk} in
                     Berlin\oindex{Berlin@\textbf{Berlin}, \emph{Hauptstadt}|pwk}}}}\label{K_L02788-1}{ }ſo nahe iſt. Ich heiße Dich \label{K_L02788-2v}\edtext{willkommen in Berlin\oindex{Berlin@\textbf{Berlin}, \emph{Hauptstadt}|pw}}{\lemma{\textnormal{\emph{willkommen in Berlin}}}\Cendnote{\textnormal{Schnitzler hielt sich vom 26. 10. 1896 bis zum
                     9. 11. 1896 in
                     Berlin\oindex{Berlin@\textbf{Berlin}, \emph{Hauptstadt}|pwk} auf.}}}\label{K_L02788-2} und wünſche Dir einen
               frohen und glücklichen Aufenthalt. Nächſtens antworte ich Dir ausführlicher auf
               Deinen letzten längeren Brief, der mich{ }ſehr erfreut hat. {\pb}Warte jedenfalls nicht auf meine Antwort und{ }ſchreibe mir gleich ein kurzes Wort über Deine Berlin\oindex{Berlin@\textbf{Berlin}, \emph{Hauptstadt}|pw}er \strikeout{E\textcolor{gray}{×}} Eindrücke und insbeſondere \strikeout{aber} darüber, wie
               Dein Stück\pwindex{Schnitzler, Arthur 15.\,5.\,1862 Wien – 21.\,10.\,1931 ebd.@\textsc{Schnitzler, Arthur} (15.\,5.\,1862 Wien – 21.\,10.\,1931 ebd.), \emph{Schriftsteller, Mediziner}!Freiwild. Schauspiel in 3 Akten@\strich\emph{Freiwild. Schauspiel in 3 Akten}|pwv} Dir \label{K_L02788-3v}\edtext{auf den Proben gefällt}{\lemma{\textnormal{\emph{auf den Proben gefällt}}}\Cendnote{\textnormal{Schnitzler notierte im \emph{Tagebuch}\pwindex{Schnitzler, Arthur 15.\,5.\,1862 Wien – 21.\,10.\,1931 ebd.@\textsc{Schnitzler, Arthur} (15.\,5.\,1862 Wien – 21.\,10.\,1931 ebd.), \emph{Schriftsteller, Mediziner}!Tagebuch@\strich\emph{Tagebuch}|pwk} zunächst äußerst negative (vgl. A. S.: \emph{Tagebuch}, 28. 10. 1896), später aber
                  auch positivere (2. 11. 1896) Eindrücke von den \emph{Freiwild}\pwindex{Schnitzler, Arthur 15.\,5.\,1862 Wien – 21.\,10.\,1931 ebd.@\textsc{Schnitzler, Arthur} (15.\,5.\,1862 Wien – 21.\,10.\,1931 ebd.), \emph{Schriftsteller, Mediziner}!Freiwild. Schauspiel in 3 Akten@\strich\emph{Freiwild. Schauspiel in 3 Akten}|pwk}-Proben.}}}\label{K_L02788-3}. Einen Rath nur in Kürze: Ganz Deutſchland\oindex{Deutschland@\textbf{Deutschland}|pw}{ }ſteht unter dem Banne des
                  Eindruck\textcolor{gray}{e}s, den die \label{K_L02788-4v}\edtext{Affaire \textsc{Bruesewitz\pwindex{Brüsewitz, Henning von 1862 – 24.\,1.\,1900@\textsc{Brüsewitz, Henning von} (1862 – 24.\,1.\,1900), \emph{Militär, Offizier}|pw}}}{\lemma{\textnormal{\emph{Affaire Bruesewitz}}}\Cendnote{\textnormal{Siehe XXXX Auszeichnungsfehler: Dokument L02787 nicht gefunden. }}}\label{K_L02788-4} gemacht
               hat. Man lechzt nach einem Wort, das dieſe{ }ſchurkiſchen Officiers-Feiglinge geißelt.
               Keiner kann beſſer dieſes Wort aus{\pb}ſprechen, als Du.
               Leg’ es Deinem anſtändigen Officier in den Mund, in der Scene\pwindex{Schnitzler, Arthur 15.\,5.\,1862 Wien – 21.\,10.\,1931 ebd.@\textsc{Schnitzler, Arthur} (15.\,5.\,1862 Wien – 21.\,10.\,1931 ebd.), \emph{Schriftsteller, Mediziner}!Freiwild. Schauspiel in 3 Akten@\strich\emph{Freiwild. Schauspiel in 3 Akten}|pwv}, wo er{ }ſagt: \label{K_L02788-5v}\edtext{Solche Leute haben im Frieden eigentlich
               gar keine Exiſtenz-Berechtigung}{\lemma{\textnormal{\emph{Solche … Existenz-Berechtigung}}}\Cendnote{\textnormal{Aussage
                  des Offiziers Rohnstedt\pwindex{Schnitzler, Arthur 15.\,5.\,1862 Wien – 21.\,10.\,1931 ebd.@\textsc{Schnitzler, Arthur} (15.\,5.\,1862 Wien – 21.\,10.\,1931 ebd.), \emph{Schriftsteller, Mediziner}!Freiwild. Schauspiel in 3 Akten@\strich\emph{Freiwild. Schauspiel in 3 Akten}|pwkv} am
                  Ende des ersten Akts\pwindex{Schnitzler, Arthur 15.\,5.\,1862 Wien – 21.\,10.\,1931 ebd.@\textsc{Schnitzler, Arthur} (15.\,5.\,1862 Wien – 21.\,10.\,1931 ebd.), \emph{Schriftsteller, Mediziner}!Freiwild. Schauspiel in 3 Akten@\strich\emph{Freiwild. Schauspiel in 3 Akten}|pwkv}}}}\label{K_L02788-5}. Laß ihn noch etwas Allgemeines, Kräftiges, Erlöſendes{ }ſagen. Dieſes Wort
               allein kann den Erfolg des Sück\pwindex{Schnitzler, Arthur 15.\,5.\,1862 Wien – 21.\,10.\,1931 ebd.@\textsc{Schnitzler, Arthur} (15.\,5.\,1862 Wien – 21.\,10.\,1931 ebd.), \emph{Schriftsteller, Mediziner}!Freiwild. Schauspiel in 3 Akten@\strich\emph{Freiwild. Schauspiel in 3 Akten}|pwv}es entſcheiden. Nimm’ meinen \label{K_L02788-6v}\edtext{Rath}{\lemma{\textnormal{\emph{Rath}}}\Cendnote{\textnormal{Über eine
                  Einarbeitung des Vorschlags ist nichts bekannt. Zumindest der Bezug zu der Affäre
                  wurde noch Jahre später hergestellt, beispielsweise: »\begin{otherlanguage}{english}The most celebrated of these was ›Freiwild\pwindex{Schnitzler, Arthur 15.\,5.\,1862 Wien – 21.\,10.\,1931 ebd.@\textsc{Schnitzler, Arthur} (15.\,5.\,1862 Wien – 21.\,10.\,1931 ebd.), \emph{Schriftsteller, Mediziner}!Freiwild. Schauspiel in 3 Akten@\strich\emph{Freiwild. Schauspiel in 3 Akten}|pw}‹, an attack on the duel, that received
                        enormous advertizing from the strange coincidence that, while the play\pwindex{Schnitzler, Arthur 15.\,5.\,1862 Wien – 21.\,10.\,1931 ebd.@\textsc{Schnitzler, Arthur} (15.\,5.\,1862 Wien – 21.\,10.\,1931 ebd.), \emph{Schriftsteller, Mediziner}!Freiwild. Schauspiel in 3 Akten@\strich\emph{Freiwild. Schauspiel in 3 Akten}|pwv} was in rehearsal,
                        Lieut. von Brüsewitz\pwindex{Brüsewitz, Henning von 1862 – 24.\,1.\,1900@\textsc{Brüsewitz, Henning von} (1862 – 24.\,1.\,1900), \emph{Militär, Offizier}|pw}, by the brutal
                        killing of a civilian\pwindex{Siepmann, Theodor †~11.\,10.\,1896 Karlsruhe@\textsc{Siepmann, Theodor} (†~11.\,10.\,1896 Karlsruhe), \emph{Handwerker, Mechaniker}|pwv} in a Carlsruhe\oindex{Karlsruhe@\textbf{Karlsruhe}, \emph{Hauptstadt}|pw}{ }restaurant\oindex{Café Tannhäuser@\textbf{Café Tannhäuser}, \emph{Kaffeehaus}|pwv},
                        vindicated his ›military honor‹ exactly as the play\pwindex{Schnitzler, Arthur 15.\,5.\,1862 Wien – 21.\,10.\,1931 ebd.@\textsc{Schnitzler, Arthur} (15.\,5.\,1862 Wien – 21.\,10.\,1931 ebd.), \emph{Schriftsteller, Mediziner}!Freiwild. Schauspiel in 3 Akten@\strich\emph{Freiwild. Schauspiel in 3 Akten}|pwv} had foretold an officer would
                        be obliged to do. The excitement over the Carlsruhe\oindex{Karlsruhe@\textbf{Karlsruhe}, \emph{Hauptstadt}|pw} incident rushed the play\pwindex{Schnitzler, Arthur 15.\,5.\,1862 Wien – 21.\,10.\,1931 ebd.@\textsc{Schnitzler, Arthur} (15.\,5.\,1862 Wien – 21.\,10.\,1931 ebd.), \emph{Schriftsteller, Mediziner}!Freiwild. Schauspiel in 3 Akten@\strich\emph{Freiwild. Schauspiel in 3 Akten}|pwv} to such a huge popularity that one of the Germ\oindex{Deutschland@\textbf{Deutschland}|pwv}an comic papers
                        showed a cartoon of Manager Brahm\pwindex{Brahm, Otto 5.\,2.\,1856 Hamburg – 28.\,11.\,1912 Berlin@\textsc{Brahm, Otto} (5.\,2.\,1856 Hamburg – 28.\,11.\,1912 Berlin), \emph{Theaterleiter, Regisseur}|pw}, of
                        the Deutsches Theater\orgindex{Deutsches Theater Berlin@Deutsches Theater Berlin|pw}, paying out
                        royalties to the leading playwrights of the season, when Lieut. Brüsewitz\pwindex{Brüsewitz, Henning von 1862 – 24.\,1.\,1900@\textsc{Brüsewitz, Henning von} (1862 – 24.\,1.\,1900), \emph{Militär, Offizier}|pw} enters saying: ›I’ve come
                        for my share of the royalties on ›Freiwild\pwindex{Schnitzler, Arthur 15.\,5.\,1862 Wien – 21.\,10.\,1931 ebd.@\textsc{Schnitzler, Arthur} (15.\,5.\,1862 Wien – 21.\,10.\,1931 ebd.), \emph{Schriftsteller, Mediziner}!Freiwild. Schauspiel in 3 Akten@\strich\emph{Freiwild. Schauspiel in 3 Akten}|pw}‹!\end{otherlanguage}« ([O. V.]: \emph{Arthur Schnitzler.
                        Dramatist of the Twilight Soul}\pwindex{Arthur Schnitzler. Dramatist of the Twilight Soul.@\emph{Arthur Schnitzler. Dramatist of the Twilight Soul.}|pwk}. In: \emph{Current Literature}\pwindex{Current Literature@\emph{Current Literature}|pwk}, Bd. 51, H. 6, Dezember 1911, S. 670–672, hier: S. 671.)}}}\label{K_L02788-6} an, ich
               glaube, ich habe Dir{ }ſelten{ }ſo gut gerathen! {\dotsfour}\pend
           
\pstart
           Auf ein Telegramm am Tage nach der \textsc{Première\pwindex{Schnitzler, Arthur 15.\,5.\,1862 Wien – 21.\,10.\,1931 ebd.@\textsc{Schnitzler, Arthur} (15.\,5.\,1862 Wien – 21.\,10.\,1931 ebd.), \emph{Schriftsteller, Mediziner}!Freiwild. Schauspiel in 3 Akten@\strich\emph{Freiwild. Schauspiel in 3 Akten}|pwv}}{ }{\pb}rechne ich mit Sicherheit.\pend
           
\pstart
           Viele treue Grüße!\pend
           
\pstart
           Und ein inniges Glückauf!\pend
           
\pstart
           Dein treuer {\\[\baselineskip]}\spacefill\mbox{Paul Goldmann}\pend
           \leftskip=0em{}
\pstart
           \noindent{}Schönen Gruß an den \textsc{Dr. Bie\pwindex{Bie, Oskar 9.\,2.\,1864 Breslau – 21.\,4.\,1938 Berlin@\textsc{Bie, Oskar} (9.\,2.\,1864 Breslau – 21.\,4.\,1938 Berlin), \emph{Schriftsteller, Journalist, Redakteur}|pw}}, wenn Du \label{K_L02788-7v}\edtext{ihn{ }ſiehſt}{\lemma{\textnormal{\emph{ihn siehst}}}\Cendnote{\textnormal{Schnitzler traf am 31. 10. 1896, 5. 11. 1896 und
                        7. 11. 1896
                     auf Oskar Bie\pwindex{Bie, Oskar 9.\,2.\,1864 Breslau – 21.\,4.\,1938 Berlin@\textsc{Bie, Oskar} (9.\,2.\,1864 Breslau – 21.\,4.\,1938 Berlin), \emph{Schriftsteller, Journalist, Redakteur}|pwk}.}}}\label{K_L02788-7}\pend
           \selectlanguage{ngerman}\endnumbering\briefempfaengerindex{Schnitzler, Arthur@\textsc{Schnitzler, Arthur}!zzzGoldmann, Paul@\emph{von Paul Goldmann}!1896-10-272@{27. 10. [1896]}|)be}\mylabel{L02788h}  \newcommand{\dateiname}{L02788}\newcommand{\titel}{Paul Goldmann an Arthur Schnitzler, 27. 10. [1896]}\newcommand{\editorInnen}{Martin Anton Müller und Laura Untner}%% latex-leseansicht-abspann.tex
%% Abspann für die Leseansicht.
%% Der Schalter \ifkorrekturansicht ist bereits durch den Vorspann gesetzt.

%% latex-abspann.tex
%% Gemeinsamer Abspann für Korrekturansicht und Leseansicht.
%% Setzt den Schalter \ifkorrekturansicht voraus (gesetzt in den
%% einbindenden Dateien latex-korrekturansicht-abspann.tex bzw.
%% latex-leseansicht-abspann.tex).
%% ---------------------------------------------------------------

\normalsize

% Das esempio-Environment wird nur in der Leseansicht benötigt
\ifkorrekturansicht\else
\newenvironment{esempio}[3]%
{
    \vspace{1.5ex}
    \rlap{\underline{#1}}
    \par
    \setlength{\parindent}{0cm}
    \nopagebreak
    \leftskip=#2cm
    \rightskip=#3cm
}
{
    \par
}
\fi

\doendnotes{C}
\bigskip
\vfill

\clearpage

\footnotesize

\ifkorrekturansicht
  \lohead{\textsc{register}}
\fi

% theindex-Environment neu definieren ohne reledmac
\makeatletter
\renewenvironment{theindex}{%
  \ifkorrekturansicht
    \section*{\indexname}%
  \else
    \subsubsection*{Index der erwähnten Entitäten}%
  \fi
  \setlength{\parindent}{0pt}%
  \setlength{\parskip}{0pt plus 0.3pt}%
  \let\item\@idxitem
}{%
  \ifkorrekturansicht\clearpage\fi
}
\makeatother

\IfFileExists{\jobname-pw.ind}{\input{\jobname-pw.ind}}{}

% Quellenangabe nur in der Leseansicht
\ifkorrekturansicht\else
% Fallback-Definitionen, falls die .tex-Datei \titel etc. nicht gesetzt hat
\providecommand{\titel}{}
\providecommand{\editorInnen}{}
\providecommand{\dateiname}{\jobname}

\vspace{3cm}

\vfill

\footnotesize
\textsc{Quelle}: \titel. Herausgegeben von {\editorInnen}. In: \emph{Arthur Schnitzler: Briefwechsel mit Autorinnen und Autoren}.
 Digitale Edition, https://schnitzler-briefe.acdh.oeaw.ac.at/{\dateiname}.html (Stand \today)
\fi

\end{document}


