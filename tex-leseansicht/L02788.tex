%% latex-leseansicht-vorspann.tex
%% Vorspann für die Leseansicht.
%% Lädt die gemeinsame Datei latex-vorspann.tex mit nicht gesetztem Schalter.

\newif\ifkorrekturansicht
\korrekturansichtfalse

\input{../tex-inputs/latex-vorspann}


         
         \renewcommand{\erwaehntePersonen}{Personen: Oskar Bie, Otto Brahm, Henning von Brüsewitz, Theodor Siepmann, Leopold Sonnemann}
         \renewcommand{\erwaehnteInstitutionen}{Institutionen: Deutsches Theater Berlin, Frankfurter Zeitung}
         \renewcommand{\erwaehnteOrte}{Orte: Berlin, Café Tannhäuser, Deutsches Theater Berlin, Deutschland, Karlsruhe, Paris, rue Feydeau}
         \renewcommand{\erwaehnteWerke}{Werke: Arthur Schnitzler. Dramatist of the Twilight Soul., Current Literature, Frankfurter Zeitung, Freiwild. Schauspiel in 3 Akten, Tagebuch}
               \section[ Paul Goldmann an Arthur Schnitzler, 27. 10. {[}1896{]}]{ Paul Goldmann an Arthur Schnitzler, 27. 10. {[}1896{]}}\nopagebreak\mylabel{v}\rehead{ }\begin{ledgroupsized}[t]{13cm}\normalsize\beginnumbering \toendnotes[C]{\smallbreak\pagebreak[2]} \Standort{DLA, A:Schnitzler, HS.NZ85.1.3166.}
\physDesc{Brief, 1 Blatt, 4 Seiten, 1346 Zeichen
\newline{}Handschrift: blaue Tinte, deutsche Kurrent
\newline{}Schnitzler: mit Bleistift das Jahr »96« vermerkt }\toendnotes[C]{\smallbreak}\pstart
           \noindent{}{\pb}\textcolor{gray}{\textbf{\textbf{Frankfurter Zeitung\orgindex{Frankfurter Zeitung@Frankfurter Zeitung|pw}}}}\pend
           \pstart
           \textcolor{gray}{\textbf{(\begin{otherlanguage}{french}Gazette de Francfort\end{otherlanguage}\orgindex{Frankfurter Zeitung@Frankfurter Zeitung|pw}).}}\pend
           \pstart
           \textcolor{gray}{\textbf{\textbf{\begin{otherlanguage}{french}Fondateur M.\end{otherlanguage}{ }L. Sonnemann\pwindex{Sonnemann, Leopold 1831-10-29 – 1909-10-30@\textsc{Sonnemann, Leopold} (1831-10-29 – 1909-10-30), \emph{Journalist, Herausgeber}|pw}.}}}\pend
           \pstart
           \begin{otherlanguage}{french}\textcolor{gray}{\textbf{Journal\pwindex{?? Werk@Nicht ermittelte Verfasserinnen und Verfasser!Frankfurter Zeitung1856 – 1943@\emph{Frankfurter Zeitung} {[}1856 – 1943{]}|pwv} politique,
                        financier,}}\end{otherlanguage}\pend
           \pstart
           \begin{otherlanguage}{french}\textcolor{gray}{\textbf{commercial et littéraire.}}\end{otherlanguage}\pend
           \pstart
           \begin{otherlanguage}{french}\textcolor{gray}{\textbf{\textbf{Paraissant trois fois par jour.}}}\end{otherlanguage}\hfill \textsc{Paris\oindex{Paris@\textbf{Paris}|pw}}, 27. October.\pend
           \pstart
           \begin{otherlanguage}{french}\textcolor{gray}{\textbf{\textbf{Bureau à Paris\oindex{Paris@\textbf{Paris}|pw}}}}\end{otherlanguage}\pend
           \pstart
           \begin{otherlanguage}{french}\textcolor{gray}{\textbf{\textbf{24. Rue Feydeau\oindex{rue Feydeau@\textbf{rue Feydeau}|pw}.}}}\end{otherlanguage}\pend
           \pstart{}Mein lieber Freund,\pend\pstart
           Deine lieben Briefe treffen mich in einer Zeit größter Arbeit. Ich kann Dir
               einſtweilen nur mit flüchtigen Worten ſagen, wie ſehr ich mich freue, daß \label{K_L02788-1v}\edtext{der große Tag}{\lemma{\textnormal{\emph{der große Tag}}}\Cendnote{\textnormal{die Uraufführung von \emph{Freiwild}\pwindex{Schnitzler, Arthur 15.05.1862 – 21.10.1931@\textsc{Schnitzler, Arthur} (15.05.1862 – 21.10.1931), \emph{Schriftsteller, Mediziner}!Freiwild. Schauspiel in 3 Akten1896@\strich\emph{Freiwild. Schauspiel in 3 Akten} {[}1896{]}|pwk} am 3. 11. 1896 am Deutschen Theater\oindex{Deutsches Theater Berlin@\textbf{Deutsches Theater Berlin}|pwk} in Berlin\oindex{Berlin@\textbf{Berlin}|pwk}}}}\label{K_L02788-1h} ſo nahe iſt. Ich heiße Dich \label{K_L02788-2v}\edtext{willkommen in Berlin\oindex{Berlin@\textbf{Berlin}|pw}}{\lemma{\textnormal{\emph{willkommen in Berlin}}}\Cendnote{\textnormal{Schnitzler\pwindex{Schnitzler, Arthur 15.05.1862 – 21.10.1931@\textsc{Schnitzler, Arthur} (15.05.1862 – 21.10.1931), \emph{Schriftsteller, Mediziner}|pwk} hielt sich von 26. 10. 1896 bis 9. 11. 1896 in Berlin\oindex{Berlin@\textbf{Berlin}|pwk} auf.}}}\label{K_L02788-2h} und wünſche Dir einen frohen
               und glücklichen Aufenthalt. Nächſtens antworte ich Dir ausführlicher auf Deinen
               letzten längeren Brief, der mich ſehr erfreut hat. {\pb}Warte jedenfalls nicht auf meine Antwort und ſchreibe mir gleich ein kurzes Wort
               über Deine Berlin\oindex{Berlin@\textbf{Berlin}|pw}er \strikeout{E\textcolor{gray}{×}} Eindrücke und insbeſondere \strikeout{aber} darüber, wie
               Dein Stück\pwindex{Schnitzler, Arthur 15.05.1862 – 21.10.1931@\textsc{Schnitzler, Arthur} (15.05.1862 – 21.10.1931), \emph{Schriftsteller, Mediziner}!Freiwild. Schauspiel in 3 Akten1896@\strich\emph{Freiwild. Schauspiel in 3 Akten} {[}1896{]}|pwv} Dir \label{K_L02788-3v}\edtext{auf den Proben gefällt}{\lemma{\textnormal{\emph{auf den Proben gefällt}}}\Cendnote{\textnormal{Schnitzler\pwindex{Schnitzler, Arthur 15.05.1862 – 21.10.1931@\textsc{Schnitzler, Arthur} (15.05.1862 – 21.10.1931), \emph{Schriftsteller, Mediziner}|pwk} notierte sich im \emph{Tagebuch}\pwindex{Schnitzler, Arthur 15.05.1862 – 21.10.1931@\textsc{Schnitzler, Arthur} (15.05.1862 – 21.10.1931), \emph{Schriftsteller, Mediziner}!Tagebuch1981 – 2000@\strich\emph{Tagebuch} {[}1981 – 2000{]}|pwk} zunächst äußerst negative (vgl. A. S.: \emph{Tagebuch}, 28. 10. 1896), später aber
                  auch positivere (2. 11. 1896) Eindrücke von den \emph{Freiwild}\pwindex{Schnitzler, Arthur 15.05.1862 – 21.10.1931@\textsc{Schnitzler, Arthur} (15.05.1862 – 21.10.1931), \emph{Schriftsteller, Mediziner}!Freiwild. Schauspiel in 3 Akten1896@\strich\emph{Freiwild. Schauspiel in 3 Akten} {[}1896{]}|pwk}-Proben.}}}\label{K_L02788-3h}. Einen Rath nur in Kürze: Ganz Deutſchland\oindex{Deutschland@\textbf{Deutschland}|pw} ſteht unter dem Banne des
                  Eindruck\textcolor{gray}{e}s, den die \label{K_L02788-4v}\edtext{Affaire \textsc{Bruesewitz\pwindex{Bruesewitz, Henning von 1862 – 1900-01-24@\textsc{Brüsewitz, Henning von} (1862 – 1900-01-24), \emph{Militär}|pw}}}{\lemma{\textnormal{\emph{Affaire Bruesewitz}}}\Cendnote{\textnormal{siehe Paul Goldmann an Arthur Schnitzler, 17. 10. [1896]}}}\label{K_L02788-4h} gemacht hat. Man lechzt nach einem Wort, das dieſe ſchurkiſchen
               Officiers-Feiglinge geißelt. Keiner kann beſſer dieſes Wort aus{\pb}ſprechen, als Du. Leg’ es Deinem anſtändigen
               Officier in den Mund, in der Scene\pwindex{Schnitzler, Arthur 15.05.1862 – 21.10.1931@\textsc{Schnitzler, Arthur} (15.05.1862 – 21.10.1931), \emph{Schriftsteller, Mediziner}!Freiwild. Schauspiel in 3 Akten1896@\strich\emph{Freiwild. Schauspiel in 3 Akten} {[}1896{]}|pwv}, wo er ſagt: \label{K_L02788-5v}\edtext{Solche
               Leute haben im Frieden eigentlich gar keine Exiſtenz-Berechtigung}{\lemma{\textnormal{\emph{Solche … Exiſtenz-Berechtigung}}}\Cendnote{\textnormal{Aussage des Offiziers Rohnstedt\pwindex{Schnitzler, Arthur 15.05.1862 – 21.10.1931@\textsc{Schnitzler, Arthur} (15.05.1862 – 21.10.1931), \emph{Schriftsteller, Mediziner}!Freiwild. Schauspiel in 3 Akten1896@\strich\emph{Freiwild. Schauspiel in 3 Akten} {[}1896{]}|pwkv} am Ende des ersten Akt\pwindex{Schnitzler, Arthur 15.05.1862 – 21.10.1931@\textsc{Schnitzler, Arthur} (15.05.1862 – 21.10.1931), \emph{Schriftsteller, Mediziner}!Freiwild. Schauspiel in 3 Akten1896@\strich\emph{Freiwild. Schauspiel in 3 Akten} {[}1896{]}|pwkv}s}}}\label{K_L02788-5h}. Laß ihn noch
               etwas Allgemeines, Kräftiges, Erlöſendes ſagen. Dieſes Wort allein kann den Erfolg
               des Sück\pwindex{Schnitzler, Arthur 15.05.1862 – 21.10.1931@\textsc{Schnitzler, Arthur} (15.05.1862 – 21.10.1931), \emph{Schriftsteller, Mediziner}!Freiwild. Schauspiel in 3 Akten1896@\strich\emph{Freiwild. Schauspiel in 3 Akten} {[}1896{]}|pwv}es entſcheiden. Nimm’
               meinen \label{K_L02788-6v}\edtext{Rath}{\lemma{\textnormal{\emph{Rath}}}\Cendnote{\textnormal{Über eine Einarbeitung des Vorschlags ist nichts bekannt.
                  Zumindest der Bezug zu der Affäre wurde noch Jahre später hergestellt,
                  beispielsweise: »\begin{otherlanguage}{english}The most celebrated of these was ›Freiwild\pwindex{Schnitzler, Arthur 15.05.1862 – 21.10.1931@\textsc{Schnitzler, Arthur} (15.05.1862 – 21.10.1931), \emph{Schriftsteller, Mediziner}!Freiwild. Schauspiel in 3 Akten1896@\strich\emph{Freiwild. Schauspiel in 3 Akten} {[}1896{]}|pw}‹, an attack on the duel, that received
                        enormous advertizing from the strange coincidence that, while the play\pwindex{Schnitzler, Arthur 15.05.1862 – 21.10.1931@\textsc{Schnitzler, Arthur} (15.05.1862 – 21.10.1931), \emph{Schriftsteller, Mediziner}!Freiwild. Schauspiel in 3 Akten1896@\strich\emph{Freiwild. Schauspiel in 3 Akten} {[}1896{]}|pwv} was in rehearsal,
                        Lieut. von Brüsewitz\pwindex{Bruesewitz, Henning von 1862 – 1900-01-24@\textsc{Brüsewitz, Henning von} (1862 – 1900-01-24), \emph{Militär}|pw}, by the brutal
                        killing of a civilian\pwindex{Siepmann, Theodor †~1896-10-11@\textsc{Siepmann, Theodor} (†~1896-10-11), \emph{Handwerker, Mechaniker}|pwv} in a Carlsruhe\oindex{Karlsruhe@\textbf{Karlsruhe}|pw}{ }restaurant\oindex{Cafe Tannhaeuser@\textbf{Café Tannhäuser}|pwv},
                        vindicated his ›military honor‹ exactly as the play\pwindex{Schnitzler, Arthur 15.05.1862 – 21.10.1931@\textsc{Schnitzler, Arthur} (15.05.1862 – 21.10.1931), \emph{Schriftsteller, Mediziner}!Freiwild. Schauspiel in 3 Akten1896@\strich\emph{Freiwild. Schauspiel in 3 Akten} {[}1896{]}|pwv} had foretold an officer would
                        be obliged to do. The excitement over the Carlsruhe\oindex{Karlsruhe@\textbf{Karlsruhe}|pw} incident rushed the play\pwindex{Schnitzler, Arthur 15.05.1862 – 21.10.1931@\textsc{Schnitzler, Arthur} (15.05.1862 – 21.10.1931), \emph{Schriftsteller, Mediziner}!Freiwild. Schauspiel in 3 Akten1896@\strich\emph{Freiwild. Schauspiel in 3 Akten} {[}1896{]}|pwv} to such a huge popularity that one of the Germ\oindex{Deutschland@\textbf{Deutschland}|pwv}an comic papers
                        showed a cartoon of Manager Brahm\pwindex{Brahm, Otto 05.02.1856 – 28.11.1912@\textsc{Brahm, Otto} (05.02.1856 – 28.11.1912), \emph{Theaterleiter, Regisseur}|pw}, of
                        the Deutsches Theater\orgindex{Deutsches Theater Berlin@Deutsches Theater Berlin|pw}, paying out
                        royalties to the leading playwrights of the season, when Lieut. Brüsewitz\pwindex{Bruesewitz, Henning von 1862 – 1900-01-24@\textsc{Brüsewitz, Henning von} (1862 – 1900-01-24), \emph{Militär}|pw} enters saying: ›I’ve come
                        for my share of the royalties on ›Freiwild\pwindex{Schnitzler, Arthur 15.05.1862 – 21.10.1931@\textsc{Schnitzler, Arthur} (15.05.1862 – 21.10.1931), \emph{Schriftsteller, Mediziner}!Freiwild. Schauspiel in 3 Akten1896@\strich\emph{Freiwild. Schauspiel in 3 Akten} {[}1896{]}|pw}‹!\end{otherlanguage}« ([O. V.]: \emph{Arthur Schnitzler.
                        Dramatist of the Twilight Soul}\pwindex{?? Werk@Nicht ermittelte Verfasserinnen und Verfasser!Arthur Schnitzler. Dramatist of the Twilight Soul.1911-12-01@\emph{Arthur Schnitzler. Dramatist of the Twilight Soul.} {[}1911-12-01{]}|pwk}. In: \emph{Current Literature}\pwindex{?? Werk@Nicht ermittelte Verfasserinnen und Verfasser!Current Literature1888 – 1925@\emph{Current Literature} {[}1888 – 1925{]}|pwk}, Bd. 51, H. 6, Dezember 1911, S. 670–672, hier: S. 671)}}}\label{K_L02788-6h} an, ich
               glaube, ich habe Dir ſelten ſo gut gerathen! {\dotsfour}\pend
           \pstart
           Auf ein Telegramm am Tage nach der \textsc{Première\pwindex{Schnitzler, Arthur 15.05.1862 – 21.10.1931@\textsc{Schnitzler, Arthur} (15.05.1862 – 21.10.1931), \emph{Schriftsteller, Mediziner}!Freiwild. Schauspiel in 3 Akten1896@\strich\emph{Freiwild. Schauspiel in 3 Akten} {[}1896{]}|pwv}}{ }{\pb}rechne ich mit Sicherheit.\pend
           \pstart
           Viele treue Grüße!\pend
           \pstart
           Und ein inniges Glückauf!\pend
           \pstart
           Dein treuer {\\[\baselineskip]}\spacefill\mbox{Paul Goldmann}\pend
           \leftskip=0em{}\pstart
           \noindent{}Schönen Gruß an den \textsc{Dr. Bie\pwindex{Bie, Oskar 09.02.1864 – 21.04.1938@\textsc{Bie, Oskar} (09.02.1864 – 21.04.1938), \emph{Schriftsteller, Journalist, Redakteur}|pw}}, wenn Du \label{K_L02788-7v}\edtext{ihn ſiehſt}{\lemma{\textnormal{\emph{ihn ſiehſt}}}\Cendnote{\textnormal{Schnitzler\pwindex{Schnitzler, Arthur 15.05.1862 – 21.10.1931@\textsc{Schnitzler, Arthur} (15.05.1862 – 21.10.1931), \emph{Schriftsteller, Mediziner}|pwk} traf am 31. 10. 1896, 5. 11. 1896 und
                        7. 11. 1896
                     auf Oskar Bie\pwindex{Bie, Oskar 09.02.1864 – 21.04.1938@\textsc{Bie, Oskar} (09.02.1864 – 21.04.1938), \emph{Schriftsteller, Journalist, Redakteur}|pwk}.}}}\label{K_L02788-7h}\pend
           
         
         \endnumbering\mylabel{h}\end{ledgroupsized}  \newcommand{\dateiname}{L02788}\newcommand{\titel}{Paul Goldmann an Arthur Schnitzler, 27. 10. [1896]}\newcommand{\editorInnen}{Martin Anton Müller und Laura Untner}%% latex-leseansicht-abspann.tex
%% Abspann für die Leseansicht.
%% Der Schalter \ifkorrekturansicht ist bereits durch den Vorspann gesetzt.

%% latex-abspann.tex
%% Gemeinsamer Abspann für Korrekturansicht und Leseansicht.
%% Setzt den Schalter \ifkorrekturansicht voraus (gesetzt in den
%% einbindenden Dateien latex-korrekturansicht-abspann.tex bzw.
%% latex-leseansicht-abspann.tex).
%% ---------------------------------------------------------------

\normalsize

% Das esempio-Environment wird nur in der Leseansicht benötigt
\ifkorrekturansicht\else
\newenvironment{esempio}[3]%
{
    \vspace{1.5ex}
    \rlap{\underline{#1}}
    \par
    \setlength{\parindent}{0cm}
    \nopagebreak
    \leftskip=#2cm
    \rightskip=#3cm
}
{
    \par
}
\fi

\doendnotes{C}
\bigskip
\vfill

\clearpage

\footnotesize

\ifkorrekturansicht
  \lohead{\textsc{register}}
\fi

% theindex-Environment neu definieren ohne reledmac
\makeatletter
\renewenvironment{theindex}{%
  \ifkorrekturansicht
    \section*{\indexname}%
  \else
    \subsubsection*{Index der erwähnten Entitäten}%
  \fi
  \setlength{\parindent}{0pt}%
  \setlength{\parskip}{0pt plus 0.3pt}%
  \let\item\@idxitem
}{%
  \ifkorrekturansicht\clearpage\fi
}
\makeatother

\IfFileExists{\jobname-pw.ind}{\input{\jobname-pw.ind}}{}

% Quellenangabe nur in der Leseansicht
\ifkorrekturansicht\else
% Fallback-Definitionen, falls die .tex-Datei \titel etc. nicht gesetzt hat
\providecommand{\titel}{}
\providecommand{\editorInnen}{}
\providecommand{\dateiname}{\jobname}

\vspace{3cm}

\vfill

\footnotesize
\textsc{Quelle}: \titel. Herausgegeben von {\editorInnen}. In: \emph{Arthur Schnitzler: Briefwechsel mit Autorinnen und Autoren}.
 Digitale Edition, https://schnitzler-briefe.acdh.oeaw.ac.at/{\dateiname}.html (Stand \today)
\fi

\end{document}


      