%% latex-leseansicht-vorspann.tex
%% Vorspann für die Leseansicht.
%% Lädt die gemeinsame Datei latex-vorspann.tex mit nicht gesetztem Schalter.

\newif\ifkorrekturansicht
\korrekturansichtfalse

\input{../tex-inputs/latex-vorspann}


\section[Arthur Schnitzler an Richard Beer-Hofmann, 26. 2. 1901]{L01099 Arthur Schnitzler an Richard Beer-Hofmann, 26. 2. 1901}
\nopagebreak\mylabel{L01099v}
\rehead{ }\normalsize\beginnumbering\briefempfaengerindex{Beer-Hofmann, Richard@\textsc{Beer-Hofmann, Richard}!zzzSchnitzler, Arthur@\emph{von Arthur Schnitzler}!1901-02-261@{26. 2. 1901}|(be}
\toendnotes[C]{\smallbreak\pagebreak[2]}
\correspDesc{Versand  durch Arthur Schnitzler am 26. 2. 1901 in Wien
\newline{}Erhalt  durch Richard Beer-Hofmann am 26. 2. 1901 in Wien}\toendnotes[C]{\smallbreak}
\Standort{YCGL, MSS 31.}
\physDesc{Postkarte, 159 Zeichen
\newline{}Handschrift: Bleistift, deutsche Kurrent
\newline{}Versand: 1) Rohrpost  2) Stempel: »\nobreak{}26 II. 01, 1110V\nobreak{}«.  3) Stempel: »\nobreak{}\oindex{I., Innere Stadt@\textbf{I., Innere Stadt}, \emph{Verwaltungsgebiet}|pwk}Wien 1/1, 26. II. 01, 1130V\nobreak{}«. }\toendnotes[C]{\smallbreak}\pstart{}{\pb}Herrn Dr. \textsc{Rich.
                     Beer-Hofmann}\pend{}\pstart{}Wien\oindex{Wien@\textbf{Wien}, \emph{Verwaltungsgebiet}|pw}\pend{}\pstart{}\textsc{I. Wollzeile 15}\oindex{Wien@\textbf{Wien}!I., Innere Stadt@\textbf{I., Innere Stadt}!Wollzeile 15 (»Berthahof«)@\textbf{Wollzeile 15 (»Berthahof«)}, \emph{Wohngebäude}|pw}\pend{}{\bigskip}\vspace{1em}
\pstart
           \noindent{}{\pb}lieber Richard, bitte{ }ſchicken Sie mir mit den Büchern gleich den
                  \label{K_L01099-1v}\edtext{\textsc{Cicerone}\pwindex{Ebe, Gustav 1.\,11.\,1834 Halberstadt – 15.\,5.\,1916 Berlin@\textsc{Ebe, Gustav} (1.\,11.\,1834 Halberstadt – 15.\,5.\,1916 Berlin), \emph{Architekt}!deutsche Cicerone@\strich\emph{Der deutsche Cicerone}|pwu}}{\lemma{\textnormal{\emph{Cicerone}}}\Cendnote{\textnormal{eventuell einer der vier Bände der von
                     Gustav Ebe\pwindex{Ebe, Gustav 1.\,11.\,1834 Halberstadt – 15.\,5.\,1916 Berlin@\textsc{Ebe, Gustav} (1.\,11.\,1834 Halberstadt – 15.\,5.\,1916 Berlin), \emph{Architekt}|pwk} von
                     1897–1901 unter dem Titel \emph{Der deutsche Cicerone}\pwindex{Ebe, Gustav 1.\,11.\,1834 Halberstadt – 15.\,5.\,1916 Berlin@\textsc{Ebe, Gustav} (1.\,11.\,1834 Halberstadt – 15.\,5.\,1916 Berlin), \emph{Architekt}!deutsche Cicerone@\strich\emph{Der deutsche Cicerone}|pwk} herausgegebenen Reihe von
                  Kunstführern}}}\label{K_L01099-1} mit ja?\pend
           
\pstart
           Herzlichſt Ihr{\\[\baselineskip]}\spacefill\mbox{Arthur.}\pend
           \leftskip=0em{}
\pstart
           \noindent{}\label{K_L01099-2v}\edtext{Mittwoch}{\lemma{\textnormal{\emph{Mittwoch}}}\Cendnote{\textnormal{27. 2. 1901.
                  }}}\label{K_L01099-2}{ }Club\orgindex{Wiener Schachclub@Wiener Schachclub|pwv}\pend
           \selectlanguage{ngerman}\endnumbering\briefempfaengerindex{Beer-Hofmann, Richard@\textsc{Beer-Hofmann, Richard}!zzzSchnitzler, Arthur@\emph{von Arthur Schnitzler}!1901-02-261@{26. 2. 1901}|)be}\mylabel{L01099h}  \newcommand{\dateiname}{L01099}\newcommand{\titel}{Arthur Schnitzler an Richard Beer-Hofmann, 26. 2. 1901}\newcommand{\editorInnen}{Martin Anton Müller und Gerd-Hermann Susen}%% latex-leseansicht-abspann.tex
%% Abspann für die Leseansicht.
%% Der Schalter \ifkorrekturansicht ist bereits durch den Vorspann gesetzt.

%% latex-abspann.tex
%% Gemeinsamer Abspann für Korrekturansicht und Leseansicht.
%% Setzt den Schalter \ifkorrekturansicht voraus (gesetzt in den
%% einbindenden Dateien latex-korrekturansicht-abspann.tex bzw.
%% latex-leseansicht-abspann.tex).
%% ---------------------------------------------------------------

\normalsize

% Das esempio-Environment wird nur in der Leseansicht benötigt
\ifkorrekturansicht\else
\newenvironment{esempio}[3]%
{
    \vspace{1.5ex}
    \rlap{\underline{#1}}
    \par
    \setlength{\parindent}{0cm}
    \nopagebreak
    \leftskip=#2cm
    \rightskip=#3cm
}
{
    \par
}
\fi

\doendnotes{C}
\bigskip
\vfill

\clearpage

\footnotesize

\ifkorrekturansicht
  \lohead{\textsc{register}}
\fi

% theindex-Environment neu definieren ohne reledmac
\makeatletter
\renewenvironment{theindex}{%
  \ifkorrekturansicht
    \section*{\indexname}%
  \else
    \subsubsection*{Index der erwähnten Entitäten}%
  \fi
  \setlength{\parindent}{0pt}%
  \setlength{\parskip}{0pt plus 0.3pt}%
  \let\item\@idxitem
}{%
  \ifkorrekturansicht\clearpage\fi
}
\makeatother

\IfFileExists{\jobname-pw.ind}{\input{\jobname-pw.ind}}{}

% Quellenangabe nur in der Leseansicht
\ifkorrekturansicht\else
% Fallback-Definitionen, falls die .tex-Datei \titel etc. nicht gesetzt hat
\providecommand{\titel}{}
\providecommand{\editorInnen}{}
\providecommand{\dateiname}{\jobname}

\vspace{3cm}

\vfill

\footnotesize
\textsc{Quelle}: \titel. Herausgegeben von {\editorInnen}. In: \emph{Arthur Schnitzler: Briefwechsel mit Autorinnen und Autoren}.
 Digitale Edition, https://schnitzler-briefe.acdh.oeaw.ac.at/{\dateiname}.html (Stand \today)
\fi

\end{document}


