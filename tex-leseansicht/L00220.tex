%% latex-korrekturansicht-vorspann.tex
%% Vorspann für die Korrekturansicht.
%% Lädt die gemeinsame Datei latex-vorspann.tex mit gesetztem Schalter.

\newif\ifkorrekturansicht
\korrekturansichttrue

\input{../tex-inputs/latex-vorspann}


\section[Arthur Schnitzler an Wilhelm Bölsche, 11. 6. 1893]{L00220 Arthur Schnitzler an Wilhelm Bölsche, 11. 6. 1893}
\nopagebreak\mylabel{L00220v}
\rehead{ }\normalsize\beginnumbering\briefempfaengerindex{Boelsche, Wilhelm@\textsc{Bölsche, Wilhelm}!zzzSchnitzler, Arthur@\emph{von Arthur Schnitzler}!1893-06-111@{11. 6. 1893}|(be}
\toendnotes[C]{\smallbreak\pagebreak[2]}\Standort{Wrocław, Biblioteka Uniwersytecka, Böl.Pis 1768.}
\physDesc{Brief, 1 Blatt, 4 Seiten, 1220 Zeichen (Briefpapier mit Trauerrand)
\newline{}Handschrift: schwarze Tinte, deutsche Kurrent
\newline{}Bölsche: als »Erl{[}edigt{]}« gezeichnet }
\buchAbdrucke{\weitereDrucke{1) \emph{Germanica Wratislaviensia} (1987) Nr. 77, S. 462–463.} \weitereDrucke{2) Wilhelm Bölsche: \emph{Briefwechsel. Mit Autoren der Freien Bühne}. Berlin: \emph{Weidler} 2010, S. 686–687.} }\toendnotes[C]{\smallbreak}
\pstart
           
\pstart
           {\pb}\textsc{Wien\oindex{Wien@\textbf{Wien}, \emph{A.ADM2}|pw}}{ }11. 6. 93.\pend
           
\pstart
           \raggedleft{}\textsc{I. Grillparzerstr 7}\oindex{Grillparzerstrasse@\textbf{Grillparzerstraße}, \emph{R.ST}|pw}.\pend
           \pend
           
\pstart{}Sehr geehrter Herr Doktor!\pend\vspace{0.5em}
\pstart
           Vor mehr als 2 Monaten hab ich Ihnen eine Skizze\pwindex{Braut@\emph{Die Braut}|pwv} zur eventuellen Veröffentlichung eingeſandt »\uline{Die Braut}\pwindex{Braut@\emph{Die Braut}|pw}«. – Vor \uline{ca} 2 Wochen hab ich die Frage an Sie
               gerichtet, ob Sie geneigt wären, mein 3 aktiges für die nächſte Saiſon am Leſſingtheater\orgindex{Lessing-Theater@Lessing-Theater|pw} zur Aufführung beſti{\geminationm}tes Schauſpiel »\uline{Das Märchen}\pwindex{Maerchen. Schauspiel in drei Aufzuegen@\emph{Das Märchen. Schauspiel in drei Aufzügen}|pw}« {\pb}in der \textsc{Freien Bühne}\pwindex{Freie Buehne fuer den Entwickelungskampf der Zeit@\emph{Freie Bühne für den Entwickelungskampf der Zeit}|pw} zu veröffentlichen. Warum, erlaube ich mir zu fragen, laſſen Sie mich denn ſo
               lange auf Antwort warten? Meine Skizze\pwindex{Braut@\emph{Die Braut}|pwv} iſt in einer viertel Stunde geleſen, und was nun gar mein Stück\pwindex{Maerchen. Schauspiel in drei Aufzuegen@\emph{Das Märchen. Schauspiel in drei Aufzügen}|pwv} anlangt, ſo bedarf es ja
               vorläufig nur eines principiellen Ja oder Nein. Sie, verehrteſter Herr Doktor, {\pb}der Sie ſelbſt Schriftſteller ſind, Sie wiſſen ja, wie
               nervös das Warten macht; und ich, der ſelbſt Redakteur einer (mediz.) Zeitſchrift\orgindex{Internationale klinische Rundschau@Internationale klinische Rundschau|pw} bin, beantworte jeden Einlauf in
               ſpäteſtens 8 Tagen. Es mag ja Leute geben, deren Briefe man unberückſichtigt zur
               Seite werfen kann; ich gehöre {\pb}nicht zu dieſen, wovon Sie
               verehrteſter Herr Doktor, gewiß ſelbſt überzeugt ſind. –\pend
           
\pstart
           – Ich wiederhole alſo meine beiden Fragen: Nehmen Sie die »Die \uline{Braut}\pwindex{Braut@\emph{Die Braut}|pw}«? – Und zweitens, wollen Sie das Das
                  Märchen\pwindex{Maerchen. Schauspiel in drei Aufzuegen@\emph{Das Märchen. Schauspiel in drei Aufzügen}|pw} im Laufe dieſes So{\geminationm}ers
               drucken? –\pend
           
\pstart
           Ich bin mit ausgezeichneter Hochachtung{\\[\baselineskip]}Ihr ſehr ergebner{\\[\baselineskip]}\spacefill\mbox{Dr. Arthur Schnitzler}\pend
           \leftskip=0em{}\selectlanguage{ngerman}\endnumbering\briefempfaengerindex{Boelsche, Wilhelm@\textsc{Bölsche, Wilhelm}!zzzSchnitzler, Arthur@\emph{von Arthur Schnitzler}!1893-06-111@{11. 6. 1893}|)be}\mylabel{L00220h}  \normalsize

\doendnotes{C}
\bigskip
\vfill

\clearpage

\footnotesize

\lohead{\textsc{register}}

% Definiere theindex-Environment komplett neu ohne reledmac
\makeatletter
\renewenvironment{theindex}{%
  \section*{\indexname}%
  \setlength{\parindent}{0pt}%
  \setlength{\parskip}{0pt plus 0.3pt}%
  \let\item\@idxitem
}{%
  \clearpage
}
\makeatother

\IfFileExists{\jobname-pw.ind}{\input{\jobname-pw.ind}}{}

\end{document}

      