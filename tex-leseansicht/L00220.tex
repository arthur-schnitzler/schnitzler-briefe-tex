%% latex-leseansicht-vorspann.tex
%% Vorspann für die Leseansicht.
%% Lädt die gemeinsame Datei latex-vorspann.tex mit nicht gesetztem Schalter.

\newif\ifkorrekturansicht
\korrekturansichtfalse

\input{../tex-inputs/latex-vorspann}


\section[Arthur Schnitzler an Wilhelm Bölsche, 11. 6. 1893]{L00220 Arthur Schnitzler an Wilhelm Bölsche, 11. 6. 1893}
\nopagebreak\mylabel{L00220v}
\rehead{ }\normalsize\beginnumbering\briefempfaengerindex{Bölsche, Wilhelm@\textsc{Bölsche, Wilhelm}!zzzSchnitzler, Arthur@\emph{von Arthur Schnitzler}!1893-06-111@{11. 6. 1893}|(be}
\toendnotes[C]{\smallbreak\pagebreak[2]}
\correspDesc{Versand  durch Arthur Schnitzler am 11. 6. 1893 in Wien
\newline{}Erhalt  durch Wilhelm Bölsche im Zeitraum [12. 6. 1893
                  – 16. 6. 1893?] in Berlin}\toendnotes[C]{\smallbreak}
\Standort{Wrocław, Biblioteka Uniwersytecka, Böl.Pis 1768.}
\physDesc{Brief, 1 Blatt, 4 Seiten, 1220 Zeichen (Briefpapier mit Trauerrand)
\newline{}Handschrift: schwarze Tinte, deutsche Kurrent
\newline{}Bölsche: als »Erl{[}edigt{]}« gezeichnet }
\buchAbdrucke{\weitereDrucke{1) Alois Woldan: \emph{Arthur Schnitzler – Briefe an Wilhelm Bölsche.} In: \emph{Germanica Wratislaviensia} (1987) Nr. 77, S. 462–463.} \weitereDrucke{2) Wilhelm Bölsche: \emph{Briefwechsel. Mit Autoren der Freien Bühne}. Herausgegeben von Gerd-Hermann Susen. Berlin: \emph{Weidler} 2010, S. 686–687 (Werke und Briefe. Wissenschaftliche Ausgabe, Briefe I).} }\toendnotes[C]{\smallbreak}
\pstart
           
\pstart
           {\pb}\textsc{Wien\oindex{Wien@\textbf{Wien}, \emph{Verwaltungsgebiet}|pw}}{ }11. 6. 93.\pend
           
\pstart
           \raggedleft{}\textsc{I. Grillparzerstr 7}\oindex{Wien@\textbf{Wien}!I., Innere Stadt@\textbf{I., Innere Stadt}!Grillparzerstraße@\textbf{Grillparzerstraße}, \emph{Straße}|pw}.\pend
           \pend
           
\pstart{}Sehr geehrter Herr Doktor!\pend\vspace{0.5em}
\pstart
           Vor mehr als 2 Monaten hab ich Ihnen eine Skizze\pwindex{Schnitzler, Arthur 15.\,5.\,1862 Wien – 21.\,10.\,1931 ebd.@\textsc{Schnitzler, Arthur} (15.\,5.\,1862 Wien – 21.\,10.\,1931 ebd.), \emph{Schriftsteller, Mediziner}!Braut@\strich\emph{Die Braut}|pwv} zur eventuellen Veröffentlichung eingeſandt »\uline{Die Braut}\pwindex{Schnitzler, Arthur 15.\,5.\,1862 Wien – 21.\,10.\,1931 ebd.@\textsc{Schnitzler, Arthur} (15.\,5.\,1862 Wien – 21.\,10.\,1931 ebd.), \emph{Schriftsteller, Mediziner}!Braut@\strich\emph{Die Braut}|pw}«. – Vor \uline{ca} 2 Wochen hab ich die Frage an Sie
               gerichtet, ob Sie geneigt wären, mein 3 aktiges für die nächſte Saiſon am Leſſingtheater\orgindex{Lessing-Theater@Lessing-Theater|pw} zur Aufführung beſti{\geminationm}tes Schauſpiel »\uline{Das Märchen}\pwindex{Schnitzler, Arthur 15.\,5.\,1862 Wien – 21.\,10.\,1931 ebd.@\textsc{Schnitzler, Arthur} (15.\,5.\,1862 Wien – 21.\,10.\,1931 ebd.), \emph{Schriftsteller, Mediziner}!Märchen. Schauspiel in drei Aufzügen@\strich\emph{Das Märchen. Schauspiel in drei Aufzügen}|pw}« {\pb}in der \textsc{Freien Bühne}\pwindex{Freie Bühne für den Entwickelungskampf der Zeit@\emph{Freie Bühne für den Entwickelungskampf der Zeit}|pw} zu veröffentlichen. Warum, erlaube ich mir zu fragen, laſſen Sie mich denn{ }ſo
               lange auf Antwort warten? Meine Skizze\pwindex{Schnitzler, Arthur 15.\,5.\,1862 Wien – 21.\,10.\,1931 ebd.@\textsc{Schnitzler, Arthur} (15.\,5.\,1862 Wien – 21.\,10.\,1931 ebd.), \emph{Schriftsteller, Mediziner}!Braut@\strich\emph{Die Braut}|pwv} iſt in einer viertel Stunde geleſen, und was nun gar mein Stück\pwindex{Schnitzler, Arthur 15.\,5.\,1862 Wien – 21.\,10.\,1931 ebd.@\textsc{Schnitzler, Arthur} (15.\,5.\,1862 Wien – 21.\,10.\,1931 ebd.), \emph{Schriftsteller, Mediziner}!Märchen. Schauspiel in drei Aufzügen@\strich\emph{Das Märchen. Schauspiel in drei Aufzügen}|pwv} anlangt,{ }ſo bedarf es ja
               vorläufig nur eines principiellen Ja oder Nein. Sie, verehrteſter Herr Doktor, {\pb}der Sie{ }ſelbſt Schriftſteller{ }ſind, Sie wiſſen ja, wie
               nervös das Warten macht; und ich, der{ }ſelbſt Redakteur einer (mediz.) Zeitſchrift\orgindex{Internationale klinische Rundschau@Internationale klinische Rundschau|pw} bin, beantworte jeden Einlauf in{ }ſpäteſtens 8 Tagen. Es mag ja Leute geben, deren Briefe man unberückſichtigt zur
               Seite werfen kann; ich gehöre {\pb}nicht zu dieſen, wovon Sie
               verehrteſter Herr Doktor, gewiß{ }ſelbſt überzeugt{ }ſind. –\pend
           
\pstart
           – Ich wiederhole alſo meine beiden Fragen: Nehmen Sie die »Die \uline{Braut}\pwindex{Schnitzler, Arthur 15.\,5.\,1862 Wien – 21.\,10.\,1931 ebd.@\textsc{Schnitzler, Arthur} (15.\,5.\,1862 Wien – 21.\,10.\,1931 ebd.), \emph{Schriftsteller, Mediziner}!Braut@\strich\emph{Die Braut}|pw}«? – Und zweitens, wollen Sie das Das
                  Märchen\pwindex{Schnitzler, Arthur 15.\,5.\,1862 Wien – 21.\,10.\,1931 ebd.@\textsc{Schnitzler, Arthur} (15.\,5.\,1862 Wien – 21.\,10.\,1931 ebd.), \emph{Schriftsteller, Mediziner}!Märchen. Schauspiel in drei Aufzügen@\strich\emph{Das Märchen. Schauspiel in drei Aufzügen}|pw} im Laufe dieſes So{\geminationm}ers
               drucken? –\pend
           
\pstart
           Ich bin mit ausgezeichneter Hochachtung{\\[\baselineskip]}Ihr{ }ſehr ergebner{\\[\baselineskip]}\spacefill\mbox{Dr. Arthur Schnitzler}\pend
           \leftskip=0em{}\selectlanguage{ngerman}\endnumbering\briefempfaengerindex{Bölsche, Wilhelm@\textsc{Bölsche, Wilhelm}!zzzSchnitzler, Arthur@\emph{von Arthur Schnitzler}!1893-06-111@{11. 6. 1893}|)be}\mylabel{L00220h}  \newcommand{\dateiname}{L00220}\newcommand{\titel}{Arthur Schnitzler an Wilhelm Bölsche, 11. 6. 1893}\newcommand{\editorInnen}{Martin Anton Müller und Gerd-Hermann Susen}%% latex-leseansicht-abspann.tex
%% Abspann für die Leseansicht.
%% Der Schalter \ifkorrekturansicht ist bereits durch den Vorspann gesetzt.

%% latex-abspann.tex
%% Gemeinsamer Abspann für Korrekturansicht und Leseansicht.
%% Setzt den Schalter \ifkorrekturansicht voraus (gesetzt in den
%% einbindenden Dateien latex-korrekturansicht-abspann.tex bzw.
%% latex-leseansicht-abspann.tex).
%% ---------------------------------------------------------------

\normalsize

% Das esempio-Environment wird nur in der Leseansicht benötigt
\ifkorrekturansicht\else
\newenvironment{esempio}[3]%
{
    \vspace{1.5ex}
    \rlap{\underline{#1}}
    \par
    \setlength{\parindent}{0cm}
    \nopagebreak
    \leftskip=#2cm
    \rightskip=#3cm
}
{
    \par
}
\fi

\doendnotes{C}
\bigskip
\vfill

\clearpage

\footnotesize

\ifkorrekturansicht
  \lohead{\textsc{register}}
\fi

% theindex-Environment neu definieren ohne reledmac
\makeatletter
\renewenvironment{theindex}{%
  \ifkorrekturansicht
    \section*{\indexname}%
  \else
    \subsubsection*{Index der erwähnten Entitäten}%
  \fi
  \setlength{\parindent}{0pt}%
  \setlength{\parskip}{0pt plus 0.3pt}%
  \let\item\@idxitem
}{%
  \ifkorrekturansicht\clearpage\fi
}
\makeatother

\IfFileExists{\jobname-pw.ind}{\input{\jobname-pw.ind}}{}

% Quellenangabe nur in der Leseansicht
\ifkorrekturansicht\else
% Fallback-Definitionen, falls die .tex-Datei \titel etc. nicht gesetzt hat
\providecommand{\titel}{}
\providecommand{\editorInnen}{}
\providecommand{\dateiname}{\jobname}

\vspace{3cm}

\vfill

\footnotesize
\textsc{Quelle}: \titel. Herausgegeben von {\editorInnen}. In: \emph{Arthur Schnitzler: Briefwechsel mit Autorinnen und Autoren}.
 Digitale Edition, https://schnitzler-briefe.acdh.oeaw.ac.at/{\dateiname}.html (Stand \today)
\fi

\end{document}


