%% latex-korrekturansicht-vorspann.tex
%% Vorspann für die Korrekturansicht.
%% Lädt die gemeinsame Datei latex-vorspann.tex mit gesetztem Schalter.

\newif\ifkorrekturansicht
\korrekturansichttrue

\input{../tex-inputs/latex-vorspann}


\section[Arthur Schnitzler an Gertrud Rung, 9. 3. 1925]{L02438 Arthur Schnitzler an Gertrud Rung, 9. 3. 1925}
\nopagebreak\mylabel{L02438v}
\rehead{ }\normalsize\beginnumbering\briefempfaengerindex{Rung, Gertrud@\textsc{Rung, Gertrud}!zzzSchnitzler, Arthur@\emph{von Arthur Schnitzler}!1925-03-091@{9. 3. 1925}|(be}
\toendnotes[C]{\smallbreak\pagebreak[2]}\Standort{Kopenhagen, Det Kongelige Bibliotek, Georg Brandes Arkiv, box 125.}
\physDesc{Postkarte, 549 Zeichen
\newline{}Handschrift: schwarze Tinte, lateinische Kurrent
\newline{}Versand: Stempel: »\nobreak{}\oindex{XVIII., Waehring@\textbf{XVIII., Währing}, \emph{A.ADM3}|pwk}18/1 Wien 110, 11. III. 25, 9\nobreak{}«.  
\newline{}Ordnung: 1) mit Bleistift von unbekannter Hand nummeriert: »\strikeout{48a}«  2) mit Bleistift von unbekannter Hand nummeriert:
                                    »51.a«}
\buchAbdrucke{\weitereDrucke{Georg Brandes, Arthur Schnitzler: \emph{Ein Briefwechsel}. Bern: \emph{Francke} 1956, S. 145.} }\toendnotes[C]{\smallbreak}\pstart{}{\pb}\label{T_L02438-1v}\edtext{\textcolor{gray}{\textbf{A. S.}}}{\lemma{\textnormal{\emph{A. S.}}}\Cendnote{\textnormal{ovaler Absenderkleber}}}\label{T_L02438-1}\pend{}\pstart{}\textcolor{gray}{\textbf{WIEN, XVIII.}}\oindex{XVIII., Waehring@\textbf{XVIII., Währing}, \emph{A.ADM3}|pw}\pend{}\pstart{}\textcolor{gray}{\textbf{STERNWARTESTR. 71}}\oindex{Sternwartestrasse 71@\textbf{Sternwartestraße 71}, \emph{Wohngebäude (K.WHS)}|pw}\pend{}{\bigskip}\pstart{}An Frau Rung\pend{}\pstart{}per Adr. Georg Brandes\pend{}\pstart{}Kopenhagen\oindex{Kopenhagen@\textbf{Kopenhagen}, \emph{P.PPLC}|pw}. \pend{}{\bigskip}\vspace{1em}
\pstart
           \raggedleft{}{\pb}Wien\oindex{Wien@\textbf{Wien}, \emph{A.ADM2}|pw}, 9. 3. 25\pend
           
\pstart{}Verehrte Frau Rung,\pend\vspace{0.5em}
\pstart
           schönen Dank für Ihre freundl Nachricht; – da ich schon früher nach Berlin\oindex{Berlin@\textbf{Berlin}, \emph{P.PPLC}|pw} fahren muſs, ist es unsicher ob ich Professor Brandes\pwindex{Brandes, Georg 04.02.1842 – 19.02.1927@\textsc{Brandes, Georg} (04.02.1842 – 19.02.1927)|pw} Ankunft werde abwarten können. Doch
               lese ich in der Zeitung, dſs G. B.\pwindex{Brandes, Georg 04.02.1842 – 19.02.1927@\textsc{Brandes, Georg} (04.02.1842 – 19.02.1927)|pw} auch nach
                  \uline{Wien}\oindex{Wien@\textbf{Wien}, \emph{A.ADM2}|pw} reisen wird – \uline{bewahrheitet sich das}? Wie froh
               wäre ich. Ich bitte um Nachricht nach Berlin\oindex{Berlin@\textbf{Berlin}, \emph{P.PPLC}|pw}, an
               die Adresse meines Sohnes Heinrich Schnitzler\pwindex{Schnitzler, Heinrich 09.08.1902 – 12.07.1982@\textsc{Schnitzler, Heinrich} (09.08.1902 – 12.07.1982), \emph{Regisseur/Regisseurin, Schauspieler/Schauspielerin}|pw}{ }Matthäikirchstraße 4\oindex{Herbert-von-Karajan-Strasse@\textbf{Herbert-von-Karajan-Straße}, \emph{Straße (K.STR)}|pw}, bei Dernburg\pwindex{Dernburg, Ilse 13.05.1880 – 1964/1965@\textsc{Dernburg, Ilse} (13.05.1880 – 1964/1965), \emph{Innenarchitekt/Innenarchitektin}|pw}. Meine herzlichsten Grüße an Georg {\pb}Brandes\pwindex{Brandes, Georg 04.02.1842 – 19.02.1927@\textsc{Brandes, Georg} (04.02.1842 – 19.02.1927)|pw},\pend
           
\pstart
           mit vielen Empfehlungen{\\[\baselineskip]}Ihr ergebner \spacefill\mbox{Arthur Schnitzler}\pend
           \leftskip=0em{}\selectlanguage{ngerman}\endnumbering\briefempfaengerindex{Rung, Gertrud@\textsc{Rung, Gertrud}!zzzSchnitzler, Arthur@\emph{von Arthur Schnitzler}!1925-03-091@{9. 3. 1925}|)be}\mylabel{L02438h}  \normalsize

\doendnotes{C}
\bigskip
\vfill

\clearpage

\footnotesize

\lohead{\textsc{register}}

% Definiere theindex-Environment komplett neu ohne reledmac
\makeatletter
\renewenvironment{theindex}{%
  \section*{\indexname}%
  \setlength{\parindent}{0pt}%
  \setlength{\parskip}{0pt plus 0.3pt}%
  \let\item\@idxitem
}{%
  \clearpage
}
\makeatother

\IfFileExists{\jobname-pw.ind}{\input{\jobname-pw.ind}}{}

\end{document}

      