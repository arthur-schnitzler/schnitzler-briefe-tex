%% latex-leseansicht-vorspann.tex
%% Vorspann für die Leseansicht.
%% Lädt die gemeinsame Datei latex-vorspann.tex mit nicht gesetztem Schalter.

\newif\ifkorrekturansicht
\korrekturansichtfalse

\input{../tex-inputs/latex-vorspann}


         
         \renewcommand{\erwaehntePersonen}{Personen: Hermann Bahr, Richard Beer-Hofmann, Otto Brahm, Max Eugen Burckhard, Gerhart Hauptmann, Hugo von Hofmannsthal, Heinrich Kanner, Hermine von Schaffgotsch, Isidor Singer}
         \renewcommand{\erwaehnteInstitutionen}{Institutionen: Die Zeit. Wiener Wochenschrift}
         \renewcommand{\erwaehnteOrte}{Orte: Kärnten, Marienbad, Thüringen, Velden am Wörthersee, Wien}
         \renewcommand{\erwaehnteWerke}{}
               \section[Arthur Schnitzler an Hugo von Hofmannsthal, 6. 7. 1899]{ Arthur Schnitzler an Hugo von Hofmannsthal, 6. 7. 1899}\nopagebreak\mylabel{v}\rehead{ }\begin{ledgroupsized}[t]{13cm}\normalsize\beginnumbering \toendnotes[C]{\smallbreak\pagebreak[2]} \Standort{FDH, Hs-30885,82.}
\physDesc{Brief, 1 Blatt, 4 Seiten
\newline{}Handschrift: schwarze Tinte, deutsche Kurrent\newline{}Ordnung: mit Bleistift von Schnitzler mutmaßlich während der
                                            Durchsicht der Briefe 1929 am oberen
                                            Blattrand zusätzlich datiert: »6/7 99« }\buchAbdrucke{\weitereDrucke{1) Hugo von Hofmannsthal, Arthur Schnitzler: \emph{Briefwechsel}. Hg. Therese Nickl und Heinrich Schnitzler. Frankfurt am Main: \emph{S. Fischer} 1964, S. 123.} \weitereDrucke{2) Hermann Bahr, Arthur Schnitzler: \emph{Briefwechsel, Aufzeichnungen, Dokumente
                                (1891–1931)}. Hg. Kurt Ifkovits und Martin Anton Müller. Göttingen: \emph{Wallstein} 2018, S. 170.} }\toendnotes[C]{\smallbreak}\pstart{}{\pb}lieber Hugo,\pend\pstart
           folgendes iſt mit \uuline{\edtext{vollkommener Discretion}{\Cendnote{dreifach unterstrichen}}} zu
                    behandeln: \uline{Bahr}\pwindex{Bahr, Hermann 19.07.1863 – 15.01.1934@\textsc{Bahr, Hermann} (19.07.1863 – 15.01.1934), \emph{Schriftsteller, Kritiker}|pw}\uline{ verläßt die }\uline{Zeit}\orgindex{Zeit. Wiener Wochenschrift@Die Zeit. Wiener Wochenschrift|pw}. Singer\pwindex{Singer, Isidor 16.01.1857 – 08.12.1927@\textsc{Singer, Isidor} (16.01.1857 – 08.12.1927), \emph{Journalist, Herausgeber, Soziologe}|pw} und Kanner\pwindex{Kanner, Heinrich 09.11.1864 – 15.02.1930@\textsc{Kanner, Heinrich} (09.11.1864 – 15.02.1930), \emph{Herausgeber, Publizist}|pw} waren bei mir. Lange Unterredung ohne Intereſſe
                    für Sie (nur mich.) Das weſentliche: ſie möchten auf das Blatt ſtellen: unter
                    Mitwirkung von – \textsc{etc etc} nur erſte Namen, ich möchte
                    Sie fragen, ob Sie im Princip damit {\pb}einverſtanden
                    wären, auch als »Mitwirkender{[}«{]} oder »ſtändg Mitwirkender«
                    aufs Blatt zu ko{\geminationm}en, neben \textsc{Burckhard\pwindex{Burckhard, Max Eugen 14.07.1854 – 16.03.1912@\textsc{Burckhard, Max Eugen} (14.07.1854 – 16.03.1912), \emph{Schriftsteller, Rechtswissenschaftler, Theaterleiter}|pw}}, mich, – event. \textsc{Hauptmann}\pwindex{Hauptmann, Gerhart 15.11.1862 – 06.06.1946@\textsc{Hauptmann, Gerhart} (15.11.1862 – 06.06.1946), \emph{Schriftsteller}|pw} (\label{K_L00934_1v}\edtext{an den ich mich über Brahm\pwindex{Brahm, Otto 05.02.1856 – 28.11.1912@\textsc{Brahm, Otto} (05.02.1856 – 28.11.1912), \emph{Theaterleiter, Regisseur}|pw} wende}{\lemma{\textnormal{\emph{an … wende}}}\Cendnote{\textnormal{siehe Arthur Schnitzler an Gerhart Hauptmann,
                    15. 7. 1899}}}\label{K_L00934_1h}.) Sie können natürlich ohne weiters zuſagen. Für die Herausgeber ſcheint
                    mir die Sache allerdings überflüſſig: ſie brauchten Arbeitskräfte, nicht
                    Namen. –\pend
           \pstart
           Ich bin noch hier; und will über meine {\pb}Sti{\geminationm}ung nichts ſagen, da nichts neues u nicht
                    erfreuliches vorliegt. Gerade dſs ſich das Leben da und dort wieder zu melden
                    anfängt, iſt das traurige; es iſt ein Leben dritter Ordnung, das beſte iſt
                    vorbei.\pend
           \pstart
           Das Wetter ist ſchändlich. Mitte Juli reiſe ich nach Kärnthen\oindex{Kaernten@\textbf{Kärnten}|pw}; zuerſt \textsc{Velden}\oindex{Velden am Woerthersee@\textbf{Velden am Wörthersee}|pw}, dann zu Richard\pwindex{Beer-Hofmann, Richard 1866-07-11 – 1945-09-26@\textsc{Beer-Hofmann, Richard} (1866-07-11 – 1945-09-26), \emph{Schriftsteller}|pw}, von dem ich eine
                    kurze Karte habe. – Hat ſich in den Chancen für Mitte Auguſt (Thü{\pb}ringen\oindex{Thueringen@\textbf{Thüringen}|pw}{ }\textsc{etc}) was geändert? – Arbeiten Sie? Sehn Sie Minnie\pwindex{Schaffgotsch, Hermine von 25.11.1871 – 25.11.1928@\textsc{Schaffgotsch, Hermine von} (25.11.1871 – 25.11.1928)|pw}? –\pend
           \pstart Leben Sie wohl. Von Herzen Ihr \spacefill\mbox{Arthur Sch}\pend{}\pstart
           Wien\oindex{Wien@\textbf{Wien}|pw}{ }6. 7. 99.\pend
           
         
         \endnumbering\mylabel{h}\end{ledgroupsized}  \newcommand{\dateiname}{L00934}\newcommand{\titel}{Arthur Schnitzler an Hugo von Hofmannsthal, 6. 7. 1899}\newcommand{\editorInnen}{ Martin Anton Müller und Gerd-Hermann Susen}%% latex-leseansicht-abspann.tex
%% Abspann für die Leseansicht.
%% Der Schalter \ifkorrekturansicht ist bereits durch den Vorspann gesetzt.

%% latex-abspann.tex
%% Gemeinsamer Abspann für Korrekturansicht und Leseansicht.
%% Setzt den Schalter \ifkorrekturansicht voraus (gesetzt in den
%% einbindenden Dateien latex-korrekturansicht-abspann.tex bzw.
%% latex-leseansicht-abspann.tex).
%% ---------------------------------------------------------------

\normalsize

% Das esempio-Environment wird nur in der Leseansicht benötigt
\ifkorrekturansicht\else
\newenvironment{esempio}[3]%
{
    \vspace{1.5ex}
    \rlap{\underline{#1}}
    \par
    \setlength{\parindent}{0cm}
    \nopagebreak
    \leftskip=#2cm
    \rightskip=#3cm
}
{
    \par
}
\fi

\doendnotes{C}
\bigskip
\vfill

\clearpage

\footnotesize

\ifkorrekturansicht
  \lohead{\textsc{register}}
\fi

% theindex-Environment neu definieren ohne reledmac
\makeatletter
\renewenvironment{theindex}{%
  \ifkorrekturansicht
    \section*{\indexname}%
  \else
    \subsubsection*{Index der erwähnten Entitäten}%
  \fi
  \setlength{\parindent}{0pt}%
  \setlength{\parskip}{0pt plus 0.3pt}%
  \let\item\@idxitem
}{%
  \ifkorrekturansicht\clearpage\fi
}
\makeatother

\IfFileExists{\jobname-pw.ind}{\input{\jobname-pw.ind}}{}

% Quellenangabe nur in der Leseansicht
\ifkorrekturansicht\else
% Fallback-Definitionen, falls die .tex-Datei \titel etc. nicht gesetzt hat
\providecommand{\titel}{}
\providecommand{\editorInnen}{}
\providecommand{\dateiname}{\jobname}

\vspace{3cm}

\vfill

\footnotesize
\textsc{Quelle}: \titel. Herausgegeben von {\editorInnen}. In: \emph{Arthur Schnitzler: Briefwechsel mit Autorinnen und Autoren}.
 Digitale Edition, https://schnitzler-briefe.acdh.oeaw.ac.at/{\dateiname}.html (Stand \today)
\fi

\end{document}


      