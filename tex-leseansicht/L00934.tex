%% latex-korrekturansicht-vorspann.tex
%% Vorspann für die Korrekturansicht.
%% Lädt die gemeinsame Datei latex-vorspann.tex mit gesetztem Schalter.

\newif\ifkorrekturansicht
\korrekturansichttrue

\input{../tex-inputs/latex-vorspann}


\section[Arthur Schnitzler an Hugo von Hofmannsthal, 6. 7. 1899]{L00934 Arthur Schnitzler an Hugo von Hofmannsthal, 6. 7. 1899}
\nopagebreak\mylabel{L00934v}
\rehead{ }\normalsize\beginnumbering\briefempfaengerindex{Hofmannsthal, Hugo von@\textsc{Hofmannsthal, Hugo von}!zzzSchnitzler, Arthur@\emph{von Arthur Schnitzler}!1899-07-062@{6. 7. 1899}|(be}
\toendnotes[C]{\smallbreak\pagebreak[2]}\Standort{FDH, Hs-30885,82.}
\physDesc{Brief, 1 Blatt, 4 Seiten, 1157 Zeichen
\newline{}Handschrift: schwarze Tinte, deutsche Kurrent
\newline{}Ordnung: mit Bleistift von Schnitzler mutmaßlich während der Durchsicht
                                 der Briefe 1929 am oberen Blattrand zusätzlich
                                 datiert: »6/7 99« }
\buchAbdrucke{\weitereDrucke{1) Hugo von Hofmannsthal, Arthur Schnitzler: \emph{Briefwechsel}. Frankfurt am Main: \emph{S. Fischer} 1964, S. 123.} \weitereDrucke{2) Hermann Bahr, Arthur Schnitzler: \emph{Briefwechsel, Aufzeichnungen, Dokumente (1891–1931)}. Göttingen: \emph{Wallstein} 2018, S. 170.} }\toendnotes[C]{\smallbreak}
\pstart{}{\pb}lieber Hugo,\pend\vspace{0.5em}
\pstart
           folgendes iſt mit \uuline{\edtext{vollkommener Discretion}{\Cendnote{dreifach unterstrichen}}} zu
               behandeln: \uline{Bahr}\pwindex{Bahr, Hermann 19.07.1863 – 15.01.1934@\textsc{Bahr, Hermann} (19.07.1863 – 15.01.1934), \emph{Schriftsteller/Schriftstellerin, Kritiker/Kritikerin}|pw}\uline{{ }verläßt die }\uline{Zeit}\orgindex{Zeit. Wiener Wochenschrift@Die Zeit. Wiener Wochenschrift|pw}. Singer\pwindex{Singer, Isidor 16.01.1857 – 08.12.1927@\textsc{Singer, Isidor} (16.01.1857 – 08.12.1927), \emph{Journalist/Journalistin, Herausgeber/Herausgeberin, Soziologe/Soziologin}|pw} und Kanner\pwindex{Kanner, Heinrich 09.11.1864 – 15.02.1930@\textsc{Kanner, Heinrich} (09.11.1864 – 15.02.1930), \emph{Herausgeber/Herausgeberin, Publizist/Publizistin}|pw} waren bei mir. Lange Unterredung ohne Intereſſe für Sie
               (nur mich.) Das weſentliche: ſie möchten auf das Blatt ſtellen: unter Mitwirkung von
               – \textsc{etc etc} nur erſte Namen, ich möchte Sie fragen, ob Sie im
               Princip damit {\pb}einverſtanden wären, auch als
                  »Mitwirkender{[}«{]} oder »ſtändg Mitwirkender« aufs Blatt zu ko{\geminationm}en, neben \textsc{Burckhard\pwindex{Burckhard, Max Eugen 14.07.1854 – 16.03.1912@\textsc{Burckhard, Max Eugen} (14.07.1854 – 16.03.1912), \emph{Schriftsteller/Schriftstellerin, Rechtswissenschaftler/Rechtswissenschaftlerin, Theaterleiter/Theaterleiterin}|pw}}, mich, – event. \textsc{Hauptmann}\pwindex{Hauptmann, Gerhart 15.11.1862 – 06.06.1946@\textsc{Hauptmann, Gerhart} (15.11.1862 – 06.06.1946), \emph{Schriftsteller/Schriftstellerin}|pw} (\label{K_L00934-1v}\edtext{an den ich mich über Brahm\pwindex{Brahm, Otto 05.02.1856 – 28.11.1912@\textsc{Brahm, Otto} (05.02.1856 – 28.11.1912), \emph{Theaterleiter/Theaterleiterin, Regisseur/Regisseurin}|pw} wende}{\lemma{\textnormal{\emph{an … wende}}}\Cendnote{\textnormal{Siehe Arthur Schnitzler an Gerhart Hauptmann, 15. 7. 1899.
               }}}\label{K_L00934-1}.) Sie können natürlich ohne weiters zuſagen. Für die Herausgeber ſcheint mir
               die Sache allerdings überflüſſig: ſie brauchten Arbeitskräfte, nicht Namen. –\pend
           
\pstart
           Ich bin noch hier; und will über meine {\pb}Sti{\geminationm}ung nichts ſagen, da nichts neues u nicht erfreuliches
               vorliegt. Gerade dſs ſich das Leben da und dort wieder zu melden anfängt, iſt das
               traurige; es iſt ein Leben dritter Ordnung, das beſte iſt vorbei.\pend
           
\pstart
           Das Wetter ist ſchändlich. Mitte Juli reiſe ich nach Kärnthen\oindex{Kaernten@\textbf{Kärnten}, \emph{A.ADM1}|pw}; zuerſt \textsc{Velden}\oindex{Velden am Woerthersee@\textbf{Velden am Wörthersee}, \emph{P.PPL}|pw}, dann zu Richard\pwindex{Beer-Hofmann, Richard 1866-07-11 – 1945-09-26@\textsc{Beer-Hofmann, Richard} (1866-07-11 – 1945-09-26), \emph{Schriftsteller/Schriftstellerin}|pw}, von dem ich eine kurze
               \label{K_L00934-2v}\edtext{Karte}{\lemma{\textnormal{\emph{Karte}}}\Cendnote{\textnormal{Siehe Arthur Schnitzler an Richard Beer-Hofmann, 6. 7. 1899.
               }}}\label{K_L00934-2} habe. – Hat ſich in den Chancen für Mitte Auguſt (Thü{\pb}ringen\oindex{Thueringen@\textbf{Thüringen}, \emph{A.ADM1}|pw}{ }\textsc{etc}) was geändert? – Arbeiten Sie? Sehn Sie Minnie\pwindex{Schaffgotsch, Hermine von 25.11.1871 – 25.11.1928@\textsc{Schaffgotsch, Hermine von} (25.11.1871 – 25.11.1928)|pw}? –\pend
           \pstart Leben Sie wohl. Von Herzen Ihr \spacefill\mbox{Arthur Sch}\pend{}
\pstart
           Wien\oindex{Wien@\textbf{Wien}, \emph{A.ADM2}|pw}{ }6. 7. 99.\pend
           \selectlanguage{ngerman}\endnumbering\briefempfaengerindex{Hofmannsthal, Hugo von@\textsc{Hofmannsthal, Hugo von}!zzzSchnitzler, Arthur@\emph{von Arthur Schnitzler}!1899-07-062@{6. 7. 1899}|)be}\mylabel{L00934h}  \normalsize

\doendnotes{C}
\bigskip
\vfill

\clearpage

\footnotesize

\lohead{\textsc{register}}

% Definiere theindex-Environment komplett neu ohne reledmac
\makeatletter
\renewenvironment{theindex}{%
  \section*{\indexname}%
  \setlength{\parindent}{0pt}%
  \setlength{\parskip}{0pt plus 0.3pt}%
  \let\item\@idxitem
}{%
  \clearpage
}
\makeatother

\IfFileExists{\jobname-pw.ind}{\input{\jobname-pw.ind}}{}

\end{document}

      