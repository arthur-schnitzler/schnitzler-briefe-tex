%% latex-leseansicht-vorspann.tex
%% Vorspann für die Leseansicht.
%% Lädt die gemeinsame Datei latex-vorspann.tex mit nicht gesetztem Schalter.

\newif\ifkorrekturansicht
\korrekturansichtfalse

\input{../tex-inputs/latex-vorspann}


\section[Arthur Schnitzler an Hugo von Hofmannsthal, 6. 7. 1899]{L00934 Arthur Schnitzler an Hugo von Hofmannsthal, 6. 7. 1899}
\nopagebreak\mylabel{L00934v}
\rehead{ }\normalsize\beginnumbering\briefempfaengerindex{Hofmannsthal, Hugo von@\textsc{Hofmannsthal, Hugo von}!zzzSchnitzler, Arthur@\emph{von Arthur Schnitzler}!1899-07-062@{6. 7. 1899}|(be}
\toendnotes[C]{\smallbreak\pagebreak[2]}
\correspDesc{Versand  durch Arthur Schnitzler am 6. 7. 1899 in Wien
\newline{}Erhalt  durch Hugo von Hofmannsthal im Zeitraum [7. 7. 1899
                  – 11. 7. 1899?] in Marienbad}\toendnotes[C]{\smallbreak}
\Standort{FDH, Hs-30885,82.}
\physDesc{Brief, 1 Blatt, 4 Seiten, 1157 Zeichen
\newline{}Handschrift: schwarze Tinte, deutsche Kurrent
\newline{}Ordnung: mit Bleistift von Schnitzler mutmaßlich während der Durchsicht
                                 der Briefe 1929 am oberen Blattrand zusätzlich
                                 datiert: »6/7 99« }
\buchAbdrucke{\weitereDrucke{1) Hugo von Hofmannsthal, Arthur Schnitzler: \emph{Briefwechsel}. Herausgegeben von Therese Nickl und Heinrich Schnitzler. Frankfurt am Main: \emph{S. Fischer} 1964, S. 123.} \weitereDrucke{2) Hermann Bahr, Arthur Schnitzler: \emph{Briefwechsel, Aufzeichnungen, Dokumente (1891–1931)}. Herausgegeben von Kurt Ifkovits und Martin Anton Müller. Göttingen: \emph{Wallstein} 2018, S. 170.} }\toendnotes[C]{\smallbreak}
\pstart{}{\pb}lieber Hugo,\pend\vspace{0.5em}
\pstart
           folgendes iſt mit \uuline{\edtext{vollkommener Discretion}{\Cendnote{dreifach unterstrichen}}} zu
               behandeln: \uline{Bahr}\pwindex{Bahr, Hermann 19.\,7.\,1863 Linz – 15.\,1.\,1934 München@\textsc{Bahr, Hermann} (19.\,7.\,1863 Linz – 15.\,1.\,1934 München), \emph{Schriftsteller, Kritiker}|pw}\uline{{ }verläßt die{ }}\uline{Zeit}\orgindex{Zeit. Wiener Wochenschrift@Die Zeit. Wiener Wochenschrift|pw}. Singer\pwindex{Singer, Isidor 16.\,1.\,1857 Budapest – 8.\,12.\,1927 Wien@\textsc{Singer, Isidor} (16.\,1.\,1857 Budapest – 8.\,12.\,1927 Wien), \emph{Journalist, Herausgeber, Soziologe}|pw} und Kanner\pwindex{Kanner, Heinrich 9.\,11.\,1864 Galați – 15.\,2.\,1930 Wien@\textsc{Kanner, Heinrich} (9.\,11.\,1864 Galați – 15.\,2.\,1930 Wien), \emph{Herausgeber, Publizist}|pw} waren bei mir. Lange Unterredung ohne Intereſſe für Sie
               (nur mich.) Das weſentliche:{ }ſie möchten auf das Blatt{ }ſtellen: unter Mitwirkung von
               – \textsc{etc etc} nur erſte Namen, ich möchte Sie fragen, ob Sie im
               Princip damit {\pb}einverſtanden wären, auch als
                  »Mitwirkender{[}«{]} oder »ſtändg Mitwirkender« aufs Blatt zu ko{\geminationm}en, neben \textsc{Burckhard\pwindex{Burckhard, Max Eugen 14.\,7.\,1854 Korneuburg – 16.\,3.\,1912 Wien@\textsc{Burckhard, Max Eugen} (14.\,7.\,1854 Korneuburg – 16.\,3.\,1912 Wien), \emph{Schriftsteller, Rechtswissenschaftler, Theaterleiter}|pw}}, mich, – event. \textsc{Hauptmann}\pwindex{Hauptmann, Gerhart 15.\,11.\,1862 Szczawno-Zdrój – 6.\,6.\,1946 Jagniątków@\textsc{Hauptmann, Gerhart} (15.\,11.\,1862 Szczawno-Zdrój – 6.\,6.\,1946 Jagniątków), \emph{Schriftsteller}|pw} (\label{K_L00934-1v}\edtext{an den ich mich über Brahm\pwindex{Brahm, Otto 5.\,2.\,1856 Hamburg – 28.\,11.\,1912 Berlin@\textsc{Brahm, Otto} (5.\,2.\,1856 Hamburg – 28.\,11.\,1912 Berlin), \emph{Theaterleiter, Regisseur}|pw} wende}{\lemma{\textnormal{\emph{an … wende}}}\Cendnote{\textnormal{Siehe XXXX Auszeichnungsfehler: Dokument L00943 nicht gefunden.
               }}}\label{K_L00934-1}.) Sie können natürlich ohne weiters zuſagen. Für die Herausgeber{ }ſcheint mir
               die Sache allerdings überflüſſig:{ }ſie brauchten Arbeitskräfte, nicht Namen. –\pend
           
\pstart
           Ich bin noch hier; und will über meine {\pb}Sti{\geminationm}ung nichts{ }ſagen, da nichts neues u nicht erfreuliches
               vorliegt. Gerade dſs{ }ſich das Leben da und dort wieder zu melden anfängt, iſt das
               traurige; es iſt ein Leben dritter Ordnung, das beſte iſt vorbei.\pend
           
\pstart
           Das Wetter ist{ }ſchändlich. Mitte Juli reiſe ich nach Kärnthen\oindex{Kärnten@\textbf{Kärnten}, \emph{Land}|pw}; zuerſt \textsc{Velden}\oindex{Velden am Wörthersee@\textbf{Velden am Wörthersee}|pw}, dann zu Richard\pwindex{Beer-Hofmann, Richard 11.\,7.\,1866 Wien – 26.\,9.\,1945 New York City@\textsc{Beer-Hofmann, Richard} (11.\,7.\,1866 Wien – 26.\,9.\,1945 New York City), \emph{Schriftsteller}|pw}, von dem ich eine kurze
               \label{K_L00934-2v}\edtext{Karte}{\lemma{\textnormal{\emph{Karte}}}\Cendnote{\textnormal{Siehe XXXX Auszeichnungsfehler: Dokument L00933 nicht gefunden.
               }}}\label{K_L00934-2} habe. – Hat{ }ſich in den Chancen für Mitte Auguſt (Thü{\pb}ringen\oindex{Thüringen@\textbf{Thüringen}, \emph{Land}|pw}{ }\textsc{etc}) was geändert? – Arbeiten Sie? Sehn Sie Minnie\pwindex{Schaffgotsch, Hermine von 25.\,11.\,1871 Wien – 25.\,11.\,1928 Purgstall@\textsc{Schaffgotsch, Hermine von} (25.\,11.\,1871 Wien – 25.\,11.\,1928 Purgstall)|pw}? –\pend
           \pstart Leben Sie wohl. Von Herzen Ihr \spacefill\mbox{Arthur Sch}\pend{}
\pstart
           Wien\oindex{Wien@\textbf{Wien}, \emph{Verwaltungsgebiet}|pw}{ }6. 7. 99.\pend
           \selectlanguage{ngerman}\endnumbering\briefempfaengerindex{Hofmannsthal, Hugo von@\textsc{Hofmannsthal, Hugo von}!zzzSchnitzler, Arthur@\emph{von Arthur Schnitzler}!1899-07-062@{6. 7. 1899}|)be}\mylabel{L00934h}  \newcommand{\dateiname}{L00934}\newcommand{\titel}{Arthur Schnitzler an Hugo von Hofmannsthal, 6. 7. 1899}\newcommand{\editorInnen}{Herausgegeben von Martin Anton Müller}%% latex-leseansicht-abspann.tex
%% Abspann für die Leseansicht.
%% Der Schalter \ifkorrekturansicht ist bereits durch den Vorspann gesetzt.

%% latex-abspann.tex
%% Gemeinsamer Abspann für Korrekturansicht und Leseansicht.
%% Setzt den Schalter \ifkorrekturansicht voraus (gesetzt in den
%% einbindenden Dateien latex-korrekturansicht-abspann.tex bzw.
%% latex-leseansicht-abspann.tex).
%% ---------------------------------------------------------------

\normalsize

% Das esempio-Environment wird nur in der Leseansicht benötigt
\ifkorrekturansicht\else
\newenvironment{esempio}[3]%
{
    \vspace{1.5ex}
    \rlap{\underline{#1}}
    \par
    \setlength{\parindent}{0cm}
    \nopagebreak
    \leftskip=#2cm
    \rightskip=#3cm
}
{
    \par
}
\fi

\doendnotes{C}
\bigskip
\vfill

\clearpage

\footnotesize

\ifkorrekturansicht
  \lohead{\textsc{register}}
\fi

% theindex-Environment neu definieren ohne reledmac
\makeatletter
\renewenvironment{theindex}{%
  \ifkorrekturansicht
    \section*{\indexname}%
  \else
    \subsubsection*{Index der erwähnten Entitäten}%
  \fi
  \setlength{\parindent}{0pt}%
  \setlength{\parskip}{0pt plus 0.3pt}%
  \let\item\@idxitem
}{%
  \ifkorrekturansicht\clearpage\fi
}
\makeatother

\IfFileExists{\jobname-pw.ind}{\input{\jobname-pw.ind}}{}

% Quellenangabe nur in der Leseansicht
\ifkorrekturansicht\else
% Fallback-Definitionen, falls die .tex-Datei \titel etc. nicht gesetzt hat
\providecommand{\titel}{}
\providecommand{\editorInnen}{}
\providecommand{\dateiname}{\jobname}

\vspace{3cm}

\vfill

\footnotesize
\textsc{Quelle}: \titel. Herausgegeben von {\editorInnen}. In: \emph{Arthur Schnitzler: Briefwechsel mit Autorinnen und Autoren}.
 Digitale Edition, https://schnitzler-briefe.acdh.oeaw.ac.at/{\dateiname}.html (Stand \today)
\fi

\end{document}


