%% latex-leseansicht-vorspann.tex
%% Vorspann für die Leseansicht.
%% Lädt die gemeinsame Datei latex-vorspann.tex mit nicht gesetztem Schalter.

\newif\ifkorrekturansicht
\korrekturansichtfalse

\input{../tex-inputs/latex-vorspann}


\section[Arthur Schnitzler und Hugo von Hofmannsthal an Gerty von Hofmannsthal, 15. 3. 1905]{L01509 Arthur Schnitzler und Hugo von Hofmannsthal an Gerty von Hofmannsthal, 15. 3. 1905}
\nopagebreak\mylabel{L01509v}
\rehead{ }\normalsize\beginnumbering\briefempfaengerindex{Hofmannsthal, Gertrude von@\textsc{Hofmannsthal, Gertrude von}!zzzHofmannsthal, Hugo von@\emph{von Hugo von Hofmannsthal}!1905-03-151@{15. 3. 1905}|(be}\briefempfaengerindex{Hofmannsthal, Gertrude von@\textsc{Hofmannsthal, Gertrude von}!zzzSchnitzler, Arthur@\emph{von Arthur Schnitzler}!1905-03-151@{15. 3. 1905}|(be}
\toendnotes[C]{\smallbreak\pagebreak[2]}
\correspDesc{Versand  durch Arthur Schnitzler, Hugo von Hofmannsthal am 15. 3. 1905 in Dubrovnik
\newline{}Erhalt  durch Gerty von Hofmannsthal im Zeitraum [16. 3. 1905
                  – 20. 3. 1905?] in Rodaun}\toendnotes[C]{\smallbreak}
\Standort{FDH, Hs-30997,159.}
\physDesc{Bildpostkarte, 112 Zeichen
\newline{}Handschrift Arthur Schnitzler: Bleistift, deutsche Kurrent
\newline{}Handschrift Hugo von Hofmannsthal: Bleistift, lateinische Kurrent
\newline{}Versand: 1) Stempel: »\nobreak{}\oindex{Rijeka@\textbf{Rijeka}|pwk}Gravos{[}a{]}\nobreak{}«.   2) Stempel: »\nobreak{}\oindex{Wien@\textbf{Wien}!XXIII., Liesing@\textbf{XXIII., Liesing}!Rodaun@\textbf{Rodaun}, \emph{Region}|pwk}Rodaun, \textcolor{gray}{17. 3. 05}, 4–6N\nobreak{}«. }\pstart{}\textsc{{\pb}Frau Gerty v Hofma{\geminationn}sthal}\pend{}\pstart{}\textsc{Rodaun b/Wien\oindex{Wien@\textbf{Wien}!XXIII., Liesing@\textbf{XXIII., Liesing}!Rodaun@\textbf{Rodaun}, \emph{Region}|pw}}\pend{}\pstart{}\textsc{Badgasse 5\oindex{Wien@\textbf{Wien}!XXIII., Liesing@\textbf{XXIII., Liesing}!Badgasse@\textbf{Badgasse}, \emph{Straße}|pw}.}\pend{}{\bigskip}
\pstart
           \noindent{}\centering{}{\pb}\textcolor{gray}{\textbf{Am Bord des Doppelschrauben-Dampfers »METEOR«}}\pend
           
\pstart
           \centering{}\textcolor{gray}{\textbf{Hamburg\oindex{Hamburg@\textbf{Hamburg}|pw}–Amerika\oindex{Amerika@\textbf{Amerika}|pw} Linie}}\pend
           \vspace{1em}
\pstart
           {\pb}\textsc{Ragusa}\oindex{Dubrovnik@\textbf{Dubrovnik}|pw}{ }15. 3. 905\pend
           \vspace{0.5em}
\pstart
           Herzliche Grüße.\pend
           \pstart \spacefill\mbox{A. S.}\pend{}\selectlanguage{ngerman}\vspace{1em}
\pstart
           \noindent{}{[}hs. Hofmannsthal:{]} \textsc{Hier ist gutes Essen.}\pend
           \pstart \spacefill\mbox{Hugo.}\pend{}\selectlanguage{ngerman}\endnumbering\briefempfaengerindex{Hofmannsthal, Gertrude von@\textsc{Hofmannsthal, Gertrude von}!zzzHofmannsthal, Hugo von@\emph{von Hugo von Hofmannsthal}!1905-03-151@{15. 3. 1905}|)be}\briefempfaengerindex{Hofmannsthal, Gertrude von@\textsc{Hofmannsthal, Gertrude von}!zzzSchnitzler, Arthur@\emph{von Arthur Schnitzler}!1905-03-151@{15. 3. 1905}|)be}\mylabel{L01509h}  \newcommand{\dateiname}{L01509}\newcommand{\titel}{Arthur Schnitzler und Hugo von Hofmannsthal an Gerty von Hofmannsthal, 15. 3. 1905}\newcommand{\editorInnen}{Martin Anton Müller und Gerd-Hermann Susen}%% latex-leseansicht-abspann.tex
%% Abspann für die Leseansicht.
%% Der Schalter \ifkorrekturansicht ist bereits durch den Vorspann gesetzt.

%% latex-abspann.tex
%% Gemeinsamer Abspann für Korrekturansicht und Leseansicht.
%% Setzt den Schalter \ifkorrekturansicht voraus (gesetzt in den
%% einbindenden Dateien latex-korrekturansicht-abspann.tex bzw.
%% latex-leseansicht-abspann.tex).
%% ---------------------------------------------------------------

\normalsize

% Das esempio-Environment wird nur in der Leseansicht benötigt
\ifkorrekturansicht\else
\newenvironment{esempio}[3]%
{
    \vspace{1.5ex}
    \rlap{\underline{#1}}
    \par
    \setlength{\parindent}{0cm}
    \nopagebreak
    \leftskip=#2cm
    \rightskip=#3cm
}
{
    \par
}
\fi

\doendnotes{C}
\bigskip
\vfill

\clearpage

\footnotesize

\ifkorrekturansicht
  \lohead{\textsc{register}}
\fi

% theindex-Environment neu definieren ohne reledmac
\makeatletter
\renewenvironment{theindex}{%
  \ifkorrekturansicht
    \section*{\indexname}%
  \else
    \subsubsection*{Index der erwähnten Entitäten}%
  \fi
  \setlength{\parindent}{0pt}%
  \setlength{\parskip}{0pt plus 0.3pt}%
  \let\item\@idxitem
}{%
  \ifkorrekturansicht\clearpage\fi
}
\makeatother

\IfFileExists{\jobname-pw.ind}{\input{\jobname-pw.ind}}{}

% Quellenangabe nur in der Leseansicht
\ifkorrekturansicht\else
% Fallback-Definitionen, falls die .tex-Datei \titel etc. nicht gesetzt hat
\providecommand{\titel}{}
\providecommand{\editorInnen}{}
\providecommand{\dateiname}{\jobname}

\vspace{3cm}

\vfill

\footnotesize
\textsc{Quelle}: \titel. Herausgegeben von {\editorInnen}. In: \emph{Arthur Schnitzler: Briefwechsel mit Autorinnen und Autoren}.
 Digitale Edition, https://schnitzler-briefe.acdh.oeaw.ac.at/{\dateiname}.html (Stand \today)
\fi

\end{document}


