%% latex-korrekturansicht-vorspann.tex
%% Vorspann für die Korrekturansicht.
%% Lädt die gemeinsame Datei latex-vorspann.tex mit gesetztem Schalter.

\newif\ifkorrekturansicht
\korrekturansichttrue

\input{../tex-inputs/latex-vorspann}


\section[Arthur Schnitzler an Hugo von Hofmannsthal, 20. 8. 1904]{L01430 Arthur Schnitzler an Hugo von Hofmannsthal, 20. 8. 1904}
\nopagebreak\mylabel{L01430v}
\rehead{ }\normalsize\beginnumbering\briefempfaengerindex{Hofmannsthal, Hugo von@\textsc{Hofmannsthal, Hugo von}!zzzSchnitzler, Arthur@\emph{von Arthur Schnitzler}!1904-08-201@{20. 8. 1904}|(be}
\toendnotes[C]{\smallbreak\pagebreak[2]}\Standort{FDH, Hs-30885,112.}
\physDesc{Brief, 2 Blätter, 6 Seiten, 2548 Zeichen
\newline{}Handschrift: schwarze Tinte, deutsche Kurrent
\newline{}Ordnung: mit Bleistift von Schnitzler mutmaßlich bei der Durchsicht der
                                 Korrespondenz 1929 das zweite Blatt beschrieben:
                                    »II 20/8 904« }
\buchAbdrucke{\weitereDrucke{1) Hugo von Hofmannsthal, Arthur Schnitzler: \emph{Briefwechsel}. Frankfurt am Main: \emph{S. Fischer} 1964, S. 197–199.} \weitereDrucke{2) Hermann Bahr, Arthur Schnitzler: \emph{Briefwechsel, Aufzeichnungen, Dokumente (1891–1931)}. Göttingen: \emph{Wallstein} 2018, S. 316.} }\toendnotes[C]{\smallbreak}
\pstart
           \raggedleft{}{\pb}Wien\oindex{Wien@\textbf{Wien}, \emph{A.ADM2}|pw}{ }20. 8. 904\pend
           \vspace{0.5em}
\pstart
           lieber Hugo, mit der Salzk.gut\oindex{Salzkammergut@\textbf{Salzkammergut}, \emph{L.RGN}|pw}reiſe ſteht es wie folgt: in dieſen Tagen beende ich die erſte
               flüchtige Niederſchrift eines neuen dreiaktigen Stücks\pwindex{Zwischenspiel. Komoedie in drei Akten@\emph{Zwischenspiel. Komödie in drei Akten}|pwv}; die Grünwald\pwindex{Gruenwald, Ida 28.06.1873 – Mai 1908@\textsc{Grünwald, Ida} (28.06.1873 – Mai 1908), \emph{Stenotypistin/Stenotypistin}|pw}
                  ko{\geminationm}t etwa 25., 26., und
               dann muſs ich es, um es überſichtlich vor mir zu haben, und weil das überhaupt zu den
               Etappen meiner Arbeitsweiſe gehört u mich ſehr fördert, dictiren. Nun ka{\geminationn} ich, auch weil der Anfangstag der Grünwald\pwindex{Gruenwald, Ida 28.06.1873 – Mai 1908@\textsc{Grünwald, Ida} (28.06.1873 – Mai 1908), \emph{Stenotypistin/Stenotypistin}|pw}{ }\substVorne{}\textsuperscript{\textcolor{gray}{sich}}\substDazwischen{}noch nicht feſtſteht\substHinten{} (ich bin ohne Nachricht, \textsc{resp} Antwort von ihr),
               nicht {\pb}auf den Tag beſtimmen, wann ich fertig bin. Ich
                  \uline{hoffe}, es wird ſich fügen, daſs wir schon am
                  3.{ }Wien\oindex{Wien@\textbf{Wien}, \emph{A.ADM2}|pw} verlaſſen können; wird aber \textsc{Gerty\pwindex{Hofmannsthal, Gertrude von 16.03.1880 – 09.11.1959@\textsc{Hofmannsthal, Gertrude von} (16.03.1880 – 09.11.1959)|pw}} auch warten, wenn der 4. oder gar der 5. September
               draus wird? Wir möchten natürlich auch ſehr gern mit ihr zuſammen fahren; ich ka{\geminationn} nur heute mich zur Beſti{\geminationm}ung des Tages nicht verpflichten. Immerhin werde ich am erſten Dictirtag ſchon
               wiſſen können, wa{\geminationn} wir bereit ſind. Ich hoffe ja ſehr,
               daſs es der 3.{ }ſein wird. Sie erſehen daraus {\pb}jedenfalls, daſs wir zu Iſchl\oindex{Hotel zum goldenen Kreuz [Bad Ischl]@\textbf{Hotel zum goldenen Kreuz [Bad Ischl]}, \emph{Hotel (K.HTL)}|pw} entſchloſſen iſt, wo wir fürs erſte Quartier nehmen, Ausflüge machen
                  (Olga\pwindex{Schnitzler, Olga 17.01.1882 – 13.01.1970@\textsc{Schnitzler, Olga} (17.01.1882 – 13.01.1970), \emph{Schauspieler/Schauspielerin, Sänger/Sängerin}|pw} kennt das Salzka{\geminationm}ergut\oindex{Salzkammergut@\textbf{Salzkammergut}, \emph{L.RGN}|pw} gar nicht), und ich
               ſehne mich auch ſehr nach ein paar ſchönen Radtouren mit Ihnen. Auch zu einer
               Fußpartie (Ruckſack!) wär ich zu haben. Nicht unmöglich iſt es, daſs ich da{\geminationn} auch noch mit Olga\pwindex{Schnitzler, Olga 17.01.1882 – 13.01.1970@\textsc{Schnitzler, Olga} (17.01.1882 – 13.01.1970), \emph{Schauspieler/Schauspielerin, Sänger/Sängerin}|pw} weiterfahre, Tirol\oindex{Tirol@\textbf{Tirol}, \emph{A.ADM1}|pw}, Bozner\oindex{Bozen@\textbf{Bozen}, \emph{P.PPLA2}|pw} Gegend, und falls das Wetter allzu
               herbſtlich wird, München\oindex{Muenchen@\textbf{München}, \emph{P.PPLA}|pw}. Wir ſehen uns ja
               jedenfalls ſchon am ersten {\pb}Iſchl\oindex{Bad Ischl@\textbf{Bad Ischl}, \emph{P.PPL}|pw}er Tag, aber ſagen Sie mir doch gleich, wa{\geminationn}{ }Sie wieder in Rodaun\oindex{Rodaun@\textbf{Rodaun}, \emph{A.ADM4}|pw} zurück ſein müſſen oder wollen. Wohnen wollen wir in der Kaiſerkrone\oindex{Hotel Kaiserkrone@\textbf{Hotel Kaiserkrone}, \emph{Hotel (K.HTL)}|pw}. –\pend
           
\pstart
           Sind Sie mit dem »geretteten\pwindex{gerettete Venedig. Trauerspiel in fuenf Aufzuegen@\emph{Das gerettete Venedig. Trauerspiel in fünf Aufzügen}|pw}« fertig? Mir geht
               es mit dem Arbeiten nicht übel und ginge mir gewiſs noch beſſer, we{\geminationn} nicht mein Widerwillen gegen den phyſ. Akt des
               Schreibens immer beträchtlicher würde und ſich oft genug in leichten Schreibkrämpfen
               äußerte.\pend
           
\pstart
           Danke ſehr betreffs V. S.\pwindex{Vansittart, Robert Gilbert 25.06.1881 – 14.02.1957@\textsc{Vansittart, Robert Gilbert} (25.06.1881 – 14.02.1957), \emph{Diplomat/Diplomatin}|pw}, mein Aerger hat
               ſich natürlich ſchon gelegt – natürlich würde es mich aber {\pb}ſehr freuen, wenn Ordnung in die ganze Angelegenheit
               gebracht werden könnte und ich von England\oindex{England@\textbf{England}, \emph{A.ADM1}|pw}, Irland\oindex{Irland@\textbf{Irland}, \emph{A.PCLI}|pw} u Schottland\oindex{Schottland@\textbf{Schottland}, \emph{A.ADM1}|pw} nicht länger misverſtanden \introOben{}verfolgt u
                  geächtet\introOben{} würde. –\pend
           
\pstart
           – \textsc{Vehse\pwindex{Geschichte der deutschen Hoefe seit der Reformation@\emph{Geschichte der deutschen Höfe seit der Reformation}|pwv}} iſt und bleibt ein koſtbares Buch. Zudem studier ich, des Überblickes halber,
               Geſchichte \introOben{}wie\introOben{} zur Matura. Ich wäre weiter als ich bin, we{\geminationn} ich ein gebildeter Menſch wäre!\pend
           
\pstart
           Was iſts mit Richard\pwindex{Beer-Hofmann, Richard 1866-07-11 – 1945-09-26@\textsc{Beer-Hofmann, Richard} (1866-07-11 – 1945-09-26), \emph{Schriftsteller/Schriftstellerin}|pw}? Seine Karte mit Paula\pwindex{Beer-Hofmann, Paula 25.02.1879 – 30.10.1939@\textsc{Beer-Hofmann, Paula} (25.02.1879 – 30.10.1939)|pw}{ }\textcolor{gray}{wie} den \textcolor{gray}{Kindern\pwindex{Beer-Hofmann, Naemah 20.12.1898 – 10.11.1971@\textsc{Beer-Hofmann, Naëmah} (20.12.1898 – 10.11.1971)|pwv}\pwindex{Beer-Hofmann, Mirjam 04.09.1897 – 24.12.1984@\textsc{Beer-Hofmann, Mirjam} (04.09.1897 – 24.12.1984)|pwv}\pwindex{Beer-Hofmann, Gabriel 09.01.1901 – 24.03.1971@\textsc{Beer-Hofmann, Gabriel} (09.01.1901 – 24.03.1971), \emph{Schriftsteller/Schriftstellerin, Filmagent/Filmagentin}|pwv}}{ }\textcolor{gray}{an}{ }\textcolor{gray}{×}\-\textcolor{gray}{×}\-\textcolor{gray}{×} hab ich bekommen. Von ſich
               ſchreibt er nichts. Grüßen Sie alle, die mir lieb ſind.\pend
           \pstart Herzlichſt Ihr\spacefill\mbox{A.{\pb}}\pend{}
\pstart
           \noindent{}{\pb}\textsc{Gerty\pwindex{Hofmannsthal, Gertrude von 16.03.1880 – 09.11.1959@\textsc{Hofmannsthal, Gertrude von} (16.03.1880 – 09.11.1959)|pw}} wird wohl auch am liebſten mit dem Zehn Uhr Früh Zug fahren?{\\}A.\pend
           
\pstart
           Geſtern Abend waren wir mit Bahr\pwindex{Bahr, Hermann 19.07.1863 – 15.01.1934@\textsc{Bahr, Hermann} (19.07.1863 – 15.01.1934), \emph{Schriftsteller/Schriftstellerin, Kritiker/Kritikerin}|pw}, (Hietzing\oindex{XIII., Hietzing@\textbf{XIII., Hietzing}, \emph{A.ADM3}|pw}) dem’s recht gut, und was das
                  weſentlichſte iſt, hoffnungsvoll zu gehen ſcheint.\pend
           \selectlanguage{ngerman}\endnumbering\briefempfaengerindex{Hofmannsthal, Hugo von@\textsc{Hofmannsthal, Hugo von}!zzzSchnitzler, Arthur@\emph{von Arthur Schnitzler}!1904-08-201@{20. 8. 1904}|)be}\mylabel{L01430h}  \normalsize

\doendnotes{C}
\bigskip
\vfill

\clearpage

\footnotesize

\lohead{\textsc{register}}

% Definiere theindex-Environment komplett neu ohne reledmac
\makeatletter
\renewenvironment{theindex}{%
  \section*{\indexname}%
  \setlength{\parindent}{0pt}%
  \setlength{\parskip}{0pt plus 0.3pt}%
  \let\item\@idxitem
}{%
  \clearpage
}
\makeatother

\IfFileExists{\jobname-pw.ind}{\input{\jobname-pw.ind}}{}

\end{document}

      