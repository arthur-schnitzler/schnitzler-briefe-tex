%% latex-leseansicht-vorspann.tex
%% Vorspann für die Leseansicht.
%% Lädt die gemeinsame Datei latex-vorspann.tex mit nicht gesetztem Schalter.

\newif\ifkorrekturansicht
\korrekturansichtfalse

\input{../tex-inputs/latex-vorspann}


\section[Arthur Schnitzler an Hugo von Hofmannsthal, 20. 8. 1904]{L01430 Arthur Schnitzler an Hugo von Hofmannsthal, 20. 8. 1904}
\nopagebreak\mylabel{L01430v}
\rehead{ }\normalsize\beginnumbering\briefempfaengerindex{Hofmannsthal, Hugo von@\textsc{Hofmannsthal, Hugo von}!zzzSchnitzler, Arthur@\emph{von Arthur Schnitzler}!1904-08-201@{20. 8. 1904}|(be}
\toendnotes[C]{\smallbreak\pagebreak[2]}
\correspDesc{Versand  durch Arthur Schnitzler am 20. 8. 1904 in Wien
\newline{}Erhalt  durch Hugo von Hofmannsthal im Zeitraum [20. 8. 1904
                  – 24. 8. 1904?] \textbf{Ort fehlend} }\toendnotes[C]{\smallbreak}
\Standort{FDH, Hs-30885,112.}
\physDesc{Brief, 2 Blätter, 6 Seiten, 2548 Zeichen
\newline{}Handschrift: schwarze Tinte, deutsche Kurrent
\newline{}Ordnung: mit Bleistift von Schnitzler mutmaßlich bei der Durchsicht der
                                 Korrespondenz 1929 das zweite Blatt beschrieben:
                                    »II 20/8 904« }
\buchAbdrucke{\weitereDrucke{1) Hugo von Hofmannsthal, Arthur Schnitzler: \emph{Briefwechsel}. Herausgegeben von Therese Nickl und Heinrich Schnitzler. Frankfurt am Main: \emph{S. Fischer} 1964, S. 197–199.} \weitereDrucke{2) Hermann Bahr, Arthur Schnitzler: \emph{Briefwechsel, Aufzeichnungen, Dokumente (1891–1931)}. Herausgegeben von Kurt Ifkovits und Martin Anton Müller. Göttingen: \emph{Wallstein} 2018, S. 316.} }\toendnotes[C]{\smallbreak}
\pstart
           \raggedleft{}{\pb}Wien\oindex{Wien@\textbf{Wien}, \emph{Verwaltungsgebiet}|pw}{ }20. 8. 904\pend
           \vspace{0.5em}
\pstart
           lieber Hugo, mit der Salzk.gut\oindex{Salzkammergut@\textbf{Salzkammergut}, \emph{Region}|pw}reiſe{ }ſteht es wie folgt: in dieſen Tagen beende ich die erſte
               flüchtige Niederſchrift eines neuen dreiaktigen Stücks\pwindex{Schnitzler, Arthur 15.\,5.\,1862 Wien – 21.\,10.\,1931 ebd.@\textsc{Schnitzler, Arthur} (15.\,5.\,1862 Wien – 21.\,10.\,1931 ebd.), \emph{Schriftsteller, Mediziner}!Zwischenspiel. Komödie in drei Akten@\strich\emph{Zwischenspiel. Komödie in drei Akten}|pwv}; die Grünwald\pwindex{Grünwald, Ida 28.\,6.\,1873 Wien – Mai 1908 Alland@\textsc{Grünwald, Ida} (28.\,6.\,1873 Wien – Mai 1908 Alland), \emph{Stenotypistin}|pw}
                  ko{\geminationm}t etwa 25., 26., und
               dann muſs ich es, um es überſichtlich vor mir zu haben, und weil das überhaupt zu den
               Etappen meiner Arbeitsweiſe gehört u mich{ }ſehr fördert, dictiren. Nun ka{\geminationn} ich, auch weil der Anfangstag der Grünwald\pwindex{Grünwald, Ida 28.\,6.\,1873 Wien – Mai 1908 Alland@\textsc{Grünwald, Ida} (28.\,6.\,1873 Wien – Mai 1908 Alland), \emph{Stenotypistin}|pw}{ }\substVorne{}\textsuperscript{\textcolor{gray}{sich}}\substDazwischen{}noch nicht feſtſteht\substHinten{} (ich bin ohne Nachricht, \textsc{resp} Antwort von ihr),
               nicht {\pb}auf den Tag beſtimmen, wann ich fertig bin. Ich
                  \uline{hoffe}, es wird{ }ſich fügen, daſs wir schon am
                  3.{ }Wien\oindex{Wien@\textbf{Wien}, \emph{Verwaltungsgebiet}|pw} verlaſſen können; wird aber \textsc{Gerty\pwindex{Hofmannsthal, Gertrude von 16.\,3.\,1880 Wien – 9.\,11.\,1959 Paddington@\textsc{Hofmannsthal, Gertrude von} (16.\,3.\,1880 Wien – 9.\,11.\,1959 Paddington)|pw}} auch warten, wenn der 4. oder gar der 5. September
               draus wird? Wir möchten natürlich auch{ }ſehr gern mit ihr zuſammen fahren; ich ka{\geminationn} nur heute mich zur Beſti{\geminationm}ung des Tages nicht verpflichten. Immerhin werde ich am erſten Dictirtag{ }ſchon
               wiſſen können, wa{\geminationn} wir bereit{ }ſind. Ich hoffe ja{ }ſehr,
               daſs es der 3.{ }ſein wird. Sie erſehen daraus {\pb}jedenfalls, daſs wir zu Iſchl\oindex{Hotel zum goldenen Kreuz [Bad Ischl]@\textbf{Hotel zum goldenen Kreuz [Bad Ischl]}, \emph{Hotel}|pw} entſchloſſen iſt, wo wir fürs erſte Quartier nehmen, Ausflüge machen
                  (Olga\pwindex{Schnitzler, Olga 17.\,1.\,1882 Wien – 13.\,1.\,1970 Lugano@\textsc{Schnitzler, Olga} (17.\,1.\,1882 Wien – 13.\,1.\,1970 Lugano), \emph{Schauspielerin, Sängerin}|pw} kennt das Salzka{\geminationm}ergut\oindex{Salzkammergut@\textbf{Salzkammergut}, \emph{Region}|pw} gar nicht), und ich{ }ſehne mich auch{ }ſehr nach ein paar{ }ſchönen Radtouren mit Ihnen. Auch zu einer
               Fußpartie (Ruckſack!) wär ich zu haben. Nicht unmöglich iſt es, daſs ich da{\geminationn} auch noch mit Olga\pwindex{Schnitzler, Olga 17.\,1.\,1882 Wien – 13.\,1.\,1970 Lugano@\textsc{Schnitzler, Olga} (17.\,1.\,1882 Wien – 13.\,1.\,1970 Lugano), \emph{Schauspielerin, Sängerin}|pw} weiterfahre, Tirol\oindex{Tirol@\textbf{Tirol}, \emph{Land}|pw}, Bozner\oindex{Bozen@\textbf{Bozen}, \emph{Hauptstadt}|pw} Gegend, und falls das Wetter allzu
               herbſtlich wird, München\oindex{München@\textbf{München}|pw}. Wir{ }ſehen uns ja
               jedenfalls{ }ſchon am ersten {\pb}Iſchl\oindex{Bad Ischl@\textbf{Bad Ischl}|pw}er Tag, aber{ }ſagen Sie mir doch gleich, wa{\geminationn}{ }Sie wieder in Rodaun\oindex{Wien@\textbf{Wien}!XXIII., Liesing@\textbf{XXIII., Liesing}!Rodaun@\textbf{Rodaun}, \emph{Region}|pw} zurück{ }ſein müſſen oder wollen. Wohnen wollen wir in der Kaiſerkrone\oindex{Hotel Kaiserkrone@\textbf{Hotel Kaiserkrone}, \emph{Hotel}|pw}. –\pend
           
\pstart
           Sind Sie mit dem »geretteten\pwindex{Hofmannsthal, Hugo von 1.\,2.\,1874 Wien – 15.\,7.\,1929 Rodaun@\textsc{Hofmannsthal, Hugo von} (1.\,2.\,1874 Wien – 15.\,7.\,1929 Rodaun), \emph{Schriftsteller}!gerettete Venedig. Trauerspiel in fünf Aufzügen@\strich\emph{Das gerettete Venedig. Trauerspiel in fünf Aufzügen}|pw}« fertig? Mir geht
               es mit dem Arbeiten nicht übel und ginge mir gewiſs noch beſſer, we{\geminationn} nicht mein Widerwillen gegen den phyſ. Akt des
               Schreibens immer beträchtlicher würde und{ }ſich oft genug in leichten Schreibkrämpfen
               äußerte.\pend
           
\pstart
           Danke{ }ſehr betreffs V. S.\pwindex{Vansittart, Robert Gilbert 25.\,6.\,1881 Farnham – 14.\,2.\,1957 Denham@\textsc{Vansittart, Robert Gilbert} (25.\,6.\,1881 Farnham – 14.\,2.\,1957 Denham), \emph{Diplomat}|pw}, mein Aerger hat{ }ſich natürlich{ }ſchon gelegt – natürlich würde es mich aber {\pb}ſehr freuen, wenn Ordnung in die ganze Angelegenheit
               gebracht werden könnte und ich von England\oindex{England@\textbf{England}, \emph{Land}|pw}, Irland\oindex{Irland@\textbf{Irland}|pw} u Schottland\oindex{Schottland@\textbf{Schottland}, \emph{Land}|pw} nicht länger misverſtanden \introOben{}verfolgt u
                  geächtet\introOben{} würde. –\pend
           
\pstart
           – \textsc{Vehse\pwindex{\textcolor{red}{\textsuperscript{XXXX indx1}}!Geschichte der deutschen Höfe seit der Reformation@\strich\emph{Geschichte der deutschen Höfe seit der Reformation}|pwv}} iſt und bleibt ein koſtbares Buch. Zudem studier ich, des Überblickes halber,
               Geſchichte \introOben{}wie\introOben{} zur Matura. Ich wäre weiter als ich bin, we{\geminationn} ich ein gebildeter Menſch wäre!\pend
           
\pstart
           Was iſts mit Richard\pwindex{Beer-Hofmann, Richard 11.\,7.\,1866 Wien – 26.\,9.\,1945 New York City@\textsc{Beer-Hofmann, Richard} (11.\,7.\,1866 Wien – 26.\,9.\,1945 New York City), \emph{Schriftsteller}|pw}? Seine Karte mit Paula\pwindex{Beer-Hofmann, Paula 25.\,2.\,1879 Wien – 30.\,10.\,1939 Zürich@\textsc{Beer-Hofmann, Paula} (25.\,2.\,1879 Wien – 30.\,10.\,1939 Zürich)|pw}{ }\textcolor{gray}{wie} den \textcolor{gray}{Kindern\pwindex{Beer-Hofmann, Naëmah 20.\,12.\,1898 Wien – 10.\,11.\,1971 New York City@\textsc{Beer-Hofmann, Naëmah} (20.\,12.\,1898 Wien – 10.\,11.\,1971 New York City)|pwv}\pwindex{Beer-Hofmann, Mirjam 4.\,9.\,1897 Wien – 24.\,12.\,1984 New York City@\textsc{Beer-Hofmann, Mirjam} (4.\,9.\,1897 Wien – 24.\,12.\,1984 New York City)|pwv}\pwindex{Beer-Hofmann, Gabriel 9.\,1.\,1901 Wien – 24.\,3.\,1971 St Albans@\textsc{Beer-Hofmann, Gabriel} (9.\,1.\,1901 Wien – 24.\,3.\,1971 St Albans), \emph{Schriftsteller, Filmagent}|pwv}}{ }\textcolor{gray}{an}{ }\textcolor{gray}{×}\-\textcolor{gray}{×}\-\textcolor{gray}{×} hab ich bekommen. Von{ }ſich{ }ſchreibt er nichts. Grüßen Sie alle, die mir lieb{ }ſind.\pend
           \pstart Herzlichſt Ihr\spacefill\mbox{A.}\pend{}
\pstart
           \noindent{}{\pb}\textsc{Gerty\pwindex{Hofmannsthal, Gertrude von 16.\,3.\,1880 Wien – 9.\,11.\,1959 Paddington@\textsc{Hofmannsthal, Gertrude von} (16.\,3.\,1880 Wien – 9.\,11.\,1959 Paddington)|pw}} wird wohl auch am liebſten mit dem Zehn Uhr Früh Zug fahren?{\\}A.\pend
           
\pstart
           Geſtern Abend waren wir mit Bahr\pwindex{Bahr, Hermann 19.\,7.\,1863 Linz – 15.\,1.\,1934 München@\textsc{Bahr, Hermann} (19.\,7.\,1863 Linz – 15.\,1.\,1934 München), \emph{Schriftsteller, Kritiker}|pw}, (Hietzing\oindex{XIII., Hietzing@\textbf{XIII., Hietzing}, \emph{Verwaltungsgebiet}|pw}) dem’s recht gut, und was das
                  weſentlichſte iſt, hoffnungsvoll zu gehen{ }ſcheint.\pend
           \selectlanguage{ngerman}\endnumbering\briefempfaengerindex{Hofmannsthal, Hugo von@\textsc{Hofmannsthal, Hugo von}!zzzSchnitzler, Arthur@\emph{von Arthur Schnitzler}!1904-08-201@{20. 8. 1904}|)be}\mylabel{L01430h}  \newcommand{\dateiname}{L01430}\newcommand{\titel}{Arthur Schnitzler an Hugo von Hofmannsthal, 20. 8. 1904}\newcommand{\editorInnen}{Herausgegeben von Martin Anton Müller}%% latex-leseansicht-abspann.tex
%% Abspann für die Leseansicht.
%% Der Schalter \ifkorrekturansicht ist bereits durch den Vorspann gesetzt.

%% latex-abspann.tex
%% Gemeinsamer Abspann für Korrekturansicht und Leseansicht.
%% Setzt den Schalter \ifkorrekturansicht voraus (gesetzt in den
%% einbindenden Dateien latex-korrekturansicht-abspann.tex bzw.
%% latex-leseansicht-abspann.tex).
%% ---------------------------------------------------------------

\normalsize

% Das esempio-Environment wird nur in der Leseansicht benötigt
\ifkorrekturansicht\else
\newenvironment{esempio}[3]%
{
    \vspace{1.5ex}
    \rlap{\underline{#1}}
    \par
    \setlength{\parindent}{0cm}
    \nopagebreak
    \leftskip=#2cm
    \rightskip=#3cm
}
{
    \par
}
\fi

\doendnotes{C}
\bigskip
\vfill

\clearpage

\footnotesize

\ifkorrekturansicht
  \lohead{\textsc{register}}
\fi

% theindex-Environment neu definieren ohne reledmac
\makeatletter
\renewenvironment{theindex}{%
  \ifkorrekturansicht
    \section*{\indexname}%
  \else
    \subsubsection*{Index der erwähnten Entitäten}%
  \fi
  \setlength{\parindent}{0pt}%
  \setlength{\parskip}{0pt plus 0.3pt}%
  \let\item\@idxitem
}{%
  \ifkorrekturansicht\clearpage\fi
}
\makeatother

\IfFileExists{\jobname-pw.ind}{\input{\jobname-pw.ind}}{}

% Quellenangabe nur in der Leseansicht
\ifkorrekturansicht\else
% Fallback-Definitionen, falls die .tex-Datei \titel etc. nicht gesetzt hat
\providecommand{\titel}{}
\providecommand{\editorInnen}{}
\providecommand{\dateiname}{\jobname}

\vspace{3cm}

\vfill

\footnotesize
\textsc{Quelle}: \titel. Herausgegeben von {\editorInnen}. In: \emph{Arthur Schnitzler: Briefwechsel mit Autorinnen und Autoren}.
 Digitale Edition, https://schnitzler-briefe.acdh.oeaw.ac.at/{\dateiname}.html (Stand \today)
\fi

\end{document}


