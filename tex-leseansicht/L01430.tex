%% latex-leseansicht-vorspann.tex
%% Vorspann für die Leseansicht.
%% Lädt die gemeinsame Datei latex-vorspann.tex mit nicht gesetztem Schalter.

\newif\ifkorrekturansicht
\korrekturansichtfalse

\input{../tex-inputs/latex-vorspann}


               \section[Arthur Schnitzler an Hugo von Hofmannsthal, 20. 8. 1904]{ Arthur Schnitzler an Hugo von Hofmannsthal, 20. 8. 1904}\nopagebreak\mylabel{v}\rehead{ }\begin{ledgroupsized}[t]{13cm}\normalsize\beginnumbering\briefempfaengerindex{Hofmannsthal, Hugo von@\textsc{Hofmannsthal, Hugo von}!zzzSchnitzler, Arthur@\emph{von Arthur Schnitzler}!1904-08-201@{20. 8. 1904}|(be} \toendnotes[C]{\smallbreak\pagebreak[2]} \Standort{FDH, Hs-30885,112.}
\physDesc{Brief, 2 Blätter, 6 Seiten
\newline{}Handschrift: schwarze Tinte, deutsche Kurrent\newline{}Ordnung: mit Bleistift von Schnitzler mutmaßlich bei der Durchsicht der
                                 Korrespondenz 1929 das zweite Blatt beschrieben:
                                    »II 20/8 904« }\buchAbdrucke{\weitereDrucke{1) Hugo von Hofmannsthal, Arthur Schnitzler: \emph{Briefwechsel}. Hg. Therese Nickl und Heinrich Schnitzler. Frankfurt am Main: \emph{S. Fischer} 1964, S. 197–199.} \weitereDrucke{2) Hermann Bahr, Arthur Schnitzler: \emph{Briefwechsel, Aufzeichnungen, Dokumente (1891–1931)}. Hg. Kurt Ifkovits und Martin Anton Müller. Göttingen: \emph{Wallstein} 2018, S. 316.} }\toendnotes[C]{\smallbreak}\pstart
           \raggedleft{}{\pb}Wien\oindex{Wien@\textbf{Wien}|pw}{ }20. 8. 904\pend
           \pstart
           lieber Hugo, mit der Salzk.gut\oindex{Salzkammergut@\textbf{Salzkammergut}|pw}reiſe
               ſteht es wie folgt: in dieſen Tagen beende ich die erſte flüchtige Niederſchrift
               eines neuen dreiaktigen Stücks\pwindex{Schnitzler, Arthur 15.05.1862 – 21.10.1931@\textsc{Schnitzler, Arthur} (15.05.1862 – 21.10.1931), \emph{Schriftsteller, Mediziner}!Zwischenspiel. Komoedie in drei Akten1905-10-12 – 1905-10-12@\strich\emph{Zwischenspiel. Komödie in drei Akten} {[}1905-10-12 – 1905-10-12{]}|pwv};
               die Grünwald\pwindex{Gruenwald, Ida 28.06.1873 – Mai 1908@\textsc{Grünwald, Ida} (28.06.1873 – Mai 1908), \emph{Stenotypistin}|pw} ko{\geminationm}t
               etwa 25., 26., und dann muſs ich es, um es überſichtlich
               vor mir zu haben, und weil das überhaupt zu den Etappen meiner Arbeitsweiſe gehört u
               mich ſehr fördert, dictiren. Nun ka{\geminationn} ich, auch weil der
               Anfangstag der Grünwald\pwindex{Gruenwald, Ida 28.06.1873 – Mai 1908@\textsc{Grünwald, Ida} (28.06.1873 – Mai 1908), \emph{Stenotypistin}|pw}{ }\substVorne{}\textsuperscript{\textcolor{gray}{sich}}\substDazwischen{}noch nicht feſtſteht\substHinten{} (ich bin ohne Nachricht, \textsc{resp} Antwort von ihr),
               nicht {\pb}auf den Tag beſtimmen, wann ich fertig bin. Ich
                  \uline{hoffe}, es wird ſich fügen, daſs wir schon am
                  3.{ }Wien\oindex{Wien@\textbf{Wien}|pw} verlaſſen können; wird aber \textsc{Gerty\pwindex{Hofmannsthal, Gertrude von 16.03.1880 – 09.11.1959@\textsc{Hofmannsthal, Gertrude von} (16.03.1880 – 09.11.1959)|pw}} auch warten, wenn der 4. oder gar der 5. September
               draus wird? Wir möchten natürlich auch ſehr gern mit ihr zuſammen fahren; ich ka{\geminationn} nur heute mich zur Beſti{\geminationm}ung des Tages nicht verpflichten. Immerhin werde ich am erſten Dictirtag ſchon
               wiſſen können, wa{\geminationn} wir bereit ſind. Ich hoffe ja ſehr,
               daſs es der 3.{ }ſein wird. Sie erſehen daraus {\pb}jedenfalls, daſs wir zu Iſchl\oindex{Goldenes Kreuz@\textbf{Goldenes Kreuz}|pw} entſchloſſen iſt, wo wir fürs erſte Quartier nehmen, Ausflüge machen
                  (Olga\pwindex{Schnitzler, Olga 17.01.1882 – 13.01.1970@\textsc{Schnitzler, Olga} (17.01.1882 – 13.01.1970), \emph{Schauspielerin, Sängerin}|pw} kennt das Salzka{\geminationm}ergut\oindex{Salzkammergut@\textbf{Salzkammergut}|pw} gar nicht), und ich ſehne mich
               auch ſehr nach ein paar ſchönen Radtouren mit Ihnen. Auch zu einer Fußpartie
               (Ruckſack!) wär ich zu haben. Nicht unmöglich iſt es, daſs ich da{\geminationn} auch noch mit Olga\pwindex{Schnitzler, Olga 17.01.1882 – 13.01.1970@\textsc{Schnitzler, Olga} (17.01.1882 – 13.01.1970), \emph{Schauspielerin, Sängerin}|pw}
               weiterfahre, Tirol\oindex{Tirol@\textbf{Tirol}|pw}, Bozner\oindex{Bozen@\textbf{Bozen}|pw} Gegend, und falls das Wetter allzu herbſtlich wird, München\oindex{Muenchen@\textbf{München}|pw}. Wir ſehen uns ja jedenfalls ſchon am ersten {\pb}Iſchl\oindex{Bad Ischl@\textbf{Bad Ischl}|pw}er Tag, aber ſagen Sie mir doch gleich, wa{\geminationn}{ }Sie wieder in Rodaun\oindex{Rodaun@\textbf{Rodaun}|pw} zurück ſein müſſen oder wollen. Wohnen wollen wir in der Kaiſerkrone\oindex{Hotel Kaiserkrone@\textbf{Hotel Kaiserkrone}|pw}. –\pend
           \pstart
           Sind Sie mit dem »geretteten\pwindex{Hofmannsthal, Hugo von 01.02.1874 – 15.07.1929@\textsc{Hofmannsthal, Hugo von} (01.02.1874 – 15.07.1929), \emph{Schriftsteller}!gerettete Venedig. Trauerspiel in fuenf Aufzuegen1905@\strich\emph{Das gerettete Venedig. Trauerspiel in fünf Aufzügen} {[}1905{]}|pw}« fertig? Mir geht es
               mit dem Arbeiten nicht übel und ginge mir gewiſs noch beſſer, we{\geminationn} nicht mein Widerwillen gegen den phyſ. Akt des
               Schreibens immer beträchtlicher würde und ſich oft genug in leichten Schreibkrämpfen
               äußerte.\pend
           \pstart
           Danke ſehr betreffs V. S.\pwindex{Vansittart, Robert Gilbert 25.06.1881 – 14.02.1957@\textsc{Vansittart, Robert Gilbert} (25.06.1881 – 14.02.1957), \emph{Diplomat}|pw}, mein Aerger hat ſich
               natürlich ſchon gelegt – natürlich würde es mich aber {\pb}ſehr freuen, wenn Ordnung in die ganze Angelegenheit gebracht werden könnte und ich
               von England\oindex{England@\textbf{England}|pw}, Irland\oindex{Irland@\textbf{Irland}|pw}
               u Schottland\oindex{Schottland@\textbf{Schottland}|pw} nicht länger misverſtanden \introOben{}verfolgt u geächtet\introOben{} würde. –\pend
           \pstart
           – \textsc{Vehse\pwindex{\textcolor{red}{\textsuperscript{XXXX1 indx}}!Geschichte der deutschen Hoefe seit der Reformation1851 – 1858@\strich\emph{Geschichte der deutschen Höfe seit der Reformation} {[}1851 – 1858{]}|pwv}} iſt und bleibt ein koſtbares Buch. Zudem studier ich, des Überblickes halber,
               Geſchichte \introOben{}wie\introOben{} zur Matura. Ich wäre weiter als ich bin, we{\geminationn} ich ein gebildeter Menſch wäre!\pend
           \pstart
           Was iſts mit Richard\pwindex{Beer-Hofmann, Richard 11.07.1866 – 26.09.1945@\textsc{Beer-Hofmann, Richard} (11.07.1866 – 26.09.1945), \emph{Schriftsteller}|pw}? Seine Karte mit Paula\pwindex{Beer-Hofmann, Paula 25.02.1879 – 30.10.1939@\textsc{Beer-Hofmann, Paula} (25.02.1879 – 30.10.1939)|pw}{ }\textcolor{gray}{wie} den \textcolor{gray}{Kindern\pwindex{Beer-Hofmann, Naemah 20.12.1898 – 10.11.1971@\textsc{Beer-Hofmann, Naëmah} (20.12.1898 – 10.11.1971)|pwv}\pwindex{Beer-Hofmann, Mirjam 04.09.1897 – 24.12.1984@\textsc{Beer-Hofmann, Mirjam} (04.09.1897 – 24.12.1984)|pwv}\pwindex{Beer-Hofmann, Gabriel 09.01.1901 – 24.03.1971@\textsc{Beer-Hofmann, Gabriel} (09.01.1901 – 24.03.1971), \emph{Schriftsteller, Filmagent}|pwv}}{ }\textcolor{gray}{an}{ }\textcolor{gray}{×}\-\textcolor{gray}{×}\-\textcolor{gray}{×} hab ich bekommen. Von ſich
               ſchreibt er nichts. Grüßen Sie alle, die mir lieb ſind.\pend
           \pstart Herzlichſt Ihr\spacefill\mbox{A.{\pb}}\pend{}\pstart
           \noindent{}{\pb}\textsc{Gerty\pwindex{Hofmannsthal, Gertrude von 16.03.1880 – 09.11.1959@\textsc{Hofmannsthal, Gertrude von} (16.03.1880 – 09.11.1959)|pw}} wird wohl auch am liebſten mit dem Zehn Uhr Früh Zug fahren?{\\}A.\pend
           \pstart
           Geſtern Abend waren wir mit Bahr\pwindex{Bahr, Hermann 19.07.1863 – 15.01.1934@\textsc{Bahr, Hermann} (19.07.1863 – 15.01.1934), \emph{Schriftsteller, Kritiker}|pw}, (Hietzing\oindex{XIII., Hietzing@\textbf{XIII., Hietzing}|pw}) dem’s recht gut, und was das
                  weſentlichſte iſt, hoffnungsvoll zu gehen ſcheint.\pend
           \endnumbering\briefempfaengerindex{Hofmannsthal, Hugo von@\textsc{Hofmannsthal, Hugo von}!zzzSchnitzler, Arthur@\emph{von Arthur Schnitzler}!1904-08-201@{20. 8. 1904}|)be}\mylabel{h}\end{ledgroupsized}  \newcommand{\dateiname}{L01430}\newcommand{\titel}{Arthur Schnitzler an Hugo von Hofmannsthal, 20. 8. 1904}\newcommand{\editorInnen}{ Martin Anton Müller und Gerd-Hermann Susen}%% latex-leseansicht-abspann.tex
%% Abspann für die Leseansicht.
%% Der Schalter \ifkorrekturansicht ist bereits durch den Vorspann gesetzt.

%% latex-abspann.tex
%% Gemeinsamer Abspann für Korrekturansicht und Leseansicht.
%% Setzt den Schalter \ifkorrekturansicht voraus (gesetzt in den
%% einbindenden Dateien latex-korrekturansicht-abspann.tex bzw.
%% latex-leseansicht-abspann.tex).
%% ---------------------------------------------------------------

\normalsize

% Das esempio-Environment wird nur in der Leseansicht benötigt
\ifkorrekturansicht\else
\newenvironment{esempio}[3]%
{
    \vspace{1.5ex}
    \rlap{\underline{#1}}
    \par
    \setlength{\parindent}{0cm}
    \nopagebreak
    \leftskip=#2cm
    \rightskip=#3cm
}
{
    \par
}
\fi

\doendnotes{C}
\bigskip
\vfill

\clearpage

\footnotesize

\ifkorrekturansicht
  \lohead{\textsc{register}}
\fi

% theindex-Environment neu definieren ohne reledmac
\makeatletter
\renewenvironment{theindex}{%
  \ifkorrekturansicht
    \section*{\indexname}%
  \else
    \subsubsection*{Index der erwähnten Entitäten}%
  \fi
  \setlength{\parindent}{0pt}%
  \setlength{\parskip}{0pt plus 0.3pt}%
  \let\item\@idxitem
}{%
  \ifkorrekturansicht\clearpage\fi
}
\makeatother

\IfFileExists{\jobname-pw.ind}{\input{\jobname-pw.ind}}{}

% Quellenangabe nur in der Leseansicht
\ifkorrekturansicht\else
% Fallback-Definitionen, falls die .tex-Datei \titel etc. nicht gesetzt hat
\providecommand{\titel}{}
\providecommand{\editorInnen}{}
\providecommand{\dateiname}{\jobname}

\vspace{3cm}

\vfill

\footnotesize
\textsc{Quelle}: \titel. Herausgegeben von {\editorInnen}. In: \emph{Arthur Schnitzler: Briefwechsel mit Autorinnen und Autoren}.
 Digitale Edition, https://schnitzler-briefe.acdh.oeaw.ac.at/{\dateiname}.html (Stand \today)
\fi

\end{document}


      