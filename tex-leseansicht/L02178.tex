%% latex-leseansicht-vorspann.tex
%% Vorspann für die Leseansicht.
%% Lädt die gemeinsame Datei latex-vorspann.tex mit nicht gesetztem Schalter.

\newif\ifkorrekturansicht
\korrekturansichtfalse

\input{../tex-inputs/latex-vorspann}


         
         \renewcommand{\erwaehntePersonen}{Personen: Richard Beer-Hofmann, Paula Beer-Hofmann, Olga Schnitzler}
         \renewcommand{\erwaehnteOrte}{Orte: Algiers, Bahnhof Genua, Florenz, Genua, Hasenauerstraße, Piazza Raffaele de Ferrari, Wien, XVIII., Währing}
         \renewcommand{\erwaehnteWerke}{}
               \section[Arthur und Olga Schnitzler an Richard und Paula Beer-Hofmann, 1{[}2?{]}. 5. 1914]{ Arthur und Olga Schnitzler an Richard und Paula Beer-Hofmann,
               1{[}2?{]}. 5. 1914}\nopagebreak\mylabel{v}\rehead{ }\begin{ledgroupsized}[t]{13cm}\normalsize\beginnumbering\briefempfaengerindex{Beer-Hofmann, Paula@\textsc{Beer-Hofmann, Paula}!zzzSchnitzler, Olga@\emph{von Olga Schnitzler}!1914-05-121@{1{[}2?{]}. 5. 1914}|(be}\briefempfaengerindex{Beer-Hofmann, Paula@\textsc{Beer-Hofmann, Paula}!zzzSchnitzler, Arthur@\emph{von Arthur Schnitzler}!1914-05-121@{1{[}2?{]}. 5. 1914}|(be}\briefempfaengerindex{Beer-Hofmann, Richard@\textsc{Beer-Hofmann, Richard}!zzzSchnitzler, Olga@\emph{von Olga Schnitzler}!1914-05-121@{1{[}2?{]}. 5. 1914}|(be}\briefempfaengerindex{Beer-Hofmann, Richard@\textsc{Beer-Hofmann, Richard}!zzzSchnitzler, Arthur@\emph{von Arthur Schnitzler}!1914-05-121@{1{[}2?{]}. 5. 1914}|(be} \toendnotes[C]{\smallbreak\pagebreak[2]} \Standort{YCGL, MSS 31.}
\physDesc{Bildpostkarte, 218 Zeichen
\newline{}Handschrift Arthur Schnitzler: Bleistift, deutsche Kurrent\newline{}Handschrift Olga Schnitzler: Bleistift, lateinische Kurrent
\newline{}Versand: 1) Stempel: »\nobreak{}\oindex{Genua@\textbf{Genua}|pwk}Genova 1914, Esposizione internazionale Igiene – Marina –
                                       Colonie\nobreak{}«.   2) Stempel: »\nobreak{}\oindex{Bahnhof Genua@\textbf{Bahnhof Genua}|pwk}Genova Ferrovia, 13. V. 14, 16\nobreak{}«.  3) mit blauem Buntstift von unbekannter Hand der Postrayon zur
                                 Bezirksangabe in der Adressierung ergänzt:
                                 »/2«}\buchAbdrucke{\weitereDrucke{Arthur Schnitzler, Richard Beer-Hofmann: \emph{Briefwechsel 1891–1931}. Hg. Konstanze Fliedl. Wien, Zürich: \emph{Europaverlag} 1992, S. 219.} }\toendnotes[C]{\smallbreak}\pstart{}{\pb}Hrn \textsc{Dr. Richard Beer
                     Hofmann}\pend{}\pstart{}und Frau.\pend{}\pstart{}\textsc{Wien XVIII\oindex{XVIII., Waehring@\textbf{XVIII., Währing}|pw}}\pend{}\pstart{}\textsc{Hasenauerstr 59}\oindex{Hasenauerstrasse@\textbf{Hasenauerstraße}|pw}\pend{}{\bigskip}\pstart
           \noindent{}\centering{}{\pb}\textcolor{gray}{\textbf{Genova – Piazza De Ferrari\oindex{Piazza Raffaele de Ferrari@\textbf{Piazza Raffaele de Ferrari}|pw}}}\pend
           \pstart
           {\pb}Herzliche Grüße!\pend
           \pstart Ihr \spacefill\mbox{Arthur}\pend{}\pstart
           \noindent{}{[}hs. Olga Schnitzler:{]} Haben in Florenz\oindex{Florenz@\textbf{Florenz}|pw} in
               einem \label{K_L02178-1v}\edtext{Varieté}{\lemma{\textnormal{\emph{Varieté}}}\Cendnote{\textnormal{Vgl. A. S.: \emph{Tagebuch}, 9. 5. 1914.
               }}}\label{K_L02178-1h} besonders heftig Ihrer gedacht. – was Sie nun neugierig machen möge!\pend
           \pstart Herzlichst \spacefill\mbox{Olga.}\pend{}\pstart
           \noindent{}{[}hs. Arthur Schnitzler:{]} \label{K_L02178-2v}\edtext{Morgen}{\lemma{\textnormal{\emph{Morgen}}}\Cendnote{\textnormal{Das erlaubt, die Karte am Tag
                  vor dem Poststempel zu datieren, da die Abfahrt am 13. 5. 1914 stattfand.}}}\label{K_L02178-2h} ab nach Algier\oindex{Algiers@\textbf{Algiers}|pw}\pend
           
         
         \endnumbering\mylabel{h}\end{ledgroupsized}  \newcommand{\dateiname}{L02178}\newcommand{\titel}{Arthur und Olga Schnitzler an Richard und Paula Beer-Hofmann, 1[2?]. 5. 1914}\newcommand{\editorInnen}{Martin Anton Müller und Gerd-Hermann Susen}%% latex-leseansicht-abspann.tex
%% Abspann für die Leseansicht.
%% Der Schalter \ifkorrekturansicht ist bereits durch den Vorspann gesetzt.

%% latex-abspann.tex
%% Gemeinsamer Abspann für Korrekturansicht und Leseansicht.
%% Setzt den Schalter \ifkorrekturansicht voraus (gesetzt in den
%% einbindenden Dateien latex-korrekturansicht-abspann.tex bzw.
%% latex-leseansicht-abspann.tex).
%% ---------------------------------------------------------------

\normalsize

% Das esempio-Environment wird nur in der Leseansicht benötigt
\ifkorrekturansicht\else
\newenvironment{esempio}[3]%
{
    \vspace{1.5ex}
    \rlap{\underline{#1}}
    \par
    \setlength{\parindent}{0cm}
    \nopagebreak
    \leftskip=#2cm
    \rightskip=#3cm
}
{
    \par
}
\fi

\doendnotes{C}
\bigskip
\vfill

\clearpage

\footnotesize

\ifkorrekturansicht
  \lohead{\textsc{register}}
\fi

% theindex-Environment neu definieren ohne reledmac
\makeatletter
\renewenvironment{theindex}{%
  \ifkorrekturansicht
    \section*{\indexname}%
  \else
    \subsubsection*{Index der erwähnten Entitäten}%
  \fi
  \setlength{\parindent}{0pt}%
  \setlength{\parskip}{0pt plus 0.3pt}%
  \let\item\@idxitem
}{%
  \ifkorrekturansicht\clearpage\fi
}
\makeatother

\IfFileExists{\jobname-pw.ind}{\input{\jobname-pw.ind}}{}

% Quellenangabe nur in der Leseansicht
\ifkorrekturansicht\else
% Fallback-Definitionen, falls die .tex-Datei \titel etc. nicht gesetzt hat
\providecommand{\titel}{}
\providecommand{\editorInnen}{}
\providecommand{\dateiname}{\jobname}

\vspace{3cm}

\vfill

\footnotesize
\textsc{Quelle}: \titel. Herausgegeben von {\editorInnen}. In: \emph{Arthur Schnitzler: Briefwechsel mit Autorinnen und Autoren}.
 Digitale Edition, https://schnitzler-briefe.acdh.oeaw.ac.at/{\dateiname}.html (Stand \today)
\fi

\end{document}


      