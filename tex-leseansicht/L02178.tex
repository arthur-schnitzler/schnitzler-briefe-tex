%% latex-korrekturansicht-vorspann.tex
%% Vorspann für die Korrekturansicht.
%% Lädt die gemeinsame Datei latex-vorspann.tex mit gesetztem Schalter.

\newif\ifkorrekturansicht
\korrekturansichttrue

\input{../tex-inputs/latex-vorspann}


\section[Arthur und Olga Schnitzler an Richard und Paula Beer-Hofmann, 1{[}2?{]}. 5. 1914]{L02178 Arthur und Olga Schnitzler an Richard und Paula Beer-Hofmann,
               1{[}2?{]}. 5. 1914}
\nopagebreak\mylabel{L02178v}
\rehead{ }\normalsize\beginnumbering\briefempfaengerindex{Beer-Hofmann, Paula@\textsc{Beer-Hofmann, Paula}!zzzSchnitzler, Olga@\emph{von Olga Schnitzler}!1914-05-121@{1{[}2?{]}. 5. 1914}|(be}\briefempfaengerindex{Beer-Hofmann, Paula@\textsc{Beer-Hofmann, Paula}!zzzSchnitzler, Arthur@\emph{von Arthur Schnitzler}!1914-05-121@{1{[}2?{]}. 5. 1914}|(be}\briefempfaengerindex{Beer-Hofmann, Richard@\textsc{Beer-Hofmann, Richard}!zzzSchnitzler, Olga@\emph{von Olga Schnitzler}!1914-05-121@{1{[}2?{]}. 5. 1914}|(be}\briefempfaengerindex{Beer-Hofmann, Richard@\textsc{Beer-Hofmann, Richard}!zzzSchnitzler, Arthur@\emph{von Arthur Schnitzler}!1914-05-121@{1{[}2?{]}. 5. 1914}|(be}
\toendnotes[C]{\smallbreak\pagebreak[2]}\Standort{YCGL, MSS 31.}
\physDesc{Bildpostkarte, 218 Zeichen
\newline{}Handschrift Arthur Schnitzler: Bleistift, deutsche Kurrent
\newline{}Handschrift Olga Schnitzler: Bleistift, lateinische Kurrent
\newline{}Versand: 1) Stempel: »\nobreak{}\oindex{Genua@\textbf{Genua}, \emph{P.PPLA}|pwk}Genova 1914, Esposizione internazionale Igiene – Marina –
                                       Colonie\nobreak{}«.   2) Stempel: »\nobreak{}\oindex{Bahnhof Genua@\textbf{Bahnhof Genua}, \emph{Bahnhofsgebäude (K.BHF)}|pwk}Genova Ferrovia, 13. V. 14, 16\nobreak{}«.  3) mit blauem Buntstift von unbekannter Hand der Postrayon zur
                                 Bezirksangabe in der Adressierung ergänzt:
                                 »/2«}
\buchAbdrucke{\weitereDrucke{Arthur Schnitzler, Richard Beer-Hofmann: \emph{Briefwechsel 1891–1931}. Wien, Zürich: \emph{Europaverlag} 1992, S. 219.} }\toendnotes[C]{\smallbreak}\pstart{}{\pb}Hrn \textsc{Dr. Richard Beer
                     Hofmann}\pend{}\pstart{}und Frau.\pend{}\pstart{}\textsc{Wien XVIII\oindex{XVIII., Waehring@\textbf{XVIII., Währing}, \emph{A.ADM3}|pw}}\pend{}\pstart{}\textsc{Hasenauerstr 59}\oindex{Hasenauerstrasse 59@\textbf{Hasenauerstraße 59}, \emph{Wohngebäude (K.WHS)}|pw}\pend{}{\bigskip}
\pstart
           \noindent{}\centering{}{\pb}\textcolor{gray}{\textbf{Genova – Piazza De Ferrari\oindex{Piazza Raffaele de Ferrari@\textbf{Piazza Raffaele de Ferrari}, \emph{Platz (K.PLT)}|pw}}}\pend
           \vspace{1em}
\pstart
           \noindent{}{\pb}Herzliche Grüße!\pend
           \pstart Ihr \spacefill\mbox{Arthur}\pend{}\selectlanguage{ngerman}\vspace{1em}
\pstart
           \noindent{}{[}hs. :{]} Haben in Florenz\oindex{Florenz@\textbf{Florenz}, \emph{P.PPLA}|pw} in
               einem \label{K_L02178-1v}\edtext{Varieté}{\lemma{\textnormal{\emph{Varieté}}}\Cendnote{\textnormal{Vgl. A. S.: \emph{Tagebuch}, 9. 5. 1914.
               }}}\label{K_L02178-1} besonders heftig Ihrer gedacht. – was Sie nun neugierig machen möge!\pend
           \pstart Herzlichst \spacefill\mbox{Olga.}\pend{}\selectlanguage{ngerman}\vspace{1em}
\pstart
           \noindent{}{[}hs. :{]} \label{K_L02178-2v}\edtext{Morgen}{\lemma{\textnormal{\emph{Morgen}}}\Cendnote{\textnormal{Das erlaubt, die Karte am Tag
                  vor dem Poststempel zu datieren, da die Abfahrt am 13. 5. 1914 stattfand.}}}\label{K_L02178-2} ab nach Algier\oindex{Algier@\textbf{Algier}, \emph{P.PPLC}|pw}\pend
           \selectlanguage{ngerman}\endnumbering\briefempfaengerindex{Beer-Hofmann, Paula@\textsc{Beer-Hofmann, Paula}!zzzSchnitzler, Olga@\emph{von Olga Schnitzler}!1914-05-121@{1{[}2?{]}. 5. 1914}|)be}\briefempfaengerindex{Beer-Hofmann, Paula@\textsc{Beer-Hofmann, Paula}!zzzSchnitzler, Arthur@\emph{von Arthur Schnitzler}!1914-05-121@{1{[}2?{]}. 5. 1914}|)be}\briefempfaengerindex{Beer-Hofmann, Richard@\textsc{Beer-Hofmann, Richard}!zzzSchnitzler, Olga@\emph{von Olga Schnitzler}!1914-05-121@{1{[}2?{]}. 5. 1914}|)be}\briefempfaengerindex{Beer-Hofmann, Richard@\textsc{Beer-Hofmann, Richard}!zzzSchnitzler, Arthur@\emph{von Arthur Schnitzler}!1914-05-121@{1{[}2?{]}. 5. 1914}|)be}\mylabel{L02178h}  \normalsize

\doendnotes{C}
\bigskip
\vfill

\clearpage

\footnotesize

\lohead{\textsc{register}}

% Definiere theindex-Environment komplett neu ohne reledmac
\makeatletter
\renewenvironment{theindex}{%
  \section*{\indexname}%
  \setlength{\parindent}{0pt}%
  \setlength{\parskip}{0pt plus 0.3pt}%
  \let\item\@idxitem
}{%
  \clearpage
}
\makeatother

\IfFileExists{\jobname-pw.ind}{\input{\jobname-pw.ind}}{}

\end{document}

      