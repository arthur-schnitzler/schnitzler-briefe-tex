%% latex-korrekturansicht-vorspann.tex
%% Vorspann für die Korrekturansicht.
%% Lädt die gemeinsame Datei latex-vorspann.tex mit gesetztem Schalter.

\newif\ifkorrekturansicht
\korrekturansichttrue

\input{../tex-inputs/latex-vorspann}


\section[Robert Adam an Arthur Schnitzler, 29. 10. 1910]{L01972 Robert Adam an Arthur Schnitzler, 29. 10. 1910}
\nopagebreak\mylabel{L01972v}
\rehead{ }\normalsize\beginnumbering\briefempfaengerindex{Schnitzler, Arthur@\textsc{Schnitzler, Arthur}!zzzAdam, Robert@\emph{von Robert Adam}!1910-10-291@{29. 10. 1910}|(be}
\toendnotes[C]{\smallbreak\pagebreak[2]}\Standort{DLA, A:Schnitzler, HS.NZ85.1.4230,2.}
\physDesc{Brief, 1 Blatt, 2 Seiten, 1003 Zeichen
\newline{}Handschrift: schwarze Tinte, deutsche Kurrent
\newline{}Schnitzler: 1) mit Bleistift beschriftet: »\textsc{Adam}«  2) mit rotem Buntstift eine Unterstreichung}\toendnotes[C]{\smallbreak}
\pstart
           \raggedleft{}{\pb}Wien\oindex{Wien@\textbf{Wien}, \emph{A.ADM2}|pw}, am 29. Oktober 1910\pend
           
\pstart{}Hochverehrter Herr Doktor!\pend\vspace{0.5em}
\pstart
           Die wohlwollenden Zeilen, die Sie mir im vorigen Jahre anläßlich der Überſendung
               meiner Komödie: »Die Geſchichte Alî ibn Bekkârs mit
                  Schams an-Nahâr\pwindex{Geschichte des Alî ibn Bekkâr mit Schams an-Nahâr@\emph{Die Geschichte des Alî ibn Bekkâr mit Schams an-Nahâr}|pw}« ſandten, geben mir den Mut, mit einer Bitte an Sie
               heranzutreten.\pend
           
\pstart
           Ich habe eine neue Komödie zum Abſchluſſe gebracht, die den Titel \textsc{Neidhard}\pwindex{Neidhard@\emph{Neidhard}|pw} führt, und möchte gerne, bevor ich mit ihr in die Öffentlichkeit trete, {\pb}Ihren Rat, hochverehrter Herr Doktor, einholen,
               welcher Weg wohl einzuſchlagen wäre, um dieſer Komödie, an der ich ſehr lang mit
               ganzem Herzen arbeitete und die ich ſelbſt für reifer und intereſſanter halte als die
               Ihnen bekannte arabiſche\pwindex{Geschichte des Alî ibn Bekkâr mit Schams an-Nahâr@\emph{Die Geschichte des Alî ibn Bekkâr mit Schams an-Nahâr}|pwv}, mehr
               Publizität zu ſichern, als jener zuteil geworden iſt.\pend
           
\pstart
           Sollten Sie die Güte haben, einem ratloſen Poeten freundlich beizuſtehen, ſo bitte
               ich um kurze Nachricht, wann ich bei Ihnen vorſprechen könnte.\pend
           
\pstart
           Seien Sie, hochverehrter Herr Doktor, meiner Dankbarkeit und unbegrenzten
               Hochſchätzung gewiß!\pend
           
\pstart
           Ihr ergebener{\\[\baselineskip]}\spacefill\mbox{Robert Adam}\pend
           \leftskip=0em{}
\pstart
           \noindent{}Wien XII/\textsubscript{1} Meidlinger
                     Hauptſtr. 56\oindex{Meidlinger Hauptstrasse@\textbf{Meidlinger Hauptstraße}, \emph{Straße (K.STR)}|pw}\pend
           \selectlanguage{ngerman}\endnumbering\briefempfaengerindex{Schnitzler, Arthur@\textsc{Schnitzler, Arthur}!zzzAdam, Robert@\emph{von Robert Adam}!1910-10-291@{29. 10. 1910}|)be}\mylabel{L01972h}  \normalsize

\doendnotes{C}
\bigskip
\vfill

\clearpage

\footnotesize

\lohead{\textsc{register}}

% Definiere theindex-Environment komplett neu ohne reledmac
\makeatletter
\renewenvironment{theindex}{%
  \section*{\indexname}%
  \setlength{\parindent}{0pt}%
  \setlength{\parskip}{0pt plus 0.3pt}%
  \let\item\@idxitem
}{%
  \clearpage
}
\makeatother

\IfFileExists{\jobname-pw.ind}{\input{\jobname-pw.ind}}{}

\end{document}

      