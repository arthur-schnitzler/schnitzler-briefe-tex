%% latex-leseansicht-vorspann.tex
%% Vorspann für die Leseansicht.
%% Lädt die gemeinsame Datei latex-vorspann.tex mit nicht gesetztem Schalter.

\newif\ifkorrekturansicht
\korrekturansichtfalse

\input{../tex-inputs/latex-vorspann}


         
         \renewcommand{\erwaehntePersonen}{Personen: Richard Beer-Hofmann, Efraim Frisch, Ludwig Hartau, Friedrich Kayssler, Max Reinhardt}
         \renewcommand{\erwaehnteOrte}{Orte: Liesingerstraße, Reichenau an der Rax, Rodaun, Wien, XVIII., Währing, XXIII., Liesing}
         \renewcommand{\erwaehnteWerke}{Werke: Das Verlöbnis. Geschichte eines Knaben, Der Graf von Charolais. Ein Trauerspiel}
               \section[Arthur Schnitzler an Richard Beer-Hofmann, 23. 5. 1905]{ Arthur Schnitzler an Richard Beer-Hofmann, 23. 5. 1905}\nopagebreak\mylabel{v}\rehead{ }\begin{ledgroupsized}[t]{13cm}\normalsize\beginnumbering \toendnotes[C]{\smallbreak\pagebreak[2]} \Standort{YCGL, MSS 31.}
\physDesc{Kartenbrief, 649 Zeichen
\newline{}Handschrift: schwarze Tinte, deutsche Kurrent
\newline{}Versand: 1) Stempel: »\nobreak{}\oindex{XVIII., Waehring@\textbf{XVIII., Währing}|pwk}18/1 Wien 110, 2\textcolor{gray}{3}. V. 05, X\nobreak{}«.   2) Stempel: »\nobreak{}\oindex{Rodaun@\textbf{Rodaun}|pwk}R{[}odaun{]}, 2\textcolor{gray}{3. 5. 05}, 2–4\textcolor{gray}{N}\nobreak{}«. }\buchAbdrucke{\weitereDrucke{Arthur Schnitzler, Richard Beer-Hofmann: \emph{Briefwechsel 1891–1931}. Hg. Konstanze Fliedl. Wien, Zürich: \emph{Europaverlag} 1992, S. 172.} }\toendnotes[C]{\smallbreak}\pstart{}{\pb}Herrn \textsc{Dr. Richard
                     Beer-Hofmann}\pend{}\pstart{}Rodaun\oindex{Rodaun@\textbf{Rodaun}|pw}\pend{}\pstart{}\textsc{bei Liesing\oindex{XXIII., Liesing@\textbf{XXIII., Liesing}|pw}}\pend{}\pstart{}\textsc{Liesingerstraße 1}\oindex{Liesingerstrasse@\textbf{Liesingerstraße}|pw}.\pend{}{\bigskip}\pstart
           \raggedleft{}{\pb}23. 5. 905\pend
           \pstart
           lieber Richard, ich beſtätige den unerwarteten Empfang des \textsc{Frisch}\pwindex{Frisch, Efraim 01.03.1873 – 26.11.1942@\textsc{Frisch, Efraim} (01.03.1873 – 26.11.1942), \emph{Schriftsteller, Publizist}|pw}ſchen Buches\pwindex{Frisch, Efraim 01.03.1873 – 26.11.1942@\textsc{Frisch, Efraim} (01.03.1873 – 26.11.1942), \emph{Schriftsteller, Publizist}!Verloebnis. Geschichte eines Knaben1901-11-01@\strich\emph{Das Verlöbnis. Geschichte eines Knaben} {[}1901-11-01{]}|pwuv}; – bedeutet das vielleicht den \substVorne{}\textsuperscript{Empfang}{\allowbreak}\substDazwischen{}Anfang\substHinten{} der Überſiedlung? Haben Sie den Grund ſchon gekauft? Könnte man ſich nicht
               wieder einmal, in Ruhe, ſehen? Sprechen? Ihre So{\geminationm}erpläne? Wir auf 3–4 Wochen Reichenau\oindex{Reichenau an der Rax@\textbf{Reichenau an der Rax}|pw}; mehr
               dürfte nicht herausko{\geminationm}en. –\pend
           \pstart
           – Zum \textsc{Charolais}\pwindex{Beer-Hofmann, Richard 1866-07-11 – 1945-09-26@\textsc{Beer-Hofmann, Richard} (1866-07-11 – 1945-09-26), \emph{Schriftsteller}!Graf von Charolais. Ein Trauerspiel1904-12-23@\strich\emph{Der Graf von Charolais. Ein Trauerspiel} {[}1904-12-23{]}|pw} (nicht gerade zur Aufführung, in der ich nur \textsc{Kayssler}\pwindex{Kayssler, Friedrich 07.04.1874 – 24.04.1945@\textsc{Kayssler, Friedrich} (07.04.1874 – 24.04.1945), \emph{Schauspieler}|pw} und \textsc{Reinhardt}\pwindex{Reinhardt, Max 09.09.1873 – 30.10.1943@\textsc{Reinhardt, Max} (09.09.1873 – 30.10.1943), \emph{Theaterleiter, Regisseur, Schauspieler}|pw} hervorragend fand, – zunächſt: \textsc{Hartau}\pwindex{Hartau, Ludwig 19.2.1877 – 24.11.1922@\textsc{Hartau, Ludwig} (19.2.1877 – 24.11.1922), \emph{Schauspieler}|pw}) ka{\geminationn} ich Sie immer wied\textcolor{gray}{er} nur
               beglückwünſchen. Gewiſſe Einwendungen bleiben beſtehen; meine Liebe zu dem Werk
               erhöht und vertieft sich.\pend
           \pstart
           Herzlichst Ihr{\\[\baselineskip]}\spacefill\mbox{A.}\pend
           \leftskip=0em{}
         
         \endnumbering\mylabel{h}\end{ledgroupsized}  \newcommand{\dateiname}{L01518}\newcommand{\titel}{Arthur Schnitzler an Richard Beer-Hofmann, 23. 5. 1905}\newcommand{\editorInnen}{Martin Anton Müller und Gerd-Hermann Susen}%% latex-leseansicht-abspann.tex
%% Abspann für die Leseansicht.
%% Der Schalter \ifkorrekturansicht ist bereits durch den Vorspann gesetzt.

%% latex-abspann.tex
%% Gemeinsamer Abspann für Korrekturansicht und Leseansicht.
%% Setzt den Schalter \ifkorrekturansicht voraus (gesetzt in den
%% einbindenden Dateien latex-korrekturansicht-abspann.tex bzw.
%% latex-leseansicht-abspann.tex).
%% ---------------------------------------------------------------

\normalsize

% Das esempio-Environment wird nur in der Leseansicht benötigt
\ifkorrekturansicht\else
\newenvironment{esempio}[3]%
{
    \vspace{1.5ex}
    \rlap{\underline{#1}}
    \par
    \setlength{\parindent}{0cm}
    \nopagebreak
    \leftskip=#2cm
    \rightskip=#3cm
}
{
    \par
}
\fi

\doendnotes{C}
\bigskip
\vfill

\clearpage

\footnotesize

\ifkorrekturansicht
  \lohead{\textsc{register}}
\fi

% theindex-Environment neu definieren ohne reledmac
\makeatletter
\renewenvironment{theindex}{%
  \ifkorrekturansicht
    \section*{\indexname}%
  \else
    \subsubsection*{Index der erwähnten Entitäten}%
  \fi
  \setlength{\parindent}{0pt}%
  \setlength{\parskip}{0pt plus 0.3pt}%
  \let\item\@idxitem
}{%
  \ifkorrekturansicht\clearpage\fi
}
\makeatother

\IfFileExists{\jobname-pw.ind}{\input{\jobname-pw.ind}}{}

% Quellenangabe nur in der Leseansicht
\ifkorrekturansicht\else
% Fallback-Definitionen, falls die .tex-Datei \titel etc. nicht gesetzt hat
\providecommand{\titel}{}
\providecommand{\editorInnen}{}
\providecommand{\dateiname}{\jobname}

\vspace{3cm}

\vfill

\footnotesize
\textsc{Quelle}: \titel. Herausgegeben von {\editorInnen}. In: \emph{Arthur Schnitzler: Briefwechsel mit Autorinnen und Autoren}.
 Digitale Edition, https://schnitzler-briefe.acdh.oeaw.ac.at/{\dateiname}.html (Stand \today)
\fi

\end{document}


      