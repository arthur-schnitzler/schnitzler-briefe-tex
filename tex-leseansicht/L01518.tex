%% latex-korrekturansicht-vorspann.tex
%% Vorspann für die Korrekturansicht.
%% Lädt die gemeinsame Datei latex-vorspann.tex mit gesetztem Schalter.

\newif\ifkorrekturansicht
\korrekturansichttrue

\input{../tex-inputs/latex-vorspann}


\section[Arthur Schnitzler an Richard Beer-Hofmann, 23. 5. 1905]{L01518 Arthur Schnitzler an Richard Beer-Hofmann, 23. 5. 1905}
\nopagebreak\mylabel{L01518v}
\rehead{ }\normalsize\beginnumbering\briefempfaengerindex{Beer-Hofmann, Richard@\textsc{Beer-Hofmann, Richard}!zzzSchnitzler, Arthur@\emph{von Arthur Schnitzler}!1905-05-231@{23. 5. 1905}|(be}
\toendnotes[C]{\smallbreak\pagebreak[2]}\Standort{YCGL, MSS 31.}
\physDesc{Kartenbrief, 649 Zeichen
\newline{}Handschrift: schwarze Tinte, deutsche Kurrent
\newline{}Versand: 1) Stempel: »\nobreak{}\oindex{XVIII., Waehring@\textbf{XVIII., Währing}, \emph{A.ADM3}|pwk}18/1 Wien 110, 2\textcolor{gray}{3}. V. 05, X\nobreak{}«.   2) Stempel: »\nobreak{}\oindex{Rodaun@\textbf{Rodaun}, \emph{A.ADM4}|pwk}R{[}odaun{]}, 2\textcolor{gray}{3. 5. 05}, 2–4\textcolor{gray}{N}\nobreak{}«. }
\buchAbdrucke{\weitereDrucke{Arthur Schnitzler, Richard Beer-Hofmann: \emph{Briefwechsel 1891–1931}. Wien, Zürich: \emph{Europaverlag} 1992, S. 172.} }\toendnotes[C]{\smallbreak}\pstart{}{\pb}Herrn \textsc{Dr. Richard
                     Beer-Hofmann}\pend{}\pstart{}Rodaun\oindex{Rodaun@\textbf{Rodaun}, \emph{A.ADM4}|pw}\pend{}\pstart{}\textsc{bei Liesing\oindex{XXIII., Liesing@\textbf{XXIII., Liesing}, \emph{A.ADM3}|pw}}\pend{}\pstart{}\textsc{Liesingerstraße 1}\oindex{Liesingerstrasse@\textbf{Liesingerstraße}, \emph{Straße (K.STR)}|pw}.\pend{}{\bigskip}\vspace{1em}
\pstart
           \raggedleft{}{\pb}23. 5. 905\pend
           \vspace{0.5em}
\pstart
           lieber Richard, ich beſtätige den unerwarteten Empfang des \textsc{Frisch}\pwindex{Frisch, Efraim 01.03.1873 – 26.11.1942@\textsc{Frisch, Efraim} (01.03.1873 – 26.11.1942), \emph{Schriftsteller/Schriftstellerin, Publizist/Publizistin}|pw}ſchen Buches\pwindex{Verloebnis. Geschichte eines Knaben@\emph{Das Verlöbnis. Geschichte eines Knaben}|pwuv}; – bedeutet das vielleicht den \substVorne{}\textsuperscript{Empfang}\substDazwischen{}Anfang\substHinten{} der Überſiedlung? Haben Sie den Grund ſchon gekauft? Könnte man ſich nicht
               wieder einmal, in Ruhe, ſehen? Sprechen? Ihre So{\geminationm}erpläne? Wir auf 3–4 Wochen Reichenau\oindex{Reichenau an der Rax@\textbf{Reichenau an der Rax}, \emph{A.ADM3}|pw}; mehr
               dürfte nicht herausko{\geminationm}en. –\pend
           
\pstart
           – Zum \textsc{Charolais}\pwindex{Graf von Charolais. Ein Trauerspiel@\emph{Der Graf von Charolais. Ein Trauerspiel}|pw} (nicht gerade zur Aufführung, in der ich nur \textsc{Kayssler}\pwindex{Kayssler, Friedrich 07.04.1874 – 24.04.1945@\textsc{Kayssler, Friedrich} (07.04.1874 – 24.04.1945), \emph{Schauspieler/Schauspielerin}|pw} und \textsc{Reinhardt}\pwindex{Reinhardt, Max 09.09.1873 – 30.10.1943@\textsc{Reinhardt, Max} (09.09.1873 – 30.10.1943), \emph{Theaterleiter/Theaterleiterin, Regisseur/Regisseurin, Schauspieler/Schauspielerin}|pw} hervorragend fand, – zunächſt: \textsc{Hartau}\pwindex{Hartau, Ludwig 19.2.1877 – 24.11.1922@\textsc{Hartau, Ludwig} (19.2.1877 – 24.11.1922), \emph{Schauspieler/Schauspielerin}|pw}) ka{\geminationn} ich Sie immer wied\textcolor{gray}{er} nur
               beglückwünſchen. Gewiſſe Einwendungen bleiben beſtehen; meine Liebe zu dem Werk
               erhöht und vertieft sich.\pend
           
\pstart
           Herzlichst Ihr{\\[\baselineskip]}\spacefill\mbox{A.}\pend
           \leftskip=0em{}\selectlanguage{ngerman}\endnumbering\briefempfaengerindex{Beer-Hofmann, Richard@\textsc{Beer-Hofmann, Richard}!zzzSchnitzler, Arthur@\emph{von Arthur Schnitzler}!1905-05-231@{23. 5. 1905}|)be}\mylabel{L01518h}  \normalsize

\doendnotes{C}
\bigskip
\vfill

\clearpage

\footnotesize

\lohead{\textsc{register}}

% Definiere theindex-Environment komplett neu ohne reledmac
\makeatletter
\renewenvironment{theindex}{%
  \section*{\indexname}%
  \setlength{\parindent}{0pt}%
  \setlength{\parskip}{0pt plus 0.3pt}%
  \let\item\@idxitem
}{%
  \clearpage
}
\makeatother

\IfFileExists{\jobname-pw.ind}{\input{\jobname-pw.ind}}{}

\end{document}

      