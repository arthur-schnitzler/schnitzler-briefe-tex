%% latex-leseansicht-vorspann.tex
%% Vorspann für die Leseansicht.
%% Lädt die gemeinsame Datei latex-vorspann.tex mit nicht gesetztem Schalter.

\newif\ifkorrekturansicht
\korrekturansichtfalse

\input{../tex-inputs/latex-vorspann}


\section[Arthur Schnitzler an Richard Beer-Hofmann, 23. 5. 1905]{L01518 Arthur Schnitzler an Richard Beer-Hofmann, 23. 5. 1905}
\nopagebreak\mylabel{L01518v}
\rehead{ }\normalsize\beginnumbering\briefempfaengerindex{Beer-Hofmann, Richard@\textsc{Beer-Hofmann, Richard}!zzzSchnitzler, Arthur@\emph{von Arthur Schnitzler}!1905-05-231@{23. 5. 1905}|(be}
\toendnotes[C]{\smallbreak\pagebreak[2]}
\correspDesc{Versand  durch Arthur Schnitzler am 23. 5. 1905 in Wien
\newline{}Erhalt  durch Richard Beer-Hofmann am 23. 5. 1905 in Rodaun}\toendnotes[C]{\smallbreak}
\Standort{YCGL, MSS 31.}
\physDesc{Kartenbrief, 649 Zeichen
\newline{}Handschrift: schwarze Tinte, deutsche Kurrent
\newline{}Versand: 1) Stempel: »\nobreak{}\oindex{XVIII., Währing@\textbf{XVIII., Währing}, \emph{Verwaltungsgebiet}|pwk}18/1 Wien 110, 2\textcolor{gray}{3}. V. 05, X\nobreak{}«.   2) Stempel: »\nobreak{}\oindex{Wien@\textbf{Wien}!XXIII., Liesing@\textbf{XXIII., Liesing}!Rodaun@\textbf{Rodaun}, \emph{Region}|pwk}R{[}odaun{]}, 2\textcolor{gray}{3. 5. 05}, 2–4\textcolor{gray}{N}\nobreak{}«. }
\buchAbdrucke{\weitereDrucke{Arthur Schnitzler, Richard Beer-Hofmann: \emph{Briefwechsel 1891–1931}. Herausgegeben von Konstanze Fliedl. Wien, Zürich: \emph{Europaverlag} 1992, S. 172.} }\toendnotes[C]{\smallbreak}\pstart{}{\pb}Herrn \textsc{Dr. Richard
                     Beer-Hofmann}\pend{}\pstart{}Rodaun\oindex{Wien@\textbf{Wien}!XXIII., Liesing@\textbf{XXIII., Liesing}!Rodaun@\textbf{Rodaun}, \emph{Region}|pw}\pend{}\pstart{}\textsc{bei Liesing\oindex{XXIII., Liesing@\textbf{XXIII., Liesing}, \emph{Verwaltungsgebiet}|pw}}\pend{}\pstart{}\textsc{Liesingerstraße 1}\oindex{Liesingerstraße@\textbf{Liesingerstraße}, \emph{Straße}|pw}.\pend{}{\bigskip}\vspace{1em}
\pstart
           \raggedleft{}{\pb}23. 5. 905\pend
           \vspace{0.5em}
\pstart
           lieber Richard, ich beſtätige den unerwarteten Empfang des \textsc{Frisch}\pwindex{Frisch, Efraim 1.\,3.\,1873 Stryj – 26.\,11.\,1942 Ascona@\textsc{Frisch, Efraim} (1.\,3.\,1873 Stryj – 26.\,11.\,1942 Ascona), \emph{Schriftsteller, Publizist}|pw}ſchen Buches\pwindex{Frisch, Efraim 1.\,3.\,1873 Stryj – 26.\,11.\,1942 Ascona@\textsc{Frisch, Efraim} (1.\,3.\,1873 Stryj – 26.\,11.\,1942 Ascona), \emph{Schriftsteller, Publizist}!Verlöbnis. Geschichte eines Knaben@\strich\emph{Das Verlöbnis. Geschichte eines Knaben}|pwuv}; – bedeutet das vielleicht den \substVorne{}\textsuperscript{Empfang}\substDazwischen{}Anfang\substHinten{} der Überſiedlung? Haben Sie den Grund{ }ſchon gekauft? Könnte man{ }ſich nicht
               wieder einmal, in Ruhe,{ }ſehen? Sprechen? Ihre So{\geminationm}erpläne? Wir auf 3–4 Wochen Reichenau\oindex{Reichenau an der Rax@\textbf{Reichenau an der Rax}, \emph{Verwaltungsgebiet}|pw}; mehr
               dürfte nicht herausko{\geminationm}en. –\pend
           
\pstart
           – Zum \textsc{Charolais}\pwindex{Beer-Hofmann, Richard 11.\,7.\,1866 Wien – 26.\,9.\,1945 New York City@\textsc{Beer-Hofmann, Richard} (11.\,7.\,1866 Wien – 26.\,9.\,1945 New York City), \emph{Schriftsteller}!Graf von Charolais. Ein Trauerspiel@\strich\emph{Der Graf von Charolais. Ein Trauerspiel}|pw} (nicht gerade zur Aufführung, in der ich nur \textsc{Kayssler}\pwindex{Kayssler, Friedrich 7.\,4.\,1874 Nowa Ruda – 24.\,4.\,1945 Kleinmachnow@\textsc{Kayssler, Friedrich} (7.\,4.\,1874 Nowa Ruda – 24.\,4.\,1945 Kleinmachnow), \emph{Schauspieler}|pw} und \textsc{Reinhardt}\pwindex{Reinhardt, Max 9.\,9.\,1873 Baden bei Wien – 30.\,10.\,1943 New York City@\textsc{Reinhardt, Max} (9.\,9.\,1873 Baden bei Wien – 30.\,10.\,1943 New York City), \emph{Theaterleiter, Regisseur, Schauspieler}|pw} hervorragend fand, – zunächſt: \textsc{Hartau}\pwindex{Hartau, Ludwig 19.\,2.\,1877 Żmigród – 24.\,11.\,1922 Berlin@\textsc{Hartau, Ludwig} (19.\,2.\,1877 Żmigród – 24.\,11.\,1922 Berlin), \emph{Schauspieler}|pw}) ka{\geminationn} ich Sie immer wied\textcolor{gray}{er} nur
               beglückwünſchen. Gewiſſe Einwendungen bleiben beſtehen; meine Liebe zu dem Werk
               erhöht und vertieft sich.\pend
           
\pstart
           Herzlichst Ihr{\\[\baselineskip]}\spacefill\mbox{A.}\pend
           \leftskip=0em{}\selectlanguage{ngerman}\endnumbering\briefempfaengerindex{Beer-Hofmann, Richard@\textsc{Beer-Hofmann, Richard}!zzzSchnitzler, Arthur@\emph{von Arthur Schnitzler}!1905-05-231@{23. 5. 1905}|)be}\mylabel{L01518h}  \newcommand{\dateiname}{L01518}\newcommand{\titel}{Arthur Schnitzler an Richard Beer-Hofmann, 23. 5. 1905}\newcommand{\editorInnen}{Martin Anton Müller und Gerd-Hermann Susen}%% latex-leseansicht-abspann.tex
%% Abspann für die Leseansicht.
%% Der Schalter \ifkorrekturansicht ist bereits durch den Vorspann gesetzt.

%% latex-abspann.tex
%% Gemeinsamer Abspann für Korrekturansicht und Leseansicht.
%% Setzt den Schalter \ifkorrekturansicht voraus (gesetzt in den
%% einbindenden Dateien latex-korrekturansicht-abspann.tex bzw.
%% latex-leseansicht-abspann.tex).
%% ---------------------------------------------------------------

\normalsize

% Das esempio-Environment wird nur in der Leseansicht benötigt
\ifkorrekturansicht\else
\newenvironment{esempio}[3]%
{
    \vspace{1.5ex}
    \rlap{\underline{#1}}
    \par
    \setlength{\parindent}{0cm}
    \nopagebreak
    \leftskip=#2cm
    \rightskip=#3cm
}
{
    \par
}
\fi

\doendnotes{C}
\bigskip
\vfill

\clearpage

\footnotesize

\ifkorrekturansicht
  \lohead{\textsc{register}}
\fi

% theindex-Environment neu definieren ohne reledmac
\makeatletter
\renewenvironment{theindex}{%
  \ifkorrekturansicht
    \section*{\indexname}%
  \else
    \subsubsection*{Index der erwähnten Entitäten}%
  \fi
  \setlength{\parindent}{0pt}%
  \setlength{\parskip}{0pt plus 0.3pt}%
  \let\item\@idxitem
}{%
  \ifkorrekturansicht\clearpage\fi
}
\makeatother

\IfFileExists{\jobname-pw.ind}{\input{\jobname-pw.ind}}{}

% Quellenangabe nur in der Leseansicht
\ifkorrekturansicht\else
% Fallback-Definitionen, falls die .tex-Datei \titel etc. nicht gesetzt hat
\providecommand{\titel}{}
\providecommand{\editorInnen}{}
\providecommand{\dateiname}{\jobname}

\vspace{3cm}

\vfill

\footnotesize
\textsc{Quelle}: \titel. Herausgegeben von {\editorInnen}. In: \emph{Arthur Schnitzler: Briefwechsel mit Autorinnen und Autoren}.
 Digitale Edition, https://schnitzler-briefe.acdh.oeaw.ac.at/{\dateiname}.html (Stand \today)
\fi

\end{document}


