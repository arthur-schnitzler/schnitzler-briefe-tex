%% latex-korrekturansicht-vorspann.tex
%% Vorspann für die Korrekturansicht.
%% Lädt die gemeinsame Datei latex-vorspann.tex mit gesetztem Schalter.

\newif\ifkorrekturansicht
\korrekturansichttrue

\input{../tex-inputs/latex-vorspann}


\section[ Paul Goldmann an Arthur Schnitzler, 22. 6. {[}1896{]}]{L02778 Paul Goldmann an Arthur Schnitzler, 22. 6. {[}1896{]}}
\nopagebreak\mylabel{L02778v}
\rehead{ }\normalsize\beginnumbering\briefempfaengerindex{Schnitzler, Arthur@\textsc{Schnitzler, Arthur}!zzzGoldmann, Paul@\emph{von Paul Goldmann}!1896-06-221@{22. 6. {[}1896{]}}|(be}
\toendnotes[C]{\smallbreak\pagebreak[2]}\Standort{DLA, A:Schnitzler, HS.NZ85.1.3166.}
\physDesc{Brief, 3 Blätter, 11 Seiten, 3758 Zeichen
\newline{}Handschrift: blaue Tinte, deutsche Kurrent
\newline{}Beilage: handschriftlicher Brief, 1 Blatt, 1 Seite, schwarze Tinte,
                                 lateinische Kurrent 
\newline{}Schnitzler: 1) mit Bleistift das Jahr »96« vermerkt  2) mit rotem Buntstift fünf Unterstreichungen}\toendnotes[C]{\smallbreak}
\pstart
           {\pb}\textcolor{gray}{\textbf{\textbf{Frankfurter Zeitung\orgindex{Frankfurter Zeitung@Frankfurter Zeitung|pw}}}}\pend
           
\pstart
           \textcolor{gray}{\textbf{(\begin{otherlanguage}{french}Gazette de Francfort\end{otherlanguage}\orgindex{Frankfurter Zeitung@Frankfurter Zeitung|pw}).}}\pend
           
\pstart
           \textcolor{gray}{\textbf{\textbf{\begin{otherlanguage}{french}Fondateur M.\end{otherlanguage}{ }L. Sonnemann\pwindex{Sonnemann, Leopold 1831-10-29 – 1909-10-30@\textsc{Sonnemann, Leopold} (1831-10-29 – 1909-10-30), \emph{Journalist/Journalistin, Herausgeber/Herausgeberin}|pw}.}}}\pend
           
\pstart
           \begin{otherlanguage}{french}\textcolor{gray}{\textbf{Journal\pwindex{Frankfurter Zeitung@\emph{Frankfurter Zeitung}|pwv} politique,
                        financier,}}\end{otherlanguage}\pend
           
\pstart
           \begin{otherlanguage}{french}\textcolor{gray}{\textbf{commercial et littéraire.}}\end{otherlanguage}\pend
           
\pstart
           \begin{otherlanguage}{french}\textcolor{gray}{\textbf{\textbf{Paraissant trois fois par jour.}}}\end{otherlanguage}\hfill \textsc{Paris\oindex{Paris@\textbf{Paris}, \emph{P.PPLC}|pw}}, 22. Juni.\pend
           
\pstart
           \begin{otherlanguage}{french}\textcolor{gray}{\textbf{\textbf{Bureau à Paris\oindex{Paris@\textbf{Paris}, \emph{P.PPLC}|pw}}}}\end{otherlanguage}\pend
           
\pstart
           \begin{otherlanguage}{french}\textcolor{gray}{\textbf{\textbf{24. Rue Feydeau\oindex{rue Feydeau@\textbf{rue Feydeau}, \emph{Straße (K.STR)}|pw}.}}}\end{otherlanguage}\pend
           
\pstart{}Mein lieber Freund,\pend\vspace{0.5em}
\pstart
           Es iſt ſehr lieb und freundſchaftlich von Dir, daß Du ſo auf dem Zuſammentreffen mit
               mir beſtehſt. Auch mir kannſt Du glauben, daß ich Dich nicht mit leichtem Herzen
               »aufgeben« würde und daß ich ſehr betrübt ſein würde, wenn ich Dich in dieſem Jahre
               nicht ſehen könnte\textcolor{gray}{.} Aber es wird ſich doch ſchwer machen laſſen.
               Da iſt zunächſt der materielle Grund. Ich habe weniger Geld als je, {\pb}und wenn ich auch mich im Princip nicht \strikeout{f} fürchten würde, mir etwas von Dir auszuleihen, ſo
               heißt doch »ausleihen« ſoviel\strikeout{,} als: Geld nehmen, um
               es wiederzugeben. Nach meinen \label{T_L02778-1v}\edtext{jetzigen}{\lemma{\textnormal{\emph{jetzigen}}}\Cendnote{\textnormal{In der Vorlage steht:
                     »jeztigen«.}}}\label{T_L02778-1} finanziellen Zuſtänden ſehe ich aber abſolut
               kein Mittel, \strikeout{D} das Ausgeliehene in abſehbarer Zeit
               zurückzugeben. Dazu kommt noch Allerlei an ſonſtigen Gründen: Ich bin ſehr müde und
               nervös, und die weite Eiſenbahn-Reiſe erſchreckt mich. {\pb}Ich kann ferner weder Seeluft noch \strikeout{See\textcolor{gray}{d}} Seebad vertragen, ſondern brauche zu meiner Erholung Gebirgsluft. Außerdem
               habe ich über die Preiſe in \textsc{Scodsborg\oindex{Skodsborg@\textbf{Skodsborg}, \emph{P.PPL}|pw}} von einem Dänen\pwindex{?? [Daene in Paris] @\textsc{?? [Däne in Paris]}|pwv}, der
               jedes Jahr hingeht, ganz andere Auskünfte erhalten, als Ihr: er meint, es ſei das
               theuerſte dän\oindex{Daenemark@\textbf{Dänemark}, \emph{A.PCLI}|pwv}iſche Seebad\oindex{Skodsborg@\textbf{Skodsborg}, \emph{P.PPL}|pw}. Endlich \strikeout{iſt
                     \textcolor{gray}{mir}} intereſſirt mich der ſkandinavi\oindex{Skandinavien@\textbf{Skandinavien}, \emph{Region}|pw}ſche Norden
               wenig, Dänemark\oindex{Daenemark@\textbf{Dänemark}, \emph{A.PCLI}|pw} ganz beſonders wenig, {\pb}und durch das Dän\oindex{Daenemark@\textbf{Dänemark}, \emph{A.PCLI}|pwv}en-Geſindel, das ich um \strikeout{\textsc{Al\textcolor{gray}{b}}}{ }\textsc{Albert Langen\pwindex{Langen, Albert 1869-07-08 – 1909-04-30@\textsc{Langen, Albert} (1869-07-08 – 1909-04-30), \emph{Verleger/Verlegerin}|pw}} habe kriechen ſehen, habe ich ſogar einen ſtarken – vielleicht ungerechten –
               Widerwillen gegen Dän\oindex{Daenemark@\textbf{Dänemark}, \emph{A.PCLI}|pwv}enthum
               bekommen. Nun glaube ich \strikeout{f\textcolor{gray}{erner}} ſo: Du wirſt nach vier Wochen ſchwed\oindex{Schweden@\textbf{Schweden}, \emph{A.PCLI}|pwv}iſch-norweg\oindex{Norwegen@\textbf{Norwegen}, \emph{A.PCLI}|pwv}iſcher Reiſe ausgiebig genug von Skandinavien\oindex{Skandinavien@\textbf{Skandinavien}, \emph{Region}|pw} haben, desgleichen \textsc{Richard\pwindex{Beer-Hofmann, Richard 1866-07-11 – 1945-09-26@\textsc{Beer-Hofmann, Richard} (1866-07-11 – 1945-09-26), \emph{Schriftsteller/Schriftstellerin}|pw}}, wenn er bereits im {\pb}Juli hingeht. Da Ihr\pwindex{Beer-Hofmann, Richard 1866-07-11 – 1945-09-26@\textsc{Beer-Hofmann, Richard} (1866-07-11 – 1945-09-26), \emph{Schriftsteller/Schriftstellerin}|pwv} nun ſo wie ſo nach Mittel-Europa\oindex{Europa@\textbf{Europa}, \emph{Kontinent (A.KNT)}|pw} zurück müßt, wie wäre es, wenn wir uns im Auguſt{ }\label{K_L02778-1v}\edtext{in der Schweiz\oindex{Schweiz@\textbf{Schweiz}, \emph{A.PCLI}|pw} träfen}{\lemma{\textnormal{\emph{in der Schweiz träfen}}}\Cendnote{\textnormal{Dazu kam es nicht, siehe Paul Goldmann an Arthur Schnitzler, 29. 4. [1896].
               }}}\label{K_L02778-1}? Einen großen Umweg macht Ihr nicht. Auch iſt es gar nicht übel: vier Wochen
               zu reiſen und ſich dann in der Schweiz\oindex{Schweiz@\textbf{Schweiz}, \emph{A.PCLI}|pw}, im Engadin\oindex{Engadin@\textbf{Engadin}, \emph{T.VAL}|pw}{ }\strikeout{z\textcolor{gray}{u}} etwa, auszuruhen. Warum ſeid Ihr denn {\pb}gar ſo
               ſehr auf das verfluchte Dänemark\oindex{Daenemark@\textbf{Dänemark}, \emph{A.PCLI}|pw}{ }\strikeout{erpicht,} erpicht, wo es nicht einmal Kunſt gibt,
               außer \textsc{Thorwaldsen\pwindex{Thorvaldsen, Bertel 13.11.1768 – 24.03.1844@\textsc{Thorvaldsen, Bertel} (13.11.1768 – 24.03.1844), \emph{Bildhauer/Bildhauerin}|pw}}, den man doch beſſer \uline{nicht} kennt. Und \textsc{Hamlet\pwindex{Hamlet@\emph{Hamlet}|pwv}}, welcher der einzig intereſſante Dän\oindex{Daenemark@\textbf{Dänemark}, \emph{A.PCLI}|pwv}e war, iſt auch ſchon todt. Wenn Ihr nun darauf beſteht,
               ſo werde ich doch mein Möglichſtes thun, um zu kommen. Aber Ihr ſolltet auch Einwände
               hören.\pend
           
\pstart
           {\pb}Daß man von \textsc{Albert Langen\pwindex{Langen, Albert 1869-07-08 – 1909-04-30@\textsc{Langen, Albert} (1869-07-08 – 1909-04-30), \emph{Verleger/Verlegerin}|pw}} überhaupt \strikeout{Einwän\textcolor{gray}{d}} Eindrücke empfängt, überraſcht mich. Das zählt doch gar nicht mit. Das iſt ein
               dummer Bube\pwindex{Langen, Albert 1869-07-08 – 1909-04-30@\textsc{Langen, Albert} (1869-07-08 – 1909-04-30), \emph{Verleger/Verlegerin}|pwv}, \strikeout{deſſ\textcolor{gray}{te}n} deſſen geiſtige Unfähigkeit
               hart an Blödſinn grenzt\substVorne{}\textsuperscript{,}\substDazwischen{}.\substHinten{} Das iſt zugleich frech und infam. Ich bitte Dich: laß’ Dich mit dem Burſchen\pwindex{Langen, Albert 1869-07-08 – 1909-04-30@\textsc{Langen, Albert} (1869-07-08 – 1909-04-30), \emph{Verleger/Verlegerin}|pwv} in keiner Weiſe ein,
               gib’ ihm keinen Rath und verhilf’ ihm zu \strikeout{kei} keinen
               Bekanntſchaften. {\pb}Er\pwindex{Langen, Albert 1869-07-08 – 1909-04-30@\textsc{Langen, Albert} (1869-07-08 – 1909-04-30), \emph{Verleger/Verlegerin}|pwv} wird Dich ausnutzen und
               Dich mit Bübereien entlohnen{\dotsfive}\pend
           
\pstart
           Ich habe den \textsc{Richard Mandl\pwindex{Mandl, Richard 1859-05-09 – 1918-03-31@\textsc{Mandl, Richard} (1859-05-09 – 1918-03-31), \emph{Komponist/Komponistin}|pw}} nun endlich kennen gelernt. Begeiſtert bin ich nicht. Ein netter und ganz
               geſcheiter Menſch, aber ſehr egoiſtiſch, ſehr berechnet, ſehr kalt, ſehr von ſich
               eingenommen, ſehr ſtolz auf ſeine \label{K_L02778-2v}\edtext{\begin{otherlanguage}{french}\textsc{relations mondaines}\end{otherlanguage}}{\lemma{\textnormal{\emph{relations mondaines}}}\Cendnote{\textnormal{französisch: weltliche
                  Beziehungen}}}\label{K_L02778-2}. Talent? Einiges jedenfalls, {\pb}viel aber wahrſcheinlich nicht. Er\pwindex{Mandl, Richard 1859-05-09 – 1918-03-31@\textsc{Mandl, Richard} (1859-05-09 – 1918-03-31), \emph{Komponist/Komponistin}|pwv} hat ein \label{K_L02778-3v}\edtext{Lied\pwindex{Anfang vom Ende@\emph{Anfang vom Ende}|pwv}}{\lemma{\textnormal{\emph{Lied}}}\Cendnote{\textnormal{Es handelte sich um eine Vertonung\pwindex{Anfang vom Ende@\emph{Anfang vom Ende}|pwkv} von Schnitzlers Gedicht \emph{Anfang vom Ende}\pwindex{Anfang vom Ende@\emph{Anfang vom Ende}|pwk}.
                     Schnitzler dürfte sie erst am 4. 1. 1898 zu hören
                  bekommen haben.}}}\label{K_L02778-3} von Dir componirt, wie Du weißt. Ich halte das für
               mißlungen. Die leichte Trauer des Lied\pwindex{Anfang vom Ende@\emph{Anfang vom Ende}|pwv}es hat er in die ſchwerſten Accente überſetzt. Das Lied\pwindex{Anfang vom Ende@\emph{Anfang vom Ende}|pwv} iſt melancholiſch, die Muſik tragiſch,
               Verſe und Compoſition ſehen ſich an und können ſich nicht verſtehen.\pend
           
\pstart
           Bitte, danke \textsc{Richard\pwindex{Beer-Hofmann, Richard 1866-07-11 – 1945-09-26@\textsc{Beer-Hofmann, Richard} (1866-07-11 – 1945-09-26), \emph{Schriftsteller/Schriftstellerin}|pw}} für ſeine Correſpondenz-{\pb}Karte. Ich hoffe, das
               hat ihn nicht zu ſehr \label{K_L02778-4v}\edtext{ermüdet}{\lemma{\textnormal{\emph{ermüdet}}}\Cendnote{\textnormal{Spott über die Schreibfaulheit Beer-Hofmanns\pwindex{Beer-Hofmann, Richard 1866-07-11 – 1945-09-26@\textsc{Beer-Hofmann, Richard} (1866-07-11 – 1945-09-26), \emph{Schriftsteller/Schriftstellerin}|pwk}}}}\label{K_L02778-4}. Am Tage, wo er dieſe
               Correnſpondenz-Karte verfaßt, hat er gewiß nicht mehr am »Götterliebling\pwindex{Tod Georgs@\emph{Der Tod Georgs}|pwv}« weitergeſchrieben, –
               hoffentlich aber hat \strikeout{ſich} er ſich am nächſten Tage
               wieder dieſem Werke\pwindex{Tod Georgs@\emph{Der Tod Georgs}|pwv}
               zugewendet, deſſen \strikeout{z\textcolor{gray}{w}} zweites Capitel\pwindex{Tod Georgs@\emph{Der Tod Georgs}|pwv} jetzt
                  \strikeout{ſich\textcolor{gray}{e}} ſicher bereits der {\pb}Vollendung
               entgegenreift.\pend
           
\pstart
           Grüß’ Dich Gott, liebſter Freund!\pend
           
\pstart
           Dein {\\[\baselineskip]}\spacefill\mbox{P. Goldmn}\pend
           \leftskip=0em{}\selectlanguage{ngerman}\vspace{1em}{\vspace{1\baselineskip}}
\pstart
           \raggedleft{}{\pb}Le \label{K_L02778-5v}\edtext{19 Juin ’96}{\lemma{\textnormal{\emph{19 Juin ’96}}}\Cendnote{\textnormal{Die Beilage ist diesem Brief
                     ausschließlich auf Grundlage der Datierung auf den 19. 6. 1896 zugeordnet. Weder in diesem noch in einem anderen Brief
                     geht Goldmann\pwindex{Goldmann, Paul 31.01.1865 – 25.09.1935@\textsc{Goldmann, Paul} (31.01.1865 – 25.09.1935), \emph{Schriftsteller/Schriftstellerin, Journalist/Journalistin}|pwk} auf das Schreiben
                     ein.}}}\label{K_L02778-5}\pend
           
\pstart{}Mon cher confrère \pend\vspace{0.5em}
\pstart
           \label{K_L02778-6v}\edtext{Ci-joint \label{K_L02778-7v}\edtext{l’article\pwindex{Lettres, Sciences et Arts [Mourir]@\emph{Lettres, Sciences et Arts [Mourir]}|pwv}}{\lemma{\textnormal{\emph{l’article}}}\Cendnote{\textnormal{eventuell die knappe Würdigung\pwindex{Lettres, Sciences et Arts [Mourir]@\emph{Lettres, Sciences et Arts [Mourir]}|pwkv} von Schnitzlers bisherigem Schaffen anlässlich des Erscheinens
                  von \emph{Mourir}\pwindex{Mourir. Roman@\emph{Mourir. Roman}|pwk}, die ohne Angabe eines Verfassers
                     (Abel Hermant\pwindex{Hermant, Abel 03.02.1862 – 28.09.1950@\textsc{Hermant, Abel} (03.02.1862 – 28.09.1950), \emph{Schriftsteller/Schriftstellerin}|pwk}?) erschien: \emph{Lettres, Sciences et Arts}\pwindex{Lettres, Sciences et Arts [Mourir]@\emph{Lettres, Sciences et Arts [Mourir]}|pwk}. In: \emph{Journal des débats}\pwindex{Journal des debats. Politiques et litteraires@\emph{Journal des débats. Politiques et littéraires}|pwk}, Jg. 108, Nr. 168,
                        16. 6. 1896, S. 3}}}\label{K_L02778-7} dont je vous ai parlé. Peut-être M. Schnitzler en aura déjà pris
               connaissance, si par exemple vos confrères à Vienne\oindex{Wien@\textbf{Wien}, \emph{A.ADM2}|pw} ou à Berlin\oindex{Berlin@\textbf{Berlin}, \emph{P.PPLC}|pw} ont eu l’obligeance
               de le lui faire parvenir.}{\lemma{\textnormal{\emph{Ci-joint … parvenir.}}}\Cendnote{\textnormal{französisch:
                     Anbei der Artikel\pwindex{Lettres, Sciences et Arts [Mourir]@\emph{Lettres, Sciences et Arts [Mourir]}|pwv}, den ich Ihnen gegenüber erwähnt habe. Vielleicht ist er Herrn
                        Schnitzler schon zur Kenntnis gelangt,
                     wenn beispielsweise Ihre Kollegen in Wien\oindex{Wien@\textbf{Wien}, \emph{A.ADM2}|pw}
                     oder Berlin\oindex{Berlin@\textbf{Berlin}, \emph{P.PPLC}|pw} die Freundlichkeit besaßen, ihn
                     ihm zukommen zu lassen.}}}\label{K_L02778-6}\pend
           
\pstart
           Mille amitiés{\\[\baselineskip]} Votre dévoué{\\[\baselineskip]}\spacefill\mbox{AHermant.\pwindex{Hermant, Abel 03.02.1862 – 28.09.1950@\textsc{Hermant, Abel} (03.02.1862 – 28.09.1950), \emph{Schriftsteller/Schriftstellerin}|pw}}\pend
           \leftskip=0em{}\selectlanguage{ngerman}\endnumbering\briefempfaengerindex{Schnitzler, Arthur@\textsc{Schnitzler, Arthur}!zzzGoldmann, Paul@\emph{von Paul Goldmann}!1896-06-221@{22. 6. {[}1896{]}}|)be}\mylabel{L02778h}  \normalsize

\doendnotes{C}
\bigskip
\vfill

\clearpage

\footnotesize

\lohead{\textsc{register}}

% Definiere theindex-Environment komplett neu ohne reledmac
\makeatletter
\renewenvironment{theindex}{%
  \section*{\indexname}%
  \setlength{\parindent}{0pt}%
  \setlength{\parskip}{0pt plus 0.3pt}%
  \let\item\@idxitem
}{%
  \clearpage
}
\makeatother

\IfFileExists{\jobname-pw.ind}{\input{\jobname-pw.ind}}{}

\end{document}

      