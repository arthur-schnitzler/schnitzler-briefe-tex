%% latex-leseansicht-vorspann.tex
%% Vorspann für die Leseansicht.
%% Lädt die gemeinsame Datei latex-vorspann.tex mit nicht gesetztem Schalter.

\newif\ifkorrekturansicht
\korrekturansichtfalse

\input{../tex-inputs/latex-vorspann}


\section[ Paul Goldmann an Arthur Schnitzler, 22. 6. [1896]]{L02778 Paul Goldmann an Arthur Schnitzler,  22. 6. [1896]}
\nopagebreak\mylabel{L02778v}
\rehead{ }\normalsize\beginnumbering\briefempfaengerindex{Schnitzler, Arthur@\textsc{Schnitzler, Arthur}!zzzGoldmann, Paul@\emph{von Paul Goldmann}!1896-06-221@{22. 6. [1896]}|(be}
\toendnotes[C]{\smallbreak\pagebreak[2]}
\correspDesc{Versand  durch Paul Goldmann am 22. 6. [1896] in Paris
\newline{}Erhalt  durch Arthur Schnitzler im Zeitraum [23. 6. 1896
                  – 27. 6. 1896?] in Wien}\toendnotes[C]{\smallbreak}
\Standort{DLA, A:Schnitzler, HS.NZ85.1.3166.}
\physDesc{Brief, 3 Blätter, 11 Seiten, 3758 Zeichen
\newline{}Handschrift: blaue Tinte, deutsche Kurrent
\newline{}Beilage: handschriftlicher Brief, 1 Blatt, 1 Seite, schwarze Tinte,
                                 lateinische Kurrent 
\newline{}Schnitzler: 1) mit Bleistift das Jahr »96« vermerkt  2) mit rotem Buntstift fünf Unterstreichungen}\toendnotes[C]{\smallbreak}
\pstart
           {\pb}\textcolor{gray}{\textbf{\textbf{Frankfurter Zeitung\orgindex{Frankfurter Zeitung@Frankfurter Zeitung|pw}}}}\pend
           
\pstart
           \textcolor{gray}{\textbf{(\begin{otherlanguage}{french}Gazette de Francfort\end{otherlanguage}\orgindex{Frankfurter Zeitung@Frankfurter Zeitung|pw}).}}\pend
           
\pstart
           \textcolor{gray}{\textbf{\textbf{\begin{otherlanguage}{french}Fondateur M.\end{otherlanguage}{ }L. Sonnemann\pwindex{Sonnemann, Leopold 29.\,10.\,1831 Höchberg – 30.\,10.\,1909 Frankfurt am Main@\textsc{Sonnemann, Leopold} (29.\,10.\,1831 Höchberg – 30.\,10.\,1909 Frankfurt am Main), \emph{Journalist, Herausgeber}|pw}.}}}\pend
           
\pstart
           \begin{otherlanguage}{french}\textcolor{gray}{\textbf{Journal\pwindex{Frankfurter Zeitung@\emph{Frankfurter Zeitung}|pwv} politique,
                        financier,}}\end{otherlanguage}\pend
           
\pstart
           \begin{otherlanguage}{french}\textcolor{gray}{\textbf{commercial et littéraire.}}\end{otherlanguage}\pend
           
\pstart
           \begin{otherlanguage}{french}\textcolor{gray}{\textbf{\textbf{Paraissant trois fois par jour.}}}\end{otherlanguage}\hfill \textsc{Paris\oindex{Paris@\textbf{Paris}, \emph{Hauptstadt}|pw}}, 22. Juni.\pend
           
\pstart
           \begin{otherlanguage}{french}\textcolor{gray}{\textbf{\textbf{Bureau à Paris\oindex{Paris@\textbf{Paris}, \emph{Hauptstadt}|pw}}}}\end{otherlanguage}\pend
           
\pstart
           \begin{otherlanguage}{french}\textcolor{gray}{\textbf{\textbf{24. Rue Feydeau\oindex{rue Feydeau@\textbf{rue Feydeau}, \emph{Straße}|pw}.}}}\end{otherlanguage}\pend
           
\pstart{}Mein lieber Freund,\pend\vspace{0.5em}
\pstart
           Es iſt{ }ſehr lieb und freundſchaftlich von Dir, daß Du{ }ſo auf dem Zuſammentreffen mit
               mir beſtehſt. Auch mir kannſt Du glauben, daß ich Dich nicht mit leichtem Herzen
               »aufgeben« würde und daß ich{ }ſehr betrübt{ }ſein würde, wenn ich Dich in dieſem Jahre
               nicht{ }ſehen könnte\textcolor{gray}{.} Aber es wird{ }ſich doch{ }ſchwer machen laſſen.
               Da iſt zunächſt der materielle Grund. Ich habe weniger Geld als je, {\pb}und wenn ich auch mich im Princip nicht \strikeout{f} fürchten würde, mir etwas von Dir auszuleihen,{ }ſo
               heißt doch »ausleihen«{ }ſoviel\strikeout{,} als: Geld nehmen, um
               es wiederzugeben. Nach meinen \label{T_L02778-1v}\edtext{jetzigen}{\lemma{\textnormal{\emph{jetzigen}}}\Cendnote{\textnormal{In der Vorlage steht:
                     »jeztigen«.}}}\label{T_L02778-1} finanziellen Zuſtänden{ }ſehe ich aber abſolut
               kein Mittel, \strikeout{D} das Ausgeliehene in abſehbarer Zeit
               zurückzugeben. Dazu kommt noch Allerlei an{ }ſonſtigen Gründen: Ich bin{ }ſehr müde und
               nervös, und die weite Eiſenbahn-Reiſe erſchreckt mich. {\pb}Ich kann ferner weder Seeluft noch \strikeout{See\textcolor{gray}{d}} Seebad vertragen,{ }ſondern brauche zu meiner Erholung Gebirgsluft. Außerdem
               habe ich über die Preiſe in \textsc{Scodsborg\oindex{Skodsborg@\textbf{Skodsborg}|pw}} von einem Dänen\pwindex{?? [Däne in Paris] @\textsc{?? [Däne in Paris]}|pwv}, der
               jedes Jahr hingeht, ganz andere Auskünfte erhalten, als Ihr: er meint, es{ }ſei das
               theuerſte dän\oindex{Dänemark@\textbf{Dänemark}|pwv}iſche Seebad\oindex{Skodsborg@\textbf{Skodsborg}|pw}. Endlich \strikeout{iſt
                     \textcolor{gray}{mir}} intereſſirt mich der ſkandinavi\oindex{Skandinavien@\textbf{Skandinavien}|pw}ſche Norden
               wenig, Dänemark\oindex{Dänemark@\textbf{Dänemark}|pw} ganz beſonders wenig, {\pb}und durch das Dän\oindex{Dänemark@\textbf{Dänemark}|pwv}en-Geſindel, das ich um \strikeout{\textsc{Al\textcolor{gray}{b}}}{ }\textsc{Albert Langen\pwindex{Langen, Albert 8.\,7.\,1869 Antwerpen – 30.\,4.\,1909 München@\textsc{Langen, Albert} (8.\,7.\,1869 Antwerpen – 30.\,4.\,1909 München), \emph{Verleger}|pw}} habe kriechen{ }ſehen, habe ich{ }ſogar einen{ }ſtarken – vielleicht ungerechten –
               Widerwillen gegen Dän\oindex{Dänemark@\textbf{Dänemark}|pwv}enthum
               bekommen. Nun glaube ich \strikeout{f\textcolor{gray}{erner}}{ }ſo: Du wirſt nach vier Wochen ſchwed\oindex{Schweden@\textbf{Schweden}|pwv}iſch-norweg\oindex{Norwegen@\textbf{Norwegen}|pwv}iſcher Reiſe ausgiebig genug von Skandinavien\oindex{Skandinavien@\textbf{Skandinavien}|pw} haben, desgleichen \textsc{Richard\pwindex{Beer-Hofmann, Richard 11.\,7.\,1866 Wien – 26.\,9.\,1945 New York City@\textsc{Beer-Hofmann, Richard} (11.\,7.\,1866 Wien – 26.\,9.\,1945 New York City), \emph{Schriftsteller}|pw}}, wenn er bereits im {\pb}Juli hingeht. Da Ihr\pwindex{Beer-Hofmann, Richard 11.\,7.\,1866 Wien – 26.\,9.\,1945 New York City@\textsc{Beer-Hofmann, Richard} (11.\,7.\,1866 Wien – 26.\,9.\,1945 New York City), \emph{Schriftsteller}|pwv} nun{ }ſo wie{ }ſo nach Mittel-Europa\oindex{Europa@\textbf{Europa}|pw} zurück müßt, wie wäre es, wenn wir uns im Auguſt{ }\label{K_L02778-1v}\edtext{in der Schweiz\oindex{Schweiz@\textbf{Schweiz}|pw} träfen}{\lemma{\textnormal{\emph{in der Schweiz träfen}}}\Cendnote{\textnormal{Dazu kam es nicht, siehe XXXX Auszeichnungsfehler: Dokument L02772 nicht gefunden.
               }}}\label{K_L02778-1}? Einen großen Umweg macht Ihr nicht. Auch iſt es gar nicht übel: vier Wochen
               zu reiſen und{ }ſich dann in der Schweiz\oindex{Schweiz@\textbf{Schweiz}|pw}, im Engadin\oindex{Engadin@\textbf{Engadin}, \emph{Tal}|pw}{ }\strikeout{z\textcolor{gray}{u}} etwa, auszuruhen. Warum{ }ſeid Ihr denn {\pb}gar{ }ſo{ }ſehr auf das verfluchte Dänemark\oindex{Dänemark@\textbf{Dänemark}|pw}{ }\strikeout{erpicht,} erpicht, wo es nicht einmal Kunſt gibt,
               außer \textsc{Thorwaldsen\pwindex{Thorvaldsen, Bertel 13.\,11.\,1768 Kopenhagen – 24.\,3.\,1844 ebd.@\textsc{Thorvaldsen, Bertel} (13.\,11.\,1768 Kopenhagen – 24.\,3.\,1844 ebd.), \emph{Bildhauer}|pw}}, den man doch beſſer \uline{nicht} kennt. Und \textsc{Hamlet\pwindex{\textcolor{red}{\textsuperscript{XXXX indx1}}!Hamlet@\strich\emph{Hamlet}|pwv}}, welcher der einzig intereſſante Dän\oindex{Dänemark@\textbf{Dänemark}|pwv}e war, iſt auch{ }ſchon todt. Wenn Ihr nun darauf beſteht,{ }ſo werde ich doch mein Möglichſtes thun, um zu kommen. Aber Ihr{ }ſolltet auch Einwände
               hören.\pend
           
\pstart
           {\pb}Daß man von \textsc{Albert Langen\pwindex{Langen, Albert 8.\,7.\,1869 Antwerpen – 30.\,4.\,1909 München@\textsc{Langen, Albert} (8.\,7.\,1869 Antwerpen – 30.\,4.\,1909 München), \emph{Verleger}|pw}} überhaupt \strikeout{Einwän\textcolor{gray}{d}} Eindrücke empfängt, überraſcht mich. Das zählt doch gar nicht mit. Das iſt ein
               dummer Bube\pwindex{Langen, Albert 8.\,7.\,1869 Antwerpen – 30.\,4.\,1909 München@\textsc{Langen, Albert} (8.\,7.\,1869 Antwerpen – 30.\,4.\,1909 München), \emph{Verleger}|pwv}, \strikeout{deſſ\textcolor{gray}{te}n} deſſen geiſtige Unfähigkeit
               hart an Blödſinn grenzt\substVorne{}\textsuperscript{,}\substDazwischen{}.\substHinten{} Das iſt zugleich frech und infam. Ich bitte Dich: laß’ Dich mit dem Burſchen\pwindex{Langen, Albert 8.\,7.\,1869 Antwerpen – 30.\,4.\,1909 München@\textsc{Langen, Albert} (8.\,7.\,1869 Antwerpen – 30.\,4.\,1909 München), \emph{Verleger}|pwv} in keiner Weiſe ein,
               gib’ ihm keinen Rath und verhilf’ ihm zu \strikeout{kei} keinen
               Bekanntſchaften. {\pb}Er\pwindex{Langen, Albert 8.\,7.\,1869 Antwerpen – 30.\,4.\,1909 München@\textsc{Langen, Albert} (8.\,7.\,1869 Antwerpen – 30.\,4.\,1909 München), \emph{Verleger}|pwv} wird Dich ausnutzen und
               Dich mit Bübereien entlohnen{\dotsfive}\pend
           
\pstart
           Ich habe den \textsc{Richard Mandl\pwindex{Mandl, Richard 9.\,5.\,1859 Prostějov – 31.\,3.\,1918 Wien@\textsc{Mandl, Richard} (9.\,5.\,1859 Prostějov – 31.\,3.\,1918 Wien), \emph{Komponist}|pw}} nun endlich kennen gelernt. Begeiſtert bin ich nicht. Ein netter und ganz
               geſcheiter Menſch, aber{ }ſehr egoiſtiſch,{ }ſehr berechnet,{ }ſehr kalt,{ }ſehr von{ }ſich
               eingenommen,{ }ſehr{ }ſtolz auf{ }ſeine \label{K_L02778-2v}\edtext{\begin{otherlanguage}{french}\textsc{relations mondaines}\end{otherlanguage}}{\lemma{\textnormal{\emph{relations mondaines}}}\Cendnote{\textnormal{französisch: weltliche
                  Beziehungen}}}\label{K_L02778-2}. Talent? Einiges jedenfalls, {\pb}viel aber wahrſcheinlich nicht. Er\pwindex{Mandl, Richard 9.\,5.\,1859 Prostějov – 31.\,3.\,1918 Wien@\textsc{Mandl, Richard} (9.\,5.\,1859 Prostějov – 31.\,3.\,1918 Wien), \emph{Komponist}|pwv} hat ein \label{K_L02778-3v}\edtext{Lied\pwindex{Anfang vom Ende@\emph{Anfang vom Ende}|pwv}}{\lemma{\textnormal{\emph{Lied}}}\Cendnote{\textnormal{Es handelte sich um eine Vertonung\pwindex{Anfang vom Ende@\emph{Anfang vom Ende}|pwkv} von Schnitzlers Gedicht \emph{Anfang vom Ende}\pwindex{Schnitzler, Arthur 15.\,5.\,1862 Wien – 21.\,10.\,1931 ebd.@\textsc{Schnitzler, Arthur} (15.\,5.\,1862 Wien – 21.\,10.\,1931 ebd.), \emph{Schriftsteller, Mediziner}!Anfang vom Ende@\strich\emph{Anfang vom Ende}|pwk}.
                     Schnitzler dürfte sie erst am 4. 1. 1898 zu hören
                  bekommen haben.}}}\label{K_L02778-3} von Dir componirt, wie Du weißt. Ich halte das für
               mißlungen. Die leichte Trauer des Lied\pwindex{Anfang vom Ende@\emph{Anfang vom Ende}|pwv}es hat er in die{ }ſchwerſten Accente überſetzt. Das Lied\pwindex{Anfang vom Ende@\emph{Anfang vom Ende}|pwv} iſt melancholiſch, die Muſik tragiſch,
               Verſe und Compoſition{ }ſehen{ }ſich an und können{ }ſich nicht verſtehen.\pend
           
\pstart
           Bitte, danke \textsc{Richard\pwindex{Beer-Hofmann, Richard 11.\,7.\,1866 Wien – 26.\,9.\,1945 New York City@\textsc{Beer-Hofmann, Richard} (11.\,7.\,1866 Wien – 26.\,9.\,1945 New York City), \emph{Schriftsteller}|pw}} für{ }ſeine Correſpondenz-{\pb}Karte. Ich hoffe, das
               hat ihn nicht zu{ }ſehr \label{K_L02778-4v}\edtext{ermüdet}{\lemma{\textnormal{\emph{ermüdet}}}\Cendnote{\textnormal{Spott über die Schreibfaulheit Beer-Hofmanns\pwindex{Beer-Hofmann, Richard 11.\,7.\,1866 Wien – 26.\,9.\,1945 New York City@\textsc{Beer-Hofmann, Richard} (11.\,7.\,1866 Wien – 26.\,9.\,1945 New York City), \emph{Schriftsteller}|pwk}}}}\label{K_L02778-4}. Am Tage, wo er dieſe
               Correnſpondenz-Karte verfaßt, hat er gewiß nicht mehr am »Götterliebling\pwindex{Beer-Hofmann, Richard 11.\,7.\,1866 Wien – 26.\,9.\,1945 New York City@\textsc{Beer-Hofmann, Richard} (11.\,7.\,1866 Wien – 26.\,9.\,1945 New York City), \emph{Schriftsteller}!Tod Georgs@\strich\emph{Der Tod Georgs}|pwv}« weitergeſchrieben, –
               hoffentlich aber hat \strikeout{ſich} er{ }ſich am nächſten Tage
               wieder dieſem Werke\pwindex{Beer-Hofmann, Richard 11.\,7.\,1866 Wien – 26.\,9.\,1945 New York City@\textsc{Beer-Hofmann, Richard} (11.\,7.\,1866 Wien – 26.\,9.\,1945 New York City), \emph{Schriftsteller}!Tod Georgs@\strich\emph{Der Tod Georgs}|pwv}
               zugewendet, deſſen \strikeout{z\textcolor{gray}{w}} zweites Capitel\pwindex{Beer-Hofmann, Richard 11.\,7.\,1866 Wien – 26.\,9.\,1945 New York City@\textsc{Beer-Hofmann, Richard} (11.\,7.\,1866 Wien – 26.\,9.\,1945 New York City), \emph{Schriftsteller}!Tod Georgs@\strich\emph{Der Tod Georgs}|pwv} jetzt
                  \strikeout{ſich\textcolor{gray}{e}}{ }ſicher bereits der {\pb}Vollendung
               entgegenreift.\pend
           
\pstart
           Grüß’ Dich Gott, liebſter Freund!\pend
           
\pstart
           Dein {\\[\baselineskip]}\spacefill\mbox{P. Goldmn}\pend
           \leftskip=0em{}\selectlanguage{ngerman}\vspace{1em}{\vspace{1\baselineskip}}
\pstart
           \raggedleft{}{\pb}Le \label{K_L02778-5v}\edtext{19 Juin ’96}{\lemma{\textnormal{\emph{19 Juin ’96}}}\Cendnote{\textnormal{Die Beilage ist diesem Brief
                     ausschließlich auf Grundlage der Datierung auf den 19. 6. 1896 zugeordnet. Weder in diesem noch in einem anderen Brief
                     geht Goldmann\pwindex{Goldmann, Paul 31.\,1.\,1865 Breslau – 25.\,9.\,1935 Wien@\textsc{Goldmann, Paul} (31.\,1.\,1865 Breslau – 25.\,9.\,1935 Wien), \emph{Schriftsteller, Journalist}|pwk} auf das Schreiben
                     ein.}}}\label{K_L02778-5}\pend
           
\pstart{}Mon cher confrère\pend\vspace{0.5em}
\pstart
           \label{K_L02778-6v}\edtext{Ci-joint \label{K_L02778-7v}\edtext{l’article\pwindex{Lettres, Sciences et Arts [Mourir]@\emph{Lettres, Sciences et Arts [Mourir]}|pwv}}{\lemma{\textnormal{\emph{l’article}}}\Cendnote{\textnormal{eventuell die knappe Würdigung\pwindex{Lettres, Sciences et Arts [Mourir]@\emph{Lettres, Sciences et Arts [Mourir]}|pwkv} von Schnitzlers bisherigem Schaffen anlässlich des Erscheinens
                  von \emph{Mourir}\pwindex{Schnitzler, Arthur 15.\,5.\,1862 Wien – 21.\,10.\,1931 ebd.@\textsc{Schnitzler, Arthur} (15.\,5.\,1862 Wien – 21.\,10.\,1931 ebd.), \emph{Schriftsteller, Mediziner}!Mourir. Roman@\strich\emph{Mourir. Roman}|pwk}, die ohne Angabe eines Verfassers
                     (Abel Hermant\pwindex{Hermant, Abel 3.\,2.\,1862 Paris – 28.\,9.\,1950@\textsc{Hermant, Abel} (3.\,2.\,1862 Paris – 28.\,9.\,1950), \emph{Schriftsteller}|pwk}?) erschien: \emph{Lettres, Sciences et Arts}\pwindex{Lettres, Sciences et Arts [Mourir]@\emph{Lettres, Sciences et Arts [Mourir]}|pwk}. In: \emph{Journal des débats}\pwindex{Journal des débats. Politiques et littéraires@\emph{Journal des débats. Politiques et littéraires}|pwk}, Jg. 108, Nr. 168,
                        16. 6. 1896, S. 3}}}\label{K_L02778-7} dont je vous ai parlé. Peut-être M. Schnitzler en aura déjà pris
               connaissance, si par exemple vos confrères à Vienne\oindex{Wien@\textbf{Wien}, \emph{Verwaltungsgebiet}|pw} ou à Berlin\oindex{Berlin@\textbf{Berlin}, \emph{Hauptstadt}|pw} ont eu l’obligeance
               de le lui faire parvenir.}{\lemma{\textnormal{\emph{Ci-joint … parvenir.}}}\Cendnote{\textnormal{französisch:
                     Anbei der Artikel\pwindex{Lettres, Sciences et Arts [Mourir]@\emph{Lettres, Sciences et Arts [Mourir]}|pwv}, den ich Ihnen gegenüber erwähnt habe. Vielleicht ist er Herrn
                        Schnitzler schon zur Kenntnis gelangt,
                     wenn beispielsweise Ihre Kollegen in Wien\oindex{Wien@\textbf{Wien}, \emph{Verwaltungsgebiet}|pw}
                     oder Berlin\oindex{Berlin@\textbf{Berlin}, \emph{Hauptstadt}|pw} die Freundlichkeit besaßen, ihn
                     ihm zukommen zu lassen.}}}\label{K_L02778-6}\pend
           
\pstart
           Mille amitiés{\\[\baselineskip]} Votre dévoué{\\[\baselineskip]}\spacefill\mbox{AHermant.\pwindex{Hermant, Abel 3.\,2.\,1862 Paris – 28.\,9.\,1950@\textsc{Hermant, Abel} (3.\,2.\,1862 Paris – 28.\,9.\,1950), \emph{Schriftsteller}|pw}}\pend
           \leftskip=0em{}\selectlanguage{ngerman}\endnumbering\briefempfaengerindex{Schnitzler, Arthur@\textsc{Schnitzler, Arthur}!zzzGoldmann, Paul@\emph{von Paul Goldmann}!1896-06-221@{22. 6. [1896]}|)be}\mylabel{L02778h}  \newcommand{\dateiname}{L02778}\newcommand{\titel}{Paul Goldmann an Arthur Schnitzler, 22. 6. [1896]}\newcommand{\editorInnen}{Martin Anton Müller und Laura Untner}%% latex-leseansicht-abspann.tex
%% Abspann für die Leseansicht.
%% Der Schalter \ifkorrekturansicht ist bereits durch den Vorspann gesetzt.

%% latex-abspann.tex
%% Gemeinsamer Abspann für Korrekturansicht und Leseansicht.
%% Setzt den Schalter \ifkorrekturansicht voraus (gesetzt in den
%% einbindenden Dateien latex-korrekturansicht-abspann.tex bzw.
%% latex-leseansicht-abspann.tex).
%% ---------------------------------------------------------------

\normalsize

% Das esempio-Environment wird nur in der Leseansicht benötigt
\ifkorrekturansicht\else
\newenvironment{esempio}[3]%
{
    \vspace{1.5ex}
    \rlap{\underline{#1}}
    \par
    \setlength{\parindent}{0cm}
    \nopagebreak
    \leftskip=#2cm
    \rightskip=#3cm
}
{
    \par
}
\fi

\doendnotes{C}
\bigskip
\vfill

\clearpage

\footnotesize

\ifkorrekturansicht
  \lohead{\textsc{register}}
\fi

% theindex-Environment neu definieren ohne reledmac
\makeatletter
\renewenvironment{theindex}{%
  \ifkorrekturansicht
    \section*{\indexname}%
  \else
    \subsubsection*{Index der erwähnten Entitäten}%
  \fi
  \setlength{\parindent}{0pt}%
  \setlength{\parskip}{0pt plus 0.3pt}%
  \let\item\@idxitem
}{%
  \ifkorrekturansicht\clearpage\fi
}
\makeatother

\IfFileExists{\jobname-pw.ind}{\input{\jobname-pw.ind}}{}

% Quellenangabe nur in der Leseansicht
\ifkorrekturansicht\else
% Fallback-Definitionen, falls die .tex-Datei \titel etc. nicht gesetzt hat
\providecommand{\titel}{}
\providecommand{\editorInnen}{}
\providecommand{\dateiname}{\jobname}

\vspace{3cm}

\vfill

\footnotesize
\textsc{Quelle}: \titel. Herausgegeben von {\editorInnen}. In: \emph{Arthur Schnitzler: Briefwechsel mit Autorinnen und Autoren}.
 Digitale Edition, https://schnitzler-briefe.acdh.oeaw.ac.at/{\dateiname}.html (Stand \today)
\fi

\end{document}


