%% latex-korrekturansicht-vorspann.tex
%% Vorspann für die Korrekturansicht.
%% Lädt die gemeinsame Datei latex-vorspann.tex mit gesetztem Schalter.

\newif\ifkorrekturansicht
\korrekturansichttrue

\input{../tex-inputs/latex-vorspann}


\section[Hugo von Hofmannsthal an Arthur Schnitzler, {[}3.?{]} 9. 1900]{L01070 Hugo von Hofmannsthal an Arthur Schnitzler, {[}3.?{]} 9. 1900}
\nopagebreak\mylabel{L01070v}
\rehead{ }\normalsize\beginnumbering\briefempfaengerindex{Schnitzler, Arthur@\textsc{Schnitzler, Arthur}!zzzHofmannsthal, Hugo von@\emph{von Hugo von Hofmannsthal}!1900-09-031@{{[}3.?{]} 9. 1900}|(be}
\toendnotes[C]{\smallbreak\pagebreak[2]}\Standort{CUL, Schnitzler, B 43.}
\physDesc{Bildpostkarte, 210 Zeichen
\newline{}Handschrift: Bleistift, deutsche Kurrent
\newline{}Versand: 1) Stempel: »\nobreak{}\oindex{Sambir@\textbf{Sambir}, \emph{P.PPLA2}|pwk}{[}Sambor{]}\nobreak{}«.   2) Stempel: »\nobreak{}\oindex{IX., Alsergrund@\textbf{IX., Alsergrund}, \emph{A.ADM3}|pwk}Wien 9/3, 4. 9. 00, 5.{[}N{]}, Bestellt\nobreak{}«. 
\newline{}Schnitzler: mit Bleistift datiert: »Aug 900« 
\newline{}Ordnung: mit Bleistift von unbekannter Hand doppelt nummeriert:
                                    »165« und »\strikeout{172}« }
\buchAbdrucke{\weitereDrucke{Hugo von Hofmannsthal, Arthur Schnitzler: \emph{Briefwechsel}. Frankfurt am Main: \emph{S. Fischer} 1964, S. 144.} }\pstart{}{\pb}\textsc{Herrn D\textsuperscript{r} Arthur Schnitzler}\pend{}\pstart{}\textsc{Wien}\oindex{Wien@\textbf{Wien}, \emph{A.ADM2}|pw}\pend{}\pstart{}IX Franckgasse 1\oindex{Frankgasse 1@\textbf{Frankgasse 1}, \emph{Wohngebäude (K.WHS)}|pw}.\pend{}{\bigskip}
\pstart
           \noindent{}\centering{}{\pb}\textcolor{gray}{\textbf{Sambor\oindex{Sambir@\textbf{Sambir}, \emph{P.PPLA2}|pw}}}\pend
           
\pstart
           \centering{}\textcolor{gray}{\textbf{z wiezy koscielnej widziany.}}\pend
           
\pstart
           \centering{}\textcolor{gray}{\textbf{von dem Kirchenthurme gesehen.}}\pend
           \vspace{1em}
\pstart
           \noindent{}{\pb}Was Sie machen? Ich bin
               10 Stunden im Sattel, ſchlafe im Heu, jeden Tag in einem andern Neſt, und bin
               eigentlich ſehr zufrieden und gut aufgelegt.\pend
           
\pstart
           Von Herzen{\\[\baselineskip]}Ihr\spacefill\mbox{Hugo.}\pend
           \leftskip=0em{}\selectlanguage{ngerman}\endnumbering\briefempfaengerindex{Schnitzler, Arthur@\textsc{Schnitzler, Arthur}!zzzHofmannsthal, Hugo von@\emph{von Hugo von Hofmannsthal}!1900-09-031@{{[}3.?{]} 9. 1900}|)be}\mylabel{L01070h}  \normalsize

\doendnotes{C}
\bigskip
\vfill

\clearpage

\footnotesize

\lohead{\textsc{register}}

% Definiere theindex-Environment komplett neu ohne reledmac
\makeatletter
\renewenvironment{theindex}{%
  \section*{\indexname}%
  \setlength{\parindent}{0pt}%
  \setlength{\parskip}{0pt plus 0.3pt}%
  \let\item\@idxitem
}{%
  \clearpage
}
\makeatother

\IfFileExists{\jobname-pw.ind}{\input{\jobname-pw.ind}}{}

\end{document}

      