%% latex-leseansicht-vorspann.tex
%% Vorspann für die Leseansicht.
%% Lädt die gemeinsame Datei latex-vorspann.tex mit nicht gesetztem Schalter.

\newif\ifkorrekturansicht
\korrekturansichtfalse

\input{../tex-inputs/latex-vorspann}


\section[Hermann Bahr an Arthur Schnitzler, 8. 7. 1897]{L00695 Hermann Bahr an Arthur Schnitzler, 8. 7. 1897}
\nopagebreak\mylabel{L00695v}
\rehead{ }\normalsize\beginnumbering\briefempfaengerindex{Schnitzler, Arthur@\textsc{Schnitzler, Arthur}!zzzBahr, Hermann@\emph{von Hermann Bahr}!1897-07-082@{8. 7. 1897}|(be}
\toendnotes[C]{\smallbreak\pagebreak[2]}
\correspDesc{Versand  durch Hermann Bahr am 8. 7. 1897 in Wien
\newline{}Erhalt  durch Arthur Schnitzler im Zeitraum [8. 7. 1897
                  – 12. 7. 1897?] in Wien}\toendnotes[C]{\smallbreak}
\Standort{CUL, Schnitzler, B 5b.}
\physDesc{Brief, 1 Blatt, 4 Seiten, 1343 Zeichen
\newline{}Handschrift: schwarze Tinte, deutsche Kurrent
\newline{}Ordnung: mit Bleistift von unbekannter Hand nummeriert:
                                    »53« }
\buchAbdrucke{\weitereDrucke{Hermann Bahr, Arthur Schnitzler: \emph{Briefwechsel, Aufzeichnungen, Dokumente (1891–1931)}. Herausgegeben von Kurt Ifkovits und Martin Anton Müller. Göttingen: \emph{Wallstein} 2018, S. 148.} }\toendnotes[C]{\smallbreak}
\pstart
           {\pb}\textcolor{gray}{\textbf{»Die Zeit\orgindex{Zeit. Wiener Wochenschrift@Die Zeit. Wiener Wochenschrift|pw}«}}\hfill \textcolor{gray}{\textbf{\textbf{Wien\oindex{Wien@\textbf{Wien}, \emph{Verwaltungsgebiet}|pw}}, den}}{ }8. Juli \textcolor{gray}{\textbf{189}}7\pend
           
\pstart
           \textcolor{gray}{\textbf{Wiener Wochenſchrift}}\hfill \textcolor{gray}{\textbf{IX/3, Günthergaſſe 1\oindex{Wien@\textbf{Wien}!IX., Alsergrund@\textbf{IX., Alsergrund}!Günthergasse@\textbf{Günthergasse}, \emph{Straße}|pw}.}}\pend
           
\pstart
           \textcolor{gray}{\textbf{\textbf{Herausgeber}:}}{\\}\textcolor{gray}{\textbf{Profeſſor Dr. I. Singer\pwindex{Singer, Isidor 16.\,1.\,1857 Budapest – 8.\,12.\,1927 Wien@\textsc{Singer, Isidor} (16.\,1.\,1857 Budapest – 8.\,12.\,1927 Wien), \emph{Journalist, Herausgeber, Soziologe}|pw}, Hermann Bahr\pwindex{Bahr, Hermann 19.\,7.\,1863 Linz – 15.\,1.\,1934 München@\textsc{Bahr, Hermann} (19.\,7.\,1863 Linz – 15.\,1.\,1934 München), \emph{Schriftsteller, Kritiker}|pw},
                        Dr. Heinrich Kanner\pwindex{Kanner, Heinrich 9.\,11.\,1864 Galați – 15.\,2.\,1930 Wien@\textsc{Kanner, Heinrich} (9.\,11.\,1864 Galați – 15.\,2.\,1930 Wien), \emph{Herausgeber, Publizist}|pw}.}}\pend
           
\pstart
           \textcolor{gray}{\textbf{Telephon Nr. 6415.}}\pend
           
\pstart\center{}Lieber Freund!\pend\vspace{0.5em}
\pstart
           Neumann-Hofers\pwindex{Neumann-Hofer, Gilbert Otto 4.\,2.\,1857 Bol’shiye Berezhki – 14.\,4.\,1941 Detmold@\textsc{Neumann-Hofer, Gilbert Otto} (4.\,2.\,1857 Bol’shiye Berezhki – 14.\,4.\,1941 Detmold), \emph{Kritiker, Theaterleiter}|pw} Drängen nachgebend, der mich
               noch immer mit Dir plagt, frage ich noch einmal bei Dir an, ob Du denn nicht doch
               irgendwie zu beſtimmen wäreſt, einen Vertrag mit ihm einzugehen, der Dich für drei
               oder fünf Jahre an{ }ſein Theater\oindex{Lessing-Theater@\textbf{Lessing-Theater}, \emph{Theater}|pwv}
               bindet. Ich habe Dir schon geſagt: er bietet Dir 12{\%}
               Tantièmen an, oder wenn Du es vorziehſt, ein Einreichungs{\pb}honorar; eventuell ließe er{ }ſich wohl zu beidem
               bereden. Es iſt ihm{ }ſehr wichtig, gerade Dich zu haben. Stelle Deine Forderungen; ich
               habe neulich in den paar Minuten Dir nicht{ }ſo recht zureden können u. weiß nicht, ob
               ich Dich in Iſchl\oindex{Bad Ischl@\textbf{Bad Ischl}|pw}{ }ſehen werde. Ich bitte Dich alſo brieflich, Dir die
               Sache doch noch einmal zu überlegen. Sie hat gewiß ihre Bedenken. Aber überlege Dir,
               ob{ }ſie{ }ſich nicht{ }ſo drehen läßt, daß{ }ſie die größten Vorzüge für Dich hat. Suche Dir
               etwa Termine aus, wie Du{ }ſie{ }ſonſt an keinem Theater {\pb}kriegſt, oder was{ }ſonſt etwa in Deinen Wünſchen
               liegt. Ich weiß ja nicht, worauf Du am meiſten Werth legſt. Schreib mir das dann. Ich
               würde{ }ſehr wünſchen, daß Du doch irgendwie mit Neumannhofer\pwindex{Neumann-Hofer, Gilbert Otto 4.\,2.\,1857 Bol’shiye Berezhki – 14.\,4.\,1941 Detmold@\textsc{Neumann-Hofer, Gilbert Otto} (4.\,2.\,1857 Bol’shiye Berezhki – 14.\,4.\,1941 Detmold), \emph{Kritiker, Theaterleiter}|pw} zuſammen kommſt: denn ich hoffe{ }ſo diesen allmälig dahin zu
               bringen, daß er aus dem Leſſingtheater\orgindex{Lessing-Theater@Lessing-Theater|pw} eine gut
                  öſtreichiſche\oindex{Österreich@\textbf{Österreich}|pw} Bühne macht. Dies würde ich
               von Herzen wünſchen.\pend
           
\pstart
           In der Hoffnung, daß es Dir immer gut geht, bin ich, mit vielen Grüßen {\pb}an Richard\pwindex{Beer-Hofmann, Richard 11.\,7.\,1866 Wien – 26.\,9.\,1945 New York City@\textsc{Beer-Hofmann, Richard} (11.\,7.\,1866 Wien – 26.\,9.\,1945 New York City), \emph{Schriftsteller}|pw},\pend
           
\pstart
           Dein alter treuer{\\[\baselineskip]}\spacefill\mbox{Hermann}\pend
           \leftskip=0em{}
\pstart
           \textcolor{gray}{\textbf{\label{T_L00695-1v}\edtext{Alle für »Die Zeit\orgindex{Zeit. Wiener Wochenschrift@Die Zeit. Wiener Wochenschrift|pw}« beſtimmten Zuſchriften und Sendungen{ }ſind an die
                  Redaction der »Zeit\orgindex{Zeit. Wiener Wochenschrift@Die Zeit. Wiener Wochenschrift|pw}« und \textbf{nicht} an die Perſon eines der Herausgeber zu richten.}{\lemma{\textnormal{\emph{Alle … richten.}}}\Cendnote{\textnormal{am unteren Rand der ersten Seite}}}\label{T_L00695-1}}}\pend
           \selectlanguage{ngerman}\endnumbering\briefempfaengerindex{Schnitzler, Arthur@\textsc{Schnitzler, Arthur}!zzzBahr, Hermann@\emph{von Hermann Bahr}!1897-07-082@{8. 7. 1897}|)be}\mylabel{L00695h}  \newcommand{\dateiname}{L00695}\newcommand{\titel}{Hermann Bahr an Arthur Schnitzler, 8. 7. 1897}\newcommand{\editorInnen}{Herausgegeben von Martin Anton Müller}%% latex-leseansicht-abspann.tex
%% Abspann für die Leseansicht.
%% Der Schalter \ifkorrekturansicht ist bereits durch den Vorspann gesetzt.

%% latex-abspann.tex
%% Gemeinsamer Abspann für Korrekturansicht und Leseansicht.
%% Setzt den Schalter \ifkorrekturansicht voraus (gesetzt in den
%% einbindenden Dateien latex-korrekturansicht-abspann.tex bzw.
%% latex-leseansicht-abspann.tex).
%% ---------------------------------------------------------------

\normalsize

% Das esempio-Environment wird nur in der Leseansicht benötigt
\ifkorrekturansicht\else
\newenvironment{esempio}[3]%
{
    \vspace{1.5ex}
    \rlap{\underline{#1}}
    \par
    \setlength{\parindent}{0cm}
    \nopagebreak
    \leftskip=#2cm
    \rightskip=#3cm
}
{
    \par
}
\fi

\doendnotes{C}
\bigskip
\vfill

\clearpage

\footnotesize

\ifkorrekturansicht
  \lohead{\textsc{register}}
\fi

% theindex-Environment neu definieren ohne reledmac
\makeatletter
\renewenvironment{theindex}{%
  \ifkorrekturansicht
    \section*{\indexname}%
  \else
    \subsubsection*{Index der erwähnten Entitäten}%
  \fi
  \setlength{\parindent}{0pt}%
  \setlength{\parskip}{0pt plus 0.3pt}%
  \let\item\@idxitem
}{%
  \ifkorrekturansicht\clearpage\fi
}
\makeatother

\IfFileExists{\jobname-pw.ind}{\input{\jobname-pw.ind}}{}

% Quellenangabe nur in der Leseansicht
\ifkorrekturansicht\else
% Fallback-Definitionen, falls die .tex-Datei \titel etc. nicht gesetzt hat
\providecommand{\titel}{}
\providecommand{\editorInnen}{}
\providecommand{\dateiname}{\jobname}

\vspace{3cm}

\vfill

\footnotesize
\textsc{Quelle}: \titel. Herausgegeben von {\editorInnen}. In: \emph{Arthur Schnitzler: Briefwechsel mit Autorinnen und Autoren}.
 Digitale Edition, https://schnitzler-briefe.acdh.oeaw.ac.at/{\dateiname}.html (Stand \today)
\fi

\end{document}


