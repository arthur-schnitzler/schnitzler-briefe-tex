%% latex-leseansicht-vorspann.tex
%% Vorspann für die Leseansicht.
%% Lädt die gemeinsame Datei latex-vorspann.tex mit nicht gesetztem Schalter.

\newif\ifkorrekturansicht
\korrekturansichtfalse

\input{../tex-inputs/latex-vorspann}


\section[Max Burckhard: Widmungsexemplar Franz Stelzhamer Charakterbilder für Arthur Schnitzler, 27. 10. 1905]{L01565 Max Burckhard: Widmungsexemplar Franz Stelzhamer Charakterbilder für
               Arthur Schnitzler, 27. 10. 1905}
\nopagebreak\mylabel{L01565v}
\rehead{ }\normalsize\beginnumbering\briefempfaengerindex{Schnitzler, Arthur@\textsc{Schnitzler, Arthur}!zzzBurckhard, Max Eugen@\emph{von Max Eugen Burckhard}!1905-10-271@{27. 10. 1905}|(be}
\toendnotes[C]{\smallbreak\pagebreak[2]}
\correspDesc{Versand  durch Max Burckhard am 27. 10. 1905 \textbf{Ort fehlend} 
\newline{}Erhalt  durch Arthur Schnitzler im Zeitraum [27. 10. 1905 – 31. 10. 1905?] in Wien}\toendnotes[C]{\smallbreak}
\Standort{DLA, G:Schnitzler, Arthur (Sammlung Heinrich Schnitzler).}
\physDesc{Widmung am Schmutztitel, 66 Zeichen
\newline{}Handschrift: schwarze Tinte, deutsche Kurrent
\newline{}Ordnung: bei der Enteignung des Exemplars 1938 von
                                 unbekannter Hand mit Bleistift ergänzte Informationen: »\noindent{}= 440752-B{ / }s{[}ine{]}.a{[}nno{]}.!« }\toendnotes[C]{\smallbreak}
\pstart
           \noindent{}{\pb}Arthur Schnitzler\pend
           \pstart herzlichſt d. Herausgeber\spacefill\mbox{DrBurckhard}\pend{}
\pstart
           27. 10. 05\pend
           {\vspace{1\baselineskip}}
\pstart
           \centering{}\textcolor{gray}{\textbf{Charakterbilder aus Oberöſterreich\pwindex{Stelzhamer, Franz 29.\,11.\,1802 Großpiesenham – 14.\,7.\,1874 Henndorf am Wallersee@\textsc{Stelzhamer, Franz} (29.\,11.\,1802 Großpiesenham – 14.\,7.\,1874 Henndorf am Wallersee), \emph{Schriftsteller}!Charakterbilder aus Oberösterreich@\strich\emph{Charakterbilder aus Oberösterreich}|pw}}}\pend
           \selectlanguage{ngerman}\vspace{1em}{\vspace{1\baselineskip}}
\pstart
           \centering{}{\pb}\textcolor{gray}{\textbf{Franz Stelzhamer\pwindex{Stelzhamer, Franz 29.\,11.\,1802 Großpiesenham – 14.\,7.\,1874 Henndorf am Wallersee@\textsc{Stelzhamer, Franz} (29.\,11.\,1802 Großpiesenham – 14.\,7.\,1874 Henndorf am Wallersee), \emph{Schriftsteller}|pw}.}}\pend
           
\pstart
           \centering{}\textcolor{gray}{\textbf{Charakterbilder aus{\\}Oberoeſterreich\pwindex{Stelzhamer, Franz 29.\,11.\,1802 Großpiesenham – 14.\,7.\,1874 Henndorf am Wallersee@\textsc{Stelzhamer, Franz} (29.\,11.\,1802 Großpiesenham – 14.\,7.\,1874 Henndorf am Wallersee), \emph{Schriftsteller}!Charakterbilder aus Oberösterreich@\strich\emph{Charakterbilder aus Oberösterreich}|pw}.–
                  Mit}}{\\}\textcolor{gray}{\textbf{einem Geleitspruch}}{\\}\textcolor{gray}{\textbf{von}}\pend
           
\pstart
           \centering{}\textcolor{gray}{\textbf{Gerhart Hauptmann\pwindex{Hauptmann, Gerhart 15.\,11.\,1862 Szczawno-Zdrój – 6.\,6.\,1946 Jagniątków@\textsc{Hauptmann, Gerhart} (15.\,11.\,1862 Szczawno-Zdrój – 6.\,6.\,1946 Jagniątków), \emph{Schriftsteller}|pw}.}}\pend
           {\vspace{1\baselineskip}}
\pstart
           \centering{}\textcolor{gray}{\textbf{\label{K_L01565-1v}\edtext{Wiener Verlag\orgindex{Wiener Verlag@Wiener Verlag|pw}}{\lemma{\textnormal{\emph{Wiener Verlag}}}\Cendnote{\textnormal{Am 13. 11. 1905 wurde das Erscheinen vom \emph{Börsenblatt für den deutschen Buchhandel}\pwindex{Börsenblatt für den Deutschen Buchhandel@\emph{Börsenblatt für den Deutschen Buchhandel}|pwk}
                      gemeldet.}}}\label{K_L01565-1}{ }Wien\oindex{Wien@\textbf{Wien}, \emph{Verwaltungsgebiet}|pw}{ }u.{ }Leipzig\oindex{Leipzig@\textbf{Leipzig}, \emph{Hauptstadt}|pw}}}\pend
           \selectlanguage{ngerman}\endnumbering\briefempfaengerindex{Schnitzler, Arthur@\textsc{Schnitzler, Arthur}!zzzBurckhard, Max Eugen@\emph{von Max Eugen Burckhard}!1905-10-271@{27. 10. 1905}|)be}\mylabel{L01565h}  \newcommand{\dateiname}{L01565}\newcommand{\titel}{Max Burckhard: Widmungsexemplar Franz Stelzhamer Charakterbilder für Arthur Schnitzler, 27. 10. 1905}\newcommand{\editorInnen}{Martin Anton Müller und Gerd-Hermann Susen}%% latex-leseansicht-abspann.tex
%% Abspann für die Leseansicht.
%% Der Schalter \ifkorrekturansicht ist bereits durch den Vorspann gesetzt.

%% latex-abspann.tex
%% Gemeinsamer Abspann für Korrekturansicht und Leseansicht.
%% Setzt den Schalter \ifkorrekturansicht voraus (gesetzt in den
%% einbindenden Dateien latex-korrekturansicht-abspann.tex bzw.
%% latex-leseansicht-abspann.tex).
%% ---------------------------------------------------------------

\normalsize

% Das esempio-Environment wird nur in der Leseansicht benötigt
\ifkorrekturansicht\else
\newenvironment{esempio}[3]%
{
    \vspace{1.5ex}
    \rlap{\underline{#1}}
    \par
    \setlength{\parindent}{0cm}
    \nopagebreak
    \leftskip=#2cm
    \rightskip=#3cm
}
{
    \par
}
\fi

\doendnotes{C}
\bigskip
\vfill

\clearpage

\footnotesize

\ifkorrekturansicht
  \lohead{\textsc{register}}
\fi

% theindex-Environment neu definieren ohne reledmac
\makeatletter
\renewenvironment{theindex}{%
  \ifkorrekturansicht
    \section*{\indexname}%
  \else
    \subsubsection*{Index der erwähnten Entitäten}%
  \fi
  \setlength{\parindent}{0pt}%
  \setlength{\parskip}{0pt plus 0.3pt}%
  \let\item\@idxitem
}{%
  \ifkorrekturansicht\clearpage\fi
}
\makeatother

\IfFileExists{\jobname-pw.ind}{\input{\jobname-pw.ind}}{}

% Quellenangabe nur in der Leseansicht
\ifkorrekturansicht\else
% Fallback-Definitionen, falls die .tex-Datei \titel etc. nicht gesetzt hat
\providecommand{\titel}{}
\providecommand{\editorInnen}{}
\providecommand{\dateiname}{\jobname}

\vspace{3cm}

\vfill

\footnotesize
\textsc{Quelle}: \titel. Herausgegeben von {\editorInnen}. In: \emph{Arthur Schnitzler: Briefwechsel mit Autorinnen und Autoren}.
 Digitale Edition, https://schnitzler-briefe.acdh.oeaw.ac.at/{\dateiname}.html (Stand \today)
\fi

\end{document}


