%% latex-korrekturansicht-vorspann.tex
%% Vorspann für die Korrekturansicht.
%% Lädt die gemeinsame Datei latex-vorspann.tex mit gesetztem Schalter.

\newif\ifkorrekturansicht
\korrekturansichttrue

\input{../tex-inputs/latex-vorspann}


\section[Hermann Bahr an Arthur Schnitzler, 29. 7. 1894]{L00358 Hermann Bahr an Arthur Schnitzler, 29. 7. 1894}
\nopagebreak\mylabel{L00358v}
\rehead{ }\normalsize\beginnumbering\briefempfaengerindex{Schnitzler, Arthur@\textsc{Schnitzler, Arthur}!zzzBahr, Hermann@\emph{von Hermann Bahr}!1894-07-291@{29. 7. 1894}|(be}
\toendnotes[C]{\smallbreak\pagebreak[2]}\Standort{CUL, Schnitzler, B 5b.}
\physDesc{Postkarte, 164 Zeichen
\newline{}Handschrift: schwarze Tinte, deutsche Kurrent
\newline{}Versand: Stempel: »\nobreak{}\oindex{VIII., Josefstadt@\textbf{VIII., Josefstadt}, \emph{A.ADM3}|pwk}Wien 8/1, 29 7 94, 3–4 N\nobreak{}«.  
\newline{}Ordnung: 1) mit rotem Buntstift von unbekannter Hand nummeriert:
                                    »25«  2) mit Bleistift von unbekannter Hand nummeriert:
                                    »25«}
\buchAbdrucke{\weitereDrucke{Hermann Bahr, Arthur Schnitzler: \emph{Briefwechsel, Aufzeichnungen, Dokumente (1891–1931)}. Göttingen: \emph{Wallstein} 2018, S. 77.} }\toendnotes[C]{\smallbreak}\pstart{}{\pb}Herrn \textsc{D\textsuperscript{r} Arthur Schnitzler}\pend{}\pstart{}Schriftsteller\pend{}\pstart{}Wien IX\oindex{IX., Alsergrund@\textbf{IX., Alsergrund}, \emph{A.ADM3}|pw}\pend{}\pstart{}\textsc{Franckgasse 1}\oindex{Frankgasse 1@\textbf{Frankgasse 1}, \emph{Wohngebäude (K.WHS)}|pw}\pend{}{\bigskip}\vspace{1em}
\pstart
           \noindent{}{\pb}Mein Telephon iſt \label{K_L00358-1v}\edtext{6415}{\lemma{\textnormal{\emph{6415}}}\Cendnote{\textnormal{Die Nummer der
                  Redaktion der \emph{Zeit}\orgindex{Zeit. Wiener Wochenschrift@Die Zeit. Wiener Wochenschrift|pwk}. Privat war Bahr\pwindex{Bahr, Hermann 19.07.1863 – 15.01.1934@\textsc{Bahr, Hermann} (19.07.1863 – 15.01.1934), \emph{Schriftsteller/Schriftstellerin, Kritiker/Kritikerin}|pwk} am 8. 5. 1894 in die Lammgasse 3\oindex{Lammgasse@\textbf{Lammgasse}, \emph{Straße (K.STR)}|pwk} umgezogen. Hier weist ihn das Adressverzeichnis \emph{Lehmann}\pwindex{Lehmann s Allgemeiner Wohnungs-Anzeiger@\emph{Lehmann’s Allgemeiner Wohnungs-Anzeiger}|pwk}{ }1895 ebenfalls als »Telephonabonnent« aus, Nr. 4531.}}}\label{K_L00358-1}.\pend
           
\pstart
           Herzlichſt{\\[\baselineskip]}\spacefill\mbox{Bahr}\pend
           \leftskip=0em{}
\pstart
           \noindent{}\label{LL171-1v}D.\pwindex{Sandrock, Adele 1863-08-19 – 1937-08-30@\textsc{Sandrock, Adele} (1863-08-19 – 1937-08-30), \emph{Schauspieler/Schauspielerin}|pw}{ }ſchreibt mir heute, daß ſie am 5. »auf zwei
                     Minuten« nach Wien\oindex{Wien@\textbf{Wien}, \emph{A.ADM2}|pw} kommt.\label{LL171-1h}\pend
           \selectlanguage{ngerman}\endnumbering\briefempfaengerindex{Schnitzler, Arthur@\textsc{Schnitzler, Arthur}!zzzBahr, Hermann@\emph{von Hermann Bahr}!1894-07-291@{29. 7. 1894}|)be}\mylabel{L00358h}  \normalsize

\doendnotes{C}
\bigskip
\vfill

\clearpage

\footnotesize

\lohead{\textsc{register}}

% Definiere theindex-Environment komplett neu ohne reledmac
\makeatletter
\renewenvironment{theindex}{%
  \section*{\indexname}%
  \setlength{\parindent}{0pt}%
  \setlength{\parskip}{0pt plus 0.3pt}%
  \let\item\@idxitem
}{%
  \clearpage
}
\makeatother

\IfFileExists{\jobname-pw.ind}{\input{\jobname-pw.ind}}{}

\end{document}

      