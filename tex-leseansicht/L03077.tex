%% latex-leseansicht-vorspann.tex
%% Vorspann für die Leseansicht.
%% Lädt die gemeinsame Datei latex-vorspann.tex mit nicht gesetztem Schalter.

\newif\ifkorrekturansicht
\korrekturansichtfalse

\input{../tex-inputs/latex-vorspann}


\section[ Paul Goldmann an Arthur Schnitzler, 5. 8. {[}1901{]}]{L03077 Paul Goldmann an Arthur Schnitzler,  5. 8. [1901]}
\nopagebreak\mylabel{L03077v}
\rehead{ }\normalsize\beginnumbering\briefempfaengerindex{Schnitzler, Arthur@\textsc{Schnitzler, Arthur}!zzzGoldmann, Paul@\emph{von Paul Goldmann}!1901-08-051@{5. 8. [1901]}|(be}
\toendnotes[C]{\smallbreak\pagebreak[2]}
\correspDesc{Versand  durch Paul Goldmann am 5. 8. [1901] in Höhlenstein
\newline{}Erhalt  durch Arthur Schnitzler im Zeitraum [6. 8. 1901
                  – 10. 8. 1901?] in Vahrn}\toendnotes[C]{\smallbreak}
\Standort{DLA, A:Schnitzler, HS.NZ85.1.3171.}
\physDesc{Brief, 1 Blatt, 2 Seiten, 1185 Zeichen
\newline{}Handschrift: schwarze Tinte, deutsche Kurrent
\newline{}Schnitzler: 1) mit Bleistift das Jahr »901« vermerkt  2) mit rotem Buntstift vier Unterstreichungen}\toendnotes[C]{\smallbreak}
\pstart
           \centering{}{\pb}\textsc{Landro\oindex{Höhlenstein@\textbf{Höhlenstein}|pw}}, 5. Auguſt.\pend
           
\pstart{}Mein lieber Freund,\pend\vspace{0.5em}
\pstart
           \label{K_L03077-1v}\edtext{\textsc{Richards\pwindex{Beer-Hofmann, Richard 11.\,7.\,1866 Wien – 26.\,9.\,1945 New York City@\textsc{Beer-Hofmann, Richard} (11.\,7.\,1866 Wien – 26.\,9.\,1945 New York City), \emph{Schriftsteller}|pw}} Telegramm}{\lemma{\textnormal{\emph{Richards Telegramm}}}\Cendnote{\textnormal{Daraus ist zu schließen, dass Beer-Hofmann\pwindex{Beer-Hofmann, Richard 11.\,7.\,1866 Wien – 26.\,9.\,1945 New York City@\textsc{Beer-Hofmann, Richard} (11.\,7.\,1866 Wien – 26.\,9.\,1945 New York City), \emph{Schriftsteller}|pwk} das Telegramm vom XXXX Auszeichnungsfehler: Dokument L01156 nicht gefunden am XXXX Auszeichnungsfehler: Dokument L01157 nicht gefunden, als er mit Goldmann\pwindex{Goldmann, Paul 31.\,1.\,1865 Breslau – 25.\,9.\,1935 Wien@\textsc{Goldmann, Paul} (31.\,1.\,1865 Breslau – 25.\,9.\,1935 Wien), \emph{Schriftsteller, Journalist}|pwk} persönlich zusammentraf, noch nicht
                  erhalten hatte.}}}\label{K_L03077-1}, in dem er mir mittheilte, daß Du einen hochgelegenen Ort\oindex{Klobenstein@\textbf{Klobenstein}|pwv} gefunden, erreichte mich
               leider zu spät. Ich hatte mich bereits in \textsc{Landro\oindex{Höhlenstein@\textbf{Höhlenstein}|pw}} eingemiethet; ein Zimmer hatte ich in dem \textsc{Hôtel\oindex{Hotel Baur@\textbf{Hotel Baur}, \emph{Hotel}|pwv}} nämlich nur \strikeout{\textcolor{gray}{×}\-\textcolor{gray}{×}} unter der Bedingung bekommen, daß ich mindeſtens eine Woche zu bleiben mich
               verpflichtete. So werde ich alſo nicht vor Ablauf dieſer Woche \label{K_L03077-2v}\edtext{zu Dir kommen}{\lemma{\textnormal{\emph{zu Dir kommen}}}\Cendnote{\textnormal{Siehe XXXX Auszeichnungsfehler: Dokument L03064 nicht gefunden.
               }}}\label{K_L03077-2} können, und ich bitte\substVorne{}\textsuperscript{,}\substDazwischen{} D\substHinten{}ich, mich{ }ſogleich von Deinem Aufenthaltsort zu \label{K_L03077-3v}\edtext{beſtändigen}{\lemma{\textnormal{\emph{beständigen}}}\Cendnote{\textnormal{Schreibirrtum, Goldmann\pwindex{Goldmann, Paul 31.\,1.\,1865 Breslau – 25.\,9.\,1935 Wien@\textsc{Goldmann, Paul} (31.\,1.\,1865 Breslau – 25.\,9.\,1935 Wien), \emph{Schriftsteller, Journalist}|pwk} meinte wohl
                  »verständigen«.}}}\label{K_L03077-3}. Die kühle und{ }ſtarke Luft hier bekommt mir gut; die trüben
               Gedanken vermag freilich keine noch{ }ſo kühle Luft zu bannen. Ich hatte gehofft, hier
               ein paar liebe Wien\oindex{Wien@\textbf{Wien}, \emph{Verwaltungsgebiet}|pw}er Mädeln zu finden. Aber es
               iſt nichts vorhanden als die Familie {\pb}\textsc{Speyer\pwindex{Speyer, Nanette 5.\,1.\,1846 Iserlohn – 15.\,1.\,1925 Wien@\textsc{Speyer, Nanette} (5.\,1.\,1846 Iserlohn – 15.\,1.\,1925 Wien)|pw}\pwindex{Speyer, Albert 8.\,4.\,1836 Breslau – 25.\,3.\,1905 Opatija@\textsc{Speyer, Albert} (8.\,4.\,1836 Breslau – 25.\,3.\,1905 Opatija), \emph{Industrieller}|pw}}. Und angeſichts des \textsc{Monte Cristallo\oindex{Monte Cristallo@\textbf{Monte Cristallo}, \emph{Berg}|pw}}{ }ſich über die literariſche Bedeutung von \textsc{Hoffmannsthal\pwindex{Hofmannsthal, Hugo von 1.\,2.\,1874 Wien – 15.\,7.\,1929 Rodaun@\textsc{Hofmannsthal, Hugo von} (1.\,2.\,1874 Wien – 15.\,7.\,1929 Rodaun), \emph{Schriftsteller}|pw}} und \textsc{Wassermann\pwindex{Wassermann, Jakob 10.\,3.\,1873 Fürth – 1.\,1.\,1934 Altaussee@\textsc{Wassermann, Jakob} (10.\,3.\,1873 Fürth – 1.\,1.\,1934 Altaussee), \emph{Schriftsteller}|pw}} zu unterhalten, hat keinen beſonderen Reiz. Geſtern bin ich gekommen, und heut möchte
               ich{ }ſchon wieder fort. Aber ich muß bis Sonntag
               feſtſitzen und hoffe nur, daß Du es mir durch Auffindung eines hohen und kühlen
               Aufenthaltsortes dann wenigſtens möglich\strikeout{\textcolor{gray}{ſt}} machſt zu Dir zu kommen.\pend
           
\pstart
           Ich grüße Dich und die Begleiterinnen\pwindex{Schnitzler, Olga 17.\,1.\,1882 Wien – 13.\,1.\,1970 Lugano@\textsc{Schnitzler, Olga} (17.\,1.\,1882 Wien – 13.\,1.\,1970 Lugano), \emph{Schauspielerin, Sängerin}|pwv}\pwindex{Steinrück, Elisabeth 19.\,11.\,1885 – 7.\,4.\,1920 Partenkirchen@\textsc{Steinrück, Elisabeth} (19.\,11.\,1885 – 7.\,4.\,1920 Partenkirchen)|pwv} vielmals und herzlichſt. {\\[\baselineskip]}Dein {\\[\baselineskip]}\spacefill\mbox{Paul Goldmann}\pend
           \leftskip=0em{}
\pstart
           \noindent{}Adreſſe: \textsc{Landro\oindex{Höhlenstein@\textbf{Höhlenstein}|pw}}, \textsc{Hôtel Baur\oindex{Hotel Baur@\textbf{Hotel Baur}, \emph{Hotel}|pw}}.\pend
           \selectlanguage{ngerman}\endnumbering\briefempfaengerindex{Schnitzler, Arthur@\textsc{Schnitzler, Arthur}!zzzGoldmann, Paul@\emph{von Paul Goldmann}!1901-08-051@{5. 8. [1901]}|)be}\mylabel{L03077h}  \newcommand{\dateiname}{L03077}\newcommand{\titel}{Paul Goldmann an Arthur Schnitzler, 5. 8. [1901]}\newcommand{\editorInnen}{Martin Anton Müller und Laura Untner}%% latex-leseansicht-abspann.tex
%% Abspann für die Leseansicht.
%% Der Schalter \ifkorrekturansicht ist bereits durch den Vorspann gesetzt.

%% latex-abspann.tex
%% Gemeinsamer Abspann für Korrekturansicht und Leseansicht.
%% Setzt den Schalter \ifkorrekturansicht voraus (gesetzt in den
%% einbindenden Dateien latex-korrekturansicht-abspann.tex bzw.
%% latex-leseansicht-abspann.tex).
%% ---------------------------------------------------------------

\normalsize

% Das esempio-Environment wird nur in der Leseansicht benötigt
\ifkorrekturansicht\else
\newenvironment{esempio}[3]%
{
    \vspace{1.5ex}
    \rlap{\underline{#1}}
    \par
    \setlength{\parindent}{0cm}
    \nopagebreak
    \leftskip=#2cm
    \rightskip=#3cm
}
{
    \par
}
\fi

\doendnotes{C}
\bigskip
\vfill

\clearpage

\footnotesize

\ifkorrekturansicht
  \lohead{\textsc{register}}
\fi

% theindex-Environment neu definieren ohne reledmac
\makeatletter
\renewenvironment{theindex}{%
  \ifkorrekturansicht
    \section*{\indexname}%
  \else
    \subsubsection*{Index der erwähnten Entitäten}%
  \fi
  \setlength{\parindent}{0pt}%
  \setlength{\parskip}{0pt plus 0.3pt}%
  \let\item\@idxitem
}{%
  \ifkorrekturansicht\clearpage\fi
}
\makeatother

\IfFileExists{\jobname-pw.ind}{\input{\jobname-pw.ind}}{}

% Quellenangabe nur in der Leseansicht
\ifkorrekturansicht\else
% Fallback-Definitionen, falls die .tex-Datei \titel etc. nicht gesetzt hat
\providecommand{\titel}{}
\providecommand{\editorInnen}{}
\providecommand{\dateiname}{\jobname}

\vspace{3cm}

\vfill

\footnotesize
\textsc{Quelle}: \titel. Herausgegeben von {\editorInnen}. In: \emph{Arthur Schnitzler: Briefwechsel mit Autorinnen und Autoren}.
 Digitale Edition, https://schnitzler-briefe.acdh.oeaw.ac.at/{\dateiname}.html (Stand \today)
\fi

\end{document}


