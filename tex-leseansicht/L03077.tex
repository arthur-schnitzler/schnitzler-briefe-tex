%% latex-leseansicht-vorspann.tex
%% Vorspann für die Leseansicht.
%% Lädt die gemeinsame Datei latex-vorspann.tex mit nicht gesetztem Schalter.

\newif\ifkorrekturansicht
\korrekturansichtfalse

\input{../tex-inputs/latex-vorspann}


         
         \renewcommand{\erwaehntePersonen}{Personen: Richard Beer-Hofmann, Paul Goldmann, Hugo von Hofmannsthal, Olga Schnitzler, Nanette Speyer, Albert Speyer, Elisabeth Steinrück, Jakob Wassermann}
         \renewcommand{\erwaehnteOrte}{Orte: Hotel Baur, Höhlenstein, Klobenstein, Monte Cristallo, Vahrn, Wien}
         \renewcommand{\erwaehnteWerke}{}
               \section[ Paul Goldmann an Arthur Schnitzler, 5. 8. {[}1901{]}]{ Paul Goldmann an Arthur Schnitzler, 5. 8. {[}1901{]}}\nopagebreak\mylabel{v}\rehead{ }\begin{ledgroupsized}[t]{13cm}\normalsize\beginnumbering\briefempfaengerindex{Schnitzler, Arthur@\textsc{Schnitzler, Arthur}!zzzGoldmann, Paul@\emph{von Paul Goldmann}!1901-08-051@{5. 8. {[}1901{]}}|(be} \toendnotes[C]{\smallbreak\pagebreak[2]} \Standort{DLA, A:Schnitzler, HS.NZ85.1.3171.}
\physDesc{Brief, 1 Blatt, 2 Seiten, 1185 Zeichen
\newline{}Handschrift: schwarze Tinte, deutsche Kurrent
\newline{}Schnitzler: 1) mit Bleistift das Jahr »901« vermerkt  2) mit rotem Buntstift vier Unterstreichungen}\toendnotes[C]{\smallbreak}\pstart
           \centering{}{\pb}\textsc{Landro\oindex{Hoehlenstein@\textbf{Höhlenstein}|pw}}, 5. Auguſt.\pend
           \pstart{}Mein lieber Freund,\pend\pstart
           \label{K_L03077-1v}\edtext{\textsc{Richard\pwindex{Beer-Hofmann, Richard 1866-07-11 – 1945-09-26@\textsc{Beer-Hofmann, Richard} (1866-07-11 – 1945-09-26), \emph{Schriftsteller}|pw}s} Telegramm}{\lemma{\textnormal{\emph{Richards Telegramm}}}\Cendnote{\textnormal{Daraus ist zu schließen, dass Beer-Hofmann\pwindex{Beer-Hofmann, Richard 1866-07-11 – 1945-09-26@\textsc{Beer-Hofmann, Richard} (1866-07-11 – 1945-09-26), \emph{Schriftsteller}|pwk} das Telegramm vom 1. 8. 1901 am 2. 8. 1901, als er mit Goldmann\pwindex{Goldmann, Paul 31.01.1865 – 25.09.1935@\textsc{Goldmann, Paul} (31.01.1865 – 25.09.1935), \emph{Schriftsteller, Journalist}|pwk} persönlich zusammentraf, noch nicht
                  erhalten hatte.}}}\label{K_L03077-1h}, in dem er mir mittheilte, daß Du einen hochgelegenen Ort\oindex{Klobenstein@\textbf{Klobenstein}|pwv} gefunden, erreichte mich
               leider zu spät. Ich hatte mich bereits in \textsc{Landro\oindex{Hoehlenstein@\textbf{Höhlenstein}|pw}} eingemiethet; ein Zimmer hatte ich in dem \textsc{Hôtel\oindex{Hotel Baur@\textbf{Hotel Baur}|pwv}} nämlich nur \strikeout{\textcolor{gray}{×}\-\textcolor{gray}{×}} unter der Bedingung bekommen, daß ich mindeſtens eine Woche zu bleiben mich
               verpflichtete. So werde ich alſo nicht vor Ablauf dieſer Woche \label{K_L03077-2v}\edtext{zu Dir kommen}{\lemma{\textnormal{\emph{zu Dir kommen}}}\Cendnote{\textnormal{siehe Paul Goldmann an Arthur Schnitzler, 26. 4. [1901]}}}\label{K_L03077-2h} können, und ich bitte\substVorne{}\textsuperscript{,}\substDazwischen{} D\substHinten{}ich, mich ſogleich von Deinem Aufenthaltsort zu \label{K_L03077-3v}\edtext{beſtändigen}{\lemma{\textnormal{\emph{beſtändigen}}}\Cendnote{\textnormal{Schreibirrtum, Goldmann\pwindex{Goldmann, Paul 31.01.1865 – 25.09.1935@\textsc{Goldmann, Paul} (31.01.1865 – 25.09.1935), \emph{Schriftsteller, Journalist}|pwk} meinte wohl
                  »verständigen«}}}\label{K_L03077-3h}. Die kühle und ſtarke Luft hier bekommt mir gut; die trüben
               Gedanken vermag freilich keine noch ſo kühle Luft zu bannen. Ich hatte gehofft, hier
               ein paar liebe Wien\oindex{Wien@\textbf{Wien}|pw}er Mädeln zu finden. Aber es
               iſt nichts vorhanden als die Familie {\pb}\textsc{Speyer\pwindex{Speyer, Nanette 05.01.1846 – 15.1.1925@\textsc{Speyer, Nanette} (05.01.1846 – 15.1.1925)|pw}\pwindex{Speyer, Albert 08.04.1836 – 25.03.1905@\textsc{Speyer, Albert} (08.04.1836 – 25.03.1905), \emph{Industrieller}|pw}}. Und angeſichts des \textsc{Monte Cristallo\oindex{Monte Cristallo@\textbf{Monte Cristallo}|pw}} ſich über die literariſche Bedeutung von \textsc{Hoffmannsthal\pwindex{Hofmannsthal, Hugo von 1874-02-01 – 1929-07-15@\textsc{Hofmannsthal, Hugo von} (1874-02-01 – 1929-07-15), \emph{Schriftsteller}|pw}} und \textsc{Wassermann\pwindex{Wassermann, Jakob 10.03.1873 – 01.01.1934@\textsc{Wassermann, Jakob} (10.03.1873 – 01.01.1934), \emph{Schriftsteller}|pw}} zu unterhalten, hat keinen beſonderen Reiz. Geſtern bin ich gekommen, und heut möchte
               ich ſchon wieder fort. Aber ich muß bis Sonntag
               feſtſitzen und hoffe nur, daß Du es mir durch Auffindung eines hohen und kühlen
               Aufenthaltsortes dann wenigſtens möglich\strikeout{\textcolor{gray}{ſt}} machſt zu Dir zu kommen.\pend
           \pstart
           Ich grüße Dich und die Begleiterinnen\pwindex{Schnitzler, Olga 17.01.1882 – 13.01.1970@\textsc{Schnitzler, Olga} (17.01.1882 – 13.01.1970), \emph{Schauspielerin, Sängerin}|pwv}\pwindex{Steinrueck, Elisabeth 19.11.1885 – 07.04.1920@\textsc{Steinrück, Elisabeth} (19.11.1885 – 07.04.1920)|pwv} vielmals und herzlichſt. {\\[\baselineskip]}Dein {\\[\baselineskip]}\spacefill\mbox{Paul Goldmann}\pend
           \leftskip=0em{}\pstart
           \noindent{}Adreſſe: \textsc{Landro\oindex{Hoehlenstein@\textbf{Höhlenstein}|pw}}, \textsc{Hôtel Baur\oindex{Hotel Baur@\textbf{Hotel Baur}|pw}}.\pend
           
         
         \endnumbering\mylabel{h}\end{ledgroupsized}  \newcommand{\dateiname}{L03077}\newcommand{\titel}{Paul Goldmann an Arthur Schnitzler, 5. 8. [1901]}\newcommand{\editorInnen}{Martin Anton Müller und Laura Untner}%% latex-leseansicht-abspann.tex
%% Abspann für die Leseansicht.
%% Der Schalter \ifkorrekturansicht ist bereits durch den Vorspann gesetzt.

%% latex-abspann.tex
%% Gemeinsamer Abspann für Korrekturansicht und Leseansicht.
%% Setzt den Schalter \ifkorrekturansicht voraus (gesetzt in den
%% einbindenden Dateien latex-korrekturansicht-abspann.tex bzw.
%% latex-leseansicht-abspann.tex).
%% ---------------------------------------------------------------

\normalsize

% Das esempio-Environment wird nur in der Leseansicht benötigt
\ifkorrekturansicht\else
\newenvironment{esempio}[3]%
{
    \vspace{1.5ex}
    \rlap{\underline{#1}}
    \par
    \setlength{\parindent}{0cm}
    \nopagebreak
    \leftskip=#2cm
    \rightskip=#3cm
}
{
    \par
}
\fi

\doendnotes{C}
\bigskip
\vfill

\clearpage

\footnotesize

\ifkorrekturansicht
  \lohead{\textsc{register}}
\fi

% theindex-Environment neu definieren ohne reledmac
\makeatletter
\renewenvironment{theindex}{%
  \ifkorrekturansicht
    \section*{\indexname}%
  \else
    \subsubsection*{Index der erwähnten Entitäten}%
  \fi
  \setlength{\parindent}{0pt}%
  \setlength{\parskip}{0pt plus 0.3pt}%
  \let\item\@idxitem
}{%
  \ifkorrekturansicht\clearpage\fi
}
\makeatother

\IfFileExists{\jobname-pw.ind}{\input{\jobname-pw.ind}}{}

% Quellenangabe nur in der Leseansicht
\ifkorrekturansicht\else
% Fallback-Definitionen, falls die .tex-Datei \titel etc. nicht gesetzt hat
\providecommand{\titel}{}
\providecommand{\editorInnen}{}
\providecommand{\dateiname}{\jobname}

\vspace{3cm}

\vfill

\footnotesize
\textsc{Quelle}: \titel. Herausgegeben von {\editorInnen}. In: \emph{Arthur Schnitzler: Briefwechsel mit Autorinnen und Autoren}.
 Digitale Edition, https://schnitzler-briefe.acdh.oeaw.ac.at/{\dateiname}.html (Stand \today)
\fi

\end{document}


      