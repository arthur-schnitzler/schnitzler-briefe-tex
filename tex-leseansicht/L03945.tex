%% latex-leseansicht-vorspann.tex
%% Vorspann für die Leseansicht.
%% Lädt die gemeinsame Datei latex-vorspann.tex mit nicht gesetztem Schalter.

\newif\ifkorrekturansicht
\korrekturansichtfalse

\input{../tex-inputs/latex-vorspann}


\section[Arthur Schnitzler an Berta Zuckerkandl, 8. 1. 1912]{L03945 Arthur Schnitzler an Berta Zuckerkandl, 8. 1. 1912}
\nopagebreak\mylabel{L03945v}
\rehead{ }\normalsize\beginnumbering\briefempfaengerindex{Zuckerkandl, Berta@\textsc{Zuckerkandl, Berta}!zzzSchnitzler, Arthur@\emph{von Arthur Schnitzler}!1912-01-081@{8. 1. 1912}|(be}
\toendnotes[C]{\smallbreak\pagebreak[2]}
\correspDesc{Versand  durch Arthur Schnitzler am 8. 1. 1912 in Wien
\newline{}Erhalt  durch Berta Zuckerkandl im Zeitraum [8. 1. 1912 –
            11. 1. 1912?] in Wien}\toendnotes[C]{\smallbreak}
\Standort{DLA, HS.1985.1.2282.}
\physDesc{Brief, Durchschlag, 1 Blatt, 2 Seiten, 989 Zeichen
\newline{}Schreibmaschine
\newline{}Handschrift: roter Buntstift, lateinische Kurrent (\noindent{}beschriftet: »\uline{Zuckerkandl}« und »Frk«, sechs Unterstreichungen)}\toendnotes[C]{\smallbreak}
\pstart
           \raggedleft{}{\pb}8. 1. 1912.\pend
           
\pstart\center{}Verehrte gnädige Frau.\pend\vspace{0.5em}
\pstart
           Aus dem \label{K_L03945-1v}\edtext{beiliegenden Schreiben}{\lemma{\textnormal{\emph{beiliegenden Schreiben}}}\Cendnote{\textnormal{nicht
            überliefert}}}\label{K_L03945-1},
          dessen gelegentliche freundliche Rücksendung ich erbitte, ersehen Sie, dass Frau Lefreve\pwindex{Lefèvre, A. @\textsc{Lefèvre, A.}, \emph{Übersetzerin}|pw} das »Weite
            Land\pwindex{Schnitzler, Arthur 15. 5. 1862 Wien – 21. 10. 1931 ebd.@\textsc{Schnitzler, Arthur} (15. 5. 1862 Wien – 21. 10. 1931 ebd.), \emph{Schriftsteller, Mediziner}!weite Land. Tragikomödie in fünf Akten@\strich\emph{Das weite Land. Tragikomödie in fünf Akten}|pw}« nicht übersetzen will. Indess hat mir \label{K_L03945-2v}\edtext{auch Herr Remon\pwindex{Rémon, Maurice 27.\,11.\,1861 Paris – 20.\,6.\,1945 Mérignac@\textsc{Rémon, Maurice} (27.\,11.\,1861 Paris – 20.\,6.\,1945 Mérignac), \emph{Übersetzer}|pw}}{\lemma{\textnormal{\emph{auch Herr Remon}}}\Cendnote{\textnormal{Maurice
              Remon\pwindex{Rémon, Maurice 27.\,11.\,1861 Paris – 20.\,6.\,1945 Mérignac@\textsc{Rémon, Maurice} (27.\,11.\,1861 Paris – 20.\,6.\,1945 Mérignac), \emph{Übersetzer}|pwk} an Arthur Schnitzler,
              20. 12. 1911, vgl. Karl Zieger: \emph{Arthur
                Schnitzler et la France 1894–1938. Enquête sur une réception},
              Villeneuve d’Ascq: \emph{Presses Universitaires du
                Septentrion} 2012, S. 191.
            }}}\label{K_L03945-2}, derjenige der sich auf seine
          persönlichen zu Guitry\pwindex{Guitry, Lucien 13.\,12.\,1860 Paris – 1.\,6.\,1925 ebd.@\textsc{Guitry, Lucien} (13.\,12.\,1860 Paris – 1.\,6.\,1925 ebd.), \emph{Schriftsteller, Schauspieler}|pw} berief, seine Zweifel
          hinsichtlich der Chancen meines Stückes\pwindex{Schnitzler, Arthur 15. 5. 1862 Wien – 21. 10. 1931 ebd.@\textsc{Schnitzler, Arthur} (15. 5. 1862 Wien – 21. 10. 1931 ebd.), \emph{Schriftsteller, Mediziner}!weite Land. Tragikomödie in fünf Akten@\strich\emph{Das weite Land. Tragikomödie in fünf Akten}|pwv} für Paris\oindex{Paris@\textbf{Paris}, \emph{Hauptstadt}|pw} ausgedrückt und so werde ich
          wohl die Uebersetzung\pwindex{Schnitzler, Arthur 15. 5. 1862 Wien – 21. 10. 1931 ebd.@\textsc{Schnitzler, Arthur} (15. 5. 1862 Wien – 21. 10. 1931 ebd.), \emph{Schriftsteller, Mediziner}!Le Pays Inconnu@\strich\emph{Le Pays Inconnu}|pwv} Herrn Morisse\pwindex{Morisse, Paul 11.\,3.\,1866 Rouen – 28.\,9.\,1946 Paris@\textsc{Morisse, Paul} (11.\,3.\,1866 Rouen – 28.\,9.\,1946 Paris), \emph{Übersetzer}|pw} übertragen, wenn der es nicht auch vorzieht
          abzurücken. Aus einer \label{K_L03945-3v}\edtext{Notiz\pwindex{P. C. @\textsc{P. C.}, \emph{Journalist/Journalistin}!Bei Le Bargy@\strich\emph{Bei Le Bargy}|pwv} im Neuen Wiener Journal\pwindex{Neues Wiener Journal@\emph{Neues Wiener Journal}|pw}}{\lemma{\textnormal{\emph{Notiz … Journal}}}\Cendnote{\textnormal{P. C.\pwindex{P. C. @\textsc{P. C.}, \emph{Journalist/Journalistin}|pwk}: \emph{Bei Le
                Bargy}\pwindex{P. C. @\textsc{P. C.}, \emph{Journalist/Journalistin}!Bei Le Bargy@\strich\emph{Bei Le Bargy}|pwk}. In: \emph{Neues Wiener Journal}\pwindex{Neues Wiener Journal@\emph{Neues Wiener Journal}|pwk}, Jg. 20,
              Nr. 6539, 6. 1. 1912, S. 3–4. Darin wird geschildert, wie Le Bargy\pwindex{Le Bargy, Charles Gustave 28.\,7.\,1858 La Chapelle – 5.\,2.\,1936 Nizza@\textsc{Le Bargy, Charles Gustave} (28.\,7.\,1858 La Chapelle – 5.\,2.\,1936 Nizza), \emph{Schauspieler}|pwk} im \emph{Burgtheater}\orgindex{Burgtheater@Burgtheater|pwk} eine Aufführung von \emph{Das weite
              Land}\pwindex{Schnitzler, Arthur 15. 5. 1862 Wien – 21. 10. 1931 ebd.@\textsc{Schnitzler, Arthur} (15. 5. 1862 Wien – 21. 10. 1931 ebd.), \emph{Schriftsteller, Mediziner}!weite Land. Tragikomödie in fünf Akten@\strich\emph{Das weite Land. Tragikomödie in fünf Akten}|pwk} besuchte und genoss, ohne deutsch zu können.}}}\label{K_L03945-3} und was wohl
          massgebender ist auch von \label{K_L03945-4v}\edtext{privater Seite\pwindex{Trebitsch, Siegfried 22.\,12.\,1868 Wien – 3.\,6.\,1956 Zürich@\textsc{Trebitsch, Siegfried} (22.\,12.\,1868 Wien – 3.\,6.\,1956 Zürich), \emph{Schriftsteller, Übersetzer}|pwv} habe ich erfahren,
          dass Le Bargy\pwindex{Le Bargy, Charles Gustave 28.\,7.\,1858 La Chapelle – 5.\,2.\,1936 Nizza@\textsc{Le Bargy, Charles Gustave} (28.\,7.\,1858 La Chapelle – 5.\,2.\,1936 Nizza), \emph{Schauspieler}|pw}}{\lemma{\textnormal{\emph{privater … Bargy}}}\Cendnote{\textnormal{Vgl. A. S.: \emph{Tagebuch}, 6. 1. 1912.}}}\label{K_L03945-4} sich sehr
          lebhaft für das Stück\pwindex{Schnitzler, Arthur 15. 5. 1862 Wien – 21. 10. 1931 ebd.@\textsc{Schnitzler, Arthur} (15. 5. 1862 Wien – 21. 10. 1931 ebd.), \emph{Schriftsteller, Mediziner}!weite Land. Tragikomödie in fünf Akten@\strich\emph{Das weite Land. Tragikomödie in fünf Akten}|pwv}
            interessiert{[}.{]} Er hat es hier gesehen. Da er nicht deutsch versteht,
          ist diese Interesse keine besondere Bedeutung beizulegen; immerhin wäre zu überlegen, ob
          man mit ihm nicht in eine persönliche Verbindung treten könnte, wenn er im
            März hier gastieren wird.\pend
           
\pstart
           Bald hoffe ich Gelegenheit zu haben Sie {\pb}wiederzusehen
          und bin mit herzlichen Grüssen{\\[\baselineskip]} Ihr sehr ergebener\pend
           \leftskip=0em{}{\vspace{1\baselineskip}}
\pstart
           \noindent{}Frau Berta Zuckerkandl, Wien\oindex{Wien@\textbf{Wien}, \emph{Verwaltungsgebiet}|pw}.\pend
           \selectlanguage{ngerman}\endnumbering\briefempfaengerindex{Zuckerkandl, Berta@\textsc{Zuckerkandl, Berta}!zzzSchnitzler, Arthur@\emph{von Arthur Schnitzler}!1912-01-081@{8. 1. 1912}|)be}\mylabel{L03945h}
\begin{anhang}
\end{anhang}\newcommand{\dateiname}{L03945}\newcommand{\titel}{Arthur Schnitzler an Berta Zuckerkandl, 8. 1. 1912}\newcommand{\editorInnen}{Herausgegeben von Jahnke, SelmaMüller, Martin Anton}%% latex-leseansicht-abspann.tex
%% Abspann für die Leseansicht.
%% Der Schalter \ifkorrekturansicht ist bereits durch den Vorspann gesetzt.

%% latex-abspann.tex
%% Gemeinsamer Abspann für Korrekturansicht und Leseansicht.
%% Setzt den Schalter \ifkorrekturansicht voraus (gesetzt in den
%% einbindenden Dateien latex-korrekturansicht-abspann.tex bzw.
%% latex-leseansicht-abspann.tex).
%% ---------------------------------------------------------------

\normalsize

% Das esempio-Environment wird nur in der Leseansicht benötigt
\ifkorrekturansicht\else
\newenvironment{esempio}[3]%
{
    \vspace{1.5ex}
    \rlap{\underline{#1}}
    \par
    \setlength{\parindent}{0cm}
    \nopagebreak
    \leftskip=#2cm
    \rightskip=#3cm
}
{
    \par
}
\fi

\doendnotes{C}
\bigskip
\vfill

\clearpage

\footnotesize

\ifkorrekturansicht
  \lohead{\textsc{register}}
\fi

% theindex-Environment neu definieren ohne reledmac
\makeatletter
\renewenvironment{theindex}{%
  \ifkorrekturansicht
    \section*{\indexname}%
  \else
    \subsubsection*{Index der erwähnten Entitäten}%
  \fi
  \setlength{\parindent}{0pt}%
  \setlength{\parskip}{0pt plus 0.3pt}%
  \let\item\@idxitem
}{%
  \ifkorrekturansicht\clearpage\fi
}
\makeatother

\IfFileExists{\jobname-pw.ind}{\input{\jobname-pw.ind}}{}

% Quellenangabe nur in der Leseansicht
\ifkorrekturansicht\else
% Fallback-Definitionen, falls die .tex-Datei \titel etc. nicht gesetzt hat
\providecommand{\titel}{}
\providecommand{\editorInnen}{}
\providecommand{\dateiname}{\jobname}

\vspace{3cm}

\vfill

\footnotesize
\textsc{Quelle}: \titel. Herausgegeben von {\editorInnen}. In: \emph{Arthur Schnitzler: Briefwechsel mit Autorinnen und Autoren}.
 Digitale Edition, https://schnitzler-briefe.acdh.oeaw.ac.at/{\dateiname}.html (Stand \today)
\fi

\end{document}


