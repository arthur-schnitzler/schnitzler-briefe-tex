%% latex-korrekturansicht-vorspann.tex
%% Vorspann für die Korrekturansicht.
%% Lädt die gemeinsame Datei latex-vorspann.tex mit gesetztem Schalter.

\newif\ifkorrekturansicht
\korrekturansichttrue

\input{../tex-inputs/latex-vorspann}


\section[ Felix Salten an Arthur Schnitzler, {[}1. 8. 1895{]}]{L03161 Felix Salten an Arthur Schnitzler, {[}1. 8. 1895{]}}
\nopagebreak\mylabel{L03161v}
\rehead{ }\normalsize\beginnumbering\briefempfaengerindex{Schnitzler, Arthur@\textsc{Schnitzler, Arthur}!zzzSalten, Felix@\emph{von Felix Salten}!1895-08-011@{{[}1. 8. 1895{]}}|(be}
\toendnotes[C]{\smallbreak\pagebreak[2]}\Standort{CUL, Schnitzler, B 89, A 1.}
\physDesc{Brief, 2 Blätter, 5 Seiten, 1678 Zeichen
\newline{}Handschrift: Bleistift, lateinische Kurrent
\newline{}Schnitzler: mit Bleistift datiert: »1/8 95.« 
\newline{}Ordnung: mit Bleistift von unbekannter Hand nummeriert: »61« }\toendnotes[C]{\smallbreak}
\pstart
           \noindent{}{\pb}Lieber Freund, ich bin, wenn ich das \label{K_L03161-1v}\edtext{B.}{\lemma{\textnormal{\emph{B.}}}\Cendnote{\textnormal{Bicycle, Fahrrad. Es ging um die bevorstehende gemeinsame
                  Radtour, siehe Felix Salten an Arthur Schnitzler, 22. 7. 1895.}}}\label{K_L03161-1} nur
               sonst in Ordnung habe, mit all dem einverstanden bis auf München\oindex{Muenchen@\textbf{München}, \emph{P.PPLA}|pw}. Das werden wir aber \label{K_L03161-2v}\edtext{Montag, wenn ich zu Ihnen komme}{\lemma{\textnormal{\emph{Montag, … komme}}}\Cendnote{\textnormal{Vgl. A. S.: \emph{Tagebuch}, 5. 8. 1895.
               }}}\label{K_L03161-2} noch näher besprechen.\pend
           
\pstart
           Was die \label{K_L03161-3v}\edtext{Feuill.\pwindex{Muenchener Kunstausstellungen. I. Im koenigl. Glaspalast@\emph{Die Münchener Kunstausstellungen. I. Im königl. Glaspalast}|pwv}\pwindex{Muenchener Kunstausstellungen. II. Im koenigl. Glaspalast@\emph{Die Münchener Kunstausstellungen. II. Im königl. Glaspalast}|pwv}\pwindex{Muenchener Brief. (Orig.-Corr. der »Wiener Allg. Ztg.«)@\emph{Münchener Brief. (Orig.-Corr. der »Wiener Allg. Ztg.«)}|pwv}}{\lemma{\textnormal{\emph{Feuill.}}}\Cendnote{\textnormal{Siehe Felix Salten an Arthur Schnitzler, 22. 7. 1895 und [30. 7. 1895].
               }}}\label{K_L03161-3} betrifft, so hätten sie – 
               wie \label{K_L03161-4v}\edtext{Speiszetteln}{\lemma{\textnormal{\emph{Speiszetteln}}}\Cendnote{\textnormal{Menükarten}}}\label{K_L03161-4} ausgesehen, wenn ich mehr
               Bilder genannt hätte. Ich wollte also {\pb}nur wichtige Stationen geben,
               die gewissermaßen die durchwanderte Gegend charakterisiren. Dann schrieb ich doch
               auch für Leute, welche München\oindex{Muenchen@\textbf{München}, \emph{P.PPLA}|pw} nicht gesehen
               haben, ich möchte also mehr beschreiben, als unkontrollirbare Kritik üben. Die Secession\orgindex{Muenchener Secession@Münchener Secession|pw} erhält übrigens noch ein \label{K_L03161-5v}\edtext{zweites (sachlicheres) Feuilleton\pwindex{Muenchener Kunstausstellungen. IV. Die Secession@\emph{Die Münchener Kunstausstellungen. IV. Die Secession}|pwv}}{\lemma{\textnormal{\emph{zweites … Feuilleton}}}\Cendnote{\textnormal{Felix Salten\pwindex{Salten, Felix 06.09.1869 – 08.10.1945@\textsc{Salten, Felix} (06.09.1869 – 08.10.1945), \emph{Schriftsteller/Schriftstellerin, Journalist/Journalistin, Chefredakteur/Chefredakteurin}|pwk}: \emph{Die Münchener Kunstausstellungen. IV. Die Secession}\pwindex{Muenchener Kunstausstellungen. IV. Die Secession@\emph{Die Münchener Kunstausstellungen. IV. Die Secession}|pwk}.
                     In: \emph{Wiener Allgemeine Zeitung}\pwindex{Wiener Allgemeine Zeitung@\emph{Wiener Allgemeine Zeitung}|pwk}, Nr. 5234,
                        15. 8. 1895, S. 8.}}}\label{K_L03161-5}.\pend
           
\pstart
           Dass Ihr \label{K_L03161-6v}\edtext{Theaterleben Sie {\pb}in Freiwild\pwindex{Freiwild. Schauspiel in 3 Akten@\emph{Freiwild. Schauspiel in 3 Akten}|pw}{ }\substVorne{}\textsuperscript{stört}\substDazwischen{}hätte stören können\substHinten{}{ }}{\lemma{\textnormal{\emph{Theaterleben … können}}}\Cendnote{\textnormal{Schnitzler stand kurz vor der Fertigstellung
                  des 1. Akts von \emph{Freiwild}\pwindex{Freiwild. Schauspiel in 3 Akten@\emph{Freiwild. Schauspiel in 3 Akten}|pwk} am 2. 8. 1895. Die in
                  diesem Stück behandelte Theaterprostitution, die er durch seinen Kontakt mit
                  Schauspielerinnen aus eigener Anschauung kannte, dürfte narrative Zweifel in ihm
                  ausgelöst haben. }}}\label{K_L03161-6}, ist sonderbar. Es kommt ja garnicht darauf an, dass
               diese Mädeln Männer fangen wollen, sondern auf die Umstände, die ihnen ein solches
               Leben zur Notwendigkeit machen. Dass sich manche willig manche mit vielem Geschick
               darin finden, ändert doch an der Freiwild\pwindex{Freiwild. Schauspiel in 3 Akten@\emph{Freiwild. Schauspiel in 3 Akten}|pw}-\uline{Idee} nicht das mindeste, selbst dann nicht wenn man
               gelegentlich wirklich der Jäger wäre.\pend
           
\pstart
           =\pend
           
\pstart
           Meine \label{K_L03161-7v}\edtext{Tochter\pwindex{Lamberg, Maria Charlotte 1895-03-24 – 1895-07-27@\textsc{Lamberg, Maria Charlotte} (1895-03-24 – 1895-07-27)|pwv}}{\lemma{\textnormal{\emph{Tochter}}}\Cendnote{\textnormal{Das gemeinsame Kind mit Charlotte Glas\pwindex{Pohl-Glas, Charlotte 1873-01-01 – 1944-02-15@\textsc{Pohl-Glas, Charlotte} (1873-01-01 – 1944-02-15), \emph{Schriftsteller/Schriftstellerin, Politiker/Politikerin, Sozialist/Sozialistin}|pwk} trug den Namen Maria Charlotte Lamberg\pwindex{Lamberg, Maria Charlotte 1895-03-24 – 1895-07-27@\textsc{Lamberg, Maria Charlotte} (1895-03-24 – 1895-07-27)|pwk} und war gerade vier
                  Monate alt, als es am 27. 7. 1895 bei der Kostfrau\pwindex{?? [Kostfrau von Charlotte Lamberg] @\textsc{?? [Kostfrau von Charlotte Lamberg]}|pwkv} in Gerasdorf bei Wien\oindex{Gerasdorf bei Wien@\textbf{Gerasdorf bei Wien}, \emph{A.ADM3}|pwk} verstarb.}}}\label{K_L03161-7} ist
               gestorben. Damit fällt {\pb}ein
               starkes Band zwischen Lotte\pwindex{Pohl-Glas, Charlotte 1873-01-01 – 1944-02-15@\textsc{Pohl-Glas, Charlotte} (1873-01-01 – 1944-02-15), \emph{Schriftsteller/Schriftstellerin, Politiker/Politikerin, Sozialist/Sozialistin}|pw} u. mir. Als die
               alte Frau\pwindex{?? [Kostfrau von Charlotte Lamberg] @\textsc{?? [Kostfrau von Charlotte Lamberg]}|pwv}, welche mir die
               Nachricht brachte, mit Thränen an meinem Redaction\orgindex{Wiener Allgemeine Zeitung@Wiener Allgemeine Zeitung|pwv}stisch saß, und ich an die Fahrt mit Lotte\pwindex{Pohl-Glas, Charlotte 1873-01-01 – 1944-02-15@\textsc{Pohl-Glas, Charlotte} (1873-01-01 – 1944-02-15), \emph{Schriftsteller/Schriftstellerin, Politiker/Politikerin, Sozialist/Sozialistin}|pw} nach Gerasdorf\oindex{Gerasdorf bei Wien@\textbf{Gerasdorf bei Wien}, \emph{A.ADM3}|pw}, an den kleinen Friedhof\oindex{Friedhof Gerasdorf@\textbf{Friedhof Gerasdorf}, \emph{Friedhof (K.FRH)}|pw}{[},{]} an den Kranz, den wir mitnehmen werden, und an das Kreuz,
               welches wir draußen kaufen werden, dachte, musste ich gleich daran denken, wie
               prachtvoll das alles für die Novelle passt. BeerH.\pwindex{Beer-Hofmann, Richard 1866-07-11 – 1945-09-26@\textsc{Beer-Hofmann, Richard} (1866-07-11 – 1945-09-26), \emph{Schriftsteller/Schriftstellerin}|pw} wird sagen, es ist sein\strikeout{\textcolor{gray}{e}} »Kind\pwindex{Kind@\emph{Das Kind}|pw}«. Viel {\pb}davon ist ja dabei, aber es
               ist doch etwas ganz, ganz anderes, wenn man die Gestalt der Lotte\pwindex{Pohl-Glas, Charlotte 1873-01-01 – 1944-02-15@\textsc{Pohl-Glas, Charlotte} (1873-01-01 – 1944-02-15), \emph{Schriftsteller/Schriftstellerin, Politiker/Politikerin, Sozialist/Sozialistin}|pw}, die \label{K_L03161-8v}\edtext{München\oindex{Muenchen@\textbf{München}, \emph{P.PPLA}|pw}er Affaire}{\lemma{\textnormal{\emph{Münchener Affaire}}}\Cendnote{\textnormal{Vgl. Felix Salten an Arthur Schnitzler, 18. 2. 1895.
               }}}\label{K_L03161-8}, und unsere jetzigen Beziehungen nimmt.\pend
           
\pstart
           Leben Sie wol, auf Wiedersehen Montag{ }früh. Herzlich {\\[\baselineskip]}Ihr \spacefill\mbox{Salten}\pend
           \leftskip=0em{}\selectlanguage{ngerman}\endnumbering\briefempfaengerindex{Schnitzler, Arthur@\textsc{Schnitzler, Arthur}!zzzSalten, Felix@\emph{von Felix Salten}!1895-08-011@{{[}1. 8. 1895{]}}|)be}\mylabel{L03161h}  \normalsize

\doendnotes{C}
\bigskip
\vfill

\clearpage

\footnotesize

\lohead{\textsc{register}}

% Definiere theindex-Environment komplett neu ohne reledmac
\makeatletter
\renewenvironment{theindex}{%
  \section*{\indexname}%
  \setlength{\parindent}{0pt}%
  \setlength{\parskip}{0pt plus 0.3pt}%
  \let\item\@idxitem
}{%
  \clearpage
}
\makeatother

\IfFileExists{\jobname-pw.ind}{\input{\jobname-pw.ind}}{}

\end{document}

      