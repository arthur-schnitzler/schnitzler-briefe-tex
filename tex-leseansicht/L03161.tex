%% latex-leseansicht-vorspann.tex
%% Vorspann für die Leseansicht.
%% Lädt die gemeinsame Datei latex-vorspann.tex mit nicht gesetztem Schalter.

\newif\ifkorrekturansicht
\korrekturansichtfalse

\input{../tex-inputs/latex-vorspann}

\begin{center}
            \textcolor{red}{ENTWURF, NICHT FERTIG KORRIGIERT}
                      \end{center}
            
         
         \renewcommand{\erwaehntePersonen}{Personen:  ?? [Kostfrau von Charlotte Lamberg], Richard Beer-Hofmann, Maria Charlotte Lamberg, Charlotte Pohl-Glas}
         \renewcommand{\erwaehnteInstitutionen}{Institutionen: Münchener Secession, Wiener Allgemeine Zeitung}
         \renewcommand{\erwaehnteOrte}{Orte: Friedhof Gerasdorf, Gerasdorf bei Wien, München, Wien}
         \renewcommand{\erwaehnteWerke}{Werke: Das Kind, Die Münchener Kunstausstellungen. I. Im königl. Glaspalast, Die Münchener Kunstausstellungen. II. Im königl. Glaspalast, Die Münchener Kunstausstellungen. IV. Die Secession, Freiwild. Schauspiel in 3 Akten, Münchener Brief. (Orig.-Corr. der »Wiener Allg. Ztg.«), Wiener Allgemeine Zeitung}
               \section[Felix Salten an Arthur Schnitzler, {[}1. 8. 1895{]}]{ Felix Salten an Arthur Schnitzler, {[}1. 8. 1895{]}}\nopagebreak\mylabel{v}\rehead{ }\begin{ledgroupsized}[t]{13cm}\normalsize\beginnumbering \toendnotes[C]{\smallbreak\pagebreak[2]} \Standort{CUL, Schnitzler, B 89, A 1.}
\physDesc{Brief, 2 Blätter, 5 Seiten
\newline{}Handschrift: Bleistift, lateinische Kurrent}\toendnotes[C]{\smallbreak}\pstart
           \noindent{}{\pb}Lieber Freund, ich bin, wenn ich das B. nur sonst in Ordnung habe,
               mit all dem einverstanden bis auf München\oindex{Muenchen@\textbf{München}|pw}. Das
               werden wir aber Montag, wenn ich zu Ihnen komme noch näher besprechen. \pend
           \pstart
           Was die \label{K_L03160-1v}\edtext{Feuill.\pwindex{Salten, Felix 06.09.1869 – 08.10.1945@\textsc{Salten, Felix} (06.09.1869 – 08.10.1945), \emph{Schriftsteller, Journalist}!Muenchener Kunstausstellungen. I. Im koenigl. Glaspalast1895-07-24@\strich\emph{Die Münchener Kunstausstellungen. I. Im königl. Glaspalast} {[}1895-07-24{]}|pwv}\pwindex{Salten, Felix 06.09.1869 – 08.10.1945@\textsc{Salten, Felix} (06.09.1869 – 08.10.1945), \emph{Schriftsteller, Journalist}!Muenchener Kunstausstellungen. II. Im koenigl. Glaspalast1895-07-25@\strich\emph{Die Münchener Kunstausstellungen. II. Im königl. Glaspalast} {[}1895-07-25{]}|pwv}\pwindex{Muenchener Brief. (Orig.-Corr. der »Wiener Allg. Ztg.«)1895-07-06@\emph{Münchener Brief. (Orig.-Corr. der »Wiener Allg. Ztg.«)} {[}1895-07-06{]}|pwv}}{\lemma{\textnormal{\emph{Feuill.}}}\Cendnote{\textnormal{f. s.\pwindex{Salten, Felix 06.09.1869 – 08.10.1945@\textsc{Salten, Felix} (06.09.1869 – 08.10.1945), \emph{Schriftsteller, Journalist}|pwk} [= Felix
                     Salten\pwindex{Salten, Felix 06.09.1869 – 08.10.1945@\textsc{Salten, Felix} (06.09.1869 – 08.10.1945), \emph{Schriftsteller, Journalist}|pwk}]: \emph{Münchener Brief. (Orig.-Corr.
                        der »Wiener Allg. Ztg.«)}\pwindex{Muenchener Brief. (Orig.-Corr. der »Wiener Allg. Ztg.«)1895-07-06@\emph{Münchener Brief. (Orig.-Corr. der »Wiener Allg. Ztg.«)} {[}1895-07-06{]}|pwk}. In: \emph{Wiener
                        Allgemeine Zeitung}\pwindex{?? Werk@Nicht ermittelte Verfasserinnen und Verfasser!Wiener Allgemeine Zeitung1.3.1880 – 11.2.1934@\emph{Wiener Allgemeine Zeitung} {[}1.3.1880 – 11.2.1934{]}|pwk}, Nr. 5.200, 6. 7. 1895,
                     S. 8. Felix Salten\pwindex{Salten, Felix 06.09.1869 – 08.10.1945@\textsc{Salten, Felix} (06.09.1869 – 08.10.1945), \emph{Schriftsteller, Journalist}|pwk}: \emph{Die Münchener Kunstausstellungen. I. Im königl.
                        Glaspalast}\pwindex{Salten, Felix 06.09.1869 – 08.10.1945@\textsc{Salten, Felix} (06.09.1869 – 08.10.1945), \emph{Schriftsteller, Journalist}!Muenchener Kunstausstellungen. I. Im koenigl. Glaspalast1895-07-24@\strich\emph{Die Münchener Kunstausstellungen. I. Im königl. Glaspalast} {[}1895-07-24{]}|pwk}. In: \emph{Wiener Allgemeine
                        Zeitung}\pwindex{?? Werk@Nicht ermittelte Verfasserinnen und Verfasser!Wiener Allgemeine Zeitung1.3.1880 – 11.2.1934@\emph{Wiener Allgemeine Zeitung} {[}1.3.1880 – 11.2.1934{]}|pwk}, Nr. 5.215, 24. 7. 1895,
                     S. 2. Felix Salten\pwindex{Salten, Felix 06.09.1869 – 08.10.1945@\textsc{Salten, Felix} (06.09.1869 – 08.10.1945), \emph{Schriftsteller, Journalist}|pwk}: \emph{Die Münchener Kunstausstellungen. II. Im
                        königl. Glaspalast}\pwindex{Salten, Felix 06.09.1869 – 08.10.1945@\textsc{Salten, Felix} (06.09.1869 – 08.10.1945), \emph{Schriftsteller, Journalist}!Muenchener Kunstausstellungen. II. Im koenigl. Glaspalast1895-07-25@\strich\emph{Die Münchener Kunstausstellungen. II. Im königl. Glaspalast} {[}1895-07-25{]}|pwk}. In: \emph{Wiener
                        Allgemeine Zeitung}\pwindex{?? Werk@Nicht ermittelte Verfasserinnen und Verfasser!Wiener Allgemeine Zeitung1.3.1880 – 11.2.1934@\emph{Wiener Allgemeine Zeitung} {[}1.3.1880 – 11.2.1934{]}|pwk}, Nr. 5.216, 25. 7. 1895,
                     S. 2–3.}}}\label{K_L03160-1h} betrifft, so hätten sie wie Speiszetteln ausgesehen,
               wenn ich mehr Bilder genommen hätte. Ich wollte also {\pb}nur wichtige Stationen geben,
               die gewissermaßen die durchwanderte Gegend charakterisiren. Dann schrieb ich doch
               auch für Leute, welche München\oindex{Muenchen@\textbf{München}|pw} nicht gesehen
               haben, ich möchte also mehr beschreiben, als unkontrollirbare Kritik üben. Die Secession\orgindex{Muenchener Secession@Münchener Secession|pw} erhält übrigens noch ein \label{K_L03161-11v}\edtext{zweites (sachlicheres) Feuilleton\pwindex{Salten, Felix 06.09.1869 – 08.10.1945@\textsc{Salten, Felix} (06.09.1869 – 08.10.1945), \emph{Schriftsteller, Journalist}!Muenchener Kunstausstellungen. IV. Die Secession1895-08-15@\strich\emph{Die Münchener Kunstausstellungen. IV. Die Secession} {[}1895-08-15{]}|pwv}}{\lemma{\textnormal{\emph{zweites … Feuilleton}}}\Cendnote{\textnormal{Felix Salten\pwindex{Salten, Felix 06.09.1869 – 08.10.1945@\textsc{Salten, Felix} (06.09.1869 – 08.10.1945), \emph{Schriftsteller, Journalist}|pwk}:
                        \emph{Die Münchener Kunstausstellungen. IV. Die
                        Secession}\pwindex{Salten, Felix 06.09.1869 – 08.10.1945@\textsc{Salten, Felix} (06.09.1869 – 08.10.1945), \emph{Schriftsteller, Journalist}!Muenchener Kunstausstellungen. IV. Die Secession1895-08-15@\strich\emph{Die Münchener Kunstausstellungen. IV. Die Secession} {[}1895-08-15{]}|pwk}. In: \emph{Wiener Allgemeine
                        Zeitung}\pwindex{?? Werk@Nicht ermittelte Verfasserinnen und Verfasser!Wiener Allgemeine Zeitung1.3.1880 – 11.2.1934@\emph{Wiener Allgemeine Zeitung} {[}1.3.1880 – 11.2.1934{]}|pwk}, Nr. 5.234, 15. 8. 1895,
                     S. 8.}}}\label{K_L03161-11h}. \pend
           \pstart
           Dass Ihr Theaterleben Sie {\pb}\introOben{}hätte stören können\introOben{} in Freiwild\pwindex{Schnitzler, Arthur 15.05.1862 – 21.10.1931@\textsc{Schnitzler, Arthur} (15.05.1862 – 21.10.1931), \emph{Schriftsteller, Mediziner}!Freiwild. Schauspiel in 3 Akten1896@\strich\emph{Freiwild. Schauspiel in 3 Akten} {[}1896{]}|pw}{ }\strikeout{stört}, ist sonderbar. Es kommt ja garnicht darauf an,
               dass diese Mädeln Männer fangen wollen, sondern auf die Umstände, die ihnen ein
               solches Leben zur Notwendigkeit machen. Dass sich manche willig manche mit vielem
               Geschick darin finden, ändert doch an der Freiwild\pwindex{Schnitzler, Arthur 15.05.1862 – 21.10.1931@\textsc{Schnitzler, Arthur} (15.05.1862 – 21.10.1931), \emph{Schriftsteller, Mediziner}!Freiwild. Schauspiel in 3 Akten1896@\strich\emph{Freiwild. Schauspiel in 3 Akten} {[}1896{]}|pw}-\uline{Idee} nicht das mindeste, selbst
               dann nicht, wenn man gelegentlich wirklich der Jäger wäre. \pend
           \pstart
           =\pend
           \pstart
           Meine \label{K_L03161-111v}\edtext{Tochter\pwindex{Lamberg, Maria Charlotte 1895-03-24 – 1895-07-27@\textsc{Lamberg, Maria Charlotte} (1895-03-24 – 1895-07-27)|pwv}}{\lemma{\textnormal{\emph{Tochter}}}\Cendnote{\textnormal{Das gemeinsame Kind mit Charlotte Glas\pwindex{Pohl-Glas, Charlotte 1873-01-01 – 1944-02-15@\textsc{Pohl-Glas, Charlotte} (1873-01-01 – 1944-02-15), \emph{Schriftstellerin, Politikerin, Sozialistin}|pwk} trug den Namen Maria
                     Charlotte Lamberg\pwindex{Lamberg, Maria Charlotte 1895-03-24 – 1895-07-27@\textsc{Lamberg, Maria Charlotte} (1895-03-24 – 1895-07-27)|pwk} und war gerade vier Monate alt, als es am
                     27. 7. 1895 bei der Kostfrau\pwindex{?? [Kostfrau von Charlotte Lamberg] @\textsc{?? [Kostfrau von Charlotte Lamberg]}|pwkv} in Gerasdorf
                     bei Wien\oindex{Gerasdorf bei Wien@\textbf{Gerasdorf bei Wien}|pwk} starb.}}}\label{K_L03161-111h} ist gestorben. Damit fällt {\pb}ein starkes Band zwischen Lotte\pwindex{Pohl-Glas, Charlotte 1873-01-01 – 1944-02-15@\textsc{Pohl-Glas, Charlotte} (1873-01-01 – 1944-02-15), \emph{Schriftstellerin, Politikerin, Sozialistin}|pw} u. mir. Als die alte Frau\pwindex{?? [Kostfrau von Charlotte Lamberg] @\textsc{?? [Kostfrau von Charlotte Lamberg]}|pwv}, welche mir die Nachricht brachte,
               mit Thränen an meinem Redaktionstisch\orgindex{Wiener Allgemeine Zeitung@Wiener Allgemeine Zeitung|pwv} saß, und ich an die Fahrt mit Lotte\pwindex{Pohl-Glas, Charlotte 1873-01-01 – 1944-02-15@\textsc{Pohl-Glas, Charlotte} (1873-01-01 – 1944-02-15), \emph{Schriftstellerin, Politikerin, Sozialistin}|pw} nach Gerasdorf\oindex{Gerasdorf bei Wien@\textbf{Gerasdorf bei Wien}|pw},
               an den kleinen Friedhof\oindex{Friedhof Gerasdorf@\textbf{Friedhof Gerasdorf}|pw}, an den Kranz, den wir
               mitnehmen werden, und an das Kreuz, welches wir draußen kaufen werden, dachte, musste
               ich gleich daran denken, wie prachtvoll das alles für die Novelle passt. BeerH.\pwindex{Beer-Hofmann, Richard 1866-07-11 – 1945-09-26@\textsc{Beer-Hofmann, Richard} (1866-07-11 – 1945-09-26), \emph{Schriftsteller}|pw} wird sagen, es ist sein »Kind\pwindex{Beer-Hofmann, Richard 1866-07-11 – 1945-09-26@\textsc{Beer-Hofmann, Richard} (1866-07-11 – 1945-09-26), \emph{Schriftsteller}!Kind1893@\strich\emph{Das Kind} {[}1893{]}|pw}«. Viel {\pb}davon ist ja dabei, aber es
               ist doch etwas ganz, ganz anderes, wenn man die Gestalt der Lotte\pwindex{Pohl-Glas, Charlotte 1873-01-01 – 1944-02-15@\textsc{Pohl-Glas, Charlotte} (1873-01-01 – 1944-02-15), \emph{Schriftstellerin, Politikerin, Sozialistin}|pw}, die \label{K_L03161-4v}\edtext{Münchener\oindex{Muenchen@\textbf{München}|pw}{ }Affaire}{\lemma{\textnormal{\emph{Münchener Affaire}}}\Cendnote{\textnormal{vgl. Felix Salten an Arthur Schnitzler, 18. 2. 1895}}}\label{K_L03161-4h}, und
               unsere jetzigen Beziehungen nimmt. \pend
           \pstart
           Leben Sie wol, auf Wiedersehen Montag früh.\pend
           \pstart
           Herzlich {\\[\baselineskip]}Ihr \spacefill\mbox{Salten}\pend
           \leftskip=0em{}
         
         \endnumbering\mylabel{h}\end{ledgroupsized}\begin{anhang}\end{anhang}\newcommand{\dateiname}{L03161}\newcommand{\titel}{Felix Salten an Arthur Schnitzler, [1. 8. 1895]}\newcommand{\editorInnen}{Martin Anton Müller und Laura Untner}%% latex-leseansicht-abspann.tex
%% Abspann für die Leseansicht.
%% Der Schalter \ifkorrekturansicht ist bereits durch den Vorspann gesetzt.

%% latex-abspann.tex
%% Gemeinsamer Abspann für Korrekturansicht und Leseansicht.
%% Setzt den Schalter \ifkorrekturansicht voraus (gesetzt in den
%% einbindenden Dateien latex-korrekturansicht-abspann.tex bzw.
%% latex-leseansicht-abspann.tex).
%% ---------------------------------------------------------------

\normalsize

% Das esempio-Environment wird nur in der Leseansicht benötigt
\ifkorrekturansicht\else
\newenvironment{esempio}[3]%
{
    \vspace{1.5ex}
    \rlap{\underline{#1}}
    \par
    \setlength{\parindent}{0cm}
    \nopagebreak
    \leftskip=#2cm
    \rightskip=#3cm
}
{
    \par
}
\fi

\doendnotes{C}
\bigskip
\vfill

\clearpage

\footnotesize

\ifkorrekturansicht
  \lohead{\textsc{register}}
\fi

% theindex-Environment neu definieren ohne reledmac
\makeatletter
\renewenvironment{theindex}{%
  \ifkorrekturansicht
    \section*{\indexname}%
  \else
    \subsubsection*{Index der erwähnten Entitäten}%
  \fi
  \setlength{\parindent}{0pt}%
  \setlength{\parskip}{0pt plus 0.3pt}%
  \let\item\@idxitem
}{%
  \ifkorrekturansicht\clearpage\fi
}
\makeatother

\IfFileExists{\jobname-pw.ind}{\input{\jobname-pw.ind}}{}

% Quellenangabe nur in der Leseansicht
\ifkorrekturansicht\else
% Fallback-Definitionen, falls die .tex-Datei \titel etc. nicht gesetzt hat
\providecommand{\titel}{}
\providecommand{\editorInnen}{}
\providecommand{\dateiname}{\jobname}

\vspace{3cm}

\vfill

\footnotesize
\textsc{Quelle}: \titel. Herausgegeben von {\editorInnen}. In: \emph{Arthur Schnitzler: Briefwechsel mit Autorinnen und Autoren}.
 Digitale Edition, https://schnitzler-briefe.acdh.oeaw.ac.at/{\dateiname}.html (Stand \today)
\fi

\end{document}


      