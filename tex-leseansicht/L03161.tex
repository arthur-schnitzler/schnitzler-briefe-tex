%% latex-leseansicht-vorspann.tex
%% Vorspann für die Leseansicht.
%% Lädt die gemeinsame Datei latex-vorspann.tex mit nicht gesetztem Schalter.

\newif\ifkorrekturansicht
\korrekturansichtfalse

\input{../tex-inputs/latex-vorspann}


\section[ Felix Salten an Arthur Schnitzler, [1. 8. 1895]]{L03161 Felix Salten an Arthur Schnitzler,  [1. 8. 1895]}
\nopagebreak\mylabel{L03161v}
\rehead{ }\normalsize\beginnumbering\briefempfaengerindex{Schnitzler, Arthur@\textsc{Schnitzler, Arthur}!zzzSalten, Felix@\emph{von Felix Salten}!1895-08-011@{{[}1. 8. 1895{]}}|(be}
\toendnotes[C]{\smallbreak\pagebreak[2]}
\correspDesc{Versand  durch Felix Salten am [1. 8. 1895] in Wien
\newline{}Erhalt  durch Arthur Schnitzler im Zeitraum [2. 8. 1895
                  – 4. 8. 1895?] in Bad Ischl}\toendnotes[C]{\smallbreak}
\Standort{CUL, Schnitzler, B 89, A 1.}
\physDesc{Brief, 2 Blätter, 5 Seiten, 1678 Zeichen
\newline{}Handschrift: Bleistift, lateinische Kurrent
\newline{}Schnitzler: mit Bleistift datiert: »1/8 95.« 
\newline{}Ordnung: mit Bleistift von unbekannter Hand nummeriert: »61« }\toendnotes[C]{\smallbreak}
\pstart
           \noindent{}{\pb}Lieber Freund, ich bin, wenn ich das \label{K_L03161-1v}\edtext{B.}{\lemma{\textnormal{\emph{B.}}}\Cendnote{\textnormal{Bicycle, Fahrrad. Es ging um die bevorstehende gemeinsame
                  Radtour, siehe XXXX Auszeichnungsfehler: Dokument L03159 nicht gefunden.}}}\label{K_L03161-1} nur
               sonst in Ordnung habe, mit all dem einverstanden bis auf München\oindex{München@\textbf{München}|pw}. Das werden wir aber \label{K_L03161-2v}\edtext{Montag, wenn ich zu Ihnen komme}{\lemma{\textnormal{\emph{Montag, … komme}}}\Cendnote{\textnormal{Vgl. A. S.: \emph{Tagebuch}, 5. 8. 1895.
               }}}\label{K_L03161-2} noch näher besprechen.\pend
           
\pstart
           Was die \label{K_L03161-3v}\edtext{Feuill.\pwindex{Salten, Felix 6.\,9.\,1869 Budapest – 8.\,10.\,1945 Zürich@\textsc{Salten, Felix} (6.\,9.\,1869 Budapest – 8.\,10.\,1945 Zürich), \emph{Schriftsteller, Journalist, Chefredakteur}!Münchener Kunstausstellungen. I. Im königl. Glaspalast@\strich\emph{Die Münchener Kunstausstellungen. I. Im königl. Glaspalast}|pwv}\pwindex{Salten, Felix 6.\,9.\,1869 Budapest – 8.\,10.\,1945 Zürich@\textsc{Salten, Felix} (6.\,9.\,1869 Budapest – 8.\,10.\,1945 Zürich), \emph{Schriftsteller, Journalist, Chefredakteur}!Münchener Kunstausstellungen. II. Im königl. Glaspalast@\strich\emph{Die Münchener Kunstausstellungen. II. Im königl. Glaspalast}|pwv}\pwindex{Salten, Felix 6.\,9.\,1869 Budapest – 8.\,10.\,1945 Zürich@\textsc{Salten, Felix} (6.\,9.\,1869 Budapest – 8.\,10.\,1945 Zürich), \emph{Schriftsteller, Journalist, Chefredakteur}!Münchener Brief. (Orig.-Corr. der »Wiener Allg. Ztg.«)@\strich\emph{Münchener Brief. (Orig.-Corr. der »Wiener Allg. Ztg.«)}|pwv}}{\lemma{\textnormal{\emph{Feuill.}}}\Cendnote{\textnormal{Siehe XXXX Auszeichnungsfehler: Dokument L03159 nicht gefunden und XXXX Auszeichnungsfehler: Dokument L03160 nicht gefunden.
               }}}\label{K_L03161-3} betrifft, so hätten sie – 
               wie \label{K_L03161-4v}\edtext{Speiszetteln}{\lemma{\textnormal{\emph{Speiszetteln}}}\Cendnote{\textnormal{Menükarten}}}\label{K_L03161-4} ausgesehen, wenn ich mehr
               Bilder genannt hätte. Ich wollte also {\pb}nur wichtige Stationen geben,
               die gewissermaßen die durchwanderte Gegend charakterisiren. Dann schrieb ich doch
               auch für Leute, welche München\oindex{München@\textbf{München}|pw} nicht gesehen
               haben, ich möchte also mehr beschreiben, als unkontrollirbare Kritik üben. Die Secession\orgindex{Münchener Secession@Münchener Secession|pw} erhält übrigens noch ein \label{K_L03161-5v}\edtext{zweites (sachlicheres) Feuilleton\pwindex{Salten, Felix 6.\,9.\,1869 Budapest – 8.\,10.\,1945 Zürich@\textsc{Salten, Felix} (6.\,9.\,1869 Budapest – 8.\,10.\,1945 Zürich), \emph{Schriftsteller, Journalist, Chefredakteur}!Münchener Kunstausstellungen. IV. Die Secession@\strich\emph{Die Münchener Kunstausstellungen. IV. Die Secession}|pwv}}{\lemma{\textnormal{\emph{zweites … Feuilleton}}}\Cendnote{\textnormal{Felix Salten\pwindex{Salten, Felix 6.\,9.\,1869 Budapest – 8.\,10.\,1945 Zürich@\textsc{Salten, Felix} (6.\,9.\,1869 Budapest – 8.\,10.\,1945 Zürich), \emph{Schriftsteller, Journalist, Chefredakteur}|pwk}: \emph{Die Münchener Kunstausstellungen. IV. Die Secession}\pwindex{Salten, Felix 6.\,9.\,1869 Budapest – 8.\,10.\,1945 Zürich@\textsc{Salten, Felix} (6.\,9.\,1869 Budapest – 8.\,10.\,1945 Zürich), \emph{Schriftsteller, Journalist, Chefredakteur}!Münchener Kunstausstellungen. IV. Die Secession@\strich\emph{Die Münchener Kunstausstellungen. IV. Die Secession}|pwk}.
                     In: \emph{Wiener Allgemeine Zeitung}\pwindex{Wiener Allgemeine Zeitung@\emph{Wiener Allgemeine Zeitung}|pwk}, Nr. 5234,
                        15. 8. 1895, S. 8.}}}\label{K_L03161-5}.\pend
           
\pstart
           Dass Ihr \label{K_L03161-6v}\edtext{Theaterleben Sie {\pb}in Freiwild\pwindex{Schnitzler, Arthur 15.\,5.\,1862 Wien – 21.\,10.\,1931 ebd.@\textsc{Schnitzler, Arthur} (15.\,5.\,1862 Wien – 21.\,10.\,1931 ebd.), \emph{Schriftsteller, Mediziner}!Freiwild. Schauspiel in 3 Akten@\strich\emph{Freiwild. Schauspiel in 3 Akten}|pw}{ }\substVorne{}\textsuperscript{stört}\substDazwischen{}hätte stören können\substHinten{}{ }}{\lemma{\textnormal{\emph{Theaterleben … können}}}\Cendnote{\textnormal{Schnitzler stand kurz vor der Fertigstellung
                  des 1. Akts von \emph{Freiwild}\pwindex{Schnitzler, Arthur 15.\,5.\,1862 Wien – 21.\,10.\,1931 ebd.@\textsc{Schnitzler, Arthur} (15.\,5.\,1862 Wien – 21.\,10.\,1931 ebd.), \emph{Schriftsteller, Mediziner}!Freiwild. Schauspiel in 3 Akten@\strich\emph{Freiwild. Schauspiel in 3 Akten}|pwk} am 2. 8. 1895. Die in
                  diesem Stück behandelte Theaterprostitution, die er durch seinen Kontakt mit
                  Schauspielerinnen aus eigener Anschauung kannte, dürfte narrative Zweifel in ihm
                  ausgelöst haben. }}}\label{K_L03161-6}, ist sonderbar. Es kommt ja garnicht darauf an, dass
               diese Mädeln Männer fangen wollen, sondern auf die Umstände, die ihnen ein solches
               Leben zur Notwendigkeit machen. Dass sich manche willig manche mit vielem Geschick
               darin finden, ändert doch an der Freiwild\pwindex{Schnitzler, Arthur 15.\,5.\,1862 Wien – 21.\,10.\,1931 ebd.@\textsc{Schnitzler, Arthur} (15.\,5.\,1862 Wien – 21.\,10.\,1931 ebd.), \emph{Schriftsteller, Mediziner}!Freiwild. Schauspiel in 3 Akten@\strich\emph{Freiwild. Schauspiel in 3 Akten}|pw}-\uline{Idee} nicht das mindeste, selbst dann nicht wenn man
               gelegentlich wirklich der Jäger wäre.\pend
           
\pstart
           =\pend
           
\pstart
           Meine \label{K_L03161-7v}\edtext{Tochter\pwindex{Lamberg, Maria Charlotte 24.\,3.\,1895 Wien – 27.\,7.\,1895 Gerasdorf bei Wien@\textsc{Lamberg, Maria Charlotte} (24.\,3.\,1895 Wien – 27.\,7.\,1895 Gerasdorf bei Wien)|pwv}}{\lemma{\textnormal{\emph{Tochter}}}\Cendnote{\textnormal{Das gemeinsame Kind mit Charlotte Glas\pwindex{Pohl-Glas, Charlotte 1.\,1.\,1873 Wien – 15.\,2.\,1944 Zürich@\textsc{Pohl-Glas, Charlotte} (1.\,1.\,1873 Wien – 15.\,2.\,1944 Zürich), \emph{Schriftstellerin, Politikerin, Sozialistin}|pwk} trug den Namen Maria Charlotte Lamberg\pwindex{Lamberg, Maria Charlotte 24.\,3.\,1895 Wien – 27.\,7.\,1895 Gerasdorf bei Wien@\textsc{Lamberg, Maria Charlotte} (24.\,3.\,1895 Wien – 27.\,7.\,1895 Gerasdorf bei Wien)|pwk} und war gerade vier
                  Monate alt, als es am 27. 7. 1895 bei der Kostfrau\pwindex{?? [Kostfrau von Charlotte Lamberg] @\textsc{?? [Kostfrau von Charlotte Lamberg]}|pwkv} in Gerasdorf bei Wien\oindex{Gerasdorf bei Wien@\textbf{Gerasdorf bei Wien}, \emph{Verwaltungsgebiet}|pwk} verstarb.}}}\label{K_L03161-7} ist
               gestorben. Damit fällt {\pb}ein
               starkes Band zwischen Lotte\pwindex{Pohl-Glas, Charlotte 1.\,1.\,1873 Wien – 15.\,2.\,1944 Zürich@\textsc{Pohl-Glas, Charlotte} (1.\,1.\,1873 Wien – 15.\,2.\,1944 Zürich), \emph{Schriftstellerin, Politikerin, Sozialistin}|pw} u. mir. Als die
               alte Frau\pwindex{?? [Kostfrau von Charlotte Lamberg] @\textsc{?? [Kostfrau von Charlotte Lamberg]}|pwv}, welche mir die
               Nachricht brachte, mit Thränen an meinem Redaction\orgindex{Wiener Allgemeine Zeitung@Wiener Allgemeine Zeitung|pwv}stisch saß, und ich an die Fahrt mit Lotte\pwindex{Pohl-Glas, Charlotte 1.\,1.\,1873 Wien – 15.\,2.\,1944 Zürich@\textsc{Pohl-Glas, Charlotte} (1.\,1.\,1873 Wien – 15.\,2.\,1944 Zürich), \emph{Schriftstellerin, Politikerin, Sozialistin}|pw} nach Gerasdorf\oindex{Gerasdorf bei Wien@\textbf{Gerasdorf bei Wien}, \emph{Verwaltungsgebiet}|pw}, an den kleinen Friedhof\oindex{Friedhof Gerasdorf@\textbf{Friedhof Gerasdorf}, \emph{Friedhof}|pw}{[},{]} an den Kranz, den wir mitnehmen werden, und an das Kreuz,
               welches wir draußen kaufen werden, dachte, musste ich gleich daran denken, wie
               prachtvoll das alles für die Novelle passt. BeerH.\pwindex{Beer-Hofmann, Richard 11.\,7.\,1866 Wien – 26.\,9.\,1945 New York City@\textsc{Beer-Hofmann, Richard} (11.\,7.\,1866 Wien – 26.\,9.\,1945 New York City), \emph{Schriftsteller}|pw} wird sagen, es ist sein\strikeout{\textcolor{gray}{e}} »Kind\pwindex{Beer-Hofmann, Richard 11.\,7.\,1866 Wien – 26.\,9.\,1945 New York City@\textsc{Beer-Hofmann, Richard} (11.\,7.\,1866 Wien – 26.\,9.\,1945 New York City), \emph{Schriftsteller}!Kind@\strich\emph{Das Kind}|pw}«. Viel {\pb}davon ist ja dabei, aber es
               ist doch etwas ganz, ganz anderes, wenn man die Gestalt der Lotte\pwindex{Pohl-Glas, Charlotte 1.\,1.\,1873 Wien – 15.\,2.\,1944 Zürich@\textsc{Pohl-Glas, Charlotte} (1.\,1.\,1873 Wien – 15.\,2.\,1944 Zürich), \emph{Schriftstellerin, Politikerin, Sozialistin}|pw}, die \label{K_L03161-8v}\edtext{München\oindex{München@\textbf{München}|pw}er Affaire}{\lemma{\textnormal{\emph{Münchener Affaire}}}\Cendnote{\textnormal{Vgl. XXXX Auszeichnungsfehler: Dokument L03152 nicht gefunden.
               }}}\label{K_L03161-8}, und unsere jetzigen Beziehungen nimmt.\pend
           
\pstart
           Leben Sie wol, auf Wiedersehen Montag{ }früh. Herzlich {\\[\baselineskip]}Ihr \spacefill\mbox{Salten}\pend
           \leftskip=0em{}\selectlanguage{ngerman}\endnumbering\briefempfaengerindex{Schnitzler, Arthur@\textsc{Schnitzler, Arthur}!zzzSalten, Felix@\emph{von Felix Salten}!1895-08-011@{{[}1. 8. 1895{]}}|)be}\mylabel{L03161h}  \newcommand{\dateiname}{L03161}\newcommand{\titel}{Felix Salten an Arthur Schnitzler, [1. 8. 1895]}\newcommand{\editorInnen}{Martin Anton Müller und Laura Untner}%% latex-leseansicht-abspann.tex
%% Abspann für die Leseansicht.
%% Der Schalter \ifkorrekturansicht ist bereits durch den Vorspann gesetzt.

%% latex-abspann.tex
%% Gemeinsamer Abspann für Korrekturansicht und Leseansicht.
%% Setzt den Schalter \ifkorrekturansicht voraus (gesetzt in den
%% einbindenden Dateien latex-korrekturansicht-abspann.tex bzw.
%% latex-leseansicht-abspann.tex).
%% ---------------------------------------------------------------

\normalsize

% Das esempio-Environment wird nur in der Leseansicht benötigt
\ifkorrekturansicht\else
\newenvironment{esempio}[3]%
{
    \vspace{1.5ex}
    \rlap{\underline{#1}}
    \par
    \setlength{\parindent}{0cm}
    \nopagebreak
    \leftskip=#2cm
    \rightskip=#3cm
}
{
    \par
}
\fi

\doendnotes{C}
\bigskip
\vfill

\clearpage

\footnotesize

\ifkorrekturansicht
  \lohead{\textsc{register}}
\fi

% theindex-Environment neu definieren ohne reledmac
\makeatletter
\renewenvironment{theindex}{%
  \ifkorrekturansicht
    \section*{\indexname}%
  \else
    \subsubsection*{Index der erwähnten Entitäten}%
  \fi
  \setlength{\parindent}{0pt}%
  \setlength{\parskip}{0pt plus 0.3pt}%
  \let\item\@idxitem
}{%
  \ifkorrekturansicht\clearpage\fi
}
\makeatother

\IfFileExists{\jobname-pw.ind}{\input{\jobname-pw.ind}}{}

% Quellenangabe nur in der Leseansicht
\ifkorrekturansicht\else
% Fallback-Definitionen, falls die .tex-Datei \titel etc. nicht gesetzt hat
\providecommand{\titel}{}
\providecommand{\editorInnen}{}
\providecommand{\dateiname}{\jobname}

\vspace{3cm}

\vfill

\footnotesize
\textsc{Quelle}: \titel. Herausgegeben von {\editorInnen}. In: \emph{Arthur Schnitzler: Briefwechsel mit Autorinnen und Autoren}.
 Digitale Edition, https://schnitzler-briefe.acdh.oeaw.ac.at/{\dateiname}.html (Stand \today)
\fi

\end{document}


