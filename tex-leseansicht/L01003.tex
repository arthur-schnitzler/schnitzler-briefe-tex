%% latex-leseansicht-vorspann.tex
%% Vorspann für die Leseansicht.
%% Lädt die gemeinsame Datei latex-vorspann.tex mit nicht gesetztem Schalter.

\newif\ifkorrekturansicht
\korrekturansichtfalse

\input{../tex-inputs/latex-vorspann}


\section[Arthur Schnitzler an Richard Beer-Hofmann, 4. 12. 1899]{L01003 Arthur Schnitzler an Richard Beer-Hofmann, 4. 12. 1899}
\nopagebreak\mylabel{L01003v}
\rehead{ }\normalsize\beginnumbering\briefempfaengerindex{Beer-Hofmann, Richard@\textsc{Beer-Hofmann, Richard}!zzzSchnitzler, Arthur@\emph{von Arthur Schnitzler}!1899-12-041@{4. 12. 1899}|(be}
\toendnotes[C]{\smallbreak\pagebreak[2]}
\correspDesc{Versand  durch Arthur Schnitzler am 4. 12. 1899 in Wien
\newline{}Erhalt  durch Richard Beer-Hofmann im Zeitraum [4. 12. 1899
                  – 8. 12. 1899?] in Wien}\toendnotes[C]{\smallbreak}
\Standort{YCGL, MSS 31.}
\physDesc{Postkarte, 237 Zeichen
\newline{}Handschrift: Bleistift, deutsche Kurrent
\newline{}Versand: Stempel: »\nobreak{}\oindex{I., Innere Stadt@\textbf{I., Innere Stadt}, \emph{Verwaltungsgebiet}|pwk}Wien 1/1, 4. 12. 99, 10–11N\nobreak{}«.  }\toendnotes[C]{\smallbreak}\pstart{}{\pb}Herrn \textsc{Dr Richard
                     Beer-Hofmann}\pend{}\pstart{}Wien\oindex{Wien@\textbf{Wien}, \emph{Verwaltungsgebiet}|pw}\pend{}\pstart{}\textsc{I. Wollzeile 15\oindex{Wien@\textbf{Wien}!I., Innere Stadt@\textbf{I., Innere Stadt}!Wollzeile 15 (»Berthahof«)@\textbf{Wollzeile 15 (»Berthahof«)}, \emph{Wohngebäude}|pw}}.\pend{}{\bigskip}\vspace{1em}
\pstart
           \noindent{}{\pb}lieber Richard,{ }ſehr wahrſcheinlich dſs ich morgen \label{K_L01003-1v}\edtext{Dinſtag im Club\orgindex{Wiener Schachclub@Wiener Schachclub|pwv}
                  nachtmahle}{\lemma{\textnormal{\emph{Dinstag … nachtmahle}}}\Cendnote{\textnormal{Dieser Besuch findet sich
                  nicht im \emph{Tagebuch}\pwindex{Schnitzler, Arthur 15.\,5.\,1862 Wien – 21.\,10.\,1931 ebd.@\textsc{Schnitzler, Arthur} (15.\,5.\,1862 Wien – 21.\,10.\,1931 ebd.), \emph{Schriftsteller, Mediziner}!Tagebuch@\strich\emph{Tagebuch}|pwk}, aber am XXXX Auszeichnungsfehler: Dokument L04173 nicht gefunden berichtet er,
                  dass er bereits zweimal im Schachclub\oindex{Wien@\textbf{Wien}!I., Innere Stadt@\textbf{I., Innere Stadt}!Wiener Schachclub [Wallnerstraße 2]@\textbf{Wiener Schachclub [Wallnerstraße 2]}|pwkv} war. Er dürfte die Ankündigung also umgesetzt haben.}}}\label{K_L01003-1}
               und dann bis gegen ½ 12 dort bleibe.\pend
           
\pstart
           \label{K_L01003-2v}\edtext{Mittwoch will ich zu Siegfried\pwindex{\textcolor{red}{\textsuperscript{XXXX indx1}}!Siegfried@\strich\emph{Siegfried}|pw}\eventindex{Oper@\textbf{Oper}!Aufführung von Siegfried, 6.12.1899@Aufführung von Siegfried, 6.12.1899|pwv}, komme nachher wohl auch hin}{\lemma{\textnormal{\emph{Mittwoch … hin}}}\Cendnote{\textnormal{Beides
                  ist belegt, vgl. A. S.: \emph{Kulturveranstaltungen}, 6. 12. 1899 und
                     vgl. A. S.: \emph{Tagebuch}, 6. 12. 1899.}}}\label{K_L01003-2}.\pend
           \pstart Herzlic\textcolor{gray}{hen} Gruſs Ihr \spacefill\mbox{Arthur}\pend{}\selectlanguage{ngerman}\endnumbering\briefempfaengerindex{Beer-Hofmann, Richard@\textsc{Beer-Hofmann, Richard}!zzzSchnitzler, Arthur@\emph{von Arthur Schnitzler}!1899-12-041@{4. 12. 1899}|)be}\mylabel{L01003h}  \newcommand{\dateiname}{L01003}\newcommand{\titel}{Arthur Schnitzler an Richard Beer-Hofmann, 4. 12. 1899}\newcommand{\editorInnen}{Martin Anton Müller und Gerd-Hermann Susen}%% latex-leseansicht-abspann.tex
%% Abspann für die Leseansicht.
%% Der Schalter \ifkorrekturansicht ist bereits durch den Vorspann gesetzt.

%% latex-abspann.tex
%% Gemeinsamer Abspann für Korrekturansicht und Leseansicht.
%% Setzt den Schalter \ifkorrekturansicht voraus (gesetzt in den
%% einbindenden Dateien latex-korrekturansicht-abspann.tex bzw.
%% latex-leseansicht-abspann.tex).
%% ---------------------------------------------------------------

\normalsize

% Das esempio-Environment wird nur in der Leseansicht benötigt
\ifkorrekturansicht\else
\newenvironment{esempio}[3]%
{
    \vspace{1.5ex}
    \rlap{\underline{#1}}
    \par
    \setlength{\parindent}{0cm}
    \nopagebreak
    \leftskip=#2cm
    \rightskip=#3cm
}
{
    \par
}
\fi

\doendnotes{C}
\bigskip
\vfill

\clearpage

\footnotesize

\ifkorrekturansicht
  \lohead{\textsc{register}}
\fi

% theindex-Environment neu definieren ohne reledmac
\makeatletter
\renewenvironment{theindex}{%
  \ifkorrekturansicht
    \section*{\indexname}%
  \else
    \subsubsection*{Index der erwähnten Entitäten}%
  \fi
  \setlength{\parindent}{0pt}%
  \setlength{\parskip}{0pt plus 0.3pt}%
  \let\item\@idxitem
}{%
  \ifkorrekturansicht\clearpage\fi
}
\makeatother

\IfFileExists{\jobname-pw.ind}{\input{\jobname-pw.ind}}{}

% Quellenangabe nur in der Leseansicht
\ifkorrekturansicht\else
% Fallback-Definitionen, falls die .tex-Datei \titel etc. nicht gesetzt hat
\providecommand{\titel}{}
\providecommand{\editorInnen}{}
\providecommand{\dateiname}{\jobname}

\vspace{3cm}

\vfill

\footnotesize
\textsc{Quelle}: \titel. Herausgegeben von {\editorInnen}. In: \emph{Arthur Schnitzler: Briefwechsel mit Autorinnen und Autoren}.
 Digitale Edition, https://schnitzler-briefe.acdh.oeaw.ac.at/{\dateiname}.html (Stand \today)
\fi

\end{document}


