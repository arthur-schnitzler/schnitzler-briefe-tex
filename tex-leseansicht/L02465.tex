%% latex-korrekturansicht-vorspann.tex
%% Vorspann für die Korrekturansicht.
%% Lädt die gemeinsame Datei latex-vorspann.tex mit gesetztem Schalter.

\newif\ifkorrekturansicht
\korrekturansichttrue

\input{../tex-inputs/latex-vorspann}


\section[Arthur Schnitzler an Georg Brandes, 3. 1. 1926]{L02465 Arthur Schnitzler an Georg Brandes, 3. 1. 1926}
\nopagebreak\mylabel{L02465v}
\rehead{ }\normalsize\beginnumbering\briefempfaengerindex{Brandes, Georg@\textsc{Brandes, Georg}!zzzSchnitzler, Arthur@\emph{von Arthur Schnitzler}!1926-01-031@{3. 1. 1926}|(be}
\toendnotes[C]{\smallbreak\pagebreak[2]}\Standort{Kopenhagen, Det Kongelige Bibliotek, Georg Brandes Arkiv, box 125.}
\physDesc{Brief, 2 Blätter, 4 Seiten, 1800 Zeichen
\newline{}Handschrift: schwarze Tinte, lateinische Kurrent
\newline{}Ordnung: mit Bleistift von unbekannter Hand nummeriert:
                                    »54.«, auf der ersten Seite
                                 »Schn«, auf dem zweiten Blatt das Datum ergänzt: »3/1 26« }
\buchAbdrucke{\weitereDrucke{Georg Brandes, Arthur Schnitzler: \emph{Ein Briefwechsel}. Bern: \emph{Francke} 1956, S. 151–152.} }
\pstart
           \raggedleft{}{\pb}Wien\oindex{Wien@\textbf{Wien}, \emph{A.ADM2}|pw}, 3. 1. 926\pend
           
\pstart{}Mein verehrter und lieber Freund,\pend\vspace{0.5em}
\pstart
           Ihr Neujahrsgruſs hat mich beschämt. Was ko{\geminationn}te ich in
                  Wien\oindex{Wien@\textbf{Wien}, \emph{A.ADM2}|pw} besondres für Sie thun – die höchst
               bescheidene Gastfreundschaft die ich Ihnen und Frau Rung\pwindex{Rung, Gertrud 26.03.1882 – 25.04.1959@\textsc{Rung, Gertrud} (26.03.1882 – 25.04.1959), \emph{Übersetzer/Übersetzerin, Sekretär/Sekretärin}|pw} erweisen durfte, \substVorne{}\textsuperscript{von}\substDazwischen{}bedeutete\substHinten{} mir mindestens so viel Freude als Ihnen – und wie wenig war es in jedem Fall
               im Verhältnis zu der tiefen Dankbarkeit und Liebe, die ich für Sie empfinde. Und ich
               halte Sie immer zumindest geistig in meiner Nähe: kaum ein Abend ist im Lauf des
               letzten oder der letzten Jahre vergangen, ohne daſs ich ein paar, und öfters recht
               viele Seiten von Ihnen gelesen. Und in Ihnen ist eine so wunder{\pb}bare Identität des Menschen und des
               Schriftstellers, daſs man immer \uline{mit} Ihnen ist, we{\geminationn} man Sie liest.\pend
           
\pstart
           Die »Frau des Richters\pwindex{Frau des Richters. Novelle@\emph{Die Frau des Richters. Novelle}|pw}« war zuerst in der Vossischen\orgindex{Vossische Zeitung@Vossische Zeitung|pw} gedruckt – und eigentlich als Einakter
               intendirt. Das Stück wollte mir nicht gelingen, so hab ich die Handlung zu erzählen
               versucht. Mein Herz hängt \uline{nicht} an dieser kleinen
               Geschichte. Viele Jahre aber hat mich ein fünfactiges Versdrama »Der Gang zum Weiher\pwindex{Gang zum Weiher. Dramatische Dichtung@\emph{Der Gang zum Weiher. Dramatische Dichtung}|pw}« begleitet, das schon im Druck ist und das
               ich Ihnen hoffentlich bald schicken kann, ebenso wie eine größere »Traumnovelle\pwindex{Traumnovelle@\emph{Traumnovelle}|pw}« (die eben in Fortsetzungen in der
                  Dame\orgindex{Dame@Die Dame|pw} erscheint –){[}.{]}{ }{\pb}Und ganz besonders viel beschäftigt mich – auch
               seit Jahren schon – allerlei aphoristisch–fragmentistisches, – worunter vielleicht
               zwei Diagramme »Der Geist im Wort« und »der Geist in
                  der That\pwindex{Geist im Wort und der Geist in der Tat@\emph{Der Geist im Wort und der Geist in der Tat}|pw}«, philosophische Spielereien nicht ohne tieferen Sinn, Sie
               unterhalten werden.\pend
           
\pstart
           Ich hoffe Sie sind so wohl und gesund als Ihre jungen und festen Schriftzüge
               vermuthen lassen. Darf ich Sie bitten, der liebenswürdigen Frau Gertrud Rung\pwindex{Rung, Gertrud 26.03.1882 – 25.04.1959@\textsc{Rung, Gertrud} (26.03.1882 – 25.04.1959), \emph{Übersetzer/Übersetzerin, Sekretär/Sekretärin}|pw} meine herzlichsten Gegengrüße zu bestellen? Und an
               Sie, mein lieber und wahrhaft verehrter Freund, gehen meine innigsten Wünsche {\pb}Tag für Tag.\pend
           
\pstart
           Auf Wiedersehen.{\\[\baselineskip]}Ihr getreuer{\\[\baselineskip]}\spacefill\mbox{Arthur Schnitzler}\pend
           \leftskip=0em{}\selectlanguage{ngerman}\endnumbering\briefempfaengerindex{Brandes, Georg@\textsc{Brandes, Georg}!zzzSchnitzler, Arthur@\emph{von Arthur Schnitzler}!1926-01-031@{3. 1. 1926}|)be}\mylabel{L02465h}  \normalsize

\doendnotes{C}
\bigskip
\vfill

\clearpage

\footnotesize

\lohead{\textsc{register}}

% Definiere theindex-Environment komplett neu ohne reledmac
\makeatletter
\renewenvironment{theindex}{%
  \section*{\indexname}%
  \setlength{\parindent}{0pt}%
  \setlength{\parskip}{0pt plus 0.3pt}%
  \let\item\@idxitem
}{%
  \clearpage
}
\makeatother

\IfFileExists{\jobname-pw.ind}{\input{\jobname-pw.ind}}{}

\end{document}

      