%% latex-korrekturansicht-vorspann.tex
%% Vorspann für die Korrekturansicht.
%% Lädt die gemeinsame Datei latex-vorspann.tex mit gesetztem Schalter.

\newif\ifkorrekturansicht
\korrekturansichttrue

\input{../tex-inputs/latex-vorspann}


\section[Hugo von Hofmannsthal an Arthur Schnitzler, 2. 5. {[}1897{]}]{L00673 Hugo von Hofmannsthal an Arthur Schnitzler, 2. 5. {[}1897{]}}
\nopagebreak\mylabel{L00673v}
\rehead{ }\normalsize\beginnumbering\briefempfaengerindex{Schnitzler, Arthur@\textsc{Schnitzler, Arthur}!zzzHofmannsthal, Hugo von@\emph{von Hugo von Hofmannsthal}!1897-05-021@{2. 5. {[}1897{]}}|(be}
\toendnotes[C]{\smallbreak\pagebreak[2]}\Standort{CUL, Schnitzler, B 43.}
\physDesc{Brief, 1 Blatt, 4 Seiten, 1508 Zeichen
\newline{}Handschrift: schwarze Tinte, deutsche Kurrent
\newline{}Schnitzler: mit Bleistift die Jahreszahl ergänzt: »97« 
\newline{}Ordnung: mit Bleistift von unbekannter Hand nummeriert:
                                    »89« }
\buchAbdrucke{\weitereDrucke{Hugo von Hofmannsthal, Arthur Schnitzler: \emph{Briefwechsel}. Frankfurt am Main: \emph{S. Fischer} 1964, S. 83–84.} }\toendnotes[C]{\smallbreak}
\pstart
           {\pb}\textcolor{gray}{\textbf{\label{T_L00673-1v}\edtext{hvH}{\lemma{\textnormal{\emph{hvH}}}\Cendnote{\textnormal{gedrucktes Monogramm mit Krone in roter Farbe}}}\label{T_L00673-1}}}\pend
           
\pstart
           \raggedleft{}Sonntag 2\textsuperscript{ten} Mai\pend
           
\pstart{}lieber Arthur,\pend\vspace{0.5em}
\pstart
           wie komiſch man eigentlich iſt: es hat mich einen Moment ganz ſtark geärgert zu
               hören, daſs Sie wieder gemiſchtes Hausbrot eſſen. Ich hätte ſo gern gehört, daſs Sie
               auf einmal etwas ganz anderes eſſen! Aber das iſt natürlich eine Kinderei.\pend
           
\pstart
           Hier iſt es jetzt ſehr ſchön. (Nur gerade heute regnet es zufällig.) Der Frühling war
                  {\pb}durch eine lange kühle Zeit
               zurückgehalten und dann war er auf einmal da und ſo warm und ſo farbig, daſs die
               Farben der Blumenbeete, der Baumwipfel und des Himmels mit ihren Contouren
               auszutreten und die Luft zu überſchwemmen ſchienen. Das Radfahren macht mir eine
               große Freude: es iſt wunderſchön, ein biſſel ermüdet und erhitzt ſich irgendwo ſtill
               hinzuſetzen {\pb}und über die
               Sträuche, die Wieſen und die Hügel hinzuſchauen, und abends iſt es ſogar wunderſchön,
               in den Straßen der Vorſtädte zu fahren.\pend
           
\pstart
           Schreiben Sie mir doch ein paar ſchöne kleine Ausflüge, an die \substVorne{}\textsuperscript{s}\substDazwischen{}S\substHinten{}ie ſich erinnern. Ich war erſt in Weidling
                  am Bach\oindex{Weidlingbach@\textbf{Weidlingbach}, \emph{P.PPL}|pw}, und in Heiligenkreuz\oindex{Heiligenkreuz@\textbf{Heiligenkreuz}, \emph{A.ADM3}|pw}.\pend
           
\pstart
           Ihre Bemerkungen über das franzöſiſche\oindex{Frankreich@\textbf{Frankreich}, \emph{A.PCLI}|pw} Theater
               verſtehe ich ſehr gut, weil jetzt gerade {\pb}eine franzöſiſche\oindex{Frankreich@\textbf{Frankreich}, \emph{A.PCLI}|pw} Truppe im Carltheater\oindex{Carl-Theater@\textbf{Carl-Theater}, \emph{Theater (K.THE)}|pw} war und lauter ſolche \textsc{Vie-paris\oindex{Paris@\textbf{Paris}, \emph{P.PPLC}|pw}ienne}{ }Stücke geſpielt hat. Vergeſſen Sie doch nicht, die
                  Delna\pwindex{Delna, Marie 03.04.1875 – 23.07.1932@\textsc{Delna, Marie} (03.04.1875 – 23.07.1932), \emph{Sänger/Sängerin}|pw} als Orpheus\pwindex{Orpheus und Eurydike@\emph{Orpheus und Eurydike}|pwv} zu hören.\pend
           
\pstart
           Ich arbeite noch immer nichts, lerne nur fleißig an meinen romaniſchen Texten. Aber
               ich fühle mich doch nun recht viel freier und weniger verworren und bin viel
               zufriedener.\pend
           
\pstart
           Ich freue mich recht auf Ihre Rückkehr. »Götterliebling\pwindex{Tod Georgs@\emph{Der Tod Georgs}|pw}« dürfte bald fertig ſein, auch das Stück\pwindex{Agnes Jordan. Schauspiel in fuenf Akten@\emph{Agnes Jordan. Schauspiel in fünf Akten}|pwv} vom Hirſchfeld\pwindex{Hirschfeld, Georg 11.02.1873 – 17.01.1942@\textsc{Hirschfeld, Georg} (11.02.1873 – 17.01.1942), \emph{Schriftsteller/Schriftstellerin}|pw}.\pend
           \pstart Ihr\spacefill\mbox{Hugo.}\pend{}\selectlanguage{ngerman}\endnumbering\briefempfaengerindex{Schnitzler, Arthur@\textsc{Schnitzler, Arthur}!zzzHofmannsthal, Hugo von@\emph{von Hugo von Hofmannsthal}!1897-05-021@{2. 5. {[}1897{]}}|)be}\mylabel{L00673h}  \normalsize

\doendnotes{C}
\bigskip
\vfill

\clearpage

\footnotesize

\lohead{\textsc{register}}

% Definiere theindex-Environment komplett neu ohne reledmac
\makeatletter
\renewenvironment{theindex}{%
  \section*{\indexname}%
  \setlength{\parindent}{0pt}%
  \setlength{\parskip}{0pt plus 0.3pt}%
  \let\item\@idxitem
}{%
  \clearpage
}
\makeatother

\IfFileExists{\jobname-pw.ind}{\input{\jobname-pw.ind}}{}

\end{document}

      