%% latex-leseansicht-vorspann.tex
%% Vorspann für die Leseansicht.
%% Lädt die gemeinsame Datei latex-vorspann.tex mit nicht gesetztem Schalter.

\newif\ifkorrekturansicht
\korrekturansichtfalse

\input{../tex-inputs/latex-vorspann}


         
         \renewcommand{\erwaehntePersonen}{Personen: Marie Delna, Georg Hirschfeld, Hugo von Hofmannsthal}
         \renewcommand{\erwaehnteOrte}{Orte: Carl-Theater, Frankreich, Heiligenkreuz, Paris, Weidlingbach, Wien}
         \renewcommand{\erwaehnteWerke}{Werke: Agnes Jordan. Schauspiel in fünf Akten, Der Tod Georgs, Orpheus und Eurydike}
               \section[Hugo von Hofmannsthal an Arthur Schnitzler, 2. 5. {[}1897{]}]{ Hugo von Hofmannsthal an Arthur Schnitzler, 2. 5. {[}1897{]}}\nopagebreak\mylabel{v}\rehead{ }\begin{ledgroupsized}[t]{13cm}\normalsize\beginnumbering\briefempfaengerindex{Schnitzler, Arthur@\textsc{Schnitzler, Arthur}!zzzHofmannsthal, Hugo von@\emph{von Hugo von Hofmannsthal}!1897-05-021@{2. 5. {[}1897{]}}|(be} \toendnotes[C]{\smallbreak\pagebreak[2]} \Standort{CUL, Schnitzler, B 43.}
\physDesc{Brief, 1 Blatt, 4 Seiten, 1508 Zeichen
\newline{}Handschrift: schwarze Tinte, deutsche Kurrent
\newline{}Schnitzler: mit Bleistift die Jahreszahl ergänzt: »97« 
\newline{}Ordnung: mit Bleistift von unbekannter Hand nummeriert:
                                    »89« }\buchAbdrucke{\weitereDrucke{Hugo von Hofmannsthal, Arthur Schnitzler: \emph{Briefwechsel}. Hg. Therese Nickl und Heinrich Schnitzler. Frankfurt am Main: \emph{S. Fischer} 1964, S. 83–84.} }\toendnotes[C]{\smallbreak}\pstart
           \noindent{}{\pb}\textcolor{gray}{\textbf{\label{T_L00673-1v}\edtext{hvH}{\lemma{\textnormal{\emph{hvH}}}\Cendnote{\textnormal{gedrucktes Monogramm mit Krone in roter Farbe}}}\label{T_L00673-1h}}}\pend
           \pstart
           \raggedleft{}Sonntag 2\textsuperscript{ten} Mai\pend
           \pstart{}lieber Arthur,\pend\pstart
           wie komiſch man eigentlich iſt: es hat mich einen Moment ganz ſtark geärgert zu
               hören, daſs Sie wieder gemiſchtes Hausbrot eſſen. Ich hätte ſo gern gehört, daſs Sie
               auf einmal etwas ganz anderes eſſen! Aber das iſt natürlich eine Kinderei.\pend
           \pstart
           Hier iſt es jetzt ſehr ſchön. (Nur gerade heute regnet es zufällig.) Der Frühling war
                  {\pb}durch eine lange kühle Zeit
               zurückgehalten und dann war er auf einmal da und ſo warm und ſo farbig, daſs die
               Farben der Blumenbeete, der Baumwipfel und des Himmels mit ihren Contouren
               auszutreten und die Luft zu überſchwemmen ſchienen. Das Radfahren macht mir eine
               große Freude: es iſt wunderſchön, ein biſſel ermüdet und erhitzt ſich irgendwo ſtill
               hinzuſetzen {\pb}und über die
               Sträuche, die Wieſen und die Hügel hinzuſchauen, und abends iſt es ſogar wunderſchön,
               in den Straßen der Vorſtädte zu fahren.\pend
           \pstart
           Schreiben Sie mir doch ein paar ſchöne kleine Ausflüge, an die \substVorne{}\textsuperscript{s}\substDazwischen{}S\substHinten{}ie ſich erinnern. Ich war erſt in Weidling
                  am Bach\oindex{Weidlingbach@\textbf{Weidlingbach}|pw}, und in Heiligenkreuz\oindex{Heiligenkreuz@\textbf{Heiligenkreuz}|pw}.\pend
           \pstart
           Ihre Bemerkungen über das franzöſiſche\oindex{Frankreich@\textbf{Frankreich}|pw} Theater
               verſtehe ich ſehr gut, weil jetzt gerade {\pb}eine franzöſiſche\oindex{Frankreich@\textbf{Frankreich}|pw} Truppe im Carltheater\oindex{Carl-Theater@\textbf{Carl-Theater}|pw} war und lauter ſolche \textsc{Vie-paris\oindex{Paris@\textbf{Paris}|pw}ienne}{ }Stücke geſpielt hat. Vergeſſen Sie doch nicht, die
                  Delna\pwindex{Delna, Marie 03.04.1875 – 23.07.1932@\textsc{Delna, Marie} (03.04.1875 – 23.07.1932), \emph{Sängerin}|pw} als Orpheus\pwindex{\textcolor{red}{\textsuperscript{XXXX1 indx}}!Orpheus und Eurydike5. 10. 1762@\strich\emph{Orpheus und Eurydike} {[}5. 10. 1762{]}|pwv} zu hören.\pend
           \pstart
           Ich arbeite noch immer nichts, lerne nur fleißig an meinen romaniſchen Texten. Aber
               ich fühle mich doch nun recht viel freier und weniger verworren und bin viel
               zufriedener.\pend
           \pstart
           Ich freue mich recht auf Ihre Rückkehr. »Götterliebling\pwindex{\textcolor{red}{\textsuperscript{XXXX1 indx}}!Tod Georgs1900@\strich\emph{Der Tod Georgs} {[}1900{]}|pw}« dürfte bald fertig ſein, auch das Stück\pwindex{Hirschfeld, Georg 11.02.1873 – 17.01.1942@\textsc{Hirschfeld, Georg} (11.02.1873 – 17.01.1942), \emph{Schriftsteller}!Agnes Jordan. Schauspiel in fuenf Akten1897@\strich\emph{Agnes Jordan. Schauspiel in fünf Akten} {[}1897{]}|pwv} vom Hirſchfeld\pwindex{Hirschfeld, Georg 11.02.1873 – 17.01.1942@\textsc{Hirschfeld, Georg} (11.02.1873 – 17.01.1942), \emph{Schriftsteller}|pw}.\pend
           \pstart Ihr\spacefill\mbox{Hugo.}\pend{}
         
         \endnumbering\mylabel{h}\end{ledgroupsized}  \newcommand{\dateiname}{L00673}\newcommand{\titel}{Hugo von Hofmannsthal an Arthur Schnitzler, 2. 5. [1897]}\newcommand{\editorInnen}{Martin Anton Müller und Gerd-Hermann Susen}%% latex-leseansicht-abspann.tex
%% Abspann für die Leseansicht.
%% Der Schalter \ifkorrekturansicht ist bereits durch den Vorspann gesetzt.

%% latex-abspann.tex
%% Gemeinsamer Abspann für Korrekturansicht und Leseansicht.
%% Setzt den Schalter \ifkorrekturansicht voraus (gesetzt in den
%% einbindenden Dateien latex-korrekturansicht-abspann.tex bzw.
%% latex-leseansicht-abspann.tex).
%% ---------------------------------------------------------------

\normalsize

% Das esempio-Environment wird nur in der Leseansicht benötigt
\ifkorrekturansicht\else
\newenvironment{esempio}[3]%
{
    \vspace{1.5ex}
    \rlap{\underline{#1}}
    \par
    \setlength{\parindent}{0cm}
    \nopagebreak
    \leftskip=#2cm
    \rightskip=#3cm
}
{
    \par
}
\fi

\doendnotes{C}
\bigskip
\vfill

\clearpage

\footnotesize

\ifkorrekturansicht
  \lohead{\textsc{register}}
\fi

% theindex-Environment neu definieren ohne reledmac
\makeatletter
\renewenvironment{theindex}{%
  \ifkorrekturansicht
    \section*{\indexname}%
  \else
    \subsubsection*{Index der erwähnten Entitäten}%
  \fi
  \setlength{\parindent}{0pt}%
  \setlength{\parskip}{0pt plus 0.3pt}%
  \let\item\@idxitem
}{%
  \ifkorrekturansicht\clearpage\fi
}
\makeatother

\IfFileExists{\jobname-pw.ind}{\input{\jobname-pw.ind}}{}

% Quellenangabe nur in der Leseansicht
\ifkorrekturansicht\else
% Fallback-Definitionen, falls die .tex-Datei \titel etc. nicht gesetzt hat
\providecommand{\titel}{}
\providecommand{\editorInnen}{}
\providecommand{\dateiname}{\jobname}

\vspace{3cm}

\vfill

\footnotesize
\textsc{Quelle}: \titel. Herausgegeben von {\editorInnen}. In: \emph{Arthur Schnitzler: Briefwechsel mit Autorinnen und Autoren}.
 Digitale Edition, https://schnitzler-briefe.acdh.oeaw.ac.at/{\dateiname}.html (Stand \today)
\fi

\end{document}


      