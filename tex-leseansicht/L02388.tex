%% latex-korrekturansicht-vorspann.tex
%% Vorspann für die Korrekturansicht.
%% Lädt die gemeinsame Datei latex-vorspann.tex mit gesetztem Schalter.

\newif\ifkorrekturansicht
\korrekturansichttrue

\input{../tex-inputs/latex-vorspann}


\section[Hermann Bahr an Arthur Schnitzler, 9. 6. 1922]{L02388 Hermann Bahr an Arthur Schnitzler, 9. 6. 1922}
\nopagebreak\mylabel{L02388v}
\rehead{ }\normalsize\beginnumbering\briefempfaengerindex{Schnitzler, Arthur@\textsc{Schnitzler, Arthur}!zzzBahr, Hermann@\emph{von Hermann Bahr}!1922-06-091@{9. 6. 1922}|(be}
\toendnotes[C]{\smallbreak\pagebreak[2]}\Standort{CUL, Schnitzler, B 5b.}
\physDesc{Postkarte, 799 Zeichen
\newline{}Handschrift: schwarze Tinte, deutsche Kurrent
\newline{}Versand: 1) Stempel: »\nobreak{}\oindex{Muenchen@\textbf{München}, \emph{P.PPLA}|pwk}München, 10 6 22, 1–2 N\nobreak{}«.   2) mit Bleistift von unbekannter Hand die unvollständige Hausnummer
                                 in der Adressierung korrigiert zu »71«
\newline{}Schnitzler: mit Bleistift Vermerk: »\textsc{A}«, vermutlich für »Abzuschreiben«/»Abschrift« 
\newline{}Ordnung: mit Bleistift von unbekannter Hand nummeriert:
                                    »185« }
\buchAbdrucke{\weitereDrucke{Hermann Bahr, Arthur Schnitzler: \emph{Briefwechsel, Aufzeichnungen, Dokumente (1891–1931)}. Göttingen: \emph{Wallstein} 2018, S. 561.} }\toendnotes[C]{\smallbreak}\pstart{}{\pb}Hermann Bahr\pend{}\pstart{}München\oindex{Muenchen@\textbf{München}, \emph{P.PPLA}|pw}\pend{}\pstart{}Barerſtraße 50\oindex{Barerstrasse@\textbf{Barerstraße}, \emph{Straße (K.STR)}|pw}\pend{}{\bigskip}\pstart{}Arthur Schnitzler\pend{}\pstart{}\textsc{Wien XVIII\oindex{XVIII., Waehring@\textbf{XVIII., Währing}, \emph{A.ADM3}|pw}}\pend{}\pstart{}Sternwarteſtr. 1\oindex{Sternwartestrasse 71@\textbf{Sternwartestraße 71}, \emph{Wohngebäude (K.WHS)}|pw}\pend{}{\bigskip}\vspace{1em}
\pstart
           \raggedleft{}{\pb}9. 6. 22\pend
           
\pstart{}Lieber Arthur!\pend\vspace{0.5em}
\pstart
           Herzlichſten Dank für Deine mich herzlichſt erfreuende Karte! Ich hatte vor, Dir zu
               dieſem ominöſen Tag, der mir am End auch noch bevorſteht, nicht blos öffentlich,
               ſondern auch direkt zu ſagen, ein welch wichtiger Beſitz meines Lebens Dein
               Vorhandenſein iſt: ein Reichtum. Aber es ging beim beſten Willen nicht. Auszudrücken,
               was ich wirklich empfinde, war nie meine ſtarke Seite und je älter ich werde, deſto
               mehr kommt mir alles, ſo bald es ausgeſprochen wird, verlogen vor.\pend
           
\pstart
           Ich denke den ganzen Sommer (außer am 11.–13. Auguſt, wo
               ich \label{K_L02388-1v}\edtext{nach Salzburg\oindex{Salzburg@\textbf{Salzburg}, \emph{A.ADM2}|pw}}{\lemma{\textnormal{\emph{nach Salzburg}}}\Cendnote{\textnormal{Zur Eröffnung der \emph{Salzburger Festspiele}\orgindex{Salzburger Festspiele@Salzburger Festspiele|pwk}. Seine Frau\pwindex{Bahr-Mildenburg, Anna 29.11.1872 – 27.01.1947@\textsc{Bahr-Mildenburg, Anna} (29.11.1872 – 27.01.1947), \emph{Sänger/Sängerin}|pwkv} war für Hofmannsthals\pwindex{Hofmannsthal, Hugo von 1874-02-01 – 1929-07-15@\textsc{Hofmannsthal, Hugo von} (1874-02-01 – 1929-07-15), \emph{Schriftsteller/Schriftstellerin}|pwk}{ }\emph{Das Salzburger große
                     Welttheater}\pwindex{Salzburger grosse Welttheater@\emph{Das Salzburger große Welttheater}|pwk} engagiert.}}}\label{K_L02388-1}, und am
                  27.–30. Auguſt, wo ich \label{K_L02388-2v}\edtext{nach Heidelberg\oindex{Heidelberg@\textbf{Heidelberg}, \emph{P.PPLA3}|pw}}{\lemma{\textnormal{\emph{nach Heidelberg}}}\Cendnote{\textnormal{Er trat, etwas später, als hier
                  angedeutet, erst am 3. 9. 1922 als Redner am \emph{Verbandstag katholischer Akademiker}\orgindex{Kartellverband katholischer deutscher Studentenvereine@Kartellverband katholischer deutscher Studentenvereine|pwk} auf.}}}\label{K_L02388-2}{ }ſoll) hier {[}zu{]}{ }ſein und es wäre mir eine große Freude, Dich
               endlich wiederzuſehen.\pend
           
\pstart
           Herzlichſt Dein alter{\\[\baselineskip]}\spacefill\mbox{Hermann}\pend
           \leftskip=0em{}\selectlanguage{ngerman}\endnumbering\briefempfaengerindex{Schnitzler, Arthur@\textsc{Schnitzler, Arthur}!zzzBahr, Hermann@\emph{von Hermann Bahr}!1922-06-091@{9. 6. 1922}|)be}\mylabel{L02388h}  \normalsize

\doendnotes{C}
\bigskip
\vfill

\clearpage

\footnotesize

\lohead{\textsc{register}}

% Definiere theindex-Environment komplett neu ohne reledmac
\makeatletter
\renewenvironment{theindex}{%
  \section*{\indexname}%
  \setlength{\parindent}{0pt}%
  \setlength{\parskip}{0pt plus 0.3pt}%
  \let\item\@idxitem
}{%
  \clearpage
}
\makeatother

\IfFileExists{\jobname-pw.ind}{\input{\jobname-pw.ind}}{}

\end{document}

      