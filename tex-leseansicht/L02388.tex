%% latex-leseansicht-vorspann.tex
%% Vorspann für die Leseansicht.
%% Lädt die gemeinsame Datei latex-vorspann.tex mit nicht gesetztem Schalter.

\newif\ifkorrekturansicht
\korrekturansichtfalse

\input{../tex-inputs/latex-vorspann}


\section[Hermann Bahr an Arthur Schnitzler, 9. 6. 1922]{L02388 Hermann Bahr an Arthur Schnitzler, 9. 6. 1922}
\nopagebreak\mylabel{L02388v}
\rehead{ }\normalsize\beginnumbering\briefempfaengerindex{Schnitzler, Arthur@\textsc{Schnitzler, Arthur}!zzzBahr, Hermann@\emph{von Hermann Bahr}!1922-06-091@{9. 6. 1922}|(be}
\toendnotes[C]{\smallbreak\pagebreak[2]}
\correspDesc{Versand  durch Hermann Bahr am 9. 6. 1922 in München
\newline{}Erhalt  durch Arthur Schnitzler am 10 6 22 in Wien}\toendnotes[C]{\smallbreak}
\Standort{CUL, Schnitzler, B 5b.}
\physDesc{Postkarte, 799 Zeichen
\newline{}Handschrift: schwarze Tinte, deutsche Kurrent
\newline{}Versand: 1) Stempel: »\nobreak{}\oindex{München@\textbf{München}|pwk}München, 10 6 22, 1–2 N\nobreak{}«.   2) mit Bleistift von unbekannter Hand die unvollständige Hausnummer
                                 in der Adressierung korrigiert zu »71«
\newline{}Schnitzler: mit Bleistift Vermerk: »\textsc{A}«, vermutlich für »Abzuschreiben«/»Abschrift« 
\newline{}Ordnung: mit Bleistift von unbekannter Hand nummeriert:
                                    »185« }
\buchAbdrucke{\weitereDrucke{Hermann Bahr, Arthur Schnitzler: \emph{Briefwechsel, Aufzeichnungen, Dokumente (1891–1931)}. Herausgegeben von Kurt Ifkovits und Martin Anton Müller. Göttingen: \emph{Wallstein} 2018, S. 561.} }\toendnotes[C]{\smallbreak}\pstart{}{\pb}Hermann Bahr\pend{}\pstart{}München\oindex{München@\textbf{München}|pw}\pend{}\pstart{}Barerſtraße 50\oindex{Barerstraße@\textbf{Barerstraße}, \emph{Straße}|pw}\pend{}{\bigskip}\pstart{}Arthur Schnitzler\pend{}\pstart{}\textsc{Wien XVIII\oindex{XVIII., Währing@\textbf{XVIII., Währing}, \emph{Verwaltungsgebiet}|pw}}\pend{}\pstart{}Sternwarteſtr. 1\oindex{Wien@\textbf{Wien}!XVIII., Währing@\textbf{XVIII., Währing}!Sternwartestraße 71@\textbf{Sternwartestraße 71}, \emph{Wohngebäude}|pw}\pend{}{\bigskip}\vspace{1em}
\pstart
           \raggedleft{}{\pb}9. 6. 22\pend
           
\pstart{}Lieber Arthur!\pend\vspace{0.5em}
\pstart
           Herzlichſten Dank für Deine mich herzlichſt erfreuende Karte! Ich hatte vor, Dir zu
               dieſem ominöſen Tag, der mir am End auch noch bevorſteht, nicht blos öffentlich,{ }ſondern auch direkt zu{ }ſagen, ein welch wichtiger Beſitz meines Lebens Dein
               Vorhandenſein iſt: ein Reichtum. Aber es ging beim beſten Willen nicht. Auszudrücken,
               was ich wirklich empfinde, war nie meine{ }ſtarke Seite und je älter ich werde, deſto
               mehr kommt mir alles,{ }ſo bald es ausgeſprochen wird, verlogen vor.\pend
           
\pstart
           Ich denke den ganzen Sommer (außer am 11.–13. Auguſt, wo
               ich \label{K_L02388-1v}\edtext{nach Salzburg\oindex{Salzburg@\textbf{Salzburg}, \emph{Verwaltungsgebiet}|pw}}{\lemma{\textnormal{\emph{nach Salzburg}}}\Cendnote{\textnormal{Zur Eröffnung der \emph{Salzburger Festspiele}\orgindex{Salzburger Festspiele@Salzburger Festspiele|pwk}. Seine Frau\pwindex{Bahr-Mildenburg, Anna 29.\,11.\,1872 Wien – 27.\,1.\,1947 ebd.@\textsc{Bahr-Mildenburg, Anna} (29.\,11.\,1872 Wien – 27.\,1.\,1947 ebd.), \emph{Sängerin}|pwkv} war für Hofmannsthals\pwindex{Hofmannsthal, Hugo von 1.\,2.\,1874 Wien – 15.\,7.\,1929 Rodaun@\textsc{Hofmannsthal, Hugo von} (1.\,2.\,1874 Wien – 15.\,7.\,1929 Rodaun), \emph{Schriftsteller}|pwk}{ }\emph{Das Salzburger große
                     Welttheater}\pwindex{Hofmannsthal, Hugo von 1.\,2.\,1874 Wien – 15.\,7.\,1929 Rodaun@\textsc{Hofmannsthal, Hugo von} (1.\,2.\,1874 Wien – 15.\,7.\,1929 Rodaun), \emph{Schriftsteller}!Salzburger große Welttheater@\strich\emph{Das Salzburger große Welttheater}|pwk} engagiert.}}}\label{K_L02388-1}, und am
                  27.–30. Auguſt, wo ich \label{K_L02388-2v}\edtext{nach Heidelberg\oindex{Heidelberg@\textbf{Heidelberg}, \emph{Hauptstadt}|pw}}{\lemma{\textnormal{\emph{nach Heidelberg}}}\Cendnote{\textnormal{Er trat, etwas später, als hier
                  angedeutet, erst am 3. 9. 1922 als Redner am \emph{Verbandstag katholischer Akademiker}\orgindex{Kartellverband katholischer deutscher Studentenvereine@Kartellverband katholischer deutscher Studentenvereine|pwk} auf.}}}\label{K_L02388-2}{ }ſoll) hier {[}zu{]}{ }ſein und es wäre mir eine große Freude, Dich
               endlich wiederzuſehen.\pend
           
\pstart
           Herzlichſt Dein alter{\\[\baselineskip]}\spacefill\mbox{Hermann}\pend
           \leftskip=0em{}\selectlanguage{ngerman}\endnumbering\briefempfaengerindex{Schnitzler, Arthur@\textsc{Schnitzler, Arthur}!zzzBahr, Hermann@\emph{von Hermann Bahr}!1922-06-091@{9. 6. 1922}|)be}\mylabel{L02388h}  \newcommand{\dateiname}{L02388}\newcommand{\titel}{Hermann Bahr an Arthur Schnitzler, 9. 6. 1922}\newcommand{\editorInnen}{Herausgegeben von Martin Anton Müller}%% latex-leseansicht-abspann.tex
%% Abspann für die Leseansicht.
%% Der Schalter \ifkorrekturansicht ist bereits durch den Vorspann gesetzt.

%% latex-abspann.tex
%% Gemeinsamer Abspann für Korrekturansicht und Leseansicht.
%% Setzt den Schalter \ifkorrekturansicht voraus (gesetzt in den
%% einbindenden Dateien latex-korrekturansicht-abspann.tex bzw.
%% latex-leseansicht-abspann.tex).
%% ---------------------------------------------------------------

\normalsize

% Das esempio-Environment wird nur in der Leseansicht benötigt
\ifkorrekturansicht\else
\newenvironment{esempio}[3]%
{
    \vspace{1.5ex}
    \rlap{\underline{#1}}
    \par
    \setlength{\parindent}{0cm}
    \nopagebreak
    \leftskip=#2cm
    \rightskip=#3cm
}
{
    \par
}
\fi

\doendnotes{C}
\bigskip
\vfill

\clearpage

\footnotesize

\ifkorrekturansicht
  \lohead{\textsc{register}}
\fi

% theindex-Environment neu definieren ohne reledmac
\makeatletter
\renewenvironment{theindex}{%
  \ifkorrekturansicht
    \section*{\indexname}%
  \else
    \subsubsection*{Index der erwähnten Entitäten}%
  \fi
  \setlength{\parindent}{0pt}%
  \setlength{\parskip}{0pt plus 0.3pt}%
  \let\item\@idxitem
}{%
  \ifkorrekturansicht\clearpage\fi
}
\makeatother

\IfFileExists{\jobname-pw.ind}{\input{\jobname-pw.ind}}{}

% Quellenangabe nur in der Leseansicht
\ifkorrekturansicht\else
% Fallback-Definitionen, falls die .tex-Datei \titel etc. nicht gesetzt hat
\providecommand{\titel}{}
\providecommand{\editorInnen}{}
\providecommand{\dateiname}{\jobname}

\vspace{3cm}

\vfill

\footnotesize
\textsc{Quelle}: \titel. Herausgegeben von {\editorInnen}. In: \emph{Arthur Schnitzler: Briefwechsel mit Autorinnen und Autoren}.
 Digitale Edition, https://schnitzler-briefe.acdh.oeaw.ac.at/{\dateiname}.html (Stand \today)
\fi

\end{document}


