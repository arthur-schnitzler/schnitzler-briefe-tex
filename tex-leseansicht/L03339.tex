%% latex-leseansicht-vorspann.tex
%% Vorspann für die Leseansicht.
%% Lädt die gemeinsame Datei latex-vorspann.tex mit nicht gesetztem Schalter.

\newif\ifkorrekturansicht
\korrekturansichtfalse

\input{../tex-inputs/latex-vorspann}


         
         \renewcommand{\erwaehntePersonen}{Personen:  ?? [Englischlehrerin von Felix Salten], Kurt Aram, Mirjam Horwitz, Theodor Horwitz, Siegfried Jacobsohn, Felix Salten, George Bernard Shaw, Elisabeth Steinrück, Carl Wiene}
         \renewcommand{\erwaehnteInstitutionen}{Institutionen: Die Zeit. Wiener Wochenschrift}
         \renewcommand{\erwaehnteOrte}{Orte: Berlin, Bosnien und Herzegowina, Dalmatien, Deutsches Theater Berlin, London, Raimund-Theater, Wien}
         \renewcommand{\erwaehnteWerke}{Werke: Der Hund von Florenz, Der Schleier der Beatrice. Schauspiel in fünf Akten, Die Gespräche des göttlichen Pietro Aretino, Die Zeit, Ein Teufelskerl. Schauspiel in drei Akten, The Devil’s Disciple, Vom göttlichen Aretino, »Schleier der Beatrice.« Man telegraphirt uns aus Berlin, 7. d.}
               \section[ Felix Salten an Arthur Schnitzler, 3. 3. 1903]{ Felix Salten an Arthur Schnitzler, 3. 3. 1903}\nopagebreak\mylabel{v}\rehead{ }\begin{ledgroupsized}[t]{13cm}\normalsize\beginnumbering\briefempfaengerindex{Schnitzler, Arthur@\textsc{Schnitzler, Arthur}!zzzSalten, Felix@\emph{von Felix Salten}!1903-03-031@{3. 3. 1903}|(be} \toendnotes[C]{\smallbreak\pagebreak[2]} \Standort{CUL, Schnitzler, B 89, A 2.}
\physDesc{Brief, 1 Blatt, 4 Seiten, 2821 Zeichen
\newline{}Handschrift: Bleistift, lateinische Kurrent
\newline{}Schnitzler: mit Bleistift »\textsc{Salten}« vermerkt 
\newline{}Ordnung: mit Bleistift von unbekannter Hand nummeriert: »164« }\toendnotes[C]{\smallbreak}\pstart
           \raggedleft{}{\pb}Wien\oindex{Wien@\textbf{Wien}|pw}, 3. III. 03\pend
           \pstart
           Lieber, zur \label{K_L03339-1v}\edtext{Premiere\pwindex{Schnitzler, Arthur 15.05.1862 – 21.10.1931@\textsc{Schnitzler, Arthur} (15.05.1862 – 21.10.1931), \emph{Schriftsteller, Mediziner}!Schleier der Beatrice. Schauspiel in fuenf Akten1900-12-01@\strich\emph{Der Schleier der Beatrice. Schauspiel in fünf Akten} {[}1900-12-01{]}|pwv}}{\lemma{\textnormal{\emph{Premiere}}}\Cendnote{\textnormal{Schnitzler\pwindex{Schnitzler, Arthur 15.05.1862 – 21.10.1931@\textsc{Schnitzler, Arthur} (15.05.1862 – 21.10.1931), \emph{Schriftsteller, Mediziner}|pwk} weilte zur Vorbereitung der
                  Premiere von \emph{Der Schleier der Beatrice}\pwindex{Schnitzler, Arthur 15.05.1862 – 21.10.1931@\textsc{Schnitzler, Arthur} (15.05.1862 – 21.10.1931), \emph{Schriftsteller, Mediziner}!Schleier der Beatrice. Schauspiel in fuenf Akten1900-12-01@\strich\emph{Der Schleier der Beatrice. Schauspiel in fünf Akten} {[}1900-12-01{]}|pwk} in Berlin\oindex{Berlin@\textbf{Berlin}|pwk}. Diese fand am 7. 3. 1903 am Deutschen Theater\oindex{Deutsches Theater Berlin@\textbf{Deutsches Theater Berlin}|pwk} in seiner Anwesenheit
                  statt.}}}\label{K_L03339-1h} kann ich nun leider doch nicht nach Berlin\oindex{Berlin@\textbf{Berlin}|pw}; schade. Ich werde erst so gegen 14.\textsuperscript{ten} März reisen; und habe vorher noch enorm viel
               zu thun. Was sagen Sie zum \label{K_L03339-2v}\edtext{Teufelskerl\pwindex{Shaw, George Bernard 26.07.1856 – 02.11.1950@\textsc{Shaw, George Bernard} (26.07.1856 – 02.11.1950), \emph{Schriftsteller}!Teufelskerl. Schauspiel in drei Akten1903@\strich\emph{Ein Teufelskerl. Schauspiel in drei Akten} {[}1903{]}|pw}? Das Stück hat Herr Wiene\pwindex{Wiene, Carl 1852-05-08 – 1913-02-13@\textsc{Wiene, Carl} (1852-05-08 – 1913-02-13), \emph{Schauspieler}|pw}}{\lemma{\textnormal{\emph{Teufelskerl? … Wiene}}}\Cendnote{\textnormal{Carl Wiene\pwindex{Wiene, Carl 1852-05-08 – 1913-02-13@\textsc{Wiene, Carl} (1852-05-08 – 1913-02-13), \emph{Schauspieler}|pwk} trat als Gastschauspieler am
                     25. 2. 1903 im Raimund-Theater\oindex{Raimund-Theater@\textbf{Raimund-Theater}|pwk} in der Hauptrolle in \emph{Ein
                     Teufelskerl}\pwindex{Shaw, George Bernard 26.07.1856 – 02.11.1950@\textsc{Shaw, George Bernard} (26.07.1856 – 02.11.1950), \emph{Schriftsteller}!Teufelskerl. Schauspiel in drei Akten1903@\strich\emph{Ein Teufelskerl. Schauspiel in drei Akten} {[}1903{]}|pwk} (\emph{The Devil’s Disciple}\pwindex{Shaw, George Bernard 26.07.1856 – 02.11.1950@\textsc{Shaw, George Bernard} (26.07.1856 – 02.11.1950), \emph{Schriftsteller}!Devil s Disciple1897@\strich\emph{The Devil’s Disciple} {[}1897{]}|pwk})
                  von George Bernard Shaw\pwindex{Shaw, George Bernard 26.07.1856 – 02.11.1950@\textsc{Shaw, George Bernard} (26.07.1856 – 02.11.1950), \emph{Schriftsteller}|pwk} auf. Schnitzler\pwindex{Schnitzler, Arthur 15.05.1862 – 21.10.1931@\textsc{Schnitzler, Arthur} (15.05.1862 – 21.10.1931), \emph{Schriftsteller, Mediziner}|pwk} war zu diesem Zeitpunkt bereits in
                     Berlin\oindex{Berlin@\textbf{Berlin}|pwk} und sah die Vorstellung
                  nicht.}}}\label{K_L03339-2h} ruinirt, wie vorauszusehen war. Sehr fühlbar wurde mir die tiefe
                  Unmoral\textcolor{gray}{,} die darin steckt, wenn das Alter sich als Jugend
               verkleidet und geberdet. Der Widerwille, den man bei solchem Schauspiel empfindet
               geht bis an ein sexuelles Missbehagen, wenigstens begreift man die Nervenzerrüttung
               einer Frau, an der ein impotenter Mann heuchlerische Versuche vornimmt, denn mit
               ähnlicher Bereitwilligkeit zur Empfängnis sitzt so ein Publikum im Theater. Mir wäre
               es sehr lieb, wenn Sie mir statt einer Ansichtskarte einmal näheres über die Proben
               ec. Berlin\oindex{Berlin@\textbf{Berlin}|pw} ec. schrieben, falls es Ihre Zeit
               gestattet.\pend
           \pstart
           In Angelegenheit der \label{K_L03339-3v}\edtext{Mirjam H.\pwindex{Horwitz, Mirjam 1882-06-15 – 1967-09-26@\textsc{Horwitz, Mirjam} (1882-06-15 – 1967-09-26), \emph{Theaterleiterin, Schauspielerin}|pw}}{\lemma{\textnormal{\emph{Mirjam H.}}}\Cendnote{\textnormal{Salten\pwindex{Salten, Felix 06.09.1869 – 08.10.1945@\textsc{Salten, Felix} (06.09.1869 – 08.10.1945), \emph{Schriftsteller, Journalist, Chefredakteur}|pwk} und die Schauspielerin Mirjam Horwitz\pwindex{Horwitz, Mirjam 1882-06-15 – 1967-09-26@\textsc{Horwitz, Mirjam} (1882-06-15 – 1967-09-26), \emph{Theaterleiterin, Schauspielerin}|pwk}
                  hatten eine Affäre, die, wenn man die Hinweise zusammenliest, von ihrem Vater\pwindex{Horwitz, Theodor 1845/1846 – 1913-02-08@\textsc{Horwitz, Theodor} (1845/1846 – 1913-02-08)|pwkv} beendet wurde, indem er eine Entscheidung von Salten\pwindex{Salten, Felix 06.09.1869 – 08.10.1945@\textsc{Salten, Felix} (06.09.1869 – 08.10.1945), \emph{Schriftsteller, Journalist, Chefredakteur}|pwk} 
                  forderte. Salten\pwindex{Salten, Felix 06.09.1869 – 08.10.1945@\textsc{Salten, Felix} (06.09.1869 – 08.10.1945), \emph{Schriftsteller, Journalist, Chefredakteur}|pwk} sah darin die Möglichkeit, die Affäre hinter sich zu lassen, 
                  und bat Schnitzler\pwindex{Schnitzler, Arthur 15.05.1862 – 21.10.1931@\textsc{Schnitzler, Arthur} (15.05.1862 – 21.10.1931), \emph{Schriftsteller, Mediziner}|pwk} um Vermittlung, was er während seines Berlin\oindex{Berlin@\textbf{Berlin}|pwk}-Aufenthalts tat. Horwitz\pwindex{Horwitz, Mirjam 1882-06-15 – 1967-09-26@\textsc{Horwitz, Mirjam} (1882-06-15 – 1967-09-26), \emph{Theaterleiterin, Schauspielerin}|pwk} war
                  auch eine Freundin von Schnitzlers\pwindex{Schnitzler, Arthur 15.05.1862 – 21.10.1931@\textsc{Schnitzler, Arthur} (15.05.1862 – 21.10.1931), \emph{Schriftsteller, Mediziner}|pwk} Schwägerin Elisabeth Gussmann\pwindex{Steinrueck, Elisabeth 19.11.1885 – 07.04.1920@\textsc{Steinrück, Elisabeth} (19.11.1885 – 07.04.1920)|pwk}.}}}\label{K_L03339-3h} muß ich Sie
               nochmals bemühen: bald und möglichst schonend. Sie schreibt mir heute einen confusen Brief; ob sie »nach hier« kommen
               soll, oder wann ich »nach dort« komme, ferner, dass ich nicht durch mein Wort an
               ihren Vater\pwindex{Horwitz, Theodor 1845/1846 – 1913-02-08@\textsc{Horwitz, Theodor} (1845/1846 – 1913-02-08)|pwv} gebunden bin,
               falls \uline{sie} mit \uline{mir}
               verkehrt, endlich, dass ich an einen Vertrauten von ihr schreiben soll, das sei auch
               nicht gegen mein Versprechen ec. Dann noch recht enervirende Dinge
               von »sich angehören vor aller {\pb}Welt –« »den Leuten zum Trotz« ec. und in diesem Stil, der die Liebe recht
               unangenehm macht.\pend
           \pstart
           Das Wesentliche an der Sache: dass ich ihrem Vater\pwindex{Horwitz, Theodor 1845/1846 – 1913-02-08@\textsc{Horwitz, Theodor} (1845/1846 – 1913-02-08)|pwv} wahrscheinlich kein Versprechen gegeben hätte, wenn
               ich Mirjam\pwindex{Horwitz, Mirjam 1882-06-15 – 1967-09-26@\textsc{Horwitz, Mirjam} (1882-06-15 – 1967-09-26), \emph{Theaterleiterin, Schauspielerin}|pw} sehr lieb hätte. Ferner: dass ich
               aber, nun ich das Versprechen gab, keine Lust habe Geschichten zu machen. Bringen Sie
               ihr das bitte schonend bei. D\textcolor{gray}{a}s mit dem Versprechen nämlich, und
               vor allem, dass sie nichts gewinnt, wenn sie gewaltsame Streiche macht, da mir solche
               von jeher zuwider waren. Aber bitte, seien Sie sehr schonend, weil sie mir mit
               Selbstmord droht, was auch eine hübsche Gewohnheit von ihr ist.\pend
           \pstart
           Am 14. fahre ich auf 8 Tage nach Berlin\oindex{Berlin@\textbf{Berlin}|pw}. Im April voraussichtlich
               nach Bosnien\oindex{Bosnien und Herzegowina@\textbf{Bosnien und Herzegowina}|pw} und Dalmatien\oindex{Dalmatien@\textbf{Dalmatien}|pw}. Im Mai nach London\oindex{London@\textbf{London}|pw} auf 14 Tage.\pend
           \pstart
           {\pb}Ich lese jetzt die »\label{K_L03339-4v}\edtext{Gespräche des göttlichen Aretino\pwindex{\textcolor{red}{\textsuperscript{XXXX1 indx}}!Gespraeche des goettlichen Pietro Aretino1903@\strich\emph{Die Gespräche des göttlichen Pietro Aretino} {[}1903{]}|pw}}{\lemma{\textnormal{\emph{Gespräche … Aretino}}}\Cendnote{\textnormal{Salten\pwindex{Salten, Felix 06.09.1869 – 08.10.1945@\textsc{Salten, Felix} (06.09.1869 – 08.10.1945), \emph{Schriftsteller, Journalist, Chefredakteur}|pwk} schrieb auch ein Feuilleton\pwindex{Salten, Felix 06.09.1869 – 08.10.1945@\textsc{Salten, Felix} (06.09.1869 – 08.10.1945), \emph{Schriftsteller, Journalist, Chefredakteur}!Vom goettlichen Aretino1903-03-15@\strich\emph{Vom göttlichen Aretino} {[}1903-03-15{]}|pwkv} darüber: Felix Salten\pwindex{Salten, Felix 06.09.1869 – 08.10.1945@\textsc{Salten, Felix} (06.09.1869 – 08.10.1945), \emph{Schriftsteller, Journalist, Chefredakteur}|pwk}: \emph{Vom göttlichen Aretino}\pwindex{Salten, Felix 06.09.1869 – 08.10.1945@\textsc{Salten, Felix} (06.09.1869 – 08.10.1945), \emph{Schriftsteller, Journalist, Chefredakteur}!Vom goettlichen Aretino1903-03-15@\strich\emph{Vom göttlichen Aretino} {[}1903-03-15{]}|pwk}. In: \emph{Die Zeit}\pwindex{Zeit1902-09-27 – 1919@\emph{Die Zeit} {[}1902-09-27 – 1919{]}|pwk}, Jg. 2, Nr. 165, 15. 3. 1903, Morgenblatt, S. 1–2.}}}\label{K_L03339-4h},« und finde darin zu
               meinem Erstaunen die römische Buhlerin, die Bekenntnisse ablegt. Sie wissen, dass ich
               ein solches Buch schreiben wollte. Arbeiten kann ich nur wenig, da mir die Zeit\orgindex{Zeit. Wiener Wochenschrift@Die Zeit. Wiener Wochenschrift|pw} fast
               alles weg nimmt. Nun soll Aram\pwindex{Aram, Kurt 1869-01-28 – 1934-07-10@\textsc{Aram, Kurt} (1869-01-28 – 1934-07-10), \emph{Schriftsteller, Journalist}|pw} fort, und ich
                  für 8400fl\textcolor{gray}{.} jährlich auch das Feuilleton übernehmen; außerdem
               heißt es, – mit mir wurde \textcolor{gray}{noch} nicht davon gesprochen – dass ich
               Chef-Stellvertreter werden soll. Ich wünschte mir, dass der Tag dann – 36 Stunden haben möge, eine
               Erhöhung, mit der ich noch mehr einverstanden wäre. Für London\oindex{London@\textbf{London}|pw} habe ich mir jetzt eine 
               Engländerin\pwindex{?? [Englischlehrerin von Felix Salten] @\textsc{?? [Englischlehrerin von Felix Salten]}|pwv}
                angeschafft,
               die 3mal die Woche kommt. Ich \label{K_L03339-5v}\edtext{beginne
               den {[}»{]}Hund {\pb}von Florenz\pwindex{Salten, Felix 06.09.1869 – 08.10.1945@\textsc{Salten, Felix} (06.09.1869 – 08.10.1945), \emph{Schriftsteller, Journalist, Chefredakteur}!Hund von Florenz1923@\strich\emph{Der Hund von Florenz} {[}1923{]}|pw}«}{\lemma{\textnormal{\emph{beginne … Florenz«}}}\Cendnote{\textnormal{Salten\pwindex{Salten, Felix 06.09.1869 – 08.10.1945@\textsc{Salten, Felix} (06.09.1869 – 08.10.1945), \emph{Schriftsteller, Journalist, Chefredakteur}|pwk} arbeitete noch Jahre an der Novelle\pwindex{Salten, Felix 06.09.1869 – 08.10.1945@\textsc{Salten, Felix} (06.09.1869 – 08.10.1945), \emph{Schriftsteller, Journalist, Chefredakteur}!Hund von Florenz1923@\strich\emph{Der Hund von Florenz} {[}1923{]}|pwkv}, die erst 1923 erschien. Siehe Felix Salten an Arthur Schnitzler, 15. 8. 1907 und 18. 3. 1921.}}}\label{K_L03339-5h} den ich vielleicht dann in Bosnien\oindex{Bosnien und Herzegowina@\textbf{Bosnien und Herzegowina}|pw} fertig mache.\pend
           \pstart
           Schreiben Sie mir bitte recht bald. Bin neugierig, wie sich Herr \label{K_L03339-6v}\edtext{Jacobsohn\pwindex{Jacobsohn, Siegfried 28.01.1881 – 03.12.1926@\textsc{Jacobsohn, Siegfried} (28.01.1881 – 03.12.1926), \emph{Journalist, Kritiker, Publizist}|pw} benehmen}{\lemma{\textnormal{\emph{Jacobsohn benehmen}}}\Cendnote{\textnormal{Jacobsohn\pwindex{Jacobsohn, Siegfried 28.01.1881 – 03.12.1926@\textsc{Jacobsohn, Siegfried} (28.01.1881 – 03.12.1926), \emph{Journalist, Kritiker, Publizist}|pwk} war der Berlin\oindex{Berlin@\textbf{Berlin}|pwk}er Theaterkorrespondent der
                  \emph{Zeit}\orgindex{Zeit. Wiener Wochenschrift@Die Zeit. Wiener Wochenschrift|pwk}. Salten\pwindex{Salten, Felix 06.09.1869 – 08.10.1945@\textsc{Salten, Felix} (06.09.1869 – 08.10.1945), \emph{Schriftsteller, Journalist, Chefredakteur}|pwk} dürfte hier seiner Neugier Ausdruck
                  verliehen haben, wie Jacobsohn\pwindex{Jacobsohn, Siegfried 28.01.1881 – 03.12.1926@\textsc{Jacobsohn, Siegfried} (28.01.1881 – 03.12.1926), \emph{Journalist, Kritiker, Publizist}|pwk} die Premiere von \emph{Der Schleier der Beatrice}\pwindex{Schnitzler, Arthur 15.05.1862 – 21.10.1931@\textsc{Schnitzler, Arthur} (15.05.1862 – 21.10.1931), \emph{Schriftsteller, Mediziner}!Schleier der Beatrice. Schauspiel in fuenf Akten1900-12-01@\strich\emph{Der Schleier der Beatrice. Schauspiel in fünf Akten} {[}1900-12-01{]}|pwk} besprechen würde. Die Depesche lautetete:
                  »Der Dichter wurde häufig gerufen, und ſtarker Beifall behauptete ſich ſiegreich gegen einzelne energiſche Ziſcher«. ([Siegfried Jacobsohn\pwindex{Jacobsohn, Siegfried 28.01.1881 – 03.12.1926@\textsc{Jacobsohn, Siegfried} (28.01.1881 – 03.12.1926), \emph{Journalist, Kritiker, Publizist}|pwk}]: \emph{»Schleier der Beatrice.« Man telegraphirt uns aus Berlin, 7. d.}\pwindex{Schleier der Beatrice.« Man telegraphirt uns aus Berlin, 7. d.1903-03-08@\emph{»Schleier der Beatrice.« Man telegraphirt uns aus Berlin, 7. d.} {[}1903-03-08{]}|pwk} 
                     In: \emph{Die Zeit}\pwindex{Zeit1902-09-27 – 1919@\emph{Die Zeit} {[}1902-09-27 – 1919{]}|pwk}, Jg. 2, Nr. 158, S. 5.)
               }}}\label{K_L03339-6h} wird.\pend
           \pstart
           herzlichst Ihr {\\[\baselineskip]}\spacefill\mbox{Salten}\pend
           \leftskip=0em{}
         
         \endnumbering\mylabel{h}\end{ledgroupsized}  \newcommand{\dateiname}{L03339}\newcommand{\titel}{Felix Salten an Arthur Schnitzler, 3. 3. 1903}\newcommand{\editorInnen}{Martin Anton Müller und Laura Untner}%% latex-leseansicht-abspann.tex
%% Abspann für die Leseansicht.
%% Der Schalter \ifkorrekturansicht ist bereits durch den Vorspann gesetzt.

%% latex-abspann.tex
%% Gemeinsamer Abspann für Korrekturansicht und Leseansicht.
%% Setzt den Schalter \ifkorrekturansicht voraus (gesetzt in den
%% einbindenden Dateien latex-korrekturansicht-abspann.tex bzw.
%% latex-leseansicht-abspann.tex).
%% ---------------------------------------------------------------

\normalsize

% Das esempio-Environment wird nur in der Leseansicht benötigt
\ifkorrekturansicht\else
\newenvironment{esempio}[3]%
{
    \vspace{1.5ex}
    \rlap{\underline{#1}}
    \par
    \setlength{\parindent}{0cm}
    \nopagebreak
    \leftskip=#2cm
    \rightskip=#3cm
}
{
    \par
}
\fi

\doendnotes{C}
\bigskip
\vfill

\clearpage

\footnotesize

\ifkorrekturansicht
  \lohead{\textsc{register}}
\fi

% theindex-Environment neu definieren ohne reledmac
\makeatletter
\renewenvironment{theindex}{%
  \ifkorrekturansicht
    \section*{\indexname}%
  \else
    \subsubsection*{Index der erwähnten Entitäten}%
  \fi
  \setlength{\parindent}{0pt}%
  \setlength{\parskip}{0pt plus 0.3pt}%
  \let\item\@idxitem
}{%
  \ifkorrekturansicht\clearpage\fi
}
\makeatother

\IfFileExists{\jobname-pw.ind}{\input{\jobname-pw.ind}}{}

% Quellenangabe nur in der Leseansicht
\ifkorrekturansicht\else
% Fallback-Definitionen, falls die .tex-Datei \titel etc. nicht gesetzt hat
\providecommand{\titel}{}
\providecommand{\editorInnen}{}
\providecommand{\dateiname}{\jobname}

\vspace{3cm}

\vfill

\footnotesize
\textsc{Quelle}: \titel. Herausgegeben von {\editorInnen}. In: \emph{Arthur Schnitzler: Briefwechsel mit Autorinnen und Autoren}.
 Digitale Edition, https://schnitzler-briefe.acdh.oeaw.ac.at/{\dateiname}.html (Stand \today)
\fi

\end{document}


      