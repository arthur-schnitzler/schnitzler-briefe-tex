%% latex-leseansicht-vorspann.tex
%% Vorspann für die Leseansicht.
%% Lädt die gemeinsame Datei latex-vorspann.tex mit nicht gesetztem Schalter.

\newif\ifkorrekturansicht
\korrekturansichtfalse

\input{../tex-inputs/latex-vorspann}


\section[ Felix Salten an Arthur Schnitzler, 3. 3. 1903]{L03339 Felix Salten an Arthur Schnitzler,  3. 3. 1903}
\nopagebreak\mylabel{L03339v}
\rehead{ }\normalsize\beginnumbering\briefempfaengerindex{Schnitzler, Arthur@\textsc{Schnitzler, Arthur}!zzzSalten, Felix@\emph{von Felix Salten}!1903-03-031@{3. 3. 1903}|(be}
\toendnotes[C]{\smallbreak\pagebreak[2]}
\correspDesc{Versand  durch Felix Salten am 3. 3. 1903 in Wien
\newline{}Erhalt  durch Arthur Schnitzler am 4. 3. 1903 in Berlin}\toendnotes[C]{\smallbreak}
\Standort{CUL, Schnitzler, B 89, A 2.}
\physDesc{Brief, 1 Blatt, 4 Seiten, 2821 Zeichen
\newline{}Handschrift: Bleistift, lateinische Kurrent
\newline{}Schnitzler: mit Bleistift »\textsc{Salten}« vermerkt 
\newline{}Ordnung: mit Bleistift von unbekannter Hand nummeriert: »164« }\toendnotes[C]{\smallbreak}
\pstart
           \raggedleft{}{\pb}Wien\oindex{Wien@\textbf{Wien}, \emph{Verwaltungsgebiet}|pw}, 3. III. 03\pend
           \vspace{0.5em}
\pstart
           Lieber, zur \label{K_L03339-1v}\edtext{Premiere\pwindex{Schnitzler, Arthur 15.\,5.\,1862 Wien – 21.\,10.\,1931 ebd.@\textsc{Schnitzler, Arthur} (15.\,5.\,1862 Wien – 21.\,10.\,1931 ebd.), \emph{Schriftsteller, Mediziner}!Schleier der Beatrice. Schauspiel in fünf Akten@\strich\emph{Der Schleier der Beatrice. Schauspiel in fünf Akten}|pwv}}{\lemma{\textnormal{\emph{Premiere}}}\Cendnote{\textnormal{Schnitzler weilte zur Vorbereitung der
                  Premiere von \emph{Der Schleier der Beatrice}\pwindex{Schnitzler, Arthur 15.\,5.\,1862 Wien – 21.\,10.\,1931 ebd.@\textsc{Schnitzler, Arthur} (15.\,5.\,1862 Wien – 21.\,10.\,1931 ebd.), \emph{Schriftsteller, Mediziner}!Schleier der Beatrice. Schauspiel in fünf Akten@\strich\emph{Der Schleier der Beatrice. Schauspiel in fünf Akten}|pwk} in Berlin\oindex{Berlin@\textbf{Berlin}, \emph{Hauptstadt}|pwk}. Diese fand am 7. 3. 1903 am Deutschen Theater\oindex{Deutsches Theater Berlin@\textbf{Deutsches Theater Berlin}, \emph{Theater}|pwk} in seiner Anwesenheit
                  statt.}}}\label{K_L03339-1} kann ich nun leider doch nicht nach Berlin\oindex{Berlin@\textbf{Berlin}, \emph{Hauptstadt}|pw}; schade. Ich werde erst so gegen 14.\textsuperscript{ten} März reisen; und habe vorher noch enorm viel
               zu thun. Was sagen Sie zum \label{K_L03339-2v}\edtext{Teufelskerl\pwindex{Shaw, George Bernard 26.\,7.\,1856 Dublin – 2.\,11.\,1950 Ayot Saint Lawrence@\textsc{Shaw, George Bernard} (26.\,7.\,1856 Dublin – 2.\,11.\,1950 Ayot Saint Lawrence), \emph{Schriftsteller}!Teufelskerl. Schauspiel in drei Akten@\strich\emph{Ein Teufelskerl. Schauspiel in drei Akten}|pw}? Das Stück hat Herr Wiene\pwindex{Wiene, Carl 8.\,5.\,1852 Wien – 13.\,2.\,1913 Berlin@\textsc{Wiene, Carl} (8.\,5.\,1852 Wien – 13.\,2.\,1913 Berlin), \emph{Schauspieler}|pw}}{\lemma{\textnormal{\emph{Teufelskerl? … Wiene}}}\Cendnote{\textnormal{Carl Wiene\pwindex{Wiene, Carl 8.\,5.\,1852 Wien – 13.\,2.\,1913 Berlin@\textsc{Wiene, Carl} (8.\,5.\,1852 Wien – 13.\,2.\,1913 Berlin), \emph{Schauspieler}|pwk} trat als Gastschauspieler am
                     25. 2. 1903 im Raimund-Theater\oindex{Wien@\textbf{Wien}!VI., Mariahilf@\textbf{VI., Mariahilf}!Raimund-Theater@\textbf{Raimund-Theater}, \emph{Theater}|pwk} in der Hauptrolle in \emph{Ein
                     Teufelskerl}\pwindex{Shaw, George Bernard 26.\,7.\,1856 Dublin – 2.\,11.\,1950 Ayot Saint Lawrence@\textsc{Shaw, George Bernard} (26.\,7.\,1856 Dublin – 2.\,11.\,1950 Ayot Saint Lawrence), \emph{Schriftsteller}!Teufelskerl. Schauspiel in drei Akten@\strich\emph{Ein Teufelskerl. Schauspiel in drei Akten}|pwk} (\emph{The Devil’s Disciple}\pwindex{Shaw, George Bernard 26.\,7.\,1856 Dublin – 2.\,11.\,1950 Ayot Saint Lawrence@\textsc{Shaw, George Bernard} (26.\,7.\,1856 Dublin – 2.\,11.\,1950 Ayot Saint Lawrence), \emph{Schriftsteller}!Devil’s Disciple@\strich\emph{The Devil’s Disciple}|pwk})
                  von George Bernard Shaw\pwindex{Shaw, George Bernard 26.\,7.\,1856 Dublin – 2.\,11.\,1950 Ayot Saint Lawrence@\textsc{Shaw, George Bernard} (26.\,7.\,1856 Dublin – 2.\,11.\,1950 Ayot Saint Lawrence), \emph{Schriftsteller}|pwk} auf. Schnitzler war zu diesem Zeitpunkt bereits in
                     Berlin\oindex{Berlin@\textbf{Berlin}, \emph{Hauptstadt}|pwk} und sah die Vorstellung
                  nicht.}}}\label{K_L03339-2} ruinirt, wie vorauszusehen war. Sehr fühlbar wurde mir die tiefe
                  Unmoral\textcolor{gray}{,} die darin steckt, wenn das Alter sich als Jugend
               verkleidet und geberdet. Der Widerwille, den man bei solchem Schauspiel empfindet
               geht bis an ein sexuelles Missbehagen, wenigstens begreift man die Nervenzerrüttung
               einer Frau, an der ein impotenter Mann heuchlerische Versuche vornimmt, denn mit
               ähnlicher Bereitwilligkeit zur Empfängnis sitzt so ein Publikum im Theater. Mir wäre
               es sehr lieb, wenn Sie mir statt einer Ansichtskarte einmal näheres über die Proben
               ec. Berlin\oindex{Berlin@\textbf{Berlin}, \emph{Hauptstadt}|pw} ec. schrieben, falls es Ihre Zeit
               gestattet.\pend
           
\pstart
           In Angelegenheit der \label{K_L03339-3v}\edtext{Mirjam H.\pwindex{Horwitz, Mirjam 15.\,6.\,1882 Berlin – 26.\,9.\,1967 Lütjensee@\textsc{Horwitz, Mirjam} (15.\,6.\,1882 Berlin – 26.\,9.\,1967 Lütjensee), \emph{Theaterleiterin, Schauspielerin}|pw}}{\lemma{\textnormal{\emph{Mirjam H.}}}\Cendnote{\textnormal{Salten\pwindex{Salten, Felix 6.\,9.\,1869 Budapest – 8.\,10.\,1945 Zürich@\textsc{Salten, Felix} (6.\,9.\,1869 Budapest – 8.\,10.\,1945 Zürich), \emph{Schriftsteller, Journalist, Chefredakteur}|pwk} und die Schauspielerin Mirjam Horwitz\pwindex{Horwitz, Mirjam 15.\,6.\,1882 Berlin – 26.\,9.\,1967 Lütjensee@\textsc{Horwitz, Mirjam} (15.\,6.\,1882 Berlin – 26.\,9.\,1967 Lütjensee), \emph{Theaterleiterin, Schauspielerin}|pwk}
                  hatten eine Affäre, die, wenn man die Hinweise zusammenliest, von ihrem Vater\pwindex{Horwitz, Theodor 1845/1846 – 8.\,2.\,1913@\textsc{Horwitz, Theodor} (1845/1846 – 8.\,2.\,1913)|pwkv} beendet wurde, indem er eine Entscheidung von Salten\pwindex{Salten, Felix 6.\,9.\,1869 Budapest – 8.\,10.\,1945 Zürich@\textsc{Salten, Felix} (6.\,9.\,1869 Budapest – 8.\,10.\,1945 Zürich), \emph{Schriftsteller, Journalist, Chefredakteur}|pwk} 
                  forderte. Salten\pwindex{Salten, Felix 6.\,9.\,1869 Budapest – 8.\,10.\,1945 Zürich@\textsc{Salten, Felix} (6.\,9.\,1869 Budapest – 8.\,10.\,1945 Zürich), \emph{Schriftsteller, Journalist, Chefredakteur}|pwk} sah darin die Möglichkeit, die Affäre hinter sich zu lassen, 
                  und bat Schnitzler um Vermittlung, was er während seines Berlin\oindex{Berlin@\textbf{Berlin}, \emph{Hauptstadt}|pwk}-Aufenthalts tat. Horwitz\pwindex{Horwitz, Mirjam 15.\,6.\,1882 Berlin – 26.\,9.\,1967 Lütjensee@\textsc{Horwitz, Mirjam} (15.\,6.\,1882 Berlin – 26.\,9.\,1967 Lütjensee), \emph{Theaterleiterin, Schauspielerin}|pwk} war
                  auch eine Freundin von Schnitzlers Schwägerin Elisabeth Gussmann\pwindex{Steinrück, Elisabeth 19.\,11.\,1885 – 7.\,4.\,1920 Partenkirchen@\textsc{Steinrück, Elisabeth} (19.\,11.\,1885 – 7.\,4.\,1920 Partenkirchen)|pwk}.}}}\label{K_L03339-3} muß ich Sie
               nochmals bemühen: bald und möglichst schonend. Sie schreibt mir heute einen confusen Brief; ob sie »nach hier« kommen
               soll, oder wann ich »nach dort« komme, ferner, dass ich nicht durch mein Wort an
               ihren Vater\pwindex{Horwitz, Theodor 1845/1846 – 8.\,2.\,1913@\textsc{Horwitz, Theodor} (1845/1846 – 8.\,2.\,1913)|pwv} gebunden bin,
               falls \uline{sie} mit \uline{mir}
               verkehrt, endlich, dass ich an einen Vertrauten von ihr schreiben soll, das sei auch
               nicht gegen mein Versprechen ec. Dann noch recht enervirende Dinge
               von »sich angehören vor aller {\pb}Welt –« »den Leuten zum Trotz« ec. und in diesem Stil, der die Liebe recht
               unangenehm macht.\pend
           
\pstart
           Das Wesentliche an der Sache: dass ich ihrem Vater\pwindex{Horwitz, Theodor 1845/1846 – 8.\,2.\,1913@\textsc{Horwitz, Theodor} (1845/1846 – 8.\,2.\,1913)|pwv} wahrscheinlich kein Versprechen gegeben hätte, wenn
               ich Mirjam\pwindex{Horwitz, Mirjam 15.\,6.\,1882 Berlin – 26.\,9.\,1967 Lütjensee@\textsc{Horwitz, Mirjam} (15.\,6.\,1882 Berlin – 26.\,9.\,1967 Lütjensee), \emph{Theaterleiterin, Schauspielerin}|pw} sehr lieb hätte. Ferner: dass ich
               aber, nun ich das Versprechen gab, keine Lust habe Geschichten zu machen. Bringen Sie
               ihr das bitte schonend bei. D\textcolor{gray}{a}s mit dem Versprechen nämlich, und
               vor allem, dass sie nichts gewinnt, wenn sie gewaltsame Streiche macht, da mir solche
               von jeher zuwider waren. Aber bitte, seien Sie sehr schonend, weil sie mir mit
               Selbstmord droht, was auch eine hübsche Gewohnheit von ihr ist.\pend
           
\pstart
           Am 14. fahre ich auf 8 Tage nach Berlin\oindex{Berlin@\textbf{Berlin}, \emph{Hauptstadt}|pw}. Im April voraussichtlich
               nach Bosnien\oindex{Bosnien und Herzegowina@\textbf{Bosnien und Herzegowina}, \emph{Land}|pw} und Dalmatien\oindex{Dalmatien@\textbf{Dalmatien}, \emph{Ehemalige Region}|pw}. Im Mai nach London\oindex{London@\textbf{London}, \emph{Hauptstadt}|pw} auf 14 Tage.\pend
           
\pstart
           {\pb}Ich lese jetzt die »\label{K_L03339-4v}\edtext{Gespräche des göttlichen Aretino\pwindex{\textcolor{red}{\textsuperscript{XXXX indx1}}!Gespräche des göttlichen Pietro Aretino@\strich\emph{Die Gespräche des göttlichen Pietro Aretino}|pw}}{\lemma{\textnormal{\emph{Gespräche … Aretino}}}\Cendnote{\textnormal{Salten\pwindex{Salten, Felix 6.\,9.\,1869 Budapest – 8.\,10.\,1945 Zürich@\textsc{Salten, Felix} (6.\,9.\,1869 Budapest – 8.\,10.\,1945 Zürich), \emph{Schriftsteller, Journalist, Chefredakteur}|pwk} schrieb auch ein Feuilleton\pwindex{Salten, Felix 6.\,9.\,1869 Budapest – 8.\,10.\,1945 Zürich@\textsc{Salten, Felix} (6.\,9.\,1869 Budapest – 8.\,10.\,1945 Zürich), \emph{Schriftsteller, Journalist, Chefredakteur}!Vom göttlichen Aretino@\strich\emph{Vom göttlichen Aretino}|pwkv} darüber: Felix Salten\pwindex{Salten, Felix 6.\,9.\,1869 Budapest – 8.\,10.\,1945 Zürich@\textsc{Salten, Felix} (6.\,9.\,1869 Budapest – 8.\,10.\,1945 Zürich), \emph{Schriftsteller, Journalist, Chefredakteur}|pwk}: \emph{Vom göttlichen Aretino}\pwindex{Salten, Felix 6.\,9.\,1869 Budapest – 8.\,10.\,1945 Zürich@\textsc{Salten, Felix} (6.\,9.\,1869 Budapest – 8.\,10.\,1945 Zürich), \emph{Schriftsteller, Journalist, Chefredakteur}!Vom göttlichen Aretino@\strich\emph{Vom göttlichen Aretino}|pwk}. In: \emph{Die Zeit}\pwindex{Zeit@\emph{Die Zeit}|pwk}, Jg. 2, Nr. 165, 15. 3. 1903, Morgenblatt, S. 1–2.}}}\label{K_L03339-4},« und finde darin zu
               meinem Erstaunen die römische Buhlerin, die Bekenntnisse ablegt. Sie wissen, dass ich
               ein solches Buch schreiben wollte. Arbeiten kann ich nur wenig, da mir die Zeit\orgindex{Zeit. Wiener Wochenschrift@Die Zeit. Wiener Wochenschrift|pw} fast
               alles weg nimmt. Nun soll Aram\pwindex{Aram, Kurt 28.\,1.\,1869 Lennep – 10.\,7.\,1934 Berlin@\textsc{Aram, Kurt} (28.\,1.\,1869 Lennep – 10.\,7.\,1934 Berlin), \emph{Schriftsteller, Journalist}|pw} fort, und ich
                  für 8400fl\textcolor{gray}{.} jährlich auch das Feuilleton übernehmen; außerdem
               heißt es, – mit mir wurde \textcolor{gray}{noch} nicht davon gesprochen – dass ich
               Chef-Stellvertreter werden soll. Ich wünschte mir, dass der Tag dann – 36 Stunden haben möge, eine
               Erhöhung, mit der ich noch mehr einverstanden wäre. Für London\oindex{London@\textbf{London}, \emph{Hauptstadt}|pw} habe ich mir jetzt eine 
               Engländerin\pwindex{?? [Englischlehrerin von Felix Salten] @\textsc{?? [Englischlehrerin von Felix Salten]}|pwv}
                angeschafft,
               die 3mal die Woche kommt. Ich \label{K_L03339-5v}\edtext{beginne
               den {[}»{]}Hund {\pb}von Florenz\pwindex{Salten, Felix 6.\,9.\,1869 Budapest – 8.\,10.\,1945 Zürich@\textsc{Salten, Felix} (6.\,9.\,1869 Budapest – 8.\,10.\,1945 Zürich), \emph{Schriftsteller, Journalist, Chefredakteur}!Hund von Florenz@\strich\emph{Der Hund von Florenz}|pw}«}{\lemma{\textnormal{\emph{beginne … Florenz«}}}\Cendnote{\textnormal{Salten\pwindex{Salten, Felix 6.\,9.\,1869 Budapest – 8.\,10.\,1945 Zürich@\textsc{Salten, Felix} (6.\,9.\,1869 Budapest – 8.\,10.\,1945 Zürich), \emph{Schriftsteller, Journalist, Chefredakteur}|pwk} arbeitete noch Jahre an der Novelle\pwindex{Salten, Felix 6.\,9.\,1869 Budapest – 8.\,10.\,1945 Zürich@\textsc{Salten, Felix} (6.\,9.\,1869 Budapest – 8.\,10.\,1945 Zürich), \emph{Schriftsteller, Journalist, Chefredakteur}!Hund von Florenz@\strich\emph{Der Hund von Florenz}|pwkv}, die erst 1923 erschien. Siehe XXXX Auszeichnungsfehler: Dokument L03510 nicht gefunden und XXXX Auszeichnungsfehler: Dokument L03569 nicht gefunden.}}}\label{K_L03339-5} den ich vielleicht dann in Bosnien\oindex{Bosnien und Herzegowina@\textbf{Bosnien und Herzegowina}, \emph{Land}|pw} fertig mache.\pend
           
\pstart
           Schreiben Sie mir bitte recht bald. Bin neugierig, wie sich Herr \label{K_L03339-6v}\edtext{Jacobsohn\pwindex{Jacobsohn, Siegfried 28.\,1.\,1881 Berlin – 3.\,12.\,1926 ebd.@\textsc{Jacobsohn, Siegfried} (28.\,1.\,1881 Berlin – 3.\,12.\,1926 ebd.), \emph{Journalist, Kritiker, Publizist}|pw} benehmen}{\lemma{\textnormal{\emph{Jacobsohn benehmen}}}\Cendnote{\textnormal{Jacobsohn\pwindex{Jacobsohn, Siegfried 28.\,1.\,1881 Berlin – 3.\,12.\,1926 ebd.@\textsc{Jacobsohn, Siegfried} (28.\,1.\,1881 Berlin – 3.\,12.\,1926 ebd.), \emph{Journalist, Kritiker, Publizist}|pwk} war der Berlin\oindex{Berlin@\textbf{Berlin}, \emph{Hauptstadt}|pwk}er Theaterkorrespondent der
                  \emph{Zeit}\orgindex{Zeit. Wiener Wochenschrift@Die Zeit. Wiener Wochenschrift|pwk}. Salten\pwindex{Salten, Felix 6.\,9.\,1869 Budapest – 8.\,10.\,1945 Zürich@\textsc{Salten, Felix} (6.\,9.\,1869 Budapest – 8.\,10.\,1945 Zürich), \emph{Schriftsteller, Journalist, Chefredakteur}|pwk} dürfte hier seiner Neugier Ausdruck
                  verliehen haben, wie Jacobsohn\pwindex{Jacobsohn, Siegfried 28.\,1.\,1881 Berlin – 3.\,12.\,1926 ebd.@\textsc{Jacobsohn, Siegfried} (28.\,1.\,1881 Berlin – 3.\,12.\,1926 ebd.), \emph{Journalist, Kritiker, Publizist}|pwk} die Premiere von \emph{Der Schleier der Beatrice}\pwindex{Schnitzler, Arthur 15.\,5.\,1862 Wien – 21.\,10.\,1931 ebd.@\textsc{Schnitzler, Arthur} (15.\,5.\,1862 Wien – 21.\,10.\,1931 ebd.), \emph{Schriftsteller, Mediziner}!Schleier der Beatrice. Schauspiel in fünf Akten@\strich\emph{Der Schleier der Beatrice. Schauspiel in fünf Akten}|pwk} besprechen würde. Die Depesche lautetete:
                  »Der Dichter wurde häufig gerufen, und{ }ſtarker Beifall behauptete{ }ſich{ }ſiegreich gegen einzelne energiſche Ziſcher«. ([Siegfried Jacobsohn\pwindex{Jacobsohn, Siegfried 28.\,1.\,1881 Berlin – 3.\,12.\,1926 ebd.@\textsc{Jacobsohn, Siegfried} (28.\,1.\,1881 Berlin – 3.\,12.\,1926 ebd.), \emph{Journalist, Kritiker, Publizist}|pwk}]: \emph{»Schleier der Beatrice.« Man telegraphirt uns aus Berlin, 7. d.}\pwindex{Schleier der Beatrice.« Man telegraphirt uns aus Berlin, 7. d.@\emph{»Schleier der Beatrice.« Man telegraphirt uns aus Berlin, 7. d.}|pwk} 
                     In: \emph{Die Zeit}\pwindex{Zeit@\emph{Die Zeit}|pwk}, Jg. 2, Nr. 158, S. 5.)
               }}}\label{K_L03339-6} wird.\pend
           
\pstart
           herzlichst Ihr {\\[\baselineskip]}\spacefill\mbox{Salten}\pend
           \leftskip=0em{}\selectlanguage{ngerman}\endnumbering\briefempfaengerindex{Schnitzler, Arthur@\textsc{Schnitzler, Arthur}!zzzSalten, Felix@\emph{von Felix Salten}!1903-03-031@{3. 3. 1903}|)be}\mylabel{L03339h}  \newcommand{\dateiname}{L03339}\newcommand{\titel}{Felix Salten an Arthur Schnitzler, 3. 3. 1903}\newcommand{\editorInnen}{Martin Anton Müller und Laura Untner}%% latex-leseansicht-abspann.tex
%% Abspann für die Leseansicht.
%% Der Schalter \ifkorrekturansicht ist bereits durch den Vorspann gesetzt.

%% latex-abspann.tex
%% Gemeinsamer Abspann für Korrekturansicht und Leseansicht.
%% Setzt den Schalter \ifkorrekturansicht voraus (gesetzt in den
%% einbindenden Dateien latex-korrekturansicht-abspann.tex bzw.
%% latex-leseansicht-abspann.tex).
%% ---------------------------------------------------------------

\normalsize

% Das esempio-Environment wird nur in der Leseansicht benötigt
\ifkorrekturansicht\else
\newenvironment{esempio}[3]%
{
    \vspace{1.5ex}
    \rlap{\underline{#1}}
    \par
    \setlength{\parindent}{0cm}
    \nopagebreak
    \leftskip=#2cm
    \rightskip=#3cm
}
{
    \par
}
\fi

\doendnotes{C}
\bigskip
\vfill

\clearpage

\footnotesize

\ifkorrekturansicht
  \lohead{\textsc{register}}
\fi

% theindex-Environment neu definieren ohne reledmac
\makeatletter
\renewenvironment{theindex}{%
  \ifkorrekturansicht
    \section*{\indexname}%
  \else
    \subsubsection*{Index der erwähnten Entitäten}%
  \fi
  \setlength{\parindent}{0pt}%
  \setlength{\parskip}{0pt plus 0.3pt}%
  \let\item\@idxitem
}{%
  \ifkorrekturansicht\clearpage\fi
}
\makeatother

\IfFileExists{\jobname-pw.ind}{\input{\jobname-pw.ind}}{}

% Quellenangabe nur in der Leseansicht
\ifkorrekturansicht\else
% Fallback-Definitionen, falls die .tex-Datei \titel etc. nicht gesetzt hat
\providecommand{\titel}{}
\providecommand{\editorInnen}{}
\providecommand{\dateiname}{\jobname}

\vspace{3cm}

\vfill

\footnotesize
\textsc{Quelle}: \titel. Herausgegeben von {\editorInnen}. In: \emph{Arthur Schnitzler: Briefwechsel mit Autorinnen und Autoren}.
 Digitale Edition, https://schnitzler-briefe.acdh.oeaw.ac.at/{\dateiname}.html (Stand \today)
\fi

\end{document}


