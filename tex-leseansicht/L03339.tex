%% latex-leseansicht-vorspann.tex
%% Vorspann für die Leseansicht.
%% Lädt die gemeinsame Datei latex-vorspann.tex mit nicht gesetztem Schalter.

\newif\ifkorrekturansicht
\korrekturansichtfalse

\input{../tex-inputs/latex-vorspann}

\begin{center}
            \textcolor{red}{ENTWURF, NICHT FERTIG KORRIGIERT}
                      \end{center}
            
         
         \renewcommand{\erwaehntePersonen}{Personen:  ?? [Englischlehrerin von Felix Salten], Kurt Aram, Mirjam Horwitz,  Horwitz, Siegfried Jacobsohn, George Bernard Shaw, Carl Wiene}
         \renewcommand{\erwaehnteInstitutionen}{Institutionen: Deutsches Theater Berlin, Raimund-Theater}
         \renewcommand{\erwaehnteOrte}{Orte: Berlin, Bosnien und Herzegowina, Dalmatien, London, Wien}
         \renewcommand{\erwaehnteWerke}{Werke: Der Hund von Florenz, Der Schleier der Beatrice. Schauspiel in fünf Akten, Die Gespräche des göttlichen Pietro Aretino, Ein Teufelskerl. Schauspiel in drei Akten, The Devil’s Disciple}
               \section[Felix Salten an Arthur Schnitzler, 3. 3. 1903]{ Felix Salten an Arthur Schnitzler, 3. 3. 1903}\nopagebreak\mylabel{v}\rehead{ }\begin{ledgroupsized}[t]{13cm}\normalsize\beginnumbering \toendnotes[C]{\smallbreak\pagebreak[2]} \Standort{CUL, Schnitzler, B 89, A 2.}
\physDesc{Brief, 1 Blatt, 4 Seiten
\newline{}Handschrift: Bleistift, lateinische Kurrent
\newline{}Schnitzler: mit Bleistift Vermerk: »\textsc{Salten}« \newline{}Ordnung: mit Bleistift von unbekannter Hand nummeriert:
                                    »164« }\toendnotes[C]{\smallbreak}\pstart
           \raggedleft{}{\pb}Wien\oindex{Wien@\textbf{Wien}|pw}, 3. III. 03\pend
           \pstart
           Lieber, zur \label{K_L03339-41v}\edtext{Premiere\pwindex{Schnitzler, Arthur 15.05.1862 – 21.10.1931@\textsc{Schnitzler, Arthur} (15.05.1862 – 21.10.1931), \emph{Schriftsteller, Mediziner}!Schleier der Beatrice. Schauspiel in fuenf Akten1900-12-01@\strich\emph{Der Schleier der Beatrice. Schauspiel in fünf Akten} {[}1900-12-01{]}|pwv}}{\lemma{\textnormal{\emph{Premiere}}}\Cendnote{\textnormal{Schnitzler\pwindex{Schnitzler, Arthur 15.05.1862 – 21.10.1931@\textsc{Schnitzler, Arthur} (15.05.1862 – 21.10.1931), \emph{Schriftsteller, Mediziner}|pwk} weilte zur
                  Vorbereitung der Premiere von \emph{Der Schleier der
                     Beatrice}\pwindex{Schnitzler, Arthur 15.05.1862 – 21.10.1931@\textsc{Schnitzler, Arthur} (15.05.1862 – 21.10.1931), \emph{Schriftsteller, Mediziner}!Schleier der Beatrice. Schauspiel in fuenf Akten1900-12-01@\strich\emph{Der Schleier der Beatrice. Schauspiel in fünf Akten} {[}1900-12-01{]}|pwk} in Berlin\oindex{Berlin@\textbf{Berlin}|pwk}. Diese fand am 7. 3. 1903 am \emph{Deutschen Theater}\orgindex{Deutsches Theater Berlin@Deutsches Theater Berlin|pwk} in seiner Anwesenheit
                  statt.}}}\label{K_L03339-41h} kann ich nun leider doch nicht nach Berlin\oindex{Berlin@\textbf{Berlin}|pw}; schade. Ich werde erst so gegen 14.\textsuperscript{ten} März reisen, und habe vorher noch enorm viel zu thun. Was
               sagen Sie zum \label{K_L03339-3v}\edtext{Teufelskerl\pwindex{Shaw, George Bernard 26.07.1856 – 02.11.1950@\textsc{Shaw, George Bernard} (26.07.1856 – 02.11.1950), \emph{Schriftsteller}!Teufelskerl. Schauspiel in drei Akten1903@\strich\emph{Ein Teufelskerl. Schauspiel in drei Akten} {[}1903{]}|pw}? Das Stück hat Herr Wiene\pwindex{Wiene, Carl 1852-05-08 – 1913-02-13@\textsc{Wiene, Carl} (1852-05-08 – 1913-02-13), \emph{Schauspieler}|pw}}{\lemma{\textnormal{\emph{Teufelskerl? … Wiene}}}\Cendnote{\textnormal{Carl Wiene\pwindex{Wiene, Carl 1852-05-08 – 1913-02-13@\textsc{Wiene, Carl} (1852-05-08 – 1913-02-13), \emph{Schauspieler}|pwk} übernahm als Gastschauspieler am
                     25. 2. 1903 die Hauptrolle von \emph{Ein Teufelskerl}\pwindex{Shaw, George Bernard 26.07.1856 – 02.11.1950@\textsc{Shaw, George Bernard} (26.07.1856 – 02.11.1950), \emph{Schriftsteller}!Teufelskerl. Schauspiel in drei Akten1903@\strich\emph{Ein Teufelskerl. Schauspiel in drei Akten} {[}1903{]}|pwk} (\emph{The
                     Devil’s Disciple}\pwindex{Shaw, George Bernard 26.07.1856 – 02.11.1950@\textsc{Shaw, George Bernard} (26.07.1856 – 02.11.1950), \emph{Schriftsteller}!Devil s Disciple1897@\strich\emph{The Devil’s Disciple} {[}1897{]}|pwk}) von George Bernard
                     Shaw\pwindex{Shaw, George Bernard 26.07.1856 – 02.11.1950@\textsc{Shaw, George Bernard} (26.07.1856 – 02.11.1950), \emph{Schriftsteller}|pwk}, das am \emph{Raimund-Theater}\orgindex{Raimund-Theater@Raimund-Theater|pwk} gegeben
                  wurde. Schnitzler\pwindex{Schnitzler, Arthur 15.05.1862 – 21.10.1931@\textsc{Schnitzler, Arthur} (15.05.1862 – 21.10.1931), \emph{Schriftsteller, Mediziner}|pwk} war bereits in Berlin\oindex{Berlin@\textbf{Berlin}|pwk} und sah die Inszenierung nicht.}}}\label{K_L03339-3h}
               ruinirt, wie vorauszusehen war. Sehr fühlbar wurde mir die tiefe Unmoral, die darin
               steckt, wenn das Alter sich als Jugend verkleidet und geberdet. Der Widerwille, den
               man bei solchem Schauspiel empfindet, geht bis an ein sexuelles Missbehagen,
               wenigstens begreift man die Nervenzerrüttung einer Frau, an der ein impotenter Mann
               heuchlerische Versuche vornimmt, denn mit ähnlicher Bereitwilligkeit zur Empfängnis
               sitzt so ein Publikum im Theater. Mir wäre es sehr lieb, wenn Sie mir statt einer
               Ansichtskarte einmal näheres über die Proben ec. Berlin\oindex{Berlin@\textbf{Berlin}|pw} ec. schreiben, falls es Ihre Zeit gestattet. \pend
           \pstart
           In Angelegenheit der Mirjam H.\pwindex{Horwitz, Mirjam 1882-06-15 – 1967-09-26@\textsc{Horwitz, Mirjam} (1882-06-15 – 1967-09-26), \emph{Theaterleiterin, Schauspielerin}|pw} muß ich Sie
               nochmals bemühen: bald und möglichst schonend. Sie schreibt mir heute einen confusen
               Brief, ob sie »nach hier« kommen soll, oder wann ich »nach dort« komme. Ferner, dass
               ich nicht durch mein Wort an ihren Vater\pwindex{Horwitz @\textsc{Horwitz}|pwv} gebunden bin, falls \uline{sie} mit \uline{mir} verkehrt, endlich, dass ich an
               einen Vertrauten von ihr schreiben solle, dass sei auch nicht gegen mein Versprechen,
               ec, dann noch recht enervirende Dinge von »sich angehören vor aller {\pb}Welt –« »den Leuten zum
               Trotz« ec. und in diesem Stil, der die Liebe recht unangenehm macht. \pend
           \pstart
           Das Wesentlich an der Sache: dass ich ihrem Vater\pwindex{Horwitz @\textsc{Horwitz}|pwv} wahrscheinlich kein Versprechen gegeben hätte, wenn
               ich Mirjam\pwindex{Horwitz, Mirjam 1882-06-15 – 1967-09-26@\textsc{Horwitz, Mirjam} (1882-06-15 – 1967-09-26), \emph{Theaterleiterin, Schauspielerin}|pw} sehr lieb hätte. Ferner: dass ich
               aber, nun ich das Versprechen gab, keine Lust habe Geschichten zu machen. Bringen Sie
               ihr das bitte schonend bei. Dies mit dem Versprechen nämlich, und vor allem, dass sie
               nichts gewinnt, wenn sie gewaltsame Streiche macht, da mir solche von jeher zuwider
               waren. Aber bitte, seien Sie sehr schonend, weil sie mir mit Selbstmord droht, was
               auch eine hübsche Gewohnheit von ihr ist. \pend
           \pstart
           Am 14. fahre ich auf 8 Tage nach Berlin\oindex{Berlin@\textbf{Berlin}|pw}. Im April voraussichtlich nach Bosnien\oindex{Bosnien und Herzegowina@\textbf{Bosnien und Herzegowina}|pw} und Dalmatien\oindex{Dalmatien@\textbf{Dalmatien}|pw}. Im Mai nach
               London\oindex{London@\textbf{London}|pw} auf 14 Tage. \pend
           \pstart
           {\pb}Ich lese jetzt die »Gespräche des göttlichen Aretino\pwindex{\textcolor{red}{\textsuperscript{XXXX1 indx}}!Gespraeche des goettlichen Pietro Aretino1903@\strich\emph{Die Gespräche des göttlichen Pietro Aretino} {[}1903{]}|pw},« und finde
               darin zu meinem Erstaunen die römische Buhlerin, die Bekenntnisse ablegt. Sie wissen,
               dass ich ein solches Buch schreiben wollte. Arbeiten kann ich nur wenig, da mir die
               Zeit fast alles weg nimmt. \pend
           \pstart
           Nun soll Aram\pwindex{Aram, Kurt 1869-01-28 – 1934-07-10@\textsc{Aram, Kurt} (1869-01-28 – 1934-07-10), \emph{Schriftsteller, Journalist}|pw} fort, und ich für 8400fl.
               jährlich auch das Feuilleton übernehmen, außerdem heißt es, – mit mir wurde noch
               nicht davon gesprochen – dass ich Chef-Stellvertreter werden soll. Ich wünschte mir,
               dass der Tag dann 36 Stunden haben möge, eine Erhöhung, mit der ich noch mehr
               einverstanden wäre. Für London\oindex{London@\textbf{London}|pw} habe ich mir jetzt eine
               Engländerin\pwindex{?? [Englischlehrerin von Felix Salten] @\textsc{?? [Englischlehrerin von Felix Salten]}|pwv} angeschafft, die 3mal
               die Woche kommt. Ich beginne den »Hund {\pb}von Florenz\pwindex{Salten, Felix 06.09.1869 – 08.10.1945@\textsc{Salten, Felix} (06.09.1869 – 08.10.1945), \emph{Schriftsteller, Journalist}!Hund von Florenz1923@\strich\emph{Der Hund von Florenz} {[}1923{]}|pw}«, den ich
               vielleicht dann in Bosnien\oindex{Bosnien und Herzegowina@\textbf{Bosnien und Herzegowina}|pw} fertig mache. \pend
           \pstart
           Schreiben Sie mir bitte recht bald. Bin neugierig, wie sich Herr Jacobson\pwindex{Jacobsohn, Siegfried 28.01.1881 – 03.12.1926@\textsc{Jacobsohn, Siegfried} (28.01.1881 – 03.12.1926), \emph{Journalist, Kritiker, Publizist}|pw} benehmen wird. \pend
           \pstart
           Herzlichst Ihr {\\[\baselineskip]}\spacefill\mbox{Salten }\pend
           \leftskip=0em{}
         
         \endnumbering\mylabel{h}\end{ledgroupsized}\begin{anhang}\end{anhang}\newcommand{\dateiname}{L03339}\newcommand{\titel}{Felix Salten an Arthur Schnitzler, 3. 3. 1903}\newcommand{\editorInnen}{Martin Anton Müller und Laura Untner}%% latex-leseansicht-abspann.tex
%% Abspann für die Leseansicht.
%% Der Schalter \ifkorrekturansicht ist bereits durch den Vorspann gesetzt.

%% latex-abspann.tex
%% Gemeinsamer Abspann für Korrekturansicht und Leseansicht.
%% Setzt den Schalter \ifkorrekturansicht voraus (gesetzt in den
%% einbindenden Dateien latex-korrekturansicht-abspann.tex bzw.
%% latex-leseansicht-abspann.tex).
%% ---------------------------------------------------------------

\normalsize

% Das esempio-Environment wird nur in der Leseansicht benötigt
\ifkorrekturansicht\else
\newenvironment{esempio}[3]%
{
    \vspace{1.5ex}
    \rlap{\underline{#1}}
    \par
    \setlength{\parindent}{0cm}
    \nopagebreak
    \leftskip=#2cm
    \rightskip=#3cm
}
{
    \par
}
\fi

\doendnotes{C}
\bigskip
\vfill

\clearpage

\footnotesize

\ifkorrekturansicht
  \lohead{\textsc{register}}
\fi

% theindex-Environment neu definieren ohne reledmac
\makeatletter
\renewenvironment{theindex}{%
  \ifkorrekturansicht
    \section*{\indexname}%
  \else
    \subsubsection*{Index der erwähnten Entitäten}%
  \fi
  \setlength{\parindent}{0pt}%
  \setlength{\parskip}{0pt plus 0.3pt}%
  \let\item\@idxitem
}{%
  \ifkorrekturansicht\clearpage\fi
}
\makeatother

\IfFileExists{\jobname-pw.ind}{\input{\jobname-pw.ind}}{}

% Quellenangabe nur in der Leseansicht
\ifkorrekturansicht\else
% Fallback-Definitionen, falls die .tex-Datei \titel etc. nicht gesetzt hat
\providecommand{\titel}{}
\providecommand{\editorInnen}{}
\providecommand{\dateiname}{\jobname}

\vspace{3cm}

\vfill

\footnotesize
\textsc{Quelle}: \titel. Herausgegeben von {\editorInnen}. In: \emph{Arthur Schnitzler: Briefwechsel mit Autorinnen und Autoren}.
 Digitale Edition, https://schnitzler-briefe.acdh.oeaw.ac.at/{\dateiname}.html (Stand \today)
\fi

\end{document}


      