%% latex-leseansicht-vorspann.tex
%% Vorspann für die Leseansicht.
%% Lädt die gemeinsame Datei latex-vorspann.tex mit nicht gesetztem Schalter.

\newif\ifkorrekturansicht
\korrekturansichtfalse

\input{../tex-inputs/latex-vorspann}


\section[Arthur Schnitzler an Stefan Zweig, 5. 7. 1908]{L03800 Arthur Schnitzler an Stefan Zweig, 5. 7. 1908}
\nopagebreak\mylabel{L03800v}
\rehead{ }\normalsize\beginnumbering\briefempfaengerindex{Zweig, Stefan@\textsc{Zweig, Stefan}!zzzSchnitzler, Arthur@\emph{von Arthur Schnitzler}!1908-07-051@{5. 7. 1908}|(be}
\toendnotes[C]{\smallbreak\pagebreak[2]}
\correspDesc{Versand  durch Arthur Schnitzler am 5. 7. 1908 in Seis am Schlern
\newline{}Erhalt  durch Stefan Zweig im Zeitraum [6. 7. 1908 – 10. 7. 1908?] in Wien}\toendnotes[C]{\smallbreak}
\Standort{Jerusalem, National Library of Israel, ARC. Ms. Var. 305 1 58 Stefan Zweig Collection.}
\physDesc{Bildpostkarte, 217 Zeichen
\newline{}Handschrift: schwarze Tinte, deutsche Kurrent
\newline{}Versand: Stempel: »\nobreak{}5. 7. {[}1908{]}\nobreak{}«.  }\toendnotes[C]{\smallbreak}\pstart{}{\pb}\textsc{Dr. Stefan Zweig}\pend{}\pstart{}Wien VIII\oindex{VIII., Josefstadt@\textbf{VIII., Josefstadt}, \emph{Verwaltungsgebiet}|pw}\pend{}\pstart{}\textsc{Kochgasse 8}\oindex{Wien@\textbf{Wien}!VIII., Josefstadt@\textbf{VIII., Josefstadt}!Kochgasse 8@\textbf{Kochgasse 8}, \emph{Wohngebäude}|pw}\pend{}{\bigskip}
\pstart
           {\pb}\textcolor{gray}{\textbf{Santnerspitze\oindex{Punta Santner@\textbf{Punta Santner}, \emph{Bergspitze}|pw}}}\hspace*{2.5em}\textcolor{gray}{\textbf{Burgstall\oindex{Burgstall@\textbf{Burgstall}, \emph{Berg}|pw}}}\hfill \textcolor{gray}{\textbf{Jung-Schlern\oindex{Piccolo Sciliar@\textbf{Piccolo Sciliar}, \emph{Berg}|pw}}}\pend
           
\pstart
           \centering{}\textcolor{gray}{\textbf{Tirol\oindex{Südtirol@\textbf{Südtirol}, \emph{Verwaltungsgebiet}|pw}: Seis am Schlern\oindex{Seis am Schlern@\textbf{Seis am Schlern}|pw}, 1000m. Nach dem Aquarell\pwindex{Reisch, Franz August Carl Maria 1.\,5.\,1862 Wien – 1942? Meran@\textsc{Reisch, Franz August Carl Maria} (1.\,5.\,1862 Wien – 1942? Meran), \emph{Maler}!Tirol: Seis am Schlern@\strich\emph{Tirol: Seis am Schlern}|pwv} von F. A. C. M. Reisch\pwindex{Reisch, Franz August Carl Maria 1.\,5.\,1862 Wien – 1942? Meran@\textsc{Reisch, Franz August Carl Maria} (1.\,5.\,1862 Wien – 1942? Meran), \emph{Maler}|pw},
                     Meran\oindex{Meran@\textbf{Meran}, \emph{Hauptstadt}|pw}.}}\pend
           \vspace{1em}
\pstart
           {\pb}5. 7. 08\pend
           \vspace{0.5em}
\pstart
           herzlichen Dank, lieber Herr Doktor, für das \label{K_L03800-1v}\edtext{\textsc{Balzac\pwindex{Balzac, Honoré de 20.\,5.\,1799 Tours – 18.\,8.\,1850 Paris@\textsc{Balzac, Honoré de} (20.\,5.\,1799 Tours – 18.\,8.\,1850 Paris), \emph{Schriftsteller}|pw}}-Büchlein\pwindex{Zweig, Stefan 28.\,11.\,1881 Wien – 23.\,2.\,1942 Petrópolis@\textsc{Zweig, Stefan} (28.\,11.\,1881 Wien – 23.\,2.\,1942 Petrópolis), \emph{Schriftsteller}!Balzac. Sein Weltbild aus den Werken@\strich\emph{Balzac. Sein Weltbild aus den Werken}|pwv}}{\lemma{\textnormal{\emph{Balzac-Büchlein}}}\Cendnote{\textnormal{Stefan Zweig\pwindex{Zweig, Stefan 28.\,11.\,1881 Wien – 23.\,2.\,1942 Petrópolis@\textsc{Zweig, Stefan} (28.\,11.\,1881 Wien – 23.\,2.\,1942 Petrópolis), \emph{Schriftsteller}|pwk}:
                        \emph{Balzac. Sein Weltbild aus den Werken}\pwindex{Zweig, Stefan 28.\,11.\,1881 Wien – 23.\,2.\,1942 Petrópolis@\textsc{Zweig, Stefan} (28.\,11.\,1881 Wien – 23.\,2.\,1942 Petrópolis), \emph{Schriftsteller}!Balzac. Sein Weltbild aus den Werken@\strich\emph{Balzac. Sein Weltbild aus den Werken}|pwk}.
                     Stuttgart: \emph{Verlag von
                        Robert Lutz}\orgindex{Robert Lutz@Robert Lutz|pwk}{ }{[}1908{]} (\emph{Aus
                        der Gedankenwelt großer Geister. Eine Sammlung von Auswahlbänden}\pwindex{Aus der Gedankenwelt großer Geister. Eine Sammlung von Auswahlbänden@\emph{Aus der Gedankenwelt großer Geister. Eine Sammlung von Auswahlbänden}|pwk}.
                     Herausgegeben von Lothar
                        Brieger-Wasservogel\pwindex{Brieger, Lothar 6.\,9.\,1879 Zwickau – 23.\,3.\,1949 Berlin@\textsc{Brieger, Lothar} (6.\,9.\,1879 Zwickau – 23.\,3.\,1949 Berlin), \emph{Herausgeber, Kunstsammler}|pwk}. Band 11: Balzac\pwindex{Balzac, Honoré de 20.\,5.\,1799 Tours – 18.\,8.\,1850 Paris@\textsc{Balzac, Honoré de} (20.\,5.\,1799 Tours – 18.\,8.\,1850 Paris), \emph{Schriftsteller}|pwk}).}}}\label{K_L03800-1}; es begleitet mich in den Wald, wo ich geſtern
               Ihre{ }ſchöne Vorrede mit Vergnügen geleſen habe.\pend
           
\pstart
           Ihr{\\[\baselineskip]}\spacefill\mbox{Arthur Schnitzler}\pend
           \leftskip=0em{}\selectlanguage{ngerman}\endnumbering\briefempfaengerindex{Zweig, Stefan@\textsc{Zweig, Stefan}!zzzSchnitzler, Arthur@\emph{von Arthur Schnitzler}!1908-07-051@{5. 7. 1908}|)be}\mylabel{L03800h}  \newcommand{\dateiname}{L03800}\newcommand{\titel}{Arthur Schnitzler an Stefan Zweig, 5. 7. 1908}\newcommand{\editorInnen}{Selma Jahnke und Martin Anton Müller}%% latex-leseansicht-abspann.tex
%% Abspann für die Leseansicht.
%% Der Schalter \ifkorrekturansicht ist bereits durch den Vorspann gesetzt.

%% latex-abspann.tex
%% Gemeinsamer Abspann für Korrekturansicht und Leseansicht.
%% Setzt den Schalter \ifkorrekturansicht voraus (gesetzt in den
%% einbindenden Dateien latex-korrekturansicht-abspann.tex bzw.
%% latex-leseansicht-abspann.tex).
%% ---------------------------------------------------------------

\normalsize

% Das esempio-Environment wird nur in der Leseansicht benötigt
\ifkorrekturansicht\else
\newenvironment{esempio}[3]%
{
    \vspace{1.5ex}
    \rlap{\underline{#1}}
    \par
    \setlength{\parindent}{0cm}
    \nopagebreak
    \leftskip=#2cm
    \rightskip=#3cm
}
{
    \par
}
\fi

\doendnotes{C}
\bigskip
\vfill

\clearpage

\footnotesize

\ifkorrekturansicht
  \lohead{\textsc{register}}
\fi

% theindex-Environment neu definieren ohne reledmac
\makeatletter
\renewenvironment{theindex}{%
  \ifkorrekturansicht
    \section*{\indexname}%
  \else
    \subsubsection*{Index der erwähnten Entitäten}%
  \fi
  \setlength{\parindent}{0pt}%
  \setlength{\parskip}{0pt plus 0.3pt}%
  \let\item\@idxitem
}{%
  \ifkorrekturansicht\clearpage\fi
}
\makeatother

\IfFileExists{\jobname-pw.ind}{\input{\jobname-pw.ind}}{}

% Quellenangabe nur in der Leseansicht
\ifkorrekturansicht\else
% Fallback-Definitionen, falls die .tex-Datei \titel etc. nicht gesetzt hat
\providecommand{\titel}{}
\providecommand{\editorInnen}{}
\providecommand{\dateiname}{\jobname}

\vspace{3cm}

\vfill

\footnotesize
\textsc{Quelle}: \titel. Herausgegeben von {\editorInnen}. In: \emph{Arthur Schnitzler: Briefwechsel mit Autorinnen und Autoren}.
 Digitale Edition, https://schnitzler-briefe.acdh.oeaw.ac.at/{\dateiname}.html (Stand \today)
\fi

\end{document}


