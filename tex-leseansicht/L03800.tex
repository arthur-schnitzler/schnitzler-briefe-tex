%% latex-korrekturansicht-vorspann.tex
%% Vorspann für die Korrekturansicht.
%% Lädt die gemeinsame Datei latex-vorspann.tex mit gesetztem Schalter.

\newif\ifkorrekturansicht
\korrekturansichttrue

\input{../tex-inputs/latex-vorspann}


\section[Arthur Schnitzler an Stefan Zweig, 5. 7. 1908]{L03800 Arthur Schnitzler an Stefan Zweig, 5. 7. 1908}
\nopagebreak\mylabel{L03800v}
\rehead{ }\normalsize\beginnumbering\briefempfaengerindex{Zweig, Stefan@\textsc{Zweig, Stefan}!zzzSchnitzler, Arthur@\emph{von Arthur Schnitzler}!1908-07-051@{5. 7. 1908}|(be}
\toendnotes[C]{\smallbreak\pagebreak[2]}\Standort{Jerusalem, National Library of Israel, ARC. Ms. Var. 305 1 58 Stefan Zweig Collection.}
\physDesc{Bildpostkarte, 1 Blatt, 2 Seiten, 217 Zeichen
\newline{}Handschrift: schwarze Tinte, deutsche Kurrent
\newline{}Versand: Stempel: »\nobreak{}5. 7. {[}1908{]}\nobreak{}«.  }\toendnotes[C]{\smallbreak}\pstart{}{\pb}\textsc{Dr. Stefan Zweig}\pend{}\pstart{}Wien VIII\oindex{VIII., Josefstadt@\textbf{VIII., Josefstadt}, \emph{A.ADM3}|pw}\pend{}\pstart{}\textsc{Kochgasse 8}\oindex{Kochgasse 8@\textbf{Kochgasse 8}, \emph{Wohngebäude (K.WHS)}|pw}\pend{}{\bigskip}
\pstart
           {\pb}\textcolor{gray}{\textbf{Santnerspitze\oindex{Punta Santner@\textbf{Punta Santner}, \emph{T.PK}|pw}}}\hspace*{2.5em}\textcolor{gray}{\textbf{Burgstall\oindex{Burgstall@\textbf{Burgstall}, \emph{Berg (N.BRG)}|pw}}}\hfill \textcolor{gray}{\textbf{Jung-Schlern\oindex{Piccolo Sciliar@\textbf{Piccolo Sciliar}, \emph{Berg (N.BRG)}|pw}}}\pend
           
\pstart
           \centering{}\textcolor{gray}{\textbf{Tirol\oindex{Suedtirol@\textbf{Südtirol}, \emph{A.ADM2}|pw}: Seis am Schlern\oindex{Seis am Schlern@\textbf{Seis am Schlern}, \emph{P.PPL}|pw}, 1000m. Nach dem Aquarell\pwindex{Tirol: Seis am Schlern@\emph{Tirol: Seis am Schlern}|pwv} von F. A. C. M. Reisch\pwindex{Reisch, Franz August Carl Maria 1862-05-01 – 1942?@\textsc{Reisch, Franz August Carl Maria} (1862-05-01 – 1942?), \emph{Maler/Malerin}|pw},
                     Meran\oindex{Meran@\textbf{Meran}, \emph{P.PPLA3}|pw}.}}\pend
           \vspace{1em}
\pstart
           {\pb}5. 7. 08\pend
           \vspace{0.5em}
\pstart
           herzlichen Dank, lieber Herr Doktor, für das \label{K_L03800-1v}\edtext{\textsc{Balzac\pwindex{Balzac, Honore de 20.05.1799 – 18.08.1850@\textsc{Balzac, Honoré de} (20.05.1799 – 18.08.1850), \emph{Schriftsteller/Schriftstellerin}|pw}}-Büchlein\pwindex{Balzac. Sein Weltbild aus den Werken@\emph{Balzac. Sein Weltbild aus den Werken}|pwv}}{\lemma{\textnormal{\emph{Balzac-Büchlein}}}\Cendnote{\textnormal{Stefan Zweig\pwindex{Zweig, Stefan 28.11.1881 – 23.02.1942@\textsc{Zweig, Stefan} (28.11.1881 – 23.02.1942), \emph{Schriftsteller/Schriftstellerin}|pwk}:
                        \emph{Balzac. Sein Weltbild aus den Werken}\pwindex{Balzac. Sein Weltbild aus den Werken@\emph{Balzac. Sein Weltbild aus den Werken}|pwk}.
                     Stuttgart: \emph{Verlag von
                        Robert Lutz}\orgindex{Robert Lutz@Robert Lutz|pwk}{ }{[}1908{]} (\emph{Aus
                        der Gedankenwelt großer Geister. Eine Sammlung von Auswahlbänden}\pwindex{Aus der Gedankenwelt grosser Geister. Eine Sammlung von Auswahlbaenden@\emph{Aus der Gedankenwelt großer Geister. Eine Sammlung von Auswahlbänden}|pwk}.
                     Herausgegeben von Lothar
                        Brieger-Wasservogel\pwindex{Brieger, Lothar 1879-09-06 – 1949-03-23@\textsc{Brieger, Lothar} (1879-09-06 – 1949-03-23), \emph{Herausgeber/Herausgeberin, Kunstsammler/Kunstsammlerin}|pwk}. Band 11: Balzac\pwindex{Balzac, Honore de 20.05.1799 – 18.08.1850@\textsc{Balzac, Honoré de} (20.05.1799 – 18.08.1850), \emph{Schriftsteller/Schriftstellerin}|pwk}).}}}\label{K_L03800-1}; es begleitet mich in den Wald, wo ich geſtern
               Ihre ſchöne Vorrede mit Vergnügen geleſen habe.\pend
           
\pstart
           Ihr{\\[\baselineskip]}\spacefill\mbox{Arthur Schnitzler}\pend
           \leftskip=0em{}\selectlanguage{ngerman}\endnumbering\briefempfaengerindex{Zweig, Stefan@\textsc{Zweig, Stefan}!zzzSchnitzler, Arthur@\emph{von Arthur Schnitzler}!1908-07-051@{5. 7. 1908}|)be}\mylabel{L03800h}  \normalsize

\doendnotes{C}
\bigskip
\vfill

\clearpage

\footnotesize

\lohead{\textsc{register}}

% Definiere theindex-Environment komplett neu ohne reledmac
\makeatletter
\renewenvironment{theindex}{%
  \section*{\indexname}%
  \setlength{\parindent}{0pt}%
  \setlength{\parskip}{0pt plus 0.3pt}%
  \let\item\@idxitem
}{%
  \clearpage
}
\makeatother

\IfFileExists{\jobname-pw.ind}{\input{\jobname-pw.ind}}{}

\end{document}

      