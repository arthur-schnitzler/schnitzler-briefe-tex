%% latex-korrekturansicht-vorspann.tex
%% Vorspann für die Korrekturansicht.
%% Lädt die gemeinsame Datei latex-vorspann.tex mit gesetztem Schalter.

\newif\ifkorrekturansicht
\korrekturansichttrue

\input{../tex-inputs/latex-vorspann}


\section[ Paul Goldmann an Arthur Schnitzler, 2. 12. {[}1900{]}]{L02942 Paul Goldmann an Arthur Schnitzler, 2. 12. {[}1900{]}}
\nopagebreak\mylabel{L02942v}
\rehead{ }\normalsize\beginnumbering\briefempfaengerindex{Schnitzler, Arthur@\textsc{Schnitzler, Arthur}!zzzGoldmann, Paul@\emph{von Paul Goldmann}!1900-12-021@{2. 12. {[}1900{]}}|(be}
\toendnotes[C]{\smallbreak\pagebreak[2]}\Standort{DLA, A:Schnitzler, HS.NZ85.1.3170.}
\physDesc{Brief, 1 Blatt, 2 Seiten, 520 Zeichen
\newline{}Handschrift: blaue Tinte, deutsche Kurrent
\newline{}Schnitzler: 1) mit Bleistift das Jahr »900« vermerkt  2) mit rotem Buntstift eine seitliche Markierung}\toendnotes[C]{\smallbreak}
\pstart
           \raggedleft{}{\pb}Berlin\oindex{Berlin@\textbf{Berlin}, \emph{P.PPLC}|pw}, 2. December.\pend
           
\pstart\center{}Mein lieber Freund,\pend\vspace{0.5em}
\pstart
           Soweit aus den Referaten der Berlin\oindex{Berlin@\textbf{Berlin}, \emph{P.PPLC}|pw}er Blätter
               klug zu werden iſt, hat die Breslau\oindex{Breslau@\textbf{Breslau}, \emph{P.PPLA}|pw}er \textsc{Première\pwindex{Schleier der Beatrice. Schauspiel in fuenf Akten@\emph{Der Schleier der Beatrice. Schauspiel in fünf Akten}|pwv}} das Reſultat gehabt, daß durch die \label{K_L02942-1v}\edtext{ſchlechte Aufführung}{\lemma{\textnormal{\emph{ſchlechte Aufführung}}}\Cendnote{\textnormal{Die Uraufführung von \emph{Der
                     Schleier der Beatrice}\pwindex{Schleier der Beatrice. Schauspiel in fuenf Akten@\emph{Der Schleier der Beatrice. Schauspiel in fünf Akten}|pwk} wurde kritisch aufgenommen, siehe A. S.: \emph{Tagebuch}, 1. 12. 1900. }}}\label{K_L02942-1}
               hindurch der Werth des Stück\pwindex{Schleier der Beatrice. Schauspiel in fuenf Akten@\emph{Der Schleier der Beatrice. Schauspiel in fünf Akten}|pwv}es
                  \strikeout{klar geword} offenbar geworden iſt. Somit hat Breslau\oindex{Breslau@\textbf{Breslau}, \emph{P.PPLA}|pw} ſeine Schuldigkeit gethan\substVorne{}\textsuperscript{. U\textcolor{gray}{n}}\substDazwischen{},\substHinten{} und wir werden das Stück\pwindex{Schleier der Beatrice. Schauspiel in fuenf Akten@\emph{Der Schleier der Beatrice. Schauspiel in fünf Akten}|pwv} jetzt wohl \label{K_L02942-2v}\edtext{bald}{\lemma{\textnormal{\emph{bald}}}\Cendnote{\textnormal{Siehe Paul Goldmann an Arthur Schnitzler, 23. 12. [1899] und 21. 6. [1900].
               }}}\label{K_L02942-2} auf einer großen Berlin\oindex{Berlin@\textbf{Berlin}, \emph{P.PPLC}|pw}er oder Wien\oindex{Wien@\textbf{Wien}, \emph{A.ADM2}|pw}er Bühne ſehen. Ich habe geſtern{ }Abend viel an Dich {\pb}gedacht, und es
               that mir unendlich leid, daß ich nicht bei Dir ſein konnte.\pend
           
\pstart
           Viele treue Grüße! {\\[\baselineskip]}Dein {\\[\baselineskip]}\spacefill\mbox{Paul Goldmann.}\pend
           \leftskip=0em{}\selectlanguage{ngerman}\endnumbering\briefempfaengerindex{Schnitzler, Arthur@\textsc{Schnitzler, Arthur}!zzzGoldmann, Paul@\emph{von Paul Goldmann}!1900-12-021@{2. 12. {[}1900{]}}|)be}\mylabel{L02942h}  \normalsize

\doendnotes{C}
\bigskip
\vfill

\clearpage

\footnotesize

\lohead{\textsc{register}}

% Definiere theindex-Environment komplett neu ohne reledmac
\makeatletter
\renewenvironment{theindex}{%
  \section*{\indexname}%
  \setlength{\parindent}{0pt}%
  \setlength{\parskip}{0pt plus 0.3pt}%
  \let\item\@idxitem
}{%
  \clearpage
}
\makeatother

\IfFileExists{\jobname-pw.ind}{\input{\jobname-pw.ind}}{}

\end{document}

      