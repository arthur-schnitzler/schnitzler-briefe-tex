%% latex-leseansicht-vorspann.tex
%% Vorspann für die Leseansicht.
%% Lädt die gemeinsame Datei latex-vorspann.tex mit nicht gesetztem Schalter.

\newif\ifkorrekturansicht
\korrekturansichtfalse

\input{../tex-inputs/latex-vorspann}


\section[ Paul Goldmann an Arthur Schnitzler, 2. 12. [1900]]{L02942 Paul Goldmann an Arthur Schnitzler,  2. 12. [1900]}
\nopagebreak\mylabel{L02942v}
\rehead{ }\normalsize\beginnumbering\briefempfaengerindex{Schnitzler, Arthur@\textsc{Schnitzler, Arthur}!zzzGoldmann, Paul@\emph{von Paul Goldmann}!1900-12-021@{2. 12. [1900]}|(be}
\toendnotes[C]{\smallbreak\pagebreak[2]}
\correspDesc{Versand  durch Paul Goldmann am 2. 12. [1900] in Berlin
\newline{}Erhalt  durch Arthur Schnitzler im Zeitraum [3. 12. 1900
                  – 7. 12. 1900?] in Wien}\toendnotes[C]{\smallbreak}
\Standort{DLA, A:Schnitzler, HS.NZ85.1.3170.}
\physDesc{Brief, 1 Blatt, 2 Seiten, 520 Zeichen
\newline{}Handschrift: blaue Tinte, deutsche Kurrent
\newline{}Schnitzler: 1) mit Bleistift das Jahr »900« vermerkt  2) mit rotem Buntstift eine seitliche Markierung}\toendnotes[C]{\smallbreak}
\pstart
           \raggedleft{}{\pb}Berlin\oindex{Berlin@\textbf{Berlin}, \emph{Hauptstadt}|pw}, 2. December.\pend
           
\pstart\center{}Mein lieber Freund,\pend\vspace{0.5em}
\pstart
           Soweit aus den Referaten der Berlin\oindex{Berlin@\textbf{Berlin}, \emph{Hauptstadt}|pw}er Blätter
               klug zu werden iſt, hat die Breslau\oindex{Breslau@\textbf{Breslau}|pw}er \textsc{Première\eventindex{Lobe-Theater@\textbf{Lobe-Theater}!Uraufführung von Der Schleier der Beatrice, 1.12.1900@Uraufführung von Der Schleier der Beatrice, 1.12.1900|pwv}\pwindex{Schnitzler, Arthur 15.\,5.\,1862 Wien – 21.\,10.\,1931 ebd.@\textsc{Schnitzler, Arthur} (15.\,5.\,1862 Wien – 21.\,10.\,1931 ebd.), \emph{Schriftsteller, Mediziner}!Schleier der Beatrice. Schauspiel in fünf Akten@\strich\emph{Der Schleier der Beatrice. Schauspiel in fünf Akten}|pwv}} das Reſultat gehabt, daß durch die \label{K_L02942-1v}\edtext{ſchlechte Aufführung\eventindex{Lobe-Theater@\textbf{Lobe-Theater}!Uraufführung von Der Schleier der Beatrice, 1.12.1900@Uraufführung von Der Schleier der Beatrice, 1.12.1900|pwv}}{\lemma{\textnormal{\emph{schlechte Aufführung}}}\Cendnote{\textnormal{Die Uraufführung von \emph{Der
                        Schleier der Beatrice}\pwindex{Schnitzler, Arthur 15.\,5.\,1862 Wien – 21.\,10.\,1931 ebd.@\textsc{Schnitzler, Arthur} (15.\,5.\,1862 Wien – 21.\,10.\,1931 ebd.), \emph{Schriftsteller, Mediziner}!Schleier der Beatrice. Schauspiel in fünf Akten@\strich\emph{Der Schleier der Beatrice. Schauspiel in fünf Akten}|pwk}\eventindex{Lobe-Theater@\textbf{Lobe-Theater}!Uraufführung von Der Schleier der Beatrice, 1.12.1900@Uraufführung von Der Schleier der Beatrice, 1.12.1900|pwk} wurde kritisch aufgenommen, siehe A. S.: \emph{Tagebuch}, 1. 12. 1900. }}}\label{K_L02942-1}
               hindurch der Werth des Stück\pwindex{Schnitzler, Arthur 15.\,5.\,1862 Wien – 21.\,10.\,1931 ebd.@\textsc{Schnitzler, Arthur} (15.\,5.\,1862 Wien – 21.\,10.\,1931 ebd.), \emph{Schriftsteller, Mediziner}!Schleier der Beatrice. Schauspiel in fünf Akten@\strich\emph{Der Schleier der Beatrice. Schauspiel in fünf Akten}|pwv}es
                  \strikeout{klar geword} offenbar geworden iſt. Somit hat Breslau\oindex{Breslau@\textbf{Breslau}|pw}{ }ſeine Schuldigkeit gethan\substVorne{}\textsuperscript{. U\textcolor{gray}{n}}\substDazwischen{},\substHinten{} und wir werden das Stück\pwindex{Schnitzler, Arthur 15.\,5.\,1862 Wien – 21.\,10.\,1931 ebd.@\textsc{Schnitzler, Arthur} (15.\,5.\,1862 Wien – 21.\,10.\,1931 ebd.), \emph{Schriftsteller, Mediziner}!Schleier der Beatrice. Schauspiel in fünf Akten@\strich\emph{Der Schleier der Beatrice. Schauspiel in fünf Akten}|pwv} jetzt wohl \label{K_L02942-2v}\edtext{bald}{\lemma{\textnormal{\emph{bald}}}\Cendnote{\textnormal{Siehe XXXX Auszeichnungsfehler: Dokument L02900 nicht gefunden und XXXX Auszeichnungsfehler: Dokument L02921 nicht gefunden.
               }}}\label{K_L02942-2} auf einer großen Berlin\oindex{Berlin@\textbf{Berlin}, \emph{Hauptstadt}|pw}er oder Wien\oindex{Wien@\textbf{Wien}, \emph{Verwaltungsgebiet}|pw}er Bühne{ }ſehen. Ich habe geſtern{ }Abend viel an Dich {\pb}gedacht, und es
               that mir unendlich leid, daß ich nicht bei Dir{ }ſein konnte.\pend
           
\pstart
           Viele treue Grüße! {\\[\baselineskip]}Dein {\\[\baselineskip]}\spacefill\mbox{Paul Goldmann.}\pend
           \leftskip=0em{}\selectlanguage{ngerman}\endnumbering\briefempfaengerindex{Schnitzler, Arthur@\textsc{Schnitzler, Arthur}!zzzGoldmann, Paul@\emph{von Paul Goldmann}!1900-12-021@{2. 12. [1900]}|)be}\mylabel{L02942h}  \newcommand{\dateiname}{L02942}\newcommand{\titel}{Paul Goldmann an Arthur Schnitzler, 2. 12. [1900]}\newcommand{\editorInnen}{Martin Anton Müller und Laura Untner}%% latex-leseansicht-abspann.tex
%% Abspann für die Leseansicht.
%% Der Schalter \ifkorrekturansicht ist bereits durch den Vorspann gesetzt.

%% latex-abspann.tex
%% Gemeinsamer Abspann für Korrekturansicht und Leseansicht.
%% Setzt den Schalter \ifkorrekturansicht voraus (gesetzt in den
%% einbindenden Dateien latex-korrekturansicht-abspann.tex bzw.
%% latex-leseansicht-abspann.tex).
%% ---------------------------------------------------------------

\normalsize

% Das esempio-Environment wird nur in der Leseansicht benötigt
\ifkorrekturansicht\else
\newenvironment{esempio}[3]%
{
    \vspace{1.5ex}
    \rlap{\underline{#1}}
    \par
    \setlength{\parindent}{0cm}
    \nopagebreak
    \leftskip=#2cm
    \rightskip=#3cm
}
{
    \par
}
\fi

\doendnotes{C}
\bigskip
\vfill

\clearpage

\footnotesize

\ifkorrekturansicht
  \lohead{\textsc{register}}
\fi

% theindex-Environment neu definieren ohne reledmac
\makeatletter
\renewenvironment{theindex}{%
  \ifkorrekturansicht
    \section*{\indexname}%
  \else
    \subsubsection*{Index der erwähnten Entitäten}%
  \fi
  \setlength{\parindent}{0pt}%
  \setlength{\parskip}{0pt plus 0.3pt}%
  \let\item\@idxitem
}{%
  \ifkorrekturansicht\clearpage\fi
}
\makeatother

\IfFileExists{\jobname-pw.ind}{\input{\jobname-pw.ind}}{}

% Quellenangabe nur in der Leseansicht
\ifkorrekturansicht\else
% Fallback-Definitionen, falls die .tex-Datei \titel etc. nicht gesetzt hat
\providecommand{\titel}{}
\providecommand{\editorInnen}{}
\providecommand{\dateiname}{\jobname}

\vspace{3cm}

\vfill

\footnotesize
\textsc{Quelle}: \titel. Herausgegeben von {\editorInnen}. In: \emph{Arthur Schnitzler: Briefwechsel mit Autorinnen und Autoren}.
 Digitale Edition, https://schnitzler-briefe.acdh.oeaw.ac.at/{\dateiname}.html (Stand \today)
\fi

\end{document}


