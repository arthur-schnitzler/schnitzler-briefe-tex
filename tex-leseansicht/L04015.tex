%% latex-leseansicht-vorspann.tex
%% Vorspann für die Leseansicht.
%% Lädt die gemeinsame Datei latex-vorspann.tex mit nicht gesetztem Schalter.

\newif\ifkorrekturansicht
\korrekturansichtfalse

\input{../tex-inputs/latex-vorspann}


\section[Berta Zuckerkandl an Arthur Schnitzler, {{[}}16. oder 17. 5. 1912?{{]}}]{L04015 Berta Zuckerkandl an Arthur Schnitzler, {[}16. oder 17. 5. 1912?{]}}
\nopagebreak\mylabel{L04015v}
\rehead{ }\normalsize\beginnumbering\briefempfaengerindex{Schnitzler, Arthur@\textsc{Schnitzler, Arthur}!zzzZuckerkandl, Berta@\emph{von Berta Zuckerkandl}!1912-05-172@{{[}16. oder 17. 5. 1912?{]}}|(be}
\toendnotes[C]{\smallbreak\pagebreak[2]}
\correspDesc{Versand  durch Berta Zuckerkandl im Zeitraum [16.
                  oder 17. 5. 1912?] in Wien
\newline{}Erhalt  durch Arthur Schnitzler in Wien}\toendnotes[C]{\smallbreak}
\Standort{CUL, Schnitzler, B 200.}
\physDesc{Brief, 1 Blatt, 3 Seiten, 536 Zeichen
\newline{}Handschrift: schwarze Tinte, lateinische Kurrent
\newline{}Schnitzler: beschriftet: »Zuckerkandl« }\toendnotes[C]{\smallbreak}
\pstart{}{\pb}Hochverehrter Herr
                  Doktor!\pend\vspace{0.5em}
\pstart
           Erlauben sie mir Ihnen meinen herzlichsten, innigsten Dank für Ihre wirklich rührende
                  \label{K_L04015-1v}\edtext{Güte}{\lemma{\textnormal{\emph{Güte}}}\Cendnote{\textnormal{Wahrscheinlich stellt dieser undatierte Brief den Dank dafür
                  dar, dass Schnitzlers Pantomime \emph{Der Schleier der Pierrette}\pwindex{Schnitzler, Arthur 15. 5. 1862 Wien – 21. 10. 1931 ebd.@\textsc{Schnitzler, Arthur} (15. 5. 1862 Wien – 21. 10. 1931 ebd.), \emph{Schriftsteller, Mediziner}!Schleier der Pierrette. Pantomime in drei Bildern@\strich\emph{Der Schleier der Pierrette. Pantomime in drei Bildern}|pwk} mit Musik von Ernst von Dohnány\pwindex{Dohnányi, Ernst von 27.\,7.\,1877 Bratislava – 9.\,2.\,1960 New York City@\textsc{Dohnányi, Ernst von} (27.\,7.\,1877 Bratislava – 9.\,2.\,1960 New York City), \emph{Komponist, Pianist}|pwk} im Rahmen einer
                  Wohltätigkeitsveranstaltung im Urania
                     Theater\oindex{Wien@\textbf{Wien}!I., Innere Stadt@\textbf{I., Innere Stadt}!Urania Theater@\textbf{Urania Theater}, \emph{Theater}|pwk} zugunsten der \emph{Gesellschaft zur
                     Erforschung der Krebskrankheit}\orgindex{Österreichische Krebshilfe@Österreichische Krebshilfe|pwk} aufgeführt werden konnte. Ein kurzer,
                  namentlich nicht gezeichneter Bericht über den Abend\eventindex{Urania Theater@\textbf{Urania Theater}!Premiere von Der Schleier der Pierrette, 16.3.1912@Premiere von Der Schleier der Pierrette, 16.3.1912|pwkv} findet sich im \emph{Neuen Wiener Journal}\pwindex{Neues Wiener Journal@\emph{Neues Wiener Journal}|pwk}, für das Berta
                     Zuckerkandl\pwindex{Zuckerkandl, Berta 13.\,4.\,1864 Wien – 16.\,10.\,1945 Paris@\textsc{Zuckerkandl, Berta} (13.\,4.\,1864 Wien – 16.\,10.\,1945 Paris), \emph{Schriftstellerin, Journalistin, Übersetzerin}|pwk} regelmäßig schrieb: \emph{(Wohltätigkeitssoiree in der Urania.)}. In: \emph{Neues Wiener Journal}\pwindex{Neues Wiener Journal@\emph{Neues Wiener Journal}|pwk}, Jg. 20, Nr. 6609,
                        17. 3. 1912, S. 10. Schnitzler verließ die Veranstaltung noch vor Schluss,
                     vgl. A. S.: \emph{Tagebuch}, 16. 5. 1912, sodass ein
                  schriftlicher Dank noch am selben Abend oder direkt am
                     Folgetag wahrscheinlich scheint.}}}\label{K_L04015-1} auszusprechen. Sie haben
               mit der edelsten Bereitwilligkeit Ihr grosses Talent Ihre grosse Zugkraft uns zur
               Verfügung gestellt. Ich bin wirklich ganz be{\pb}schämt, und hoffe nun dass der \uline{kolossale} Erfolg den ihr geradezu verblüffend
               gescheuter Akt\pwindex{Schnitzler, Arthur 15. 5. 1862 Wien – 21. 10. 1931 ebd.@\textsc{Schnitzler, Arthur} (15. 5. 1862 Wien – 21. 10. 1931 ebd.), \emph{Schriftsteller, Mediziner}!Schleier der Pierrette. Pantomime in drei Bildern@\strich\emph{Der Schleier der Pierrette. Pantomime in drei Bildern}|pwuv}\eventindex{Urania Theater@\textbf{Urania Theater}!Premiere von Der Schleier der Pierrette, 16.3.1912@Premiere von Der Schleier der Pierrette, 16.3.1912|pwuv} errungen
               hat, Sie einigermaassen für \label{K_L04015-2v}\edtext{die grosse
               Mühe}{\lemma{\textnormal{\emph{die grosse
               Mühe}}}\Cendnote{\textnormal{Schnitzler dokumentiert im \emph{Tagebuch}\pwindex{Schnitzler, Arthur 15. 5. 1862 Wien – 21. 10. 1931 ebd.@\textsc{Schnitzler, Arthur} (15. 5. 1862 Wien – 21. 10. 1931 ebd.), \emph{Schriftsteller, Mediziner}!Tagebuch@\strich\emph{Tagebuch}|pwk} die Teilnahme an Proben am 10. 3. 1912, 11. 3. 1912, 15. 3. 1912 und 16. 3. 1912.}}}\label{K_L04015-2} entschädigt hat. Nehmen Sie
               geehrter Herr die Versicherung meiner tiefsten Dankbarkeit entgegen.\pend
           
\pstart
           {\pb}Ihre sehr ergebene
                  {\\[\baselineskip]}\spacefill\mbox{Berta Zuckerkandl}\pend
           \leftskip=0em{}\selectlanguage{ngerman}\endnumbering\briefempfaengerindex{Schnitzler, Arthur@\textsc{Schnitzler, Arthur}!zzzZuckerkandl, Berta@\emph{von Berta Zuckerkandl}!1912-05-162@{{[}16. oder 17. 5. 1912?{]}}|)be}\mylabel{L04015h}
\begin{anhang}
\end{anhang}\newcommand{\dateiname}{L04015}\newcommand{\titel}{Berta Zuckerkandl an Arthur Schnitzler, [16. oder 17. 5. 1912?]}\newcommand{\editorInnen}{Herausgegeben von Jahnke, SelmaMüller, Martin Anton}%% latex-leseansicht-abspann.tex
%% Abspann für die Leseansicht.
%% Der Schalter \ifkorrekturansicht ist bereits durch den Vorspann gesetzt.

%% latex-abspann.tex
%% Gemeinsamer Abspann für Korrekturansicht und Leseansicht.
%% Setzt den Schalter \ifkorrekturansicht voraus (gesetzt in den
%% einbindenden Dateien latex-korrekturansicht-abspann.tex bzw.
%% latex-leseansicht-abspann.tex).
%% ---------------------------------------------------------------

\normalsize

% Das esempio-Environment wird nur in der Leseansicht benötigt
\ifkorrekturansicht\else
\newenvironment{esempio}[3]%
{
    \vspace{1.5ex}
    \rlap{\underline{#1}}
    \par
    \setlength{\parindent}{0cm}
    \nopagebreak
    \leftskip=#2cm
    \rightskip=#3cm
}
{
    \par
}
\fi

\doendnotes{C}
\bigskip
\vfill

\clearpage

\footnotesize

\ifkorrekturansicht
  \lohead{\textsc{register}}
\fi

% theindex-Environment neu definieren ohne reledmac
\makeatletter
\renewenvironment{theindex}{%
  \ifkorrekturansicht
    \section*{\indexname}%
  \else
    \subsubsection*{Index der erwähnten Entitäten}%
  \fi
  \setlength{\parindent}{0pt}%
  \setlength{\parskip}{0pt plus 0.3pt}%
  \let\item\@idxitem
}{%
  \ifkorrekturansicht\clearpage\fi
}
\makeatother

\IfFileExists{\jobname-pw.ind}{\input{\jobname-pw.ind}}{}

% Quellenangabe nur in der Leseansicht
\ifkorrekturansicht\else
% Fallback-Definitionen, falls die .tex-Datei \titel etc. nicht gesetzt hat
\providecommand{\titel}{}
\providecommand{\editorInnen}{}
\providecommand{\dateiname}{\jobname}

\vspace{3cm}

\vfill

\footnotesize
\textsc{Quelle}: \titel. Herausgegeben von {\editorInnen}. In: \emph{Arthur Schnitzler: Briefwechsel mit Autorinnen und Autoren}.
 Digitale Edition, https://schnitzler-briefe.acdh.oeaw.ac.at/{\dateiname}.html (Stand \today)
\fi

\end{document}


