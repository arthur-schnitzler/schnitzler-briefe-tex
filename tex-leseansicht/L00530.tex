%% latex-korrekturansicht-vorspann.tex
%% Vorspann für die Korrekturansicht.
%% Lädt die gemeinsame Datei latex-vorspann.tex mit gesetztem Schalter.

\newif\ifkorrekturansicht
\korrekturansichttrue

\input{../tex-inputs/latex-vorspann}


\section[Lou Andreas-Salomé an Arthur Schnitzler, 28. 1. 1896]{L00530 Lou Andreas-Salomé an Arthur Schnitzler, 28. 1. 1896}
\nopagebreak\mylabel{L00530v}
\rehead{ }\normalsize\beginnumbering\briefempfaengerindex{Schnitzler, Arthur@\textsc{Schnitzler, Arthur}!zzzAndreas-Salome, Lou@\emph{von Lou Andreas-Salomé}!1896-01-281@{28. 1. 1896}|(be}
\toendnotes[C]{\smallbreak\pagebreak[2]}\Standort{CUL, Schnitzler, B 3.}
\physDesc{Kartenbrief, 532 Zeichen
\newline{}Handschrift: schwarze Tinte, deutsche Kurrent
\newline{}Versand: 1) Stempel: »\nobreak{}\oindex{I., Innere Stadt@\textbf{I., Innere Stadt}, \emph{A.ADM3}|pwk}Wien 1/1, 28. 1. 96, 9 10 N\nobreak{}«.   2) Stempel: »\nobreak{}\oindex{IX., Alsergrund@\textbf{IX., Alsergrund}, \emph{A.ADM3}|pwk}Wien {[}9/3{]}, 29.1{[}.96{]}, 8 {[}V{]}\nobreak{}«. 
\newline{}Ordnung: mit Bleistift von unbekannter Hand nummeriert:
                                    »17« }\toendnotes[C]{\smallbreak}\pstart{}{\pb}Herrn \textsc{D\textsuperscript{r}}\pend{}\pstart{}\textsc{Arthur Schnitzler}\pend{}\pstart{}\textsc{Wien IX\oindex{IX., Alsergrund@\textbf{IX., Alsergrund}, \emph{A.ADM3}|pw}}\pend{}\pstart{}Frankgasse 1\oindex{Frankgasse 1@\textbf{Frankgasse 1}, \emph{Wohngebäude (K.WHS)}|pw}. \pend{}{\bigskip}\vspace{1em}
\pstart
           \noindent{}{\pb}Lieber Herr \textsc{D\textsuperscript{r}}, danke für Ihren Beſuch. ich ſchlief ſo feſt, daß ich Sie nicht
               einmal klopfen gehört habe. Sie werden vor mir in Berlin\oindex{Berlin@\textbf{Berlin}, \emph{P.PPLC}|pw}{ }ſein: wollen Sie ſo gut ſein, mir \uline{hierher nach Wien\oindex{Wien@\textbf{Wien}, \emph{A.ADM2}|pw}} eine Karte mit Angabe Ihrer Hôteladreſſe zu ſchicken? ich ſuche Sie gleich auf,
               ſobald ich ankomme, – \uline{wenn} ich ankomme. Aber ich weiß
               es, von Stunde zu Stunde, nicht, wann das ſein wird.\pend
           
\pstart
           Sie werden gewiß viel Freude in Berlin\oindex{Berlin@\textbf{Berlin}, \emph{P.PPLC}|pw} erleben;
               ich wünſche Ihnen eine gute Beſetzung\pwindex{Liebelei. Schauspiel in drei Akten@\emph{Liebelei. Schauspiel in drei Akten}|pwv} und viel, viel Glück.\pend
           
\pstart
           Herzlich Ihre{\\[\baselineskip]}\spacefill\mbox{LouAS.}\pend
           \leftskip=0em{}\selectlanguage{ngerman}\endnumbering\briefempfaengerindex{Schnitzler, Arthur@\textsc{Schnitzler, Arthur}!zzzAndreas-Salome, Lou@\emph{von Lou Andreas-Salomé}!1896-01-281@{28. 1. 1896}|)be}\mylabel{L00530h}  \normalsize

\doendnotes{C}
\bigskip
\vfill

\clearpage

\footnotesize

\lohead{\textsc{register}}

% Definiere theindex-Environment komplett neu ohne reledmac
\makeatletter
\renewenvironment{theindex}{%
  \section*{\indexname}%
  \setlength{\parindent}{0pt}%
  \setlength{\parskip}{0pt plus 0.3pt}%
  \let\item\@idxitem
}{%
  \clearpage
}
\makeatother

\IfFileExists{\jobname-pw.ind}{\input{\jobname-pw.ind}}{}

\end{document}

      