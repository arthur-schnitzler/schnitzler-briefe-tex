%% latex-leseansicht-vorspann.tex
%% Vorspann für die Leseansicht.
%% Lädt die gemeinsame Datei latex-vorspann.tex mit nicht gesetztem Schalter.

\newif\ifkorrekturansicht
\korrekturansichtfalse

\input{../tex-inputs/latex-vorspann}


\section[Friedrich M. Fels an Arthur Schnitzler, {[}vor dem 1. 1. 1893?{]}]{L00152 Friedrich M. Fels an Arthur Schnitzler, {[}vor dem 1. 1. 1893?{]}}
\nopagebreak\mylabel{L00152v}
\rehead{ }\normalsize\beginnumbering\briefempfaengerindex{Schnitzler, Arthur@\textsc{Schnitzler, Arthur}!zzzFels, Friedrich Michael@\emph{von Friedrich Michael Fels}!1892-12-313@{{[}vor dem 1. 1. 1893?{]}}|(be}
\toendnotes[C]{\smallbreak\pagebreak[2]}
\correspDesc{Versand  durch Friedrich M. Fels am [vor dem 1. 1. 1893?] in Wien
\newline{}Erhalt  durch Arthur Schnitzler im Zeitraum [31. 12. 1892 – 4. 1. 1893?] in Wien}\toendnotes[C]{\smallbreak}
\Standort{DLA, A:Schnitzler, HS.NZ85.1.2956.}
\physDesc{Brief, 1 Blatt, 1 Seite, 421 Zeichen
\newline{}Handschrift: schwarze Tinte, lateinische Kurrent
\newline{}Schnitzler: mit Bleistift datiert: »93« und nummeriert: »4« }\toendnotes[C]{\smallbreak}
\pstart
           \noindent{}{\pb}Lieber Dr\hspace*{1.5em}Wie man sich bisweilen irren ka{\geminationn}: Gestern kam ich gar nicht ins Café, sondern um
                  5 Uhr lag ich im Bett. – Warum sah ich sie heute Frühe nicht? Und es
               wäre grade so dringend gewesen! Ich muſs vielleicht heute noch \label{K_L00152-1v}\edtext{ausziehen}{\lemma{\textnormal{\emph{ausziehen}}}\Cendnote{\textnormal{Da sich Fels\pwindex{Fels, Friedrich Michael *~1864 Bad Dürkheim@\textsc{Fels, Friedrich Michael} (*~1864 Bad Dürkheim), \emph{Journalist}|pwk} am
                     1. 1. 1893 für eine neue Wohnung entschieden hat, dürfte es sich
                  bei dieser um die vorhergehende Adresse in der Leopoldstadt\oindex{II., Leopoldstadt@\textbf{II., Leopoldstadt}, \emph{Verwaltungsgebiet}|pwk} handeln.}}}\label{K_L00152-1}: das hätte mit Ihnen gesprochen.\pend
           
\pstart
           – Bitte, \uline{nach 5 Uhr} auf \uline{einen Augenblick ins Central\oindex{Wien@\textbf{Wien}!I., Innere Stadt@\textbf{I., Innere Stadt}!Café Central@\textbf{Café Central}, \emph{Kaffeehaus}|pw}}, nicht ins groſse Lokal, sondern ins erste der langen Reihe. Ich bitte Sie so
               dringend wie herzlich darum.\pend
           \pstart \spacefill\mbox{Fels}\pend{}\selectlanguage{ngerman}\endnumbering\briefempfaengerindex{Schnitzler, Arthur@\textsc{Schnitzler, Arthur}!zzzFels, Friedrich Michael@\emph{von Friedrich Michael Fels}!1892-12-313@{{[}vor dem 1. 1. 1893?{]}}|)be}\mylabel{L00152h}  \newcommand{\dateiname}{L00152}\newcommand{\titel}{Friedrich M. Fels an Arthur Schnitzler, [vor dem 1. 1. 1893?]}\newcommand{\editorInnen}{Martin Anton Müller und Gerd-Hermann Susen}%% latex-leseansicht-abspann.tex
%% Abspann für die Leseansicht.
%% Der Schalter \ifkorrekturansicht ist bereits durch den Vorspann gesetzt.

%% latex-abspann.tex
%% Gemeinsamer Abspann für Korrekturansicht und Leseansicht.
%% Setzt den Schalter \ifkorrekturansicht voraus (gesetzt in den
%% einbindenden Dateien latex-korrekturansicht-abspann.tex bzw.
%% latex-leseansicht-abspann.tex).
%% ---------------------------------------------------------------

\normalsize

% Das esempio-Environment wird nur in der Leseansicht benötigt
\ifkorrekturansicht\else
\newenvironment{esempio}[3]%
{
    \vspace{1.5ex}
    \rlap{\underline{#1}}
    \par
    \setlength{\parindent}{0cm}
    \nopagebreak
    \leftskip=#2cm
    \rightskip=#3cm
}
{
    \par
}
\fi

\doendnotes{C}
\bigskip
\vfill

\clearpage

\footnotesize

\ifkorrekturansicht
  \lohead{\textsc{register}}
\fi

% theindex-Environment neu definieren ohne reledmac
\makeatletter
\renewenvironment{theindex}{%
  \ifkorrekturansicht
    \section*{\indexname}%
  \else
    \subsubsection*{Index der erwähnten Entitäten}%
  \fi
  \setlength{\parindent}{0pt}%
  \setlength{\parskip}{0pt plus 0.3pt}%
  \let\item\@idxitem
}{%
  \ifkorrekturansicht\clearpage\fi
}
\makeatother

\IfFileExists{\jobname-pw.ind}{\input{\jobname-pw.ind}}{}

% Quellenangabe nur in der Leseansicht
\ifkorrekturansicht\else
% Fallback-Definitionen, falls die .tex-Datei \titel etc. nicht gesetzt hat
\providecommand{\titel}{}
\providecommand{\editorInnen}{}
\providecommand{\dateiname}{\jobname}

\vspace{3cm}

\vfill

\footnotesize
\textsc{Quelle}: \titel. Herausgegeben von {\editorInnen}. In: \emph{Arthur Schnitzler: Briefwechsel mit Autorinnen und Autoren}.
 Digitale Edition, https://schnitzler-briefe.acdh.oeaw.ac.at/{\dateiname}.html (Stand \today)
\fi

\end{document}


