%% latex-korrekturansicht-vorspann.tex
%% Vorspann für die Korrekturansicht.
%% Lädt die gemeinsame Datei latex-vorspann.tex mit gesetztem Schalter.

\newif\ifkorrekturansicht
\korrekturansichttrue

\input{../tex-inputs/latex-vorspann}


\section[Arthur Schnitzler an Richard Beer-Hofmann, 4. 1. 1894]{L00289 Arthur Schnitzler an Richard Beer-Hofmann, 4. 1. 1894}
\nopagebreak\mylabel{L00289v}
\rehead{ }\normalsize\beginnumbering\briefempfaengerindex{Beer-Hofmann, Richard@\textsc{Beer-Hofmann, Richard}!zzzSchnitzler, Arthur@\emph{von Arthur Schnitzler}!1894-01-041@{4. 1. 1894}|(be}
\toendnotes[C]{\smallbreak\pagebreak[2]}\Standort{YCGL, MSS 31.}
\physDesc{Briefkarte, 407 Zeichen (Briefkarte mit Trauerrand )
\newline{}Handschrift: schwarze Tinte, deutsche Kurrent
\newline{}Ordnung: mit Bleistift von unbekannter Hand unterhalb der Monatsangabe
                                 die alternative Datierung »5.« vermerkt }\toendnotes[C]{\smallbreak}
\pstart
           \noindent{}{\pb}Lieber Richard, bitte ſenden Sie dem \textsc{Fels}\pwindex{Fels, Friedrich Michael *~1864@\textsc{Fels, Friedrich Michael} (*~1864), \emph{Journalist/Journalistin}|pw} möglichſt bald die beſprochenen Sachen; – auch das Geld können Sie direct an
               ihn ſenden; ich habe mich vergewiſſert, dß es ihn nicht beleidigen wird. –\pend
           
\pstart
           Es iſt traurig, dß wir uns ſo ſelten ſehn. –\pend
           
\pstart
           Morgen will ich entweder zur böſen Nacht\pwindex{Eine boese Nacht. Lustspiel in 3 Acten@\emph{Eine böse Nacht. Lustspiel in 3 Acten}|pw} oder
               zum \label{K_L00289-1v}\edtext{Bild des Signorelli\pwindex{Bild des Signorelli. Schauspiel in 4 Acten@\emph{Das Bild des Signorelli. Schauspiel in 4 Acten}|pw}}{\lemma{\textnormal{\emph{Bild des Signorelli}}}\Cendnote{\textnormal{Er entschied sich dafür und ging in
                   die Uraufführung\eventindex{Raimund-Theater@\textbf{Raimund-Theater}!Urauffuehrung Das Bild des Signorelli, 5.1.1894@Uraufführung Das Bild des Signorelli, 5.1.1894|pwkv} ins Raimund-Theater\oindex{Raimund-Theater@\textbf{Raimund-Theater}, \emph{Theater (K.THE)}|pwk}.}}}\label{K_L00289-1}:
               Jedenfalls {\pb}könnten wir uns alle wieder einmal gegen
               eilf im Central\oindex{Cafe Central@\textbf{Café Central}, \emph{Kaffeehaus (K.KAF)}|pw} finden.\pend
           
\pstart
           Herzliche Grüße{\\[\baselineskip]}Ihr{\\[\baselineskip]}\spacefill\mbox{Arthur}\pend
           \leftskip=0em{}
\pstart
           4. 1. 94.\pend
           \selectlanguage{ngerman}\endnumbering\briefempfaengerindex{Beer-Hofmann, Richard@\textsc{Beer-Hofmann, Richard}!zzzSchnitzler, Arthur@\emph{von Arthur Schnitzler}!1894-01-041@{4. 1. 1894}|)be}\mylabel{L00289h}  \normalsize

\doendnotes{C}
\bigskip
\vfill

\clearpage

\footnotesize

\lohead{\textsc{register}}

% Definiere theindex-Environment komplett neu ohne reledmac
\makeatletter
\renewenvironment{theindex}{%
  \section*{\indexname}%
  \setlength{\parindent}{0pt}%
  \setlength{\parskip}{0pt plus 0.3pt}%
  \let\item\@idxitem
}{%
  \clearpage
}
\makeatother

\IfFileExists{\jobname-pw.ind}{\input{\jobname-pw.ind}}{}

\end{document}

      