%% latex-leseansicht-vorspann.tex
%% Vorspann für die Leseansicht.
%% Lädt die gemeinsame Datei latex-vorspann.tex mit nicht gesetztem Schalter.

\newif\ifkorrekturansicht
\korrekturansichtfalse

\input{../tex-inputs/latex-vorspann}


\section[Arthur Schnitzler an Richard Beer-Hofmann, 4. 1. 1894]{L00289 Arthur Schnitzler an Richard Beer-Hofmann, 4. 1. 1894}
\nopagebreak\mylabel{L00289v}
\rehead{ }\normalsize\beginnumbering\briefempfaengerindex{Beer-Hofmann, Richard@\textsc{Beer-Hofmann, Richard}!zzzSchnitzler, Arthur@\emph{von Arthur Schnitzler}!1894-01-041@{4. 1. 1894}|(be}
\toendnotes[C]{\smallbreak\pagebreak[2]}
\correspDesc{Versand  durch Arthur Schnitzler am 4. 1. 1894 in Wien
\newline{}Erhalt  durch Richard Beer-Hofmann im Zeitraum [4. 1. 1894
                  – 8. 1. 1894?] in Wien}\toendnotes[C]{\smallbreak}
\Standort{YCGL, MSS 31.}
\physDesc{Briefkarte, 407 Zeichen (Briefkarte mit Trauerrand )
\newline{}Handschrift: schwarze Tinte, deutsche Kurrent
\newline{}Ordnung: mit Bleistift von unbekannter Hand unterhalb der Monatsangabe
                                 die alternative Datierung »5.« vermerkt }\toendnotes[C]{\smallbreak}
\pstart
           \noindent{}{\pb}Lieber Richard, bitte{ }ſenden Sie dem \textsc{Fels}\pwindex{Fels, Friedrich Michael *~1864 Bad Dürkheim@\textsc{Fels, Friedrich Michael} (*~1864 Bad Dürkheim), \emph{Journalist}|pw} möglichſt bald die beſprochenen Sachen; – auch das Geld können Sie direct an
               ihn{ }ſenden; ich habe mich vergewiſſert, dß es ihn nicht beleidigen wird. –\pend
           
\pstart
           Es iſt traurig, dß wir uns{ }ſo{ }ſelten{ }ſehn. –\pend
           
\pstart
           Morgen will ich entweder zur böſen Nacht\pwindex{\textcolor{red}{\textsuperscript{XXXX indx1}}!Eine böse Nacht. Lustspiel in 3 Acten@\strich\emph{Eine böse Nacht. Lustspiel in 3 Acten}|pw} oder
               zum \label{K_L00289-1v}\edtext{Bild des Signorelli\pwindex{\textcolor{red}{\textsuperscript{XXXX indx1}}!Bild des Signorelli. Schauspiel in 4 Acten@\strich\emph{Das Bild des Signorelli. Schauspiel in 4 Acten}|pw}}{\lemma{\textnormal{\emph{Bild des Signorelli}}}\Cendnote{\textnormal{Er entschied sich dafür und ging in
                   die Uraufführung\eventindex{Raimund-Theater@\textbf{Raimund-Theater}!Uraufführung Das Bild des Signorelli, 5.1.1894@Uraufführung Das Bild des Signorelli, 5.1.1894|pwkv} ins Raimund-Theater\oindex{Wien@\textbf{Wien}!VI., Mariahilf@\textbf{VI., Mariahilf}!Raimund-Theater@\textbf{Raimund-Theater}, \emph{Theater}|pwk}.}}}\label{K_L00289-1}:
               Jedenfalls {\pb}könnten wir uns alle wieder einmal gegen
               eilf im Central\oindex{Wien@\textbf{Wien}!I., Innere Stadt@\textbf{I., Innere Stadt}!Café Central@\textbf{Café Central}, \emph{Kaffeehaus}|pw} finden.\pend
           
\pstart
           Herzliche Grüße{\\[\baselineskip]}Ihr{\\[\baselineskip]}\spacefill\mbox{Arthur}\pend
           \leftskip=0em{}
\pstart
           4. 1. 94.\pend
           \selectlanguage{ngerman}\endnumbering\briefempfaengerindex{Beer-Hofmann, Richard@\textsc{Beer-Hofmann, Richard}!zzzSchnitzler, Arthur@\emph{von Arthur Schnitzler}!1894-01-041@{4. 1. 1894}|)be}\mylabel{L00289h}  \newcommand{\dateiname}{L00289}\newcommand{\titel}{Arthur Schnitzler an Richard Beer-Hofmann, 4. 1. 1894}\newcommand{\editorInnen}{Martin Anton Müller und Gerd-Hermann Susen}%% latex-leseansicht-abspann.tex
%% Abspann für die Leseansicht.
%% Der Schalter \ifkorrekturansicht ist bereits durch den Vorspann gesetzt.

%% latex-abspann.tex
%% Gemeinsamer Abspann für Korrekturansicht und Leseansicht.
%% Setzt den Schalter \ifkorrekturansicht voraus (gesetzt in den
%% einbindenden Dateien latex-korrekturansicht-abspann.tex bzw.
%% latex-leseansicht-abspann.tex).
%% ---------------------------------------------------------------

\normalsize

% Das esempio-Environment wird nur in der Leseansicht benötigt
\ifkorrekturansicht\else
\newenvironment{esempio}[3]%
{
    \vspace{1.5ex}
    \rlap{\underline{#1}}
    \par
    \setlength{\parindent}{0cm}
    \nopagebreak
    \leftskip=#2cm
    \rightskip=#3cm
}
{
    \par
}
\fi

\doendnotes{C}
\bigskip
\vfill

\clearpage

\footnotesize

\ifkorrekturansicht
  \lohead{\textsc{register}}
\fi

% theindex-Environment neu definieren ohne reledmac
\makeatletter
\renewenvironment{theindex}{%
  \ifkorrekturansicht
    \section*{\indexname}%
  \else
    \subsubsection*{Index der erwähnten Entitäten}%
  \fi
  \setlength{\parindent}{0pt}%
  \setlength{\parskip}{0pt plus 0.3pt}%
  \let\item\@idxitem
}{%
  \ifkorrekturansicht\clearpage\fi
}
\makeatother

\IfFileExists{\jobname-pw.ind}{\input{\jobname-pw.ind}}{}

% Quellenangabe nur in der Leseansicht
\ifkorrekturansicht\else
% Fallback-Definitionen, falls die .tex-Datei \titel etc. nicht gesetzt hat
\providecommand{\titel}{}
\providecommand{\editorInnen}{}
\providecommand{\dateiname}{\jobname}

\vspace{3cm}

\vfill

\footnotesize
\textsc{Quelle}: \titel. Herausgegeben von {\editorInnen}. In: \emph{Arthur Schnitzler: Briefwechsel mit Autorinnen und Autoren}.
 Digitale Edition, https://schnitzler-briefe.acdh.oeaw.ac.at/{\dateiname}.html (Stand \today)
\fi

\end{document}


