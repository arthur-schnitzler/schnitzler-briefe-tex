%% latex-leseansicht-vorspann.tex
%% Vorspann für die Leseansicht.
%% Lädt die gemeinsame Datei latex-vorspann.tex mit nicht gesetztem Schalter.

\newif\ifkorrekturansicht
\korrekturansichtfalse

\input{../tex-inputs/latex-vorspann}


         
         \renewcommand{\erwaehntePersonen}{Personen: Paul Goldmann, Clementine Goldmann, Theodore Rottenberg, Paul Schlenther, Olga Schnitzler, Ida d’Albert}
         \renewcommand{\erwaehnteInstitutionen}{Institutionen: Bauernfeld-Preis, Burgtheater, Deutsches Theater Berlin, Neue Freie Presse}
         \renewcommand{\erwaehnteOrte}{Orte: Berlin, Dessauer Straße, Deutsches Theater Berlin, Volkstheater, Wien}
         \renewcommand{\erwaehnteWerke}{Werke: Berliner Theater. (»Der Schleier der Beatrice« von Arthur Schnitzler.), Berliner Theater. (»Lebendige Stunden« von Arthur Schnitzler.), Der Schleier der Beatrice. Schauspiel in fünf Akten, Lebendige Stunden. Vier Einakter, Neue Freie Presse, Tagebuch}
               \section[ Paul Goldmann an Arthur Schnitzler, 17. 3. {[}1903{]}]{ Paul Goldmann an Arthur Schnitzler, 17. 3. {[}1903{]}}\nopagebreak\mylabel{v}\rehead{ }\begin{ledgroupsized}[t]{13cm}\normalsize\beginnumbering\briefempfaengerindex{Schnitzler, Arthur@\textsc{Schnitzler, Arthur}!zzzGoldmann, Paul@\emph{von Paul Goldmann}!1903-03-171@{17. 3. {[}1903{]}}|(be} \toendnotes[C]{\smallbreak\pagebreak[2]} \Standort{DLA, A:Schnitzler, HS.NZ85.1.3173.}
\physDesc{Brief, 1 Blatt, 4 Seiten, 1465 Zeichen
\newline{}Handschrift: blaue Tinte, deutsche Kurrent
\newline{}Schnitzler: 1) mit Bleistift das Jahr »903« vermerkt  2) mit rotem Buntstift eine Unterstreichung}\toendnotes[C]{\smallbreak}\pstart
           \noindent{}\raggedleft{}{\pb}\textcolor{gray}{\textbf{DESSAUERSTRASSE 19\oindex{Dessauer Strasse@\textbf{Dessauer Straße}|pw}}}\pend
           \pstart
           Berlin\oindex{Berlin@\textbf{Berlin}|pw}, 17. März.\pend
           \pstart\center{}Mein lieber Freund,\pend\pstart
           Ich habe mit großer Freude \strikeout{\textcolor{gray}{ver}} geleſen, daß Du den \label{K_L03369-1v}\edtext{\textsc{Bauernfeld}-Preis\orgindex{Bauernfeld-Preis@Bauernfeld-Preis|pw}}{\lemma{\textnormal{\emph{Bauernfeld-Preis}}}\Cendnote{\textnormal{Den \emph{Bauernfeld-Preis}\orgindex{Bauernfeld-Preis@Bauernfeld-Preis|pwk} erhielt Schnitzler\pwindex{Schnitzler, Arthur 15.05.1862 – 21.10.1931@\textsc{Schnitzler, Arthur} (15.05.1862 – 21.10.1931), \emph{Schriftsteller, Mediziner}|pwk}
                  am 17. 3. 1903 für
                  seinen Einakterzyklus \emph{Lebendige Stunden}\pwindex{Schnitzler, Arthur 15.05.1862 – 21.10.1931@\textsc{Schnitzler, Arthur} (15.05.1862 – 21.10.1931), \emph{Schriftsteller, Mediziner}!Lebendige Stunden. Vier Einakter1901-12-23@\strich\emph{Lebendige Stunden. Vier Einakter} {[}1901-12-23{]}|pwk}.
                     1899 hatte er bereits eine Ehrengabe\orgindex{Bauernfeld-Preis@Bauernfeld-Preis|pwkv}.}}}\label{K_L03369-1h} erhalten haſt, u.
               beglückwünſche Dich (auch im Namen meiner Mutter\pwindex{Goldmann, Clementine 1842-05-15 – 1924-02-24@\textsc{Goldmann, Clementine} (1842-05-15 – 1924-02-24)|pwv}) auf das Herzlichſte.\pend
           \pstart
           Auch höre ich, daß die \label{K_L03369-2v}\edtext{»\textsc{Beatrice\pwindex{Schnitzler, Arthur 15.05.1862 – 21.10.1931@\textsc{Schnitzler, Arthur} (15.05.1862 – 21.10.1931), \emph{Schriftsteller, Mediziner}!Schleier der Beatrice. Schauspiel in fuenf Akten1900-12-01@\strich\emph{Der Schleier der Beatrice. Schauspiel in fünf Akten} {[}1900-12-01{]}|pw}}«}{\lemma{\textnormal{\emph{»Beatrice«}}}\Cendnote{\textnormal{am \emph{Deutschen Theater Berlin}\orgindex{Deutsches Theater Berlin@Deutsches Theater Berlin|pwk}}}}\label{K_L03369-2h} gut geht. Frau \textsc{Fulda\pwindex{DAlbert, Ida 05.12.1869 – 1926-10-06@\textsc{d’Albert, Ida} (05.12.1869 – 1926-10-06)|pw}} ſagte es mir; ſie fügte hinzu, Sonntag ſei das
                  Haus\oindex{Deutsches Theater Berlin@\textbf{Deutsches Theater Berlin}|pwv} ausverkauft geweſen. {\pb}Auch das freut mich von Herzen.\pend
           \pstart
           Heut habe ich nun endlich mein \label{K_L03369-3v}\edtext{Feuilleton\pwindex{Goldmann, Paul 31.01.1865 – 25.09.1935@\textsc{Goldmann, Paul} (31.01.1865 – 25.09.1935), \emph{Schriftsteller, Journalist}!Berliner Theater. (»Der Schleier der Beatrice« von Arthur Schnitzler.)1903-03-19@\strich\emph{Berliner Theater. (»Der Schleier der Beatrice« von Arthur Schnitzler.)} {[}1903-03-19{]}|pwv}}{\lemma{\textnormal{\emph{Feuilleton}}}\Cendnote{\textnormal{Paul Goldmann\pwindex{Goldmann, Paul 31.01.1865 – 25.09.1935@\textsc{Goldmann, Paul} (31.01.1865 – 25.09.1935), \emph{Schriftsteller, Journalist}|pwk}: \emph{Berliner Theater. (»Der Schleier der Beatrice« von Arthur
                        Schnitzler.)}\pwindex{Goldmann, Paul 31.01.1865 – 25.09.1935@\textsc{Goldmann, Paul} (31.01.1865 – 25.09.1935), \emph{Schriftsteller, Journalist}!Berliner Theater. (»Der Schleier der Beatrice« von Arthur Schnitzler.)1903-03-19@\strich\emph{Berliner Theater. (»Der Schleier der Beatrice« von Arthur Schnitzler.)} {[}1903-03-19{]}|pwk}. In: \emph{Neue Freie
                        Presse}\pwindex{Neue Freie Presse1864 – 1939@\emph{Neue Freie Presse} {[}1864 – 1939{]}|pwk}, Nr. 13.851, 19. 3. 1903,
                     Morgenblatt, S. 1–5. Dieses negativ urteilende Feuilleton\pwindex{Goldmann, Paul 31.01.1865 – 25.09.1935@\textsc{Goldmann, Paul} (31.01.1865 – 25.09.1935), \emph{Schriftsteller, Journalist}!Berliner Theater. (»Der Schleier der Beatrice« von Arthur Schnitzler.)1903-03-19@\strich\emph{Berliner Theater. (»Der Schleier der Beatrice« von Arthur Schnitzler.)} {[}1903-03-19{]}|pwkv} stellt eine Zäsur in der
                  Beziehung zwischen Goldmann\pwindex{Goldmann, Paul 31.01.1865 – 25.09.1935@\textsc{Goldmann, Paul} (31.01.1865 – 25.09.1935), \emph{Schriftsteller, Journalist}|pwk} und Schnitzler\pwindex{Schnitzler, Arthur 15.05.1862 – 21.10.1931@\textsc{Schnitzler, Arthur} (15.05.1862 – 21.10.1931), \emph{Schriftsteller, Mediziner}|pwk} dar. Nach Goldmann\pwindex{Goldmann, Paul 31.01.1865 – 25.09.1935@\textsc{Goldmann, Paul} (31.01.1865 – 25.09.1935), \emph{Schriftsteller, Journalist}|pwk}s kritischem Feuilleton\pwindex{Goldmann, Paul 31.01.1865 – 25.09.1935@\textsc{Goldmann, Paul} (31.01.1865 – 25.09.1935), \emph{Schriftsteller, Journalist}!Berliner Theater. (»Lebendige Stunden« von Arthur Schnitzler.)1902-01-22@\strich\emph{Berliner Theater. (»Lebendige Stunden« von Arthur Schnitzler.)} {[}1902-01-22{]}|pwkv} zu \emph{Lebendige
                     Stunden}\pwindex{Schnitzler, Arthur 15.05.1862 – 21.10.1931@\textsc{Schnitzler, Arthur} (15.05.1862 – 21.10.1931), \emph{Schriftsteller, Mediziner}!Lebendige Stunden. Vier Einakter1901-12-23@\strich\emph{Lebendige Stunden. Vier Einakter} {[}1901-12-23{]}|pwk} im Jahr zuvor war es in den folgenden
                  Jahren der zweite zentrale Punkt in deren Streit. In Schnitzler\pwindex{Schnitzler, Arthur 15.05.1862 – 21.10.1931@\textsc{Schnitzler, Arthur} (15.05.1862 – 21.10.1931), \emph{Schriftsteller, Mediziner}|pwk}s \emph{Tagebuch}\pwindex{\textcolor{red}{\textsuperscript{XXXX1 indx}}!Tagebuch1981 – 2000@\strich\emph{Tagebuch} {[}Hrsg., 1981 – 2000{]}|pwk}
                  finden sich ab dem 19. 3. 1903 mehrfach Notizen dazu.}}}\label{K_L03369-3h} abgeſandt. Ich habe zehn
               Tage lang damit gerungen – wahrhaft gerungen – habe allein den Anfang vier Mal neu
               geſchrieben. Das Stück\pwindex{Schnitzler, Arthur 15.05.1862 – 21.10.1931@\textsc{Schnitzler, Arthur} (15.05.1862 – 21.10.1931), \emph{Schriftsteller, Mediziner}!Schleier der Beatrice. Schauspiel in fuenf Akten1900-12-01@\strich\emph{Der Schleier der Beatrice. Schauspiel in fünf Akten} {[}1900-12-01{]}|pwv} hat
               mir, je mehr ich darauf einging, immer weniger gefallen. Ich finde es, bei allen
               dichteriſchen Eigenſchaften, innerlich klein. Nun habe ich mich aufs Äußerſte
               angeſtrengt, {\pb}gerecht zu ſein, mit jedem Worte. Mein
               Gewiſſen ſagt mir, daß ich es geweſen bin. Was Du ſagen wirſt, weiß ich nicht. Aber
               ich verwünſche mein Schickſal und ich frage mich, ob man dazu einen einzigen nahen
               und lieben Freund hat, um gegen ihn – öffentlich, vor allen Leuten – gerecht zu ſein?
               Vielleicht übrigens mißfällt das Feuilleton\pwindex{Goldmann, Paul 31.01.1865 – 25.09.1935@\textsc{Goldmann, Paul} (31.01.1865 – 25.09.1935), \emph{Schriftsteller, Journalist}!Berliner Theater. (»Der Schleier der Beatrice« von Arthur Schnitzler.)1903-03-19@\strich\emph{Berliner Theater. (»Der Schleier der Beatrice« von Arthur Schnitzler.)} {[}1903-03-19{]}|pwv} in der Redaktion\orgindex{Neue Freie Presse@Neue Freie Presse|pwv} und es erſcheint {\pb}gar nicht. Das
               wäre mir das Liebſte.\pend
           \pstart
           Auch zu dem \label{K_L03369-4v}\edtext{Erfolge der »Lebendigen St.\pwindex{Schnitzler, Arthur 15.05.1862 – 21.10.1931@\textsc{Schnitzler, Arthur} (15.05.1862 – 21.10.1931), \emph{Schriftsteller, Mediziner}!Lebendige Stunden. Vier Einakter1901-12-23@\strich\emph{Lebendige Stunden. Vier Einakter} {[}1901-12-23{]}|pw}« in Wien\oindex{Wien@\textbf{Wien}|pw}}{\lemma{\textnormal{\emph{Erfolge … Wien}}}\Cendnote{\textnormal{\emph{Lebendige Stunden}\pwindex{Schnitzler, Arthur 15.05.1862 – 21.10.1931@\textsc{Schnitzler, Arthur} (15.05.1862 – 21.10.1931), \emph{Schriftsteller, Mediziner}!Lebendige Stunden. Vier Einakter1901-12-23@\strich\emph{Lebendige Stunden. Vier Einakter} {[}1901-12-23{]}|pwk} hatte am 14. 3. 1903 am Deutschen Volkstheater\oindex{Volkstheater@\textbf{Volkstheater}|pwk} in Wien\oindex{Wien@\textbf{Wien}|pwk} Premiere.}}}\label{K_L03369-4h} beglückwünſche ich Dich auf das
               Herzlichſte. Wird nun der Herr \label{K_L03369-5v}\edtext{\textsc{Schlenther\pwindex{Schlenther, Paul 20.08.1854 – 30.04.1916@\textsc{Schlenther, Paul} (20.08.1854 – 30.04.1916), \emph{Schriftsteller, Kritiker, Theaterleiter}|pw}}}{\lemma{\textnormal{\emph{Schlenther}}}\Cendnote{\textnormal{Paul Schlenther\pwindex{Schlenther, Paul 20.08.1854 – 30.04.1916@\textsc{Schlenther, Paul} (20.08.1854 – 30.04.1916), \emph{Schriftsteller, Kritiker, Theaterleiter}|pwk} hatte 1900
                  abgelehnt, \emph{Der Schleier der Beatrice}\pwindex{Schnitzler, Arthur 15.05.1862 – 21.10.1931@\textsc{Schnitzler, Arthur} (15.05.1862 – 21.10.1931), \emph{Schriftsteller, Mediziner}!Schleier der Beatrice. Schauspiel in fuenf Akten1900-12-01@\strich\emph{Der Schleier der Beatrice. Schauspiel in fünf Akten} {[}1900-12-01{]}|pwk} am \emph{Burgtheater}\orgindex{Burgtheater@Burgtheater|pwk} aufzuführen. Der teilweise
                  öffentlich ausgetragene Konflikt führte dazu, dass für fünf Jahre keine neuen
                  Stücke Schnitzler\pwindex{Schnitzler, Arthur 15.05.1862 – 21.10.1931@\textsc{Schnitzler, Arthur} (15.05.1862 – 21.10.1931), \emph{Schriftsteller, Mediziner}|pwk}s auf der Bühne zu sehen
                  waren.}}}\label{K_L03369-5h} ſich nicht endlich rühren?\pend
           \pstart
           Dank für Deine lieben Zeilen aus Wien\oindex{Wien@\textbf{Wien}|pw}! Ich bin
               traurig, wie zuvor. Mein ganzes Leben iſt voll von dieſer \label{K_L03369-6v}\edtext{Frau\pwindex{Rottenberg, Theodore 1875-09-07 – 1945-04-05@\textsc{Rottenberg, Theodore} (1875-09-07 – 1945-04-05)|pwv}}{\lemma{\textnormal{\emph{Frau}}}\Cendnote{\textnormal{Theodore Rottenberg\pwindex{Rottenberg, Theodore 1875-09-07 – 1945-04-05@\textsc{Rottenberg, Theodore} (1875-09-07 – 1945-04-05)|pwk}, die Goldmann\pwindex{Goldmann, Paul 31.01.1865 – 25.09.1935@\textsc{Goldmann, Paul} (31.01.1865 – 25.09.1935), \emph{Schriftsteller, Journalist}|pwk} Anfang 1903 verlassen
                  hatte (vgl. Paul Goldmann an Arthur Schnitzler, 3. 1. [1903])}}}\label{K_L03369-6h}, die
               mich längſt vergeſſen hat.\pend
           \pstart
           Leb’ wohl, mein lieber Freund! Grüße \textsc{Olga\pwindex{Schnitzler, Olga 17.01.1882 – 13.01.1970@\textsc{Schnitzler, Olga} (17.01.1882 – 13.01.1970), \emph{Schauspielerin, Sängerin}|pw}} u. ſei Du ſelbſt vielmals gegrüßt von Deinem {\\[\baselineskip]}getreuen \spacefill\mbox{Paul
                  Goldm}\pend
           \leftskip=0em{}
         
         \endnumbering\mylabel{h}\end{ledgroupsized}  \newcommand{\dateiname}{L03369}\newcommand{\titel}{Paul Goldmann an Arthur Schnitzler, 17. 3. [1903]}\newcommand{\editorInnen}{Martin Anton Müller und Laura Untner}%% latex-leseansicht-abspann.tex
%% Abspann für die Leseansicht.
%% Der Schalter \ifkorrekturansicht ist bereits durch den Vorspann gesetzt.

%% latex-abspann.tex
%% Gemeinsamer Abspann für Korrekturansicht und Leseansicht.
%% Setzt den Schalter \ifkorrekturansicht voraus (gesetzt in den
%% einbindenden Dateien latex-korrekturansicht-abspann.tex bzw.
%% latex-leseansicht-abspann.tex).
%% ---------------------------------------------------------------

\normalsize

% Das esempio-Environment wird nur in der Leseansicht benötigt
\ifkorrekturansicht\else
\newenvironment{esempio}[3]%
{
    \vspace{1.5ex}
    \rlap{\underline{#1}}
    \par
    \setlength{\parindent}{0cm}
    \nopagebreak
    \leftskip=#2cm
    \rightskip=#3cm
}
{
    \par
}
\fi

\doendnotes{C}
\bigskip
\vfill

\clearpage

\footnotesize

\ifkorrekturansicht
  \lohead{\textsc{register}}
\fi

% theindex-Environment neu definieren ohne reledmac
\makeatletter
\renewenvironment{theindex}{%
  \ifkorrekturansicht
    \section*{\indexname}%
  \else
    \subsubsection*{Index der erwähnten Entitäten}%
  \fi
  \setlength{\parindent}{0pt}%
  \setlength{\parskip}{0pt plus 0.3pt}%
  \let\item\@idxitem
}{%
  \ifkorrekturansicht\clearpage\fi
}
\makeatother

\IfFileExists{\jobname-pw.ind}{\input{\jobname-pw.ind}}{}

% Quellenangabe nur in der Leseansicht
\ifkorrekturansicht\else
% Fallback-Definitionen, falls die .tex-Datei \titel etc. nicht gesetzt hat
\providecommand{\titel}{}
\providecommand{\editorInnen}{}
\providecommand{\dateiname}{\jobname}

\vspace{3cm}

\vfill

\footnotesize
\textsc{Quelle}: \titel. Herausgegeben von {\editorInnen}. In: \emph{Arthur Schnitzler: Briefwechsel mit Autorinnen und Autoren}.
 Digitale Edition, https://schnitzler-briefe.acdh.oeaw.ac.at/{\dateiname}.html (Stand \today)
\fi

\end{document}


      