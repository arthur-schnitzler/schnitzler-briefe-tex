%% latex-leseansicht-vorspann.tex
%% Vorspann für die Leseansicht.
%% Lädt die gemeinsame Datei latex-vorspann.tex mit nicht gesetztem Schalter.

\newif\ifkorrekturansicht
\korrekturansichtfalse

\input{../tex-inputs/latex-vorspann}


\section[Hermann Bahr an Arthur Schnitzler, 28. 9. 1912]{L02092 Hermann Bahr an Arthur Schnitzler, 28. 9. 1912}
\nopagebreak\mylabel{L02092v}
\rehead{ }\normalsize\beginnumbering\briefempfaengerindex{Schnitzler, Arthur@\textsc{Schnitzler, Arthur}!zzzBahr, Hermann@\emph{von Hermann Bahr}!1912-09-281@{28. 9. 1912}|(be}
\toendnotes[C]{\smallbreak\pagebreak[2]}
\correspDesc{Versand  durch Hermann Bahr am 28. 9. 1912 in Semmering
\newline{}Erhalt  durch Arthur Schnitzler am 29. IX. 12 in Wien}\toendnotes[C]{\smallbreak}
\Standort{CUL, Schnitzler, B 5b.}
\physDesc{Bildpostkarte, 437 Zeichen
\newline{}Handschrift: schwarze Tinte, deutsche Kurrent
\newline{}Versand: Stempel: »\nobreak{}\oindex{Semmering@\textbf{Semmering}, \emph{Verwaltungsgebiet}|pwk}Semmering, 29. IX. 12, XII\nobreak{}«.  
\newline{}Ordnung: mit Bleistift von unbekannter Hand nummeriert:
                                    »174« }
\buchAbdrucke{\weitereDrucke{Hermann Bahr, Arthur Schnitzler: \emph{Briefwechsel, Aufzeichnungen, Dokumente (1891–1931)}. Herausgegeben von Kurt Ifkovits und Martin Anton Müller. Göttingen: \emph{Wallstein} 2018, S. 478.} }\toendnotes[C]{\smallbreak}\pstart{}{\pb}Herrn \textsc{D\textsuperscript{r} Artur Schnitzler}\pend{}\pstart{}XVIII Sternwarteſtr 71\oindex{Wien@\textbf{Wien}!XVIII., Währing@\textbf{XVIII., Währing}!Sternwartestraße 71@\textbf{Sternwartestraße 71}, \emph{Wohngebäude}|pw}\pend{}\pstart{}\textsc{Wien XVIII\oindex{XVIII., Währing@\textbf{XVIII., Währing}, \emph{Verwaltungsgebiet}|pw}}\pend{}{\bigskip}
\pstart
           \noindent{}\centering{}{\pb}\textcolor{gray}{\textbf{Venezia – Cortile Palazzo Ducale\oindex{Palazzo Ducale@\textbf{Palazzo Ducale}, \emph{Gebäude}|pw}}}\pend
           
\pstart
           {\pb}Semmering Villa Mautner\oindex{Villa Mauthner-Markhof@\textbf{Villa Mauthner-Markhof}, \emph{Wohngebäude}|pw}\pend
           \vspace{1em}
\pstart
           \raggedleft{}{\pb}28. 9. 12\pend
           \vspace{0.5em}
\pstart
           Herzlichen Dank, lieber Artur, für Deine Karte und die eben
               einlangenden \label{K_L02092-1v}\edtext{vier Bände Theater\pwindex{Schnitzler, Arthur 15.\,5.\,1862 Wien – 21.\,10.\,1931 ebd.@\textsc{Schnitzler, Arthur} (15.\,5.\,1862 Wien – 21.\,10.\,1931 ebd.), \emph{Schriftsteller, Mediziner}!Gesammelte Werke@\strich\emph{Gesammelte Werke}|pwv}}{\lemma{\textnormal{\emph{vier Bände Theater}}}\Cendnote{\textnormal{Die \emph{Gesammelten Werke}\pwindex{Schnitzler, Arthur 15.\,5.\,1862 Wien – 21.\,10.\,1931 ebd.@\textsc{Schnitzler, Arthur} (15.\,5.\,1862 Wien – 21.\,10.\,1931 ebd.), \emph{Schriftsteller, Mediziner}!Gesammelte Werke@\strich\emph{Gesammelte Werke}|pwk} erschienen 1912 bei \emph{S. Fischer}\orgindex{S. Fischer Verlag@S. Fischer Verlag|pwk} in sieben Bänden, vier enthielten die \emph{Theaterstücke}\pwindex{Schnitzler, Arthur 15.\,5.\,1862 Wien – 21.\,10.\,1931 ebd.@\textsc{Schnitzler, Arthur} (15.\,5.\,1862 Wien – 21.\,10.\,1931 ebd.), \emph{Schriftsteller, Mediziner}!Theaterstücke@\strich\emph{Die Theaterstücke}|pwk}, drei die \emph{Erzählenden Schriften}\pwindex{Schnitzler, Arthur 15.\,5.\,1862 Wien – 21.\,10.\,1931 ebd.@\textsc{Schnitzler, Arthur} (15.\,5.\,1862 Wien – 21.\,10.\,1931 ebd.), \emph{Schriftsteller, Mediziner}!Erzählende Schriften@\strich\emph{Erzählende Schriften}|pwk}.}}}\label{K_L02092-1}. Ich \label{K_L02092-2v}\edtext{vagabundiere durch die Welt}{\lemma{\textnormal{\emph{vagabundiere … Welt}}}\Cendnote{\textnormal{Die Vortragstourneen in Deutschland\oindex{Deutschland@\textbf{Deutschland}|pwk} fanden vom 4. 11. 1912 bis zum
                     27. 11. 1912 und vom 13. 12. 1912 bis zum
                     30. 1. 1913 statt. Ob er den für Dezember geplanten
                  Aufenthalt in Rom\oindex{Rom@\textbf{Rom}, \emph{Hauptstadt}|pwk} verwirklichte, ist nicht
                  geklärt.}}}\label{K_L02092-2} (zunächſt von hier nach Deutſchland\oindex{Deutschland@\textbf{Deutschland}|pw},{ }ſechs Wochen Vorleſungen, dann nach Rom\oindex{Rom@\textbf{Rom}, \emph{Hauptstadt}|pw}, Januar und Februar wieder Vorleſungen), bis ich am
                  1. März in Salzburg,
                  Arenbergſchloß\oindex{Schloss Arenberg@\textbf{Schloss Arenberg}, \emph{Schloss}|pw} zu landen hoffe.\pend
           \pstart Herzlichſt immer Dein \spacefill\mbox{Hermann}\pend{}
\pstart
           \noindent{}Schönſte Grüße an Frau Olga\pwindex{Schnitzler, Olga 17.\,1.\,1882 Wien – 13.\,1.\,1970 Lugano@\textsc{Schnitzler, Olga} (17.\,1.\,1882 Wien – 13.\,1.\,1970 Lugano), \emph{Schauspielerin, Sängerin}|pw}!\pend
           \selectlanguage{ngerman}\endnumbering\briefempfaengerindex{Schnitzler, Arthur@\textsc{Schnitzler, Arthur}!zzzBahr, Hermann@\emph{von Hermann Bahr}!1912-09-281@{28. 9. 1912}|)be}\mylabel{L02092h}  \newcommand{\dateiname}{L02092}\newcommand{\titel}{Hermann Bahr an Arthur Schnitzler, 28. 9. 1912}\newcommand{\editorInnen}{Herausgegeben von Martin Anton Müller}%% latex-leseansicht-abspann.tex
%% Abspann für die Leseansicht.
%% Der Schalter \ifkorrekturansicht ist bereits durch den Vorspann gesetzt.

%% latex-abspann.tex
%% Gemeinsamer Abspann für Korrekturansicht und Leseansicht.
%% Setzt den Schalter \ifkorrekturansicht voraus (gesetzt in den
%% einbindenden Dateien latex-korrekturansicht-abspann.tex bzw.
%% latex-leseansicht-abspann.tex).
%% ---------------------------------------------------------------

\normalsize

% Das esempio-Environment wird nur in der Leseansicht benötigt
\ifkorrekturansicht\else
\newenvironment{esempio}[3]%
{
    \vspace{1.5ex}
    \rlap{\underline{#1}}
    \par
    \setlength{\parindent}{0cm}
    \nopagebreak
    \leftskip=#2cm
    \rightskip=#3cm
}
{
    \par
}
\fi

\doendnotes{C}
\bigskip
\vfill

\clearpage

\footnotesize

\ifkorrekturansicht
  \lohead{\textsc{register}}
\fi

% theindex-Environment neu definieren ohne reledmac
\makeatletter
\renewenvironment{theindex}{%
  \ifkorrekturansicht
    \section*{\indexname}%
  \else
    \subsubsection*{Index der erwähnten Entitäten}%
  \fi
  \setlength{\parindent}{0pt}%
  \setlength{\parskip}{0pt plus 0.3pt}%
  \let\item\@idxitem
}{%
  \ifkorrekturansicht\clearpage\fi
}
\makeatother

\IfFileExists{\jobname-pw.ind}{\input{\jobname-pw.ind}}{}

% Quellenangabe nur in der Leseansicht
\ifkorrekturansicht\else
% Fallback-Definitionen, falls die .tex-Datei \titel etc. nicht gesetzt hat
\providecommand{\titel}{}
\providecommand{\editorInnen}{}
\providecommand{\dateiname}{\jobname}

\vspace{3cm}

\vfill

\footnotesize
\textsc{Quelle}: \titel. Herausgegeben von {\editorInnen}. In: \emph{Arthur Schnitzler: Briefwechsel mit Autorinnen und Autoren}.
 Digitale Edition, https://schnitzler-briefe.acdh.oeaw.ac.at/{\dateiname}.html (Stand \today)
\fi

\end{document}


