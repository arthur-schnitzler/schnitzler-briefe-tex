%% latex-korrekturansicht-vorspann.tex
%% Vorspann für die Korrekturansicht.
%% Lädt die gemeinsame Datei latex-vorspann.tex mit gesetztem Schalter.

\newif\ifkorrekturansicht
\korrekturansichttrue

\input{../tex-inputs/latex-vorspann}


\section[Hermann Bahr an Arthur Schnitzler, 28. 9. 1912]{L02092 Hermann Bahr an Arthur Schnitzler, 28. 9. 1912}
\nopagebreak\mylabel{L02092v}
\rehead{ }\normalsize\beginnumbering\briefempfaengerindex{Schnitzler, Arthur@\textsc{Schnitzler, Arthur}!zzzBahr, Hermann@\emph{von Hermann Bahr}!1912-09-281@{28. 9. 1912}|(be}
\toendnotes[C]{\smallbreak\pagebreak[2]}\Standort{CUL, Schnitzler, B 5b.}
\physDesc{Bildpostkarte, 437 Zeichen
\newline{}Handschrift: schwarze Tinte, deutsche Kurrent
\newline{}Versand: Stempel: »\nobreak{}\oindex{Semmering@\textbf{Semmering}, \emph{A.ADM3}|pwk}Semmering, 29. IX. 12, XII\nobreak{}«.  
\newline{}Ordnung: mit Bleistift von unbekannter Hand nummeriert:
                                    »174« }
\buchAbdrucke{\weitereDrucke{Hermann Bahr, Arthur Schnitzler: \emph{Briefwechsel, Aufzeichnungen, Dokumente (1891–1931)}. Göttingen: \emph{Wallstein} 2018, S. 478.} }\toendnotes[C]{\smallbreak}\pstart{}{\pb}Herrn \textsc{D\textsuperscript{r} Artur Schnitzler}\pend{}\pstart{}XVIII Sternwarteſtr 71\oindex{Sternwartestrasse 71@\textbf{Sternwartestraße 71}, \emph{Wohngebäude (K.WHS)}|pw}\pend{}\pstart{}\textsc{Wien XVIII\oindex{XVIII., Waehring@\textbf{XVIII., Währing}, \emph{A.ADM3}|pw}}\pend{}{\bigskip}
\pstart
           \noindent{}\centering{}{\pb}\textcolor{gray}{\textbf{Venezia – Cortile Palazzo Ducale\oindex{Palazzo Ducale@\textbf{Palazzo Ducale}, \emph{Gebäude (K.GBD)}|pw}}}\pend
           
\pstart
           {\pb}Semmering Villa Mautner\oindex{Villa Mauthner-Markhof@\textbf{Villa Mauthner-Markhof}, \emph{Wohngebäude (K.WHS)}|pw}\pend
           \vspace{1em}
\pstart
           \raggedleft{}{\pb}28. 9. 12\pend
           \vspace{0.5em}
\pstart
           Herzlichen Dank, lieber Artur, für Deine Karte und die eben
               einlangenden \label{K_L02092-1v}\edtext{vier Bände Theater\pwindex{Gesammelte Werke@\emph{Gesammelte Werke}|pwv}}{\lemma{\textnormal{\emph{vier Bände Theater}}}\Cendnote{\textnormal{Die \emph{Gesammelten Werke}\pwindex{Gesammelte Werke@\emph{Gesammelte Werke}|pwk} erschienen 1912 bei \emph{S. Fischer}\orgindex{S. Fischer Verlag@S. Fischer Verlag|pwk} in sieben Bänden, vier enthielten die \emph{Theaterstücke}\pwindex{Theaterstuecke@\emph{Die Theaterstücke}|pwk}, drei die \emph{Erzählenden Schriften}\pwindex{Erzaehlende Schriften@\emph{Erzählende Schriften}|pwk}.}}}\label{K_L02092-1}. Ich \label{K_L02092-2v}\edtext{vagabundiere durch die Welt}{\lemma{\textnormal{\emph{vagabundiere … Welt}}}\Cendnote{\textnormal{Die Vortragstourneen in Deutschland\oindex{Deutschland@\textbf{Deutschland}, \emph{A.PCLI}|pwk} fanden vom 4. 11. 1912 bis zum
                     27. 11. 1912 und vom 13. 12. 1912 bis zum
                     30. 1. 1913 statt. Ob er den für Dezember geplanten
                  Aufenthalt in Rom\oindex{Rom@\textbf{Rom}, \emph{P.PPLC}|pwk} verwirklichte, ist nicht
                  geklärt.}}}\label{K_L02092-2} (zunächſt von hier nach Deutſchland\oindex{Deutschland@\textbf{Deutschland}, \emph{A.PCLI}|pw}, ſechs Wochen Vorleſungen, dann nach Rom\oindex{Rom@\textbf{Rom}, \emph{P.PPLC}|pw}, Januar und Februar wieder Vorleſungen), bis ich am
                  1. März in Salzburg,
                  Arenbergſchloß\oindex{Schloss Arenberg@\textbf{Schloss Arenberg}, \emph{Schloss (K.SLS)}|pw} zu landen hoffe.\pend
           \pstart Herzlichſt immer Dein \spacefill\mbox{Hermann}\pend{}
\pstart
           \noindent{}Schönſte Grüße an Frau Olga\pwindex{Schnitzler, Olga 17.01.1882 – 13.01.1970@\textsc{Schnitzler, Olga} (17.01.1882 – 13.01.1970), \emph{Schauspieler/Schauspielerin, Sänger/Sängerin}|pw}!\pend
           \selectlanguage{ngerman}\endnumbering\briefempfaengerindex{Schnitzler, Arthur@\textsc{Schnitzler, Arthur}!zzzBahr, Hermann@\emph{von Hermann Bahr}!1912-09-281@{28. 9. 1912}|)be}\mylabel{L02092h}  \normalsize

\doendnotes{C}
\bigskip
\vfill

\clearpage

\footnotesize

\lohead{\textsc{register}}

% Definiere theindex-Environment komplett neu ohne reledmac
\makeatletter
\renewenvironment{theindex}{%
  \section*{\indexname}%
  \setlength{\parindent}{0pt}%
  \setlength{\parskip}{0pt plus 0.3pt}%
  \let\item\@idxitem
}{%
  \clearpage
}
\makeatother

\IfFileExists{\jobname-pw.ind}{\input{\jobname-pw.ind}}{}

\end{document}

      