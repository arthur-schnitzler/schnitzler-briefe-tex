%% latex-leseansicht-vorspann.tex
%% Vorspann für die Leseansicht.
%% Lädt die gemeinsame Datei latex-vorspann.tex mit nicht gesetztem Schalter.

\newif\ifkorrekturansicht
\korrekturansichtfalse

\input{../tex-inputs/latex-vorspann}


         
         \renewcommand{\erwaehntePersonen}{Personen: Richard Beer-Hofmann, Alexander Engel, Leo Feld, Victor Léon, Gustav Schwarzkopf, Max Schwarzkopf, Emil Schwarzkopf}
         \renewcommand{\erwaehnteInstitutionen}{Institutionen: Deutsches Theater Berlin}
         \renewcommand{\erwaehnteOrte}{Orte: Berlin, Wien}
         \renewcommand{\erwaehnteWerke}{Werke: Freiwild. Schauspiel in 3 Akten}
               \section[Richard Beer-Hofmann an Arthur Schnitzler, {[}4. 11. 1896{]}]{ Richard Beer-Hofmann an Arthur Schnitzler, {[}4. 11. 1896{]}}\nopagebreak\mylabel{v}\rehead{ }\begin{ledgroupsized}[t]{13cm}\normalsize\beginnumbering\briefempfaengerindex{Schnitzler, Arthur@\textsc{Schnitzler, Arthur}!zzzBeer-Hofmann, Richard@\emph{von Richard Beer-Hofmann}!1896-11-041@{{[}4. 11. 1896{]}}|(be} \toendnotes[C]{\smallbreak\pagebreak[2]} \Standort{CUL, Schnitzler, B 8.}
\physDesc{Telegramm, 106 Zeichen
\newline{}maschinell
\newline{}Ordnung: mit Bleistift von unbekannter Hand
                                    nummeriert: »87« }\buchAbdrucke{\weitereDrucke{Arthur Schnitzler, Richard Beer-Hofmann: \emph{Briefwechsel 1891–1931}. Hg. Konstanze Fliedl. Wien, Zürich: \emph{Europaverlag} 1992, S. 99.} }\toendnotes[C]{\smallbreak}\pstart
           {\pb}\textcolor{gray}{b}\oindex{Berlin@\textbf{Berlin}|pw}{ }fr{ }\textcolor{gray}{wien}\oindex{Wien@\textbf{Wien}|pw}{ }502{ }21{ }12/50=\pend
           \pstart
           ich freue mich sehr. es \label{K_L00615-1v}\edtext{gratuliren}{\lemma{\textnormal{\emph{gratuliren}}}\Cendnote{\textnormal{Die Gratulation erfolgte anlässlich der Uraufführung von \emph{Freiwild}\pwindex{Schnitzler, Arthur 15.05.1862 – 21.10.1931@\textsc{Schnitzler, Arthur} (15.05.1862 – 21.10.1931), \emph{Schriftsteller, Mediziner}!Freiwild. Schauspiel in 3 Akten1896@\strich\emph{Freiwild. Schauspiel in 3 Akten} {[}1896{]}|pwk} am 3. 11. 1896 am \emph{Deutschen Theater}\orgindex{Deutsches Theater Berlin@Deutsches Theater Berlin|pwk} in Berlin\oindex{Berlin@\textbf{Berlin}|pwk}.
               }}}\label{K_L00615-1h}{ }schwarze koepfe\pwindex{Schwarzkopf, Gustav 07.11.1853 – 13.11.1939@\textsc{Schwarzkopf, Gustav} (07.11.1853 – 13.11.1939), \emph{Schriftsteller}|pw}\pwindex{Schwarzkopf, Max 12.06.1857 – 14.04.1928@\textsc{Schwarzkopf, Max} (12.06.1857 – 14.04.1928), \emph{Rechtsanwalt}|pw}\pwindex{Schwarzkopf, Emil 17.09.1851 – 28.01.1928@\textsc{Schwarzkopf, Emil} (17.09.1851 – 28.01.1928), \emph{Übersetzer, Komponist}|pw}, leon\pwindex{Leon, Victor 4.1.1858 – 23.2.1940@\textsc{Léon, Victor} (4.1.1858 – 23.2.1940), \emph{Schriftsteller, Dramaturg}|pw} und bruder\pwindex{Feld, Leo 14.02.1869 – 05.09.1924@\textsc{Feld, Leo} (14.02.1869 – 05.09.1924), \emph{Schriftsteller, Übersetzer, Dirigent}|pwv} doktor engl\pwindex{Engel, Alexander 10.04.1868 – 17.11.1940@\textsc{Engel, Alexander} (10.04.1868 – 17.11.1940), \emph{Schriftsteller, Journalist}|pwu} = \spacefill\mbox{richard +}\pend
           
         
         \endnumbering\mylabel{h}\end{ledgroupsized}  \newcommand{\dateiname}{L00615}\newcommand{\titel}{Richard Beer-Hofmann an Arthur Schnitzler, [4. 11. 1896]}\newcommand{\editorInnen}{Martin Anton Müller und Gerd-Hermann Susen}%% latex-leseansicht-abspann.tex
%% Abspann für die Leseansicht.
%% Der Schalter \ifkorrekturansicht ist bereits durch den Vorspann gesetzt.

%% latex-abspann.tex
%% Gemeinsamer Abspann für Korrekturansicht und Leseansicht.
%% Setzt den Schalter \ifkorrekturansicht voraus (gesetzt in den
%% einbindenden Dateien latex-korrekturansicht-abspann.tex bzw.
%% latex-leseansicht-abspann.tex).
%% ---------------------------------------------------------------

\normalsize

% Das esempio-Environment wird nur in der Leseansicht benötigt
\ifkorrekturansicht\else
\newenvironment{esempio}[3]%
{
    \vspace{1.5ex}
    \rlap{\underline{#1}}
    \par
    \setlength{\parindent}{0cm}
    \nopagebreak
    \leftskip=#2cm
    \rightskip=#3cm
}
{
    \par
}
\fi

\doendnotes{C}
\bigskip
\vfill

\clearpage

\footnotesize

\ifkorrekturansicht
  \lohead{\textsc{register}}
\fi

% theindex-Environment neu definieren ohne reledmac
\makeatletter
\renewenvironment{theindex}{%
  \ifkorrekturansicht
    \section*{\indexname}%
  \else
    \subsubsection*{Index der erwähnten Entitäten}%
  \fi
  \setlength{\parindent}{0pt}%
  \setlength{\parskip}{0pt plus 0.3pt}%
  \let\item\@idxitem
}{%
  \ifkorrekturansicht\clearpage\fi
}
\makeatother

\IfFileExists{\jobname-pw.ind}{\input{\jobname-pw.ind}}{}

% Quellenangabe nur in der Leseansicht
\ifkorrekturansicht\else
% Fallback-Definitionen, falls die .tex-Datei \titel etc. nicht gesetzt hat
\providecommand{\titel}{}
\providecommand{\editorInnen}{}
\providecommand{\dateiname}{\jobname}

\vspace{3cm}

\vfill

\footnotesize
\textsc{Quelle}: \titel. Herausgegeben von {\editorInnen}. In: \emph{Arthur Schnitzler: Briefwechsel mit Autorinnen und Autoren}.
 Digitale Edition, https://schnitzler-briefe.acdh.oeaw.ac.at/{\dateiname}.html (Stand \today)
\fi

\end{document}


      