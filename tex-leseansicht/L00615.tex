%% latex-leseansicht-vorspann.tex
%% Vorspann für die Leseansicht.
%% Lädt die gemeinsame Datei latex-vorspann.tex mit nicht gesetztem Schalter.

\newif\ifkorrekturansicht
\korrekturansichtfalse

\input{../tex-inputs/latex-vorspann}


\section[Richard Beer-Hofmann an Arthur Schnitzler, {[}4. 11. 1896{]}]{L00615 Richard Beer-Hofmann an Arthur Schnitzler, {[}4. 11. 1896{]}}
\nopagebreak\mylabel{L00615v}
\rehead{ }\normalsize\beginnumbering\briefempfaengerindex{Schnitzler, Arthur@\textsc{Schnitzler, Arthur}!zzzBeer-Hofmann, Richard@\emph{von Richard Beer-Hofmann}!1896-11-041@{{[}4. 11. 1896{]}}|(be}
\toendnotes[C]{\smallbreak\pagebreak[2]}
\correspDesc{Versand  durch Richard Beer-Hofmann am [4. 11. 1896] in Wien
\newline{}Erhalt  durch Arthur Schnitzler am [4. 11. 1896] in Berlin}\toendnotes[C]{\smallbreak}
\Standort{CUL, Schnitzler, B 8.}
\physDesc{Telegramm, 106 Zeichen
\newline{}maschinell
\newline{}Ordnung: mit Bleistift von unbekannter Hand
                                    nummeriert: »87« }
\buchAbdrucke{\weitereDrucke{Arthur Schnitzler, Richard Beer-Hofmann: \emph{Briefwechsel 1891–1931}. Herausgegeben von Konstanze Fliedl. Wien, Zürich: \emph{Europaverlag} 1992, S. 99.} }\toendnotes[C]{\smallbreak}
\pstart
           {\pb}\textcolor{gray}{b}\oindex{Berlin@\textbf{Berlin}, \emph{Hauptstadt}|pw}{ }fr{ }\textcolor{gray}{wien}\oindex{Wien@\textbf{Wien}, \emph{Verwaltungsgebiet}|pw}{ }502{ }21{ }12/50=\pend
           \vspace{0.5em}
\pstart
           ich freue mich sehr. es \label{K_L00615-1v}\edtext{gratuliren}{\lemma{\textnormal{\emph{gratuliren}}}\Cendnote{\textnormal{Die Gratulation erfolgte anlässlich der Uraufführung\eventindex{Deutsches Theater Berlin@\textbf{Deutsches Theater Berlin}!Uraufführung von Freiwild, 3.11.1896@Uraufführung von Freiwild, 3.11.1896|pwkv} von \emph{Freiwild}\pwindex{Schnitzler, Arthur 15.\,5.\,1862 Wien – 21.\,10.\,1931 ebd.@\textsc{Schnitzler, Arthur} (15.\,5.\,1862 Wien – 21.\,10.\,1931 ebd.), \emph{Schriftsteller, Mediziner}!Freiwild. Schauspiel in 3 Akten@\strich\emph{Freiwild. Schauspiel in 3 Akten}|pwk} am 3. 11. 1896 am \emph{Deutschen Theater}\orgindex{Deutsches Theater Berlin@Deutsches Theater Berlin|pwk} in Berlin\oindex{Berlin@\textbf{Berlin}, \emph{Hauptstadt}|pwk}.
               }}}\label{K_L00615-1}{ }schwarze koepfe\pwindex{Schwarzkopf, Gustav 7.\,11.\,1853 Wien – 13.\,11.\,1939 ebd.@\textsc{Schwarzkopf, Gustav} (7.\,11.\,1853 Wien – 13.\,11.\,1939 ebd.), \emph{Schriftsteller}|pw}\pwindex{Schwarzkopf, Max 12.\,6.\,1857 Wien – 14.\,4.\,1928 ebd.@\textsc{Schwarzkopf, Max} (12.\,6.\,1857 Wien – 14.\,4.\,1928 ebd.), \emph{Rechtsanwalt}|pw}\pwindex{Schwarzkopf, Emil 17.\,9.\,1851 Wien – 28.\,1.\,1928 ebd.@\textsc{Schwarzkopf, Emil} (17.\,9.\,1851 Wien – 28.\,1.\,1928 ebd.), \emph{Übersetzer, Komponist, Musiklehrer}|pw}, leon\pwindex{Léon, Victor 4.\,1.\,1858 Senica – 23.\,2.\,1940 Wien@\textsc{Léon, Victor} (4.\,1.\,1858 Senica – 23.\,2.\,1940 Wien), \emph{Schriftsteller, Dramaturg}|pw} und bruder\pwindex{Feld, Leo 14.\,2.\,1869 Augsburg – 5.\,9.\,1924 Florenz@\textsc{Feld, Leo} (14.\,2.\,1869 Augsburg – 5.\,9.\,1924 Florenz), \emph{Schriftsteller, Übersetzer, Dirigent}|pwv} doktor engl\pwindex{Engel, Alexander 10.\,4.\,1868 Necpaly – 17.\,11.\,1940 Wien@\textsc{Engel, Alexander} (10.\,4.\,1868 Necpaly – 17.\,11.\,1940 Wien), \emph{Schriftsteller, Journalist}|pwu} = \spacefill\mbox{richard +}\pend
           \selectlanguage{ngerman}\endnumbering\briefempfaengerindex{Schnitzler, Arthur@\textsc{Schnitzler, Arthur}!zzzBeer-Hofmann, Richard@\emph{von Richard Beer-Hofmann}!1896-11-041@{{[}4. 11. 1896{]}}|)be}\mylabel{L00615h}  \newcommand{\dateiname}{L00615}\newcommand{\titel}{Richard Beer-Hofmann an Arthur Schnitzler, [4. 11. 1896]}\newcommand{\editorInnen}{Martin Anton Müller und Gerd-Hermann Susen}%% latex-leseansicht-abspann.tex
%% Abspann für die Leseansicht.
%% Der Schalter \ifkorrekturansicht ist bereits durch den Vorspann gesetzt.

%% latex-abspann.tex
%% Gemeinsamer Abspann für Korrekturansicht und Leseansicht.
%% Setzt den Schalter \ifkorrekturansicht voraus (gesetzt in den
%% einbindenden Dateien latex-korrekturansicht-abspann.tex bzw.
%% latex-leseansicht-abspann.tex).
%% ---------------------------------------------------------------

\normalsize

% Das esempio-Environment wird nur in der Leseansicht benötigt
\ifkorrekturansicht\else
\newenvironment{esempio}[3]%
{
    \vspace{1.5ex}
    \rlap{\underline{#1}}
    \par
    \setlength{\parindent}{0cm}
    \nopagebreak
    \leftskip=#2cm
    \rightskip=#3cm
}
{
    \par
}
\fi

\doendnotes{C}
\bigskip
\vfill

\clearpage

\footnotesize

\ifkorrekturansicht
  \lohead{\textsc{register}}
\fi

% theindex-Environment neu definieren ohne reledmac
\makeatletter
\renewenvironment{theindex}{%
  \ifkorrekturansicht
    \section*{\indexname}%
  \else
    \subsubsection*{Index der erwähnten Entitäten}%
  \fi
  \setlength{\parindent}{0pt}%
  \setlength{\parskip}{0pt plus 0.3pt}%
  \let\item\@idxitem
}{%
  \ifkorrekturansicht\clearpage\fi
}
\makeatother

\IfFileExists{\jobname-pw.ind}{\input{\jobname-pw.ind}}{}

% Quellenangabe nur in der Leseansicht
\ifkorrekturansicht\else
% Fallback-Definitionen, falls die .tex-Datei \titel etc. nicht gesetzt hat
\providecommand{\titel}{}
\providecommand{\editorInnen}{}
\providecommand{\dateiname}{\jobname}

\vspace{3cm}

\vfill

\footnotesize
\textsc{Quelle}: \titel. Herausgegeben von {\editorInnen}. In: \emph{Arthur Schnitzler: Briefwechsel mit Autorinnen und Autoren}.
 Digitale Edition, https://schnitzler-briefe.acdh.oeaw.ac.at/{\dateiname}.html (Stand \today)
\fi

\end{document}


