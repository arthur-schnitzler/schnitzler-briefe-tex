%% latex-korrekturansicht-vorspann.tex
%% Vorspann für die Korrekturansicht.
%% Lädt die gemeinsame Datei latex-vorspann.tex mit gesetztem Schalter.

\newif\ifkorrekturansicht
\korrekturansichttrue

\input{../tex-inputs/latex-vorspann}


\section[Gerhart Hauptmann an Arthur Schnitzler, 24. 1. 1905]{L01494 Gerhart Hauptmann an Arthur Schnitzler, 24. 1. 1905}
\nopagebreak\mylabel{L01494v}
\rehead{ }\normalsize\beginnumbering\briefempfaengerindex{Schnitzler, Arthur@\textsc{Schnitzler, Arthur}!zzzHauptmann, Gerhart@\emph{von Gerhart Hauptmann}!1905-01-241@{24. 1. 1905}|(be}
\toendnotes[C]{\smallbreak\pagebreak[2]}\Standort{CUL, Schnitzler, B 36.}
\physDesc{Brief, 1 Blatt, 3 Seiten, 930 Zeichen
\newline{}Handschrift: schwarze Tinte, lateinische Kurrent}\toendnotes[C]{\smallbreak}
\pstart{}{\pb}Lieber Herr Schnitzler.\pend\vspace{0.5em}
\pstart
           Ich war in den Berlin\oindex{Berlin@\textbf{Berlin}, \emph{P.PPLC}|pw}er Trubel gerathen, sonst
               hätte ich Ihnen gleich geantwortet und gedankt, für das Gute und Herzliche, was Sie
               mir erwiesen haben, durch Ihren Brief. So sehr wir geneigt sein mögen, eine erfahrene
               Auszeichnung nicht als unverdient zu erachten, so sehr bin ich mir doch auch der
               Verdienste bewusst, die Sie, verehrter Herr Schnitzler, und andere gleichstrebende
               deutsche Dichter in Oesterreich\oindex{Oesterreich@\textbf{Österreich}, \emph{A.PCLI}|pw}, haben: und es
               fällt mir nicht ein, sie geringer anzuschlagen, als die Meinen.\pend
           
\pstart
           Ich sage es, obgleich ich {\pb}annehme, Sie
               wissen das ungesagt. Und ich wünschte auch nichts sehnlicher, als fortan eine schöne
               Reihe von Gratulationen nach Wien\oindex{Wien@\textbf{Wien}, \emph{A.ADM2}|pw} richten zu
               können. Wahrhaftig! Wenn ich an \label{K_L01494-1v}\edtext{Preise\orgindex{Franz-Grillparzer-Preis@Franz-Grillparzer-Preis|pw}}{\lemma{\textnormal{\emph{Preise}}}\Cendnote{\textnormal{Die Zuerkennung des \emph{Grillparzer-Preises}\orgindex{Franz-Grillparzer-Preis@Franz-Grillparzer-Preis|pwk} für \emph{Der
                     arme Heinrich}\pwindex{arme Heinrich – Eine deutsche Sage@\emph{Der arme Heinrich – Eine deutsche Sage}|pwk} wurde Mitte Januar 1905 bekannt gegeben.}}}\label{K_L01494-1}
               überhaupt gedacht hätte, so würde ich es schon früher gewusst haben. Seien Sie
               vielmals gegrüsst! Alles Glück für Leben und Wirken und auf gesundes Wiedersehen!\pend
           
\pstart
           Herzlich{\\[\baselineskip]} Ihr{\\[\baselineskip]}\spacefill\mbox{Gerhart Hauptmann}\pend
           \leftskip=0em{}
\pstart
           \noindent{}{\pb}Agnetendorf\oindex{Jagniątków@\textbf{Jagniątków}, \emph{P.PPL}|pw}\pend
           
\pstart
           d 24.{\\}Januar{\\}1905.\pend
           \selectlanguage{ngerman}\endnumbering\briefempfaengerindex{Schnitzler, Arthur@\textsc{Schnitzler, Arthur}!zzzHauptmann, Gerhart@\emph{von Gerhart Hauptmann}!1905-01-241@{24. 1. 1905}|)be}\mylabel{L01494h}  \normalsize

\doendnotes{C}
\bigskip
\vfill

\clearpage

\footnotesize

\lohead{\textsc{register}}

% Definiere theindex-Environment komplett neu ohne reledmac
\makeatletter
\renewenvironment{theindex}{%
  \section*{\indexname}%
  \setlength{\parindent}{0pt}%
  \setlength{\parskip}{0pt plus 0.3pt}%
  \let\item\@idxitem
}{%
  \clearpage
}
\makeatother

\IfFileExists{\jobname-pw.ind}{\input{\jobname-pw.ind}}{}

\end{document}

      