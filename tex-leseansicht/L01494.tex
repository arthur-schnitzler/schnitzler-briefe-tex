%% latex-leseansicht-vorspann.tex
%% Vorspann für die Leseansicht.
%% Lädt die gemeinsame Datei latex-vorspann.tex mit nicht gesetztem Schalter.

\newif\ifkorrekturansicht
\korrekturansichtfalse

\input{../tex-inputs/latex-vorspann}


         \newcommand{\erwaehnteInstitutionen}{Institutionen: Franz-Grillparzer-Preis}
         \newcommand{\erwaehnteOrte}{Orte: Agnetendorf, Berlin, Wien, Österreich}
         \newcommand{\erwaehnteWerke}{Werke: Der arme Heinrich – Eine deutsche Sage}
               \section[Gerhart Hauptmann an Arthur Schnitzler, 24. 1. 1905]{ Gerhart Hauptmann an Arthur Schnitzler, 24. 1. 1905}\nopagebreak\mylabel{v}\rehead{ }\begin{ledgroupsized}[t]{13cm}\normalsize\beginnumbering \toendnotes[C]{\smallbreak\pagebreak[2]} \Standort{CUL, Schnitzler, B 36.}
\physDesc{Brief, 1 Blatt, 3 Seiten
\newline{}Handschrift: schwarze Tinte, lateinische Kurrent}\toendnotes[C]{\smallbreak}\pstart{}{\pb}Lieber Herr Schnitzler.\pend\pstart
           Ich war in den Berlin\oindex{Berlin@\textbf{Berlin}|pw}er Trubel gerathen, sonst
                    hätte ich Ihnen gleich geantwortet und gedankt, für das Gute und Herzliche, was
                    Sie mir erwiesen haben, durch Ihren Brief. So sehr wir geneigt sein mögen, eine
                    erfahrene Auszeichnung nicht als unverdient zu erachten, so sehr bin ich mir
                    doch auch der Verdienste bewusst, die Sie, verehrter Herr Schnitzler, und andere
                    gleichstrebende deutsche Dichter in Oesterreich\oindex{Oesterreich@\textbf{Österreich}|pw}, haben: und es fällt mir nicht ein, sie geringer
                    anzuschlagen, als die Meinen.\pend
           \pstart
           Ich sage es, obgleich ich {\pb}annehme, Sie wissen das ungesagt.
                    Und ich wünschte auch nichts sehnlicher, als fortan eine schöne Reihe von
                    Gratulationen nach Wien\oindex{Wien@\textbf{Wien}|pw} richten zu können.
                    Wahrhaftig! Wenn ich an \label{K_L01494_1v}\edtext{Preise\orgindex{Franz-Grillparzer-Preis@Franz-Grillparzer-Preis|pw}}{\lemma{\textnormal{\emph{Preise}}}\Cendnote{\textnormal{Die Zuerkennung des \emph{Grillparzer-Preises}\orgindex{Franz-Grillparzer-Preis@Franz-Grillparzer-Preis|pwk} für \emph{Der
                            arme Heinrich}\pwindex{Hauptmann, Gerhart 15.11.1862 – 06.06.1946@\textsc{Hauptmann, Gerhart} (15.11.1862 – 06.06.1946), \emph{Schriftsteller}!arme Heinrich – Eine deutsche Sage29. 11. 1902@\strich\emph{Der arme Heinrich – Eine deutsche Sage} {[}29. 11. 1902{]}|pwk} wurde Mitte Januar 1905 bekannt
                        gegeben.}}}\label{K_L01494_1h} überhaupt gedacht hätte, so würde ich es schon früher
                    gewusst haben. Seien Sie vielmals gegrüsst! Alles Glück für Leben und Wirken und
                    auf gesundes Wiedersehen!\pend
           \pstart
           Herzlich{\\[\baselineskip]} Ihr{\\[\baselineskip]}\spacefill\mbox{Gerhart Hauptmann}\pend
           \leftskip=0em{}\pstart
           \noindent{}{\pb}Agnetendorf\oindex{Agnetendorf@\textbf{Agnetendorf}|pw}\pend
           \pstart
           d 24.{\\}Januar{\\}1905.\pend
           
         
         \endnumbering\mylabel{h}\end{ledgroupsized}  \newcommand{\dateiname}{L01494}\newcommand{\titel}{Gerhart Hauptmann an Arthur Schnitzler, 24. 1. 1905}\newcommand{\editorInnen}{Martin Anton Müller und Gerd-Hermann Susen}%% latex-leseansicht-abspann.tex
%% Abspann für die Leseansicht.
%% Der Schalter \ifkorrekturansicht ist bereits durch den Vorspann gesetzt.

%% latex-abspann.tex
%% Gemeinsamer Abspann für Korrekturansicht und Leseansicht.
%% Setzt den Schalter \ifkorrekturansicht voraus (gesetzt in den
%% einbindenden Dateien latex-korrekturansicht-abspann.tex bzw.
%% latex-leseansicht-abspann.tex).
%% ---------------------------------------------------------------

\normalsize

% Das esempio-Environment wird nur in der Leseansicht benötigt
\ifkorrekturansicht\else
\newenvironment{esempio}[3]%
{
    \vspace{1.5ex}
    \rlap{\underline{#1}}
    \par
    \setlength{\parindent}{0cm}
    \nopagebreak
    \leftskip=#2cm
    \rightskip=#3cm
}
{
    \par
}
\fi

\doendnotes{C}
\bigskip
\vfill

\clearpage

\footnotesize

\ifkorrekturansicht
  \lohead{\textsc{register}}
\fi

% theindex-Environment neu definieren ohne reledmac
\makeatletter
\renewenvironment{theindex}{%
  \ifkorrekturansicht
    \section*{\indexname}%
  \else
    \subsubsection*{Index der erwähnten Entitäten}%
  \fi
  \setlength{\parindent}{0pt}%
  \setlength{\parskip}{0pt plus 0.3pt}%
  \let\item\@idxitem
}{%
  \ifkorrekturansicht\clearpage\fi
}
\makeatother

\IfFileExists{\jobname-pw.ind}{\input{\jobname-pw.ind}}{}

% Quellenangabe nur in der Leseansicht
\ifkorrekturansicht\else
% Fallback-Definitionen, falls die .tex-Datei \titel etc. nicht gesetzt hat
\providecommand{\titel}{}
\providecommand{\editorInnen}{}
\providecommand{\dateiname}{\jobname}

\vspace{3cm}

\vfill

\footnotesize
\textsc{Quelle}: \titel. Herausgegeben von {\editorInnen}. In: \emph{Arthur Schnitzler: Briefwechsel mit Autorinnen und Autoren}.
 Digitale Edition, https://schnitzler-briefe.acdh.oeaw.ac.at/{\dateiname}.html (Stand \today)
\fi

\end{document}


      