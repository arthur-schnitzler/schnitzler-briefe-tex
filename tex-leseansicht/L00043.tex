%% latex-korrekturansicht-vorspann.tex
%% Vorspann für die Korrekturansicht.
%% Lädt die gemeinsame Datei latex-vorspann.tex mit gesetztem Schalter.

\newif\ifkorrekturansicht
\korrekturansichttrue

\input{../tex-inputs/latex-vorspann}


\section[Arthur Schnitzler an Richard Beer-Hofmann, {[}zwischen 7. 10. 1891 und Ende April 1892{]}]{L00043 Arthur Schnitzler an Richard Beer-Hofmann, {[}zwischen 7. 10. 1891 und
               Ende April 1892{]}}
\nopagebreak\mylabel{L00043v}
\rehead{ }\normalsize\beginnumbering\briefempfaengerindex{Beer-Hofmann, Richard@\textsc{Beer-Hofmann, Richard}!zzzSchnitzler, Arthur@\emph{von Arthur Schnitzler}!1892-04-301@{{[}zwischen 7. 10. 1891 und
                  30. 4. 1892{]}}|(be}
\toendnotes[C]{\smallbreak\pagebreak[2]}\Standort{YCGL, MSS 31.}
\physDesc{Briefkarte, , Umschlag, 187 Zeichen
\newline{}Handschrift: Bleistift, deutsche Kurrent
\newline{}Versand: ohne postalischen Übermittlungsvermerk }\toendnotes[C]{\smallbreak}\pstart{}{\pb}\textsc{Herrn Dr. Rich Beer-Hofma{\geminationn}}\pend{}\pstart{}\textsc{Wien\oindex{Wien@\textbf{Wien}, \emph{A.ADM2}|pw}}\pend{}\pstart{}\textsc{III Seidlgasse 30\oindex{Seidlgasse@\textbf{Seidlgasse}, \emph{Straße (K.STR)}|pw}}.\pend{}{\bigskip}\vspace{1em}
\pstart{}{\pb}Lieber Richard,\pend\vspace{0.5em}
\pstart
           Ich bin \label{K_L00043-1v}\edtext{heute}{\lemma{\textnormal{\emph{heute}}}\Cendnote{\textnormal{Das undatierte Korrespondenzstück ist
                  womöglich auf den ersten Nachmittagsaufenthalt Beer-Hofmanns\pwindex{Beer-Hofmann, Richard 1866-07-11 – 1945-09-26@\textsc{Beer-Hofmann, Richard} (1866-07-11 – 1945-09-26), \emph{Schriftsteller/Schriftstellerin}|pwk} bei Schnitzler,
                  jedenfalls aber frühestens auf diesen einzuordnen. Da Beer-Hofmann\pwindex{Beer-Hofmann, Richard 1866-07-11 – 1945-09-26@\textsc{Beer-Hofmann, Richard} (1866-07-11 – 1945-09-26), \emph{Schriftsteller/Schriftstellerin}|pwk} nur bis Ende April 1892 in der
                     Seidlgasse\oindex{Seidlgasse@\textbf{Seidlgasse}, \emph{Straße (K.STR)}|pwk} wohnte, gibt dieses Datum die hintere
                  zeitliche Grenze an.}}}\label{K_L00043-1} Nachmittag zu Hauſe u habe auch die andern {\pb}verſtändigt. We{\geminationn} Sie
               nichts beſſeres vorhaben, ko{\geminationm}en Sie?\pend
           
\pstart
           Herzlich{\\[\baselineskip]}\spacefill\mbox{Arthur}\pend
           \leftskip=0em{}\selectlanguage{ngerman}\endnumbering\briefempfaengerindex{Beer-Hofmann, Richard@\textsc{Beer-Hofmann, Richard}!zzzSchnitzler, Arthur@\emph{von Arthur Schnitzler}!1891-10-071@{{[}zwischen 7. 10. 1891 und
                  30. 4. 1892{]}}|)be}\mylabel{L00043h}  \normalsize

\doendnotes{C}
\bigskip
\vfill

\clearpage

\footnotesize

\lohead{\textsc{register}}

% Definiere theindex-Environment komplett neu ohne reledmac
\makeatletter
\renewenvironment{theindex}{%
  \section*{\indexname}%
  \setlength{\parindent}{0pt}%
  \setlength{\parskip}{0pt plus 0.3pt}%
  \let\item\@idxitem
}{%
  \clearpage
}
\makeatother

\IfFileExists{\jobname-pw.ind}{\input{\jobname-pw.ind}}{}

\end{document}

      