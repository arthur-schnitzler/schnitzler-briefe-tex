%% latex-korrekturansicht-vorspann.tex
%% Vorspann für die Korrekturansicht.
%% Lädt die gemeinsame Datei latex-vorspann.tex mit gesetztem Schalter.

\newif\ifkorrekturansicht
\korrekturansichttrue

\input{../tex-inputs/latex-vorspann}


\section[Gerhart Hauptmann an Arthur Schnitzler, 29. {[}1. 1905?{]}]{L01496 Gerhart Hauptmann an Arthur Schnitzler, 29. {[}1. 1905?{]}}
\nopagebreak\mylabel{L01496v}
\rehead{ }\normalsize\beginnumbering\briefempfaengerindex{Schnitzler, Arthur@\textsc{Schnitzler, Arthur}!zzzHauptmann, Gerhart@\emph{von Gerhart Hauptmann}!1905-01-291@{{[}29. 1. 1905?{]}}|(be}
\toendnotes[C]{\smallbreak\pagebreak[2]}\Standort{CUL, Schnitzler, B 36.}
\physDesc{Telegramm, 171 Zeichen
\newline{}maschinell
\newline{}Versand: Stempel des Telegrafenbeamten, der Telegrafenbeamtin: »Fischer\pwindex{Fischer @\textsc{Fischer}, \emph{Telegrafenbeamter/Telegrafenbeamtin}|pw}« 
\newline{}Ordnung: beschnitten }\toendnotes[C]{\smallbreak}
\pstart
           {\pb}fr agnetendorf\oindex{Jagniątków@\textbf{Jagniątków}, \emph{P.PPL}|pw} 128 28 29{ }12.25 n\pend
           \vspace{0.5em}
\pstart
           lieber herr schnitzler ich werde gern den gewuenschten \label{K_L01496-1v}\edtext{prolog\pwindex{Prolog einer musikalischen Feier zum Gedaechtnisse Schillers@\emph{Prolog einer musikalischen Feier zum Gedächtnisse Schillers}|pwv}}{\lemma{\textnormal{\emph{prolog}}}\Cendnote{\textnormal{Das undatierte Telegramm dürfte am 29.
                  eines Monats versandt worden sein. Wahrscheinlich steht es in Zusammenhang mit dem von Hauptmann\pwindex{Hauptmann, Gerhart 15.11.1862 – 06.06.1946@\textsc{Hauptmann, Gerhart} (15.11.1862 – 06.06.1946), \emph{Schriftsteller/Schriftstellerin}|pwk} verfassten Prolog\pwindex{Prolog einer musikalischen Feier zum Gedaechtnisse Schillers@\emph{Prolog einer musikalischen Feier zum Gedächtnisse Schillers}|pwkv}, der am
                     22. 3. 1905 bei der Schiller\pwindex{Schiller, Friedrich von 10.11.1759 – 09.05.1805@\textsc{Schiller, Friedrich von} (10.11.1759 – 09.05.1805), \emph{Schriftsteller/Schriftstellerin, Historiker/Historikerin, Philosoph/Philosophin}|pwk}feier des \emph{Wiener
                     Konzertvereins}\orgindex{Wiener Konzertverein@Wiener Konzertverein|pwk} vorgetragen wurde. Da der 29. 2. 1905 zu
                  kurzfristig für eine solche Zusage erscheint, könnte es am
                     29. 1. 1905 geschickt worden sein. Das wiederum würde es
                  nahelegen, dass Hofmannsthal\pwindex{Hofmannsthal, Hugo von 1874-02-01 – 1929-07-15@\textsc{Hofmannsthal, Hugo von} (1874-02-01 – 1929-07-15), \emph{Schriftsteller/Schriftstellerin}|pwk} mit der
                  Kommission betraut war, die Anfrage zu stellen.}}}\label{K_L01496-1} so gut es geht verfaszen.
               herzliche gruesze von haus zu haus ihr \spacefill\mbox{gerhart hauptmann +}\pend
           \selectlanguage{ngerman}\endnumbering\briefempfaengerindex{Schnitzler, Arthur@\textsc{Schnitzler, Arthur}!zzzHauptmann, Gerhart@\emph{von Gerhart Hauptmann}!1905-01-291@{{[}29. 1. 1905?{]}}|)be}\mylabel{L01496h}  \normalsize

\doendnotes{C}
\bigskip
\vfill

\clearpage

\footnotesize

\lohead{\textsc{register}}

% Definiere theindex-Environment komplett neu ohne reledmac
\makeatletter
\renewenvironment{theindex}{%
  \section*{\indexname}%
  \setlength{\parindent}{0pt}%
  \setlength{\parskip}{0pt plus 0.3pt}%
  \let\item\@idxitem
}{%
  \clearpage
}
\makeatother

\IfFileExists{\jobname-pw.ind}{\input{\jobname-pw.ind}}{}

\end{document}

      