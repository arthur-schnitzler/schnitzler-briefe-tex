%% latex-leseansicht-vorspann.tex
%% Vorspann für die Leseansicht.
%% Lädt die gemeinsame Datei latex-vorspann.tex mit nicht gesetztem Schalter.

\newif\ifkorrekturansicht
\korrekturansichtfalse

\input{../tex-inputs/latex-vorspann}


\section[Gerhart Hauptmann an Arthur Schnitzler, 29. {[}1. 1905?{]}]{L01496 Gerhart Hauptmann an Arthur Schnitzler, 29. [1. 1905?]}
\nopagebreak\mylabel{L01496v}
\rehead{ }\normalsize\beginnumbering\briefempfaengerindex{Schnitzler, Arthur@\textsc{Schnitzler, Arthur}!zzzHauptmann, Gerhart@\emph{von Gerhart Hauptmann}!1905-01-291@{{[}29. 1. 1905?{]}}|(be}
\toendnotes[C]{\smallbreak\pagebreak[2]}
\correspDesc{Versand  durch Gerhart Hauptmann am [29. 1. 1905?] in Agnetendorf
\newline{}Erhalt  durch Arthur Schnitzler am [29. 1. 1905?] in Wien}\toendnotes[C]{\smallbreak}
\Standort{CUL, Schnitzler, B 36.}
\physDesc{Telegramm, 171 Zeichen
\newline{}maschinell
\newline{}Versand: Stempel des Telegrafenbeamten, der Telegrafenbeamtin: »Fischer\pwindex{Fischer @\textsc{Fischer}, \emph{Telegrafenbeamter/Telegrafenbeamtin}|pw}« 
\newline{}Ordnung: beschnitten }\toendnotes[C]{\smallbreak}
\pstart
           {\pb}fr agnetendorf\oindex{Jagniątków@\textbf{Jagniątków}|pw} 128 28 29{ }12.25 n\pend
           \vspace{0.5em}
\pstart
           lieber herr schnitzler ich werde gern den gewuenschten \label{K_L01496-1v}\edtext{prolog\pwindex{Hauptmann, Gerhart 15.\,11.\,1862 Szczawno-Zdrój – 6.\,6.\,1946 Jagniątków@\textsc{Hauptmann, Gerhart} (15.\,11.\,1862 Szczawno-Zdrój – 6.\,6.\,1946 Jagniątków), \emph{Schriftsteller}!Prolog einer musikalischen Feier zum Gedächtnisse Schillers@\strich\emph{Prolog einer musikalischen Feier zum Gedächtnisse Schillers}|pwv}}{\lemma{\textnormal{\emph{prolog}}}\Cendnote{\textnormal{Das undatierte Telegramm dürfte am 29.
                  eines Monats versandt worden sein. Wahrscheinlich steht es in Zusammenhang mit dem von Hauptmann\pwindex{Hauptmann, Gerhart 15.\,11.\,1862 Szczawno-Zdrój – 6.\,6.\,1946 Jagniątków@\textsc{Hauptmann, Gerhart} (15.\,11.\,1862 Szczawno-Zdrój – 6.\,6.\,1946 Jagniątków), \emph{Schriftsteller}|pwk} verfassten Prolog\pwindex{Hauptmann, Gerhart 15.\,11.\,1862 Szczawno-Zdrój – 6.\,6.\,1946 Jagniątków@\textsc{Hauptmann, Gerhart} (15.\,11.\,1862 Szczawno-Zdrój – 6.\,6.\,1946 Jagniątków), \emph{Schriftsteller}!Prolog einer musikalischen Feier zum Gedächtnisse Schillers@\strich\emph{Prolog einer musikalischen Feier zum Gedächtnisse Schillers}|pwkv}, der am
                     22. 3. 1905 bei der Schiller\pwindex{Schiller, Friedrich von 10.\,11.\,1759 Marbach am Neckar – 9.\,5.\,1805 Weimar@\textsc{Schiller, Friedrich von} (10.\,11.\,1759 Marbach am Neckar – 9.\,5.\,1805 Weimar), \emph{Schriftsteller, Historiker, Philosoph}|pwk}feier des \emph{Wiener
                     Konzertvereins}\orgindex{Wiener Konzertverein@Wiener Konzertverein|pwk} vorgetragen wurde. Da der 29. 2. 1905 zu
                  kurzfristig für eine solche Zusage erscheint, könnte es am
                     29. 1. 1905 geschickt worden sein. Das wiederum würde es
                  nahelegen, dass Hofmannsthal\pwindex{Hofmannsthal, Hugo von 1.\,2.\,1874 Wien – 15.\,7.\,1929 Rodaun@\textsc{Hofmannsthal, Hugo von} (1.\,2.\,1874 Wien – 15.\,7.\,1929 Rodaun), \emph{Schriftsteller}|pwk} mit der
                  Kommission betraut war, die Anfrage zu stellen.}}}\label{K_L01496-1} so gut es geht verfaszen.
               herzliche gruesze von haus zu haus ihr \spacefill\mbox{gerhart hauptmann +}\pend
           \selectlanguage{ngerman}\endnumbering\briefempfaengerindex{Schnitzler, Arthur@\textsc{Schnitzler, Arthur}!zzzHauptmann, Gerhart@\emph{von Gerhart Hauptmann}!1905-01-291@{{[}29. 1. 1905?{]}}|)be}\mylabel{L01496h}  \newcommand{\dateiname}{L01496}\newcommand{\titel}{Gerhart Hauptmann an Arthur Schnitzler, 29. [1. 1905?]}\newcommand{\editorInnen}{Martin Anton Müller und Gerd-Hermann Susen}%% latex-leseansicht-abspann.tex
%% Abspann für die Leseansicht.
%% Der Schalter \ifkorrekturansicht ist bereits durch den Vorspann gesetzt.

%% latex-abspann.tex
%% Gemeinsamer Abspann für Korrekturansicht und Leseansicht.
%% Setzt den Schalter \ifkorrekturansicht voraus (gesetzt in den
%% einbindenden Dateien latex-korrekturansicht-abspann.tex bzw.
%% latex-leseansicht-abspann.tex).
%% ---------------------------------------------------------------

\normalsize

% Das esempio-Environment wird nur in der Leseansicht benötigt
\ifkorrekturansicht\else
\newenvironment{esempio}[3]%
{
    \vspace{1.5ex}
    \rlap{\underline{#1}}
    \par
    \setlength{\parindent}{0cm}
    \nopagebreak
    \leftskip=#2cm
    \rightskip=#3cm
}
{
    \par
}
\fi

\doendnotes{C}
\bigskip
\vfill

\clearpage

\footnotesize

\ifkorrekturansicht
  \lohead{\textsc{register}}
\fi

% theindex-Environment neu definieren ohne reledmac
\makeatletter
\renewenvironment{theindex}{%
  \ifkorrekturansicht
    \section*{\indexname}%
  \else
    \subsubsection*{Index der erwähnten Entitäten}%
  \fi
  \setlength{\parindent}{0pt}%
  \setlength{\parskip}{0pt plus 0.3pt}%
  \let\item\@idxitem
}{%
  \ifkorrekturansicht\clearpage\fi
}
\makeatother

\IfFileExists{\jobname-pw.ind}{\input{\jobname-pw.ind}}{}

% Quellenangabe nur in der Leseansicht
\ifkorrekturansicht\else
% Fallback-Definitionen, falls die .tex-Datei \titel etc. nicht gesetzt hat
\providecommand{\titel}{}
\providecommand{\editorInnen}{}
\providecommand{\dateiname}{\jobname}

\vspace{3cm}

\vfill

\footnotesize
\textsc{Quelle}: \titel. Herausgegeben von {\editorInnen}. In: \emph{Arthur Schnitzler: Briefwechsel mit Autorinnen und Autoren}.
 Digitale Edition, https://schnitzler-briefe.acdh.oeaw.ac.at/{\dateiname}.html (Stand \today)
\fi

\end{document}


