%% latex-korrekturansicht-vorspann.tex
%% Vorspann für die Korrekturansicht.
%% Lädt die gemeinsame Datei latex-vorspann.tex mit gesetztem Schalter.

\newif\ifkorrekturansicht
\korrekturansichttrue

\input{../tex-inputs/latex-vorspann}


\section[Lou Andreas-Salomé an Arthur Schnitzler, {[}22. 4. 1901{]}]{L01111 Lou Andreas-Salomé an Arthur Schnitzler, {[}22. 4. 1901{]}}
\nopagebreak\mylabel{L01111v}
\rehead{ }\normalsize\beginnumbering\briefempfaengerindex{Schnitzler, Arthur@\textsc{Schnitzler, Arthur}!zzzAndreas-Salome, Lou@\emph{von Lou Andreas-Salomé}!1901-04-221@{{[}22. 4. 1901{]}}|(be}
\toendnotes[C]{\smallbreak\pagebreak[2]}\Standort{CUL, Schnitzler, B 3.}
\physDesc{Briefkarte, 715 Zeichen
\newline{}Handschrift: schwarze Tinte, deutsche Kurrent
\newline{}Schnitzler: 1) mit Bleistift datiert: »22/4 901«  2) mit rotem Buntstift eine Unterstreichung
\newline{}Ordnung: mit Bleistift von unbekannter Hand nummeriert:
                                    »18« }\toendnotes[C]{\smallbreak}
\pstart{}{\pb}Lieber Herr Doktor,\pend\vspace{0.5em}
\pstart
           ſehr freu ich mich darüber, Ihr neues Buch\pwindex{Frau Bertha Garlan. Roman@\emph{Frau Bertha Garlan. Roman}|pwv} von Ihnen zu empfangen, nachdem ich die Bekanntſchaft
               mit Frau \textsc{Bertha Garlan}\pwindex{Frau Bertha Garlan. Roman@\emph{Frau Bertha Garlan. Roman}|pwv} und Frau \textsc{Rupius}\pwindex{Frau Bertha Garlan. Roman@\emph{Frau Bertha Garlan. Roman}|pwv} in der \label{K_L01111-1v}\edtext{\textsc{N. D. Rundschau}\pwindex{Neue Deutsche Rundschau@\emph{Neue Deutsche Rundschau}|pw}}{\lemma{\textnormal{\emph{N. D. Rundschau}}}\Cendnote{\textnormal{Nachdem \emph{Frau Bertha Garlan}\pwindex{Frau Bertha Garlan. Roman@\emph{Frau Bertha Garlan. Roman}|pwk} in drei Teilen zwischen Januar und
                     März 1901 in der \emph{Neuen Deutschen
                     Rundschau}\pwindex{Neue Deutsche Rundschau@\emph{Neue Deutsche Rundschau}|pwk} erschienen war, wurde die Buchausgabe Mitte April
                  ausgeliefert (\emph{Frau Bertha Garlan}\pwindex{Frau Bertha Garlan. Roman@\emph{Frau Bertha Garlan. Roman}|pwk}. Roman. Berlin: \emph{S. Fischer}\orgindex{S. Fischer Verlag@S. Fischer Verlag|pwk}{ }1901).
               }}}\label{K_L01111-1} gemacht habe. Um Frau \textsc{Rupius}\pwindex{Frau Bertha Garlan. Roman@\emph{Frau Bertha Garlan. Roman}|pwv} focht ich ſogar mit Frieda Bülow\pwindex{Buelow, Frieda von 12.10.1857 – 12.03.1909@\textsc{Bülow, Frieda von} (12.10.1857 – 12.03.1909), \emph{Schriftsteller/Schriftstellerin}|pw} einen
               großen Streit aus; ich hielt es mit Herrn \textsc{Rupius}\pwindex{Frau Bertha Garlan. Roman@\emph{Frau Bertha Garlan. Roman}|pwv}.\pend
           
\pstart
           Hoffentlich geht es Ihnen drüben in Wien\oindex{Wien@\textbf{Wien}, \emph{A.ADM2}|pw}{ }ſo gut, wie mir hier, wo ich zwar nur zur Hälfte
               bin, denn {\pb}am liebſten ſind mein Mann\pwindex{Andreas, Friedrich Carl 14.04.1846 – 03.10.1930@\textsc{Andreas, Friedrich Carl} (14.04.1846 – 03.10.1930), \emph{Orientalist/Orientalistin}|pwv} und ich in Rußland\oindex{Russland@\textbf{Russland}, \emph{A.PCLI}|pw} und reiſen auch demnächſt wieder auf
               lange dorthin. Erſt ſeit ein paar Jahren kenne ich meine ruſſiſche\oindex{Russland@\textbf{Russland}, \emph{A.PCLI}|pw} Heimath in ihrem weitern Umkreis, mit ihren
               Landſchaften und Menſchen; ſeitdem weiß ich erſt, daß ſie meine Heimath iſt, und daß
               ich eigentlich dort lebe.\pend
           
\pstart
           Herzlichen Gruß Ihnen allen!{\\[\baselineskip]}\spacefill\mbox{Frau Lou.}\pend
           \leftskip=0em{}\selectlanguage{ngerman}\endnumbering\briefempfaengerindex{Schnitzler, Arthur@\textsc{Schnitzler, Arthur}!zzzAndreas-Salome, Lou@\emph{von Lou Andreas-Salomé}!1901-04-221@{{[}22. 4. 1901{]}}|)be}\mylabel{L01111h}  \normalsize

\doendnotes{C}
\bigskip
\vfill

\clearpage

\footnotesize

\lohead{\textsc{register}}

% Definiere theindex-Environment komplett neu ohne reledmac
\makeatletter
\renewenvironment{theindex}{%
  \section*{\indexname}%
  \setlength{\parindent}{0pt}%
  \setlength{\parskip}{0pt plus 0.3pt}%
  \let\item\@idxitem
}{%
  \clearpage
}
\makeatother

\IfFileExists{\jobname-pw.ind}{\input{\jobname-pw.ind}}{}

\end{document}

      