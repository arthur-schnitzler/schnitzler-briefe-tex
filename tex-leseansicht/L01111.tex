%% latex-leseansicht-vorspann.tex
%% Vorspann für die Leseansicht.
%% Lädt die gemeinsame Datei latex-vorspann.tex mit nicht gesetztem Schalter.

\newif\ifkorrekturansicht
\korrekturansichtfalse

\input{../tex-inputs/latex-vorspann}


\section[Lou Andreas-Salomé an Arthur Schnitzler, {{[}}22. 4. 1901{{]}}]{L01111 Lou Andreas-Salomé an Arthur Schnitzler, {[}22. 4. 1901{]}}
\nopagebreak\mylabel{L01111v}
\rehead{ }\normalsize\beginnumbering\briefempfaengerindex{Schnitzler, Arthur@\textsc{Schnitzler, Arthur}!zzzAndreas-Salomé, Lou@\emph{von Lou Andreas-Salomé}!1901-04-221@{{[}22. 4. 1901{]}}|(be}
\toendnotes[C]{\smallbreak\pagebreak[2]}
\correspDesc{Versand  durch Lou Andreas-Salomé am [22. 4. 1901] \textbf{Ort fehlend} 
\newline{}Erhalt  durch Arthur Schnitzler im Zeitraum [22. 4. 1901
                  – 26. 4. 1901?] in Wien}\toendnotes[C]{\smallbreak}
\Standort{CUL, Schnitzler, B 3.}
\physDesc{Briefkarte, 715 Zeichen
\newline{}Handschrift: schwarze Tinte, deutsche Kurrent
\newline{}Schnitzler: 1) mit Bleistift datiert: »22/4 901«  2) mit rotem Buntstift eine Unterstreichung
\newline{}Ordnung: mit Bleistift von unbekannter Hand nummeriert:
                                    »18« }\toendnotes[C]{\smallbreak}
\pstart{}{\pb}Lieber Herr Doktor,\pend\vspace{0.5em}
\pstart
           ſehr freu ich mich darüber, Ihr neues Buch\pwindex{Schnitzler, Arthur 15.\,5.\,1862 Wien – 21.\,10.\,1931 ebd.@\textsc{Schnitzler, Arthur} (15.\,5.\,1862 Wien – 21.\,10.\,1931 ebd.), \emph{Schriftsteller, Mediziner}!Frau Bertha Garlan. Roman@\strich\emph{Frau Bertha Garlan. Roman}|pwv} von Ihnen zu empfangen, nachdem ich die Bekanntſchaft
               mit Frau \textsc{Bertha Garlan}\pwindex{Schnitzler, Arthur 15.\,5.\,1862 Wien – 21.\,10.\,1931 ebd.@\textsc{Schnitzler, Arthur} (15.\,5.\,1862 Wien – 21.\,10.\,1931 ebd.), \emph{Schriftsteller, Mediziner}!Frau Bertha Garlan. Roman@\strich\emph{Frau Bertha Garlan. Roman}|pwv} und Frau \textsc{Rupius}\pwindex{Schnitzler, Arthur 15.\,5.\,1862 Wien – 21.\,10.\,1931 ebd.@\textsc{Schnitzler, Arthur} (15.\,5.\,1862 Wien – 21.\,10.\,1931 ebd.), \emph{Schriftsteller, Mediziner}!Frau Bertha Garlan. Roman@\strich\emph{Frau Bertha Garlan. Roman}|pwv} in der \label{K_L01111-1v}\edtext{\textsc{N. D. Rundschau}\pwindex{Neue Deutsche Rundschau@\emph{Neue Deutsche Rundschau}|pw}}{\lemma{\textnormal{\emph{N. D. Rundschau}}}\Cendnote{\textnormal{Nachdem \emph{Frau Bertha Garlan}\pwindex{Schnitzler, Arthur 15.\,5.\,1862 Wien – 21.\,10.\,1931 ebd.@\textsc{Schnitzler, Arthur} (15.\,5.\,1862 Wien – 21.\,10.\,1931 ebd.), \emph{Schriftsteller, Mediziner}!Frau Bertha Garlan. Roman@\strich\emph{Frau Bertha Garlan. Roman}|pwk} in drei Teilen zwischen Januar und
                     März 1901 in der \emph{Neuen Deutschen
                     Rundschau}\pwindex{Neue Deutsche Rundschau@\emph{Neue Deutsche Rundschau}|pwk} erschienen war, wurde die Buchausgabe Mitte April
                  ausgeliefert (\emph{Frau Bertha Garlan}\pwindex{Schnitzler, Arthur 15.\,5.\,1862 Wien – 21.\,10.\,1931 ebd.@\textsc{Schnitzler, Arthur} (15.\,5.\,1862 Wien – 21.\,10.\,1931 ebd.), \emph{Schriftsteller, Mediziner}!Frau Bertha Garlan. Roman@\strich\emph{Frau Bertha Garlan. Roman}|pwk}. Roman. Berlin: \emph{S. Fischer}\orgindex{S. Fischer Verlag@S. Fischer Verlag|pwk}{ }1901).
               }}}\label{K_L01111-1} gemacht habe. Um Frau \textsc{Rupius}\pwindex{Schnitzler, Arthur 15.\,5.\,1862 Wien – 21.\,10.\,1931 ebd.@\textsc{Schnitzler, Arthur} (15.\,5.\,1862 Wien – 21.\,10.\,1931 ebd.), \emph{Schriftsteller, Mediziner}!Frau Bertha Garlan. Roman@\strich\emph{Frau Bertha Garlan. Roman}|pwv} focht ich{ }ſogar mit Frieda Bülow\pwindex{Bülow, Frieda von 12.\,10.\,1857 Berlin – 12.\,3.\,1909 Dornburg@\textsc{Bülow, Frieda von} (12.\,10.\,1857 Berlin – 12.\,3.\,1909 Dornburg), \emph{Schriftstellerin}|pw} einen
               großen Streit aus; ich hielt es mit Herrn \textsc{Rupius}\pwindex{Schnitzler, Arthur 15.\,5.\,1862 Wien – 21.\,10.\,1931 ebd.@\textsc{Schnitzler, Arthur} (15.\,5.\,1862 Wien – 21.\,10.\,1931 ebd.), \emph{Schriftsteller, Mediziner}!Frau Bertha Garlan. Roman@\strich\emph{Frau Bertha Garlan. Roman}|pwv}.\pend
           
\pstart
           Hoffentlich geht es Ihnen drüben in Wien\oindex{Wien@\textbf{Wien}, \emph{Verwaltungsgebiet}|pw}{ }ſo gut, wie mir hier, wo ich zwar nur zur Hälfte
               bin, denn {\pb}am liebſten{ }ſind mein Mann\pwindex{Andreas, Friedrich Carl 14.\,4.\,1846 Jakarta – 3.\,10.\,1930 Göttingen@\textsc{Andreas, Friedrich Carl} (14.\,4.\,1846 Jakarta – 3.\,10.\,1930 Göttingen), \emph{Orientalist}|pwv} und ich in Rußland\oindex{Russland@\textbf{Russland}|pw} und reiſen auch demnächſt wieder auf
               lange dorthin. Erſt{ }ſeit ein paar Jahren kenne ich meine ruſſiſche\oindex{Russland@\textbf{Russland}|pw} Heimath in ihrem weitern Umkreis, mit ihren
               Landſchaften und Menſchen;{ }ſeitdem weiß ich erſt, daß{ }ſie meine Heimath iſt, und daß
               ich eigentlich dort lebe.\pend
           
\pstart
           Herzlichen Gruß Ihnen allen!{\\[\baselineskip]}\spacefill\mbox{Frau Lou.}\pend
           \leftskip=0em{}\selectlanguage{ngerman}\endnumbering\briefempfaengerindex{Schnitzler, Arthur@\textsc{Schnitzler, Arthur}!zzzAndreas-Salomé, Lou@\emph{von Lou Andreas-Salomé}!1901-04-221@{{[}22. 4. 1901{]}}|)be}\mylabel{L01111h}  \newcommand{\dateiname}{L01111}\newcommand{\titel}{Lou Andreas-Salomé an Arthur Schnitzler, [22. 4. 1901]}\newcommand{\editorInnen}{Martin Anton Müller und Gerd-Hermann Susen}%% latex-leseansicht-abspann.tex
%% Abspann für die Leseansicht.
%% Der Schalter \ifkorrekturansicht ist bereits durch den Vorspann gesetzt.

%% latex-abspann.tex
%% Gemeinsamer Abspann für Korrekturansicht und Leseansicht.
%% Setzt den Schalter \ifkorrekturansicht voraus (gesetzt in den
%% einbindenden Dateien latex-korrekturansicht-abspann.tex bzw.
%% latex-leseansicht-abspann.tex).
%% ---------------------------------------------------------------

\normalsize

% Das esempio-Environment wird nur in der Leseansicht benötigt
\ifkorrekturansicht\else
\newenvironment{esempio}[3]%
{
    \vspace{1.5ex}
    \rlap{\underline{#1}}
    \par
    \setlength{\parindent}{0cm}
    \nopagebreak
    \leftskip=#2cm
    \rightskip=#3cm
}
{
    \par
}
\fi

\doendnotes{C}
\bigskip
\vfill

\clearpage

\footnotesize

\ifkorrekturansicht
  \lohead{\textsc{register}}
\fi

% theindex-Environment neu definieren ohne reledmac
\makeatletter
\renewenvironment{theindex}{%
  \ifkorrekturansicht
    \section*{\indexname}%
  \else
    \subsubsection*{Index der erwähnten Entitäten}%
  \fi
  \setlength{\parindent}{0pt}%
  \setlength{\parskip}{0pt plus 0.3pt}%
  \let\item\@idxitem
}{%
  \ifkorrekturansicht\clearpage\fi
}
\makeatother

\IfFileExists{\jobname-pw.ind}{\input{\jobname-pw.ind}}{}

% Quellenangabe nur in der Leseansicht
\ifkorrekturansicht\else
% Fallback-Definitionen, falls die .tex-Datei \titel etc. nicht gesetzt hat
\providecommand{\titel}{}
\providecommand{\editorInnen}{}
\providecommand{\dateiname}{\jobname}

\vspace{3cm}

\vfill

\footnotesize
\textsc{Quelle}: \titel. Herausgegeben von {\editorInnen}. In: \emph{Arthur Schnitzler: Briefwechsel mit Autorinnen und Autoren}.
 Digitale Edition, https://schnitzler-briefe.acdh.oeaw.ac.at/{\dateiname}.html (Stand \today)
\fi

\end{document}


