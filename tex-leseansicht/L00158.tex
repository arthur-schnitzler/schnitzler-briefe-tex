%% latex-leseansicht-vorspann.tex
%% Vorspann für die Leseansicht.
%% Lädt die gemeinsame Datei latex-vorspann.tex mit nicht gesetztem Schalter.

\newif\ifkorrekturansicht
\korrekturansichtfalse

\input{../tex-inputs/latex-vorspann}


               \section[Eduard Michael Kafka an Arthur Schnitzler, 12. 1. 1893]{ Eduard Michael Kafka an Arthur Schnitzler, 12. 1. 1893}\nopagebreak\mylabel{v}\rehead{ }\begin{ledgroupsized}[t]{13cm}\normalsize\beginnumbering\briefempfaengerindex{Schnitzler, Arthur@\textsc{Schnitzler, Arthur}!zzzKafka, Eduard Michael@\emph{von Eduard Michael Kafka}!1893-01-121@{12. 1. 1893}|(be} \toendnotes[C]{\smallbreak\pagebreak[2]} \Standort{DLA, A:Schnitzler, HS.NZ85.1.3604.}
\physDesc{Brief, 1 Blatt, 4 Seiten
\newline{}Handschrift: schwarze Tinte, deutsche Kurrent
\newline{}Schnitzler: mit rotem Buntstift mehrere Unterstreichungen }\toendnotes[C]{\smallbreak}\pstart
           \raggedleft{}{\pb}12/1 93.\pend
           \pstart{}Lieber Freund,\pend\pstart
           vorgeſtern – bei einer Soiree des Rechtsanwalts D\textsuperscript{r}{ }Grelling\pwindex{Grelling, Richard 11.06.1853 – 15.01.1929@\textsc{Grelling, Richard} (11.06.1853 – 15.01.1929), \emph{Schriftsteller, Rechtsanwalt, Publizist}|pw} in \textsc{Berlin}\oindex{Berlin@\textbf{Berlin}|pw} – wurde Ihre »Frage an das Schickſal\pwindex{Schnitzler, Arthur 15.05.1862 – 21.10.1931@\textsc{Schnitzler, Arthur} (15.05.1862 – 21.10.1931), \emph{Schriftsteller, Mediziner}!Frage an das Schicksal01. 05. 1890@\strich\emph{Die Frage an das Schicksal} {[}01. 05. 1890{]}|pw}«
                    aufgeführt. Reicher\pwindex{Reicher, Emanuel 18.06.1849 – 15.05.1924@\textsc{Reicher, Emanuel} (18.06.1849 – 15.05.1924), \emph{Schauspieler}|pw} brillirte als Anatol\pwindex{Schnitzler, Arthur 15.05.1862 – 21.10.1931@\textsc{Schnitzler, Arthur} (15.05.1862 – 21.10.1931), \emph{Schriftsteller, Mediziner}!Anatol1892-10-29 – 1892-10-29@\strich\emph{Anatol} {[}1892-10-29 – 1892-10-29{]}|pw} – ich kann Ihnen nicht ſchildern, wie
                    vorzüglich er war: einfach ganz \uline{einzig}, der Anatol\pwindex{Schnitzler, Arthur 15.05.1862 – 21.10.1931@\textsc{Schnitzler, Arthur} (15.05.1862 – 21.10.1931), \emph{Schriftsteller, Mediziner}!Anatol1892-10-29 – 1892-10-29@\strich\emph{Anatol} {[}1892-10-29 – 1892-10-29{]}|pw}{ }\textsc{par excellence}. – Es hat mich ungemein gefreut, daſs
                    ich der Aufführung Ihres Stückes – in ſo meiſterlicher Darſtellung – habe
                    perſönlich beiwohnen können. Es waren mehr {\pb}als 100 Perſonen anweſend; die
                    hervorragendſten \textsc{literarischen} u künſtleriſchen Kreiſe
                    waren vertreten: von Sudermann\pwindex{Sudermann, Hermann 30.09.1857 – 21.11.1928@\textsc{Sudermann, Hermann} (30.09.1857 – 21.11.1928), \emph{Schriftsteller}|pw} bis Träger\pwindex{Traeger, Albert 1830-06-12 – 1912-03-26@\textsc{Traeger, Albert} (1830-06-12 – 1912-03-26), \emph{Schriftsteller, Politiker}|pw}. Sudermann\pwindex{Sudermann, Hermann 30.09.1857 – 21.11.1928@\textsc{Sudermann, Hermann} (30.09.1857 – 21.11.1928), \emph{Schriftsteller}|pw}\introOben{}inſonderheit\introOben{} war ganz entzückt u. wurde nicht müde,
                    ſeinen Beifall in der allerlebhafteſten Weiſe, durch beſtändige Zwischenrufe \substVorne{}\textsuperscript{\textcolor{gray}{von}}\substDazwischen{}aufrichtiger\substHinten{} Bewunderung, Ausdruck zu geben.\pend
           \pstart
           Reicher\pwindex{Reicher, Emanuel 18.06.1849 – 15.05.1924@\textsc{Reicher, Emanuel} (18.06.1849 – 15.05.1924), \emph{Schauspieler}|pw} läßt Sie grüßen. Er bat mich Ihnen
                        \introOben{}zugleich\introOben{} mitzuteilen, daſs Blumenthal\pwindex{Blumenthal, Oskar 13.03.1852 – 24.04.1917@\textsc{Blumenthal, Oskar} (13.03.1852 – 24.04.1917), \emph{Schriftsteller, Journalist, Theaterleiter}|pw}{ }\substVorne{}\textsuperscript{\textcolor{gray}{angeg}}\substDazwischen{}bezüglich\substHinten{} der Aufführung des »Märchen\pwindex{Schnitzler, Arthur 15.05.1862 – 21.10.1931@\textsc{Schnitzler, Arthur} (15.05.1862 – 21.10.1931), \emph{Schriftsteller, Mediziner}!Maerchen. Schauspiel in drei Aufzuegen1891 – 1891@\strich\emph{Das Märchen. Schauspiel in drei Aufzügen} {[}1891 – 1891{]}|pw}« darauf
                    {\pb}hinweiſt, daſs Sie ihm ſeinerzeit
                    geſagt hätten, das Stück werde in Prag\oindex{Prag@\textbf{Prag}|pw} gegeben
                    werden. Er\pwindex{Blumenthal, Oskar 13.03.1852 – 24.04.1917@\textsc{Blumenthal, Oskar} (13.03.1852 – 24.04.1917), \emph{Schriftsteller, Journalist, Theaterleiter}|pwv} möchte erst
                    dieſe Aufführung abwarten, – Sie ſollen daher zuſehen, daſs Sie die Prag\oindex{Prag@\textbf{Prag}|pw}er Première beſchleunigen. – Notabene,
                    Lieber Freund, – dieſes Berlin\oindex{Berlin@\textbf{Berlin}|pw} iſt eine
                    herrliche Stadt: ich fühle mich hier, obwol ich erſt einige Tage da bin, ſo
                    heimiſch, als wäre {\pb}ich \substVorne{}\textsuperscript{hier}\substDazwischen{}dort\substHinten{} geboren. Wir wiſſen in Wien\oindex{Wien@\textbf{Wien}|pw} nicht, was
                    geiſtiges u künſtleriſches Leben bedeutet: man muſs hieher kommen, wenn man dies
                    erfahren will.\pend
           \pstart
           Raten Sie, bitte, ſchleunigſt allen unſeren lieben Freunden: Sie ſollen ohne
                    Zaudern, ohne eine Minute zu verlieren, ihr Bündel packen und nach Berlin\oindex{Berlin@\textbf{Berlin}|pw} ko{\geminationm}en –
                    Alle, – es iſt hier Boden genug für ſie u. in Wien\oindex{Wien@\textbf{Wien}|pw} werden ſie \introOben{}ja\introOben{} doch alle verkü{\geminationm}ern! \pend
           \pstart
           Herzlichſt Ihr{\\[\baselineskip]}EMKafka\pend
           \leftskip=0em{}\pstart
           \noindent{}\label{T_L00158_1v}\edtext{Hotel \textsc{Wienerhof}\oindex{Wienerhof@\textbf{Wienerhof}|pw}, Marienstraße\oindex{Marienstrasse@\textbf{Marienstraße}|pw} 20}{\lemma{\textnormal{\emph{Hotel … 20}}}\Cendnote{\textnormal{quer am Rand der letzten Seite}}}\label{T_L00158_1h}\pend
                     \endnumbering\briefempfaengerindex{Schnitzler, Arthur@\textsc{Schnitzler, Arthur}!zzzKafka, Eduard Michael@\emph{von Eduard Michael Kafka}!1893-01-121@{12. 1. 1893}|)be}\mylabel{h}\end{ledgroupsized}  \newcommand{\dateiname}{L00158}\newcommand{\titel}{Eduard Michael Kafka an Arthur Schnitzler, 12. 1. 1893}\newcommand{\editorInnen}{Martin Anton Müller und Gerd-Hermann Susen}
            \footnotesize
\begin{ledgroupsized}[t]{11.5cm}
\doendnotes{C}
\end{ledgroupsized}
         %% latex-leseansicht-abspann.tex
%% Abspann für die Leseansicht.
%% Der Schalter \ifkorrekturansicht ist bereits durch den Vorspann gesetzt.

%% latex-abspann.tex
%% Gemeinsamer Abspann für Korrekturansicht und Leseansicht.
%% Setzt den Schalter \ifkorrekturansicht voraus (gesetzt in den
%% einbindenden Dateien latex-korrekturansicht-abspann.tex bzw.
%% latex-leseansicht-abspann.tex).
%% ---------------------------------------------------------------

\normalsize

% Das esempio-Environment wird nur in der Leseansicht benötigt
\ifkorrekturansicht\else
\newenvironment{esempio}[3]%
{
    \vspace{1.5ex}
    \rlap{\underline{#1}}
    \par
    \setlength{\parindent}{0cm}
    \nopagebreak
    \leftskip=#2cm
    \rightskip=#3cm
}
{
    \par
}
\fi

\doendnotes{C}
\bigskip
\vfill

\clearpage

\footnotesize

\ifkorrekturansicht
  \lohead{\textsc{register}}
\fi

% theindex-Environment neu definieren ohne reledmac
\makeatletter
\renewenvironment{theindex}{%
  \ifkorrekturansicht
    \section*{\indexname}%
  \else
    \subsubsection*{Index der erwähnten Entitäten}%
  \fi
  \setlength{\parindent}{0pt}%
  \setlength{\parskip}{0pt plus 0.3pt}%
  \let\item\@idxitem
}{%
  \ifkorrekturansicht\clearpage\fi
}
\makeatother

\IfFileExists{\jobname-pw.ind}{\input{\jobname-pw.ind}}{}

% Quellenangabe nur in der Leseansicht
\ifkorrekturansicht\else
% Fallback-Definitionen, falls die .tex-Datei \titel etc. nicht gesetzt hat
\providecommand{\titel}{}
\providecommand{\editorInnen}{}
\providecommand{\dateiname}{\jobname}

\vspace{3cm}

\vfill

\footnotesize
\textsc{Quelle}: \titel. Herausgegeben von {\editorInnen}. In: \emph{Arthur Schnitzler: Briefwechsel mit Autorinnen und Autoren}.
 Digitale Edition, https://schnitzler-briefe.acdh.oeaw.ac.at/{\dateiname}.html (Stand \today)
\fi

\end{document}


      