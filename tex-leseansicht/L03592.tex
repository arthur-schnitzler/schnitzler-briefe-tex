%% latex-leseansicht-vorspann.tex
%% Vorspann für die Leseansicht.
%% Lädt die gemeinsame Datei latex-vorspann.tex mit nicht gesetztem Schalter.

\newif\ifkorrekturansicht
\korrekturansichtfalse

\input{../tex-inputs/latex-vorspann}


\section[ Felix Salten an Arthur Schnitzler, 18. 6. 1930]{L03592 Felix Salten an Arthur Schnitzler,  18. 6. 1930}
\nopagebreak\mylabel{L03592v}
\rehead{ }\normalsize\beginnumbering\briefempfaengerindex{Schnitzler, Arthur@\textsc{Schnitzler, Arthur}!zzzSalten, Felix@\emph{von Felix Salten}!1930-06-181@{18. 6. 1930}|(be}
\toendnotes[C]{\smallbreak\pagebreak[2]}
\correspDesc{Versand  durch Felix Salten am 18. 6. 1930 in San Francisco
\newline{}Erhalt  durch Arthur Schnitzler im Zeitraum [1. 7. 1930
                  – 25. 7. 1930?] in Wien}\toendnotes[C]{\smallbreak}
\Standort{CUL, Schnitzler, B 89, B 2.}
\physDesc{Bildpostkarte, 253 Zeichen
\newline{}Handschrift: blaue Tinte, lateinische Kurrent
\newline{}Versand: 1) Stempel: »\nobreak{}Notify your correspondents of change of
                                       address\nobreak{}«.   2) Stempel: »\nobreak{}\oindex{San Francisco@\textbf{San Francisco}, \emph{Hauptstadt}|pwk}San Francisco, Calif. 3, Jun 18 1930, 9\textsuperscript{30}\nobreak{}«. 
\newline{}Ordnung: mit Bleistift von unbekannter Hand nummeriert: »306« }\pstart{}{\pb}EUROPE\oindex{Europa@\textbf{Europa}|pw} – AUSTRIA\oindex{Österreich@\textbf{Österreich}|pw}\pend{}\pstart{}Herrn\pend{}\pstart{}D\textsuperscript{r} Arthur Schnitzler\pend{}\pstart{}XVIII. Sternwartestrasse 71\oindex{Wien@\textbf{Wien}!XVIII., Währing@\textbf{XVIII., Währing}!Sternwartestraße 71@\textbf{Sternwartestraße 71}, \emph{Wohngebäude}|pw}\pend{}\pstart{}Wien\oindex{Wien@\textbf{Wien}, \emph{Verwaltungsgebiet}|pw}\pend{}{\bigskip}
\pstart
           \noindent{}\centering{}{\pb}\textcolor{gray}{\textbf{\begin{otherlanguage}{english}SAN FRANCISCO\oindex{San Francisco@\textbf{San Francisco}, \emph{Hauptstadt}|pw}’S FAMOUS CHINATOWN\oindex{Chinatown@\textbf{Chinatown}, \emph{Ehemaliger Ort}|pw}, SAN FRANCISCO\oindex{San Francisco@\textbf{San Francisco}, \emph{Hauptstadt}|pw}, CALIF.\oindex{Kalifornien@\textbf{Kalifornien}, \emph{Land}|pw}\end{otherlanguage}}}\pend
           \vspace{1em}
\pstart
           \noindent{}{\pb}Nach Ihnen fragen mich so viele
               Menschen. Neulich, in Pasadena\oindex{Pasadena@\textbf{Pasadena}|pw}\textcolor{gray}{,} auch Upton Sinclair\pwindex{Sinclair, Upton 20.\,9.\,1878 Baltimore – 25.\,11.\,1968 Bound Brook@\textsc{Sinclair, Upton} (20.\,9.\,1878 Baltimore – 25.\,11.\,1968 Bound Brook), \emph{Schriftsteller}|pw}. Sie
               können sich denken, wie gern ich von Ihnen spreche.\pend
           
\pstart
           Herzlichst Ihr {\\[\baselineskip]}\spacefill\mbox{Felix Salten}\pend
           \leftskip=0em{}
\pstart
           San Francisco\oindex{San Francisco@\textbf{San Francisco}, \emph{Hauptstadt}|pw}, 18\textcolor{gray}{-}6-30\pend
           \selectlanguage{ngerman}\endnumbering\briefempfaengerindex{Schnitzler, Arthur@\textsc{Schnitzler, Arthur}!zzzSalten, Felix@\emph{von Felix Salten}!1930-06-181@{18. 6. 1930}|)be}\mylabel{L03592h}  \newcommand{\dateiname}{L03592}\newcommand{\titel}{Felix Salten an Arthur Schnitzler, 18. 6. 1930}\newcommand{\editorInnen}{Martin Anton Müller und Laura Untner}%% latex-leseansicht-abspann.tex
%% Abspann für die Leseansicht.
%% Der Schalter \ifkorrekturansicht ist bereits durch den Vorspann gesetzt.

%% latex-abspann.tex
%% Gemeinsamer Abspann für Korrekturansicht und Leseansicht.
%% Setzt den Schalter \ifkorrekturansicht voraus (gesetzt in den
%% einbindenden Dateien latex-korrekturansicht-abspann.tex bzw.
%% latex-leseansicht-abspann.tex).
%% ---------------------------------------------------------------

\normalsize

% Das esempio-Environment wird nur in der Leseansicht benötigt
\ifkorrekturansicht\else
\newenvironment{esempio}[3]%
{
    \vspace{1.5ex}
    \rlap{\underline{#1}}
    \par
    \setlength{\parindent}{0cm}
    \nopagebreak
    \leftskip=#2cm
    \rightskip=#3cm
}
{
    \par
}
\fi

\doendnotes{C}
\bigskip
\vfill

\clearpage

\footnotesize

\ifkorrekturansicht
  \lohead{\textsc{register}}
\fi

% theindex-Environment neu definieren ohne reledmac
\makeatletter
\renewenvironment{theindex}{%
  \ifkorrekturansicht
    \section*{\indexname}%
  \else
    \subsubsection*{Index der erwähnten Entitäten}%
  \fi
  \setlength{\parindent}{0pt}%
  \setlength{\parskip}{0pt plus 0.3pt}%
  \let\item\@idxitem
}{%
  \ifkorrekturansicht\clearpage\fi
}
\makeatother

\IfFileExists{\jobname-pw.ind}{\input{\jobname-pw.ind}}{}

% Quellenangabe nur in der Leseansicht
\ifkorrekturansicht\else
% Fallback-Definitionen, falls die .tex-Datei \titel etc. nicht gesetzt hat
\providecommand{\titel}{}
\providecommand{\editorInnen}{}
\providecommand{\dateiname}{\jobname}

\vspace{3cm}

\vfill

\footnotesize
\textsc{Quelle}: \titel. Herausgegeben von {\editorInnen}. In: \emph{Arthur Schnitzler: Briefwechsel mit Autorinnen und Autoren}.
 Digitale Edition, https://schnitzler-briefe.acdh.oeaw.ac.at/{\dateiname}.html (Stand \today)
\fi

\end{document}


