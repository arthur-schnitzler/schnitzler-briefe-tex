%% latex-korrekturansicht-vorspann.tex
%% Vorspann für die Korrekturansicht.
%% Lädt die gemeinsame Datei latex-vorspann.tex mit gesetztem Schalter.

\newif\ifkorrekturansicht
\korrekturansichttrue

\input{../tex-inputs/latex-vorspann}


\section[ Felix Salten an Arthur Schnitzler, 18. 6. 1930]{L03592 Felix Salten an Arthur Schnitzler, 18. 6. 1930}
\nopagebreak\mylabel{L03592v}
\rehead{ }\normalsize\beginnumbering\briefempfaengerindex{Schnitzler, Arthur@\textsc{Schnitzler, Arthur}!zzzSalten, Felix@\emph{von Felix Salten}!1930-06-181@{18. 6. 1930}|(be}
\toendnotes[C]{\smallbreak\pagebreak[2]}\Standort{CUL, Schnitzler, B 89, B 2.}
\physDesc{Bildpostkarte, 253 Zeichen
\newline{}Handschrift: blaue Tinte, lateinische Kurrent
\newline{}Versand: 1) Stempel: »\nobreak{}Notify your correspondents of change of
                                       address\nobreak{}«.   2) Stempel: »\nobreak{}\oindex{San Francisco@\textbf{San Francisco}, \emph{P.PPLA2}|pwk}San Francisco, Calif. 3, Jun 18 1930, 9\textsuperscript{30}\nobreak{}«. 
\newline{}Ordnung: mit Bleistift von unbekannter Hand nummeriert: »306« }\pstart{}{\pb}EUROPE\oindex{Europa@\textbf{Europa}, \emph{Kontinent (A.KNT)}|pw} – AUSTRIA\oindex{Oesterreich@\textbf{Österreich}, \emph{A.PCLI}|pw}\pend{}\pstart{}Herrn\pend{}\pstart{}D\textsuperscript{r} Arthur Schnitzler\pend{}\pstart{}XVIII. Sternwartestrasse 71\oindex{Sternwartestrasse 71@\textbf{Sternwartestraße 71}, \emph{Wohngebäude (K.WHS)}|pw}\pend{}\pstart{}Wien\oindex{Wien@\textbf{Wien}, \emph{A.ADM2}|pw}\pend{}{\bigskip}
\pstart
           \noindent{}\centering{}{\pb}\textcolor{gray}{\textbf{\begin{otherlanguage}{english}SAN FRANCISCO\oindex{San Francisco@\textbf{San Francisco}, \emph{P.PPLA2}|pw}’S FAMOUS CHINATOWN\oindex{Chinatown@\textbf{Chinatown}, \emph{P.PPLX}|pw}, SAN FRANCISCO\oindex{San Francisco@\textbf{San Francisco}, \emph{P.PPLA2}|pw}, CALIF.\oindex{Kalifornien@\textbf{Kalifornien}, \emph{A.ADM1}|pw}\end{otherlanguage}}}\pend
           \vspace{1em}
\pstart
           \noindent{}{\pb}Nach Ihnen fragen mich so viele
               Menschen. Neulich, in Pasadena\oindex{Pasadena@\textbf{Pasadena}, \emph{P.PPL}|pw}\textcolor{gray}{,} auch Upton Sinclair\pwindex{Sinclair, Upton 20.09.1878 – 25.11.1968@\textsc{Sinclair, Upton} (20.09.1878 – 25.11.1968), \emph{Schriftsteller/Schriftstellerin}|pw}. Sie
               können sich denken, wie gern ich von Ihnen spreche.\pend
           
\pstart
           Herzlichst Ihr {\\[\baselineskip]}\spacefill\mbox{Felix Salten}\pend
           \leftskip=0em{}
\pstart
           San Francisco\oindex{San Francisco@\textbf{San Francisco}, \emph{P.PPLA2}|pw}, 18\textcolor{gray}{-}6-30\pend
           \selectlanguage{ngerman}\endnumbering\briefempfaengerindex{Schnitzler, Arthur@\textsc{Schnitzler, Arthur}!zzzSalten, Felix@\emph{von Felix Salten}!1930-06-181@{18. 6. 1930}|)be}\mylabel{L03592h}  \normalsize

\doendnotes{C}
\bigskip
\vfill

\clearpage

\footnotesize

\lohead{\textsc{register}}

% Definiere theindex-Environment komplett neu ohne reledmac
\makeatletter
\renewenvironment{theindex}{%
  \section*{\indexname}%
  \setlength{\parindent}{0pt}%
  \setlength{\parskip}{0pt plus 0.3pt}%
  \let\item\@idxitem
}{%
  \clearpage
}
\makeatother

\IfFileExists{\jobname-pw.ind}{\input{\jobname-pw.ind}}{}

\end{document}

      