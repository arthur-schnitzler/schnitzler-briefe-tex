%% latex-korrekturansicht-vorspann.tex
%% Vorspann für die Korrekturansicht.
%% Lädt die gemeinsame Datei latex-vorspann.tex mit gesetztem Schalter.

\newif\ifkorrekturansicht
\korrekturansichttrue

\input{../tex-inputs/latex-vorspann}


\section[Elsa Plessner an Arthur Schnitzler, 9. 1. 1897]{L03711 Elsa Plessner an Arthur Schnitzler, 9. 1. 1897}
\nopagebreak\mylabel{L03711v}
\rehead{ }\normalsize\beginnumbering\briefempfaengerindex{Schnitzler, Arthur@\textsc{Schnitzler, Arthur}!zzzPlessner, Elsa@\emph{von Elsa Plessner}!1897-01-091@{9. 1. 1897}|(be}
\toendnotes[C]{\smallbreak\pagebreak[2]}\Standort{DLA, A:Schnitzler, HS.1985.1.419.}
\physDesc{Brief,  Blätter, 3 Seiten, 1472 Zeichen
\newline{}Handschrift: , lateinische Kurrent}\toendnotes[C]{\smallbreak}
\pstart
           {\pb}Meran, Pension Wolf\oindex{Hotel Meranerhof@\textbf{Hotel Meranerhof}, \emph{Hotel (K.HTL)}|pw}, den
                     9. 1. 97.\pend
           
\pstart{}Hochverehrter Herr Doctor!\pend\vspace{0.5em}
\pstart
           Herzlichen Dank für Ihre freundlichen \label{K_L03711-1v}\edtext{Zeilen}{\lemma{\textnormal{\emph{Zeilen}}}\Cendnote{\textnormal{nicht überliefert}}}\label{K_L03711-1}. Sie
               sagen mir nichts Überraschendes. Eine halbe Stunde nachdem die Arbeit\pwindex{Orchideen [Schauspiel in drei Akten]@\emph{Orchideen [Schauspiel in drei Akten]}|pwv} an Sie abgeschickt war, habe ich es
               auch schon selber gewusst: – Wenn ich aufrichtig sein soll – im Schreiben selbst hat
               mein Gewissen »veto« geschrien. Aber ich schrie noch
               lauter – vide{ }\label{K_L03711-2v}\edtext{2 Briefe}{\lemma{\textnormal{\emph{2 Briefe}}}\Cendnote{\textnormal{Elsa Plessner an Arthur Schnitzler, 23. 12. 1896, und Elsa Plessner an Arthur Schnitzler, 29. 12. 1896.
               }}}\label{K_L03711-2} an Sie! – Na – \begin{otherlanguage}{french}passé\end{otherlanguage}! – – Daß es rapid abwärts ging, habe ich im
                  letzten Jahr genug oft bemerkt, nachdem ich kaum ein bißchen hinaufgekommen
               war, daß ich fertig bin, total fertig, weiß ich {\pb}\uline{seit einem halben Jahr} – also Ihr Urtheil über »Orchideen\pwindex{Orchideen [Schauspiel in drei Akten]@\emph{Orchideen [Schauspiel in drei Akten]}|pw}« nur das Siegel auf der Urkunde! Ich
               habe mich an die Arbeit \introOben{}»Orch.\pwindex{Orchideen [Schauspiel in drei Akten]@\emph{Orchideen [Schauspiel in drei Akten]}|pw}«\introOben{} geklammert – denn ich dachte entweder – oder! – Aber es ist
               – oder! Und das ist mir nicht neu! – Ich sehe es ja auch ganz deutlich ein und weiß
               trotzdem Sie daran zweifeln, wie recht Sie haben! – Also lassen wir die Tinte – ! Es
               kommt für mich nichts dabei heraus – das weiß ich auch besser wie Sie – wenn Sie {\pb}mich auch, gut wie Sie sind, mit einem talentirten Schüler vergleichen! –
               Ich weist ja auch, woher das kommt und Sie können es nicht wissen – Das ist eine
               Wurzelkrankheit bei mir! Darum lauter, mißlungene Blüten! – Und da hilft nichts! Also
               nochmals herzlichen, herzlichen Dank für Ihre Geduld und Güte! – Besser machen können
               Sie freilich nichts, als es ist! – Darum werde ich Sie auch in Zukunft verschonen –
               und sehen, wie s ohne Feder geht. Herzlichen innigen Dank!\pend
           \pstart \spacefill\mbox{Elsa Plessner}\pend{}\selectlanguage{ngerman}\endnumbering\briefempfaengerindex{Schnitzler, Arthur@\textsc{Schnitzler, Arthur}!zzzPlessner, Elsa@\emph{von Elsa Plessner}!1897-01-091@{9. 1. 1897}|)be}\mylabel{L03711h}
\begin{anhang}
\end{anhang}\normalsize

\doendnotes{C}
\bigskip
\vfill

\clearpage

\footnotesize

\lohead{\textsc{register}}

% Definiere theindex-Environment komplett neu ohne reledmac
\makeatletter
\renewenvironment{theindex}{%
  \section*{\indexname}%
  \setlength{\parindent}{0pt}%
  \setlength{\parskip}{0pt plus 0.3pt}%
  \let\item\@idxitem
}{%
  \clearpage
}
\makeatother

\IfFileExists{\jobname-pw.ind}{\input{\jobname-pw.ind}}{}

\end{document}

      