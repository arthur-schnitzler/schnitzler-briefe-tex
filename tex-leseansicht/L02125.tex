%% latex-korrekturansicht-vorspann.tex
%% Vorspann für die Korrekturansicht.
%% Lädt die gemeinsame Datei latex-vorspann.tex mit gesetztem Schalter.

\newif\ifkorrekturansicht
\korrekturansichttrue

\input{../tex-inputs/latex-vorspann}


\section[Arthur Schnitzler an Georg Engländer, 20. 4. 1913]{L02125 Arthur Schnitzler an Georg Engländer, 20. 4. 1913}
\nopagebreak\mylabel{L02125v}
\rehead{ }\normalsize\beginnumbering\briefempfaengerindex{Englaender, Georg@\textsc{Engländer, Georg}!zzzSchnitzler, Arthur@\emph{von Arthur Schnitzler}!1913-04-201@{20. 4. 1913}|(be}
\toendnotes[C]{\smallbreak\pagebreak[2]}\Standort{Wien, Österreichische Nationalbibliothek, 228/B8/1-3 LIT MAG.}
\physDesc{Postkarte, 290 Zeichen
\newline{}Handschrift: schwarze Tinte, deutsche Kurrent
\newline{}Versand: Stempel: »\nobreak{}\oindex{XVIII., Waehring@\textbf{XVIII., Währing}, \emph{A.ADM3}|pwk}18/1 WIEN 111, \textcolor{gray}{20. 4.} 13, 4\nobreak{}«.  }\pstart{}{\pb}\textcolor{gray}{\textbf{Dr. Arthur Schnitzler}}\pend{}\pstart{}\textcolor{gray}{\textbf{Wien XVIII. Sternwartestrasse 71\oindex{Sternwartestrasse 71@\textbf{Sternwartestraße 71}, \emph{Wohngebäude (K.WHS)}|pw}}}\pend{}{\bigskip}\pstart{}Herrn\pend{}\pstart{}\textsc{Georg Engländer}\pend{}\pstart{}Wien IX\oindex{IX., Alsergrund@\textbf{IX., Alsergrund}, \emph{A.ADM3}|pw}\pend{}\pstart{}Nußdorferstr \strikeout{\textcolor{gray}{4}} 10.\oindex{Nussdorfer Strasse@\textbf{Nussdorfer Straße}, \emph{Straße (K.STR)}|pw}\pend{}{\bigskip}\vspace{1em}
\pstart
           \raggedleft{}{\pb}20. 4. 913\pend
           
\pstart{}Sehr geehrter Herr,\pend\vspace{0.5em}
\pstart
           in Angelegenheit von Pet.\pwindex{Altenberg, Peter 09.03.1859 – 08.01.1919@\textsc{Altenberg, Peter} (09.03.1859 – 08.01.1919), \emph{Schriftsteller/Schriftstellerin}|pw} den ich heute ſprach
               (auch \textsc{Primar} Dr. R.\pwindex{Richter, Karl 09.03.1862 – 25.06.1937@\textsc{Richter, Karl} (09.03.1862 – 25.06.1937), \emph{Mediziner/Medizinerin, Sanatoriumsleiter/Sanatoriumsleiterin}|pw}{ }ſprach ich) möchte ich gern mit Ihnen reden. We{\geminationn} ich Sie morgen Montag um
                  1 Uhr oder um ½ 7–7 Abend erwarten darf, bedarf es
               keiner Antwort.\pend
           \pstart Ihr ſehr ergebner \spacefill\mbox{A. S.}\pend{}\selectlanguage{ngerman}\endnumbering\briefempfaengerindex{Englaender, Georg@\textsc{Engländer, Georg}!zzzSchnitzler, Arthur@\emph{von Arthur Schnitzler}!1913-04-201@{20. 4. 1913}|)be}\mylabel{L02125h}  \normalsize

\doendnotes{C}
\bigskip
\vfill

\clearpage

\footnotesize

\lohead{\textsc{register}}

% Definiere theindex-Environment komplett neu ohne reledmac
\makeatletter
\renewenvironment{theindex}{%
  \section*{\indexname}%
  \setlength{\parindent}{0pt}%
  \setlength{\parskip}{0pt plus 0.3pt}%
  \let\item\@idxitem
}{%
  \clearpage
}
\makeatother

\IfFileExists{\jobname-pw.ind}{\input{\jobname-pw.ind}}{}

\end{document}

      