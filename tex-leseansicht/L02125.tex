%% latex-leseansicht-vorspann.tex
%% Vorspann für die Leseansicht.
%% Lädt die gemeinsame Datei latex-vorspann.tex mit nicht gesetztem Schalter.

\newif\ifkorrekturansicht
\korrekturansichtfalse

\input{../tex-inputs/latex-vorspann}


\section[Arthur Schnitzler an Georg Engländer, 20. 4. 1913]{L02125 Arthur Schnitzler an Georg Engländer, 20. 4. 1913}
\nopagebreak\mylabel{L02125v}
\rehead{ }\normalsize\beginnumbering\briefempfaengerindex{Engländer, Georg@\textsc{Engländer, Georg}!zzzSchnitzler, Arthur@\emph{von Arthur Schnitzler}!1913-04-201@{20. 4. 1913}|(be}
\toendnotes[C]{\smallbreak\pagebreak[2]}
\correspDesc{Versand  durch Arthur Schnitzler am 20. 4. 1913 in Wien
\newline{}Erhalt  durch Georg Engländer im Zeitraum [20. 4. 1913
                  – 24. 4. 1913?] in Wien}\toendnotes[C]{\smallbreak}
\Standort{Wien, Österreichische Nationalbibliothek, 228/B8/1-3 LIT MAG.}
\physDesc{Postkarte, 290 Zeichen
\newline{}Handschrift: schwarze Tinte, deutsche Kurrent
\newline{}Versand: Stempel: »\nobreak{}\oindex{XVIII., Währing@\textbf{XVIII., Währing}, \emph{Verwaltungsgebiet}|pwk}18/1 WIEN 111, \textcolor{gray}{20. 4.} 13, 4\nobreak{}«.  }\pstart{}{\pb}\textcolor{gray}{\textbf{Dr. Arthur Schnitzler}}\pend{}\pstart{}\textcolor{gray}{\textbf{Wien XVIII. Sternwartestrasse 71\oindex{Wien@\textbf{Wien}!XVIII., Währing@\textbf{XVIII., Währing}!Sternwartestraße 71@\textbf{Sternwartestraße 71}, \emph{Wohngebäude}|pw}}}\pend{}{\bigskip}\pstart{}Herrn\pend{}\pstart{}\textsc{Georg Engländer}\pend{}\pstart{}Wien IX\oindex{IX., Alsergrund@\textbf{IX., Alsergrund}, \emph{Verwaltungsgebiet}|pw}\pend{}\pstart{}Nußdorferstr \strikeout{\textcolor{gray}{4}} 10.\oindex{Wien@\textbf{Wien}!IX., Alsergrund@\textbf{IX., Alsergrund}!Nussdorfer Straße@\textbf{Nussdorfer Straße}, \emph{Straße}|pw}\pend{}{\bigskip}\vspace{1em}
\pstart
           \raggedleft{}{\pb}20. 4. 913\pend
           
\pstart{}Sehr geehrter Herr,\pend\vspace{0.5em}
\pstart
           in Angelegenheit von Pet.\pwindex{Altenberg, Peter 9.\,3.\,1859 Wien – 8.\,1.\,1919 ebd.@\textsc{Altenberg, Peter} (9.\,3.\,1859 Wien – 8.\,1.\,1919 ebd.), \emph{Schriftsteller}|pw} den ich heute{ }ſprach
               (auch \textsc{Primar} Dr. R.\pwindex{Richter, Karl 9.\,3.\,1862 Bruntál – 25.\,6.\,1937 Wien@\textsc{Richter, Karl} (9.\,3.\,1862 Bruntál – 25.\,6.\,1937 Wien), \emph{Mediziner, Sanatoriumsleiter}|pw}{ }ſprach ich) möchte ich gern mit Ihnen reden. We{\geminationn} ich Sie morgen Montag um
                  1 Uhr oder um ½ 7–7 Abend erwarten darf, bedarf es
               keiner Antwort.\pend
           \pstart Ihr{ }ſehr ergebner \spacefill\mbox{A. S.}\pend{}\selectlanguage{ngerman}\endnumbering\briefempfaengerindex{Engländer, Georg@\textsc{Engländer, Georg}!zzzSchnitzler, Arthur@\emph{von Arthur Schnitzler}!1913-04-201@{20. 4. 1913}|)be}\mylabel{L02125h}  \newcommand{\dateiname}{L02125}\newcommand{\titel}{Arthur Schnitzler an Georg Engländer, 20. 4. 1913}\newcommand{\editorInnen}{Martin Anton Müller und Gerd-Hermann Susen}%% latex-leseansicht-abspann.tex
%% Abspann für die Leseansicht.
%% Der Schalter \ifkorrekturansicht ist bereits durch den Vorspann gesetzt.

%% latex-abspann.tex
%% Gemeinsamer Abspann für Korrekturansicht und Leseansicht.
%% Setzt den Schalter \ifkorrekturansicht voraus (gesetzt in den
%% einbindenden Dateien latex-korrekturansicht-abspann.tex bzw.
%% latex-leseansicht-abspann.tex).
%% ---------------------------------------------------------------

\normalsize

% Das esempio-Environment wird nur in der Leseansicht benötigt
\ifkorrekturansicht\else
\newenvironment{esempio}[3]%
{
    \vspace{1.5ex}
    \rlap{\underline{#1}}
    \par
    \setlength{\parindent}{0cm}
    \nopagebreak
    \leftskip=#2cm
    \rightskip=#3cm
}
{
    \par
}
\fi

\doendnotes{C}
\bigskip
\vfill

\clearpage

\footnotesize

\ifkorrekturansicht
  \lohead{\textsc{register}}
\fi

% theindex-Environment neu definieren ohne reledmac
\makeatletter
\renewenvironment{theindex}{%
  \ifkorrekturansicht
    \section*{\indexname}%
  \else
    \subsubsection*{Index der erwähnten Entitäten}%
  \fi
  \setlength{\parindent}{0pt}%
  \setlength{\parskip}{0pt plus 0.3pt}%
  \let\item\@idxitem
}{%
  \ifkorrekturansicht\clearpage\fi
}
\makeatother

\IfFileExists{\jobname-pw.ind}{\input{\jobname-pw.ind}}{}

% Quellenangabe nur in der Leseansicht
\ifkorrekturansicht\else
% Fallback-Definitionen, falls die .tex-Datei \titel etc. nicht gesetzt hat
\providecommand{\titel}{}
\providecommand{\editorInnen}{}
\providecommand{\dateiname}{\jobname}

\vspace{3cm}

\vfill

\footnotesize
\textsc{Quelle}: \titel. Herausgegeben von {\editorInnen}. In: \emph{Arthur Schnitzler: Briefwechsel mit Autorinnen und Autoren}.
 Digitale Edition, https://schnitzler-briefe.acdh.oeaw.ac.at/{\dateiname}.html (Stand \today)
\fi

\end{document}


