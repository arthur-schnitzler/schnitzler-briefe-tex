%% latex-korrekturansicht-vorspann.tex
%% Vorspann für die Korrekturansicht.
%% Lädt die gemeinsame Datei latex-vorspann.tex mit gesetztem Schalter.

\newif\ifkorrekturansicht
\korrekturansichttrue

\input{../tex-inputs/latex-vorspann}


\section[Arthur Schnitzler an Felix Salten, {[}21. 5. 1892?{]}]{L02956 Arthur Schnitzler an Felix Salten, {[}21. 5. 1892?{]}}
\nopagebreak\mylabel{L02956v}
\rehead{ }\normalsize\beginnumbering\briefempfaengerindex{Salten, Felix@\textsc{Salten, Felix}!zzzSchnitzler, Arthur@\emph{von Arthur Schnitzler}!1892-05-211@{{[}21. 5. 1892?{]}}|(be}
\toendnotes[C]{\smallbreak\pagebreak[2]}\Standort{Wienbibliothek im Rathaus, ZPH 1681, 2.1.516.}
\physDesc{Brief, 1 Blatt, 2 Seiten, 297 Zeichen
\newline{}Handschrift: Bleistift, deutsche Kurrent
\newline{}Ordnung: mit Bleistift von unbekannter Hand nummeriert: »24« }\toendnotes[C]{\smallbreak}
\pstart
           \raggedleft{}{\pb}\uline{\label{K_L02956-1v}\edtext{Samſtag}{\lemma{\textnormal{\emph{Samſtag}}}\Cendnote{\textnormal{Das Erscheinen des Artikels\pwindex{Theater-Briefe. Wien@\emph{Theater-Briefe. Wien}|pwkv} von Bahr\pwindex{Bahr, Hermann 19.07.1863 – 15.01.1934@\textsc{Bahr, Hermann} (19.07.1863 – 15.01.1934), \emph{Schriftsteller/Schriftstellerin, Kritiker/Kritikerin}|pwk} gibt eine zeitliche Einordnung.}}}\label{K_L02956-1}.}\pend
           
\pstart{}Lieber Freund,\pend\vspace{0.5em}
\pstart
           es wäre mir ſehr angenehm, Sie beim Schneider\oindex{Cafe Schneider@\textbf{Café Schneider}, \emph{Kaffeehaus (K.KAF)}|pw}{ }heut{ }Abend zu ſehen (ich habe einen \label{K_L02956-2v}\edtext{Sitz ins Theater\oindex{Internationales Ausstellungstheater im k.k. Prater@\textbf{Internationales Ausstellungstheater im k.k. Prater}, \emph{Theater (K.THE)}|pwv}}{\lemma{\textnormal{\emph{Sitz ins Theater}}}\Cendnote{\textnormal{Siehe A. S.: \emph{Tagebuch}, 21. 5. 1892.
               }}}\label{K_L02956-2}.)\pend
           
\pstart
           – Ich werde wahrſcheinlich \uline{morgen}{ }Nachmttg frei ſein.\pend
           
\pstart
           {\pb}– Eben den \label{K_L02956-3v}\edtext{Artikel\pwindex{Theater-Briefe. Wien@\emph{Theater-Briefe. Wien}|pwv}}{\lemma{\textnormal{\emph{Artikel}}}\Cendnote{\textnormal{Hermann Bahr\pwindex{Bahr, Hermann 19.07.1863 – 15.01.1934@\textsc{Bahr, Hermann} (19.07.1863 – 15.01.1934), \emph{Schriftsteller/Schriftstellerin, Kritiker/Kritikerin}|pwk}: \emph{Theater-Briefe. Wien}\pwindex{Theater-Briefe. Wien@\emph{Theater-Briefe. Wien}|pwk}. In: \emph{Allgemeine Theater-Revue für Bühne und Welt}\pwindex{Allgemeine Theater-Revue fuer Buehne und Welt@\emph{Allgemeine Theater-Revue für Bühne und Welt}|pwk}, Jg. 1,
                     Nr. 4, Mitte Mai 1892, S. 40–41.}}}\label{K_L02956-3}
               von \textsc{Bahr\pwindex{Bahr, Hermann 19.07.1863 – 15.01.1934@\textsc{Bahr, Hermann} (19.07.1863 – 15.01.1934), \emph{Schriftsteller/Schriftstellerin, Kritiker/Kritikerin}|pw}} geleſen in der \textsc{Theater revue\pwindex{Allgemeine Theater-Revue fuer Buehne und Welt@\emph{Allgemeine Theater-Revue für Bühne und Welt}|pw}}, den ich ſehr luſtig finde; es iſt wenigſtens echter \textcolor{gray}{Bahr\pwindex{Bahr, Hermann 19.07.1863 – 15.01.1934@\textsc{Bahr, Hermann} (19.07.1863 – 15.01.1934), \emph{Schriftsteller/Schriftstellerin, Kritiker/Kritikerin}|pw}}.–\pend
           
\pstart
           Herzlichſt Ihr {\\[\baselineskip]}\spacefill\mbox{Arth}\pend
           \leftskip=0em{}\selectlanguage{ngerman}\endnumbering\briefempfaengerindex{Salten, Felix@\textsc{Salten, Felix}!zzzSchnitzler, Arthur@\emph{von Arthur Schnitzler}!1892-05-211@{{[}21. 5. 1892?{]}}|)be}\mylabel{L02956h}  \normalsize

\doendnotes{C}
\bigskip
\vfill

\clearpage

\footnotesize

\lohead{\textsc{register}}

% Definiere theindex-Environment komplett neu ohne reledmac
\makeatletter
\renewenvironment{theindex}{%
  \section*{\indexname}%
  \setlength{\parindent}{0pt}%
  \setlength{\parskip}{0pt plus 0.3pt}%
  \let\item\@idxitem
}{%
  \clearpage
}
\makeatother

\IfFileExists{\jobname-pw.ind}{\input{\jobname-pw.ind}}{}

\end{document}

      