%% latex-leseansicht-vorspann.tex
%% Vorspann für die Leseansicht.
%% Lädt die gemeinsame Datei latex-vorspann.tex mit nicht gesetztem Schalter.

\newif\ifkorrekturansicht
\korrekturansichtfalse

\input{../tex-inputs/latex-vorspann}

\begin{center}
            \textcolor{red}{ENTWURF, NICHT FERTIG KORRIGIERT}
                      \end{center}
            
         
         \renewcommand{\erwaehntePersonen}{Personen: Hermann Bahr, Felix Salten}
         \renewcommand{\erwaehnteOrte}{Orte: Internationales Ausstellungstheater im k.k. Prater, Wien}
         \renewcommand{\erwaehnteWerke}{Werke: Allgemeine Theater-Revue für Bühne und Welt, Theater-Briefe. Wien, Thermidor. Drama in vier Akten}
               \section[Arthur Schnitzler an Felix Salten, {[}21. 5. 1892?{]}]{ Arthur Schnitzler an Felix Salten, {[}21. 5. 1892?{]}}\nopagebreak\mylabel{v}\rehead{ }\begin{ledgroupsized}[t]{13cm}\normalsize\beginnumbering \toendnotes[C]{\smallbreak\pagebreak[2]} \Standort{Wienbibliothek im Rathaus, ZPH 1681, 2.1.516.}
\physDesc{
\newline{}Handschrift: , deutsche Kurrent}\toendnotes[C]{\smallbreak}\pstart
           \raggedleft{}{\pb}\label{K_L02956-2v}\edtext{Samſtag}{\lemma{\textnormal{\emph{Samſtag}}}\Cendnote{\textnormal{Das Erscheinen des Artikel\pwindex{Bahr, Hermann 19.07.1863 – 15.01.1934@\textsc{Bahr, Hermann} (19.07.1863 – 15.01.1934), \emph{Schriftsteller, Kritiker}!Theater-Briefe. Wien1892-05-15@\strich\emph{Theater-Briefe. Wien} {[}1892-05-15{]}|pwkv}s von Bahr\pwindex{Bahr, Hermann 19.07.1863 – 15.01.1934@\textsc{Bahr, Hermann} (19.07.1863 – 15.01.1934), \emph{Schriftsteller, Kritiker}|pwk} gibt eine zeitliche Einordnung, mit der auch
                        der ansonsten schwer zu entziffernde Titel des Theaterstücks gelesen werden
                        kann.}}}\label{K_L02956-2h}. \pend
           \pstart{}Lieber Freund,\pend\pstart
           es wäre mir ſehr angenehm, Sie beim Thmid\textcolor{gray}{or}\pwindex{\textcolor{red}{\textsuperscript{XXXX1 indx}}!Thermidor. Drama in vier AktenNone@\strich\emph{Thermidor. Drama in vier Akten} {[}None{]}|pw} heut Abend zu ſehen (ich habe einen Sitz ins
                  Theater\oindex{Internationales Ausstellungstheater im k.k. Prater@\textbf{Internationales Ausstellungstheater im k.k. Prater}|pwv}.) – Ich werde
               wahrſcheinlich \uline{morgen} Nachmttg frei ſein\pend
           \pstart
           {\pb}– Eben den \label{K_L02956-1v}\edtext{Artikel\pwindex{Bahr, Hermann 19.07.1863 – 15.01.1934@\textsc{Bahr, Hermann} (19.07.1863 – 15.01.1934), \emph{Schriftsteller, Kritiker}!Theater-Briefe. Wien1892-05-15@\strich\emph{Theater-Briefe. Wien} {[}1892-05-15{]}|pwv}}{\lemma{\textnormal{\emph{Artikel}}}\Cendnote{\textnormal{Hermann Bahr\pwindex{Bahr, Hermann 19.07.1863 – 15.01.1934@\textsc{Bahr, Hermann} (19.07.1863 – 15.01.1934), \emph{Schriftsteller, Kritiker}|pwk}: \emph{Theater-Briefe. Wien}\pwindex{Bahr, Hermann 19.07.1863 – 15.01.1934@\textsc{Bahr, Hermann} (19.07.1863 – 15.01.1934), \emph{Schriftsteller, Kritiker}!Theater-Briefe. Wien1892-05-15@\strich\emph{Theater-Briefe. Wien} {[}1892-05-15{]}|pwk}. In: \emph{Allgemeine Theater-Revue für Bühne und Welt}\pwindex{?? Werk@Nicht ermittelte Verfasserinnen und Verfasser!Allgemeine Theater-Revue fuer Buehne und Welt1892-04-01 – 1892-06-30@\emph{Allgemeine Theater-Revue für Bühne und Welt} {[}1892-04-01 – 1892-06-30{]}|pwk},
                     Jg. 1, Nr. 4, Mitte Mai 1892,
                  S. 40–41.}}}\label{K_L02956-1h} von \textsc{Bahr\pwindex{Bahr, Hermann 19.07.1863 – 15.01.1934@\textsc{Bahr, Hermann} (19.07.1863 – 15.01.1934), \emph{Schriftsteller, Kritiker}|pw}} geleſen in der \textsc{Theaterrevue\pwindex{?? Werk@Nicht ermittelte Verfasserinnen und Verfasser!Allgemeine Theater-Revue fuer Buehne und Welt1892-04-01 – 1892-06-30@\emph{Allgemeine Theater-Revue für Bühne und Welt} {[}1892-04-01 – 1892-06-30{]}|pw}}, den ich ſehr luſtig finde; es iſt wenigſtens echter \textcolor{gray}{Bahr}\pwindex{Bahr, Hermann 19.07.1863 – 15.01.1934@\textsc{Bahr, Hermann} (19.07.1863 – 15.01.1934), \emph{Schriftsteller, Kritiker}|pw}.– \pend
           \pstart
           Herzlichſt Ihr {\\[\baselineskip]}\spacefill\mbox{Arth}\pend
           \leftskip=0em{}
         
         \endnumbering\mylabel{h}\end{ledgroupsized}\begin{anhang}\end{anhang}\newcommand{\dateiname}{L02956}\newcommand{\titel}{Arthur Schnitzler an Felix Salten, [21. 5. 1892?]}\newcommand{\editorInnen}{Martin Anton Müller und Laura Untner}%% latex-leseansicht-abspann.tex
%% Abspann für die Leseansicht.
%% Der Schalter \ifkorrekturansicht ist bereits durch den Vorspann gesetzt.

%% latex-abspann.tex
%% Gemeinsamer Abspann für Korrekturansicht und Leseansicht.
%% Setzt den Schalter \ifkorrekturansicht voraus (gesetzt in den
%% einbindenden Dateien latex-korrekturansicht-abspann.tex bzw.
%% latex-leseansicht-abspann.tex).
%% ---------------------------------------------------------------

\normalsize

% Das esempio-Environment wird nur in der Leseansicht benötigt
\ifkorrekturansicht\else
\newenvironment{esempio}[3]%
{
    \vspace{1.5ex}
    \rlap{\underline{#1}}
    \par
    \setlength{\parindent}{0cm}
    \nopagebreak
    \leftskip=#2cm
    \rightskip=#3cm
}
{
    \par
}
\fi

\doendnotes{C}
\bigskip
\vfill

\clearpage

\footnotesize

\ifkorrekturansicht
  \lohead{\textsc{register}}
\fi

% theindex-Environment neu definieren ohne reledmac
\makeatletter
\renewenvironment{theindex}{%
  \ifkorrekturansicht
    \section*{\indexname}%
  \else
    \subsubsection*{Index der erwähnten Entitäten}%
  \fi
  \setlength{\parindent}{0pt}%
  \setlength{\parskip}{0pt plus 0.3pt}%
  \let\item\@idxitem
}{%
  \ifkorrekturansicht\clearpage\fi
}
\makeatother

\IfFileExists{\jobname-pw.ind}{\input{\jobname-pw.ind}}{}

% Quellenangabe nur in der Leseansicht
\ifkorrekturansicht\else
% Fallback-Definitionen, falls die .tex-Datei \titel etc. nicht gesetzt hat
\providecommand{\titel}{}
\providecommand{\editorInnen}{}
\providecommand{\dateiname}{\jobname}

\vspace{3cm}

\vfill

\footnotesize
\textsc{Quelle}: \titel. Herausgegeben von {\editorInnen}. In: \emph{Arthur Schnitzler: Briefwechsel mit Autorinnen und Autoren}.
 Digitale Edition, https://schnitzler-briefe.acdh.oeaw.ac.at/{\dateiname}.html (Stand \today)
\fi

\end{document}


      