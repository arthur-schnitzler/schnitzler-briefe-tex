%% latex-leseansicht-vorspann.tex
%% Vorspann für die Leseansicht.
%% Lädt die gemeinsame Datei latex-vorspann.tex mit nicht gesetztem Schalter.

\newif\ifkorrekturansicht
\korrekturansichtfalse

\input{../tex-inputs/latex-vorspann}


\section[Arthur Schnitzler an Felix Salten, {{[}}21. 5. 1892?{{]}}]{L02956 Arthur Schnitzler an Felix Salten, {[}21. 5. 1892?{]}}
\nopagebreak\mylabel{L02956v}
\rehead{ }\normalsize\beginnumbering\briefempfaengerindex{Salten, Felix@\textsc{Salten, Felix}!zzzSchnitzler, Arthur@\emph{von Arthur Schnitzler}!1892-05-211@{{[}21. 5. 1892?{]}}|(be}
\toendnotes[C]{\smallbreak\pagebreak[2]}
\correspDesc{Versand  durch Arthur Schnitzler am [21. 5. 1892?] in Wien
\newline{}Erhalt  durch Felix Salten am [21. 5. 1892?] in Wien}\toendnotes[C]{\smallbreak}
\Standort{Wienbibliothek im Rathaus, ZPH 1681, 2.1.516.}
\physDesc{Brief, 1 Blatt, 2 Seiten, 297 Zeichen
\newline{}Handschrift: Bleistift, deutsche Kurrent
\newline{}Ordnung: mit Bleistift von unbekannter Hand nummeriert: »24« }\toendnotes[C]{\smallbreak}
\pstart
           \raggedleft{}{\pb}\uline{\label{K_L02956-1v}\edtext{Samſtag}{\lemma{\textnormal{\emph{Samstag}}}\Cendnote{\textnormal{Das Erscheinen des Artikels\pwindex{Bahr, Hermann 19.\,7.\,1863 Linz – 15.\,1.\,1934 München@\textsc{Bahr, Hermann} (19.\,7.\,1863 Linz – 15.\,1.\,1934 München), \emph{Schriftsteller, Kritiker}!Theater-Briefe. Wien@\strich\emph{Theater-Briefe. Wien}|pwkv} von Bahr\pwindex{Bahr, Hermann 19.\,7.\,1863 Linz – 15.\,1.\,1934 München@\textsc{Bahr, Hermann} (19.\,7.\,1863 Linz – 15.\,1.\,1934 München), \emph{Schriftsteller, Kritiker}|pwk} gibt eine zeitliche Einordnung.}}}\label{K_L02956-1}.}\pend
           
\pstart{}Lieber Freund,\pend\vspace{0.5em}
\pstart
           es wäre mir{ }ſehr angenehm, Sie beim Schneider\oindex{Café Schneider@\textbf{Café Schneider}, \emph{Kaffeehaus}|pw}{ }heut{ }Abend zu{ }ſehen (ich habe einen \label{K_L02956-2v}\edtext{Sitz ins Theater\oindex{Wien@\textbf{Wien}!II., Leopoldstadt@\textbf{II., Leopoldstadt}!Internationales Ausstellungstheater im k.k. Prater@\textbf{Internationales Ausstellungstheater im k.k. Prater}, \emph{Theater}|pwv}}{\lemma{\textnormal{\emph{Sitz ins Theater}}}\Cendnote{\textnormal{Siehe A. S.: \emph{Tagebuch}, 21. 5. 1892.
               }}}\label{K_L02956-2}.)\pend
           
\pstart
           – Ich werde wahrſcheinlich \uline{morgen}{ }Nachmttg frei{ }ſein.\pend
           
\pstart
           {\pb}– Eben den \label{K_L02956-3v}\edtext{Artikel\pwindex{Bahr, Hermann 19.\,7.\,1863 Linz – 15.\,1.\,1934 München@\textsc{Bahr, Hermann} (19.\,7.\,1863 Linz – 15.\,1.\,1934 München), \emph{Schriftsteller, Kritiker}!Theater-Briefe. Wien@\strich\emph{Theater-Briefe. Wien}|pwv}}{\lemma{\textnormal{\emph{Artikel}}}\Cendnote{\textnormal{Hermann Bahr\pwindex{Bahr, Hermann 19.\,7.\,1863 Linz – 15.\,1.\,1934 München@\textsc{Bahr, Hermann} (19.\,7.\,1863 Linz – 15.\,1.\,1934 München), \emph{Schriftsteller, Kritiker}|pwk}: \emph{Theater-Briefe. Wien}\pwindex{Bahr, Hermann 19.\,7.\,1863 Linz – 15.\,1.\,1934 München@\textsc{Bahr, Hermann} (19.\,7.\,1863 Linz – 15.\,1.\,1934 München), \emph{Schriftsteller, Kritiker}!Theater-Briefe. Wien@\strich\emph{Theater-Briefe. Wien}|pwk}. In: \emph{Allgemeine Theater-Revue für Bühne und Welt}\pwindex{Allgemeine Theater-Revue für Bühne und Welt@\emph{Allgemeine Theater-Revue für Bühne und Welt}|pwk}, Jg. 1,
                     Nr. 4, Mitte Mai 1892, S. 40–41.}}}\label{K_L02956-3}
               von \textsc{Bahr\pwindex{Bahr, Hermann 19.\,7.\,1863 Linz – 15.\,1.\,1934 München@\textsc{Bahr, Hermann} (19.\,7.\,1863 Linz – 15.\,1.\,1934 München), \emph{Schriftsteller, Kritiker}|pw}} geleſen in der \textsc{Theater revue\pwindex{Allgemeine Theater-Revue für Bühne und Welt@\emph{Allgemeine Theater-Revue für Bühne und Welt}|pw}}, den ich{ }ſehr luſtig finde; es iſt wenigſtens echter \textcolor{gray}{Bahr\pwindex{Bahr, Hermann 19.\,7.\,1863 Linz – 15.\,1.\,1934 München@\textsc{Bahr, Hermann} (19.\,7.\,1863 Linz – 15.\,1.\,1934 München), \emph{Schriftsteller, Kritiker}|pw}}.–\pend
           
\pstart
           Herzlichſt Ihr {\\[\baselineskip]}\spacefill\mbox{Arth}\pend
           \leftskip=0em{}\selectlanguage{ngerman}\endnumbering\briefempfaengerindex{Salten, Felix@\textsc{Salten, Felix}!zzzSchnitzler, Arthur@\emph{von Arthur Schnitzler}!1892-05-211@{{[}21. 5. 1892?{]}}|)be}\mylabel{L02956h}  \newcommand{\dateiname}{L02956}\newcommand{\titel}{Arthur Schnitzler an Felix Salten, [21. 5. 1892?]}\newcommand{\editorInnen}{Martin Anton Müller und Laura Untner}%% latex-leseansicht-abspann.tex
%% Abspann für die Leseansicht.
%% Der Schalter \ifkorrekturansicht ist bereits durch den Vorspann gesetzt.

%% latex-abspann.tex
%% Gemeinsamer Abspann für Korrekturansicht und Leseansicht.
%% Setzt den Schalter \ifkorrekturansicht voraus (gesetzt in den
%% einbindenden Dateien latex-korrekturansicht-abspann.tex bzw.
%% latex-leseansicht-abspann.tex).
%% ---------------------------------------------------------------

\normalsize

% Das esempio-Environment wird nur in der Leseansicht benötigt
\ifkorrekturansicht\else
\newenvironment{esempio}[3]%
{
    \vspace{1.5ex}
    \rlap{\underline{#1}}
    \par
    \setlength{\parindent}{0cm}
    \nopagebreak
    \leftskip=#2cm
    \rightskip=#3cm
}
{
    \par
}
\fi

\doendnotes{C}
\bigskip
\vfill

\clearpage

\footnotesize

\ifkorrekturansicht
  \lohead{\textsc{register}}
\fi

% theindex-Environment neu definieren ohne reledmac
\makeatletter
\renewenvironment{theindex}{%
  \ifkorrekturansicht
    \section*{\indexname}%
  \else
    \subsubsection*{Index der erwähnten Entitäten}%
  \fi
  \setlength{\parindent}{0pt}%
  \setlength{\parskip}{0pt plus 0.3pt}%
  \let\item\@idxitem
}{%
  \ifkorrekturansicht\clearpage\fi
}
\makeatother

\IfFileExists{\jobname-pw.ind}{\input{\jobname-pw.ind}}{}

% Quellenangabe nur in der Leseansicht
\ifkorrekturansicht\else
% Fallback-Definitionen, falls die .tex-Datei \titel etc. nicht gesetzt hat
\providecommand{\titel}{}
\providecommand{\editorInnen}{}
\providecommand{\dateiname}{\jobname}

\vspace{3cm}

\vfill

\footnotesize
\textsc{Quelle}: \titel. Herausgegeben von {\editorInnen}. In: \emph{Arthur Schnitzler: Briefwechsel mit Autorinnen und Autoren}.
 Digitale Edition, https://schnitzler-briefe.acdh.oeaw.ac.at/{\dateiname}.html (Stand \today)
\fi

\end{document}


