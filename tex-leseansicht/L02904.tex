%% latex-leseansicht-vorspann.tex
%% Vorspann für die Leseansicht.
%% Lädt die gemeinsame Datei latex-vorspann.tex mit nicht gesetztem Schalter.

\newif\ifkorrekturansicht
\korrekturansichtfalse

\input{../tex-inputs/latex-vorspann}


\section[ Paul Goldmann an Arthur Schnitzler, 11. 2. 1900]{L02904 Paul Goldmann an Arthur Schnitzler,  11. 2. 1900}
\nopagebreak\mylabel{L02904v}
\rehead{ }\normalsize\beginnumbering\briefempfaengerindex{Schnitzler, Arthur@\textsc{Schnitzler, Arthur}!zzzGoldmann, Paul@\emph{von Paul Goldmann}!1900-02-113@{11. 2. 1900}|(be}
\toendnotes[C]{\smallbreak\pagebreak[2]}
\correspDesc{Versand  durch Paul Goldmann am 11. 2. 1900 in Berlin
\newline{}Erhalt  durch Arthur Schnitzler im Zeitraum [12. 2. 1900
                  – 16. 2. 1900?] in Wien}\toendnotes[C]{\smallbreak}
\Standort{DLA, A:Schnitzler, HS.NZ85.1.3170.}
\physDesc{Brief, 2 Blätter, 8 Seiten, 3912 Zeichen
\newline{}Handschrift: schwarze Tinte, deutsche Kurrent}\toendnotes[C]{\smallbreak}
\pstart
           \centering{}{\pb}\textcolor{gray}{\textbf{\textbf{HOTEL SAXONIA\oindex{Hotel Saxonia@\textbf{Hotel Saxonia}, \emph{Hotel}|pw}}}}\pend
           
\pstart
           \raggedleft{}\textcolor{gray}{\textbf{am Potsdamer Platz\oindex{Potsdamer Platz@\textbf{Potsdamer Platz}, \emph{Platz}|pw} und
                        Thiergarten\oindex{Tiergarten@\textbf{Tiergarten}, \emph{Ehemaliger Ort}|pw}}}\pend
           
\pstart
           \centering{}\textcolor{gray}{\textbf{D. W. SCHRÖDER\pwindex{Schröder, D. W. @\textsc{Schröder, D. W.}, \emph{Hotelbesitzer/Hotelbesitzerin}|pw}.}}\pend
           
\pstart
           \textcolor{gray}{\textbf{Fernsprecher:}}\pend
           
\pstart
           \textcolor{gray}{\textbf{\textbf{Amt VI. No. 2838.}}}\pend
           
\pstart
           \raggedleft{}\textcolor{gray}{\textbf{\emph{BERLIN W.}\oindex{Berlin@\textbf{Berlin}, \emph{Hauptstadt}|pw}, den}}{ }11. Februar \textcolor{gray}{\textbf{1}}900.\pend
           
\pstart
           \raggedleft{}\textcolor{gray}{\textbf{Königgrätzerstrasse 10\oindex{Stresemannstraße@\textbf{Stresemannstraße}, \emph{Straße}|pw}.}}\pend
           
\pstart{}Mein lieber Freund,\pend\vspace{0.5em}
\pstart
           Ich danke Dir von Herzen für Dein Stück\pwindex{Schnitzler, Arthur 15.\,5.\,1862 Wien – 21.\,10.\,1931 ebd.@\textsc{Schnitzler, Arthur} (15.\,5.\,1862 Wien – 21.\,10.\,1931 ebd.), \emph{Schriftsteller, Mediziner}!Schleier der Beatrice. Schauspiel in fünf Akten@\strich\emph{Der Schleier der Beatrice. Schauspiel in fünf Akten}|pwv}. In den Nächten, die auf die{ }ſchwere Arbeit dieſer Tage folgten, habe
               ich es geleſen.\pend
           
\pstart
           Ich glaube, es iſt das Bedeutendſte, was Du geſchrieben haſt. Die Sprache, Poeſie und
               Proſa, iſt prachtvoll. Die Verſe\pwindex{Schnitzler, Arthur 15.\,5.\,1862 Wien – 21.\,10.\,1931 ebd.@\textsc{Schnitzler, Arthur} (15.\,5.\,1862 Wien – 21.\,10.\,1931 ebd.), \emph{Schriftsteller, Mediziner}!Schleier der Beatrice. Schauspiel in fünf Akten@\strich\emph{Der Schleier der Beatrice. Schauspiel in fünf Akten}|pwv} namentlich find von einer goldenen Reife, – zum Theil von wunderbarer
               Schönheit. Und dabei ganz {\pb}Du{ }ſelbſt. Kein Ton von
               einem Andern (Ich denke dabei an \textsc{Gerhart Hauptmann\pwindex{Hauptmann, Gerhart 15.\,11.\,1862 Szczawno-Zdrój – 6.\,6.\,1946 Jagniątków@\textsc{Hauptmann, Gerhart} (15.\,11.\,1862 Szczawno-Zdrój – 6.\,6.\,1946 Jagniątków), \emph{Schriftsteller}|pw}}, den ich erſt vor Kurzem gehört habe, wie er \label{K_L02904-1v}\edtext{\textsc{Shakespeare\pwindex{Shakespeare, William 23.\,4.\,1564? Stratford-upon-Avon – 3.\,5.\,1616 ebd.@\textsc{Shakespeare, William} (23.\,4.\,1564? Stratford-upon-Avon – 3.\,5.\,1616 ebd.), \emph{Schauspieler, Dramatiker}|pw}} nachſtammelte}{\lemma{\textnormal{\emph{Shakespeare nachstammelte}}}\Cendnote{\textnormal{Goldmann\pwindex{Goldmann, Paul 31.\,1.\,1865 Breslau – 25.\,9.\,1935 Wien@\textsc{Goldmann, Paul} (31.\,1.\,1865 Breslau – 25.\,9.\,1935 Wien), \emph{Schriftsteller, Journalist}|pwk} dürfte sich auf Hauptmanns\pwindex{Hauptmann, Gerhart 15.\,11.\,1862 Szczawno-Zdrój – 6.\,6.\,1946 Jagniątków@\textsc{Hauptmann, Gerhart} (15.\,11.\,1862 Szczawno-Zdrój – 6.\,6.\,1946 Jagniątków), \emph{Schriftsteller}|pwk} Komödie \emph{Schluck
                     und Jau}\pwindex{Hauptmann, Gerhart 15.\,11.\,1862 Szczawno-Zdrój – 6.\,6.\,1946 Jagniątków@\textsc{Hauptmann, Gerhart} (15.\,11.\,1862 Szczawno-Zdrój – 6.\,6.\,1946 Jagniątków), \emph{Schriftsteller}!Schluck und Jau@\strich\emph{Schluck und Jau}|pwk} bezogen haben, die am 3. 2. 1900 am
                     Deutschen Theater Berlin\oindex{Deutsches Theater Berlin@\textbf{Deutsches Theater Berlin}, \emph{Theater}|pwk} uraufgeführt worden
                  und von Shakespeare\pwindex{Shakespeare, William 23.\,4.\,1564? Stratford-upon-Avon – 3.\,5.\,1616 ebd.@\textsc{Shakespeare, William} (23.\,4.\,1564? Stratford-upon-Avon – 3.\,5.\,1616 ebd.), \emph{Schauspieler, Dramatiker}|pwk} inspiriert
               war.}}}\label{K_L02904-1}.)\pend
           
\pstart
           Was die Bühnenwirkung anlangt,{ }ſo habe ich noch nie vor einem Drama{ }ſo rathlos
               geſtanden. Vielleicht wird es mir bei längerem Nachdenken klarer. Denn ich bin eben
               erſt zu Ende. Es{ }ſind Szenen\pwindex{Schnitzler, Arthur 15.\,5.\,1862 Wien – 21.\,10.\,1931 ebd.@\textsc{Schnitzler, Arthur} (15.\,5.\,1862 Wien – 21.\,10.\,1931 ebd.), \emph{Schriftsteller, Mediziner}!Schleier der Beatrice. Schauspiel in fünf Akten@\strich\emph{Der Schleier der Beatrice. Schauspiel in fünf Akten}|pwv}
               darin, die Einem{ }ſchon beim Leſen den dramatiſchen Schauer geben, – die ergreifendſte
               iſt{ }ſicherlich die zwiſchen \textsc{Filippo\pwindex{Schnitzler, Arthur 15.\,5.\,1862 Wien – 21.\,10.\,1931 ebd.@\textsc{Schnitzler, Arthur} (15.\,5.\,1862 Wien – 21.\,10.\,1931 ebd.), \emph{Schriftsteller, Mediziner}!Schleier der Beatrice. Schauspiel in fünf Akten@\strich\emph{Der Schleier der Beatrice. Schauspiel in fünf Akten}|pwv}} und \textsc{Beatrice\pwindex{Schnitzler, Arthur 15.\,5.\,1862 Wien – 21.\,10.\,1931 ebd.@\textsc{Schnitzler, Arthur} (15.\,5.\,1862 Wien – 21.\,10.\,1931 ebd.), \emph{Schriftsteller, Mediziner}!Schleier der Beatrice. Schauspiel in fünf Akten@\strich\emph{Der Schleier der Beatrice. Schauspiel in fünf Akten}|pwv}} am Schluß des dritten Akts\pwindex{Schnitzler, Arthur 15.\,5.\,1862 Wien – 21.\,10.\,1931 ebd.@\textsc{Schnitzler, Arthur} (15.\,5.\,1862 Wien – 21.\,10.\,1931 ebd.), \emph{Schriftsteller, Mediziner}!Schleier der Beatrice. Schauspiel in fünf Akten@\strich\emph{Der Schleier der Beatrice. Schauspiel in fünf Akten}|pwv}. Aber einige Charaktere\pwindex{Schnitzler, Arthur 15.\,5.\,1862 Wien – 21.\,10.\,1931 ebd.@\textsc{Schnitzler, Arthur} (15.\,5.\,1862 Wien – 21.\,10.\,1931 ebd.), \emph{Schriftsteller, Mediziner}!Schleier der Beatrice. Schauspiel in fünf Akten@\strich\emph{Der Schleier der Beatrice. Schauspiel in fünf Akten}|pwv} verſtehe ich nicht. Und ich weiß nicht: werden{ }ſie auf der Bühne,
               von bedeutenden Künſtlern {\pb}dargeſtellt, \strikeout{\textcolor{gray}{es}} erſt \strikeout{\textcolor{gray}{in}} zu Leben und Wahrheit erwachſen, oder werden{ }ſie auf der Bühne erſt recht
               unbegreiflich{ }ſcheinen, weil die feinen pſychologiſchen \textsc{Nuancen} auf dem Theater{ }ſo gut wie unſichtbar \strikeout{w\textcolor{gray}{er}} werden? In dieſer Frage ruht, meiner Anſicht nach, die Frage der
               Bühnenwirkſamkeit des Stück\pwindex{Schnitzler, Arthur 15.\,5.\,1862 Wien – 21.\,10.\,1931 ebd.@\textsc{Schnitzler, Arthur} (15.\,5.\,1862 Wien – 21.\,10.\,1931 ebd.), \emph{Schriftsteller, Mediziner}!Schleier der Beatrice. Schauspiel in fünf Akten@\strich\emph{Der Schleier der Beatrice. Schauspiel in fünf Akten}|pwv}es.
               Und ich bin außer Stande,{ }ſie zu beantworten.\pend
           
\pstart
           Die \textsc{Beatrice\pwindex{Schnitzler, Arthur 15.\,5.\,1862 Wien – 21.\,10.\,1931 ebd.@\textsc{Schnitzler, Arthur} (15.\,5.\,1862 Wien – 21.\,10.\,1931 ebd.), \emph{Schriftsteller, Mediziner}!Schleier der Beatrice. Schauspiel in fünf Akten@\strich\emph{Der Schleier der Beatrice. Schauspiel in fünf Akten}|pwv}} verſtehe ich \strikeout{z\textcolor{gray}{×}\-\textcolor{gray}{×}} noch ganz gut. Kann die weibliche \label{K_L02904-2v}\edtext{\begin{otherlanguage}{french}\textsc{inconscience}\end{otherlanguage}}{\lemma{\textnormal{\emph{inconscience}}}\Cendnote{\textnormal{französisch: Gedankenlosigkeit,
                  Unbewusstsein}}}\label{K_L02904-2}{ }ſo weit gehen? Ich würde es nicht für möglich halten, aber es
               wird durch das \strikeout{\textcolor{gray}{Dr}}{ }Drama\pwindex{Schnitzler, Arthur 15.\,5.\,1862 Wien – 21.\,10.\,1931 ebd.@\textsc{Schnitzler, Arthur} (15.\,5.\,1862 Wien – 21.\,10.\,1931 ebd.), \emph{Schriftsteller, Mediziner}!Schleier der Beatrice. Schauspiel in fünf Akten@\strich\emph{Der Schleier der Beatrice. Schauspiel in fünf Akten}|pwv} beinahe wahrſcheinlich.
               Ich beuge mich vor der Geſtaltungskraft des Dichters, obwohl im Grunde meines Herzens
               einige {\pb}Zweifel verbleiben. Aber den \textsc{Filippo\pwindex{Schnitzler, Arthur 15.\,5.\,1862 Wien – 21.\,10.\,1931 ebd.@\textsc{Schnitzler, Arthur} (15.\,5.\,1862 Wien – 21.\,10.\,1931 ebd.), \emph{Schriftsteller, Mediziner}!Schleier der Beatrice. Schauspiel in fünf Akten@\strich\emph{Der Schleier der Beatrice. Schauspiel in fünf Akten}|pwv}} verſtehe ich nicht. Wie? \strikeout{W\textcolor{gray}{enn}
                     di\textcolor{gray}{e}} Die Heißgeliebte und Heißerſehnte kommt, und man{ }ſchickt{ }ſie wieder weg –
               wegen eines Traumes? Wenn ich mein Mädchen \introOben{}heut\introOben{} in den Armen
               halte, kann{ }ſie \strikeout{\textcolor{gray}{×}\-\textcolor{gray}{×}} geſtern geträumt haben, was{ }ſie will. Und dann kommt{ }ſie wieder, – kommt
               wieder aus dem Brautgemach des Herzogs\pwindex{Schnitzler, Arthur 15.\,5.\,1862 Wien – 21.\,10.\,1931 ebd.@\textsc{Schnitzler, Arthur} (15.\,5.\,1862 Wien – 21.\,10.\,1931 ebd.), \emph{Schriftsteller, Mediziner}!Schleier der Beatrice. Schauspiel in fünf Akten@\strich\emph{Der Schleier der Beatrice. Schauspiel in fünf Akten}|pwv} heraus. \textsc{Filippo\pwindex{Schnitzler, Arthur 15.\,5.\,1862 Wien – 21.\,10.\,1931 ebd.@\textsc{Schnitzler, Arthur} (15.\,5.\,1862 Wien – 21.\,10.\,1931 ebd.), \emph{Schriftsteller, Mediziner}!Schleier der Beatrice. Schauspiel in fünf Akten@\strich\emph{Der Schleier der Beatrice. Schauspiel in fünf Akten}|pwv}} will mit ihr{ }ſterben. Sie hat Furcht vor dem Tode und will am Leben bleiben.
               Schön! Aber warum bringt \uline{er}{ }ſich dann um? Sie iſt
               menſchlich und wahr. Und er{ }ſieht das nicht ein, – er, der ein Dichter iſt? Man kann
               Jemanden immer noch ungeheuer lieb haben,{ }ſelbſt wenn man nicht mit ihm{ }ſterben will.
               Es geht {\pb}nun einmal nicht{ }ſo leicht mit dem Sterben.
               Das Alles{ }ſagt \textsc{Filippo\pwindex{Schnitzler, Arthur 15.\,5.\,1862 Wien – 21.\,10.\,1931 ebd.@\textsc{Schnitzler, Arthur} (15.\,5.\,1862 Wien – 21.\,10.\,1931 ebd.), \emph{Schriftsteller, Mediziner}!Schleier der Beatrice. Schauspiel in fünf Akten@\strich\emph{Der Schleier der Beatrice. Schauspiel in fünf Akten}|pwv}}{ }ſelber mit den herrlichſten Worten. Und auf einmal bringt er{ }ſich um. Weshalb?
               Ich kann es nicht begreifen. Und ich finde, wenn man ein{ }ſchönes Liebchen hat, und
               wenn{ }ſie in der Nacht zu Einem kommt, und wenn man nicht weiß, was morgen{ }ſein wird,{ }ſo greift man, weiß Gott, nicht zum Giftbecher. \strikeout{Ich
                  mag} Ich mag die jungen {\pb}Leute nicht, die{ }ſich aus Pſychologie vergiften.\pend
           
\pstart
           Auch den Herzog\pwindex{Schnitzler, Arthur 15.\,5.\,1862 Wien – 21.\,10.\,1931 ebd.@\textsc{Schnitzler, Arthur} (15.\,5.\,1862 Wien – 21.\,10.\,1931 ebd.), \emph{Schriftsteller, Mediziner}!Schleier der Beatrice. Schauspiel in fünf Akten@\strich\emph{Der Schleier der Beatrice. Schauspiel in fünf Akten}|pwv} verſtehe ich
               nicht. Ich hätte ihn verſtanden, wenn die Trauung mit \textsc{Beatrice\pwindex{Schnitzler, Arthur 15.\,5.\,1862 Wien – 21.\,10.\,1931 ebd.@\textsc{Schnitzler, Arthur} (15.\,5.\,1862 Wien – 21.\,10.\,1931 ebd.), \emph{Schriftsteller, Mediziner}!Schleier der Beatrice. Schauspiel in fünf Akten@\strich\emph{Der Schleier der Beatrice. Schauspiel in fünf Akten}|pwv}}{ }\strikeout{\textcolor{gray}{die wirkl}ich} ein \label{K_L02904-3v}\edtext{Faſtnachts-Scherz}{\lemma{\textnormal{\emph{Fastnachts-Scherz}}}\Cendnote{\textnormal{traditioneller Scherz zur Fastnacht (Fasching, Karneval)}}}\label{K_L02904-3} geweſen wäre\substVorne{}\textsuperscript{,}\substDazwischen{}.\substHinten{} Aber ich begreife nicht, daß dieſer Renaiſſance-Despot{ }ſentimental genug
               iſt, das Mädchen wirklich zu heirathen. \strikeout{\textcolor{gray}{Überhaupt iſt}{ }\textcolor{gray}{[4 Zeilen unleserlich{]} }} Gewiß, es iſt nur für eine Nacht, und man weiß nicht, was morgen{ }ſein wird.
               Und doch hat er unverkennbar{ }ſentimentale Anwandlungen, und die {\pb}paſſen nicht zum Bilde eines Mannes, der
               entſchloſſen iſt, das Leben in{ }ſeiner Fülle zu genießen. \strikeout{\textcolor{gray}{×}\-\textcolor{gray}{×}\-\textcolor{gray}{×}\-\textcolor{gray}{×}\-\textcolor{gray}{×}\-\textcolor{gray}{×}\-\textcolor{gray}{×}\-\textcolor{gray}{×}\-\textcolor{gray}{×}\-\textcolor{gray}{×}\-\textcolor{gray}{×}\-\textcolor{gray}{×}\-\textcolor{gray}{×}\-\textcolor{gray}{×}\-\textcolor{gray}{×}}{ }\strikeout{\textcolor{gray}{×}\-\textcolor{gray}{×}\-\textcolor{gray}{×}\-\textcolor{gray}{×}\-\textcolor{gray}{×}\-\textcolor{gray}{×}{ }ſei\textcolor{gray}{.}}\pend
           
\pstart
           Bewundernswürdig aber iſt wieder die Fülle der \introOben{}andern\introOben{}
               Figuren, die \uline{Alle} leben, die \substVorne{}\textsuperscript{G}\substDazwischen{}g\substHinten{}roßen und die kleinen. Den \textsc{Francesco\pwindex{Schnitzler, Arthur 15.\,5.\,1862 Wien – 21.\,10.\,1931 ebd.@\textsc{Schnitzler, Arthur} (15.\,5.\,1862 Wien – 21.\,10.\,1931 ebd.), \emph{Schriftsteller, Mediziner}!Schleier der Beatrice. Schauspiel in fünf Akten@\strich\emph{Der Schleier der Beatrice. Schauspiel in fünf Akten}|pwv}} mag ich freilich auch nicht und es kommt mir vor, als{ }ſei er nur da, damit{ }ſich
               am Schluß doch noch Jemand finde, welcher die \textsc{Beatrice\pwindex{Schnitzler, Arthur 15.\,5.\,1862 Wien – 21.\,10.\,1931 ebd.@\textsc{Schnitzler, Arthur} (15.\,5.\,1862 Wien – 21.\,10.\,1931 ebd.), \emph{Schriftsteller, Mediziner}!Schleier der Beatrice. Schauspiel in fünf Akten@\strich\emph{Der Schleier der Beatrice. Schauspiel in fünf Akten}|pwv}} erſticht. Ob es unumgänglich iſt, \strikeout{\textcolor{gray}{da}} daß{ }ſie erſtochen wird, iſt mir ebenfalls nicht klar.\pend
           
\pstart
           Höchſt eindrucksvoll iſt es, wie{ }ſich alle dieſe Ereigniſſe in der \uline{einen} Nacht zuſammendrängen und wie während {\pb}des \strikeout{g\textcolor{gray}{roß}} ganzen Dramas\pwindex{Schnitzler, Arthur 15.\,5.\,1862 Wien – 21.\,10.\,1931 ebd.@\textsc{Schnitzler, Arthur} (15.\,5.\,1862 Wien – 21.\,10.\,1931 ebd.), \emph{Schriftsteller, Mediziner}!Schleier der Beatrice. Schauspiel in fünf Akten@\strich\emph{Der Schleier der Beatrice. Schauspiel in fünf Akten}|pwv}{ }\textsc{Cesar Borgia\pwindex{Schnitzler, Arthur 15.\,5.\,1862 Wien – 21.\,10.\,1931 ebd.@\textsc{Schnitzler, Arthur} (15.\,5.\,1862 Wien – 21.\,10.\,1931 ebd.), \emph{Schriftsteller, Mediziner}!Schleier der Beatrice. Schauspiel in fünf Akten@\strich\emph{Der Schleier der Beatrice. Schauspiel in fünf Akten}|pwv}\pwindex{Borgia, Cesare 1475/1476 Rom – 22.\,3.\,1507 Viana@\textsc{Borgia, Cesare} (1475/1476 Rom – 22.\,3.\,1507 Viana), \emph{Fürst}|pw}} vor den Thoren von \textsc{Bologna\oindex{Bologna@\textbf{Bologna}|pw}}{ }ſteht. Auch habe ich auf mancher Seite des Buch\pwindex{Schnitzler, Arthur 15.\,5.\,1862 Wien – 21.\,10.\,1931 ebd.@\textsc{Schnitzler, Arthur} (15.\,5.\,1862 Wien – 21.\,10.\,1931 ebd.), \emph{Schriftsteller, Mediziner}!Schleier der Beatrice. Schauspiel in fünf Akten@\strich\emph{Der Schleier der Beatrice. Schauspiel in fünf Akten}|pwv}es die Kraft und die Fülle der Zeit empfunden, in welche
               die Handlung verlegt iſt{\dotsfive}\pend
           
\pstart
           Das{ }ſind wenige, flüchtige Worte, – mit müdem und{ }ſchmerzendem Kopfe geſchrieben.\pend
           
\pstart
           Ich grüße Dich von Herzen {\\[\baselineskip]}Dein {\\[\baselineskip]}\spacefill\mbox{Paul Goldmann.}\pend
           \leftskip=0em{}\selectlanguage{ngerman}\endnumbering\briefempfaengerindex{Schnitzler, Arthur@\textsc{Schnitzler, Arthur}!zzzGoldmann, Paul@\emph{von Paul Goldmann}!1900-02-113@{11. 2. 1900}|)be}\mylabel{L02904h}  \newcommand{\dateiname}{L02904}\newcommand{\titel}{Paul Goldmann an Arthur Schnitzler, 11. 2. 1900}\newcommand{\editorInnen}{Martin Anton Müller und Laura Untner}%% latex-leseansicht-abspann.tex
%% Abspann für die Leseansicht.
%% Der Schalter \ifkorrekturansicht ist bereits durch den Vorspann gesetzt.

%% latex-abspann.tex
%% Gemeinsamer Abspann für Korrekturansicht und Leseansicht.
%% Setzt den Schalter \ifkorrekturansicht voraus (gesetzt in den
%% einbindenden Dateien latex-korrekturansicht-abspann.tex bzw.
%% latex-leseansicht-abspann.tex).
%% ---------------------------------------------------------------

\normalsize

% Das esempio-Environment wird nur in der Leseansicht benötigt
\ifkorrekturansicht\else
\newenvironment{esempio}[3]%
{
    \vspace{1.5ex}
    \rlap{\underline{#1}}
    \par
    \setlength{\parindent}{0cm}
    \nopagebreak
    \leftskip=#2cm
    \rightskip=#3cm
}
{
    \par
}
\fi

\doendnotes{C}
\bigskip
\vfill

\clearpage

\footnotesize

\ifkorrekturansicht
  \lohead{\textsc{register}}
\fi

% theindex-Environment neu definieren ohne reledmac
\makeatletter
\renewenvironment{theindex}{%
  \ifkorrekturansicht
    \section*{\indexname}%
  \else
    \subsubsection*{Index der erwähnten Entitäten}%
  \fi
  \setlength{\parindent}{0pt}%
  \setlength{\parskip}{0pt plus 0.3pt}%
  \let\item\@idxitem
}{%
  \ifkorrekturansicht\clearpage\fi
}
\makeatother

\IfFileExists{\jobname-pw.ind}{\input{\jobname-pw.ind}}{}

% Quellenangabe nur in der Leseansicht
\ifkorrekturansicht\else
% Fallback-Definitionen, falls die .tex-Datei \titel etc. nicht gesetzt hat
\providecommand{\titel}{}
\providecommand{\editorInnen}{}
\providecommand{\dateiname}{\jobname}

\vspace{3cm}

\vfill

\footnotesize
\textsc{Quelle}: \titel. Herausgegeben von {\editorInnen}. In: \emph{Arthur Schnitzler: Briefwechsel mit Autorinnen und Autoren}.
 Digitale Edition, https://schnitzler-briefe.acdh.oeaw.ac.at/{\dateiname}.html (Stand \today)
\fi

\end{document}


