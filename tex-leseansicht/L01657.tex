%% latex-leseansicht-vorspann.tex
%% Vorspann für die Leseansicht.
%% Lädt die gemeinsame Datei latex-vorspann.tex mit nicht gesetztem Schalter.

\newif\ifkorrekturansicht
\korrekturansichtfalse

\input{../tex-inputs/latex-vorspann}


\section[Arthur Schnitzler an Hermann Bahr, 15. 2. 1907]{L01657 Arthur Schnitzler an Hermann Bahr, 15. 2. 1907}
\nopagebreak\mylabel{L01657v}
\rehead{ }\normalsize\beginnumbering\briefempfaengerindex{Bahr, Hermann@\textsc{Bahr, Hermann}!zzzSchnitzler, Arthur@\emph{von Arthur Schnitzler}!1907-02-151@{15. 2. 1907}|(be}
\toendnotes[C]{\smallbreak\pagebreak[2]}
\correspDesc{Versand  durch Arthur Schnitzler am 15. 2. 1907 in Wien
\newline{}Erhalt  durch Hermann Bahr im Zeitraum [16. 2. 1907
                  – 20. 2. 1907?] in Berlin}\toendnotes[C]{\smallbreak}
\Standort{TMW, HS AM 23382 Ba.}
\physDesc{Brief, 1 Blatt, 3 Seiten, 726 Zeichen
\newline{}Handschrift: schwarze Tinte, deutsche Kurrent
\newline{}Ordnung: Lochung }
\buchAbdrucke{\weitereDrucke{1) \emph{15. 2. 1907.} In: Arthur Schnitzler: \emph{The Letters of Arthur Schnitzler to Hermann Bahr}. Edited, annotated, and with an introduction, by Donald G. Daviau. Chapel Hill: \emph{The University of North Carolina Press} 1978, S. 96–97 (University of North Carolina studies in the Germanic languages
                        and literatures, 89).} \weitereDrucke{2) Hermann Bahr, Arthur Schnitzler: \emph{Briefwechsel, Aufzeichnungen, Dokumente (1891–1931)}. Herausgegeben von Kurt Ifkovits und Martin Anton Müller. Göttingen: \emph{Wallstein} 2018, S. 389.} }\toendnotes[C]{\smallbreak}
\pstart
           \raggedleft{}{\pb}Wien\oindex{Wien@\textbf{Wien}, \emph{Verwaltungsgebiet}|pw}, 15. 2. 907\pend
           
\pstart{}lieber Hermann,\pend\vspace{0.5em}
\pstart
           vielen Dank. \textsc{Lbl\pwindex{Schnitzler, Arthur 15.\,5.\,1862 Wien – 21.\,10.\,1931 ebd.@\textsc{Schnitzler, Arthur} (15.\,5.\,1862 Wien – 21.\,10.\,1931 ebd.), \emph{Schriftsteller, Mediziner}!Liebelei. Schauspiel in drei Akten@\strich\emph{Liebelei. Schauspiel in drei Akten}|pw}} ein Exemplar geſtern an dich geſandt. Ich bitte dich nur recht{ }ſehr, dir
               keinerlei Ungelegenheiten zu machen. Wenn \textsc{R}.\pwindex{Reinhardt, Max 9.\,9.\,1873 Baden bei Wien – 30.\,10.\,1943 New York City@\textsc{Reinhardt, Max} (9.\,9.\,1873 Baden bei Wien – 30.\,10.\,1943 New York City), \emph{Theaterleiter, Regisseur, Schauspieler}|pw} gern daran geht, ja. Aber wenns ihm nicht von
               Herzen iſt, da{\geminationn} lieber nicht. Wie denkſt du dir die{ }ſonſtigen Beſetzungsmöglichkeiten? Iſt Pagay\pwindex{Pagay, Hans 11.\,11.\,1843 Wien – 21.\,1.\,1915 Berlin@\textsc{Pagay, Hans} (11.\,11.\,1843 Wien – 21.\,1.\,1915 Berlin), \emph{Schauspieler}|pw}
               für den Alten nicht zu trocken?\pend
           
\pstart
           {\pb}\textsc{Valentin}\pwindex{Vallentin, Richard 3.\,2.\,1874 Luzern – 14.\,1.\,1908 Berlin@\textsc{Vallentin, Richard} (3.\,2.\,1874 Luzern – 14.\,1.\,1908 Berlin), \emph{Regisseur, Schauspieler}|pw} hat mir neuerdings wegen der \textsc{Bea}\pwindex{Schnitzler, Arthur 15.\,5.\,1862 Wien – 21.\,10.\,1931 ebd.@\textsc{Schnitzler, Arthur} (15.\,5.\,1862 Wien – 21.\,10.\,1931 ebd.), \emph{Schriftsteller, Mediziner}!Schleier der Beatrice. Schauspiel in fünf Akten@\strich\emph{Der Schleier der Beatrice. Schauspiel in fünf Akten}|pw}. geſchrieben; ich hab mich noch nicht endgiltig ausgeſprochen.\pend
           
\pstart
           Bin im übrigen ziemlich fleißig und hoffe zu nächſtem Herbst mit etlichem bereit zu{ }ſein.\pend
           
\pstart
           Famos dein »\label{K_L01657-1v}\edtext{Grillparzer\pwindex{Grillparzer, Franz 15.\,1.\,1791 Wien – 21.\,1.\,1872 ebd.@\textsc{Grillparzer, Franz} (15.\,1.\,1791 Wien – 21.\,1.\,1872 ebd.), \emph{Schriftsteller, Beamter}|pw}\pwindex{Bahr, Hermann 19.\,7.\,1863 Linz – 15.\,1.\,1934 München@\textsc{Bahr, Hermann} (19.\,7.\,1863 Linz – 15.\,1.\,1934 München), \emph{Schriftsteller, Kritiker}!Grillparzer@\strich\emph{Grillparzer}|pw}}{\lemma{\textnormal{\emph{Grillparzer}}}\Cendnote{\textnormal{Hermann Bahr\pwindex{Bahr, Hermann 19.\,7.\,1863 Linz – 15.\,1.\,1934 München@\textsc{Bahr, Hermann} (19.\,7.\,1863 Linz – 15.\,1.\,1934 München), \emph{Schriftsteller, Kritiker}|pwk}: \emph{Grillparzer}\pwindex{Bahr, Hermann 19.\,7.\,1863 Linz – 15.\,1.\,1934 München@\textsc{Bahr, Hermann} (19.\,7.\,1863 Linz – 15.\,1.\,1934 München), \emph{Schriftsteller, Kritiker}!Grillparzer@\strich\emph{Grillparzer}|pwk}. In: \emph{Die
                        Schaubühne}\pwindex{Schaubühne@\emph{Die Schaubühne}|pwk}, Jg. 3, H. 7, 14. 2. 1907, S. 163–170,
                  als Vorabdruck aus \emph{Wien}\pwindex{Bahr, Hermann 19.\,7.\,1863 Linz – 15.\,1.\,1934 München@\textsc{Bahr, Hermann} (19.\,7.\,1863 Linz – 15.\,1.\,1934 München), \emph{Schriftsteller, Kritiker}!Wien. Mit acht Vollbildern@\strich\emph{Wien. Mit acht Vollbildern}|pwk}
                  gekennzeichnet.}}}\label{K_L01657-1}« in der Schaubühne\orgindex{Schaubühne / Die Weltbühne@Die Schaubühne / Die Weltbühne|pw}.
                  Freu\damage{e} mich auf das ganze \label{K_L01657-2v}\edtext{Buch\pwindex{Bahr, Hermann 19.\,7.\,1863 Linz – 15.\,1.\,1934 München@\textsc{Bahr, Hermann} (19.\,7.\,1863 Linz – 15.\,1.\,1934 München), \emph{Schriftsteller, Kritiker}!Wien. Mit acht Vollbildern@\strich\emph{Wien. Mit acht Vollbildern}|pwv}}{\lemma{\textnormal{\emph{Buch}}}\Cendnote{\textnormal{Hermann Bahr\pwindex{Bahr, Hermann 19.\,7.\,1863 Linz – 15.\,1.\,1934 München@\textsc{Bahr, Hermann} (19.\,7.\,1863 Linz – 15.\,1.\,1934 München), \emph{Schriftsteller, Kritiker}|pwk}: \emph{Wien}\pwindex{Bahr, Hermann 19.\,7.\,1863 Linz – 15.\,1.\,1934 München@\textsc{Bahr, Hermann} (19.\,7.\,1863 Linz – 15.\,1.\,1934 München), \emph{Schriftsteller, Kritiker}!Wien. Mit acht Vollbildern@\strich\emph{Wien. Mit acht Vollbildern}|pwk}. Stuttgart: \emph{Karl Krabbe}{ }1907 (erschienen in der
                  zweiten Mai-Hälfte).}}}\label{K_L01657-2}.\pend
           
\pstart
           Was machſt du nach Berlin\oindex{Berlin@\textbf{Berlin}, \emph{Hauptstadt}|pw}? Sollte die \label{K_L01657-3v}\edtext{\textsc{Neue Freie}\orgindex{Neue Freie Presse@Neue Freie Presse|pw}\pwindex{Bahr, Hermann 19.\,7.\,1863 Linz – 15.\,1.\,1934 München@\textsc{Bahr, Hermann} (19.\,7.\,1863 Linz – 15.\,1.\,1934 München), \emph{Schriftsteller, Kritiker}!Laiengedanken über die Wahlen in Österreich@\strich\emph{Laiengedanken über die Wahlen in Österreich}|pwv} den {\pb}Beginn
               deiner Wiederkehr}{\lemma{\textnormal{\emph{Neue … Wiederkehr}}}\Cendnote{\textnormal{Das Feuilleton \emph{Laiengedanken über die Wahlen in Österreich}\pwindex{Bahr, Hermann 19.\,7.\,1863 Linz – 15.\,1.\,1934 München@\textsc{Bahr, Hermann} (19.\,7.\,1863 Linz – 15.\,1.\,1934 München), \emph{Schriftsteller, Kritiker}!Laiengedanken über die Wahlen in Österreich@\strich\emph{Laiengedanken über die Wahlen in Österreich}|pwk} am
                     2. 2. 1907 (Nr. 15.249, Morgenblatt, S. 3–4) eröffnete
                  die bis zum Tod anhaltende Mitarbeit Bahrs\pwindex{Bahr, Hermann 19.\,7.\,1863 Linz – 15.\,1.\,1934 München@\textsc{Bahr, Hermann} (19.\,7.\,1863 Linz – 15.\,1.\,1934 München), \emph{Schriftsteller, Kritiker}|pwk} an der \emph{Neuen
                     Freien Presse}\orgindex{Neue Freie Presse@Neue Freie Presse|pwk}.}}}\label{K_L01657-3} bedeuten?\pend
           
\pstart
           Meine Frau\pwindex{Schnitzler, Olga 17.\,1.\,1882 Wien – 13.\,1.\,1970 Lugano@\textsc{Schnitzler, Olga} (17.\,1.\,1882 Wien – 13.\,1.\,1970 Lugano), \emph{Schauspielerin, Sängerin}|pwv} grüßt dich
               vielmals. Von Herzen{\\[\baselineskip]}Dein{\\[\baselineskip]}\spacefill\mbox{Arthur}\pend
           \leftskip=0em{}\selectlanguage{ngerman}\endnumbering\briefempfaengerindex{Bahr, Hermann@\textsc{Bahr, Hermann}!zzzSchnitzler, Arthur@\emph{von Arthur Schnitzler}!1907-02-151@{15. 2. 1907}|)be}\mylabel{L01657h}  \newcommand{\dateiname}{L01657}\newcommand{\titel}{Arthur Schnitzler an Hermann Bahr, 15. 2. 1907}\newcommand{\editorInnen}{Herausgegeben von Martin Anton Müller}%% latex-leseansicht-abspann.tex
%% Abspann für die Leseansicht.
%% Der Schalter \ifkorrekturansicht ist bereits durch den Vorspann gesetzt.

%% latex-abspann.tex
%% Gemeinsamer Abspann für Korrekturansicht und Leseansicht.
%% Setzt den Schalter \ifkorrekturansicht voraus (gesetzt in den
%% einbindenden Dateien latex-korrekturansicht-abspann.tex bzw.
%% latex-leseansicht-abspann.tex).
%% ---------------------------------------------------------------

\normalsize

% Das esempio-Environment wird nur in der Leseansicht benötigt
\ifkorrekturansicht\else
\newenvironment{esempio}[3]%
{
    \vspace{1.5ex}
    \rlap{\underline{#1}}
    \par
    \setlength{\parindent}{0cm}
    \nopagebreak
    \leftskip=#2cm
    \rightskip=#3cm
}
{
    \par
}
\fi

\doendnotes{C}
\bigskip
\vfill

\clearpage

\footnotesize

\ifkorrekturansicht
  \lohead{\textsc{register}}
\fi

% theindex-Environment neu definieren ohne reledmac
\makeatletter
\renewenvironment{theindex}{%
  \ifkorrekturansicht
    \section*{\indexname}%
  \else
    \subsubsection*{Index der erwähnten Entitäten}%
  \fi
  \setlength{\parindent}{0pt}%
  \setlength{\parskip}{0pt plus 0.3pt}%
  \let\item\@idxitem
}{%
  \ifkorrekturansicht\clearpage\fi
}
\makeatother

\IfFileExists{\jobname-pw.ind}{\input{\jobname-pw.ind}}{}

% Quellenangabe nur in der Leseansicht
\ifkorrekturansicht\else
% Fallback-Definitionen, falls die .tex-Datei \titel etc. nicht gesetzt hat
\providecommand{\titel}{}
\providecommand{\editorInnen}{}
\providecommand{\dateiname}{\jobname}

\vspace{3cm}

\vfill

\footnotesize
\textsc{Quelle}: \titel. Herausgegeben von {\editorInnen}. In: \emph{Arthur Schnitzler: Briefwechsel mit Autorinnen und Autoren}.
 Digitale Edition, https://schnitzler-briefe.acdh.oeaw.ac.at/{\dateiname}.html (Stand \today)
\fi

\end{document}


