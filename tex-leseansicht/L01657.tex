%% latex-korrekturansicht-vorspann.tex
%% Vorspann für die Korrekturansicht.
%% Lädt die gemeinsame Datei latex-vorspann.tex mit gesetztem Schalter.

\newif\ifkorrekturansicht
\korrekturansichttrue

\input{../tex-inputs/latex-vorspann}


\section[Arthur Schnitzler an Hermann Bahr, 15. 2. 1907]{L01657 Arthur Schnitzler an Hermann Bahr, 15. 2. 1907}
\nopagebreak\mylabel{L01657v}
\rehead{ }\normalsize\beginnumbering\briefempfaengerindex{Bahr, Hermann@\textsc{Bahr, Hermann}!zzzSchnitzler, Arthur@\emph{von Arthur Schnitzler}!1907-02-151@{15. 2. 1907}|(be}
\toendnotes[C]{\smallbreak\pagebreak[2]}\Standort{TMW, HS AM 23382 Ba.}
\physDesc{Brief, 1 Blatt, 3 Seiten, 726 Zeichen
\newline{}Handschrift: schwarze Tinte, deutsche Kurrent
\newline{}Ordnung: Lochung }
\buchAbdrucke{\weitereDrucke{1) Arthur Schnitzler: \emph{The Letters of Arthur Schnitzler to Hermann Bahr}. Chapel Hill: \emph{The University of North Carolina Press} 1978, S. 96–97.} \weitereDrucke{2) Hermann Bahr, Arthur Schnitzler: \emph{Briefwechsel, Aufzeichnungen, Dokumente (1891–1931)}. Göttingen: \emph{Wallstein} 2018, S. 389.} }\toendnotes[C]{\smallbreak}
\pstart
           \raggedleft{}{\pb}Wien\oindex{Wien@\textbf{Wien}, \emph{A.ADM2}|pw}, 15. 2. 907\pend
           
\pstart{}lieber Hermann,\pend\vspace{0.5em}
\pstart
           vielen Dank. \textsc{Lbl\pwindex{Liebelei. Schauspiel in drei Akten@\emph{Liebelei. Schauspiel in drei Akten}|pw}} ein Exemplar geſtern an dich geſandt. Ich bitte dich nur recht ſehr, dir
               keinerlei Ungelegenheiten zu machen. Wenn \textsc{R}.\pwindex{Reinhardt, Max 09.09.1873 – 30.10.1943@\textsc{Reinhardt, Max} (09.09.1873 – 30.10.1943), \emph{Theaterleiter/Theaterleiterin, Regisseur/Regisseurin, Schauspieler/Schauspielerin}|pw} gern daran geht, ja. Aber wenns ihm nicht von
               Herzen iſt, da{\geminationn} lieber nicht. Wie denkſt du dir die
               ſonſtigen Beſetzungsmöglichkeiten? Iſt Pagay\pwindex{Pagay, Hans 1843-11-11 – 1915-01-21@\textsc{Pagay, Hans} (1843-11-11 – 1915-01-21), \emph{Schauspieler/Schauspielerin}|pw}
               für den Alten nicht zu trocken?\pend
           
\pstart
           {\pb}\textsc{Valentin}\pwindex{Vallentin, Richard 03.02.1874 – 14.01.1908@\textsc{Vallentin, Richard} (03.02.1874 – 14.01.1908), \emph{Regisseur/Regisseurin, Schauspieler/Schauspielerin}|pw} hat mir neuerdings wegen der \textsc{Bea}\pwindex{Schleier der Beatrice. Schauspiel in fuenf Akten@\emph{Der Schleier der Beatrice. Schauspiel in fünf Akten}|pw}. geſchrieben; ich hab mich noch nicht endgiltig ausgeſprochen.\pend
           
\pstart
           Bin im übrigen ziemlich fleißig und hoffe zu nächſtem Herbst mit etlichem bereit zu
               ſein.\pend
           
\pstart
           Famos dein »\label{K_L01657-1v}\edtext{Grillparzer\pwindex{Grillparzer, Franz 15.01.1791 – 21.01.1872@\textsc{Grillparzer, Franz} (15.01.1791 – 21.01.1872), \emph{Schriftsteller/Schriftstellerin, Beamter/Beamte}|pw}\pwindex{Grillparzer@\emph{Grillparzer}|pw}}{\lemma{\textnormal{\emph{Grillparzer}}}\Cendnote{\textnormal{Hermann Bahr\pwindex{Bahr, Hermann 19.07.1863 – 15.01.1934@\textsc{Bahr, Hermann} (19.07.1863 – 15.01.1934), \emph{Schriftsteller/Schriftstellerin, Kritiker/Kritikerin}|pwk}: \emph{Grillparzer}\pwindex{Grillparzer@\emph{Grillparzer}|pwk}. In: \emph{Die
                        Schaubühne}\pwindex{Schaubuehne@\emph{Die Schaubühne}|pwk}, Jg. 3, H. 7, 14. 2. 1907, S. 163–170,
                  als Vorabdruck aus \emph{Wien}\pwindex{Wien. Mit acht Vollbildern@\emph{Wien. Mit acht Vollbildern}|pwk}
                  gekennzeichnet.}}}\label{K_L01657-1}« in der Schaubühne\orgindex{Schaubuehne / Die Weltbuehne@Die Schaubühne / Die Weltbühne|pw}.
                  Freu\damage{e} mich auf das ganze \label{K_L01657-2v}\edtext{Buch\pwindex{Wien. Mit acht Vollbildern@\emph{Wien. Mit acht Vollbildern}|pwv}}{\lemma{\textnormal{\emph{Buch}}}\Cendnote{\textnormal{Hermann Bahr\pwindex{Bahr, Hermann 19.07.1863 – 15.01.1934@\textsc{Bahr, Hermann} (19.07.1863 – 15.01.1934), \emph{Schriftsteller/Schriftstellerin, Kritiker/Kritikerin}|pwk}: \emph{Wien}\pwindex{Wien. Mit acht Vollbildern@\emph{Wien. Mit acht Vollbildern}|pwk}. Stuttgart: \emph{Karl Krabbe}{ }1907 (erschienen in der
                  zweiten Mai-Hälfte).}}}\label{K_L01657-2}.\pend
           
\pstart
           Was machſt du nach Berlin\oindex{Berlin@\textbf{Berlin}, \emph{P.PPLC}|pw}? Sollte die \label{K_L01657-3v}\edtext{\textsc{Neue Freie}\orgindex{Neue Freie Presse@Neue Freie Presse|pw}\pwindex{Laiengedanken ueber die Wahlen in Oesterreich@\emph{Laiengedanken über die Wahlen in Österreich}|pwv} den {\pb}Beginn
               deiner Wiederkehr}{\lemma{\textnormal{\emph{Neue … Wiederkehr}}}\Cendnote{\textnormal{Das Feuilleton \emph{Laiengedanken über die Wahlen in Österreich}\pwindex{Laiengedanken ueber die Wahlen in Oesterreich@\emph{Laiengedanken über die Wahlen in Österreich}|pwk} am
                     2. 2. 1907 (Nr. 15.249, Morgenblatt, S. 3–4) eröffnete
                  die bis zum Tod anhaltende Mitarbeit Bahrs\pwindex{Bahr, Hermann 19.07.1863 – 15.01.1934@\textsc{Bahr, Hermann} (19.07.1863 – 15.01.1934), \emph{Schriftsteller/Schriftstellerin, Kritiker/Kritikerin}|pwk} an der \emph{Neuen
                     Freien Presse}\orgindex{Neue Freie Presse@Neue Freie Presse|pwk}.}}}\label{K_L01657-3} bedeuten? \pend
           
\pstart
           Meine Frau\pwindex{Schnitzler, Olga 17.01.1882 – 13.01.1970@\textsc{Schnitzler, Olga} (17.01.1882 – 13.01.1970), \emph{Schauspieler/Schauspielerin, Sänger/Sängerin}|pwv} grüßt dich
               vielmals. Von Herzen{\\[\baselineskip]}Dein{\\[\baselineskip]}\spacefill\mbox{Arthur}\pend
           \leftskip=0em{}\selectlanguage{ngerman}\endnumbering\briefempfaengerindex{Bahr, Hermann@\textsc{Bahr, Hermann}!zzzSchnitzler, Arthur@\emph{von Arthur Schnitzler}!1907-02-151@{15. 2. 1907}|)be}\mylabel{L01657h}  \normalsize

\doendnotes{C}
\bigskip
\vfill

\clearpage

\footnotesize

\lohead{\textsc{register}}

% Definiere theindex-Environment komplett neu ohne reledmac
\makeatletter
\renewenvironment{theindex}{%
  \section*{\indexname}%
  \setlength{\parindent}{0pt}%
  \setlength{\parskip}{0pt plus 0.3pt}%
  \let\item\@idxitem
}{%
  \clearpage
}
\makeatother

\IfFileExists{\jobname-pw.ind}{\input{\jobname-pw.ind}}{}

\end{document}

      