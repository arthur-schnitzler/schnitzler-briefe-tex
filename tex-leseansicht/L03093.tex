%% latex-leseansicht-vorspann.tex
%% Vorspann für die Leseansicht.
%% Lädt die gemeinsame Datei latex-vorspann.tex mit nicht gesetztem Schalter.

\newif\ifkorrekturansicht
\korrekturansichtfalse

\input{../tex-inputs/latex-vorspann}


\section[ Paul Goldmann an Arthur Schnitzler, 4. 12. [1901]]{L03093 Paul Goldmann an Arthur Schnitzler,  4. 12. [1901]}
\nopagebreak\mylabel{L03093v}
\rehead{ }\normalsize\beginnumbering\briefempfaengerindex{Schnitzler, Arthur@\textsc{Schnitzler, Arthur}!zzzGoldmann, Paul@\emph{von Paul Goldmann}!1901-12-041@{4. 12. [1901]}|(be}
\toendnotes[C]{\smallbreak\pagebreak[2]}
\correspDesc{Versand  durch Paul Goldmann am 4. 12. [1901] in Berlin
\newline{}Erhalt  durch Arthur Schnitzler im Zeitraum [5. 12. 1901
                  – 9. 12. 1901?] in Wien}\toendnotes[C]{\smallbreak}
\Standort{DLA, A:Schnitzler, HS.NZ85.1.3171.}
\physDesc{Brief, 1 Blatt, 4 Seiten, 1165 Zeichen
\newline{}Handschrift: blaue Tinte, deutsche Kurrent
\newline{}Beilage: Zeitungsausschnitt (zwei Teile), beschnitten und
                                 aufgeklebt 
\newline{}Schnitzler: 1) mit Bleistift das Jahr »901« vermerkt  2) mit rotem Buntstift zwei Unterstreichungen}\toendnotes[C]{\smallbreak}
\pstart
           \raggedleft{}{\pb}\textcolor{gray}{\textbf{DESSAUERSTRASSE 19}}\oindex{Dessauer Straße@\textbf{Dessauer Straße}, \emph{Straße}|pw}\pend
           
\pstart
           Berlin\oindex{Berlin@\textbf{Berlin}, \emph{Hauptstadt}|pw}, 4. Dezember.\pend
           
\pstart\center{}Mein lieber Freund,\pend\vspace{0.5em}
\pstart
           Zolltarif im Reichſtag\orgindex{Reichstag@Reichstag|pw}. Ich habe keine freie
               Minute.\pend
           
\pstart
           Tauſend Dank für Deinen lieben Brief.\pend
           
\pstart
           Über Deine Auslegung, daß \textsc{Hauptmann\pwindex{Hauptmann, Gerhart 15.\,11.\,1862 Szczawno-Zdrój – 6.\,6.\,1946 Jagniątków@\textsc{Hauptmann, Gerhart} (15.\,11.\,1862 Szczawno-Zdrój – 6.\,6.\,1946 Jagniątków), \emph{Schriftsteller}|pw}} eine geiſtige Krankheit durchmacht, habe ich den Kopf geſchüttelt. Warum eine
               Erklärung {\pb}an den Haaren herbeiziehen? Warum das
               Eigentliche nicht{ }ſehen wollen? Wenn Einer geiſtig leer iſt,{ }ſo iſt er immer geiſtig
               leer geweſen. Man kann ein Stück verfehlen, man kann aber nicht auf einmal weder
               Geiſt noch Talent haben. Und was Deine Anſicht betrifft, \textsc{Hannele\pwindex{Hauptmann, Gerhart 15.\,11.\,1862 Szczawno-Zdrój – 6.\,6.\,1946 Jagniątków@\textsc{Hauptmann, Gerhart} (15.\,11.\,1862 Szczawno-Zdrój – 6.\,6.\,1946 Jagniątków), \emph{Schriftsteller}!Hanneles Himmelfahrt. Traumdichtung in zwei Teilen@\strich\emph{Hanneles Himmelfahrt. Traumdichtung in zwei Teilen}|pw}}{ }ſei für »alle Zeiten« ein{ }ſchönes Stück,{ }ſo{ }ſprichſt Du im Namen von »allen
               Zeiten« ein künſtleriſches Urtheil aus, zu dem »alle Zeiten« Dich gewiß {\pb}nicht ermächtigt haben.\pend
           
\pstart
           \label{K_L03093-1v}\edtext{Wann kommſt Du?}{\lemma{\textnormal{\emph{Wann kommst Du?}}}\Cendnote{\textnormal{Schnitzler war vom 28. 12. 1901 bis zum 6. 1. 1902 in Berlin\oindex{Berlin@\textbf{Berlin}, \emph{Hauptstadt}|pwk}. Er und Goldmann\pwindex{Goldmann, Paul 31.\,1.\,1865 Breslau – 25.\,9.\,1935 Wien@\textsc{Goldmann, Paul} (31.\,1.\,1865 Breslau – 25.\,9.\,1935 Wien), \emph{Schriftsteller, Journalist}|pwk} sahen sich jedenfalls am 5. 1. 1902 und 6. 1. 1902,
                  höchstwahrscheinlich auch am 4. 1. 1902 bei der Uraufführung von \emph{Lebendige Stunden}\pwindex{Schnitzler, Arthur 15.\,5.\,1862 Wien – 21.\,10.\,1931 ebd.@\textsc{Schnitzler, Arthur} (15.\,5.\,1862 Wien – 21.\,10.\,1931 ebd.), \emph{Schriftsteller, Mediziner}!Lebendige Stunden. Vier Einakter@\strich\emph{Lebendige Stunden. Vier Einakter}|pwk}\eventindex{Berliner Theater@\textbf{Berliner Theater}!Uraufführung von Alt-Heidelberg, 22.11.1901@Uraufführung von Alt-Heidelberg, 22.11.1901|pwk}.}}}\label{K_L03093-1} Ich freue mich{ }ſehr darauf, Dich
               wiederzuſehen.\pend
           
\pstart
           Haſt Du \label{K_L03093-2v}\edtext{\textsc{Hirschfelds\pwindex{Hirschfeld, Robert 17.\,9.\,1857 Žďár nad Sázavou – 2.\,4.\,1914 Salzburg@\textsc{Hirschfeld, Robert} (17.\,9.\,1857 Žďár nad Sázavou – 2.\,4.\,1914 Salzburg), \emph{Journalist, Musikkritiker}|pw}}{ }Feuilleton\pwindex{Hirschfeld, Robert 17.\,9.\,1857 Žďár nad Sázavou – 2.\,4.\,1914 Salzburg@\textsc{Hirschfeld, Robert} (17.\,9.\,1857 Žďár nad Sázavou – 2.\,4.\,1914 Salzburg), \emph{Journalist, Musikkritiker}!Wiener Leben@\strich\emph{Wiener Leben}|pwv}}{\lemma{\textnormal{\emph{Hirschfelds Feuilleton}}}\Cendnote{\textnormal{Robert Hirschfeld\pwindex{Hirschfeld, Robert 17.\,9.\,1857 Žďár nad Sázavou – 2.\,4.\,1914 Salzburg@\textsc{Hirschfeld, Robert} (17.\,9.\,1857 Žďár nad Sázavou – 2.\,4.\,1914 Salzburg), \emph{Journalist, Musikkritiker}|pwk}: \emph{Wiener Leben}\pwindex{Hirschfeld, Robert 17.\,9.\,1857 Žďár nad Sázavou – 2.\,4.\,1914 Salzburg@\textsc{Hirschfeld, Robert} (17.\,9.\,1857 Žďár nad Sázavou – 2.\,4.\,1914 Salzburg), \emph{Journalist, Musikkritiker}!Wiener Leben@\strich\emph{Wiener Leben}|pwk}. In: \emph{Frankfurter Zeitung}\pwindex{Frankfurter Zeitung@\emph{Frankfurter Zeitung}|pwk}, Jg. 46, Nr. 333, 1. 12. 1901, Erstes
                     Morgenblatt, S. 1–2. }}}\label{K_L03093-2} in der Frkf. Ztg.\pwindex{Frankfurter Zeitung@\emph{Frankfurter Zeitung}|pw} geleſen? Wenn das Jung-Wiener
                  Theater\orgindex{Jung-Wiener Theater zum Lieben Augustin@Jung-Wiener Theater zum Lieben Augustin|pw}{ }ſo erbärmlich war, wie es darin geſchildert wird,{ }ſo kann ich auch
               der N. Fr. Pr.\orgindex{Neue Freie Presse@Neue Freie Presse|pw} und dem alten \label{K_L03093-3v}\edtext{\textsc{Neuda\pwindex{Neuda, Moriz 22.\,11.\,1842 Loštice – 4.\,2.\,1917 Wien@\textsc{Neuda, Moriz} (22.\,11.\,1842 Loštice – 4.\,2.\,1917 Wien), \emph{Journalist}|pw}\pwindex{Neuda, Moriz 22.\,11.\,1842 Loštice – 4.\,2.\,1917 Wien@\textsc{Neuda, Moriz} (22.\,11.\,1842 Loštice – 4.\,2.\,1917 Wien), \emph{Journalist}!Theater- und Kunstnachrichten. Jung-Wiener-Theater »Zum lieben Augustin«@\strich\emph{Theater- und Kunstnachrichten. Jung-Wiener-Theater »Zum lieben Augustin«}|pwv}}}{\lemma{\textnormal{\emph{Neuda}}}\Cendnote{\textnormal{Siehe XXXX Auszeichnungsfehler: Dokument L03091 nicht gefunden.
               }}}\label{K_L03093-3} nicht Unrecht geben.\pend
           
\pstart
           Ich{ }ſende Dir einen \label{K_L03093-4v}\edtext{Ausſchnitt\pwindex{Hauptmanns Niedergang und die Berliner Litteratur-Tyrannei@\emph{Hauptmanns Niedergang und die Berliner Litteratur-Tyrannei}|pwv}}{\lemma{\textnormal{\emph{Ausschnitt}}}\Cendnote{\textnormal{[O. V.]: \emph{Die Berliner
                        Litteratur-Tyrannei}\pwindex{Hauptmanns Niedergang und die Berliner Litteratur-Tyrannei@\emph{Hauptmanns Niedergang und die Berliner Litteratur-Tyrannei}|pwk}. In: \emph{Frankfurter
                        Zeitung}\pwindex{Frankfurter Zeitung@\emph{Frankfurter Zeitung}|pwk}, Jg. 46, Nr. 332, 30. 11. 1901, Abendblatt,
                     S. 1.}}}\label{K_L03093-4}{ }{\pb}aus einem \label{K_L03093-5v}\edtext{Referat\pwindex{Gerhart Hauptmanns Tragikomödie »Der rothe Hahn«@\emph{Gerhart Hauptmanns Tragikomödie »Der rothe Hahn«}|pwv}}{\lemma{\textnormal{\emph{Referat}}}\Cendnote{\textnormal{[Karl von Perfall\pwindex{Perfall, Karl von 24.\,3.\,1851 Landsberg am Lech – 31.\,8.\,1924 Düsseldorf@\textsc{Perfall, Karl von} (24.\,3.\,1851 Landsberg am Lech – 31.\,8.\,1924 Düsseldorf), \emph{Schriftsteller, Herausgeber}|pwk}]: \emph{Gerhart Hauptmanns Tragikomödie »Der rothe Hahn«}\pwindex{Gerhart Hauptmanns Tragikomödie »Der rothe Hahn«@\emph{Gerhart Hauptmanns Tragikomödie »Der rothe Hahn«}|pwk}. In:
                        \emph{Kölnische Zeitung}\pwindex{Kölnische Zeitung@\emph{Kölnische Zeitung}|pwk}, Nr. 931, 28. 11. 1901, Abend-Ausgabe, S. [2].}}}\label{K_L03093-5}{ }\textsc{Perfalls\pwindex{Perfall, Karl von 24.\,3.\,1851 Landsberg am Lech – 31.\,8.\,1924 Düsseldorf@\textsc{Perfall, Karl von} (24.\,3.\,1851 Landsberg am Lech – 31.\,8.\,1924 Düsseldorf), \emph{Schriftsteller, Herausgeber}|pw}} in der Kölniſchen Zeitung\pwindex{Kölnische Zeitung@\emph{Kölnische Zeitung}|pw}, nur damit Du{ }ſiehſt, daß es außer Herrn
                  \label{K_L03093-6v}\edtext{\textsc{Ebermann\pwindex{Ebermann, Leo 16.\,7.\,1863 Draganovka – 9.\,10.\,1914 Wien@\textsc{Ebermann, Leo} (16.\,7.\,1863 Draganovka – 9.\,10.\,1914 Wien), \emph{Schriftsteller, Journalist, Rechtswissenschaftler}|pw}}}{\lemma{\textnormal{\emph{Ebermann}}}\Cendnote{\textnormal{Siehe XXXX Auszeichnungsfehler: Dokument L03090 nicht gefunden.
               }}}\label{K_L03093-6} auch noch andere Leute gibt, die meine Anſicht theilen.\pend
           
\pstart
           Viele treue Grüße! {\\[\baselineskip]}Dein \spacefill\mbox{Paul Goldmann.}\pend
           \leftskip=0em{}{\vspace{1\baselineskip}}
\pstart
           \textcolor{gray}{\textbf{\textbf{Hauptmanns\pwindex{Hauptmann, Gerhart 15.\,11.\,1862 Szczawno-Zdrój – 6.\,6.\,1946 Jagniątków@\textsc{Hauptmann, Gerhart} (15.\,11.\,1862 Szczawno-Zdrój – 6.\,6.\,1946 Jagniątków), \emph{Schriftsteller}|pw}s Niedergang und die Berlin\oindex{Berlin@\textbf{Berlin}, \emph{Hauptstadt}|pw}er Litteratur-Tyrannei.} In der
                        »Kölniſchen Zeitung\pwindex{Kölnische Zeitung@\emph{Kölnische Zeitung}|pw}« leſen wir: »{\dots} Der Mißerfolg des ›roten
                     Hahns\pwindex{Hauptmann, Gerhart 15.\,11.\,1862 Szczawno-Zdrój – 6.\,6.\,1946 Jagniątków@\textsc{Hauptmann, Gerhart} (15.\,11.\,1862 Szczawno-Zdrój – 6.\,6.\,1946 Jagniątków), \emph{Schriftsteller}!rothe Hahn. Tragikomödie in vier Akten@\strich\emph{Der rothe Hahn. Tragikomödie in vier Akten}|pw}‹,
                     der dem Mißerfolge des ›Michael Kramer\pwindex{Hauptmann, Gerhart 15.\,11.\,1862 Szczawno-Zdrój – 6.\,6.\,1946 Jagniątków@\textsc{Hauptmann, Gerhart} (15.\,11.\,1862 Szczawno-Zdrój – 6.\,6.\,1946 Jagniątków), \emph{Schriftsteller}!Michael Kramer. Drama@\strich\emph{Michael Kramer. Drama}|pw}‹
                     folgt, läßt kaum noch die Hoffnung übrig, daß Hauptmann\pwindex{Hauptmann, Gerhart 15.\,11.\,1862 Szczawno-Zdrój – 6.\,6.\,1946 Jagniątków@\textsc{Hauptmann, Gerhart} (15.\,11.\,1862 Szczawno-Zdrój – 6.\,6.\,1946 Jagniątków), \emph{Schriftsteller}|pw} über{ }ſeine früheren Werke zu einer großen Dramatik
                     aufſteigen wird. Es iſt vielmehr ziemlich{ }ſicher, daß er beſtenfalls{ }ſich noch
                     einmal auf halber Höhe aufrichtet, aber der Hauptmann\pwindex{Hauptmann, Gerhart 15.\,11.\,1862 Szczawno-Zdrój – 6.\,6.\,1946 Jagniątków@\textsc{Hauptmann, Gerhart} (15.\,11.\,1862 Szczawno-Zdrój – 6.\,6.\,1946 Jagniątków), \emph{Schriftsteller}|pw}, über den eine ganze Litteratur entſtanden iſt, der Hauptmann\pwindex{Hauptmann, Gerhart 15.\,11.\,1862 Szczawno-Zdrój – 6.\,6.\,1946 Jagniątków@\textsc{Hauptmann, Gerhart} (15.\,11.\,1862 Szczawno-Zdrój – 6.\,6.\,1946 Jagniątków), \emph{Schriftsteller}|pw}, in dem man die Zukunft des
                     Deutſchen Dramas ahnen wollte, dieſer Hauptmann\pwindex{Hauptmann, Gerhart 15.\,11.\,1862 Szczawno-Zdrój – 6.\,6.\,1946 Jagniątków@\textsc{Hauptmann, Gerhart} (15.\,11.\,1862 Szczawno-Zdrój – 6.\,6.\,1946 Jagniątków), \emph{Schriftsteller}|pw} iſt geweſen, und die deutſche Litteratur geht über ihn
                     hinweg, weil{ }ſie{ }ſchon über manchen kurzlebigen Stern, der an dem Theaterhimmel
                     glänzte, hinweggegangen iſt. Aber Hauptmanns\pwindex{Hauptmann, Gerhart 15.\,11.\,1862 Szczawno-Zdrój – 6.\,6.\,1946 Jagniątków@\textsc{Hauptmann, Gerhart} (15.\,11.\,1862 Szczawno-Zdrój – 6.\,6.\,1946 Jagniątków), \emph{Schriftsteller}|pw}s Niedergang bedeutet, wie die Dinge einmal liegen, noch
                     mehr. Hauptmann\pwindex{Hauptmann, Gerhart 15.\,11.\,1862 Szczawno-Zdrój – 6.\,6.\,1946 Jagniątków@\textsc{Hauptmann, Gerhart} (15.\,11.\,1862 Szczawno-Zdrój – 6.\,6.\,1946 Jagniątków), \emph{Schriftsteller}|pw} war ohne{ }ſeinen Willen
                     der große Neuerer, um den{ }ſich ein ganzes Programm, eine ganze Bewegung
                     gebildet hat; er war der heimliche Diktator der deutſch\oindex{Deutschland@\textbf{Deutschland}|pwv}en Theaterlitteratur. Das alles
                     hat ein \label{T_L03093-1v}\edtext{Ende,}{\lemma{\textnormal{\emph{Ende,}}}\Cendnote{\textnormal{In der Vorlage steht
                           »Ende.«}}}\label{T_L03093-1} und mit ihm bricht ein Gebäude zuſammen,
                     in dem eine ganze Schar schwächerer, aber{ }ſehr lauter Geiſter Obdach gefunden
                     hat. Der Durchfall des ›roten Hahns\pwindex{Hauptmann, Gerhart 15.\,11.\,1862 Szczawno-Zdrój – 6.\,6.\,1946 Jagniątków@\textsc{Hauptmann, Gerhart} (15.\,11.\,1862 Szczawno-Zdrój – 6.\,6.\,1946 Jagniątków), \emph{Schriftsteller}!rothe Hahn. Tragikomödie in vier Akten@\strich\emph{Der rothe Hahn. Tragikomödie in vier Akten}|pw}‹ iſt{ }ſo etwas wie ein litterariſcher Börſenſturz, wie eine Kataſtrophe, die ihre
                     Wirkung ausüben muß, wenn auch noch frecher als nach dem ›Michael Kramer\pwindex{Hauptmann, Gerhart 15.\,11.\,1862 Szczawno-Zdrój – 6.\,6.\,1946 Jagniątków@\textsc{Hauptmann, Gerhart} (15.\,11.\,1862 Szczawno-Zdrój – 6.\,6.\,1946 Jagniątków), \emph{Schriftsteller}!Michael Kramer. Drama@\strich\emph{Michael Kramer. Drama}|pw}‹ der Verſuch gemacht werden{ }ſollte, das
                        deutſch\oindex{Deutschland@\textbf{Deutschland}|pwv}e Publikum über
                     die Wahrheit zu täuſchen. Die Berlin\oindex{Berlin@\textbf{Berlin}, \emph{Hauptstadt}|pw}er
                     Litteratur-Tyrannei hat am 27. November ihr
                        \label{T_L03093-2v}\edtext{Ende gefunden}{\lemma{\textnormal{\emph{Ende gefunden}}}\Cendnote{\textnormal{In der Vorlage steht
                           »Endegefunden«.}}}\label{T_L03093-2}.« –\pwindex{Hauptmanns Niedergang und die Berliner Litteratur-Tyrannei@\emph{Hauptmanns Niedergang und die Berliner Litteratur-Tyrannei}|pwv}}}\pend
           \selectlanguage{ngerman}\endnumbering\briefempfaengerindex{Schnitzler, Arthur@\textsc{Schnitzler, Arthur}!zzzGoldmann, Paul@\emph{von Paul Goldmann}!1901-12-041@{4. 12. [1901]}|)be}\mylabel{L03093h}  \newcommand{\dateiname}{L03093}\newcommand{\titel}{Paul Goldmann an Arthur Schnitzler, 4. 12. [1901]}\newcommand{\editorInnen}{Martin Anton Müller und Laura Untner}%% latex-leseansicht-abspann.tex
%% Abspann für die Leseansicht.
%% Der Schalter \ifkorrekturansicht ist bereits durch den Vorspann gesetzt.

%% latex-abspann.tex
%% Gemeinsamer Abspann für Korrekturansicht und Leseansicht.
%% Setzt den Schalter \ifkorrekturansicht voraus (gesetzt in den
%% einbindenden Dateien latex-korrekturansicht-abspann.tex bzw.
%% latex-leseansicht-abspann.tex).
%% ---------------------------------------------------------------

\normalsize

% Das esempio-Environment wird nur in der Leseansicht benötigt
\ifkorrekturansicht\else
\newenvironment{esempio}[3]%
{
    \vspace{1.5ex}
    \rlap{\underline{#1}}
    \par
    \setlength{\parindent}{0cm}
    \nopagebreak
    \leftskip=#2cm
    \rightskip=#3cm
}
{
    \par
}
\fi

\doendnotes{C}
\bigskip
\vfill

\clearpage

\footnotesize

\ifkorrekturansicht
  \lohead{\textsc{register}}
\fi

% theindex-Environment neu definieren ohne reledmac
\makeatletter
\renewenvironment{theindex}{%
  \ifkorrekturansicht
    \section*{\indexname}%
  \else
    \subsubsection*{Index der erwähnten Entitäten}%
  \fi
  \setlength{\parindent}{0pt}%
  \setlength{\parskip}{0pt plus 0.3pt}%
  \let\item\@idxitem
}{%
  \ifkorrekturansicht\clearpage\fi
}
\makeatother

\IfFileExists{\jobname-pw.ind}{\input{\jobname-pw.ind}}{}

% Quellenangabe nur in der Leseansicht
\ifkorrekturansicht\else
% Fallback-Definitionen, falls die .tex-Datei \titel etc. nicht gesetzt hat
\providecommand{\titel}{}
\providecommand{\editorInnen}{}
\providecommand{\dateiname}{\jobname}

\vspace{3cm}

\vfill

\footnotesize
\textsc{Quelle}: \titel. Herausgegeben von {\editorInnen}. In: \emph{Arthur Schnitzler: Briefwechsel mit Autorinnen und Autoren}.
 Digitale Edition, https://schnitzler-briefe.acdh.oeaw.ac.at/{\dateiname}.html (Stand \today)
\fi

\end{document}


