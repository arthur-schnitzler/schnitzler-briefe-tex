%% latex-korrekturansicht-vorspann.tex
%% Vorspann für die Korrekturansicht.
%% Lädt die gemeinsame Datei latex-vorspann.tex mit gesetztem Schalter.

\newif\ifkorrekturansicht
\korrekturansichttrue

\input{../tex-inputs/latex-vorspann}


\section[ Paul Goldmann an Arthur Schnitzler, 4. 12. {[}1901{]}]{L03093 Paul Goldmann an Arthur Schnitzler, 4. 12. {[}1901{]}}
\nopagebreak\mylabel{L03093v}
\rehead{ }\normalsize\beginnumbering\briefempfaengerindex{Schnitzler, Arthur@\textsc{Schnitzler, Arthur}!zzzGoldmann, Paul@\emph{von Paul Goldmann}!1901-12-041@{4. 12. {[}1901{]}}|(be}
\toendnotes[C]{\smallbreak\pagebreak[2]}\Standort{DLA, A:Schnitzler, HS.NZ85.1.3171.}
\physDesc{Brief, 1 Blatt, 4 Seiten, 1165 Zeichen
\newline{}Handschrift: blaue Tinte, deutsche Kurrent
\newline{}Beilage: Zeitungsausschnitt (zwei Teile), beschnitten und
                                 aufgeklebt 
\newline{}Schnitzler: 1) mit Bleistift das Jahr »901« vermerkt  2) mit rotem Buntstift zwei Unterstreichungen}\toendnotes[C]{\smallbreak}
\pstart
           \raggedleft{}{\pb}\textcolor{gray}{\textbf{DESSAUERSTRASSE 19}}\oindex{Dessauer Strasse@\textbf{Dessauer Straße}, \emph{Straße (K.STR)}|pw}\pend
           
\pstart
           Berlin\oindex{Berlin@\textbf{Berlin}, \emph{P.PPLC}|pw}, 4. Dezember.\pend
           
\pstart\center{}Mein lieber Freund,\pend\vspace{0.5em}
\pstart
           Zolltarif im Reichſtag\orgindex{Reichstag@Reichstag|pw}. Ich habe keine freie
               Minute.\pend
           
\pstart
           Tauſend Dank für Deinen lieben Brief.\pend
           
\pstart
           Über Deine Auslegung, daß \textsc{Hauptmann\pwindex{Hauptmann, Gerhart 15.11.1862 – 06.06.1946@\textsc{Hauptmann, Gerhart} (15.11.1862 – 06.06.1946), \emph{Schriftsteller/Schriftstellerin}|pw}} eine geiſtige Krankheit durchmacht, habe ich den Kopf geſchüttelt. Warum eine
               Erklärung {\pb}an den Haaren herbeiziehen? Warum das
               Eigentliche nicht ſehen wollen? Wenn Einer geiſtig leer iſt, ſo iſt er immer geiſtig
               leer geweſen. Man kann ein Stück verfehlen, man kann aber nicht auf einmal weder
               Geiſt noch Talent haben. Und was Deine Anſicht betrifft, \textsc{Hannele\pwindex{Hanneles Himmelfahrt. Traumdichtung in zwei Teilen@\emph{Hanneles Himmelfahrt. Traumdichtung in zwei Teilen}|pw}} ſei für »alle Zeiten« ein ſchönes Stück, ſo ſprichſt Du im Namen von »allen
               Zeiten« ein künſtleriſches Urtheil aus, zu dem »alle Zeiten« Dich gewiß {\pb}nicht ermächtigt haben.\pend
           
\pstart
           \label{K_L03093-1v}\edtext{Wann kommſt Du?}{\lemma{\textnormal{\emph{Wann kommſt Du?}}}\Cendnote{\textnormal{Schnitzler war vom 28. 12. 1901 bis zum 6. 1. 1902 in Berlin\oindex{Berlin@\textbf{Berlin}, \emph{P.PPLC}|pwk}. Er und Goldmann\pwindex{Goldmann, Paul 31.01.1865 – 25.09.1935@\textsc{Goldmann, Paul} (31.01.1865 – 25.09.1935), \emph{Schriftsteller/Schriftstellerin, Journalist/Journalistin}|pwk} sahen sich jedenfalls am 5. 1. 1902 und 6. 1. 1902,
                  höchstwahrscheinlich auch am 4. 1. 1902 bei der Uraufführung von \emph{Lebendige Stunden}\pwindex{Lebendige Stunden. Vier Einakter@\emph{Lebendige Stunden. Vier Einakter}|pwk}.}}}\label{K_L03093-1} Ich freue mich ſehr darauf, Dich
               wiederzuſehen.\pend
           
\pstart
           Haſt Du \label{K_L03093-2v}\edtext{\textsc{Hirschfelds\pwindex{Hirschfeld, Robert 17.09.1857 – 02.04.1914@\textsc{Hirschfeld, Robert} (17.09.1857 – 02.04.1914), \emph{Journalist/Journalistin, Musikkritiker/Musikkritikerin}|pw}}{ }Feuilleton\pwindex{Wiener Leben@\emph{Wiener Leben}|pwv}}{\lemma{\textnormal{\emph{Hirschfelds Feuilleton}}}\Cendnote{\textnormal{Robert Hirschfeld\pwindex{Hirschfeld, Robert 17.09.1857 – 02.04.1914@\textsc{Hirschfeld, Robert} (17.09.1857 – 02.04.1914), \emph{Journalist/Journalistin, Musikkritiker/Musikkritikerin}|pwk}: \emph{Wiener Leben}\pwindex{Wiener Leben@\emph{Wiener Leben}|pwk}. In: \emph{Frankfurter Zeitung}\pwindex{Frankfurter Zeitung@\emph{Frankfurter Zeitung}|pwk}, Jg. 46, Nr. 333, 1. 12. 1901, Erstes
                     Morgenblatt, S. 1–2. }}}\label{K_L03093-2} in der Frkf. Ztg.\pwindex{Frankfurter Zeitung@\emph{Frankfurter Zeitung}|pw} geleſen? Wenn das Jung-Wiener
                  Theater\orgindex{Jung-Wiener Theater zum Lieben Augustin@Jung-Wiener Theater zum Lieben Augustin|pw} ſo erbärmlich war, wie es darin geſchildert wird, ſo kann ich auch
               der N. Fr. Pr.\orgindex{Neue Freie Presse@Neue Freie Presse|pw} und dem alten \label{K_L03093-3v}\edtext{\textsc{Neuda\pwindex{Neuda, Moriz 1842-11-22 – 1917-02-04@\textsc{Neuda, Moriz} (1842-11-22 – 1917-02-04), \emph{Journalist/Journalistin}|pw}\pwindex{Theater- und Kunstnachrichten. Jung-Wiener-Theater »Zum lieben Augustin«@\emph{Theater- und Kunstnachrichten. Jung-Wiener-Theater »Zum lieben Augustin«}|pwv}}}{\lemma{\textnormal{\emph{Neuda}}}\Cendnote{\textnormal{Siehe Paul Goldmann an Arthur Schnitzler, 23. 11. [1901].
               }}}\label{K_L03093-3} nicht Unrecht geben.\pend
           
\pstart
           Ich ſende Dir einen \label{K_L03093-4v}\edtext{Ausſchnitt\pwindex{Hauptmanns Niedergang und die Berliner Litteratur-Tyrannei@\emph{Hauptmanns Niedergang und die Berliner Litteratur-Tyrannei}|pwv}}{\lemma{\textnormal{\emph{Ausſchnitt}}}\Cendnote{\textnormal{[O. V.]: \emph{Die Berliner
                        Litteratur-Tyrannei}\pwindex{Hauptmanns Niedergang und die Berliner Litteratur-Tyrannei@\emph{Hauptmanns Niedergang und die Berliner Litteratur-Tyrannei}|pwk}. In: \emph{Frankfurter
                        Zeitung}\pwindex{Frankfurter Zeitung@\emph{Frankfurter Zeitung}|pwk}, Jg. 46, Nr. 332, 30. 11. 1901, Abendblatt,
                     S. 1.}}}\label{K_L03093-4}{ }{\pb}aus einem \label{K_L03093-5v}\edtext{Referat\pwindex{Gerhart Hauptmanns Tragikomoedie »Der rothe Hahn«@\emph{Gerhart Hauptmanns Tragikomödie »Der rothe Hahn«}|pwv}}{\lemma{\textnormal{\emph{Referat}}}\Cendnote{\textnormal{[Karl von Perfall\pwindex{Perfall, Karl von 24.03.1851 – 31.08.1924@\textsc{Perfall, Karl von} (24.03.1851 – 31.08.1924), \emph{Schriftsteller/Schriftstellerin, Herausgeber/Herausgeberin}|pwk}]: \emph{Gerhart Hauptmanns Tragikomödie »Der rothe Hahn«}\pwindex{Gerhart Hauptmanns Tragikomoedie »Der rothe Hahn«@\emph{Gerhart Hauptmanns Tragikomödie »Der rothe Hahn«}|pwk}. In:
                        \emph{Kölnische Zeitung}\pwindex{Koelnische Zeitung@\emph{Kölnische Zeitung}|pwk}, Nr. 931, 28. 11. 1901, Abend-Ausgabe, S. [2].}}}\label{K_L03093-5}{ }\textsc{Perfalls\pwindex{Perfall, Karl von 24.03.1851 – 31.08.1924@\textsc{Perfall, Karl von} (24.03.1851 – 31.08.1924), \emph{Schriftsteller/Schriftstellerin, Herausgeber/Herausgeberin}|pw}} in der Kölniſchen Zeitung\pwindex{Koelnische Zeitung@\emph{Kölnische Zeitung}|pw}, nur damit Du ſiehſt, daß es außer Herrn
                  \label{K_L03093-6v}\edtext{\textsc{Ebermann\pwindex{Ebermann, Leo 16.07.1863 – 09.10.1914@\textsc{Ebermann, Leo} (16.07.1863 – 09.10.1914), \emph{Schriftsteller/Schriftstellerin, Journalist/Journalistin, Rechtswissenschaftler/Rechtswissenschaftlerin}|pw}}}{\lemma{\textnormal{\emph{Ebermann}}}\Cendnote{\textnormal{Siehe Paul Goldmann an Arthur Schnitzler, 9. 11. [1901].
               }}}\label{K_L03093-6} auch noch andere Leute gibt, die meine Anſicht theilen.\pend
           
\pstart
           Viele treue Grüße! {\\[\baselineskip]}Dein \spacefill\mbox{Paul Goldmann.}\pend
           \leftskip=0em{}{\vspace{1\baselineskip}}
\pstart
           \textcolor{gray}{\textbf{\textbf{Hauptmanns\pwindex{Hauptmann, Gerhart 15.11.1862 – 06.06.1946@\textsc{Hauptmann, Gerhart} (15.11.1862 – 06.06.1946), \emph{Schriftsteller/Schriftstellerin}|pw}s Niedergang und die Berlin\oindex{Berlin@\textbf{Berlin}, \emph{P.PPLC}|pw}er Litteratur-Tyrannei.} In der
                        »Kölniſchen Zeitung\pwindex{Koelnische Zeitung@\emph{Kölnische Zeitung}|pw}« leſen wir: »{\dots} Der Mißerfolg des ›roten
                     Hahns\pwindex{rothe Hahn. Tragikomoedie in vier Akten@\emph{Der rothe Hahn. Tragikomödie in vier Akten}|pw}‹,
                     der dem Mißerfolge des ›Michael Kramer\pwindex{Michael Kramer. Drama@\emph{Michael Kramer. Drama}|pw}‹
                     folgt, läßt kaum noch die Hoffnung übrig, daß Hauptmann\pwindex{Hauptmann, Gerhart 15.11.1862 – 06.06.1946@\textsc{Hauptmann, Gerhart} (15.11.1862 – 06.06.1946), \emph{Schriftsteller/Schriftstellerin}|pw} über ſeine früheren Werke zu einer großen Dramatik
                     aufſteigen wird. Es iſt vielmehr ziemlich ſicher, daß er beſtenfalls ſich noch
                     einmal auf halber Höhe aufrichtet, aber der Hauptmann\pwindex{Hauptmann, Gerhart 15.11.1862 – 06.06.1946@\textsc{Hauptmann, Gerhart} (15.11.1862 – 06.06.1946), \emph{Schriftsteller/Schriftstellerin}|pw}, über den eine ganze Litteratur entſtanden iſt, der Hauptmann\pwindex{Hauptmann, Gerhart 15.11.1862 – 06.06.1946@\textsc{Hauptmann, Gerhart} (15.11.1862 – 06.06.1946), \emph{Schriftsteller/Schriftstellerin}|pw}, in dem man die Zukunft des
                     Deutſchen Dramas ahnen wollte, dieſer Hauptmann\pwindex{Hauptmann, Gerhart 15.11.1862 – 06.06.1946@\textsc{Hauptmann, Gerhart} (15.11.1862 – 06.06.1946), \emph{Schriftsteller/Schriftstellerin}|pw} iſt geweſen, und die deutſche Litteratur geht über ihn
                     hinweg, weil ſie ſchon über manchen kurzlebigen Stern, der an dem Theaterhimmel
                     glänzte, hinweggegangen iſt. Aber Hauptmanns\pwindex{Hauptmann, Gerhart 15.11.1862 – 06.06.1946@\textsc{Hauptmann, Gerhart} (15.11.1862 – 06.06.1946), \emph{Schriftsteller/Schriftstellerin}|pw}s Niedergang bedeutet, wie die Dinge einmal liegen, noch
                     mehr. Hauptmann\pwindex{Hauptmann, Gerhart 15.11.1862 – 06.06.1946@\textsc{Hauptmann, Gerhart} (15.11.1862 – 06.06.1946), \emph{Schriftsteller/Schriftstellerin}|pw} war ohne ſeinen Willen
                     der große Neuerer, um den ſich ein ganzes Programm, eine ganze Bewegung
                     gebildet hat; er war der heimliche Diktator der deutſch\oindex{Deutschland@\textbf{Deutschland}, \emph{A.PCLI}|pwv}en Theaterlitteratur. Das alles
                     hat ein \label{T_L03093-1v}\edtext{Ende,}{\lemma{\textnormal{\emph{Ende,}}}\Cendnote{\textnormal{In der Vorlage steht
                           »Ende.«}}}\label{T_L03093-1} und mit ihm bricht ein Gebäude zuſammen,
                     in dem eine ganze Schar schwächerer, aber ſehr lauter Geiſter Obdach gefunden
                     hat. Der Durchfall des ›roten Hahns\pwindex{rothe Hahn. Tragikomoedie in vier Akten@\emph{Der rothe Hahn. Tragikomödie in vier Akten}|pw}‹ iſt
                     ſo etwas wie ein litterariſcher Börſenſturz, wie eine Kataſtrophe, die ihre
                     Wirkung ausüben muß, wenn auch noch frecher als nach dem ›Michael Kramer\pwindex{Michael Kramer. Drama@\emph{Michael Kramer. Drama}|pw}‹ der Verſuch gemacht werden ſollte, das
                        deutſch\oindex{Deutschland@\textbf{Deutschland}, \emph{A.PCLI}|pwv}e Publikum über
                     die Wahrheit zu täuſchen. Die Berlin\oindex{Berlin@\textbf{Berlin}, \emph{P.PPLC}|pw}er
                     Litteratur-Tyrannei hat am 27. November ihr
                        \label{T_L03093-2v}\edtext{Ende gefunden}{\lemma{\textnormal{\emph{Ende gefunden}}}\Cendnote{\textnormal{In der Vorlage steht
                           »Endegefunden«.}}}\label{T_L03093-2}.« –\pwindex{Hauptmanns Niedergang und die Berliner Litteratur-Tyrannei@\emph{Hauptmanns Niedergang und die Berliner Litteratur-Tyrannei}|pwv}}}\pend
           \selectlanguage{ngerman}\endnumbering\briefempfaengerindex{Schnitzler, Arthur@\textsc{Schnitzler, Arthur}!zzzGoldmann, Paul@\emph{von Paul Goldmann}!1901-12-041@{4. 12. {[}1901{]}}|)be}\mylabel{L03093h}  \normalsize

\doendnotes{C}
\bigskip
\vfill

\clearpage

\footnotesize

\lohead{\textsc{register}}

% Definiere theindex-Environment komplett neu ohne reledmac
\makeatletter
\renewenvironment{theindex}{%
  \section*{\indexname}%
  \setlength{\parindent}{0pt}%
  \setlength{\parskip}{0pt plus 0.3pt}%
  \let\item\@idxitem
}{%
  \clearpage
}
\makeatother

\IfFileExists{\jobname-pw.ind}{\input{\jobname-pw.ind}}{}

\end{document}

      