%% latex-leseansicht-vorspann.tex
%% Vorspann für die Leseansicht.
%% Lädt die gemeinsame Datei latex-vorspann.tex mit nicht gesetztem Schalter.

\newif\ifkorrekturansicht
\korrekturansichtfalse

\input{../tex-inputs/latex-vorspann}


\section[Arthur Schnitzler an Gustav Schwarzkopf, 28. 4. 1899]{L04140 Arthur Schnitzler an Gustav Schwarzkopf, 28. 4. 1899}
\nopagebreak\mylabel{L04140v}
\rehead{ }\normalsize\beginnumbering\briefempfaengerindex{Schwarzkopf, Gustav@\textsc{Schwarzkopf, Gustav}!zzzSchnitzler, Arthur@\emph{von Arthur Schnitzler}!1899-04-281@{28. 4. 1899}|(be}
\toendnotes[C]{\smallbreak\pagebreak[2]}
\correspDesc{Versand  durch Arthur Schnitzler am 28. 4. 1899 in Berlin
\newline{}Erhalt  durch Gustav Schwarzkopf im Zeitraum [29. 4. 1899 – 3. 5. 1899?] in Wien}\toendnotes[C]{\smallbreak}
\Standort{CUL, Schnitzler, B 96.}
\physDesc{Postkarte, 516 Zeichen
\newline{}Handschrift: Bleistift, deutsche Kurrent
\newline{}Versand: Stempel: »\nobreak{}\oindex{Berlin@\textbf{Berlin}, \emph{Hauptstadt}|pwk}Berlin N. W., 28. 4. 99, 6–7N\nobreak{}«.  }\toendnotes[C]{\smallbreak}\pstart{}{\pb}Herrn \textsc{Gustav Schwarzkopf}\pend{}\pstart{}\substVorne{}\textsuperscript{B}\substDazwischen{}W\oindex{I., Innere Stadt@\textbf{I., Innere Stadt}, \emph{Verwaltungsgebiet}|pw}\substHinten{}ien I\oindex{I., Innere Stadt@\textbf{I., Innere Stadt}, \emph{Verwaltungsgebiet}|pw}\pend{}\pstart{}\textsc{Tiefer
                        Graben 23}\oindex{Wien@\textbf{Wien}!I., Innere Stadt@\textbf{I., Innere Stadt}!Tiefer Graben 23@\textbf{Tiefer Graben 23}, \emph{Wohngebäude}|pw}.\pend{}{\bigskip}\vspace{1em}
\pstart
           \noindent{}{\pb}Mein lieber Guſtav, heute hatten wir Generalprobe\pwindex{Schnitzler, Arthur 15. 5. 1862 Wien – 21. 10. 1931 ebd.@\textsc{Schnitzler, Arthur} (15. 5. 1862 Wien – 21. 10. 1931 ebd.), \emph{Schriftsteller, Mediziner}!grüne Kakadu – Paracelsus – Die Gefährtin. Drei Einakter@\strich\emph{Der grüne Kakadu – Paracelsus – Die Gefährtin. Drei Einakter}|pwv}\eventindex{Deutsches Theater Berlin@\textbf{Deutsches Theater Berlin}!Generalprobe von Der grüne Kakadu, 28.4.1899@Generalprobe von Der grüne Kakadu, 28.4.1899|pwv} (viel gutes, manches gerade zu dilettantiſch) morgen \textsc{Première}\pwindex{Schnitzler, Arthur 15. 5. 1862 Wien – 21. 10. 1931 ebd.@\textsc{Schnitzler, Arthur} (15. 5. 1862 Wien – 21. 10. 1931 ebd.), \emph{Schriftsteller, Mediziner}!grüne Kakadu – Paracelsus – Die Gefährtin. Drei Einakter@\strich\emph{Der grüne Kakadu – Paracelsus – Die Gefährtin. Drei Einakter}|pwv}\eventindex{Deutsches Theater Berlin@\textbf{Deutsches Theater Berlin}!Premiere von Der grüne Kakadu – Paracelsus – Die Gefährtin. Drei Einakter, 29.4.1899@Premiere von Der grüne Kakadu – Paracelsus – Die Gefährtin. Drei Einakter, 29.4.1899|pwv}; \label{K_L04140-1v}\edtext{Montag oder Dinſtag}{\lemma{\textnormal{\emph{Montag oder Dinstag}}}\Cendnote{\textnormal{Er reiste
                  am Dienstag, dem 2. 5. 1899 abends ab.}}}\label{K_L04140-1} verlaſſe ich Berlin\oindex{Berlin@\textbf{Berlin}, \emph{Hauptstadt}|pw} und fahre wohl direct nach Wien\oindex{Wien@\textbf{Wien}, \emph{Verwaltungsgebiet}|pw}. Die Wiener\oindex{Wien@\textbf{Wien}, \emph{Verwaltungsgebiet}|pw} Nachmittage fehlen mir gewiſs
               tauſendmal mehr als Ihnen!– »Hans\pwindex{Dreyer, Max 25.\,9.\,1862 Rostock – 27.\,11.\,1946 Göhren@\textsc{Dreyer, Max} (25.\,9.\,1862 Rostock – 27.\,11.\,1946 Göhren), \emph{Schriftsteller}!Hans. Drama in drei Aufzügen@\strich\emph{Hans. Drama in drei Aufzügen}|pw}« hab ich, mit viel
               Befriedigung geſehen\eventindex{Deutsches Theater Berlin@\textbf{Deutsches Theater Berlin}!Aufführung von Hans, Mutterherz, 25.4.1899@Aufführung von Hans, Mutterherz, 25.4.1899|pwv}, da{\geminationn}
               die »\textsc{pouppée\pwindex{Audran, Edmond 11.\,4.\,1842 Lyon – 17.\,8.\,1901 Tierceville@\textsc{Audran, Edmond} (11.\,4.\,1842 Lyon – 17.\,8.\,1901 Tierceville), \emph{Schauspieler, Komponist, Musiker}!Puppe@\strich\emph{Die Puppe}|pw}}\eventindex{Central-Theater@\textbf{Central-Theater}!Aufführung von Die Puppe, 26.4.1899@Aufführung von Die Puppe, 26.4.1899|pwv}«; ſonſt gibts hier eigentlich nichts.
                   Der Winter iſt ſchön. – Grüßen Sie Richard\pwindex{Beer-Hofmann, Richard 11.\,7.\,1866 Wien – 26.\,9.\,1945 New York City@\textsc{Beer-Hofmann, Richard} (11.\,7.\,1866 Wien – 26.\,9.\,1945 New York City), \emph{Schriftsteller}|pw}{ }\introOben{}und Waſſermann\pwindex{Wassermann, Jakob 10.\,3.\,1873 Fürth – 1.\,1.\,1934 Altaussee@\textsc{Wassermann, Jakob} (10.\,3.\,1873 Fürth – 1.\,1.\,1934 Altaussee), \emph{Schriftsteller}|pw}\introOben{}; haben Sie von Salten\pwindex{Salten, Felix 6.\,9.\,1869 Budapest – 8.\,10.\,1945 Zürich@\textsc{Salten, Felix} (6.\,9.\,1869 Budapest – 8.\,10.\,1945 Zürich), \emph{Schriftsteller, Journalist, Chefredakteur}|pw} irgend
               was gehört? Ich nicht. –\pend
           \pstart Herzlichſt Ihr \spacefill\mbox{A. S.}\pend{}\selectlanguage{ngerman}\endnumbering\briefempfaengerindex{Schwarzkopf, Gustav@\textsc{Schwarzkopf, Gustav}!zzzSchnitzler, Arthur@\emph{von Arthur Schnitzler}!1899-04-281@{28. 4. 1899}|)be}\mylabel{L04140h}
\begin{anhang}
\end{anhang}\newcommand{\dateiname}{L04140}\newcommand{\titel}{Arthur Schnitzler an Gustav Schwarzkopf, 28. 4. 1899}\newcommand{\editorInnen}{Herausgegeben von Jahnke, SelmaMüller, Martin Anton}%% latex-leseansicht-abspann.tex
%% Abspann für die Leseansicht.
%% Der Schalter \ifkorrekturansicht ist bereits durch den Vorspann gesetzt.

%% latex-abspann.tex
%% Gemeinsamer Abspann für Korrekturansicht und Leseansicht.
%% Setzt den Schalter \ifkorrekturansicht voraus (gesetzt in den
%% einbindenden Dateien latex-korrekturansicht-abspann.tex bzw.
%% latex-leseansicht-abspann.tex).
%% ---------------------------------------------------------------

\normalsize

% Das esempio-Environment wird nur in der Leseansicht benötigt
\ifkorrekturansicht\else
\newenvironment{esempio}[3]%
{
    \vspace{1.5ex}
    \rlap{\underline{#1}}
    \par
    \setlength{\parindent}{0cm}
    \nopagebreak
    \leftskip=#2cm
    \rightskip=#3cm
}
{
    \par
}
\fi

\doendnotes{C}
\bigskip
\vfill

\clearpage

\footnotesize

\ifkorrekturansicht
  \lohead{\textsc{register}}
\fi

% theindex-Environment neu definieren ohne reledmac
\makeatletter
\renewenvironment{theindex}{%
  \ifkorrekturansicht
    \section*{\indexname}%
  \else
    \subsubsection*{Index der erwähnten Entitäten}%
  \fi
  \setlength{\parindent}{0pt}%
  \setlength{\parskip}{0pt plus 0.3pt}%
  \let\item\@idxitem
}{%
  \ifkorrekturansicht\clearpage\fi
}
\makeatother

\IfFileExists{\jobname-pw.ind}{\input{\jobname-pw.ind}}{}

% Quellenangabe nur in der Leseansicht
\ifkorrekturansicht\else
% Fallback-Definitionen, falls die .tex-Datei \titel etc. nicht gesetzt hat
\providecommand{\titel}{}
\providecommand{\editorInnen}{}
\providecommand{\dateiname}{\jobname}

\vspace{3cm}

\vfill

\footnotesize
\textsc{Quelle}: \titel. Herausgegeben von {\editorInnen}. In: \emph{Arthur Schnitzler: Briefwechsel mit Autorinnen und Autoren}.
 Digitale Edition, https://schnitzler-briefe.acdh.oeaw.ac.at/{\dateiname}.html (Stand \today)
\fi

\end{document}


