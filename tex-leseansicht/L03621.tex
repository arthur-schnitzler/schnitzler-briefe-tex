%% latex-leseansicht-vorspann.tex
%% Vorspann für die Leseansicht.
%% Lädt die gemeinsame Datei latex-vorspann.tex mit nicht gesetztem Schalter.

\newif\ifkorrekturansicht
\korrekturansichtfalse

\input{../tex-inputs/latex-vorspann}


\section[Stefan Zweig an Arthur Schnitzler, 15. 1. 190{[}8{]}]{L03621 Stefan Zweig an Arthur Schnitzler, 15. 1. 190[8]}
\nopagebreak\mylabel{L03621v}
\rehead{ }\normalsize\beginnumbering\briefempfaengerindex{Schnitzler, Arthur@\textsc{Schnitzler, Arthur}!zzzZweig, Stefan@\emph{von Stefan Zweig}!1908-01-152@{15. 1. 190[8]}|(be}
\toendnotes[C]{\smallbreak\pagebreak[2]}
\correspDesc{Versand  durch Stefan Zweig am 15. 1. 190[8] in Wien
\newline{}Erhalt  durch Arthur Schnitzler im Zeitraum [15. 1. 1908
                  – 18. 1. 1908?] in Wien}\toendnotes[C]{\smallbreak}
\Standort{CUL, Schnitzler, B 118.}
\physDesc{Brief, 1 Blatt, 2 Seiten, 702 Zeichen
\newline{}Handschrift: schwarze Tinte, lateinische Kurrent
\newline{}Schnitzler: 1) mit rotem Buntstift eine Markierung, eventuell der falschen Datumsangabe  2) mit Bleistift »\textsc{Zweig}«}
\buchAbdrucke{\weitereDrucke{Stefan Zweig: \emph{Briefwechsel mit Hermann Bahr, Sigmund Freud, Rainer Maria
                        Rilke und Arthur Schnitzler}. Herausgegeben von Jeffrey B. Berlin, Hans-Ulrich Lindken und Donald A. Prater. Frankfurt am Main: \emph{S. Fischer} 1987, S. 353.} }\toendnotes[C]{\smallbreak}
\pstart
           {\pb}Wien VIII Kochgasse 8\oindex{Wien@\textbf{Wien}!VIII., Josefstadt@\textbf{VIII., Josefstadt}!Kochgasse 8@\textbf{Kochgasse 8}, \emph{Wohngebäude}|pw}\pend
           
\pstart
           \label{K_L03621-1v}\edtext{15. Januar 1907}{\lemma{\textnormal{\emph{15. Januar 1907}}}\Cendnote{\textnormal{Mit der Jahreszahl
                           »1907« unterlief Zweig\pwindex{Zweig, Stefan 28.\,11.\,1881 Wien – 23.\,2.\,1942 Petrópolis@\textsc{Zweig, Stefan} (28.\,11.\,1881 Wien – 23.\,2.\,1942 Petrópolis), \emph{Schriftsteller}|pwk} ein Schreibfehler. Aus dem Inhalt geht hervor,
                     dass er vom 15. 1. 1908 stammt.}}}\label{K_L03621-1}.\pend
           {\vspace{1\baselineskip}}
\pstart{}Sehr verehrter Herr Doktor,\pend\vspace{0.5em}
\pstart
           gestatten Sie mir als persönlich Unbekanntem Ihnen heute meine aufrichtigen \label{K_L03621-2v}\edtext{Glückwünsche\orgindex{Franz-Grillparzer-Preis@Franz-Grillparzer-Preis|pwv}}{\lemma{\textnormal{\emph{Glückwünsche}}}\Cendnote{\textnormal{Am 15. 1. 1908 erhielt Schnitzler
                  den \emph{Grillparzer-Preis}\orgindex{Franz-Grillparzer-Preis@Franz-Grillparzer-Preis|pwk} für seine Komödie \emph{Zwischenspiel}\pwindex{Schnitzler, Arthur 15.\,5.\,1862 Wien – 21.\,10.\,1931 ebd.@\textsc{Schnitzler, Arthur} (15.\,5.\,1862 Wien – 21.\,10.\,1931 ebd.), \emph{Schriftsteller, Mediziner}!Zwischenspiel. Komödie in drei Akten@\strich\emph{Zwischenspiel. Komödie in drei Akten}|pwk}.}}}\label{K_L03621-2} zu übermitteln. Ich
               glaube, für uns jüngere Leute, die wir in der Bewunderung Ihres Werkes gewissermassen
               aufgewachsen sind, kann es keine grössere Freude geben, als zu sehen, wie Ihnen nun
               auch aus den älteren kälteren Kreisen endlich die grosse Zustimmung wird, die wir so
               lange schon als ein Selbstverständliches ersehnen. Und so einen Tag wollte ich nicht
               vorübergehen zu lassen, ohne Ihnen zu sagen, dass {\pb}es für uns ein Tag der freudigsten
               Genugtuung gewesen ist, unsere Liebe bestätigt zu wissen.\pend
           
\pstart
           In Verehrung getreu{\\[\baselineskip]} Ihr sehr ergebener{\\[\baselineskip]}\spacefill\mbox{Stefan Zweig}\pend
           \leftskip=0em{}\selectlanguage{ngerman}\endnumbering\briefempfaengerindex{Schnitzler, Arthur@\textsc{Schnitzler, Arthur}!zzzZweig, Stefan@\emph{von Stefan Zweig}!1908-01-152@{15. 1. 190[8]}|)be}\mylabel{L03621h}  \newcommand{\dateiname}{L03621}\newcommand{\titel}{Stefan Zweig an Arthur Schnitzler, 15. 1. 190[8]}\newcommand{\editorInnen}{Selma Jahnke und Martin Anton Müller}%% latex-leseansicht-abspann.tex
%% Abspann für die Leseansicht.
%% Der Schalter \ifkorrekturansicht ist bereits durch den Vorspann gesetzt.

%% latex-abspann.tex
%% Gemeinsamer Abspann für Korrekturansicht und Leseansicht.
%% Setzt den Schalter \ifkorrekturansicht voraus (gesetzt in den
%% einbindenden Dateien latex-korrekturansicht-abspann.tex bzw.
%% latex-leseansicht-abspann.tex).
%% ---------------------------------------------------------------

\normalsize

% Das esempio-Environment wird nur in der Leseansicht benötigt
\ifkorrekturansicht\else
\newenvironment{esempio}[3]%
{
    \vspace{1.5ex}
    \rlap{\underline{#1}}
    \par
    \setlength{\parindent}{0cm}
    \nopagebreak
    \leftskip=#2cm
    \rightskip=#3cm
}
{
    \par
}
\fi

\doendnotes{C}
\bigskip
\vfill

\clearpage

\footnotesize

\ifkorrekturansicht
  \lohead{\textsc{register}}
\fi

% theindex-Environment neu definieren ohne reledmac
\makeatletter
\renewenvironment{theindex}{%
  \ifkorrekturansicht
    \section*{\indexname}%
  \else
    \subsubsection*{Index der erwähnten Entitäten}%
  \fi
  \setlength{\parindent}{0pt}%
  \setlength{\parskip}{0pt plus 0.3pt}%
  \let\item\@idxitem
}{%
  \ifkorrekturansicht\clearpage\fi
}
\makeatother

\IfFileExists{\jobname-pw.ind}{\input{\jobname-pw.ind}}{}

% Quellenangabe nur in der Leseansicht
\ifkorrekturansicht\else
% Fallback-Definitionen, falls die .tex-Datei \titel etc. nicht gesetzt hat
\providecommand{\titel}{}
\providecommand{\editorInnen}{}
\providecommand{\dateiname}{\jobname}

\vspace{3cm}

\vfill

\footnotesize
\textsc{Quelle}: \titel. Herausgegeben von {\editorInnen}. In: \emph{Arthur Schnitzler: Briefwechsel mit Autorinnen und Autoren}.
 Digitale Edition, https://schnitzler-briefe.acdh.oeaw.ac.at/{\dateiname}.html (Stand \today)
\fi

\end{document}


