%% latex-leseansicht-vorspann.tex
%% Vorspann für die Leseansicht.
%% Lädt die gemeinsame Datei latex-vorspann.tex mit nicht gesetztem Schalter.

\newif\ifkorrekturansicht
\korrekturansichtfalse

\input{../tex-inputs/latex-vorspann}


\section[Arthur Schnitzler an Gustav Schwarzkopf, 29. 9. 1899]{L04093 Arthur Schnitzler an Gustav Schwarzkopf, 29. 9. 1899}
\nopagebreak\mylabel{L04093v}
\rehead{ }\normalsize\beginnumbering\briefempfaengerindex{Schwarzkopf, Gustav@\textsc{Schwarzkopf, Gustav}!zzzSchnitzler, Arthur@\emph{von Arthur Schnitzler}!1899-09-293@{29. 9. 1899}|(be}
\toendnotes[C]{\smallbreak\pagebreak[2]}
\correspDesc{Versand  durch Arthur Schnitzler am 29. 9. 1899 in Wiesbaden
\newline{}Erhalt  durch Gustav Schwarzkopf im Zeitraum [30. 9. 1899 – 4. 10. 1899?] in Wien}\toendnotes[C]{\smallbreak}
\Standort{CUL, Schnitzler, B 96.}
\physDesc{Brief, 2 Blätter, 7 Seiten, 4268 Zeichen
\newline{}Handschrift: Bleistift, deutsche Kurrent}
\buchAbdrucke{\weitereDrucke{Arthur Schnitzler: \emph{Briefe 1875–1912}. Herausgegeben von Therese Nickl und Heinrich Schnitzler. Frankfurt am Main: \emph{S. Fischer} 1981, S. 378–380.} }\toendnotes[C]{\smallbreak}
\pstart
           \raggedleft{}{\pb}\textsc{Wiesbaden\oindex{Wiesbaden@\textbf{Wiesbaden}|pw}}, 29. 9. 99.\pend
           
\pstart
           \raggedleft{}Wiener Café\oindex{Wiener Café@\textbf{Wiener Café}, \emph{Kaffeehaus}|pw}, 10 Uhr Abds\pend
           \vspace{0.5em}
\pstart
           lieber Guſtav, nun ſitz ich hier, in Wiesbdn\oindex{Wiesbaden@\textbf{Wiesbaden}|pw} nicht im Wiener Café seit So{\geminationn}tag. Gerade in Wiesbaden\oindex{Wiesbaden@\textbf{Wiesbaden}|pw}, wieſo? Ich
               bin ins Rollen gekommen, von München\oindex{München@\textbf{München}|pw} aus. Dort
               habe ich, in den Tagen, wo die \label{K_L04093-1v}\edtext{Brücken\oindex{Max-Joseph-Brücke [III]@\textbf{Max-Joseph-Brücke [III]}, \emph{Brücke}|pwv}\oindex{Luitpoldbrücke [I]@\textbf{Luitpoldbrücke [I]}, \emph{Brücke}|pwv}
                  einſtürzten}{\lemma{\textnormal{\emph{Brücken
                  einstürzten}}}\Cendnote{\textnormal{Im Zuge eines Hochwassers
                  stürzte am 13. 9. 1899 die Max-Joseph-Brücke\oindex{Max-Joseph-Brücke [III]@\textbf{Max-Joseph-Brücke [III]}, \emph{Brücke}|pwk} und am Folgetag die Luitpoldbrücke\oindex{Luitpoldbrücke [I]@\textbf{Luitpoldbrücke [I]}, \emph{Brücke}|pwk} ein. }}}\label{K_L04093-1}, verabredetermaßen, M. E.\pwindex{Elsinger, Marie *~28.\,2.\,1874 St. Pölten@\textsc{Elsinger, Marie} (*~28.\,2.\,1874 St. Pölten), \emph{Schauspielerin}|pw} getroffen; beide kamen wir grad an dem Tag an, bevor wir
               beide, ich von Iſchl\oindex{Bad Ischl@\textbf{Bad Ischl}|pw} aus,{ }ſie von Partenkirchen\oindex{Partenkirchen@\textbf{Partenkirchen}, \emph{Teil eines besiedelten Ortes}|pw} aus von der übrigen Welt
               abgeſchnitten geweſen wären. Er folgten einige recht hübſcher Tage wollte ich
               pikant ſein, ſo könnte ich hinzuſetzen »und Nächte«. Aber ich will nicht pikant ſein
               u auch nicht reno{\geminationm}iren – alſo ſagen wir »Abende«. Sie
               hatte überraſchend viele lichte Momente. Manchmal allerdings war ſie vollko{\geminationm}en irrſinnnig. Beiſpielsweiſe: ich fragte ſie nach
               ihrem Alter. Etwas unſicher antwortete ſie: 23. – Ich erinnerte mich {\pb}dass sie in Wien\oindex{Wien@\textbf{Wien}, \emph{Verwaltungsgebiet}|pw} 26 war – und ſagte ihrs. Sie darauf vollko{\geminationm}en ernſt, »Ich weiſs nicht ſo genau, – ich bin aber{ }ſo
                  70, 72, 74 geboren, ſo um die Jahreszeit –«
               Ich: !!!!!\pend
           
\pstart
           Sie; wieder ganz ernſt. Ja, wir Schweſtern\pwindex{Elsinger, Marie *~28.\,2.\,1874 St. Pölten@\textsc{Elsinger, Marie} (*~28.\,2.\,1874 St. Pölten), \emph{Schauspielerin}|pwv}\pwindex{Elsinger, Milli @\textsc{Elsinger, Milli}, \emph{Schauspielerin}|pwv}\pwindex{Elsinger, Jenny @\textsc{Elsinger, Jenny}, \emph{Schauspielerin}|pwv}{ }ſind alle
               ungefähr in den Jahren geboren. –\pend
           
\pstart
           (Nicht styliſirt, bitte, abſolut wörtlich, ich habs mir gleich extra für Sie in
               gemerkt)\pend
           
\pstart
           – Nachdem ich in München\oindex{München@\textbf{München}|pw} incognito meine 3 Einakter\pwindex{Schnitzler, Arthur 15. 5. 1862 Wien – 21. 10. 1931 ebd.@\textsc{Schnitzler, Arthur} (15. 5. 1862 Wien – 21. 10. 1931 ebd.), \emph{Schriftsteller, Mediziner}!grüne Kakadu – Paracelsus – Die Gefährtin. Drei Einakter@\strich\emph{Der grüne Kakadu – Paracelsus – Die Gefährtin. Drei Einakter}|pwv}\eventindex{Residenztheater München@\textbf{Residenztheater München}!Aufführung von Paracelsus, Die Gefährtin, Der grüne Kakadu, 14.9.1899@Aufführung von Paracelsus, Die Gefährtin, Der grüne Kakadu, 14.9.1899|pw} (gar nicht übel) im Reſidtheater\oindex{Residenztheater München@\textbf{Residenztheater München}, \emph{Theater}|pw} und das
                  »Vermächtnis\pwindex{Schnitzler, Arthur 15. 5. 1862 Wien – 21. 10. 1931 ebd.@\textsc{Schnitzler, Arthur} (15. 5. 1862 Wien – 21. 10. 1931 ebd.), \emph{Schriftsteller, Mediziner}!Vermächtnis. Schauspiel in drei Akten@\strich\emph{Das Vermächtnis. Schauspiel in drei Akten}|pw}\eventindex{Münchner Schauspielhaus@\textbf{Münchner Schauspielhaus}!Premiere von Das Vermächtnis, 15.9.1899@Premiere von Das Vermächtnis, 15.9.1899|pwv}« noch incogniter auf der \strikeout{\textcolor{gray}{2}} Gallerie \introOben{}Schauſpielhau\textcolor{gray}{s}\oindex{Münchner Schauspielhaus@\textbf{Münchner Schauspielhaus}, \emph{Theater}|pw}\introOben{} (ſah wenig u hörte nahezu nichts) genoſſen (ich möchte das Vermächtnis\pwindex{Schnitzler, Arthur 15. 5. 1862 Wien – 21. 10. 1931 ebd.@\textsc{Schnitzler, Arthur} (15. 5. 1862 Wien – 21. 10. 1931 ebd.), \emph{Schriftsteller, Mediziner}!Vermächtnis. Schauspiel in drei Akten@\strich\emph{Das Vermächtnis. Schauspiel in drei Akten}|pw} durchaus noch einmal ſchreiben, – es könnte
               ein ſo ſchönes Stück ſein u. ist ſo ſcheußlich {\pb}(abgeſehen von \textsc{etc. etc.}) – fuhren wir nach Nürnberg\oindex{Nürnberg@\textbf{Nürnberg}|pw}; ſie von dort nach Berlin\oindex{Berlin@\textbf{Berlin}, \emph{Hauptstadt}|pw} (um zu
                  ihre\textcolor{gray}{n}{ }Blumenthal\pwindex{Blumenthal, Oskar 13.\,3.\,1852 Berlin – 24.\,4.\,1917 ebd.@\textsc{Blumenthal, Oskar} (13.\,3.\,1852 Berlin – 24.\,4.\,1917 ebd.), \emph{Schriftsteller, Journalist, Theaterleiter}|pw}{ }Kadelburg\pwindex{Kadelburg, Gustav 26.\,7.\,1851 Budapest – 11.\,9.\,1925 Berlin@\textsc{Kadelburg, Gustav} (26.\,7.\,1851 Budapest – 11.\,9.\,1925 Berlin), \emph{Schriftsteller, Schauspieler}|pw} Proben einzutreffen) ich,{ }ſchon in
               der Nähe (auf der Reiſe iſt alles nah) nach Frankfurt\oindex{Frankfurt am Main@\textbf{Frankfurt am Main}, \emph{Hauptstadt}|pw}, wo ich ein paar Tage mit Paul
                  Goldm.\pwindex{Goldmann, Paul 31.\,1.\,1865 Breslau – 25.\,9.\,1935 Wien@\textsc{Goldmann, Paul} (31.\,1.\,1865 Breslau – 25.\,9.\,1935 Wien), \emph{Schriftsteller, Journalist}|pw} verbrachte. Ich fand ihn im ganzen beſſer geſti{\geminationm}t als je (\label{K_L04093-2v}\edtext{\begin{otherlanguage}{french}\textsc{chercher la femme\pwindex{Rottenberg, Theodore 7.\,9.\,1875 – 5.\,4.\,1945 Limburg an der Lahn@\textsc{Rottenberg, Theodore} (7.\,9.\,1875 – 5.\,4.\,1945 Limburg an der Lahn)|pwv} – d’un autre\pwindex{Rottenberg, Ludwig 11.\,10.\,1864 Czernowitz – 6.\,5.\,1932 Frankfurt am Main@\textsc{Rottenberg, Ludwig} (11.\,10.\,1864 Czernowitz – 6.\,5.\,1932 Frankfurt am Main), \emph{Kapellmeister}|pwv}}\end{otherlanguage}}{\lemma{\textnormal{\emph{chercher … autre}}}\Cendnote{\textnormal{französisch: man suche die Frau – eines
                  Anderen (die dahintersteckt)}}}\label{K_L04093-2}); er reiſt nach Florenz\oindex{Luzern@\textbf{Luzern}|pw}, ich, um einige Zeit ganz allein, in Ruh, zu arbeiten, hieher. Hier,
               im Park-hotel\oindex{Hôtel du Parc {\kaufmannsund} Bristol@\textbf{Hôtel du Parc {\kaufmannsund} Bristol}, \emph{Hotel}|pw}, angenehmes Zi{\geminationm}er, vortreffliches Eſſen, bin ich nun daran, den Schleier der \textsc{Beatrice}\pwindex{Schnitzler, Arthur 15. 5. 1862 Wien – 21. 10. 1931 ebd.@\textsc{Schnitzler, Arthur} (15. 5. 1862 Wien – 21. 10. 1931 ebd.), \emph{Schriftsteller, Mediziner}!Schleier der Beatrice. Schauspiel in fünf Akten@\strich\emph{Der Schleier der Beatrice. Schauspiel in fünf Akten}|pw} wieder einmal, und wieder nur »vorläufig« abzuſchließen. Ich denk’ es ſo weit
               zu bringen, daſs ichs dem Brahm\pwindex{Brahm, Otto 5.\,2.\,1856 Hamburg – 28.\,11.\,1912 Berlin@\textsc{Brahm, Otto} (5.\,2.\,1856 Hamburg – 28.\,11.\,1912 Berlin), \emph{Theaterleiter, Regisseur}|pw} in Berlin\oindex{Berlin@\textbf{Berlin}, \emph{Hauptstadt}|pw} vorleſen\eventindex{Luisenplatz 2@\textbf{Luisenplatz 2}!Private Lesung von Der Schleier der Beatrice, 7.10.1899@Private Lesung von Der Schleier der Beatrice, 7.10.1899|pwv} kann. In der Anlage scheint’s mir {\pb}nicht ohne Kühnheit; aber in den
               wichtigen Momenten verſagt – doch ich will Ihrem Urtheil, wenigſtens dem tadelnden
               nicht vorgreifen. – Wie ich ſonſt die Zeit verbringe? Das Wetter war{ }ſchlecht; heut
               erſt ko{\geminationn}t ich ein biſchen radeln. In den erſten Morgen-
               und in manchen Abendſtunden ko{\geminationm}t eine drückende
               Traurigkeit über mich; und an gewiſſe Örtlichkeiten, Straßen, Plätze in Wien\oindex{Wien@\textbf{Wien}, \emph{Verwaltungsgebiet}|pw} denk’ ich mit Angſt; ja wie an etwas
               geſpenſtiſches. Geleſen hab ich die \textsc{Madame Bovary\pwindex{Flaubert, Gustave 12.\,12.\,1821 Rouen – 8.\,5.\,1880 Canteleu@\textsc{Flaubert, Gustave} (12.\,12.\,1821 Rouen – 8.\,5.\,1880 Canteleu), \emph{Schriftsteller}!Madame Bovary. Mœurs de province@\strich\emph{Madame Bovary. Mœurs de province}|pw}}; jetzt \textsc{Orme du Mail\pwindex{France, Anatole 16.\,4.\,1844 Paris – 12.\,10.\,1924 Saint-Cyr-sur-Loire@\textsc{France, Anatole} (16.\,4.\,1844 Paris – 12.\,10.\,1924 Saint-Cyr-sur-Loire), \emph{Schriftsteller}!Orme du mail@\strich\emph{L’Orme du mail}|pw}} von \textsc{France\pwindex{France, Anatole 16.\,4.\,1844 Paris – 12.\,10.\,1924 Saint-Cyr-sur-Loire@\textsc{France, Anatole} (16.\,4.\,1844 Paris – 12.\,10.\,1924 Saint-Cyr-sur-Loire), \emph{Schriftsteller}|pw}}, frei, ſehr frei, aber ein wenig langweilig. – Im Theater: geſtern
               bei \textsc{Carmen\pwindex{Meilhac, Henri 21.\,2.\,1831 Paris – 6.\,6.\,1897 ebd.@\textsc{Meilhac, Henri} (21.\,2.\,1831 Paris – 6.\,6.\,1897 ebd.), \emph{Schriftsteller}!Carmen. Opéra-Comique en quatre actes@\strich\emph{Carmen. Opéra-Comique en quatre actes}|pw}\pwindex{Halévy, Ludovic 1.\,1.\,1834 Paris – 7.\,5.\,1908 ebd.@\textsc{Halévy, Ludovic} (1.\,1.\,1834 Paris – 7.\,5.\,1908 ebd.), \emph{Schriftsteller}!Carmen. Opéra-Comique en quatre actes@\strich\emph{Carmen. Opéra-Comique en quatre actes}|pw}}\eventindex{Hessisches Staatstheater Wiesbaden@\textbf{Hessisches Staatstheater Wiesbaden}!Aufführung von Carmen, 28.9.1899@Aufführung von Carmen, 28.9.1899|pwv}, neulich, neulich bei \textsc{Undine\pwindex{Lortzing, Albert 23.\,10.\,1801 Berlin – 21.\,1.\,1851 ebd.@\textsc{Lortzing, Albert} (23.\,10.\,1801 Berlin – 21.\,1.\,1851 ebd.), \emph{Komponist}!Undine. Romantische Zauberoper in vier Akten@\strich\emph{Undine. Romantische Zauberoper in vier Akten}|pw}}\eventindex{Hessisches Staatstheater Wiesbaden@\textbf{Hessisches Staatstheater Wiesbaden}!Aufführung von Undine, 24.9.1899@Aufführung von Undine, 24.9.1899|pwv}. Sehr ſchön, das Hoftheater\oindex{Hessisches Staatstheater Wiesbaden@\textbf{Hessisches Staatstheater Wiesbaden}, \emph{Theater}|pw}.\pend
           
\pstart
           {\pb}Nebſtbei brauch ich eine Art Cur;
               trinke »Kochbrunnen\oindex{Kochbrunnen@\textbf{Kochbrunnen}, \emph{Brunnen}|pw}«, was beka{\geminationn}tlich alle Leiden beſeitigt. (Warum man nur an alle
               Curplätze i{\geminationm}er wieder kommt?) – Ich habe die Abſicht
                  \label{K_L04093-3v}\edtext{Dinſtg von hier nach Berlin\oindex{Berlin@\textbf{Berlin}, \emph{Hauptstadt}|pw}}{\lemma{\textnormal{\emph{Dinstg … Berlin}}}\Cendnote{\textnormal{Siehe A. S.: \emph{Wiener Schnitzler}, 3. 10. 1899.}}}\label{K_L04093-3} zu
               reiſen, bleibe dort (vielleicht) 8 Tage, ſo dſs ich um den 10. October
               herum nach Wien\oindex{Wien@\textbf{Wien}, \emph{Verwaltungsgebiet}|pw} komme. Entſchuldigen Sie: ich
               freu’ mich beinah nur auf Sie. – \label{K_L04093-4v}\edtext{\textsc{Polna}\oindex{Polná@\textbf{Polná}|pw} regt mich \substVorne{}\textsuperscript{\textcolor{gray}{×}\-\textcolor{gray}{×}\-\textcolor{gray}{×}\-\textcolor{gray}{×}\-\textcolor{gray}{×}}\substDazwischen{}ſelbſt in\substHinten{} dieſer Entfernung auf}{\lemma{\textnormal{\emph{Polna … auf}}}\Cendnote{\textnormal{Am 29. 3. 1899 wurde in Polna\oindex{Polná@\textbf{Polná}|pwk} (heute Polnà\oindex{Polná@\textbf{Polná}|pwk})
                  die neunzehnjährige 
                  Agnes Hruza\pwindex{Hrůza, Anežka 16.\,4.\,1879 Věžnička – 29.\,3.\,1899 Polná@\textsc{Hrůza, Anežka} (16.\,4.\,1879 Věžnička – 29.\,3.\,1899 Polná)|pwk} (Anežka Hrůza\pwindex{Hrůza, Anežka 16.\,4.\,1879 Věžnička – 29.\,3.\,1899 Polná@\textsc{Hrůza, Anežka} (16.\,4.\,1879 Věžnička – 29.\,3.\,1899 Polná)|pwk}) getötet.
                  Ab dem 11. 9. 1899 fand der Geschworenenprozess gegen den jüdischen Schuster Leopold
                     Hilsner\pwindex{Hilsner, Leopold 10.\,7.\,1876 Polná – 9.\,1.\,1928 Wien@\textsc{Hilsner, Leopold} (10.\,7.\,1876 Polná – 9.\,1.\,1928 Wien), \emph{Schuster}|pwk} statt, der eine geistige Behinderung hatte und in Folge zum Tod verurteilt wurde. Hilsner\pwindex{Hilsner, Leopold 10.\,7.\,1876 Polná – 9.\,1.\,1928 Wien@\textsc{Hilsner, Leopold} (10.\,7.\,1876 Polná – 9.\,1.\,1928 Wien), \emph{Schuster}|pwk}
                  entging zwar in der Revision der Todesstrafe, blieb aber bis 1918 unschuldig in Haft.
                  Von rechten und antisemitischen 
                  Politikern wurde der an Ostern geschehene Mord als »Ritualmord« dargestellt und politisch ausgeschlachtet.}}}\label{K_L04093-4}; man{ }ſpürt in Deutſchland\oindex{Deutschland@\textbf{Deutschland}|pw} doch immer, dſs Oeſterreich das vertrottelſte Land der Welt ist.
               Die letzte \label{K_L04093-5v}\edtext{Nummer\pwindex{Fackel@\emph{Die Fackel}|pwv} vom kleinen Kraus\pwindex{Kraus, Karl 28.\,4.\,1874 Jičín – 12.\,6.\,1936 Wien@\textsc{Kraus, Karl} (28.\,4.\,1874 Jičín – 12.\,6.\,1936 Wien), \emph{Schriftsteller, Publizist, Schriftsteller}|pw}}{\lemma{\textnormal{\emph{Nummer vom kleinen Kraus}}}\Cendnote{\textnormal{Er dürfte sich auf 
                     die Nr. 17 (»Mitte September«) beziehen. Schnitzler selbst ist
                     darin nicht genannt.Kraus\pwindex{Kraus, Karl 28.\,4.\,1874 Jičín – 12.\,6.\,1936 Wien@\textsc{Kraus, Karl} (28.\,4.\,1874 Jičín – 12.\,6.\,1936 Wien), \emph{Schriftsteller, Publizist, Schriftsteller}|pwk}}}}\label{K_L04093-5} hab ich geſehen; ich ſag  Ihnen, der Kerl
               iſt und bleibt, we{\geminationn} er auch viele gute Einfälle u in
               manchem Recht hat, nun niederträchtiger, nein, ein {\pb}modriger Kerl. – Dann dieſe \label{K_L04093-6v}\edtext{Verſa{\geminationm}lg\eventindex{Musikverein@\textbf{Musikverein}!Antisemitische Versammlung im Musikvereinssaal, 27.9.1899@Antisemitische Versammlung im Musikvereinssaal, 27.9.1899|pw} im Muſikvereinsſaal\oindex{Wien@\textbf{Wien}!I., Innere Stadt@\textbf{I., Innere Stadt}!Musikverein@\textbf{Musikverein}, \emph{Konzertsaal}|pw}}{\lemma{\textnormal{\emph{Versammlg im Musikvereinssaal}}}\Cendnote{\textnormal{Am 27. 9. 1899 veranstaltete
                           der \emph{Volkswirtschaftliche Verein}\orgindex{Volkswirtschaftlicher Verein@Volkswirtschaftlicher Verein|pwk} eine antisemitische Kundgebung im 
                           Musikvereinssaal\oindex{Wien@\textbf{Wien}!I., Innere Stadt@\textbf{I., Innere Stadt}!Musikverein@\textbf{Musikverein}, \emph{Konzertsaal}|pwk}, bei der führende Antisemiten wie Hermann Bielohlawek\pwindex{Bielohlawek, Hermann 2.\,8.\,1861 Wien – 30.\,6.\,1918 ebd.@\textsc{Bielohlawek, Hermann} (2.\,8.\,1861 Wien – 30.\,6.\,1918 ebd.), \emph{Politiker}|pwk}, Richard Weiskirchner\pwindex{Weiskirchner, Richard 24.\,3.\,1861 Wien – 30.\,4.\,1926 ebd.@\textsc{Weiskirchner, Richard} (24.\,3.\,1861 Wien – 30.\,4.\,1926 ebd.), \emph{Politiker}|pwk} und
                           Ernst Vergani\pwindex{Vergani, Ernst 15.\,3.\,1848 Stebnik – 19.\,2.\,1915 Emmersdorf an der Donau@\textsc{Vergani, Ernst} (15.\,3.\,1848 Stebnik – 19.\,2.\,1915 Emmersdorf an der Donau), \emph{Herausgeber}|pwk} auftraten und unter anderem zum »Ritualmord« von 
                           Polna\oindex{Polná@\textbf{Polná}|pwk} agitierten. Die Veranstaltung war überbucht,
                        dass Leute mit Eintrittskarten abgewiesen werden mussten.}}}\label{K_L04093-6}! So etwas iſt möglich!!– Man merkt, ich bin lange von
                  Wien\oindex{Wien@\textbf{Wien}, \emph{Verwaltungsgebiet}|pw} fort. – Was{ }ſagen Sie (nun kommt das
               wichtigſte) zu dem »\label{K_L04093-7v}\edtext{Verſchwinden}{\lemma{\textnormal{\emph{Verschwinden}}}\Cendnote{\textnormal{\emph{Der grüne Kakadu}\pwindex{Schnitzler, Arthur 15. 5. 1862 Wien – 21. 10. 1931 ebd.@\textsc{Schnitzler, Arthur} (15. 5. 1862 Wien – 21. 10. 1931 ebd.), \emph{Schriftsteller, Mediziner}!grüne Kakadu. Groteske in einem Akt@\strich\emph{Der grüne Kakadu. Groteske in einem Akt}|pwk} wurde nach nur sechs
                     Aufführungen vom Spielplan\orgindex{Burgtheater@Burgtheater|pwkv}
                     genommen. Am 26. 10. 1899 war Direktor Paul
                        Schlenther\pwindex{Schlenther, Paul 20.\,8.\,1854 Chernyakhovsk – 30.\,4.\,1916 Berlin@\textsc{Schlenther, Paul} (20.\,8.\,1854 Chernyakhovsk – 30.\,4.\,1916 Berlin), \emph{Schriftsteller, Kritiker, Theaterleiter}|pwk} bei Schnitzler zu Hause
                     und teilte ihm mit, dass die \emph{Zensurbehörde}\orgindex{K. u. k. Zensurstelle@K. u. k. Zensurstelle|pwk} die
                     weitere Aufführung verbiete, ohne das aber mit einem schriftlichen Urteil zu
                     bestätigen, worüber sich Schnitzler
                     zusätzlich ärgerte. Erst Jahre später, am 4. 12. 1905, erfuhr Schnitzler den eigentlichen Grund: »Erzh. Gisela\pwindex{Gisela von Österreich 12.\,7.\,1856 Laxenburg – 27.\,7.\,1932 München@\textsc{Gisela von Österreich} (12.\,7.\,1856 Laxenburg – 27.\,7.\,1932 München), \emph{Erzherzogin}|pw} war drin und indignirt, weil Haeberle\pwindex{Speidel-Haeberle, Else 11.\,7.\,1877 Stuttgart – 21.\,7.\,1937 Augustenfeld@\textsc{Speidel-Haeberle, Else} (11.\,7.\,1877 Stuttgart – 21.\,7.\,1937 Augustenfeld), \emph{Schauspielerin}|pw} (Michette) sich an den Dessous
                        der Marquise (Mitterwurzer\pwindex{Mitterwurzer, Wilhelmine 27.\,3.\,1848 Freiburg im Breisgau – 3.\,8.\,1909 Wien@\textsc{Mitterwurzer, Wilhelmine} (27.\,3.\,1848 Freiburg im Breisgau – 3.\,8.\,1909 Wien), \emph{Schauspielerin}|pw}) zu schaffen
                        machte. –« }}}\label{K_L04093-7}« meiner 3 Einakter\pwindex{Schnitzler, Arthur 15. 5. 1862 Wien – 21. 10. 1931 ebd.@\textsc{Schnitzler, Arthur} (15. 5. 1862 Wien – 21. 10. 1931 ebd.), \emph{Schriftsteller, Mediziner}!grüne Kakadu – Paracelsus – Die Gefährtin. Drei Einakter@\strich\emph{Der grüne Kakadu – Paracelsus – Die Gefährtin. Drei Einakter}|pwv}. – Ich habe \label{K_L04093-8v}\edtext{\textsc{Schlenther\pwindex{Schlenther, Paul 20.\,8.\,1854 Chernyakhovsk – 30.\,4.\,1916 Berlin@\textsc{Schlenther, Paul} (20.\,8.\,1854 Chernyakhovsk – 30.\,4.\,1916 Berlin), \emph{Schriftsteller, Kritiker, Theaterleiter}|pw}} geſchrieben}{\lemma{\textnormal{\emph{Schlenther geschrieben}}}\Cendnote{\textnormal{Schnitzler schrieb aus Wiesbaden\oindex{Wiesbaden@\textbf{Wiesbaden}|pwk} an Schlenther\pwindex{Schlenther, Paul 20.\,8.\,1854 Chernyakhovsk – 30.\,4.\,1916 Berlin@\textsc{Schlenther, Paul} (20.\,8.\,1854 Chernyakhovsk – 30.\,4.\,1916 Berlin), \emph{Schriftsteller, Kritiker, Theaterleiter}|pwk},
                  die Antwort erhielt er von Richard Rosenbaum\pwindex{Rosenbaum, Richard 4.\,11.\,1867 Žikov – 25.\,6.\,1942 Konzentrationslager Theresienstadt@\textsc{Rosenbaum, Richard} (4.\,11.\,1867 Žikov – 25.\,6.\,1942 Konzentrationslager Theresienstadt), \emph{Dramaturg, Verleger}|pwk}, datiert vom 30. 9. 1899: »\textcolor{gray}{\textbf{K. u. K.
                        \so{DIRECTION} DES K. K. HOFBURGTHEATERS\orgindex{Burgtheater@Burgtheater|pw}.}}{ / }Wien\oindex{Wien@\textbf{Wien}, \emph{Verwaltungsgebiet}|pw}, 30/9 1899.{ / }Verehrter Herr Doctor,{ / }Ich bin beauftragt, Ihnen in Erwidrung auf
                           Ihre freundliche Zuschrift aus Wiesbaden\oindex{Wiesbaden@\textbf{Wiesbaden}|pw} von
                        26. d. M. mitzuteilen, dass es leider nicht möglich
                        war, Ihre Einakter\pwindex{Schnitzler, Arthur 15. 5. 1862 Wien – 21. 10. 1931 ebd.@\textsc{Schnitzler, Arthur} (15. 5. 1862 Wien – 21. 10. 1931 ebd.), \emph{Schriftsteller, Mediziner}!grüne Kakadu – Paracelsus – Die Gefährtin. Drei Einakter@\strich\emph{Der grüne Kakadu – Paracelsus – Die Gefährtin. Drei Einakter}|pwv} schon auf das Repertoire
                        der nächsten Woche zu setzen. Aber in der darauffolgenden Woche soll es bestimmt geschehen.{ / }Mit den herzlichsten Grüßen und in der Hoffnung,
                        Sie in Wien\oindex{Wien@\textbf{Wien}, \emph{Verwaltungsgebiet}|pw} bald begrüßen zu können, bin ich Ihr
                        hochachtungsvoll ergebener{ / }Richard Rosenbaum« In Folge setzte Schlenther\pwindex{Schlenther, Paul 20.\,8.\,1854 Chernyakhovsk – 30.\,4.\,1916 Berlin@\textsc{Schlenther, Paul} (20.\,8.\,1854 Chernyakhovsk – 30.\,4.\,1916 Berlin), \emph{Schriftsteller, Kritiker, Theaterleiter}|pwk} aber nur
                  mehr zwei\pwindex{Schnitzler, Arthur 15. 5. 1862 Wien – 21. 10. 1931 ebd.@\textsc{Schnitzler, Arthur} (15. 5. 1862 Wien – 21. 10. 1931 ebd.), \emph{Schriftsteller, Mediziner}!Gefährtin. Schauspiel in einem Akt@\strich\emph{Die Gefährtin. Schauspiel in einem Akt}|pwkv}\pwindex{Schnitzler, Arthur 15. 5. 1862 Wien – 21. 10. 1931 ebd.@\textsc{Schnitzler, Arthur} (15. 5. 1862 Wien – 21. 10. 1931 ebd.), \emph{Schriftsteller, Mediziner}!Paracelsus. Versspiel in einem Akt@\strich\emph{Paracelsus. Versspiel in einem Akt}|pwkv} der drei Einakter\pwindex{Schnitzler, Arthur 15. 5. 1862 Wien – 21. 10. 1931 ebd.@\textsc{Schnitzler, Arthur} (15. 5. 1862 Wien – 21. 10. 1931 ebd.), \emph{Schriftsteller, Mediziner}!grüne Kakadu – Paracelsus – Die Gefährtin. Drei Einakter@\strich\emph{Der grüne Kakadu – Paracelsus – Die Gefährtin. Drei Einakter}|pwkv} auf den Spielplan,  \emph{Der grüne Kakadu}\pwindex{Schnitzler, Arthur 15. 5. 1862 Wien – 21. 10. 1931 ebd.@\textsc{Schnitzler, Arthur} (15. 5. 1862 Wien – 21. 10. 1931 ebd.), \emph{Schriftsteller, Mediziner}!grüne Kakadu. Groteske in einem Akt@\strich\emph{Der grüne Kakadu. Groteske in einem Akt}|pwk} blieb abgesetzt.}}}\label{K_L04093-8}, ob jenes Anſinnen wiederholt wurde? – Sehen Sie unſre Bekannten?
               Ich hör gar nichts von ihnen, von \textsc{Wassermann\pwindex{Wassermann, Jakob 10.\,3.\,1873 Fürth – 1.\,1.\,1934 Altaussee@\textsc{Wassermann, Jakob} (10.\,3.\,1873 Fürth – 1.\,1.\,1934 Altaussee), \emph{Schriftsteller}|pw}}, \textsc{Salten\pwindex{Salten, Felix 6.\,9.\,1869 Budapest – 8.\,10.\,1945 Zürich@\textsc{Salten, Felix} (6.\,9.\,1869 Budapest – 8.\,10.\,1945 Zürich), \emph{Schriftsteller, Journalist, Chefredakteur}|pw}}; etc. – \label{K_L04093-9v}\edtext{\textsc{Hugo\pwindex{Hofmannsthal, Hugo von 1.\,2.\,1874 Wien – 15.\,7.\,1929 Rodaun@\textsc{Hofmannsthal, Hugo von} (1.\,2.\,1874 Wien – 15.\,7.\,1929 Rodaun), \emph{Schriftsteller}|pw}} ist nach Venedig\oindex{Venedig@\textbf{Venedig}|pw}, Richard\pwindex{Beer-Hofmann, Richard 11.\,7.\,1866 Wien – 26.\,9.\,1945 New York City@\textsc{Beer-Hofmann, Richard} (11.\,7.\,1866 Wien – 26.\,9.\,1945 New York City), \emph{Schriftsteller}|pw} nach einem unbegreiflichen\textsc{St. Michael in Eppan\oindex{Sankt Michael@\textbf{Sankt Michael}, \emph{Bezirk}|pw}}}{\lemma{\textnormal{\emph{Hugo … Eppan}}}\Cendnote{\textnormal{Vgl. XXXX Auszeichnungsfehler: Dokument L00981 nicht gefunden.}}}\label{K_L04093-9} gereiſt. –\pend
           
\pstart
           Wenn Sie einmal eine überflüſſige Viertelſtunde haben (»Gott, hab ich denn
               andre?!«), ſo ſchaun Sie vielleicht zu meiner Mama\pwindex{Schnitzler, Louise 8.\,7.\,1840 Kőszeg – 9.\,9.\,1911 Wien@\textsc{Schnitzler, Louise} (8.\,7.\,1840 Kőszeg – 9.\,9.\,1911 Wien)|pwv} hinauf. {\pb}Sie würde ſich ſehr freuen. – Und nun leben Sie wohl. Ein paar Worte ins \textsc{Hotel Savoy, Berlin\oindex{Hotel Savoy [Berlin]@\textbf{Hotel Savoy [Berlin]}, \emph{Hotel}|pw}}, wären mir, wie Sie ſich denken können, mehr als willkommen.\pend
           
\pstart
           Von Herzen Ihr{\\[\baselineskip]}\spacefill\mbox{Arthur}\pend
           \leftskip=0em{}
\pstart
           \noindent{}»Wiener Café\oindex{Wiener Café@\textbf{Wiener Café}, \emph{Kaffeehaus}|pw}« - alſo lauter Juden und neben
                  mir wird »\label{K_L04093-10v}\edtext{geklabbert}{\lemma{\textnormal{\emph{geklabbert}}}\Cendnote{\textnormal{Klabrias: Kartenspiel, das spätestens
                     durch den Erfolg des Stücks \emph{Eine Partie
                        Klabrias im Café Spitzer}\pwindex{Bergmann, Adolf 1850 Budapest – 28.\,4.\,1903 Wien@\textsc{Bergmann, Adolf} (1850 Budapest – 28.\,4.\,1903 Wien), \emph{Schauspieler}!Eine Partie Klabrias im Café Spitzer@\strich\emph{Eine Partie Klabrias im Café Spitzer}|pwk} vor allem mit Juden in Verbindung gebracht
                     wurde.}}}\label{K_L04093-10}«. Der kleine \textsc{Kraus\pwindex{Kraus, Karl 28.\,4.\,1874 Jičín – 12.\,6.\,1936 Wien@\textsc{Kraus, Karl} (28.\,4.\,1874 Jičín – 12.\,6.\,1936 Wien), \emph{Schriftsteller, Publizist, Schriftsteller}|pw}} und \textsc{\label{K_L04093-11v}\edtext{Gregorig\pwindex{Gregorig, Josef 27.\,4.\,1846 Bisamberg – 2.\,7.\,1909 Maria Enzersdorf@\textsc{Gregorig, Josef} (27.\,4.\,1846 Bisamberg – 2.\,7.\,1909 Maria Enzersdorf), \emph{Politiker}|pw}}{\lemma{\textnormal{\emph{Gregorig}}}\Cendnote{\textnormal{Josef Gregorig\pwindex{Gregorig, Josef 27.\,4.\,1846 Bisamberg – 2.\,7.\,1909 Maria Enzersdorf@\textsc{Gregorig, Josef} (27.\,4.\,1846 Bisamberg – 2.\,7.\,1909 Maria Enzersdorf), \emph{Politiker}|pwk} war ein
                        christlich-sozialer Politiker und offener Antisemit.}}}\label{K_L04093-11}}{ }ſtürben an Ekel. Was ich ſehr erfreulich fände. –\pend
           \selectlanguage{ngerman}\endnumbering\briefempfaengerindex{Schwarzkopf, Gustav@\textsc{Schwarzkopf, Gustav}!zzzSchnitzler, Arthur@\emph{von Arthur Schnitzler}!1899-09-293@{29. 9. 1899}|)be}\mylabel{L04093h}
\begin{anhang}
\end{anhang}\newcommand{\dateiname}{L04093}\newcommand{\titel}{Arthur Schnitzler an Gustav Schwarzkopf, 29. 9. 1899}\newcommand{\editorInnen}{Herausgegeben von Jahnke, SelmaMüller, Martin Anton}%% latex-leseansicht-abspann.tex
%% Abspann für die Leseansicht.
%% Der Schalter \ifkorrekturansicht ist bereits durch den Vorspann gesetzt.

%% latex-abspann.tex
%% Gemeinsamer Abspann für Korrekturansicht und Leseansicht.
%% Setzt den Schalter \ifkorrekturansicht voraus (gesetzt in den
%% einbindenden Dateien latex-korrekturansicht-abspann.tex bzw.
%% latex-leseansicht-abspann.tex).
%% ---------------------------------------------------------------

\normalsize

% Das esempio-Environment wird nur in der Leseansicht benötigt
\ifkorrekturansicht\else
\newenvironment{esempio}[3]%
{
    \vspace{1.5ex}
    \rlap{\underline{#1}}
    \par
    \setlength{\parindent}{0cm}
    \nopagebreak
    \leftskip=#2cm
    \rightskip=#3cm
}
{
    \par
}
\fi

\doendnotes{C}
\bigskip
\vfill

\clearpage

\footnotesize

\ifkorrekturansicht
  \lohead{\textsc{register}}
\fi

% theindex-Environment neu definieren ohne reledmac
\makeatletter
\renewenvironment{theindex}{%
  \ifkorrekturansicht
    \section*{\indexname}%
  \else
    \subsubsection*{Index der erwähnten Entitäten}%
  \fi
  \setlength{\parindent}{0pt}%
  \setlength{\parskip}{0pt plus 0.3pt}%
  \let\item\@idxitem
}{%
  \ifkorrekturansicht\clearpage\fi
}
\makeatother

\IfFileExists{\jobname-pw.ind}{\input{\jobname-pw.ind}}{}

% Quellenangabe nur in der Leseansicht
\ifkorrekturansicht\else
% Fallback-Definitionen, falls die .tex-Datei \titel etc. nicht gesetzt hat
\providecommand{\titel}{}
\providecommand{\editorInnen}{}
\providecommand{\dateiname}{\jobname}

\vspace{3cm}

\vfill

\footnotesize
\textsc{Quelle}: \titel. Herausgegeben von {\editorInnen}. In: \emph{Arthur Schnitzler: Briefwechsel mit Autorinnen und Autoren}.
 Digitale Edition, https://schnitzler-briefe.acdh.oeaw.ac.at/{\dateiname}.html (Stand \today)
\fi

\end{document}


