%% latex-leseansicht-vorspann.tex
%% Vorspann für die Leseansicht.
%% Lädt die gemeinsame Datei latex-vorspann.tex mit nicht gesetztem Schalter.

\newif\ifkorrekturansicht
\korrekturansichtfalse

\input{../tex-inputs/latex-vorspann}


\section[Berta Zuckerkandl an Arthur Schnitzler, {{[}}zwischen 11. und 31. 5. 1911?{{]}}]{L03997 Berta Zuckerkandl an Arthur Schnitzler, {[}zwischen 11. und 31. 5. 1911?{]}}
\nopagebreak\mylabel{L03997v}
\rehead{ }\normalsize\beginnumbering\briefempfaengerindex{Schnitzler, Arthur@\textsc{Schnitzler, Arthur}!zzzZuckerkandl, Berta@\emph{von Berta Zuckerkandl}!1911-05-311@{{[}zwischen 11. und 31. 5. 1911?{]}}|(be}
\toendnotes[C]{\smallbreak\pagebreak[2]}
\correspDesc{Versand  durch Berta Zuckerkandl im Zeitraum [zwischen 11. und
                  31. 5. 1911?] in Wien
\newline{}Erhalt  durch Arthur Schnitzler in Wien}\toendnotes[C]{\smallbreak}
\Standort{CUL, Schnitzler, B 200.}
\physDesc{Brief, 1 Blatt, 4 Seiten, 721 Zeichen
\newline{}Handschrift: schwarze Tinte, lateinische Kurrent
\newline{}Schnitzler: mit Bleistift beschriftet: »Zuckerkandl.« }\toendnotes[C]{\smallbreak}
\pstart{}{\pb}Hochverehrter Herr Doktor!\pend\vspace{0.5em}
\pstart
           Ich habe es\pwindex{Schnitzler, Arthur 15.\,5.\,1862 Wien – 21.\,10.\,1931 ebd.@\textsc{Schnitzler, Arthur} (15.\,5.\,1862 Wien – 21.\,10.\,1931 ebd.), \emph{Schriftsteller, Mediziner}!weite Land. Tragikomödie in fünf Akten@\strich\emph{Das weite Land. Tragikomödie in fünf Akten}|pwv} erst in einem Zug
               gelesen. Und dann langsam nachgekostet. Es ist ganz wunderbar. Wirklich \label{K_L03997-1v}\edtext{ein weites – weites Land}{\lemma{\textnormal{\emph{ein weites – weites Land}}}\Cendnote{\textnormal{Die Druckausgabe von \emph{Das weite Land}\pwindex{Schnitzler, Arthur 15.\,5.\,1862 Wien – 21.\,10.\,1931 ebd.@\textsc{Schnitzler, Arthur} (15.\,5.\,1862 Wien – 21.\,10.\,1931 ebd.), \emph{Schriftsteller, Mediziner}!weite Land. Tragikomödie in fünf Akten@\strich\emph{Das weite Land. Tragikomödie in fünf Akten}|pwk} erschien erst im September 1911, es muss sich hier um das Bühnenmanuskript handeln, das Schnitzler auch Stefan Zweig\pwindex{Zweig, Stefan 28.\,11.\,1881 Wien – 23.\,2.\,1942 Petrópolis@\textsc{Zweig, Stefan} (28.\,11.\,1881 Wien – 23.\,2.\,1942 Petrópolis), \emph{Schriftsteller}|pwk} bereits zur Verfügung gestellt hatte (vgl. XXXX Auszeichnungsfehler: Dokument L03626 nicht gefunden).}}}\label{K_L03997-1} in das wir
               durch Dichter’s Gnaden {\pb}blicken können.
               Wie viel wird lebendig in der eigenen Seele!\pend
           
\pstart
           – Ich glaube dass gerade dieses Thema – das Verhältniss zwischen Mann und Frau – in
               der so eigenthümlichen {\pb}Beleuchtung – für
                  Paris\oindex{Paris@\textbf{Paris}, \emph{Hauptstadt}|pw} wie geschaffen wäre. Erlauben Sie mir
               jedenfalls wenn ich hinreise – dort darüber unverbindlich zu sprechen. Übrigens
               erwarte ich täglich einen Brief – \label{K_L03997-2v}\edtext{wegen des Medardus\pwindex{Schnitzler, Arthur 15.\,5.\,1862 Wien – 21.\,10.\,1931 ebd.@\textsc{Schnitzler, Arthur} (15.\,5.\,1862 Wien – 21.\,10.\,1931 ebd.), \emph{Schriftsteller, Mediziner}!junge Medardus. Dramatische Historie in einem Vorspiel und fünf Aufzügen@\strich\emph{Der junge Medardus. Dramatische Historie in einem Vorspiel und fünf Aufzügen}|pw}}{\lemma{\textnormal{\emph{wegen des Medardus}}}\Cendnote{\textnormal{Im April 1911 war Berta Zuckerkandl\pwindex{Zuckerkandl, Berta 13.\,4.\,1864 Wien – 16.\,10.\,1945 Paris@\textsc{Zuckerkandl, Berta} (13.\,4.\,1864 Wien – 16.\,10.\,1945 Paris), \emph{Schriftstellerin, Journalistin, Übersetzerin}|pwk} mit dem Vorschlag an Schnitzler herangetreten, sein Schauspiel \emph{Der junge Medardus}\pwindex{Schnitzler, Arthur 15.\,5.\,1862 Wien – 21.\,10.\,1931 ebd.@\textsc{Schnitzler, Arthur} (15.\,5.\,1862 Wien – 21.\,10.\,1931 ebd.), \emph{Schriftsteller, Mediziner}!junge Medardus. Dramatische Historie in einem Vorspiel und fünf Aufzügen@\strich\emph{Der junge Medardus. Dramatische Historie in einem Vorspiel und fünf Aufzügen}|pwk} zur Aufführung in Paris\oindex{Paris@\textbf{Paris}, \emph{Hauptstadt}|pwk} vermitteln (vgl. XXXX Auszeichnungsfehler: Dokument L03998 nicht gefunden), am 25. 7. 1911 unterrichtete sie ihn, dass der
                  Versuch vorerst gescheitert war (vgl. XXXX Auszeichnungsfehler: Dokument L04009 nicht gefunden). Der vorliegende undatierte Brief, in dem Zuckerkandl\pwindex{Zuckerkandl, Berta 13.\,4.\,1864 Wien – 16.\,10.\,1945 Paris@\textsc{Zuckerkandl, Berta} (13.\,4.\,1864 Wien – 16.\,10.\,1945 Paris), \emph{Schriftstellerin, Journalistin, Übersetzerin}|pwk} angibt auf Nachrichten bezüglich des Medardus\pwindex{Schnitzler, Arthur 15.\,5.\,1862 Wien – 21.\,10.\,1931 ebd.@\textsc{Schnitzler, Arthur} (15.\,5.\,1862 Wien – 21.\,10.\,1931 ebd.), \emph{Schriftsteller, Mediziner}!junge Medardus. Dramatische Historie in einem Vorspiel und fünf Aufzügen@\strich\emph{Der junge Medardus. Dramatische Historie in einem Vorspiel und fünf Aufzügen}|pwkv} zu warten, fällt
                  also in diese Zeitspanne. Genauer: Am 18. 4. 1911 schlug Schnitzler vor, Anfang Mai über
                  die Vermittlung von \emph{Der junge Medardus}\pwindex{Schnitzler, Arthur 15.\,5.\,1862 Wien – 21.\,10.\,1931 ebd.@\textsc{Schnitzler, Arthur} (15.\,5.\,1862 Wien – 21.\,10.\,1931 ebd.), \emph{Schriftsteller, Mediziner}!junge Medardus. Dramatische Historie in einem Vorspiel und fünf Aufzügen@\strich\emph{Der junge Medardus. Dramatische Historie in einem Vorspiel und fünf Aufzügen}|pwk} aber
                  auch von seinem neuen Stück \emph{Das weite Land}\pwindex{Schnitzler, Arthur 15.\,5.\,1862 Wien – 21.\,10.\,1931 ebd.@\textsc{Schnitzler, Arthur} (15.\,5.\,1862 Wien – 21.\,10.\,1931 ebd.), \emph{Schriftsteller, Mediziner}!weite Land. Tragikomödie in fünf Akten@\strich\emph{Das weite Land. Tragikomödie in fünf Akten}|pwk}
                  nach Frankreich\oindex{Frankreich@\textbf{Frankreich}|pwk} zu sprechen: »Dieses
                        Stück\pwindex{Schnitzler, Arthur 15.\,5.\,1862 Wien – 21.\,10.\,1931 ebd.@\textsc{Schnitzler, Arthur} (15.\,5.\,1862 Wien – 21.\,10.\,1931 ebd.), \emph{Schriftsteller, Mediziner}!weite Land. Tragikomödie in fünf Akten@\strich\emph{Das weite Land. Tragikomödie in fünf Akten}|pwv} scheint mir
                     nach den internationalen Seite mehr zu versprechen als der Medardus\pwindex{Schnitzler, Arthur 15.\,5.\,1862 Wien – 21.\,10.\,1931 ebd.@\textsc{Schnitzler, Arthur} (15.\,5.\,1862 Wien – 21.\,10.\,1931 ebd.), \emph{Schriftsteller, Mediziner}!junge Medardus. Dramatische Historie in einem Vorspiel und fünf Aufzügen@\strich\emph{Der junge Medardus. Dramatische Historie in einem Vorspiel und fünf Aufzügen}|pw}«, siehe XXXX Auszeichnungsfehler: Dokument L03981 nicht gefunden. Das
                  Treffen fand am 11. 5. 1911 statt. Der vorliegende Brief dürfte nach dem Treffen
                  entstanden sein (möglicherweise gab Schnitzler{ }Zuckerkandl\pwindex{Zuckerkandl, Berta 13.\,4.\,1864 Wien – 16.\,10.\,1945 Paris@\textsc{Zuckerkandl, Berta} (13.\,4.\,1864 Wien – 16.\,10.\,1945 Paris), \emph{Schriftstellerin, Journalistin, Übersetzerin}|pwk} den Text\pwindex{Schnitzler, Arthur 15.\,5.\,1862 Wien – 21.\,10.\,1931 ebd.@\textsc{Schnitzler, Arthur} (15.\,5.\,1862 Wien – 21.\,10.\,1931 ebd.), \emph{Schriftsteller, Mediziner}!weite Land. Tragikomödie in fünf Akten@\strich\emph{Das weite Land. Tragikomödie in fünf Akten}|pwkv} gleich mit) und vor Zuckerkandls\pwindex{Zuckerkandl, Berta 13.\,4.\,1864 Wien – 16.\,10.\,1945 Paris@\textsc{Zuckerkandl, Berta} (13.\,4.\,1864 Wien – 16.\,10.\,1945 Paris), \emph{Schriftstellerin, Journalistin, Übersetzerin}|pwk} erstem Brief aus dem
                     Juni, in dem sie berichtet Nachricht aus Paris\oindex{Paris@\textbf{Paris}, \emph{Hauptstadt}|pwk} erhalten zu haben, und zwar mindestens 10 Tage vor
                  diesem Brief, denn dort entschuldigt sie sich für die lange Pause: »Ich war
                     sehr leidend und arbeitsunfähig. Auch hatte ich nichts Weiteres aus Paris\oindex{Paris@\textbf{Paris}, \emph{Hauptstadt}|pw} gehört.« (siehe XXXX Auszeichnungsfehler: Dokument L03996 nicht gefunden). So ergibt sich die
                  ungefähre Datierung zwischen 11. und 31. 5. 1911.}}}\label{K_L03997-2}.\pend
           
\pstart
           {\pb}Man sagt i{\geminationm}er Frauen seien unergründlich. Ihre Männertypen – schillern noch rätselhafter.
                  \label{K_L03997-3v}\edtext{Friedrich\pwindex{Schnitzler, Arthur 15.\,5.\,1862 Wien – 21.\,10.\,1931 ebd.@\textsc{Schnitzler, Arthur} (15.\,5.\,1862 Wien – 21.\,10.\,1931 ebd.), \emph{Schriftsteller, Mediziner}!weite Land. Tragikomödie in fünf Akten@\strich\emph{Das weite Land. Tragikomödie in fünf Akten}|pwv}}{\lemma{\textnormal{\emph{Friedrich}}}\Cendnote{\textnormal{Hauptfigur von \emph{Das weite Land}\pwindex{Schnitzler, Arthur 15.\,5.\,1862 Wien – 21.\,10.\,1931 ebd.@\textsc{Schnitzler, Arthur} (15.\,5.\,1862 Wien – 21.\,10.\,1931 ebd.), \emph{Schriftsteller, Mediziner}!weite Land. Tragikomödie in fünf Akten@\strich\emph{Das weite Land. Tragikomödie in fünf Akten}|pwk}}}}\label{K_L03997-3} – ! Alles giebt so viel Denck- und Gefühls-Nahrung.\pend
           
\pstart
           Viel herzlich Danck, {\\[\baselineskip]}\spacefill\mbox{B. Zuckerkandl}\pend
           \leftskip=0em{}\selectlanguage{ngerman}\endnumbering\briefempfaengerindex{Schnitzler, Arthur@\textsc{Schnitzler, Arthur}!zzzZuckerkandl, Berta@\emph{von Berta Zuckerkandl}!1911-05-111@{{[}zwischen 11. und 31. 5. 1911?{]}}|)be}\mylabel{L03997h}
\begin{anhang}
\end{anhang}\newcommand{\dateiname}{L03997}\newcommand{\titel}{Berta Zuckerkandl an Arthur Schnitzler, [zwischen 11. und 31. 5. 1911?]}\newcommand{\editorInnen}{Herausgegeben von Jahnke, SelmaMüller, Martin Anton}%% latex-leseansicht-abspann.tex
%% Abspann für die Leseansicht.
%% Der Schalter \ifkorrekturansicht ist bereits durch den Vorspann gesetzt.

%% latex-abspann.tex
%% Gemeinsamer Abspann für Korrekturansicht und Leseansicht.
%% Setzt den Schalter \ifkorrekturansicht voraus (gesetzt in den
%% einbindenden Dateien latex-korrekturansicht-abspann.tex bzw.
%% latex-leseansicht-abspann.tex).
%% ---------------------------------------------------------------

\normalsize

% Das esempio-Environment wird nur in der Leseansicht benötigt
\ifkorrekturansicht\else
\newenvironment{esempio}[3]%
{
    \vspace{1.5ex}
    \rlap{\underline{#1}}
    \par
    \setlength{\parindent}{0cm}
    \nopagebreak
    \leftskip=#2cm
    \rightskip=#3cm
}
{
    \par
}
\fi

\doendnotes{C}
\bigskip
\vfill

\clearpage

\footnotesize

\ifkorrekturansicht
  \lohead{\textsc{register}}
\fi

% theindex-Environment neu definieren ohne reledmac
\makeatletter
\renewenvironment{theindex}{%
  \ifkorrekturansicht
    \section*{\indexname}%
  \else
    \subsubsection*{Index der erwähnten Entitäten}%
  \fi
  \setlength{\parindent}{0pt}%
  \setlength{\parskip}{0pt plus 0.3pt}%
  \let\item\@idxitem
}{%
  \ifkorrekturansicht\clearpage\fi
}
\makeatother

\IfFileExists{\jobname-pw.ind}{\input{\jobname-pw.ind}}{}

% Quellenangabe nur in der Leseansicht
\ifkorrekturansicht\else
% Fallback-Definitionen, falls die .tex-Datei \titel etc. nicht gesetzt hat
\providecommand{\titel}{}
\providecommand{\editorInnen}{}
\providecommand{\dateiname}{\jobname}

\vspace{3cm}

\vfill

\footnotesize
\textsc{Quelle}: \titel. Herausgegeben von {\editorInnen}. In: \emph{Arthur Schnitzler: Briefwechsel mit Autorinnen und Autoren}.
 Digitale Edition, https://schnitzler-briefe.acdh.oeaw.ac.at/{\dateiname}.html (Stand \today)
\fi

\end{document}


