%% latex-korrekturansicht-vorspann.tex
%% Vorspann für die Korrekturansicht.
%% Lädt die gemeinsame Datei latex-vorspann.tex mit gesetztem Schalter.

\newif\ifkorrekturansicht
\korrekturansichttrue

\input{../tex-inputs/latex-vorspann}


\section[Hermann Bahr an Arthur Schnitzler, 28. 6. 1909]{L01851 Hermann Bahr an Arthur Schnitzler, 28. 6. 1909}
\nopagebreak\mylabel{L01851v}
\rehead{ }\normalsize\beginnumbering\briefempfaengerindex{Schnitzler, Arthur@\textsc{Schnitzler, Arthur}!zzzBahr, Hermann@\emph{von Hermann Bahr}!1909-06-281@{28. 6. 1909}|(be}
\toendnotes[C]{\smallbreak\pagebreak[2]}\Standort{CUL, Schnitzler, B 5b.}
\physDesc{Kartenbrief, 673 Zeichen
\newline{}Handschrift: schwarze Tinte, deutsche Kurrent
\newline{}Versand: 1) Stempel: »\nobreak{}\oindex{Bayreuth@\textbf{Bayreuth}, \emph{P.PPLA2}|pwk}Bayreuth, 29 Juni 09\nobreak{}«.   2) Stempel: »\nobreak{}\oindex{Edlach@\textbf{Edlach}, \emph{P.PPL}|pwk}Edlach b. Reichenau in
                                       N.Oe., 30 6 09, 2–6 N\nobreak{}«. 
\newline{}Schnitzler: mit Bleistift ergänzt »Bahr« 
\newline{}Ordnung: mit Bleistift von unbekannter Hand nummeriert:
                                    »158« }
\buchAbdrucke{\weitereDrucke{Hermann Bahr, Arthur Schnitzler: \emph{Briefwechsel, Aufzeichnungen, Dokumente (1891–1931)}. Göttingen: \emph{Wallstein} 2018, S. 420.} }\toendnotes[C]{\smallbreak}\pstart{}{\pb}Hermann\textsc{Bahr}\pend{}\pstart{}\textsc{Bayreuth Parsifalgasse 12\oindex{Parsifalstrasse@\textbf{Parsifalstraße}, \emph{Straße (K.STR)}|pw}}\pend{}{\bigskip}\pstart{}Herrn \textsc{D\textsuperscript{r} Artur
                     Schnitzler}\pend{}\pstart{}aus \textsc{Wien XVIII Spöttelgasse 7}\oindex{Edmund-Weiss-Gasse 7@\textbf{Edmund-Weiß-Gasse 7}, \emph{Wohngebäude (K.WHS)}|pw}\pend{}\pstart{}\textsc{Edlach} b. Wien\oindex{Edlach@\textbf{Edlach}, \emph{P.PPL}|pw}\pend{}\pstart{}\textsc{Südbahn}\pend{}{\bigskip}\vspace{1em}
\pstart
           \raggedleft{}{\pb}Bayreuth\oindex{Bayreuth@\textbf{Bayreuth}, \emph{P.PPLA2}|pw}{ }28. 6. 09\pend
           \vspace{0.5em}
\pstart
           Dank ſchön, lieber Arthur, für Deine ſo lieben Zeilen!\pend
           
\pstart
           Ich denke, daß dann vielleicht nicht blos Du ſagen wirſt: \label{LL319-1v}Schad!\label{LL319-1h} Oft denke ich das.\pend
           
\pstart
           Hoffentlich gehts Deinem Buben\pwindex{Schnitzler, Heinrich 09.08.1902 – 12.07.1982@\textsc{Schnitzler, Heinrich} (09.08.1902 – 12.07.1982), \emph{Regisseur/Regisseurin, Schauspieler/Schauspielerin}|pwv}{ }ſchon wieder gut.\pend
           
\pstart
           Hier iſts jetzt, noch ganz ohne »Fremde« (und die »Künſtler« findet auch nur, wer ſie
               ſehr ſucht), unbeſchreiblich ſchön und man ſpürt in dieſer einzigen Landſchaft doch,
               daß es ums Deutſche ſchon was iſt, dort wos aus der Erde wächſt (aber nicht in Prag\oindex{Prag@\textbf{Prag}, \emph{A.ADM1}|pw}).\pend
           
\pstart
           Wärſt Du hier!\pend
           
\pstart
           Hier könnte man reden.\pend
           
\pstart
           Grüß herzlichſt Deine liebe Frau\pwindex{Schnitzler, Olga 17.01.1882 – 13.01.1970@\textsc{Schnitzler, Olga} (17.01.1882 – 13.01.1970), \emph{Schauspieler/Schauspielerin, Sänger/Sängerin}|pwv}.{\\[\baselineskip]}In alter, ſehr wirklicher Freundſchaft{\\[\baselineskip]}\spacefill\mbox{Hermann}\pend
           \leftskip=0em{}\selectlanguage{ngerman}\endnumbering\briefempfaengerindex{Schnitzler, Arthur@\textsc{Schnitzler, Arthur}!zzzBahr, Hermann@\emph{von Hermann Bahr}!1909-06-281@{28. 6. 1909}|)be}\mylabel{L01851h}  \normalsize

\doendnotes{C}
\bigskip
\vfill

\clearpage

\footnotesize

\lohead{\textsc{register}}

% Definiere theindex-Environment komplett neu ohne reledmac
\makeatletter
\renewenvironment{theindex}{%
  \section*{\indexname}%
  \setlength{\parindent}{0pt}%
  \setlength{\parskip}{0pt plus 0.3pt}%
  \let\item\@idxitem
}{%
  \clearpage
}
\makeatother

\IfFileExists{\jobname-pw.ind}{\input{\jobname-pw.ind}}{}

\end{document}

      