%% latex-leseansicht-vorspann.tex
%% Vorspann für die Leseansicht.
%% Lädt die gemeinsame Datei latex-vorspann.tex mit nicht gesetztem Schalter.

\newif\ifkorrekturansicht
\korrekturansichtfalse

\input{../tex-inputs/latex-vorspann}


\section[Hermann Bahr an Arthur Schnitzler, 28. 6. 1909]{L01851 Hermann Bahr an Arthur Schnitzler, 28. 6. 1909}
\nopagebreak\mylabel{L01851v}
\rehead{ }\normalsize\beginnumbering\briefempfaengerindex{Schnitzler, Arthur@\textsc{Schnitzler, Arthur}!zzzBahr, Hermann@\emph{von Hermann Bahr}!1909-06-281@{28. 6. 1909}|(be}
\toendnotes[C]{\smallbreak\pagebreak[2]}
\correspDesc{Versand  durch Hermann Bahr am 28. 6. 1909 in Bayreuth
\newline{}Erhalt  durch Arthur Schnitzler am 30 6 09 in Edlach}\toendnotes[C]{\smallbreak}
\Standort{CUL, Schnitzler, B 5b.}
\physDesc{Kartenbrief, 673 Zeichen
\newline{}Handschrift: schwarze Tinte, deutsche Kurrent
\newline{}Versand: 1) Stempel: »\nobreak{}\oindex{Bayreuth@\textbf{Bayreuth}, \emph{Hauptstadt}|pwk}Bayreuth, 29 Juni 09\nobreak{}«.   2) Stempel: »\nobreak{}\oindex{Edlach@\textbf{Edlach}|pwk}Edlach b. Reichenau in
                                       N.Oe., 30 6 09, 2–6 N\nobreak{}«. 
\newline{}Schnitzler: mit Bleistift ergänzt »Bahr« 
\newline{}Ordnung: mit Bleistift von unbekannter Hand nummeriert:
                                    »158« }
\buchAbdrucke{\weitereDrucke{Hermann Bahr, Arthur Schnitzler: \emph{Briefwechsel, Aufzeichnungen, Dokumente (1891–1931)}. Herausgegeben von Kurt Ifkovits und Martin Anton Müller. Göttingen: \emph{Wallstein} 2018, S. 420.} }\toendnotes[C]{\smallbreak}\pstart{}{\pb}Hermann\textsc{Bahr}\pend{}\pstart{}\textsc{Bayreuth Parsifalgasse 12\oindex{Parsifalstraße@\textbf{Parsifalstraße}, \emph{Straße}|pw}}\pend{}{\bigskip}\pstart{}Herrn \textsc{D\textsuperscript{r} Artur
                     Schnitzler}\pend{}\pstart{}aus \textsc{Wien XVIII Spöttelgasse 7}\oindex{Wien@\textbf{Wien}!XVIII., Währing@\textbf{XVIII., Währing}!Edmund-Weiß-Gasse 7@\textbf{Edmund-Weiß-Gasse 7}, \emph{Wohngebäude}|pw}\pend{}\pstart{}\textsc{Edlach} b. Wien\oindex{Edlach@\textbf{Edlach}|pw}\pend{}\pstart{}\textsc{Südbahn}\pend{}{\bigskip}\vspace{1em}
\pstart
           \raggedleft{}{\pb}Bayreuth\oindex{Bayreuth@\textbf{Bayreuth}, \emph{Hauptstadt}|pw}{ }28. 6. 09\pend
           \vspace{0.5em}
\pstart
           Dank{ }ſchön, lieber Arthur, für Deine{ }ſo lieben Zeilen!\pend
           
\pstart
           Ich denke, daß dann vielleicht nicht blos Du{ }ſagen wirſt: \label{LL319-1v}Schad!\label{LL319-1h} Oft denke ich das.\pend
           
\pstart
           Hoffentlich gehts Deinem Buben\pwindex{Schnitzler, Heinrich 9.\,8.\,1902 Hinterbrühl – 12.\,7.\,1982 Wien@\textsc{Schnitzler, Heinrich} (9.\,8.\,1902 Hinterbrühl – 12.\,7.\,1982 Wien), \emph{Regisseur, Schauspieler}|pwv}{ }ſchon wieder gut.\pend
           
\pstart
           Hier iſts jetzt, noch ganz ohne »Fremde« (und die »Künſtler« findet auch nur, wer{ }ſie{ }ſehr{ }ſucht), unbeſchreiblich{ }ſchön und man{ }ſpürt in dieſer einzigen Landſchaft doch,
               daß es ums Deutſche{ }ſchon was iſt, dort wos aus der Erde wächſt (aber nicht in Prag\oindex{Prag@\textbf{Prag}, \emph{Land}|pw}).\pend
           
\pstart
           Wärſt Du hier!\pend
           
\pstart
           Hier könnte man reden.\pend
           
\pstart
           Grüß herzlichſt Deine liebe Frau\pwindex{Schnitzler, Olga 17.\,1.\,1882 Wien – 13.\,1.\,1970 Lugano@\textsc{Schnitzler, Olga} (17.\,1.\,1882 Wien – 13.\,1.\,1970 Lugano), \emph{Schauspielerin, Sängerin}|pwv}.{\\[\baselineskip]}In alter,{ }ſehr wirklicher Freundſchaft{\\[\baselineskip]}\spacefill\mbox{Hermann}\pend
           \leftskip=0em{}\selectlanguage{ngerman}\endnumbering\briefempfaengerindex{Schnitzler, Arthur@\textsc{Schnitzler, Arthur}!zzzBahr, Hermann@\emph{von Hermann Bahr}!1909-06-281@{28. 6. 1909}|)be}\mylabel{L01851h}  \newcommand{\dateiname}{L01851}\newcommand{\titel}{Hermann Bahr an Arthur Schnitzler, 28. 6. 1909}\newcommand{\editorInnen}{Herausgegeben von Martin Anton Müller}%% latex-leseansicht-abspann.tex
%% Abspann für die Leseansicht.
%% Der Schalter \ifkorrekturansicht ist bereits durch den Vorspann gesetzt.

%% latex-abspann.tex
%% Gemeinsamer Abspann für Korrekturansicht und Leseansicht.
%% Setzt den Schalter \ifkorrekturansicht voraus (gesetzt in den
%% einbindenden Dateien latex-korrekturansicht-abspann.tex bzw.
%% latex-leseansicht-abspann.tex).
%% ---------------------------------------------------------------

\normalsize

% Das esempio-Environment wird nur in der Leseansicht benötigt
\ifkorrekturansicht\else
\newenvironment{esempio}[3]%
{
    \vspace{1.5ex}
    \rlap{\underline{#1}}
    \par
    \setlength{\parindent}{0cm}
    \nopagebreak
    \leftskip=#2cm
    \rightskip=#3cm
}
{
    \par
}
\fi

\doendnotes{C}
\bigskip
\vfill

\clearpage

\footnotesize

\ifkorrekturansicht
  \lohead{\textsc{register}}
\fi

% theindex-Environment neu definieren ohne reledmac
\makeatletter
\renewenvironment{theindex}{%
  \ifkorrekturansicht
    \section*{\indexname}%
  \else
    \subsubsection*{Index der erwähnten Entitäten}%
  \fi
  \setlength{\parindent}{0pt}%
  \setlength{\parskip}{0pt plus 0.3pt}%
  \let\item\@idxitem
}{%
  \ifkorrekturansicht\clearpage\fi
}
\makeatother

\IfFileExists{\jobname-pw.ind}{\input{\jobname-pw.ind}}{}

% Quellenangabe nur in der Leseansicht
\ifkorrekturansicht\else
% Fallback-Definitionen, falls die .tex-Datei \titel etc. nicht gesetzt hat
\providecommand{\titel}{}
\providecommand{\editorInnen}{}
\providecommand{\dateiname}{\jobname}

\vspace{3cm}

\vfill

\footnotesize
\textsc{Quelle}: \titel. Herausgegeben von {\editorInnen}. In: \emph{Arthur Schnitzler: Briefwechsel mit Autorinnen und Autoren}.
 Digitale Edition, https://schnitzler-briefe.acdh.oeaw.ac.at/{\dateiname}.html (Stand \today)
\fi

\end{document}


