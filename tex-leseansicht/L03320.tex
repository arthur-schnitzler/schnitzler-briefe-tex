%% latex-leseansicht-vorspann.tex
%% Vorspann für die Leseansicht.
%% Lädt die gemeinsame Datei latex-vorspann.tex mit nicht gesetztem Schalter.

\newif\ifkorrekturansicht
\korrekturansichtfalse

\input{../tex-inputs/latex-vorspann}


         
         \renewcommand{\erwaehntePersonen}{Personen: Carl Kohlis, Felix Salten, Olga Schnitzler}
         \renewcommand{\erwaehnteInstitutionen}{Institutionen: Jung-Wiener Theater zum Lieben Augustin}
         \renewcommand{\erwaehnteOrte}{Orte: Berlin, Hotel Kronprinz, Jung-Wiener Theater zum Lieben Augustin, Luisenstraße, Marschallbrücke, Reichstag, Schiffbauerdamm, Wien}
         \renewcommand{\erwaehnteWerke}{Werke: Die Insel. Monatsschrift mit Buchschmuck und Illustrationen, Schlange}
               \section[ Felix Salten an Arthur Schnitzler, 9. 10. 1901]{ Felix Salten an Arthur Schnitzler, 9. 10. 1901}\nopagebreak\mylabel{v}\rehead{ }\begin{ledgroupsized}[t]{13cm}\normalsize\beginnumbering\briefempfaengerindex{Schnitzler, Arthur@\textsc{Schnitzler, Arthur}!zzzSalten, Felix@\emph{von Felix Salten}!1901-10-091@{9. 10. 1901}|(be} \toendnotes[C]{\smallbreak\pagebreak[2]} \Standort{CUL, Schnitzler, B 89, A 2.}
\physDesc{Brief, 1 Blatt, 1 Seite, 319 Zeichen
\newline{}Handschrift: schwarze Tinte, lateinische Kurrent
\newline{}Ordnung: mit Bleistift von unbekannter Hand nummeriert: »144« }\toendnotes[C]{\smallbreak}\pstart
           \noindent{}\centering{}{\pb}\textcolor{gray}{\textbf{Hôtel Kronprinz\oindex{Hotel Kronprinz@\textbf{Hotel Kronprinz}|pw}}}\pend
           \pstart
           \noindent{}\centering{}\textcolor{gray}{\textbf{\textsc{Berlin} N.W. 6\oindex{Berlin@\textbf{Berlin}|pw}.}}\pend
           \pstart
           \noindent{}\raggedleft{}\textcolor{gray}{\textbf{\emph{Luisen-Str. 30\oindex{Luisenstrasse@\textbf{Luisenstraße}|pw},}}}\pend
           \pstart
           \noindent{}\raggedleft{}\textcolor{gray}{\textbf{\emph{nahe dem Reichstagspalast\oindex{Reichstag@\textbf{Reichstag}|pw},}}}\pend
           \pstart
           \noindent{}\textcolor{gray}{\textbf{\emph{Direktion: C.
                              Kohlis\pwindex{Kohlis, Carl 1857-09-25 – 1910-12-21@\textsc{Kohlis, Carl} (1857-09-25 – 1910-12-21), \emph{Journalist, Dichter, Hoteldirektor}|pw}.}}}\hfill \textcolor{gray}{\textbf{\emph{Ecke Schiffbauerdamm\oindex{Schiffbauerdamm@\textbf{Schiffbauerdamm}|pw} (a. d. Marschall-Brücke\oindex{Marschallbruecke@\textbf{Marschallbrücke}|pw}).}}}\pend
           \pstart
           \textcolor{gray}{\textbf{Telegr. Adr.: \textsc{Kronprinzhôtel\oindex{Hotel Kronprinz@\textbf{Hotel Kronprinz}|pw}, Berlin\oindex{Berlin@\textbf{Berlin}|pw}.}}}\pend
           \pstart
           \textcolor{gray}{\textbf{Fernsprech-Anschluss: Amt III. N\textsuperscript{o} 8871.}}\pend
           \pstart
           \raggedleft{}\textcolor{gray}{\textbf{Berlin\oindex{Berlin@\textbf{Berlin}|pw}, den}}{ }9 October 01\pend
           \pstart
           Lieber Arthur, herzlichen Dank für die Besorgung der Schlange\pwindex{?? Werk@Nicht ermittelte Verfasserinnen und Verfasser!Schlange@\emph{Schlange}|pw}\textcolor{gray}{,}{ }{\kaufmannsund} für die Insel\pwindex{Insel. Monatsschrift mit Buchschmuck und Illustrationen1899 – 1902@\emph{Die Insel. Monatsschrift mit Buchschmuck und Illustrationen} {[}1899 – 1902{]}|pw}. Da
               ich erst Samstag zurückkomme, (früh)
               können Sie’s vielleicht so einrichten, dass ich Sie Mittag verständigen
               kann, ob {\kaufmannsund} um wie viel Uhr wir Nachmittg
               die \label{K_L03320-1v}\edtext{Bühne\oindex{Jung-Wiener Theater zum Lieben Augustin@\textbf{Jung-Wiener Theater zum Lieben Augustin}|pwuv} haben, und dass
               Sie dann es gleich dem Fräulein\pwindex{Schnitzler, Olga 17.01.1882 – 13.01.1970@\textsc{Schnitzler, Olga} (17.01.1882 – 13.01.1970), \emph{Schauspielerin, Sängerin}|pwuv}}{\lemma{\textnormal{\emph{Bühne … Fräulein}}}\Cendnote{\textnormal{Olga
                     Gussmann\pwindex{Schnitzler, Olga 17.01.1882 – 13.01.1970@\textsc{Schnitzler, Olga} (17.01.1882 – 13.01.1970), \emph{Schauspielerin, Sängerin}|pwk}, Schnitzlers\pwindex{Schnitzler, Arthur 15.05.1862 – 21.10.1931@\textsc{Schnitzler, Arthur} (15.05.1862 – 21.10.1931), \emph{Schriftsteller, Mediziner}|pwk} Lebensgefährtin und nachmalige Ehefrau, dürfte für einen Auftritt beim \emph{Jung-Wiener Theater zum Lieben Augustin}\orgindex{Jung-Wiener Theater zum Lieben Augustin@Jung-Wiener Theater zum Lieben Augustin|pwk}
                  vorgesprochen haben, vgl. Paul Goldmann an Arthur Schnitzler, 7. 10. [1901].}}}\label{K_L03320-1h} mittheilen.\pend
           \pstart
           herzlichst Ihr {\\[\baselineskip]}\spacefill\mbox{Salten}\pend
           \leftskip=0em{}
         
         \endnumbering\mylabel{h}\end{ledgroupsized}  \newcommand{\dateiname}{L03320}\newcommand{\titel}{Felix Salten an Arthur Schnitzler, 9. 10. 1901}\newcommand{\editorInnen}{Martin Anton Müller und Laura Untner}%% latex-leseansicht-abspann.tex
%% Abspann für die Leseansicht.
%% Der Schalter \ifkorrekturansicht ist bereits durch den Vorspann gesetzt.

%% latex-abspann.tex
%% Gemeinsamer Abspann für Korrekturansicht und Leseansicht.
%% Setzt den Schalter \ifkorrekturansicht voraus (gesetzt in den
%% einbindenden Dateien latex-korrekturansicht-abspann.tex bzw.
%% latex-leseansicht-abspann.tex).
%% ---------------------------------------------------------------

\normalsize

% Das esempio-Environment wird nur in der Leseansicht benötigt
\ifkorrekturansicht\else
\newenvironment{esempio}[3]%
{
    \vspace{1.5ex}
    \rlap{\underline{#1}}
    \par
    \setlength{\parindent}{0cm}
    \nopagebreak
    \leftskip=#2cm
    \rightskip=#3cm
}
{
    \par
}
\fi

\doendnotes{C}
\bigskip
\vfill

\clearpage

\footnotesize

\ifkorrekturansicht
  \lohead{\textsc{register}}
\fi

% theindex-Environment neu definieren ohne reledmac
\makeatletter
\renewenvironment{theindex}{%
  \ifkorrekturansicht
    \section*{\indexname}%
  \else
    \subsubsection*{Index der erwähnten Entitäten}%
  \fi
  \setlength{\parindent}{0pt}%
  \setlength{\parskip}{0pt plus 0.3pt}%
  \let\item\@idxitem
}{%
  \ifkorrekturansicht\clearpage\fi
}
\makeatother

\IfFileExists{\jobname-pw.ind}{\input{\jobname-pw.ind}}{}

% Quellenangabe nur in der Leseansicht
\ifkorrekturansicht\else
% Fallback-Definitionen, falls die .tex-Datei \titel etc. nicht gesetzt hat
\providecommand{\titel}{}
\providecommand{\editorInnen}{}
\providecommand{\dateiname}{\jobname}

\vspace{3cm}

\vfill

\footnotesize
\textsc{Quelle}: \titel. Herausgegeben von {\editorInnen}. In: \emph{Arthur Schnitzler: Briefwechsel mit Autorinnen und Autoren}.
 Digitale Edition, https://schnitzler-briefe.acdh.oeaw.ac.at/{\dateiname}.html (Stand \today)
\fi

\end{document}


      