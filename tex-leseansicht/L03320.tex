%% latex-korrekturansicht-vorspann.tex
%% Vorspann für die Korrekturansicht.
%% Lädt die gemeinsame Datei latex-vorspann.tex mit gesetztem Schalter.

\newif\ifkorrekturansicht
\korrekturansichttrue

\input{../tex-inputs/latex-vorspann}


\section[ Felix Salten an Arthur Schnitzler, 9. 10. 1901]{L03320 Felix Salten an Arthur Schnitzler, 9. 10. 1901}
\nopagebreak\mylabel{L03320v}
\rehead{ }\normalsize\beginnumbering\briefempfaengerindex{Schnitzler, Arthur@\textsc{Schnitzler, Arthur}!zzzSalten, Felix@\emph{von Felix Salten}!1901-10-091@{9. 10. 1901}|(be}
\toendnotes[C]{\smallbreak\pagebreak[2]}\Standort{CUL, Schnitzler, B 89, A 2.}
\physDesc{Brief, 1 Blatt, 1 Seite, 319 Zeichen
\newline{}Handschrift: schwarze Tinte, lateinische Kurrent
\newline{}Ordnung: mit Bleistift von unbekannter Hand nummeriert: »144« }\toendnotes[C]{\smallbreak}
\pstart
           \centering{}{\pb}\textcolor{gray}{\textbf{Hôtel Kronprinz\oindex{Hotel Kronprinz [Berlin]@\textbf{Hotel Kronprinz [Berlin]}, \emph{Hotel (K.HTL)}|pw}}}\pend
           
\pstart
           \centering{}\textcolor{gray}{\textbf{\textsc{Berlin} N.W. 6\oindex{Berlin@\textbf{Berlin}, \emph{P.PPLC}|pw}.}}\pend
           
\pstart
           \raggedleft{}\textcolor{gray}{\textbf{\emph{Luisen-Str. 30\oindex{Luisenstrasse@\textbf{Luisenstraße}, \emph{Straße (K.STR)}|pw},}}}\pend
           
\pstart
           \raggedleft{}\textcolor{gray}{\textbf{\emph{nahe dem Reichstagspalast\oindex{Reichstag@\textbf{Reichstag}, \emph{Regierungsgebäude (K.RGB)}|pw},}}}\pend
           
\pstart
           \textcolor{gray}{\textbf{\emph{Direktion: C.
                              Kohlis\pwindex{Kohlis, Carl 1857-09-25 – 1910-12-21@\textsc{Kohlis, Carl} (1857-09-25 – 1910-12-21), \emph{Journalist/Journalistin, Dichter/Dichterin, Hoteldirektor/Hoteldirektorin}|pw}.}}}\hfill \textcolor{gray}{\textbf{\emph{Ecke Schiffbauerdamm\oindex{Schiffbauerdamm@\textbf{Schiffbauerdamm}, \emph{Straße (K.STR)}|pw} (a. d. Marschall-Brücke\oindex{Marschallbruecke@\textbf{Marschallbrücke}, \emph{Brücke (K.BRK)}|pw}).}}}\pend
           
\pstart
           \textcolor{gray}{\textbf{Telegr. Adr.: \textsc{Kronprinzhôtel\oindex{Hotel Kronprinz [Berlin]@\textbf{Hotel Kronprinz [Berlin]}, \emph{Hotel (K.HTL)}|pw}, Berlin\oindex{Berlin@\textbf{Berlin}, \emph{P.PPLC}|pw}.}}}\pend
           
\pstart
           \textcolor{gray}{\textbf{Fernsprech-Anschluss: Amt III. N\textsuperscript{o} 8871.}}\pend
           
\pstart
           \raggedleft{}\textcolor{gray}{\textbf{Berlin\oindex{Berlin@\textbf{Berlin}, \emph{P.PPLC}|pw}, den}}{ }9 October 01\pend
           \vspace{0.5em}
\pstart
           Lieber Arthur, herzlichen Dank für die Besorgung der Schlange\pwindex{Schlange@\emph{Schlange}|pw}\textcolor{gray}{,}{ }{\kaufmannsund} für die Insel\pwindex{Insel. Monatsschrift mit Buchschmuck und Illustrationen@\emph{Die Insel. Monatsschrift mit Buchschmuck und Illustrationen}|pw}. Da
               ich erst Samstag zurückkomme, (früh)
               können Sie’s vielleicht so einrichten, dass ich Sie Mittag verständigen
               kann, ob {\kaufmannsund} um wie viel Uhr wir Nachmittg
               die \label{K_L03320-1v}\edtext{Bühne\oindex{Jung-Wiener Theater zum Lieben Augustin@\textbf{Jung-Wiener Theater zum Lieben Augustin}, \emph{Kabarett (K.KBR)}|pwuv} haben, und dass
               Sie dann es gleich dem Fräulein\pwindex{Schnitzler, Olga 17.01.1882 – 13.01.1970@\textsc{Schnitzler, Olga} (17.01.1882 – 13.01.1970), \emph{Schauspieler/Schauspielerin, Sänger/Sängerin}|pwuv}}{\lemma{\textnormal{\emph{Bühne … Fräulein}}}\Cendnote{\textnormal{Olga
                     Gussmann\pwindex{Schnitzler, Olga 17.01.1882 – 13.01.1970@\textsc{Schnitzler, Olga} (17.01.1882 – 13.01.1970), \emph{Schauspieler/Schauspielerin, Sänger/Sängerin}|pwk}, Schnitzlers Lebensgefährtin und nachmalige Ehefrau, dürfte für einen Auftritt beim \emph{Jung-Wiener Theater zum Lieben Augustin}\orgindex{Jung-Wiener Theater zum Lieben Augustin@Jung-Wiener Theater zum Lieben Augustin|pwk}
                  vorgesprochen haben, vgl. Paul Goldmann an Arthur Schnitzler, 7. 10. [1901].}}}\label{K_L03320-1} mittheilen.\pend
           
\pstart
           herzlichst Ihr {\\[\baselineskip]}\spacefill\mbox{Salten}\pend
           \leftskip=0em{}\selectlanguage{ngerman}\endnumbering\briefempfaengerindex{Schnitzler, Arthur@\textsc{Schnitzler, Arthur}!zzzSalten, Felix@\emph{von Felix Salten}!1901-10-091@{9. 10. 1901}|)be}\mylabel{L03320h}  \normalsize

\doendnotes{C}
\bigskip
\vfill

\clearpage

\footnotesize

\lohead{\textsc{register}}

% Definiere theindex-Environment komplett neu ohne reledmac
\makeatletter
\renewenvironment{theindex}{%
  \section*{\indexname}%
  \setlength{\parindent}{0pt}%
  \setlength{\parskip}{0pt plus 0.3pt}%
  \let\item\@idxitem
}{%
  \clearpage
}
\makeatother

\IfFileExists{\jobname-pw.ind}{\input{\jobname-pw.ind}}{}

\end{document}

      