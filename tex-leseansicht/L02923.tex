%% latex-leseansicht-vorspann.tex
%% Vorspann für die Leseansicht.
%% Lädt die gemeinsame Datei latex-vorspann.tex mit nicht gesetztem Schalter.

\newif\ifkorrekturansicht
\korrekturansichtfalse

\input{../tex-inputs/latex-vorspann}

\begin{center}
            \textcolor{red}{ENTWURF, NICHT FERTIG KORRIGIERT}
                      \end{center}
            
         
         \newcommand{\erwaehntePersonen}{Personen: Richard Beer-Hofmann, Ludwig Fulda, Alfred Kerr, Leo Van-Jung}
         \newcommand{\erwaehnteOrte}{Orte: Berlin, China, Dessauer Straße, Innsbruck, Reichenau an der Rax}
         \newcommand{\erwaehnteWerke}{
               \section[ Paul Goldmann an Arthur Schnitzler, 5. 7. {[}1900{]}]{ Paul Goldmann an Arthur Schnitzler, 5. 7. {[}1900{]}}\nopagebreak\mylabel{v}\rehead{ }\begin{ledgroupsized}[t]{13cm}\normalsize\beginnumbering \toendnotes[C]{\smallbreak\pagebreak[2]} \Standort{DLA, A:Schnitzler, HS.NZ85.1.3170.}
\physDesc{Brief, 1 Blatt, 2 Seiten
\newline{}Handschrift: blaue Tinte, deutsche Kurrent
\newline{}Schnitzler: mit Bleistift das Jahr »{[}1{]}900« vermerkt }\toendnotes[C]{\smallbreak}\pstart
           \raggedleft{}{\pb}Berlin\oindex{Berlin@\textbf{Berlin}|pw}, 5. Juli.\pend
           \pstart
           \textcolor{gray}{\textbf{DESSAUERSTRASSE 19}}\oindex{Dessauer Strasse@\textbf{Dessauer Straße}|pw}\pend
           \pstart\center{}Mein lieber Freund,\pend\pstart
           Mit dem \label{K_L02923-1v}\edtext{Rendevous in Innsbruck\oindex{Innsbruck@\textbf{Innsbruck}|pw} Mitte Auguſt}{\lemma{\textnormal{\emph{Rendevous … Auguſt}}}\Cendnote{\textnormal{siehe Paul Goldmann an Arthur Schnitzler, 16. 6. [1900] und A. S.: \emph{Tagebuch}, 16. 8. 1900}}}\label{K_L02923-1h} behufs Antritts der Fußparthie wäre ich einverſtanden. Freilich wird es durch
               die \label{K_L02923-3v}\edtext{chin\oindex{China@\textbf{China}|pwv}eſiſchen Ereigniſſe}{\lemma{\textnormal{\emph{chineſiſchen Ereigniſſe}}}\Cendnote{\textnormal{Bezug auf den sich im Sommer 1900 zuspitzenden chin\oindex{China@\textbf{China}|pwkv}esischen Boxeraufstand}}}\label{K_L02923-3h} immer fraglicher, ob ich
               überhaupt fort kann. Es wäre ſehr ſchön, wenn \textsc{Leo\pwindex{Van-Jung, Leo 15.10.1866 – 02.07.1939@\textsc{Van-Jung, Leo} (15.10.1866 – 02.07.1939), \emph{Gesangspädagoge, Mathematiker}|pw}} und \textsc{Richard\pwindex{Beer-Hofmann, Richard 1866-07-11 – 1945-09-26@\textsc{Beer-Hofmann, Richard} (1866-07-11 – 1945-09-26), \emph{Schriftsteller}|pw}} mit kämen. Wohin wollen wir wandern? Und wie lange, glaubſt Du, wird das
               dauern?\pend
           \pstart
           Wie geht es Dir? Bitte, laß’ bald wieder von Dir {\pb}hören. Haſt Du von \label{K_L02923-2v}\edtext{\textsc{Fulda\pwindex{Fulda, Ludwig 15.07.1862 – 30.03.1939@\textsc{Fulda, Ludwig} (15.07.1862 – 30.03.1939), \emph{Schriftsteller, Übersetzer}|pw}}}{\lemma{\textnormal{\emph{Fulda}}}\Cendnote{\textnormal{vermutlich Bezug auf gemeinsame
                  Sommerpläne (siehe A. S.: \emph{Tagebuch}, 27. 8. 1900,
                     28. 8. 1900 und
                     29. 8. 1900)}}}\label{K_L02923-2h} ſchon Beſcheid?\pend
           \pstart
           \textsc{Kerr\pwindex{Kerr, Alfred 25.12.1867 – 12.10.1948@\textsc{Kerr, Alfred} (25.12.1867 – 12.10.1948), \emph{Schriftsteller, Kritiker}|pw}} dürfte Mitte Auguſt auch mithalten.\pend
           \pstart
           Viele treue Grüße! {\\[\baselineskip]}Dein {\\[\baselineskip]}\spacefill\mbox{Paul Goldmnn.}\pend
           \leftskip=0em{}
         
         \endnumbering\mylabel{h}\end{ledgroupsized}\begin{anhang}\end{anhang}\newcommand{\dateiname}{L02923}\newcommand{\titel}{Paul Goldmann an Arthur Schnitzler, 5. 7. [1900]}\newcommand{\editorInnen}{Martin Anton Müller und Laura Untner}%% latex-leseansicht-abspann.tex
%% Abspann für die Leseansicht.
%% Der Schalter \ifkorrekturansicht ist bereits durch den Vorspann gesetzt.

%% latex-abspann.tex
%% Gemeinsamer Abspann für Korrekturansicht und Leseansicht.
%% Setzt den Schalter \ifkorrekturansicht voraus (gesetzt in den
%% einbindenden Dateien latex-korrekturansicht-abspann.tex bzw.
%% latex-leseansicht-abspann.tex).
%% ---------------------------------------------------------------

\normalsize

% Das esempio-Environment wird nur in der Leseansicht benötigt
\ifkorrekturansicht\else
\newenvironment{esempio}[3]%
{
    \vspace{1.5ex}
    \rlap{\underline{#1}}
    \par
    \setlength{\parindent}{0cm}
    \nopagebreak
    \leftskip=#2cm
    \rightskip=#3cm
}
{
    \par
}
\fi

\doendnotes{C}
\bigskip
\vfill

\clearpage

\footnotesize

\ifkorrekturansicht
  \lohead{\textsc{register}}
\fi

% theindex-Environment neu definieren ohne reledmac
\makeatletter
\renewenvironment{theindex}{%
  \ifkorrekturansicht
    \section*{\indexname}%
  \else
    \subsubsection*{Index der erwähnten Entitäten}%
  \fi
  \setlength{\parindent}{0pt}%
  \setlength{\parskip}{0pt plus 0.3pt}%
  \let\item\@idxitem
}{%
  \ifkorrekturansicht\clearpage\fi
}
\makeatother

\IfFileExists{\jobname-pw.ind}{\input{\jobname-pw.ind}}{}

% Quellenangabe nur in der Leseansicht
\ifkorrekturansicht\else
% Fallback-Definitionen, falls die .tex-Datei \titel etc. nicht gesetzt hat
\providecommand{\titel}{}
\providecommand{\editorInnen}{}
\providecommand{\dateiname}{\jobname}

\vspace{3cm}

\vfill

\footnotesize
\textsc{Quelle}: \titel. Herausgegeben von {\editorInnen}. In: \emph{Arthur Schnitzler: Briefwechsel mit Autorinnen und Autoren}.
 Digitale Edition, https://schnitzler-briefe.acdh.oeaw.ac.at/{\dateiname}.html (Stand \today)
\fi

\end{document}


      