%% latex-leseansicht-vorspann.tex
%% Vorspann für die Leseansicht.
%% Lädt die gemeinsame Datei latex-vorspann.tex mit nicht gesetztem Schalter.

\newif\ifkorrekturansicht
\korrekturansichtfalse

\input{../tex-inputs/latex-vorspann}


\section[ Paul Goldmann an Arthur Schnitzler, 5. 7. {[}1900{]}]{L02923 Paul Goldmann an Arthur Schnitzler,  5. 7. [1900]}
\nopagebreak\mylabel{L02923v}
\rehead{ }\normalsize\beginnumbering\briefempfaengerindex{Schnitzler, Arthur@\textsc{Schnitzler, Arthur}!zzzGoldmann, Paul@\emph{von Paul Goldmann}!1900-07-051@{5. 7. [1900]}|(be}
\toendnotes[C]{\smallbreak\pagebreak[2]}
\correspDesc{Versand  durch Paul Goldmann am 5. 7. [1900] in Berlin
\newline{}Erhalt  durch Arthur Schnitzler im Zeitraum [6. 7. 1900 –
                  7. 7. 1900] in Reichenau an der Rax}\toendnotes[C]{\smallbreak}
\Standort{DLA, A:Schnitzler, HS.NZ85.1.3170.}
\physDesc{Brief, 1 Blatt, 2 Seiten, 498 Zeichen
\newline{}Handschrift: blaue Tinte, deutsche Kurrent
\newline{}Schnitzler: mit Bleistift das Jahr »900« vermerkt }\toendnotes[C]{\smallbreak}
\pstart
           {\pb}\textcolor{gray}{\textbf{DESSAUERSTRASSE 19}}\oindex{Dessauer Straße@\textbf{Dessauer Straße}, \emph{Straße}|pw}\hfill Berlin\oindex{Berlin@\textbf{Berlin}, \emph{Hauptstadt}|pw}, 5. Juli.\pend
           
\pstart\center{}Mein lieber Freund,\pend\vspace{0.5em}
\pstart
           Mit dem \label{K_L02923-1v}\edtext{Rendevous in Innsbruck\oindex{Innsbruck@\textbf{Innsbruck}, \emph{Verwaltungsgebiet}|pw} Mitte Auguſt}{\lemma{\textnormal{\emph{Rendevous … August}}}\Cendnote{\textnormal{Siehe XXXX Auszeichnungsfehler: Dokument L02920 nicht gefunden und A. S.: \emph{Tagebuch}, 16. 8. 1900.
               }}}\label{K_L02923-1} behufs Antritts der Fußparthie wäre ich einverſtanden. Freilich wird es durch
               die \label{K_L02923-2v}\edtext{chineſiſchen\oindex{China@\textbf{China}|pwv} Ereigniſſe}{\lemma{\textnormal{\emph{chinesischen Ereignisse}}}\Cendnote{\textnormal{Im Sommer 1900
                  hatte sich der chinesische\oindex{China@\textbf{China}|pwkv} Boxeraufstand zugespitzt.}}}\label{K_L02923-2} immer fraglicher, ob ich überhaupt fort
               kann. Es wäre{ }ſehr{ }ſchön, wenn \textsc{Leo\pwindex{Van-Jung, Leo 15.\,10.\,1866 Odessa – 2.\,7.\,1939 Riga@\textsc{Van-Jung, Leo} (15.\,10.\,1866 Odessa – 2.\,7.\,1939 Riga), \emph{Gesangspädagoge, Mathematiker}|pw}} und \textsc{Richard\pwindex{Beer-Hofmann, Richard 11.\,7.\,1866 Wien – 26.\,9.\,1945 New York City@\textsc{Beer-Hofmann, Richard} (11.\,7.\,1866 Wien – 26.\,9.\,1945 New York City), \emph{Schriftsteller}|pw}} mitkämen. Wohin wollen wir wandern? Und wie lange, glaubſt Du, wird das
               dauern?\pend
           
\pstart
           Wie geht es Dir? Bitte, laß’ bald wieder von Dir {\pb}hören! Haſt Du von \label{K_L02923-3v}\edtext{\textsc{Fulda\pwindex{Fulda, Ludwig 15.\,7.\,1862 Frankfurt am Main – 30.\,3.\,1939 Berlin@\textsc{Fulda, Ludwig} (15.\,7.\,1862 Frankfurt am Main – 30.\,3.\,1939 Berlin), \emph{Schriftsteller, Übersetzer}|pw}}}{\lemma{\textnormal{\emph{Fulda}}}\Cendnote{\textnormal{Fulda\pwindex{Fulda, Ludwig 15.\,7.\,1862 Frankfurt am Main – 30.\,3.\,1939 Berlin@\textsc{Fulda, Ludwig} (15.\,7.\,1862 Frankfurt am Main – 30.\,3.\,1939 Berlin), \emph{Schriftsteller, Übersetzer}|pwk} bemühte sich, den \emph{Schleier der Beatrice}\pwindex{Schnitzler, Arthur 15.\,5.\,1862 Wien – 21.\,10.\,1931 ebd.@\textsc{Schnitzler, Arthur} (15.\,5.\,1862 Wien – 21.\,10.\,1931 ebd.), \emph{Schriftsteller, Mediziner}!Schleier der Beatrice. Schauspiel in fünf Akten@\strich\emph{Der Schleier der Beatrice. Schauspiel in fünf Akten}|pwk} an das \emph{Berliner Schauspielhaus}\orgindex{Schauspielhaus Berlin@Schauspielhaus Berlin|pwk} zu vermitteln.}}}\label{K_L02923-3}{ }ſchon
               Beſcheid?\pend
           
\pstart
           \textsc{Kerr\pwindex{Kerr, Alfred 25.\,12.\,1867 Breslau – 12.\,10.\,1948 Hamburg@\textsc{Kerr, Alfred} (25.\,12.\,1867 Breslau – 12.\,10.\,1948 Hamburg), \emph{Schriftsteller, Kritiker}|pw}} dürfte Mitte Auguſt auch mithalten.\pend
           
\pstart
           Viele treue Grüße! {\\[\baselineskip]}Dein {\\[\baselineskip]}\spacefill\mbox{Paul Goldmann.}\pend
           \leftskip=0em{}\selectlanguage{ngerman}\endnumbering\briefempfaengerindex{Schnitzler, Arthur@\textsc{Schnitzler, Arthur}!zzzGoldmann, Paul@\emph{von Paul Goldmann}!1900-07-051@{5. 7. [1900]}|)be}\mylabel{L02923h}  \newcommand{\dateiname}{L02923}\newcommand{\titel}{Paul Goldmann an Arthur Schnitzler, 5. 7. [1900]}\newcommand{\editorInnen}{Martin Anton Müller und Laura Untner}%% latex-leseansicht-abspann.tex
%% Abspann für die Leseansicht.
%% Der Schalter \ifkorrekturansicht ist bereits durch den Vorspann gesetzt.

%% latex-abspann.tex
%% Gemeinsamer Abspann für Korrekturansicht und Leseansicht.
%% Setzt den Schalter \ifkorrekturansicht voraus (gesetzt in den
%% einbindenden Dateien latex-korrekturansicht-abspann.tex bzw.
%% latex-leseansicht-abspann.tex).
%% ---------------------------------------------------------------

\normalsize

% Das esempio-Environment wird nur in der Leseansicht benötigt
\ifkorrekturansicht\else
\newenvironment{esempio}[3]%
{
    \vspace{1.5ex}
    \rlap{\underline{#1}}
    \par
    \setlength{\parindent}{0cm}
    \nopagebreak
    \leftskip=#2cm
    \rightskip=#3cm
}
{
    \par
}
\fi

\doendnotes{C}
\bigskip
\vfill

\clearpage

\footnotesize

\ifkorrekturansicht
  \lohead{\textsc{register}}
\fi

% theindex-Environment neu definieren ohne reledmac
\makeatletter
\renewenvironment{theindex}{%
  \ifkorrekturansicht
    \section*{\indexname}%
  \else
    \subsubsection*{Index der erwähnten Entitäten}%
  \fi
  \setlength{\parindent}{0pt}%
  \setlength{\parskip}{0pt plus 0.3pt}%
  \let\item\@idxitem
}{%
  \ifkorrekturansicht\clearpage\fi
}
\makeatother

\IfFileExists{\jobname-pw.ind}{\input{\jobname-pw.ind}}{}

% Quellenangabe nur in der Leseansicht
\ifkorrekturansicht\else
% Fallback-Definitionen, falls die .tex-Datei \titel etc. nicht gesetzt hat
\providecommand{\titel}{}
\providecommand{\editorInnen}{}
\providecommand{\dateiname}{\jobname}

\vspace{3cm}

\vfill

\footnotesize
\textsc{Quelle}: \titel. Herausgegeben von {\editorInnen}. In: \emph{Arthur Schnitzler: Briefwechsel mit Autorinnen und Autoren}.
 Digitale Edition, https://schnitzler-briefe.acdh.oeaw.ac.at/{\dateiname}.html (Stand \today)
\fi

\end{document}


