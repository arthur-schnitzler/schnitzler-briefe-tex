%% latex-korrekturansicht-vorspann.tex
%% Vorspann für die Korrekturansicht.
%% Lädt die gemeinsame Datei latex-vorspann.tex mit gesetztem Schalter.

\newif\ifkorrekturansicht
\korrekturansichttrue

\input{../tex-inputs/latex-vorspann}


\section[ Paul Goldmann an Arthur Schnitzler, 5. 7. {[}1900{]}]{L02923 Paul Goldmann an Arthur Schnitzler, 5. 7. {[}1900{]}}
\nopagebreak\mylabel{L02923v}
\rehead{ }\normalsize\beginnumbering\briefempfaengerindex{Schnitzler, Arthur@\textsc{Schnitzler, Arthur}!zzzGoldmann, Paul@\emph{von Paul Goldmann}!1900-07-051@{5. 7. {[}1900{]}}|(be}
\toendnotes[C]{\smallbreak\pagebreak[2]}\Standort{DLA, A:Schnitzler, HS.NZ85.1.3170.}
\physDesc{Brief, 1 Blatt, 2 Seiten, 498 Zeichen
\newline{}Handschrift: blaue Tinte, deutsche Kurrent
\newline{}Schnitzler: mit Bleistift das Jahr »900« vermerkt }\toendnotes[C]{\smallbreak}
\pstart
           {\pb}\textcolor{gray}{\textbf{DESSAUERSTRASSE 19}}\oindex{Dessauer Strasse@\textbf{Dessauer Straße}, \emph{Straße (K.STR)}|pw}\hfill Berlin\oindex{Berlin@\textbf{Berlin}, \emph{P.PPLC}|pw}, 5. Juli.\pend
           
\pstart\center{}Mein lieber Freund,\pend\vspace{0.5em}
\pstart
           Mit dem \label{K_L02923-1v}\edtext{Rendevous in Innsbruck\oindex{Innsbruck@\textbf{Innsbruck}, \emph{A.ADM2}|pw} Mitte Auguſt}{\lemma{\textnormal{\emph{Rendevous … Auguſt}}}\Cendnote{\textnormal{Siehe Paul Goldmann an Arthur Schnitzler, 16. 6. [1900] und A. S.: \emph{Tagebuch}, 16. 8. 1900.
               }}}\label{K_L02923-1} behufs Antritts der Fußparthie wäre ich einverſtanden. Freilich wird es durch
               die \label{K_L02923-2v}\edtext{chineſiſchen\oindex{China@\textbf{China}, \emph{A.PCLI}|pwv} Ereigniſſe}{\lemma{\textnormal{\emph{chineſiſchen Ereigniſſe}}}\Cendnote{\textnormal{Im Sommer 1900
                  hatte sich der chinesische\oindex{China@\textbf{China}, \emph{A.PCLI}|pwkv} Boxeraufstand zugespitzt.}}}\label{K_L02923-2} immer fraglicher, ob ich überhaupt fort
               kann. Es wäre ſehr ſchön, wenn \textsc{Leo\pwindex{Van-Jung, Leo 15.10.1866 – 02.07.1939@\textsc{Van-Jung, Leo} (15.10.1866 – 02.07.1939), \emph{Gesangspädagoge/Gesangspädagogin, Mathematiker/Mathematikerin}|pw}} und \textsc{Richard\pwindex{Beer-Hofmann, Richard 1866-07-11 – 1945-09-26@\textsc{Beer-Hofmann, Richard} (1866-07-11 – 1945-09-26), \emph{Schriftsteller/Schriftstellerin}|pw}} mitkämen. Wohin wollen wir wandern? Und wie lange, glaubſt Du, wird das
               dauern?\pend
           
\pstart
           Wie geht es Dir? Bitte, laß’ bald wieder von Dir {\pb}hören! Haſt Du von \label{K_L02923-3v}\edtext{\textsc{Fulda\pwindex{Fulda, Ludwig 15.07.1862 – 30.03.1939@\textsc{Fulda, Ludwig} (15.07.1862 – 30.03.1939), \emph{Schriftsteller/Schriftstellerin, Übersetzer/Übersetzerin}|pw}}}{\lemma{\textnormal{\emph{Fulda}}}\Cendnote{\textnormal{Fulda\pwindex{Fulda, Ludwig 15.07.1862 – 30.03.1939@\textsc{Fulda, Ludwig} (15.07.1862 – 30.03.1939), \emph{Schriftsteller/Schriftstellerin, Übersetzer/Übersetzerin}|pwk} bemühte sich, den \emph{Schleier der Beatrice}\pwindex{Schleier der Beatrice. Schauspiel in fuenf Akten@\emph{Der Schleier der Beatrice. Schauspiel in fünf Akten}|pwk} an das \emph{Berliner Schauspielhaus}\orgindex{Schauspielhaus Berlin@Schauspielhaus Berlin|pwk} zu vermitteln.}}}\label{K_L02923-3} ſchon
               Beſcheid?\pend
           
\pstart
           \textsc{Kerr\pwindex{Kerr, Alfred 25.12.1867 – 12.10.1948@\textsc{Kerr, Alfred} (25.12.1867 – 12.10.1948), \emph{Schriftsteller/Schriftstellerin, Kritiker/Kritikerin}|pw}} dürfte Mitte Auguſt auch mithalten.\pend
           
\pstart
           Viele treue Grüße! {\\[\baselineskip]}Dein {\\[\baselineskip]}\spacefill\mbox{Paul Goldmann.}\pend
           \leftskip=0em{}\selectlanguage{ngerman}\endnumbering\briefempfaengerindex{Schnitzler, Arthur@\textsc{Schnitzler, Arthur}!zzzGoldmann, Paul@\emph{von Paul Goldmann}!1900-07-051@{5. 7. {[}1900{]}}|)be}\mylabel{L02923h}  \normalsize

\doendnotes{C}
\bigskip
\vfill

\clearpage

\footnotesize

\lohead{\textsc{register}}

% Definiere theindex-Environment komplett neu ohne reledmac
\makeatletter
\renewenvironment{theindex}{%
  \section*{\indexname}%
  \setlength{\parindent}{0pt}%
  \setlength{\parskip}{0pt plus 0.3pt}%
  \let\item\@idxitem
}{%
  \clearpage
}
\makeatother

\IfFileExists{\jobname-pw.ind}{\input{\jobname-pw.ind}}{}

\end{document}

      