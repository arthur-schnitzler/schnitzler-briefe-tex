%% latex-leseansicht-vorspann.tex
%% Vorspann für die Leseansicht.
%% Lädt die gemeinsame Datei latex-vorspann.tex mit nicht gesetztem Schalter.

\newif\ifkorrekturansicht
\korrekturansichtfalse

\input{../tex-inputs/latex-vorspann}


\section[Berta Zuckerkandl an Arthur Schnitzler, {{[}}zwischen 11. und 13. 6. 1911?{{]}}]{L03996 Berta Zuckerkandl an Arthur Schnitzler, {[}zwischen 11. und 13. 6. 1911?{]}}
\nopagebreak\mylabel{L03996v}
\rehead{ }\normalsize\beginnumbering\briefempfaengerindex{Schnitzler, Arthur@\textsc{Schnitzler, Arthur}!zzzZuckerkandl, Berta@\emph{von Berta Zuckerkandl}!1911-06-131@{{[}zwischen 11. und 13. 6. 1911?{]}}|(be}
\toendnotes[C]{\smallbreak\pagebreak[2]}
\correspDesc{Versand  durch Berta Zuckerkandl im Zeitraum [zwischen 11. und
                  13. 6. 1911?] in Wien
\newline{}Erhalt  durch Arthur Schnitzler in Wien}\toendnotes[C]{\smallbreak}
\Standort{CUL, Schnitzler, B 200.}
\physDesc{Brief, 1 Blatt, 3 Seiten, 641 Zeichen
\newline{}Handschrift: schwarze Tinte, lateinische Kurrent
\newline{}Schnitzler: mit Bleistift beschriftet: »Zucker« }\toendnotes[C]{\smallbreak}
\pstart{}{\pb}Hochverehrter Herr Doktor!\pend\vspace{0.5em}
\pstart
           Ich war sehr leidend und arbeitsunfähig. Auch hatte ich nichts Weiteres aus Paris\oindex{Paris@\textbf{Paris}, \emph{Hauptstadt}|pw} gehört. Nun erhalte ich eben einen \label{K_L03996-2v}\edtext{Brief
              meiner Schwester\pwindex{Clemenceau, Sophie 25.\,5.\,1862 – 24.\,9.\,1937@\textsc{Clemenceau, Sophie} (25.\,5.\,1862 – 24.\,9.\,1937)|pwv}}{\lemma{\textnormal{\emph{Brief
              meiner Schwester}}}\Cendnote{\textnormal{nicht überliefert}}}\label{K_L03996-2} den ich
               Ihnen vorlesen muss — da ich Samstag{ }Früh nach Paris\oindex{Paris@\textbf{Paris}, \emph{Hauptstadt}|pw} reise. {\pb}Und zwar wäre es dringend, dass ich Sie
                  heute spräche – denn ich soll ein französisches\oindex{Frankreich@\textbf{Frankreich}|pw} Scenarium des »Medardus\pwindex{Schnitzler, Arthur 15.\,5.\,1862 Wien – 21.\,10.\,1931 ebd.@\textsc{Schnitzler, Arthur} (15.\,5.\,1862 Wien – 21.\,10.\,1931 ebd.), \emph{Schriftsteller, Mediziner}!junge Medardus. Dramatische Historie in einem Vorspiel und fünf Aufzügen@\strich\emph{Der junge Medardus. Dramatische Historie in einem Vorspiel und fünf Aufzügen}|pw}«
               machen – eine grosse Arbeit – für welche ich Ihre Hilfe brauchte. – Wann könnte ich
               Sie nun \label{K_L03996-1v}\edtext{heute}{\lemma{\textnormal{\emph{heute}}}\Cendnote{\textnormal{Am 13. 6. 1911 notiert Schnitzler im \emph{Tagebuch}\pwindex{Schnitzler, Arthur 15.\,5.\,1862 Wien – 21.\,10.\,1931 ebd.@\textsc{Schnitzler, Arthur} (15.\,5.\,1862 Wien – 21.\,10.\,1931 ebd.), \emph{Schriftsteller, Mediziner}!Tagebuch@\strich\emph{Tagebuch}|pwk}
                  für den Nachmittag: »Frau Hofr. Zuckerkandl\pwindex{Zuckerkandl, Berta 13.\,4.\,1864 Wien – 16.\,10.\,1945 Paris@\textsc{Zuckerkandl, Berta} (13.\,4.\,1864 Wien – 16.\,10.\,1945 Paris), \emph{Schriftstellerin, Journalistin, Übersetzerin}|pw}; in Sachen Medardus\pwindex{Schnitzler, Arthur 15.\,5.\,1862 Wien – 21.\,10.\,1931 ebd.@\textsc{Schnitzler, Arthur} (15.\,5.\,1862 Wien – 21.\,10.\,1931 ebd.), \emph{Schriftsteller, Mediziner}!junge Medardus. Dramatische Historie in einem Vorspiel und fünf Aufzügen@\strich\emph{Der junge Medardus. Dramatische Historie in einem Vorspiel und fünf Aufzügen}|pw}
                     für Paris\oindex{Paris@\textbf{Paris}, \emph{Hauptstadt}|pw}.« Vorliegender nicht
                  datierter Brief, der um ein solches Treffen, wenn möglich noch am Nachmittag des
                  Tages der Abfassung bittet, ist demnach auf diesen Tag oder einen der direkt
                  vorangegangenen zu datieren, nicht jedoch vor Sonntag, dem
                     10. 6. 1911, denn Zuckerkandl erwähnt ihre Abreise »Samstag Früh«,
                  sodass der Zeitraum nicht mehr als sieben Tage vor einem Samstag umfasst.}}}\label{K_L03996-1} länger
               {\pb}sprechen. Ausser von halb
                  fünf – bis halbsechs Uhr wo ich bereits von meinem Arzt\pwindex{?? [Arzt von Berta Zuckerkandl] @\textsc{?? [Arzt von Berta Zuckerkandl]}|pwv} erwartet werde – bin
               ich ganz frei.\pend
           
\pstart
           Bitte gütigst Antwort. Geht es heute nicht – vielleicht dann
                  morgen{ }Vormittag?\pend
           \pstart Beste Grüsse \spacefill\mbox{B. Zuckerkandl}\pend{}\selectlanguage{ngerman}\endnumbering\briefempfaengerindex{Schnitzler, Arthur@\textsc{Schnitzler, Arthur}!zzzZuckerkandl, Berta@\emph{von Berta Zuckerkandl}!1911-06-111@{{[}zwischen 11. und 13. 6. 1911?{]}}|)be}\mylabel{L03996h}
\begin{anhang}
\end{anhang}\newcommand{\dateiname}{L03996}\newcommand{\titel}{Berta Zuckerkandl an Arthur Schnitzler, [zwischen 11. und 13. 6. 1911?]}\newcommand{\editorInnen}{Herausgegeben von Jahnke, SelmaMüller, Martin Anton}%% latex-leseansicht-abspann.tex
%% Abspann für die Leseansicht.
%% Der Schalter \ifkorrekturansicht ist bereits durch den Vorspann gesetzt.

%% latex-abspann.tex
%% Gemeinsamer Abspann für Korrekturansicht und Leseansicht.
%% Setzt den Schalter \ifkorrekturansicht voraus (gesetzt in den
%% einbindenden Dateien latex-korrekturansicht-abspann.tex bzw.
%% latex-leseansicht-abspann.tex).
%% ---------------------------------------------------------------

\normalsize

% Das esempio-Environment wird nur in der Leseansicht benötigt
\ifkorrekturansicht\else
\newenvironment{esempio}[3]%
{
    \vspace{1.5ex}
    \rlap{\underline{#1}}
    \par
    \setlength{\parindent}{0cm}
    \nopagebreak
    \leftskip=#2cm
    \rightskip=#3cm
}
{
    \par
}
\fi

\doendnotes{C}
\bigskip
\vfill

\clearpage

\footnotesize

\ifkorrekturansicht
  \lohead{\textsc{register}}
\fi

% theindex-Environment neu definieren ohne reledmac
\makeatletter
\renewenvironment{theindex}{%
  \ifkorrekturansicht
    \section*{\indexname}%
  \else
    \subsubsection*{Index der erwähnten Entitäten}%
  \fi
  \setlength{\parindent}{0pt}%
  \setlength{\parskip}{0pt plus 0.3pt}%
  \let\item\@idxitem
}{%
  \ifkorrekturansicht\clearpage\fi
}
\makeatother

\IfFileExists{\jobname-pw.ind}{\input{\jobname-pw.ind}}{}

% Quellenangabe nur in der Leseansicht
\ifkorrekturansicht\else
% Fallback-Definitionen, falls die .tex-Datei \titel etc. nicht gesetzt hat
\providecommand{\titel}{}
\providecommand{\editorInnen}{}
\providecommand{\dateiname}{\jobname}

\vspace{3cm}

\vfill

\footnotesize
\textsc{Quelle}: \titel. Herausgegeben von {\editorInnen}. In: \emph{Arthur Schnitzler: Briefwechsel mit Autorinnen und Autoren}.
 Digitale Edition, https://schnitzler-briefe.acdh.oeaw.ac.at/{\dateiname}.html (Stand \today)
\fi

\end{document}


