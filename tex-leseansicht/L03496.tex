%% latex-leseansicht-vorspann.tex
%% Vorspann für die Leseansicht.
%% Lädt die gemeinsame Datei latex-vorspann.tex mit nicht gesetztem Schalter.

\newif\ifkorrekturansicht
\korrekturansichtfalse

\input{../tex-inputs/latex-vorspann}


\section[Felix Salten an Arthur Schnitzler, {{[}}zwischen 19. und 21. 4. 1908{{]}}]{L03496 Felix Salten an Arthur Schnitzler, {[}zwischen 19. und 21. 4. 1908{]}}
\nopagebreak\mylabel{L03496v}
\rehead{ }\normalsize\beginnumbering\briefempfaengerindex{Schnitzler, Arthur@\textsc{Schnitzler, Arthur}!zzzSalten, Felix@\emph{von Felix Salten}!1908-04-211@{{[}zwischen 19. und 21. 4. 1908{]}}|(be}
\toendnotes[C]{\smallbreak\pagebreak[2]}
\correspDesc{Versand  durch Felix Salten im Zeitraum [zwischen 19. und 21. 4. 1908] in Wien
\newline{}Erhalt  durch Arthur Schnitzler im Zeitraum [zwischen 19. und 22. 4. 1908] in Wien}\toendnotes[C]{\smallbreak}
\Standort{CUL, Schnitzler, B 89, B 1.}
\physDesc{Brief, 1 Blatt, 1 Seite, 255 Zeichen
\newline{}Handschrift: Bleistift, lateinische Kurrent
\newline{}Schnitzler: mit Bleistift datiert: »Ende April \textcolor{gray}{08}« und Vermerk »\textsc{Salten}« 
\newline{}Ordnung: mit Bleistift von unbekannter Hand nummeriert: »245?« }\toendnotes[C]{\smallbreak}
\pstart{}{\pb}Lieber,\pend\vspace{0.5em}
\pstart
           bitte geben Sie dem Boten das \label{K_L03496-1v}\edtext{dalmatinische\oindex{Dalmatien@\textbf{Dalmatien}, \emph{Ehemalige Region}|pw} Buch\pwindex{Petermann, Reinhard Ernst 21.\,1.\,1859 Bruntál – 26.\,2.\,1932 Wien@\textsc{Petermann, Reinhard Ernst} (21.\,1.\,1859 Bruntál – 26.\,2.\,1932 Wien), \emph{Schriftsteller, Journalist, Versicherungsangestellte}!Führer durch Dalmatien@\strich\emph{Führer durch Dalmatien}|pwu}}{\lemma{\textnormal{\emph{dalmatinische Buch}}}\Cendnote{\textnormal{Das angesprochene Werk kann nicht verlässlich identifiziert werden. Eventuell
                  handelt es sich um \emph{Führer durch Dalmatien}\pwindex{Petermann, Reinhard Ernst 21.\,1.\,1859 Bruntál – 26.\,2.\,1932 Wien@\textsc{Petermann, Reinhard Ernst} (21.\,1.\,1859 Bruntál – 26.\,2.\,1932 Wien), \emph{Schriftsteller, Journalist, Versicherungsangestellte}!Führer durch Dalmatien@\strich\emph{Führer durch Dalmatien}|pwk} von Reinhard E.
                     Petermann\pwindex{Petermann, Reinhard Ernst 21.\,1.\,1859 Bruntál – 26.\,2.\,1932 Wien@\textsc{Petermann, Reinhard Ernst} (21.\,1.\,1859 Bruntál – 26.\,2.\,1932 Wien), \emph{Schriftsteller, Journalist, Versicherungsangestellte}|pwk}
                  (1899) für die bevorstehende Reise.}}}\label{K_L03496-1} und seien Sie bestens dafür
               bedankt. Die »\label{K_L03496-2v}\edtext{Komtesse Mizzi\pwindex{Schnitzler, Arthur 15.\,5.\,1862 Wien – 21.\,10.\,1931 ebd.@\textsc{Schnitzler, Arthur} (15.\,5.\,1862 Wien – 21.\,10.\,1931 ebd.), \emph{Schriftsteller, Mediziner}!Komtesse Mizzi oder: Der Familientag@\strich\emph{Komtesse Mizzi oder: Der Familientag}|pw}}{\lemma{\textnormal{\emph{Komtesse Mizzi}}}\Cendnote{\textnormal{Arthur Schnitzler: \emph{Komtesse Mizzi oder: Der Familientag}\pwindex{Schnitzler, Arthur 15.\,5.\,1862 Wien – 21.\,10.\,1931 ebd.@\textsc{Schnitzler, Arthur} (15.\,5.\,1862 Wien – 21.\,10.\,1931 ebd.), \emph{Schriftsteller, Mediziner}!Komtesse Mizzi oder: Der Familientag@\strich\emph{Komtesse Mizzi oder: Der Familientag}|pwk}. In: \emph{Neue Freie Presse}\pwindex{Neue Freie Presse@\emph{Neue Freie Presse}|pwk}, Nr. 15.684, 19. 4. 1908, Osterbeilage, S. 31–35. Durch
                  das Erscheinungsdatum kann Schnitzlers Datierung auf »Ende April \textcolor{gray}{08}« eingeschränkt werden. Nach hinten lässt sich ebenfalls eine zeitliche Einschränkung treffen. 
                  Saltens\pwindex{Salten, Felix 6.\,9.\,1869 Budapest – 8.\,10.\,1945 Zürich@\textsc{Salten, Felix} (6.\,9.\,1869 Budapest – 8.\,10.\,1945 Zürich), \emph{Schriftsteller, Journalist, Chefredakteur}|pwk} erstes Feuilleton von der Reise ist mit »Venedig\oindex{Venedig@\textbf{Venedig}|pw}, 23. April«
                  datiert (Felix Salten\pwindex{Salten, Felix 6.\,9.\,1869 Budapest – 8.\,10.\,1945 Zürich@\textsc{Salten, Felix} (6.\,9.\,1869 Budapest – 8.\,10.\,1945 Zürich), \emph{Schriftsteller, Journalist, Chefredakteur}|pwk}: \emph{Unsichere Reise}\pwindex{Salten, Felix 6.\,9.\,1869 Budapest – 8.\,10.\,1945 Zürich@\textsc{Salten, Felix} (6.\,9.\,1869 Budapest – 8.\,10.\,1945 Zürich), \emph{Schriftsteller, Journalist, Chefredakteur}!Unsichere Reise@\strich\emph{Unsichere Reise}|pwk}. In: \emph{Die Zeit}\pwindex{Zeit@\emph{Die Zeit}|pwk}, 
                     Jg. 7, Nr. 2008, 26. 4. 1908, Morgenblatt, S. 1–3, hier 3). Geschildert wird, dass der Erzähler/Salten\pwindex{Salten, Felix 6.\,9.\,1869 Budapest – 8.\,10.\,1945 Zürich@\textsc{Salten, Felix} (6.\,9.\,1869 Budapest – 8.\,10.\,1945 Zürich), \emph{Schriftsteller, Journalist, Chefredakteur}|pwk} von Triest\oindex{Triest@\textbf{Triest}, \emph{Verwaltungsgebiet}|pwk} aus eine Schifffahrt entlang der dalmatischen Küste\oindex{Dalmatien@\textbf{Dalmatien}, \emph{Ehemalige Region}|pwkv}
                  unternehmen wollte, aber das Schiff bereits in Pula\oindex{Pula@\textbf{Pula}, \emph{Hauptstadt}|pwk} verlassen hat. Er fuhr mit dem Zug zurück
                  nach Triest\oindex{Triest@\textbf{Triest}, \emph{Verwaltungsgebiet}|pwk}, wo er in der Nacht den Dampfer nach Venedig\oindex{Venedig@\textbf{Venedig}|pwk} bestieg. Sofern die Reise akkurat 
                  beschrieben ist, müsste er spätestens am 21. 4. 1908 in Wien\oindex{Wien@\textbf{Wien}, \emph{Verwaltungsgebiet}|pwk} den Nachtzug bestiegen
                  haben.}}}\label{K_L03496-2}«, die ich eben las, ist reizend. Die andere \label{K_L03496-3v}\edtext{Geschichte\pwindex{Schnitzler, Arthur 15.\,5.\,1862 Wien – 21.\,10.\,1931 ebd.@\textsc{Schnitzler, Arthur} (15.\,5.\,1862 Wien – 21.\,10.\,1931 ebd.), \emph{Schriftsteller, Mediziner}!Tod des Junggesellen. Novelle@\strich\emph{Der Tod des Junggesellen. Novelle}|pwv}
                in der »Zeit\pwindex{Österreichische Rundschau@\emph{Österreichische Rundschau}|pw}\pwindex{Zeit. Wiener Wochenschrift@\emph{Die Zeit. Wiener Wochenschrift}|pw}}{\lemma{\textnormal{\emph{Geschichte
                in der »Zeit}}}\Cendnote{\textnormal{Arthur Schnitzler: \emph{Der Tod des Junggesellen. Novelle}\pwindex{Schnitzler, Arthur 15.\,5.\,1862 Wien – 21.\,10.\,1931 ebd.@\textsc{Schnitzler, Arthur} (15.\,5.\,1862 Wien – 21.\,10.\,1931 ebd.), \emph{Schriftsteller, Mediziner}!Tod des Junggesellen. Novelle@\strich\emph{Der Tod des Junggesellen. Novelle}|pwk}. In: \emph{Österreichische Rundschau}\pwindex{Österreichische Rundschau@\emph{Österreichische Rundschau}|pwk}, Bd. 15, 1. 4. 1908, S. 19–26. Die \emph{Österreichische Rundschau}\pwindex{Österreichische Rundschau@\emph{Österreichische Rundschau}|pwk} galt als Nachfolger der Wochenschrift \emph{Die Zeit}\pwindex{Zeit. Wiener Wochenschrift@\emph{Die Zeit. Wiener Wochenschrift}|pwk}.}}}\label{K_L03496-3}« nehm’ ich mir auf die Reise
               mit.\pend
           
\pstart
           Viele herzliche Grüße von uns\pwindex{Salten, Ottilie 7.\,3.\,1868 Prag – 22.\,6.\,1942 Zürich@\textsc{Salten, Ottilie} (7.\,3.\,1868 Prag – 22.\,6.\,1942 Zürich), \emph{Schauspielerin}|pwv} zu Ihnen {\\[\baselineskip]}Ihr \spacefill\mbox{Salten}\pend
           \leftskip=0em{}\selectlanguage{ngerman}\endnumbering\briefempfaengerindex{Schnitzler, Arthur@\textsc{Schnitzler, Arthur}!zzzSalten, Felix@\emph{von Felix Salten}!1908-04-191@{{[}zwischen 19. und 21. 4. 1908{]}}|)be}\mylabel{L03496h}  \newcommand{\dateiname}{L03496}\newcommand{\titel}{Felix Salten an Arthur Schnitzler, [zwischen 19. und 21. 4. 1908]}\newcommand{\editorInnen}{Martin Anton Müller und Laura Untner}%% latex-leseansicht-abspann.tex
%% Abspann für die Leseansicht.
%% Der Schalter \ifkorrekturansicht ist bereits durch den Vorspann gesetzt.

%% latex-abspann.tex
%% Gemeinsamer Abspann für Korrekturansicht und Leseansicht.
%% Setzt den Schalter \ifkorrekturansicht voraus (gesetzt in den
%% einbindenden Dateien latex-korrekturansicht-abspann.tex bzw.
%% latex-leseansicht-abspann.tex).
%% ---------------------------------------------------------------

\normalsize

% Das esempio-Environment wird nur in der Leseansicht benötigt
\ifkorrekturansicht\else
\newenvironment{esempio}[3]%
{
    \vspace{1.5ex}
    \rlap{\underline{#1}}
    \par
    \setlength{\parindent}{0cm}
    \nopagebreak
    \leftskip=#2cm
    \rightskip=#3cm
}
{
    \par
}
\fi

\doendnotes{C}
\bigskip
\vfill

\clearpage

\footnotesize

\ifkorrekturansicht
  \lohead{\textsc{register}}
\fi

% theindex-Environment neu definieren ohne reledmac
\makeatletter
\renewenvironment{theindex}{%
  \ifkorrekturansicht
    \section*{\indexname}%
  \else
    \subsubsection*{Index der erwähnten Entitäten}%
  \fi
  \setlength{\parindent}{0pt}%
  \setlength{\parskip}{0pt plus 0.3pt}%
  \let\item\@idxitem
}{%
  \ifkorrekturansicht\clearpage\fi
}
\makeatother

\IfFileExists{\jobname-pw.ind}{\input{\jobname-pw.ind}}{}

% Quellenangabe nur in der Leseansicht
\ifkorrekturansicht\else
% Fallback-Definitionen, falls die .tex-Datei \titel etc. nicht gesetzt hat
\providecommand{\titel}{}
\providecommand{\editorInnen}{}
\providecommand{\dateiname}{\jobname}

\vspace{3cm}

\vfill

\footnotesize
\textsc{Quelle}: \titel. Herausgegeben von {\editorInnen}. In: \emph{Arthur Schnitzler: Briefwechsel mit Autorinnen und Autoren}.
 Digitale Edition, https://schnitzler-briefe.acdh.oeaw.ac.at/{\dateiname}.html (Stand \today)
\fi

\end{document}


