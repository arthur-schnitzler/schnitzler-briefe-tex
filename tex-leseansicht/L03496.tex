%% latex-korrekturansicht-vorspann.tex
%% Vorspann für die Korrekturansicht.
%% Lädt die gemeinsame Datei latex-vorspann.tex mit gesetztem Schalter.

\newif\ifkorrekturansicht
\korrekturansichttrue

\input{../tex-inputs/latex-vorspann}


\section[Felix Salten an Arthur Schnitzler, {[}zwischen 19. und 21. 4. 1908{]}]{L03496 Felix Salten an Arthur Schnitzler,
               {[}zwischen 19. und 21. 4. 1908{]}}
\nopagebreak\mylabel{L03496v}
\rehead{ }\normalsize\beginnumbering\briefempfaengerindex{Schnitzler, Arthur@\textsc{Schnitzler, Arthur}!zzzSalten, Felix@\emph{von Felix Salten}!1908-04-211@{{[}zwischen 19. und 21. 4. 1908{]}}|(be}
\toendnotes[C]{\smallbreak\pagebreak[2]}\Standort{CUL, Schnitzler, B 89, B 1.}
\physDesc{Brief, 1 Blatt, 1 Seite, 255 Zeichen
\newline{}Handschrift: Bleistift, lateinische Kurrent
\newline{}Schnitzler: mit Bleistift datiert: »Ende April \textcolor{gray}{08}« und Vermerk »\textsc{Salten}« 
\newline{}Ordnung: mit Bleistift von unbekannter Hand nummeriert: »245?« }\toendnotes[C]{\smallbreak}
\pstart{}{\pb}Lieber,\pend\vspace{0.5em}
\pstart
           bitte geben Sie dem Boten das \label{K_L03496-1v}\edtext{dalmatinische\oindex{Dalmatien@\textbf{Dalmatien}, \emph{L.RGNH}|pw} Buch\pwindex{Fuehrer durch Dalmatien@\emph{Führer durch Dalmatien}|pwu}}{\lemma{\textnormal{\emph{dalmatinische Buch}}}\Cendnote{\textnormal{Das angesprochene Werk kann nicht verlässlich identifiziert werden. Eventuell
                  handelt es sich um \emph{Führer durch Dalmatien}\pwindex{Fuehrer durch Dalmatien@\emph{Führer durch Dalmatien}|pwk} von Reinhard E.
                     Petermann\pwindex{Petermann, Reinhard Ernst 1859-01-21 – 1932-02-26@\textsc{Petermann, Reinhard Ernst} (1859-01-21 – 1932-02-26), \emph{Schriftsteller/Schriftstellerin, Journalist/Journalistin, Versicherungsangestellte/Versicherungsangestellte}|pwk}
                  (1899) für die bevorstehende Reise.}}}\label{K_L03496-1} und seien Sie bestens dafür
               bedankt. Die »\label{K_L03496-2v}\edtext{Komtesse Mizzi\pwindex{Komtesse Mizzi oder: Der Familientag@\emph{Komtesse Mizzi oder: Der Familientag}|pw}}{\lemma{\textnormal{\emph{Komtesse Mizzi}}}\Cendnote{\textnormal{Arthur Schnitzler: \emph{Komtesse Mizzi oder: Der Familientag}\pwindex{Komtesse Mizzi oder: Der Familientag@\emph{Komtesse Mizzi oder: Der Familientag}|pwk}. In: \emph{Neue Freie Presse}\pwindex{Neue Freie Presse@\emph{Neue Freie Presse}|pwk}, Nr. 15.684, 19. 4. 1908, Osterbeilage, S. 31–35. Durch
                  das Erscheinungsdatum kann Schnitzlers Datierung auf »Ende April \textcolor{gray}{08}« eingeschränkt werden. Nach hinten lässt sich ebenfalls eine zeitliche Einschränkung treffen. 
                  Saltens\pwindex{Salten, Felix 06.09.1869 – 08.10.1945@\textsc{Salten, Felix} (06.09.1869 – 08.10.1945), \emph{Schriftsteller/Schriftstellerin, Journalist/Journalistin, Chefredakteur/Chefredakteurin}|pwk} erstes Feuilleton von der Reise ist mit »Venedig\oindex{Venedig@\textbf{Venedig}, \emph{P.PPLA}|pw}, 23. April«
                  datiert (Felix Salten\pwindex{Salten, Felix 06.09.1869 – 08.10.1945@\textsc{Salten, Felix} (06.09.1869 – 08.10.1945), \emph{Schriftsteller/Schriftstellerin, Journalist/Journalistin, Chefredakteur/Chefredakteurin}|pwk}: \emph{Unsichere Reise}\pwindex{Unsichere Reise@\emph{Unsichere Reise}|pwk}. In: \emph{Die Zeit}\pwindex{Zeit@\emph{Die Zeit}|pwk}, 
                     Jg. 7, Nr. 2008, 26. 4. 1908, Morgenblatt, S. 1–3, hier 3). Geschildert wird, dass der Erzähler/Salten\pwindex{Salten, Felix 06.09.1869 – 08.10.1945@\textsc{Salten, Felix} (06.09.1869 – 08.10.1945), \emph{Schriftsteller/Schriftstellerin, Journalist/Journalistin, Chefredakteur/Chefredakteurin}|pwk} von Triest\oindex{Triest@\textbf{Triest}, \emph{A.ADM3}|pwk} aus eine Schifffahrt entlang der dalmatischen Küste\oindex{Dalmatien@\textbf{Dalmatien}, \emph{L.RGNH}|pwkv}
                  unternehmen wollte, aber das Schiff bereits in Pula\oindex{Pula@\textbf{Pula}, \emph{P.PPLA2}|pwk} verlassen hat. Er fuhr mit dem Zug zurück
                  nach Triest\oindex{Triest@\textbf{Triest}, \emph{A.ADM3}|pwk}, wo er in der Nacht den Dampfer nach Venedig\oindex{Venedig@\textbf{Venedig}, \emph{P.PPLA}|pwk} bestieg. Sofern die Reise akkurat 
                  beschrieben ist, müsste er spätestens am 21. 4. 1908 in Wien\oindex{Wien@\textbf{Wien}, \emph{A.ADM2}|pwk} den Nachtzug bestiegen
                  haben.}}}\label{K_L03496-2}«, die ich eben las, ist reizend. Die andere \label{K_L03496-3v}\edtext{Geschichte\pwindex{Tod des Junggesellen. Novelle@\emph{Der Tod des Junggesellen. Novelle}|pwv}
                in der »Zeit\pwindex{Oesterreichische Rundschau@\emph{Österreichische Rundschau}|pw}\pwindex{Zeit. Wiener Wochenschrift@\emph{Die Zeit. Wiener Wochenschrift}|pw}}{\lemma{\textnormal{\emph{Geschichte
                in der »Zeit}}}\Cendnote{\textnormal{Arthur Schnitzler: \emph{Der Tod des Junggesellen. Novelle}\pwindex{Tod des Junggesellen. Novelle@\emph{Der Tod des Junggesellen. Novelle}|pwk}. In: \emph{Österreichische Rundschau}\pwindex{Oesterreichische Rundschau@\emph{Österreichische Rundschau}|pwk}, Bd. 15, 1. 4. 1908, S. 19–26. Die \emph{Österreichische Rundschau}\pwindex{Oesterreichische Rundschau@\emph{Österreichische Rundschau}|pwk} galt als Nachfolger der Wochenschrift \emph{Die Zeit}\pwindex{Zeit. Wiener Wochenschrift@\emph{Die Zeit. Wiener Wochenschrift}|pwk}.}}}\label{K_L03496-3}« nehm’ ich mir auf die Reise
               mit.\pend
           
\pstart
           Viele herzliche Grüße von uns\pwindex{Salten, Ottilie 07.03.1868 – 22.06.1942@\textsc{Salten, Ottilie} (07.03.1868 – 22.06.1942), \emph{Schauspieler/Schauspielerin}|pwv} zu Ihnen {\\[\baselineskip]}Ihr \spacefill\mbox{Salten}\pend
           \leftskip=0em{}\selectlanguage{ngerman}\endnumbering\briefempfaengerindex{Schnitzler, Arthur@\textsc{Schnitzler, Arthur}!zzzSalten, Felix@\emph{von Felix Salten}!1908-04-191@{{[}zwischen 19. und 21. 4. 1908{]}}|)be}\mylabel{L03496h}  \normalsize

\doendnotes{C}
\bigskip
\vfill

\clearpage

\footnotesize

\lohead{\textsc{register}}

% Definiere theindex-Environment komplett neu ohne reledmac
\makeatletter
\renewenvironment{theindex}{%
  \section*{\indexname}%
  \setlength{\parindent}{0pt}%
  \setlength{\parskip}{0pt plus 0.3pt}%
  \let\item\@idxitem
}{%
  \clearpage
}
\makeatother

\IfFileExists{\jobname-pw.ind}{\input{\jobname-pw.ind}}{}

\end{document}

      