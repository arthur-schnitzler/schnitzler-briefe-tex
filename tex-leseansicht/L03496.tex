%% latex-leseansicht-vorspann.tex
%% Vorspann für die Leseansicht.
%% Lädt die gemeinsame Datei latex-vorspann.tex mit nicht gesetztem Schalter.

\newif\ifkorrekturansicht
\korrekturansichtfalse

\input{../tex-inputs/latex-vorspann}


         
         \renewcommand{\erwaehntePersonen}{Personen: Reinhard Ernst Petermann, Felix Salten, Ottilie Salten}
         \renewcommand{\erwaehnteOrte}{Orte: Dalmatien, Pula, Triest, Venedig, Wien}
         \renewcommand{\erwaehnteWerke}{Werke: Der Tod des Junggesellen. Novelle, Die Zeit, Die Zeit. Wiener Wochenschrift, Führer durch Dalmatien, Komtesse Mizzi oder: Der Familientag, Neue Freie Presse, Unsichere Reise, Österreichische Rundschau}
               \section[Felix Salten an Arthur Schnitzler, {[}zwischen 19. und 21. 4. 1908{]}]{ Felix Salten an Arthur Schnitzler,
               {[}zwischen 19. und 21. 4. 1908{]}}\nopagebreak\mylabel{v}\rehead{ }\begin{ledgroupsized}[t]{13cm}\normalsize\beginnumbering\briefempfaengerindex{Schnitzler, Arthur@\textsc{Schnitzler, Arthur}!zzzSalten, Felix@\emph{von Felix Salten}!1908-04-211@{{[}zwischen 19. und 21. 4. 1908{]}}|(be} \toendnotes[C]{\smallbreak\pagebreak[2]} \Standort{CUL, Schnitzler, B 89, B 1.}
\physDesc{Brief, 1 Blatt, 1 Seite, 255 Zeichen
\newline{}Handschrift: Bleistift, lateinische Kurrent
\newline{}Schnitzler: mit Bleistift datiert: »Ende April \textcolor{gray}{08}« und Vermerk »\textsc{Salten}« 
\newline{}Ordnung: mit Bleistift von unbekannter Hand nummeriert: »245?« }\toendnotes[C]{\smallbreak}\pstart{}{\pb}Lieber,\pend\pstart
           bitte geben Sie dem Boten das \label{K_L03496-1v}\edtext{dalmatinische\oindex{Dalmatien@\textbf{Dalmatien}|pw} Buch\pwindex{Petermann, Reinhard Ernst 1859-01-21 – 1932-02-26@\textsc{Petermann, Reinhard Ernst} (1859-01-21 – 1932-02-26), \emph{Schriftsteller, Journalist, Versicherungsangestellte}!Fuehrer durch Dalmatien1899@\strich\emph{Führer durch Dalmatien} {[}1899{]}|pwu}}{\lemma{\textnormal{\emph{dalmatinische Buch}}}\Cendnote{\textnormal{Das angesprochene Werk kann nicht verlässlich identifiziert werden. Eventuell
                  handelt es sich um \emph{Führer durch Dalmatien}\pwindex{Petermann, Reinhard Ernst 1859-01-21 – 1932-02-26@\textsc{Petermann, Reinhard Ernst} (1859-01-21 – 1932-02-26), \emph{Schriftsteller, Journalist, Versicherungsangestellte}!Fuehrer durch Dalmatien1899@\strich\emph{Führer durch Dalmatien} {[}1899{]}|pwk} von Reinhard E.
                     Petermann\pwindex{Petermann, Reinhard Ernst 1859-01-21 – 1932-02-26@\textsc{Petermann, Reinhard Ernst} (1859-01-21 – 1932-02-26), \emph{Schriftsteller, Journalist, Versicherungsangestellte}|pwk}
                  (1899) für die bevorstehende Reise.}}}\label{K_L03496-1h} und seien Sie bestens dafür
               bedankt. Die »\label{K_L03496-2v}\edtext{Komtesse Mizzi\pwindex{Schnitzler, Arthur 15.05.1862 – 21.10.1931@\textsc{Schnitzler, Arthur} (15.05.1862 – 21.10.1931), \emph{Schriftsteller, Mediziner}!Komtesse Mizzi oder: Der Familientag1908-04-19@\strich\emph{Komtesse Mizzi oder: Der Familientag} {[}1908-04-19{]}|pw}}{\lemma{\textnormal{\emph{Komtesse Mizzi}}}\Cendnote{\textnormal{Arthur Schnitzler\pwindex{Schnitzler, Arthur 15.05.1862 – 21.10.1931@\textsc{Schnitzler, Arthur} (15.05.1862 – 21.10.1931), \emph{Schriftsteller, Mediziner}|pwk}: \emph{Komtesse Mizzi oder: Der Familientag}\pwindex{Schnitzler, Arthur 15.05.1862 – 21.10.1931@\textsc{Schnitzler, Arthur} (15.05.1862 – 21.10.1931), \emph{Schriftsteller, Mediziner}!Komtesse Mizzi oder: Der Familientag1908-04-19@\strich\emph{Komtesse Mizzi oder: Der Familientag} {[}1908-04-19{]}|pwk}. In: \emph{Neue Freie Presse}\pwindex{Neue Freie Presse1864 – 1939@\emph{Neue Freie Presse} {[}1864 – 1939{]}|pwk}, Nr. 15.684, 19. 4. 1908, Osterbeilage, S. 31–35. Durch
                  das Erscheinungsdatum kann Schnitzlers\pwindex{Schnitzler, Arthur 15.05.1862 – 21.10.1931@\textsc{Schnitzler, Arthur} (15.05.1862 – 21.10.1931), \emph{Schriftsteller, Mediziner}|pwk} Datierung auf »Ende April \textcolor{gray}{08}« eingeschränkt werden. Nach hinten lässt sich ebenfalls eine zeitliche Einschränkung treffen. 
                  Saltens\pwindex{Salten, Felix 06.09.1869 – 08.10.1945@\textsc{Salten, Felix} (06.09.1869 – 08.10.1945), \emph{Schriftsteller, Journalist, Chefredakteur}|pwk} erstes Feuilleton von der Reise ist mit »Venedig\oindex{Venedig@\textbf{Venedig}|pw}, 23. April«
                  datiert (Felix Salten\pwindex{Salten, Felix 06.09.1869 – 08.10.1945@\textsc{Salten, Felix} (06.09.1869 – 08.10.1945), \emph{Schriftsteller, Journalist, Chefredakteur}|pwk}: \emph{Unsichere Reise}\pwindex{Salten, Felix 06.09.1869 – 08.10.1945@\textsc{Salten, Felix} (06.09.1869 – 08.10.1945), \emph{Schriftsteller, Journalist, Chefredakteur}!Unsichere Reise1908-04-26@\strich\emph{Unsichere Reise} {[}1908-04-26{]}|pwk}. In: \emph{Die Zeit}\pwindex{Zeit1902-09-27 – 1919@\emph{Die Zeit} {[}1902-09-27 – 1919{]}|pwk}, 
                     Jg. 7, Nr. 2008, 26. 4. 1908, Morgenblatt, S. 1–3, hier 3). Geschildert wird, dass der Erzähler/Salten\pwindex{Salten, Felix 06.09.1869 – 08.10.1945@\textsc{Salten, Felix} (06.09.1869 – 08.10.1945), \emph{Schriftsteller, Journalist, Chefredakteur}|pwk} von Triest\oindex{Triest@\textbf{Triest}|pwk} aus eine Schifffahrt entlang der dalmatischen Küste\oindex{Dalmatien@\textbf{Dalmatien}|pwkv}
                  unternehmen wollte, aber das Schiff bereits in Pula\oindex{Pula@\textbf{Pula}|pwk} verlassen hat. Er fuhr mit dem Zug zurück
                  nach Triest\oindex{Triest@\textbf{Triest}|pwk}, wo er in der Nacht den Dampfer nach Venedig\oindex{Venedig@\textbf{Venedig}|pwk} bestieg. Sofern die Reise akkurat 
                  beschrieben ist, müsste er spätestens am 21. 4. 1908 in Wien\oindex{Wien@\textbf{Wien}|pwk} den Nachtzug bestiegen
                  haben.}}}\label{K_L03496-2h}«, die ich eben las, ist reizend. Die andere \label{K_L03496-3v}\edtext{Geschichte\pwindex{Schnitzler, Arthur 15.05.1862 – 21.10.1931@\textsc{Schnitzler, Arthur} (15.05.1862 – 21.10.1931), \emph{Schriftsteller, Mediziner}!Tod des Junggesellen. Novelle1. 4. 1908@\strich\emph{Der Tod des Junggesellen. Novelle} {[}1. 4. 1908{]}|pwv}
                in der »Zeit\pwindex{Oesterreichische Rundschau1904 – 1924@\emph{Österreichische Rundschau} {[}1904 – 1924{]}|pw}\pwindex{Zeit. Wiener Wochenschrift1894 – 1904@\emph{Die Zeit. Wiener Wochenschrift} {[}1894 – 1904{]}|pw}}{\lemma{\textnormal{\emph{Geschichte
                in der »Zeit}}}\Cendnote{\textnormal{Arthur Schnitzler\pwindex{Schnitzler, Arthur 15.05.1862 – 21.10.1931@\textsc{Schnitzler, Arthur} (15.05.1862 – 21.10.1931), \emph{Schriftsteller, Mediziner}|pwk}: \emph{Der Tod des Junggesellen. Novelle}\pwindex{Schnitzler, Arthur 15.05.1862 – 21.10.1931@\textsc{Schnitzler, Arthur} (15.05.1862 – 21.10.1931), \emph{Schriftsteller, Mediziner}!Tod des Junggesellen. Novelle1. 4. 1908@\strich\emph{Der Tod des Junggesellen. Novelle} {[}1. 4. 1908{]}|pwk}. In: \emph{Österreichische Rundschau}\pwindex{Oesterreichische Rundschau1904 – 1924@\emph{Österreichische Rundschau} {[}1904 – 1924{]}|pwk}, Bd. 15, 1. 4. 1908, S. 19–26. Die \emph{Österreichische Rundschau}\pwindex{Oesterreichische Rundschau1904 – 1924@\emph{Österreichische Rundschau} {[}1904 – 1924{]}|pwk} galt als Nachfolger der Wochenschrift \emph{Die Zeit}\pwindex{Zeit. Wiener Wochenschrift1894 – 1904@\emph{Die Zeit. Wiener Wochenschrift} {[}1894 – 1904{]}|pwk}.}}}\label{K_L03496-3h}« nehm’ ich mir auf die Reise
               mit.\pend
           \pstart
           Viele herzliche Grüße von uns\pwindex{Salten, Ottilie 07.03.1868 – 22.06.1942@\textsc{Salten, Ottilie} (07.03.1868 – 22.06.1942), \emph{Schauspielerin}|pwv} zu Ihnen {\\[\baselineskip]}Ihr \spacefill\mbox{Salten}\pend
           \leftskip=0em{}
         
         \endnumbering\mylabel{h}\end{ledgroupsized}  \newcommand{\dateiname}{L03496}\newcommand{\titel}{Felix Salten an Arthur Schnitzler, [zwischen 19. und 21. 4. 1908]}\newcommand{\editorInnen}{Martin Anton Müller und Laura Untner}%% latex-leseansicht-abspann.tex
%% Abspann für die Leseansicht.
%% Der Schalter \ifkorrekturansicht ist bereits durch den Vorspann gesetzt.

%% latex-abspann.tex
%% Gemeinsamer Abspann für Korrekturansicht und Leseansicht.
%% Setzt den Schalter \ifkorrekturansicht voraus (gesetzt in den
%% einbindenden Dateien latex-korrekturansicht-abspann.tex bzw.
%% latex-leseansicht-abspann.tex).
%% ---------------------------------------------------------------

\normalsize

% Das esempio-Environment wird nur in der Leseansicht benötigt
\ifkorrekturansicht\else
\newenvironment{esempio}[3]%
{
    \vspace{1.5ex}
    \rlap{\underline{#1}}
    \par
    \setlength{\parindent}{0cm}
    \nopagebreak
    \leftskip=#2cm
    \rightskip=#3cm
}
{
    \par
}
\fi

\doendnotes{C}
\bigskip
\vfill

\clearpage

\footnotesize

\ifkorrekturansicht
  \lohead{\textsc{register}}
\fi

% theindex-Environment neu definieren ohne reledmac
\makeatletter
\renewenvironment{theindex}{%
  \ifkorrekturansicht
    \section*{\indexname}%
  \else
    \subsubsection*{Index der erwähnten Entitäten}%
  \fi
  \setlength{\parindent}{0pt}%
  \setlength{\parskip}{0pt plus 0.3pt}%
  \let\item\@idxitem
}{%
  \ifkorrekturansicht\clearpage\fi
}
\makeatother

\IfFileExists{\jobname-pw.ind}{\input{\jobname-pw.ind}}{}

% Quellenangabe nur in der Leseansicht
\ifkorrekturansicht\else
% Fallback-Definitionen, falls die .tex-Datei \titel etc. nicht gesetzt hat
\providecommand{\titel}{}
\providecommand{\editorInnen}{}
\providecommand{\dateiname}{\jobname}

\vspace{3cm}

\vfill

\footnotesize
\textsc{Quelle}: \titel. Herausgegeben von {\editorInnen}. In: \emph{Arthur Schnitzler: Briefwechsel mit Autorinnen und Autoren}.
 Digitale Edition, https://schnitzler-briefe.acdh.oeaw.ac.at/{\dateiname}.html (Stand \today)
\fi

\end{document}


      