%% latex-leseansicht-vorspann.tex
%% Vorspann für die Leseansicht.
%% Lädt die gemeinsame Datei latex-vorspann.tex mit nicht gesetztem Schalter.

\newif\ifkorrekturansicht
\korrekturansichtfalse

\input{../tex-inputs/latex-vorspann}

\begin{center}
            \textcolor{red}{ENTWURF, NICHT FERTIG KORRIGIERT}
                      \end{center}
            
         
         \renewcommand{\erwaehntePersonen}{Personen: Heinrich Kanner, Ernst Raupach, Isidor Singer}
         \renewcommand{\erwaehnteInstitutionen}{Institutionen: Die Zeit, Raimund-Theater}
         \renewcommand{\erwaehnteOrte}{Orte: Wien, Wipplingerstraße}
         \renewcommand{\erwaehnteWerke}{Werke: Der Müller und sein Kind. Volksdrama in fünf Aufzügen}
               \section[Felix Salten an Arthur Schnitzler, {[}zwischen 26. und 30. 10. 1903{]}]{ Felix Salten an Arthur Schnitzler, {[}zwischen 26. und
               30. 10. 1903{]}}\nopagebreak\mylabel{v}\rehead{ }\begin{ledgroupsized}[t]{13cm}\normalsize\beginnumbering \toendnotes[C]{\smallbreak\pagebreak[2]} \Standort{CUL, Schnitzler, B 89, A 2.}
\physDesc{Brief, 1 Blatt, 1 Seite, 88 Zeichen
\newline{}Handschrift: Bleistift, lateinische Kurrent
\newline{}Schnitzler: mit Bleistift datiert: »Oct 903« 
\newline{}Ordnung: mit Bleistift von unbekannter Hand nummeriert:
                                    »174« }\toendnotes[C]{\smallbreak}\pstart
           \noindent{}{\pb}\textcolor{gray}{\textbf{DIE}}\pend
           \pstart
           \textcolor{gray}{\textbf{ZEIT\orgindex{Zeit@Die Zeit|pw}}}\pend
           \pstart
           \textcolor{gray}{\textbf{Wien\oindex{Wien@\textbf{Wien}|pw}er Tageszeitung}}\hfill \textcolor{gray}{\textbf{WIEN\oindex{Wien@\textbf{Wien}|pw}}}\pend
           \pstart
           \textcolor{gray}{\textbf{Herausgeber: }}\hfill \textcolor{gray}{\textbf{I. Wipplingerstrasse 38\oindex{Wipplingerstrasse@\textbf{Wipplingerstraße}|pw}}}\pend
           \pstart
           \textcolor{gray}{\textbf{Prof. Dr. I. Singer\pwindex{Singer, Isidor 16.01.1857 – 08.12.1927@\textsc{Singer, Isidor} (16.01.1857 – 08.12.1927), \emph{Journalist, Herausgeber, Soziologe}|pw}}}\pend
           \pstart
           \textcolor{gray}{\textbf{Dr. Heinrich Kanner\pwindex{Kanner, Heinrich 09.11.1864 – 15.02.1930@\textsc{Kanner, Heinrich} (09.11.1864 – 15.02.1930), \emph{Herausgeber, Publizist}|pw}}}\pend
           \pstart
           \textcolor{gray}{\textbf{Redaction.}}\pend
           \pstart
           \textcolor{gray}{\textbf{Telegramm-Adresse: \so{Zeit}\orgindex{Zeit@Die Zeit|pw}\so{,{ }}\so{Wien}\oindex{Wien@\textbf{Wien}|pw}}}\pend
           \pstart
           \textcolor{gray}{\textbf{Interurbanes Telephon Nr. 15.988}}\pend
           \pstart
           \textcolor{gray}{\textbf{= Telephone Nr. 17.040, 17.041 =}}\pend
           \pstart
           Lieber,  wir kommen also (mit \label{K_L03348-1v}\edtext{Fourage}{\lemma{\textnormal{\emph{Fourage}}}\Cendnote{\textnormal{eigtl.
                  Pferdefutter, hier im Sinn von: mit Essen}}}\label{K_L03348-1h}) Sonntag nach dem
                  »\label{K_L03348-21v}\edtext{Müller\pwindex{Raupach, Ernst 1784-05-21 – 1852-03-18@\textsc{Raupach, Ernst} (1784-05-21 – 1852-03-18), \emph{Schriftsteller, Wissenschaftler, Dichter}!Mueller und sein Kind. Volksdrama in fuenf Aufzuegen1830@\strich\emph{Der Müller und sein Kind. Volksdrama in fünf Aufzügen} {[}1830{]}|pw}}{\lemma{\textnormal{\emph{Müller}}}\Cendnote{\textnormal{\emph{Der Müller und sein Kind. Volksdrama in fünf
                     Aufzügen}\pwindex{Raupach, Ernst 1784-05-21 – 1852-03-18@\textsc{Raupach, Ernst} (1784-05-21 – 1852-03-18), \emph{Schriftsteller, Wissenschaftler, Dichter}!Mueller und sein Kind. Volksdrama in fuenf Aufzuegen1830@\strich\emph{Der Müller und sein Kind. Volksdrama in fünf Aufzügen} {[}1830{]}|pwk} von Ernst Raupach\pwindex{Raupach, Ernst 1784-05-21 – 1852-03-18@\textsc{Raupach, Ernst} (1784-05-21 – 1852-03-18), \emph{Schriftsteller, Wissenschaftler, Dichter}|pwk} wurde am
                     1. 11. 1903 am \emph{Raimundtheater}\orgindex{Raimund-Theater@Raimund-Theater|pwk}
                  als Nachmittagsvorstellung (Beginnzeit halb 3 Uhr) gegeben. Das erlaubt die
                  Datierung des Korrespondenzstückes in die Woche vor diesem Sonntag. Zwar lief das
                  Stück auch am 25. 10. 1903 als Nachmittagsvorstellung, doch lässt
                  sich das nicht mit dem Schreiben vom [23./24.? 10. 1903] vereinbaren. Damit bleibt nur das Treffen am 1. 11. 1903 übrig, in
                  Vorbereitung dessen zuerst dieses und dann der Brief vom [zwischen 27. und
                  31. 10. 1903] gelaufen sein
                  dürften.}}}\label{K_L03348-21h}« zu Ihnen.\pend
           \pstart
           Herzlichst {\\[\baselineskip]}Ihr {\\[\baselineskip]}\spacefill\mbox{Salten}\pend
           \leftskip=0em{}
         
         \endnumbering\mylabel{h}\end{ledgroupsized}\begin{anhang}\end{anhang}\newcommand{\dateiname}{L03348}\newcommand{\titel}{Felix Salten an Arthur Schnitzler, [zwischen 26. und 30. 10. 1903]}\newcommand{\editorInnen}{Martin Anton Müller und Laura Untner}%% latex-leseansicht-abspann.tex
%% Abspann für die Leseansicht.
%% Der Schalter \ifkorrekturansicht ist bereits durch den Vorspann gesetzt.

%% latex-abspann.tex
%% Gemeinsamer Abspann für Korrekturansicht und Leseansicht.
%% Setzt den Schalter \ifkorrekturansicht voraus (gesetzt in den
%% einbindenden Dateien latex-korrekturansicht-abspann.tex bzw.
%% latex-leseansicht-abspann.tex).
%% ---------------------------------------------------------------

\normalsize

% Das esempio-Environment wird nur in der Leseansicht benötigt
\ifkorrekturansicht\else
\newenvironment{esempio}[3]%
{
    \vspace{1.5ex}
    \rlap{\underline{#1}}
    \par
    \setlength{\parindent}{0cm}
    \nopagebreak
    \leftskip=#2cm
    \rightskip=#3cm
}
{
    \par
}
\fi

\doendnotes{C}
\bigskip
\vfill

\clearpage

\footnotesize

\ifkorrekturansicht
  \lohead{\textsc{register}}
\fi

% theindex-Environment neu definieren ohne reledmac
\makeatletter
\renewenvironment{theindex}{%
  \ifkorrekturansicht
    \section*{\indexname}%
  \else
    \subsubsection*{Index der erwähnten Entitäten}%
  \fi
  \setlength{\parindent}{0pt}%
  \setlength{\parskip}{0pt plus 0.3pt}%
  \let\item\@idxitem
}{%
  \ifkorrekturansicht\clearpage\fi
}
\makeatother

\IfFileExists{\jobname-pw.ind}{\input{\jobname-pw.ind}}{}

% Quellenangabe nur in der Leseansicht
\ifkorrekturansicht\else
% Fallback-Definitionen, falls die .tex-Datei \titel etc. nicht gesetzt hat
\providecommand{\titel}{}
\providecommand{\editorInnen}{}
\providecommand{\dateiname}{\jobname}

\vspace{3cm}

\vfill

\footnotesize
\textsc{Quelle}: \titel. Herausgegeben von {\editorInnen}. In: \emph{Arthur Schnitzler: Briefwechsel mit Autorinnen und Autoren}.
 Digitale Edition, https://schnitzler-briefe.acdh.oeaw.ac.at/{\dateiname}.html (Stand \today)
\fi

\end{document}


      