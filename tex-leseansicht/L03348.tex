%% latex-leseansicht-vorspann.tex
%% Vorspann für die Leseansicht.
%% Lädt die gemeinsame Datei latex-vorspann.tex mit nicht gesetztem Schalter.

\newif\ifkorrekturansicht
\korrekturansichtfalse

\input{../tex-inputs/latex-vorspann}


\section[ Felix Salten an Arthur Schnitzler, {[}zwischen 26. und 30. 10. 1903{]}]{L03348 Felix Salten an Arthur
               Schnitzler,  [zwischen 26. und 30. 10. 1903]}
\nopagebreak\mylabel{L03348v}
\rehead{ }\normalsize\beginnumbering\briefempfaengerindex{Schnitzler, Arthur@\textsc{Schnitzler, Arthur}!zzzSalten, Felix@\emph{von Felix Salten}!1903-10-301@{{[}zwischen 26. und 30. 10. 1903{]}}|(be}
\toendnotes[C]{\smallbreak\pagebreak[2]}
\correspDesc{Versand  durch Felix Salten im Zeitraum [zwischen 26. und 30. 10. 1903] in Wien
\newline{}Erhalt  durch Arthur Schnitzler im Zeitraum [zwischen 26. und 30. 10. 1903] in Wien}\toendnotes[C]{\smallbreak}
\Standort{CUL, Schnitzler, B 89, A 2.}
\physDesc{Brief, 1 Blatt, 1 Seite, 88 Zeichen
\newline{}Handschrift: Bleistift, lateinische Kurrent
\newline{}Schnitzler: mit Bleistift datiert: »Oct 903« 
\newline{}Ordnung: mit Bleistift von unbekannter Hand nummeriert: »174« }\toendnotes[C]{\smallbreak}
\pstart
           {\pb}\textcolor{gray}{\textbf{DIE}}\pend
           
\pstart
           \textcolor{gray}{\textbf{ZEIT\orgindex{Zeit@Die Zeit|pw}}}\hfill \textcolor{gray}{\textbf{\emph{WIEN}\oindex{Wien@\textbf{Wien}, \emph{Verwaltungsgebiet}|pw}}}\pend
           
\pstart
           \textcolor{gray}{\textbf{Wien\oindex{Wien@\textbf{Wien}, \emph{Verwaltungsgebiet}|pw}er Tageszeitung}}\hfill \textcolor{gray}{\textbf{\emph{I. Wipplingerstrasse 38\oindex{Wien@\textbf{Wien}!I., Innere Stadt@\textbf{I., Innere Stadt}!Wipplingerstraße@\textbf{Wipplingerstraße}, \emph{Straße}|pw}}}}\pend
           
\pstart
           \textcolor{gray}{\textbf{Herausgeber:}}\pend
           
\pstart
           \textcolor{gray}{\textbf{\textbf{Prof. Dr. I. Singer\pwindex{Singer, Isidor 16.\,1.\,1857 Budapest – 8.\,12.\,1927 Wien@\textsc{Singer, Isidor} (16.\,1.\,1857 Budapest – 8.\,12.\,1927 Wien), \emph{Journalist, Herausgeber, Soziologe}|pw}}}}\pend
           
\pstart
           \textcolor{gray}{\textbf{\textbf{Dr. Heinrich Kanner\pwindex{Kanner, Heinrich 9.\,11.\,1864 Galați – 15.\,2.\,1930 Wien@\textsc{Kanner, Heinrich} (9.\,11.\,1864 Galați – 15.\,2.\,1930 Wien), \emph{Herausgeber, Publizist}|pw}}}}\pend
           
\pstart
           \textcolor{gray}{\textbf{\textbf{Redaction}}}\pend
           
\pstart
           \textcolor{gray}{\textbf{Telegramm-Adresse: \so{Zeit}\orgindex{Zeit@Die Zeit|pw}\so{,}{ }\so{Wien}\oindex{Wien@\textbf{Wien}, \emph{Verwaltungsgebiet}|pw}}}\pend
           
\pstart
           \textcolor{gray}{\textbf{Interurbanes Telephon Nr. 15.988}}\pend
           
\pstart
           \textcolor{gray}{\textbf{= Telephone Nr. 17.040, 17.041 =}}\pend
           \vspace{0.5em}
\pstart
           Lieber, wir kommen also (mit \label{K_L03348-1v}\edtext{fourage}{\lemma{\textnormal{\emph{fourage}}}\Cendnote{\textnormal{eigentlich Pferdefutter, hier im Sinne von: mitgebrachtes Essen}}}\label{K_L03348-1}) Sonntag nach dem »\label{K_L03348-2v}\edtext{Müller\pwindex{Raupach, Ernst 21.\,5.\,1784 – 18.\,3.\,1852@\textsc{Raupach, Ernst} (21.\,5.\,1784 – 18.\,3.\,1852), \emph{Schriftsteller, Literaturwissenschaftler, Dichter}!Müller und sein Kind. Volksdrama in fünf Aufzügen@\strich\emph{Der Müller und sein Kind. Volksdrama in fünf Aufzügen}|pw}}{\lemma{\textnormal{\emph{Müller}}}\Cendnote{\textnormal{\emph{Der Müller und sein Kind. Volksdrama in fünf
                     Aufzügen}\pwindex{Raupach, Ernst 21.\,5.\,1784 – 18.\,3.\,1852@\textsc{Raupach, Ernst} (21.\,5.\,1784 – 18.\,3.\,1852), \emph{Schriftsteller, Literaturwissenschaftler, Dichter}!Müller und sein Kind. Volksdrama in fünf Aufzügen@\strich\emph{Der Müller und sein Kind. Volksdrama in fünf Aufzügen}|pwk} von Ernst Raupach\pwindex{Raupach, Ernst 21.\,5.\,1784 – 18.\,3.\,1852@\textsc{Raupach, Ernst} (21.\,5.\,1784 – 18.\,3.\,1852), \emph{Schriftsteller, Literaturwissenschaftler, Dichter}|pwk} wurde am
                  1. 11. 1903 am \emph{Raimundtheater}\orgindex{Raimund-Theater@Raimund-Theater|pwk} als Nachmittagsvorstellung (Beginn
                     14 Uhr 30) gegeben. Das erlaubt die Datierung des
                  Korrespondenzstücks in die Woche vor Sonntag, dem 1. 11. 1903. Der Brief
                   XXXX Auszeichnungsfehler: Dokument L03349 nicht gefunden wiederum
                  folgt auf den vorliegenden und ist ebenfalls vor dem Sonntag zu datieren.}}}\label{K_L03348-2}«
               zu Ihnen.\pend
           
\pstart
           Herzlichst {\\[\baselineskip]}Ihr {\\[\baselineskip]}\spacefill\mbox{Salten}\pend
           \leftskip=0em{}\selectlanguage{ngerman}\endnumbering\briefempfaengerindex{Schnitzler, Arthur@\textsc{Schnitzler, Arthur}!zzzSalten, Felix@\emph{von Felix Salten}!1903-10-261@{{[}zwischen 26. und 30. 10. 1903{]}}|)be}\mylabel{L03348h}  \newcommand{\dateiname}{L03348}\newcommand{\titel}{Felix Salten an Arthur Schnitzler, [zwischen 26. und 30. 10. 1903]}\newcommand{\editorInnen}{Martin Anton Müller und Laura Untner}%% latex-leseansicht-abspann.tex
%% Abspann für die Leseansicht.
%% Der Schalter \ifkorrekturansicht ist bereits durch den Vorspann gesetzt.

%% latex-abspann.tex
%% Gemeinsamer Abspann für Korrekturansicht und Leseansicht.
%% Setzt den Schalter \ifkorrekturansicht voraus (gesetzt in den
%% einbindenden Dateien latex-korrekturansicht-abspann.tex bzw.
%% latex-leseansicht-abspann.tex).
%% ---------------------------------------------------------------

\normalsize

% Das esempio-Environment wird nur in der Leseansicht benötigt
\ifkorrekturansicht\else
\newenvironment{esempio}[3]%
{
    \vspace{1.5ex}
    \rlap{\underline{#1}}
    \par
    \setlength{\parindent}{0cm}
    \nopagebreak
    \leftskip=#2cm
    \rightskip=#3cm
}
{
    \par
}
\fi

\doendnotes{C}
\bigskip
\vfill

\clearpage

\footnotesize

\ifkorrekturansicht
  \lohead{\textsc{register}}
\fi

% theindex-Environment neu definieren ohne reledmac
\makeatletter
\renewenvironment{theindex}{%
  \ifkorrekturansicht
    \section*{\indexname}%
  \else
    \subsubsection*{Index der erwähnten Entitäten}%
  \fi
  \setlength{\parindent}{0pt}%
  \setlength{\parskip}{0pt plus 0.3pt}%
  \let\item\@idxitem
}{%
  \ifkorrekturansicht\clearpage\fi
}
\makeatother

\IfFileExists{\jobname-pw.ind}{\input{\jobname-pw.ind}}{}

% Quellenangabe nur in der Leseansicht
\ifkorrekturansicht\else
% Fallback-Definitionen, falls die .tex-Datei \titel etc. nicht gesetzt hat
\providecommand{\titel}{}
\providecommand{\editorInnen}{}
\providecommand{\dateiname}{\jobname}

\vspace{3cm}

\vfill

\footnotesize
\textsc{Quelle}: \titel. Herausgegeben von {\editorInnen}. In: \emph{Arthur Schnitzler: Briefwechsel mit Autorinnen und Autoren}.
 Digitale Edition, https://schnitzler-briefe.acdh.oeaw.ac.at/{\dateiname}.html (Stand \today)
\fi

\end{document}


