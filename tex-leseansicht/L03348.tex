%% latex-korrekturansicht-vorspann.tex
%% Vorspann für die Korrekturansicht.
%% Lädt die gemeinsame Datei latex-vorspann.tex mit gesetztem Schalter.

\newif\ifkorrekturansicht
\korrekturansichttrue

\input{../tex-inputs/latex-vorspann}


\section[ Felix Salten an Arthur Schnitzler, {[}zwischen 26. und 30. 10. 1903{]}]{L03348 Felix Salten an Arthur
               Schnitzler, {[}zwischen 26. und 30. 10. 1903{]}}
\nopagebreak\mylabel{L03348v}
\rehead{ }\normalsize\beginnumbering\briefempfaengerindex{Schnitzler, Arthur@\textsc{Schnitzler, Arthur}!zzzSalten, Felix@\emph{von Felix Salten}!1903-10-301@{{[}zwischen 26. und 30. 10. 1903{]}}|(be}
\toendnotes[C]{\smallbreak\pagebreak[2]}\Standort{CUL, Schnitzler, B 89, A 2.}
\physDesc{Brief, 1 Blatt, 1 Seite, 88 Zeichen
\newline{}Handschrift: Bleistift, lateinische Kurrent
\newline{}Schnitzler: mit Bleistift datiert: »Oct 903« 
\newline{}Ordnung: mit Bleistift von unbekannter Hand nummeriert: »174« }\toendnotes[C]{\smallbreak}
\pstart
           {\pb}\textcolor{gray}{\textbf{DIE}}\pend
           
\pstart
           \textcolor{gray}{\textbf{ZEIT\orgindex{Zeit@Die Zeit|pw}}}\hfill \textcolor{gray}{\textbf{\emph{WIEN}\oindex{Wien@\textbf{Wien}, \emph{A.ADM2}|pw}}}\pend
           
\pstart
           \textcolor{gray}{\textbf{Wien\oindex{Wien@\textbf{Wien}, \emph{A.ADM2}|pw}er Tageszeitung}}\hfill \textcolor{gray}{\textbf{\emph{I. Wipplingerstrasse 38\oindex{Wipplingerstrasse@\textbf{Wipplingerstraße}, \emph{Straße (K.STR)}|pw}}}}\pend
           
\pstart
           \textcolor{gray}{\textbf{Herausgeber:}}\pend
           
\pstart
           \textcolor{gray}{\textbf{\textbf{Prof. Dr. I. Singer\pwindex{Singer, Isidor 16.01.1857 – 08.12.1927@\textsc{Singer, Isidor} (16.01.1857 – 08.12.1927), \emph{Journalist/Journalistin, Herausgeber/Herausgeberin, Soziologe/Soziologin}|pw}}}}\pend
           
\pstart
           \textcolor{gray}{\textbf{\textbf{Dr. Heinrich Kanner\pwindex{Kanner, Heinrich 09.11.1864 – 15.02.1930@\textsc{Kanner, Heinrich} (09.11.1864 – 15.02.1930), \emph{Herausgeber/Herausgeberin, Publizist/Publizistin}|pw}}}}\pend
           
\pstart
           \textcolor{gray}{\textbf{\textbf{Redaction}}}\pend
           
\pstart
           \textcolor{gray}{\textbf{Telegramm-Adresse: \so{Zeit}\orgindex{Zeit@Die Zeit|pw}\so{,}{ }\so{Wien}\oindex{Wien@\textbf{Wien}, \emph{A.ADM2}|pw}}}\pend
           
\pstart
           \textcolor{gray}{\textbf{Interurbanes Telephon Nr. 15.988}}\pend
           
\pstart
           \textcolor{gray}{\textbf{= Telephone Nr. 17.040, 17.041 =}}\pend
           \vspace{0.5em}
\pstart
           Lieber, wir kommen also (mit \label{K_L03348-1v}\edtext{fourage}{\lemma{\textnormal{\emph{fourage}}}\Cendnote{\textnormal{eigentlich Pferdefutter, hier im Sinne von: mitgebrachtes Essen}}}\label{K_L03348-1}) Sonntag nach dem »\label{K_L03348-2v}\edtext{Müller\pwindex{Mueller und sein Kind. Volksdrama in fuenf Aufzuegen@\emph{Der Müller und sein Kind. Volksdrama in fünf Aufzügen}|pw}}{\lemma{\textnormal{\emph{Müller}}}\Cendnote{\textnormal{\emph{Der Müller und sein Kind. Volksdrama in fünf
                     Aufzügen}\pwindex{Mueller und sein Kind. Volksdrama in fuenf Aufzuegen@\emph{Der Müller und sein Kind. Volksdrama in fünf Aufzügen}|pwk} von Ernst Raupach\pwindex{Raupach, Ernst 1784-05-21 – 1852-03-18@\textsc{Raupach, Ernst} (1784-05-21 – 1852-03-18), \emph{Schriftsteller/Schriftstellerin, Literaturwissenschaftler/Literaturwissenschaftlerin, Dichter/Dichterin}|pwk} wurde am
                  1. 11. 1903 am \emph{Raimundtheater}\orgindex{Raimund-Theater@Raimund-Theater|pwk} als Nachmittagsvorstellung (Beginn
                     14 Uhr 30) gegeben. Das erlaubt die Datierung des
                  Korrespondenzstücks in die Woche vor Sonntag, dem 1. 11. 1903. Der Brief
                   [zwischen 27. und 31. 10. 1903] wiederum
                  folgt auf den vorliegenden und ist ebenfalls vor dem Sonntag zu datieren.}}}\label{K_L03348-2}«
               zu Ihnen.\pend
           
\pstart
           Herzlichst {\\[\baselineskip]}Ihr {\\[\baselineskip]}\spacefill\mbox{Salten}\pend
           \leftskip=0em{}\selectlanguage{ngerman}\endnumbering\briefempfaengerindex{Schnitzler, Arthur@\textsc{Schnitzler, Arthur}!zzzSalten, Felix@\emph{von Felix Salten}!1903-10-261@{{[}zwischen 26. und 30. 10. 1903{]}}|)be}\mylabel{L03348h}  \normalsize

\doendnotes{C}
\bigskip
\vfill

\clearpage

\footnotesize

\lohead{\textsc{register}}

% Definiere theindex-Environment komplett neu ohne reledmac
\makeatletter
\renewenvironment{theindex}{%
  \section*{\indexname}%
  \setlength{\parindent}{0pt}%
  \setlength{\parskip}{0pt plus 0.3pt}%
  \let\item\@idxitem
}{%
  \clearpage
}
\makeatother

\IfFileExists{\jobname-pw.ind}{\input{\jobname-pw.ind}}{}

\end{document}

      