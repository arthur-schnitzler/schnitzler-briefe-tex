%% latex-korrekturansicht-vorspann.tex
%% Vorspann für die Korrekturansicht.
%% Lädt die gemeinsame Datei latex-vorspann.tex mit gesetztem Schalter.

\newif\ifkorrekturansicht
\korrekturansichttrue

\input{../tex-inputs/latex-vorspann}


\section[ Paul Goldmann an Arthur Schnitzler, 16. 9. {[}1901{]}]{L03084 Paul Goldmann an Arthur Schnitzler, 16. 9. {[}1901{]}}
\nopagebreak\mylabel{L03084v}
\rehead{ }\normalsize\beginnumbering\briefempfaengerindex{Schnitzler, Arthur@\textsc{Schnitzler, Arthur}!zzzGoldmann, Paul@\emph{von Paul Goldmann}!1901-09-162@{16. 9. {[}1901{]}}|(be}
\toendnotes[C]{\smallbreak\pagebreak[2]}\Standort{DLA, A:Schnitzler, HS.NZ85.1.3171.}
\physDesc{Brief, 1 Blatt, 3 Seiten, 822 Zeichen
\newline{}Handschrift: blaue Tinte, deutsche Kurrent
\newline{}Schnitzler: 1) mit Bleistift das Jahr »901« vermerkt  2) mit rotem Buntstift drei Unterstreichungen}\toendnotes[C]{\smallbreak}
\pstart
           \raggedleft{}{\pb}\textcolor{gray}{\textbf{DESSAUERSTRASSE 19}}\oindex{Dessauer Strasse@\textbf{Dessauer Straße}, \emph{Straße (K.STR)}|pw}\pend
           
\pstart
           Berlin\oindex{Berlin@\textbf{Berlin}, \emph{P.PPLC}|pw}, 16. September.\pend
           
\pstart\center{}Mein lieber Freund,\pend\vspace{0.5em}
\pstart
           Bin aus Danzig\oindex{Danzig@\textbf{Danzig}, \emph{A.ADM3}|pw} zurück, finde Deinen lieben
               Brief, habe ſehr viel zu thun und kann einſtweilen nur in Eile antworten: Habe geſtern die \textsc{Triesch\pwindex{Triesch, Irene 13.04.1877 – 24.11.1964@\textsc{Triesch, Irene} (13.04.1877 – 24.11.1964), \emph{Schauspieler/Schauspielerin}|pw}} geſprochen, die mit Sehnſucht auf Deine Stücke\pwindex{Lebendige Stunden. Vier Einakter@\emph{Lebendige Stunden. Vier Einakter}|pwv} wartet und auch ſehr gern die \label{K_L03084-1v}\edtext{\textcolor{gray}{L}uſtſpielrolle \textcolor{gray}{im}{ }dritten\pwindex{Literatur@\emph{Literatur}|pwv} ſpielen}{\lemma{\textnormal{\emph{Luſtſpielrolle … ſpielen}}}\Cendnote{\textnormal{Irene Triesch\pwindex{Triesch, Irene 13.04.1877 – 24.11.1964@\textsc{Triesch, Irene} (13.04.1877 – 24.11.1964), \emph{Schauspieler/Schauspielerin}|pwk} übernahm bei der Uraufführung
                  von \emph{Lebendige Stunden}\pwindex{Lebendige Stunden. Vier Einakter@\emph{Lebendige Stunden. Vier Einakter}|pwk} am 4. 1. 1902 am \emph{Deutschen Theater Berlin}\orgindex{Deutsches Theater Berlin@Deutsches Theater Berlin|pwk} die Rolle der Margarete\pwindex{Literatur@\emph{Literatur}|pwkv} in \emph{Literatur}\pwindex{Literatur@\emph{Literatur}|pwk}.}}}\label{K_L03084-1} möchte. Außer {\pb}dem neuen \label{K_L03084-2v}\edtext{Stück\pwindex{Es lebe das Leben@\emph{Es lebe das Leben}|pwv}}{\lemma{\textnormal{\emph{Stück}}}\Cendnote{\textnormal{Hermann Sudermanns\pwindex{Sudermann, Hermann 30.09.1857 – 21.11.1928@\textsc{Sudermann, Hermann} (30.09.1857 – 21.11.1928), \emph{Schriftsteller/Schriftstellerin}|pwk} Fünfakter \emph{Es lebe das Leben}\pwindex{Es lebe das Leben@\emph{Es lebe das Leben}|pwk} wurde am 1. 2. 1902 am Deutschen
                     Theater Berlin\oindex{Deutsches Theater Berlin@\textbf{Deutsches Theater Berlin}, \emph{Theater (K.THE)}|pwk} uraufgeführt.}}}\label{K_L03084-2} von \textsc{Sudermann\pwindex{Sudermann, Hermann 30.09.1857 – 21.11.1928@\textsc{Sudermann, Hermann} (30.09.1857 – 21.11.1928), \emph{Schriftsteller/Schriftstellerin}|pw}} hat \textsc{Brahm\pwindex{Brahm, Otto 05.02.1856 – 28.11.1912@\textsc{Brahm, Otto} (05.02.1856 – 28.11.1912), \emph{Theaterleiter/Theaterleiterin, Regisseur/Regisseurin}|pw}} gar nichts.\pend
           
\pstart
           Das \label{K_L03084-3v}\edtext{Urtheil}{\lemma{\textnormal{\emph{Urtheil}}}\Cendnote{\textnormal{Zu den Einaktern \emph{Die Frau mit
                     dem Dolche}\pwindex{Frau mit dem Dolche@\emph{Die Frau mit dem Dolche}|pwk} und \emph{Lebendige Stunden}\pwindex{Lebendige Stunden@\emph{Lebendige Stunden}|pwk},
                     vgl. A. S.: \emph{Tagebuch}, 4. 9. 1901.}}}\label{K_L03084-3}, das \textsc{Schwarzkopf\pwindex{Schwarzkopf, Gustav 07.11.1853 – 13.11.1939@\textsc{Schwarzkopf, Gustav} (07.11.1853 – 13.11.1939), \emph{Schriftsteller/Schriftstellerin}|pw}} und \textsc{Salten\pwindex{Salten, Felix 06.09.1869 – 08.10.1945@\textsc{Salten, Felix} (06.09.1869 – 08.10.1945), \emph{Schriftsteller/Schriftstellerin, Journalist/Journalistin, Chefredakteur/Chefredakteurin}|pw}} gefällt haben, halte ich für durchaus unrichtig.\pend
           
\pstart
           Dagegen billige ich durchaus den \label{K_L03084-4v}\edtext{Standpunkt, den \textsc{Burckhardt\pwindex{Burckhard, Max Eugen 14.07.1854 – 16.03.1912@\textsc{Burckhard, Max Eugen} (14.07.1854 – 16.03.1912), \emph{Schriftsteller/Schriftstellerin, Rechtswissenschaftler/Rechtswissenschaftlerin, Theaterleiter/Theaterleiterin}|pw}}}{\lemma{\textnormal{\emph{Standpunkt, den Burckhardt}}}\Cendnote{\textnormal{Max Burckhard\pwindex{Burckhard, Max Eugen 14.07.1854 – 16.03.1912@\textsc{Burckhard, Max Eugen} (14.07.1854 – 16.03.1912), \emph{Schriftsteller/Schriftstellerin, Rechtswissenschaftler/Rechtswissenschaftlerin, Theaterleiter/Theaterleiterin}|pwk}, ehemaliger Direktor des \emph{Burgtheaters}\orgindex{Burgtheater@Burgtheater|pwk}, war seit 1901 am \emph{Verwaltungsgerichtshof}\orgindex{Verwaltungsgerichtshof@Verwaltungsgerichtshof|pwk}
                  tätig und beriet Schnitzler in der \emph{Lieutenant Gustl}\pwindex{Lieutenant Gustl. Novelle@\emph{Lieutenant Gustl. Novelle}|pwk}-Affäre. Er empfahl Schnitzler, nicht zu reagieren, vgl. Hermann Bahr, Arthur Schnitzler: \emph{Briefwechsel, Aufzeichnungen, Dokumente (1891–1931)}, Arthur Schnitzler: Max Burckhard, April 1912.}}}\label{K_L03084-4}{ }\strikeout{ein} in der Militär-Affaire einnimmt. Laß’ die Leute
               nur reden! Und ſchreib’ weiter gute {\pb}Stücke! Das iſt
               die beſte Antwort und ärgert ſie am Meiſten.\pend
           
\pstart
           Ich danke vielmals für die Zuſendung der alten Hoſen, die ich bei Euch vergeſſen
               hatte. Hättet ſie auch behalten können.\pend
           
\pstart
           Empfiehl’ mich Deiner Frau Mutter\pwindex{Schnitzler, Louise 1840-07-08 – 1911-09-09@\textsc{Schnitzler, Louise} (1840-07-08 – 1911-09-09)|pwv} und ſei herzlichſt gegrüßt!\pend
           
\pstart
           Dein {\\[\baselineskip]}\spacefill\mbox{Paul Goldmann}\pend
           \leftskip=0em{}\selectlanguage{ngerman}\endnumbering\briefempfaengerindex{Schnitzler, Arthur@\textsc{Schnitzler, Arthur}!zzzGoldmann, Paul@\emph{von Paul Goldmann}!1901-09-162@{16. 9. {[}1901{]}}|)be}\mylabel{L03084h}  \normalsize

\doendnotes{C}
\bigskip
\vfill

\clearpage

\footnotesize

\lohead{\textsc{register}}

% Definiere theindex-Environment komplett neu ohne reledmac
\makeatletter
\renewenvironment{theindex}{%
  \section*{\indexname}%
  \setlength{\parindent}{0pt}%
  \setlength{\parskip}{0pt plus 0.3pt}%
  \let\item\@idxitem
}{%
  \clearpage
}
\makeatother

\IfFileExists{\jobname-pw.ind}{\input{\jobname-pw.ind}}{}

\end{document}

      