\input{../tex-inputs/latex-pdf-vorspann}
\begin{center}
            \textcolor{red}{ENTWURF. ENTZIFFERUNG NOCH NICHT KORREKTURGELESEN}
                      \end{center}
            
               \section[Hugo Hofmannsthal an Arthur Schnitzler, 13. 3. 1920]{ Hugo Hofmannsthal an Arthur Schnitzler, 13. 3. 1920}\nopagebreak\mylabel{v}\rehead{ }\begin{ledgroupsized}[t]{13cm}\normalsize\beginnumbering\briefempfaengerindex{Schnitzler, Arthur@\textsc{Schnitzler, Arthur}!zzzHofmannsthal, Hugo von@\emph{von Hugo von Hofmannsthal}!1920-03-131@{13. 3. 1920}|(be} \toendnotes[C]{\smallbreak\pagebreak[2]} \Standort{CUL, Schnitzler, B 43.}
\physDesc{Postkarte
\newline{}Handschrift: schwarze Tinte, deutsche Kurrent\newline{}Versand: Stempel: »\nobreak{}\oindex{Rodaun@\textbf{Rodaun}|pwk}Rodaun\nobreak{}«.  \newline{}Ordnung: 1) mit Bleistift von Frieda Pollak\pwindex{Pollak, Frieda 08.12.1881 – 13.07.1937@\textsc{Pollak, Frieda} (08.12.1881 – 13.07.1937), \emph{Sekretärin}|pw} (?) mit dem Buchstaben »A« (Abgeschrieben/Abschrift) gekennzeichnet 2) mit Bleistift von unbekannter Hand nummeriert: »\strikeout{260}«3) mit Bleistift von unbekannter Hand nummeriert: »364«}\buchAbdrucke{\weitereDrucke{Hugo von Hofmannsthal, Arthur Schnitzler: \emph{Briefwechsel}. Hg. Therese Nickl und Heinrich Schnitzler. Frankfurt am Main: \emph{S. Fischer} 1964, S. 291.} }\toendnotes[C]{\smallbreak}\pstart{}{\pb}\textsc{Herrn D\textsuperscript{r} Arthur Schnitzler}\pend{}\pstart{}\textsc{Wien\oindex{Wien@\textbf{Wien}|pw}}\pend{}\pstart{}\textsc{XVIII. Sternwartestrasse 71\oindex{Sternwartestrasse@\textbf{Sternwartestraße}|pw}}\pend{}{\bigskip}\pstart
           \raggedleft{}{\pb}Rodaun\oindex{Rodaun@\textbf{Rodaun}|pw}{ }13 III 20\pend
           \pstart{}mein lieber Arthur, \pend\pstart
           seit 5 Wochen vegetiere ich hier zwiſchen Bett u. Fauteuil (mehr Bett als Fauteuil)
               mit Grippe in Form von Rheumatismen vom Genick bis in die Fußzehen. \pend
           \pstart
           {\pb}Hab ſeit 5 Wochen Gerty\pwindex{Hofmannsthal, Gertrude von 16.03.1880 – 09.11.1959@\textsc{Hofmannsthal, Gertrude von} (16.03.1880 – 09.11.1959)|pw} nicht geſehen, die drinnen, aber indeſſen hergeſtellt. –
               Hab ich, um mein Vergnügen an dem Luſtſpiel\pwindex{Schnitzler, Arthur 15.05.1862 – 21.10.1931@\textsc{Schnitzler, Arthur} (15.05.1862 – 21.10.1931), \emph{Schriftsteller, Mediziner}!Schwestern oder Casanova in Spa. Lustspiel in Versen01. 10. 1919@\strich\emph{Die Schwestern oder Casanova in Spa. Lustspiel in Versen} {[}01. 10. 1919{]}|pwv} zu bezeichnen, das Wort »unterhaltend« gebraucht? u. war Ihnen das
               Wort unlieb? (faſt ſcheint’s mir ſo.) Ich gebrauchte es, um etwas Seltenes
               auszudrücken, den freien leichten Silberglanz des Geiſtes, den zu empfangen woltuend
               iſt. Natürlich hat ein Dichterwerk noch viele andere Eigenſchaften!\pend
           \pstart
           Alles Gute Ihnen für die Proben\pwindex{Schnitzler, Arthur 15.05.1862 – 21.10.1931@\textsc{Schnitzler, Arthur} (15.05.1862 – 21.10.1931), \emph{Schriftsteller, Mediziner}!Schwestern oder Casanova in Spa. Lustspiel in Versen01. 10. 1919@\strich\emph{Die Schwestern oder Casanova in Spa. Lustspiel in Versen} {[}01. 10. 1919{]}|pwv}
               u. überhaupt! Von Herzen Ihr{\\}\spacefill\mbox{Hugo.}\pend
           \endnumbering\briefempfaengerindex{Schnitzler, Arthur@\textsc{Schnitzler, Arthur}!zzzHofmannsthal, Hugo von@\emph{von Hugo von Hofmannsthal}!1920-03-131@{13. 3. 1920}|)be}\mylabel{h}\end{ledgroupsized}  \newcommand{\dateiname}{L02338}\newcommand{\titel}{Hugo Hofmannsthal an Arthur Schnitzler, 13. 3. 1920}\newcommand{\editorInnen}{Martin Anton Müller und Gerd-Hermann Susen}\input{../tex-inputs/latex-pdf-abspann}
      