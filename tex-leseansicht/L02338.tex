%% latex-korrekturansicht-vorspann.tex
%% Vorspann für die Korrekturansicht.
%% Lädt die gemeinsame Datei latex-vorspann.tex mit gesetztem Schalter.

\newif\ifkorrekturansicht
\korrekturansichttrue

\input{../tex-inputs/latex-vorspann}


\section[Hugo Hofmannsthal an Arthur Schnitzler, 13. 3. 1920]{L02338 Hugo Hofmannsthal an Arthur Schnitzler, 13. 3. 1920}
\nopagebreak\mylabel{L02338v}
\rehead{ }\normalsize\beginnumbering\briefempfaengerindex{Schnitzler, Arthur@\textsc{Schnitzler, Arthur}!zzzHofmannsthal, Hugo von@\emph{von Hugo von Hofmannsthal}!1920-03-131@{13. 3. 1920}|(be}
\toendnotes[C]{\smallbreak\pagebreak[2]}\Standort{CUL, Schnitzler, B 43.}
\physDesc{Postkarte, 720 Zeichen
\newline{}Handschrift: 1) schwarze Tinte, deutsche Kurrent\hspace{1em}2) schwarze Tinte, lateinische Kurrent (\noindent{}Adresse)\hspace{1em}
\newline{}Versand: Stempel: »\nobreak{}\oindex{Rodaun@\textbf{Rodaun}, \emph{A.ADM4}|pwk}Rodaun\nobreak{}«.  
\newline{}Ordnung: 1) mit Bleistift von Frieda
                                    Pollak\pwindex{Pollak, Frieda 08.12.1881 – 13.07.1937@\textsc{Pollak, Frieda} (08.12.1881 – 13.07.1937), \emph{Sekretär/Sekretärin}|pw} (?) mit dem Buchstaben »A«
                                 (Abgeschrieben/Abschrift) gekennzeichnet  2) mit Bleistift von unbekannter Hand nummeriert: »\strikeout{260}« 3) mit Bleistift von unbekannter Hand nummeriert:
                                    »364«}
\buchAbdrucke{\weitereDrucke{Hugo von Hofmannsthal, Arthur Schnitzler: \emph{Briefwechsel}. Frankfurt am Main: \emph{S. Fischer} 1964, S. 291.} }\toendnotes[C]{\smallbreak}\pstart{}{\pb}Herrn D\textsuperscript{r} Arthur Schnitzler\pend{}\pstart{}Wien\oindex{Wien@\textbf{Wien}, \emph{A.ADM2}|pw}\pend{}\pstart{}XVIII. Sternwartestrasse 71\oindex{Sternwartestrasse 71@\textbf{Sternwartestraße 71}, \emph{Wohngebäude (K.WHS)}|pw}\pend{}{\bigskip}\vspace{1em}
\pstart
           \raggedleft{}{\pb}Rodaun\oindex{Rodaun@\textbf{Rodaun}, \emph{A.ADM4}|pw}{ }13 III 20\pend
           
\pstart{}mein lieber Arthur, \pend\vspace{0.5em}
\pstart
           seit 5 Wochen vegetiere ich hier zwiſchen Bett u. Fauteuil (mehr Bett als Fauteuil)
               mit Grippe in Form von Rheumatismen vom Genick bis in die Fußzehen. \pend
           
\pstart
           {\pb}Hab ſeit 5 Wochen Gerty\pwindex{Hofmannsthal, Gertrude von 16.03.1880 – 09.11.1959@\textsc{Hofmannsthal, Gertrude von} (16.03.1880 – 09.11.1959)|pw} nicht geſehen, die drinnen, aber indeſſen
               hergeſtellt. – Hab ich, um mein Vergnügen an dem Luſtſpiel\pwindex{Schwestern oder Casanova in Spa. Lustspiel in Versen@\emph{Die Schwestern oder Casanova in Spa. Lustspiel in Versen}|pwv} zu bezeichnen, das Wort »unterhaltend« gebraucht?
               u. war Ihnen das Wort unlieb? (faſt ſcheint’s mir ſo.) Ich gebrauchte es, um etwas
               Seltenes auszudrücken, den freien leichten Silberglanz des Geiſtes, den zu empfangen
               woltuend iſt. Natürlich hat ein Dichterwerk noch viele andere Eigenſchaften!\pend
           
\pstart
           Alles Gute Ihnen für die Proben\pwindex{Schwestern oder Casanova in Spa. Lustspiel in Versen@\emph{Die Schwestern oder Casanova in Spa. Lustspiel in Versen}|pwv} u. überhaupt! Von Herzen Ihr{\\}\spacefill\mbox{Hugo.}\pend
           \selectlanguage{ngerman}\endnumbering\briefempfaengerindex{Schnitzler, Arthur@\textsc{Schnitzler, Arthur}!zzzHofmannsthal, Hugo von@\emph{von Hugo von Hofmannsthal}!1920-03-131@{13. 3. 1920}|)be}\mylabel{L02338h}  \normalsize

\doendnotes{C}
\bigskip
\vfill

\clearpage

\footnotesize

\lohead{\textsc{register}}

% Definiere theindex-Environment komplett neu ohne reledmac
\makeatletter
\renewenvironment{theindex}{%
  \section*{\indexname}%
  \setlength{\parindent}{0pt}%
  \setlength{\parskip}{0pt plus 0.3pt}%
  \let\item\@idxitem
}{%
  \clearpage
}
\makeatother

\IfFileExists{\jobname-pw.ind}{\input{\jobname-pw.ind}}{}

\end{document}

      