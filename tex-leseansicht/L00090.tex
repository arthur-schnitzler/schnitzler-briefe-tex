%% latex-leseansicht-vorspann.tex
%% Vorspann für die Leseansicht.
%% Lädt die gemeinsame Datei latex-vorspann.tex mit nicht gesetztem Schalter.

\newif\ifkorrekturansicht
\korrekturansichtfalse

\input{../tex-inputs/latex-vorspann}


\section[Arthur Schnitzler an Hugo von Hofmannsthal, 27. 3. 1892]{L00090 Arthur Schnitzler an Hugo von Hofmannsthal, 27. 3. 1892}
\nopagebreak\mylabel{L00090v}
\rehead{ }\normalsize\beginnumbering\briefempfaengerindex{Hofmannsthal, Hugo von@\textsc{Hofmannsthal, Hugo von}!zzzSchnitzler, Arthur@\emph{von Arthur Schnitzler}!1892-03-272@{27. 3. 1892}|(be}
\toendnotes[C]{\smallbreak\pagebreak[2]}
\correspDesc{Versand  durch Arthur Schnitzler am 27. 3. 1892 in Wien
\newline{}Erhalt  durch Hugo von Hofmannsthal im Zeitraum [27. 3. 1892
                  – 31. 3. 1892?] in Wien}\toendnotes[C]{\smallbreak}
\Standort{FDH, Hs-30885,19.}
\physDesc{Brief, 1 Blatt, 4 Seiten, 1132 Zeichen
\newline{}Handschrift: Bleistift, deutsche Kurrent}
\buchAbdrucke{\weitereDrucke{1) Hugo von Hofmannsthal, Arthur Schnitzler: \emph{Briefwechsel}. Herausgegeben von Therese Nickl und Heinrich Schnitzler. Frankfurt am Main: \emph{S. Fischer} 1964, S. 18–19.} \weitereDrucke{2) Hermann Bahr, Arthur Schnitzler: \emph{Briefwechsel, Aufzeichnungen, Dokumente (1891–1931)}. Herausgegeben von Kurt Ifkovits und Martin Anton Müller. Göttingen: \emph{Wallstein} 2018.} }\toendnotes[C]{\smallbreak}
\pstart
           \raggedleft{}{\pb}27/3 92\pend
           
\pstart{}Lieber Freund,\pend\vspace{0.5em}
\pstart
           es war mir{ }ſehr leid, daſs Sie heute nicht kamen. \textsc{Bölsche}\pwindex{Bölsche, Wilhelm 2.\,1.\,1861 Köln – 31.\,8.\,1939 Szklarska Poręba@\textsc{Bölsche, Wilhelm} (2.\,1.\,1861 Köln – 31.\,8.\,1939 Szklarska Poręba), \emph{Schriftsteller, Publizist}|pw} hat auch mir geſchrieben – auf eine Anfrage, ob man Gedichte einſenden kann u
               was mit meinen »Elixiren\pwindex{Schnitzler, Arthur 15.\,5.\,1862 Wien – 21.\,10.\,1931 ebd.@\textsc{Schnitzler, Arthur} (15.\,5.\,1862 Wien – 21.\,10.\,1931 ebd.), \emph{Schriftsteller, Mediziner}!drei Elixire@\strich\emph{Die drei Elixire}|pw}« los{ }ſei. – Er will die
                  Elixire\pwindex{Schnitzler, Arthur 15.\,5.\,1862 Wien – 21.\,10.\,1931 ebd.@\textsc{Schnitzler, Arthur} (15.\,5.\,1862 Wien – 21.\,10.\,1931 ebd.), \emph{Schriftsteller, Mediziner}!drei Elixire@\strich\emph{Die drei Elixire}|pw} bringen »ſobald es geht«, aber »offen
               geſtanden{ }ſind{ }ſie ihm nicht{ }ſo lieb {\pb}wie die erſte Novelle\pwindex{Schnitzler, Arthur 15.\,5.\,1862 Wien – 21.\,10.\,1931 ebd.@\textsc{Schnitzler, Arthur} (15.\,5.\,1862 Wien – 21.\,10.\,1931 ebd.), \emph{Schriftsteller, Mediziner}!Sohn. Aus den Papieren eines Arztes@\strich\emph{Der Sohn. Aus den Papieren eines Arztes}|pwv},{ }ſie{ }ſind lange nicht{ }ſo aktuell.« – Sagt’ ich’s nicht? Auch \uline{die} Herren
               haben{ }ſchon ihren Zopf. Wir brauchen ja doch »unſer« Blatt! – Ich will übrigens das
                  »Hi{\geminationm}elbett\pwindex{Schnitzler, Arthur 15.\,5.\,1862 Wien – 21.\,10.\,1931 ebd.@\textsc{Schnitzler, Arthur} (15.\,5.\,1862 Wien – 21.\,10.\,1931 ebd.), \emph{Schriftsteller, Mediziner}!Himmelbett@\strich\emph{Das Himmelbett}|pw}« an
                  \textsc{Bölsche}\pwindex{Bölsche, Wilhelm 2.\,1.\,1861 Köln – 31.\,8.\,1939 Szklarska Poręba@\textsc{Bölsche, Wilhelm} (2.\,1.\,1861 Köln – 31.\,8.\,1939 Szklarska Poręba), \emph{Schriftsteller, Publizist}|pw}{ }ſchicken. – Geſtern{ }ſprach ich Herrn \textsc{Leo Geiringer}\pwindex{Geiringer, Leopold 27.\,6.\,1851 Wien – 29.\,5.\,1900 ebd.@\textsc{Geiringer, Leopold} (27.\,6.\,1851 Wien – 29.\,5.\,1900 ebd.), \emph{Schriftsteller, Dramaturg}|pw}, den Dramaturgen des Dtſch Volksth.\oindex{Wien@\textbf{Wien}!VII., Neubau@\textbf{VII., Neubau}!Volkstheater@\textbf{Volkstheater}, \emph{Theater}|pw}, der
               mich um mein Märchen\pwindex{Schnitzler, Arthur 15.\,5.\,1862 Wien – 21.\,10.\,1931 ebd.@\textsc{Schnitzler, Arthur} (15.\,5.\,1862 Wien – 21.\,10.\,1931 ebd.), \emph{Schriftsteller, Mediziner}!Märchen. Schauspiel in drei Aufzügen@\strich\emph{Das Märchen. Schauspiel in drei Aufzügen}|pw} gebeten hatte – ich{ }ſandte
               es ihm {\pb}als »Privatmann«. – Er{ }ſagte: »Wirklich ein
               hübſches Talent, ich muſs nur bedauern, daß Sie{ }ſich \uline{dieſer} Richtung zugewandt haben!{[}«{]}\pend
           
\pstart
           \uline{Ich}{ }{\dotstwo}?{\dotsfour}! – ?\pend
           
\pstart
           \uline{Er}. Nun ja, Sie werden doch zugeben, der Schluſs ist
                  unbefriedigend{\dots}\pend
           
\pstart
           \uline{Ich}.{ }{\dotstwo}!{\dots}in den Charakteren{\dots}\pend
           
\pstart
           \uline{Er}. Die Erfahrung lehrt nun einmal, daß unſer
               Publicum \textsc{etc etc}.\pend
           
\pstart
           {\pb}\uline{Ich.}{ }{\dots}{ }Wildente\pwindex{\textcolor{red}{\textsuperscript{XXXX indx1}}!Wildente. Schauspiel in fünf Akten@\strich\emph{Die Wildente. Schauspiel in fünf Akten}|pw}!!{\dotsfour}\pend
           
\pstart
           \uline{Er}. Den Einfluſs merkt man auch deutlich {\dotstwo} ich will nicht gerade{ }ſagen, daß Sie abgeſchrieben
                  haben{\dotsfour}\pend
           
\pstart
           \label{T_L00090-1v}\edtext{!!.Ich.}{\lemma{\textnormal{\emph{!!.Ich.}}}\Cendnote{\textnormal{kopfüber zum Text}}}\label{T_L00090-1}\pend
           
\pstart
           Herzlichſt der Ihre, und ko{\geminationm}en Sie Dienſtag gef. zur \textsc{Bahr}\pwindex{Bahr, Hermann 19.\,7.\,1863 Linz – 15.\,1.\,1934 München@\textsc{Bahr, Hermann} (19.\,7.\,1863 Linz – 15.\,1.\,1934 München), \emph{Schriftsteller, Kritiker}|pw}’ſchen \label{K_L00090-1v}\edtext{Myſtik}{\lemma{\textnormal{\emph{Mystik}}}\Cendnote{\textnormal{Gemeint ist Bahrs\pwindex{Bahr, Hermann 19.\,7.\,1863 Linz – 15.\,1.\,1934 München@\textsc{Bahr, Hermann} (19.\,7.\,1863 Linz – 15.\,1.\,1934 München), \emph{Schriftsteller, Kritiker}|pwk} Vortrag über »Moderne Mystik«, den er am
                     29. 3. 1892 bei einer Veranstaltung der \emph{Freien Bühne}\orgindex{»Freie Bühne« Verein für moderne Literatur@»Freie Bühne« Verein für moderne Literatur|pwk} hielt.}}}\label{K_L00090-1}!\pend
           \selectlanguage{ngerman}\endnumbering\briefempfaengerindex{Hofmannsthal, Hugo von@\textsc{Hofmannsthal, Hugo von}!zzzSchnitzler, Arthur@\emph{von Arthur Schnitzler}!1892-03-272@{27. 3. 1892}|)be}\mylabel{L00090h}  \newcommand{\dateiname}{L00090}\newcommand{\titel}{Arthur Schnitzler an Hugo von Hofmannsthal, 27. 3. 1892}\newcommand{\editorInnen}{Herausgegeben von Martin Anton Müller}%% latex-leseansicht-abspann.tex
%% Abspann für die Leseansicht.
%% Der Schalter \ifkorrekturansicht ist bereits durch den Vorspann gesetzt.

%% latex-abspann.tex
%% Gemeinsamer Abspann für Korrekturansicht und Leseansicht.
%% Setzt den Schalter \ifkorrekturansicht voraus (gesetzt in den
%% einbindenden Dateien latex-korrekturansicht-abspann.tex bzw.
%% latex-leseansicht-abspann.tex).
%% ---------------------------------------------------------------

\normalsize

% Das esempio-Environment wird nur in der Leseansicht benötigt
\ifkorrekturansicht\else
\newenvironment{esempio}[3]%
{
    \vspace{1.5ex}
    \rlap{\underline{#1}}
    \par
    \setlength{\parindent}{0cm}
    \nopagebreak
    \leftskip=#2cm
    \rightskip=#3cm
}
{
    \par
}
\fi

\doendnotes{C}
\bigskip
\vfill

\clearpage

\footnotesize

\ifkorrekturansicht
  \lohead{\textsc{register}}
\fi

% theindex-Environment neu definieren ohne reledmac
\makeatletter
\renewenvironment{theindex}{%
  \ifkorrekturansicht
    \section*{\indexname}%
  \else
    \subsubsection*{Index der erwähnten Entitäten}%
  \fi
  \setlength{\parindent}{0pt}%
  \setlength{\parskip}{0pt plus 0.3pt}%
  \let\item\@idxitem
}{%
  \ifkorrekturansicht\clearpage\fi
}
\makeatother

\IfFileExists{\jobname-pw.ind}{\input{\jobname-pw.ind}}{}

% Quellenangabe nur in der Leseansicht
\ifkorrekturansicht\else
% Fallback-Definitionen, falls die .tex-Datei \titel etc. nicht gesetzt hat
\providecommand{\titel}{}
\providecommand{\editorInnen}{}
\providecommand{\dateiname}{\jobname}

\vspace{3cm}

\vfill

\footnotesize
\textsc{Quelle}: \titel. Herausgegeben von {\editorInnen}. In: \emph{Arthur Schnitzler: Briefwechsel mit Autorinnen und Autoren}.
 Digitale Edition, https://schnitzler-briefe.acdh.oeaw.ac.at/{\dateiname}.html (Stand \today)
\fi

\end{document}


