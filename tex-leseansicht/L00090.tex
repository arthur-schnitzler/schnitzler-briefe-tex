%% latex-korrekturansicht-vorspann.tex
%% Vorspann für die Korrekturansicht.
%% Lädt die gemeinsame Datei latex-vorspann.tex mit gesetztem Schalter.

\newif\ifkorrekturansicht
\korrekturansichttrue

\input{../tex-inputs/latex-vorspann}


\section[Arthur Schnitzler an Hugo von Hofmannsthal, 27. 3. 1892]{L00090 Arthur Schnitzler an Hugo von Hofmannsthal, 27. 3. 1892}
\nopagebreak\mylabel{L00090v}
\rehead{ }\normalsize\beginnumbering\briefempfaengerindex{Hofmannsthal, Hugo von@\textsc{Hofmannsthal, Hugo von}!zzzSchnitzler, Arthur@\emph{von Arthur Schnitzler}!1892-03-272@{27. 3. 1892}|(be}
\toendnotes[C]{\smallbreak\pagebreak[2]}\Standort{FDH, Hs-30885,19.}
\physDesc{Brief, 1 Blatt, 4 Seiten, 1132 Zeichen
\newline{}Handschrift: Bleistift, deutsche Kurrent}
\buchAbdrucke{\weitereDrucke{1) Hugo von Hofmannsthal, Arthur Schnitzler: \emph{Briefwechsel}. Frankfurt am Main: \emph{S. Fischer} 1964, S. 18–19.} \weitereDrucke{2) Hermann Bahr, Arthur Schnitzler: \emph{Briefwechsel, Aufzeichnungen, Dokumente (1891–1931)}. Göttingen: \emph{Wallstein} 2018.} }\toendnotes[C]{\smallbreak}
\pstart
           \raggedleft{}{\pb}27/3 92\pend
           
\pstart{}Lieber Freund,\pend\vspace{0.5em}
\pstart
           es war mir ſehr leid, daſs Sie heute nicht kamen. \textsc{Bölsche}\pwindex{Boelsche, Wilhelm 02.01.1861 – 31.08.1939@\textsc{Bölsche, Wilhelm} (02.01.1861 – 31.08.1939), \emph{Schriftsteller/Schriftstellerin, Publizist/Publizistin}|pw} hat auch mir geſchrieben – auf eine Anfrage, ob man Gedichte einſenden kann u
               was mit meinen »Elixiren\pwindex{drei Elixire@\emph{Die drei Elixire}|pw}« los ſei. – Er will die
                  Elixire\pwindex{drei Elixire@\emph{Die drei Elixire}|pw} bringen »ſobald es geht«, aber »offen
               geſtanden ſind ſie ihm nicht ſo lieb {\pb}wie die erſte Novelle\pwindex{Sohn. Aus den Papieren eines Arztes@\emph{Der Sohn. Aus den Papieren eines Arztes}|pwv}, ſie ſind lange nicht
               ſo aktuell.« – Sagt’ ich’s nicht? Auch \uline{die} Herren
               haben ſchon ihren Zopf. Wir brauchen ja doch »unſer« Blatt! – Ich will übrigens das
                  »Hi{\geminationm}elbett\pwindex{Himmelbett@\emph{Das Himmelbett}|pw}« an
                  \textsc{Bölsche}\pwindex{Boelsche, Wilhelm 02.01.1861 – 31.08.1939@\textsc{Bölsche, Wilhelm} (02.01.1861 – 31.08.1939), \emph{Schriftsteller/Schriftstellerin, Publizist/Publizistin}|pw}{ }ſchicken. – Geſtern ſprach ich Herrn \textsc{Leo Geiringer}\pwindex{Geiringer, Leopold 27.06.1851 – 29.05.1900@\textsc{Geiringer, Leopold} (27.06.1851 – 29.05.1900), \emph{Schriftsteller/Schriftstellerin, Dramaturg/Dramaturgin}|pw}, den Dramaturgen des Dtſch Volksth.\oindex{Volkstheater@\textbf{Volkstheater}, \emph{Theater (K.THE)}|pw}, der
               mich um mein Märchen\pwindex{Maerchen. Schauspiel in drei Aufzuegen@\emph{Das Märchen. Schauspiel in drei Aufzügen}|pw} gebeten hatte – ich ſandte
               es ihm {\pb}als »Privatmann«. – Er ſagte: »Wirklich ein
               hübſches Talent, ich muſs nur bedauern, daß Sie ſich \uline{dieſer} Richtung zugewandt haben!{[}«{]}\pend
           
\pstart
           \uline{Ich}{ }{\dotstwo}?{\dotsfour}! – ?\pend
           
\pstart
           \uline{Er}. Nun ja, Sie werden doch zugeben, der Schluſs ist
                  unbefriedigend{\dots}\pend
           
\pstart
           \uline{Ich}.{ }{\dotstwo}!{\dots}in den Charakteren{\dots}\pend
           
\pstart
           \uline{Er}. Die Erfahrung lehrt nun einmal, daß unſer
               Publicum \textsc{etc etc}.\pend
           
\pstart
           {\pb}\uline{Ich.}{ }{\dots}{ }Wildente\pwindex{Wildente. Schauspiel in fuenf Akten@\emph{Die Wildente. Schauspiel in fünf Akten}|pw}!!{\dotsfour}\pend
           
\pstart
           \uline{Er}. Den Einfluſs merkt man auch deutlich {\dotstwo} ich will nicht gerade ſagen, daß Sie abgeſchrieben
                  haben{\dotsfour}\pend
           
\pstart
           \label{T_L00090-1v}\edtext{!!.Ich.}{\lemma{\textnormal{\emph{!!.Ich.}}}\Cendnote{\textnormal{kopfüber zum Text}}}\label{T_L00090-1}\pend
           
\pstart
           Herzlichſt der Ihre, und ko{\geminationm}en Sie Dienſtag gef. zur \textsc{Bahr}\pwindex{Bahr, Hermann 19.07.1863 – 15.01.1934@\textsc{Bahr, Hermann} (19.07.1863 – 15.01.1934), \emph{Schriftsteller/Schriftstellerin, Kritiker/Kritikerin}|pw}’ſchen \label{K_L00090-1v}\edtext{Myſtik}{\lemma{\textnormal{\emph{Myſtik}}}\Cendnote{\textnormal{Gemeint ist Bahrs\pwindex{Bahr, Hermann 19.07.1863 – 15.01.1934@\textsc{Bahr, Hermann} (19.07.1863 – 15.01.1934), \emph{Schriftsteller/Schriftstellerin, Kritiker/Kritikerin}|pwk} Vortrag über »Moderne Mystik«, den er am
                     29. 3. 1892 bei einer Veranstaltung der \emph{Freien Bühne}\orgindex{»Freie Buehne« Verein fuer moderne Literatur@»Freie Bühne« Verein für moderne Literatur|pwk} hielt.}}}\label{K_L00090-1}!\pend
           \selectlanguage{ngerman}\endnumbering\briefempfaengerindex{Hofmannsthal, Hugo von@\textsc{Hofmannsthal, Hugo von}!zzzSchnitzler, Arthur@\emph{von Arthur Schnitzler}!1892-03-272@{27. 3. 1892}|)be}\mylabel{L00090h}  \normalsize

\doendnotes{C}
\bigskip
\vfill

\clearpage

\footnotesize

\lohead{\textsc{register}}

% Definiere theindex-Environment komplett neu ohne reledmac
\makeatletter
\renewenvironment{theindex}{%
  \section*{\indexname}%
  \setlength{\parindent}{0pt}%
  \setlength{\parskip}{0pt plus 0.3pt}%
  \let\item\@idxitem
}{%
  \clearpage
}
\makeatother

\IfFileExists{\jobname-pw.ind}{\input{\jobname-pw.ind}}{}

\end{document}

      