%% latex-leseansicht-vorspann.tex
%% Vorspann für die Leseansicht.
%% Lädt die gemeinsame Datei latex-vorspann.tex mit nicht gesetztem Schalter.

\newif\ifkorrekturansicht
\korrekturansichtfalse

\input{../tex-inputs/latex-vorspann}


         
         \renewcommand{\erwaehntePersonen}{Personen: Richard Beer-Hofmann, Arthur Kaufmann, Emil Redlich}
         \renewcommand{\erwaehnteOrte}{Orte: Bad Gastein, Bad Ischl, Grazer Straße, Purkersdorf, Sternwartestraße 71, Wien}
         \renewcommand{\erwaehnteWerke}{Werke: [Philosophisches Werk]}
               \section[Arthur Schnitzler an Richard Beer-Hofmann, 29. 6. 1917]{ Arthur Schnitzler an Richard Beer-Hofmann, 29. 6. 1917}\nopagebreak\mylabel{v}\rehead{ }\begin{ledgroupsized}[t]{13cm}\normalsize\beginnumbering\briefempfaengerindex{Beer-Hofmann, Richard@\textsc{Beer-Hofmann, Richard}!zzzSchnitzler, Arthur@\emph{von Arthur Schnitzler}!1917-06-291@{29. 6. 1917}|(be} \toendnotes[C]{\smallbreak\pagebreak[2]} \Standort{YCGL, MSS 31.}
\physDesc{Kartenbrief, 1409 Zeichen
\newline{}Handschrift: Bleistift, lateinische Kurrent
\newline{}Versand: Stempel: »\nobreak{}Wien, 30 VI 17\nobreak{}«.  
\newline{}Beer-Hofmann: mit blauem Buntstift Empfang und Beantwortung vermerkt:
                                    »E. B. 19./VII 17« }\buchAbdrucke{\weitereDrucke{Arthur Schnitzler, Richard Beer-Hofmann: \emph{Briefwechsel 1891–1931}. Hg. Konstanze Fliedl. Wien, Zürich: \emph{Europaverlag} 1992, S. 223.} }\toendnotes[C]{\smallbreak}\pstart{}{\pb}Abſ. Schnitzler, Wien XVIII Sternwartestr 71\oindex{Sternwartestrasse 71@\textbf{Sternwartestraße 71}|pw}.\pend{}{\bigskip}\pstart{}Herrn Doctor Richard Beer\substVorne{}\textsuperscript{h}\substDazwischen{}-H\substHinten{}ofmann\pend{}\pstart{}Bad Ischl\oindex{Bad Ischl@\textbf{Bad Ischl}|pw}\pend{}\pstart{}Grazerstr. 56\oindex{Grazer Strasse@\textbf{Grazer Straße}|pw}\pend{}{\bigskip}\pstart
           \raggedleft{}{\pb}Wien\oindex{Wien@\textbf{Wien}|pw}, 29. 6. 1917\pend
           \pstart
           lieber Richard, ich nehme an es wird Sie interessiren, \label{K_L02265-1v}\edtext{näheres über Arthur Kfm.\pwindex{Kaufmann, Arthur 04.04.1872 – 25.07.1938@\textsc{Kaufmann, Arthur} (04.04.1872 – 25.07.1938), \emph{Rechtswissenschaftler, Privatgelehrte, Privatier}|pw}}{\lemma{\textnormal{\emph{näheres über Arthur Kfm.}}}\Cendnote{\textnormal{Vgl. A. S.: \emph{Tagebuch}, 24. 6. 1917.
               }}}\label{K_L02265-1h} zu erfahren. Vorgestern war \introOben{}Prof.\introOben{}{ }Redlich\pwindex{Redlich, Emil 18.01.1866 – 07.06.1930@\textsc{Redlich, Emil} (18.01.1866 – 07.06.1930), \emph{Psychiater, Neurologe, Arzt}|pw} bei ihm; er stellte die Diagnose \introOben{}(ich wohnte bei)\introOben{}, die wir schon nach den 2 Briefen, die ich
               von A. K.\pwindex{Kaufmann, Arthur 04.04.1872 – 25.07.1938@\textsc{Kaufmann, Arthur} (04.04.1872 – 25.07.1938), \emph{Rechtswissenschaftler, Privatgelehrte, Privatier}|pw} nach Gastein\oindex{Bad Gastein@\textbf{Bad Gastein}|pw} erhalten hatte höchst wahrscheinlich war: (acute \introOben{}Manie\introOben{}) \uline{Manie}, »Hypomanie« wie er hinzu
               setzte – eine leichtere Form \introOben{}(Paranoia – keine Spur!)\introOben{}. Im
               19. Lebensjahr hat K.\pwindex{Kaufmann, Arthur 04.04.1872 – 25.07.1938@\textsc{Kaufmann, Arthur} (04.04.1872 – 25.07.1938), \emph{Rechtswissenschaftler, Privatgelehrte, Privatier}|pw} einen ähnlichen Anfall
               gehabt, – damals trat die Krankheit als schwere Melancholie auf; – da der
               Zwischenraum ein so langer war – ist die Prognose günstig – we{\geminationn}{ }\introOben{}auch\introOben{} natürlich eine Wiederkehr in absehbarer Zeit keineswegs
               ausgeschlossen erscheint. Subjectiv befindet sich A.\pwindex{Kaufmann, Arthur 04.04.1872 – 25.07.1938@\textsc{Kaufmann, Arthur} (04.04.1872 – 25.07.1938), \emph{Rechtswissenschaftler, Privatgelehrte, Privatier}|pw} wohl – nicht mehr \uline{zu wohl} – wie uns beim
               ersten Besuch \introOben{}in Purkersdorf\oindex{Purkersdorf@\textbf{Purkersdorf}|pw}\introOben{} beinah vorkam; kein zwanghaftes Denken mehr, kein Grübeln, – er \uline{will} gesund werden, möglichst bald und vollko{\geminationm}en, – vor allem um sein Werk\pwindex{Kaufmann, Arthur 04.04.1872 – 25.07.1938@\textsc{Kaufmann, Arthur} (04.04.1872 – 25.07.1938), \emph{Rechtswissenschaftler, Privatgelehrte, Privatier}!Philosophisches Werk]@\strich\emph{[Philosophisches Werk]}|pwv} in aller Ruhe schreiben zu können. Wir wollen hoffen –
               und ich halte es für sehr möglich – daß er gerade in der Hauptsache gar nicht
               verrückt war – denn wer sollte die Philosophie weiter bringen können als er –
               insbesondre, da er die schöne Absicht hat sie überflüssig zu machen. Uns gehts
                  \label{T_L02265-1v}\edtext{recht gut, Gastein\oindex{Bad Gastein@\textbf{Bad Gastein}|pw} war sehr erholend, ich arbeite und
               wünschte ähnliches und andres auch von Ihnen zu hören. Wir grüßen}{\lemma{\textnormal{\emph{recht … grüßen}}}\Cendnote{\textnormal{am Seitenkopf, verkehrt zum
                  Text}}}\label{T_L02265-1h}{ }\label{T_L02265-2v}\edtext{Sie herzlichst Ihr
                  \spacefill\mbox{Arthur}}{\lemma{\textnormal{\emph{Sie … Arthur}}}\Cendnote{\textnormal{weiter am Seitenrand}}}\label{T_L02265-2h}\pend
           
         
         \endnumbering\mylabel{h}\end{ledgroupsized}  \newcommand{\dateiname}{L02265}\newcommand{\titel}{Arthur Schnitzler an Richard Beer-Hofmann, 29. 6. 1917}\newcommand{\editorInnen}{Martin Anton Müller und Gerd-Hermann Susen}%% latex-leseansicht-abspann.tex
%% Abspann für die Leseansicht.
%% Der Schalter \ifkorrekturansicht ist bereits durch den Vorspann gesetzt.

%% latex-abspann.tex
%% Gemeinsamer Abspann für Korrekturansicht und Leseansicht.
%% Setzt den Schalter \ifkorrekturansicht voraus (gesetzt in den
%% einbindenden Dateien latex-korrekturansicht-abspann.tex bzw.
%% latex-leseansicht-abspann.tex).
%% ---------------------------------------------------------------

\normalsize

% Das esempio-Environment wird nur in der Leseansicht benötigt
\ifkorrekturansicht\else
\newenvironment{esempio}[3]%
{
    \vspace{1.5ex}
    \rlap{\underline{#1}}
    \par
    \setlength{\parindent}{0cm}
    \nopagebreak
    \leftskip=#2cm
    \rightskip=#3cm
}
{
    \par
}
\fi

\doendnotes{C}
\bigskip
\vfill

\clearpage

\footnotesize

\ifkorrekturansicht
  \lohead{\textsc{register}}
\fi

% theindex-Environment neu definieren ohne reledmac
\makeatletter
\renewenvironment{theindex}{%
  \ifkorrekturansicht
    \section*{\indexname}%
  \else
    \subsubsection*{Index der erwähnten Entitäten}%
  \fi
  \setlength{\parindent}{0pt}%
  \setlength{\parskip}{0pt plus 0.3pt}%
  \let\item\@idxitem
}{%
  \ifkorrekturansicht\clearpage\fi
}
\makeatother

\IfFileExists{\jobname-pw.ind}{\input{\jobname-pw.ind}}{}

% Quellenangabe nur in der Leseansicht
\ifkorrekturansicht\else
% Fallback-Definitionen, falls die .tex-Datei \titel etc. nicht gesetzt hat
\providecommand{\titel}{}
\providecommand{\editorInnen}{}
\providecommand{\dateiname}{\jobname}

\vspace{3cm}

\vfill

\footnotesize
\textsc{Quelle}: \titel. Herausgegeben von {\editorInnen}. In: \emph{Arthur Schnitzler: Briefwechsel mit Autorinnen und Autoren}.
 Digitale Edition, https://schnitzler-briefe.acdh.oeaw.ac.at/{\dateiname}.html (Stand \today)
\fi

\end{document}


      