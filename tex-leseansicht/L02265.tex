%% latex-korrekturansicht-vorspann.tex
%% Vorspann für die Korrekturansicht.
%% Lädt die gemeinsame Datei latex-vorspann.tex mit gesetztem Schalter.

\newif\ifkorrekturansicht
\korrekturansichttrue

\input{../tex-inputs/latex-vorspann}


\section[Arthur Schnitzler an Richard Beer-Hofmann, 29. 6. 1917]{L02265 Arthur Schnitzler an Richard Beer-Hofmann, 29. 6. 1917}
\nopagebreak\mylabel{L02265v}
\rehead{ }\normalsize\beginnumbering\briefempfaengerindex{Beer-Hofmann, Richard@\textsc{Beer-Hofmann, Richard}!zzzSchnitzler, Arthur@\emph{von Arthur Schnitzler}!1917-06-291@{29. 6. 1917}|(be}
\toendnotes[C]{\smallbreak\pagebreak[2]}\Standort{YCGL, MSS 31.}
\physDesc{Kartenbrief, 1409 Zeichen
\newline{}Handschrift: Bleistift, lateinische Kurrent
\newline{}Versand: Stempel: »\nobreak{}Wien, 30 VI 17\nobreak{}«.  
\newline{}Beer-Hofmann: mit blauem Buntstift Empfang und Beantwortung vermerkt:
                                    »E. B. 19./VII 17« }
\buchAbdrucke{\weitereDrucke{Arthur Schnitzler, Richard Beer-Hofmann: \emph{Briefwechsel 1891–1931}. Wien, Zürich: \emph{Europaverlag} 1992, S. 223.} }\toendnotes[C]{\smallbreak}\pstart{}{\pb}Abſ. Schnitzler, Wien XVIII Sternwartestr 71\oindex{Sternwartestrasse 71@\textbf{Sternwartestraße 71}, \emph{Wohngebäude (K.WHS)}|pw}.\pend{}{\bigskip}\pstart{}Herrn Doctor Richard Beer\substVorne{}\textsuperscript{h}\substDazwischen{}-H\substHinten{}ofmann\pend{}\pstart{}Bad Ischl\oindex{Bad Ischl@\textbf{Bad Ischl}, \emph{P.PPL}|pw}\pend{}\pstart{}Grazerstr. 56\oindex{Grazer Strasse [Bad Ischl]@\textbf{Grazer Straße [Bad Ischl]}, \emph{Straße (K.STR)}|pw}\pend{}{\bigskip}\vspace{1em}
\pstart
           \raggedleft{}{\pb}Wien\oindex{Wien@\textbf{Wien}, \emph{A.ADM2}|pw}, 29. 6. 1917\pend
           \vspace{0.5em}
\pstart
           lieber Richard, ich nehme an es wird Sie interessiren, \label{K_L02265-1v}\edtext{näheres über Arthur Kfm.\pwindex{Kaufmann, Arthur 04.04.1872 – 25.07.1938@\textsc{Kaufmann, Arthur} (04.04.1872 – 25.07.1938), \emph{Rechtswissenschaftler/Rechtswissenschaftlerin, Privatgelehrte/Privatgelehrte, Privatier/Privatière}|pw}}{\lemma{\textnormal{\emph{näheres über Arthur Kfm.}}}\Cendnote{\textnormal{Vgl. A. S.: \emph{Tagebuch}, 24. 6. 1917.
               }}}\label{K_L02265-1} zu erfahren. Vorgestern war \introOben{}Prof.\introOben{}{ }Redlich\pwindex{Redlich, Emil 18.01.1866 – 07.06.1930@\textsc{Redlich, Emil} (18.01.1866 – 07.06.1930), \emph{Psychiater/Psychiaterin, Neurologe/Neurologin, Arzt/Ärztin}|pw} bei ihm; er stellte die Diagnose \introOben{}(ich wohnte bei)\introOben{}, die wir schon nach den 2 Briefen, die ich
               von A. K.\pwindex{Kaufmann, Arthur 04.04.1872 – 25.07.1938@\textsc{Kaufmann, Arthur} (04.04.1872 – 25.07.1938), \emph{Rechtswissenschaftler/Rechtswissenschaftlerin, Privatgelehrte/Privatgelehrte, Privatier/Privatière}|pw} nach Gastein\oindex{Bad Gastein@\textbf{Bad Gastein}, \emph{P.PPLA3}|pw} erhalten hatte höchst wahrscheinlich war: (acute \introOben{}Manie\introOben{}) \uline{Manie}, »Hypomanie« wie er hinzu
               setzte – eine leichtere Form \introOben{}(Paranoia – keine Spur!)\introOben{}. Im
               19. Lebensjahr hat K.\pwindex{Kaufmann, Arthur 04.04.1872 – 25.07.1938@\textsc{Kaufmann, Arthur} (04.04.1872 – 25.07.1938), \emph{Rechtswissenschaftler/Rechtswissenschaftlerin, Privatgelehrte/Privatgelehrte, Privatier/Privatière}|pw} einen ähnlichen Anfall
               gehabt, – damals trat die Krankheit als schwere Melancholie auf; – da der
               Zwischenraum ein so langer war – ist die Prognose günstig – we{\geminationn}{ }\introOben{}auch\introOben{} natürlich eine Wiederkehr in absehbarer Zeit keineswegs
               ausgeschlossen erscheint. Subjectiv befindet sich A.\pwindex{Kaufmann, Arthur 04.04.1872 – 25.07.1938@\textsc{Kaufmann, Arthur} (04.04.1872 – 25.07.1938), \emph{Rechtswissenschaftler/Rechtswissenschaftlerin, Privatgelehrte/Privatgelehrte, Privatier/Privatière}|pw} wohl – nicht mehr \uline{zu wohl} – wie uns beim
               ersten Besuch \introOben{}in Purkersdorf\oindex{Purkersdorf@\textbf{Purkersdorf}, \emph{A.ADM3}|pw}\introOben{} beinah vorkam; kein zwanghaftes Denken mehr, kein Grübeln, – er \uline{will} gesund werden, möglichst bald und vollko{\geminationm}en, – vor allem um sein Werk\pwindex{Philosophisches Werk]@\emph{[Philosophisches Werk]}|pwv} in aller Ruhe schreiben zu können. Wir wollen hoffen –
               und ich halte es für sehr möglich – daß er gerade in der Hauptsache gar nicht
               verrückt war – denn wer sollte die Philosophie weiter bringen können als er –
               insbesondre, da er die schöne Absicht hat sie überflüssig zu machen. Uns gehts
                  \label{T_L02265-1v}\edtext{recht gut, Gastein\oindex{Bad Gastein@\textbf{Bad Gastein}, \emph{P.PPLA3}|pw} war sehr erholend, ich arbeite und
               wünschte ähnliches und andres auch von Ihnen zu hören. Wir grüßen}{\lemma{\textnormal{\emph{recht … grüßen}}}\Cendnote{\textnormal{am Seitenkopf, verkehrt zum
                  Text}}}\label{T_L02265-1}{ }\label{T_L02265-2v}\edtext{Sie herzlichst Ihr
                  \spacefill\mbox{Arthur}}{\lemma{\textnormal{\emph{Sie … Arthur}}}\Cendnote{\textnormal{weiter am Seitenrand}}}\label{T_L02265-2}\pend
           \selectlanguage{ngerman}\endnumbering\briefempfaengerindex{Beer-Hofmann, Richard@\textsc{Beer-Hofmann, Richard}!zzzSchnitzler, Arthur@\emph{von Arthur Schnitzler}!1917-06-291@{29. 6. 1917}|)be}\mylabel{L02265h}  \normalsize

\doendnotes{C}
\bigskip
\vfill

\clearpage

\footnotesize

\lohead{\textsc{register}}

% Definiere theindex-Environment komplett neu ohne reledmac
\makeatletter
\renewenvironment{theindex}{%
  \section*{\indexname}%
  \setlength{\parindent}{0pt}%
  \setlength{\parskip}{0pt plus 0.3pt}%
  \let\item\@idxitem
}{%
  \clearpage
}
\makeatother

\IfFileExists{\jobname-pw.ind}{\input{\jobname-pw.ind}}{}

\end{document}

      