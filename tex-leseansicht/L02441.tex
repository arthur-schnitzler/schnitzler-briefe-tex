%% latex-leseansicht-vorspann.tex
%% Vorspann für die Leseansicht.
%% Lädt die gemeinsame Datei latex-vorspann.tex mit nicht gesetztem Schalter.

\newif\ifkorrekturansicht
\korrekturansichtfalse

\input{../tex-inputs/latex-vorspann}


\section[Gertrud Rung an Arthur Schnitzler, 24. 5. 1925]{L02441 Gertrud Rung an Arthur Schnitzler, 24. 5. 1925}
\nopagebreak\mylabel{L02441v}
\rehead{ }\normalsize\beginnumbering\briefempfaengerindex{Schnitzler, Arthur@\textsc{Schnitzler, Arthur}!zzzRung, Gertrud@\emph{von Gertrud Rung}!1925-05-241@{24. 5. 1925}|(be}
\toendnotes[C]{\smallbreak\pagebreak[2]}
\correspDesc{Versand  durch Gertrud Rung am 24. 5. 1925 in Salzburg
\newline{}Erhalt  durch Arthur Schnitzler im Zeitraum [25. 5. 1925
                  – 29. 5. 1925?] in Wien}\toendnotes[C]{\smallbreak}
\Standort{CUL, Schnitzler, B 17.}
\physDesc{Brief, 1 Blatt, 2 Seiten, 1248 Zeichen
\newline{}Handschrift: schwarze Tinte, lateinische Kurrent
\newline{}Schnitzler: mit Bleistift beschriftet: »\noindent{}\textsc{Brandes}{ / }\textsc{(Rung}{[}){]}« 
\newline{}Ordnung: mit Bleistift von unbekannter Hand nummeriert:
                                    »58« }
\buchAbdrucke{\weitereDrucke{Georg Brandes, Arthur Schnitzler: \emph{Ein Briefwechsel}. Herausgegeben von Kurt Bergel. Bern: \emph{Francke} 1956, S. 146.} }
\pstart
           \raggedleft{}{\pb}Oesterreichischer Hof, Salzburg\oindex{Österreichischer Hof@\textbf{Österreichischer Hof}, \emph{Hotel}|pw}{\\}24/5. 25\pend
           
\pstart{}Hochverehrter Herr Dr Schnitzler.\pend\vspace{0.5em}
\pstart
           Dr Brandes\pwindex{Brandes, Georg 4.\,2.\,1842 Kopenhagen – 19.\,2.\,1927 ebd.@\textsc{Brandes, Georg} (4.\,2.\,1842 Kopenhagen – 19.\,2.\,1927 ebd.)|pw} dankt Ihnen ergebenst für Ihren
               freundlichen Brief. Wie Sie wahrscheinlich aus den Zeitungen erfahren haben,
               erkrankte Dr Brandes\pwindex{Brandes, Georg 4.\,2.\,1842 Kopenhagen – 19.\,2.\,1927 ebd.@\textsc{Brandes, Georg} (4.\,2.\,1842 Kopenhagen – 19.\,2.\,1927 ebd.)|pw} gleich nach seiner
               Ankunft hier an Bronchitis, und es sah für ein paar Tage recht ernst aus, aber
               glücklicherweise ist es gut gegangen, die Krankheit ist beinahe vorüber und Morgen
                  {\pb}wird er, wenn das Wetter schön
               bleibt, spazieren fahren.\pend
           
\pstart
           Mit Ausnahme der ersten Woche hat die Sonne jeden Tag von einem wolkenlosen Himmel
               niedergeschienen, und Salzburg\oindex{Salzburg@\textbf{Salzburg}, \emph{Verwaltungsgebiet}|pw} hat sich in aller
               ihrer Schönheit dargeboten; die Stadt ist ja entzückend und ich hoffe, daß Dr Brandes\pwindex{Brandes, Georg 4.\,2.\,1842 Kopenhagen – 19.\,2.\,1927 ebd.@\textsc{Brandes, Georg} (4.\,2.\,1842 Kopenhagen – 19.\,2.\,1927 ebd.)|pw} bald im Stande sein wird kleinere
               Ausflüge zu machen und etwas von der Schönheit zu genießen.\pend
           
\pstart
           Dr Brandes beauftragt mich Sie {\pb}zu
               sagen, daß auch für ihn war das Zusammensein mit Ihnen, hochverehrter Herr Doktor,
               eine große Freude, und daß er sich bei Ihnen außerordentlich wohl befunden habe. Er
               würde sich sehr freuen wenn Sie, wie Sie andeuteten, im Herbst nach Kopenhagen\oindex{Kopenhagen@\textbf{Kopenhagen}, \emph{Hauptstadt}|pw} kämen.\pend
           
\pstart
           Ich möchte gern die Gelegenheit benützen und Ihnen, verehrter und lieber Herr Doktor,
               vom Herzen danken für die schönen Stunden die ich bei Ihnen verbrachte.\pend
           
\pstart
           Mit besten Grüßen von Dr Brandes\pwindex{Brandes, Georg 4.\,2.\,1842 Kopenhagen – 19.\,2.\,1927 ebd.@\textsc{Brandes, Georg} (4.\,2.\,1842 Kopenhagen – 19.\,2.\,1927 ebd.)|pw}
                  und\hspace*{2.5em}Ihrer{\\[\baselineskip]}\spacefill\mbox{Gertrud Rung}\pend
           \leftskip=0em{}\selectlanguage{ngerman}\endnumbering\briefempfaengerindex{Schnitzler, Arthur@\textsc{Schnitzler, Arthur}!zzzRung, Gertrud@\emph{von Gertrud Rung}!1925-05-241@{24. 5. 1925}|)be}\mylabel{L02441h}  \newcommand{\dateiname}{L02441}\newcommand{\titel}{Gertrud Rung an Arthur Schnitzler, 24. 5. 1925}\newcommand{\editorInnen}{Martin Anton Müller und Gerd-Hermann Susen}%% latex-leseansicht-abspann.tex
%% Abspann für die Leseansicht.
%% Der Schalter \ifkorrekturansicht ist bereits durch den Vorspann gesetzt.

%% latex-abspann.tex
%% Gemeinsamer Abspann für Korrekturansicht und Leseansicht.
%% Setzt den Schalter \ifkorrekturansicht voraus (gesetzt in den
%% einbindenden Dateien latex-korrekturansicht-abspann.tex bzw.
%% latex-leseansicht-abspann.tex).
%% ---------------------------------------------------------------

\normalsize

% Das esempio-Environment wird nur in der Leseansicht benötigt
\ifkorrekturansicht\else
\newenvironment{esempio}[3]%
{
    \vspace{1.5ex}
    \rlap{\underline{#1}}
    \par
    \setlength{\parindent}{0cm}
    \nopagebreak
    \leftskip=#2cm
    \rightskip=#3cm
}
{
    \par
}
\fi

\doendnotes{C}
\bigskip
\vfill

\clearpage

\footnotesize

\ifkorrekturansicht
  \lohead{\textsc{register}}
\fi

% theindex-Environment neu definieren ohne reledmac
\makeatletter
\renewenvironment{theindex}{%
  \ifkorrekturansicht
    \section*{\indexname}%
  \else
    \subsubsection*{Index der erwähnten Entitäten}%
  \fi
  \setlength{\parindent}{0pt}%
  \setlength{\parskip}{0pt plus 0.3pt}%
  \let\item\@idxitem
}{%
  \ifkorrekturansicht\clearpage\fi
}
\makeatother

\IfFileExists{\jobname-pw.ind}{\input{\jobname-pw.ind}}{}

% Quellenangabe nur in der Leseansicht
\ifkorrekturansicht\else
% Fallback-Definitionen, falls die .tex-Datei \titel etc. nicht gesetzt hat
\providecommand{\titel}{}
\providecommand{\editorInnen}{}
\providecommand{\dateiname}{\jobname}

\vspace{3cm}

\vfill

\footnotesize
\textsc{Quelle}: \titel. Herausgegeben von {\editorInnen}. In: \emph{Arthur Schnitzler: Briefwechsel mit Autorinnen und Autoren}.
 Digitale Edition, https://schnitzler-briefe.acdh.oeaw.ac.at/{\dateiname}.html (Stand \today)
\fi

\end{document}


