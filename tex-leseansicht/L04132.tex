%% latex-leseansicht-vorspann.tex
%% Vorspann für die Leseansicht.
%% Lädt die gemeinsame Datei latex-vorspann.tex mit nicht gesetztem Schalter.

\newif\ifkorrekturansicht
\korrekturansichtfalse

\input{../tex-inputs/latex-vorspann}


\section[Arthur Schnitzler an Gustav Schwarzkopf, 1. 1. 1899]{L04132 Arthur Schnitzler an Gustav Schwarzkopf, 1. 1. 1899}
\nopagebreak\mylabel{L04132v}
\rehead{ }\normalsize\beginnumbering\briefempfaengerindex{Schwarzkopf, Gustav@\textsc{Schwarzkopf, Gustav}!zzzSchnitzler, Arthur@\emph{von Arthur Schnitzler}!1899-01-011@{1. 1. 1899}|(be}
\toendnotes[C]{\smallbreak\pagebreak[2]}
\correspDesc{Versand  durch Arthur Schnitzler am 1. 1. 1899 in Wien
\newline{}Erhalt  durch Gustav Schwarzkopf im Zeitraum [1. 1. 1899 – 4. 1. 1899?] in Wien}\toendnotes[C]{\smallbreak}
\Standort{CUL, Schnitzler, B 96.}
\physDesc{Karte, 237 Zeichen
\newline{}Handschrift: Bleistift, deutsche Kurrent}\toendnotes[C]{\smallbreak}
\pstart
           \noindent{}{\pb}Lieber Guſtav; es bleibt dabei. Sie{ }ſind{ }ſo gut, Samſtag{ }Abend um 8 (nicht ſpäter){ }ſagen wir gleich: ¾ 8 bei mir \textsc{resp}
      uns zu eſſen, da{\geminationn} fahren wir{\dots} »\label{K_L04132-1v}\edtext{alſo
               ich bitte, ich ſage nichts als: in den zweiten Be{\pb}zirk\oindex{Wien@\textbf{Wien}!II., Leopoldstadt@\textbf{II., Leopoldstadt}!Budapest-Orpheum@\textbf{Budapest-Orpheum}, \emph{Veranstaltungsgebäude}|pwv}\eventindex{Budapest-Orpheum@\textbf{Budapest-Orpheum}!Besuch des Budapest-Orpheums, 7.1.1899@Besuch des Budapest-Orpheums, 7.1.1899|pwv}}{\lemma{\textnormal{\emph{also … Bezirk}}}\Cendnote{\textnormal{Die Quelle des Zitats ist nicht ermittelt/ermittelbar. Am
                  7. 1. 1899 waren Schnitzler, Schwarzkopf\pwindex{Schwarzkopf, Gustav 7.\,11.\,1853 Wien – 13.\,11.\,1939 ebd.@\textsc{Schwarzkopf, Gustav} (7.\,11.\,1853 Wien – 13.\,11.\,1939 ebd.), \emph{Schriftsteller}|pwk} und
                  Richard Beer-Hofmann\pwindex{Beer-Hofmann, Richard 11.\,7.\,1866 Wien – 26.\,9.\,1945 New York City@\textsc{Beer-Hofmann, Richard} (11.\,7.\,1866 Wien – 26.\,9.\,1945 New York City), \emph{Schriftsteller}|pwk} im Budapest-Orpheum\oindex{Wien@\textbf{Wien}!II., Leopoldstadt@\textbf{II., Leopoldstadt}!Budapest-Orpheum@\textbf{Budapest-Orpheum}, \emph{Veranstaltungsgebäude}|pwk}, das im 2. Wiener 
                  Gemeindebezirk\oindex{II., Leopoldstadt@\textbf{II., Leopoldstadt}, \emph{Verwaltungsgebiet}|pwk} angesiedelt war.}}}\label{K_L04132-1}.«\pend
           
\pstart
           Herzlich grüßend{\\[\baselineskip]} Ihr{\\[\baselineskip]}\spacefill\mbox{Arthur S}\pend
           \leftskip=0em{}
\pstart
           1. 1. 99.\pend
           \selectlanguage{ngerman}\endnumbering\briefempfaengerindex{Schwarzkopf, Gustav@\textsc{Schwarzkopf, Gustav}!zzzSchnitzler, Arthur@\emph{von Arthur Schnitzler}!1899-01-011@{1. 1. 1899}|)be}\mylabel{L04132h}
\begin{anhang}
\end{anhang}\newcommand{\dateiname}{L04132}\newcommand{\titel}{Arthur Schnitzler an Gustav Schwarzkopf, 1. 1. 1899}\newcommand{\editorInnen}{Herausgegeben von Jahnke, SelmaMüller, Martin Anton}%% latex-leseansicht-abspann.tex
%% Abspann für die Leseansicht.
%% Der Schalter \ifkorrekturansicht ist bereits durch den Vorspann gesetzt.

%% latex-abspann.tex
%% Gemeinsamer Abspann für Korrekturansicht und Leseansicht.
%% Setzt den Schalter \ifkorrekturansicht voraus (gesetzt in den
%% einbindenden Dateien latex-korrekturansicht-abspann.tex bzw.
%% latex-leseansicht-abspann.tex).
%% ---------------------------------------------------------------

\normalsize

% Das esempio-Environment wird nur in der Leseansicht benötigt
\ifkorrekturansicht\else
\newenvironment{esempio}[3]%
{
    \vspace{1.5ex}
    \rlap{\underline{#1}}
    \par
    \setlength{\parindent}{0cm}
    \nopagebreak
    \leftskip=#2cm
    \rightskip=#3cm
}
{
    \par
}
\fi

\doendnotes{C}
\bigskip
\vfill

\clearpage

\footnotesize

\ifkorrekturansicht
  \lohead{\textsc{register}}
\fi

% theindex-Environment neu definieren ohne reledmac
\makeatletter
\renewenvironment{theindex}{%
  \ifkorrekturansicht
    \section*{\indexname}%
  \else
    \subsubsection*{Index der erwähnten Entitäten}%
  \fi
  \setlength{\parindent}{0pt}%
  \setlength{\parskip}{0pt plus 0.3pt}%
  \let\item\@idxitem
}{%
  \ifkorrekturansicht\clearpage\fi
}
\makeatother

\IfFileExists{\jobname-pw.ind}{\input{\jobname-pw.ind}}{}

% Quellenangabe nur in der Leseansicht
\ifkorrekturansicht\else
% Fallback-Definitionen, falls die .tex-Datei \titel etc. nicht gesetzt hat
\providecommand{\titel}{}
\providecommand{\editorInnen}{}
\providecommand{\dateiname}{\jobname}

\vspace{3cm}

\vfill

\footnotesize
\textsc{Quelle}: \titel. Herausgegeben von {\editorInnen}. In: \emph{Arthur Schnitzler: Briefwechsel mit Autorinnen und Autoren}.
 Digitale Edition, https://schnitzler-briefe.acdh.oeaw.ac.at/{\dateiname}.html (Stand \today)
\fi

\end{document}


