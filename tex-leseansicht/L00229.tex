%% latex-leseansicht-vorspann.tex
%% Vorspann für die Leseansicht.
%% Lädt die gemeinsame Datei latex-vorspann.tex mit nicht gesetztem Schalter.

\newif\ifkorrekturansicht
\korrekturansichtfalse

\input{../tex-inputs/latex-vorspann}


\section[Wilhelm Bölsche an Arthur Schnitzler, 1. 7. 1893]{L00229 Wilhelm Bölsche an Arthur Schnitzler, 1. 7. 1893}
\nopagebreak\mylabel{L00229v}
\rehead{ }\normalsize\beginnumbering\briefempfaengerindex{Schnitzler, Arthur@\textsc{Schnitzler, Arthur}!zzzBölsche, Wilhelm@\emph{von Wilhelm Bölsche}!1893-07-012@{1. 7. 1893}|(be}
\toendnotes[C]{\smallbreak\pagebreak[2]}
\correspDesc{Versand  durch Wilhelm Bölsche am 1. 7. 1893 in Berlin
\newline{}Erhalt  durch Arthur Schnitzler im Zeitraum [2. 7. 1893
                  – 6. 7. 1893?] in Wien}\toendnotes[C]{\smallbreak}
\Standort{DLA, A:Schnitzler, HS.NZ85.1.2577,8.}
\physDesc{Brief, 1 Blatt, 1 Seite, 590 Zeichen
\newline{}Handschrift: schwarze Tinte, deutsche Kurrent
\newline{}Schnitzler: mit rotem Buntstift nummeriert: »9« }
\buchAbdrucke{\weitereDrucke{Wilhelm Bölsche: \emph{Briefwechsel. Mit Autoren der Freien Bühne}. Herausgegeben von Gerd-Hermann Susen. Berlin: \emph{Weidler} 2010, S. 690 (Werke und Briefe. Wissenschaftliche Ausgabe, Briefe I).} }\toendnotes[C]{\smallbreak}
\pstart
           \raggedleft{}{\pb}\textcolor{gray}{\textbf{\textit{Wilhelm Bölsche}}}\pend
           
\pstart
           \raggedleft{}\textcolor{gray}{\textbf{\textit{Friedrichshagen\oindex{Friedrichshagen@\textbf{Friedrichshagen}, \emph{Ehemaliger Ort}|pw}}}}\pend
           
\pstart
           \raggedleft{}1. VII. 93.\pend
           
\pstart\center{}Hochgeehrter Herr Dr.!\pend\vspace{0.5em}
\pstart
           Ihre erſte, frühere Anfrage muß, zu meinem lebhaften Bedauern, wohl von mir überſehen
               worden{ }ſein. Auf Ihre neuere kann ich jetzt definitiv antworten, daß in dieſem Sommer
               eine Möglichkeit, \substVorne{}\textsuperscript{für die}\substDazwischen{}in der\substHinten{}{ }Fr. B.\pwindex{Freie Bühne für den Entwickelungskampf der Zeit@\emph{Freie Bühne für den Entwickelungskampf der Zeit}|pw} noch ein Drama zu veröffentlichen, leider
               nicht beſteht. Rosmer\pwindex{Bernstein, Elsa 28.\,10.\,1866 Wien – 12.\,7.\,1949 Hamburg@\textsc{Bernstein, Elsa} (28.\,10.\,1866 Wien – 12.\,7.\,1949 Hamburg), \emph{Schriftstellerin}|pw}’s »Dämmerung\pwindex{Bernstein, Elsa 28.\,10.\,1866 Wien – 12.\,7.\,1949 Hamburg@\textsc{Bernstein, Elsa} (28.\,10.\,1866 Wien – 12.\,7.\,1949 Hamburg), \emph{Schriftstellerin}!Dämmerung@\strich\emph{Dämmerung}|pw}« füllt noch Juli und Auguſt,
               dann kommt \label{K_L00229-1v}\edtext{Halbe\pwindex{Halbe, Max 4.\,10.\,1865 Gmina Suchy Dąb – 30.\,11.\,1944 Neuötting@\textsc{Halbe, Max} (4.\,10.\,1865 Gmina Suchy Dąb – 30.\,11.\,1944 Neuötting), \emph{Schriftsteller}|pw}’s neues Stück\pwindex{Halbe, Max 4.\,10.\,1865 Gmina Suchy Dąb – 30.\,11.\,1944 Neuötting@\textsc{Halbe, Max} (4.\,10.\,1865 Gmina Suchy Dąb – 30.\,11.\,1944 Neuötting), \emph{Schriftsteller}!Amerikafahrer@\strich\emph{Der Amerikafahrer}|pwv}}{\lemma{\textnormal{\emph{Halbe’s neues Stück}}}\Cendnote{\textnormal{\emph{Der Amerikafahrer}\pwindex{Halbe, Max 4.\,10.\,1865 Gmina Suchy Dąb – 30.\,11.\,1944 Neuötting@\textsc{Halbe, Max} (4.\,10.\,1865 Gmina Suchy Dąb – 30.\,11.\,1944 Neuötting), \emph{Schriftsteller}!Amerikafahrer@\strich\emph{Der Amerikafahrer}|pwk} erschien nicht in der \emph{Freien Bühne}\pwindex{Freie Bühne für den Entwickelungskampf der Zeit@\emph{Freie Bühne für den Entwickelungskampf der Zeit}|pwk}.}}}\label{K_L00229-1}. Zwei Theaterſtücke
               nebeneinander aber geht nicht gut!\pend
           
\pstart
           Mit vorzüglichſter Hochachtung und der nochmaligen Bitte, Verzögerungen nicht als
               Wertungen perſönlicher Art aufzufaſſen\pend
           \pstart Ihr \spacefill\mbox{W. Bölsche}\pend{}\selectlanguage{ngerman}\endnumbering\briefempfaengerindex{Schnitzler, Arthur@\textsc{Schnitzler, Arthur}!zzzBölsche, Wilhelm@\emph{von Wilhelm Bölsche}!1893-07-012@{1. 7. 1893}|)be}\mylabel{L00229h}  \newcommand{\dateiname}{L00229}\newcommand{\titel}{Wilhelm Bölsche an Arthur Schnitzler, 1. 7. 1893}\newcommand{\editorInnen}{Martin Anton Müller und Gerd-Hermann Susen}%% latex-leseansicht-abspann.tex
%% Abspann für die Leseansicht.
%% Der Schalter \ifkorrekturansicht ist bereits durch den Vorspann gesetzt.

%% latex-abspann.tex
%% Gemeinsamer Abspann für Korrekturansicht und Leseansicht.
%% Setzt den Schalter \ifkorrekturansicht voraus (gesetzt in den
%% einbindenden Dateien latex-korrekturansicht-abspann.tex bzw.
%% latex-leseansicht-abspann.tex).
%% ---------------------------------------------------------------

\normalsize

% Das esempio-Environment wird nur in der Leseansicht benötigt
\ifkorrekturansicht\else
\newenvironment{esempio}[3]%
{
    \vspace{1.5ex}
    \rlap{\underline{#1}}
    \par
    \setlength{\parindent}{0cm}
    \nopagebreak
    \leftskip=#2cm
    \rightskip=#3cm
}
{
    \par
}
\fi

\doendnotes{C}
\bigskip
\vfill

\clearpage

\footnotesize

\ifkorrekturansicht
  \lohead{\textsc{register}}
\fi

% theindex-Environment neu definieren ohne reledmac
\makeatletter
\renewenvironment{theindex}{%
  \ifkorrekturansicht
    \section*{\indexname}%
  \else
    \subsubsection*{Index der erwähnten Entitäten}%
  \fi
  \setlength{\parindent}{0pt}%
  \setlength{\parskip}{0pt plus 0.3pt}%
  \let\item\@idxitem
}{%
  \ifkorrekturansicht\clearpage\fi
}
\makeatother

\IfFileExists{\jobname-pw.ind}{\input{\jobname-pw.ind}}{}

% Quellenangabe nur in der Leseansicht
\ifkorrekturansicht\else
% Fallback-Definitionen, falls die .tex-Datei \titel etc. nicht gesetzt hat
\providecommand{\titel}{}
\providecommand{\editorInnen}{}
\providecommand{\dateiname}{\jobname}

\vspace{3cm}

\vfill

\footnotesize
\textsc{Quelle}: \titel. Herausgegeben von {\editorInnen}. In: \emph{Arthur Schnitzler: Briefwechsel mit Autorinnen und Autoren}.
 Digitale Edition, https://schnitzler-briefe.acdh.oeaw.ac.at/{\dateiname}.html (Stand \today)
\fi

\end{document}


