%% latex-korrekturansicht-vorspann.tex
%% Vorspann für die Korrekturansicht.
%% Lädt die gemeinsame Datei latex-vorspann.tex mit gesetztem Schalter.

\newif\ifkorrekturansicht
\korrekturansichttrue

\input{../tex-inputs/latex-vorspann}


\section[Wilhelm Bölsche an Arthur Schnitzler, 1. 7. 1893]{L00229 Wilhelm Bölsche an Arthur Schnitzler, 1. 7. 1893}
\nopagebreak\mylabel{L00229v}
\rehead{ }\normalsize\beginnumbering\briefempfaengerindex{Schnitzler, Arthur@\textsc{Schnitzler, Arthur}!zzzBoelsche, Wilhelm@\emph{von Wilhelm Bölsche}!1893-07-011@{1. 7. 1893}|(be}
\toendnotes[C]{\smallbreak\pagebreak[2]}\Standort{DLA, A:Schnitzler, HS.NZ85.1.2577,8.}
\physDesc{Brief, 1 Blatt, 1 Seite, 590 Zeichen
\newline{}Handschrift: schwarze Tinte, deutsche Kurrent
\newline{}Schnitzler: mit rotem Buntstift nummeriert: »9« }
\buchAbdrucke{\weitereDrucke{Wilhelm Bölsche: \emph{Briefwechsel. Mit Autoren der Freien Bühne}. Berlin: \emph{Weidler} 2010, S. 690.} }\toendnotes[C]{\smallbreak}
\pstart
           \raggedleft{}{\pb}\textcolor{gray}{\textbf{\textit{Wilhelm Bölsche}}}\pend
           
\pstart
           \raggedleft{}\textcolor{gray}{\textbf{\textit{Friedrichshagen\oindex{Friedrichshagen@\textbf{Friedrichshagen}, \emph{P.PPLX}|pw}}}}\pend
           
\pstart
           \raggedleft{}1. VII. 93.\pend
           
\pstart\center{}Hochgeehrter Herr Dr.!\pend\vspace{0.5em}
\pstart
           Ihre erſte, frühere Anfrage muß, zu meinem lebhaften Bedauern, wohl von mir überſehen
               worden ſein. Auf Ihre neuere kann ich jetzt definitiv antworten, daß in dieſem Sommer
               eine Möglichkeit, \substVorne{}\textsuperscript{für die}\substDazwischen{}in der\substHinten{}{ }Fr. B.\pwindex{Freie Buehne fuer den Entwickelungskampf der Zeit@\emph{Freie Bühne für den Entwickelungskampf der Zeit}|pw} noch ein Drama zu veröffentlichen, leider
               nicht beſteht. Rosmer\pwindex{Bernstein, Elsa 28.10.1866 – 12.07.1949@\textsc{Bernstein, Elsa} (28.10.1866 – 12.07.1949), \emph{Schriftsteller/Schriftstellerin}|pw}’s »Dämmerung\pwindex{Daemmerung@\emph{Dämmerung}|pw}« füllt noch Juli und Auguſt,
               dann kommt \label{K_L00229-1v}\edtext{Halbe\pwindex{Halbe, Max 04.10.1865 – 30.11.1944@\textsc{Halbe, Max} (04.10.1865 – 30.11.1944), \emph{Schriftsteller/Schriftstellerin}|pw}’s neues Stück\pwindex{Amerikafahrer@\emph{Der Amerikafahrer}|pwv}}{\lemma{\textnormal{\emph{Halbe’s neues Stück}}}\Cendnote{\textnormal{\emph{Der Amerikafahrer}\pwindex{Amerikafahrer@\emph{Der Amerikafahrer}|pwk} erschien nicht in der \emph{Freien Bühne}\pwindex{Freie Buehne fuer den Entwickelungskampf der Zeit@\emph{Freie Bühne für den Entwickelungskampf der Zeit}|pwk}.}}}\label{K_L00229-1}. Zwei Theaterſtücke
               nebeneinander aber geht nicht gut!\pend
           
\pstart
           Mit vorzüglichſter Hochachtung und der nochmaligen Bitte, Verzögerungen nicht als
               Wertungen perſönlicher Art aufzufaſſen\pend
           \pstart Ihr \spacefill\mbox{W. Bölsche}\pend{}\selectlanguage{ngerman}\endnumbering\briefempfaengerindex{Schnitzler, Arthur@\textsc{Schnitzler, Arthur}!zzzBoelsche, Wilhelm@\emph{von Wilhelm Bölsche}!1893-07-011@{1. 7. 1893}|)be}\mylabel{L00229h}  \normalsize

\doendnotes{C}
\bigskip
\vfill

\clearpage

\footnotesize

\lohead{\textsc{register}}

% Definiere theindex-Environment komplett neu ohne reledmac
\makeatletter
\renewenvironment{theindex}{%
  \section*{\indexname}%
  \setlength{\parindent}{0pt}%
  \setlength{\parskip}{0pt plus 0.3pt}%
  \let\item\@idxitem
}{%
  \clearpage
}
\makeatother

\IfFileExists{\jobname-pw.ind}{\input{\jobname-pw.ind}}{}

\end{document}

      