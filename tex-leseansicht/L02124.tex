%% latex-leseansicht-vorspann.tex
%% Vorspann für die Leseansicht.
%% Lädt die gemeinsame Datei latex-vorspann.tex mit nicht gesetztem Schalter.

\newif\ifkorrekturansicht
\korrekturansichtfalse

\input{../tex-inputs/latex-vorspann}


         
         \renewcommand{\erwaehntePersonen}{Personen: Peter Altenberg, Georg Engländer}
         \renewcommand{\erwaehnteOrte}{Orte: Otto-Wagner-Spital, Wien}
         \renewcommand{\erwaehnteWerke}{}
               \section[Peter Altenberg an Arthur Schnitzler, {[}19.? 4. 1913{]}]{ Peter Altenberg an Arthur Schnitzler, {[}19.? 4. 1913{]}}\nopagebreak\mylabel{v}\rehead{ }\begin{ledgroupsized}[t]{13cm}\normalsize\beginnumbering \toendnotes[C]{\smallbreak\pagebreak[2]} \Standort{DLA, A:Schnitzler, HS.NZ85.1.2342, S. 14–15.}
\physDesc{Brief, maschinenschriftliche Abschrift, 2 Blätter, 2 Seiten, 554 Zeichen
\newline{}Schreibmaschine}\toendnotes[C]{\smallbreak}\pstart
           \raggedleft{}{\pb}\label{K_L02124-1v}\edtext{20. 4. 1913}{\lemma{\textnormal{\emph{20. 4. 1913}}}\Cendnote{\textnormal{Die Datierung der Abschrift dürfte
                        falsch sein und dieser Brief unmittelbar vor dem Besuch Schnitzler\pwindex{Schnitzler, Arthur 15.05.1862 – 21.10.1931@\textsc{Schnitzler, Arthur} (15.05.1862 – 21.10.1931), \emph{Schriftsteller, Mediziner}|pwk}s in der Psychiatrie\oindex{Otto-Wagner-Spital@\textbf{Otto-Wagner-Spital}|pwkv} am 20. 4. anzusiedeln
                        sein. Umgekehrt datiert die Abschrift einen Brief, der nach dem Besuch
                        abgefasst sein muss, mit 19.. Folglich wird eine Verwechslung
                        angenommen und dieser Brief auf 19., der andere auf
                           20. datiert.}}}\label{K_L02124-1h}\pend
           \pstart
           Liebster bester Dr. Arthur Schnitzler, ich wende mich nun, in meiner
               tiefsten \uuline{Lebens-Noth} an Sie, den \uline{Dichter} vor allem, den \uline{Menschen!}\pend
           \pstart
           Hilfe, Hilfe! Erbarmen! Gnade! Ich \uuline{\edtext{muss}{\Cendnote{dreifach unterstrichen}}} meine \uline{süsse unentbehrliche} Freiheit haben, ich \uuline{\edtext{muss}{\Cendnote{dreifach unterstrichen}}}! Da gibt es kein \uline{Zögern}, keine \uline{Bedenken}, kein \uline{Paktieren}! Jede \uline{Verzögerung}
               ist Mord an meinem {\pb}\uline{dadurch allein} verzweifelnden Gehirne! Sprechen Sie
                  \uuline{\edtext{nicht}{\Cendnote{dreifach unterstrichen}}} mit den hiesigen Aerzten! Ich \uuline{\edtext{muss}{\Cendnote{dreifach unterstrichen}}} meine \uline{volle bedingungslose
                  ganze} Freiheit haben. Man muss sie mir \uuline{\edtext{sofort}{\Cendnote{dreifach unterstrichen}}}
               geben! Hilfe, Erbarmen, Gnade!\pend
           \pstart
           Ihr durch einen \uline{feig-stupiden}{ }Bruder\pwindex{Englaender, Georg 03.04.1862 – 10.04.1927@\textsc{Engländer, Georg} (03.04.1862 – 10.04.1927), \emph{Beamter}|pwv} Eingekerkerten{\\[\baselineskip]}\spacefill\mbox{P. A.}\pend
           \leftskip=0em{}
         
         \endnumbering\mylabel{h}\end{ledgroupsized}  \newcommand{\dateiname}{L02124}\newcommand{\titel}{Peter Altenberg an Arthur Schnitzler, [19.? 4. 1913]}\newcommand{\editorInnen}{Martin Anton Müller und Gerd-Hermann Susen}%% latex-leseansicht-abspann.tex
%% Abspann für die Leseansicht.
%% Der Schalter \ifkorrekturansicht ist bereits durch den Vorspann gesetzt.

%% latex-abspann.tex
%% Gemeinsamer Abspann für Korrekturansicht und Leseansicht.
%% Setzt den Schalter \ifkorrekturansicht voraus (gesetzt in den
%% einbindenden Dateien latex-korrekturansicht-abspann.tex bzw.
%% latex-leseansicht-abspann.tex).
%% ---------------------------------------------------------------

\normalsize

% Das esempio-Environment wird nur in der Leseansicht benötigt
\ifkorrekturansicht\else
\newenvironment{esempio}[3]%
{
    \vspace{1.5ex}
    \rlap{\underline{#1}}
    \par
    \setlength{\parindent}{0cm}
    \nopagebreak
    \leftskip=#2cm
    \rightskip=#3cm
}
{
    \par
}
\fi

\doendnotes{C}
\bigskip
\vfill

\clearpage

\footnotesize

\ifkorrekturansicht
  \lohead{\textsc{register}}
\fi

% theindex-Environment neu definieren ohne reledmac
\makeatletter
\renewenvironment{theindex}{%
  \ifkorrekturansicht
    \section*{\indexname}%
  \else
    \subsubsection*{Index der erwähnten Entitäten}%
  \fi
  \setlength{\parindent}{0pt}%
  \setlength{\parskip}{0pt plus 0.3pt}%
  \let\item\@idxitem
}{%
  \ifkorrekturansicht\clearpage\fi
}
\makeatother

\IfFileExists{\jobname-pw.ind}{\input{\jobname-pw.ind}}{}

% Quellenangabe nur in der Leseansicht
\ifkorrekturansicht\else
% Fallback-Definitionen, falls die .tex-Datei \titel etc. nicht gesetzt hat
\providecommand{\titel}{}
\providecommand{\editorInnen}{}
\providecommand{\dateiname}{\jobname}

\vspace{3cm}

\vfill

\footnotesize
\textsc{Quelle}: \titel. Herausgegeben von {\editorInnen}. In: \emph{Arthur Schnitzler: Briefwechsel mit Autorinnen und Autoren}.
 Digitale Edition, https://schnitzler-briefe.acdh.oeaw.ac.at/{\dateiname}.html (Stand \today)
\fi

\end{document}


      