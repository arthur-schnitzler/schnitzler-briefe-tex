%% latex-korrekturansicht-vorspann.tex
%% Vorspann für die Korrekturansicht.
%% Lädt die gemeinsame Datei latex-vorspann.tex mit gesetztem Schalter.

\newif\ifkorrekturansicht
\korrekturansichttrue

\input{../tex-inputs/latex-vorspann}


\section[Peter Altenberg an Arthur Schnitzler, {[}19.? 4. 1913{]}]{L02124 Peter Altenberg an Arthur Schnitzler, {[}19.? 4. 1913{]}}
\nopagebreak\mylabel{L02124v}
\rehead{ }\normalsize\beginnumbering\briefempfaengerindex{Schnitzler, Arthur@\textsc{Schnitzler, Arthur}!zzzAltenberg, Peter@\emph{von Peter Altenberg}!1913-04-191@{{[}19.? 4. 1913{]}}|(be}
\toendnotes[C]{\smallbreak\pagebreak[2]}\Standort{DLA, A:Schnitzler, HS.NZ85.1.2342, S. 14–15.}
\physDesc{Brief, maschinenschriftliche Abschrift2 Blätter, 2 Seiten, 554 Zeichen
\newline{}Schreibmaschine}\toendnotes[C]{\smallbreak}
\pstart
           \raggedleft{}{\pb}\label{K_L02124-1v}\edtext{20. 4. 1913}{\lemma{\textnormal{\emph{20. 4. 1913}}}\Cendnote{\textnormal{Die Datierung der Abschrift dürfte
                        falsch sein und dieser Brief unmittelbar vor dem Besuch Schnitzlers in der Psychiatrie\oindex{Otto-Wagner-Spital@\textbf{Otto-Wagner-Spital}, \emph{Krankenhaus (K.KKH)}|pwkv} am 20. 4. 1913 anzusiedeln
                        sein. Umgekehrt datiert die Abschrift einen Brief, der nach dem Besuch
                        abgefasst sein muss, mit 19. 4. 1913 (siehe Peter Altenberg an Arthur Schnitzler, [20.? 4. 1913]). Folglich wird eine Verwechslung
                        angenommen und dieser Brief auf den 19. 4. 1913, der andere auf
                        20. 4. 1913 datiert.}}}\label{K_L02124-1}\pend
           \vspace{0.5em}
\pstart
           Liebster bester Dr. Arthur Schnitzler, ich wende mich nun, in meiner
               tiefsten \uuline{Lebens-Noth} an Sie, den \uline{Dichter} vor allem, den \uline{Menschen!}\pend
           
\pstart
           Hilfe, Hilfe! Erbarmen! Gnade! Ich \uuline{\edtext{muss}{\Cendnote{dreifach unterstrichen}}} meine \uline{süsse unentbehrliche} Freiheit haben, ich \uuline{\edtext{muss}{\Cendnote{dreifach unterstrichen}}}! Da gibt es kein \uline{Zögern}, keine \uline{Bedenken}, kein \uline{Paktieren}! Jede \uline{Verzögerung}
               ist Mord an meinem {\pb}\uline{dadurch allein} verzweifelnden Gehirne! Sprechen Sie
                  \uuline{\edtext{nicht}{\Cendnote{dreifach unterstrichen}}} mit den hiesigen Aerzten! Ich \uuline{\edtext{muss}{\Cendnote{dreifach unterstrichen}}} meine \uline{volle bedingungslose
                  ganze} Freiheit haben. Man muss sie mir \uuline{\edtext{sofort}{\Cendnote{dreifach unterstrichen}}}
               geben! Hilfe, Erbarmen, Gnade!\pend
           
\pstart
           Ihr durch einen \uline{feig-stupiden}{ }Bruder\pwindex{Englaender, Georg 03.04.1862 – 10.04.1927@\textsc{Engländer, Georg} (03.04.1862 – 10.04.1927), \emph{Privatbeamter/Privatbeamtin}|pwv} Eingekerkerten{\\[\baselineskip]}\spacefill\mbox{P. A.}\pend
           \leftskip=0em{}\selectlanguage{ngerman}\endnumbering\briefempfaengerindex{Schnitzler, Arthur@\textsc{Schnitzler, Arthur}!zzzAltenberg, Peter@\emph{von Peter Altenberg}!1913-04-191@{{[}19.? 4. 1913{]}}|)be}\mylabel{L02124h}  \normalsize

\doendnotes{C}
\bigskip
\vfill

\clearpage

\footnotesize

\lohead{\textsc{register}}

% Definiere theindex-Environment komplett neu ohne reledmac
\makeatletter
\renewenvironment{theindex}{%
  \section*{\indexname}%
  \setlength{\parindent}{0pt}%
  \setlength{\parskip}{0pt plus 0.3pt}%
  \let\item\@idxitem
}{%
  \clearpage
}
\makeatother

\IfFileExists{\jobname-pw.ind}{\input{\jobname-pw.ind}}{}

\end{document}

      