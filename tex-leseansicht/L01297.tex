%% latex-korrekturansicht-vorspann.tex
%% Vorspann für die Korrekturansicht.
%% Lädt die gemeinsame Datei latex-vorspann.tex mit gesetztem Schalter.

\newif\ifkorrekturansicht
\korrekturansichttrue

\input{../tex-inputs/latex-vorspann}


\section[Hugo von Hofmannsthal an Arthur Schnitzler, 19. 6. {[}1903{]}]{L01297 Hugo von Hofmannsthal an Arthur Schnitzler, 19. 6. {[}1903{]}}
\nopagebreak\mylabel{L01297v}
\rehead{ }\normalsize\beginnumbering\briefempfaengerindex{Schnitzler, Arthur@\textsc{Schnitzler, Arthur}!zzzHofmannsthal, Hugo von@\emph{von Hugo von Hofmannsthal}!1903-06-191@{19. 6. {[}1903{]}}|(be}
\toendnotes[C]{\smallbreak\pagebreak[2]}\Standort{CUL, Schnitzler, B 43.}
\physDesc{Brief, 1 Blatt, 2 Seiten, 2192 Zeichen
\newline{}Handschrift: schwarze Tinte, deutsche Kurrent
\newline{}Ordnung: 1) mit Bleistift von unbekannter Hand nummeriert: »\strikeout{217}«  2) mit Bleistift von unbekannter Hand nummeriert:
                                    »198«}
\buchAbdrucke{\weitereDrucke{Hugo von Hofmannsthal, Arthur Schnitzler: \emph{Briefwechsel}. Frankfurt am Main: \emph{S. Fischer} 1964, S. 169.} }\toendnotes[C]{\smallbreak}
\pstart
           \centering{}{\pb}\textcolor{gray}{\textbf{Hôtel und Pension Faloria\oindex{Hôtel und Pension Faloria@\textbf{Hôtel und Pension Faloria}, \emph{Hotel (K.HTL)}|pw}}}\pend
           
\pstart
           \centering{}\textcolor{gray}{\textbf{mit Dependance Bellevue\oindex{Hotel Bellevue [Cortina DAmpezzo]@\textbf{Hotel Bellevue [Cortina d’Ampezzo]}, \emph{Hotel (K.HTL)}|pw}}}\pend
           
\pstart
           \raggedleft{}\textcolor{gray}{\textbf{Cortina d’Ampezzo\oindex{Cortina DAmpezzo@\textbf{Cortina d’Ampezzo}, \emph{P.PPLA3}|pw}}}\pend
           
\pstart
           \raggedleft{}\textcolor{gray}{\textbf{Tirolo\oindex{Suedtirol@\textbf{Südtirol}, \emph{A.ADM2}|pw}.}}\pend
           
\pstart
           \raggedleft{}\textcolor{gray}{\textbf{L. Menardi\pwindex{Menardi, Luigi @\textsc{Menardi, Luigi}, \emph{Hotelbesitzer/Hotelbesitzerin}|pw}.}}\pend
           
\pstart
           \centering{}Freitag 19. Juni\pend
           
\pstart{}Lieber Arthur,\pend\vspace{0.5em}
\pstart
           bei dem völligen Mangel an Nachricht muſs ich denken, daß Sie faſt den gleichen Tag,
               wo wir abgereiſt ſind, angekommen ſein dürften.\hspace*{1.5em}Es
               iſt nun faſt ein Jahr her, daß wir zuſammen gereiſt ſind und wenn man es
               zuſammenrechnet, wie oft wir, in dem dazwiſchenliegenden Jahr, uns geſehen haben, ſo
               wird wohl kaum ſo viele Zeit herauskommen, als wir miteinander verbracht hätten, wenn
               wir, ich in Petersburg\oindex{Sankt Petersburg@\textbf{Sankt Petersburg}, \emph{P.PPLA}|pw} und Sie in London\oindex{London@\textbf{London}, \emph{P.PPLC}|pw}, leben würden und wir uns auf 8 oder
               10 Tage etwa in Berlin\oindex{Berlin@\textbf{Berlin}, \emph{P.PPLC}|pw}{ }\textsc{rendez-vous} gegeben hätten.\hspace*{1.5em}Und doch ſind wir weder ſo reich an Freunden und wohlthuenden Menschen, noch ſo
               ſtumpfſinnig überzeugt von der endloſen Dauer des Lebens, noch so begraben in dem
               Reichthum unſerer Arbeit, daß wir auf das verzichten {\pb}könnten – was vielleicht das
               einzige Geſchenk iſt womit unſer Schickſal uns für eine unfreundliche Gegenwart
               entſchädigen wollte: die Freude uns aneinander als Lebendige zu erfreuen.\pend
           
\pstart
           Faſt beneide ich diejenigen, die nach uns einmal in Ihren ausführlichen Tagebüchern
               leſen und wochenlang ganz darin leben werden – wie es mir jetzt mit dem prachtvollen
                  Briefwechſel Hebbels\pwindex{Hebbel, Friedrich 18.03.1813 – 13.12.1863@\textsc{Hebbel, Friedrich} (18.03.1813 – 13.12.1863), \emph{Schriftsteller/Schriftstellerin}|pw}\pwindex{Briefwechsel mit Freunden und beruehmten Zeitgenossen@\emph{Briefwechsel mit Freunden und berühmten Zeitgenossen}|pwv} geht.\pend
           
\pstart
           Wirklich hier geht es ſo weit – ein ganz einziger Fall – daß uns das Alltagsgeſicht
               einer Sti{\geminationm}ung überliefert iſt, dann der Brief, der ſich
               dieſer Sti{\geminationm}ung nachmittags abringen ließ, und endlich
               als ſie abends ſich von innen erleuchtete und erwärmte, das Gedicht, das aus ihr
                  entſtand.\hspace*{1.5em}Über Goethe\pwindex{Goethe, Johann Wolfgang von 1749-08-28 – 1832-03-22@\textsc{Goethe, Johann Wolfgang von} (1749-08-28 – 1832-03-22), \emph{Schriftsteller/Schriftstellerin}|pw} iſt uns ſo viel überliefert: aber an keinem Punkt ſchließt ſich’s ſo
               zum Kreiſe; Nirgends können wir ganz deutlich den Übergang aus dem Leben und Leiden
               ins Geſtalten gewahren. Die Jugend erſcheint uns traumhaft und befremdlich, ſelbſt
               wie ein Gedicht; \substVorne{}\textsuperscript{auf}\substDazwischen{}in\substHinten{} dem ſpäteren Alter ist Poeſie und Reflexion freilich eins, aber auf Koſten
               der erſteren. Was aber in dem, der die ſtärkſten Theile des Fauſt\pwindex{Faust. Eine Tragoedie@\emph{Faust. Eine Tragödie}|pw} ſchrieb, vorgegangen iſt, an den Tagen wo er ſie ſchrieb,
               wie ſich damals das Fühlen in Schaffen umſetzte, das würde ich lieber erfahren als
               vieles andere, aber freilich ſo erfahren wie mans bei Hebbel\pwindex{Hebbel, Friedrich 18.03.1813 – 13.12.1863@\textsc{Hebbel, Friedrich} (18.03.1813 – 13.12.1863), \emph{Schriftsteller/Schriftstellerin}|pw} erfährt, wo man’s ſieht, wie durch ein Glasfenſter.\pend
           
\pstart
           Wie aus dieſem Brief zu entnehmen, regnet es. Aber ich wüßte \label{T_L01297-1v}\edtext{gern etwas von Ihnen,}{\lemma{\textnormal{\emph{gern etwas von Ihnen,}}}\Cendnote{\textnormal{weiter quer am rechten Rand}}}\label{T_L01297-1} bitte
               Arthur, ſchreiben Sie mir.\pend
           \pstart Von Herzen\spacefill\mbox{Hugo.}\pend{}\selectlanguage{ngerman}\endnumbering\briefempfaengerindex{Schnitzler, Arthur@\textsc{Schnitzler, Arthur}!zzzHofmannsthal, Hugo von@\emph{von Hugo von Hofmannsthal}!1903-06-191@{19. 6. {[}1903{]}}|)be}\mylabel{L01297h}  \normalsize

\doendnotes{C}
\bigskip
\vfill

\clearpage

\footnotesize

\lohead{\textsc{register}}

% Definiere theindex-Environment komplett neu ohne reledmac
\makeatletter
\renewenvironment{theindex}{%
  \section*{\indexname}%
  \setlength{\parindent}{0pt}%
  \setlength{\parskip}{0pt plus 0.3pt}%
  \let\item\@idxitem
}{%
  \clearpage
}
\makeatother

\IfFileExists{\jobname-pw.ind}{\input{\jobname-pw.ind}}{}

\end{document}

      