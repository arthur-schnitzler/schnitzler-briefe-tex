%% latex-leseansicht-vorspann.tex
%% Vorspann für die Leseansicht.
%% Lädt die gemeinsame Datei latex-vorspann.tex mit nicht gesetztem Schalter.

\newif\ifkorrekturansicht
\korrekturansichtfalse

\input{../tex-inputs/latex-vorspann}


\section[Arthur Schnitzler an Robert Adam, 20. 11. 1920]{L02358 Arthur Schnitzler an Robert Adam, 20. 11. 1920}
\nopagebreak\mylabel{L02358v}
\rehead{ }\normalsize\beginnumbering\briefempfaengerindex{Adam, Robert@\textsc{Adam, Robert}!zzzSchnitzler, Arthur@\emph{von Arthur Schnitzler}!1920-11-201@{20. 11. 1920}|(be}
\toendnotes[C]{\smallbreak\pagebreak[2]}
\correspDesc{Versand  durch Arthur Schnitzler am 20. 11. 1920 in Wien
\newline{}Erhalt  durch Robert Adam im Zeitraum [20. 11. 1920 – 24. 11. 1920?] in Wien}\toendnotes[C]{\smallbreak}
\Standort{DLA, 96.34.2/24.}
\physDesc{Brief, 1 Blatt, 1 Seite, Kuvert, 629 Zeichen
\newline{}Schreibmaschine
\newline{}Handschrift: schwarze Tinte, deutsche Kurrent (\noindent{}Unterschrift)
\newline{}Versand: Stempel: »\nobreak{}\oindex{IX., Alsergrund@\textbf{IX., Alsergrund}, \emph{Verwaltungsgebiet}|pwk}9/3 \textcolor{gray}{Wien}, \textcolor{gray}{2}0. 11. {[}20{]}\nobreak{}«.  }\toendnotes[C]{\smallbreak}\pstart{}{\pb}\textcolor{gray}{\textbf{\textit{Dr ARTHUR SCHNITZLER}}}\pend{}\pstart{}\textcolor{gray}{\textbf{\textit{WIEN XVIII\oindex{XVIII., Währing@\textbf{XVIII., Währing}, \emph{Verwaltungsgebiet}|pw}}}}\pend{}\pstart{}\textcolor{gray}{\textbf{\textit{STERNWARTESTRASSE 71\oindex{Wien@\textbf{Wien}!XVIII., Währing@\textbf{XVIII., Währing}!Sternwartestraße 71@\textbf{Sternwartestraße 71}, \emph{Wohngebäude}|pw}}}}\pend{}{\bigskip}\pstart{}{\pb}Herrn\pend{}\pstart{}Oberlandesgerichtsrat\pend{}\pstart{}Dr. Robert Adam \so{Pollak},\pend{}\pstart{}Wien XII\oindex{XII., Meidling@\textbf{XII., Meidling}, \emph{Verwaltungsgebiet}|pw}.\pend{}\pstart{}Meidlinger Hauptstraße 58\oindex{Wien@\textbf{Wien}!XII., Meidling@\textbf{XII., Meidling}!Meidlinger Hauptstraße@\textbf{Meidlinger Hauptstraße}, \emph{Straße}|pw}.\pend{}{\bigskip}\vspace{1em}
\pstart
           {\pb}\textcolor{gray}{\textbf{\textit{Dr. ARTHUR SCHNITZLER}}}\hfill 20. 11. 1920.\pend
           
\pstart
           \textcolor{gray}{\textbf{\textit{WIEN XVIII\oindex{XVIII., Währing@\textbf{XVIII., Währing}, \emph{Verwaltungsgebiet}|pw}}}}\pend
           
\pstart
           \textcolor{gray}{\textbf{\textit{STERNWARTESTRASSE 71\oindex{Wien@\textbf{Wien}!XVIII., Währing@\textbf{XVIII., Währing}!Sternwartestraße 71@\textbf{Sternwartestraße 71}, \emph{Wohngebäude}|pw}}}}\pend
           
\pstart{}Sehr verehrter Herr Oberlandesgerichtsrat.\pend\vspace{0.5em}
\pstart
           Ihr freundliches Schreiben habe ich bei unserer letzten Vorstandssitzung\orgindex{Genossenschaft dramatischer Autoren und Komponisten in Österreich@Genossenschaft dramatischer Autoren und Komponisten in Österreich|pwv} mitgeteilt, es wurde mit
               lebhaftem Bedauern zur Kenntnis genommen und man bittet mich Ihnen den herzlichsten
               Dank für Ihre Bemühung und für Ihre liebenswürdige Absicht auszusprechen.\pend
           
\pstart
           Ich füge noch meinen persönlichen Dank hinzu und hoffe sehr Sie bald
               wiederzusehen.\pend
           
\pstart
           Mit verbindlichem Gruss{\\[\baselineskip]}Ihr sehr ergebener{\\[\baselineskip]}\spacefill\mbox{{[}hs.:{]} Arthur Schnitzler}\pend
           \leftskip=0em{}\selectlanguage{ngerman}\endnumbering\briefempfaengerindex{Adam, Robert@\textsc{Adam, Robert}!zzzSchnitzler, Arthur@\emph{von Arthur Schnitzler}!1920-11-201@{20. 11. 1920}|)be}\mylabel{L02358h}  \newcommand{\dateiname}{L02358}\newcommand{\titel}{Arthur Schnitzler an Robert Adam, 20. 11. 1920}\newcommand{\editorInnen}{Martin Anton Müller und Gerd-Hermann Susen}%% latex-leseansicht-abspann.tex
%% Abspann für die Leseansicht.
%% Der Schalter \ifkorrekturansicht ist bereits durch den Vorspann gesetzt.

%% latex-abspann.tex
%% Gemeinsamer Abspann für Korrekturansicht und Leseansicht.
%% Setzt den Schalter \ifkorrekturansicht voraus (gesetzt in den
%% einbindenden Dateien latex-korrekturansicht-abspann.tex bzw.
%% latex-leseansicht-abspann.tex).
%% ---------------------------------------------------------------

\normalsize

% Das esempio-Environment wird nur in der Leseansicht benötigt
\ifkorrekturansicht\else
\newenvironment{esempio}[3]%
{
    \vspace{1.5ex}
    \rlap{\underline{#1}}
    \par
    \setlength{\parindent}{0cm}
    \nopagebreak
    \leftskip=#2cm
    \rightskip=#3cm
}
{
    \par
}
\fi

\doendnotes{C}
\bigskip
\vfill

\clearpage

\footnotesize

\ifkorrekturansicht
  \lohead{\textsc{register}}
\fi

% theindex-Environment neu definieren ohne reledmac
\makeatletter
\renewenvironment{theindex}{%
  \ifkorrekturansicht
    \section*{\indexname}%
  \else
    \subsubsection*{Index der erwähnten Entitäten}%
  \fi
  \setlength{\parindent}{0pt}%
  \setlength{\parskip}{0pt plus 0.3pt}%
  \let\item\@idxitem
}{%
  \ifkorrekturansicht\clearpage\fi
}
\makeatother

\IfFileExists{\jobname-pw.ind}{\input{\jobname-pw.ind}}{}

% Quellenangabe nur in der Leseansicht
\ifkorrekturansicht\else
% Fallback-Definitionen, falls die .tex-Datei \titel etc. nicht gesetzt hat
\providecommand{\titel}{}
\providecommand{\editorInnen}{}
\providecommand{\dateiname}{\jobname}

\vspace{3cm}

\vfill

\footnotesize
\textsc{Quelle}: \titel. Herausgegeben von {\editorInnen}. In: \emph{Arthur Schnitzler: Briefwechsel mit Autorinnen und Autoren}.
 Digitale Edition, https://schnitzler-briefe.acdh.oeaw.ac.at/{\dateiname}.html (Stand \today)
\fi

\end{document}


