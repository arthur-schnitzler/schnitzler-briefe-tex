%% latex-korrekturansicht-vorspann.tex
%% Vorspann für die Korrekturansicht.
%% Lädt die gemeinsame Datei latex-vorspann.tex mit gesetztem Schalter.

\newif\ifkorrekturansicht
\korrekturansichttrue

\input{../tex-inputs/latex-vorspann}


\section[Arthur Schnitzler an Robert Adam, 20. 11. 1920]{L02358 Arthur Schnitzler an Robert Adam, 20. 11. 1920}
\nopagebreak\mylabel{L02358v}
\rehead{ }\normalsize\beginnumbering\briefempfaengerindex{Adam, Robert@\textsc{Adam, Robert}!zzzSchnitzler, Arthur@\emph{von Arthur Schnitzler}!1920-11-201@{20. 11. 1920}|(be}
\toendnotes[C]{\smallbreak\pagebreak[2]}\Standort{DLA, 96.34.2/24.}
\physDesc{Brief, 1 Blatt, 1 Seite, Umschlag, 629 Zeichen
\newline{}Schreibmaschine
\newline{}Handschrift: schwarze Tinte, deutsche Kurrent (\noindent{}Unterschrift)
\newline{}Versand: Stempel: »\nobreak{}\oindex{IX., Alsergrund@\textbf{IX., Alsergrund}, \emph{A.ADM3}|pwk}9/3 \textcolor{gray}{Wien}, \textcolor{gray}{2}0. 11. {[}20{]}\nobreak{}«.  }\toendnotes[C]{\smallbreak}\pstart{}{\pb}\textcolor{gray}{\textbf{\textit{Dr ARTHUR SCHNITZLER}}}\pend{}\pstart{}\textcolor{gray}{\textbf{\textit{WIEN XVIII\oindex{XVIII., Waehring@\textbf{XVIII., Währing}, \emph{A.ADM3}|pw}}}}\pend{}\pstart{}\textcolor{gray}{\textbf{\textit{STERNWARTESTRASSE 71\oindex{Sternwartestrasse 71@\textbf{Sternwartestraße 71}, \emph{Wohngebäude (K.WHS)}|pw}}}}\pend{}{\bigskip}\pstart{}{\pb}Herrn\pend{}\pstart{}Oberlandesgerichtsrat\pend{}\pstart{}Dr. Robert Adam \so{Pollak},\pend{}\pstart{}Wien XII\oindex{XII., Meidling@\textbf{XII., Meidling}, \emph{A.ADM3}|pw}.\pend{}\pstart{}Meidlinger Hauptstraße 58\oindex{Meidlinger Hauptstrasse@\textbf{Meidlinger Hauptstraße}, \emph{Straße (K.STR)}|pw}.\pend{}{\bigskip}\vspace{1em}
\pstart
           {\pb}\textcolor{gray}{\textbf{\textit{Dr. ARTHUR SCHNITZLER}}}\hfill 20. 11. 1920.\pend
           
\pstart
           \textcolor{gray}{\textbf{\textit{WIEN XVIII\oindex{XVIII., Waehring@\textbf{XVIII., Währing}, \emph{A.ADM3}|pw}}}}\pend
           
\pstart
           \textcolor{gray}{\textbf{\textit{STERNWARTESTRASSE 71\oindex{Sternwartestrasse 71@\textbf{Sternwartestraße 71}, \emph{Wohngebäude (K.WHS)}|pw}}}}\pend
           
\pstart{}Sehr verehrter Herr Oberlandesgerichtsrat.\pend\vspace{0.5em}
\pstart
           Ihr freundliches Schreiben habe ich bei unserer letzten VorstandssitzungXXXX ORGangabe fehlt mitgeteilt, es wurde mit
               lebhaftem Bedauern zur Kenntnis genommen und man bittet mich Ihnen den herzlichsten
               Dank für Ihre Bemühung und für Ihre liebenswürdige Absicht auszusprechen.\pend
           
\pstart
           Ich füge noch meinen persönlichen Dank hinzu und hoffe sehr Sie bald
               wiederzusehen.\pend
           
\pstart
           Mit verbindlichem Gruss{\\[\baselineskip]}Ihr sehr ergebener{\\[\baselineskip]}\spacefill\mbox{{[}hs.:{]} Arthur Schnitzler}\pend
           \leftskip=0em{}\selectlanguage{ngerman}\endnumbering\briefempfaengerindex{Adam, Robert@\textsc{Adam, Robert}!zzzSchnitzler, Arthur@\emph{von Arthur Schnitzler}!1920-11-201@{20. 11. 1920}|)be}\mylabel{L02358h}  \normalsize

\doendnotes{C}
\bigskip
\vfill

\clearpage

\footnotesize

\lohead{\textsc{register}}

% Definiere theindex-Environment komplett neu ohne reledmac
\makeatletter
\renewenvironment{theindex}{%
  \section*{\indexname}%
  \setlength{\parindent}{0pt}%
  \setlength{\parskip}{0pt plus 0.3pt}%
  \let\item\@idxitem
}{%
  \clearpage
}
\makeatother

\IfFileExists{\jobname-pw.ind}{\input{\jobname-pw.ind}}{}

\end{document}

      