%% latex-leseansicht-vorspann.tex
%% Vorspann für die Leseansicht.
%% Lädt die gemeinsame Datei latex-vorspann.tex mit nicht gesetztem Schalter.

\newif\ifkorrekturansicht
\korrekturansichtfalse

\input{../tex-inputs/latex-vorspann}


         
         \renewcommand{\erwaehntePersonen}{Personen: Robert Adam}
         \renewcommand{\erwaehnteInstitutionen}{Institutionen: Deutschösterreichischer Autorenverband}
         \renewcommand{\erwaehnteOrte}{Orte: IX., Alsergrund, Meidlinger Hauptstraße, Sternwartestraße, Wien, XII., Meidling, XVIII., Währing}
         \renewcommand{\erwaehnteWerke}{}
               \section[Arthur Schnitzler an Robert Adam, 20. 11. 1920]{ Arthur Schnitzler an Robert Adam, 20. 11. 1920}\nopagebreak\mylabel{v}\rehead{ }\begin{ledgroupsized}[t]{13cm}\normalsize\beginnumbering \toendnotes[C]{\smallbreak\pagebreak[2]} \Standort{DLA, 96.34.2/24.}
\physDesc{Brief, 1 Blatt, 1 Seite, , , , Umschlag
\newline{}Schreibmaschine
\newline{}Handschrift: schwarze Tinte, deutsche Kurrent (\noindent{}Unterschrift)\newline{}Versand: Stempel: »\nobreak{}\oindex{IX., Alsergrund@\textbf{IX., Alsergrund}|pwk}9/3 \textcolor{gray}{Wien}, \textcolor{gray}{2}0. 11. {[}20{]}\nobreak{}«.  }\toendnotes[C]{\smallbreak}\pstart{}{\pb}\textcolor{gray}{\textbf{\textit{Dr ARTHUR SCHNITZLER}}}\pend{}\pstart{}\textcolor{gray}{\textbf{\textit{WIEN XVIII\oindex{XVIII., Waehring@\textbf{XVIII., Währing}|pw}}}}\pend{}\pstart{}\textcolor{gray}{\textbf{\textit{STERNWARTESTRASSE 71\oindex{Sternwartestrasse@\textbf{Sternwartestraße}|pw}}}}\pend{}{\bigskip}\pstart{}{\pb}Herrn\pend{}\pstart{}Oberlandesgerichtsrat\pend{}\pstart{}Dr. Robert Adam \so{Pollak},\pend{}\pstart{}Wien XII\oindex{XII., Meidling@\textbf{XII., Meidling}|pw}.\pend{}\pstart{}Meidlinger Hauptstraße 58\oindex{Meidlinger Hauptstrasse@\textbf{Meidlinger Hauptstraße}|pw}.\pend{}{\bigskip}\pstart
           \noindent{}{\pb}\textcolor{gray}{\textbf{\textit{Dr. ARTHUR SCHNITZLER}}}\hfill 20. 11. 1920.\pend
           \pstart
           \textcolor{gray}{\textbf{\textit{WIEN XVIII\oindex{XVIII., Waehring@\textbf{XVIII., Währing}|pw}}}}\pend
           \pstart
           \textcolor{gray}{\textbf{\textit{STERNWARTESTRASSE 71\oindex{Sternwartestrasse@\textbf{Sternwartestraße}|pw}}}}\pend
           \pstart{}Sehr verehrter Herr Oberlandesgerichtsrat.\pend\pstart
           Ihr freundliches Schreiben habe ich bei unserer letzten Vorstandssitzung\orgindex{Deutschoesterreichischer Autorenverband@Deutschösterreichischer Autorenverband|pwv} mitgeteilt, es wurde
                    mit lebhaftem Bedauern zur Kenntnis genommen und man bittet mich Ihnen den
                    herzlichsten Dank für Ihre Bemühung und für Ihre liebenswürdige Absicht
                    auszusprechen.\pend
           \pstart
           Ich füge noch meinen persönlichen Dank hinzu und hoffe sehr Sie bald
                    wiederzusehen.\pend
           \pstart
           Mit verbindlichem Gruss{\\[\baselineskip]}Ihr sehr ergebener{\\[\baselineskip]}\spacefill\mbox{{[}hs.:{]} Arthur Schnitzler}\pend
           \leftskip=0em{}
         
         \endnumbering\mylabel{h}\end{ledgroupsized}  \newcommand{\dateiname}{L02358}\newcommand{\titel}{Arthur Schnitzler an Robert Adam, 20. 11. 1920}\newcommand{\editorInnen}{Martin Anton Müller und Gerd-Hermann Susen}%% latex-leseansicht-abspann.tex
%% Abspann für die Leseansicht.
%% Der Schalter \ifkorrekturansicht ist bereits durch den Vorspann gesetzt.

%% latex-abspann.tex
%% Gemeinsamer Abspann für Korrekturansicht und Leseansicht.
%% Setzt den Schalter \ifkorrekturansicht voraus (gesetzt in den
%% einbindenden Dateien latex-korrekturansicht-abspann.tex bzw.
%% latex-leseansicht-abspann.tex).
%% ---------------------------------------------------------------

\normalsize

% Das esempio-Environment wird nur in der Leseansicht benötigt
\ifkorrekturansicht\else
\newenvironment{esempio}[3]%
{
    \vspace{1.5ex}
    \rlap{\underline{#1}}
    \par
    \setlength{\parindent}{0cm}
    \nopagebreak
    \leftskip=#2cm
    \rightskip=#3cm
}
{
    \par
}
\fi

\doendnotes{C}
\bigskip
\vfill

\clearpage

\footnotesize

\ifkorrekturansicht
  \lohead{\textsc{register}}
\fi

% theindex-Environment neu definieren ohne reledmac
\makeatletter
\renewenvironment{theindex}{%
  \ifkorrekturansicht
    \section*{\indexname}%
  \else
    \subsubsection*{Index der erwähnten Entitäten}%
  \fi
  \setlength{\parindent}{0pt}%
  \setlength{\parskip}{0pt plus 0.3pt}%
  \let\item\@idxitem
}{%
  \ifkorrekturansicht\clearpage\fi
}
\makeatother

\IfFileExists{\jobname-pw.ind}{\input{\jobname-pw.ind}}{}

% Quellenangabe nur in der Leseansicht
\ifkorrekturansicht\else
% Fallback-Definitionen, falls die .tex-Datei \titel etc. nicht gesetzt hat
\providecommand{\titel}{}
\providecommand{\editorInnen}{}
\providecommand{\dateiname}{\jobname}

\vspace{3cm}

\vfill

\footnotesize
\textsc{Quelle}: \titel. Herausgegeben von {\editorInnen}. In: \emph{Arthur Schnitzler: Briefwechsel mit Autorinnen und Autoren}.
 Digitale Edition, https://schnitzler-briefe.acdh.oeaw.ac.at/{\dateiname}.html (Stand \today)
\fi

\end{document}


      