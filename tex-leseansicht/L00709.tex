%% latex-leseansicht-vorspann.tex
%% Vorspann für die Leseansicht.
%% Lädt die gemeinsame Datei latex-vorspann.tex mit nicht gesetztem Schalter.

\newif\ifkorrekturansicht
\korrekturansichtfalse

\input{../tex-inputs/latex-vorspann}


         
         \newcommand{\erwaehntePersonen}{Personen: }
         \newcommand{\erwaehnteInstitutionen}{}
         \newcommand{\erwaehnteOrte}{}
         \newcommand{\erwaehnteWerke}{
               \section[Arthur Schnitzler an Hugo von Hofmannsthal, 21. 7. 1897]{ Arthur Schnitzler an Hugo von Hofmannsthal, 21. 7. 1897}\nopagebreak\mylabel{v}\rehead{ }\begin{ledgroupsized}[t]{13cm}\normalsize\beginnumbering \toendnotes[C]{\smallbreak\pagebreak[2]} \Standort{FDH, Hs-30885,62.}
\physDesc{Brief, 1 Blatt, 4 Seiten
\newline{}Handschrift: Bleistift, deutsche Kurrent}\buchAbdrucke{\weitereDrucke{Hugo von Hofmannsthal, Arthur Schnitzler: \emph{Briefwechsel}. Hg. Therese Nickl und Heinrich Schnitzler. Frankfurt am Main: \emph{S. Fischer} 1964, S. 94.} }\toendnotes[C]{\smallbreak}\pstart
           \raggedleft{}{\pb}21/7\pend
           \pstart{}Mein lieber Hugo, \pend\pstart
           daſs wir uns erſt im Herbſt ſehn werden, iſt mir ſehr leid. – Laſſen Sie nur von
                    ſich hören; auch zeigen Sie mir an, wohin ich Ihnen die 2 letzten \textsc{Mozart}\pwindex{\textcolor{red}{\textsuperscript{XXXX1 indx}}|pw}bände\textcolor{red}{\textsuperscript{XXXX indx}}{ }ſchicken ſoll.\pend
           \pstart
           Richard\pwindex{\textcolor{red}{\textsuperscript{XXXX1 indx}}|pw} iſt nun zu einer wirklichen
                    Radpartie nicht zu bewegen; {\pb}ich aber fahre, we{\geminationn} das Wetter gut iſt, Freitag (mit
                    einem kleinen Schwager\pwindex{\textcolor{red}{\textsuperscript{XXXX1 indx}}|pwuv}\pwindex{\textcolor{red}{\textsuperscript{XXXX1 indx}}|pwuv}) nach Salzburg\oindex{XXXX Ortsangabe fehlt|pw}.
                        Samſtag: \textsc{Salzb.\oindex{XXXX Ortsangabe fehlt|pw} – Berchtesgaden\oindex{XXXX Ortsangabe fehlt|pw} – Ramsau\oindex{XXXX Ortsangabe fehlt|pw} – Zell am See\oindex{XXXX Ortsangabe fehlt|pw}}. So{\geminationn}tag – an der Bahn, ſo weit
                    ich komme, um Mittgs einzuſteigen und am Abend in Wien\oindex{XXXX Ortsangabe fehlt|pw} einzutreffen. –\pend
           \pstart
           {\pb}Neulich war ich in \textsc{Aussee}\oindex{XXXX Ortsangabe fehlt|pw} bei den \textsc{Loebs}\pwindex{\textcolor{red}{\textsuperscript{XXXX1 indx}}|pw}\pwindex{\textcolor{red}{\textsuperscript{XXXX1 indx}}|pw}; geſtern waren ſie in \textsc{Ischl}\oindex{XXXX Ortsangabe fehlt|pw}. \textsc{Clara}\pwindex{\textcolor{red}{\textsuperscript{XXXX1 indx}}|pw} fühlt ſich ſehr verlaſſen von Ihnen. Sie hat es anders ausgedrückt; aber
                    das iſt der Sinn. –\pend
           \pstart
           Sie wiſſen wohl, dſs \textsc{Burckhard}\pwindex{\textcolor{red}{\textsuperscript{XXXX1 indx}}|pw} die \textsc{Jordan}\textcolor{red}{\textsuperscript{XXXX indx}} nicht aufführt? – Ich ärgere mich
                    ſehr; umſomehr als ich zu ahnen glau{\pb}be, wo \uline{die} Gründe liegen und wer\pwindex{\textcolor{red}{\textsuperscript{XXXX1 indx}}|pwv} eigentlich {\dots}{ }ſagen wir »mit«ſchuldig iſt. –\pend
           \pstart
           – Sie ſchreiben mir bald nach Wien\oindex{XXXX Ortsangabe fehlt|pw}, nicht wahr? \pend
           \pstart Ihr \spacefill\mbox{Arthur.}\pend{}\pstart
           \textsc{Ischl\oindex{XXXX Ortsangabe fehlt|pw}}, 21/7 97.\pend
           \pstart
           Grüßen Sie P. A.\pwindex{\textcolor{red}{\textsuperscript{XXXX1 indx}}|pw}, we{\geminationn} er ſchon bei Ihnen iſt.\pend
           
         
         \endnumbering\mylabel{h}\end{ledgroupsized}  \newcommand{\dateiname}{L00709}\newcommand{\titel}{Arthur Schnitzler an Hugo von Hofmannsthal, 21. 7. 1897}\newcommand{\editorInnen}{Martin Anton Müller und Gerd-Hermann Susen}%% latex-leseansicht-abspann.tex
%% Abspann für die Leseansicht.
%% Der Schalter \ifkorrekturansicht ist bereits durch den Vorspann gesetzt.

%% latex-abspann.tex
%% Gemeinsamer Abspann für Korrekturansicht und Leseansicht.
%% Setzt den Schalter \ifkorrekturansicht voraus (gesetzt in den
%% einbindenden Dateien latex-korrekturansicht-abspann.tex bzw.
%% latex-leseansicht-abspann.tex).
%% ---------------------------------------------------------------

\normalsize

% Das esempio-Environment wird nur in der Leseansicht benötigt
\ifkorrekturansicht\else
\newenvironment{esempio}[3]%
{
    \vspace{1.5ex}
    \rlap{\underline{#1}}
    \par
    \setlength{\parindent}{0cm}
    \nopagebreak
    \leftskip=#2cm
    \rightskip=#3cm
}
{
    \par
}
\fi

\doendnotes{C}
\bigskip
\vfill

\clearpage

\footnotesize

\ifkorrekturansicht
  \lohead{\textsc{register}}
\fi

% theindex-Environment neu definieren ohne reledmac
\makeatletter
\renewenvironment{theindex}{%
  \ifkorrekturansicht
    \section*{\indexname}%
  \else
    \subsubsection*{Index der erwähnten Entitäten}%
  \fi
  \setlength{\parindent}{0pt}%
  \setlength{\parskip}{0pt plus 0.3pt}%
  \let\item\@idxitem
}{%
  \ifkorrekturansicht\clearpage\fi
}
\makeatother

\IfFileExists{\jobname-pw.ind}{\input{\jobname-pw.ind}}{}

% Quellenangabe nur in der Leseansicht
\ifkorrekturansicht\else
% Fallback-Definitionen, falls die .tex-Datei \titel etc. nicht gesetzt hat
\providecommand{\titel}{}
\providecommand{\editorInnen}{}
\providecommand{\dateiname}{\jobname}

\vspace{3cm}

\vfill

\footnotesize
\textsc{Quelle}: \titel. Herausgegeben von {\editorInnen}. In: \emph{Arthur Schnitzler: Briefwechsel mit Autorinnen und Autoren}.
 Digitale Edition, https://schnitzler-briefe.acdh.oeaw.ac.at/{\dateiname}.html (Stand \today)
\fi

\end{document}


      