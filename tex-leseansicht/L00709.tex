%% latex-leseansicht-vorspann.tex
%% Vorspann für die Leseansicht.
%% Lädt die gemeinsame Datei latex-vorspann.tex mit nicht gesetztem Schalter.

\newif\ifkorrekturansicht
\korrekturansichtfalse

\input{../tex-inputs/latex-vorspann}

\begin{center}
            \textcolor{red}{ENTWURF. ENTZIFFERUNG NOCH NICHT KORREKTURGELESEN}
                      \end{center}
            
               \section[Arthur Schnitzler an Hugo von Hofmannsthal, 21. 7. 1897]{ Arthur Schnitzler an Hugo von Hofmannsthal, 21. 7. 1897}\nopagebreak\mylabel{v}\rehead{ }\begin{ledgroupsized}[t]{13cm}\normalsize\beginnumbering\briefempfaengerindex{Hofmannsthal, Hugo von@\textsc{Hofmannsthal, Hugo von}!zzzSchnitzler, Arthur@\emph{von Arthur Schnitzler}!1897-07-211@{21. 7. 1897}|(be} \toendnotes[C]{\smallbreak\pagebreak[2]} \Standort{FDH, Hs-30885,62.}
\physDesc{Brief, 1 Blatt, 4 Seiten
\newline{}Handschrift: Bleistift, deutsche Kurrent}\buchAbdrucke{\weitereDrucke{Hugo von Hofmannsthal, Arthur Schnitzler: \emph{Briefwechsel}. Hg. Therese Nickl und Heinrich Schnitzler. Frankfurt am Main: \emph{S. Fischer} 1964, S. 94.} }\toendnotes[C]{\smallbreak}\pstart
           \raggedleft{}{\pb}21/7\pend
           \pstart{} Mein lieber Hugo, \pend\pstart
           daſs wir uns erſt im Herbſt ſehn werden, iſt mir ſehr leid. – Laſſen Sie nur von
                    ſich hören; auch zeigen Sie mir an, wohin ich Ihnen die 2 letzten \textsc{Mozart}\pwindex{Mozart, Wolfgang Amadeus 27.01.1756 – 05.12.1791@\textsc{Mozart, Wolfgang Amadeus} (27.01.1756 – 05.12.1791), \emph{Komponist}|pw}bände\pwindex{\textcolor{red}{\textsuperscript{XXXX1 indx}}!W. A. Mozart1856 – 1859@\strich\emph{W. A. Mozart} {[}1856 – 1859{]}|pwv}{ }ſchicken ſoll.\pend
           \pstart
           Richard\pwindex{Beer-Hofmann, Richard 11.07.1866 – 26.09.1945@\textsc{Beer-Hofmann, Richard} (11.07.1866 – 26.09.1945), \emph{Schriftsteller}|pw} iſt nun zu einer wirklichen
                    Radpartie nicht zu bewegen; {\pb}ich aber fahre, we{\geminationn} das Wetter gut iſt, Freitag (mit
                    einem kleinen Schwager\pwindex{Reinhard, Carl 01.03.1868 – 1904-09-29@\textsc{Reinhard, Carl} (01.03.1868 – 1904-09-29), \emph{Kapellmeister}|pwuv}\pwindex{Reinhard, Franz 28.05.1874 – 15.09.1939@\textsc{Reinhard, Franz} (28.05.1874 – 15.09.1939), \emph{Versicherungsbeamter}|pwuv}) nach Salzburg\oindex{Salzburg@\textbf{Salzburg}|pw}.
                        Samſtag: \textsc{Salzb.\oindex{Salzburg@\textbf{Salzburg}|pw} – Berchtesgaden\oindex{Berchtesgaden@\textbf{Berchtesgaden}|pw} – Ramsau\oindex{Ramsau bei Berchtesgaden@\textbf{Ramsau bei Berchtesgaden}|pw} – Zell am See\oindex{Zell am See@\textbf{Zell am See}|pw}}. So{\geminationn}tag – an der Bahn, ſo weit
                    ich komme, um Mittgs einzuſteigen und am Abend in Wien\oindex{Wien@\textbf{Wien}|pw} einzutreffen. –\pend
           \pstart
           {\pb}Neulich war ich in \textsc{Aussee}\oindex{Bad Aussee@\textbf{Bad Aussee}|pw} bei den \textsc{Loebs}\pwindex{Loeb, Louis 29.06.1842 – 06.06.1921@\textsc{Loeb, Louis} (29.06.1842 – 06.06.1921), \emph{Bankier}|pw}\pwindex{Loeb, Regina 1850 – 5.2.1918@\textsc{Loeb, Regina} (1850 – 5.2.1918)|pw}; geſtern waren ſie in \textsc{Ischl}\oindex{Bad Ischl@\textbf{Bad Ischl}|pw}. \textsc{Clara}\pwindex{Pollaczek, Clara Katharina 15.01.1875 – 22.07.1951@\textsc{Pollaczek, Clara Katharina} (15.01.1875 – 22.07.1951), \emph{Schriftstellerin}|pw} fühlt ſich ſehr verlaſſen von Ihnen. Sie hat es anders ausgedrückt; aber
                    das iſt der Sinn. –\pend
           \pstart
           Sie wiſſen wohl, dſs \textsc{Burckhard}\pwindex{Burckhard, Max Eugen 14.07.1854 – 16.03.1912@\textsc{Burckhard, Max Eugen} (14.07.1854 – 16.03.1912), \emph{Schriftsteller, Rechtswissenschaftler, Theaterleiter}|pw} die \textsc{Jordan}\pwindex{\textcolor{red}{\textsuperscript{XXXX1 indx}}!Agnes Jordan. Schauspiel in fuenf Akten1897@\strich\emph{Agnes Jordan. Schauspiel in fünf Akten} {[}1897{]}|pw} nicht aufführt? – Ich ärgere mich
                    ſehr; umſomehr als ich zu ahnen glau{\pb}be, wo \uline{die} Gründe liegen und wer\pwindex{Bahr, Hermann 19.07.1863 – 15.01.1934@\textsc{Bahr, Hermann} (19.07.1863 – 15.01.1934), \emph{Schriftsteller, Kritiker}|pwv} eigentlich {\dots}{ }ſagen wir »mit«ſchuldig iſt. –\pend
           \pstart
           – Sie ſchreiben mir bald nach Wien\oindex{Wien@\textbf{Wien}|pw}, nicht wahr? \pend
           \pstart Ihr \spacefill\mbox{Arthur.}\pend{}\pstart
           \textsc{Ischl\oindex{Bad Ischl@\textbf{Bad Ischl}|pw}}, 21/7 97.\pend
           \pstart
           Grüßen Sie P. A.\pwindex{Altenberg, Peter 09.03.1859 – 08.01.1919@\textsc{Altenberg, Peter} (09.03.1859 – 08.01.1919), \emph{Schriftsteller}|pw}, we{\geminationn} er ſchon bei Ihnen iſt.\pend
           \endnumbering\briefempfaengerindex{Hofmannsthal, Hugo von@\textsc{Hofmannsthal, Hugo von}!zzzSchnitzler, Arthur@\emph{von Arthur Schnitzler}!1897-07-211@{21. 7. 1897}|)be}\mylabel{h}\end{ledgroupsized}  \newcommand{\dateiname}{L00709}\newcommand{\titel}{Arthur Schnitzler an Hugo von Hofmannsthal, 21. 7. 1897}\newcommand{\editorInnen}{Martin Anton Müller und Gerd-Hermann Susen}%% latex-leseansicht-abspann.tex
%% Abspann für die Leseansicht.
%% Der Schalter \ifkorrekturansicht ist bereits durch den Vorspann gesetzt.

%% latex-abspann.tex
%% Gemeinsamer Abspann für Korrekturansicht und Leseansicht.
%% Setzt den Schalter \ifkorrekturansicht voraus (gesetzt in den
%% einbindenden Dateien latex-korrekturansicht-abspann.tex bzw.
%% latex-leseansicht-abspann.tex).
%% ---------------------------------------------------------------

\normalsize

% Das esempio-Environment wird nur in der Leseansicht benötigt
\ifkorrekturansicht\else
\newenvironment{esempio}[3]%
{
    \vspace{1.5ex}
    \rlap{\underline{#1}}
    \par
    \setlength{\parindent}{0cm}
    \nopagebreak
    \leftskip=#2cm
    \rightskip=#3cm
}
{
    \par
}
\fi

\doendnotes{C}
\bigskip
\vfill

\clearpage

\footnotesize

\ifkorrekturansicht
  \lohead{\textsc{register}}
\fi

% theindex-Environment neu definieren ohne reledmac
\makeatletter
\renewenvironment{theindex}{%
  \ifkorrekturansicht
    \section*{\indexname}%
  \else
    \subsubsection*{Index der erwähnten Entitäten}%
  \fi
  \setlength{\parindent}{0pt}%
  \setlength{\parskip}{0pt plus 0.3pt}%
  \let\item\@idxitem
}{%
  \ifkorrekturansicht\clearpage\fi
}
\makeatother

\IfFileExists{\jobname-pw.ind}{\input{\jobname-pw.ind}}{}

% Quellenangabe nur in der Leseansicht
\ifkorrekturansicht\else
% Fallback-Definitionen, falls die .tex-Datei \titel etc. nicht gesetzt hat
\providecommand{\titel}{}
\providecommand{\editorInnen}{}
\providecommand{\dateiname}{\jobname}

\vspace{3cm}

\vfill

\footnotesize
\textsc{Quelle}: \titel. Herausgegeben von {\editorInnen}. In: \emph{Arthur Schnitzler: Briefwechsel mit Autorinnen und Autoren}.
 Digitale Edition, https://schnitzler-briefe.acdh.oeaw.ac.at/{\dateiname}.html (Stand \today)
\fi

\end{document}


      