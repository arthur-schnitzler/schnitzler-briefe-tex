%% latex-korrekturansicht-vorspann.tex
%% Vorspann für die Korrekturansicht.
%% Lädt die gemeinsame Datei latex-vorspann.tex mit gesetztem Schalter.

\newif\ifkorrekturansicht
\korrekturansichttrue

\input{../tex-inputs/latex-vorspann}


\section[Arthur Schnitzler an Hugo von Hofmannsthal, 21. 7. 1897]{L00709 Arthur Schnitzler an Hugo von Hofmannsthal, 21. 7. 1897}
\nopagebreak\mylabel{L00709v}
\rehead{ }\normalsize\beginnumbering\briefempfaengerindex{Hofmannsthal, Hugo von@\textsc{Hofmannsthal, Hugo von}!zzzSchnitzler, Arthur@\emph{von Arthur Schnitzler}!1897-07-211@{21. 7. 1897}|(be}
\toendnotes[C]{\smallbreak\pagebreak[2]}\Standort{FDH, Hs-30885,62.}
\physDesc{Brief, 1 Blatt, 4 Seiten, 936 Zeichen
\newline{}Handschrift: Bleistift, deutsche Kurrent}
\buchAbdrucke{\weitereDrucke{Hugo von Hofmannsthal, Arthur Schnitzler: \emph{Briefwechsel}. Frankfurt am Main: \emph{S. Fischer} 1964, S. 94.} }\toendnotes[C]{\smallbreak}
\pstart
           \raggedleft{}{\pb}21/7\pend
           
\pstart{}Mein lieber Hugo, \pend\vspace{0.5em}
\pstart
           daſs wir uns erſt im Herbſt ſehn werden, iſt mir ſehr leid. – Laſſen Sie nur von ſich
               hören; auch zeigen Sie mir an, wohin ich Ihnen die 2 letzten \textsc{Mozart}\pwindex{Mozart, Wolfgang Amadeus 27.01.1756 – 05.12.1791@\textsc{Mozart, Wolfgang Amadeus} (27.01.1756 – 05.12.1791), \emph{Komponist/Komponistin}|pw}bände\pwindex{W. A. Mozart@\emph{W. A. Mozart}|pwv}{ }ſchicken ſoll.\pend
           
\pstart
           Richard\pwindex{Beer-Hofmann, Richard 1866-07-11 – 1945-09-26@\textsc{Beer-Hofmann, Richard} (1866-07-11 – 1945-09-26), \emph{Schriftsteller/Schriftstellerin}|pw} iſt nun zu einer wirklichen Radpartie
               nicht zu bewegen; {\pb}ich aber fahre, we{\geminationn} das Wetter gut iſt, Freitag (mit einem
               kleinen Schwager\pwindex{Reinhard, Carl 01.03.1868 – 1904-09-29@\textsc{Reinhard, Carl} (01.03.1868 – 1904-09-29), \emph{Kapellmeister/Kapellmeisterin}|pwuv}\pwindex{Reinhard, Franz 28.05.1874 – 15.09.1939@\textsc{Reinhard, Franz} (28.05.1874 – 15.09.1939), \emph{Versicherungsbeamter/Versicherungsbeamtin}|pwuv}) nach Salzburg\oindex{Salzburg@\textbf{Salzburg}, \emph{A.ADM2}|pw}.
                  Samſtag: \textsc{Salzb.\oindex{Salzburg@\textbf{Salzburg}, \emph{A.ADM2}|pw} – Berchtesgaden\oindex{Berchtesgaden@\textbf{Berchtesgaden}, \emph{P.PPL}|pw} – Ramsau\oindex{Ramsau bei Berchtesgaden@\textbf{Ramsau bei Berchtesgaden}, \emph{P.PPL}|pw} – Zell am See\oindex{Zell am See@\textbf{Zell am See}, \emph{P.PPLA3}|pw}}. So{\geminationn}tag – an der Bahn, ſo weit ich
               komme, um Mittgs einzuſteigen und am Abend in Wien\oindex{Wien@\textbf{Wien}, \emph{A.ADM2}|pw} einzutreffen. –\pend
           
\pstart
           {\pb}Neulich war ich in \textsc{Aussee}\oindex{Bad Aussee@\textbf{Bad Aussee}, \emph{P.PPLA3}|pw} bei den \textsc{Loebs}\pwindex{Loeb, Louis 29.06.1842 – 06.06.1921@\textsc{Loeb, Louis} (29.06.1842 – 06.06.1921), \emph{Bankier/Bankierin}|pw}\pwindex{Loeb, Regina 1850 – 5.2.1918@\textsc{Loeb, Regina} (1850 – 5.2.1918)|pw}; geſtern waren ſie in \textsc{Ischl}\oindex{Bad Ischl@\textbf{Bad Ischl}, \emph{P.PPL}|pw}. \textsc{Clara}\pwindex{Pollaczek, Clara Katharina 15.01.1875 – 22.07.1951@\textsc{Pollaczek, Clara Katharina} (15.01.1875 – 22.07.1951), \emph{Schriftsteller/Schriftstellerin}|pw} fühlt ſich ſehr verlaſſen von Ihnen. Sie hat es anders ausgedrückt; aber das
               iſt der Sinn. –\pend
           
\pstart
           Sie wiſſen wohl, dſs \label{K_L00709-1v}\edtext{\textsc{Burckhard}\pwindex{Burckhard, Max Eugen 14.07.1854 – 16.03.1912@\textsc{Burckhard, Max Eugen} (14.07.1854 – 16.03.1912), \emph{Schriftsteller/Schriftstellerin, Rechtswissenschaftler/Rechtswissenschaftlerin, Theaterleiter/Theaterleiterin}|pw} die \textsc{Jordan}\pwindex{Agnes Jordan. Schauspiel in fuenf Akten@\emph{Agnes Jordan. Schauspiel in fünf Akten}|pw} nicht aufführt? – Ich ärgere mich ſehr; umſomehr als ich zu ahnen glau{\pb}be, wo \uline{die} Gründe liegen
               und wer\pwindex{Bahr, Hermann 19.07.1863 – 15.01.1934@\textsc{Bahr, Hermann} (19.07.1863 – 15.01.1934), \emph{Schriftsteller/Schriftstellerin, Kritiker/Kritikerin}|pwv} eigentlich {\dots}{ }ſagen wir »mit«ſchuldig}{\lemma{\textnormal{\emph{Burckhard … »mit«ſchuldig}}}\Cendnote{\textnormal{Siehe Felix Salten an Arthur Schnitzler, 17. 7. 1897.
               }}}\label{K_L00709-1} iſt. –\pend
           
\pstart
           – Sie ſchreiben mir bald nach Wien\oindex{Wien@\textbf{Wien}, \emph{A.ADM2}|pw}, nicht wahr? \pend
           \pstart Ihr \spacefill\mbox{Arthur.}\pend{}
\pstart
           \textsc{Ischl\oindex{Bad Ischl@\textbf{Bad Ischl}, \emph{P.PPL}|pw}}, 21/7 97.\pend
           
\pstart
           Grüßen Sie P. A.\pwindex{Altenberg, Peter 09.03.1859 – 08.01.1919@\textsc{Altenberg, Peter} (09.03.1859 – 08.01.1919), \emph{Schriftsteller/Schriftstellerin}|pw}, we{\geminationn} er ſchon bei Ihnen iſt.\pend
           \selectlanguage{ngerman}\endnumbering\briefempfaengerindex{Hofmannsthal, Hugo von@\textsc{Hofmannsthal, Hugo von}!zzzSchnitzler, Arthur@\emph{von Arthur Schnitzler}!1897-07-211@{21. 7. 1897}|)be}\mylabel{L00709h}  \normalsize

\doendnotes{C}
\bigskip
\vfill

\clearpage

\footnotesize

\lohead{\textsc{register}}

% Definiere theindex-Environment komplett neu ohne reledmac
\makeatletter
\renewenvironment{theindex}{%
  \section*{\indexname}%
  \setlength{\parindent}{0pt}%
  \setlength{\parskip}{0pt plus 0.3pt}%
  \let\item\@idxitem
}{%
  \clearpage
}
\makeatother

\IfFileExists{\jobname-pw.ind}{\input{\jobname-pw.ind}}{}

\end{document}

      