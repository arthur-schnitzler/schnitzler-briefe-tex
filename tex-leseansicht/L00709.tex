%% latex-leseansicht-vorspann.tex
%% Vorspann für die Leseansicht.
%% Lädt die gemeinsame Datei latex-vorspann.tex mit nicht gesetztem Schalter.

\newif\ifkorrekturansicht
\korrekturansichtfalse

\input{../tex-inputs/latex-vorspann}


\section[Arthur Schnitzler an Hugo von Hofmannsthal, 21. 7. 1897]{L00709 Arthur Schnitzler an Hugo von Hofmannsthal, 21. 7. 1897}
\nopagebreak\mylabel{L00709v}
\rehead{ }\normalsize\beginnumbering\briefempfaengerindex{Hofmannsthal, Hugo von@\textsc{Hofmannsthal, Hugo von}!zzzSchnitzler, Arthur@\emph{von Arthur Schnitzler}!1897-07-211@{21. 7. 1897}|(be}
\toendnotes[C]{\smallbreak\pagebreak[2]}
\correspDesc{Versand  durch Arthur Schnitzler am 21. 7. 1897 in Bad Ischl
\newline{}Erhalt  durch Hugo von Hofmannsthal im Zeitraum [22. 7. 1897
                  – 26. 7. 1897?] in Bad Fusch}\toendnotes[C]{\smallbreak}
\Standort{FDH, Hs-30885,62.}
\physDesc{Brief, 1 Blatt, 4 Seiten, 936 Zeichen
\newline{}Handschrift: Bleistift, deutsche Kurrent}
\buchAbdrucke{\weitereDrucke{Hugo von Hofmannsthal, Arthur Schnitzler: \emph{Briefwechsel}. Herausgegeben von Therese Nickl und Heinrich Schnitzler. Frankfurt am Main: \emph{S. Fischer} 1964, S. 94.} }\toendnotes[C]{\smallbreak}
\pstart
           \raggedleft{}{\pb}21/7\pend
           
\pstart{}Mein lieber Hugo,\pend\vspace{0.5em}
\pstart
           daſs wir uns erſt im Herbſt{ }ſehn werden, iſt mir{ }ſehr leid. – Laſſen Sie nur von{ }ſich
               hören; auch zeigen Sie mir an, wohin ich Ihnen die 2 letzten \textsc{Mozart}\pwindex{Mozart, Wolfgang Amadeus 27.\,1.\,1756 Salzburg – 5.\,12.\,1791 Wien@\textsc{Mozart, Wolfgang Amadeus} (27.\,1.\,1756 Salzburg – 5.\,12.\,1791 Wien), \emph{Komponist}|pw}bände\pwindex{\textcolor{red}{\textsuperscript{XXXX indx1}}!W. A. Mozart@\strich\emph{W. A. Mozart}|pwv}{ }ſchicken{ }ſoll.\pend
           
\pstart
           Richard\pwindex{Beer-Hofmann, Richard 11.\,7.\,1866 Wien – 26.\,9.\,1945 New York City@\textsc{Beer-Hofmann, Richard} (11.\,7.\,1866 Wien – 26.\,9.\,1945 New York City), \emph{Schriftsteller}|pw} iſt nun zu einer wirklichen Radpartie
               nicht zu bewegen; {\pb}ich aber fahre, we{\geminationn} das Wetter gut iſt, Freitag (mit einem
               kleinen Schwager\pwindex{Reinhard, Carl 1.\,3.\,1868 Wien – 29.\,9.\,1904 ebd.@\textsc{Reinhard, Carl} (1.\,3.\,1868 Wien – 29.\,9.\,1904 ebd.), \emph{Kapellmeister}|pwuv}\pwindex{Reinhard, Franz 28.\,5.\,1874 Maria Enzersdorf – 15.\,9.\,1939 Wien@\textsc{Reinhard, Franz} (28.\,5.\,1874 Maria Enzersdorf – 15.\,9.\,1939 Wien), \emph{Versicherungsbeamter}|pwuv}) nach Salzburg\oindex{Salzburg@\textbf{Salzburg}, \emph{Verwaltungsgebiet}|pw}.
                  Samſtag: \textsc{Salzb.\oindex{Salzburg@\textbf{Salzburg}, \emph{Verwaltungsgebiet}|pw} – Berchtesgaden\oindex{Berchtesgaden@\textbf{Berchtesgaden}|pw} – Ramsau\oindex{Ramsau bei Berchtesgaden@\textbf{Ramsau bei Berchtesgaden}|pw} – Zell am See\oindex{Zell am See@\textbf{Zell am See}, \emph{Hauptstadt}|pw}}. So{\geminationn}tag – an der Bahn,{ }ſo weit ich
               komme, um Mittgs einzuſteigen und am Abend in Wien\oindex{Wien@\textbf{Wien}, \emph{Verwaltungsgebiet}|pw} einzutreffen. –\pend
           
\pstart
           {\pb}Neulich war ich in \textsc{Aussee}\oindex{Bad Aussee@\textbf{Bad Aussee}, \emph{Hauptstadt}|pw} bei den \textsc{Loebs}\pwindex{Loeb, Louis 29.\,6.\,1842 Mattersdorf – 6.\,6.\,1921 Wien@\textsc{Loeb, Louis} (29.\,6.\,1842 Mattersdorf – 6.\,6.\,1921 Wien), \emph{Bankier}|pw}\pwindex{Loeb, Regina 1850 – 5.\,2.\,1918 Wien@\textsc{Loeb, Regina} (1850 – 5.\,2.\,1918 Wien)|pw}; geſtern waren{ }ſie in \textsc{Ischl}\oindex{Bad Ischl@\textbf{Bad Ischl}|pw}. \textsc{Clara}\pwindex{Pollaczek, Clara Katharina 15.\,1.\,1875 Wien – 22.\,7.\,1951 ebd.@\textsc{Pollaczek, Clara Katharina} (15.\,1.\,1875 Wien – 22.\,7.\,1951 ebd.), \emph{Schriftstellerin}|pw} fühlt{ }ſich{ }ſehr verlaſſen von Ihnen. Sie hat es anders ausgedrückt; aber das
               iſt der Sinn. –\pend
           
\pstart
           Sie wiſſen wohl, dſs \label{K_L00709-1v}\edtext{\textsc{Burckhard}\pwindex{Burckhard, Max Eugen 14.\,7.\,1854 Korneuburg – 16.\,3.\,1912 Wien@\textsc{Burckhard, Max Eugen} (14.\,7.\,1854 Korneuburg – 16.\,3.\,1912 Wien), \emph{Schriftsteller, Rechtswissenschaftler, Theaterleiter}|pw} die \textsc{Jordan}\pwindex{\textcolor{red}{\textsuperscript{XXXX indx1}}!Agnes Jordan. Schauspiel in fünf Akten@\strich\emph{Agnes Jordan. Schauspiel in fünf Akten}|pw} nicht aufführt? – Ich ärgere mich{ }ſehr; umſomehr als ich zu ahnen glau{\pb}be, wo \uline{die} Gründe liegen
               und wer\pwindex{Bahr, Hermann 19.\,7.\,1863 Linz – 15.\,1.\,1934 München@\textsc{Bahr, Hermann} (19.\,7.\,1863 Linz – 15.\,1.\,1934 München), \emph{Schriftsteller, Kritiker}|pwv} eigentlich {\dots}{ }ſagen wir »mit«ſchuldig}{\lemma{\textnormal{\emph{Burckhard … »mit«schuldig}}}\Cendnote{\textnormal{Siehe XXXX Auszeichnungsfehler: Dokument L03269 nicht gefunden.
               }}}\label{K_L00709-1} iſt. –\pend
           
\pstart
           – Sie{ }ſchreiben mir bald nach Wien\oindex{Wien@\textbf{Wien}, \emph{Verwaltungsgebiet}|pw}, nicht wahr?\pend
           \pstart Ihr \spacefill\mbox{Arthur.}\pend{}
\pstart
           \textsc{Ischl\oindex{Bad Ischl@\textbf{Bad Ischl}|pw}}, 21/7 97.\pend
           
\pstart
           Grüßen Sie P. A.\pwindex{Altenberg, Peter 9.\,3.\,1859 Wien – 8.\,1.\,1919 ebd.@\textsc{Altenberg, Peter} (9.\,3.\,1859 Wien – 8.\,1.\,1919 ebd.), \emph{Schriftsteller}|pw}, we{\geminationn} er{ }ſchon bei Ihnen iſt.\pend
           \selectlanguage{ngerman}\endnumbering\briefempfaengerindex{Hofmannsthal, Hugo von@\textsc{Hofmannsthal, Hugo von}!zzzSchnitzler, Arthur@\emph{von Arthur Schnitzler}!1897-07-211@{21. 7. 1897}|)be}\mylabel{L00709h}  \newcommand{\dateiname}{L00709}\newcommand{\titel}{Arthur Schnitzler an Hugo von Hofmannsthal, 21. 7. 1897}\newcommand{\editorInnen}{Martin Anton Müller und Gerd-Hermann Susen}%% latex-leseansicht-abspann.tex
%% Abspann für die Leseansicht.
%% Der Schalter \ifkorrekturansicht ist bereits durch den Vorspann gesetzt.

%% latex-abspann.tex
%% Gemeinsamer Abspann für Korrekturansicht und Leseansicht.
%% Setzt den Schalter \ifkorrekturansicht voraus (gesetzt in den
%% einbindenden Dateien latex-korrekturansicht-abspann.tex bzw.
%% latex-leseansicht-abspann.tex).
%% ---------------------------------------------------------------

\normalsize

% Das esempio-Environment wird nur in der Leseansicht benötigt
\ifkorrekturansicht\else
\newenvironment{esempio}[3]%
{
    \vspace{1.5ex}
    \rlap{\underline{#1}}
    \par
    \setlength{\parindent}{0cm}
    \nopagebreak
    \leftskip=#2cm
    \rightskip=#3cm
}
{
    \par
}
\fi

\doendnotes{C}
\bigskip
\vfill

\clearpage

\footnotesize

\ifkorrekturansicht
  \lohead{\textsc{register}}
\fi

% theindex-Environment neu definieren ohne reledmac
\makeatletter
\renewenvironment{theindex}{%
  \ifkorrekturansicht
    \section*{\indexname}%
  \else
    \subsubsection*{Index der erwähnten Entitäten}%
  \fi
  \setlength{\parindent}{0pt}%
  \setlength{\parskip}{0pt plus 0.3pt}%
  \let\item\@idxitem
}{%
  \ifkorrekturansicht\clearpage\fi
}
\makeatother

\IfFileExists{\jobname-pw.ind}{\input{\jobname-pw.ind}}{}

% Quellenangabe nur in der Leseansicht
\ifkorrekturansicht\else
% Fallback-Definitionen, falls die .tex-Datei \titel etc. nicht gesetzt hat
\providecommand{\titel}{}
\providecommand{\editorInnen}{}
\providecommand{\dateiname}{\jobname}

\vspace{3cm}

\vfill

\footnotesize
\textsc{Quelle}: \titel. Herausgegeben von {\editorInnen}. In: \emph{Arthur Schnitzler: Briefwechsel mit Autorinnen und Autoren}.
 Digitale Edition, https://schnitzler-briefe.acdh.oeaw.ac.at/{\dateiname}.html (Stand \today)
\fi

\end{document}


