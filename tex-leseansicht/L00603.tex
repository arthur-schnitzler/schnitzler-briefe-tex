%% latex-korrekturansicht-vorspann.tex
%% Vorspann für die Korrekturansicht.
%% Lädt die gemeinsame Datei latex-vorspann.tex mit gesetztem Schalter.

\newif\ifkorrekturansicht
\korrekturansichttrue

\input{../tex-inputs/latex-vorspann}


\section[Hermann Bahr an Arthur Schnitzler, 8. 10. 1896]{L00603 Hermann Bahr an Arthur Schnitzler, 8. 10. 1896}
\nopagebreak\mylabel{L00603v}
\rehead{ }\normalsize\beginnumbering\briefempfaengerindex{Schnitzler, Arthur@\textsc{Schnitzler, Arthur}!zzzBahr, Hermann@\emph{von Hermann Bahr}!1896-10-082@{8. 10. 1896}|(be}
\toendnotes[C]{\smallbreak\pagebreak[2]}\Standort{CUL, Schnitzler, B 5b.}
\physDesc{Brief, 1 Blatt, 3 Seiten, 805 Zeichen
\newline{}Handschrift: schwarze Tinte, deutsche Kurrent
\newline{}Ordnung: mit Bleistift von unbekannter Hand nummeriert:
                                    »43« }
\buchAbdrucke{\weitereDrucke{Hermann Bahr, Arthur Schnitzler: \emph{Briefwechsel, Aufzeichnungen, Dokumente (1891–1931)}. Göttingen: \emph{Wallstein} 2018, S. 127–128.} }\toendnotes[C]{\smallbreak}
\pstart
           {\pb}\textcolor{gray}{\textbf{»Die Zeit\orgindex{Zeit. Wiener Wochenschrift@Die Zeit. Wiener Wochenschrift|pw}«}}\hfill \textcolor{gray}{\textbf{\textbf{Wien\oindex{Wien@\textbf{Wien}, \emph{A.ADM2}|pw}}, den }}8/10 \textcolor{gray}{\textbf{189}}\pend
           
\pstart
           \textcolor{gray}{\textbf{Wiener Wochenſchrift}}\hfill \textcolor{gray}{\textbf{IX/3, Günthergaſſe 1\oindex{Guenthergasse@\textbf{Günthergasse}, \emph{Straße (K.STR)}|pw}.}}\pend
           
\pstart
           \textcolor{gray}{\textbf{\textbf{Herausgeber}:}}{\\}\textcolor{gray}{\textbf{Profeſſor Dr. I. Singer\pwindex{Singer, Isidor 16.01.1857 – 08.12.1927@\textsc{Singer, Isidor} (16.01.1857 – 08.12.1927), \emph{Journalist/Journalistin, Herausgeber/Herausgeberin, Soziologe/Soziologin}|pw}, Hermann Bahr\pwindex{Bahr, Hermann 19.07.1863 – 15.01.1934@\textsc{Bahr, Hermann} (19.07.1863 – 15.01.1934), \emph{Schriftsteller/Schriftstellerin, Kritiker/Kritikerin}|pw},
                        Dr. Heinrich Kanner\pwindex{Kanner, Heinrich 09.11.1864 – 15.02.1930@\textsc{Kanner, Heinrich} (09.11.1864 – 15.02.1930), \emph{Herausgeber/Herausgeberin, Publizist/Publizistin}|pw}.}}\pend
           
\pstart
           \textcolor{gray}{\textbf{Telephon Nr. 6415.}}\pend
           
\pstart\center{}Lieber Arthur!\pend\vspace{0.5em}
\pstart
           Ich habe Brandes\pwindex{Brandes, Georg 04.02.1842 – 19.02.1927@\textsc{Brandes, Georg} (04.02.1842 – 19.02.1927)|pw}{ }ſofort ausführlich geſchrieben. Ich kann ihm
               belegen, daß ich den Artikel\pwindex{Censur in Polen@\emph{Censur in Polen}|pwv}
               von einer ihm u. mir bekannten, ſehr angeſehenen Berlin\oindex{Berlin@\textbf{Berlin}, \emph{P.PPLC}|pw}er Dame\pwindex{Neustaedter, Adele *~1862-05-16@\textsc{Neustädter, Adele} (*~1862-05-16), \emph{Übersetzer/Übersetzerin}|pwv}
               erhielt, als aus einem Buche\pwindex{Polen@\emph{Polen}|pwv}{ }ſta{\geminationm}end, das den
               nächſten Winter erst deutſch erſcheinen ſoll, von ihm autoriſiert, ja mit der
               Ermächtigung, {\pb}für ein beſonderes Honorar das
               Fragment als Originalartikel zu bringen. Ich bin alſo unſchuldig.\pend
           
\pstart
           Dir danke ich jedenfalls ſehr, daß Du ſo lieb geweſen biſt, mich gleich zu
               verſtändigen. Intereſſiert Dich die Sache, ſo kannſt Du die ganze Correspondenz mit
               der Berlinerin\pwindex{Neustaedter, Adele *~1862-05-16@\textsc{Neustädter, Adele} (*~1862-05-16), \emph{Übersetzer/Übersetzerin}|pwv} in unſerem
               Copierbuche ſehen.\pend
           
\pstart
           Was macht Deine \label{K_L00603-1v}\edtext{Novelle\pwindex{Frau des Weisen. Erzaehlung@\emph{Die Frau des Weisen. Erzählung}|pwv}}{\lemma{\textnormal{\emph{Novelle}}}\Cendnote{\textnormal{Daraus wird: Arthur Schnitzler: \emph{Die Frau des Weisen}\pwindex{Frau des Weisen. Erzaehlung@\emph{Die Frau des Weisen. Erzählung}|pwk}. In: \emph{Die Zeit}\pwindex{Zeit. Wiener Wochenschrift@\emph{Die Zeit. Wiener Wochenschrift}|pwk}, Bd. 10, H. 118, 2. 1. 1897, S. 15–16; H. 119,
                        9. 1. 1897, S. 31–32; H. 129, 16. 1. 1897,
                     S. 47–48.}}}\label{K_L00603-1}? Ich rechne beſtimmt auf ſie! Auch bin ich ſehr {\pb}neugierig, was aus dem »Freiwild\pwindex{Freiwild. Schauspiel in 3 Akten@\emph{Freiwild. Schauspiel in 3 Akten}|pw}« wird.\pend
           
\pstart
           Nochmals dankt herzlich{\\[\baselineskip]}mit beſten Grüßen{\\[\baselineskip]}Dein{\\[\baselineskip]}\spacefill\mbox{Hermann}\pend
           \leftskip=0em{}
\pstart
           \noindent{}Herrn \textsc{D\textsuperscript{r} Arthur Schnitzler}{\\}\textsc{Wien\oindex{Wien@\textbf{Wien}, \emph{A.ADM2}|pw}{ }IX Frankgasse 1\oindex{Frankgasse 1@\textbf{Frankgasse 1}, \emph{Wohngebäude (K.WHS)}|pw}.}\pend
           
\pstart
           \textcolor{gray}{\textbf{\label{T_L00603-1v}\edtext{Alle für »Die Zeit\orgindex{Zeit. Wiener Wochenschrift@Die Zeit. Wiener Wochenschrift|pw}« beſtimmten Zuſchriften und Sendungen ſind an die
                  Redaction der »Zeit\orgindex{Zeit. Wiener Wochenschrift@Die Zeit. Wiener Wochenschrift|pw}« und \textbf{nicht} an die Perſon eines der Herausgeber zu richten.}{\lemma{\textnormal{\emph{Alle … richten.}}}\Cendnote{\textnormal{am unteren Rand der ersten Seite}}}\label{T_L00603-1}}}\pend
           \selectlanguage{ngerman}\endnumbering\briefempfaengerindex{Schnitzler, Arthur@\textsc{Schnitzler, Arthur}!zzzBahr, Hermann@\emph{von Hermann Bahr}!1896-10-082@{8. 10. 1896}|)be}\mylabel{L00603h}  \normalsize

\doendnotes{C}
\bigskip
\vfill

\clearpage

\footnotesize

\lohead{\textsc{register}}

% Definiere theindex-Environment komplett neu ohne reledmac
\makeatletter
\renewenvironment{theindex}{%
  \section*{\indexname}%
  \setlength{\parindent}{0pt}%
  \setlength{\parskip}{0pt plus 0.3pt}%
  \let\item\@idxitem
}{%
  \clearpage
}
\makeatother

\IfFileExists{\jobname-pw.ind}{\input{\jobname-pw.ind}}{}

\end{document}

      