%% latex-leseansicht-vorspann.tex
%% Vorspann für die Leseansicht.
%% Lädt die gemeinsame Datei latex-vorspann.tex mit nicht gesetztem Schalter.

\newif\ifkorrekturansicht
\korrekturansichtfalse

\input{../tex-inputs/latex-vorspann}


\section[Arthur Schnitzler an Berta Zuckerkandl, 31. 10. 1925]{L03961 Arthur Schnitzler an Berta Zuckerkandl, 31. 10. 1925}
\nopagebreak\mylabel{L03961v}
\rehead{ }\normalsize\beginnumbering\briefempfaengerindex{Zuckerkandl, Berta@\textsc{Zuckerkandl, Berta}!zzzSchnitzler, Arthur@\emph{von Arthur Schnitzler}!1925-10-311@{31. 10. 1925}|(be}
\toendnotes[C]{\smallbreak\pagebreak[2]}
\correspDesc{Versand  durch Arthur Schnitzler am 31. 10. 1925 in Wien
\newline{}Erhalt  durch Berta Zuckerkandl im Zeitraum [1. 11. 1925 – 5. 11. 1925?] in Paris}\toendnotes[C]{\smallbreak}
\Standort{DLA, HS.1985.1.2282.}
\physDesc{Brief, Durchschlag, 1 Blatt, 2 Seiten, 2319 Zeichen
\newline{}Schreibmaschine
\newline{}Handschrift: roter Buntstift, lateinische Kurrent (\noindent{}beschriftet: »\uline{Zuckerkandl}« und »\uline{Frankreich}«, elf Unterstreichungen)}\toendnotes[C]{\smallbreak}
\pstart
           \raggedleft{}{\pb}31. 10. 1925.\pend
           
\pstart{}Liebe und verehrte Frau Hofrätin.\pend\vspace{0.5em}
\pstart
           Offenbar ist ein \label{K_L03961-1v}\edtext{Brief von mir}{\lemma{\textnormal{\emph{Brief von mir}}}\Cendnote{\textnormal{Siehe XXXX Auszeichnungsfehler: Dokument L03960 nicht gefunden.}}}\label{K_L03961-1}
      verloren gegangen, denn \label{K_L03961-2v}\edtext{der vom 28. d.}{\lemma{\textnormal{\emph{der vom 28. d.}}}\Cendnote{\textnormal{nicht überliefert}}}\label{K_L03961-2}, in dem
      Sie annehmen, dass unsere Briefe sich gekreuzt hätten, ist der erste, den ich seit Monaten von Ihnen erhalte. So erfahre ich auch
      erst heute, dass Gemier\pwindex{Gémier, Firmin 21.\,2.\,1865 Aubervilliers – 26.\,11.\,1933 Paris@\textsc{Gémier, Firmin} (21.\,2.\,1865 Aubervilliers – 26.\,11.\,1933 Paris), \emph{Theaterleiter, Schauspieler, Drehbuchautor}|pw} nun doch das »Weite Land\pwindex{Schnitzler, Arthur 15. 5. 1862 Wien – 21. 10. 1931 ebd.@\textsc{Schnitzler, Arthur} (15. 5. 1862 Wien – 21. 10. 1931 ebd.), \emph{Schriftsteller, Mediziner}!weite Land. Tragikomödie in fünf Akten@\strich\emph{Das weite Land. Tragikomödie in fünf Akten}|pw}« und nicht die »Liebelei\pwindex{Schnitzler, Arthur 15. 5. 1862 Wien – 21. 10. 1931 ebd.@\textsc{Schnitzler, Arthur} (15. 5. 1862 Wien – 21. 10. 1931 ebd.), \emph{Schriftsteller, Mediziner}!Liebelei. Schauspiel in drei Akten@\strich\emph{Liebelei. Schauspiel in drei Akten}|pw}« spielen will.
      Aber wenn er auch dazu keine besondere Lust
      haben sollte, so wäre es mir lieber, er liesse auch davon die Hand; ich bin in keinen
      Weise aufführungshungrig, wie Sie wissen
      liebe Frau Hofrätin. Dass man verpflichtet
      ist auf die Rückgabe eines Manuscriptes zu
      verzichten, auch wenn der Direktor, dem man
      es vorgelegt hat, von der Aufführung absieht,
      diese Annahme ist eine jener direktorialen
      Ueberheblichkeiten, über die Sie wohl eines
      Sinnes mit mir sind. Ich sende Ihnen mit gleicher Post ein anderes Exemplar\pwindex{Schnitzler, Arthur 15. 5. 1862 Wien – 21. 10. 1931 ebd.@\textsc{Schnitzler, Arthur} (15. 5. 1862 Wien – 21. 10. 1931 ebd.), \emph{Schriftsteller, Mediziner}!Liebelei. Schauspiel in drei Akten@\strich\emph{Liebelei. Schauspiel in drei Akten}|pwv}\pwindex{Schnitzler, Arthur 15. 5. 1862 Wien – 21. 10. 1931 ebd.@\textsc{Schnitzler, Arthur} (15. 5. 1862 Wien – 21. 10. 1931 ebd.), \emph{Schriftsteller, Mediziner}!Amourette. Pièce en trois actes. Adaptée de Arthur Schnitzler@\strich\emph{Amourette. Pièce en trois actes. Adaptée de Arthur Schnitzler}|pwv} und es wird
               mich freuen, wenn Jouvet\pwindex{Jouvet, Louis 24.\,12.\,1887 Crozon – 16.\,8.\,1951 Paris@\textsc{Jouvet, Louis} (24.\,12.\,1887 Crozon – 16.\,8.\,1951 Paris), \emph{Theaterleiter, Regisseur, Schauspieler}|pw} es spielen wollte.\pend
           
\pstart
           Als Einakter dazu käme, wie wir ja schon besprochen, die »Literatur\pwindex{Schnitzler, Arthur 15. 5. 1862 Wien – 21. 10. 1931 ebd.@\textsc{Schnitzler, Arthur} (15. 5. 1862 Wien – 21. 10. 1931 ebd.), \emph{Schriftsteller, Mediziner}!Literatur@\strich\emph{Literatur}|pw}« in Betracht, die ja
               Rémon\pwindex{Rémon, Maurice 27.\,11.\,1861 Paris – 20.\,6.\,1945 Mérignac@\textsc{Rémon, Maurice} (27.\,11.\,1861 Paris – 20.\,6.\,1945 Mérignac), \emph{Übersetzer}|pw}{ }übersetzt\pwindex{Schnitzler, Arthur 15. 5. 1862 Wien – 21. 10. 1931 ebd.@\textsc{Schnitzler, Arthur} (15. 5. 1862 Wien – 21. 10. 1931 ebd.), \emph{Schriftsteller, Mediziner}!Littérature. Comédie en en act@\strich\emph{Littérature. Comédie en en act}|pwv} hat.\pend
           
\pstart
           Wenn Sie es für richtig halten,
      liebe Frau Hofrätin, werde ich Rémon\pwindex{Rémon, Maurice 27.\,11.\,1861 Paris – 20.\,6.\,1945 Mérignac@\textsc{Rémon, Maurice} (27.\,11.\,1861 Paris – 20.\,6.\,1945 Mérignac), \emph{Übersetzer}|pw} gern
               die Uebersetzung von »Fräulein Else\pwindex{Schnitzler, Arthur 15. 5. 1862 Wien – 21. 10. 1931 ebd.@\textsc{Schnitzler, Arthur} (15. 5. 1862 Wien – 21. 10. 1931 ebd.), \emph{Schriftsteller, Mediziner}!Fräulein Else@\strich\emph{Fräulein Else}|pw}« anvertrauen; aber wir sollten es doch nicht {[}ohne{]} Forderung eines Vorschusses tun, den natürlich
      der Verleger zahlen müsste. Kein französischer\oindex{Frankreich@\textbf{Frankreich}|pw}
      Verleger und kein französischer\oindex{Frankreich@\textbf{Frankreich}|pw} Autor von
      Rang erteil die Autorisation zur Uebersetzung
      eines seiner Werke (besonders eines, das erfolgreich war{[}){]}, ohne eine Sicherstellung und
      zwar werden Beträge von 5–10.000 Francs ver{\pb}langt. Auf die Höhe der Summe käme es mir nun
      weniger an, aber der Verleger müsste doch
      zum mindesten durch dem Erlag eines mehr
      oder minder angemessenen Betrages seinen
      guten Willen dokumentieren.\pend
           
\pstart
           Die
               französische\oindex{Frankreich@\textbf{Frankreich}|pw}{ }Uebersetzung\pwindex{Schnitzler, Arthur 15. 5. 1862 Wien – 21. 10. 1931 ebd.@\textsc{Schnitzler, Arthur} (15. 5. 1862 Wien – 21. 10. 1931 ebd.), \emph{Schriftsteller, Mediziner}!Au Perroquet Vert@\strich\emph{Au Perroquet Vert}|pwv} des
         »Kakadu\pwindex{Schnitzler, Arthur 15. 5. 1862 Wien – 21. 10. 1931 ebd.@\textsc{Schnitzler, Arthur} (15. 5. 1862 Wien – 21. 10. 1931 ebd.), \emph{Schriftsteller, Mediziner}!grüne Kakadu. Groteske in einem Akt@\strich\emph{Der grüne Kakadu. Groteske in einem Akt}|pw}« von Lutz\pwindex{Lutz, Émile 8.\,4.\,1868 Saint-Étienne-du-Rouvray – 18.\,1.\,1940 Paris@\textsc{Lutz, Émile} (8.\,4.\,1868 Saint-Étienne-du-Rouvray – 18.\,1.\,1940 Paris), \emph{Übersetzer, Dichter}|pw} und Etienne\pwindex{Epstein, Stephan 12.\,11.\,1866 Warschau – 1941 Paris@\textsc{Epstein, Stephan} (12.\,11.\,1866 Warschau – 1941 Paris), \emph{Schriftsteller, Dramaturg, Übersetzer}|pw} (Epstein\pwindex{Epstein, Stephan 12.\,11.\,1866 Warschau – 1941 Paris@\textsc{Epstein, Stephan} (12.\,11.\,1866 Warschau – 1941 Paris), \emph{Schriftsteller, Dramaturg, Übersetzer}|pw}) ist
      nie gedruckt worden. Dass die Autorisation,
      zum mindesten die alleinige Autorisation
      für diese französische\oindex{Frankreich@\textbf{Frankreich}|pw}{ }Uebersetzung\pwindex{Schnitzler, Arthur 15. 5. 1862 Wien – 21. 10. 1931 ebd.@\textsc{Schnitzler, Arthur} (15. 5. 1862 Wien – 21. 10. 1931 ebd.), \emph{Schriftsteller, Mediziner}!Au Perroquet Vert@\strich\emph{Au Perroquet Vert}|pwv} längst abgelaufen, wie alle Autorisationen, die vor dem
      Krieg erteilt worden sind, haben wir schon
      seinerzeit konstatiert. Für alle Fälle lasse ich an Rémon\pwindex{Rémon, Maurice 27.\,11.\,1861 Paris – 20.\,6.\,1945 Mérignac@\textsc{Rémon, Maurice} (27.\,11.\,1861 Paris – 20.\,6.\,1945 Mérignac), \emph{Übersetzer}|pw} ein deutsches Bühnenexemplar
               des »Kakadu\pwindex{Schnitzler, Arthur 15. 5. 1862 Wien – 21. 10. 1931 ebd.@\textsc{Schnitzler, Arthur} (15. 5. 1862 Wien – 21. 10. 1931 ebd.), \emph{Schriftsteller, Mediziner}!grüne Kakadu. Groteske in einem Akt@\strich\emph{Der grüne Kakadu. Groteske in einem Akt}|pw}« senden.\pend
           
\pstart
           Verfügen Sie, liebe Freundin, in jeder
      Hinsicht, wie es Ihnen richtig dünkt{\dotstwo} Sie
      wissen, wie dankbar ich Ihnen für alle Ihre
      Bemühungen bin. Ich freue mich sehr Sie so
      bald wiederzusehen.\pend
           
\pstart
           Mit den herzlichsten Grüssen{\\[\baselineskip]} Ihr\pend
           \leftskip=0em{}{\vspace{1\baselineskip}}
\pstart
           \noindent{}Frau Hofrätin Berta Zuckerkandl,{\\}Paris\oindex{Paris@\textbf{Paris}, \emph{Hauptstadt}|pw}.\pend
           \selectlanguage{ngerman}\endnumbering\briefempfaengerindex{Zuckerkandl, Berta@\textsc{Zuckerkandl, Berta}!zzzSchnitzler, Arthur@\emph{von Arthur Schnitzler}!1925-10-311@{31. 10. 1925}|)be}\mylabel{L03961h}  \newcommand{\dateiname}{L03961}\newcommand{\titel}{Arthur Schnitzler an Berta Zuckerkandl, 31. 10. 1925}\newcommand{\editorInnen}{Herausgegeben von Jahnke, SelmaMüller, Martin Anton}%% latex-leseansicht-abspann.tex
%% Abspann für die Leseansicht.
%% Der Schalter \ifkorrekturansicht ist bereits durch den Vorspann gesetzt.

%% latex-abspann.tex
%% Gemeinsamer Abspann für Korrekturansicht und Leseansicht.
%% Setzt den Schalter \ifkorrekturansicht voraus (gesetzt in den
%% einbindenden Dateien latex-korrekturansicht-abspann.tex bzw.
%% latex-leseansicht-abspann.tex).
%% ---------------------------------------------------------------

\normalsize

% Das esempio-Environment wird nur in der Leseansicht benötigt
\ifkorrekturansicht\else
\newenvironment{esempio}[3]%
{
    \vspace{1.5ex}
    \rlap{\underline{#1}}
    \par
    \setlength{\parindent}{0cm}
    \nopagebreak
    \leftskip=#2cm
    \rightskip=#3cm
}
{
    \par
}
\fi

\doendnotes{C}
\bigskip
\vfill

\clearpage

\footnotesize

\ifkorrekturansicht
  \lohead{\textsc{register}}
\fi

% theindex-Environment neu definieren ohne reledmac
\makeatletter
\renewenvironment{theindex}{%
  \ifkorrekturansicht
    \section*{\indexname}%
  \else
    \subsubsection*{Index der erwähnten Entitäten}%
  \fi
  \setlength{\parindent}{0pt}%
  \setlength{\parskip}{0pt plus 0.3pt}%
  \let\item\@idxitem
}{%
  \ifkorrekturansicht\clearpage\fi
}
\makeatother

\IfFileExists{\jobname-pw.ind}{\input{\jobname-pw.ind}}{}

% Quellenangabe nur in der Leseansicht
\ifkorrekturansicht\else
% Fallback-Definitionen, falls die .tex-Datei \titel etc. nicht gesetzt hat
\providecommand{\titel}{}
\providecommand{\editorInnen}{}
\providecommand{\dateiname}{\jobname}

\vspace{3cm}

\vfill

\footnotesize
\textsc{Quelle}: \titel. Herausgegeben von {\editorInnen}. In: \emph{Arthur Schnitzler: Briefwechsel mit Autorinnen und Autoren}.
 Digitale Edition, https://schnitzler-briefe.acdh.oeaw.ac.at/{\dateiname}.html (Stand \today)
\fi

\end{document}


