%% latex-leseansicht-vorspann.tex
%% Vorspann für die Leseansicht.
%% Lädt die gemeinsame Datei latex-vorspann.tex mit nicht gesetztem Schalter.

\newif\ifkorrekturansicht
\korrekturansichtfalse

\input{../tex-inputs/latex-vorspann}


\section[Arthur Schnitzler an Richard Beer-Hofmann, 12. 9. 1895]{L00481 Arthur Schnitzler an Richard Beer-Hofmann, 12. 9. 1895}
\nopagebreak\mylabel{L00481v}
\rehead{ }\normalsize\beginnumbering\briefempfaengerindex{Beer-Hofmann, Richard@\textsc{Beer-Hofmann, Richard}!zzzSchnitzler, Arthur@\emph{von Arthur Schnitzler}!1895-09-121@{12. 9. 1895}|(be}
\toendnotes[C]{\smallbreak\pagebreak[2]}
\correspDesc{Versand  durch Arthur Schnitzler am 12. 9. 1895 in Wien
\newline{}Erhalt  durch Richard Beer-Hofmann am 13.  9. 1895 in Schönberg im Stubaital}\toendnotes[C]{\smallbreak}
\Standort{YCGL, MSS 31.}
\physDesc{Brief, 1 Blatt, 4 Seiten, Kuvert, 842 Zeichen
\newline{}Handschrift: Bleistift, deutsche Kurrent
\newline{}Versand: 1) Stempel: »\nobreak{}\oindex{IX., Alsergrund@\textbf{IX., Alsergrund}, \emph{Verwaltungsgebiet}|pwk}Wien 9/3, 12. 9. 95, 2–3V\nobreak{}«.   2) Stempel: »\nobreak{}\oindex{Schönberg im Stubaital@\textbf{Schönberg im Stubaital}, \emph{Hauptstadt}|pwk}\textcolor{gray}{Schön}{[}berg{]} in Tirol, \textcolor{gray}{13}{[}9{]}\textcolor{gray}{95}\nobreak{}«. }
\buchAbdrucke{\weitereDrucke{Arthur Schnitzler, Richard Beer-Hofmann: \emph{Briefwechsel 1891–1931}. Herausgegeben von Konstanze Fliedl. Wien, Zürich: \emph{Europaverlag} 1992, S. 79–80.} }\toendnotes[C]{\smallbreak}\pstart{}\textsc{{\pb}Herrn Dr Rich Beer-Hofmann}\pend{}\pstart{}\textsc{Tirol\oindex{Tirol@\textbf{Tirol}, \emph{Land}|pw}}\pend{}\pstart{}\textsc{Schönberg im Stubaithal\oindex{Schönberg im Stubaital@\textbf{Schönberg im Stubaital}, \emph{Hauptstadt}|pw}}\pend{}{\bigskip}\vspace{1em}
\pstart
           \noindent{}{\pb}Lieber Richard, Sie werden{ }ſich hoffentlich \substVorne{}\textsuperscript{hier}\substDazwischen{}dort\substHinten{}{ }ſehr wohl fühlen. We{\geminationn}
               es nur{ }ſchön bleibt – hier iſt der Umſchlag{ }ſchon, regnet, iſt kalt. Was werden Sie
               da thun bis Ende October? Ich glaube, Sie werden vom 16. an plötzlich in
               irgend einer Stadt {\pb}ſein und früher als Sie ahnten in
                  Wien\oindex{Wien@\textbf{Wien}, \emph{Verwaltungsgebiet}|pw}. –\pend
           
\pstart
           Viel neues gibts nicht. \textsc{Liebelei}\pwindex{Schnitzler, Arthur 15.\,5.\,1862 Wien – 21.\,10.\,1931 ebd.@\textsc{Schnitzler, Arthur} (15.\,5.\,1862 Wien – 21.\,10.\,1931 ebd.), \emph{Schriftsteller, Mediziner}!Liebelei. Schauspiel in drei Akten@\strich\emph{Liebelei. Schauspiel in drei Akten}|pw}{ }ſoll wirklich die 1. \label{K_L00481-1v}\edtext{Nov.}{\lemma{\textnormal{\emph{Nov.}}}\Cendnote{\textnormal{Novität}}}\label{K_L00481-1}{ }ſein, Anfang October. – Die \textsc{Trag}\pwindex{Sandrock, Adele 19.\,8.\,1863 Rotterdam – 30.\,8.\,1937 Berlin@\textsc{Sandrock, Adele} (19.\,8.\,1863 Rotterdam – 30.\,8.\,1937 Berlin), \emph{Schauspielerin}|pwv} hat{ }ſchon wieder ihre Feindſeligkeiten eröffnet in kindiſcher u hilfloſer
               Weise. – Kleine Aergerlichkeiten durch das »Zu Hauſe« – die Schlüſſel {\pb}klappern zu viel. (\textsc{Symbol}.)\pend
           
\pstart
           – Aerztlich zu thun. Ja! – Zufall natürlich. –\pend
           
\pstart
           Geschrieben noch nichts. –\pend
           
\pstart
           Bitte grüßen Sie Frau Lou\pwindex{Andreas-Salomé, Lou 12.\,2.\,1861 Sankt Petersburg – 5.\,2.\,1937 Göttingen@\textsc{Andreas-Salomé, Lou} (12.\,2.\,1861 Sankt Petersburg – 5.\,2.\,1937 Göttingen), \emph{Schriftstellerin}|pw} recht herzlich, wenn{ }ſie noch da iſt; we{\geminationn} Sie mir ein Wort gleich{ }ſchreiben,
                  {\pb}hören Sie{ }ſofort wieder, etwas ausführlicher, von
               mir\pend
           \pstart Ihr \spacefill\mbox{Arth}\pend{}
\pstart
           12. 9. 95. Wien\oindex{Wien@\textbf{Wien}, \emph{Verwaltungsgebiet}|pw}\pend
           \selectlanguage{ngerman}\endnumbering\briefempfaengerindex{Beer-Hofmann, Richard@\textsc{Beer-Hofmann, Richard}!zzzSchnitzler, Arthur@\emph{von Arthur Schnitzler}!1895-09-121@{12. 9. 1895}|)be}\mylabel{L00481h}  \newcommand{\dateiname}{L00481}\newcommand{\titel}{Arthur Schnitzler an Richard Beer-Hofmann, 12. 9. 1895}\newcommand{\editorInnen}{Martin Anton Müller und Gerd-Hermann Susen}%% latex-leseansicht-abspann.tex
%% Abspann für die Leseansicht.
%% Der Schalter \ifkorrekturansicht ist bereits durch den Vorspann gesetzt.

%% latex-abspann.tex
%% Gemeinsamer Abspann für Korrekturansicht und Leseansicht.
%% Setzt den Schalter \ifkorrekturansicht voraus (gesetzt in den
%% einbindenden Dateien latex-korrekturansicht-abspann.tex bzw.
%% latex-leseansicht-abspann.tex).
%% ---------------------------------------------------------------

\normalsize

% Das esempio-Environment wird nur in der Leseansicht benötigt
\ifkorrekturansicht\else
\newenvironment{esempio}[3]%
{
    \vspace{1.5ex}
    \rlap{\underline{#1}}
    \par
    \setlength{\parindent}{0cm}
    \nopagebreak
    \leftskip=#2cm
    \rightskip=#3cm
}
{
    \par
}
\fi

\doendnotes{C}
\bigskip
\vfill

\clearpage

\footnotesize

\ifkorrekturansicht
  \lohead{\textsc{register}}
\fi

% theindex-Environment neu definieren ohne reledmac
\makeatletter
\renewenvironment{theindex}{%
  \ifkorrekturansicht
    \section*{\indexname}%
  \else
    \subsubsection*{Index der erwähnten Entitäten}%
  \fi
  \setlength{\parindent}{0pt}%
  \setlength{\parskip}{0pt plus 0.3pt}%
  \let\item\@idxitem
}{%
  \ifkorrekturansicht\clearpage\fi
}
\makeatother

\IfFileExists{\jobname-pw.ind}{\input{\jobname-pw.ind}}{}

% Quellenangabe nur in der Leseansicht
\ifkorrekturansicht\else
% Fallback-Definitionen, falls die .tex-Datei \titel etc. nicht gesetzt hat
\providecommand{\titel}{}
\providecommand{\editorInnen}{}
\providecommand{\dateiname}{\jobname}

\vspace{3cm}

\vfill

\footnotesize
\textsc{Quelle}: \titel. Herausgegeben von {\editorInnen}. In: \emph{Arthur Schnitzler: Briefwechsel mit Autorinnen und Autoren}.
 Digitale Edition, https://schnitzler-briefe.acdh.oeaw.ac.at/{\dateiname}.html (Stand \today)
\fi

\end{document}


