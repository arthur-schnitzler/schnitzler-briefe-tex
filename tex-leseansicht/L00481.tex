%% latex-korrekturansicht-vorspann.tex
%% Vorspann für die Korrekturansicht.
%% Lädt die gemeinsame Datei latex-vorspann.tex mit gesetztem Schalter.

\newif\ifkorrekturansicht
\korrekturansichttrue

\input{../tex-inputs/latex-vorspann}


\section[Arthur Schnitzler an Richard Beer-Hofmann, 12. 9. 1895]{L00481 Arthur Schnitzler an Richard Beer-Hofmann, 12. 9. 1895}
\nopagebreak\mylabel{L00481v}
\rehead{ }\normalsize\beginnumbering\briefempfaengerindex{Beer-Hofmann, Richard@\textsc{Beer-Hofmann, Richard}!zzzSchnitzler, Arthur@\emph{von Arthur Schnitzler}!1895-09-121@{12. 9. 1895}|(be}
\toendnotes[C]{\smallbreak\pagebreak[2]}\Standort{YCGL, MSS 31.}
\physDesc{Brief, 1 Blatt, 4 Seiten, Umschlag, 842 Zeichen
\newline{}Handschrift: 1) Bleistift, deutsche Kurrent\hspace{1em}2) Bleistift, lateinische Kurrent (\noindent{}Adresse)\hspace{1em}
\newline{}Versand: 1) Stempel: »\nobreak{}\oindex{IX., Alsergrund@\textbf{IX., Alsergrund}, \emph{A.ADM3}|pwk}Wien 9/3, 12. 9. 95, 2–3V\nobreak{}«.   2) Stempel: »\nobreak{}\oindex{Schoenberg im Stubaital@\textbf{Schönberg im Stubaital}, \emph{P.PPLA3}|pwk}\textcolor{gray}{Schön}{[}berg{]} in Tirol, \textcolor{gray}{13}{[}9{]}\textcolor{gray}{95}\nobreak{}«. }
\buchAbdrucke{\weitereDrucke{Arthur Schnitzler, Richard Beer-Hofmann: \emph{Briefwechsel 1891–1931}. Wien, Zürich: \emph{Europaverlag} 1992, S. 79–80.} }\toendnotes[C]{\smallbreak}\pstart{}{\pb}Herrn Dr Rich Beer-Hofmann\pend{}\pstart{}Tirol\oindex{Tirol@\textbf{Tirol}, \emph{A.ADM1}|pw}\pend{}\pstart{}Schönberg im Stubaithal\oindex{Schoenberg im Stubaital@\textbf{Schönberg im Stubaital}, \emph{P.PPLA3}|pw}\pend{}{\bigskip}\vspace{1em}
\pstart
           \noindent{}{\pb}Lieber Richard, Sie werden ſich hoffentlich \substVorne{}\textsuperscript{hier}\substDazwischen{}dort\substHinten{}{ }ſehr wohl fühlen. We{\geminationn}
               es nur ſchön bleibt – hier iſt der Umſchlag ſchon, regnet, iſt kalt. Was werden Sie
               da thun bis Ende October? Ich glaube, Sie werden vom 16. an plötzlich in
               irgend einer Stadt {\pb}ſein und früher als Sie ahnten in
                  Wien\oindex{Wien@\textbf{Wien}, \emph{A.ADM2}|pw}. –\pend
           
\pstart
           Viel neues gibts nicht. \textsc{Liebelei}\pwindex{Liebelei. Schauspiel in drei Akten@\emph{Liebelei. Schauspiel in drei Akten}|pw}{ }ſoll wirklich die 1. \label{K_L00481-1v}\edtext{Nov.}{\lemma{\textnormal{\emph{Nov.}}}\Cendnote{\textnormal{Novität}}}\label{K_L00481-1}{ }ſein, Anfang October. – Die \textsc{Trag}\pwindex{Sandrock, Adele 1863-08-19 – 1937-08-30@\textsc{Sandrock, Adele} (1863-08-19 – 1937-08-30), \emph{Schauspieler/Schauspielerin}|pwv} hat ſchon wieder ihre Feindſeligkeiten eröffnet in kindiſcher u hilfloſer
               Weise. – Kleine Aergerlichkeiten durch das »Zu Hauſe« – die Schlüſſel {\pb}klappern zu viel. (\textsc{Symbol}.)\pend
           
\pstart
           – Aerztlich zu thun. Ja! – Zufall natürlich. –\pend
           
\pstart
           Geschrieben noch nichts. –\pend
           
\pstart
           Bitte grüßen Sie Frau Lou\pwindex{Andreas-Salome, Lou 12.02.1861 – 05.02.1937@\textsc{Andreas-Salomé, Lou} (12.02.1861 – 05.02.1937), \emph{Schriftsteller/Schriftstellerin}|pw} recht herzlich, wenn
               ſie noch da iſt; we{\geminationn} Sie mir ein Wort gleich ſchreiben,
                  {\pb}hören Sie ſofort wieder, etwas ausführlicher, von
               mir\pend
           \pstart Ihr \spacefill\mbox{Arth}\pend{}
\pstart
           12. 9. 95. Wien\oindex{Wien@\textbf{Wien}, \emph{A.ADM2}|pw}\pend
           \selectlanguage{ngerman}\endnumbering\briefempfaengerindex{Beer-Hofmann, Richard@\textsc{Beer-Hofmann, Richard}!zzzSchnitzler, Arthur@\emph{von Arthur Schnitzler}!1895-09-121@{12. 9. 1895}|)be}\mylabel{L00481h}  \normalsize

\doendnotes{C}
\bigskip
\vfill

\clearpage

\footnotesize

\lohead{\textsc{register}}

% Definiere theindex-Environment komplett neu ohne reledmac
\makeatletter
\renewenvironment{theindex}{%
  \section*{\indexname}%
  \setlength{\parindent}{0pt}%
  \setlength{\parskip}{0pt plus 0.3pt}%
  \let\item\@idxitem
}{%
  \clearpage
}
\makeatother

\IfFileExists{\jobname-pw.ind}{\input{\jobname-pw.ind}}{}

\end{document}

      