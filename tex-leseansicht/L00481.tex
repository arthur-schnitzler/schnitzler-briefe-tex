%% latex-leseansicht-vorspann.tex
%% Vorspann für die Leseansicht.
%% Lädt die gemeinsame Datei latex-vorspann.tex mit nicht gesetztem Schalter.

\newif\ifkorrekturansicht
\korrekturansichtfalse

\input{../tex-inputs/latex-vorspann}


         
         \renewcommand{\erwaehntePersonen}{Personen: Lou Andreas-Salomé, Richard Beer-Hofmann, Adele Sandrock}
         \renewcommand{\erwaehnteOrte}{Orte: IX., Alsergrund, Schönberg im Stubaital, Tirol, Wien}
         \renewcommand{\erwaehnteWerke}{Werke: Liebelei. Schauspiel in drei Akten}
               \section[Arthur Schnitzler an Richard Beer-Hofmann, 12. 9. 1895]{ Arthur Schnitzler an Richard Beer-Hofmann, 12. 9. 1895}\nopagebreak\mylabel{v}\rehead{ }\begin{ledgroupsized}[t]{13cm}\normalsize\beginnumbering \toendnotes[C]{\smallbreak\pagebreak[2]} \Standort{YCGL, MSS 31.}
\physDesc{Brief, 1 Blatt, 4 Seiten, Umschlag
\newline{}Handschrift: 1) Bleistift, deutsche Kurrent\hspace{1em}2) Bleistift, lateinische Kurrent (\noindent{}Adresse)\hspace{1em}\newline{}Versand: 1) Stempel: »\nobreak{}\oindex{IX., Alsergrund@\textbf{IX., Alsergrund}|pwk}Wien 9/3, 12. 9. 95, 2–3V\nobreak{}«.   2) Stempel: »\nobreak{}\oindex{Schoenberg im Stubaital@\textbf{Schönberg im Stubaital}|pwk}\textcolor{gray}{Schön}{[}berg{]} in Tirol, \textcolor{gray}{13} {[}9{]} \textcolor{gray}{95}\nobreak{}«. }\buchAbdrucke{\weitereDrucke{Arthur Schnitzler, Richard Beer-Hofmann: \emph{Briefwechsel 1891–1931}. Hg. Konstanze Fliedl. Wien, Zürich: \emph{Europaverlag} 1992, S. 79–80.} }\toendnotes[C]{\smallbreak}\pstart{}{\pb}Herrn Dr Rich Beer-Hofmann\pend{}\pstart{}Tirol\oindex{Tirol@\textbf{Tirol}|pw}\pend{}\pstart{}Schönberg im Stubaithal\oindex{Schoenberg im Stubaital@\textbf{Schönberg im Stubaital}|pw}\pend{}{\bigskip}\pstart
           \noindent{}{\pb}Lieber Richard, Sie werden ſich hoffentlich \substVorne{}\textsuperscript{hier}\substDazwischen{}dort\substHinten{}{ }ſehr wohl fühlen. We{\geminationn}
               es nur ſchön bleibt – hier iſt der Umſchlag ſchon, regnet, iſt kalt. Was werden Sie
               da thun bis Ende October? Ich glaube, Sie werden vom 16. an plötzlich in
               irgend einer Stadt {\pb}ſein und früher als Sie ahnten in
                  Wien\oindex{Wien@\textbf{Wien}|pw}. –\pend
           \pstart
           Viel neues gibts nicht. \textsc{Liebelei}\pwindex{Schnitzler, Arthur 15.05.1862 – 21.10.1931@\textsc{Schnitzler, Arthur} (15.05.1862 – 21.10.1931), \emph{Schriftsteller, Mediziner}!Liebelei. Schauspiel in drei Akten1895-10-09@\strich\emph{Liebelei. Schauspiel in drei Akten} {[}1895-10-09{]}|pw}{ }ſoll wirklich die 1. \label{K_L00481_1v}\edtext{Nov.}{\lemma{\textnormal{\emph{Nov.}}}\Cendnote{\textnormal{Novität}}}\label{K_L00481_1h}{ }ſein, Anfang October. – Die \textsc{Trag}\pwindex{Sandrock, Adele 1863-08-19 – 1937-08-30@\textsc{Sandrock, Adele} (1863-08-19 – 1937-08-30), \emph{Schauspielerin}|pwv} hat ſchon wieder ihre Feindſeligkeiten eröffnet in kindiſcher u hilfloſer
               Weise. – Kleine Aergerlichkeiten durch das »Zu Hauſe« – die Schlüſſel {\pb}klappern zu viel. (\textsc{Symbol}.)\pend
           \pstart
           – Aerztlich zu thun. Ja! – Zufall natürlich. –\pend
           \pstart
           Geschrieben noch nichts. –\pend
           \pstart
           Bitte grüßen Sie Frau Lou\pwindex{Andreas-Salome, Lou 12.02.1861 – 05.02.1937@\textsc{Andreas-Salomé, Lou} (12.02.1861 – 05.02.1937), \emph{Schriftstellerin}|pw} recht herzlich, wenn
               ſie noch da iſt; we{\geminationn} Sie mir ein Wort gleich ſchreiben,
                  {\pb}hören Sie ſofort wieder, etwas ausführlicher, von
               mir\pend
           \pstart Ihr \spacefill\mbox{Arth}\pend{}\pstart
           12. 9. 95. Wien\oindex{Wien@\textbf{Wien}|pw}\pend
           
         
         \endnumbering\mylabel{h}\end{ledgroupsized}  \newcommand{\dateiname}{L00481}\newcommand{\titel}{Arthur Schnitzler an Richard Beer-Hofmann, 12. 9. 1895}\newcommand{\editorInnen}{Martin Anton Müller und Gerd-Hermann Susen}%% latex-leseansicht-abspann.tex
%% Abspann für die Leseansicht.
%% Der Schalter \ifkorrekturansicht ist bereits durch den Vorspann gesetzt.

%% latex-abspann.tex
%% Gemeinsamer Abspann für Korrekturansicht und Leseansicht.
%% Setzt den Schalter \ifkorrekturansicht voraus (gesetzt in den
%% einbindenden Dateien latex-korrekturansicht-abspann.tex bzw.
%% latex-leseansicht-abspann.tex).
%% ---------------------------------------------------------------

\normalsize

% Das esempio-Environment wird nur in der Leseansicht benötigt
\ifkorrekturansicht\else
\newenvironment{esempio}[3]%
{
    \vspace{1.5ex}
    \rlap{\underline{#1}}
    \par
    \setlength{\parindent}{0cm}
    \nopagebreak
    \leftskip=#2cm
    \rightskip=#3cm
}
{
    \par
}
\fi

\doendnotes{C}
\bigskip
\vfill

\clearpage

\footnotesize

\ifkorrekturansicht
  \lohead{\textsc{register}}
\fi

% theindex-Environment neu definieren ohne reledmac
\makeatletter
\renewenvironment{theindex}{%
  \ifkorrekturansicht
    \section*{\indexname}%
  \else
    \subsubsection*{Index der erwähnten Entitäten}%
  \fi
  \setlength{\parindent}{0pt}%
  \setlength{\parskip}{0pt plus 0.3pt}%
  \let\item\@idxitem
}{%
  \ifkorrekturansicht\clearpage\fi
}
\makeatother

\IfFileExists{\jobname-pw.ind}{\input{\jobname-pw.ind}}{}

% Quellenangabe nur in der Leseansicht
\ifkorrekturansicht\else
% Fallback-Definitionen, falls die .tex-Datei \titel etc. nicht gesetzt hat
\providecommand{\titel}{}
\providecommand{\editorInnen}{}
\providecommand{\dateiname}{\jobname}

\vspace{3cm}

\vfill

\footnotesize
\textsc{Quelle}: \titel. Herausgegeben von {\editorInnen}. In: \emph{Arthur Schnitzler: Briefwechsel mit Autorinnen und Autoren}.
 Digitale Edition, https://schnitzler-briefe.acdh.oeaw.ac.at/{\dateiname}.html (Stand \today)
\fi

\end{document}


      