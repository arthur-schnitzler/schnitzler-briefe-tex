%% latex-leseansicht-vorspann.tex
%% Vorspann für die Leseansicht.
%% Lädt die gemeinsame Datei latex-vorspann.tex mit nicht gesetztem Schalter.

\newif\ifkorrekturansicht
\korrekturansichtfalse

\input{../tex-inputs/latex-vorspann}


\section[Arthur Schnitzler an Berta Zuckerkandl, 24. 12. 1925]{L03962 Arthur Schnitzler an Berta Zuckerkandl, 24. 12. 1925}
\nopagebreak\mylabel{L03962v}
\rehead{ }\normalsize\beginnumbering\briefempfaengerindex{Zuckerkandl, Berta@\textsc{Zuckerkandl, Berta}!zzzSchnitzler, Arthur@\emph{von Arthur Schnitzler}!1925-12-243@{24. 12. 1925}|(be}
\toendnotes[C]{\smallbreak\pagebreak[2]}
\correspDesc{Versand  durch Arthur Schnitzler am 24. 12. 1925 in Wien
\newline{}Erhalt  durch Berta Zuckerkandl im Zeitraum [24. 12. 1925 – 27. 12. 1925?] in Wien}\toendnotes[C]{\smallbreak}
\Standort{DLA, HS.1985.1.2282.}
\physDesc{Brief, Durchschlag, 1 Blatt, 2 Seiten, 3390 Zeichen
\newline{}Schreibmaschine
\newline{}Handschrift: roter Buntstift, lateinische Kurrent (\noindent{}beschriftet: »\uline{Zuckerkandl}«, neun Unterstreichungen)
\newline{}Beilage: maschinenschriftlicher Brief, Durchschlag, 2 Blätter, 1 Seite;
                                 von Schnitzler mit rotem Buntstift beschriftet: »\textsc{\uline{Zuckerkandl}}«, acht Unterstreichungen; von Schnitzler mit Bleistift
                                 Korrekturen in lateinischer Kurrentschrift }\toendnotes[C]{\smallbreak}
\pstart
           \raggedleft{}{\pb}24. 12. 1925.\pend
           
\pstart{}Liebe und verehrte Frau Hofrätin.\pend\vspace{0.5em}
\pstart
           Beigeschlossen die beiden \label{K_L03962-1v}\edtext{\begin{otherlanguage}{french}Bulletins\end{otherlanguage}}{\lemma{\textnormal{\emph{Bulletins}}}\Cendnote{\textnormal{nicht überliefert}}}\label{K_L03962-1}{ }unterzeichnet
               und so weit ausgefüllt als bisher möglich. Auf dem \label{K_L03962-2v}\edtext{\begin{otherlanguage}{french}Bull. de Decl.\end{otherlanguage}}{\lemma{\textnormal{\emph{Bull. de Decl.}}}\Cendnote{\textnormal{\begin{otherlanguage}{french}bulletin de déclaration\end{otherlanguage}, französisch:
                  Formular einer justiziablen Erklärung, in diesem Fall
                  Urheberschaftserklärung}}}\label{K_L03962-2} ist, wie Sie sehen, eine \label{K_L03962-3v}\edtext{\begin{otherlanguage}{french}Repartition des droits\end{otherlanguage}}{\lemma{\textnormal{\emph{Repartition des droits}}}\Cendnote{\textnormal{französisch: Verteilung der
                  Rechte}}}\label{K_L03962-3}{ }bereits vorgesehen und so habe ich die Teilung von 9 zu 3 {\%} in der Voraussetzung vorgenommen, dass die Tantiemen sich
               tatsächlich auf 12 {\%} belaufen.\pend
           
\pstart
           Mme. Cabire\pwindex{Cabire, Emma @\textsc{Cabire, Emma}, \emph{Übersetzerin, Redakteurin, Literaturagentin}|pw} wohnt Paris, 27 Rue Lemercier\oindex{27, Rue Lemercier@\textbf{27, Rue Lemercier}, \emph{Wohngebäude}|pw}. Sie besitzt, wie ich nochmals
               erwähne, in diesem Moment überhaupt noch keine \label{K_L03962-4v}\edtext{formelle Autorisation}{\lemma{\textnormal{\emph{formelle Autorisation}}}\Cendnote{\textnormal{Es geht um die \emph{Übersetzung}\pwindex{Schnitzler, Arthur 15. 5. 1862 Wien – 21. 10. 1931 ebd.@\textsc{Schnitzler, Arthur} (15. 5. 1862 Wien – 21. 10. 1931 ebd.), \emph{Schriftsteller, Mediziner}!Le Pays de l’âme. Drame en 5 actes@\strich\emph{Le Pays de l’âme. Drame en 5 actes}|pwk} von \emph{Das weite
                  Land}\pwindex{Schnitzler, Arthur 15. 5. 1862 Wien – 21. 10. 1931 ebd.@\textsc{Schnitzler, Arthur} (15. 5. 1862 Wien – 21. 10. 1931 ebd.), \emph{Schriftsteller, Mediziner}!weite Land. Tragikomödie in fünf Akten@\strich\emph{Das weite Land. Tragikomödie in fünf Akten}|pwk}.}}}\label{K_L03962-4}. Ich lege übrigens einen \label{K_L03962-5v}\edtext{ostensiblen Brief}{\lemma{\textnormal{\emph{ostensiblen Brief}}}\Cendnote{\textnormal{Beilage, siehe unten}}}\label{K_L03962-5} gleich in zwei Exemplaren bei, von dem Sie sowohl
               Frau Cabire\pwindex{Cabire, Emma @\textsc{Cabire, Emma}, \emph{Übersetzerin, Redakteurin, Literaturagentin}|pw} gegenüber als für Gemier\pwindex{Gémier, Firmin 21.\,2.\,1865 Aubervilliers – 26.\,11.\,1933 Paris@\textsc{Gémier, Firmin} (21.\,2.\,1865 Aubervilliers – 26.\,11.\,1933 Paris), \emph{Theaterleiter, Schauspieler, Drehbuchautor}|pw} und wo Sie es sonst noch für richtig
               finden (Besnard\pwindex{Besnard, Lucien 19.\,1.\,1872 Nonancourt – 1955 Paris@\textsc{Besnard, Lucien} (19.\,1.\,1872 Nonancourt – 1955 Paris), \emph{Schriftsteller}|pw}) Gebrauch machen können.\pend
           
\pstart
           Und nun erlauben sie mir nochmals Sie zu bitten in diesem Fall Ihre Provision nicht
               mit 15, sondern mit 20 {\%} ansetzen zu dürfen und verzeihen Sie
               meine Halsstarrigkeit.\pend
           
\pstart
           Es wird mich sehr freuen, wenn nicht nur in der Angelegenheit des »Weiten Landes\pwindex{Schnitzler, Arthur 15. 5. 1862 Wien – 21. 10. 1931 ebd.@\textsc{Schnitzler, Arthur} (15. 5. 1862 Wien – 21. 10. 1931 ebd.), \emph{Schriftsteller, Mediziner}!weite Land. Tragikomödie in fünf Akten@\strich\emph{Das weite Land. Tragikomödie in fünf Akten}|pw}«, sondern eventuell auch in den
                  and{[}r{]}en sozusagen schwebenden durch Ihre gütige Vermittlung
               Fortschritte zu verzeichnen sein werden. Es handelt sich da um eventuelle Aufführung
               der »Liebelei\pwindex{Schnitzler, Arthur 15. 5. 1862 Wien – 21. 10. 1931 ebd.@\textsc{Schnitzler, Arthur} (15. 5. 1862 Wien – 21. 10. 1931 ebd.), \emph{Schriftsteller, Mediziner}!Liebelei. Schauspiel in drei Akten@\strich\emph{Liebelei. Schauspiel in drei Akten}|pw}\textcolor{red}{\textsuperscript{XXXX indx2}}« zusammen mit »Literatur\pwindex{Schnitzler, Arthur 15. 5. 1862 Wien – 21. 10. 1931 ebd.@\textsc{Schnitzler, Arthur} (15. 5. 1862 Wien – 21. 10. 1931 ebd.), \emph{Schriftsteller, Mediziner}!Literatur@\strich\emph{Literatur}|pw}\textcolor{red}{\textsuperscript{XXXX indx2}}« (Rémon\pwindex{Rémon, Maurice 27.\,11.\,1861 Paris – 20.\,6.\,1945 Mérignac@\textsc{Rémon, Maurice} (27.\,11.\,1861 Paris – 20.\,6.\,1945 Mérignac), \emph{Übersetzer}|pw}!),
               um das endliche Erscheinen von »Casanovas
                  Heimfahrt\pwindex{Schnitzler, Arthur 15. 5. 1862 Wien – 21. 10. 1931 ebd.@\textsc{Schnitzler, Arthur} (15. 5. 1862 Wien – 21. 10. 1931 ebd.), \emph{Schriftsteller, Mediziner}!Casanovas Heimfahrt@\strich\emph{Casanovas Heimfahrt}|pw}« (Nathan\pwindex{Nathan, Nicolas @\textsc{Nathan, Nicolas}, \emph{Übersetzer}|pw}{ }{\pb}Paris, Hotel Ronceray, 10, Boulevard
               Monmartre\oindex{Hôtel Ronceray@\textbf{Hôtel Ronceray}, \emph{Hotel}|pw}), »Fräulein Else\pwindex{Schnitzler, Arthur 15. 5. 1862 Wien – 21. 10. 1931 ebd.@\textsc{Schnitzler, Arthur} (15. 5. 1862 Wien – 21. 10. 1931 ebd.), \emph{Schriftsteller, Mediziner}!Fräulein Else@\strich\emph{Fräulein Else}|pw}« etc.\pend
           
\pstart
           Mit herzlichen Grüssen und vielen Wünschen für Erfolg in allen Gebieten{\\[\baselineskip]} Ihr
               aufrichtig ergebener\pend
           \leftskip=0em{}{\vspace{1\baselineskip}}
\pstart
           \noindent{}Frau Hofrätin Berta Zuckerkandl,{\\}Wien\oindex{Wien@\textbf{Wien}, \emph{Verwaltungsgebiet}|pw}.\pend
           \selectlanguage{ngerman}\vspace{1em}
\pstart
           {\pb}24. 12. 1925.\pend
           
\pstart{}Sehr verehrte Frau Hofrätin.\pend\vspace{0.5em}
\pstart
           Ich bitte Sie als meine Vertreterin für Frankreich\oindex{Frankreich@\textbf{Frankreich}|pw} die Angelegenheit »Weites
                  Land\pwindex{Schnitzler, Arthur 15. 5. 1862 Wien – 21. 10. 1931 ebd.@\textsc{Schnitzler, Arthur} (15. 5. 1862 Wien – 21. 10. 1931 ebd.), \emph{Schriftsteller, Mediziner}!weite Land. Tragikomödie in fünf Akten@\strich\emph{Das weite Land. Tragikomödie in fünf Akten}|pw}« anlässlich Ihres bevorstehenden Aufenthaltes in Paris\oindex{Paris@\textbf{Paris}, \emph{Hauptstadt}|pw} wenn möglich zu endgültigem Abschluss zu bringen.\pend
           
\pstart
           Insbesondere bitte ich Sie mit Mme. Cabire\pwindex{Cabire, Emma @\textsc{Cabire, Emma}, \emph{Übersetzerin, Redakteurin, Literaturagentin}|pw},
               die formell überhaupt noch keine Autorisation besitzt, in Verbindung zu treten. Es
               wäre mir sehr wichtig in die Uebersetzung\pwindex{Schnitzler, Arthur 15. 5. 1862 Wien – 21. 10. 1931 ebd.@\textsc{Schnitzler, Arthur} (15. 5. 1862 Wien – 21. 10. 1931 ebd.), \emph{Schriftsteller, Mediziner}!Le Pays de l’âme. Drame en 5 actes@\strich\emph{Le Pays de l’âme. Drame en 5 actes}|pwv} der Mme. Cabire\pwindex{Cabire, Emma @\textsc{Cabire, Emma}, \emph{Übersetzerin, Redakteurin, Literaturagentin}|pw} Einsicht
               zu nehmen. Ich bin durchaus geneigt ihr die formelle Autorisation unter der
               Voraussetzung zu erteilen, dass Monsieur Gemier\pwindex{Gémier, Firmin 21.\,2.\,1865 Aubervilliers – 26.\,11.\,1933 Paris@\textsc{Gémier, Firmin} (21.\,2.\,1865 Aubervilliers – 26.\,11.\,1933 Paris), \emph{Theaterleiter, Schauspieler, Drehbuchautor}|pw} selbst sich bereit erklärt diese Uebersetzung\pwindex{Schnitzler, Arthur 15. 5. 1862 Wien – 21. 10. 1931 ebd.@\textsc{Schnitzler, Arthur} (15. 5. 1862 Wien – 21. 10. 1931 ebd.), \emph{Schriftsteller, Mediziner}!Le Pays de l’âme. Drame en 5 actes@\strich\emph{Le Pays de l’âme. Drame en 5 actes}|pwv} (eventuell mit Retouchen) im Odéon\oindex{Odéon@\textbf{Odéon}, \emph{Theater}|pw} zur Aufführung zu bringen.\pend
           
\pstart
           Was die Tantiemen anbelangt, so denke ich, dass die von mir im \begin{otherlanguage}{french}Bulletin\end{otherlanguage} eingesetzte
               Verteilung (¾ der Autor, ¼ derer Uebersetzer) von allen Beteiligten ohneweiters
               genehmigt werden wird.\pend
           
\pstart
           So weit mir bekannt ist, wird an jeden französischen\oindex{Frankreich@\textbf{Frankreich}|pw} Autor in Deutschland\oindex{Deutschland@\textbf{Deutschland}|pw} und
               Oesterreich\oindex{Österreich@\textbf{Österreich}|pw}, sobald eine Bühne sein Stück zu
               erwerben wünscht, ein \label{K_L03962-6v}\edtext{\substVorne{}\textsuperscript{\begin{otherlanguage}{french}A\end{otherlanguage}}\substDazwischen{}\begin{otherlanguage}{french}à\end{otherlanguage}\substHinten{}\begin{otherlanguage}{french}{ }valoi\end{otherlanguage}\substVorne{}\textsuperscript{\begin{otherlanguage}{french}t\end{otherlanguage}}\substDazwischen{}\begin{otherlanguage}{french}r\end{otherlanguage}\substHinten{}}{\lemma{\textnormal{\emph{à valoir}}}\Cendnote{\textnormal{französisch: Vorschuss}}}\label{K_L03962-6} gezahlt,
               das von den ersten Tantiemen abgezogen wird. Es wäre mir lieb, wenn auch in unserem
               Falle durch das Theater Odeéon\orgindex{Odéon@Odéon|pw} ein \substVorne{}\textsuperscript{\begin{otherlanguage}{french}A\end{otherlanguage}}\substDazwischen{}\begin{otherlanguage}{french}à\end{otherlanguage}\substHinten{}\begin{otherlanguage}{french}{ }valoi\end{otherlanguage}\substVorne{}\textsuperscript{\begin{otherlanguage}{french}t\end{otherlanguage}}\substDazwischen{}\begin{otherlanguage}{french}r\end{otherlanguage}\substHinten{} in einer von Gemier\pwindex{Gémier, Firmin 21.\,2.\,1865 Aubervilliers – 26.\,11.\,1933 Paris@\textsc{Gémier, Firmin} (21.\,2.\,1865 Aubervilliers – 26.\,11.\,1933 Paris), \emph{Theaterleiter, Schauspieler, Drehbuchautor}|pw} selbst zu
               bestimmenden Höhe gezahlt würde.\pend
           
\pstart
           Als eine weitere noch wichtigere Sicherheit erscheint mir die Festsetzung {\pb}eines Termins, bis zu welchem spätestens das Stück\pwindex{Schnitzler, Arthur 15. 5. 1862 Wien – 21. 10. 1931 ebd.@\textsc{Schnitzler, Arthur} (15. 5. 1862 Wien – 21. 10. 1931 ebd.), \emph{Schriftsteller, Mediziner}!weite Land. Tragikomödie in fünf Akten@\strich\emph{Das weite Land. Tragikomödie in fünf Akten}|pwv}\pwindex{Schnitzler, Arthur 15. 5. 1862 Wien – 21. 10. 1931 ebd.@\textsc{Schnitzler, Arthur} (15. 5. 1862 Wien – 21. 10. 1931 ebd.), \emph{Schriftsteller, Mediziner}!Le Pays de l’âme. Drame en 5 actes@\strich\emph{Le Pays de l’âme. Drame en 5 actes}|pwv} aufgeführt
               sein müsste, widrigenfalls die Rechte wieder ungeschmälert an mich zurückfielen. Ich
               würde als diesen Termin 31. Mai 1926 vorschlagen.\pend
           
\pstart
           Als selbstverständlich nehme ich weiter an, dass Mme. Cabire\pwindex{Cabire, Emma @\textsc{Cabire, Emma}, \emph{Übersetzerin, Redakteurin, Literaturagentin}|pw} Ihnen, verehrte Frau Hofrätin, die gleiche
               Vermittlungsprovision wie ich, nämlich 20 {\%} von ihren eigenen
               Einnahmen auszuzahlen sich verpflichtet.\pend
           
\pstart
           Wegen einer eventuellen Buchausgabe des »Weiten Land\pwindex{Schnitzler, Arthur 15. 5. 1862 Wien – 21. 10. 1931 ebd.@\textsc{Schnitzler, Arthur} (15. 5. 1862 Wien – 21. 10. 1931 ebd.), \emph{Schriftsteller, Mediziner}!weite Land. Tragikomödie in fünf Akten@\strich\emph{Das weite Land. Tragikomödie in fünf Akten}|pw}\pwindex{Schnitzler, Arthur 15. 5. 1862 Wien – 21. 10. 1931 ebd.@\textsc{Schnitzler, Arthur} (15. 5. 1862 Wien – 21. 10. 1931 ebd.), \emph{Schriftsteller, Mediziner}!Le Pays de l’âme. Drame en 5 actes@\strich\emph{Le Pays de l’âme. Drame en 5 actes}|pw}« erwarte ich gerne Vorschläge von verlegerischer Seite.\pend
           
\pstart
           In der bestimmten Erwartung, dass Ihre freundlichen Bemühungen von Erfolg
               begleitet sein werden, bin ich, verehrte Frau Hofrätin, mit herzlichem Dank und
               Gruss{\\[\baselineskip]} Ihr aufrichtig ergebener\pend
           \leftskip=0em{}{\vspace{1\baselineskip}}
\pstart
           \noindent{}Frau Hofrät in Berta Zuckerkandl,{\\}Wien\oindex{Wien@\textbf{Wien}, \emph{Verwaltungsgebiet}|pw}.\pend
           \selectlanguage{ngerman}\endnumbering\briefempfaengerindex{Zuckerkandl, Berta@\textsc{Zuckerkandl, Berta}!zzzSchnitzler, Arthur@\emph{von Arthur Schnitzler}!1925-12-243@{24. 12. 1925}|)be}\mylabel{L03962h}
\begin{anhang}
\end{anhang}\newcommand{\dateiname}{L03962}\newcommand{\titel}{Arthur Schnitzler an Berta Zuckerkandl, 24. 12. 1925}\newcommand{\editorInnen}{Herausgegeben von Jahnke, SelmaMüller, Martin Anton}%% latex-leseansicht-abspann.tex
%% Abspann für die Leseansicht.
%% Der Schalter \ifkorrekturansicht ist bereits durch den Vorspann gesetzt.

%% latex-abspann.tex
%% Gemeinsamer Abspann für Korrekturansicht und Leseansicht.
%% Setzt den Schalter \ifkorrekturansicht voraus (gesetzt in den
%% einbindenden Dateien latex-korrekturansicht-abspann.tex bzw.
%% latex-leseansicht-abspann.tex).
%% ---------------------------------------------------------------

\normalsize

% Das esempio-Environment wird nur in der Leseansicht benötigt
\ifkorrekturansicht\else
\newenvironment{esempio}[3]%
{
    \vspace{1.5ex}
    \rlap{\underline{#1}}
    \par
    \setlength{\parindent}{0cm}
    \nopagebreak
    \leftskip=#2cm
    \rightskip=#3cm
}
{
    \par
}
\fi

\doendnotes{C}
\bigskip
\vfill

\clearpage

\footnotesize

\ifkorrekturansicht
  \lohead{\textsc{register}}
\fi

% theindex-Environment neu definieren ohne reledmac
\makeatletter
\renewenvironment{theindex}{%
  \ifkorrekturansicht
    \section*{\indexname}%
  \else
    \subsubsection*{Index der erwähnten Entitäten}%
  \fi
  \setlength{\parindent}{0pt}%
  \setlength{\parskip}{0pt plus 0.3pt}%
  \let\item\@idxitem
}{%
  \ifkorrekturansicht\clearpage\fi
}
\makeatother

\IfFileExists{\jobname-pw.ind}{\input{\jobname-pw.ind}}{}

% Quellenangabe nur in der Leseansicht
\ifkorrekturansicht\else
% Fallback-Definitionen, falls die .tex-Datei \titel etc. nicht gesetzt hat
\providecommand{\titel}{}
\providecommand{\editorInnen}{}
\providecommand{\dateiname}{\jobname}

\vspace{3cm}

\vfill

\footnotesize
\textsc{Quelle}: \titel. Herausgegeben von {\editorInnen}. In: \emph{Arthur Schnitzler: Briefwechsel mit Autorinnen und Autoren}.
 Digitale Edition, https://schnitzler-briefe.acdh.oeaw.ac.at/{\dateiname}.html (Stand \today)
\fi

\end{document}


