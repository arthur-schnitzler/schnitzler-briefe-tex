%% latex-leseansicht-vorspann.tex
%% Vorspann für die Leseansicht.
%% Lädt die gemeinsame Datei latex-vorspann.tex mit nicht gesetztem Schalter.

\newif\ifkorrekturansicht
\korrekturansichtfalse

\input{../tex-inputs/latex-vorspann}


         
         \renewcommand{\erwaehntePersonen}{Personen:  ?? [Schauspieler], Peter Altenberg, Richard Beer-Hofmann, Mirjam Beer-Hofmann, Paula Beer-Hofmann, Georg Hirschfeld, Ricarda Huch, Ernst Moriz Kronfeld, Felix Rappaport, Paul Rittinger, Wilhelm Schölermann, Gustav Schönaich}
         \renewcommand{\erwaehnteOrte}{Orte: Ebensee, Spiegelgasse, Steindorf am Ossiacher See, Wien}
         \renewcommand{\erwaehnteWerke}{Werke: Burne-Jones, Der botanische Poet (Anton Kerner v. Marilaun†), Der grüne Kakadu – Paracelsus – Die Gefährtin. Drei Einakter, Der grüne Kakadu. Groteske in einem Akt, Dichter, Die Engländer und die Franzosen in der Jubiläums-Ausstellung, Die Maiwiese, Diese ist sein, Neue Freie Presse, Riesengebirge, Wiener Rundschau}
               \section[Richard Beer-Hofmann an Arthur Schnitzler, 3. 7. 1898]{ Richard Beer-Hofmann an Arthur Schnitzler, 3. 7. 1898}\nopagebreak\mylabel{v}\rehead{ }\begin{ledgroupsized}[t]{13cm}\normalsize\beginnumbering\briefempfaengerindex{Schnitzler, Arthur@\textsc{Schnitzler, Arthur}!zzzBeer-Hofmann, Richard@\emph{von Richard Beer-Hofmann}!1898-07-031@{3. 7. 1898}|(be} \toendnotes[C]{\smallbreak\pagebreak[2]} \Standort{CUL, Schnitzler, B 8.}
\physDesc{Brief, 1 Blatt, 4 Seiten, 952 Zeichen
\newline{}Handschrift: Bleistift, lateinische Kurrent
\newline{}Ordnung: mit Bleistift von unbekannter Hand nummeriert:
                                    »118« }\buchAbdrucke{\weitereDrucke{Arthur Schnitzler, Richard Beer-Hofmann: \emph{Briefwechsel 1891–1931}. Hg. Konstanze Fliedl. Wien, Zürich: \emph{Europaverlag} 1992, S. 121–122.} }\toendnotes[C]{\smallbreak}\pstart
           \raggedleft{}{\pb}3/7 98\pend
           \pstart
           Lieber Arthur! Brief Cigaretten, Tasche, erhalten, – danke sehr.\pend
           \pstart
           Im August werden wir uns hoffentlich treffen nur wird sich das Nähere
               voraussichtlich erst im August feststellen lassen. Mirjam\pwindex{Beer-Hofmann, Mirjam 04.09.1897 – 24.12.1984@\textsc{Beer-Hofmann, Mirjam} (04.09.1897 – 24.12.1984)|pw} und Paula\pwindex{Beer-Hofmann, Paula 25.02.1879 – 30.10.1939@\textsc{Beer-Hofmann, Paula} (25.02.1879 – 30.10.1939)|pw} hab ich
               Ihren Traum erzählt; man {\pb}dankt.
               Der zudringliche Mime\pwindex{?? [Schauspieler] 3.7.1898 – 3.7.1898@\textsc{?? [Schauspieler]} (3.7.1898 – 3.7.1898)|pwv} hat
                  \uline{mir} richtig von Ebensee\oindex{Ebensee@\textbf{Ebensee}|pw} aus eine Ansichtskarte mit Grüßen gesandt – Ein Viech! – Ich
               arbeite, aber nicht genug – leider schlaf ich auch nur täglich von ½ 11
               bis 2–3 Uhr nachts. Zu wenig. Ich erhalte {\pb}soeben die N. Fr. Presse\pwindex{Neue Freie Presse1864 – 1939@\emph{Neue Freie Presse} {[}1864 – 1939{]}|pw} von heute – (Sonntag
                  3/VII){[}.{]} Lese darin die Inhaltsangabe der »Wiener Rundschau\pwindex{Wiener Rundschau1896 – 1901@\emph{Wiener Rundschau} {[}1896 – 1901{]}|pw}« und werde nervös. Wenn Sie die
                  \label{K_L00811-1v}\edtext{Inhaltsangabe}{\lemma{\textnormal{\emph{Inhaltsangabe}}}\Cendnote{\textnormal{›— ›\so{Wiener Rundschau}\pwindex{Wiener Rundschau1896 – 1901@\emph{Wiener Rundschau} {[}1896 – 1901{]}|pw}.‹ (Herausgeber Gustav \so{Schoenaich}\pwindex{Schoenaich, Gustav 1840-11-24 – 1906-04-04@\textsc{Schönaich, Gustav} (1840-11-24 – 1906-04-04), \emph{Journalist, Musikwissenschaftler, Jurist}|pw}, Felix \so{Rappaport}\pwindex{Rappaport, Felix 1874-11-17 – 1939-12-08@\textsc{Rappaport, Felix} (1874-11-17 – 1939-12-08), \emph{Herausgeber, Finanzier, Lyriker}|pw}.) Nr. 16 (II. Jahrgang) vom 1. Juli 1898 hat folgenden
                        Inhalt: Die Maiwiese\pwindex{Huch, Ricarda 18.07.1864 – 17.11.1947@\textsc{Huch, Ricarda} (18.07.1864 – 17.11.1947), \emph{Schriftstellerin}!Maiwiese1898-07-01@\strich\emph{Die Maiwiese} {[}1898-07-01{]}|pw}. Von Ricarda \so{Huch}\pwindex{Huch, Ricarda 18.07.1864 – 17.11.1947@\textsc{Huch, Ricarda} (18.07.1864 – 17.11.1947), \emph{Schriftstellerin}|pw}. — Burne-Jones\pwindex{Schoelermann, Wilhelm 1865-01-06 – 1923-05-04@\textsc{Schölermann, Wilhelm} (1865-01-06 – 1923-05-04), \emph{Schriftsteller, Übersetzer}!Burne-Jones1898-07-01@\strich\emph{Burne-Jones} {[}1898-07-01{]}|pw}. Von Wilhelm \so{Schölermann}\pwindex{Schoelermann, Wilhelm 1865-01-06 – 1923-05-04@\textsc{Schölermann, Wilhelm} (1865-01-06 – 1923-05-04), \emph{Schriftsteller, Übersetzer}|pw}. — Riesengebirge\pwindex{Hirschfeld, Georg 11.02.1873 – 17.01.1942@\textsc{Hirschfeld, Georg} (11.02.1873 – 17.01.1942), \emph{Schriftsteller}!Riesengebirge1898-07-01@\strich\emph{Riesengebirge} {[}1898-07-01{]}|pw}. Dichter\pwindex{Hirschfeld, Georg 11.02.1873 – 17.01.1942@\textsc{Hirschfeld, Georg} (11.02.1873 – 17.01.1942), \emph{Schriftsteller}!Dichter1898-07-01@\strich\emph{Dichter} {[}1898-07-01{]}|pw}. Von Georg
                              \so{Hirschfeld}\pwindex{Hirschfeld, Georg 11.02.1873 – 17.01.1942@\textsc{Hirschfeld, Georg} (11.02.1873 – 17.01.1942), \emph{Schriftsteller}|pw}. — Der botanische Poet. (Anton Kerner
                           v. Marilaun †.)\pwindex{Kronfeld, Ernst Moriz 1865-06-03 – 1942-03-16@\textsc{Kronfeld, Ernst Moriz} (1865-06-03 – 1942-03-16), \emph{Schriftsteller, Redakteur, Sammler}!botanische Poet (Anton Kerner v. Marilaun†)1898-07-01@\strich\emph{Der botanische Poet (Anton Kerner v. Marilaun†)} {[}1898-07-01{]}|pw} Von M. \so{Kronfeld}\pwindex{Kronfeld, Ernst Moriz 1865-06-03 – 1942-03-16@\textsc{Kronfeld, Ernst Moriz} (1865-06-03 – 1942-03-16), \emph{Schriftsteller, Redakteur, Sammler}|pw}. — Diese ist sein.\pwindex{Altenberg, Peter 09.03.1859 – 08.01.1919@\textsc{Altenberg, Peter} (09.03.1859 – 08.01.1919), \emph{Schriftsteller}!Diese ist sein1898-07-01@\strich\emph{Diese ist sein} {[}1898-07-01{]}|pw} Von Peter \so{Altenberg}\pwindex{Altenberg, Peter 09.03.1859 – 08.01.1919@\textsc{Altenberg, Peter} (09.03.1859 – 08.01.1919), \emph{Schriftsteller}|pw}. — Die Engländer und die Franzosen in
                           der Jubiläums-Ausstellung.\pwindex{Rittinger, Paul 1879-04-08 – 1953-01-23@\textsc{Rittinger, Paul} (1879-04-08 – 1953-01-23), \emph{Maler, Musikwissenschaftler, Zeichner}!Englaender und die Franzosen in der Jubilaeums-Ausstellung1898-07-01@\strich\emph{Die Engländer und die Franzosen in der Jubiläums-Ausstellung} {[}1898-07-01{]}|pw} Von Paul Ritter v. \so{Rittinger}\pwindex{Rittinger, Paul 1879-04-08 – 1953-01-23@\textsc{Rittinger, Paul} (1879-04-08 – 1953-01-23), \emph{Maler, Musikwissenschaftler, Zeichner}|pw}. — Notizen. — Preis per Quartal 2 fl. Redaction und Administration:
                           Wien, 1/1, Spiegelgasse Nr. 11\oindex{Spiegelgasse@\textbf{Spiegelgasse}|pw}.‹ (\emph{Neue Freie Presse}\pwindex{Neue Freie Presse1864 – 1939@\emph{Neue Freie Presse} {[}1864 – 1939{]}|pwk}, Nr. 12162,
                        3. 7. 1898, S. 9.) Vermutlich dürfte er
                  irrtümlicherweise den Text Altenberg\pwindex{Altenberg, Peter 09.03.1859 – 08.01.1919@\textsc{Altenberg, Peter} (09.03.1859 – 08.01.1919), \emph{Schriftsteller}|pwk}s auf
                  sich bezogen haben.}}}\label{K_L00811-1h} lesen werden Sie ahnen warum: Verfolgungswahn? –
               Schicken Sie mir jedenfalls gleich – bitte – die betreffende Nu{\geminationm}er (N\textsuperscript{r.} 16).\pend
           \pstart
           {\pb}Ich habe eben nur die Empfindung
               daß von dieser Seite etwas gegen mich vorbereitet wird. Wenn möglich lachen Sie mich
               aus – hoffentlich ist Grund dazu – zum Auslachen\pend
           \pstart
           Ihre Stücke\pwindex{Schnitzler, Arthur 15.05.1862 – 21.10.1931@\textsc{Schnitzler, Arthur} (15.05.1862 – 21.10.1931), \emph{Schriftsteller, Mediziner}!gruene Kakadu – Paracelsus – Die Gefaehrtin. Drei Einakter1898 – 1899@\strich\emph{Der grüne Kakadu – Paracelsus – Die Gefährtin. Drei Einakter} {[}1898 – 1899{]}|pwv}? Wie heißen sie?
                  Kakadu\pwindex{Schnitzler, Arthur 15.05.1862 – 21.10.1931@\textsc{Schnitzler, Arthur} (15.05.1862 – 21.10.1931), \emph{Schriftsteller, Mediziner}!gruene Kakadu. Groteske in einem Akt1. 3. 1899@\strich\emph{Der grüne Kakadu. Groteske in einem Akt} {[}1. 3. 1899{]}|pw}\pwindex{Schnitzler, Arthur 15.05.1862 – 21.10.1931@\textsc{Schnitzler, Arthur} (15.05.1862 – 21.10.1931), \emph{Schriftsteller, Mediziner}!gruene Kakadu. Groteske in einem Akt1. 3. 1899@\strich\emph{Der grüne Kakadu. Groteske in einem Akt} {[}1. 3. 1899{]}|pw} und – –?\pend
           \pstart
           Herzlichst Ihr {\\[\baselineskip]}\spacefill\mbox{Richard}\pend
           \leftskip=0em{}
         
         \endnumbering\mylabel{h}\end{ledgroupsized}  \newcommand{\dateiname}{L00811}\newcommand{\titel}{Richard Beer-Hofmann an Arthur Schnitzler, 3. 7. 1898}\newcommand{\editorInnen}{Martin Anton Müller und Gerd-Hermann Susen}%% latex-leseansicht-abspann.tex
%% Abspann für die Leseansicht.
%% Der Schalter \ifkorrekturansicht ist bereits durch den Vorspann gesetzt.

%% latex-abspann.tex
%% Gemeinsamer Abspann für Korrekturansicht und Leseansicht.
%% Setzt den Schalter \ifkorrekturansicht voraus (gesetzt in den
%% einbindenden Dateien latex-korrekturansicht-abspann.tex bzw.
%% latex-leseansicht-abspann.tex).
%% ---------------------------------------------------------------

\normalsize

% Das esempio-Environment wird nur in der Leseansicht benötigt
\ifkorrekturansicht\else
\newenvironment{esempio}[3]%
{
    \vspace{1.5ex}
    \rlap{\underline{#1}}
    \par
    \setlength{\parindent}{0cm}
    \nopagebreak
    \leftskip=#2cm
    \rightskip=#3cm
}
{
    \par
}
\fi

\doendnotes{C}
\bigskip
\vfill

\clearpage

\footnotesize

\ifkorrekturansicht
  \lohead{\textsc{register}}
\fi

% theindex-Environment neu definieren ohne reledmac
\makeatletter
\renewenvironment{theindex}{%
  \ifkorrekturansicht
    \section*{\indexname}%
  \else
    \subsubsection*{Index der erwähnten Entitäten}%
  \fi
  \setlength{\parindent}{0pt}%
  \setlength{\parskip}{0pt plus 0.3pt}%
  \let\item\@idxitem
}{%
  \ifkorrekturansicht\clearpage\fi
}
\makeatother

\IfFileExists{\jobname-pw.ind}{\input{\jobname-pw.ind}}{}

% Quellenangabe nur in der Leseansicht
\ifkorrekturansicht\else
% Fallback-Definitionen, falls die .tex-Datei \titel etc. nicht gesetzt hat
\providecommand{\titel}{}
\providecommand{\editorInnen}{}
\providecommand{\dateiname}{\jobname}

\vspace{3cm}

\vfill

\footnotesize
\textsc{Quelle}: \titel. Herausgegeben von {\editorInnen}. In: \emph{Arthur Schnitzler: Briefwechsel mit Autorinnen und Autoren}.
 Digitale Edition, https://schnitzler-briefe.acdh.oeaw.ac.at/{\dateiname}.html (Stand \today)
\fi

\end{document}


      