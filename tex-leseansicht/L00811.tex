%% latex-leseansicht-vorspann.tex
%% Vorspann für die Leseansicht.
%% Lädt die gemeinsame Datei latex-vorspann.tex mit nicht gesetztem Schalter.

\newif\ifkorrekturansicht
\korrekturansichtfalse

\input{../tex-inputs/latex-vorspann}


\section[Richard Beer-Hofmann an Arthur Schnitzler, 3. 7. 1898]{L00811 Richard Beer-Hofmann an Arthur Schnitzler, 3. 7. 1898}
\nopagebreak\mylabel{L00811v}
\rehead{ }\normalsize\beginnumbering\briefempfaengerindex{Schnitzler, Arthur@\textsc{Schnitzler, Arthur}!zzzBeer-Hofmann, Richard@\emph{von Richard Beer-Hofmann}!1898-07-031@{3. 7. 1898}|(be}
\toendnotes[C]{\smallbreak\pagebreak[2]}
\correspDesc{Versand  durch Richard Beer-Hofmann am 3. 7. 1898 in Steindorf am Ossiacher See
\newline{}Erhalt  durch Arthur Schnitzler im Zeitraum [4. 7. 1898
                  – 8. 7. 1898?] in Wien}\toendnotes[C]{\smallbreak}
\Standort{CUL, Schnitzler, B 8.}
\physDesc{Brief, 1 Blatt, 4 Seiten, 952 Zeichen
\newline{}Handschrift: Bleistift, lateinische Kurrent
\newline{}Ordnung: mit Bleistift von unbekannter Hand nummeriert:
                                    »118« }
\buchAbdrucke{\weitereDrucke{Arthur Schnitzler, Richard Beer-Hofmann: \emph{Briefwechsel 1891–1931}. Herausgegeben von Konstanze Fliedl. Wien, Zürich: \emph{Europaverlag} 1992, S. 121–122.} }\toendnotes[C]{\smallbreak}
\pstart
           \raggedleft{}{\pb}3/7 98\pend
           \vspace{0.5em}
\pstart
           Lieber Arthur! Brief Cigaretten, Tasche, erhalten, – danke sehr.\pend
           
\pstart
           Im August werden wir uns hoffentlich treffen nur wird sich das Nähere
               voraussichtlich erst im August feststellen lassen. Mirjam\pwindex{Beer-Hofmann, Mirjam 4.\,9.\,1897 Wien – 24.\,12.\,1984 New York City@\textsc{Beer-Hofmann, Mirjam} (4.\,9.\,1897 Wien – 24.\,12.\,1984 New York City)|pw} und Paula\pwindex{Beer-Hofmann, Paula 25.\,2.\,1879 Wien – 30.\,10.\,1939 Zürich@\textsc{Beer-Hofmann, Paula} (25.\,2.\,1879 Wien – 30.\,10.\,1939 Zürich)|pw} hab ich
               Ihren Traum erzählt; man {\pb}dankt.
               Der zudringliche Mime\pwindex{?? [Schauspieler] 3.\,7.\,1898 – 3.\,7.\,1898@\textsc{?? [Schauspieler]} (3.\,7.\,1898 – 3.\,7.\,1898)|pwv} hat
                  \uline{mir} richtig von Ebensee\oindex{Ebensee am Traunsee@\textbf{Ebensee am Traunsee}|pw} aus eine Ansichtskarte mit Grüßen gesandt – Ein Viech! – Ich
               arbeite, aber nicht genug – leider schlaf ich auch nur täglich von ½ 11
               bis 2–3 Uhr nachts. Zu wenig. Ich erhalte {\pb}soeben die N. Fr. Presse\pwindex{Neue Freie Presse@\emph{Neue Freie Presse}|pw} von heute – (Sonntag 3/VII){[}.{]} Lese darin die Inhaltsangabe der »Wiener Rundschau\pwindex{Wiener Rundschau@\emph{Wiener Rundschau}|pw}« und werde nervös. Wenn Sie die
               \label{K_L00811-1v}\edtext{Inhaltsangabe}{\lemma{\textnormal{\emph{Inhaltsangabe}}}\Cendnote{\textnormal{\emph{Neue Freie Presse}\pwindex{Neue Freie Presse@\emph{Neue Freie Presse}|pwk}, Nr. 12.162,
                  3. 7. 1898, S. 9: »– ›\so{Wiener Rundschau}\pwindex{Wiener Rundschau@\emph{Wiener Rundschau}|pw}.‹ (Herausgeber Gustav \so{Schoenaich}\pwindex{Schönaich, Gustav 24.\,11.\,1840 Wien – 4.\,4.\,1906@\textsc{Schönaich, Gustav} (24.\,11.\,1840 Wien – 4.\,4.\,1906), \emph{Journalist, Musikwissenschaftler, Jurist}|pw}, Felix \so{Rappaport}\pwindex{Rappaport, Felix 17.\,11.\,1874 Wien – 8.\,12.\,1939 ebd.@\textsc{Rappaport, Felix} (17.\,11.\,1874 Wien – 8.\,12.\,1939 ebd.), \emph{Herausgeber, Finanzier, Lyriker}|pw}.) Nr. 16 (II. Jahrgang) vom 1. Juli 1898 hat folgenden
                        Inhalt: Die Maiwiese\pwindex{Huch, Ricarda 18.\,7.\,1864 Braunschweig – 17.\,11.\,1947 Schönberg@\textsc{Huch, Ricarda} (18.\,7.\,1864 Braunschweig – 17.\,11.\,1947 Schönberg), \emph{Schriftstellerin}!Maiwiese@\strich\emph{Die Maiwiese}|pw}. Von Ricarda \so{Huch}\pwindex{Huch, Ricarda 18.\,7.\,1864 Braunschweig – 17.\,11.\,1947 Schönberg@\textsc{Huch, Ricarda} (18.\,7.\,1864 Braunschweig – 17.\,11.\,1947 Schönberg), \emph{Schriftstellerin}|pw}. – Burne-Jones\pwindex{Schölermann, Wilhelm 6.\,1.\,1865 Hamburg – 4.\,5.\,1923 Weimar@\textsc{Schölermann, Wilhelm} (6.\,1.\,1865 Hamburg – 4.\,5.\,1923 Weimar), \emph{Schriftsteller, Übersetzer}!Burne-Jones@\strich\emph{Burne-Jones}|pw}. Von Wilhelm \so{Schölermann}\pwindex{Schölermann, Wilhelm 6.\,1.\,1865 Hamburg – 4.\,5.\,1923 Weimar@\textsc{Schölermann, Wilhelm} (6.\,1.\,1865 Hamburg – 4.\,5.\,1923 Weimar), \emph{Schriftsteller, Übersetzer}|pw}. – Riesengebirge\pwindex{Hirschfeld, Georg 11.\,2.\,1873 Berlin – 17.\,1.\,1942 München@\textsc{Hirschfeld, Georg} (11.\,2.\,1873 Berlin – 17.\,1.\,1942 München), \emph{Schriftsteller}!Riesengebirge@\strich\emph{Riesengebirge}|pw}. Dichter\pwindex{Hirschfeld, Georg 11.\,2.\,1873 Berlin – 17.\,1.\,1942 München@\textsc{Hirschfeld, Georg} (11.\,2.\,1873 Berlin – 17.\,1.\,1942 München), \emph{Schriftsteller}!Dichter@\strich\emph{Dichter}|pw}. Von Georg
                              \so{Hirschfeld}\pwindex{Hirschfeld, Georg 11.\,2.\,1873 Berlin – 17.\,1.\,1942 München@\textsc{Hirschfeld, Georg} (11.\,2.\,1873 Berlin – 17.\,1.\,1942 München), \emph{Schriftsteller}|pw}. – Der botanische Poet. (Anton Kerner
                           v. Marilaun †.)\pwindex{Kronfeld, Ernst Moriz 3.\,6.\,1865 Lviv – 16.\,3.\,1942@\textsc{Kronfeld, Ernst Moriz} (3.\,6.\,1865 Lviv – 16.\,3.\,1942), \emph{Schriftsteller, Redakteur, Sammler}!botanische Poet (Anton Kerner v. Marilaun†)@\strich\emph{Der botanische Poet (Anton Kerner v. Marilaun†)}|pw} Von M. \so{Kronfeld}\pwindex{Kronfeld, Ernst Moriz 3.\,6.\,1865 Lviv – 16.\,3.\,1942@\textsc{Kronfeld, Ernst Moriz} (3.\,6.\,1865 Lviv – 16.\,3.\,1942), \emph{Schriftsteller, Redakteur, Sammler}|pw}. – Diese ist sein.\pwindex{Altenberg, Peter 9.\,3.\,1859 Wien – 8.\,1.\,1919 ebd.@\textsc{Altenberg, Peter} (9.\,3.\,1859 Wien – 8.\,1.\,1919 ebd.), \emph{Schriftsteller}!Diese ist sein@\strich\emph{Diese ist sein}|pw} Von Peter \so{Altenberg}\pwindex{Altenberg, Peter 9.\,3.\,1859 Wien – 8.\,1.\,1919 ebd.@\textsc{Altenberg, Peter} (9.\,3.\,1859 Wien – 8.\,1.\,1919 ebd.), \emph{Schriftsteller}|pw}. – Die Engländer und die Franzosen in
                           der Jubiläums-Ausstellung.\pwindex{Rittinger, Paul 8.\,4.\,1879 Hollabrunn – 23.\,1.\,1953 Innsbruck@\textsc{Rittinger, Paul} (8.\,4.\,1879 Hollabrunn – 23.\,1.\,1953 Innsbruck), \emph{Maler, Musikwissenschaftler, Zeichner}!Engländer und die Franzosen in der Jubiläums-Ausstellung@\strich\emph{Die Engländer und die Franzosen in der Jubiläums-Ausstellung}|pw} Von Paul Ritter v. \so{Rittinger}\pwindex{Rittinger, Paul 8.\,4.\,1879 Hollabrunn – 23.\,1.\,1953 Innsbruck@\textsc{Rittinger, Paul} (8.\,4.\,1879 Hollabrunn – 23.\,1.\,1953 Innsbruck), \emph{Maler, Musikwissenschaftler, Zeichner}|pw}. – Notizen. – Preis per Quartal 2 fl. Redaction und Administration:
                           Wien, 1/1, Spiegelgasse Nr. 11\oindex{Wien@\textbf{Wien}!I., Innere Stadt@\textbf{I., Innere Stadt}!Spiegelgasse@\textbf{Spiegelgasse}, \emph{Straße}|pw}.« Vermutlich hat Beer-Hofmann\pwindex{Beer-Hofmann, Richard 11.\,7.\,1866 Wien – 26.\,9.\,1945 New York City@\textsc{Beer-Hofmann, Richard} (11.\,7.\,1866 Wien – 26.\,9.\,1945 New York City), \emph{Schriftsteller}|pwk}
                  irrtümlicherweise den Text Altenbergs\pwindex{Altenberg, Peter 9.\,3.\,1859 Wien – 8.\,1.\,1919 ebd.@\textsc{Altenberg, Peter} (9.\,3.\,1859 Wien – 8.\,1.\,1919 ebd.), \emph{Schriftsteller}|pwk} auf
                  sich bezogen.}}}\label{K_L00811-1} lesen werden Sie ahnen warum: Verfolgungswahn? –
               Schicken Sie mir jedenfalls gleich – bitte – die betreffende Nu{\geminationm}er (N\textsuperscript{r.} 16).\pend
           
\pstart
           {\pb}Ich habe eben nur die Empfindung
               daß von dieser Seite etwas gegen mich vorbereitet wird. Wenn möglich lachen Sie mich
               aus – hoffentlich ist Grund dazu – zum Auslachen\pend
           
\pstart
           Ihre Stücke\pwindex{Schnitzler, Arthur 15.\,5.\,1862 Wien – 21.\,10.\,1931 ebd.@\textsc{Schnitzler, Arthur} (15.\,5.\,1862 Wien – 21.\,10.\,1931 ebd.), \emph{Schriftsteller, Mediziner}!grüne Kakadu – Paracelsus – Die Gefährtin. Drei Einakter@\strich\emph{Der grüne Kakadu – Paracelsus – Die Gefährtin. Drei Einakter}|pwv}? Wie heißen sie?
                  Kakadu\pwindex{Schnitzler, Arthur 15.\,5.\,1862 Wien – 21.\,10.\,1931 ebd.@\textsc{Schnitzler, Arthur} (15.\,5.\,1862 Wien – 21.\,10.\,1931 ebd.), \emph{Schriftsteller, Mediziner}!grüne Kakadu. Groteske in einem Akt@\strich\emph{Der grüne Kakadu. Groteske in einem Akt}|pw} und – –?\pend
           
\pstart
           Herzlichst Ihr {\\[\baselineskip]}\spacefill\mbox{Richard}\pend
           \leftskip=0em{}\selectlanguage{ngerman}\endnumbering\briefempfaengerindex{Schnitzler, Arthur@\textsc{Schnitzler, Arthur}!zzzBeer-Hofmann, Richard@\emph{von Richard Beer-Hofmann}!1898-07-031@{3. 7. 1898}|)be}\mylabel{L00811h}  \newcommand{\dateiname}{L00811}\newcommand{\titel}{Richard Beer-Hofmann an Arthur Schnitzler, 3. 7. 1898}\newcommand{\editorInnen}{Martin Anton Müller und Gerd-Hermann Susen}%% latex-leseansicht-abspann.tex
%% Abspann für die Leseansicht.
%% Der Schalter \ifkorrekturansicht ist bereits durch den Vorspann gesetzt.

%% latex-abspann.tex
%% Gemeinsamer Abspann für Korrekturansicht und Leseansicht.
%% Setzt den Schalter \ifkorrekturansicht voraus (gesetzt in den
%% einbindenden Dateien latex-korrekturansicht-abspann.tex bzw.
%% latex-leseansicht-abspann.tex).
%% ---------------------------------------------------------------

\normalsize

% Das esempio-Environment wird nur in der Leseansicht benötigt
\ifkorrekturansicht\else
\newenvironment{esempio}[3]%
{
    \vspace{1.5ex}
    \rlap{\underline{#1}}
    \par
    \setlength{\parindent}{0cm}
    \nopagebreak
    \leftskip=#2cm
    \rightskip=#3cm
}
{
    \par
}
\fi

\doendnotes{C}
\bigskip
\vfill

\clearpage

\footnotesize

\ifkorrekturansicht
  \lohead{\textsc{register}}
\fi

% theindex-Environment neu definieren ohne reledmac
\makeatletter
\renewenvironment{theindex}{%
  \ifkorrekturansicht
    \section*{\indexname}%
  \else
    \subsubsection*{Index der erwähnten Entitäten}%
  \fi
  \setlength{\parindent}{0pt}%
  \setlength{\parskip}{0pt plus 0.3pt}%
  \let\item\@idxitem
}{%
  \ifkorrekturansicht\clearpage\fi
}
\makeatother

\IfFileExists{\jobname-pw.ind}{\input{\jobname-pw.ind}}{}

% Quellenangabe nur in der Leseansicht
\ifkorrekturansicht\else
% Fallback-Definitionen, falls die .tex-Datei \titel etc. nicht gesetzt hat
\providecommand{\titel}{}
\providecommand{\editorInnen}{}
\providecommand{\dateiname}{\jobname}

\vspace{3cm}

\vfill

\footnotesize
\textsc{Quelle}: \titel. Herausgegeben von {\editorInnen}. In: \emph{Arthur Schnitzler: Briefwechsel mit Autorinnen und Autoren}.
 Digitale Edition, https://schnitzler-briefe.acdh.oeaw.ac.at/{\dateiname}.html (Stand \today)
\fi

\end{document}


