%% latex-korrekturansicht-vorspann.tex
%% Vorspann für die Korrekturansicht.
%% Lädt die gemeinsame Datei latex-vorspann.tex mit gesetztem Schalter.

\newif\ifkorrekturansicht
\korrekturansichttrue

\input{../tex-inputs/latex-vorspann}


\section[Richard Beer-Hofmann an Arthur Schnitzler, 3. 7. 1898]{L00811 Richard Beer-Hofmann an Arthur Schnitzler, 3. 7. 1898}
\nopagebreak\mylabel{L00811v}
\rehead{ }\normalsize\beginnumbering\briefempfaengerindex{Schnitzler, Arthur@\textsc{Schnitzler, Arthur}!zzzBeer-Hofmann, Richard@\emph{von Richard Beer-Hofmann}!1898-07-031@{3. 7. 1898}|(be}
\toendnotes[C]{\smallbreak\pagebreak[2]}\Standort{CUL, Schnitzler, B 8.}
\physDesc{Brief, 1 Blatt, 4 Seiten, 952 Zeichen
\newline{}Handschrift: Bleistift, lateinische Kurrent
\newline{}Ordnung: mit Bleistift von unbekannter Hand nummeriert:
                                    »118« }
\buchAbdrucke{\weitereDrucke{Arthur Schnitzler, Richard Beer-Hofmann: \emph{Briefwechsel 1891–1931}. Wien, Zürich: \emph{Europaverlag} 1992, S. 121–122.} }\toendnotes[C]{\smallbreak}
\pstart
           \raggedleft{}{\pb}3/7 98\pend
           \vspace{0.5em}
\pstart
           Lieber Arthur! Brief Cigaretten, Tasche, erhalten, – danke sehr.\pend
           
\pstart
           Im August werden wir uns hoffentlich treffen nur wird sich das Nähere
               voraussichtlich erst im August feststellen lassen. Mirjam\pwindex{Beer-Hofmann, Mirjam 04.09.1897 – 24.12.1984@\textsc{Beer-Hofmann, Mirjam} (04.09.1897 – 24.12.1984)|pw} und Paula\pwindex{Beer-Hofmann, Paula 25.02.1879 – 30.10.1939@\textsc{Beer-Hofmann, Paula} (25.02.1879 – 30.10.1939)|pw} hab ich
               Ihren Traum erzählt; man {\pb}dankt.
               Der zudringliche Mime\pwindex{?? [Schauspieler] 3.7.1898 – 3.7.1898@\textsc{?? [Schauspieler]} (3.7.1898 – 3.7.1898)|pwv} hat
                  \uline{mir} richtig von Ebensee\oindex{Ebensee am Traunsee@\textbf{Ebensee am Traunsee}, \emph{P.PPL}|pw} aus eine Ansichtskarte mit Grüßen gesandt – Ein Viech! – Ich
               arbeite, aber nicht genug – leider schlaf ich auch nur täglich von ½ 11
               bis 2–3 Uhr nachts. Zu wenig. Ich erhalte {\pb}soeben die N. Fr. Presse\pwindex{Neue Freie Presse@\emph{Neue Freie Presse}|pw} von heute – (Sonntag
                  3/VII){[}.{]} Lese darin die Inhaltsangabe der »Wiener Rundschau\pwindex{Wiener Rundschau@\emph{Wiener Rundschau}|pw}« und werde nervös. Wenn Sie die
               \label{K_L00811-1v}\edtext{Inhaltsangabe}{\lemma{\textnormal{\emph{Inhaltsangabe}}}\Cendnote{\textnormal{\emph{Neue Freie Presse}\pwindex{Neue Freie Presse@\emph{Neue Freie Presse}|pwk}, Nr. 12.162,
                  3. 7. 1898, S. 9: »– ›\so{Wiener Rundschau}\pwindex{Wiener Rundschau@\emph{Wiener Rundschau}|pw}.‹ (Herausgeber Gustav \so{Schoenaich}\pwindex{Schoenaich, Gustav 1840-11-24 – 1906-04-04@\textsc{Schönaich, Gustav} (1840-11-24 – 1906-04-04), \emph{Journalist/Journalistin, Musikwissenschaftler/Musikwissenschaftlerin, Jurist/Juristin}|pw}, Felix \so{Rappaport}\pwindex{Rappaport, Felix 1874-11-17 – 1939-12-08@\textsc{Rappaport, Felix} (1874-11-17 – 1939-12-08), \emph{Herausgeber/Herausgeberin, Finanzier/Finanzierin, Lyriker/Lyrikerin}|pw}.) Nr. 16 (II. Jahrgang) vom 1. Juli 1898 hat folgenden
                        Inhalt: Die Maiwiese\pwindex{Maiwiese@\emph{Die Maiwiese}|pw}. Von Ricarda \so{Huch}\pwindex{Huch, Ricarda 18.07.1864 – 17.11.1947@\textsc{Huch, Ricarda} (18.07.1864 – 17.11.1947), \emph{Schriftsteller/Schriftstellerin}|pw}. – Burne-Jones\pwindex{Burne-Jones@\emph{Burne-Jones}|pw}. Von Wilhelm \so{Schölermann}\pwindex{Schoelermann, Wilhelm 1865-01-06 – 1923-05-04@\textsc{Schölermann, Wilhelm} (1865-01-06 – 1923-05-04), \emph{Schriftsteller/Schriftstellerin, Übersetzer/Übersetzerin}|pw}. – Riesengebirge\pwindex{Riesengebirge@\emph{Riesengebirge}|pw}. Dichter\pwindex{Dichter@\emph{Dichter}|pw}. Von Georg
                              \so{Hirschfeld}\pwindex{Hirschfeld, Georg 11.02.1873 – 17.01.1942@\textsc{Hirschfeld, Georg} (11.02.1873 – 17.01.1942), \emph{Schriftsteller/Schriftstellerin}|pw}. – Der botanische Poet. (Anton Kerner
                           v. Marilaun †.)\pwindex{botanische Poet (Anton Kerner v. Marilaun†)@\emph{Der botanische Poet (Anton Kerner v. Marilaun†)}|pw} Von M. \so{Kronfeld}\pwindex{Kronfeld, Ernst Moriz 1865-06-03 – 1942-03-16@\textsc{Kronfeld, Ernst Moriz} (1865-06-03 – 1942-03-16), \emph{Schriftsteller/Schriftstellerin, Redakteur/Redakteurin, Sammler/Sammlerin}|pw}. – Diese ist sein.\pwindex{Diese ist sein@\emph{Diese ist sein}|pw} Von Peter \so{Altenberg}\pwindex{Altenberg, Peter 09.03.1859 – 08.01.1919@\textsc{Altenberg, Peter} (09.03.1859 – 08.01.1919), \emph{Schriftsteller/Schriftstellerin}|pw}. – Die Engländer und die Franzosen in
                           der Jubiläums-Ausstellung.\pwindex{Englaender und die Franzosen in der Jubilaeums-Ausstellung@\emph{Die Engländer und die Franzosen in der Jubiläums-Ausstellung}|pw} Von Paul Ritter v. \so{Rittinger}\pwindex{Rittinger, Paul 1879-04-08 – 1953-01-23@\textsc{Rittinger, Paul} (1879-04-08 – 1953-01-23), \emph{Maler/Malerin, Musikwissenschaftler/Musikwissenschaftlerin, Zeichner/Zeichnerin}|pw}. – Notizen. – Preis per Quartal 2 fl. Redaction und Administration:
                           Wien, 1/1, Spiegelgasse Nr. 11\oindex{Spiegelgasse@\textbf{Spiegelgasse}, \emph{Straße (K.STR)}|pw}.« Vermutlich hat Beer-Hofmann\pwindex{Beer-Hofmann, Richard 1866-07-11 – 1945-09-26@\textsc{Beer-Hofmann, Richard} (1866-07-11 – 1945-09-26), \emph{Schriftsteller/Schriftstellerin}|pwk}
                  irrtümlicherweise den Text Altenbergs\pwindex{Altenberg, Peter 09.03.1859 – 08.01.1919@\textsc{Altenberg, Peter} (09.03.1859 – 08.01.1919), \emph{Schriftsteller/Schriftstellerin}|pwk} auf
                  sich bezogen.}}}\label{K_L00811-1} lesen werden Sie ahnen warum: Verfolgungswahn? –
               Schicken Sie mir jedenfalls gleich – bitte – die betreffende Nu{\geminationm}er (N\textsuperscript{r.} 16).\pend
           
\pstart
           {\pb}Ich habe eben nur die Empfindung
               daß von dieser Seite etwas gegen mich vorbereitet wird. Wenn möglich lachen Sie mich
               aus – hoffentlich ist Grund dazu – zum Auslachen\pend
           
\pstart
           Ihre Stücke\pwindex{gruene Kakadu – Paracelsus – Die Gefaehrtin. Drei Einakter@\emph{Der grüne Kakadu – Paracelsus – Die Gefährtin. Drei Einakter}|pwv}? Wie heißen sie?
                  Kakadu\pwindex{gruene Kakadu. Groteske in einem Akt@\emph{Der grüne Kakadu. Groteske in einem Akt}|pw} und – –?\pend
           
\pstart
           Herzlichst Ihr {\\[\baselineskip]}\spacefill\mbox{Richard}\pend
           \leftskip=0em{}\selectlanguage{ngerman}\endnumbering\briefempfaengerindex{Schnitzler, Arthur@\textsc{Schnitzler, Arthur}!zzzBeer-Hofmann, Richard@\emph{von Richard Beer-Hofmann}!1898-07-031@{3. 7. 1898}|)be}\mylabel{L00811h}  \normalsize

\doendnotes{C}
\bigskip
\vfill

\clearpage

\footnotesize

\lohead{\textsc{register}}

% Definiere theindex-Environment komplett neu ohne reledmac
\makeatletter
\renewenvironment{theindex}{%
  \section*{\indexname}%
  \setlength{\parindent}{0pt}%
  \setlength{\parskip}{0pt plus 0.3pt}%
  \let\item\@idxitem
}{%
  \clearpage
}
\makeatother

\IfFileExists{\jobname-pw.ind}{\input{\jobname-pw.ind}}{}

\end{document}

      