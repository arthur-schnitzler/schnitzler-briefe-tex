%% latex-leseansicht-vorspann.tex
%% Vorspann für die Leseansicht.
%% Lädt die gemeinsame Datei latex-vorspann.tex mit nicht gesetztem Schalter.

\newif\ifkorrekturansicht
\korrekturansichtfalse

\input{../tex-inputs/latex-vorspann}


         
         \newcommand{\erwaehntePersonen}{Personen: }
         \newcommand{\erwaehnteInstitutionen}{}
         \newcommand{\erwaehnteOrte}{}
         \newcommand{\erwaehnteWerke}{
               \section[Richard Beer-Hofmann an Arthur Schnitzler, 3. 7. 1898]{ Richard Beer-Hofmann an Arthur Schnitzler, 3. 7. 1898}\nopagebreak\mylabel{v}\rehead{ }\begin{ledgroupsized}[t]{13cm}\normalsize\beginnumbering \toendnotes[C]{\smallbreak\pagebreak[2]} \Standort{CUL, Schnitzler, B 8.}
\physDesc{Brief, 1 Blatt, 4 Seiten
\newline{}Handschrift: Bleistift, lateinische Kurrent\newline{}Ordnung: mit Bleistift von unbekannter Hand nummeriert:
                                    »118« }\buchAbdrucke{\weitereDrucke{Arthur Schnitzler, Richard Beer-Hofmann: \emph{Briefwechsel 1891–1931}. Hg. Konstanze Fliedl. Wien, Zürich: \emph{Europaverlag} 1992, S. 121–122.} }\toendnotes[C]{\smallbreak}\pstart
           \raggedleft{}{\pb}3/7 98\pend
           \pstart
           Lieber Arthur! Brief Cigaretten, Tasche, erhalten, – danke sehr.\pend
           \pstart
           Im August werden wir uns hoffentlich treffen nur wird sich das Nähere
               voraussichtlich erst im August feststellen lassen. Mirjam\pwindex{\textcolor{red}{\textsuperscript{XXXX1 indx}}|pw} und Paula\pwindex{\textcolor{red}{\textsuperscript{XXXX1 indx}}|pw} hab ich
               Ihren Traum erzählt; man {\pb}dankt.
               Der zudringliche Mime\pwindex{\textcolor{red}{\textsuperscript{XXXX1 indx}}|pwv} hat \uline{mir} richtig von Ebensee\oindex{XXXX Ortsangabe fehlt|pw} aus eine Ansichtskarte mit Grüßen gesandt – Ein Viech! – Ich
               arbeite, aber nicht genug – leider schlaf ich auch nur täglich von ½ 11
               bis 2–3 Uhr nachts. Zu wenig. Ich erhalte {\pb}soeben die N. Fr. Presse\textcolor{red}{\textsuperscript{XXXX indx}} von heute – (Sonntag
                  3/VII){[}.{]} Lese darin die Inhaltsangabe der »Wiener Rundschau\textcolor{red}{\textsuperscript{XXXX indx}}« und werde nervös. Wenn Sie die
                  \label{K_L00811_1v}\edtext{Inhaltsangabe}{\lemma{\textnormal{\emph{Inhaltsangabe}}}\Cendnote{\textnormal{›— ›\so{Wiener Rundschau}\textcolor{red}{\textsuperscript{XXXX indx}}.‹ (Herausgeber Gustav \so{Schoenaich}\pwindex{\textcolor{red}{\textsuperscript{XXXX1 indx}}|pw}, Felix \so{Rappaport}\pwindex{\textcolor{red}{\textsuperscript{XXXX1 indx}}|pw}.) Nr. 16 (II. Jahrgang) vom 1. Juli 1898 hat folgenden
                        Inhalt: Die Maiwiese\textcolor{red}{\textsuperscript{XXXX indx}}. Von Ricarda \so{Huch}\pwindex{\textcolor{red}{\textsuperscript{XXXX1 indx}}|pw}. — Burne-Jones\textcolor{red}{\textsuperscript{XXXX indx}}. Von Wilhelm \so{Schölermann}\pwindex{\textcolor{red}{\textsuperscript{XXXX1 indx}}|pw}. — Riesengebirge\textcolor{red}{\textsuperscript{XXXX indx}}. Dichter\textcolor{red}{\textsuperscript{XXXX indx}}. Von Georg \so{Hirschfeld}\pwindex{\textcolor{red}{\textsuperscript{XXXX1 indx}}|pw}. — Der botanische Poet. (Anton Kerner v.
                           Marilaun †.)\textcolor{red}{\textsuperscript{XXXX indx}} Von M. \so{Kronfeld}\pwindex{\textcolor{red}{\textsuperscript{XXXX1 indx}}|pw}. — Diese ist sein.\textcolor{red}{\textsuperscript{XXXX indx}} Von Peter \so{Altenberg}\pwindex{\textcolor{red}{\textsuperscript{XXXX1 indx}}|pw}. — Die Engländer und die Franzosen in der
                           Jubiläums-Ausstellung.\textcolor{red}{\textsuperscript{XXXX indx}} Von Paul
                           Ritter v. \so{Rittinger}\pwindex{\textcolor{red}{\textsuperscript{XXXX1 indx}}|pw}. — Notizen. — Preis per Quartal 2 fl. Redaction und Administration:
                           Wien, 1/1, Spiegelgasse Nr. 11\oindex{XXXX Ortsangabe fehlt|pw}.‹ (\emph{Neue Freie Presse}\textcolor{red}{\textsuperscript{XXXX indx}}, Nr. 12162,
                        3. 7. 1898, S. 9.) Vermutlich dürfte er irrtümlicherweise den Text Altenberg\pwindex{\textcolor{red}{\textsuperscript{XXXX1 indx}}|pwk}s auf sich bezogen haben.}}}\label{K_L00811_1h} lesen werden Sie ahnen warum: Verfolgungswahn? – Schicken Sie
               mir jedenfalls gleich – bitte – die betreffende Nu{\geminationm}er
                  (N\textsuperscript{r.} 16).\pend
           \pstart
           {\pb}Ich habe eben nur die Empfindung
               daß von dieser Seite etwas gegen mich vorbereitet wird. Wenn möglich lachen Sie mich
               aus – hoffentlich ist Grund dazu – zum Auslachen\pend
           \pstart
           Ihre Stücke\textcolor{red}{\textsuperscript{XXXX indx}}? Wie heißen sie? Kakadu\textcolor{red}{\textsuperscript{XXXX indx}} und – –?\pend
           \pstart
           Herzlichst Ihr {\\[\baselineskip]}\spacefill\mbox{Richard}\pend
           \leftskip=0em{}
         
         \endnumbering\mylabel{h}\end{ledgroupsized}  \newcommand{\dateiname}{L00811}\newcommand{\titel}{Richard Beer-Hofmann an Arthur Schnitzler, 3. 7. 1898}\newcommand{\editorInnen}{Martin Anton Müller und Gerd-Hermann Susen}%% latex-leseansicht-abspann.tex
%% Abspann für die Leseansicht.
%% Der Schalter \ifkorrekturansicht ist bereits durch den Vorspann gesetzt.

%% latex-abspann.tex
%% Gemeinsamer Abspann für Korrekturansicht und Leseansicht.
%% Setzt den Schalter \ifkorrekturansicht voraus (gesetzt in den
%% einbindenden Dateien latex-korrekturansicht-abspann.tex bzw.
%% latex-leseansicht-abspann.tex).
%% ---------------------------------------------------------------

\normalsize

% Das esempio-Environment wird nur in der Leseansicht benötigt
\ifkorrekturansicht\else
\newenvironment{esempio}[3]%
{
    \vspace{1.5ex}
    \rlap{\underline{#1}}
    \par
    \setlength{\parindent}{0cm}
    \nopagebreak
    \leftskip=#2cm
    \rightskip=#3cm
}
{
    \par
}
\fi

\doendnotes{C}
\bigskip
\vfill

\clearpage

\footnotesize

\ifkorrekturansicht
  \lohead{\textsc{register}}
\fi

% theindex-Environment neu definieren ohne reledmac
\makeatletter
\renewenvironment{theindex}{%
  \ifkorrekturansicht
    \section*{\indexname}%
  \else
    \subsubsection*{Index der erwähnten Entitäten}%
  \fi
  \setlength{\parindent}{0pt}%
  \setlength{\parskip}{0pt plus 0.3pt}%
  \let\item\@idxitem
}{%
  \ifkorrekturansicht\clearpage\fi
}
\makeatother

\IfFileExists{\jobname-pw.ind}{\input{\jobname-pw.ind}}{}

% Quellenangabe nur in der Leseansicht
\ifkorrekturansicht\else
% Fallback-Definitionen, falls die .tex-Datei \titel etc. nicht gesetzt hat
\providecommand{\titel}{}
\providecommand{\editorInnen}{}
\providecommand{\dateiname}{\jobname}

\vspace{3cm}

\vfill

\footnotesize
\textsc{Quelle}: \titel. Herausgegeben von {\editorInnen}. In: \emph{Arthur Schnitzler: Briefwechsel mit Autorinnen und Autoren}.
 Digitale Edition, https://schnitzler-briefe.acdh.oeaw.ac.at/{\dateiname}.html (Stand \today)
\fi

\end{document}


      