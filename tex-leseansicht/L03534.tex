%% latex-leseansicht-vorspann.tex
%% Vorspann für die Leseansicht.
%% Lädt die gemeinsame Datei latex-vorspann.tex mit nicht gesetztem Schalter.

\newif\ifkorrekturansicht
\korrekturansichtfalse

\input{../tex-inputs/latex-vorspann}


\section[ Paul Goldmann an Olga Gussmann, 29. 9. {[}1902{]}]{L03534 Paul Goldmann an Olga Gussmann,  29. 9. [1902]}
\nopagebreak\mylabel{L03534v}
\rehead{ }\normalsize\beginnumbering\briefempfaengerindex{Schnitzler, Olga@\textsc{Schnitzler, Olga}!zzzGoldmann, Paul@\emph{von Paul Goldmann}!1902-09-291@{29. 9. [1902]}|(be}
\toendnotes[C]{\smallbreak\pagebreak[2]}
\correspDesc{Versand  durch Paul Goldmann am 29. 9. [1902] in Berlin
\newline{}Erhalt  durch Olga Gussmann im Zeitraum [30. 9. 1902
                  – 4. 10. 1902?] in Wien}\toendnotes[C]{\smallbreak}
\Standort{DLA, A:Schnitzler, HS.NZ85.1.5247.}
\physDesc{Brief, 1 Blatt, 3 Seiten, 857 Zeichen
\newline{}Handschrift: blaue Tinte, deutsche Kurrent}\toendnotes[C]{\smallbreak}
\pstart
           \raggedleft{}{\pb}\textcolor{gray}{\textbf{DESSAUERSTRASSE 19\oindex{Dessauer Straße@\textbf{Dessauer Straße}, \emph{Straße}|pw}}}\pend
           
\pstart
           Berlin\oindex{Berlin@\textbf{Berlin}, \emph{Hauptstadt}|pw}, 29. September.\pend
           
\pstart\center{}Liebe Freundin,\pend\vspace{0.5em}
\pstart
           Ich habe mich{ }ſehr gefreut, einen Brief von Ihnen zu erhalten, weil dies das beſte
               Zeichen iſt, daß es Ihnen wohl ergeht.\pend
           
\pstart
           Das \label{K_L03534-1v}\edtext{Gewitter, das über \textsc{Liesls\pwindex{Steinrück, Elisabeth 19.\,11.\,1885 – 7.\,4.\,1920 Partenkirchen@\textsc{Steinrück, Elisabeth} (19.\,11.\,1885 – 7.\,4.\,1920 Partenkirchen)|pw}} Haupt{ }ſchwebte}{\lemma{\textnormal{\emph{Gewitter, … schwebte}}}\Cendnote{\textnormal{Elisabeth Gussmann\pwindex{Steinrück, Elisabeth 19.\,11.\,1885 – 7.\,4.\,1920 Partenkirchen@\textsc{Steinrück, Elisabeth} (19.\,11.\,1885 – 7.\,4.\,1920 Partenkirchen)|pwk} war ohne entsprechende
                  Dokumente für ihre Anstellung am \emph{Schiller-Theater}\orgindex{Schiller-Theater@Schiller-Theater|pwk} nach Berlin\oindex{Berlin@\textbf{Berlin}, \emph{Hauptstadt}|pwk} gezogen,
                     siehe A. S.: \emph{Tagebuch}, 25. 9. 1902.}}}\label{K_L03534-1},
               iſt einſtweilen beſchworen. Wir haben eine Friſt von einem Monat durch Intervention
               der Botſchaft erreicht. In
               dieſem Monat muß aber das fehlende Dokument {\pb}unbedingt beſchafft werden. Mit der preuß\oindex{Preußen@\textbf{Preußen}|pwv}iſchen Polizei\orgindex{Preußische Polzei@Preußische Polzei|pw} iſt nicht zu{ }ſchaffen. Es genügt, daß Ihr Vater\pwindex{Gussmann, Rudolf 5.\,3.\,1842 Veprovac – 24.\,1.\,1921 Wien@\textsc{Gussmann, Rudolf} (5.\,3.\,1842 Veprovac – 24.\,1.\,1921 Wien), \emph{Handelsagent}|pwv} das Verfahren wegen Erlangung{ }ſeiner
               Zuſtändigkeit \uline{einleitet}, um die Ausſtellung eines \uline{Interims}paſſes zu ermöglichen. Dazu wird man ihn doch
               wohl zwingen können?\pend
           
\pstart
           Auf die Frage: ob es mich »noch immer« intereſſirt, wenn Sie mir von{ }ſich und Ihrem
                  Buben\pwindex{Schnitzler, Heinrich 9.\,8.\,1902 Hinterbrühl – 12.\,7.\,1982 Wien@\textsc{Schnitzler, Heinrich} (9.\,8.\,1902 Hinterbrühl – 12.\,7.\,1982 Wien), \emph{Regisseur, Schauspieler}|pwv} erzählen, finde ich
                  {\pb}keine Antwort.\pend
           
\pstart
           Ich wünſche Ihnen einen glücklichen \label{K_L03534-2v}\edtext{Einzug in Wien\oindex{Wien@\textbf{Wien}, \emph{Verwaltungsgebiet}|pw}}{\lemma{\textnormal{\emph{Einzug in Wien}}}\Cendnote{\textnormal{Olga Gussmann\pwindex{Schnitzler, Olga 17.\,1.\,1882 Wien – 13.\,1.\,1970 Lugano@\textsc{Schnitzler, Olga} (17.\,1.\,1882 Wien – 13.\,1.\,1970 Lugano), \emph{Schauspielerin, Sängerin}|pwk} hatte für die meiste Zeit der
                  Schwangerschaft und die Geburt des gemeinsamen Sohnes Heinrich\pwindex{Schnitzler, Heinrich 9.\,8.\,1902 Hinterbrühl – 12.\,7.\,1982 Wien@\textsc{Schnitzler, Heinrich} (9.\,8.\,1902 Hinterbrühl – 12.\,7.\,1982 Wien), \emph{Regisseur, Schauspieler}|pwk} in Hinterbrühl\oindex{Hinterbrühl@\textbf{Hinterbrühl}, \emph{Hauptstadt}|pwk}
                  gelebt. Am 29. 9. 1902 übersiedelten sie und das Kind in die Gentzgasse 110\oindex{Wien@\textbf{Wien}!XVIII., Währing@\textbf{XVIII., Währing}!Gentzgasse@\textbf{Gentzgasse}, \emph{Straße}|pwk}. Hier blieb sie und das Kind bis zu ihrer Eheschließung wohnhaft, die am 26. 8. 1903 stattfand.}}}\label{K_L03534-2} und bin mit herzlichen Grüßen an Sie und \textsc{Arthur}{ }{\\[\baselineskip]}Ihr ergebener {\\[\baselineskip]}\spacefill\mbox{Dr. Paul Goldmann.}\pend
           \leftskip=0em{}\selectlanguage{ngerman}\endnumbering\briefempfaengerindex{Schnitzler, Olga@\textsc{Schnitzler, Olga}!zzzGoldmann, Paul@\emph{von Paul Goldmann}!1902-09-291@{29. 9. [1902]}|)be}\mylabel{L03534h}  \newcommand{\dateiname}{L03534}\newcommand{\titel}{Paul Goldmann an Olga Gussmann, 29. 9. [1902]}\newcommand{\editorInnen}{Martin Anton Müller und Laura Untner}%% latex-leseansicht-abspann.tex
%% Abspann für die Leseansicht.
%% Der Schalter \ifkorrekturansicht ist bereits durch den Vorspann gesetzt.

%% latex-abspann.tex
%% Gemeinsamer Abspann für Korrekturansicht und Leseansicht.
%% Setzt den Schalter \ifkorrekturansicht voraus (gesetzt in den
%% einbindenden Dateien latex-korrekturansicht-abspann.tex bzw.
%% latex-leseansicht-abspann.tex).
%% ---------------------------------------------------------------

\normalsize

% Das esempio-Environment wird nur in der Leseansicht benötigt
\ifkorrekturansicht\else
\newenvironment{esempio}[3]%
{
    \vspace{1.5ex}
    \rlap{\underline{#1}}
    \par
    \setlength{\parindent}{0cm}
    \nopagebreak
    \leftskip=#2cm
    \rightskip=#3cm
}
{
    \par
}
\fi

\doendnotes{C}
\bigskip
\vfill

\clearpage

\footnotesize

\ifkorrekturansicht
  \lohead{\textsc{register}}
\fi

% theindex-Environment neu definieren ohne reledmac
\makeatletter
\renewenvironment{theindex}{%
  \ifkorrekturansicht
    \section*{\indexname}%
  \else
    \subsubsection*{Index der erwähnten Entitäten}%
  \fi
  \setlength{\parindent}{0pt}%
  \setlength{\parskip}{0pt plus 0.3pt}%
  \let\item\@idxitem
}{%
  \ifkorrekturansicht\clearpage\fi
}
\makeatother

\IfFileExists{\jobname-pw.ind}{\input{\jobname-pw.ind}}{}

% Quellenangabe nur in der Leseansicht
\ifkorrekturansicht\else
% Fallback-Definitionen, falls die .tex-Datei \titel etc. nicht gesetzt hat
\providecommand{\titel}{}
\providecommand{\editorInnen}{}
\providecommand{\dateiname}{\jobname}

\vspace{3cm}

\vfill

\footnotesize
\textsc{Quelle}: \titel. Herausgegeben von {\editorInnen}. In: \emph{Arthur Schnitzler: Briefwechsel mit Autorinnen und Autoren}.
 Digitale Edition, https://schnitzler-briefe.acdh.oeaw.ac.at/{\dateiname}.html (Stand \today)
\fi

\end{document}


