%% latex-leseansicht-vorspann.tex
%% Vorspann für die Leseansicht.
%% Lädt die gemeinsame Datei latex-vorspann.tex mit nicht gesetztem Schalter.

\newif\ifkorrekturansicht
\korrekturansichtfalse

\input{../tex-inputs/latex-vorspann}

\begin{center}
            \textcolor{red}{ENTWURF, NICHT FERTIG KORRIGIERT}
                      \end{center}
            
         
         \renewcommand{\erwaehntePersonen}{Personen: Paul Goldmann, Rudolf Gussmann, Olga Schnitzler, Heinrich Schnitzler, Elisabeth Steinrück}
         \renewcommand{\erwaehnteInstitutionen}{Institutionen: Preußische Polzei, Schiller-Theater}
         \renewcommand{\erwaehnteOrte}{Orte: Berlin, Dessauer Straße, Gentzgasse, Hinterbrühl, Preußen, Wien}
         \renewcommand{\erwaehnteWerke}{}
               \section[ Paul Goldmann an Olga Gussmann, 29. 9. {[}1902{]}]{ Paul Goldmann an Olga Gussmann, 29. 9. {[}1902{]}}\nopagebreak\mylabel{v}\rehead{ }\begin{ledgroupsized}[t]{13cm}\normalsize\beginnumbering\briefempfaengerindex{Schnitzler, Olga@\textsc{Schnitzler, Olga}!zzzGoldmann, Paul@\emph{von Paul Goldmann}!1902-09-291@{29. 9. {[}1902{]}}|(be} \toendnotes[C]{\smallbreak\pagebreak[2]} \Standort{DLA, A:Schnitzler, HS.NZ85.1.5247.}
\physDesc{Brief, 1 Blatt, 3 Seiten, 853 Zeichen
\newline{}Handschrift: blaue Tinte, deutsche Kurrent}\toendnotes[C]{\smallbreak}\pstart
           \noindent{}\raggedleft{}{\pb}\textcolor{gray}{\textbf{DESSAUERSTRASSE 19\oindex{Dessauer Strasse@\textbf{Dessauer Straße}|pw}}}\pend
           \pstart
           Berlin\oindex{Berlin@\textbf{Berlin}|pw}, 29. September.\pend
           \pstart\center{}Liebe Freundin,\pend\pstart
           Ich habe mich ſehr gefreut, einen Brief von Ihnen zu erhalten, weil dies das beſte
               Zeichen iſt, daß es Ihnen wohl ergeht.\pend
           \pstart
           Das \label{K_L03534-1v}\edtext{Gewitter, das über \textsc{Liesl\pwindex{Steinrueck, Elisabeth 19.11.1885 – 07.04.1920@\textsc{Steinrück, Elisabeth} (19.11.1885 – 07.04.1920)|pw}s} Haupt ſchwebte}{\lemma{\textnormal{\emph{Gewitter, … ſchwebte}}}\Cendnote{\textnormal{Elisabeth Gussmann\pwindex{Steinrueck, Elisabeth 19.11.1885 – 07.04.1920@\textsc{Steinrück, Elisabeth} (19.11.1885 – 07.04.1920)|pwk} war ohne entsprechende
                  Dokumente für ihre Anstellung am \emph{Schiller-Theater}\orgindex{Schiller-Theater@Schiller-Theater|pwk} nach Berlin\oindex{Berlin@\textbf{Berlin}|pwk} gezogen,
                     siehe A. S.: \emph{Tagebuch}, 25. 9. 1902.}}}\label{K_L03534-1h},
               iſt einſtweilen beſchworen. Wir haben eine Friſt von einem Monat durch Intervention
               der Ort\oindex{Berlin@\textbf{Berlin}|pwv}ſchaft erreicht. In
               dieſem Monat muß aber das fehlende Dokument {\pb}unbedingt beſchafft werden. Mit der preuß\oindex{Preussen@\textbf{Preußen}|pwv}iſchen Polizei\orgindex{Preussische Polzei@Preußische Polzei|pw} iſt nicht zu ſchaffen. Es genügt, daß Ihr Vater\pwindex{Gussmann, Rudolf 05.03.1842 – 24.01.1921@\textsc{Gussmann, Rudolf} (05.03.1842 – 24.01.1921), \emph{Handelsagent}|pwv} das Verfahren wegen Erlangung ſeiner
               Zuſtändigkeit \uline{einleitet}, um die Ausſtellung eines \uline{Interims}paſſes zu ermöglichen. Dazu wird man in doch
               wohl zwingen können?\pend
           \pstart
           Auf die Frage: ob es mich noch immer intereſſirt, wenn Sie mir von ſich und Ihrem Buben\pwindex{Schnitzler, Heinrich 09.08.1902 – 12.07.1982@\textsc{Schnitzler, Heinrich} (09.08.1902 – 12.07.1982), \emph{Regisseur, Schauspieler}|pwv} erzählen, finde ich {\pb}keine Antwort.\pend
           \pstart
           Ich wünſche Ihnen einen glücklichen \label{K_L03534-2v}\edtext{Einzug in Wien\oindex{Wien@\textbf{Wien}|pw}}{\lemma{\textnormal{\emph{Einzug in Wien}}}\Cendnote{\textnormal{Olga Gussmann\pwindex{Schnitzler, Olga 17.01.1882 – 13.01.1970@\textsc{Schnitzler, Olga} (17.01.1882 – 13.01.1970), \emph{Schauspielerin, Sängerin}|pwk} war mit dem gemeinsamen Sohn
                     Heinrich\pwindex{Schnitzler, Heinrich 09.08.1902 – 12.07.1982@\textsc{Schnitzler, Heinrich} (09.08.1902 – 12.07.1982), \emph{Regisseur, Schauspieler}|pwk} am 29. 9. 1902 aus der
                     Hinterbrühl\oindex{Hinterbruehl@\textbf{Hinterbrühl}|pwk} zurückgekehrt. Fortan lebte sie
                  in Schnitzler\pwindex{Schnitzler, Arthur 15.05.1862 – 21.10.1931@\textsc{Schnitzler, Arthur} (15.05.1862 – 21.10.1931), \emph{Schriftsteller, Mediziner}|pwk}s alter Junggesellenwohnung in
                  der Gentzgasse 110\oindex{Gentzgasse@\textbf{Gentzgasse}|pwk}.}}}\label{K_L03534-2h} und bin mit
               herzlichen Grüßen an Sie und \textsc{Arthur\pwindex{Schnitzler, Arthur 15.05.1862 – 21.10.1931@\textsc{Schnitzler, Arthur} (15.05.1862 – 21.10.1931), \emph{Schriftsteller, Mediziner}|pw}}{ }{\\[\baselineskip]}Ihr ergebener {\\[\baselineskip]}\spacefill\mbox{Dr. Paul Goldmann.}\pend
           \leftskip=0em{}
         
         \endnumbering\mylabel{h}\end{ledgroupsized}\begin{anhang}\end{anhang}\newcommand{\dateiname}{L03534}\newcommand{\titel}{Paul Goldmann an Olga Gussmann, 29. 9. [1902]}\newcommand{\editorInnen}{Martin Anton Müller und Laura Untner}%% latex-leseansicht-abspann.tex
%% Abspann für die Leseansicht.
%% Der Schalter \ifkorrekturansicht ist bereits durch den Vorspann gesetzt.

%% latex-abspann.tex
%% Gemeinsamer Abspann für Korrekturansicht und Leseansicht.
%% Setzt den Schalter \ifkorrekturansicht voraus (gesetzt in den
%% einbindenden Dateien latex-korrekturansicht-abspann.tex bzw.
%% latex-leseansicht-abspann.tex).
%% ---------------------------------------------------------------

\normalsize

% Das esempio-Environment wird nur in der Leseansicht benötigt
\ifkorrekturansicht\else
\newenvironment{esempio}[3]%
{
    \vspace{1.5ex}
    \rlap{\underline{#1}}
    \par
    \setlength{\parindent}{0cm}
    \nopagebreak
    \leftskip=#2cm
    \rightskip=#3cm
}
{
    \par
}
\fi

\doendnotes{C}
\bigskip
\vfill

\clearpage

\footnotesize

\ifkorrekturansicht
  \lohead{\textsc{register}}
\fi

% theindex-Environment neu definieren ohne reledmac
\makeatletter
\renewenvironment{theindex}{%
  \ifkorrekturansicht
    \section*{\indexname}%
  \else
    \subsubsection*{Index der erwähnten Entitäten}%
  \fi
  \setlength{\parindent}{0pt}%
  \setlength{\parskip}{0pt plus 0.3pt}%
  \let\item\@idxitem
}{%
  \ifkorrekturansicht\clearpage\fi
}
\makeatother

\IfFileExists{\jobname-pw.ind}{\input{\jobname-pw.ind}}{}

% Quellenangabe nur in der Leseansicht
\ifkorrekturansicht\else
% Fallback-Definitionen, falls die .tex-Datei \titel etc. nicht gesetzt hat
\providecommand{\titel}{}
\providecommand{\editorInnen}{}
\providecommand{\dateiname}{\jobname}

\vspace{3cm}

\vfill

\footnotesize
\textsc{Quelle}: \titel. Herausgegeben von {\editorInnen}. In: \emph{Arthur Schnitzler: Briefwechsel mit Autorinnen und Autoren}.
 Digitale Edition, https://schnitzler-briefe.acdh.oeaw.ac.at/{\dateiname}.html (Stand \today)
\fi

\end{document}


      