%% latex-korrekturansicht-vorspann.tex
%% Vorspann für die Korrekturansicht.
%% Lädt die gemeinsame Datei latex-vorspann.tex mit gesetztem Schalter.

\newif\ifkorrekturansicht
\korrekturansichttrue

\input{../tex-inputs/latex-vorspann}


\section[ Paul Goldmann an Olga Gussmann, 29. 9. {[}1902{]}]{L03534 Paul Goldmann an Olga Gussmann, 29. 9. {[}1902{]}}
\nopagebreak\mylabel{L03534v}
\rehead{ }\normalsize\beginnumbering\briefempfaengerindex{Schnitzler, Olga@\textsc{Schnitzler, Olga}!zzzGoldmann, Paul@\emph{von Paul Goldmann}!1902-09-291@{29. 9. {[}1902{]}}|(be}
\toendnotes[C]{\smallbreak\pagebreak[2]}\Standort{DLA, A:Schnitzler, HS.NZ85.1.5247.}
\physDesc{Brief, 1 Blatt, 3 Seiten, 857 Zeichen
\newline{}Handschrift: blaue Tinte, deutsche Kurrent}\toendnotes[C]{\smallbreak}
\pstart
           \raggedleft{}{\pb}\textcolor{gray}{\textbf{DESSAUERSTRASSE 19\oindex{Dessauer Strasse@\textbf{Dessauer Straße}, \emph{Straße (K.STR)}|pw}}}\pend
           
\pstart
           Berlin\oindex{Berlin@\textbf{Berlin}, \emph{P.PPLC}|pw}, 29. September.\pend
           
\pstart\center{}Liebe Freundin,\pend\vspace{0.5em}
\pstart
           Ich habe mich ſehr gefreut, einen Brief von Ihnen zu erhalten, weil dies das beſte
               Zeichen iſt, daß es Ihnen wohl ergeht.\pend
           
\pstart
           Das \label{K_L03534-1v}\edtext{Gewitter, das über \textsc{Liesls\pwindex{Steinrueck, Elisabeth 19.11.1885 – 07.04.1920@\textsc{Steinrück, Elisabeth} (19.11.1885 – 07.04.1920)|pw}} Haupt ſchwebte}{\lemma{\textnormal{\emph{Gewitter, … ſchwebte}}}\Cendnote{\textnormal{Elisabeth Gussmann\pwindex{Steinrueck, Elisabeth 19.11.1885 – 07.04.1920@\textsc{Steinrück, Elisabeth} (19.11.1885 – 07.04.1920)|pwk} war ohne entsprechende
                  Dokumente für ihre Anstellung am \emph{Schiller-Theater}\orgindex{Schiller-Theater@Schiller-Theater|pwk} nach Berlin\oindex{Berlin@\textbf{Berlin}, \emph{P.PPLC}|pwk} gezogen,
                     siehe A. S.: \emph{Tagebuch}, 25. 9. 1902.}}}\label{K_L03534-1},
               iſt einſtweilen beſchworen. Wir haben eine Friſt von einem Monat durch Intervention
               der Botſchaft erreicht. In
               dieſem Monat muß aber das fehlende Dokument {\pb}unbedingt beſchafft werden. Mit der preuß\oindex{Preussen@\textbf{Preußen}, \emph{Land (A.LND)}|pwv}iſchen Polizei\orgindex{Preussische Polzei@Preußische Polzei|pw} iſt nicht zu ſchaffen. Es genügt, daß Ihr Vater\pwindex{Gussmann, Rudolf 05.03.1842 – 24.01.1921@\textsc{Gussmann, Rudolf} (05.03.1842 – 24.01.1921), \emph{Handelsagent/Handelsagentin}|pwv} das Verfahren wegen Erlangung ſeiner
               Zuſtändigkeit \uline{einleitet}, um die Ausſtellung eines \uline{Interims}paſſes zu ermöglichen. Dazu wird man ihn doch
               wohl zwingen können?\pend
           
\pstart
           Auf die Frage: ob es mich »noch immer« intereſſirt, wenn Sie mir von ſich und Ihrem
                  Buben\pwindex{Schnitzler, Heinrich 09.08.1902 – 12.07.1982@\textsc{Schnitzler, Heinrich} (09.08.1902 – 12.07.1982), \emph{Regisseur/Regisseurin, Schauspieler/Schauspielerin}|pwv} erzählen, finde ich
                  {\pb}keine Antwort.\pend
           
\pstart
           Ich wünſche Ihnen einen glücklichen \label{K_L03534-2v}\edtext{Einzug in Wien\oindex{Wien@\textbf{Wien}, \emph{A.ADM2}|pw}}{\lemma{\textnormal{\emph{Einzug in Wien}}}\Cendnote{\textnormal{Olga Gussmann\pwindex{Schnitzler, Olga 17.01.1882 – 13.01.1970@\textsc{Schnitzler, Olga} (17.01.1882 – 13.01.1970), \emph{Schauspieler/Schauspielerin, Sänger/Sängerin}|pwk} hatte für die meiste Zeit der
                  Schwangerschaft und die Geburt des gemeinsamen Sohnes Heinrich\pwindex{Schnitzler, Heinrich 09.08.1902 – 12.07.1982@\textsc{Schnitzler, Heinrich} (09.08.1902 – 12.07.1982), \emph{Regisseur/Regisseurin, Schauspieler/Schauspielerin}|pwk} in Hinterbrühl\oindex{Hinterbruehl@\textbf{Hinterbrühl}, \emph{P.PPLA3}|pwk}
                  gelebt. Am 29. 9. 1902 übersiedelten sie und das Kind in die Gentzgasse 110\oindex{Gentzgasse@\textbf{Gentzgasse}, \emph{Straße (K.STR)}|pwk}. Hier blieb sie und das Kind bis zu ihrer Eheschließung wohnhaft, die am 26. 8. 1903 stattfand.}}}\label{K_L03534-2} und bin mit herzlichen Grüßen an Sie und \textsc{Arthur}{ }{\\[\baselineskip]}Ihr ergebener {\\[\baselineskip]}\spacefill\mbox{Dr. Paul Goldmann.}\pend
           \leftskip=0em{}\selectlanguage{ngerman}\endnumbering\briefempfaengerindex{Schnitzler, Olga@\textsc{Schnitzler, Olga}!zzzGoldmann, Paul@\emph{von Paul Goldmann}!1902-09-291@{29. 9. {[}1902{]}}|)be}\mylabel{L03534h}  \normalsize

\doendnotes{C}
\bigskip
\vfill

\clearpage

\footnotesize

\lohead{\textsc{register}}

% Definiere theindex-Environment komplett neu ohne reledmac
\makeatletter
\renewenvironment{theindex}{%
  \section*{\indexname}%
  \setlength{\parindent}{0pt}%
  \setlength{\parskip}{0pt plus 0.3pt}%
  \let\item\@idxitem
}{%
  \clearpage
}
\makeatother

\IfFileExists{\jobname-pw.ind}{\input{\jobname-pw.ind}}{}

\end{document}

      