%% latex-leseansicht-vorspann.tex
%% Vorspann für die Leseansicht.
%% Lädt die gemeinsame Datei latex-vorspann.tex mit nicht gesetztem Schalter.

\newif\ifkorrekturansicht
\korrekturansichtfalse

\input{../tex-inputs/latex-vorspann}


         
         \renewcommand{\erwaehntePersonen}{Personen: Richard Beer-Hofmann, Paul Goldmann, Marie Reinhard}
         \renewcommand{\erwaehnteInstitutionen}{Institutionen: Norddeutscher Lloyd, Preussen}
         \renewcommand{\erwaehnteOrte}{Orte: Bad Ischl, Bremen, China, Deutsches Postamt in Shanghai, Indischer Ozean, Rotes Meer, Sri Lanka, Wien}
         \renewcommand{\erwaehnteWerke}{}
               \section[ Paul Goldmann an Arthur Schnitzler, 21. 4. {[}1898{]}]{ Paul Goldmann an Arthur Schnitzler, 21. 4. {[}1898{]}}\nopagebreak\mylabel{v}\rehead{ }\begin{ledgroupsized}[t]{13cm}\normalsize\beginnumbering\briefempfaengerindex{Schnitzler, Arthur@\textsc{Schnitzler, Arthur}!zzzGoldmann, Paul@\emph{von Paul Goldmann}!1898-04-212@{21. 4. {[}1898{]}}|(be} \toendnotes[C]{\smallbreak\pagebreak[2]} \Standort{DLA, A:Schnitzler, HS.NZ85.1.3168.}
\physDesc{Brief, 2 Blätter, 6 Seiten, 1891 Zeichen
\newline{}Handschrift: schwarze Tinte, deutsche Kurrent
\newline{}Schnitzler: mit Bleistift das Datum »21/4 98« vermerkt }\toendnotes[C]{\smallbreak}\pstart
           \noindent{}\centering{}{\pb}\textcolor{gray}{\textbf{DAMPFER »PREUSSEN«\orgindex{Preussen@Preussen|pw}}}\pend
           \pstart
           \noindent{}\textcolor{gray}{\textbf{NORDDEUTSCHER LLOYD\orgindex{Norddeutscher Lloyd@Norddeutscher Lloyd|pw} * BREMEN\oindex{Bremen@\textbf{Bremen}|pw} *}}\pend
           \pstart
           \raggedleft{}21. April, Indiſcher Ocean\oindex{Indischer Ozean@\textbf{Indischer Ozean}|pw}.\pend
           \pstart\center{}Mein lieber Freund,\pend\pstart
           Morgen iſt Poſtanſchluß in \textsc{Ceylon\oindex{Sri Lanka@\textbf{Sri Lanka}|pw}}, und ich will Dir einen herzlichen Gruß ſenden.\pend
           \pstart
           Die Reiſe iſt bisher wenig erfreulich. Ich leide abwechſelnd unter der Seekrankheit
               und unter der namenloſen Hitze. Das geht ſo ſeit dem Rothen Meer\oindex{Rotes Meer@\textbf{Rotes Meer}|pw}, alſo ſeit zehn Tagen {\pb}und es wird täglich ſchlimmer, je mehr wir an den
                  \textsc{Aequator} herankommen. Heut haben wir 36 Grad (\textsc{Celsius}), und dazu nicht
               ein Lüftchen Wind. In der Nacht gibt es keine Abkühlung, und die enge Cabine iſt ein
               entſetzlicher Aufenthalt. An Schlafen iſt kaum zu denken. Man dämmert ein paar
               Stunden hin zwiſchen Wachen u. Schlaf und ſpr\textcolor{gray}{i}ngt beim erſten
               Lichtſtrahl wieder auf die Beine, froh aus {\pb}dem
               dumpfen Kerkerloch herauszukommen. Dazu habe ich einen \strikeout{\textcolor{gray}{du}} durch Seekrankheit u. heißes Trinken unheilbar verdorbenen Magen. Und in \textsc{China\oindex{China@\textbf{China}|pw}} ſollen wir in den heißen Sommer hineinkommen! Das kann gut werden. Das
               Schlimmſte aber iſt, daß mir das Arbeiten ſo ſchlecht von der Hand geht. Ich zwinge
               mich dazu mit Aufwendung aller meiner Energie. {\pb}Jeden Satz quäle ich mir heraus, und es iſt ſchrecklich, wie unlebendig,
               unperſönlich und conventionell Alles herauskommt. Ich reihe mühſam Eindrückchen an
               Eindrückchen, und ich fühle, daß das Ganze kein Bild gibt. Das iſt tief verſtimmend,
               und ich fürchte, meine Reiſe wird journaliſtiſch ein \textsc{Fiasco}.\pend
           \pstart
           Sehr fehlen mir auch Deine lieben Nachrichten. Ich bitte Dich, mir gleich {\pb}nach \textsc{Shanghai\oindex{Deutsches Postamt in Shanghai@\textbf{Deutsches Postamt in Shanghai}|pw}}, \textsc{Deutsches Post-Amt\oindex{Deutsches Postamt in Shanghai@\textbf{Deutsches Postamt in Shanghai}|pw}, Poste Restante} zu
               ſchreiben u. dieſe Adreſſe auch für ſpäter beizubehalten, bis ich Dir Gegentheiliges
               angebe.\pend
           \pstart
           Was wirſt Du dieſen \label{K_L02846-1v}\edtext{Sommer
                  unternehmen}{\lemma{\textnormal{\emph{Sommer
                  unternehmen}}}\Cendnote{\textnormal{siehe Paul Goldmann an Arthur Schnitzler, 16. 5. 1898}}}\label{K_L02846-1h}? \textsc{Ischl\oindex{Bad Ischl@\textbf{Bad Ischl}|pw}}? Der Gedanke an einen \textsc{Ischl\oindex{Bad Ischl@\textbf{Bad Ischl}|pw}er} Tannen-Wald \strikeout{i\textcolor{gray}{n}} iſt {\pb}wahrhaft ſchmerzlich an einem verſengenden
                  Indiſchen-Ocean\oindex{Indischer Ozean@\textbf{Indischer Ozean}|pw}-Tage, wo man nach Luft und
               Kühlung ſchmachtet. Warum bin ich auch auf dieſes verfluchte Meer\oindex{Indischer Ozean@\textbf{Indischer Ozean}|pwv} hinausgefahren!\pend
           \pstart
           Ich grüße Dich u. den lieben \textsc{Richard\pwindex{Beer-Hofmann, Richard 1866-07-11 – 1945-09-26@\textsc{Beer-Hofmann, Richard} (1866-07-11 – 1945-09-26), \emph{Schriftsteller}|pw}} von ganzem Herzen.\pend
           \pstart
           Dein treuer {\\[\baselineskip]}\spacefill\mbox{Paul Goldmn}\pend
           \leftskip=0em{}\pstart
           \noindent{}Herzlichen Gruß an Deine Freundin\pwindex{Reinhard, Marie 1871-03-13 – 1899-03-18@\textsc{Reinhard, Marie} (1871-03-13 – 1899-03-18), \emph{Gesangspädagogin}|pwv}!\pend
           
         
         \endnumbering\mylabel{h}\end{ledgroupsized}  \newcommand{\dateiname}{L02846}\newcommand{\titel}{Paul Goldmann an Arthur Schnitzler, 21. 4. [1898]}\newcommand{\editorInnen}{Martin Anton Müller und Laura Untner}%% latex-leseansicht-abspann.tex
%% Abspann für die Leseansicht.
%% Der Schalter \ifkorrekturansicht ist bereits durch den Vorspann gesetzt.

%% latex-abspann.tex
%% Gemeinsamer Abspann für Korrekturansicht und Leseansicht.
%% Setzt den Schalter \ifkorrekturansicht voraus (gesetzt in den
%% einbindenden Dateien latex-korrekturansicht-abspann.tex bzw.
%% latex-leseansicht-abspann.tex).
%% ---------------------------------------------------------------

\normalsize

% Das esempio-Environment wird nur in der Leseansicht benötigt
\ifkorrekturansicht\else
\newenvironment{esempio}[3]%
{
    \vspace{1.5ex}
    \rlap{\underline{#1}}
    \par
    \setlength{\parindent}{0cm}
    \nopagebreak
    \leftskip=#2cm
    \rightskip=#3cm
}
{
    \par
}
\fi

\doendnotes{C}
\bigskip
\vfill

\clearpage

\footnotesize

\ifkorrekturansicht
  \lohead{\textsc{register}}
\fi

% theindex-Environment neu definieren ohne reledmac
\makeatletter
\renewenvironment{theindex}{%
  \ifkorrekturansicht
    \section*{\indexname}%
  \else
    \subsubsection*{Index der erwähnten Entitäten}%
  \fi
  \setlength{\parindent}{0pt}%
  \setlength{\parskip}{0pt plus 0.3pt}%
  \let\item\@idxitem
}{%
  \ifkorrekturansicht\clearpage\fi
}
\makeatother

\IfFileExists{\jobname-pw.ind}{\input{\jobname-pw.ind}}{}

% Quellenangabe nur in der Leseansicht
\ifkorrekturansicht\else
% Fallback-Definitionen, falls die .tex-Datei \titel etc. nicht gesetzt hat
\providecommand{\titel}{}
\providecommand{\editorInnen}{}
\providecommand{\dateiname}{\jobname}

\vspace{3cm}

\vfill

\footnotesize
\textsc{Quelle}: \titel. Herausgegeben von {\editorInnen}. In: \emph{Arthur Schnitzler: Briefwechsel mit Autorinnen und Autoren}.
 Digitale Edition, https://schnitzler-briefe.acdh.oeaw.ac.at/{\dateiname}.html (Stand \today)
\fi

\end{document}


      