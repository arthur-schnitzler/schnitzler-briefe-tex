%% latex-leseansicht-vorspann.tex
%% Vorspann für die Leseansicht.
%% Lädt die gemeinsame Datei latex-vorspann.tex mit nicht gesetztem Schalter.

\newif\ifkorrekturansicht
\korrekturansichtfalse

\input{../tex-inputs/latex-vorspann}


\section[ Paul Goldmann an Arthur Schnitzler, 19. 11. {[}1897{]}]{L02831 Paul Goldmann an Arthur Schnitzler,  19. 11. [1897]}
\nopagebreak\mylabel{L02831v}
\rehead{ }\normalsize\beginnumbering\briefempfaengerindex{Schnitzler, Arthur@\textsc{Schnitzler, Arthur}!zzzGoldmann, Paul@\emph{von Paul Goldmann}!1897-11-192@{19. 11. [1897]}|(be}
\toendnotes[C]{\smallbreak\pagebreak[2]}
\correspDesc{Versand  durch Paul Goldmann am 19. 11. [1897] in Paris
\newline{}Erhalt  durch Arthur Schnitzler im Zeitraum [20. 11. 1897 – 24. 11. 1897?] in Wien}\toendnotes[C]{\smallbreak}
\Standort{DLA, A:Schnitzler, HS.NZ85.1.3167.}
\physDesc{Brief, 2 Blätter, 8 Seiten, 3356 Zeichen
\newline{}Handschrift: blaue Tinte, deutsche Kurrent
\newline{}Schnitzler: 1) mit Bleistift das Jahr »97« vermerkt  2) mit rotem Buntstift drei Unterstreichungen}\toendnotes[C]{\smallbreak}
\pstart
           {\pb}\textcolor{gray}{\textbf{\textbf{Frankfurter Zeitung\orgindex{Frankfurter Zeitung@Frankfurter Zeitung|pw}}}}\pend
           
\pstart
           \textcolor{gray}{\textbf{(\begin{otherlanguage}{french}Gazette de Francfort\end{otherlanguage}\orgindex{Frankfurter Zeitung@Frankfurter Zeitung|pw}).}}\pend
           
\pstart
           \textcolor{gray}{\textbf{\textbf{\begin{otherlanguage}{french}Fondateur M.\end{otherlanguage}{ }L. Sonnemann\pwindex{Sonnemann, Leopold 29.\,10.\,1831 Höchberg – 30.\,10.\,1909 Frankfurt am Main@\textsc{Sonnemann, Leopold} (29.\,10.\,1831 Höchberg – 30.\,10.\,1909 Frankfurt am Main), \emph{Journalist, Herausgeber}|pw}.}}}\pend
           
\pstart
           \begin{otherlanguage}{french}\textcolor{gray}{\textbf{Journal politique, financier,}}\end{otherlanguage}\pend
           
\pstart
           \begin{otherlanguage}{french}\textcolor{gray}{\textbf{commercial et littéraire.}}\end{otherlanguage}\pend
           
\pstart
           \begin{otherlanguage}{french}\textcolor{gray}{\textbf{\textbf{Paraissant trois fois par jour.}}}\end{otherlanguage}\pend
           
\pstart
           \begin{otherlanguage}{french}\textcolor{gray}{\textbf{\textbf{Bureau à Paris\oindex{Paris@\textbf{Paris}, \emph{Hauptstadt}|pw}}}}\end{otherlanguage}\hfill \textsc{Paris\oindex{Paris@\textbf{Paris}, \emph{Hauptstadt}|pw}}, 19. Nov.\pend
           
\pstart
           \begin{otherlanguage}{french}\textcolor{gray}{\textbf{\textbf{10 \so{Rue de la Bourse}\oindex{rue de la Bourse@\textbf{rue de la Bourse}, \emph{Straße}|pw}.}}}\end{otherlanguage}\pend
           
\pstart\center{}Mein lieber Freund,\pend\vspace{0.5em}
\pstart
           Ich{ }ſchreibe Dir heut nur in Kürze, um mich zu
                  entſchuldigen\strikeout{\textcolor{gray}{.}} und Dir für Deine Nachſicht zu danken. Seit Wochen warte ich vergebens auf
               eine freie Stunde, um \strikeout{\textcolor{gray}{×}\-\textcolor{gray}{×}\-\textcolor{gray}{×}} Dir zu \strikeout{ſ\textcolor{gray}{c}h\textcolor{gray}{×}}{ }ſchreiben. Seit ich Deinen letzten,{ }ſo{ }ſchönen und ergreifenden Brief mit der
               traurigen Nachricht erhielt, vergeht kein Tag, wo ich nicht mit der Abſicht aufſtehe:
               Heut wird geſchrieben. Aber die Ereigniſſe{ }ſind erbarmungslos und laſſen mich nicht
               zu Athem kommen. \strikeout{Du} Du kannſt Dir nicht vorſtellen,
               welche Zeit wir {\pb}hier durchmachen. Es geht zu wie im
               Tollhaus. Seit Wochen leiſte ich übermenſchliche Arbeits-Anſtrengungen. Du verfolgſt
               ja vielleicht auch von fern das Wiedererwachen der Affaire \textsc{Dreyfus\pwindex{Dreyfus, Alfred 9.\,10.\,1859 Mulhouse – 12.\,7.\,1935 Paris@\textsc{Dreyfus, Alfred} (9.\,10.\,1859 Mulhouse – 12.\,7.\,1935 Paris), \emph{Militär}|pw}}. Seit ich Journaliſt bin, habe ich etwas{ }ſo Aufregendes nicht miterlebt. Es
               wird allmälig eine Kriſis daraus, die das ganze Land\oindex{Frankreich@\textbf{Frankreich}|pwv} zu ergreifen beginnt. Es herrſcht eine Fieber-Athmoſphäre,
               und wenn man da mitten drin lebt und außerdem die Pflichten des Berufes erfüllen, das
               heißt{ }ſich Meinungen bilden und das Publicum informiren muß, und wenn man außerdem
               eine \label{K_L02831-1v}\edtext{perſönliche {\pb}Stellung in der Angelegenheit eingenommen}{\lemma{\textnormal{\emph{persönliche … eingenommen}}}\Cendnote{\textnormal{Siehe XXXX Auszeichnungsfehler: Dokument L02785 nicht gefunden.
               }}}\label{K_L02831-1} hat und keinen Tag die Zeitungen in die Hand nehmen kann, ohne fürchten zu
               müſſen,{ }ſich als Spion oder Verräther entehrt zu{ }ſehen, – wenn das Alles und noch
               mehr auf Einen einſtürmt,{ }ſo kannſt Du Dir denken, in welcher Gemüths- und
               Nerven-Verfaſſung man{ }ſich befindet. Die Ruhe, um auf Deine{ }ſo lieben und{ }ſchönen
               Briefe auch nur annähernd in einem \strikeout{ent} entſprechenden
               Tone zu antworten, iſt unmöglich zu finden. Nachdem {\pb}Du mir{ }ſolange verziehen haſt, verzeihſt Du mir wohl noch ein wenig, bis endlich,
               endlich \strikeout{d} die Stunde der Sammlung kommt, um Dir den{ }ſeit Wochen geplanten langen Brief zu{ }ſchreiben.\pend
           
\pstart
           Und nun habe ich noch eine große Bitte. Mit der \label{K_L02831-2v}\edtext{Familie \textsc{B.}\pwindex{Bondy, Charlotte 25.\,3.\,1854 Bielsko-Biała – 7.\,3.\,1914 Prag@\textsc{Bondy, Charlotte} (25.\,3.\,1854 Bielsko-Biała – 7.\,3.\,1914 Prag), \emph{Schauspielerin}|pwv}\pwindex{Bondy, Vít Šalomoun 9.\,12.\,1831 Prag – 31.\,10.\,1909 ebd.@\textsc{Bondy, Vít Šalomoun} (9.\,12.\,1831 Prag – 31.\,10.\,1909 ebd.), \emph{Fabrikant}|pwv}\pwindex{Ziegler, Alice 5.\,1.\,1880 Prag – Dezember 1943 Konzentrationslager Auschwitz-Birkenau@\textsc{Ziegler, Alice} (5.\,1.\,1880 Prag – Dezember 1943 Konzentrationslager Auschwitz-Birkenau)|pwv}}{\lemma{\textnormal{\emph{Familie B.}}}\Cendnote{\textnormal{Vít Šalomoun und Charlotte Bondy\pwindex{Bondy, Charlotte 25.\,3.\,1854 Bielsko-Biała – 7.\,3.\,1914 Prag@\textsc{Bondy, Charlotte} (25.\,3.\,1854 Bielsko-Biała – 7.\,3.\,1914 Prag), \emph{Schauspielerin}|pwk}\pwindex{Bondy, Vít Šalomoun 9.\,12.\,1831 Prag – 31.\,10.\,1909 ebd.@\textsc{Bondy, Vít Šalomoun} (9.\,12.\,1831 Prag – 31.\,10.\,1909 ebd.), \emph{Fabrikant}|pwk} und
                   die jüngere Tochter Alice\pwindex{Ziegler, Alice 5.\,1.\,1880 Prag – Dezember 1943 Konzentrationslager Auschwitz-Birkenau@\textsc{Ziegler, Alice} (5.\,1.\,1880 Prag – Dezember 1943 Konzentrationslager Auschwitz-Birkenau)|pwk} (nachmalig
                  verheiratete Ziegler\pwindex{Ziegler, Alice 5.\,1.\,1880 Prag – Dezember 1943 Konzentrationslager Auschwitz-Birkenau@\textsc{Ziegler, Alice} (5.\,1.\,1880 Prag – Dezember 1943 Konzentrationslager Auschwitz-Birkenau)|pwk})}}}\label{K_L02831-2} in \textsc{Prag\oindex{Prag@\textbf{Prag}, \emph{Land}|pw}} unterhalte ich eine Correſpondenz. Die Mutter\pwindex{Bondy, Charlotte 25.\,3.\,1854 Bielsko-Biała – 7.\,3.\,1914 Prag@\textsc{Bondy, Charlotte} (25.\,3.\,1854 Bielsko-Biała – 7.\,3.\,1914 Prag), \emph{Schauspielerin}|pwv}{ }ſcheint eine blöde Gans zu{ }ſein, das Mädchen\pwindex{Ziegler, Alice 5.\,1.\,1880 Prag – Dezember 1943 Konzentrationslager Auschwitz-Birkenau@\textsc{Ziegler, Alice} (5.\,1.\,1880 Prag – Dezember 1943 Konzentrationslager Auschwitz-Birkenau)|pwv} aber iſt wohl ein liebes Kind. Ich
               kann mir kaum \strikeout{\textcolor{gray}{de}} denken, daß alle Träume, welche ich{ }ſeit dieſer kurzen \textsc{Ischl\oindex{Bad Ischl@\textbf{Bad Ischl}|pw}er} Bekanntſchaft in mir herumtrage,
               jemals {\pb}zu Wirklichkeiten werden{ }ſollten. Aber es
               iſt mir eine Wohlthat, hier in der Heimatloſigkeit, in dieſer Hölle von Anſtrengungen
               und Aufregungen, an ein liebes Mädchen-Geſicht denken zu können, wie an eine
               Hoffnung. Darum bitte ich Dich recht{ }ſehr: \label{K_L02831-3v}\edtext{Geh’ zu den Leuten\pwindex{Bondy, Charlotte 25.\,3.\,1854 Bielsko-Biała – 7.\,3.\,1914 Prag@\textsc{Bondy, Charlotte} (25.\,3.\,1854 Bielsko-Biała – 7.\,3.\,1914 Prag), \emph{Schauspielerin}|pwv}\pwindex{Bondy, Vít Šalomoun 9.\,12.\,1831 Prag – 31.\,10.\,1909 ebd.@\textsc{Bondy, Vít Šalomoun} (9.\,12.\,1831 Prag – 31.\,10.\,1909 ebd.), \emph{Fabrikant}|pwv}\pwindex{Ziegler, Alice 5.\,1.\,1880 Prag – Dezember 1943 Konzentrationslager Auschwitz-Birkenau@\textsc{Ziegler, Alice} (5.\,1.\,1880 Prag – Dezember 1943 Konzentrationslager Auschwitz-Birkenau)|pwv} hin}{\lemma{\textnormal{\emph{Geh’ zu den Leuten hin}}}\Cendnote{\textnormal{Schnitzler traf Charlotte und Vít Bondy\pwindex{Bondy, Charlotte 25.\,3.\,1854 Bielsko-Biała – 7.\,3.\,1914 Prag@\textsc{Bondy, Charlotte} (25.\,3.\,1854 Bielsko-Biała – 7.\,3.\,1914 Prag), \emph{Schauspielerin}|pwk}\pwindex{Bondy, Vít Šalomoun 9.\,12.\,1831 Prag – 31.\,10.\,1909 ebd.@\textsc{Bondy, Vít Šalomoun} (9.\,12.\,1831 Prag – 31.\,10.\,1909 ebd.), \emph{Fabrikant}|pwk} bei seinem
                  Aufenthalt mehrfach, am 24. 11. 1897, 25. 11. 1897, 27. 11. 1897 und 28. 11. 1897.}}}\label{K_L02831-3} (\textsc{Mariengaſse 45\oindex{Wien@\textbf{Wien}!IX., Alsergrund@\textbf{IX., Alsergrund}!Mariannengasse@\textbf{Mariannengasse}, \emph{Straße}|pw}}),{ }ſchau Dir an, wer{ }ſie{ }ſind, höre auch, was die Anderen über{ }ſie{ }ſagen, und,
               wenn Du es für gut findeſt,{ }ſprich ein freundliches Wort über mich. Jedenfalls {\pb}aber{ }ſende mir einen recht ausführlichen Bericht!
               Ja? Das iſt ein wahrer Freundſchaftsdienſt, den ich verlange.\pend
           
\pstart
           Ich wünſche Dir von Herzen Glück zu Deiner \label{K_L02831-4v}\edtext{Vorleſung\pwindex{Schnitzler, Arthur 15.\,5.\,1862 Wien – 21.\,10.\,1931 ebd.@\textsc{Schnitzler, Arthur} (15.\,5.\,1862 Wien – 21.\,10.\,1931 ebd.), \emph{Schriftsteller, Mediziner}!Toten schweigen@\strich\emph{Die Toten schweigen}|pwv}\pwindex{Schnitzler, Arthur 15.\,5.\,1862 Wien – 21.\,10.\,1931 ebd.@\textsc{Schnitzler, Arthur} (15.\,5.\,1862 Wien – 21.\,10.\,1931 ebd.), \emph{Schriftsteller, Mediziner}!Weihnachts-Einkäufe@\strich\emph{Weihnachts-Einkäufe}|pwv} und Deiner
                  \textsc{Première\pwindex{Schnitzler, Arthur 15.\,5.\,1862 Wien – 21.\,10.\,1931 ebd.@\textsc{Schnitzler, Arthur} (15.\,5.\,1862 Wien – 21.\,10.\,1931 ebd.), \emph{Schriftsteller, Mediziner}!Freiwild. Schauspiel in 3 Akten@\strich\emph{Freiwild. Schauspiel in 3 Akten}|pwv}} in \textsc{Prag\oindex{Prag@\textbf{Prag}, \emph{Land}|pw}}}{\lemma{\textnormal{\emph{Vorlesung … Prag}}}\Cendnote{\textnormal{Siehe XXXX Auszeichnungsfehler: Dokument L02829 nicht gefunden.
               }}}\label{K_L02831-4} und grüße Dich Tauſend Mal in Treue\pend
           
\pstart
           Dein {\\[\baselineskip]}\spacefill\mbox{Paul Goldm}\pend
           \leftskip=0em{}
\pstart
           \noindent{}Ich{ }ſchreibe in höchſter Eile und kann Dir nur mit einem {\pb}Wort{ }ſagen, wie{ }ſehr mich die Nachricht vom
                     \label{K_L02831-5v}\edtext{Tode der armen Frau\pwindex{Waissnix, Olga 3.\,11.\,1862 Wien – 4.\,11.\,1897 ebd.@\textsc{Waissnix, Olga} (3.\,11.\,1862 Wien – 4.\,11.\,1897 ebd.), \emph{Hotelière}|pwv}}{\lemma{\textnormal{\emph{Tode der armen Frau}}}\Cendnote{\textnormal{Olga Waissnix\pwindex{Waissnix, Olga 3.\,11.\,1862 Wien – 4.\,11.\,1897 ebd.@\textsc{Waissnix, Olga} (3.\,11.\,1862 Wien – 4.\,11.\,1897 ebd.), \emph{Hotelière}|pwk} war am 4. 11. 1897 in Wien\oindex{Wien@\textbf{Wien}, \emph{Verwaltungsgebiet}|pwk} verstorben. Schnitzler hatte davon am
                        6. 11. 1897 erfahren.}}}\label{K_L02831-5} ergriffen hat. Wieder ein Stück Jugend unwiederbringlich verloren!
                  Wie{ }ſich um uns \strikeout{her} herum die Vergangenheit
                  auszudehnen beginnt, das Geweſene, – das nie mehr wieder{ }ſein wird, – das bereits
                  verbrauchte Leben! Und dieſe Ärmſte\pwindex{Waissnix, Olga 3.\,11.\,1862 Wien – 4.\,11.\,1897 ebd.@\textsc{Waissnix, Olga} (3.\,11.\,1862 Wien – 4.\,11.\,1897 ebd.), \emph{Hotelière}|pwv}, die fort mußte, ehe{ }ſie{ }ſich ausleben gekonnt, die wahrſcheinlich
                  erwartete, daß das Eigentliche noch kommen würde! Wie man{ }ſich alſo darauf
                  vorbereiten muß, daß das Ende eines {\pb}ſchönen
                  Tages kommen kann, ohne daß man Zeit gehabt hat, auch nur mit irgend etwas fertig
                  zu werden! Und dann, ohne lange Worte: die arme, liebe,{ }ſchöne Frau\pwindex{Waissnix, Olga 3.\,11.\,1862 Wien – 4.\,11.\,1897 ebd.@\textsc{Waissnix, Olga} (3.\,11.\,1862 Wien – 4.\,11.\,1897 ebd.), \emph{Hotelière}|pwv}!!\pend
           \selectlanguage{ngerman}\endnumbering\briefempfaengerindex{Schnitzler, Arthur@\textsc{Schnitzler, Arthur}!zzzGoldmann, Paul@\emph{von Paul Goldmann}!1897-11-192@{19. 11. [1897]}|)be}\mylabel{L02831h}  \newcommand{\dateiname}{L02831}\newcommand{\titel}{Paul Goldmann an Arthur Schnitzler, 19. 11. [1897]}\newcommand{\editorInnen}{Martin Anton Müller und Laura Untner}%% latex-leseansicht-abspann.tex
%% Abspann für die Leseansicht.
%% Der Schalter \ifkorrekturansicht ist bereits durch den Vorspann gesetzt.

%% latex-abspann.tex
%% Gemeinsamer Abspann für Korrekturansicht und Leseansicht.
%% Setzt den Schalter \ifkorrekturansicht voraus (gesetzt in den
%% einbindenden Dateien latex-korrekturansicht-abspann.tex bzw.
%% latex-leseansicht-abspann.tex).
%% ---------------------------------------------------------------

\normalsize

% Das esempio-Environment wird nur in der Leseansicht benötigt
\ifkorrekturansicht\else
\newenvironment{esempio}[3]%
{
    \vspace{1.5ex}
    \rlap{\underline{#1}}
    \par
    \setlength{\parindent}{0cm}
    \nopagebreak
    \leftskip=#2cm
    \rightskip=#3cm
}
{
    \par
}
\fi

\doendnotes{C}
\bigskip
\vfill

\clearpage

\footnotesize

\ifkorrekturansicht
  \lohead{\textsc{register}}
\fi

% theindex-Environment neu definieren ohne reledmac
\makeatletter
\renewenvironment{theindex}{%
  \ifkorrekturansicht
    \section*{\indexname}%
  \else
    \subsubsection*{Index der erwähnten Entitäten}%
  \fi
  \setlength{\parindent}{0pt}%
  \setlength{\parskip}{0pt plus 0.3pt}%
  \let\item\@idxitem
}{%
  \ifkorrekturansicht\clearpage\fi
}
\makeatother

\IfFileExists{\jobname-pw.ind}{\input{\jobname-pw.ind}}{}

% Quellenangabe nur in der Leseansicht
\ifkorrekturansicht\else
% Fallback-Definitionen, falls die .tex-Datei \titel etc. nicht gesetzt hat
\providecommand{\titel}{}
\providecommand{\editorInnen}{}
\providecommand{\dateiname}{\jobname}

\vspace{3cm}

\vfill

\footnotesize
\textsc{Quelle}: \titel. Herausgegeben von {\editorInnen}. In: \emph{Arthur Schnitzler: Briefwechsel mit Autorinnen und Autoren}.
 Digitale Edition, https://schnitzler-briefe.acdh.oeaw.ac.at/{\dateiname}.html (Stand \today)
\fi

\end{document}


