%% latex-korrekturansicht-vorspann.tex
%% Vorspann für die Korrekturansicht.
%% Lädt die gemeinsame Datei latex-vorspann.tex mit gesetztem Schalter.

\newif\ifkorrekturansicht
\korrekturansichttrue

\input{../tex-inputs/latex-vorspann}


\section[Arthur Schnitzler an Robert Adam, 28. 5. 1917]{L02261 Arthur Schnitzler an Robert Adam, 28. 5. 1917}
\nopagebreak\mylabel{L02261v}
\rehead{ }\normalsize\beginnumbering\briefempfaengerindex{Adam, Robert@\textsc{Adam, Robert}!zzzSchnitzler, Arthur@\emph{von Arthur Schnitzler}!1917-05-281@{28. 5. 1917}|(be}
\toendnotes[C]{\smallbreak\pagebreak[2]}\Standort{DLA, 96.34.2/2.}
\physDesc{Kartenbrief, 700 Zeichen
\newline{}Handschrift: schwarze Tinte, lateinische Kurrent
\newline{}Versand: Stempel: »\nobreak{}Wien, 29. V. 17, 7\nobreak{}«.  }\toendnotes[C]{\smallbreak}\pstart{}{\pb}Hrn Dr. Robert Adam Pollak\pend{}\pstart{}Wien XII\oindex{XII., Meidling@\textbf{XII., Meidling}, \emph{A.ADM3}|pw}\pend{}\pstart{}Meidlinger Hptstr 56\oindex{Meidlinger Hauptstrasse@\textbf{Meidlinger Hauptstraße}, \emph{Straße (K.STR)}|pw}.\pend{}{\bigskip}\vspace{1em}
\pstart
           \raggedleft{}{\pb}28. 5. 1917\pend
           
\pstart{}verehrter Herr Doktor,\pend\vspace{0.5em}
\pstart
           es thut mir sehr leid, daß Sie schon wieder eine theatralische Enttäuschung erleben
               mußten; – da gibts nun einmal nichts andres, als weiter arbeiten – vielleicht glückt
               es mit dem nächsten besser, und da{\geminationn} rücken die Vorgänger
               nach.\pend
           
\pstart
           Ich sehe Sie hoffentlich bald wieder, nicht wahr? Ende dieser Woche wollen wir auf
               circa 14 Tage nach Gastein\oindex{Bad Gastein@\textbf{Bad Gastein}, \emph{P.PPLA3}|pw} (wir waren schon in
                  Salzburg\oindex{Salzburg@\textbf{Salzburg}, \emph{A.ADM2}|pw} – auf dem Weg – und wurden durch die
               Nachricht vom \label{K_L02261-1v}\edtext{Tode}{\lemma{\textnormal{\emph{Tode}}}\Cendnote{\textnormal{Stefanie Bachrach\pwindex{Bachrach, Stefanie 22.05.1887 – 16.05.1917@\textsc{Bachrach, Stefanie} (22.05.1887 – 16.05.1917), \emph{Krankenpfleger/Krankenpflegerin}|pwk} hatte sich am 16. 5. 1917 das Leben
                  genommen.}}}\label{K_L02261-1} einer sehr lieben Freundin\pwindex{Bachrach, Stefanie 22.05.1887 – 16.05.1917@\textsc{Bachrach, Stefanie} (22.05.1887 – 16.05.1917), \emph{Krankenpfleger/Krankenpflegerin}|pwv} zurückgerufen) – Mitte Juni aber
               dürften wir wieder zu Hause sein. Ich schicke Ihnen den sehr amüsanten Dumas\pwindex{Meine Memoiren@\emph{Meine Memoiren}|pwv}\pwindex{Dumas, Alexandre pere 24.07.1802 – 05.12.1870@\textsc{Dumas, Alexandre père} (24.07.1802 – 05.12.1870), \emph{Schriftsteller/Schriftstellerin}|pw} mit vielem Dank zurück.\pend
           \pstart Herzlichſt grüßend Ihr \spacefill\mbox{Arthur Schnitzler}\pend{}\selectlanguage{ngerman}\endnumbering\briefempfaengerindex{Adam, Robert@\textsc{Adam, Robert}!zzzSchnitzler, Arthur@\emph{von Arthur Schnitzler}!1917-05-281@{28. 5. 1917}|)be}\mylabel{L02261h}  \normalsize

\doendnotes{C}
\bigskip
\vfill

\clearpage

\footnotesize

\lohead{\textsc{register}}

% Definiere theindex-Environment komplett neu ohne reledmac
\makeatletter
\renewenvironment{theindex}{%
  \section*{\indexname}%
  \setlength{\parindent}{0pt}%
  \setlength{\parskip}{0pt plus 0.3pt}%
  \let\item\@idxitem
}{%
  \clearpage
}
\makeatother

\IfFileExists{\jobname-pw.ind}{\input{\jobname-pw.ind}}{}

\end{document}

      