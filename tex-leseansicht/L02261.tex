%% latex-leseansicht-vorspann.tex
%% Vorspann für die Leseansicht.
%% Lädt die gemeinsame Datei latex-vorspann.tex mit nicht gesetztem Schalter.

\newif\ifkorrekturansicht
\korrekturansichtfalse

\input{../tex-inputs/latex-vorspann}


\section[Arthur Schnitzler an Robert Adam, 28. 5. 1917]{L02261 Arthur Schnitzler an Robert Adam, 28. 5. 1917}
\nopagebreak\mylabel{L02261v}
\rehead{ }\normalsize\beginnumbering\briefempfaengerindex{Adam, Robert@\textsc{Adam, Robert}!zzzSchnitzler, Arthur@\emph{von Arthur Schnitzler}!1917-05-281@{28. 5. 1917}|(be}
\toendnotes[C]{\smallbreak\pagebreak[2]}
\correspDesc{Versand  durch Arthur Schnitzler am 28. 5. 1917 in Wien
\newline{}Erhalt  durch Robert Adam im Zeitraum [28. 5. 1917
                  – 1. 6. 1917?] in Wien}\toendnotes[C]{\smallbreak}
\Standort{DLA, 96.34.2/2.}
\physDesc{Kartenbrief, 700 Zeichen
\newline{}Handschrift: schwarze Tinte, lateinische Kurrent
\newline{}Versand: Stempel: »\nobreak{}\oindex{Wien@\textbf{Wien}, \emph{Verwaltungsgebiet}|pwk}Wien, 29. V. 17, 7\nobreak{}«.  }\toendnotes[C]{\smallbreak}\pstart{}{\pb}Hrn Dr. Robert Adam Pollak\pend{}\pstart{}Wien XII\oindex{XII., Meidling@\textbf{XII., Meidling}, \emph{Verwaltungsgebiet}|pw}\pend{}\pstart{}Meidlinger Hptstr 56\oindex{Wien@\textbf{Wien}!XII., Meidling@\textbf{XII., Meidling}!Meidlinger Hauptstraße@\textbf{Meidlinger Hauptstraße}, \emph{Straße}|pw}.\pend{}{\bigskip}\vspace{1em}
\pstart
           \raggedleft{}{\pb}28. 5. 1917\pend
           
\pstart{}verehrter Herr Doktor,\pend\vspace{0.5em}
\pstart
           es thut mir sehr leid, daß Sie schon wieder eine theatralische Enttäuschung erleben
               mußten; – da gibts nun einmal nichts andres, als weiter arbeiten – vielleicht glückt
               es mit dem nächsten besser, und da{\geminationn} rücken die Vorgänger
               nach.\pend
           
\pstart
           Ich sehe Sie hoffentlich bald wieder, nicht wahr? Ende dieser Woche wollen wir auf
               circa 14 Tage nach Gastein\oindex{Bad Gastein@\textbf{Bad Gastein}, \emph{Hauptstadt}|pw} (wir waren schon in
                  Salzburg\oindex{Salzburg@\textbf{Salzburg}, \emph{Verwaltungsgebiet}|pw} – auf dem Weg – und wurden durch die
               Nachricht vom \label{K_L02261-1v}\edtext{Tode}{\lemma{\textnormal{\emph{Tode}}}\Cendnote{\textnormal{Stefanie Bachrach\pwindex{Bachrach, Stefanie 22.\,5.\,1887 Wien – 16.\,5.\,1917 ebd.@\textsc{Bachrach, Stefanie} (22.\,5.\,1887 Wien – 16.\,5.\,1917 ebd.), \emph{Krankenpflegerin}|pwk} hatte sich am 16. 5. 1917 das Leben
                  genommen.}}}\label{K_L02261-1} einer sehr lieben Freundin\pwindex{Bachrach, Stefanie 22.\,5.\,1887 Wien – 16.\,5.\,1917 ebd.@\textsc{Bachrach, Stefanie} (22.\,5.\,1887 Wien – 16.\,5.\,1917 ebd.), \emph{Krankenpflegerin}|pwv} zurückgerufen) – Mitte Juni aber
               dürften wir wieder zu Hause sein. Ich schicke Ihnen den sehr amüsanten Dumas\pwindex{Dumas, Alexandre père 24.\,7.\,1802 Villers-Cotterêts – 5.\,12.\,1870 Puys@\textsc{Dumas, Alexandre père} (24.\,7.\,1802 Villers-Cotterêts – 5.\,12.\,1870 Puys), \emph{Schriftsteller}!Meine Memoiren@\strich\emph{Meine Memoiren}|pwv}\pwindex{Dumas, Alexandre père 24.\,7.\,1802 Villers-Cotterêts – 5.\,12.\,1870 Puys@\textsc{Dumas, Alexandre père} (24.\,7.\,1802 Villers-Cotterêts – 5.\,12.\,1870 Puys), \emph{Schriftsteller}|pw} mit vielem Dank zurück.\pend
           \pstart Herzlichſt grüßend Ihr \spacefill\mbox{Arthur Schnitzler}\pend{}\selectlanguage{ngerman}\endnumbering\briefempfaengerindex{Adam, Robert@\textsc{Adam, Robert}!zzzSchnitzler, Arthur@\emph{von Arthur Schnitzler}!1917-05-281@{28. 5. 1917}|)be}\mylabel{L02261h}  \newcommand{\dateiname}{L02261}\newcommand{\titel}{Arthur Schnitzler an Robert Adam, 28. 5. 1917}\newcommand{\editorInnen}{Martin Anton Müller und Gerd-Hermann Susen}%% latex-leseansicht-abspann.tex
%% Abspann für die Leseansicht.
%% Der Schalter \ifkorrekturansicht ist bereits durch den Vorspann gesetzt.

%% latex-abspann.tex
%% Gemeinsamer Abspann für Korrekturansicht und Leseansicht.
%% Setzt den Schalter \ifkorrekturansicht voraus (gesetzt in den
%% einbindenden Dateien latex-korrekturansicht-abspann.tex bzw.
%% latex-leseansicht-abspann.tex).
%% ---------------------------------------------------------------

\normalsize

% Das esempio-Environment wird nur in der Leseansicht benötigt
\ifkorrekturansicht\else
\newenvironment{esempio}[3]%
{
    \vspace{1.5ex}
    \rlap{\underline{#1}}
    \par
    \setlength{\parindent}{0cm}
    \nopagebreak
    \leftskip=#2cm
    \rightskip=#3cm
}
{
    \par
}
\fi

\doendnotes{C}
\bigskip
\vfill

\clearpage

\footnotesize

\ifkorrekturansicht
  \lohead{\textsc{register}}
\fi

% theindex-Environment neu definieren ohne reledmac
\makeatletter
\renewenvironment{theindex}{%
  \ifkorrekturansicht
    \section*{\indexname}%
  \else
    \subsubsection*{Index der erwähnten Entitäten}%
  \fi
  \setlength{\parindent}{0pt}%
  \setlength{\parskip}{0pt plus 0.3pt}%
  \let\item\@idxitem
}{%
  \ifkorrekturansicht\clearpage\fi
}
\makeatother

\IfFileExists{\jobname-pw.ind}{\input{\jobname-pw.ind}}{}

% Quellenangabe nur in der Leseansicht
\ifkorrekturansicht\else
% Fallback-Definitionen, falls die .tex-Datei \titel etc. nicht gesetzt hat
\providecommand{\titel}{}
\providecommand{\editorInnen}{}
\providecommand{\dateiname}{\jobname}

\vspace{3cm}

\vfill

\footnotesize
\textsc{Quelle}: \titel. Herausgegeben von {\editorInnen}. In: \emph{Arthur Schnitzler: Briefwechsel mit Autorinnen und Autoren}.
 Digitale Edition, https://schnitzler-briefe.acdh.oeaw.ac.at/{\dateiname}.html (Stand \today)
\fi

\end{document}


