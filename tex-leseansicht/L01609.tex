%% latex-leseansicht-vorspann.tex
%% Vorspann für die Leseansicht.
%% Lädt die gemeinsame Datei latex-vorspann.tex mit nicht gesetztem Schalter.

\newif\ifkorrekturansicht
\korrekturansichtfalse

\input{../tex-inputs/latex-vorspann}


\section[Georg Brandes an Arthur Schnitzler, 11. 7. 1906]{L01609 Georg Brandes an Arthur Schnitzler, 11. 7. 1906}
\nopagebreak\mylabel{L01609v}
\rehead{ }\normalsize\beginnumbering\briefempfaengerindex{Schnitzler, Arthur@\textsc{Schnitzler, Arthur}!zzzBrandes, Georg@\emph{von Georg Brandes}!1906-07-112@{11. 7. 1906}|(be}
\toendnotes[C]{\smallbreak\pagebreak[2]}
\correspDesc{Versand  durch Georg Brandes am 11. 7. 1906 in Kopenhagen
\newline{}Erhalt  durch Arthur Schnitzler im Zeitraum [11. 7. 1906
                  – 15. 7. 1906?] in Marienlyst}\toendnotes[C]{\smallbreak}
\Standort{CUL, Schnitzler, B 17.}
\physDesc{Brief, 1 Blatt, 3 Seiten, 828 Zeichen
\newline{}Handschrift: schwarze Tinte, lateinische Kurrent
\newline{}Ordnung: mit Bleistift von unbekannter Hand nummeriert:
                                    »31« }
\buchAbdrucke{\weitereDrucke{Georg Brandes, Arthur Schnitzler: \emph{Ein Briefwechsel}. Herausgegeben von Kurt Bergel. Bern: \emph{Francke} 1956, S. 93.} }\toendnotes[C]{\smallbreak}
\pstart
           \raggedleft{}{\pb}\strikeout{2}{ }11 Juli 06\pend
           
\pstart{}Verehrter Freund\pend\vspace{0.5em}
\pstart
           Wenn ich Ihre Karte einigermassen richtig dechiffrire – die Schrift ist räthselhaft –
               so fragen Sie nach meinem Befinden und sagen mir dass {\dots}
               Jemand mich grüssen lässt.\pend
           
\pstart
           Ich bin heute aus dem Spital heraus, nur noch sehr, sehr matt, stolpere aber umher,
               um mich zum Gehen wieder zu gewöhnen.\pend
           
\pstart
           Ich wurde sehr gerührt, dass {\pb}Sie
               meiner gedacht hatten. Hoffmannsthal\pwindex{Hofmannsthal, Hugo von 1.\,2.\,1874 Wien – 15.\,7.\,1929 Rodaun@\textsc{Hofmannsthal, Hugo von} (1.\,2.\,1874 Wien – 15.\,7.\,1929 Rodaun), \emph{Schriftsteller}|pw} schickte
               mir Thor und Tod\pwindex{Hofmannsthal, Hugo von 1.\,2.\,1874 Wien – 15.\,7.\,1929 Rodaun@\textsc{Hofmannsthal, Hugo von} (1.\,2.\,1874 Wien – 15.\,7.\,1929 Rodaun), \emph{Schriftsteller}!Thor und der Tod@\strich\emph{Der Thor und der Tod}|pw}. Es ist schön und fein, machte
               mir aber lange nicht den Eindruck wie die zwei antikisirenden Schauspiele\pwindex{Hofmannsthal, Hugo von 1.\,2.\,1874 Wien – 15.\,7.\,1929 Rodaun@\textsc{Hofmannsthal, Hugo von} (1.\,2.\,1874 Wien – 15.\,7.\,1929 Rodaun), \emph{Schriftsteller}!Elektra. Tragödie in einem Aufzug@\strich\emph{Elektra. Tragödie in einem Aufzug}|pwv}\pwindex{Hofmannsthal, Hugo von 1.\,2.\,1874 Wien – 15.\,7.\,1929 Rodaun@\textsc{Hofmannsthal, Hugo von} (1.\,2.\,1874 Wien – 15.\,7.\,1929 Rodaun), \emph{Schriftsteller}!Oedipus und die Sphinx. Tragödie in drei Aufzügen@\strich\emph{Oedipus und die Sphinx. Tragödie in drei Aufzügen}|pwv}.\pend
           
\pstart
           Ueber Ihre eigenen Arbeiten kam ich das letzte \label{K_L01609-1v}\edtext{Mal}{\lemma{\textnormal{\emph{Mal}}}\Cendnote{\textnormal{Schnitzler hatte Brandes\pwindex{Brandes, Georg 4.\,2.\,1842 Kopenhagen – 19.\,2.\,1927 ebd.@\textsc{Brandes, Georg} (4.\,2.\,1842 Kopenhagen – 19.\,2.\,1927 ebd.)|pwk} am 2. 7. 1906 im Kommunehospitalet\oindex{Kommunehospitalet@\textbf{Kommunehospitalet}, \emph{Krankenhaus}|pwk} besucht.}}}\label{K_L01609-1} gar nicht dazu, mit Ihnen
               zu reden, wollte es doch sehr gern.\pend
           
\pstart
           Ich komme wohl eines Tages nach Helsingør\oindex{Helsingør@\textbf{Helsingør}, \emph{Hauptstadt}|pw} und
               versuche an Ihre Thür zu klopfen. Aber etwas kräftiger muss {\pb}ich erst sein.\pend
           
\pstart
           Vorläufig soll ich arme Sau Empfangsrede an das \label{K_L01609-2v}\edtext{Allthing}{\lemma{\textnormal{\emph{Allthing}}}\Cendnote{\textnormal{Am
                     17. 8. 1906 hielt Brandes\pwindex{Brandes, Georg 4.\,2.\,1842 Kopenhagen – 19.\,2.\,1927 ebd.@\textsc{Brandes, Georg} (4.\,2.\,1842 Kopenhagen – 19.\,2.\,1927 ebd.)|pwk}
                  eine Festrede für die Mitglieder des \emph{Althing}\orgindex{Althing@Althing|pwk},
                  für die isländischen\oindex{Island@\textbf{Island}|pwk} Volksvertreter, die
                  sich in Kopenhagen\oindex{Kopenhagen@\textbf{Kopenhagen}, \emph{Hauptstadt}|pwk} aufhielten.}}}\label{K_L01609-2}
               halten.\pend
           
\pstart
           Ihr ergebener{\\[\baselineskip]}\spacefill\mbox{Georg Brandes}\pend
           \leftskip=0em{}\selectlanguage{ngerman}\endnumbering\briefempfaengerindex{Schnitzler, Arthur@\textsc{Schnitzler, Arthur}!zzzBrandes, Georg@\emph{von Georg Brandes}!1906-07-112@{11. 7. 1906}|)be}\mylabel{L01609h}  \newcommand{\dateiname}{L01609}\newcommand{\titel}{Georg Brandes an Arthur Schnitzler, 11. 7. 1906}\newcommand{\editorInnen}{Martin Anton Müller und Gerd-Hermann Susen}%% latex-leseansicht-abspann.tex
%% Abspann für die Leseansicht.
%% Der Schalter \ifkorrekturansicht ist bereits durch den Vorspann gesetzt.

%% latex-abspann.tex
%% Gemeinsamer Abspann für Korrekturansicht und Leseansicht.
%% Setzt den Schalter \ifkorrekturansicht voraus (gesetzt in den
%% einbindenden Dateien latex-korrekturansicht-abspann.tex bzw.
%% latex-leseansicht-abspann.tex).
%% ---------------------------------------------------------------

\normalsize

% Das esempio-Environment wird nur in der Leseansicht benötigt
\ifkorrekturansicht\else
\newenvironment{esempio}[3]%
{
    \vspace{1.5ex}
    \rlap{\underline{#1}}
    \par
    \setlength{\parindent}{0cm}
    \nopagebreak
    \leftskip=#2cm
    \rightskip=#3cm
}
{
    \par
}
\fi

\doendnotes{C}
\bigskip
\vfill

\clearpage

\footnotesize

\ifkorrekturansicht
  \lohead{\textsc{register}}
\fi

% theindex-Environment neu definieren ohne reledmac
\makeatletter
\renewenvironment{theindex}{%
  \ifkorrekturansicht
    \section*{\indexname}%
  \else
    \subsubsection*{Index der erwähnten Entitäten}%
  \fi
  \setlength{\parindent}{0pt}%
  \setlength{\parskip}{0pt plus 0.3pt}%
  \let\item\@idxitem
}{%
  \ifkorrekturansicht\clearpage\fi
}
\makeatother

\IfFileExists{\jobname-pw.ind}{\input{\jobname-pw.ind}}{}

% Quellenangabe nur in der Leseansicht
\ifkorrekturansicht\else
% Fallback-Definitionen, falls die .tex-Datei \titel etc. nicht gesetzt hat
\providecommand{\titel}{}
\providecommand{\editorInnen}{}
\providecommand{\dateiname}{\jobname}

\vspace{3cm}

\vfill

\footnotesize
\textsc{Quelle}: \titel. Herausgegeben von {\editorInnen}. In: \emph{Arthur Schnitzler: Briefwechsel mit Autorinnen und Autoren}.
 Digitale Edition, https://schnitzler-briefe.acdh.oeaw.ac.at/{\dateiname}.html (Stand \today)
\fi

\end{document}


