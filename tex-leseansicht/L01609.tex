%% latex-korrekturansicht-vorspann.tex
%% Vorspann für die Korrekturansicht.
%% Lädt die gemeinsame Datei latex-vorspann.tex mit gesetztem Schalter.

\newif\ifkorrekturansicht
\korrekturansichttrue

\input{../tex-inputs/latex-vorspann}


\section[Georg Brandes an Arthur Schnitzler, 11. 7. 1906]{L01609 Georg Brandes an Arthur Schnitzler, 11. 7. 1906}
\nopagebreak\mylabel{L01609v}
\rehead{ }\normalsize\beginnumbering\briefempfaengerindex{Schnitzler, Arthur@\textsc{Schnitzler, Arthur}!zzzBrandes, Georg@\emph{von Georg Brandes}!1906-07-112@{11. 7. 1906}|(be}
\toendnotes[C]{\smallbreak\pagebreak[2]}\Standort{CUL, Schnitzler, B 17.}
\physDesc{Brief, 1 Blatt, 3 Seiten, 828 Zeichen
\newline{}Handschrift: schwarze Tinte, lateinische Kurrent
\newline{}Ordnung: mit Bleistift von unbekannter Hand nummeriert:
                                    »31« }
\buchAbdrucke{\weitereDrucke{Georg Brandes, Arthur Schnitzler: \emph{Ein Briefwechsel}. Bern: \emph{Francke} 1956, S. 93.} }\toendnotes[C]{\smallbreak}
\pstart
           \raggedleft{}{\pb}\strikeout{2}{ }11 Juli 06\pend
           
\pstart{}Verehrter Freund\pend\vspace{0.5em}
\pstart
           Wenn ich Ihre Karte einigermassen richtig dechiffrire – die Schrift ist räthselhaft –
               so fragen Sie nach meinem Befinden und sagen mir dass {\dots}
               Jemand mich grüssen lässt.\pend
           
\pstart
           Ich bin heute aus dem Spital heraus, nur noch sehr, sehr matt, stolpere aber umher,
               um mich zum Gehen wieder zu gewöhnen.\pend
           
\pstart
           Ich wurde sehr gerührt, dass {\pb}Sie
               meiner gedacht hatten. Hoffmannsthal\pwindex{Hofmannsthal, Hugo von 1874-02-01 – 1929-07-15@\textsc{Hofmannsthal, Hugo von} (1874-02-01 – 1929-07-15), \emph{Schriftsteller/Schriftstellerin}|pw} schickte
               mir Thor und Tod\pwindex{Thor und der Tod@\emph{Der Thor und der Tod}|pw}. Es ist schön und fein, machte
               mir aber lange nicht den Eindruck wie die zwei antikisirenden Schauspiele\pwindex{Elektra. Tragoedie in einem Aufzug@\emph{Elektra. Tragödie in einem Aufzug}|pwv}\pwindex{Oedipus und die Sphinx. Tragoedie in drei Aufzuegen@\emph{Oedipus und die Sphinx. Tragödie in drei Aufzügen}|pwv}.\pend
           
\pstart
           Ueber Ihre eigenen Arbeiten kam ich das letzte \label{K_L01609-1v}\edtext{Mal}{\lemma{\textnormal{\emph{Mal}}}\Cendnote{\textnormal{Schnitzler hatte Brandes\pwindex{Brandes, Georg 04.02.1842 – 19.02.1927@\textsc{Brandes, Georg} (04.02.1842 – 19.02.1927)|pwk} am 2. 7. 1906 im Kommunehospitalet\oindex{Kommunehospitalet@\textbf{Kommunehospitalet}, \emph{Krankenhaus (K.KKH)}|pwk} besucht.}}}\label{K_L01609-1} gar nicht dazu, mit Ihnen
               zu reden, wollte es doch sehr gern.\pend
           
\pstart
           Ich komme wohl eines Tages nach Helsingør\oindex{Helsingør@\textbf{Helsingør}, \emph{P.PPLA2}|pw} und
               versuche an Ihre Thür zu klopfen. Aber etwas kräftiger muss {\pb}ich erst sein.\pend
           
\pstart
           Vorläufig soll ich arme Sau Empfangsrede an das \label{K_L01609-2v}\edtext{Allthing}{\lemma{\textnormal{\emph{Allthing}}}\Cendnote{\textnormal{Am
                     17. 8. 1906 hielt Brandes\pwindex{Brandes, Georg 04.02.1842 – 19.02.1927@\textsc{Brandes, Georg} (04.02.1842 – 19.02.1927)|pwk}
                  eine Festrede für die Mitglieder des \emph{Althing}\orgindex{Althing@Althing|pwk},
                  für die isländischen\oindex{Island@\textbf{Island}, \emph{A.PCLI}|pwk} Volksvertreter, die
                  sich in Kopenhagen\oindex{Kopenhagen@\textbf{Kopenhagen}, \emph{P.PPLC}|pwk} aufhielten.}}}\label{K_L01609-2}
               halten.\pend
           
\pstart
           Ihr ergebener{\\[\baselineskip]}\spacefill\mbox{Georg Brandes}\pend
           \leftskip=0em{}\selectlanguage{ngerman}\endnumbering\briefempfaengerindex{Schnitzler, Arthur@\textsc{Schnitzler, Arthur}!zzzBrandes, Georg@\emph{von Georg Brandes}!1906-07-112@{11. 7. 1906}|)be}\mylabel{L01609h}  \normalsize

\doendnotes{C}
\bigskip
\vfill

\clearpage

\footnotesize

\lohead{\textsc{register}}

% Definiere theindex-Environment komplett neu ohne reledmac
\makeatletter
\renewenvironment{theindex}{%
  \section*{\indexname}%
  \setlength{\parindent}{0pt}%
  \setlength{\parskip}{0pt plus 0.3pt}%
  \let\item\@idxitem
}{%
  \clearpage
}
\makeatother

\IfFileExists{\jobname-pw.ind}{\input{\jobname-pw.ind}}{}

\end{document}

      