%% latex-korrekturansicht-vorspann.tex
%% Vorspann für die Korrekturansicht.
%% Lädt die gemeinsame Datei latex-vorspann.tex mit gesetztem Schalter.

\newif\ifkorrekturansicht
\korrekturansichttrue

\input{../tex-inputs/latex-vorspann}


\section[Arthur Schnitzler an Hugo von Hofmannsthal, 17. 8. 1895]{L00474 Arthur Schnitzler an Hugo von Hofmannsthal, 17. 8. 1895}
\nopagebreak\mylabel{L00474v}
\rehead{ }\normalsize\beginnumbering\briefempfaengerindex{Hofmannsthal, Hugo von@\textsc{Hofmannsthal, Hugo von}!zzzSchnitzler, Arthur@\emph{von Arthur Schnitzler}!1895-08-171@{17. 8. 1895}|(be}
\toendnotes[C]{\smallbreak\pagebreak[2]}\Standort{FDH, Hs-30885,45.}
\physDesc{Brief, 1 Blatt, 4 Seiten, 2002 Zeichen
\newline{}Handschrift: schwarze Tinte, deutsche Kurrent}
\buchAbdrucke{\weitereDrucke{Hugo von Hofmannsthal, Arthur Schnitzler: \emph{Briefwechsel}. Frankfurt am Main: \emph{S. Fischer} 1964, S. 59–60.} }\toendnotes[C]{\smallbreak}
\pstart
           \raggedleft{}{\pb}\textsc{Ischl}\oindex{Bad Ischl@\textbf{Bad Ischl}, \emph{P.PPL}|pw}, \uline{17. 8. 95.}\pend
           \vspace{0.5em}
\pstart
           Mein Lieber Hugo, Ihren Brief habe ich beim Zurückko{\geminationm}en aus Wien\oindex{Wien@\textbf{Wien}, \emph{A.ADM2}|pw}
               gefunden. Dort bin ich 2 Tage geweſen und habe die Marionetten in \textsc{Venedig}\oindex{Venedig in Wien@\textbf{Venedig in Wien}, \emph{Vergnügungspark (K.VGN)}|pw} u \textsc{Hänsel u Grethel}\pwindex{Haensel und Gretel. Maerchenspiel in drei Bildern@\emph{Hänsel und Gretel. Märchenspiel in drei Bildern}|pw} geſehen. An einzelne von dieſen Marionetten denke ich zurück wie an lebendige
               Schauſpieler, die ſich auch an mich erinnern müſſen. Im übrigen iſt Wien\oindex{Wien@\textbf{Wien}, \emph{A.ADM2}|pw} jetzt dumpf und übelriechend und es iſt gut, daſs ich
               wieder weg konnte. In Iſchl\oindex{Bad Ischl@\textbf{Bad Ischl}, \emph{P.PPL}|pw} bleib ich nur noch
               bis Montag. Dann fahr ich per Rad nach Salzburg\oindex{Salzburg@\textbf{Salzburg}, \emph{A.ADM2}|pw},
               mit Salten\pwindex{Salten, Felix 06.09.1869 – 08.10.1945@\textsc{Salten, Felix} (06.09.1869 – 08.10.1945), \emph{Schriftsteller/Schriftstellerin, Journalist/Journalistin, Chefredakteur/Chefredakteurin}|pw}. {\pb}Auch
                  Richard\pwindex{Beer-Hofmann, Richard 1866-07-11 – 1945-09-26@\textsc{Beer-Hofmann, Richard} (1866-07-11 – 1945-09-26), \emph{Schriftsteller/Schriftstellerin}|pw}, dem ich Ihre Kränkung beſtellt
               habe, ko{\geminationm}t wohl hin, und die Frau Lou\pwindex{Andreas-Salome, Lou 12.02.1861 – 05.02.1937@\textsc{Andreas-Salomé, Lou} (12.02.1861 – 05.02.1937), \emph{Schriftsteller/Schriftstellerin}|pw} wird ſchon dort ſein. Wenn Sie mir gleich zwei Zeilen
               ſchreiben, ſo kann ich ſie mir noch in Salzburg\oindex{Salzburg@\textbf{Salzburg}, \emph{A.ADM2}|pw}{ }\textsc{post restante} abholen u hätte eine große Freude.
                  Donnerſtag radle ich nämlich weiter, auf einem bisher noch nicht
               definitiv feſtgeſtellten Weg nach \textsc{München}\oindex{Muenchen@\textbf{München}, \emph{P.PPLA}|pw}, wo das Rendezvous mit Goldma{\geminationn}\pwindex{Goldmann, Paul 31.01.1865 – 25.09.1935@\textsc{Goldmann, Paul} (31.01.1865 – 25.09.1935), \emph{Schriftsteller/Schriftstellerin, Journalist/Journalistin}|pw} iſt. In M.\oindex{Muenchen@\textbf{München}, \emph{P.PPLA}|pw} bin ich mindeſtens bis
                  3. September (Briefe dahin auch \textsc{post
                  restante}. Aber ich {\pb}werd Ihnen von meiner Radtour
               noch öfters ein paar Worte ſchreiben)\pend
           
\pstart
           – Ich hab hier den erſten Akt\pwindex{Freiwild. Schauspiel in 3 Akten@\emph{Freiwild. Schauspiel in 3 Akten}|pwv}
               zu Ende geſchrieben, und ein paar kleine Geſchichten\pwindex{Frau des Weisen. Erzaehlung@\emph{Die Frau des Weisen. Erzählung}|pwv}\pwindex{Abschied@\emph{Ein Abschied}|pwv}, an denen mir vielleicht ſchon manches gelungen
               iſt. Sie wiſſen ja, meine große Sehnſucht: die ſehr einfache Geſchichte, die in ſich
               ſelbſt ganz fertig iſt. Eine Flaſche, die man ausgießt, ohne daſs es nachtröpfeln
               darf und ohne daſs was zurückbleibt. – Auch geht es mir heuer innerlich gut – es
               gelingt mir faſt jedesmal kleine Eitelkeiten und große {\pb}Hypochondrien davon zujagen, wenn ſie ſich melden wollen. Im ganzen fühl ich mich
               in dieſem Jahre um fünf Jahre jünger als im vorigen, was darin begründet iſt, daſs
               ich in weniger falſchen Verhältniſſen lebe als damals. Was Sie einmal von der Seele,
               die i{\geminationm}er eine kindliche bleibt, ſagten, fällt mir ein.
               Es mag ſein, daſs Altwerden wirklich nur eine Schwäche iſt, von der man ſich befreien
                  kann{\dotsfour}{ }ſolang man eben doch eigentlich nur 33 Jahre alt
               iſt.\pend
           
\pstart
           Leben Sie wohl, ſeien Sie herzlich gegrüßt. Und ſchreiben Sie eine Zeile nach Salzb.\oindex{Salzburg@\textbf{Salzburg}, \emph{A.ADM2}|pw}\pend
           \pstart Ihr \spacefill\mbox{Arthur}\pend{}
\pstart
           \noindent{}\label{T_L00474-1v}\edtext{Ich habe an Goldm.\pwindex{Goldmann, Paul 31.01.1865 – 25.09.1935@\textsc{Goldmann, Paul} (31.01.1865 – 25.09.1935), \emph{Schriftsteller/Schriftstellerin, Journalist/Journalistin}|pw} wegen Mamroth\pwindex{Mamroth, Fedor 21.02.1851 – 25.06.1907@\textsc{Mamroth, Fedor} (21.02.1851 – 25.06.1907), \emph{Journalist/Journalistin, Kritiker/Kritikerin}|pw}
                     geſchrieben.}{\lemma{\textnormal{\emph{Ich … geſchrieben.}}}\Cendnote{\textnormal{Das Postscript befindet
                     sich neben der Ortsangabe auf der ersten Seite auf dem Kopf.}}}\label{T_L00474-1}\pend
           \selectlanguage{ngerman}\endnumbering\briefempfaengerindex{Hofmannsthal, Hugo von@\textsc{Hofmannsthal, Hugo von}!zzzSchnitzler, Arthur@\emph{von Arthur Schnitzler}!1895-08-171@{17. 8. 1895}|)be}\mylabel{L00474h}  \normalsize

\doendnotes{C}
\bigskip
\vfill

\clearpage

\footnotesize

\lohead{\textsc{register}}

% Definiere theindex-Environment komplett neu ohne reledmac
\makeatletter
\renewenvironment{theindex}{%
  \section*{\indexname}%
  \setlength{\parindent}{0pt}%
  \setlength{\parskip}{0pt plus 0.3pt}%
  \let\item\@idxitem
}{%
  \clearpage
}
\makeatother

\IfFileExists{\jobname-pw.ind}{\input{\jobname-pw.ind}}{}

\end{document}

      