%% latex-leseansicht-vorspann.tex
%% Vorspann für die Leseansicht.
%% Lädt die gemeinsame Datei latex-vorspann.tex mit nicht gesetztem Schalter.

\newif\ifkorrekturansicht
\korrekturansichtfalse

\input{../tex-inputs/latex-vorspann}

\begin{center}
            \textcolor{red}{ENTWURF. ENTZIFFERUNG NOCH NICHT KORREKTURGELESEN}
                      \end{center}
            
               \section[Arthur Schnitzler an Hugo von Hofmannsthal, 17. 8. 1895]{ Arthur Schnitzler an Hugo von Hofmannsthal, 17. 8. 1895}\nopagebreak\mylabel{v}\rehead{ }\begin{ledgroupsized}[t]{13cm}\normalsize\beginnumbering\briefempfaengerindex{Hofmannsthal, Hugo von@\textsc{Hofmannsthal, Hugo von}!zzzSchnitzler, Arthur@\emph{von Arthur Schnitzler}!1895-08-171@{17. 8. 1895}|(be} \toendnotes[C]{\smallbreak\pagebreak[2]} \Standort{FDH, Hs-30885,45.}
\physDesc{Brief, 1 Blatt, 4 Seiten
\newline{}Handschrift: schwarze Tinte, deutsche Kurrent}\buchAbdrucke{\weitereDrucke{Hugo von Hofmannsthal, Arthur Schnitzler: \emph{Briefwechsel}. Hg. Therese Nickl und Heinrich Schnitzler. Frankfurt am Main: \emph{S. Fischer} 1964, S. 59–60.} }\toendnotes[C]{\smallbreak}\pstart
           \raggedleft{}{\pb}\textsc{Ischl}\oindex{Bad Ischl@\textbf{Bad Ischl}|pw}, \uline{17. 8. 95.}\pend
           \pstart
           Mein Lieber Hugo, Ihren Brief habe ich beim Zurückko{\geminationm}en aus Wien\oindex{Wien@\textbf{Wien}|pw}
                    gefunden. Dort bin ich 2 Tage geweſen und habe die Marionetten in \textsc{Venedig}\oindex{Venedig in Wien@\textbf{Venedig in Wien}|pw} u \textsc{Hänsel u Grethel}\pwindex{\textcolor{red}{\textsuperscript{XXXX1 indx}}!Haensel und Grethel1893@\strich\emph{Hänsel und Grethel} {[}1893{]}|pw} geſehen. An einzelne von dieſen Marionetten denke ich zurück wie an
                    lebendige Schauſpieler, die ſich auch an mich erinnern müſſen. Im übrigen iſt
                        Wien\oindex{Wien@\textbf{Wien}|pw} jetzt dumpf und übelriechend und es
                    iſt gut, daſs ich wieder weg konnte. In Iſchl\oindex{Bad Ischl@\textbf{Bad Ischl}|pw} bleib ich nur noch bis Montag. Dann fahr ich per Rad nach
                        Salzburg\oindex{Salzburg@\textbf{Salzburg}|pw}, mit Salten\pwindex{Salten, Felix 06.09.1869 – 08.10.1945@\textsc{Salten, Felix} (06.09.1869 – 08.10.1945), \emph{Schriftsteller, Journalist}|pw}. {\pb}Auch Richard\pwindex{Beer-Hofmann, Richard 11.07.1866 – 26.09.1945@\textsc{Beer-Hofmann, Richard} (11.07.1866 – 26.09.1945), \emph{Schriftsteller}|pw}, dem ich Ihre Kränkung beſtellt habe, ko{\geminationm}t wohl hin, und die Frau Lou\pwindex{Andreas-Salome, Lou 12.02.1861 – 05.02.1937@\textsc{Andreas-Salomé, Lou} (12.02.1861 – 05.02.1937), \emph{Schriftstellerin}|pw} wird ſchon dort ſein. Wenn Sie mir gleich zwei Zeilen
                    ſchreiben, ſo kann ich ſie mir noch in Salzburg\oindex{Salzburg@\textbf{Salzburg}|pw}{ }\textsc{post restante} abholen u hätte eine große Freude.
                        Donnerſtag radle ich nämlich weiter, auf einem bisher noch
                    nicht definitiv feſtgeſtellten Weg nach \textsc{München}\oindex{Muenchen@\textbf{München}|pw}, wo das Rendezvous mit
                        Goldma{\geminationn}\pwindex{Goldmann, Paul 31.01.1865 – 25.09.1935@\textsc{Goldmann, Paul} (31.01.1865 – 25.09.1935), \emph{Schriftsteller, Journalist}|pw} iſt. In M.\oindex{Muenchen@\textbf{München}|pw} bin ich mindeſtens
                    bis 3. September (Briefe dahin auch \textsc{post
                        restante}. Aber ich {\pb}werd Ihnen von
                    meiner Radtour noch öfters ein paar Worte ſchreiben)\pend
           \pstart
           – Ich hab hier den erſten Akt\pwindex{Schnitzler, Arthur 15.05.1862 – 21.10.1931@\textsc{Schnitzler, Arthur} (15.05.1862 – 21.10.1931), \emph{Schriftsteller, Mediziner}!Freiwild. Schauspiel in 3 Akten1896@\strich\emph{Freiwild. Schauspiel in 3 Akten} {[}1896{]}|pwv} zu Ende
                    geſchrieben, und ein paar kleine Geſchichten\pwindex{Schnitzler, Arthur 15.05.1862 – 21.10.1931@\textsc{Schnitzler, Arthur} (15.05.1862 – 21.10.1931), \emph{Schriftsteller, Mediziner}!Frau des Weisen. Erzaehlung1897-01-02 – 1897-01-16@\strich\emph{Die Frau des Weisen. Erzählung} {[}1897-01-02 – 1897-01-16{]}|pwv}\pwindex{Schnitzler, Arthur 15.05.1862 – 21.10.1931@\textsc{Schnitzler, Arthur} (15.05.1862 – 21.10.1931), \emph{Schriftsteller, Mediziner}!Abschied1896@\strich\emph{Ein Abschied} {[}1896{]}|pwv}, an denen mir vielleicht ſchon manches gelungen iſt. Sie
                    wiſſen ja, meine große Sehnſucht: die ſehr einfache Geſchichte, die in ſich
                    ſelbſt ganz fertig iſt. Eine Flaſche, die man ausgießt, ohne daſs es
                    nachtröpfeln darf und ohne daſs was zurückbleibt. – Auch geht es mir heuer
                    innerlich gut – es gelingt mir faſt jedesmal kleine Eitelkeiten und große {\pb}Hypochondrien davon zujagen, wenn ſie ſich
                    melden wollen. Im ganzen fühl ich mich in dieſem Jahre um fünf Jahre jünger als
                    im vorigen, was darin begründet iſt, daſs ich in weniger falſchen Verhältniſſen
                    lebe als damals. Was Sie einmal von der Seele, die i{\geminationm}er eine kindliche bleibt, ſagten, fällt mir ein. Es mag ſein, daſs Altwerden
                    wirklich nur eine Schwäche iſt, von der man ſich befreien kann{\dotsfour}{ }ſolang man eben doch eigentlich nur 33 Jahre alt
                    iſt.\pend
           \pstart
           Leben Sie wohl, ſeien Sie herzlich gegrüßt. Und ſchreiben Sie eine Zeile nach
                        Salzb.\oindex{Salzburg@\textbf{Salzburg}|pw}\pend
           \pstart Ihr \spacefill\mbox{Arthur}\pend{}\pstart
           \noindent{}\label{T_L00474_1v}\edtext{Ich habe an Goldm.\pwindex{Goldmann, Paul 31.01.1865 – 25.09.1935@\textsc{Goldmann, Paul} (31.01.1865 – 25.09.1935), \emph{Schriftsteller, Journalist}|pw} wegen Mamroth\pwindex{Mamroth, Fedor 21.02.1851 – 25.06.1907@\textsc{Mamroth, Fedor} (21.02.1851 – 25.06.1907), \emph{Journalist, Kritiker}|pw} geſchrieben.}{\lemma{\textnormal{\emph{Ich … geſchrieben.}}}\Cendnote{\textnormal{Das Postscript befindet sich neben der Ortsangabe auf
                            der ersten Seite auf dem Kopf.}}}\label{T_L00474_1h}\pend
           \endnumbering\briefempfaengerindex{Hofmannsthal, Hugo von@\textsc{Hofmannsthal, Hugo von}!zzzSchnitzler, Arthur@\emph{von Arthur Schnitzler}!1895-08-171@{17. 8. 1895}|)be}\mylabel{h}\end{ledgroupsized}  \newcommand{\dateiname}{L00474}\newcommand{\titel}{Arthur Schnitzler an Hugo von Hofmannsthal, 17. 8. 1895}\newcommand{\editorInnen}{Martin Anton Müller und Gerd-Hermann Susen}%% latex-leseansicht-abspann.tex
%% Abspann für die Leseansicht.
%% Der Schalter \ifkorrekturansicht ist bereits durch den Vorspann gesetzt.

%% latex-abspann.tex
%% Gemeinsamer Abspann für Korrekturansicht und Leseansicht.
%% Setzt den Schalter \ifkorrekturansicht voraus (gesetzt in den
%% einbindenden Dateien latex-korrekturansicht-abspann.tex bzw.
%% latex-leseansicht-abspann.tex).
%% ---------------------------------------------------------------

\normalsize

% Das esempio-Environment wird nur in der Leseansicht benötigt
\ifkorrekturansicht\else
\newenvironment{esempio}[3]%
{
    \vspace{1.5ex}
    \rlap{\underline{#1}}
    \par
    \setlength{\parindent}{0cm}
    \nopagebreak
    \leftskip=#2cm
    \rightskip=#3cm
}
{
    \par
}
\fi

\doendnotes{C}
\bigskip
\vfill

\clearpage

\footnotesize

\ifkorrekturansicht
  \lohead{\textsc{register}}
\fi

% theindex-Environment neu definieren ohne reledmac
\makeatletter
\renewenvironment{theindex}{%
  \ifkorrekturansicht
    \section*{\indexname}%
  \else
    \subsubsection*{Index der erwähnten Entitäten}%
  \fi
  \setlength{\parindent}{0pt}%
  \setlength{\parskip}{0pt plus 0.3pt}%
  \let\item\@idxitem
}{%
  \ifkorrekturansicht\clearpage\fi
}
\makeatother

\IfFileExists{\jobname-pw.ind}{\input{\jobname-pw.ind}}{}

% Quellenangabe nur in der Leseansicht
\ifkorrekturansicht\else
% Fallback-Definitionen, falls die .tex-Datei \titel etc. nicht gesetzt hat
\providecommand{\titel}{}
\providecommand{\editorInnen}{}
\providecommand{\dateiname}{\jobname}

\vspace{3cm}

\vfill

\footnotesize
\textsc{Quelle}: \titel. Herausgegeben von {\editorInnen}. In: \emph{Arthur Schnitzler: Briefwechsel mit Autorinnen und Autoren}.
 Digitale Edition, https://schnitzler-briefe.acdh.oeaw.ac.at/{\dateiname}.html (Stand \today)
\fi

\end{document}


      