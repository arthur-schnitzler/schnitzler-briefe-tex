%% latex-leseansicht-vorspann.tex
%% Vorspann für die Leseansicht.
%% Lädt die gemeinsame Datei latex-vorspann.tex mit nicht gesetztem Schalter.

\newif\ifkorrekturansicht
\korrekturansichtfalse

\input{../tex-inputs/latex-vorspann}


               \section[Arthur Schnitzler an Richard Beer-Hofmann, 22. 8. 1914]{ Arthur Schnitzler an Richard Beer-Hofmann, 22. 8. 1914}\nopagebreak\mylabel{v}\rehead{ }\begin{ledgroupsized}[t]{13cm}\normalsize\beginnumbering\briefempfaengerindex{Beer-Hofmann, Richard@\textsc{Beer-Hofmann, Richard}!zzzSchnitzler, Arthur@\emph{von Arthur Schnitzler}!1914-08-221@{22. 8. 1914}|(be} \toendnotes[C]{\smallbreak\pagebreak[2]} \Standort{YCGL, MSS 31.}
\physDesc{Kartenbrief
\newline{}Handschrift: Bleistift, deutsche Kurrent\newline{}Versand: Stempel: »\nobreak{}\oindex{Bad Ischl@\textbf{Bad Ischl}|pwk}{[}Bad{]} Ischl, 2\textcolor{gray}{2}. VIII. {[}1914{]}\nobreak{}«.  
\newline{}Beer-Hofmann: mit blauem Buntstift den Erhalt und die Beantwortung markiert: »\noindent{}E.B{ / }24/VIII 14{ }\textsc{Telegr.}« }\buchAbdrucke{\weitereDrucke{Arthur Schnitzler, Richard Beer-Hofmann: \emph{Briefwechsel 1891–1931}. Hg. Konstanze Fliedl. Wien, Zürich: \emph{Europaverlag} 1992, S. 220.} }\toendnotes[C]{\smallbreak}\pstart{}{\pb}Abſ. \textsc{Schnitzler, Ischl\oindex{Bad Ischl@\textbf{Bad Ischl}|pw}, Kaiserkrone\oindex{Hotel Kaiserkrone@\textbf{Hotel Kaiserkrone}|pw}}\pend{}{\bigskip}\pstart{}Herrn \textsc{Dr. Richard Beer-Hofmann}\pend{}\pstart{}\strikeout{\textsc{Untera}\oindex{Unterach am Attersee@\textbf{Unterach am Attersee}|pw}}\pend{}\pstart{}\textsc{Weißenbach\oindex{Weissenbach am Attersee@\textbf{Weißenbach am Attersee}|pw}.}\pend{}\pstart{}\textsc{Am}{ }\textsc{Atter}ſee\oindex{Attersee@\textbf{Attersee}|pw}\pend{}{\bigskip}\pstart
           \raggedleft{}{\pb}\textsc{Ischl}\oindex{Bad Ischl@\textbf{Bad Ischl}|pw}, 22/8 914. \pend
           \pstart{}lieber Richard,\pend\pstart
           wir ſind recht reiſemüde nach dieſer höchſt unbequemen überlangen Fahrt – wollen hier
               eigentlich nur ein paar Tage ausruhn und nicht mehr hin u her radeln. Vielleicht
               entſchließen Sie ſich mit Paula\pwindex{Beer-Hofmann, Paula 25.02.1879 – 30.10.1939@\textsc{Beer-Hofmann, Paula} (25.02.1879 – 30.10.1939)|pw},
                  Montag oder \label{KLL02194_Beer-Hofmann-1v}\edtext{Dinſtag}{\lemma{\textnormal{\emph{Dinſtag}}}\Cendnote{\textnormal{vgl. A. S.: \emph{Tagebuch}, 25. 8. 1914}}}\label{KLL02194_Beer-Hofmann-1h} herüberzufahren? Es wäre ſehr ſchön! Wir dürften Mittwoch oder
                  Donnerſtg\label{KLL02194_Beer-Hofmann-2v}\edtext{{ }heimfahren}{\lemma{\textnormal{\emph{ heimfahren}}}\Cendnote{\textnormal{Das verzögerte sich bis 30. 9. 1914.}}}\label{KLL02194_Beer-Hofmann-2h}.\pend
           \pstart
           Wie lange bleiben Sie überhaupt noch?\pend
           \pstart
           Wir grüßen Sie alle herzlichſt!\pend
           \pstart
           Ihr{\\[\baselineskip]}\spacefill\mbox{Arthur}\pend
           \leftskip=0em{}\pstart
           Vielleicht machen Sie \introOben{}etwas\introOben{} mit Saltens\pwindex{Salten, Felix 06.09.1869 – 08.10.1945@\textsc{Salten, Felix} (06.09.1869 – 08.10.1945), \emph{Schriftsteller, Journalist}|pw}\pwindex{Salten, Ottilie 07.03.1868 – 22.06.1942@\textsc{Salten, Ottilie} (07.03.1868 – 22.06.1942), \emph{Schauspielerin}|pw} ab, dem ich in ähnlichem Sinn ſchreibe\pend
           \endnumbering\briefempfaengerindex{Beer-Hofmann, Richard@\textsc{Beer-Hofmann, Richard}!zzzSchnitzler, Arthur@\emph{von Arthur Schnitzler}!1914-08-221@{22. 8. 1914}|)be}\mylabel{h}\end{ledgroupsized}  \newcommand{\dateiname}{L02194}\newcommand{\titel}{Arthur Schnitzler an Richard Beer-Hofmann, 22. 8. 1914}\newcommand{\editorInnen}{Martin Anton Müller und Gerd-Hermann Susen}
            \footnotesize
\begin{ledgroupsized}[t]{11.5cm}
\doendnotes{C}
\end{ledgroupsized}
         %% latex-leseansicht-abspann.tex
%% Abspann für die Leseansicht.
%% Der Schalter \ifkorrekturansicht ist bereits durch den Vorspann gesetzt.

%% latex-abspann.tex
%% Gemeinsamer Abspann für Korrekturansicht und Leseansicht.
%% Setzt den Schalter \ifkorrekturansicht voraus (gesetzt in den
%% einbindenden Dateien latex-korrekturansicht-abspann.tex bzw.
%% latex-leseansicht-abspann.tex).
%% ---------------------------------------------------------------

\normalsize

% Das esempio-Environment wird nur in der Leseansicht benötigt
\ifkorrekturansicht\else
\newenvironment{esempio}[3]%
{
    \vspace{1.5ex}
    \rlap{\underline{#1}}
    \par
    \setlength{\parindent}{0cm}
    \nopagebreak
    \leftskip=#2cm
    \rightskip=#3cm
}
{
    \par
}
\fi

\doendnotes{C}
\bigskip
\vfill

\clearpage

\footnotesize

\ifkorrekturansicht
  \lohead{\textsc{register}}
\fi

% theindex-Environment neu definieren ohne reledmac
\makeatletter
\renewenvironment{theindex}{%
  \ifkorrekturansicht
    \section*{\indexname}%
  \else
    \subsubsection*{Index der erwähnten Entitäten}%
  \fi
  \setlength{\parindent}{0pt}%
  \setlength{\parskip}{0pt plus 0.3pt}%
  \let\item\@idxitem
}{%
  \ifkorrekturansicht\clearpage\fi
}
\makeatother

\IfFileExists{\jobname-pw.ind}{\input{\jobname-pw.ind}}{}

% Quellenangabe nur in der Leseansicht
\ifkorrekturansicht\else
% Fallback-Definitionen, falls die .tex-Datei \titel etc. nicht gesetzt hat
\providecommand{\titel}{}
\providecommand{\editorInnen}{}
\providecommand{\dateiname}{\jobname}

\vspace{3cm}

\vfill

\footnotesize
\textsc{Quelle}: \titel. Herausgegeben von {\editorInnen}. In: \emph{Arthur Schnitzler: Briefwechsel mit Autorinnen und Autoren}.
 Digitale Edition, https://schnitzler-briefe.acdh.oeaw.ac.at/{\dateiname}.html (Stand \today)
\fi

\end{document}


      