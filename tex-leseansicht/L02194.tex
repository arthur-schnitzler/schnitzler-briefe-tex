%% latex-korrekturansicht-vorspann.tex
%% Vorspann für die Korrekturansicht.
%% Lädt die gemeinsame Datei latex-vorspann.tex mit gesetztem Schalter.

\newif\ifkorrekturansicht
\korrekturansichttrue

\input{../tex-inputs/latex-vorspann}


\section[Arthur Schnitzler an Richard Beer-Hofmann, 22. 8. 1914]{L02194 Arthur Schnitzler an Richard Beer-Hofmann, 22. 8. 1914}
\nopagebreak\mylabel{L02194v}
\rehead{ }\normalsize\beginnumbering\briefempfaengerindex{Beer-Hofmann, Richard@\textsc{Beer-Hofmann, Richard}!zzzSchnitzler, Arthur@\emph{von Arthur Schnitzler}!1914-08-221@{22. 8. 1914}|(be}
\toendnotes[C]{\smallbreak\pagebreak[2]}\Standort{YCGL, MSS 31.}
\physDesc{Kartenbrief, 566 Zeichen
\newline{}Handschrift: Bleistift, deutsche Kurrent
\newline{}Versand: Stempel: »\nobreak{}\oindex{Bad Ischl@\textbf{Bad Ischl}, \emph{P.PPL}|pwk}{[}Bad{]} Ischl, 2\textcolor{gray}{2}. VIII. {[}1914{]}\nobreak{}«.  
\newline{}Beer-Hofmann: mit blauem Buntstift den Erhalt und die Beantwortung markiert: »\noindent{}E.B{ / }24/VIII 14{ }\textsc{Telegr.}« }
\buchAbdrucke{\weitereDrucke{Arthur Schnitzler, Richard Beer-Hofmann: \emph{Briefwechsel 1891–1931}. Wien, Zürich: \emph{Europaverlag} 1992, S. 220.} }\toendnotes[C]{\smallbreak}\pstart{}{\pb}Abſ. \textsc{Schnitzler, Ischl\oindex{Bad Ischl@\textbf{Bad Ischl}, \emph{P.PPL}|pw}, Kaiserkrone\oindex{Hotel Kaiserkrone@\textbf{Hotel Kaiserkrone}, \emph{Hotel (K.HTL)}|pw}}\pend{}{\bigskip}\pstart{}Herrn \textsc{Dr. Richard Beer-Hofmann}\pend{}\pstart{}\strikeout{\textsc{Untera}\oindex{Unterach am Attersee@\textbf{Unterach am Attersee}, \emph{P.PPL}|pw}}\pend{}\pstart{}\textsc{Weißenbach\oindex{Weissenbach am Attersee@\textbf{Weißenbach am Attersee}, \emph{A.ADM3}|pw}.}\pend{}\pstart{}\textsc{Am}{ }\textsc{Atter}ſee\oindex{Attersee@\textbf{Attersee}, \emph{H.LK}|pw}\pend{}{\bigskip}\vspace{1em}
\pstart
           \raggedleft{}{\pb}\textsc{Ischl}\oindex{Bad Ischl@\textbf{Bad Ischl}, \emph{P.PPL}|pw}, 22/8 914. \pend
           
\pstart{}lieber Richard,\pend\vspace{0.5em}
\pstart
           wir ſind recht reiſemüde nach dieſer höchſt unbequemen überlangen Fahrt – wollen hier
               eigentlich nur ein paar Tage ausruhn und nicht mehr hin u her radeln. Vielleicht
               entſchließen Sie ſich mit Paula\pwindex{Beer-Hofmann, Paula 25.02.1879 – 30.10.1939@\textsc{Beer-Hofmann, Paula} (25.02.1879 – 30.10.1939)|pw},
                  Montag oder \label{K_L02194-1v}\edtext{Dinſtag}{\lemma{\textnormal{\emph{Dinſtag}}}\Cendnote{\textnormal{Vgl. A. S.: \emph{Tagebuch}, 25. 8. 1914.
               }}}\label{K_L02194-1} herüberzufahren? Es wäre ſehr ſchön! Wir dürften Mittwoch oder
                  Donnerſtg\label{K_L02194-2v}\edtext{{ }heimfahren}{\lemma{\textnormal{\emph{ heimfahren}}}\Cendnote{\textnormal{Das verzögerte sich bis 30. 9. 1914.}}}\label{K_L02194-2}.\pend
           
\pstart
           Wie lange bleiben Sie überhaupt noch?\pend
           
\pstart
           Wir grüßen Sie alle herzlichſt!\pend
           
\pstart
           Ihr{\\[\baselineskip]}\spacefill\mbox{Arthur}\pend
           \leftskip=0em{}
\pstart
           Vielleicht machen Sie \introOben{}etwas\introOben{} mit Saltens\pwindex{Salten, Felix 06.09.1869 – 08.10.1945@\textsc{Salten, Felix} (06.09.1869 – 08.10.1945), \emph{Schriftsteller/Schriftstellerin, Journalist/Journalistin, Chefredakteur/Chefredakteurin}|pw}\pwindex{Salten, Ottilie 07.03.1868 – 22.06.1942@\textsc{Salten, Ottilie} (07.03.1868 – 22.06.1942), \emph{Schauspieler/Schauspielerin}|pw} ab, dem ich in ähnlichem Sinn ſchreibe\pend
           \selectlanguage{ngerman}\endnumbering\briefempfaengerindex{Beer-Hofmann, Richard@\textsc{Beer-Hofmann, Richard}!zzzSchnitzler, Arthur@\emph{von Arthur Schnitzler}!1914-08-221@{22. 8. 1914}|)be}\mylabel{L02194h}  \normalsize

\doendnotes{C}
\bigskip
\vfill

\clearpage

\footnotesize

\lohead{\textsc{register}}

% Definiere theindex-Environment komplett neu ohne reledmac
\makeatletter
\renewenvironment{theindex}{%
  \section*{\indexname}%
  \setlength{\parindent}{0pt}%
  \setlength{\parskip}{0pt plus 0.3pt}%
  \let\item\@idxitem
}{%
  \clearpage
}
\makeatother

\IfFileExists{\jobname-pw.ind}{\input{\jobname-pw.ind}}{}

\end{document}

      