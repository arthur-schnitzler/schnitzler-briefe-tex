%% latex-korrekturansicht-vorspann.tex
%% Vorspann für die Korrekturansicht.
%% Lädt die gemeinsame Datei latex-vorspann.tex mit gesetztem Schalter.

\newif\ifkorrekturansicht
\korrekturansichttrue

\input{../tex-inputs/latex-vorspann}


\section[Georg Brandes an Arthur Schnitzler, 20. 11. 1912]{L02102 Georg Brandes an Arthur Schnitzler, 20. 11. 1912}
\nopagebreak\mylabel{L02102v}
\rehead{ }\normalsize\beginnumbering\briefempfaengerindex{Schnitzler, Arthur@\textsc{Schnitzler, Arthur}!zzzBrandes, Georg@\emph{von Georg Brandes}!1912-11-202@{20. 11. 1912}|(be}
\toendnotes[C]{\smallbreak\pagebreak[2]}\Standort{CUL, Schnitzler, B 17-2.}
\physDesc{Karte, maschinenschriftliche Abschrift1 Blatt, 1 Seite, 351 Zeichen
\newline{}Schreibmaschine
\newline{}Ordnung: von unbekannter Hand als Briefnummer »34«
                                 gekennzeichnet und die Seitenzahl »40«
                                 vermerkt }
\buchAbdrucke{\weitereDrucke{Georg Brandes, Arthur Schnitzler: \emph{Ein Briefwechsel}. Bern: \emph{Francke} 1956, S. 105.} }
\pstart
           \raggedleft{}{\pb}Wien\oindex{Wien@\textbf{Wien}, \emph{A.ADM2}|pw}, 20. 11. 1912. \pend
           
\pstart{}Mein lieber Schnitzler.\pend\vspace{0.5em}
\pstart
           Wie schade, dass Sie weggehen, wenn ich komme. Ich will natürlich mit grosser Freude
                  Freitag Abend bei Ihnen sein.\pend
           
\pstart
           Ich soll heute Abend, morgen und Sonnabend
               reden, habe also eben Freitag frei. Glauben Sie doch nicht, dass man
               sich um mich reisst, ich werde sehr still hier einige Tage leben.\pend
           \pstart Ihr alter Freund \spacefill\mbox{Georg Brandes}\pend{}\selectlanguage{ngerman}\endnumbering\briefempfaengerindex{Schnitzler, Arthur@\textsc{Schnitzler, Arthur}!zzzBrandes, Georg@\emph{von Georg Brandes}!1912-11-202@{20. 11. 1912}|)be}\mylabel{L02102h}  \normalsize

\doendnotes{C}
\bigskip
\vfill

\clearpage

\footnotesize

\lohead{\textsc{register}}

% Definiere theindex-Environment komplett neu ohne reledmac
\makeatletter
\renewenvironment{theindex}{%
  \section*{\indexname}%
  \setlength{\parindent}{0pt}%
  \setlength{\parskip}{0pt plus 0.3pt}%
  \let\item\@idxitem
}{%
  \clearpage
}
\makeatother

\IfFileExists{\jobname-pw.ind}{\input{\jobname-pw.ind}}{}

\end{document}

      