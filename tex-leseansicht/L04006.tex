%% latex-leseansicht-vorspann.tex
%% Vorspann für die Leseansicht.
%% Lädt die gemeinsame Datei latex-vorspann.tex mit nicht gesetztem Schalter.

\newif\ifkorrekturansicht
\korrekturansichtfalse

\input{../tex-inputs/latex-vorspann}


\section[Berta Zuckerkandl an Arthur Schnitzler, {[}23. 5. 1925?{]}]{L04006 Berta Zuckerkandl an Arthur Schnitzler, {[}23. 5. 1925?{]}}
\nopagebreak\mylabel{L04006v}
\rehead{ }\normalsize\beginnumbering\briefempfaengerindex{Schnitzler, Arthur@\textsc{Schnitzler, Arthur}!zzzZuckerkandl, Berta@\emph{von Berta Zuckerkandl}!1925-05-231@{{[}23. 5. 1925?{]}}|(be}
\toendnotes[C]{\smallbreak\pagebreak[2]}
\correspDesc{Versand  durch Berta Zuckerkandl am [23. 5. 1925?] in Wien
\newline{}Erhalt  durch Arthur Schnitzler im Zeitraum [23. 5. 1925
                  – 26. 5. 1925?] in Wien}\toendnotes[C]{\smallbreak}
\Standort{CUL, Schnitzler, B 200.}
\physDesc{Karte, 2 Seiten, , 703 Zeichen
\newline{}Handschrift: schwarze Tinte, lateinische Kurrent}\toendnotes[C]{\smallbreak}
\pstart
           {\pb}Samstag\pend
           \vspace{0.5em}
\pstart
           Verehrter Freund! Ich wollte Sie \label{K_L04006-1v}\edtext{nach der Probe\eventindex{Burgtheater@\textbf{Burgtheater}!Generalprobe von Der Schleier der Beatrice, 22.5.1925@Generalprobe von Der Schleier der Beatrice, 22.5.1925|pwv}}{\lemma{\textnormal{\emph{nach der Probe}}}\Cendnote{\textnormal{Das Korrespondenzstück ist nicht
                  datiert. Der Hinweis auf das Theaterstück über eine traurige Liebesbeziehung, das
                  nach 25 Jahren noch aktuell wirkt, lässt darauf schließen, dass der Brief Bezug
                  nimmt auf die Generalprobe von \emph{Der Schleier der Beatrice}\pwindex{Schnitzler, Arthur 15. 5. 1862 Wien – 21. 10. 1931 ebd.@\textsc{Schnitzler, Arthur} (15. 5. 1862 Wien – 21. 10. 1931 ebd.), \emph{Schriftsteller, Mediziner}!Schleier der Beatrice. Schauspiel in fünf Akten@\strich\emph{Der Schleier der Beatrice. Schauspiel in fünf Akten}|pwk}\eventindex{Burgtheater@\textbf{Burgtheater}!Generalprobe von Der Schleier der Beatrice, 22.5.1925@Generalprobe von Der Schleier der Beatrice, 22.5.1925|pwk}, denn dieses Schauspiel\pwindex{Schnitzler, Arthur 15. 5. 1862 Wien – 21. 10. 1931 ebd.@\textsc{Schnitzler, Arthur} (15. 5. 1862 Wien – 21. 10. 1931 ebd.), \emph{Schriftsteller, Mediziner}!Schleier der Beatrice. Schauspiel in fünf Akten@\strich\emph{Der Schleier der Beatrice. Schauspiel in fünf Akten}|pwkv} erlebte erst ein Vierteljahrhundert nach Entstehung seine Wiener\oindex{Wien@\textbf{Wien}, \emph{Verwaltungsgebiet}|pwk}{ }Erstaufführung\eventindex{Burgtheater@\textbf{Burgtheater}!Premiere von Der Schleier der Beatrice, 23.5.1925@Premiere von Der Schleier der Beatrice, 23.5.1925|pwkv}. Die
                  Angabe des Wochentages Samstag in der Datumszeile passt zum Wochentag der Premiere\eventindex{Burgtheater@\textbf{Burgtheater}!Premiere von Der Schleier der Beatrice, 23.5.1925@Premiere von Der Schleier der Beatrice, 23.5.1925|pwkv} am
                     23. 5. 1925 mit am Tag zuvor vorangegangner Generalprobe\eventindex{Burgtheater@\textbf{Burgtheater}!Generalprobe von Der Schleier der Beatrice, 22.5.1925@Generalprobe von Der Schleier der Beatrice, 22.5.1925|pwkv}.}}}\label{K_L04006-1} nicht stören. Und
               will nicht warten bis ich Sie wiedersehe um Ihnen zu sagen welch einen starken
               Eindruck uns Ihr trauriges Liebesspiel\pwindex{Schnitzler, Arthur 15. 5. 1862 Wien – 21. 10. 1931 ebd.@\textsc{Schnitzler, Arthur} (15. 5. 1862 Wien – 21. 10. 1931 ebd.), \emph{Schriftsteller, Mediziner}!Schleier der Beatrice. Schauspiel in fünf Akten@\strich\emph{Der Schleier der Beatrice. Schauspiel in fünf Akten}|pwv} gemacht hat. Es ist Ewigkeitszug darin. Es weht die Eisluft
               schmerzlicher Einsamkeit, in all dieser Zweisamkeit.\pend
           
\pstart
           Dass ich gesagte darüber nicht schreiben kann \label{K_L04006-2v}\edtext{sondern der Herr – X\pwindex{Jacobson, Leopold 30.\,6.\,1878 Czernowitz – 1942@\textsc{Jacobson, Leopold} (30.\,6.\,1878 Czernowitz – 1942), \emph{Schriftsteller, Journalist}|pwuv} –}{\lemma{\textnormal{\emph{sondern der Herr – X –}}}\Cendnote{\textnormal{Vermutlich ist Leopold Jacobson\pwindex{Jacobson, Leopold 30.\,6.\,1878 Czernowitz – 1942@\textsc{Jacobson, Leopold} (30.\,6.\,1878 Czernowitz – 1942), \emph{Schriftsteller, Journalist}|pwk} gemeint,
                  der im \emph{Neuen Wiener Journal}\pwindex{Neues Wiener Journal@\emph{Neues Wiener Journal}|pwk}, in dem Zuckerkandl\pwindex{Zuckerkandl, Berta 13.\,4.\,1864 Wien – 16.\,10.\,1945 Paris@\textsc{Zuckerkandl, Berta} (13.\,4.\,1864 Wien – 16.\,10.\,1945 Paris), \emph{Schriftstellerin, Journalistin, Übersetzerin}|pwk} in diesem Jahr
                  vorwiegend über Architektur- und Frankreich\oindex{Frankreich@\textbf{Frankreich}|pwk}themen schrieb, die ausführliche Theaterkritik zur Premiere von \emph{Der
                        Schleier der Beatrice}\pwindex{Schnitzler, Arthur 15. 5. 1862 Wien – 21. 10. 1931 ebd.@\textsc{Schnitzler, Arthur} (15. 5. 1862 Wien – 21. 10. 1931 ebd.), \emph{Schriftsteller, Mediziner}!Schleier der Beatrice. Schauspiel in fünf Akten@\strich\emph{Der Schleier der Beatrice. Schauspiel in fünf Akten}|pwk}\eventindex{Burgtheater@\textbf{Burgtheater}!Premiere von Der Schleier der Beatrice, 23.5.1925@Premiere von Der Schleier der Beatrice, 23.5.1925|pwk} verfasste, siehe \emph{Burgtheater. Zum erstenmal: »Der Schleier der
                        Beatrice«}\pwindex{Jacobson, Leopold 30.\,6.\,1878 Czernowitz – 1942@\textsc{Jacobson, Leopold} (30.\,6.\,1878 Czernowitz – 1942), \emph{Schriftsteller, Journalist}!Burgtheater. Zum erstenmal: »Der Schleier der Beatrice«, Schauspiel in fünf Akten von Artur Schnitzler@\strich\emph{Burgtheater. Zum erstenmal: »Der Schleier der Beatrice«, Schauspiel in fünf Akten von Artur Schnitzler}|pwk}. In: \emph{Neues Wiener
                        Journal}\pwindex{Neues Wiener Journal@\emph{Neues Wiener Journal}|pwk}, Jg. 33, Nr. 11.317, 24. 5. 1925,
                  S. 3–4.}}}\label{K_L04006-2}{ }{\pb}ist idiotische Metier-Konvention. Es wird
               ein grosser Erfolg sein, und die Zartheit der \label{K_L04006-3v}\edtext{Regie}{\lemma{\textnormal{\emph{Regie}}}\Cendnote{\textnormal{Regie führte Franz Herterich\pwindex{Herterich, Franz 3.\,10.\,1877 München – 28.\,10.\,1966 Wien@\textsc{Herterich, Franz} (3.\,10.\,1877 München – 28.\,10.\,1966 Wien), \emph{Theaterleiter, Schauspieler}|pwk}.}}}\label{K_L04006-3} muss jedes Missverstehen
               unmöglich machen. – Wie schön – wenn ein Werk nach 25 Jahren – noch immer von –
               morgen ist!!! Wollen Sie \label{K_L04006-4v}\edtext{morgen Nachmittag}{\lemma{\textnormal{\emph{morgen Nachmittag}}}\Cendnote{\textnormal{Erst am darauffolgenden
                     Freitag dokumentiert das \emph{Tagebuch}\pwindex{Schnitzler, Arthur 15. 5. 1862 Wien – 21. 10. 1931 ebd.@\textsc{Schnitzler, Arthur} (15. 5. 1862 Wien – 21. 10. 1931 ebd.), \emph{Schriftsteller, Mediziner}!Tagebuch@\strich\emph{Tagebuch}|pwk} einen Besuch Schnitzlers
                  bei Zuckerkandl\pwindex{Zuckerkandl, Berta 13.\,4.\,1864 Wien – 16.\,10.\,1945 Paris@\textsc{Zuckerkandl, Berta} (13.\,4.\,1864 Wien – 16.\,10.\,1945 Paris), \emph{Schriftstellerin, Journalistin, Übersetzerin}|pwk}, siehe A. S.: \emph{Tagebuch}, 29. 5. 1925.}}}\label{K_L04006-4} ein
               bisserl plauschen ko{\geminationm}en? Sonst frage ich mich in den
               nächsten Tagen bei Ihnen an. Herzlich{[}s{]}t\pend
           \pstart \spacefill\mbox{B. Z.}\pend{}\selectlanguage{ngerman}\endnumbering\briefempfaengerindex{Schnitzler, Arthur@\textsc{Schnitzler, Arthur}!zzzZuckerkandl, Berta@\emph{von Berta Zuckerkandl}!1925-05-231@{{[}23. 5. 1925?{]}}|)be}\mylabel{L04006h}
\begin{anhang}
\end{anhang}\newcommand{\dateiname}{L04006}\newcommand{\titel}{Berta Zuckerkandl an Arthur Schnitzler, [23. 5. 1925?]}\newcommand{\editorInnen}{Herausgegeben von Jahnke, SelmaMüller, Martin Anton}%% latex-leseansicht-abspann.tex
%% Abspann für die Leseansicht.
%% Der Schalter \ifkorrekturansicht ist bereits durch den Vorspann gesetzt.

%% latex-abspann.tex
%% Gemeinsamer Abspann für Korrekturansicht und Leseansicht.
%% Setzt den Schalter \ifkorrekturansicht voraus (gesetzt in den
%% einbindenden Dateien latex-korrekturansicht-abspann.tex bzw.
%% latex-leseansicht-abspann.tex).
%% ---------------------------------------------------------------

\normalsize

% Das esempio-Environment wird nur in der Leseansicht benötigt
\ifkorrekturansicht\else
\newenvironment{esempio}[3]%
{
    \vspace{1.5ex}
    \rlap{\underline{#1}}
    \par
    \setlength{\parindent}{0cm}
    \nopagebreak
    \leftskip=#2cm
    \rightskip=#3cm
}
{
    \par
}
\fi

\doendnotes{C}
\bigskip
\vfill

\clearpage

\footnotesize

\ifkorrekturansicht
  \lohead{\textsc{register}}
\fi

% theindex-Environment neu definieren ohne reledmac
\makeatletter
\renewenvironment{theindex}{%
  \ifkorrekturansicht
    \section*{\indexname}%
  \else
    \subsubsection*{Index der erwähnten Entitäten}%
  \fi
  \setlength{\parindent}{0pt}%
  \setlength{\parskip}{0pt plus 0.3pt}%
  \let\item\@idxitem
}{%
  \ifkorrekturansicht\clearpage\fi
}
\makeatother

\IfFileExists{\jobname-pw.ind}{\input{\jobname-pw.ind}}{}

% Quellenangabe nur in der Leseansicht
\ifkorrekturansicht\else
% Fallback-Definitionen, falls die .tex-Datei \titel etc. nicht gesetzt hat
\providecommand{\titel}{}
\providecommand{\editorInnen}{}
\providecommand{\dateiname}{\jobname}

\vspace{3cm}

\vfill

\footnotesize
\textsc{Quelle}: \titel. Herausgegeben von {\editorInnen}. In: \emph{Arthur Schnitzler: Briefwechsel mit Autorinnen und Autoren}.
 Digitale Edition, https://schnitzler-briefe.acdh.oeaw.ac.at/{\dateiname}.html (Stand \today)
\fi

\end{document}


