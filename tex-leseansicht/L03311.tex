%% latex-leseansicht-vorspann.tex
%% Vorspann für die Leseansicht.
%% Lädt die gemeinsame Datei latex-vorspann.tex mit nicht gesetztem Schalter.

\newif\ifkorrekturansicht
\korrekturansichtfalse

\input{../tex-inputs/latex-vorspann}

\begin{center}
            \textcolor{red}{ENTWURF, NICHT FERTIG KORRIGIERT}
                      \end{center}
            
         
         \renewcommand{\erwaehntePersonen}{Personen: Caroline Kotter, Ottmar Peter Kotter, Leopoldine Müller, Louise Schnitzler}
         \renewcommand{\erwaehnteOrte}{Orte: Bad Ischl, Meran, Schruns, Traunkai}
         \renewcommand{\erwaehnteWerke}{}
               \section[Felix Salten an Arthur Schnitzler, 18. 8. 1900]{ Felix Salten an Arthur Schnitzler, 18. 8. 1900}\nopagebreak\mylabel{v}\rehead{ }\begin{ledgroupsized}[t]{13cm}\normalsize\beginnumbering \toendnotes[C]{\smallbreak\pagebreak[2]} \Standort{CUL, Schnitzler, B 89, A 2.}
\physDesc{Brief, 1 Blatt, 2 Seiten, 744 Zeichen
\newline{}Handschrift: schwarze Tinte, lateinische Kurrent
\newline{}Ordnung: mit Bleistift von unbekannter Hand nummeriert:
                                    »135« }\toendnotes[C]{\smallbreak}\pstart
           \raggedleft{}{\pb}Ischl, Traunquai 27\oindex{Traunkai@\textbf{Traunkai}|pw}, \uline{nicht} 11. \pend
           \pstart
           \raggedleft{}18. VIII. 00. \pend
           \pstart
           Lieber Freund, Ihre Frau Mama\pwindex{Schnitzler, Louise 1840-07-08 – 1911-09-09@\textsc{Schnitzler, Louise} (1840-07-08 – 1911-09-09)|pwv} sagte mir heute, Sie hätten sie gefragt, wann ich nach
                  Meran\oindex{Meran@\textbf{Meran}|pw} komme. Da ich daraus entnehme, dass
               Sie meinen Brief in Schruns\oindex{Schruns@\textbf{Schruns}|pw} schon erhalten
               haben, bitte ich Sie nochmals um Nachricht, wann Sie in Meran\oindex{Meran@\textbf{Meran}|pw} sind. Ich kann vom 24. an (auch früher)
               jeden Tag. Ich bitte Sie, mir genau die Tour zu schreiben, die Sie vorschlagen, weil
               ich mir von hier aus die Eisenbahnkarte darnach bestellen muß. Das dauert auch 3–4
               Tage und je früher ich’s weiß, desto besser ist es. Wie geht es? Es thut mir leid,
               dass ich nicht mit konnte. \pend
           \pstart
            Herzlichst Ihr {\\[\baselineskip]}\spacefill\mbox{Salten.}\pend
           \leftskip=0em{}\pstart
           \noindent{}Ellychen\pwindex{Kotter, Caroline 1893-07-07 – 1964-07-01@\textsc{Kotter, Caroline} (1893-07-07 – 1964-07-01)|pw} und Peter\pwindex{Kotter, Ottmar Peter *~1898@\textsc{Kotter, Ottmar Peter} (*~1898)|pw} befinden sich wol, nur heißt Peter\pwindex{Kotter, Ottmar Peter *~1898@\textsc{Kotter, Ottmar Peter} (*~1898)|pw} jetzt »Pumpi«. \pend
           \pstart
           Richtig! vor einer {\pb}halben
                  Stunde hab ich Frl. Poldi\pwindex{Mueller, Leopoldine 01.11.1873 – 18.01.1946@\textsc{Müller, Leopoldine} (01.11.1873 – 18.01.1946)|pw} gesehen, sie sah
                  bildhübsch aus! \pend
           
         
         \endnumbering\mylabel{h}\end{ledgroupsized}\begin{anhang}\end{anhang}\newcommand{\dateiname}{L03311}\newcommand{\titel}{Felix Salten an Arthur Schnitzler, 18. 8. 1900}\newcommand{\editorInnen}{Martin Anton Müller und Laura Untner}%% latex-leseansicht-abspann.tex
%% Abspann für die Leseansicht.
%% Der Schalter \ifkorrekturansicht ist bereits durch den Vorspann gesetzt.

%% latex-abspann.tex
%% Gemeinsamer Abspann für Korrekturansicht und Leseansicht.
%% Setzt den Schalter \ifkorrekturansicht voraus (gesetzt in den
%% einbindenden Dateien latex-korrekturansicht-abspann.tex bzw.
%% latex-leseansicht-abspann.tex).
%% ---------------------------------------------------------------

\normalsize

% Das esempio-Environment wird nur in der Leseansicht benötigt
\ifkorrekturansicht\else
\newenvironment{esempio}[3]%
{
    \vspace{1.5ex}
    \rlap{\underline{#1}}
    \par
    \setlength{\parindent}{0cm}
    \nopagebreak
    \leftskip=#2cm
    \rightskip=#3cm
}
{
    \par
}
\fi

\doendnotes{C}
\bigskip
\vfill

\clearpage

\footnotesize

\ifkorrekturansicht
  \lohead{\textsc{register}}
\fi

% theindex-Environment neu definieren ohne reledmac
\makeatletter
\renewenvironment{theindex}{%
  \ifkorrekturansicht
    \section*{\indexname}%
  \else
    \subsubsection*{Index der erwähnten Entitäten}%
  \fi
  \setlength{\parindent}{0pt}%
  \setlength{\parskip}{0pt plus 0.3pt}%
  \let\item\@idxitem
}{%
  \ifkorrekturansicht\clearpage\fi
}
\makeatother

\IfFileExists{\jobname-pw.ind}{\input{\jobname-pw.ind}}{}

% Quellenangabe nur in der Leseansicht
\ifkorrekturansicht\else
% Fallback-Definitionen, falls die .tex-Datei \titel etc. nicht gesetzt hat
\providecommand{\titel}{}
\providecommand{\editorInnen}{}
\providecommand{\dateiname}{\jobname}

\vspace{3cm}

\vfill

\footnotesize
\textsc{Quelle}: \titel. Herausgegeben von {\editorInnen}. In: \emph{Arthur Schnitzler: Briefwechsel mit Autorinnen und Autoren}.
 Digitale Edition, https://schnitzler-briefe.acdh.oeaw.ac.at/{\dateiname}.html (Stand \today)
\fi

\end{document}


      