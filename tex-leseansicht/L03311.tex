%% latex-korrekturansicht-vorspann.tex
%% Vorspann für die Korrekturansicht.
%% Lädt die gemeinsame Datei latex-vorspann.tex mit gesetztem Schalter.

\newif\ifkorrekturansicht
\korrekturansichttrue

\input{../tex-inputs/latex-vorspann}


\section[ Felix Salten an Arthur Schnitzler, 18. 8. 1900]{L03311 Felix Salten an Arthur Schnitzler, 18. 8. 1900}
\nopagebreak\mylabel{L03311v}
\rehead{ }\normalsize\beginnumbering\briefempfaengerindex{Schnitzler, Arthur@\textsc{Schnitzler, Arthur}!zzzSalten, Felix@\emph{von Felix Salten}!1900-08-181@{18. 8. 1900}|(be}
\toendnotes[C]{\smallbreak\pagebreak[2]}\Standort{CUL, Schnitzler, B 89, A 2.}
\physDesc{Brief, 1 Blatt, 2 Seiten, 744 Zeichen
\newline{}Handschrift: schwarze Tinte, lateinische Kurrent
\newline{}Ordnung: mit Bleistift von unbekannter Hand nummeriert: »135« }\toendnotes[C]{\smallbreak}
\pstart
           \raggedleft{}{\pb}Ischl, Traunquai 27\oindex{Traunkai@\textbf{Traunkai}, \emph{Straße (K.STR)}|pw}, \uline{nicht} 11. {\\}18. VIII. 00.\pend
           \vspace{0.5em}
\pstart
           Lieber Freund, Ihre Frau Mama\pwindex{Schnitzler, Louise 1840-07-08 – 1911-09-09@\textsc{Schnitzler, Louise} (1840-07-08 – 1911-09-09)|pwv} sagte mir heute, Sie
               hätten sie gefragt, wann ich nach Meran\oindex{Meran@\textbf{Meran}, \emph{P.PPLA3}|pw} komme.
               Da ich daraus entnehme, dass Sie meinen \label{K_L03311-1v}\edtext{Brief in 
               Schruns\oindex{Schruns@\textbf{Schruns}, \emph{A.ADM3}|pw}}{\lemma{\textnormal{\emph{Brief in 
               Schruns}}}\Cendnote{\textnormal{Siehe Felix Salten an Arthur Schnitzler, 14. 8. 1900.
               }}}\label{K_L03311-1} schon erhalten haben, bitte ich Sie nochmals um Nachricht, wann Sie in Meran\oindex{Meran@\textbf{Meran}, \emph{P.PPLA3}|pw} sind? Ich kann vom
                  24. an, (auch früher) jeden Tag. Ich bitte Sie, mir
               genau die Tour zu schreiben, die Sie vorschlagen, weil ich mir von hier aus die
               Eisenbahnkarte danach bestellen muß. Das dauert auch 3–4 Tage und je früher ich’s
               weiß, desto besser ist es.\hspace*{2em}Wie geht es? Es thut mir leid, dass ich nicht mit
               konnte.\pend
           
\pstart
            Herzlichst Ihr {\\[\baselineskip]}\spacefill\mbox{Salten.}\pend
           \leftskip=0em{}
\pstart
           \noindent{}\label{K_L03311-2v}\edtext{Ellychen\pwindex{Kotter, Caroline 1893-07-07 – 1964-07-01@\textsc{Kotter, Caroline} (1893-07-07 – 1964-07-01)|pw} und Peter\pwindex{Kotter, Ottmar Peter *~1898@\textsc{Kotter, Ottmar Peter} (*~1898)|pw}}{\lemma{\textnormal{\emph{Ellychen und Peter}}}\Cendnote{\textnormal{Caroline\pwindex{Kotter, Caroline 1893-07-07 – 1964-07-01@\textsc{Kotter, Caroline} (1893-07-07 – 1964-07-01)|pwk} und Ottmar Peter Kotter\pwindex{Kotter, Ottmar Peter *~1898@\textsc{Kotter, Ottmar Peter} (*~1898)|pwk}, Saltens\pwindex{Salten, Felix 06.09.1869 – 08.10.1945@\textsc{Salten, Felix} (06.09.1869 – 08.10.1945), \emph{Schriftsteller/Schriftstellerin, Journalist/Journalistin, Chefredakteur/Chefredakteurin}|pwk} Kinder mit Elisabeth
                        Kotter\pwindex{Kotter, Elisabeth *~1873@\textsc{Kotter, Elisabeth} (*~1873), \emph{Haushaltshilfe/Haushaltshilfe}|pwk}}}}\label{K_L03311-2} befinden sich wol, nur heißt Peter\pwindex{Kotter, Ottmar Peter *~1898@\textsc{Kotter, Ottmar Peter} (*~1898)|pw}
                  jetzt »Pumpi«.\pend
           
\pstart
           Richtig! vor einer {\pb}halben
                  Stunde hab ich Frl. \label{K_L03311-3v}\edtext{Poldi\pwindex{Mueller, Leopoldine 01.11.1873 – 18.01.1946@\textsc{Müller, Leopoldine} (01.11.1873 – 18.01.1946)|pw}}{\lemma{\textnormal{\emph{Poldi}}}\Cendnote{\textnormal{Leopoldine Müller\pwindex{Mueller, Leopoldine 01.11.1873 – 18.01.1946@\textsc{Müller, Leopoldine} (01.11.1873 – 18.01.1946)|pwk}, eine Geliebte Schnitzlers}}}\label{K_L03311-3} gesehen. Sie sah
                  bildhübsch aus!\pend
           \selectlanguage{ngerman}\endnumbering\briefempfaengerindex{Schnitzler, Arthur@\textsc{Schnitzler, Arthur}!zzzSalten, Felix@\emph{von Felix Salten}!1900-08-181@{18. 8. 1900}|)be}\mylabel{L03311h}  \normalsize

\doendnotes{C}
\bigskip
\vfill

\clearpage

\footnotesize

\lohead{\textsc{register}}

% Definiere theindex-Environment komplett neu ohne reledmac
\makeatletter
\renewenvironment{theindex}{%
  \section*{\indexname}%
  \setlength{\parindent}{0pt}%
  \setlength{\parskip}{0pt plus 0.3pt}%
  \let\item\@idxitem
}{%
  \clearpage
}
\makeatother

\IfFileExists{\jobname-pw.ind}{\input{\jobname-pw.ind}}{}

\end{document}

      