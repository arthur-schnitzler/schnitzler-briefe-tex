%% latex-leseansicht-vorspann.tex
%% Vorspann für die Leseansicht.
%% Lädt die gemeinsame Datei latex-vorspann.tex mit nicht gesetztem Schalter.

\newif\ifkorrekturansicht
\korrekturansichtfalse

\input{../tex-inputs/latex-vorspann}


\section[ Felix Salten an Arthur Schnitzler, 18. 8. 1900]{L03311 Felix Salten an Arthur Schnitzler,  18. 8. 1900}
\nopagebreak\mylabel{L03311v}
\rehead{ }\normalsize\beginnumbering\briefempfaengerindex{Schnitzler, Arthur@\textsc{Schnitzler, Arthur}!zzzSalten, Felix@\emph{von Felix Salten}!1900-08-181@{18. 8. 1900}|(be}
\toendnotes[C]{\smallbreak\pagebreak[2]}
\correspDesc{Versand  durch Felix Salten am 18. 8. 1900 in Bad Ischl
\newline{}Erhalt  durch Arthur Schnitzler im Zeitraum [19. 8. 1900
                  – 23. 8. 1900?] in Thusis?}\toendnotes[C]{\smallbreak}
\Standort{CUL, Schnitzler, B 89, A 2.}
\physDesc{Brief, 1 Blatt, 2 Seiten, 744 Zeichen
\newline{}Handschrift: schwarze Tinte, lateinische Kurrent
\newline{}Ordnung: mit Bleistift von unbekannter Hand nummeriert: »135« }\toendnotes[C]{\smallbreak}
\pstart
           \raggedleft{}{\pb}Ischl, Traunquai 27\oindex{Traunkai@\textbf{Traunkai}, \emph{Straße}|pw}, \uline{nicht} 11. {\\}18. VIII. 00.\pend
           \vspace{0.5em}
\pstart
           Lieber Freund, Ihre Frau Mama\pwindex{Schnitzler, Louise 8.\,7.\,1840 Kőszeg – 9.\,9.\,1911 Wien@\textsc{Schnitzler, Louise} (8.\,7.\,1840 Kőszeg – 9.\,9.\,1911 Wien)|pwv} sagte mir heute, Sie
               hätten sie gefragt, wann ich nach Meran\oindex{Meran@\textbf{Meran}, \emph{Hauptstadt}|pw} komme.
               Da ich daraus entnehme, dass Sie meinen \label{K_L03311-1v}\edtext{Brief in 
               Schruns\oindex{Schruns@\textbf{Schruns}, \emph{Verwaltungsgebiet}|pw}}{\lemma{\textnormal{\emph{Brief in 
               Schruns}}}\Cendnote{\textnormal{Siehe XXXX Auszeichnungsfehler: Dokument L03310 nicht gefunden.
               }}}\label{K_L03311-1} schon erhalten haben, bitte ich Sie nochmals um Nachricht, wann Sie in Meran\oindex{Meran@\textbf{Meran}, \emph{Hauptstadt}|pw} sind? Ich kann vom
                  24. an, (auch früher) jeden Tag. Ich bitte Sie, mir
               genau die Tour zu schreiben, die Sie vorschlagen, weil ich mir von hier aus die
               Eisenbahnkarte danach bestellen muß. Das dauert auch 3–4 Tage und je früher ich’s
               weiß, desto besser ist es.\hspace*{2em}Wie geht es? Es thut mir leid, dass ich nicht mit
               konnte.\pend
           
\pstart
           Herzlichst Ihr {\\[\baselineskip]}\spacefill\mbox{Salten.}\pend
           \leftskip=0em{}
\pstart
           \noindent{}\label{K_L03311-2v}\edtext{Ellychen\pwindex{Kotter, Caroline 7.\,7.\,1893 Wien – 1.\,7.\,1964 ebd.@\textsc{Kotter, Caroline} (7.\,7.\,1893 Wien – 1.\,7.\,1964 ebd.)|pw} und Peter\pwindex{Kotter, Ottmar Peter *~1898@\textsc{Kotter, Ottmar Peter} (*~1898)|pw}}{\lemma{\textnormal{\emph{Ellychen und Peter}}}\Cendnote{\textnormal{Caroline\pwindex{Kotter, Caroline 7.\,7.\,1893 Wien – 1.\,7.\,1964 ebd.@\textsc{Kotter, Caroline} (7.\,7.\,1893 Wien – 1.\,7.\,1964 ebd.)|pwk} und Ottmar Peter Kotter\pwindex{Kotter, Ottmar Peter *~1898@\textsc{Kotter, Ottmar Peter} (*~1898)|pwk}, Saltens\pwindex{Salten, Felix 6.\,9.\,1869 Budapest – 8.\,10.\,1945 Zürich@\textsc{Salten, Felix} (6.\,9.\,1869 Budapest – 8.\,10.\,1945 Zürich), \emph{Schriftsteller, Journalist, Chefredakteur}|pwk} Kinder mit Elisabeth
                        Kotter\pwindex{Kotter, Elisabeth *~1873 Groß-Enzersdorf@\textsc{Kotter, Elisabeth} (*~1873 Groß-Enzersdorf), \emph{Haushaltshilfe}|pwk}}}}\label{K_L03311-2} befinden sich wol, nur heißt Peter\pwindex{Kotter, Ottmar Peter *~1898@\textsc{Kotter, Ottmar Peter} (*~1898)|pw}
                  jetzt »Pumpi«.\pend
           
\pstart
           Richtig! vor einer {\pb}halben
                  Stunde hab ich Frl. \label{K_L03311-3v}\edtext{Poldi\pwindex{Müller, Leopoldine 1.\,11.\,1873 Wien – 18.\,1.\,1946 ebd.@\textsc{Müller, Leopoldine} (1.\,11.\,1873 Wien – 18.\,1.\,1946 ebd.)|pw}}{\lemma{\textnormal{\emph{Poldi}}}\Cendnote{\textnormal{Leopoldine Müller\pwindex{Müller, Leopoldine 1.\,11.\,1873 Wien – 18.\,1.\,1946 ebd.@\textsc{Müller, Leopoldine} (1.\,11.\,1873 Wien – 18.\,1.\,1946 ebd.)|pwk}, eine Geliebte Schnitzlers}}}\label{K_L03311-3} gesehen. Sie sah
                  bildhübsch aus!\pend
           \selectlanguage{ngerman}\endnumbering\briefempfaengerindex{Schnitzler, Arthur@\textsc{Schnitzler, Arthur}!zzzSalten, Felix@\emph{von Felix Salten}!1900-08-181@{18. 8. 1900}|)be}\mylabel{L03311h}  \newcommand{\dateiname}{L03311}\newcommand{\titel}{Felix Salten an Arthur Schnitzler, 18. 8. 1900}\newcommand{\editorInnen}{Martin Anton Müller und Laura Untner}%% latex-leseansicht-abspann.tex
%% Abspann für die Leseansicht.
%% Der Schalter \ifkorrekturansicht ist bereits durch den Vorspann gesetzt.

%% latex-abspann.tex
%% Gemeinsamer Abspann für Korrekturansicht und Leseansicht.
%% Setzt den Schalter \ifkorrekturansicht voraus (gesetzt in den
%% einbindenden Dateien latex-korrekturansicht-abspann.tex bzw.
%% latex-leseansicht-abspann.tex).
%% ---------------------------------------------------------------

\normalsize

% Das esempio-Environment wird nur in der Leseansicht benötigt
\ifkorrekturansicht\else
\newenvironment{esempio}[3]%
{
    \vspace{1.5ex}
    \rlap{\underline{#1}}
    \par
    \setlength{\parindent}{0cm}
    \nopagebreak
    \leftskip=#2cm
    \rightskip=#3cm
}
{
    \par
}
\fi

\doendnotes{C}
\bigskip
\vfill

\clearpage

\footnotesize

\ifkorrekturansicht
  \lohead{\textsc{register}}
\fi

% theindex-Environment neu definieren ohne reledmac
\makeatletter
\renewenvironment{theindex}{%
  \ifkorrekturansicht
    \section*{\indexname}%
  \else
    \subsubsection*{Index der erwähnten Entitäten}%
  \fi
  \setlength{\parindent}{0pt}%
  \setlength{\parskip}{0pt plus 0.3pt}%
  \let\item\@idxitem
}{%
  \ifkorrekturansicht\clearpage\fi
}
\makeatother

\IfFileExists{\jobname-pw.ind}{\input{\jobname-pw.ind}}{}

% Quellenangabe nur in der Leseansicht
\ifkorrekturansicht\else
% Fallback-Definitionen, falls die .tex-Datei \titel etc. nicht gesetzt hat
\providecommand{\titel}{}
\providecommand{\editorInnen}{}
\providecommand{\dateiname}{\jobname}

\vspace{3cm}

\vfill

\footnotesize
\textsc{Quelle}: \titel. Herausgegeben von {\editorInnen}. In: \emph{Arthur Schnitzler: Briefwechsel mit Autorinnen und Autoren}.
 Digitale Edition, https://schnitzler-briefe.acdh.oeaw.ac.at/{\dateiname}.html (Stand \today)
\fi

\end{document}


