%% latex-leseansicht-vorspann.tex
%% Vorspann für die Leseansicht.
%% Lädt die gemeinsame Datei latex-vorspann.tex mit nicht gesetztem Schalter.

\newif\ifkorrekturansicht
\korrekturansichtfalse

\input{../tex-inputs/latex-vorspann}


               \section[Arthur Schnitzler an Hermann Bahr, 3. 1. 1902]{ Arthur Schnitzler an Hermann Bahr, 3. 1. 1902}\nopagebreak\mylabel{v}\rehead{ }\begin{ledgroupsized}[t]{13cm}\normalsize\beginnumbering\briefempfaengerindex{Bahr, Hermann@\textsc{Bahr, Hermann}!zzzSchnitzler, Arthur@\emph{von Arthur Schnitzler}!1902-01-031@{3. 1. 1902}|(be} \toendnotes[C]{\smallbreak\pagebreak[2]} \Standort{TMW, HS AM 23348 Ba.}
\physDesc{Brief, 2 Blätter, 7 Seiten
\newline{}Handschrift: Bleistift, deutsche Kurrent\newline{}Ordnung: Lochung }\buchAbdrucke{\weitereDrucke{1) \emph{3. 1. 1902.} In: Arthur Schnitzler: \emph{The Letters of Arthur Schnitzler to Hermann Bahr}. Edited, annotated, and with an introduction, by Donald G.
                        Daviau. Chapel Hill: \emph{The University of North Carolina Press} 1978, S. 73–74 (University of North Carolina studies in the Germanic languages
                        and literatures, 89).} \weitereDrucke{2) Hermann Bahr, Arthur Schnitzler: \emph{Briefwechsel, Aufzeichnungen, Dokumente (1891–1931)}. Hg. Kurt Ifkovits und Martin Anton Müller. Göttingen: \emph{Wallstein} 2018, S. 222–223.} }\toendnotes[C]{\smallbreak}\pstart
           \raggedleft{}{\pb}3. 1. 902{\\}\textsc{Berlin}\oindex{Berlin@\textbf{Berlin}|pw}\pend
           \pstart
           lieber Hermann, ich habe Brahm\pwindex{Brahm, Otto 05.02.1856 – 28.11.1912@\textsc{Brahm, Otto} (05.02.1856 – 28.11.1912), \emph{Theaterleiter, Regisseur}|pw}
               geſprochen, er äußerte ſich anerkennend über den Krampus\pwindex{Bahr, Hermann 19.07.1863 – 15.01.1934@\textsc{Bahr, Hermann} (19.07.1863 – 15.01.1934), \emph{Schriftsteller, Kritiker}!Krampus1901@\strich\emph{Der Krampus} {[}1901{]}|pw}, findet nur, daſs gerade das Deutſche
                  Theater\oindex{Deutsches Theater Berlin@\textbf{Deutsches Theater Berlin}|pw} nicht der rechte Boden für das Stück sei. Ich glaube alſo nicht, daſs
               er zu der Aufführg nach Hamburg\oindex{Hamburg@\textbf{Hamburg}|pw} fahren wird, hielte
               es aber doch für ganz gut, we{\geminationn} du ihn unverbindlich mit
               ein paar {\pb}Worten dazu
               einladen möchteſt. Gegen deine Bemerkung über den literar. Stempel, den doch erſt das
                  Deutſche Theater\oindex{Deutsches Theater Berlin@\textbf{Deutsches Theater Berlin}|pw} verleihe (die ihm mitzutheilen ich
               mich wohl für befugt halten durfte?) ſchien er nicht unempfindlich zu ſein, und ich
               zweifle nicht daran, daſs er deine nächſten Stücke ohne vorgefaſſte Meinung leſen
               wird. Ich bin übrigens mor{\pb}gen Nachmittag bei ihm
               und habe ſicher Gelegenheit, nochmals in deinem Sinne zu reden. Er gehört doch, bei
               allen Begrenztheiten und Eigenſinnigkeiten zu den weitaus verſtändigſten
               Theatermenſchen \introOben{}(vielleicht auch Menſchen ſchlichtweg –)\introOben{},
               die es gibt, und iſt derjenige, mit dem man am gradlinigſten und verläßlichſten
               verkehren kann. Man darf von ihm ſagen, daſs {\pb}er nie lügt. Du
               sollteſt dich einmal perſönlich mit ihm ausſprechen. We{\geminationn}
               er nicht nach Hamburg\oindex{Hamburg@\textbf{Hamburg}|pw} ko{\geminationm}t, vielleicht beſuchst du ihn auf der Hin- oder Rückfahrt? – \pend
           \pstart
           \label{K_L01195_1v}\edtext{Dieſer Tage}{\lemma{\textnormal{\emph{Dieſer Tage}}}\Cendnote{\textnormal{vgl. A. S.: \emph{Tagebuch}, 1. 1. 1902}}}\label{K_L01195_1h}{ }ſprach ich \textsc{Harden\pwindex{Harden, Maximilian 20.10.1861 – 30.10.1927@\textsc{Harden, Maximilian} (20.10.1861 – 30.10.1927), \emph{Schriftsteller, Publizist}|pw}}\damage{,} der jetzt ſehr gegen den kleinen Kraus\pwindex{Kraus, Karl 28.04.1874 – 12.06.1936@\textsc{Kraus, Karl} (28.04.1874 – 12.06.1936), \emph{Schriftsteller, Publizist}|pw}
                  eingeno{\geminationm}en iſt und findet, daſs ein ſolches Blatt\pwindex{Fackel1899 – 1936@\emph{Die Fackel}|pwv} in Berlin\oindex{Berlin@\textbf{Berlin}|pw}{ }ſich nicht halten kö{\geminationn}te. {\pb}Anläßlich der
                  \label{K_L01195_2v}\edtext{Krausiſchen Kritik\pwindex{Kraus, Karl 28.04.1874 – 12.06.1936@\textsc{Kraus, Karl} (28.04.1874 – 12.06.1936), \emph{Schriftsteller, Publizist}!Wie mich Herr Bahr beneidet]01.10.1901 – 01.10.1901@\strich\emph{[Wie mich Herr Bahr beneidet]} {[}01.10.1901 – 01.10.1901{]}|pwv} über die \textsc{veine\pwindex{Capus, Alfred 25.11.1858 – 01.11.1922@\textsc{Capus, Alfred} (25.11.1858 – 01.11.1922), \emph{Schriftsteller, Journalist}!Glueck1901@\strich\emph{Das Glück} {[}1901{]}|pw}}}{\lemma{\textnormal{\emph{Krausiſchen … veine}}}\Cendnote{\textnormal{Kraus\pwindex{Kraus, Karl 28.04.1874 – 12.06.1936@\textsc{Kraus, Karl} (28.04.1874 – 12.06.1936), \emph{Schriftsteller, Publizist}|pwk}{ }schreibt in der \emph{Fackel}\pwindex{Fackel1899 – 1936@\emph{Die Fackel}|pwk} (Bd. 10, H. 82, Anfang October, S. 19\pwindex{Kraus, Karl 28.04.1874 – 12.06.1936@\textsc{Kraus, Karl} (28.04.1874 – 12.06.1936), \emph{Schriftsteller, Publizist}!Wie mich Herr Bahr beneidet]01.10.1901 – 01.10.1901@\strich\emph{[Wie mich Herr Bahr beneidet]} {[}01.10.1901 – 01.10.1901{]}|pwkv}): »Herr Bahr\pwindex{Bahr, Hermann 19.07.1863 – 15.01.1934@\textsc{Bahr, Hermann} (19.07.1863 – 15.01.1934), \emph{Schriftsteller, Kritiker}|pw}, der wiederum das
                     Referat über das Deutsche Volkstheater\oindex{Volkstheater@\textbf{Volkstheater}|pw}
                     übernommen hat, berichtet, dass in dem neuen Stücke\pwindex{Capus, Alfred 25.11.1858 – 01.11.1922@\textsc{Capus, Alfred} (25.11.1858 – 01.11.1922), \emph{Schriftsteller, Journalist}!Glueck1901@\strich\emph{Das Glück} {[}1901{]}|pwv} von Capus\pwindex{Capus, Alfred 25.11.1858 – 01.11.1922@\textsc{Capus, Alfred} (25.11.1858 – 01.11.1922), \emph{Schriftsteller, Journalist}|pw}
                     ein ›mit zwei Strichen wunderbar gezeichneter‹ Journalist vorkomme, der sich
                     nicht verkauft, weil ›ihm das nie so viel tragen kann wie seine
                     Unbestechlichkeit‹. Man versichert mir – ich kann die Mittheilung leider nicht
                     überprüfen –, dass diese Stelle, die Herr Bahr\pwindex{Bahr, Hermann 19.07.1863 – 15.01.1934@\textsc{Bahr, Hermann} (19.07.1863 – 15.01.1934), \emph{Schriftsteller, Kritiker}|pw} mit so munterem Behagen citiert, nachträglich in die Uebersetzung
                     der französischen\oindex{Frankreich@\textbf{Frankreich}|pw} Comödie hineingeflickt
                     worden sei und dass Herr Bahr\pwindex{Bahr, Hermann 19.07.1863 – 15.01.1934@\textsc{Bahr, Hermann} (19.07.1863 – 15.01.1934), \emph{Schriftsteller, Kritiker}|pw}{ }sich selbst citiere.« Bahrs\pwindex{Bahr, Hermann 19.07.1863 – 15.01.1934@\textsc{Bahr, Hermann} (19.07.1863 – 15.01.1934), \emph{Schriftsteller, Kritiker}|pwk} Besprechung, in der sich das Zitat
                  findet: \emph{Das Glück. (La veine. Komödie in vier Aufzügen von
                           Alfred Capus\pwindex{Capus, Alfred 25.11.1858 – 01.11.1922@\textsc{Capus, Alfred} (25.11.1858 – 01.11.1922), \emph{Schriftsteller, Journalist}|pwk}. Deutsch von Theodor Wolff\pwindex{Wolff, Theodor 02.08.1868 – 23.09.1943@\textsc{Wolff, Theodor} (02.08.1868 – 23.09.1943), \emph{Schriftsteller, Journalist}|pwk}. Zum erstenmal aufgeführt
                        im Deutschen Volkstheater\oindex{Volkstheater@\textbf{Volkstheater}|pwk} am 28.
                           September 1901)}\pwindex{Bahr, Hermann 19.07.1863 – 15.01.1934@\textsc{Bahr, Hermann} (19.07.1863 – 15.01.1934), \emph{Schriftsteller, Kritiker}!Glueck. (La veine. Komoedie in vier Aufzuegen von Alfred Capus. Deutsch von Theodor Wolff. Zum ersten Mal aufgefuehrt im Deutschen Volkstheater am 28. September 1901)29. 09. 1901@\strich\emph{Das Glück. (La veine. Komödie in vier Aufzügen von Alfred Capus. Deutsch von Theodor Wolff. Zum ersten Mal aufgeführt im Deutschen Volkstheater am 28. September 1901)} {[}29. 09. 1901{]}|pwk}. In: \emph{Neues
                        Wiener Tagblatt}\pwindex{Neues Wiener Tagblatt1867 – 1945@\emph{Neues Wiener Tagblatt}|pwk}, Jg. 35, Nr. 267, 29. 9. 1901,
                     S. 2–4.}}}\label{K_L01195_2h}, in der Kr.\pwindex{Kraus, Karl 28.04.1874 – 12.06.1936@\textsc{Kraus, Karl} (28.04.1874 – 12.06.1936), \emph{Schriftsteller, Publizist}|pw} von einer
               angeblich extra von dir \introOben{}(?)\introOben{} gegen ihn hineingedichteten
               Stelle erzählte, hat er ihm (\textsc{Harden\pwindex{Harden, Maximilian 20.10.1861 – 30.10.1927@\textsc{Harden, Maximilian} (20.10.1861 – 30.10.1927), \emph{Schriftsteller, Publizist}|pw}} dem Kraus\pwindex{Kraus, Karl 28.04.1874 – 12.06.1936@\textsc{Kraus, Karl} (28.04.1874 – 12.06.1936), \emph{Schriftsteller, Publizist}|pw}) eine Karte geſchrieben, er müſſe
               gelegentlich diesen Irrthum richtigſtellen, da die betreffende Stelle ſich \label{K_L01195_3v}\edtext{im Original}{\lemma{\textnormal{\emph{im Original}}}\Cendnote{\textnormal{»Car pourquoi se vendrait-il? Ça ne lui rapporterait
                     jamais autant que d’être incorruptible.« Alfred Capus\pwindex{Capus, Alfred 25.11.1858 – 01.11.1922@\textsc{Capus, Alfred} (25.11.1858 – 01.11.1922), \emph{Schriftsteller, Journalist}|pwk}: \emph{La veine. Comédie en quatre actes}\pwindex{Capus, Alfred 25.11.1858 – 01.11.1922@\textsc{Capus, Alfred} (25.11.1858 – 01.11.1922), \emph{Schriftsteller, Journalist}!Glueck1901@\strich\emph{Das Glück} {[}1901{]}|pwk}. Paris: \emph{Éditions de la
                        Revue Blanche}{ }{[}1901?{]}, S. 149 (III, 9).}}}\label{K_L01195_3h} fände; – Kraus\pwindex{Kraus, Karl 28.04.1874 – 12.06.1936@\textsc{Kraus, Karl} (28.04.1874 – 12.06.1936), \emph{Schriftsteller, Publizist}|pw}{ }ſoll es auch zugeſagt \strikeout{\textcolor{gray}{haben}, aber} bisher nicht {\pb}gethan haben. – \pend
           \pstart
           Heute war Generalprobe der Lebendigen Stunden\pwindex{Schnitzler, Arthur 15.05.1862 – 21.10.1931@\textsc{Schnitzler, Arthur} (15.05.1862 – 21.10.1931), \emph{Schriftsteller, Mediziner}!Lebendige Stunden. Vier Einakter1901-12-23@\strich\emph{Lebendige Stunden. Vier Einakter} {[}1901-12-23{]}|pw}. Sie
               fiel günſtig – für abergläubiſch\damage{e} Gemüther zu günſtig \substVorne{}\textsuperscript{\textcolor{gray}{ohne}}\substDazwischen{}aus\substHinten{}. –\pend
           \pstart
           Ganz entzückt bin ich von \textsc{Bassermann}\pwindex{Bassermann, Albert 07.09.1867 – 15.05.1952@\textsc{Bassermann, Albert} (07.09.1867 – 15.05.1952), \emph{Schauspieler}|pw}. \label{K_L01195_4v}\edtext{Neulich ſah ich ihn als \textsc{H\introOben{}j\introOben{}a\substVorne{}\textsuperscript{\textcolor{gray}{jm}}\substDazwischen{}lm\substHinten{}ar\pwindex{Ibsen, Henrik 20.03.1828 – 23.05.1906@\textsc{Ibsen, Henrik} (20.03.1828 – 23.05.1906), \emph{Schriftsteller}!Wildente1884@\strich\emph{Die Wildente} {[}1884{]}|pwv}}, \textsc{Sauer}\pwindex{Sauer, Oskar 05.12.1856 – 03.04.1918@\textsc{Sauer, Oskar} (05.12.1856 – 03.04.1918), \emph{Schauspieler}|pw} als \textsc{Gregers Werle}\pwindex{Ibsen, Henrik 20.03.1828 – 23.05.1906@\textsc{Ibsen, Henrik} (20.03.1828 – 23.05.1906), \emph{Schriftsteller}!Wildente1884@\strich\emph{Die Wildente} {[}1884{]}|pwv}}{\lemma{\textnormal{\emph{Neulich … Werle}}}\Cendnote{\textnormal{Am 30. 12. 1901 spielte er
                  in Ibsens\pwindex{Ibsen, Henrik 20.03.1828 – 23.05.1906@\textsc{Ibsen, Henrik} (20.03.1828 – 23.05.1906), \emph{Schriftsteller}|pwk}{ }\emph{Wildente}\pwindex{Ibsen, Henrik 20.03.1828 – 23.05.1906@\textsc{Ibsen, Henrik} (20.03.1828 – 23.05.1906), \emph{Schriftsteller}!Wildente1884@\strich\emph{Die Wildente} {[}1884{]}|pwk}.}}}\label{K_L01195_4h}; ich habe ſelten ſo ſtarke
               ſchauſpieleriſche Eindrücke erlebt. Die Trieſch\pwindex{Triesch, Irene 13.04.1877 – 24.11.1964@\textsc{Triesch, Irene} (13.04.1877 – 24.11.1964), \emph{Schauspielerin}|pw}{ }{\pb}kann überraſchend
               viel. –\pend
           \pstart
           – Ich ſeh dich hoffentlich bald wieder. Herzlichen Gruſs. Dein \pend
           \pstart \spacefill\mbox{Arth Sch}\pend{}          \endnumbering\briefempfaengerindex{Bahr, Hermann@\textsc{Bahr, Hermann}!zzzSchnitzler, Arthur@\emph{von Arthur Schnitzler}!1902-01-031@{3. 1. 1902}|)be}\mylabel{h}\end{ledgroupsized}  \newcommand{\dateiname}{L01195}\newcommand{\titel}{Arthur Schnitzler an Hermann Bahr, 3. 1. 1902}\newcommand{\editorInnen}{ Kurt Ifkovits,  Martin Anton Müller}
            \footnotesize
\begin{ledgroupsized}[t]{11.5cm}
\doendnotes{C}
\end{ledgroupsized}
         %% latex-leseansicht-abspann.tex
%% Abspann für die Leseansicht.
%% Der Schalter \ifkorrekturansicht ist bereits durch den Vorspann gesetzt.

%% latex-abspann.tex
%% Gemeinsamer Abspann für Korrekturansicht und Leseansicht.
%% Setzt den Schalter \ifkorrekturansicht voraus (gesetzt in den
%% einbindenden Dateien latex-korrekturansicht-abspann.tex bzw.
%% latex-leseansicht-abspann.tex).
%% ---------------------------------------------------------------

\normalsize

% Das esempio-Environment wird nur in der Leseansicht benötigt
\ifkorrekturansicht\else
\newenvironment{esempio}[3]%
{
    \vspace{1.5ex}
    \rlap{\underline{#1}}
    \par
    \setlength{\parindent}{0cm}
    \nopagebreak
    \leftskip=#2cm
    \rightskip=#3cm
}
{
    \par
}
\fi

\doendnotes{C}
\bigskip
\vfill

\clearpage

\footnotesize

\ifkorrekturansicht
  \lohead{\textsc{register}}
\fi

% theindex-Environment neu definieren ohne reledmac
\makeatletter
\renewenvironment{theindex}{%
  \ifkorrekturansicht
    \section*{\indexname}%
  \else
    \subsubsection*{Index der erwähnten Entitäten}%
  \fi
  \setlength{\parindent}{0pt}%
  \setlength{\parskip}{0pt plus 0.3pt}%
  \let\item\@idxitem
}{%
  \ifkorrekturansicht\clearpage\fi
}
\makeatother

\IfFileExists{\jobname-pw.ind}{\input{\jobname-pw.ind}}{}

% Quellenangabe nur in der Leseansicht
\ifkorrekturansicht\else
% Fallback-Definitionen, falls die .tex-Datei \titel etc. nicht gesetzt hat
\providecommand{\titel}{}
\providecommand{\editorInnen}{}
\providecommand{\dateiname}{\jobname}

\vspace{3cm}

\vfill

\footnotesize
\textsc{Quelle}: \titel. Herausgegeben von {\editorInnen}. In: \emph{Arthur Schnitzler: Briefwechsel mit Autorinnen und Autoren}.
 Digitale Edition, https://schnitzler-briefe.acdh.oeaw.ac.at/{\dateiname}.html (Stand \today)
\fi

\end{document}


      