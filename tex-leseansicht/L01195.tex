%% latex-korrekturansicht-vorspann.tex
%% Vorspann für die Korrekturansicht.
%% Lädt die gemeinsame Datei latex-vorspann.tex mit gesetztem Schalter.

\newif\ifkorrekturansicht
\korrekturansichttrue

\input{../tex-inputs/latex-vorspann}


\section[Arthur Schnitzler an Hermann Bahr, 3. 1. 1902]{L01195 Arthur Schnitzler an Hermann Bahr, 3. 1. 1902}
\nopagebreak\mylabel{L01195v}
\rehead{ }\normalsize\beginnumbering\briefempfaengerindex{Bahr, Hermann@\textsc{Bahr, Hermann}!zzzSchnitzler, Arthur@\emph{von Arthur Schnitzler}!1902-01-031@{3. 1. 1902}|(be}
\toendnotes[C]{\smallbreak\pagebreak[2]}\Standort{TMW, HS AM 23348 Ba.}
\physDesc{Brief, 2 Blätter, 7 Seiten, 2056 Zeichen
\newline{}Handschrift: Bleistift, deutsche Kurrent
\newline{}Ordnung: Lochung }
\buchAbdrucke{\weitereDrucke{1) Arthur Schnitzler: \emph{The Letters of Arthur Schnitzler to Hermann Bahr}. Chapel Hill: \emph{The University of North Carolina Press} 1978, S. 73–74.} \weitereDrucke{2) Hermann Bahr, Arthur Schnitzler: \emph{Briefwechsel, Aufzeichnungen, Dokumente (1891–1931)}. Göttingen: \emph{Wallstein} 2018, S. 222–223.} }\toendnotes[C]{\smallbreak}
\pstart
           \raggedleft{}{\pb}3. 1. 902{\\}\textsc{Berlin}\oindex{Berlin@\textbf{Berlin}, \emph{P.PPLC}|pw}\pend
           \vspace{0.5em}
\pstart
           lieber Hermann, ich habe Brahm\pwindex{Brahm, Otto 05.02.1856 – 28.11.1912@\textsc{Brahm, Otto} (05.02.1856 – 28.11.1912), \emph{Theaterleiter/Theaterleiterin, Regisseur/Regisseurin}|pw} geſprochen, er äußerte ſich anerkennend über den Krampus\pwindex{Krampus. Lustspiel in drei Aufzuegen@\emph{Der Krampus. Lustspiel in drei Aufzügen}|pw}, findet nur, daſs gerade das Deutſche Theater\oindex{Deutsches Theater Berlin@\textbf{Deutsches Theater Berlin}, \emph{Theater (K.THE)}|pw} nicht der rechte Boden für das Stück sei. Ich glaube alſo
               nicht, daſs er zu der Aufführg nach Hamburg\oindex{Hamburg@\textbf{Hamburg}, \emph{P.PPLA}|pw}
               fahren wird, hielte es aber doch für ganz gut, we{\geminationn} du
               ihn unverbindlich mit ein paar {\pb}Worten dazu einladen
               möchteſt. Gegen deine Bemerkung über den literar. Stempel, den doch erſt das Deutſche Theater\oindex{Deutsches Theater Berlin@\textbf{Deutsches Theater Berlin}, \emph{Theater (K.THE)}|pw} verleihe (die ihm mitzutheilen ich
               mich wohl für befugt halten durfte?) ſchien er nicht unempfindlich zu ſein, und ich
               zweifle nicht daran, daſs er deine nächſten Stücke ohne vorgefaſſte Meinung leſen
               wird. Ich bin übrigens mor{\pb}gen Nachmittag bei ihm
               und habe ſicher Gelegenheit, nochmals in deinem Sinne zu reden. Er gehört doch, bei
               allen Begrenztheiten und Eigenſinnigkeiten zu den weitaus verſtändigſten
               Theatermenſchen \introOben{}(vielleicht auch Menſchen ſchlichtweg –)\introOben{},
               die es gibt, und iſt derjenige, mit dem man am gradlinigſten und verläßlichſten
               verkehren kann. Man darf von ihm ſagen, daſs {\pb}er nie lügt. Du
               sollteſt dich einmal perſönlich mit ihm ausſprechen. We{\geminationn}
               er nicht nach Hamburg\oindex{Hamburg@\textbf{Hamburg}, \emph{P.PPLA}|pw} ko{\geminationm}t, vielleicht beſuchst du ihn auf der Hin- oder
               Rückfahrt? – \pend
           
\pstart
           \label{K_L01195-1v}\edtext{Dieſer Tage}{\lemma{\textnormal{\emph{Dieſer Tage}}}\Cendnote{\textnormal{Vgl. A. S.: \emph{Tagebuch}, 1. 1. 1902.
               }}}\label{K_L01195-1}{ }ſprach ich \textsc{Harden\pwindex{Harden, Maximilian 20.10.1861 – 30.10.1927@\textsc{Harden, Maximilian} (20.10.1861 – 30.10.1927), \emph{Schriftsteller/Schriftstellerin, Publizist/Publizistin}|pw}}\damage{,} der jetzt ſehr gegen den kleinen Kraus\pwindex{Kraus, Karl 28.04.1874 – 12.06.1936@\textsc{Kraus, Karl} (28.04.1874 – 12.06.1936), \emph{Schriftsteller/Schriftstellerin, Publizist/Publizistin, Schriftsteller/Schriftstellerin}|pw} eingeno{\geminationm}en iſt und findet, daſs ein
               ſolches Blatt\pwindex{Fackel@\emph{Die Fackel}|pwv} in Berlin\oindex{Berlin@\textbf{Berlin}, \emph{P.PPLC}|pw}{ }ſich nicht halten kö{\geminationn}te. {\pb}Anläßlich der
                  \label{K_L01195-2v}\edtext{Krausiſchen Kritik\pwindex{Wie mich Herr Bahr beneidet]@\emph{[Wie mich Herr Bahr beneidet]}|pwv} über die
                  \textsc{veine\pwindex{Glueck@\emph{Das Glück}|pw}}}{\lemma{\textnormal{\emph{Krausiſchen … veine}}}\Cendnote{\textnormal{Kraus\pwindex{Kraus, Karl 28.04.1874 – 12.06.1936@\textsc{Kraus, Karl} (28.04.1874 – 12.06.1936), \emph{Schriftsteller/Schriftstellerin, Publizist/Publizistin, Schriftsteller/Schriftstellerin}|pwk}{ }schreibt in der \emph{Fackel}\pwindex{Fackel@\emph{Die Fackel}|pwk} (Bd. 10, H. 82, Anfang October, S. 19\pwindex{Wie mich Herr Bahr beneidet]@\emph{[Wie mich Herr Bahr beneidet]}|pwkv}): »Herr Bahr\pwindex{Bahr, Hermann 19.07.1863 – 15.01.1934@\textsc{Bahr, Hermann} (19.07.1863 – 15.01.1934), \emph{Schriftsteller/Schriftstellerin, Kritiker/Kritikerin}|pw}, der wiederum das
                     Referat über das Deutsche Volkstheater\oindex{Volkstheater@\textbf{Volkstheater}, \emph{Theater (K.THE)}|pw}
                     übernommen hat, berichtet, dass in dem neuen Stücke\pwindex{Glueck@\emph{Das Glück}|pwv} von Capus\pwindex{Capus, Alfred 25.11.1858 – 01.11.1922@\textsc{Capus, Alfred} (25.11.1858 – 01.11.1922), \emph{Schriftsteller/Schriftstellerin, Journalist/Journalistin}|pw} ein ›mit zwei Strichen wunderbar gezeichneter‹ Journalist
                     vorkomme, der sich nicht verkauft, weil ›ihm das nie so viel tragen kann wie
                     seine Unbestechlichkeit‹. Man versichert mir – ich kann die Mittheilung leider
                     nicht überprüfen –, dass diese Stelle, die Herr Bahr\pwindex{Bahr, Hermann 19.07.1863 – 15.01.1934@\textsc{Bahr, Hermann} (19.07.1863 – 15.01.1934), \emph{Schriftsteller/Schriftstellerin, Kritiker/Kritikerin}|pw} mit so munterem Behagen citiert, nachträglich in die
                     Uebersetzung der französischen\oindex{Frankreich@\textbf{Frankreich}, \emph{A.PCLI}|pw} Comödie
                     hineingeflickt worden sei und dass Herr Bahr\pwindex{Bahr, Hermann 19.07.1863 – 15.01.1934@\textsc{Bahr, Hermann} (19.07.1863 – 15.01.1934), \emph{Schriftsteller/Schriftstellerin, Kritiker/Kritikerin}|pw}{ }sich selbst citiere.« Bahrs\pwindex{Bahr, Hermann 19.07.1863 – 15.01.1934@\textsc{Bahr, Hermann} (19.07.1863 – 15.01.1934), \emph{Schriftsteller/Schriftstellerin, Kritiker/Kritikerin}|pwk} Besprechung, in der sich das Zitat
                  findet: \emph{Das Glück. (La veine. Komödie in vier Aufzügen
                        von Alfred Capus\pwindex{Capus, Alfred 25.11.1858 – 01.11.1922@\textsc{Capus, Alfred} (25.11.1858 – 01.11.1922), \emph{Schriftsteller/Schriftstellerin, Journalist/Journalistin}|pwk}. Deutsch von Theodor Wolff\pwindex{Wolff, Theodor 1868-08-02 – 1943-09-23@\textsc{Wolff, Theodor} (1868-08-02 – 1943-09-23), \emph{Schriftsteller/Schriftstellerin, Journalist/Journalistin}|pwk}. Zum erstenmal
                        aufgeführt im Deutschen Volkstheater\oindex{Volkstheater@\textbf{Volkstheater}, \emph{Theater (K.THE)}|pwk} am
                           28. September 1901)}\pwindex{Glueck. (La veine. Komoedie in vier Aufzuegen von Alfred Capus. Deutsch von Theodor Wolff. Zum ersten Mal aufgefuehrt im Deutschen Volkstheater am 28. September 1901)@\emph{Das Glück. (La veine. Komödie in vier Aufzügen von Alfred Capus. Deutsch von Theodor Wolff. Zum ersten Mal aufgeführt im Deutschen Volkstheater am 28. September 1901)}|pwk}. In: \emph{Neues Wiener Tagblatt}\pwindex{Neues Wiener Tagblatt@\emph{Neues Wiener Tagblatt}|pwk}, Jg. 35, Nr. 267,
                        29. 9. 1901, S. 2–4.}}}\label{K_L01195-2}, in der Kr.\pwindex{Kraus, Karl 28.04.1874 – 12.06.1936@\textsc{Kraus, Karl} (28.04.1874 – 12.06.1936), \emph{Schriftsteller/Schriftstellerin, Publizist/Publizistin, Schriftsteller/Schriftstellerin}|pw} von einer angeblich extra von dir \introOben{}(?)\introOben{} gegen ihn hineingedichteten Stelle erzählte, hat er ihm (\textsc{Harden\pwindex{Harden, Maximilian 20.10.1861 – 30.10.1927@\textsc{Harden, Maximilian} (20.10.1861 – 30.10.1927), \emph{Schriftsteller/Schriftstellerin, Publizist/Publizistin}|pw}} dem Kraus\pwindex{Kraus, Karl 28.04.1874 – 12.06.1936@\textsc{Kraus, Karl} (28.04.1874 – 12.06.1936), \emph{Schriftsteller/Schriftstellerin, Publizist/Publizistin, Schriftsteller/Schriftstellerin}|pw}) eine Karte geſchrieben, er
               müſſe gelegentlich diesen Irrthum richtigſtellen, da die betreffende Stelle ſich
                  \label{K_L01195-3v}\edtext{im Original}{\lemma{\textnormal{\emph{im Original}}}\Cendnote{\textnormal{»Car pourquoi se vendrait-il? Ça ne
                     lui rapporterait jamais autant que d’être incorruptible.« Alfred Capus\pwindex{Capus, Alfred 25.11.1858 – 01.11.1922@\textsc{Capus, Alfred} (25.11.1858 – 01.11.1922), \emph{Schriftsteller/Schriftstellerin, Journalist/Journalistin}|pwk}: \emph{La veine. Comédie en quatre actes}\pwindex{Glueck@\emph{Das Glück}|pwk}. Paris:
                        \emph{Éditions de la Revue Blanche}{ }{[}1901?{]}, S. 149 (III, 9).}}}\label{K_L01195-3} fände; – Kraus\pwindex{Kraus, Karl 28.04.1874 – 12.06.1936@\textsc{Kraus, Karl} (28.04.1874 – 12.06.1936), \emph{Schriftsteller/Schriftstellerin, Publizist/Publizistin, Schriftsteller/Schriftstellerin}|pw}{ }ſoll es auch zugeſagt \strikeout{\textcolor{gray}{haben}, aber} bisher nicht {\pb}gethan haben. – \pend
           
\pstart
           Heute war Generalprobe der Lebendigen Stunden\pwindex{Lebendige Stunden. Vier Einakter@\emph{Lebendige Stunden. Vier Einakter}|pw}.
               Sie fiel günſtig – für abergläubiſch\damage{e} Gemüther zu günſtig \substVorne{}\textsuperscript{\textcolor{gray}{ohne}}\substDazwischen{}aus\substHinten{}. –\pend
           
\pstart
           Ganz entzückt bin ich von \textsc{Bassermann}\pwindex{Bassermann, Albert 07.09.1867 – 15.05.1952@\textsc{Bassermann, Albert} (07.09.1867 – 15.05.1952), \emph{Schauspieler/Schauspielerin}|pw}. \label{K_L01195-4v}\edtext{Neulich ſah ich ihn als \textsc{H\introOben{}j\introOben{}a\substVorne{}\textsuperscript{\textcolor{gray}{jm}}\substDazwischen{}lm\substHinten{}ar\pwindex{Wildente. Schauspiel in fuenf Akten@\emph{Die Wildente. Schauspiel in fünf Akten}|pwv}}, \textsc{Sauer}\pwindex{Sauer, Oskar 05.12.1856 – 03.04.1918@\textsc{Sauer, Oskar} (05.12.1856 – 03.04.1918), \emph{Schauspieler/Schauspielerin}|pw} als \textsc{Gregers Werle}\pwindex{Wildente. Schauspiel in fuenf Akten@\emph{Die Wildente. Schauspiel in fünf Akten}|pwv}}{\lemma{\textnormal{\emph{Neulich … Werle}}}\Cendnote{\textnormal{Am 30. 12. 1901 spielte er
                  in Ibsens\pwindex{Ibsen, Henrik 20.03.1828 – 23.05.1906@\textsc{Ibsen, Henrik} (20.03.1828 – 23.05.1906), \emph{Schriftsteller/Schriftstellerin}|pwk}{ }\emph{Wildente}\pwindex{Wildente. Schauspiel in fuenf Akten@\emph{Die Wildente. Schauspiel in fünf Akten}|pwk}.}}}\label{K_L01195-4}; ich habe ſelten ſo ſtarke
               ſchauſpieleriſche Eindrücke erlebt. Die Trieſch\pwindex{Triesch, Irene 13.04.1877 – 24.11.1964@\textsc{Triesch, Irene} (13.04.1877 – 24.11.1964), \emph{Schauspieler/Schauspielerin}|pw}{ }{\pb}kann überraſchend
               viel. –\pend
           
\pstart
           – Ich ſeh dich hoffentlich bald wieder. Herzlichen Gruſs. Dein \pend
           \pstart \spacefill\mbox{Arth Sch}\pend{}\selectlanguage{ngerman}\endnumbering\briefempfaengerindex{Bahr, Hermann@\textsc{Bahr, Hermann}!zzzSchnitzler, Arthur@\emph{von Arthur Schnitzler}!1902-01-031@{3. 1. 1902}|)be}\mylabel{L01195h}  \normalsize

\doendnotes{C}
\bigskip
\vfill

\clearpage

\footnotesize

\lohead{\textsc{register}}

% Definiere theindex-Environment komplett neu ohne reledmac
\makeatletter
\renewenvironment{theindex}{%
  \section*{\indexname}%
  \setlength{\parindent}{0pt}%
  \setlength{\parskip}{0pt plus 0.3pt}%
  \let\item\@idxitem
}{%
  \clearpage
}
\makeatother

\IfFileExists{\jobname-pw.ind}{\input{\jobname-pw.ind}}{}

\end{document}

      