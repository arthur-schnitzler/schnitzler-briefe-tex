%% latex-leseansicht-vorspann.tex
%% Vorspann für die Leseansicht.
%% Lädt die gemeinsame Datei latex-vorspann.tex mit nicht gesetztem Schalter.

\newif\ifkorrekturansicht
\korrekturansichtfalse

\input{../tex-inputs/latex-vorspann}


\section[Arthur Schnitzler an Hermann Bahr, 3. 1. 1902]{L01195 Arthur Schnitzler an Hermann Bahr, 3. 1. 1902}
\nopagebreak\mylabel{L01195v}
\rehead{ }\normalsize\beginnumbering\briefempfaengerindex{Bahr, Hermann@\textsc{Bahr, Hermann}!zzzSchnitzler, Arthur@\emph{von Arthur Schnitzler}!1902-01-031@{3. 1. 1902}|(be}
\toendnotes[C]{\smallbreak\pagebreak[2]}
\correspDesc{Versand  durch Arthur Schnitzler am 3. 1. 1902 in Berlin
\newline{}Erhalt  durch Hermann Bahr im Zeitraum [4. 1. 1902
                  – 8. 1. 1902?] in Wien}\toendnotes[C]{\smallbreak}
\Standort{TMW, HS AM 23348 Ba.}
\physDesc{Brief, 2 Blätter, 7 Seiten, 2056 Zeichen
\newline{}Handschrift: Bleistift, deutsche Kurrent
\newline{}Ordnung: Lochung }
\buchAbdrucke{\weitereDrucke{1) \emph{3. 1. 1902.} In: Arthur Schnitzler: \emph{The Letters of Arthur Schnitzler to Hermann Bahr}. Edited, annotated, and with an introduction, by Donald G. Daviau. Chapel Hill: \emph{The University of North Carolina Press} 1978, S. 73–74 (University of North Carolina studies in the Germanic languages
                        and literatures, 89).} \weitereDrucke{2) Hermann Bahr, Arthur Schnitzler: \emph{Briefwechsel, Aufzeichnungen, Dokumente (1891–1931)}. Herausgegeben von Kurt Ifkovits und Martin Anton Müller. Göttingen: \emph{Wallstein} 2018, S. 222–223.} }\toendnotes[C]{\smallbreak}
\pstart
           \raggedleft{}{\pb}3. 1. 902{\\}\textsc{Berlin}\oindex{Berlin@\textbf{Berlin}, \emph{Hauptstadt}|pw}\pend
           \vspace{0.5em}
\pstart
           lieber Hermann, ich habe Brahm\pwindex{Brahm, Otto 5.\,2.\,1856 Hamburg – 28.\,11.\,1912 Berlin@\textsc{Brahm, Otto} (5.\,2.\,1856 Hamburg – 28.\,11.\,1912 Berlin), \emph{Theaterleiter, Regisseur}|pw} geſprochen, er äußerte{ }ſich anerkennend über den Krampus\pwindex{Bahr, Hermann 19.\,7.\,1863 Linz – 15.\,1.\,1934 München@\textsc{Bahr, Hermann} (19.\,7.\,1863 Linz – 15.\,1.\,1934 München), \emph{Schriftsteller, Kritiker}!Krampus. Lustspiel in drei Aufzügen@\strich\emph{Der Krampus. Lustspiel in drei Aufzügen}|pw}, findet nur, daſs gerade das Deutſche Theater\oindex{Deutsches Theater Berlin@\textbf{Deutsches Theater Berlin}, \emph{Theater}|pw} nicht der rechte Boden für das Stück sei. Ich glaube alſo
               nicht, daſs er zu der Aufführg nach Hamburg\oindex{Hamburg@\textbf{Hamburg}|pw}
               fahren wird, hielte es aber doch für ganz gut, we{\geminationn} du
               ihn unverbindlich mit ein paar {\pb}Worten dazu einladen
               möchteſt. Gegen deine Bemerkung über den literar. Stempel, den doch erſt das Deutſche Theater\oindex{Deutsches Theater Berlin@\textbf{Deutsches Theater Berlin}, \emph{Theater}|pw} verleihe (die ihm mitzutheilen ich
               mich wohl für befugt halten durfte?){ }ſchien er nicht unempfindlich zu{ }ſein, und ich
               zweifle nicht daran, daſs er deine nächſten Stücke ohne vorgefaſſte Meinung leſen
               wird. Ich bin übrigens mor{\pb}gen Nachmittag bei ihm
               und habe{ }ſicher Gelegenheit, nochmals in deinem Sinne zu reden. Er gehört doch, bei
               allen Begrenztheiten und Eigenſinnigkeiten zu den weitaus verſtändigſten
               Theatermenſchen \introOben{}(vielleicht auch Menſchen{ }ſchlichtweg –)\introOben{},
               die es gibt, und iſt derjenige, mit dem man am gradlinigſten und verläßlichſten
               verkehren kann. Man darf von ihm{ }ſagen, daſs {\pb}er nie lügt. Du
               sollteſt dich einmal perſönlich mit ihm ausſprechen. We{\geminationn}
               er nicht nach Hamburg\oindex{Hamburg@\textbf{Hamburg}|pw} ko{\geminationm}t, vielleicht beſuchst du ihn auf der Hin- oder
               Rückfahrt? –\pend
           
\pstart
           \label{K_L01195-1v}\edtext{Dieſer Tage}{\lemma{\textnormal{\emph{Dieser Tage}}}\Cendnote{\textnormal{Vgl. A. S.: \emph{Tagebuch}, 1. 1. 1902.
               }}}\label{K_L01195-1}{ }ſprach ich \textsc{Harden\pwindex{Harden, Maximilian 20.\,10.\,1861 Berlin – 30.\,10.\,1927 Montana@\textsc{Harden, Maximilian} (20.\,10.\,1861 Berlin – 30.\,10.\,1927 Montana), \emph{Schriftsteller, Publizist}|pw}}\damage{,} der jetzt{ }ſehr gegen den kleinen Kraus\pwindex{Kraus, Karl 28.\,4.\,1874 Jičín – 12.\,6.\,1936 Wien@\textsc{Kraus, Karl} (28.\,4.\,1874 Jičín – 12.\,6.\,1936 Wien), \emph{Schriftsteller, Publizist, Schriftsteller}|pw} eingeno{\geminationm}en iſt und findet, daſs ein{ }ſolches Blatt\pwindex{Fackel@\emph{Die Fackel}|pwv} in Berlin\oindex{Berlin@\textbf{Berlin}, \emph{Hauptstadt}|pw}{ }ſich nicht halten kö{\geminationn}te. {\pb}Anläßlich der
                  \label{K_L01195-2v}\edtext{Krausiſchen Kritik\pwindex{Kraus, Karl 28.\,4.\,1874 Jičín – 12.\,6.\,1936 Wien@\textsc{Kraus, Karl} (28.\,4.\,1874 Jičín – 12.\,6.\,1936 Wien), \emph{Schriftsteller, Publizist, Schriftsteller}!Wie mich Herr Bahr beneidet]@\strich\emph{[Wie mich Herr Bahr beneidet]}|pwv} über die
                  \textsc{veine\pwindex{Capus, Alfred 25.\,11.\,1858 Aix-en-Provence – 1.\,11.\,1922 Neuilly-sur-Seine@\textsc{Capus, Alfred} (25.\,11.\,1858 Aix-en-Provence – 1.\,11.\,1922 Neuilly-sur-Seine), \emph{Schriftsteller, Journalist}!Glück@\strich\emph{Das Glück}|pw}}}{\lemma{\textnormal{\emph{Krausischen … veine}}}\Cendnote{\textnormal{Kraus\pwindex{Kraus, Karl 28.\,4.\,1874 Jičín – 12.\,6.\,1936 Wien@\textsc{Kraus, Karl} (28.\,4.\,1874 Jičín – 12.\,6.\,1936 Wien), \emph{Schriftsteller, Publizist, Schriftsteller}|pwk}{ }schreibt in der \emph{Fackel}\pwindex{Fackel@\emph{Die Fackel}|pwk} (Bd. 10, H. 82, Anfang October, S. 19\pwindex{Kraus, Karl 28.\,4.\,1874 Jičín – 12.\,6.\,1936 Wien@\textsc{Kraus, Karl} (28.\,4.\,1874 Jičín – 12.\,6.\,1936 Wien), \emph{Schriftsteller, Publizist, Schriftsteller}!Wie mich Herr Bahr beneidet]@\strich\emph{[Wie mich Herr Bahr beneidet]}|pwkv}): »Herr Bahr\pwindex{Bahr, Hermann 19.\,7.\,1863 Linz – 15.\,1.\,1934 München@\textsc{Bahr, Hermann} (19.\,7.\,1863 Linz – 15.\,1.\,1934 München), \emph{Schriftsteller, Kritiker}|pw}, der wiederum das
                     Referat über das Deutsche Volkstheater\oindex{Wien@\textbf{Wien}!VII., Neubau@\textbf{VII., Neubau}!Volkstheater@\textbf{Volkstheater}, \emph{Theater}|pw}
                     übernommen hat, berichtet, dass in dem neuen Stücke\pwindex{Capus, Alfred 25.\,11.\,1858 Aix-en-Provence – 1.\,11.\,1922 Neuilly-sur-Seine@\textsc{Capus, Alfred} (25.\,11.\,1858 Aix-en-Provence – 1.\,11.\,1922 Neuilly-sur-Seine), \emph{Schriftsteller, Journalist}!Glück@\strich\emph{Das Glück}|pwv} von Capus\pwindex{Capus, Alfred 25.\,11.\,1858 Aix-en-Provence – 1.\,11.\,1922 Neuilly-sur-Seine@\textsc{Capus, Alfred} (25.\,11.\,1858 Aix-en-Provence – 1.\,11.\,1922 Neuilly-sur-Seine), \emph{Schriftsteller, Journalist}|pw} ein ›mit zwei Strichen wunderbar gezeichneter‹ Journalist
                     vorkomme, der sich nicht verkauft, weil ›ihm das nie so viel tragen kann wie
                     seine Unbestechlichkeit‹. Man versichert mir – ich kann die Mittheilung leider
                     nicht überprüfen –, dass diese Stelle, die Herr Bahr\pwindex{Bahr, Hermann 19.\,7.\,1863 Linz – 15.\,1.\,1934 München@\textsc{Bahr, Hermann} (19.\,7.\,1863 Linz – 15.\,1.\,1934 München), \emph{Schriftsteller, Kritiker}|pw} mit so munterem Behagen citiert, nachträglich in die
                     Uebersetzung der französischen\oindex{Frankreich@\textbf{Frankreich}|pw} Comödie
                     hineingeflickt worden sei und dass Herr Bahr\pwindex{Bahr, Hermann 19.\,7.\,1863 Linz – 15.\,1.\,1934 München@\textsc{Bahr, Hermann} (19.\,7.\,1863 Linz – 15.\,1.\,1934 München), \emph{Schriftsteller, Kritiker}|pw}{ }sich selbst citiere.« Bahrs\pwindex{Bahr, Hermann 19.\,7.\,1863 Linz – 15.\,1.\,1934 München@\textsc{Bahr, Hermann} (19.\,7.\,1863 Linz – 15.\,1.\,1934 München), \emph{Schriftsteller, Kritiker}|pwk} Besprechung, in der sich das Zitat
                  findet: \emph{Das Glück. (La veine. Komödie in vier Aufzügen
                        von Alfred Capus\pwindex{Capus, Alfred 25.\,11.\,1858 Aix-en-Provence – 1.\,11.\,1922 Neuilly-sur-Seine@\textsc{Capus, Alfred} (25.\,11.\,1858 Aix-en-Provence – 1.\,11.\,1922 Neuilly-sur-Seine), \emph{Schriftsteller, Journalist}|pwk}. Deutsch von Theodor Wolff\pwindex{Wolff, Theodor 2.\,8.\,1868 Berlin – 23.\,9.\,1943 ebd.@\textsc{Wolff, Theodor} (2.\,8.\,1868 Berlin – 23.\,9.\,1943 ebd.), \emph{Schriftsteller, Journalist}|pwk}. Zum erstenmal
                        aufgeführt im Deutschen Volkstheater\oindex{Wien@\textbf{Wien}!VII., Neubau@\textbf{VII., Neubau}!Volkstheater@\textbf{Volkstheater}, \emph{Theater}|pwk} am
                           28. September 1901)}\pwindex{Bahr, Hermann 19.\,7.\,1863 Linz – 15.\,1.\,1934 München@\textsc{Bahr, Hermann} (19.\,7.\,1863 Linz – 15.\,1.\,1934 München), \emph{Schriftsteller, Kritiker}!Glück. (La veine. Komödie in vier Aufzügen von Alfred Capus. Deutsch von Theodor Wolff. Zum ersten Mal aufgeführt im Deutschen Volkstheater am 28. September 1901)@\strich\emph{Das Glück. (La veine. Komödie in vier Aufzügen von Alfred Capus. Deutsch von Theodor Wolff. Zum ersten Mal aufgeführt im Deutschen Volkstheater am 28. September 1901)}|pwk}. In: \emph{Neues Wiener Tagblatt}\pwindex{Neues Wiener Tagblatt@\emph{Neues Wiener Tagblatt}|pwk}, Jg. 35, Nr. 267,
                        29. 9. 1901, S. 2–4.}}}\label{K_L01195-2}, in der Kr.\pwindex{Kraus, Karl 28.\,4.\,1874 Jičín – 12.\,6.\,1936 Wien@\textsc{Kraus, Karl} (28.\,4.\,1874 Jičín – 12.\,6.\,1936 Wien), \emph{Schriftsteller, Publizist, Schriftsteller}|pw} von einer angeblich extra von dir \introOben{}(?)\introOben{} gegen ihn hineingedichteten Stelle erzählte, hat er ihm (\textsc{Harden\pwindex{Harden, Maximilian 20.\,10.\,1861 Berlin – 30.\,10.\,1927 Montana@\textsc{Harden, Maximilian} (20.\,10.\,1861 Berlin – 30.\,10.\,1927 Montana), \emph{Schriftsteller, Publizist}|pw}} dem Kraus\pwindex{Kraus, Karl 28.\,4.\,1874 Jičín – 12.\,6.\,1936 Wien@\textsc{Kraus, Karl} (28.\,4.\,1874 Jičín – 12.\,6.\,1936 Wien), \emph{Schriftsteller, Publizist, Schriftsteller}|pw}) eine Karte geſchrieben, er
               müſſe gelegentlich diesen Irrthum richtigſtellen, da die betreffende Stelle{ }ſich
                  \label{K_L01195-3v}\edtext{im Original}{\lemma{\textnormal{\emph{im Original}}}\Cendnote{\textnormal{»\begin{otherlanguage}{french}Car pourquoi se vendrait-il? Ça ne
                     lui rapporterait jamais autant que d’être incorruptible.\end{otherlanguage}« Alfred Capus\pwindex{Capus, Alfred 25.\,11.\,1858 Aix-en-Provence – 1.\,11.\,1922 Neuilly-sur-Seine@\textsc{Capus, Alfred} (25.\,11.\,1858 Aix-en-Provence – 1.\,11.\,1922 Neuilly-sur-Seine), \emph{Schriftsteller, Journalist}|pwk}: \emph{La veine. Comédie en quatre actes}\pwindex{Capus, Alfred 25.\,11.\,1858 Aix-en-Provence – 1.\,11.\,1922 Neuilly-sur-Seine@\textsc{Capus, Alfred} (25.\,11.\,1858 Aix-en-Provence – 1.\,11.\,1922 Neuilly-sur-Seine), \emph{Schriftsteller, Journalist}!Glück@\strich\emph{Das Glück}|pwk}. Paris:
                        \emph{Éditions de la Revue Blanche}{ }{[}1901?{]}, S. 149 (III, 9).}}}\label{K_L01195-3} fände; – Kraus\pwindex{Kraus, Karl 28.\,4.\,1874 Jičín – 12.\,6.\,1936 Wien@\textsc{Kraus, Karl} (28.\,4.\,1874 Jičín – 12.\,6.\,1936 Wien), \emph{Schriftsteller, Publizist, Schriftsteller}|pw}{ }ſoll es auch zugeſagt \strikeout{\textcolor{gray}{haben}, aber} bisher nicht {\pb}gethan haben. –\pend
           
\pstart
           Heute war Generalprobe der Lebendigen Stunden\pwindex{Schnitzler, Arthur 15.\,5.\,1862 Wien – 21.\,10.\,1931 ebd.@\textsc{Schnitzler, Arthur} (15.\,5.\,1862 Wien – 21.\,10.\,1931 ebd.), \emph{Schriftsteller, Mediziner}!Lebendige Stunden. Vier Einakter@\strich\emph{Lebendige Stunden. Vier Einakter}|pw}.
               Sie fiel günſtig – für abergläubiſch\damage{e} Gemüther zu günſtig \substVorne{}\textsuperscript{\textcolor{gray}{ohne}}\substDazwischen{}aus\substHinten{}. –\pend
           
\pstart
           Ganz entzückt bin ich von \textsc{Bassermann}\pwindex{Bassermann, Albert 7.\,9.\,1867 Mannheim – 15.\,5.\,1952 Atlantischer Ozean@\textsc{Bassermann, Albert} (7.\,9.\,1867 Mannheim – 15.\,5.\,1952 Atlantischer Ozean), \emph{Schauspieler}|pw}. \label{K_L01195-4v}\edtext{Neulich{ }ſah ich ihn als \textsc{H\introOben{}j\introOben{}a\substVorne{}\textsuperscript{\textcolor{gray}{jm}}\substDazwischen{}lm\substHinten{}ar\pwindex{Ibsen, Henrik 20.\,3.\,1828 Skien – 23.\,5.\,1906 Oslo@\textsc{Ibsen, Henrik} (20.\,3.\,1828 Skien – 23.\,5.\,1906 Oslo), \emph{Schriftsteller}!Wildente. Schauspiel in fünf Akten@\strich\emph{Die Wildente. Schauspiel in fünf Akten}|pwv}}, \textsc{Sauer}\pwindex{Sauer, Oskar 5.\,12.\,1856 Berlin – 3.\,4.\,1918 ebd.@\textsc{Sauer, Oskar} (5.\,12.\,1856 Berlin – 3.\,4.\,1918 ebd.), \emph{Schauspieler}|pw} als \textsc{Gregers Werle}\pwindex{Ibsen, Henrik 20.\,3.\,1828 Skien – 23.\,5.\,1906 Oslo@\textsc{Ibsen, Henrik} (20.\,3.\,1828 Skien – 23.\,5.\,1906 Oslo), \emph{Schriftsteller}!Wildente. Schauspiel in fünf Akten@\strich\emph{Die Wildente. Schauspiel in fünf Akten}|pwv}}{\lemma{\textnormal{\emph{Neulich … Werle}}}\Cendnote{\textnormal{Am 30. 12. 1901 spielte er
                  in Ibsens\pwindex{Ibsen, Henrik 20.\,3.\,1828 Skien – 23.\,5.\,1906 Oslo@\textsc{Ibsen, Henrik} (20.\,3.\,1828 Skien – 23.\,5.\,1906 Oslo), \emph{Schriftsteller}|pwk}{ }\emph{Wildente}\pwindex{Ibsen, Henrik 20.\,3.\,1828 Skien – 23.\,5.\,1906 Oslo@\textsc{Ibsen, Henrik} (20.\,3.\,1828 Skien – 23.\,5.\,1906 Oslo), \emph{Schriftsteller}!Wildente. Schauspiel in fünf Akten@\strich\emph{Die Wildente. Schauspiel in fünf Akten}|pwk}.}}}\label{K_L01195-4}; ich habe{ }ſelten{ }ſo{ }ſtarke{ }ſchauſpieleriſche Eindrücke erlebt. Die Trieſch\pwindex{Triesch, Irene 13.\,4.\,1877 Wien – 24.\,11.\,1964 Basel@\textsc{Triesch, Irene} (13.\,4.\,1877 Wien – 24.\,11.\,1964 Basel), \emph{Schauspielerin}|pw}{ }{\pb}kann überraſchend
               viel. –\pend
           
\pstart
           – Ich{ }ſeh dich hoffentlich bald wieder. Herzlichen Gruſs. Dein\pend
           \pstart \spacefill\mbox{Arth Sch}\pend{}\selectlanguage{ngerman}\endnumbering\briefempfaengerindex{Bahr, Hermann@\textsc{Bahr, Hermann}!zzzSchnitzler, Arthur@\emph{von Arthur Schnitzler}!1902-01-031@{3. 1. 1902}|)be}\mylabel{L01195h}  \newcommand{\dateiname}{L01195}\newcommand{\titel}{Arthur Schnitzler an Hermann Bahr, 3. 1. 1902}\newcommand{\editorInnen}{Herausgegeben von Martin Anton Müller}%% latex-leseansicht-abspann.tex
%% Abspann für die Leseansicht.
%% Der Schalter \ifkorrekturansicht ist bereits durch den Vorspann gesetzt.

%% latex-abspann.tex
%% Gemeinsamer Abspann für Korrekturansicht und Leseansicht.
%% Setzt den Schalter \ifkorrekturansicht voraus (gesetzt in den
%% einbindenden Dateien latex-korrekturansicht-abspann.tex bzw.
%% latex-leseansicht-abspann.tex).
%% ---------------------------------------------------------------

\normalsize

% Das esempio-Environment wird nur in der Leseansicht benötigt
\ifkorrekturansicht\else
\newenvironment{esempio}[3]%
{
    \vspace{1.5ex}
    \rlap{\underline{#1}}
    \par
    \setlength{\parindent}{0cm}
    \nopagebreak
    \leftskip=#2cm
    \rightskip=#3cm
}
{
    \par
}
\fi

\doendnotes{C}
\bigskip
\vfill

\clearpage

\footnotesize

\ifkorrekturansicht
  \lohead{\textsc{register}}
\fi

% theindex-Environment neu definieren ohne reledmac
\makeatletter
\renewenvironment{theindex}{%
  \ifkorrekturansicht
    \section*{\indexname}%
  \else
    \subsubsection*{Index der erwähnten Entitäten}%
  \fi
  \setlength{\parindent}{0pt}%
  \setlength{\parskip}{0pt plus 0.3pt}%
  \let\item\@idxitem
}{%
  \ifkorrekturansicht\clearpage\fi
}
\makeatother

\IfFileExists{\jobname-pw.ind}{\input{\jobname-pw.ind}}{}

% Quellenangabe nur in der Leseansicht
\ifkorrekturansicht\else
% Fallback-Definitionen, falls die .tex-Datei \titel etc. nicht gesetzt hat
\providecommand{\titel}{}
\providecommand{\editorInnen}{}
\providecommand{\dateiname}{\jobname}

\vspace{3cm}

\vfill

\footnotesize
\textsc{Quelle}: \titel. Herausgegeben von {\editorInnen}. In: \emph{Arthur Schnitzler: Briefwechsel mit Autorinnen und Autoren}.
 Digitale Edition, https://schnitzler-briefe.acdh.oeaw.ac.at/{\dateiname}.html (Stand \today)
\fi

\end{document}


