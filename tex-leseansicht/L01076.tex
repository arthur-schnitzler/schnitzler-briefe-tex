%% latex-korrekturansicht-vorspann.tex
%% Vorspann für die Korrekturansicht.
%% Lädt die gemeinsame Datei latex-vorspann.tex mit gesetztem Schalter.

\newif\ifkorrekturansicht
\korrekturansichttrue

\input{../tex-inputs/latex-vorspann}


\section[Arthur Schnitzler an Hermann Bahr, 11. 10. 1900]{L01076 Arthur Schnitzler an Hermann Bahr, 11. 10. 1900}
\nopagebreak\mylabel{L01076v}
\rehead{ }\normalsize\beginnumbering\briefempfaengerindex{Bahr, Hermann@\textsc{Bahr, Hermann}!zzzSchnitzler, Arthur@\emph{von Arthur Schnitzler}!1900-10-111@{11. 10. 1900}|(be}
\toendnotes[C]{\smallbreak\pagebreak[2]}\Standort{TMW, HS AM 60152 Ba.}
\physDesc{Briefkarte, 475 Zeichen
\newline{}Handschrift: schwarze Tinte, deutsche Kurrent
\newline{}Ordnung: Lochung }
\buchAbdrucke{\weitereDrucke{1) Arthur Schnitzler: \emph{The Letters of Arthur Schnitzler to Hermann Bahr}. Chapel Hill: \emph{The University of North Carolina Press} 1978, S. 66–67.} \weitereDrucke{2) Hermann Bahr, Arthur Schnitzler: \emph{Briefwechsel, Aufzeichnungen, Dokumente (1891–1931)}. Göttingen: \emph{Wallstein} 2018, S. 182.} }\toendnotes[C]{\smallbreak}
\pstart
           \noindent{}{\pb}Lieber Hermann, ich danke dir vielmals für den »\label{K_L01076-1v}\edtext{Franzl\pwindex{Franzl. Fuenf Bilder aus dem Leben eines guten Mannes@\emph{Der Franzl. Fünf Bilder aus dem Leben eines guten Mannes}|pw}}{\lemma{\textnormal{\emph{Franzl}}}\Cendnote{\textnormal{Hermann Bahr\pwindex{Bahr, Hermann 19.07.1863 – 15.01.1934@\textsc{Bahr, Hermann} (19.07.1863 – 15.01.1934), \emph{Schriftsteller/Schriftstellerin, Kritiker/Kritikerin}|pwk}: \emph{Der Franzl. Fünf Bilder eines guten Mannes}\pwindex{Franzl. Fuenf Bilder aus dem Leben eines guten Mannes@\emph{Der Franzl. Fünf Bilder aus dem Leben eines guten Mannes}|pwk}.}}}\label{K_L01076-1}«, den ich mir auf einen kurzen Landaufenthalt mitnehme, um ihn mit
               Muße u Vergnügen zu leſen. Ich will dich gleich was fragen. Im Sommer hab ich eine
               mäßig {\pb}lange Geſchichte\pwindex{Lieutenant Gustl. Novelle@\emph{Lieutenant Gustl. Novelle}|pwv} geſchrieben, die
               ſich ausnehmend zum Vorleſen eignet, und die niemand beſſer vorleſen könnte als du.
               Bevor ich dir das \textsc{Mscrpt\pwindex{Lieutenant Gustl. Novelle@\emph{Lieutenant Gustl. Novelle}|pwv}}{ }ſchicke (\textsc{typewritten})
               möchte ich nur dein \uline{principielles} Einverſtändnis
               haben. Herzlichen Gruß. Dein{\\}\spacefill\mbox{Arthur Schnitzler}\pend
           
\pstart
           \damage{1}1. 10. 900.\pend
           \selectlanguage{ngerman}\endnumbering\briefempfaengerindex{Bahr, Hermann@\textsc{Bahr, Hermann}!zzzSchnitzler, Arthur@\emph{von Arthur Schnitzler}!1900-10-111@{11. 10. 1900}|)be}\mylabel{L01076h}  \normalsize

\doendnotes{C}
\bigskip
\vfill

\clearpage

\footnotesize

\lohead{\textsc{register}}

% Definiere theindex-Environment komplett neu ohne reledmac
\makeatletter
\renewenvironment{theindex}{%
  \section*{\indexname}%
  \setlength{\parindent}{0pt}%
  \setlength{\parskip}{0pt plus 0.3pt}%
  \let\item\@idxitem
}{%
  \clearpage
}
\makeatother

\IfFileExists{\jobname-pw.ind}{\input{\jobname-pw.ind}}{}

\end{document}

      