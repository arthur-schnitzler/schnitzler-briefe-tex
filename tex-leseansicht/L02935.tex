%% latex-korrekturansicht-vorspann.tex
%% Vorspann für die Korrekturansicht.
%% Lädt die gemeinsame Datei latex-vorspann.tex mit gesetztem Schalter.

\newif\ifkorrekturansicht
\korrekturansichttrue

\input{../tex-inputs/latex-vorspann}


\section[ Paul Goldmann an Arthur Schnitzler, 5. 10. {[}1900{]}]{L02935 Paul Goldmann an Arthur Schnitzler, 5. 10. {[}1900{]}}
\nopagebreak\mylabel{L02935v}
\rehead{ }\normalsize\beginnumbering\briefempfaengerindex{Schnitzler, Arthur@\textsc{Schnitzler, Arthur}!zzzGoldmann, Paul@\emph{von Paul Goldmann}!1900-10-051@{5. 10. {[}1900{]}}|(be}
\toendnotes[C]{\smallbreak\pagebreak[2]}\Standort{DLA, A:Schnitzler, HS.NZ85.1.3170.}
\physDesc{Brief, 1 Blatt, 3 Seiten, 490 Zeichen
\newline{}Handschrift: blaue Tinte, deutsche Kurrent
\newline{}Beilage: ein aufgeklebter beschnittener Zeitungsausschnitt 
\newline{}Schnitzler: 1) mit Bleistift das Jahr »900.« vermerkt  2) mit rotem Buntstift zwei Unterstreichungen}\toendnotes[C]{\smallbreak}
\pstart
           {\pb}Berlin\oindex{Berlin@\textbf{Berlin}, \emph{P.PPLC}|pw}, 5. Oktober.\hfill \textcolor{gray}{\textbf{DESSAUERSTRASSE 19}}\oindex{Dessauer Strasse@\textbf{Dessauer Straße}, \emph{Straße (K.STR)}|pw}\pend
           
\pstart\center{}Mein lieber Freund,\pend\vspace{0.5em}
\pstart
           Ein Herr \textsc{Anton Reitler\pwindex{Reitler, Anton *~1856-10-31@\textsc{Reitler, Anton} (*~1856-10-31), \emph{Dichter/Dichterin, Jurist/Juristin}|pw}} (?) läßt ſich in einem \label{K_L02935-1v}\edtext{Wiener Briefe\pwindex{Wiener Leben@\emph{Wiener Leben}|pwv}}{\lemma{\textnormal{\emph{Wiener Briefe}}}\Cendnote{\textnormal{Anton Reitler\pwindex{Reitler, Anton *~1856-10-31@\textsc{Reitler, Anton} (*~1856-10-31), \emph{Dichter/Dichterin, Jurist/Juristin}|pwk}: \emph{Wiener Leben}\pwindex{Wiener Leben@\emph{Wiener Leben}|pwk}. In: \emph{Vossische Zeitung}\pwindex{Vossische Zeitung@\emph{Vossische Zeitung}|pwk}, Nr. 466, 5. 10. 1900, Morgen-Ausgabe, S. [16].}}}\label{K_L02935-1} in der »Voſſiſchen Zeitung\pwindex{Vossische Zeitung@\emph{Vossische Zeitung}|pw}« heut folgendermaßen aus:\pend
           {\vspace{1\baselineskip}}
\pstart
           \textcolor{gray}{\textbf{Ein anderes Ereigniß, das mit dem Theater\orgindex{Burgtheater@Burgtheater|pwv} in Zuſammenhang ſtand, beginnt
                  bereits dem Gedächtnis der Zeitgenoſſen zu entſchwinden: Die \label{K_L02935-2v}\edtext{Affaire Schnitzler–Schlenther\pwindex{Schlenther, Paul 20.08.1854 – 30.04.1916@\textsc{Schlenther, Paul} (20.08.1854 – 30.04.1916), \emph{Schriftsteller/Schriftstellerin, Kritiker/Kritikerin, Theaterleiter/Theaterleiterin}|pw}}{\lemma{\textnormal{\emph{Affaire Schnitzler–Schlenther}}}\Cendnote{\textnormal{Siehe Richard Beer-Hofmann an Arthur Schnitzler, 14. 9. 1900.
                  }}}\label{K_L02935-2}. Schlenther\pwindex{Schlenther, Paul 20.08.1854 – 30.04.1916@\textsc{Schlenther, Paul} (20.08.1854 – 30.04.1916), \emph{Schriftsteller/Schriftstellerin, Kritiker/Kritikerin, Theaterleiter/Theaterleiterin}|pw} ſoll das neue Stück\pwindex{Schleier der Beatrice. Schauspiel in fuenf Akten@\emph{Der Schleier der Beatrice. Schauspiel in fünf Akten}|pwv} Schnitzlers »Der Schleier der Beatrice\pwindex{Schleier der Beatrice. Schauspiel in fuenf Akten@\emph{Der Schleier der Beatrice. Schauspiel in fünf Akten}|pw}« im Januar für das Burgtheater\orgindex{Burgtheater@Burgtheater|pw} angenommen, im September
                  abgelehnt haben, was die Vormünder der öſterreich\oindex{Oesterreich@\textbf{Österreich}, \emph{A.PCLI}|pw}iſchen dramatiſchen Produktion zu einem flammenden Proteſte
                  gegen das Vorgehen Schlenthers\pwindex{Schlenther, Paul 20.08.1854 – 30.04.1916@\textsc{Schlenther, Paul} (20.08.1854 – 30.04.1916), \emph{Schriftsteller/Schriftstellerin, Kritiker/Kritikerin, Theaterleiter/Theaterleiterin}|pw} veranlaßte.
                  Aus den der Oeffentlichkeit mitgetheilten, gewiß nicht für die Oeffentlichkeit
                  bestimmt geweſenen \label{K_L02935-3v}\edtext{Briefen\pwindex{Erklaerung [Schleier der Beatrice]@\emph{Erklärung [Schleier der Beatrice]}|pwv}}{\lemma{\textnormal{\emph{Briefen}}}\Cendnote{\textnormal{Siehe Paul Goldmann an Arthur Schnitzler, 19. 9. [1900].
                  }}}\label{K_L02935-3} wird der Unbefangene das angebliche Schlenther\pwindex{Schlenther, Paul 20.08.1854 – 30.04.1916@\textsc{Schlenther, Paul} (20.08.1854 – 30.04.1916), \emph{Schriftsteller/Schriftstellerin, Kritiker/Kritikerin, Theaterleiter/Theaterleiterin}|pw}ſche Verſchulden nicht ableiten können; aus den Briefen\pwindex{Erklaerung [Schleier der Beatrice]@\emph{Erklärung [Schleier der Beatrice]}|pwv} geht nichts anderes hervor, als
                  daß Schlenther\pwindex{Schlenther, Paul 20.08.1854 – 30.04.1916@\textsc{Schlenther, Paul} (20.08.1854 – 30.04.1916), \emph{Schriftsteller/Schriftstellerin, Kritiker/Kritikerin, Theaterleiter/Theaterleiterin}|pw} ſich das Recht der
                  Erſtaufführung des Stück\pwindex{Schleier der Beatrice. Schauspiel in fuenf Akten@\emph{Der Schleier der Beatrice. Schauspiel in fünf Akten}|pwv}es
                     \so{für den Fall} der Annahme ſichern wollte und
                  ſicherte, keineswegs aber, daß das Stück\pwindex{Schleier der Beatrice. Schauspiel in fuenf Akten@\emph{Der Schleier der Beatrice. Schauspiel in fünf Akten}|pwv} ſchon angenommen war. Da man auf Seite Schlenthers\pwindex{Schlenther, Paul 20.08.1854 – 30.04.1916@\textsc{Schlenther, Paul} (20.08.1854 – 30.04.1916), \emph{Schriftsteller/Schriftstellerin, Kritiker/Kritikerin, Theaterleiter/Theaterleiterin}|pw} böſe Abſicht gewiß nicht
                  vermuthet, ſo kann der Auslegung, die die Schlenther\pwindex{Schlenther, Paul 20.08.1854 – 30.04.1916@\textsc{Schlenther, Paul} (20.08.1854 – 30.04.1916), \emph{Schriftsteller/Schriftstellerin, Kritiker/Kritikerin, Theaterleiter/Theaterleiterin}|pw}ſchen Briefe\pwindex{Erklaerung [Schleier der Beatrice]@\emph{Erklärung [Schleier der Beatrice]}|pwv} bei Schnitzler fanden, nichts anderes als ein Mißverſtändniß zu
                  Grunde liegen. Die literariſchen Freunde\pwindex{Bahr, Hermann 19.07.1863 – 15.01.1934@\textsc{Bahr, Hermann} (19.07.1863 – 15.01.1934), \emph{Schriftsteller/Schriftstellerin, Kritiker/Kritikerin}|pwv}\pwindex{Salten, Felix 06.09.1869 – 08.10.1945@\textsc{Salten, Felix} (06.09.1869 – 08.10.1945), \emph{Schriftsteller/Schriftstellerin, Journalist/Journalistin, Chefredakteur/Chefredakteurin}|pwv}\pwindex{Bauer, Julius 15.10.1853 – 11.06.1941@\textsc{Bauer, Julius} (15.10.1853 – 11.06.1941), \emph{Schriftsteller/Schriftstellerin, Journalist/Journalistin, Kritiker/Kritikerin}|pwv}\pwindex{Hirschfeld, Robert 17.09.1857 – 02.04.1914@\textsc{Hirschfeld, Robert} (17.09.1857 – 02.04.1914), \emph{Journalist/Journalistin, Musikkritiker/Musikkritikerin}|pwv}\pwindex{Speidel, Ludwig 1830-04-11 – 1906-02-03@\textsc{Speidel, Ludwig} (1830-04-11 – 1906-02-03), \emph{Journalist/Journalistin, Kritiker/Kritikerin}|pwv}\pwindex{David, Jakob Julius 1859-02-06 – 1906-11-20@\textsc{David, Jakob Julius} (1859-02-06 – 1906-11-20), \emph{Schriftsteller/Schriftstellerin, Journalist/Journalistin}|pwv}
                  Schnitzlers ließen aber ſofort ſchweres Geſchütz\pwindex{Erklaerung [Schleier der Beatrice]@\emph{Erklärung [Schleier der Beatrice]}|pwv} gegen Schlenther\pwindex{Schlenther, Paul 20.08.1854 – 30.04.1916@\textsc{Schlenther, Paul} (20.08.1854 – 30.04.1916), \emph{Schriftsteller/Schriftstellerin, Kritiker/Kritikerin, Theaterleiter/Theaterleiterin}|pw} auffahren und ſtellten ohne weiteres auf ſeiner Seite die böſe
                  Abſicht feſt.}}\pend
           
\pstart
           {\pb}Die Parteilichkeit der Darſtellung darf Dich mit
               Rückſicht auf die Beziehungen \textsc{Schlenthers\pwindex{Schlenther, Paul 20.08.1854 – 30.04.1916@\textsc{Schlenther, Paul} (20.08.1854 – 30.04.1916), \emph{Schriftsteller/Schriftstellerin, Kritiker/Kritikerin, Theaterleiter/Theaterleiterin}|pw}} zur »Voſſiſchen Zeitung\orgindex{Vossische Zeitung@Vossische Zeitung|pw}« nicht verwundern. Ich theile Dir das nur
               mit, damit Du Dir dieſen Herrn \textsc{Anton Reitler\pwindex{Reitler, Anton *~1856-10-31@\textsc{Reitler, Anton} (*~1856-10-31), \emph{Dichter/Dichterin, Jurist/Juristin}|pw}}{ }\label{K_L02935-4v}\edtext{\textsc{ad notam}}{\lemma{\textnormal{\emph{ad notam}}}\Cendnote{\textnormal{lateinisch: zur Kenntnis}}}\label{K_L02935-4}
               nimmſt.\pend
           
\pstart
           Ich vergaß \label{K_L02935-5v}\edtext{geſtern}{\lemma{\textnormal{\emph{geſtern}}}\Cendnote{\textnormal{Paul Goldmann an Arthur Schnitzler, 4. 10. [1900].
               }}}\label{K_L02935-5}, Dir Grüße aufzutragen an die ſtrebſamen \label{K_L02935-6v}\edtext{Fräulein\pwindex{Schnitzler, Olga 17.01.1882 – 13.01.1970@\textsc{Schnitzler, Olga} (17.01.1882 – 13.01.1970), \emph{Schauspieler/Schauspielerin, Sänger/Sängerin}|pwv}\pwindex{Steinrueck, Elisabeth 19.11.1885 – 07.04.1920@\textsc{Steinrück, Elisabeth} (19.11.1885 – 07.04.1920)|pwv} aus {\pb}der Rothen-Stern-Gaſſe\oindex{Rotensterngasse@\textbf{Rotensterngasse}, \emph{Straße (K.STR)}|pw}}{\lemma{\textnormal{\emph{Fräulein … Rothen-Stern-Gaſſe}}}\Cendnote{\textnormal{Vgl. Paul Goldmann an Arthur Schnitzler, 19. 9. [1900].
               }}}\label{K_L02935-6}.\pend
           
\pstart
           Viele Grüße auch an Dich!\pend
           
\pstart
           Dein {\\[\baselineskip]}\spacefill\mbox{Paul Goldmnn}\pend
           \leftskip=0em{}\selectlanguage{ngerman}\endnumbering\briefempfaengerindex{Schnitzler, Arthur@\textsc{Schnitzler, Arthur}!zzzGoldmann, Paul@\emph{von Paul Goldmann}!1900-10-051@{5. 10. {[}1900{]}}|)be}\mylabel{L02935h}  \normalsize

\doendnotes{C}
\bigskip
\vfill

\clearpage

\footnotesize

\lohead{\textsc{register}}

% Definiere theindex-Environment komplett neu ohne reledmac
\makeatletter
\renewenvironment{theindex}{%
  \section*{\indexname}%
  \setlength{\parindent}{0pt}%
  \setlength{\parskip}{0pt plus 0.3pt}%
  \let\item\@idxitem
}{%
  \clearpage
}
\makeatother

\IfFileExists{\jobname-pw.ind}{\input{\jobname-pw.ind}}{}

\end{document}

      