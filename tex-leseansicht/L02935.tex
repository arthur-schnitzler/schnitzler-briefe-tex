%% latex-leseansicht-vorspann.tex
%% Vorspann für die Leseansicht.
%% Lädt die gemeinsame Datei latex-vorspann.tex mit nicht gesetztem Schalter.

\newif\ifkorrekturansicht
\korrekturansichtfalse

\input{../tex-inputs/latex-vorspann}


\section[ Paul Goldmann an Arthur Schnitzler, 5. 10. [1900]]{L02935 Paul Goldmann an Arthur Schnitzler,  5. 10. [1900]}
\nopagebreak\mylabel{L02935v}
\rehead{ }\normalsize\beginnumbering\briefempfaengerindex{Schnitzler, Arthur@\textsc{Schnitzler, Arthur}!zzzGoldmann, Paul@\emph{von Paul Goldmann}!1900-10-051@{5. 10. [1900]}|(be}
\toendnotes[C]{\smallbreak\pagebreak[2]}
\correspDesc{Versand  durch Paul Goldmann am 5. 10. [1900] in Berlin
\newline{}Erhalt  durch Arthur Schnitzler im Zeitraum [6. 10. 1900
                  – 10. 10. 1900?] in Wien}\toendnotes[C]{\smallbreak}
\Standort{DLA, A:Schnitzler, HS.NZ85.1.3170.}
\physDesc{Brief, 1 Blatt, 3 Seiten, 490 Zeichen
\newline{}Handschrift: blaue Tinte, deutsche Kurrent
\newline{}Beilage: ein aufgeklebter beschnittener Zeitungsausschnitt 
\newline{}Schnitzler: 1) mit Bleistift das Jahr »900.« vermerkt  2) mit rotem Buntstift zwei Unterstreichungen}\toendnotes[C]{\smallbreak}
\pstart
           {\pb}Berlin\oindex{Berlin@\textbf{Berlin}, \emph{Hauptstadt}|pw}, 5. Oktober.\hfill \textcolor{gray}{\textbf{DESSAUERSTRASSE 19}}\oindex{Dessauer Straße@\textbf{Dessauer Straße}, \emph{Straße}|pw}\pend
           
\pstart\center{}Mein lieber Freund,\pend\vspace{0.5em}
\pstart
           Ein Herr \textsc{Anton Reitler\pwindex{Reitler, Anton *~31.\,10.\,1856 Prag@\textsc{Reitler, Anton} (*~31.\,10.\,1856 Prag), \emph{Dichter, Jurist}|pw}} (?) läßt{ }ſich in einem \label{K_L02935-1v}\edtext{Wiener Briefe\pwindex{Reitler, Anton *~31.\,10.\,1856 Prag@\textsc{Reitler, Anton} (*~31.\,10.\,1856 Prag), \emph{Dichter, Jurist}!Wiener Leben@\strich\emph{Wiener Leben}|pwv}}{\lemma{\textnormal{\emph{Wiener Briefe}}}\Cendnote{\textnormal{Anton Reitler\pwindex{Reitler, Anton *~31.\,10.\,1856 Prag@\textsc{Reitler, Anton} (*~31.\,10.\,1856 Prag), \emph{Dichter, Jurist}|pwk}: \emph{Wiener Leben}\pwindex{Reitler, Anton *~31.\,10.\,1856 Prag@\textsc{Reitler, Anton} (*~31.\,10.\,1856 Prag), \emph{Dichter, Jurist}!Wiener Leben@\strich\emph{Wiener Leben}|pwk}. In: \emph{Vossische Zeitung}\pwindex{Vossische Zeitung@\emph{Vossische Zeitung}|pwk}, Nr. 466, 5. 10. 1900, Morgen-Ausgabe, S. [16].}}}\label{K_L02935-1} in der »Voſſiſchen Zeitung\pwindex{Vossische Zeitung@\emph{Vossische Zeitung}|pw}« heut folgendermaßen aus:\pend
           {\vspace{1\baselineskip}}
\pstart
           \textcolor{gray}{\textbf{Ein anderes Ereigniß, das mit dem Theater\orgindex{Burgtheater@Burgtheater|pwv} in Zuſammenhang{ }ſtand, beginnt
                  bereits dem Gedächtnis der Zeitgenoſſen zu entſchwinden: Die \label{K_L02935-2v}\edtext{Affaire Schnitzler–Schlenther\pwindex{Schlenther, Paul 20.\,8.\,1854 Chernyakhovsk – 30.\,4.\,1916 Berlin@\textsc{Schlenther, Paul} (20.\,8.\,1854 Chernyakhovsk – 30.\,4.\,1916 Berlin), \emph{Schriftsteller, Kritiker, Theaterleiter}|pw}}{\lemma{\textnormal{\emph{Affaire Schnitzler–Schlenther}}}\Cendnote{\textnormal{Siehe XXXX Auszeichnungsfehler: Dokument L01073 nicht gefunden.
                  }}}\label{K_L02935-2}. Schlenther\pwindex{Schlenther, Paul 20.\,8.\,1854 Chernyakhovsk – 30.\,4.\,1916 Berlin@\textsc{Schlenther, Paul} (20.\,8.\,1854 Chernyakhovsk – 30.\,4.\,1916 Berlin), \emph{Schriftsteller, Kritiker, Theaterleiter}|pw}{ }ſoll das neue Stück\pwindex{Schnitzler, Arthur 15.\,5.\,1862 Wien – 21.\,10.\,1931 ebd.@\textsc{Schnitzler, Arthur} (15.\,5.\,1862 Wien – 21.\,10.\,1931 ebd.), \emph{Schriftsteller, Mediziner}!Schleier der Beatrice. Schauspiel in fünf Akten@\strich\emph{Der Schleier der Beatrice. Schauspiel in fünf Akten}|pwv} Schnitzlers »Der Schleier der Beatrice\pwindex{Schnitzler, Arthur 15.\,5.\,1862 Wien – 21.\,10.\,1931 ebd.@\textsc{Schnitzler, Arthur} (15.\,5.\,1862 Wien – 21.\,10.\,1931 ebd.), \emph{Schriftsteller, Mediziner}!Schleier der Beatrice. Schauspiel in fünf Akten@\strich\emph{Der Schleier der Beatrice. Schauspiel in fünf Akten}|pw}« im Januar für das Burgtheater\orgindex{Burgtheater@Burgtheater|pw} angenommen, im September
                  abgelehnt haben, was die Vormünder der öſterreich\oindex{Österreich@\textbf{Österreich}|pw}iſchen dramatiſchen Produktion zu einem flammenden Proteſte
                  gegen das Vorgehen Schlenthers\pwindex{Schlenther, Paul 20.\,8.\,1854 Chernyakhovsk – 30.\,4.\,1916 Berlin@\textsc{Schlenther, Paul} (20.\,8.\,1854 Chernyakhovsk – 30.\,4.\,1916 Berlin), \emph{Schriftsteller, Kritiker, Theaterleiter}|pw} veranlaßte.
                  Aus den der Oeffentlichkeit mitgetheilten, gewiß nicht für die Oeffentlichkeit
                  bestimmt geweſenen \label{K_L02935-3v}\edtext{Briefen\pwindex{Bahr, Hermann 19.\,7.\,1863 Linz – 15.\,1.\,1934 München@\textsc{Bahr, Hermann} (19.\,7.\,1863 Linz – 15.\,1.\,1934 München), \emph{Schriftsteller, Kritiker}!Erklärung [Schleier der Beatrice]@\strich\emph{Erklärung [Schleier der Beatrice]}|pwv}\pwindex{Salten, Felix 6.\,9.\,1869 Budapest – 8.\,10.\,1945 Zürich@\textsc{Salten, Felix} (6.\,9.\,1869 Budapest – 8.\,10.\,1945 Zürich), \emph{Schriftsteller, Journalist, Chefredakteur}!Erklärung [Schleier der Beatrice]@\strich\emph{Erklärung [Schleier der Beatrice]}|pwv}\pwindex{Bauer, Julius 15.\,10.\,1853 Szigetvár – 11.\,6.\,1941 Wien@\textsc{Bauer, Julius} (15.\,10.\,1853 Szigetvár – 11.\,6.\,1941 Wien), \emph{Schriftsteller, Journalist, Kritiker}!Erklärung [Schleier der Beatrice]@\strich\emph{Erklärung [Schleier der Beatrice]}|pwv}\pwindex{Hirschfeld, Robert 17.\,9.\,1857 Žďár nad Sázavou – 2.\,4.\,1914 Salzburg@\textsc{Hirschfeld, Robert} (17.\,9.\,1857 Žďár nad Sázavou – 2.\,4.\,1914 Salzburg), \emph{Journalist, Musikkritiker}!Erklärung [Schleier der Beatrice]@\strich\emph{Erklärung [Schleier der Beatrice]}|pwv}\pwindex{Speidel, Ludwig 11.\,4.\,1830 Ulm – 3.\,2.\,1906 Wien@\textsc{Speidel, Ludwig} (11.\,4.\,1830 Ulm – 3.\,2.\,1906 Wien), \emph{Journalist, Kritiker}!Erklärung [Schleier der Beatrice]@\strich\emph{Erklärung [Schleier der Beatrice]}|pwv}\pwindex{David, Jakob Julius 6.\,2.\,1859 Hranice – 20.\,11.\,1906 Wien@\textsc{David, Jakob Julius} (6.\,2.\,1859 Hranice – 20.\,11.\,1906 Wien), \emph{Schriftsteller, Journalist}!Erklärung [Schleier der Beatrice]@\strich\emph{Erklärung [Schleier der Beatrice]}|pwv}}{\lemma{\textnormal{\emph{Briefen}}}\Cendnote{\textnormal{Siehe XXXX Auszeichnungsfehler: Dokument L02931 nicht gefunden.
                  }}}\label{K_L02935-3} wird der Unbefangene das angebliche Schlenther\pwindex{Schlenther, Paul 20.\,8.\,1854 Chernyakhovsk – 30.\,4.\,1916 Berlin@\textsc{Schlenther, Paul} (20.\,8.\,1854 Chernyakhovsk – 30.\,4.\,1916 Berlin), \emph{Schriftsteller, Kritiker, Theaterleiter}|pw}ſche Verſchulden nicht ableiten können; aus den Briefen\pwindex{Bahr, Hermann 19.\,7.\,1863 Linz – 15.\,1.\,1934 München@\textsc{Bahr, Hermann} (19.\,7.\,1863 Linz – 15.\,1.\,1934 München), \emph{Schriftsteller, Kritiker}!Erklärung [Schleier der Beatrice]@\strich\emph{Erklärung [Schleier der Beatrice]}|pwv}\pwindex{Salten, Felix 6.\,9.\,1869 Budapest – 8.\,10.\,1945 Zürich@\textsc{Salten, Felix} (6.\,9.\,1869 Budapest – 8.\,10.\,1945 Zürich), \emph{Schriftsteller, Journalist, Chefredakteur}!Erklärung [Schleier der Beatrice]@\strich\emph{Erklärung [Schleier der Beatrice]}|pwv}\pwindex{Bauer, Julius 15.\,10.\,1853 Szigetvár – 11.\,6.\,1941 Wien@\textsc{Bauer, Julius} (15.\,10.\,1853 Szigetvár – 11.\,6.\,1941 Wien), \emph{Schriftsteller, Journalist, Kritiker}!Erklärung [Schleier der Beatrice]@\strich\emph{Erklärung [Schleier der Beatrice]}|pwv}\pwindex{Hirschfeld, Robert 17.\,9.\,1857 Žďár nad Sázavou – 2.\,4.\,1914 Salzburg@\textsc{Hirschfeld, Robert} (17.\,9.\,1857 Žďár nad Sázavou – 2.\,4.\,1914 Salzburg), \emph{Journalist, Musikkritiker}!Erklärung [Schleier der Beatrice]@\strich\emph{Erklärung [Schleier der Beatrice]}|pwv}\pwindex{Speidel, Ludwig 11.\,4.\,1830 Ulm – 3.\,2.\,1906 Wien@\textsc{Speidel, Ludwig} (11.\,4.\,1830 Ulm – 3.\,2.\,1906 Wien), \emph{Journalist, Kritiker}!Erklärung [Schleier der Beatrice]@\strich\emph{Erklärung [Schleier der Beatrice]}|pwv}\pwindex{David, Jakob Julius 6.\,2.\,1859 Hranice – 20.\,11.\,1906 Wien@\textsc{David, Jakob Julius} (6.\,2.\,1859 Hranice – 20.\,11.\,1906 Wien), \emph{Schriftsteller, Journalist}!Erklärung [Schleier der Beatrice]@\strich\emph{Erklärung [Schleier der Beatrice]}|pwv} geht nichts anderes hervor, als
                  daß Schlenther\pwindex{Schlenther, Paul 20.\,8.\,1854 Chernyakhovsk – 30.\,4.\,1916 Berlin@\textsc{Schlenther, Paul} (20.\,8.\,1854 Chernyakhovsk – 30.\,4.\,1916 Berlin), \emph{Schriftsteller, Kritiker, Theaterleiter}|pw}{ }ſich das Recht der
                  Erſtaufführung des Stück\pwindex{Schnitzler, Arthur 15.\,5.\,1862 Wien – 21.\,10.\,1931 ebd.@\textsc{Schnitzler, Arthur} (15.\,5.\,1862 Wien – 21.\,10.\,1931 ebd.), \emph{Schriftsteller, Mediziner}!Schleier der Beatrice. Schauspiel in fünf Akten@\strich\emph{Der Schleier der Beatrice. Schauspiel in fünf Akten}|pwv}es
                     \so{für den Fall} der Annahme{ }ſichern wollte und{ }ſicherte, keineswegs aber, daß das Stück\pwindex{Schnitzler, Arthur 15.\,5.\,1862 Wien – 21.\,10.\,1931 ebd.@\textsc{Schnitzler, Arthur} (15.\,5.\,1862 Wien – 21.\,10.\,1931 ebd.), \emph{Schriftsteller, Mediziner}!Schleier der Beatrice. Schauspiel in fünf Akten@\strich\emph{Der Schleier der Beatrice. Schauspiel in fünf Akten}|pwv}{ }ſchon angenommen war. Da man auf Seite Schlenthers\pwindex{Schlenther, Paul 20.\,8.\,1854 Chernyakhovsk – 30.\,4.\,1916 Berlin@\textsc{Schlenther, Paul} (20.\,8.\,1854 Chernyakhovsk – 30.\,4.\,1916 Berlin), \emph{Schriftsteller, Kritiker, Theaterleiter}|pw} böſe Abſicht gewiß nicht
                  vermuthet,{ }ſo kann der Auslegung, die die Schlenther\pwindex{Schlenther, Paul 20.\,8.\,1854 Chernyakhovsk – 30.\,4.\,1916 Berlin@\textsc{Schlenther, Paul} (20.\,8.\,1854 Chernyakhovsk – 30.\,4.\,1916 Berlin), \emph{Schriftsteller, Kritiker, Theaterleiter}|pw}ſchen Briefe\pwindex{Bahr, Hermann 19.\,7.\,1863 Linz – 15.\,1.\,1934 München@\textsc{Bahr, Hermann} (19.\,7.\,1863 Linz – 15.\,1.\,1934 München), \emph{Schriftsteller, Kritiker}!Erklärung [Schleier der Beatrice]@\strich\emph{Erklärung [Schleier der Beatrice]}|pwv}\pwindex{Salten, Felix 6.\,9.\,1869 Budapest – 8.\,10.\,1945 Zürich@\textsc{Salten, Felix} (6.\,9.\,1869 Budapest – 8.\,10.\,1945 Zürich), \emph{Schriftsteller, Journalist, Chefredakteur}!Erklärung [Schleier der Beatrice]@\strich\emph{Erklärung [Schleier der Beatrice]}|pwv}\pwindex{Bauer, Julius 15.\,10.\,1853 Szigetvár – 11.\,6.\,1941 Wien@\textsc{Bauer, Julius} (15.\,10.\,1853 Szigetvár – 11.\,6.\,1941 Wien), \emph{Schriftsteller, Journalist, Kritiker}!Erklärung [Schleier der Beatrice]@\strich\emph{Erklärung [Schleier der Beatrice]}|pwv}\pwindex{Hirschfeld, Robert 17.\,9.\,1857 Žďár nad Sázavou – 2.\,4.\,1914 Salzburg@\textsc{Hirschfeld, Robert} (17.\,9.\,1857 Žďár nad Sázavou – 2.\,4.\,1914 Salzburg), \emph{Journalist, Musikkritiker}!Erklärung [Schleier der Beatrice]@\strich\emph{Erklärung [Schleier der Beatrice]}|pwv}\pwindex{Speidel, Ludwig 11.\,4.\,1830 Ulm – 3.\,2.\,1906 Wien@\textsc{Speidel, Ludwig} (11.\,4.\,1830 Ulm – 3.\,2.\,1906 Wien), \emph{Journalist, Kritiker}!Erklärung [Schleier der Beatrice]@\strich\emph{Erklärung [Schleier der Beatrice]}|pwv}\pwindex{David, Jakob Julius 6.\,2.\,1859 Hranice – 20.\,11.\,1906 Wien@\textsc{David, Jakob Julius} (6.\,2.\,1859 Hranice – 20.\,11.\,1906 Wien), \emph{Schriftsteller, Journalist}!Erklärung [Schleier der Beatrice]@\strich\emph{Erklärung [Schleier der Beatrice]}|pwv} bei Schnitzler fanden, nichts anderes als ein Mißverſtändniß zu
                  Grunde liegen. Die literariſchen Freunde\pwindex{Bahr, Hermann 19.\,7.\,1863 Linz – 15.\,1.\,1934 München@\textsc{Bahr, Hermann} (19.\,7.\,1863 Linz – 15.\,1.\,1934 München), \emph{Schriftsteller, Kritiker}|pwv}\pwindex{Salten, Felix 6.\,9.\,1869 Budapest – 8.\,10.\,1945 Zürich@\textsc{Salten, Felix} (6.\,9.\,1869 Budapest – 8.\,10.\,1945 Zürich), \emph{Schriftsteller, Journalist, Chefredakteur}|pwv}\pwindex{Bauer, Julius 15.\,10.\,1853 Szigetvár – 11.\,6.\,1941 Wien@\textsc{Bauer, Julius} (15.\,10.\,1853 Szigetvár – 11.\,6.\,1941 Wien), \emph{Schriftsteller, Journalist, Kritiker}|pwv}\pwindex{Hirschfeld, Robert 17.\,9.\,1857 Žďár nad Sázavou – 2.\,4.\,1914 Salzburg@\textsc{Hirschfeld, Robert} (17.\,9.\,1857 Žďár nad Sázavou – 2.\,4.\,1914 Salzburg), \emph{Journalist, Musikkritiker}|pwv}\pwindex{Speidel, Ludwig 11.\,4.\,1830 Ulm – 3.\,2.\,1906 Wien@\textsc{Speidel, Ludwig} (11.\,4.\,1830 Ulm – 3.\,2.\,1906 Wien), \emph{Journalist, Kritiker}|pwv}\pwindex{David, Jakob Julius 6.\,2.\,1859 Hranice – 20.\,11.\,1906 Wien@\textsc{David, Jakob Julius} (6.\,2.\,1859 Hranice – 20.\,11.\,1906 Wien), \emph{Schriftsteller, Journalist}|pwv}
                  Schnitzlers ließen aber{ }ſofort{ }ſchweres Geſchütz\pwindex{Bahr, Hermann 19.\,7.\,1863 Linz – 15.\,1.\,1934 München@\textsc{Bahr, Hermann} (19.\,7.\,1863 Linz – 15.\,1.\,1934 München), \emph{Schriftsteller, Kritiker}!Erklärung [Schleier der Beatrice]@\strich\emph{Erklärung [Schleier der Beatrice]}|pwv}\pwindex{Salten, Felix 6.\,9.\,1869 Budapest – 8.\,10.\,1945 Zürich@\textsc{Salten, Felix} (6.\,9.\,1869 Budapest – 8.\,10.\,1945 Zürich), \emph{Schriftsteller, Journalist, Chefredakteur}!Erklärung [Schleier der Beatrice]@\strich\emph{Erklärung [Schleier der Beatrice]}|pwv}\pwindex{Bauer, Julius 15.\,10.\,1853 Szigetvár – 11.\,6.\,1941 Wien@\textsc{Bauer, Julius} (15.\,10.\,1853 Szigetvár – 11.\,6.\,1941 Wien), \emph{Schriftsteller, Journalist, Kritiker}!Erklärung [Schleier der Beatrice]@\strich\emph{Erklärung [Schleier der Beatrice]}|pwv}\pwindex{Hirschfeld, Robert 17.\,9.\,1857 Žďár nad Sázavou – 2.\,4.\,1914 Salzburg@\textsc{Hirschfeld, Robert} (17.\,9.\,1857 Žďár nad Sázavou – 2.\,4.\,1914 Salzburg), \emph{Journalist, Musikkritiker}!Erklärung [Schleier der Beatrice]@\strich\emph{Erklärung [Schleier der Beatrice]}|pwv}\pwindex{Speidel, Ludwig 11.\,4.\,1830 Ulm – 3.\,2.\,1906 Wien@\textsc{Speidel, Ludwig} (11.\,4.\,1830 Ulm – 3.\,2.\,1906 Wien), \emph{Journalist, Kritiker}!Erklärung [Schleier der Beatrice]@\strich\emph{Erklärung [Schleier der Beatrice]}|pwv}\pwindex{David, Jakob Julius 6.\,2.\,1859 Hranice – 20.\,11.\,1906 Wien@\textsc{David, Jakob Julius} (6.\,2.\,1859 Hranice – 20.\,11.\,1906 Wien), \emph{Schriftsteller, Journalist}!Erklärung [Schleier der Beatrice]@\strich\emph{Erklärung [Schleier der Beatrice]}|pwv} gegen Schlenther\pwindex{Schlenther, Paul 20.\,8.\,1854 Chernyakhovsk – 30.\,4.\,1916 Berlin@\textsc{Schlenther, Paul} (20.\,8.\,1854 Chernyakhovsk – 30.\,4.\,1916 Berlin), \emph{Schriftsteller, Kritiker, Theaterleiter}|pw} auffahren und{ }ſtellten ohne weiteres auf{ }ſeiner Seite die böſe
                  Abſicht feſt.}}\pend
           
\pstart
           {\pb}Die Parteilichkeit der Darſtellung darf Dich mit
               Rückſicht auf die Beziehungen \textsc{Schlenthers\pwindex{Schlenther, Paul 20.\,8.\,1854 Chernyakhovsk – 30.\,4.\,1916 Berlin@\textsc{Schlenther, Paul} (20.\,8.\,1854 Chernyakhovsk – 30.\,4.\,1916 Berlin), \emph{Schriftsteller, Kritiker, Theaterleiter}|pw}} zur »Voſſiſchen Zeitung\orgindex{Vossische Zeitung@Vossische Zeitung|pw}« nicht verwundern. Ich theile Dir das nur
               mit, damit Du Dir dieſen Herrn \textsc{Anton Reitler\pwindex{Reitler, Anton *~31.\,10.\,1856 Prag@\textsc{Reitler, Anton} (*~31.\,10.\,1856 Prag), \emph{Dichter, Jurist}|pw}}{ }\label{K_L02935-4v}\edtext{\textsc{ad notam}}{\lemma{\textnormal{\emph{ad notam}}}\Cendnote{\textnormal{lateinisch: zur Kenntnis}}}\label{K_L02935-4}
               nimmſt.\pend
           
\pstart
           Ich vergaß \label{K_L02935-5v}\edtext{geſtern}{\lemma{\textnormal{\emph{gestern}}}\Cendnote{\textnormal{XXXX Auszeichnungsfehler: Dokument L02934 nicht gefunden.
               }}}\label{K_L02935-5}, Dir Grüße aufzutragen an die{ }ſtrebſamen \label{K_L02935-6v}\edtext{Fräulein\pwindex{Schnitzler, Olga 17.\,1.\,1882 Wien – 13.\,1.\,1970 Lugano@\textsc{Schnitzler, Olga} (17.\,1.\,1882 Wien – 13.\,1.\,1970 Lugano), \emph{Schauspielerin, Sängerin}|pwv}\pwindex{Steinrück, Elisabeth 19.\,11.\,1885 – 7.\,4.\,1920 Partenkirchen@\textsc{Steinrück, Elisabeth} (19.\,11.\,1885 – 7.\,4.\,1920 Partenkirchen)|pwv} aus {\pb}der Rothen-Stern-Gaſſe\oindex{Wien@\textbf{Wien}!II., Leopoldstadt@\textbf{II., Leopoldstadt}!Rotensterngasse@\textbf{Rotensterngasse}, \emph{Straße}|pw}}{\lemma{\textnormal{\emph{Fräulein … Rothen-Stern-Gasse}}}\Cendnote{\textnormal{Vgl. XXXX Auszeichnungsfehler: Dokument L02931 nicht gefunden.
               }}}\label{K_L02935-6}.\pend
           
\pstart
           Viele Grüße auch an Dich!\pend
           
\pstart
           Dein {\\[\baselineskip]}\spacefill\mbox{Paul Goldmnn}\pend
           \leftskip=0em{}\selectlanguage{ngerman}\endnumbering\briefempfaengerindex{Schnitzler, Arthur@\textsc{Schnitzler, Arthur}!zzzGoldmann, Paul@\emph{von Paul Goldmann}!1900-10-051@{5. 10. [1900]}|)be}\mylabel{L02935h}  \newcommand{\dateiname}{L02935}\newcommand{\titel}{Paul Goldmann an Arthur Schnitzler, 5. 10. [1900]}\newcommand{\editorInnen}{Martin Anton Müller und Laura Untner}%% latex-leseansicht-abspann.tex
%% Abspann für die Leseansicht.
%% Der Schalter \ifkorrekturansicht ist bereits durch den Vorspann gesetzt.

%% latex-abspann.tex
%% Gemeinsamer Abspann für Korrekturansicht und Leseansicht.
%% Setzt den Schalter \ifkorrekturansicht voraus (gesetzt in den
%% einbindenden Dateien latex-korrekturansicht-abspann.tex bzw.
%% latex-leseansicht-abspann.tex).
%% ---------------------------------------------------------------

\normalsize

% Das esempio-Environment wird nur in der Leseansicht benötigt
\ifkorrekturansicht\else
\newenvironment{esempio}[3]%
{
    \vspace{1.5ex}
    \rlap{\underline{#1}}
    \par
    \setlength{\parindent}{0cm}
    \nopagebreak
    \leftskip=#2cm
    \rightskip=#3cm
}
{
    \par
}
\fi

\doendnotes{C}
\bigskip
\vfill

\clearpage

\footnotesize

\ifkorrekturansicht
  \lohead{\textsc{register}}
\fi

% theindex-Environment neu definieren ohne reledmac
\makeatletter
\renewenvironment{theindex}{%
  \ifkorrekturansicht
    \section*{\indexname}%
  \else
    \subsubsection*{Index der erwähnten Entitäten}%
  \fi
  \setlength{\parindent}{0pt}%
  \setlength{\parskip}{0pt plus 0.3pt}%
  \let\item\@idxitem
}{%
  \ifkorrekturansicht\clearpage\fi
}
\makeatother

\IfFileExists{\jobname-pw.ind}{\input{\jobname-pw.ind}}{}

% Quellenangabe nur in der Leseansicht
\ifkorrekturansicht\else
% Fallback-Definitionen, falls die .tex-Datei \titel etc. nicht gesetzt hat
\providecommand{\titel}{}
\providecommand{\editorInnen}{}
\providecommand{\dateiname}{\jobname}

\vspace{3cm}

\vfill

\footnotesize
\textsc{Quelle}: \titel. Herausgegeben von {\editorInnen}. In: \emph{Arthur Schnitzler: Briefwechsel mit Autorinnen und Autoren}.
 Digitale Edition, https://schnitzler-briefe.acdh.oeaw.ac.at/{\dateiname}.html (Stand \today)
\fi

\end{document}


