%% latex-leseansicht-vorspann.tex
%% Vorspann für die Leseansicht.
%% Lädt die gemeinsame Datei latex-vorspann.tex mit nicht gesetztem Schalter.

\newif\ifkorrekturansicht
\korrekturansichtfalse

\input{../tex-inputs/latex-vorspann}


\section[ Paul Goldmann an Arthur Schnitzler, 31. 12. {[}1901{]}]{L03099 Paul Goldmann an Arthur Schnitzler,  31. 12. [1901]}
\nopagebreak\mylabel{L03099v}
\rehead{ }\normalsize\beginnumbering\briefempfaengerindex{Schnitzler, Arthur@\textsc{Schnitzler, Arthur}!zzzGoldmann, Paul@\emph{von Paul Goldmann}!1901-12-311@{31. 12. [1901]}|(be}
\toendnotes[C]{\smallbreak\pagebreak[2]}
\correspDesc{Versand  durch Paul Goldmann am 31. 12. [1901] in Frankfurt am Main
\newline{}Erhalt  durch Arthur Schnitzler im Zeitraum [1. 1. 1902
                  – 5. 1. 1902?] in Berlin}\toendnotes[C]{\smallbreak}
\Standort{DLA, A:Schnitzler, HS.NZ85.1.3171.}
\physDesc{Brief, 1 Blatt, 1 Seite, 438 Zeichen
\newline{}Handschrift: blaue Tinte, deutsche Kurrent
\newline{}Schnitzler: 1) mit Bleistift das Jahr »901.« vermerkt  2) mit rotem Buntstift eine Unterstreichung}\toendnotes[C]{\smallbreak}
\pstart
           \centering{}{\pb}Frankfurt\oindex{Frankfurt am Main@\textbf{Frankfurt am Main}, \emph{Hauptstadt}|pw}{ }31. Dezember\pend
           
\pstart\center{}Mein lieber Freund,\pend\vspace{0.5em}
\pstart
           Dank für das Billet! Ich freue mich{ }ſehr über den guten Fortgang der Proben\eventindex{Deutsches Theater Berlin@\textbf{Deutsches Theater Berlin}!Probe von Lebendige Stunden, 28.12.1901@Probe von Lebendige Stunden, 28.12.1901|pwv}\eventindex{Deutsches Theater Berlin@\textbf{Deutsches Theater Berlin}!Probe von Lebendige Stunden, 30.12.1901@Probe von Lebendige Stunden, 30.12.1901|pwv}\eventindex{Deutsches Theater Berlin@\textbf{Deutsches Theater Berlin}!Probe von Lebendige Stunden, 2.1.1902@Probe von Lebendige Stunden, 2.1.1902|pwv}\eventindex{Deutsches Theater Berlin@\textbf{Deutsches Theater Berlin}!Generalprobe von Lebendige Stunden, 3.1.1902@Generalprobe von Lebendige Stunden, 3.1.1902|pwv}\pwindex{Schnitzler, Arthur 15.\,5.\,1862 Wien – 21.\,10.\,1931 ebd.@\textsc{Schnitzler, Arthur} (15.\,5.\,1862 Wien – 21.\,10.\,1931 ebd.), \emph{Schriftsteller, Mediziner}!Lebendige Stunden. Vier Einakter@\strich\emph{Lebendige Stunden. Vier Einakter}|pwv}. \label{K_L03099-1v}\edtext{Samſtag{ }Abend\eventindex{Deutsches Theater Berlin@\textbf{Deutsches Theater Berlin}!Uraufführung von Lebendige Stunden, 4.1.1902@Uraufführung von Lebendige Stunden, 4.1.1902|pwv} bin ich im Theater\oindex{Deutsches Theater Berlin@\textbf{Deutsches Theater Berlin}, \emph{Theater}|pwv}}{\lemma{\textnormal{\emph{Samstag … Theater}}}\Cendnote{\textnormal{Am Samstag, dem 4. 1. 1902 fand am
                  Deutschen Theater Berlin\oindex{Deutsches Theater Berlin@\textbf{Deutsches Theater Berlin}, \emph{Theater}|pwk} die Uraufführung der
                     vier Einakter \emph{Lebendige Stunden}\pwindex{Schnitzler, Arthur 15.\,5.\,1862 Wien – 21.\,10.\,1931 ebd.@\textsc{Schnitzler, Arthur} (15.\,5.\,1862 Wien – 21.\,10.\,1931 ebd.), \emph{Schriftsteller, Mediziner}!Lebendige Stunden. Vier Einakter@\strich\emph{Lebendige Stunden. Vier Einakter}|pwk}\eventindex{Deutsches Theater Berlin@\textbf{Deutsches Theater Berlin}!Uraufführung von Lebendige Stunden, 4.1.1902@Uraufführung von Lebendige Stunden, 4.1.1902|pwk}
               statt.}}}\label{K_L03099-1}. Vorher werde ich Dich kaum{ }ſehen, da ich erſt{ }ſpät ankomme. Paß’ bei
               den Proben\eventindex{Deutsches Theater Berlin@\textbf{Deutsches Theater Berlin}!Probe von Lebendige Stunden, 28.12.1901@Probe von Lebendige Stunden, 28.12.1901|pwv}\eventindex{Deutsches Theater Berlin@\textbf{Deutsches Theater Berlin}!Probe von Lebendige Stunden, 30.12.1901@Probe von Lebendige Stunden, 30.12.1901|pwv}\eventindex{Deutsches Theater Berlin@\textbf{Deutsches Theater Berlin}!Probe von Lebendige Stunden, 2.1.1902@Probe von Lebendige Stunden, 2.1.1902|pwv}\eventindex{Deutsches Theater Berlin@\textbf{Deutsches Theater Berlin}!Generalprobe von Lebendige Stunden, 3.1.1902@Generalprobe von Lebendige Stunden, 3.1.1902|pwv} nur auf die \textsc{Triesch\pwindex{Triesch, Irene 13.\,4.\,1877 Wien – 24.\,11.\,1964 Basel@\textsc{Triesch, Irene} (13.\,4.\,1877 Wien – 24.\,11.\,1964 Basel), \emph{Schauspielerin}|pw}} auf, daß{ }ſie nicht zu viel thut! Sie iſt bei aller Begabung von einer
               unglaublichen Geſchmackloſigkeit. Laß’ es Dir in Berlin\oindex{Berlin@\textbf{Berlin}, \emph{Hauptstadt}|pw} gut gehen! Gückliches neues Jahr! Viele
               treue Grüße!\pend
           
\pstart
           Dein {\\[\baselineskip]}\spacefill\mbox{Paul Goldmann}\pend
           \leftskip=0em{}\selectlanguage{ngerman}\endnumbering\briefempfaengerindex{Schnitzler, Arthur@\textsc{Schnitzler, Arthur}!zzzGoldmann, Paul@\emph{von Paul Goldmann}!1901-12-311@{31. 12. [1901]}|)be}\mylabel{L03099h}  \newcommand{\dateiname}{L03099}\newcommand{\titel}{Paul Goldmann an Arthur Schnitzler, 31. 12. [1901]}\newcommand{\editorInnen}{Martin Anton Müller und Laura Untner}%% latex-leseansicht-abspann.tex
%% Abspann für die Leseansicht.
%% Der Schalter \ifkorrekturansicht ist bereits durch den Vorspann gesetzt.

%% latex-abspann.tex
%% Gemeinsamer Abspann für Korrekturansicht und Leseansicht.
%% Setzt den Schalter \ifkorrekturansicht voraus (gesetzt in den
%% einbindenden Dateien latex-korrekturansicht-abspann.tex bzw.
%% latex-leseansicht-abspann.tex).
%% ---------------------------------------------------------------

\normalsize

% Das esempio-Environment wird nur in der Leseansicht benötigt
\ifkorrekturansicht\else
\newenvironment{esempio}[3]%
{
    \vspace{1.5ex}
    \rlap{\underline{#1}}
    \par
    \setlength{\parindent}{0cm}
    \nopagebreak
    \leftskip=#2cm
    \rightskip=#3cm
}
{
    \par
}
\fi

\doendnotes{C}
\bigskip
\vfill

\clearpage

\footnotesize

\ifkorrekturansicht
  \lohead{\textsc{register}}
\fi

% theindex-Environment neu definieren ohne reledmac
\makeatletter
\renewenvironment{theindex}{%
  \ifkorrekturansicht
    \section*{\indexname}%
  \else
    \subsubsection*{Index der erwähnten Entitäten}%
  \fi
  \setlength{\parindent}{0pt}%
  \setlength{\parskip}{0pt plus 0.3pt}%
  \let\item\@idxitem
}{%
  \ifkorrekturansicht\clearpage\fi
}
\makeatother

\IfFileExists{\jobname-pw.ind}{\input{\jobname-pw.ind}}{}

% Quellenangabe nur in der Leseansicht
\ifkorrekturansicht\else
% Fallback-Definitionen, falls die .tex-Datei \titel etc. nicht gesetzt hat
\providecommand{\titel}{}
\providecommand{\editorInnen}{}
\providecommand{\dateiname}{\jobname}

\vspace{3cm}

\vfill

\footnotesize
\textsc{Quelle}: \titel. Herausgegeben von {\editorInnen}. In: \emph{Arthur Schnitzler: Briefwechsel mit Autorinnen und Autoren}.
 Digitale Edition, https://schnitzler-briefe.acdh.oeaw.ac.at/{\dateiname}.html (Stand \today)
\fi

\end{document}


