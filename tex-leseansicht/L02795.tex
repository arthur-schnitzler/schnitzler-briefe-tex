%% latex-leseansicht-vorspann.tex
%% Vorspann für die Leseansicht.
%% Lädt die gemeinsame Datei latex-vorspann.tex mit nicht gesetztem Schalter.

\newif\ifkorrekturansicht
\korrekturansichtfalse

\input{../tex-inputs/latex-vorspann}


\section[ Clementine Goldmann und Vally Rosengart an Arthur Schnitzler, {[}11. 1. 1896{]}]{L02795 Clementine Goldmann und Vally Rosengart an Arthur
               Schnitzler,  [11. 1. 1896]}
\nopagebreak\mylabel{L02795v}
\rehead{ }\normalsize\beginnumbering\briefempfaengerindex{Schnitzler, Arthur@\textsc{Schnitzler, Arthur}!zzzRosengart, Vally@\emph{von Vally Rosengart}!1896-01-112@{{[}11. 1. 1896{]}}|(be}\briefempfaengerindex{Schnitzler, Arthur@\textsc{Schnitzler, Arthur}!zzzGoldmann, Clementine@\emph{von Clementine Goldmann}!1896-01-112@{{[}11. 1. 1896{]}}|(be}
\toendnotes[C]{\smallbreak\pagebreak[2]}
\correspDesc{Versand  durch Clementine Goldmann, Vally Rosengart am [11. 1. 1896] in Frankfurt am Main
\newline{}Erhalt  durch Arthur Schnitzler im Zeitraum [11. 1. 1896
                  – 12. 1. 1896?] in Frankfurt am Main}\toendnotes[C]{\smallbreak}
\Standort{DLA, A:Schnitzler, HS.NZ85.1.3159.}
\physDesc{Briefkarte, 566 Zeichen (Umseitig gestrichene vorgedruckte Adresse: »Große
                                       Eſchenheimerſtraße 1\oindex{Große Eschenheimer Straße@\textbf{Große Eschenheimer Straße}, \emph{Straße}|pw}.« )
\newline{}Handschrift Clementine Goldmann: blaue Tinte, deutsche Kurrent
\newline{}Handschrift Vally Rosengart: blaue Tinte, deutsche Kurrent
\newline{}Schnitzler: mit Bleistift das Datum »11/1 96« vermerkt }\toendnotes[C]{\smallbreak}
\pstart
           \raggedleft{}{\pb}\textsc{Samstag}{ }Abend\pend
           
\pstart\center{}Sehr geehrter Herr \textsc{Doctor}!\pend\vspace{0.5em}
\pstart
           Nehmen Sie wärmſten Glückwunſch zu Ihrem großen \label{K_L02795-1v}\edtext{Erfolge\pwindex{Schnitzler, Arthur 15. 5. 1862 Wien – 21. 10. 1931 ebd.@\textsc{Schnitzler, Arthur} (15. 5. 1862 Wien – 21. 10. 1931 ebd.), \emph{Schriftsteller, Mediziner}!Liebelei. Schauspiel in drei Akten@\strich\emph{Liebelei. Schauspiel in drei Akten}|pwv}}{\lemma{\textnormal{\emph{Erfolge}}}\Cendnote{\textnormal{Diese Karte wurde nach der Premiere von
                     \emph{Liebelei}\pwindex{Schnitzler, Arthur 15. 5. 1862 Wien – 21. 10. 1931 ebd.@\textsc{Schnitzler, Arthur} (15. 5. 1862 Wien – 21. 10. 1931 ebd.), \emph{Schriftsteller, Mediziner}!Liebelei. Schauspiel in drei Akten@\strich\emph{Liebelei. Schauspiel in drei Akten}|pwk} am \emph{Frankfurter Schauspielhaus}\orgindex{Frankfurter Stadttheater@Frankfurter Stadttheater|pwk} verfasst. Schnitzler war zu dieser angereist.}}}\label{K_L02795-1} ud. noch beſonderen Dank für
               den{ }ſeltenen Genuß, den Sie mir mit Ihrem geiſtvollen, {\pb}intereſſanten Stück\pwindex{Schnitzler, Arthur 15. 5. 1862 Wien – 21. 10. 1931 ebd.@\textsc{Schnitzler, Arthur} (15. 5. 1862 Wien – 21. 10. 1931 ebd.), \emph{Schriftsteller, Mediziner}!Liebelei. Schauspiel in drei Akten@\strich\emph{Liebelei. Schauspiel in drei Akten}|pwv}
               bereitet. Wer ein{ }ſo feiner Beobachter des Lebens iſt – wie Sie – der wird noch
               vieles Bedeutende{ }ſchaffen!\pend
           
\pstart
           Auf Wiederſehen bis morgen ud. herzliche Grüße{\\[\baselineskip]}von Ihrer{\\[\baselineskip]}\spacefill\mbox{Clementine Goldmann.}\pend
           \leftskip=0em{}\selectlanguage{ngerman}\vspace{1em}
\pstart
           \noindent{}{[}hs. Rosengart:{]} Sehr verehrter Herr \textsc{Dr}. –
               ich{ }ſchließe mich den Glückwünſchen meiner Mutter auf’s herzlichſte an. Mein \label{K_L02795-2v}\edtext{Mann\pwindex{Rosengart, Josef 8.\,2.\,1860 Laupheim – 4.\,8.\,1927 Frankfurt am Main@\textsc{Rosengart, Josef} (8.\,2.\,1860 Laupheim – 4.\,8.\,1927 Frankfurt am Main), \emph{Arzt}|pwv} wird morgen früh
               perſönlich bei Ihnen vorſprechen}{\lemma{\textnormal{\emph{Mann … vorsprechen}}}\Cendnote{\textnormal{Siehe A. S.: \emph{Tagebuch}, 12. 1. 1896.
               }}}\label{K_L02795-2}. Mit warmem Gruß{\\}Ihre{\\}\spacefill\mbox{Vally Rosengart.}\pend
           \selectlanguage{ngerman}\endnumbering\briefempfaengerindex{Schnitzler, Arthur@\textsc{Schnitzler, Arthur}!zzzRosengart, Vally@\emph{von Vally Rosengart}!1896-01-112@{{[}11. 1. 1896{]}}|)be}\briefempfaengerindex{Schnitzler, Arthur@\textsc{Schnitzler, Arthur}!zzzGoldmann, Clementine@\emph{von Clementine Goldmann}!1896-01-112@{{[}11. 1. 1896{]}}|)be}\mylabel{L02795h}  \newcommand{\dateiname}{L02795}\newcommand{\titel}{Clementine Goldmann und Vally Rosengart an Arthur Schnitzler, [11. 1. 1896]}\newcommand{\editorInnen}{Martin Anton Müller und Laura Untner}%% latex-leseansicht-abspann.tex
%% Abspann für die Leseansicht.
%% Der Schalter \ifkorrekturansicht ist bereits durch den Vorspann gesetzt.

%% latex-abspann.tex
%% Gemeinsamer Abspann für Korrekturansicht und Leseansicht.
%% Setzt den Schalter \ifkorrekturansicht voraus (gesetzt in den
%% einbindenden Dateien latex-korrekturansicht-abspann.tex bzw.
%% latex-leseansicht-abspann.tex).
%% ---------------------------------------------------------------

\normalsize

% Das esempio-Environment wird nur in der Leseansicht benötigt
\ifkorrekturansicht\else
\newenvironment{esempio}[3]%
{
    \vspace{1.5ex}
    \rlap{\underline{#1}}
    \par
    \setlength{\parindent}{0cm}
    \nopagebreak
    \leftskip=#2cm
    \rightskip=#3cm
}
{
    \par
}
\fi

\doendnotes{C}
\bigskip
\vfill

\clearpage

\footnotesize

\ifkorrekturansicht
  \lohead{\textsc{register}}
\fi

% theindex-Environment neu definieren ohne reledmac
\makeatletter
\renewenvironment{theindex}{%
  \ifkorrekturansicht
    \section*{\indexname}%
  \else
    \subsubsection*{Index der erwähnten Entitäten}%
  \fi
  \setlength{\parindent}{0pt}%
  \setlength{\parskip}{0pt plus 0.3pt}%
  \let\item\@idxitem
}{%
  \ifkorrekturansicht\clearpage\fi
}
\makeatother

\IfFileExists{\jobname-pw.ind}{\input{\jobname-pw.ind}}{}

% Quellenangabe nur in der Leseansicht
\ifkorrekturansicht\else
% Fallback-Definitionen, falls die .tex-Datei \titel etc. nicht gesetzt hat
\providecommand{\titel}{}
\providecommand{\editorInnen}{}
\providecommand{\dateiname}{\jobname}

\vspace{3cm}

\vfill

\footnotesize
\textsc{Quelle}: \titel. Herausgegeben von {\editorInnen}. In: \emph{Arthur Schnitzler: Briefwechsel mit Autorinnen und Autoren}.
 Digitale Edition, https://schnitzler-briefe.acdh.oeaw.ac.at/{\dateiname}.html (Stand \today)
\fi

\end{document}


