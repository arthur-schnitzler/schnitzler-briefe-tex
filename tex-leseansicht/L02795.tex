%% latex-korrekturansicht-vorspann.tex
%% Vorspann für die Korrekturansicht.
%% Lädt die gemeinsame Datei latex-vorspann.tex mit gesetztem Schalter.

\newif\ifkorrekturansicht
\korrekturansichttrue

\input{../tex-inputs/latex-vorspann}


\section[ Clementine Goldmann und Vally Rosengart an Arthur Schnitzler, {[}11. 1. 1896{]}]{L02795 Clementine Goldmann und Vally Rosengart an Arthur
               Schnitzler, {[}11. 1. 1896{]}}
\nopagebreak\mylabel{L02795v}
\rehead{ }\normalsize\beginnumbering\briefempfaengerindex{Schnitzler, Arthur@\textsc{Schnitzler, Arthur}!zzzRosengart, Vally@\emph{von Vally Rosengart}!1896-01-112@{{[}11. 1. 1896{]}}|(be}\briefempfaengerindex{Schnitzler, Arthur@\textsc{Schnitzler, Arthur}!zzzGoldmann, Clementine@\emph{von Clementine Goldmann}!1896-01-112@{{[}11. 1. 1896{]}}|(be}
\toendnotes[C]{\smallbreak\pagebreak[2]}\Standort{DLA, A:Schnitzler, HS.NZ85.1.3159.}
\physDesc{Briefkarte, 566 Zeichen (Umseitig gestrichene vorgedruckte Adresse: »Große
                                       Eſchenheimerſtraße 1\oindex{Grosse Eschenheimer Strasse@\textbf{Große Eschenheimer Straße}, \emph{Straße (K.STR)}|pw}.« )
\newline{}Handschrift Clementine Goldmann: blaue Tinte, deutsche Kurrent
\newline{}Handschrift Vally Rosengart: blaue Tinte, deutsche Kurrent
\newline{}Schnitzler: mit Bleistift das Datum »11/1 96« vermerkt }\toendnotes[C]{\smallbreak}
\pstart
           \raggedleft{}{\pb}\textsc{Samstag}{ }Abend\pend
           
\pstart\center{}Sehr geehrter Herr \textsc{Doctor}!\pend\vspace{0.5em}
\pstart
           Nehmen Sie wärmſten Glückwunſch zu Ihrem großen \label{K_L02795-1v}\edtext{Erfolge\pwindex{Liebelei. Schauspiel in drei Akten@\emph{Liebelei. Schauspiel in drei Akten}|pwv}}{\lemma{\textnormal{\emph{Erfolge}}}\Cendnote{\textnormal{Diese Karte wurde nach der Premiere von
                     \emph{Liebelei}\pwindex{Liebelei. Schauspiel in drei Akten@\emph{Liebelei. Schauspiel in drei Akten}|pwk} am \emph{Frankfurter Schauspielhaus}XXXX ORGangabe fehlt verfasst. Schnitzler war zu dieser angereist.}}}\label{K_L02795-1} ud. noch beſonderen Dank für
               den ſeltenen Genuß, den Sie mir mit Ihrem geiſtvollen, {\pb}intereſſanten Stück\pwindex{Liebelei. Schauspiel in drei Akten@\emph{Liebelei. Schauspiel in drei Akten}|pwv}
               bereitet. Wer ein ſo feiner Beobachter des Lebens iſt – wie Sie – der wird noch
               vieles Bedeutende ſchaffen!\pend
           
\pstart
           Auf Wiederſehen bis morgen ud. herzliche Grüße{\\[\baselineskip]}von Ihrer{\\[\baselineskip]}\spacefill\mbox{Clementine Goldmann.}\pend
           \leftskip=0em{}\selectlanguage{ngerman}\vspace{1em}
\pstart
           \noindent{}{[}hs. :{]} Sehr verehrter Herr \textsc{Dr}. –
               ich ſchließe mich den Glückwünſchen meiner Mutter auf’s herzlichſte an. Mein \label{K_L02795-2v}\edtext{Mann\pwindex{Rosengart, Josef 1860-02-08 – 1927-08-04@\textsc{Rosengart, Josef} (1860-02-08 – 1927-08-04), \emph{Arzt/Ärztin}|pwv} wird morgen früh
               perſönlich bei Ihnen vorſprechen}{\lemma{\textnormal{\emph{Mann … vorſprechen}}}\Cendnote{\textnormal{Siehe A. S.: \emph{Tagebuch}, 12. 1. 1896.
               }}}\label{K_L02795-2}. Mit warmem Gruß{\\}Ihre{\\}\spacefill\mbox{Vally Rosengart.}\pend
           \selectlanguage{ngerman}\endnumbering\briefempfaengerindex{Schnitzler, Arthur@\textsc{Schnitzler, Arthur}!zzzRosengart, Vally@\emph{von Vally Rosengart}!1896-01-112@{{[}11. 1. 1896{]}}|)be}\briefempfaengerindex{Schnitzler, Arthur@\textsc{Schnitzler, Arthur}!zzzGoldmann, Clementine@\emph{von Clementine Goldmann}!1896-01-112@{{[}11. 1. 1896{]}}|)be}\mylabel{L02795h}  \normalsize

\doendnotes{C}
\bigskip
\vfill

\clearpage

\footnotesize

\lohead{\textsc{register}}

% Definiere theindex-Environment komplett neu ohne reledmac
\makeatletter
\renewenvironment{theindex}{%
  \section*{\indexname}%
  \setlength{\parindent}{0pt}%
  \setlength{\parskip}{0pt plus 0.3pt}%
  \let\item\@idxitem
}{%
  \clearpage
}
\makeatother

\IfFileExists{\jobname-pw.ind}{\input{\jobname-pw.ind}}{}

\end{document}

      