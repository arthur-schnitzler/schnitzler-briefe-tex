%% latex-korrekturansicht-vorspann.tex
%% Vorspann für die Korrekturansicht.
%% Lädt die gemeinsame Datei latex-vorspann.tex mit gesetztem Schalter.

\newif\ifkorrekturansicht
\korrekturansichttrue

\input{../tex-inputs/latex-vorspann}


\section[Eduard Michael Kafka an Arthur Schnitzler, 24. 1. 1893]{L00162 Eduard Michael Kafka an Arthur Schnitzler, 24. 1. 1893}
\nopagebreak\mylabel{L00162v}
\rehead{ }\normalsize\beginnumbering\briefempfaengerindex{Schnitzler, Arthur@\textsc{Schnitzler, Arthur}!zzzKafka, Eduard Michael@\emph{von Eduard Michael Kafka}!1893-01-241@{24. 1. 1893}|(be}
\toendnotes[C]{\smallbreak\pagebreak[2]}\Standort{DLA, A:Schnitzler, HS.NZ85.1.3604.}
\physDesc{Brief, 1 Blatt, 4 Seiten, 986 Zeichen
\newline{}Handschrift: schwarze Tinte, deutsche Kurrent
\newline{}Schnitzler: mit rotem Buntstift eine Unterstreichung }\toendnotes[C]{\smallbreak}
\pstart
           \raggedleft{}{\pb}Prag\oindex{Prag@\textbf{Prag}, \emph{A.ADM1}|pw}{ }24/I 93\pend
           
\pstart{}Lieber Schnitzler,\pend\vspace{0.5em}
\pstart
           ich bin in Prag\oindex{Prag@\textbf{Prag}, \emph{A.ADM1}|pw}; wenn Sie mir was mitzuteilen
               haben: meine Adreſſe ist \textsc{Grand Hotel}\oindex{Grand Hotel Prag@\textbf{Grand Hotel Prag}, \emph{Hotel (K.HTL)}|pw}. Ich bleibe noch mehrere Tage. –\pend
           
\pstart
           Reicher\pwindex{Reicher, Emanuel 18.06.1849 – 15.05.1924@\textsc{Reicher, Emanuel} (18.06.1849 – 15.05.1924), \emph{Schauspieler/Schauspielerin}|pw} bat mich, Ihnen zu ſchreiben, daß er
               von Blumenthal\pwindex{Blumenthal, Oskar 13.03.1852 – 24.04.1917@\textsc{Blumenthal, Oskar} (13.03.1852 – 24.04.1917), \emph{Schriftsteller/Schriftstellerin, Journalist/Journalistin, Theaterleiter/Theaterleiterin}|pw} die beſtimmte Zuſicherung
               erhalten, daß Ihr Stück\pwindex{Maerchen. Schauspiel in drei Aufzuegen@\emph{Das Märchen. Schauspiel in drei Aufzügen}|pwv} bis
               längſtens {\pb}im April in Berlin\oindex{Berlin@\textbf{Berlin}, \emph{P.PPLC}|pw} zur Aufführung ko{\geminationm}t.\pend
           
\pstart
           Ferner kann ich Ihnen mittheilen, daſs Ihre »Frage an
                  das Schickſal\pwindex{Frage an das Schicksal@\emph{Die Frage an das Schicksal}|pw}« nächsten Tage \introOben{}(2 Februar)\introOben{} in Hamburg\oindex{Hamburg@\textbf{Hamburg}, \emph{P.PPLA}|pw} (in der Freien \textsc{literarischen} Geſellschaft\orgindex{Freie literarische Gesellschaft Hamburg@Freie literarische Gesellschaft Hamburg|pw}) u. Mitte \introOben{}(16.)\introOben{}{ }Februar in Königsberg\oindex{Kaliningrad@\textbf{Kaliningrad}, \emph{P.PPLA}|pw} zum Vortrag
               gelangt: beidemale durch Reicher\pwindex{Reicher, Emanuel 18.06.1849 – 15.05.1924@\textsc{Reicher, Emanuel} (18.06.1849 – 15.05.1924), \emph{Schauspieler/Schauspielerin}|pw}.\pend
           
\pstart
           Sonntag habe ich die »\label{K_L00162-1v}\edtext{Gläubiger\pwindex{Glaeubiger. Schauspiel in einem Act@\emph{Gläubiger. Schauspiel in einem Act}|pw}-\textsc{Pre{\pb}mière}}{\lemma{\textnormal{\emph{Gläubiger-Première}}}\Cendnote{\textnormal{Die Premiere von \emph{Gläubiger}\pwindex{Glaeubiger. Schauspiel in einem Act@\emph{Gläubiger. Schauspiel in einem Act}|pwk} fand zusammen mit zwei anderen Einaktern\pwindex{Vor dem Tode@\emph{Vor dem Tode}|pwkv}\pwindex{Herbstzeichen@\emph{Herbstzeichen}|pwkv} von Strindberg\pwindex{Strindberg, August 22.01.1849 – 14.05.1912@\textsc{Strindberg, August} (22.01.1849 – 14.05.1912), \emph{Schriftsteller/Schriftstellerin}|pwk} am 22. 1. 1893 am
                  \emph{Residenztheater}\orgindex{Residenz-Theater Berlin@Residenz-Theater Berlin|pwk} in Berlin\oindex{Berlin@\textbf{Berlin}, \emph{P.PPLC}|pwk} statt.}}}\label{K_L00162-1} mitgemacht: ein gewaltiger Eindruck.\pend
           
\pstart
           Auch die \label{K_L00162-2v}\edtext{Baumeister \textsc{Solneß}\pwindex{Baumeister Solness@\emph{Baumeister Solness}|pw}-\textsc{Première}}{\lemma{\textnormal{\emph{Baumeister Solneß-Première}}}\Cendnote{\textnormal{Die Premiere fand am 19. 1. 1893 im \emph{Deutschen Theater}\orgindex{Deutsches Theater Berlin@Deutsches Theater Berlin|pwk} in Berlin\oindex{Berlin@\textbf{Berlin}, \emph{P.PPLC}|pwk} statt.
               }}}\label{K_L00162-2} war ein bedeutſames Erlebnis.\pend
           
\pstart
           Was ich in Berlin\oindex{Berlin@\textbf{Berlin}, \emph{P.PPLC}|pw}{ }\introOben{}machte oder\introOben{} mache? Ein gütiges Schickſal, in Geſtalt eines
                  \uline{lieben Mannes\pwindex{?? [Bekannter von E. M. Kafka] @\textsc{?? [Bekannter von E. M. Kafka]}|pwv}}, hat mich dahin \strikeout{ge} entführt. Nächſtens {\pb}übrigens können Sie auch aus einer \uline{anderen Welt} auf ein Lebenszeichen von mir rechnen.
               Vorher \substVorne{}\textsuperscript{aber}\substDazwischen{}allerdings\substHinten{} will ich Sie \introOben{}aber\introOben{} noch vom \textsc{Nordcap}\oindex{Nordkap@\textbf{Nordkap}, \emph{Kap (N.KAP)}|pw} grüßen. Nächſtens!\pend
           
\pstart
           \textsc{Servus}! Mit herzlichen Grüßen{\\[\baselineskip]}Ihr Sie hochſchätzender{\\[\baselineskip]}\spacefill\mbox{Kafka}\pend
           \leftskip=0em{}\selectlanguage{ngerman}\endnumbering\briefempfaengerindex{Schnitzler, Arthur@\textsc{Schnitzler, Arthur}!zzzKafka, Eduard Michael@\emph{von Eduard Michael Kafka}!1893-01-241@{24. 1. 1893}|)be}\mylabel{L00162h}  \normalsize

\doendnotes{C}
\bigskip
\vfill

\clearpage

\footnotesize

\lohead{\textsc{register}}

% Definiere theindex-Environment komplett neu ohne reledmac
\makeatletter
\renewenvironment{theindex}{%
  \section*{\indexname}%
  \setlength{\parindent}{0pt}%
  \setlength{\parskip}{0pt plus 0.3pt}%
  \let\item\@idxitem
}{%
  \clearpage
}
\makeatother

\IfFileExists{\jobname-pw.ind}{\input{\jobname-pw.ind}}{}

\end{document}

      