%% latex-leseansicht-vorspann.tex
%% Vorspann für die Leseansicht.
%% Lädt die gemeinsame Datei latex-vorspann.tex mit nicht gesetztem Schalter.

\newif\ifkorrekturansicht
\korrekturansichtfalse

\input{../tex-inputs/latex-vorspann}


         
         \newcommand{\erwaehntePersonen}{Personen:  ?? [Bekannter von E. M. Kafka], Oskar Blumenthal, Emanuel Reicher, August Strindberg}
         \newcommand{\erwaehnteInstitutionen}{Institutionen: Freie literarische Gesellschaft Hamburg}
         \newcommand{\erwaehnteOrte}{Orte: Berlin, Deutsches Theater Berlin, Grand Hotel Prag, Hamburg, Kaliningrad, Nordkap, Prag, Residenztheater Berlin, Wien}
         \newcommand{\erwaehnteWerke}{Werke: Baumeister Solness, Das Märchen. Schauspiel in drei Aufzügen, Die Frage an das Schicksal, Gläubiger, Herbstzeichen, Vor dem Tode}
               \section[Eduard Michael Kafka an Arthur Schnitzler, 24. 1. 1893]{ Eduard Michael Kafka an Arthur Schnitzler, 24. 1. 1893}\nopagebreak\mylabel{v}\rehead{ }\begin{ledgroupsized}[t]{13cm}\normalsize\beginnumbering \toendnotes[C]{\smallbreak\pagebreak[2]} \Standort{DLA, A:Schnitzler, HS.NZ85.1.3604.}
\physDesc{Brief, 1 Blatt, 4 Seiten
\newline{}Handschrift: schwarze Tinte, deutsche Kurrent
\newline{}Schnitzler: mit rotem Buntstift eine Unterstreichung }\toendnotes[C]{\smallbreak}\pstart
           \raggedleft{}{\pb}Prag\oindex{Prag@\textbf{Prag}|pw}{ }24/I 93\pend
           \pstart{}Lieber Schnitzler,\pend\pstart
           ich bin in Prag\oindex{Prag@\textbf{Prag}|pw}; wenn Sie mir was mitzuteilen haben:
               meine Adreſſe ist \textsc{Grand Hotel}\oindex{Grand Hotel Prag@\textbf{Grand Hotel Prag}|pw}. Ich bleibe noch mehrere Tage. –\pend
           \pstart
           Reicher\pwindex{Reicher, Emanuel 18.06.1849 – 15.05.1924@\textsc{Reicher, Emanuel} (18.06.1849 – 15.05.1924), \emph{Schauspieler}|pw} bat mich, Ihnen zu ſchreiben, daß er von
                  Blumenthal\pwindex{Blumenthal, Oskar 13.03.1852 – 24.04.1917@\textsc{Blumenthal, Oskar} (13.03.1852 – 24.04.1917), \emph{Schriftsteller, Journalist, Theaterleiter}|pw} die beſtimmte Zuſicherung erhalten,
               daß Ihr Stück\pwindex{Schnitzler, Arthur 15.05.1862 – 21.10.1931@\textsc{Schnitzler, Arthur} (15.05.1862 – 21.10.1931), \emph{Schriftsteller, Mediziner}!Maerchen. Schauspiel in drei Aufzuegen1893-12-01@\strich\emph{Das Märchen. Schauspiel in drei Aufzügen} {[}1893-12-01{]}|pwv} bis längſtens {\pb}im April in Berlin\oindex{Berlin@\textbf{Berlin}|pw} zur Aufführung ko{\geminationm}t.\pend
           \pstart
           Ferner kann ich Ihnen mittheilen, daſs Ihre »Frage an
                  das Schickſal\pwindex{Schnitzler, Arthur 15.05.1862 – 21.10.1931@\textsc{Schnitzler, Arthur} (15.05.1862 – 21.10.1931), \emph{Schriftsteller, Mediziner}!Frage an das Schicksal01. 05. 1890@\strich\emph{Die Frage an das Schicksal} {[}01. 05. 1890{]}|pw}« nächsten Tage \introOben{}(2 Februar)\introOben{} in Hamburg\oindex{Hamburg@\textbf{Hamburg}|pw} (in der Freien
                     \textsc{literarischen} Geſellschaft\orgindex{Freie literarische Gesellschaft Hamburg@Freie literarische Gesellschaft Hamburg|pw}) u. Mitte \introOben{}(16.)\introOben{}{ }Februar in Königsberg\oindex{Kaliningrad@\textbf{Kaliningrad}|pw} zum Vortrag
               gelangt: beidemale durch Reicher\pwindex{Reicher, Emanuel 18.06.1849 – 15.05.1924@\textsc{Reicher, Emanuel} (18.06.1849 – 15.05.1924), \emph{Schauspieler}|pw}.\pend
           \pstart
           Sonntag habe ich die »\label{K_L00162_1v}\edtext{Gläubiger\pwindex{Strindberg, August 22.01.1849 – 14.05.1912@\textsc{Strindberg, August} (22.01.1849 – 14.05.1912), \emph{Schriftsteller}!Glaeubiger1892@\strich\emph{Gläubiger} {[}1892{]}|pw}-\textsc{Pre{\pb}mière}}{\lemma{\textnormal{\emph{Gläubiger-Première}}}\Cendnote{\textnormal{Zusammen mit zwei anderen Einaktern\pwindex{Strindberg, August 22.01.1849 – 14.05.1912@\textsc{Strindberg, August} (22.01.1849 – 14.05.1912), \emph{Schriftsteller}!Vor dem Tode22. 1. 1893@\strich\emph{Vor dem Tode} {[}22. 1. 1893{]}|pwkv}\pwindex{Strindberg, August 22.01.1849 – 14.05.1912@\textsc{Strindberg, August} (22.01.1849 – 14.05.1912), \emph{Schriftsteller}!Herbstzeichen22. 1. 1893@\strich\emph{Herbstzeichen} {[}22. 1. 1893{]}|pwkv} von Strindberg\pwindex{Strindberg, August 22.01.1849 – 14.05.1912@\textsc{Strindberg, August} (22.01.1849 – 14.05.1912), \emph{Schriftsteller}|pwk} am 22. 1. 1893 im Residenztheater\oindex{Residenztheater Berlin@\textbf{Residenztheater Berlin}|pwk} in Berlin\oindex{Berlin@\textbf{Berlin}|pwk}.}}}\label{K_L00162_1h} mitgemacht: ein gewaltiger Eindruck.\pend
           \pstart
           Auch die \label{K_L00162_2v}\edtext{Baumeister \textsc{Solneß}\pwindex{\textcolor{red}{\textsuperscript{XXXX1 indx}}!Baumeister Solness1892@\strich\emph{Baumeister Solness} {[}1892{]}|pw}-\textsc{Première}}{\lemma{\textnormal{\emph{Baumeister Solneß-Première}}}\Cendnote{\textnormal{am 19. 1. 1893 am Deutschen Theater\oindex{Deutsches Theater Berlin@\textbf{Deutsches Theater Berlin}|pwk} in Berlin\oindex{Berlin@\textbf{Berlin}|pwk}}}}\label{K_L00162_2h} war ein bedeutſames Erlebnis.\pend
           \pstart
           Was ich in Berlin\oindex{Berlin@\textbf{Berlin}|pw}{ }\introOben{}machte oder\introOben{} mache? Ein gütiges Schickſal, in Geſtalt eines
                  \uline{lieben Mannes\pwindex{?? [Bekannter von E. M. Kafka] @\textsc{?? [Bekannter von E. M. Kafka]}|pwv}}, hat mich dahin \strikeout{ge} entführt. Nächſtens {\pb}übrigens können Sie auch aus einer \uline{anderen Welt} auf ein Lebenszeichen von mir rechnen.
               Vorher \substVorne{}\textsuperscript{aber}\substDazwischen{}allerdings\substHinten{} will ich Sie \introOben{}aber\introOben{} noch vom \textsc{Nordcap}\oindex{Nordkap@\textbf{Nordkap}|pw} grüßen. Nächſtens!\pend
           \pstart
           \textsc{Servus}! Mit herzlichen Grüßen{\\[\baselineskip]}Ihr Sie hochſchätzender{\\[\baselineskip]}\spacefill\mbox{Kafka}\pend
           \leftskip=0em{}
         
         \endnumbering\mylabel{h}\end{ledgroupsized}  \newcommand{\dateiname}{L00162}\newcommand{\titel}{Eduard Michael Kafka an Arthur Schnitzler, 24. 1. 1893}\newcommand{\editorInnen}{Martin Anton Müller und Gerd-Hermann Susen}%% latex-leseansicht-abspann.tex
%% Abspann für die Leseansicht.
%% Der Schalter \ifkorrekturansicht ist bereits durch den Vorspann gesetzt.

%% latex-abspann.tex
%% Gemeinsamer Abspann für Korrekturansicht und Leseansicht.
%% Setzt den Schalter \ifkorrekturansicht voraus (gesetzt in den
%% einbindenden Dateien latex-korrekturansicht-abspann.tex bzw.
%% latex-leseansicht-abspann.tex).
%% ---------------------------------------------------------------

\normalsize

% Das esempio-Environment wird nur in der Leseansicht benötigt
\ifkorrekturansicht\else
\newenvironment{esempio}[3]%
{
    \vspace{1.5ex}
    \rlap{\underline{#1}}
    \par
    \setlength{\parindent}{0cm}
    \nopagebreak
    \leftskip=#2cm
    \rightskip=#3cm
}
{
    \par
}
\fi

\doendnotes{C}
\bigskip
\vfill

\clearpage

\footnotesize

\ifkorrekturansicht
  \lohead{\textsc{register}}
\fi

% theindex-Environment neu definieren ohne reledmac
\makeatletter
\renewenvironment{theindex}{%
  \ifkorrekturansicht
    \section*{\indexname}%
  \else
    \subsubsection*{Index der erwähnten Entitäten}%
  \fi
  \setlength{\parindent}{0pt}%
  \setlength{\parskip}{0pt plus 0.3pt}%
  \let\item\@idxitem
}{%
  \ifkorrekturansicht\clearpage\fi
}
\makeatother

\IfFileExists{\jobname-pw.ind}{\input{\jobname-pw.ind}}{}

% Quellenangabe nur in der Leseansicht
\ifkorrekturansicht\else
% Fallback-Definitionen, falls die .tex-Datei \titel etc. nicht gesetzt hat
\providecommand{\titel}{}
\providecommand{\editorInnen}{}
\providecommand{\dateiname}{\jobname}

\vspace{3cm}

\vfill

\footnotesize
\textsc{Quelle}: \titel. Herausgegeben von {\editorInnen}. In: \emph{Arthur Schnitzler: Briefwechsel mit Autorinnen und Autoren}.
 Digitale Edition, https://schnitzler-briefe.acdh.oeaw.ac.at/{\dateiname}.html (Stand \today)
\fi

\end{document}


      