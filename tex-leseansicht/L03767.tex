%% latex-korrekturansicht-vorspann.tex
%% Vorspann für die Korrekturansicht.
%% Lädt die gemeinsame Datei latex-vorspann.tex mit gesetztem Schalter.

\newif\ifkorrekturansicht
\korrekturansichttrue

\input{../tex-inputs/latex-vorspann}


\section[Olga Schnitzler an Stefan Zweig, 22. 11. 1916]{L03767 Olga Schnitzler an Stefan Zweig, 22. 11. 1916}
\nopagebreak\mylabel{L03767v}
\rehead{ }\normalsize\beginnumbering\briefempfaengerindex{Zweig, Stefan@\textsc{Zweig, Stefan}!zzzSchnitzler, Olga@\emph{von Olga Schnitzler}!1916-11-221@{22. 11. 1916}|(be}
\toendnotes[C]{\smallbreak\pagebreak[2]}\Standort{Jerusalem, National Library of Israel, ARC. Ms. Var. 305 1 58 Stefan Zweig Collection.}
\physDesc{Briefkarte, 1 Blatt, 2 Seiten, 714 Zeichen
\newline{}Handschrift: schwarze Tinte, lateinische Kurrent}\toendnotes[C]{\smallbreak}
\pstart
           \raggedleft{}{\pb}22. Nov. 1916. \pend
           \vspace{0.5em}
\pstart
           Lieber Herr Doctor, Ihr \label{K_L03767-1v}\edtext{Brief}{\lemma{\textnormal{\emph{Brief}}}\Cendnote{\textnormal{nicht überliefert}}}\label{K_L03767-1} kam, wie ich den \label{K_L03767-2v}\edtext{meinen}{\lemma{\textnormal{\emph{meinen}}}\Cendnote{\textnormal{Olga Schnitzler an Stefan Zweig, 20. 11. 1916.}}}\label{K_L03767-2} eben
               abgeschickt hatte. Ich kann nichts sagen als: ich danke Ihnen. Wüssten Sie, wie es
               mich berührt, wenn mir einmal Jemand meinen »Eigensinn« nicht zum Vorwurf macht. Das
               ist bisher nicht oft geschehen. Aber ich habe {\pb}kein
               Verdienst: meine Arbeit war nur Lebenselement, Quell aller Freudigkeit, trotz so
               vieler schwerer Stunden des Zweifels – was Wunder wenn ich sie, wenn sie mich nicht
               losgelassen hat?!\pend
           
\pstart
           Auch dieser Abend\eventindex{Wiener Konzerthaus@\textbf{Wiener Konzerthaus}!Gesangskonzert von Olga Schnitzler, 18.11.1916@Gesangskonzert von Olga Schnitzler, 18.11.1916|pwv}: nur ein Schritt weiter. Jetzt freu ich mich schon unsagbar auf
               alle herrlichen Lieder, die ich gleich – morgen –, neu studieren werde.\pend
           
\pstart
           Seien Sie herzlichst gegrüsst und auf Wiedersehen!{\\[\baselineskip]}Arthur grüsst Sie bestens!{\\[\baselineskip]}Ihre
                  \spacefill\mbox{OlgaSchnitzler.}\pend
           \leftskip=0em{}\selectlanguage{ngerman}\endnumbering\briefempfaengerindex{Zweig, Stefan@\textsc{Zweig, Stefan}!zzzSchnitzler, Olga@\emph{von Olga Schnitzler}!1916-11-221@{22. 11. 1916}|)be}\mylabel{L03767h}
\begin{anhang}
\end{anhang}\normalsize

\doendnotes{C}
\bigskip
\vfill

\clearpage

\footnotesize

\lohead{\textsc{register}}

% Definiere theindex-Environment komplett neu ohne reledmac
\makeatletter
\renewenvironment{theindex}{%
  \section*{\indexname}%
  \setlength{\parindent}{0pt}%
  \setlength{\parskip}{0pt plus 0.3pt}%
  \let\item\@idxitem
}{%
  \clearpage
}
\makeatother

\IfFileExists{\jobname-pw.ind}{\input{\jobname-pw.ind}}{}

\end{document}

      