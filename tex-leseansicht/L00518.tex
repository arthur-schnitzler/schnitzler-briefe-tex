%% latex-leseansicht-vorspann.tex
%% Vorspann für die Leseansicht.
%% Lädt die gemeinsame Datei latex-vorspann.tex mit nicht gesetztem Schalter.

\newif\ifkorrekturansicht
\korrekturansichtfalse

\input{../tex-inputs/latex-vorspann}


         
         \renewcommand{\erwaehntePersonen}{Personen: Hugo von Hofmannsthal}
         \renewcommand{\erwaehnteOrte}{Orte: Wien}
         \renewcommand{\erwaehnteWerke}{Werke: Das Märchen der 672. Nacht}
               \section[Arthur Schnitzler an Hugo von Hofmannsthal, 26. 11. 1895]{ Arthur Schnitzler an Hugo von Hofmannsthal, 26. 11. 1895}\nopagebreak\mylabel{v}\rehead{ }\begin{ledgroupsized}[t]{13cm}\normalsize\beginnumbering \toendnotes[C]{\smallbreak\pagebreak[2]} \Standort{FDH, Hs-30885,47.}
\physDesc{Brief, 1 Blatt, 3 Seiten
\newline{}Handschrift: Bleistift, deutsche Kurrent
\newline{}Hofmannsthal: mit rotem Buntstift mit einem »X« markiert }\buchAbdrucke{\weitereDrucke{Hugo von Hofmannsthal, Arthur Schnitzler: \emph{Briefwechsel}. Hg. Therese Nickl und Heinrich Schnitzler. Frankfurt am Main: \emph{S. Fischer} 1964, S. 63–64.} }\toendnotes[C]{\smallbreak}\pstart
           \raggedleft{}{\pb}26. 11. 95.\pend
           \pstart
           Lieber Hugo, eben hab ich den Kaufmannsſohn\pwindex{Hofmannsthal, Hugo von 1874-02-01 – 1929-07-15@\textsc{Hofmannsthal, Hugo von} (1874-02-01 – 1929-07-15), \emph{Schriftsteller}!Maerchen der 672. Nacht2.11.1895 – 16.11.1895@\strich\emph{Das Märchen der 672. Nacht} {[}2.11.1895 – 16.11.1895{]}|pw} geleſen. Folgendes find ich: die Geſchichte hat nichts von der
               Wärme und dem Glanz eines Märchens, wohl aber in wunderbarer Weiſe das fahle Licht
               des Traums, deſſen räthſelhafte wie verwiſchte Uebergänge und das eigene Gemiſch von
               Deutlichkeit der geringen und Bläſſe der beſondern Dinge, das eben dem Traum zuko{\geminationm}t. Sobald ich mir die Erlebniſſe des Kaufm.ſ.\pwindex{Hofmannsthal, Hugo von 1874-02-01 – 1929-07-15@\textsc{Hofmannsthal, Hugo von} (1874-02-01 – 1929-07-15), \emph{Schriftsteller}!Maerchen der 672. Nacht2.11.1895 – 16.11.1895@\strich\emph{Das Märchen der 672. Nacht} {[}2.11.1895 – 16.11.1895{]}|pwv} als Traum vorſtelle, werden ſie mir
               höchſt ergreifend; denn es gibt ſolche Träume, ſie ſind eigentlich auch Schickſale,
               und man könnte verſtehen, daſs ſich Menſchen, die von ſolchen Träumen geplagt {\pb}werden, aus Verzweiflung umbringen. Auch iſt nicht zu
               vergeſſen: die Empfindungen des Kaufmannsſohnes ſind wie im Traum geſchildert; die
               unſägliche Unheimlichkeit, die irgend ein Weg, ein Kindergeſicht, eine Thür annehmen
               kann, wenn man ſie träumt, finden kaum im wachen Leben ein Analogon. Ihre tiefere
               Bedeutung verliert die Geſchichte durchaus nicht, wenn der Kaufma{\geminationn}sſoh{[}n{]}\pwindex{Hofmannsthal, Hugo von 1874-02-01 – 1929-07-15@\textsc{Hofmannsthal, Hugo von} (1874-02-01 – 1929-07-15), \emph{Schriftsteller}!Maerchen der 672. Nacht2.11.1895 – 16.11.1895@\strich\emph{Das Märchen der 672. Nacht} {[}2.11.1895 – 16.11.1895{]}|pwv} aus ihr erwacht ſtatt a\strikeout{u}n ihr zu ſterben; ich
               würd ihn ſogar mehr beklagen; denn das tödtliche fühlen wir beſſer mit als den Tod. –
               Ich will mit alldem {\pb}nicht ſagen, daſs mir \introOben{}nicht\introOben{} auch ein Märchen desſelben Inhalts, \uline{ganz} desſelben \strikeout{zu}recht wäre; aber Sie
               haben die Geſchichte\pwindex{Hofmannsthal, Hugo von 1874-02-01 – 1929-07-15@\textsc{Hofmannsthal, Hugo von} (1874-02-01 – 1929-07-15), \emph{Schriftsteller}!Maerchen der 672. Nacht2.11.1895 – 16.11.1895@\strich\emph{Das Märchen der 672. Nacht} {[}2.11.1895 – 16.11.1895{]}|pwv} beſti{\geminationm}t als Traum erzählt; – erinnere ich mich jetzt zurück,
               ſo ſehe ich den Kaufma{\geminationn}sſohn im Bett \strikeout{sich}{ }ſtöhnend ſich wälzen, und er thut mir ſehr leid.
               –\pend
           \pstart
           Damit wäre auch alles \substVorne{}\textsuperscript{\textcolor{gray}{×}\-\textcolor{gray}{×}\-\textcolor{gray}{×}{ }\textcolor{gray}{×}\-\textcolor{gray}{×}\-\textcolor{gray}{×}\-\textcolor{gray}{×}\-\textcolor{gray}{×}\-\textcolor{gray}{×}\-\textcolor{gray}{×}\-\textcolor{gray}{×}\-\textcolor{gray}{×}}\substDazwischen{}zum Vorzug gewandelt\substHinten{}, was ſonſt befremden müßte: eine ſeltſame Trockenheit, etwas
               hinſchleichendes im Stil – was die Stimmung des Traums unvergleichlich malt, der
               Märchenwirklichkeit aber zum Nachtheil iſt.\pend
           \pstart
           Viele herzliche Grüße. Es wird ſich noch manches ſagen laſſen.\pend
           \pstart Ihr \spacefill\mbox{Arthur}\pend{}
         
         \endnumbering\mylabel{h}\end{ledgroupsized}  \newcommand{\dateiname}{L00518}\newcommand{\titel}{Arthur Schnitzler an Hugo von Hofmannsthal, 26. 11. 1895}\newcommand{\editorInnen}{Martin Anton Müller und Gerd-Hermann Susen}%% latex-leseansicht-abspann.tex
%% Abspann für die Leseansicht.
%% Der Schalter \ifkorrekturansicht ist bereits durch den Vorspann gesetzt.

%% latex-abspann.tex
%% Gemeinsamer Abspann für Korrekturansicht und Leseansicht.
%% Setzt den Schalter \ifkorrekturansicht voraus (gesetzt in den
%% einbindenden Dateien latex-korrekturansicht-abspann.tex bzw.
%% latex-leseansicht-abspann.tex).
%% ---------------------------------------------------------------

\normalsize

% Das esempio-Environment wird nur in der Leseansicht benötigt
\ifkorrekturansicht\else
\newenvironment{esempio}[3]%
{
    \vspace{1.5ex}
    \rlap{\underline{#1}}
    \par
    \setlength{\parindent}{0cm}
    \nopagebreak
    \leftskip=#2cm
    \rightskip=#3cm
}
{
    \par
}
\fi

\doendnotes{C}
\bigskip
\vfill

\clearpage

\footnotesize

\ifkorrekturansicht
  \lohead{\textsc{register}}
\fi

% theindex-Environment neu definieren ohne reledmac
\makeatletter
\renewenvironment{theindex}{%
  \ifkorrekturansicht
    \section*{\indexname}%
  \else
    \subsubsection*{Index der erwähnten Entitäten}%
  \fi
  \setlength{\parindent}{0pt}%
  \setlength{\parskip}{0pt plus 0.3pt}%
  \let\item\@idxitem
}{%
  \ifkorrekturansicht\clearpage\fi
}
\makeatother

\IfFileExists{\jobname-pw.ind}{\input{\jobname-pw.ind}}{}

% Quellenangabe nur in der Leseansicht
\ifkorrekturansicht\else
% Fallback-Definitionen, falls die .tex-Datei \titel etc. nicht gesetzt hat
\providecommand{\titel}{}
\providecommand{\editorInnen}{}
\providecommand{\dateiname}{\jobname}

\vspace{3cm}

\vfill

\footnotesize
\textsc{Quelle}: \titel. Herausgegeben von {\editorInnen}. In: \emph{Arthur Schnitzler: Briefwechsel mit Autorinnen und Autoren}.
 Digitale Edition, https://schnitzler-briefe.acdh.oeaw.ac.at/{\dateiname}.html (Stand \today)
\fi

\end{document}


      