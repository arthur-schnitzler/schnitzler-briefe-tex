%% latex-korrekturansicht-vorspann.tex
%% Vorspann für die Korrekturansicht.
%% Lädt die gemeinsame Datei latex-vorspann.tex mit gesetztem Schalter.

\newif\ifkorrekturansicht
\korrekturansichttrue

\input{../tex-inputs/latex-vorspann}


\section[Hugo von Hofmannsthal an Arthur Schnitzler, 1. 2. {[}1893{]}]{L00171 Hugo von Hofmannsthal an Arthur Schnitzler, 1. 2. {[}1893{]}}
\nopagebreak\mylabel{L00171v}
\rehead{ }\normalsize\beginnumbering\briefempfaengerindex{Schnitzler, Arthur@\textsc{Schnitzler, Arthur}!zzzHofmannsthal, Hugo von@\emph{von Hugo von Hofmannsthal}!1893-02-013@{1. 2. {[}1893{]}}|(be}
\toendnotes[C]{\smallbreak\pagebreak[2]}\Standort{CUL, Schnitzler, B 43.}
\physDesc{Brief, 1 Blatt, 2 Seiten, 816 Zeichen
\newline{}Handschrift: schwarze Tinte, deutsche Kurrent
\newline{}Schnitzler: mit Bleistift nummeriert: »43« und umdatiert zu: »1. III.« }
\buchAbdrucke{\weitereDrucke{1) Hugo von Hofmannsthal, Arthur Schnitzler: \emph{Briefwechsel}. Frankfurt am Main: \emph{S. Fischer} 1964, S. 35–36.} \weitereDrucke{2) Hermann Bahr, Arthur Schnitzler: \emph{Briefwechsel, Aufzeichnungen, Dokumente (1891–1931)}. Göttingen: \emph{Wallstein} 2018, S. 33.} }
\pstart
           \raggedleft{}{\pb}1. II\pend
           
\pstart{}lieber Arthur.\pend\vspace{0.5em}
\pstart
           Bahr\pwindex{Bahr, Hermann 19.07.1863 – 15.01.1934@\textsc{Bahr, Hermann} (19.07.1863 – 15.01.1934), \emph{Schriftsteller/Schriftstellerin, Kritiker/Kritikerin}|pw}{ }ſtellte mir zu meiner Freude folgenden Antrag: er
               ſei im Stande und gern bereit, Fels\pwindex{Fels, Friedrich Michael *~1864@\textsc{Fels, Friedrich Michael} (*~1864), \emph{Journalist/Journalistin}|pw} von
                  Anfang März an mit einem Gehalt von 100 fl in der Deutſchen Zeitung\orgindex{Deutsche Zeitung@Deutsche Zeitung|pw} als Redacteur unterzubringen. Es handelt ſich
               nur um Fähigkeit und Bereitwilligkeit. Dritten Perſonen werden Sie es vorläufig
               ebenſowenig erzählen, wie ich.\pend
           
\pstart
           Falls wir Sonntag bei Ihnen Zuſammenkommen, zu welchem {\pb}Zweck ich wenigſtens vorläufig
               eine Einladung abgelehnt habe, ſeien Sie doch ſogut, Robert Ehrhardt\pwindex{Ehrhart-Ehrhartstein, Robert 12.09.1870 – 11.11.1956@\textsc{Ehrhart-Ehrhartstein, Robert} (12.09.1870 – 11.11.1956), \emph{Schriftsteller/Schriftstellerin, Ministerialbeamter/Ministerialbeamte}|pw} (\textsc{V. Siebenbrunng.\oindex{Siebenbrunnengasse@\textbf{Siebenbrunnengasse}, \emph{Straße (K.STR)}|pw} 29}) ausdrücklich einzuladen. Er geht der Trauer wegen
               faſt nicht in Geſellschaft und würde gewiſs gern kommen.\pend
           
\pstart
           Herzlichſt\hspace*{2.5em}Ihr{\\[\baselineskip]}\spacefill\mbox{Loris.}\pend
           \leftskip=0em{}
\pstart
           \noindent{}P. S.{\\}Ich denke ſehr oft an die Novelle vom Sterben\pwindex{Sterben. Novelle@\emph{Sterben. Novelle}|pw} und möchte viel mehr davon reden, als geſchieht. Sie haben was
                  gegen die Geſchichte. Wenigſtens ſcheinen Sie ſie todtſchweigen zu wollen.\pend
           \selectlanguage{ngerman}\endnumbering\briefempfaengerindex{Schnitzler, Arthur@\textsc{Schnitzler, Arthur}!zzzHofmannsthal, Hugo von@\emph{von Hugo von Hofmannsthal}!1893-02-013@{1. 2. {[}1893{]}}|)be}\mylabel{L00171h}  \normalsize

\doendnotes{C}
\bigskip
\vfill

\clearpage

\footnotesize

\lohead{\textsc{register}}

% Definiere theindex-Environment komplett neu ohne reledmac
\makeatletter
\renewenvironment{theindex}{%
  \section*{\indexname}%
  \setlength{\parindent}{0pt}%
  \setlength{\parskip}{0pt plus 0.3pt}%
  \let\item\@idxitem
}{%
  \clearpage
}
\makeatother

\IfFileExists{\jobname-pw.ind}{\input{\jobname-pw.ind}}{}

\end{document}

      