%% latex-leseansicht-vorspann.tex
%% Vorspann für die Leseansicht.
%% Lädt die gemeinsame Datei latex-vorspann.tex mit nicht gesetztem Schalter.

\newif\ifkorrekturansicht
\korrekturansichtfalse

\input{../tex-inputs/latex-vorspann}


\section[Hugo von Hofmannsthal an Arthur Schnitzler, 1. 2. [1893]]{L00171 Hugo von Hofmannsthal an Arthur Schnitzler, 1. 2. [1893]}
\nopagebreak\mylabel{L00171v}
\rehead{ }\normalsize\beginnumbering\briefempfaengerindex{Schnitzler, Arthur@\textsc{Schnitzler, Arthur}!zzzHofmannsthal, Hugo von@\emph{von Hugo von Hofmannsthal}!1893-02-013@{1. 2. [1893]}|(be}
\toendnotes[C]{\smallbreak\pagebreak[2]}
\correspDesc{Versand  durch Hugo von Hofmannsthal am 1. 2. [1893] in Wien
\newline{}Erhalt  durch Arthur Schnitzler im Zeitraum [1. 2. 1893
                  – 5. 2. 1893?] in Wien}\toendnotes[C]{\smallbreak}
\Standort{CUL, Schnitzler, B 43.}
\physDesc{Brief, 1 Blatt, 2 Seiten, 816 Zeichen
\newline{}Handschrift: schwarze Tinte, deutsche Kurrent
\newline{}Schnitzler: mit Bleistift nummeriert: »43« und umdatiert zu: »1. III.« }
\buchAbdrucke{\weitereDrucke{1) Hugo von Hofmannsthal, Arthur Schnitzler: \emph{Briefwechsel}. Herausgegeben von Therese Nickl und Heinrich Schnitzler. Frankfurt am Main: \emph{S. Fischer} 1964, S. 35–36.} \weitereDrucke{2) Hermann Bahr, Arthur Schnitzler: \emph{Briefwechsel, Aufzeichnungen, Dokumente (1891–1931)}. Herausgegeben von Kurt Ifkovits und Martin Anton Müller. Göttingen: \emph{Wallstein} 2018, S. 33.} }
\pstart
           \raggedleft{}{\pb}1. II\pend
           
\pstart{}lieber Arthur.\pend\vspace{0.5em}
\pstart
           Bahr\pwindex{Bahr, Hermann 19.\,7.\,1863 Linz – 15.\,1.\,1934 München@\textsc{Bahr, Hermann} (19.\,7.\,1863 Linz – 15.\,1.\,1934 München), \emph{Schriftsteller, Kritiker}|pw}{ }ſtellte mir zu meiner Freude folgenden Antrag: er{ }ſei im Stande und gern bereit, Fels\pwindex{Fels, Friedrich Michael *~1864 Bad Dürkheim@\textsc{Fels, Friedrich Michael} (*~1864 Bad Dürkheim), \emph{Journalist}|pw} von
                  Anfang März an mit einem Gehalt von 100 fl in der Deutſchen Zeitung\orgindex{Deutsche Zeitung@Deutsche Zeitung|pw} als Redacteur unterzubringen. Es handelt{ }ſich
               nur um Fähigkeit und Bereitwilligkeit. Dritten Perſonen werden Sie es vorläufig
               ebenſowenig erzählen, wie ich.\pend
           
\pstart
           Falls wir Sonntag bei Ihnen Zuſammenkommen, zu welchem {\pb}Zweck ich wenigſtens vorläufig
               eine Einladung abgelehnt habe,{ }ſeien Sie doch{ }ſogut, Robert Ehrhardt\pwindex{Ehrhart-Ehrhartstein, Robert 12.\,9.\,1870 Innsbruck – 11.\,11.\,1956 Baden bei Wien@\textsc{Ehrhart-Ehrhartstein, Robert} (12.\,9.\,1870 Innsbruck – 11.\,11.\,1956 Baden bei Wien), \emph{Schriftsteller, Ministerialbeamter}|pw} (\textsc{V. Siebenbrunng.\oindex{Wien@\textbf{Wien}!V., Margareten@\textbf{V., Margareten}!Siebenbrunnengasse@\textbf{Siebenbrunnengasse}, \emph{Straße}|pw} 29}) ausdrücklich einzuladen. Er geht der Trauer wegen
               faſt nicht in Geſellschaft und würde gewiſs gern kommen.\pend
           
\pstart
           Herzlichſt\hspace*{2.5em}Ihr{\\[\baselineskip]}\spacefill\mbox{Loris.}\pend
           \leftskip=0em{}
\pstart
           \noindent{}P. S.{\\}Ich denke{ }ſehr oft an die Novelle vom Sterben\pwindex{Schnitzler, Arthur 15.\,5.\,1862 Wien – 21.\,10.\,1931 ebd.@\textsc{Schnitzler, Arthur} (15.\,5.\,1862 Wien – 21.\,10.\,1931 ebd.), \emph{Schriftsteller, Mediziner}!Sterben. Novelle@\strich\emph{Sterben. Novelle}|pw} und möchte viel mehr davon reden, als geſchieht. Sie haben was
                  gegen die Geſchichte. Wenigſtens{ }ſcheinen Sie{ }ſie todtſchweigen zu wollen.\pend
           \selectlanguage{ngerman}\endnumbering\briefempfaengerindex{Schnitzler, Arthur@\textsc{Schnitzler, Arthur}!zzzHofmannsthal, Hugo von@\emph{von Hugo von Hofmannsthal}!1893-02-013@{1. 2. [1893]}|)be}\mylabel{L00171h}  \newcommand{\dateiname}{L00171}\newcommand{\titel}{Hugo von Hofmannsthal an Arthur Schnitzler, 1. 2. [1893]}\newcommand{\editorInnen}{Herausgegeben von Martin Anton Müller}%% latex-leseansicht-abspann.tex
%% Abspann für die Leseansicht.
%% Der Schalter \ifkorrekturansicht ist bereits durch den Vorspann gesetzt.

%% latex-abspann.tex
%% Gemeinsamer Abspann für Korrekturansicht und Leseansicht.
%% Setzt den Schalter \ifkorrekturansicht voraus (gesetzt in den
%% einbindenden Dateien latex-korrekturansicht-abspann.tex bzw.
%% latex-leseansicht-abspann.tex).
%% ---------------------------------------------------------------

\normalsize

% Das esempio-Environment wird nur in der Leseansicht benötigt
\ifkorrekturansicht\else
\newenvironment{esempio}[3]%
{
    \vspace{1.5ex}
    \rlap{\underline{#1}}
    \par
    \setlength{\parindent}{0cm}
    \nopagebreak
    \leftskip=#2cm
    \rightskip=#3cm
}
{
    \par
}
\fi

\doendnotes{C}
\bigskip
\vfill

\clearpage

\footnotesize

\ifkorrekturansicht
  \lohead{\textsc{register}}
\fi

% theindex-Environment neu definieren ohne reledmac
\makeatletter
\renewenvironment{theindex}{%
  \ifkorrekturansicht
    \section*{\indexname}%
  \else
    \subsubsection*{Index der erwähnten Entitäten}%
  \fi
  \setlength{\parindent}{0pt}%
  \setlength{\parskip}{0pt plus 0.3pt}%
  \let\item\@idxitem
}{%
  \ifkorrekturansicht\clearpage\fi
}
\makeatother

\IfFileExists{\jobname-pw.ind}{\input{\jobname-pw.ind}}{}

% Quellenangabe nur in der Leseansicht
\ifkorrekturansicht\else
% Fallback-Definitionen, falls die .tex-Datei \titel etc. nicht gesetzt hat
\providecommand{\titel}{}
\providecommand{\editorInnen}{}
\providecommand{\dateiname}{\jobname}

\vspace{3cm}

\vfill

\footnotesize
\textsc{Quelle}: \titel. Herausgegeben von {\editorInnen}. In: \emph{Arthur Schnitzler: Briefwechsel mit Autorinnen und Autoren}.
 Digitale Edition, https://schnitzler-briefe.acdh.oeaw.ac.at/{\dateiname}.html (Stand \today)
\fi

\end{document}


