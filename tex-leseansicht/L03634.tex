%% latex-leseansicht-vorspann.tex
%% Vorspann für die Leseansicht.
%% Lädt die gemeinsame Datei latex-vorspann.tex mit nicht gesetztem Schalter.

\newif\ifkorrekturansicht
\korrekturansichtfalse

\input{../tex-inputs/latex-vorspann}


\section[Stefan Zweig an Arthur Schnitzler, {{[}}zwischen 5. und 12. 6. 1911?{{]}}]{L03634 Stefan Zweig an Arthur Schnitzler, {[}zwischen 5. und 12. 6. 1911?{]}}
\nopagebreak\mylabel{L03634v}
\rehead{ }\normalsize\beginnumbering\briefempfaengerindex{Schnitzler, Arthur@\textsc{Schnitzler, Arthur}!zzzZweig, Stefan@\emph{von Stefan Zweig}!1911-06-121@{{[}zwischen 5. und 12. 6. 1911?{]}}|(be}
\toendnotes[C]{\smallbreak\pagebreak[2]}
\correspDesc{Versand  durch Stefan Zweig im Zeitraum [zwischen 5. und
                  12. 6. 1911?] in Wien
\newline{}Erhalt  durch Arthur Schnitzler im Zeitraum [zwischen 5. und
                     12. 6. 1911?] in Wien}\toendnotes[C]{\smallbreak}
\Standort{CUL, Schnitzler, B 118.}
\physDesc{Kartenbrief, 494 Zeichen
\newline{}Handschrift: schwarze Tinte, lateinische Kurrent}
\buchAbdrucke{\weitereDrucke{Stefan Zweig: \emph{Briefwechsel mit Hermann Bahr, Sigmund Freud, Rainer Maria
                        Rilke und Arthur Schnitzler}. Herausgegeben von Jeffrey B. Berlin, Hans-Ulrich Lindken und Donald A. Prater. Frankfurt am Main: \emph{S. Fischer} 1987, S. 364.} }\toendnotes[C]{\smallbreak}\pstart{}{\pb}Herrn D\textsuperscript{r}
                  Artur Schnitzler\pend{}\pstart{}Wien – Cottage\oindex{Wien@\textbf{Wien}!XVIII., Währing@\textbf{XVIII., Währing}!Währinger Cottage@\textbf{Währinger Cottage}, \emph{Teil eines besiedelten Ortes}|pw}\pend{}\pstart{}\label{K_L03634-1v}\edtext{Sternwartestrasse 72}{\lemma{\textnormal{\emph{Sternwartestrasse 72}}}\Cendnote{\textnormal{Zweig\pwindex{Zweig, Stefan 28.\,11.\,1881 Wien – 23.\,2.\,1942 Petrópolis@\textsc{Zweig, Stefan} (28.\,11.\,1881 Wien – 23.\,2.\,1942 Petrópolis), \emph{Schriftsteller}|pwk} wechselt bei der Adressierung
                        seiner Schreiben an Schnitzler immer
                        wieder zwischen der falschen Hausnummer »72« und der
                        richtigen »71«.}}}\label{K_L03634-1}\oindex{Wien@\textbf{Wien}!XVIII., Währing@\textbf{XVIII., Währing}!Sternwartestraße 71@\textbf{Sternwartestraße 71}, \emph{Wohngebäude}|pw}\pend{}{\bigskip}\vspace{1em}
\pstart
           {\pb}VIII. Kochgasse 8\oindex{Wien@\textbf{Wien}!VIII., Josefstadt@\textbf{VIII., Josefstadt}!Kochgasse 8@\textbf{Kochgasse 8}, \emph{Wohngebäude}|pw}\pend
           {\vspace{1\baselineskip}}
\pstart{}Sehr verehrter Herr Doktor,\pend\vspace{0.5em}
\pstart
           \label{K_L03634-2v}\edtext{meine Amerika\oindex{Amerika@\textbf{Amerika}|pw}reise}{\lemma{\textnormal{\emph{meine Amerikareise}}}\Cendnote{\textnormal{Vom
                     22. 2. 1911 bis zum 21. 4. 1911 unternahm Stefan Zweig\pwindex{Zweig, Stefan 28.\,11.\,1881 Wien – 23.\,2.\,1942 Petrópolis@\textsc{Zweig, Stefan} (28.\,11.\,1881 Wien – 23.\,2.\,1942 Petrópolis), \emph{Schriftsteller}|pwk} eine amerikanische\oindex{Amerika@\textbf{Amerika}|pwk} Reise, beginnend in New York\oindex{New York City@\textbf{New York City}|pwk}. Von dort reiste er in mehrere Städte an der nordamerikanischen\oindex{Nordamerika@\textbf{Nordamerika}, \emph{Region}|pwk} Ostküste, dann nach Chicago\oindex{Chicago@\textbf{Chicago}, \emph{Hauptstadt}|pwk} und Kanada\oindex{Kanada@\textbf{Kanada}|pwk}, um über Bermuda\oindex{Bermuda@\textbf{Bermuda}, \emph{Exterritoriales Gebiet}|pwk} und Kuba\oindex{Kuba@\textbf{Kuba}|pwk} bis nach Südamerika\oindex{Südamerika@\textbf{Südamerika}|pwk} zu gelangen.}}}\label{K_L03634-2} und dann eine ärgerliche
               langwierige kleine \label{K_L03634-3v}\edtext{Operation}{\lemma{\textnormal{\emph{Operation}}}\Cendnote{\textnormal{Stefan Zweig\pwindex{Zweig, Stefan 28.\,11.\,1881 Wien – 23.\,2.\,1942 Petrópolis@\textsc{Zweig, Stefan} (28.\,11.\,1881 Wien – 23.\,2.\,1942 Petrópolis), \emph{Schriftsteller}|pwk} musste sich im Mai 1911 wegen einer Rippenfellentzündung einer Operation
                  unterziehen.}}}\label{K_L03634-3} hat mir lange das Vergnügen genommen, Sie und Ihre verehrte
               Frau Gemahlin\pwindex{Schnitzler, Olga 17.\,1.\,1882 Wien – 13.\,1.\,1970 Lugano@\textsc{Schnitzler, Olga} (17.\,1.\,1882 Wien – 13.\,1.\,1970 Lugano), \emph{Schauspielerin, Sängerin}|pwv} sehen zu
               können. Jetzt aber wäre ich sehr froh, wollten Sie mich es einmal wissen lassen, wenn
               Sie \label{K_L03634-4v}\edtext{einen Abend im freien}{\lemma{\textnormal{\emph{einen Abend im freien}}}\Cendnote{\textnormal{Der Brief ist nicht datiert. Der spontane
                  Gestus des Schreibens Schnitzlers vom XXXX Auszeichnungsfehler: Dokument L03794 nicht gefunden, der ein
                  gemeinsames Nachtmahl noch an diesem Abend im Türkenschanzpark\oindex{Wien@\textbf{Wien}!XVIII., Währing@\textbf{XVIII., Währing}!Restauration Türkenschanz-Park@\textbf{Restauration Türkenschanz-Park}, \emph{Lokal}|pwk} veranlasste (vgl. A. S.: \emph{Tagebuch}, 12. 6. 1911), lässt vermuten, dass Zweigs\pwindex{Zweig, Stefan 28.\,11.\,1881 Wien – 23.\,2.\,1942 Petrópolis@\textsc{Zweig, Stefan} (28.\,11.\,1881 Wien – 23.\,2.\,1942 Petrópolis), \emph{Schriftsteller}|pwk} Anfrage frühestens wenige Tage vor
                  der Verabredung und spätestens am Tag von Schnitzlers Einladung selbst gestellt wurde.}}}\label{K_L03634-4} verbringen und ich
               mich, ohne zu stören, anschliessen dürfte. Mit vielen herzlichen Grüssen und in
               getreuer Ergebenheit\pend
           \pstart \spacefill\mbox{Stefan Zweig}\pend{}\selectlanguage{ngerman}\endnumbering\briefempfaengerindex{Schnitzler, Arthur@\textsc{Schnitzler, Arthur}!zzzZweig, Stefan@\emph{von Stefan Zweig}!1911-06-051@{{[}zwischen 5. und 12. 6. 1911?{]}}|)be}\mylabel{L03634h}  \newcommand{\dateiname}{L03634}\newcommand{\titel}{Stefan Zweig an Arthur Schnitzler, [zwischen 5. und 12. 6. 1911?]}\newcommand{\editorInnen}{Selma Jahnke und Martin Anton Müller}%% latex-leseansicht-abspann.tex
%% Abspann für die Leseansicht.
%% Der Schalter \ifkorrekturansicht ist bereits durch den Vorspann gesetzt.

%% latex-abspann.tex
%% Gemeinsamer Abspann für Korrekturansicht und Leseansicht.
%% Setzt den Schalter \ifkorrekturansicht voraus (gesetzt in den
%% einbindenden Dateien latex-korrekturansicht-abspann.tex bzw.
%% latex-leseansicht-abspann.tex).
%% ---------------------------------------------------------------

\normalsize

% Das esempio-Environment wird nur in der Leseansicht benötigt
\ifkorrekturansicht\else
\newenvironment{esempio}[3]%
{
    \vspace{1.5ex}
    \rlap{\underline{#1}}
    \par
    \setlength{\parindent}{0cm}
    \nopagebreak
    \leftskip=#2cm
    \rightskip=#3cm
}
{
    \par
}
\fi

\doendnotes{C}
\bigskip
\vfill

\clearpage

\footnotesize

\ifkorrekturansicht
  \lohead{\textsc{register}}
\fi

% theindex-Environment neu definieren ohne reledmac
\makeatletter
\renewenvironment{theindex}{%
  \ifkorrekturansicht
    \section*{\indexname}%
  \else
    \subsubsection*{Index der erwähnten Entitäten}%
  \fi
  \setlength{\parindent}{0pt}%
  \setlength{\parskip}{0pt plus 0.3pt}%
  \let\item\@idxitem
}{%
  \ifkorrekturansicht\clearpage\fi
}
\makeatother

\IfFileExists{\jobname-pw.ind}{\input{\jobname-pw.ind}}{}

% Quellenangabe nur in der Leseansicht
\ifkorrekturansicht\else
% Fallback-Definitionen, falls die .tex-Datei \titel etc. nicht gesetzt hat
\providecommand{\titel}{}
\providecommand{\editorInnen}{}
\providecommand{\dateiname}{\jobname}

\vspace{3cm}

\vfill

\footnotesize
\textsc{Quelle}: \titel. Herausgegeben von {\editorInnen}. In: \emph{Arthur Schnitzler: Briefwechsel mit Autorinnen und Autoren}.
 Digitale Edition, https://schnitzler-briefe.acdh.oeaw.ac.at/{\dateiname}.html (Stand \today)
\fi

\end{document}


