%% latex-korrekturansicht-vorspann.tex
%% Vorspann für die Korrekturansicht.
%% Lädt die gemeinsame Datei latex-vorspann.tex mit gesetztem Schalter.

\newif\ifkorrekturansicht
\korrekturansichttrue

\input{../tex-inputs/latex-vorspann}


\section[Stefan Zweig an Arthur Schnitzler, {[}zwischen 5. und 12. 6. 1911?{]}]{L03634 Stefan Zweig an Arthur Schnitzler, {[}zwischen 5. und
               12. 6. 1911?{]}}
\nopagebreak\mylabel{L03634v}
\rehead{ }\normalsize\beginnumbering\briefempfaengerindex{Schnitzler, Arthur@\textsc{Schnitzler, Arthur}!zzzZweig, Stefan@\emph{von Stefan Zweig}!1911-06-121@{{[}zwischen 5. und
                  12. 6. 1911?{]}}|(be}
\toendnotes[C]{\smallbreak\pagebreak[2]}\Standort{CUL, Schnitzler, B 118.}
\physDesc{Kartenbrief, 1 Blatt, 1 Seite, 492 Zeichen
\newline{}Handschrift: schwarze Tinte, lateinische Kurrent}
\buchAbdrucke{\weitereDrucke{Stefan Zweig: \emph{Briefwechsel mit Hermann Bahr, Sigmund Freud, Rainer Maria
                        Rilke und Arthur Schnitzler}. Frankfurt am Main: \emph{S. Fischer} 1987, S. 364.} }\toendnotes[C]{\smallbreak}\pstart{}{\pb}Herrn D\textsuperscript{r}
                  Artur Schnitzler\pend{}\pstart{}Wien – Cottage\oindex{Waehringer Cottage@\textbf{Währinger Cottage}, \emph{Teil eines besiedelten Ortes (A.BSOX)}|pw}\pend{}\pstart{}\label{K_L03634-1v}\edtext{Sternwartestrasse 72}{\lemma{\textnormal{\emph{Sternwartestrasse 72}}}\Cendnote{\textnormal{Zweig\pwindex{Zweig, Stefan 28.11.1881 – 23.02.1942@\textsc{Zweig, Stefan} (28.11.1881 – 23.02.1942), \emph{Schriftsteller/Schriftstellerin}|pwk} wechselt bei der Adressierung
                        seiner Schreiben an Schnitzler immer
                        wieder zwischen der falschen Hausnummer »72« und der
                        richtigen »71«.}}}\label{K_L03634-1}\oindex{Sternwartestrasse 71@\textbf{Sternwartestraße 71}, \emph{Wohngebäude (K.WHS)}|pw}\pend{}{\bigskip}\vspace{1em}
\pstart
           {\pb}VIII. Kochgasse\oindex{Kochgasse 8@\textbf{Kochgasse 8}, \emph{Wohngebäude (K.WHS)}|pw}\pend
           {\vspace{1\baselineskip}}
\pstart{}Sehr verehrter Herr Doktor,\pend\vspace{0.5em}
\pstart
           \label{K_L03634-2v}\edtext{meine Amerika\oindex{Amerika@\textbf{Amerika}, \emph{kein passender Code gefunden}|pw}reise}{\lemma{\textnormal{\emph{meine Amerikareise}}}\Cendnote{\textnormal{Vom
                     22. 2. 1911 bis zum 21. 4. 1911 unternahm Stefan Zweig\pwindex{Zweig, Stefan 28.11.1881 – 23.02.1942@\textsc{Zweig, Stefan} (28.11.1881 – 23.02.1942), \emph{Schriftsteller/Schriftstellerin}|pwk} eine amerikanische\oindex{Amerika@\textbf{Amerika}, \emph{kein passender Code gefunden}|pwk} Reise, beginnend in New York\oindex{New York City@\textbf{New York City}, \emph{P.PPL}|pwk}. Von dort reiste er in mehrere Städte an der nordamerikanischen\oindex{Nordamerika@\textbf{Nordamerika}, \emph{L.RGN}|pwk} Ostküste, dann nach Chicago\oindex{Chicago@\textbf{Chicago}, \emph{P.PPLA2}|pwk} und Kanada\oindex{Kanada@\textbf{Kanada}, \emph{A.PCLI}|pwk}, um über Bermuda\oindex{Bermuda@\textbf{Bermuda}, \emph{A.PCLD}|pwk} und Kuba\oindex{Cuba@\textbf{Cuba}, \emph{A.PCLI}|pwk} bis nach Südamerika\oindex{Suedamerika@\textbf{Südamerika}, \emph{Kontinent (A.KNT)}|pwk} zu gelangen.}}}\label{K_L03634-2} und dann eine ärgerliche
               langwierige kleine \label{K_L03634-3v}\edtext{Operation}{\lemma{\textnormal{\emph{Operation}}}\Cendnote{\textnormal{Stefan Zweig\pwindex{Zweig, Stefan 28.11.1881 – 23.02.1942@\textsc{Zweig, Stefan} (28.11.1881 – 23.02.1942), \emph{Schriftsteller/Schriftstellerin}|pwk} musste sich im Mai
                     1911 wegen einer Rippenfellentzündung einer Operation
                  unterziehen.}}}\label{K_L03634-3} hat mir lange das Vergnügen genommen, Sie und Ihre verehrte
               Frau Gemahlin\pwindex{Schnitzler, Olga 17.01.1882 – 13.01.1970@\textsc{Schnitzler, Olga} (17.01.1882 – 13.01.1970), \emph{Schauspieler/Schauspielerin, Sänger/Sängerin}|pwv} sehen zu
               können. Jetzt aber wäre ich sehr froh, wollten Sie mich es einmal wissen lassen, wenn
               Sie \label{K_L03634-4v}\edtext{einen Abend im freien}{\lemma{\textnormal{\emph{einen Abend im freien}}}\Cendnote{\textnormal{Der Brief ist nicht datiert. Der spontane
                  Gestus des Schreibens Schnitzlers vom 12. 6. 1911, die ein
                  gemeinsames Nachtmahl noch an diesem Abend im Türkenschanzpark\oindex{Restauration Tuerkenschanz-Park@\textbf{Restauration Türkenschanz-Park}, \emph{Lokal (K.LKL)}|pwk} veranlasste (vgl. A. S.: \emph{Tagebuch}, 12. 6. 1911), lässt vermuten, dass Zweigs\pwindex{Zweig, Stefan 28.11.1881 – 23.02.1942@\textsc{Zweig, Stefan} (28.11.1881 – 23.02.1942), \emph{Schriftsteller/Schriftstellerin}|pwk} Anfrage frühestens wenige Tage vor
                  der Verabredung und spätestens am Tag von Schnitzlers Einladung selbst gestellt wurde.}}}\label{K_L03634-4} verbringen und ich
               mich, ohne zu stören, anschliessen dürfte. Mit vielen herzlichen Grüssen und in
               getreuer Ergebenheit\pend
           \pstart \spacefill\mbox{Stefan Zweig}\pend{}\selectlanguage{ngerman}\endnumbering\briefempfaengerindex{Schnitzler, Arthur@\textsc{Schnitzler, Arthur}!zzzZweig, Stefan@\emph{von Stefan Zweig}!1911-06-051@{{[}zwischen 5. und
                  12. 6. 1911?{]}}|)be}\mylabel{L03634h}  \normalsize

\doendnotes{C}
\bigskip
\vfill

\clearpage

\footnotesize

\lohead{\textsc{register}}

% Definiere theindex-Environment komplett neu ohne reledmac
\makeatletter
\renewenvironment{theindex}{%
  \section*{\indexname}%
  \setlength{\parindent}{0pt}%
  \setlength{\parskip}{0pt plus 0.3pt}%
  \let\item\@idxitem
}{%
  \clearpage
}
\makeatother

\IfFileExists{\jobname-pw.ind}{\input{\jobname-pw.ind}}{}

\end{document}

      