%% latex-leseansicht-vorspann.tex
%% Vorspann für die Leseansicht.
%% Lädt die gemeinsame Datei latex-vorspann.tex mit nicht gesetztem Schalter.

\newif\ifkorrekturansicht
\korrekturansichtfalse

\input{../tex-inputs/latex-vorspann}


\section[Arthur Schnitzler an Richard Beer-Hofmann, {[}5.?{]} 5. 1902]{L01218 Arthur Schnitzler an Richard Beer-Hofmann, [5.?] 5. 1902}
\nopagebreak\mylabel{L01218v}
\rehead{ }\normalsize\beginnumbering\briefempfaengerindex{Beer-Hofmann, Richard@\textsc{Beer-Hofmann, Richard}!zzzSchnitzler, Arthur@\emph{von Arthur Schnitzler}!1903-05-051@{[5.?] 5. 1902}|(be}
\toendnotes[C]{\smallbreak\pagebreak[2]}
\correspDesc{Versand  durch Arthur Schnitzler am [5.?] 5. 1902 in Wien
\newline{}Erhalt  durch Richard Beer-Hofmann am 5. 5. 1902 in Rodaun}\toendnotes[C]{\smallbreak}
\Standort{YCGL, MSS 31.}
\physDesc{Brief, 1 Blatt, 1 Seite, Kuvert, 124 Zeichen
\newline{}Handschrift: 1) Bleistift, deutsche Kurrent\hspace{1em}2) schwarze Tinte, deutsche Kurrent (\noindent{}Umschlag)\hspace{1em}
\newline{}Versand: 1) Einschreiben  2) Stempel: »\nobreak{}\oindex{IX., Alsergrund@\textbf{IX., Alsergrund}, \emph{Verwaltungsgebiet}|pwk}\textcolor{gray}{Wien 9/3}, \textcolor{gray}{5}{[}. 5. 1902{]}, 5\nobreak{}«.  3) Stempel: »\nobreak{}\oindex{Wien@\textbf{Wien}!XXIII., Liesing@\textbf{XXIII., Liesing}!Rodaun@\textbf{Rodaun}, \emph{Region}|pwk}{\pb}Rodaun, 5/5 02\nobreak{}«. }\toendnotes[C]{\smallbreak}\pstart{}{\pb}\textsc{rec.}\pend{}\pstart{}\textsc{Dr. Richard Beer-Hofmann}\pend{}\pstart{}\textsc{Rodaun}\oindex{Wien@\textbf{Wien}!XXIII., Liesing@\textbf{XXIII., Liesing}!Rodaun@\textbf{Rodaun}, \emph{Region}|pw}\pend{}\pstart{}\textsc{Liesinger}ſtraße 2\oindex{Liesingerstraße@\textbf{Liesingerstraße}, \emph{Straße}|pw}. \pend{}\pstart{}\textsc{Reco{\geminationm}andirt}. \pend{}{\bigskip}\vspace{1em}
\pstart
           \noindent{}{\pb}Auf Wiederſehen\pend
           
\pstart
           herzlichſt{\\[\baselineskip]}\spacefill\mbox{A.}\pend
           \leftskip=0em{}
\pstart
           \noindent{}An Eger\pwindex{Eger, Paul 23.\,1.\,1881 Wien – 9.\,4.\,1947 Luzern@\textsc{Eger, Paul} (23.\,1.\,1881 Wien – 9.\,4.\,1947 Luzern), \emph{Schriftsteller, Theaterleiter, Regisseur}|pw} (\label{K_L01218-1v}\edtext{\textsc{Peer Gynt}\pwindex{\textcolor{red}{\textsuperscript{XXXX indx1}}!Peer Gynt@\strich\emph{Peer Gynt}|pw}}{\lemma{\textnormal{\emph{Peer Gynt}}}\Cendnote{\textnormal{Am 7. 5. 1902 veranstaltete der \emph{Akademische Verein für Kunst und Literatur}\orgindex{Akademischer Verein für Kunst und Literatur@Akademischer Verein für Kunst und Literatur|pwk}
                     im Deutschen Volkstheater\oindex{Wien@\textbf{Wien}!VII., Neubau@\textbf{VII., Neubau}!Volkstheater@\textbf{Volkstheater}, \emph{Theater}|pwk} die Generalprobe
                     für eine zweimalige Wohltätigkeitsaufführung.}}}\label{K_L01218-1}{ }ſchrieb ich eben)\pend
           \selectlanguage{ngerman}\endnumbering\briefempfaengerindex{Beer-Hofmann, Richard@\textsc{Beer-Hofmann, Richard}!zzzSchnitzler, Arthur@\emph{von Arthur Schnitzler}!1903-05-051@{[5.?] 5. 1902}|)be}\mylabel{L01218h}  \newcommand{\dateiname}{L01218}\newcommand{\titel}{Arthur Schnitzler an Richard Beer-Hofmann, [5.?] 5. 1902}\newcommand{\editorInnen}{Martin Anton Müller und Gerd-Hermann Susen}%% latex-leseansicht-abspann.tex
%% Abspann für die Leseansicht.
%% Der Schalter \ifkorrekturansicht ist bereits durch den Vorspann gesetzt.

%% latex-abspann.tex
%% Gemeinsamer Abspann für Korrekturansicht und Leseansicht.
%% Setzt den Schalter \ifkorrekturansicht voraus (gesetzt in den
%% einbindenden Dateien latex-korrekturansicht-abspann.tex bzw.
%% latex-leseansicht-abspann.tex).
%% ---------------------------------------------------------------

\normalsize

% Das esempio-Environment wird nur in der Leseansicht benötigt
\ifkorrekturansicht\else
\newenvironment{esempio}[3]%
{
    \vspace{1.5ex}
    \rlap{\underline{#1}}
    \par
    \setlength{\parindent}{0cm}
    \nopagebreak
    \leftskip=#2cm
    \rightskip=#3cm
}
{
    \par
}
\fi

\doendnotes{C}
\bigskip
\vfill

\clearpage

\footnotesize

\ifkorrekturansicht
  \lohead{\textsc{register}}
\fi

% theindex-Environment neu definieren ohne reledmac
\makeatletter
\renewenvironment{theindex}{%
  \ifkorrekturansicht
    \section*{\indexname}%
  \else
    \subsubsection*{Index der erwähnten Entitäten}%
  \fi
  \setlength{\parindent}{0pt}%
  \setlength{\parskip}{0pt plus 0.3pt}%
  \let\item\@idxitem
}{%
  \ifkorrekturansicht\clearpage\fi
}
\makeatother

\IfFileExists{\jobname-pw.ind}{\input{\jobname-pw.ind}}{}

% Quellenangabe nur in der Leseansicht
\ifkorrekturansicht\else
% Fallback-Definitionen, falls die .tex-Datei \titel etc. nicht gesetzt hat
\providecommand{\titel}{}
\providecommand{\editorInnen}{}
\providecommand{\dateiname}{\jobname}

\vspace{3cm}

\vfill

\footnotesize
\textsc{Quelle}: \titel. Herausgegeben von {\editorInnen}. In: \emph{Arthur Schnitzler: Briefwechsel mit Autorinnen und Autoren}.
 Digitale Edition, https://schnitzler-briefe.acdh.oeaw.ac.at/{\dateiname}.html (Stand \today)
\fi

\end{document}


