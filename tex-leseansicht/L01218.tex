%% latex-korrekturansicht-vorspann.tex
%% Vorspann für die Korrekturansicht.
%% Lädt die gemeinsame Datei latex-vorspann.tex mit gesetztem Schalter.

\newif\ifkorrekturansicht
\korrekturansichttrue

\input{../tex-inputs/latex-vorspann}


\section[Arthur Schnitzler an Richard Beer-Hofmann, {[}5.?{]} 5. 1902]{L01218 Arthur Schnitzler an Richard Beer-Hofmann, {[}5.?{]} 5. 1902}
\nopagebreak\mylabel{L01218v}
\rehead{ }\normalsize\beginnumbering\briefempfaengerindex{Beer-Hofmann, Richard@\textsc{Beer-Hofmann, Richard}!zzzSchnitzler, Arthur@\emph{von Arthur Schnitzler}!1903-05-051@{{[}5.?{]} 5. 1902}|(be}
\toendnotes[C]{\smallbreak\pagebreak[2]}\Standort{YCGL, MSS 31.}
\physDesc{Brief, 1 Blatt, 1 Seite, Umschlag, 124 Zeichen
\newline{}Handschrift: 1) Bleistift, deutsche Kurrent\hspace{1em}2) schwarze Tinte, deutsche Kurrent (\noindent{}Umschlag)\hspace{1em}
\newline{}Versand: 1) Einschreiben  2) Stempel: »\nobreak{}\oindex{IX., Alsergrund@\textbf{IX., Alsergrund}, \emph{A.ADM3}|pwk}\textcolor{gray}{Wien 9/3}, \textcolor{gray}{5}{[}. 5. 1902{]}, 5\nobreak{}«.  3) Stempel: »\nobreak{}\oindex{Rodaun@\textbf{Rodaun}, \emph{A.ADM4}|pwk}{\pb}Rodaun, 5/5 02\nobreak{}«. }\toendnotes[C]{\smallbreak}\pstart{}{\pb}\textsc{rec.}\pend{}\pstart{}\textsc{Dr. Richard Beer-Hofmann}\pend{}\pstart{}\textsc{Rodaun}\oindex{Rodaun@\textbf{Rodaun}, \emph{A.ADM4}|pw}\pend{}\pstart{}\textsc{Liesinger}ſtraße 2\oindex{Liesingerstrasse@\textbf{Liesingerstraße}, \emph{Straße (K.STR)}|pw}. \pend{}\pstart{}\textsc{Reco{\geminationm}andirt}. \pend{}{\bigskip}\vspace{1em}
\pstart
           \noindent{}{\pb}Auf Wiederſehen\pend
           
\pstart
           herzlichſt{\\[\baselineskip]}\spacefill\mbox{A.}\pend
           \leftskip=0em{}
\pstart
           \noindent{}An Eger\pwindex{Eger, Paul 23.01.1881 – 09.04.1947@\textsc{Eger, Paul} (23.01.1881 – 09.04.1947), \emph{Schriftsteller/Schriftstellerin, Theaterleiter/Theaterleiterin, Regisseur/Regisseurin}|pw} (\label{K_L01218-1v}\edtext{\textsc{Peer Gynt}\pwindex{Peer Gynt@\emph{Peer Gynt}|pw}}{\lemma{\textnormal{\emph{Peer Gynt}}}\Cendnote{\textnormal{Am 7. 5. 1902 veranstaltete der \emph{Akademische Verein für Kunst und Literatur}\orgindex{Akademischer Verein fuer Kunst und Literatur@Akademischer Verein für Kunst und Literatur|pwk}
                     im Deutschen Volkstheater\oindex{Volkstheater@\textbf{Volkstheater}, \emph{Theater (K.THE)}|pwk} die Generalprobe
                     für eine zweimalige Wohltätigkeitsaufführung.}}}\label{K_L01218-1} ſchrieb ich eben)\pend
           \selectlanguage{ngerman}\endnumbering\briefempfaengerindex{Beer-Hofmann, Richard@\textsc{Beer-Hofmann, Richard}!zzzSchnitzler, Arthur@\emph{von Arthur Schnitzler}!1903-05-051@{{[}5.?{]} 5. 1902}|)be}\mylabel{L01218h}  \normalsize

\doendnotes{C}
\bigskip
\vfill

\clearpage

\footnotesize

\lohead{\textsc{register}}

% Definiere theindex-Environment komplett neu ohne reledmac
\makeatletter
\renewenvironment{theindex}{%
  \section*{\indexname}%
  \setlength{\parindent}{0pt}%
  \setlength{\parskip}{0pt plus 0.3pt}%
  \let\item\@idxitem
}{%
  \clearpage
}
\makeatother

\IfFileExists{\jobname-pw.ind}{\input{\jobname-pw.ind}}{}

\end{document}

      