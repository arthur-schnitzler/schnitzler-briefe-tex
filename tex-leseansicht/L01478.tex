%% latex-korrekturansicht-vorspann.tex
%% Vorspann für die Korrekturansicht.
%% Lädt die gemeinsame Datei latex-vorspann.tex mit gesetztem Schalter.

\newif\ifkorrekturansicht
\korrekturansichttrue

\input{../tex-inputs/latex-vorspann}


\section[Hermann Bahr an Arthur Schnitzler, 14. 12. 1904]{L01478 Hermann Bahr an Arthur Schnitzler, 14. 12. 1904}
\nopagebreak\mylabel{L01478v}
\rehead{ }\normalsize\beginnumbering\briefempfaengerindex{Schnitzler, Arthur@\textsc{Schnitzler, Arthur}!zzzBahr, Hermann@\emph{von Hermann Bahr}!1904-12-142@{14. 12. 1904}|(be}
\toendnotes[C]{\smallbreak\pagebreak[2]}\Standort{CUL, Schnitzler, B 5b.}
\physDesc{Brief, 1 Blatt, 4 Seiten, 2216 Zeichen
\newline{}Handschrift: schwarze Tinte, deutsche Kurrent
\newline{}Ordnung: mit Bleistift von unbekannter Hand nummeriert:
                                    »124« }
\buchAbdrucke{\weitereDrucke{Hermann Bahr, Arthur Schnitzler: \emph{Briefwechsel, Aufzeichnungen, Dokumente (1891–1931)}. Göttingen: \emph{Wallstein} 2018, S. 334–335.} }\toendnotes[C]{\smallbreak}
\pstart
           \raggedleft{}{\pb}14. 12. 04{\\}Nachts\pend
           
\pstart\center{}Lieber Arthur!\pend\vspace{0.5em}
\pstart
           Ich hab Dich nach der Symphonie\pwindex{3. Sinfonie in d-Moll@\emph{3. Sinfonie in d-Moll}|pwv} heut überall geſucht, aber Du warſt wie in die Erde verſunken. So
               laß mich Dir ſchriftlich geſchwind (denn ich bin todtmüd vor Muſik, geſtern auch nach
                  \label{K_L01478-1v}\edtext{Walküre\pwindex{Walkuere@\emph{Die Walküre}|pw}}{\lemma{\textnormal{\emph{Walküre}}}\Cendnote{\textnormal{am 3. 12. 1904 in der Hofoper\oindex{Oper@\textbf{Oper}, \emph{Oper (K.OPR)}|pwk}, mit Anna von Mildenburg\pwindex{Bahr-Mildenburg, Anna 29.11.1872 – 27.01.1947@\textsc{Bahr-Mildenburg, Anna} (29.11.1872 – 27.01.1947), \emph{Sänger/Sängerin}|pwk}}}}\label{K_L01478-1}, die mich ſo wahnſinnig aufgeregt hat, daß ich heut erſt in der Früh gegen
               fünf einſchlafen konnte) herzlichſt für Deinen lieben Brief danken. Es iſt möglich,
               daß Du recht haſt (mit dem, was Du über Deine Intention ſagſt, haſt Du natürlich
               gewiß recht, fraglich bleibt nur, ob nicht bei der Ausführung, Dir ſelbſt unbewußt,
               etwas von einer Untergrundſtimmung in Dir, die ſich nach dem Philiſter ſehnt,
               eingefloſſen iſt), ich mußte mein Gefühl aber einmal ausſprechen, mit einiger
               Schärfe, die nicht Dir gilt, ſondern mir ſelbſt, einer inneren Schwäche in
                  \textcolor{gray}{mir}{ }ſelbſt, {\pb}an der
               ich Jahre lang gelitten habe (Manches, was ich jetzt im »Franzl\pwindex{Franzl. Fuenf Bilder aus dem Leben eines guten Mannes@\emph{Der Franzl. Fünf Bilder aus dem Leben eines guten Mannes}|pw}« nicht mehr mag und dieſe blödſinnige letzte Scene des
                  »Apoſtels\pwindex{Apostel. Schauspiel in drei Aufzuegen@\emph{Der Apostel. Schauspiel in drei Aufzügen}|pw}« iſt aus ihr) und von der ich mich
               nur durch eine erbitterte Anrufung meiner innerſten Inſtinkte \label{LL287-2v}frei gemacht habe – ganz frei freilich erſt, ſeit ich mit dem
                  Tode ſo vertraut bin, ſeit der Tod wirklich mein beſter Freund geworden
                  iſt\label{LL287-2h}, der einzige nemlich, den ich mir noch wirklich verdienen will,
                  \label{LL287-3v}aber über dies alles einmal mündlich in
                  einer guten Stunde, denn es iſt tiefer, als ſich ſo hinſchreiben läßt, viel
                     »\label{K_L01478-2v}\edtext{tiefer als der Tag
                     gedacht\pwindex{3. Sinfonie in d-Moll@\emph{3. Sinfonie in d-Moll}|pwv}\pwindex{Also sprach Zarathustra@\emph{Also sprach Zarathustra}|pwv}}{\lemma{\textnormal{\emph{tiefer … gedacht}}}\Cendnote{\textnormal{Zitat aus dem Lied »Vor
                     Sonnen-Aufgang« in Friedrich Nietzsche\pwindex{Nietzsche, Friedrich 15.10.1844 – 25.08.1900@\textsc{Nietzsche, Friedrich} (15.10.1844 – 25.08.1900), \emph{Schriftsteller/Schriftstellerin, Philosoph/Philosophin}|pwk}:
                        \emph{Also sprach Zarathustra. Ein Buch für Alle
                        und Keinen}\pwindex{Also sprach Zarathustra@\emph{Also sprach Zarathustra}|pwk} (3. Band. Chemnitz: \emph{Schmeitzner}{ }1884), hier wohl nach der Vertonung durch Gustav Mahler\pwindex{Mahler, Gustav 07.07.1860 – 18.05.1911@\textsc{Mahler, Gustav} (07.07.1860 – 18.05.1911), \emph{Theaterleiter/Theaterleiterin, Komponist/Komponistin, Dirigent/Dirigentin}|pwk} im 4. Satz der \emph{3.
                        Sinfonie}\pwindex{3. Sinfonie in d-Moll@\emph{3. Sinfonie in d-Moll}|pwk}.}}}\label{K_L01478-2}«, Triſtan\pwindex{Tristan und Isolde@\emph{Tristan und Isolde}|pwv}tief, wo Du es jetzt, im zweiten Akt, viel ſchöner finden {\pb}wirſt, als ichs jemals werd ausſprechen
                  können\label{LL287-3h}.\pend
           
\pstart
           Sehr leid tut mir, daß ich Samſtag nicht zu Euch kommen kann, 1) weil
               ich Hugo\pwindex{Hofmannsthal, Hugo von 1874-02-01 – 1929-07-15@\textsc{Hofmannsthal, Hugo von} (1874-02-01 – 1929-07-15), \emph{Schriftsteller/Schriftstellerin}|pw} verſprochen habe, nach Rodaun\oindex{Rodaun@\textbf{Rodaun}, \emph{A.ADM4}|pw} zu kommen und 2) weil ich auch dort abſagen
               muß, weil ich 3) gerade jetzt, bei froheſter innerer Geneſung\pwindex{Genesung. Roman@\emph{Genesung. Roman}|pw} (der Teufel ſoll den \label{K_L01478-3v}\edtext{Trebitſch\pwindex{Trebitsch, Siegfried 22.12.1868 – 03.06.1956@\textsc{Trebitsch, Siegfried} (22.12.1868 – 03.06.1956), \emph{Schriftsteller/Schriftstellerin, Übersetzer/Übersetzerin}|pw} holen, der die ſchönſten Worte ſo
                  beſchmutzt}{\lemma{\textnormal{\emph{Trebitſch … beſchmutzt}}}\Cendnote{\textnormal{Vgl. Hermann Bahr an Arthur Schnitzler, 22. 2. 1903.
                  }}}\label{K_L01478-3}, daß einem grauſt, ſie anzurühren), \label{LL287-1v}äußerlich in einem rechten \label{K_L01478-4v}\edtext{Durcheinander lebe, den}{\lemma{\textnormal{\emph{Durcheinander lebe, den}}}\Cendnote{\textnormal{Durcheinander: dialektal auch als Maskulinum}}}\label{K_L01478-4} ich nicht ändern kann und
                  nicht ändern möchte\label{LL287-1h}, kurz: ſo ſehr ich mich wirklich ſehne, wieder einmal
               ruhig bei Euch zu ſitzen, jetzt gerade gehts in den nächſten Tagen leider nicht.\pend
           
\pstart
           {\pb}Herzlichſt danke ich auch für den Gruß Deiner
               lieben Frau\pwindex{Schnitzler, Olga 17.01.1882 – 13.01.1970@\textsc{Schnitzler, Olga} (17.01.1882 – 13.01.1970), \emph{Schauspieler/Schauspielerin, Sänger/Sängerin}|pwv} und erwiedere ihn
               herzlichſt.\pend
           
\pstart
           Ich wünſche mir ſehr, daß ſichs ſo treffen möchte, daß wir doch zwei drei Tage in Lueg\oindex{Lueg@\textbf{Lueg}, \emph{Teil eines besiedelten Ortes (A.BSOX)}|pw}{ }\label{LL287-4v}beiſammen ſind\label{LL287-4h}.\pend
           
\pstart
           Dein alter{\\[\baselineskip]}\spacefill\mbox{H.}\pend
           \leftskip=0em{}\selectlanguage{ngerman}\endnumbering\briefempfaengerindex{Schnitzler, Arthur@\textsc{Schnitzler, Arthur}!zzzBahr, Hermann@\emph{von Hermann Bahr}!1904-12-142@{14. 12. 1904}|)be}\mylabel{L01478h}  \normalsize

\doendnotes{C}
\bigskip
\vfill

\clearpage

\footnotesize

\lohead{\textsc{register}}

% Definiere theindex-Environment komplett neu ohne reledmac
\makeatletter
\renewenvironment{theindex}{%
  \section*{\indexname}%
  \setlength{\parindent}{0pt}%
  \setlength{\parskip}{0pt plus 0.3pt}%
  \let\item\@idxitem
}{%
  \clearpage
}
\makeatother

\IfFileExists{\jobname-pw.ind}{\input{\jobname-pw.ind}}{}

\end{document}

      