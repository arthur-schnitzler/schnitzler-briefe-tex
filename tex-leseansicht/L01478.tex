%% latex-leseansicht-vorspann.tex
%% Vorspann für die Leseansicht.
%% Lädt die gemeinsame Datei latex-vorspann.tex mit nicht gesetztem Schalter.

\newif\ifkorrekturansicht
\korrekturansichtfalse

\input{../tex-inputs/latex-vorspann}


               \section[Hermann Bahr an Arthur Schnitzler, 14. 12. 1904]{ Hermann Bahr an Arthur Schnitzler, 14. 12. 1904}\nopagebreak\mylabel{v}\rehead{ }\begin{ledgroupsized}[t]{13cm}\normalsize\beginnumbering\briefempfaengerindex{Schnitzler, Arthur@\textsc{Schnitzler, Arthur}!zzzBahr, Hermann@\emph{von Hermann Bahr}!1904-12-142@{14. 12. 1904}|(be} \toendnotes[C]{\smallbreak\pagebreak[2]} \Standort{CUL, Schnitzler, B 5b.}
\physDesc{Brief, 1 Blatt, 4 Seiten
\newline{}Handschrift: schwarze Tinte, deutsche Kurrent\newline{}Ordnung: mit Bleistift von unbekannter Hand nummeriert: »124« }\buchAbdrucke{\weitereDrucke{Hermann Bahr, Arthur Schnitzler: \emph{Briefwechsel, Aufzeichnungen, Dokumente (1891–1931)}. Hg. Kurt Ifkovits und Martin Anton Müller. Göttingen: \emph{Wallstein} 2018, S. 334–335.} }\toendnotes[C]{\smallbreak}\pstart
           \raggedleft{}{\pb}14. 12. 04{\\}Nachts\pend
           \pstart\center{}Lieber Arthur!\pend\pstart
           Ich hab Dich nach der Symphonie\pwindex{Mahler, Gustav 07.07.1860 – 18.05.1911@\textsc{Mahler, Gustav} (07.07.1860 – 18.05.1911), \emph{Theaterleiter, Komponist, Dirigent}!Symphonie Nr. 3 D-Moll1902@\strich\emph{Symphonie Nr. 3 D-Moll} {[}1902{]}|pwv}
               heut überall geſucht, aber Du warſt wie in die Erde verſunken. So laß mich Dir
               ſchriftlich geſchwind (denn ich bin todtmüd vor Muſik, geſtern auch nach \label{K_L01478_1v}\edtext{Walküre\pwindex{\textcolor{red}{\textsuperscript{XXXX1 indx}}!Walkuere1870@\strich\emph{Die Walküre} {[}1870{]}|pw}}{\lemma{\textnormal{\emph{Walküre}}}\Cendnote{\textnormal{am 3. 12. 1904 in der Hofoper\oindex{Oper@\textbf{Oper}|pwk}, mit Anna
                     von Mildenburg\pwindex{Bahr-Mildenburg, Anna 29.11.1872 – 27.01.1947@\textsc{Bahr-Mildenburg, Anna} (29.11.1872 – 27.01.1947), \emph{Sängerin}|pwk}}}}\label{K_L01478_1h}, die mich ſo wahnſinnig aufgeregt hat, daß ich heut
               erſt in der Früh gegen fünf einſchlafen konnte) herzlichſt für Deinen lieben Brief
               danken. Es iſt möglich, daß Du recht haſt (mit dem, was Du über Deine Intention
               ſagſt, haſt Du natürlich gewiß recht, fraglich bleibt nur, ob nicht bei der
               Ausführung, Dir ſelbſt unbewußt, etwas von einer Untergrundſtimmung in Dir, die ſich
               nach dem Philiſter ſehnt, eingefloſſen iſt), ich mußte mein Gefühl aber einmal
               ausſprechen, mit einiger Schärfe, die nicht Dir gilt, ſondern mir ſelbſt, einer
               inneren Schwäche in \textcolor{gray}{mir}{ }ſelbſt, {\pb}an der
               ich Jahre lang gelitten habe (Manches, was ich jetzt im »Franzl\pwindex{Bahr, Hermann 19.07.1863 – 15.01.1934@\textsc{Bahr, Hermann} (19.07.1863 – 15.01.1934), \emph{Schriftsteller, Kritiker}!Franzl. Fuenf Bilder aus dem Leben eines guten Mannes1900@\strich\emph{Der Franzl. Fünf Bilder aus dem Leben eines guten Mannes} {[}1900{]}|pw}« nicht mehr mag und dieſe blödſinnige letzte Scene des »Apoſtels\pwindex{Bahr, Hermann 19.07.1863 – 15.01.1934@\textsc{Bahr, Hermann} (19.07.1863 – 15.01.1934), \emph{Schriftsteller, Kritiker}!Apostel1901@\strich\emph{Der Apostel} {[}1901{]}|pw}« iſt aus ihr) und von der ich mich nur
               durch eine erbitterte Anrufung meiner innerſten Inſtinkte \label{LL287-2v}frei gemacht habe – ganz frei freilich erſt, ſeit ich mit dem
                  Tode ſo vertraut bin, ſeit der Tod wirklich mein beſter Freund geworden
                  iſt\label{LL287-2h}, der einzige nemlich, den ich mir noch wirklich verdienen will,
                  \label{LL287-3v}aber über dies alles einmal mündlich in
                  einer guten Stunde, denn es iſt tiefer, als ſich ſo hinſchreiben läßt, viel
                     »\label{K_L01478_2v}\edtext{tiefer als der Tag
                     gedacht\pwindex{Mahler, Gustav 07.07.1860 – 18.05.1911@\textsc{Mahler, Gustav} (07.07.1860 – 18.05.1911), \emph{Theaterleiter, Komponist, Dirigent}!Symphonie Nr. 3 D-Moll1902@\strich\emph{Symphonie Nr. 3 D-Moll} {[}1902{]}|pwv}\pwindex{Nietzsche, Friedrich 15.10.1844 – 25.08.1900@\textsc{Nietzsche, Friedrich} (15.10.1844 – 25.08.1900), \emph{Schriftsteller, Philosoph}!Also sprach Zarathustra1883 – 1885@\strich\emph{Also sprach Zarathustra} {[}1883 – 1885{]}|pwv}}{\lemma{\textnormal{\emph{tiefer … gedacht}}}\Cendnote{\textnormal{Zitat aus dem Lied »Vor
                     Sonnen-Aufgang« in Friedrich Nietzsche\pwindex{Nietzsche, Friedrich 15.10.1844 – 25.08.1900@\textsc{Nietzsche, Friedrich} (15.10.1844 – 25.08.1900), \emph{Schriftsteller, Philosoph}|pwk}: \emph{Also sprach Zarathustra. Ein Buch für Alle und
                        Keinen}\pwindex{Nietzsche, Friedrich 15.10.1844 – 25.08.1900@\textsc{Nietzsche, Friedrich} (15.10.1844 – 25.08.1900), \emph{Schriftsteller, Philosoph}!Also sprach Zarathustra1883 – 1885@\strich\emph{Also sprach Zarathustra} {[}1883 – 1885{]}|pwk} (3. Band. Chemnitz: \emph{Schmeitzner}{ }1884), hier wohl nach der Vertonung durch Gustav Mahler\pwindex{Mahler, Gustav 07.07.1860 – 18.05.1911@\textsc{Mahler, Gustav} (07.07.1860 – 18.05.1911), \emph{Theaterleiter, Komponist, Dirigent}|pwk} im 4. Satz der \emph{3.
                        Sinfonie}\pwindex{Mahler, Gustav 07.07.1860 – 18.05.1911@\textsc{Mahler, Gustav} (07.07.1860 – 18.05.1911), \emph{Theaterleiter, Komponist, Dirigent}!Symphonie Nr. 3 D-Moll1902@\strich\emph{Symphonie Nr. 3 D-Moll} {[}1902{]}|pwk}.}}}\label{K_L01478_2h}«, Triſtan\pwindex{\textcolor{red}{\textsuperscript{XXXX1 indx}}!Tristan und Isolde1865@\strich\emph{Tristan und Isolde} {[}1865{]}|pwv}tief, wo Du es jetzt, im zweiten Akt, viel ſchöner finden {\pb}wirſt, als ichs jemals werd ausſprechen
                  können\label{LL287-3h}.\pend
           \pstart
           Sehr leid tut mir, daß ich Samſtag nicht zu Euch kommen kann, 1) weil ich Hugo\pwindex{Hofmannsthal, Hugo von 01.02.1874 – 15.07.1929@\textsc{Hofmannsthal, Hugo von} (01.02.1874 – 15.07.1929), \emph{Schriftsteller}|pw} verſprochen habe, nach Rodaun\oindex{Rodaun@\textbf{Rodaun}|pw} zu kommen und 2) weil ich auch dort abſagen muß, weil ich
               3) gerade jetzt, bei froheſter innerer Geneſung\pwindex{Trebitsch, Siegfried 22.12.1868 – 03.06.1956@\textsc{Trebitsch, Siegfried} (22.12.1868 – 03.06.1956), \emph{Schriftsteller, Übersetzer}!Genesung. Roman1902@\strich\emph{Genesung. Roman} {[}1902{]}|pw}
               (der Teufel ſoll den \label{K_L01478_3v}\edtext{Trebitſch\pwindex{Trebitsch, Siegfried 22.12.1868 – 03.06.1956@\textsc{Trebitsch, Siegfried} (22.12.1868 – 03.06.1956), \emph{Schriftsteller, Übersetzer}|pw} holen, der die ſchönſten Worte ſo
                  beſchmutzt}{\lemma{\textnormal{\emph{Trebitſch … beſchmutzt}}}\Cendnote{\textnormal{vgl. Hermann Bahr an Arthur Schnitzler, 22. 2. 1903}}}\label{K_L01478_3h}, daß einem grauſt, ſie anzurühren), \label{LL287-1v}äußerlich in einem rechten Durcheinander lebe, \label{K_L01478_4v}\edtext{den ich nicht ändern}{\lemma{\textnormal{\emph{den ich nicht ändern}}}\Cendnote{\textnormal{Durcheinander: dialektal auch als Maskulinum.}}}\label{K_L01478_4h} kann und
                  nicht ändern möchte\label{LL287-1h}, kurz: ſo ſehr ich mich wirklich ſehne, wieder einmal
               ruhig bei Euch zu ſitzen, jetzt gerade gehts in den nächſten Tagen leider nicht.\pend
           \pstart
           {\pb}Herzlichſt danke ich auch für den Gruß Deiner
               lieben Frau\pwindex{Schnitzler, Olga 17.01.1882 – 13.01.1970@\textsc{Schnitzler, Olga} (17.01.1882 – 13.01.1970), \emph{Schauspielerin, Sängerin}|pwv} und erwiedere ihn
               herzlichſt.\pend
           \pstart
           Ich wünſche mir ſehr, daß ſichs ſo treffen möchte, daß wir doch zwei drei Tage in Lueg\oindex{Lueg am Wolfgangsee@\textbf{Lueg am Wolfgangsee}|pw}{ }\label{LL287-4v}beiſammen ſind\label{LL287-4h}.\pend
           \pstart
           Dein alter{\\[\baselineskip]}\spacefill\mbox{H.}\pend
           \leftskip=0em{}\endnumbering\briefempfaengerindex{Schnitzler, Arthur@\textsc{Schnitzler, Arthur}!zzzBahr, Hermann@\emph{von Hermann Bahr}!1904-12-142@{14. 12. 1904}|)be}\mylabel{h}\end{ledgroupsized}  \newcommand{\dateiname}{L01478}\newcommand{\titel}{Hermann Bahr an Arthur Schnitzler, 14. 12. 1904}\newcommand{\editorInnen}{ Kurt Ifkovits,  Martin Anton Müller}%% latex-leseansicht-abspann.tex
%% Abspann für die Leseansicht.
%% Der Schalter \ifkorrekturansicht ist bereits durch den Vorspann gesetzt.

%% latex-abspann.tex
%% Gemeinsamer Abspann für Korrekturansicht und Leseansicht.
%% Setzt den Schalter \ifkorrekturansicht voraus (gesetzt in den
%% einbindenden Dateien latex-korrekturansicht-abspann.tex bzw.
%% latex-leseansicht-abspann.tex).
%% ---------------------------------------------------------------

\normalsize

% Das esempio-Environment wird nur in der Leseansicht benötigt
\ifkorrekturansicht\else
\newenvironment{esempio}[3]%
{
    \vspace{1.5ex}
    \rlap{\underline{#1}}
    \par
    \setlength{\parindent}{0cm}
    \nopagebreak
    \leftskip=#2cm
    \rightskip=#3cm
}
{
    \par
}
\fi

\doendnotes{C}
\bigskip
\vfill

\clearpage

\footnotesize

\ifkorrekturansicht
  \lohead{\textsc{register}}
\fi

% theindex-Environment neu definieren ohne reledmac
\makeatletter
\renewenvironment{theindex}{%
  \ifkorrekturansicht
    \section*{\indexname}%
  \else
    \subsubsection*{Index der erwähnten Entitäten}%
  \fi
  \setlength{\parindent}{0pt}%
  \setlength{\parskip}{0pt plus 0.3pt}%
  \let\item\@idxitem
}{%
  \ifkorrekturansicht\clearpage\fi
}
\makeatother

\IfFileExists{\jobname-pw.ind}{\input{\jobname-pw.ind}}{}

% Quellenangabe nur in der Leseansicht
\ifkorrekturansicht\else
% Fallback-Definitionen, falls die .tex-Datei \titel etc. nicht gesetzt hat
\providecommand{\titel}{}
\providecommand{\editorInnen}{}
\providecommand{\dateiname}{\jobname}

\vspace{3cm}

\vfill

\footnotesize
\textsc{Quelle}: \titel. Herausgegeben von {\editorInnen}. In: \emph{Arthur Schnitzler: Briefwechsel mit Autorinnen und Autoren}.
 Digitale Edition, https://schnitzler-briefe.acdh.oeaw.ac.at/{\dateiname}.html (Stand \today)
\fi

\end{document}


      