%% latex-korrekturansicht-vorspann.tex
%% Vorspann für die Korrekturansicht.
%% Lädt die gemeinsame Datei latex-vorspann.tex mit gesetztem Schalter.

\newif\ifkorrekturansicht
\korrekturansichttrue

\input{../tex-inputs/latex-vorspann}


\section[ Felix Salten an Arthur Schnitzler, 2. 9. 1912]{L03559 Felix Salten an Arthur Schnitzler, 2. 9. 1912}
\nopagebreak\mylabel{L03559v}
\rehead{ }\normalsize\beginnumbering\briefempfaengerindex{Schnitzler, Arthur@\textsc{Schnitzler, Arthur}!zzzSalten, Felix@\emph{von Felix Salten}!1912-09-021@{2. 9. 1912}|(be}
\toendnotes[C]{\smallbreak\pagebreak[2]}\Standort{CUL, Schnitzler, B 89, B 2.}
\physDesc{Briefkarte, 1319 Zeichen
\newline{}Handschrift: schwarze Tinte, lateinische Kurrent
\newline{}Ordnung: mit Bleistift von unbekannter Hand nummeriert: »274a« }\toendnotes[C]{\smallbreak}
\pstart
           \raggedleft{}{\pb}Berghof\oindex{Berghof@\textbf{Berghof}, \emph{Wohngebäude (K.WHS)}|pw}, 2. IX. 12\pend
           
\pstart{}Lieber,\pend\vspace{0.5em}
\pstart
           ich hoffe sehr, dass \label{K_L03559-1v}\edtext{Reinhardts\pwindex{Reinhardt, Max 09.09.1873 – 30.10.1943@\textsc{Reinhardt, Max} (09.09.1873 – 30.10.1943), \emph{Theaterleiter/Theaterleiterin, Regisseur/Regisseurin, Schauspieler/Schauspielerin}|pw}{ }Mirakel\pwindex{Mirakel@\emph{Das Mirakel}|pw}}{\lemma{\textnormal{\emph{Reinhardts Mirakel}}}\Cendnote{\textnormal{\emph{Das Mirakel}\pwindex{Mirakel@\emph{Das Mirakel}|pwk} von Karl Gustav Vollmoeller\pwindex{Vollmoeller, Karl Gustav 07.05.1878 – 18.10.1948@\textsc{Vollmoeller, Karl Gustav} (07.05.1878 – 18.10.1948), \emph{Schriftsteller/Schriftstellerin}|pwk} wurde am 18. 9. 1912 in der Rotunde\oindex{Rotunde@\textbf{Rotunde}, \emph{Gebäude (K.GBD)}|pwk} im
                     Wien\oindex{Wien@\textbf{Wien}, \emph{A.ADM2}|pwk}er Prater\oindex{Prater@\textbf{Prater}, \emph{Park (K.PRK)}|pwk} erstmals auf Deutsch gegeben, wo Platz für 8000 Zuschauerinnen
                  und Zuschauer war. Die Inszenierung stammte von Max Reinhardt\pwindex{Reinhardt, Max 09.09.1873 – 30.10.1943@\textsc{Reinhardt, Max} (09.09.1873 – 30.10.1943), \emph{Theaterleiter/Theaterleiterin, Regisseur/Regisseurin, Schauspieler/Schauspielerin}|pwk}. Schnitzler besuchte
                  die Aufführung am 5. 10. 1912.}}}\label{K_L03559-1} verspätet aufgeführt wird, und dass mich also
               nichts dazu zwingt, die \label{K_L03559-2v}\edtext{Eucharistische Luft}{\lemma{\textnormal{\emph{Eucharistische Luft}}}\Cendnote{\textnormal{In Wien\oindex{Wien@\textbf{Wien}, \emph{A.ADM2}|pwk} fand zwischen dem
                     12. 9. 1912 und dem 15. 9. 1912
                  der XXIII. internationale Eucharistische Kongress statt.}}}\label{K_L03559-2} in Wien\oindex{Wien@\textbf{Wien}, \emph{A.ADM2}|pw} zu atmen. Wenn Otti\pwindex{Salten, Ottilie 07.03.1868 – 22.06.1942@\textsc{Salten, Ottilie} (07.03.1868 – 22.06.1942), \emph{Schauspieler/Schauspielerin}|pw} wieder da und der Berghof\oindex{Berghof@\textbf{Berghof}, \emph{Wohngebäude (K.WHS)}|pw} ruhiger geworden ist, möchte ich wol gerne noch ein paar Wochen
               still hier arbeiten. Was sagen Sie zum \label{K_L03559-3v}\edtext{Burgtheater\orgindex{Burgtheater@Burgtheater|pw}}{\lemma{\textnormal{\emph{Burgtheater}}}\Cendnote{\textnormal{Alfred von Berger\pwindex{Berger, Alfred von 30.04.1853 – 24.08.1912@\textsc{Berger, Alfred von} (30.04.1853 – 24.08.1912), \emph{Schriftsteller/Schriftstellerin, Journalist/Journalistin, Theaterleiter/Theaterleiterin}|pwk}, der Direktor des \emph{Burgtheaters}\orgindex{Burgtheater@Burgtheater|pwk}, war am 24. 8. 1912
                  verstorben. Am 1. 9. 1912 wurde Hugo Thimig\pwindex{Thimig, Hugo 16.06.1854 – 24.09.1944@\textsc{Thimig, Hugo} (16.06.1854 – 24.09.1944), \emph{Theaterleiter/Theaterleiterin, Schauspieler/Schauspielerin}|pwk} zum provisorischen – später dann zum
                  ordentlichen Leiter ernannt.}}}\label{K_L03559-3}? Der arme Berger\pwindex{Berger, Alfred von 30.04.1853 – 24.08.1912@\textsc{Berger, Alfred von} (30.04.1853 – 24.08.1912), \emph{Schriftsteller/Schriftstellerin, Journalist/Journalistin, Theaterleiter/Theaterleiterin}|pw} tut mir leid, aber ich kann mir nicht helfen – wenn auch ein
                  Fi\textcolor{gray}{asco} oftmals besser ist als das Sterben, hier hat der Tod
               doch einen an sich schon nicht übermäßig glücklichen Menschen vor sehr unglücklichen
               Enttäuschungen bewahrt. Könnten wir Brahm\pwindex{Brahm, Otto 05.02.1856 – 28.11.1912@\textsc{Brahm, Otto} (05.02.1856 – 28.11.1912), \emph{Theaterleiter/Theaterleiterin, Regisseur/Regisseurin}|pw} oder
               vielleicht sogar Rudolf Rittner\pwindex{Rittner, Rudolf 30.06.1869 – 04.02.1943@\textsc{Rittner, Rudolf} (30.06.1869 – 04.02.1943), \emph{Theaterleiter/Theaterleiterin, Schauspieler/Schauspielerin}|pw} bekommen, dann
               wäre doch vielleicht für die Zukunft ein gutes menschliches und künstlerisches
               Verhältnis zum Burgtheater\orgindex{Burgtheater@Burgtheater|pw} möglich. Aber
                  das{[}s{]} Herr von Kralik\pwindex{Kralik, Richard 1852-10-01 – 1934-02-04@\textsc{Kralik, Richard} (1852-10-01 – 1934-02-04), \emph{Schriftsteller/Schriftstellerin}|pw}
               als Director auch nur genannt werden {\pb}kann, dass die Leo-Gesellschaft\orgindex{Oesterreichische Leo-Gesellschaft@Österreichische Leo-Gesellschaft|pw} ihre Zeit schon so sehr für gekommen hält,
               das ist ein böses Zeichen. Franz Ferdinand\pwindex{Franz Ferdinand von Oesterreich-Este 18.12.1863 – 28.06.1914@\textsc{Franz Ferdinand von Österreich-Este} (18.12.1863 – 28.06.1914), \emph{Erzherzog/Erzherzogin, Thronfolger/Thronfolgerin}|pw}
               wirft eben auch hier schon seine schwarzen Schatten voraus! Wie ich die Gesellschaft
               im Burgtheater\orgindex{Burgtheater@Burgtheater|pw} zu kennen glaube, werden sie mit
               Wonne und Schadenfreude und mit allen Übertreibungen der Strebsamkeit an der \label{K_L03559-4v}\edtext{Katholisisirung}{\lemma{\textnormal{\emph{Katholisisirung}}}\Cendnote{\textnormal{Die \emph{Österreichische
                     Leo-Gesellschaft}\orgindex{Oesterreichische Leo-Gesellschaft@Österreichische Leo-Gesellschaft|pwk} förderte explizit katholische Kunst und
                  Wissenschaft.}}}\label{K_L03559-4} des Repertoires mithelfen. Ich habe sehr das Gefühl, dass in
               dieser Beziehung ungeahnte Dinge bevorstehen. \label{K_L03559-5v}\edtext{Wer ljäben wird, wird sehen}{\lemma{\textnormal{\emph{Wer … sehen}}}\Cendnote{\textnormal{vermutlich eine jiddelnde Eindeutschung der französischen
                  Phrase »\begin{otherlanguage}{french}qui vivra, verra\end{otherlanguage}« (wer leben wird, wird
                  sehen)}}}\label{K_L03559-5}!\pend
           
\pstart
           Auf gutes Wiedersehen und viele herzliche Grüße {\\[\baselineskip]}Ihr \spacefill\mbox{Salten}\pend
           \leftskip=0em{}\selectlanguage{ngerman}\endnumbering\briefempfaengerindex{Schnitzler, Arthur@\textsc{Schnitzler, Arthur}!zzzSalten, Felix@\emph{von Felix Salten}!1912-09-021@{2. 9. 1912}|)be}\mylabel{L03559h}  \normalsize

\doendnotes{C}
\bigskip
\vfill

\clearpage

\footnotesize

\lohead{\textsc{register}}

% Definiere theindex-Environment komplett neu ohne reledmac
\makeatletter
\renewenvironment{theindex}{%
  \section*{\indexname}%
  \setlength{\parindent}{0pt}%
  \setlength{\parskip}{0pt plus 0.3pt}%
  \let\item\@idxitem
}{%
  \clearpage
}
\makeatother

\IfFileExists{\jobname-pw.ind}{\input{\jobname-pw.ind}}{}

\end{document}

      