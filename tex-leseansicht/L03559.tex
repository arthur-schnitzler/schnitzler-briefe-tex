%% latex-leseansicht-vorspann.tex
%% Vorspann für die Leseansicht.
%% Lädt die gemeinsame Datei latex-vorspann.tex mit nicht gesetztem Schalter.

\newif\ifkorrekturansicht
\korrekturansichtfalse

\input{../tex-inputs/latex-vorspann}

\begin{center}
            \textcolor{red}{ENTWURF, NICHT FERTIG KORRIGIERT}
                      \end{center}
            
         
         \renewcommand{\erwaehntePersonen}{Personen: Alfred von Berger, Otto Brahm,  Franz Ferdinand von Österreich-Este, Richard Kralik, Max Reinhardt, Rudolf Rittner, Ottilie Salten, Karl Gustav Vollmoeller}
         \renewcommand{\erwaehnteInstitutionen}{Institutionen: Burgtheater, Österreichische Leo-Gesellschaft}
         \renewcommand{\erwaehnteOrte}{Orte: Berghof, Prater, Rotunde, Unterach am Attersee, Wien}
         \renewcommand{\erwaehnteWerke}{Werke: Das Mirakel}
               \section[Felix Salten an Arthur Schnitzler, 2. 9. 1912]{ Felix Salten an Arthur Schnitzler, 2. 9. 1912}\nopagebreak\mylabel{v}\rehead{ }\begin{ledgroupsized}[t]{13cm}\normalsize\beginnumbering \toendnotes[C]{\smallbreak\pagebreak[2]} \Standort{CUL, Schnitzler, B 89, B 2.}
\physDesc{Briefkarte, 1321 Zeichen
\newline{}Handschrift: schwarze Tinte, lateinische Kurrent
\newline{}Ordnung: mit Bleistift von unbekannter Hand nummeriert:
                                    »274a« }\toendnotes[C]{\smallbreak}\pstart
           {\pb}Berghof\oindex{Berghof@\textbf{Berghof}|pw}, 2. IX. 12\pend
           \pstart{}Lieber,\pend\pstart
           ich hoffe sehr, dass \label{K_L03559-1v}\edtext{Reinhardt\pwindex{Reinhardt, Max 09.09.1873 – 30.10.1943@\textsc{Reinhardt, Max} (09.09.1873 – 30.10.1943), \emph{Theaterleiter, Regisseur, Schauspieler}|pw}s Mirakel\pwindex{Vollmoeller, Karl Gustav 07.05.1878 – 18.10.1948@\textsc{Vollmoeller, Karl Gustav} (07.05.1878 – 18.10.1948), \emph{Schriftsteller}!Mirakel1911-12-23@\strich\emph{Das Mirakel} {[}1911-12-23{]}|pw}}{\lemma{\textnormal{\emph{Reinhardts Mirakel}}}\Cendnote{\textnormal{\emph{Das Mirakel}\pwindex{Vollmoeller, Karl Gustav 07.05.1878 – 18.10.1948@\textsc{Vollmoeller, Karl Gustav} (07.05.1878 – 18.10.1948), \emph{Schriftsteller}!Mirakel1911-12-23@\strich\emph{Das Mirakel} {[}1911-12-23{]}|pwk} von Karl Gustav Vollmoeller\pwindex{Vollmoeller, Karl Gustav 07.05.1878 – 18.10.1948@\textsc{Vollmoeller, Karl Gustav} (07.05.1878 – 18.10.1948), \emph{Schriftsteller}|pwk} wurde
                  am 18. 9. 1912 in der Rotunde\oindex{Rotunde@\textbf{Rotunde}|pwk}
               im Wien\oindex{Wien@\textbf{Wien}|pwk}er Prater\oindex{Prater@\textbf{Prater}|pwk} erstmals deutsch gegeben. Die Inszenierung stammte von
                  Max Reinhardt\pwindex{Reinhardt, Max 09.09.1873 – 30.10.1943@\textsc{Reinhardt, Max} (09.09.1873 – 30.10.1943), \emph{Theaterleiter, Regisseur, Schauspieler}|pwk}, der Platz für 8.000 Zuschauer
                  geschaffen hatte.}}}\label{K_L03559-1h} verspätet aufgeführt wird, und dass mich also
               nichts dazu zwingt, die Eucharistische Luft in Wien\oindex{Wien@\textbf{Wien}|pw} zu atmen. Wenn Otti\pwindex{Salten, Ottilie 07.03.1868 – 22.06.1942@\textsc{Salten, Ottilie} (07.03.1868 – 22.06.1942), \emph{Schauspielerin}|pw} wieder da
               und der Berghof\oindex{Berghof@\textbf{Berghof}|pw} ruhiger geworden ist, möchte ich
               wol gern noch ein naar Wochen still hier arbeiten. Was sagen Sie zum Burgtheater\orgindex{Burgtheater@Burgtheater|pw}? Der arme Berger\pwindex{Berger, Alfred von 30.04.1853 – 24.08.1912@\textsc{Berger, Alfred von} (30.04.1853 – 24.08.1912), \emph{Schriftsteller, Journalist, Theaterleiter}|pw} tut mir leid, aber ich kann mir nicht helfen – wenn auch ein
                  Fi\textcolor{gray}{n}is oftmals besser ist als das Sterben, hier hat der Tod doch
               einen an sich schon nicht übermäßig gemüthlichen Menschen vor sehr unglücklichen
               Enttäuschungen bewahrt. Könnten wirBrahm\pwindex{Brahm, Otto 05.02.1856 – 28.11.1912@\textsc{Brahm, Otto} (05.02.1856 – 28.11.1912), \emph{Theaterleiter, Regisseur}|pw} oder
               vielleicht sogar Rudolf Rittner\pwindex{Rittner, Rudolf 30.06.1869 – 04.02.1943@\textsc{Rittner, Rudolf} (30.06.1869 – 04.02.1943), \emph{Theaterleiter, Schauspieler}|pw} bekommen, dann
               wäre doch vielleicht für die Zukunft ein gutes menschliches und künstlerisches
               Verhältnis zum Burgtheater\orgindex{Burgtheater@Burgtheater|pw} möglich. Aber das Herr
                  von Kralik\pwindex{Kralik, Richard 1852-10-01 – 1934-02-04@\textsc{Kralik, Richard} (1852-10-01 – 1934-02-04), \emph{Schriftsteller}|pw} als Director auch nur genannt
               werden {\pb}kann, dass die Leo-Gesellschaft\orgindex{Oesterreichische Leo-Gesellschaft@Österreichische Leo-Gesellschaft|pw} ihre Zeit schon so sehr für
               gekommen hält, das ist ein böses Zeichen. Franz
                  Ferdinand\pwindex{Franz Ferdinand von Oesterreich-Este 18.12.1863 – 28.06.1914@\textsc{Franz Ferdinand von Österreich-Este} (18.12.1863 – 28.06.1914), \emph{Erzherzog, Thronfolger}|pw} wirft eben auch hier schon seine schwarzen Schatten voraus! Wie ich
               die Gesellschaft\orgindex{Oesterreichische Leo-Gesellschaft@Österreichische Leo-Gesellschaft|pw} im Burgtheater\orgindex{Burgtheater@Burgtheater|pw} zu kennen glaube, werden sie mit Wonne und
               Schadenfreude und mit allen Übertreibungen der Strebsamkeit in der Katholisisirung
               des Burgtheaters\orgindex{Burgtheater@Burgtheater|pw} mithelfen. Ich habe sehr das
               Gefühl, dass in dieser Beziehung ungeahnte Dinge bevorstehen. Wer ljäben wird, wird
               sehen!\pend
           \pstart
           Auf gutes Wiedersehen und viele herzliche Grüße {\\[\baselineskip]}Ihr \spacefill\mbox{Salten}\pend
           \leftskip=0em{}
         
         \endnumbering\mylabel{h}\end{ledgroupsized}\begin{anhang}\end{anhang}\newcommand{\dateiname}{L03559}\newcommand{\titel}{Felix Salten an Arthur Schnitzler, 2. 9. 1912}\newcommand{\editorInnen}{Martin Anton Müller und Laura Untner}%% latex-leseansicht-abspann.tex
%% Abspann für die Leseansicht.
%% Der Schalter \ifkorrekturansicht ist bereits durch den Vorspann gesetzt.

%% latex-abspann.tex
%% Gemeinsamer Abspann für Korrekturansicht und Leseansicht.
%% Setzt den Schalter \ifkorrekturansicht voraus (gesetzt in den
%% einbindenden Dateien latex-korrekturansicht-abspann.tex bzw.
%% latex-leseansicht-abspann.tex).
%% ---------------------------------------------------------------

\normalsize

% Das esempio-Environment wird nur in der Leseansicht benötigt
\ifkorrekturansicht\else
\newenvironment{esempio}[3]%
{
    \vspace{1.5ex}
    \rlap{\underline{#1}}
    \par
    \setlength{\parindent}{0cm}
    \nopagebreak
    \leftskip=#2cm
    \rightskip=#3cm
}
{
    \par
}
\fi

\doendnotes{C}
\bigskip
\vfill

\clearpage

\footnotesize

\ifkorrekturansicht
  \lohead{\textsc{register}}
\fi

% theindex-Environment neu definieren ohne reledmac
\makeatletter
\renewenvironment{theindex}{%
  \ifkorrekturansicht
    \section*{\indexname}%
  \else
    \subsubsection*{Index der erwähnten Entitäten}%
  \fi
  \setlength{\parindent}{0pt}%
  \setlength{\parskip}{0pt plus 0.3pt}%
  \let\item\@idxitem
}{%
  \ifkorrekturansicht\clearpage\fi
}
\makeatother

\IfFileExists{\jobname-pw.ind}{\input{\jobname-pw.ind}}{}

% Quellenangabe nur in der Leseansicht
\ifkorrekturansicht\else
% Fallback-Definitionen, falls die .tex-Datei \titel etc. nicht gesetzt hat
\providecommand{\titel}{}
\providecommand{\editorInnen}{}
\providecommand{\dateiname}{\jobname}

\vspace{3cm}

\vfill

\footnotesize
\textsc{Quelle}: \titel. Herausgegeben von {\editorInnen}. In: \emph{Arthur Schnitzler: Briefwechsel mit Autorinnen und Autoren}.
 Digitale Edition, https://schnitzler-briefe.acdh.oeaw.ac.at/{\dateiname}.html (Stand \today)
\fi

\end{document}


      