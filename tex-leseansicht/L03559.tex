%% latex-leseansicht-vorspann.tex
%% Vorspann für die Leseansicht.
%% Lädt die gemeinsame Datei latex-vorspann.tex mit nicht gesetztem Schalter.

\newif\ifkorrekturansicht
\korrekturansichtfalse

\input{../tex-inputs/latex-vorspann}


         
         \renewcommand{\erwaehntePersonen}{Personen: Alfred von Berger, Otto Brahm,  Franz Ferdinand von Österreich-Este, Richard Kralik, Max Reinhardt, Rudolf Rittner, Felix Salten, Ottilie Salten, Hugo Thimig, Karl Gustav Vollmoeller}
         \renewcommand{\erwaehnteInstitutionen}{Institutionen: Burgtheater, Österreichische Leo-Gesellschaft}
         \renewcommand{\erwaehnteOrte}{Orte: Berghof, Prater, Rotunde, Tutzing, Unterach am Attersee, Wien}
         \renewcommand{\erwaehnteWerke}{Werke: Das Mirakel}
               \section[ Felix Salten an Arthur Schnitzler, 2. 9. 1912]{ Felix Salten an Arthur Schnitzler, 2. 9. 1912}\nopagebreak\mylabel{v}\rehead{ }\begin{ledgroupsized}[t]{13cm}\normalsize\beginnumbering\briefempfaengerindex{Schnitzler, Arthur@\textsc{Schnitzler, Arthur}!zzzSalten, Felix@\emph{von Felix Salten}!1912-09-021@{2. 9. 1912}|(be} \toendnotes[C]{\smallbreak\pagebreak[2]} \Standort{CUL, Schnitzler, B 89, B 2.}
\physDesc{Briefkarte, 1319 Zeichen
\newline{}Handschrift: schwarze Tinte, lateinische Kurrent
\newline{}Ordnung: mit Bleistift von unbekannter Hand nummeriert: »274a« }\toendnotes[C]{\smallbreak}\pstart
           \raggedleft{}{\pb}Berghof\oindex{Berghof@\textbf{Berghof}|pw}, 2. IX. 12\pend
           \pstart{}Lieber,\pend\pstart
           ich hoffe sehr, dass \label{K_L03559-1v}\edtext{Reinhardts\pwindex{Reinhardt, Max 09.09.1873 – 30.10.1943@\textsc{Reinhardt, Max} (09.09.1873 – 30.10.1943), \emph{Theaterleiter, Regisseur, Schauspieler}|pw}{ }Mirakel\pwindex{\textcolor{red}{\textsuperscript{XXXX1 indx}}!Mirakel1911-12-23@\strich\emph{Das Mirakel} {[}Vertonung, 1911-12-23{]}|pw}}{\lemma{\textnormal{\emph{Reinhardts Mirakel}}}\Cendnote{\textnormal{\emph{Das Mirakel}\pwindex{\textcolor{red}{\textsuperscript{XXXX1 indx}}!Mirakel1911-12-23@\strich\emph{Das Mirakel} {[}Vertonung, 1911-12-23{]}|pwk} von Karl Gustav Vollmoeller\pwindex{Vollmoeller, Karl Gustav 07.05.1878 – 18.10.1948@\textsc{Vollmoeller, Karl Gustav} (07.05.1878 – 18.10.1948), \emph{Schriftsteller}|pwk} wurde am 18. 9. 1912 in der Rotunde\oindex{Rotunde@\textbf{Rotunde}|pwk} im
                     Wien\oindex{Wien@\textbf{Wien}|pwk}er Prater\oindex{Prater@\textbf{Prater}|pwk} erstmals auf Deutsch gegeben, wo Platz für 8000 Zuschauerinnen
                  und Zuschauer war. Die Inszenierung stammte von Max Reinhardt\pwindex{Reinhardt, Max 09.09.1873 – 30.10.1943@\textsc{Reinhardt, Max} (09.09.1873 – 30.10.1943), \emph{Theaterleiter, Regisseur, Schauspieler}|pwk}. Schnitzler\pwindex{Schnitzler, Arthur 15.05.1862 – 21.10.1931@\textsc{Schnitzler, Arthur} (15.05.1862 – 21.10.1931), \emph{Schriftsteller, Mediziner}|pwk} besuchte
                  die Aufführung am 5. 10. 1912.}}}\label{K_L03559-1h} verspätet aufgeführt wird, und dass mich also
               nichts dazu zwingt, die \label{K_L03559-2v}\edtext{Eucharistische Luft}{\lemma{\textnormal{\emph{Eucharistische Luft}}}\Cendnote{\textnormal{In Wien\oindex{Wien@\textbf{Wien}|pwk} fand zwischen dem
                     12. 9. 1912 und dem 15. 9. 1912
                  der XXIII. internationale Eucharistische Kongress statt.}}}\label{K_L03559-2h} in Wien\oindex{Wien@\textbf{Wien}|pw} zu atmen. Wenn Otti\pwindex{Salten, Ottilie 07.03.1868 – 22.06.1942@\textsc{Salten, Ottilie} (07.03.1868 – 22.06.1942), \emph{Schauspielerin}|pw} wieder da und der Berghof\oindex{Berghof@\textbf{Berghof}|pw} ruhiger geworden ist, möchte ich wol gerne noch ein paar Wochen
               still hier arbeiten. Was sagen Sie zum \label{K_L03559-3v}\edtext{Burgtheater\orgindex{Burgtheater@Burgtheater|pw}}{\lemma{\textnormal{\emph{Burgtheater}}}\Cendnote{\textnormal{Alfred von Berger\pwindex{Berger, Alfred von 30.04.1853 – 24.08.1912@\textsc{Berger, Alfred von} (30.04.1853 – 24.08.1912), \emph{Schriftsteller, Journalist, Theaterleiter}|pwk}, der Direktor des \emph{Burgtheaters}\orgindex{Burgtheater@Burgtheater|pwk}, war am 24. 8. 1912
                  verstorben. Am 1. 9. 1912 wurde Hugo Thimig\pwindex{Thimig, Hugo 16.06.1854 – 24.09.1944@\textsc{Thimig, Hugo} (16.06.1854 – 24.09.1944), \emph{Theaterleiter, Schauspieler}|pwk} zum provisorischen – später dann zum
                  ordentlichen Leiter ernannt.}}}\label{K_L03559-3h}? Der arme Berger\pwindex{Berger, Alfred von 30.04.1853 – 24.08.1912@\textsc{Berger, Alfred von} (30.04.1853 – 24.08.1912), \emph{Schriftsteller, Journalist, Theaterleiter}|pw} tut mir leid, aber ich kann mir nicht helfen – wenn auch ein
                  Fi\textcolor{gray}{asco} oftmals besser ist als das Sterben, hier hat der Tod
               doch einen an sich schon nicht übermäßig glücklichen Menschen vor sehr unglücklichen
               Enttäuschungen bewahrt. Könnten wir Brahm\pwindex{Brahm, Otto 05.02.1856 – 28.11.1912@\textsc{Brahm, Otto} (05.02.1856 – 28.11.1912), \emph{Theaterleiter, Regisseur}|pw} oder
               vielleicht sogar Rudolf Rittner\pwindex{Rittner, Rudolf 30.06.1869 – 04.02.1943@\textsc{Rittner, Rudolf} (30.06.1869 – 04.02.1943), \emph{Theaterleiter, Schauspieler}|pw} bekommen, dann
               wäre doch vielleicht für die Zukunft ein gutes menschliches und künstlerisches
               Verhältnis zum Burgtheater\orgindex{Burgtheater@Burgtheater|pw} möglich. Aber
                  das{[}s{]} Herr von Kralik\pwindex{Kralik, Richard 1852-10-01 – 1934-02-04@\textsc{Kralik, Richard} (1852-10-01 – 1934-02-04), \emph{Schriftsteller}|pw}
               als Director auch nur genannt werden {\pb}kann, dass die Leo-Gesellschaft\orgindex{Oesterreichische Leo-Gesellschaft@Österreichische Leo-Gesellschaft|pw} ihre Zeit schon so sehr für gekommen hält,
               das ist ein böses Zeichen. Franz Ferdinand\pwindex{Franz Ferdinand von Oesterreich-Este 18.12.1863 – 28.06.1914@\textsc{Franz Ferdinand von Österreich-Este} (18.12.1863 – 28.06.1914), \emph{Erzherzog, Thronfolger}|pw}
               wirft eben auch hier schon seine schwarzen Schatten voraus! Wie ich die Gesellschaft
               im Burgtheater\orgindex{Burgtheater@Burgtheater|pw} zu kennen glaube, werden sie mit
               Wonne und Schadenfreude und mit allen Übertreibungen der Strebsamkeit an der \label{K_L03559-4v}\edtext{Katholisisirung}{\lemma{\textnormal{\emph{Katholisisirung}}}\Cendnote{\textnormal{Die \emph{Österreichische
                     Leo-Gesellschaft}\orgindex{Oesterreichische Leo-Gesellschaft@Österreichische Leo-Gesellschaft|pwk} förderte explizit katholische Kunst und
                  Wissenschaft.}}}\label{K_L03559-4h} des Repertoires mithelfen. Ich habe sehr das Gefühl, dass in
               dieser Beziehung ungeahnte Dinge bevorstehen. \label{K_L03559-5v}\edtext{Wer ljäben wird, wird sehen}{\lemma{\textnormal{\emph{Wer … sehen}}}\Cendnote{\textnormal{vermutlich eine jiddelnde Eindeutschung der französischen
                  Phrase »\begin{otherlanguage}{french}qui vivra, verra\end{otherlanguage}« (wer leben wird, wird
                  sehen)}}}\label{K_L03559-5h}!\pend
           \pstart
           Auf gutes Wiedersehen und viele herzliche Grüße {\\[\baselineskip]}Ihr \spacefill\mbox{Salten}\pend
           \leftskip=0em{}
         
         \endnumbering\mylabel{h}\end{ledgroupsized}  \newcommand{\dateiname}{L03559}\newcommand{\titel}{Felix Salten an Arthur Schnitzler, 2. 9. 1912}\newcommand{\editorInnen}{Martin Anton Müller und Laura Untner}%% latex-leseansicht-abspann.tex
%% Abspann für die Leseansicht.
%% Der Schalter \ifkorrekturansicht ist bereits durch den Vorspann gesetzt.

%% latex-abspann.tex
%% Gemeinsamer Abspann für Korrekturansicht und Leseansicht.
%% Setzt den Schalter \ifkorrekturansicht voraus (gesetzt in den
%% einbindenden Dateien latex-korrekturansicht-abspann.tex bzw.
%% latex-leseansicht-abspann.tex).
%% ---------------------------------------------------------------

\normalsize

% Das esempio-Environment wird nur in der Leseansicht benötigt
\ifkorrekturansicht\else
\newenvironment{esempio}[3]%
{
    \vspace{1.5ex}
    \rlap{\underline{#1}}
    \par
    \setlength{\parindent}{0cm}
    \nopagebreak
    \leftskip=#2cm
    \rightskip=#3cm
}
{
    \par
}
\fi

\doendnotes{C}
\bigskip
\vfill

\clearpage

\footnotesize

\ifkorrekturansicht
  \lohead{\textsc{register}}
\fi

% theindex-Environment neu definieren ohne reledmac
\makeatletter
\renewenvironment{theindex}{%
  \ifkorrekturansicht
    \section*{\indexname}%
  \else
    \subsubsection*{Index der erwähnten Entitäten}%
  \fi
  \setlength{\parindent}{0pt}%
  \setlength{\parskip}{0pt plus 0.3pt}%
  \let\item\@idxitem
}{%
  \ifkorrekturansicht\clearpage\fi
}
\makeatother

\IfFileExists{\jobname-pw.ind}{\input{\jobname-pw.ind}}{}

% Quellenangabe nur in der Leseansicht
\ifkorrekturansicht\else
% Fallback-Definitionen, falls die .tex-Datei \titel etc. nicht gesetzt hat
\providecommand{\titel}{}
\providecommand{\editorInnen}{}
\providecommand{\dateiname}{\jobname}

\vspace{3cm}

\vfill

\footnotesize
\textsc{Quelle}: \titel. Herausgegeben von {\editorInnen}. In: \emph{Arthur Schnitzler: Briefwechsel mit Autorinnen und Autoren}.
 Digitale Edition, https://schnitzler-briefe.acdh.oeaw.ac.at/{\dateiname}.html (Stand \today)
\fi

\end{document}


      