%% latex-korrekturansicht-vorspann.tex
%% Vorspann für die Korrekturansicht.
%% Lädt die gemeinsame Datei latex-vorspann.tex mit gesetztem Schalter.

\newif\ifkorrekturansicht
\korrekturansichttrue

\input{../tex-inputs/latex-vorspann}


\section[Felix Salten an Arthur Schnitzler, {[}30. 8. 1905?{]}]{L03411 Felix Salten an Arthur Schnitzler, {[}30. 8. 1905?{]}}
\nopagebreak\mylabel{L03411v}
\rehead{ }\normalsize\beginnumbering\briefempfaengerindex{Schnitzler, Arthur@\textsc{Schnitzler, Arthur}!zzzSalten, Felix@\emph{von Felix Salten}!1905-08-301@{{[}30. 8. 1905?{]}}|(be}
\toendnotes[C]{\smallbreak\pagebreak[2]}\Standort{CUL, Schnitzler, B 89, B 1.}
\physDesc{Bildpostkarte, 66 Zeichen
\newline{}Handschrift: Bleistift, lateinische Kurrent
\newline{}Ordnung: mit Bleistift von unbekannter Hand nummeriert: »203« }\toendnotes[C]{\smallbreak}\pstart{}{\pb}Herrn D\textsuperscript{r} Arthur Schnitzler\pend{}\pstart{}Wien XVIII.\oindex{XVIII., Waehring@\textbf{XVIII., Währing}, \emph{A.ADM3}|pw}\pend{}\pstart{}Spöttelgasse 7\oindex{Edmund-Weiss-Gasse 7@\textbf{Edmund-Weiß-Gasse 7}, \emph{Wohngebäude (K.WHS)}|pw}\pend{}{\bigskip}
\pstart
           \noindent{}\centering{}{\pb}\textcolor{gray}{\textbf{Auf dem \label{K_L03411-1v}\edtext{Penegal\oindex{Monte Penegal@\textbf{Monte Penegal}, \emph{T.MT}|pw}}{\lemma{\textnormal{\emph{Penegal}}}\Cendnote{\textnormal{Die Postkarte ist undatiert und der
                     Poststempel nicht zu entziffern, weswegen externe Faktoren für die Datierung
                     herangezogen werden müssen. Innerhalb der weitgehend chronologischen
                     Reihenfolge der überlieferten Korrespondenzstücke Saltens\pwindex{Salten, Felix 06.09.1869 – 08.10.1945@\textsc{Salten, Felix} (06.09.1869 – 08.10.1945), \emph{Schriftsteller/Schriftstellerin, Journalist/Journalistin, Chefredakteur/Chefredakteurin}|pwk} an Schnitzler liegt die Karte im Sommer 1905. Für den 23. 8. 1905 erwähnt
                        Schnitzlers{ }\emph{Tagebuch}\pwindex{Tagebuch@\emph{Tagebuch}|pwk}, dass Salten\pwindex{Salten, Felix 06.09.1869 – 08.10.1945@\textsc{Salten, Felix} (06.09.1869 – 08.10.1945), \emph{Schriftsteller/Schriftstellerin, Journalist/Journalistin, Chefredakteur/Chefredakteurin}|pwk} nach Südtirol\oindex{Suedtirol@\textbf{Südtirol}, \emph{A.ADM2}|pwk}
                     fahre. Für den 4. 9. 1905 ist die nächste Begegnung festgehalten, sodass sich
                     die Karte im dazwischenliegenden Zeitraum verorten lässt. In Saltens\pwindex{Salten, Felix 06.09.1869 – 08.10.1945@\textsc{Salten, Felix} (06.09.1869 – 08.10.1945), \emph{Schriftsteller/Schriftstellerin, Journalist/Journalistin, Chefredakteur/Chefredakteurin}|pwk}{ } Nachlass wird unter seinen Texten ein
                     ungezeichneter Zeitungsausschnitt über das Kaisermanöver in Romeno\oindex{Romeno@\textbf{Romeno}, \emph{A.ADM3}|pwk} überliefert, datiert mit
                        28. 8. 1905 ([O. V. = Felix Salten\pwindex{Salten, Felix 06.09.1869 – 08.10.1945@\textsc{Salten, Felix} (06.09.1869 – 08.10.1945), \emph{Schriftsteller/Schriftstellerin, Journalist/Journalistin, Chefredakteur/Chefredakteurin}|pwk}]: \emph{Manöverfahrt. (Von
                           unserem Spezialberichterstatter)}\pwindex{Manoeverfahrt. (Von unserem Spezialberichterstatter)@\emph{Manöverfahrt. (Von unserem Spezialberichterstatter)}|pwk}. In: \emph{Die Zeit}\pwindex{Zeit@\emph{Die Zeit}|pwk}, Jg. 4, Nr. 1052, 30. 8. 1905,
                        S. 2). Im Nachlass Hermann
                        Bahrs\pwindex{Bahr, Hermann 19.07.1863 – 15.01.1934@\textsc{Bahr, Hermann} (19.07.1863 – 15.01.1934), \emph{Schriftsteller/Schriftstellerin, Kritiker/Kritikerin}|pwk} hat sich das selbe Postkartenmotiv erhalten, mit weitgehend
                     identer, knapper handschriftlicher Beschriftung (\emph{Theatermuseum}, AM 60.042Ba). Auch diese Karte
                     ist nicht datiert, vom Poststempel können aber die Tagesangabe
                        (»30«) und der Beginn der Ortsangabe (»Me{[}ndel{]}\oindex{Mendelgebirge@\textbf{Mendelgebirge}, \emph{Gebirge (N.GBR)}|pwk}«) entnommen werden, so dass geschlossen werden kann, dass die vorliegende
                     Karte an Schnitzler zeitgleich abgefasst
                     und abgesandt sein dürfte.}}}\label{K_L03411-1} (Mendel\oindex{Mendelgebirge@\textbf{Mendelgebirge}, \emph{Gebirge (N.GBR)}|pw}).}}\pend
           \vspace{1em}
\pstart
           \noindent{}{\pb}herzlichst Ihr
                  \spacefill\mbox{S.}\pend
           \selectlanguage{ngerman}\endnumbering\briefempfaengerindex{Schnitzler, Arthur@\textsc{Schnitzler, Arthur}!zzzSalten, Felix@\emph{von Felix Salten}!1905-08-301@{{[}30. 8. 1905?{]}}|)be}\mylabel{L03411h}  \normalsize

\doendnotes{C}
\bigskip
\vfill

\clearpage

\footnotesize

\lohead{\textsc{register}}

% Definiere theindex-Environment komplett neu ohne reledmac
\makeatletter
\renewenvironment{theindex}{%
  \section*{\indexname}%
  \setlength{\parindent}{0pt}%
  \setlength{\parskip}{0pt plus 0.3pt}%
  \let\item\@idxitem
}{%
  \clearpage
}
\makeatother

\IfFileExists{\jobname-pw.ind}{\input{\jobname-pw.ind}}{}

\end{document}

      