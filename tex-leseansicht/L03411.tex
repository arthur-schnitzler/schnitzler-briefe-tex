%% latex-leseansicht-vorspann.tex
%% Vorspann für die Leseansicht.
%% Lädt die gemeinsame Datei latex-vorspann.tex mit nicht gesetztem Schalter.

\newif\ifkorrekturansicht
\korrekturansichtfalse

\input{../tex-inputs/latex-vorspann}


\section[Felix Salten an Arthur Schnitzler, {{[}}30. 8. 1905?{{]}}]{L03411 Felix Salten an Arthur Schnitzler, {[}30. 8. 1905?{]}}
\nopagebreak\mylabel{L03411v}
\rehead{ }\normalsize\beginnumbering\briefempfaengerindex{Schnitzler, Arthur@\textsc{Schnitzler, Arthur}!zzzSalten, Felix@\emph{von Felix Salten}!1905-08-301@{{[}30. 8. 1905?{]}}|(be}
\toendnotes[C]{\smallbreak\pagebreak[2]}
\correspDesc{Versand  durch Felix Salten am [30. 8. 1905?] in Mendel
\newline{}Erhalt  durch Arthur Schnitzler im Zeitraum [zwischen 1. 9. und
                  5. 9. 1905?] in Wien}\toendnotes[C]{\smallbreak}
\Standort{CUL, Schnitzler, B 89, B 1.}
\physDesc{Bildpostkarte, 66 Zeichen
\newline{}Handschrift: Bleistift, lateinische Kurrent
\newline{}Ordnung: mit Bleistift von unbekannter Hand nummeriert: »203« }\toendnotes[C]{\smallbreak}\pstart{}{\pb}Herrn D\textsuperscript{r} Arthur Schnitzler\pend{}\pstart{}Wien XVIII.\oindex{XVIII., Währing@\textbf{XVIII., Währing}, \emph{Verwaltungsgebiet}|pw}\pend{}\pstart{}Spöttelgasse 7\oindex{Wien@\textbf{Wien}!XVIII., Währing@\textbf{XVIII., Währing}!Edmund-Weiß-Gasse 7@\textbf{Edmund-Weiß-Gasse 7}, \emph{Wohngebäude}|pw}\pend{}{\bigskip}
\pstart
           \noindent{}\centering{}{\pb}\textcolor{gray}{\textbf{Auf dem \label{K_L03411-1v}\edtext{Penegal\oindex{Monte Penegal@\textbf{Monte Penegal}, \emph{Berg}|pw}}{\lemma{\textnormal{\emph{Penegal}}}\Cendnote{\textnormal{Die Postkarte ist undatiert und der
                     Poststempel nicht zu entziffern, weswegen externe Faktoren für die Datierung
                     herangezogen werden müssen. Innerhalb der weitgehend chronologischen
                     Reihenfolge der überlieferten Korrespondenzstücke Saltens\pwindex{Salten, Felix 6.\,9.\,1869 Budapest – 8.\,10.\,1945 Zürich@\textsc{Salten, Felix} (6.\,9.\,1869 Budapest – 8.\,10.\,1945 Zürich), \emph{Schriftsteller, Journalist, Chefredakteur}|pwk} an Schnitzler liegt die Karte im Sommer 1905. Für den 23. 8. 1905 erwähnt
                        Schnitzlers{ }\emph{Tagebuch}\pwindex{Schnitzler, Arthur 15.\,5.\,1862 Wien – 21.\,10.\,1931 ebd.@\textsc{Schnitzler, Arthur} (15.\,5.\,1862 Wien – 21.\,10.\,1931 ebd.), \emph{Schriftsteller, Mediziner}!Tagebuch@\strich\emph{Tagebuch}|pwk}, dass Salten\pwindex{Salten, Felix 6.\,9.\,1869 Budapest – 8.\,10.\,1945 Zürich@\textsc{Salten, Felix} (6.\,9.\,1869 Budapest – 8.\,10.\,1945 Zürich), \emph{Schriftsteller, Journalist, Chefredakteur}|pwk} nach Südtirol\oindex{Südtirol@\textbf{Südtirol}, \emph{Verwaltungsgebiet}|pwk}
                     fahre. Für den 4. 9. 1905 ist die nächste Begegnung festgehalten, sodass sich
                     die Karte im dazwischenliegenden Zeitraum verorten lässt. In Saltens\pwindex{Salten, Felix 6.\,9.\,1869 Budapest – 8.\,10.\,1945 Zürich@\textsc{Salten, Felix} (6.\,9.\,1869 Budapest – 8.\,10.\,1945 Zürich), \emph{Schriftsteller, Journalist, Chefredakteur}|pwk}{ } Nachlass wird unter seinen Texten ein
                     ungezeichneter Zeitungsausschnitt über das Kaisermanöver in Romeno\oindex{Romeno@\textbf{Romeno}, \emph{Verwaltungsgebiet}|pwk} überliefert, datiert mit
                        28. 8. 1905 ([O. V. = Felix Salten\pwindex{Salten, Felix 6.\,9.\,1869 Budapest – 8.\,10.\,1945 Zürich@\textsc{Salten, Felix} (6.\,9.\,1869 Budapest – 8.\,10.\,1945 Zürich), \emph{Schriftsteller, Journalist, Chefredakteur}|pwk}]: \emph{Manöverfahrt. (Von
                           unserem Spezialberichterstatter)}\pwindex{Manöverfahrt. (Von unserem Spezialberichterstatter)@\emph{Manöverfahrt. (Von unserem Spezialberichterstatter)}|pwk}. In: \emph{Die Zeit}\pwindex{Zeit@\emph{Die Zeit}|pwk}, Jg. 4, Nr. 1052, 30. 8. 1905,
                        S. 2). Im Nachlass Hermann
                        Bahrs\pwindex{Bahr, Hermann 19.\,7.\,1863 Linz – 15.\,1.\,1934 München@\textsc{Bahr, Hermann} (19.\,7.\,1863 Linz – 15.\,1.\,1934 München), \emph{Schriftsteller, Kritiker}|pwk} hat sich das selbe Postkartenmotiv erhalten, mit weitgehend
                     identer, knapper handschriftlicher Beschriftung (\emph{Theatermuseum}, AM 60.042Ba). Auch diese Karte
                     ist nicht datiert, vom Poststempel können aber die Tagesangabe
                        (»30«) und der Beginn der Ortsangabe (»Me{[}ndel{]}\oindex{Mendelgebirge@\textbf{Mendelgebirge}, \emph{Gebirge}|pwk}«) entnommen werden, so dass geschlossen werden kann, dass die vorliegende
                     Karte an Schnitzler zeitgleich abgefasst
                     und abgesandt sein dürfte.}}}\label{K_L03411-1} (Mendel\oindex{Mendelgebirge@\textbf{Mendelgebirge}, \emph{Gebirge}|pw}).}}\pend
           \vspace{1em}
\pstart
           \noindent{}{\pb}herzlichst Ihr
                  \spacefill\mbox{S.}\pend
           \selectlanguage{ngerman}\endnumbering\briefempfaengerindex{Schnitzler, Arthur@\textsc{Schnitzler, Arthur}!zzzSalten, Felix@\emph{von Felix Salten}!1905-08-301@{{[}30. 8. 1905?{]}}|)be}\mylabel{L03411h}  \newcommand{\dateiname}{L03411}\newcommand{\titel}{Felix Salten an Arthur Schnitzler, [30. 8. 1905?]}\newcommand{\editorInnen}{Martin Anton Müller und Laura Untner}%% latex-leseansicht-abspann.tex
%% Abspann für die Leseansicht.
%% Der Schalter \ifkorrekturansicht ist bereits durch den Vorspann gesetzt.

%% latex-abspann.tex
%% Gemeinsamer Abspann für Korrekturansicht und Leseansicht.
%% Setzt den Schalter \ifkorrekturansicht voraus (gesetzt in den
%% einbindenden Dateien latex-korrekturansicht-abspann.tex bzw.
%% latex-leseansicht-abspann.tex).
%% ---------------------------------------------------------------

\normalsize

% Das esempio-Environment wird nur in der Leseansicht benötigt
\ifkorrekturansicht\else
\newenvironment{esempio}[3]%
{
    \vspace{1.5ex}
    \rlap{\underline{#1}}
    \par
    \setlength{\parindent}{0cm}
    \nopagebreak
    \leftskip=#2cm
    \rightskip=#3cm
}
{
    \par
}
\fi

\doendnotes{C}
\bigskip
\vfill

\clearpage

\footnotesize

\ifkorrekturansicht
  \lohead{\textsc{register}}
\fi

% theindex-Environment neu definieren ohne reledmac
\makeatletter
\renewenvironment{theindex}{%
  \ifkorrekturansicht
    \section*{\indexname}%
  \else
    \subsubsection*{Index der erwähnten Entitäten}%
  \fi
  \setlength{\parindent}{0pt}%
  \setlength{\parskip}{0pt plus 0.3pt}%
  \let\item\@idxitem
}{%
  \ifkorrekturansicht\clearpage\fi
}
\makeatother

\IfFileExists{\jobname-pw.ind}{\input{\jobname-pw.ind}}{}

% Quellenangabe nur in der Leseansicht
\ifkorrekturansicht\else
% Fallback-Definitionen, falls die .tex-Datei \titel etc. nicht gesetzt hat
\providecommand{\titel}{}
\providecommand{\editorInnen}{}
\providecommand{\dateiname}{\jobname}

\vspace{3cm}

\vfill

\footnotesize
\textsc{Quelle}: \titel. Herausgegeben von {\editorInnen}. In: \emph{Arthur Schnitzler: Briefwechsel mit Autorinnen und Autoren}.
 Digitale Edition, https://schnitzler-briefe.acdh.oeaw.ac.at/{\dateiname}.html (Stand \today)
\fi

\end{document}


