%% latex-leseansicht-vorspann.tex
%% Vorspann für die Leseansicht.
%% Lädt die gemeinsame Datei latex-vorspann.tex mit nicht gesetztem Schalter.

\newif\ifkorrekturansicht
\korrekturansichtfalse

\input{../tex-inputs/latex-vorspann}

\begin{center}
            \textcolor{red}{ENTWURF, NICHT FERTIG KORRIGIERT}
                      \end{center}
            
         \renewcommand{\erwaehnteOrte}{Orte: Edmund-Weiß-Gasse 7, Mendelgebirge, Monte Penegal, Südtirol, Wien, XVIII., Währing}
         \renewcommand{\erwaehnteWerke}{Werke: Tagebuch}
               \section[Felix Salten an Arthur Schnitzler, {[}zwischen 25. 8. und 3. 9. 1905?{]}]{ Felix Salten an Arthur Schnitzler,
               {[}zwischen 25. 8. und 3. 9. 1905?{]}}\nopagebreak\mylabel{v}\rehead{ }\begin{ledgroupsized}[t]{13cm}\normalsize\beginnumbering \toendnotes[C]{\smallbreak\pagebreak[2]} \Standort{CUL, Schnitzler, B 89, B 1.}
\physDesc{Bildpostkarte, 66 Zeichen
\newline{}Handschrift: Bleistift, lateinische Kurrent
\newline{}Ordnung: mit Bleistift von unbekannter Hand nummeriert: »203« }\toendnotes[C]{\smallbreak}\pstart{}{\pb}Herrn D\textsuperscript{r} Arthur Schnitzler\pend{}\pstart{}Wien XVIII.\oindex{XVIII., Waehring@\textbf{XVIII., Währing}|pw}\pend{}\pstart{}Spöttelgasse 7\oindex{Edmund-Weiss-Gasse 7@\textbf{Edmund-Weiß-Gasse 7}|pw}\pend{}{\bigskip}\pstart
           \noindent{}\centering{}{\pb}\textcolor{gray}{\textbf{Auf dem \label{K_L03411-1v}\edtext{Penegal\oindex{Monte Penegal@\textbf{Monte Penegal}|pw}}{\lemma{\textnormal{\emph{Penegal}}}\Cendnote{\textnormal{Die Postkarte ist undatiert und
                        der Poststempel nicht zu entziffern, weswegen externe Faktoren für die
                        Datierung herangezogen werden müssen. Innerhalb der weitgehend
                        chronologischen Reihenfolge der überlieferten Korrespondenzstücke Salten\pwindex{Salten, Felix 06.09.1869 – 08.10.1945@\textsc{Salten, Felix} (06.09.1869 – 08.10.1945), \emph{Schriftsteller, Journalist}|pwk}s an Schnitzler\pwindex{Schnitzler, Arthur 15.05.1862 – 21.10.1931@\textsc{Schnitzler, Arthur} (15.05.1862 – 21.10.1931), \emph{Schriftsteller, Mediziner}|pwk} liegt die Karte im
                           Sommer 1905. Für den 23. 8. 1905 erwähnt Schnitzler\pwindex{Schnitzler, Arthur 15.05.1862 – 21.10.1931@\textsc{Schnitzler, Arthur} (15.05.1862 – 21.10.1931), \emph{Schriftsteller, Mediziner}|pwk}s \emph{Tagebuch}\pwindex{\textcolor{red}{\textsuperscript{XXXX1 indx}}!Tagebuch1981 – 2000@\strich\emph{Tagebuch} {[}Hrsg., 1981 – 2000{]}|pwk}, dass Salten\pwindex{Salten, Felix 06.09.1869 – 08.10.1945@\textsc{Salten, Felix} (06.09.1869 – 08.10.1945), \emph{Schriftsteller, Journalist}|pwk} nach
                           Südtirol\oindex{Suedtirol@\textbf{Südtirol}|pwk} fahre. Für den 4. 9. 1905 ist
                        die nächste Begegnung festgehalten, sodass die Karte im dazwischenliegenden
                        Zeitraum zu verorten sein dürfte. Eine weitere Bestätigung erhält das durch
                        die Erwähnung von Penegal\oindex{Monte Penegal@\textbf{Monte Penegal}|pwk} in einem
                        ungezeichneten Bericht über das Kaisermanöver in Romeno\oindex{XXXX Ortsangabe fehlt|pwk}, die von Salten\pwindex{Salten, Felix 06.09.1869 – 08.10.1945@\textsc{Salten, Felix} (06.09.1869 – 08.10.1945), \emph{Schriftsteller, Journalist}|pwk} verfasst sein dürfte. ([O. V. = Felix Salten\pwindex{Salten, Felix 06.09.1869 – 08.10.1945@\textsc{Salten, Felix} (06.09.1869 – 08.10.1945), \emph{Schriftsteller, Journalist}|pwk}?:] \emph{Manöverfahrt. (Von unserem
                              Spezialberichterstatter)}\textcolor{red}{\textsuperscript{XXXX indx}}. In: \emph{Die
                              Zeit}\textcolor{red}{\textsuperscript{XXXX indx}}, Jg. 4, Nr. 1.052,
                              30. 8. 1905, S. 2.)}}}\label{K_L03411-1h} (Mendel\oindex{Mendelgebirge@\textbf{Mendelgebirge}|pw}).}}\pend
           \pstart
           herzlichst Ihr \spacefill\mbox{S.}\pend
           
         
         \endnumbering\mylabel{h}\end{ledgroupsized}  \newcommand{\dateiname}{L03411}\newcommand{\titel}{Felix Salten an Arthur Schnitzler, [zwischen 25. 8. und 3. 9. 1905?]}\newcommand{\editorInnen}{Martin Anton Müller und Laura Untner}%% latex-leseansicht-abspann.tex
%% Abspann für die Leseansicht.
%% Der Schalter \ifkorrekturansicht ist bereits durch den Vorspann gesetzt.

%% latex-abspann.tex
%% Gemeinsamer Abspann für Korrekturansicht und Leseansicht.
%% Setzt den Schalter \ifkorrekturansicht voraus (gesetzt in den
%% einbindenden Dateien latex-korrekturansicht-abspann.tex bzw.
%% latex-leseansicht-abspann.tex).
%% ---------------------------------------------------------------

\normalsize

% Das esempio-Environment wird nur in der Leseansicht benötigt
\ifkorrekturansicht\else
\newenvironment{esempio}[3]%
{
    \vspace{1.5ex}
    \rlap{\underline{#1}}
    \par
    \setlength{\parindent}{0cm}
    \nopagebreak
    \leftskip=#2cm
    \rightskip=#3cm
}
{
    \par
}
\fi

\doendnotes{C}
\bigskip
\vfill

\clearpage

\footnotesize

\ifkorrekturansicht
  \lohead{\textsc{register}}
\fi

% theindex-Environment neu definieren ohne reledmac
\makeatletter
\renewenvironment{theindex}{%
  \ifkorrekturansicht
    \section*{\indexname}%
  \else
    \subsubsection*{Index der erwähnten Entitäten}%
  \fi
  \setlength{\parindent}{0pt}%
  \setlength{\parskip}{0pt plus 0.3pt}%
  \let\item\@idxitem
}{%
  \ifkorrekturansicht\clearpage\fi
}
\makeatother

\IfFileExists{\jobname-pw.ind}{\input{\jobname-pw.ind}}{}

% Quellenangabe nur in der Leseansicht
\ifkorrekturansicht\else
% Fallback-Definitionen, falls die .tex-Datei \titel etc. nicht gesetzt hat
\providecommand{\titel}{}
\providecommand{\editorInnen}{}
\providecommand{\dateiname}{\jobname}

\vspace{3cm}

\vfill

\footnotesize
\textsc{Quelle}: \titel. Herausgegeben von {\editorInnen}. In: \emph{Arthur Schnitzler: Briefwechsel mit Autorinnen und Autoren}.
 Digitale Edition, https://schnitzler-briefe.acdh.oeaw.ac.at/{\dateiname}.html (Stand \today)
\fi

\end{document}


      