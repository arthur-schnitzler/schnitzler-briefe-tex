%% latex-leseansicht-vorspann.tex
%% Vorspann für die Leseansicht.
%% Lädt die gemeinsame Datei latex-vorspann.tex mit nicht gesetztem Schalter.

\newif\ifkorrekturansicht
\korrekturansichtfalse

\input{../tex-inputs/latex-vorspann}


\section[ Paul Goldmann an Arthur Schnitzler, 2. [1.? 1897]]{L02792 Paul Goldmann an Arthur Schnitzler,  2. [1.? 1897]}
\nopagebreak\mylabel{L02792v}
\rehead{ }\normalsize\beginnumbering\briefempfaengerindex{Schnitzler, Arthur@\textsc{Schnitzler, Arthur}!zzzGoldmann, Paul@\emph{von Paul Goldmann}!1897-01-021@{2. [1.? 1897]}|(be}
\toendnotes[C]{\smallbreak\pagebreak[2]}
\correspDesc{Versand  durch Paul Goldmann am 2. [1.? 1897] in Paris
\newline{}Erhalt  durch Arthur Schnitzler im Zeitraum [3. 1. 1897
                  – 7. 1. 1897?] in Wien}\toendnotes[C]{\smallbreak}
\Standort{DLA, A:Schnitzler, HS.NZ85.1.3166.}
\physDesc{Brief, 5 Blätter, 18 Seiten, 8748 Zeichen
\newline{}Handschrift: blaue Tinte, deutsche Kurrent
\newline{}Schnitzler: 1) mit Bleistift das Jahr »96« vermerkt sowie die Tagesangabe des Datums unterstrichen
                                 und mit »?« kommentiert  2) mit rotem Buntstift neun Unterstreichungen}\toendnotes[C]{\smallbreak}
\pstart
           {\pb}\textcolor{gray}{\textbf{\textbf{Frankfurter Zeitung\orgindex{Frankfurter Zeitung@Frankfurter Zeitung|pw}}}}\pend
           
\pstart
           \textcolor{gray}{\textbf{(\begin{otherlanguage}{french}Gazette de Francfort\end{otherlanguage}\orgindex{Frankfurter Zeitung@Frankfurter Zeitung|pw}).}}\pend
           
\pstart
           \textcolor{gray}{\textbf{\textbf{\begin{otherlanguage}{french}Fondateur M.\end{otherlanguage}{ }L. Sonnemann\pwindex{Sonnemann, Leopold 29.\,10.\,1831 Höchberg – 30.\,10.\,1909 Frankfurt am Main@\textsc{Sonnemann, Leopold} (29.\,10.\,1831 Höchberg – 30.\,10.\,1909 Frankfurt am Main), \emph{Journalist, Herausgeber}|pw}.}}}\pend
           
\pstart
           \begin{otherlanguage}{french}\textcolor{gray}{\textbf{Journal\pwindex{Frankfurter Zeitung@\emph{Frankfurter Zeitung}|pwv} politique,
                        financier,}}\end{otherlanguage}\pend
           
\pstart
           \begin{otherlanguage}{french}\textcolor{gray}{\textbf{commercial et littéraire.}}\end{otherlanguage}\pend
           
\pstart
           \begin{otherlanguage}{french}\textcolor{gray}{\textbf{\textbf{Paraissant trois fois par jour.}}}\end{otherlanguage}\hfill \textsc{Paris\oindex{Paris@\textbf{Paris}, \emph{Hauptstadt}|pw}}, \label{K_L02792-1v}\edtext{2. December}{\lemma{\textnormal{\emph{2. December}}}\Cendnote{\textnormal{Es ist davon auszugehen, dass Goldmann\pwindex{Goldmann, Paul 31.\,1.\,1865 Breslau – 25.\,9.\,1935 Wien@\textsc{Goldmann, Paul} (31.\,1.\,1865 Breslau – 25.\,9.\,1935 Wien), \emph{Schriftsteller, Journalist}|pwk} den Brief falsch datiert und
                        nicht am 2. 12. 1896, sondern am 2. 1. 1897 verfasst hat. Dafür spricht, dass er
                           Schnitzler eingangs ein frohes
                        neues Jahr wünscht.}}}\label{K_L02792-1}.\pend
           
\pstart
           \begin{otherlanguage}{french}\textcolor{gray}{\textbf{\textbf{Bureau à Paris\oindex{Paris@\textbf{Paris}, \emph{Hauptstadt}|pw}}}}\end{otherlanguage}\pend
           
\pstart
           \begin{otherlanguage}{french}\textcolor{gray}{\textbf{\textbf{24. Rue Feydeau\oindex{rue Feydeau@\textbf{rue Feydeau}, \emph{Straße}|pw}.}}}\end{otherlanguage}\pend
           
\pstart{}Mein lieber Freund,\pend\vspace{0.5em}
\pstart
           Ich wünſche Dir von Herzen ein glückliches neues Jahr. Im
               alten Jahr waren die Tage, die ich mit Dir verlebt, für mich
               wohl das Beſte. Ich danke Dir \strikeout{\textcolor{gray}{×}\-\textcolor{gray}{×}\-\textcolor{gray}{×}} vielmals für alle Deine Treue und Güte{\dotsseven}\pend
           
\pstart
           Sehr habe ich mich mit Deinem lieben ausführlichen Briefe gefreut. Er hätte gleich
               beantwortet werden{ }ſollen. In jenen Tagen hatte ich keine Zeit dazu, und dann kam ein{ }ſchrecklicher \strikeout{Zuſ\textcolor{gray}{am}} Zuſammenbruch: neue Erſcheinungen der gewiſſen \label{K_L02792-2v}\edtext{Krankheit}{\lemma{\textnormal{\emph{Krankheit}}}\Cendnote{\textnormal{vermutlich Syphilis}}}\label{K_L02792-2}, Verſchlimmerung des Augenübels, eine vom Arzt
               conſtatirte unheilbare \label{K_L02792-3v}\edtext{\textsc{Mydriase}}{\lemma{\textnormal{\emph{Mydriase}}}\Cendnote{\textnormal{Pupillenerweiterung}}}\label{K_L02792-3}, {\pb}mit Möglichkeit der Verſchlimmerung, vielleicht gar
               des Sehverluſtes. Was{ }ſoll ich das Alles aufzählen? Seitdem habe ich nicht mehr die
               Kraft, irgend etwas zu thun. Ich gehe nirgends hin, weiſe alle Beſuche ab, bleibe bis
                  Mittag im Bett liegen und denke nur über das Sterben nach. In den
               Schmerz miſcht{ }ſich die Reue, in die Todes- und Selbſtmord-Gedanken die Sehnſucht
               nach dem Leben, nach dem ich heißer begehre als je. Das{ }ſind{ }ſchlimme Tage, und Du
               begreifſt, daß \strikeout{\textcolor{gray}{×}h} Dein Brief
               unbeantwortet bleiben mußte. Nun möchte ich Dir aber trotzdem{ }ſagen, daß ich oft an
               Dich denke, und{ }ſo raffe ich mich auf und{ }ſchreibe Dir doch{\dotssix}\pend
           
\pstart
           {\pb}Vor einiger Zeit war ich bei \textsc{Thorel\pwindex{Thorel, Jean 11.\,9.\,1859 Éragny – 20.\,8.\,1916 Enghien-les-Bains@\textsc{Thorel, Jean} (11.\,9.\,1859 Éragny – 20.\,8.\,1916 Enghien-les-Bains), \emph{Übersetzer, Dramatiker}|pw}}. Durch die \label{K_L02792-4v}\edtext{Directons-Kriſis im
                  »\textsc{Odéon\orgindex{Odéon@Odéon|pw}}«}{\lemma{\textnormal{\emph{Directons-Krisis im »Odéon«}}}\Cendnote{\textnormal{Zwischen 14. 6. 1896 und 29. 10. 1896 waren Paul Ginisty\pwindex{Ginisty, Paul 4.\,4.\,1855 Paris – 5.\,3.\,1932 ebd.@\textsc{Ginisty, Paul} (4.\,4.\,1855 Paris – 5.\,3.\,1932 ebd.), \emph{Schriftsteller, Theaterleiter}|pwk} und André Antoine\pwindex{Antoine, André 31.\,1.\,1858 Limoges – 23.\,10.\,1943 Le Pouliguen@\textsc{Antoine, André} (31.\,1.\,1858 Limoges – 23.\,10.\,1943 Le Pouliguen), \emph{Theaterleiter, Schauspieler}|pwk} die Direktoren des \emph{Odéon}\orgindex{Odéon@Odéon|pwk}-Theaters. Danach hatte Ginisty\pwindex{Ginisty, Paul 4.\,4.\,1855 Paris – 5.\,3.\,1932 ebd.@\textsc{Ginisty, Paul} (4.\,4.\,1855 Paris – 5.\,3.\,1932 ebd.), \emph{Schriftsteller, Theaterleiter}|pwk} die Funktion alleine inne.}}}\label{K_L02792-4} und den Weggang
                  \textsc{Antoines\pwindex{Antoine, André 31.\,1.\,1858 Limoges – 23.\,10.\,1943 Le Pouliguen@\textsc{Antoine, André} (31.\,1.\,1858 Limoges – 23.\,10.\,1943 Le Pouliguen), \emph{Theaterleiter, Schauspieler}|pw}} iſt \label{K_L02792-5v}\edtext{eine unſerer Combinationen}{\lemma{\textnormal{\emph{eine … Combinationen}}}\Cendnote{\textnormal{Hier im Sinne von: Überlegungen, siehe XXXX Auszeichnungsfehler: Dokument L02776 nicht gefunden. }}}\label{K_L02792-5} geſtört
               worden. \textsc{Thorel\pwindex{Thorel, Jean 11.\,9.\,1859 Éragny – 20.\,8.\,1916 Enghien-les-Bains@\textsc{Thorel, Jean} (11.\,9.\,1859 Éragny – 20.\,8.\,1916 Enghien-les-Bains), \emph{Übersetzer, Dramatiker}|pw}} hat dem übrigbleibenden Director \textsc{Ginisty\pwindex{Ginisty, Paul 4.\,4.\,1855 Paris – 5.\,3.\,1932 ebd.@\textsc{Ginisty, Paul} (4.\,4.\,1855 Paris – 5.\,3.\,1932 ebd.), \emph{Schriftsteller, Theaterleiter}|pw}} zwar das Stück\pwindex{Schnitzler, Arthur 15.\,5.\,1862 Wien – 21.\,10.\,1931 ebd.@\textsc{Schnitzler, Arthur} (15.\,5.\,1862 Wien – 21.\,10.\,1931 ebd.), \emph{Schriftsteller, Mediziner}!Amourette. Pièce en trois actes. Adaptée de Arthur Schnitzler@\strich\emph{Amourette. Pièce en trois actes. Adaptée de Arthur Schnitzler}|pwv}
               überreicht; aber das iſt ein Flachkopf, und er wird es kaum acceptiren. Ein anderes
                  \label{K_L02792-6v}\edtext{Manuſkript\pwindex{Schnitzler, Arthur 15.\,5.\,1862 Wien – 21.\,10.\,1931 ebd.@\textsc{Schnitzler, Arthur} (15.\,5.\,1862 Wien – 21.\,10.\,1931 ebd.), \emph{Schriftsteller, Mediziner}!Amourette. Pièce en trois actes. Adaptée de Arthur Schnitzler@\strich\emph{Amourette. Pièce en trois actes. Adaptée de Arthur Schnitzler}|pwv}}{\lemma{\textnormal{\emph{Manuskript}}}\Cendnote{\textnormal{Goldmann\pwindex{Goldmann, Paul 31.\,1.\,1865 Breslau – 25.\,9.\,1935 Wien@\textsc{Goldmann, Paul} (31.\,1.\,1865 Breslau – 25.\,9.\,1935 Wien), \emph{Schriftsteller, Journalist}|pwk} meinte ein weiteres Exemplar von
                     \emph{Amourette}\pwindex{Schnitzler, Arthur 15.\,5.\,1862 Wien – 21.\,10.\,1931 ebd.@\textsc{Schnitzler, Arthur} (15.\,5.\,1862 Wien – 21.\,10.\,1931 ebd.), \emph{Schriftsteller, Mediziner}!Amourette. Pièce en trois actes. Adaptée de Arthur Schnitzler@\strich\emph{Amourette. Pièce en trois actes. Adaptée de Arthur Schnitzler}|pwk}, der Übersetzung von \emph{Liebelei}\pwindex{Schnitzler, Arthur 15.\,5.\,1862 Wien – 21.\,10.\,1931 ebd.@\textsc{Schnitzler, Arthur} (15.\,5.\,1862 Wien – 21.\,10.\,1931 ebd.), \emph{Schriftsteller, Mediziner}!Liebelei. Schauspiel in drei Akten@\strich\emph{Liebelei. Schauspiel in drei Akten}|pwk}. Albert Carré\pwindex{Carré, Albert 22.\,6.\,1852 Straßburg – 11.\,12.\,1938 Paris@\textsc{Carré, Albert} (22.\,6.\,1852 Straßburg – 11.\,12.\,1938 Paris), \emph{Schriftsteller, Theaterleiter, Schauspieler}|pwk} lobte dieses einige Monate später (vgl. A. S.: \emph{Tagebuch}, 7. 5. 1897).}}}\label{K_L02792-6} iſt zur
               Zeit bei \textsc{Carré\pwindex{Carré, Albert 22.\,6.\,1852 Straßburg – 11.\,12.\,1938 Paris@\textsc{Carré, Albert} (22.\,6.\,1852 Straßburg – 11.\,12.\,1938 Paris), \emph{Schriftsteller, Theaterleiter, Schauspieler}|pw}}, dem Director\pwindex{Carré, Albert 22.\,6.\,1852 Straßburg – 11.\,12.\,1938 Paris@\textsc{Carré, Albert} (22.\,6.\,1852 Straßburg – 11.\,12.\,1938 Paris), \emph{Schriftsteller, Theaterleiter, Schauspieler}|pwv} des »\textsc{Vaudeville\orgindex{Théâtre du Vaudeville@Théâtre du Vaudeville|pw}}«. \textsc{Thorel\pwindex{Thorel, Jean 11.\,9.\,1859 Éragny – 20.\,8.\,1916 Enghien-les-Bains@\textsc{Thorel, Jean} (11.\,9.\,1859 Éragny – 20.\,8.\,1916 Enghien-les-Bains), \emph{Übersetzer, Dramatiker}|pw}}{ }\strikeout{\textcolor{gray}{iſt}} wird auf dieſer Seite mit allen Mitteln arbeiten. Freunde \textsc{Carrés\pwindex{Carré, Albert 22.\,6.\,1852 Straßburg – 11.\,12.\,1938 Paris@\textsc{Carré, Albert} (22.\,6.\,1852 Straßburg – 11.\,12.\,1938 Paris), \emph{Schriftsteller, Theaterleiter, Schauspieler}|pw}}{ }ſollen in Bewegung geſetzt
               werden, \textsc{Pierre Loti\pwindex{Loti, Pierre 14.\,1.\,1850 Rochefort – 10.\,6.\,1923 Hendaye@\textsc{Loti, Pierre} (14.\,1.\,1850 Rochefort – 10.\,6.\,1923 Hendaye), \emph{Schriftsteller}|pw}}, \textsc{Thorels\pwindex{Thorel, Jean 11.\,9.\,1859 Éragny – 20.\,8.\,1916 Enghien-les-Bains@\textsc{Thorel, Jean} (11.\,9.\,1859 Éragny – 20.\,8.\,1916 Enghien-les-Bains), \emph{Übersetzer, Dramatiker}|pw}} intimer Freund\pwindex{Loti, Pierre 14.\,1.\,1850 Rochefort – 10.\,6.\,1923 Hendaye@\textsc{Loti, Pierre} (14.\,1.\,1850 Rochefort – 10.\,6.\,1923 Hendaye), \emph{Schriftsteller}|pwv},{ }ſoll auch ein Wort mitreden. In den
               nächſten Wochen werden wir Bericht über das Ergebniß erhalten.\pend
           
\pstart
           Du findeſt in dieſem Briefe 1.) eine Beſprechung\pwindex{Het Tooneel. Groote Schouwburg. Minnespel. (Liebelei, van Arthur Schnitzler.)@\emph{Het Tooneel. Groote Schouwburg. Minnespel. (Liebelei, van Arthur Schnitzler.)}|pwv} der »Liebelei\pwindex{Schnitzler, Arthur 15.\,5.\,1862 Wien – 21.\,10.\,1931 ebd.@\textsc{Schnitzler, Arthur} (15.\,5.\,1862 Wien – 21.\,10.\,1931 ebd.), \emph{Schriftsteller, Mediziner}!Liebelei. Schauspiel in drei Akten@\strich\emph{Liebelei. Schauspiel in drei Akten}|pw}« {\pb}im \label{K_L02792-7v}\edtext{»\textsc{Rotterdamsche Courant\pwindex{Nieuwe Rotterdamsche Courant@\emph{Nieuwe Rotterdamsche Courant}|pw}}«}{\lemma{\textnormal{\emph{»Rotterdamsche Courant«}}}\Cendnote{\textnormal{[O. V.]: \emph{Het Tooneel. Groote Schouwburg.
                        Minnespel. (Liebelei, van Arthur Schnitzler)}\pwindex{Het Tooneel. Groote Schouwburg. Minnespel. (Liebelei, van Arthur Schnitzler.)@\emph{Het Tooneel. Groote Schouwburg. Minnespel. (Liebelei, van Arthur Schnitzler.)}|pwk}. In: \emph{Nieuwe Rotterdamsche Courant}\pwindex{Nieuwe Rotterdamsche Courant@\emph{Nieuwe Rotterdamsche Courant}|pwk}, Jg. 53, Nr. 300, 15. 12. 1896, S. 1. Schnitzler bewahrte diese Besprechung\pwindex{Het Tooneel. Groote Schouwburg. Minnespel. (Liebelei, van Arthur Schnitzler.)@\emph{Het Tooneel. Groote Schouwburg. Minnespel. (Liebelei, van Arthur Schnitzler.)}|pwkv} in seiner Zeitungsausschnittsammlung auf. Die
                  Premiere des Stücks\pwindex{Schnitzler, Arthur 15.\,5.\,1862 Wien – 21.\,10.\,1931 ebd.@\textsc{Schnitzler, Arthur} (15.\,5.\,1862 Wien – 21.\,10.\,1931 ebd.), \emph{Schriftsteller, Mediziner}!Minne-spel@\strich\emph{Minne-spel}|pwkv} (\emph{Minne-spel}\pwindex{Schnitzler, Arthur 15.\,5.\,1862 Wien – 21.\,10.\,1931 ebd.@\textsc{Schnitzler, Arthur} (15.\,5.\,1862 Wien – 21.\,10.\,1931 ebd.), \emph{Schriftsteller, Mediziner}!Minne-spel@\strich\emph{Minne-spel}|pwk}) in der Übersetzung von Frans Mijnssen\pwindex{Mijnssen, François Henri Jacques 28.\,2.\,1872 Amsterdam – 20.\,1.\,1954 Baarn@\textsc{Mijnssen, François Henri Jacques} (28.\,2.\,1872 Amsterdam – 20.\,1.\,1954 Baarn), \emph{Schriftsteller, Übersetzer, Versicherungsdirektor}|pwk} und veranstaltet von \emph{Vereenigde Rotterdamsche Tooneelisten}\orgindex{Vereenigde Rotterdamsche Tooneelisten@Vereenigde Rotterdamsche Tooneelisten|pwk} fand am
                     11. 12. 1896 in der Groote Schouwburg\oindex{Groote Schwouwburg@\textbf{Groote Schwouwburg}, \emph{Theater}|pwk} statt.}}}\label{K_L02792-7}, die mir der hieſige
                  \label{K_L02792-8v}\edtext{Correſpondent\pwindex{Wesly, Émile 1.\,11.\,1858 Maastricht – 26.\,3.\,1926@\textsc{Wesly, Émile} (1.\,11.\,1858 Maastricht – 26.\,3.\,1926), \emph{Komponist, Auslandskorrespondent}|pwuv}}{\lemma{\textnormal{\emph{Correspondent}}}\Cendnote{\textnormal{möglicherweise der Komponist und
                  Journalist Émile Wesly\pwindex{Wesly, Émile 1.\,11.\,1858 Maastricht – 26.\,3.\,1926@\textsc{Wesly, Émile} (1.\,11.\,1858 Maastricht – 26.\,3.\,1926), \emph{Komponist, Auslandskorrespondent}|pwk}}}}\label{K_L02792-8} des Blatt\orgindex{Nieuwe Rotterdamsche Courant@Nieuwe Rotterdamsche Courant|pwv}es, ein guter
                  Freund\pwindex{Wesly, Émile 1.\,11.\,1858 Maastricht – 26.\,3.\,1926@\textsc{Wesly, Émile} (1.\,11.\,1858 Maastricht – 26.\,3.\,1926), \emph{Komponist, Auslandskorrespondent}|pwuv} von
               mir, übergeben hat, um{ }ſie an Dich zu befördern. 2.) Einen Brief von \label{K_L02792-9v}\edtext{\textsc{Brandes\pwindex{Brandes, Georg 4.\,2.\,1842 Kopenhagen – 19.\,2.\,1927 ebd.@\textsc{Brandes, Georg} (4.\,2.\,1842 Kopenhagen – 19.\,2.\,1927 ebd.)|pw}} an mich 3.) Einen Brief von \textsc{Nansen\pwindex{Nansen, Peter 20.\,1.\,1861 Kopenhagen – 31.\,7.\,1918 Mariager@\textsc{Nansen, Peter} (20.\,1.\,1861 Kopenhagen – 31.\,7.\,1918 Mariager), \emph{Schriftsteller, Journalist, Verleger}|pw}}}{\lemma{\textnormal{\emph{Brandes … Nansen}}}\Cendnote{\textnormal{Beide Briefbeilagen sind nicht
                  überliefert und dürften Goldmann\pwindex{Goldmann, Paul 31.\,1.\,1865 Breslau – 25.\,9.\,1935 Wien@\textsc{Goldmann, Paul} (31.\,1.\,1865 Breslau – 25.\,9.\,1935 Wien), \emph{Schriftsteller, Journalist}|pwk}
                  zurückgesandt worden sein.}}}\label{K_L02792-9} an mich. Beide Briefe bitte ich Dich, mir \uline{zurückzuſenden}. Beide Briefe \strikeout{\textcolor{gray}{×}\-\textcolor{gray}{×}\-\textcolor{gray}{×}} hätte ich Dir{ }ſchon längſt{ }ſenden{ }ſollen, aber ich wollte{ }ſie erſt
               beantworten. Beide Briefe geben auch Dir wohl \label{K_L02792-10v}\edtext{Anlaß zu einer Antwort an die Abſender\pwindex{Nansen, Peter 20.\,1.\,1861 Kopenhagen – 31.\,7.\,1918 Mariager@\textsc{Nansen, Peter} (20.\,1.\,1861 Kopenhagen – 31.\,7.\,1918 Mariager), \emph{Schriftsteller, Journalist, Verleger}|pwv}\pwindex{Brandes, Georg 4.\,2.\,1842 Kopenhagen – 19.\,2.\,1927 ebd.@\textsc{Brandes, Georg} (4.\,2.\,1842 Kopenhagen – 19.\,2.\,1927 ebd.)|pwv}}{\lemma{\textnormal{\emph{Anlaß … Absender}}}\Cendnote{\textnormal{Der nächste Brief Schnitzlers an Brandes\pwindex{Brandes, Georg 4.\,2.\,1842 Kopenhagen – 19.\,2.\,1927 ebd.@\textsc{Brandes, Georg} (4.\,2.\,1842 Kopenhagen – 19.\,2.\,1927 ebd.)|pwk} (vom XXXX Auszeichnungsfehler: Dokument L00636 nicht gefunden) enthält keinen Hinweis, dass diese Aufforderung
                  motivierend wirkte. Der nächste Brief der überlieferten Korrespondenz Schnitzler–Nansen\pwindex{Nansen, Peter 20.\,1.\,1861 Kopenhagen – 31.\,7.\,1918 Mariager@\textsc{Nansen, Peter} (20.\,1.\,1861 Kopenhagen – 31.\,7.\,1918 Mariager), \emph{Schriftsteller, Journalist, Verleger}|pwk} ist auf den 15. 3. 1897
                  datiert.}}}\label{K_L02792-10}.\pend
           
\pstart
           Die \label{K_L02792-11v}\edtext{Kritik\pwindex{Faguet, Émile 17.\,12.\,1847 La Roche-sur-Yon – 7.\,6.\,1916 Paris@\textsc{Faguet, Émile} (17.\,12.\,1847 La Roche-sur-Yon – 7.\,6.\,1916 Paris), \emph{Kritiker}!Le livre à Paris. Francis de Pressensé: Le Cardinal Manning. – Arthur Schnitzler (traduction Gaspard Vallette): Mourir@\strich\emph{Le livre à Paris. Francis de Pressensé: Le Cardinal Manning. – Arthur Schnitzler (traduction Gaspard Vallette): Mourir}|pwv} in »\textsc{Cosmopolis\pwindex{Cosmopolis@\emph{Cosmopolis}|pw}}}{\lemma{\textnormal{\emph{Kritik in »Cosmopolis}}}\Cendnote{\textnormal{Émile Faguet\pwindex{Faguet, Émile 17.\,12.\,1847 La Roche-sur-Yon – 7.\,6.\,1916 Paris@\textsc{Faguet, Émile} (17.\,12.\,1847 La Roche-sur-Yon – 7.\,6.\,1916 Paris), \emph{Kritiker}|pwk}: \emph{Le livre à Paris. Francis de Pressensé: Le Cardinal
                        Manning. – Arthur Schnitzler (traduction Gaspard Vallette): Mourir}\pwindex{Faguet, Émile 17.\,12.\,1847 La Roche-sur-Yon – 7.\,6.\,1916 Paris@\textsc{Faguet, Émile} (17.\,12.\,1847 La Roche-sur-Yon – 7.\,6.\,1916 Paris), \emph{Kritiker}!Le livre à Paris. Francis de Pressensé: Le Cardinal Manning. – Arthur Schnitzler (traduction Gaspard Vallette): Mourir@\strich\emph{Le livre à Paris. Francis de Pressensé: Le Cardinal Manning. – Arthur Schnitzler (traduction Gaspard Vallette): Mourir}|pwk}. In:
                        \emph{Cosmopolis}\pwindex{Cosmopolis@\emph{Cosmopolis}|pwk}, Jg. 4, H. 12, Dezember 1896, S. 792–803.}}}\label{K_L02792-11}« hat mich \substVorne{}\textsuperscript{rieſig}\substDazwischen{}rieſig\substHinten{} gefreut. \textsc{Faguet\pwindex{Faguet, Émile 17.\,12.\,1847 La Roche-sur-Yon – 7.\,6.\,1916 Paris@\textsc{Faguet, Émile} (17.\,12.\,1847 La Roche-sur-Yon – 7.\,6.\,1916 Paris), \emph{Kritiker}|pw}} iſt, wie Du wohl weißt, der \strikeout{Nachf}{ }Nachfolger\pwindex{Faguet, Émile 17.\,12.\,1847 La Roche-sur-Yon – 7.\,6.\,1916 Paris@\textsc{Faguet, Émile} (17.\,12.\,1847 La Roche-sur-Yon – 7.\,6.\,1916 Paris), \emph{Kritiker}|pwv} von \textsc{Jules Lemaître\pwindex{Lemaître, Jules 27.\,4.\,1853 Vennecy – 4.\,8.\,1914 Tavers@\textsc{Lemaître, Jules} (27.\,4.\,1853 Vennecy – 4.\,8.\,1914 Tavers), \emph{Schriftsteller, Librettist}|pw}} als Theater-Kritiker im »\textsc{Journal des Débats\orgindex{Journal des débats@Journal des débats|pw}}« und einer der größten Literatur-\substVorne{}\textsuperscript{Bo}\substDazwischen{}Bonzen\substHinten{} von \textsc{Paris\oindex{Paris@\textbf{Paris}, \emph{Hauptstadt}|pw}}.\pend
           
\pstart
           {\pb}Die \label{K_L02792-12v}\edtext{Aufnahme\pwindex{demolirte Literatur@\emph{Die demolirte Literatur}|pwv} der Lausbüberei\pwindex{Kraus, Karl 28.\,4.\,1874 Jičín – 12.\,6.\,1936 Wien@\textsc{Kraus, Karl} (28.\,4.\,1874 Jičín – 12.\,6.\,1936 Wien), \emph{Schriftsteller, Publizist, Schriftsteller}!demolirte Literatur@\strich\emph{Die demolirte Literatur}|pwv} des \textsc{Kraus\pwindex{Kraus, Karl 28.\,4.\,1874 Jičín – 12.\,6.\,1936 Wien@\textsc{Kraus, Karl} (28.\,4.\,1874 Jičín – 12.\,6.\,1936 Wien), \emph{Schriftsteller, Publizist, Schriftsteller}|pw}} in die Frankf. Zeit.\pwindex{Frankfurter Zeitung@\emph{Frankfurter Zeitung}|pw}}{\lemma{\textnormal{\emph{Aufnahme … Zeit.}}}\Cendnote{\textnormal{[O. V.]: \emph{Die demolirte Literatur}\pwindex{demolirte Literatur@\emph{Die demolirte Literatur}|pwk}. In:
                        \emph{Frankfurter Zeitung}\pwindex{Frankfurter Zeitung@\emph{Frankfurter Zeitung}|pwk}, Jg. 41, Nr. 352,
                        19. 12. 1896, Abendblatt, S. 1.}}}\label{K_L02792-12}
               hat mich bitter gekränkt. Ich habe mich{ }ſofort bei meinem Onkel\pwindex{Mamroth, Fedor 21.\,2.\,1851 Breslau – 25.\,6.\,1907 Frankfurt am Main@\textsc{Mamroth, Fedor} (21.\,2.\,1851 Breslau – 25.\,6.\,1907 Frankfurt am Main), \emph{Journalist, Kritiker}|pwv} beſchwert. Dieſer iſt vollſtändig
                  \label{K_L02792-13v}\edtext{\textsc{bona fide}}{\lemma{\textnormal{\emph{bona fide}}}\Cendnote{\textnormal{lateinisch: guten Glaubens}}}\label{K_L02792-13}, hat
               keine Ahnung gehabt, um wen es{ }ſich handelt, und hat die Sache\pwindex{demolirte Literatur@\emph{Die demolirte Literatur}|pwv}, wie er mir mittheilt, nur
               aufgenommen, weil er{ }ſie »vorzüglich geſchrieben fand«. Ich vermuthe, daß meines Onkels\pwindex{Mamroth, Fedor 21.\,2.\,1851 Breslau – 25.\,6.\,1907 Frankfurt am Main@\textsc{Mamroth, Fedor} (21.\,2.\,1851 Breslau – 25.\,6.\,1907 Frankfurt am Main), \emph{Journalist, Kritiker}|pwv}{ }Frau\pwindex{Mamroth, Johanna 19.\,5.\,1872 Frankfurt am Main – 12.\,9.\,1910@\textsc{Mamroth, Johanna} (19.\,5.\,1872 Frankfurt am Main – 12.\,9.\,1910)|pwv} dahinterſteckt;{ }ſie dürfte das neue
               Genie \textsc{Kraus\pwindex{Kraus, Karl 28.\,4.\,1874 Jičín – 12.\,6.\,1936 Wien@\textsc{Kraus, Karl} (28.\,4.\,1874 Jičín – 12.\,6.\,1936 Wien), \emph{Schriftsteller, Publizist, Schriftsteller}|pw}} entdeckt haben, das{ }ſieht ihr{ }ſchon ähnlich; und mein Onkel\pwindex{Mamroth, Fedor 21.\,2.\,1851 Breslau – 25.\,6.\,1907 Frankfurt am Main@\textsc{Mamroth, Fedor} (21.\,2.\,1851 Breslau – 25.\,6.\,1907 Frankfurt am Main), \emph{Journalist, Kritiker}|pwv}{ }ſieht in dieſen \strikeout{Fälle\textcolor{gray}{m}} Fällen nur mit {\pb}ihren Augen. \strikeout{\textcolor{gray}{Auch}} Oder auch iſt die Sache\pwindex{demolirte Literatur@\emph{Die demolirte Literatur}|pwv}{ }\textsc{via Altenberg\pwindex{Altenberg, Peter 9.\,3.\,1859 Wien – 8.\,1.\,1919 ebd.@\textsc{Altenberg, Peter} (9.\,3.\,1859 Wien – 8.\,1.\,1919 ebd.), \emph{Schriftsteller}|pw}} gekommen, mit welchem die große \label{K_L02792-14v}\edtext{Kritikerin\pwindex{Mamroth, Johanna 19.\,5.\,1872 Frankfurt am Main – 12.\,9.\,1910@\textsc{Mamroth, Johanna} (19.\,5.\,1872 Frankfurt am Main – 12.\,9.\,1910)|pwv} im
                  Briefwechſel}{\lemma{\textnormal{\emph{Kritikerin im Briefwechsel}}}\Cendnote{\textnormal{Vgl. den Brief Peter Altenbergs\pwindex{Altenberg, Peter 9.\,3.\,1859 Wien – 8.\,1.\,1919 ebd.@\textsc{Altenberg, Peter} (9.\,3.\,1859 Wien – 8.\,1.\,1919 ebd.), \emph{Schriftsteller}|pwk} an Hermann Bahr\pwindex{Bahr, Hermann 19.\,7.\,1863 Linz – 15.\,1.\,1934 München@\textsc{Bahr, Hermann} (19.\,7.\,1863 Linz – 15.\,1.\,1934 München), \emph{Schriftsteller, Kritiker}|pwk}, Dezember 1898:
                     »Frau Johanna Schwarz-Mamroth\pwindex{Mamroth, Johanna 19.\,5.\,1872 Frankfurt am Main – 12.\,9.\,1910@\textsc{Mamroth, Johanna} (19.\,5.\,1872 Frankfurt am Main – 12.\,9.\,1910)|pw},
                     welche über mein 2. Buch\pwindex{Altenberg, Peter 9.\,3.\,1859 Wien – 8.\,1.\,1919 ebd.@\textsc{Altenberg, Peter} (9.\,3.\,1859 Wien – 8.\,1.\,1919 ebd.), \emph{Schriftsteller}!Ashantee@\strich\emph{Ashantee}|pwv}
                     in der Frankfurter Zeitung\pwindex{Frankfurter Zeitung@\emph{Frankfurter Zeitung}|pw} sehr lobend
                        {[}g{]}eschrieben hat, bittet mich von Florenz\oindex{Florenz@\textbf{Florenz}|pw} aus {[}{\dots}{]}« (Hermann Bahr und Peter Altenberg: \emph{Korrespondenz (1895–1913)}. Herausgegeben von Heinz
                     Lunzer und Victoria Lunzer-Talos. In: Jeanne Bennay und Alfred Pfabigan
                     (Herausgeber): \emph{Hermann Bahr – Für eine andere Moderne}.
                     Bern: \emph{Peter Lang}{ }2004, S. 249–262, hier: S. 258.) Nachgewiesen
                  ist nur eine Rezension\pwindex{Mamroth, Johanna 19.\,5.\,1872 Frankfurt am Main – 12.\,9.\,1910@\textsc{Mamroth, Johanna} (19.\,5.\,1872 Frankfurt am Main – 12.\,9.\,1910)!Wie ich es sehe«@\strich\emph{»Wie ich es sehe«}|pwkv} des
                  ersten Buch\pwindex{Altenberg, Peter 9.\,3.\,1859 Wien – 8.\,1.\,1919 ebd.@\textsc{Altenberg, Peter} (9.\,3.\,1859 Wien – 8.\,1.\,1919 ebd.), \emph{Schriftsteller}!Wie ich es sehe@\strich\emph{Wie ich es sehe}|pwkv}es, nicht des
                  zweiten (\emph{Ashantee}\pwindex{Altenberg, Peter 9.\,3.\,1859 Wien – 8.\,1.\,1919 ebd.@\textsc{Altenberg, Peter} (9.\,3.\,1859 Wien – 8.\,1.\,1919 ebd.), \emph{Schriftsteller}!Ashantee@\strich\emph{Ashantee}|pwk}): J. S.\pwindex{Mamroth, Johanna 19.\,5.\,1872 Frankfurt am Main – 12.\,9.\,1910@\textsc{Mamroth, Johanna} (19.\,5.\,1872 Frankfurt am Main – 12.\,9.\,1910)|pwkv} [ = Johanna Mamroth\pwindex{Mamroth, Johanna 19.\,5.\,1872 Frankfurt am Main – 12.\,9.\,1910@\textsc{Mamroth, Johanna} (19.\,5.\,1872 Frankfurt am Main – 12.\,9.\,1910)|pwk}]: \emph{»Wie ich es sehe«}\pwindex{Mamroth, Johanna 19.\,5.\,1872 Frankfurt am Main – 12.\,9.\,1910@\textsc{Mamroth, Johanna} (19.\,5.\,1872 Frankfurt am Main – 12.\,9.\,1910)!Wie ich es sehe«@\strich\emph{»Wie ich es sehe«}|pwk}. In: \emph{Frankfurter Zeitung}\pwindex{Frankfurter Zeitung@\emph{Frankfurter Zeitung}|pwk}, Jg. 40, Nr. 158, 8. 6. 1896,
                     Morgenblatt, S. 1–2. }}}\label{K_L02792-14}{ }ſteht,{ }ſeit{ }ſie ihn als Dichter gekrönt
               hat. Ich bin machtlos gegen{ }ſolche Dinge, kann nur hinterher wüthend{ }ſein und kann
               nicht einmal einer Wiederholung vorbeugen{\dotsfour}\pend
           
\pstart
           Mit großer Theilnahme habe ich die Skizze von Deinem Tagewerk geleſen, die Du mir
               entworfen haſt. Daß auch Du von körperlichen Leiden geplagt biſt, iſt recht garſtig.
               Soviel ich von Medicin verſtehe, will mir freilich ein \label{K_L02792-15v}\edtext{Ohren-Katarrh}{\lemma{\textnormal{\emph{Ohren-Katarrh}}}\Cendnote{\textnormal{Schnitzler litt seit
                     Herbst 1896 an Otosklerose – einer Verknöcherung des Innenohrs mit
                  zunehmender Schwerhörigkeit.}}}\label{K_L02792-15} nicht{ }ſchlimm erſcheinen. Wer weiß, ob Du ihn
               überhaupt entdeckt hätteſt, {\pb}wenn Du nicht Arzt
               wäreſt? Wie gern möchte ich ihn noch zu alle dem dazu nehmen, was ich habe! Auf einen
               Ohren-Katarrh mehr oder weniger käme es mir, weiß Gott, nicht an, wenn ich Dich \strikeout{\textcolor{gray}{von}} um dieſen Preis davon befreien könnte! Aber ich meine, das Ganze iſt doch{ }ſo
               unbedeutend, daß Du Unrecht hätteſt, Dir deßwegen auch nur eine Minute Deines Lebens
               zu verſtören.\pend
           
\pstart
           Merkwürdig iſt, daß Du trotz all’ dem Schönen, was Du haſt, Deines Lebens nicht froh
               wirſt. Ich komme um vor Sehnſucht und Reue – und Du, der Du Vieles von dem haſt, was
               ich erſehne, und Vieles noch haſt von dem, deſſen Verluſt ich bereue, – Du biſt darum
               doch {\pb}anſcheinend nicht ruhiger noch zufriedener.
               Ich werde von der Angſt gequält, daß ich werde{ }ſterben müſſen, ohne je gelebt zu
               haben, – und Du, Du lebſt und leideſt darunter, daß Du Dich nicht leben fühlſt. Was{ }ſind das für Räthſel? Deine und meine und \strikeout{a}
               wahrſcheinlich aller Menſchen Lebensthätigkeit kommt auf dieſe Weiſe darauf hinaus,
               daß wir, Jeder in{ }ſeiner Art, unſer Leben vertrödeln und verlieren. Was Dich anlangt,{ }ſo meine ich, Du grübelſt zuviel. Du haſt zuviel Raum vor Deinen Blicken. \strikeout{I\textcolor{gray}{c}h}{ }\strikeout{\textcolor{gray}{×}} Du{ }ſollteſt Dir{ }ſelbſt Grenzen aufſtellen. Die Löſung aller dieſer Probleme
                  {\pb}liegt vielleicht darin, daß man{ }ſich ein Bett im
               Gewöhnlichen graben und ruhig zwiſchen zwei Ufern hinfließen{ }ſoll. Das iſt zu
               bildlich ausgedrückt. Für Dich heißt die reale Überſetzung vielleicht: Du{ }ſollteſt
               doch heirathen. Heirathen und Kinder haben – das iſt vielleicht der einzige Weg, jene
               Übereinſtimmung mit dem dunklen Willen der Natur herzuſtellen, die{ }ſich durch inneren
               Frieden belohnt. Die Freiheit? Was hat das zu{ }ſagen? Sie iſt doch nur dazu gut, um
                  {\pb}\strikeout{\textcolor{gray}{e}} einmal Jemandem ein großes Geſchenk damit zu machen, und wir machen \strikeout{ei} eigentlich nur fortwährend Verſuche,{ }ſie dem oder
               Jenem oder vielmehr Dieſer oder Jener \strikeout{h} wegzugeben, –
               die Freiheit{\dotssix}\pend
           
\pstart
           Arbeiteſt Du nun wieder? \strikeout{Hu\textcolor{gray}{b}} Hübſch iſt die Idee, ein \label{K_L02792-16v}\edtext{Schlußſtück\pwindex{Schnitzler, Arthur 15.\,5.\,1862 Wien – 21.\,10.\,1931 ebd.@\textsc{Schnitzler, Arthur} (15.\,5.\,1862 Wien – 21.\,10.\,1931 ebd.), \emph{Schriftsteller, Mediziner}!Anatols Größenwahn@\strich\emph{Anatols Größenwahn}|pwv} zum »\textsc{Anatol\pwindex{Schnitzler, Arthur 15.\,5.\,1862 Wien – 21.\,10.\,1931 ebd.@\textsc{Schnitzler, Arthur} (15.\,5.\,1862 Wien – 21.\,10.\,1931 ebd.), \emph{Schriftsteller, Mediziner}!Anatol@\strich\emph{Anatol}|pw}}«}{\lemma{\textnormal{\emph{Schlußstück zum »Anatol«}}}\Cendnote{\textnormal{Unzufrieden mit dem letzten
                  Einakter \emph{Anatols Hochzeitsmorgen}\pwindex{Schnitzler, Arthur 15.\,5.\,1862 Wien – 21.\,10.\,1931 ebd.@\textsc{Schnitzler, Arthur} (15.\,5.\,1862 Wien – 21.\,10.\,1931 ebd.), \emph{Schriftsteller, Mediziner}!Anatols Hochzeitsmorgen@\strich\emph{Anatols Hochzeitsmorgen}|pwk}, wünschte
                  sich Mitterwurzer\pwindex{Mitterwurzer, Friedrich 16.\,10.\,1844 Dresden – 13.\,2.\,1897 Wien@\textsc{Mitterwurzer, Friedrich} (16.\,10.\,1844 Dresden – 13.\,2.\,1897 Wien), \emph{Schauspieler}|pwk} »ein anderes
                     letztes Stück ›Anatols
                        Tod\pwindex{Schnitzler, Arthur 15.\,5.\,1862 Wien – 21.\,10.\,1931 ebd.@\textsc{Schnitzler, Arthur} (15.\,5.\,1862 Wien – 21.\,10.\,1931 ebd.), \emph{Schriftsteller, Mediziner}!Anatol@\strich\emph{Anatol}|pwv}‹: Warum soll so ein Lump nicht sterben?« Schnitzler verfasste in Folge \emph{Anatols Größenwahn}\pwindex{Schnitzler, Arthur 15.\,5.\,1862 Wien – 21.\,10.\,1931 ebd.@\textsc{Schnitzler, Arthur} (15.\,5.\,1862 Wien – 21.\,10.\,1931 ebd.), \emph{Schriftsteller, Mediziner}!Anatols Größenwahn@\strich\emph{Anatols Größenwahn}|pwk}, das aber weder Mitterwurzer\pwindex{Mitterwurzer, Friedrich 16.\,10.\,1844 Dresden – 13.\,2.\,1897 Wien@\textsc{Mitterwurzer, Friedrich} (16.\,10.\,1844 Dresden – 13.\,2.\,1897 Wien), \emph{Schauspieler}|pwk} noch Schnitzler gefiel und nicht in die Buchausgabe\pwindex{Schnitzler, Arthur 15.\,5.\,1862 Wien – 21.\,10.\,1931 ebd.@\textsc{Schnitzler, Arthur} (15.\,5.\,1862 Wien – 21.\,10.\,1931 ebd.), \emph{Schriftsteller, Mediziner}!Anatol@\strich\emph{Anatol}|pwkv} aufgenommen wurde. (\emph{Anatol. Historisch-kritische Ausgabe}\pwindex{Schnitzler, Arthur 15.\,5.\,1862 Wien – 21.\,10.\,1931 ebd.@\textsc{Schnitzler, Arthur} (15.\,5.\,1862 Wien – 21.\,10.\,1931 ebd.), \emph{Schriftsteller, Mediziner}!Anatol@\strich\emph{Anatol}|pwk}. Herausgegeben 
                     von Evelyne Polt-Heinzl und Isabella Schwentner unter Mitarbeit von Gerhard
                     Hubmann. Berlin, Boston:
                        \emph{De Gruyter}{ }2012, S. 18.)}}}\label{K_L02792-16} zu{ }ſchreiben. Auch{ }ſoll \textsc{Mitterwurzer\pwindex{Mitterwurzer, Friedrich 16.\,10.\,1844 Dresden – 13.\,2.\,1897 Wien@\textsc{Mitterwurzer, Friedrich} (16.\,10.\,1844 Dresden – 13.\,2.\,1897 Wien), \emph{Schauspieler}|pw}} ruhig den \label{K_L02792-17v}\edtext{Cyclus\pwindex{Schnitzler, Arthur 15.\,5.\,1862 Wien – 21.\,10.\,1931 ebd.@\textsc{Schnitzler, Arthur} (15.\,5.\,1862 Wien – 21.\,10.\,1931 ebd.), \emph{Schriftsteller, Mediziner}!Anatol@\strich\emph{Anatol}|pwv} der kleinen Stücke}{\lemma{\textnormal{\emph{Cyclus … Stücke}}}\Cendnote{\textnormal{\emph{Anatol}\pwindex{Schnitzler, Arthur 15.\,5.\,1862 Wien – 21.\,10.\,1931 ebd.@\textsc{Schnitzler, Arthur} (15.\,5.\,1862 Wien – 21.\,10.\,1931 ebd.), \emph{Schriftsteller, Mediziner}!Anatol@\strich\emph{Anatol}|pwk}, dessen Szenen noch nie gemeinsam
                  gespielt worden waren}}}\label{K_L02792-17}{ }ſpielen. Deine ganze Eigenart{ }ſteckt doch darin, wenn{ }ſie auch klein{ }ſind. Die \label{K_L02792-18v}\edtext{Idee der
                  »Entrüſteten\pwindex{Schnitzler, Arthur 15.\,5.\,1862 Wien – 21.\,10.\,1931 ebd.@\textsc{Schnitzler, Arthur} (15.\,5.\,1862 Wien – 21.\,10.\,1931 ebd.), \emph{Schriftsteller, Mediziner}!Weg ins Freie. Roman@\strich\emph{Der Weg ins Freie. Roman}|pwv}«}{\lemma{\textnormal{\emph{Idee der
                  »Entrüsteten«}}}\Cendnote{\textnormal{Stoff, der sich über ein Jahrzehnt
                  entwickelte und der zum Roman \emph{Der Weg ins
                     Freie}\pwindex{Schnitzler, Arthur 15.\,5.\,1862 Wien – 21.\,10.\,1931 ebd.@\textsc{Schnitzler, Arthur} (15.\,5.\,1862 Wien – 21.\,10.\,1931 ebd.), \emph{Schriftsteller, Mediziner}!Weg ins Freie. Roman@\strich\emph{Der Weg ins Freie. Roman}|pwk} wurde. Die Idee (noch als Bühnenstück) notierte Schnitzler am 24. 3. 1895 ins \emph{Tagebuch}\pwindex{Schnitzler, Arthur 15.\,5.\,1862 Wien – 21.\,10.\,1931 ebd.@\textsc{Schnitzler, Arthur} (15.\,5.\,1862 Wien – 21.\,10.\,1931 ebd.), \emph{Schriftsteller, Mediziner}!Tagebuch@\strich\emph{Tagebuch}|pwk}.}}}\label{K_L02792-18} gefällt mir{ }ſehr. Es{ }ſollte {\pb}einmal \strikeout{\textcolor{gray}{v}}{ }ſchlankweg ein Luſtſpiel werden. Dazu gehört freilich Ruhe und
               Seelen-Heiterkeit; aber Du wirſt{ }ſie{ }ſchon wieder finden. Könnteſt Du nicht auf ein
               paar Wochen nach dem Süden fahren? Der \label{K_L02792-19v}\edtext{Theater-Roman\pwindex{Schnitzler, Arthur 15.\,5.\,1862 Wien – 21.\,10.\,1931 ebd.@\textsc{Schnitzler, Arthur} (15.\,5.\,1862 Wien – 21.\,10.\,1931 ebd.), \emph{Schriftsteller, Mediziner}!Roman-Fragment@\strich\emph{Roman-Fragment}|pwv}}{\lemma{\textnormal{\emph{Theater-Roman}}}\Cendnote{\textnormal{Roman\pwindex{Schnitzler, Arthur 15.\,5.\,1862 Wien – 21.\,10.\,1931 ebd.@\textsc{Schnitzler, Arthur} (15.\,5.\,1862 Wien – 21.\,10.\,1931 ebd.), \emph{Schriftsteller, Mediziner}!Roman-Fragment@\strich\emph{Roman-Fragment}|pwkv}idee, die Schnitzler bis zu seinem Tod weiterverfolgte,
                  aber erst 1967 publiziert wurde.}}}\label{K_L02792-19} muß wohl erſt \substVorne{}\textsuperscript{reifen}\substDazwischen{}reifen\substHinten{}. Laß’ den \label{K_L02792-20v}\edtext{\textsc{Bahr\pwindex{Bahr, Hermann 19.\,7.\,1863 Linz – 15.\,1.\,1934 München@\textsc{Bahr, Hermann} (19.\,7.\,1863 Linz – 15.\,1.\,1934 München), \emph{Schriftsteller, Kritiker}|pw}} nur ruhig \strikeout{\textcolor{gray}{vo}} vorangehen}{\lemma{\textnormal{\emph{Bahr … vorangehen}}}\Cendnote{\textnormal{Am
                     20. 3. 1897 erschien von Bahr\pwindex{Bahr, Hermann 19.\,7.\,1863 Linz – 15.\,1.\,1934 München@\textsc{Bahr, Hermann} (19.\,7.\,1863 Linz – 15.\,1.\,1934 München), \emph{Schriftsteller, Kritiker}|pwk} ein im Theatermilieu angesiedelter Text: \emph{Theater. Ein Wiener Roman}\pwindex{Bahr, Hermann 19.\,7.\,1863 Linz – 15.\,1.\,1934 München@\textsc{Bahr, Hermann} (19.\,7.\,1863 Linz – 15.\,1.\,1934 München), \emph{Schriftsteller, Kritiker}!Theater. Ein Wiener Roman@\strich\emph{Theater. Ein Wiener Roman}|pwk} im \emph{S. Fischer Verlag}\orgindex{S. Fischer Verlag@S. Fischer Verlag|pwk}.}}}\label{K_L02792-20}! Was hat denn das für Belang,
               was der \strikeout{\textcolor{gray}{M}}{ }Hanswurſt{ }ſchreibt? Du{ }ſcheinſt übrigens wieder gut
               mit ihm \strikeout{\textcolor{gray}{g}} zu{ }ſtehen? Die »Zeit\pwindex{Zeit. Wiener Wochenschrift@\emph{Die Zeit. Wiener Wochenschrift}|pw}« iſt{ }ſo zuckerſüß
               für Dich. Was der \label{K_L02792-21v}\edtext{\textsc{Servaes\pwindex{Servaes, Franz 17.\,6.\,1862 Köln – 14.\,7.\,1947 Wien@\textsc{Servaes, Franz} (17.\,6.\,1862 Köln – 14.\,7.\,1947 Wien), \emph{Journalist, Kritiker}|pw}} dort über Dich geſchrieben\pwindex{Servaes, Franz 17.\,6.\,1862 Köln – 14.\,7.\,1947 Wien@\textsc{Servaes, Franz} (17.\,6.\,1862 Köln – 14.\,7.\,1947 Wien), \emph{Journalist, Kritiker}!Jung Wien. Berliner Eindrücke@\strich\emph{Jung Wien. Berliner Eindrücke}|pwv}}{\lemma{\textnormal{\emph{Servaes … geschrieben}}}\Cendnote{\textnormal{Franz Servaes\pwindex{Servaes, Franz 17.\,6.\,1862 Köln – 14.\,7.\,1947 Wien@\textsc{Servaes, Franz} (17.\,6.\,1862 Köln – 14.\,7.\,1947 Wien), \emph{Journalist, Kritiker}|pwk}: \emph{Jung Wien. Berliner Eindrücke}\pwindex{Servaes, Franz 17.\,6.\,1862 Köln – 14.\,7.\,1947 Wien@\textsc{Servaes, Franz} (17.\,6.\,1862 Köln – 14.\,7.\,1947 Wien), \emph{Journalist, Kritiker}!Jung Wien. Berliner Eindrücke@\strich\emph{Jung Wien. Berliner Eindrücke}|pwk}. In: \emph{Die Zeit}\pwindex{Zeit. Wiener Wochenschrift@\emph{Die Zeit. Wiener Wochenschrift}|pwk}, Bd. 10, Nr. 118, 2. 1. 1897,
                     S. 6–8: »Der erste, der kam, war \so{Arthur Schnitzler}, und damit kam gleich ein echtes Stück vom guten, alten, nun
                     wieder jung gewordenen Wien\oindex{Wien@\textbf{Wien}, \emph{Verwaltungsgebiet}|pw}. Er ist nicht
                     gar zu schnell berühmt geworden, und das war sein Glück. So bewahrte er sich
                     umso länger seine Naivetät, die gerade bei ihm von unschätzbarem Juwelenglanz
                     ist. Er hat etwas Goeth\pwindex{Goethe, Johann Wolfgang von 28.\,8.\,1749 Frankfurt am Main – 22.\,3.\,1832 Weimar@\textsc{Goethe, Johann Wolfgang von} (28.\,8.\,1749 Frankfurt am Main – 22.\,3.\,1832 Weimar), \emph{Schriftsteller}|pw}isches in seinem
                     Naturell, etwas vom frühen Goethe\pwindex{Goethe, Johann Wolfgang von 28.\,8.\,1749 Frankfurt am Main – 22.\,3.\,1832 Weimar@\textsc{Goethe, Johann Wolfgang von} (28.\,8.\,1749 Frankfurt am Main – 22.\,3.\,1832 Weimar), \emph{Schriftsteller}|pw}, in
                     der Art, wie er im Volke wurzelt, wie er das Volk fühlt und liebt und wie er
                     doch wieder als der vornehme Herr und denkende Mensch zum Volke sich herab
                     lässt. Diese Innigkeit der Gemüthsverbindung macht seine Naivetät. Er hat so
                     schöne, schlichte Worte für seine ›süßen Mädln‹, und die süßen Mädln haben die
                     gleichen Worte für ihn. Trotzdem ist er ein neugieriger, wissbegieriger
                     Experimentator. Aber das ist der Unterschied gegen Berlin\oindex{Berlin@\textbf{Berlin}, \emph{Hauptstadt}|pw}: hier experimentiert man mit dem Verstande, Schnitzler thut es mit dem Herzen; bei uns
                     experimentiert man an sorglich zubereiteten Präparaten, Schnitzler thut es am lebenden Organismus. Und niemals
                     verwischt er beim Experimentieren den \so{Duft} des
                     Lebens. Er lässt es auf sich wirken in seiner Ganzheit, Unberührbarkeit, er
                     schlürft mit feiner prüfender Zunge seine Poesie. Ja, wenn man es recht nimmt,
                     experimentiert er eigentlich nur an sich selber. Das Draußen liegt heiter,
                     gelassen, nur wenig in Mitleidenschaft gezogen, schaukelt in seinen Bahnen
                     ruhig auf und nieder. Aber in ihm selber sitzt der Nerv, der feine,
                     empfindliche, der bei jeder Berührung zuckt, und der stets in der Wonne bereits
                     die Qual, in der Lust die Unlust spürt. Und dann wieder die Freude, solche
                     Schmerzen empfinden zu können, weil man soviel edler darum ist, soviel weiser.
                     Und die noch viel höhere Freude, den ganzen Complex von Schmerzen und
                     Seligkeiten, diesen wüsten durcheinandergeschlungenen Ballen
                     ineinanderverbissener Amphibien, den mit zarter fühlender Hand sachte
                     aufdröseln zu können, Worte dafür zu finden, malende Ausdrücke, spiegelnde
                     Verdichtungen! Die Sprache zu zwingen, dass sie den Erlebnissen unseres Inneren
                     folgt, die spröde, geizige, verschämte deutsche Sprache, die doch einen
                     Reichthum in sich birgt und ein fesselloses Jauchzen, eine Biegsamkeit und
                     herrische Uebergewalt wie – ja, das meine ich wirklich! – wie keine zweite
                     Sprache der Welt. Und Schnitzler hat vor
                     allem die Wärme und die Anmuth unserer Sprache und ihre leise, singende
                     Wehmuth.«}}}\label{K_L02792-21}, iſt {\pb}gewiß{ }ſehr{ }ſchön; aber der Unſinn{ }ſonſt in dem Artikel\pwindex{Servaes, Franz 17.\,6.\,1862 Köln – 14.\,7.\,1947 Wien@\textsc{Servaes, Franz} (17.\,6.\,1862 Köln – 14.\,7.\,1947 Wien), \emph{Journalist, Kritiker}!Jung Wien. Berliner Eindrücke@\strich\emph{Jung Wien. Berliner Eindrücke}|pwv}! Und \textsc{Bahr\pwindex{Bahr, Hermann 19.\,7.\,1863 Linz – 15.\,1.\,1934 München@\textsc{Bahr, Hermann} (19.\,7.\,1863 Linz – 15.\,1.\,1934 München), \emph{Schriftsteller, Kritiker}|pw}} als der Entbinder, der \label{K_L02792-22v}\edtext{\textsc{Georg Brandes\pwindex{Brandes, Georg 4.\,2.\,1842 Kopenhagen – 19.\,2.\,1927 ebd.@\textsc{Brandes, Georg} (4.\,2.\,1842 Kopenhagen – 19.\,2.\,1927 ebd.)|pw}} von \textsc{Wien\oindex{Wien@\textbf{Wien}, \emph{Verwaltungsgebiet}|pw}}}{\lemma{\textnormal{\emph{Georg Brandes von Wien}}}\Cendnote{\textnormal{Hermann Bahr\pwindex{Bahr, Hermann 19.\,7.\,1863 Linz – 15.\,1.\,1934 München@\textsc{Bahr, Hermann} (19.\,7.\,1863 Linz – 15.\,1.\,1934 München), \emph{Schriftsteller, Kritiker}|pwk} wird von Franz Servaes\pwindex{Servaes, Franz 17.\,6.\,1862 Köln – 14.\,7.\,1947 Wien@\textsc{Servaes, Franz} (17.\,6.\,1862 Köln – 14.\,7.\,1947 Wien), \emph{Journalist, Kritiker}|pwk} als der Erfinder von Jung-Wien\oindex{Wien@\textbf{Wien}, \emph{Verwaltungsgebiet}|pwk} geschildert, als ihr Sprachrohr. Das war eine
                  historische Ungenauigkeit, zu der Bahr\pwindex{Bahr, Hermann 19.\,7.\,1863 Linz – 15.\,1.\,1934 München@\textsc{Bahr, Hermann} (19.\,7.\,1863 Linz – 15.\,1.\,1934 München), \emph{Schriftsteller, Kritiker}|pwk}
                  seinen Beitrag geleistet hat. Eine junge Wien\oindex{Wien@\textbf{Wien}, \emph{Verwaltungsgebiet}|pwk}er
                  Literaturbewegung entwickelte sich tatsächlich noch bevor Bahr\pwindex{Bahr, Hermann 19.\,7.\,1863 Linz – 15.\,1.\,1934 München@\textsc{Bahr, Hermann} (19.\,7.\,1863 Linz – 15.\,1.\,1934 München), \emph{Schriftsteller, Kritiker}|pwk}{ }1891 aus Berlin\oindex{Berlin@\textbf{Berlin}, \emph{Hauptstadt}|pwk} nach
                     Wien\oindex{Wien@\textbf{Wien}, \emph{Verwaltungsgebiet}|pwk} übersiedelte. Bahr\pwindex{Bahr, Hermann 19.\,7.\,1863 Linz – 15.\,1.\,1934 München@\textsc{Bahr, Hermann} (19.\,7.\,1863 Linz – 15.\,1.\,1934 München), \emph{Schriftsteller, Kritiker}|pwk} war es aber, der die Literaturbewegung im
                  deutschsprachigen Feuilleton bewarb und bekannt machte – und insofern erst recht
                  wieder als ihr Erfinder gelten kann.}}}\label{K_L02792-22}! Das kränkt mich immer bitter, weil
               ich{ }ſehe, daß der Kerl\pwindex{Bahr, Hermann 19.\,7.\,1863 Linz – 15.\,1.\,1934 München@\textsc{Bahr, Hermann} (19.\,7.\,1863 Linz – 15.\,1.\,1934 München), \emph{Schriftsteller, Kritiker}|pwv}{ }\label{K_L02792-23v}\edtext{mir perſönlich etwas \strikeout{ſtie}{ }ſtiehlt}{\lemma{\textnormal{\emph{mir … stiehlt}}}\Cendnote{\textnormal{Goldmann\pwindex{Goldmann, Paul 31.\,1.\,1865 Breslau – 25.\,9.\,1935 Wien@\textsc{Goldmann, Paul} (31.\,1.\,1865 Breslau – 25.\,9.\,1935 Wien), \emph{Schriftsteller, Journalist}|pwk} konnte durch seine Tätigkeit als
                  Redakteur von \emph{An der schönen blauen Donau}\orgindex{der schönen blauen Donau@An der schönen blauen Donau|pwk} bis
                  zum Jahresende 1890 Anspruch darauf erheben, dem
                  schriftstellerischen Nachwuchs eine Publikationsmöglichkeit geschaffen zu haben.
                  Zudem könnte er sich auf eine geplante Vereinsbildung beziehen, von der am 2. 4. 1890 im \emph{Tagebuch}\pwindex{Schnitzler, Arthur 15.\,5.\,1862 Wien – 21.\,10.\,1931 ebd.@\textsc{Schnitzler, Arthur} (15.\,5.\,1862 Wien – 21.\,10.\,1931 ebd.), \emph{Schriftsteller, Mediziner}!Tagebuch@\strich\emph{Tagebuch}|pwk} berichtet wird: »Ansätze zu
                     einem lit. Verein Jung Wien\oindex{Wien@\textbf{Wien}, \emph{Verwaltungsgebiet}|pw}: Poestion\pwindex{Poestion, Josef Calasanz 7.\,6.\,1853 Bad Aussee – 5.\,5.\,1922 Wien@\textsc{Poestion, Josef Calasanz} (7.\,6.\,1853 Bad Aussee – 5.\,5.\,1922 Wien), \emph{Schriftsteller, Ministerialbeamter, Bibliotheksleiter}|pw}, Lemmermayer\pwindex{Lemmermayer, Fritz 26.\,3.\,1857 Wien – 11.\,9.\,1932 ebd.@\textsc{Lemmermayer, Fritz} (26.\,3.\,1857 Wien – 11.\,9.\,1932 ebd.), \emph{Schriftsteller}|pw}, Steiner\pwindex{Steiner, Rudolf 27.\,2.\,1861 Donji Kraljevec – 30.\,3.\,1925 Dornach@\textsc{Steiner, Rudolf} (27.\,2.\,1861 Donji Kraljevec – 30.\,3.\,1925 Dornach), \emph{Schriftsteller, Philosoph, Philosophiehistoriker}|pw}, List\pwindex{List, Guido von 5.\,10.\,1848 Wien – 21.\,5.\,1919 ebd.@\textsc{List, Guido von} (5.\,10.\,1848 Wien – 21.\,5.\,1919 ebd.), \emph{Privatgelehrte}|pw}, Wodiczka\pwindex{Wodiczka, Viktor 9.\,1.\,1851 Lichtenstein – 8.\,7.\,1898 Brunn am Gebirge@\textsc{Wodiczka, Viktor} (9.\,1.\,1851 Lichtenstein – 8.\,7.\,1898 Brunn am Gebirge), \emph{Schriftsteller, Beamter}|pw}, Ludaßy\pwindex{Gans-Ludassy, Julius von 13.\,4.\,1858 Wien – 30.\,9.\,1922 ebd.@\textsc{Gans-Ludassy, Julius von} (13.\,4.\,1858 Wien – 30.\,9.\,1922 ebd.), \emph{Schriftsteller, Journalist, Herausgeber}|pw}, Klein\pwindex{Klein, Hugo 21.\,7.\,1853 Szeged – 29.\,6.\,1915 Karlsbad@\textsc{Klein, Hugo} (21.\,7.\,1853 Szeged – 29.\,6.\,1915 Karlsbad), \emph{Schriftsteller, Journalist, Kritiker}|pw}, Breitenstein\pwindex{Breitenstein, Max 10.\,2.\,1853 Brtnice – 22.\,9.\,1926 Wien@\textsc{Breitenstein, Max} (10.\,2.\,1853 Brtnice – 22.\,9.\,1926 Wien), \emph{Journalist}|pw}, Goldmann\pwindex{Goldmann, Paul 31.\,1.\,1865 Breslau – 25.\,9.\,1935 Wien@\textsc{Goldmann, Paul} (31.\,1.\,1865 Breslau – 25.\,9.\,1935 Wien), \emph{Schriftsteller, Journalist}|pw}, ich.« Spannend ist, dass bei diesem frühen
                  Zusammenschluss mit Guido von List\pwindex{List, Guido von 5.\,10.\,1848 Wien – 21.\,5.\,1919 ebd.@\textsc{List, Guido von} (5.\,10.\,1848 Wien – 21.\,5.\,1919 ebd.), \emph{Privatgelehrte}|pwk} und Rudolf Steiner\pwindex{Steiner, Rudolf 27.\,2.\,1861 Donji Kraljevec – 30.\,3.\,1925 Dornach@\textsc{Steiner, Rudolf} (27.\,2.\,1861 Donji Kraljevec – 30.\,3.\,1925 Dornach), \emph{Schriftsteller, Philosoph, Philosophiehistoriker}|pwk} deutschnationale und
                  anthroposophische Mythenschmieder beteiligt waren.}}}\label{K_L02792-23}. Die Jungen Wien\oindex{Wien@\textbf{Wien}, \emph{Verwaltungsgebiet}|pw}er haben keines Entbinders bedurft; aber wenn{ }ſchon \strikeout{ei} Einer da war, der{ }ſie zuſammengeſucht hat,{ }ſo war
                  \uline{ich} es. Als \textsc{Bahr\pwindex{Bahr, Hermann 19.\,7.\,1863 Linz – 15.\,1.\,1934 München@\textsc{Bahr, Hermann} (19.\,7.\,1863 Linz – 15.\,1.\,1934 München), \emph{Schriftsteller, Kritiker}|pw}} nach \textsc{Wien\oindex{Wien@\textbf{Wien}, \emph{Verwaltungsgebiet}|pw}} kam, waren{ }ſchon \strikeout{Alle} Alle da; und{ }ſeine
               Wirkſamkeit hat{ }ſich darauf beſchränkt, daß er Dich beſchimpft {\pb}und verkannt hat; daß er den \textsc{Loris\pwindex{Hofmannsthal, Hugo von 1.\,2.\,1874 Wien – 15.\,7.\,1929 Rodaun@\textsc{Hofmannsthal, Hugo von} (1.\,2.\,1874 Wien – 15.\,7.\,1929 Rodaun), \emph{Schriftsteller}|pw}} mißverſtanden und verdorben hat; und daß er als neues Genie den grotesken
               Zieraffen \textsc{Andrian\pwindex{Andrian-Werburg, Leopold von 9.\,5.\,1875 Berlin – 19.\,11.\,1951 Fribourg@\textsc{Andrian-Werburg, Leopold von} (9.\,5.\,1875 Berlin – 19.\,11.\,1951 Fribourg), \emph{Schriftsteller, Diplomat}|pw}} gefunden hat. Und das läßt{ }ſich \label{K_L02792-24v}\edtext{als Begründer\pwindex{Bahr, Hermann 19.\,7.\,1863 Linz – 15.\,1.\,1934 München@\textsc{Bahr, Hermann} (19.\,7.\,1863 Linz – 15.\,1.\,1934 München), \emph{Schriftsteller, Kritiker}|pwv} der Wien\oindex{Wien@\textbf{Wien}, \emph{Verwaltungsgebiet}|pw}er Bewegung preiſen}{\lemma{\textnormal{\emph{als … preisen}}}\Cendnote{\textnormal{Siehe XXXX Auszeichnungsfehler: Dokument L02623 nicht gefunden. }}}\label{K_L02792-24}, deren gute
               Leiſtungen immer nur \uline{trotz}{ }\textsc{Bahr\pwindex{Bahr, Hermann 19.\,7.\,1863 Linz – 15.\,1.\,1934 München@\textsc{Bahr, Hermann} (19.\,7.\,1863 Linz – 15.\,1.\,1934 München), \emph{Schriftsteller, Kritiker}|pw}} entſtanden{ }ſind! {\dotsfour}\pend
           
\pstart
           Dieſer \textsc{Dr. Graf\pwindex{Graf, Max 1.\,10.\,1873 Wien – 24.\,6.\,1958 ebd.@\textsc{Graf, Max} (1.\,10.\,1873 Wien – 24.\,6.\,1958 ebd.), \emph{Kritiker}|pw}}, den mir \textsc{Richard\pwindex{Beer-Hofmann, Richard 11.\,7.\,1866 Wien – 26.\,9.\,1945 New York City@\textsc{Beer-Hofmann, Richard} (11.\,7.\,1866 Wien – 26.\,9.\,1945 New York City), \emph{Schriftsteller}|pw}} geſchickt hat, gefällt mir recht gut. Er hat eine angenehme Art, iſt aber wohl
               keine {\pb}ſtarke Perſönlichkeit und kein{ }ſehr klarer
               Kopf. Er{ }ſtreckt unſicher{ }ſeine Fühlhörner ins \strikeout{Leben}
               Leben aus. \strikeout{\textcolor{gray}{Wa}} Seine \textsc{Bahr\pwindex{Bahr, Hermann 19.\,7.\,1863 Linz – 15.\,1.\,1934 München@\textsc{Bahr, Hermann} (19.\,7.\,1863 Linz – 15.\,1.\,1934 München), \emph{Schriftsteller, Kritiker}|pw}}-Bewunderung habe ich bereits ein wenig erſchüttert; aber es iſt nicht gut
               möglich, ihm auszureden, daß \textsc{Altenberg\pwindex{Altenberg, Peter 9.\,3.\,1859 Wien – 8.\,1.\,1919 ebd.@\textsc{Altenberg, Peter} (9.\,3.\,1859 Wien – 8.\,1.\,1919 ebd.), \emph{Schriftsteller}|pw}} ein genialer Dichtergeiſt iſt. Wollen{ }ſehen, was man aus ihm machen kann.
               Einſtweilen habe ich ihm kleine Arbeiten für unſer Blatt\orgindex{Frankfurter Zeitung@Frankfurter Zeitung|pwv} verſchafft.\pend
           
\pstart
           \strikeout{Di\textcolor{gray}{e}} Die Fragen, die Du an mich{ }ſtellſt, \label{K_L02792-25v}\edtext{\textsc{\begin{otherlanguage}{french}me concernant\end{otherlanguage}}}{\lemma{\textnormal{\emph{me concernant}}}\Cendnote{\textnormal{französisch: mich betreffend}}}\label{K_L02792-25},
               beantworten{ }ſich von{ }ſelbſt durch den Eingang dieſes Briefes {\pb}(zu deſſen Fertigftellung ich drei Tage gebraucht).
               Stimmung: verzweifelt (ich werde nie dazu kommen, den tiefen Riß in meinem Leben \strikeout{a\textcolor{gray}{×}} auszufüllen); Stellung: unerfreulich; Arbeit: null; Freunde: ein paar brave
                  \label{K_L02792-26v}\edtext{Leute}{\lemma{\textnormal{\emph{Leute}}}\Cendnote{\textnormal{nicht identifiziert}}}\label{K_L02792-26} auf \textsc{Montmartre\oindex{18. arrondissement [Paris]@\textbf{18. arrondissement [Paris]}, \emph{Teil eines besiedelten Ortes}|pw}}, ehrliche und{ }ſimple Menſchen, die mich in ihrer kühlen Weiſe gern haben und –
               nicht verſtehen; Geliebte:{ }ſchwere pſychiſche (?) Impotenz{\dotsfour}\pend
           
\pstart
           Willſt Du mir einen Gefallen thun? Ich möchte gern den »\label{K_L02792-27v}\edtext{\textsc{Lorenzaccio\pwindex{Musset, Alfred de 11.\,12.\,1810 Paris – 2.\,5.\,1857 ebd.@\textsc{Musset, Alfred de} (11.\,12.\,1810 Paris – 2.\,5.\,1857 ebd.), \emph{Schriftsteller}!Lorenzaccio. Drame romantique en cinq actes@\strich\emph{Lorenzaccio. Drame romantique en cinq actes}|pw}}}{\lemma{\textnormal{\emph{Lorenzaccio}}}\Cendnote{\textnormal{Am 3. 12. 1896
                  wurde 
                  \emph{Lorenzaccio. Drame romantique en cinq actes}\pwindex{Musset, Alfred de 11.\,12.\,1810 Paris – 2.\,5.\,1857 ebd.@\textsc{Musset, Alfred de} (11.\,12.\,1810 Paris – 2.\,5.\,1857 ebd.), \emph{Schriftsteller}!Lorenzaccio. Drame romantique en cinq actes@\strich\emph{Lorenzaccio. Drame romantique en cinq actes}|pwk}  am \emph{Théâtre de la Renaissance}\orgindex{Théâtre de la Renaissance@Théâtre de la Renaissance|pwk}
                  uraufgeführt. Die Hauptrolle spielte Sarah Bernhardt\pwindex{Bernhardt, Sarah 22.\,10.\,1844 Paris – 26.\,3.\,1923 ebd.@\textsc{Bernhardt, Sarah} (22.\,10.\,1844 Paris – 26.\,3.\,1923 ebd.), \emph{Schauspielerin}|pwk}. Veröffentlicht wurde es 62 Jahre zuvor.}}}\label{K_L02792-27}« von \textsc{Musset\pwindex{Musset, Alfred de 11.\,12.\,1810 Paris – 2.\,5.\,1857 ebd.@\textsc{Musset, Alfred de} (11.\,12.\,1810 Paris – 2.\,5.\,1857 ebd.), \emph{Schriftsteller}|pw}}{ }\label{K_L02792-28v}\edtext{für die deutſche {\pb}Bühne bearbeiten}{\lemma{\textnormal{\emph{für … bearbeiten}}}\Cendnote{\textnormal{Die Idee bestand jedenfalls seit 1894 (vgl. A. S.: \emph{Tagebuch}, 8. 9. 1894 und XXXX Auszeichnungsfehler: Dokument L02614 nicht gefunden). Schnitzler fühlte bei Otto Brahm\pwindex{Brahm, Otto 5.\,2.\,1856 Hamburg – 28.\,11.\,1912 Berlin@\textsc{Brahm, Otto} (5.\,2.\,1856 Hamburg – 28.\,11.\,1912 Berlin), \emph{Theaterleiter, Regisseur}|pwk} vor, der ihm am 13. 5. 1897
                  antwortete: »Wegen einer \uline{Lorenzaccio\pwindex{Musset, Alfred de 11.\,12.\,1810 Paris – 2.\,5.\,1857 ebd.@\textsc{Musset, Alfred de} (11.\,12.\,1810 Paris – 2.\,5.\,1857 ebd.), \emph{Schriftsteller}!Lorenzaccio. Drame romantique en cinq actes@\strich\emph{Lorenzaccio. Drame romantique en cinq actes}|pw}}-Übersetzung bin ich Ihnen auch noch eine Antwort schuldig. Es ist
                     inzwischen eine bei uns eingelaufen und abgelehnt worden. Ist das die Ihres
                     Protegés? Ich glaube kaum, daß das Stück\pwindex{Musset, Alfred de 11.\,12.\,1810 Paris – 2.\,5.\,1857 ebd.@\textsc{Musset, Alfred de} (11.\,12.\,1810 Paris – 2.\,5.\,1857 ebd.), \emph{Schriftsteller}!Lorenzaccio. Drame romantique en cinq actes@\strich\emph{Lorenzaccio. Drame romantique en cinq actes}|pwv} bei uns Chancen hätte; aber wenn die Sache für
                     Ihren Unbekannten noch nicht erledigt ist – einreichen kann er ja immer, das
                     ist Menschenrecht.« (\emph{Der Briefwechsel Arthur Schnitzler – Otto Brahm}.
                     Vollständige Ausgabe. Herausgegeben, eingeleitet und erläutert von Oskar
                     Seidlin. Tübingen: \emph{Niemeyer}{ }1975, S. 33.)}}}\label{K_L02792-28}. Ich{ }ſende Dir anbei das \label{K_L02792-29v}\edtext{Feuilleton\pwindex{Goldmann, Paul 31.\,1.\,1865 Breslau – 25.\,9.\,1935 Wien@\textsc{Goldmann, Paul} (31.\,1.\,1865 Breslau – 25.\,9.\,1935 Wien), \emph{Schriftsteller, Journalist}!Lorenzaccio@\strich\emph{Lorenzaccio}|pwv}}{\lemma{\textnormal{\emph{Feuilleton}}}\Cendnote{\textnormal{Paul Goldmann\pwindex{Goldmann, Paul 31.\,1.\,1865 Breslau – 25.\,9.\,1935 Wien@\textsc{Goldmann, Paul} (31.\,1.\,1865 Breslau – 25.\,9.\,1935 Wien), \emph{Schriftsteller, Journalist}|pwk}: \emph{Lorenzaccio}\pwindex{Goldmann, Paul 31.\,1.\,1865 Breslau – 25.\,9.\,1935 Wien@\textsc{Goldmann, Paul} (31.\,1.\,1865 Breslau – 25.\,9.\,1935 Wien), \emph{Schriftsteller, Journalist}!Lorenzaccio@\strich\emph{Lorenzaccio}|pwk}. In: \emph{Frankfurter Zeitung}\pwindex{Frankfurter Zeitung@\emph{Frankfurter Zeitung}|pwk}, Jg. 41, Nr. 346, 13. 12. 1896,
                     Erstes Morgenblatt, S. 3; Nr. 347, 14. 12. 1896, Morgenblatt,
                     S. 1–2.}}}\label{K_L02792-29}, das ich darüber geſchrieben. Könnte ich vielleicht vom
                  »Burgtheater\orgindex{Burgtheater@Burgtheater|pw}« den Auftrag zu dieſer
               Bearbeitung bekommen? Könnteſt Du ein Wort mit \textsc{Burckhardt\pwindex{Burckhard, Max Eugen 14.\,7.\,1854 Korneuburg – 16.\,3.\,1912 Wien@\textsc{Burckhard, Max Eugen} (14.\,7.\,1854 Korneuburg – 16.\,3.\,1912 Wien), \emph{Schriftsteller, Rechtswissenschaftler, Theaterleiter}|pw}} oder mit \textsc{Uhl\pwindex{Uhl, Friedrich 14.\,5.\,1825 Cieszyn – 20.\,1.\,1906 Mondsee@\textsc{Uhl, Friedrich} (14.\,5.\,1825 Cieszyn – 20.\,1.\,1906 Mondsee), \emph{Journalist}|pw}} reden? In meinem Feuilleton\pwindex{Goldmann, Paul 31.\,1.\,1865 Breslau – 25.\,9.\,1935 Wien@\textsc{Goldmann, Paul} (31.\,1.\,1865 Breslau – 25.\,9.\,1935 Wien), \emph{Schriftsteller, Journalist}!Lorenzaccio@\strich\emph{Lorenzaccio}|pwv} finden{ }ſie alle nöthigen{ }ſachlichen Angaben über das Stück\pwindex{Musset, Alfred de 11.\,12.\,1810 Paris – 2.\,5.\,1857 ebd.@\textsc{Musset, Alfred de} (11.\,12.\,1810 Paris – 2.\,5.\,1857 ebd.), \emph{Schriftsteller}!Lorenzaccio. Drame romantique en cinq actes@\strich\emph{Lorenzaccio. Drame romantique en cinq actes}|pwv}. Das iſt{ }ſo eine
               phantaſtiſche Idee, die ich habe; ausführbar wird{ }ſie natürlich nicht{ }ſein; und es
               lohnt nicht der Mühe, daß Du Dir deßwegen auch nur einen überflüßigen Weg machſt{\dotsfive}\pend
           
\pstart
           {\pb}Wie gern würde ich Dich bald einmal wiederſehen\substVorne{}\textsuperscript{?}\substDazwischen{}!\substHinten{} Iſt gar keine Ausſicht, daß Du \label{K_L02792-30v}\edtext{nach \textsc{Paris\oindex{Paris@\textbf{Paris}, \emph{Hauptstadt}|pw}}}{\lemma{\textnormal{\emph{nach Paris}}}\Cendnote{\textnormal{Schnitzler und Marie Reinhard\pwindex{Reinhard, Marie 13.\,3.\,1871 Wien – 18.\,3.\,1899 ebd.@\textsc{Reinhard, Marie} (13.\,3.\,1871 Wien – 18.\,3.\,1899 ebd.), \emph{Gesangspädagogin}|pwk} kamen am 12. 4. 1897 nach Paris\oindex{Paris@\textbf{Paris}, \emph{Hauptstadt}|pwk}. Er blieb bis zum 24. 5. 1897, sie reiste einen Tag früher
                  ab.}}}\label{K_L02792-30} kommſt?\pend
           
\pstart
           Grüß’ mir den lieben \textsc{Richard\pwindex{Beer-Hofmann, Richard 11.\,7.\,1866 Wien – 26.\,9.\,1945 New York City@\textsc{Beer-Hofmann, Richard} (11.\,7.\,1866 Wien – 26.\,9.\,1945 New York City), \emph{Schriftsteller}|pw}} und auch \textsc{Leo Vanjung\pwindex{Van-Jung, Leo 15.\,10.\,1866 Odessa – 2.\,7.\,1939 Riga@\textsc{Van-Jung, Leo} (15.\,10.\,1866 Odessa – 2.\,7.\,1939 Riga), \emph{Gesangspädagoge, Mathematiker}|pw}}, wenn Du ihn \label{K_L02792-31v}\edtext{ſiehſt}{\lemma{\textnormal{\emph{siehst}}}\Cendnote{\textnormal{Das nächste Mal trafen sich Schnitzler und Leo Van-Jung\pwindex{Van-Jung, Leo 15.\,10.\,1866 Odessa – 2.\,7.\,1939 Riga@\textsc{Van-Jung, Leo} (15.\,10.\,1866 Odessa – 2.\,7.\,1939 Riga), \emph{Gesangspädagoge, Mathematiker}|pwk} vermutlich am 12. 1. 1897.}}}\label{K_L02792-31}!\pend
           
\pstart
           Allen den Deinigen wünſche ich ein glückliches neues Jahr; empfiehl’ mich
               insbeſondere Deiner Frau Mutter\pwindex{Schnitzler, Louise 8.\,7.\,1840 Kőszeg – 9.\,9.\,1911 Wien@\textsc{Schnitzler, Louise} (8.\,7.\,1840 Kőszeg – 9.\,9.\,1911 Wien)|pwv} und grüße mir recht herzlich Deinen Bruder\pwindex{Schnitzler, Julius 13.\,7.\,1865 Wien – 29.\,6.\,1939 ebd.@\textsc{Schnitzler, Julius} (13.\,7.\,1865 Wien – 29.\,6.\,1939 ebd.), \emph{Chirurg}|pwv} und Deine Schwägerin\pwindex{Schnitzler, Helene 16.\,7.\,1871 Budapest – September 1941 Atlantischer Ozean@\textsc{Schnitzler, Helene} (16.\,7.\,1871 Budapest – September 1941 Atlantischer Ozean)|pwv}.\pend
           
\pstart
           {\pb}Und{ }ſei’ Du{ }ſelbſt von Herzen gegrüßt!\pend
           
\pstart
           In Treue {\\[\baselineskip]}Dein {\\[\baselineskip]}\spacefill\mbox{Paul Goldmann.}\pend
           \leftskip=0em{}
\pstart
           \noindent{}Nicht wahr, Du{ }ſchreibſt mir bald wieder einmal?\pend
           \selectlanguage{ngerman}\endnumbering\briefempfaengerindex{Schnitzler, Arthur@\textsc{Schnitzler, Arthur}!zzzGoldmann, Paul@\emph{von Paul Goldmann}!1897-01-021@{2. [1.? 1897]}|)be}\mylabel{L02792h}  \newcommand{\dateiname}{L02792}\newcommand{\titel}{Paul Goldmann an Arthur Schnitzler, 2. [1.? 1897]}\newcommand{\editorInnen}{Martin Anton Müller und Laura Untner}%% latex-leseansicht-abspann.tex
%% Abspann für die Leseansicht.
%% Der Schalter \ifkorrekturansicht ist bereits durch den Vorspann gesetzt.

%% latex-abspann.tex
%% Gemeinsamer Abspann für Korrekturansicht und Leseansicht.
%% Setzt den Schalter \ifkorrekturansicht voraus (gesetzt in den
%% einbindenden Dateien latex-korrekturansicht-abspann.tex bzw.
%% latex-leseansicht-abspann.tex).
%% ---------------------------------------------------------------

\normalsize

% Das esempio-Environment wird nur in der Leseansicht benötigt
\ifkorrekturansicht\else
\newenvironment{esempio}[3]%
{
    \vspace{1.5ex}
    \rlap{\underline{#1}}
    \par
    \setlength{\parindent}{0cm}
    \nopagebreak
    \leftskip=#2cm
    \rightskip=#3cm
}
{
    \par
}
\fi

\doendnotes{C}
\bigskip
\vfill

\clearpage

\footnotesize

\ifkorrekturansicht
  \lohead{\textsc{register}}
\fi

% theindex-Environment neu definieren ohne reledmac
\makeatletter
\renewenvironment{theindex}{%
  \ifkorrekturansicht
    \section*{\indexname}%
  \else
    \subsubsection*{Index der erwähnten Entitäten}%
  \fi
  \setlength{\parindent}{0pt}%
  \setlength{\parskip}{0pt plus 0.3pt}%
  \let\item\@idxitem
}{%
  \ifkorrekturansicht\clearpage\fi
}
\makeatother

\IfFileExists{\jobname-pw.ind}{\input{\jobname-pw.ind}}{}

% Quellenangabe nur in der Leseansicht
\ifkorrekturansicht\else
% Fallback-Definitionen, falls die .tex-Datei \titel etc. nicht gesetzt hat
\providecommand{\titel}{}
\providecommand{\editorInnen}{}
\providecommand{\dateiname}{\jobname}

\vspace{3cm}

\vfill

\footnotesize
\textsc{Quelle}: \titel. Herausgegeben von {\editorInnen}. In: \emph{Arthur Schnitzler: Briefwechsel mit Autorinnen und Autoren}.
 Digitale Edition, https://schnitzler-briefe.acdh.oeaw.ac.at/{\dateiname}.html (Stand \today)
\fi

\end{document}


