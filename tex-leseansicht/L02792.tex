%% latex-leseansicht-vorspann.tex
%% Vorspann für die Leseansicht.
%% Lädt die gemeinsame Datei latex-vorspann.tex mit nicht gesetztem Schalter.

\newif\ifkorrekturansicht
\korrekturansichtfalse

\input{../tex-inputs/latex-vorspann}


         
         \renewcommand{\erwaehntePersonen}{Personen: Peter Altenberg, Leopold von Andrian-Werburg, André Antoine, Hermann Bahr, Richard Beer-Hofmann, Sarah Bernhardt, Otto Brahm, Georg Brandes, Max Breitenstein, Max Eugen Burckhard, Albert Carré, Émile Faguet, Julius von Gans-Ludassy, Paul Ginisty, Johann Wolfgang von Goethe, Max Graf, Hugo von Hofmannsthal, Hugo Klein, Karl Kraus, Jules Lemaître, Fritz Lemmermayer, Guido von List, Pierre Loti, Fedor Mamroth, Johanna Mamroth, François Henri Jacques Mijnssen, Friedrich Mitterwurzer, Alfred de Musset, Peter Nansen, Josef Calasanz Poestion, Marie Reinhard, Louise Schnitzler, Julius Schnitzler, Helene Schnitzler, Franz Servaes, Leopold Sonnemann, Rudolf Steiner, Jean Thorel, Friedrich Uhl, Leo Van-Jung, Émile Wesly, Viktor Wodiczka}
         \renewcommand{\erwaehnteInstitutionen}{Institutionen: An der schönen blauen Donau, Burgtheater, Frankfurter Zeitung, Journal des débats, Nieuwe Rotterdamsche Courant, Odéon, S. Fischer Verlag, Théâtre de la Renaissance, Théâtre du Vaudeville, Vereenigde Rotterdamsche Tooneelisten}
         \renewcommand{\erwaehnteOrte}{Orte: Berlin, Florenz, Groote Schwouwburg, Paris, Paris 18 Buttes-Montmartre, Wien, rue Feydeau}
         \renewcommand{\erwaehnteWerke}{Werke: Amourette. Pièce en trois actes. Adaptée de Arthur Schnitzler, Anatol, Anatols Größenwahn, Anatols Hochzeitsmorgen, Ashantee, Cosmopolis, Der Weg ins Freie. Roman, Die Zeit. Wiener Wochenschrift, Die demolirte Literatur, Die demolirte Literatur, Frankfurter Zeitung, Het Tooneel. Groote Schouwburg. Minnespel. (Liebelei, van Arthur Schnitzler.), Jung Wien. Berliner Eindrücke, Le livre à Paris. Francis de Pressensé: Le Cardinal Manning. – Arthur Schnitzler (traduction Gaspard Vallette): Mourir, Liebelei. Schauspiel in drei Akten, Lorenzaccio, Lorenzaccio. Drame romantique en cinq actes, Minne-spel, Nieuwe Rotterdamsche Courant, Tagebuch, Theater. Ein Wiener Roman, Theaterroman, Wie ich es sehe, »Wie ich es sehe«}
               \section[ Paul Goldmann an Arthur Schnitzler, 2. {[}1.? 1897{]}]{ Paul Goldmann an Arthur Schnitzler, 2. {[}1.? 1897{]}}\nopagebreak\mylabel{v}\rehead{ }\begin{ledgroupsized}[t]{13cm}\normalsize\beginnumbering \toendnotes[C]{\smallbreak\pagebreak[2]} \Standort{DLA, A:Schnitzler, HS.NZ85.1.3166.}
\physDesc{Brief, 5 Blätter, 18 Seiten
\newline{}Handschrift: blaue Tinte, deutsche Kurrent
\newline{}Schnitzler: 1) mit Bleistift das Jahr »96« vermerkt sowie die Tagesangabe des Datums unterstrichen
                                 und mit »?« kommentiert  2) mit rotem Buntstift neun Unterstreichungen}\toendnotes[C]{\smallbreak}\pstart
           \noindent{}{\pb}\textcolor{gray}{\textbf{\textbf{Frankfurter Zeitung\orgindex{Frankfurter Zeitung@Frankfurter Zeitung|pw}}}}\pend
           \pstart
           \textcolor{gray}{\textbf{(\begin{otherlanguage}{french}Gazette de Francfort\end{otherlanguage}\orgindex{Frankfurter Zeitung@Frankfurter Zeitung|pw}).}}\pend
           \pstart
           \textcolor{gray}{\textbf{\textbf{\begin{otherlanguage}{french}Fondateur M.\end{otherlanguage}{ }L. Sonnemann\pwindex{Sonnemann, Leopold 1831-10-29 – 1909-10-30@\textsc{Sonnemann, Leopold} (1831-10-29 – 1909-10-30), \emph{Journalist, Herausgeber}|pw}.}}}\pend
           \pstart
           \begin{otherlanguage}{french}\textcolor{gray}{\textbf{Journal\pwindex{?? Werk@Nicht ermittelte Verfasserinnen und Verfasser!Frankfurter Zeitung1856 – 1943@\emph{Frankfurter Zeitung} {[}1856 – 1943{]}|pwv} politique,
                        financier,}}\end{otherlanguage}\pend
           \pstart
           \begin{otherlanguage}{french}\textcolor{gray}{\textbf{commercial et littéraire.}}\end{otherlanguage}\pend
           \pstart
           \begin{otherlanguage}{french}\textcolor{gray}{\textbf{\textbf{Paraissant trois fois par jour.}}}\end{otherlanguage}\hfill \textsc{Paris\oindex{Paris@\textbf{Paris}|pw}}, \label{K_L02792-48v}\edtext{2. December}{\lemma{\textnormal{\emph{2. December}}}\Cendnote{\textnormal{Es ist davon auszugehen, dass Goldmann\pwindex{Goldmann, Paul 31.01.1865 – 25.09.1935@\textsc{Goldmann, Paul} (31.01.1865 – 25.09.1935), \emph{Schriftsteller, Journalist}|pwk} den Brief falsch datierte und
                        nicht am 2. 12. 1896, sondern am 2. 1. 1897 verfasste. Dafür spricht, dass er Schnitzler\pwindex{Schnitzler, Arthur 15.05.1862 – 21.10.1931@\textsc{Schnitzler, Arthur} (15.05.1862 – 21.10.1931), \emph{Schriftsteller, Mediziner}|pwk} eingangs ein frohes neues
                           Jahr wünscht.}}}\label{K_L02792-48h}.\pend
           \pstart
           \begin{otherlanguage}{french}\textcolor{gray}{\textbf{\textbf{Bureau à Paris\oindex{Paris@\textbf{Paris}|pw}}}}\end{otherlanguage}\pend
           \pstart
           \begin{otherlanguage}{french}\textcolor{gray}{\textbf{\textbf{24. Rue Feydeau\oindex{rue Feydeau@\textbf{rue Feydeau}|pw}.}}}\end{otherlanguage}\pend
           \pstart{}Mein lieber Freund,\pend\pstart
           Ich wünſche Dir von Herzen ein glückliches neues Jahr. Im
               alten Jahr waren die Tage, die ich mit Dir verlebt, für mich
               wohl das Beſte. Ich danke Dir \strikeout{\textcolor{gray}{×}\-\textcolor{gray}{×}\-\textcolor{gray}{×}} vielmals für alle Deine Treue und Güte{\dotsseven}\pend
           \pstart
           Sehr habe ich mich mit Deinem lieben ausführlichen Briefe gefreut. Er hätte gleich
               beantwortet werden ſollen. In jenen Tagen hatte ich keine Zeit dazu, und dann kam ein
               ſchrecklicher \strikeout{Zuſ\textcolor{gray}{am}} Zuſammenbruch: neue Erſcheinungen der gewiſſen \label{K_L02792-1v}\edtext{Krankheit}{\lemma{\textnormal{\emph{Krankheit}}}\Cendnote{\textnormal{vermutlich Syphilis}}}\label{K_L02792-1h}, Verſchlimmerung des Augenübels, eine vom Arzt
               conſtatirte unheilbare \label{K_L02792-2v}\edtext{\textsc{Mydriase}}{\lemma{\textnormal{\emph{Mydriase}}}\Cendnote{\textnormal{Pupillenerweiterung}}}\label{K_L02792-2h}, {\pb}mit Möglichkeit der Verſchlimmerung, vielleicht gar
               des Sehverluſtes. Was ſoll ich das Alles aufzählen? Seitdem habe ich nicht mehr die
               Kraft, irgend etwas zu thun. Ich gehe nirgends hin, weiſe alle Beſuche ab, bleibe bis
                  Mittag im Bett liegen und denke nur über das Sterben nach. In den
               Schmerz miſcht ſich die Reue, in die Todes- und Selbſtmord-Gedanken die Sehnſucht
               nach dem Leben, nach dem ich heißer begehre als je. Das ſind ſchlimme Tage, und Du
               begreifſt, daß \strikeout{\textcolor{gray}{×}h} Dein Brief
               unbeantwortet bleiben mußte. Nun möchte ich Dir aber trotzdem ſagen, daß ich oft an
               Dich denke, und ſo raffe ich mich auf und ſchreibe Dir doch{\dotssix}\pend
           \pstart
           {\pb}Vor einiger Zeit war ich bei \textsc{Thorel\pwindex{Thorel, Jean 1859-09-11 – 1916-08-20@\textsc{Thorel, Jean} (1859-09-11 – 1916-08-20), \emph{Übersetzer, Dramatiker}|pw}}. Durch die \label{K_L02792-3v}\edtext{Directons-Kriſis im
                  »\textsc{Odéon\orgindex{Odeon@Odéon|pw}}«}{\lemma{\textnormal{\emph{Directons-Kriſis im »Odéon«}}}\Cendnote{\textnormal{Zwischen 14. 6. 1896 und 29. 10. 1896 waren Paul Ginisty\pwindex{Ginisty, Paul 1855-04-04 – 1932-03-05@\textsc{Ginisty, Paul} (1855-04-04 – 1932-03-05), \emph{Schriftsteller, Theaterleiter}|pwk} und André Antoine\pwindex{Antoine, Andre 1858-01-31 – 1943-10-23@\textsc{Antoine, André} (1858-01-31 – 1943-10-23), \emph{Theaterleiter, Schauspieler}|pwk} die Direktoren des \emph{Odéon}\orgindex{Odeon@Odéon|pwk}-Theaters. Danach hatte Ginisty\pwindex{Ginisty, Paul 1855-04-04 – 1932-03-05@\textsc{Ginisty, Paul} (1855-04-04 – 1932-03-05), \emph{Schriftsteller, Theaterleiter}|pwk} die Funktion alleine inne.}}}\label{K_L02792-3h} und den Weggang
                  \textsc{Antoine\pwindex{Antoine, Andre 1858-01-31 – 1943-10-23@\textsc{Antoine, André} (1858-01-31 – 1943-10-23), \emph{Theaterleiter, Schauspieler}|pw}s} iſt \label{K_L02792-4v}\edtext{eine unſerer Combinationen}{\lemma{\textnormal{\emph{eine … Combinationen}}}\Cendnote{\textnormal{hier im Sinne von: Überlegungen, siehe Paul Goldmann an Arthur Schnitzler, 4. 6. [1896]}}}\label{K_L02792-4h} geſtört worden. \textsc{Thorel\pwindex{Thorel, Jean 1859-09-11 – 1916-08-20@\textsc{Thorel, Jean} (1859-09-11 – 1916-08-20), \emph{Übersetzer, Dramatiker}|pw}} hat dem übrigbleibenden Director \textsc{Ginisty\pwindex{Ginisty, Paul 1855-04-04 – 1932-03-05@\textsc{Ginisty, Paul} (1855-04-04 – 1932-03-05), \emph{Schriftsteller, Theaterleiter}|pw}} zwar das Stück\pwindex{Thorel, Jean 1859-09-11 – 1916-08-20@\textsc{Thorel, Jean} (1859-09-11 – 1916-08-20), \emph{Übersetzer, Dramatiker}!Amourette. Piece en trois actes. Adaptee de Arthur Schnitzler1897@\strich\emph{Amourette. Pièce en trois actes. Adaptée de Arthur Schnitzler} {[}Übersetzung, 1897{]}|pwv}
               überreicht; aber das iſt ein Flachkopf, und er wird es kaum acceptiren. Ein anderes
                  \label{K_L02792-84v}\edtext{Manuſkript\pwindex{Thorel, Jean 1859-09-11 – 1916-08-20@\textsc{Thorel, Jean} (1859-09-11 – 1916-08-20), \emph{Übersetzer, Dramatiker}!Amourette. Piece en trois actes. Adaptee de Arthur Schnitzler1897@\strich\emph{Amourette. Pièce en trois actes. Adaptée de Arthur Schnitzler} {[}Übersetzung, 1897{]}|pwv}}{\lemma{\textnormal{\emph{Manuſkript}}}\Cendnote{\textnormal{Goldmann\pwindex{Goldmann, Paul 31.01.1865 – 25.09.1935@\textsc{Goldmann, Paul} (31.01.1865 – 25.09.1935), \emph{Schriftsteller, Journalist}|pwk} meinte ein weiteres Exemplar von
                     \emph{Amourette}\pwindex{Thorel, Jean 1859-09-11 – 1916-08-20@\textsc{Thorel, Jean} (1859-09-11 – 1916-08-20), \emph{Übersetzer, Dramatiker}!Amourette. Piece en trois actes. Adaptee de Arthur Schnitzler1897@\strich\emph{Amourette. Pièce en trois actes. Adaptée de Arthur Schnitzler} {[}Übersetzung, 1897{]}|pwk}, der Übersetzung von \emph{Liebelei}\pwindex{Schnitzler, Arthur 15.05.1862 – 21.10.1931@\textsc{Schnitzler, Arthur} (15.05.1862 – 21.10.1931), \emph{Schriftsteller, Mediziner}!Liebelei. Schauspiel in drei Akten1895-10-09@\strich\emph{Liebelei. Schauspiel in drei Akten} {[}1895-10-09{]}|pwk}. Albert Carré\pwindex{Carre, Albert 22.06.1852 – 11.12.1938@\textsc{Carré, Albert} (22.06.1852 – 11.12.1938), \emph{Schriftsteller, Theaterleiter, Schauspieler}|pwk} lobte dieses einige Monate später (vgl. A. S.: \emph{Tagebuch}, 7. 5. 1897).}}}\label{K_L02792-84h} iſt zur
               Zeit bei \textsc{Carré\pwindex{Carre, Albert 22.06.1852 – 11.12.1938@\textsc{Carré, Albert} (22.06.1852 – 11.12.1938), \emph{Schriftsteller, Theaterleiter, Schauspieler}|pw}}, dem Director\pwindex{Carre, Albert 22.06.1852 – 11.12.1938@\textsc{Carré, Albert} (22.06.1852 – 11.12.1938), \emph{Schriftsteller, Theaterleiter, Schauspieler}|pwv} des »\textsc{Vaudeville\orgindex{Theâtre du Vaudeville@Théâtre du Vaudeville|pw}}«. \textsc{Thorel\pwindex{Thorel, Jean 1859-09-11 – 1916-08-20@\textsc{Thorel, Jean} (1859-09-11 – 1916-08-20), \emph{Übersetzer, Dramatiker}|pw}}{ }\strikeout{\textcolor{gray}{iſt}} wird auf dieſer Seite mit allen Mitteln arbeiten. Freunde \textsc{Carré\pwindex{Carre, Albert 22.06.1852 – 11.12.1938@\textsc{Carré, Albert} (22.06.1852 – 11.12.1938), \emph{Schriftsteller, Theaterleiter, Schauspieler}|pw}s} ſollen in Bewegung geſetzt
               werden, \textsc{Pierre Loti\pwindex{Loti, Pierre 14.01.1850 – 10.06.1923@\textsc{Loti, Pierre} (14.01.1850 – 10.06.1923), \emph{Schriftsteller}|pw}}, \textsc{Thorel\pwindex{Thorel, Jean 1859-09-11 – 1916-08-20@\textsc{Thorel, Jean} (1859-09-11 – 1916-08-20), \emph{Übersetzer, Dramatiker}|pw}s} intimer Freund\pwindex{Loti, Pierre 14.01.1850 – 10.06.1923@\textsc{Loti, Pierre} (14.01.1850 – 10.06.1923), \emph{Schriftsteller}|pwv}, ſoll auch ein Wort mitreden. In den
               nächſten Wochen werden wir Bericht über das Ergebniß erhalten.\pend
           \pstart
           Du findeſt in dieſem Briefe 1.) eine Beſprechung\pwindex{?? Werk@Nicht ermittelte Verfasserinnen und Verfasser!Het Tooneel. Groote Schouwburg. Minnespel. (Liebelei, van Arthur Schnitzler.)1896-12-15@\emph{Het Tooneel. Groote Schouwburg. Minnespel. (Liebelei, van Arthur Schnitzler.)} {[}1896-12-15{]}|pwv} der »Liebelei\pwindex{Schnitzler, Arthur 15.05.1862 – 21.10.1931@\textsc{Schnitzler, Arthur} (15.05.1862 – 21.10.1931), \emph{Schriftsteller, Mediziner}!Liebelei. Schauspiel in drei Akten1895-10-09@\strich\emph{Liebelei. Schauspiel in drei Akten} {[}1895-10-09{]}|pw}« {\pb}im \label{K_L02792-78v}\edtext{»\textsc{Rotterdamsche Courant\pwindex{?? Werk@Nicht ermittelte Verfasserinnen und Verfasser!Nieuwe Rotterdamsche Courant1844-01-01 – 1970@\emph{Nieuwe Rotterdamsche Courant} {[}1844-01-01 – 1970{]}|pw}}«}{\lemma{\textnormal{\emph{»Rotterdamsche Courant«}}}\Cendnote{\textnormal{[O. V.]: \emph{Het Tooneel. Groote Schouwburg.
                        Minnespel. (Liebelei, van Arthur Schnitzler.)}\pwindex{?? Werk@Nicht ermittelte Verfasserinnen und Verfasser!Het Tooneel. Groote Schouwburg. Minnespel. (Liebelei, van Arthur Schnitzler.)1896-12-15@\emph{Het Tooneel. Groote Schouwburg. Minnespel. (Liebelei, van Arthur Schnitzler.)} {[}1896-12-15{]}|pwk}. In: \emph{Nieuwe Rotterdamsche Courant}\pwindex{?? Werk@Nicht ermittelte Verfasserinnen und Verfasser!Nieuwe Rotterdamsche Courant1844-01-01 – 1970@\emph{Nieuwe Rotterdamsche Courant} {[}1844-01-01 – 1970{]}|pwk}, Jg. 53, Nr. 300, 15. 12. 1896, S. 1. Schnitzler\pwindex{Schnitzler, Arthur 15.05.1862 – 21.10.1931@\textsc{Schnitzler, Arthur} (15.05.1862 – 21.10.1931), \emph{Schriftsteller, Mediziner}|pwk} bewahrte diese Besprechung\pwindex{?? Werk@Nicht ermittelte Verfasserinnen und Verfasser!Het Tooneel. Groote Schouwburg. Minnespel. (Liebelei, van Arthur Schnitzler.)1896-12-15@\emph{Het Tooneel. Groote Schouwburg. Minnespel. (Liebelei, van Arthur Schnitzler.)} {[}1896-12-15{]}|pwkv} in seiner Zeitungsausschnittssammlung auf.
                  Die Premiere des Stück\pwindex{Schnitzler, Arthur 15.05.1862 – 21.10.1931@\textsc{Schnitzler, Arthur} (15.05.1862 – 21.10.1931), \emph{Schriftsteller, Mediziner}!Minne-spel1896-12-11@\strich\emph{Minne-spel} {[}1896-12-11{]}|pwkv}s
                     (\emph{Minne-spel}\pwindex{Schnitzler, Arthur 15.05.1862 – 21.10.1931@\textsc{Schnitzler, Arthur} (15.05.1862 – 21.10.1931), \emph{Schriftsteller, Mediziner}!Minne-spel1896-12-11@\strich\emph{Minne-spel} {[}1896-12-11{]}|pwk}) in der Übersetzung von Frans Mijnssen\pwindex{Mijnssen, François Henri Jacques 28.02.1872 – 20.01.1954@\textsc{Mijnssen, François Henri Jacques} (28.02.1872 – 20.01.1954), \emph{Schriftsteller, Übersetzer, Versicherungsdirektor}|pwk} und veranstaltet von \emph{Vereenigde Rotterdamsche Tooneelisten}\orgindex{Vereenigde Rotterdamsche Tooneelisten@Vereenigde Rotterdamsche Tooneelisten|pwk} fand am
                     11. 12. 1896 in der Groote Schouwburg\oindex{Groote Schwouwburg@\textbf{Groote Schwouwburg}|pwk} statt.}}}\label{K_L02792-78h}, die mir der hieſige
                  \label{K_L02792-67v}\edtext{Correſpondent\pwindex{Wesly, Emile 1858-11-01 – 1926-03-26@\textsc{Wesly, Émile} (1858-11-01 – 1926-03-26), \emph{Komponist, Auslandskorrespondent}|pwuv}}{\lemma{\textnormal{\emph{Correſpondent}}}\Cendnote{\textnormal{möglicherweise der Komponist und
                  Journalist Émile Wesly\pwindex{Wesly, Emile 1858-11-01 – 1926-03-26@\textsc{Wesly, Émile} (1858-11-01 – 1926-03-26), \emph{Komponist, Auslandskorrespondent}|pwk}}}}\label{K_L02792-67h} des Blatt\orgindex{Nieuwe Rotterdamsche Courant@Nieuwe Rotterdamsche Courant|pwv}es, ein guter
                  Freund\pwindex{Wesly, Emile 1858-11-01 – 1926-03-26@\textsc{Wesly, Émile} (1858-11-01 – 1926-03-26), \emph{Komponist, Auslandskorrespondent}|pwuv} von
               mir, übergeben hat, um ſie an Dich zu befördern. 2.) Einen Brief von \label{K_L02792-889v}\edtext{\textsc{Brandes\pwindex{Brandes, Georg 04.02.1842 – 19.02.1927@\textsc{Brandes, Georg} (04.02.1842 – 19.02.1927)|pw}} an mich 3.) Einen Brief von \textsc{Nansen\pwindex{Nansen, Peter 20.01.1861 – 31.07.1918@\textsc{Nansen, Peter} (20.01.1861 – 31.07.1918), \emph{Schriftsteller, Journalist, Verleger}|pw}}}{\lemma{\textnormal{\emph{Brandes … Nansen}}}\Cendnote{\textnormal{Beide Briefbeilagen sind nicht
                  überliefert und dürften Goldmann\pwindex{Goldmann, Paul 31.01.1865 – 25.09.1935@\textsc{Goldmann, Paul} (31.01.1865 – 25.09.1935), \emph{Schriftsteller, Journalist}|pwk}
                  zurückgesandt worden sein.}}}\label{K_L02792-889h} an mich. Beide Briefe bitte ich Dich, mir \uline{zurückzuſenden}. Beide Briefe \strikeout{\textcolor{gray}{×}\-\textcolor{gray}{×}\-\textcolor{gray}{×}} hätte ich Dir ſchon längſt ſenden ſollen, aber ich wollte ſie erſt
               beantworten. Beide Briefe geben auch Dir wohl \label{K_L02792-74v}\edtext{Anlaß zu einer Antwort an die Abſender\pwindex{Nansen, Peter 20.01.1861 – 31.07.1918@\textsc{Nansen, Peter} (20.01.1861 – 31.07.1918), \emph{Schriftsteller, Journalist, Verleger}|pwv}\pwindex{Brandes, Georg 04.02.1842 – 19.02.1927@\textsc{Brandes, Georg} (04.02.1842 – 19.02.1927)|pwv}}{\lemma{\textnormal{\emph{Anlaß … Abſender}}}\Cendnote{\textnormal{Der nächste Brief Schnitzler\pwindex{Schnitzler, Arthur 15.05.1862 – 21.10.1931@\textsc{Schnitzler, Arthur} (15.05.1862 – 21.10.1931), \emph{Schriftsteller, Mediziner}|pwk}s an Brandes\pwindex{Brandes, Georg 04.02.1842 – 19.02.1927@\textsc{Brandes, Georg} (04.02.1842 – 19.02.1927)|pwk} (vom 11. 1. 1897) enthält keinen Hinweis, dass diese Aufforderung
                  motivierend wirkte. Der nächste Brief der überlieferten Korrespondenz Schnitzler\pwindex{Schnitzler, Arthur 15.05.1862 – 21.10.1931@\textsc{Schnitzler, Arthur} (15.05.1862 – 21.10.1931), \emph{Schriftsteller, Mediziner}|pwk}–Nansen\pwindex{Nansen, Peter 20.01.1861 – 31.07.1918@\textsc{Nansen, Peter} (20.01.1861 – 31.07.1918), \emph{Schriftsteller, Journalist, Verleger}|pwk} ist auf den 15. 3. 1897
                  datiert.}}}\label{K_L02792-74h}.\pend
           \pstart
           Die \label{K_L02792-88v}\edtext{Kritik\pwindex{Faguet, Emile 17.12.1847 – 07.06.1916@\textsc{Faguet, Émile} (17.12.1847 – 07.06.1916), \emph{Kritiker}!Le livre à Paris. Francis de Pressense: Le Cardinal Manning. – Arthur Schnitzler (traduction Gaspard Vallette): Mourir1896-12-01@\strich\emph{Le livre à Paris. Francis de Pressensé: Le Cardinal Manning. – Arthur Schnitzler (traduction Gaspard Vallette): Mourir} {[}1896-12-01{]}|pwv} in »\textsc{Cosmopolis\pwindex{?? Werk@Nicht ermittelte Verfasserinnen und Verfasser!Cosmopolis1896 – 1898@\emph{Cosmopolis} {[}1896 – 1898{]}|pw}}}{\lemma{\textnormal{\emph{Kritik in »Cosmopolis}}}\Cendnote{\textnormal{Émile Faguet\pwindex{Faguet, Emile 17.12.1847 – 07.06.1916@\textsc{Faguet, Émile} (17.12.1847 – 07.06.1916), \emph{Kritiker}|pwk}: \emph{Le livre à Paris. Francis de Pressensé: Le Cardinal
                        Manning. – Arthur Schnitzler (traduction Gaspard Vallette): Mourir}\pwindex{Faguet, Emile 17.12.1847 – 07.06.1916@\textsc{Faguet, Émile} (17.12.1847 – 07.06.1916), \emph{Kritiker}!Le livre à Paris. Francis de Pressense: Le Cardinal Manning. – Arthur Schnitzler (traduction Gaspard Vallette): Mourir1896-12-01@\strich\emph{Le livre à Paris. Francis de Pressensé: Le Cardinal Manning. – Arthur Schnitzler (traduction Gaspard Vallette): Mourir} {[}1896-12-01{]}|pwk}. In:
                        \emph{Cosmopolis}\pwindex{?? Werk@Nicht ermittelte Verfasserinnen und Verfasser!Cosmopolis1896 – 1898@\emph{Cosmopolis} {[}1896 – 1898{]}|pwk}, Jg. 4, H. 12, Dezember 1896, S. 792–803.}}}\label{K_L02792-88h}« hat mich \substVorne{}\textsuperscript{rieſig}{\allowbreak}\substDazwischen{}rieſig\substHinten{} gefreut. \textsc{Faguet\pwindex{Faguet, Emile 17.12.1847 – 07.06.1916@\textsc{Faguet, Émile} (17.12.1847 – 07.06.1916), \emph{Kritiker}|pw}} iſt, wie Du wohl weißt, der \strikeout{Nachf}{ }Nachfolger\pwindex{Faguet, Emile 17.12.1847 – 07.06.1916@\textsc{Faguet, Émile} (17.12.1847 – 07.06.1916), \emph{Kritiker}|pwv} von \textsc{Jules Lemaître\pwindex{Lemaître, Jules 1853-04-27 – 1914-08-04@\textsc{Lemaître, Jules} (1853-04-27 – 1914-08-04), \emph{Schriftsteller, Librettist}|pw}} als Theater-Kritiker im »\textsc{Journal des Débats\orgindex{Journal des debats@Journal des débats|pw}}« und einer der größten Literatur-\substVorne{}\textsuperscript{Bo}\substDazwischen{}Bonzen\substHinten{} von \textsc{Paris\oindex{Paris@\textbf{Paris}|pw}}. \pend
           \pstart
           {\pb}Die \label{K_L02792-89v}\edtext{Aufnahme\pwindex{?? Werk@Nicht ermittelte Verfasserinnen und Verfasser!demolirte Literatur1896-12-19@\emph{Die demolirte Literatur} {[}1896-12-19{]}|pwv} der Lausbüberei\pwindex{Kraus, Karl 28.04.1874 – 12.06.1936@\textsc{Kraus, Karl} (28.04.1874 – 12.06.1936), \emph{Schriftsteller, Publizist}!demolirte Literatur15.11.1896 – 1.12.1896@\strich\emph{Die demolirte Literatur} {[}15.11.1896 – 1.12.1896{]}|pwv} des \textsc{Kraus\pwindex{Kraus, Karl 28.04.1874 – 12.06.1936@\textsc{Kraus, Karl} (28.04.1874 – 12.06.1936), \emph{Schriftsteller, Publizist}|pw}} in die Frankf. Zeit.\pwindex{?? Werk@Nicht ermittelte Verfasserinnen und Verfasser!Frankfurter Zeitung1856 – 1943@\emph{Frankfurter Zeitung} {[}1856 – 1943{]}|pw}}{\lemma{\textnormal{\emph{Aufnahme … Zeit.}}}\Cendnote{\textnormal{[O. V.]: \emph{Die demolirte Literatur}\pwindex{?? Werk@Nicht ermittelte Verfasserinnen und Verfasser!demolirte Literatur1896-12-19@\emph{Die demolirte Literatur} {[}1896-12-19{]}|pwk}. In:
                        \emph{Frankfurter Zeitung}\pwindex{?? Werk@Nicht ermittelte Verfasserinnen und Verfasser!Frankfurter Zeitung1856 – 1943@\emph{Frankfurter Zeitung} {[}1856 – 1943{]}|pwk}, Jg. 41, Nr. 352,
                        19. 12. 1896, Abendblatt, S. 1.}}}\label{K_L02792-89h}
               hat mich bitter gekränkt. Ich habe mich ſofort bei meinem Onkel\pwindex{Mamroth, Fedor 21.02.1851 – 25.06.1907@\textsc{Mamroth, Fedor} (21.02.1851 – 25.06.1907), \emph{Journalist, Kritiker}|pwv} beſchwert. Dieſer iſt vollſtändig
                  \label{K_L02792-21v}\edtext{\textsc{bona fide}}{\lemma{\textnormal{\emph{bona fide}}}\Cendnote{\textnormal{lateinisch: guten Glaubens}}}\label{K_L02792-21h}, hat
               keine Ahnung gehabt, um wen es ſich handelt, und hat die Sache\pwindex{?? Werk@Nicht ermittelte Verfasserinnen und Verfasser!demolirte Literatur1896-12-19@\emph{Die demolirte Literatur} {[}1896-12-19{]}|pwv}, wie er mir mittheilt, nur
               aufgenommen, weil er ſie »vorzüglich geſchrieben fand«. Ich vermuthe, daß meines Onkel\pwindex{Mamroth, Fedor 21.02.1851 – 25.06.1907@\textsc{Mamroth, Fedor} (21.02.1851 – 25.06.1907), \emph{Journalist, Kritiker}|pwv}s Frau\pwindex{Mamroth, Johanna 1872-05-19 – 1910-09-12@\textsc{Mamroth, Johanna} (1872-05-19 – 1910-09-12)|pwv} dahinterſteckt; ſie dürfte das neue
               Genie \textsc{Kraus\pwindex{Kraus, Karl 28.04.1874 – 12.06.1936@\textsc{Kraus, Karl} (28.04.1874 – 12.06.1936), \emph{Schriftsteller, Publizist}|pw}} entdeckt haben, das ſieht ihr ſchon ähnlich; und mein Onkel\pwindex{Mamroth, Fedor 21.02.1851 – 25.06.1907@\textsc{Mamroth, Fedor} (21.02.1851 – 25.06.1907), \emph{Journalist, Kritiker}|pwv} ſieht in dieſen \strikeout{Fälle\textcolor{gray}{m}} Fällen nur mit {\pb}ihren Augen. \strikeout{\textcolor{gray}{Auch}} Oder auch iſt die Sache\pwindex{?? Werk@Nicht ermittelte Verfasserinnen und Verfasser!demolirte Literatur1896-12-19@\emph{Die demolirte Literatur} {[}1896-12-19{]}|pwv}{ }\textsc{via Altenberg\pwindex{Altenberg, Peter 09.03.1859 – 08.01.1919@\textsc{Altenberg, Peter} (09.03.1859 – 08.01.1919), \emph{Schriftsteller}|pw}} gekommen, mit welchem die große \label{K_L02792-123v}\edtext{Kritikerin\pwindex{Mamroth, Johanna 1872-05-19 – 1910-09-12@\textsc{Mamroth, Johanna} (1872-05-19 – 1910-09-12)|pwv} im
                  Briefwechſel}{\lemma{\textnormal{\emph{Kritikerin im Briefwechſel}}}\Cendnote{\textnormal{Vgl. den Brief Peter Altenberg\pwindex{Altenberg, Peter 09.03.1859 – 08.01.1919@\textsc{Altenberg, Peter} (09.03.1859 – 08.01.1919), \emph{Schriftsteller}|pwk}s an Hermann Bahr\pwindex{Bahr, Hermann 19.07.1863 – 15.01.1934@\textsc{Bahr, Hermann} (19.07.1863 – 15.01.1934), \emph{Schriftsteller, Kritiker}|pwk}, Dezember 1898:
                     »Frau Johanna Schwarz-Mamroth\pwindex{Mamroth, Johanna 1872-05-19 – 1910-09-12@\textsc{Mamroth, Johanna} (1872-05-19 – 1910-09-12)|pw},
                     welche über mein 2. Buch\pwindex{Altenberg, Peter 09.03.1859 – 08.01.1919@\textsc{Altenberg, Peter} (09.03.1859 – 08.01.1919), \emph{Schriftsteller}!Ashantee1897@\strich\emph{Ashantee} {[}1897{]}|pwv}
                     in der Frankfurter Zeitung\pwindex{?? Werk@Nicht ermittelte Verfasserinnen und Verfasser!Frankfurter Zeitung1856 – 1943@\emph{Frankfurter Zeitung} {[}1856 – 1943{]}|pw} sehr lobend
                        {[}g{]}eschrieben hat, bittet mich von Florenz\oindex{Florenz@\textbf{Florenz}|pw} aus {[}{\dots}{]}« (Hermann Bahr und Peter Altenberg: \emph{Korrespondenz
                        von Peter Altenberg an H. B. (1895-1913)}. Hg. v. Heinz Lunzer und
                     Victoria Lunzer-Talos. In: Jeanne Bennay und Alfred Pfabigan (Hg.): \emph{Hermann Bahr – Für eine andere Moderne}.
                     Bern: \emph{Peter Lang}{ }2004, S. 249–262, hier: S. 258.) Nachgewiesen
                  ist nur eine Rezension\pwindex{Wie ich es sehe«1896-06-08@\emph{»Wie ich es sehe«} {[}1896-06-08{]}|pwkv} des
                  ersten Buch\pwindex{Altenberg, Peter 09.03.1859 – 08.01.1919@\textsc{Altenberg, Peter} (09.03.1859 – 08.01.1919), \emph{Schriftsteller}!Wie ich es sehe1896@\strich\emph{Wie ich es sehe} {[}1896{]}|pwkv}es, nicht des
                  zweiten (\emph{Ashantee}\pwindex{Altenberg, Peter 09.03.1859 – 08.01.1919@\textsc{Altenberg, Peter} (09.03.1859 – 08.01.1919), \emph{Schriftsteller}!Ashantee1897@\strich\emph{Ashantee} {[}1897{]}|pwk}): J. S.\pwindex{Mamroth, Johanna 1872-05-19 – 1910-09-12@\textsc{Mamroth, Johanna} (1872-05-19 – 1910-09-12)|pwkv} [=Johanna Mamroth\pwindex{Mamroth, Johanna 1872-05-19 – 1910-09-12@\textsc{Mamroth, Johanna} (1872-05-19 – 1910-09-12)|pwk}]: \emph{»Wie ich es sehe«}\pwindex{Wie ich es sehe«1896-06-08@\emph{»Wie ich es sehe«} {[}1896-06-08{]}|pwk}. In: \emph{Frankfurter Zeitung}\pwindex{?? Werk@Nicht ermittelte Verfasserinnen und Verfasser!Frankfurter Zeitung1856 – 1943@\emph{Frankfurter Zeitung} {[}1856 – 1943{]}|pwk}, Jg. 40, Nr. 158, 8. 6. 1896,
                     Morgenblatt, S. 1–2. }}}\label{K_L02792-123h} ſteht, ſeit ſie ihn als Dichter gekrönt
               hat. Ich bin machtlos gegen ſolche Dinge, kann nur hinterher wüthend ſein und kann
               nicht einmal einer Wiederholung vorbeugen{\dotsfour}\pend
           \pstart
           Mit großer Theilnahme habe ich die Skizze von Deinem Tagewerk geleſen, die Du mir
               entworfen haſt. Daß auch Du von körperlichen Leiden geplagt biſt, iſt recht garſtig.
               Soviel ich von Medicin verſtehe, will mir freilich ein \label{K_L02792-12v}\edtext{Ohren-Katarrh}{\lemma{\textnormal{\emph{Ohren-Katarrh}}}\Cendnote{\textnormal{Schnitzler\pwindex{Schnitzler, Arthur 15.05.1862 – 21.10.1931@\textsc{Schnitzler, Arthur} (15.05.1862 – 21.10.1931), \emph{Schriftsteller, Mediziner}|pwk} litt seit
                     Herbst 1896 an Otosklerose – einer Verknöcherung des Innenohrs mit
                  zunehmender Schwerhörigkeit.}}}\label{K_L02792-12h} nicht ſchlimm erſcheinen. Wer weiß, ob Du ihn
               überhaupt entdeckt hätteſt, {\pb}wenn Du nicht Arzt
               wäreſt? Wie gern möchte ich ihn noch zu alle dem dazu nehmen, was ich habe! Auf einen
               Ohren-Katarrh mehr oder weniger käme es mir, weiß Gott, nicht an, wenn ich Dich \strikeout{\textcolor{gray}{von}} um dieſen Preis davon befreien könnte! Aber ich meine, das Ganze iſt doch ſo
               unbedeutend, daß Du Unrecht hätteſt, Dir deßwegen auch nur eine Minute Deines Lebens
               zu verſtören.\pend
           \pstart
           Merkwürdig iſt, daß Du trotz all’ dem Schönen, was Du haſt, Deines Lebens nicht froh
               wirſt. Ich komme um vor Sehnſucht und Reue – und Du, der Du Vieles von dem haſt, was
               ich erſehne, und Vieles noch haſt von dem, deſſen Verluſt ich bereue, – Du biſt darum
               doch {\pb}anſcheinend nicht ruhiger noch zufriedener.
               Ich werde von der Angſt gequält, daß ich werde ſterben müſſen, ohne je gelebt zu
               haben, – und Du, Du lebſt und leideſt darunter, daß Du Dich nicht leben fühlſt. Was
               ſind das für Räthſel? Deine und meine und \strikeout{a}
               wahrſcheinlich aller Menſchen Lebensthätigkeit kommt auf dieſe Weiſe darauf hinaus,
               daß wir, Jeder in ſeiner Art, unſer Leben vertrödeln und verlieren. Was Dich anlangt,
               ſo meine ich, Du grübelſt zuviel. Du haſt zuviel Raum vor Deinen Blicken. \strikeout{I\textcolor{gray}{c}h}{ }\strikeout{\textcolor{gray}{×}} Du ſollteſt Dir ſelbſt Grenzen aufſtellen. Die Löſung aller dieſer Probleme
                  {\pb}liegt vielleicht darin, daß man ſich ein Bett im
               Gewöhnlichen graben und ruhig zwiſchen zwei Ufern hinfließen ſoll. Das iſt zu
               bildlich ausgedrückt. Für Dich heißt die reale Überſetzung vielleicht: Du ſollteſt
               doch heirathen. Heirathen und Kinder haben – das iſt vielleicht der einzige Weg, jene
               Übereinſtimmung mit dem dunklen Willen der Natur herzuſtellen, die ſich durch inneren
               Frieden belohnt. Die Freiheit? Was hat das zu ſagen? Sie iſt doch nur dazu gut, um
                  {\pb}\strikeout{\textcolor{gray}{e}} einmal Jemandem ein großes Geſchenk damit zu machen, und wir machen \strikeout{ei} eigentlich nur fortwährend Verſuche, ſie dem oder
               Jenem oder vielmehr Dieſer oder Jener \strikeout{h} wegzugeben, –
               die Freiheit{\dotssix}\pend
           \pstart
           Arbeiteſt Du nun wieder? \strikeout{Hu\textcolor{gray}{b}} Hübſch iſt die Idee, ein \label{K_L02792-45v}\edtext{Schlußſtück\pwindex{Schnitzler, Arthur 15.05.1862 – 21.10.1931@\textsc{Schnitzler, Arthur} (15.05.1862 – 21.10.1931), \emph{Schriftsteller, Mediziner}!Anatols Groessenwahn1931@\strich\emph{Anatols Größenwahn} {[}1931{]}|pwv} zum »\textsc{Anatol\pwindex{Schnitzler, Arthur 15.05.1862 – 21.10.1931@\textsc{Schnitzler, Arthur} (15.05.1862 – 21.10.1931), \emph{Schriftsteller, Mediziner}!Anatol1892-10-29@\strich\emph{Anatol} {[}1892-10-29{]}|pw}}«}{\lemma{\textnormal{\emph{Schlußſtück zum »Anatol«}}}\Cendnote{\textnormal{Unzufrieden mit dem letzten
                  Einakter \emph{Anatols Hochzeitsmorgen}\pwindex{Schnitzler, Arthur 15.05.1862 – 21.10.1931@\textsc{Schnitzler, Arthur} (15.05.1862 – 21.10.1931), \emph{Schriftsteller, Mediziner}!Anatols Hochzeitsmorgen01. 07. 1890@\strich\emph{Anatols Hochzeitsmorgen} {[}01. 07. 1890{]}|pwk}, wünschte
                  sich Mitterwurzer\pwindex{Mitterwurzer, Friedrich 16.10.1844 – 13.02.1897@\textsc{Mitterwurzer, Friedrich} (16.10.1844 – 13.02.1897), \emph{Schauspieler}|pwk} »ein anderes
                     letztes Stück ›Anatol\pwindex{Schnitzler, Arthur 15.05.1862 – 21.10.1931@\textsc{Schnitzler, Arthur} (15.05.1862 – 21.10.1931), \emph{Schriftsteller, Mediziner}!Anatol1892-10-29@\strich\emph{Anatol} {[}1892-10-29{]}|pw}s Tod‹: Warum soll so
                     ein Lump nicht sterben?« Schnitzler\pwindex{Schnitzler, Arthur 15.05.1862 – 21.10.1931@\textsc{Schnitzler, Arthur} (15.05.1862 – 21.10.1931), \emph{Schriftsteller, Mediziner}|pwk} verfasste in Folge \emph{Anatols
                     Größenwahn}\pwindex{Schnitzler, Arthur 15.05.1862 – 21.10.1931@\textsc{Schnitzler, Arthur} (15.05.1862 – 21.10.1931), \emph{Schriftsteller, Mediziner}!Anatols Groessenwahn1931@\strich\emph{Anatols Größenwahn} {[}1931{]}|pwk}, das aber weder Mitterwurzer\pwindex{Mitterwurzer, Friedrich 16.10.1844 – 13.02.1897@\textsc{Mitterwurzer, Friedrich} (16.10.1844 – 13.02.1897), \emph{Schauspieler}|pwk} noch Schnitzler\pwindex{Schnitzler, Arthur 15.05.1862 – 21.10.1931@\textsc{Schnitzler, Arthur} (15.05.1862 – 21.10.1931), \emph{Schriftsteller, Mediziner}|pwk} gefiel
                  und nicht in die Buchausgabe\pwindex{Schnitzler, Arthur 15.05.1862 – 21.10.1931@\textsc{Schnitzler, Arthur} (15.05.1862 – 21.10.1931), \emph{Schriftsteller, Mediziner}!Anatol1892-10-29@\strich\emph{Anatol} {[}1892-10-29{]}|pwkv} aufgenommen wurde. (\emph{Anatol. Historisch-kritische Ausgabe}\pwindex{Schnitzler, Arthur 15.05.1862 – 21.10.1931@\textsc{Schnitzler, Arthur} (15.05.1862 – 21.10.1931), \emph{Schriftsteller, Mediziner}!Anatol1892-10-29@\strich\emph{Anatol} {[}1892-10-29{]}|pwk}
                     18)}}}\label{K_L02792-45h} zu ſchreiben. Auch ſoll \textsc{Mitterwurzer\pwindex{Mitterwurzer, Friedrich 16.10.1844 – 13.02.1897@\textsc{Mitterwurzer, Friedrich} (16.10.1844 – 13.02.1897), \emph{Schauspieler}|pw}} ruhig den \label{K_L02792-878v}\edtext{Cyclus\pwindex{Schnitzler, Arthur 15.05.1862 – 21.10.1931@\textsc{Schnitzler, Arthur} (15.05.1862 – 21.10.1931), \emph{Schriftsteller, Mediziner}!Anatol1892-10-29@\strich\emph{Anatol} {[}1892-10-29{]}|pwv} der kleinen Stücke}{\lemma{\textnormal{\emph{Cyclus … Stücke}}}\Cendnote{\textnormal{\emph{Anatol}\pwindex{Schnitzler, Arthur 15.05.1862 – 21.10.1931@\textsc{Schnitzler, Arthur} (15.05.1862 – 21.10.1931), \emph{Schriftsteller, Mediziner}!Anatol1892-10-29@\strich\emph{Anatol} {[}1892-10-29{]}|pwk}, dessen Szenen noch nie gemeinsam
                  gespielt worden waren}}}\label{K_L02792-878h} ſpielen. Deine ganze Eigenart ſteckt doch darin, wenn
               ſie auch klein ſind. Die \label{K_L02792-56v}\edtext{Idee der
                  »Entrüſteten\pwindex{Schnitzler, Arthur 15.05.1862 – 21.10.1931@\textsc{Schnitzler, Arthur} (15.05.1862 – 21.10.1931), \emph{Schriftsteller, Mediziner}!Weg ins Freie. Roman1.1.1908 – 1.6.1908@\strich\emph{Der Weg ins Freie. Roman} {[}1.1.1908 – 1.6.1908{]}|pwv}«}{\lemma{\textnormal{\emph{Idee der
                  »Entrüſteten«}}}\Cendnote{\textnormal{Stoff, der sich über ein Jahrzehnt
                  entwickelte und der zum Roman \emph{Der Weg ins
                     Freie}\pwindex{Schnitzler, Arthur 15.05.1862 – 21.10.1931@\textsc{Schnitzler, Arthur} (15.05.1862 – 21.10.1931), \emph{Schriftsteller, Mediziner}!Weg ins Freie. Roman1.1.1908 – 1.6.1908@\strich\emph{Der Weg ins Freie. Roman} {[}1.1.1908 – 1.6.1908{]}|pwk} wurde. Die Idee (noch als Bühnenstück) notierte Schnitzler\pwindex{Schnitzler, Arthur 15.05.1862 – 21.10.1931@\textsc{Schnitzler, Arthur} (15.05.1862 – 21.10.1931), \emph{Schriftsteller, Mediziner}|pwk} am 24. 3. 1895 im \emph{Tagebuch}\pwindex{Schnitzler, Arthur 15.05.1862 – 21.10.1931@\textsc{Schnitzler, Arthur} (15.05.1862 – 21.10.1931), \emph{Schriftsteller, Mediziner}!Tagebuch1981 – 2000@\strich\emph{Tagebuch} {[}1981 – 2000{]}|pwk}.}}}\label{K_L02792-56h} gefällt mir ſehr. Es ſollte {\pb}einmal \strikeout{\textcolor{gray}{v}} ſchlankweg ein Luſtſpiel werden. Dazu gehört freilich Ruhe und
               Seelen-Heiterkeit; aber Du wirſt ſie ſchon wieder finden. Könnteſt Du nicht auf ein
               paar Wochen nach dem Süden fahren? Der \label{K_L02792-987v}\edtext{Theater-Roman\pwindex{Schnitzler, Arthur 15.05.1862 – 21.10.1931@\textsc{Schnitzler, Arthur} (15.05.1862 – 21.10.1931), \emph{Schriftsteller, Mediziner}!Theaterroman1967@\strich\emph{Theaterroman} {[}1967{]}|pwv}}{\lemma{\textnormal{\emph{Theater-Roman}}}\Cendnote{\textnormal{Roman\pwindex{Schnitzler, Arthur 15.05.1862 – 21.10.1931@\textsc{Schnitzler, Arthur} (15.05.1862 – 21.10.1931), \emph{Schriftsteller, Mediziner}!Theaterroman1967@\strich\emph{Theaterroman} {[}1967{]}|pwkv}idee, die Schnitzler\pwindex{Schnitzler, Arthur 15.05.1862 – 21.10.1931@\textsc{Schnitzler, Arthur} (15.05.1862 – 21.10.1931), \emph{Schriftsteller, Mediziner}|pwk} bis zu seinem Tod weiterverfolgte,
                  aber erst 1967 publiziert wurde.}}}\label{K_L02792-987h} muß wohl erſt \substVorne{}\textsuperscript{reifen}{\allowbreak}\substDazwischen{}reifen\substHinten{}. Laß’ den \label{K_L02792-8976v}\edtext{\textsc{Bahr\pwindex{Bahr, Hermann 19.07.1863 – 15.01.1934@\textsc{Bahr, Hermann} (19.07.1863 – 15.01.1934), \emph{Schriftsteller, Kritiker}|pw}} nur ruhig \strikeout{\textcolor{gray}{vo}} vorangehen}{\lemma{\textnormal{\emph{Bahr … vorangehen}}}\Cendnote{\textnormal{Am
                     20. 3. 1897 erschien von Bahr\pwindex{Bahr, Hermann 19.07.1863 – 15.01.1934@\textsc{Bahr, Hermann} (19.07.1863 – 15.01.1934), \emph{Schriftsteller, Kritiker}|pwk} ein im Theatermilieu angesiedelter Text: \emph{Theater. Ein Wiener Roman}\pwindex{Bahr, Hermann 19.07.1863 – 15.01.1934@\textsc{Bahr, Hermann} (19.07.1863 – 15.01.1934), \emph{Schriftsteller, Kritiker}!Theater. Ein Wiener Roman1897-03-20@\strich\emph{Theater. Ein Wiener Roman} {[}1897-03-20{]}|pwk} im \emph{S. Fischer-Verlag}\orgindex{S. Fischer Verlag@S. Fischer Verlag|pwk}.}}}\label{K_L02792-8976h}! Was hat denn das für Belang,
               was der \strikeout{\textcolor{gray}{M}}{ }Hanswurſt ſchreibt? Du ſcheinſt übrigens wieder gut
               mit ihm \strikeout{\textcolor{gray}{g}} zu ſtehen? Die »Zeit\pwindex{Zeit. Wiener Wochenschrift1894 – 1904@\emph{Die Zeit. Wiener Wochenschrift} {[}1894 – 1904{]}|pw}« iſt ſo zuckerſüß
               für Dich. Was der \label{K_L02792-24v}\edtext{\textsc{Servaes\pwindex{Servaes, Franz 17.06.1862 – 14.07.1947@\textsc{Servaes, Franz} (17.06.1862 – 14.07.1947), \emph{Journalist, Kritiker}|pw}} dort über Dich geſchrieben\pwindex{Servaes, Franz 17.06.1862 – 14.07.1947@\textsc{Servaes, Franz} (17.06.1862 – 14.07.1947), \emph{Journalist, Kritiker}!Jung Wien. Berliner Eindruecke1897-01-02@\strich\emph{Jung Wien. Berliner Eindrücke} {[}1897-01-02{]}|pwv}}{\lemma{\textnormal{\emph{Servaes … geſchrieben}}}\Cendnote{\textnormal{Franz Servaes\pwindex{Servaes, Franz 17.06.1862 – 14.07.1947@\textsc{Servaes, Franz} (17.06.1862 – 14.07.1947), \emph{Journalist, Kritiker}|pwk}: \emph{Jung Wien. Berliner Eindrücke}\pwindex{Servaes, Franz 17.06.1862 – 14.07.1947@\textsc{Servaes, Franz} (17.06.1862 – 14.07.1947), \emph{Journalist, Kritiker}!Jung Wien. Berliner Eindruecke1897-01-02@\strich\emph{Jung Wien. Berliner Eindrücke} {[}1897-01-02{]}|pwk}. In: \emph{Die Zeit}\pwindex{Zeit. Wiener Wochenschrift1894 – 1904@\emph{Die Zeit. Wiener Wochenschrift} {[}1894 – 1904{]}|pwk}, Bd. 10, Nr. 118, 2. 1. 1897,
                     S. 6–8: »Der erste, der kam, war \so{Arthur Schnitzler}, und damit kam gleich ein echtes Stück vom guten, alten, nun
                     wieder jung gewordenen Wien\oindex{Wien@\textbf{Wien}|pw}. Er ist nicht
                     gar zu schnell berühmt geworden, und das war sein Glück. So bewahrte er sich
                     umso länger seine Naivetät, die gerade bei ihm von unschätzbarem Juwelenglanz
                     ist. Er hat etwas Goeth\pwindex{Goethe, Johann Wolfgang von 1749-08-28 – 1832-03-22@\textsc{Goethe, Johann Wolfgang von} (1749-08-28 – 1832-03-22), \emph{Schriftsteller}|pw}isches in seinem
                     Naturell, etwas vom frühen Goethe\pwindex{Goethe, Johann Wolfgang von 1749-08-28 – 1832-03-22@\textsc{Goethe, Johann Wolfgang von} (1749-08-28 – 1832-03-22), \emph{Schriftsteller}|pw}, in
                     der Art, wie er im Volke wurzelt, wie er das Volk fühlt und liebt und wie er
                     doch wieder als der vornehme Herr und denkende Mensch zum Volke sich herab
                     lässt. Diese Innigkeit der Gemüthsverbindung macht seine Naivetät. Er hat so
                     schöne, schlichte Worte für seine ›süßen Mädln‹, und die süßen Mädln haben die
                     gleichen Worte für ihn. Trotzdem ist er ein neugieriger, wissbegieriger
                     Experimentator. Aber das ist der Unterschied gegen Berlin\oindex{Berlin@\textbf{Berlin}|pw}: hier experimentiert man mit dem Verstande, Schnitzler\pwindex{Schnitzler, Arthur 15.05.1862 – 21.10.1931@\textsc{Schnitzler, Arthur} (15.05.1862 – 21.10.1931), \emph{Schriftsteller, Mediziner}|pw} thut es mit dem Herzen; bei uns
                     experimentiert man an sorglich zubereiteten Präparaten, Schnitzler\pwindex{Schnitzler, Arthur 15.05.1862 – 21.10.1931@\textsc{Schnitzler, Arthur} (15.05.1862 – 21.10.1931), \emph{Schriftsteller, Mediziner}|pw} thut es am lebenden Organismus. Und niemals
                     verwischt er beim Experimentieren den \so{Duft} des
                     Lebens. Er lässt es auf sich wirken in seiner Ganzheit, Unberührbarkeit, er
                     schlürft mit feiner prüfender Zunge seine Poesie. Ja, wenn man es recht nimmt,
                     experimentiert er eigentlich nur an sich selber. Das Draußen liegt heiter,
                     gelassen, nur wenig in Mitleidenschaft gezogen, schaukelt in seinen Bahnen
                     ruhig auf und nieder. Aber in ihm selber sitzt der Nerv, der feine,
                     empfindliche, der bei jeder Berührung zuckt, und der stets in der Wonne bereits
                     die Qual, in der Lust die Unlust spürt. Und dann wieder die Freude, solche
                     Schmerzen empfinden zu können, weil man soviel edler darum ist, soviel weiser.
                     Und die noch viel höhere Freude, den ganzen Complex von Schmerzen und
                     Seligkeiten, diesen wüsten durcheinandergeschlungenen Ballen
                     ineinanderverbissener Amphibien, den mit zarter fühlender Hand sachte
                     aufdröseln zu können, Worte dafür zu finden, malende Ausdrücke, spiegelnde
                     Verdichtungen! Die Sprache zu zwingen, dass sie den Erlebnissen unseres Inneren
                     folgt, die spröde, geizige, verschämte deutsche Sprache, die doch einen
                     Reichthum in sich birgt und ein fesselloses Jauchzen, eine Biegsamkeit und
                     herrische Uebergewalt wie – ja, das meine ich wirklich! – wie keine zweite
                     Sprache der Welt. Und Schnitzler\pwindex{Schnitzler, Arthur 15.05.1862 – 21.10.1931@\textsc{Schnitzler, Arthur} (15.05.1862 – 21.10.1931), \emph{Schriftsteller, Mediziner}|pw} hat vor
                     allem die Wärme und die Anmuth unserer Sprache und ihre leise, singende
                     Wehmuth.«}}}\label{K_L02792-24h}, iſt {\pb}gewiß ſehr
               ſchön; aber der Unſinn ſonſt in dem Artikel\pwindex{Servaes, Franz 17.06.1862 – 14.07.1947@\textsc{Servaes, Franz} (17.06.1862 – 14.07.1947), \emph{Journalist, Kritiker}!Jung Wien. Berliner Eindruecke1897-01-02@\strich\emph{Jung Wien. Berliner Eindrücke} {[}1897-01-02{]}|pwv}! Und \textsc{Bahr\pwindex{Bahr, Hermann 19.07.1863 – 15.01.1934@\textsc{Bahr, Hermann} (19.07.1863 – 15.01.1934), \emph{Schriftsteller, Kritiker}|pw}} als der Entbinder, der \label{K_L02792-87v}\edtext{\textsc{Georg Brandes\pwindex{Brandes, Georg 04.02.1842 – 19.02.1927@\textsc{Brandes, Georg} (04.02.1842 – 19.02.1927)|pw}} von \textsc{Wien\oindex{Wien@\textbf{Wien}|pw}}}{\lemma{\textnormal{\emph{Georg Brandes von Wien}}}\Cendnote{\textnormal{Hermann Bahr\pwindex{Bahr, Hermann 19.07.1863 – 15.01.1934@\textsc{Bahr, Hermann} (19.07.1863 – 15.01.1934), \emph{Schriftsteller, Kritiker}|pwk} wird von Franz Servaes\pwindex{Servaes, Franz 17.06.1862 – 14.07.1947@\textsc{Servaes, Franz} (17.06.1862 – 14.07.1947), \emph{Journalist, Kritiker}|pwk} als der Erfinder von Jung-Wien\oindex{Wien@\textbf{Wien}|pwk} geschildert, als ihr Sprachrohr. Das war eine
                  historische Ungenauigkeit, zu der Bahr\pwindex{Bahr, Hermann 19.07.1863 – 15.01.1934@\textsc{Bahr, Hermann} (19.07.1863 – 15.01.1934), \emph{Schriftsteller, Kritiker}|pwk}
                  seinen Beitrag geleistet hat. Eine junge Wien\oindex{Wien@\textbf{Wien}|pwk}er
                  Literaturbewegung entwickelte sich tatsächlich noch bevor Bahr\pwindex{Bahr, Hermann 19.07.1863 – 15.01.1934@\textsc{Bahr, Hermann} (19.07.1863 – 15.01.1934), \emph{Schriftsteller, Kritiker}|pwk}{ }1891 aus Berlin\oindex{Berlin@\textbf{Berlin}|pwk} nach
                     Wien\oindex{Wien@\textbf{Wien}|pwk} übersiedelte. Bahr\pwindex{Bahr, Hermann 19.07.1863 – 15.01.1934@\textsc{Bahr, Hermann} (19.07.1863 – 15.01.1934), \emph{Schriftsteller, Kritiker}|pwk} war es aber, der die Literaturbewegung im
                  deutschsprachigen Feuilleton bewarb und bekannt machte – und insofern erst recht
                  wieder als ihr Erfinder gelten kann.}}}\label{K_L02792-87h}! Das kränkt mich immer bitter, weil
               ich ſehe, daß der Kerl\pwindex{Bahr, Hermann 19.07.1863 – 15.01.1934@\textsc{Bahr, Hermann} (19.07.1863 – 15.01.1934), \emph{Schriftsteller, Kritiker}|pwv}{ }\label{K_L02792-98765v}\edtext{mir perſönlich etwas \strikeout{ſtie} ſtiehlt}{\lemma{\textnormal{\emph{mir … ſtiehlt}}}\Cendnote{\textnormal{Goldmann\pwindex{Goldmann, Paul 31.01.1865 – 25.09.1935@\textsc{Goldmann, Paul} (31.01.1865 – 25.09.1935), \emph{Schriftsteller, Journalist}|pwk} konnte durch seine Tätigkeit als
                  Redakteur von \emph{An der schönen blauen Donau}\orgindex{der schoenen blauen Donau@An der schönen blauen Donau|pwk} bis
                  zum Jahresende 1890 Anspruch darauf erheben, dem
                  schriftstellerischen Nachwuchs eine Publikationsmöglichkeit geschaffen zu haben.
                  Zudem könnte er sich auf eine geplante Vereinsbildung beziehen, von der am 2. 4. 1890 im \emph{Tagebuch}\pwindex{Schnitzler, Arthur 15.05.1862 – 21.10.1931@\textsc{Schnitzler, Arthur} (15.05.1862 – 21.10.1931), \emph{Schriftsteller, Mediziner}!Tagebuch1981 – 2000@\strich\emph{Tagebuch} {[}1981 – 2000{]}|pwk} berichtet wird: »Ansätze zu
                     einem lit. Verein Jung Wien\oindex{Wien@\textbf{Wien}|pw}: Poestion\pwindex{Poestion, Josef Calasanz 07.06.1853 – 05.05.1922@\textsc{Poestion, Josef Calasanz} (07.06.1853 – 05.05.1922), \emph{Schriftsteller, Ministerialbeamter, Bibliotheksleiter}|pw}, Lemmermayer\pwindex{Lemmermayer, Fritz 26.03.1857 – 11.09.1932@\textsc{Lemmermayer, Fritz} (26.03.1857 – 11.09.1932), \emph{Schriftsteller}|pw}, Steiner\pwindex{Steiner, Rudolf 27.02.1861 – 30.03.1925@\textsc{Steiner, Rudolf} (27.02.1861 – 30.03.1925), \emph{Philosoph}|pw}, List\pwindex{List, Guido von 05.10.1848 – 21.05.1919@\textsc{List, Guido von} (05.10.1848 – 21.05.1919), \emph{Privatgelehrte}|pw}, Wodiczka\pwindex{Wodiczka, Viktor 09.01.1851 – 08.07.1898@\textsc{Wodiczka, Viktor} (09.01.1851 – 08.07.1898), \emph{Schriftsteller, Beamter}|pw}, Ludaßy\pwindex{Gans-Ludassy, Julius von 13.04.1858 – 30.09.1922@\textsc{Gans-Ludassy, Julius von} (13.04.1858 – 30.09.1922), \emph{Schriftsteller, Journalist, Herausgeber}|pw}, Klein\pwindex{Klein, Hugo 21.07.1853 – 29.06.1915@\textsc{Klein, Hugo} (21.07.1853 – 29.06.1915), \emph{Schriftsteller, Kritiker, Journalist}|pw}, Breitenstein\pwindex{Breitenstein, Max 10.02.1853 – 22.09.1926@\textsc{Breitenstein, Max} (10.02.1853 – 22.09.1926), \emph{Journalist}|pw}, Goldmann\pwindex{Goldmann, Paul 31.01.1865 – 25.09.1935@\textsc{Goldmann, Paul} (31.01.1865 – 25.09.1935), \emph{Schriftsteller, Journalist}|pw}, ich.« Spannend ist, dass bei diesem frühen
                  Zusammenschluss mit Guido von List\pwindex{List, Guido von 05.10.1848 – 21.05.1919@\textsc{List, Guido von} (05.10.1848 – 21.05.1919), \emph{Privatgelehrte}|pwk} und Rudolf Steiner\pwindex{Steiner, Rudolf 27.02.1861 – 30.03.1925@\textsc{Steiner, Rudolf} (27.02.1861 – 30.03.1925), \emph{Philosoph}|pwk} deutschnationale und
                  antroposophische Mythenmetze beteiligt waren.}}}\label{K_L02792-98765h}. Die Jungen Wien\oindex{Wien@\textbf{Wien}|pw}er haben keines Entbinders bedurft; aber wenn ſchon \strikeout{ei} Einer da war, der ſie zuſammengeſucht hat, ſo war
                  \uline{ich} es. Als \textsc{Bahr\pwindex{Bahr, Hermann 19.07.1863 – 15.01.1934@\textsc{Bahr, Hermann} (19.07.1863 – 15.01.1934), \emph{Schriftsteller, Kritiker}|pw}} nach \textsc{Wien\oindex{Wien@\textbf{Wien}|pw}} kam, waren ſchon \strikeout{Alle} Alle da; und ſeine
               Wirkſamkeit hat ſich darauf beſchränkt, daß er Dich beſchimpft {\pb}und verkannt hat; daß er den \textsc{Loris\pwindex{Hofmannsthal, Hugo von 1874-02-01 – 1929-07-15@\textsc{Hofmannsthal, Hugo von} (1874-02-01 – 1929-07-15), \emph{Schriftsteller}|pw}} mißverſtanden und verdorben hat; und daß er als neues Genie den grotesken
               Zieraffen \textsc{Andrian\pwindex{Andrian-Werburg, Leopold von 09.05.1875 – 19.11.1951@\textsc{Andrian-Werburg, Leopold von} (09.05.1875 – 19.11.1951), \emph{Schriftsteller, Diplomat}|pw}} gefunden hat. Und das läßt ſich \label{K_L02792-98v}\edtext{als Begründer\pwindex{Bahr, Hermann 19.07.1863 – 15.01.1934@\textsc{Bahr, Hermann} (19.07.1863 – 15.01.1934), \emph{Schriftsteller, Kritiker}|pwv} der Wien\oindex{Wien@\textbf{Wien}|pw}er Bewegung preiſen}{\lemma{\textnormal{\emph{als … preiſen}}}\Cendnote{\textnormal{siehe Paul Goldmann an Arthur Schnitzler, 1. 6. [1894]}}}\label{K_L02792-98h}, deren gute Leiſtungen immer nur \uline{trotz}{ }\textsc{Bahr\pwindex{Bahr, Hermann 19.07.1863 – 15.01.1934@\textsc{Bahr, Hermann} (19.07.1863 – 15.01.1934), \emph{Schriftsteller, Kritiker}|pw}} entſtanden ſind! {\dotsfour}\pend
           \pstart
           Dieſer \textsc{Dr. Graf\pwindex{Graf, Max 01.10.1873 – 24.06.1958@\textsc{Graf, Max} (01.10.1873 – 24.06.1958), \emph{Kritiker}|pw}}, den mir \textsc{Richard\pwindex{Beer-Hofmann, Richard 1866-07-11 – 1945-09-26@\textsc{Beer-Hofmann, Richard} (1866-07-11 – 1945-09-26), \emph{Schriftsteller}|pw}} geſchickt hat, gefällt mir recht gut. Er hat eine angenehme Art, iſt aber wohl
               keine {\pb}ſtarke Perſönlichkeit und kein ſehr klarer
               Kopf. Er ſtreckt unſicher ſeine Fühlhörner ins \strikeout{Leben}
               Leben aus. \strikeout{\textcolor{gray}{Wa}} Seine \textsc{Bahr\pwindex{Bahr, Hermann 19.07.1863 – 15.01.1934@\textsc{Bahr, Hermann} (19.07.1863 – 15.01.1934), \emph{Schriftsteller, Kritiker}|pw}}-Bewunderung habe ich bereits ein wenig erſchüttert; aber es iſt nicht gut
               möglich, ihm auszureden, daß \textsc{Altenberg\pwindex{Altenberg, Peter 09.03.1859 – 08.01.1919@\textsc{Altenberg, Peter} (09.03.1859 – 08.01.1919), \emph{Schriftsteller}|pw}} ein genialer Dichtergeiſt iſt. Wollen ſehen, was man aus ihm machen kann.
               Einſtweilen habe ich ihm kleine Arbeiten für unſer Blatt\orgindex{Frankfurter Zeitung@Frankfurter Zeitung|pwv} verſchafft.\pend
           \pstart
           \strikeout{Di\textcolor{gray}{e}} Die Fragen, die Du an mich ſtellſt, \label{K_L02792-77v}\edtext{\textsc{\begin{otherlanguage}{french}me concernant\end{otherlanguage}}}{\lemma{\textnormal{\emph{me concernant}}}\Cendnote{\textnormal{französisch: mich betreffend}}}\label{K_L02792-77h},
               beantworten ſich von ſelbſt durch den Eingang dieſes Briefes {\pb}(zu deſſen Fertigftellung ich drei Tage gebraucht).
               Stimmung: verzweifelt (ich werde nie dazu kommen, den tiefen Riß in meinem Leben \strikeout{a\textcolor{gray}{×}} auszufüllen); Stellung: unerfreulich; Arbeit: null; Freunde: ein paar brave
                  \label{K_L02792-34v}\edtext{Leute}{\lemma{\textnormal{\emph{Leute}}}\Cendnote{\textnormal{nicht identifiziert}}}\label{K_L02792-34h} auf \textsc{Montmartre\oindex{Paris 18 Buttes-Montmartre@\textbf{Paris 18 Buttes-Montmartre}|pw}}, ehrliche und ſimple Menſchen, die mich in ihrer kühlen Weiſe gern haben und –
               nicht verſtehen; Geliebte: ſchwere pſychiſche (?) Impotenz{\dotsfour}\pend
           \pstart
           Willſt Du mir einen Gefallen thun? Ich möchte gern den »\label{K_L02792-888v}\edtext{\textsc{Lorenzaccio\pwindex{Musset, Alfred de 11.12.1810 – 02.05.1857@\textsc{Musset, Alfred de} (11.12.1810 – 02.05.1857), \emph{Schriftsteller}!Lorenzaccio. Drame romantique en cinq actes1834@\strich\emph{Lorenzaccio. Drame romantique en cinq actes} {[}1834{]}|pw}}}{\lemma{\textnormal{\emph{Lorenzaccio}}}\Cendnote{\textnormal{\emph{Lorenzaccio. Drame romantique en cinq actes}\pwindex{Musset, Alfred de 11.12.1810 – 02.05.1857@\textsc{Musset, Alfred de} (11.12.1810 – 02.05.1857), \emph{Schriftsteller}!Lorenzaccio. Drame romantique en cinq actes1834@\strich\emph{Lorenzaccio. Drame romantique en cinq actes} {[}1834{]}|pwk}
                  wurde postum am 3. 12. 1896 uraufgeführt –
                  zweiundsechzig Jahre nach der Veröffentlichung am \emph{Théâtre de la Renaissance}\orgindex{Theâtre de la Renaissance@Théâtre de la Renaissance|pwk}. Die Hauptrolle spielte Sarah Bernhardt\pwindex{Bernhardt, Sarah 22.10.1844 – 26.03.1923@\textsc{Bernhardt, Sarah} (22.10.1844 – 26.03.1923), \emph{Schauspielerin}|pwk}.}}}\label{K_L02792-888h}« von \textsc{Musset\pwindex{Musset, Alfred de 11.12.1810 – 02.05.1857@\textsc{Musset, Alfred de} (11.12.1810 – 02.05.1857), \emph{Schriftsteller}|pw}}{ }\label{K_L02792-677v}\edtext{für die deutſche {\pb}Bühne bearbeiten}{\lemma{\textnormal{\emph{für … bearbeiten}}}\Cendnote{\textnormal{Die Idee bestand jedenfalls seit 1894 (vgl. A. S.: \emph{Tagebuch}, 8. 9. 1894 und Paul Goldmann an Arthur Schnitzler, 21. 9. [1894]. Schnitzler\pwindex{Schnitzler, Arthur 15.05.1862 – 21.10.1931@\textsc{Schnitzler, Arthur} (15.05.1862 – 21.10.1931), \emph{Schriftsteller, Mediziner}|pwk} fühlte bei Otto Brahm\pwindex{Brahm, Otto 05.02.1856 – 28.11.1912@\textsc{Brahm, Otto} (05.02.1856 – 28.11.1912), \emph{Theaterleiter, Regisseur}|pwk} vor, der ihm am 13. 5. 1897
                  antwortete: »Wegen einer \uline{Lorenzaccio\pwindex{Musset, Alfred de 11.12.1810 – 02.05.1857@\textsc{Musset, Alfred de} (11.12.1810 – 02.05.1857), \emph{Schriftsteller}!Lorenzaccio. Drame romantique en cinq actes1834@\strich\emph{Lorenzaccio. Drame romantique en cinq actes} {[}1834{]}|pw}}-Übersetzung bin ich Ihnen auch noch eine Antwort schuldig. Es ist
                     inzwischen eine bei uns eingelaufen und abgelehnt worden. Ist das die Ihres
                     Protegés? Ich glaube kaum, daß das Stück\pwindex{Musset, Alfred de 11.12.1810 – 02.05.1857@\textsc{Musset, Alfred de} (11.12.1810 – 02.05.1857), \emph{Schriftsteller}!Lorenzaccio. Drame romantique en cinq actes1834@\strich\emph{Lorenzaccio. Drame romantique en cinq actes} {[}1834{]}|pwv} bei uns Chancen hätte; aber wenn die Sache für
                     Ihren Unbekannten noch nicht erledigt ist – einreichen kann er ja immer, das
                     ist Menschenrecht.« (Brahm/Schnitzler, 33)}}}\label{K_L02792-677h}. Ich
               ſende Dir anbei das \label{K_L02792-858v}\edtext{Feuilleton\pwindex{Goldmann, Paul 31.01.1865 – 25.09.1935@\textsc{Goldmann, Paul} (31.01.1865 – 25.09.1935), \emph{Schriftsteller, Journalist}!Lorenzaccio1896-12-13 – 1896-12-14@\strich\emph{Lorenzaccio} {[}1896-12-13 – 1896-12-14{]}|pwv}}{\lemma{\textnormal{\emph{Feuilleton}}}\Cendnote{\textnormal{Paul Goldmann\pwindex{Goldmann, Paul 31.01.1865 – 25.09.1935@\textsc{Goldmann, Paul} (31.01.1865 – 25.09.1935), \emph{Schriftsteller, Journalist}|pwk}: \emph{Lorenzaccio}\pwindex{Goldmann, Paul 31.01.1865 – 25.09.1935@\textsc{Goldmann, Paul} (31.01.1865 – 25.09.1935), \emph{Schriftsteller, Journalist}!Lorenzaccio1896-12-13 – 1896-12-14@\strich\emph{Lorenzaccio} {[}1896-12-13 – 1896-12-14{]}|pwk}. In: \emph{Frankfurter Zeitung}\pwindex{?? Werk@Nicht ermittelte Verfasserinnen und Verfasser!Frankfurter Zeitung1856 – 1943@\emph{Frankfurter Zeitung} {[}1856 – 1943{]}|pwk}, Jg. 41, Nr. 346, 13. 12. 1896,
                     Erstes Morgenblatt, S. 3; Nr. 347, 14. 12. 1896, Morgenblatt,
                     S. 1–2.}}}\label{K_L02792-858h}, das ich darüber geſchrieben. Könnte ich vielleicht vom
                  »Burgtheater\orgindex{Burgtheater@Burgtheater|pw}« den Auftrag zu dieſer
               Bearbeitung bekommen? Könnteſt Du ein Wort mit \textsc{Burckhardt\pwindex{Burckhard, Max Eugen 14.07.1854 – 16.03.1912@\textsc{Burckhard, Max Eugen} (14.07.1854 – 16.03.1912), \emph{Schriftsteller, Rechtswissenschaftler, Theaterleiter}|pw}} oder mit \textsc{Uhl\pwindex{Uhl, Friedrich 14.05.1825 – 20.01.1906@\textsc{Uhl, Friedrich} (14.05.1825 – 20.01.1906), \emph{Journalist}|pw}} reden? In meinem Feuilleton\pwindex{Goldmann, Paul 31.01.1865 – 25.09.1935@\textsc{Goldmann, Paul} (31.01.1865 – 25.09.1935), \emph{Schriftsteller, Journalist}!Lorenzaccio1896-12-13 – 1896-12-14@\strich\emph{Lorenzaccio} {[}1896-12-13 – 1896-12-14{]}|pwv} finden ſie alle nöthigen ſachlichen Angaben über das Stück\pwindex{Musset, Alfred de 11.12.1810 – 02.05.1857@\textsc{Musset, Alfred de} (11.12.1810 – 02.05.1857), \emph{Schriftsteller}!Lorenzaccio. Drame romantique en cinq actes1834@\strich\emph{Lorenzaccio. Drame romantique en cinq actes} {[}1834{]}|pwv}. Das iſt ſo eine
               phantaſtiſche Idee, die ich habe; ausführbar wird ſie natürlich nicht ſein; und es
               lohnt nicht der Mühe, daß Du Dir deßwegen auch nur einen überflüßigen Weg machſt{\dotsfive}\pend
           \pstart
           {\pb}Wie gern würde ich Dich bald einmal wiederſehen\substVorne{}\textsuperscript{?}\substDazwischen{}!\substHinten{} Iſt gar keine Ausſicht, daß Du \label{K_L02792-54v}\edtext{nach \textsc{Paris\oindex{Paris@\textbf{Paris}|pw}}}{\lemma{\textnormal{\emph{nach Paris}}}\Cendnote{\textnormal{Schnitzler\pwindex{Schnitzler, Arthur 15.05.1862 – 21.10.1931@\textsc{Schnitzler, Arthur} (15.05.1862 – 21.10.1931), \emph{Schriftsteller, Mediziner}|pwk} und Marie Reinhard\pwindex{Reinhard, Marie 1871-03-13 – 1899-03-18@\textsc{Reinhard, Marie} (1871-03-13 – 1899-03-18), \emph{Gesangspädagogin}|pwk} kamen am 12. 4. 1897 nach Paris\oindex{Paris@\textbf{Paris}|pwk}. Er blieb bis zum 24. 5. 1897, sie reiste einen Tag früher
                  ab.}}}\label{K_L02792-54h} kommſt?\pend
           \pstart
           Grüß’ mir den lieben \textsc{Richard\pwindex{Beer-Hofmann, Richard 1866-07-11 – 1945-09-26@\textsc{Beer-Hofmann, Richard} (1866-07-11 – 1945-09-26), \emph{Schriftsteller}|pw}} und auch \textsc{Leo Vanjung\pwindex{Van-Jung, Leo 15.10.1866 – 02.07.1939@\textsc{Van-Jung, Leo} (15.10.1866 – 02.07.1939), \emph{Gesangspädagoge, Mathematiker}|pw}}, wenn Du ihn \label{K_L02792-90v}\edtext{ſiehſt}{\lemma{\textnormal{\emph{ſiehſt}}}\Cendnote{\textnormal{Das nächste Mal trafen sich Schnitzler\pwindex{Schnitzler, Arthur 15.05.1862 – 21.10.1931@\textsc{Schnitzler, Arthur} (15.05.1862 – 21.10.1931), \emph{Schriftsteller, Mediziner}|pwk} und Leo Van-Jung\pwindex{Van-Jung, Leo 15.10.1866 – 02.07.1939@\textsc{Van-Jung, Leo} (15.10.1866 – 02.07.1939), \emph{Gesangspädagoge, Mathematiker}|pwk} vermutlich am 12. 1. 1897.}}}\label{K_L02792-90h}!\pend
           \pstart
           Allen den Deinigen wünſche ich ein glückliches neues Jahr; empfiehl’ mich
               insbeſondere Deiner Frau Mutter\pwindex{Schnitzler, Louise 1840-07-08 – 1911-09-09@\textsc{Schnitzler, Louise} (1840-07-08 – 1911-09-09)|pwv} und grüße mir recht herzlich Deinen Bruder\pwindex{Schnitzler, Julius 13.07.1865 – 29.06.1939@\textsc{Schnitzler, Julius} (13.07.1865 – 29.06.1939), \emph{Chirurg}|pwv} und Deine Schwägerin\pwindex{Schnitzler, Helene 16.07.1871 – September 1941@\textsc{Schnitzler, Helene} (16.07.1871 – September 1941)|pwv}.\pend
           \pstart
           {\pb}Und ſei’ Du ſelbſt von Herzen gegrüßt!\pend
           \pstart
           In Treue {\\[\baselineskip]}Dein {\\[\baselineskip]}\spacefill\mbox{Paul Goldmann.}\pend
           \leftskip=0em{}\pstart
           \noindent{}Nicht wahr, Du ſchreibſt mir bald wieder einmal?\pend
           
         
         \endnumbering\mylabel{h}\end{ledgroupsized}  \newcommand{\dateiname}{L02792}\newcommand{\titel}{Paul Goldmann an Arthur Schnitzler, 2. [1.? 1897]}\newcommand{\editorInnen}{Martin Anton Müller und Laura Untner}%% latex-leseansicht-abspann.tex
%% Abspann für die Leseansicht.
%% Der Schalter \ifkorrekturansicht ist bereits durch den Vorspann gesetzt.

%% latex-abspann.tex
%% Gemeinsamer Abspann für Korrekturansicht und Leseansicht.
%% Setzt den Schalter \ifkorrekturansicht voraus (gesetzt in den
%% einbindenden Dateien latex-korrekturansicht-abspann.tex bzw.
%% latex-leseansicht-abspann.tex).
%% ---------------------------------------------------------------

\normalsize

% Das esempio-Environment wird nur in der Leseansicht benötigt
\ifkorrekturansicht\else
\newenvironment{esempio}[3]%
{
    \vspace{1.5ex}
    \rlap{\underline{#1}}
    \par
    \setlength{\parindent}{0cm}
    \nopagebreak
    \leftskip=#2cm
    \rightskip=#3cm
}
{
    \par
}
\fi

\doendnotes{C}
\bigskip
\vfill

\clearpage

\footnotesize

\ifkorrekturansicht
  \lohead{\textsc{register}}
\fi

% theindex-Environment neu definieren ohne reledmac
\makeatletter
\renewenvironment{theindex}{%
  \ifkorrekturansicht
    \section*{\indexname}%
  \else
    \subsubsection*{Index der erwähnten Entitäten}%
  \fi
  \setlength{\parindent}{0pt}%
  \setlength{\parskip}{0pt plus 0.3pt}%
  \let\item\@idxitem
}{%
  \ifkorrekturansicht\clearpage\fi
}
\makeatother

\IfFileExists{\jobname-pw.ind}{\input{\jobname-pw.ind}}{}

% Quellenangabe nur in der Leseansicht
\ifkorrekturansicht\else
% Fallback-Definitionen, falls die .tex-Datei \titel etc. nicht gesetzt hat
\providecommand{\titel}{}
\providecommand{\editorInnen}{}
\providecommand{\dateiname}{\jobname}

\vspace{3cm}

\vfill

\footnotesize
\textsc{Quelle}: \titel. Herausgegeben von {\editorInnen}. In: \emph{Arthur Schnitzler: Briefwechsel mit Autorinnen und Autoren}.
 Digitale Edition, https://schnitzler-briefe.acdh.oeaw.ac.at/{\dateiname}.html (Stand \today)
\fi

\end{document}


      