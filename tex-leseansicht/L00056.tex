%% latex-leseansicht-vorspann.tex
%% Vorspann für die Leseansicht.
%% Lädt die gemeinsame Datei latex-vorspann.tex mit nicht gesetztem Schalter.

\newif\ifkorrekturansicht
\korrekturansichtfalse

\input{../tex-inputs/latex-vorspann}


         
         \renewcommand{\erwaehntePersonen}{Personen: Ferry Bératon, Hugo von Hofmannsthal, Maurice Maeterlinck}
         \renewcommand{\erwaehnteOrte}{Orte: Wien}
         \renewcommand{\erwaehnteWerke}{Werke: L’Intruse, Prinzessin Maleine}
               \section[Hermann Bahr an Arthur Schnitzler, {[}22. 12. 1891{]}]{ Hermann Bahr an Arthur Schnitzler, {[}22. 12. 1891{]}}\nopagebreak\mylabel{v}\rehead{ }\begin{ledgroupsized}[t]{13cm}\normalsize\beginnumbering \toendnotes[C]{\smallbreak\pagebreak[2]} \Standort{CUL, Schnitzler, B 5b.}
\physDesc{Brief, 1 Blatt, 1 Seite, 387 Zeichen
\newline{}Handschrift: Bleistift, deutsche Kurrent
\newline{}Schnitzler: 1) mit Bleistift datiert: »22/12 91. «  2) mit rotem Buntstift nummeriert: »1.«
\newline{}Ordnung: mit Bleistift von unbekannter Hand nummeriert:
                                    »1.« und verso »\textsc{Bahr}« beschriftet }\buchAbdrucke{\weitereDrucke{Hermann Bahr, Arthur Schnitzler: \emph{Briefwechsel, Aufzeichnungen, Dokumente (1891–1931)}. Hg. Kurt Ifkovits und Martin Anton Müller. Göttingen: \emph{Wallstein} 2018, S. 16.} }\toendnotes[C]{\smallbreak}\pstart{}{\pb}Lieber Herr Dr!\pend\pstart
           Bitte, teilen Sie mir we{\geminationn} möglich mit, ob es Ihnen paßt,
               daß uns morgen \introOben{}Mittwoch\introOben{} Abend von 6–8 (ſei es bei Ihnen,
               oder bei mir) \textsc{Bératon}\pwindex{Beraton, Ferry 06.12.1859 – 11.02.1900@\textsc{Bératon, Ferry} (06.12.1859 – 11.02.1900), \emph{Schriftsteller, Journalist, Bildender Künstler}|pw}{ }\label{K_L00056_1v}\edtext{ſein Stück}{\lemma{\textnormal{\emph{ſein Stück}}}\Cendnote{\textnormal{Unklar. Nachdem am 2. 5. 1892{ }\emph{L’intruse}\pwindex{Maeterlinck, Maurice 29.08.1862 – 06.05.1949@\textsc{Maeterlinck, Maurice} (29.08.1862 – 06.05.1949), \emph{Schriftsteller}!Intruse1891@\strich\emph{L’Intruse} {[}1891{]}|pwk} von Maurice Maeterlinck\pwindex{Maeterlinck, Maurice 29.08.1862 – 06.05.1949@\textsc{Maeterlinck, Maurice} (29.08.1862 – 06.05.1949), \emph{Schriftsteller}|pwk} in Bératons\pwindex{Beraton, Ferry 06.12.1859 – 11.02.1900@\textsc{Bératon, Ferry} (06.12.1859 – 11.02.1900), \emph{Schriftsteller, Journalist, Bildender Künstler}|pwk} Übersetzung gegeben wurde und zuvor weitere Dramen des Autors
                  zur Inszenierung angedacht waren, könnte es sich um eine Übertragung von \emph{La Princesse Maleine}\pwindex{Maeterlinck, Maurice 29.08.1862 – 06.05.1949@\textsc{Maeterlinck, Maurice} (29.08.1862 – 06.05.1949), \emph{Schriftsteller}!Prinzessin Maleine1889@\strich\emph{Prinzessin Maleine} {[}1889{]}|pwk} handeln.}}}\label{K_L00056_1h} vorlieſt.
               Ich möchte Sie bitten, mich etwa bis 5 zu verſtändigen, da ich noch zu \textsc{Loris}\pwindex{Hofmannsthal, Hugo von 1874-02-01 – 1929-07-15@\textsc{Hofmannsthal, Hugo von} (1874-02-01 – 1929-07-15), \emph{Schriftsteller}|pw}{ }ſchicken u \textsc{Beraton}\pwindex{Beraton, Ferry 06.12.1859 – 11.02.1900@\textsc{Bératon, Ferry} (06.12.1859 – 11.02.1900), \emph{Schriftsteller, Journalist, Bildender Künstler}|pw} Antwort ſagen muß.\pend
           \pstart
           \substVorne{}\textsuperscript{\textcolor{gray}{M}}\substDazwischen{}Im\substHinten{} übrigen bitte größte Discretion! B.\pwindex{Beraton, Ferry 06.12.1859 – 11.02.1900@\textsc{Bératon, Ferry} (06.12.1859 – 11.02.1900), \emph{Schriftsteller, Journalist, Bildender Künstler}|pw}
               will nicht, daß »die Welt« etwas von ſr Miſſetat erfahre.\pend
           \pstart
           Herzlichſt{\\[\baselineskip]}\spacefill\mbox{Bahr.}\pend
           \leftskip=0em{}
         
         \endnumbering\mylabel{h}\end{ledgroupsized}  \newcommand{\dateiname}{L00056}\newcommand{\titel}{Hermann Bahr an Arthur Schnitzler, [22. 12. 1891]}\newcommand{\editorInnen}{ Kurt Ifkovits,  Martin Anton Müller}%% latex-leseansicht-abspann.tex
%% Abspann für die Leseansicht.
%% Der Schalter \ifkorrekturansicht ist bereits durch den Vorspann gesetzt.

%% latex-abspann.tex
%% Gemeinsamer Abspann für Korrekturansicht und Leseansicht.
%% Setzt den Schalter \ifkorrekturansicht voraus (gesetzt in den
%% einbindenden Dateien latex-korrekturansicht-abspann.tex bzw.
%% latex-leseansicht-abspann.tex).
%% ---------------------------------------------------------------

\normalsize

% Das esempio-Environment wird nur in der Leseansicht benötigt
\ifkorrekturansicht\else
\newenvironment{esempio}[3]%
{
    \vspace{1.5ex}
    \rlap{\underline{#1}}
    \par
    \setlength{\parindent}{0cm}
    \nopagebreak
    \leftskip=#2cm
    \rightskip=#3cm
}
{
    \par
}
\fi

\doendnotes{C}
\bigskip
\vfill

\clearpage

\footnotesize

\ifkorrekturansicht
  \lohead{\textsc{register}}
\fi

% theindex-Environment neu definieren ohne reledmac
\makeatletter
\renewenvironment{theindex}{%
  \ifkorrekturansicht
    \section*{\indexname}%
  \else
    \subsubsection*{Index der erwähnten Entitäten}%
  \fi
  \setlength{\parindent}{0pt}%
  \setlength{\parskip}{0pt plus 0.3pt}%
  \let\item\@idxitem
}{%
  \ifkorrekturansicht\clearpage\fi
}
\makeatother

\IfFileExists{\jobname-pw.ind}{\input{\jobname-pw.ind}}{}

% Quellenangabe nur in der Leseansicht
\ifkorrekturansicht\else
% Fallback-Definitionen, falls die .tex-Datei \titel etc. nicht gesetzt hat
\providecommand{\titel}{}
\providecommand{\editorInnen}{}
\providecommand{\dateiname}{\jobname}

\vspace{3cm}

\vfill

\footnotesize
\textsc{Quelle}: \titel. Herausgegeben von {\editorInnen}. In: \emph{Arthur Schnitzler: Briefwechsel mit Autorinnen und Autoren}.
 Digitale Edition, https://schnitzler-briefe.acdh.oeaw.ac.at/{\dateiname}.html (Stand \today)
\fi

\end{document}


      