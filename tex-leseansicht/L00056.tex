%% latex-leseansicht-vorspann.tex
%% Vorspann für die Leseansicht.
%% Lädt die gemeinsame Datei latex-vorspann.tex mit nicht gesetztem Schalter.

\newif\ifkorrekturansicht
\korrekturansichtfalse

\input{../tex-inputs/latex-vorspann}


\section[Hermann Bahr an Arthur Schnitzler, {{[}}22. 12. 1891{{]}}]{L00056 Hermann Bahr an Arthur Schnitzler, {[}22. 12. 1891{]}}
\nopagebreak\mylabel{L00056v}
\rehead{ }\normalsize\beginnumbering\briefempfaengerindex{Schnitzler, Arthur@\textsc{Schnitzler, Arthur}!zzzBahr, Hermann@\emph{von Hermann Bahr}!1891-12-221@{22. 12. 1891}|(be}
\toendnotes[C]{\smallbreak\pagebreak[2]}
\correspDesc{Versand  durch Hermann Bahr am 22. 12. 1891 in Wien
\newline{}Erhalt  durch Arthur Schnitzler im Zeitraum [22. 12. 1891 – 26. 12. 1891?] in Wien}\toendnotes[C]{\smallbreak}
\Standort{CUL, Schnitzler, B 5b.}
\physDesc{Brief, 1 Blatt, 1 Seite, 387 Zeichen
\newline{}Handschrift: Bleistift, deutsche Kurrent
\newline{}Schnitzler: 1) mit Bleistift datiert: »22/12 91. «  2) mit rotem Buntstift nummeriert: »1.«
\newline{}Ordnung: mit Bleistift von unbekannter Hand nummeriert:
                                    »1.« und verso »\textsc{Bahr}« beschriftet }
\buchAbdrucke{\weitereDrucke{Hermann Bahr, Arthur Schnitzler: \emph{Briefwechsel, Aufzeichnungen, Dokumente (1891–1931)}. Herausgegeben von Kurt Ifkovits und Martin Anton Müller. Göttingen: \emph{Wallstein} 2018, S. 16.} }\toendnotes[C]{\smallbreak}
\pstart{}{\pb}Lieber Herr Dr!\pend\vspace{0.5em}
\pstart
           Bitte, teilen Sie mir we{\geminationn} möglich mit, ob es Ihnen paßt,
               daß uns morgen \introOben{}Mittwoch\introOben{} Abend von 6–8 (ſei es bei Ihnen,
               oder bei mir) \textsc{Bératon}\pwindex{Bératon, Ferry 6.\,12.\,1859 Wien – 11.\,2.\,1900 Venedig@\textsc{Bératon, Ferry} (6.\,12.\,1859 Wien – 11.\,2.\,1900 Venedig), \emph{Schriftsteller, Journalist, Maler}|pw}{ }\label{K_L00056-1v}\edtext{ſein Stück}{\lemma{\textnormal{\emph{sein Stück}}}\Cendnote{\textnormal{Unklar. Nachdem am 2. 5. 1892{ }\emph{L’intruse}\pwindex{Maeterlinck, Maurice 29.\,8.\,1862 Gent – 6.\,5.\,1949 Nizza@\textsc{Maeterlinck, Maurice} (29.\,8.\,1862 Gent – 6.\,5.\,1949 Nizza), \emph{Schriftsteller}!Intruse@\strich\emph{L’Intruse}|pwk} von Maurice Maeterlinck\pwindex{Maeterlinck, Maurice 29.\,8.\,1862 Gent – 6.\,5.\,1949 Nizza@\textsc{Maeterlinck, Maurice} (29.\,8.\,1862 Gent – 6.\,5.\,1949 Nizza), \emph{Schriftsteller}|pwk} in Bératons\pwindex{Bératon, Ferry 6.\,12.\,1859 Wien – 11.\,2.\,1900 Venedig@\textsc{Bératon, Ferry} (6.\,12.\,1859 Wien – 11.\,2.\,1900 Venedig), \emph{Schriftsteller, Journalist, Maler}|pwk} Übersetzung gegeben und zuvor weitere Dramen des Autors
                  zur Inszenierung angedacht waren, könnte es sich um eine Übertragung von \emph{La Princesse Maleine}\pwindex{Maeterlinck, Maurice 29.\,8.\,1862 Gent – 6.\,5.\,1949 Nizza@\textsc{Maeterlinck, Maurice} (29.\,8.\,1862 Gent – 6.\,5.\,1949 Nizza), \emph{Schriftsteller}!Prinzessin Maleine@\strich\emph{Prinzessin Maleine}|pwk} handeln.}}}\label{K_L00056-1} vorlieſt.
               Ich möchte Sie bitten, mich etwa bis 5 zu verſtändigen, da ich noch zu \textsc{Loris}\pwindex{Hofmannsthal, Hugo von 1.\,2.\,1874 Wien – 15.\,7.\,1929 Rodaun@\textsc{Hofmannsthal, Hugo von} (1.\,2.\,1874 Wien – 15.\,7.\,1929 Rodaun), \emph{Schriftsteller}|pw}{ }ſchicken u \textsc{Beraton}\pwindex{Bératon, Ferry 6.\,12.\,1859 Wien – 11.\,2.\,1900 Venedig@\textsc{Bératon, Ferry} (6.\,12.\,1859 Wien – 11.\,2.\,1900 Venedig), \emph{Schriftsteller, Journalist, Maler}|pw} Antwort{ }ſagen muß.\pend
           
\pstart
           \substVorne{}\textsuperscript{\textcolor{gray}{M}}\substDazwischen{}Im\substHinten{} übrigen bitte größte Discretion! B.\pwindex{Bératon, Ferry 6.\,12.\,1859 Wien – 11.\,2.\,1900 Venedig@\textsc{Bératon, Ferry} (6.\,12.\,1859 Wien – 11.\,2.\,1900 Venedig), \emph{Schriftsteller, Journalist, Maler}|pw}
               will nicht, daß »die Welt« etwas von{ }ſr Miſſetat erfahre.\pend
           
\pstart
           Herzlichſt{\\[\baselineskip]}\spacefill\mbox{Bahr.}\pend
           \leftskip=0em{}\selectlanguage{ngerman}\endnumbering\briefempfaengerindex{Schnitzler, Arthur@\textsc{Schnitzler, Arthur}!zzzBahr, Hermann@\emph{von Hermann Bahr}!1891-12-221@{22. 12. 1891}|)be}\mylabel{L00056h}  \newcommand{\dateiname}{L00056}\newcommand{\titel}{Hermann Bahr an Arthur Schnitzler, [22. 12. 1891]}\newcommand{\editorInnen}{Herausgegeben von Martin Anton Müller}%% latex-leseansicht-abspann.tex
%% Abspann für die Leseansicht.
%% Der Schalter \ifkorrekturansicht ist bereits durch den Vorspann gesetzt.

%% latex-abspann.tex
%% Gemeinsamer Abspann für Korrekturansicht und Leseansicht.
%% Setzt den Schalter \ifkorrekturansicht voraus (gesetzt in den
%% einbindenden Dateien latex-korrekturansicht-abspann.tex bzw.
%% latex-leseansicht-abspann.tex).
%% ---------------------------------------------------------------

\normalsize

% Das esempio-Environment wird nur in der Leseansicht benötigt
\ifkorrekturansicht\else
\newenvironment{esempio}[3]%
{
    \vspace{1.5ex}
    \rlap{\underline{#1}}
    \par
    \setlength{\parindent}{0cm}
    \nopagebreak
    \leftskip=#2cm
    \rightskip=#3cm
}
{
    \par
}
\fi

\doendnotes{C}
\bigskip
\vfill

\clearpage

\footnotesize

\ifkorrekturansicht
  \lohead{\textsc{register}}
\fi

% theindex-Environment neu definieren ohne reledmac
\makeatletter
\renewenvironment{theindex}{%
  \ifkorrekturansicht
    \section*{\indexname}%
  \else
    \subsubsection*{Index der erwähnten Entitäten}%
  \fi
  \setlength{\parindent}{0pt}%
  \setlength{\parskip}{0pt plus 0.3pt}%
  \let\item\@idxitem
}{%
  \ifkorrekturansicht\clearpage\fi
}
\makeatother

\IfFileExists{\jobname-pw.ind}{\input{\jobname-pw.ind}}{}

% Quellenangabe nur in der Leseansicht
\ifkorrekturansicht\else
% Fallback-Definitionen, falls die .tex-Datei \titel etc. nicht gesetzt hat
\providecommand{\titel}{}
\providecommand{\editorInnen}{}
\providecommand{\dateiname}{\jobname}

\vspace{3cm}

\vfill

\footnotesize
\textsc{Quelle}: \titel. Herausgegeben von {\editorInnen}. In: \emph{Arthur Schnitzler: Briefwechsel mit Autorinnen und Autoren}.
 Digitale Edition, https://schnitzler-briefe.acdh.oeaw.ac.at/{\dateiname}.html (Stand \today)
\fi

\end{document}


