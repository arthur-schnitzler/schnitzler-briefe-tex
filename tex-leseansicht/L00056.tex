%% latex-korrekturansicht-vorspann.tex
%% Vorspann für die Korrekturansicht.
%% Lädt die gemeinsame Datei latex-vorspann.tex mit gesetztem Schalter.

\newif\ifkorrekturansicht
\korrekturansichttrue

\input{../tex-inputs/latex-vorspann}


\section[Hermann Bahr an Arthur Schnitzler, {[}22. 12. 1891{]}]{L00056 Hermann Bahr an Arthur Schnitzler, {[}22. 12. 1891{]}}
\nopagebreak\mylabel{L00056v}
\rehead{ }\normalsize\beginnumbering\briefempfaengerindex{Schnitzler, Arthur@\textsc{Schnitzler, Arthur}!zzzBahr, Hermann@\emph{von Hermann Bahr}!1891-12-221@{22. 12. 1891}|(be}
\toendnotes[C]{\smallbreak\pagebreak[2]}\Standort{CUL, Schnitzler, B 5b.}
\physDesc{Brief, 1 Blatt, 1 Seite, 387 Zeichen
\newline{}Handschrift: Bleistift, deutsche Kurrent
\newline{}Schnitzler: 1) mit Bleistift datiert: »22/12 91. «  2) mit rotem Buntstift nummeriert: »1.«
\newline{}Ordnung: mit Bleistift von unbekannter Hand nummeriert:
                                    »1.« und verso »\textsc{Bahr}« beschriftet }
\buchAbdrucke{\weitereDrucke{Hermann Bahr, Arthur Schnitzler: \emph{Briefwechsel, Aufzeichnungen, Dokumente (1891–1931)}. Göttingen: \emph{Wallstein} 2018, S. 16.} }\toendnotes[C]{\smallbreak}
\pstart{}{\pb}Lieber Herr Dr!\pend\vspace{0.5em}
\pstart
           Bitte, teilen Sie mir we{\geminationn} möglich mit, ob es Ihnen paßt,
               daß uns morgen \introOben{}Mittwoch\introOben{} Abend von 6–8 (ſei es bei Ihnen,
               oder bei mir) \textsc{Bératon}\pwindex{Beraton, Ferry 06.12.1859 – 11.02.1900@\textsc{Bératon, Ferry} (06.12.1859 – 11.02.1900), \emph{Schriftsteller/Schriftstellerin, Journalist/Journalistin, Maler/Malerin}|pw}{ }\label{K_L00056-1v}\edtext{ſein Stück}{\lemma{\textnormal{\emph{ſein Stück}}}\Cendnote{\textnormal{Unklar. Nachdem am 2. 5. 1892{ }\emph{L’intruse}\pwindex{Intruse@\emph{L’Intruse}|pwk} von Maurice Maeterlinck\pwindex{Maeterlinck, Maurice 29.08.1862 – 06.05.1949@\textsc{Maeterlinck, Maurice} (29.08.1862 – 06.05.1949), \emph{Schriftsteller/Schriftstellerin}|pwk} in Bératons\pwindex{Beraton, Ferry 06.12.1859 – 11.02.1900@\textsc{Bératon, Ferry} (06.12.1859 – 11.02.1900), \emph{Schriftsteller/Schriftstellerin, Journalist/Journalistin, Maler/Malerin}|pwk} Übersetzung gegeben und zuvor weitere Dramen des Autors
                  zur Inszenierung angedacht waren, könnte es sich um eine Übertragung von \emph{La Princesse Maleine}\pwindex{Prinzessin Maleine@\emph{Prinzessin Maleine}|pwk} handeln.}}}\label{K_L00056-1} vorlieſt.
               Ich möchte Sie bitten, mich etwa bis 5 zu verſtändigen, da ich noch zu \textsc{Loris}\pwindex{Hofmannsthal, Hugo von 1874-02-01 – 1929-07-15@\textsc{Hofmannsthal, Hugo von} (1874-02-01 – 1929-07-15), \emph{Schriftsteller/Schriftstellerin}|pw}{ }ſchicken u \textsc{Beraton}\pwindex{Beraton, Ferry 06.12.1859 – 11.02.1900@\textsc{Bératon, Ferry} (06.12.1859 – 11.02.1900), \emph{Schriftsteller/Schriftstellerin, Journalist/Journalistin, Maler/Malerin}|pw} Antwort ſagen muß.\pend
           
\pstart
           \substVorne{}\textsuperscript{\textcolor{gray}{M}}\substDazwischen{}Im\substHinten{} übrigen bitte größte Discretion! B.\pwindex{Beraton, Ferry 06.12.1859 – 11.02.1900@\textsc{Bératon, Ferry} (06.12.1859 – 11.02.1900), \emph{Schriftsteller/Schriftstellerin, Journalist/Journalistin, Maler/Malerin}|pw}
               will nicht, daß »die Welt« etwas von ſr Miſſetat erfahre.\pend
           
\pstart
           Herzlichſt{\\[\baselineskip]}\spacefill\mbox{Bahr.}\pend
           \leftskip=0em{}\selectlanguage{ngerman}\endnumbering\briefempfaengerindex{Schnitzler, Arthur@\textsc{Schnitzler, Arthur}!zzzBahr, Hermann@\emph{von Hermann Bahr}!1891-12-221@{22. 12. 1891}|)be}\mylabel{L00056h}  \normalsize

\doendnotes{C}
\bigskip
\vfill

\clearpage

\footnotesize

\lohead{\textsc{register}}

% Definiere theindex-Environment komplett neu ohne reledmac
\makeatletter
\renewenvironment{theindex}{%
  \section*{\indexname}%
  \setlength{\parindent}{0pt}%
  \setlength{\parskip}{0pt plus 0.3pt}%
  \let\item\@idxitem
}{%
  \clearpage
}
\makeatother

\IfFileExists{\jobname-pw.ind}{\input{\jobname-pw.ind}}{}

\end{document}

      