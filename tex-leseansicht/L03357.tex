%% latex-korrekturansicht-vorspann.tex
%% Vorspann für die Korrekturansicht.
%% Lädt die gemeinsame Datei latex-vorspann.tex mit gesetztem Schalter.

\newif\ifkorrekturansicht
\korrekturansichttrue

\input{../tex-inputs/latex-vorspann}


\section[ Felix Salten an Arthur Schnitzler, 20. 5. 1902]{L03357 Felix Salten an Arthur Schnitzler, 20. 5. 1902}
\nopagebreak\mylabel{L03357v}
\rehead{ }\normalsize\beginnumbering\briefempfaengerindex{Schnitzler, Arthur@\textsc{Schnitzler, Arthur}!zzzSalten, Felix@\emph{von Felix Salten}!1902-05-202@{20. 5. 1902}|(be}
\toendnotes[C]{\smallbreak\pagebreak[2]}\Standort{CUL, Schnitzler, B 89, A 2.}
\physDesc{Postkarte, 185 Zeichen
\newline{}Handschrift: Bleistift, lateinische Kurrent
\newline{}Versand: 1) Stempel: »\nobreak{}\oindex{Bologna@\textbf{Bologna}, \emph{P.PPLA}|pwk}Bologna Ferrovia, 20 5 – 02, 5S\nobreak{}«.   2) Stempel: »\nobreak{}\oindex{IX., Alsergrund@\textbf{IX., Alsergrund}, \emph{A.ADM3}|pwk}9/3 Wien 72, 22. 5. 02, 8. V, Bestellt\nobreak{}«. 
\newline{}Ordnung: mit Bleistift von unbekannter Hand nummeriert: »154« }\toendnotes[C]{\smallbreak}\pstart{}{\pb}Herrn D\textsuperscript{r} Arthur Schnitzler\pend{}\pstart{}IX. Frankgaße 1\oindex{Frankgasse 1@\textbf{Frankgasse 1}, \emph{Wohngebäude (K.WHS)}|pw}\pend{}\pstart{}Wien\oindex{Wien@\textbf{Wien}, \emph{A.ADM2}|pw}\pend{}\pstart{}Austria\oindex{Oesterreich@\textbf{Österreich}, \emph{A.PCLI}|pw}\pend{}{\bigskip}\vspace{1em}
\pstart
           {\pb}Bologna\oindex{Bologna@\textbf{Bologna}, \emph{P.PPLA}|pw}, 20. Mai 02.\pend
           \vspace{0.5em}
\pstart
           \label{K_L03357-1v}\edtext{Bentivoglio\pwindex{Schleier der Beatrice. Schauspiel in fuenf Akten@\emph{Der Schleier der Beatrice. Schauspiel in fünf Akten}|pwv} – San Petron\oindex{Basilika San Petronio@\textbf{Basilika San Petronio}, \emph{Kirche (K.KRC)}|pw}, – Beatrice\pwindex{Schleier der Beatrice. Schauspiel in fuenf Akten@\emph{Der Schleier der Beatrice. Schauspiel in fünf Akten}|pwv} u. s. w. Filippo Loschi\pwindex{Schleier der Beatrice. Schauspiel in fuenf Akten@\emph{Der Schleier der Beatrice. Schauspiel in fünf Akten}|pwv}}{\lemma{\textnormal{\emph{Bentivoglio … Loschi}}}\Cendnote{\textnormal{Orte und Personen aus Schnitzlers Theaterstück \emph{Der Schleier der Beatrice}\pwindex{Schleier der Beatrice. Schauspiel in fuenf Akten@\emph{Der Schleier der Beatrice. Schauspiel in fünf Akten}|pwk}, das in Bologna\oindex{Bologna@\textbf{Bologna}, \emph{P.PPLA}|pwk} angesiedelt ist.}}}\label{K_L03357-1} nicht zu vergessen, und dann der \label{K_L03357-2v}\edtext{durchgängige Hund}{\lemma{\textnormal{\emph{durchgängige Hund}}}\Cendnote{\textnormal{Eventuell reiste Salten\pwindex{Salten, Felix 06.09.1869 – 08.10.1945@\textsc{Salten, Felix} (06.09.1869 – 08.10.1945), \emph{Schriftsteller/Schriftstellerin, Journalist/Journalistin, Chefredakteur/Chefredakteurin}|pwk} mit seinem Hund? Oder eine Anspielung auf Vorarbeiten für \emph{Der Hund von Florenz}\pwindex{Hund von Florenz@\emph{Der Hund von Florenz}|pwk}? Explizit spricht er den Beginn der Arbeit
                     erst am 3. 3. 1903 aus.}}}\label{K_L03357-2}.\pend
           
\pstart
           herzl. {\\[\baselineskip]}\spacefill\mbox{F S.}\pend
           \leftskip=0em{}\selectlanguage{ngerman}\endnumbering\briefempfaengerindex{Schnitzler, Arthur@\textsc{Schnitzler, Arthur}!zzzSalten, Felix@\emph{von Felix Salten}!1902-05-202@{20. 5. 1902}|)be}\mylabel{L03357h}  \normalsize

\doendnotes{C}
\bigskip
\vfill

\clearpage

\footnotesize

\lohead{\textsc{register}}

% Definiere theindex-Environment komplett neu ohne reledmac
\makeatletter
\renewenvironment{theindex}{%
  \section*{\indexname}%
  \setlength{\parindent}{0pt}%
  \setlength{\parskip}{0pt plus 0.3pt}%
  \let\item\@idxitem
}{%
  \clearpage
}
\makeatother

\IfFileExists{\jobname-pw.ind}{\input{\jobname-pw.ind}}{}

\end{document}

      