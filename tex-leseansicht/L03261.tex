%% latex-korrekturansicht-vorspann.tex
%% Vorspann für die Korrekturansicht.
%% Lädt die gemeinsame Datei latex-vorspann.tex mit gesetztem Schalter.

\newif\ifkorrekturansicht
\korrekturansichttrue

\input{../tex-inputs/latex-vorspann}


\section[ Felix Salten an Arthur Schnitzler, {[}6. 1. 1897{]}]{L03261 Felix Salten an Arthur Schnitzler, {[}6. 1. 1897{]}}
\nopagebreak\mylabel{L03261v}
\rehead{ }\normalsize\beginnumbering\briefempfaengerindex{Schnitzler, Arthur@\textsc{Schnitzler, Arthur}!zzzSalten, Felix@\emph{von Felix Salten}!1897-01-061@{{[}6. 1. 1897{]}}|(be}
\toendnotes[C]{\smallbreak\pagebreak[2]}\Standort{CUL, Schnitzler, B 89, A 2.}
\physDesc{Brief, 1 Blatt, 1 Seite, 177 Zeichen
\newline{}Handschrift: Bleistift, lateinische Kurrent
\newline{}Schnitzler: mit Bleistift datiert: »6/1 97« 
\newline{}Ordnung: mit Bleistift von unbekannter Hand nummeriert: »83« }\toendnotes[C]{\smallbreak}
\pstart
           \noindent{}{\pb}lieber Arthur, in der \label{K_L03261-1v}\edtext{Affaire Kraus\pwindex{Kraus, Karl 28.04.1874 – 12.06.1936@\textsc{Kraus, Karl} (28.04.1874 – 12.06.1936), \emph{Schriftsteller/Schriftstellerin, Publizist/Publizistin, Schriftsteller/Schriftstellerin}|pw}}{\lemma{\textnormal{\emph{Affaire Kraus}}}\Cendnote{\textnormal{Karl
                     Kraus\pwindex{Kraus, Karl 28.04.1874 – 12.06.1936@\textsc{Kraus, Karl} (28.04.1874 – 12.06.1936), \emph{Schriftsteller/Schriftstellerin, Publizist/Publizistin, Schriftsteller/Schriftstellerin}|pwk} hatte Salten\pwindex{Salten, Felix 06.09.1869 – 08.10.1945@\textsc{Salten, Felix} (06.09.1869 – 08.10.1945), \emph{Schriftsteller/Schriftstellerin, Journalist/Journalistin, Chefredakteur/Chefredakteurin}|pwk} mehrfach wegen
                  seines schlechten Schreibstils angegriffen, zuletzt im dritten Teil von \emph{Die demolirte Literatur}\pwindex{demolirte Literatur@\emph{Die demolirte Literatur}|pwk}: »Im journalistischen
                     Dienste hart mitgenommen, hat sich der Literat bis heute doch seine Eigenart zu
                     wahren gewusst. Die Verwechslung des Dativs mit dem Accusativ gelingt ihm noch
                     immer mit unverminderter Jugendfrische. Anfänglich hatte er wohl mit dem
                     Widerstand der Setzer zu kämpfen, die bekanntlich immer klüger sein wollen als
                     der Schriftsteller und gerne corrigiren, weil sie für undeutsch ansehen, was
                     individuellster Ausdruck einer künstlerischen Persönlichkeit ist, aber bald
                     lernten sie die Eigenart unseres Autors respectiren, und sein Talent setzte
                     sich durch.« (In: \emph{Wiener
                        Rundschau}\pwindex{Wiener Rundschau@\emph{Wiener Rundschau}|pwk}, Jg. 1, H. 3, 15. 12. 1896,
                     S. 115.) Im gleichen Text spielte Kraus\pwindex{Kraus, Karl 28.04.1874 – 12.06.1936@\textsc{Kraus, Karl} (28.04.1874 – 12.06.1936), \emph{Schriftsteller/Schriftstellerin, Publizist/Publizistin, Schriftsteller/Schriftstellerin}|pwk} auch auf die Beziehung Saltens\pwindex{Salten, Felix 06.09.1869 – 08.10.1945@\textsc{Salten, Felix} (06.09.1869 – 08.10.1945), \emph{Schriftsteller/Schriftstellerin, Journalist/Journalistin, Chefredakteur/Chefredakteurin}|pwk} mit Ottilie Metzl\pwindex{Salten, Ottilie 07.03.1868 – 22.06.1942@\textsc{Salten, Ottilie} (07.03.1868 – 22.06.1942), \emph{Schauspieler/Schauspielerin}|pwk} an. Salten\pwindex{Salten, Felix 06.09.1869 – 08.10.1945@\textsc{Salten, Felix} (06.09.1869 – 08.10.1945), \emph{Schriftsteller/Schriftstellerin, Journalist/Journalistin, Chefredakteur/Chefredakteurin}|pwk} hatte also berufliche wie persönliche
                  Gründe, verärgert zu sein. Am 14. 12. 1896,
                  offenbar bereits im Besitz der neuen Ausgabe der \emph{Wiener Rundschau}\pwindex{Wiener Rundschau@\emph{Wiener Rundschau}|pwk}, ohrfeigte er Kraus\pwindex{Kraus, Karl 28.04.1874 – 12.06.1936@\textsc{Kraus, Karl} (28.04.1874 – 12.06.1936), \emph{Schriftsteller/Schriftstellerin, Publizist/Publizistin, Schriftsteller/Schriftstellerin}|pwk} im Kaffeehaus. Am 25. 2. 1897 wurde er wegen Beleidigung
                  zu 20 Gulden Bußgeld verurteilt.}}}\label{K_L03261-1} ist eine merkwürdige Wendung eingetreten,
               die uns den Herrn möglicherweise total ausliefert.\pend
           
\pstart
           Können Sie auf einen Sprung \label{K_L03261-2v}\edtext{zu mir
                  kommen}{\lemma{\textnormal{\emph{zu mir
                  kommen}}}\Cendnote{\textnormal{Salten\pwindex{Salten, Felix 06.09.1869 – 08.10.1945@\textsc{Salten, Felix} (06.09.1869 – 08.10.1945), \emph{Schriftsteller/Schriftstellerin, Journalist/Journalistin, Chefredakteur/Chefredakteurin}|pwk} besuchte abends{ }Schnitzler, vgl. A. S.: \emph{Tagebuch}, 6. 1. 1897.}}}\label{K_L03261-2}?\pend
           \pstart Herzl. \spacefill\mbox{Salten}\pend{}\selectlanguage{ngerman}\endnumbering\briefempfaengerindex{Schnitzler, Arthur@\textsc{Schnitzler, Arthur}!zzzSalten, Felix@\emph{von Felix Salten}!1897-01-061@{{[}6. 1. 1897{]}}|)be}\mylabel{L03261h}  \normalsize

\doendnotes{C}
\bigskip
\vfill

\clearpage

\footnotesize

\lohead{\textsc{register}}

% Definiere theindex-Environment komplett neu ohne reledmac
\makeatletter
\renewenvironment{theindex}{%
  \section*{\indexname}%
  \setlength{\parindent}{0pt}%
  \setlength{\parskip}{0pt plus 0.3pt}%
  \let\item\@idxitem
}{%
  \clearpage
}
\makeatother

\IfFileExists{\jobname-pw.ind}{\input{\jobname-pw.ind}}{}

\end{document}

      