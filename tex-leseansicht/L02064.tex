%% latex-leseansicht-vorspann.tex
%% Vorspann für die Leseansicht.
%% Lädt die gemeinsame Datei latex-vorspann.tex mit nicht gesetztem Schalter.

\newif\ifkorrekturansicht
\korrekturansichtfalse

\input{../tex-inputs/latex-vorspann}

\begin{center}
            \textcolor{red}{ENTWURF. ENTZIFFERUNG NOCH NICHT KORREKTURGELESEN}
                      \end{center}
            
               \section[Oscar Blumenthal an Arthur Schnitzler, 14. 5. 1912]{ Oscar Blumenthal an Arthur Schnitzler, 14. 5. 1912}\nopagebreak\mylabel{v}\rehead{ }\begin{ledgroupsized}[t]{13cm}\normalsize\beginnumbering\briefempfaengerindex{Schnitzler, Arthur@\textsc{Schnitzler, Arthur}!zzzBlumenthal, Oskar@\emph{von Oskar Blumenthal}!1912-05-141@{14. 5. 1912}|(be} \toendnotes[C]{\smallbreak\pagebreak[2]} \Standort{TMW, HS Schn 1/59/4.}
\physDesc{Brief, 1 Blatt, 2 Seiten
\newline{}Handschrift: schwarze Tinte, deutsche Kurrent
\newline{}Schnitzler: mit rotem Buntstift eine Unterstreichung und mit rotem Buntstift
            nummeriert: »1« }\toendnotes[C]{\smallbreak}\pstart
           {\pb}\textcolor{gray}{\textbf{D\textsuperscript{r} Oscar
                                Blumenthal}}\hfill \textcolor{gray}{\textbf{Berlin W. 15\oindex{Berlin@\textbf{Berlin}|pw} den}}{ }14 Mai 1912\pend
           \pstart
           \raggedleft{}\textcolor{gray}{\textbf{Kaiser-Allee 20}}\oindex{Bundesallee@\textbf{Bundesallee}|pw}.{\\}z. Z. Laufen bei Ischl\oindex{Lauffen@\textbf{Lauffen}|pw}\pend
           \pstart\center{}Werther Herr Doctor!\pend\pstart
           Dem vielſtimmigen Chor der Dankbarkeit und Verehrung, dem Sie \label{K_L02064_1v}\edtext{heute}{\lemma{\textnormal{\emph{heute}}}\Cendnote{\textnormal{Schnitzler\pwindex{Schnitzler, Arthur 15.05.1862 – 21.10.1931@\textsc{Schnitzler, Arthur} (15.05.1862 – 21.10.1931), \emph{Schriftsteller, Mediziner}|pwk}s 50. Geburtstag am
                            15. 5. 1912.}}}\label{K_L02064_1h} rettungslos ausgeliefert sind, bitte
                    ich einen Zuruf der herzlichſten Sympathie einfügen zu dürfen, die sich mir mit
                    jeder Ihrer neuen Schöpfungen gefeſtigt, vertieft und geſteigert hat. Vielleicht
                    empfinden Sie das Bedürfniß, sich von der frohen Bürde Ihres Lebensjubiläums
                    hier auszuruhen und geben mir dann Gelegenheit, Ihnen meine wärmſten Wünſche
                    mündlich zu wiederholen. Sie finden jetzt hier alles, was man zur Erholung
                    braucht: Frühlingswetter, Bergfrieden und – keine Menſchen. Herzlichſt Ihr\pend
           \pstart \spacefill\mbox{Osc. Blumenthal.}\pend{}\endnumbering\briefempfaengerindex{Schnitzler, Arthur@\textsc{Schnitzler, Arthur}!zzzBlumenthal, Oskar@\emph{von Oskar Blumenthal}!1912-05-141@{14. 5. 1912}|)be}\mylabel{h}\end{ledgroupsized}  \newcommand{\dateiname}{L02064}\newcommand{\titel}{Oscar Blumenthal an Arthur Schnitzler, 14. 5. 1912}\newcommand{\editorInnen}{Martin Anton Müller und Gerd-Hermann Susen}%% latex-leseansicht-abspann.tex
%% Abspann für die Leseansicht.
%% Der Schalter \ifkorrekturansicht ist bereits durch den Vorspann gesetzt.

%% latex-abspann.tex
%% Gemeinsamer Abspann für Korrekturansicht und Leseansicht.
%% Setzt den Schalter \ifkorrekturansicht voraus (gesetzt in den
%% einbindenden Dateien latex-korrekturansicht-abspann.tex bzw.
%% latex-leseansicht-abspann.tex).
%% ---------------------------------------------------------------

\normalsize

% Das esempio-Environment wird nur in der Leseansicht benötigt
\ifkorrekturansicht\else
\newenvironment{esempio}[3]%
{
    \vspace{1.5ex}
    \rlap{\underline{#1}}
    \par
    \setlength{\parindent}{0cm}
    \nopagebreak
    \leftskip=#2cm
    \rightskip=#3cm
}
{
    \par
}
\fi

\doendnotes{C}
\bigskip
\vfill

\clearpage

\footnotesize

\ifkorrekturansicht
  \lohead{\textsc{register}}
\fi

% theindex-Environment neu definieren ohne reledmac
\makeatletter
\renewenvironment{theindex}{%
  \ifkorrekturansicht
    \section*{\indexname}%
  \else
    \subsubsection*{Index der erwähnten Entitäten}%
  \fi
  \setlength{\parindent}{0pt}%
  \setlength{\parskip}{0pt plus 0.3pt}%
  \let\item\@idxitem
}{%
  \ifkorrekturansicht\clearpage\fi
}
\makeatother

\IfFileExists{\jobname-pw.ind}{\input{\jobname-pw.ind}}{}

% Quellenangabe nur in der Leseansicht
\ifkorrekturansicht\else
% Fallback-Definitionen, falls die .tex-Datei \titel etc. nicht gesetzt hat
\providecommand{\titel}{}
\providecommand{\editorInnen}{}
\providecommand{\dateiname}{\jobname}

\vspace{3cm}

\vfill

\footnotesize
\textsc{Quelle}: \titel. Herausgegeben von {\editorInnen}. In: \emph{Arthur Schnitzler: Briefwechsel mit Autorinnen und Autoren}.
 Digitale Edition, https://schnitzler-briefe.acdh.oeaw.ac.at/{\dateiname}.html (Stand \today)
\fi

\end{document}


      