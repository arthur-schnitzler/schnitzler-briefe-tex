%% latex-korrekturansicht-vorspann.tex
%% Vorspann für die Korrekturansicht.
%% Lädt die gemeinsame Datei latex-vorspann.tex mit gesetztem Schalter.

\newif\ifkorrekturansicht
\korrekturansichttrue

\input{../tex-inputs/latex-vorspann}


\section[Oscar Blumenthal an Arthur Schnitzler, 14. 5. 1912]{L02064 Oscar Blumenthal an Arthur Schnitzler, 14. 5. 1912}
\nopagebreak\mylabel{L02064v}
\rehead{ }\normalsize\beginnumbering\briefempfaengerindex{Schnitzler, Arthur@\textsc{Schnitzler, Arthur}!zzzBlumenthal, Oskar@\emph{von Oskar Blumenthal}!1912-05-141@{14. 5. 1912}|(be}
\toendnotes[C]{\smallbreak\pagebreak[2]}\Standort{TMW, HS Schn 1/59/4.}
\physDesc{Brief, 1 Blatt, 2 Seiten, 637 Zeichen
\newline{}Handschrift: schwarze Tinte, deutsche Kurrent
\newline{}Schnitzler: mit rotem Buntstift eine Unterstreichung und mit rotem Buntstift
                                 nummeriert: »1« }\toendnotes[C]{\smallbreak}
\pstart
           
\pstart
           {\pb}\textcolor{gray}{\textbf{D\textsuperscript{r} Oscar Blumenthal}}\pend
           
\pstart
           \raggedleft{}\textcolor{gray}{\textbf{Berlin W. 15\oindex{Berlin@\textbf{Berlin}, \emph{P.PPLC}|pw} den}}{ }14 Mai 1912\pend
           \pend
           
\pstart
           \raggedleft{}\textcolor{gray}{\textbf{Kaiser-Allee 20}}\oindex{Bundesallee@\textbf{Bundesallee}, \emph{Straße (K.STR)}|pw}.{\\}z. Z. Laufen bei Ischl\oindex{Lauffen@\textbf{Lauffen}, \emph{P.PPL}|pw}\pend
           
\pstart\center{}Werther Herr Doctor!\pend\vspace{0.5em}
\pstart
           Dem vielſtimmigen Chor der Dankbarkeit und Verehrung, dem Sie \label{K_L02064-1v}\edtext{heute}{\lemma{\textnormal{\emph{heute}}}\Cendnote{\textnormal{Am
               15. 5. 1912 beging Schnitzler
               seinen 50. Geburtstag.}}}\label{K_L02064-1} rettungslos ausgeliefert sind, bitte ich einen
               Zuruf der herzlichſten Sympathie einfügen zu dürfen, die sich mir mit jeder Ihrer
               neuen Schöpfungen gefeſtigt, vertieft und geſteigert hat. Vielleicht empfinden Sie
               das Bedürfniß, sich von der frohen Bürde Ihres Lebensjubiläums hier auszuruhen und
               geben mir dann Gelegenheit, Ihnen meine wärmſten Wünſche mündlich zu wiederholen. Sie
               finden jetzt hier alles, was man zur Erholung braucht: Frühlingswetter, Bergfrieden
               und – keine Menſchen. Herzlichſt Ihr\pend
           \pstart \spacefill\mbox{Osc. Blumenthal.}\pend{}\selectlanguage{ngerman}\endnumbering\briefempfaengerindex{Schnitzler, Arthur@\textsc{Schnitzler, Arthur}!zzzBlumenthal, Oskar@\emph{von Oskar Blumenthal}!1912-05-141@{14. 5. 1912}|)be}\mylabel{L02064h}  \normalsize

\doendnotes{C}
\bigskip
\vfill

\clearpage

\footnotesize

\lohead{\textsc{register}}

% Definiere theindex-Environment komplett neu ohne reledmac
\makeatletter
\renewenvironment{theindex}{%
  \section*{\indexname}%
  \setlength{\parindent}{0pt}%
  \setlength{\parskip}{0pt plus 0.3pt}%
  \let\item\@idxitem
}{%
  \clearpage
}
\makeatother

\IfFileExists{\jobname-pw.ind}{\input{\jobname-pw.ind}}{}

\end{document}

      