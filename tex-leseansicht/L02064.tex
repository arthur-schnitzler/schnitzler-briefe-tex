%% latex-leseansicht-vorspann.tex
%% Vorspann für die Leseansicht.
%% Lädt die gemeinsame Datei latex-vorspann.tex mit nicht gesetztem Schalter.

\newif\ifkorrekturansicht
\korrekturansichtfalse

\input{../tex-inputs/latex-vorspann}


\section[Oscar Blumenthal an Arthur Schnitzler, 14. 5. 1912]{L02064 Oscar Blumenthal an Arthur Schnitzler, 14. 5. 1912}
\nopagebreak\mylabel{L02064v}
\rehead{ }\normalsize\beginnumbering\briefempfaengerindex{Schnitzler, Arthur@\textsc{Schnitzler, Arthur}!zzzBlumenthal, Oskar@\emph{von Oskar Blumenthal}!1912-05-141@{14. 5. 1912}|(be}
\toendnotes[C]{\smallbreak\pagebreak[2]}
\correspDesc{Versand  durch Oscar Blumenthal am 14. 5. 1912 in Lauffen
\newline{}Erhalt  durch Arthur Schnitzler im Zeitraum [15. 5. 1912
                  – 19. 5. 1912?] in Wien}\toendnotes[C]{\smallbreak}
\Standort{TMW, HS Schn 1/59/4.}
\physDesc{Brief, 1 Blatt, 2 Seiten, 637 Zeichen
\newline{}Handschrift: schwarze Tinte, deutsche Kurrent
\newline{}Schnitzler: mit rotem Buntstift eine Unterstreichung und mit rotem Buntstift
                                 nummeriert: »1« }\toendnotes[C]{\smallbreak}
\pstart
           
\pstart
           {\pb}\textcolor{gray}{\textbf{D\textsuperscript{r} Oscar Blumenthal}}\pend
           
\pstart
           \raggedleft{}\textcolor{gray}{\textbf{Berlin W. 15\oindex{Berlin@\textbf{Berlin}, \emph{Hauptstadt}|pw} den}}{ }14 Mai 1912\pend
           \pend
           
\pstart
           \raggedleft{}\textcolor{gray}{\textbf{Kaiser-Allee 20}}\oindex{Bundesallee@\textbf{Bundesallee}, \emph{Straße}|pw}.{\\}z. Z. Laufen bei Ischl\oindex{Lauffen@\textbf{Lauffen}|pw}\pend
           
\pstart\center{}Werther Herr Doctor!\pend\vspace{0.5em}
\pstart
           Dem vielſtimmigen Chor der Dankbarkeit und Verehrung, dem Sie \label{K_L02064-1v}\edtext{heute}{\lemma{\textnormal{\emph{heute}}}\Cendnote{\textnormal{Am
               15. 5. 1912 beging Schnitzler
               seinen 50. Geburtstag.}}}\label{K_L02064-1} rettungslos ausgeliefert sind, bitte ich einen
               Zuruf der herzlichſten Sympathie einfügen zu dürfen, die sich mir mit jeder Ihrer
               neuen Schöpfungen gefeſtigt, vertieft und geſteigert hat. Vielleicht empfinden Sie
               das Bedürfniß, sich von der frohen Bürde Ihres Lebensjubiläums hier auszuruhen und
               geben mir dann Gelegenheit, Ihnen meine wärmſten Wünſche mündlich zu wiederholen. Sie
               finden jetzt hier alles, was man zur Erholung braucht: Frühlingswetter, Bergfrieden
               und – keine Menſchen. Herzlichſt Ihr\pend
           \pstart \spacefill\mbox{Osc. Blumenthal.}\pend{}\selectlanguage{ngerman}\endnumbering\briefempfaengerindex{Schnitzler, Arthur@\textsc{Schnitzler, Arthur}!zzzBlumenthal, Oskar@\emph{von Oskar Blumenthal}!1912-05-141@{14. 5. 1912}|)be}\mylabel{L02064h}  \newcommand{\dateiname}{L02064}\newcommand{\titel}{Oscar Blumenthal an Arthur Schnitzler, 14. 5. 1912}\newcommand{\editorInnen}{Martin Anton Müller und Gerd-Hermann Susen}%% latex-leseansicht-abspann.tex
%% Abspann für die Leseansicht.
%% Der Schalter \ifkorrekturansicht ist bereits durch den Vorspann gesetzt.

%% latex-abspann.tex
%% Gemeinsamer Abspann für Korrekturansicht und Leseansicht.
%% Setzt den Schalter \ifkorrekturansicht voraus (gesetzt in den
%% einbindenden Dateien latex-korrekturansicht-abspann.tex bzw.
%% latex-leseansicht-abspann.tex).
%% ---------------------------------------------------------------

\normalsize

% Das esempio-Environment wird nur in der Leseansicht benötigt
\ifkorrekturansicht\else
\newenvironment{esempio}[3]%
{
    \vspace{1.5ex}
    \rlap{\underline{#1}}
    \par
    \setlength{\parindent}{0cm}
    \nopagebreak
    \leftskip=#2cm
    \rightskip=#3cm
}
{
    \par
}
\fi

\doendnotes{C}
\bigskip
\vfill

\clearpage

\footnotesize

\ifkorrekturansicht
  \lohead{\textsc{register}}
\fi

% theindex-Environment neu definieren ohne reledmac
\makeatletter
\renewenvironment{theindex}{%
  \ifkorrekturansicht
    \section*{\indexname}%
  \else
    \subsubsection*{Index der erwähnten Entitäten}%
  \fi
  \setlength{\parindent}{0pt}%
  \setlength{\parskip}{0pt plus 0.3pt}%
  \let\item\@idxitem
}{%
  \ifkorrekturansicht\clearpage\fi
}
\makeatother

\IfFileExists{\jobname-pw.ind}{\input{\jobname-pw.ind}}{}

% Quellenangabe nur in der Leseansicht
\ifkorrekturansicht\else
% Fallback-Definitionen, falls die .tex-Datei \titel etc. nicht gesetzt hat
\providecommand{\titel}{}
\providecommand{\editorInnen}{}
\providecommand{\dateiname}{\jobname}

\vspace{3cm}

\vfill

\footnotesize
\textsc{Quelle}: \titel. Herausgegeben von {\editorInnen}. In: \emph{Arthur Schnitzler: Briefwechsel mit Autorinnen und Autoren}.
 Digitale Edition, https://schnitzler-briefe.acdh.oeaw.ac.at/{\dateiname}.html (Stand \today)
\fi

\end{document}


