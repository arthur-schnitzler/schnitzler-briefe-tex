%% latex-korrekturansicht-vorspann.tex
%% Vorspann für die Korrekturansicht.
%% Lädt die gemeinsame Datei latex-vorspann.tex mit gesetztem Schalter.

\newif\ifkorrekturansicht
\korrekturansichttrue

\input{../tex-inputs/latex-vorspann}


\section[ Felix Salten an Arthur Schnitzler, {[}28. 1. 1899{]}]{L03286 Felix Salten an Arthur Schnitzler, {[}28. 1. 1899{]}}
\nopagebreak\mylabel{L03286v}
\rehead{ }\normalsize\beginnumbering\briefempfaengerindex{Schnitzler, Arthur@\textsc{Schnitzler, Arthur}!zzzSalten, Felix@\emph{von Felix Salten}!1899-01-281@{{[}28. 1. 1899{]}}|(be}
\toendnotes[C]{\smallbreak\pagebreak[2]}\Standort{CUL, Schnitzler, B 89, A 2.}
\physDesc{Karte, 198 Zeichen
\newline{}Handschrift: schwarze Tinte, lateinische Kurrent
\newline{}Schnitzler: mit Bleistift datiert: »28/1 99« 
\newline{}Ordnung: mit Bleistift von unbekannter Hand nummeriert: »110« }\toendnotes[C]{\smallbreak}
\pstart
           \noindent{}{\pb}lieber Arthur, wenn Sie eine verfügbare halbe Stunde
               haben, lesen Sie, bitte, meine »\label{K_L03286-1v}\edtext{Literatur\pwindex{Franz Joseph I. und seine Zeit.« (Culturhistorischer Rueckblick auf die Francisco-Josephinische Epoche. – Unter dem Protectorate des Erzherzogs Franz Ferdinand, herausgegeben von J. Schnitzer. Wien, bei R. Lechner.)@\emph{»Franz Joseph I. und seine Zeit.« (Culturhistorischer Rückblick auf die Francisco-Josephinische Epoche. – Unter dem Protectorate des Erzherzogs Franz Ferdinand, herausgegeben von J. Schnitzer. Wien, bei R. Lechner.)}|pwv}}{\lemma{\textnormal{\emph{Literatur}}}\Cendnote{\textnormal{Wohl: –X. –\pwindex{Salten, Felix 06.09.1869 – 08.10.1945@\textsc{Salten, Felix} (06.09.1869 – 08.10.1945), \emph{Schriftsteller/Schriftstellerin, Journalist/Journalistin, Chefredakteur/Chefredakteurin}|pwk} [ = Felix Salten\pwindex{Salten, Felix 06.09.1869 – 08.10.1945@\textsc{Salten, Felix} (06.09.1869 – 08.10.1945), \emph{Schriftsteller/Schriftstellerin, Journalist/Journalistin, Chefredakteur/Chefredakteurin}|pwk}]: \emph{»Franz Joseph I. und
                        seine Zeit.« (Culturhistorischer Rückblick auf die Francisco-Josephinische
                        Epoche. – Unter dem Protectorate des Erzherzogs Franz Ferdinand,
                        herausgegeben von J. Schnitzer. Wien, bei R. Lechner.)}\pwindex{Franz Joseph I. und seine Zeit.« (Culturhistorischer Rueckblick auf die Francisco-Josephinische Epoche. – Unter dem Protectorate des Erzherzogs Franz Ferdinand, herausgegeben von J. Schnitzer. Wien, bei R. Lechner.)@\emph{»Franz Joseph I. und seine Zeit.« (Culturhistorischer Rückblick auf die Francisco-Josephinische Epoche. – Unter dem Protectorate des Erzherzogs Franz Ferdinand, herausgegeben von J. Schnitzer. Wien, bei R. Lechner.)}|pwk} In: \emph{Wiener Allgemeine Zeitung}\pwindex{Wiener Allgemeine Zeitung@\emph{Wiener Allgemeine Zeitung}|pwk}, Nr. 6272, 28. 1. 1899, S. 2.}}}\label{K_L03286-1}«. Ich bin heute{ }Abend
               im Schrangl\oindex{Cafe Pfob@\textbf{Café Pfob}, \emph{Kaffeehaus (K.KAF)}|pw}, und es ist mir natürlich sehr um Ihre
               Meinung zu thun.\pend
           
\pstart
           Herzlich Ihr {\\[\baselineskip]}\spacefill\mbox{Salten}\pend
           \leftskip=0em{}\selectlanguage{ngerman}\endnumbering\briefempfaengerindex{Schnitzler, Arthur@\textsc{Schnitzler, Arthur}!zzzSalten, Felix@\emph{von Felix Salten}!1899-01-281@{{[}28. 1. 1899{]}}|)be}\mylabel{L03286h}  \normalsize

\doendnotes{C}
\bigskip
\vfill

\clearpage

\footnotesize

\lohead{\textsc{register}}

% Definiere theindex-Environment komplett neu ohne reledmac
\makeatletter
\renewenvironment{theindex}{%
  \section*{\indexname}%
  \setlength{\parindent}{0pt}%
  \setlength{\parskip}{0pt plus 0.3pt}%
  \let\item\@idxitem
}{%
  \clearpage
}
\makeatother

\IfFileExists{\jobname-pw.ind}{\input{\jobname-pw.ind}}{}

\end{document}

      