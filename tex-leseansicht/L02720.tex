%% latex-korrekturansicht-vorspann.tex
%% Vorspann für die Korrekturansicht.
%% Lädt die gemeinsame Datei latex-vorspann.tex mit gesetztem Schalter.

\newif\ifkorrekturansicht
\korrekturansichttrue

\input{../tex-inputs/latex-vorspann}


\section[Paul Goldmann an Arthur Schnitzler, 28. 11. {[}1893{]}]{L02720 Paul Goldmann an Arthur Schnitzler, 28. 11. {[}1893{]}}
\nopagebreak\mylabel{L02720v}
\rehead{ }\normalsize\beginnumbering\briefempfaengerindex{Schnitzler, Arthur@\textsc{Schnitzler, Arthur}!zzzGoldmann, Paul@\emph{von Paul Goldmann}!1893-11-281@{28. 11. {[}1893{]}}|(be}
\toendnotes[C]{\smallbreak\pagebreak[2]}\Standort{DLA, A:Schnitzler, HS.NZ85.1.3163.}
\physDesc{Brief, 1 Blatt, 3 Seiten, 706 Zeichen
\newline{}Handschrift: schwarze Tinte, deutsche Kurrent
\newline{}Schnitzler: mit Bleistift das Jahr »93« vermerkt }\toendnotes[C]{\smallbreak}
\pstart
           \raggedleft{}{\pb}\textsc{Paris\oindex{Paris@\textbf{Paris}, \emph{P.PPLC}|pw}}, 28. November.\pend
           
\pstart\center{}Mein lieber Freund!\pend\vspace{0.5em}
\pstart
           Ich \label{K_L02720-1v}\edtext{freue mich}{\lemma{\textnormal{\emph{freue mich}}}\Cendnote{\textnormal{Goldmann\pwindex{Goldmann, Paul 31.01.1865 – 25.09.1935@\textsc{Goldmann, Paul} (31.01.1865 – 25.09.1935), \emph{Schriftsteller/Schriftstellerin, Journalist/Journalistin}|pwk} dürfte sich hier auf den
                  Probenbeginn für die Uraufführung des \emph{Märchens}\pwindex{Maerchen. Schauspiel in drei Aufzuegen@\emph{Das Märchen. Schauspiel in drei Aufzügen}|pwk} beziehen, der am 24. 11. 1893 stattfand.}}}\label{K_L02720-1} von Herzen und
               wünſche Dir ſo viel Glück, ſo viel Glück – ach\textcolor{gray}{’} es iſt ſchwer zu
               ſagen, wieviel Glück ich Dir wünſche. Wir ſind mitten in einer \label{K_L02720-2v}\edtext{Miniſterkriſis}{\lemma{\textnormal{\emph{Miniſterkriſis}}}\Cendnote{\textnormal{Innerhalb der \emph{französischen
                     Regierung}\orgindex{Franzoesische Regierung@Französische Regierung|pwk} herrschte Uneinigkeit, wie die Trennung zwischen Kirche und
                  Staat zu erreichen sei, weswegen das Kabinett\orgindex{Franzoesische Regierung@Französische Regierung|pwkv} personell umstrukturiert wurde.}}}\label{K_L02720-2}, und ich
               muß mir mit tauſend Liſten eine Minute ſtehlen, um Dir die Hand drücken zu können.
                  {\pb}Ich kann Dir all’ das nicht ſagen, was ich Dir
               ſagen möchte. Ich habe keine Zeit. Es iſt vielleicht auch beſſer ſo. Mit einem Worte:
               Es iſt erreicht, – und das iſt genug. Und \strikeout{\textcolor{gray}{×}\-\textcolor{gray}{×}\-\textcolor{gray}{×}{ }\textcolor{gray}{×}\-\textcolor{gray}{×}\-\textcolor{gray}{×}\-\textcolor{gray}{×}\-\textcolor{gray}{×}} nun eine Bitte: Am Tage nach der Aufführung\pwindex{Maerchen. Schauspiel in drei Aufzuegen@\emph{Das Märchen. Schauspiel in drei Aufzügen}|pwv}, ſo zeitig als Du
               kannſt, ſchickſt Du mir wohl ein Telegramm über Aufnahme durch Publicum und Preſſe?
               Und einen ausführlichen {\pb}\label{K_L02720-3v}\edtext{Brief hinterdrein}{\lemma{\textnormal{\emph{Brief hinterdrein}}}\Cendnote{\textnormal{Vgl. Paul Goldmann an Arthur Schnitzler, 5. 12. [1893].
               }}}\label{K_L02720-3}, nicht wahr?\pend
           
\pstart
           Alſo glückauf!!!\pend
           
\pstart
           Dein treuer {\\[\baselineskip]}\spacefill\mbox{Paul Goldm}\pend
           \leftskip=0em{}\selectlanguage{ngerman}\endnumbering\briefempfaengerindex{Schnitzler, Arthur@\textsc{Schnitzler, Arthur}!zzzGoldmann, Paul@\emph{von Paul Goldmann}!1893-11-281@{28. 11. {[}1893{]}}|)be}\mylabel{L02720h}  \normalsize

\doendnotes{C}
\bigskip
\vfill

\clearpage

\footnotesize

\lohead{\textsc{register}}

% Definiere theindex-Environment komplett neu ohne reledmac
\makeatletter
\renewenvironment{theindex}{%
  \section*{\indexname}%
  \setlength{\parindent}{0pt}%
  \setlength{\parskip}{0pt plus 0.3pt}%
  \let\item\@idxitem
}{%
  \clearpage
}
\makeatother

\IfFileExists{\jobname-pw.ind}{\input{\jobname-pw.ind}}{}

\end{document}

      