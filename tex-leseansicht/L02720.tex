%% latex-leseansicht-vorspann.tex
%% Vorspann für die Leseansicht.
%% Lädt die gemeinsame Datei latex-vorspann.tex mit nicht gesetztem Schalter.

\newif\ifkorrekturansicht
\korrekturansichtfalse

\input{../tex-inputs/latex-vorspann}


\section[Paul Goldmann an Arthur Schnitzler, 28. 11. {[}1893{]}]{L02720 Paul Goldmann an Arthur Schnitzler, 28. 11. [1893]}
\nopagebreak\mylabel{L02720v}
\rehead{ }\normalsize\beginnumbering\briefempfaengerindex{Schnitzler, Arthur@\textsc{Schnitzler, Arthur}!zzzGoldmann, Paul@\emph{von Paul Goldmann}!1893-11-282@{28. 11. [1893]}|(be}
\toendnotes[C]{\smallbreak\pagebreak[2]}
\correspDesc{Versand  durch Paul Goldmann am 28. 11. [1893] in Paris
\newline{}Erhalt  durch Arthur Schnitzler im Zeitraum [29. 11. 1893 – 3. 12. 1893?] in Wien}\toendnotes[C]{\smallbreak}
\Standort{DLA, A:Schnitzler, HS.NZ85.1.3163.}
\physDesc{Brief, 1 Blatt, 3 Seiten, 706 Zeichen
\newline{}Handschrift: schwarze Tinte, deutsche Kurrent
\newline{}Schnitzler: mit Bleistift das Jahr »93« vermerkt }\toendnotes[C]{\smallbreak}
\pstart
           \raggedleft{}{\pb}\textsc{Paris\oindex{Paris@\textbf{Paris}, \emph{Hauptstadt}|pw}}, 28. November.\pend
           
\pstart\center{}Mein lieber Freund!\pend\vspace{0.5em}
\pstart
           Ich \label{K_L02720-1v}\edtext{freue mich}{\lemma{\textnormal{\emph{freue mich}}}\Cendnote{\textnormal{Goldmann\pwindex{Goldmann, Paul 31.\,1.\,1865 Breslau – 25.\,9.\,1935 Wien@\textsc{Goldmann, Paul} (31.\,1.\,1865 Breslau – 25.\,9.\,1935 Wien), \emph{Schriftsteller, Journalist}|pwk} dürfte sich hier auf den
                  Probenbeginn\eventindex{Volkstheater@\textbf{Volkstheater}!1. Probe von Das Märchen, 24.11.1893@1. Probe von Das Märchen, 24.11.1893|pwkv} für die Uraufführung\eventindex{Volkstheater@\textbf{Volkstheater}!Uraufführung von Das Märchen, 1.12.1893@Uraufführung von Das Märchen, 1.12.1893|pwkv} des \emph{Märchens}\pwindex{Schnitzler, Arthur 15.\,5.\,1862 Wien – 21.\,10.\,1931 ebd.@\textsc{Schnitzler, Arthur} (15.\,5.\,1862 Wien – 21.\,10.\,1931 ebd.), \emph{Schriftsteller, Mediziner}!Märchen. Schauspiel in drei Aufzügen@\strich\emph{Das Märchen. Schauspiel in drei Aufzügen}|pwk} beziehen, der am 24. 11. 1893 stattfand.}}}\label{K_L02720-1} von Herzen und
               wünſche Dir{ }ſo viel Glück,{ }ſo viel Glück – ach\textcolor{gray}{’} es iſt{ }ſchwer zu{ }ſagen, wieviel Glück ich Dir wünſche. Wir{ }ſind mitten in einer \label{K_L02720-2v}\edtext{Miniſterkriſis}{\lemma{\textnormal{\emph{Ministerkrisis}}}\Cendnote{\textnormal{Innerhalb der \emph{französischen
                     Regierung}\orgindex{Französische Regierung@Französische Regierung|pwk} herrschte Uneinigkeit, wie die Trennung zwischen Kirche und
                  Staat zu erreichen sei, weswegen das Kabinett\orgindex{Französische Regierung@Französische Regierung|pwkv} personell umstrukturiert wurde.}}}\label{K_L02720-2}, und ich
               muß mir mit tauſend Liſten eine Minute{ }ſtehlen, um Dir die Hand drücken zu können.
                  {\pb}Ich kann Dir all’ das nicht{ }ſagen, was ich Dir{ }ſagen möchte. Ich habe keine Zeit. Es iſt vielleicht auch beſſer{ }ſo. Mit einem Worte:
               Es iſt erreicht, – und das iſt genug. Und \strikeout{\textcolor{gray}{×}\-\textcolor{gray}{×}\-\textcolor{gray}{×}{ }\textcolor{gray}{×}\-\textcolor{gray}{×}\-\textcolor{gray}{×}\-\textcolor{gray}{×}\-\textcolor{gray}{×}} nun eine Bitte: Am Tage nach der Aufführung\pwindex{Schnitzler, Arthur 15.\,5.\,1862 Wien – 21.\,10.\,1931 ebd.@\textsc{Schnitzler, Arthur} (15.\,5.\,1862 Wien – 21.\,10.\,1931 ebd.), \emph{Schriftsteller, Mediziner}!Märchen. Schauspiel in drei Aufzügen@\strich\emph{Das Märchen. Schauspiel in drei Aufzügen}|pwv},{ }ſo zeitig als Du
               kannſt,{ }ſchickſt Du mir wohl ein Telegramm über Aufnahme durch Publicum und Preſſe?
               Und einen ausführlichen {\pb}\label{K_L02720-3v}\edtext{Brief hinterdrein}{\lemma{\textnormal{\emph{Brief hinterdrein}}}\Cendnote{\textnormal{Vgl. XXXX Auszeichnungsfehler: Dokument L02721 nicht gefunden.
               }}}\label{K_L02720-3}, nicht wahr?\pend
           
\pstart
           Alſo glückauf!!!\pend
           
\pstart
           Dein treuer {\\[\baselineskip]}\spacefill\mbox{Paul Goldm}\pend
           \leftskip=0em{}\selectlanguage{ngerman}\endnumbering\briefempfaengerindex{Schnitzler, Arthur@\textsc{Schnitzler, Arthur}!zzzGoldmann, Paul@\emph{von Paul Goldmann}!1893-11-282@{28. 11. [1893]}|)be}\mylabel{L02720h}  \newcommand{\dateiname}{L02720}\newcommand{\titel}{Paul Goldmann an Arthur Schnitzler, 28. 11. [1893]}\newcommand{\editorInnen}{Martin Anton Müller und Laura Untner}%% latex-leseansicht-abspann.tex
%% Abspann für die Leseansicht.
%% Der Schalter \ifkorrekturansicht ist bereits durch den Vorspann gesetzt.

%% latex-abspann.tex
%% Gemeinsamer Abspann für Korrekturansicht und Leseansicht.
%% Setzt den Schalter \ifkorrekturansicht voraus (gesetzt in den
%% einbindenden Dateien latex-korrekturansicht-abspann.tex bzw.
%% latex-leseansicht-abspann.tex).
%% ---------------------------------------------------------------

\normalsize

% Das esempio-Environment wird nur in der Leseansicht benötigt
\ifkorrekturansicht\else
\newenvironment{esempio}[3]%
{
    \vspace{1.5ex}
    \rlap{\underline{#1}}
    \par
    \setlength{\parindent}{0cm}
    \nopagebreak
    \leftskip=#2cm
    \rightskip=#3cm
}
{
    \par
}
\fi

\doendnotes{C}
\bigskip
\vfill

\clearpage

\footnotesize

\ifkorrekturansicht
  \lohead{\textsc{register}}
\fi

% theindex-Environment neu definieren ohne reledmac
\makeatletter
\renewenvironment{theindex}{%
  \ifkorrekturansicht
    \section*{\indexname}%
  \else
    \subsubsection*{Index der erwähnten Entitäten}%
  \fi
  \setlength{\parindent}{0pt}%
  \setlength{\parskip}{0pt plus 0.3pt}%
  \let\item\@idxitem
}{%
  \ifkorrekturansicht\clearpage\fi
}
\makeatother

\IfFileExists{\jobname-pw.ind}{\input{\jobname-pw.ind}}{}

% Quellenangabe nur in der Leseansicht
\ifkorrekturansicht\else
% Fallback-Definitionen, falls die .tex-Datei \titel etc. nicht gesetzt hat
\providecommand{\titel}{}
\providecommand{\editorInnen}{}
\providecommand{\dateiname}{\jobname}

\vspace{3cm}

\vfill

\footnotesize
\textsc{Quelle}: \titel. Herausgegeben von {\editorInnen}. In: \emph{Arthur Schnitzler: Briefwechsel mit Autorinnen und Autoren}.
 Digitale Edition, https://schnitzler-briefe.acdh.oeaw.ac.at/{\dateiname}.html (Stand \today)
\fi

\end{document}


