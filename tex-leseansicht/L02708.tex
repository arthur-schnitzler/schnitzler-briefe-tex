%% latex-leseansicht-vorspann.tex
%% Vorspann für die Leseansicht.
%% Lädt die gemeinsame Datei latex-vorspann.tex mit nicht gesetztem Schalter.

\newif\ifkorrekturansicht
\korrekturansichtfalse

\input{../tex-inputs/latex-vorspann}

\begin{center}
            \textcolor{red}{ENTWURF, NICHT FERTIG KORRIGIERT}
                      \end{center}
            
               \section[Paul Goldmann an Arthur Schnitzler, 22. 5. {[}1893{]}]{ Paul Goldmann an Arthur Schnitzler, 22. 5. {[}1893{]}}\nopagebreak\mylabel{v}\rehead{ }\begin{ledgroupsized}[t]{13cm}\normalsize\beginnumbering\briefempfaengerindex{Schnitzler, Arthur@\textsc{Schnitzler, Arthur}!zzzGoldmann, Paul@\emph{von Paul Goldmann}!1893-05-221@{22. 5. {[}1893{]}}|(be} \toendnotes[C]{\smallbreak\pagebreak[2]} \Standort{DLA, A:Schnitzler, HS.NZ85.1.3163.}
\physDesc{Brief, 1 Blatt, 4 Seiten
\newline{}Handschrift: schwarze Tinte, deutsche Kurrent
\newline{}Schnitzler: 1) mit Bleistift das Jahr »93« vermerkt 2) mit rotem Buntstift drei Unterstreichungen}\toendnotes[C]{\smallbreak}\pstart
           \noindent{}{\pb}\textcolor{gray}{\textbf{\textbf{Frankfurter Zeitung\orgindex{Frankfurter Zeitung@Frankfurter Zeitung|pw}.}}}\pend
           \pstart
           \textcolor{gray}{\textbf{\textbf{(\begin{otherlanguage}{french}Gazette de Francfort\end{otherlanguage}\orgindex{Frankfurter Zeitung@Frankfurter Zeitung|pw}.)}}}\pend
           \pstart
           \textcolor{gray}{\textbf{\begin{otherlanguage}{french}Directeur\pwindex{Sonnemann, Leopold 1831-10-29 – 1909-10-30@\textsc{Sonnemann, Leopold} (1831-10-29 – 1909-10-30), \emph{Journalist, Herausgeber}|pwv}\end{otherlanguage}{ }\textbf{M. L. Sonnemann\pwindex{Sonnemann, Leopold 1831-10-29 – 1909-10-30@\textsc{Sonnemann, Leopold} (1831-10-29 – 1909-10-30), \emph{Journalist, Herausgeber}|pw}.}}}\hfill \textsc{Paris\oindex{Paris@\textbf{Paris}|pw}}, 22. Mai.\pend
           \pstart
           \begin{otherlanguage}{french}\textcolor{gray}{\textbf{Journal\pwindex{Frankfurter Zeitung1856 – 1943@\emph{Frankfurter Zeitung}|pw} politique, financier,}}\end{otherlanguage}\pend
           \pstart
           \begin{otherlanguage}{french}\textcolor{gray}{\textbf{commercial et litteraire.}}\end{otherlanguage}\pend
           \pstart
           \begin{otherlanguage}{french}\textcolor{gray}{\textbf{\textbf{Paraissant trois fois par jour}}}\end{otherlanguage}\pend
           \pstart
           \begin{otherlanguage}{french}\textcolor{gray}{\textbf{\textbf{Bureaux à Paris\oindex{Paris@\textbf{Paris}|pw}:}}}\end{otherlanguage}\pend
           \pstart
           \begin{otherlanguage}{french}\textcolor{gray}{\textbf{\textbf{rue Richelieu 75\oindex{rue Richelieu@\textbf{rue Richelieu}|pw}.}}}\end{otherlanguage}\pend
           \pstart
           Mein lieber Arthur!\pend
           \pstart
           Dein lieber Brief, für den ich Dir herzlichſt danke, hat mich im Weſentlichen
               beruhigt. Die Hauptſache iſt, daß Dir die niedrigen \label{K_L02708-1v}\edtext{Brodſorgen fern bleiben}{\lemma{\textnormal{\emph{Brodſorgen fern bleiben}}}\Cendnote{\textnormal{Schnitzler\pwindex{Schnitzler, Arthur 15.05.1862 – 21.10.1931@\textsc{Schnitzler, Arthur} (15.05.1862 – 21.10.1931), \emph{Schriftsteller, Mediziner}|pwk}s Anteil am Erbe seines Vater\pwindex{Schnitzler, Johann 10.04.1835 – 02.05.1893@\textsc{Schnitzler, Johann} (10.04.1835 – 02.05.1893), \emph{Laryngologe}|pwkv}s ermöglichte ihm
                  einige Zeit finanzielle Sicherheit.}}}\label{K_L02708-1h}. Alles übrige Weh’, das ich tief
               beklage, ſoweit es Dich als Menſchen betrifft, wird Dir vielleicht doch zum \label{T_L02708-1v}\edtext{Z{[}ie{]}le}{\lemma{\textnormal{\emph{Ziele}}}\Cendnote{\textnormal{Goldmann\pwindex{Goldmann, Paul 31.01.1865 – 25.09.1935@\textsc{Goldmann, Paul} (31.01.1865 – 25.09.1935), \emph{Schriftsteller, Journalist}|pwk} schreibt
                  »Zeile«}}}\label{T_L02708-1h} ſein. Und mit jenem künſtleriſchen Egoismus, der
               Alles unter dem Geſichtspunkte ſeiner eigenſten Zwecke ſieht, denke ich mir, daß ein
               wenig Härtung und Hämmerung von Seiten des Lebens Deiner ſchönen Begabung gar
               herrlich zuſtatten kommen wird. Auch \textsc{Herzl\pwindex{Herzl, Theodor 02.05.1860 – 03.07.1904@\textsc{Herzl, Theodor} (02.05.1860 – 03.07.1904), \emph{Schriftsteller, Journalist}|pw}{ }}{\pb}iſt dierſer Anficht, der Dich jetzt \label{K_L02708-2v}\edtext{zu lieben und zu verſtehen begonnen}{\lemma{\textnormal{\emph{zu … begonnen}}}\Cendnote{\textnormal{Schnitzler\pwindex{Schnitzler, Arthur 15.05.1862 – 21.10.1931@\textsc{Schnitzler, Arthur} (15.05.1862 – 21.10.1931), \emph{Schriftsteller, Mediziner}|pwk} und Theodor Herzl\pwindex{Herzl, Theodor 02.05.1860 – 03.07.1904@\textsc{Herzl, Theodor} (02.05.1860 – 03.07.1904), \emph{Schriftsteller, Journalist}|pwk} korrespondierten korrespondierten zwischen
                     Mai und September{ }1893 auch häufig miteinander. Siehe Theodor Herzl\pwindex{Herzl, Theodor 02.05.1860 – 03.07.1904@\textsc{Herzl, Theodor} (02.05.1860 – 03.07.1904), \emph{Schriftsteller, Journalist}|pwk}. Briefe und Tagebücher.
                     Hg. v. Alex Bein, Hermann Greive, Moshe Schaerf und Julius H. Schoeps. Bd. 1.:
                        Theodor Herzl\pwindex{Herzl, Theodor 02.05.1860 – 03.07.1904@\textsc{Herzl, Theodor} (02.05.1860 – 03.07.1904), \emph{Schriftsteller, Journalist}|pwk}. Briefe und
                     autobiographische Notizen. 1866–1895. Bearb. v. Johannes Wachten. In Zusammenarbeit m. Chaya Harel,
                     Daisy Tycho und Manfred Winkler. Berlin\oindex{Berlin@\textbf{Berlin}|pwk}/Frankfurt a. M.\oindex{Frankfurt am Main@\textbf{Frankfurt am Main}|pwk}/Wien\oindex{Wien@\textbf{Wien}|pwk}: \emph{Ullstein}\orgindex{Ullstein Verlag@Ullstein Verlag|pwk}/\emph{Propyläen}\orgindex{Propylaeen Verlag@Propyläen Verlag|pwk} 1983,
                     S. 526–541.}}}\label{K_L02708-2h} hat und mit dem ich oft über Dich ſpreche. Hier und da erfahre ich auf dieſem
               Wege etwas über Dein Ergehen, wenn er einen Brief von Dir bekommen hat. Und dann
               denke ich mir: »Der hat aber ein Glück.« Auch \textsc{Isidor Fuchs\pwindex{Fuchs, Isidor 25.09.1849 – 1920.09@\textsc{Fuchs, Isidor} (25.09.1849 – 1920.09), \emph{Schriftsteller/Schriftstellerin}|pw}} hat mir viel über Wien\oindex{Wien@\textbf{Wien}|pw} erzählt. Und ſo \strikeout{h\textcolor{gray}{a}} bin ich denn durch fleißiges \strikeout{Euch} Betreiben
               dieſes Nachrichtendienſtes ein wenig auf dem Laufenden der Veränderungen, die ſich in
               den äußeren Wien\oindex{Wien@\textbf{Wien}|pw}er Dingen vollzogen, und weiß vor
               allen Dingen von Deinen Erfolgen, die mich mit wahrer Freude {\pb}erfüllt. Immerhin gibt es in meinem Wiſſen gewaltige
               Lücken. Und wenn Du mir nur ein wenig Näheres über die inneren Dinge ſchreiben
               könnteſt – über die Natur der \label{K_L02708-3v}\edtext{Unfälle}{\lemma{\textnormal{\emph{Unfälle}}}\Cendnote{\textnormal{Nicht nur mit dem Tod des
                     Vater\pwindex{Schnitzler, Johann 10.04.1835 – 02.05.1893@\textsc{Schnitzler, Johann} (10.04.1835 – 02.05.1893), \emph{Laryngologe}|pwkv}s am 2. 5. 1893 hatte Schnitzler\pwindex{Schnitzler, Arthur 15.05.1862 – 21.10.1931@\textsc{Schnitzler, Arthur} (15.05.1862 – 21.10.1931), \emph{Schriftsteller, Mediziner}|pwk} seit Anfang des Jahres 1893 zu kämpfen, auch sein Liebesleben gestaltete sich
                  unverhofft schwierig, erhielt er doch am 28. 1. 1893 erste Hinweise auf Marie Glümer\pwindex{Gluemer, Marie 03.07.1867 – 16.11.1925@\textsc{Glümer, Marie} (03.07.1867 – 16.11.1925), \emph{Schauspielerin}|pwk}s Untreue.}}}\label{K_L02708-3h}, die Dich
               betroffen, über Stimmungen und Pläne – ein wenig, ein ganz klein wenig, damit ich
               wieder Dein liebes Bild etwas klarer vor Augen habe und damit ich nicht blos auf die
               Erinnerungen angewieſen bin, um es mir zu verdeutlichen, – ſo wäre ich Dir recht ſehr
               dankbar.\pend
           \pstart
           Auch ein Paar Nachrichten über die Freunde\pwindex{Beer-Hofmann, Richard 11.07.1866 – 26.09.1945@\textsc{Beer-Hofmann, Richard} (11.07.1866 – 26.09.1945), \emph{Schriftsteller}|pwv}\pwindex{Hofmannsthal, Hugo von 01.02.1874 – 15.07.1929@\textsc{Hofmannsthal, Hugo von} (01.02.1874 – 15.07.1929), \emph{Schriftsteller}|pwv}, von denen ich kein Wort mehr weiß, über
                  \textsc{Richard\pwindex{Beer-Hofmann, Richard 11.07.1866 – 26.09.1945@\textsc{Beer-Hofmann, Richard} (11.07.1866 – 26.09.1945), \emph{Schriftsteller}|pw}} und {\pb}\textsc{Loris\pwindex{Hofmannsthal, Hugo von 01.02.1874 – 15.07.1929@\textsc{Hofmannsthal, Hugo von} (01.02.1874 – 15.07.1929), \emph{Schriftsteller}|pw}}, würden mir hochwillkommen ſein, ſowie über dieſen \label{K_L02708-4v}\edtext{Tauſendkünſtler\pwindex{Bahr, Hermann 19.07.1863 – 15.01.1934@\textsc{Bahr, Hermann} (19.07.1863 – 15.01.1934), \emph{Schriftsteller, Kritiker}|pwv}}{\lemma{\textnormal{\emph{Tauſendkünſtler}}}\Cendnote{\textnormal{Anspielung auf Hermann Bahr\pwindex{Bahr, Hermann 19.07.1863 – 15.01.1934@\textsc{Bahr, Hermann} (19.07.1863 – 15.01.1934), \emph{Schriftsteller, Kritiker}|pwk}s vielseitige journalistische und literarische
                  Betätigung}}}\label{K_L02708-4h}{ }\textsc{Hermann Bahr\pwindex{Bahr, Hermann 19.07.1863 – 15.01.1934@\textsc{Bahr, Hermann} (19.07.1863 – 15.01.1934), \emph{Schriftsteller, Kritiker}|pw}}, der \strikeout{\textcolor{gray}{×}} es alſo doch fertig gebracht zu haben ſcheint, in Wien\oindex{Wien@\textbf{Wien}|pw}{ }\textsc{Carrière} zu machen, worum ich ihn aufrichtig beneide.\pend
           \pstart
           Daran, Dir meine Dienſte in den ſchwierigen Zeiten, die Du jetzt durch machſt,
               anzubieten, habe ich \strikeout{\textcolor{gray}{×}} gedacht, aber ich habe \strikeout{mich} auch gemeint, daß
               Du mich leider kaum wirſt brauchen können. Iſt Dir aber doch zu etwas eine
               bedingungsloſe Ergebenheit nützlich, ſo denke daran, daß es für mich keine größere
               Freude geben könnte, als ſie Dir zu beweiſen.\pend
           \pstart In Treue Dein \spacefill\mbox{Paul Goldm}\pend{}          \endnumbering\briefempfaengerindex{Schnitzler, Arthur@\textsc{Schnitzler, Arthur}!zzzGoldmann, Paul@\emph{von Paul Goldmann}!1893-05-221@{22. 5. {[}1893{]}}|)be}\mylabel{h}\end{ledgroupsized}\begin{anhang}\end{anhang}\newcommand{\dateiname}{L02708}\newcommand{\titel}{Paul Goldmann an Arthur Schnitzler, 22. 5. [1893]}\newcommand{\editorInnen}{Martin Anton Müller und Laura Untner}
            \footnotesize
\begin{ledgroupsized}[t]{11.5cm}
\doendnotes{C}
\end{ledgroupsized}
         %% latex-leseansicht-abspann.tex
%% Abspann für die Leseansicht.
%% Der Schalter \ifkorrekturansicht ist bereits durch den Vorspann gesetzt.

%% latex-abspann.tex
%% Gemeinsamer Abspann für Korrekturansicht und Leseansicht.
%% Setzt den Schalter \ifkorrekturansicht voraus (gesetzt in den
%% einbindenden Dateien latex-korrekturansicht-abspann.tex bzw.
%% latex-leseansicht-abspann.tex).
%% ---------------------------------------------------------------

\normalsize

% Das esempio-Environment wird nur in der Leseansicht benötigt
\ifkorrekturansicht\else
\newenvironment{esempio}[3]%
{
    \vspace{1.5ex}
    \rlap{\underline{#1}}
    \par
    \setlength{\parindent}{0cm}
    \nopagebreak
    \leftskip=#2cm
    \rightskip=#3cm
}
{
    \par
}
\fi

\doendnotes{C}
\bigskip
\vfill

\clearpage

\footnotesize

\ifkorrekturansicht
  \lohead{\textsc{register}}
\fi

% theindex-Environment neu definieren ohne reledmac
\makeatletter
\renewenvironment{theindex}{%
  \ifkorrekturansicht
    \section*{\indexname}%
  \else
    \subsubsection*{Index der erwähnten Entitäten}%
  \fi
  \setlength{\parindent}{0pt}%
  \setlength{\parskip}{0pt plus 0.3pt}%
  \let\item\@idxitem
}{%
  \ifkorrekturansicht\clearpage\fi
}
\makeatother

\IfFileExists{\jobname-pw.ind}{\input{\jobname-pw.ind}}{}

% Quellenangabe nur in der Leseansicht
\ifkorrekturansicht\else
% Fallback-Definitionen, falls die .tex-Datei \titel etc. nicht gesetzt hat
\providecommand{\titel}{}
\providecommand{\editorInnen}{}
\providecommand{\dateiname}{\jobname}

\vspace{3cm}

\vfill

\footnotesize
\textsc{Quelle}: \titel. Herausgegeben von {\editorInnen}. In: \emph{Arthur Schnitzler: Briefwechsel mit Autorinnen und Autoren}.
 Digitale Edition, https://schnitzler-briefe.acdh.oeaw.ac.at/{\dateiname}.html (Stand \today)
\fi

\end{document}


      