%% latex-leseansicht-vorspann.tex
%% Vorspann für die Leseansicht.
%% Lädt die gemeinsame Datei latex-vorspann.tex mit nicht gesetztem Schalter.

\newif\ifkorrekturansicht
\korrekturansichtfalse

\input{../tex-inputs/latex-vorspann}


\section[Paul Goldmann an Arthur Schnitzler, 17. 5. [1893]]{L02708 Paul Goldmann an Arthur Schnitzler, 17. 5. [1893]}
\nopagebreak\mylabel{L02708v}
\rehead{ }\normalsize\beginnumbering\briefempfaengerindex{Schnitzler, Arthur@\textsc{Schnitzler, Arthur}!zzzGoldmann, Paul@\emph{von Paul Goldmann}!1893-05-171@{17. 5. [1893]}|(be}
\toendnotes[C]{\smallbreak\pagebreak[2]}
\correspDesc{Versand  durch Paul Goldmann am 17. 5. [1893] in Paris
\newline{}Erhalt  durch Arthur Schnitzler im Zeitraum [18. 5. 1893
                  – 22. 5. 1893?] in Wien}\toendnotes[C]{\smallbreak}
\Standort{DLA, A:Schnitzler, HS.NZ85.1.3163.}
\physDesc{Brief, 1 Blatt, 4 Seiten, 2209 Zeichen
\newline{}Handschrift: schwarze Tinte, deutsche Kurrent
\newline{}Schnitzler: 1) mit Bleistift das Jahr »93« vermerkt  2) mit rotem Buntstift drei Unterstreichungen}\toendnotes[C]{\smallbreak}
\pstart
           {\pb}\textcolor{gray}{\textbf{\textbf{Frankfurter Zeitung\orgindex{Frankfurter Zeitung@Frankfurter Zeitung|pw}.}}}\pend
           
\pstart
           \textcolor{gray}{\textbf{\textbf{(\begin{otherlanguage}{french}Gazette de Francfort\end{otherlanguage}\orgindex{Frankfurter Zeitung@Frankfurter Zeitung|pw}.)}}}\pend
           
\pstart
           \textcolor{gray}{\textbf{\begin{otherlanguage}{french}Directeur\end{otherlanguage}{ }\textbf{M. L. Sonnemann\pwindex{Sonnemann, Leopold 29.\,10.\,1831 Höchberg – 30.\,10.\,1909 Frankfurt am Main@\textsc{Sonnemann, Leopold} (29.\,10.\,1831 Höchberg – 30.\,10.\,1909 Frankfurt am Main), \emph{Journalist, Herausgeber}|pw}.}}}\hfill \textsc{Paris\oindex{Paris@\textbf{Paris}, \emph{Hauptstadt}|pw}}, 17. Mai.\pend
           
\pstart
           \begin{otherlanguage}{french}\textcolor{gray}{\textbf{Journal politique, financier,}}\end{otherlanguage}\pend
           
\pstart
           \begin{otherlanguage}{french}\textcolor{gray}{\textbf{commercial et litteraire.}}\end{otherlanguage}\pend
           
\pstart
           \begin{otherlanguage}{french}\textcolor{gray}{\textbf{\textbf{Paraissant trois fois par jour}}}\end{otherlanguage}\pend
           
\pstart
           \begin{otherlanguage}{french}\textcolor{gray}{\textbf{\textbf{Bureaux à Paris\oindex{Paris@\textbf{Paris}, \emph{Hauptstadt}|pw}:}}}\end{otherlanguage}\pend
           
\pstart
           \begin{otherlanguage}{french}\textcolor{gray}{\textbf{\textbf{rue Richelieu 75\oindex{rue Richelieu@\textbf{rue Richelieu}, \emph{Straße}|pw}.}}}\end{otherlanguage}\pend
           
\pstart\center{}Mein lieber Arthur!\pend\vspace{0.5em}
\pstart
           Dein lieber Brief, für den ich Dir herzlichſt danke, hat mich im Weſentlichen
               beruhigt. Die Hauptſache iſt, daß Dir die niedrigen \label{K_L02708-1v}\edtext{Brodſorgen fern bleiben}{\lemma{\textnormal{\emph{Brodsorgen fern bleiben}}}\Cendnote{\textnormal{Schnitzlers Anteil am Erbe seines Vaters\pwindex{Schnitzler, Johann 10.\,4.\,1835 Nagykanizsa – 2.\,5.\,1893 Wien@\textsc{Schnitzler, Johann} (10.\,4.\,1835 Nagykanizsa – 2.\,5.\,1893 Wien), \emph{Laryngologe}|pwkv} ermöglichte ihm
                  einige Zeit finanzielle Sicherheit.}}}\label{K_L02708-1}. Alles übrige Weh’, das ich tief
               beklage,{ }ſoweit es Dich als Menſchen betrifft, wird Dir vielleicht doch zum Heile{ }ſein. Und mit jenem künſtleriſchen Egoismus, der Alles unter dem Geſichtspunkte{ }ſeiner eigenſten Zwecke{ }ſieht, denke ich mir, daß ein wenig Härtung und Hämmerung von
               Seiten des Lebens Deiner{ }ſchönen Begabung gar herrlich zuſtatten kommen wird. Auch
                  \textsc{Herzl\pwindex{Herzl, Theodor 2.\,5.\,1860 Budapest – 3.\,7.\,1904 Edlach@\textsc{Herzl, Theodor} (2.\,5.\,1860 Budapest – 3.\,7.\,1904 Edlach), \emph{Schriftsteller, Journalist}|pw}{ }}{\pb}iſt dieſer Anficht, der Dich jetzt \label{K_L02708-2v}\edtext{zu lieben und zu verſtehen begonnen}{\lemma{\textnormal{\emph{zu … begonnen}}}\Cendnote{\textnormal{Nachdem Theodor Herzl\pwindex{Herzl, Theodor 2.\,5.\,1860 Budapest – 3.\,7.\,1904 Edlach@\textsc{Herzl, Theodor} (2.\,5.\,1860 Budapest – 3.\,7.\,1904 Edlach), \emph{Schriftsteller, Journalist}|pwk} am 4. 5. 1893{ }Schnitzler anlässlich des Ablebens des Vaters\pwindex{Schnitzler, Johann 10.\,4.\,1835 Nagykanizsa – 2.\,5.\,1893 Wien@\textsc{Schnitzler, Johann} (10.\,4.\,1835 Nagykanizsa – 2.\,5.\,1893 Wien), \emph{Laryngologe}|pwkv} kondoliert hatte,
                  antwortete Schnitzler am
                     11. 5. 1893, worauf Herzl\pwindex{Herzl, Theodor 2.\,5.\,1860 Budapest – 3.\,7.\,1904 Edlach@\textsc{Herzl, Theodor} (2.\,5.\,1860 Budapest – 3.\,7.\,1904 Edlach), \emph{Schriftsteller, Journalist}|pwk}
                  zwei Tage später replizierte. Danach dürfte ein Brief Schnitzlers verloren gegangen sein, jedenfalls gab es am
                     19. 5. 1893 neuerlich ein Schreiben Herzls\pwindex{Herzl, Theodor 2.\,5.\,1860 Budapest – 3.\,7.\,1904 Edlach@\textsc{Herzl, Theodor} (2.\,5.\,1860 Budapest – 3.\,7.\,1904 Edlach), \emph{Schriftsteller, Journalist}|pwk}. Vgl. Theodor Herzl\pwindex{Herzl, Theodor 2.\,5.\,1860 Budapest – 3.\,7.\,1904 Edlach@\textsc{Herzl, Theodor} (2.\,5.\,1860 Budapest – 3.\,7.\,1904 Edlach), \emph{Schriftsteller, Journalist}|pwk}: \emph{Briefe und Tagebücher}.
                     Herausgegeben von Alex Bein, Hermann Greive, Moshe Schaerf und Julius
                     H. Schoeps. Bd. 1: \emph{Briefe
                     und autobiographische Notizen}. 1866–1895. Bearbeitet von Johannes Wachten. In Zusammenarbeit mit Chaya
                     Harel, Daisy Tycho und Manfred Winkler.
                     Berlin/Frankfurt am Main/Wien:
                        \emph{Ullstein}/\emph{Propyläen}{ }1983, S. 526–541. }}}\label{K_L02708-2} hat und mit dem ich oft über Dich{ }ſpreche. Hier und da erfahre ich auf dieſem Wege etwas über Dein Ergehen, wenn er
               einen Brief von Dir bekommen hat. Und dann denke ich mir: »Der hat aber ein Glück.«
               Auch \textsc{Isidor Fuchs\pwindex{Fuchs, Isidor 25.\,9.\,1849 Lipnik Górny – um den 20.8.1920 Schruns@\textsc{Fuchs, Isidor} (25.\,9.\,1849 Lipnik Górny – um den 20.8.1920 Schruns), \emph{Schriftsteller, Journalist}|pw}} hat mir viel über Wien\oindex{Wien@\textbf{Wien}, \emph{Verwaltungsgebiet}|pw} erzählt. Und{ }ſo \strikeout{h\textcolor{gray}{a}} bin ich denn durch fleißiges \strikeout{E\textcolor{gray}{r}h} Betreiben dieſes Nachrichtendienſtes ein wenig auf
               dem Laufenden der Veränderungen, die{ }ſich in den äußeren Wien\oindex{Wien@\textbf{Wien}, \emph{Verwaltungsgebiet}|pw}er Dingen vollzogen, und weiß vor allen Dingen von Deinen
               Erfolgen, die mich mit wahrer Freude {\pb}erfüllt.
               Immerhin gibt es in meinem Wiſſen gewaltige Lücken. Und wenn Du mir nur ein wenig
               Näheres über die inneren Dinge{ }ſchreiben könnteſt – über die Natur der \label{K_L02708-3v}\edtext{Unfälle}{\lemma{\textnormal{\emph{Unfälle}}}\Cendnote{\textnormal{Das Liebesleben Schnitzlers gestaltete sich seit Jahresanfang unerwartet
                  schwierig, hatte er doch am 28. 1. 1893 erste Hinweise auf Marie Glümers\pwindex{Glümer, Marie 3.\,7.\,1867 Wien – 16.\,11.\,1925 München@\textsc{Glümer, Marie} (3.\,7.\,1867 Wien – 16.\,11.\,1925 München), \emph{Schauspielerin}|pwk} Untreue erhalten.}}}\label{K_L02708-3}, die Dich betroffen, über Stimmungen und
               Pläne – ein wenig, ein ganz klein wenig, damit ich wieder Dein liebes Bild etwas
               klarer vor Augen habe und damit ich nicht blos auf die Erinnerungen angewieſen bin,
               um es mir zu verdeutlichen, –{ }ſo wäre ich Dir recht{ }ſehr dankbar.\pend
           
\pstart
           Auch ein paar Nachrichten über die Freunde\pwindex{Beer-Hofmann, Richard 11.\,7.\,1866 Wien – 26.\,9.\,1945 New York City@\textsc{Beer-Hofmann, Richard} (11.\,7.\,1866 Wien – 26.\,9.\,1945 New York City), \emph{Schriftsteller}|pwv}\pwindex{Hofmannsthal, Hugo von 1.\,2.\,1874 Wien – 15.\,7.\,1929 Rodaun@\textsc{Hofmannsthal, Hugo von} (1.\,2.\,1874 Wien – 15.\,7.\,1929 Rodaun), \emph{Schriftsteller}|pwv}, von denen ich kein Wort mehr weiß, über
                  \textsc{Richard\pwindex{Beer-Hofmann, Richard 11.\,7.\,1866 Wien – 26.\,9.\,1945 New York City@\textsc{Beer-Hofmann, Richard} (11.\,7.\,1866 Wien – 26.\,9.\,1945 New York City), \emph{Schriftsteller}|pw}} und {\pb}\textsc{Loris\pwindex{Hofmannsthal, Hugo von 1.\,2.\,1874 Wien – 15.\,7.\,1929 Rodaun@\textsc{Hofmannsthal, Hugo von} (1.\,2.\,1874 Wien – 15.\,7.\,1929 Rodaun), \emph{Schriftsteller}|pw}}, würden mir hochwillkommen{ }ſein,{ }ſowie über dieſen \label{K_L02708-4v}\edtext{Tauſendkünſtler\pwindex{Bahr, Hermann 19.\,7.\,1863 Linz – 15.\,1.\,1934 München@\textsc{Bahr, Hermann} (19.\,7.\,1863 Linz – 15.\,1.\,1934 München), \emph{Schriftsteller, Kritiker}|pwv}}{\lemma{\textnormal{\emph{Tausendkünstler}}}\Cendnote{\textnormal{Anspielung auf Hermann Bahrs\pwindex{Bahr, Hermann 19.\,7.\,1863 Linz – 15.\,1.\,1934 München@\textsc{Bahr, Hermann} (19.\,7.\,1863 Linz – 15.\,1.\,1934 München), \emph{Schriftsteller, Kritiker}|pwk} vielseitige journalistische und literarische
                  Betätigung}}}\label{K_L02708-4}{ }\textsc{Hermann Bahr\pwindex{Bahr, Hermann 19.\,7.\,1863 Linz – 15.\,1.\,1934 München@\textsc{Bahr, Hermann} (19.\,7.\,1863 Linz – 15.\,1.\,1934 München), \emph{Schriftsteller, Kritiker}|pw}}, der \strikeout{\textcolor{gray}{×}} es alſo doch fertig gebracht zu haben{ }ſcheint, in Wien\oindex{Wien@\textbf{Wien}, \emph{Verwaltungsgebiet}|pw}{ }\textsc{Carrière} zu machen, worum ich ihn aufrichtig beneide.\pend
           
\pstart
           Daran, Dir meine Dienſte in den{ }ſchwierigen Zeiten, die Du jetzt durchmachſt,
               anzubieten, habe ich \strikeout{\textcolor{gray}{×}} gedacht, aber ich habe \strikeout{mich} auch gemeint, daß
               Du mich leider kaum wirſt brauchen können. Iſt Dir aber doch zu etwas eine
               bedingungsloſe Ergebenheit nützlich,{ }ſo denke daran, daß es für mich keine größere
               Freude geben könnte, als{ }ſie Dir zu beweiſen.\pend
           \pstart In Treue Dein \spacefill\mbox{Paul Goldm}\pend{}\selectlanguage{ngerman}\endnumbering\briefempfaengerindex{Schnitzler, Arthur@\textsc{Schnitzler, Arthur}!zzzGoldmann, Paul@\emph{von Paul Goldmann}!1893-05-171@{17. 5. [1893]}|)be}\mylabel{L02708h}  \newcommand{\dateiname}{L02708}\newcommand{\titel}{Paul Goldmann an Arthur Schnitzler, 17. 5. [1893]}\newcommand{\editorInnen}{Martin Anton Müller und Laura Untner}%% latex-leseansicht-abspann.tex
%% Abspann für die Leseansicht.
%% Der Schalter \ifkorrekturansicht ist bereits durch den Vorspann gesetzt.

%% latex-abspann.tex
%% Gemeinsamer Abspann für Korrekturansicht und Leseansicht.
%% Setzt den Schalter \ifkorrekturansicht voraus (gesetzt in den
%% einbindenden Dateien latex-korrekturansicht-abspann.tex bzw.
%% latex-leseansicht-abspann.tex).
%% ---------------------------------------------------------------

\normalsize

% Das esempio-Environment wird nur in der Leseansicht benötigt
\ifkorrekturansicht\else
\newenvironment{esempio}[3]%
{
    \vspace{1.5ex}
    \rlap{\underline{#1}}
    \par
    \setlength{\parindent}{0cm}
    \nopagebreak
    \leftskip=#2cm
    \rightskip=#3cm
}
{
    \par
}
\fi

\doendnotes{C}
\bigskip
\vfill

\clearpage

\footnotesize

\ifkorrekturansicht
  \lohead{\textsc{register}}
\fi

% theindex-Environment neu definieren ohne reledmac
\makeatletter
\renewenvironment{theindex}{%
  \ifkorrekturansicht
    \section*{\indexname}%
  \else
    \subsubsection*{Index der erwähnten Entitäten}%
  \fi
  \setlength{\parindent}{0pt}%
  \setlength{\parskip}{0pt plus 0.3pt}%
  \let\item\@idxitem
}{%
  \ifkorrekturansicht\clearpage\fi
}
\makeatother

\IfFileExists{\jobname-pw.ind}{\input{\jobname-pw.ind}}{}

% Quellenangabe nur in der Leseansicht
\ifkorrekturansicht\else
% Fallback-Definitionen, falls die .tex-Datei \titel etc. nicht gesetzt hat
\providecommand{\titel}{}
\providecommand{\editorInnen}{}
\providecommand{\dateiname}{\jobname}

\vspace{3cm}

\vfill

\footnotesize
\textsc{Quelle}: \titel. Herausgegeben von {\editorInnen}. In: \emph{Arthur Schnitzler: Briefwechsel mit Autorinnen und Autoren}.
 Digitale Edition, https://schnitzler-briefe.acdh.oeaw.ac.at/{\dateiname}.html (Stand \today)
\fi

\end{document}


