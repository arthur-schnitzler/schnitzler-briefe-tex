%% latex-leseansicht-vorspann.tex
%% Vorspann für die Leseansicht.
%% Lädt die gemeinsame Datei latex-vorspann.tex mit nicht gesetztem Schalter.

\newif\ifkorrekturansicht
\korrekturansichtfalse

\input{../tex-inputs/latex-vorspann}


\section[Arthur Schnitzler an Gustav Schwarzkopf, 25. 8. 1899]{L04131 Arthur Schnitzler an Gustav Schwarzkopf, 25. 8. 1899}
\nopagebreak\mylabel{L04131v}
\rehead{ }\normalsize\beginnumbering\briefempfaengerindex{Schwarzkopf, Gustav@\textsc{Schwarzkopf, Gustav}!zzzSchnitzler, Arthur@\emph{von Arthur Schnitzler}!1899-08-252@{25. 8. 1899}|(be}
\toendnotes[C]{\smallbreak\pagebreak[2]}
\correspDesc{Versand  durch Arthur Schnitzler am 25. 8. 1899 in Bad Ischl
\newline{}Erhalt  durch Gustav Schwarzkopf im Zeitraum [26. 8. 1899
                  – 30. 8. 1899?] in Wien}\toendnotes[C]{\smallbreak}
\Standort{CUL, Schnitzler, B 96.}
\physDesc{Brief, 2 Blätter, 6 Seiten, 2606 Zeichen
\newline{}Handschrift: schwarze Tinte, deutsche Kurrent}
\buchAbdrucke{\weitereDrucke{Arthur Schnitzler: \emph{Briefe 1875–1912}. Herausgegeben von Therese Nickl und Heinrich Schnitzler. Frankfurt am Main: \emph{S. Fischer} 1981, S. 374.} }\toendnotes[C]{\smallbreak}
\pstart
           \noindent{}{\pb}Lieber Guſtav, ich hab es ja geahnt, dſs Sie ſchließlich doch nicht
                  ko{\geminationm}en werden – aber Sie müſſen mir trotzdem erlauben,
               enttäuſcht zu ſein. Wenn ich ſagen würde: Ich ſehne mich nach Ihnen,{ }ſo könnten Sie
               erwidern: »Ko{\geminationm}en Sie nach Wien\oindex{Wien@\textbf{Wien}, \emph{Verwaltungsgebiet}|pw} – we{\geminationn} das wahr iſt« — aber Sie wiſſen ja,
               daſs allerlei Wahrheiten in mir zur gleichen Zeit beſtehn können, und Sie werden mir
               daher auch glauben, dſs ich vor Wien\oindex{Wien@\textbf{Wien}, \emph{Verwaltungsgebiet}|pw} – ich fand
               gar keinen \introOben{}andern\introOben{} Ausdruck, eine förm{\pb}liche Angſt habe. Heute hatte ich den{ }ſonderbaren Traum, daſs ich (– laſſen Sie mich ſagen:) die Entſchwundene\pwindex{Reinhard, Marie 13.\,3.\,1871 Wien – 18.\,3.\,1899 ebd.@\textsc{Reinhard, Marie} (13.\,3.\,1871 Wien – 18.\,3.\,1899 ebd.), \emph{Gesangspädagogin}|pwv} vor einem Hutgeſchäft, das im
                  \textsc{Jockeyclub\orgindex{Jockey-Club für Österreich@Jockey-Club für Österreich|pw}} war, erwartete, und wunderte mich zugleich, daſs ich ſie \substVorne{}\textsuperscript{dort e\textcolor{gray}{ve}}\substDazwischen{}ſo nah\substHinten{} vor dem Hauſe erwartete, wo ſie geſtorben war. Sie kam heraus, als ganz alte
               Frau, und ſchien ſich zugleich wegen aller dieſer Dinge zu entſchuldigen; daſs ſie
               mich ſo lang habe warten laſſen, daſs ſie ſo alt und daſs ſie ſchon todt ſei. Es war
               ganz ent{\pb}ſetzlich. – Die beſten
               Stunden hier ſind noch die, in denen ich arbeite; zuweilen geht es leidlich. Nun wird
               auch das Wetter wieder hübſch, und es radelt ſich angenehm, we{\geminationn} nicht die \textsc{Pneumatik} platzt,
               wie geſtern. Sie wiſſen jedenfalls, dſs auch Hugo\pwindex{Hofmannsthal, Hugo von 1.\,2.\,1874 Wien – 15.\,7.\,1929 Rodaun@\textsc{Hofmannsthal, Hugo von} (1.\,2.\,1874 Wien – 15.\,7.\,1929 Rodaun), \emph{Schriftsteller}|pw} da iſt und fleiß\textcolor{gray}{g} an ſeinem Stück ſchreibt, »Die Bergwerke von \textsc{Falun}\pwindex{Hofmannsthal, Hugo von 1.\,2.\,1874 Wien – 15.\,7.\,1929 Rodaun@\textsc{Hofmannsthal, Hugo von} (1.\,2.\,1874 Wien – 15.\,7.\,1929 Rodaun), \emph{Schriftsteller}!Bergwerk zu Falun@\strich\emph{Das Bergwerk zu Falun}|pw}«. Ich freue mich{ }ſehr, dſs er da iſt. Von Familie wi{\geminationm}elt es und man dankt Ihnen für Ihre lieben Grüße – aber
               – Sie können mir glauben, alle bedauern, daſs Sie {\pb}ſich zu keinem Herko{\geminationm}en (dieſer Doppelſinn des Wortes Herko{\geminationm}en fällt mir jetzt erſt auf) entſchließen konnten. Vor
               ein paar Tagen iſt meine \label{K_L04131-1v}\edtext{Tante \textsc{Marie Schey\pwindex{Schey, Marie 8.\,5.\,1821 Nagykanizsa – 22.\,8.\,1899 Bad Ischl@\textsc{Schey, Marie} (8.\,5.\,1821 Nagykanizsa – 22.\,8.\,1899 Bad Ischl)|pw}} geſtorben}{\lemma{\textnormal{\emph{Tante … gestorben}}}\Cendnote{\textnormal{am 22. 8. 1899}}}\label{K_L04131-1}; der Bequemlichkeit halber iſt ihr hier noch ein Speiſeröhrenkrebs
               andiagnoſtizirt worden. Ich war in Ungnade, weil das Dienſtperſonal erklärte, ich
               habe ſie nach Iſchl\oindex{Bad Ischl@\textbf{Bad Ischl}|pw} geſchickt, und in Wien\oindex{Wien@\textbf{Wien}, \emph{Verwaltungsgebiet}|pw} wäre ſie nie geſtorben. – Ganz Boz\pwindex{Dickens, Charles 7.\,2.\,1812 Landport – 9.\,6.\,1870 Gads Hill Place@\textsc{Dickens, Charles} (7.\,2.\,1812 Landport – 9.\,6.\,1870 Gads Hill Place), \emph{Schriftsteller, Schriftsteller}|pw}iſche Sachen; davon mündlich. – {\pb}Wahrſcheinlich bleib ich bis etwa
                  8. hier. Es iſt möglich, dſs ich da{\geminationn} auf
               einige Tage mit M. E.\pwindex{Elsinger, Marie *~28.\,2.\,1874 St. Pölten@\textsc{Elsinger, Marie} (*~28.\,2.\,1874 St. Pölten), \emph{Schauspielerin}|pw} zuſa{\geminationm}enko{\geminationm}e, we{\geminationn}{ }ſie nicht auf Verſehen nach Madrid\oindex{Madrid@\textbf{Madrid}, \emph{Hauptstadt}|pw} reiſt ſtatt nach Innsbruck\oindex{Innsbruck@\textbf{Innsbruck}, \emph{Verwaltungsgebiet}|pw} oder in der Zwiſchenzeit von einem Detectiv erwürgt wird. Ich
               bekomme Briefe von ihr, in denen der Schwachſinn die Verlogenheit überwiegt und ka{\geminationn} mich leider nur mit dem letztern, we{\geminationn} auch da ohne Elan, revanchiren. – Trotzdem werd ich
               wieder alle die Unbequemlichkeiten auf mich nehmen, {\pb}– »\label{K_L04131-2v}\edtext{und alles dies für eine einzige
                  Nacht\pwindex{Schnitzler, Arthur 15. 5. 1862 Wien – 21. 10. 1931 ebd.@\textsc{Schnitzler, Arthur} (15. 5. 1862 Wien – 21. 10. 1931 ebd.), \emph{Schriftsteller, Mediziner}!Schleier der Beatrice. Schauspiel in fünf Akten@\strich\emph{Der Schleier der Beatrice. Schauspiel in fünf Akten}|pwv}}{\lemma{\textnormal{\emph{und … Nacht}}}\Cendnote{\textnormal{Im 2. Akt sagt der Herzog : »Sie sollen alle Dir
                           gehören: Steine\pwindex{Schnitzler, Arthur 15. 5. 1862 Wien – 21. 10. 1931 ebd.@\textsc{Schnitzler, Arthur} (15. 5. 1862 Wien – 21. 10. 1931 ebd.), \emph{Schriftsteller, Mediziner}!Schleier der Beatrice. Schauspiel in fünf Akten@\strich\emph{Der Schleier der Beatrice. Schauspiel in fünf Akten}|pwv}{ / }Und Kleider aus Damast und
                           Perlenschnüre\pwindex{Schnitzler, Arthur 15. 5. 1862 Wien – 21. 10. 1931 ebd.@\textsc{Schnitzler, Arthur} (15. 5. 1862 Wien – 21. 10. 1931 ebd.), \emph{Schriftsteller, Mediziner}!Schleier der Beatrice. Schauspiel in fünf Akten@\strich\emph{Der Schleier der Beatrice. Schauspiel in fünf Akten}|pwv}{ / }Sind alle Dein, und zu dem
                           Allem noch\pwindex{Schnitzler, Arthur 15. 5. 1862 Wien – 21. 10. 1931 ebd.@\textsc{Schnitzler, Arthur} (15. 5. 1862 Wien – 21. 10. 1931 ebd.), \emph{Schriftsteller, Mediziner}!Schleier der Beatrice. Schauspiel in fünf Akten@\strich\emph{Der Schleier der Beatrice. Schauspiel in fünf Akten}|pwv}{ / }Ein Schleier von so
                           wunderbarer Schönheit,\pwindex{Schnitzler, Arthur 15. 5. 1862 Wien – 21. 10. 1931 ebd.@\textsc{Schnitzler, Arthur} (15. 5. 1862 Wien – 21. 10. 1931 ebd.), \emph{Schriftsteller, Mediziner}!Schleier der Beatrice. Schauspiel in fünf Akten@\strich\emph{Der Schleier der Beatrice. Schauspiel in fünf Akten}|pwv}{ / }Wie keiner, den ein
                           Mädchen dieses Land’s\pwindex{Schnitzler, Arthur 15. 5. 1862 Wien – 21. 10. 1931 ebd.@\textsc{Schnitzler, Arthur} (15. 5. 1862 Wien – 21. 10. 1931 ebd.), \emph{Schriftsteller, Mediziner}!Schleier der Beatrice. Schauspiel in fünf Akten@\strich\emph{Der Schleier der Beatrice. Schauspiel in fünf Akten}|pwv}{ / }Und niemals eine Herzogin
                           getragen.\pwindex{Schnitzler, Arthur 15. 5. 1862 Wien – 21. 10. 1931 ebd.@\textsc{Schnitzler, Arthur} (15. 5. 1862 Wien – 21. 10. 1931 ebd.), \emph{Schriftsteller, Mediziner}!Schleier der Beatrice. Schauspiel in fünf Akten@\strich\emph{Der Schleier der Beatrice. Schauspiel in fünf Akten}|pwv}{ / }So kostbar, daß der Fürst
                           von Pergamum\pwindex{Schnitzler, Arthur 15. 5. 1862 Wien – 21. 10. 1931 ebd.@\textsc{Schnitzler, Arthur} (15. 5. 1862 Wien – 21. 10. 1931 ebd.), \emph{Schriftsteller, Mediziner}!Schleier der Beatrice. Schauspiel in fünf Akten@\strich\emph{Der Schleier der Beatrice. Schauspiel in fünf Akten}|pwv}{ / }Ihn und nur ihn allein als
                           Hochzeitsgabe\pwindex{Schnitzler, Arthur 15. 5. 1862 Wien – 21. 10. 1931 ebd.@\textsc{Schnitzler, Arthur} (15. 5. 1862 Wien – 21. 10. 1931 ebd.), \emph{Schriftsteller, Mediziner}!Schleier der Beatrice. Schauspiel in fünf Akten@\strich\emph{Der Schleier der Beatrice. Schauspiel in fünf Akten}|pwv}{ / }Der Fürstin schenkte, die
                           er sich erwählt.\pwindex{Schnitzler, Arthur 15. 5. 1862 Wien – 21. 10. 1931 ebd.@\textsc{Schnitzler, Arthur} (15. 5. 1862 Wien – 21. 10. 1931 ebd.), \emph{Schriftsteller, Mediziner}!Schleier der Beatrice. Schauspiel in fünf Akten@\strich\emph{Der Schleier der Beatrice. Schauspiel in fünf Akten}|pwv}{ / }Ich geb’ ihn Dir für eine
                           einz’ge Nacht.\pwindex{Schnitzler, Arthur 15. 5. 1862 Wien – 21. 10. 1931 ebd.@\textsc{Schnitzler, Arthur} (15. 5. 1862 Wien – 21. 10. 1931 ebd.), \emph{Schriftsteller, Mediziner}!Schleier der Beatrice. Schauspiel in fünf Akten@\strich\emph{Der Schleier der Beatrice. Schauspiel in fünf Akten}|pwv}«}}}\label{K_L04131-2}«, wie der Herzog von Bologna\pwindex{Schnitzler, Arthur 15. 5. 1862 Wien – 21. 10. 1931 ebd.@\textsc{Schnitzler, Arthur} (15. 5. 1862 Wien – 21. 10. 1931 ebd.), \emph{Schriftsteller, Mediziner}!Schleier der Beatrice. Schauspiel in fünf Akten@\strich\emph{Der Schleier der Beatrice. Schauspiel in fünf Akten}|pwv} weniger originell als fünffüßig im 2. Akt der Beatrice\pwindex{Schnitzler, Arthur 15. 5. 1862 Wien – 21. 10. 1931 ebd.@\textsc{Schnitzler, Arthur} (15. 5. 1862 Wien – 21. 10. 1931 ebd.), \emph{Schriftsteller, Mediziner}!Schleier der Beatrice. Schauspiel in fünf Akten@\strich\emph{Der Schleier der Beatrice. Schauspiel in fünf Akten}|pw} bemerkt. – Hier lebe ich vollko{\geminationm}en zurückgezogen,– bis zur Unhöflichkeit. Schreiben Sie
               mir doch{ }ſehr bald wieder!\pend
           \pstart Von Herzen Ihr\spacefill\mbox{Arth Sch}\pend{}
\pstart
           \textsc{Ischl\oindex{Bad Ischl@\textbf{Bad Ischl}|pw}}{ }25/8 99.\pend
           
\pstart
           Kapper\pwindex{Kapper, Friedrich 21.\,4.\,1861 Wien – 22.\,7.\,1939 ebd.@\textsc{Kapper, Friedrich} (21.\,4.\,1861 Wien – 22.\,7.\,1939 ebd.), \emph{Mediziner}|pw} erzählte viel von Ihnen. Auch von
                     Ebermann\pwindex{Ebermann, Leo 16.\,7.\,1863 Draganovka – 9.\,10.\,1914 Wien@\textsc{Ebermann, Leo} (16.\,7.\,1863 Draganovka – 9.\,10.\,1914 Wien), \emph{Schriftsteller, Journalist, Rechtswissenschaftler}|pw}; hier war der
                  Enthuſiasmus geringer.\pend
           \selectlanguage{ngerman}\endnumbering\briefempfaengerindex{Schwarzkopf, Gustav@\textsc{Schwarzkopf, Gustav}!zzzSchnitzler, Arthur@\emph{von Arthur Schnitzler}!1899-08-252@{25. 8. 1899}|)be}\mylabel{L04131h}
\begin{anhang}
\end{anhang}\newcommand{\dateiname}{L04131}\newcommand{\titel}{Arthur Schnitzler an Gustav Schwarzkopf, 25. 8. 1899}\newcommand{\editorInnen}{Herausgegeben von Jahnke, SelmaMüller, Martin Anton}%% latex-leseansicht-abspann.tex
%% Abspann für die Leseansicht.
%% Der Schalter \ifkorrekturansicht ist bereits durch den Vorspann gesetzt.

%% latex-abspann.tex
%% Gemeinsamer Abspann für Korrekturansicht und Leseansicht.
%% Setzt den Schalter \ifkorrekturansicht voraus (gesetzt in den
%% einbindenden Dateien latex-korrekturansicht-abspann.tex bzw.
%% latex-leseansicht-abspann.tex).
%% ---------------------------------------------------------------

\normalsize

% Das esempio-Environment wird nur in der Leseansicht benötigt
\ifkorrekturansicht\else
\newenvironment{esempio}[3]%
{
    \vspace{1.5ex}
    \rlap{\underline{#1}}
    \par
    \setlength{\parindent}{0cm}
    \nopagebreak
    \leftskip=#2cm
    \rightskip=#3cm
}
{
    \par
}
\fi

\doendnotes{C}
\bigskip
\vfill

\clearpage

\footnotesize

\ifkorrekturansicht
  \lohead{\textsc{register}}
\fi

% theindex-Environment neu definieren ohne reledmac
\makeatletter
\renewenvironment{theindex}{%
  \ifkorrekturansicht
    \section*{\indexname}%
  \else
    \subsubsection*{Index der erwähnten Entitäten}%
  \fi
  \setlength{\parindent}{0pt}%
  \setlength{\parskip}{0pt plus 0.3pt}%
  \let\item\@idxitem
}{%
  \ifkorrekturansicht\clearpage\fi
}
\makeatother

\IfFileExists{\jobname-pw.ind}{\input{\jobname-pw.ind}}{}

% Quellenangabe nur in der Leseansicht
\ifkorrekturansicht\else
% Fallback-Definitionen, falls die .tex-Datei \titel etc. nicht gesetzt hat
\providecommand{\titel}{}
\providecommand{\editorInnen}{}
\providecommand{\dateiname}{\jobname}

\vspace{3cm}

\vfill

\footnotesize
\textsc{Quelle}: \titel. Herausgegeben von {\editorInnen}. In: \emph{Arthur Schnitzler: Briefwechsel mit Autorinnen und Autoren}.
 Digitale Edition, https://schnitzler-briefe.acdh.oeaw.ac.at/{\dateiname}.html (Stand \today)
\fi

\end{document}


