%% latex-leseansicht-vorspann.tex
%% Vorspann für die Leseansicht.
%% Lädt die gemeinsame Datei latex-vorspann.tex mit nicht gesetztem Schalter.

\newif\ifkorrekturansicht
\korrekturansichtfalse

\input{../tex-inputs/latex-vorspann}


\section[ Felix Salten an Arthur Schnitzler, {[}30. 7. 1895{]}]{L03160 Felix Salten an Arthur Schnitzler,  [30. 7. 1895]}
\nopagebreak\mylabel{L03160v}
\rehead{ }\normalsize\beginnumbering\briefempfaengerindex{Schnitzler, Arthur@\textsc{Schnitzler, Arthur}!zzzSalten, Felix@\emph{von Felix Salten}!1895-07-302@{{[}30. 7. 1895{]}}|(be}
\toendnotes[C]{\smallbreak\pagebreak[2]}
\correspDesc{Versand  durch Felix Salten am [30. 7. 1895] in Wien
\newline{}Erhalt  durch Arthur Schnitzler am [31. 7. 1895?] in Bad Ischl}\toendnotes[C]{\smallbreak}
\Standort{CUL, Schnitzler, B 89, A 1.}
\physDesc{Brief, 1 Blatt, 4 Seiten, 1092 Zeichen
\newline{}Handschrift: Bleistift, lateinische Kurrent
\newline{}Schnitzler: mit Bleistift datiert: »30/7 95.« 
\newline{}Ordnung: mit Bleistift von unbekannter Hand nummeriert: »60« }\toendnotes[C]{\smallbreak}
\pstart
           \noindent{}{\pb}Lieber Freund! Dank für den Brief. Ich bin hier so auf mich allein
               gestellt, und durch alle die traurigen Agonie-Stimmungen die ich täglich mitmache, so
               herabgedrückt, dass ich es noch weit angenehmer empfinde, als Sie, wenn man mir {\pb}Briefe schreibt. Dass \label{K_L03160-1v}\edtext{Freiwild\pwindex{Schnitzler, Arthur 15.\,5.\,1862 Wien – 21.\,10.\,1931 ebd.@\textsc{Schnitzler, Arthur} (15.\,5.\,1862 Wien – 21.\,10.\,1931 ebd.), \emph{Schriftsteller, Mediziner}!Freiwild. Schauspiel in 3 Akten@\strich\emph{Freiwild. Schauspiel in 3 Akten}|pw} fortschreitet}{\lemma{\textnormal{\emph{Freiwild fortschreitet}}}\Cendnote{\textnormal{Am 15. 6. 1895 hatte Schnitzler
                  die Arbeit an \emph{Freiwild}\pwindex{Schnitzler, Arthur 15.\,5.\,1862 Wien – 21.\,10.\,1931 ebd.@\textsc{Schnitzler, Arthur} (15.\,5.\,1862 Wien – 21.\,10.\,1931 ebd.), \emph{Schriftsteller, Mediziner}!Freiwild. Schauspiel in 3 Akten@\strich\emph{Freiwild. Schauspiel in 3 Akten}|pwk} wiederaufgenommen. Am
                     2. 8. 1895
                  stellte er den ersten Akt\pwindex{Schnitzler, Arthur 15.\,5.\,1862 Wien – 21.\,10.\,1931 ebd.@\textsc{Schnitzler, Arthur} (15.\,5.\,1862 Wien – 21.\,10.\,1931 ebd.), \emph{Schriftsteller, Mediziner}!Freiwild. Schauspiel in 3 Akten@\strich\emph{Freiwild. Schauspiel in 3 Akten}|pwkv}
                  fertig.}}}\label{K_L03160-1} ist recht. Auch dem \label{K_L03160-2v}\edtext{Götterliebling\pwindex{Beer-Hofmann, Richard 11.\,7.\,1866 Wien – 26.\,9.\,1945 New York City@\textsc{Beer-Hofmann, Richard} (11.\,7.\,1866 Wien – 26.\,9.\,1945 New York City), \emph{Schriftsteller}!Tod Georgs@\strich\emph{Der Tod Georgs}|pw}}{\lemma{\textnormal{\emph{Götterliebling}}}\Cendnote{\textnormal{Richard Beer-Hofmann\pwindex{Beer-Hofmann, Richard 11.\,7.\,1866 Wien – 26.\,9.\,1945 New York City@\textsc{Beer-Hofmann, Richard} (11.\,7.\,1866 Wien – 26.\,9.\,1945 New York City), \emph{Schriftsteller}|pwk} arbeitete in dieser
                  Zeit intensiv an der Erzählung, die er später unter dem Titel \emph{Der Tod Georgs}\pwindex{Beer-Hofmann, Richard 11.\,7.\,1866 Wien – 26.\,9.\,1945 New York City@\textsc{Beer-Hofmann, Richard} (11.\,7.\,1866 Wien – 26.\,9.\,1945 New York City), \emph{Schriftsteller}!Tod Georgs@\strich\emph{Der Tod Georgs}|pwk} publizierte.}}}\label{K_L03160-2} wär das schon sehr zu
               wünschen. Möchten doch beide Sachen\pwindex{Beer-Hofmann, Richard 11.\,7.\,1866 Wien – 26.\,9.\,1945 New York City@\textsc{Beer-Hofmann, Richard} (11.\,7.\,1866 Wien – 26.\,9.\,1945 New York City), \emph{Schriftsteller}!Tod Georgs@\strich\emph{Der Tod Georgs}|pwv}\pwindex{Schnitzler, Arthur 15.\,5.\,1862 Wien – 21.\,10.\,1931 ebd.@\textsc{Schnitzler, Arthur} (15.\,5.\,1862 Wien – 21.\,10.\,1931 ebd.), \emph{Schriftsteller, Mediziner}!Freiwild. Schauspiel in 3 Akten@\strich\emph{Freiwild. Schauspiel in 3 Akten}|pwv} bis zum Herbste fertig sein. Pusterthal\oindex{Pustertal@\textbf{Pustertal}, \emph{Tal}|pw} wäre sehr schön, ob wir uns nicht aber
               doch lieber ruhig in Ischl\oindex{Bad Ischl@\textbf{Bad Ischl}|pw} aufhalten und in den
               gewissen behaglichen Parthien die \label{K_L03160-3v}\edtext{Gegend abfahren}{\lemma{\textnormal{\emph{Gegend abfahren}}}\Cendnote{\textnormal{Sie einigten sich auf
                  eine Radtour von Salzburg\oindex{Salzburg@\textbf{Salzburg}, \emph{Verwaltungsgebiet}|pwk} nach München\oindex{München@\textbf{München}|pwk}, siehe XXXX Auszeichnungsfehler: Dokument L03159 nicht gefunden.}}}\label{K_L03160-3} wollen. Dann {\pb}noch Eins. Ich werde sehr
               gequält nach Rügen\oindex{Rügen@\textbf{Rügen}, \emph{Insel}|pw} zu fahren. \uline{E.\pwindex{Kotter, Elisabeth *~1873 Groß-Enzersdorf@\textsc{Kotter, Elisabeth} (*~1873 Groß-Enzersdorf), \emph{Haushaltshilfe}|pwu}}, die in Heringsdorf\oindex{Heringsdorf@\textbf{Heringsdorf}, \emph{Hauptstadt}|pw} ist, schreibt
               rührende Briefe. Vielleicht finde ich mich also dann doch bestimmt so gegen den 27 od. 28. August dahin zu
               reisen. Aber das wird sich ja alles noch
               entscheiden, bis ich nach Ischl\oindex{Bad Ischl@\textbf{Bad Ischl}|pw}{ }{\pb}komme. Vorerst freue ich
               mich auf den Montag, oder Sonntag. Ich verständige Sie jedenfalls noch vorher. Für heute sende ich die gewünschten \label{K_L03160-4v}\edtext{Feuilletons\pwindex{Salten, Felix 6.\,9.\,1869 Budapest – 8.\,10.\,1945 Zürich@\textsc{Salten, Felix} (6.\,9.\,1869 Budapest – 8.\,10.\,1945 Zürich), \emph{Schriftsteller, Journalist, Chefredakteur}!Münchener Kunstausstellungen. I. Im königl. Glaspalast@\strich\emph{Die Münchener Kunstausstellungen. I. Im königl. Glaspalast}|pwv}\pwindex{Salten, Felix 6.\,9.\,1869 Budapest – 8.\,10.\,1945 Zürich@\textsc{Salten, Felix} (6.\,9.\,1869 Budapest – 8.\,10.\,1945 Zürich), \emph{Schriftsteller, Journalist, Chefredakteur}!Münchener Kunstausstellungen. II. Im königl. Glaspalast@\strich\emph{Die Münchener Kunstausstellungen. II. Im königl. Glaspalast}|pwv}\pwindex{Salten, Felix 6.\,9.\,1869 Budapest – 8.\,10.\,1945 Zürich@\textsc{Salten, Felix} (6.\,9.\,1869 Budapest – 8.\,10.\,1945 Zürich), \emph{Schriftsteller, Journalist, Chefredakteur}!Münchener Brief. (Orig.-Corr. der »Wiener Allg. Ztg.«)@\strich\emph{Münchener Brief. (Orig.-Corr. der »Wiener Allg. Ztg.«)}|pwv}}{\lemma{\textnormal{\emph{Feuilletons}}}\Cendnote{\textnormal{f. s.\pwindex{Salten, Felix 6.\,9.\,1869 Budapest – 8.\,10.\,1945 Zürich@\textsc{Salten, Felix} (6.\,9.\,1869 Budapest – 8.\,10.\,1945 Zürich), \emph{Schriftsteller, Journalist, Chefredakteur}|pwk} [ = Felix Salten\pwindex{Salten, Felix 6.\,9.\,1869 Budapest – 8.\,10.\,1945 Zürich@\textsc{Salten, Felix} (6.\,9.\,1869 Budapest – 8.\,10.\,1945 Zürich), \emph{Schriftsteller, Journalist, Chefredakteur}|pwk}]: \emph{Münchener Brief. (Orig.-Corr. der »Wiener Allg. Ztg.«)}\pwindex{Salten, Felix 6.\,9.\,1869 Budapest – 8.\,10.\,1945 Zürich@\textsc{Salten, Felix} (6.\,9.\,1869 Budapest – 8.\,10.\,1945 Zürich), \emph{Schriftsteller, Journalist, Chefredakteur}!Münchener Brief. (Orig.-Corr. der »Wiener Allg. Ztg.«)@\strich\emph{Münchener Brief. (Orig.-Corr. der »Wiener Allg. Ztg.«)}|pwk}. In: \emph{Wiener Allgemeine Zeitung}\pwindex{Wiener Allgemeine Zeitung@\emph{Wiener Allgemeine Zeitung}|pwk}, Nr. 5200, 6. 7. 1895, S. 8; Felix Salten\pwindex{Salten, Felix 6.\,9.\,1869 Budapest – 8.\,10.\,1945 Zürich@\textsc{Salten, Felix} (6.\,9.\,1869 Budapest – 8.\,10.\,1945 Zürich), \emph{Schriftsteller, Journalist, Chefredakteur}|pwk}: \emph{Die Münchener Kunstausstellungen. I. Im königl.
                        Glaspalast}\pwindex{Salten, Felix 6.\,9.\,1869 Budapest – 8.\,10.\,1945 Zürich@\textsc{Salten, Felix} (6.\,9.\,1869 Budapest – 8.\,10.\,1945 Zürich), \emph{Schriftsteller, Journalist, Chefredakteur}!Münchener Kunstausstellungen. I. Im königl. Glaspalast@\strich\emph{Die Münchener Kunstausstellungen. I. Im königl. Glaspalast}|pwk}. In: \emph{Wiener Allgemeine Zeitung}\pwindex{Wiener Allgemeine Zeitung@\emph{Wiener Allgemeine Zeitung}|pwk}, Nr. 5215,
                        24. 7. 1895, S. 2; Felix Salten\pwindex{Salten, Felix 6.\,9.\,1869 Budapest – 8.\,10.\,1945 Zürich@\textsc{Salten, Felix} (6.\,9.\,1869 Budapest – 8.\,10.\,1945 Zürich), \emph{Schriftsteller, Journalist, Chefredakteur}|pwk}: \emph{Die Münchener Kunstausstellungen. II. Im königl. Glaspalast}\pwindex{Salten, Felix 6.\,9.\,1869 Budapest – 8.\,10.\,1945 Zürich@\textsc{Salten, Felix} (6.\,9.\,1869 Budapest – 8.\,10.\,1945 Zürich), \emph{Schriftsteller, Journalist, Chefredakteur}!Münchener Kunstausstellungen. II. Im königl. Glaspalast@\strich\emph{Die Münchener Kunstausstellungen. II. Im königl. Glaspalast}|pwk}. In: \emph{Wiener Allgemeine Zeitung}\pwindex{Wiener Allgemeine Zeitung@\emph{Wiener Allgemeine Zeitung}|pwk}, Nr. 5216, 25. 7. 1895, S. 2–3. Siehe XXXX Auszeichnungsfehler: Dokument L03159 nicht gefunden.}}}\label{K_L03160-4}. Auch die für Goldmann\pwindex{Goldmann, Paul 31.\,1.\,1865 Breslau – 25.\,9.\,1935 Wien@\textsc{Goldmann, Paul} (31.\,1.\,1865 Breslau – 25.\,9.\,1935 Wien), \emph{Schriftsteller, Journalist}|pw} bestimmten, welche Sie absenden werden, falls \substVorne{}\textsuperscript{s}\substDazwischen{}es\substHinten{} noch Zeit ist, ja?\pend
           
\pstart
           Also auf baldiges Wiedersehen, herzlichst {\\[\baselineskip]}Ihr \spacefill\mbox{Salten.}\pend
           \leftskip=0em{}\selectlanguage{ngerman}\endnumbering\briefempfaengerindex{Schnitzler, Arthur@\textsc{Schnitzler, Arthur}!zzzSalten, Felix@\emph{von Felix Salten}!1895-07-302@{{[}30. 7. 1895{]}}|)be}\mylabel{L03160h}  \newcommand{\dateiname}{L03160}\newcommand{\titel}{Felix Salten an Arthur Schnitzler, [30. 7. 1895]}\newcommand{\editorInnen}{Martin Anton Müller und Laura Untner}%% latex-leseansicht-abspann.tex
%% Abspann für die Leseansicht.
%% Der Schalter \ifkorrekturansicht ist bereits durch den Vorspann gesetzt.

%% latex-abspann.tex
%% Gemeinsamer Abspann für Korrekturansicht und Leseansicht.
%% Setzt den Schalter \ifkorrekturansicht voraus (gesetzt in den
%% einbindenden Dateien latex-korrekturansicht-abspann.tex bzw.
%% latex-leseansicht-abspann.tex).
%% ---------------------------------------------------------------

\normalsize

% Das esempio-Environment wird nur in der Leseansicht benötigt
\ifkorrekturansicht\else
\newenvironment{esempio}[3]%
{
    \vspace{1.5ex}
    \rlap{\underline{#1}}
    \par
    \setlength{\parindent}{0cm}
    \nopagebreak
    \leftskip=#2cm
    \rightskip=#3cm
}
{
    \par
}
\fi

\doendnotes{C}
\bigskip
\vfill

\clearpage

\footnotesize

\ifkorrekturansicht
  \lohead{\textsc{register}}
\fi

% theindex-Environment neu definieren ohne reledmac
\makeatletter
\renewenvironment{theindex}{%
  \ifkorrekturansicht
    \section*{\indexname}%
  \else
    \subsubsection*{Index der erwähnten Entitäten}%
  \fi
  \setlength{\parindent}{0pt}%
  \setlength{\parskip}{0pt plus 0.3pt}%
  \let\item\@idxitem
}{%
  \ifkorrekturansicht\clearpage\fi
}
\makeatother

\IfFileExists{\jobname-pw.ind}{\input{\jobname-pw.ind}}{}

% Quellenangabe nur in der Leseansicht
\ifkorrekturansicht\else
% Fallback-Definitionen, falls die .tex-Datei \titel etc. nicht gesetzt hat
\providecommand{\titel}{}
\providecommand{\editorInnen}{}
\providecommand{\dateiname}{\jobname}

\vspace{3cm}

\vfill

\footnotesize
\textsc{Quelle}: \titel. Herausgegeben von {\editorInnen}. In: \emph{Arthur Schnitzler: Briefwechsel mit Autorinnen und Autoren}.
 Digitale Edition, https://schnitzler-briefe.acdh.oeaw.ac.at/{\dateiname}.html (Stand \today)
\fi

\end{document}


