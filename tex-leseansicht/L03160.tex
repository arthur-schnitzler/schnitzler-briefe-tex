%% latex-leseansicht-vorspann.tex
%% Vorspann für die Leseansicht.
%% Lädt die gemeinsame Datei latex-vorspann.tex mit nicht gesetztem Schalter.

\newif\ifkorrekturansicht
\korrekturansichtfalse

\input{../tex-inputs/latex-vorspann}


         
         \renewcommand{\erwaehntePersonen}{Personen: Richard Beer-Hofmann, Paul Goldmann, Elisabeth Kotter}
         \renewcommand{\erwaehnteOrte}{Orte: Bad Ischl, Heringsdorf, München, Pustertal, Rügen, Salzburg, Wien}
         \renewcommand{\erwaehnteWerke}{Werke: Der Tod Georgs, Die Münchener Kunstausstellungen. I. Im königl. Glaspalast, Die Münchener Kunstausstellungen. II. Im königl. Glaspalast, Freiwild. Schauspiel in 3 Akten, Münchener Brief. (Orig.-Corr. der »Wiener Allg. Ztg.«), Wiener Allgemeine Zeitung}
               \section[ Felix Salten an Arthur Schnitzler, {[}30. 7. 1895{]}]{ Felix Salten an Arthur Schnitzler, {[}30. 7. 1895{]}}\nopagebreak\mylabel{v}\rehead{ }\begin{ledgroupsized}[t]{13cm}\normalsize\beginnumbering \toendnotes[C]{\smallbreak\pagebreak[2]} \Standort{CUL, Schnitzler, B 89, A 1.}
\physDesc{Brief, 1 Blatt, 4 Seiten, 1092 Zeichen
\newline{}Handschrift: Bleistift, lateinische Kurrent
\newline{}Schnitzler: mit Bleistift datiert: »30/7 95.« 
\newline{}Ordnung: mit Bleistift von unbekannter Hand nummeriert: »60« }\toendnotes[C]{\smallbreak}\pstart
           \noindent{}{\pb}Lieber Freund! Dank für den Brief. Ich bin hier so auf
               mich allein gestellt, und durch alle die traurigen Agonie-Stimmungen die ich täglich
               mitmache, so herabgedrückt, dass ich es noch weit angenehmer empfinde, als Sie, wenn
               man mir {\pb}Briefe schreibt. Dass
                  \label{K_L03160-1v}\edtext{Freiwild\pwindex{Schnitzler, Arthur 15.05.1862 – 21.10.1931@\textsc{Schnitzler, Arthur} (15.05.1862 – 21.10.1931), \emph{Schriftsteller, Mediziner}!Freiwild. Schauspiel in 3 Akten1896@\strich\emph{Freiwild. Schauspiel in 3 Akten} {[}1896{]}|pw} fortschreitet}{\lemma{\textnormal{\emph{Freiwild fortschreitet}}}\Cendnote{\textnormal{Am 15. 6. 1895 hatte Schnitzler\pwindex{Schnitzler, Arthur 15.05.1862 – 21.10.1931@\textsc{Schnitzler, Arthur} (15.05.1862 – 21.10.1931), \emph{Schriftsteller, Mediziner}|pwk}
                  die Arbeit an \emph{Freiwild}\pwindex{Schnitzler, Arthur 15.05.1862 – 21.10.1931@\textsc{Schnitzler, Arthur} (15.05.1862 – 21.10.1931), \emph{Schriftsteller, Mediziner}!Freiwild. Schauspiel in 3 Akten1896@\strich\emph{Freiwild. Schauspiel in 3 Akten} {[}1896{]}|pwk} wiederaufgenommen. Am
                     2. 8. 1895
                  stellte er den ersten Akt\pwindex{Schnitzler, Arthur 15.05.1862 – 21.10.1931@\textsc{Schnitzler, Arthur} (15.05.1862 – 21.10.1931), \emph{Schriftsteller, Mediziner}!Freiwild. Schauspiel in 3 Akten1896@\strich\emph{Freiwild. Schauspiel in 3 Akten} {[}1896{]}|pwkv}
                  fertig.}}}\label{K_L03160-1h} ist recht. Auch dem \label{K_L03160-2v}\edtext{Götterliebling\pwindex{Beer-Hofmann, Richard 1866-07-11 – 1945-09-26@\textsc{Beer-Hofmann, Richard} (1866-07-11 – 1945-09-26), \emph{Schriftsteller}!Tod Georgs1900@\strich\emph{Der Tod Georgs} {[}1900{]}|pw}}{\lemma{\textnormal{\emph{Götterliebling}}}\Cendnote{\textnormal{Richard Beer-Hofmann\pwindex{Beer-Hofmann, Richard 1866-07-11 – 1945-09-26@\textsc{Beer-Hofmann, Richard} (1866-07-11 – 1945-09-26), \emph{Schriftsteller}|pwk} arbeitete in dieser
                  Zeit intensiv an der Erzählung, die er später unter dem Titel \emph{Der Tod Georgs}\pwindex{Beer-Hofmann, Richard 1866-07-11 – 1945-09-26@\textsc{Beer-Hofmann, Richard} (1866-07-11 – 1945-09-26), \emph{Schriftsteller}!Tod Georgs1900@\strich\emph{Der Tod Georgs} {[}1900{]}|pwk} publizierte.}}}\label{K_L03160-2h} wär das schon sehr zu
               wünschen. Möchten doch beide Sachen\pwindex{Beer-Hofmann, Richard 1866-07-11 – 1945-09-26@\textsc{Beer-Hofmann, Richard} (1866-07-11 – 1945-09-26), \emph{Schriftsteller}!Tod Georgs1900@\strich\emph{Der Tod Georgs} {[}1900{]}|pwv}\pwindex{Schnitzler, Arthur 15.05.1862 – 21.10.1931@\textsc{Schnitzler, Arthur} (15.05.1862 – 21.10.1931), \emph{Schriftsteller, Mediziner}!Freiwild. Schauspiel in 3 Akten1896@\strich\emph{Freiwild. Schauspiel in 3 Akten} {[}1896{]}|pwv} bis zum Herbste fertig sein. Pusterthal\oindex{Pustertal@\textbf{Pustertal}|pw} wäre sehr schön, ob wir uns nicht aber
               doch lieber ruhig in Ischl\oindex{Bad Ischl@\textbf{Bad Ischl}|pw} aufhalten und in den
               gewissen behaglichen Parthien die \label{K_L03160-3v}\edtext{Gegend abfahren}{\lemma{\textnormal{\emph{Gegend abfahren}}}\Cendnote{\textnormal{Sie einigten sich auf eine Radtour von Salzburg\oindex{Salzburg@\textbf{Salzburg}|pwk} nach München\oindex{Muenchen@\textbf{München}|pwk}, siehe Felix Salten an Arthur Schnitzler, 22. 7. 1895.}}}\label{K_L03160-3h} wollen. Dann {\pb}noch Eins. Ich werde sehr
               gequält nach Rügen\oindex{Ruegen@\textbf{Rügen}|pw} zu fahren. \uline{E.\pwindex{Kotter, Elisabeth *~1873@\textsc{Kotter, Elisabeth} (*~1873), \emph{Haushaltshilfe}|pwu}}, die in Heringsdorf\oindex{Heringsdorf@\textbf{Heringsdorf}|pw} ist, schreibt
               rührende Briefe. Vielleicht finde ich mich also dann doch bestimmt so gegen den 27 od. 28. August dahin zu
               reisen. Aber das wird sich ja alles noch
               entscheiden, bis ich nach Ischl\oindex{Bad Ischl@\textbf{Bad Ischl}|pw}{ }{\pb}komme. Vorerst freue ich
               mich auf den Montag, oder Sonntag. Ich verständige Sie jedenfalls noch vorher. Für heute sende ich die gewünschten \label{K_L03160-4v}\edtext{Feuilletons\pwindex{Salten, Felix 06.09.1869 – 08.10.1945@\textsc{Salten, Felix} (06.09.1869 – 08.10.1945), \emph{Schriftsteller, Journalist}!Muenchener Kunstausstellungen. I. Im koenigl. Glaspalast1895-07-24@\strich\emph{Die Münchener Kunstausstellungen. I. Im königl. Glaspalast} {[}1895-07-24{]}|pwv}\pwindex{Salten, Felix 06.09.1869 – 08.10.1945@\textsc{Salten, Felix} (06.09.1869 – 08.10.1945), \emph{Schriftsteller, Journalist}!Muenchener Kunstausstellungen. II. Im koenigl. Glaspalast1895-07-25@\strich\emph{Die Münchener Kunstausstellungen. II. Im königl. Glaspalast} {[}1895-07-25{]}|pwv}\pwindex{Muenchener Brief. (Orig.-Corr. der »Wiener Allg. Ztg.«)1895-07-06@\emph{Münchener Brief. (Orig.-Corr. der »Wiener Allg. Ztg.«)} {[}1895-07-06{]}|pwv}}{\lemma{\textnormal{\emph{Feuilletons}}}\Cendnote{\textnormal{f. s.\pwindex{Salten, Felix 06.09.1869 – 08.10.1945@\textsc{Salten, Felix} (06.09.1869 – 08.10.1945), \emph{Schriftsteller, Journalist}|pwk} [ = Felix Salten\pwindex{Salten, Felix 06.09.1869 – 08.10.1945@\textsc{Salten, Felix} (06.09.1869 – 08.10.1945), \emph{Schriftsteller, Journalist}|pwk}]: \emph{Münchener Brief. (Orig.-Corr. der »Wiener Allg. Ztg.«)}\pwindex{Muenchener Brief. (Orig.-Corr. der »Wiener Allg. Ztg.«)1895-07-06@\emph{Münchener Brief. (Orig.-Corr. der »Wiener Allg. Ztg.«)} {[}1895-07-06{]}|pwk}. In: \emph{Wiener Allgemeine Zeitung}\pwindex{?? Werk@Nicht ermittelte Verfasserinnen und Verfasser!Wiener Allgemeine Zeitung1.3.1880 – 11.2.1934@\emph{Wiener Allgemeine Zeitung} {[}1.3.1880 – 11.2.1934{]}|pwk}, Nr. 5.200, 6. 7. 1895, S. 8; Felix Salten\pwindex{Salten, Felix 06.09.1869 – 08.10.1945@\textsc{Salten, Felix} (06.09.1869 – 08.10.1945), \emph{Schriftsteller, Journalist}|pwk}: \emph{Die Münchener Kunstausstellungen. I. Im königl.
                        Glaspalast}\pwindex{Salten, Felix 06.09.1869 – 08.10.1945@\textsc{Salten, Felix} (06.09.1869 – 08.10.1945), \emph{Schriftsteller, Journalist}!Muenchener Kunstausstellungen. I. Im koenigl. Glaspalast1895-07-24@\strich\emph{Die Münchener Kunstausstellungen. I. Im königl. Glaspalast} {[}1895-07-24{]}|pwk}. In: \emph{ebd.}\pwindex{?? Werk@Nicht ermittelte Verfasserinnen und Verfasser!Wiener Allgemeine Zeitung1.3.1880 – 11.2.1934@\emph{Wiener Allgemeine Zeitung} {[}1.3.1880 – 11.2.1934{]}|pwk}, Nr. 5.215,
                        24. 7. 1895, S. 2; ders.\pwindex{Salten, Felix 06.09.1869 – 08.10.1945@\textsc{Salten, Felix} (06.09.1869 – 08.10.1945), \emph{Schriftsteller, Journalist}|pwk}: \emph{Die Münchener Kunstausstellungen. II. Im königl. Glaspalast}\pwindex{Salten, Felix 06.09.1869 – 08.10.1945@\textsc{Salten, Felix} (06.09.1869 – 08.10.1945), \emph{Schriftsteller, Journalist}!Muenchener Kunstausstellungen. II. Im koenigl. Glaspalast1895-07-25@\strich\emph{Die Münchener Kunstausstellungen. II. Im königl. Glaspalast} {[}1895-07-25{]}|pwk}. In: \emph{ebd.}\pwindex{?? Werk@Nicht ermittelte Verfasserinnen und Verfasser!Wiener Allgemeine Zeitung1.3.1880 – 11.2.1934@\emph{Wiener Allgemeine Zeitung} {[}1.3.1880 – 11.2.1934{]}|pwk}, Nr. 5.216, 25. 7. 1895, S. 2–3. Siehe Felix Salten an Arthur Schnitzler, 22. 7. 1895.}}}\label{K_L03160-4h}. Auch die für Goldmann\pwindex{Goldmann, Paul 31.01.1865 – 25.09.1935@\textsc{Goldmann, Paul} (31.01.1865 – 25.09.1935), \emph{Schriftsteller, Journalist}|pw} bestimmten, welche Sie absenden werden, falls \substVorne{}\textsuperscript{s}\substDazwischen{}es\substHinten{} noch Zeit ist, ja?\pend
           \pstart
           Also auf baldiges Wiedersehen, herzlichst {\\[\baselineskip]}Ihr \spacefill\mbox{Salten.}\pend
           \leftskip=0em{}
         
         \endnumbering\mylabel{h}\end{ledgroupsized}  \newcommand{\dateiname}{L03160}\newcommand{\titel}{Felix Salten an Arthur Schnitzler, [30. 7. 1895]}\newcommand{\editorInnen}{Martin Anton Müller und Laura Untner}%% latex-leseansicht-abspann.tex
%% Abspann für die Leseansicht.
%% Der Schalter \ifkorrekturansicht ist bereits durch den Vorspann gesetzt.

%% latex-abspann.tex
%% Gemeinsamer Abspann für Korrekturansicht und Leseansicht.
%% Setzt den Schalter \ifkorrekturansicht voraus (gesetzt in den
%% einbindenden Dateien latex-korrekturansicht-abspann.tex bzw.
%% latex-leseansicht-abspann.tex).
%% ---------------------------------------------------------------

\normalsize

% Das esempio-Environment wird nur in der Leseansicht benötigt
\ifkorrekturansicht\else
\newenvironment{esempio}[3]%
{
    \vspace{1.5ex}
    \rlap{\underline{#1}}
    \par
    \setlength{\parindent}{0cm}
    \nopagebreak
    \leftskip=#2cm
    \rightskip=#3cm
}
{
    \par
}
\fi

\doendnotes{C}
\bigskip
\vfill

\clearpage

\footnotesize

\ifkorrekturansicht
  \lohead{\textsc{register}}
\fi

% theindex-Environment neu definieren ohne reledmac
\makeatletter
\renewenvironment{theindex}{%
  \ifkorrekturansicht
    \section*{\indexname}%
  \else
    \subsubsection*{Index der erwähnten Entitäten}%
  \fi
  \setlength{\parindent}{0pt}%
  \setlength{\parskip}{0pt plus 0.3pt}%
  \let\item\@idxitem
}{%
  \ifkorrekturansicht\clearpage\fi
}
\makeatother

\IfFileExists{\jobname-pw.ind}{\input{\jobname-pw.ind}}{}

% Quellenangabe nur in der Leseansicht
\ifkorrekturansicht\else
% Fallback-Definitionen, falls die .tex-Datei \titel etc. nicht gesetzt hat
\providecommand{\titel}{}
\providecommand{\editorInnen}{}
\providecommand{\dateiname}{\jobname}

\vspace{3cm}

\vfill

\footnotesize
\textsc{Quelle}: \titel. Herausgegeben von {\editorInnen}. In: \emph{Arthur Schnitzler: Briefwechsel mit Autorinnen und Autoren}.
 Digitale Edition, https://schnitzler-briefe.acdh.oeaw.ac.at/{\dateiname}.html (Stand \today)
\fi

\end{document}


      