%% latex-korrekturansicht-vorspann.tex
%% Vorspann für die Korrekturansicht.
%% Lädt die gemeinsame Datei latex-vorspann.tex mit gesetztem Schalter.

\newif\ifkorrekturansicht
\korrekturansichttrue

\input{../tex-inputs/latex-vorspann}


\section[ Felix Salten an Arthur Schnitzler, {[}30. 7. 1895{]}]{L03160 Felix Salten an Arthur Schnitzler, {[}30. 7. 1895{]}}
\nopagebreak\mylabel{L03160v}
\rehead{ }\normalsize\beginnumbering\briefempfaengerindex{Schnitzler, Arthur@\textsc{Schnitzler, Arthur}!zzzSalten, Felix@\emph{von Felix Salten}!1895-07-302@{{[}30. 7. 1895{]}}|(be}
\toendnotes[C]{\smallbreak\pagebreak[2]}\Standort{CUL, Schnitzler, B 89, A 1.}
\physDesc{Brief, 1 Blatt, 4 Seiten, 1092 Zeichen
\newline{}Handschrift: Bleistift, lateinische Kurrent
\newline{}Schnitzler: mit Bleistift datiert: »30/7 95.« 
\newline{}Ordnung: mit Bleistift von unbekannter Hand nummeriert: »60« }\toendnotes[C]{\smallbreak}
\pstart
           \noindent{}{\pb}Lieber Freund! Dank für den Brief. Ich bin hier so auf mich allein
               gestellt, und durch alle die traurigen Agonie-Stimmungen die ich täglich mitmache, so
               herabgedrückt, dass ich es noch weit angenehmer empfinde, als Sie, wenn man mir {\pb}Briefe schreibt. Dass \label{K_L03160-1v}\edtext{Freiwild\pwindex{Freiwild. Schauspiel in 3 Akten@\emph{Freiwild. Schauspiel in 3 Akten}|pw} fortschreitet}{\lemma{\textnormal{\emph{Freiwild fortschreitet}}}\Cendnote{\textnormal{Am 15. 6. 1895 hatte Schnitzler
                  die Arbeit an \emph{Freiwild}\pwindex{Freiwild. Schauspiel in 3 Akten@\emph{Freiwild. Schauspiel in 3 Akten}|pwk} wiederaufgenommen. Am
                     2. 8. 1895
                  stellte er den ersten Akt\pwindex{Freiwild. Schauspiel in 3 Akten@\emph{Freiwild. Schauspiel in 3 Akten}|pwkv}
                  fertig.}}}\label{K_L03160-1} ist recht. Auch dem \label{K_L03160-2v}\edtext{Götterliebling\pwindex{Tod Georgs@\emph{Der Tod Georgs}|pw}}{\lemma{\textnormal{\emph{Götterliebling}}}\Cendnote{\textnormal{Richard Beer-Hofmann\pwindex{Beer-Hofmann, Richard 1866-07-11 – 1945-09-26@\textsc{Beer-Hofmann, Richard} (1866-07-11 – 1945-09-26), \emph{Schriftsteller/Schriftstellerin}|pwk} arbeitete in dieser
                  Zeit intensiv an der Erzählung, die er später unter dem Titel \emph{Der Tod Georgs}\pwindex{Tod Georgs@\emph{Der Tod Georgs}|pwk} publizierte.}}}\label{K_L03160-2} wär das schon sehr zu
               wünschen. Möchten doch beide Sachen\pwindex{Tod Georgs@\emph{Der Tod Georgs}|pwv}\pwindex{Freiwild. Schauspiel in 3 Akten@\emph{Freiwild. Schauspiel in 3 Akten}|pwv} bis zum Herbste fertig sein. Pusterthal\oindex{Pustertal@\textbf{Pustertal}, \emph{T.VAL}|pw} wäre sehr schön, ob wir uns nicht aber
               doch lieber ruhig in Ischl\oindex{Bad Ischl@\textbf{Bad Ischl}, \emph{P.PPL}|pw} aufhalten und in den
               gewissen behaglichen Parthien die \label{K_L03160-3v}\edtext{Gegend abfahren}{\lemma{\textnormal{\emph{Gegend abfahren}}}\Cendnote{\textnormal{Sie einigten sich auf
                  eine Radtour von Salzburg\oindex{Salzburg@\textbf{Salzburg}, \emph{A.ADM2}|pwk} nach München\oindex{Muenchen@\textbf{München}, \emph{P.PPLA}|pwk}, siehe Felix Salten an Arthur Schnitzler, 22. 7. 1895.}}}\label{K_L03160-3} wollen. Dann {\pb}noch Eins. Ich werde sehr
               gequält nach Rügen\oindex{Ruegen@\textbf{Rügen}, \emph{T.ISL}|pw} zu fahren. \uline{E.\pwindex{Kotter, Elisabeth *~1873@\textsc{Kotter, Elisabeth} (*~1873), \emph{Haushaltshilfe/Haushaltshilfe}|pwu}}, die in Heringsdorf\oindex{Heringsdorf@\textbf{Heringsdorf}, \emph{P.PPLA4}|pw} ist, schreibt
               rührende Briefe. Vielleicht finde ich mich also dann doch bestimmt so gegen den 27 od. 28. August dahin zu
               reisen. Aber das wird sich ja alles noch
               entscheiden, bis ich nach Ischl\oindex{Bad Ischl@\textbf{Bad Ischl}, \emph{P.PPL}|pw}{ }{\pb}komme. Vorerst freue ich
               mich auf den Montag, oder Sonntag. Ich verständige Sie jedenfalls noch vorher. Für heute sende ich die gewünschten \label{K_L03160-4v}\edtext{Feuilletons\pwindex{Muenchener Kunstausstellungen. I. Im koenigl. Glaspalast@\emph{Die Münchener Kunstausstellungen. I. Im königl. Glaspalast}|pwv}\pwindex{Muenchener Kunstausstellungen. II. Im koenigl. Glaspalast@\emph{Die Münchener Kunstausstellungen. II. Im königl. Glaspalast}|pwv}\pwindex{Muenchener Brief. (Orig.-Corr. der »Wiener Allg. Ztg.«)@\emph{Münchener Brief. (Orig.-Corr. der »Wiener Allg. Ztg.«)}|pwv}}{\lemma{\textnormal{\emph{Feuilletons}}}\Cendnote{\textnormal{f. s.\pwindex{Salten, Felix 06.09.1869 – 08.10.1945@\textsc{Salten, Felix} (06.09.1869 – 08.10.1945), \emph{Schriftsteller/Schriftstellerin, Journalist/Journalistin, Chefredakteur/Chefredakteurin}|pwk} [ = Felix Salten\pwindex{Salten, Felix 06.09.1869 – 08.10.1945@\textsc{Salten, Felix} (06.09.1869 – 08.10.1945), \emph{Schriftsteller/Schriftstellerin, Journalist/Journalistin, Chefredakteur/Chefredakteurin}|pwk}]: \emph{Münchener Brief. (Orig.-Corr. der »Wiener Allg. Ztg.«)}\pwindex{Muenchener Brief. (Orig.-Corr. der »Wiener Allg. Ztg.«)@\emph{Münchener Brief. (Orig.-Corr. der »Wiener Allg. Ztg.«)}|pwk}. In: \emph{Wiener Allgemeine Zeitung}\pwindex{Wiener Allgemeine Zeitung@\emph{Wiener Allgemeine Zeitung}|pwk}, Nr. 5200, 6. 7. 1895, S. 8; Felix Salten\pwindex{Salten, Felix 06.09.1869 – 08.10.1945@\textsc{Salten, Felix} (06.09.1869 – 08.10.1945), \emph{Schriftsteller/Schriftstellerin, Journalist/Journalistin, Chefredakteur/Chefredakteurin}|pwk}: \emph{Die Münchener Kunstausstellungen. I. Im königl.
                        Glaspalast}\pwindex{Muenchener Kunstausstellungen. I. Im koenigl. Glaspalast@\emph{Die Münchener Kunstausstellungen. I. Im königl. Glaspalast}|pwk}. In: \emph{Wiener Allgemeine Zeitung}\pwindex{Wiener Allgemeine Zeitung@\emph{Wiener Allgemeine Zeitung}|pwk}, Nr. 5215,
                        24. 7. 1895, S. 2; Felix Salten\pwindex{Salten, Felix 06.09.1869 – 08.10.1945@\textsc{Salten, Felix} (06.09.1869 – 08.10.1945), \emph{Schriftsteller/Schriftstellerin, Journalist/Journalistin, Chefredakteur/Chefredakteurin}|pwk}: \emph{Die Münchener Kunstausstellungen. II. Im königl. Glaspalast}\pwindex{Muenchener Kunstausstellungen. II. Im koenigl. Glaspalast@\emph{Die Münchener Kunstausstellungen. II. Im königl. Glaspalast}|pwk}. In: \emph{Wiener Allgemeine Zeitung}\pwindex{Wiener Allgemeine Zeitung@\emph{Wiener Allgemeine Zeitung}|pwk}, Nr. 5216, 25. 7. 1895, S. 2–3. Siehe Felix Salten an Arthur Schnitzler, 22. 7. 1895.}}}\label{K_L03160-4}. Auch die für Goldmann\pwindex{Goldmann, Paul 31.01.1865 – 25.09.1935@\textsc{Goldmann, Paul} (31.01.1865 – 25.09.1935), \emph{Schriftsteller/Schriftstellerin, Journalist/Journalistin}|pw} bestimmten, welche Sie absenden werden, falls \substVorne{}\textsuperscript{s}\substDazwischen{}es\substHinten{} noch Zeit ist, ja?\pend
           
\pstart
           Also auf baldiges Wiedersehen, herzlichst {\\[\baselineskip]}Ihr \spacefill\mbox{Salten.}\pend
           \leftskip=0em{}\selectlanguage{ngerman}\endnumbering\briefempfaengerindex{Schnitzler, Arthur@\textsc{Schnitzler, Arthur}!zzzSalten, Felix@\emph{von Felix Salten}!1895-07-302@{{[}30. 7. 1895{]}}|)be}\mylabel{L03160h}  \normalsize

\doendnotes{C}
\bigskip
\vfill

\clearpage

\footnotesize

\lohead{\textsc{register}}

% Definiere theindex-Environment komplett neu ohne reledmac
\makeatletter
\renewenvironment{theindex}{%
  \section*{\indexname}%
  \setlength{\parindent}{0pt}%
  \setlength{\parskip}{0pt plus 0.3pt}%
  \let\item\@idxitem
}{%
  \clearpage
}
\makeatother

\IfFileExists{\jobname-pw.ind}{\input{\jobname-pw.ind}}{}

\end{document}

      