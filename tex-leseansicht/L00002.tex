%% latex-korrekturansicht-vorspann.tex
%% Vorspann für die Korrekturansicht.
%% Lädt die gemeinsame Datei latex-vorspann.tex mit gesetztem Schalter.

\newif\ifkorrekturansicht
\korrekturansichttrue

\input{../tex-inputs/latex-vorspann}


\section[Jaques Joachim an Arthur Schnitzler, 31. 1. 1890]{L00002 Jaques Joachim an Arthur Schnitzler,31. 1. 1890}
\nopagebreak\mylabel{L00002v}
\rehead{ }\normalsize\beginnumbering\briefempfaengerindex{Schnitzler, Arthur@\textsc{Schnitzler, Arthur}!zzzJoachim, Jaques@\emph{von Jaques Joachim}!1890-01-311@{31. 1. 1890}|(be}
\toendnotes[C]{\smallbreak\pagebreak[2]}\Standort{DLA, A:Schnitzler, HS.NZ85.1.3571,1.}
\physDesc{Brief, 2 Blätter, 2 Seiten, 486 Zeichen
\newline{}Handschrift: schwarze Tinte, lateinische Kurrent
\newline{}Schnitzler: 1) mit Bleistift beschriftet: »\textsc{DrJoachim}«  2) mit rotem Buntstift »Mod Dicht\orgindex{Moderne Dichtung/Moderne Rundschau@Moderne Dichtung/Moderne Rundschau|pw}« und zwei Unterstreichungen}\toendnotes[C]{\smallbreak}
\pstart
           \raggedleft{}{\pb}Wien I Fr. J. Quai 31\oindex{Franz-Josefs-Kai@\textbf{Franz-Josefs-Kai}|pw}{\\}31. Januar 1890\pend
           
\pstart{}Sehr geehrter Herr Doctor!\pend\vspace{0.5em}
\pstart
           Unter Berufung auf Herrn D\textsuperscript{r}{ }Goldmann\pwindex{Goldmann, Paul 31.\,1.\,1865 Breslau – 25.\,9.\,1935 Wien@\textsc{Goldmann, Paul} (31.\,1.\,1865 Breslau – 25.\,9.\,1935 Wien), \emph{Schriftsteller, Journalist}|pw} erlaube ich mir als
					Redactions-Mitglied der in Brünn\oindex{Bruenn@\textbf{Brünn}|pw}
					erscheinenden \label{K_L00002-1v}\edtext{neuen
						Zeitschrift}{\lemma{\textnormal{\emph{neuen
						Zeitschrift}}}\Cendnote{\textnormal{Das erste Heft der
							\emph{Modernen Dichtung}\pwindex{Moderne Dichtung. Monatsschrift fuer Literatur und Kritik@\emph{Moderne Dichtung. Monatsschrift für Literatur und Kritik}|pwk} war am
							1. 1. 1891 erschienen.}}}\label{K_L00002-1} »Moderne Dichtung\pwindex{Moderne Dichtung. Monatsschrift fuer Literatur und Kritik@\emph{Moderne Dichtung. Monatsschrift für Literatur und Kritik}|pw}« zur Mitarbeiterschaft an derselben
					aufzufordern. Herr D\textsuperscript{r}{ }Goldmann\pwindex{Goldmann, Paul 31.\,1.\,1865 Breslau – 25.\,9.\,1935 Wien@\textsc{Goldmann, Paul} (31.\,1.\,1865 Breslau – 25.\,9.\,1935 Wien), \emph{Schriftsteller, Journalist}|pw}{ }{\pb}theilte mir freundlichst mit, daß Sie eine
					Novelle \label{K_L00002-2v}\edtext{»BelastetSEXref\pwindex{Schnitzler, Arthur 15.\,5.\,1862 Wien – 21.\,10.\,1931 ebd.@\textsc{Schnitzler, Arthur} (15.\,5.\,1862 Wien – 21.\,10.\,1931 ebd.), \emph{Schriftsteller*in, Mediziner*in}!Belastet@\strich\emph{Belastet}|pw}«}{\lemma{\textnormal{\emph{»Belastet«}}}\Cendnote{\textnormal{Die
						Novelle blieb zu Lebzeiten Schnitzlers ungedruckt. Eine Inhaltsangabe findet
							sich in der Autobiografie (Arthur Schnitzler: \emph{Jugend in Wien. Eine Autobiographie}SEXref\pwindex{Schnitzler, Arthur 15.\,5.\,1862 Wien – 21.\,10.\,1931 ebd.@\textsc{Schnitzler, Arthur} (15.\,5.\,1862 Wien – 21.\,10.\,1931 ebd.), \emph{Schriftsteller*in, Mediziner*in}!Jugend in Wien@\strich\emph{Jugend in Wien}|pwk}. Mit einem Nachwort
								von Friedrich Torberg. Wien, München,
								Zürich, New York:
								\emph{S. Fischer}{ }1968, S. 213).}}}\label{K_L00002-2} und
					einen CyclusSEXref\pwindex{Schnitzler, Arthur 15.\,5.\,1862 Wien – 21.\,10.\,1931 ebd.@\textsc{Schnitzler, Arthur} (15.\,5.\,1862 Wien – 21.\,10.\,1931 ebd.), \emph{Schriftsteller*in, Mediziner*in}!Anatol@\strich\emph{Anatol}|pwv} von
					Einaktern geschrieben haben – ich wäre sehr erfreut, wenn Sie sich entschliessen
					würden mir selbe bald zu übersenden.\pend
           
\pstart
           Hochachtungsvoll{\\[\baselineskip]}\spacefill\mbox{D\textsuperscript{r}JJoachim}\pend
           \leftskip=0em{}\selectlanguage{ngerman}\endnumbering\briefempfaengerindex{Schnitzler, Arthur@\textsc{Schnitzler, Arthur}!zzzJoachim, Jaques@\emph{von Jaques Joachim}!1890-01-311@{31. 1. 1890}|)be}\mylabel{L00002h}  \normalsize

\doendnotes{C}
\bigskip
\vfill

\clearpage

\footnotesize

\lohead{\textsc{register}}

% Definiere theindex-Environment komplett neu ohne reledmac
\makeatletter
\renewenvironment{theindex}{%
  \section*{\indexname}%
  \setlength{\parindent}{0pt}%
  \setlength{\parskip}{0pt plus 0.3pt}%
  \let\item\@idxitem
}{%
  \clearpage
}
\makeatother

\IfFileExists{\jobname-pw.ind}{\input{\jobname-pw.ind}}{}

\end{document}

      