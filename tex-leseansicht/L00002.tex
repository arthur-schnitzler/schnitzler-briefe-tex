%% latex-leseansicht-vorspann.tex
%% Vorspann für die Leseansicht.
%% Lädt die gemeinsame Datei latex-vorspann.tex mit nicht gesetztem Schalter.

\newif\ifkorrekturansicht
\korrekturansichtfalse

\input{../tex-inputs/latex-vorspann}


         
         \renewcommand{\erwaehntePersonen}{Personen: Paul Goldmann, Jaques Joachim}
         \renewcommand{\erwaehnteInstitutionen}{Institutionen: Moderne Dichtung/Moderne Rundschau}
         \renewcommand{\erwaehnteOrte}{Orte: Brünn, Franz-Josefs-Kai, Wien}
         \renewcommand{\erwaehnteWerke}{Werke: Anatol, Belastet, Jugend in Wien, Moderne Dichtung. Monatsschrift für Literatur und Kritik}
               \section[Jaques Joachim an Arthur Schnitzler, 31. 1. 1890]{ Jaques Joachim an Arthur Schnitzler, 31. 1. 1890}\nopagebreak\mylabel{v}\rehead{ }\begin{ledgroupsized}[t]{13cm}\normalsize\beginnumbering \toendnotes[C]{\smallbreak\pagebreak[2]} \Standort{DLA, A:Schnitzler, HS.NZ85.1.3571,1.}
\physDesc{Brief, 2 Blätter, 2 Seiten, 486 Zeichen
\newline{}Handschrift: schwarze Tinte, lateinische Kurrent
\newline{}Schnitzler: 1) mit Bleistift beschriftet: »\textsc{DrJoachim}«  2) mit rotem Buntstift »Mod Dicht\orgindex{Moderne Dichtung/Moderne Rundschau@Moderne Dichtung/Moderne Rundschau|pw}« und zwei Unterstreichungen}\toendnotes[C]{\smallbreak}\pstart
           \raggedleft{}{\pb}Wien I Fr. J. Quai 31\oindex{Franz-Josefs-Kai@\textbf{Franz-Josefs-Kai}|pw}{\\}31. Januar 1890\pend
           \pstart{}Sehr geehrter Herr Doctor!\pend\pstart
           Unter Berufung auf Herrn D\textsuperscript{r}{ }Goldmann\pwindex{Goldmann, Paul 31.01.1865 – 25.09.1935@\textsc{Goldmann, Paul} (31.01.1865 – 25.09.1935), \emph{Schriftsteller, Journalist}|pw} erlaube ich mir als
               Redactions-Mitglied der in Brünn\oindex{Bruenn@\textbf{Brünn}|pw} erscheinenden
                  \label{K_L00002-1v}\edtext{neuen Zeitschrift}{\lemma{\textnormal{\emph{neuen Zeitschrift}}}\Cendnote{\textnormal{Das erste Heft der \emph{Modernen Dichtung}\pwindex{Moderne Dichtung. Monatsschrift fuer Literatur und Kritik1890-01-01 – 1890-12-31@\emph{Moderne Dichtung. Monatsschrift für Literatur und Kritik} {[}1890-01-01 – 1890-12-31{]}|pwk} war am 1. 1. 1891
                  erschienen.}}}\label{K_L00002-1h} »Moderne Dichtung\pwindex{Moderne Dichtung. Monatsschrift fuer Literatur und Kritik1890-01-01 – 1890-12-31@\emph{Moderne Dichtung. Monatsschrift für Literatur und Kritik} {[}1890-01-01 – 1890-12-31{]}|pw}« zur
               Mitarbeiterschaft an derselben aufzufordern. Herr D\textsuperscript{r}{ }Goldmann\pwindex{Goldmann, Paul 31.01.1865 – 25.09.1935@\textsc{Goldmann, Paul} (31.01.1865 – 25.09.1935), \emph{Schriftsteller, Journalist}|pw}{ }{\pb}theilte mir freundlichst mit, daß Sie eine Novelle
                  \label{K_L00002-2v}\edtext{»Belastet\pwindex{Schnitzler, Arthur 15.05.1862 – 21.10.1931@\textsc{Schnitzler, Arthur} (15.05.1862 – 21.10.1931), \emph{Schriftsteller, Mediziner}!BelastetNone@\strich\emph{Belastet} {[}None{]}|pw}«}{\lemma{\textnormal{\emph{»Belastet«}}}\Cendnote{\textnormal{Die Novelle blieb zu
                  Lebzeiten Schnitzlers ungedruckt. Eine Inhaltsangabe findet sich in \emph{Jugend in Wien}\pwindex{Schnitzler, Arthur 15.05.1862 – 21.10.1931@\textsc{Schnitzler, Arthur} (15.05.1862 – 21.10.1931), \emph{Schriftsteller, Mediziner}!Jugend in Wien1968@\strich\emph{Jugend in Wien} {[}1968{]}|pwk}.}}}\label{K_L00002-2h} und einen Cyclus\pwindex{Schnitzler, Arthur 15.05.1862 – 21.10.1931@\textsc{Schnitzler, Arthur} (15.05.1862 – 21.10.1931), \emph{Schriftsteller, Mediziner}!Anatol1892-10-29@\strich\emph{Anatol} {[}1892-10-29{]}|pwv} von Einaktern
               geschrieben haben – ich wäre sehr erfreut, wenn Sie sich entschliessen würden mir
               selbe bald zu übersenden.\pend
           \pstart
           Hochachtungsvoll{\\[\baselineskip]}\spacefill\mbox{D\textsuperscript{r}JJoachim}\pend
           \leftskip=0em{}
         
         \endnumbering\mylabel{h}\end{ledgroupsized}  \newcommand{\dateiname}{L00002}\newcommand{\titel}{Jaques Joachim an Arthur Schnitzler, 31. 1. 1890}\newcommand{\editorInnen}{Martin Anton Müller und Gerd-Hermann Susen}%% latex-leseansicht-abspann.tex
%% Abspann für die Leseansicht.
%% Der Schalter \ifkorrekturansicht ist bereits durch den Vorspann gesetzt.

%% latex-abspann.tex
%% Gemeinsamer Abspann für Korrekturansicht und Leseansicht.
%% Setzt den Schalter \ifkorrekturansicht voraus (gesetzt in den
%% einbindenden Dateien latex-korrekturansicht-abspann.tex bzw.
%% latex-leseansicht-abspann.tex).
%% ---------------------------------------------------------------

\normalsize

% Das esempio-Environment wird nur in der Leseansicht benötigt
\ifkorrekturansicht\else
\newenvironment{esempio}[3]%
{
    \vspace{1.5ex}
    \rlap{\underline{#1}}
    \par
    \setlength{\parindent}{0cm}
    \nopagebreak
    \leftskip=#2cm
    \rightskip=#3cm
}
{
    \par
}
\fi

\doendnotes{C}
\bigskip
\vfill

\clearpage

\footnotesize

\ifkorrekturansicht
  \lohead{\textsc{register}}
\fi

% theindex-Environment neu definieren ohne reledmac
\makeatletter
\renewenvironment{theindex}{%
  \ifkorrekturansicht
    \section*{\indexname}%
  \else
    \subsubsection*{Index der erwähnten Entitäten}%
  \fi
  \setlength{\parindent}{0pt}%
  \setlength{\parskip}{0pt plus 0.3pt}%
  \let\item\@idxitem
}{%
  \ifkorrekturansicht\clearpage\fi
}
\makeatother

\IfFileExists{\jobname-pw.ind}{\input{\jobname-pw.ind}}{}

% Quellenangabe nur in der Leseansicht
\ifkorrekturansicht\else
% Fallback-Definitionen, falls die .tex-Datei \titel etc. nicht gesetzt hat
\providecommand{\titel}{}
\providecommand{\editorInnen}{}
\providecommand{\dateiname}{\jobname}

\vspace{3cm}

\vfill

\footnotesize
\textsc{Quelle}: \titel. Herausgegeben von {\editorInnen}. In: \emph{Arthur Schnitzler: Briefwechsel mit Autorinnen und Autoren}.
 Digitale Edition, https://schnitzler-briefe.acdh.oeaw.ac.at/{\dateiname}.html (Stand \today)
\fi

\end{document}


      