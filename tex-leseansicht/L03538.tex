%% latex-leseansicht-vorspann.tex
%% Vorspann für die Leseansicht.
%% Lädt die gemeinsame Datei latex-vorspann.tex mit nicht gesetztem Schalter.

\newif\ifkorrekturansicht
\korrekturansichtfalse

\input{../tex-inputs/latex-vorspann}


\section[ Paul Goldmann an Olga und Elisabeth Gussmann, 28. 12. {[}1900?{]}]{L03538 Paul Goldmann an Olga und Elisabeth Gussmann,  28. 12. [1900?]}
\nopagebreak\mylabel{L03538v}
\rehead{ }\normalsize\beginnumbering\briefempfaengerindex{Steinrück, Elisabeth@\textsc{Steinrück, Elisabeth}!zzzGoldmann, Paul@\emph{von Paul Goldmann}!1900-12-281@{28. 12. [1900?]}|(be}\briefempfaengerindex{Schnitzler, Olga@\textsc{Schnitzler, Olga}!zzzGoldmann, Paul@\emph{von Paul Goldmann}!1900-12-281@{28. 12. [1900?]}|(be}
\toendnotes[C]{\smallbreak\pagebreak[2]}
\correspDesc{Versand  durch Paul Goldmann am 28. 12. [1900?] in Frankfurt am Main
\newline{}Erhalt  durch Olga Gussmann, Elisabeth Gussmann im Zeitraum [29. 12. 1900 – 2. 1. 1901?] in Wien}\toendnotes[C]{\smallbreak}
\Standort{DLA, A:Schnitzler, HS.NZ85.1.5247.}
\physDesc{Brief, 1 Blatt, 4 Seiten, 1573 Zeichen
\newline{}Handschrift: blaue Tinte, deutsche Kurrent}\toendnotes[C]{\smallbreak}
\pstart
           Frankfurt\oindex{Frankfurt am Main@\textbf{Frankfurt am Main}, \emph{Hauptstadt}|pw}, 28. December.\hfill {\pb}\textcolor{gray}{\textbf{Reuterweg 59.\oindex{Reuterweg@\textbf{Reuterweg}, \emph{Straße}|pw}}}\pend
           
\pstart\center{}Liebes Fräulein \textsc{Olga},\pend\vspace{0.5em}
\pstart
           Ihr neues Briefpapier, das Herr \textsc{Paul\pwindex{Marx, Paul 21.\,7.\,1879 Wien – 30.\,10.\,1956 ebd.@\textsc{Marx, Paul} (21.\,7.\,1879 Wien – 30.\,10.\,1956 ebd.), \emph{Regisseur, Schauspieler}|pw}} Ihnen geſchenkt hat, iſt{ }ſehr{ }ſchön, und über Ihren \label{K_L03538-1v}\edtext{Erfolg}{\lemma{\textnormal{\emph{Erfolg}}}\Cendnote{\textnormal{Siehe A. S.: \emph{Tagebuch}, 21. 12. 1900.
               }}}\label{K_L03538-1} habe ich mich{ }ſehr gefreut. Ich habe es nicht anders erwartet, und ich meine,
               Sie{ }ſind auf dem Wege, etwas Großes zu werden. Laſſen Sie{ }ſich von Herzen
               beglückwünſchen! Die große Dummheit, die gewiſſe Leute gemacht haben, die ich näher
               kenne, – die Dummheit nämlich, dem Ehrgeiz allzuſehr nachzugeben und über dem Streben
               das Leben zu vergeſſen – werden Sie ja wohl vermeiden. Und{ }ſo iſt Alles gut. Ich bin
               zu Weihnachten in Frankfurt\oindex{Frankfurt am Main@\textbf{Frankfurt am Main}, \emph{Hauptstadt}|pw} bei Schweſter\pwindex{Rosengart, Vally 29.\,12.\,1866 Breslau – nach 1926@\textsc{Rosengart, Vally} (29.\,12.\,1866 Breslau – nach 1926)|pwv}, Schwager\pwindex{Rosengart, Josef 8.\,2.\,1860 Laupheim – 4.\,8.\,1927 Frankfurt am Main@\textsc{Rosengart, Josef} (8.\,2.\,1860 Laupheim – 4.\,8.\,1927 Frankfurt am Main), \emph{Arzt}|pwv}{ }{\pb}und Onkel\pwindex{Mamroth, Fedor 21.\,2.\,1851 Breslau – 25.\,6.\,1907 Frankfurt am Main@\textsc{Mamroth, Fedor} (21.\,2.\,1851 Breslau – 25.\,6.\,1907 Frankfurt am Main), \emph{Journalist, Kritiker}|pwv}. Hatte Allerlei von dieſem Aufenthalt gehofft. Aber
               vergebens. Traurig, wie ich gegangen, komme ich nach Berlin\oindex{Berlin@\textbf{Berlin}, \emph{Hauptstadt}|pw} zurück. Schreiben Sie mir bald wieder!\pend
           
\pstart
           Herzlichſt {\\[\baselineskip]}Ihr {\\[\baselineskip]}\spacefill\mbox{Dr. Paul Goldmann.}\pend
           \leftskip=0em{}
\pstart
           \noindent{}Bitte, grüßen Sie den \textsc{Dr. Schnitzler}!\pend
           \selectlanguage{ngerman}\vspace{1em}{\vspace{1\baselineskip}}
\pstart{}{\pb}Liebes Fräulein \textsc{Liesl},\pend\vspace{0.5em}
\pstart
           Einen Brief, den Sie mir{ }ſchreiben, brauchen Sie Ihrer Schweſter nicht zur Kritik
               vorzulegen. Das wäre noch{ }ſchöner! Schweſtern verſtehen nichts von Briefen!\pend
           
\pstart
           Der »\label{K_L03538-2v}\edtext{Roſenmontag\pwindex{\textcolor{red}{\textsuperscript{XXXX indx1}}!Rosenmontag@\strich\emph{Rosenmontag}|pw}}{\lemma{\textnormal{\emph{Rosenmontag}}}\Cendnote{\textnormal{Goldmann\pwindex{Goldmann, Paul 31.\,1.\,1865 Breslau – 25.\,9.\,1935 Wien@\textsc{Goldmann, Paul} (31.\,1.\,1865 Breslau – 25.\,9.\,1935 Wien), \emph{Schriftsteller, Journalist}|pwk} hatte das Stück\pwindex{\textcolor{red}{\textsuperscript{XXXX indx1}}!Rosenmontag@\strich\emph{Rosenmontag}|pwkv} am 26. 11. 1900 womöglich gemeinsam mit Schnitzler gesehen.}}}\label{K_L03538-2}« iſt ein
               blödſinniges Stück. \textsc{Altenberg\pwindex{Altenberg, Peter 9.\,3.\,1859 Wien – 8.\,1.\,1919 ebd.@\textsc{Altenberg, Peter} (9.\,3.\,1859 Wien – 8.\,1.\,1919 ebd.), \emph{Schriftsteller}|pw}}{ }ſollen Sie nicht leſen, \textsc{Maupassant\pwindex{Maupassant, Guy de 5.\,8.\,1850 Tourville-sur-Arques – 7.\,7.\,1893 Paris@\textsc{Maupassant, Guy de} (5.\,8.\,1850 Tourville-sur-Arques – 7.\,7.\,1893 Paris), \emph{Schriftsteller}|pw}}{ }ſo viel als möglich (obwohl Sie eigentlich noch zu jung dazu{ }ſind).\pend
           
\pstart
           Meine Mutter\pwindex{Goldmann, Clementine 15.\,5.\,1842 Breslau – 24.\,2.\,1924 Frankfurt am Main@\textsc{Goldmann, Clementine} (15.\,5.\,1842 Breslau – 24.\,2.\,1924 Frankfurt am Main)|pwv} iſt die Güte
               und Selbſtaufopferung in Perſon. Gerade das, was Sie brauchten. Ich aber bin wenig
               dankbar dafür und{ }ſehne mich nach etwas ganz, ganz Anderem, als nach einer
               Mutter.\pend
           
\pstart
           Nach Wien\oindex{Wien@\textbf{Wien}, \emph{Verwaltungsgebiet}|pw} werde ich lange nicht kommen. Wozu auch
               das ewige Herumreiſen? {\pb}Man fährt und fährt und
               kommt doch nicht weiter.\pend
           
\pstart
           Ihr Brief war{ }ſehr lieb. Ich bitte um einen andern.\pend
           
\pstart
           Grüß’ Sie Gott! {\\[\baselineskip]}Ihr {\\[\baselineskip]}\spacefill\mbox{Dr. Paul Goldmann.}\pend
           \leftskip=0em{}
\pstart
           \noindent{}Bitte, grüßen Sie den \textsc{Dr. Schnitzler}!\pend
           \selectlanguage{ngerman}\endnumbering\briefempfaengerindex{Steinrück, Elisabeth@\textsc{Steinrück, Elisabeth}!zzzGoldmann, Paul@\emph{von Paul Goldmann}!1900-12-281@{28. 12. [1900?]}|)be}\briefempfaengerindex{Schnitzler, Olga@\textsc{Schnitzler, Olga}!zzzGoldmann, Paul@\emph{von Paul Goldmann}!1900-12-281@{28. 12. [1900?]}|)be}\mylabel{L03538h}  \newcommand{\dateiname}{L03538}\newcommand{\titel}{Paul Goldmann an Olga und Elisabeth Gussmann, 28. 12. [1900?]}\newcommand{\editorInnen}{Martin Anton Müller und Laura Untner}%% latex-leseansicht-abspann.tex
%% Abspann für die Leseansicht.
%% Der Schalter \ifkorrekturansicht ist bereits durch den Vorspann gesetzt.

%% latex-abspann.tex
%% Gemeinsamer Abspann für Korrekturansicht und Leseansicht.
%% Setzt den Schalter \ifkorrekturansicht voraus (gesetzt in den
%% einbindenden Dateien latex-korrekturansicht-abspann.tex bzw.
%% latex-leseansicht-abspann.tex).
%% ---------------------------------------------------------------

\normalsize

% Das esempio-Environment wird nur in der Leseansicht benötigt
\ifkorrekturansicht\else
\newenvironment{esempio}[3]%
{
    \vspace{1.5ex}
    \rlap{\underline{#1}}
    \par
    \setlength{\parindent}{0cm}
    \nopagebreak
    \leftskip=#2cm
    \rightskip=#3cm
}
{
    \par
}
\fi

\doendnotes{C}
\bigskip
\vfill

\clearpage

\footnotesize

\ifkorrekturansicht
  \lohead{\textsc{register}}
\fi

% theindex-Environment neu definieren ohne reledmac
\makeatletter
\renewenvironment{theindex}{%
  \ifkorrekturansicht
    \section*{\indexname}%
  \else
    \subsubsection*{Index der erwähnten Entitäten}%
  \fi
  \setlength{\parindent}{0pt}%
  \setlength{\parskip}{0pt plus 0.3pt}%
  \let\item\@idxitem
}{%
  \ifkorrekturansicht\clearpage\fi
}
\makeatother

\IfFileExists{\jobname-pw.ind}{\input{\jobname-pw.ind}}{}

% Quellenangabe nur in der Leseansicht
\ifkorrekturansicht\else
% Fallback-Definitionen, falls die .tex-Datei \titel etc. nicht gesetzt hat
\providecommand{\titel}{}
\providecommand{\editorInnen}{}
\providecommand{\dateiname}{\jobname}

\vspace{3cm}

\vfill

\footnotesize
\textsc{Quelle}: \titel. Herausgegeben von {\editorInnen}. In: \emph{Arthur Schnitzler: Briefwechsel mit Autorinnen und Autoren}.
 Digitale Edition, https://schnitzler-briefe.acdh.oeaw.ac.at/{\dateiname}.html (Stand \today)
\fi

\end{document}


