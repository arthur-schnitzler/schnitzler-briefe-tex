%% latex-korrekturansicht-vorspann.tex
%% Vorspann für die Korrekturansicht.
%% Lädt die gemeinsame Datei latex-vorspann.tex mit gesetztem Schalter.

\newif\ifkorrekturansicht
\korrekturansichttrue

\input{../tex-inputs/latex-vorspann}


\section[ Paul Goldmann an Olga und Elisabeth Gussmann, 28. 12. {[}1900?{]}]{L03538 Paul Goldmann an Olga und Elisabeth Gussmann, 28. 12. {[}1900?{]}}
\nopagebreak\mylabel{L03538v}
\rehead{ }\normalsize\beginnumbering\briefempfaengerindex{Steinrueck, Elisabeth@\textsc{Steinrück, Elisabeth}!zzzGoldmann, Paul@\emph{von Paul Goldmann}!1900-12-281@{28. 12. {[}1900?{]}}|(be}\briefempfaengerindex{Schnitzler, Olga@\textsc{Schnitzler, Olga}!zzzGoldmann, Paul@\emph{von Paul Goldmann}!1900-12-281@{28. 12. {[}1900?{]}}|(be}
\toendnotes[C]{\smallbreak\pagebreak[2]}\Standort{DLA, A:Schnitzler, HS.NZ85.1.5247.}
\physDesc{Brief, 1 Blatt, 4 Seiten, 1573 Zeichen
\newline{}Handschrift: blaue Tinte, deutsche Kurrent}\toendnotes[C]{\smallbreak}
\pstart
           Frankfurt\oindex{Frankfurt am Main@\textbf{Frankfurt am Main}, \emph{P.PPLA3}|pw}, 28. December.\hfill {\pb}\textcolor{gray}{\textbf{Reuterweg 59.\oindex{Reuterweg@\textbf{Reuterweg}, \emph{Straße (K.STR)}|pw}}}\pend
           
\pstart\center{}Liebes Fräulein \textsc{Olga},\pend\vspace{0.5em}
\pstart
           Ihr neues Briefpapier, das Herr \textsc{Paul\pwindex{Marx, Paul 21.07.1879 – 1956-10-30@\textsc{Marx, Paul} (21.07.1879 – 1956-10-30), \emph{Regisseur/Regisseurin, Schauspieler/Schauspielerin}|pw}} Ihnen geſchenkt hat, iſt ſehr ſchön, und über Ihren \label{K_L03538-1v}\edtext{Erfolg}{\lemma{\textnormal{\emph{Erfolg}}}\Cendnote{\textnormal{Siehe A. S.: \emph{Tagebuch}, 21. 12. 1900.
               }}}\label{K_L03538-1} habe ich mich ſehr gefreut. Ich habe es nicht anders erwartet, und ich meine,
               Sie ſind auf dem Wege, etwas Großes zu werden. Laſſen Sie ſich von Herzen
               beglückwünſchen! Die große Dummheit, die gewiſſe Leute gemacht haben, die ich näher
               kenne, – die Dummheit nämlich, dem Ehrgeiz allzuſehr nachzugeben und über dem Streben
               das Leben zu vergeſſen – werden Sie ja wohl vermeiden. Und ſo iſt Alles gut. Ich bin
               zu Weihnachten in Frankfurt\oindex{Frankfurt am Main@\textbf{Frankfurt am Main}, \emph{P.PPLA3}|pw} bei Schweſter\pwindex{Rosengart, Vally 1866-12-29 – nach 1926@\textsc{Rosengart, Vally} (1866-12-29 – nach 1926)|pwv}, Schwager\pwindex{Rosengart, Josef 1860-02-08 – 1927-08-04@\textsc{Rosengart, Josef} (1860-02-08 – 1927-08-04), \emph{Arzt/Ärztin}|pwv}{ }{\pb}und Onkel\pwindex{Mamroth, Fedor 21.02.1851 – 25.06.1907@\textsc{Mamroth, Fedor} (21.02.1851 – 25.06.1907), \emph{Journalist/Journalistin, Kritiker/Kritikerin}|pwv}. Hatte Allerlei von dieſem Aufenthalt gehofft. Aber
               vergebens. Traurig, wie ich gegangen, komme ich nach Berlin\oindex{Berlin@\textbf{Berlin}, \emph{P.PPLC}|pw} zurück. Schreiben Sie mir bald wieder!\pend
           
\pstart
           Herzlichſt {\\[\baselineskip]}Ihr {\\[\baselineskip]}\spacefill\mbox{Dr. Paul Goldmann.}\pend
           \leftskip=0em{}
\pstart
           \noindent{}Bitte, grüßen Sie den \textsc{Dr. Schnitzler}!\pend
           \selectlanguage{ngerman}\vspace{1em}{\vspace{1\baselineskip}}
\pstart{}{\pb}Liebes Fräulein \textsc{Liesl},\pend\vspace{0.5em}
\pstart
           Einen Brief, den Sie mir ſchreiben, brauchen Sie Ihrer Schweſter nicht zur Kritik
               vorzulegen. Das wäre noch ſchöner! Schweſtern verſtehen nichts von Briefen!\pend
           
\pstart
           Der »\label{K_L03538-2v}\edtext{Roſenmontag\pwindex{Rosenmontag@\emph{Rosenmontag}|pw}}{\lemma{\textnormal{\emph{Roſenmontag}}}\Cendnote{\textnormal{Goldmann\pwindex{Goldmann, Paul 31.01.1865 – 25.09.1935@\textsc{Goldmann, Paul} (31.01.1865 – 25.09.1935), \emph{Schriftsteller/Schriftstellerin, Journalist/Journalistin}|pwk} hatte das Stück\pwindex{Rosenmontag@\emph{Rosenmontag}|pwkv} am 26. 11. 1900 womöglich gemeinsam mit Schnitzler gesehen.}}}\label{K_L03538-2}« iſt ein
               blödſinniges Stück. \textsc{Altenberg\pwindex{Altenberg, Peter 09.03.1859 – 08.01.1919@\textsc{Altenberg, Peter} (09.03.1859 – 08.01.1919), \emph{Schriftsteller/Schriftstellerin}|pw}} ſollen Sie nicht leſen, \textsc{Maupassant\pwindex{Maupassant, Guy de 05.08.1850 – 07.07.1893@\textsc{Maupassant, Guy de} (05.08.1850 – 07.07.1893), \emph{Schriftsteller/Schriftstellerin}|pw}} ſo viel als möglich (obwohl Sie eigentlich noch zu jung dazu ſind).\pend
           
\pstart
           Meine Mutter\pwindex{Goldmann, Clementine 1842-05-15 – 1924-02-24@\textsc{Goldmann, Clementine} (1842-05-15 – 1924-02-24)|pwv} iſt die Güte
               und Selbſtaufopferung in Perſon. Gerade das, was Sie brauchten. Ich aber bin wenig
               dankbar dafür und ſehne mich nach etwas ganz, ganz Anderem, als nach einer
               Mutter.\pend
           
\pstart
           Nach Wien\oindex{Wien@\textbf{Wien}, \emph{A.ADM2}|pw} werde ich lange nicht kommen. Wozu auch
               das ewige Herumreiſen? {\pb}Man fährt und fährt und
               kommt doch nicht weiter.\pend
           
\pstart
           Ihr Brief war ſehr lieb. Ich bitte um einen andern.\pend
           
\pstart
           Grüß’ Sie Gott! {\\[\baselineskip]}Ihr {\\[\baselineskip]}\spacefill\mbox{Dr. Paul Goldmann.}\pend
           \leftskip=0em{}
\pstart
           \noindent{}Bitte, grüßen Sie den \textsc{Dr. Schnitzler}!\pend
           \selectlanguage{ngerman}\endnumbering\briefempfaengerindex{Steinrueck, Elisabeth@\textsc{Steinrück, Elisabeth}!zzzGoldmann, Paul@\emph{von Paul Goldmann}!1900-12-281@{28. 12. {[}1900?{]}}|)be}\briefempfaengerindex{Schnitzler, Olga@\textsc{Schnitzler, Olga}!zzzGoldmann, Paul@\emph{von Paul Goldmann}!1900-12-281@{28. 12. {[}1900?{]}}|)be}\mylabel{L03538h}  \normalsize

\doendnotes{C}
\bigskip
\vfill

\clearpage

\footnotesize

\lohead{\textsc{register}}

% Definiere theindex-Environment komplett neu ohne reledmac
\makeatletter
\renewenvironment{theindex}{%
  \section*{\indexname}%
  \setlength{\parindent}{0pt}%
  \setlength{\parskip}{0pt plus 0.3pt}%
  \let\item\@idxitem
}{%
  \clearpage
}
\makeatother

\IfFileExists{\jobname-pw.ind}{\input{\jobname-pw.ind}}{}

\end{document}

      