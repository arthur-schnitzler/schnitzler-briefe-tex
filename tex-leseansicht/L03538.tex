%% latex-leseansicht-vorspann.tex
%% Vorspann für die Leseansicht.
%% Lädt die gemeinsame Datei latex-vorspann.tex mit nicht gesetztem Schalter.

\newif\ifkorrekturansicht
\korrekturansichtfalse

\input{../tex-inputs/latex-vorspann}

\begin{center}
            \textcolor{red}{ENTWURF, NICHT FERTIG KORRIGIERT}
                      \end{center}
            
         
         \renewcommand{\erwaehntePersonen}{Personen: Peter Altenberg, Paul Goldmann, Clementine Goldmann, Fedor Mamroth, Paul Marx, Guy de Maupassant, Vally Rosengart, Josef Rosengart, Olga Schnitzler, Elisabeth Steinrück}
         \renewcommand{\erwaehnteOrte}{Orte: Berlin, Frankfurt am Main, Reuterweg, Wien}
         \renewcommand{\erwaehnteWerke}{Werke: Rosenmontag}
               \section[ Paul Goldmann an Olga und Elisabeth Gussmann, 28. 12. {[}1900?{]}]{ Paul Goldmann an Olga und Elisabeth Gussmann, 28. 12. {[}1900?{]}}\nopagebreak\mylabel{v}\rehead{ }\begin{ledgroupsized}[t]{13cm}\normalsize\beginnumbering \toendnotes[C]{\smallbreak\pagebreak[2]} \Standort{DLA, A:Schnitzler, HS.NZ85.1.5247.}
\physDesc{Brief, 1 Blatt, 4 Seiten, 1568 Zeichen
\newline{}Handschrift: blaue Tinte, deutsche Kurrent}\toendnotes[C]{\smallbreak}\pstart
           \noindent{}Frankfurt\oindex{Frankfurt am Main@\textbf{Frankfurt am Main}|pw}, 28. December.\hfill {\pb}\textcolor{gray}{\textbf{Reuterweg 59.\oindex{Reuterweg@\textbf{Reuterweg}|pw}}}\pend
           \pstart\center{}Liebes Fräulein \textsc{Olga},\pend\pstart
           Ihr neues Briefgapier, das Herr \textsc{Paul\pwindex{Marx, Paul 21.07.1879 – 1956-10-30@\textsc{Marx, Paul} (21.07.1879 – 1956-10-30), \emph{Regisseur, Schauspieler}|pw}} Ihnen geſchenkt hat, iſt ſehr ſchön, und über Ihren \label{K_L03538-1v}\edtext{Erfolg}{\lemma{\textnormal{\emph{Erfolg}}}\Cendnote{\textnormal{siehe A. S.: \emph{Tagebuch}, 21. 12. 1900}}}\label{K_L03538-1h} habe ich mich ſehr gefreut. Ich habe es nicht anders erwartet, und ich meine,
               Sie ſind auf dem Wege, etwas Großes zu werden. Laſſen Sie ſich von Herzen
               beglückwünſchen! Die große Dummheit, die gewiſſe Leute gemacht haben, die ich näher
               kenne, – die Dummheit nämlich, dem Ehrgeiz allzuſehr nachzugeben und über dem Streben
               das Leben zu vergeſſen – werden Sie ja wohl vermeiden. Und ſo iſt Alles gut. Ich bin
               zu Weihnachten in Frankfurt\oindex{Frankfurt am Main@\textbf{Frankfurt am Main}|pw} bei Schweſter\pwindex{Rosengart, Vally *~1866-12-29@\textsc{Rosengart, Vally} (*~1866-12-29)|pwv}, Schwager\pwindex{Rosengart, Josef 1860-02-08 – 1927-08-04@\textsc{Rosengart, Josef} (1860-02-08 – 1927-08-04), \emph{Arzt}|pwv}{ }{\pb}und Onkel\pwindex{Mamroth, Fedor 21.02.1851 – 25.06.1907@\textsc{Mamroth, Fedor} (21.02.1851 – 25.06.1907), \emph{Journalist, Kritiker}|pwv}. Hatte Allerlei von dieſem Aufenthalt gehofft. Aber
               vergebens. Traurig, wie ich gegangen, komme ich nach Berlin\oindex{Berlin@\textbf{Berlin}|pw} zurück. Schreiben Sie mir bald wieder!\pend
           \pstart
           Herzlichſt {\\[\baselineskip]}Ihr {\\[\baselineskip]}\spacefill\mbox{Dr. Paul Goldmann.}\pend
           \leftskip=0em{}\pstart
           \noindent{}Bitte, grüßen Sie den \textsc{Dr. Schnitzler\pwindex{Schnitzler, Arthur 15.05.1862 – 21.10.1931@\textsc{Schnitzler, Arthur} (15.05.1862 – 21.10.1931), \emph{Schriftsteller, Mediziner}|pw}}!\pend
           \pstart{}{\pb}Liebes Fräulein \textsc{Liesl},\pend\pstart
           Einen Brief, den Sie mir ſchreiben, brauchen Sie Ihrer Schweſter nicht zur Kritik
               vorzulegen. Das wäre noch ſchöner! Schweſtern verſtehen nichts von Briefen!\pend
           \pstart
           Der \label{K_L03538-2v}\edtext{Roſenmontag\pwindex{\textcolor{red}{\textsuperscript{XXXX1 indx}}!Rosenmontag1900@\strich\emph{Rosenmontag} {[}1900{]}|pw}}{\lemma{\textnormal{\emph{Roſenmontag}}}\Cendnote{\textnormal{Da Goldmann\pwindex{Goldmann, Paul 31.01.1865 – 25.09.1935@\textsc{Goldmann, Paul} (31.01.1865 – 25.09.1935), \emph{Schriftsteller, Journalist}|pwk} das Stück\pwindex{\textcolor{red}{\textsuperscript{XXXX1 indx}}!Rosenmontag1900@\strich\emph{Rosenmontag} {[}1900{]}|pwkv}
                  am 26. 11. 1900
                  womöglich gemeinsam mit Schnitzler\pwindex{Schnitzler, Arthur 15.05.1862 – 21.10.1931@\textsc{Schnitzler, Arthur} (15.05.1862 – 21.10.1931), \emph{Schriftsteller, Mediziner}|pwk} gesehen
                  hatte und im Dezember 1900 jedenfalls in Frankfurt am Main\oindex{Frankfurt am Main@\textbf{Frankfurt am Main}|pwk} war (vgl. Paul Goldmann an Arthur Schnitzler, 27. 12. [1900]), ist es, vor allem in
                  Kombination mit dem Bezug auf Olga Gussmann\pwindex{Schnitzler, Olga 17.01.1882 – 13.01.1970@\textsc{Schnitzler, Olga} (17.01.1882 – 13.01.1970), \emph{Schauspielerin, Sängerin}|pwk}s
                     »Erfolg«, wahrscheinlich, dass der Brief aus dem Jahr 1900 stammt.}}}\label{K_L03538-2h} iſt ein blödſinniges Stück. \textsc{Altenberg\pwindex{Altenberg, Peter 09.03.1859 – 08.01.1919@\textsc{Altenberg, Peter} (09.03.1859 – 08.01.1919), \emph{Schriftsteller}|pw}} ſollen Sie nicht leſen, \textsc{Maupassant\pwindex{Maupassant, Guy de 05.08.1850 – 07.07.1893@\textsc{Maupassant, Guy de} (05.08.1850 – 07.07.1893), \emph{Schriftsteller}|pw}} ſo viel als möglich (obwohl Sie eigentlich noch zu jung dazu ſind).\pend
           \pstart
           Meine Mutter\pwindex{Goldmann, Clementine 1842-05-15 – 1924-02-24@\textsc{Goldmann, Clementine} (1842-05-15 – 1924-02-24)|pwv} iſt die Güte
               und Selbſtaufopferung in Perſon. Gerade das, was Sie brauchten. Ich aber bin wenig
               dankbar dafür und ſehne mich nach etwas ganz, ganz Anderem, als nach einer
               Mutter.\pend
           \pstart
           Nach Wien\oindex{Wien@\textbf{Wien}|pw} werde ich lange nicht kommen. Wozu auch
               das ewige Herumreiſen? {\pb}Man fährt und fährt und
               kommt doch nicht weiter.\pend
           \pstart
           Ihr Brief war ſehr lieb. Ich bitte um einen andern.\pend
           \pstart
           Grüß’ Sie Gott! {\\[\baselineskip]}Ihr {\\[\baselineskip]}\spacefill\mbox{Dr. Paul Goldmann.}\pend
           \leftskip=0em{}\pstart
           \noindent{}Bitte, grüßen Sie den \textsc{Dr. Schnitzler\pwindex{Schnitzler, Arthur 15.05.1862 – 21.10.1931@\textsc{Schnitzler, Arthur} (15.05.1862 – 21.10.1931), \emph{Schriftsteller, Mediziner}|pw}}!\pend
           
         
         \endnumbering\mylabel{h}\end{ledgroupsized}\begin{anhang}\end{anhang}\newcommand{\dateiname}{L03538}\newcommand{\titel}{Paul Goldmann an Olga und Elisabeth Gussmann, 28. 12. [1900?]}\newcommand{\editorInnen}{Martin Anton Müller und Laura Untner}%% latex-leseansicht-abspann.tex
%% Abspann für die Leseansicht.
%% Der Schalter \ifkorrekturansicht ist bereits durch den Vorspann gesetzt.

%% latex-abspann.tex
%% Gemeinsamer Abspann für Korrekturansicht und Leseansicht.
%% Setzt den Schalter \ifkorrekturansicht voraus (gesetzt in den
%% einbindenden Dateien latex-korrekturansicht-abspann.tex bzw.
%% latex-leseansicht-abspann.tex).
%% ---------------------------------------------------------------

\normalsize

% Das esempio-Environment wird nur in der Leseansicht benötigt
\ifkorrekturansicht\else
\newenvironment{esempio}[3]%
{
    \vspace{1.5ex}
    \rlap{\underline{#1}}
    \par
    \setlength{\parindent}{0cm}
    \nopagebreak
    \leftskip=#2cm
    \rightskip=#3cm
}
{
    \par
}
\fi

\doendnotes{C}
\bigskip
\vfill

\clearpage

\footnotesize

\ifkorrekturansicht
  \lohead{\textsc{register}}
\fi

% theindex-Environment neu definieren ohne reledmac
\makeatletter
\renewenvironment{theindex}{%
  \ifkorrekturansicht
    \section*{\indexname}%
  \else
    \subsubsection*{Index der erwähnten Entitäten}%
  \fi
  \setlength{\parindent}{0pt}%
  \setlength{\parskip}{0pt plus 0.3pt}%
  \let\item\@idxitem
}{%
  \ifkorrekturansicht\clearpage\fi
}
\makeatother

\IfFileExists{\jobname-pw.ind}{\input{\jobname-pw.ind}}{}

% Quellenangabe nur in der Leseansicht
\ifkorrekturansicht\else
% Fallback-Definitionen, falls die .tex-Datei \titel etc. nicht gesetzt hat
\providecommand{\titel}{}
\providecommand{\editorInnen}{}
\providecommand{\dateiname}{\jobname}

\vspace{3cm}

\vfill

\footnotesize
\textsc{Quelle}: \titel. Herausgegeben von {\editorInnen}. In: \emph{Arthur Schnitzler: Briefwechsel mit Autorinnen und Autoren}.
 Digitale Edition, https://schnitzler-briefe.acdh.oeaw.ac.at/{\dateiname}.html (Stand \today)
\fi

\end{document}


      