%% latex-leseansicht-vorspann.tex
%% Vorspann für die Leseansicht.
%% Lädt die gemeinsame Datei latex-vorspann.tex mit nicht gesetztem Schalter.

\newif\ifkorrekturansicht
\korrekturansichtfalse

\input{../tex-inputs/latex-vorspann}

\begin{center}
            \textcolor{red}{ENTWURF, NICHT FERTIG KORRIGIERT}
                      \end{center}
            
         
         \renewcommand{\erwaehntePersonen}{Personen: Peter Altenberg, Olga Schnitzler}
         \renewcommand{\erwaehnteOrte}{Orte: Berlin, Frankfurt am Main, Wien}
         \renewcommand{\erwaehnteWerke}{}
               \section[ Paul Goldmann an Olga XXXX Gussmann/Schnitzler, 28. 12. {[}XXXX{]}]{ Paul Goldmann an Olga XXXX Gussmann/Schnitzler, 28. 12. {[}XXXX{]}}\nopagebreak\mylabel{v}\rehead{ }\begin{ledgroupsized}[t]{13cm}\normalsize\beginnumbering \toendnotes[C]{\smallbreak\pagebreak[2]} \Standort{DLA, A:Schnitzler, HS.1985.1.5247.}
\physDesc{,  Blätter,  Seiten
\newline{}Handschrift: , deutsche Kurrent}\toendnotes[C]{\smallbreak}\pstart
           \noindent{}{\pb}\pend
           \textcolor{gray}{\textbf{Reuterweg 59.}}\textcolor{red}{\textsuperscript{\textbf{KEY}}}\pstart
           Frankfurt\oindex{Frankfurt am Main@\textbf{Frankfurt am Main}|pw}, 28.
                  December.\pend
           \pstart{}Liebes Fräulein \textsc{Olga},\pend\pstart
           Ihr neues Briefgapier, das Herr \textsc{Paul\textcolor{red}{\textsuperscript{\textbf{KEY}}}} Ihnen geſchenkt hat, iſt ſehr ſchön, und über Ihren Erfolg habe ich mich ſehr
               gefreut. Ich habe es nicht anders erwartet, und ich meine, Sie ſind auf dem Wege,
               etwas Großes zu werden. Laſſen Sie ſich von Herzen beglückwünſchen! Die große
               Dummheit, die gewiſſe Leute gemacht haben, die ich näher kenne, – die Dummheit
               nämlich, dem Ehrgeiz allzuſehr nachzugeben und über dem Streben das Leben zu
               vergeſſen – werden Sie ja wohl vermeiden. Und ſo iſt Alles gut. Ich bin zu
                  Weihnachten in Frankfurt\oindex{Frankfurt am Main@\textbf{Frankfurt am Main}|pw} bei Schweſter\textcolor{red}{\textsuperscript{\textbf{KEY}}}, Schwager\textcolor{red}{\textsuperscript{\textbf{KEY}}}{\pb} und Onkel\textcolor{red}{\textsuperscript{\textbf{KEY}}}. Hatte Allerlei von dieſem Aufenthalt gehofft.
               Aber vergebens. Traurig, wie ich gegangen, komme ich nach Berlin\oindex{Berlin@\textbf{Berlin}|pw} zurück. Schreiben Sie mir bald wieder!
               {\\[\baselineskip]}Herzlichſt\pend
           \leftskip=0em{}\pstart
           {\\[\baselineskip]}Ihr\pend
           \leftskip=0em{}\pstart
           {\\[\baselineskip]}\spacefill\mbox{Dr. Paul Goldmann.}\pend
           \leftskip=0em{}Bitte, grüßen Sie den \textsc{Dr. Schnitzler\pwindex{Schnitzler, Arthur 15.05.1862 – 21.10.1931@\textsc{Schnitzler, Arthur} (15.05.1862 – 21.10.1931), \emph{Schriftsteller, Mediziner}|pw}}!{\pb}\pstart{}Liebes Fräulein \textsc{Liesl},\pend\pstart
           Einen Brief, den Sie mir ſchreiben, brauchen Sie Ihrer Schweſter\textcolor{red}{\textsuperscript{\textbf{KEY}}} nicht zur Kritik vorzulegen. Das wäre noch
               ſchöner! Schweſtern verſtehen nichts von Briefen!\pend
           \pstart
           Der Roſenmontag\textcolor{red}{\textsuperscript{\textbf{KEY}}} iſt ein blödſinniges Stück. \textsc{Altenberg\pwindex{Altenberg, Peter 09.03.1859 – 08.01.1919@\textsc{Altenberg, Peter} (09.03.1859 – 08.01.1919), \emph{Schriftsteller}|pw}} ſollen Sie nicht leſen, \textsc{Maupassant\textcolor{red}{\textsuperscript{\textbf{KEY}}}} ſo viel als möglich (obwohl Sie eigentlich noch zu jung dazu ſind).\pend
           \pstart
           Meine Mutter iſt die Güte und Selbſtaufopferung in Perſon. Gerade das, was Sie
               brauchten. Ich aber bin wenig dankbar dafür und ſehne mich nach etwas ganz, ganz
               Anderem, als nach einer Mutter.\pend
           \pstart
           Nach Wien\oindex{Wien@\textbf{Wien}|pw} werde ich lange nicht kommen. Wozu auch
               das ewige Herumreiſen? {\pb}Man
               fährt und fährt und kommt doch nicht weiter.\pend
           \pstart
           Ihr Brief war ſehr lieb. Ich bitte um einen andern. {\\[\baselineskip]}Grüß’ Sie
               Gott!\pend
           \leftskip=0em{}\pstart
           {\\[\baselineskip]}Ihr\pend
           \leftskip=0em{}\pstart
           {\\[\baselineskip]}\spacefill\mbox{Dr. Paul Goldmann.}\pend
           \leftskip=0em{}\pstart
           Bitte, grüßen Sie den \textsc{Dr. Schnitzler\pwindex{Schnitzler, Arthur 15.05.1862 – 21.10.1931@\textsc{Schnitzler, Arthur} (15.05.1862 – 21.10.1931), \emph{Schriftsteller, Mediziner}|pw}}!\pend
           
         
         \endnumbering\mylabel{h}\end{ledgroupsized}\begin{anhang}\end{anhang}\newcommand{\dateiname}{L03538}\newcommand{\titel}{Paul Goldmann an Olga XXXX Gussmann/Schnitzler, 28. 12. [XXXX]}\newcommand{\editorInnen}{Martin Anton Müller und Laura Untner}%% latex-leseansicht-abspann.tex
%% Abspann für die Leseansicht.
%% Der Schalter \ifkorrekturansicht ist bereits durch den Vorspann gesetzt.

%% latex-abspann.tex
%% Gemeinsamer Abspann für Korrekturansicht und Leseansicht.
%% Setzt den Schalter \ifkorrekturansicht voraus (gesetzt in den
%% einbindenden Dateien latex-korrekturansicht-abspann.tex bzw.
%% latex-leseansicht-abspann.tex).
%% ---------------------------------------------------------------

\normalsize

% Das esempio-Environment wird nur in der Leseansicht benötigt
\ifkorrekturansicht\else
\newenvironment{esempio}[3]%
{
    \vspace{1.5ex}
    \rlap{\underline{#1}}
    \par
    \setlength{\parindent}{0cm}
    \nopagebreak
    \leftskip=#2cm
    \rightskip=#3cm
}
{
    \par
}
\fi

\doendnotes{C}
\bigskip
\vfill

\clearpage

\footnotesize

\ifkorrekturansicht
  \lohead{\textsc{register}}
\fi

% theindex-Environment neu definieren ohne reledmac
\makeatletter
\renewenvironment{theindex}{%
  \ifkorrekturansicht
    \section*{\indexname}%
  \else
    \subsubsection*{Index der erwähnten Entitäten}%
  \fi
  \setlength{\parindent}{0pt}%
  \setlength{\parskip}{0pt plus 0.3pt}%
  \let\item\@idxitem
}{%
  \ifkorrekturansicht\clearpage\fi
}
\makeatother

\IfFileExists{\jobname-pw.ind}{\input{\jobname-pw.ind}}{}

% Quellenangabe nur in der Leseansicht
\ifkorrekturansicht\else
% Fallback-Definitionen, falls die .tex-Datei \titel etc. nicht gesetzt hat
\providecommand{\titel}{}
\providecommand{\editorInnen}{}
\providecommand{\dateiname}{\jobname}

\vspace{3cm}

\vfill

\footnotesize
\textsc{Quelle}: \titel. Herausgegeben von {\editorInnen}. In: \emph{Arthur Schnitzler: Briefwechsel mit Autorinnen und Autoren}.
 Digitale Edition, https://schnitzler-briefe.acdh.oeaw.ac.at/{\dateiname}.html (Stand \today)
\fi

\end{document}


      