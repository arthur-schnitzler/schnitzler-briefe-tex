%% latex-leseansicht-vorspann.tex
%% Vorspann für die Leseansicht.
%% Lädt die gemeinsame Datei latex-vorspann.tex mit nicht gesetztem Schalter.

\newif\ifkorrekturansicht
\korrekturansichtfalse

\input{../tex-inputs/latex-vorspann}


\section[Sigmund Freud an Arthur Schnitzler, 14. 5. 1912]{L03886 Sigmund Freud an Arthur Schnitzler, 14. 5. 1912}
\nopagebreak\mylabel{L03886v}
\rehead{ }\normalsize\beginnumbering\briefempfaengerindex{Schnitzler, Arthur@\textsc{Schnitzler, Arthur}!zzzFreud, Sigmund@\emph{von Sigmund Freud}!1912-05-142@{14. 5. 1912}|(be}
\toendnotes[C]{\smallbreak\pagebreak[2]}
\correspDesc{Versand  durch Sigmund Freud am 14. 5. 1912 in Wien
\newline{}Erhalt  durch Arthur Schnitzler im Zeitraum [14. 5. 1912
                  – 17. 5. 1912?] in Wien}\toendnotes[C]{\smallbreak}
\Standort{–, Privatbesitz, –.}
\physDesc{Brief, 1 Blatt, 2 Seiten, 1503 Zeichen
\newline{}Handschrift: schwarze Tinte, deutsche Kurrent
\newline{}Schnitzler: mit rotem Buntstift eine Streichung seitlich der Datumsangabe 
\newline{}Zusatz: Der derzeitige Aufbewahrungsort des Briefes ist nicht bekannt.
                                 Zum Zeitpunkt der ersten Edition (1955) befand er sich im Besitz
                                 von Heinrich Schnitzler. 1967 wurde die erste Seite reproduziert, als Leihgeber wurde
                                 dabei das Auktionshaus Stargardt genannt. Der Brief wurde sowohl 2004 (Katalog 680,
                                 Lot 395) wie 2017 (Katalog 704, Lot 301) von Stargardt versteigert.
                                 Die Wiedergabe der ersten Seite folgt dem Katalogfaksimile von
                                 Stargardt 2017. }\Standort{Washington, DC, Library of Congress, Freud Archives, C41F8.}
\physDesc{Brief, Fotokopie, 2 Blätter, 2 Seiten, 1503 Zeichen
\newline{}Handschrift: schwarze Tinte, deutsche Kurrent}
\buchAbdrucke{\weitereDrucke{1) Sigmund Freud: \emph{Briefe an Arthur Schnitzler.}Herausgegeben von Henry Schnitzler In: \emph{Neue deutsche Rundschau}, Jg. 66 (Januar 1955) Nr. 1, S. 95–96.} \weitereDrucke{2) \emph{[Faksimile des ersten Blattes].} In: \emph{Die Brücke. Eine Hauszeitschrift der Pharmazeutisch-Medizinischen Abteilung der
                     Farbwerke Hoechst}, Nr. 27, Februar 1967, S. [IV. Umschlagseite].} \weitereDrucke{3) Sigmund Freud: \emph{Sigmund Freud Edition. Digitale historisch-kritische
                        Gesamtausgabe}. Herausgegeben von Christine Diercks, Arkadi Blatow und Elisabeth Skale. (2014–2025) \url{https://www.freudedition.net/briefe/freud-sigmund/schnitzler-arthur/1912/05/14}.} }\toendnotes[C]{\smallbreak}
\pstart
           \raggedleft{}{\pb}14. 5. 12\pend
           
\pstart
           \textcolor{gray}{\textbf{PROF. D\textsuperscript{R.} FREUD}}\hfill \textcolor{gray}{\textbf{WIEN, IX. BERGGASSE 19\oindex{Wien@\textbf{Wien}!IX., Alsergrund@\textbf{IX., Alsergrund}!Berggasse 19@\textbf{Berggasse 19}, \emph{Wohngebäude}|pw}. }}\pend
           
\pstart\center{}Verehrter Herr College\pend\vspace{0.5em}
\pstart
           Geſtatten Sie mir, die obige Anrede durch die Berufung auf Ihr \label{K_L03886-1v}\edtext{\textsc{rite}}{\lemma{\textnormal{\emph{rite}}}\Cendnote{\textnormal{lateinisch: rechtmäßig}}}\label{K_L03886-1} erworbenes
               Doktordiplom der Medizin zu rechtfertigen und dann mich unter die vielen
               Glückwünſchenden zu mengen, die Ihren 50ſten Geburtstag feiern wollen.\pend
           
\pstart
           Es iſt mehr als ein Akt der Revanche von meiner Seite. Ich glaube mich zu erinnern,
               daß ich in der \label{K_L03886-2v}\edtext{Antwort}{\lemma{\textnormal{\emph{Antwort}}}\Cendnote{\textnormal{XXXX Auszeichnungsfehler: Dokument L03819 nicht gefunden. }}}\label{K_L03886-2} auf Ihre liebenswürdige
                  \label{K_L03886-3v}\edtext{Zuſchrift}{\lemma{\textnormal{\emph{Zuschrift}}}\Cendnote{\textnormal{XXXX Auszeichnungsfehler: Dokument L03815 nicht gefunden. }}}\label{K_L03886-3} bei analogem Anlaße vor 6
               Jahren ausgeführt habe, wie{ }ſehr ich immer Ihrer Teilnahme und Ihres Verständnißes
               bei meinen Arbeiten{ }ſicher geweſen bin, obwol ich \label{K_L03886-55v}\edtext{niemals in die Lage gekommen bin,
               ein Wort mit Ihnen zu wechſeln}{\lemma{\textnormal{\emph{niemals … wechseln}}}\Cendnote{\textnormal{Das dürfte eine Übertreibung sein, nachdem 
               die Bekanntschaft nachweislich bis in die 1880erjahre zurückverfolgbar ist und Schnitzler
                  für den A. S.: \emph{Tagebuch}, 16. 6. 1922 in seinem \emph{Tagebuch}\pwindex{Schnitzler, Arthur 15.\,5.\,1862 Wien – 21.\,10.\,1931 ebd.@\textsc{Schnitzler, Arthur} (15.\,5.\,1862 Wien – 21.\,10.\,1931 ebd.), \emph{Schriftsteller, Mediziner}!Tagebuch@\strich\emph{Tagebuch}|pwk} notierte:
                  »Hatte ihn bisher nur ein paar Mal flüchtig gesprochen. –«.
               }}}\label{K_L03886-55}. Ebenſo, habe ich mich immer zu denen gerechnet, die
               Ihre{ }ſchönen und ernſten poetiſchen Schöpfungen in ganz beſonderem Maße verſtehen und
               genießen können. Ja, ich habe mir eingebildet, daß ein Reflex der thörichten und
               frevelhaften Gering{\pb}ſchätzung, welche die Menſchen
               heute für die Erotik bereit halten, auch auf Ihr Wirken gefallen{ }ſei, und daß Sie mir
               darum beſonders wert{ }ſein dürften.\pend
           
\pstart
           Lachen Sie nicht darüber, daß ich{ }ſo in die Lage komme, die feiernde Mitwelt an
               dieſem Tage bei Ihnen zu verſchwärzen – oder beſſer, lachen Sie nur darüber und
               denken Sie daß keiner von uns von{ }ſeinen »Komplexen« frei kommt, wie meine Freunde{ }ſagen.\pend
           
\pstart
           Zum Schluße aber – ich weiß nicht, ob Sie dieſes Troſtes bedürfen – laſſen Sie{ }ſich{ }ſagen, daß der Dichter{ }ſpäter altert als gewöhnliche Menſchenkinder, und daß nach dem
               Dichter noch der Denker herauskommt.\pend
           
\pstart
           Mit herzlichen Glückwünſchen{\\[\baselineskip]} Ihr ergebener{\\[\baselineskip]}\spacefill\mbox{Freud}\pend
           \leftskip=0em{}\selectlanguage{ngerman}\endnumbering\briefempfaengerindex{Schnitzler, Arthur@\textsc{Schnitzler, Arthur}!zzzFreud, Sigmund@\emph{von Sigmund Freud}!1912-05-142@{14. 5. 1912}|)be}\mylabel{L03886h}
\begin{anhang}
\end{anhang}\newcommand{\dateiname}{L03886}\newcommand{\titel}{Sigmund Freud an Arthur Schnitzler, 14. 5. 1912}\newcommand{\editorInnen}{Selma Jahnke und Martin Anton Müller}%% latex-leseansicht-abspann.tex
%% Abspann für die Leseansicht.
%% Der Schalter \ifkorrekturansicht ist bereits durch den Vorspann gesetzt.

%% latex-abspann.tex
%% Gemeinsamer Abspann für Korrekturansicht und Leseansicht.
%% Setzt den Schalter \ifkorrekturansicht voraus (gesetzt in den
%% einbindenden Dateien latex-korrekturansicht-abspann.tex bzw.
%% latex-leseansicht-abspann.tex).
%% ---------------------------------------------------------------

\normalsize

% Das esempio-Environment wird nur in der Leseansicht benötigt
\ifkorrekturansicht\else
\newenvironment{esempio}[3]%
{
    \vspace{1.5ex}
    \rlap{\underline{#1}}
    \par
    \setlength{\parindent}{0cm}
    \nopagebreak
    \leftskip=#2cm
    \rightskip=#3cm
}
{
    \par
}
\fi

\doendnotes{C}
\bigskip
\vfill

\clearpage

\footnotesize

\ifkorrekturansicht
  \lohead{\textsc{register}}
\fi

% theindex-Environment neu definieren ohne reledmac
\makeatletter
\renewenvironment{theindex}{%
  \ifkorrekturansicht
    \section*{\indexname}%
  \else
    \subsubsection*{Index der erwähnten Entitäten}%
  \fi
  \setlength{\parindent}{0pt}%
  \setlength{\parskip}{0pt plus 0.3pt}%
  \let\item\@idxitem
}{%
  \ifkorrekturansicht\clearpage\fi
}
\makeatother

\IfFileExists{\jobname-pw.ind}{\input{\jobname-pw.ind}}{}

% Quellenangabe nur in der Leseansicht
\ifkorrekturansicht\else
% Fallback-Definitionen, falls die .tex-Datei \titel etc. nicht gesetzt hat
\providecommand{\titel}{}
\providecommand{\editorInnen}{}
\providecommand{\dateiname}{\jobname}

\vspace{3cm}

\vfill

\footnotesize
\textsc{Quelle}: \titel. Herausgegeben von {\editorInnen}. In: \emph{Arthur Schnitzler: Briefwechsel mit Autorinnen und Autoren}.
 Digitale Edition, https://schnitzler-briefe.acdh.oeaw.ac.at/{\dateiname}.html (Stand \today)
\fi

\end{document}


