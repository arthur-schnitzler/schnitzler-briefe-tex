%% latex-korrekturansicht-vorspann.tex
%% Vorspann für die Korrekturansicht.
%% Lädt die gemeinsame Datei latex-vorspann.tex mit gesetztem Schalter.

\newif\ifkorrekturansicht
\korrekturansichttrue

\input{../tex-inputs/latex-vorspann}


\section[Arthur Schnitzler an Richard Beer-Hofmann, 1. 6. 1899]{L00921 Arthur Schnitzler an Richard Beer-Hofmann, 1. 6. 1899}
\nopagebreak\mylabel{L00921v}
\rehead{ }\normalsize\beginnumbering\briefempfaengerindex{Beer-Hofmann, Richard@\textsc{Beer-Hofmann, Richard}!zzzSchnitzler, Arthur@\emph{von Arthur Schnitzler}!1899-06-011@{1. 6. 1899}|(be}
\toendnotes[C]{\smallbreak\pagebreak[2]}\Standort{YCGL, MSS 31.}
\physDesc{Brief, 2 Blätter, 8 Seiten, Umschlag, 2400 Zeichen
\newline{}Handschrift: 1) Bleistift, deutsche Kurrent\hspace{1em}2) schwarze Tinte, deutsche Kurrent (\noindent{}Umschlag)\hspace{1em}
\newline{}Versand: 1) Stempel: »\nobreak{}\oindex{IX., Alsergrund@\textbf{IX., Alsergrund}, \emph{A.ADM3}|pwk}Wien 9/3, 2. 6. 99, 9–10V\nobreak{}«.   2) Stempel: »\nobreak{}\oindex{Seeboden@\textbf{Seeboden}, \emph{A.ADM3}|pwk}{\pb}See{[}boden{]}, 3. 6. {[}1899{]}\nobreak{}«. }
\buchAbdrucke{\weitereDrucke{Arthur Schnitzler, Richard Beer-Hofmann: \emph{Briefwechsel 1891–1931}. Wien, Zürich: \emph{Europaverlag} 1992, S. 128–129.} }\toendnotes[C]{\smallbreak}\pstart{}{\pb}\textsc{Herrn Dr Richard Beer-Hofmann}\pend{}\pstart{}\textsc{Kärnthen}\oindex{Kaernten@\textbf{Kärnten}, \emph{A.ADM1}|pw}\pend{}\pstart{}\textsc{Seeboden}\oindex{Seeboden@\textbf{Seeboden}, \emph{A.ADM3}|pw}\pend{}\pstart{}am \textsc{Millstätter}ſee\oindex{Millstaetter See@\textbf{Millstätter See}, \emph{See (N.SEE)}|pw}\pend{}\pstart{}\textsc{Villa Platzer\oindex{Villa Platzer@\textbf{Villa Platzer}, \emph{Gebäude (K.GBD)}|pw}}\pend{}{\bigskip}\vspace{1em}
\pstart
           \raggedleft{}{\pb}1. 6. 99. \pend
           \vspace{0.5em}
\pstart
           Mein lieber Richard,\pend
           
\pstart
           die \label{K_L00921-1v}\edtext{Rieſenkarte}{\lemma{\textnormal{\emph{Rieſenkarte}}}\Cendnote{\textnormal{Die Karte vom 29. 5. 1899 ist größer als eine normale
                  Postkarte.}}}\label{K_L00921-1} hab ich beko{\geminationm}en und danke für den
               lieben \label{K_L00921-2v}\edtext{Frozelgruſs}{\lemma{\textnormal{\emph{Frozelgruſs}}}\Cendnote{\textnormal{frotzeln, umgangssprachlich für:
                  necken}}}\label{K_L00921-2}. – Hier iſt es traurig – immer trauriger – Frühling und einſam – und
               ich weiſs nicht was ich mit mir beginnen ſoll –\pend
           
\pstart
           Jetzt eben, \label{K_L00921-3v}\edtext{Feiertag}{\lemma{\textnormal{\emph{Feiertag}}}\Cendnote{\textnormal{Fronleichnam}}}\label{K_L00921-3}, Nachmittg, ſehr ſchön
               – und der Abend vor mir – und nebſtbei das »ganze« Leben – vollko{\geminationm}en {\pb}überflüſſig. –\pend
           
\pstart
           \label{K_L00921-4v}\edtext{Neulich}{\lemma{\textnormal{\emph{Neulich}}}\Cendnote{\textnormal{Siehe A. S.: \emph{Tagebuch}, 28. 5. 1899.
               }}}\label{K_L00921-4} war ich mit Hugo\pwindex{Hofmannsthal, Hugo von 1874-02-01 – 1929-07-15@\textsc{Hofmannsthal, Hugo von} (1874-02-01 – 1929-07-15), \emph{Schriftsteller/Schriftstellerin}|pw}{ }Kampthal\oindex{Kamptal@\textbf{Kamptal}, \emph{Tal (N.TAL)}|pw} und Wachau\oindex{Wachau@\textbf{Wachau}, \emph{L.RGN}|pw}, die Abende auf dem Land ſind ſchauerlich – was da alles in der Luft
               ſchwebt – da verſtummen die Worte und verſiegen die Thränen. Ich habe Angſt vor dem
               Sommer, beſonders vor den Abenden, vor den Abenden am See –\pend
           
\pstart
           – Zuckungen, als we{\geminationn} ich {\pb}arbeiten wollte hab ich ſchon zuweilen, aber weiter noch nichts. Vorläufig ſteht es
               noch immer ſo, daſs nur der \uline{eine} Gedanke mildert –
               nun, Sie wiſſen ja.\pend
           
\pstart
           Nebstbei, ganz nebſtbei bringt mich auch das Ohrenſauſen langſam um – es iſt wahrhaft
               gräßlich, nicht eine Sekunde Ruhe zu haben und jeden Tag ein wenig nur {\pb}ein ganz klein wenig ſchlechter zu hören. –\pend
           
\pstart
           Sie wiſſen ſchon, dſs der Direktor Schleſinger\pwindex{Schlesinger, Emil 10.05.1844 – 31.05.1899@\textsc{Schlesinger, Emil} (10.05.1844 – 31.05.1899), \emph{Bankdirektor/Bankdirektorin}|pw}
               geſtern geſtorben ist. \label{K_L00921-5v}\edtext{Morgen vor 14
                  Tagen}{\lemma{\textnormal{\emph{Morgen vor 14
                  Tagen}}}\Cendnote{\textnormal{Siehe A. S.: \emph{Tagebuch}, 19. 5. 1899.
               }}}\label{K_L00921-5} waren Hugo\pwindex{Hofmannsthal, Hugo von 1874-02-01 – 1929-07-15@\textsc{Hofmannsthal, Hugo von} (1874-02-01 – 1929-07-15), \emph{Schriftsteller/Schriftstellerin}|pw} und ich mit ihm\pwindex{Schlesinger, Emil 10.05.1844 – 31.05.1899@\textsc{Schlesinger, Emil} (10.05.1844 – 31.05.1899), \emph{Bankdirektor/Bankdirektorin}|pwv} auf der Rohrerhütte\oindex{Rohrerhuette@\textbf{Rohrerhütte}, \emph{Gastgewerbegebäude (K.GGW)}|pw} zuſammen; er war heiſer und ſonſt »ganz
               geſund«. –\pend
           
\pstart
           \label{K_L00921-6v}\edtext{Geſtern war \introOben{}auch\introOben{} das »Vermächtnis\pwindex{Vermaechtnis. Schauspiel in drei Akten@\emph{Das Vermächtnis. Schauspiel in drei Akten}|pw}«}{\lemma{\textnormal{\emph{Geſtern … »Vermächtnis«}}}\Cendnote{\textnormal{Es stand am \emph{Burgtheater}\orgindex{Burgtheater@Burgtheater|pwk} noch immer am Spielplan.}}}\label{K_L00921-6}. Kein gutes Klima, unſre
               Stücke. – \label{K_L00921-7v}\edtext{Zweimal}{\lemma{\textnormal{\emph{Zweimal}}}\Cendnote{\textnormal{am 25. 5. 1899 und am 30. 5. 1899}}}\label{K_L00921-7} war ich {\pb}in Kaltenleutgeben\oindex{Kaltenleutgeben@\textbf{Kaltenleutgeben}, \emph{P.PPLA3}|pw}, bei Brahm\pwindex{Brahm, Otto 05.02.1856 – 28.11.1912@\textsc{Brahm, Otto} (05.02.1856 – 28.11.1912), \emph{Theaterleiter/Theaterleiterin, Regisseur/Regisseurin}|pw}. Er iſt
               ein nahezu wohlthuender Menſch. –\pend
           
\pstart
           \label{K_L00921-8v}\edtext{Samſtag}{\lemma{\textnormal{\emph{Samſtag}}}\Cendnote{\textnormal{Vgl. A. S.: \emph{Tagebuch}, 27. 5. 1899.
               }}}\label{K_L00921-8} beim »Richter von Zalamea\pwindex{Richter von Zalamea@\emph{Der Richter von Zalamea}|pw}«. Baumeiſter\pwindex{Baumeister, Bernhard 28.09.1827 – 25.10.1917@\textsc{Baumeister, Bernhard} (28.09.1827 – 25.10.1917), \emph{Schauspieler/Schauspielerin}|pw} unbeſchreiblich. Und das Stück! Hugo\pwindex{Hofmannsthal, Hugo von 1874-02-01 – 1929-07-15@\textsc{Hofmannsthal, Hugo von} (1874-02-01 – 1929-07-15), \emph{Schriftsteller/Schriftstellerin}|pw} findet, daſs Sie noch am eheſten ſo eins
               ſchreiben könnten (er meint, unter »uns«, alſo: Sie, er, ich, Leo Hirſchfeld\pwindex{Feld, Leo 14.02.1869 – 05.09.1924@\textsc{Feld, Leo} (14.02.1869 – 05.09.1924), \emph{Schriftsteller/Schriftstellerin, Übersetzer/Übersetzerin, Dirigent/Dirigentin}|pw}, Oskar
                  Friedmann\pwindex{Friedmann, Oskar 13.07.1872 – 03.11.1929@\textsc{Friedmann, Oskar} (13.07.1872 – 03.11.1929), \emph{Schriftsteller/Schriftstellerin, Regisseur/Regisseurin, Dramaturg/Dramaturgin}|pw}, Karlweis\pwindex{Karlweis, Carl 23.11.1850 – 27.10.1901@\textsc{Karlweis, Carl} (23.11.1850 – 27.10.1901), \emph{Schriftsteller/Schriftstellerin}|pw}) – ich hoffe {\pb}Sie laſſen ihn nicht in dem Glauben, – ſondern
               ſchreiben wirklich ein Stück.\pend
           
\pstart
           Hören Sie: Ein jüdiſcher Selcher will \introOben{}im\introOben{}{ }So{\geminationm}er einmal auf ein
               paar Augenblicke ſein Local verlaſſen – die Thür iſt offen, wie er hinaustritt –
               liegt ein großer Hund da. Der Selcher denkt: Mach ich jetzt die Thür zu, ſo merkt
               doch jenner (der Hund) daſs {\pb}ich fort bin und ſpringt
               ſich durch die Glasſcheiben in mein Geſchäft und friſſt ſich meine Würſtel – ich
               laſſe doch lieber die Thür offen, werd er glauben, ich bin gar nicht eweg
               gegangen. –\pend
           
\pstart
           – Er geht, ko{\geminationm}t nach einer Weile zurück, der Hund iſt im
               Geschäft und hat ſich richtig alle Würſtel aufgefreſſen. Der Selcher schüttelt {\pb}den Kopf und ſagt: »A ſo ä Dreh von dem Hund!«\pend
           
\pstart
           – Schöneres ka{\geminationn} ich Ihnen heut nicht mehr \substVorne{}\textsuperscript{ſagen}\substDazwischen{}erzählen\substHinten{}! –\pend
           
\pstart
           – Grüß Sie Gott. Schreiben Sie mir bald.\pend
           \pstart Ihr \spacefill\mbox{Arthur}\pend{}\selectlanguage{ngerman}\endnumbering\briefempfaengerindex{Beer-Hofmann, Richard@\textsc{Beer-Hofmann, Richard}!zzzSchnitzler, Arthur@\emph{von Arthur Schnitzler}!1899-06-011@{1. 6. 1899}|)be}\mylabel{L00921h}  \normalsize

\doendnotes{C}
\bigskip
\vfill

\clearpage

\footnotesize

\lohead{\textsc{register}}

% Definiere theindex-Environment komplett neu ohne reledmac
\makeatletter
\renewenvironment{theindex}{%
  \section*{\indexname}%
  \setlength{\parindent}{0pt}%
  \setlength{\parskip}{0pt plus 0.3pt}%
  \let\item\@idxitem
}{%
  \clearpage
}
\makeatother

\IfFileExists{\jobname-pw.ind}{\input{\jobname-pw.ind}}{}

\end{document}

      