%% latex-leseansicht-vorspann.tex
%% Vorspann für die Leseansicht.
%% Lädt die gemeinsame Datei latex-vorspann.tex mit nicht gesetztem Schalter.

\newif\ifkorrekturansicht
\korrekturansichtfalse

\input{../tex-inputs/latex-vorspann}

\begin{center}
            \textcolor{red}{ENTWURF. ENTZIFFERUNG NOCH NICHT KORREKTURGELESEN}
                      \end{center}
            
               \section[Arthur Schnitzler an Richard Beer-Hofmann, 1. 6. 1899]{ Arthur Schnitzler an Richard Beer-Hofmann, 1. 6. 1899}\nopagebreak\mylabel{v}\rehead{ }\begin{ledgroupsized}[t]{13cm}\normalsize\beginnumbering\briefempfaengerindex{Beer-Hofmann, Richard@\textsc{Beer-Hofmann, Richard}!zzzSchnitzler, Arthur@\emph{von Arthur Schnitzler}!1899-06-011@{1. 6. 1899}|(be} \toendnotes[C]{\smallbreak\pagebreak[2]} \Standort{YCGL, MSS 31.}
\physDesc{Brief, 2 Blätter, 8 Seiten, Umschlag
\newline{}Handschrift: 1) Bleistift, deutsche Kurrent\hspace{1em}2) schwarze Tinte, deutsche Kurrent (\noindent{}Umschlag)\hspace{1em}\newline{}Versand: 1) Stempel: »\nobreak{}\oindex{IX., Alsergrund@\textbf{IX., Alsergrund}|pwk}Wien 9/3, 2. 6. 99, 9–10V\nobreak{}«.  2) Stempel: »\nobreak{}\oindex{Seeboden@\textbf{Seeboden}|pwk}{\pb}See{[}boden{]}, 3. 6. {[}1899{]}\nobreak{}«. }\buchAbdrucke{\weitereDrucke{Arthur Schnitzler, Richard Beer-Hofmann: \emph{Briefwechsel 1891–1931}. Hg. Konstanze Fliedl. Wien, Zürich: \emph{Europaverlag} 1992, S. 128–129.} }\toendnotes[C]{\smallbreak}\pstart{}{\pb}\textsc{Herrn Dr Richard Beer-Hofmann }\pend{}\pstart{}\textsc{Kärnthen}\oindex{Kaernten@\textbf{Kärnten}|pw}\pend{}\pstart{}\textsc{Seeboden}\oindex{Seeboden@\textbf{Seeboden}|pw}\pend{}\pstart{}am \textsc{Millstätter}ſee\oindex{Millstaetter See@\textbf{Millstätter See}|pw}\pend{}\pstart{}\textsc{Villa Platzer\oindex{Villa Platzer@\textbf{Villa Platzer}|pw}}\pend{}{\bigskip}\pstart
           \raggedleft{}{\pb}1. 6. 99. \pend
           \pstart
           Mein lieber Richard,\pend
           \pstart
           die \label{K_L00921_1v}\edtext{Rieſenkarte}{\lemma{\textnormal{\emph{Rieſenkarte}}}\Cendnote{\textnormal{Die Karte vom 29. 5. 1899 ist größer als eine normale
                  Postkarte.}}}\label{K_L00921_1h} hab ich beko{\geminationm}en und danke für den
               lieben \label{K_L00921_2v}\edtext{Frozelgruſs}{\lemma{\textnormal{\emph{Frozelgruſs}}}\Cendnote{\textnormal{frotzeln, umgangssprachlich für:
                  necken}}}\label{K_L00921_2h}. – Hier iſt es traurig – immer trauriger – Frühling und einſam –
               und ich weiſs nicht was ich mit mir beginnen ſoll –\pend
           \pstart
           Jetzt eben, \label{K_L00921_3v}\edtext{Feiertag}{\lemma{\textnormal{\emph{Feiertag}}}\Cendnote{\textnormal{Fronleichnam}}}\label{K_L00921_3h}, Nachmittg, ſehr ſchön
               – und der Abend vor mir – und nebſtbei das »ganze« Leben – vollko{\geminationm}en {\pb}überflüſſig. –\pend
           \pstart
           \label{K_L00921_4v}\edtext{Neulich}{\lemma{\textnormal{\emph{Neulich}}}\Cendnote{\textnormal{siehe A. S.: \emph{Tagebuch}, 28. 5. 1899}}}\label{K_L00921_4h} war ich mit Hugo\pwindex{Hofmannsthal, Hugo von 01.02.1874 – 15.07.1929@\textsc{Hofmannsthal, Hugo von} (01.02.1874 – 15.07.1929), \emph{Schriftsteller}|pw}{ }Kampthal\oindex{Kamptal@\textbf{Kamptal}|pw} und Wachau\oindex{Wachau@\textbf{Wachau}|pw}, die Abende auf dem Land ſind ſchauerlich – was da alles in der Luft
               ſchwebt – da verſtummen die Worte und verſiegen die Thränen. Ich habe Angſt vor dem
               Sommer, beſonders vor den Abenden, vor den Abenden am See –\pend
           \pstart
           – Zuckungen, als we{\geminationn} ich {\pb}arbeiten wollte hab ich ſchon zuweilen, aber weiter noch nichts. Vorläufig ſteht es
               noch immer ſo, daſs nur der \uline{eine} Gedanke mildert –
               nun, Sie wiſſen ja.\pend
           \pstart
           Nebstbei, ganz nebſtbei bringt mich auch das Ohrenſauſen langſam um – es iſt wahrhaft
               gräßlich, nicht eine Sekunde Ruhe zu haben und jeden Tag ein wenig nur {\pb}ein ganz klein wenig ſchlechter zu hören. –\pend
           \pstart
           Sie wiſſen ſchon, dſs der Direktor Schleſinger\pwindex{Schlesinger, Emil 10.05.1844 – 31.05.1899@\textsc{Schlesinger, Emil} (10.05.1844 – 31.05.1899), \emph{Bankdirektor}|pw}
               geſtern geſtorben ist. \label{K_L00921_5v}\edtext{Morgen vor 14
                  Tagen}{\lemma{\textnormal{\emph{Morgen vor 14
                  Tagen}}}\Cendnote{\textnormal{siehe A. S.: \emph{Tagebuch}, 19. 5. 1899}}}\label{K_L00921_5h} waren Hugo\pwindex{Hofmannsthal, Hugo von 01.02.1874 – 15.07.1929@\textsc{Hofmannsthal, Hugo von} (01.02.1874 – 15.07.1929), \emph{Schriftsteller}|pw} und ich mit ihm\pwindex{Schlesinger, Emil 10.05.1844 – 31.05.1899@\textsc{Schlesinger, Emil} (10.05.1844 – 31.05.1899), \emph{Bankdirektor}|pwv} auf der Rohrerhütte\oindex{Rohrerhuette@\textbf{Rohrerhütte}|pw} zuſammen; er war heiſer und ſonſt »ganz geſund«. –\pend
           \pstart
           \label{K_L00921_6v}\edtext{Geſtern war \introOben{}auch\introOben{} das »Vermächtnis\pwindex{Schnitzler, Arthur 15.05.1862 – 21.10.1931@\textsc{Schnitzler, Arthur} (15.05.1862 – 21.10.1931), \emph{Schriftsteller, Mediziner}!Vermaechtnis. Schauspiel in drei Akten1898@\strich\emph{Das Vermächtnis. Schauspiel in drei Akten} {[}1898{]}|pw}«}{\lemma{\textnormal{\emph{Geſtern … »Vermächtnis«}}}\Cendnote{\textnormal{Es stand am \emph{Burgtheater}\orgindex{Burgtheater@Burgtheater|pwk} noch immer am Spielplan.}}}\label{K_L00921_6h}. Kein gutes Klima, unſre
               Stücke. – \label{K_L00921_7v}\edtext{Zweimal}{\lemma{\textnormal{\emph{Zweimal}}}\Cendnote{\textnormal{am 25. 5. 1899 und am 30. 5. 1899}}}\label{K_L00921_7h} war ich {\pb}in Kaltenleutgeben\oindex{Kaltenleutgeben@\textbf{Kaltenleutgeben}|pw}, bei Brahm\pwindex{Brahm, Otto 05.02.1856 – 28.11.1912@\textsc{Brahm, Otto} (05.02.1856 – 28.11.1912), \emph{Theaterleiter, Regisseur}|pw}. Er iſt ein
               nahezu wohlthuender Menſch. –\pend
           \pstart
           \label{K_L00921_8v}\edtext{Samſtag}{\lemma{\textnormal{\emph{Samſtag}}}\Cendnote{\textnormal{vgl. A. S.: \emph{Tagebuch}, 27. 5. 1899}}}\label{K_L00921_8h} beim »Richter von Zalamea\pwindex{\textcolor{red}{\textsuperscript{XXXX1 indx}}!Richter von Zalamea1640@\strich\emph{Der Richter von Zalamea} {[}1640{]}|pw}«. Baumeiſter\pwindex{Baumeister, Bernhard 28.09.1827 – 25.10.1917@\textsc{Baumeister, Bernhard} (28.09.1827 – 25.10.1917), \emph{Schauspieler}|pw} unbeſchreiblich. Und das Stück! Hugo\pwindex{Hofmannsthal, Hugo von 01.02.1874 – 15.07.1929@\textsc{Hofmannsthal, Hugo von} (01.02.1874 – 15.07.1929), \emph{Schriftsteller}|pw} findet, daſs Sie noch am eheſten ſo eins ſchreiben könnten
               (er meint, unter »uns«, alſo: Sie, er, ich, Leo
                  Hirſchfeld\pwindex{Feld, Leo 14.02.1869 – 05.09.1924@\textsc{Feld, Leo} (14.02.1869 – 05.09.1924), \emph{Schriftsteller}|pw}, Oskar Friedmann\pwindex{Friedmann, Oskar 13.07.1872 – 03.11.1929@\textsc{Friedmann, Oskar} (13.07.1872 – 03.11.1929), \emph{Schriftsteller, Regisseur, Dramaturg}|pw}, Karlweis\pwindex{Karlweis, Carl 23.11.1850 – 27.10.1901@\textsc{Karlweis, Carl} (23.11.1850 – 27.10.1901), \emph{Schriftsteller}|pw}) – ich hoffe {\pb}Sie laſſen ihn nicht in dem Glauben, – ſondern
               ſchreiben wirklich ein Stück.\pend
           \pstart
           Hören Sie: Ein jüdiſcher Selcher will \introOben{}im\introOben{}{ }So{\geminationm}er einmal auf ein
               paar Augenblicke ſein Local verlaſſen – die Thür iſt offen, wie er hinaustritt –
               liegt ein großer Hund da. Der Selcher denkt: Mach ich jetzt die Thür zu, ſo merkt
               doch jenner (der Hund) daſs {\pb}ich fort bin und ſpringt
               ſich durch die Glasſcheiben in mein Geſchäft und friſſt ſich meine Würſtel – ich
               laſſe doch lieber die Thür offen, werd er glauben, ich bin gar nicht eweg
               gegangen. –\pend
           \pstart
           – Er geht, ko{\geminationm}t nach einer Weile zurück, der Hund iſt im
               Geschäft und hat ſich richtig alle Würſtel aufgefreſſen. Der Selcher schüttelt {\pb}den Kopf und ſagt: »A ſo ä Dreh von dem Hund!«\pend
           \pstart
           – Schöneres ka{\geminationn} ich Ihnen heut nicht mehr \substVorne{}\textsuperscript{ſagen}\substDazwischen{}erzählen\substHinten{}! –\pend
           \pstart
           – Grüß Sie Gott. Schreiben Sie mir bald.\pend
           \pstart Ihr \spacefill\mbox{Arthur}\pend{}\endnumbering\briefempfaengerindex{Beer-Hofmann, Richard@\textsc{Beer-Hofmann, Richard}!zzzSchnitzler, Arthur@\emph{von Arthur Schnitzler}!1899-06-011@{1. 6. 1899}|)be}\mylabel{h}\end{ledgroupsized}  \newcommand{\dateiname}{L00921}\newcommand{\titel}{Arthur Schnitzler an Richard Beer-Hofmann, 1. 6. 1899}\newcommand{\editorInnen}{Martin Anton Müller und Gerd-Hermann Susen}%% latex-leseansicht-abspann.tex
%% Abspann für die Leseansicht.
%% Der Schalter \ifkorrekturansicht ist bereits durch den Vorspann gesetzt.

%% latex-abspann.tex
%% Gemeinsamer Abspann für Korrekturansicht und Leseansicht.
%% Setzt den Schalter \ifkorrekturansicht voraus (gesetzt in den
%% einbindenden Dateien latex-korrekturansicht-abspann.tex bzw.
%% latex-leseansicht-abspann.tex).
%% ---------------------------------------------------------------

\normalsize

% Das esempio-Environment wird nur in der Leseansicht benötigt
\ifkorrekturansicht\else
\newenvironment{esempio}[3]%
{
    \vspace{1.5ex}
    \rlap{\underline{#1}}
    \par
    \setlength{\parindent}{0cm}
    \nopagebreak
    \leftskip=#2cm
    \rightskip=#3cm
}
{
    \par
}
\fi

\doendnotes{C}
\bigskip
\vfill

\clearpage

\footnotesize

\ifkorrekturansicht
  \lohead{\textsc{register}}
\fi

% theindex-Environment neu definieren ohne reledmac
\makeatletter
\renewenvironment{theindex}{%
  \ifkorrekturansicht
    \section*{\indexname}%
  \else
    \subsubsection*{Index der erwähnten Entitäten}%
  \fi
  \setlength{\parindent}{0pt}%
  \setlength{\parskip}{0pt plus 0.3pt}%
  \let\item\@idxitem
}{%
  \ifkorrekturansicht\clearpage\fi
}
\makeatother

\IfFileExists{\jobname-pw.ind}{\input{\jobname-pw.ind}}{}

% Quellenangabe nur in der Leseansicht
\ifkorrekturansicht\else
% Fallback-Definitionen, falls die .tex-Datei \titel etc. nicht gesetzt hat
\providecommand{\titel}{}
\providecommand{\editorInnen}{}
\providecommand{\dateiname}{\jobname}

\vspace{3cm}

\vfill

\footnotesize
\textsc{Quelle}: \titel. Herausgegeben von {\editorInnen}. In: \emph{Arthur Schnitzler: Briefwechsel mit Autorinnen und Autoren}.
 Digitale Edition, https://schnitzler-briefe.acdh.oeaw.ac.at/{\dateiname}.html (Stand \today)
\fi

\end{document}


      