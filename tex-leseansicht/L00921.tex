%% latex-leseansicht-vorspann.tex
%% Vorspann für die Leseansicht.
%% Lädt die gemeinsame Datei latex-vorspann.tex mit nicht gesetztem Schalter.

\newif\ifkorrekturansicht
\korrekturansichtfalse

\input{../tex-inputs/latex-vorspann}


\section[Arthur Schnitzler an Richard Beer-Hofmann, 1. 6. 1899]{L00921 Arthur Schnitzler an Richard Beer-Hofmann, 1. 6. 1899}
\nopagebreak\mylabel{L00921v}
\rehead{ }\normalsize\beginnumbering\briefempfaengerindex{Beer-Hofmann, Richard@\textsc{Beer-Hofmann, Richard}!zzzSchnitzler, Arthur@\emph{von Arthur Schnitzler}!1899-06-011@{1. 6. 1899}|(be}
\toendnotes[C]{\smallbreak\pagebreak[2]}
\correspDesc{Versand  durch Arthur Schnitzler am 1. 6. 1899 in Wien
\newline{}Erhalt  durch Richard Beer-Hofmann am 3. 6. 1899 in Seeboden}\toendnotes[C]{\smallbreak}
\Standort{YCGL, MSS 31.}
\physDesc{Brief, 2 Blätter, 8 Seiten, Kuvert, 2400 Zeichen
\newline{}Handschrift: 1) Bleistift, deutsche Kurrent\hspace{1em}2) schwarze Tinte, deutsche Kurrent (\noindent{}Umschlag)\hspace{1em}
\newline{}Versand: 1) Stempel: »\nobreak{}\oindex{IX., Alsergrund@\textbf{IX., Alsergrund}, \emph{Verwaltungsgebiet}|pwk}Wien 9/3, 2. 6. 99, 9–10V\nobreak{}«.   2) Stempel: »\nobreak{}\oindex{Seeboden am Millstättersee@\textbf{Seeboden am Millstättersee}|pwk}{\pb}See{[}boden{]}, 3. 6. {[}1899{]}\nobreak{}«. }
\buchAbdrucke{\weitereDrucke{Arthur Schnitzler, Richard Beer-Hofmann: \emph{Briefwechsel 1891–1931}. Herausgegeben von Konstanze Fliedl. Wien, Zürich: \emph{Europaverlag} 1992, S. 128–129.} }\toendnotes[C]{\smallbreak}\pstart{}{\pb}\textsc{Herrn Dr Richard Beer-Hofmann}\pend{}\pstart{}\textsc{Kärnthen}\oindex{Kärnten@\textbf{Kärnten}, \emph{Land}|pw}\pend{}\pstart{}\textsc{Seeboden}\oindex{Seeboden am Millstättersee@\textbf{Seeboden am Millstättersee}|pw}\pend{}\pstart{}am \textsc{Millstätter}ſee\oindex{Millstätter See@\textbf{Millstätter See}, \emph{See}|pw}\pend{}\pstart{}\textsc{Villa Platzer\oindex{Villa Platzer@\textbf{Villa Platzer}, \emph{Gebäude}|pw}}\pend{}{\bigskip}\vspace{1em}
\pstart
           \raggedleft{}{\pb}1. 6. 99.\pend
           \vspace{0.5em}
\pstart
           Mein lieber Richard,\pend
           
\pstart
           die \label{K_L00921-1v}\edtext{Rieſenkarte}{\lemma{\textnormal{\emph{Riesenkarte}}}\Cendnote{\textnormal{Die Karte vom XXXX Auszeichnungsfehler: Dokument L00919 nicht gefunden ist größer als eine normale
                  Postkarte.}}}\label{K_L00921-1} hab ich beko{\geminationm}en und danke für den
               lieben \label{K_L00921-2v}\edtext{Frozelgruſs}{\lemma{\textnormal{\emph{Frozelgruss}}}\Cendnote{\textnormal{frotzeln, umgangssprachlich für:
                  necken}}}\label{K_L00921-2}. – Hier iſt es traurig – immer trauriger – Frühling und einſam – und
               ich weiſs nicht was ich mit mir beginnen{ }ſoll –\pend
           
\pstart
           Jetzt eben, \label{K_L00921-3v}\edtext{Feiertag}{\lemma{\textnormal{\emph{Feiertag}}}\Cendnote{\textnormal{Fronleichnam}}}\label{K_L00921-3}, Nachmittg,{ }ſehr{ }ſchön
               – und der Abend vor mir – und nebſtbei das »ganze« Leben – vollko{\geminationm}en {\pb}überflüſſig. –\pend
           
\pstart
           \label{K_L00921-4v}\edtext{Neulich}{\lemma{\textnormal{\emph{Neulich}}}\Cendnote{\textnormal{Siehe A. S.: \emph{Tagebuch}, 28. 5. 1899.
               }}}\label{K_L00921-4} war ich mit Hugo\pwindex{Hofmannsthal, Hugo von 1.\,2.\,1874 Wien – 15.\,7.\,1929 Rodaun@\textsc{Hofmannsthal, Hugo von} (1.\,2.\,1874 Wien – 15.\,7.\,1929 Rodaun), \emph{Schriftsteller}|pw}{ }Kampthal\oindex{Kamptal@\textbf{Kamptal}, \emph{Tal}|pw} und Wachau\oindex{Wachau@\textbf{Wachau}, \emph{Region}|pw}, die Abende auf dem Land{ }ſind{ }ſchauerlich – was da alles in der Luft{ }ſchwebt – da verſtummen die Worte und verſiegen die Thränen. Ich habe Angſt vor dem
               Sommer, beſonders vor den Abenden, vor den Abenden am See –\pend
           
\pstart
           – Zuckungen, als we{\geminationn} ich {\pb}arbeiten wollte hab ich{ }ſchon zuweilen, aber weiter noch nichts. Vorläufig{ }ſteht es
               noch immer{ }ſo, daſs nur der \uline{eine} Gedanke mildert –
               nun, Sie wiſſen ja.\pend
           
\pstart
           Nebstbei, ganz nebſtbei bringt mich auch das Ohrenſauſen langſam um – es iſt wahrhaft
               gräßlich, nicht eine Sekunde Ruhe zu haben und jeden Tag ein wenig nur {\pb}ein ganz klein wenig{ }ſchlechter zu hören. –\pend
           
\pstart
           Sie wiſſen{ }ſchon, dſs der Direktor Schleſinger\pwindex{Schlesinger, Emil 10.\,5.\,1844 Wien – 31.\,5.\,1899 ebd.@\textsc{Schlesinger, Emil} (10.\,5.\,1844 Wien – 31.\,5.\,1899 ebd.), \emph{Bankdirektor}|pw}
               geſtern geſtorben ist. \label{K_L00921-5v}\edtext{Morgen vor 14
                  Tagen}{\lemma{\textnormal{\emph{Morgen vor 14
                  Tagen}}}\Cendnote{\textnormal{Siehe A. S.: \emph{Tagebuch}, 19. 5. 1899.
               }}}\label{K_L00921-5} waren Hugo\pwindex{Hofmannsthal, Hugo von 1.\,2.\,1874 Wien – 15.\,7.\,1929 Rodaun@\textsc{Hofmannsthal, Hugo von} (1.\,2.\,1874 Wien – 15.\,7.\,1929 Rodaun), \emph{Schriftsteller}|pw} und ich mit ihm\pwindex{Schlesinger, Emil 10.\,5.\,1844 Wien – 31.\,5.\,1899 ebd.@\textsc{Schlesinger, Emil} (10.\,5.\,1844 Wien – 31.\,5.\,1899 ebd.), \emph{Bankdirektor}|pwv} auf der Rohrerhütte\oindex{Wien@\textbf{Wien}!XVII., Hernals@\textbf{XVII., Hernals}!Rohrerhütte@\textbf{Rohrerhütte}, \emph{Gastgewerbegebäude}|pw} zuſammen; er war heiſer und{ }ſonſt »ganz
               geſund«. –\pend
           
\pstart
           \label{K_L00921-6v}\edtext{Geſtern war \introOben{}auch\introOben{} das »Vermächtnis\pwindex{Schnitzler, Arthur 15.\,5.\,1862 Wien – 21.\,10.\,1931 ebd.@\textsc{Schnitzler, Arthur} (15.\,5.\,1862 Wien – 21.\,10.\,1931 ebd.), \emph{Schriftsteller, Mediziner}!Vermächtnis. Schauspiel in drei Akten@\strich\emph{Das Vermächtnis. Schauspiel in drei Akten}|pw}«}{\lemma{\textnormal{\emph{Gestern … »Vermächtnis«}}}\Cendnote{\textnormal{Es stand am \emph{Burgtheater}\orgindex{Burgtheater@Burgtheater|pwk} noch immer am Spielplan.}}}\label{K_L00921-6}. Kein gutes Klima, unſre
               Stücke. – \label{K_L00921-7v}\edtext{Zweimal}{\lemma{\textnormal{\emph{Zweimal}}}\Cendnote{\textnormal{am 25. 5. 1899 und am 30. 5. 1899}}}\label{K_L00921-7} war ich {\pb}in Kaltenleutgeben\oindex{Kaltenleutgeben@\textbf{Kaltenleutgeben}, \emph{Hauptstadt}|pw}, bei Brahm\pwindex{Brahm, Otto 5.\,2.\,1856 Hamburg – 28.\,11.\,1912 Berlin@\textsc{Brahm, Otto} (5.\,2.\,1856 Hamburg – 28.\,11.\,1912 Berlin), \emph{Theaterleiter, Regisseur}|pw}. Er iſt
               ein nahezu wohlthuender Menſch. –\pend
           
\pstart
           \label{K_L00921-8v}\edtext{Samſtag}{\lemma{\textnormal{\emph{Samstag}}}\Cendnote{\textnormal{Vgl. A. S.: \emph{Tagebuch}, 27. 5. 1899.
               }}}\label{K_L00921-8} beim »Richter von Zalamea\pwindex{\textcolor{red}{\textsuperscript{XXXX indx1}}!Richter von Zalamea@\strich\emph{Der Richter von Zalamea}|pw}«. Baumeiſter\pwindex{Baumeister, Bernhard 28.\,9.\,1827 Poznan – 25.\,10.\,1917 Baden bei Wien@\textsc{Baumeister, Bernhard} (28.\,9.\,1827 Poznan – 25.\,10.\,1917 Baden bei Wien), \emph{Schauspieler}|pw} unbeſchreiblich. Und das Stück! Hugo\pwindex{Hofmannsthal, Hugo von 1.\,2.\,1874 Wien – 15.\,7.\,1929 Rodaun@\textsc{Hofmannsthal, Hugo von} (1.\,2.\,1874 Wien – 15.\,7.\,1929 Rodaun), \emph{Schriftsteller}|pw} findet, daſs Sie noch am eheſten{ }ſo eins{ }ſchreiben könnten (er meint, unter »uns«, alſo: Sie, er, ich, Leo Hirſchfeld\pwindex{Feld, Leo 14.\,2.\,1869 Augsburg – 5.\,9.\,1924 Florenz@\textsc{Feld, Leo} (14.\,2.\,1869 Augsburg – 5.\,9.\,1924 Florenz), \emph{Schriftsteller, Übersetzer, Dirigent}|pw}, Oskar
                  Friedmann\pwindex{Friedmann, Oskar 13.\,7.\,1872 Wien – 3.\,11.\,1929 ebd.@\textsc{Friedmann, Oskar} (13.\,7.\,1872 Wien – 3.\,11.\,1929 ebd.), \emph{Schriftsteller, Regisseur, Dramaturg}|pw}, Karlweis\pwindex{Karlweis, Carl 23.\,11.\,1850 Wien – 27.\,10.\,1901 ebd.@\textsc{Karlweis, Carl} (23.\,11.\,1850 Wien – 27.\,10.\,1901 ebd.), \emph{Schriftsteller}|pw}) – ich hoffe {\pb}Sie laſſen ihn nicht in dem Glauben, –{ }ſondern{ }ſchreiben wirklich ein Stück.\pend
           
\pstart
           Hören Sie: Ein jüdiſcher Selcher will \introOben{}im\introOben{}{ }So{\geminationm}er einmal auf ein
               paar Augenblicke{ }ſein Local verlaſſen – die Thür iſt offen, wie er hinaustritt –
               liegt ein großer Hund da. Der Selcher denkt: Mach ich jetzt die Thür zu,{ }ſo merkt
               doch jenner (der Hund) daſs {\pb}ich fort bin und{ }ſpringt{ }ſich durch die Glasſcheiben in mein Geſchäft und friſſt{ }ſich meine Würſtel – ich
               laſſe doch lieber die Thür offen, werd er glauben, ich bin gar nicht eweg
               gegangen. –\pend
           
\pstart
           – Er geht, ko{\geminationm}t nach einer Weile zurück, der Hund iſt im
               Geschäft und hat{ }ſich richtig alle Würſtel aufgefreſſen. Der Selcher schüttelt {\pb}den Kopf und{ }ſagt: »A{ }ſo ä Dreh von dem Hund!«\pend
           
\pstart
           – Schöneres ka{\geminationn} ich Ihnen heut nicht mehr \substVorne{}\textsuperscript{ſagen}\substDazwischen{}erzählen\substHinten{}! –\pend
           
\pstart
           – Grüß Sie Gott. Schreiben Sie mir bald.\pend
           \pstart Ihr \spacefill\mbox{Arthur}\pend{}\selectlanguage{ngerman}\endnumbering\briefempfaengerindex{Beer-Hofmann, Richard@\textsc{Beer-Hofmann, Richard}!zzzSchnitzler, Arthur@\emph{von Arthur Schnitzler}!1899-06-011@{1. 6. 1899}|)be}\mylabel{L00921h}  \newcommand{\dateiname}{L00921}\newcommand{\titel}{Arthur Schnitzler an Richard Beer-Hofmann, 1. 6. 1899}\newcommand{\editorInnen}{Martin Anton Müller und Gerd-Hermann Susen}%% latex-leseansicht-abspann.tex
%% Abspann für die Leseansicht.
%% Der Schalter \ifkorrekturansicht ist bereits durch den Vorspann gesetzt.

%% latex-abspann.tex
%% Gemeinsamer Abspann für Korrekturansicht und Leseansicht.
%% Setzt den Schalter \ifkorrekturansicht voraus (gesetzt in den
%% einbindenden Dateien latex-korrekturansicht-abspann.tex bzw.
%% latex-leseansicht-abspann.tex).
%% ---------------------------------------------------------------

\normalsize

% Das esempio-Environment wird nur in der Leseansicht benötigt
\ifkorrekturansicht\else
\newenvironment{esempio}[3]%
{
    \vspace{1.5ex}
    \rlap{\underline{#1}}
    \par
    \setlength{\parindent}{0cm}
    \nopagebreak
    \leftskip=#2cm
    \rightskip=#3cm
}
{
    \par
}
\fi

\doendnotes{C}
\bigskip
\vfill

\clearpage

\footnotesize

\ifkorrekturansicht
  \lohead{\textsc{register}}
\fi

% theindex-Environment neu definieren ohne reledmac
\makeatletter
\renewenvironment{theindex}{%
  \ifkorrekturansicht
    \section*{\indexname}%
  \else
    \subsubsection*{Index der erwähnten Entitäten}%
  \fi
  \setlength{\parindent}{0pt}%
  \setlength{\parskip}{0pt plus 0.3pt}%
  \let\item\@idxitem
}{%
  \ifkorrekturansicht\clearpage\fi
}
\makeatother

\IfFileExists{\jobname-pw.ind}{\input{\jobname-pw.ind}}{}

% Quellenangabe nur in der Leseansicht
\ifkorrekturansicht\else
% Fallback-Definitionen, falls die .tex-Datei \titel etc. nicht gesetzt hat
\providecommand{\titel}{}
\providecommand{\editorInnen}{}
\providecommand{\dateiname}{\jobname}

\vspace{3cm}

\vfill

\footnotesize
\textsc{Quelle}: \titel. Herausgegeben von {\editorInnen}. In: \emph{Arthur Schnitzler: Briefwechsel mit Autorinnen und Autoren}.
 Digitale Edition, https://schnitzler-briefe.acdh.oeaw.ac.at/{\dateiname}.html (Stand \today)
\fi

\end{document}


