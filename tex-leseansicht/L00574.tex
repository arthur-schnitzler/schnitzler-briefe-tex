%% latex-leseansicht-vorspann.tex
%% Vorspann für die Leseansicht.
%% Lädt die gemeinsame Datei latex-vorspann.tex mit nicht gesetztem Schalter.

\newif\ifkorrekturansicht
\korrekturansichtfalse

\input{../tex-inputs/latex-vorspann}


\section[Arthur Schnitzler an Richard Beer-Hofmann, 30.\,7.\,1896]{L00574 Arthur Schnitzler an Richard Beer-Hofmann, 30.\,7.\,1896}
\nopagebreak\mylabel{L00574v}
\rehead{ }\normalsize\beginnumbering\briefempfaengerindex{Beer-Hofmann, Richard@\textsc{Beer-Hofmann, Richard}!zzzSchnitzler, Arthur@\emph{von Arthur Schnitzler}!1896-07-301@{30. 07. 1896}|(be}
\toendnotes[C]{\smallbreak\pagebreak[2]}
\correspDesc{Versand  durch Arthur Schnitzler am 30. 07. 1896 in Stockholm
\newline{}Erhalt  durch Richard Beer-Hofmann am 30. 07. 1896 in Kopenhagen}\toendnotes[C]{\smallbreak}
\Standort{YCGL, MSS 31.}
\physDesc{Telegramm, 179 Zeichen
\newline{}Handschrift Schreibkraft: 1) schwarze Tinte, lateinische Kurrent (\noindent{}Umschlag)\hspace{1em}2) blauer Buntstift, lateinische Kurrent\hspace{1em}}\pstart{}{\pb}Richard Beer-Hofmann\pend{}\pstart{}Hotel Kongen af Danmark\oindex{Hotel Kongen af Danmark@\textbf{Hotel Kongen af Danmark}, \emph{Hotel}|pw}\pend{}{\bigskip}\vspace{1em}
\pstart
           \noindent{}\centering{}{\pb}\textcolor{gray}{\textbf{Telegram fra}}{ }Stockholm\oindex{Stockholm@\textbf{Stockholm}, \emph{Hauptstadt}|pw}\pend
           
\pstart
           \textcolor{gray}{\textbf{Nr.}} 21/2107, 15 \textcolor{gray}{\textbf{Ord, indleveret
                  den}}{ }30\textcolor{gray}{\textbf{/}}7 \textcolor{gray}{\textbf{189}}6{ }12\textcolor{gray}{\textbf{T.}}{ }53\textcolor{gray}{\textbf{M.F.}}\pend
           
\pstart
           \raggedleft{}Richard Beer-Hofman{\\}Kbh\oindex{Kopenhagen@\textbf{Kopenhagen}, \emph{Hauptstadt}|pw}{\\}Hotel Konig Dänemark\oindex{Hotel Kongen af Danmark@\textbf{Hotel Kongen af Danmark}, \emph{Hotel}|pw}\pend
           
\pstart
           Unmöglich vor{ }Sonntag wegen Gothenburg\oindex{Göteborg@\textbf{Göteborg}|pw}er
               Billet\pend
           
\pstart
           herzlichst{\\[\baselineskip]}\spacefill\mbox{Arthur}\pend
           \leftskip=0em{}\selectlanguage{ngerman}\endnumbering\briefempfaengerindex{Beer-Hofmann, Richard@\textsc{Beer-Hofmann, Richard}!zzzSchnitzler, Arthur@\emph{von Arthur Schnitzler}!1896-07-301@{30. 07. 1896}|)be}\mylabel{L00574h}  \newcommand{\dateiname}{L00574}\newcommand{\titel}{Arthur Schnitzler an Richard Beer-Hofmann, 30. 7. 1896}\newcommand{\editorInnen}{Martin Anton Müller und Gerd-Hermann Susen}%% latex-leseansicht-abspann.tex
%% Abspann für die Leseansicht.
%% Der Schalter \ifkorrekturansicht ist bereits durch den Vorspann gesetzt.

%% latex-abspann.tex
%% Gemeinsamer Abspann für Korrekturansicht und Leseansicht.
%% Setzt den Schalter \ifkorrekturansicht voraus (gesetzt in den
%% einbindenden Dateien latex-korrekturansicht-abspann.tex bzw.
%% latex-leseansicht-abspann.tex).
%% ---------------------------------------------------------------

\normalsize

% Das esempio-Environment wird nur in der Leseansicht benötigt
\ifkorrekturansicht\else
\newenvironment{esempio}[3]%
{
    \vspace{1.5ex}
    \rlap{\underline{#1}}
    \par
    \setlength{\parindent}{0cm}
    \nopagebreak
    \leftskip=#2cm
    \rightskip=#3cm
}
{
    \par
}
\fi

\doendnotes{C}
\bigskip
\vfill

\clearpage

\footnotesize

\ifkorrekturansicht
  \lohead{\textsc{register}}
\fi

% theindex-Environment neu definieren ohne reledmac
\makeatletter
\renewenvironment{theindex}{%
  \ifkorrekturansicht
    \section*{\indexname}%
  \else
    \subsubsection*{Index der erwähnten Entitäten}%
  \fi
  \setlength{\parindent}{0pt}%
  \setlength{\parskip}{0pt plus 0.3pt}%
  \let\item\@idxitem
}{%
  \ifkorrekturansicht\clearpage\fi
}
\makeatother

\IfFileExists{\jobname-pw.ind}{\input{\jobname-pw.ind}}{}

% Quellenangabe nur in der Leseansicht
\ifkorrekturansicht\else
% Fallback-Definitionen, falls die .tex-Datei \titel etc. nicht gesetzt hat
\providecommand{\titel}{}
\providecommand{\editorInnen}{}
\providecommand{\dateiname}{\jobname}

\vspace{3cm}

\vfill

\footnotesize
\textsc{Quelle}: \titel. Herausgegeben von {\editorInnen}. In: \emph{Arthur Schnitzler: Briefwechsel mit Autorinnen und Autoren}.
 Digitale Edition, https://schnitzler-briefe.acdh.oeaw.ac.at/{\dateiname}.html (Stand \today)
\fi

\end{document}


