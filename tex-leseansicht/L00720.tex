%% latex-korrekturansicht-vorspann.tex
%% Vorspann für die Korrekturansicht.
%% Lädt die gemeinsame Datei latex-vorspann.tex mit gesetztem Schalter.

\newif\ifkorrekturansicht
\korrekturansichttrue

\input{../tex-inputs/latex-vorspann}


\section[Richard Beer-Hofmann an Arthur Schnitzler, {[}31. 8. 1897{]}]{L00720 Richard Beer-Hofmann an Arthur Schnitzler, {[}31. 8. 1897{]}}
\nopagebreak\mylabel{L00720v}
\rehead{ }\normalsize\beginnumbering\briefempfaengerindex{Schnitzler, Arthur@\textsc{Schnitzler, Arthur}!zzzBeer-Hofmann, Richard@\emph{von Richard Beer-Hofmann}!1897-08-312@{{[}31. 8. 1897{]}}|(be}
\toendnotes[C]{\smallbreak\pagebreak[2]}\Standort{CUL, Schnitzler, B 8.}
\physDesc{Brief, 1 Blatt, 1 Seite, 332 Zeichen
\newline{}Handschrift: Bleistift, lateinische Kurrent
\newline{}Schnitzler: mit Bleistift datiert: »31/8 97« }\toendnotes[C]{\smallbreak}
\pstart
           \raggedleft{}{\pb}8 Uhr Abend\pend
           \vspace{0.5em}
\pstart
           Lieber Arthur! Vor einer Stunde Ihre Karte erhalten; Leo\pwindex{Van-Jung, Leo 15.10.1866 – 02.07.1939@\textsc{Van-Jung, Leo} (15.10.1866 – 02.07.1939), \emph{Gesangspädagoge/Gesangspädagogin, Mathematiker/Mathematikerin}|pw}, der \label{K_L00720-1v}\edtext{Goldmann\pwindex{Goldmann, Paul 31.01.1865 – 25.09.1935@\textsc{Goldmann, Paul} (31.01.1865 – 25.09.1935), \emph{Schriftsteller/Schriftstellerin, Journalist/Journalistin}|pw} morgen noch
               in Salzburg\oindex{Salzburg@\textbf{Salzburg}, \emph{A.ADM2}|pw}}{\lemma{\textnormal{\emph{Goldmann … Salzburg}}}\Cendnote{\textnormal{Vgl. Felix Salten an Arthur Schnitzler, 3. 9. [1897]
                  und Paul Goldmann an Arthur Schnitzler, 15. 10. [1897].
               }}}\label{K_L00720-1} treffen möchte, reist
               heute Abends nach Salzburg\oindex{Salzburg@\textbf{Salzburg}, \emph{A.ADM2}|pw}, möchte
               Sie gerne auch noch sprechen und packt momentan. Bis 9 Uhr sind wir hier
                  Kolingasse 9\oindex{Kolingasse@\textbf{Kolingasse}, \emph{Straße (K.STR)}|pw}, dann begleite ich Leo\pwindex{Van-Jung, Leo 15.10.1866 – 02.07.1939@\textsc{Van-Jung, Leo} (15.10.1866 – 02.07.1939), \emph{Gesangspädagoge/Gesangspädagogin, Mathematiker/Mathematikerin}|pw} zur Bahn und kann erst um 11 ½
               im Caffee\oindex{Cafe Arkaden@\textbf{Café Arkaden}, \emph{Kaffeehaus (K.KAF)}|pwv} sein. Hoffe Sie zu
               treffen.\pend
           
\pstart
           Ihr{\\[\baselineskip]}\spacefill\mbox{Richard}\pend
           \leftskip=0em{}\selectlanguage{ngerman}\endnumbering\briefempfaengerindex{Schnitzler, Arthur@\textsc{Schnitzler, Arthur}!zzzBeer-Hofmann, Richard@\emph{von Richard Beer-Hofmann}!1897-08-312@{{[}31. 8. 1897{]}}|)be}\mylabel{L00720h}  \normalsize

\doendnotes{C}
\bigskip
\vfill

\clearpage

\footnotesize

\lohead{\textsc{register}}

% Definiere theindex-Environment komplett neu ohne reledmac
\makeatletter
\renewenvironment{theindex}{%
  \section*{\indexname}%
  \setlength{\parindent}{0pt}%
  \setlength{\parskip}{0pt plus 0.3pt}%
  \let\item\@idxitem
}{%
  \clearpage
}
\makeatother

\IfFileExists{\jobname-pw.ind}{\input{\jobname-pw.ind}}{}

\end{document}

      