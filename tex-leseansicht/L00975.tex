\input{../tex-inputs/latex-pdf-vorspann}
\begin{center}
            \textcolor{red}{ENTWURF. ENTZIFFERUNG NOCH NICHT KORREKTURGELESEN}
                      \end{center}
            
               \section[Arthur Schnitzler an Richard Beer-Hofmann, 17. 9. 1899]{ Arthur Schnitzler an Richard Beer-Hofmann, 17. 9. 1899}\nopagebreak\mylabel{v}\rehead{ }\begin{ledgroupsized}[t]{13cm}\normalsize\beginnumbering\briefempfaengerindex{Beer-Hofmann, Richard@\textsc{Beer-Hofmann, Richard}!zzzSchnitzler, Arthur@\emph{von Arthur Schnitzler}!1899-09-171@{17. 9. 1899}|(be} \toendnotes[C]{\smallbreak\pagebreak[2]} \Standort{CUL, Schnitzler, B 8.}
\physDesc{Bildpostkarte
\newline{}Handschrift: Bleistift, deutsche Kurrent\newline{}Versand: 1) Stempel: »\nobreak{}\oindex{Zirndorf@\textbf{Zirndorf}|pwk}Zirndorf, 17. {[}Sep{]} 99, 6–7\textcolor{gray}{NM}\nobreak{}«.  2) Stempel: »\nobreak{}\oindex{Vahrn@\textbf{Vahrn}|pwk}{[}Vahr{]}n, 18. 9. 99\nobreak{}«. \newline{}Ordnung: mit Bleistift von unbekannter Hand datiert: »17. 9.« }\buchAbdrucke{\weitereDrucke{Arthur Schnitzler, Richard Beer-Hofmann: \emph{Briefwechsel 1891–1931}. Hg. Konstanze Fliedl. Wien, Zürich: \emph{Europaverlag} 1992, S. 137.} }\pstart{}{\pb}Hrn \textsc{Dr. Rich.
                     Beer-Hofmann}\pend{}\pstart{}\textsc{Vahrn}\oindex{Vahrn@\textbf{Vahrn}|pw}\pend{}\pstart{}bei \textsc{Brixen\oindex{Brixen@\textbf{Brixen}|pw}}\pend{}\pstart{}\textsc{Tirol}\oindex{Tirol@\textbf{Tirol}|pw}\pend{}{\bigskip}\pstart
           \noindent{}\centering{}\textcolor{gray}{\textbf{{\pb}Gruss aus Zirndorf\oindex{Zirndorf@\textbf{Zirndorf}|pw}. Alte Veste\oindex{Alte Veste@\textbf{Alte Veste}|pw}. Scheidlers Haus\oindex{Scheidlers Haus@\textbf{Scheidlers Haus}|pw}}}\pend
           \pstart
           Daſs ich einmal hieher käme, hab ich nicht geahnt.\pend
           \pstart
           \substVorne{}\textsuperscript{\textcolor{gray}{Send}}\substDazwischen{}Schrei\substHinten{}ben Sie mir nach Frankfurt\oindex{Frankfurt am Main@\textbf{Frankfurt am Main}|pw}{ }\textsc{post. rest.}\pend
           \endnumbering\briefempfaengerindex{Beer-Hofmann, Richard@\textsc{Beer-Hofmann, Richard}!zzzSchnitzler, Arthur@\emph{von Arthur Schnitzler}!1899-09-171@{17. 9. 1899}|)be}\mylabel{h}\end{ledgroupsized}  \newcommand{\dateiname}{L00975}\newcommand{\titel}{Arthur Schnitzler an Richard Beer-Hofmann, 17. 9. 1899}\newcommand{\editorInnen}{Martin Anton Müller und Gerd-Hermann Susen}\input{../tex-inputs/latex-pdf-abspann}
      