%% latex-korrekturansicht-vorspann.tex
%% Vorspann für die Korrekturansicht.
%% Lädt die gemeinsame Datei latex-vorspann.tex mit gesetztem Schalter.

\newif\ifkorrekturansicht
\korrekturansichttrue

\input{../tex-inputs/latex-vorspann}


\section[Arthur Schnitzler an Richard Beer-Hofmann, 17. 9. 1899]{L00975 Arthur Schnitzler an Richard Beer-Hofmann, 17. 9. 1899}
\nopagebreak\mylabel{L00975v}
\rehead{ }\normalsize\beginnumbering\briefempfaengerindex{Beer-Hofmann, Richard@\textsc{Beer-Hofmann, Richard}!zzzSchnitzler, Arthur@\emph{von Arthur Schnitzler}!1899-09-171@{17. 9. 1899}|(be}
\toendnotes[C]{\smallbreak\pagebreak[2]}\Standort{YCGL, MSS 31.}
\physDesc{Bildpostkarte, 141 Zeichen
\newline{}Handschrift: Bleistift, deutsche Kurrent
\newline{}Versand: 1) Stempel: »\nobreak{}\oindex{Zirndorf@\textbf{Zirndorf}, \emph{P.PPL}|pwk}Zirndorf, 17. {[}Sep{]} 99, 6–7\textcolor{gray}{NM}\nobreak{}«.   2) Stempel: »\nobreak{}\oindex{Vahrn@\textbf{Vahrn}, \emph{P.PPLA3}|pwk}{[}Vahr{]}n, 18. 9. 99\nobreak{}«. 
\newline{}Ordnung: mit Bleistift von unbekannter Hand datiert: »17. 9.« }
\buchAbdrucke{\weitereDrucke{Arthur Schnitzler, Richard Beer-Hofmann: \emph{Briefwechsel 1891–1931}. Wien, Zürich: \emph{Europaverlag} 1992, S. 137.} }\pstart{}{\pb}Hrn \textsc{Dr. Rich.
                     Beer-Hofmann}\pend{}\pstart{}\textsc{Vahrn}\oindex{Vahrn@\textbf{Vahrn}, \emph{P.PPLA3}|pw}\pend{}\pstart{}bei \textsc{Brixen\oindex{Brixen@\textbf{Brixen}, \emph{P.PPLA3}|pw}}\pend{}\pstart{}\textsc{Tirol}\oindex{Tirol@\textbf{Tirol}, \emph{A.ADM1}|pw}\pend{}{\bigskip}
\pstart
           \noindent{}\centering{}{\pb}\textcolor{gray}{\textbf{Gruss aus Zirndorf\oindex{Zirndorf@\textbf{Zirndorf}, \emph{P.PPL}|pw}. Alte Veste\oindex{Alte Veste@\textbf{Alte Veste}, \emph{Monument (K.MON)}|pw}. Scheidlers Haus\oindex{Scheidlers Haus@\textbf{Scheidlers Haus}, \emph{Gebäude (K.GBD)}|pw}}}\pend
           \vspace{1em}
\pstart
           \noindent{}{\pb}Daſs ich einmal hieher käme, hab ich nicht geahnt.\pend
           
\pstart
           \substVorne{}\textsuperscript{\textcolor{gray}{Send}}\substDazwischen{}Schrei\substHinten{}ben Sie mir nach Frankfurt\oindex{Frankfurt am Main@\textbf{Frankfurt am Main}, \emph{P.PPLA3}|pw}{ }\textsc{post. rest.}\pend
           \selectlanguage{ngerman}\endnumbering\briefempfaengerindex{Beer-Hofmann, Richard@\textsc{Beer-Hofmann, Richard}!zzzSchnitzler, Arthur@\emph{von Arthur Schnitzler}!1899-09-171@{17. 9. 1899}|)be}\mylabel{L00975h}  \normalsize

\doendnotes{C}
\bigskip
\vfill

\clearpage

\footnotesize

\lohead{\textsc{register}}

% Definiere theindex-Environment komplett neu ohne reledmac
\makeatletter
\renewenvironment{theindex}{%
  \section*{\indexname}%
  \setlength{\parindent}{0pt}%
  \setlength{\parskip}{0pt plus 0.3pt}%
  \let\item\@idxitem
}{%
  \clearpage
}
\makeatother

\IfFileExists{\jobname-pw.ind}{\input{\jobname-pw.ind}}{}

\end{document}

      