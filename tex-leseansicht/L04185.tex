%% latex-leseansicht-vorspann.tex
%% Vorspann für die Leseansicht.
%% Lädt die gemeinsame Datei latex-vorspann.tex mit nicht gesetztem Schalter.

\newif\ifkorrekturansicht
\korrekturansichtfalse

\input{../tex-inputs/latex-vorspann}


\section[Arthur Schnitzler an Gustav Schwarzkopf, 10. 9. 1917]{L04185 Arthur Schnitzler an Gustav Schwarzkopf, 10. 9. 1917}
\nopagebreak\mylabel{L04185v}
\rehead{ }\normalsize\beginnumbering\briefempfaengerindex{Schwarzkopf, Gustav@\textsc{Schwarzkopf, Gustav}!zzzMarx, Paul@\emph{von Paul Marx}!1917-09-101@{10. 9. 1917}|(be}\briefempfaengerindex{Schwarzkopf, Gustav@\textsc{Schwarzkopf, Gustav}!zzzSteinrück, Elisabeth@\emph{von Elisabeth Steinrück}!1917-09-101@{10. 9. 1917}|(be}\briefempfaengerindex{Schwarzkopf, Gustav@\textsc{Schwarzkopf, Gustav}!zzzSchnitzler, Olga@\emph{von Olga Schnitzler}!1917-09-101@{10. 9. 1917}|(be}\briefempfaengerindex{Schwarzkopf, Gustav@\textsc{Schwarzkopf, Gustav}!zzzSchnitzler, Arthur@\emph{von Arthur Schnitzler}!1917-09-101@{10. 9. 1917}|(be}
\toendnotes[C]{\smallbreak\pagebreak[2]}
\correspDesc{Versand  durch Arthur Schnitzler, Olga Schnitzler, Elisabeth Steinrück, Paul Marx am 10. 9. 1917 in Eibsee
\newline{}Erhalt  durch Gustav Schwarzkopf im Zeitraum [11. 9. 1917 – 15. 9. 1917?] in Wien}\toendnotes[C]{\smallbreak}
\Standort{CUL, Schnitzler, B 96.}
\physDesc{Postkarte, 425 Zeichen
\newline{}Handschrift Arthur Schnitzler: Bleistift, lateinische Kurrent
\newline{}Handschrift Olga Schnitzler: Bleistift, lateinische Kurrent
\newline{}Handschrift Elisabeth Steinrück: Bleistift, lateinische Kurrent
\newline{}Handschrift Paul Marx: Bleistift, lateinische Kurrent}\pstart{}{\pb}Arthur Schnitzler,\pend{}\pstart{}Partenkirchen\oindex{Partenkirchen@\textbf{Partenkirchen}, \emph{Teil eines besiedelten Ortes}|pw}\pend{}\pstart{}Haus Tannenberg\oindex{Haus Tannenberg@\textbf{Haus Tannenberg}, \emph{Beherbergungsgebäude}|pw}\pend{}{\bigskip}\pstart{}Herrn Gustav Schwarzkopf\pend{}\pstart{}Wien I\oindex{I., Innere Stadt@\textbf{I., Innere Stadt}, \emph{Verwaltungsgebiet}|pw}\pend{}\pstart{}Tiefer Graben 17\oindex{Wien@\textbf{Wien}!I., Innere Stadt@\textbf{I., Innere Stadt}!Tiefer Graben 17@\textbf{Tiefer Graben 17}, \emph{Wohngebäude}|pw}\pend{}{\bigskip}\vspace{1em}
\pstart
           \raggedleft{}{\pb}Am Eibsee\oindex{Eibsee@\textbf{Eibsee}, \emph{See}|pw}{ }10. 9. 1917\pend
           \vspace{0.5em}
\pstart
           von einem{ }ſchönen Ausflug{ }ſenden wir Ihnen die herzlichſten Grüße.\pend
           
\pstart
           Ihr{\\[\baselineskip]}\spacefill\mbox{Arthur}\pend
           \leftskip=0em{}\selectlanguage{ngerman}\vspace{1em}\pstart {[}hs. Schnitzler:{]} Alles Herzliche, Ihre \spacefill\mbox{OlgaS.}\pend{}\selectlanguage{ngerman}\vspace{1em}
\pstart
           \noindent{}{[}hs. Steinrück:{]} In aller Verehrung, die ehemals kleine\pend
           \pstart \spacefill\mbox{Lisl}.\pend{}\selectlanguage{ngerman}\vspace{1em}
\pstart
           \noindent{}{[}hs. Marx:{]} Sehr verehrter Herr Schwarzkopf, nach
               unendlich lange\introOben{}r\introOben{}{ }\introOben{}Zeit\introOben{} wieder einmal herzlichſte Grüße und beſte Empfehlungen.
               Wann wird man{ }ſich wieder einmal{ }ſehen?\pend
           \pstart Ihr \spacefill\mbox{\substVorne{}\textsuperscript{Marx}\substDazwischen{}Paul\substHinten{}Marx}\pend{}\selectlanguage{ngerman}\endnumbering\briefempfaengerindex{Schwarzkopf, Gustav@\textsc{Schwarzkopf, Gustav}!zzzMarx, Paul@\emph{von Paul Marx}!1917-09-101@{10. 9. 1917}|)be}\briefempfaengerindex{Schwarzkopf, Gustav@\textsc{Schwarzkopf, Gustav}!zzzSteinrück, Elisabeth@\emph{von Elisabeth Steinrück}!1917-09-101@{10. 9. 1917}|)be}\briefempfaengerindex{Schwarzkopf, Gustav@\textsc{Schwarzkopf, Gustav}!zzzSchnitzler, Olga@\emph{von Olga Schnitzler}!1917-09-101@{10. 9. 1917}|)be}\briefempfaengerindex{Schwarzkopf, Gustav@\textsc{Schwarzkopf, Gustav}!zzzSchnitzler, Arthur@\emph{von Arthur Schnitzler}!1917-09-101@{10. 9. 1917}|)be}\mylabel{L04185h}
\begin{anhang}
\end{anhang}\newcommand{\dateiname}{L04185}\newcommand{\titel}{Arthur Schnitzler an Gustav Schwarzkopf, 10. 9. 1917}\newcommand{\editorInnen}{Herausgegeben von Jahnke, SelmaMüller, Martin Anton}%% latex-leseansicht-abspann.tex
%% Abspann für die Leseansicht.
%% Der Schalter \ifkorrekturansicht ist bereits durch den Vorspann gesetzt.

%% latex-abspann.tex
%% Gemeinsamer Abspann für Korrekturansicht und Leseansicht.
%% Setzt den Schalter \ifkorrekturansicht voraus (gesetzt in den
%% einbindenden Dateien latex-korrekturansicht-abspann.tex bzw.
%% latex-leseansicht-abspann.tex).
%% ---------------------------------------------------------------

\normalsize

% Das esempio-Environment wird nur in der Leseansicht benötigt
\ifkorrekturansicht\else
\newenvironment{esempio}[3]%
{
    \vspace{1.5ex}
    \rlap{\underline{#1}}
    \par
    \setlength{\parindent}{0cm}
    \nopagebreak
    \leftskip=#2cm
    \rightskip=#3cm
}
{
    \par
}
\fi

\doendnotes{C}
\bigskip
\vfill

\clearpage

\footnotesize

\ifkorrekturansicht
  \lohead{\textsc{register}}
\fi

% theindex-Environment neu definieren ohne reledmac
\makeatletter
\renewenvironment{theindex}{%
  \ifkorrekturansicht
    \section*{\indexname}%
  \else
    \subsubsection*{Index der erwähnten Entitäten}%
  \fi
  \setlength{\parindent}{0pt}%
  \setlength{\parskip}{0pt plus 0.3pt}%
  \let\item\@idxitem
}{%
  \ifkorrekturansicht\clearpage\fi
}
\makeatother

\IfFileExists{\jobname-pw.ind}{\input{\jobname-pw.ind}}{}

% Quellenangabe nur in der Leseansicht
\ifkorrekturansicht\else
% Fallback-Definitionen, falls die .tex-Datei \titel etc. nicht gesetzt hat
\providecommand{\titel}{}
\providecommand{\editorInnen}{}
\providecommand{\dateiname}{\jobname}

\vspace{3cm}

\vfill

\footnotesize
\textsc{Quelle}: \titel. Herausgegeben von {\editorInnen}. In: \emph{Arthur Schnitzler: Briefwechsel mit Autorinnen und Autoren}.
 Digitale Edition, https://schnitzler-briefe.acdh.oeaw.ac.at/{\dateiname}.html (Stand \today)
\fi

\end{document}


