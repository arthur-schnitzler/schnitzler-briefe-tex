\input{../tex-inputs/latex-pdf-vorspann}
\begin{center}
            \textcolor{red}{ENTWURF. ENTZIFFERUNG NOCH NICHT KORREKTURGELESEN}
                      \end{center}
            
               \section[Paul Goldmann an Arthur Schnitzler, 4. 9. {[}1893{]}]{ Paul Goldmann an Arthur Schnitzler, 4. 9. {[}1893{]}}\nopagebreak\mylabel{v}\rehead{ }\begin{ledgroupsized}[t]{13cm}\normalsize\beginnumbering\briefempfaengerindex{Schnitzler, Arthur@\textsc{Schnitzler, Arthur}!zzzGoldmann, Paul@\emph{von Paul Goldmann}!1893-09-041@{4. 9. {[}1893{]}}|(be} \toendnotes[C]{\smallbreak\pagebreak[2]} \Standort{DLA, A:Schnitzler, HS.NZ85.1.3163.}
\physDesc{Brief, 1 Blatt, 2 Seiten
\newline{}Handschrift: schwarze Tinte, deutsche Kurrent
\newline{}Schnitzler: mit Bleistift das Jahr »93« vermerkt }\toendnotes[C]{\smallbreak}\pstart
           \noindent{}\centering{}{\pb}\textcolor{gray}{\textbf{\textbf{\begin{otherlanguage}{french}HÔTEL DU DAUPHIN\end{otherlanguage}}}}\orgindex{Hôtel du Dauphin@Hôtel du Dauphin|pw}\pend
           \pstart
           \noindent{}\centering{}\textcolor{gray}{\textbf{\begin{otherlanguage}{french}FRÈRES OMARINI\end{otherlanguage}}}\pwindex{Omarini, Romeo @\textsc{Omarini, Romeo}, \emph{Hotelbesitzer}|pw}\pwindex{Omarini, Antonio @\textsc{Omarini, Antonio}, \emph{Hotelbesitzer}|pw}\pend
           \pstart
           \noindent{}\centering{}\textcolor{gray}{\textbf{\textbf{\begin{otherlanguage}{french}ISOLA BELLA\end{otherlanguage}}}}\oindex{Isola Bella@\textbf{Isola Bella}|pw}\pend
           \pstart
           \noindent{}\centering{}\textcolor{gray}{\textbf{\begin{otherlanguage}{french}ĴLES BORROMÈES\end{otherlanguage}\oindex{Borromean Islands@\textbf{Borromean Islands}|pw} - \begin{otherlanguage}{french}LAC MAJEUR\end{otherlanguage}\oindex{Lake Maggiore@\textbf{Lake Maggiore}|pw} - \begin{otherlanguage}{french}ITALIE\end{otherlanguage}\oindex{Italien@\textbf{Italien}|pw}}}\pend
           \pstart
           \noindent{}\textcolor{gray}{\textbf{\label{K_L02714-2v}\edtext{\textcolor{gray}{Stab. Sit P\textcolor{gray}{×}\-\textcolor{gray}{×}icco
                     Salvatore, Jutra}}{\lemma{\textnormal{\emph{Stab. … Jutra}}}\Cendnote{\textnormal{XXXX (vermutlich ein Name)}}}\label{K_L02714-2h}}}\pend
           \pstart
           \label{T_L02714-1v}\edtext{\textcolor{gray}{\textbf{TENU PAR LES}}}{\lemma{\textnormal{\emph{TENU PAR LES}}}\Cendnote{\textnormal{seitlich am linken Rand}}}\label{T_L02714-1h}\pend
           \pstart
           \raggedleft{}\label{T_L02714-2v}\edtext{\textcolor{gray}{\textbf{PROPRIÉTARIES}}\pwindex{Omarini, Romeo @\textsc{Omarini, Romeo}, \emph{Hotelbesitzer}|pwv}\pwindex{Omarini, Antonio @\textsc{Omarini, Antonio}, \emph{Hotelbesitzer}|pwv}}{\lemma{\textnormal{\emph{PROPRIÉTARIES}}}\Cendnote{\textnormal{seitlich am rechten Rand}}}\label{T_L02714-2h}\pend
           \pstart
           \noindent{}\centering{}\textsc{Isola Bella\oindex{Isola Bella@\textbf{Isola Bella}|pw}}, 4. September. Mein lieber
                  Arthur!\pend
           \pstart
           \noindent{}Es iſt ſchade, daß aus der gemeinſamen Reiſe nichts geworden iſt. Nun bleibe ich noch
               ein Paar Tage hier am \textsc{Lago Maggiore\oindex{Lake Maggiore@\textbf{Lake Maggiore}|pw}} und in Mailand\oindex{Mailand@\textbf{Mailand}|pw}. Dann fahre ich nach \textsc{Salzburg\oindex{Salzburg@\textbf{Salzburg}|pw}}. Wenn Du mir alſo die große Freude machen willſt, \label{K_L02714-1v}\edtext{hinüber zu kommen}{\lemma{\textnormal{\emph{hinüber zu kommen}}}\Cendnote{\textnormal{Siehe Paul Goldmann an Arthur Schnitzler, 18. 8. [1893]}}}\label{K_L02714-1h}, ſo halte Dich ſo um den 15. September herum
               bereit. {\pb}Sobald ich in Salzburg\oindex{Salzburg@\textbf{Salzburg}|pw} bin, telegraphire ich Dir meine Adreſſe und \strikeout{\textcolor{gray}{×}} erwarte dann die Nachricht von dem Datum Deiner Ankunft. Nach Wien\oindex{Wien@\textbf{Wien}|pw} komme ich nicht. Es thut mir noch Alles zu weh dort, und
               ich fürchte mich gar zu ſehr vor dem Wieder-Wegfahren. Wenn \textsc{Richard\pwindex{Beer-Hofmann, Richard 11.07.1866 – 26.09.1945@\textsc{Beer-Hofmann, Richard} (11.07.1866 – 26.09.1945), \emph{Schriftsteller}|pw}} oder \textsc{Loris\pwindex{Hofmannsthal, Hugo von 01.02.1874 – 15.07.1929@\textsc{Hofmannsthal, Hugo von} (01.02.1874 – 15.07.1929), \emph{Schriftsteller}|pw}} auch nach \textsc{Salzburg\oindex{Salzburg@\textbf{Salzburg}|pw}} kämen, ſo wäre das gar lieb von ihnen.\pend
           \pstart
           Auf baldiges Wieder ſehen alſo, mein lieber Freund!\pend
           \pstart
           Dein {\\[\baselineskip]}treuer {\\[\baselineskip]}\spacefill\mbox{Paul Goldmann}\pend
           \leftskip=0em{}\endnumbering\briefempfaengerindex{Schnitzler, Arthur@\textsc{Schnitzler, Arthur}!zzzGoldmann, Paul@\emph{von Paul Goldmann}!1893-09-041@{4. 9. {[}1893{]}}|)be}\mylabel{h}\end{ledgroupsized}\begin{anhang}\end{anhang}\newcommand{\dateiname}{L02714}\newcommand{\titel}{Paul Goldmann an Arthur Schnitzler, 4. 9. [1893]}\newcommand{\editorInnen}{Martin Anton Müller und Laura Untner}\input{../tex-inputs/latex-pdf-abspann}
      