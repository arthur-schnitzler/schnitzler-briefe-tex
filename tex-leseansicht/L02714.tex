%% latex-leseansicht-vorspann.tex
%% Vorspann für die Leseansicht.
%% Lädt die gemeinsame Datei latex-vorspann.tex mit nicht gesetztem Schalter.

\newif\ifkorrekturansicht
\korrekturansichtfalse

\input{../tex-inputs/latex-vorspann}


\section[Paul Goldmann an Arthur Schnitzler, 4. 9. [1893]]{L02714 Paul Goldmann an Arthur Schnitzler, 4. 9. [1893]}
\nopagebreak\mylabel{L02714v}
\rehead{ }\normalsize\beginnumbering\briefempfaengerindex{Schnitzler, Arthur@\textsc{Schnitzler, Arthur}!zzzGoldmann, Paul@\emph{von Paul Goldmann}!1893-09-041@{4. 9. [1893]}|(be}
\toendnotes[C]{\smallbreak\pagebreak[2]}
\correspDesc{Versand  durch Paul Goldmann am 4. 9. [1893] in Isola Bella
\newline{}Erhalt  durch Arthur Schnitzler im Zeitraum [5. 9. 1893
                  – 9. 9. 1893?] in Wien}\toendnotes[C]{\smallbreak}
\Standort{DLA, A:Schnitzler, HS.NZ85.1.3163.}
\physDesc{Brief, 1 Blatt, 2 Seiten, 713 Zeichen
\newline{}Handschrift: schwarze Tinte, deutsche Kurrent
\newline{}Schnitzler: mit Bleistift das Jahr »93« vermerkt }\toendnotes[C]{\smallbreak}
\pstart
           \centering{}{\pb}\textcolor{gray}{\textbf{\textbf{\begin{otherlanguage}{french}HÔTEL DU DAUPHIN\end{otherlanguage}}}}\orgindex{Hôtel du Dauphin@Hôtel du Dauphin|pw}\pend
           
\pstart
           \centering{}\textcolor{gray}{\textbf{\begin{otherlanguage}{french}FRÈRES OMARINI\end{otherlanguage}}}\pwindex{Omarini, Romeo @\textsc{Omarini, Romeo}, \emph{Hotelbesitzer}|pw}\pwindex{Omarini, Antonio @\textsc{Omarini, Antonio}, \emph{Hotelbesitzer}|pw}\pend
           
\pstart
           \centering{}\textcolor{gray}{\textbf{TENU PAR LES PROPRIÉTAIRES}}\pend
           
\pstart
           \centering{}\textcolor{gray}{\textbf{\textbf{\begin{otherlanguage}{french}ISOLA BELLA\end{otherlanguage}}}}\oindex{Isola Bella@\textbf{Isola Bella}, \emph{Insel}|pw}\pend
           
\pstart
           \centering{}\textcolor{gray}{\textbf{\begin{otherlanguage}{french}ÎLES BORROMÈES\end{otherlanguage}\oindex{Borromäische Inseln@\textbf{Borromäische Inseln}, \emph{Insel}|pw} – \begin{otherlanguage}{french}LAC MAJEUR\end{otherlanguage}\oindex{Lago Maggiore@\textbf{Lago Maggiore}, \emph{See}|pw} – \begin{otherlanguage}{french}ITALIE\end{otherlanguage}\oindex{Italien@\textbf{Italien}|pw}}}\pend
           
\pstart
           \centering{}\textsc{Isola Bella\oindex{Isola Bella@\textbf{Isola Bella}, \emph{Insel}|pw}}, 4. September.\pend
           
\pstart\center{}Mein lieber Arthur!\pend\vspace{0.5em}
\pstart
           Es iſt{ }ſchade, daß aus der gemeinſamen Reiſe nichts geworden iſt. Nun bleibe ich noch
               ein paar Tage hier am \textsc{Lago Maggiore\oindex{Lago Maggiore@\textbf{Lago Maggiore}, \emph{See}|pw}} und in Mailand\oindex{Mailand@\textbf{Mailand}|pw}. Dann fahre ich nach \textsc{Salzburg\oindex{Salzburg@\textbf{Salzburg}, \emph{Verwaltungsgebiet}|pw}}. Wenn Du mir alſo die große Freude machen willſt, \label{K_L02714-1v}\edtext{hinüber zu kommen}{\lemma{\textnormal{\emph{hinüber zu kommen}}}\Cendnote{\textnormal{Siehe XXXX Auszeichnungsfehler: Dokument L02712 nicht gefunden.
               }}}\label{K_L02714-1},{ }ſo halte Dich{ }ſo um den 15. September herum
               bereit. {\pb}Sobald ich in Salzburg\oindex{Salzburg@\textbf{Salzburg}, \emph{Verwaltungsgebiet}|pw} bin, telegraphire ich Dir meine Adreſſe und \strikeout{\textcolor{gray}{×}} erwarte dann die Nachricht von dem Datum Deiner Ankunft. Nach Wien\oindex{Wien@\textbf{Wien}, \emph{Verwaltungsgebiet}|pw} komme ich nicht. Es thut mir noch Alles zu weh dort, und
               ich fürchte mich gar zu{ }ſehr vor dem Wieder-Wegfahren. Wenn \textsc{Richard\pwindex{Beer-Hofmann, Richard 11.\,7.\,1866 Wien – 26.\,9.\,1945 New York City@\textsc{Beer-Hofmann, Richard} (11.\,7.\,1866 Wien – 26.\,9.\,1945 New York City), \emph{Schriftsteller}|pw}} oder \textsc{Loris\pwindex{Hofmannsthal, Hugo von 1.\,2.\,1874 Wien – 15.\,7.\,1929 Rodaun@\textsc{Hofmannsthal, Hugo von} (1.\,2.\,1874 Wien – 15.\,7.\,1929 Rodaun), \emph{Schriftsteller}|pw}} auch nach \textsc{Salzburg\oindex{Salzburg@\textbf{Salzburg}, \emph{Verwaltungsgebiet}|pw}} kämen,{ }ſo wäre das gar lieb von ihnen.\pend
           
\pstart
           Auf baldiges Wiederſehen alſo, mein lieber Freund!\pend
           
\pstart
           Dein {\\[\baselineskip]}treuer {\\[\baselineskip]}\spacefill\mbox{Paul Goldmann}\pend
           \leftskip=0em{}\selectlanguage{ngerman}\endnumbering\briefempfaengerindex{Schnitzler, Arthur@\textsc{Schnitzler, Arthur}!zzzGoldmann, Paul@\emph{von Paul Goldmann}!1893-09-041@{4. 9. [1893]}|)be}\mylabel{L02714h}  \newcommand{\dateiname}{L02714}\newcommand{\titel}{Paul Goldmann an Arthur Schnitzler, 4. 9. [1893]}\newcommand{\editorInnen}{Martin Anton Müller und Laura Untner}%% latex-leseansicht-abspann.tex
%% Abspann für die Leseansicht.
%% Der Schalter \ifkorrekturansicht ist bereits durch den Vorspann gesetzt.

%% latex-abspann.tex
%% Gemeinsamer Abspann für Korrekturansicht und Leseansicht.
%% Setzt den Schalter \ifkorrekturansicht voraus (gesetzt in den
%% einbindenden Dateien latex-korrekturansicht-abspann.tex bzw.
%% latex-leseansicht-abspann.tex).
%% ---------------------------------------------------------------

\normalsize

% Das esempio-Environment wird nur in der Leseansicht benötigt
\ifkorrekturansicht\else
\newenvironment{esempio}[3]%
{
    \vspace{1.5ex}
    \rlap{\underline{#1}}
    \par
    \setlength{\parindent}{0cm}
    \nopagebreak
    \leftskip=#2cm
    \rightskip=#3cm
}
{
    \par
}
\fi

\doendnotes{C}
\bigskip
\vfill

\clearpage

\footnotesize

\ifkorrekturansicht
  \lohead{\textsc{register}}
\fi

% theindex-Environment neu definieren ohne reledmac
\makeatletter
\renewenvironment{theindex}{%
  \ifkorrekturansicht
    \section*{\indexname}%
  \else
    \subsubsection*{Index der erwähnten Entitäten}%
  \fi
  \setlength{\parindent}{0pt}%
  \setlength{\parskip}{0pt plus 0.3pt}%
  \let\item\@idxitem
}{%
  \ifkorrekturansicht\clearpage\fi
}
\makeatother

\IfFileExists{\jobname-pw.ind}{\input{\jobname-pw.ind}}{}

% Quellenangabe nur in der Leseansicht
\ifkorrekturansicht\else
% Fallback-Definitionen, falls die .tex-Datei \titel etc. nicht gesetzt hat
\providecommand{\titel}{}
\providecommand{\editorInnen}{}
\providecommand{\dateiname}{\jobname}

\vspace{3cm}

\vfill

\footnotesize
\textsc{Quelle}: \titel. Herausgegeben von {\editorInnen}. In: \emph{Arthur Schnitzler: Briefwechsel mit Autorinnen und Autoren}.
 Digitale Edition, https://schnitzler-briefe.acdh.oeaw.ac.at/{\dateiname}.html (Stand \today)
\fi

\end{document}


