%% latex-korrekturansicht-vorspann.tex
%% Vorspann für die Korrekturansicht.
%% Lädt die gemeinsame Datei latex-vorspann.tex mit gesetztem Schalter.

\newif\ifkorrekturansicht
\korrekturansichttrue

\input{../tex-inputs/latex-vorspann}


\section[Paul Goldmann an Arthur Schnitzler, 4. 9. {[}1893{]}]{L02714 Paul Goldmann an Arthur Schnitzler, 4. 9. {[}1893{]}}
\nopagebreak\mylabel{L02714v}
\rehead{ }\normalsize\beginnumbering\briefempfaengerindex{Schnitzler, Arthur@\textsc{Schnitzler, Arthur}!zzzGoldmann, Paul@\emph{von Paul Goldmann}!1893-09-041@{4. 9. {[}1893{]}}|(be}
\toendnotes[C]{\smallbreak\pagebreak[2]}\Standort{DLA, A:Schnitzler, HS.NZ85.1.3163.}
\physDesc{Brief, 1 Blatt, 2 Seiten, 713 Zeichen
\newline{}Handschrift: schwarze Tinte, deutsche Kurrent
\newline{}Schnitzler: mit Bleistift das Jahr »93« vermerkt }\toendnotes[C]{\smallbreak}
\pstart
           \centering{}{\pb}\textcolor{gray}{\textbf{\textbf{\begin{otherlanguage}{french}HÔTEL DU DAUPHIN\end{otherlanguage}}}}\orgindex{Hôtel du Dauphin@Hôtel du Dauphin|pw}\pend
           
\pstart
           \centering{}\textcolor{gray}{\textbf{\begin{otherlanguage}{french}FRÈRES OMARINI\end{otherlanguage}}}\pwindex{Omarini, Romeo @\textsc{Omarini, Romeo}, \emph{Hotelbesitzer/Hotelbesitzerin}|pw}\pwindex{Omarini, Antonio @\textsc{Omarini, Antonio}, \emph{Hotelbesitzer/Hotelbesitzerin}|pw}\pend
           
\pstart
           \centering{}\textcolor{gray}{\textbf{TENU PAR LES PROPRIÉTAIRES}}\pend
           
\pstart
           \centering{}\textcolor{gray}{\textbf{\textbf{\begin{otherlanguage}{french}ISOLA BELLA\end{otherlanguage}}}}\oindex{Isola Bella@\textbf{Isola Bella}, \emph{T.ISL}|pw}\pend
           
\pstart
           \centering{}\textcolor{gray}{\textbf{\begin{otherlanguage}{french}ÎLES BORROMÈES\end{otherlanguage}\oindex{Borromaeische Inseln@\textbf{Borromäische Inseln}, \emph{Insel (N.INS)}|pw} – \begin{otherlanguage}{french}LAC MAJEUR\end{otherlanguage}\oindex{Lago Maggiore@\textbf{Lago Maggiore}, \emph{H.LK}|pw} – \begin{otherlanguage}{french}ITALIE\end{otherlanguage}\oindex{Italien@\textbf{Italien}, \emph{A.PCLI}|pw}}}\pend
           
\pstart
           \centering{}\textsc{Isola Bella\oindex{Isola Bella@\textbf{Isola Bella}, \emph{T.ISL}|pw}}, 4. September. \pend
           
\pstart\center{}Mein lieber Arthur!\pend\vspace{0.5em}
\pstart
           Es iſt ſchade, daß aus der gemeinſamen Reiſe nichts geworden iſt. Nun bleibe ich noch
               ein paar Tage hier am \textsc{Lago Maggiore\oindex{Lago Maggiore@\textbf{Lago Maggiore}, \emph{H.LK}|pw}} und in Mailand\oindex{Mailand@\textbf{Mailand}, \emph{P.PPLA}|pw}. Dann fahre ich nach \textsc{Salzburg\oindex{Salzburg@\textbf{Salzburg}, \emph{A.ADM2}|pw}}. Wenn Du mir alſo die große Freude machen willſt, \label{K_L02714-1v}\edtext{hinüber zu kommen}{\lemma{\textnormal{\emph{hinüber zu kommen}}}\Cendnote{\textnormal{Siehe Paul Goldmann an Arthur Schnitzler, 18. 8. [1893].
               }}}\label{K_L02714-1}, ſo halte Dich ſo um den 15. September herum
               bereit. {\pb}Sobald ich in Salzburg\oindex{Salzburg@\textbf{Salzburg}, \emph{A.ADM2}|pw} bin, telegraphire ich Dir meine Adreſſe und \strikeout{\textcolor{gray}{×}} erwarte dann die Nachricht von dem Datum Deiner Ankunft. Nach Wien\oindex{Wien@\textbf{Wien}, \emph{A.ADM2}|pw} komme ich nicht. Es thut mir noch Alles zu weh dort, und
               ich fürchte mich gar zu ſehr vor dem Wieder-Wegfahren. Wenn \textsc{Richard\pwindex{Beer-Hofmann, Richard 1866-07-11 – 1945-09-26@\textsc{Beer-Hofmann, Richard} (1866-07-11 – 1945-09-26), \emph{Schriftsteller/Schriftstellerin}|pw}} oder \textsc{Loris\pwindex{Hofmannsthal, Hugo von 1874-02-01 – 1929-07-15@\textsc{Hofmannsthal, Hugo von} (1874-02-01 – 1929-07-15), \emph{Schriftsteller/Schriftstellerin}|pw}} auch nach \textsc{Salzburg\oindex{Salzburg@\textbf{Salzburg}, \emph{A.ADM2}|pw}} kämen, ſo wäre das gar lieb von ihnen.\pend
           
\pstart
           Auf baldiges Wiederſehen alſo, mein lieber Freund!\pend
           
\pstart
           Dein {\\[\baselineskip]}treuer {\\[\baselineskip]}\spacefill\mbox{Paul Goldmann}\pend
           \leftskip=0em{}\selectlanguage{ngerman}\endnumbering\briefempfaengerindex{Schnitzler, Arthur@\textsc{Schnitzler, Arthur}!zzzGoldmann, Paul@\emph{von Paul Goldmann}!1893-09-041@{4. 9. {[}1893{]}}|)be}\mylabel{L02714h}  \normalsize

\doendnotes{C}
\bigskip
\vfill

\clearpage

\footnotesize

\lohead{\textsc{register}}

% Definiere theindex-Environment komplett neu ohne reledmac
\makeatletter
\renewenvironment{theindex}{%
  \section*{\indexname}%
  \setlength{\parindent}{0pt}%
  \setlength{\parskip}{0pt plus 0.3pt}%
  \let\item\@idxitem
}{%
  \clearpage
}
\makeatother

\IfFileExists{\jobname-pw.ind}{\input{\jobname-pw.ind}}{}

\end{document}

      