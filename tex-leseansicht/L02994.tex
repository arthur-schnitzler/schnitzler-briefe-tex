%% latex-korrekturansicht-vorspann.tex
%% Vorspann für die Korrekturansicht.
%% Lädt die gemeinsame Datei latex-vorspann.tex mit gesetztem Schalter.

\newif\ifkorrekturansicht
\korrekturansichttrue

\input{../tex-inputs/latex-vorspann}


\section[ Arthur Schnitzler an Felix Salten, {[}23. 12. 1904?{]}]{L02994 Arthur Schnitzler an Felix Salten, {[}23. 12. 1904?{]}}
\nopagebreak\mylabel{L02994v}
\rehead{ }\normalsize\beginnumbering\briefempfaengerindex{Salten, Felix@\textsc{Salten, Felix}!zzzSchnitzler, Arthur@\emph{von Arthur Schnitzler}!1904-12-232@{{[}23. 12. 1904?{]}}|(be}
\toendnotes[C]{\smallbreak\pagebreak[2]}\Standort{Wienbibliothek im Rathaus, ZPH 1681, 2.1.516.}
\physDesc{Brief, 1 Blatt, 3 Seiten, 683 Zeichen
\newline{}Handschrift: Bleistift, deutsche Kurrent
\newline{}Ordnung: mit Bleistift von unbekannter Hand Nummerierung der Doppelseiten des
                                 Konvoluts: »11«–»12« }\toendnotes[C]{\smallbreak}
\pstart
           \noindent{}{\pb}lieber, wir haben geſtern{ }Abend ¾ Stunden gewartet, dachten umſoweniger dran, dſs Sie noch kommen
               würden, als Sie mir ja \label{K_L02994-1v}\edtext{geſchrieben}{\lemma{\textnormal{\emph{geſchrieben}}}\Cendnote{\textnormal{Siehe Felix Salten an Arthur Schnitzler, [20. 12. 1904].
               }}}\label{K_L02994-1} hatten, daſs Sie auch im Concert\pwindex{3. Sinfonie in d-Moll@\emph{3. Sinfonie in d-Moll}|pwv} wären und vom Concert\pwindex{3. Sinfonie in d-Moll@\emph{3. Sinfonie in d-Moll}|pwv} aus \substVorne{}\textsuperscript{\textcolor{gray}{kämen}}\substDazwischen{}in den\substHinten{}{ }Riedhof\oindex{Riedhof@\textbf{Riedhof}, \emph{Lokal (K.LKL)}|pw} gehen {\pb}würden. Ich dachte natürlich an eine
               redactionelle oder ſonſtige Verhinderung Ihrerſeits, und ſo gingen wir, zwar mit
               Bedauern, aber höchſt unſchuldsvoll, nach Hause.\pend
           
\pstart
           Ich grüße Sie herzlich und wünſche Ihnen, nebſt allem ſchönen, daſs der Genius Ihrer
                  {\pb}Empfindlichkeit zur Hölle fahre.\pend
           
\pstart
           Ihr {\\[\baselineskip]}\spacefill\mbox{A.}\pend
           \leftskip=0em{}
\pstart
           \noindent{}Heute wollten wir zu \label{K_L02994-2v}\edtext{Triſtan\pwindex{Tristan und Isolde@\emph{Tristan und Isolde}|pw}}{\lemma{\textnormal{\emph{Triſtan}}}\Cendnote{\textnormal{Richard Wagners\pwindex{Wagner, Richard 22.05.1813 – 13.02.1883@\textsc{Wagner, Richard} (22.05.1813 – 13.02.1883), \emph{Komponist/Komponistin}|pwk}{ }\emph{Tristan und Isolde}\pwindex{Tristan und Isolde@\emph{Tristan und Isolde}|pwk} wurde in der Oper\oindex{Oper@\textbf{Oper}, \emph{Oper (K.OPR)}|pwk} gegeben. Die weibliche Titelrolle sang Anna von
                     Mildenburg\pwindex{Bahr-Mildenburg, Anna 29.11.1872 – 27.01.1947@\textsc{Bahr-Mildenburg, Anna} (29.11.1872 – 27.01.1947), \emph{Sänger/Sängerin}|pwk}.}}}\label{K_L02994-2}\textcolor{gray}{,} haben nichts mehr bekommen, ſind
                  wieder Erwarten heim{[};{]} theilen Sie mir bitte ein Wort \introOben{}\textsc{pneumatisch}\introOben{} ob Sie und Otti\pwindex{Salten, Ottilie 07.03.1868 – 22.06.1942@\textsc{Salten, Ottilie} (07.03.1868 – 22.06.1942), \emph{Schauspieler/Schauspielerin}|pw}{ }heute{ }Abend 9 Uhr im Riedhof\oindex{Riedhof@\textbf{Riedhof}, \emph{Lokal (K.LKL)}|pw} mit uns
                  nachtmahlen wollen.\pend
           
\pstart
           \spacefill\mbox{A.}\pend
           \selectlanguage{ngerman}\endnumbering\briefempfaengerindex{Salten, Felix@\textsc{Salten, Felix}!zzzSchnitzler, Arthur@\emph{von Arthur Schnitzler}!1904-12-232@{{[}23. 12. 1904?{]}}|)be}\mylabel{L02994h}  \normalsize

\doendnotes{C}
\bigskip
\vfill

\clearpage

\footnotesize

\lohead{\textsc{register}}

% Definiere theindex-Environment komplett neu ohne reledmac
\makeatletter
\renewenvironment{theindex}{%
  \section*{\indexname}%
  \setlength{\parindent}{0pt}%
  \setlength{\parskip}{0pt plus 0.3pt}%
  \let\item\@idxitem
}{%
  \clearpage
}
\makeatother

\IfFileExists{\jobname-pw.ind}{\input{\jobname-pw.ind}}{}

\end{document}

      