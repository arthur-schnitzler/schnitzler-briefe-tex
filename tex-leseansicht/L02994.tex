%% latex-leseansicht-vorspann.tex
%% Vorspann für die Leseansicht.
%% Lädt die gemeinsame Datei latex-vorspann.tex mit nicht gesetztem Schalter.

\newif\ifkorrekturansicht
\korrekturansichtfalse

\input{../tex-inputs/latex-vorspann}


\section[ Arthur Schnitzler an Felix Salten, {[}23. 12. 1904?{]}]{L02994 Arthur Schnitzler an Felix Salten,  [23. 12. 1904?]}
\nopagebreak\mylabel{L02994v}
\rehead{ }\normalsize\beginnumbering\briefempfaengerindex{Salten, Felix@\textsc{Salten, Felix}!zzzSchnitzler, Arthur@\emph{von Arthur Schnitzler}!1904-12-232@{{[}23. 12. 1904?{]}}|(be}
\toendnotes[C]{\smallbreak\pagebreak[2]}
\correspDesc{Versand  durch Arthur Schnitzler am [23. 12. 1904?] in Wien
\newline{}Erhalt  durch Felix Salten am [23. 12. 1904?] in Wien}\toendnotes[C]{\smallbreak}
\Standort{Wienbibliothek im Rathaus, ZPH 1681, 2.1.516.}
\physDesc{Brief, 1 Blatt, 3 Seiten, 683 Zeichen
\newline{}Handschrift: Bleistift, deutsche Kurrent
\newline{}Ordnung: mit Bleistift von unbekannter Hand Nummerierung der Doppelseiten des
                                 Konvoluts: »11«–»12« }\toendnotes[C]{\smallbreak}
\pstart
           \noindent{}{\pb}lieber, wir haben geſtern{ }Abend ¾ Stunden gewartet, dachten umſoweniger dran, dſs Sie noch kommen
               würden, als Sie mir ja \label{K_L02994-1v}\edtext{geſchrieben}{\lemma{\textnormal{\emph{geschrieben}}}\Cendnote{\textnormal{Siehe XXXX Auszeichnungsfehler: Dokument L03402 nicht gefunden.
               }}}\label{K_L02994-1} hatten, daſs Sie auch im Concert\pwindex{\textcolor{red}{\textsuperscript{XXXX indx1}}!3. Sinfonie in d-Moll@\strich\emph{3. Sinfonie in d-Moll}|pwv} wären und vom Concert\pwindex{\textcolor{red}{\textsuperscript{XXXX indx1}}!3. Sinfonie in d-Moll@\strich\emph{3. Sinfonie in d-Moll}|pwv} aus \substVorne{}\textsuperscript{\textcolor{gray}{kämen}}\substDazwischen{}in den\substHinten{}{ }Riedhof\oindex{Wien@\textbf{Wien}!VIII., Josefstadt@\textbf{VIII., Josefstadt}!Riedhof@\textbf{Riedhof}, \emph{Lokal}|pw} gehen {\pb}würden. Ich dachte natürlich an eine
               redactionelle oder{ }ſonſtige Verhinderung Ihrerſeits, und{ }ſo gingen wir, zwar mit
               Bedauern, aber höchſt unſchuldsvoll, nach Hause.\pend
           
\pstart
           Ich grüße Sie herzlich und wünſche Ihnen, nebſt allem{ }ſchönen, daſs der Genius Ihrer
                  {\pb}Empfindlichkeit zur Hölle fahre.\pend
           
\pstart
           Ihr {\\[\baselineskip]}\spacefill\mbox{A.}\pend
           \leftskip=0em{}
\pstart
           \noindent{}Heute wollten wir zu \label{K_L02994-2v}\edtext{Triſtan\pwindex{Wagner, Richard 22.\,5.\,1813 Leipzig – 13.\,2.\,1883 Venedig@\textsc{Wagner, Richard} (22.\,5.\,1813 Leipzig – 13.\,2.\,1883 Venedig), \emph{Komponist}!Tristan und Isolde@\strich\emph{Tristan und Isolde}|pw}}{\lemma{\textnormal{\emph{Tristan}}}\Cendnote{\textnormal{Richard Wagners\pwindex{Wagner, Richard 22.\,5.\,1813 Leipzig – 13.\,2.\,1883 Venedig@\textsc{Wagner, Richard} (22.\,5.\,1813 Leipzig – 13.\,2.\,1883 Venedig), \emph{Komponist}|pwk}{ }\emph{Tristan und Isolde}\pwindex{Wagner, Richard 22.\,5.\,1813 Leipzig – 13.\,2.\,1883 Venedig@\textsc{Wagner, Richard} (22.\,5.\,1813 Leipzig – 13.\,2.\,1883 Venedig), \emph{Komponist}!Tristan und Isolde@\strich\emph{Tristan und Isolde}|pwk} wurde in der Oper\oindex{Wien@\textbf{Wien}!I., Innere Stadt@\textbf{I., Innere Stadt}!Oper@\textbf{Oper}, \emph{Oper}|pwk} gegeben. Die weibliche Titelrolle sang Anna von
                     Mildenburg\pwindex{Bahr-Mildenburg, Anna 29.\,11.\,1872 Wien – 27.\,1.\,1947 ebd.@\textsc{Bahr-Mildenburg, Anna} (29.\,11.\,1872 Wien – 27.\,1.\,1947 ebd.), \emph{Sängerin}|pwk}.}}}\label{K_L02994-2}\textcolor{gray}{,} haben nichts mehr bekommen,{ }ſind
                  wieder Erwarten heim{[};{]} theilen Sie mir bitte ein Wort \introOben{}\textsc{pneumatisch}\introOben{} ob Sie und Otti\pwindex{Salten, Ottilie 7.\,3.\,1868 Prag – 22.\,6.\,1942 Zürich@\textsc{Salten, Ottilie} (7.\,3.\,1868 Prag – 22.\,6.\,1942 Zürich), \emph{Schauspielerin}|pw}{ }heute{ }Abend 9 Uhr im Riedhof\oindex{Wien@\textbf{Wien}!VIII., Josefstadt@\textbf{VIII., Josefstadt}!Riedhof@\textbf{Riedhof}, \emph{Lokal}|pw} mit uns
                  nachtmahlen wollen.\pend
           
\pstart
           \spacefill\mbox{A.}\pend
           \selectlanguage{ngerman}\endnumbering\briefempfaengerindex{Salten, Felix@\textsc{Salten, Felix}!zzzSchnitzler, Arthur@\emph{von Arthur Schnitzler}!1904-12-232@{{[}23. 12. 1904?{]}}|)be}\mylabel{L02994h}  \newcommand{\dateiname}{L02994}\newcommand{\titel}{Arthur Schnitzler an Felix Salten, [23. 12. 1904?]}\newcommand{\editorInnen}{Martin Anton Müller und Laura Untner}%% latex-leseansicht-abspann.tex
%% Abspann für die Leseansicht.
%% Der Schalter \ifkorrekturansicht ist bereits durch den Vorspann gesetzt.

%% latex-abspann.tex
%% Gemeinsamer Abspann für Korrekturansicht und Leseansicht.
%% Setzt den Schalter \ifkorrekturansicht voraus (gesetzt in den
%% einbindenden Dateien latex-korrekturansicht-abspann.tex bzw.
%% latex-leseansicht-abspann.tex).
%% ---------------------------------------------------------------

\normalsize

% Das esempio-Environment wird nur in der Leseansicht benötigt
\ifkorrekturansicht\else
\newenvironment{esempio}[3]%
{
    \vspace{1.5ex}
    \rlap{\underline{#1}}
    \par
    \setlength{\parindent}{0cm}
    \nopagebreak
    \leftskip=#2cm
    \rightskip=#3cm
}
{
    \par
}
\fi

\doendnotes{C}
\bigskip
\vfill

\clearpage

\footnotesize

\ifkorrekturansicht
  \lohead{\textsc{register}}
\fi

% theindex-Environment neu definieren ohne reledmac
\makeatletter
\renewenvironment{theindex}{%
  \ifkorrekturansicht
    \section*{\indexname}%
  \else
    \subsubsection*{Index der erwähnten Entitäten}%
  \fi
  \setlength{\parindent}{0pt}%
  \setlength{\parskip}{0pt plus 0.3pt}%
  \let\item\@idxitem
}{%
  \ifkorrekturansicht\clearpage\fi
}
\makeatother

\IfFileExists{\jobname-pw.ind}{\input{\jobname-pw.ind}}{}

% Quellenangabe nur in der Leseansicht
\ifkorrekturansicht\else
% Fallback-Definitionen, falls die .tex-Datei \titel etc. nicht gesetzt hat
\providecommand{\titel}{}
\providecommand{\editorInnen}{}
\providecommand{\dateiname}{\jobname}

\vspace{3cm}

\vfill

\footnotesize
\textsc{Quelle}: \titel. Herausgegeben von {\editorInnen}. In: \emph{Arthur Schnitzler: Briefwechsel mit Autorinnen und Autoren}.
 Digitale Edition, https://schnitzler-briefe.acdh.oeaw.ac.at/{\dateiname}.html (Stand \today)
\fi

\end{document}


