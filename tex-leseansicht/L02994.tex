%% latex-leseansicht-vorspann.tex
%% Vorspann für die Leseansicht.
%% Lädt die gemeinsame Datei latex-vorspann.tex mit nicht gesetztem Schalter.

\newif\ifkorrekturansicht
\korrekturansichtfalse

\input{../tex-inputs/latex-vorspann}


         
         \renewcommand{\erwaehntePersonen}{Personen: Felix Salten, Ottilie Salten, Richard Wagner}
         \renewcommand{\erwaehnteOrte}{Orte: Oper, Riedhof, Wien}
         \renewcommand{\erwaehnteWerke}{Werke: Symphonie Nr. 3 D-Moll, Tristan und Isolde}
               \section[ Arthur Schnitzler an Felix Salten, {[}23. 12. 1904?{]}]{ Arthur Schnitzler an Felix Salten, {[}23. 12. 1904?{]}}\nopagebreak\mylabel{v}\rehead{ }\begin{ledgroupsized}[t]{13cm}\normalsize\beginnumbering\briefempfaengerindex{Salten, Felix@\textsc{Salten, Felix}!zzzSchnitzler, Arthur@\emph{von Arthur Schnitzler}!1904-12-232@{{[}23. 12. 1904?{]}}|(be} \toendnotes[C]{\smallbreak\pagebreak[2]} \Standort{Wienbibliothek im Rathaus, ZPH 1681, 2.1.516.}
\physDesc{Brief, 1 Blatt, 3 Seiten, 683 Zeichen
\newline{}Handschrift: Bleistift, deutsche Kurrent
\newline{}Ordnung: mit Bleistift von unbekannter Hand Nummerierung der Doppelseiten des
                                 Konvoluts: »11«–»12« }\toendnotes[C]{\smallbreak}\pstart
           \noindent{}{\pb}lieber, wir haben geſtern{ }Abend ¾ Stunden gewartet, dachten umſoweniger dran, dſs Sie noch kommen
               würden, als Sie mir ja \label{K_L02994-1v}\edtext{geſchrieben}{\lemma{\textnormal{\emph{geſchrieben}}}\Cendnote{\textnormal{siehe Felix Salten an Arthur Schnitzler, [20. 12. 1904]}}}\label{K_L02994-1h} hatten, daſs Sie auch im Concert\pwindex{\textcolor{red}{\textsuperscript{XXXX1 indx}}!Symphonie Nr. 3 D-Moll1902@\strich\emph{Symphonie Nr. 3 D-Moll} {[}1902{]}|pwv} wären und vom Concert\pwindex{\textcolor{red}{\textsuperscript{XXXX1 indx}}!Symphonie Nr. 3 D-Moll1902@\strich\emph{Symphonie Nr. 3 D-Moll} {[}1902{]}|pwv} aus \substVorne{}\textsuperscript{\textcolor{gray}{kämen}}\substDazwischen{}in den\substHinten{}{ }Riedhof\oindex{Riedhof@\textbf{Riedhof}|pw} gehen {\pb}würden. Ich dachte natürlich an eine
               redactionelle oder ſonſtige Verhinderung Ihrerſeits, und ſo gingen wir, zwar mit
               Bedauern, aber höchſt unſchuldsvoll, nach Hause.\pend
           \pstart
           Ich grüße Sie herzlich und wünſche Ihnen, nebſt allem ſchönen, daſs der Genius Ihrer
                  {\pb}Empfindlichkeit zur Hölle fahre.\pend
           \pstart
           Ihr {\\[\baselineskip]}\spacefill\mbox{A.}\pend
           \leftskip=0em{}\pstart
           \noindent{}Heute wollten wir zu \label{K_L02994-2v}\edtext{Triſtan\pwindex{Wagner, Richard 22.05.1813 – 13.02.1883@\textsc{Wagner, Richard} (22.05.1813 – 13.02.1883), \emph{Komponist}!Tristan und Isolde1865@\strich\emph{Tristan und Isolde} {[}1865{]}|pw}}{\lemma{\textnormal{\emph{Triſtan}}}\Cendnote{\textnormal{Richard Wagner\pwindex{Wagner, Richard 22.05.1813 – 13.02.1883@\textsc{Wagner, Richard} (22.05.1813 – 13.02.1883), \emph{Komponist}|pwk}s \emph{Tristan und Isolde}\pwindex{Wagner, Richard 22.05.1813 – 13.02.1883@\textsc{Wagner, Richard} (22.05.1813 – 13.02.1883), \emph{Komponist}!Tristan und Isolde1865@\strich\emph{Tristan und Isolde} {[}1865{]}|pwk} wurde in der Oper\oindex{Oper@\textbf{Oper}|pwk} gegeben. Die weibliche Titelrolle sang Anna von
                     Mildenburg\pwindex{\textcolor{red}{\textsuperscript{XXXX1 indx}}|pwk}.}}}\label{K_L02994-2h}\textcolor{gray}{,} haben nichts mehr bekommen, ſind
                  wieder Erwarten heim{[};{]} theilen Sie mir bitte ein Wort \introOben{}\textsc{pneumatisch}\introOben{} ob Sie und Otti\pwindex{Salten, Ottilie 07.03.1868 – 22.06.1942@\textsc{Salten, Ottilie} (07.03.1868 – 22.06.1942), \emph{Schauspielerin}|pw}{ }heute{ }Abend 9 Uhr im Riedhof\oindex{Riedhof@\textbf{Riedhof}|pw} mit uns
                  nachtmahlen wollen.\pend
           \pstart
           \spacefill\mbox{A.}\pend
           
         
         \endnumbering\mylabel{h}\end{ledgroupsized}  \newcommand{\dateiname}{L02994}\newcommand{\titel}{Arthur Schnitzler an Felix Salten, [23. 12. 1904?]}\newcommand{\editorInnen}{Martin Anton Müller und Laura Untner}%% latex-leseansicht-abspann.tex
%% Abspann für die Leseansicht.
%% Der Schalter \ifkorrekturansicht ist bereits durch den Vorspann gesetzt.

%% latex-abspann.tex
%% Gemeinsamer Abspann für Korrekturansicht und Leseansicht.
%% Setzt den Schalter \ifkorrekturansicht voraus (gesetzt in den
%% einbindenden Dateien latex-korrekturansicht-abspann.tex bzw.
%% latex-leseansicht-abspann.tex).
%% ---------------------------------------------------------------

\normalsize

% Das esempio-Environment wird nur in der Leseansicht benötigt
\ifkorrekturansicht\else
\newenvironment{esempio}[3]%
{
    \vspace{1.5ex}
    \rlap{\underline{#1}}
    \par
    \setlength{\parindent}{0cm}
    \nopagebreak
    \leftskip=#2cm
    \rightskip=#3cm
}
{
    \par
}
\fi

\doendnotes{C}
\bigskip
\vfill

\clearpage

\footnotesize

\ifkorrekturansicht
  \lohead{\textsc{register}}
\fi

% theindex-Environment neu definieren ohne reledmac
\makeatletter
\renewenvironment{theindex}{%
  \ifkorrekturansicht
    \section*{\indexname}%
  \else
    \subsubsection*{Index der erwähnten Entitäten}%
  \fi
  \setlength{\parindent}{0pt}%
  \setlength{\parskip}{0pt plus 0.3pt}%
  \let\item\@idxitem
}{%
  \ifkorrekturansicht\clearpage\fi
}
\makeatother

\IfFileExists{\jobname-pw.ind}{\input{\jobname-pw.ind}}{}

% Quellenangabe nur in der Leseansicht
\ifkorrekturansicht\else
% Fallback-Definitionen, falls die .tex-Datei \titel etc. nicht gesetzt hat
\providecommand{\titel}{}
\providecommand{\editorInnen}{}
\providecommand{\dateiname}{\jobname}

\vspace{3cm}

\vfill

\footnotesize
\textsc{Quelle}: \titel. Herausgegeben von {\editorInnen}. In: \emph{Arthur Schnitzler: Briefwechsel mit Autorinnen und Autoren}.
 Digitale Edition, https://schnitzler-briefe.acdh.oeaw.ac.at/{\dateiname}.html (Stand \today)
\fi

\end{document}


      