%% latex-korrekturansicht-vorspann.tex
%% Vorspann für die Korrekturansicht.
%% Lädt die gemeinsame Datei latex-vorspann.tex mit gesetztem Schalter.

\newif\ifkorrekturansicht
\korrekturansichttrue

\input{../tex-inputs/latex-vorspann}


\section[Lou Andreas-Salomé an Arthur Schnitzler, {[}25. –26.? 11. 1895{]}]{L00517 Lou Andreas-Salomé an Arthur Schnitzler, {[}25. –26.? 11. 1895{]}}
\nopagebreak\mylabel{L00517v}
\rehead{ }\normalsize\beginnumbering\briefempfaengerindex{Schnitzler, Arthur@\textsc{Schnitzler, Arthur}!zzzAndreas-Salome, Lou@\emph{von Lou Andreas-Salomé}!1895-11-252@{{[}25. –26.? 11. 1895{]}}|(be}
\toendnotes[C]{\smallbreak\pagebreak[2]}\Standort{CUL, Schnitzler, B 3.}
\physDesc{Brief, 1 Blatt, 1 Seite, 273 Zeichen
\newline{}Handschrift: schwarze Tinte, deutsche Kurrent
\newline{}Schnitzler: 1) mit Bleistift datiert: »Nov 95«  2) mit rotem Buntstift eine Unterstreichung
\newline{}Ordnung: mit rotem Buntstift von unbekannter Hand nummeriert:
                                    »8« }\toendnotes[C]{\smallbreak}
\pstart{}{\pb}Lieber Herr \textsc{D\textsuperscript{r}},\pend\vspace{0.5em}
\pstart
           wäre es Ihnen möglich, noch \label{K_L00517-1v}\edtext{heute}{\lemma{\textnormal{\emph{heute}}}\Cendnote{\textnormal{Die Datierung basiert auf der Annahme, dass
                  die Krankheit die täglichen Treffen unterbrach, die zwischen dem 23. 11. 1895 und 27. 11. 1895
                  stattfanden.}}}\label{K_L00517-1} Abend einen Augenblick hier vorzuſprechen? Frieda\pwindex{Buelow, Frieda von 12.10.1857 – 12.03.1909@\textsc{Bülow, Frieda von} (12.10.1857 – 12.03.1909), \emph{Schriftsteller/Schriftstellerin}|pw} iſt krank geworden, heute Nacht, ſie hat Fieber und
               Halsſchmerzen, und läßt bei Ihnen anfragen, ob Sie kommen mögen.\pend
           
\pstart
           Mit herzlichem Gruß{\\[\baselineskip]}Ihre \spacefill\mbox{LouAS.}\pend
           \leftskip=0em{}
\pstart
           \noindent{}Klopfen Sie bei mir, bitte, N\textsuperscript{o} 36.\pend
           \selectlanguage{ngerman}\endnumbering\briefempfaengerindex{Schnitzler, Arthur@\textsc{Schnitzler, Arthur}!zzzAndreas-Salome, Lou@\emph{von Lou Andreas-Salomé}!1895-11-252@{{[}25. –26.? 11. 1895{]}}|)be}\mylabel{L00517h}  \normalsize

\doendnotes{C}
\bigskip
\vfill

\clearpage

\footnotesize

\lohead{\textsc{register}}

% Definiere theindex-Environment komplett neu ohne reledmac
\makeatletter
\renewenvironment{theindex}{%
  \section*{\indexname}%
  \setlength{\parindent}{0pt}%
  \setlength{\parskip}{0pt plus 0.3pt}%
  \let\item\@idxitem
}{%
  \clearpage
}
\makeatother

\IfFileExists{\jobname-pw.ind}{\input{\jobname-pw.ind}}{}

\end{document}

      