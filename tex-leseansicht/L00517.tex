%% latex-leseansicht-vorspann.tex
%% Vorspann für die Leseansicht.
%% Lädt die gemeinsame Datei latex-vorspann.tex mit nicht gesetztem Schalter.

\newif\ifkorrekturansicht
\korrekturansichtfalse

\input{../tex-inputs/latex-vorspann}


         
         \renewcommand{\erwaehntePersonen}{Personen: Frieda von Bülow}
         \renewcommand{\erwaehnteOrte}{Orte: Wien}
         \renewcommand{\erwaehnteWerke}{}
               \section[Lou Andreas-Salomé an Arthur Schnitzler, {[}25. –26.? 11. 1895{]}]{ Lou Andreas-Salomé an Arthur Schnitzler, {[}25. –26.? 11. 1895{]}}\nopagebreak\mylabel{v}\rehead{ }\begin{ledgroupsized}[t]{13cm}\normalsize\beginnumbering \toendnotes[C]{\smallbreak\pagebreak[2]} \Standort{CUL, Schnitzler, B 3.}
\physDesc{Brief, 1 Blatt, 1 Seite, 273 Zeichen
\newline{}Handschrift: schwarze Tinte, deutsche Kurrent
\newline{}Schnitzler: 1) mit Bleistift datiert: »Nov 95«  2) mit rotem Buntstift eine Unterstreichung
\newline{}Ordnung: mit rotem Buntstift von unbekannter Hand nummeriert:
                                    »8« }\toendnotes[C]{\smallbreak}\pstart{}{\pb}Lieber Herr \textsc{D\textsuperscript{r}},\pend\pstart
           wäre es Ihnen möglich, noch \label{K_L00517_1v}\edtext{heute}{\lemma{\textnormal{\emph{heute}}}\Cendnote{\textnormal{Die Datierung basiert auf der Annahme, dass
                  die Krankheit die täglichen Treffen unterbricht, die zwischen dem 23. 11. 1895 und 23. 11. 1895
                  stattfanden.}}}\label{K_L00517_1h} Abend einen Augenblick hier vorzuſprechen? Frieda\pwindex{Buelow, Frieda von 12.10.1857 – 12.03.1909@\textsc{Bülow, Frieda von} (12.10.1857 – 12.03.1909), \emph{Schriftstellerin}|pw} iſt krank geworden, heute Nacht, ſie hat Fieber und
               Halsſchmerzen, und läßt bei Ihnen anfragen, ob Sie kommen mögen.\pend
           \pstart
           Mit herzlichem Gruß{\\[\baselineskip]}Ihre \spacefill\mbox{LouAS.}\pend
           \leftskip=0em{}\pstart
           \noindent{}Klopfen Sie bei mir, bitte, N\textsuperscript{o} 36.\pend
           
         
         \endnumbering\mylabel{h}\end{ledgroupsized}  \newcommand{\dateiname}{L00517}\newcommand{\titel}{Lou Andreas-Salomé an Arthur Schnitzler, [25. –26.? 11. 1895]}\newcommand{\editorInnen}{Martin Anton Müller und Gerd-Hermann Susen}%% latex-leseansicht-abspann.tex
%% Abspann für die Leseansicht.
%% Der Schalter \ifkorrekturansicht ist bereits durch den Vorspann gesetzt.

%% latex-abspann.tex
%% Gemeinsamer Abspann für Korrekturansicht und Leseansicht.
%% Setzt den Schalter \ifkorrekturansicht voraus (gesetzt in den
%% einbindenden Dateien latex-korrekturansicht-abspann.tex bzw.
%% latex-leseansicht-abspann.tex).
%% ---------------------------------------------------------------

\normalsize

% Das esempio-Environment wird nur in der Leseansicht benötigt
\ifkorrekturansicht\else
\newenvironment{esempio}[3]%
{
    \vspace{1.5ex}
    \rlap{\underline{#1}}
    \par
    \setlength{\parindent}{0cm}
    \nopagebreak
    \leftskip=#2cm
    \rightskip=#3cm
}
{
    \par
}
\fi

\doendnotes{C}
\bigskip
\vfill

\clearpage

\footnotesize

\ifkorrekturansicht
  \lohead{\textsc{register}}
\fi

% theindex-Environment neu definieren ohne reledmac
\makeatletter
\renewenvironment{theindex}{%
  \ifkorrekturansicht
    \section*{\indexname}%
  \else
    \subsubsection*{Index der erwähnten Entitäten}%
  \fi
  \setlength{\parindent}{0pt}%
  \setlength{\parskip}{0pt plus 0.3pt}%
  \let\item\@idxitem
}{%
  \ifkorrekturansicht\clearpage\fi
}
\makeatother

\IfFileExists{\jobname-pw.ind}{\input{\jobname-pw.ind}}{}

% Quellenangabe nur in der Leseansicht
\ifkorrekturansicht\else
% Fallback-Definitionen, falls die .tex-Datei \titel etc. nicht gesetzt hat
\providecommand{\titel}{}
\providecommand{\editorInnen}{}
\providecommand{\dateiname}{\jobname}

\vspace{3cm}

\vfill

\footnotesize
\textsc{Quelle}: \titel. Herausgegeben von {\editorInnen}. In: \emph{Arthur Schnitzler: Briefwechsel mit Autorinnen und Autoren}.
 Digitale Edition, https://schnitzler-briefe.acdh.oeaw.ac.at/{\dateiname}.html (Stand \today)
\fi

\end{document}


      