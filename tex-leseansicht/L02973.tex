%% latex-korrekturansicht-vorspann.tex
%% Vorspann für die Korrekturansicht.
%% Lädt die gemeinsame Datei latex-vorspann.tex mit gesetztem Schalter.

\newif\ifkorrekturansicht
\korrekturansichttrue

\input{../tex-inputs/latex-vorspann}


\section[ Arthur Schnitzler an Felix Salten, {[}10. 4. 1902{]}]{L02973 Arthur Schnitzler an Felix Salten, {[}10. 4. 1902{]}}
\nopagebreak\mylabel{L02973v}
\rehead{ }\normalsize\beginnumbering\briefempfaengerindex{Salten, Felix@\textsc{Salten, Felix}!zzzSchnitzler, Arthur@\emph{von Arthur Schnitzler}!1902-04-101@{{[}10. 4. 1902{]}}|(be}
\toendnotes[C]{\smallbreak\pagebreak[2]}\Standort{Wienbibliothek im Rathaus, ZPH 1681, 2.1.516.}
\physDesc{Brief, 1 Blatt, 2 Seiten, 317 Zeichen
\newline{}Handschrift: Bleistift, deutsche Kurrent
\newline{}Ordnung: mit Bleistift von unbekannter Hand nummeriert: »10« }\toendnotes[C]{\smallbreak}
\pstart
           \raggedleft{}{\pb}Donnerſtg\pend
           \vspace{0.5em}
\pstart
           lieber, ich gehe \label{K_L02973-1v}\edtext{heut zum Erbförſter\pwindex{Erbfoerster@\emph{Der Erbförster}|pw}}{\lemma{\textnormal{\emph{heut zum Erbförſter}}}\Cendnote{\textnormal{Dadurch gelingt die Datierung mit Hilfe
                  des \emph{Tagebuchs}\pwindex{Tagebuch@\emph{Tagebuch}|pwk}, vgl. A. S.: \emph{Tagebuch}, 10. 4. 1902.}}}\label{K_L02973-1}, bin da{\geminationn} im Café (nachtmahle etwa im Riedhof\oindex{Riedhof@\textbf{Riedhof}, \emph{Lokal (K.LKL)}|pw}) wäre ſehr erfreut Sie zu ſehen; ferner: für Samſtag hab ich mir eine \label{K_L02973-2v}\edtext{Impfstunde {\pb}bei Dr.
                  \textsc{Schlichter\pwindex{Schlichter, Felix 11.04.1865 – 03.11.1924@\textsc{Schlichter, Felix} (11.04.1865 – 03.11.1924), \emph{Pädiater/Pädiaterin}|pw}}}{\lemma{\textnormal{\emph{Impfstunde … Schlichter}}}\Cendnote{\textnormal{Siehe A. S.: \emph{Tagebuch}, 12. 4. 1902.
               }}}\label{K_L02973-2}{ }4 Uhr N. M. beſtellt, und auch Ihr wahrſcheinliches Ko{\geminationm}en in Ausſicht geſtellt. Ich würd Sie um
                  ½ 4 abholen.\pend
           
\pstart
           Auf Wiederſehen {\\[\baselineskip]}Herzlich\textcolor{gray}{ſt} Ihr {\\[\baselineskip]}\spacefill\mbox{Arth}\pend
           \leftskip=0em{}\selectlanguage{ngerman}\endnumbering\briefempfaengerindex{Salten, Felix@\textsc{Salten, Felix}!zzzSchnitzler, Arthur@\emph{von Arthur Schnitzler}!1902-04-101@{{[}10. 4. 1902{]}}|)be}\mylabel{L02973h}  \normalsize

\doendnotes{C}
\bigskip
\vfill

\clearpage

\footnotesize

\lohead{\textsc{register}}

% Definiere theindex-Environment komplett neu ohne reledmac
\makeatletter
\renewenvironment{theindex}{%
  \section*{\indexname}%
  \setlength{\parindent}{0pt}%
  \setlength{\parskip}{0pt plus 0.3pt}%
  \let\item\@idxitem
}{%
  \clearpage
}
\makeatother

\IfFileExists{\jobname-pw.ind}{\input{\jobname-pw.ind}}{}

\end{document}

      