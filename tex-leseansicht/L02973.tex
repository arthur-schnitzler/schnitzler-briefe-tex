%% latex-leseansicht-vorspann.tex
%% Vorspann für die Leseansicht.
%% Lädt die gemeinsame Datei latex-vorspann.tex mit nicht gesetztem Schalter.

\newif\ifkorrekturansicht
\korrekturansichtfalse

\input{../tex-inputs/latex-vorspann}


         
         \renewcommand{\erwaehntePersonen}{Personen: Felix Salten, Felix Schlichter}
         \renewcommand{\erwaehnteOrte}{Orte: Riedhof, Wien}
         \renewcommand{\erwaehnteWerke}{Werke: Der Erbförster, Tagebuch}
               \section[ Arthur Schnitzler an Felix Salten, {[}10. 4. 1902{]}]{ Arthur Schnitzler an Felix Salten, {[}10. 4. 1902{]}}\nopagebreak\mylabel{v}\rehead{ }\begin{ledgroupsized}[t]{13cm}\normalsize\beginnumbering\briefempfaengerindex{Salten, Felix@\textsc{Salten, Felix}!zzzSchnitzler, Arthur@\emph{von Arthur Schnitzler}!1902-04-101@{{[}10. 4. 1902{]}}|(be} \toendnotes[C]{\smallbreak\pagebreak[2]} \Standort{Wienbibliothek im Rathaus, ZPH 1681, 2.1.516.}
\physDesc{Brief, 1 Blatt, 2 Seiten, 317 Zeichen
\newline{}Handschrift: Bleistift, deutsche Kurrent
\newline{}Ordnung: mit Bleistift von unbekannter Hand nummeriert: »10« }\toendnotes[C]{\smallbreak}\pstart
           \raggedleft{}{\pb}Donnerſtg\pend
           \pstart
           lieber, ich gehe \label{K_L02973-1v}\edtext{heut zum Erbförſter\pwindex{\textcolor{red}{\textsuperscript{XXXX1 indx}}!Erbfoerster@\strich\emph{Der Erbförster}|pw}}{\lemma{\textnormal{\emph{heut zum Erbförſter}}}\Cendnote{\textnormal{Dadurch gelingt die Datierung mit Hilfe
                  des \emph{Tagebuchs}\pwindex{\textcolor{red}{\textsuperscript{XXXX1 indx}}!Tagebuch1981 – 2000@\strich\emph{Tagebuch} {[}Hrsg., 1981 – 2000{]}|pwk}, vgl. A. S.: \emph{Tagebuch}, 10. 4. 1902.}}}\label{K_L02973-1h}, bin da{\geminationn} im Café (nachtmahle etwa im Riedhof\oindex{Riedhof@\textbf{Riedhof}|pw}) wäre ſehr erfreut Sie zu ſehen; ferner: für Samſtag hab ich mir eine \label{K_L02973-2v}\edtext{Impfstunde {\pb}bei Dr.
                  \textsc{Schlichter\pwindex{Schlichter, Felix 11.04.1865 – 03.11.1924@\textsc{Schlichter, Felix} (11.04.1865 – 03.11.1924), \emph{Pädiater}|pw}}}{\lemma{\textnormal{\emph{Impfstunde … Schlichter}}}\Cendnote{\textnormal{Siehe A. S.: \emph{Tagebuch}, 12. 4. 1902.
               }}}\label{K_L02973-2h}{ }4 Uhr N. M. beſtellt, und auch Ihr wahrſcheinliches Ko{\geminationm}en in Ausſicht geſtellt. Ich würd Sie um
                  ½ 4 abholen.\pend
           \pstart
           Auf Wiederſehen {\\[\baselineskip]}Herzlich\textcolor{gray}{ſt} Ihr {\\[\baselineskip]}\spacefill\mbox{Arth}\pend
           \leftskip=0em{}
         
         \endnumbering\mylabel{h}\end{ledgroupsized}  \newcommand{\dateiname}{L02973}\newcommand{\titel}{Arthur Schnitzler an Felix Salten, [10. 4. 1902]}\newcommand{\editorInnen}{Martin Anton Müller und Laura Untner}%% latex-leseansicht-abspann.tex
%% Abspann für die Leseansicht.
%% Der Schalter \ifkorrekturansicht ist bereits durch den Vorspann gesetzt.

%% latex-abspann.tex
%% Gemeinsamer Abspann für Korrekturansicht und Leseansicht.
%% Setzt den Schalter \ifkorrekturansicht voraus (gesetzt in den
%% einbindenden Dateien latex-korrekturansicht-abspann.tex bzw.
%% latex-leseansicht-abspann.tex).
%% ---------------------------------------------------------------

\normalsize

% Das esempio-Environment wird nur in der Leseansicht benötigt
\ifkorrekturansicht\else
\newenvironment{esempio}[3]%
{
    \vspace{1.5ex}
    \rlap{\underline{#1}}
    \par
    \setlength{\parindent}{0cm}
    \nopagebreak
    \leftskip=#2cm
    \rightskip=#3cm
}
{
    \par
}
\fi

\doendnotes{C}
\bigskip
\vfill

\clearpage

\footnotesize

\ifkorrekturansicht
  \lohead{\textsc{register}}
\fi

% theindex-Environment neu definieren ohne reledmac
\makeatletter
\renewenvironment{theindex}{%
  \ifkorrekturansicht
    \section*{\indexname}%
  \else
    \subsubsection*{Index der erwähnten Entitäten}%
  \fi
  \setlength{\parindent}{0pt}%
  \setlength{\parskip}{0pt plus 0.3pt}%
  \let\item\@idxitem
}{%
  \ifkorrekturansicht\clearpage\fi
}
\makeatother

\IfFileExists{\jobname-pw.ind}{\input{\jobname-pw.ind}}{}

% Quellenangabe nur in der Leseansicht
\ifkorrekturansicht\else
% Fallback-Definitionen, falls die .tex-Datei \titel etc. nicht gesetzt hat
\providecommand{\titel}{}
\providecommand{\editorInnen}{}
\providecommand{\dateiname}{\jobname}

\vspace{3cm}

\vfill

\footnotesize
\textsc{Quelle}: \titel. Herausgegeben von {\editorInnen}. In: \emph{Arthur Schnitzler: Briefwechsel mit Autorinnen und Autoren}.
 Digitale Edition, https://schnitzler-briefe.acdh.oeaw.ac.at/{\dateiname}.html (Stand \today)
\fi

\end{document}


      