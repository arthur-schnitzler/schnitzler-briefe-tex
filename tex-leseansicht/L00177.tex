%% latex-korrekturansicht-vorspann.tex
%% Vorspann für die Korrekturansicht.
%% Lädt die gemeinsame Datei latex-vorspann.tex mit gesetztem Schalter.

\newif\ifkorrekturansicht
\korrekturansichttrue

\input{../tex-inputs/latex-vorspann}


\section[Friedrich M. Fels an Arthur Schnitzler, 1{[}7{]}. 2. 1893]{L00177 Friedrich M. Fels an Arthur Schnitzler, 1{[}7{]}. 2. 1893}
\nopagebreak\mylabel{L00177v}
\rehead{ }\normalsize\beginnumbering\briefempfaengerindex{Schnitzler, Arthur@\textsc{Schnitzler, Arthur}!zzzFels, Friedrich Michael@\emph{von Friedrich Michael Fels}!1893-02-171@{1{[}7{]}. 2. 1893}|(be}
\toendnotes[C]{\smallbreak\pagebreak[2]}\Standort{DLA, A:Schnitzler, HS.NZ85.1.2956.}
\physDesc{Brief, 1 Blatt, 2 Seiten, 1251 Zeichen
\newline{}Handschrift: schwarze Tinte, lateinische Kurrent
\newline{}Schnitzler: mit Bleistift nummeriert: »9.« und unterhalb der
                                 Datumsangabe klein »17« vermerkt }\toendnotes[C]{\smallbreak}
\pstart
           \raggedleft{}{\pb}Meran-Obermais, Hotel Erzherz. Rainer\oindex{Hotel Erzherzog Rainer [Meran]@\textbf{Hotel Erzherzog Rainer [Meran]}, \emph{Hotel (K.HTL)}|pw}{\\}18. II. 1893\pend
           
\pstart{}Lieber Doktor!\pend\vspace{0.5em}
\pstart
           Zu meinem gesterigen Brief trage ich noch einiges nach, was ich dort vergeſsen
               habe.\pend
           
\pstart
           Ihre Medizin, die \uline{Schreiber}\pwindex{Schreiber, Joseph 17.03.1835 – 28.09.1908@\textsc{Schreiber, Joseph} (17.03.1835 – 28.09.1908), \emph{Mediziner/Medizinerin, Sanatoriumsleiter/Sanatoriumsleiterin, Arzt/Ärztin}|pw} für sehr gut erklärt, nehme ich weiter; später soll da{\geminationn} ein Eisenpräparat folgen.\pend
           
\pstart
           Hier im Hotel habe ich einen Beka{\geminationn}ten aus Wien\oindex{Wien@\textbf{Wien}, \emph{A.ADM2}|pw} getroffen, den Sie auch ke{\geminationn}en, den Schwager von Moriz Rosenthal\pwindex{Rosenthal, Moriz 17.12.1862 – 03.09.1946@\textsc{Rosenthal, Moriz} (17.12.1862 – 03.09.1946), \emph{Komponist/Komponistin, Pianist/Pianistin}|pw}, Dr. med. Schrager\pwindex{Schraga, Sigmund 19.1.1866 – 1915@\textsc{Schraga, Sigmund} (19.1.1866 – 1915), \emph{Mediziner/Medizinerin}|pw}.
               Er kam hierher, sich von einer Lungenentzündung zu erholen, ist schon zwei Monate
               hier und bleibt bis Ende Februar. Auſserdem verkehre ich mit dem Erzieher\pwindex{?? [Erzieher von Max von Fuerstenberg] @\textsc{?? [Erzieher von Max von Fürstenberg]}|pwv} des Erbprinzen von Fürstenberg\pwindex{Fuerstenberg, Maximilian Egon von 13.10.1863 – 11.08.1941@\textsc{Fürstenberg, Maximilian Egon von} (13.10.1863 – 11.08.1941), \emph{Politiker/Politikerin}|pw},
               einem Philologen, der kürzlich sein Examen gemacht hat und mich durch Gestalt,
               Benehmen usw sehr an meine München\oindex{Muenchen@\textbf{München}, \emph{P.PPLA}|pw}er
               Studierzeit eri{\geminationn}ert. Übrigens ist er ein wütender
               Naturalist.\pend
           
\pstart
           Am Tag, da ich hier ankam, als wir mit dem Bu{\geminationm}elzug von
                  Bozen\oindex{Bozen@\textbf{Bozen}, \emph{P.PPLA2}|pw} herüber fuhren, hatte es 28° in der
                  So{\geminationn}e; gestern ebenso. Sonst circa 24°. {\pb}Trotzdem ka{\geminationn} ich es
               absolut zu keinem Gefühl der Wärme bringen. Ich trage wollene Unterkleider, warme
               Oberkleider, Mantel, Plaid – und mir ist, we{\geminationn} ich mir
               die So{\geminationn}e direkt in den Magen scheinen laſse, als hätte
               es 14°.\pend
           
\pstart
           Sie wiſsen, daſs ich angeschwollene Füſse habe, die auch schmerzen. Ich dachte i{\geminationm}er, es sei vom vielen Gehen; aber Schreiber\pwindex{Schreiber, Joseph 17.03.1835 – 28.09.1908@\textsc{Schreiber, Joseph} (17.03.1835 – 28.09.1908), \emph{Mediziner/Medizinerin, Sanatoriumsleiter/Sanatoriumsleiterin, Arzt/Ärztin}|pw}{ }ſagt: Anämie! alles Anämie!\pend
           
\pstart
           Herzl. {\\[\baselineskip]}\spacefill\mbox{Fels}\pend
           \leftskip=0em{}\selectlanguage{ngerman}\endnumbering\briefempfaengerindex{Schnitzler, Arthur@\textsc{Schnitzler, Arthur}!zzzFels, Friedrich Michael@\emph{von Friedrich Michael Fels}!1893-02-171@{1{[}7{]}. 2. 1893}|)be}\mylabel{L00177h}  \normalsize

\doendnotes{C}
\bigskip
\vfill

\clearpage

\footnotesize

\lohead{\textsc{register}}

% Definiere theindex-Environment komplett neu ohne reledmac
\makeatletter
\renewenvironment{theindex}{%
  \section*{\indexname}%
  \setlength{\parindent}{0pt}%
  \setlength{\parskip}{0pt plus 0.3pt}%
  \let\item\@idxitem
}{%
  \clearpage
}
\makeatother

\IfFileExists{\jobname-pw.ind}{\input{\jobname-pw.ind}}{}

\end{document}

      