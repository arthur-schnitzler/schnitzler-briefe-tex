%% latex-leseansicht-vorspann.tex
%% Vorspann für die Leseansicht.
%% Lädt die gemeinsame Datei latex-vorspann.tex mit nicht gesetztem Schalter.

\newif\ifkorrekturansicht
\korrekturansichtfalse

\input{../tex-inputs/latex-vorspann}

\begin{center}
            \textcolor{red}{ENTWURF. ENTZIFFERUNG NOCH NICHT KORREKTURGELESEN}
                      \end{center}
            
               \section[Friedrich M. Fels an Arthur Schnitzler, 1{[}7{]}. 2. 1893]{ Friedrich M. Fels an Arthur Schnitzler, 1{[}7{]}. 2. 1893}\nopagebreak\mylabel{v}\rehead{ }\begin{ledgroupsized}[t]{13cm}\normalsize\beginnumbering\briefempfaengerindex{Schnitzler, Arthur@\textsc{Schnitzler, Arthur}!zzzFels, Friedrich Michael@\emph{von Friedrich Michael Fels}!1893-02-171@{1{[}7{]}. 2. 1893}|(be} \toendnotes[C]{\smallbreak\pagebreak[2]} \Standort{DLA, A:Schnitzler, HS.NZ85.1.2956.}
\physDesc{Brief, 1 Blatt, 2 Seiten
\newline{}Handschrift: schwarze Tinte, lateinische Kurrent
\newline{}Schnitzler: mit Bleistift nummeriert: »9.« und unterhalb der
            Datumsangabe klein »17« vermerkt }\toendnotes[C]{\smallbreak}\pstart
           \raggedleft{}{\pb}Meran-Obermais, Hotel Erzherz. Rainer\oindex{Erzherzog Rainer@\textbf{Erzherzog Rainer}|pw}{\\}18. II. 1893\pend
           \pstart{}Lieber Doktor!\pend\pstart
           Zu meinem gesterigen Brief trage ich noch einiges nach, was ich dort vergeſsen
                    habe.\pend
           \pstart
           Ihre Medizin, die \uline{Schreiber}\pwindex{Schreiber, Joseph 17.03.1835 – 28.09.1908@\textsc{Schreiber, Joseph} (17.03.1835 – 28.09.1908), \emph{Mediziner, Sanatoriumsleiter}|pw} für sehr gut erklärt, nehme ich weiter; später soll da{\geminationn} ein Eisenpräparat folgen.\pend
           \pstart
           Hier im Hotel habe ich einen Beka{\geminationn}ten aus Wien\oindex{Wien@\textbf{Wien}|pw} getroffen, den Sie auch ke{\geminationn}en, den Schwager von Moriz Rosenthal\pwindex{Rosenthal, Moritz 17.12.1862 – 03.09.1946@\textsc{Rosenthal, Moritz} (17.12.1862 – 03.09.1946), \emph{Pianist}|pw}, Dr. med. Schrager\pwindex{Schraga, Sigmund 19.1.1866 – 1915@\textsc{Schraga, Sigmund} (19.1.1866 – 1915), \emph{Mediziner}|pw}. Er kam hierher, sich von einer Lungenentzündung zu erholen,
                    ist schon zwei Monate hier und bleibt bis Ende Februar. Auſserdem
                    verkehre ich mit dem Erzieher\pwindex{?? [Erzieher von Max von Fuerstenberg] 17.2.1893 – 17.2.1893@\textsc{?? [Erzieher von Max von Fürstenberg]} (17.2.1893 – 17.2.1893)|pwv} des Erbprinzen von Fürstenberg\pwindex{Fuerstenberg, Maximilian Egon von 13.10.1863 – 11.08.1941@\textsc{Fürstenberg, Maximilian Egon von} (13.10.1863 – 11.08.1941), \emph{Politiker}|pwv}, einem Philologen, der kürzlich sein Examen
                    gemacht hat und mich durch Gestalt, Benehmen usw sehr an meine München\oindex{Muenchen@\textbf{München}|pw}er Studierzeit eri{\geminationn}ert. Übrigens ist er ein wütender Naturalist.\pend
           \pstart
           Am Tag, da ich hier ankam, als wir mit dem Bu{\geminationm}elzug
                    von Bozen\oindex{Bozen@\textbf{Bozen}|pw} herüber fuhren, hatte es 28° in der
                        So{\geminationn}e; gestern ebenso. Sonst circa 24°. {\pb}Trotzdem ka{\geminationn} ich es
                    absolut zu keinem Gefühl der Wärme bringen. Ich trage wollene Unterkleider,
                    warme Oberkleider, Mantel, Plaid – und mir ist, we{\geminationn}
                    ich mir die So{\geminationn}e direkt in den Magen scheinen
                    laſse, als hätte es 14°.\pend
           \pstart
           Sie wiſsen, daſs ich angeschwollene Füſse habe, die auch schmerzen. Ich dachte
                        i{\geminationm}er, es sei vom vielen Gehen; aber Schreiber\pwindex{Schreiber, Joseph 17.03.1835 – 28.09.1908@\textsc{Schreiber, Joseph} (17.03.1835 – 28.09.1908), \emph{Mediziner, Sanatoriumsleiter}|pw}{ }ſagt: Anämie! alles Anämie!\pend
           \pstart
           Herzl. {\\[\baselineskip]}\spacefill\mbox{Fels}\pend
           \leftskip=0em{}\endnumbering\briefempfaengerindex{Schnitzler, Arthur@\textsc{Schnitzler, Arthur}!zzzFels, Friedrich Michael@\emph{von Friedrich Michael Fels}!1893-02-171@{1{[}7{]}. 2. 1893}|)be}\mylabel{h}\end{ledgroupsized}  \newcommand{\dateiname}{L00177}\newcommand{\titel}{Friedrich M. Fels an Arthur Schnitzler, 1[7]. 2. 1893}\newcommand{\editorInnen}{Martin Anton Müller und Gerd-Hermann Susen}%% latex-leseansicht-abspann.tex
%% Abspann für die Leseansicht.
%% Der Schalter \ifkorrekturansicht ist bereits durch den Vorspann gesetzt.

%% latex-abspann.tex
%% Gemeinsamer Abspann für Korrekturansicht und Leseansicht.
%% Setzt den Schalter \ifkorrekturansicht voraus (gesetzt in den
%% einbindenden Dateien latex-korrekturansicht-abspann.tex bzw.
%% latex-leseansicht-abspann.tex).
%% ---------------------------------------------------------------

\normalsize

% Das esempio-Environment wird nur in der Leseansicht benötigt
\ifkorrekturansicht\else
\newenvironment{esempio}[3]%
{
    \vspace{1.5ex}
    \rlap{\underline{#1}}
    \par
    \setlength{\parindent}{0cm}
    \nopagebreak
    \leftskip=#2cm
    \rightskip=#3cm
}
{
    \par
}
\fi

\doendnotes{C}
\bigskip
\vfill

\clearpage

\footnotesize

\ifkorrekturansicht
  \lohead{\textsc{register}}
\fi

% theindex-Environment neu definieren ohne reledmac
\makeatletter
\renewenvironment{theindex}{%
  \ifkorrekturansicht
    \section*{\indexname}%
  \else
    \subsubsection*{Index der erwähnten Entitäten}%
  \fi
  \setlength{\parindent}{0pt}%
  \setlength{\parskip}{0pt plus 0.3pt}%
  \let\item\@idxitem
}{%
  \ifkorrekturansicht\clearpage\fi
}
\makeatother

\IfFileExists{\jobname-pw.ind}{\input{\jobname-pw.ind}}{}

% Quellenangabe nur in der Leseansicht
\ifkorrekturansicht\else
% Fallback-Definitionen, falls die .tex-Datei \titel etc. nicht gesetzt hat
\providecommand{\titel}{}
\providecommand{\editorInnen}{}
\providecommand{\dateiname}{\jobname}

\vspace{3cm}

\vfill

\footnotesize
\textsc{Quelle}: \titel. Herausgegeben von {\editorInnen}. In: \emph{Arthur Schnitzler: Briefwechsel mit Autorinnen und Autoren}.
 Digitale Edition, https://schnitzler-briefe.acdh.oeaw.ac.at/{\dateiname}.html (Stand \today)
\fi

\end{document}


      