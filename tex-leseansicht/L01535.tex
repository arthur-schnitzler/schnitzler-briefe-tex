%% latex-korrekturansicht-vorspann.tex
%% Vorspann für die Korrekturansicht.
%% Lädt die gemeinsame Datei latex-vorspann.tex mit gesetztem Schalter.

\newif\ifkorrekturansicht
\korrekturansichttrue

\input{../tex-inputs/latex-vorspann}


\section[Hermann Bahr an Arthur Schnitzler, 3. 8. 1905]{L01535 Hermann Bahr an Arthur Schnitzler, 3. 8. 1905}
\nopagebreak\mylabel{L01535v}
\rehead{ }\normalsize\beginnumbering\briefempfaengerindex{Schnitzler, Arthur@\textsc{Schnitzler, Arthur}!zzzBahr, Hermann@\emph{von Hermann Bahr}!1905-08-031@{3. 8. 1905}|(be}
\toendnotes[C]{\smallbreak\pagebreak[2]}\Standort{CUL, Schnitzler, B 5b.}
\physDesc{Bildpostkarte, 95 Zeichen
\newline{}Handschrift: 1) Bleistift, deutsche Kurrent\hspace{1em}2) Bleistift, lateinische Kurrent (\noindent{}Adresse)\hspace{1em}
\newline{}Versand: Stempel: »\nobreak{}\oindex{Wuerzburg@\textbf{Würzburg}, \emph{P.PPLA2}|pwk}Wuerzburg, 3 Aug 05, 8 Nm.\nobreak{}«.  
\newline{}Ordnung: mit Bleistift von unbekannter Hand nummeriert:
                                    »129« }
\buchAbdrucke{\weitereDrucke{Hermann Bahr, Arthur Schnitzler: \emph{Briefwechsel, Aufzeichnungen, Dokumente (1891–1931)}. Göttingen: \emph{Wallstein} 2018, S. 349.} }\toendnotes[C]{\smallbreak}\pstart{}{\pb}Arthur Schnitzler\pend{}\pstart{}Wien XVIII\oindex{XVIII., Waehring@\textbf{XVIII., Währing}, \emph{A.ADM3}|pw}\pend{}\pstart{}Spöttlgasse 7\oindex{Edmund-Weiss-Gasse 7@\textbf{Edmund-Weiß-Gasse 7}, \emph{Wohngebäude (K.WHS)}|pw}\pend{}{\bigskip}
\pstart
           \noindent{}\centering{}{\pb}\textcolor{gray}{\textbf{Platz’scher Garten\oindex{Platz scher Garten@\textbf{Platz’scher Garten}, \emph{Lokal (K.LKL)}|pw}}}\pend
           
\pstart
           \centering{}\textcolor{gray}{\textbf{Bes. Fr. Kneuer\pwindex{Kneuer, Franz 1856 – 1932@\textsc{Kneuer, Franz} (1856 – 1932), \emph{Gastwirt/Gastwirtin}|pw}}}\pend
           
\pstart
           \centering{}\textcolor{gray}{\textbf{Gruss aus Würzburg\oindex{Wuerzburg@\textbf{Würzburg}, \emph{P.PPLA2}|pw}}}\pend
           \vspace{1em}
\pstart
           \raggedleft{}{\pb}3. \label{T_L01535-1v}\edtext{7.}{\lemma{\textnormal{\emph{7.}}}\Cendnote{\textnormal{Schreibirrtum}}}\label{T_L01535-1} 05\pend
           \vspace{0.5em}
\pstart
           Dich u Deine liebe Frau\pwindex{Schnitzler, Olga 17.01.1882 – 13.01.1970@\textsc{Schnitzler, Olga} (17.01.1882 – 13.01.1970), \emph{Schauspieler/Schauspielerin, Sänger/Sängerin}|pwv} grüßt
               herzlichſt\pend
           \pstart \spacefill\mbox{HermannB.}\pend{}\selectlanguage{ngerman}\endnumbering\briefempfaengerindex{Schnitzler, Arthur@\textsc{Schnitzler, Arthur}!zzzBahr, Hermann@\emph{von Hermann Bahr}!1905-08-031@{3. 8. 1905}|)be}\mylabel{L01535h}  \normalsize

\doendnotes{C}
\bigskip
\vfill

\clearpage

\footnotesize

\lohead{\textsc{register}}

% Definiere theindex-Environment komplett neu ohne reledmac
\makeatletter
\renewenvironment{theindex}{%
  \section*{\indexname}%
  \setlength{\parindent}{0pt}%
  \setlength{\parskip}{0pt plus 0.3pt}%
  \let\item\@idxitem
}{%
  \clearpage
}
\makeatother

\IfFileExists{\jobname-pw.ind}{\input{\jobname-pw.ind}}{}

\end{document}

      