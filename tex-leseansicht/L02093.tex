%% latex-korrekturansicht-vorspann.tex
%% Vorspann für die Korrekturansicht.
%% Lädt die gemeinsame Datei latex-vorspann.tex mit gesetztem Schalter.

\newif\ifkorrekturansicht
\korrekturansichttrue

\input{../tex-inputs/latex-vorspann}


\section[Hugo von Hofmannsthal an Arthur Schnitzler, {[}6. 11. 1912{]}]{L02093 Hugo von Hofmannsthal an Arthur Schnitzler, {[}6. 11. 1912{]}}
\nopagebreak\mylabel{L02093v}
\rehead{ }\normalsize\beginnumbering\briefempfaengerindex{Schnitzler, Arthur@\textsc{Schnitzler, Arthur}!zzzHofmannsthal, Hugo von@\emph{von Hugo von Hofmannsthal}!1912-11-061@{6. 11. 1912}|(be}
\toendnotes[C]{\smallbreak\pagebreak[2]}\Standort{CUL, Schnitzler, B 43.}
\physDesc{Brief, 1 Blatt, 1 Seite, 571 Zeichen
\newline{}Handschrift: schwarze Tinte, deutsche Kurrent
\newline{}Schnitzler: mit Bleistift datiert: »6/11 912« 
\newline{}Ordnung: mit Bleistift von unbekannter Hand doppelt nummeriert:
                                    »341« }
\buchAbdrucke{\weitereDrucke{Hugo von Hofmannsthal, Arthur Schnitzler: \emph{Briefwechsel}. Frankfurt am Main: \emph{S. Fischer} 1964, S. 269.} }\toendnotes[C]{\smallbreak}
\pstart
           \raggedleft{}{\pb}Mittwoch.\pend
           
\pstart{}mein lieber Arthur\pend\vspace{0.5em}
\pstart
           ich bin \label{K_L02093-1v}\edtext{zurück}{\lemma{\textnormal{\emph{zurück}}}\Cendnote{\textnormal{Er kam am 2. 11. 1912 aus Neubeuern\oindex{Neubeuern@\textbf{Neubeuern}, \emph{P.PPL}|pwk} retour.}}}\label{K_L02093-1}, muſs \label{K_L02093-2v}\edtext{nächſtens}{\lemma{\textnormal{\emph{nächſtens}}}\Cendnote{\textnormal{Am 30. 11. 1912 reiste er nach Dresden\oindex{Dresden@\textbf{Dresden}, \emph{P.PPLA}|pwk} ab.}}}\label{K_L02093-2} wieder fort, und wünſche mir recht ſehr,
               Sie zu ſehen und daſs es womöglich wieder einmal ganz ohne andere Menſchen wäre.\pend
           
\pstart
           Es iſt nun wieder faſt ein halbes Jahr, daſs man ſich nicht geſehen hat.\hspace*{1.5em}Die nahen, mit der eigenen Jugend verknüpften Menſchen
               und die Natur – dieſe beiden ſind mir immer wie der Gegenſtand eines nie ganz
               geſtillten Durſtes, immer bleibt etwas zu wünſchen übrig – nach dieſem So{\geminationm}er doppelt.\pend
           
\pstart
           Paſst Ihnen u. Olga\pwindex{Schnitzler, Olga 17.01.1882 – 13.01.1970@\textsc{Schnitzler, Olga} (17.01.1882 – 13.01.1970), \emph{Schauspieler/Schauspielerin, Sänger/Sängerin}|pw} daſs wir \label{K_L02093-3v}\edtext{Freitag}{\lemma{\textnormal{\emph{Freitag}}}\Cendnote{\textnormal{Das gewünschte Treffen am 8. 11. 1912 fand nicht
                  statt, Schnitzler dürfte, weil anderweitig
                  verpflichtet, abgesagt haben. Am übernächsten Tag reiste er nach Berlin\oindex{Berlin@\textbf{Berlin}, \emph{P.PPLC}|pwk}.}}}\label{K_L02093-3}{ }abend zu Euch kämen. Es wäre mir recht lieb. Bitte um Depeſche.\pend
           \pstart Ihr \spacefill\mbox{Hugo.}\pend{}\selectlanguage{ngerman}\endnumbering\briefempfaengerindex{Schnitzler, Arthur@\textsc{Schnitzler, Arthur}!zzzHofmannsthal, Hugo von@\emph{von Hugo von Hofmannsthal}!1912-11-061@{6. 11. 1912}|)be}\mylabel{L02093h}  \normalsize

\doendnotes{C}
\bigskip
\vfill

\clearpage

\footnotesize

\lohead{\textsc{register}}

% Definiere theindex-Environment komplett neu ohne reledmac
\makeatletter
\renewenvironment{theindex}{%
  \section*{\indexname}%
  \setlength{\parindent}{0pt}%
  \setlength{\parskip}{0pt plus 0.3pt}%
  \let\item\@idxitem
}{%
  \clearpage
}
\makeatother

\IfFileExists{\jobname-pw.ind}{\input{\jobname-pw.ind}}{}

\end{document}

      