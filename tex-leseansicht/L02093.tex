%% latex-leseansicht-vorspann.tex
%% Vorspann für die Leseansicht.
%% Lädt die gemeinsame Datei latex-vorspann.tex mit nicht gesetztem Schalter.

\newif\ifkorrekturansicht
\korrekturansichtfalse

\input{../tex-inputs/latex-vorspann}


\section[Hugo von Hofmannsthal an Arthur Schnitzler, {[}6. 11. 1912{]}]{L02093 Hugo von Hofmannsthal an Arthur Schnitzler, {[}6. 11. 1912{]}}
\nopagebreak\mylabel{L02093v}
\rehead{ }\normalsize\beginnumbering\briefempfaengerindex{Schnitzler, Arthur@\textsc{Schnitzler, Arthur}!zzzHofmannsthal, Hugo von@\emph{von Hugo von Hofmannsthal}!1912-11-061@{6. 11. 1912}|(be}
\toendnotes[C]{\smallbreak\pagebreak[2]}
\correspDesc{Versand  durch Hugo von Hofmannsthal am 6. 11. 1912 \textbf{Ort fehlend} 
\newline{}Erhalt  durch Arthur Schnitzler im Zeitraum [6. 11. 1912
                  – 10. 11. 1912?] in Wien}\toendnotes[C]{\smallbreak}
\Standort{CUL, Schnitzler, B 43.}
\physDesc{Brief, 1 Blatt, 1 Seite, 571 Zeichen
\newline{}Handschrift: schwarze Tinte, deutsche Kurrent
\newline{}Schnitzler: mit Bleistift datiert: »6/11 912« 
\newline{}Ordnung: mit Bleistift von unbekannter Hand doppelt nummeriert:
                                    »341« }
\buchAbdrucke{\weitereDrucke{Hugo von Hofmannsthal, Arthur Schnitzler: \emph{Briefwechsel}. Herausgegeben von Therese Nickl und Heinrich Schnitzler. Frankfurt am Main: \emph{S. Fischer} 1964, S. 269.} }\toendnotes[C]{\smallbreak}
\pstart
           \raggedleft{}{\pb}Mittwoch.\pend
           
\pstart{}mein lieber Arthur\pend\vspace{0.5em}
\pstart
           ich bin \label{K_L02093-1v}\edtext{zurück}{\lemma{\textnormal{\emph{zurück}}}\Cendnote{\textnormal{Er kam am 2. 11. 1912 aus Neubeuern\oindex{Neubeuern@\textbf{Neubeuern}|pwk} retour.}}}\label{K_L02093-1}, muſs \label{K_L02093-2v}\edtext{nächſtens}{\lemma{\textnormal{\emph{nächstens}}}\Cendnote{\textnormal{Am 30. 11. 1912 reiste er nach Dresden\oindex{Dresden@\textbf{Dresden}|pwk} ab.}}}\label{K_L02093-2} wieder fort, und wünſche mir recht{ }ſehr,
               Sie zu{ }ſehen und daſs es womöglich wieder einmal ganz ohne andere Menſchen wäre.\pend
           
\pstart
           Es iſt nun wieder faſt ein halbes Jahr, daſs man{ }ſich nicht geſehen hat.\hspace*{1.5em}Die nahen, mit der eigenen Jugend verknüpften Menſchen
               und die Natur – dieſe beiden{ }ſind mir immer wie der Gegenſtand eines nie ganz
               geſtillten Durſtes, immer bleibt etwas zu wünſchen übrig – nach dieſem So{\geminationm}er doppelt.\pend
           
\pstart
           Paſst Ihnen u. Olga\pwindex{Schnitzler, Olga 17.\,1.\,1882 Wien – 13.\,1.\,1970 Lugano@\textsc{Schnitzler, Olga} (17.\,1.\,1882 Wien – 13.\,1.\,1970 Lugano), \emph{Schauspielerin, Sängerin}|pw} daſs wir \label{K_L02093-3v}\edtext{Freitag}{\lemma{\textnormal{\emph{Freitag}}}\Cendnote{\textnormal{Das gewünschte Treffen am 8. 11. 1912 fand nicht
                  statt, Schnitzler dürfte, weil anderweitig
                  verpflichtet, abgesagt haben. Am übernächsten Tag reiste er nach Berlin\oindex{Berlin@\textbf{Berlin}, \emph{Hauptstadt}|pwk}.}}}\label{K_L02093-3}{ }abend zu Euch kämen. Es wäre mir recht lieb. Bitte um Depeſche.\pend
           \pstart Ihr \spacefill\mbox{Hugo.}\pend{}\selectlanguage{ngerman}\endnumbering\briefempfaengerindex{Schnitzler, Arthur@\textsc{Schnitzler, Arthur}!zzzHofmannsthal, Hugo von@\emph{von Hugo von Hofmannsthal}!1912-11-061@{6. 11. 1912}|)be}\mylabel{L02093h}  \newcommand{\dateiname}{L02093}\newcommand{\titel}{Hugo von Hofmannsthal an Arthur Schnitzler, [6. 11. 1912]}\newcommand{\editorInnen}{Martin Anton Müller und Gerd-Hermann Susen}%% latex-leseansicht-abspann.tex
%% Abspann für die Leseansicht.
%% Der Schalter \ifkorrekturansicht ist bereits durch den Vorspann gesetzt.

%% latex-abspann.tex
%% Gemeinsamer Abspann für Korrekturansicht und Leseansicht.
%% Setzt den Schalter \ifkorrekturansicht voraus (gesetzt in den
%% einbindenden Dateien latex-korrekturansicht-abspann.tex bzw.
%% latex-leseansicht-abspann.tex).
%% ---------------------------------------------------------------

\normalsize

% Das esempio-Environment wird nur in der Leseansicht benötigt
\ifkorrekturansicht\else
\newenvironment{esempio}[3]%
{
    \vspace{1.5ex}
    \rlap{\underline{#1}}
    \par
    \setlength{\parindent}{0cm}
    \nopagebreak
    \leftskip=#2cm
    \rightskip=#3cm
}
{
    \par
}
\fi

\doendnotes{C}
\bigskip
\vfill

\clearpage

\footnotesize

\ifkorrekturansicht
  \lohead{\textsc{register}}
\fi

% theindex-Environment neu definieren ohne reledmac
\makeatletter
\renewenvironment{theindex}{%
  \ifkorrekturansicht
    \section*{\indexname}%
  \else
    \subsubsection*{Index der erwähnten Entitäten}%
  \fi
  \setlength{\parindent}{0pt}%
  \setlength{\parskip}{0pt plus 0.3pt}%
  \let\item\@idxitem
}{%
  \ifkorrekturansicht\clearpage\fi
}
\makeatother

\IfFileExists{\jobname-pw.ind}{\input{\jobname-pw.ind}}{}

% Quellenangabe nur in der Leseansicht
\ifkorrekturansicht\else
% Fallback-Definitionen, falls die .tex-Datei \titel etc. nicht gesetzt hat
\providecommand{\titel}{}
\providecommand{\editorInnen}{}
\providecommand{\dateiname}{\jobname}

\vspace{3cm}

\vfill

\footnotesize
\textsc{Quelle}: \titel. Herausgegeben von {\editorInnen}. In: \emph{Arthur Schnitzler: Briefwechsel mit Autorinnen und Autoren}.
 Digitale Edition, https://schnitzler-briefe.acdh.oeaw.ac.at/{\dateiname}.html (Stand \today)
\fi

\end{document}


