%% latex-leseansicht-vorspann.tex
%% Vorspann für die Leseansicht.
%% Lädt die gemeinsame Datei latex-vorspann.tex mit nicht gesetztem Schalter.

\newif\ifkorrekturansicht
\korrekturansichtfalse

\input{../tex-inputs/latex-vorspann}


         
         \renewcommand{\erwaehntePersonen}{Personen: Leopold Sonnemann}
         \renewcommand{\erwaehnteInstitutionen}{Institutionen: Frankfurter Zeitung}
         \renewcommand{\erwaehnteOrte}{Orte: Paris, Wien, rue Feydeau}
         \renewcommand{\erwaehnteWerke}{}
               \section[Paul Goldmann an Arthur Schnitzler, 16. 12. {[}1895{]}]{ Paul Goldmann an Arthur Schnitzler, 16. 12. {[}1895{]}}\nopagebreak\mylabel{v}\rehead{ }\begin{ledgroupsized}[t]{13cm}\normalsize\beginnumbering \toendnotes[C]{\smallbreak\pagebreak[2]} \Standort{DLA, A:Schnitzler, HS.NZ85.1.3165.}
\physDesc{Brief, 1 Blatt, 3 Seiten, 1192 Zeichen
\newline{}Handschrift: blaue Tinte, deutsche Kurrent
\newline{}Schnitzler: mit Bleistift das Jahr » 95« vermerkt }\toendnotes[C]{\smallbreak}\pstart
           \noindent{}{\pb}\textcolor{gray}{\textbf{\textbf{Frankfurter Zeitung\orgindex{Frankfurter Zeitung@Frankfurter Zeitung|pw}}}}\pend
           \pstart
           \textcolor{gray}{\textbf{(\begin{otherlanguage}{french}Gazette de Francfort\end{otherlanguage}\orgindex{Frankfurter Zeitung@Frankfurter Zeitung|pw}). }}\pend
           \pstart
           \textcolor{gray}{\textbf{\textbf{\begin{otherlanguage}{french}Fondateur M. L.
                              Sonnemann\pwindex{Sonnemann, Leopold 1831-10-29 – 1909-10-30@\textsc{Sonnemann, Leopold} (1831-10-29 – 1909-10-30), \emph{Journalist, Herausgeber}|pw}\end{otherlanguage}.}}}\pend
           \pstart
           \begin{otherlanguage}{french}\textcolor{gray}{\textbf{Journal politique, financier,}}\end{otherlanguage}\pend
           \pstart
           \begin{otherlanguage}{french}\textcolor{gray}{\textbf{commercial et littéraire.}}\end{otherlanguage}\pend
           \pstart
           \begin{otherlanguage}{french}\textcolor{gray}{\textbf{\textbf{Paraissant trois fois par jour.}}}\end{otherlanguage}\pend
           \pstart
           \begin{otherlanguage}{french}\textcolor{gray}{\textbf{\textbf{Bureau à Paris\oindex{Paris@\textbf{Paris}|pw}:}}}\end{otherlanguage}\pend
           \pstart
           \begin{otherlanguage}{french}\textcolor{gray}{\textbf{\textbf{24. Rue Feydeau\oindex{rue Feydeau@\textbf{rue Feydeau}|pw}.}}}\end{otherlanguage}\hfill \textsc{Paris\oindex{Paris@\textbf{Paris}|pw}}, 16. December.\pend
           \pstart\center{}Mein lieber Freund,\pend\pstart
           Die Opernglas-Definitionen Deines letzten lieben Briefes reichen nicht aus. Was
               verſtehſt Du unter »billig«? Ich habe mich umgethan\strikeout{,}
               und habe folgende Preiſe feſtgeſtellt: Ein kleines Damen-Opernglas aus buntfarbigem
               Perlmutter, innen vergoldet, koſtet von 35 \textsc{frcs} aufwärts;
               etwas kleiner iſt es auch zu 25 \textsc{frcs} zu haben. \label{K_L02759-22v}\edtext{Beifolgendes Blatt Papier}{\lemma{\textnormal{\emph{Beifolgendes Blatt Papier}}}\Cendnote{\textnormal{Beilage nicht erhalten}}}\label{K_L02759-22h} gibt die
               Größe der unteren Gläſer an; die {\pb}Tintenſtriche
               bezeichnen die Längen-Dimenſion, wenn es geſchloſſen iſt. Das ſieht ganz niedlich
               aus, aber die Gläſer ſind nicht gerade hervorragend, wie es natürlich iſt bei ſo
               kleinen Inſtrumenten. Würde das Deinem Wunſche entſprechen? Das iſt das billigſte
               Preis-Niveau; ſonſt natürlich ſind Inſtrumente von 100 \textsc{frcs}
               aufwärts zu haben. Ich habe eines für 150 mit zwölf Gläſern geſehen, das ſehr ſchön
               angibt; aber das iſt {\pb}natürlich zu theuer.\pend
           \pstart
           Laß’ mir umgehend Deine Aufträge zukommen. Nimm’ ruhig das für \textsc{35 Frcs}. Das Geld darfſt Du mir ſchicken, denn ich habe keinen \label{K_L02759-1v}\edtext{\textsc{Sou}}{\lemma{\textnormal{\emph{Sou}}}\Cendnote{\textnormal{im Sinne von: Cent}}}\label{K_L02759-1h} mehr.\pend
           \pstart
           Kann Dir heute nicht mehr ſchreiben. Mein Kopf geht auseinander. Ich erlebe unſagbar
               traurige Dinge.\pend
           \pstart
           Grüß’ Dich Gott, liebſter {\\[\baselineskip]}Freund! Dein {\\[\baselineskip]}\spacefill\mbox{Paul Goldmann\textcolor{gray}{.}}\pend
           \leftskip=0em{}\pstart
           \noindent{}\label{T_L02759-4v}\edtext{Wenn die Zeit zu kurz wird,
                  telegraphire mir!}{\lemma{\textnormal{\emph{Wenn … mir!}}}\Cendnote{\textnormal{oberhalb der letzten
                     beschriebenen Seite, verkehrt zum Text}}}\label{T_L02759-4h}\pend
           
         
         \endnumbering\mylabel{h}\end{ledgroupsized}  \newcommand{\dateiname}{L02759}\newcommand{\titel}{Paul Goldmann an Arthur Schnitzler, 16. 12. [1895]}\newcommand{\editorInnen}{Martin Anton Müller und Laura Untner}%% latex-leseansicht-abspann.tex
%% Abspann für die Leseansicht.
%% Der Schalter \ifkorrekturansicht ist bereits durch den Vorspann gesetzt.

%% latex-abspann.tex
%% Gemeinsamer Abspann für Korrekturansicht und Leseansicht.
%% Setzt den Schalter \ifkorrekturansicht voraus (gesetzt in den
%% einbindenden Dateien latex-korrekturansicht-abspann.tex bzw.
%% latex-leseansicht-abspann.tex).
%% ---------------------------------------------------------------

\normalsize

% Das esempio-Environment wird nur in der Leseansicht benötigt
\ifkorrekturansicht\else
\newenvironment{esempio}[3]%
{
    \vspace{1.5ex}
    \rlap{\underline{#1}}
    \par
    \setlength{\parindent}{0cm}
    \nopagebreak
    \leftskip=#2cm
    \rightskip=#3cm
}
{
    \par
}
\fi

\doendnotes{C}
\bigskip
\vfill

\clearpage

\footnotesize

\ifkorrekturansicht
  \lohead{\textsc{register}}
\fi

% theindex-Environment neu definieren ohne reledmac
\makeatletter
\renewenvironment{theindex}{%
  \ifkorrekturansicht
    \section*{\indexname}%
  \else
    \subsubsection*{Index der erwähnten Entitäten}%
  \fi
  \setlength{\parindent}{0pt}%
  \setlength{\parskip}{0pt plus 0.3pt}%
  \let\item\@idxitem
}{%
  \ifkorrekturansicht\clearpage\fi
}
\makeatother

\IfFileExists{\jobname-pw.ind}{\input{\jobname-pw.ind}}{}

% Quellenangabe nur in der Leseansicht
\ifkorrekturansicht\else
% Fallback-Definitionen, falls die .tex-Datei \titel etc. nicht gesetzt hat
\providecommand{\titel}{}
\providecommand{\editorInnen}{}
\providecommand{\dateiname}{\jobname}

\vspace{3cm}

\vfill

\footnotesize
\textsc{Quelle}: \titel. Herausgegeben von {\editorInnen}. In: \emph{Arthur Schnitzler: Briefwechsel mit Autorinnen und Autoren}.
 Digitale Edition, https://schnitzler-briefe.acdh.oeaw.ac.at/{\dateiname}.html (Stand \today)
\fi

\end{document}


      