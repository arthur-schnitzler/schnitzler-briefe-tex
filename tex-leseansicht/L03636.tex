%% latex-leseansicht-vorspann.tex
%% Vorspann für die Leseansicht.
%% Lädt die gemeinsame Datei latex-vorspann.tex mit nicht gesetztem Schalter.

\newif\ifkorrekturansicht
\korrekturansichtfalse

\input{../tex-inputs/latex-vorspann}


\section[Stefan Zweig an Arthur Schnitzler, 21. 2. 1911]{L03636 Stefan Zweig an Arthur Schnitzler, 21. 2. 1911}
\nopagebreak\mylabel{L03636v}
\rehead{ }\normalsize\beginnumbering\briefempfaengerindex{Schnitzler, Arthur@\textsc{Schnitzler, Arthur}!zzzZweig, Stefan@\emph{von Stefan Zweig}!1911-02-211@{21. 2. 1911}|(be}
\toendnotes[C]{\smallbreak\pagebreak[2]}
\correspDesc{Versand  durch Stefan Zweig am 21. 2. 1911 in Paris
\newline{}Erhalt  durch Arthur Schnitzler im Zeitraum [22. 2. 1911
                  – 26. 2. 1911?] in Wien}\toendnotes[C]{\smallbreak}
\Standort{CUL, Schnitzler, B 118.}
\physDesc{Bildpostkarte, 635 Zeichen
\newline{}Handschrift: schwarze Tinte, lateinische Kurrent
\newline{}Versand: Stempel: »\nobreak{}\oindex{Boulevard des Italiens@\textbf{Boulevard des Italiens}, \emph{Straße}|pwk}Paris B\textsuperscript{d} des Itali\textcolor{gray}{ens}, 21 Fevr 11, 12 H\nobreak{}«.  }
\buchAbdrucke{\weitereDrucke{Stefan Zweig: \emph{Briefwechsel mit Hermann Bahr, Sigmund Freud, Rainer Maria
                        Rilke und Arthur Schnitzler}. Herausgegeben von Jeffrey B. Berlin, Hans-Ulrich Lindken und Donald A. Prater. Frankfurt am Main: \emph{S. Fischer} 1987, S. 363.} }\toendnotes[C]{\smallbreak}\pstart{}{\pb}D\textsuperscript{r} Artur
                  Schnitzler\pend{}\pstart{}Vienne (Autriche)\oindex{Wien@\textbf{Wien}, \emph{Verwaltungsgebiet}|pw}\pend{}\pstart{}XVIII. \label{K_L03636-1v}\edtext{Sternwartestrasse 72}{\lemma{\textnormal{\emph{Sternwartestrasse 72}}}\Cendnote{\textnormal{Zweig\pwindex{Zweig, Stefan 28.\,11.\,1881 Wien – 23.\,2.\,1942 Petrópolis@\textsc{Zweig, Stefan} (28.\,11.\,1881 Wien – 23.\,2.\,1942 Petrópolis), \emph{Schriftsteller}|pwk} wechselt bei der Adressierung
                        seiner Schreiben an Schnitzler immer
                        wieder zwischen der falschen Hausnummer »72« und der
                        richtigen »71«.}}}\label{K_L03636-1}\oindex{Wien@\textbf{Wien}!XVIII., Währing@\textbf{XVIII., Währing}!Sternwartestraße 71@\textbf{Sternwartestraße 71}, \emph{Wohngebäude}|pw}\pend{}{\bigskip}
\pstart
           \noindent{}\centering{}{\pb}\textcolor{gray}{\textbf{175 \hspace*{1em}PARIS\oindex{Paris@\textbf{Paris}, \emph{Hauptstadt}|pw}. – La Place de la Bastille\oindex{Bastille@\textbf{Bastille}, \emph{Gebäude}|pw}. – LL}}\pend
           \vspace{1em}
\pstart
           \noindent{}{\pb}Verehrter Herr Doktor, ich sende Ihnen und Ihrer werten Frau Gemahlin\pwindex{Schnitzler, Olga 17.\,1.\,1882 Wien – 13.\,1.\,1970 Lugano@\textsc{Schnitzler, Olga} (17.\,1.\,1882 Wien – 13.\,1.\,1970 Lugano), \emph{Schauspielerin, Sängerin}|pwv} von hier aus die
               herzlichsten Abschiedsgrüsse vor meiner \label{K_L03636-2v}\edtext{Amerika\oindex{Amerika@\textbf{Amerika}|pw}fahrt}{\lemma{\textnormal{\emph{Amerikafahrt}}}\Cendnote{\textnormal{Vom 22. 2. 1911 bis zum 21. 4. 1911
                  unternahm Stefan Zweig\pwindex{Zweig, Stefan 28.\,11.\,1881 Wien – 23.\,2.\,1942 Petrópolis@\textsc{Zweig, Stefan} (28.\,11.\,1881 Wien – 23.\,2.\,1942 Petrópolis), \emph{Schriftsteller}|pwk} eine amerikanische\oindex{Amerika@\textbf{Amerika}|pwk} Reise. Die erste Station war New York\oindex{New York City@\textbf{New York City}|pwk}. Von dort reiste er in mehrere
                  Städte an der nordamerikanischen\oindex{Nordamerika@\textbf{Nordamerika}, \emph{Region}|pwk} Ostküste,
                  dann nach Chicago\oindex{Chicago@\textbf{Chicago}, \emph{Hauptstadt}|pwk} und Kanada\oindex{Kanada@\textbf{Kanada}|pwk}, um über Bermuda\oindex{Bermuda@\textbf{Bermuda}, \emph{Exterritoriales Gebiet}|pwk} und Kuba\oindex{Kuba@\textbf{Kuba}|pwk} bis nach Südamerika\oindex{Südamerika@\textbf{Südamerika}|pwk} zu gelangen.}}}\label{K_L03636-2}: Gestern
               sprach ich \label{K_L03636-3v}\edtext{Paul Morisse\pwindex{Morisse, Paul 11.\,3.\,1866 Rouen – 28.\,9.\,1946 Paris@\textsc{Morisse, Paul} (11.\,3.\,1866 Rouen – 28.\,9.\,1946 Paris), \emph{Übersetzer}|pw}}{\lemma{\textnormal{\emph{Paul Morisse}}}\Cendnote{\textnormal{Paul Morisse\pwindex{Morisse, Paul 11.\,3.\,1866 Rouen – 28.\,9.\,1946 Paris@\textsc{Morisse, Paul} (11.\,3.\,1866 Rouen – 28.\,9.\,1946 Paris), \emph{Übersetzer}|pwk} war Dichter, Übersetzer und
                  Redaktionsmitglied des \emph{Mercure de France}\orgindex{Mercure de France@Mercure de France|pwk}. Er
                  verfasste mehrere Übersetzungen von Werken Zweigs\pwindex{Zweig, Stefan 28.\,11.\,1881 Wien – 23.\,2.\,1942 Petrópolis@\textsc{Zweig, Stefan} (28.\,11.\,1881 Wien – 23.\,2.\,1942 Petrópolis), \emph{Schriftsteller}|pwk}. Die hier geplante Übersetzung von \emph{Das weite Land}\pwindex{Schnitzler, Arthur 15.\,5.\,1862 Wien – 21.\,10.\,1931 ebd.@\textsc{Schnitzler, Arthur} (15.\,5.\,1862 Wien – 21.\,10.\,1931 ebd.), \emph{Schriftsteller, Mediziner}!weite Land. Tragikomödie in fünf Akten@\strich\emph{Das weite Land. Tragikomödie in fünf Akten}|pwk} dürfte nie publiziert oder aufgeführt worden sein. 
                  Auf die vorliegende briefliche Einführung folgte ein Brief von Morisse\pwindex{Morisse, Paul 11.\,3.\,1866 Rouen – 28.\,9.\,1946 Paris@\textsc{Morisse, Paul} (11.\,3.\,1866 Rouen – 28.\,9.\,1946 Paris), \emph{Übersetzer}|pwk} an Schnitzler, datiert mit
                  23. 2. 1911. Er beginnt folgendermaßen: »\begin{otherlanguage}{french}Je crois que mon nom ne vous est pas tout a fait inconnu,
                        puisque M. Stefan Zweig\pwindex{Zweig, Stefan 28.\,11.\,1881 Wien – 23.\,2.\,1942 Petrópolis@\textsc{Zweig, Stefan} (28.\,11.\,1881 Wien – 23.\,2.\,1942 Petrópolis), \emph{Schriftsteller}|pw}, dont j’ai
                        traduit l’ouvrage\pwindex{Zweig, Stefan 28.\,11.\,1881 Wien – 23.\,2.\,1942 Petrópolis@\textsc{Zweig, Stefan} (28.\,11.\,1881 Wien – 23.\,2.\,1942 Petrópolis), \emph{Schriftsteller}!Émile Verhaeren@\strich\emph{Émile Verhaeren}|pwv}\pwindex{Zweig, Stefan 28.\,11.\,1881 Wien – 23.\,2.\,1942 Petrópolis@\textsc{Zweig, Stefan} (28.\,11.\,1881 Wien – 23.\,2.\,1942 Petrópolis), \emph{Schriftsteller}!Émile Verhaeren. Sa vie, son oeuvre@\strich\emph{Émile Verhaeren. Sa vie, son oeuvre}|pwv} sur le poète Émile
                              Verhaeren\pwindex{Verhaeren, Émile 21.\,5.\,1855 Sint-Amands – 27.\,11.\,1916 Rouen@\textsc{Verhaeren, Émile} (21.\,5.\,1855 Sint-Amands – 27.\,11.\,1916 Rouen), \emph{Schriftsteller, Schriftsteller, Krimiautor}|pw}, m’a dit vous avois parlé de
                        moi.\end{otherlanguage}« (»Ich glaube, mein Name ist ihnen nicht vollständig unbekannt, da
                           Stefan Zweig\pwindex{Zweig, Stefan 28.\,11.\,1881 Wien – 23.\,2.\,1942 Petrópolis@\textsc{Zweig, Stefan} (28.\,11.\,1881 Wien – 23.\,2.\,1942 Petrópolis), \emph{Schriftsteller}|pw}, dessen Werk\pwindex{Zweig, Stefan 28.\,11.\,1881 Wien – 23.\,2.\,1942 Petrópolis@\textsc{Zweig, Stefan} (28.\,11.\,1881 Wien – 23.\,2.\,1942 Petrópolis), \emph{Schriftsteller}!Émile Verhaeren@\strich\emph{Émile Verhaeren}|pwv}\pwindex{Zweig, Stefan 28.\,11.\,1881 Wien – 23.\,2.\,1942 Petrópolis@\textsc{Zweig, Stefan} (28.\,11.\,1881 Wien – 23.\,2.\,1942 Petrópolis), \emph{Schriftsteller}!Émile Verhaeren. Sa vie, son oeuvre@\strich\emph{Émile Verhaeren. Sa vie, son oeuvre}|pwv} über den Dichter Émile
                                 Verhaeren\pwindex{Verhaeren, Émile 21.\,5.\,1855 Sint-Amands – 27.\,11.\,1916 Rouen@\textsc{Verhaeren, Émile} (21.\,5.\,1855 Sint-Amands – 27.\,11.\,1916 Rouen), \emph{Schriftsteller, Schriftsteller, Krimiautor}|pw} ich übersetzt habe, mir sagte, er hätte vor Ihnen von mir gesprochen.«)
                  Schnitzler dürfte hinhaltend geantwortet haben und die 
                  Sache wurde erst im Herbst/Winter des Jahres wieder aufgenommen, siehe XXXX Auszeichnungsfehler: Dokument L03630 nicht gefunden.
                  }}}\label{K_L03636-3}, den Secretär des
                  »Mercure de France\orgindex{Mercure de France@Mercure de France|pw}«, der sehr gerne – ich
               erzählte ihm davon – das Weite Land\pwindex{Schnitzler, Arthur 15.\,5.\,1862 Wien – 21.\,10.\,1931 ebd.@\textsc{Schnitzler, Arthur} (15.\,5.\,1862 Wien – 21.\,10.\,1931 ebd.), \emph{Schriftsteller, Mediziner}!weite Land. Tragikomödie in fünf Akten@\strich\emph{Das weite Land. Tragikomödie in fünf Akten}|pw} übersetzen
               möchte und sich an Sie wenden will. Ich kann ihn \uline{aufrichtigst} empfehlen{[}:{]}{ }{\pb}er ist sehr tüchtig und hat auch die
               nötigen Verbindungen mit den Theatern. Es  ist mir leid, dass ich über den \label{K_L03636-4v}\edtext{Berliner\oindex{Berlin@\textbf{Berlin}, \emph{Hauptstadt}|pw} Erfolg}{\lemma{\textnormal{\emph{Berliner Erfolg}}}\Cendnote{\textnormal{Am 23. 2. 1911 gab Olga Schnitzler\pwindex{Schnitzler, Olga 17.\,1.\,1882 Wien – 13.\,1.\,1970 Lugano@\textsc{Schnitzler, Olga} (17.\,1.\,1882 Wien – 13.\,1.\,1970 Lugano), \emph{Schauspielerin, Sängerin}|pwk} ein Gesangskonzert\eventindex{Klindworth-Scharwenka-Saal@\textbf{Klindworth-Scharwenka-Saal}!Gesangskonzert von Olga Schnitzler, 23.2.1911@Gesangskonzert von Olga Schnitzler, 23.2.1911|pwkv} im Klindworth-Scharwenka-Saal\oindex{Klindworth-Scharwenka-Saal@\textbf{Klindworth-Scharwenka-Saal}, \emph{Veranstaltungsgebäude}|pwk}.}}}\label{K_L03636-4} Ihrer Frau Gemahlin\pwindex{Schnitzler, Olga 17.\,1.\,1882 Wien – 13.\,1.\,1970 Lugano@\textsc{Schnitzler, Olga} (17.\,1.\,1882 Wien – 13.\,1.\,1970 Lugano), \emph{Schauspielerin, Sängerin}|pwv} nichts mehr hören
               kann, hoffentlich dann bald in Wien\oindex{Wien@\textbf{Wien}, \emph{Verwaltungsgebiet}|pw}!\pend
           
\pstart
           In Treue Ihr ergebener{\\[\baselineskip]}\spacefill\mbox{Stefan Zweig}\pend
           \leftskip=0em{}\selectlanguage{ngerman}\endnumbering\briefempfaengerindex{Schnitzler, Arthur@\textsc{Schnitzler, Arthur}!zzzZweig, Stefan@\emph{von Stefan Zweig}!1911-02-211@{21. 2. 1911}|)be}\mylabel{L03636h}  \newcommand{\dateiname}{L03636}\newcommand{\titel}{Stefan Zweig an Arthur Schnitzler, 21. 2. 1911}\newcommand{\editorInnen}{Selma Jahnke und Martin Anton Müller}%% latex-leseansicht-abspann.tex
%% Abspann für die Leseansicht.
%% Der Schalter \ifkorrekturansicht ist bereits durch den Vorspann gesetzt.

%% latex-abspann.tex
%% Gemeinsamer Abspann für Korrekturansicht und Leseansicht.
%% Setzt den Schalter \ifkorrekturansicht voraus (gesetzt in den
%% einbindenden Dateien latex-korrekturansicht-abspann.tex bzw.
%% latex-leseansicht-abspann.tex).
%% ---------------------------------------------------------------

\normalsize

% Das esempio-Environment wird nur in der Leseansicht benötigt
\ifkorrekturansicht\else
\newenvironment{esempio}[3]%
{
    \vspace{1.5ex}
    \rlap{\underline{#1}}
    \par
    \setlength{\parindent}{0cm}
    \nopagebreak
    \leftskip=#2cm
    \rightskip=#3cm
}
{
    \par
}
\fi

\doendnotes{C}
\bigskip
\vfill

\clearpage

\footnotesize

\ifkorrekturansicht
  \lohead{\textsc{register}}
\fi

% theindex-Environment neu definieren ohne reledmac
\makeatletter
\renewenvironment{theindex}{%
  \ifkorrekturansicht
    \section*{\indexname}%
  \else
    \subsubsection*{Index der erwähnten Entitäten}%
  \fi
  \setlength{\parindent}{0pt}%
  \setlength{\parskip}{0pt plus 0.3pt}%
  \let\item\@idxitem
}{%
  \ifkorrekturansicht\clearpage\fi
}
\makeatother

\IfFileExists{\jobname-pw.ind}{\input{\jobname-pw.ind}}{}

% Quellenangabe nur in der Leseansicht
\ifkorrekturansicht\else
% Fallback-Definitionen, falls die .tex-Datei \titel etc. nicht gesetzt hat
\providecommand{\titel}{}
\providecommand{\editorInnen}{}
\providecommand{\dateiname}{\jobname}

\vspace{3cm}

\vfill

\footnotesize
\textsc{Quelle}: \titel. Herausgegeben von {\editorInnen}. In: \emph{Arthur Schnitzler: Briefwechsel mit Autorinnen und Autoren}.
 Digitale Edition, https://schnitzler-briefe.acdh.oeaw.ac.at/{\dateiname}.html (Stand \today)
\fi

\end{document}


