%% latex-leseansicht-vorspann.tex
%% Vorspann für die Leseansicht.
%% Lädt die gemeinsame Datei latex-vorspann.tex mit nicht gesetztem Schalter.

\newif\ifkorrekturansicht
\korrekturansichtfalse

\input{../tex-inputs/latex-vorspann}


         
         \renewcommand{\erwaehntePersonen}{Personen: Gabriel Beer-Hofmann, Harley Granville-Barker, Frieda Pollak, Joseph Schildkraut}
         \renewcommand{\erwaehnteOrte}{Orte: Central Park West, Lyceum Theatre, Mayflower Hotel, New York City, Vereinigte Staaten von Amerika (USA), Wien}
         \renewcommand{\erwaehnteWerke}{Werke: Anatol}
               \section[Arthur Schnitzler an Gabriel Beer-Hofmann, 14. 1. 1931]{ Arthur Schnitzler an Gabriel Beer-Hofmann, 14. 1. 1931}\nopagebreak\mylabel{v}\rehead{ }\begin{ledgroupsized}[t]{13cm}\normalsize\beginnumbering\briefempfaengerindex{Beer-Hofmann, Gabriel@\textsc{Beer-Hofmann, Gabriel}!zzzSchnitzler, Arthur@\emph{von Arthur Schnitzler}!1931-01-141@{14. 1. 1931}|(be} \toendnotes[C]{\smallbreak\pagebreak[2]} \Standort{DLA, A:Schnitzler, HS.NZ.1.339.}
\physDesc{Brief, Durchschlag, 1 Blatt, 1 Seite, 343 Zeichen (Durchschlag? )
\newline{}Schreibmaschine
\newline{}Handschrift Frieda Pollak: roter Buntstift, deutsche Kurrent (\noindent{}zwei Unterstreichungen, Beschriftung: »Beer-Hofmann«
                                 und »U.S.A\oindex{Vereinigte Staaten von Amerika (USA)@\textbf{Vereinigte Staaten von Amerika (USA)}|pw}«)
\newline{}Ordnung: mit schwarzer Tinte von unbekannter Hand die maschinschriftliche
                                 Unterschrift »Arthur« um
                                       »Sc\textcolor{gray}{h}« erweitert }\toendnotes[C]{\smallbreak}\pstart
           \raggedleft{}{\pb}14. 1. 1931. \pend
           \pstart
           Gabriel Beer-Hofmann Mayflower Hotel\oindex{Mayflower Hotel@\textbf{Mayflower Hotel}|pw}{ }Centralpark West New York\oindex{Central Park West@\textbf{Central Park West}|pw}\pend
           \pstart
           In froher Zuversicht, mein lieber Gabriel, dass Deine junge liebevoll sichere Führung
               im Verein mit den vortrefflichen Schauspielern, dem guten alten Anatol\pwindex{Schnitzler, Arthur 15.05.1862 – 21.10.1931@\textsc{Schnitzler, Arthur} (15.05.1862 – 21.10.1931), \emph{Schriftsteller, Mediziner}!Anatol1892-10-29@\strich\emph{Anatol} {[}1892-10-29{]}|pw} einen \label{K_L02541-1v}\edtext{neuen Erfolg}{\lemma{\textnormal{\emph{neuen Erfolg}}}\Cendnote{\textnormal{Am
                     16. 1. 1931 hatte \emph{Anatol}\pwindex{Schnitzler, Arthur 15.05.1862 – 21.10.1931@\textsc{Schnitzler, Arthur} (15.05.1862 – 21.10.1931), \emph{Schriftsteller, Mediziner}!Anatol1892-10-29@\strich\emph{Anatol} {[}1892-10-29{]}|pwk} in
                  der Bearbeitung von Harley Granville-Parker\pwindex{Granville-Barker, Harley 25.11.1877 – 31.08.1946@\textsc{Granville-Barker, Harley} (25.11.1877 – 31.08.1946), \emph{Theaterleiter, Schauspieler, Übersetzer}|pwk}
                  und mit Joseph Schildkraut\pwindex{Schildkraut, Joseph 22.03.1896 – 1964-01-21@\textsc{Schildkraut, Joseph} (22.03.1896 – 1964-01-21), \emph{Schauspieler}|pwk} am Lyceum-Theatre\oindex{Lyceum Theatre@\textbf{Lyceum Theatre}|pwk} in New York\oindex{New York City@\textbf{New York City}|pwk} Premiere.}}}\label{K_L02541-1h} bringen wird bin ich mit den
               herzlichsten Wünschen und allen freundschaftlichen Gefühlen\pend
           \pstart
           Dein{\\[\baselineskip]}\spacefill\mbox{Arthur}\pend
           \leftskip=0em{}
         
         \endnumbering\mylabel{h}\end{ledgroupsized}  \newcommand{\dateiname}{L02541}\newcommand{\titel}{Arthur Schnitzler an Gabriel Beer-Hofmann, 14. 1. 1931}\newcommand{\editorInnen}{Martin Anton Müller und Gerd-Hermann Susen}%% latex-leseansicht-abspann.tex
%% Abspann für die Leseansicht.
%% Der Schalter \ifkorrekturansicht ist bereits durch den Vorspann gesetzt.

%% latex-abspann.tex
%% Gemeinsamer Abspann für Korrekturansicht und Leseansicht.
%% Setzt den Schalter \ifkorrekturansicht voraus (gesetzt in den
%% einbindenden Dateien latex-korrekturansicht-abspann.tex bzw.
%% latex-leseansicht-abspann.tex).
%% ---------------------------------------------------------------

\normalsize

% Das esempio-Environment wird nur in der Leseansicht benötigt
\ifkorrekturansicht\else
\newenvironment{esempio}[3]%
{
    \vspace{1.5ex}
    \rlap{\underline{#1}}
    \par
    \setlength{\parindent}{0cm}
    \nopagebreak
    \leftskip=#2cm
    \rightskip=#3cm
}
{
    \par
}
\fi

\doendnotes{C}
\bigskip
\vfill

\clearpage

\footnotesize

\ifkorrekturansicht
  \lohead{\textsc{register}}
\fi

% theindex-Environment neu definieren ohne reledmac
\makeatletter
\renewenvironment{theindex}{%
  \ifkorrekturansicht
    \section*{\indexname}%
  \else
    \subsubsection*{Index der erwähnten Entitäten}%
  \fi
  \setlength{\parindent}{0pt}%
  \setlength{\parskip}{0pt plus 0.3pt}%
  \let\item\@idxitem
}{%
  \ifkorrekturansicht\clearpage\fi
}
\makeatother

\IfFileExists{\jobname-pw.ind}{\input{\jobname-pw.ind}}{}

% Quellenangabe nur in der Leseansicht
\ifkorrekturansicht\else
% Fallback-Definitionen, falls die .tex-Datei \titel etc. nicht gesetzt hat
\providecommand{\titel}{}
\providecommand{\editorInnen}{}
\providecommand{\dateiname}{\jobname}

\vspace{3cm}

\vfill

\footnotesize
\textsc{Quelle}: \titel. Herausgegeben von {\editorInnen}. In: \emph{Arthur Schnitzler: Briefwechsel mit Autorinnen und Autoren}.
 Digitale Edition, https://schnitzler-briefe.acdh.oeaw.ac.at/{\dateiname}.html (Stand \today)
\fi

\end{document}


      