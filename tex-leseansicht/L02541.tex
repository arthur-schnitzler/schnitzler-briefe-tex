%% latex-korrekturansicht-vorspann.tex
%% Vorspann für die Korrekturansicht.
%% Lädt die gemeinsame Datei latex-vorspann.tex mit gesetztem Schalter.

\newif\ifkorrekturansicht
\korrekturansichttrue

\input{../tex-inputs/latex-vorspann}


\section[Arthur Schnitzler an Gabriel Beer-Hofmann, 14. 1. 1931]{L02541 Arthur Schnitzler an Gabriel Beer-Hofmann, 14. 1. 1931}
\nopagebreak\mylabel{L02541v}
\rehead{ }\normalsize\beginnumbering\briefempfaengerindex{Beer-Hofmann, Gabriel@\textsc{Beer-Hofmann, Gabriel}!zzzSchnitzler, Arthur@\emph{von Arthur Schnitzler}!1931-01-141@{14. 1. 1931}|(be}
\toendnotes[C]{\smallbreak\pagebreak[2]}\Standort{DLA, A:Schnitzler, HS.NZ.1.339.}
\physDesc{Brief, Durchschlag1 Blatt, 1 Seite, 343 Zeichen (Durchschlag? )
\newline{}Schreibmaschine
\newline{}Handschrift Frieda Pollak: roter Buntstift, deutsche Kurrent (\noindent{}zwei Unterstreichungen, Beschriftung: »Beer-Hofmann«
                                 und »U.S.A\oindex{Vereinigte Staaten von Amerika [USA]@\textbf{Vereinigte Staaten von Amerika [USA]}, \emph{A.PCLI}|pw}«)
\newline{}Ordnung: mit schwarzer Tinte von unbekannter Hand die maschinschriftliche
                                 Unterschrift »Arthur« um
                                       »Sc\textcolor{gray}{h}« erweitert }\toendnotes[C]{\smallbreak}
\pstart
           \raggedleft{}{\pb}14. 1. 1931. \pend
           
\pstart
           Gabriel Beer-Hofmann Mayflower Hotel\oindex{Mayflower Hotel@\textbf{Mayflower Hotel}, \emph{Hotel (K.HTL)}|pw}{ }Centralpark West New York\oindex{Central Park West@\textbf{Central Park West}, \emph{Straße (K.STR)}|pw}\pend
           \vspace{0.5em}
\pstart
           In froher Zuversicht, mein lieber Gabriel, dass Deine junge liebevoll sichere Führung
               im Verein mit den vortrefflichen Schauspielern, dem guten alten Anatol\pwindex{Anatol@\emph{Anatol}|pw} einen \label{K_L02541-1v}\edtext{neuen Erfolg}{\lemma{\textnormal{\emph{neuen Erfolg}}}\Cendnote{\textnormal{Am
                     16. 1. 1931 hatte \emph{Anatol}\pwindex{Anatol@\emph{Anatol}|pwk} in
                  der Bearbeitung von Harley Granville-Parker\pwindex{Granville-Barker, Harley 25.11.1877 – 31.08.1946@\textsc{Granville-Barker, Harley} (25.11.1877 – 31.08.1946), \emph{Theaterleiter/Theaterleiterin, Schauspieler/Schauspielerin, Übersetzer/Übersetzerin}|pwk}
                  und mit Joseph Schildkraut\pwindex{Schildkraut, Joseph 22.03.1896 – 1964-01-21@\textsc{Schildkraut, Joseph} (22.03.1896 – 1964-01-21), \emph{Schauspieler/Schauspielerin}|pwk} am Lyceum-Theatre\oindex{Lyceum Theatre@\textbf{Lyceum Theatre}, \emph{Theater (K.THE)}|pwk} in New York\oindex{New York City@\textbf{New York City}, \emph{P.PPL}|pwk} Premiere.}}}\label{K_L02541-1} bringen wird bin ich mit den
               herzlichsten Wünschen und allen freundschaftlichen Gefühlen\pend
           
\pstart
           Dein{\\[\baselineskip]}\spacefill\mbox{Arthur}\pend
           \leftskip=0em{}\selectlanguage{ngerman}\endnumbering\briefempfaengerindex{Beer-Hofmann, Gabriel@\textsc{Beer-Hofmann, Gabriel}!zzzSchnitzler, Arthur@\emph{von Arthur Schnitzler}!1931-01-141@{14. 1. 1931}|)be}\mylabel{L02541h}  \normalsize

\doendnotes{C}
\bigskip
\vfill

\clearpage

\footnotesize

\lohead{\textsc{register}}

% Definiere theindex-Environment komplett neu ohne reledmac
\makeatletter
\renewenvironment{theindex}{%
  \section*{\indexname}%
  \setlength{\parindent}{0pt}%
  \setlength{\parskip}{0pt plus 0.3pt}%
  \let\item\@idxitem
}{%
  \clearpage
}
\makeatother

\IfFileExists{\jobname-pw.ind}{\input{\jobname-pw.ind}}{}

\end{document}

      