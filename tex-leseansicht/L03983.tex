%% latex-leseansicht-vorspann.tex
%% Vorspann für die Leseansicht.
%% Lädt die gemeinsame Datei latex-vorspann.tex mit nicht gesetztem Schalter.

\newif\ifkorrekturansicht
\korrekturansichtfalse

\input{../tex-inputs/latex-vorspann}


\section[Arthur Schnitzler an Berta Zuckerkandl, 3. 9. 1929]{L03983 Arthur Schnitzler an Berta Zuckerkandl, 3. 9. 1929}
\nopagebreak\mylabel{L03983v}
\rehead{ }\normalsize\beginnumbering\briefempfaengerindex{Zuckerkandl, Berta@\textsc{Zuckerkandl, Berta}!zzzSchnitzler, Arthur@\emph{von Arthur Schnitzler}!1929-09-031@{3. 9. 1929}|(be}
\toendnotes[C]{\smallbreak\pagebreak[2]}
\correspDesc{Versand  durch Arthur Schnitzler am 3. 9. 1929 in Territet
\newline{}Erhalt  durch Berta Zuckerkandl im Zeitraum [4. 9. 1929
                  – 8. 9. 1929?] in Wien}\toendnotes[C]{\smallbreak}
\Standort{Wien, Österreichische Nationalbibliothek, 405/B78/4 LIT MAG.}
\physDesc{Brief, 1 Blatt, 2 Seiten, 733 Zeichen (Briefpapier mit Trauerrand)
\newline{}Handschrift: Bleistift, lateinische Kurrent}\toendnotes[C]{\smallbreak}
\pstart
           \raggedleft{}{\pb}Territet\oindex{Territet@\textbf{Territet}|pw}, 3. 9. 29\pend
           \vspace{0.5em}
\pstart
           liebe und verehrte Freundin,{ }\label{K_L03983-1v}\edtext{Ihren Brief}{\lemma{\textnormal{\emph{Ihren Brief}}}\Cendnote{\textnormal{nicht überliefert}}}\label{K_L03983-1}, den ich ausgezeichnet finde, hab
               ich zu Mauget\pwindex{Mauget, Irénée 1881 Angoulême – 1976@\textsc{Mauget, Irénée} (1881 Angoulême – 1976), \emph{Herausgeber, Theaterdirektor, Schriftsteller}|pw} weiter geschickt. Bourdet\pwindex{Bourdet, Édouard 26.\,10.\,1887 Saint-Germain-en-Laye – 17.\,1.\,1945 Paris@\textsc{Bourdet, Édouard} (26.\,10.\,1887 Saint-Germain-en-Laye – 17.\,1.\,1945 Paris), \emph{Schriftsteller}|pw} bemüht sich wunderbar; – ich will ihm
               aus \label{K_L03983-2v}\edtext{Wien\oindex{Wien@\textbf{Wien}, \emph{Verwaltungsgebiet}|pw}}{\lemma{\textnormal{\emph{Wien}}}\Cendnote{\textnormal{Arthur Schnitzler an Édouard Bourdet\pwindex{Bourdet, Édouard 26.\,10.\,1887 Saint-Germain-en-Laye – 17.\,1.\,1945 Paris@\textsc{Bourdet, Édouard} (26.\,10.\,1887 Saint-Germain-en-Laye – 17.\,1.\,1945 Paris), \emph{Schriftsteller}|pwk}, 24. 10. 1929, \emph{Deutsches Literaturarchiv Marbach},
                  HS.1985.1.434.}}}\label{K_L03983-2} schreiben, aus Leserlichkeitsgründen – wollen Sie so
               gut sein ihm vorerst in meinem Namen zu danken? Wie froh bin ich, daſs wir nicht
               voreilig gewesen sind. Nun können wir über alles sehr bald in Wien\oindex{Wien@\textbf{Wien}, \emph{Verwaltungsgebiet}|pw} weiterreden – ich hoffe \label{K_L03983-3v}\edtext{vor 15. dort}{\lemma{\textnormal{\emph{vor 15. dort}}}\Cendnote{\textnormal{Schnitzler hatte Wien\oindex{Wien@\textbf{Wien}, \emph{Verwaltungsgebiet}|pwk} am 19. 8. 1929 verlassen, um mit Clara Katharina Pollaczek\pwindex{Pollaczek, Clara Katharina 15.\,1.\,1875 Wien – 22.\,7.\,1951 ebd.@\textsc{Pollaczek, Clara Katharina} (15.\,1.\,1875 Wien – 22.\,7.\,1951 ebd.), \emph{Schriftstellerin}|pwk} Ferien in Caux\oindex{Caux@\textbf{Caux}|pwk} und Terriet\oindex{Territet@\textbf{Territet}|pwk} zu
                  machen. Im Anschluss entschloss er sich, ab dem 12. 9. 1929 allein nach Marienbad\oindex{Marienbad@\textbf{Marienbad}|pwk} zu reisen, um seine Exfrau Olga
                     Schnitzler\pwindex{Schnitzler, Olga 17.\,1.\,1882 Wien – 13.\,1.\,1970 Lugano@\textsc{Schnitzler, Olga} (17.\,1.\,1882 Wien – 13.\,1.\,1970 Lugano), \emph{Schauspielerin, Sängerin}|pwk} in Franzensbad\oindex{Franzensbad@\textbf{Franzensbad}|pwk} zu treffen,
                  und kehrte erst am 23. 9. 1929 nach Wien\oindex{Wien@\textbf{Wien}, \emph{Verwaltungsgebiet}|pwk}
                  zurück.}}}\label{K_L03983-3} zu sein.\pend
           
\pstart
           Heute war ich mit Frau P.–\pwindex{Pollaczek, Clara Katharina 15.\,1.\,1875 Wien – 22.\,7.\,1951 ebd.@\textsc{Pollaczek, Clara Katharina} (15.\,1.\,1875 Wien – 22.\,7.\,1951 ebd.), \emph{Schriftstellerin}|pw}\label{K_L03983-4v}\edtext{im Auto}{\lemma{\textnormal{\emph{im Auto}}}\Cendnote{\textnormal{Im \emph{Tagebuch}\pwindex{Schnitzler, Arthur 15. 5. 1862 Wien – 21. 10. 1931 ebd.@\textsc{Schnitzler, Arthur} (15. 5. 1862 Wien – 21. 10. 1931 ebd.), \emph{Schriftsteller, Mediziner}!Tagebuch@\strich\emph{Tagebuch}|pwk} ist diese Ausfahrt erst unter dem Datum des 4. 9. 1929 eingetragen.}}}\label{K_L03983-4} an der See\oindex{Genfer See@\textbf{Genfer See}, \emph{See}|pwv}, – und, denken Sie, – erst
               in Evian\oindex{Évian-les-Bains@\textbf{Évian-les-Bains}|pw} fiel mir ein, daſs \label{K_L03983-5v}\edtext{der Völkerbund\orgindex{Völkerbund@Völkerbund|pw}}{\lemma{\textnormal{\emph{der Völkerbund}}}\Cendnote{\textnormal{Ab dem 2. 9. 1929 fand in
                     Genf\oindex{Genf@\textbf{Genf}|pwk} die zehnte Vollversammlung des \emph{Völkerbundes}\orgindex{Völkerbund@Völkerbund|pwk}\eventindex{Genf@\textbf{Genf}!X. Vollversammlung des Völkerbundes@X. Vollversammlung des Völkerbundes|pwk} statt.}}}\label{K_L03983-5} in meiner nächsten Nähe {\pb}tagte\eventindex{Genf@\textbf{Genf}!X. Vollversammlung des Völkerbundes@X. Vollversammlung des Völkerbundes|pwv}. Darf man so apolitisch sein – Und dabei bin ichs
               nicht einmal. –\pend
           
\pstart
           Ich grüſse Sie von ganzen Herzen, und soll Ihnen auch von Frau Clara\pwindex{Pollaczek, Clara Katharina 15.\,1.\,1875 Wien – 22.\,7.\,1951 ebd.@\textsc{Pollaczek, Clara Katharina} (15.\,1.\,1875 Wien – 22.\,7.\,1951 ebd.), \emph{Schriftstellerin}|pw} viele Grüſse bestellen.{\\[\baselineskip]} Ihr getreuer{\\[\baselineskip]}\spacefill\mbox{Arthur Schnitzler}\pend
           \leftskip=0em{}\selectlanguage{ngerman}\endnumbering\briefempfaengerindex{Zuckerkandl, Berta@\textsc{Zuckerkandl, Berta}!zzzSchnitzler, Arthur@\emph{von Arthur Schnitzler}!1929-09-031@{3. 9. 1929}|)be}\mylabel{L03983h}
\begin{anhang}
\end{anhang}\newcommand{\dateiname}{L03983}\newcommand{\titel}{Arthur Schnitzler an Berta Zuckerkandl, 3. 9. 1929}\newcommand{\editorInnen}{Herausgegeben von Jahnke, SelmaMüller, Martin Anton}%% latex-leseansicht-abspann.tex
%% Abspann für die Leseansicht.
%% Der Schalter \ifkorrekturansicht ist bereits durch den Vorspann gesetzt.

%% latex-abspann.tex
%% Gemeinsamer Abspann für Korrekturansicht und Leseansicht.
%% Setzt den Schalter \ifkorrekturansicht voraus (gesetzt in den
%% einbindenden Dateien latex-korrekturansicht-abspann.tex bzw.
%% latex-leseansicht-abspann.tex).
%% ---------------------------------------------------------------

\normalsize

% Das esempio-Environment wird nur in der Leseansicht benötigt
\ifkorrekturansicht\else
\newenvironment{esempio}[3]%
{
    \vspace{1.5ex}
    \rlap{\underline{#1}}
    \par
    \setlength{\parindent}{0cm}
    \nopagebreak
    \leftskip=#2cm
    \rightskip=#3cm
}
{
    \par
}
\fi

\doendnotes{C}
\bigskip
\vfill

\clearpage

\footnotesize

\ifkorrekturansicht
  \lohead{\textsc{register}}
\fi

% theindex-Environment neu definieren ohne reledmac
\makeatletter
\renewenvironment{theindex}{%
  \ifkorrekturansicht
    \section*{\indexname}%
  \else
    \subsubsection*{Index der erwähnten Entitäten}%
  \fi
  \setlength{\parindent}{0pt}%
  \setlength{\parskip}{0pt plus 0.3pt}%
  \let\item\@idxitem
}{%
  \ifkorrekturansicht\clearpage\fi
}
\makeatother

\IfFileExists{\jobname-pw.ind}{\input{\jobname-pw.ind}}{}

% Quellenangabe nur in der Leseansicht
\ifkorrekturansicht\else
% Fallback-Definitionen, falls die .tex-Datei \titel etc. nicht gesetzt hat
\providecommand{\titel}{}
\providecommand{\editorInnen}{}
\providecommand{\dateiname}{\jobname}

\vspace{3cm}

\vfill

\footnotesize
\textsc{Quelle}: \titel. Herausgegeben von {\editorInnen}. In: \emph{Arthur Schnitzler: Briefwechsel mit Autorinnen und Autoren}.
 Digitale Edition, https://schnitzler-briefe.acdh.oeaw.ac.at/{\dateiname}.html (Stand \today)
\fi

\end{document}


