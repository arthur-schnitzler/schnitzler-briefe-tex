%% latex-korrekturansicht-vorspann.tex
%% Vorspann für die Korrekturansicht.
%% Lädt die gemeinsame Datei latex-vorspann.tex mit gesetztem Schalter.

\newif\ifkorrekturansicht
\korrekturansichttrue

\input{../tex-inputs/latex-vorspann}


\section[ Felix Salten an Arthur Schnitzler, 15. 10. 1907]{L03513 Felix Salten an Arthur Schnitzler, 15. 10. 1907}
\nopagebreak\mylabel{L03513v}
\rehead{ }\normalsize\beginnumbering\briefempfaengerindex{Schnitzler, Arthur@\textsc{Schnitzler, Arthur}!zzzSalten, Felix@\emph{von Felix Salten}!1907-10-151@{15. 10. 1907}|(be}
\toendnotes[C]{\smallbreak\pagebreak[2]}\Standort{CUL, Schnitzler, B 89, B 1.}
\physDesc{Postkarte, 881 Zeichen
\newline{}Handschrift: schwarze Tinte, lateinische Kurrent
\newline{}Versand: Stempel: »\nobreak{}\oindex{Berlin@\textbf{Berlin}, \emph{P.PPLC}|pwk}Berlin W. 50, 15. 10. 07, 6–7N.\nobreak{}«. Stempel: »\nobreak{}\oindex{XVIII., Waehring@\textbf{XVIII., Währing}, \emph{A.ADM3}|pwk}18/\textsubscript{1} Wien 110 , 17. X. 07, VIII\nobreak{}«.  
\newline{}Ordnung: mit Bleistift von unbekannter Hand nummeriert: »236« }\toendnotes[C]{\smallbreak}\pstart{}{\pb}Herrn D\textsuperscript{r} Arthur Schnitzler\pend{}\pstart{}Wien XVIII.\oindex{XVIII., Waehring@\textbf{XVIII., Währing}, \emph{A.ADM3}|pw}\pend{}\pstart{}Spöttelgasse 7\oindex{Edmund-Weiss-Gasse 7@\textbf{Edmund-Weiß-Gasse 7}, \emph{Wohngebäude (K.WHS)}|pw}\pend{}{\bigskip}\vspace{1em}
\pstart
           \raggedleft{}{\pb}Berlin\oindex{Berlin@\textbf{Berlin}, \emph{P.PPLC}|pw}, 15. X. 07\pend
           
\pstart{}Lieber,\pend\vspace{0.5em}
\pstart
           gestern waren wir in den \label{K_L03513-1v}\edtext{Kammerspielen\oindex{Kammerspiele Berlin@\textbf{Kammerspiele Berlin}, \emph{Theater (K.THE)}|pw} bei der
                  »Liebelei\pwindex{Liebelei. Schauspiel in drei Akten@\emph{Liebelei. Schauspiel in drei Akten}|pw}}{\lemma{\textnormal{\emph{Kammerspielen … »Liebelei}}}\Cendnote{\textnormal{Seit dem 19. 9. 1907 wurde \emph{Liebelei}\pwindex{Liebelei. Schauspiel in drei Akten@\emph{Liebelei. Schauspiel in drei Akten}|pwk} in einer Neuinszenierung an den Berliner Kammerspielen\oindex{Kammerspiele Berlin@\textbf{Kammerspiele Berlin}, \emph{Theater (K.THE)}|pwk} gegeben. Vgl. Hermann Bahr an Arthur Schnitzler, 12. 2. 1907. }}}\label{K_L03513-1}«. Ich möchte
               Ihnen sagen, wie sehr mich dieses Stück\pwindex{Liebelei. Schauspiel in drei Akten@\emph{Liebelei. Schauspiel in drei Akten}|pwv} wieder ergriffen hat. Übrigens nicht mich allein, sondern alle. Otti\pwindex{Salten, Ottilie 07.03.1868 – 22.06.1942@\textsc{Salten, Ottilie} (07.03.1868 – 22.06.1942), \emph{Schauspieler/Schauspielerin}|pw}, Wollf\pwindex{Wollf, Julius Ferdinand 22.05.1871 – 01.03.1942@\textsc{Wollf, Julius Ferdinand} (22.05.1871 – 01.03.1942), \emph{Journalist/Journalistin, Herausgeber/Herausgeberin, Verleger/Verlegerin}|pw}, und das ganze Publicum. Bei mir waren da natürlich noch andere Dinge,
               die mich im Anhören tief gerührt haben. Aber daneben und drüber hinaus hab ich doch
               gesehen, wie schön dieses Werk\pwindex{Liebelei. Schauspiel in drei Akten@\emph{Liebelei. Schauspiel in drei Akten}|pwv}
               ist, und habe vor allem gespürt, dass es sicherlich bleiben wird. Es ist ein Ausdruck
               unserer Epoche darin und dabei etwas so zeitlos Wahres und im Gefühl Starkes. Die Höflich\pwindex{Hoeflich, Lucie 20.02.1883 – 08.10.1956@\textsc{Höflich, Lucie} (20.02.1883 – 08.10.1956), \emph{Schauspieler/Schauspielerin}|pw} über alle Begriffe herrlich. Pagay\pwindex{Pagay, Hans 1843-11-11 – 1915-01-21@\textsc{Pagay, Hans} (1843-11-11 – 1915-01-21), \emph{Schauspieler/Schauspielerin}|pw} einfach wundervoll. Die Anderen fast
               unmöglich. – Heute war \label{K_L03513-2v}\edtext{Generalprobe\pwindex{Vom andern Ufer. Einakter@\emph{Vom andern Ufer. Einakter}|pwv}}{\lemma{\textnormal{\emph{Generalprobe}}}\Cendnote{\textnormal{Saltens\pwindex{Salten, Felix 06.09.1869 – 08.10.1945@\textsc{Salten, Felix} (06.09.1869 – 08.10.1945), \emph{Schriftsteller/Schriftstellerin, Journalist/Journalistin, Chefredakteur/Chefredakteurin}|pwk} Einakterreihe \emph{Vom andern
                  Ufer}\pwindex{Vom andern Ufer. Einakter@\emph{Vom andern Ufer. Einakter}|pwk} wurde noch am selben Tag am \emph{Lessing-Theater}\orgindex{Lessing-Theater@Lessing-Theater|pwk} uraufgeführt.}}}\label{K_L03513-2}, und ich weiß noch
               garnichts. Bassermann\pwindex{Bassermann, Albert 07.09.1867 – 15.05.1952@\textsc{Bassermann, Albert} (07.09.1867 – 15.05.1952), \emph{Schauspieler/Schauspielerin}|pw} beinahe schlecht. Die
               Wirkung auf mich matt. Ich bin bald in Wien\oindex{Wien@\textbf{Wien}, \emph{A.ADM2}|pw}.\pend
           
\pstart
           Inzwischen viele schöne Grüße {\\[\baselineskip]}von uns\pwindex{Salten, Ottilie 07.03.1868 – 22.06.1942@\textsc{Salten, Ottilie} (07.03.1868 – 22.06.1942), \emph{Schauspieler/Schauspielerin}|pwv} zu Ihnen, herzlichst {\\[\baselineskip]}Ihr {\\[\baselineskip]}\spacefill\mbox{Salten}\pend
           \leftskip=0em{}\selectlanguage{ngerman}\endnumbering\briefempfaengerindex{Schnitzler, Arthur@\textsc{Schnitzler, Arthur}!zzzSalten, Felix@\emph{von Felix Salten}!1907-10-151@{15. 10. 1907}|)be}\mylabel{L03513h}  \normalsize

\doendnotes{C}
\bigskip
\vfill

\clearpage

\footnotesize

\lohead{\textsc{register}}

% Definiere theindex-Environment komplett neu ohne reledmac
\makeatletter
\renewenvironment{theindex}{%
  \section*{\indexname}%
  \setlength{\parindent}{0pt}%
  \setlength{\parskip}{0pt plus 0.3pt}%
  \let\item\@idxitem
}{%
  \clearpage
}
\makeatother

\IfFileExists{\jobname-pw.ind}{\input{\jobname-pw.ind}}{}

\end{document}

      