%% latex-leseansicht-vorspann.tex
%% Vorspann für die Leseansicht.
%% Lädt die gemeinsame Datei latex-vorspann.tex mit nicht gesetztem Schalter.

\newif\ifkorrekturansicht
\korrekturansichtfalse

\input{../tex-inputs/latex-vorspann}


               \section[Albert Ehrenstein an Arthur Schnitzler, 18. 7. 1910]{ Albert Ehrenstein an Arthur Schnitzler, 18. 7. 1910}\nopagebreak\mylabel{v}\rehead{ }\begin{ledgroupsized}[t]{13cm}\normalsize\beginnumbering\briefempfaengerindex{Schnitzler, Arthur@\textsc{Schnitzler, Arthur}!zzzEhrenstein, Albert@\emph{von Albert Ehrenstein}!1910-07-181@{18. 7. 1910}|(be} \toendnotes[C]{\smallbreak\pagebreak[2]} \Standort{TMW, HS Schn 2/7/1.}
\physDesc{Brief, 1 Blatt, 2 Seiten
\newline{}Handschrift: schwarze Tinte, deutsche Kurrent
\newline{}Schnitzler: mit Bleistift beschrieben: »\textsc{Ehrenst\textcolor{gray}{ein}}« }\toendnotes[C]{\smallbreak}\pstart
           {\pb}\textsc{Vradist bei Holics}\oindex{Vrádište@\textbf{Vrádište}|pw}.\hfill \textsc{18. Juli 1910}.\pend
           \pstart{}\textsc{Hochverehrter Herr Doktor,}\pend\pstart
           in der Meinung, meiner Unluſt zu jeden Studium lägen äußere Umſtände zugrunde, bin
               ich in die Slowakei\oindex{Slowakei@\textbf{Slowakei}|pw} gefahren, in eine wald- und
               reizloſe Gegend, in der auch die Menſchen nur Land ſind, bewegliche Erde, vermodernde
               Pflanzen. Aber mit dem Lernen geht es auch hier nicht beſonders, und ſo dürfte ich
                  Anfang September wieder in Wien\oindex{Wien@\textbf{Wien}|pw}{ }ſein. – Wenn das nicht gerade eine Zeit ſein
               ſollte, wo Sie durch Proben zu sehr in Anſpruch genommen ſind, möchte ich Ihnen gerne
               meine Aufwartung machen. Sehr angenehm wäre es mir aber, falls Sie, hochverehrter
               Herr Doktor, {\pb}mir
               früher, wenn einmal Ihre Möbelwanderungen – Völkerwanderungen ſind übrigens
               mindeſtens ebenſo unangenehm – zu einem Abschluſſe gekommen ſein werden, etwas über
               meine Sachen zu ſagen die Güte hätten.\pend
           \pstart
           Ich glaube nämlich nicht, daß hierbei auch bei mir ein inneres Manco vorliegt, was
                  \label{K_L01948_1v}\edtext{Gumppenberg\pwindex{Gumppenberg, Hanns von 04.12.1866 – 29.03.1928@\textsc{Gumppenberg, Hanns von} (04.12.1866 – 29.03.1928), \emph{Schriftsteller/Schriftstellerin, Kritiker/Kritikerin}|pw} andeutete}{\lemma{\textnormal{\emph{Gumppenberg andeutete}}}\Cendnote{\textnormal{eine im unmittelbaren Verkehr getätigte Aussage}}}\label{K_L01948_1h}, indem
               er dem »Grafen Cilli\pwindex{Ehrenstein, Albert 23.12.1886 – 08.04.1950@\textsc{Ehrenstein, Albert} (23.12.1886 – 08.04.1950), \emph{Schriftsteller}!Graf Cilli1910@\strich\emph{Graf Cilli} {[}1910{]}|pw}« eine kunſtloſe, rohe,
               gefliſſentlich derbe Sprache vorwarf, der »März\orgindex{Maerz@März|pw}«,
               indem er rein artiſtiſche Gebarung meinerſeits als Hindernis einer Annahme meiner
               Arbeiten \label{K_L01948_2v}\edtext{deklarierte}{\lemma{\textnormal{\emph{deklarierte}}}\Cendnote{\textnormal{die Ablehnung gleichfalls kein publiziertes
                  Urteil}}}\label{K_L01948_2h}. –\pend
           \pstart
           Hochachtungsvoll{\\[\baselineskip]}Ihr ergebenſter{\\[\baselineskip]}\spacefill\mbox{Albert Ehrenstein.}\pend
           \leftskip=0em{}\endnumbering\briefempfaengerindex{Schnitzler, Arthur@\textsc{Schnitzler, Arthur}!zzzEhrenstein, Albert@\emph{von Albert Ehrenstein}!1910-07-181@{18. 7. 1910}|)be}\mylabel{h}\end{ledgroupsized}  \newcommand{\dateiname}{L01948}\newcommand{\titel}{Albert Ehrenstein an Arthur Schnitzler, 18. 7. 1910}\newcommand{\editorInnen}{Martin Anton Müller und Gerd-Hermann Susen}%% latex-leseansicht-abspann.tex
%% Abspann für die Leseansicht.
%% Der Schalter \ifkorrekturansicht ist bereits durch den Vorspann gesetzt.

%% latex-abspann.tex
%% Gemeinsamer Abspann für Korrekturansicht und Leseansicht.
%% Setzt den Schalter \ifkorrekturansicht voraus (gesetzt in den
%% einbindenden Dateien latex-korrekturansicht-abspann.tex bzw.
%% latex-leseansicht-abspann.tex).
%% ---------------------------------------------------------------

\normalsize

% Das esempio-Environment wird nur in der Leseansicht benötigt
\ifkorrekturansicht\else
\newenvironment{esempio}[3]%
{
    \vspace{1.5ex}
    \rlap{\underline{#1}}
    \par
    \setlength{\parindent}{0cm}
    \nopagebreak
    \leftskip=#2cm
    \rightskip=#3cm
}
{
    \par
}
\fi

\doendnotes{C}
\bigskip
\vfill

\clearpage

\footnotesize

\ifkorrekturansicht
  \lohead{\textsc{register}}
\fi

% theindex-Environment neu definieren ohne reledmac
\makeatletter
\renewenvironment{theindex}{%
  \ifkorrekturansicht
    \section*{\indexname}%
  \else
    \subsubsection*{Index der erwähnten Entitäten}%
  \fi
  \setlength{\parindent}{0pt}%
  \setlength{\parskip}{0pt plus 0.3pt}%
  \let\item\@idxitem
}{%
  \ifkorrekturansicht\clearpage\fi
}
\makeatother

\IfFileExists{\jobname-pw.ind}{\input{\jobname-pw.ind}}{}

% Quellenangabe nur in der Leseansicht
\ifkorrekturansicht\else
% Fallback-Definitionen, falls die .tex-Datei \titel etc. nicht gesetzt hat
\providecommand{\titel}{}
\providecommand{\editorInnen}{}
\providecommand{\dateiname}{\jobname}

\vspace{3cm}

\vfill

\footnotesize
\textsc{Quelle}: \titel. Herausgegeben von {\editorInnen}. In: \emph{Arthur Schnitzler: Briefwechsel mit Autorinnen und Autoren}.
 Digitale Edition, https://schnitzler-briefe.acdh.oeaw.ac.at/{\dateiname}.html (Stand \today)
\fi

\end{document}


      