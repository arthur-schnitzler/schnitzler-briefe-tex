%% latex-korrekturansicht-vorspann.tex
%% Vorspann für die Korrekturansicht.
%% Lädt die gemeinsame Datei latex-vorspann.tex mit gesetztem Schalter.

\newif\ifkorrekturansicht
\korrekturansichttrue

\input{../tex-inputs/latex-vorspann}


\section[Albert Ehrenstein an Arthur Schnitzler, 18. 7. 1910]{L01948 Albert Ehrenstein an Arthur Schnitzler, 18. 7. 1910}
\nopagebreak\mylabel{L01948v}
\rehead{ }\normalsize\beginnumbering\briefempfaengerindex{Schnitzler, Arthur@\textsc{Schnitzler, Arthur}!zzzEhrenstein, Albert@\emph{von Albert Ehrenstein}!1910-07-181@{18. 7. 1910}|(be}
\toendnotes[C]{\smallbreak\pagebreak[2]}\Standort{TMW, HS Schn 2/7/1.}
\physDesc{Brief, 1 Blatt, 2 Seiten, 1170 Zeichen
\newline{}Handschrift: schwarze Tinte, deutsche Kurrent
\newline{}Schnitzler: mit Bleistift beschrieben: »\textsc{Ehrenst\textcolor{gray}{ein}}« }\toendnotes[C]{\smallbreak}
\pstart
           
\pstart
           {\pb}\textsc{Vradist bei Holics}\oindex{Vrádište@\textbf{Vrádište}, \emph{P.PPL}|pw}.\pend
           
\pstart
           \raggedleft{}\textsc{18. Juli 1910}.\pend
           \pend
           
\pstart{}\textsc{Hochverehrter Herr Doktor,}\pend\vspace{0.5em}
\pstart
           in der Meinung, meiner Unluſt zu jeden Studium lägen äußere Umſtände zugrunde, bin
               ich in die Slowakei\oindex{Slowakei@\textbf{Slowakei}, \emph{A.PCLI}|pw} gefahren, in eine wald- und
               reizloſe Gegend, in der auch die Menſchen nur Land ſind, bewegliche Erde, vermodernde
               Pflanzen. Aber mit dem Lernen geht es auch hier nicht beſonders, und ſo dürfte ich
                  Anfang September wieder in Wien\oindex{Wien@\textbf{Wien}, \emph{A.ADM2}|pw}{ }ſein. – Wenn das nicht gerade eine Zeit ſein
               ſollte, wo Sie durch Proben zu sehr in Anſpruch genommen ſind, möchte ich Ihnen gerne
               meine Aufwartung machen. Sehr angenehm wäre es mir aber, falls Sie, hochverehrter
               Herr Doktor, {\pb}mir
               früher, wenn einmal Ihre Möbelwanderungen – Völkerwanderungen ſind übrigens
               mindeſtens ebenſo unangenehm – zu einem Abschluſſe gekommen ſein werden, etwas über
               meine Sachen zu ſagen die Güte hätten.\pend
           
\pstart
           Ich glaube nämlich nicht, daß hierbei auch bei mir ein inneres Manco vorliegt, was
                  \label{K_L01948-1v}\edtext{Gumppenberg\pwindex{Gumppenberg, Hanns von 04.12.1866 – 29.03.1928@\textsc{Gumppenberg, Hanns von} (04.12.1866 – 29.03.1928), \emph{Schriftsteller/Schriftstellerin, Kritiker/Kritikerin}|pw} andeutete}{\lemma{\textnormal{\emph{Gumppenberg andeutete}}}\Cendnote{\textnormal{eine im unmittelbaren Verkehr getätigte
                  Aussage}}}\label{K_L01948-1}, indem er dem »Grafen Cilli\pwindex{Graf Cilli@\emph{Graf Cilli}|pw}«
               eine kunſtloſe, rohe, gefliſſentlich derbe Sprache vorwarf, der »März\orgindex{Maerz@März|pw}«, indem er rein artiſtiſche Gebarung meinerſeits als
               Hindernis einer Annahme meiner Arbeiten \label{K_L01948-2v}\edtext{deklarierte}{\lemma{\textnormal{\emph{deklarierte}}}\Cendnote{\textnormal{Bei der Ablehnung
                  handelte es sich gleichfalls um kein publiziertes Urteil.}}}\label{K_L01948-2}. –\pend
           
\pstart
           Hochachtungsvoll{\\[\baselineskip]}Ihr ergebenſter{\\[\baselineskip]}\spacefill\mbox{Albert Ehrenstein.}\pend
           \leftskip=0em{}\selectlanguage{ngerman}\endnumbering\briefempfaengerindex{Schnitzler, Arthur@\textsc{Schnitzler, Arthur}!zzzEhrenstein, Albert@\emph{von Albert Ehrenstein}!1910-07-181@{18. 7. 1910}|)be}\mylabel{L01948h}  \normalsize

\doendnotes{C}
\bigskip
\vfill

\clearpage

\footnotesize

\lohead{\textsc{register}}

% Definiere theindex-Environment komplett neu ohne reledmac
\makeatletter
\renewenvironment{theindex}{%
  \section*{\indexname}%
  \setlength{\parindent}{0pt}%
  \setlength{\parskip}{0pt plus 0.3pt}%
  \let\item\@idxitem
}{%
  \clearpage
}
\makeatother

\IfFileExists{\jobname-pw.ind}{\input{\jobname-pw.ind}}{}

\end{document}

      