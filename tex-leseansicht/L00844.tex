%% latex-leseansicht-vorspann.tex
%% Vorspann für die Leseansicht.
%% Lädt die gemeinsame Datei latex-vorspann.tex mit nicht gesetztem Schalter.

\newif\ifkorrekturansicht
\korrekturansichtfalse

\input{../tex-inputs/latex-vorspann}


\section[Richard Beer-Hofmann und Hugo von Hofmannsthal an Arthur Schnitzler, 5. 9. 1898]{L00844 Richard Beer-Hofmann und Hugo von Hofmannsthal an Arthur Schnitzler, 5. 9. 1898}
\nopagebreak\mylabel{L00844v}
\rehead{ }\normalsize\beginnumbering\briefempfaengerindex{Schnitzler, Arthur@\textsc{Schnitzler, Arthur}!zzzHofmannsthal, Hugo von@\emph{von Hugo von Hofmannsthal}!1898-09-051@{5. 9. 1898}|(be}\briefempfaengerindex{Schnitzler, Arthur@\textsc{Schnitzler, Arthur}!zzzBeer-Hofmann, Richard@\emph{von Richard Beer-Hofmann}!1898-09-051@{5. 9. 1898}|(be}
\toendnotes[C]{\smallbreak\pagebreak[2]}
\correspDesc{Versand  durch Richard Beer-Hofmann, Hugo von Hofmannsthal am 5. 9. 1898 in Lugano
\newline{}Erhalt  durch Arthur Schnitzler am 7. 9. 1898 in Wien}\toendnotes[C]{\smallbreak}
\Standort{CUL, Schnitzler, B 8.}
\physDesc{Bildpostkarte, 787 Zeichen
\newline{}Handschrift Richard Beer-Hofmann: Bleistift, lateinische Kurrent
\newline{}Handschrift Hugo von Hofmannsthal: Bleistift, lateinische Kurrent
\newline{}Versand: 1) Stempel: »\nobreak{}\oindex{Lugano@\textbf{Lugano}, \emph{Hauptstadt}|pwk}Lugano, 5. IX. 98, IX\nobreak{}«.   2) Stempel: »\nobreak{}\oindex{IX., Alsergrund@\textbf{IX., Alsergrund}, \emph{Verwaltungsgebiet}|pwk}Wien 9/3 72, 7. 9. 98, 8.N, Bestellt\nobreak{}«. 
\newline{}Ordnung: mit Bleistift von unbekannter Hand nummeriert:
                                    »122« }
\buchAbdrucke{\weitereDrucke{Arthur Schnitzler, Richard Beer-Hofmann: \emph{Briefwechsel 1891–1931}. Herausgegeben von Konstanze Fliedl. Wien, Zürich: \emph{Europaverlag} 1992, S. 124–125.} }\toendnotes[C]{\smallbreak}\pstart{}{\pb}Herrn Arthur D\textsuperscript{r} Schnitzler\pend{}\pstart{}Wien\oindex{Wien@\textbf{Wien}, \emph{Verwaltungsgebiet}|pw}\pend{}\pstart{}\strikeout{Wien} im IX.\pend{}\pstart{}Frankgasse 1\oindex{Wien@\textbf{Wien}!IX., Alsergrund@\textbf{IX., Alsergrund}!Frankgasse 1@\textbf{Frankgasse 1}, \emph{Wohngebäude}|pw}\pend{}\pstart{}Autriche\oindex{Österreich@\textbf{Österreich}|pw}\pend{}\pstart{}Austria\oindex{Österreich@\textbf{Österreich}|pw}\pend{}{\bigskip}
\pstart
           \noindent{}\centering{}{\pb}\textcolor{gray}{\textbf{Villa Ceresio\oindex{Villa Ceresio@\textbf{Villa Ceresio}, \emph{Gebäude}|pw}}}\pend
           
\pstart
           \centering{}\textcolor{gray}{\textbf{Hôtel du Park\oindex{Hôtel du Parc@\textbf{Hôtel du Parc}, \emph{Hotel}|pw}}}\pend
           
\pstart
           \centering{}\textcolor{gray}{\textbf{Lugano\oindex{Lugano@\textbf{Lugano}, \emph{Hauptstadt}|pw}}}\pend
           
\pstart
           \centering{}\textcolor{gray}{\textbf{Villa Beauséjour\oindex{Villa Beauséjour@\textbf{Villa Beauséjour}, \emph{Gebäude}|pw}}}\pend
           
\pstart
           \centering{}\textcolor{gray}{\textbf{Belvédère\oindex{Belvédère@\textbf{Belvédère}, \emph{Hotel}|pw}}}\pend
           \vspace{1em}
\pstart
           \noindent{}{\pb}Lieber Arthur, ich hab mir den größeren Thurm geno{\geminationm}en. Wir fahren Mittwoch von Mailand\oindex{Mailand@\textbf{Mailand}|pw} hin um die beiden ab\introOben{}zu\introOben{}holen – Hugo\pwindex{Hofmannsthal, Hugo von 1.\,2.\,1874 Wien – 15.\,7.\,1929 Rodaun@\textsc{Hofmannsthal, Hugo von} (1.\,2.\,1874 Wien – 15.\,7.\,1929 Rodaun), \emph{Schriftsteller}|pw} hat heute in
               2 Operationen (Vor × Nachm.) den »Götterlibling\pwindex{Beer-Hofmann, Richard 11.\,7.\,1866 Wien – 26.\,9.\,1945 New York City@\textsc{Beer-Hofmann, Richard} (11.\,7.\,1866 Wien – 26.\,9.\,1945 New York City), \emph{Schriftsteller}!Tod Georgs@\strich\emph{Der Tod Georgs}|pw}«
               (jetzt heißt er »Der Tod Georgs\pwindex{Beer-Hofmann, Richard 11.\,7.\,1866 Wien – 26.\,9.\,1945 New York City@\textsc{Beer-Hofmann, Richard} (11.\,7.\,1866 Wien – 26.\,9.\,1945 New York City), \emph{Schriftsteller}!Tod Georgs@\strich\emph{Der Tod Georgs}|pw}«) erlitten.
               Vorher hat er sich die Hühneraugen {[}({]}\label{T_L00844-1v}\edtext{Der Hugo\pwindex{Hofmannsthal, Hugo von 1.\,2.\,1874 Wien – 15.\,7.\,1929 Rodaun@\textsc{Hofmannsthal, Hugo von} (1.\,2.\,1874 Wien – 15.\,7.\,1929 Rodaun), \emph{Schriftsteller}|pw} behauptet »Hühneraugen« kann man gar nicht lesen. Dazu ist doch der
                  »Secolo\pwindex{Secolo@\emph{Il Secolo}|pw}« da. R\textcolor{gray}{d} Der Hugo\pwindex{Hofmannsthal, Hugo von 1.\,2.\,1874 Wien – 15.\,7.\,1929 Rodaun@\textsc{Hofmannsthal, Hugo von} (1.\,2.\,1874 Wien – 15.\,7.\,1929 Rodaun), \emph{Schriftsteller}|pw} sagt das versteht kein Mensch. Ich mein
               zum lesen ist der Secolo\pwindex{Secolo@\emph{Il Secolo}|pw} da.}{\lemma{\textnormal{\emph{Der … da.}}}\Cendnote{\textnormal{über die Abbildung geschrieben und mit einem
                  Pfeil zum Wort »Hühneraugen« verbunden}}}\label{T_L00844-1}{[}){]} schneiden lassen. Diese Operation gelang auch. Der Götterl.\pwindex{Beer-Hofmann, Richard 11.\,7.\,1866 Wien – 26.\,9.\,1945 New York City@\textsc{Beer-Hofmann, Richard} (11.\,7.\,1866 Wien – 26.\,9.\,1945 New York City), \emph{Schriftsteller}!Tod Georgs@\strich\emph{Der Tod Georgs}|pw} ist ein »\label{K_L00844-1v}\edtext{meschugener Fisch}{\lemma{\textnormal{\emph{meschugener Fisch}}}\Cendnote{\textnormal{stehender Ausdruck in der jüdischen Kultur, sinngemäß: verrückter Kerl}}}\label{K_L00844-1}«
               darin scheint sich Hugos\pwindex{Hofmannsthal, Hugo von 1.\,2.\,1874 Wien – 15.\,7.\,1929 Rodaun@\textsc{Hofmannsthal, Hugo von} (1.\,2.\,1874 Wien – 15.\,7.\,1929 Rodaun), \emph{Schriftsteller}|pw} Urtheil zu resumiren.
                  \spacefill\mbox{R.}\pend
           \selectlanguage{ngerman}\vspace{1em}
\pstart
           \noindent{}{[}hs. Hofmannsthal:{]} \label{T_L00844-2v}\edtext{Das Schwein lasst mir keinen Platz und
               sagt mir auch keinen Stoff.}{\lemma{\textnormal{\emph{Das … Stoff.}}}\Cendnote{\textnormal{am oberen Rand
                  auf dem Kopf}}}\label{T_L00844-2}\pend
           \pstart \label{T_L00844-3v}\edtext{Herzlich Hugo kleinerer
                  Thurmbesitzer}{\lemma{\textnormal{\emph{Herzlich … Thurmbesitzer}}}\Cendnote{\textnormal{quer am linken
                  Rand}}}\label{T_L00844-3}\pend{}\selectlanguage{ngerman}\vspace{1em}
\pstart
           \noindent{}{[}hs. Beer-Hofmann:{]} Er will \label{T_L00844-4v}\edtext{i{\geminationm}er einen Stoff von mir haben weil ich ein alter
               Jud bin.}{\lemma{\textnormal{\emph{immer … bin.}}}\Cendnote{\textnormal{diagonal über den Text
                  geschrieben}}}\label{T_L00844-4}\pend
           \selectlanguage{ngerman}\endnumbering\briefempfaengerindex{Schnitzler, Arthur@\textsc{Schnitzler, Arthur}!zzzHofmannsthal, Hugo von@\emph{von Hugo von Hofmannsthal}!1898-09-051@{5. 9. 1898}|)be}\briefempfaengerindex{Schnitzler, Arthur@\textsc{Schnitzler, Arthur}!zzzBeer-Hofmann, Richard@\emph{von Richard Beer-Hofmann}!1898-09-051@{5. 9. 1898}|)be}\mylabel{L00844h}  \newcommand{\dateiname}{L00844}\newcommand{\titel}{Richard Beer-Hofmann und Hugo von Hofmannsthal an Arthur Schnitzler, 5. 9. 1898}\newcommand{\editorInnen}{Martin Anton Müller und Gerd-Hermann Susen}%% latex-leseansicht-abspann.tex
%% Abspann für die Leseansicht.
%% Der Schalter \ifkorrekturansicht ist bereits durch den Vorspann gesetzt.

%% latex-abspann.tex
%% Gemeinsamer Abspann für Korrekturansicht und Leseansicht.
%% Setzt den Schalter \ifkorrekturansicht voraus (gesetzt in den
%% einbindenden Dateien latex-korrekturansicht-abspann.tex bzw.
%% latex-leseansicht-abspann.tex).
%% ---------------------------------------------------------------

\normalsize

% Das esempio-Environment wird nur in der Leseansicht benötigt
\ifkorrekturansicht\else
\newenvironment{esempio}[3]%
{
    \vspace{1.5ex}
    \rlap{\underline{#1}}
    \par
    \setlength{\parindent}{0cm}
    \nopagebreak
    \leftskip=#2cm
    \rightskip=#3cm
}
{
    \par
}
\fi

\doendnotes{C}
\bigskip
\vfill

\clearpage

\footnotesize

\ifkorrekturansicht
  \lohead{\textsc{register}}
\fi

% theindex-Environment neu definieren ohne reledmac
\makeatletter
\renewenvironment{theindex}{%
  \ifkorrekturansicht
    \section*{\indexname}%
  \else
    \subsubsection*{Index der erwähnten Entitäten}%
  \fi
  \setlength{\parindent}{0pt}%
  \setlength{\parskip}{0pt plus 0.3pt}%
  \let\item\@idxitem
}{%
  \ifkorrekturansicht\clearpage\fi
}
\makeatother

\IfFileExists{\jobname-pw.ind}{\input{\jobname-pw.ind}}{}

% Quellenangabe nur in der Leseansicht
\ifkorrekturansicht\else
% Fallback-Definitionen, falls die .tex-Datei \titel etc. nicht gesetzt hat
\providecommand{\titel}{}
\providecommand{\editorInnen}{}
\providecommand{\dateiname}{\jobname}

\vspace{3cm}

\vfill

\footnotesize
\textsc{Quelle}: \titel. Herausgegeben von {\editorInnen}. In: \emph{Arthur Schnitzler: Briefwechsel mit Autorinnen und Autoren}.
 Digitale Edition, https://schnitzler-briefe.acdh.oeaw.ac.at/{\dateiname}.html (Stand \today)
\fi

\end{document}


