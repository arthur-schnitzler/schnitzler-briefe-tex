%% latex-korrekturansicht-vorspann.tex
%% Vorspann für die Korrekturansicht.
%% Lädt die gemeinsame Datei latex-vorspann.tex mit gesetztem Schalter.

\newif\ifkorrekturansicht
\korrekturansichttrue

\input{../tex-inputs/latex-vorspann}


\section[ Felix Salten an Arthur Schnitzler, 2{[}3{]}. 5. 1902]{L03331 Felix Salten an Arthur Schnitzler, 2{[}3{]}. 5. 1902}
\nopagebreak\mylabel{L03331v}
\rehead{ }\normalsize\beginnumbering\briefempfaengerindex{Schnitzler, Arthur@\textsc{Schnitzler, Arthur}!zzzSalten, Felix@\emph{von Felix Salten}!1902-05-231@{2{[}3{]}. 5. 1902}|(be}
\toendnotes[C]{\smallbreak\pagebreak[2]}\Standort{CUL, Schnitzler, B 89, A 2.}
\physDesc{Bildpostkarte, 253 Zeichen
\newline{}Handschrift: schwarze Tinte, lateinische Kurrent
\newline{}Versand: 1) Stempel: »\nobreak{}\oindex{Florenz@\textbf{Florenz}, \emph{P.PPLA}|pwk}Firenze Ferrovia, 25 5 02\nobreak{}«.   2) Stempel: »\nobreak{}\oindex{IX., Alsergrund@\textbf{IX., Alsergrund}, \emph{A.ADM3}|pwk}9/3 W\textcolor{gray}{ien 72}, 27. 5. \textcolor{gray}{02}, 8. V, Beste{[}llt{]}\nobreak{}«. 
\newline{}Ordnung: mit Bleistift von unbekannter Hand nummeriert: »156« }\toendnotes[C]{\smallbreak}\pstart{}{\pb}Herrn D\textsuperscript{r} Arthur Schnitzler\pend{}\pstart{}Wien IX.\oindex{IX., Alsergrund@\textbf{IX., Alsergrund}, \emph{A.ADM3}|pw}\pend{}\pstart{}Frankgaße 1\oindex{Frankgasse 1@\textbf{Frankgasse 1}, \emph{Wohngebäude (K.WHS)}|pw}\pend{}\pstart{}Austria\oindex{Oesterreich@\textbf{Österreich}, \emph{A.PCLI}|pw}\pend{}{\bigskip}
\pstart
           {\pb}\textcolor{gray}{\textbf{Firenze\oindex{Florenz@\textbf{Florenz}, \emph{P.PPLA}|pw}}}\hfill \textcolor{gray}{\textbf{Passeggiata delle Cascine\hspace*{1.5em}Viale del Re}}\oindex{Parco delle Cascine@\textbf{Parco delle Cascine}, \emph{L.PRK}|pw}\pend
           \vspace{1em}
\pstart
           \noindent{}{\pb}Vielen Dank für den \label{K_L03331-1v}\edtext{Kerr\pwindex{Kerr, Alfred 25.12.1867 – 12.10.1948@\textsc{Kerr, Alfred} (25.12.1867 – 12.10.1948), \emph{Schriftsteller/Schriftstellerin, Kritiker/Kritikerin}|pw}-Ausschnitt\pwindex{Abschluss@\emph{Abschluß}|pwv}}{\lemma{\textnormal{\emph{Kerr-Ausschnitt}}}\Cendnote{\textnormal{Beilage nicht erhalten. Es handelte sich
                  wohl um diese Sammelrezension über die neuen Theaterstücke des vergangenen
                  Winters: Alfred Kerr\pwindex{Kerr, Alfred 25.12.1867 – 12.10.1948@\textsc{Kerr, Alfred} (25.12.1867 – 12.10.1948), \emph{Schriftsteller/Schriftstellerin, Kritiker/Kritikerin}|pwk}: \emph{Abschluß}\pwindex{Abschluss@\emph{Abschluß}|pwk}. In: \emph{Neue
                        Deutsche Rundschau}\pwindex{Neue Deutsche Rundschau@\emph{Neue Deutsche Rundschau}|pwk}, Jg. 13, H. 5, Mai 1902,
                     S. 545–553. Insofern das Wort »Ausschnitt« wörtlich zu
                  nehmen ist, könnte Schnitzler auch nur die
                  Seiten 551–553 gesandt haben, die (trotz allgemeinen Lobs für Schnitzler) die vier Einakter der \emph{Lebendigen Stunden}\pwindex{Lebendige Stunden. Vier Einakter@\emph{Lebendige Stunden. Vier Einakter}|pwk} abwertend beurteilen.}}}\label{K_L03331-1}. Natürlich
               würde ich mich der \label{K_L03331-2v}\edtext{N. fr. Pr.\orgindex{Neue Freie Presse@Neue Freie Presse|pw} gegenüber – prinzipiell – \uline{nicht} ablehnend}{\lemma{\textnormal{\emph{N. … ablehnend}}}\Cendnote{\textnormal{Aus dem
                  Engagement Saltens\pwindex{Salten, Felix 06.09.1869 – 08.10.1945@\textsc{Salten, Felix} (06.09.1869 – 08.10.1945), \emph{Schriftsteller/Schriftstellerin, Journalist/Journalistin, Chefredakteur/Chefredakteurin}|pwk} für die \emph{Neue Freie Presse}\orgindex{Neue Freie Presse@Neue Freie Presse|pwk} wurde zu dieser Zeit nichts, erst über
                  ein Jahrzehnt später realisierte sich eine Mitarbeit. Vgl. Arthur Schnitzler an Felix Salten, 27. 5. 1902.}}}\label{K_L03331-2} verhalten.
                  \label{K_L03331-3v}\edtext{Schrieb Ihnen gestern}{\lemma{\textnormal{\emph{Schrieb Ihnen gestern}}}\Cendnote{\textnormal{Felix Salten an Arthur Schnitzler, 22. 5. 1902. Das erlaubt die Datierung dieser
                  Karte auf Freitag, den 23. 5. 1902. Der Versand erfolgte erst nach
                  dem Wochenende, am 25. 5. 1902.}}}\label{K_L03331-3} wegen »Dämmerseele\pwindex{Daemmerseele@\emph{Dämmerseele}|pw}«. herzlichst {\\}\spacefill\mbox{Salten}\pend
           
\pstart
           \noindent{}\label{K_L03331-4v}\edtext{h. Gruß an P. Goldmann\pwindex{Goldmann, Paul 31.01.1865 – 25.09.1935@\textsc{Goldmann, Paul} (31.01.1865 – 25.09.1935), \emph{Schriftsteller/Schriftstellerin, Journalist/Journalistin}|pw}}{\lemma{\textnormal{\emph{h. Gruß an P. Goldmann}}}\Cendnote{\textnormal{Dieser weilte in Wien\oindex{Wien@\textbf{Wien}, \emph{A.ADM2}|pwk}, vgl. A. S.: \emph{Tagebuch}, 25. 5. 1902. }}}\label{K_L03331-4}.\pend
           \selectlanguage{ngerman}\endnumbering\briefempfaengerindex{Schnitzler, Arthur@\textsc{Schnitzler, Arthur}!zzzSalten, Felix@\emph{von Felix Salten}!1902-05-231@{2{[}3{]}. 5. 1902}|)be}\mylabel{L03331h}  \normalsize

\doendnotes{C}
\bigskip
\vfill

\clearpage

\footnotesize

\lohead{\textsc{register}}

% Definiere theindex-Environment komplett neu ohne reledmac
\makeatletter
\renewenvironment{theindex}{%
  \section*{\indexname}%
  \setlength{\parindent}{0pt}%
  \setlength{\parskip}{0pt plus 0.3pt}%
  \let\item\@idxitem
}{%
  \clearpage
}
\makeatother

\IfFileExists{\jobname-pw.ind}{\input{\jobname-pw.ind}}{}

\end{document}

      