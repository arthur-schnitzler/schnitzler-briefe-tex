%% latex-korrekturansicht-vorspann.tex
%% Vorspann für die Korrekturansicht.
%% Lädt die gemeinsame Datei latex-vorspann.tex mit gesetztem Schalter.

\newif\ifkorrekturansicht
\korrekturansichttrue

\input{../tex-inputs/latex-vorspann}


\section[Hugo von Hofmannsthal an Arthur Schnitzler, 21. 8. 1898]{L00835 Hugo von Hofmannsthal an Arthur Schnitzler, 21. 8. 1898}
\nopagebreak\mylabel{L00835v}
\rehead{ }\normalsize\beginnumbering\briefempfaengerindex{Schnitzler, Arthur@\textsc{Schnitzler, Arthur}!zzzHofmannsthal, Hugo von@\emph{von Hugo von Hofmannsthal}!1898-08-211@{21. 8. 1898}|(be}
\toendnotes[C]{\smallbreak\pagebreak[2]}\Standort{CUL, Schnitzler, B 43.}
\physDesc{Postkarte, 267 Zeichen
\newline{}Handschrift: 1) schwarze Tinte, deutsche Kurrent\hspace{1em}2) schwarze Tinte, lateinische Kurrent (\noindent{}Adresse)\hspace{1em}
\newline{}Versand: 1) Stempel: »\nobreak{}\oindex{Lugano@\textbf{Lugano}, \emph{P.PPLA2}|pwk}Lugano, 21. VIII. 98, 1\nobreak{}«.   2) Stempel: »\nobreak{}\oindex{Luzern@\textbf{Luzern}, \emph{P.PPLA}|pwk}Luzern Brf. Dist., 21. VIII. 98, 8\nobreak{}«. 
\newline{}Schnitzler: mit Bleistift datiert: »21/8 98« 
\newline{}Ordnung: 1) mit Bleistift von unbekannter Hand nummeriert »\strikeout{126}«  2) mit Bleistift von unbekannter Hand nummeriert
                                    »120«}
\buchAbdrucke{\weitereDrucke{Hugo von Hofmannsthal, Arthur Schnitzler: \emph{Briefwechsel}. Frankfurt am Main: \emph{S. Fischer} 1964, S. 110.} }\toendnotes[C]{\smallbreak}\pstart{}{\pb}Herrn D\textsuperscript{r} Arthur Schnitzler\pend{}\pstart{}Lucerne\oindex{Luzern@\textbf{Luzern}, \emph{P.PPLA}|pw}\pend{}\pstart{}poste rest.\pend{}\pstart{}Suisse\oindex{Schweiz@\textbf{Schweiz}, \emph{A.PCLI}|pw}\pend{}{\bigskip}\vspace{1em}
\pstart
           {\pb}Lugano, du Parc\oindex{Hôtel du Parc@\textbf{Hôtel du Parc}, \emph{Hotel (K.HTL)}|pw}, Sonntag{ }Früh. \pend
           \vspace{0.5em}
\pstart
           Bin über \textsc{Zermatt}\oindex{Zermatt@\textbf{Zermatt}, \emph{A.ADM3}|pw} und \textsc{Simplon}\oindex{Simplon@\textbf{Simplon}, \emph{A.ADM3}|pw} gut angeko{\geminationm}en, wohne\oindex{Hôtel du Parc@\textbf{Hôtel du Parc}, \emph{Hotel (K.HTL)}|pwv}{ }ſchön und angenehm. Hoffe ſehr auf Nachricht von
               Ihnen und bitte vielmals um Recepiſſe der Taſche, das bis jetzt nicht in meinen
               Händen.\pend
           \pstart Ihr \spacefill\mbox{Hugo.}\pend{}\selectlanguage{ngerman}\endnumbering\briefempfaengerindex{Schnitzler, Arthur@\textsc{Schnitzler, Arthur}!zzzHofmannsthal, Hugo von@\emph{von Hugo von Hofmannsthal}!1898-08-211@{21. 8. 1898}|)be}\mylabel{L00835h}  \normalsize

\doendnotes{C}
\bigskip
\vfill

\clearpage

\footnotesize

\lohead{\textsc{register}}

% Definiere theindex-Environment komplett neu ohne reledmac
\makeatletter
\renewenvironment{theindex}{%
  \section*{\indexname}%
  \setlength{\parindent}{0pt}%
  \setlength{\parskip}{0pt plus 0.3pt}%
  \let\item\@idxitem
}{%
  \clearpage
}
\makeatother

\IfFileExists{\jobname-pw.ind}{\input{\jobname-pw.ind}}{}

\end{document}

      