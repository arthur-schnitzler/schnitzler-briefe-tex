%% latex-leseansicht-vorspann.tex
%% Vorspann für die Leseansicht.
%% Lädt die gemeinsame Datei latex-vorspann.tex mit nicht gesetztem Schalter.

\newif\ifkorrekturansicht
\korrekturansichtfalse

\input{../tex-inputs/latex-vorspann}

\begin{center}
            \textcolor{red}{ENTWURF. ENTZIFFERUNG NOCH NICHT KORREKTURGELESEN}
                      \end{center}
            
               \section[Hugo von Hofmannsthal an Arthur Schnitzler, 21. 8. 1898]{ Hugo von Hofmannsthal an Arthur Schnitzler, 21. 8. 1898}\nopagebreak\mylabel{v}\rehead{ }\begin{ledgroupsized}[t]{13cm}\normalsize\beginnumbering\briefempfaengerindex{Schnitzler, Arthur@\textsc{Schnitzler, Arthur}!zzzHofmannsthal, Hugo von@\emph{von Hugo von Hofmannsthal}!1898-08-211@{21. 8. 1898}|(be} \toendnotes[C]{\smallbreak\pagebreak[2]} \Standort{CUL, Schnitzler, B 43.}
\physDesc{Postkarte
\newline{}Handschrift: schwarze Tinte, deutsche Kurrent\newline{}Versand: 1) Stempel: »\nobreak{}\oindex{Lugano@\textbf{Lugano}|pwk}Lugano, 21. VIII. 98, 1\nobreak{}«.  2) Stempel: »\nobreak{}\oindex{Luzern@\textbf{Luzern}|pwk}Luzern Brf. Dist., 21. VIII. 98, 8\nobreak{}«. 
\newline{}Schnitzler: mit Bleistift datiert: »21/8 98« \newline{}Ordnung: 1) mit Bleistift von unbekannter Hand nummeriert »\strikeout{126}« 2) mit Bleistift von unbekannter Hand nummeriert
                                    »120«}\buchAbdrucke{\weitereDrucke{Hugo von Hofmannsthal, Arthur Schnitzler: \emph{Briefwechsel}. Hg. Therese Nickl und Heinrich Schnitzler. Frankfurt am Main: \emph{S. Fischer} 1964, S. 110.} }\toendnotes[C]{\smallbreak}\pstart{}{\pb}\textsc{Herrn D\textsuperscript{r} Arthur Schnitzler}\pend{}\pstart{}\textsc{Lucerne}\oindex{Luzern@\textbf{Luzern}|pw}\pend{}\pstart{}\textsc{poste rest.}\pend{}\pstart{}\textsc{Suisse}\oindex{Schweiz@\textbf{Schweiz}|pw}\pend{}{\bigskip}\pstart
           {\pb}Lugano, du Parc\oindex{Hôtel du Parc@\textbf{Hôtel du Parc}|pw}, Sonntag{ }Früh.
               \pend
           \pstart
           Bin über \textsc{Zermatt}\oindex{Zermatt@\textbf{Zermatt}|pw} und \textsc{Simplon}\oindex{Simplon@\textbf{Simplon}|pw} gut angeko{\geminationm}en, wohne\oindex{Hôtel du Parc@\textbf{Hôtel du Parc}|pwv}{ }ſchön und angenehm. Hoffe ſehr auf Nachricht von
               Ihnen und bitte vielmals um Recepiſſe der Taſche, das bis jetzt nicht in meinen
               Händen.\pend
           \pstart Ihr \spacefill\mbox{Hugo.}\pend{}\endnumbering\briefempfaengerindex{Schnitzler, Arthur@\textsc{Schnitzler, Arthur}!zzzHofmannsthal, Hugo von@\emph{von Hugo von Hofmannsthal}!1898-08-211@{21. 8. 1898}|)be}\mylabel{h}\end{ledgroupsized}  \newcommand{\dateiname}{L00835}\newcommand{\titel}{Hugo von Hofmannsthal an Arthur Schnitzler, 21. 8. 1898}\newcommand{\editorInnen}{Martin Anton Müller und Gerd-Hermann Susen}%% latex-leseansicht-abspann.tex
%% Abspann für die Leseansicht.
%% Der Schalter \ifkorrekturansicht ist bereits durch den Vorspann gesetzt.

%% latex-abspann.tex
%% Gemeinsamer Abspann für Korrekturansicht und Leseansicht.
%% Setzt den Schalter \ifkorrekturansicht voraus (gesetzt in den
%% einbindenden Dateien latex-korrekturansicht-abspann.tex bzw.
%% latex-leseansicht-abspann.tex).
%% ---------------------------------------------------------------

\normalsize

% Das esempio-Environment wird nur in der Leseansicht benötigt
\ifkorrekturansicht\else
\newenvironment{esempio}[3]%
{
    \vspace{1.5ex}
    \rlap{\underline{#1}}
    \par
    \setlength{\parindent}{0cm}
    \nopagebreak
    \leftskip=#2cm
    \rightskip=#3cm
}
{
    \par
}
\fi

\doendnotes{C}
\bigskip
\vfill

\clearpage

\footnotesize

\ifkorrekturansicht
  \lohead{\textsc{register}}
\fi

% theindex-Environment neu definieren ohne reledmac
\makeatletter
\renewenvironment{theindex}{%
  \ifkorrekturansicht
    \section*{\indexname}%
  \else
    \subsubsection*{Index der erwähnten Entitäten}%
  \fi
  \setlength{\parindent}{0pt}%
  \setlength{\parskip}{0pt plus 0.3pt}%
  \let\item\@idxitem
}{%
  \ifkorrekturansicht\clearpage\fi
}
\makeatother

\IfFileExists{\jobname-pw.ind}{\input{\jobname-pw.ind}}{}

% Quellenangabe nur in der Leseansicht
\ifkorrekturansicht\else
% Fallback-Definitionen, falls die .tex-Datei \titel etc. nicht gesetzt hat
\providecommand{\titel}{}
\providecommand{\editorInnen}{}
\providecommand{\dateiname}{\jobname}

\vspace{3cm}

\vfill

\footnotesize
\textsc{Quelle}: \titel. Herausgegeben von {\editorInnen}. In: \emph{Arthur Schnitzler: Briefwechsel mit Autorinnen und Autoren}.
 Digitale Edition, https://schnitzler-briefe.acdh.oeaw.ac.at/{\dateiname}.html (Stand \today)
\fi

\end{document}


      