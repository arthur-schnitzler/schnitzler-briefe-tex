%% latex-leseansicht-vorspann.tex
%% Vorspann für die Leseansicht.
%% Lädt die gemeinsame Datei latex-vorspann.tex mit nicht gesetztem Schalter.

\newif\ifkorrekturansicht
\korrekturansichtfalse

\input{../tex-inputs/latex-vorspann}


\section[Theodor Herzl an Arthur Schnitzler, 23. 12. 1900]{L03877 Theodor Herzl an Arthur Schnitzler, 23. 12. 1900}
\nopagebreak\mylabel{L03877v}
\rehead{ }\normalsize\beginnumbering\briefempfaengerindex{Schnitzler, Arthur@\textsc{Schnitzler, Arthur}!zzzHerzl, Theodor@\emph{von Theodor Herzl}!1900-12-231@{23. 12. 1900}|(be}
\toendnotes[C]{\smallbreak\pagebreak[2]}
\correspDesc{Versand  durch Theodor Herzl am 23. 12. 1900 in Wien
\newline{}Erhalt  durch Arthur Schnitzler in Wien}\toendnotes[C]{\smallbreak}
\Standort{CUL, Schnitzler, B 39.}
\physDesc{Brief, 1 Blatt, 2 Seiten
\newline{}Handschrift: schwarze Tinte, lateinische Kurrent
\newline{}Ordnung: mit Bleistift von unbekannter Hand nummeriert: »56« }
\buchAbdrucke{\weitereDrucke{Theodor Herzl: \emph{Briefe Ende August 1900 – Ende Dezember 1902}. Bearbeitet von Barbara Schäfer in Zusammenarbeit mit Sofia Gelmann, Chaya Harel und Ines Rubin. Berlin, Frankfurt am Main, Wien: \emph{Propyläen} 1993, S. 124–125 (Briefe und Tagebücher. Herausgegeben von Alex Bein, Hermann Greive, Moshe Schaerf, Julius H. Schoeps und Johannes Wachten, 6).} }\toendnotes[C]{\smallbreak}
\pstart
           {\pb}\textcolor{gray}{\textbf{NEUE FREIE PRESSE\orgindex{Neue Freie Presse@Neue Freie Presse|pw}. }}\hfill Privatbrief\pend
           
\pstart
           \textcolor{gray}{\textbf{\textsc{Redaction}:}}\pend
           
\pstart
           \textcolor{gray}{\textbf{WIEN\oindex{Wien@\textbf{Wien}, \emph{Verwaltungsgebiet}|pw}}}\hfill 23 XII 900\pend
           
\pstart
           \textcolor{gray}{\textbf{Kolowratring, Fichtegasse Nr. 11\oindex{Wien@\textbf{Wien}!I., Innere Stadt@\textbf{I., Innere Stadt}!Fichtegasse 11@\textbf{Fichtegasse 11}, \emph{Gebäude}|pw}.}}\pend
           
\pstart{}Lieber Doctor,\pend\vspace{0.5em}
\pstart
           gestern hatte ich Schreibtag, kam erst Nachts ins Bureau\oindex{Wien@\textbf{Wien}!I., Innere Stadt@\textbf{I., Innere Stadt}!Fichtegasse 11@\textbf{Fichtegasse 11}, \emph{Gebäude}|pwv}, wo ich Ihren \label{K_L03877-1v}\edtext{Brief}{\lemma{\textnormal{\emph{Brief}}}\Cendnote{\textnormal{XXXX22.12.1900}}}\label{K_L03877-1} fand.\pend
           
\pstart
           Dass wir Ihre Novelle\pwindex{Schnitzler, Arthur 15.\,5.\,1862 Wien – 21.\,10.\,1931 ebd.@\textsc{Schnitzler, Arthur} (15.\,5.\,1862 Wien – 21.\,10.\,1931 ebd.), \emph{Schriftsteller, Mediziner}!Lieutenant Gustl. Novelle@\strich\emph{Lieutenant Gustl. Novelle}|pwv} nicht
               ins Weihnachtsblatt\pwindex{Neue Freie Presse@\emph{Neue Freie Presse}|pwv} bringen
               können, entdeckten wir, wie wunderlich Ihnen das auch scheinen mag, erst an dem Tage
               wo ich Ihnen schrieb – Donnerstag oder Freitag. Ich setzte
               mich gleich hin, um Ihnen diese unangenehme Mittheilung zu machen, nachdem Benedikt\pwindex{Benedikt, Moriz 27.\,5.\,1849 Kvačice – 18.\,3.\,1920 Wien@\textsc{Benedikt, Moriz} (27.\,5.\,1849 Kvačice – 18.\,3.\,1920 Wien), \emph{Journalist, Herausgeber}|pw} aus der Setzerei gekommen war u. es
               mir \introOben{}ge\introOben{}sagt\substVorne{}\textsuperscript{e}\substDazwischen{}{ }hatte\substHinten{}. Eine unfreundliche Absicht hatte dabei weder er noch ich, wozu
               hätten wir Sie sonst aufgefordert? Es war vielmehr beschlossen, dass Ihre Novelle\pwindex{Schnitzler, Arthur 15.\,5.\,1862 Wien – 21.\,10.\,1931 ebd.@\textsc{Schnitzler, Arthur} (15.\,5.\,1862 Wien – 21.\,10.\,1931 ebd.), \emph{Schriftsteller, Mediziner}!Lieutenant Gustl. Novelle@\strich\emph{Lieutenant Gustl. Novelle}|pwv} in der Weihnachtsbeilage\pwindex{Neue Freie Presse@\emph{Neue Freie Presse}|pwv} unterm Strich kommen u. weiterlaufen \introOben{}ich glaube, unter den Annonceblättern weiter so wie die
                  Fachblätter, muss aber gestehen, dass ich darüber nicht ganz genau unterrichtet
                  wurde; es geht mich auch nichts an.\introOben{} sollte. Diese technischen {\pb}Details werden Sie nicht interessiren.
               Genug, das Wegbleiben Ihrer Arbeit\pwindex{Schnitzler, Arthur 15.\,5.\,1862 Wien – 21.\,10.\,1931 ebd.@\textsc{Schnitzler, Arthur} (15.\,5.\,1862 Wien – 21.\,10.\,1931 ebd.), \emph{Schriftsteller, Mediziner}!Lieutenant Gustl. Novelle@\strich\emph{Lieutenant Gustl. Novelle}|pwv} hat rein typographische Gründe. Wir hätten nicht einen, sondern alle
               anderen Beiträge weglassen müssen.\pend
           
\pstart
           Für die Weihnachtsbeilage\pwindex{Neue Freie Presse@\emph{Neue Freie Presse}|pwv}
               mussten wir also auf Ihre Novelle\pwindex{Schnitzler, Arthur 15.\,5.\,1862 Wien – 21.\,10.\,1931 ebd.@\textsc{Schnitzler, Arthur} (15.\,5.\,1862 Wien – 21.\,10.\,1931 ebd.), \emph{Schriftsteller, Mediziner}!Lieutenant Gustl. Novelle@\strich\emph{Lieutenant Gustl. Novelle}|pwv} verzichten. Ihrem Wunsch, die Novelle\pwindex{Schnitzler, Arthur 15.\,5.\,1862 Wien – 21.\,10.\,1931 ebd.@\textsc{Schnitzler, Arthur} (15.\,5.\,1862 Wien – 21.\,10.\,1931 ebd.), \emph{Schriftsteller, Mediziner}!Lieutenant Gustl. Novelle@\strich\emph{Lieutenant Gustl. Novelle}|pwv} nur in einem Stück, nicht in Fortsetzungen
               erscheinen zu lassen, werde ich mit den Herausgebern\pwindex{Benedikt, Moriz 27.\,5.\,1849 Kvačice – 18.\,3.\,1920 Wien@\textsc{Benedikt, Moriz} (27.\,5.\,1849 Kvačice – 18.\,3.\,1920 Wien), \emph{Journalist, Herausgeber}|pwv}\pwindex{Bacher, Eduard 7.\,3.\,1846 Postoloprty – 16.\,1.\,1908 Wien@\textsc{Bacher, Eduard} (7.\,3.\,1846 Postoloprty – 16.\,1.\,1908 Wien), \emph{Journalist, Herausgeber}|pwv} besprechen u. werde mich bemühen, sie zur
               Einlegung eines Blattes zu bewegen, u. zw. so bald als möglich. Sie werden meine
               Antwort in den nächsten Tagen haben.\pend
           
\pstart
           Ich wundere mich nur, dass Sie diese Sache so übel aufnehmen, nachdem Ihnen die N Fr Presse\orgindex{Neue Freie Presse@Neue Freie Presse|pw} doch wiederholte und genügende Beweise einer
               freundlichen Gesinnung gegeben hat. Von mir persönlich will ich da gar nicht
               sprechen.\pend
           
\pstart
           Mit bestem Gruß{\\[\baselineskip]}Ihr ergebener{\\[\baselineskip]}\spacefill\mbox{Herzl}\pend
           \leftskip=0em{}\selectlanguage{ngerman}\endnumbering\briefempfaengerindex{Schnitzler, Arthur@\textsc{Schnitzler, Arthur}!zzzHerzl, Theodor@\emph{von Theodor Herzl}!1900-12-231@{23. 12. 1900}|)be}\mylabel{L03877h}
\begin{anhang}
\end{anhang}\newcommand{\dateiname}{L03877}\newcommand{\titel}{Theodor Herzl an Arthur Schnitzler, 23. 12. 1900}\newcommand{\editorInnen}{Selma Jahnke und Martin Anton Müller}%% latex-leseansicht-abspann.tex
%% Abspann für die Leseansicht.
%% Der Schalter \ifkorrekturansicht ist bereits durch den Vorspann gesetzt.

%% latex-abspann.tex
%% Gemeinsamer Abspann für Korrekturansicht und Leseansicht.
%% Setzt den Schalter \ifkorrekturansicht voraus (gesetzt in den
%% einbindenden Dateien latex-korrekturansicht-abspann.tex bzw.
%% latex-leseansicht-abspann.tex).
%% ---------------------------------------------------------------

\normalsize

% Das esempio-Environment wird nur in der Leseansicht benötigt
\ifkorrekturansicht\else
\newenvironment{esempio}[3]%
{
    \vspace{1.5ex}
    \rlap{\underline{#1}}
    \par
    \setlength{\parindent}{0cm}
    \nopagebreak
    \leftskip=#2cm
    \rightskip=#3cm
}
{
    \par
}
\fi

\doendnotes{C}
\bigskip
\vfill

\clearpage

\footnotesize

\ifkorrekturansicht
  \lohead{\textsc{register}}
\fi

% theindex-Environment neu definieren ohne reledmac
\makeatletter
\renewenvironment{theindex}{%
  \ifkorrekturansicht
    \section*{\indexname}%
  \else
    \subsubsection*{Index der erwähnten Entitäten}%
  \fi
  \setlength{\parindent}{0pt}%
  \setlength{\parskip}{0pt plus 0.3pt}%
  \let\item\@idxitem
}{%
  \ifkorrekturansicht\clearpage\fi
}
\makeatother

\IfFileExists{\jobname-pw.ind}{\input{\jobname-pw.ind}}{}

% Quellenangabe nur in der Leseansicht
\ifkorrekturansicht\else
% Fallback-Definitionen, falls die .tex-Datei \titel etc. nicht gesetzt hat
\providecommand{\titel}{}
\providecommand{\editorInnen}{}
\providecommand{\dateiname}{\jobname}

\vspace{3cm}

\vfill

\footnotesize
\textsc{Quelle}: \titel. Herausgegeben von {\editorInnen}. In: \emph{Arthur Schnitzler: Briefwechsel mit Autorinnen und Autoren}.
 Digitale Edition, https://schnitzler-briefe.acdh.oeaw.ac.at/{\dateiname}.html (Stand \today)
\fi

\end{document}


