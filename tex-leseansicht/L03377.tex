%% latex-leseansicht-vorspann.tex
%% Vorspann für die Leseansicht.
%% Lädt die gemeinsame Datei latex-vorspann.tex mit nicht gesetztem Schalter.

\newif\ifkorrekturansicht
\korrekturansichtfalse

\input{../tex-inputs/latex-vorspann}


         
         \renewcommand{\erwaehntePersonen}{Personen: Richard Beer-Hofmann, Paul Goldmann, Rudyard Kipling, Bertold Löffler, Friedrich Wilhelm Riemer, Theodore Rottenberg, Olga Schnitzler}
         \renewcommand{\erwaehnteInstitutionen}{Institutionen: Wiener Verlag}
         \renewcommand{\erwaehnteOrte}{Orte: Berlin, Dessauer Straße, Eppan an der Weinstraße, Frankfurt am Main, Leipzig, Südtirol, Welsberg-Taisten, Wien}
         \renewcommand{\erwaehnteWerke}{Werke: Bibliothek berühmter Autoren, Das Mädchen aus Birma, Das Mädchen aus Birma und andere Geschichten, Mittheilungen über Goethe. Aus mündlichen und schriftlichen, gedruckten und ungedruckten Quellen}
               \section[ Paul Goldmann an Arthur Schnitzler, 19. 7. {[}1903{]}]{ Paul Goldmann an Arthur Schnitzler, 19. 7. {[}1903{]}}\nopagebreak\mylabel{v}\rehead{ }\begin{ledgroupsized}[t]{13cm}\normalsize\beginnumbering\briefempfaengerindex{Schnitzler, Arthur@\textsc{Schnitzler, Arthur}!zzzGoldmann, Paul@\emph{von Paul Goldmann}!1903-07-192@{19. 7. {[}1903{]}}|(be} \toendnotes[C]{\smallbreak\pagebreak[2]} \Standort{DLA, A:Schnitzler, HS.NZ85.1.3173.}
\physDesc{Brief, 1 Blatt, 3 Seiten, 1448 Zeichen
\newline{}Handschrift: blaue Tinte, deutsche Kurrent
\newline{}Schnitzler: 1) mit Bleistift das Jahr »903« vermerkt  2) mit rotem Buntstift eine Unterstreichung}\toendnotes[C]{\smallbreak}\pstart
           \noindent{}\raggedleft{}{\pb}\textcolor{gray}{\textbf{DESSAUERSTRASSE 19\oindex{Dessauer Strasse@\textbf{Dessauer Straße}|pw}}}\pend
           \pstart
           Berlin\oindex{Berlin@\textbf{Berlin}|pw}, 19. Juli.\pend
           \pstart\center{}Mein lieber Freund,\pend\pstart
           Ich war in Frankfurt\oindex{Frankfurt am Main@\textbf{Frankfurt am Main}|pw}, ich habe \label{K_L03377-1v}\edtext{ſie\pwindex{Rottenberg, Theodore 1875-09-07 – 1945-04-05@\textsc{Rottenberg, Theodore} (1875-09-07 – 1945-04-05)|pwv}}{\lemma{\textnormal{\emph{ſie}}}\Cendnote{\textnormal{Theodore Rottenberg\pwindex{Rottenberg, Theodore 1875-09-07 – 1945-04-05@\textsc{Rottenberg, Theodore} (1875-09-07 – 1945-04-05)|pwk}, die das seit 1899 andauernde Verhältnis mit Goldmann\pwindex{Goldmann, Paul 31.01.1865 – 25.09.1935@\textsc{Goldmann, Paul} (31.01.1865 – 25.09.1935), \emph{Schriftsteller, Journalist}|pwk} Anfang 1903 beendet
                  hatte (vgl. Paul Goldmann an Arthur Schnitzler, 3. 1. [1903])}}}\label{K_L03377-1h}
               wiedergeſehen, und ich weiß jetzt: daß dieſe Frau\pwindex{Rottenberg, Theodore 1875-09-07 – 1945-04-05@\textsc{Rottenberg, Theodore} (1875-09-07 – 1945-04-05)|pwv} (trotz Allem) rein und wahr und ein Engel von Güte iſt.
               Ich war Jahre lang ein blinder Thor und ich habe mein Glück mit Füßen von mir
               geſtoßen. Sie liebt mich nicht mehr, weil die Verachtung die Liebe in ihr ertödtet
               hat. Aber ſie hat den Wunſch, mich wieder lieben zu können. Wenn ich in Frankfurt\oindex{Frankfurt am Main@\textbf{Frankfurt am Main}|pw} lebte, könnte ich ſie vielleicht
               wiedergewinnen. Die Entfernung verurtheilt mich zur Ohnmacht. Aber ich habe \strikeout{ih} ihr geſagt, daß mein Leben jetzt ihr gehört; und ſie
               hat dieſe Gabe angenommen, ohne ſich einſtweilen jedoch ihrerſeits zu {\pb}binden. Das Alles kann ich Dir nur mündlich
               erklären. Zum Schreiben fehlt mir die Zeit und die Kraft.\pend
           \pstart
           Meine Sommerpläne hängen von ihr\pwindex{Rottenberg, Theodore 1875-09-07 – 1945-04-05@\textsc{Rottenberg, Theodore} (1875-09-07 – 1945-04-05)|pwv} ab. Es iſt nämlich eine, allerdings ſehr ſchwache Möglichkeit, daß ſie
                  \label{K_L03377-2v}\edtext{mit mir auf 14 Tage nach Südtirol\oindex{Suedtirol@\textbf{Südtirol}|pw}}{\lemma{\textnormal{\emph{mit … Südtirol}}}\Cendnote{\textnormal{Rottenberg\pwindex{Rottenberg, Theodore 1875-09-07 – 1945-04-05@\textsc{Rottenberg, Theodore} (1875-09-07 – 1945-04-05)|pwk} kam mit, vgl. Paul Goldmann an Arthur Schnitzler, 27. 6. [1903]. }}}\label{K_L03377-2h} kommt. Weißt
               Du einen ſchönen, kühlen, \uline{billigen} Ort, abſeits von
               der Touriſten-Heerſtraße? \textsc{Welsberg\oindex{Welsberg-Taisten@\textbf{Welsberg-Taisten}|pw}} iſt ausgeſchloſſen, weil dort \label{K_L03377-3v}\edtext{Berlin\oindex{Berlin@\textbf{Berlin}|pw}er Bekannte}{\lemma{\textnormal{\emph{Berliner Bekannte}}}\Cendnote{\textnormal{Rottenberg\pwindex{Rottenberg, Theodore 1875-09-07 – 1945-04-05@\textsc{Rottenberg, Theodore} (1875-09-07 – 1945-04-05)|pwk} war verheiratet, die Beziehung
                  also nicht so, dass man sich in der Öffentlichkeit gemeinsam zeigen konnte.}}}\label{K_L03377-3h}
               von mir ſind. Wenn die Reiſe zuſtandekommt, wirſt Du, wie ich hoffe, es einrichten
               können, mit uns \label{K_L03377-4v}\edtext{zuſammenzutreffen}{\lemma{\textnormal{\emph{zuſammenzutreffen}}}\Cendnote{\textnormal{siehe Paul Goldmann an Arthur Schnitzler, 27. 6. [1903]}}}\label{K_L03377-4h}. Aber, wie geſagt, das liegt Alles noch ſehr im Nebel.\pend
           \pstart
           {\pb}Jedenfalls gib’ mir einen Rath, wo man ſich
               wiedertreffen könnte. Iſt \textsc{Eppan\oindex{Eppan an der Weinstrasse@\textbf{Eppan an der Weinstraße}|pw}} ſchön, wo \label{K_L03377-5v}\edtext{\textsc{Richard\pwindex{Beer-Hofmann, Richard 1866-07-11 – 1945-09-26@\textsc{Beer-Hofmann, Richard} (1866-07-11 – 1945-09-26), \emph{Schriftsteller}|pw}}}{\lemma{\textnormal{\emph{Richard}}}\Cendnote{\textnormal{Beer-Hofmann\pwindex{Beer-Hofmann, Richard 1866-07-11 – 1945-09-26@\textsc{Beer-Hofmann, Richard} (1866-07-11 – 1945-09-26), \emph{Schriftsteller}|pwk} war im Herbst 1899 in Eppan\oindex{Eppan an der Weinstrasse@\textbf{Eppan an der Weinstraße}|pwk} gewesen,
                     vgl. Richard Beer-Hofmann an Arthur Schnitzler, 1. 10. 1899.}}}\label{K_L03377-5h} war?\pend
           \pstart
           Grüße mir \textsc{Olga\pwindex{Schnitzler, Olga 17.01.1882 – 13.01.1970@\textsc{Schnitzler, Olga} (17.01.1882 – 13.01.1970), \emph{Schauspielerin, Sängerin}|pw}} (seid \strikeout{\textcolor{gray}{×}\-\textcolor{gray}{×}} Ihr nun \label{K_L03377-6v}\edtext{verheirathet}{\lemma{\textnormal{\emph{verheirathet}}}\Cendnote{\textnormal{Sie heirateten am 26. 8. 1903.}}}\label{K_L03377-6h}
               oder nicht?) und ſei ſelbſt tauſendmal gegrüßt von {\\[\baselineskip]}Deinem getreuen {\\[\baselineskip]}\spacefill\mbox{Paul Goldmann}\pend
           \leftskip=0em{}\pstart
           \noindent{}Dank für \label{K_L03377-7v}\edtext{\textsc{Riemer\pwindex{Riemer, Friedrich Wilhelm 1774-04-19 – 1845-12-19@\textsc{Riemer, Friedrich Wilhelm} (1774-04-19 – 1845-12-19), \emph{Schriftsteller, Philologe, Bibliothekar}|pw}}}{\lemma{\textnormal{\emph{Riemer}}}\Cendnote{\textnormal{Obwohl kein Titel genannt wurde,
                     dürfte es sich um dessen Hauptwerk \emph{Mittheilungen über Goethe. Aus mündlichen und schriftlichen, gedruckten und
                        ungedruckten Quellen}\pwindex{Riemer, Friedrich Wilhelm 1774-04-19 – 1845-12-19@\textsc{Riemer, Friedrich Wilhelm} (1774-04-19 – 1845-12-19), \emph{Schriftsteller, Philologe, Bibliothekar}!Mittheilungen ueber Goethe. Aus muendlichen und schriftlichen, gedruckten und ungedruckten Quellen1843@\strich\emph{Mittheilungen über Goethe. Aus mündlichen und schriftlichen, gedruckten und ungedruckten Quellen} {[}1843{]}|pwk} aus dem Jahr 1843
                     gehandelt haben. }}}\label{K_L03377-7h}!\pend
           \pstart
           Lies: \label{K_L03377-8v}\edtext{\textsc{Kipling\pwindex{Kipling, Rudyard 30.12.1865 – 18.01.1936@\textsc{Kipling, Rudyard} (30.12.1865 – 18.01.1936), \emph{Schriftsteller}|pw}}, Das Mädchen von \textsc{Birma}\pwindex{Kipling, Rudyard 30.12.1865 – 18.01.1936@\textsc{Kipling, Rudyard} (30.12.1865 – 18.01.1936), \emph{Schriftsteller}!Maedchen aus Birma1903@\strich\emph{Das Mädchen aus Birma} {[}1903{]}|pw}}{\lemma{\textnormal{\emph{Kipling, … Birma}}}\Cendnote{\textnormal{\emph{Das Mädchen aus Birma}\pwindex{Kipling, Rudyard 30.12.1865 – 18.01.1936@\textsc{Kipling, Rudyard} (30.12.1865 – 18.01.1936), \emph{Schriftsteller}!Maedchen aus Birma1903@\strich\emph{Das Mädchen aus Birma} {[}1903{]}|pwk} ist enthalten in: Rudyard Kipling\pwindex{Kipling, Rudyard 30.12.1865 – 18.01.1936@\textsc{Kipling, Rudyard} (30.12.1865 – 18.01.1936), \emph{Schriftsteller}|pwk}: \emph{Das Mädchen aus Birma und andere Geschichten}\pwindex{Loeffler, Bertold 28.09.1874 – 23.03.1960@\textsc{Löffler, Bertold} (28.09.1874 – 23.03.1960), \emph{Maler, Grafiker}!Maedchen aus Birma und andere Geschichten1903@\strich\emph{Das Mädchen aus Birma und andere Geschichten} {[}Illustration, 1903{]}|pwk}.
                        Autorisierte Übersetzung aus dem Englischen. Umschlag von Berthold Löffler\pwindex{Loeffler, Bertold 28.09.1874 – 23.03.1960@\textsc{Löffler, Bertold} (28.09.1874 – 23.03.1960), \emph{Maler, Grafiker}|pwk}. Wien\oindex{Wien@\textbf{Wien}|pwk}/Leipzig\oindex{Leipzig@\textbf{Leipzig}|pwk}: \emph{Wiener Verlag}\orgindex{Wiener Verlag@Wiener Verlag|pwk}{ }1903. (\emph{Bibliothek
                           berühmter Autoren}\pwindex{?? Werk@Nicht ermittelte Verfasserinnen und Verfasser!Bibliothek beruehmter Autoren1903 – 1905@\emph{Bibliothek berühmter Autoren} {[}1903 – 1905{]}|pwk} 8) Eine Lektüre durch Schnitzler\pwindex{Schnitzler, Arthur 15.05.1862 – 21.10.1931@\textsc{Schnitzler, Arthur} (15.05.1862 – 21.10.1931), \emph{Schriftsteller, Mediziner}|pwk} ist nicht bekannt.}}}\label{K_L03377-8h}.\pend
           
         
         \endnumbering\mylabel{h}\end{ledgroupsized}  \newcommand{\dateiname}{L03377}\newcommand{\titel}{Paul Goldmann an Arthur Schnitzler, 19. 7. [1903]}\newcommand{\editorInnen}{Martin Anton Müller und Laura Untner}%% latex-leseansicht-abspann.tex
%% Abspann für die Leseansicht.
%% Der Schalter \ifkorrekturansicht ist bereits durch den Vorspann gesetzt.

%% latex-abspann.tex
%% Gemeinsamer Abspann für Korrekturansicht und Leseansicht.
%% Setzt den Schalter \ifkorrekturansicht voraus (gesetzt in den
%% einbindenden Dateien latex-korrekturansicht-abspann.tex bzw.
%% latex-leseansicht-abspann.tex).
%% ---------------------------------------------------------------

\normalsize

% Das esempio-Environment wird nur in der Leseansicht benötigt
\ifkorrekturansicht\else
\newenvironment{esempio}[3]%
{
    \vspace{1.5ex}
    \rlap{\underline{#1}}
    \par
    \setlength{\parindent}{0cm}
    \nopagebreak
    \leftskip=#2cm
    \rightskip=#3cm
}
{
    \par
}
\fi

\doendnotes{C}
\bigskip
\vfill

\clearpage

\footnotesize

\ifkorrekturansicht
  \lohead{\textsc{register}}
\fi

% theindex-Environment neu definieren ohne reledmac
\makeatletter
\renewenvironment{theindex}{%
  \ifkorrekturansicht
    \section*{\indexname}%
  \else
    \subsubsection*{Index der erwähnten Entitäten}%
  \fi
  \setlength{\parindent}{0pt}%
  \setlength{\parskip}{0pt plus 0.3pt}%
  \let\item\@idxitem
}{%
  \ifkorrekturansicht\clearpage\fi
}
\makeatother

\IfFileExists{\jobname-pw.ind}{\input{\jobname-pw.ind}}{}

% Quellenangabe nur in der Leseansicht
\ifkorrekturansicht\else
% Fallback-Definitionen, falls die .tex-Datei \titel etc. nicht gesetzt hat
\providecommand{\titel}{}
\providecommand{\editorInnen}{}
\providecommand{\dateiname}{\jobname}

\vspace{3cm}

\vfill

\footnotesize
\textsc{Quelle}: \titel. Herausgegeben von {\editorInnen}. In: \emph{Arthur Schnitzler: Briefwechsel mit Autorinnen und Autoren}.
 Digitale Edition, https://schnitzler-briefe.acdh.oeaw.ac.at/{\dateiname}.html (Stand \today)
\fi

\end{document}


      