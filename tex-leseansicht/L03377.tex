%% latex-korrekturansicht-vorspann.tex
%% Vorspann für die Korrekturansicht.
%% Lädt die gemeinsame Datei latex-vorspann.tex mit gesetztem Schalter.

\newif\ifkorrekturansicht
\korrekturansichttrue

\input{../tex-inputs/latex-vorspann}


\section[ Paul Goldmann an Arthur Schnitzler, 19. 7. {[}1903{]}]{L03377 Paul Goldmann an Arthur Schnitzler, 19. 7. {[}1903{]}}
\nopagebreak\mylabel{L03377v}
\rehead{ }\normalsize\beginnumbering\briefempfaengerindex{Schnitzler, Arthur@\textsc{Schnitzler, Arthur}!zzzGoldmann, Paul@\emph{von Paul Goldmann}!1903-07-192@{19. 7. {[}1903{]}}|(be}
\toendnotes[C]{\smallbreak\pagebreak[2]}\Standort{DLA, A:Schnitzler, HS.NZ85.1.3173.}
\physDesc{Brief, 1 Blatt, 3 Seiten, 1448 Zeichen
\newline{}Handschrift: blaue Tinte, deutsche Kurrent
\newline{}Schnitzler: 1) mit Bleistift das Jahr »903« vermerkt  2) mit rotem Buntstift eine Unterstreichung}\toendnotes[C]{\smallbreak}
\pstart
           \raggedleft{}{\pb}\textcolor{gray}{\textbf{DESSAUERSTRASSE 19\oindex{Dessauer Strasse@\textbf{Dessauer Straße}, \emph{Straße (K.STR)}|pw}}}\pend
           
\pstart
           Berlin\oindex{Berlin@\textbf{Berlin}, \emph{P.PPLC}|pw}, 19. Juli.\pend
           
\pstart\center{}Mein lieber Freund,\pend\vspace{0.5em}
\pstart
           Ich war in Frankfurt\oindex{Frankfurt am Main@\textbf{Frankfurt am Main}, \emph{P.PPLA3}|pw}, ich habe \label{K_L03377-1v}\edtext{ſie\pwindex{Rottenberg, Theodore 1875-09-07 – 1945-04-05@\textsc{Rottenberg, Theodore} (1875-09-07 – 1945-04-05)|pwv}}{\lemma{\textnormal{\emph{ſie}}}\Cendnote{\textnormal{Theodore Rottenberg\pwindex{Rottenberg, Theodore 1875-09-07 – 1945-04-05@\textsc{Rottenberg, Theodore} (1875-09-07 – 1945-04-05)|pwk}, die das seit 1899 andauernde Verhältnis mit Goldmann\pwindex{Goldmann, Paul 31.01.1865 – 25.09.1935@\textsc{Goldmann, Paul} (31.01.1865 – 25.09.1935), \emph{Schriftsteller/Schriftstellerin, Journalist/Journalistin}|pwk} Anfang 1903 beendet
                  hatte (vgl. Paul Goldmann an Arthur Schnitzler, 3. 1. [1903]).}}}\label{K_L03377-1}
               wiedergeſehen, und ich weiß jetzt: daß dieſe Frau\pwindex{Rottenberg, Theodore 1875-09-07 – 1945-04-05@\textsc{Rottenberg, Theodore} (1875-09-07 – 1945-04-05)|pwv} (trotz Allem) rein und wahr und ein Engel von Güte iſt.
               Ich war Jahre lang ein blinder Thor und ich habe mein Glück mit Füßen von mir
               geſtoßen. Sie liebt mich nicht mehr, weil die Verachtung die Liebe in ihr ertödtet
               hat. Aber ſie hat den Wunſch, mich wieder lieben zu können. Wenn ich in Frankfurt\oindex{Frankfurt am Main@\textbf{Frankfurt am Main}, \emph{P.PPLA3}|pw} lebte, könnte ich ſie vielleicht
               wiedergewinnen. Die Entfernung verurtheilt mich zur Ohnmacht. Aber ich habe \strikeout{ih} ihr geſagt, daß mein Leben jetzt ihr gehört; und ſie
               hat dieſe Gabe angenommen, ohne ſich einſtweilen jedoch ihrerſeits zu {\pb}binden. Das Alles kann ich Dir nur mündlich
               erklären. Zum Schreiben fehlt mir die Zeit und die Kraft.\pend
           
\pstart
           Meine Sommerpläne hängen von ihr\pwindex{Rottenberg, Theodore 1875-09-07 – 1945-04-05@\textsc{Rottenberg, Theodore} (1875-09-07 – 1945-04-05)|pwv} ab. Es iſt nämlich eine, allerdings ſehr ſchwache Möglichkeit, daß ſie
                  \label{K_L03377-2v}\edtext{mit mir auf 14 Tage nach Südtirol\oindex{Suedtirol@\textbf{Südtirol}, \emph{A.ADM2}|pw}}{\lemma{\textnormal{\emph{mit … Südtirol}}}\Cendnote{\textnormal{Rottenberg\pwindex{Rottenberg, Theodore 1875-09-07 – 1945-04-05@\textsc{Rottenberg, Theodore} (1875-09-07 – 1945-04-05)|pwk} kam mit, vgl. Paul Goldmann an Arthur Schnitzler, 27. 6. [1903]. }}}\label{K_L03377-2} kommt. Weißt
               Du einen ſchönen, kühlen, \uline{billigen} Ort, abſeits von
               der Touriſten-Heerſtraße? \textsc{Welsberg\oindex{Welsberg-Taisten@\textbf{Welsberg-Taisten}, \emph{A.ADM3}|pw}} iſt ausgeſchloſſen, weil dort \label{K_L03377-3v}\edtext{Berlin\oindex{Berlin@\textbf{Berlin}, \emph{P.PPLC}|pw}er Bekannte}{\lemma{\textnormal{\emph{Berliner Bekannte}}}\Cendnote{\textnormal{Rottenberg\pwindex{Rottenberg, Theodore 1875-09-07 – 1945-04-05@\textsc{Rottenberg, Theodore} (1875-09-07 – 1945-04-05)|pwk} war verheiratet, die Beziehung
                  also nicht so, dass man sich in der Öffentlichkeit gemeinsam zeigen konnte.}}}\label{K_L03377-3}
               von mir ſind. Wenn die Reiſe zuſtandekommt, wirſt Du, wie ich hoffe, es einrichten
               können, mit uns \label{K_L03377-4v}\edtext{zuſammenzutreffen}{\lemma{\textnormal{\emph{zuſammenzutreffen}}}\Cendnote{\textnormal{Siehe Paul Goldmann an Arthur Schnitzler, 27. 6. [1903].
               }}}\label{K_L03377-4}. Aber, wie geſagt, das liegt Alles noch ſehr im Nebel.\pend
           
\pstart
           {\pb}Jedenfalls gib’ mir einen Rath, wo man ſich
               wiedertreffen könnte. Iſt \textsc{Eppan\oindex{Eppan an der Weinstrasse@\textbf{Eppan an der Weinstraße}, \emph{A.ADM3}|pw}} ſchön, wo \label{K_L03377-5v}\edtext{\textsc{Richard\pwindex{Beer-Hofmann, Richard 1866-07-11 – 1945-09-26@\textsc{Beer-Hofmann, Richard} (1866-07-11 – 1945-09-26), \emph{Schriftsteller/Schriftstellerin}|pw}}}{\lemma{\textnormal{\emph{Richard}}}\Cendnote{\textnormal{Beer-Hofmann\pwindex{Beer-Hofmann, Richard 1866-07-11 – 1945-09-26@\textsc{Beer-Hofmann, Richard} (1866-07-11 – 1945-09-26), \emph{Schriftsteller/Schriftstellerin}|pwk} war im Herbst 1899 in Eppan\oindex{Eppan an der Weinstrasse@\textbf{Eppan an der Weinstraße}, \emph{A.ADM3}|pwk} gewesen,
                     vgl. Richard Beer-Hofmann an Arthur Schnitzler, 1. 10. 1899.}}}\label{K_L03377-5} war?\pend
           
\pstart
           Grüße mir \textsc{Olga\pwindex{Schnitzler, Olga 17.01.1882 – 13.01.1970@\textsc{Schnitzler, Olga} (17.01.1882 – 13.01.1970), \emph{Schauspieler/Schauspielerin, Sänger/Sängerin}|pw}} (seid \strikeout{\textcolor{gray}{×}\-\textcolor{gray}{×}} Ihr nun \label{K_L03377-6v}\edtext{verheirathet}{\lemma{\textnormal{\emph{verheirathet}}}\Cendnote{\textnormal{Arhtur Schnitzler und Olga Gussmann\pwindex{Schnitzler, Olga 17.01.1882 – 13.01.1970@\textsc{Schnitzler, Olga} (17.01.1882 – 13.01.1970), \emph{Schauspieler/Schauspielerin, Sänger/Sängerin}|pwk} heirateten am 26. 8. 1903.}}}\label{K_L03377-6}
               oder nicht?) und ſei ſelbſt tauſendmal gegrüßt von {\\[\baselineskip]}Deinem getreuen {\\[\baselineskip]}\spacefill\mbox{Paul Goldmann}\pend
           \leftskip=0em{}
\pstart
           \noindent{}Dank für \label{K_L03377-7v}\edtext{\textsc{Riemer\pwindex{Riemer, Friedrich Wilhelm 1774-04-19 – 1845-12-19@\textsc{Riemer, Friedrich Wilhelm} (1774-04-19 – 1845-12-19), \emph{Schriftsteller/Schriftstellerin, Philologe/Philologin, Bibliothekar/Bibliothekarin}|pw}}}{\lemma{\textnormal{\emph{Riemer}}}\Cendnote{\textnormal{Obwohl kein Titel genannt wurde,
                     dürfte es sich um dessen Hauptwerk \emph{Mittheilungen über Goethe. Aus mündlichen und schriftlichen, gedruckten und
                        ungedruckten Quellen}\pwindex{Mittheilungen ueber Goethe. Aus muendlichen und schriftlichen, gedruckten und ungedruckten Quellen@\emph{Mittheilungen über Goethe. Aus mündlichen und schriftlichen, gedruckten und ungedruckten Quellen}|pwk} aus dem Jahr 1843
                     gehandelt haben. }}}\label{K_L03377-7}!\pend
           
\pstart
           Lies: \label{K_L03377-8v}\edtext{\textsc{Kipling\pwindex{Kipling, Rudyard 30.12.1865 – 18.01.1936@\textsc{Kipling, Rudyard} (30.12.1865 – 18.01.1936), \emph{Schriftsteller/Schriftstellerin}|pw}}, Das Mädchen von \textsc{Birma}\pwindex{Maedchen aus Birma@\emph{Das Mädchen aus Birma}|pw}}{\lemma{\textnormal{\emph{Kipling, … Birma}}}\Cendnote{\textnormal{\emph{Das Mädchen aus Birma}\pwindex{Maedchen aus Birma@\emph{Das Mädchen aus Birma}|pwk} ist enthalten in: Rudyard Kipling\pwindex{Kipling, Rudyard 30.12.1865 – 18.01.1936@\textsc{Kipling, Rudyard} (30.12.1865 – 18.01.1936), \emph{Schriftsteller/Schriftstellerin}|pwk}: \emph{Das Mädchen aus Birma und andere Geschichten}\pwindex{Maedchen aus Birma und andere Geschichten@\emph{Das Mädchen aus Birma und andere Geschichten}|pwk}.
                        Autorisierte Übersetzung aus dem Englischen. Umschlag von Berthold Löffler\pwindex{Loeffler, Bertold 28.09.1874 – 23.03.1960@\textsc{Löffler, Bertold} (28.09.1874 – 23.03.1960), \emph{Maler/Malerin, Grafiker/Grafikerin}|pwk}. Wien\oindex{Wien@\textbf{Wien}, \emph{A.ADM2}|pwk}/Leipzig\oindex{Leipzig@\textbf{Leipzig}, \emph{P.PPLA3}|pwk}: \emph{Wiener Verlag}\orgindex{Wiener Verlag@Wiener Verlag|pwk}{ }1903 (\emph{Bibliothek
                           berühmter Autoren}\pwindex{Bibliothek beruehmter Autoren@\emph{Bibliothek berühmter Autoren}|pwk} 8). Eine Lektüre durch Schnitzler ist nicht bekannt.}}}\label{K_L03377-8}.\pend
           \selectlanguage{ngerman}\endnumbering\briefempfaengerindex{Schnitzler, Arthur@\textsc{Schnitzler, Arthur}!zzzGoldmann, Paul@\emph{von Paul Goldmann}!1903-07-192@{19. 7. {[}1903{]}}|)be}\mylabel{L03377h}  \normalsize

\doendnotes{C}
\bigskip
\vfill

\clearpage

\footnotesize

\lohead{\textsc{register}}

% Definiere theindex-Environment komplett neu ohne reledmac
\makeatletter
\renewenvironment{theindex}{%
  \section*{\indexname}%
  \setlength{\parindent}{0pt}%
  \setlength{\parskip}{0pt plus 0.3pt}%
  \let\item\@idxitem
}{%
  \clearpage
}
\makeatother

\IfFileExists{\jobname-pw.ind}{\input{\jobname-pw.ind}}{}

\end{document}

      