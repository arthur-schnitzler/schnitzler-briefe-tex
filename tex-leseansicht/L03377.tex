%% latex-leseansicht-vorspann.tex
%% Vorspann für die Leseansicht.
%% Lädt die gemeinsame Datei latex-vorspann.tex mit nicht gesetztem Schalter.

\newif\ifkorrekturansicht
\korrekturansichtfalse

\input{../tex-inputs/latex-vorspann}


\section[ Paul Goldmann an Arthur Schnitzler, 19. 7. [1903]]{L03377 Paul Goldmann an Arthur Schnitzler,  19. 7. [1903]}
\nopagebreak\mylabel{L03377v}
\rehead{ }\normalsize\beginnumbering\briefempfaengerindex{Schnitzler, Arthur@\textsc{Schnitzler, Arthur}!zzzGoldmann, Paul@\emph{von Paul Goldmann}!1903-07-192@{19. 7. [1903]}|(be}
\toendnotes[C]{\smallbreak\pagebreak[2]}
\correspDesc{Versand  durch Paul Goldmann am 19. 7. [1903] in Berlin
\newline{}Erhalt  durch Arthur Schnitzler im Zeitraum [20. 7. 1903
                  – 22. 7. 1903?] in Wien}\toendnotes[C]{\smallbreak}
\Standort{DLA, A:Schnitzler, HS.NZ85.1.3173.}
\physDesc{Brief, 1 Blatt, 3 Seiten, 1448 Zeichen
\newline{}Handschrift: blaue Tinte, deutsche Kurrent
\newline{}Schnitzler: 1) mit Bleistift das Jahr »903« vermerkt  2) mit rotem Buntstift eine Unterstreichung}\toendnotes[C]{\smallbreak}
\pstart
           \raggedleft{}{\pb}\textcolor{gray}{\textbf{DESSAUERSTRASSE 19\oindex{Dessauer Straße@\textbf{Dessauer Straße}, \emph{Straße}|pw}}}\pend
           
\pstart
           Berlin\oindex{Berlin@\textbf{Berlin}, \emph{Hauptstadt}|pw}, 19. Juli.\pend
           
\pstart\center{}Mein lieber Freund,\pend\vspace{0.5em}
\pstart
           Ich war in Frankfurt\oindex{Frankfurt am Main@\textbf{Frankfurt am Main}, \emph{Hauptstadt}|pw}, ich habe \label{K_L03377-1v}\edtext{ſie\pwindex{Rottenberg, Theodore 7.\,9.\,1875 – 5.\,4.\,1945 Limburg an der Lahn@\textsc{Rottenberg, Theodore} (7.\,9.\,1875 – 5.\,4.\,1945 Limburg an der Lahn)|pwv}}{\lemma{\textnormal{\emph{sie}}}\Cendnote{\textnormal{Theodore Rottenberg\pwindex{Rottenberg, Theodore 7.\,9.\,1875 – 5.\,4.\,1945 Limburg an der Lahn@\textsc{Rottenberg, Theodore} (7.\,9.\,1875 – 5.\,4.\,1945 Limburg an der Lahn)|pwk}, die das seit 1899 andauernde Verhältnis mit Goldmann\pwindex{Goldmann, Paul 31.\,1.\,1865 Breslau – 25.\,9.\,1935 Wien@\textsc{Goldmann, Paul} (31.\,1.\,1865 Breslau – 25.\,9.\,1935 Wien), \emph{Schriftsteller, Journalist}|pwk} Anfang 1903 beendet
                  hatte (vgl. XXXX Auszeichnungsfehler: Dokument L03360 nicht gefunden).}}}\label{K_L03377-1}
               wiedergeſehen, und ich weiß jetzt: daß dieſe Frau\pwindex{Rottenberg, Theodore 7.\,9.\,1875 – 5.\,4.\,1945 Limburg an der Lahn@\textsc{Rottenberg, Theodore} (7.\,9.\,1875 – 5.\,4.\,1945 Limburg an der Lahn)|pwv} (trotz Allem) rein und wahr und ein Engel von Güte iſt.
               Ich war Jahre lang ein blinder Thor und ich habe mein Glück mit Füßen von mir
               geſtoßen. Sie liebt mich nicht mehr, weil die Verachtung die Liebe in ihr ertödtet
               hat. Aber{ }ſie hat den Wunſch, mich wieder lieben zu können. Wenn ich in Frankfurt\oindex{Frankfurt am Main@\textbf{Frankfurt am Main}, \emph{Hauptstadt}|pw} lebte, könnte ich{ }ſie vielleicht
               wiedergewinnen. Die Entfernung verurtheilt mich zur Ohnmacht. Aber ich habe \strikeout{ih} ihr geſagt, daß mein Leben jetzt ihr gehört; und{ }ſie
               hat dieſe Gabe angenommen, ohne{ }ſich einſtweilen jedoch ihrerſeits zu {\pb}binden. Das Alles kann ich Dir nur mündlich
               erklären. Zum Schreiben fehlt mir die Zeit und die Kraft.\pend
           
\pstart
           Meine Sommerpläne hängen von ihr\pwindex{Rottenberg, Theodore 7.\,9.\,1875 – 5.\,4.\,1945 Limburg an der Lahn@\textsc{Rottenberg, Theodore} (7.\,9.\,1875 – 5.\,4.\,1945 Limburg an der Lahn)|pwv} ab. Es iſt nämlich eine, allerdings{ }ſehr{ }ſchwache Möglichkeit, daß{ }ſie
                  \label{K_L03377-2v}\edtext{mit mir auf 14 Tage nach Südtirol\oindex{Südtirol@\textbf{Südtirol}, \emph{Verwaltungsgebiet}|pw}}{\lemma{\textnormal{\emph{mit … Südtirol}}}\Cendnote{\textnormal{Rottenberg\pwindex{Rottenberg, Theodore 7.\,9.\,1875 – 5.\,4.\,1945 Limburg an der Lahn@\textsc{Rottenberg, Theodore} (7.\,9.\,1875 – 5.\,4.\,1945 Limburg an der Lahn)|pwk} kam mit, vgl. XXXX Auszeichnungsfehler: Dokument L03375 nicht gefunden. }}}\label{K_L03377-2} kommt. Weißt
               Du einen{ }ſchönen, kühlen, \uline{billigen} Ort, abſeits von
               der Touriſten-Heerſtraße? \textsc{Welsberg\oindex{Welsberg-Taisten@\textbf{Welsberg-Taisten}, \emph{Verwaltungsgebiet}|pw}} iſt ausgeſchloſſen, weil dort \label{K_L03377-3v}\edtext{Berlin\oindex{Berlin@\textbf{Berlin}, \emph{Hauptstadt}|pw}er Bekannte}{\lemma{\textnormal{\emph{Berliner Bekannte}}}\Cendnote{\textnormal{Rottenberg\pwindex{Rottenberg, Theodore 7.\,9.\,1875 – 5.\,4.\,1945 Limburg an der Lahn@\textsc{Rottenberg, Theodore} (7.\,9.\,1875 – 5.\,4.\,1945 Limburg an der Lahn)|pwk} war verheiratet, die Beziehung
                  also nicht so, dass man sich in der Öffentlichkeit gemeinsam zeigen konnte.}}}\label{K_L03377-3}
               von mir{ }ſind. Wenn die Reiſe zuſtandekommt, wirſt Du, wie ich hoffe, es einrichten
               können, mit uns \label{K_L03377-4v}\edtext{zuſammenzutreffen}{\lemma{\textnormal{\emph{zusammenzutreffen}}}\Cendnote{\textnormal{Siehe XXXX Auszeichnungsfehler: Dokument L03375 nicht gefunden.
               }}}\label{K_L03377-4}. Aber, wie geſagt, das liegt Alles noch{ }ſehr im Nebel.\pend
           
\pstart
           {\pb}Jedenfalls gib’ mir einen Rath, wo man{ }ſich
               wiedertreffen könnte. Iſt \textsc{Eppan\oindex{Eppan an der Weinstraße@\textbf{Eppan an der Weinstraße}, \emph{Verwaltungsgebiet}|pw}}{ }ſchön, wo \label{K_L03377-5v}\edtext{\textsc{Richard\pwindex{Beer-Hofmann, Richard 11.\,7.\,1866 Wien – 26.\,9.\,1945 New York City@\textsc{Beer-Hofmann, Richard} (11.\,7.\,1866 Wien – 26.\,9.\,1945 New York City), \emph{Schriftsteller}|pw}}}{\lemma{\textnormal{\emph{Richard}}}\Cendnote{\textnormal{Beer-Hofmann\pwindex{Beer-Hofmann, Richard 11.\,7.\,1866 Wien – 26.\,9.\,1945 New York City@\textsc{Beer-Hofmann, Richard} (11.\,7.\,1866 Wien – 26.\,9.\,1945 New York City), \emph{Schriftsteller}|pwk} war im Herbst 1899 in Eppan\oindex{Eppan an der Weinstraße@\textbf{Eppan an der Weinstraße}, \emph{Verwaltungsgebiet}|pwk} gewesen,
                     vgl. XXXX Auszeichnungsfehler: Dokument L00985 nicht gefunden.}}}\label{K_L03377-5} war?\pend
           
\pstart
           Grüße mir \textsc{Olga\pwindex{Schnitzler, Olga 17.\,1.\,1882 Wien – 13.\,1.\,1970 Lugano@\textsc{Schnitzler, Olga} (17.\,1.\,1882 Wien – 13.\,1.\,1970 Lugano), \emph{Schauspielerin, Sängerin}|pw}} (seid \strikeout{\textcolor{gray}{×}\-\textcolor{gray}{×}} Ihr nun \label{K_L03377-6v}\edtext{verheirathet}{\lemma{\textnormal{\emph{verheirathet}}}\Cendnote{\textnormal{Arhtur Schnitzler und Olga Gussmann\pwindex{Schnitzler, Olga 17.\,1.\,1882 Wien – 13.\,1.\,1970 Lugano@\textsc{Schnitzler, Olga} (17.\,1.\,1882 Wien – 13.\,1.\,1970 Lugano), \emph{Schauspielerin, Sängerin}|pwk} heirateten am 26. 8. 1903.}}}\label{K_L03377-6}
               oder nicht?) und{ }ſei{ }ſelbſt tauſendmal gegrüßt von {\\[\baselineskip]}Deinem getreuen {\\[\baselineskip]}\spacefill\mbox{Paul Goldmann}\pend
           \leftskip=0em{}
\pstart
           \noindent{}Dank für \label{K_L03377-7v}\edtext{\textsc{Riemer\pwindex{Riemer, Friedrich Wilhelm 19.\,4.\,1774 Kłodzko – 19.\,12.\,1845 Weimar@\textsc{Riemer, Friedrich Wilhelm} (19.\,4.\,1774 Kłodzko – 19.\,12.\,1845 Weimar), \emph{Schriftsteller, Philologe, Bibliothekar}|pw}}}{\lemma{\textnormal{\emph{Riemer}}}\Cendnote{\textnormal{Obwohl kein Titel genannt wurde,
                     dürfte es sich um dessen Hauptwerk \emph{Mittheilungen über Goethe. Aus mündlichen und schriftlichen, gedruckten und
                        ungedruckten Quellen}\pwindex{Riemer, Friedrich Wilhelm 19.\,4.\,1774 Kłodzko – 19.\,12.\,1845 Weimar@\textsc{Riemer, Friedrich Wilhelm} (19.\,4.\,1774 Kłodzko – 19.\,12.\,1845 Weimar), \emph{Schriftsteller, Philologe, Bibliothekar}!Mittheilungen über Goethe. Aus mündlichen und schriftlichen, gedruckten und ungedruckten Quellen@\strich\emph{Mittheilungen über Goethe. Aus mündlichen und schriftlichen, gedruckten und ungedruckten Quellen}|pwk} aus dem Jahr 1843
                     gehandelt haben. }}}\label{K_L03377-7}!\pend
           
\pstart
           Lies: \label{K_L03377-8v}\edtext{\textsc{Kipling\pwindex{Kipling, Rudyard 30.\,12.\,1865 Mumbai – 18.\,1.\,1936 London@\textsc{Kipling, Rudyard} (30.\,12.\,1865 Mumbai – 18.\,1.\,1936 London), \emph{Schriftsteller}|pw}}, Das Mädchen von \textsc{Birma}\pwindex{Kipling, Rudyard 30.\,12.\,1865 Mumbai – 18.\,1.\,1936 London@\textsc{Kipling, Rudyard} (30.\,12.\,1865 Mumbai – 18.\,1.\,1936 London), \emph{Schriftsteller}!Mädchen aus Birma@\strich\emph{Das Mädchen aus Birma}|pw}}{\lemma{\textnormal{\emph{Kipling, … Birma}}}\Cendnote{\textnormal{\emph{Das Mädchen aus Birma}\pwindex{Kipling, Rudyard 30.\,12.\,1865 Mumbai – 18.\,1.\,1936 London@\textsc{Kipling, Rudyard} (30.\,12.\,1865 Mumbai – 18.\,1.\,1936 London), \emph{Schriftsteller}!Mädchen aus Birma@\strich\emph{Das Mädchen aus Birma}|pwk} ist enthalten in: Rudyard Kipling\pwindex{Kipling, Rudyard 30.\,12.\,1865 Mumbai – 18.\,1.\,1936 London@\textsc{Kipling, Rudyard} (30.\,12.\,1865 Mumbai – 18.\,1.\,1936 London), \emph{Schriftsteller}|pwk}: \emph{Das Mädchen aus Birma und andere Geschichten}\pwindex{Kipling, Rudyard 30.\,12.\,1865 Mumbai – 18.\,1.\,1936 London@\textsc{Kipling, Rudyard} (30.\,12.\,1865 Mumbai – 18.\,1.\,1936 London), \emph{Schriftsteller}!Mädchen aus Birma und andere Geschichten@\strich\emph{Das Mädchen aus Birma und andere Geschichten}|pwk}.
                        Autorisierte Übersetzung aus dem Englischen. Umschlag von Berthold Löffler\pwindex{Löffler, Bertold 28.\,9.\,1874 Liberec – 23.\,3.\,1960 Wien@\textsc{Löffler, Bertold} (28.\,9.\,1874 Liberec – 23.\,3.\,1960 Wien), \emph{Maler, Grafiker}|pwk}. Wien\oindex{Wien@\textbf{Wien}, \emph{Verwaltungsgebiet}|pwk}/Leipzig\oindex{Leipzig@\textbf{Leipzig}, \emph{Hauptstadt}|pwk}: \emph{Wiener Verlag}\orgindex{Wiener Verlag@Wiener Verlag|pwk}{ }1903 (\emph{Bibliothek
                           berühmter Autoren}\pwindex{Bibliothek berühmter Autoren@\emph{Bibliothek berühmter Autoren}|pwk} 8). Eine Lektüre durch Schnitzler ist nicht bekannt.}}}\label{K_L03377-8}.\pend
           \selectlanguage{ngerman}\endnumbering\briefempfaengerindex{Schnitzler, Arthur@\textsc{Schnitzler, Arthur}!zzzGoldmann, Paul@\emph{von Paul Goldmann}!1903-07-192@{19. 7. [1903]}|)be}\mylabel{L03377h}  \newcommand{\dateiname}{L03377}\newcommand{\titel}{Paul Goldmann an Arthur Schnitzler, 19. 7. [1903]}\newcommand{\editorInnen}{Martin Anton Müller und Laura Untner}%% latex-leseansicht-abspann.tex
%% Abspann für die Leseansicht.
%% Der Schalter \ifkorrekturansicht ist bereits durch den Vorspann gesetzt.

%% latex-abspann.tex
%% Gemeinsamer Abspann für Korrekturansicht und Leseansicht.
%% Setzt den Schalter \ifkorrekturansicht voraus (gesetzt in den
%% einbindenden Dateien latex-korrekturansicht-abspann.tex bzw.
%% latex-leseansicht-abspann.tex).
%% ---------------------------------------------------------------

\normalsize

% Das esempio-Environment wird nur in der Leseansicht benötigt
\ifkorrekturansicht\else
\newenvironment{esempio}[3]%
{
    \vspace{1.5ex}
    \rlap{\underline{#1}}
    \par
    \setlength{\parindent}{0cm}
    \nopagebreak
    \leftskip=#2cm
    \rightskip=#3cm
}
{
    \par
}
\fi

\doendnotes{C}
\bigskip
\vfill

\clearpage

\footnotesize

\ifkorrekturansicht
  \lohead{\textsc{register}}
\fi

% theindex-Environment neu definieren ohne reledmac
\makeatletter
\renewenvironment{theindex}{%
  \ifkorrekturansicht
    \section*{\indexname}%
  \else
    \subsubsection*{Index der erwähnten Entitäten}%
  \fi
  \setlength{\parindent}{0pt}%
  \setlength{\parskip}{0pt plus 0.3pt}%
  \let\item\@idxitem
}{%
  \ifkorrekturansicht\clearpage\fi
}
\makeatother

\IfFileExists{\jobname-pw.ind}{\input{\jobname-pw.ind}}{}

% Quellenangabe nur in der Leseansicht
\ifkorrekturansicht\else
% Fallback-Definitionen, falls die .tex-Datei \titel etc. nicht gesetzt hat
\providecommand{\titel}{}
\providecommand{\editorInnen}{}
\providecommand{\dateiname}{\jobname}

\vspace{3cm}

\vfill

\footnotesize
\textsc{Quelle}: \titel. Herausgegeben von {\editorInnen}. In: \emph{Arthur Schnitzler: Briefwechsel mit Autorinnen und Autoren}.
 Digitale Edition, https://schnitzler-briefe.acdh.oeaw.ac.at/{\dateiname}.html (Stand \today)
\fi

\end{document}


