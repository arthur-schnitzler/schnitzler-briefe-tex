%% latex-leseansicht-vorspann.tex
%% Vorspann für die Leseansicht.
%% Lädt die gemeinsame Datei latex-vorspann.tex mit nicht gesetztem Schalter.

\newif\ifkorrekturansicht
\korrekturansichtfalse

\input{../tex-inputs/latex-vorspann}


\section[Paul Goldmann an Arthur Schnitzler, 21. 3. {[}1895{]}]{L02731 Paul Goldmann an Arthur Schnitzler, 21. 3. [1895]}
\nopagebreak\mylabel{L02731v}
\rehead{ }\normalsize\beginnumbering\briefempfaengerindex{Schnitzler, Arthur@\textsc{Schnitzler, Arthur}!zzzGoldmann, Paul@\emph{von Paul Goldmann}!1895-03-212@{21. 3. [1895]}|(be}
\toendnotes[C]{\smallbreak\pagebreak[2]}
\correspDesc{Versand  durch Paul Goldmann am 21. 3. [1895] in Paris
\newline{}Erhalt  durch Arthur Schnitzler im Zeitraum [22. 3. 1895
                  – 26. 3. 1895?] in Wien}\toendnotes[C]{\smallbreak}
\Standort{DLA, A:Schnitzler, HS.NZ85.1.3165.}
\physDesc{Brief, 1 Blatt, 2 Seiten, 728 Zeichen
\newline{}Handschrift: schwarze Tinte, deutsche Kurrent
\newline{}Schnitzler: 1) mit Bleistift das Jahr »95« vermerkt  2) mit rotem Buntstift eine Unterstreichung
\newline{}Editorischer Hinweis: Die Wiedergabe des Zeitungsausschnitts\pwindex{Lalo, Pierre 6.\,9.\,1866 Puteaux – 9.\,6.\,1943 Paris@\textsc{Lalo, Pierre} (6.\,9.\,1866 Puteaux – 9.\,6.\,1943 Paris), \emph{Kritiker}!Au jour le jour. M. Arthur Schnitzler@\strich\emph{Au jour le jour. M. Arthur Schnitzler}|pwv} stellt eine
                                 Rekonstruktion dar. Auf dem originalen Brief ist nur eine Klebespur
                                 mit der Rückseite des Texts\pwindex{Lalo, Pierre 6.\,9.\,1866 Puteaux – 9.\,6.\,1943 Paris@\textsc{Lalo, Pierre} (6.\,9.\,1866 Puteaux – 9.\,6.\,1943 Paris), \emph{Kritiker}!Au jour le jour. M. Arthur Schnitzler@\strich\emph{Au jour le jour. M. Arthur Schnitzler}|pwv} von Pierre Lalo\pwindex{Lalo, Pierre 6.\,9.\,1866 Puteaux – 9.\,6.\,1943 Paris@\textsc{Lalo, Pierre} (6.\,9.\,1866 Puteaux – 9.\,6.\,1943 Paris), \emph{Kritiker}|pw} überliefert, die belegt, dass hier
                                 ursprünglich die Besprechung\pwindex{Lalo, Pierre 6.\,9.\,1866 Puteaux – 9.\,6.\,1943 Paris@\textsc{Lalo, Pierre} (6.\,9.\,1866 Puteaux – 9.\,6.\,1943 Paris), \emph{Kritiker}!Au jour le jour. M. Arthur Schnitzler@\strich\emph{Au jour le jour. M. Arthur Schnitzler}|pwv} angebracht war. }\toendnotes[C]{\smallbreak}
\pstart
           {\pb}\textcolor{gray}{\textbf{\textbf{Frankfurter Zeitung\orgindex{Frankfurter Zeitung@Frankfurter Zeitung|pw}}}}\pend
           
\pstart
           \textcolor{gray}{\textbf{(\begin{otherlanguage}{french}Gazette de Francfort\end{otherlanguage}\orgindex{Frankfurter Zeitung@Frankfurter Zeitung|pw}).}}\pend
           
\pstart
           \textcolor{gray}{\textbf{\textbf{\begin{otherlanguage}{french}Fondateur M. L.
                              Sonnemann\pwindex{Sonnemann, Leopold 29.\,10.\,1831 Höchberg – 30.\,10.\,1909 Frankfurt am Main@\textsc{Sonnemann, Leopold} (29.\,10.\,1831 Höchberg – 30.\,10.\,1909 Frankfurt am Main), \emph{Journalist, Herausgeber}|pw}\end{otherlanguage}.}}}\pend
           
\pstart
           \begin{otherlanguage}{french}\textcolor{gray}{\textbf{Journal politique, financier,}}\end{otherlanguage}\pend
           
\pstart
           \begin{otherlanguage}{french}\textcolor{gray}{\textbf{commercial et littéraire.}}\end{otherlanguage}\pend
           
\pstart
           \begin{otherlanguage}{french}\textcolor{gray}{\textbf{\textbf{Paraissant trois fois par jour.}}}\end{otherlanguage}\pend
           
\pstart
           \begin{otherlanguage}{french}\textcolor{gray}{\textbf{\textbf{Bureau à Paris\oindex{Paris@\textbf{Paris}, \emph{Hauptstadt}|pw}:}}}\end{otherlanguage}\pend
           
\pstart
           \begin{otherlanguage}{french}\textcolor{gray}{\textbf{\textbf{24. Rue Feydeau\oindex{rue Feydeau@\textbf{rue Feydeau}, \emph{Straße}|pw}.}}}\end{otherlanguage}\pend
           \vspace{0.5em}
\pstart
           \centering{}\textcolor{gray}{\textbf{\label{K_L02731-1v}\edtext{\begin{otherlanguage}{french}AU JOUR LE JOUR\end{otherlanguage}}{\lemma{\textnormal{\emph{Au Jour le Jour}}}\Cendnote{\textnormal{P. L.\pwindex{Lalo, Pierre 6.\,9.\,1866 Puteaux – 9.\,6.\,1943 Paris@\textsc{Lalo, Pierre} (6.\,9.\,1866 Puteaux – 9.\,6.\,1943 Paris), \emph{Kritiker}|pwkv} [ = Pierre Lalo\pwindex{Lalo, Pierre 6.\,9.\,1866 Puteaux – 9.\,6.\,1943 Paris@\textsc{Lalo, Pierre} (6.\,9.\,1866 Puteaux – 9.\,6.\,1943 Paris), \emph{Kritiker}|pwk}]: \emph{Au jour le jour. M. Arthur Schnitzler}\pwindex{Lalo, Pierre 6.\,9.\,1866 Puteaux – 9.\,6.\,1943 Paris@\textsc{Lalo, Pierre} (6.\,9.\,1866 Puteaux – 9.\,6.\,1943 Paris), \emph{Kritiker}!Au jour le jour. M. Arthur Schnitzler@\strich\emph{Au jour le jour. M. Arthur Schnitzler}|pwk}. In: \emph{Journal des débats}\pwindex{Journal des débats. Politiques et littéraires@\emph{Journal des débats. Politiques et littéraires}|pwk}, Jg. 107,
                           21. 3. 1895, S. 1. Deutsche
                     Übersetzung: »VON TAG ZU TAG{ / }ARTHUR SCHNITZLER{ / }Arthur Schnitzler ist einer der
                           jüngsten Zugänge zu den Schriftstellern des jungen Deutschlands\oindex{Deutschland@\textbf{Deutschland}|pw}. Bisher kannte man von ihm eine Sammlung von
                              Erzählungen\pwindex{Schnitzler, Arthur 15.\,5.\,1862 Wien – 21.\,10.\,1931 ebd.@\textsc{Schnitzler, Arthur} (15.\,5.\,1862 Wien – 21.\,10.\,1931 ebd.), \emph{Schriftsteller, Mediziner}!Anatol@\strich\emph{Anatol}|pwv} und ein dreiaktiges Stück\pwindex{Schnitzler, Arthur 15.\,5.\,1862 Wien – 21.\,10.\,1931 ebd.@\textsc{Schnitzler, Arthur} (15.\,5.\,1862 Wien – 21.\,10.\,1931 ebd.), \emph{Schriftsteller, Mediziner}!Märchen. Schauspiel in drei Aufzügen@\strich\emph{Das Märchen. Schauspiel in drei Aufzügen}|pwv}, in denen sich seine
                           besonderen Qualitäten zeigten. Sie hoben ihn aber noch nicht hervor, das
                           geschah erst durch die kürzlich erfolgte Veröffentlichtung eines Romans\pwindex{Schnitzler, Arthur 15.\,5.\,1862 Wien – 21.\,10.\,1931 ebd.@\textsc{Schnitzler, Arthur} (15.\,5.\,1862 Wien – 21.\,10.\,1931 ebd.), \emph{Schriftsteller, Mediziner}!Sterben. Novelle@\strich\emph{Sterben. Novelle}|pwv} mit dem
                           Titel \emph{Sterben}\pwindex{Schnitzler, Arthur 15.\,5.\,1862 Wien – 21.\,10.\,1931 ebd.@\textsc{Schnitzler, Arthur} (15.\,5.\,1862 Wien – 21.\,10.\,1931 ebd.), \emph{Schriftsteller, Mediziner}!Sterben. Novelle@\strich\emph{Sterben. Novelle}|pw} in der \emph{Neuen Deutschen Rundschau}\pwindex{Neue Deutsche Rundschau@\emph{Neue Deutsche Rundschau}|pw}. Der Erfolg war sehr groß, und es scheint, dass er in jeder
                           Hinsicht verdient ist. \emph{Sterben}\pwindex{Schnitzler, Arthur 15.\,5.\,1862 Wien – 21.\,10.\,1931 ebd.@\textsc{Schnitzler, Arthur} (15.\,5.\,1862 Wien – 21.\,10.\,1931 ebd.), \emph{Schriftsteller, Mediziner}!Sterben. Novelle@\strich\emph{Sterben. Novelle}|pw} ist ein sehr kurzer Roman oder, wenn man so will, eine lange
                           Novelle: kaum 150 Seiten. Es gibt nur drei Personen: einen jungen Mann
                           und eine junge Frau, Felix\pwindex{Schnitzler, Arthur 15.\,5.\,1862 Wien – 21.\,10.\,1931 ebd.@\textsc{Schnitzler, Arthur} (15.\,5.\,1862 Wien – 21.\,10.\,1931 ebd.), \emph{Schriftsteller, Mediziner}!Sterben. Novelle@\strich\emph{Sterben. Novelle}|pwv} und Marie\pwindex{Schnitzler, Arthur 15.\,5.\,1862 Wien – 21.\,10.\,1931 ebd.@\textsc{Schnitzler, Arthur} (15.\,5.\,1862 Wien – 21.\,10.\,1931 ebd.), \emph{Schriftsteller, Mediziner}!Sterben. Novelle@\strich\emph{Sterben. Novelle}|pwv}, die zärtlich miteinander verbunden sind, und einen Arzt.
                           In der ersten Szene\pwindex{Schnitzler, Arthur 15.\,5.\,1862 Wien – 21.\,10.\,1931 ebd.@\textsc{Schnitzler, Arthur} (15.\,5.\,1862 Wien – 21.\,10.\,1931 ebd.), \emph{Schriftsteller, Mediziner}!Sterben. Novelle@\strich\emph{Sterben. Novelle}|pwv}, die durch die sichere Darstellung und die Wahl der
                           Details beeindruckt, erfährt Felix\pwindex{Schnitzler, Arthur 15.\,5.\,1862 Wien – 21.\,10.\,1931 ebd.@\textsc{Schnitzler, Arthur} (15.\,5.\,1862 Wien – 21.\,10.\,1931 ebd.), \emph{Schriftsteller, Mediziner}!Sterben. Novelle@\strich\emph{Sterben. Novelle}|pwv}, dass er unheilbar krank ist und nur noch
                           ein Jahr zu leben hat. Er teilt dies Marie\pwindex{Schnitzler, Arthur 15.\,5.\,1862 Wien – 21.\,10.\,1931 ebd.@\textsc{Schnitzler, Arthur} (15.\,5.\,1862 Wien – 21.\,10.\,1931 ebd.), \emph{Schriftsteller, Mediziner}!Sterben. Novelle@\strich\emph{Sterben. Novelle}|pwv} mit, und sie ruft verzweifelt, dass sie mit
                           ihrem Freund sterben werde. Er versucht sie zu besänftigen und ihr
                           klarzumachen, dass sie leben muss und noch glücklich sein kann, aber sie
                           will nicht hören{\dots} Auf den letzten Seiten des
                              Romans\pwindex{Schnitzler, Arthur 15.\,5.\,1862 Wien – 21.\,10.\,1931 ebd.@\textsc{Schnitzler, Arthur} (15.\,5.\,1862 Wien – 21.\,10.\,1931 ebd.), \emph{Schriftsteller, Mediziner}!Sterben. Novelle@\strich\emph{Sterben. Novelle}|pwv}, in den
                           letzten Tagen von Felix\pwindex{Schnitzler, Arthur 15.\,5.\,1862 Wien – 21.\,10.\,1931 ebd.@\textsc{Schnitzler, Arthur} (15.\,5.\,1862 Wien – 21.\,10.\,1931 ebd.), \emph{Schriftsteller, Mediziner}!Sterben. Novelle@\strich\emph{Sterben. Novelle}|pwv}’ Krankheit, ist er es, der sie leidenschaftlich mit in den
                           Tod nehmen möchte, und sie ist es, die leben möchte. Dieser langsame
                           Zerfall von Gefühlen und Zuneigung ist das Thema von \emph{Sterben}\pwindex{Schnitzler, Arthur 15.\,5.\,1862 Wien – 21.\,10.\,1931 ebd.@\textsc{Schnitzler, Arthur} (15.\,5.\,1862 Wien – 21.\,10.\,1931 ebd.), \emph{Schriftsteller, Mediziner}!Sterben. Novelle@\strich\emph{Sterben. Novelle}|pw}. Stellen Sie sich vor, dieses Thema würde von einem unserer
                           Romanautoren behandelt: Er würde zweifellos dazu neigen, die moralische
                           Hässlichkeit seiner Figuren zu übertreiben. Bei Schnitzler gibt es nichts dergleichen: keine
                           Exzesse, keine Gewalt, keine Brutalität; die Darstellung, so stark sie
                           auch sein mag, behält perfekt Maß und Genauigkeit. Was in Marie\pwindex{Schnitzler, Arthur 15.\,5.\,1862 Wien – 21.\,10.\,1931 ebd.@\textsc{Schnitzler, Arthur} (15.\,5.\,1862 Wien – 21.\,10.\,1931 ebd.), \emph{Schriftsteller, Mediziner}!Sterben. Novelle@\strich\emph{Sterben. Novelle}|pwv} vorgeht, was an
                           unbewusster Ungeduld und Überdruss unter ihrer Zärtlichkeit und ihrem
                           Mitleid erwacht und sich einschleicht, all das wird tiefgehend beobachtet
                           und akzentuiert mit seltener Präzision{\dots} Wenn
                           ich noch hinzufüge, dass die Entwicklung der Erzählung\pwindex{Schnitzler, Arthur 15.\,5.\,1862 Wien – 21.\,10.\,1931 ebd.@\textsc{Schnitzler, Arthur} (15.\,5.\,1862 Wien – 21.\,10.\,1931 ebd.), \emph{Schriftsteller, Mediziner}!Sterben. Novelle@\strich\emph{Sterben. Novelle}|pwv} kurz und nüchtern
                           ist, dass die Komposition nahezu klassisch ist in Befolgung von Logik,
                           Reihung und Klarheit aufweist, habe ich genug gesagt, um den Erfolg von
                              \emph{Sterben}\pwindex{Schnitzler, Arthur 15.\,5.\,1862 Wien – 21.\,10.\,1931 ebd.@\textsc{Schnitzler, Arthur} (15.\,5.\,1862 Wien – 21.\,10.\,1931 ebd.), \emph{Schriftsteller, Mediziner}!Sterben. Novelle@\strich\emph{Sterben. Novelle}|pw} zu erklären und zu zeigen, dass die deutsche Literatur von nun an
                           viel von Herrn Schnitzler erhoffen
                           darf. – P. L.«}}}\label{K_L02731-1}}}\pend
           
\pstart
           \centering{}\textcolor{gray}{\textbf{M. ARTHUR SCHNITZLER}}\pend
           
\pstart
           \textcolor{gray}{\textbf{\begin{otherlanguage}{french}M. Arthur Schnitzler est un des derniers venus parmi les
                     écrivains de la Jeune Allemagne\oindex{Deutschland@\textbf{Deutschland}|pw}. On
                     connaissait jusqu’ici de lui un recueil de nouvelles\pwindex{Schnitzler, Arthur 15.\,5.\,1862 Wien – 21.\,10.\,1931 ebd.@\textsc{Schnitzler, Arthur} (15.\,5.\,1862 Wien – 21.\,10.\,1931 ebd.), \emph{Schriftsteller, Mediziner}!Anatol@\strich\emph{Anatol}|pwv} et une pièce en trois actes\pwindex{Schnitzler, Arthur 15.\,5.\,1862 Wien – 21.\,10.\,1931 ebd.@\textsc{Schnitzler, Arthur} (15.\,5.\,1862 Wien – 21.\,10.\,1931 ebd.), \emph{Schriftsteller, Mediziner}!Märchen. Schauspiel in drei Aufzügen@\strich\emph{Das Märchen. Schauspiel in drei Aufzügen}|pwv}, où se
                     révélaient des qualités éminentes, mais qui ne l’avaient point encore fait
                     sortir du rang, lorsque, récemment, il publia dans la \emph{Neue Deutsche Rundschau}\pwindex{Neue Deutsche Rundschau@\emph{Neue Deutsche Rundschau}|pw} un roman\pwindex{Schnitzler, Arthur 15.\,5.\,1862 Wien – 21.\,10.\,1931 ebd.@\textsc{Schnitzler, Arthur} (15.\,5.\,1862 Wien – 21.\,10.\,1931 ebd.), \emph{Schriftsteller, Mediziner}!Sterben. Novelle@\strich\emph{Sterben. Novelle}|pwv} intitulé:
                        \emph{Sterben}\pwindex{Schnitzler, Arthur 15.\,5.\,1862 Wien – 21.\,10.\,1931 ebd.@\textsc{Schnitzler, Arthur} (15.\,5.\,1862 Wien – 21.\,10.\,1931 ebd.), \emph{Schriftsteller, Mediziner}!Sterben. Novelle@\strich\emph{Sterben. Novelle}|pw} – \emph{Mourir}\pwindex{Schnitzler, Arthur 15.\,5.\,1862 Wien – 21.\,10.\,1931 ebd.@\textsc{Schnitzler, Arthur} (15.\,5.\,1862 Wien – 21.\,10.\,1931 ebd.), \emph{Schriftsteller, Mediziner}!Sterben. Novelle@\strich\emph{Sterben. Novelle}|pw}. Le succès en fut très vif; il semble bien qu’il soit de tout point
                     mérité. \emph{Sterben}\pwindex{Schnitzler, Arthur 15.\,5.\,1862 Wien – 21.\,10.\,1931 ebd.@\textsc{Schnitzler, Arthur} (15.\,5.\,1862 Wien – 21.\,10.\,1931 ebd.), \emph{Schriftsteller, Mediziner}!Sterben. Novelle@\strich\emph{Sterben. Novelle}|pw} est un très court roman ou, si l’on veut, une longue nouvelle: cent
                     cinquante pages à peine. Trois personnages seulement: un jeune homme et une
                     jeune femme tendrement unis, Félix\pwindex{Schnitzler, Arthur 15.\,5.\,1862 Wien – 21.\,10.\,1931 ebd.@\textsc{Schnitzler, Arthur} (15.\,5.\,1862 Wien – 21.\,10.\,1931 ebd.), \emph{Schriftsteller, Mediziner}!Sterben. Novelle@\strich\emph{Sterben. Novelle}|pwv} et Marie\pwindex{Schnitzler, Arthur 15.\,5.\,1862 Wien – 21.\,10.\,1931 ebd.@\textsc{Schnitzler, Arthur} (15.\,5.\,1862 Wien – 21.\,10.\,1931 ebd.), \emph{Schriftsteller, Mediziner}!Sterben. Novelle@\strich\emph{Sterben. Novelle}|pwv},
                     et un médecin. En la première scène\pwindex{Schnitzler, Arthur 15.\,5.\,1862 Wien – 21.\,10.\,1931 ebd.@\textsc{Schnitzler, Arthur} (15.\,5.\,1862 Wien – 21.\,10.\,1931 ebd.), \emph{Schriftsteller, Mediziner}!Sterben. Novelle@\strich\emph{Sterben. Novelle}|pwv}, singulièrement saisissante par la sûreté des traits et le choix
                     des détails, Félix\pwindex{Schnitzler, Arthur 15.\,5.\,1862 Wien – 21.\,10.\,1931 ebd.@\textsc{Schnitzler, Arthur} (15.\,5.\,1862 Wien – 21.\,10.\,1931 ebd.), \emph{Schriftsteller, Mediziner}!Sterben. Novelle@\strich\emph{Sterben. Novelle}|pwv} vient
                     d’apprendre qu’il est atteint d’une maladie incurable et qu’il n’a pas plus
                     d’une année de vie: il l’annonce à Marie\pwindex{Schnitzler, Arthur 15.\,5.\,1862 Wien – 21.\,10.\,1931 ebd.@\textsc{Schnitzler, Arthur} (15.\,5.\,1862 Wien – 21.\,10.\,1931 ebd.), \emph{Schriftsteller, Mediziner}!Sterben. Novelle@\strich\emph{Sterben. Novelle}|pwv}, et celle-ci, désespérée, s’écrie qu’elle mourra
                     avec son ami. Il s’efforce de l’apaiser, de lui faire comprendre qu’elle doit
                     vivre et qu’elle pourra encore être heureuse: elle ne veut rien entendre{\dots} Aux dernières pages du roman\pwindex{Schnitzler, Arthur 15.\,5.\,1862 Wien – 21.\,10.\,1931 ebd.@\textsc{Schnitzler, Arthur} (15.\,5.\,1862 Wien – 21.\,10.\,1931 ebd.), \emph{Schriftsteller, Mediziner}!Sterben. Novelle@\strich\emph{Sterben. Novelle}|pwv}, aux derniers jours de la
                     maladie de Félix\pwindex{Schnitzler, Arthur 15.\,5.\,1862 Wien – 21.\,10.\,1931 ebd.@\textsc{Schnitzler, Arthur} (15.\,5.\,1862 Wien – 21.\,10.\,1931 ebd.), \emph{Schriftsteller, Mediziner}!Sterben. Novelle@\strich\emph{Sterben. Novelle}|pwv}, c’est
                     lui qui désirera passionnément l’emmener avec lui dans la mort, c’est elle qui
                     voudra vivre. Cette lente décomposition des sentiments et des affections, tel
                     est le sujet de \emph{Sterben}\pwindex{Schnitzler, Arthur 15.\,5.\,1862 Wien – 21.\,10.\,1931 ebd.@\textsc{Schnitzler, Arthur} (15.\,5.\,1862 Wien – 21.\,10.\,1931 ebd.), \emph{Schriftsteller, Mediziner}!Sterben. Novelle@\strich\emph{Sterben. Novelle}|pw}. Imaginez ce thème traité par un de nos romanciers: sans doute il sera
                     porté à exagérer la laideur morale de ses personnages. Rien de pareil chez M.
                     Schnitzler: aucun excès, aucune violence, aucune brutalité; la peinture, si
                     forte qu’elle soit, garde une mesure et une justesse parfaites. Ce qui se passe
                     chez Marie\pwindex{Schnitzler, Arthur 15.\,5.\,1862 Wien – 21.\,10.\,1931 ebd.@\textsc{Schnitzler, Arthur} (15.\,5.\,1862 Wien – 21.\,10.\,1931 ebd.), \emph{Schriftsteller, Mediziner}!Sterben. Novelle@\strich\emph{Sterben. Novelle}|pwv}, ce qui
                     s’éveille et se glisse d’inconsciente impatience et de lassitude sous sa
                     tendresse et sa pitié, tout cela est profondément observé, nuancé avec une rare
                        précision{\dots} Si j’ajoute que les développements du
                        récit\pwindex{Schnitzler, Arthur 15.\,5.\,1862 Wien – 21.\,10.\,1931 ebd.@\textsc{Schnitzler, Arthur} (15.\,5.\,1862 Wien – 21.\,10.\,1931 ebd.), \emph{Schriftsteller, Mediziner}!Sterben. Novelle@\strich\emph{Sterben. Novelle}|pwv} sont brefs et
                     sobres, que la composition a une logique, une suite et une clarté presque
                     classiques, j’en aurai assez dit pour expliquer le succès de \emph{Sterben}\pwindex{Schnitzler, Arthur 15.\,5.\,1862 Wien – 21.\,10.\,1931 ebd.@\textsc{Schnitzler, Arthur} (15.\,5.\,1862 Wien – 21.\,10.\,1931 ebd.), \emph{Schriftsteller, Mediziner}!Sterben. Novelle@\strich\emph{Sterben. Novelle}|pw} et pour montrer que les lettres allemandes ont désormais le droit
                     d’attendre beaucoup de M. Schnitzler. – P. L.\end{otherlanguage}}}\pend
           \selectlanguage{ngerman}\vspace{1em}
\pstart
           \raggedleft{}{\pb}\textsc{Paris\oindex{Paris@\textbf{Paris}, \emph{Hauptstadt}|pw}}, 21. März.\pend
           
\pstart\center{}Mein lieber Freund,\pend\vspace{0.5em}
\pstart
           \textsc{Pierre Lalo\pwindex{Lalo, Pierre 6.\,9.\,1866 Puteaux – 9.\,6.\,1943 Paris@\textsc{Lalo, Pierre} (6.\,9.\,1866 Puteaux – 9.\,6.\,1943 Paris), \emph{Kritiker}|pw}} hat alſo endlich{ }ſein \label{K_L02731-2v}\edtext{Verſprechen}{\lemma{\textnormal{\emph{Versprechen}}}\Cendnote{\textnormal{Siehe XXXX Auszeichnungsfehler: Dokument L02727 nicht gefunden.
               }}}\label{K_L02731-2} gehalten und hat einen{ }ſchönen Artikel\pwindex{Lalo, Pierre 6.\,9.\,1866 Puteaux – 9.\,6.\,1943 Paris@\textsc{Lalo, Pierre} (6.\,9.\,1866 Puteaux – 9.\,6.\,1943 Paris), \emph{Kritiker}!Au jour le jour. M. Arthur Schnitzler@\strich\emph{Au jour le jour. M. Arthur Schnitzler}|pwv} geſchrieben. Das heißt, die Schönheit des Artikels\pwindex{Lalo, Pierre 6.\,9.\,1866 Puteaux – 9.\,6.\,1943 Paris@\textsc{Lalo, Pierre} (6.\,9.\,1866 Puteaux – 9.\,6.\,1943 Paris), \emph{Kritiker}!Au jour le jour. M. Arthur Schnitzler@\strich\emph{Au jour le jour. M. Arthur Schnitzler}|pwv} hat natürlich nichts
               mit dem Verſprechen zu thun,{ }ſondern mit der Schönheit Deines Buch\pwindex{Schnitzler, Arthur 15.\,5.\,1862 Wien – 21.\,10.\,1931 ebd.@\textsc{Schnitzler, Arthur} (15.\,5.\,1862 Wien – 21.\,10.\,1931 ebd.), \emph{Schriftsteller, Mediziner}!Sterben. Novelle@\strich\emph{Sterben. Novelle}|pwv}es, die den fran\oindex{Frankreich@\textbf{Frankreich}|pwv}zöſiſchen Kritiker\pwindex{Lalo, Pierre 6.\,9.\,1866 Puteaux – 9.\,6.\,1943 Paris@\textsc{Lalo, Pierre} (6.\,9.\,1866 Puteaux – 9.\,6.\,1943 Paris), \emph{Kritiker}|pwv} hocherfreut hat. Ich beglückwünſche
               Dich zu dem neuen Erfolge und bin recht{ }ſtolz darauf, Dich in dem ernſteſten und
               vornehmſten Blatte\pwindex{Journal des débats. Politiques et littéraires@\emph{Journal des débats. Politiques et littéraires}|pwv} der großen
                  Pariſ\oindex{Paris@\textbf{Paris}, \emph{Hauptstadt}|pw}er Tagespreſſe an erſter Stelle in{ }ſolcher Weiſe beſprochen zu{ }ſehen.\pend
           
\pstart
           {\pb}Anbei erhältſt Du einige Exemplare\pwindex{Journal des débats. Politiques et littéraires@\emph{Journal des débats. Politiques et littéraires}|pwv}. Bitte{ }ſchreibe \uline{umgehend} und recht herzlich an \textsc{Lalo\pwindex{Lalo, Pierre 6.\,9.\,1866 Puteaux – 9.\,6.\,1943 Paris@\textsc{Lalo, Pierre} (6.\,9.\,1866 Puteaux – 9.\,6.\,1943 Paris), \emph{Kritiker}|pw}} (\textsc{19. Boulevard de Courcelles}\oindex{Boulevard de Courcelles@\textbf{Boulevard de Courcelles}, \emph{Straße}|pw}).\pend
           
\pstart
           In Treue {\\[\baselineskip]}Dein {\\[\baselineskip]}\spacefill\mbox{Paul Goldmann.}\pend
           \leftskip=0em{}
\pstart
           \noindent{}Bitte,{ }ſchick’ mir bei Gelegenheit ein Exemplar von »\textsc{Alkandis} Lied\pwindex{Schnitzler, Arthur 15.\,5.\,1862 Wien – 21.\,10.\,1931 ebd.@\textsc{Schnitzler, Arthur} (15.\,5.\,1862 Wien – 21.\,10.\,1931 ebd.), \emph{Schriftsteller, Mediziner}!Alkandi’s Lied@\strich\emph{Alkandi’s Lied}|pw}«. Zu Progaganda-Zwecken!\pend
           \selectlanguage{ngerman}\endnumbering\briefempfaengerindex{Schnitzler, Arthur@\textsc{Schnitzler, Arthur}!zzzGoldmann, Paul@\emph{von Paul Goldmann}!1895-03-212@{21. 3. [1895]}|)be}\mylabel{L02731h}  \newcommand{\dateiname}{L02731}\newcommand{\titel}{Paul Goldmann an Arthur Schnitzler, 21. 3. [1895]}\newcommand{\editorInnen}{Martin Anton Müller und Laura Untner}%% latex-leseansicht-abspann.tex
%% Abspann für die Leseansicht.
%% Der Schalter \ifkorrekturansicht ist bereits durch den Vorspann gesetzt.

%% latex-abspann.tex
%% Gemeinsamer Abspann für Korrekturansicht und Leseansicht.
%% Setzt den Schalter \ifkorrekturansicht voraus (gesetzt in den
%% einbindenden Dateien latex-korrekturansicht-abspann.tex bzw.
%% latex-leseansicht-abspann.tex).
%% ---------------------------------------------------------------

\normalsize

% Das esempio-Environment wird nur in der Leseansicht benötigt
\ifkorrekturansicht\else
\newenvironment{esempio}[3]%
{
    \vspace{1.5ex}
    \rlap{\underline{#1}}
    \par
    \setlength{\parindent}{0cm}
    \nopagebreak
    \leftskip=#2cm
    \rightskip=#3cm
}
{
    \par
}
\fi

\doendnotes{C}
\bigskip
\vfill

\clearpage

\footnotesize

\ifkorrekturansicht
  \lohead{\textsc{register}}
\fi

% theindex-Environment neu definieren ohne reledmac
\makeatletter
\renewenvironment{theindex}{%
  \ifkorrekturansicht
    \section*{\indexname}%
  \else
    \subsubsection*{Index der erwähnten Entitäten}%
  \fi
  \setlength{\parindent}{0pt}%
  \setlength{\parskip}{0pt plus 0.3pt}%
  \let\item\@idxitem
}{%
  \ifkorrekturansicht\clearpage\fi
}
\makeatother

\IfFileExists{\jobname-pw.ind}{\input{\jobname-pw.ind}}{}

% Quellenangabe nur in der Leseansicht
\ifkorrekturansicht\else
% Fallback-Definitionen, falls die .tex-Datei \titel etc. nicht gesetzt hat
\providecommand{\titel}{}
\providecommand{\editorInnen}{}
\providecommand{\dateiname}{\jobname}

\vspace{3cm}

\vfill

\footnotesize
\textsc{Quelle}: \titel. Herausgegeben von {\editorInnen}. In: \emph{Arthur Schnitzler: Briefwechsel mit Autorinnen und Autoren}.
 Digitale Edition, https://schnitzler-briefe.acdh.oeaw.ac.at/{\dateiname}.html (Stand \today)
\fi

\end{document}


