%% latex-leseansicht-vorspann.tex
%% Vorspann für die Leseansicht.
%% Lädt die gemeinsame Datei latex-vorspann.tex mit nicht gesetztem Schalter.

\newif\ifkorrekturansicht
\korrekturansichtfalse

\input{../tex-inputs/latex-vorspann}


\section[Hugo von Hofmannsthal an Arthur Schnitzler, 23. 2. 1892]{L00075 Hugo von Hofmannsthal an Arthur Schnitzler, 23. 2. 1892}
\nopagebreak\mylabel{L00075v}
\rehead{ }\normalsize\beginnumbering\briefempfaengerindex{Schnitzler, Arthur@\textsc{Schnitzler, Arthur}!zzzHofmannsthal, Hugo von@\emph{von Hugo von Hofmannsthal}!1892-02-231@{23. 2. 1892}|(be}
\toendnotes[C]{\smallbreak\pagebreak[2]}
\correspDesc{Versand  durch Hugo von Hofmannsthal am 23. 2. 1892 in Wien
\newline{}Erhalt  durch Arthur Schnitzler am 24. 2. 1892 in Wien}\toendnotes[C]{\smallbreak}
\Standort{CUL, Schnitzler, B 43.}
\physDesc{Postkarte, 278 Zeichen
\newline{}Handschrift: Bleistift, deutsche Kurrent
\newline{}Versand: 1) Stempel: »\nobreak{}\oindex{Wien@\textbf{Wien}, \emph{Verwaltungsgebiet}|pwk}Wien 3/3 40, 24. 2. 92, 7–8V\nobreak{}«.   2) Stempel: »\nobreak{}\oindex{Wien@\textbf{Wien}, \emph{Verwaltungsgebiet}|pwk}Wien, 24. 2. 92, 10½–12V\nobreak{}«. 
\newline{}Schnitzler: mit Bleistift auf der Anschriftenseite: »24/2 92« und auf der Textseite datiert: »2\strikeout{4}3. 2. 92« 
\newline{}Ordnung: mit Bleistift von unbekannter Hand nummeriert: »18« }
\buchAbdrucke{\weitereDrucke{Hugo von Hofmannsthal, Arthur Schnitzler: \emph{Briefwechsel}. Herausgegeben von Therese Nickl und Heinrich Schnitzler. Frankfurt am Main: \emph{S. Fischer} 1964, S. 16.} }\toendnotes[C]{\smallbreak}\pstart{}{\pb}Herrn \textsc{D\textsuperscript{r} Arthur Schnitzler}\pend{}\pstart{}\textsc{Wien\oindex{Wien@\textbf{Wien}, \emph{Verwaltungsgebiet}|pw}}\pend{}\pstart{}\textsc{I Kärnthner\strikeout{strasse}ring 12\oindex{Wien@\textbf{Wien}!I., Innere Stadt@\textbf{I., Innere Stadt}!Kärntnerring 12/Bösendorferstraße 11@\textbf{Kärntnerring 12/Bösendorferstraße 11}, \emph{Wohngebäude}|pw}}\pend{}{\bigskip}\vspace{1em}
\pstart
           \raggedleft{}{\pb}\label{K_L00075-1v}\edtext{Dienstag}{\lemma{\textnormal{\emph{Dienstag}}}\Cendnote{\textnormal{Hofmannsthal\pwindex{Hofmannsthal, Hugo von 1.\,2.\,1874 Wien – 15.\,7.\,1929 Rodaun@\textsc{Hofmannsthal, Hugo von} (1.\,2.\,1874 Wien – 15.\,7.\,1929 Rodaun), \emph{Schriftsteller}|pwk} schrieb die Karte
                     unmittelbar nach dem Besuch von \emph{Feodora}\pwindex{\textcolor{red}{\textsuperscript{XXXX indx1}}!Fédora@\strich\emph{Fédora}|pwk},
                     dem zweiten Auftritt von Eleonora Duse\pwindex{Duse, Eleonora 3.\,10.\,1858 Vigevano – 21.\,4.\,1924 Pittsburgh@\textsc{Duse, Eleonora} (3.\,10.\,1858 Vigevano – 21.\,4.\,1924 Pittsburgh), \emph{Schauspielerin}|pwk}
                     bei ihrem ersten Wien\oindex{Wien@\textbf{Wien}, \emph{Verwaltungsgebiet}|pwk}er Gastspiel. Entgegen
                     seiner Ankündigung, auch noch \emph{Fernande}\pwindex{\textcolor{red}{\textsuperscript{XXXX indx1}}!Fernanda. Commedia in 4 atti@\strich\emph{Fernanda. Commedia in 4 atti}|pwk}
                     sehen zu wollen, wurden bis zum 26. 2. 1892 nur \emph{Nora oder Ein Puppenheim}\pwindex{\textcolor{red}{\textsuperscript{XXXX indx1}}!Nora oder ein Puppenheim. Schauspiel in drei Akten@\strich\emph{Nora oder ein Puppenheim. Schauspiel in drei Akten}|pwk} und die \emph{Kameliendame}\pwindex{\textcolor{red}{\textsuperscript{XXXX indx1}}!Dame aux camélias (théâtre)@\strich\emph{La Dame aux camélias (théâtre)}|pwk} gegeben. Schnitzler erlebte sie erst zwei Monate später, bei ihrem
                     zweiten Gastspiel: am 17. 5. 1892 und 24. 5. 1892 sah
                     er \emph{Nora}\pwindex{\textcolor{red}{\textsuperscript{XXXX indx1}}!Nora oder ein Puppenheim. Schauspiel in drei Akten@\strich\emph{Nora oder ein Puppenheim. Schauspiel in drei Akten}|pwk} und \emph{Fernande}\pwindex{\textcolor{red}{\textsuperscript{XXXX indx1}}!Fernanda. Commedia in 4 atti@\strich\emph{Fernanda. Commedia in 4 atti}|pwk}. (\emph{Cambridge University Library}, A 179a.)}}}\label{K_L00075-1}{ }11 Uhr nachts\pend
           \vspace{0.5em}
\pstart
           Wenn Sie{ }ſich die \textsc{Duse}\pwindex{Duse, Eleonora 3.\,10.\,1858 Vigevano – 21.\,4.\,1924 Pittsburgh@\textsc{Duse, Eleonora} (3.\,10.\,1858 Vigevano – 21.\,4.\,1924 Pittsburgh), \emph{Schauspielerin}|pw} nicht anſehen, wenn auch auf der letzten Gallerie und{ }ſtehend, verſäumen Sie
               mehr, als Sie{ }ſich vorſtellen können.\pend
           \pstart \spacefill\mbox{Loris.}\pend{}
\pstart
           \noindent{}Ich gehe zu \textsc{Nora}\pwindex{\textcolor{red}{\textsuperscript{XXXX indx1}}!Nora oder ein Puppenheim. Schauspiel in drei Akten@\strich\emph{Nora oder ein Puppenheim. Schauspiel in drei Akten}|pw} und \textsc{Fernande}\pwindex{\textcolor{red}{\textsuperscript{XXXX indx1}}!Fernanda. Commedia in 4 atti@\strich\emph{Fernanda. Commedia in 4 atti}|pw}\pend
           
\pstart
           Alles andere iſt jetzt gleichgiltig.\pend
           \selectlanguage{ngerman}\endnumbering\briefempfaengerindex{Schnitzler, Arthur@\textsc{Schnitzler, Arthur}!zzzHofmannsthal, Hugo von@\emph{von Hugo von Hofmannsthal}!1892-02-231@{23. 2. 1892}|)be}\mylabel{L00075h}  \newcommand{\dateiname}{L00075}\newcommand{\titel}{Hugo von Hofmannsthal an Arthur Schnitzler, 23. 2. 1892}\newcommand{\editorInnen}{Martin Anton Müller und Gerd-Hermann Susen}%% latex-leseansicht-abspann.tex
%% Abspann für die Leseansicht.
%% Der Schalter \ifkorrekturansicht ist bereits durch den Vorspann gesetzt.

%% latex-abspann.tex
%% Gemeinsamer Abspann für Korrekturansicht und Leseansicht.
%% Setzt den Schalter \ifkorrekturansicht voraus (gesetzt in den
%% einbindenden Dateien latex-korrekturansicht-abspann.tex bzw.
%% latex-leseansicht-abspann.tex).
%% ---------------------------------------------------------------

\normalsize

% Das esempio-Environment wird nur in der Leseansicht benötigt
\ifkorrekturansicht\else
\newenvironment{esempio}[3]%
{
    \vspace{1.5ex}
    \rlap{\underline{#1}}
    \par
    \setlength{\parindent}{0cm}
    \nopagebreak
    \leftskip=#2cm
    \rightskip=#3cm
}
{
    \par
}
\fi

\doendnotes{C}
\bigskip
\vfill

\clearpage

\footnotesize

\ifkorrekturansicht
  \lohead{\textsc{register}}
\fi

% theindex-Environment neu definieren ohne reledmac
\makeatletter
\renewenvironment{theindex}{%
  \ifkorrekturansicht
    \section*{\indexname}%
  \else
    \subsubsection*{Index der erwähnten Entitäten}%
  \fi
  \setlength{\parindent}{0pt}%
  \setlength{\parskip}{0pt plus 0.3pt}%
  \let\item\@idxitem
}{%
  \ifkorrekturansicht\clearpage\fi
}
\makeatother

\IfFileExists{\jobname-pw.ind}{\input{\jobname-pw.ind}}{}

% Quellenangabe nur in der Leseansicht
\ifkorrekturansicht\else
% Fallback-Definitionen, falls die .tex-Datei \titel etc. nicht gesetzt hat
\providecommand{\titel}{}
\providecommand{\editorInnen}{}
\providecommand{\dateiname}{\jobname}

\vspace{3cm}

\vfill

\footnotesize
\textsc{Quelle}: \titel. Herausgegeben von {\editorInnen}. In: \emph{Arthur Schnitzler: Briefwechsel mit Autorinnen und Autoren}.
 Digitale Edition, https://schnitzler-briefe.acdh.oeaw.ac.at/{\dateiname}.html (Stand \today)
\fi

\end{document}


