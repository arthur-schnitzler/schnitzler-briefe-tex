%% latex-leseansicht-vorspann.tex
%% Vorspann für die Leseansicht.
%% Lädt die gemeinsame Datei latex-vorspann.tex mit nicht gesetztem Schalter.

\newif\ifkorrekturansicht
\korrekturansichtfalse

\input{../tex-inputs/latex-vorspann}


               \section[Hugo von Hofmannsthal an Arthur Schnitzler, 23. 2. 1892]{ Hugo von Hofmannsthal an Arthur Schnitzler, 23. 2. 1892}\nopagebreak\mylabel{v}\rehead{ }\begin{ledgroupsized}[t]{13cm}\normalsize\beginnumbering\briefempfaengerindex{Schnitzler, Arthur@\textsc{Schnitzler, Arthur}!zzzHofmannsthal, Hugo von@\emph{von Hugo von Hofmannsthal}!1892-02-231@{23. 2. 1892}|(be} \toendnotes[C]{\smallbreak\pagebreak[2]} \Standort{CUL, Schnitzler, B 43.}
\physDesc{Postkarte
\newline{}Handschrift: Bleistift, deutsche Kurrent\newline{}Versand: 1) Stempel: »\nobreak{}Wien 3/3 40, 24. 2. 92, 7–8V\nobreak{}«.  2) Stempel: »\nobreak{}Wien, 24. 2. 92, 10½–12V\nobreak{}«. 
\newline{}Schnitzler: mit Bleistift auf der Anschriftenseite: »24/2 92« und auf der Textseite datiert: »2\strikeout{4}3. 2. 92« \newline{}Ordnung: von unbekannter Hand nummeriert: »18« }\buchAbdrucke{\weitereDrucke{Hugo von Hofmannsthal, Arthur Schnitzler: \emph{Briefwechsel}. Hg. Therese Nickl und Heinrich Schnitzler. Frankfurt am Main: \emph{S. Fischer} 1964, S. 16.} }\toendnotes[C]{\smallbreak}\pstart{}{\pb}Herrn \textsc{D\textsuperscript{r} Arthur Schnitzler}\pend{}\pstart{}\textsc{Wien\oindex{Wien@\textbf{Wien}|pw}}\pend{}\pstart{}\textsc{I Kärnthner\strikeout{strasse}ring 12\oindex{Kaerntnerring@\textbf{Kärntnerring}|pw}}\pend{}{\bigskip}\pstart
           \raggedleft{}{\pb}\label{K_L00075_1v}\edtext{Dienstag}{\lemma{\textnormal{\emph{Dienstag}}}\Cendnote{\textnormal{Hofmannsthal\pwindex{Hofmannsthal, Hugo von 01.02.1874 – 15.07.1929@\textsc{Hofmannsthal, Hugo von} (01.02.1874 – 15.07.1929), \emph{Schriftsteller}|pwk} schrieb die Karte unmittelbar
                     nach dem Besuch von \emph{Feodora}\pwindex{\textcolor{red}{\textsuperscript{XXXX1 indx}}!Fedora1882@\strich\emph{Fédora} {[}1882{]}|pwk}, dem zweiten
                     Auftritt von Eleonora Duse\pwindex{Duse, Eleonora 03.10.1858 – 21.04.1924@\textsc{Duse, Eleonora} (03.10.1858 – 21.04.1924), \emph{Schauspielerin}|pwk} bei ihrem ersten
                        Wien\oindex{Wien@\textbf{Wien}|pwk}er Gastspiel. Entgegen seiner
                     Ankündigung, auch noch \emph{Fernande}\pwindex{\textcolor{red}{\textsuperscript{XXXX1 indx}}!Fernande1844@\strich\emph{Fernande} {[}1844{]}|pwk} sehen zu
                     wollen, wurden bis zum 26. 2. 1892 nur \emph{Nora
                        oder Ein Puppenheim}\pwindex{\textcolor{red}{\textsuperscript{XXXX1 indx}}!Nora oder ein Puppenheim1879@\strich\emph{Nora oder ein Puppenheim} {[}1879{]}|pwk} und die \emph{Kameliendame}\pwindex{\textcolor{red}{\textsuperscript{XXXX1 indx}}!Kameliendame. Drama in  fuenf Akten1852@\strich\emph{Die Kameliendame. Drama in fünf Akten} {[}1852{]}|pwk} gegeben. Schnitzler\pwindex{Schnitzler, Arthur 15.05.1862 – 21.10.1931@\textsc{Schnitzler, Arthur} (15.05.1862 – 21.10.1931), \emph{Schriftsteller, Mediziner}|pwk}
                     erlebte sie erst zwei Monate später, bei ihrem zweiten Gastspiel: am
                        17. 5. 1892 und 24. 5. 1892 sah er \emph{Nora}\pwindex{\textcolor{red}{\textsuperscript{XXXX1 indx}}!Nora oder ein Puppenheim1879@\strich\emph{Nora oder ein Puppenheim} {[}1879{]}|pwk} und \emph{Fernande}\pwindex{\textcolor{red}{\textsuperscript{XXXX1 indx}}!Fernande1844@\strich\emph{Fernande} {[}1844{]}|pwk}. (\emph{Cambridge University Library}, A 179a).}}}\label{K_L00075_1h}{ }11 Uhr nachts\pend
           \pstart
           Wenn Sie ſich die \textsc{Duse}\pwindex{Duse, Eleonora 03.10.1858 – 21.04.1924@\textsc{Duse, Eleonora} (03.10.1858 – 21.04.1924), \emph{Schauspielerin}|pw} nicht anſehen, wenn auch auf der letzten Gallerie und ſtehend, verſäumen Sie
               mehr, als Sie ſich vorſtellen können.\pend
           \pstart \spacefill\mbox{Loris.}\pend{}\pstart
           \noindent{}Ich gehe zu \textsc{Nora}\pwindex{\textcolor{red}{\textsuperscript{XXXX1 indx}}!Nora oder ein Puppenheim1879@\strich\emph{Nora oder ein Puppenheim} {[}1879{]}|pw} und \textsc{Fernande}\pwindex{\textcolor{red}{\textsuperscript{XXXX1 indx}}!Fernande1844@\strich\emph{Fernande} {[}1844{]}|pw}\pend
           \pstart
           Alles andere iſt jetzt gleichgiltig.\pend
           \endnumbering\briefempfaengerindex{Schnitzler, Arthur@\textsc{Schnitzler, Arthur}!zzzHofmannsthal, Hugo von@\emph{von Hugo von Hofmannsthal}!1892-02-231@{23. 2. 1892}|)be}\mylabel{h}\end{ledgroupsized}  \newcommand{\dateiname}{L00075}\newcommand{\titel}{Hugo von Hofmannsthal an Arthur Schnitzler, 23. 2. 1892}\newcommand{\editorInnen}{Martin Anton Müller und Gerd-Hermann Susen}
            \footnotesize
\begin{ledgroupsized}[t]{11.5cm}
\doendnotes{C}
\end{ledgroupsized}
         %% latex-leseansicht-abspann.tex
%% Abspann für die Leseansicht.
%% Der Schalter \ifkorrekturansicht ist bereits durch den Vorspann gesetzt.

%% latex-abspann.tex
%% Gemeinsamer Abspann für Korrekturansicht und Leseansicht.
%% Setzt den Schalter \ifkorrekturansicht voraus (gesetzt in den
%% einbindenden Dateien latex-korrekturansicht-abspann.tex bzw.
%% latex-leseansicht-abspann.tex).
%% ---------------------------------------------------------------

\normalsize

% Das esempio-Environment wird nur in der Leseansicht benötigt
\ifkorrekturansicht\else
\newenvironment{esempio}[3]%
{
    \vspace{1.5ex}
    \rlap{\underline{#1}}
    \par
    \setlength{\parindent}{0cm}
    \nopagebreak
    \leftskip=#2cm
    \rightskip=#3cm
}
{
    \par
}
\fi

\doendnotes{C}
\bigskip
\vfill

\clearpage

\footnotesize

\ifkorrekturansicht
  \lohead{\textsc{register}}
\fi

% theindex-Environment neu definieren ohne reledmac
\makeatletter
\renewenvironment{theindex}{%
  \ifkorrekturansicht
    \section*{\indexname}%
  \else
    \subsubsection*{Index der erwähnten Entitäten}%
  \fi
  \setlength{\parindent}{0pt}%
  \setlength{\parskip}{0pt plus 0.3pt}%
  \let\item\@idxitem
}{%
  \ifkorrekturansicht\clearpage\fi
}
\makeatother

\IfFileExists{\jobname-pw.ind}{\input{\jobname-pw.ind}}{}

% Quellenangabe nur in der Leseansicht
\ifkorrekturansicht\else
% Fallback-Definitionen, falls die .tex-Datei \titel etc. nicht gesetzt hat
\providecommand{\titel}{}
\providecommand{\editorInnen}{}
\providecommand{\dateiname}{\jobname}

\vspace{3cm}

\vfill

\footnotesize
\textsc{Quelle}: \titel. Herausgegeben von {\editorInnen}. In: \emph{Arthur Schnitzler: Briefwechsel mit Autorinnen und Autoren}.
 Digitale Edition, https://schnitzler-briefe.acdh.oeaw.ac.at/{\dateiname}.html (Stand \today)
\fi

\end{document}


      