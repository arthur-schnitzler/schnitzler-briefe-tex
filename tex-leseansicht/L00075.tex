%% latex-korrekturansicht-vorspann.tex
%% Vorspann für die Korrekturansicht.
%% Lädt die gemeinsame Datei latex-vorspann.tex mit gesetztem Schalter.

\newif\ifkorrekturansicht
\korrekturansichttrue

\input{../tex-inputs/latex-vorspann}


\section[Hugo von Hofmannsthal an Arthur Schnitzler, 23. 2. 1892]{L00075 Hugo von Hofmannsthal an Arthur Schnitzler, 23. 2. 1892}
\nopagebreak\mylabel{L00075v}
\rehead{ }\normalsize\beginnumbering\briefempfaengerindex{Schnitzler, Arthur@\textsc{Schnitzler, Arthur}!zzzHofmannsthal, Hugo von@\emph{von Hugo von Hofmannsthal}!1892-02-231@{23. 2. 1892}|(be}
\toendnotes[C]{\smallbreak\pagebreak[2]}\Standort{CUL, Schnitzler, B 43.}
\physDesc{Postkarte, 278 Zeichen
\newline{}Handschrift: Bleistift, deutsche Kurrent
\newline{}Versand: 1) Stempel: »\nobreak{}Wien 3/3 40, 24. 2. 92, 7–8V\nobreak{}«.   2) Stempel: »\nobreak{}Wien, 24. 2. 92, 10½–12V\nobreak{}«. 
\newline{}Schnitzler: mit Bleistift auf der Anschriftenseite: »24/2 92« und auf der Textseite datiert: »2\strikeout{4}3. 2. 92« 
\newline{}Ordnung: mit Bleistift von unbekannter Hand nummeriert: »18« }
\buchAbdrucke{\weitereDrucke{Hugo von Hofmannsthal, Arthur Schnitzler: \emph{Briefwechsel}. Frankfurt am Main: \emph{S. Fischer} 1964, S. 16.} }\toendnotes[C]{\smallbreak}\pstart{}{\pb}Herrn \textsc{D\textsuperscript{r} Arthur Schnitzler}\pend{}\pstart{}\textsc{Wien\oindex{Wien@\textbf{Wien}, \emph{A.ADM2}|pw}}\pend{}\pstart{}\textsc{I Kärnthner\strikeout{strasse}ring 12\oindex{Kaerntnerring 12/Boesendorferstrasse 11@\textbf{Kärntnerring 12/Bösendorferstraße 11}, \emph{Wohngebäude (K.WHS)}|pw}}\pend{}{\bigskip}\vspace{1em}
\pstart
           \raggedleft{}{\pb}\label{K_L00075-1v}\edtext{Dienstag}{\lemma{\textnormal{\emph{Dienstag}}}\Cendnote{\textnormal{Hofmannsthal\pwindex{Hofmannsthal, Hugo von 1874-02-01 – 1929-07-15@\textsc{Hofmannsthal, Hugo von} (1874-02-01 – 1929-07-15), \emph{Schriftsteller/Schriftstellerin}|pwk} schrieb die Karte
                     unmittelbar nach dem Besuch von \emph{Feodora}\pwindex{Fedora@\emph{Fédora}|pwk},
                     dem zweiten Auftritt von Eleonora Duse\pwindex{Duse, Eleonora 03.10.1858 – 21.04.1924@\textsc{Duse, Eleonora} (03.10.1858 – 21.04.1924), \emph{Schauspieler/Schauspielerin}|pwk}
                     bei ihrem ersten Wien\oindex{Wien@\textbf{Wien}, \emph{A.ADM2}|pwk}er Gastspiel. Entgegen
                     seiner Ankündigung, auch noch \emph{Fernande}\pwindex{Fernanda. Commedia in 4 atti@\emph{Fernanda. Commedia in 4 atti}|pwk}
                     sehen zu wollen, wurden bis zum 26. 2. 1892 nur \emph{Nora oder Ein Puppenheim}\pwindex{Nora oder ein Puppenheim. Schauspiel in drei Akten@\emph{Nora oder ein Puppenheim. Schauspiel in drei Akten}|pwk} und die \emph{Kameliendame}\pwindex{Dame aux camelias (theâtre)@\emph{La Dame aux camélias (théâtre)}|pwk} gegeben. Schnitzler erlebte sie erst zwei Monate später, bei ihrem
                     zweiten Gastspiel: am 17. 5. 1892 und 24. 5. 1892 sah
                     er \emph{Nora}\pwindex{Nora oder ein Puppenheim. Schauspiel in drei Akten@\emph{Nora oder ein Puppenheim. Schauspiel in drei Akten}|pwk} und \emph{Fernande}\pwindex{Fernanda. Commedia in 4 atti@\emph{Fernanda. Commedia in 4 atti}|pwk}. (\emph{Cambridge University Library}, A 179a.)}}}\label{K_L00075-1}{ }11 Uhr nachts\pend
           \vspace{0.5em}
\pstart
           Wenn Sie ſich die \textsc{Duse}\pwindex{Duse, Eleonora 03.10.1858 – 21.04.1924@\textsc{Duse, Eleonora} (03.10.1858 – 21.04.1924), \emph{Schauspieler/Schauspielerin}|pw} nicht anſehen, wenn auch auf der letzten Gallerie und ſtehend, verſäumen Sie
               mehr, als Sie ſich vorſtellen können.\pend
           \pstart \spacefill\mbox{Loris.}\pend{}
\pstart
           \noindent{}Ich gehe zu \textsc{Nora}\pwindex{Nora oder ein Puppenheim. Schauspiel in drei Akten@\emph{Nora oder ein Puppenheim. Schauspiel in drei Akten}|pw} und \textsc{Fernande}\pwindex{Fernanda. Commedia in 4 atti@\emph{Fernanda. Commedia in 4 atti}|pw}\pend
           
\pstart
           Alles andere iſt jetzt gleichgiltig.\pend
           \selectlanguage{ngerman}\endnumbering\briefempfaengerindex{Schnitzler, Arthur@\textsc{Schnitzler, Arthur}!zzzHofmannsthal, Hugo von@\emph{von Hugo von Hofmannsthal}!1892-02-231@{23. 2. 1892}|)be}\mylabel{L00075h}  \normalsize

\doendnotes{C}
\bigskip
\vfill

\clearpage

\footnotesize

\lohead{\textsc{register}}

% Definiere theindex-Environment komplett neu ohne reledmac
\makeatletter
\renewenvironment{theindex}{%
  \section*{\indexname}%
  \setlength{\parindent}{0pt}%
  \setlength{\parskip}{0pt plus 0.3pt}%
  \let\item\@idxitem
}{%
  \clearpage
}
\makeatother

\IfFileExists{\jobname-pw.ind}{\input{\jobname-pw.ind}}{}

\end{document}

      