%% latex-leseansicht-vorspann.tex
%% Vorspann für die Leseansicht.
%% Lädt die gemeinsame Datei latex-vorspann.tex mit nicht gesetztem Schalter.

\newif\ifkorrekturansicht
\korrekturansichtfalse

\input{../tex-inputs/latex-vorspann}


         \renewcommand{\erwaehnteOrte}{Orte: Wien, Zistersdorf}
         \renewcommand{\erwaehnteWerke}{
               \section[Robert Adam an Arthur Schnitzler, 15. 5. 1913]{ Robert Adam an Arthur Schnitzler, 15. 5. 1913}\nopagebreak\mylabel{v}\rehead{ }\begin{ledgroupsized}[t]{13cm}\normalsize\beginnumbering \toendnotes[C]{\smallbreak\pagebreak[2]} \Standort{DLA, A:Schnitzler, HS.NZ85.1.4230,6.}
\physDesc{Briefkarte
\newline{}Handschrift: schwarze Tinte, deutsche Kurrent
\newline{}Schnitzler: mit Bleistift beschriftet: »\textsc{Adam}« }\Standort{Wien, Österreichische Nationalbibliothek, Cod.ser. 52.266, 168.}
\physDesc{handschriftliche Abschrift
\newline{}Handschrift: schwarze Tinte, Gabelsberger Kurzschrift}\Standort{Wien, Österreichische Nationalbibliothek, Cod.ser. 52.266, 168.}
\physDesc{maschinelle Abschrift
\newline{}Schreibmaschine}\pstart
           \raggedleft{}{\pb}Ziſtersdorf\oindex{Zistersdorf@\textbf{Zistersdorf}|pw}, am 15. Mai 1913\pend
           \pstart{}Hochverehrter Herr Doktor!\pend\pstart
           Nehmen Sie meinen herzlichſten Dank für Ihren liebenswürdigen Brief!\pend
           \pstart
           Sie haben mir damit allen bitteren Kleinmut, der mich befallen hatte, vom Herzen
                    genommen und mich ermuntert, fröhlich weiterzuarbeiten, an meinen poetiſchen
                        Ver{\pb}ſuchen und dadurch an mir ſelbſt,
                    unbekümmert um äußeren Erfolg, nur dem Drang meiner Eingebungen und Ideen
                    folgend, um ſchließlich einmal das große reine Kunſtwerk ſchaffen zu können, das
                    mir vorſchwebt.\pend
           \pstart
           Ich danke Ihnen aus vollem Herzen!\pend
           \pstart
           Ihr ſehr ergebener{\\[\baselineskip]}\spacefill\mbox{Robert Adam}\pend
           \leftskip=0em{}
         
         \endnumbering\mylabel{h}\end{ledgroupsized}  \newcommand{\dateiname}{L02138}\newcommand{\titel}{Robert Adam an Arthur Schnitzler, 15. 5. 1913}\newcommand{\editorInnen}{Martin Anton Müller und Gerd-Hermann Susen}%% latex-leseansicht-abspann.tex
%% Abspann für die Leseansicht.
%% Der Schalter \ifkorrekturansicht ist bereits durch den Vorspann gesetzt.

%% latex-abspann.tex
%% Gemeinsamer Abspann für Korrekturansicht und Leseansicht.
%% Setzt den Schalter \ifkorrekturansicht voraus (gesetzt in den
%% einbindenden Dateien latex-korrekturansicht-abspann.tex bzw.
%% latex-leseansicht-abspann.tex).
%% ---------------------------------------------------------------

\normalsize

% Das esempio-Environment wird nur in der Leseansicht benötigt
\ifkorrekturansicht\else
\newenvironment{esempio}[3]%
{
    \vspace{1.5ex}
    \rlap{\underline{#1}}
    \par
    \setlength{\parindent}{0cm}
    \nopagebreak
    \leftskip=#2cm
    \rightskip=#3cm
}
{
    \par
}
\fi

\doendnotes{C}
\bigskip
\vfill

\clearpage

\footnotesize

\ifkorrekturansicht
  \lohead{\textsc{register}}
\fi

% theindex-Environment neu definieren ohne reledmac
\makeatletter
\renewenvironment{theindex}{%
  \ifkorrekturansicht
    \section*{\indexname}%
  \else
    \subsubsection*{Index der erwähnten Entitäten}%
  \fi
  \setlength{\parindent}{0pt}%
  \setlength{\parskip}{0pt plus 0.3pt}%
  \let\item\@idxitem
}{%
  \ifkorrekturansicht\clearpage\fi
}
\makeatother

\IfFileExists{\jobname-pw.ind}{\input{\jobname-pw.ind}}{}

% Quellenangabe nur in der Leseansicht
\ifkorrekturansicht\else
% Fallback-Definitionen, falls die .tex-Datei \titel etc. nicht gesetzt hat
\providecommand{\titel}{}
\providecommand{\editorInnen}{}
\providecommand{\dateiname}{\jobname}

\vspace{3cm}

\vfill

\footnotesize
\textsc{Quelle}: \titel. Herausgegeben von {\editorInnen}. In: \emph{Arthur Schnitzler: Briefwechsel mit Autorinnen und Autoren}.
 Digitale Edition, https://schnitzler-briefe.acdh.oeaw.ac.at/{\dateiname}.html (Stand \today)
\fi

\end{document}


      