%% latex-leseansicht-vorspann.tex
%% Vorspann für die Leseansicht.
%% Lädt die gemeinsame Datei latex-vorspann.tex mit nicht gesetztem Schalter.

\newif\ifkorrekturansicht
\korrekturansichtfalse

\input{../tex-inputs/latex-vorspann}


\section[Arthur Schnitzler an Karl Emil Franzos, 11. 5. 1888]{L03616 Arthur Schnitzler an Karl Emil Franzos, 11. 5. 1888}
\nopagebreak\mylabel{L03616v}
\rehead{ }\normalsize\beginnumbering\briefempfaengerindex{Franzos, Karl Emil@\textsc{Franzos, Karl Emil}!zzzSchnitzler, Arthur@\emph{von Arthur Schnitzler}!1888-05-111@{11. 5. 1888}|(be}
\toendnotes[C]{\smallbreak\pagebreak[2]}
\correspDesc{Versand  durch Arthur Schnitzler am 11. 5. 1888 in Berlin
\newline{}Erhalt  durch Karl Emil Franzos am [11. 5. 1888?] in Berlin}\toendnotes[C]{\smallbreak}
\Standort{Wienbibliothek im Rathaus, H.I.N.-60193.}
\physDesc{Brief, 1 Blatt, 2 Seiten, 731 Zeichen
\newline{}Handschrift: schwarze Tinte, deutsche Kurrent}\toendnotes[C]{\smallbreak}
\pstart\center{}{\pb}Verehrteſter Herr Franzos!\pend\vspace{0.5em}
\pstart
           Es fügt ſich, daſs ich bereits morgen – wider mein Erwarten – von hier fortreiſen
               muſs, wodurch ich nicht mehr dazuko{\geminationm}e, Ihnen und Ihrer
               hochverehrten Frau Gemahlin\pwindex{Franzos, Ottilie 24.\,9.\,1856 Wien – 5.\,3.\,1932 ebd.@\textsc{Franzos, Ottilie} (24.\,9.\,1856 Wien – 5.\,3.\,1932 ebd.), \emph{Schriftstellerin}|pwv}
               meinen perſönlichen Dank für Ihre liebenswürdige \label{K_L03616-33v}\edtext{Gaſtfreundschaft}{\lemma{\textnormal{\emph{Gastfreundschaft}}}\Cendnote{\textnormal{Vgl. A. S.: \emph{Tagebuch}, 28. 4. 1888.
               }}}\label{K_L03616-33} auszudrücken. Ich muſs mich begnügen, dies auf dieſem
               Wege zu thun, und Sie{ }ſchriftlich bitten, meines Danks’ und meiner Hochachtung
               verſichert zu{ }ſein.\pend
           
\pstart
           Was die an Sie geſandten \label{K_L03616-1v}\edtext{\textsc{Manuscripte}}{\lemma{\textnormal{\emph{Manuscripte}}}\Cendnote{\textnormal{Siehe XXXX Auszeichnungsfehler: Dokument L03618 nicht gefunden.
                  }}}\label{K_L03616-1}\pwindex{Schnitzler, Arthur 15.\,5.\,1862 Wien – 21.\,10.\,1931 ebd.@\textsc{Schnitzler, Arthur} (15.\,5.\,1862 Wien – 21.\,10.\,1931 ebd.), \emph{Schriftsteller, Mediziner}!Amerika@\strich\emph{Amerika}|pwv}\pwindex{Schnitzler, Arthur 15.\,5.\,1862 Wien – 21.\,10.\,1931 ebd.@\textsc{Schnitzler, Arthur} (15.\,5.\,1862 Wien – 21.\,10.\,1931 ebd.), \emph{Schriftsteller, Mediziner}!Mein Freund Ypsilon. Aus den Papieren eines Arztes@\strich\emph{Mein Freund Ypsilon. Aus den Papieren eines Arztes}|pwv}\pwindex{Schnitzler, Arthur 15.\,5.\,1862 Wien – 21.\,10.\,1931 ebd.@\textsc{Schnitzler, Arthur} (15.\,5.\,1862 Wien – 21.\,10.\,1931 ebd.), \emph{Schriftsteller, Mediziner}!Erbschaft@\strich\emph{Erbschaft}|pwv} betrifft,{ }ſo würde ich um eine Antwort, {\pb}eventuelle \label{K_L03616-11v}\edtext{Rückſendung erſt nach \textsc{London}\oindex{London@\textbf{London}, \emph{Hauptstadt}|pw}}{\lemma{\textnormal{\emph{Rücksendung … London}}}\Cendnote{\textnormal{Hier ist »nach« nicht zeitlich, sondern
                  räumlich zu verstehen: Schnitzler bittet darum, dass
                  ihm die Texte nach London\oindex{London@\textbf{London}, \emph{Hauptstadt}|pwk} gesandt werden. (Er reiste nicht direkt von Berlin\oindex{Berlin@\textbf{Berlin}, \emph{Hauptstadt}|pwk}, sondern über Wien\oindex{Wien@\textbf{Wien}, \emph{Verwaltungsgebiet}|pwk}.)
                  Zu der hier noch angedachten Mitteilung der Londoner\oindex{London@\textbf{London}, \emph{Hauptstadt}|pwk} Adresse dürfte es nicht gekommen sein, was dafür spricht, 
                  dass Franzos\pwindex{Franzos, Karl Emil 25.\,10.\,1848 Tschortkiw – 28.\,1.\,1904 Berlin@\textsc{Franzos, Karl Emil} (25.\,10.\,1848 Tschortkiw – 28.\,1.\,1904 Berlin), \emph{Schriftsteller, Journalist}|pwk} unmittelbar auf dieses
                  Schreiben mit der Rücksendung reagierte, vgl. XXXX Auszeichnungsfehler: Dokument L03619 nicht gefunden.}}}\label{K_L03616-11}
               bitten, von wo aus ich{ }ſo frei{ }ſein werde, Ihnen meine Adreſſe mitzutheilen.\pend
           \pstart Indem ich mich Ihnen und Ihrer w. Frau Gemahlin\pwindex{Franzos, Ottilie 24.\,9.\,1856 Wien – 5.\,3.\,1932 ebd.@\textsc{Franzos, Ottilie} (24.\,9.\,1856 Wien – 5.\,3.\,1932 ebd.), \emph{Schriftstellerin}|pwv} ergebenſt empfehle, bin ich mit beſondrer
               Hochachtung Ihr \spacefill\mbox{Dr. Arthur Schnitzler}\pend{}
\pstart
           \textsc{Berlin\oindex{Berlin@\textbf{Berlin}, \emph{Hauptstadt}|pw}}, 11. 5. 88\pend
           \selectlanguage{ngerman}\endnumbering\briefempfaengerindex{Franzos, Karl Emil@\textsc{Franzos, Karl Emil}!zzzSchnitzler, Arthur@\emph{von Arthur Schnitzler}!1888-05-111@{11. 5. 1888}|)be}\mylabel{L03616h}  \newcommand{\dateiname}{L03616}\newcommand{\titel}{Arthur Schnitzler an Karl Emil Franzos, 11. 5. 1888}\newcommand{\editorInnen}{Selma Jahnke und Martin Anton Müller}%% latex-leseansicht-abspann.tex
%% Abspann für die Leseansicht.
%% Der Schalter \ifkorrekturansicht ist bereits durch den Vorspann gesetzt.

%% latex-abspann.tex
%% Gemeinsamer Abspann für Korrekturansicht und Leseansicht.
%% Setzt den Schalter \ifkorrekturansicht voraus (gesetzt in den
%% einbindenden Dateien latex-korrekturansicht-abspann.tex bzw.
%% latex-leseansicht-abspann.tex).
%% ---------------------------------------------------------------

\normalsize

% Das esempio-Environment wird nur in der Leseansicht benötigt
\ifkorrekturansicht\else
\newenvironment{esempio}[3]%
{
    \vspace{1.5ex}
    \rlap{\underline{#1}}
    \par
    \setlength{\parindent}{0cm}
    \nopagebreak
    \leftskip=#2cm
    \rightskip=#3cm
}
{
    \par
}
\fi

\doendnotes{C}
\bigskip
\vfill

\clearpage

\footnotesize

\ifkorrekturansicht
  \lohead{\textsc{register}}
\fi

% theindex-Environment neu definieren ohne reledmac
\makeatletter
\renewenvironment{theindex}{%
  \ifkorrekturansicht
    \section*{\indexname}%
  \else
    \subsubsection*{Index der erwähnten Entitäten}%
  \fi
  \setlength{\parindent}{0pt}%
  \setlength{\parskip}{0pt plus 0.3pt}%
  \let\item\@idxitem
}{%
  \ifkorrekturansicht\clearpage\fi
}
\makeatother

\IfFileExists{\jobname-pw.ind}{\input{\jobname-pw.ind}}{}

% Quellenangabe nur in der Leseansicht
\ifkorrekturansicht\else
% Fallback-Definitionen, falls die .tex-Datei \titel etc. nicht gesetzt hat
\providecommand{\titel}{}
\providecommand{\editorInnen}{}
\providecommand{\dateiname}{\jobname}

\vspace{3cm}

\vfill

\footnotesize
\textsc{Quelle}: \titel. Herausgegeben von {\editorInnen}. In: \emph{Arthur Schnitzler: Briefwechsel mit Autorinnen und Autoren}.
 Digitale Edition, https://schnitzler-briefe.acdh.oeaw.ac.at/{\dateiname}.html (Stand \today)
\fi

\end{document}


