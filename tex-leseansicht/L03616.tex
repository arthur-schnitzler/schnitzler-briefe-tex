%% latex-korrekturansicht-vorspann.tex
%% Vorspann für die Korrekturansicht.
%% Lädt die gemeinsame Datei latex-vorspann.tex mit gesetztem Schalter.

\newif\ifkorrekturansicht
\korrekturansichttrue

\input{../tex-inputs/latex-vorspann}


\section[Arthur Schnitzler an Karl Emil Franzos, 11. 5. 1888]{L03616 Arthur Schnitzler an Karl Emil Franzos, 11. 5. 1888}
\nopagebreak\mylabel{L03616v}
\rehead{ }\normalsize\beginnumbering\briefempfaengerindex{Franzos, Karl Emil@\textsc{Franzos, Karl Emil}!zzzSchnitzler, Arthur@\emph{von Arthur Schnitzler}!1888-05-111@{11. 5. 1888}|(be}
\toendnotes[C]{\smallbreak\pagebreak[2]}\Standort{Wienbibliothek im Rathaus, H.I.N.-60193.}
\physDesc{Brief, 1 Blatt, 2 Seiten, 731 Zeichen
\newline{}Handschrift: schwarze Tinte, deutsche Kurrent}\toendnotes[C]{\smallbreak}
\pstart\center{}{\pb}Verehrteſter Herr Franzos!\pend\vspace{0.5em}
\pstart
           Es fügt ſich, daſs ich bereits morgen – wider mein Erwarten – von hier fortreiſen
               muſs, wodurch ich nicht mehr dazuko{\geminationm}e, Ihnen und Ihrer
               hochverehrten Frau Gemahlin\pwindex{Franzos, Ottilie 24.09.1856 – 05.03.1932@\textsc{Franzos, Ottilie} (24.09.1856 – 05.03.1932), \emph{Schriftsteller/Schriftstellerin}|pwv}
               meinen perſönlichen Dank für Ihre liebenswürdige \label{K_L03616-33v}\edtext{Gaſtfreundschaft}{\lemma{\textnormal{\emph{Gaſtfreundschaft}}}\Cendnote{\textnormal{Vgl. A. S.: \emph{Tagebuch}, 28. 4. 1888.
               }}}\label{K_L03616-33} auszudrücken. Ich muſs mich begnügen, dies auf dieſem
               Wege zu thun, und Sie ſchriftlich bitten, meines Danks’ und meiner Hochachtung
               verſichert zu ſein.\pend
           
\pstart
           Was die an Sie geſandten \label{K_L03616-1v}\edtext{\textsc{Manuscripte}}{\lemma{\textnormal{\emph{Manuscripte}}}\Cendnote{\textnormal{Siehe Arthur Schnitzler an Karl Emil Franzos, 29. 4. 1888.
                  }}}\label{K_L03616-1}\pwindex{Amerika@\emph{Amerika}|pwv}\pwindex{Mein Freund Ypsilon. Aus den Papieren eines Arztes@\emph{Mein Freund Ypsilon. Aus den Papieren eines Arztes}|pwv}\pwindex{Erbschaft@\emph{Erbschaft}|pwv} betrifft, ſo würde ich um eine Antwort, {\pb}eventuelle \label{K_L03616-11v}\edtext{Rückſendung erſt nach \textsc{London}\oindex{London@\textbf{London}, \emph{P.PPLC}|pw}}{\lemma{\textnormal{\emph{Rückſendung … London}}}\Cendnote{\textnormal{Hier ist »nach« nicht zeitlich, sondern
                  räumlich zu verstehen: Schnitzler bittet darum, dass
                  ihm die Texte nach London\oindex{London@\textbf{London}, \emph{P.PPLC}|pwk} gesandt werden. (Er reiste nicht direkt von Berlin\oindex{Berlin@\textbf{Berlin}, \emph{P.PPLC}|pwk}, sondern über Wien\oindex{Wien@\textbf{Wien}, \emph{A.ADM2}|pwk}.)
                  Zu der hier noch angedachten Mitteilung der Londoner\oindex{London@\textbf{London}, \emph{P.PPLC}|pwk} Adresse dürfte es nicht gekommen sein, was dafür spricht, 
                  dass Franzos\pwindex{Franzos, Karl Emil 25.10.1848 – 28.01.1904@\textsc{Franzos, Karl Emil} (25.10.1848 – 28.01.1904), \emph{Schriftsteller/Schriftstellerin, Journalist/Journalistin}|pwk} unmittelbar auf dieses
                  Schreiben mit der Rücksendung reagierte, vgl. Karl Emil Franzos an Arthur Schnitzler, [3. 5. 1888 –
               11. 5. 1888?].}}}\label{K_L03616-11}
               bitten, von wo aus ich ſo frei ſein werde, Ihnen meine Adreſſe mitzutheilen.\pend
           \pstart Indem ich mich Ihnen und Ihrer w. Frau Gemahlin\pwindex{Franzos, Ottilie 24.09.1856 – 05.03.1932@\textsc{Franzos, Ottilie} (24.09.1856 – 05.03.1932), \emph{Schriftsteller/Schriftstellerin}|pwv} ergebenſt empfehle, bin ich mit beſondrer
               Hochachtung Ihr \spacefill\mbox{Dr. Arthur Schnitzler}\pend{}
\pstart
           \textsc{Berlin\oindex{Berlin@\textbf{Berlin}, \emph{P.PPLC}|pw}}, 11. 5. 88\pend
           \selectlanguage{ngerman}\endnumbering\briefempfaengerindex{Franzos, Karl Emil@\textsc{Franzos, Karl Emil}!zzzSchnitzler, Arthur@\emph{von Arthur Schnitzler}!1888-05-111@{11. 5. 1888}|)be}\mylabel{L03616h}  \normalsize

\doendnotes{C}
\bigskip
\vfill

\clearpage

\footnotesize

\lohead{\textsc{register}}

% Definiere theindex-Environment komplett neu ohne reledmac
\makeatletter
\renewenvironment{theindex}{%
  \section*{\indexname}%
  \setlength{\parindent}{0pt}%
  \setlength{\parskip}{0pt plus 0.3pt}%
  \let\item\@idxitem
}{%
  \clearpage
}
\makeatother

\IfFileExists{\jobname-pw.ind}{\input{\jobname-pw.ind}}{}

\end{document}

      