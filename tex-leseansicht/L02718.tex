%% latex-leseansicht-vorspann.tex
%% Vorspann für die Leseansicht.
%% Lädt die gemeinsame Datei latex-vorspann.tex mit nicht gesetztem Schalter.

\newif\ifkorrekturansicht
\korrekturansichtfalse

\input{../tex-inputs/latex-vorspann}


\section[Paul Goldmann an Arthur Schnitzler, {[}zwischen 29. 9. und 2. 10. 1893{]}]{L02718 Paul Goldmann an Arthur Schnitzler, {[}zwischen 29. 9. und 2. 10. 1893{]}}
\nopagebreak\mylabel{L02718v}
\rehead{ }\normalsize\beginnumbering\briefempfaengerindex{Schnitzler, Arthur@\textsc{Schnitzler, Arthur}!zzzGoldmann, Paul@\emph{von Paul Goldmann}!1893-10-021@{{[}zwischen 29. 9. und 2. 10. 1893{]}}|(be}
\toendnotes[C]{\smallbreak\pagebreak[2]}
\correspDesc{Versand  durch Paul Goldmann im Zeitraum [zwischen 29. 9. und
                  2. 10. 1893] in Paris
\newline{}Erhalt  durch Arthur Schnitzler im Zeitraum [zwischen 1.–4.] 10. 1893 in Wien}\toendnotes[C]{\smallbreak}
\Standort{DLA, A:Schnitzler, HS.NZ85.1.3163.}
\physDesc{Brief, 1 Blatt, 2 Seiten, 553 Zeichen
\newline{}Handschrift: schwarze Tinte, deutsche Kurrent
\newline{}Schnitzler: 1) mit Bleistift das Datum »Octob 93.« vermerkt  2) mit rotem Buntstift eine Unterstreichung}\toendnotes[C]{\smallbreak}
\pstart
           \noindent{}{\pb}Herzlichen Dank, liebſter Freund! Die \label{K_L02718-1v}\edtext{S. u. M.-Ztg.\orgindex{Wiener Sonn- und Montagszeitung@Wiener Sonn- und Montagszeitung|pw}}{\lemma{\textnormal{\emph{S. u. M.-Ztg.}}}\Cendnote{\textnormal{Unklarer Bezug auf die \emph{Wiener Sonn- und Montags-Zeitung}\orgindex{Wiener Sonn- und Montagszeitung@Wiener Sonn- und Montagszeitung|pwk}. Sofern die vorliegende Stelle eine Reaktion
                  auf einen Text darstellt, der in der letzten oder vorletzten Nummer\pwindex{Wiener Sonn- und Montagszeitung@\emph{Wiener Sonn- und Montagszeitung}|pwkv} enthalten war, dürfte es sich um
                  diesen handeln: G. Engelsmann\pwindex{Engelsmann, Gabriel 27.\,12.\,1855 – 18.\,11.\,1941 Wien@\textsc{Engelsmann, Gabriel} (27.\,12.\,1855 – 18.\,11.\,1941 Wien), \emph{Journalist}|pwk}: \emph{Zola über die Anonymität der Presse}\pwindex{Engelsmann, Gabriel 27.\,12.\,1855 – 18.\,11.\,1941 Wien@\textsc{Engelsmann, Gabriel} (27.\,12.\,1855 – 18.\,11.\,1941 Wien), \emph{Journalist}!Zola über die Anonymität der Presse@\strich\emph{Zola über die Anonymität der Presse}|pwk}. In: \emph{Wiener Sonn- und Montags-Zeitung}\pwindex{Wiener Sonn- und Montagszeitung@\emph{Wiener Sonn- und Montagszeitung}|pwk}, Jg. 31,
                     Nr. 38, 18. 9. 1893, S. 1–2. (Gabriel
                     Engelsmann\pwindex{Engelsmann, Gabriel 27.\,12.\,1855 – 18.\,11.\,1941 Wien@\textsc{Engelsmann, Gabriel} (27.\,12.\,1855 – 18.\,11.\,1941 Wien), \emph{Journalist}|pwk} hatte im Vorjahr auch zu der von Goldmann\pwindex{Goldmann, Paul 31.\,1.\,1865 Breslau – 25.\,9.\,1935 Wien@\textsc{Goldmann, Paul} (31.\,1.\,1865 Breslau – 25.\,9.\,1935 Wien), \emph{Schriftsteller, Journalist}|pwk} herausgegeben \emph{An
                     der schönen blauen Donau}\orgindex{der schönen blauen Donau@An der schönen blauen Donau|pwk} beigetragen.) – Sofern es sich bei der Stelle um eine Aussage
                  über Hermann Bahr\pwindex{Bahr, Hermann 19.\,7.\,1863 Linz – 15.\,1.\,1934 München@\textsc{Bahr, Hermann} (19.\,7.\,1863 Linz – 15.\,1.\,1934 München), \emph{Schriftsteller, Kritiker}|pwk} handelt, so dürfte diese
                  aus der Abrechnung\pwindex{Hirschfeld, Robert 17.\,9.\,1857 Žďár nad Sázavou – 2.\,4.\,1914 Salzburg@\textsc{Hirschfeld, Robert} (17.\,9.\,1857 Žďár nad Sázavou – 2.\,4.\,1914 Salzburg), \emph{Journalist, Musikkritiker}!Zwei Freunde Burckhards@\strich\emph{Zwei Freunde Burckhards}|pwkv} stammen,
                  die am 24. 7. 1893 im Blatt\pwindex{Wiener Sonn- und Montagszeitung@\emph{Wiener Sonn- und Montagszeitung}|pwkv} stand: L. A. Terne. (Dr. Rob. Hirschfeld)\pwindex{Hirschfeld, Robert 17.\,9.\,1857 Žďár nad Sázavou – 2.\,4.\,1914 Salzburg@\textsc{Hirschfeld, Robert} (17.\,9.\,1857 Žďár nad Sázavou – 2.\,4.\,1914 Salzburg), \emph{Journalist, Musikkritiker}|pwk}: \emph{Zwei Freunde Burkhards}\pwindex{Hirschfeld, Robert 17.\,9.\,1857 Žďár nad Sázavou – 2.\,4.\,1914 Salzburg@\textsc{Hirschfeld, Robert} (17.\,9.\,1857 Žďár nad Sázavou – 2.\,4.\,1914 Salzburg), \emph{Journalist, Musikkritiker}!Zwei Freunde Burckhards@\strich\emph{Zwei Freunde Burckhards}|pwk}. In: \emph{Wiener Sonn- und Montags-Zeitung}\pwindex{Wiener Sonn- und Montagszeitung@\emph{Wiener Sonn- und Montagszeitung}|pwk}, Jg. 31, Nr. 30,
                     S. 1–3.}}}\label{K_L02718-1} iſt ganz hübſch; ehrliche Mühe, zu verſtehen, und
               ehrlicher und gutmüthiger \strikeout{Repſ} Reſpekt vor dem
               Talent. \label{K_L02718-2v}\edtext{\textsc{Bahr\pwindex{Bahr, Hermann 19.\,7.\,1863 Linz – 15.\,1.\,1934 München@\textsc{Bahr, Hermann} (19.\,7.\,1863 Linz – 15.\,1.\,1934 München), \emph{Schriftsteller, Kritiker}|pw}\pwindex{Bahr, Hermann 19.\,7.\,1863 Linz – 15.\,1.\,1934 München@\textsc{Bahr, Hermann} (19.\,7.\,1863 Linz – 15.\,1.\,1934 München), \emph{Schriftsteller, Kritiker}!junge Österreich@\strich\emph{Das junge Österreich}|pwv}}}{\lemma{\textnormal{\emph{Bahr}}}\Cendnote{\textnormal{Gemeint war der zweite Teil\pwindex{Bahr, Hermann 19.\,7.\,1863 Linz – 15.\,1.\,1934 München@\textsc{Bahr, Hermann} (19.\,7.\,1863 Linz – 15.\,1.\,1934 München), \emph{Schriftsteller, Kritiker}!junge Österreich@\strich\emph{Das junge Österreich}|pwkv} von Hermann Bahrs\pwindex{Bahr, Hermann 19.\,7.\,1863 Linz – 15.\,1.\,1934 München@\textsc{Bahr, Hermann} (19.\,7.\,1863 Linz – 15.\,1.\,1934 München), \emph{Schriftsteller, Kritiker}|pwk} dreiteiliger Feuilleton-Serie\pwindex{Bahr, Hermann 19.\,7.\,1863 Linz – 15.\,1.\,1934 München@\textsc{Bahr, Hermann} (19.\,7.\,1863 Linz – 15.\,1.\,1934 München), \emph{Schriftsteller, Kritiker}!junge Österreich@\strich\emph{Das junge Österreich}|pwkv}{ }\emph{Das junge Österreich}\pwindex{Bahr, Hermann 19.\,7.\,1863 Linz – 15.\,1.\,1934 München@\textsc{Bahr, Hermann} (19.\,7.\,1863 Linz – 15.\,1.\,1934 München), \emph{Schriftsteller, Kritiker}!junge Österreich@\strich\emph{Das junge Österreich}|pwk}. Über Schnitzler steht darin: »Arthur \so{Schnitzler} ist anders. Er ist ein großer Virtuose, aber einer kleinen Note. Torresani\pwindex{Torresani-Lanzenfeld, Carl von 19.\,4.\,1846 Mailand – 16.\,4.\,1907 Nago-Torbole@\textsc{Torresani-Lanzenfeld, Carl von} (19.\,4.\,1846 Mailand – 16.\,4.\,1907 Nago-Torbole), \emph{Schriftsteller, Offizier}|pw} streut aus reichen Krügen, ohne die einzelne
                     Gabe zu achten. Schnitzler darf nicht
                     verschwenden. Er muß sparen. Er hat wenig. So will er es denn mit der
                     zärtlichsten Sorge, mit erfinderischer Mühe, mit geduldigem Geize schleifen,
                     bis das Geringe durch seine unermüdlichen Künste Adel und Würde verdient. Was
                     er bringt, ist nichtig. Aber wie er es bringt, darf gelten. Die großen Züge der
                     Zeit, Leidenschaften, Stürme, Erschütterungen der Menschen, die ungestüme
                     Pracht der Welt an Farben und an Klängen ist ihm versagt. Er weiß immer nur
                     einen einzigen Menschen, ja nur ein einziges Gefühl zu gestalten. Aber dieser
                     Gestalt gibt er Vollkommenheit, Vollendung. So ist er recht der \textsc{\begin{otherlanguage}{french}artiste\end{otherlanguage}} nach dem Herzen des ›Parnasses‹, jener Franz\oindex{Frankreich@\textbf{Frankreich}|pwv}osen, welche um den Werth an Gehalt nicht bekümmert,
                     nur in der Fassung Pflicht und Verdienst der Kunst erkennen und als eitel
                     verachten, was nicht seltene Nuance, malendes Objectiv, gesuchte Metapher
                     ist.« (\emph{Das junge Österreich. II}\pwindex{Bahr, Hermann 19.\,7.\,1863 Linz – 15.\,1.\,1934 München@\textsc{Bahr, Hermann} (19.\,7.\,1863 Linz – 15.\,1.\,1934 München), \emph{Schriftsteller, Kritiker}!junge Österreich@\strich\emph{Das junge Österreich}|pwk}. In: \emph{Deutsche Zeitung}\pwindex{Deutsche Zeitung@\emph{Deutsche Zeitung}|pwk}, Jg. 23, Nr. 7813, 27. 9. 1893, Morgen-Ausgabe, S. 1–3, hier
                     S. 1) Schnitzler notierte 
                  dazu am 27. 9. 1893
                  im \emph{Tagebuch}\pwindex{Schnitzler, Arthur 15.\,5.\,1862 Wien – 21.\,10.\,1931 ebd.@\textsc{Schnitzler, Arthur} (15.\,5.\,1862 Wien – 21.\,10.\,1931 ebd.), \emph{Schriftsteller, Mediziner}!Tagebuch@\strich\emph{Tagebuch}|pwk}: »Ich sei ein großer
                     Virtuos auf kleinem Ton; jedoch \begin{otherlanguage}{french}apporteur du
                        neuf\end{otherlanguage}, etc.; – ich war ärgerlich.«}}}\label{K_L02718-2} hingegen iſt
               niederträchtig, neidiſch, gemein, \label{K_L02718-3v}\edtext{perfid}{\lemma{\textnormal{\emph{perfid}}}\Cendnote{\textnormal{Dieser Ausdruck Goldmanns\pwindex{Goldmann, Paul 31.\,1.\,1865 Breslau – 25.\,9.\,1935 Wien@\textsc{Goldmann, Paul} (31.\,1.\,1865 Breslau – 25.\,9.\,1935 Wien), \emph{Schriftsteller, Journalist}|pwk} ermöglicht letztlich die ungefähre
                  Datierung des undatierten Briefes: Bahrs\pwindex{Bahr, Hermann 19.\,7.\,1863 Linz – 15.\,1.\,1934 München@\textsc{Bahr, Hermann} (19.\,7.\,1863 Linz – 15.\,1.\,1934 München), \emph{Schriftsteller, Kritiker}|pwk}{ }Kritik\pwindex{Bahr, Hermann 19.\,7.\,1863 Linz – 15.\,1.\,1934 München@\textsc{Bahr, Hermann} (19.\,7.\,1863 Linz – 15.\,1.\,1934 München), \emph{Schriftsteller, Kritiker}!junge Österreich@\strich\emph{Das junge Österreich}|pwkv} erschien am 27. 9. 1893. Schnitzler datierte den Brief beziehungsweise
                  das Empfangsdatum desselben auf »Octob 93«. Spätestens am 4. 10. 1893 muss Schnitzler den
                  Brief erhalten haben, insofern im \emph{Tagebuch}\pwindex{Schnitzler, Arthur 15.\,5.\,1862 Wien – 21.\,10.\,1931 ebd.@\textsc{Schnitzler, Arthur} (15.\,5.\,1862 Wien – 21.\,10.\,1931 ebd.), \emph{Schriftsteller, Mediziner}!Tagebuch@\strich\emph{Tagebuch}|pwk}-Eintrag des genannten Tages
                  Folgendes zu lesen ist: »Ludaßy\pwindex{Gans-Ludassy, Julius von 13.\,4.\,1858 Wien – 30.\,9.\,1922 ebd.@\textsc{Gans-Ludassy, Julius von} (13.\,4.\,1858 Wien – 30.\,9.\,1922 ebd.), \emph{Schriftsteller, Journalist, Herausgeber}|pw} findet (wie Paul G.\pwindex{Goldmann, Paul 31.\,1.\,1865 Breslau – 25.\,9.\,1935 Wien@\textsc{Goldmann, Paul} (31.\,1.\,1865 Breslau – 25.\,9.\,1935 Wien), \emph{Schriftsteller, Journalist}|pw}) die Kritik\pwindex{Bahr, Hermann 19.\,7.\,1863 Linz – 15.\,1.\,1934 München@\textsc{Bahr, Hermann} (19.\,7.\,1863 Linz – 15.\,1.\,1934 München), \emph{Schriftsteller, Kritiker}!junge Österreich@\strich\emph{Das junge Österreich}|pwv} von Bahr\pwindex{Bahr, Hermann 19.\,7.\,1863 Linz – 15.\,1.\,1934 München@\textsc{Bahr, Hermann} (19.\,7.\,1863 Linz – 15.\,1.\,1934 München), \emph{Schriftsteller, Kritiker}|pw}
                     perfid.« Anzunehmen ist, dass Schnitzler{ }Goldmann\pwindex{Goldmann, Paul 31.\,1.\,1865 Breslau – 25.\,9.\,1935 Wien@\textsc{Goldmann, Paul} (31.\,1.\,1865 Breslau – 25.\,9.\,1935 Wien), \emph{Schriftsteller, Journalist}|pwk} die Kritik\pwindex{Bahr, Hermann 19.\,7.\,1863 Linz – 15.\,1.\,1934 München@\textsc{Bahr, Hermann} (19.\,7.\,1863 Linz – 15.\,1.\,1934 München), \emph{Schriftsteller, Kritiker}!junge Österreich@\strich\emph{Das junge Österreich}|pwkv} am 27. 9. 1893 oder 28. 9. 1893 geschickt hat, sodass
                     Goldmanns\pwindex{Goldmann, Paul 31.\,1.\,1865 Breslau – 25.\,9.\,1935 Wien@\textsc{Goldmann, Paul} (31.\,1.\,1865 Breslau – 25.\,9.\,1935 Wien), \emph{Schriftsteller, Journalist}|pwk} Replik zwischen dem 29. 9. 1893 und dem 2. 10. 1893 verfasst worden sein dürfte.}}}\label{K_L02718-3}. Und dieſe unverſchämte
               Schwindelei, was \strikeout{Lit}{ }franzöſiſch\oindex{Frankreich@\textbf{Frankreich}|pwv}e Literatur-Kenntniß
               anlangt. \label{K_L02718-4v}\edtext{\textsc{Courteline\pwindex{Courteline, Georges 25.\,6.\,1858 Tours – 25.\,6.\,1929 Paris@\textsc{Courteline, Georges} (25.\,6.\,1858 Tours – 25.\,6.\,1929 Paris), \emph{Schriftsteller}|pw}}, den Militär-Humoriſten\pwindex{Courteline, Georges 25.\,6.\,1858 Tours – 25.\,6.\,1929 Paris@\textsc{Courteline, Georges} (25.\,6.\,1858 Tours – 25.\,6.\,1929 Paris), \emph{Schriftsteller}|pwv}, in einer Linie mit \textsc{Lavedan\pwindex{Lavedan, Henri Léon 9.\,4.\,1859 Orléans – 4.\,9.\,1940 Paris@\textsc{Lavedan, Henri Léon} (9.\,4.\,1859 Orléans – 4.\,9.\,1940 Paris), \emph{Schriftsteller, Journalist}|pw}}}{\lemma{\textnormal{\emph{Courteline, … Lavedan}}}\Cendnote{\textnormal{Die weiteren von Goldmann\pwindex{Goldmann, Paul 31.\,1.\,1865 Breslau – 25.\,9.\,1935 Wien@\textsc{Goldmann, Paul} (31.\,1.\,1865 Breslau – 25.\,9.\,1935 Wien), \emph{Schriftsteller, Journalist}|pwk} kritisierten Aussagen\pwindex{Bahr, Hermann 19.\,7.\,1863 Linz – 15.\,1.\,1934 München@\textsc{Bahr, Hermann} (19.\,7.\,1863 Linz – 15.\,1.\,1934 München), \emph{Schriftsteller, Kritiker}!junge Österreich@\strich\emph{Das junge Österreich}|pwkv} finden sich im ersten Teil\pwindex{Bahr, Hermann 19.\,7.\,1863 Linz – 15.\,1.\,1934 München@\textsc{Bahr, Hermann} (19.\,7.\,1863 Linz – 15.\,1.\,1934 München), \emph{Schriftsteller, Kritiker}!junge Österreich@\strich\emph{Das junge Österreich}|pwkv}: Hermann Bahr\pwindex{Bahr, Hermann 19.\,7.\,1863 Linz – 15.\,1.\,1934 München@\textsc{Bahr, Hermann} (19.\,7.\,1863 Linz – 15.\,1.\,1934 München), \emph{Schriftsteller, Kritiker}|pwk}: \emph{Das junge Österreich. I}\pwindex{Bahr, Hermann 19.\,7.\,1863 Linz – 15.\,1.\,1934 München@\textsc{Bahr, Hermann} (19.\,7.\,1863 Linz – 15.\,1.\,1934 München), \emph{Schriftsteller, Kritiker}!junge Österreich@\strich\emph{Das junge Österreich}|pwk}. In: \emph{Deutsche Zeitung}\pwindex{Deutsche Zeitung@\emph{Deutsche Zeitung}|pwk}, Jg. 23, Nr. 7806, 20. 9. 1893, Morgen-Ausgabe, S. 1–2.}}}\label{K_L02718-4}
               zu nennen! {\pb}\textsc{Aurélien Scholl\pwindex{Scholl, Aurélien 13.\,7.\,1833 Bordeaux – 1902 Paris@\textsc{Scholl, Aurélien} (13.\,7.\,1833 Bordeaux – 1902 Paris), \emph{Schriftsteller, Journalist}|pw}}, den geiſtreichen \textsc{Chroniqueur\pwindex{Scholl, Aurélien 13.\,7.\,1833 Bordeaux – 1902 Paris@\textsc{Scholl, Aurélien} (13.\,7.\,1833 Bordeaux – 1902 Paris), \emph{Schriftsteller, Journalist}|pwv} à la Daniel Spitzer\pwindex{Spitzer, Daniel 3.\,6.\,1835 Wien – 11.\,1.\,1893 ebd.@\textsc{Spitzer, Daniel} (3.\,6.\,1835 Wien – 11.\,1.\,1893 ebd.), \emph{Schriftsteller, Journalist, Rechtsanwalt}|pw}}, mit \textsc{Lavedan\pwindex{Lavedan, Henri Léon 9.\,4.\,1859 Orléans – 4.\,9.\,1940 Paris@\textsc{Lavedan, Henri Léon} (9.\,4.\,1859 Orléans – 4.\,9.\,1940 Paris), \emph{Schriftsteller, Journalist}|pw}}, dem Analytiker\pwindex{Lavedan, Henri Léon 9.\,4.\,1859 Orléans – 4.\,9.\,1940 Paris@\textsc{Lavedan, Henri Léon} (9.\,4.\,1859 Orléans – 4.\,9.\,1940 Paris), \emph{Schriftsteller, Journalist}|pwv},
               zuſammenzuſtellen \textsc{etc}. Wirklich zu frech! Und diefer
               unerträgliche Styl! {\dots}\pend
           
\pstart
           Grüß’ Dich Gott! {\\[\baselineskip]}Dein {\\[\baselineskip]}\spacefill\mbox{P. G.}\pend
           \leftskip=0em{}\selectlanguage{ngerman}\endnumbering\briefempfaengerindex{Schnitzler, Arthur@\textsc{Schnitzler, Arthur}!zzzGoldmann, Paul@\emph{von Paul Goldmann}!1893-09-291@{{[}zwischen 29. 9. und 2. 10. 1893{]}}|)be}\mylabel{L02718h}  \newcommand{\dateiname}{L02718}\newcommand{\titel}{Paul Goldmann an Arthur Schnitzler, [zwischen 29. 9. und 2. 10. 1893]}\newcommand{\editorInnen}{Martin Anton Müller und Laura Untner}%% latex-leseansicht-abspann.tex
%% Abspann für die Leseansicht.
%% Der Schalter \ifkorrekturansicht ist bereits durch den Vorspann gesetzt.

%% latex-abspann.tex
%% Gemeinsamer Abspann für Korrekturansicht und Leseansicht.
%% Setzt den Schalter \ifkorrekturansicht voraus (gesetzt in den
%% einbindenden Dateien latex-korrekturansicht-abspann.tex bzw.
%% latex-leseansicht-abspann.tex).
%% ---------------------------------------------------------------

\normalsize

% Das esempio-Environment wird nur in der Leseansicht benötigt
\ifkorrekturansicht\else
\newenvironment{esempio}[3]%
{
    \vspace{1.5ex}
    \rlap{\underline{#1}}
    \par
    \setlength{\parindent}{0cm}
    \nopagebreak
    \leftskip=#2cm
    \rightskip=#3cm
}
{
    \par
}
\fi

\doendnotes{C}
\bigskip
\vfill

\clearpage

\footnotesize

\ifkorrekturansicht
  \lohead{\textsc{register}}
\fi

% theindex-Environment neu definieren ohne reledmac
\makeatletter
\renewenvironment{theindex}{%
  \ifkorrekturansicht
    \section*{\indexname}%
  \else
    \subsubsection*{Index der erwähnten Entitäten}%
  \fi
  \setlength{\parindent}{0pt}%
  \setlength{\parskip}{0pt plus 0.3pt}%
  \let\item\@idxitem
}{%
  \ifkorrekturansicht\clearpage\fi
}
\makeatother

\IfFileExists{\jobname-pw.ind}{\input{\jobname-pw.ind}}{}

% Quellenangabe nur in der Leseansicht
\ifkorrekturansicht\else
% Fallback-Definitionen, falls die .tex-Datei \titel etc. nicht gesetzt hat
\providecommand{\titel}{}
\providecommand{\editorInnen}{}
\providecommand{\dateiname}{\jobname}

\vspace{3cm}

\vfill

\footnotesize
\textsc{Quelle}: \titel. Herausgegeben von {\editorInnen}. In: \emph{Arthur Schnitzler: Briefwechsel mit Autorinnen und Autoren}.
 Digitale Edition, https://schnitzler-briefe.acdh.oeaw.ac.at/{\dateiname}.html (Stand \today)
\fi

\end{document}


