\input{../tex-inputs/latex-pdf-vorspann}
\begin{center}
            \textcolor{red}{ENTWURF. ENTZIFFERUNG NOCH NICHT KORREKTURGELESEN}
                      \end{center}
            
               \section[Paul Goldmann an Arthur Schnitzler, {[}zwischen 29. 9. und 2. 10. 1893{]}]{ Paul Goldmann an Arthur Schnitzler, {[}zwischen 29. 9. und
               2. 10. 1893{]}}\nopagebreak\mylabel{v}\rehead{ }\begin{ledgroupsized}[t]{13cm}\normalsize\beginnumbering\briefempfaengerindex{Schnitzler, Arthur@\textsc{Schnitzler, Arthur}!zzzGoldmann, Paul@\emph{von Paul Goldmann}!1893-09-291@{{[}zwischen 29. 9. und
                  2. 10. 1893{]}}|(be} \toendnotes[C]{\smallbreak\pagebreak[2]} \Standort{DLA, A:Schnitzler, HS.NZ85.1.3163.}
\physDesc{Brief, 1 Blatt, 2 Seiten
\newline{}Handschrift: schwarze Tinte, deutsche Kurrent
\newline{}Schnitzler: 1) mit Bleistift das Datum »Octob 93.« vermerkt 2) mit rotem Buntstift eine Unterstreichung}\toendnotes[C]{\smallbreak}\pstart
           \noindent{}{\pb}Herzlichen Dank, liebſter Freund! Die \label{K_L02718-1v}\edtext{S. u. M.-Ztg.\pwindex{Wiener Sonn- und Montagszeitung1863 – 1936@\emph{Wiener Sonn- und Montagszeitung}|pw}}{\lemma{\textnormal{\emph{S. u. M.-Ztg.}}}\Cendnote{\textnormal{Unklarer Bezug. Möglich, aber
                  unwahrscheinlich ist, dass es sich um die Monate zurückliegende Anatol-Kritik\pwindex{Anatol« von Arthur Schnitzler1893-01-02 – 1893-01-02@\emph{»Anatol« von Arthur Schnitzler} {[}1893-01-02 – 1893-01-02{]}|pwkv} von Robert Hirschfeld\pwindex{Hirschfeld, Robert 17.09.1857 – 02.04.1914@\textsc{Hirschfeld, Robert} (17.09.1857 – 02.04.1914), \emph{Journalist, Kritiker}|pwk} vom 2. 1. 1893 handelte.}}}\label{K_L02718-1h} iſt ganz hübſch;
               ehrliche Mühe, zu verſtehen, und ehrlicher und gutmüthiger \strikeout{Repſ} Reſpekt vor dem Talent. \label{K_L02718-2v}\edtext{\textsc{Bahr\pwindex{Bahr, Hermann 19.07.1863 – 15.01.1934@\textsc{Bahr, Hermann} (19.07.1863 – 15.01.1934), \emph{Schriftsteller, Kritiker}|pw}\pwindex{Bahr, Hermann 19.07.1863 – 15.01.1934@\textsc{Bahr, Hermann} (19.07.1863 – 15.01.1934), \emph{Schriftsteller, Kritiker}!junge Oesterreich1893-09-20 – 1893-10-07@\strich\emph{Das junge Österreich} {[}1893-09-20 – 1893-10-07{]}|pwv}}}{\lemma{\textnormal{\emph{Bahr}}}\Cendnote{\textnormal{Gemeint war der zweite Teil\pwindex{Bahr, Hermann 19.07.1863 – 15.01.1934@\textsc{Bahr, Hermann} (19.07.1863 – 15.01.1934), \emph{Schriftsteller, Kritiker}!junge Oesterreich1893-09-20 – 1893-10-07@\strich\emph{Das junge Österreich} {[}1893-09-20 – 1893-10-07{]}|pwkv} von Hermann Bahr\pwindex{Bahr, Hermann 19.07.1863 – 15.01.1934@\textsc{Bahr, Hermann} (19.07.1863 – 15.01.1934), \emph{Schriftsteller, Kritiker}|pwk}s Feuilleton-Reihe\pwindex{Bahr, Hermann 19.07.1863 – 15.01.1934@\textsc{Bahr, Hermann} (19.07.1863 – 15.01.1934), \emph{Schriftsteller, Kritiker}!junge Oesterreich1893-09-20 – 1893-10-07@\strich\emph{Das junge Österreich} {[}1893-09-20 – 1893-10-07{]}|pwkv}{ }\emph{Das junge Österreich}\pwindex{Bahr, Hermann 19.07.1863 – 15.01.1934@\textsc{Bahr, Hermann} (19.07.1863 – 15.01.1934), \emph{Schriftsteller, Kritiker}!junge Oesterreich1893-09-20 – 1893-10-07@\strich\emph{Das junge Österreich} {[}1893-09-20 – 1893-10-07{]}|pwk}. Siehe Hermann Bahr\pwindex{Bahr, Hermann 19.07.1863 – 15.01.1934@\textsc{Bahr, Hermann} (19.07.1863 – 15.01.1934), \emph{Schriftsteller, Kritiker}|pwk}: \emph{Das junge Österreich. II}\pwindex{Bahr, Hermann 19.07.1863 – 15.01.1934@\textsc{Bahr, Hermann} (19.07.1863 – 15.01.1934), \emph{Schriftsteller, Kritiker}!junge Oesterreich1893-09-20 – 1893-10-07@\strich\emph{Das junge Österreich} {[}1893-09-20 – 1893-10-07{]}|pwk}. In: \emph{Deutsche Zeitung}\pwindex{Deutsche Zeitung1871 – 1907@\emph{Deutsche Zeitung}|pwk}. Jg. 23, Nr. 7813, 27. 9. 1893, Morgen-Ausgabe, S. 1–3. Über
                  Schnitzler schrieb er darin unter anderem: »Arthur \so{Schnitzler}\pwindex{Schnitzler, Arthur 15.05.1862 – 21.10.1931@\textsc{Schnitzler, Arthur} (15.05.1862 – 21.10.1931), \emph{Schriftsteller, Mediziner}|pw} iſt anders. Er iſt ein großer Virtuoſe\pwindex{Schnitzler, Arthur 15.05.1862 – 21.10.1931@\textsc{Schnitzler, Arthur} (15.05.1862 – 21.10.1931), \emph{Schriftsteller, Mediziner}|pwv}, aber einer kleinen Note. Torreſani\pwindex{Torresani-Lanzenfeld, Carl von 19.04.1846 – 16.04.1907@\textsc{Torresani-Lanzenfeld, Carl von} (19.04.1846 – 16.04.1907), \emph{Schriftsteller}|pw} ſtreut aus reichen Krügen, ohne die einzelne
                     Gabe zu achten. Schnitzler\pwindex{Schnitzler, Arthur 15.05.1862 – 21.10.1931@\textsc{Schnitzler, Arthur} (15.05.1862 – 21.10.1931), \emph{Schriftsteller, Mediziner}|pw} darf nicht
                     verſchwenden. Er muß ſparen. Er hat wenig. So will er es denn mit der
                     zärtlichſten Sorge, mit erfinderiſcher Mühe, mit geduldigem Geize ſchleifen,
                     bis das Geringe durch ſeine unermüdlichen Künſte Adel und Würde verdient. Was
                     er bringt, iſt nichtig. Aber wie er es bringt, darf gelten. Die großen Züge der
                     Zeit, Leidenſchaften, Stürme, Erſchütterungen der Menſchen, die ungeſtüme
                     Pracht der Welt an Farben und an Klängen iſt ihm verſagt. Er weiß immer nur
                     einen einzigen Menſchen, ja nur ein einziges Gefühl zu geſtalten. Aber dieſer
                     Geſtalt gibt er Vollkommenheit, Vollendung. So iſt er recht der \textsc{\begin{otherlanguage}{french}artiste\end{otherlanguage}} nach dem Herzen des »Parnasses«, jener Franzoſ\oindex{Frankreich@\textbf{Frankreich}|pwv}en, welche um den Werth an Gehalt nicht bekümmert,
                     nur in der Faſſung Pflicht und Verdienſt der Kunſt erkennen und als eitel
                     verachten, was nicht ſeltene Nuance, malendes Objectiv, geſuchte Metapher
                     iſt.« (S. 1) Schnitzler\pwindex{Schnitzler, Arthur 15.05.1862 – 21.10.1931@\textsc{Schnitzler, Arthur} (15.05.1862 – 21.10.1931), \emph{Schriftsteller, Mediziner}|pwk} notierte
                  sich dazu am 27. 9. 1893
                  im \emph{Tagebuch}\pwindex{Schnitzler, Arthur 15.05.1862 – 21.10.1931@\textsc{Schnitzler, Arthur} (15.05.1862 – 21.10.1931), \emph{Schriftsteller, Mediziner}!Tagebuch1981 – 2000@\strich\emph{Tagebuch} {[}1981 – 2000{]}|pwk}: »Ich sei ein großer
                     Virtuos auf kleinem Ton; jedoch \begin{otherlanguage}{french}apporteur du
                        neuf\end{otherlanguage}, etc.;― ich war ärgerlich.« Goldmann\pwindex{Goldmann, Paul 31.01.1865 – 25.09.1935@\textsc{Goldmann, Paul} (31.01.1865 – 25.09.1935), \emph{Schriftsteller, Journalist}|pwk}s Bezug auf Bahr\pwindex{Bahr, Hermann 19.07.1863 – 15.01.1934@\textsc{Bahr, Hermann} (19.07.1863 – 15.01.1934), \emph{Schriftsteller, Kritiker}|pwk}s Kritik\pwindex{Bahr, Hermann 19.07.1863 – 15.01.1934@\textsc{Bahr, Hermann} (19.07.1863 – 15.01.1934), \emph{Schriftsteller, Kritiker}!junge Oesterreich1893-09-20 – 1893-10-07@\strich\emph{Das junge Österreich} {[}1893-09-20 – 1893-10-07{]}|pwkv}
                  ermöglicht letztlich die ungefähre Datierung des Briefes: Die Kritik\pwindex{Bahr, Hermann 19.07.1863 – 15.01.1934@\textsc{Bahr, Hermann} (19.07.1863 – 15.01.1934), \emph{Schriftsteller, Kritiker}!junge Oesterreich1893-09-20 – 1893-10-07@\strich\emph{Das junge Österreich} {[}1893-09-20 – 1893-10-07{]}|pwkv} erschien am 27. 9. 1893 und Schnitzler\pwindex{Schnitzler, Arthur 15.05.1862 – 21.10.1931@\textsc{Schnitzler, Arthur} (15.05.1862 – 21.10.1931), \emph{Schriftsteller, Mediziner}|pwk} datierte den Brief bzw. das
                  Empfangsdatum auf »Octob 93«. Spätestens am 4. 10. 1893 muss Schnitzler\pwindex{Schnitzler, Arthur 15.05.1862 – 21.10.1931@\textsc{Schnitzler, Arthur} (15.05.1862 – 21.10.1931), \emph{Schriftsteller, Mediziner}|pwk} den
                  Brief erhalten haben, insofern im \emph{Tagebuch}\pwindex{Schnitzler, Arthur 15.05.1862 – 21.10.1931@\textsc{Schnitzler, Arthur} (15.05.1862 – 21.10.1931), \emph{Schriftsteller, Mediziner}!Tagebuch1981 – 2000@\strich\emph{Tagebuch} {[}1981 – 2000{]}|pwk}-Eintrag des genannten Tages
                  Folgendes zu lesen ist: »Ludaßy\pwindex{Gans-Ludassy, Julius von 13.04.1858 – 30.09.1922@\textsc{Gans-Ludassy, Julius von} (13.04.1858 – 30.09.1922), \emph{Schriftsteller, Journalist}|pw} findet (wie Paul G.\pwindex{Goldmann, Paul 31.01.1865 – 25.09.1935@\textsc{Goldmann, Paul} (31.01.1865 – 25.09.1935), \emph{Schriftsteller, Journalist}|pw}) die Kritik\pwindex{Bahr, Hermann 19.07.1863 – 15.01.1934@\textsc{Bahr, Hermann} (19.07.1863 – 15.01.1934), \emph{Schriftsteller, Kritiker}!junge Oesterreich1893-09-20 – 1893-10-07@\strich\emph{Das junge Österreich} {[}1893-09-20 – 1893-10-07{]}|pwv} von Bahr\pwindex{Bahr, Hermann 19.07.1863 – 15.01.1934@\textsc{Bahr, Hermann} (19.07.1863 – 15.01.1934), \emph{Schriftsteller, Kritiker}|pw}
                     perfid.« Möglich wäre, dass Schnitzler\pwindex{Schnitzler, Arthur 15.05.1862 – 21.10.1931@\textsc{Schnitzler, Arthur} (15.05.1862 – 21.10.1931), \emph{Schriftsteller, Mediziner}|pwk}{ }Goldmann\pwindex{Goldmann, Paul 31.01.1865 – 25.09.1935@\textsc{Goldmann, Paul} (31.01.1865 – 25.09.1935), \emph{Schriftsteller, Journalist}|pwk} die Kritiken\pwindex{Bahr, Hermann 19.07.1863 – 15.01.1934@\textsc{Bahr, Hermann} (19.07.1863 – 15.01.1934), \emph{Schriftsteller, Kritiker}!junge Oesterreich1893-09-20 – 1893-10-07@\strich\emph{Das junge Österreich} {[}1893-09-20 – 1893-10-07{]}|pwkv} zwischen dem 27. und dem 30. 9. 1893
                  schickte, sie also zwischen dem 29. 9. und dem
                     2. 10. in Paris\oindex{Paris@\textbf{Paris}|pwk} ankamen. Goldmann\pwindex{Goldmann, Paul 31.01.1865 – 25.09.1935@\textsc{Goldmann, Paul} (31.01.1865 – 25.09.1935), \emph{Schriftsteller, Journalist}|pwk}s Brief
                  wäre dann nicht vor dem 29. 9. und nicht nach dem
                     2. 10. entstanden.}}}\label{K_L02718-2h} hingegen iſt
               niederträchtig, neidiſch, gemein, perſid. Und dieſe unverſchämte Schwindelei, was
                  \strikeout{Lit}{ }franzöſiſch\oindex{Frankreich@\textbf{Frankreich}|pwv}e Literatur-Kenntniß
               anlangt. \textsc{Courteline\pwindex{Courteline, Georges 25.06.1858 – 25.06.1929@\textsc{Courteline, Georges} (25.06.1858 – 25.06.1929), \emph{Schriftsteller}|pw}}, den Militär-Humoriſten\pwindex{Courteline, Georges 25.06.1858 – 25.06.1929@\textsc{Courteline, Georges} (25.06.1858 – 25.06.1929), \emph{Schriftsteller}|pwv}, in einer Linie mit \textsc{Lavedan\pwindex{Lavedan, Henri Leon 09.04.1859 – 4.9.1940@\textsc{Lavedan, Henri Léon} (09.04.1859 – 4.9.1940), \emph{Schriftsteller, Journalist}|pw}} zu nennen! {\pb}\textsc{Aurélien Scholl\pwindex{Scholl, Aurelien 1833 – 1902@\textsc{Scholl, Aurélien} (1833 – 1902), \emph{Schriftsteller, Journalist}|pw}}, den geiſtreichen \textsc{Chroniqueur\pwindex{Scholl, Aurelien 1833 – 1902@\textsc{Scholl, Aurélien} (1833 – 1902), \emph{Schriftsteller, Journalist}|pwv} à la Daniel Spitzer\pwindex{Spitzer, Daniel 03.06.1835 – 11.01.1893@\textsc{Spitzer, Daniel} (03.06.1835 – 11.01.1893), \emph{Schriftsteller, Journalist, Rechtsanwalt}|pw}}, mit \textsc{Lavedan\pwindex{Lavedan, Henri Leon 09.04.1859 – 4.9.1940@\textsc{Lavedan, Henri Léon} (09.04.1859 – 4.9.1940), \emph{Schriftsteller, Journalist}|pw}}, dem Analyti\textcolor{gray}{c}ker\pwindex{Lavedan, Henri Leon 09.04.1859 – 4.9.1940@\textsc{Lavedan, Henri Léon} (09.04.1859 – 4.9.1940), \emph{Schriftsteller, Journalist}|pwv}, zuſammenzuſtellen \textsc{etc}. Wirklich zu frech! Und diefer unerträgliche Styl! {\dots}\pend
           \pstart
           Grüß’ Dich Gott! {\\[\baselineskip]}Dein {\\[\baselineskip]}\spacefill\mbox{P. G.}\pend
           \leftskip=0em{}\endnumbering\briefempfaengerindex{Schnitzler, Arthur@\textsc{Schnitzler, Arthur}!zzzGoldmann, Paul@\emph{von Paul Goldmann}!1893-09-291@{{[}zwischen 29. 9. und
                  2. 10. 1893{]}}|)be}\mylabel{h}\end{ledgroupsized}\begin{anhang}\end{anhang}\newcommand{\dateiname}{L02718}\newcommand{\titel}{Paul Goldmann an Arthur Schnitzler, [zwischen 29. 9. und 2. 10. 1893]}\newcommand{\editorInnen}{Martin Anton Müller und Laura Untner}\input{../tex-inputs/latex-pdf-abspann}
      