%% latex-leseansicht-vorspann.tex
%% Vorspann für die Leseansicht.
%% Lädt die gemeinsame Datei latex-vorspann.tex mit nicht gesetztem Schalter.

\newif\ifkorrekturansicht
\korrekturansichtfalse

\input{../tex-inputs/latex-vorspann}


         
         \renewcommand{\erwaehntePersonen}{Personen: Hermann Bahr, Georges Courteline, Gabriel Engelsmann, Julius von Gans-Ludassy, Paul Goldmann, Robert Hirschfeld, Henri Léon Lavedan, Aurélien Scholl, Daniel Spitzer, Carl von Torresani-Lanzenfeld}
         \renewcommand{\erwaehnteInstitutionen}{Institutionen: An der schönen blauen Donau, Wiener Sonn- und Montagszeitung}
         \renewcommand{\erwaehnteOrte}{Orte: Frankreich, Paris, Wien}
         \renewcommand{\erwaehnteWerke}{Werke: Das junge Österreich, Deutsche Zeitung, Tagebuch, Wiener Sonn- und Montagszeitung, Zola über die Anonymität der Presse, Zwei Freunde Burckhards}
               \section[Paul Goldmann an Arthur Schnitzler, {[}zwischen 29. 9. und 2. 10. 1893{]}]{ Paul Goldmann an Arthur Schnitzler, {[}zwischen 29. 9. und
               2. 10. 1893{]}}\nopagebreak\mylabel{v}\rehead{ }\begin{ledgroupsized}[t]{13cm}\normalsize\beginnumbering\briefempfaengerindex{Schnitzler, Arthur@\textsc{Schnitzler, Arthur}!zzzGoldmann, Paul@\emph{von Paul Goldmann}!1893-10-021@{{[}zwischen 29. 9. und
                  2. 10. 1893{]}}|(be} \toendnotes[C]{\smallbreak\pagebreak[2]} \Standort{DLA, A:Schnitzler, HS.NZ85.1.3163.}
\physDesc{Brief, 1 Blatt, 2 Seiten, 553 Zeichen
\newline{}Handschrift: schwarze Tinte, deutsche Kurrent
\newline{}Schnitzler: 1) mit Bleistift das Datum »Octob 93.« vermerkt  2) mit rotem Buntstift eine Unterstreichung}\toendnotes[C]{\smallbreak}\pstart
           \noindent{}{\pb}Herzlichen Dank, liebſter Freund! Die \label{K_L02718-1v}\edtext{S. u. M.-Ztg.\orgindex{Wiener Sonn- und Montagszeitung@Wiener Sonn- und Montagszeitung|pw}}{\lemma{\textnormal{\emph{S. u. M.-Ztg.}}}\Cendnote{\textnormal{Unklarer Bezug auf die \emph{Wiener Sonn- und Montagszeitung}\orgindex{Wiener Sonn- und Montagszeitung@Wiener Sonn- und Montagszeitung|pwk}. Sofern es eine Reaktion
                  auf einen Text darstellt, der in der letzten oder vorletzten Nummer\pwindex{?? Werk@Nicht ermittelte Verfasserinnen und Verfasser!Wiener Sonn- und Montagszeitung1863 – 1936@\emph{Wiener Sonn- und Montagszeitung} {[}1863 – 1936{]}|pwkv} enthalten war, dürfte es sich um
                  diesen handeln: G. Engelsmann\pwindex{Engelsmann, Gabriel 1855-12-27 – 1941-11-18@\textsc{Engelsmann, Gabriel} (1855-12-27 – 1941-11-18), \emph{Journalist}|pwk}: \emph{Zola über die Anonymität der Presse}\pwindex{Zola ueber die Anonymitaet der Presse1893-09-25@\emph{Zola über die Anonymität der Presse} {[}1893-09-25{]}|pwk}. In: \emph{Wiener Sonn- und Montagszeitung}\pwindex{?? Werk@Nicht ermittelte Verfasserinnen und Verfasser!Wiener Sonn- und Montagszeitung1863 – 1936@\emph{Wiener Sonn- und Montagszeitung} {[}1863 – 1936{]}|pwk}, Jg. 31,
                     Nr. 38, 18.9.1893, S. 1–2. (Gabriel
                     Engelsmann\pwindex{Engelsmann, Gabriel 1855-12-27 – 1941-11-18@\textsc{Engelsmann, Gabriel} (1855-12-27 – 1941-11-18), \emph{Journalist}|pwk} hatte im Vorjahr auch zu der von Goldmann\pwindex{Goldmann, Paul 31.01.1865 – 25.09.1935@\textsc{Goldmann, Paul} (31.01.1865 – 25.09.1935), \emph{Schriftsteller, Journalist}|pwk} herausgegeben \emph{An
                     der schönen blauen Donau}\orgindex{der schoenen blauen Donau@An der schönen blauen Donau|pwk} beigetragen.) – Sofern es sich um eine Aussage
                  über Hermann Bahr\pwindex{Bahr, Hermann 19.07.1863 – 15.01.1934@\textsc{Bahr, Hermann} (19.07.1863 – 15.01.1934), \emph{Schriftsteller, Kritiker}|pwk} handelt, so dürfte diese
                  aus der Abrechnung\pwindex{Zwei Freunde Burckhards1893-07-24@\emph{Zwei Freunde Burckhards} {[}1893-07-24{]}|pwkv} stammen,
                  die am 24. 7. 1893 im Blatt\pwindex{?? Werk@Nicht ermittelte Verfasserinnen und Verfasser!Wiener Sonn- und Montagszeitung1863 – 1936@\emph{Wiener Sonn- und Montagszeitung} {[}1863 – 1936{]}|pwkv} stand: L. A. Terne. (Dr. Rob. Hirschfeld)\pwindex{Hirschfeld, Robert 17.09.1857 – 02.04.1914@\textsc{Hirschfeld, Robert} (17.09.1857 – 02.04.1914), \emph{Journalist, Kritiker}|pwk}: \emph{Zwei Freunde Burkhards}\pwindex{Zwei Freunde Burckhards1893-07-24@\emph{Zwei Freunde Burckhards} {[}1893-07-24{]}|pwk}. In: \emph{Wiener Sonn- und Montags-Zeitung}\pwindex{?? Werk@Nicht ermittelte Verfasserinnen und Verfasser!Wiener Sonn- und Montagszeitung1863 – 1936@\emph{Wiener Sonn- und Montagszeitung} {[}1863 – 1936{]}|pwk}, Jg. 31, Nr. 30,
                     S. 1–3.}}}\label{K_L02718-1h} iſt ganz hübſch; ehrliche Mühe, zu verſtehen, und
               ehrlicher und gutmüthiger \strikeout{Repſ} Reſpekt vor dem
               Talent. \label{K_L02718-2v}\edtext{\textsc{Bahr\pwindex{Bahr, Hermann 19.07.1863 – 15.01.1934@\textsc{Bahr, Hermann} (19.07.1863 – 15.01.1934), \emph{Schriftsteller, Kritiker}|pw}\pwindex{Bahr, Hermann 19.07.1863 – 15.01.1934@\textsc{Bahr, Hermann} (19.07.1863 – 15.01.1934), \emph{Schriftsteller, Kritiker}!junge Oesterreich1893-09-20 – 1893-10-07@\strich\emph{Das junge Österreich} {[}1893-09-20 – 1893-10-07{]}|pwv}}}{\lemma{\textnormal{\emph{Bahr}}}\Cendnote{\textnormal{Gemeint war der zweite Teil\pwindex{Bahr, Hermann 19.07.1863 – 15.01.1934@\textsc{Bahr, Hermann} (19.07.1863 – 15.01.1934), \emph{Schriftsteller, Kritiker}!junge Oesterreich1893-09-20 – 1893-10-07@\strich\emph{Das junge Österreich} {[}1893-09-20 – 1893-10-07{]}|pwkv} von Hermann Bahr\pwindex{Bahr, Hermann 19.07.1863 – 15.01.1934@\textsc{Bahr, Hermann} (19.07.1863 – 15.01.1934), \emph{Schriftsteller, Kritiker}|pwk}s dreiteiliger Feuilleton-Reihe\pwindex{Bahr, Hermann 19.07.1863 – 15.01.1934@\textsc{Bahr, Hermann} (19.07.1863 – 15.01.1934), \emph{Schriftsteller, Kritiker}!junge Oesterreich1893-09-20 – 1893-10-07@\strich\emph{Das junge Österreich} {[}1893-09-20 – 1893-10-07{]}|pwkv}{ }\emph{Das junge Österreich}\pwindex{Bahr, Hermann 19.07.1863 – 15.01.1934@\textsc{Bahr, Hermann} (19.07.1863 – 15.01.1934), \emph{Schriftsteller, Kritiker}!junge Oesterreich1893-09-20 – 1893-10-07@\strich\emph{Das junge Österreich} {[}1893-09-20 – 1893-10-07{]}|pwk}. Über Schnitzler\pwindex{Schnitzler, Arthur 15.05.1862 – 21.10.1931@\textsc{Schnitzler, Arthur} (15.05.1862 – 21.10.1931), \emph{Schriftsteller, Mediziner}|pwk} steht darin: »Arthur \so{Schnitzler}\pwindex{Schnitzler, Arthur 15.05.1862 – 21.10.1931@\textsc{Schnitzler, Arthur} (15.05.1862 – 21.10.1931), \emph{Schriftsteller, Mediziner}|pw} ist anders. Er ist ein großer Virtuose\pwindex{Schnitzler, Arthur 15.05.1862 – 21.10.1931@\textsc{Schnitzler, Arthur} (15.05.1862 – 21.10.1931), \emph{Schriftsteller, Mediziner}|pwv}, aber einer kleinen Note. Torresani\pwindex{Torresani-Lanzenfeld, Carl von 19.04.1846 – 16.04.1907@\textsc{Torresani-Lanzenfeld, Carl von} (19.04.1846 – 16.04.1907), \emph{Schriftsteller}|pw} streut aus reichen Krügen, ohne die einzelne
                     Gabe zu achten. Schnitzler\pwindex{Schnitzler, Arthur 15.05.1862 – 21.10.1931@\textsc{Schnitzler, Arthur} (15.05.1862 – 21.10.1931), \emph{Schriftsteller, Mediziner}|pw} darf nicht
                     verschwenden. Er muß sparen. Er hat wenig. So will er es denn mit der
                     zärtlichsten Sorge, mit erfinderischer Mühe, mit geduldigem Geize schleifen,
                     bis das Geringe durch seine unermüdlichen Künste Adel und Würde verdient. Was
                     er bringt, ist nichtig. Aber wie er es bringt, darf gelten. Die großen Züge der
                     Zeit, Leidenschaften, Stürme, Erschütterungen der Menschen, die ungestüme
                     Pracht der Welt an Farben und an Klängen ist ihm versagt. Er weiß immer nur
                     einen einzigen Menschen, ja nur ein einziges Gefühl zu gestalten. Aber dieser
                     Gestalt gibt er Vollkommenheit, Vollendung. So ist er recht der \textsc{\begin{otherlanguage}{french}artiste\pwindex{Schnitzler, Arthur 15.05.1862 – 21.10.1931@\textsc{Schnitzler, Arthur} (15.05.1862 – 21.10.1931), \emph{Schriftsteller, Mediziner}|pwv}\end{otherlanguage}} nach dem Herzen des ›Parnasses‹, jener Franz\oindex{Frankreich@\textbf{Frankreich}|pwv}osen, welche um den Werth an Gehalt nicht bekümmert,
                     nur in der Fassung Pflicht und Verdienst der Kunst erkennen und als eitel
                     verachten, was nicht seltene Nuance, malendes Objectiv, gesuchte Metapher
                     ist.« (\emph{Das junge Österreich. II}\pwindex{Bahr, Hermann 19.07.1863 – 15.01.1934@\textsc{Bahr, Hermann} (19.07.1863 – 15.01.1934), \emph{Schriftsteller, Kritiker}!junge Oesterreich1893-09-20 – 1893-10-07@\strich\emph{Das junge Österreich} {[}1893-09-20 – 1893-10-07{]}|pwk}. In: \emph{Deutsche Zeitung}\pwindex{?? Werk@Nicht ermittelte Verfasserinnen und Verfasser!Deutsche Zeitung1871 – 1907@\emph{Deutsche Zeitung} {[}1871 – 1907{]}|pwk}, Jg. 23, Nr. 7.813, 27. 9. 1893, Morgen-Ausgabe, S. 1–3, hier
                     S. 1) Schnitzler\pwindex{Schnitzler, Arthur 15.05.1862 – 21.10.1931@\textsc{Schnitzler, Arthur} (15.05.1862 – 21.10.1931), \emph{Schriftsteller, Mediziner}|pwk} notierte sich
                  dazu am 27. 9. 1893
                  im \emph{Tagebuch}\pwindex{Schnitzler, Arthur 15.05.1862 – 21.10.1931@\textsc{Schnitzler, Arthur} (15.05.1862 – 21.10.1931), \emph{Schriftsteller, Mediziner}!Tagebuch1981 – 2000@\strich\emph{Tagebuch} {[}1981 – 2000{]}|pwk}: »Ich sei ein großer
                     Virtuos auf kleinem Ton; jedoch \begin{otherlanguage}{french}apporteur du
                        neuf\end{otherlanguage}, etc.;– ich war ärgerlich.«}}}\label{K_L02718-2h} hingegen iſt
               niederträchtig, neidiſch, gemein, \label{K_L02718-3v}\edtext{perfid}{\lemma{\textnormal{\emph{perfid}}}\Cendnote{\textnormal{Dieser Ausdruck Goldmann\pwindex{Goldmann, Paul 31.01.1865 – 25.09.1935@\textsc{Goldmann, Paul} (31.01.1865 – 25.09.1935), \emph{Schriftsteller, Journalist}|pwk}s ermöglicht letztlich die ungefähre
                  Datierung des undatierten Briefes: Bahr\pwindex{Bahr, Hermann 19.07.1863 – 15.01.1934@\textsc{Bahr, Hermann} (19.07.1863 – 15.01.1934), \emph{Schriftsteller, Kritiker}|pwk}s
                     Kritik\pwindex{Bahr, Hermann 19.07.1863 – 15.01.1934@\textsc{Bahr, Hermann} (19.07.1863 – 15.01.1934), \emph{Schriftsteller, Kritiker}!junge Oesterreich1893-09-20 – 1893-10-07@\strich\emph{Das junge Österreich} {[}1893-09-20 – 1893-10-07{]}|pwkv} erschien am 27. 9. 1893. Schnitzler\pwindex{Schnitzler, Arthur 15.05.1862 – 21.10.1931@\textsc{Schnitzler, Arthur} (15.05.1862 – 21.10.1931), \emph{Schriftsteller, Mediziner}|pwk} datierte den Brief beziehungsweise
                  das Empfangsdatum desselben auf »Octob 93«. Spätestens am 4. 10. 1893 muss Schnitzler\pwindex{Schnitzler, Arthur 15.05.1862 – 21.10.1931@\textsc{Schnitzler, Arthur} (15.05.1862 – 21.10.1931), \emph{Schriftsteller, Mediziner}|pwk} den
                  Brief erhalten haben, insofern im \emph{Tagebuch}\pwindex{Schnitzler, Arthur 15.05.1862 – 21.10.1931@\textsc{Schnitzler, Arthur} (15.05.1862 – 21.10.1931), \emph{Schriftsteller, Mediziner}!Tagebuch1981 – 2000@\strich\emph{Tagebuch} {[}1981 – 2000{]}|pwk}-Eintrag des genannten Tages
                  Folgendes zu lesen ist: »Ludaßy\pwindex{Gans-Ludassy, Julius von 13.04.1858 – 30.09.1922@\textsc{Gans-Ludassy, Julius von} (13.04.1858 – 30.09.1922), \emph{Schriftsteller, Journalist, Herausgeber}|pw} findet (wie Paul G.\pwindex{Goldmann, Paul 31.01.1865 – 25.09.1935@\textsc{Goldmann, Paul} (31.01.1865 – 25.09.1935), \emph{Schriftsteller, Journalist}|pw}) die Kritik\pwindex{Bahr, Hermann 19.07.1863 – 15.01.1934@\textsc{Bahr, Hermann} (19.07.1863 – 15.01.1934), \emph{Schriftsteller, Kritiker}!junge Oesterreich1893-09-20 – 1893-10-07@\strich\emph{Das junge Österreich} {[}1893-09-20 – 1893-10-07{]}|pwv} von Bahr\pwindex{Bahr, Hermann 19.07.1863 – 15.01.1934@\textsc{Bahr, Hermann} (19.07.1863 – 15.01.1934), \emph{Schriftsteller, Kritiker}|pw}
                     perfid.« Anzunehmen ist, dass Schnitzler\pwindex{Schnitzler, Arthur 15.05.1862 – 21.10.1931@\textsc{Schnitzler, Arthur} (15.05.1862 – 21.10.1931), \emph{Schriftsteller, Mediziner}|pwk}{ }Goldmann\pwindex{Goldmann, Paul 31.01.1865 – 25.09.1935@\textsc{Goldmann, Paul} (31.01.1865 – 25.09.1935), \emph{Schriftsteller, Journalist}|pwk} die Kritik\pwindex{Bahr, Hermann 19.07.1863 – 15.01.1934@\textsc{Bahr, Hermann} (19.07.1863 – 15.01.1934), \emph{Schriftsteller, Kritiker}!junge Oesterreich1893-09-20 – 1893-10-07@\strich\emph{Das junge Österreich} {[}1893-09-20 – 1893-10-07{]}|pwkv} am 27. oder 28. 9. 1893 schickte, sodass
                     Goldmann\pwindex{Goldmann, Paul 31.01.1865 – 25.09.1935@\textsc{Goldmann, Paul} (31.01.1865 – 25.09.1935), \emph{Schriftsteller, Journalist}|pwk}s Replik zwischen dem 29. 9. 1893 und dem 2. 10. 1893 verfasst sein dürfte.}}}\label{K_L02718-3h}. Und dieſe unverſchämte
               Schwindelei, was \strikeout{Lit}{ }franzöſiſch\oindex{Frankreich@\textbf{Frankreich}|pwv}e Literatur-Kenntniß
               anlangt. \label{K_L02718-4v}\edtext{\textsc{Courteline\pwindex{Courteline, Georges 25.06.1858 – 25.06.1929@\textsc{Courteline, Georges} (25.06.1858 – 25.06.1929), \emph{Schriftsteller}|pw}}, den Militär-Humoriſten\pwindex{Courteline, Georges 25.06.1858 – 25.06.1929@\textsc{Courteline, Georges} (25.06.1858 – 25.06.1929), \emph{Schriftsteller}|pwv}, in einer Linie mit \textsc{Lavedan\pwindex{Lavedan, Henri Leon 09.04.1859 – 4.9.1940@\textsc{Lavedan, Henri Léon} (09.04.1859 – 4.9.1940), \emph{Schriftsteller, Journalist}|pw}}}{\lemma{\textnormal{\emph{Courteline, … Lavedan}}}\Cendnote{\textnormal{Die weiteren von Goldmann\pwindex{Goldmann, Paul 31.01.1865 – 25.09.1935@\textsc{Goldmann, Paul} (31.01.1865 – 25.09.1935), \emph{Schriftsteller, Journalist}|pwk} kritisierten Aussagen\pwindex{Bahr, Hermann 19.07.1863 – 15.01.1934@\textsc{Bahr, Hermann} (19.07.1863 – 15.01.1934), \emph{Schriftsteller, Kritiker}!junge Oesterreich1893-09-20 – 1893-10-07@\strich\emph{Das junge Österreich} {[}1893-09-20 – 1893-10-07{]}|pwkv} finden sich im ersten Teil\pwindex{Bahr, Hermann 19.07.1863 – 15.01.1934@\textsc{Bahr, Hermann} (19.07.1863 – 15.01.1934), \emph{Schriftsteller, Kritiker}!junge Oesterreich1893-09-20 – 1893-10-07@\strich\emph{Das junge Österreich} {[}1893-09-20 – 1893-10-07{]}|pwkv}: Hermann Bahr\pwindex{Bahr, Hermann 19.07.1863 – 15.01.1934@\textsc{Bahr, Hermann} (19.07.1863 – 15.01.1934), \emph{Schriftsteller, Kritiker}|pwk}: \emph{Das junge Österreich. I}\pwindex{Bahr, Hermann 19.07.1863 – 15.01.1934@\textsc{Bahr, Hermann} (19.07.1863 – 15.01.1934), \emph{Schriftsteller, Kritiker}!junge Oesterreich1893-09-20 – 1893-10-07@\strich\emph{Das junge Österreich} {[}1893-09-20 – 1893-10-07{]}|pwk}. In: \emph{Deutsche Zeitung}\pwindex{?? Werk@Nicht ermittelte Verfasserinnen und Verfasser!Deutsche Zeitung1871 – 1907@\emph{Deutsche Zeitung} {[}1871 – 1907{]}|pwk}, Jg. 23, Nr. 7.806, 20. 9. 1893, Morgen-Ausgabe, S. 1–2.}}}\label{K_L02718-4h}
               zu nennen! {\pb}\textsc{Aurélien Scholl\pwindex{Scholl, Aurelien 1833-07-13 – 1902@\textsc{Scholl, Aurélien} (1833-07-13 – 1902), \emph{Schriftsteller, Journalist}|pw}}, den geiſtreichen \textsc{Chroniqueur\pwindex{Scholl, Aurelien 1833-07-13 – 1902@\textsc{Scholl, Aurélien} (1833-07-13 – 1902), \emph{Schriftsteller, Journalist}|pwv} à la Daniel Spitzer\pwindex{Spitzer, Daniel 03.06.1835 – 11.01.1893@\textsc{Spitzer, Daniel} (03.06.1835 – 11.01.1893), \emph{Schriftsteller, Journalist, Rechtsanwalt}|pw}}, mit \textsc{Lavedan\pwindex{Lavedan, Henri Leon 09.04.1859 – 4.9.1940@\textsc{Lavedan, Henri Léon} (09.04.1859 – 4.9.1940), \emph{Schriftsteller, Journalist}|pw}}, dem Analytiker\pwindex{Lavedan, Henri Leon 09.04.1859 – 4.9.1940@\textsc{Lavedan, Henri Léon} (09.04.1859 – 4.9.1940), \emph{Schriftsteller, Journalist}|pwv},
               zuſammenzuſtellen \textsc{etc}. Wirklich zu frech! Und diefer
               unerträgliche Styl! {\dots}\pend
           \pstart
           Grüß’ Dich Gott! {\\[\baselineskip]}Dein {\\[\baselineskip]}\spacefill\mbox{P. G.}\pend
           \leftskip=0em{}
         
         \endnumbering\mylabel{h}\end{ledgroupsized}  \newcommand{\dateiname}{L02718}\newcommand{\titel}{Paul Goldmann an Arthur Schnitzler, [zwischen 29. 9. und 2. 10. 1893]}\newcommand{\editorInnen}{Martin Anton Müller und Laura Untner}%% latex-leseansicht-abspann.tex
%% Abspann für die Leseansicht.
%% Der Schalter \ifkorrekturansicht ist bereits durch den Vorspann gesetzt.

%% latex-abspann.tex
%% Gemeinsamer Abspann für Korrekturansicht und Leseansicht.
%% Setzt den Schalter \ifkorrekturansicht voraus (gesetzt in den
%% einbindenden Dateien latex-korrekturansicht-abspann.tex bzw.
%% latex-leseansicht-abspann.tex).
%% ---------------------------------------------------------------

\normalsize

% Das esempio-Environment wird nur in der Leseansicht benötigt
\ifkorrekturansicht\else
\newenvironment{esempio}[3]%
{
    \vspace{1.5ex}
    \rlap{\underline{#1}}
    \par
    \setlength{\parindent}{0cm}
    \nopagebreak
    \leftskip=#2cm
    \rightskip=#3cm
}
{
    \par
}
\fi

\doendnotes{C}
\bigskip
\vfill

\clearpage

\footnotesize

\ifkorrekturansicht
  \lohead{\textsc{register}}
\fi

% theindex-Environment neu definieren ohne reledmac
\makeatletter
\renewenvironment{theindex}{%
  \ifkorrekturansicht
    \section*{\indexname}%
  \else
    \subsubsection*{Index der erwähnten Entitäten}%
  \fi
  \setlength{\parindent}{0pt}%
  \setlength{\parskip}{0pt plus 0.3pt}%
  \let\item\@idxitem
}{%
  \ifkorrekturansicht\clearpage\fi
}
\makeatother

\IfFileExists{\jobname-pw.ind}{\input{\jobname-pw.ind}}{}

% Quellenangabe nur in der Leseansicht
\ifkorrekturansicht\else
% Fallback-Definitionen, falls die .tex-Datei \titel etc. nicht gesetzt hat
\providecommand{\titel}{}
\providecommand{\editorInnen}{}
\providecommand{\dateiname}{\jobname}

\vspace{3cm}

\vfill

\footnotesize
\textsc{Quelle}: \titel. Herausgegeben von {\editorInnen}. In: \emph{Arthur Schnitzler: Briefwechsel mit Autorinnen und Autoren}.
 Digitale Edition, https://schnitzler-briefe.acdh.oeaw.ac.at/{\dateiname}.html (Stand \today)
\fi

\end{document}


      