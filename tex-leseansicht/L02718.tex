%% latex-korrekturansicht-vorspann.tex
%% Vorspann für die Korrekturansicht.
%% Lädt die gemeinsame Datei latex-vorspann.tex mit gesetztem Schalter.

\newif\ifkorrekturansicht
\korrekturansichttrue

\input{../tex-inputs/latex-vorspann}


\section[Paul Goldmann an Arthur Schnitzler, {[}zwischen 29. 9. und 2. 10. 1893{]}]{L02718 Paul Goldmann an Arthur Schnitzler, {[}zwischen 29. 9. und
               2. 10. 1893{]}}
\nopagebreak\mylabel{L02718v}
\rehead{ }\normalsize\beginnumbering\briefempfaengerindex{Schnitzler, Arthur@\textsc{Schnitzler, Arthur}!zzzGoldmann, Paul@\emph{von Paul Goldmann}!1893-10-021@{{[}zwischen 29. 9. und
                  2. 10. 1893{]}}|(be}
\toendnotes[C]{\smallbreak\pagebreak[2]}\Standort{DLA, A:Schnitzler, HS.NZ85.1.3163.}
\physDesc{Brief, 1 Blatt, 2 Seiten, 553 Zeichen
\newline{}Handschrift: schwarze Tinte, deutsche Kurrent
\newline{}Schnitzler: 1) mit Bleistift das Datum »Octob 93.« vermerkt  2) mit rotem Buntstift eine Unterstreichung}\toendnotes[C]{\smallbreak}
\pstart
           \noindent{}{\pb}Herzlichen Dank, liebſter Freund! Die \label{K_L02718-1v}\edtext{S. u. M.-Ztg.\orgindex{Wiener Sonn- und Montagszeitung@Wiener Sonn- und Montagszeitung|pw}}{\lemma{\textnormal{\emph{S. u. M.-Ztg.}}}\Cendnote{\textnormal{Unklarer Bezug auf die \emph{Wiener Sonn- und Montags-Zeitung}\orgindex{Wiener Sonn- und Montagszeitung@Wiener Sonn- und Montagszeitung|pwk}. Sofern die vorliegende Stelle eine Reaktion
                  auf einen Text darstellt, der in der letzten oder vorletzten Nummer\pwindex{Wiener Sonn- und Montagszeitung@\emph{Wiener Sonn- und Montagszeitung}|pwkv} enthalten war, dürfte es sich um
                  diesen handeln: G. Engelsmann\pwindex{Engelsmann, Gabriel 1855-12-27 – 1941-11-18@\textsc{Engelsmann, Gabriel} (1855-12-27 – 1941-11-18), \emph{Journalist/Journalistin}|pwk}: \emph{Zola über die Anonymität der Presse}\pwindex{Zola ueber die Anonymitaet der Presse@\emph{Zola über die Anonymität der Presse}|pwk}. In: \emph{Wiener Sonn- und Montags-Zeitung}\pwindex{Wiener Sonn- und Montagszeitung@\emph{Wiener Sonn- und Montagszeitung}|pwk}, Jg. 31,
                     Nr. 38, 18. 9. 1893, S. 1–2. (Gabriel
                     Engelsmann\pwindex{Engelsmann, Gabriel 1855-12-27 – 1941-11-18@\textsc{Engelsmann, Gabriel} (1855-12-27 – 1941-11-18), \emph{Journalist/Journalistin}|pwk} hatte im Vorjahr auch zu der von Goldmann\pwindex{Goldmann, Paul 31.01.1865 – 25.09.1935@\textsc{Goldmann, Paul} (31.01.1865 – 25.09.1935), \emph{Schriftsteller/Schriftstellerin, Journalist/Journalistin}|pwk} herausgegeben \emph{An
                     der schönen blauen Donau}\orgindex{der schoenen blauen Donau@An der schönen blauen Donau|pwk} beigetragen.) – Sofern es sich bei der Stelle um eine Aussage
                  über Hermann Bahr\pwindex{Bahr, Hermann 19.07.1863 – 15.01.1934@\textsc{Bahr, Hermann} (19.07.1863 – 15.01.1934), \emph{Schriftsteller/Schriftstellerin, Kritiker/Kritikerin}|pwk} handelt, so dürfte diese
                  aus der Abrechnung\pwindex{Zwei Freunde Burckhards@\emph{Zwei Freunde Burckhards}|pwkv} stammen,
                  die am 24. 7. 1893 im Blatt\pwindex{Wiener Sonn- und Montagszeitung@\emph{Wiener Sonn- und Montagszeitung}|pwkv} stand: L. A. Terne. (Dr. Rob. Hirschfeld)\pwindex{Hirschfeld, Robert 17.09.1857 – 02.04.1914@\textsc{Hirschfeld, Robert} (17.09.1857 – 02.04.1914), \emph{Journalist/Journalistin, Musikkritiker/Musikkritikerin}|pwk}: \emph{Zwei Freunde Burkhards}\pwindex{Zwei Freunde Burckhards@\emph{Zwei Freunde Burckhards}|pwk}. In: \emph{Wiener Sonn- und Montags-Zeitung}\pwindex{Wiener Sonn- und Montagszeitung@\emph{Wiener Sonn- und Montagszeitung}|pwk}, Jg. 31, Nr. 30,
                     S. 1–3.}}}\label{K_L02718-1} iſt ganz hübſch; ehrliche Mühe, zu verſtehen, und
               ehrlicher und gutmüthiger \strikeout{Repſ} Reſpekt vor dem
               Talent. \label{K_L02718-2v}\edtext{\textsc{Bahr\pwindex{Bahr, Hermann 19.07.1863 – 15.01.1934@\textsc{Bahr, Hermann} (19.07.1863 – 15.01.1934), \emph{Schriftsteller/Schriftstellerin, Kritiker/Kritikerin}|pw}\pwindex{junge Oesterreich@\emph{Das junge Österreich}|pwv}}}{\lemma{\textnormal{\emph{Bahr}}}\Cendnote{\textnormal{Gemeint war der zweite Teil\pwindex{junge Oesterreich@\emph{Das junge Österreich}|pwkv} von Hermann Bahrs\pwindex{Bahr, Hermann 19.07.1863 – 15.01.1934@\textsc{Bahr, Hermann} (19.07.1863 – 15.01.1934), \emph{Schriftsteller/Schriftstellerin, Kritiker/Kritikerin}|pwk} dreiteiliger Feuilleton-Serie\pwindex{junge Oesterreich@\emph{Das junge Österreich}|pwkv}{ }\emph{Das junge Österreich}\pwindex{junge Oesterreich@\emph{Das junge Österreich}|pwk}. Über Schnitzler steht darin: »Arthur \so{Schnitzler} ist anders. Er ist ein großer Virtuose, aber einer kleinen Note. Torresani\pwindex{Torresani-Lanzenfeld, Carl von 19.04.1846 – 16.04.1907@\textsc{Torresani-Lanzenfeld, Carl von} (19.04.1846 – 16.04.1907), \emph{Schriftsteller/Schriftstellerin, Offizier/Offizierin}|pw} streut aus reichen Krügen, ohne die einzelne
                     Gabe zu achten. Schnitzler darf nicht
                     verschwenden. Er muß sparen. Er hat wenig. So will er es denn mit der
                     zärtlichsten Sorge, mit erfinderischer Mühe, mit geduldigem Geize schleifen,
                     bis das Geringe durch seine unermüdlichen Künste Adel und Würde verdient. Was
                     er bringt, ist nichtig. Aber wie er es bringt, darf gelten. Die großen Züge der
                     Zeit, Leidenschaften, Stürme, Erschütterungen der Menschen, die ungestüme
                     Pracht der Welt an Farben und an Klängen ist ihm versagt. Er weiß immer nur
                     einen einzigen Menschen, ja nur ein einziges Gefühl zu gestalten. Aber dieser
                     Gestalt gibt er Vollkommenheit, Vollendung. So ist er recht der \textsc{\begin{otherlanguage}{french}artiste\end{otherlanguage}} nach dem Herzen des ›Parnasses‹, jener Franz\oindex{Frankreich@\textbf{Frankreich}, \emph{A.PCLI}|pwv}osen, welche um den Werth an Gehalt nicht bekümmert,
                     nur in der Fassung Pflicht und Verdienst der Kunst erkennen und als eitel
                     verachten, was nicht seltene Nuance, malendes Objectiv, gesuchte Metapher
                     ist.« (\emph{Das junge Österreich. II}\pwindex{junge Oesterreich@\emph{Das junge Österreich}|pwk}. In: \emph{Deutsche Zeitung}\pwindex{Deutsche Zeitung@\emph{Deutsche Zeitung}|pwk}, Jg. 23, Nr. 7813, 27. 9. 1893, Morgen-Ausgabe, S. 1–3, hier
                     S. 1) Schnitzler notierte 
                  dazu am 27. 9. 1893
                  im \emph{Tagebuch}\pwindex{Tagebuch@\emph{Tagebuch}|pwk}: »Ich sei ein großer
                     Virtuos auf kleinem Ton; jedoch \begin{otherlanguage}{french}apporteur du
                        neuf\end{otherlanguage}, etc.; – ich war ärgerlich.«}}}\label{K_L02718-2} hingegen iſt
               niederträchtig, neidiſch, gemein, \label{K_L02718-3v}\edtext{perfid}{\lemma{\textnormal{\emph{perfid}}}\Cendnote{\textnormal{Dieser Ausdruck Goldmanns\pwindex{Goldmann, Paul 31.01.1865 – 25.09.1935@\textsc{Goldmann, Paul} (31.01.1865 – 25.09.1935), \emph{Schriftsteller/Schriftstellerin, Journalist/Journalistin}|pwk} ermöglicht letztlich die ungefähre
                  Datierung des undatierten Briefes: Bahrs\pwindex{Bahr, Hermann 19.07.1863 – 15.01.1934@\textsc{Bahr, Hermann} (19.07.1863 – 15.01.1934), \emph{Schriftsteller/Schriftstellerin, Kritiker/Kritikerin}|pwk}{ }Kritik\pwindex{junge Oesterreich@\emph{Das junge Österreich}|pwkv} erschien am 27. 9. 1893. Schnitzler datierte den Brief beziehungsweise
                  das Empfangsdatum desselben auf »Octob 93«. Spätestens am 4. 10. 1893 muss Schnitzler den
                  Brief erhalten haben, insofern im \emph{Tagebuch}\pwindex{Tagebuch@\emph{Tagebuch}|pwk}-Eintrag des genannten Tages
                  Folgendes zu lesen ist: »Ludaßy\pwindex{Gans-Ludassy, Julius von 13.04.1858 – 30.09.1922@\textsc{Gans-Ludassy, Julius von} (13.04.1858 – 30.09.1922), \emph{Schriftsteller/Schriftstellerin, Journalist/Journalistin, Herausgeber/Herausgeberin}|pw} findet (wie Paul G.\pwindex{Goldmann, Paul 31.01.1865 – 25.09.1935@\textsc{Goldmann, Paul} (31.01.1865 – 25.09.1935), \emph{Schriftsteller/Schriftstellerin, Journalist/Journalistin}|pw}) die Kritik\pwindex{junge Oesterreich@\emph{Das junge Österreich}|pwv} von Bahr\pwindex{Bahr, Hermann 19.07.1863 – 15.01.1934@\textsc{Bahr, Hermann} (19.07.1863 – 15.01.1934), \emph{Schriftsteller/Schriftstellerin, Kritiker/Kritikerin}|pw}
                     perfid.« Anzunehmen ist, dass Schnitzler{ }Goldmann\pwindex{Goldmann, Paul 31.01.1865 – 25.09.1935@\textsc{Goldmann, Paul} (31.01.1865 – 25.09.1935), \emph{Schriftsteller/Schriftstellerin, Journalist/Journalistin}|pwk} die Kritik\pwindex{junge Oesterreich@\emph{Das junge Österreich}|pwkv} am 27. 9. 1893 oder 28. 9. 1893 geschickt hat, sodass
                     Goldmanns\pwindex{Goldmann, Paul 31.01.1865 – 25.09.1935@\textsc{Goldmann, Paul} (31.01.1865 – 25.09.1935), \emph{Schriftsteller/Schriftstellerin, Journalist/Journalistin}|pwk} Replik zwischen dem 29. 9. 1893 und dem 2. 10. 1893 verfasst worden sein dürfte.}}}\label{K_L02718-3}. Und dieſe unverſchämte
               Schwindelei, was \strikeout{Lit}{ }franzöſiſch\oindex{Frankreich@\textbf{Frankreich}, \emph{A.PCLI}|pwv}e Literatur-Kenntniß
               anlangt. \label{K_L02718-4v}\edtext{\textsc{Courteline\pwindex{Courteline, Georges 25.06.1858 – 25.06.1929@\textsc{Courteline, Georges} (25.06.1858 – 25.06.1929), \emph{Schriftsteller/Schriftstellerin}|pw}}, den Militär-Humoriſten\pwindex{Courteline, Georges 25.06.1858 – 25.06.1929@\textsc{Courteline, Georges} (25.06.1858 – 25.06.1929), \emph{Schriftsteller/Schriftstellerin}|pwv}, in einer Linie mit \textsc{Lavedan\pwindex{Lavedan, Henri Leon 09.04.1859 – 4.9.1940@\textsc{Lavedan, Henri Léon} (09.04.1859 – 4.9.1940), \emph{Schriftsteller/Schriftstellerin, Journalist/Journalistin}|pw}}}{\lemma{\textnormal{\emph{Courteline, … Lavedan}}}\Cendnote{\textnormal{Die weiteren von Goldmann\pwindex{Goldmann, Paul 31.01.1865 – 25.09.1935@\textsc{Goldmann, Paul} (31.01.1865 – 25.09.1935), \emph{Schriftsteller/Schriftstellerin, Journalist/Journalistin}|pwk} kritisierten Aussagen\pwindex{junge Oesterreich@\emph{Das junge Österreich}|pwkv} finden sich im ersten Teil\pwindex{junge Oesterreich@\emph{Das junge Österreich}|pwkv}: Hermann Bahr\pwindex{Bahr, Hermann 19.07.1863 – 15.01.1934@\textsc{Bahr, Hermann} (19.07.1863 – 15.01.1934), \emph{Schriftsteller/Schriftstellerin, Kritiker/Kritikerin}|pwk}: \emph{Das junge Österreich. I}\pwindex{junge Oesterreich@\emph{Das junge Österreich}|pwk}. In: \emph{Deutsche Zeitung}\pwindex{Deutsche Zeitung@\emph{Deutsche Zeitung}|pwk}, Jg. 23, Nr. 7806, 20. 9. 1893, Morgen-Ausgabe, S. 1–2.}}}\label{K_L02718-4}
               zu nennen! {\pb}\textsc{Aurélien Scholl\pwindex{Scholl, Aurelien 1833-07-13 – 1902@\textsc{Scholl, Aurélien} (1833-07-13 – 1902), \emph{Schriftsteller/Schriftstellerin, Journalist/Journalistin}|pw}}, den geiſtreichen \textsc{Chroniqueur\pwindex{Scholl, Aurelien 1833-07-13 – 1902@\textsc{Scholl, Aurélien} (1833-07-13 – 1902), \emph{Schriftsteller/Schriftstellerin, Journalist/Journalistin}|pwv} à la Daniel Spitzer\pwindex{Spitzer, Daniel 03.06.1835 – 11.01.1893@\textsc{Spitzer, Daniel} (03.06.1835 – 11.01.1893), \emph{Schriftsteller/Schriftstellerin, Journalist/Journalistin, Rechtsanwalt/Rechtsanwältin}|pw}}, mit \textsc{Lavedan\pwindex{Lavedan, Henri Leon 09.04.1859 – 4.9.1940@\textsc{Lavedan, Henri Léon} (09.04.1859 – 4.9.1940), \emph{Schriftsteller/Schriftstellerin, Journalist/Journalistin}|pw}}, dem Analytiker\pwindex{Lavedan, Henri Leon 09.04.1859 – 4.9.1940@\textsc{Lavedan, Henri Léon} (09.04.1859 – 4.9.1940), \emph{Schriftsteller/Schriftstellerin, Journalist/Journalistin}|pwv},
               zuſammenzuſtellen \textsc{etc}. Wirklich zu frech! Und diefer
               unerträgliche Styl! {\dots}\pend
           
\pstart
           Grüß’ Dich Gott! {\\[\baselineskip]}Dein {\\[\baselineskip]}\spacefill\mbox{P. G.}\pend
           \leftskip=0em{}\selectlanguage{ngerman}\endnumbering\briefempfaengerindex{Schnitzler, Arthur@\textsc{Schnitzler, Arthur}!zzzGoldmann, Paul@\emph{von Paul Goldmann}!1893-09-291@{{[}zwischen 29. 9. und
                  2. 10. 1893{]}}|)be}\mylabel{L02718h}  \normalsize

\doendnotes{C}
\bigskip
\vfill

\clearpage

\footnotesize

\lohead{\textsc{register}}

% Definiere theindex-Environment komplett neu ohne reledmac
\makeatletter
\renewenvironment{theindex}{%
  \section*{\indexname}%
  \setlength{\parindent}{0pt}%
  \setlength{\parskip}{0pt plus 0.3pt}%
  \let\item\@idxitem
}{%
  \clearpage
}
\makeatother

\IfFileExists{\jobname-pw.ind}{\input{\jobname-pw.ind}}{}

\end{document}

      