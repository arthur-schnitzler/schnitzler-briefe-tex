%% latex-leseansicht-vorspann.tex
%% Vorspann für die Leseansicht.
%% Lädt die gemeinsame Datei latex-vorspann.tex mit nicht gesetztem Schalter.

\newif\ifkorrekturansicht
\korrekturansichtfalse

\input{../tex-inputs/latex-vorspann}


\section[Georg Brandes an Arthur Schnitzler, 17. 5. 1901]{L01121 Georg Brandes an Arthur Schnitzler, 17. 5. 1901}
\nopagebreak\mylabel{L01121v}
\rehead{ }\normalsize\beginnumbering\briefempfaengerindex{Schnitzler, Arthur@\textsc{Schnitzler, Arthur}!zzzBrandes, Georg@\emph{von Georg Brandes}!1901-05-171@{17. 5. 1901}|(be}
\toendnotes[C]{\smallbreak\pagebreak[2]}
\correspDesc{Versand  durch Georg Brandes am 17. 5. 1901 in Opatija
\newline{}Erhalt  durch Arthur Schnitzler im Zeitraum [18. 5. 1901
                  – 22. 5. 1901?] in Wien}\toendnotes[C]{\smallbreak}
\Standort{CUL, Schnitzler, B 17.}
\physDesc{Brief, 1 Blatt, 2 Seiten, 938 Zeichen
\newline{}Handschrift: schwarze Tinte, lateinische Kurrent
\newline{}Ordnung: 1) mit Bleistift von unbekannter Hand nummeriert: »\strikeout{22}«  2) mit Bleistift von unbekannter Hand nummeriert:
                                    »23«}
\buchAbdrucke{\weitereDrucke{Georg Brandes, Arthur Schnitzler: \emph{Ein Briefwechsel}. Herausgegeben von Kurt Bergel. Bern: \emph{Francke} 1956, S. 87.} }
\pstart
           \centering{}{\pb}Abbazia\oindex{Opatija@\textbf{Opatija}, \emph{Hauptstadt}|pw}{\\}Hotel Quitta\oindex{Pension Quitta@\textbf{Pension Quitta}, \emph{Hotel}|pw}\pend
           
\pstart
           \raggedleft{}17 May 1901\pend
           
\pstart{}Verehrter Freund\pend\vspace{0.5em}
\pstart
           Anbei die 30 Gulden. Hier ist überzogen, durchaus nicht sehr warm und ich durch die
               Unpünktlichkeit meiner Mitmenschen völlig allein, mindestens \uline{zwei} Tage, worüber ich wüthe, da ich diese zwei Tage
               vorzüglich in Wien\oindex{Wien@\textbf{Wien}, \emph{Verwaltungsgebiet}|pw} zugebracht haben könnte,
               während ich mich hier über die verlorene Zeit ärgere.\pend
           
\pstart
           Der Weg von Mattuglie\oindex{Matulji@\textbf{Matulji}, \emph{Hauptstadt}|pw} nach Abbazia\oindex{Opatija@\textbf{Opatija}, \emph{Hauptstadt}|pw} erinnert ein wenig an den von Taormina\oindex{Taormina@\textbf{Taormina}, \emph{Hauptstadt}|pw} nach Giardini\oindex{Giardini Naxos@\textbf{Giardini Naxos}, \emph{Verwaltungsgebiet}|pw}.
               Hier blühen die Rosen, nur nicht die meinen.\pend
           
\pstart
           Haben Sie aufrichtigen und herzlichen Dank für alle mir erwiesenen Dienste. Ich, der
               ich selbst überlaufen werde, weiss {\pb}was es heisst, dass Jemand plötzlich kommt und uns die Zeit raubt.\hspace*{1.5em}Nur unsere alte Freundschaft macht die Sache etwas
               leidlicher.\pend
           
\pstart
           Nun erfuhr ich gar nicht, was Beer-Hofmann\pwindex{Beer-Hofmann, Richard 11.\,7.\,1866 Wien – 26.\,9.\,1945 New York City@\textsc{Beer-Hofmann, Richard} (11.\,7.\,1866 Wien – 26.\,9.\,1945 New York City), \emph{Schriftsteller}|pw}
               vorhat, und das interessirt mich doch lebhaft. Das ist die Folge jugendlich-seniler
               Schwatzhaftigkeit, dass die Anderen nicht zu Worte kommen.\pend
           
\pstart
           Auf Wiedersehen in 14 Tagen.\pend
           
\pstart
           Ihr{\\[\baselineskip]}\spacefill\mbox{Georg Brandes}\pend
           \leftskip=0em{}\selectlanguage{ngerman}\endnumbering\briefempfaengerindex{Schnitzler, Arthur@\textsc{Schnitzler, Arthur}!zzzBrandes, Georg@\emph{von Georg Brandes}!1901-05-171@{17. 5. 1901}|)be}\mylabel{L01121h}  \newcommand{\dateiname}{L01121}\newcommand{\titel}{Georg Brandes an Arthur Schnitzler, 17. 5. 1901}\newcommand{\editorInnen}{Martin Anton Müller und Gerd-Hermann Susen}%% latex-leseansicht-abspann.tex
%% Abspann für die Leseansicht.
%% Der Schalter \ifkorrekturansicht ist bereits durch den Vorspann gesetzt.

%% latex-abspann.tex
%% Gemeinsamer Abspann für Korrekturansicht und Leseansicht.
%% Setzt den Schalter \ifkorrekturansicht voraus (gesetzt in den
%% einbindenden Dateien latex-korrekturansicht-abspann.tex bzw.
%% latex-leseansicht-abspann.tex).
%% ---------------------------------------------------------------

\normalsize

% Das esempio-Environment wird nur in der Leseansicht benötigt
\ifkorrekturansicht\else
\newenvironment{esempio}[3]%
{
    \vspace{1.5ex}
    \rlap{\underline{#1}}
    \par
    \setlength{\parindent}{0cm}
    \nopagebreak
    \leftskip=#2cm
    \rightskip=#3cm
}
{
    \par
}
\fi

\doendnotes{C}
\bigskip
\vfill

\clearpage

\footnotesize

\ifkorrekturansicht
  \lohead{\textsc{register}}
\fi

% theindex-Environment neu definieren ohne reledmac
\makeatletter
\renewenvironment{theindex}{%
  \ifkorrekturansicht
    \section*{\indexname}%
  \else
    \subsubsection*{Index der erwähnten Entitäten}%
  \fi
  \setlength{\parindent}{0pt}%
  \setlength{\parskip}{0pt plus 0.3pt}%
  \let\item\@idxitem
}{%
  \ifkorrekturansicht\clearpage\fi
}
\makeatother

\IfFileExists{\jobname-pw.ind}{\input{\jobname-pw.ind}}{}

% Quellenangabe nur in der Leseansicht
\ifkorrekturansicht\else
% Fallback-Definitionen, falls die .tex-Datei \titel etc. nicht gesetzt hat
\providecommand{\titel}{}
\providecommand{\editorInnen}{}
\providecommand{\dateiname}{\jobname}

\vspace{3cm}

\vfill

\footnotesize
\textsc{Quelle}: \titel. Herausgegeben von {\editorInnen}. In: \emph{Arthur Schnitzler: Briefwechsel mit Autorinnen und Autoren}.
 Digitale Edition, https://schnitzler-briefe.acdh.oeaw.ac.at/{\dateiname}.html (Stand \today)
\fi

\end{document}


