%% latex-leseansicht-vorspann.tex
%% Vorspann für die Leseansicht.
%% Lädt die gemeinsame Datei latex-vorspann.tex mit nicht gesetztem Schalter.

\newif\ifkorrekturansicht
\korrekturansichtfalse

\input{../tex-inputs/latex-vorspann}


\section[Paul Goldmann, Eva Goldmann und Georg Brandes an Arthur Schnitzler, 26. 7. 1909]{L02637 Paul Goldmann, Eva Goldmann und Georg Brandes an Arthur Schnitzler, 26. 7. 1909}
\nopagebreak\mylabel{L02637v}
\rehead{ }\normalsize\beginnumbering\briefempfaengerindex{Schnitzler, Arthur@\textsc{Schnitzler, Arthur}!zzzBrandes, Georg@\emph{von Georg Brandes}!1909-07-261@{26. 7. 1909}|(be}\briefempfaengerindex{Schnitzler, Arthur@\textsc{Schnitzler, Arthur}!zzzGoldmann, Eva Marie@\emph{von Eva Marie Goldmann}!1909-07-261@{26. 7. 1909}|(be}\briefempfaengerindex{Schnitzler, Arthur@\textsc{Schnitzler, Arthur}!zzzGoldmann, Paul@\emph{von Paul Goldmann}!1909-07-261@{26. 7. 1909}|(be}
\toendnotes[C]{\smallbreak\pagebreak[2]}
\correspDesc{Versand  durch Paul Goldmann, Eva Marie Goldmann, Georg Brandes am 26. 7. 1909 in Karlsbad
\newline{}Weiterleitung  in Wien
\newline{}Erhalt  durch Arthur Schnitzler am 27. 7. 1909 in Edlach}\toendnotes[C]{\smallbreak}
\Standort{DLA, A:Schnitzler, HS.NZ85.1.3175.}
\physDesc{Bildpostkarte, 210 Zeichen
\newline{}Handschrift Paul Goldmann: blaue Tinte, deutsche Kurrent
\newline{}Handschrift Eva Marie Goldmann: blaue Tinte
\newline{}Handschrift Georg Brandes: blaue Tinte
\newline{}Versand: 1) Stempel: »\nobreak{}\oindex{Karlsbad@\textbf{Karlsbad}|pwk}Karlsbad 1, 26. VII. 0\textcolor{gray}{9}, 4\nobreak{}«.   2) mit lila Tintevon unbekannter Hand die Straßenangabe der
                                 Adresszeile gestrichen und darüber ergänzt: »\noindent{}\textsc{Edlach NÖest}\oindex{Edlach@\textbf{Edlach}|pw}{ / }Edlacherhof\oindex{Hotel Edlacherhof@\textbf{Hotel Edlacherhof}, \emph{Hotel}|pw}«  3) Stempel: »\nobreak{}\oindex{Payerbach@\textbf{Payerbach}, \emph{Hauptstadt}|pwk}{[}P{]}ayerbach, 27. VII. \textcolor{gray}{0}9, 4\nobreak{}«. 
\newline{}Schnitzler: mit Bleistift Vermerk: »\textsc{Goldma{\geminationn}}« }
\buchAbdrucke{\weitereDrucke{Georg Brandes, Arthur Schnitzler: \emph{Ein Briefwechsel}. Herausgegeben von Kurt Bergel. Bern: \emph{Francke} 1956, S. 198.} }\pstart{}\textsc{{\pb}Herrn}\pend{}\pstart{}\textsc{Dr. Arthur Schnitzler}\pend{}\pstart{}\textsc{Wien\oindex{Wien@\textbf{Wien}, \emph{Verwaltungsgebiet}|pw}}\pend{}\pstart{}\textsc{XVIII. Spöttelgaſse 7\oindex{Wien@\textbf{Wien}!XVIII., Währing@\textbf{XVIII., Währing}!Edmund-Weiß-Gasse 7@\textbf{Edmund-Weiß-Gasse 7}, \emph{Wohngebäude}|pw}.}\pend{}{\bigskip}
\pstart
           \noindent{}\centering{}{\pb}\textcolor{gray}{\textbf{Karlsbad\oindex{Karlsbad@\textbf{Karlsbad}|pw} Sprudel}}\pend
           \vspace{1em}
\pstart
           {\pb}26. 7. 09{ }Karlsbad\oindex{Karlsbad@\textbf{Karlsbad}|pw}\pend
           
\pstart{}Lieber Freund,\pend\vspace{0.5em}
\pstart
           Wir{ }ſitzen hier beiſammen, gedenken Deiner in Freundſchaft u.{ }ſenden Dir herzliche
               Grüße.\pend
           
\pstart
           \spacefill\mbox{Paul Goldmann}{\\[\baselineskip]}\spacefill\mbox{{[}hs. Goldmann:{]} EvaGoldmann}{\\[\baselineskip]}\spacefill\mbox{{[}hs. Brandes:{]} Georg Brandes}\pend
           \leftskip=0em{}\selectlanguage{ngerman}\endnumbering\briefempfaengerindex{Schnitzler, Arthur@\textsc{Schnitzler, Arthur}!zzzBrandes, Georg@\emph{von Georg Brandes}!1909-07-261@{26. 7. 1909}|)be}\briefempfaengerindex{Schnitzler, Arthur@\textsc{Schnitzler, Arthur}!zzzGoldmann, Eva Marie@\emph{von Eva Marie Goldmann}!1909-07-261@{26. 7. 1909}|)be}\briefempfaengerindex{Schnitzler, Arthur@\textsc{Schnitzler, Arthur}!zzzGoldmann, Paul@\emph{von Paul Goldmann}!1909-07-261@{26. 7. 1909}|)be}\mylabel{L02637h}  \newcommand{\dateiname}{L02637}\newcommand{\titel}{Paul Goldmann, Eva Goldmann und Georg Brandes an Arthur Schnitzler, 26. 7. 1909}\newcommand{\editorInnen}{Martin Anton Müller und Laura Untner}%% latex-leseansicht-abspann.tex
%% Abspann für die Leseansicht.
%% Der Schalter \ifkorrekturansicht ist bereits durch den Vorspann gesetzt.

%% latex-abspann.tex
%% Gemeinsamer Abspann für Korrekturansicht und Leseansicht.
%% Setzt den Schalter \ifkorrekturansicht voraus (gesetzt in den
%% einbindenden Dateien latex-korrekturansicht-abspann.tex bzw.
%% latex-leseansicht-abspann.tex).
%% ---------------------------------------------------------------

\normalsize

% Das esempio-Environment wird nur in der Leseansicht benötigt
\ifkorrekturansicht\else
\newenvironment{esempio}[3]%
{
    \vspace{1.5ex}
    \rlap{\underline{#1}}
    \par
    \setlength{\parindent}{0cm}
    \nopagebreak
    \leftskip=#2cm
    \rightskip=#3cm
}
{
    \par
}
\fi

\doendnotes{C}
\bigskip
\vfill

\clearpage

\footnotesize

\ifkorrekturansicht
  \lohead{\textsc{register}}
\fi

% theindex-Environment neu definieren ohne reledmac
\makeatletter
\renewenvironment{theindex}{%
  \ifkorrekturansicht
    \section*{\indexname}%
  \else
    \subsubsection*{Index der erwähnten Entitäten}%
  \fi
  \setlength{\parindent}{0pt}%
  \setlength{\parskip}{0pt plus 0.3pt}%
  \let\item\@idxitem
}{%
  \ifkorrekturansicht\clearpage\fi
}
\makeatother

\IfFileExists{\jobname-pw.ind}{\input{\jobname-pw.ind}}{}

% Quellenangabe nur in der Leseansicht
\ifkorrekturansicht\else
% Fallback-Definitionen, falls die .tex-Datei \titel etc. nicht gesetzt hat
\providecommand{\titel}{}
\providecommand{\editorInnen}{}
\providecommand{\dateiname}{\jobname}

\vspace{3cm}

\vfill

\footnotesize
\textsc{Quelle}: \titel. Herausgegeben von {\editorInnen}. In: \emph{Arthur Schnitzler: Briefwechsel mit Autorinnen und Autoren}.
 Digitale Edition, https://schnitzler-briefe.acdh.oeaw.ac.at/{\dateiname}.html (Stand \today)
\fi

\end{document}


