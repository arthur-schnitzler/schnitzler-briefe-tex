%% latex-leseansicht-vorspann.tex
%% Vorspann für die Leseansicht.
%% Lädt die gemeinsame Datei latex-vorspann.tex mit nicht gesetztem Schalter.

\newif\ifkorrekturansicht
\korrekturansichtfalse

\input{../tex-inputs/latex-vorspann}


\section[Arthur und Olga Schnitzler an Hermann Bahr, 5. 5. 1905]{L01513 Arthur und Olga Schnitzler an Hermann Bahr, 5. 5. 1905}
\nopagebreak\mylabel{L01513v}
\rehead{ }\normalsize\beginnumbering\briefempfaengerindex{Bahr, Hermann@\textsc{Bahr, Hermann}!zzzSchnitzler, Olga@\emph{von Olga Schnitzler}!1905-05-051@{5. 5. 1905}|(be}\briefempfaengerindex{Bahr, Hermann@\textsc{Bahr, Hermann}!zzzSchnitzler, Arthur@\emph{von Arthur Schnitzler}!1905-05-051@{5. 5. 1905}|(be}
\toendnotes[C]{\smallbreak\pagebreak[2]}
\correspDesc{Versand  durch Arthur Schnitzler, Olga Schnitzler am 5. 5. 1905 in Semmering
\newline{}Erhalt  durch Hermann Bahr im Zeitraum [6. 5. 1905
                  – 10. 5. 1905?] in Wien}\toendnotes[C]{\smallbreak}
\Standort{TMW, HS AM 60179 Ba.}
\physDesc{Bildpostkarte, 104 Zeichen
\newline{}Handschrift Arthur Schnitzler: Bleistift, deutsche Kurrent
\newline{}Handschrift Olga Schnitzler: Bleistift, deutsche Kurrent
\newline{}Versand: Stempel: »\nobreak{}\oindex{Semmering@\textbf{Semmering}, \emph{Verwaltungsgebiet}|pwk}Semmering, 5. 5. 05, 6–7\nobreak{}«.  
\newline{}Ordnung: Lochung }
\buchAbdrucke{\weitereDrucke{1) \emph{5. 5. 1905, Abschrift.} In: Arthur Schnitzler: \emph{The Letters of Arthur Schnitzler to Hermann Bahr}. Edited, annotated, and with an introduction, by Donald G. Daviau. Chapel Hill: \emph{The University of North Carolina Press} 1978, S. 88 (University of North Carolina studies in the Germanic languages
                        and literatures, 89).} \weitereDrucke{2) Hermann Bahr, Arthur Schnitzler: \emph{Briefwechsel, Aufzeichnungen, Dokumente (1891–1931)}. Herausgegeben von Kurt Ifkovits und Martin Anton Müller. Göttingen: \emph{Wallstein} 2018, S. 344.} }\pstart{}\textsc{{\pb}Herrn Hermann
                  Bahr}\pend{}\pstart{}\textsc{Wien Ob St Veit\oindex{Wien@\textbf{Wien}!XIII., Hietzing@\textbf{XIII., Hietzing}!Ober Sankt Veit@\textbf{Ober Sankt Veit}, \emph{Ehemaliger Ort}|pw}}\pend{}\pstart{}\textsc{Veitlissengasse\oindex{Wien@\textbf{Wien}!XIII., Hietzing@\textbf{XIII., Hietzing}!Veitlissengasse@\textbf{Veitlissengasse}, \emph{Straße}|pw}}\pend{}{\bigskip}
\pstart
           {\pb}\textcolor{gray}{\textbf{Südbahnhôtel\oindex{Südbahnhotel [Semmering]@\textbf{Südbahnhotel [Semmering]}, \emph{Hotel}|pw}, 1000 m Seehöhe}}\hfill \textcolor{gray}{\textbf{Semmering\oindex{Semmering@\textbf{Semmering}, \emph{Verwaltungsgebiet}|pw}}}.\pend
           \vspace{1em}
\pstart
           {\pb}5. 5. 905\pend
           \vspace{0.5em}
\pstart
           Herzlichen Gruß! \spacefill\mbox{ArthSch.}\pend
           
\pstart
           {[}hs. Schnitzler:{]} Ebenſo! \spacefill\mbox{OlgaSchnitzler.}\pend
           \selectlanguage{ngerman}\endnumbering\briefempfaengerindex{Bahr, Hermann@\textsc{Bahr, Hermann}!zzzSchnitzler, Olga@\emph{von Olga Schnitzler}!1905-05-051@{5. 5. 1905}|)be}\briefempfaengerindex{Bahr, Hermann@\textsc{Bahr, Hermann}!zzzSchnitzler, Arthur@\emph{von Arthur Schnitzler}!1905-05-051@{5. 5. 1905}|)be}\mylabel{L01513h}  \newcommand{\dateiname}{L01513}\newcommand{\titel}{Arthur und Olga Schnitzler an Hermann Bahr, 5. 5. 1905}\newcommand{\editorInnen}{Herausgegeben von Martin Anton Müller}%% latex-leseansicht-abspann.tex
%% Abspann für die Leseansicht.
%% Der Schalter \ifkorrekturansicht ist bereits durch den Vorspann gesetzt.

%% latex-abspann.tex
%% Gemeinsamer Abspann für Korrekturansicht und Leseansicht.
%% Setzt den Schalter \ifkorrekturansicht voraus (gesetzt in den
%% einbindenden Dateien latex-korrekturansicht-abspann.tex bzw.
%% latex-leseansicht-abspann.tex).
%% ---------------------------------------------------------------

\normalsize

% Das esempio-Environment wird nur in der Leseansicht benötigt
\ifkorrekturansicht\else
\newenvironment{esempio}[3]%
{
    \vspace{1.5ex}
    \rlap{\underline{#1}}
    \par
    \setlength{\parindent}{0cm}
    \nopagebreak
    \leftskip=#2cm
    \rightskip=#3cm
}
{
    \par
}
\fi

\doendnotes{C}
\bigskip
\vfill

\clearpage

\footnotesize

\ifkorrekturansicht
  \lohead{\textsc{register}}
\fi

% theindex-Environment neu definieren ohne reledmac
\makeatletter
\renewenvironment{theindex}{%
  \ifkorrekturansicht
    \section*{\indexname}%
  \else
    \subsubsection*{Index der erwähnten Entitäten}%
  \fi
  \setlength{\parindent}{0pt}%
  \setlength{\parskip}{0pt plus 0.3pt}%
  \let\item\@idxitem
}{%
  \ifkorrekturansicht\clearpage\fi
}
\makeatother

\IfFileExists{\jobname-pw.ind}{\input{\jobname-pw.ind}}{}

% Quellenangabe nur in der Leseansicht
\ifkorrekturansicht\else
% Fallback-Definitionen, falls die .tex-Datei \titel etc. nicht gesetzt hat
\providecommand{\titel}{}
\providecommand{\editorInnen}{}
\providecommand{\dateiname}{\jobname}

\vspace{3cm}

\vfill

\footnotesize
\textsc{Quelle}: \titel. Herausgegeben von {\editorInnen}. In: \emph{Arthur Schnitzler: Briefwechsel mit Autorinnen und Autoren}.
 Digitale Edition, https://schnitzler-briefe.acdh.oeaw.ac.at/{\dateiname}.html (Stand \today)
\fi

\end{document}


