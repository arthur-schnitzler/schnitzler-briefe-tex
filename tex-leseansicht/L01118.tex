%% latex-leseansicht-vorspann.tex
%% Vorspann für die Leseansicht.
%% Lädt die gemeinsame Datei latex-vorspann.tex mit nicht gesetztem Schalter.

\newif\ifkorrekturansicht
\korrekturansichtfalse

\input{../tex-inputs/latex-vorspann}


\section[Georg Brandes an Arthur Schnitzler, 13. 5. [1901]]{L01118 Georg Brandes an Arthur Schnitzler, 13. 5. [1901]}
\nopagebreak\mylabel{L01118v}
\rehead{ }\normalsize\beginnumbering\briefempfaengerindex{Schnitzler, Arthur@\textsc{Schnitzler, Arthur}!zzzBrandes, Georg@\emph{von Georg Brandes}!1901-05-131@{13. 5. 1901}|(be}
\toendnotes[C]{\smallbreak\pagebreak[2]}
\correspDesc{Versand  durch Georg Brandes am 13. 5. 1901 in Ostrava
\newline{}Erhalt  durch Arthur Schnitzler im Zeitraum [14. 5. 1901
                  – 18. 5. 1901?] in Wien}\toendnotes[C]{\smallbreak}
\Standort{CUL, Schnitzler, B 17.}
\physDesc{Brief, 1 Blatt, 2 Seiten, 744 Zeichen
\newline{}Handschrift: schwarze Tinte, lateinische Kurrent
\newline{}Schnitzler: mit Bleistift die Jahreszahl ergänzt: »901« 
\newline{}Ordnung: 1) mit Bleistift von unbekannter Hand nummeriert: »\strikeout{21}«  2) mit Bleistift von unbekannter Hand nummeriert:
                                    »22«}
\buchAbdrucke{\weitereDrucke{Georg Brandes, Arthur Schnitzler: \emph{Ein Briefwechsel}. Herausgegeben von Kurt Bergel. Bern: \emph{Francke} 1956, S. 86.} }
\pstart
           \raggedleft{}{\pb}Schloss Strzebowitz\oindex{Schloss Strzebowitz@\textbf{Schloss Strzebowitz}, \emph{Schloss}|pw}{\\}Oesterr. Schlesien\oindex{Schlesien@\textbf{Schlesien}, \emph{Region}|pw}{\\}13 May\pend
           
\pstart{}Verehrter Freund\pend\vspace{0.5em}
\pstart
           Es ist meine Absicht, am 16\textsuperscript{sten} um 3\textsubscript{45} in Wien\oindex{Wien@\textbf{Wien}, \emph{Verwaltungsgebiet}|pw} anzukommen und um 8\textsubscript{25} Abends nach Abbazia\oindex{Opatija@\textbf{Opatija}, \emph{Hauptstadt}|pw} abzureisen.\pend
           
\pstart
           Ich will sehr gern von der Nordbahn\oindex{Wien@\textbf{Wien}!II., Leopoldstadt@\textbf{II., Leopoldstadt}!Nordbahnhof@\textbf{Nordbahnhof}, \emph{Bahnhofsgebäude}|pw} zu Ihnen
               fahren, weiss nur nicht, da ich die Lage der Bahnhöfe nicht kenne, ob es nicht besser
               wäre, erst meinen Koffer nach der Südbahnstation\oindex{Wien@\textbf{Wien}!X., Favoriten@\textbf{X., Favoriten}!Südbahnhof@\textbf{Südbahnhof}, \emph{Bahnhofsgebäude}|pw}
               zu fahren.\pend
           
\pstart
           Es versteht sich von selbst, dass es mir nur lieb sein kann, Herrn Beer-Hofmann\pwindex{Beer-Hofmann, Richard 11.\,7.\,1866 Wien – 26.\,9.\,1945 New York City@\textsc{Beer-Hofmann, Richard} (11.\,7.\,1866 Wien – 26.\,9.\,1945 New York City), \emph{Schriftsteller}|pw} zu treffen. Ich weiss nicht, wann
               Sie Mittag essen, ich werde wohl im Zuge etwas frühstücken, also sagen wir um
                  5 Uhr{ }{\pb}(oder wann es Sie passt, wer
               weiss im voraus, wann man an einem bestimmten Tag Hunger hat?) Recht bedacht
               überlasse ich Ihnen die Esszeit.\pend
           
\pstart
           Haben Sie vielen Dank für Ihre freundliche Antwort.\pend
           
\pstart
           Von Herzen{\\[\baselineskip]}Ihr{\\[\baselineskip]}\spacefill\mbox{Georg Brandes}\pend
           \leftskip=0em{}\selectlanguage{ngerman}\endnumbering\briefempfaengerindex{Schnitzler, Arthur@\textsc{Schnitzler, Arthur}!zzzBrandes, Georg@\emph{von Georg Brandes}!1901-05-131@{13. 5. 1901}|)be}\mylabel{L01118h}  \newcommand{\dateiname}{L01118}\newcommand{\titel}{Georg Brandes an Arthur Schnitzler, 13. 5. [1901]}\newcommand{\editorInnen}{Martin Anton Müller und Gerd-Hermann Susen}%% latex-leseansicht-abspann.tex
%% Abspann für die Leseansicht.
%% Der Schalter \ifkorrekturansicht ist bereits durch den Vorspann gesetzt.

%% latex-abspann.tex
%% Gemeinsamer Abspann für Korrekturansicht und Leseansicht.
%% Setzt den Schalter \ifkorrekturansicht voraus (gesetzt in den
%% einbindenden Dateien latex-korrekturansicht-abspann.tex bzw.
%% latex-leseansicht-abspann.tex).
%% ---------------------------------------------------------------

\normalsize

% Das esempio-Environment wird nur in der Leseansicht benötigt
\ifkorrekturansicht\else
\newenvironment{esempio}[3]%
{
    \vspace{1.5ex}
    \rlap{\underline{#1}}
    \par
    \setlength{\parindent}{0cm}
    \nopagebreak
    \leftskip=#2cm
    \rightskip=#3cm
}
{
    \par
}
\fi

\doendnotes{C}
\bigskip
\vfill

\clearpage

\footnotesize

\ifkorrekturansicht
  \lohead{\textsc{register}}
\fi

% theindex-Environment neu definieren ohne reledmac
\makeatletter
\renewenvironment{theindex}{%
  \ifkorrekturansicht
    \section*{\indexname}%
  \else
    \subsubsection*{Index der erwähnten Entitäten}%
  \fi
  \setlength{\parindent}{0pt}%
  \setlength{\parskip}{0pt plus 0.3pt}%
  \let\item\@idxitem
}{%
  \ifkorrekturansicht\clearpage\fi
}
\makeatother

\IfFileExists{\jobname-pw.ind}{\input{\jobname-pw.ind}}{}

% Quellenangabe nur in der Leseansicht
\ifkorrekturansicht\else
% Fallback-Definitionen, falls die .tex-Datei \titel etc. nicht gesetzt hat
\providecommand{\titel}{}
\providecommand{\editorInnen}{}
\providecommand{\dateiname}{\jobname}

\vspace{3cm}

\vfill

\footnotesize
\textsc{Quelle}: \titel. Herausgegeben von {\editorInnen}. In: \emph{Arthur Schnitzler: Briefwechsel mit Autorinnen und Autoren}.
 Digitale Edition, https://schnitzler-briefe.acdh.oeaw.ac.at/{\dateiname}.html (Stand \today)
\fi

\end{document}


