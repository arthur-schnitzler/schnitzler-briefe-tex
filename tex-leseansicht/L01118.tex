%% latex-korrekturansicht-vorspann.tex
%% Vorspann für die Korrekturansicht.
%% Lädt die gemeinsame Datei latex-vorspann.tex mit gesetztem Schalter.

\newif\ifkorrekturansicht
\korrekturansichttrue

\input{../tex-inputs/latex-vorspann}


\section[Georg Brandes an Arthur Schnitzler, 13. 5. {[}1901{]}]{L01118 Georg Brandes an Arthur Schnitzler, 13. 5. {[}1901{]}}
\nopagebreak\mylabel{L01118v}
\rehead{ }\normalsize\beginnumbering\briefempfaengerindex{Schnitzler, Arthur@\textsc{Schnitzler, Arthur}!zzzBrandes, Georg@\emph{von Georg Brandes}!1901-05-131@{13. 5. 1901}|(be}
\toendnotes[C]{\smallbreak\pagebreak[2]}\Standort{CUL, Schnitzler, B 17.}
\physDesc{Brief, 1 Blatt, 2 Seiten, 744 Zeichen
\newline{}Handschrift: schwarze Tinte, lateinische Kurrent
\newline{}Schnitzler: mit Bleistift die Jahreszahl ergänzt: »901« 
\newline{}Ordnung: 1) mit Bleistift von unbekannter Hand nummeriert: »\strikeout{21}«  2) mit Bleistift von unbekannter Hand nummeriert:
                                    »22«}
\buchAbdrucke{\weitereDrucke{Georg Brandes, Arthur Schnitzler: \emph{Ein Briefwechsel}. Bern: \emph{Francke} 1956, S. 86.} }
\pstart
           \raggedleft{}{\pb}Schloss Strzebowitz\oindex{Schloss Strzebowitz@\textbf{Schloss Strzebowitz}, \emph{Schloss (K.SLS)}|pw}{\\}Oesterr. Schlesien\oindex{Schlesien@\textbf{Schlesien}, \emph{L.RGN}|pw}{\\}13 May\pend
           
\pstart{}Verehrter Freund\pend\vspace{0.5em}
\pstart
           Es ist meine Absicht, am 16\textsuperscript{sten} um 3\textsubscript{45} in Wien\oindex{Wien@\textbf{Wien}, \emph{A.ADM2}|pw} anzukommen und um 8\textsubscript{25} Abends nach Abbazia\oindex{Opatija@\textbf{Opatija}, \emph{P.PPLA2}|pw} abzureisen.\pend
           
\pstart
           Ich will sehr gern von der Nordbahn\oindex{Nordbahnhof@\textbf{Nordbahnhof}, \emph{Bahnhofsgebäude (K.BHF)}|pw} zu Ihnen
               fahren, weiss nur nicht, da ich die Lage der Bahnhöfe nicht kenne, ob es nicht besser
               wäre, erst meinen Koffer nach der Südbahnstation\oindex{Suedbahnhof@\textbf{Südbahnhof}, \emph{Bahnhofsgebäude (K.BHF)}|pw}
               zu fahren.\pend
           
\pstart
           Es versteht sich von selbst, dass es mir nur lieb sein kann, Herrn Beer-Hofmann\pwindex{Beer-Hofmann, Richard 1866-07-11 – 1945-09-26@\textsc{Beer-Hofmann, Richard} (1866-07-11 – 1945-09-26), \emph{Schriftsteller/Schriftstellerin}|pw} zu treffen. Ich weiss nicht, wann
               Sie Mittag essen, ich werde wohl im Zuge etwas frühstücken, also sagen wir um
                  5 Uhr{ }{\pb}(oder wann es Sie passt, wer
               weiss im voraus, wann man an einem bestimmten Tag Hunger hat?) Recht bedacht
               überlasse ich Ihnen die Esszeit.\pend
           
\pstart
           Haben Sie vielen Dank für Ihre freundliche Antwort.\pend
           
\pstart
           Von Herzen{\\[\baselineskip]}Ihr{\\[\baselineskip]}\spacefill\mbox{Georg Brandes}\pend
           \leftskip=0em{}\selectlanguage{ngerman}\endnumbering\briefempfaengerindex{Schnitzler, Arthur@\textsc{Schnitzler, Arthur}!zzzBrandes, Georg@\emph{von Georg Brandes}!1901-05-131@{13. 5. 1901}|)be}\mylabel{L01118h}  \normalsize

\doendnotes{C}
\bigskip
\vfill

\clearpage

\footnotesize

\lohead{\textsc{register}}

% Definiere theindex-Environment komplett neu ohne reledmac
\makeatletter
\renewenvironment{theindex}{%
  \section*{\indexname}%
  \setlength{\parindent}{0pt}%
  \setlength{\parskip}{0pt plus 0.3pt}%
  \let\item\@idxitem
}{%
  \clearpage
}
\makeatother

\IfFileExists{\jobname-pw.ind}{\input{\jobname-pw.ind}}{}

\end{document}

      