%% latex-leseansicht-vorspann.tex
%% Vorspann für die Leseansicht.
%% Lädt die gemeinsame Datei latex-vorspann.tex mit nicht gesetztem Schalter.

\newif\ifkorrekturansicht
\korrekturansichtfalse

\input{../tex-inputs/latex-vorspann}


               \section[Arthur Schnitzler an Hugo von Hofmannsthal, 2. 8. 1893]{ Arthur Schnitzler an Hugo von Hofmannsthal, 2. 8. 1893}\nopagebreak\mylabel{v}\rehead{ }\begin{ledgroupsized}[t]{13cm}\normalsize\beginnumbering\briefempfaengerindex{Hofmannsthal, Hugo von@\textsc{Hofmannsthal, Hugo von}!zzzSchnitzler, Arthur@\emph{von Arthur Schnitzler}!1893-08-021@{2. 8. 1893}|(be} \toendnotes[C]{\smallbreak\pagebreak[2]} \Standort{FDH, Hs-30885,37.}
\physDesc{Brief, 2 Blätter (Briefpapier mit Trauerrand), 6 Seiten
\newline{}Handschrift: schwarze Tinte, deutsche Kurrent\newline{}Ordnung: 1) mit rotem Buntstift das erste
                                    Blatt nummeriert: »IX« 2) mit Bleistift datiert von Schnitzler das zweite Blatt mutmaßlich bei der Durchsicht der Briefe 1929  »2. 8. 92«}\buchAbdrucke{\weitereDrucke{Hugo von Hofmannsthal, Arthur Schnitzler: \emph{Briefwechsel}. Hg. Therese Nickl und Heinrich Schnitzler. Frankfurt am Main: \emph{S. Fischer} 1964, S. 42–43.} }\toendnotes[C]{\smallbreak}\pstart
           \raggedleft{}{\pb}Wien\oindex{Wien@\textbf{Wien}|pw}, \uline{2. 8. 93}\pend
           \pstart{}Mein lieber Hugo,\pend\pstart
           ich las Ihren Brief an \textsc{Salten}\pwindex{Salten, Felix 06.09.1869 – 08.10.1945@\textsc{Salten, Felix} (06.09.1869 – 08.10.1945), \emph{Schriftsteller, Journalist}|pw}. Daſs Sie nicht in München\oindex{Muenchen@\textbf{München}|pw}, wußt’ ich, da ich \textsc{Bahr}\pwindex{Bahr, Hermann 19.07.1863 – 15.01.1934@\textsc{Bahr, Hermann} (19.07.1863 – 15.01.1934), \emph{Schriftsteller, Kritiker}|pw}{ }ſprach. Sie wollen im September hin? Nicht unmöglich, daſs ich mich
                    anſchließe; de{\geminationn} ich habe zur Waffenübung keine
                    Einberufung beko{\geminationm}en, u dürfte auch vorausſichtlich
                    keine mehr erhalten.\pend
           \pstart
           Vorläufig bleibe ich in Wien\oindex{Wien@\textbf{Wien}|pw}; Mitte
                        Auguſt fahre ich vielleicht mit Mama\pwindex{Schnitzler, Louise 08.07.1840 – 09.09.1911@\textsc{Schnitzler, Louise} (08.07.1840 – 09.09.1911)|pwv} weg, {\pb}mache auch event. eine \textsc{Bicycle}tour mit \textsc{Salten}\pwindex{Salten, Felix 06.09.1869 – 08.10.1945@\textsc{Salten, Felix} (06.09.1869 – 08.10.1945), \emph{Schriftsteller, Journalist}|pw}. Sie müſſen \textsc{Bic.} fahren lernen; ebenſo wie
                        Richard\pwindex{Beer-Hofmann, Richard 11.07.1866 – 26.09.1945@\textsc{Beer-Hofmann, Richard} (11.07.1866 – 26.09.1945), \emph{Schriftsteller}|pw}; es iſt wirklich ein großes
                    Vergnügen. –\pend
           \pstart
           Wien\oindex{Wien@\textbf{Wien}|pw} bietet mir jetzt einiges zu thun; eine
                    kleine Couſine\pwindex{Suppe, Adele von 05.12.1879 – 04.08.1893@\textsc{Suppé, Adele von} (05.12.1879 – 04.08.1893)|pwv} von mir iſt ſchwer krank;
                    die beſuch’ ich 1, 2, 3 mal im Tag; da{\geminationn} ab u zu
                    irgend was andres ärztliches, ſo daſs die Zeit zerſplittert iſt.
                        Aben\textcolor{gray}{d}s zuweilen auf dem Kahlenberg, wo Mama\pwindex{Schnitzler, Louise 08.07.1840 – 09.09.1911@\textsc{Schnitzler, Louise} (08.07.1840 – 09.09.1911)|pwv} u Schweſter\pwindex{Hajek, Gisela 20.12.1867 – 03.02.1953@\textsc{Hajek, Gisela} (20.12.1867 – 03.02.1953)|pwv}
                    wohnen oder mit dem \textsc{Bic.} da oder dorthin.\pend
           \pstart
           {\pb}– Die »luſtige« Novelle\pwindex{Schnitzler, Arthur 15.05.1862 – 21.10.1931@\textsc{Schnitzler, Arthur} (15.05.1862 – 21.10.1931), \emph{Schriftsteller, Mediziner}!kleine Komoedie01.08.1895 – 01.08.1895@\strich\emph{Die kleine Komödie} {[}01.08.1895 – 01.08.1895{]}|pwv} hab ich bis auf wenige Zeilen beendet, die ich erſt ſchreiben
                    kann, wenn ich Luſt beko{\geminationm}e, das ganze Zeug wieder
                    durchzuleſen. Was ich zunächſt ſchreiben werde, iſt unklar – am liebſten eins
                    meiner im Umriſs fertigen 3aktigen Stücke; aber ich ſtehe der dramatiſchen Kunſt
                    unglaublich muthlos gegenüber; ja ich hatte in der letzten Zeit oft die
                    Empfindung, daſs ich überhaupt nie {\pb}ein gutes
                    Stück werde ſchreiben können. Geſtalten u Scenen, einzelne, wären da; aber mir
                    iſt, als hätt’ ich jedes ſtrategiſche Talent verloren. Vielleicht hatt’ ichs
                    auch nie – und hab nur aus meinen kleinen Schmerzen die großen \substVorne{}\textsuperscript{S}\substDazwischen{}D\substHinten{}reiakter machen können; und ſeit meinen großen Schmerzen \strikeout{hab} werden mir nur die kleinen Novellettchen
                    gelingen. Wie leicht, wie mühelos hab ich vor – zehn, zwölf Jahren
                    geſchrieben, – {\pb}es kam zwar nie was gutes heraus;
                    aber ich war damals vielleicht ein echterer »Poet« als heut. Denn heut nagen an
                    meiner Poeſie viele Würmer, z. B. das Leben. –\pend
           \pstart
           – Wollen Sie mir nicht Ihre Pläne für den Reſt des So{\geminationm}ers mittheilen. Es iſt nicht unmöglich, daſs wir uns begegnen können.
                    Jedenfalls ſchreiben Sie mir einige Zeilen – oder Seiten, was mir lieber wäre.
                    Beleuchten {\pb}Sie mit einem »Flähmchen« die ganze
                    Umgebung!\pend
           \pstart
           Herzlich der Ihre{\\[\baselineskip]}\spacefill\mbox{Arthur}\pend
           \leftskip=0em{}\endnumbering\briefempfaengerindex{Hofmannsthal, Hugo von@\textsc{Hofmannsthal, Hugo von}!zzzSchnitzler, Arthur@\emph{von Arthur Schnitzler}!1893-08-021@{2. 8. 1893}|)be}\mylabel{h}\end{ledgroupsized}  \newcommand{\dateiname}{L00248}\newcommand{\titel}{Arthur Schnitzler an Hugo von Hofmannsthal, 2. 8. 1893}\newcommand{\editorInnen}{ Martin Anton Müller und Gerd-Hermann Susen}%% latex-leseansicht-abspann.tex
%% Abspann für die Leseansicht.
%% Der Schalter \ifkorrekturansicht ist bereits durch den Vorspann gesetzt.

%% latex-abspann.tex
%% Gemeinsamer Abspann für Korrekturansicht und Leseansicht.
%% Setzt den Schalter \ifkorrekturansicht voraus (gesetzt in den
%% einbindenden Dateien latex-korrekturansicht-abspann.tex bzw.
%% latex-leseansicht-abspann.tex).
%% ---------------------------------------------------------------

\normalsize

% Das esempio-Environment wird nur in der Leseansicht benötigt
\ifkorrekturansicht\else
\newenvironment{esempio}[3]%
{
    \vspace{1.5ex}
    \rlap{\underline{#1}}
    \par
    \setlength{\parindent}{0cm}
    \nopagebreak
    \leftskip=#2cm
    \rightskip=#3cm
}
{
    \par
}
\fi

\doendnotes{C}
\bigskip
\vfill

\clearpage

\footnotesize

\ifkorrekturansicht
  \lohead{\textsc{register}}
\fi

% theindex-Environment neu definieren ohne reledmac
\makeatletter
\renewenvironment{theindex}{%
  \ifkorrekturansicht
    \section*{\indexname}%
  \else
    \subsubsection*{Index der erwähnten Entitäten}%
  \fi
  \setlength{\parindent}{0pt}%
  \setlength{\parskip}{0pt plus 0.3pt}%
  \let\item\@idxitem
}{%
  \ifkorrekturansicht\clearpage\fi
}
\makeatother

\IfFileExists{\jobname-pw.ind}{\input{\jobname-pw.ind}}{}

% Quellenangabe nur in der Leseansicht
\ifkorrekturansicht\else
% Fallback-Definitionen, falls die .tex-Datei \titel etc. nicht gesetzt hat
\providecommand{\titel}{}
\providecommand{\editorInnen}{}
\providecommand{\dateiname}{\jobname}

\vspace{3cm}

\vfill

\footnotesize
\textsc{Quelle}: \titel. Herausgegeben von {\editorInnen}. In: \emph{Arthur Schnitzler: Briefwechsel mit Autorinnen und Autoren}.
 Digitale Edition, https://schnitzler-briefe.acdh.oeaw.ac.at/{\dateiname}.html (Stand \today)
\fi

\end{document}


      