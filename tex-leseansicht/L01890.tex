%% latex-leseansicht-vorspann.tex
%% Vorspann für die Leseansicht.
%% Lädt die gemeinsame Datei latex-vorspann.tex mit nicht gesetztem Schalter.

\newif\ifkorrekturansicht
\korrekturansichtfalse

\input{../tex-inputs/latex-vorspann}


         
         \renewcommand{\erwaehntePersonen}{Personen: Leopold von Andrian-Werburg, Richard Beer-Hofmann, Naëmah Beer-Hofmann, Mirjam Beer-Hofmann, Gabriel Beer-Hofmann, Lili Cappellini, Hugo von Hofmannsthal, Felix von Oppenheimer, Ludwig von Oppenheimer}
         \renewcommand{\erwaehnteOrte}{Orte: Reisnerstraße, Semmering, Wien}
         \renewcommand{\erwaehnteWerke}{}
               \section[Richard Beer-Hofmann an Arthur Schnitzler, 30. 11. 1909]{ Richard Beer-Hofmann an Arthur Schnitzler, 30. 11. 1909}\nopagebreak\mylabel{v}\rehead{ }\begin{ledgroupsized}[t]{13cm}\normalsize\beginnumbering\briefempfaengerindex{Schnitzler, Arthur@\textsc{Schnitzler, Arthur}!zzzBeer-Hofmann, Richard@\emph{von Richard Beer-Hofmann}!1909-11-301@{30. 11. 1909}|(be} \toendnotes[C]{\smallbreak\pagebreak[2]} \Standort{CUL, Schnitzler, B 8.}
\physDesc{Brief, 1 Blatt, 4 Seiten, 677 Zeichen
\newline{}Handschrift: Bleistift, lateinische Kurrent
\newline{}Schnitzler: mit Bleistift beschriftet: »\textsc{R. Beerhofm}« 
\newline{}Ordnung: 1) mit Bleistift von unbekannter Hand nummeriert: »\strikeout{219}«  2) mit Bleistift von unbekannter Hand nummeriert:
                                    »225«}\buchAbdrucke{\weitereDrucke{Arthur Schnitzler, Richard Beer-Hofmann: \emph{Briefwechsel 1891–1931}. Hg. Konstanze Fliedl. Wien, Zürich: \emph{Europaverlag} 1992, S. 195–196.} }\toendnotes[C]{\smallbreak}\pstart
           \raggedleft{}{\pb}30/XI 09{\\}10 ¾ Nachts\pend
           \pstart
           Lieber Arthur!{ }Poldi Andrian\pwindex{Andrian-Werburg, Leopold von 09.05.1875 – 19.11.1951@\textsc{Andrian-Werburg, Leopold von} (09.05.1875 – 19.11.1951), \emph{Schriftsteller, Diplomat}|pw} geht eben weg; er ist – Felix Oppenheimer\pwindex{Oppenheimer, Felix von 20.02.1874 – 15.11.1938@\textsc{Oppenheimer, Felix von} (20.02.1874 – 15.11.1938), \emph{Schriftsteller, Soziologe, Mäzen}|pw} ist vor dem \label{K_L01890-1v}\edtext{Leichenbegängnis}{\lemma{\textnormal{\emph{Leichenbegängnis}}}\Cendnote{\textnormal{Die Überführung aus dem Trauerhaus in der Reisnerstraße 28\oindex{Reisnerstrasse@\textbf{Reisnerstraße}|pwk} auf den Friedhof fand am
                     30. 11. 1909 statt.}}}\label{K_L01890-1h} seines Vaters\pwindex{Oppenheimer, Ludwig von 21.8.1843 – 27.11.1909@\textsc{Oppenheimer, Ludwig von} (21.8.1843 – 27.11.1909), \emph{Politiker, Unternehmer}|pwv} – Hugo\pwindex{Hofmannsthal, Hugo von 1874-02-01 – 1929-07-15@\textsc{Hofmannsthal, Hugo von} (1874-02-01 – 1929-07-15), \emph{Schriftsteller}|pw} auf dem Se{\geminationm}ering\oindex{Semmering@\textbf{Semmering}|pw} – von der Bahn aus – ohne in einem Hôtel gewesen zu sein, zu mir {\pb}gefahren. Irgend eine – hoffentlich
               – wiederum nur hypochondrische Sache – diesmals Zungenkrebs – hat ihn ganz verstört.
               Er möchte dass Sie ihm rathen zu {\pb}wem er gehen soll – vielleicht sogar mit ihm hingehen. Er will – um Sie sicher zu
               treffen – morgen – Mittwoch – um 10\textsuperscript{h}.
                  Vorm. zu Ihnen ko{\geminationm}en, und bat mich Sie zu
               verständigen – was {\pb}ich hiemit
               tue –\pend
           \pstart
           Herzlichst Ihr{\\[\baselineskip]}\spacefill\mbox{Richard}\pend
           \leftskip=0em{}\pstart
           \noindent{}Lili\pwindex{Cappellini, Lili 13.09.1909 – 26.07.1928@\textsc{Cappellini, Lili} (13.09.1909 – 26.07.1928)|pw} die bei uns vorfuhr hat die Kinder\pwindex{Beer-Hofmann, Naemah 20.12.1898 – 10.11.1971@\textsc{Beer-Hofmann, Naëmah} (20.12.1898 – 10.11.1971)|pwv}\pwindex{Beer-Hofmann, Mirjam 04.09.1897 – 24.12.1984@\textsc{Beer-Hofmann, Mirjam} (04.09.1897 – 24.12.1984)|pwv}\pwindex{Beer-Hofmann, Gabriel 09.01.1901 – 24.03.1971@\textsc{Beer-Hofmann, Gabriel} (09.01.1901 – 24.03.1971), \emph{Schriftsteller, Filmagent}|pwv} –
                  durch ihr elegantes und energisches Lutschen – sehr entzückt.\pend
           
         
         \endnumbering\mylabel{h}\end{ledgroupsized}  \newcommand{\dateiname}{L01890}\newcommand{\titel}{Richard Beer-Hofmann an Arthur Schnitzler, 30. 11. 1909}\newcommand{\editorInnen}{Martin Anton Müller und Gerd-Hermann Susen}%% latex-leseansicht-abspann.tex
%% Abspann für die Leseansicht.
%% Der Schalter \ifkorrekturansicht ist bereits durch den Vorspann gesetzt.

%% latex-abspann.tex
%% Gemeinsamer Abspann für Korrekturansicht und Leseansicht.
%% Setzt den Schalter \ifkorrekturansicht voraus (gesetzt in den
%% einbindenden Dateien latex-korrekturansicht-abspann.tex bzw.
%% latex-leseansicht-abspann.tex).
%% ---------------------------------------------------------------

\normalsize

% Das esempio-Environment wird nur in der Leseansicht benötigt
\ifkorrekturansicht\else
\newenvironment{esempio}[3]%
{
    \vspace{1.5ex}
    \rlap{\underline{#1}}
    \par
    \setlength{\parindent}{0cm}
    \nopagebreak
    \leftskip=#2cm
    \rightskip=#3cm
}
{
    \par
}
\fi

\doendnotes{C}
\bigskip
\vfill

\clearpage

\footnotesize

\ifkorrekturansicht
  \lohead{\textsc{register}}
\fi

% theindex-Environment neu definieren ohne reledmac
\makeatletter
\renewenvironment{theindex}{%
  \ifkorrekturansicht
    \section*{\indexname}%
  \else
    \subsubsection*{Index der erwähnten Entitäten}%
  \fi
  \setlength{\parindent}{0pt}%
  \setlength{\parskip}{0pt plus 0.3pt}%
  \let\item\@idxitem
}{%
  \ifkorrekturansicht\clearpage\fi
}
\makeatother

\IfFileExists{\jobname-pw.ind}{\input{\jobname-pw.ind}}{}

% Quellenangabe nur in der Leseansicht
\ifkorrekturansicht\else
% Fallback-Definitionen, falls die .tex-Datei \titel etc. nicht gesetzt hat
\providecommand{\titel}{}
\providecommand{\editorInnen}{}
\providecommand{\dateiname}{\jobname}

\vspace{3cm}

\vfill

\footnotesize
\textsc{Quelle}: \titel. Herausgegeben von {\editorInnen}. In: \emph{Arthur Schnitzler: Briefwechsel mit Autorinnen und Autoren}.
 Digitale Edition, https://schnitzler-briefe.acdh.oeaw.ac.at/{\dateiname}.html (Stand \today)
\fi

\end{document}


      