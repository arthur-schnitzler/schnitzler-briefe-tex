%% latex-leseansicht-vorspann.tex
%% Vorspann für die Leseansicht.
%% Lädt die gemeinsame Datei latex-vorspann.tex mit nicht gesetztem Schalter.

\newif\ifkorrekturansicht
\korrekturansichtfalse

\input{../tex-inputs/latex-vorspann}


\section[Richard Beer-Hofmann an Arthur Schnitzler, 30. 11. 1909]{L01890 Richard Beer-Hofmann an Arthur Schnitzler, 30. 11. 1909}
\nopagebreak\mylabel{L01890v}
\rehead{ }\normalsize\beginnumbering\briefempfaengerindex{Schnitzler, Arthur@\textsc{Schnitzler, Arthur}!zzzBeer-Hofmann, Richard@\emph{von Richard Beer-Hofmann}!1909-11-302@{30. 11. 1909}|(be}
\toendnotes[C]{\smallbreak\pagebreak[2]}
\correspDesc{Versand  durch Richard Beer-Hofmann am 30. 11. 1909 in Wien
\newline{}Erhalt  durch Arthur Schnitzler im Zeitraum [30. 11. 1909 – 4. 12. 1909?] in Wien}\toendnotes[C]{\smallbreak}
\Standort{CUL, Schnitzler, B 8.}
\physDesc{Brief, 1 Blatt, 4 Seiten, 677 Zeichen
\newline{}Handschrift: Bleistift, lateinische Kurrent
\newline{}Schnitzler: mit Bleistift beschriftet: »\textsc{R. Beerhofm}« 
\newline{}Ordnung: 1) mit Bleistift von unbekannter Hand nummeriert: »\strikeout{219}«  2) mit Bleistift von unbekannter Hand nummeriert:
                                    »225«}
\buchAbdrucke{\weitereDrucke{Arthur Schnitzler, Richard Beer-Hofmann: \emph{Briefwechsel 1891–1931}. Herausgegeben von Konstanze Fliedl. Wien, Zürich: \emph{Europaverlag} 1992, S. 195–196.} }\toendnotes[C]{\smallbreak}
\pstart
           \raggedleft{}{\pb}30/XI 09{\\}10 ¾ Nachts\pend
           \vspace{0.5em}
\pstart
           Lieber Arthur!{ }Poldi Andrian\pwindex{Andrian-Werburg, Leopold von 9.\,5.\,1875 Berlin – 19.\,11.\,1951 Fribourg@\textsc{Andrian-Werburg, Leopold von} (9.\,5.\,1875 Berlin – 19.\,11.\,1951 Fribourg), \emph{Schriftsteller, Diplomat}|pw} geht eben weg; er ist – Felix Oppenheimer\pwindex{Oppenheimer, Felix von 20.\,2.\,1874 Wien – 15.\,11.\,1938 ebd.@\textsc{Oppenheimer, Felix von} (20.\,2.\,1874 Wien – 15.\,11.\,1938 ebd.), \emph{Schriftsteller, Soziologe, Mäzen}|pw} ist vor dem \label{K_L01890-1v}\edtext{Leichenbegängnis}{\lemma{\textnormal{\emph{Leichenbegängnis}}}\Cendnote{\textnormal{Die Überführung aus dem Trauerhaus in der Reisnerstraße 28\oindex{Wien@\textbf{Wien}!III., Landstraße@\textbf{III., Landstraße}!Reisnerstraße@\textbf{Reisnerstraße}, \emph{Straße}|pwk} auf den Friedhof fand am
                     30. 11. 1909 statt.}}}\label{K_L01890-1} seines Vaters\pwindex{Oppenheimer, Ludwig von 21.\,8.\,1843 Leipzig – 27.\,11.\,1909 Wien@\textsc{Oppenheimer, Ludwig von} (21.\,8.\,1843 Leipzig – 27.\,11.\,1909 Wien), \emph{Politiker, Unternehmer}|pwv} – Hugo\pwindex{Hofmannsthal, Hugo von 1.\,2.\,1874 Wien – 15.\,7.\,1929 Rodaun@\textsc{Hofmannsthal, Hugo von} (1.\,2.\,1874 Wien – 15.\,7.\,1929 Rodaun), \emph{Schriftsteller}|pw} auf dem Se{\geminationm}ering\oindex{Semmering@\textbf{Semmering}, \emph{Verwaltungsgebiet}|pw} – von der Bahn aus – ohne in einem Hôtel gewesen zu sein, zu mir {\pb}gefahren. Irgend eine – hoffentlich
               – wiederum nur hypochondrische Sache – diesmals Zungenkrebs – hat ihn ganz verstört.
               Er möchte dass Sie ihm rathen zu {\pb}wem er gehen soll – vielleicht sogar mit ihm hingehen. Er will – um Sie sicher zu
               treffen – morgen – Mittwoch – um 10\textsuperscript{h}.
                  Vorm. zu Ihnen ko{\geminationm}en, und bat mich Sie zu
               verständigen – was {\pb}ich hiemit
               tue –\pend
           
\pstart
           Herzlichst Ihr{\\[\baselineskip]}\spacefill\mbox{Richard}\pend
           \leftskip=0em{}
\pstart
           \noindent{}Lili\pwindex{Cappellini, Lili 13.\,9.\,1909 Wien – 26.\,7.\,1928 Venedig@\textsc{Cappellini, Lili} (13.\,9.\,1909 Wien – 26.\,7.\,1928 Venedig)|pw} die bei uns vorfuhr hat die Kinder\pwindex{Beer-Hofmann, Naëmah 20.\,12.\,1898 Wien – 10.\,11.\,1971 New York City@\textsc{Beer-Hofmann, Naëmah} (20.\,12.\,1898 Wien – 10.\,11.\,1971 New York City)|pwv}\pwindex{Beer-Hofmann, Mirjam 4.\,9.\,1897 Wien – 24.\,12.\,1984 New York City@\textsc{Beer-Hofmann, Mirjam} (4.\,9.\,1897 Wien – 24.\,12.\,1984 New York City)|pwv}\pwindex{Beer-Hofmann, Gabriel 9.\,1.\,1901 Wien – 24.\,3.\,1971 St Albans@\textsc{Beer-Hofmann, Gabriel} (9.\,1.\,1901 Wien – 24.\,3.\,1971 St Albans), \emph{Schriftsteller, Filmagent}|pwv} –
                  durch ihr elegantes und energisches Lutschen – sehr entzückt.\pend
           \selectlanguage{ngerman}\endnumbering\briefempfaengerindex{Schnitzler, Arthur@\textsc{Schnitzler, Arthur}!zzzBeer-Hofmann, Richard@\emph{von Richard Beer-Hofmann}!1909-11-302@{30. 11. 1909}|)be}\mylabel{L01890h}  \newcommand{\dateiname}{L01890}\newcommand{\titel}{Richard Beer-Hofmann an Arthur Schnitzler, 30. 11. 1909}\newcommand{\editorInnen}{Martin Anton Müller und Gerd-Hermann Susen}%% latex-leseansicht-abspann.tex
%% Abspann für die Leseansicht.
%% Der Schalter \ifkorrekturansicht ist bereits durch den Vorspann gesetzt.

%% latex-abspann.tex
%% Gemeinsamer Abspann für Korrekturansicht und Leseansicht.
%% Setzt den Schalter \ifkorrekturansicht voraus (gesetzt in den
%% einbindenden Dateien latex-korrekturansicht-abspann.tex bzw.
%% latex-leseansicht-abspann.tex).
%% ---------------------------------------------------------------

\normalsize

% Das esempio-Environment wird nur in der Leseansicht benötigt
\ifkorrekturansicht\else
\newenvironment{esempio}[3]%
{
    \vspace{1.5ex}
    \rlap{\underline{#1}}
    \par
    \setlength{\parindent}{0cm}
    \nopagebreak
    \leftskip=#2cm
    \rightskip=#3cm
}
{
    \par
}
\fi

\doendnotes{C}
\bigskip
\vfill

\clearpage

\footnotesize

\ifkorrekturansicht
  \lohead{\textsc{register}}
\fi

% theindex-Environment neu definieren ohne reledmac
\makeatletter
\renewenvironment{theindex}{%
  \ifkorrekturansicht
    \section*{\indexname}%
  \else
    \subsubsection*{Index der erwähnten Entitäten}%
  \fi
  \setlength{\parindent}{0pt}%
  \setlength{\parskip}{0pt plus 0.3pt}%
  \let\item\@idxitem
}{%
  \ifkorrekturansicht\clearpage\fi
}
\makeatother

\IfFileExists{\jobname-pw.ind}{\input{\jobname-pw.ind}}{}

% Quellenangabe nur in der Leseansicht
\ifkorrekturansicht\else
% Fallback-Definitionen, falls die .tex-Datei \titel etc. nicht gesetzt hat
\providecommand{\titel}{}
\providecommand{\editorInnen}{}
\providecommand{\dateiname}{\jobname}

\vspace{3cm}

\vfill

\footnotesize
\textsc{Quelle}: \titel. Herausgegeben von {\editorInnen}. In: \emph{Arthur Schnitzler: Briefwechsel mit Autorinnen und Autoren}.
 Digitale Edition, https://schnitzler-briefe.acdh.oeaw.ac.at/{\dateiname}.html (Stand \today)
\fi

\end{document}


