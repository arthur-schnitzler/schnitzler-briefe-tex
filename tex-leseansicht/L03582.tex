%% latex-korrekturansicht-vorspann.tex
%% Vorspann für die Korrekturansicht.
%% Lädt die gemeinsame Datei latex-vorspann.tex mit gesetztem Schalter.

\newif\ifkorrekturansicht
\korrekturansichttrue

\input{../tex-inputs/latex-vorspann}


\section[ Felix Salten an Arthur Schnitzler, 17. 8. 1922]{L03582 Felix Salten an Arthur Schnitzler, 17. 8. 1922}
\nopagebreak\mylabel{L03582v}
\rehead{ }\normalsize\beginnumbering\briefempfaengerindex{Schnitzler, Arthur@\textsc{Schnitzler, Arthur}!zzzSalten, Felix@\emph{von Felix Salten}!1922-08-171@{17. 8. 1922}|(be}
\toendnotes[C]{\smallbreak\pagebreak[2]}\Standort{CUL, Schnitzler, B 89, B 2.}
\physDesc{Brief, 1 Blatt, 1 Seite, 388 Zeichen
\newline{}Handschrift: schwarze Tinte, lateinische Kurrent
\newline{}Ordnung: 1) mit Bleistift von Frieda Pollak\pwindex{Pollak, Frieda 08.12.1881 – 13.07.1937@\textsc{Pollak, Frieda} (08.12.1881 – 13.07.1937), \emph{Sekretär/Sekretärin}|pw} (?) mit
                                 dem Buchstaben »A« (Abgeschrieben/Abschrift)
                                 gekennzeichnet  2) mit Bleistift von unbekannter Hand nummeriert: »29\substVorne{}\textsuperscript{3}\substDazwischen{}2\substHinten{}.«}\toendnotes[C]{\smallbreak}
\pstart
           \raggedleft{}{\pb}Berghof\oindex{Berghof@\textbf{Berghof}, \emph{Wohngebäude (K.WHS)}|pw}, 17. 8. 22.\pend
           \vspace{0.5em}
\pstart
           Lieber, vielen Dank für Ihre \label{K_L03582-1v}\edtext{Karte}{\lemma{\textnormal{\emph{Karte}}}\Cendnote{\textnormal{nicht
                  erhalten}}}\label{K_L03582-1}. Es geht uns allen ganz gut. Ich bin seit drei Wochen da\oindex{Unterach am Attersee@\textbf{Unterach am Attersee}, \emph{P.PPL}|pwv} und faullenze. Lassen Sie
               sich das beiliegende kleine \label{K_L03582-2v}\edtext{Buch\pwindex{Burgtheater. Naturgeschichte eines alten Hauses@\emph{Das Burgtheater. Naturgeschichte eines alten Hauses}|pwv}}{\lemma{\textnormal{\emph{Buch}}}\Cendnote{\textnormal{Beilage nicht erhalten; vermutlich war
                  es: Felix Salten\pwindex{Salten, Felix 06.09.1869 – 08.10.1945@\textsc{Salten, Felix} (06.09.1869 – 08.10.1945), \emph{Schriftsteller/Schriftstellerin, Journalist/Journalistin, Chefredakteur/Chefredakteurin}|pwk}: \emph{Das Burgtheater. Naturgeschichte eines alten Hauses}\pwindex{Burgtheater. Naturgeschichte eines alten Hauses@\emph{Das Burgtheater. Naturgeschichte eines alten Hauses}|pwk}.
                     Wien, Leipzig: \emph{WILA Wiener literarische Anstalt}\orgindex{Wiener Literarische Anstalt@Wiener Literarische Anstalt|pwk}{ }1922.}}}\label{K_L03582-2} gefallen. Und – wenn es irgend geht, – aber es ginge gewiß! –
                  \label{K_L03582-3v}\edtext{kommen Sie doch jetzt}{\lemma{\textnormal{\emph{kommen Sie doch jetzt}}}\Cendnote{\textnormal{Zu Schnitzlers Verhältnis zum Berghof\oindex{Berghof@\textbf{Berghof}, \emph{Wohngebäude (K.WHS)}|pwk}{ }siehe Felix Salten an Arthur Schnitzler, [25.? 8. 1892].}}}\label{K_L03582-3}, da Sie so
               nahe sind, auf der Heimfahrt wenigstens für ein paar Tage zu uns. Wir würden uns alle
               so sehr mit Ihnen freuen!\pend
           
\pstart
           Herzlichst Ihr {\\[\baselineskip]}\spacefill\mbox{Salten}\pend
           \leftskip=0em{}\selectlanguage{ngerman}\endnumbering\briefempfaengerindex{Schnitzler, Arthur@\textsc{Schnitzler, Arthur}!zzzSalten, Felix@\emph{von Felix Salten}!1922-08-171@{17. 8. 1922}|)be}\mylabel{L03582h}  \normalsize

\doendnotes{C}
\bigskip
\vfill

\clearpage

\footnotesize

\lohead{\textsc{register}}

% Definiere theindex-Environment komplett neu ohne reledmac
\makeatletter
\renewenvironment{theindex}{%
  \section*{\indexname}%
  \setlength{\parindent}{0pt}%
  \setlength{\parskip}{0pt plus 0.3pt}%
  \let\item\@idxitem
}{%
  \clearpage
}
\makeatother

\IfFileExists{\jobname-pw.ind}{\input{\jobname-pw.ind}}{}

\end{document}

      