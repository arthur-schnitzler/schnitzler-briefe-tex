%% latex-leseansicht-vorspann.tex
%% Vorspann für die Leseansicht.
%% Lädt die gemeinsame Datei latex-vorspann.tex mit nicht gesetztem Schalter.

\newif\ifkorrekturansicht
\korrekturansichtfalse

\input{../tex-inputs/latex-vorspann}

\begin{center}
            \textcolor{red}{ENTWURF, NICHT FERTIG KORRIGIERT}
                      \end{center}
            
         
         \newcommand{\erwaehntePersonen}{Personen: François Bassompierre, Giovanni Boccaccio, Johann Wolfgang von Goethe, Hugo von Hofmannsthal}
         \newcommand{\erwaehnteInstitutionen}{}
         \newcommand{\erwaehnteOrte}{Orte: Berlin, Dessauer Straße, Frankreich, Wien}
         \newcommand{\erwaehnteWerke}{Werke: Das Erlebnis des Marschalls von Bassompierre, Decamerone, Der Schleier der Beatrice. Schauspiel in fünf Akten, Die Geschichte des Marschalls von Bassompierre, Die Zeit. Wiener Wochenschrift, Frankfurter Zeitung, Memoires du mareschal de Bassompierre, contenant l'histoire de sa vie et de ce qui s'est fait de plus remarquable à la cour de France pendant quelques années. 2 Bde., Unterhaltungen deutscher Ausgewanderten}
               \section[ Paul Goldmann an Arthur Schnitzler, 1. 12. {[}1900{]}]{ Paul Goldmann an Arthur Schnitzler, 1. 12. {[}1900{]}}\nopagebreak\mylabel{v}\rehead{ }\begin{ledgroupsized}[t]{13cm}\normalsize\beginnumbering \toendnotes[C]{\smallbreak\pagebreak[2]} \Standort{DLA, A:Schnitzler, HS.NZ85.1.3170.}
\physDesc{Brief, 1 Blatt, 2 Seiten
\newline{}Handschrift: blaue Tinte, deutsche Kurrent\newline{}Beilage: ein Zeitungsausschnitt, beschnitten und aufgeklebt 
\newline{}Schnitzler: 1) mit Bleistift das Jahr »{[}1{]}900« vermerkt  2) mit rotem Buntstift eine Unterstreichung}\toendnotes[C]{\smallbreak}\pstart{}{\pb}\textcolor{gray}{\textbf{DESSAUERSTRASSE 19}}\oindex{Dessauer Strasse@\textbf{Dessauer Straße}|pw}\pend{}{\bigskip}\pstart
           Berlin\oindex{Berlin@\textbf{Berlin}|pw}, 1. Dezember.\pend
           \pstart\center{}Mein lieber Freund,\pend\pstart
           Es iſt leider doch nicht gegangen. Ich muß hier\oindex{Berlin@\textbf{Berlin}|pwv} bleiben und kann Dich \label{K_L02941-1v}\edtext{heut{ }Abend}{\lemma{\textnormal{\emph{heut Abend}}}\Cendnote{\textnormal{zur Uraufführung von \emph{Der Schleier der Beatrice}\pwindex{Schnitzler, Arthur 15.05.1862 – 21.10.1931@\textsc{Schnitzler, Arthur} (15.05.1862 – 21.10.1931), \emph{Schriftsteller, Mediziner}!Schleier der Beatrice. Schauspiel in fuenf Akten1900-12-01@\strich\emph{Der Schleier der Beatrice. Schauspiel in fünf Akten} {[}1900-12-01{]}|pwk}}}}\label{K_L02941-1h} nur mit allen guten Wünſchen begleiten. Wenn Du dieſen Brief erhältſt, biſt
               Du hoffentlich wieder um einen Erfolg\pwindex{Schnitzler, Arthur 15.05.1862 – 21.10.1931@\textsc{Schnitzler, Arthur} (15.05.1862 – 21.10.1931), \emph{Schriftsteller, Mediziner}!Schleier der Beatrice. Schauspiel in fuenf Akten1900-12-01@\strich\emph{Der Schleier der Beatrice. Schauspiel in fünf Akten} {[}1900-12-01{]}|pwv} reicher.\pend
           \pstart
           Beifolgenden \label{K_L02941-2v}\edtext{Artikel\pwindex{?? Werk@Nicht ermittelte Verfasserinnen und Verfasser!Geschichte des Marschalls von Bassompierre1900-12-01@\emph{Die Geschichte des Marschalls von Bassompierre} {[}1900-12-01{]}|pwv}}{\lemma{\textnormal{\emph{Artikel}}}\Cendnote{\textnormal{XXXX}}}\label{K_L02941-2h}, der Deinen Freund \textsc{Hoffmannsthal\pwindex{Hofmannsthal, Hugo von 1874-02-01 – 1929-07-15@\textsc{Hofmannsthal, Hugo von} (1874-02-01 – 1929-07-15), \emph{Schriftsteller}|pw}} betrifft, finde ich heut in der {\pb}»Frankfurter
                  Zeitung\pwindex{?? Werk@Nicht ermittelte Verfasserinnen und Verfasser!Frankfurter Zeitung1856 – 1943@\emph{Frankfurter Zeitung} {[}1856 – 1943{]}|pw}«.\pend
           \pstart
           Viele treue Grüße! {\\[\baselineskip]}{\pb}Dein {\\[\baselineskip]}\spacefill\mbox{Paul Goldmn.}\pend
           \leftskip=0em{}\pstart
           {\pb}\textcolor{gray}{\textbf{= \textbf{[Die
                        Geſchichte des Marſchalls von Baſſompierre\pwindex{Hofmannsthal, Hugo von 1874-02-01 – 1929-07-15@\textsc{Hofmannsthal, Hugo von} (1874-02-01 – 1929-07-15), \emph{Schriftsteller}!Erlebnis des Marschalls von Bassompierre1900@\strich\emph{Das Erlebnis des Marschalls von Bassompierre} {[}1900{]}|pw}.]}}}\pend
           \pstart
           \textcolor{gray}{\textbf{Ein Vorkommniß, das in literariſchen Kreiſen von ſich reden
                  macht, verdient um der Perſonen willen, die daran betheiligt ſind, allgemeinere
                  Beachtung. Die dieswöchentliche Wien\oindex{Wien@\textbf{Wien}|pw}er »Zeit\pwindex{Zeit. Wiener Wochenschrift1894 – 1904@\emph{Die Zeit. Wiener Wochenschrift} {[}1894 – 1904{]}|pw}« enthält}}\textcolor{gray}{\textbf{den Anfang einer Erzählung\pwindex{Hofmannsthal, Hugo von 1874-02-01 – 1929-07-15@\textsc{Hofmannsthal, Hugo von} (1874-02-01 – 1929-07-15), \emph{Schriftsteller}!Erlebnis des Marschalls von Bassompierre1900@\strich\emph{Das Erlebnis des Marschalls von Bassompierre} {[}1900{]}|pwv}, die betitelt iſt: »\so{Erlebniß des Marſchalls von Baſſompierre\pwindex{Hofmannsthal, Hugo von 1874-02-01 – 1929-07-15@\textsc{Hofmannsthal, Hugo von} (1874-02-01 – 1929-07-15), \emph{Schriftsteller}!Erlebnis des Marschalls von Bassompierre1900@\strich\emph{Das Erlebnis des Marschalls von Bassompierre} {[}1900{]}|pw}}« und als \so{Verfaſſer} nennt ſich der hochſtrebende
                     Wien\oindex{Wien@\textbf{Wien}|pw}er Poet \so{Hugo v. Hofmannsthal}\pwindex{Hofmannsthal, Hugo von 1874-02-01 – 1929-07-15@\textsc{Hofmannsthal, Hugo von} (1874-02-01 – 1929-07-15), \emph{Schriftsteller}|pw}. Dieſe Erzählung\pwindex{Hofmannsthal, Hugo von 1874-02-01 – 1929-07-15@\textsc{Hofmannsthal, Hugo von} (1874-02-01 – 1929-07-15), \emph{Schriftsteller}!Erlebnis des Marschalls von Bassompierre1900@\strich\emph{Das Erlebnis des Marschalls von Bassompierre} {[}1900{]}|pwv}
                  behandelt nicht nur den nämlichen Vorfall, den in \so{Goethe}\pwindex{Goethe, Johann Wolfgang von 1749-08-28 – 1832-03-22@\textsc{Goethe, Johann Wolfgang von} (1749-08-28 – 1832-03-22), \emph{Schriftsteller}|pw}’s »\so{Unterhaltungen deutſcher Ausgewanderten}\pwindex{Goethe, Johann Wolfgang von 1749-08-28 – 1832-03-22@\textsc{Goethe, Johann Wolfgang von} (1749-08-28 – 1832-03-22), \emph{Schriftsteller}!Unterhaltungen deutscher Ausgewanderten1795@\strich\emph{Unterhaltungen deutscher Ausgewanderten} {[}1795{]}|pw}« Vetter Karl\pwindex{Goethe, Johann Wolfgang von 1749-08-28 – 1832-03-22@\textsc{Goethe, Johann Wolfgang von} (1749-08-28 – 1832-03-22), \emph{Schriftsteller}!Unterhaltungen deutscher Ausgewanderten1795@\strich\emph{Unterhaltungen deutscher Ausgewanderten} {[}1795{]}|pwv} auf dem
                     »Gut am rechten Ufer des
                     Rheins\pwindex{Goethe, Johann Wolfgang von 1749-08-28 – 1832-03-22@\textsc{Goethe, Johann Wolfgang von} (1749-08-28 – 1832-03-22), \emph{Schriftsteller}!Unterhaltungen deutscher Ausgewanderten1795@\strich\emph{Unterhaltungen deutscher Ausgewanderten} {[}1795{]}|pwv}« zum Beſten gibt, ſondern, obgleich ſie weit ausführlicher und
                  zufolge ihres näheren Eingehens ins Einzelne blühender iſt, als bei Goethe\pwindex{Goethe, Johann Wolfgang von 1749-08-28 – 1832-03-22@\textsc{Goethe, Johann Wolfgang von} (1749-08-28 – 1832-03-22), \emph{Schriftsteller}!Unterhaltungen deutscher Ausgewanderten1795@\strich\emph{Unterhaltungen deutscher Ausgewanderten} {[}1795{]}|pwv}\pwindex{Goethe, Johann Wolfgang von 1749-08-28 – 1832-03-22@\textsc{Goethe, Johann Wolfgang von} (1749-08-28 – 1832-03-22), \emph{Schriftsteller}|pw}, der die Hauptvorgänge ſtraff zuſammenzufaſſen ſich begnügt, kann es keinem
                  Zweifel unterliegen, daß Beide\pwindex{Goethe, Johann Wolfgang von 1749-08-28 – 1832-03-22@\textsc{Goethe, Johann Wolfgang von} (1749-08-28 – 1832-03-22), \emph{Schriftsteller}|pwv}\pwindex{Hofmannsthal, Hugo von 1874-02-01 – 1929-07-15@\textsc{Hofmannsthal, Hugo von} (1874-02-01 – 1929-07-15), \emph{Schriftsteller}|pwv}, der Alte\pwindex{Goethe, Johann Wolfgang von 1749-08-28 – 1832-03-22@\textsc{Goethe, Johann Wolfgang von} (1749-08-28 – 1832-03-22), \emph{Schriftsteller}|pwv} wie der Junge\pwindex{Hofmannsthal, Hugo von 1874-02-01 – 1929-07-15@\textsc{Hofmannsthal, Hugo von} (1874-02-01 – 1929-07-15), \emph{Schriftsteller}|pwv}, aus der gleichen Quellen geſchöpft haben. Und Beide\pwindex{Goethe, Johann Wolfgang von 1749-08-28 – 1832-03-22@\textsc{Goethe, Johann Wolfgang von} (1749-08-28 – 1832-03-22), \emph{Schriftsteller}|pwv}\pwindex{Hofmannsthal, Hugo von 1874-02-01 – 1929-07-15@\textsc{Hofmannsthal, Hugo von} (1874-02-01 – 1929-07-15), \emph{Schriftsteller}|pwv} lehnen
                  ſich ſo deutlich an das \label{K_L02941-34v}\edtext{fran\oindex{Frankreich@\textbf{Frankreich}|pwv}zöſiſche Original\pwindex{Bassompierre, François 1579-04-12 – 1646-10-12@\textsc{Bassompierre, François} (1579-04-12 – 1646-10-12), \emph{Diplomat, Politiker}!Memoires du mareschal de Bassompierre, contenant l'histoire de sa vie et de ce qui s'est fait de plus remarquable à la cour de France pendant quelques annees. 2 Bde.1665@\strich\emph{Memoires du mareschal de Bassompierre, contenant l'histoire de sa vie et de ce qui s'est fait de plus remarquable à la cour de France pendant quelques années. 2 Bde.} {[}1665{]}|pwv}}{\lemma{\textnormal{\emph{franzöſiſche Original}}}\Cendnote{\textnormal{gemeint sind François Bassompierre\pwindex{Bassompierre, François 1579-04-12 – 1646-10-12@\textsc{Bassompierre, François} (1579-04-12 – 1646-10-12), \emph{Diplomat, Politiker}|pwk}s \emph{Memoires du mareschal de Bassompierre}\pwindex{Bassompierre, François 1579-04-12 – 1646-10-12@\textsc{Bassompierre, François} (1579-04-12 – 1646-10-12), \emph{Diplomat, Politiker}!Memoires du mareschal de Bassompierre, contenant l'histoire de sa vie et de ce qui s'est fait de plus remarquable à la cour de France pendant quelques annees. 2 Bde.1665@\strich\emph{Memoires du mareschal de Bassompierre, contenant l'histoire de sa vie et de ce qui s'est fait de plus remarquable à la cour de France pendant quelques années. 2 Bde.} {[}1665{]}|pwk} (1665, 2 Bde.), wobei Goethe\pwindex{Goethe, Johann Wolfgang von 1749-08-28 – 1832-03-22@\textsc{Goethe, Johann Wolfgang von} (1749-08-28 – 1832-03-22), \emph{Schriftsteller}|pwk}s Rahmenhandlung\pwindex{Goethe, Johann Wolfgang von 1749-08-28 – 1832-03-22@\textsc{Goethe, Johann Wolfgang von} (1749-08-28 – 1832-03-22), \emph{Schriftsteller}!Unterhaltungen deutscher Ausgewanderten1795@\strich\emph{Unterhaltungen deutscher Ausgewanderten} {[}1795{]}|pwkv} an Giovanni
                        Boccaccio\pwindex{Boccaccio, Giovanni 1313-06-16 – 1375-12-21@\textsc{Boccaccio, Giovanni} (1313-06-16 – 1375-12-21), \emph{Schriftsteller}|pwk}s \emph{Decamerone}\pwindex{Boccaccio, Giovanni 1313-06-16 – 1375-12-21@\textsc{Boccaccio, Giovanni} (1313-06-16 – 1375-12-21), \emph{Schriftsteller}!Decamerone@\strich\emph{Decamerone}|pwk} angelehnt
                     ist}}}\label{K_L02941-34h} an, daß ihre Schilderungen in ganzen Sätzen übereinſtimmen, aber
                  ſich auch untereinander im Ton des Vortrags außerordentlich ähneln. Daß Goethe\pwindex{Goethe, Johann Wolfgang von 1749-08-28 – 1832-03-22@\textsc{Goethe, Johann Wolfgang von} (1749-08-28 – 1832-03-22), \emph{Schriftsteller}|pw}, in deſſen Decamerone-Nachbildung\pwindex{Goethe, Johann Wolfgang von 1749-08-28 – 1832-03-22@\textsc{Goethe, Johann Wolfgang von} (1749-08-28 – 1832-03-22), \emph{Schriftsteller}!Unterhaltungen deutscher Ausgewanderten1795@\strich\emph{Unterhaltungen deutscher Ausgewanderten} {[}1795{]}|pwv} das Abenteuer
                  des Marſchalls eine raſch vorübergehende Epiſode, gewiſſermaßen nur ein
                  nebenſächliches Illuſtrationsfaktum iſt, von {[}dem{]}{ }Hofmannsthal\pwindex{Hofmannsthal, Hugo von 1874-02-01 – 1929-07-15@\textsc{Hofmannsthal, Hugo von} (1874-02-01 – 1929-07-15), \emph{Schriftsteller}|pw} nichts gewußt hat, darf man
                  dreiſt vorausſetzen. Merkwürdig iſt nur, daß \so{dieſem}\pwindex{Hofmannsthal, Hugo von 1874-02-01 – 1929-07-15@\textsc{Hofmannsthal, Hugo von} (1874-02-01 – 1929-07-15), \emph{Schriftsteller}|pwv} die Behandlung des Motivs durch Goethe\pwindex{Goethe, Johann Wolfgang von 1749-08-28 – 1832-03-22@\textsc{Goethe, Johann Wolfgang von} (1749-08-28 – 1832-03-22), \emph{Schriftsteller}|pw} unbekannt geblieben iſt, denn wäre dies \so{nicht} der Fall geweſen, ſo hätte er doch ſicher auf die Arbeit\pwindex{Goethe, Johann Wolfgang von 1749-08-28 – 1832-03-22@\textsc{Goethe, Johann Wolfgang von} (1749-08-28 – 1832-03-22), \emph{Schriftsteller}!Unterhaltungen deutscher Ausgewanderten1795@\strich\emph{Unterhaltungen deutscher Ausgewanderten} {[}1795{]}|pwv} ſeines großen Vorgänger\pwindex{Goethe, Johann Wolfgang von 1749-08-28 – 1832-03-22@\textsc{Goethe, Johann Wolfgang von} (1749-08-28 – 1832-03-22), \emph{Schriftsteller}|pwv}s verwieſen. Noch merkwürdiger
                  iſt, daß ſich Hofmannsthal\pwindex{Hofmannsthal, Hugo von 1874-02-01 – 1929-07-15@\textsc{Hofmannsthal, Hugo von} (1874-02-01 – 1929-07-15), \emph{Schriftsteller}|pw} als \so{Verfaſſer} dieſer Gedichte\pwindex{Hofmannsthal, Hugo von 1874-02-01 – 1929-07-15@\textsc{Hofmannsthal, Hugo von} (1874-02-01 – 1929-07-15), \emph{Schriftsteller}!Erlebnis des Marschalls von Bassompierre1900@\strich\emph{Das Erlebnis des Marschalls von Bassompierre} {[}1900{]}|pwv} bezeichnet, da, ſelbſt wenn die allerliebſten
                  Stimmungsſchilderungen der Erzählung\pwindex{Hofmannsthal, Hugo von 1874-02-01 – 1929-07-15@\textsc{Hofmannsthal, Hugo von} (1874-02-01 – 1929-07-15), \emph{Schriftsteller}!Erlebnis des Marschalls von Bassompierre1900@\strich\emph{Das Erlebnis des Marschalls von Bassompierre} {[}1900{]}|pwv} ſein Eigenthum ſein ſollten, eine Hindeutung auf das Originalwerk\pwindex{Bassompierre, François 1579-04-12 – 1646-10-12@\textsc{Bassompierre, François} (1579-04-12 – 1646-10-12), \emph{Diplomat, Politiker}!Memoires du mareschal de Bassompierre, contenant l'histoire de sa vie et de ce qui s'est fait de plus remarquable à la cour de France pendant quelques annees. 2 Bde.1665@\strich\emph{Memoires du mareschal de Bassompierre, contenant l'histoire de sa vie et de ce qui s'est fait de plus remarquable à la cour de France pendant quelques années. 2 Bde.} {[}1665{]}|pwv} unter keinen
                  Umſtänden zu vermeiden war. Die Zeit, wo man auf das Titelblatt von Komödien und
                  Proſaſchriften einfach zu ſchreiben pflegte: »\so{Nach dem Franzöſiſchen} von X. X.« ſind vorüber, aber ſelbſt damals benützte man die
                  Phraſe »Nach dem Franzöſiſchen«, um, wenn man ſchon den Autor nicht nennen wollte,
                  wenigſtens zuzugeſtehen, daß es ſich um keine Original-Arbeit handle. Da Hugo v. Hofmannsthal\pwindex{Hofmannsthal, Hugo von 1874-02-01 – 1929-07-15@\textsc{Hofmannsthal, Hugo von} (1874-02-01 – 1929-07-15), \emph{Schriftsteller}|pw} nicht nöthig hat, bei
                  fremden Autoren zu leihen, wäre eine \so{Aufklärung} des
                  Falles gewiß von Intereſſe.}}\pend
           
         
         \endnumbering\mylabel{h}\end{ledgroupsized}\begin{anhang}\end{anhang}\newcommand{\dateiname}{L02941}\newcommand{\titel}{Paul Goldmann an Arthur Schnitzler, 1. 12. [1900]}\newcommand{\editorInnen}{Martin Anton Müller und Laura Untner}%% latex-leseansicht-abspann.tex
%% Abspann für die Leseansicht.
%% Der Schalter \ifkorrekturansicht ist bereits durch den Vorspann gesetzt.

%% latex-abspann.tex
%% Gemeinsamer Abspann für Korrekturansicht und Leseansicht.
%% Setzt den Schalter \ifkorrekturansicht voraus (gesetzt in den
%% einbindenden Dateien latex-korrekturansicht-abspann.tex bzw.
%% latex-leseansicht-abspann.tex).
%% ---------------------------------------------------------------

\normalsize

% Das esempio-Environment wird nur in der Leseansicht benötigt
\ifkorrekturansicht\else
\newenvironment{esempio}[3]%
{
    \vspace{1.5ex}
    \rlap{\underline{#1}}
    \par
    \setlength{\parindent}{0cm}
    \nopagebreak
    \leftskip=#2cm
    \rightskip=#3cm
}
{
    \par
}
\fi

\doendnotes{C}
\bigskip
\vfill

\clearpage

\footnotesize

\ifkorrekturansicht
  \lohead{\textsc{register}}
\fi

% theindex-Environment neu definieren ohne reledmac
\makeatletter
\renewenvironment{theindex}{%
  \ifkorrekturansicht
    \section*{\indexname}%
  \else
    \subsubsection*{Index der erwähnten Entitäten}%
  \fi
  \setlength{\parindent}{0pt}%
  \setlength{\parskip}{0pt plus 0.3pt}%
  \let\item\@idxitem
}{%
  \ifkorrekturansicht\clearpage\fi
}
\makeatother

\IfFileExists{\jobname-pw.ind}{\input{\jobname-pw.ind}}{}

% Quellenangabe nur in der Leseansicht
\ifkorrekturansicht\else
% Fallback-Definitionen, falls die .tex-Datei \titel etc. nicht gesetzt hat
\providecommand{\titel}{}
\providecommand{\editorInnen}{}
\providecommand{\dateiname}{\jobname}

\vspace{3cm}

\vfill

\footnotesize
\textsc{Quelle}: \titel. Herausgegeben von {\editorInnen}. In: \emph{Arthur Schnitzler: Briefwechsel mit Autorinnen und Autoren}.
 Digitale Edition, https://schnitzler-briefe.acdh.oeaw.ac.at/{\dateiname}.html (Stand \today)
\fi

\end{document}


      