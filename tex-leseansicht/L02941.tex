%% latex-leseansicht-vorspann.tex
%% Vorspann für die Leseansicht.
%% Lädt die gemeinsame Datei latex-vorspann.tex mit nicht gesetztem Schalter.

\newif\ifkorrekturansicht
\korrekturansichtfalse

\input{../tex-inputs/latex-vorspann}


\section[ Paul Goldmann an Arthur Schnitzler, 1. 12. 1900]{L02941 Paul Goldmann an Arthur Schnitzler,  1. 12. 1900}
\nopagebreak\mylabel{L02941v}
\rehead{ }\normalsize\beginnumbering\briefempfaengerindex{Schnitzler, Arthur@\textsc{Schnitzler, Arthur}!zzzGoldmann, Paul@\emph{von Paul Goldmann}!1900-12-012@{1. 12. 1900}|(be}
\toendnotes[C]{\smallbreak\pagebreak[2]}
\correspDesc{Versand  durch Paul Goldmann am 1. 12. 1900 in Berlin
\newline{}Erhalt  durch Arthur Schnitzler im Zeitraum [2. 12. 1900
                  – 4. 12. 1900?] in Breslau}\toendnotes[C]{\smallbreak}
\Standort{DLA, A:Schnitzler, HS.NZ85.1.3170.}
\physDesc{Brief, 1 Blatt, 2 Seiten, 372 Zeichen
\newline{}Handschrift: blaue Tinte, deutsche Kurrent
\newline{}Beilage: ein Zeitungsausschnitt, beschnitten und aufgeklebt 
\newline{}Schnitzler: 1) mit Bleistift das Jahr »900« vermerkt  2) mit rotem Buntstift eine Unterstreichung}\toendnotes[C]{\smallbreak}
\pstart
           \raggedleft{}{\pb}\textcolor{gray}{\textbf{DESSAUERSTRASSE 19}}\oindex{Dessauer Straße@\textbf{Dessauer Straße}, \emph{Straße}|pw}\pend
           
\pstart
           Berlin\oindex{Berlin@\textbf{Berlin}, \emph{Hauptstadt}|pw}, 1. Dezember.\pend
           
\pstart\center{}Mein lieber Freund,\pend\vspace{0.5em}
\pstart
           Es iſt leider doch nicht gegangen. Ich muß hier\oindex{Berlin@\textbf{Berlin}, \emph{Hauptstadt}|pwv} bleiben und kann Dich \label{K_L02941-1v}\edtext{heut{ }Abend\eventindex{Lobe-Theater@\textbf{Lobe-Theater}!Uraufführung von Der Schleier der Beatrice, 1.12.1900@Uraufführung von Der Schleier der Beatrice, 1.12.1900|pwv}}{\lemma{\textnormal{\emph{heut Abend}}}\Cendnote{\textnormal{zur Uraufführung von \emph{Der Schleier der Beatrice}\pwindex{Schnitzler, Arthur 15.\,5.\,1862 Wien – 21.\,10.\,1931 ebd.@\textsc{Schnitzler, Arthur} (15.\,5.\,1862 Wien – 21.\,10.\,1931 ebd.), \emph{Schriftsteller, Mediziner}!Schleier der Beatrice. Schauspiel in fünf Akten@\strich\emph{Der Schleier der Beatrice. Schauspiel in fünf Akten}|pwk}\eventindex{Lobe-Theater@\textbf{Lobe-Theater}!Uraufführung von Der Schleier der Beatrice, 1.12.1900@Uraufführung von Der Schleier der Beatrice, 1.12.1900|pwk}}}}\label{K_L02941-1} nur mit allen guten Wünſchen begleiten. Wenn Du dieſen Brief erhältſt, biſt
               Du hoffentlich wieder um einen Erfolg\pwindex{Schnitzler, Arthur 15.\,5.\,1862 Wien – 21.\,10.\,1931 ebd.@\textsc{Schnitzler, Arthur} (15.\,5.\,1862 Wien – 21.\,10.\,1931 ebd.), \emph{Schriftsteller, Mediziner}!Schleier der Beatrice. Schauspiel in fünf Akten@\strich\emph{Der Schleier der Beatrice. Schauspiel in fünf Akten}|pwv} reicher.\pend
           
\pstart
           Beifolgenden \label{K_L02941-2v}\edtext{Artikel\pwindex{Geschichte des Marschalls von Bassompierre@\emph{Die Geschichte des Marschalls von Bassompierre}|pwv}}{\lemma{\textnormal{\emph{Artikel}}}\Cendnote{\textnormal{[O. V.]: \emph{Die Geschichte des Marschalls von
                        Bassompierre}\pwindex{Geschichte des Marschalls von Bassompierre@\emph{Die Geschichte des Marschalls von Bassompierre}|pwk}. In: \emph{Frankfurter
                        Zeitung}\pwindex{Frankfurter Zeitung@\emph{Frankfurter Zeitung}|pwk}, Jg. 45, Nr. 331, 30. 11. 1900,
                     Abendblatt, S. 1. Schnitzler
                  teilte das Unverständnis gegenüber Hofmannsthals\pwindex{Hofmannsthal, Hugo von 1.\,2.\,1874 Wien – 15.\,7.\,1929 Rodaun@\textsc{Hofmannsthal, Hugo von} (1.\,2.\,1874 Wien – 15.\,7.\,1929 Rodaun), \emph{Schriftsteller}|pwk} fehlender Bekanntmachung der literarischen Aneignung, vgl. A. S.: \emph{Tagebuch}, 12. 12. 1902.}}}\label{K_L02941-2}, der
               Deinen Freund \textsc{Hoffmannsthal\pwindex{Hofmannsthal, Hugo von 1.\,2.\,1874 Wien – 15.\,7.\,1929 Rodaun@\textsc{Hofmannsthal, Hugo von} (1.\,2.\,1874 Wien – 15.\,7.\,1929 Rodaun), \emph{Schriftsteller}|pw}} betrifft, finde ich heut in der {\pb}»Frankfurter
                  Zeitung\pwindex{Frankfurter Zeitung@\emph{Frankfurter Zeitung}|pw}«.\pend
           
\pstart
           Viele treue Grüße! {\\[\baselineskip]} Dein {\\[\baselineskip]}\spacefill\mbox{Paul Goldmn.}\pend
           \leftskip=0em{}{\vspace{1\baselineskip}}
\pstart
           {\pb}\textcolor{gray}{\textbf{= \textbf{\label{T_L02941-1v}\edtext{[Die Geſchichte des Marſchalls von Baſſompierre\pwindex{Hofmannsthal, Hugo von 1.\,2.\,1874 Wien – 15.\,7.\,1929 Rodaun@\textsc{Hofmannsthal, Hugo von} (1.\,2.\,1874 Wien – 15.\,7.\,1929 Rodaun), \emph{Schriftsteller}!Erlebnis des Marschalls von Bassompierre@\strich\emph{Das Erlebnis des Marschalls von Bassompierre}|pw}.]}{\lemma{\textnormal{\emph{[Die … Bassompierre.]}}}\Cendnote{\textnormal{eckige Klammern in der Druckvorlage}}}\label{T_L02941-1}}}}\pend
           
\pstart
           \textcolor{gray}{\textbf{Ein Vorkommniß, das in literariſchen Kreiſen von{ }ſich reden
                  macht, verdient um der Perſonen willen, die daran betheiligt{ }ſind, allgemeinere
                  Beachtung. Die dieswöchentliche Wien\oindex{Wien@\textbf{Wien}, \emph{Verwaltungsgebiet}|pw}er »Zeit\pwindex{Zeit. Wiener Wochenschrift@\emph{Die Zeit. Wiener Wochenschrift}|pw}« enthält den Anfang einer Erzählung\pwindex{Hofmannsthal, Hugo von 1.\,2.\,1874 Wien – 15.\,7.\,1929 Rodaun@\textsc{Hofmannsthal, Hugo von} (1.\,2.\,1874 Wien – 15.\,7.\,1929 Rodaun), \emph{Schriftsteller}!Erlebnis des Marschalls von Bassompierre@\strich\emph{Das Erlebnis des Marschalls von Bassompierre}|pwv}, die betitelt iſt: »\so{Erlebniß des Marſchalls von Baſſompierre}\pwindex{Hofmannsthal, Hugo von 1.\,2.\,1874 Wien – 15.\,7.\,1929 Rodaun@\textsc{Hofmannsthal, Hugo von} (1.\,2.\,1874 Wien – 15.\,7.\,1929 Rodaun), \emph{Schriftsteller}!Erlebnis des Marschalls von Bassompierre@\strich\emph{Das Erlebnis des Marschalls von Bassompierre}|pw}« und als \so{Verfaſſer} nennt{ }ſich der hochſtrebende
                     Wien\oindex{Wien@\textbf{Wien}, \emph{Verwaltungsgebiet}|pw}er Poet Hugo v. \so{Hofmannsthal}\pwindex{Hofmannsthal, Hugo von 1.\,2.\,1874 Wien – 15.\,7.\,1929 Rodaun@\textsc{Hofmannsthal, Hugo von} (1.\,2.\,1874 Wien – 15.\,7.\,1929 Rodaun), \emph{Schriftsteller}|pw}. Dieſe Erzählung\pwindex{Hofmannsthal, Hugo von 1.\,2.\,1874 Wien – 15.\,7.\,1929 Rodaun@\textsc{Hofmannsthal, Hugo von} (1.\,2.\,1874 Wien – 15.\,7.\,1929 Rodaun), \emph{Schriftsteller}!Erlebnis des Marschalls von Bassompierre@\strich\emph{Das Erlebnis des Marschalls von Bassompierre}|pwv}
                  behandelt nicht nur den nämlichen Vorfall, den in \so{Goethe}\pwindex{Goethe, Johann Wolfgang von 28.\,8.\,1749 Frankfurt am Main – 22.\,3.\,1832 Weimar@\textsc{Goethe, Johann Wolfgang von} (28.\,8.\,1749 Frankfurt am Main – 22.\,3.\,1832 Weimar), \emph{Schriftsteller}|pw}’s »\so{Unterhaltungen deutſcher Ausgewanderten}\pwindex{Goethe, Johann Wolfgang von 28.\,8.\,1749 Frankfurt am Main – 22.\,3.\,1832 Weimar@\textsc{Goethe, Johann Wolfgang von} (28.\,8.\,1749 Frankfurt am Main – 22.\,3.\,1832 Weimar), \emph{Schriftsteller}!Unterhaltungen deutscher Ausgewanderten@\strich\emph{Unterhaltungen deutscher Ausgewanderten}|pw}« Vetter Karl\pwindex{Goethe, Johann Wolfgang von 28.\,8.\,1749 Frankfurt am Main – 22.\,3.\,1832 Weimar@\textsc{Goethe, Johann Wolfgang von} (28.\,8.\,1749 Frankfurt am Main – 22.\,3.\,1832 Weimar), \emph{Schriftsteller}!Unterhaltungen deutscher Ausgewanderten@\strich\emph{Unterhaltungen deutscher Ausgewanderten}|pwv} auf dem
                     »Gut am rechten Ufer des
                     Rheins\pwindex{Goethe, Johann Wolfgang von 28.\,8.\,1749 Frankfurt am Main – 22.\,3.\,1832 Weimar@\textsc{Goethe, Johann Wolfgang von} (28.\,8.\,1749 Frankfurt am Main – 22.\,3.\,1832 Weimar), \emph{Schriftsteller}!Unterhaltungen deutscher Ausgewanderten@\strich\emph{Unterhaltungen deutscher Ausgewanderten}|pwv}« zum Beſten gibt,{ }ſondern, obgleich{ }ſie weit ausführlicher und
                  zufolge ihres näheren Eingehens ins Einzelne blühender iſt, als bei Goethe\pwindex{Goethe, Johann Wolfgang von 28.\,8.\,1749 Frankfurt am Main – 22.\,3.\,1832 Weimar@\textsc{Goethe, Johann Wolfgang von} (28.\,8.\,1749 Frankfurt am Main – 22.\,3.\,1832 Weimar), \emph{Schriftsteller}!Unterhaltungen deutscher Ausgewanderten@\strich\emph{Unterhaltungen deutscher Ausgewanderten}|pwv}\pwindex{Goethe, Johann Wolfgang von 28.\,8.\,1749 Frankfurt am Main – 22.\,3.\,1832 Weimar@\textsc{Goethe, Johann Wolfgang von} (28.\,8.\,1749 Frankfurt am Main – 22.\,3.\,1832 Weimar), \emph{Schriftsteller}|pw}, der die Hauptvorgänge{ }ſtraff zuſammenzufaſſen{ }ſich begnügt, kann es keinem
                  Zweifel unterliegen, daß Beide\pwindex{Goethe, Johann Wolfgang von 28.\,8.\,1749 Frankfurt am Main – 22.\,3.\,1832 Weimar@\textsc{Goethe, Johann Wolfgang von} (28.\,8.\,1749 Frankfurt am Main – 22.\,3.\,1832 Weimar), \emph{Schriftsteller}|pwv}\pwindex{Hofmannsthal, Hugo von 1.\,2.\,1874 Wien – 15.\,7.\,1929 Rodaun@\textsc{Hofmannsthal, Hugo von} (1.\,2.\,1874 Wien – 15.\,7.\,1929 Rodaun), \emph{Schriftsteller}|pwv}, der Alte\pwindex{Goethe, Johann Wolfgang von 28.\,8.\,1749 Frankfurt am Main – 22.\,3.\,1832 Weimar@\textsc{Goethe, Johann Wolfgang von} (28.\,8.\,1749 Frankfurt am Main – 22.\,3.\,1832 Weimar), \emph{Schriftsteller}|pwv} wie der Junge\pwindex{Hofmannsthal, Hugo von 1.\,2.\,1874 Wien – 15.\,7.\,1929 Rodaun@\textsc{Hofmannsthal, Hugo von} (1.\,2.\,1874 Wien – 15.\,7.\,1929 Rodaun), \emph{Schriftsteller}|pwv}, aus der gleichen Quellen geſchöpft haben. Und Beide\pwindex{Goethe, Johann Wolfgang von 28.\,8.\,1749 Frankfurt am Main – 22.\,3.\,1832 Weimar@\textsc{Goethe, Johann Wolfgang von} (28.\,8.\,1749 Frankfurt am Main – 22.\,3.\,1832 Weimar), \emph{Schriftsteller}|pwv}\pwindex{Hofmannsthal, Hugo von 1.\,2.\,1874 Wien – 15.\,7.\,1929 Rodaun@\textsc{Hofmannsthal, Hugo von} (1.\,2.\,1874 Wien – 15.\,7.\,1929 Rodaun), \emph{Schriftsteller}|pwv} lehnen{ }ſich{ }ſo deutlich an das \label{K_L02941-3v}\edtext{fran\oindex{Frankreich@\textbf{Frankreich}|pwv}zöſiſche Original\pwindex{Bassompierre, François 12.\,4.\,1579 – 12.\,10.\,1646@\textsc{Bassompierre, François} (12.\,4.\,1579 – 12.\,10.\,1646), \emph{Politiker, Diplomat}!Memoires du mareschal de Bassompierre, contenant l'histoire de sa vie et de ce qui s'est fait de plus remarquable à la cour de France pendant quelques années. 2 Bde.@\strich\emph{Memoires du mareschal de Bassompierre, contenant l'histoire de sa vie et de ce qui s'est fait de plus remarquable à la cour de France pendant quelques années. 2 Bde.}|pwv}}{\lemma{\textnormal{\emph{französische Original}}}\Cendnote{\textnormal{Gemeint sind François Bassompierres\pwindex{Bassompierre, François 12.\,4.\,1579 – 12.\,10.\,1646@\textsc{Bassompierre, François} (12.\,4.\,1579 – 12.\,10.\,1646), \emph{Politiker, Diplomat}|pwk}{ }\emph{Memoires du mareschal de Bassompierre}\pwindex{Bassompierre, François 12.\,4.\,1579 – 12.\,10.\,1646@\textsc{Bassompierre, François} (12.\,4.\,1579 – 12.\,10.\,1646), \emph{Politiker, Diplomat}!Memoires du mareschal de Bassompierre, contenant l'histoire de sa vie et de ce qui s'est fait de plus remarquable à la cour de France pendant quelques années. 2 Bde.@\strich\emph{Memoires du mareschal de Bassompierre, contenant l'histoire de sa vie et de ce qui s'est fait de plus remarquable à la cour de France pendant quelques années. 2 Bde.}|pwk} (1665, 2 Bde.), wobei Goethes\pwindex{Goethe, Johann Wolfgang von 28.\,8.\,1749 Frankfurt am Main – 22.\,3.\,1832 Weimar@\textsc{Goethe, Johann Wolfgang von} (28.\,8.\,1749 Frankfurt am Main – 22.\,3.\,1832 Weimar), \emph{Schriftsteller}|pwk}{ }Rahmenhandlung\pwindex{Goethe, Johann Wolfgang von 28.\,8.\,1749 Frankfurt am Main – 22.\,3.\,1832 Weimar@\textsc{Goethe, Johann Wolfgang von} (28.\,8.\,1749 Frankfurt am Main – 22.\,3.\,1832 Weimar), \emph{Schriftsteller}!Unterhaltungen deutscher Ausgewanderten@\strich\emph{Unterhaltungen deutscher Ausgewanderten}|pwkv} an Giovanni
                        Boccaccios\pwindex{Boccaccio, Giovanni 16.\,6.\,1313 Certaldo – 21.\,12.\,1375 ebd.@\textsc{Boccaccio, Giovanni} (16.\,6.\,1313 Certaldo – 21.\,12.\,1375 ebd.), \emph{Schriftsteller}|pwk}{ }\emph{Decamerone}\pwindex{Boccaccio, Giovanni 16.\,6.\,1313 Certaldo – 21.\,12.\,1375 ebd.@\textsc{Boccaccio, Giovanni} (16.\,6.\,1313 Certaldo – 21.\,12.\,1375 ebd.), \emph{Schriftsteller}!Decamerone@\strich\emph{Decamerone}|pwk} angelehnt
                     ist.}}}\label{K_L02941-3} an, daß ihre Schilderungen in ganzen Sätzen übereinſtimmen, aber{ }ſich auch untereinander im Ton des Vortrags außerordentlich ähneln. Daß Goethe\pwindex{Goethe, Johann Wolfgang von 28.\,8.\,1749 Frankfurt am Main – 22.\,3.\,1832 Weimar@\textsc{Goethe, Johann Wolfgang von} (28.\,8.\,1749 Frankfurt am Main – 22.\,3.\,1832 Weimar), \emph{Schriftsteller}|pw}, in deſſen Decamerone\pwindex{Boccaccio, Giovanni 16.\,6.\,1313 Certaldo – 21.\,12.\,1375 ebd.@\textsc{Boccaccio, Giovanni} (16.\,6.\,1313 Certaldo – 21.\,12.\,1375 ebd.), \emph{Schriftsteller}!Decamerone@\strich\emph{Decamerone}|pw}-Nachbildung\pwindex{Goethe, Johann Wolfgang von 28.\,8.\,1749 Frankfurt am Main – 22.\,3.\,1832 Weimar@\textsc{Goethe, Johann Wolfgang von} (28.\,8.\,1749 Frankfurt am Main – 22.\,3.\,1832 Weimar), \emph{Schriftsteller}!Unterhaltungen deutscher Ausgewanderten@\strich\emph{Unterhaltungen deutscher Ausgewanderten}|pwv} das Abenteuer
                  des Marſchalls eine raſch vorübergehende Epiſode, gewiſſermaßen nur ein
                  nebenſächliches Illuſtrationsfaktum iſt, von 
                  Hofmannsthal\pwindex{Hofmannsthal, Hugo von 1.\,2.\,1874 Wien – 15.\,7.\,1929 Rodaun@\textsc{Hofmannsthal, Hugo von} (1.\,2.\,1874 Wien – 15.\,7.\,1929 Rodaun), \emph{Schriftsteller}|pw} nichts gewußt hat, darf man
                  dreiſt vorausſetzen. Merkwürdig iſt nur, daß \so{dieſem}\pwindex{Hofmannsthal, Hugo von 1.\,2.\,1874 Wien – 15.\,7.\,1929 Rodaun@\textsc{Hofmannsthal, Hugo von} (1.\,2.\,1874 Wien – 15.\,7.\,1929 Rodaun), \emph{Schriftsteller}|pwv} die Behandlung des Motivs durch Goethe\pwindex{Goethe, Johann Wolfgang von 28.\,8.\,1749 Frankfurt am Main – 22.\,3.\,1832 Weimar@\textsc{Goethe, Johann Wolfgang von} (28.\,8.\,1749 Frankfurt am Main – 22.\,3.\,1832 Weimar), \emph{Schriftsteller}|pw} unbekannt geblieben iſt, denn wäre dies \so{nicht} der Fall geweſen,{ }ſo hätte er doch{ }ſicher auf die Arbeit\pwindex{Goethe, Johann Wolfgang von 28.\,8.\,1749 Frankfurt am Main – 22.\,3.\,1832 Weimar@\textsc{Goethe, Johann Wolfgang von} (28.\,8.\,1749 Frankfurt am Main – 22.\,3.\,1832 Weimar), \emph{Schriftsteller}!Unterhaltungen deutscher Ausgewanderten@\strich\emph{Unterhaltungen deutscher Ausgewanderten}|pwv}{ }ſeines großen Vorgängers\pwindex{Goethe, Johann Wolfgang von 28.\,8.\,1749 Frankfurt am Main – 22.\,3.\,1832 Weimar@\textsc{Goethe, Johann Wolfgang von} (28.\,8.\,1749 Frankfurt am Main – 22.\,3.\,1832 Weimar), \emph{Schriftsteller}|pwv} verwieſen. Noch merkwürdiger
                  iſt, daß{ }ſich Hofmannsthal\pwindex{Hofmannsthal, Hugo von 1.\,2.\,1874 Wien – 15.\,7.\,1929 Rodaun@\textsc{Hofmannsthal, Hugo von} (1.\,2.\,1874 Wien – 15.\,7.\,1929 Rodaun), \emph{Schriftsteller}|pw} als \so{Verfaſſer} dieſer Geſchichte\pwindex{Hofmannsthal, Hugo von 1.\,2.\,1874 Wien – 15.\,7.\,1929 Rodaun@\textsc{Hofmannsthal, Hugo von} (1.\,2.\,1874 Wien – 15.\,7.\,1929 Rodaun), \emph{Schriftsteller}!Erlebnis des Marschalls von Bassompierre@\strich\emph{Das Erlebnis des Marschalls von Bassompierre}|pwv} bezeichnet, da,{ }ſelbſt wenn die allerliebſten
                  Stimmungsſchilderungen der Erzählung\pwindex{Hofmannsthal, Hugo von 1.\,2.\,1874 Wien – 15.\,7.\,1929 Rodaun@\textsc{Hofmannsthal, Hugo von} (1.\,2.\,1874 Wien – 15.\,7.\,1929 Rodaun), \emph{Schriftsteller}!Erlebnis des Marschalls von Bassompierre@\strich\emph{Das Erlebnis des Marschalls von Bassompierre}|pwv}{ }ſein Eigenthum{ }ſein{ }ſollten, eine Hindeutung auf das Originalwerk\pwindex{Bassompierre, François 12.\,4.\,1579 – 12.\,10.\,1646@\textsc{Bassompierre, François} (12.\,4.\,1579 – 12.\,10.\,1646), \emph{Politiker, Diplomat}!Memoires du mareschal de Bassompierre, contenant l'histoire de sa vie et de ce qui s'est fait de plus remarquable à la cour de France pendant quelques années. 2 Bde.@\strich\emph{Memoires du mareschal de Bassompierre, contenant l'histoire de sa vie et de ce qui s'est fait de plus remarquable à la cour de France pendant quelques années. 2 Bde.}|pwv} unter keinen
                  Umſtänden zu vermeiden war. Die Zeit{[}en{]}, wo man auf das
                  Titelblatt von Komödien und Proſaſchriften einfach zu{ }ſchreiben pflegte: »\so{Nach dem Franzöſiſchen} von X. X.«{ }ſind vorüber, aber{ }ſelbſt damals benützte man die Phraſe »Nach dem Franzöſiſchen«, um, wenn man{ }ſchon
                  den Autor nicht nennen wollte, wenigſtens zuzugeſtehen, daß es{ }ſich um keine
                  Original-Arbeit handle. Da Hugo v.
                     Hofmannsthal\pwindex{Hofmannsthal, Hugo von 1.\,2.\,1874 Wien – 15.\,7.\,1929 Rodaun@\textsc{Hofmannsthal, Hugo von} (1.\,2.\,1874 Wien – 15.\,7.\,1929 Rodaun), \emph{Schriftsteller}|pw} nicht nöthig hat, bei fremden Autoren zu leihen, wäre eine
                     \so{Aufklärung} des Falles gewiß von Intereſſe.}}\pend
           \selectlanguage{ngerman}\endnumbering\briefempfaengerindex{Schnitzler, Arthur@\textsc{Schnitzler, Arthur}!zzzGoldmann, Paul@\emph{von Paul Goldmann}!1900-12-012@{1. 12. 1900}|)be}\mylabel{L02941h}  \newcommand{\dateiname}{L02941}\newcommand{\titel}{Paul Goldmann an Arthur Schnitzler, 1. 12. 1900}\newcommand{\editorInnen}{Martin Anton Müller und Laura Untner}%% latex-leseansicht-abspann.tex
%% Abspann für die Leseansicht.
%% Der Schalter \ifkorrekturansicht ist bereits durch den Vorspann gesetzt.

%% latex-abspann.tex
%% Gemeinsamer Abspann für Korrekturansicht und Leseansicht.
%% Setzt den Schalter \ifkorrekturansicht voraus (gesetzt in den
%% einbindenden Dateien latex-korrekturansicht-abspann.tex bzw.
%% latex-leseansicht-abspann.tex).
%% ---------------------------------------------------------------

\normalsize

% Das esempio-Environment wird nur in der Leseansicht benötigt
\ifkorrekturansicht\else
\newenvironment{esempio}[3]%
{
    \vspace{1.5ex}
    \rlap{\underline{#1}}
    \par
    \setlength{\parindent}{0cm}
    \nopagebreak
    \leftskip=#2cm
    \rightskip=#3cm
}
{
    \par
}
\fi

\doendnotes{C}
\bigskip
\vfill

\clearpage

\footnotesize

\ifkorrekturansicht
  \lohead{\textsc{register}}
\fi

% theindex-Environment neu definieren ohne reledmac
\makeatletter
\renewenvironment{theindex}{%
  \ifkorrekturansicht
    \section*{\indexname}%
  \else
    \subsubsection*{Index der erwähnten Entitäten}%
  \fi
  \setlength{\parindent}{0pt}%
  \setlength{\parskip}{0pt plus 0.3pt}%
  \let\item\@idxitem
}{%
  \ifkorrekturansicht\clearpage\fi
}
\makeatother

\IfFileExists{\jobname-pw.ind}{\input{\jobname-pw.ind}}{}

% Quellenangabe nur in der Leseansicht
\ifkorrekturansicht\else
% Fallback-Definitionen, falls die .tex-Datei \titel etc. nicht gesetzt hat
\providecommand{\titel}{}
\providecommand{\editorInnen}{}
\providecommand{\dateiname}{\jobname}

\vspace{3cm}

\vfill

\footnotesize
\textsc{Quelle}: \titel. Herausgegeben von {\editorInnen}. In: \emph{Arthur Schnitzler: Briefwechsel mit Autorinnen und Autoren}.
 Digitale Edition, https://schnitzler-briefe.acdh.oeaw.ac.at/{\dateiname}.html (Stand \today)
\fi

\end{document}


