%% latex-korrekturansicht-vorspann.tex
%% Vorspann für die Korrekturansicht.
%% Lädt die gemeinsame Datei latex-vorspann.tex mit gesetztem Schalter.

\newif\ifkorrekturansicht
\korrekturansichttrue

\input{../tex-inputs/latex-vorspann}


\section[ Paul Goldmann an Arthur Schnitzler, 1. 12. 1900]{L02941 Paul Goldmann an Arthur Schnitzler, 1. 12. 1900}
\nopagebreak\mylabel{L02941v}
\rehead{ }\normalsize\beginnumbering\briefempfaengerindex{Schnitzler, Arthur@\textsc{Schnitzler, Arthur}!zzzGoldmann, Paul@\emph{von Paul Goldmann}!1900-12-012@{1. 12. 1900}|(be}
\toendnotes[C]{\smallbreak\pagebreak[2]}\Standort{DLA, A:Schnitzler, HS.NZ85.1.3170.}
\physDesc{Brief, 1 Blatt, 2 Seiten, 372 Zeichen
\newline{}Handschrift: blaue Tinte, deutsche Kurrent
\newline{}Beilage: ein Zeitungsausschnitt, beschnitten und aufgeklebt 
\newline{}Schnitzler: 1) mit Bleistift das Jahr »900« vermerkt  2) mit rotem Buntstift eine Unterstreichung}\toendnotes[C]{\smallbreak}
\pstart
           \raggedleft{}{\pb}\textcolor{gray}{\textbf{DESSAUERSTRASSE 19}}\oindex{Dessauer Strasse@\textbf{Dessauer Straße}, \emph{Straße (K.STR)}|pw}\pend
           
\pstart
           Berlin\oindex{Berlin@\textbf{Berlin}, \emph{P.PPLC}|pw}, 1. Dezember.\pend
           
\pstart\center{}Mein lieber Freund,\pend\vspace{0.5em}
\pstart
           Es iſt leider doch nicht gegangen. Ich muß hier\oindex{Berlin@\textbf{Berlin}, \emph{P.PPLC}|pwv} bleiben und kann Dich \label{K_L02941-1v}\edtext{heut{ }Abend}{\lemma{\textnormal{\emph{heut Abend}}}\Cendnote{\textnormal{zur Uraufführung von \emph{Der Schleier der Beatrice}\pwindex{Schleier der Beatrice. Schauspiel in fuenf Akten@\emph{Der Schleier der Beatrice. Schauspiel in fünf Akten}|pwk}}}}\label{K_L02941-1} nur mit allen guten Wünſchen begleiten. Wenn Du dieſen Brief erhältſt, biſt
               Du hoffentlich wieder um einen Erfolg\pwindex{Schleier der Beatrice. Schauspiel in fuenf Akten@\emph{Der Schleier der Beatrice. Schauspiel in fünf Akten}|pwv} reicher.\pend
           
\pstart
           Beifolgenden \label{K_L02941-2v}\edtext{Artikel\pwindex{Geschichte des Marschalls von Bassompierre@\emph{Die Geschichte des Marschalls von Bassompierre}|pwv}}{\lemma{\textnormal{\emph{Artikel}}}\Cendnote{\textnormal{[O. V.]: \emph{Die Geschichte des Marschalls von
                        Bassompierre}\pwindex{Geschichte des Marschalls von Bassompierre@\emph{Die Geschichte des Marschalls von Bassompierre}|pwk}. In: \emph{Frankfurter
                        Zeitung}\pwindex{Frankfurter Zeitung@\emph{Frankfurter Zeitung}|pwk}, Jg. 45, Nr. 331, 30. 11. 1900,
                     Abendblatt, S. 1. Schnitzler
                  teilte das Unverständnis gegenüber Hofmannsthals\pwindex{Hofmannsthal, Hugo von 1874-02-01 – 1929-07-15@\textsc{Hofmannsthal, Hugo von} (1874-02-01 – 1929-07-15), \emph{Schriftsteller/Schriftstellerin}|pwk} fehlender Bekanntmachung der literarischen Aneignung, vgl. A. S.: \emph{Tagebuch}, 12. 12. 1902.}}}\label{K_L02941-2}, der
               Deinen Freund \textsc{Hoffmannsthal\pwindex{Hofmannsthal, Hugo von 1874-02-01 – 1929-07-15@\textsc{Hofmannsthal, Hugo von} (1874-02-01 – 1929-07-15), \emph{Schriftsteller/Schriftstellerin}|pw}} betrifft, finde ich heut in der {\pb}»Frankfurter
                  Zeitung\pwindex{Frankfurter Zeitung@\emph{Frankfurter Zeitung}|pw}«.\pend
           
\pstart
           Viele treue Grüße! {\\[\baselineskip]} Dein {\\[\baselineskip]}\spacefill\mbox{Paul Goldmn.}\pend
           \leftskip=0em{}{\vspace{1\baselineskip}}
\pstart
           {\pb}\textcolor{gray}{\textbf{= \textbf{\label{T_L02941-1v}\edtext{[Die Geſchichte des Marſchalls von Baſſompierre\pwindex{Erlebnis des Marschalls von Bassompierre@\emph{Das Erlebnis des Marschalls von Bassompierre}|pw}.]}{\lemma{\textnormal{\emph{[Die … Baſſompierre.]}}}\Cendnote{\textnormal{eckige Klammern in der Druckvorlage}}}\label{T_L02941-1}}}}\pend
           
\pstart
           \textcolor{gray}{\textbf{Ein Vorkommniß, das in literariſchen Kreiſen von ſich reden
                  macht, verdient um der Perſonen willen, die daran betheiligt ſind, allgemeinere
                  Beachtung. Die dieswöchentliche Wien\oindex{Wien@\textbf{Wien}, \emph{A.ADM2}|pw}er »Zeit\pwindex{Zeit. Wiener Wochenschrift@\emph{Die Zeit. Wiener Wochenschrift}|pw}« enthält den Anfang einer Erzählung\pwindex{Erlebnis des Marschalls von Bassompierre@\emph{Das Erlebnis des Marschalls von Bassompierre}|pwv}, die betitelt iſt: »\so{Erlebniß des Marſchalls von Baſſompierre}\pwindex{Erlebnis des Marschalls von Bassompierre@\emph{Das Erlebnis des Marschalls von Bassompierre}|pw}« und als \so{Verfaſſer} nennt ſich der hochſtrebende
                     Wien\oindex{Wien@\textbf{Wien}, \emph{A.ADM2}|pw}er Poet Hugo v. \so{Hofmannsthal}\pwindex{Hofmannsthal, Hugo von 1874-02-01 – 1929-07-15@\textsc{Hofmannsthal, Hugo von} (1874-02-01 – 1929-07-15), \emph{Schriftsteller/Schriftstellerin}|pw}. Dieſe Erzählung\pwindex{Erlebnis des Marschalls von Bassompierre@\emph{Das Erlebnis des Marschalls von Bassompierre}|pwv}
                  behandelt nicht nur den nämlichen Vorfall, den in \so{Goethe}\pwindex{Goethe, Johann Wolfgang von 1749-08-28 – 1832-03-22@\textsc{Goethe, Johann Wolfgang von} (1749-08-28 – 1832-03-22), \emph{Schriftsteller/Schriftstellerin}|pw}’s »\so{Unterhaltungen deutſcher Ausgewanderten}\pwindex{Unterhaltungen deutscher Ausgewanderten@\emph{Unterhaltungen deutscher Ausgewanderten}|pw}« Vetter Karl\pwindex{Unterhaltungen deutscher Ausgewanderten@\emph{Unterhaltungen deutscher Ausgewanderten}|pwv} auf dem
                     »Gut am rechten Ufer des
                     Rheins\pwindex{Unterhaltungen deutscher Ausgewanderten@\emph{Unterhaltungen deutscher Ausgewanderten}|pwv}« zum Beſten gibt, ſondern, obgleich ſie weit ausführlicher und
                  zufolge ihres näheren Eingehens ins Einzelne blühender iſt, als bei Goethe\pwindex{Unterhaltungen deutscher Ausgewanderten@\emph{Unterhaltungen deutscher Ausgewanderten}|pwv}\pwindex{Goethe, Johann Wolfgang von 1749-08-28 – 1832-03-22@\textsc{Goethe, Johann Wolfgang von} (1749-08-28 – 1832-03-22), \emph{Schriftsteller/Schriftstellerin}|pw}, der die Hauptvorgänge ſtraff zuſammenzufaſſen ſich begnügt, kann es keinem
                  Zweifel unterliegen, daß Beide\pwindex{Goethe, Johann Wolfgang von 1749-08-28 – 1832-03-22@\textsc{Goethe, Johann Wolfgang von} (1749-08-28 – 1832-03-22), \emph{Schriftsteller/Schriftstellerin}|pwv}\pwindex{Hofmannsthal, Hugo von 1874-02-01 – 1929-07-15@\textsc{Hofmannsthal, Hugo von} (1874-02-01 – 1929-07-15), \emph{Schriftsteller/Schriftstellerin}|pwv}, der Alte\pwindex{Goethe, Johann Wolfgang von 1749-08-28 – 1832-03-22@\textsc{Goethe, Johann Wolfgang von} (1749-08-28 – 1832-03-22), \emph{Schriftsteller/Schriftstellerin}|pwv} wie der Junge\pwindex{Hofmannsthal, Hugo von 1874-02-01 – 1929-07-15@\textsc{Hofmannsthal, Hugo von} (1874-02-01 – 1929-07-15), \emph{Schriftsteller/Schriftstellerin}|pwv}, aus der gleichen Quellen geſchöpft haben. Und Beide\pwindex{Goethe, Johann Wolfgang von 1749-08-28 – 1832-03-22@\textsc{Goethe, Johann Wolfgang von} (1749-08-28 – 1832-03-22), \emph{Schriftsteller/Schriftstellerin}|pwv}\pwindex{Hofmannsthal, Hugo von 1874-02-01 – 1929-07-15@\textsc{Hofmannsthal, Hugo von} (1874-02-01 – 1929-07-15), \emph{Schriftsteller/Schriftstellerin}|pwv} lehnen
                  ſich ſo deutlich an das \label{K_L02941-3v}\edtext{fran\oindex{Frankreich@\textbf{Frankreich}, \emph{A.PCLI}|pwv}zöſiſche Original\pwindex{Memoires du mareschal de Bassompierre, contenant l'histoire de sa vie et de ce qui s'est fait de plus remarquable à la cour de France pendant quelques annees. 2 Bde.@\emph{Memoires du mareschal de Bassompierre, contenant l'histoire de sa vie et de ce qui s'est fait de plus remarquable à la cour de France pendant quelques années. 2 Bde.}|pwv}}{\lemma{\textnormal{\emph{franzöſiſche Original}}}\Cendnote{\textnormal{Gemeint sind François Bassompierres\pwindex{Bassompierre, François 1579-04-12 – 1646-10-12@\textsc{Bassompierre, François} (1579-04-12 – 1646-10-12), \emph{Politiker/Politikerin, Diplomat/Diplomatin}|pwk}{ }\emph{Memoires du mareschal de Bassompierre}\pwindex{Memoires du mareschal de Bassompierre, contenant l'histoire de sa vie et de ce qui s'est fait de plus remarquable à la cour de France pendant quelques annees. 2 Bde.@\emph{Memoires du mareschal de Bassompierre, contenant l'histoire de sa vie et de ce qui s'est fait de plus remarquable à la cour de France pendant quelques années. 2 Bde.}|pwk} (1665, 2 Bde.), wobei Goethes\pwindex{Goethe, Johann Wolfgang von 1749-08-28 – 1832-03-22@\textsc{Goethe, Johann Wolfgang von} (1749-08-28 – 1832-03-22), \emph{Schriftsteller/Schriftstellerin}|pwk}{ }Rahmenhandlung\pwindex{Unterhaltungen deutscher Ausgewanderten@\emph{Unterhaltungen deutscher Ausgewanderten}|pwkv} an Giovanni
                        Boccaccios\pwindex{Boccaccio, Giovanni 1313-06-16 – 1375-12-21@\textsc{Boccaccio, Giovanni} (1313-06-16 – 1375-12-21), \emph{Schriftsteller/Schriftstellerin}|pwk}{ }\emph{Decamerone}\pwindex{Decamerone@\emph{Decamerone}|pwk} angelehnt
                     ist.}}}\label{K_L02941-3} an, daß ihre Schilderungen in ganzen Sätzen übereinſtimmen, aber
                  ſich auch untereinander im Ton des Vortrags außerordentlich ähneln. Daß Goethe\pwindex{Goethe, Johann Wolfgang von 1749-08-28 – 1832-03-22@\textsc{Goethe, Johann Wolfgang von} (1749-08-28 – 1832-03-22), \emph{Schriftsteller/Schriftstellerin}|pw}, in deſſen Decamerone\pwindex{Decamerone@\emph{Decamerone}|pw}-Nachbildung\pwindex{Unterhaltungen deutscher Ausgewanderten@\emph{Unterhaltungen deutscher Ausgewanderten}|pwv} das Abenteuer
                  des Marſchalls eine raſch vorübergehende Epiſode, gewiſſermaßen nur ein
                  nebenſächliches Illuſtrationsfaktum iſt, von 
                  Hofmannsthal\pwindex{Hofmannsthal, Hugo von 1874-02-01 – 1929-07-15@\textsc{Hofmannsthal, Hugo von} (1874-02-01 – 1929-07-15), \emph{Schriftsteller/Schriftstellerin}|pw} nichts gewußt hat, darf man
                  dreiſt vorausſetzen. Merkwürdig iſt nur, daß \so{dieſem}\pwindex{Hofmannsthal, Hugo von 1874-02-01 – 1929-07-15@\textsc{Hofmannsthal, Hugo von} (1874-02-01 – 1929-07-15), \emph{Schriftsteller/Schriftstellerin}|pwv} die Behandlung des Motivs durch Goethe\pwindex{Goethe, Johann Wolfgang von 1749-08-28 – 1832-03-22@\textsc{Goethe, Johann Wolfgang von} (1749-08-28 – 1832-03-22), \emph{Schriftsteller/Schriftstellerin}|pw} unbekannt geblieben iſt, denn wäre dies \so{nicht} der Fall geweſen, ſo hätte er doch ſicher auf die Arbeit\pwindex{Unterhaltungen deutscher Ausgewanderten@\emph{Unterhaltungen deutscher Ausgewanderten}|pwv} ſeines großen Vorgängers\pwindex{Goethe, Johann Wolfgang von 1749-08-28 – 1832-03-22@\textsc{Goethe, Johann Wolfgang von} (1749-08-28 – 1832-03-22), \emph{Schriftsteller/Schriftstellerin}|pwv} verwieſen. Noch merkwürdiger
                  iſt, daß ſich Hofmannsthal\pwindex{Hofmannsthal, Hugo von 1874-02-01 – 1929-07-15@\textsc{Hofmannsthal, Hugo von} (1874-02-01 – 1929-07-15), \emph{Schriftsteller/Schriftstellerin}|pw} als \so{Verfaſſer} dieſer Geſchichte\pwindex{Erlebnis des Marschalls von Bassompierre@\emph{Das Erlebnis des Marschalls von Bassompierre}|pwv} bezeichnet, da, ſelbſt wenn die allerliebſten
                  Stimmungsſchilderungen der Erzählung\pwindex{Erlebnis des Marschalls von Bassompierre@\emph{Das Erlebnis des Marschalls von Bassompierre}|pwv} ſein Eigenthum ſein ſollten, eine Hindeutung auf das Originalwerk\pwindex{Memoires du mareschal de Bassompierre, contenant l'histoire de sa vie et de ce qui s'est fait de plus remarquable à la cour de France pendant quelques annees. 2 Bde.@\emph{Memoires du mareschal de Bassompierre, contenant l'histoire de sa vie et de ce qui s'est fait de plus remarquable à la cour de France pendant quelques années. 2 Bde.}|pwv} unter keinen
                  Umſtänden zu vermeiden war. Die Zeit{[}en{]}, wo man auf das
                  Titelblatt von Komödien und Proſaſchriften einfach zu ſchreiben pflegte: »\so{Nach dem Franzöſiſchen} von X. X.« ſind vorüber, aber
                  ſelbſt damals benützte man die Phraſe »Nach dem Franzöſiſchen«, um, wenn man ſchon
                  den Autor nicht nennen wollte, wenigſtens zuzugeſtehen, daß es ſich um keine
                  Original-Arbeit handle. Da Hugo v.
                     Hofmannsthal\pwindex{Hofmannsthal, Hugo von 1874-02-01 – 1929-07-15@\textsc{Hofmannsthal, Hugo von} (1874-02-01 – 1929-07-15), \emph{Schriftsteller/Schriftstellerin}|pw} nicht nöthig hat, bei fremden Autoren zu leihen, wäre eine
                     \so{Aufklärung} des Falles gewiß von Intereſſe.}}\pend
           \selectlanguage{ngerman}\endnumbering\briefempfaengerindex{Schnitzler, Arthur@\textsc{Schnitzler, Arthur}!zzzGoldmann, Paul@\emph{von Paul Goldmann}!1900-12-012@{1. 12. 1900}|)be}\mylabel{L02941h}  \normalsize

\doendnotes{C}
\bigskip
\vfill

\clearpage

\footnotesize

\lohead{\textsc{register}}

% Definiere theindex-Environment komplett neu ohne reledmac
\makeatletter
\renewenvironment{theindex}{%
  \section*{\indexname}%
  \setlength{\parindent}{0pt}%
  \setlength{\parskip}{0pt plus 0.3pt}%
  \let\item\@idxitem
}{%
  \clearpage
}
\makeatother

\IfFileExists{\jobname-pw.ind}{\input{\jobname-pw.ind}}{}

\end{document}

      