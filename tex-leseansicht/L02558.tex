%% latex-leseansicht-vorspann.tex
%% Vorspann für die Leseansicht.
%% Lädt die gemeinsame Datei latex-vorspann.tex mit nicht gesetztem Schalter.

\newif\ifkorrekturansicht
\korrekturansichtfalse

\input{../tex-inputs/latex-vorspann}


\section[Olga und Arthur Schnitzler an Richard und Paula Beer-Hofmann, 25. 5. 1910]{L02558 Olga und Arthur Schnitzler an Richard und Paula Beer-Hofmann, 25. 5. 1910}
\nopagebreak\mylabel{L02558v}
\rehead{ }\normalsize\beginnumbering\briefempfaengerindex{Beer-Hofmann, Paula@\textsc{Beer-Hofmann, Paula}!zzzSchnitzler, Arthur@\emph{von Arthur Schnitzler}!1910-05-271@{27. 5. 1910}|(be}\briefempfaengerindex{Beer-Hofmann, Paula@\textsc{Beer-Hofmann, Paula}!zzzSchnitzler, Olga@\emph{von Olga Schnitzler}!1910-05-271@{27. 5. 1910}|(be}\briefempfaengerindex{Beer-Hofmann, Richard@\textsc{Beer-Hofmann, Richard}!zzzSchnitzler, Arthur@\emph{von Arthur Schnitzler}!1910-05-271@{27. 5. 1910}|(be}\briefempfaengerindex{Beer-Hofmann, Richard@\textsc{Beer-Hofmann, Richard}!zzzSchnitzler, Olga@\emph{von Olga Schnitzler}!1910-05-271@{27. 5. 1910}|(be}
\toendnotes[C]{\smallbreak\pagebreak[2]}
\correspDesc{Versand  durch Olga Schnitzler, Arthur Schnitzler am 27. 5. 1910 in Wien
\newline{}Erhalt  durch Richard Beer-Hofmann, Paula Beer-Hofmann im Zeitraum [27. 5. 1910
                  – 31. 5. 1910?] in Wien}\toendnotes[C]{\smallbreak}
\Standort{YCGL, MSS 31.}
\physDesc{Bildpostkarte, 214 Zeichen
\newline{}Handschrift Olga Schnitzler: schwarze Tinte, lateinische Kurrent
\newline{}Handschrift Arthur Schnitzler: schwarze Tinte, deutsche Kurrent
\newline{}Versand: Stempel: »\nobreak{}\oindex{Territet@\textbf{Territet}|pwk}Territet, 27. V. 10, 10\nobreak{}«.  
\newline{}Ordnung: mit Bleistift von unbekannter Hand datiert: »27. 5.« }\toendnotes[C]{\smallbreak}\pstart{}{\pb}Herrn u. Frau\pwindex{Beer-Hofmann, Paula 25.\,2.\,1879 Wien – 30.\,10.\,1939 Zürich@\textsc{Beer-Hofmann, Paula} (25.\,2.\,1879 Wien – 30.\,10.\,1939 Zürich)|pwv}\pend{}\pstart{}D\textsuperscript{r} Richard Beer-Hofmann\pend{}\pstart{}Wien XVIII\oindex{XVIII., Währing@\textbf{XVIII., Währing}, \emph{Verwaltungsgebiet}|pw}\pend{}\pstart{}Hasenauerstrasse 59\oindex{Wien@\textbf{Wien}!XVIII., Währing@\textbf{XVIII., Währing}!Hasenauerstraße 59@\textbf{Hasenauerstraße 59}, \emph{Wohngebäude}|pw}.\pend{}{\bigskip}
\pstart
           \noindent{}\centering{}{\pb}\textcolor{gray}{\textbf{Le Grand Hôtel et Hôtel des Alpes\oindex{Hôtel des Alpes-Grand Hôtel@\textbf{Hôtel des Alpes-Grand Hôtel}, \emph{Hotel}|pw} – Territet-Montreux\oindex{Territet@\textbf{Territet}|pw}}}\pend
           \vspace{1em}
\pstart
           \noindent{}{\pb}Heut waren wir schon in Villars sur Ollon\oindex{Villars-sur-Ollon@\textbf{Villars-sur-Ollon}|pw} – hoch oben, über 1300 Meter – so ein
               schönes und bequemes Land ist das!\pend
           
\pstart
           Herzliche Grüsse!{\\[\baselineskip]}\spacefill\mbox{Olga.}\pend
           \leftskip=0em{}
\pstart
           {[}hs. Schnitzler:{]} Herzlichſt\spacefill\mbox{Arthur}\pend
           \selectlanguage{ngerman}\endnumbering\briefempfaengerindex{Beer-Hofmann, Paula@\textsc{Beer-Hofmann, Paula}!zzzSchnitzler, Arthur@\emph{von Arthur Schnitzler}!1910-05-271@{27. 5. 1910}|)be}\briefempfaengerindex{Beer-Hofmann, Paula@\textsc{Beer-Hofmann, Paula}!zzzSchnitzler, Olga@\emph{von Olga Schnitzler}!1910-05-271@{27. 5. 1910}|)be}\briefempfaengerindex{Beer-Hofmann, Richard@\textsc{Beer-Hofmann, Richard}!zzzSchnitzler, Arthur@\emph{von Arthur Schnitzler}!1910-05-271@{27. 5. 1910}|)be}\briefempfaengerindex{Beer-Hofmann, Richard@\textsc{Beer-Hofmann, Richard}!zzzSchnitzler, Olga@\emph{von Olga Schnitzler}!1910-05-271@{27. 5. 1910}|)be}\mylabel{L02558h}  \newcommand{\dateiname}{L02558}\newcommand{\titel}{Olga und Arthur Schnitzler an Richard und Paula Beer-Hofmann, 25. 5. 1910}\newcommand{\editorInnen}{Martin Anton Müller und Gerd-Hermann Susen}%% latex-leseansicht-abspann.tex
%% Abspann für die Leseansicht.
%% Der Schalter \ifkorrekturansicht ist bereits durch den Vorspann gesetzt.

%% latex-abspann.tex
%% Gemeinsamer Abspann für Korrekturansicht und Leseansicht.
%% Setzt den Schalter \ifkorrekturansicht voraus (gesetzt in den
%% einbindenden Dateien latex-korrekturansicht-abspann.tex bzw.
%% latex-leseansicht-abspann.tex).
%% ---------------------------------------------------------------

\normalsize

% Das esempio-Environment wird nur in der Leseansicht benötigt
\ifkorrekturansicht\else
\newenvironment{esempio}[3]%
{
    \vspace{1.5ex}
    \rlap{\underline{#1}}
    \par
    \setlength{\parindent}{0cm}
    \nopagebreak
    \leftskip=#2cm
    \rightskip=#3cm
}
{
    \par
}
\fi

\doendnotes{C}
\bigskip
\vfill

\clearpage

\footnotesize

\ifkorrekturansicht
  \lohead{\textsc{register}}
\fi

% theindex-Environment neu definieren ohne reledmac
\makeatletter
\renewenvironment{theindex}{%
  \ifkorrekturansicht
    \section*{\indexname}%
  \else
    \subsubsection*{Index der erwähnten Entitäten}%
  \fi
  \setlength{\parindent}{0pt}%
  \setlength{\parskip}{0pt plus 0.3pt}%
  \let\item\@idxitem
}{%
  \ifkorrekturansicht\clearpage\fi
}
\makeatother

\IfFileExists{\jobname-pw.ind}{\input{\jobname-pw.ind}}{}

% Quellenangabe nur in der Leseansicht
\ifkorrekturansicht\else
% Fallback-Definitionen, falls die .tex-Datei \titel etc. nicht gesetzt hat
\providecommand{\titel}{}
\providecommand{\editorInnen}{}
\providecommand{\dateiname}{\jobname}

\vspace{3cm}

\vfill

\footnotesize
\textsc{Quelle}: \titel. Herausgegeben von {\editorInnen}. In: \emph{Arthur Schnitzler: Briefwechsel mit Autorinnen und Autoren}.
 Digitale Edition, https://schnitzler-briefe.acdh.oeaw.ac.at/{\dateiname}.html (Stand \today)
\fi

\end{document}


