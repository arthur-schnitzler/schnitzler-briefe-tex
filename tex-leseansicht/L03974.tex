%% latex-leseansicht-vorspann.tex
%% Vorspann für die Leseansicht.
%% Lädt die gemeinsame Datei latex-vorspann.tex mit nicht gesetztem Schalter.

\newif\ifkorrekturansicht
\korrekturansichtfalse

\input{../tex-inputs/latex-vorspann}


\section[Arthur Schnitzler an Berta Zuckerkandl, 22. 12. 1928]{L03974 Arthur Schnitzler an Berta Zuckerkandl, 22. 12. 1928}
\nopagebreak\mylabel{L03974v}
\rehead{ }\normalsize\beginnumbering\briefempfaengerindex{Zuckerkandl, Berta@\textsc{Zuckerkandl, Berta}!zzzSchnitzler, Arthur@\emph{von Arthur Schnitzler}!1928-12-221@{22. 12. 1928}|(be}
\toendnotes[C]{\smallbreak\pagebreak[2]}
\correspDesc{Versand  durch Arthur Schnitzler am 22. 12. 1928 in Wien
\newline{}Erhalt  durch Berta Zuckerkandl im Zeitraum [22. 12. 1928 – 25. 12. 1928?] in Wien}\toendnotes[C]{\smallbreak}
\Standort{DLA, HS.1985.1.2282.}
\physDesc{Brief, Durchschlag, 1 Blatt, 1 Seite, 341 Zeichen
\newline{}Schreibmaschine
\newline{}Handschrift: roter Buntstift, lateinische Kurrent (\noindent{}beschriftet: »\uline{Zuckerkandl}«, fünf Unterstreichungen)}\toendnotes[C]{\smallbreak}
\pstart
           \raggedleft{}{\pb}22. 12. 1928.\pend
           
\pstart{}Liebe, verehrte Freundin.\pend\vspace{0.5em}
\pstart
           Von Alzir Hella\pwindex{Hella, Alzir 30.\,12.\,1881 Vieux Condé – 14.\,7.\,1953 Paris@\textsc{Hella, Alzir} (30.\,12.\,1881 Vieux Condé – 14.\,7.\,1953 Paris), \emph{Übersetzer}|pw} habe ich vorschussweise für
               die Veröffentlichung von »Beate und ihr
                  Sohn\pwindex{Schnitzler, Arthur 15. 5. 1862 Wien – 21. 10. 1931 ebd.@\textsc{Schnitzler, Arthur} (15. 5. 1862 Wien – 21. 10. 1931 ebd.), \emph{Schriftsteller, Mediziner}!Madame Beate et son fils@\strich\emph{Madame Beate et son fils}|pw}\pwindex{Schnitzler, Arthur 15. 5. 1862 Wien – 21. 10. 1931 ebd.@\textsc{Schnitzler, Arthur} (15. 5. 1862 Wien – 21. 10. 1931 ebd.), \emph{Schriftsteller, Mediziner}!Frau Beate und ihr Sohn. Novelle@\strich\emph{Frau Beate und ihr Sohn. Novelle}|pw}« \label{K_L03974-1v}\edtext{in der »Humanité\pwindex{L'Humanité@\emph{L'Humanité}|pw}«}{\lemma{\textnormal{\emph{in der »Humanité«}}}\Cendnote{\textnormal{\emph{Beate et son fils}\pwindex{Schnitzler, Arthur 15. 5. 1862 Wien – 21. 10. 1931 ebd.@\textsc{Schnitzler, Arthur} (15. 5. 1862 Wien – 21. 10. 1931 ebd.), \emph{Schriftsteller, Mediziner}!Madame Beate et son fils@\strich\emph{Madame Beate et son fils}|pwk} erschien zwischen dem
                     23. 10. und dem 26. 11. 1928 in
                  zweiundzwanzig Folgen in der Zeitung \emph{L' Humanité}\pwindex{L'Humanité@\emph{L'Humanité}|pwk}. Alzir Hella\pwindex{Hella, Alzir 30.\,12.\,1881 Vieux Condé – 14.\,7.\,1953 Paris@\textsc{Hella, Alzir} (30.\,12.\,1881 Vieux Condé – 14.\,7.\,1953 Paris), \emph{Übersetzer}|pwk} hatte Schnitzler nicht darüber in Kenntnis gesetzt,
                  weshalb dieser brieflich Honorar und Belegexemplar angemahnt hatte, Arthur Schnitzler an Alzir Hella\pwindex{Hella, Alzir 30.\,12.\,1881 Vieux Condé – 14.\,7.\,1953 Paris@\textsc{Hella, Alzir} (30.\,12.\,1881 Vieux Condé – 14.\,7.\,1953 Paris), \emph{Übersetzer}|pwk}, 14. 11. 1928,
                        \emph{Deutsches Literaturarchiv Marbach},
                     HS.1985.1.969.}}}\label{K_L03974-1} den Betrag von S. 129.– erhalten, wovon ich
               ordnungsgemäss beigeschlossen S. 19.35 (15{\%}) an Sie zu
               überweisen mir erlaube.\pend
           
\pstart
           Mit herzlichen Grüssen und Wünschen{\\[\baselineskip]} Ihr\pend
           \leftskip=0em{}{\vspace{1\baselineskip}}
\pstart
           \noindent{}Frau Hofrätin Bertha Zuckerhandl,{\\}Wien\oindex{Wien@\textbf{Wien}, \emph{Verwaltungsgebiet}|pw}.\pend
           \selectlanguage{ngerman}\endnumbering\briefempfaengerindex{Zuckerkandl, Berta@\textsc{Zuckerkandl, Berta}!zzzSchnitzler, Arthur@\emph{von Arthur Schnitzler}!1928-12-221@{22. 12. 1928}|)be}\mylabel{L03974h}
\begin{anhang}
\end{anhang}\newcommand{\dateiname}{L03974}\newcommand{\titel}{Arthur Schnitzler an Berta Zuckerkandl, 22. 12. 1928}\newcommand{\editorInnen}{Herausgegeben von Jahnke, SelmaMüller, Martin Anton}%% latex-leseansicht-abspann.tex
%% Abspann für die Leseansicht.
%% Der Schalter \ifkorrekturansicht ist bereits durch den Vorspann gesetzt.

%% latex-abspann.tex
%% Gemeinsamer Abspann für Korrekturansicht und Leseansicht.
%% Setzt den Schalter \ifkorrekturansicht voraus (gesetzt in den
%% einbindenden Dateien latex-korrekturansicht-abspann.tex bzw.
%% latex-leseansicht-abspann.tex).
%% ---------------------------------------------------------------

\normalsize

% Das esempio-Environment wird nur in der Leseansicht benötigt
\ifkorrekturansicht\else
\newenvironment{esempio}[3]%
{
    \vspace{1.5ex}
    \rlap{\underline{#1}}
    \par
    \setlength{\parindent}{0cm}
    \nopagebreak
    \leftskip=#2cm
    \rightskip=#3cm
}
{
    \par
}
\fi

\doendnotes{C}
\bigskip
\vfill

\clearpage

\footnotesize

\ifkorrekturansicht
  \lohead{\textsc{register}}
\fi

% theindex-Environment neu definieren ohne reledmac
\makeatletter
\renewenvironment{theindex}{%
  \ifkorrekturansicht
    \section*{\indexname}%
  \else
    \subsubsection*{Index der erwähnten Entitäten}%
  \fi
  \setlength{\parindent}{0pt}%
  \setlength{\parskip}{0pt plus 0.3pt}%
  \let\item\@idxitem
}{%
  \ifkorrekturansicht\clearpage\fi
}
\makeatother

\IfFileExists{\jobname-pw.ind}{\input{\jobname-pw.ind}}{}

% Quellenangabe nur in der Leseansicht
\ifkorrekturansicht\else
% Fallback-Definitionen, falls die .tex-Datei \titel etc. nicht gesetzt hat
\providecommand{\titel}{}
\providecommand{\editorInnen}{}
\providecommand{\dateiname}{\jobname}

\vspace{3cm}

\vfill

\footnotesize
\textsc{Quelle}: \titel. Herausgegeben von {\editorInnen}. In: \emph{Arthur Schnitzler: Briefwechsel mit Autorinnen und Autoren}.
 Digitale Edition, https://schnitzler-briefe.acdh.oeaw.ac.at/{\dateiname}.html (Stand \today)
\fi

\end{document}


