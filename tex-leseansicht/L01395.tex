%% latex-korrekturansicht-vorspann.tex
%% Vorspann für die Korrekturansicht.
%% Lädt die gemeinsame Datei latex-vorspann.tex mit gesetztem Schalter.

\newif\ifkorrekturansicht
\korrekturansichttrue

\input{../tex-inputs/latex-vorspann}


\section[Arthur Schnitzler an Richard Beer-Hofmann, 26. 4. 1904]{L01395 Arthur Schnitzler an Richard Beer-Hofmann, 26. 4. 1904}
\nopagebreak\mylabel{L01395v}
\rehead{ }\normalsize\beginnumbering\briefempfaengerindex{Beer-Hofmann, Richard@\textsc{Beer-Hofmann, Richard}!zzzSchnitzler, Arthur@\emph{von Arthur Schnitzler}!1904-04-262@{26. 4. 1904}|(be}
\toendnotes[C]{\smallbreak\pagebreak[2]}\Standort{YCGL, MSS 31.}
\physDesc{Brief, 1 Blatt, 3 Seiten, Umschlag, 471 Zeichen
\newline{}Handschrift: Bleistift, deutsche Kurrent
\newline{}Versand: 1) Stempel: »\nobreak{}\oindex{XVIII., Waehring@\textbf{XVIII., Währing}, \emph{A.ADM3}|pwk}18 Wien 110, 26. 4. 04, 11–12V\nobreak{}«.   2) Stempel: »\nobreak{}\oindex{Rodaun@\textbf{Rodaun}, \emph{A.ADM4}|pwk}{\pb}Rodaun, {[}2{]}\textcolor{gray}{7}{[}. 4. 04{]}, V\nobreak{}«. }\toendnotes[C]{\smallbreak}\pstart{}{\pb}Herrn \textsc{Dr Richard Beer-Hofmann}\pend{}\pstart{}\textsc{Rodaun}\oindex{Rodaun@\textbf{Rodaun}, \emph{A.ADM4}|pw}\pend{}\pstart{}\textsc{Liesinger Straße 2\oindex{Liesingerstrasse@\textbf{Liesingerstraße}, \emph{Straße (K.STR)}|pw}.}\pend{}{\bigskip}\vspace{1em}
\pstart
           \raggedleft{}{\pb}Dinſtg, 26. 4. 904\pend
           \vspace{0.5em}
\pstart
           lieber Richard, aus einer \label{K_L01395-1v}\edtext{Karte Hugos\pwindex{Hofmannsthal, Hugo von 1874-02-01 – 1929-07-15@\textsc{Hofmannsthal, Hugo von} (1874-02-01 – 1929-07-15), \emph{Schriftsteller/Schriftstellerin}|pw}}{\lemma{\textnormal{\emph{Karte Hugos}}}\Cendnote{\textnormal{Hugo von Hofmannsthal an Arthur Schnitzler, 25. 4. 1904.
               }}}\label{K_L01395-1} vom Semmering\oindex{Semmering@\textbf{Semmering}, \emph{A.ADM3}|pw} entnehme ich daſs er
               die meine nicht erhalten hat. Dieſe meine Karte ſchlug ein Rendezvous für \label{K_L01395-2v}\edtext{Mittwoch Abend}{\lemma{\textnormal{\emph{Mittwoch Abend}}}\Cendnote{\textnormal{Siehe A. S.: \emph{Tagebuch}, 27. 4. 1904.
               }}}\label{K_L01395-2}{ }Hietzing \textsc{Kuffner}\oindex{Ottakringer Braeu@\textbf{Ottakringer Bräu}, \emph{Bierhaus (K.BIR)}|pw} vor und bat ihn, das {\pb}auch Ihnen mitzutheilen.
               Es wär mir, \textsc{resp} uns beiden Olga\pwindex{Schnitzler, Olga 17.01.1882 – 13.01.1970@\textsc{Schnitzler, Olga} (17.01.1882 – 13.01.1970), \emph{Schauspieler/Schauspielerin, Sänger/Sängerin}|pw} u mir ſehr lieb, Sie Beide noch vor unſerer Abreiſe zu
               ſehen. \substVorne{}\textsuperscript{\textcolor{gray}{Jed}}\substDazwischen{}W\substHinten{}ir werden alſo jedenfalls in Hietzing\oindex{XIII., Hietzing@\textbf{XIII., Hietzing}, \emph{A.ADM3}|pw}
               ſein. (Auch Bahr\pwindex{Bahr, Hermann 19.07.1863 – 15.01.1934@\textsc{Bahr, Hermann} (19.07.1863 – 15.01.1934), \emph{Schriftsteller/Schriftstellerin, Kritiker/Kritikerin}|pw} hatt’ ich geſchrieben.)\pend
           
\pstart
           {\pb}Herzlich{\\[\baselineskip]}Ihr{\\[\baselineskip]}\spacefill\mbox{Arthur}\pend
           \leftskip=0em{}\selectlanguage{ngerman}\endnumbering\briefempfaengerindex{Beer-Hofmann, Richard@\textsc{Beer-Hofmann, Richard}!zzzSchnitzler, Arthur@\emph{von Arthur Schnitzler}!1904-04-262@{26. 4. 1904}|)be}\mylabel{L01395h}  \normalsize

\doendnotes{C}
\bigskip
\vfill

\clearpage

\footnotesize

\lohead{\textsc{register}}

% Definiere theindex-Environment komplett neu ohne reledmac
\makeatletter
\renewenvironment{theindex}{%
  \section*{\indexname}%
  \setlength{\parindent}{0pt}%
  \setlength{\parskip}{0pt plus 0.3pt}%
  \let\item\@idxitem
}{%
  \clearpage
}
\makeatother

\IfFileExists{\jobname-pw.ind}{\input{\jobname-pw.ind}}{}

\end{document}

      