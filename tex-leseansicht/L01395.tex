%% latex-leseansicht-vorspann.tex
%% Vorspann für die Leseansicht.
%% Lädt die gemeinsame Datei latex-vorspann.tex mit nicht gesetztem Schalter.

\newif\ifkorrekturansicht
\korrekturansichtfalse

\input{../tex-inputs/latex-vorspann}


         
         \renewcommand{\erwaehntePersonen}{Personen: Hermann Bahr, Richard Beer-Hofmann, Hugo von Hofmannsthal, Olga Schnitzler}
         \renewcommand{\erwaehnteOrte}{Orte: Liesingerstraße, Ottakringer Bräu, Rodaun, Semmering, Wien, XIII., Hietzing, XVIII., Währing}
         \renewcommand{\erwaehnteWerke}{}
               \section[Arthur Schnitzler an Richard Beer-Hofmann, 26. 4. 1904]{ Arthur Schnitzler an Richard Beer-Hofmann, 26. 4. 1904}\nopagebreak\mylabel{v}\rehead{ }\begin{ledgroupsized}[t]{13cm}\normalsize\beginnumbering \toendnotes[C]{\smallbreak\pagebreak[2]} \Standort{YCGL, MSS 31.}
\physDesc{Brief, 1 Blatt, 3 Seiten, Umschlag, 473 Zeichen
\newline{}Handschrift: Bleistift, deutsche Kurrent
\newline{}Versand: 1) Stempel: »\nobreak{}\oindex{XVIII., Waehring@\textbf{XVIII., Währing}|pwk}18 Wien 110, 26. 4. 04, 11–12V\nobreak{}«.   2) Stempel: »\nobreak{}\oindex{Rodaun@\textbf{Rodaun}|pwk}{\pb}Rodaun, {[}2{]}\textcolor{gray}{7}{[}. 4. 04{]}, V\nobreak{}«. }\toendnotes[C]{\smallbreak}\pstart{}{\pb}Herrn \textsc{Dr Richard Beer-Hofmann}\pend{}\pstart{}\textsc{Rodaun}\oindex{Rodaun@\textbf{Rodaun}|pw}\pend{}\pstart{}\textsc{Liesinger Straße 2\oindex{Liesingerstrasse@\textbf{Liesingerstraße}|pw}.}\pend{}{\bigskip}\pstart
           \raggedleft{}{\pb}Dinſtg, 26. 4. 904\pend
           \pstart
           lieber Richard, aus einer \label{K_L01395-11v}\edtext{Karte Hugo\pwindex{Hofmannsthal, Hugo von 1874-02-01 – 1929-07-15@\textsc{Hofmannsthal, Hugo von} (1874-02-01 – 1929-07-15), \emph{Schriftsteller}|pw}s}{\lemma{\textnormal{\emph{Karte Hugos}}}\Cendnote{\textnormal{Hugo von Hofmannsthal an Arthur Schnitzler, 25. 4. 1904}}}\label{K_L01395-11h} vom Semmering\oindex{Semmering@\textbf{Semmering}|pw} entnehme ich daſs er
               die meine nicht erhalten hat. Dieſe meine Karte ſchlug ein Rendezvous für \label{K_L01395-1v}\edtext{Mittwoch Abend}{\lemma{\textnormal{\emph{Mittwoch Abend}}}\Cendnote{\textnormal{siehe A. S.: \emph{Tagebuch}, 27. 4. 1904}}}\label{K_L01395-1h}{ }Hietzing \textsc{Kuffner}\oindex{Ottakringer Braeu@\textbf{Ottakringer Bräu}|pw} vor und bat ihn, das {\pb}auch Ihnen mitzutheilen.
               Es wär mir, \textsc{resp} uns beiden Olga\pwindex{Schnitzler, Olga 17.01.1882 – 13.01.1970@\textsc{Schnitzler, Olga} (17.01.1882 – 13.01.1970), \emph{Schauspielerin, Sängerin}|pw} u mir ſehr lieb, Sie Beide noch vor unſerer Abreiſe zu
               ſehen. \substVorne{}\textsuperscript{\textcolor{gray}{Jed}}\substDazwischen{}W\substHinten{}ir werden alſo jedenfalls in Hietzing\oindex{XIII., Hietzing@\textbf{XIII., Hietzing}|pw}
               ſein. (Auch Bahr\pwindex{Bahr, Hermann 19.07.1863 – 15.01.1934@\textsc{Bahr, Hermann} (19.07.1863 – 15.01.1934), \emph{Schriftsteller, Kritiker}|pw} hatt’ ich geſchrieben.)\pend
           \pstart
           {\pb}Herzlich{\\[\baselineskip]}Ihr{\\[\baselineskip]}\spacefill\mbox{Arthur}\pend
           \leftskip=0em{}
         
         \endnumbering\mylabel{h}\end{ledgroupsized}  \newcommand{\dateiname}{L01395}\newcommand{\titel}{Arthur Schnitzler an Richard Beer-Hofmann, 26. 4. 1904}\newcommand{\editorInnen}{Martin Anton Müller und Gerd-Hermann Susen}%% latex-leseansicht-abspann.tex
%% Abspann für die Leseansicht.
%% Der Schalter \ifkorrekturansicht ist bereits durch den Vorspann gesetzt.

%% latex-abspann.tex
%% Gemeinsamer Abspann für Korrekturansicht und Leseansicht.
%% Setzt den Schalter \ifkorrekturansicht voraus (gesetzt in den
%% einbindenden Dateien latex-korrekturansicht-abspann.tex bzw.
%% latex-leseansicht-abspann.tex).
%% ---------------------------------------------------------------

\normalsize

% Das esempio-Environment wird nur in der Leseansicht benötigt
\ifkorrekturansicht\else
\newenvironment{esempio}[3]%
{
    \vspace{1.5ex}
    \rlap{\underline{#1}}
    \par
    \setlength{\parindent}{0cm}
    \nopagebreak
    \leftskip=#2cm
    \rightskip=#3cm
}
{
    \par
}
\fi

\doendnotes{C}
\bigskip
\vfill

\clearpage

\footnotesize

\ifkorrekturansicht
  \lohead{\textsc{register}}
\fi

% theindex-Environment neu definieren ohne reledmac
\makeatletter
\renewenvironment{theindex}{%
  \ifkorrekturansicht
    \section*{\indexname}%
  \else
    \subsubsection*{Index der erwähnten Entitäten}%
  \fi
  \setlength{\parindent}{0pt}%
  \setlength{\parskip}{0pt plus 0.3pt}%
  \let\item\@idxitem
}{%
  \ifkorrekturansicht\clearpage\fi
}
\makeatother

\IfFileExists{\jobname-pw.ind}{\input{\jobname-pw.ind}}{}

% Quellenangabe nur in der Leseansicht
\ifkorrekturansicht\else
% Fallback-Definitionen, falls die .tex-Datei \titel etc. nicht gesetzt hat
\providecommand{\titel}{}
\providecommand{\editorInnen}{}
\providecommand{\dateiname}{\jobname}

\vspace{3cm}

\vfill

\footnotesize
\textsc{Quelle}: \titel. Herausgegeben von {\editorInnen}. In: \emph{Arthur Schnitzler: Briefwechsel mit Autorinnen und Autoren}.
 Digitale Edition, https://schnitzler-briefe.acdh.oeaw.ac.at/{\dateiname}.html (Stand \today)
\fi

\end{document}


      