%% latex-korrekturansicht-vorspann.tex
%% Vorspann für die Korrekturansicht.
%% Lädt die gemeinsame Datei latex-vorspann.tex mit gesetztem Schalter.

\newif\ifkorrekturansicht
\korrekturansichttrue

\input{../tex-inputs/latex-vorspann}


\section[Georg Brandes an Arthur Schnitzler, 30. 4. 1900]{L01033 Georg Brandes an Arthur Schnitzler, 30. 4. 1900}
\nopagebreak\mylabel{L01033v}
\rehead{ }\normalsize\beginnumbering\briefempfaengerindex{Schnitzler, Arthur@\textsc{Schnitzler, Arthur}!zzzBrandes, Georg@\emph{von Georg Brandes}!1900-04-301@{30. 4. 1900}|(be}
\toendnotes[C]{\smallbreak\pagebreak[2]}\Standort{CUL, Schnitzler, B 17.}
\physDesc{Brief, 1 Blatt, 4 Seiten, 2832 Zeichen
\newline{}Handschrift: schwarze Tinte, lateinische Kurrent
\newline{}Ordnung: mit Bleistift von unbekannter Hand nummeriert:
                                    »20« }
\buchAbdrucke{\weitereDrucke{Georg Brandes, Arthur Schnitzler: \emph{Ein Briefwechsel}. Bern: \emph{Francke} 1956, S. 80–81.} }\toendnotes[C]{\smallbreak}
\pstart
           \raggedleft{}{\pb}Kommunehospitalet\oindex{Kommunehospitalet@\textbf{Kommunehospitalet}, \emph{Krankenhaus (K.KKH)}|pw}\pend
           
\pstart
           \raggedleft{}Kopenhagen\pend
           
\pstart
           \raggedleft{}30 April 1900\pend
           
\pstart{}Verehrter Freund\pend\vspace{0.5em}
\pstart
           Sie wundern sich vielleicht, gar nicht von mir gehört zu haben, da wir doch
               verabredet hatten, uns zu treffen und uns jedenfalls in Wien\oindex{Wien@\textbf{Wien}, \emph{A.ADM2}|pw} zu sehen. Aber eben wie ich eine Reise auf Kosten des ungarischen\oindex{Ungarn@\textbf{Ungarn}, \emph{A.PCLI}|pw} Staats durch die ungarischen\oindex{Ungarn@\textbf{Ungarn}, \emph{A.PCLI}|pw} Provinzen antreten sollte, kam meine alte Krankheit,
               die Venenentzündung, wieder, ich lag erst 3–4 Tage im Hotel reiste dann nach Kopenhagen\oindex{Kopenhagen@\textbf{Kopenhagen}, \emph{P.PPLC}|pw} und habe also den ganzen Monat
               verloren. Ich habe mich ins Hospital eingelegt um sorgfältige Pflege zu haben, die
               Entzündung schien schon zwei Mal erloschen, kam aber dann wieder. Ich liege also
               vorläufig in dieser gelinden Tortur, das Bein hoch und in der Schiene {\pb}auf dem Rücken immer in derselben
               Lage ohne mich weder rechts noch links drehen zu können.\pend
           
\pstart
           Dies ist der dritte Frühling, den ich nicht sehe (97, 99,
                  1900)\pend
           
\pstart
           Die deutschen Blätter haben Dutzende von Schmähartikeln gegen mich enthalten, weil
               ich in dem Klub in Budapest\orgindex{Liberaler Club@Liberaler Club|pwv},
               aufgefordert, eine französische\oindex{Frankreich@\textbf{Frankreich}, \emph{A.PCLI}|pw} Einleitung zu
               machen (was mir lächerlich vorkam), einfach sagte »Die Sprache, deren ich mich
               bediene ist nicht die Ihre und nicht die meine, nicht Ihre Lieblingssprache und nicht
               die meine, doch es ist die, worin wir uns am leichtesten verstehen.« Das wird ein \uline{hämischer} Angriff auf Deutschland\oindex{Deutschland@\textbf{Deutschland}, \emph{A.PCLI}|pw} und die deutsche Kultur genannt. Und zwar von anonymen Bengeln,
               die nicht mehr Antheil an die deutsche Kultur haben als ein alter Stiefel. \strikeout{V}Die Verachtung, die ich für die Journalisten {\pb}hege, ist nach und nach so gross,
               dass ich förmlich einen bitteren Geschmack im Munde davon habe, wenn ich daran
               denke.\pend
           
\pstart
           Ich bin Ihnen und Beer-Hoffmann\pwindex{Beer-Hofmann, Richard 1866-07-11 – 1945-09-26@\textsc{Beer-Hofmann, Richard} (1866-07-11 – 1945-09-26), \emph{Schriftsteller/Schriftstellerin}|pw} wie gewöhnlich
               vielen Dank für Wien\oindex{Wien@\textbf{Wien}, \emph{A.ADM2}|pw} schuldig.\pend
           
\pstart
           Sie beiden und Gomperz\pwindex{Gomperz, Theodor 29.03.1832 – 29.08.1912@\textsc{Gomperz, Theodor} (29.03.1832 – 29.08.1912), \emph{Altphilologe/Altphilologin}|pw}’s Haus und Lanckoronski\pwindex{Lanckoroński, Karl 04.11.1848 – 15.07.1934@\textsc{Lanckoroński, Karl} (04.11.1848 – 15.07.1934), \emph{Schriftsteller/Schriftstellerin, Sammler/Sammlerin, Forscher/Forscherin}|pw} waren dies mal mein Wien\oindex{Wien@\textbf{Wien}, \emph{A.ADM2}|pw}. Ich habe Sie sehr lieb und freue mich, dass
               wir Freunde sind.\pend
           
\pstart
           Ich las jetzt im Bett einige Bücher: \uline{Drames de famille}\pwindex{Familiendramen@\emph{Familiendramen}|pw}, die Bourget\pwindex{Bourget, Paul 02.09.1852 – 25.12.1935@\textsc{Bourget, Paul} (02.09.1852 – 25.12.1935), \emph{Schriftsteller/Schriftstellerin}|pw} mir schickte trotzdem er so
               katholisch geworden ist; die zwei grossen Erzählungen, die in unsern nordischen\oindex{Skandinavien@\textbf{Skandinavien}, \emph{Region}|pwv} Blättern übel besprochen werden,
               gefielen mir sehr, wenn auch nicht die moralisirende Schreibweise, doch Stoff und
               Ausführung. Dann las ich einen deutschen Roman, der mir geschickt wurde und der mir
               gut scheint, Wilhelm Hegeler\pwindex{Hegeler, Wilhelm 25.02.1870 – 08.10.1943@\textsc{Hegeler, Wilhelm} (25.02.1870 – 08.10.1943), \emph{Schriftsteller/Schriftstellerin, Schriftsteller/Schriftstellerin, Krimiautor/Krimiautorin}|pw}, \uline{Ingenieur Horstmann}\pwindex{Ingenieur Horstmann@\emph{Ingenieur Horstmann}|pw}, eine {\pb}sehr tüchtige
               Leistung. Mit Interesse las ich Balzacs\pwindex{Balzac, Honore de 20.05.1799 – 18.08.1850@\textsc{Balzac, Honoré de} (20.05.1799 – 18.08.1850), \emph{Schriftsteller/Schriftstellerin}|pw}{ }Briefe À L’Etrangère\pwindex{Lettres à l etrangere (1833–1842), (1842–1844)@\emph{Lettres à l’étrangère (1833–1842), (1842–1844)}|pw} d. h. an seine zukünftige
                  Frau\pwindex{Hańska, Ewelina 06.01.1801 – 10.04.1882@\textsc{Hańska, Ewelina} (06.01.1801 – 10.04.1882)|pwv} in der ersten \label{K_L01033-1v}\edtext{\uline{vollständigen} Ausgabe}{\lemma{\textnormal{\emph{vollständigen Ausgabe}}}\Cendnote{\textnormal{H. de Balzac\pwindex{Balzac, Honore de 20.05.1799 – 18.08.1850@\textsc{Balzac, Honoré de} (20.05.1799 – 18.08.1850), \emph{Schriftsteller/Schriftstellerin}|pwk}: \emph{Œuvres
                        posthumes} I. \emph{Lettres à l’étrangère
                        (1833–1842)}\pwindex{Lettres à l etrangere (1833–1842), (1842–1844)@\emph{Lettres à l’étrangère (1833–1842), (1842–1844)}|pwk}. Paris: \emph{Calmann-Lévy,
                        Éditeurs}\orgindex{Calmann-Levy@Calmann-Lévy|pwk}{ }{[}1899{]}.}}}\label{K_L01033-1}.\pend
           
\pstart
           Es war amüsant, den Ton in Lanckoronskis\pwindex{Lanckoroński, Karl 04.11.1848 – 15.07.1934@\textsc{Lanckoroński, Karl} (04.11.1848 – 15.07.1934), \emph{Schriftsteller/Schriftstellerin, Sammler/Sammlerin, Forscher/Forscherin}|pw}{ }\uline{Rund um die Welt}\pwindex{Rund um die Erde 1888–89@\emph{Rund um die Erde 1888–89}|pw} mit dem in unseres Freundes Goldmann\pwindex{Goldmann, Paul 31.01.1865 – 25.09.1935@\textsc{Goldmann, Paul} (31.01.1865 – 25.09.1935), \emph{Schriftsteller/Schriftstellerin, Journalist/Journalistin}|pw}’s\pwindex{Sommer in China. Reisebilder@\emph{Ein Sommer in China. Reisebilder}|pwv} zu vergleichen. Goldmann\pwindex{Goldmann, Paul 31.01.1865 – 25.09.1935@\textsc{Goldmann, Paul} (31.01.1865 – 25.09.1935), \emph{Schriftsteller/Schriftstellerin, Journalist/Journalistin}|pw} hat mehr Geist und Herz, der Graf hat
               viel mehr gesehen und (erstaunlich!) er hat darin ein gutes lyrisches Gedicht
               geschrieben.\pend
           
\pstart
           Es ist trist, so oft und lange krank zu sein. Ich bin ganz ausser Stande, irgend eine
               ordentliche Arbeit vorzunehmen, meine tägliche Arbeit besteht allein darin, die
               Ausgabe meiner Sämmtlichen
                  Schriften\pwindex{Samlede Skrifter [Gesammelte Werke]@\emph{Samlede Skrifter [Gesammelte Werke]}|pwv} zu verbessern und zu corrigiren.\pend
           
\pstart
           Ihr Freund{\\[\baselineskip]}\spacefill\mbox{Georg Brandes}\pend
           \leftskip=0em{}\selectlanguage{ngerman}\endnumbering\briefempfaengerindex{Schnitzler, Arthur@\textsc{Schnitzler, Arthur}!zzzBrandes, Georg@\emph{von Georg Brandes}!1900-04-301@{30. 4. 1900}|)be}\mylabel{L01033h}  \normalsize

\doendnotes{C}
\bigskip
\vfill

\clearpage

\footnotesize

\lohead{\textsc{register}}

% Definiere theindex-Environment komplett neu ohne reledmac
\makeatletter
\renewenvironment{theindex}{%
  \section*{\indexname}%
  \setlength{\parindent}{0pt}%
  \setlength{\parskip}{0pt plus 0.3pt}%
  \let\item\@idxitem
}{%
  \clearpage
}
\makeatother

\IfFileExists{\jobname-pw.ind}{\input{\jobname-pw.ind}}{}

\end{document}

      