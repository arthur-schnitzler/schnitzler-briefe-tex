%% latex-leseansicht-vorspann.tex
%% Vorspann für die Leseansicht.
%% Lädt die gemeinsame Datei latex-vorspann.tex mit nicht gesetztem Schalter.

\newif\ifkorrekturansicht
\korrekturansichtfalse

\input{../tex-inputs/latex-vorspann}


         
         \renewcommand{\erwaehntePersonen}{Personen: Honoré de Balzac, Richard Beer-Hofmann, Paul Bourget, Paul Goldmann, Theodor Gomperz, Ewelina Hańska, Wilhelm Hegeler, Karl Lanckoroński}
         \renewcommand{\erwaehnteInstitutionen}{Institutionen: Calmann-Lévy, Liberaler Club}
         \renewcommand{\erwaehnteOrte}{Orte: Deutschland, Frankreich, Kommunehospitalet, Kopenhagen, Skandinavien, Ungarn, Wien}
         \renewcommand{\erwaehnteWerke}{Werke: Ein Sommer in China. Reisebilder, Familiendramen, Ingenieur Horstmann, Lettres à l’étrangère (1833–1842), (1842–1844), Rund um die Erde 1888–89, Samlede Skrifter [Gesammelte Werke]}
               \section[Georg Brandes an Arthur Schnitzler, 30. 4. 1900]{ Georg Brandes an Arthur Schnitzler, 30. 4. 1900}\nopagebreak\mylabel{v}\rehead{ }\begin{ledgroupsized}[t]{13cm}\normalsize\beginnumbering \toendnotes[C]{\smallbreak\pagebreak[2]} \Standort{CUL, Schnitzler, B 17.}
\physDesc{Brief, 1 Blatt, 4 Seiten, 2832 Zeichen
\newline{}Handschrift: schwarze Tinte, lateinische Kurrent
\newline{}Ordnung: mit Bleistift von unbekannter Hand nummeriert:
                                    »20« }\buchAbdrucke{\weitereDrucke{Georg Brandes, Arthur Schnitzler: \emph{Ein Briefwechsel}. Hg. Kurt Bergel. Bern: \emph{Francke} 1956, S. 80–81.} }\toendnotes[C]{\smallbreak}\pstart
           \raggedleft{}{\pb}Kommunehospitalet\oindex{Kommunehospitalet@\textbf{Kommunehospitalet}|pw}\pend
           \pstart
           \raggedleft{}Kopenhagen\pend
           \pstart
           \raggedleft{}30 April 1900\pend
           \pstart{}Verehrter Freund\pend\pstart
           Sie wundern sich vielleicht, gar nicht von mir gehört zu haben, da wir doch
               verabredet hatten, uns zu treffen und uns jedenfalls in Wien\oindex{Wien@\textbf{Wien}|pw} zu sehen. Aber eben wie ich eine Reise auf Kosten des ungarischen\oindex{Ungarn@\textbf{Ungarn}|pw} Staats durch die ungarischen\oindex{Ungarn@\textbf{Ungarn}|pw} Provinzen antreten sollte, kam meine alte Krankheit,
               die Venenentzündung, wieder, ich lag erst 3–4 Tage im Hotel reiste dann nach Kopenhagen\oindex{Kopenhagen@\textbf{Kopenhagen}|pw} und habe also den ganzen Monat
               verloren. Ich habe mich ins Hospital eingelegt um sorgfältige Pflege zu haben, die
               Entzündung schien schon zwei Mal erloschen, kam aber dann wieder. Ich liege also
               vorläufig in dieser gelinden Tortur, das Bein hoch und in der Schiene {\pb}auf dem Rücken immer in derselben
               Lage ohne mich weder rechts noch links drehen zu können.\pend
           \pstart
           Dies ist der dritte Frühling, den ich nicht sehe (97, 99,
                  1900)\pend
           \pstart
           Die deutschen Blätter haben Dutzende von Schmähartikeln gegen mich enthalten, weil
               ich in dem Klub in Budapest\orgindex{Liberaler Club@Liberaler Club|pwv},
               aufgefordert, eine französische\oindex{Frankreich@\textbf{Frankreich}|pw} Einleitung zu
               machen (was mir lächerlich vorkam), einfach sagte »Die Sprache, deren ich mich
               bediene ist nicht die Ihre und nicht die meine, nicht Ihre Lieblingssprache und nicht
               die meine, doch es ist die, worin wir uns am leichtesten verstehen.« Das wird ein \uline{hämischer} Angriff auf Deutschland\oindex{Deutschland@\textbf{Deutschland}|pw} und die deutsche Kultur genannt. Und zwar von anonymen Bengeln,
               die nicht mehr Antheil an die deutsche Kultur haben als ein alter Stiefel. \strikeout{V}Die Verachtung, die ich für die Journalisten {\pb}hege, ist nach und nach so gross,
               dass ich förmlich einen bitteren Geschmack im Munde davon habe, wenn ich daran
               denke.\pend
           \pstart
           Ich bin Ihnen und Beer-Hoffmann\pwindex{Beer-Hofmann, Richard 1866-07-11 – 1945-09-26@\textsc{Beer-Hofmann, Richard} (1866-07-11 – 1945-09-26), \emph{Schriftsteller}|pw} wie gewöhnlich
               vielen Dank für Wien\oindex{Wien@\textbf{Wien}|pw} schuldig.\pend
           \pstart
           Sie beiden und Gomperz\pwindex{Gomperz, Theodor 29.03.1832 – 29.08.1912@\textsc{Gomperz, Theodor} (29.03.1832 – 29.08.1912), \emph{Altphilologe}|pw}’s Haus und Lanckoronski\pwindex{Lanckoroński, Karl 04.11.1848 – 15.07.1934@\textsc{Lanckoroński, Karl} (04.11.1848 – 15.07.1934), \emph{Schriftsteller, Sammler, Forscher}|pw} waren dies mal mein Wien\oindex{Wien@\textbf{Wien}|pw}. Ich habe Sie sehr lieb und freue mich, dass
               wir Freunde sind.\pend
           \pstart
           Ich las jetzt im Bett einige Bücher: \uline{Drames de famille}\pwindex{Bourget, Paul 02.09.1852 – 25.12.1935@\textsc{Bourget, Paul} (02.09.1852 – 25.12.1935), \emph{Schriftsteller}!Familiendramen1900@\strich\emph{Familiendramen} {[}1900{]}|pw}, die Bourget\pwindex{Bourget, Paul 02.09.1852 – 25.12.1935@\textsc{Bourget, Paul} (02.09.1852 – 25.12.1935), \emph{Schriftsteller}|pw} mir schickte trotzdem er so
               katholisch geworden ist; die zwei grossen Erzählungen, die in unsern nordischen\oindex{Skandinavien@\textbf{Skandinavien}|pwv} Blättern übel besprochen werden,
               gefielen mir sehr, wenn auch nicht die moralisirende Schreibweise, doch Stoff und
               Ausführung. Dann las ich einen deutschen Roman, der mir geschickt wurde und der mir
               gut scheint, Wilhelm Hegeler\pwindex{Hegeler, Wilhelm 25.02.1870 – 08.10.1943@\textsc{Hegeler, Wilhelm} (25.02.1870 – 08.10.1943), \emph{Schriftsteller}|pw}, \uline{Ingenieur Horstmann}\pwindex{Hegeler, Wilhelm 25.02.1870 – 08.10.1943@\textsc{Hegeler, Wilhelm} (25.02.1870 – 08.10.1943), \emph{Schriftsteller}!Ingenieur Horstmann1900@\strich\emph{Ingenieur Horstmann} {[}1900{]}|pw}, eine {\pb}sehr tüchtige
               Leistung. Mit Interesse las ich Balzacs\pwindex{Balzac, Honore de 20.05.1799 – 18.08.1850@\textsc{Balzac, Honoré de} (20.05.1799 – 18.08.1850), \emph{Schriftsteller}|pw}{ }Briefe À L’Etrangère\pwindex{Balzac, Honore de 20.05.1799 – 18.08.1850@\textsc{Balzac, Honoré de} (20.05.1799 – 18.08.1850), \emph{Schriftsteller}!Lettres à l etrangere (1833–1842), (1842–1844)1899 – 1906@\strich\emph{Lettres à l’étrangère (1833–1842), (1842–1844)} {[}1899 – 1906{]}|pw} d. h. an seine zukünftige
                  Frau\pwindex{Hańska, Ewelina 06.01.1801 – 10.04.1882@\textsc{Hańska, Ewelina} (06.01.1801 – 10.04.1882)|pwv} in der ersten \label{K_L01033-1v}\edtext{\uline{vollständigen} Ausgabe}{\lemma{\textnormal{\emph{vollständigen Ausgabe}}}\Cendnote{\textnormal{H. de Balzac\pwindex{Balzac, Honore de 20.05.1799 – 18.08.1850@\textsc{Balzac, Honoré de} (20.05.1799 – 18.08.1850), \emph{Schriftsteller}|pwk}: \emph{Œuvres
                        posthumes} I. \emph{Lettres à l’étrangère
                        (1833–1842)}\pwindex{Balzac, Honore de 20.05.1799 – 18.08.1850@\textsc{Balzac, Honoré de} (20.05.1799 – 18.08.1850), \emph{Schriftsteller}!Lettres à l etrangere (1833–1842), (1842–1844)1899 – 1906@\strich\emph{Lettres à l’étrangère (1833–1842), (1842–1844)} {[}1899 – 1906{]}|pwk}. Paris: \emph{Calmann-Lévy,
                        Éditeurs}\orgindex{Calmann-Levy@Calmann-Lévy|pwk}{ }{[}1899{]}.}}}\label{K_L01033-1h}.\pend
           \pstart
           Es war amüsant, den Ton in Lanckoronskis\pwindex{Lanckoroński, Karl 04.11.1848 – 15.07.1934@\textsc{Lanckoroński, Karl} (04.11.1848 – 15.07.1934), \emph{Schriftsteller, Sammler, Forscher}|pw}{ }\uline{Rund um die Welt}\pwindex{Lanckoroński, Karl 04.11.1848 – 15.07.1934@\textsc{Lanckoroński, Karl} (04.11.1848 – 15.07.1934), \emph{Schriftsteller, Sammler, Forscher}!Rund um die Erde 1888–891891@\strich\emph{Rund um die Erde 1888–89} {[}1891{]}|pw} mit dem in unseres Freundes Goldmann\pwindex{Goldmann, Paul 31.01.1865 – 25.09.1935@\textsc{Goldmann, Paul} (31.01.1865 – 25.09.1935), \emph{Schriftsteller, Journalist}|pw}’s\pwindex{Goldmann, Paul 31.01.1865 – 25.09.1935@\textsc{Goldmann, Paul} (31.01.1865 – 25.09.1935), \emph{Schriftsteller, Journalist}!Sommer in China. Reisebilder1899-05-02@\strich\emph{Ein Sommer in China. Reisebilder} {[}1899-05-02{]}|pwv} zu vergleichen. Goldmann\pwindex{Goldmann, Paul 31.01.1865 – 25.09.1935@\textsc{Goldmann, Paul} (31.01.1865 – 25.09.1935), \emph{Schriftsteller, Journalist}|pw} hat mehr Geist und Herz, der Graf hat
               viel mehr gesehen und (erstaunlich!) er hat darin ein gutes lyrisches Gedicht
               geschrieben.\pend
           \pstart
           Es ist trist, so oft und lange krank zu sein. Ich bin ganz ausser Stande, irgend eine
               ordentliche Arbeit vorzunehmen, meine tägliche Arbeit besteht allein darin, die
               Ausgabe meiner Sämmtlichen
                  Schriften\pwindex{Brandes, Georg 04.02.1842 – 19.02.1927@\textsc{Brandes, Georg} (04.02.1842 – 19.02.1927)!Samlede Skrifter [Gesammelte Werke]1899 – 1910@\strich\emph{Samlede Skrifter [Gesammelte Werke]} {[}1899 – 1910{]}|pwv} zu verbessern und zu corrigiren.\pend
           \pstart
           Ihr Freund{\\[\baselineskip]}\spacefill\mbox{Georg Brandes}\pend
           \leftskip=0em{}
         
         \endnumbering\mylabel{h}\end{ledgroupsized}  \newcommand{\dateiname}{L01033}\newcommand{\titel}{Georg Brandes an Arthur Schnitzler, 30. 4. 1900}\newcommand{\editorInnen}{Martin Anton Müller und Gerd-Hermann Susen}%% latex-leseansicht-abspann.tex
%% Abspann für die Leseansicht.
%% Der Schalter \ifkorrekturansicht ist bereits durch den Vorspann gesetzt.

%% latex-abspann.tex
%% Gemeinsamer Abspann für Korrekturansicht und Leseansicht.
%% Setzt den Schalter \ifkorrekturansicht voraus (gesetzt in den
%% einbindenden Dateien latex-korrekturansicht-abspann.tex bzw.
%% latex-leseansicht-abspann.tex).
%% ---------------------------------------------------------------

\normalsize

% Das esempio-Environment wird nur in der Leseansicht benötigt
\ifkorrekturansicht\else
\newenvironment{esempio}[3]%
{
    \vspace{1.5ex}
    \rlap{\underline{#1}}
    \par
    \setlength{\parindent}{0cm}
    \nopagebreak
    \leftskip=#2cm
    \rightskip=#3cm
}
{
    \par
}
\fi

\doendnotes{C}
\bigskip
\vfill

\clearpage

\footnotesize

\ifkorrekturansicht
  \lohead{\textsc{register}}
\fi

% theindex-Environment neu definieren ohne reledmac
\makeatletter
\renewenvironment{theindex}{%
  \ifkorrekturansicht
    \section*{\indexname}%
  \else
    \subsubsection*{Index der erwähnten Entitäten}%
  \fi
  \setlength{\parindent}{0pt}%
  \setlength{\parskip}{0pt plus 0.3pt}%
  \let\item\@idxitem
}{%
  \ifkorrekturansicht\clearpage\fi
}
\makeatother

\IfFileExists{\jobname-pw.ind}{\input{\jobname-pw.ind}}{}

% Quellenangabe nur in der Leseansicht
\ifkorrekturansicht\else
% Fallback-Definitionen, falls die .tex-Datei \titel etc. nicht gesetzt hat
\providecommand{\titel}{}
\providecommand{\editorInnen}{}
\providecommand{\dateiname}{\jobname}

\vspace{3cm}

\vfill

\footnotesize
\textsc{Quelle}: \titel. Herausgegeben von {\editorInnen}. In: \emph{Arthur Schnitzler: Briefwechsel mit Autorinnen und Autoren}.
 Digitale Edition, https://schnitzler-briefe.acdh.oeaw.ac.at/{\dateiname}.html (Stand \today)
\fi

\end{document}


      