%% latex-korrekturansicht-vorspann.tex
%% Vorspann für die Korrekturansicht.
%% Lädt die gemeinsame Datei latex-vorspann.tex mit gesetztem Schalter.

\newif\ifkorrekturansicht
\korrekturansichttrue

\input{../tex-inputs/latex-vorspann}


\section[Stefan Zweig an Arthur Schnitzler, 2. 11. 1929]{L03691 Stefan Zweig an Arthur Schnitzler, 2. 11. 1929}
\nopagebreak\mylabel{L03691v}
\rehead{ }\normalsize\beginnumbering\briefempfaengerindex{Schnitzler, Arthur@\textsc{Schnitzler, Arthur}!zzzZweig, Stefan@\emph{von Stefan Zweig}!1929-11-021@{2. 11. 1929}|(be}
\toendnotes[C]{\smallbreak\pagebreak[2]}\Standort{CUL, Schnitzler, B 118.}
\physDesc{Brief, 1 Blatt, 1 Seite, 1100 Zeichen
\newline{}Schreibmaschine
\newline{}Handschrift: blaue Tinte, lateinische Kurrent (\noindent{}Korrekturen, Unterschrift)
\newline{}Schnitzler: 1) mit rotem Buntstift beschriftet: »\textsc{Spanien}«  2) mit rotem Buntstift fünf Unterstreichungen}
\buchAbdrucke{\weitereDrucke{Stefan Zweig: \emph{Briefwechsel mit Hermann Bahr, Sigmund Freud, Rainer Maria
                        Rilke und Arthur Schnitzler}. Frankfurt am Main: \emph{S. Fischer} 1987, S. 446.} }\toendnotes[C]{\smallbreak}
\pstart
           {\pb}\textcolor{gray}{\textbf{SZ}}\hfill \textcolor{gray}{\textbf{SALZBURG\oindex{Salzburg@\textbf{Salzburg}, \emph{A.ADM2}|pw}}}\pend
           
\pstart
           \raggedleft{}\textcolor{gray}{\textbf{KAPUZINERBERG 5\oindex{Paschinger Schloessl@\textbf{Paschinger Schlössl}, \emph{Wohngebäude (K.WHS)}|pw}}}\pend
           
\pstart
           \raggedleft{}2. November 1929.\pend
           
\pstart{}Lieber, verehrter Herr Doktor!\pend\vspace{0.5em}
\pstart
           Ich nütze jede Gelegenheit gern, mich an Sie zu wenden und die vorliegende ist ein
               Brief von Herrn \label{K_L03691-1v}\edtext{A. del Vayo\pwindex{Álvarez del Vayo, Julio 1891-02-09 – 1975-05-03@\textsc{Álvarez del Vayo, Julio} (1891-02-09 – 1975-05-03), \emph{Schriftsteller/Schriftstellerin, Politiker/Politikerin, Journalist/Journalistin}|pwu}, (dem Leiter des
               Verlags Editorial Espana\orgindex{Espasa-Calpe@Espasa-Calpe|pwu}}{\lemma{\textnormal{\emph{A. … Espana}}}\Cendnote{\textnormal{Vermutlich ist Espana ein Tippfehler und
                  es geht um eine Anfrage des Verlags \emph{Espasa}\orgindex{Espasa-Calpe@Espasa-Calpe|pwk}, bei dem Stefan Zweig\pwindex{Zweig, Stefan 28.11.1881 – 23.02.1942@\textsc{Zweig, Stefan} (28.11.1881 – 23.02.1942), \emph{Schriftsteller/Schriftstellerin}|pwk}
                  selbst im Jahr darauf ein Buch publizierte (Stefan Zweig\pwindex{Zweig, Stefan 28.11.1881 – 23.02.1942@\textsc{Zweig, Stefan} (28.11.1881 – 23.02.1942), \emph{Schriftsteller/Schriftstellerin}|pwk}: \emph{Fouché. Retrato di un Político}\pwindex{Fouche. Retrato di un Político@\emph{Fouché. Retrato di un Político}|pwk}. Madrid\oindex{Madrid@\textbf{Madrid}, \emph{P.PPLC}|pwk}: \emph{Espasa-Calpe}\orgindex{Espasa-Calpe@Espasa-Calpe|pwk}{ }1930). Bei dem Verleger handelt es sich wohl um den Schriftsteller Julio Alvares del Vayo\pwindex{Álvarez del Vayo, Julio 1891-02-09 – 1975-05-03@\textsc{Álvarez del Vayo, Julio} (1891-02-09 – 1975-05-03), \emph{Schriftsteller/Schriftstellerin, Politiker/Politikerin, Journalist/Journalistin}|pwk}, der schon einmal
                  nach Übersetzungsrechten für das Spanische\oindex{Spanien@\textbf{Spanien}, \emph{A.PCLI}|pwk} angefragt hatte, wie aus zwei Briefen
                     Schnitzlers an ihn aus dem Jahr 1923
                  hervorgeht (\emph{DLA}: HS.1985.1.02118,1-2).}}}\label{K_L03691-1}, Madrid\oindex{Madrid@\textbf{Madrid}, \emph{P.PPLC}|pw}, Palacio de la Prensa\oindex{Palacio de la Prensa@\textbf{Palacio de la Prensa}, \emph{Bürogebäude (K.BUR)}|pw}, Plaza del Callao 4\oindex{Plaza del Callao@\textbf{Plaza del Callao}, \emph{Platz (K.PLT)}|pw}), der sich bei mir beklagt, dass er an Fischer\pwindex{Fischer, Samuel 24.12.1859 – 15.10.1934@\textsc{Fischer, Samuel} (24.12.1859 – 15.10.1934), \emph{Verleger/Verlegerin}|pw} wegen \introOben{}des\introOben{} Uebersetzung\introOben{}srechts\introOben{} ihrer »Therese\pwindex{Therese. Chronik eines Frauenlebens@\emph{Therese. Chronik eines Frauenlebens}|pw}«
               geschrieben habe, ohne aber eine Antwort zu erhalten. Er lässt Sie nun durch mich
               bitten, erstlich, dass Sie dort nachfragen mögen, zweitens, ob Sie ihm bald etwas
               Neues von sich in Aussicht stellen könnten. Ich kenne ihn persönlich und die
               geschäftlichen Beziehungen zu dem Verlage\orgindex{Espasa-Calpe@Espasa-Calpe|pwv} sind durchaus angenehm und korrekt.\pend
           
\pstart
           Noch in den nächsten Tagen grüsst Sie ein kleines Buch\pwindex{Kleine Chronik@\emph{Kleine Chronik}|pwv}{ }Erzählungen\pwindex{Episode vom Genfer See@\emph{Episode vom Genfer See}|pwv}\pwindex{Leporella@\emph{Leporella}|pwv}\pwindex{unsichtbare Sammlung@\emph{Die unsichtbare Sammlung}|pwv}\pwindex{Buchmendel@\emph{Buchmendel}|pwv} von mir und hoffentlich habe ich endlich Gelegenheit, bei Ihnen
               vorzusprechen. Mein letzter Aufenthalt in Wien\oindex{Wien@\textbf{Wien}, \emph{A.ADM2}|pw} war
               furchtbar überhetzt und als ich endlich bei Berta
                  Zuckerkandl\pwindex{Zuckerkandl, Berta 13.04.1864 – 16.10.1945@\textsc{Zuckerkandl, Berta} (13.04.1864 – 16.10.1945), \emph{Journalist/Journalistin, Übersetzer/Übersetzerin}|pw} Ihre geheime Telefon-Nummer auskundschaftete und Sie anrief,
               meldete sich an jenem Sonntag Nachmittag niemand bei Ihnen\pend
           
\pstart
           In alter Herzlichkeit und Verehrung ergeben {\\[\baselineskip]}Ihr{\\[\baselineskip]}\spacefill\mbox{{[}hs.:{]} Stefan Zweig}\pend
           \leftskip=0em{}
\pstart
           \noindent{}Herrn Dr. Artur Schnitzler{\\}Wien\oindex{Wien@\textbf{Wien}, \emph{A.ADM2}|pw}{\\}\label{K_L03691-2v}\edtext{1 Beilage}{\lemma{\textnormal{\emph{1 Beilage}}}\Cendnote{\textnormal{nicht überliefert}}}\label{K_L03691-2}\pend
           \selectlanguage{ngerman}\endnumbering\briefempfaengerindex{Schnitzler, Arthur@\textsc{Schnitzler, Arthur}!zzzZweig, Stefan@\emph{von Stefan Zweig}!1929-11-021@{2. 11. 1929}|)be}\mylabel{L03691h}
\begin{anhang}
\end{anhang}\normalsize

\doendnotes{C}
\bigskip
\vfill

\clearpage

\footnotesize

\lohead{\textsc{register}}

% Definiere theindex-Environment komplett neu ohne reledmac
\makeatletter
\renewenvironment{theindex}{%
  \section*{\indexname}%
  \setlength{\parindent}{0pt}%
  \setlength{\parskip}{0pt plus 0.3pt}%
  \let\item\@idxitem
}{%
  \clearpage
}
\makeatother

\IfFileExists{\jobname-pw.ind}{\input{\jobname-pw.ind}}{}

\end{document}

      