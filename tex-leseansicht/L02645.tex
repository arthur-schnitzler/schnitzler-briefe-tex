%% latex-korrekturansicht-vorspann.tex
%% Vorspann für die Korrekturansicht.
%% Lädt die gemeinsame Datei latex-vorspann.tex mit gesetztem Schalter.

\newif\ifkorrekturansicht
\korrekturansichttrue

\input{../tex-inputs/latex-vorspann}


\section[Paul Goldmann an Arthur Schnitzler, 7. 12. 1889]{L02645 Paul Goldmann an Arthur Schnitzler, 7. 12. 1889}
\nopagebreak\mylabel{L02645v}
\rehead{ }\normalsize\beginnumbering\briefempfaengerindex{Schnitzler, Arthur@\textsc{Schnitzler, Arthur}!zzzGoldmann, Paul@\emph{von Paul Goldmann}!1889-12-071@{7. 12. 1889}|(be}
\toendnotes[C]{\smallbreak\pagebreak[2]}\Standort{DLA, A:Schnitzler, HS.NZ85.1.3162.}
\physDesc{Brief, 1 Blatt, 4 Seiten, 2789 Zeichen
\newline{}Handschrift: blaue Tinte, deutsche Kurrent}\toendnotes[C]{\smallbreak}
\pstart
           \centering{}{\pb}\textcolor{gray}{\textbf{\textbf{Adminiſtration: VII.
                           Seidengaſſe 7\oindex{Seidengasse@\textbf{Seidengasse}, \emph{Straße (K.STR)}|pw}} (Jos. Eberle {\kaufmannsund} Co.\orgindex{Josef Eberle Stein-, Buch und Musikaliendruckerei@Josef Eberle Stein-, Buch und Musikaliendruckerei|pw})}}\pend
           
\pstart
           \centering{}\textcolor{gray}{\textbf{An der Schönen Blauen Donau\orgindex{der schoenen blauen Donau@An der schönen blauen Donau|pw}}}\pend
           
\pstart
           \centering{}\textcolor{gray}{\textbf{Chef-Redacteur: Dr. F.
                        Mamroth\pwindex{Mamroth, Fedor 21.02.1851 – 25.06.1907@\textsc{Mamroth, Fedor} (21.02.1851 – 25.06.1907), \emph{Journalist/Journalistin, Kritiker/Kritikerin}|pw}. – Redaction: IX.,
                        Berggaſſe 31\oindex{Berggasse@\textbf{Berggasse}, \emph{Straße (K.STR)}|pw}.}}\pend
           
\pstart
           \raggedleft{}\textcolor{gray}{\textbf{Wien\oindex{Wien@\textbf{Wien}, \emph{A.ADM2}|pw}, den}}{ }7. December \textcolor{gray}{\textbf{18}}89.\pend
           
\pstart\center{}Lieber Freund!\pend\vspace{0.5em}
\pstart
           Weniger die ſchiefe und ungemein verzwickte Logik Ihres lieben Briefes, als vielmehr
               deſſen Liebenswürdigkeit bringen mich zu der Überzeugung, daß ich in einem Punkte
               jedenfalls Unrecht hatte: in meiner gereizten Auffaſſung der ganzen \label{K_L02645-1v}\edtext{Streitfrage}{\lemma{\textnormal{\emph{Streitfrage}}}\Cendnote{\textnormal{Der Vorgang, der einen Streit ausgelöst hat, ist nur durch
                  diesen und den vorangegangenen Brief (Paul Goldmann an Arthur Schnitzler, 6. 12. 1889) belegt. Wie sich daraus ergibt, hatte sich Goldmann\pwindex{Goldmann, Paul 31.01.1865 – 25.09.1935@\textsc{Goldmann, Paul} (31.01.1865 – 25.09.1935), \emph{Schriftsteller/Schriftstellerin, Journalist/Journalistin}|pwk} mit einer nicht näher
                  identifizierten Frau\pwindex{?? [Frau, die mit Goldmann in der Strassenbahn spricht, Ende November 1889] @\textsc{?? [Frau, die mit Goldmann in der Straßenbahn spricht, Ende November 1889]}|pwkv} in
                  der Straßenbahn über Schnitzler unterhalten.
                  Das Gespräch wurde belauscht und Schnitzler{ }rapportiert\pwindex{?? [Mann, der Gespraech ueber Schnitzler in der Strassenbahn belauscht, Ende November 1889] *~Ende November 1889@\textsc{?? [Mann, der Gespräch über Schnitzler in der Straßenbahn belauscht, Ende November 1889]} (*~Ende November 1889)|pwkv}. Worüber
                  genau gesprochen wurde, bleibt unklar; es könnte sich um eine der mehreren zu
                  diesem Zeitpunkt parallel laufenden aktiven Beziehungen Schnitzlers gehandelt haben: um jene mit der verheirateten
                     Olga Waissnix\pwindex{Waissnix, Olga 03.11.1862 – 04.11.1897@\textsc{Waissnix, Olga} (03.11.1862 – 04.11.1897), \emph{Hotelier/Hotelière}|pwk} oder um die jeweiligen
                  Beziehungen mit Jeanette Heeger\pwindex{Heeger, Jeanette 1865-07-01 – 1903-01-03@\textsc{Heeger, Jeanette} (1865-07-01 – 1903-01-03), \emph{Näher/Näherin}|pwk}, Marie Glümer\pwindex{Gluemer, Marie 03.07.1867 – 16.11.1925@\textsc{Glümer, Marie} (03.07.1867 – 16.11.1925), \emph{Schauspieler/Schauspielerin}|pwk} oder Helene Herz\pwindex{Binder, Helene 23.05.1865 – 11.03.1960@\textsc{Binder, Helene} (23.05.1865 – 11.03.1960)|pwk}.}}}\label{K_L02645-1}. Aber es war gerade geſtern ein Tag höchſter Nervoſität für mich – das war
               der phyſiſche Grund; und dann habe ich mich wüthend geärgert, daß in mein Verhältniß
               zu Ihnen, das mir bisher ſo viel Freude gemacht, ein Mißton gekommen war – das war
               der pſychiſche Grund. Ich will auf die Sache ſelbſt gar nicht mehr eingehen, obwohl
               ich überzeugt bin, daß auch nicht einmal der Vorwurf der Unvorſich{\pb}tigkeit trifft. Woher wiſſen Sie denn überhaupt, ob
               das Mädel\pwindex{?? [Frau, die mit Goldmann in der Strassenbahn spricht, Ende November 1889] @\textsc{?? [Frau, die mit Goldmann in der Straßenbahn spricht, Ende November 1889]}|pwv} Ihren Namen
               genannt hat, oder ob ich das wa\substVorne{}\textsuperscript{\textcolor{gray}{s}}\substDazwischen{}r\substHinten{}? Dieſes Thatbeſtandes-Moment hätten Sie doch erſt aufnehmen müſſen, ehe Sie
               Ihr Verdict fällten. Ich meine nach wie vor, daß ich nur \uline{eine} ſchuldhafte Handlung begangen habe, nämlich die, daß ich auf der
               Tramway überhaupt gefahren bin. Und ich ſehe, ich werde mir in Zukunft, um Ihnen
               Unannehmlichkeiten zu erſparen, das Tramway-Fahren abgewöhnen müſſen.\pend
           
\pstart
           Aber – laſſen wir das wirklich begraben ſein. \uline{Sie}
               haben ſich gekränkt, \uline{ich} habe mich gekränkt; ein
               Dienſtmann hat 30 und die Poſt 6 \textsc{Kr.} verdient; damit hat
               die ganze Affaire, meine ich, Wirkungen genug gehabt, und ſie kann jetzt geruhig vom
               Erdboden verſchwinden. Reden wir nicht mehr davon – ich bin ganz Ihrer Anſicht.\pend
           
\pstart
           Nur noch ein Wort für die Zukunft. Es wird ſelbſtverſtändlich wieder vorkommen, daß
               Sie Gelegenheit haben werden, ſich über mich zu ärgern, obwohl – wie Sie überzeugt
               ſein können – von meiner Seite Alles geſchehen wird, um das zu {\pb}vermeiden. Aber das iſt nun einmal ſo: wozu hätte
               man einen guten Freund anders, als um ſich hie\substVorne{}\textsuperscript{’}\substDazwischen{}r\substHinten{} und da über ihn zu ärgern! Ich bin auch ganz Ihrer Anſicht, daß man jeden
               ſolchen Zwiſchenfall zur Sprache bringen ſoll; dazu iſt man befreundet, daß man ſich
               gegenſeitig ausſpricht. Nur bitte ich Sie um Eines: keine Briefe mehr in Zukunft. Ich
               kann mir nicht \strikeout{\textcolor{gray}{hef}} helfen: ſür mich hat ſo ein Wiſch Papier, der mir \strikeout{S\textcolor{gray}{a}} allerlei unangenehme Sachen ſagt, ohne daß ich in der Lage bin, mich ihm
               gegenüber zu vertheidigen, immer etwas verteufelt Odioſes. Alſo reden Sie zu\strikeout{\textcolor{gray}{r}} Gleiche mit Ihnen thun. So ein Brief iſt wie ein Dritter, der ſich in etwas
               hineinmengt, das nur zwei allein angeht. Alſo, nicht wahr, den Gefallen thun Sie mir
                  \strikeout{\textcolor{gray}{ehe}} in Zukunft? {\dotsfour}\pend
           
\pstart
           {\pb}Und nun nehme ich eine neue Seite, \strikeout{\textcolor{gray}{aus}} wie man das immer thun ſoll, wenn man mit ſich in’s Reine gekommen iſt und
               wenn Alles wieder gut geworden. Und frage Sie, ob man heut{ }Abend auf das Vergnügen Ihrer Geſellſchaft beim \label{K_L02645-2v}\edtext{Souper}{\lemma{\textnormal{\emph{Souper}}}\Cendnote{\textnormal{Sie trafen sich erst am nächsten Tag, dem 8. 12. 1889, gemeinsam mit Julius
                     Schnitzler\pwindex{Schnitzler, Julius 13.07.1865 – 29.06.1939@\textsc{Schnitzler, Julius} (13.07.1865 – 29.06.1939), \emph{Chirurg/Chirurgin}|pwk}.}}}\label{K_L02645-2} rechnen\strikeout{\textcolor{gray}{×}} kann. Oder wann ſonſt, wenn nicht heut{ }Abend? Und wenn heut{ }Abend – wo und zu welcher Stunde?\pend
           
\pstart
           Mein Bote wartet auf Antwort.\pend
           
\pstart
           Herzlichſten Gruß! {\\[\baselineskip]}Ihr {\\[\baselineskip]}\spacefill\mbox{Paul Goldmann.}\pend
           \leftskip=0em{}\selectlanguage{ngerman}\endnumbering\briefempfaengerindex{Schnitzler, Arthur@\textsc{Schnitzler, Arthur}!zzzGoldmann, Paul@\emph{von Paul Goldmann}!1889-12-071@{7. 12. 1889}|)be}\mylabel{L02645h}  \normalsize

\doendnotes{C}
\bigskip
\vfill

\clearpage

\footnotesize

\lohead{\textsc{register}}

% Definiere theindex-Environment komplett neu ohne reledmac
\makeatletter
\renewenvironment{theindex}{%
  \section*{\indexname}%
  \setlength{\parindent}{0pt}%
  \setlength{\parskip}{0pt plus 0.3pt}%
  \let\item\@idxitem
}{%
  \clearpage
}
\makeatother

\IfFileExists{\jobname-pw.ind}{\input{\jobname-pw.ind}}{}

\end{document}

      