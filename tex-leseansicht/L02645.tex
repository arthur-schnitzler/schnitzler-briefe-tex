%% latex-leseansicht-vorspann.tex
%% Vorspann für die Leseansicht.
%% Lädt die gemeinsame Datei latex-vorspann.tex mit nicht gesetztem Schalter.

\newif\ifkorrekturansicht
\korrekturansichtfalse

\input{../tex-inputs/latex-vorspann}


\section[Paul Goldmann an Arthur Schnitzler, 7. 12. 1889]{L02645 Paul Goldmann an Arthur Schnitzler, 7. 12. 1889}
\nopagebreak\mylabel{L02645v}
\rehead{ }\normalsize\beginnumbering\briefempfaengerindex{Schnitzler, Arthur@\textsc{Schnitzler, Arthur}!zzzGoldmann, Paul@\emph{von Paul Goldmann}!1889-12-071@{7. 12. 1889}|(be}
\toendnotes[C]{\smallbreak\pagebreak[2]}
\correspDesc{Versand  durch Paul Goldmann am 7. 12. 1889 in Wien
\newline{}Erhalt  durch Arthur Schnitzler im Zeitraum [7. 12. 1889
                  – 9. 12. 1889?] in Wien}\toendnotes[C]{\smallbreak}
\Standort{DLA, A:Schnitzler, HS.NZ85.1.3162.}
\physDesc{Brief, 1 Blatt, 4 Seiten, 2789 Zeichen
\newline{}Handschrift: blaue Tinte, deutsche Kurrent}\toendnotes[C]{\smallbreak}
\pstart
           \centering{}{\pb}\textcolor{gray}{\textbf{\textbf{Adminiſtration: VII.
                           Seidengaſſe 7\oindex{Wien@\textbf{Wien}!VII., Neubau@\textbf{VII., Neubau}!Seidengasse@\textbf{Seidengasse}, \emph{Straße}|pw}} (Jos. Eberle {\kaufmannsund} Co.\orgindex{Josef Eberle Stein-, Buch und Musikaliendruckerei@Josef Eberle Stein-, Buch und Musikaliendruckerei|pw})}}\pend
           
\pstart
           \centering{}\textcolor{gray}{\textbf{An der Schönen Blauen Donau\orgindex{der schönen blauen Donau@An der schönen blauen Donau|pw}}}\pend
           
\pstart
           \centering{}\textcolor{gray}{\textbf{Chef-Redacteur: Dr. F.
                        Mamroth\pwindex{Mamroth, Fedor 21.\,2.\,1851 Breslau – 25.\,6.\,1907 Frankfurt am Main@\textsc{Mamroth, Fedor} (21.\,2.\,1851 Breslau – 25.\,6.\,1907 Frankfurt am Main), \emph{Journalist, Kritiker}|pw}. – Redaction: IX.,
                        Berggaſſe 31\oindex{Wien@\textbf{Wien}!IX., Alsergrund@\textbf{IX., Alsergrund}!Berggasse@\textbf{Berggasse}, \emph{Straße}|pw}.}}\pend
           
\pstart
           \raggedleft{}\textcolor{gray}{\textbf{Wien\oindex{Wien@\textbf{Wien}, \emph{Verwaltungsgebiet}|pw}, den}}{ }7. December \textcolor{gray}{\textbf{18}}89.\pend
           
\pstart\center{}Lieber Freund!\pend\vspace{0.5em}
\pstart
           Weniger die{ }ſchiefe und ungemein verzwickte Logik Ihres lieben Briefes, als vielmehr
               deſſen Liebenswürdigkeit bringen mich zu der Überzeugung, daß ich in einem Punkte
               jedenfalls Unrecht hatte: in meiner gereizten Auffaſſung der ganzen \label{K_L02645-1v}\edtext{Streitfrage}{\lemma{\textnormal{\emph{Streitfrage}}}\Cendnote{\textnormal{Der Vorgang, der einen Streit ausgelöst hat, ist nur durch
                  diesen und den vorangegangenen Brief (XXXX Auszeichnungsfehler: Dokument L02646 nicht gefunden) belegt. Wie sich daraus ergibt, hatte sich Goldmann\pwindex{Goldmann, Paul 31.\,1.\,1865 Breslau – 25.\,9.\,1935 Wien@\textsc{Goldmann, Paul} (31.\,1.\,1865 Breslau – 25.\,9.\,1935 Wien), \emph{Schriftsteller, Journalist}|pwk} mit einer nicht näher
                  identifizierten Frau\pwindex{?? [Frau, die mit Goldmann in der Straßenbahn spricht, Ende November 1889] @\textsc{?? [Frau, die mit Goldmann in der Straßenbahn spricht, Ende November 1889]}|pwkv} in
                  der Straßenbahn über Schnitzler unterhalten.
                  Das Gespräch wurde belauscht und Schnitzler{ }rapportiert\pwindex{?? [Mann, der Gespräch über Schnitzler in der Straßenbahn belauscht, Ende November 1889] *~Ende November 1889@\textsc{?? [Mann, der Gespräch über Schnitzler in der Straßenbahn belauscht, Ende November 1889]} (*~Ende November 1889)|pwkv}. Worüber
                  genau gesprochen wurde, bleibt unklar; es könnte sich um eine der mehreren zu
                  diesem Zeitpunkt parallel laufenden aktiven Beziehungen Schnitzlers gehandelt haben: um jene mit der verheirateten
                     Olga Waissnix\pwindex{Waissnix, Olga 3.\,11.\,1862 Wien – 4.\,11.\,1897 ebd.@\textsc{Waissnix, Olga} (3.\,11.\,1862 Wien – 4.\,11.\,1897 ebd.), \emph{Hotelière}|pwk} oder um die jeweiligen
                  Beziehungen mit Jeanette Heeger\pwindex{Heeger, Jeanette 1.\,7.\,1865 Šternberk – 3.\,1.\,1903 Wien@\textsc{Heeger, Jeanette} (1.\,7.\,1865 Šternberk – 3.\,1.\,1903 Wien), \emph{Näherin}|pwk}, Marie Glümer\pwindex{Glümer, Marie 3.\,7.\,1867 Wien – 16.\,11.\,1925 München@\textsc{Glümer, Marie} (3.\,7.\,1867 Wien – 16.\,11.\,1925 München), \emph{Schauspielerin}|pwk} oder Helene Herz\pwindex{Binder, Helene 23.\,5.\,1865 Peciu Nou – 11.\,3.\,1960 Willesden@\textsc{Binder, Helene} (23.\,5.\,1865 Peciu Nou – 11.\,3.\,1960 Willesden)|pwk}.}}}\label{K_L02645-1}. Aber es war gerade geſtern ein Tag höchſter Nervoſität für mich – das war
               der phyſiſche Grund; und dann habe ich mich wüthend geärgert, daß in mein Verhältniß
               zu Ihnen, das mir bisher{ }ſo viel Freude gemacht, ein Mißton gekommen war – das war
               der pſychiſche Grund. Ich will auf die Sache{ }ſelbſt gar nicht mehr eingehen, obwohl
               ich überzeugt bin, daß auch nicht einmal der Vorwurf der Unvorſich{\pb}tigkeit trifft. Woher wiſſen Sie denn überhaupt, ob
               das Mädel\pwindex{?? [Frau, die mit Goldmann in der Straßenbahn spricht, Ende November 1889] @\textsc{?? [Frau, die mit Goldmann in der Straßenbahn spricht, Ende November 1889]}|pwv} Ihren Namen
               genannt hat, oder ob ich das wa\substVorne{}\textsuperscript{\textcolor{gray}{s}}\substDazwischen{}r\substHinten{}? Dieſes Thatbeſtandes-Moment hätten Sie doch erſt aufnehmen müſſen, ehe Sie
               Ihr Verdict fällten. Ich meine nach wie vor, daß ich nur \uline{eine}{ }ſchuldhafte Handlung begangen habe, nämlich die, daß ich auf der
               Tramway überhaupt gefahren bin. Und ich{ }ſehe, ich werde mir in Zukunft, um Ihnen
               Unannehmlichkeiten zu erſparen, das Tramway-Fahren abgewöhnen müſſen.\pend
           
\pstart
           Aber – laſſen wir das wirklich begraben{ }ſein. \uline{Sie}
               haben{ }ſich gekränkt, \uline{ich} habe mich gekränkt; ein
               Dienſtmann hat 30 und die Poſt 6 \textsc{Kr.} verdient; damit hat
               die ganze Affaire, meine ich, Wirkungen genug gehabt, und{ }ſie kann jetzt geruhig vom
               Erdboden verſchwinden. Reden wir nicht mehr davon – ich bin ganz Ihrer Anſicht.\pend
           
\pstart
           Nur noch ein Wort für die Zukunft. Es wird{ }ſelbſtverſtändlich wieder vorkommen, daß
               Sie Gelegenheit haben werden,{ }ſich über mich zu ärgern, obwohl – wie Sie überzeugt{ }ſein können – von meiner Seite Alles geſchehen wird, um das zu {\pb}vermeiden. Aber das iſt nun einmal{ }ſo: wozu hätte
               man einen guten Freund anders, als um{ }ſich hie\substVorne{}\textsuperscript{’}\substDazwischen{}r\substHinten{} und da über ihn zu ärgern! Ich bin auch ganz Ihrer Anſicht, daß man jeden{ }ſolchen Zwiſchenfall zur Sprache bringen{ }ſoll; dazu iſt man befreundet, daß man{ }ſich
               gegenſeitig ausſpricht. Nur bitte ich Sie um Eines: keine Briefe mehr in Zukunft. Ich
               kann mir nicht \strikeout{\textcolor{gray}{hef}} helfen:{ }ſür mich hat{ }ſo ein Wiſch Papier, der mir \strikeout{S\textcolor{gray}{a}} allerlei unangenehme Sachen{ }ſagt, ohne daß ich in der Lage bin, mich ihm
               gegenüber zu vertheidigen, immer etwas verteufelt Odioſes. Alſo reden Sie zu\strikeout{\textcolor{gray}{r}} Gleiche mit Ihnen thun. So ein Brief iſt wie ein Dritter, der{ }ſich in etwas
               hineinmengt, das nur zwei allein angeht. Alſo, nicht wahr, den Gefallen thun Sie mir
                  \strikeout{\textcolor{gray}{ehe}} in Zukunft? {\dotsfour}\pend
           
\pstart
           {\pb}Und nun nehme ich eine neue Seite, \strikeout{\textcolor{gray}{aus}} wie man das immer thun{ }ſoll, wenn man mit{ }ſich in’s Reine gekommen iſt und
               wenn Alles wieder gut geworden. Und frage Sie, ob man heut{ }Abend auf das Vergnügen Ihrer Geſellſchaft beim \label{K_L02645-2v}\edtext{Souper}{\lemma{\textnormal{\emph{Souper}}}\Cendnote{\textnormal{Sie trafen sich erst am nächsten Tag, dem 8. 12. 1889, gemeinsam mit Julius
                     Schnitzler\pwindex{Schnitzler, Julius 13.\,7.\,1865 Wien – 29.\,6.\,1939 ebd.@\textsc{Schnitzler, Julius} (13.\,7.\,1865 Wien – 29.\,6.\,1939 ebd.), \emph{Chirurg}|pwk}.}}}\label{K_L02645-2} rechnen\strikeout{\textcolor{gray}{×}} kann. Oder wann{ }ſonſt, wenn nicht heut{ }Abend? Und wenn heut{ }Abend – wo und zu welcher Stunde?\pend
           
\pstart
           Mein Bote wartet auf Antwort.\pend
           
\pstart
           Herzlichſten Gruß! {\\[\baselineskip]}Ihr {\\[\baselineskip]}\spacefill\mbox{Paul Goldmann.}\pend
           \leftskip=0em{}\selectlanguage{ngerman}\endnumbering\briefempfaengerindex{Schnitzler, Arthur@\textsc{Schnitzler, Arthur}!zzzGoldmann, Paul@\emph{von Paul Goldmann}!1889-12-071@{7. 12. 1889}|)be}\mylabel{L02645h}  \newcommand{\dateiname}{L02645}\newcommand{\titel}{Paul Goldmann an Arthur Schnitzler, 7. 12. 1889}\newcommand{\editorInnen}{Martin Anton Müller und Laura Untner}%% latex-leseansicht-abspann.tex
%% Abspann für die Leseansicht.
%% Der Schalter \ifkorrekturansicht ist bereits durch den Vorspann gesetzt.

%% latex-abspann.tex
%% Gemeinsamer Abspann für Korrekturansicht und Leseansicht.
%% Setzt den Schalter \ifkorrekturansicht voraus (gesetzt in den
%% einbindenden Dateien latex-korrekturansicht-abspann.tex bzw.
%% latex-leseansicht-abspann.tex).
%% ---------------------------------------------------------------

\normalsize

% Das esempio-Environment wird nur in der Leseansicht benötigt
\ifkorrekturansicht\else
\newenvironment{esempio}[3]%
{
    \vspace{1.5ex}
    \rlap{\underline{#1}}
    \par
    \setlength{\parindent}{0cm}
    \nopagebreak
    \leftskip=#2cm
    \rightskip=#3cm
}
{
    \par
}
\fi

\doendnotes{C}
\bigskip
\vfill

\clearpage

\footnotesize

\ifkorrekturansicht
  \lohead{\textsc{register}}
\fi

% theindex-Environment neu definieren ohne reledmac
\makeatletter
\renewenvironment{theindex}{%
  \ifkorrekturansicht
    \section*{\indexname}%
  \else
    \subsubsection*{Index der erwähnten Entitäten}%
  \fi
  \setlength{\parindent}{0pt}%
  \setlength{\parskip}{0pt plus 0.3pt}%
  \let\item\@idxitem
}{%
  \ifkorrekturansicht\clearpage\fi
}
\makeatother

\IfFileExists{\jobname-pw.ind}{\input{\jobname-pw.ind}}{}

% Quellenangabe nur in der Leseansicht
\ifkorrekturansicht\else
% Fallback-Definitionen, falls die .tex-Datei \titel etc. nicht gesetzt hat
\providecommand{\titel}{}
\providecommand{\editorInnen}{}
\providecommand{\dateiname}{\jobname}

\vspace{3cm}

\vfill

\footnotesize
\textsc{Quelle}: \titel. Herausgegeben von {\editorInnen}. In: \emph{Arthur Schnitzler: Briefwechsel mit Autorinnen und Autoren}.
 Digitale Edition, https://schnitzler-briefe.acdh.oeaw.ac.at/{\dateiname}.html (Stand \today)
\fi

\end{document}


