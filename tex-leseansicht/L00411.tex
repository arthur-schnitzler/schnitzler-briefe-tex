%% latex-leseansicht-vorspann.tex
%% Vorspann für die Leseansicht.
%% Lädt die gemeinsame Datei latex-vorspann.tex mit nicht gesetztem Schalter.

\newif\ifkorrekturansicht
\korrekturansichtfalse

\input{../tex-inputs/latex-vorspann}


         \renewcommand{\erwaehnteOrte}{Orte: Wien}
         \renewcommand{\erwaehnteWerke}{Werke: Sterben. Novelle}
               \section[Jakob Julius David an Arthur Schnitzler, 23. 12. 1894]{ Jakob Julius David an Arthur Schnitzler, 23. 12. 1894}\nopagebreak\mylabel{v}\rehead{ }\begin{ledgroupsized}[t]{13cm}\normalsize\beginnumbering \toendnotes[C]{\smallbreak\pagebreak[2]} \Standort{CUL, Schnitzler, B 25.}
\physDesc{Brief, 1 Blatt, 1 Seite
\newline{}Handschrift: schwarze Tinte, lateinische Kurrent
\newline{}Schnitzler: 1) mit rotem Buntstift beschriftet: »\textsc{David}« und der Buchtitel unterstrichen  2) mit Bleistift nummeriert: »1.«}\buchAbdrucke{\weitereDrucke{Josef Körner: \emph{Herman Groeneweg, J. J. David in seinem Verhältnis zur Heimat, Geschichte, Gesellschaft und Literatur.} In: \emph{Literaturblatt für germanische und romanische
                        Philologie}, Jg. 52 (1931), Sp. 33.} }\toendnotes[C]{\smallbreak}\pstart
           \raggedleft{}{\pb}23/12 94.\pend
           \pstart\center{}Werther Herr Doctor!\pend\pstart
           Ich habe Sterben\pwindex{Schnitzler, Arthur 15.05.1862 – 21.10.1931@\textsc{Schnitzler, Arthur} (15.05.1862 – 21.10.1931), \emph{Schriftsteller, Mediziner}!Sterben. Novelle1894-10-01 – 1894-12-01@\strich\emph{Sterben. Novelle} {[}1894-10-01 – 1894-12-01{]}|pw} bis nun zwei mal gelesen, und
               werde wohl noch darauf zurückkommen. Es ist eine höchst tüchtige und eine wirklich
               merkwürdige Arbeit; in der Analyse von wirksamster Feinheit und Tiefe.
               Bewundernswerth ist die Kunst, mit welcher Sie den zeitlich so knappen und doch für
               die Vorgänge fast zu weitgesteckten Rahmen mit Leben zu erfüllen wißen. Es ist ein
               vollkommen zielbewußtes Schlendern; was Abschweifung erscheinen könnte, führt nur
               desto sicherer zum letzten Ende. Manchmal möcht’ ich mir mehr Leidenschaftlichkeit
               verlangen; besonders am Schluße könnte ein stärkeres Temperament durchbrennen. Aber:
               Sie haben in dieser Arbeit\pwindex{Schnitzler, Arthur 15.05.1862 – 21.10.1931@\textsc{Schnitzler, Arthur} (15.05.1862 – 21.10.1931), \emph{Schriftsteller, Mediziner}!Sterben. Novelle1894-10-01 – 1894-12-01@\strich\emph{Sterben. Novelle} {[}1894-10-01 – 1894-12-01{]}|pwv} einen
               mächtigen Ruck vorwärts gethan und will ich Ihnen sagen, in wie ferne mir Arbeit das
               Höchste dünkt: im Sinne der Arbeit an sich selbst. Da nun sind Sie tüchtig und
               ehrlich am Werke und darum rücken Sie vor in schönen Erfolgen und zu einer ersten
               Stellung, auf die Sie heute schon Anspruch haben.\pend
           \pstart
           Es grüßt und begrüßt Sie herzlichst{\\[\baselineskip]}Ihr{\\[\baselineskip]}\spacefill\mbox{David}\pend
           \leftskip=0em{}
         
         \endnumbering\mylabel{h}\end{ledgroupsized}  \newcommand{\dateiname}{L00411}\newcommand{\titel}{Jakob Julius David an Arthur Schnitzler, 23. 12. 1894}\newcommand{\editorInnen}{Martin Anton Müller und Gerd-Hermann Susen}%% latex-leseansicht-abspann.tex
%% Abspann für die Leseansicht.
%% Der Schalter \ifkorrekturansicht ist bereits durch den Vorspann gesetzt.

%% latex-abspann.tex
%% Gemeinsamer Abspann für Korrekturansicht und Leseansicht.
%% Setzt den Schalter \ifkorrekturansicht voraus (gesetzt in den
%% einbindenden Dateien latex-korrekturansicht-abspann.tex bzw.
%% latex-leseansicht-abspann.tex).
%% ---------------------------------------------------------------

\normalsize

% Das esempio-Environment wird nur in der Leseansicht benötigt
\ifkorrekturansicht\else
\newenvironment{esempio}[3]%
{
    \vspace{1.5ex}
    \rlap{\underline{#1}}
    \par
    \setlength{\parindent}{0cm}
    \nopagebreak
    \leftskip=#2cm
    \rightskip=#3cm
}
{
    \par
}
\fi

\doendnotes{C}
\bigskip
\vfill

\clearpage

\footnotesize

\ifkorrekturansicht
  \lohead{\textsc{register}}
\fi

% theindex-Environment neu definieren ohne reledmac
\makeatletter
\renewenvironment{theindex}{%
  \ifkorrekturansicht
    \section*{\indexname}%
  \else
    \subsubsection*{Index der erwähnten Entitäten}%
  \fi
  \setlength{\parindent}{0pt}%
  \setlength{\parskip}{0pt plus 0.3pt}%
  \let\item\@idxitem
}{%
  \ifkorrekturansicht\clearpage\fi
}
\makeatother

\IfFileExists{\jobname-pw.ind}{\input{\jobname-pw.ind}}{}

% Quellenangabe nur in der Leseansicht
\ifkorrekturansicht\else
% Fallback-Definitionen, falls die .tex-Datei \titel etc. nicht gesetzt hat
\providecommand{\titel}{}
\providecommand{\editorInnen}{}
\providecommand{\dateiname}{\jobname}

\vspace{3cm}

\vfill

\footnotesize
\textsc{Quelle}: \titel. Herausgegeben von {\editorInnen}. In: \emph{Arthur Schnitzler: Briefwechsel mit Autorinnen und Autoren}.
 Digitale Edition, https://schnitzler-briefe.acdh.oeaw.ac.at/{\dateiname}.html (Stand \today)
\fi

\end{document}


      