%% latex-korrekturansicht-vorspann.tex
%% Vorspann für die Korrekturansicht.
%% Lädt die gemeinsame Datei latex-vorspann.tex mit gesetztem Schalter.

\newif\ifkorrekturansicht
\korrekturansichttrue

\input{../tex-inputs/latex-vorspann}


\section[Arthur Schnitzler an Georg Brandes, 12. 6. 1894]{L00336 Arthur Schnitzler an Georg Brandes, 12. 6. 1894}
\nopagebreak\mylabel{L00336v}
\rehead{ }\normalsize\beginnumbering\briefempfaengerindex{Brandes, Georg@\textsc{Brandes, Georg}!zzzSchnitzler, Arthur@\emph{von Arthur Schnitzler}!1894-06-121@{12. 6. 1894}|(be}
\toendnotes[C]{\smallbreak\pagebreak[2]}\Standort{Kopenhagen, Det Kongelige Bibliotek, Georg Brandes Arkiv, box 125.}
\physDesc{Brief, 2 Blätter, 7 Seiten, 2575 Zeichen
\newline{}Handschrift: schwarze Tinte, deutsche Kurrent
\newline{}Ordnung: auf der ersten Seite mit Bleistift »Schnitzler«
                                 und Briefnummerierung: »1«, das zweite Blatt mit »12/6 94« gekennzeichnet }
\buchAbdrucke{\weitereDrucke{1) Georg Brandes, Arthur Schnitzler: \emph{Ein Briefwechsel}. Bern: \emph{Francke} 1956, S. 55–56.} \weitereDrucke{2) Arthur Schnitzler: \emph{Briefe 1875–1912}. Frankfurt am Main: \emph{S. Fischer} 1981, S. 225–227.} }\toendnotes[C]{\smallbreak}
\pstart
           \raggedleft{}{\pb}\textsc{IX. Frankgasse 1.}\oindex{Frankgasse 1@\textbf{Frankgasse 1}, \emph{Wohngebäude (K.WHS)}|pw}{\\}\textsc{Wien\oindex{Wien@\textbf{Wien}, \emph{A.ADM2}|pw}, 12. Juni 94.}\pend
           
\pstart\center{}Hochverehrter Herr,\pend\vspace{0.5em}
\pstart
           es iſt nicht ſchwer ſich vorzuſtellen, wie viel Bücher Sie zugeſandt beko{\geminationm}en, und als ich mir erlaubte, Ihnen die meinen\pwindex{Maerchen. Schauspiel in drei Aufzuegen@\emph{Das Märchen. Schauspiel in drei Aufzügen}|pwv}\pwindex{Anatol@\emph{Anatol}|pwv} zu ſchicken,
               hab ich natürlich gehofft – habe aber gewiſs nicht darauf gerechnet, daſs Sie Zeit
               und Luſt haben würden, die Bücher eines ziemlich Unbeka{\geminationn}ten zu leſen. Und nun habe ich Ihren Brief beko{\geminationm}en,
               mit all dem liebens{\pb}würdigen und ehrenvollen, das
               er enthält; und ich ka{\geminationn} Ihnen gar nicht ſagen, eine wie
               tiefe Freude er mir bedeutet hat. Auf eine kurze Reiſe, von der ich eben
               zurückgekehrt bin, hatte ich Ihr letztes mir unbeka{\geminationn}tes
               Buch »Menſchen u Werke\pwindex{Menschen und Werke@\emph{Menschen und Werke}|pw}« mitgeno{\geminationm}en. Ich bin es gewohnt, Ihre Bücher mit der ſtillen
               Bewunderung zu leſen, die man großen und fernen Geiſtern entgegen{\pb}bringt; diesmal habe ich aber auch andres
               empfunden. Ich glaube, es war eine Art von Stolz. Mit einem Male iſt meine Exiſtenz
               in das Bereich Ihres Schauens gerückt, und we{\geminationn} ich Ihnen
               ſage, daſs ich Sie verehre, ſo geht meine Stimme nicht unter den tauſenden verloren,
               deren Namen Sie nicht kennen. Dieſe vielleicht etwas hochmütige Empfindung blieb mir
                  {\pb}von der erſten bis zur letzten Zeile, – und,
               ich will es Ihnen nur geſtehn, ſie hat mir ſo wohl gethan, daſs ich mir ſehr feſt
               vorgenommen habe, von Ihnen nicht wieder vergeſſen zu werden. Ihre Worte,
               hochverehrter Herr, ſind mehr als Anerke{\geminationn}ung, Lob,
               Ermuthigung – ich betrachte ſie als Würde, die mir verliehen iſt; – laſſen Sie mich
               Ihnen aufs innigſte dafür {\pb}danken.\pend
           
\pstart
           Es iſt Ihnen, hochverehrter Herr, kaum beka{\geminationn}t geworden,
               daſs »Das Märchen\pwindex{Maerchen. Schauspiel in drei Aufzuegen@\emph{Das Märchen. Schauspiel in drei Aufzügen}|pw}« bereits aufgeführt worden
               iſt. Man hat es in Wien\oindex{Wien@\textbf{Wien}, \emph{A.ADM2}|pw}, im Deutſchen Volkstheater\oindex{Volkstheater@\textbf{Volkstheater}, \emph{Theater (K.THE)}|pw} gegeben. Die zwei erſten Akte gefielen;
               der dritte misfiel ſo gründlich, daſs er das ganze Stück mitriſs. Insbeſondere
               ſcheint man über die moraliſchen Qualitäten des Stückes wenig erbaut geweſen zu ſein;
               – ein Kritiker\pwindex{Granichstaedten, Emil 1847-07-08 – 1904-07-02@\textsc{Granichstaedten, Emil} (1847-07-08 – 1904-07-02), \emph{Journalist/Journalistin, Rechtswissenschaftler/Rechtswissenschaftlerin}|pwv} rief mir zu:
                  »\label{K_L00336-1v}\edtext{Um {\pb}Reinlichkeit wird gebeten\pwindex{Feuilleton. Deutsches Volkstheater [Maerchen]@\emph{Feuilleton. Deutsches Volkstheater [Märchen]}|pwv}}{\lemma{\textnormal{\emph{Um … gebeten}}}\Cendnote{\textnormal{Emil Granichstaedten\pwindex{Granichstaedten, Emil 1847-07-08 – 1904-07-02@\textsc{Granichstaedten, Emil} (1847-07-08 – 1904-07-02), \emph{Journalist/Journalistin, Rechtswissenschaftler/Rechtswissenschaftlerin}|pwk}: \emph{Deutsches Volkstheater}\pwindex{Feuilleton. Deutsches Volkstheater [Maerchen]@\emph{Feuilleton. Deutsches Volkstheater [Märchen]}|pwk}. In: \emph{Die Presse}\pwindex{Presse@\emph{Die Presse}|pwk}, Jg. 46, Nr. 334, 3. 12. 1893, S. 1–2, hier
                     S. 2.}}}\label{K_L00336-1}«; ein anderer\pwindex{–r– @\textsc{–r–}, \emph{Theaterkritiker/Theaterkritikerin}|pwv}{ }ſprach geradezu von der »\label{K_L00336-2v}\edtext{wahrhaft erſchreckenden ſittlichen
                  Verwahrloſung\pwindex{Theater und Kunst [Urauffuehrung Das Maerchen]@\emph{Theater und Kunst [Uraufführung Das Märchen]}|pwv}}{\lemma{\textnormal{\emph{wahrhaft … Verwahrloſung}}}\Cendnote{\textnormal{–r–\pwindex{–r– @\textsc{–r–}, \emph{Theaterkritiker/Theaterkritikerin}|pwk}: \emph{(Deutsches Volkstheater.)}\pwindex{Theater und Kunst [Urauffuehrung Das Maerchen]@\emph{Theater und Kunst [Uraufführung Das Märchen]}|pwk} In: \emph{Das
                        Vaterland}\pwindex{Vaterland@\emph{Das Vaterland}|pwk}, Jg. 34, Nr. 333, 2. 12. 1893,
                  S. 7.}}}\label{K_L00336-2}«, von der das Schauſpiel Zeugnis gebe. Eine \label{K_L00336-3v}\edtext{Berliner Bühne\oindex{Lessing-Theater@\textbf{Lessing-Theater}, \emph{Theater (K.THE)}|pwv}}{\lemma{\textnormal{\emph{Berliner Bühne}}}\Cendnote{\textnormal{Das \emph{Lessing-Theater}\orgindex{Lessing-Theater@Lessing-Theater|pwk} hatte \emph{Das Märchen}\pwindex{Maerchen. Schauspiel in drei Aufzuegen@\emph{Das Märchen. Schauspiel in drei Aufzügen}|pwk} bereits im Dezember 1891
                  angenommen.}}}\label{K_L00336-3}, die das Märchen\pwindex{Maerchen. Schauspiel in drei Aufzuegen@\emph{Das Märchen. Schauspiel in drei Aufzügen}|pw}{ }ſchon angeno{\geminationm}en hatte,
               trat auf den Wien\oindex{Wien@\textbf{Wien}, \emph{A.ADM2}|pw}er Miserfolg hin von \substVorne{}\textsuperscript{ſeiner}\substDazwischen{}ihrer\substHinten{} Verpflichtung zurück, und ſomit ka{\geminationn} ich wohl
               die Bühnenlaufbahn dieſes Stückes als abgeſchlossen anſehn. – Ich {\pb}habe mich beinahe verpflichtet gefühlt, Ihnen
               dieſe äußern Umſtände mitzutheilen, die mich anfangs wohl verſtimmt haben, die ich
               aber bald als das betrachten konnte, was ſie ſind – als \uline{äußere} Umſtände. –\pend
           
\pstart
           Nochmals, hochverehrter Herr, bitte ich Sie meiner tiefſten Dankbarkeit und
               meiner unveränderlichen Bewunderung verſichert zu ſein,{\\[\baselineskip]}\spacefill\mbox{Arthur Schnitzler}\pend
           \leftskip=0em{}\selectlanguage{ngerman}\endnumbering\briefempfaengerindex{Brandes, Georg@\textsc{Brandes, Georg}!zzzSchnitzler, Arthur@\emph{von Arthur Schnitzler}!1894-06-121@{12. 6. 1894}|)be}\mylabel{L00336h}  \normalsize

\doendnotes{C}
\bigskip
\vfill

\clearpage

\footnotesize

\lohead{\textsc{register}}

% Definiere theindex-Environment komplett neu ohne reledmac
\makeatletter
\renewenvironment{theindex}{%
  \section*{\indexname}%
  \setlength{\parindent}{0pt}%
  \setlength{\parskip}{0pt plus 0.3pt}%
  \let\item\@idxitem
}{%
  \clearpage
}
\makeatother

\IfFileExists{\jobname-pw.ind}{\input{\jobname-pw.ind}}{}

\end{document}

      