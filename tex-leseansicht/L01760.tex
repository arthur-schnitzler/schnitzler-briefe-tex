%% latex-leseansicht-vorspann.tex
%% Vorspann für die Leseansicht.
%% Lädt die gemeinsame Datei latex-vorspann.tex mit nicht gesetztem Schalter.

\newif\ifkorrekturansicht
\korrekturansichtfalse

\input{../tex-inputs/latex-vorspann}

\begin{center}
            \textcolor{red}{ENTWURF. ENTZIFFERUNG NOCH NICHT KORREKTURGELESEN}
                      \end{center}
            
               \section[Arthur und Olga Schnitzler an Richard Beer-Hofmann, 13. 2. 1908]{ Arthur und Olga Schnitzler an Richard Beer-Hofmann,
                    13. 2. 1908}\nopagebreak\mylabel{v}\rehead{ }\begin{ledgroupsized}[t]{13cm}\normalsize\beginnumbering\briefempfaengerindex{Beer-Hofmann, Richard@\textsc{Beer-Hofmann, Richard}!zzzSchnitzler, Olga@\emph{von Olga Schnitzler}!1908-02-131@{13. 2. 1908}|(be}\briefempfaengerindex{Beer-Hofmann, Richard@\textsc{Beer-Hofmann, Richard}!zzzSchnitzler, Arthur@\emph{von Arthur Schnitzler}!1908-02-131@{13. 2. 1908}|(be} \toendnotes[C]{\smallbreak\pagebreak[2]} \Standort{YCGL, MSS 31.}
\physDesc{Bildpostkarte
\newline{}Handschrift Arthur Schnitzler: schwarze Tinte, deutsche Kurrent\newline{}Handschrift Olga Schnitzler: schwarze Tinte, lateinische Kurrent\newline{}Versand: Stempel: »\nobreak{}\oindex{Semmering@\textbf{Semmering}|pwk}Semmering, 13. II. 09, 6\nobreak{}«.  }\buchAbdrucke{\weitereDrucke{Arthur Schnitzler, Richard Beer-Hofmann: \emph{Briefwechsel 1891–1931}. Hg. Konstanze Fliedl. Wien, Zürich: \emph{Europaverlag} 1992, S. 188.} }\pstart{}{\pb}Hrn\pend{}\pstart{}\textsc{Dr. Richard}\pend{}\pstart{}\textsc{Beer-Hofmann}\pend{}\pstart{}Wien XVIII\oindex{XVIII., Waehring@\textbf{XVIII., Währing}|pw}\pend{}\pstart{}\textsc{Hasenauerstraße 59}\oindex{Hasenauerstrasse@\textbf{Hasenauerstraße}|pw}.\pend{}{\bigskip}\pstart
           \noindent{}\centering{}\textcolor{gray}{\textbf{{\pb}Semmering\oindex{Semmering@\textbf{Semmering}|pw}.{ }Südbahnhotel\oindex{Suedbahnhotel@\textbf{Südbahnhotel}|pw}.}}\pend
           \pstart
           \raggedleft{}{\pb}13. 2. 908\pend
           \pstart
           Es iſt jetzt ſo ſchön hier, daſs ichs Ihnen doch noch einmal ſagen muſs. Wir
                    bleiben noch etwa 8 Tage.\pend
           \pstart
           Niemand mehr da (außer Grethe Kainz\pwindex{Kainz, Margarethe 13.12.1858 – 12.02.1950@\textsc{Kainz, Margarethe} (13.12.1858 – 12.02.1950), \emph{Schauspielerin}|pw}.)
                    Entschließen Sie ſich und ko{\geminationm}en mit Paula\pwindex{Beer-Hofmann, Paula 25.02.1879 – 30.10.1939@\textsc{Beer-Hofmann, Paula} (25.02.1879 – 30.10.1939)|pw}.\pend
           \pstart Herzlichſt Ihr \spacefill\mbox{Arthur}\pend{}\pstart
           \noindent{}{[}hs. O. Schnitzler:{]} Es ist unglaublich schön, ich möchte gar nicht nach
                    Hause kommen. Ihr seid Euch sehr im Licht, wenn Ihr nicht kommt. (Strahlendes
                    Sonnenlicht.)\pend
           \pstart  Herzlichste Grüsse, \spacefill\mbox{Olga.}\pend{}\endnumbering\briefempfaengerindex{Beer-Hofmann, Richard@\textsc{Beer-Hofmann, Richard}!zzzSchnitzler, Olga@\emph{von Olga Schnitzler}!1908-02-131@{13. 2. 1908}|)be}\briefempfaengerindex{Beer-Hofmann, Richard@\textsc{Beer-Hofmann, Richard}!zzzSchnitzler, Arthur@\emph{von Arthur Schnitzler}!1908-02-131@{13. 2. 1908}|)be}\mylabel{h}\end{ledgroupsized}  \newcommand{\dateiname}{L01760}\newcommand{\titel}{Arthur und Olga Schnitzler an Richard Beer-Hofmann, 13. 2. 1908}\newcommand{\editorInnen}{Martin Anton Müller und Gerd-Hermann Susen}%% latex-leseansicht-abspann.tex
%% Abspann für die Leseansicht.
%% Der Schalter \ifkorrekturansicht ist bereits durch den Vorspann gesetzt.

%% latex-abspann.tex
%% Gemeinsamer Abspann für Korrekturansicht und Leseansicht.
%% Setzt den Schalter \ifkorrekturansicht voraus (gesetzt in den
%% einbindenden Dateien latex-korrekturansicht-abspann.tex bzw.
%% latex-leseansicht-abspann.tex).
%% ---------------------------------------------------------------

\normalsize

% Das esempio-Environment wird nur in der Leseansicht benötigt
\ifkorrekturansicht\else
\newenvironment{esempio}[3]%
{
    \vspace{1.5ex}
    \rlap{\underline{#1}}
    \par
    \setlength{\parindent}{0cm}
    \nopagebreak
    \leftskip=#2cm
    \rightskip=#3cm
}
{
    \par
}
\fi

\doendnotes{C}
\bigskip
\vfill

\clearpage

\footnotesize

\ifkorrekturansicht
  \lohead{\textsc{register}}
\fi

% theindex-Environment neu definieren ohne reledmac
\makeatletter
\renewenvironment{theindex}{%
  \ifkorrekturansicht
    \section*{\indexname}%
  \else
    \subsubsection*{Index der erwähnten Entitäten}%
  \fi
  \setlength{\parindent}{0pt}%
  \setlength{\parskip}{0pt plus 0.3pt}%
  \let\item\@idxitem
}{%
  \ifkorrekturansicht\clearpage\fi
}
\makeatother

\IfFileExists{\jobname-pw.ind}{\input{\jobname-pw.ind}}{}

% Quellenangabe nur in der Leseansicht
\ifkorrekturansicht\else
% Fallback-Definitionen, falls die .tex-Datei \titel etc. nicht gesetzt hat
\providecommand{\titel}{}
\providecommand{\editorInnen}{}
\providecommand{\dateiname}{\jobname}

\vspace{3cm}

\vfill

\footnotesize
\textsc{Quelle}: \titel. Herausgegeben von {\editorInnen}. In: \emph{Arthur Schnitzler: Briefwechsel mit Autorinnen und Autoren}.
 Digitale Edition, https://schnitzler-briefe.acdh.oeaw.ac.at/{\dateiname}.html (Stand \today)
\fi

\end{document}


      