%% latex-korrekturansicht-vorspann.tex
%% Vorspann für die Korrekturansicht.
%% Lädt die gemeinsame Datei latex-vorspann.tex mit gesetztem Schalter.

\newif\ifkorrekturansicht
\korrekturansichttrue

\input{../tex-inputs/latex-vorspann}


\section[Arthur Schnitzler an Richard Beer-Hofmann, 17. 7. 1906]{L01613 Arthur Schnitzler an Richard Beer-Hofmann, 17. 7. 1906}
\nopagebreak\mylabel{L01613v}
\rehead{ }\normalsize\beginnumbering\briefempfaengerindex{Beer-Hofmann, Richard@\textsc{Beer-Hofmann, Richard}!zzzSchnitzler, Arthur@\emph{von Arthur Schnitzler}!1906-07-171@{17. 7. 1906}|(be}
\toendnotes[C]{\smallbreak\pagebreak[2]}\Standort{YCGL, MSS 31.}
\physDesc{Bildpostkarte, 444 Zeichen
\newline{}Handschrift: Bleistift, deutsche Kurrent
\newline{}Versand: Stempel: »\nobreak{}\oindex{Helsingør@\textbf{Helsingør}, \emph{P.PPLA2}|pwk}Hels{[}ingør{]}, 17. 7. 06, 2–5E\nobreak{}«.  
\newline{}Ordnung: mit Bleistift von unbekannter Hand datiert:
                                    »17. 7.« }
\buchAbdrucke{\weitereDrucke{Arthur Schnitzler, Richard Beer-Hofmann: \emph{Briefwechsel 1891–1931}. Wien, Zürich: \emph{Europaverlag} 1992, S. 179.} }\pstart{}{\pb}\textsc{Dr. Richard}\pend{}\pstart{}\textsc{Beer-Hofmann}\pend{}\pstart{}\textsc{Rodaun}\oindex{Rodaun@\textbf{Rodaun}, \emph{A.ADM4}|pw}\pend{}\pstart{}\textsc{bei Wien\oindex{Wien@\textbf{Wien}, \emph{A.ADM2}|pw}}. \pend{}\pstart{}\textsc{Liesingerstraße 1}\oindex{Liesingerstrasse@\textbf{Liesingerstraße}, \emph{Straße (K.STR)}|pw}.\pend{}\pstart{}Austria\oindex{Oesterreich@\textbf{Österreich}, \emph{A.PCLI}|pw}\pend{}{\bigskip}
\pstart
           \noindent{}{\pb}\textcolor{gray}{\textbf{Hamlets\pwindex{Hamlet@\emph{Hamlet}|pw} Statue}}\pend
           \vspace{1em}
\pstart
           \noindent{}{\pb}Ihre Heiligenkreuz\oindex{Heiligenkreuz@\textbf{Heiligenkreuz}, \emph{A.ADM3}|pw}erkarte beko{\geminationm}en. Herzlichen Dank u
               Gruſs. Wohin gehn Sie? Oder bleiben Sie? Wir dürften noch 3 Wochen hier verweilen;
               ich arbeite. Sind mit Natur, Kunſt und Küche ſehr zufrieden. Heini\pwindex{Schnitzler, Heinrich 09.08.1902 – 12.07.1982@\textsc{Schnitzler, Heinrich} (09.08.1902 – 12.07.1982), \emph{Regisseur/Regisseurin, Schauspieler/Schauspielerin}|pw} zeichnet täglich einen Block voll. Laſſen Sie was von
               ſich hören. Eine Karte, auf der die Hälfte von mir iſt, genügt mir nicht. Herzlichſt
               Ihr \spacefill\mbox{A.}\pend
           
\pstart
           \noindent{}{\pb}Viele Grüße von uns allen an Sie alle.\pend
           \selectlanguage{ngerman}\endnumbering\briefempfaengerindex{Beer-Hofmann, Richard@\textsc{Beer-Hofmann, Richard}!zzzSchnitzler, Arthur@\emph{von Arthur Schnitzler}!1906-07-171@{17. 7. 1906}|)be}\mylabel{L01613h}  \normalsize

\doendnotes{C}
\bigskip
\vfill

\clearpage

\footnotesize

\lohead{\textsc{register}}

% Definiere theindex-Environment komplett neu ohne reledmac
\makeatletter
\renewenvironment{theindex}{%
  \section*{\indexname}%
  \setlength{\parindent}{0pt}%
  \setlength{\parskip}{0pt plus 0.3pt}%
  \let\item\@idxitem
}{%
  \clearpage
}
\makeatother

\IfFileExists{\jobname-pw.ind}{\input{\jobname-pw.ind}}{}

\end{document}

      