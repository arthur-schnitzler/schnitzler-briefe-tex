%% latex-leseansicht-vorspann.tex
%% Vorspann für die Leseansicht.
%% Lädt die gemeinsame Datei latex-vorspann.tex mit nicht gesetztem Schalter.

\newif\ifkorrekturansicht
\korrekturansichtfalse

\input{../tex-inputs/latex-vorspann}


\section[Hugo von Hofmannsthal an Arthur Schnitzler, 5. 8. {[}1912{]}]{L02083 Hugo von Hofmannsthal an Arthur Schnitzler, 5. 8. [1912]}
\nopagebreak\mylabel{L02083v}
\rehead{ }\normalsize\beginnumbering\briefempfaengerindex{Schnitzler, Arthur@\textsc{Schnitzler, Arthur}!zzzHofmannsthal, Hugo von@\emph{von Hugo von Hofmannsthal}!1912-08-052@{5. 8. [1912]}|(be}
\toendnotes[C]{\smallbreak\pagebreak[2]}
\correspDesc{Versand  durch Hugo von Hofmannsthal am 5. 8. [1912] in Bad Aussee
\newline{}Erhalt  durch Arthur Schnitzler im Zeitraum [6. 8. 1912
                  – 10. 8. 1912?] in Wien}\toendnotes[C]{\smallbreak}
\Standort{CUL, Schnitzler, B 43.}
\physDesc{Briefkarte, 774 Zeichen
\newline{}Handschrift: schwarze Tinte, deutsche Kurrent
\newline{}Schnitzler: mit Bleistift die Jahreszahl ergänzt: »912« 
\newline{}Ordnung: 1) mit Bleistift von unbekannter Hand nummeriert: »\strikeout{329}«  2) mit Bleistift von unbekannter Hand nummeriert:
                                    »339«}
\buchAbdrucke{\weitereDrucke{Hugo von Hofmannsthal, Arthur Schnitzler: \emph{Briefwechsel}. Herausgegeben von Therese Nickl und Heinrich Schnitzler. Frankfurt am Main: \emph{S. Fischer} 1964, S. 268.} }\toendnotes[C]{\smallbreak}
\pstart
           \raggedleft{}{\pb}5 VIII.{ }\textsc{Aussee}\oindex{Bad Aussee@\textbf{Bad Aussee}, \emph{Hauptstadt}|pw}.\pend
           
\pstart{}mein lieber Arthur\pend\vspace{0.5em}
\pstart
           ich bin froh, aus Ihren Karten zu{ }ſehen daſs es Euch gut geht. Uns gehts auch gut.
               Mir iſt dieſe Landſchaft die{ }ſchönſte und liebſte, und daſs hie und da Leute{ }ſind,
               die man kennt, tut mir auch nichts, man iſt dennoch{ }ſo viel allein und{ }ſo meilenweit
               von ihnen als man will. Mir iſt{ }ſchon {\pb}Jahre lang nicht{ }ſo viel und
               vielerlei eingefallen, macht man auch nicht alles{ }ſo iſt das Einfallen doch ein
               großes Vergnügen.\pend
           
\pstart
           Unter andern Büchern les ich den Varnhagen\pwindex{Varnhagen-Ense, Karl August von 21.\,2.\,1785 Düsseldorf – 10.\,10.\,1858 Berlin@\textsc{Varnhagen-Ense, Karl August von} (21.\,2.\,1785 Düsseldorf – 10.\,10.\,1858 Berlin), \emph{Schriftsteller, Diplomat}|pw}\pwindex{Varnhagen-Ense, Karl August von 21.\,2.\,1785 Düsseldorf – 10.\,10.\,1858 Berlin@\textsc{Varnhagen-Ense, Karl August von} (21.\,2.\,1785 Düsseldorf – 10.\,10.\,1858 Berlin), \emph{Schriftsteller, Diplomat}!Tagebücher@\strich\emph{Tagebücher}|pwv}, finde ihn äuſſerſt intereſſant.\hspace*{1.5em}Kommt doch im
                  September hier vorbei, ich{ }ſag wieder mein Sprücherl: man wird auf
               einmal todt{ }ſein und dann wird einem \uline{sehr} leid{ }ſein
               daſs man{ }ſich nicht öfter geſehen hat. \label{T_L02083-1v}\edtext{Schreiben Sie wieder einmal ein kleines Karterl}{\lemma{\textnormal{\emph{Schreiben … Karterl}}}\Cendnote{\textnormal{quer am linken Rand}}}\label{T_L02083-1}.\pend
           \pstart Ihr \spacefill\mbox{Hugo}\pend{}
\pstart
           \noindent{}\label{T_L02083-2v}\edtext{Viele Grüße Olga\pwindex{Schnitzler, Olga 17.\,1.\,1882 Wien – 13.\,1.\,1970 Lugano@\textsc{Schnitzler, Olga} (17.\,1.\,1882 Wien – 13.\,1.\,1970 Lugano), \emph{Schauspielerin, Sängerin}|pw} von uns beiden.}{\lemma{\textnormal{\emph{Viele … beiden.}}}\Cendnote{\textnormal{quer am rechten Rand der ersten Seite}}}\label{T_L02083-2}\pend
           \selectlanguage{ngerman}\endnumbering\briefempfaengerindex{Schnitzler, Arthur@\textsc{Schnitzler, Arthur}!zzzHofmannsthal, Hugo von@\emph{von Hugo von Hofmannsthal}!1912-08-052@{5. 8. [1912]}|)be}\mylabel{L02083h}  \newcommand{\dateiname}{L02083}\newcommand{\titel}{Hugo von Hofmannsthal an Arthur Schnitzler, 5. 8. [1912]}\newcommand{\editorInnen}{Martin Anton Müller und Gerd-Hermann Susen}%% latex-leseansicht-abspann.tex
%% Abspann für die Leseansicht.
%% Der Schalter \ifkorrekturansicht ist bereits durch den Vorspann gesetzt.

%% latex-abspann.tex
%% Gemeinsamer Abspann für Korrekturansicht und Leseansicht.
%% Setzt den Schalter \ifkorrekturansicht voraus (gesetzt in den
%% einbindenden Dateien latex-korrekturansicht-abspann.tex bzw.
%% latex-leseansicht-abspann.tex).
%% ---------------------------------------------------------------

\normalsize

% Das esempio-Environment wird nur in der Leseansicht benötigt
\ifkorrekturansicht\else
\newenvironment{esempio}[3]%
{
    \vspace{1.5ex}
    \rlap{\underline{#1}}
    \par
    \setlength{\parindent}{0cm}
    \nopagebreak
    \leftskip=#2cm
    \rightskip=#3cm
}
{
    \par
}
\fi

\doendnotes{C}
\bigskip
\vfill

\clearpage

\footnotesize

\ifkorrekturansicht
  \lohead{\textsc{register}}
\fi

% theindex-Environment neu definieren ohne reledmac
\makeatletter
\renewenvironment{theindex}{%
  \ifkorrekturansicht
    \section*{\indexname}%
  \else
    \subsubsection*{Index der erwähnten Entitäten}%
  \fi
  \setlength{\parindent}{0pt}%
  \setlength{\parskip}{0pt plus 0.3pt}%
  \let\item\@idxitem
}{%
  \ifkorrekturansicht\clearpage\fi
}
\makeatother

\IfFileExists{\jobname-pw.ind}{\input{\jobname-pw.ind}}{}

% Quellenangabe nur in der Leseansicht
\ifkorrekturansicht\else
% Fallback-Definitionen, falls die .tex-Datei \titel etc. nicht gesetzt hat
\providecommand{\titel}{}
\providecommand{\editorInnen}{}
\providecommand{\dateiname}{\jobname}

\vspace{3cm}

\vfill

\footnotesize
\textsc{Quelle}: \titel. Herausgegeben von {\editorInnen}. In: \emph{Arthur Schnitzler: Briefwechsel mit Autorinnen und Autoren}.
 Digitale Edition, https://schnitzler-briefe.acdh.oeaw.ac.at/{\dateiname}.html (Stand \today)
\fi

\end{document}


