%% latex-korrekturansicht-vorspann.tex
%% Vorspann für die Korrekturansicht.
%% Lädt die gemeinsame Datei latex-vorspann.tex mit gesetztem Schalter.

\newif\ifkorrekturansicht
\korrekturansichttrue

\input{../tex-inputs/latex-vorspann}


\section[ Arthur Schnitzler an Felix Salten, 11. 4. 1928]{L03024 Arthur Schnitzler an Felix Salten, 11. 4. 1928}
\nopagebreak\mylabel{L03024v}
\rehead{ }\normalsize\beginnumbering\briefempfaengerindex{Salten, Felix@\textsc{Salten, Felix}!zzzSchnitzler, Arthur@\emph{von Arthur Schnitzler}!1928-04-112@{11. 4. 1928}|(be}
\toendnotes[C]{\smallbreak\pagebreak[2]}\Standort{Wienbibliothek im Rathaus, ZPH 1681, 2.1.516.}
\physDesc{Brief, 1 Blatt, 2 Seiten, 703 Zeichen
\newline{}Handschrift: Bleistift, lateinische Kurrent
\newline{}Ordnung: mit Bleistift von unbekannter Hand nummeriert: »2« }
\buchAbdrucke{\weitereDrucke{Arthur Schnitzler: \emph{Briefe 1913–1931}. Frankfurt am Main: \emph{S. Fischer} 1984, S. 541–542.} }\toendnotes[C]{\smallbreak}
\pstart
           \raggedleft{}{\pb}Wien\oindex{Wien@\textbf{Wien}, \emph{A.ADM2}|pw}{ }11. 4. 928\pend
           \vspace{0.5em}
\pstart
           lieber, der \label{K_L03024-1v}\edtext{Schrei der Liebe\pwindex{Quer durch den Wurstelprater@\emph{Quer durch den Wurstelprater}|pw}}{\lemma{\textnormal{\emph{Schrei der Liebe}}}\Cendnote{\textnormal{Vgl. Felix Salten: Widmungsexemplar Der Schrei der Liebe für Arthur
               Schnitzler, Juli 1928.
               }}}\label{K_L03024-1} ist vorläufg unauffindbar – (ich merke eben, dſs mir auch der \label{K_L03024-2v}\edtext{Wurstlprater\pwindex{Quer durch den Wurstelprater@\emph{Quer durch den Wurstelprater}|pw}}{\lemma{\textnormal{\emph{Wurstlprater}}}\Cendnote{\textnormal{Vgl. Felix Salten: Widmungsexemplar Wurstelprater für Arthur
               Schnitzler, 12. 12. 1911.
               }}}\label{K_L03024-2} verschwunden ist) – doch steht ein großes Reinemachen und Bücherklopfen bevor
               – da wird er sich hoffentlich finden. Und we{\geminationn} da nicht,
               im Mai, wo neue Regale kommen und ich überhaupt eine
               »ordentliche Ordnung« machen will. Ich zweifle nicht, daſs die Bücher\pwindex{Quer durch den Wurstelprater@\emph{Quer durch den Wurstelprater}|pw}\pwindex{Quer durch den Wurstelprater@\emph{Quer durch den Wurstelprater}|pw} in meiner Bibliothek vorhanden sind, de{\geminationn} Widmungsexemplare, und gar von Ihnen, leih ich nicht
               her.\pend
           
\pstart
           Morgen fahr ich nach Triest\oindex{Triest@\textbf{Triest}, \emph{A.ADM3}|pw}, und Samstag mit der Stella d’Italia\orgindex{Stella DItalia@Stella d’Italia|pw}{ }{\pb}in Begleitung von Lili\pwindex{Cappellini, Lili 13.09.1909 – 26.07.1928@\textsc{Cappellini, Lili} (13.09.1909 – 26.07.1928)|pw} und ihrem Gatten\pwindex{Cappellini, Arnoldo 10.08.1889 – 08.05.1954@\textsc{Cappellini, Arnoldo} (10.08.1889 – 08.05.1954)|pwv} über Athen\oindex{Athen@\textbf{Athen}, \emph{P.PPLC}|pw} – Konstantinopel\oindex{Istanbul@\textbf{Istanbul}, \emph{A.ADM1}|pw} und zurück (über Rhodus\oindex{Rhodos@\textbf{Rhodos}, \emph{P.PPLA2}|pw}, das es also zu geben scheint.)\pend
           
\pstart
           Auf ein gutes \label{K_L03024-3v}\edtext{Wiedersehn im Mai}{\lemma{\textnormal{\emph{Wiedersehn im Mai}}}\Cendnote{\textnormal{Das nächste nachweisbare Treffen fand
                  am 18. 5. 1928
                  statt.}}}\label{K_L03024-3}, u alles herzliche bis dahin {\\[\baselineskip]}Ihr {\\[\baselineskip]}\spacefill\mbox{Arth}\pend
           \leftskip=0em{}\selectlanguage{ngerman}\endnumbering\briefempfaengerindex{Salten, Felix@\textsc{Salten, Felix}!zzzSchnitzler, Arthur@\emph{von Arthur Schnitzler}!1928-04-112@{11. 4. 1928}|)be}\mylabel{L03024h}  \normalsize

\doendnotes{C}
\bigskip
\vfill

\clearpage

\footnotesize

\lohead{\textsc{register}}

% Definiere theindex-Environment komplett neu ohne reledmac
\makeatletter
\renewenvironment{theindex}{%
  \section*{\indexname}%
  \setlength{\parindent}{0pt}%
  \setlength{\parskip}{0pt plus 0.3pt}%
  \let\item\@idxitem
}{%
  \clearpage
}
\makeatother

\IfFileExists{\jobname-pw.ind}{\input{\jobname-pw.ind}}{}

\end{document}

      