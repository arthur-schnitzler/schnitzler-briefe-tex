%% latex-leseansicht-vorspann.tex
%% Vorspann für die Leseansicht.
%% Lädt die gemeinsame Datei latex-vorspann.tex mit nicht gesetztem Schalter.

\newif\ifkorrekturansicht
\korrekturansichtfalse

\input{../tex-inputs/latex-vorspann}


         
         \renewcommand{\erwaehntePersonen}{Personen: Lili Cappellini, Arnoldo Cappellini, Felix Salten}
         \renewcommand{\erwaehnteInstitutionen}{Institutionen: Stella d’Italia}
         \renewcommand{\erwaehnteOrte}{Orte: Athen, Istanbul, Rhodos, Triest, Wien}
         \renewcommand{\erwaehnteWerke}{Werke: Quer durch den Wurstelprater}
               \section[ Arthur Schnitzler an Felix Salten, 11. 4. 1928]{ Arthur Schnitzler an Felix Salten, 11. 4. 1928}\nopagebreak\mylabel{v}\rehead{ }\begin{ledgroupsized}[t]{13cm}\normalsize\beginnumbering\briefempfaengerindex{Salten, Felix@\textsc{Salten, Felix}!zzzSchnitzler, Arthur@\emph{von Arthur Schnitzler}!1928-04-112@{11. 4. 1928}|(be} \toendnotes[C]{\smallbreak\pagebreak[2]} \Standort{Wienbibliothek im Rathaus, ZPH 1681, 2.1.516.}
\physDesc{Brief, 1 Blatt, 2 Seiten, 702 Zeichen
\newline{}Handschrift: Bleistift, lateinische Kurrent
\newline{}Ordnung: mit Bleistift von unbekannter Hand nummeriert: »2« }\buchAbdrucke{\weitereDrucke{Arthur Schnitzler: \emph{Briefe 1913–1931}. Hg. Peter Michael Braunwarth, Richard Miklin, Susanne Pertlik und Heinrich Schnitzler. Frankfurt am Main: \emph{S. Fischer} 1984, S. 541–542.} }\toendnotes[C]{\smallbreak}\pstart
           \raggedleft{}{\pb}Wien\oindex{Wien@\textbf{Wien}|pw}{ }11. 4. 928\pend
           \pstart
           lieber, der \label{K_L03024-1v}\edtext{Schrei der Liebe\pwindex{Salten, Felix 06.09.1869 – 08.10.1945@\textsc{Salten, Felix} (06.09.1869 – 08.10.1945), \emph{Schriftsteller, Journalist, Chefredakteur}!Quer durch den Wurstelprater1895-06-02 – 1895-06-09@\strich\emph{Quer durch den Wurstelprater} {[}1895-06-02 – 1895-06-09{]}|pw}}{\lemma{\textnormal{\emph{Schrei der Liebe}}}\Cendnote{\textnormal{vgl. Felix Salten: Widmungsexemplar Der Schrei der Liebe für Arthur
               Schnitzler, Juli 1928}}}\label{K_L03024-1h} ist vorläufg unauffindbar – (ich merke eben, dſs mir auch der \label{K_L03024-2v}\edtext{Wurstlprater\pwindex{Salten, Felix 06.09.1869 – 08.10.1945@\textsc{Salten, Felix} (06.09.1869 – 08.10.1945), \emph{Schriftsteller, Journalist, Chefredakteur}!Quer durch den Wurstelprater1895-06-02 – 1895-06-09@\strich\emph{Quer durch den Wurstelprater} {[}1895-06-02 – 1895-06-09{]}|pw}}{\lemma{\textnormal{\emph{Wurstlprater}}}\Cendnote{\textnormal{vgl. Felix Salten: Widmungsexemplar Wurstelprater für Arthur
               Schnitzler, 12. 12. 1911}}}\label{K_L03024-2h} verschwunden ist) – doch steht ein großes Reinemachen und Bücherklopfen bevor
               – da wird er sich hoffentlich finden. Und we{\geminationn} da nicht,
               im Mai, wo neue Regale kommen und ich überhaupt eine
               »ordentliche Ordnung« machen will. Ich zweifle nicht, daſs die Bücher\pwindex{Salten, Felix 06.09.1869 – 08.10.1945@\textsc{Salten, Felix} (06.09.1869 – 08.10.1945), \emph{Schriftsteller, Journalist, Chefredakteur}!Quer durch den Wurstelprater1895-06-02 – 1895-06-09@\strich\emph{Quer durch den Wurstelprater} {[}1895-06-02 – 1895-06-09{]}|pw}\pwindex{Salten, Felix 06.09.1869 – 08.10.1945@\textsc{Salten, Felix} (06.09.1869 – 08.10.1945), \emph{Schriftsteller, Journalist, Chefredakteur}!Quer durch den Wurstelprater1895-06-02 – 1895-06-09@\strich\emph{Quer durch den Wurstelprater} {[}1895-06-02 – 1895-06-09{]}|pw} in meiner Bibliothek vorhanden sind, de{\geminationn} Widmungsexemplare, und gar von Ihnen, leih ich nicht
               her.\pend
           \pstart
           Morgen fahr ich nach Triest\oindex{Triest@\textbf{Triest}|pw}, und Samstag mit der Stella d’Italia\orgindex{Stella DItalia@Stella d’Italia|pw}{ }{\pb}in Begleitung von Lili\pwindex{Cappellini, Lili 13.09.1909 – 26.07.1928@\textsc{Cappellini, Lili} (13.09.1909 – 26.07.1928)|pw} und ihrem Gatten\pwindex{Cappellini, Arnoldo 10.08.1889 – 08.05.1954@\textsc{Cappellini, Arnoldo} (10.08.1889 – 08.05.1954)|pwv} über Athen\oindex{Athen@\textbf{Athen}|pw} – Konstantinopel\oindex{Istanbul@\textbf{Istanbul}|pw} und zurück (über Rhodus\oindex{Rhodos@\textbf{Rhodos}|pw}, das es also zu geben scheint.)\pend
           \pstart
           Auf ein gutes \label{K_L03024-3v}\edtext{Wiedersehn im Mai}{\lemma{\textnormal{\emph{Wiedersehn im Mai}}}\Cendnote{\textnormal{Das nächste nachweisbare Treffen fand
                  am 18. 5. 1928
                  statt.}}}\label{K_L03024-3h}, u alles herzliche bis dahin {\\[\baselineskip]}Ihr {\\[\baselineskip]}\spacefill\mbox{Arth}\pend
           \leftskip=0em{}
         
         \endnumbering\mylabel{h}\end{ledgroupsized}  \newcommand{\dateiname}{L03024}\newcommand{\titel}{Arthur Schnitzler an Felix Salten, 11. 4. 1928}\newcommand{\editorInnen}{Martin Anton Müller und Laura Untner}%% latex-leseansicht-abspann.tex
%% Abspann für die Leseansicht.
%% Der Schalter \ifkorrekturansicht ist bereits durch den Vorspann gesetzt.

%% latex-abspann.tex
%% Gemeinsamer Abspann für Korrekturansicht und Leseansicht.
%% Setzt den Schalter \ifkorrekturansicht voraus (gesetzt in den
%% einbindenden Dateien latex-korrekturansicht-abspann.tex bzw.
%% latex-leseansicht-abspann.tex).
%% ---------------------------------------------------------------

\normalsize

% Das esempio-Environment wird nur in der Leseansicht benötigt
\ifkorrekturansicht\else
\newenvironment{esempio}[3]%
{
    \vspace{1.5ex}
    \rlap{\underline{#1}}
    \par
    \setlength{\parindent}{0cm}
    \nopagebreak
    \leftskip=#2cm
    \rightskip=#3cm
}
{
    \par
}
\fi

\doendnotes{C}
\bigskip
\vfill

\clearpage

\footnotesize

\ifkorrekturansicht
  \lohead{\textsc{register}}
\fi

% theindex-Environment neu definieren ohne reledmac
\makeatletter
\renewenvironment{theindex}{%
  \ifkorrekturansicht
    \section*{\indexname}%
  \else
    \subsubsection*{Index der erwähnten Entitäten}%
  \fi
  \setlength{\parindent}{0pt}%
  \setlength{\parskip}{0pt plus 0.3pt}%
  \let\item\@idxitem
}{%
  \ifkorrekturansicht\clearpage\fi
}
\makeatother

\IfFileExists{\jobname-pw.ind}{\input{\jobname-pw.ind}}{}

% Quellenangabe nur in der Leseansicht
\ifkorrekturansicht\else
% Fallback-Definitionen, falls die .tex-Datei \titel etc. nicht gesetzt hat
\providecommand{\titel}{}
\providecommand{\editorInnen}{}
\providecommand{\dateiname}{\jobname}

\vspace{3cm}

\vfill

\footnotesize
\textsc{Quelle}: \titel. Herausgegeben von {\editorInnen}. In: \emph{Arthur Schnitzler: Briefwechsel mit Autorinnen und Autoren}.
 Digitale Edition, https://schnitzler-briefe.acdh.oeaw.ac.at/{\dateiname}.html (Stand \today)
\fi

\end{document}


      