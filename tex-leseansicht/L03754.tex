%% latex-korrekturansicht-vorspann.tex
%% Vorspann für die Korrekturansicht.
%% Lädt die gemeinsame Datei latex-vorspann.tex mit gesetztem Schalter.

\newif\ifkorrekturansicht
\korrekturansichttrue

\input{../tex-inputs/latex-vorspann}


\section[Arthur Schnitzler an Stefan Zweig, 6. 11. 1924]{L03754 Arthur Schnitzler an Stefan Zweig, 6. 11. 1924}
\nopagebreak\mylabel{L03754v}
\rehead{ }\normalsize\beginnumbering\briefempfaengerindex{Zweig, Stefan@\textsc{Zweig, Stefan}!zzzSchnitzler, Arthur@\emph{von Arthur Schnitzler}!1924-11-062@{6. 11. 1924}|(be}
\toendnotes[C]{\smallbreak\pagebreak[2]}\Standort{Jerusalem, National Library of Israel, ARC. Ms. Var. 305 1 58 Stefan Zweig Collection.}
\physDesc{Brief, 1 Blatt, 1 Seite, 1509 Zeichen
\newline{}Schreibmaschine
\newline{}Handschrift: Bleistift, lateinische Kurrent (\noindent{}minimale Korrekturen, Schlussformel, Unterschrift)}
\buchAbdrucke{\weitereDrucke{Arthur Schnitzler: \emph{Briefe 1913–1931}. Frankfurt am Main: \emph{S. Fischer} 1984, S. 372–373.} }\toendnotes[C]{\smallbreak}
\pstart
           {\pb}\textcolor{gray}{\textbf{D\textsuperscript{R} ARTHUR SCHNITZLER}}\hfill {\pb}6. 11. 1924.\pend
           
\pstart
           \textcolor{gray}{\textbf{WIEN, XVIII.
                           STERNWARTESTRASSE 71\oindex{Sternwartestrasse 71@\textbf{Sternwartestraße 71}, \emph{Wohngebäude (K.WHS)}|pw}.}}\pend
           
\pstart{}Lieber Herr Dr. Zweig.\pend\vspace{0.5em}
\pstart
           Es freut mich herzlich, dass Ihnen das »Fräulein
                  Else\pwindex{Fraeulein Else@\emph{Fräulein Else}|pw}« so wohlgefällt. Eine trouvaille ist es \strikeout{ja} eigentlich nicht, dieselbe
               Technik habe ich ja im »Leutnant Gustl\pwindex{Lieutenant Gustl. Novelle@\emph{Lieutenant Gustl. Novelle}|pw}« schon
               angewandt. Es ist eigentlich merkwürdig, dass sie seitdem so selten benützt wurde, da
               sie ganz ausserordentliche Möglichkeiten bietet. Freilich eignen sich nur wenige
               Sujets dazu, sonst hätte wahrscheinlich vor allem ich selbst von dieser Form öfters
               Gebrauch gemacht. Als der »Leutnant Gustl\pwindex{Lieutenant Gustl. Novelle@\emph{Lieutenant Gustl. Novelle}|pw}« neu
               war sagte man mir, dass in einer Novelle von Dujardin\pwindex{Dujardin, Edouard 10.10.1861 – 31.10.1949@\textsc{Dujardin, Édouard} (10.10.1861 – 31.10.1949), \emph{Schriftsteller/Schriftstellerin}|pw} »Les Lau\strikeout{r}riers sont coupé\strikeout{é}s\pwindex{lauriers sont coupes@\emph{Les lauriers sont coupés}|pw}«
               eine ähnliche Technik angewandt worden sei; die Angabe stimmte nicht ganz. Nach
                  \label{K_L03754-1v}\edtext{Georg Brandes\pwindex{Brandes, Georg 04.02.1842 – 19.02.1927@\textsc{Brandes, Georg} (04.02.1842 – 19.02.1927)|pw} sollte die »Krotkaja\pwindex{Sanfte@\emph{Die Sanfte}|pw}}{\lemma{\textnormal{\emph{Georg … »Krotkaja}}}\Cendnote{\textnormal{Siehe Georg Brandes an Arthur Schnitzler, 16. 6. 1901.
               }}}\label{K_L03754-1}« von Dostojewsky\pwindex{Dostojevskij, Fjodor Mihajlovic 11.11.1821 – 09.02.1881@\textsc{Dostojevskij, Fjodor Mihajlovič} (11.11.1821 – 09.02.1881), \emph{Schriftsteller/Schriftstellerin}|pw}
               sich der gleichen Technik bedienen, aber auch das trifft eigentlich nicht zu. \pend
           
\pstart
           Ihr Bedenken wegen der Summe kann ich wohl verstehen. Es ist schon möglich, dass ich,
               wie die übrigen österreichischen\oindex{Oesterreich@\textbf{Österreich}, \emph{A.PCLI}|pw} Millionäre in
               unserem Nullenwahnsinn \label{K_L03754-2v}\edtext{a priori}{\lemma{\textnormal{\emph{a priori}}}\Cendnote{\textnormal{lateinisch: von vornherein}}}\label{K_L03754-2} falsch eingestellt war; andererseits gebe ich ihnen
               zu erwägen, dass Dors\substVorne{}\textsuperscript{t}\substDazwischen{}d\substHinten{}ay immerhin an einem Bild achtzigtausend Gulden verdient hatte, was schon
               damals vorkam; ferner dass durch die Höhe der Summe auch seine Forderung für das
               Publikum gewissermassen entschuldbarer wird; – und endlich spielten gewisse
               persönliche Jugenderinnerungen in die finanzielle Partie meiner Novelle\pwindex{Fraeulein Else@\emph{Fräulein Else}|pwv} hinein, nach denen sich die von mir genannte
               Summe durchaus im Bereich des Wahrscheinlichen bewegt. \pend
           
\pstart
           Nochmals herzlichen Dank, viele Grüsse und auf baldiges Wiedersehen{\\[\baselineskip]}{[}hs.:{]} Ihr
                  \spacefill\mbox{Arthur Schnitzler}\pend
           \leftskip=0em{}\selectlanguage{ngerman}\endnumbering\briefempfaengerindex{Zweig, Stefan@\textsc{Zweig, Stefan}!zzzSchnitzler, Arthur@\emph{von Arthur Schnitzler}!1924-11-062@{6. 11. 1924}|)be}\mylabel{L03754h}
\begin{anhang}
\end{anhang}\normalsize

\doendnotes{C}
\bigskip
\vfill

\clearpage

\footnotesize

\lohead{\textsc{register}}

% Definiere theindex-Environment komplett neu ohne reledmac
\makeatletter
\renewenvironment{theindex}{%
  \section*{\indexname}%
  \setlength{\parindent}{0pt}%
  \setlength{\parskip}{0pt plus 0.3pt}%
  \let\item\@idxitem
}{%
  \clearpage
}
\makeatother

\IfFileExists{\jobname-pw.ind}{\input{\jobname-pw.ind}}{}

\end{document}

      