%% latex-leseansicht-vorspann.tex
%% Vorspann für die Leseansicht.
%% Lädt die gemeinsame Datei latex-vorspann.tex mit nicht gesetztem Schalter.

\newif\ifkorrekturansicht
\korrekturansichtfalse

\input{../tex-inputs/latex-vorspann}


\section[Arthur Schnitzler an Gustav Schwarzkopf, 9. 1. 1909]{L04152 Arthur Schnitzler an Gustav Schwarzkopf, 9. 1. 1909}
\nopagebreak\mylabel{L04152v}
\rehead{ }\normalsize\beginnumbering\briefempfaengerindex{Schwarzkopf, Gustav@\textsc{Schwarzkopf, Gustav}!zzzSchnitzler, Arthur@\emph{von Arthur Schnitzler}!1909-01-091@{9. 1. 1909}|(be}
\toendnotes[C]{\smallbreak\pagebreak[2]}
\correspDesc{Versand  durch Arthur Schnitzler am 9. 1. 1909 in Wien
\newline{}Erhalt  durch Gustav Schwarzkopf im Zeitraum [9. 1. 1909
                  – 12. 1. 1909?] in Wien}\toendnotes[C]{\smallbreak}
\Standort{CUL, Schnitzler, B 96.}
\physDesc{Brief, 1 Blatt, 1 Seite, 189 Zeichen
\newline{}Handschrift: schwarze Tinte, deutsche Kurrent}\toendnotes[C]{\smallbreak}
\pstart
           {\pb}\textcolor{gray}{\textbf{Dr. Arthur Schnitzler}}\hfill 9. 1. 09.\pend
           
\pstart
           \textcolor{gray}{\textbf{Wien XVIII. Spoettelgasse 7\oindex{Wien@\textbf{Wien}!XVIII., Währing@\textbf{XVIII., Währing}!Edmund-Weiß-Gasse@\textbf{Edmund-Weiß-Gasse}, \emph{Straße}|pw}.}}\pend
           \vspace{0.5em}
\pstart
           Hoffentlich, lieber Guſtav, können Sie von dieſen \label{K_L04152-1v}\edtext{Sitzen für morgen\eventindex{Johann Strauß-Theater@\textbf{Johann Strauß-Theater}!Concordia-Matinee, 10.1.1909@Concordia-Matinee, 10.1.1909|pwv}}{\lemma{\textnormal{\emph{Sitzen für morgen}}}\Cendnote{\textnormal{ Die \emph{Concordia}\orgindex{Concordia. Journalisten- und Schriftstellerverein@Concordia. Journalisten- und Schriftstellerverein|pwk}-Matinée\eventindex{Johann Strauß-Theater@\textbf{Johann Strauß-Theater}!Concordia-Matinee, 10.1.1909@Concordia-Matinee, 10.1.1909|pwk} mit Premieren
                  von \emph{Besuch in der Dämmerung}\pwindex{Rittner, Thaddäus 31.\,5.\,1873 Lviv – 19.\,6.\,1921 Bad Gastein@\textsc{Rittner, Thaddäus} (31.\,5.\,1873 Lviv – 19.\,6.\,1921 Bad Gastein), \emph{Schriftsteller}!Besuch in der Dämmerung@\strich\emph{Besuch in der Dämmerung}|pwk} (von Thaddäus
                     Rittner\pwindex{Rittner, Thaddäus 31.\,5.\,1873 Lviv – 19.\,6.\,1921 Bad Gastein@\textsc{Rittner, Thaddäus} (31.\,5.\,1873 Lviv – 19.\,6.\,1921 Bad Gastein), \emph{Schriftsteller}|pwk}), \emph{Eine florentinische
                     Tragödie}\pwindex{Wilde, Oscar 16.\,10.\,1854 Dublin – 30.\,11.\,1900 Paris@\textsc{Wilde, Oscar} (16.\,10.\,1854 Dublin – 30.\,11.\,1900 Paris), \emph{Schriftsteller}!Eine florentinische Tragödie@\strich\emph{Eine florentinische Tragödie}|pwk} (von Oscar Wilde\pwindex{Wilde, Oscar 16.\,10.\,1854 Dublin – 30.\,11.\,1900 Paris@\textsc{Wilde, Oscar} (16.\,10.\,1854 Dublin – 30.\,11.\,1900 Paris), \emph{Schriftsteller}|pwk}), \emph{Der Pechvogel}\pwindex{Willner, Alfred Maria 11.\,7.\,1859 Wien – 27.\,10.\,1929 ebd.@\textsc{Willner, Alfred Maria} (11.\,7.\,1859 Wien – 27.\,10.\,1929 ebd.), \emph{Schriftsteller, Journalist}!Pechvogel@\strich\emph{Der Pechvogel}|pwk} (von Alfred Maria Willner\pwindex{Willner, Alfred Maria 11.\,7.\,1859 Wien – 27.\,10.\,1929 ebd.@\textsc{Willner, Alfred Maria} (11.\,7.\,1859 Wien – 27.\,10.\,1929 ebd.), \emph{Schriftsteller, Journalist}|pwk}) und von \emph{Anatols Hochzeitsmorgen}\pwindex{Schnitzler, Arthur 15. 5. 1862 Wien – 21. 10. 1931 ebd.@\textsc{Schnitzler, Arthur} (15. 5. 1862 Wien – 21. 10. 1931 ebd.), \emph{Schriftsteller, Mediziner}!Anatols Hochzeitsmorgen@\strich\emph{Anatols Hochzeitsmorgen}|pwk} (von Schnitzler) fand am 10. 1. 1909 um 3 Uhr nachmittags im Johann Strauß-Theater\oindex{Wien@\textbf{Wien}!IV., Wieden@\textbf{IV., Wieden}!Johann Strauß-Theater@\textbf{Johann Strauß-Theater}, \emph{Theater}|pwk} statt. Schnitzler nahm nicht teil. }}}\label{K_L04152-1} Gebrauch machen. Ich ko{\geminationm}e eben von der Generalprobe\eventindex{Johann Strauß-Theater@\textbf{Johann Strauß-Theater}!Generalprobe von Anatols Hochzeitsmorgen, Besuch in der Dämmerung, Eine florentinische Tragödie, Der Pechvogel, 9.1.1909@Generalprobe von Anatols Hochzeitsmorgen, Besuch in der Dämmerung, Eine florentinische Tragödie, Der Pechvogel, 9.1.1909|pwv}. »Man lacht« (wenigſtens bei den \label{K_L04152-2v}\edtext{erſten 3 Stücken\pwindex{Rittner, Thaddäus 31.\,5.\,1873 Lviv – 19.\,6.\,1921 Bad Gastein@\textsc{Rittner, Thaddäus} (31.\,5.\,1873 Lviv – 19.\,6.\,1921 Bad Gastein), \emph{Schriftsteller}!Besuch in der Dämmerung@\strich\emph{Besuch in der Dämmerung}|pwv}\pwindex{Wilde, Oscar 16.\,10.\,1854 Dublin – 30.\,11.\,1900 Paris@\textsc{Wilde, Oscar} (16.\,10.\,1854 Dublin – 30.\,11.\,1900 Paris), \emph{Schriftsteller}!Eine florentinische Tragödie@\strich\emph{Eine florentinische Tragödie}|pwv}\pwindex{Willner, Alfred Maria 11.\,7.\,1859 Wien – 27.\,10.\,1929 ebd.@\textsc{Willner, Alfred Maria} (11.\,7.\,1859 Wien – 27.\,10.\,1929 ebd.), \emph{Schriftsteller, Journalist}!Pechvogel@\strich\emph{Der Pechvogel}|pwv}}{\lemma{\textnormal{\emph{ersten 3 Stücken}}}\Cendnote{\textnormal{Laut Programmzettel
                        wurde \emph{Anatols Hochzeitsmorgen}\pwindex{Schnitzler, Arthur 15. 5. 1862 Wien – 21. 10. 1931 ebd.@\textsc{Schnitzler, Arthur} (15. 5. 1862 Wien – 21. 10. 1931 ebd.), \emph{Schriftsteller, Mediziner}!Anatols Hochzeitsmorgen@\strich\emph{Anatols Hochzeitsmorgen}|pwk} als viertes gegeben.}}}\label{K_L04152-2}.)\pend
           
\pstart
           Herzlichſt Ihr{\\[\baselineskip]}\spacefill\mbox{A.}\pend
           \leftskip=0em{}\selectlanguage{ngerman}\endnumbering\briefempfaengerindex{Schwarzkopf, Gustav@\textsc{Schwarzkopf, Gustav}!zzzSchnitzler, Arthur@\emph{von Arthur Schnitzler}!1909-01-091@{9. 1. 1909}|)be}\mylabel{L04152h}
\begin{anhang}
\end{anhang}\newcommand{\dateiname}{L04152}\newcommand{\titel}{Arthur Schnitzler an Gustav Schwarzkopf, 9. 1. 1909}\newcommand{\editorInnen}{Herausgegeben von Jahnke, SelmaMüller, Martin Anton}%% latex-leseansicht-abspann.tex
%% Abspann für die Leseansicht.
%% Der Schalter \ifkorrekturansicht ist bereits durch den Vorspann gesetzt.

%% latex-abspann.tex
%% Gemeinsamer Abspann für Korrekturansicht und Leseansicht.
%% Setzt den Schalter \ifkorrekturansicht voraus (gesetzt in den
%% einbindenden Dateien latex-korrekturansicht-abspann.tex bzw.
%% latex-leseansicht-abspann.tex).
%% ---------------------------------------------------------------

\normalsize

% Das esempio-Environment wird nur in der Leseansicht benötigt
\ifkorrekturansicht\else
\newenvironment{esempio}[3]%
{
    \vspace{1.5ex}
    \rlap{\underline{#1}}
    \par
    \setlength{\parindent}{0cm}
    \nopagebreak
    \leftskip=#2cm
    \rightskip=#3cm
}
{
    \par
}
\fi

\doendnotes{C}
\bigskip
\vfill

\clearpage

\footnotesize

\ifkorrekturansicht
  \lohead{\textsc{register}}
\fi

% theindex-Environment neu definieren ohne reledmac
\makeatletter
\renewenvironment{theindex}{%
  \ifkorrekturansicht
    \section*{\indexname}%
  \else
    \subsubsection*{Index der erwähnten Entitäten}%
  \fi
  \setlength{\parindent}{0pt}%
  \setlength{\parskip}{0pt plus 0.3pt}%
  \let\item\@idxitem
}{%
  \ifkorrekturansicht\clearpage\fi
}
\makeatother

\IfFileExists{\jobname-pw.ind}{\input{\jobname-pw.ind}}{}

% Quellenangabe nur in der Leseansicht
\ifkorrekturansicht\else
% Fallback-Definitionen, falls die .tex-Datei \titel etc. nicht gesetzt hat
\providecommand{\titel}{}
\providecommand{\editorInnen}{}
\providecommand{\dateiname}{\jobname}

\vspace{3cm}

\vfill

\footnotesize
\textsc{Quelle}: \titel. Herausgegeben von {\editorInnen}. In: \emph{Arthur Schnitzler: Briefwechsel mit Autorinnen und Autoren}.
 Digitale Edition, https://schnitzler-briefe.acdh.oeaw.ac.at/{\dateiname}.html (Stand \today)
\fi

\end{document}


