%% latex-korrekturansicht-vorspann.tex
%% Vorspann für die Korrekturansicht.
%% Lädt die gemeinsame Datei latex-vorspann.tex mit gesetztem Schalter.

\newif\ifkorrekturansicht
\korrekturansichttrue

\input{../tex-inputs/latex-vorspann}


\section[Richard Beer-Hofmann an Arthur Schnitzler, 15. 9. 1904]{L01445 Richard Beer-Hofmann an Arthur Schnitzler, 15. 9. 1904}
\nopagebreak\mylabel{L01445v}
\rehead{ }\normalsize\beginnumbering\briefempfaengerindex{Schnitzler, Arthur@\textsc{Schnitzler, Arthur}!zzzBeer-Hofmann, Richard@\emph{von Richard Beer-Hofmann}!1904-09-151@{15. 9. 1904}|(be}
\toendnotes[C]{\smallbreak\pagebreak[2]}\Standort{CUL, Schnitzler, B 8.}
\physDesc{Brief, 1 Blatt, 1 Seite, 463 Zeichen
\newline{}Handschrift: blaue Tinte, lateinische Kurrent
\newline{}Ordnung: mit Bleistift von unbekannter Hand nummeriert:
                                    »190« }\toendnotes[C]{\smallbreak}
\pstart
           \centering{}{\pb}Aussee\oindex{Bad Aussee@\textbf{Bad Aussee}, \emph{P.PPLA3}|pw}{ }15/IX. 04\pend
           \vspace{0.5em}
\pstart
           Lieber Arthur!{ }Paula\pwindex{Beer-Hofmann, Paula 25.02.1879 – 30.10.1939@\textsc{Beer-Hofmann, Paula} (25.02.1879 – 30.10.1939)|pw} und Kinder\pwindex{Beer-Hofmann, Gabriel 09.01.1901 – 24.03.1971@\textsc{Beer-Hofmann, Gabriel} (09.01.1901 – 24.03.1971), \emph{Schriftsteller/Schriftstellerin, Filmagent/Filmagentin}|pwv}\pwindex{Beer-Hofmann, Naemah 20.12.1898 – 10.11.1971@\textsc{Beer-Hofmann, Naëmah} (20.12.1898 – 10.11.1971)|pwv}\pwindex{Beer-Hofmann, Mirjam 04.09.1897 – 24.12.1984@\textsc{Beer-Hofmann, Mirjam} (04.09.1897 – 24.12.1984)|pwv} fahren
                  morgen Mittag zu Papa\pwindex{Beer, Hermann 10.8.1835 – 03.10.1902@\textsc{Beer, Hermann} (10.8.1835 – 03.10.1902), \emph{Rechtsanwalt/Rechtsanwältin}|pwv} nach Baden\oindex{Baden bei Wien@\textbf{Baden bei Wien}, \emph{P.PPLA3}|pw}. In Lueg\oindex{Lueg@\textbf{Lueg}, \emph{Teil eines besiedelten Ortes (A.BSOX)}|pw} übernachten hat nicht viel Sinn; außerdem suche ich mit
               meinem Husten möglichst bald aus allzu feuchter Luft zu entweichen. Wir treffen uns
               also in Salzburg\oindex{Salzburg@\textbf{Salzburg}, \emph{A.ADM2}|pw}. Bin noch unentschlossen, wo ich
               wohnen soll.\pend
           
\pstart
           Ich hinterlege Brief für Sie im Caffée Tomaselli\oindex{Cafe Tomaselli@\textbf{Café Tomaselli}, \emph{Kaffeehaus (K.KAF)}|pw} mit meiner Adresse. \uline{Vielleicht}
               wohne ich »Schiff\oindex{Hotel Schiff [Salzburg]@\textbf{Hotel Schiff [Salzburg]}, \emph{Hotel (K.HTL)}|pw}«. Geben Sie mir – ebenfalls
                  \strikeout{bis} bei Tomaselli\oindex{Cafe Tomaselli@\textbf{Café Tomaselli}, \emph{Kaffeehaus (K.KAF)}|pw} – Ihre Adresse an.\pend
           
\pstart
           Herzlichst Ihr{\\[\baselineskip]}\spacefill\mbox{Richard}\pend
           \leftskip=0em{}\selectlanguage{ngerman}\endnumbering\briefempfaengerindex{Schnitzler, Arthur@\textsc{Schnitzler, Arthur}!zzzBeer-Hofmann, Richard@\emph{von Richard Beer-Hofmann}!1904-09-151@{15. 9. 1904}|)be}\mylabel{L01445h}  \normalsize

\doendnotes{C}
\bigskip
\vfill

\clearpage

\footnotesize

\lohead{\textsc{register}}

% Definiere theindex-Environment komplett neu ohne reledmac
\makeatletter
\renewenvironment{theindex}{%
  \section*{\indexname}%
  \setlength{\parindent}{0pt}%
  \setlength{\parskip}{0pt plus 0.3pt}%
  \let\item\@idxitem
}{%
  \clearpage
}
\makeatother

\IfFileExists{\jobname-pw.ind}{\input{\jobname-pw.ind}}{}

\end{document}

      