%% latex-leseansicht-vorspann.tex
%% Vorspann für die Leseansicht.
%% Lädt die gemeinsame Datei latex-vorspann.tex mit nicht gesetztem Schalter.

\newif\ifkorrekturansicht
\korrekturansichtfalse

\input{../tex-inputs/latex-vorspann}


\section[Berta Zuckerkandl an Arthur Schnitzler, {{[}}7. oder 14. 4. 1911?{{]}}]{L03998 Berta Zuckerkandl an Arthur Schnitzler, {[}7. oder 14. 4. 1911?{]}}
\nopagebreak\mylabel{L03998v}
\rehead{ }\normalsize\beginnumbering\briefempfaengerindex{Schnitzler, Arthur@\textsc{Schnitzler, Arthur}!zzzZuckerkandl, Berta@\emph{von Berta Zuckerkandl}!1911-04-141@{{[}7. oder 14. 4. 1911?{]}}|(be}
\toendnotes[C]{\smallbreak\pagebreak[2]}
\correspDesc{Versand  durch Berta Zuckerkandl im Zeitraum [7. oder 14. 4. 1911?] in Wien
\newline{}Erhalt  durch Arthur Schnitzler in Menton}\toendnotes[C]{\smallbreak}
\Standort{CUL, Schnitzler, B 200.}
\physDesc{Brief, 2 Blätter, 6 Seiten, 1371 Zeichen
\newline{}Handschrift: schwarze Tinte, lateinische Kurrent (\noindent{}Nummerierung des zweiten Bogens: »II.«)
\newline{}Schnitzler: 1) mit Bleistift datiert: »April 91\substVorne{}\textsuperscript{0}\substDazwischen{}1\substHinten{}«  2) mit Bleistift beschriftet: »Zuckerkandl«}\toendnotes[C]{\smallbreak}
\pstart
           \raggedleft{}{\pb}\label{K_L03998-1v}\edtext{\uline{Samstag}}{\lemma{\textnormal{\emph{Samstag}}}\Cendnote{\textnormal{Schnitzler hat den von Berta Zuckerkandl\pwindex{Zuckerkandl, Berta 13.\,4.\,1864 Wien – 16.\,10.\,1945 Paris@\textsc{Zuckerkandl, Berta} (13.\,4.\,1864 Wien – 16.\,10.\,1945 Paris), \emph{Schriftstellerin, Journalistin, Übersetzerin}|pwk} nur mit Angabe des Wochentags versehenen Brief auf April 1911 datiert. Er beantwortete ihn am XXXX Auszeichnungsfehler: Dokument L03981 nicht gefunden aus Menton\oindex{Menton@\textbf{Menton}|pwk} von seiner Osterreise, sodass der Brief an einem der beiden vorangegangenen Samstage verfasst worden sein müsste, also am 7. 4. 1911 oder am 14. 4. 1911, und dann von Schnitzler zur Beantwortung mit auf die Reise genommen oder ihm dorthin nachgesandt wurde.}}}\label{K_L03998-1}.\pend
           
\pstart{}Sehr verehrter Herr Doktor!\pend\vspace{0.5em}
\pstart
           Wollen Sie folgende Anfrage verzeihen – und diese
      briefliche Belästigung. –\pend
           
\pstart
           Dadurch dass meine Schwester\pwindex{Clemenceau, Sophie 25.\,5.\,1862 – 24.\,9.\,1937@\textsc{Clemenceau, Sophie} (25.\,5.\,1862 – 24.\,9.\,1937)|pwv}
      und die ganze Familie
               Clemenceau\pwindex{Clemenceau, Sophie 25.\,5.\,1862 – 24.\,9.\,1937@\textsc{Clemenceau, Sophie} (25.\,5.\,1862 – 24.\,9.\,1937)|pw}\pwindex{Clemenceau, Georges 28.\,9.\,1841 Mouilleron-en-Pareds – 24.\,11.\,1929 Paris@\textsc{Clemenceau, Georges} (28.\,9.\,1841 Mouilleron-en-Pareds – 24.\,11.\,1929 Paris), \emph{Politiker}|pw}\pwindex{Clemenceau, Albert 23.\,2.\,1861 Nantes – 23.\,7.\,1955 Sceaux@\textsc{Clemenceau, Albert} (23.\,2.\,1861 Nantes – 23.\,7.\,1955 Sceaux), \emph{Politiker, Jurist}|pw} mit \uline{Antoine}\pwindex{Antoine, André 31.\,1.\,1858 Limoges – 23.\,10.\,1943 Le Pouliguen@\textsc{Antoine, André} (31.\,1.\,1858 Limoges – 23.\,10.\,1943 Le Pouliguen), \emph{Theaterleiter, Schauspieler}|pw}
      dem einstigen Begründer
               \introOben{}des Theatre Libre\orgindex{Théâtre Libre@Théâtre Libre|pw}\introOben{}
               und jetzigen Leiter des »Odeon\orgindex{Odéon@Odéon|pw}« {\pb}befreundet ist – bin ich
      über die Repertoire-Verhältnisse des »Odeon\orgindex{Odéon@Odéon|pw}«
      und über die Art der
      Stücke welche Antoine\pwindex{Antoine, André 31.\,1.\,1858 Limoges – 23.\,10.\,1943 Le Pouliguen@\textsc{Antoine, André} (31.\,1.\,1858 Limoges – 23.\,10.\,1943 Le Pouliguen), \emph{Theaterleiter, Schauspieler}|pw}
      sucht, gut unterrichtet.
      Was er für die nächste
      Saison ambitionirt wäre
      ein Stück dass mit
      grossem literarischen {\pb}Wert, auch die Freude
      am bunten Spiel verbindet. Ich weiss nun
      nicht ob Sie geehrter Herr
      Doktor schon die Übersetzung des »Medardus\pwindex{Schnitzler, Arthur 15.\,5.\,1862 Wien – 21.\,10.\,1931 ebd.@\textsc{Schnitzler, Arthur} (15.\,5.\,1862 Wien – 21.\,10.\,1931 ebd.), \emph{Schriftsteller, Mediziner}!junge Medardus. Dramatische Historie in einem Vorspiel und fünf Aufzügen@\strich\emph{Der junge Medardus. Dramatische Historie in einem Vorspiel und fünf Aufzügen}|pw}«
                  ins Französische\oindex{Frankreich@\textbf{Frankreich}|pw} – vergeben haben. Mir ist diese
      Sprache so geläufig wie meine
      eigene – und ich habe sogar
      für den »Temps\pwindex{Le Temps@\emph{Le Temps}|pw}« \label{K_L03998-2v}\edtext{Original-Artikel}{\lemma{\textnormal{\emph{Original-Artikel}}}\Cendnote{\textnormal{\emph{Chronique théatrale. Le théatre de Vienne. Revue de l’ année}\pwindex{Zuckerkandl, Berta 13.\,4.\,1864 Wien – 16.\,10.\,1945 Paris@\textsc{Zuckerkandl, Berta} (13.\,4.\,1864 Wien – 16.\,10.\,1945 Paris), \emph{Schriftstellerin, Journalistin, Übersetzerin}!Chronique théatrale. Le théatre di Vienne. Revue de l’ année@\strich\emph{Chronique théatrale. Le théatre di Vienne. Revue de l’ année}|pwk}. In: \emph{Le Temps}\pwindex{Le Temps@\emph{Le Temps}|pwk}, Jg. 45, Nr. 16147, 4. 9. 1911, S. [1-2].}}}\label{K_L03998-2} geschrieben.\pend
           
\pstart
           {\pb}Es wäre mir Freude, den
               »Medardus\pwindex{Schnitzler, Arthur 15.\,5.\,1862 Wien – 21.\,10.\,1931 ebd.@\textsc{Schnitzler, Arthur} (15.\,5.\,1862 Wien – 21.\,10.\,1931 ebd.), \emph{Schriftsteller, Mediziner}!junge Medardus. Dramatische Historie in einem Vorspiel und fünf Aufzügen@\strich\emph{Der junge Medardus. Dramatische Historie in einem Vorspiel und fünf Aufzügen}|pw}« übersetzen
      zu dürfen. Und da mi\substVorne{}\textsuperscript{ch
                  }\substDazwischen{}r\substHinten{}
      ein länger Aufenthalt Anfangs Juni in Paris\oindex{Paris@\textbf{Paris}, \emph{Hauptstadt}|pw} beschieden sein dürfte, so
      hätte ich dort Gelegenheit überhaupt für Ihre
      Stücke zu wirken. Da ich
      dort durch George Clemenceau\pwindex{Clemenceau, Georges 28.\,9.\,1841 Mouilleron-en-Pareds – 24.\,11.\,1929 Paris@\textsc{Clemenceau, Georges} (28.\,9.\,1841 Mouilleron-en-Pareds – 24.\,11.\,1929 Paris), \emph{Politiker}|pw} alle Thüren der
      Theater-Direktoren offen {\pb}finde. Vielleicht liesse
      sich auch Ihr neuestes
      Stück das »Weite Land\pwindex{Schnitzler, Arthur 15.\,5.\,1862 Wien – 21.\,10.\,1931 ebd.@\textsc{Schnitzler, Arthur} (15.\,5.\,1862 Wien – 21.\,10.\,1931 ebd.), \emph{Schriftsteller, Mediziner}!weite Land. Tragikomödie in fünf Akten@\strich\emph{Das weite Land. Tragikomödie in fünf Akten}|pw}«
      – für die französische\oindex{Frankreich@\textbf{Frankreich}|pw}
      Bühne gewinnen?\pend
           
\pstart
           Ich weiss wie sehr Sie
      durch solche Anfragen
      gestört werden, und
      bitte Sie daher mir {\pb}nur eine telephonische
      Antwort zukommen
      zu lassen.\pend
           
\pstart
           Mit ausgezeichneter Hochachtung{\\[\baselineskip]}\spacefill\mbox{Berta Zuckerkandl}\pend
           \leftskip=0em{}
\pstart
           \noindent{}XIX. Nusswaldgasse 22\oindex{Wien@\textbf{Wien}!XIX., Döbling@\textbf{XIX., Döbling}!Nusswaldgasse 22@\textbf{Nusswaldgasse 22}, \emph{Wohngebäude}|pw}.\pend
           
\pstart
           Gesellschafts-Telephon 412 – VI.\pend
           \selectlanguage{ngerman}\endnumbering\briefempfaengerindex{Schnitzler, Arthur@\textsc{Schnitzler, Arthur}!zzzZuckerkandl, Berta@\emph{von Berta Zuckerkandl}!1911-04-071@{{[}7. oder 14. 4. 1911?{]}}|)be}\mylabel{L03998h}
\begin{anhang}
\end{anhang}\newcommand{\dateiname}{L03998}\newcommand{\titel}{Berta Zuckerkandl an Arthur Schnitzler, [7. oder 14. 4. 1911?]}\newcommand{\editorInnen}{Herausgegeben von Jahnke, SelmaMüller, Martin Anton}%% latex-leseansicht-abspann.tex
%% Abspann für die Leseansicht.
%% Der Schalter \ifkorrekturansicht ist bereits durch den Vorspann gesetzt.

%% latex-abspann.tex
%% Gemeinsamer Abspann für Korrekturansicht und Leseansicht.
%% Setzt den Schalter \ifkorrekturansicht voraus (gesetzt in den
%% einbindenden Dateien latex-korrekturansicht-abspann.tex bzw.
%% latex-leseansicht-abspann.tex).
%% ---------------------------------------------------------------

\normalsize

% Das esempio-Environment wird nur in der Leseansicht benötigt
\ifkorrekturansicht\else
\newenvironment{esempio}[3]%
{
    \vspace{1.5ex}
    \rlap{\underline{#1}}
    \par
    \setlength{\parindent}{0cm}
    \nopagebreak
    \leftskip=#2cm
    \rightskip=#3cm
}
{
    \par
}
\fi

\doendnotes{C}
\bigskip
\vfill

\clearpage

\footnotesize

\ifkorrekturansicht
  \lohead{\textsc{register}}
\fi

% theindex-Environment neu definieren ohne reledmac
\makeatletter
\renewenvironment{theindex}{%
  \ifkorrekturansicht
    \section*{\indexname}%
  \else
    \subsubsection*{Index der erwähnten Entitäten}%
  \fi
  \setlength{\parindent}{0pt}%
  \setlength{\parskip}{0pt plus 0.3pt}%
  \let\item\@idxitem
}{%
  \ifkorrekturansicht\clearpage\fi
}
\makeatother

\IfFileExists{\jobname-pw.ind}{\input{\jobname-pw.ind}}{}

% Quellenangabe nur in der Leseansicht
\ifkorrekturansicht\else
% Fallback-Definitionen, falls die .tex-Datei \titel etc. nicht gesetzt hat
\providecommand{\titel}{}
\providecommand{\editorInnen}{}
\providecommand{\dateiname}{\jobname}

\vspace{3cm}

\vfill

\footnotesize
\textsc{Quelle}: \titel. Herausgegeben von {\editorInnen}. In: \emph{Arthur Schnitzler: Briefwechsel mit Autorinnen und Autoren}.
 Digitale Edition, https://schnitzler-briefe.acdh.oeaw.ac.at/{\dateiname}.html (Stand \today)
\fi

\end{document}


