\input{../tex-inputs/latex-pdf-vorspann}
\begin{center}
            \textcolor{red}{ENTWURF. ENTZIFFERUNG NOCH NICHT KORREKTURGELESEN}
                      \end{center}
            
               \section[Hugo von Hofmannsthal an Arthur Schnitzler, {[}21. 2. 1913{]}]{ Hugo von Hofmannsthal an Arthur Schnitzler, {[}21. 2. 1913{]}}\nopagebreak\mylabel{v}\rehead{ }\begin{ledgroupsized}[t]{13cm}\normalsize\beginnumbering\briefempfaengerindex{Schnitzler, Arthur@\textsc{Schnitzler, Arthur}!zzzHofmannsthal, Hugo von@\emph{von Hugo von Hofmannsthal}!1913-02-211@{{[}21. 2. 1913{]}}|(be} \toendnotes[C]{\smallbreak\pagebreak[2]} \Standort{CUL, Schnitzler, B 43.}
\physDesc{Briefkarte
\newline{}Handschrift: schwarze Tinte, deutsche Kurrent
\newline{}Schnitzler: mit Bleistift datiert: »21/2 913« und beschriftet: »\textsc{Hugo}« \newline{}Ordnung: 1) mit Bleistift von unbekannter Hand nummeriert: »\strikeout{334}« 2) mit Bleistift von unbekannter Hand nummeriert: »347«}\buchAbdrucke{\weitereDrucke{Hugo von Hofmannsthal, Arthur Schnitzler: \emph{Briefwechsel}. Hg. Therese Nickl und Heinrich Schnitzler. Frankfurt am Main: \emph{S. Fischer} 1964, S. 272.} }\toendnotes[C]{\smallbreak}\pstart
           \raggedleft{}{\pb}Rodaun\oindex{Rodaun@\textbf{Rodaun}|pw}{ }Freitg\pend
           \pstart{}mein lieber Arthur \pend\pstart
           ganz gewiſs werde ich Montag um ¾ 6 bei Ihnen ſein – weil
               es mir eine der größten und reinſten Freuden iſt, eine neue Ihrer Arbeiten\pwindex{Schnitzler, Arthur 15.05.1862 – 21.10.1931@\textsc{Schnitzler, Arthur} (15.05.1862 – 21.10.1931), \emph{Schriftsteller, Mediziner}!Frau Beate und ihr Sohn. Novelle1.2.1913 – 1.4.1913@\strich\emph{Frau Beate und ihr Sohn. Novelle} {[}1.2.1913 – 1.4.1913{]}|pwv} von Ihrer eigenen Stimme zuerſt zu
               hören – und weil ich überhaupt beſtändig {\pb}traurig darüber bin, daſs ich Sie
               ſo wenig ſehe, daſs in dieſem Einander-ſehen gar keine Improviſation möglich iſt, gar
               keine Begegnung, kein Miteinander-ausgehen, ſondern allmählich nur dieſe einzige Form
               des Nachtmahls, faſt ein wenig starr, ſich herausgebildet hat, was vielleicht –
               bedenkt man wie kurz das Leben und wie unerſchöpflich das Individuum iſt – nicht ſo
                  \label{T_L02113_1v}\edtext{ſein müßte
               und ſollte}{\lemma{\textnormal{\emph{ſein müßte
               und ſollte}}}\Cendnote{\textnormal{weiter
                  quer am linken Rand}}}\label{T_L02113_1h}.\pend
           \pstart Von Herzen Ihr\spacefill\mbox{Hugo}\pend{}\endnumbering\briefempfaengerindex{Schnitzler, Arthur@\textsc{Schnitzler, Arthur}!zzzHofmannsthal, Hugo von@\emph{von Hugo von Hofmannsthal}!1913-02-211@{{[}21. 2. 1913{]}}|)be}\mylabel{h}\end{ledgroupsized}  \newcommand{\dateiname}{L02113}\newcommand{\titel}{Hugo von Hofmannsthal an Arthur Schnitzler, [21. 2. 1913]}\newcommand{\editorInnen}{Martin Anton Müller und Gerd-Hermann Susen}\input{../tex-inputs/latex-pdf-abspann}
      