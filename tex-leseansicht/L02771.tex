%% latex-leseansicht-vorspann.tex
%% Vorspann für die Leseansicht.
%% Lädt die gemeinsame Datei latex-vorspann.tex mit nicht gesetztem Schalter.

\newif\ifkorrekturansicht
\korrekturansichtfalse

\input{../tex-inputs/latex-vorspann}


\section[ Paul Goldmann an Arthur Schnitzler, 9. [4.] 1896]{L02771 Paul Goldmann an Arthur Schnitzler,  9. [4.] 1896}
\nopagebreak\mylabel{L02771v}
\rehead{ }\normalsize\beginnumbering\briefempfaengerindex{Schnitzler, Arthur@\textsc{Schnitzler, Arthur}!zzzGoldmann, Paul@\emph{von Paul Goldmann}!1896-04-091@{9. [4.] 1896}|(be}
\toendnotes[C]{\smallbreak\pagebreak[2]}
\correspDesc{Versand  durch Paul Goldmann am 9. [4.] 1896 in Frankfurt am Main
\newline{}Erhalt  durch Arthur Schnitzler im Zeitraum [10. 4. 1896
                  – 15. 4. 1896?] in Wien}\toendnotes[C]{\smallbreak}
\Standort{DLA, A:Schnitzler, HS.NZ85.1.3166.}
\physDesc{Brief, 2 Blätter, 8 Seiten, 2769 Zeichen
\newline{}Handschrift: blaue Tinte, deutsche Kurrent
\newline{}Schnitzler: 1) mit Bleistift Goldmanns\pwindex{Goldmann, Paul 31.\,1.\,1865 Breslau – 25.\,9.\,1935 Wien@\textsc{Goldmann, Paul} (31.\,1.\,1865 Breslau – 25.\,9.\,1935 Wien), \emph{Schriftsteller, Journalist}|pw}
                                 Datierung »März« durchgestrichen und darunter »April« vermerkt  2) mit rotem Buntstift zwei Unterstreichungen}\toendnotes[C]{\smallbreak}
\pstart
           {\pb}\textcolor{gray}{\textbf{\textbf{Frankfurter Zeitung\orgindex{Frankfurter Zeitung@Frankfurter Zeitung|pw}}}}\hfill \textcolor{gray}{\textbf{Frankfurt a. M.\oindex{Frankfurt am Main@\textbf{Frankfurt am Main}, \emph{Hauptstadt}|pw},}}{ }9. März \textcolor{gray}{\textbf{189}}6.\pend
           
\pstart
           \textcolor{gray}{\textbf{und}}\pend
           
\pstart
           \textcolor{gray}{\textbf{Handelsblatt\orgindex{Frankfurter Zeitung@Frankfurter Zeitung|pwv}.}}\pend
           
\pstart
           \textcolor{gray}{\textbf{\textbf{Redaktion\orgindex{Frankfurter Zeitung@Frankfurter Zeitung|pwv}.\footnote{\noindent{}\textcolor{gray}{\textbf{Für die Redaktion\orgindex{Frankfurter Zeitung@Frankfurter Zeitung|pwv} beſtimmte Briefe und Sendungen
                                 wolle man \so{nicht} an die Perſon eines
                                 Redakteurs,{ }ſondern{ }ſtets \textbf{an die Redaktion der Frankfurter Zeitung\orgindex{Frankfurter Zeitung@Frankfurter Zeitung|pw}} adreſſiren}}.}}}}\pend
           
\pstart
           \textcolor{gray}{\textbf{Telegramm-Adreſſe:}}\pend
           
\pstart
           \textcolor{gray}{\textbf{\textbf{Zeitung\orgindex{Frankfurter Zeitung@Frankfurter Zeitung|pwv}{ }Frankfurt Main\oindex{Frankfurt am Main@\textbf{Frankfurt am Main}, \emph{Hauptstadt}|pw}.}}}\pend
           
\pstart\center{}Mein lieber Freund,\pend\vspace{0.5em}
\pstart
           Ich bekam Deinen lieben Brief hierher nachgeſandt, kann Dir alſo den Brief, von dem
               Du{ }ſprichſt, erſt nächſte Woche nach meiner Rückkehr zurückfenden.\pend
           
\pstart
           Du{ }ſollſt nur einen kurzen Gruß von unterwegs erhalten. Ich bin hier, müde und
               ruhebedürftig. Mein \strikeout{Au} Auge iſt krank, und \strikeout{d} auch die Ruhe will nicht mehr viel nutzen. Hieſigen
               Eindrücke wenig erfreulich. Meine Familie, die {\pb}friedliche, in \strikeout{z\textcolor{gray}{×}} Parteien geſpalten, – aufgelöſt durch das neu hinzugekommene \label{K_L02771-1v}\edtext{\textsc{dissolvant}}{\lemma{\textnormal{\emph{dissolvant}}}\Cendnote{\textnormal{französisch: Lösungsmittel. Womöglich
                  ist Johanna Schwabacher\pwindex{Mamroth, Johanna 19.\,5.\,1872 Frankfurt am Main – 12.\,9.\,1910@\textsc{Mamroth, Johanna} (19.\,5.\,1872 Frankfurt am Main – 12.\,9.\,1910)|pwk} gemeint, deren
                  Heirat mit Fedor Mamroth\pwindex{Mamroth, Fedor 21.\,2.\,1851 Breslau – 25.\,6.\,1907 Frankfurt am Main@\textsc{Mamroth, Fedor} (21.\,2.\,1851 Breslau – 25.\,6.\,1907 Frankfurt am Main), \emph{Journalist, Kritiker}|pwk} bevorstand.}}}\label{K_L02771-1}.
               Schlimme Dinge,{ }ſchlimmme Dinge!\pend
           
\pstart
           Von Dir{ }ſpricht alle Welt mit wärmſter Sympathie, und während Deines Aufenhalts in
                  Frankfurt\oindex{Frankfurt am Main@\textbf{Frankfurt am Main}, \emph{Hauptstadt}|pw} haſt Du bei uns alle Herzen
               gewonnen. Freundlich grüßt mich Dein Name aus den Schaufenſtern der
               Buchhandlungen.\pend
           
\pstart
           Was Du mir über Deine Stimmungen{ }ſchreibſt, iſt gar{ }ſeltſam. Daß auch Du dieſe Idee
               haſt, Dein Leben zu verlieren{[},{]}{ }{\pb}Du, deſſen Leben reich iſt, wie kein zweites, das
               ich kenne. So{ }ſcheint es, daß \strikeout{\textcolor{gray}{×}} wir auf allen Stufen, bei allen Geſchicken, im Glück und Unglück das Gefühl
               haben, das Leben zu verlieren; und vielleicht verlieren wirs auch \substVorne{}\textsuperscript{a}\substDazwischen{}A\substHinten{}lle wirklich.\pend
           
\pstart
           Gern möchte ich Dich im Sommer wiederſehen, vorausgeſetzt, daß ich bis dahin noch in
               keinem Spital liege: Holland\oindex{Niederlande@\textbf{Niederlande}|pw}, Dänemark\oindex{Dänemark@\textbf{Dänemark}|pw}, wo Du willſt. Freilich wirſt Du bei unſerem
               Wiederſehen {\pb}merken, daß{ }ſich Manches verändert
               hat.\pend
           
\pstart
           Und warum kommſt Du nicht nach \textsc{Paris\oindex{Paris@\textbf{Paris}, \emph{Hauptstadt}|pw}}?\pend
           
\pstart
           Dem \textsc{Hugo\pwindex{Hofmannsthal, Hugo von 1.\,2.\,1874 Wien – 15.\,7.\,1929 Rodaun@\textsc{Hofmannsthal, Hugo von} (1.\,2.\,1874 Wien – 15.\,7.\,1929 Rodaun), \emph{Schriftsteller}|pw}} thue ich \uline{nicht} Unrecht. Ich{ }ſoll den \label{K_L02771-2v}\edtext{Artikel\pwindex{Hofmannsthal, Hugo von 1.\,2.\,1874 Wien – 15.\,7.\,1929 Rodaun@\textsc{Hofmannsthal, Hugo von} (1.\,2.\,1874 Wien – 15.\,7.\,1929 Rodaun), \emph{Schriftsteller}!Gedichte von Stefan George@\strich\emph{Gedichte von Stefan George}|pwv}}{\lemma{\textnormal{\emph{Artikel}}}\Cendnote{\textnormal{Hugo von Hofmannsthal\pwindex{Hofmannsthal, Hugo von 1.\,2.\,1874 Wien – 15.\,7.\,1929 Rodaun@\textsc{Hofmannsthal, Hugo von} (1.\,2.\,1874 Wien – 15.\,7.\,1929 Rodaun), \emph{Schriftsteller}|pwk}: \emph{Gedichte von Stefan George}\pwindex{Hofmannsthal, Hugo von 1.\,2.\,1874 Wien – 15.\,7.\,1929 Rodaun@\textsc{Hofmannsthal, Hugo von} (1.\,2.\,1874 Wien – 15.\,7.\,1929 Rodaun), \emph{Schriftsteller}!Gedichte von Stefan George@\strich\emph{Gedichte von Stefan George}|pwk}. In: \emph{Die Zeit}\pwindex{Zeit. Wiener Wochenschrift@\emph{Die Zeit. Wiener Wochenschrift}|pwk}, Bd. 6, Nr. 77, 21. 3. 1896, S. 189–191.}}}\label{K_L02771-2} leſen, als handle er nicht
               von \textsc{St. Georges\pwindex{George, Stefan 17.\,7.\,1868 Büdesheim – 4.\,12.\,1933 Minusio@\textsc{George, Stefan} (17.\,7.\,1868 Büdesheim – 4.\,12.\,1933 Minusio), \emph{Schriftsteller, Übersetzer}|pw}}. Ja, er handelt aber davon. Ich kann Form und Inhalt nicht{ }ſcheiden, beſonders
               nicht bei einer Kritik. Und wenn die Form gut iſt, das Urtheil aber falſch,{ }ſo iſts
               eine{ }ſchlechte Kritik. Auch iſt die Form nicht gut, – verfluchte Manier! {\pb}Hoffentlich nimmſt Du das \label{K_L02771-3v}\edtext{Burgtheater\orgindex{Burgtheater@Burgtheater|pw}-Referat in der »Zeit\pwindex{Zeit. Wiener Wochenschrift@\emph{Die Zeit. Wiener Wochenschrift}|pw}«}{\lemma{\textnormal{\emph{Burgtheater-Referat … »Zeit«}}}\Cendnote{\textnormal{Das hätte 
                  bedeutet, dass Schnitzler alle Rezensionen der \emph{Zeit}\pwindex{Zeit. Wiener Wochenschrift@\emph{Die Zeit. Wiener Wochenschrift}|pwk} von Aufführungen im 
                  \emph{Burgtheater}\orgindex{Burgtheater@Burgtheater|pwk} verantwortet hätte. Dazu kam es nicht.}}}\label{K_L02771-3} an. Du biſt der
               geborene Kritiker – wahrhaftig und unbeſtechlich, ich meine{ }ſeeliſch unbeſtechlich,
               nicht einmal ein \label{K_L02771-4v}\edtext{\begin{otherlanguage}{french}\textsc{emballé}\end{otherlanguage}}{\lemma{\textnormal{\emph{emballé}}}\Cendnote{\textnormal{französisch: Mitgerissener}}}\label{K_L02771-4}, wie
               ich. Und dann Du mit Deinem \strikeout{klug} klugen Urtheil und
               feinen Kunſtſinn! Nimms \strikeout{a} an! \strikeout{Da} Daß Du nicht journaliſtiſch thätig{ }ſein kannſt, {\pb}iſt eine Deiner Wahnideen, die am Beſten durch die
               Praxis widerlegt werden. Auch{ }ſchafft Dir eine regelmäßige kritiſche Thätigkeit
               gewiſſe Lebensgrenzen, – \begin{otherlanguage}{french}Barri\textsc{è}ren\end{otherlanguage}, welche Deine Gedanken verhindern, im Unendlichen Unfug zu
               treiben. Wenn Du genöthigt biſt, \textsc{Rudolf Lothar\pwindex{Lothar, Rudolf 23.\,2.\,1865 Budapest – 2.\,10.\,1943 ebd.@\textsc{Lothar, Rudolf} (23.\,2.\,1865 Budapest – 2.\,10.\,1943 ebd.), \emph{Schriftsteller, Journalist, Theaterdirektor}|pw}} und \textsc{Davis\pwindex{Davis, Gustav 3.\,3.\,1856 Bratislava – 21.\,8.\,1951 Hollenstein an der Ybbs@\textsc{Davis, Gustav} (3.\,3.\,1856 Bratislava – 21.\,8.\,1951 Hollenstein an der Ybbs), \emph{Journalist, Herausgeber}|pw}} kritiſch zu behandeln, wirſt Du weniger an den Tod denken.\pend
           
\pstart
           Wie wenn Du mir ein Wort hierher{ }ſchriebeſt? (\textsc{Niddastraſse 37}\oindex{Niddastraße@\textbf{Niddastraße}, \emph{Straße}|pw}.) Das wäre{ }ſchön\textcolor{gray}{.}\pend
           
\pstart
           {\pb}Iſt Dein \label{K_L02771-5v}\edtext{Stück\pwindex{Schnitzler, Arthur 15.\,5.\,1862 Wien – 21.\,10.\,1931 ebd.@\textsc{Schnitzler, Arthur} (15.\,5.\,1862 Wien – 21.\,10.\,1931 ebd.), \emph{Schriftsteller, Mediziner}!Freiwild. Schauspiel in 3 Akten@\strich\emph{Freiwild. Schauspiel in 3 Akten}|pwv} fertig}{\lemma{\textnormal{\emph{Stück fertig}}}\Cendnote{\textnormal{Es ging dem Ende zu. Schnitzler begann eine neue Niederschrift von \emph{Freiwild}\pwindex{Schnitzler, Arthur 15.\,5.\,1862 Wien – 21.\,10.\,1931 ebd.@\textsc{Schnitzler, Arthur} (15.\,5.\,1862 Wien – 21.\,10.\,1931 ebd.), \emph{Schriftsteller, Mediziner}!Freiwild. Schauspiel in 3 Akten@\strich\emph{Freiwild. Schauspiel in 3 Akten}|pwk} am 27. 4. 1896. Am 3. 5. 1896 las er das Stück\pwindex{Schnitzler, Arthur 15.\,5.\,1862 Wien – 21.\,10.\,1931 ebd.@\textsc{Schnitzler, Arthur} (15.\,5.\,1862 Wien – 21.\,10.\,1931 ebd.), \emph{Schriftsteller, Mediziner}!Freiwild. Schauspiel in 3 Akten@\strich\emph{Freiwild. Schauspiel in 3 Akten}|pwkv}{ }Felix Salten\pwindex{Salten, Felix 6.\,9.\,1869 Budapest – 8.\,10.\,1945 Zürich@\textsc{Salten, Felix} (6.\,9.\,1869 Budapest – 8.\,10.\,1945 Zürich), \emph{Schriftsteller, Journalist, Chefredakteur}|pwk} vor, dessen positive
                  Rückmeldung ihn bestärkte. Am 5. 6. 1896 hatte Schnitzler
                  das Stück\pwindex{Schnitzler, Arthur 15.\,5.\,1862 Wien – 21.\,10.\,1931 ebd.@\textsc{Schnitzler, Arthur} (15.\,5.\,1862 Wien – 21.\,10.\,1931 ebd.), \emph{Schriftsteller, Mediziner}!Freiwild. Schauspiel in 3 Akten@\strich\emph{Freiwild. Schauspiel in 3 Akten}|pwkv} »sozusagen
                     beendet«.}}}\label{K_L02771-5}? Kann man das Manuſkript\pwindex{Schnitzler, Arthur 15.\,5.\,1862 Wien – 21.\,10.\,1931 ebd.@\textsc{Schnitzler, Arthur} (15.\,5.\,1862 Wien – 21.\,10.\,1931 ebd.), \emph{Schriftsteller, Mediziner}!Freiwild. Schauspiel in 3 Akten@\strich\emph{Freiwild. Schauspiel in 3 Akten}|pwv}{ }ſehen?\pend
           
\pstart
           Bitte,{ }ſchick’ mir nach \textsc{Paris\oindex{Paris@\textbf{Paris}, \emph{Hauptstadt}|pw}} die im Buchhandel erſchienenen \textsc{Anatol}-Sachen\pwindex{Schnitzler, Arthur 15.\,5.\,1862 Wien – 21.\,10.\,1931 ebd.@\textsc{Schnitzler, Arthur} (15.\,5.\,1862 Wien – 21.\,10.\,1931 ebd.), \emph{Schriftsteller, Mediziner}!Anatol@\strich\emph{Anatol}|pw}.\pend
           
\pstart
           Grüß’ Dich Gott, mein lieber Freund!\pend
           
\pstart
           Dein {\\[\baselineskip]}\spacefill\mbox{Paul Goldmann.}\pend
           \leftskip=0em{}
\pstart
           \noindent{}Gruß an \textsc{Richard\pwindex{Beer-Hofmann, Richard 11.\,7.\,1866 Wien – 26.\,9.\,1945 New York City@\textsc{Beer-Hofmann, Richard} (11.\,7.\,1866 Wien – 26.\,9.\,1945 New York City), \emph{Schriftsteller}|pw}}.\pend
           
\pstart
           {\pb}Gefunden in einem alten deutſchen Myſtiker\pwindex{Angelus Silesius vor dem 25.12.1624 Breslau – 9.\,7.\,1677 ebd.@\textsc{Angelus Silesius} (vor dem 25.12.1624 Breslau – 9.\,7.\,1677 ebd.), \emph{Schriftsteller, Mediziner, Theologe}|pwv}\pwindex{Angelus Silesius vor dem 25.12.1624 Breslau – 9.\,7.\,1677 ebd.@\textsc{Angelus Silesius} (vor dem 25.12.1624 Breslau – 9.\,7.\,1677 ebd.), \emph{Schriftsteller, Mediziner, Theologe}|pwv}:\pend
           \stanza{} »\label{K_L02771-6v}\edtext{Der Zufall muß hinweg\pwindex{Angelus Silesius vor dem 25.12.1624 Breslau – 9.\,7.\,1677 ebd.@\textsc{Angelus Silesius} (vor dem 25.12.1624 Breslau – 9.\,7.\,1677 ebd.), \emph{Schriftsteller, Mediziner, Theologe}!Cherubinischer Wandersmann@\strich\emph{Cherubinischer Wandersmann}|pwv}}{\lemma{\textnormal{\emph{Der Zufall muß hinweg}}}\Cendnote{\textnormal{Epigramm 274 aus \emph{Geistreiche Sinn- und Schlussreime}\pwindex{Angelus Silesius vor dem 25.12.1624 Breslau – 9.\,7.\,1677 ebd.@\textsc{Angelus Silesius} (vor dem 25.12.1624 Breslau – 9.\,7.\,1677 ebd.), \emph{Schriftsteller, Mediziner, Theologe}!Geistreiche Sinn- und Schlussreime@\strich\emph{Geistreiche Sinn- und Schlussreime}|pwk} (1657) von Angelus
                           Silesius\pwindex{Angelus Silesius vor dem 25.12.1624 Breslau – 9.\,7.\,1677 ebd.@\textsc{Angelus Silesius} (vor dem 25.12.1624 Breslau – 9.\,7.\,1677 ebd.), \emph{Schriftsteller, Mediziner, Theologe}|pwk}}}}\label{K_L02771-6}\newverse{}und aller falſcher
                        Schein,\pwindex{Angelus Silesius vor dem 25.12.1624 Breslau – 9.\,7.\,1677 ebd.@\textsc{Angelus Silesius} (vor dem 25.12.1624 Breslau – 9.\,7.\,1677 ebd.), \emph{Schriftsteller, Mediziner, Theologe}!Cherubinischer Wandersmann@\strich\emph{Cherubinischer Wandersmann}|pwv}\newverse{}Du mußt ganz weſentlich\pwindex{Angelus Silesius vor dem 25.12.1624 Breslau – 9.\,7.\,1677 ebd.@\textsc{Angelus Silesius} (vor dem 25.12.1624 Breslau – 9.\,7.\,1677 ebd.), \emph{Schriftsteller, Mediziner, Theologe}!Cherubinischer Wandersmann@\strich\emph{Cherubinischer Wandersmann}|pwv}\newverse{}und ungefärbet{ }ſein.\pwindex{Angelus Silesius vor dem 25.12.1624 Breslau – 9.\,7.\,1677 ebd.@\textsc{Angelus Silesius} (vor dem 25.12.1624 Breslau – 9.\,7.\,1677 ebd.), \emph{Schriftsteller, Mediziner, Theologe}!Cherubinischer Wandersmann@\strich\emph{Cherubinischer Wandersmann}|pwv}«\stanzaend{}
\pstart
           Und was{ }ſagſt Du zu Frau \label{K_L02771-7v}\edtext{\textsc{Lou Andreas\pwindex{Andreas-Salomé, Lou 12.\,2.\,1861 Sankt Petersburg – 5.\,2.\,1937 Göttingen@\textsc{Andreas-Salomé, Lou} (12.\,2.\,1861 Sankt Petersburg – 5.\,2.\,1937 Göttingen), \emph{Schriftstellerin}|pw}}}{\lemma{\textnormal{\emph{Lou Andreas}}}\Cendnote{\textnormal{\emph{Ruth}\pwindex{Andreas-Salomé, Lou 12.\,2.\,1861 Sankt Petersburg – 5.\,2.\,1937 Göttingen@\textsc{Andreas-Salomé, Lou} (12.\,2.\,1861 Sankt Petersburg – 5.\,2.\,1937 Göttingen), \emph{Schriftstellerin}!Ruth. Erzählung@\strich\emph{Ruth. Erzählung}|pwk} hatte Schnitzler bereits am 10. 1. 1896 gelesen. Zu Lou Andreas-Salomé\pwindex{Andreas-Salomé, Lou 12.\,2.\,1861 Sankt Petersburg – 5.\,2.\,1937 Göttingen@\textsc{Andreas-Salomé, Lou} (12.\,2.\,1861 Sankt Petersburg – 5.\,2.\,1937 Göttingen), \emph{Schriftstellerin}|pwk} dürfte zu dieser Zeit kein näherer
                     Kontakt bestanden haben.}}}\label{K_L02771-7}’ Buch »Ruth\pwindex{Andreas-Salomé, Lou 12.\,2.\,1861 Sankt Petersburg – 5.\,2.\,1937 Göttingen@\textsc{Andreas-Salomé, Lou} (12.\,2.\,1861 Sankt Petersburg – 5.\,2.\,1937 Göttingen), \emph{Schriftstellerin}!Ruth. Erzählung@\strich\emph{Ruth. Erzählung}|pw}«? Hörſt Du etwas von ihr?\pend
           \selectlanguage{ngerman}\endnumbering\briefempfaengerindex{Schnitzler, Arthur@\textsc{Schnitzler, Arthur}!zzzGoldmann, Paul@\emph{von Paul Goldmann}!1896-04-091@{9. [4.] 1896}|)be}\mylabel{L02771h}  \newcommand{\dateiname}{L02771}\newcommand{\titel}{Paul Goldmann an Arthur Schnitzler, 9. [4.] 1896}\newcommand{\editorInnen}{Martin Anton Müller und Laura Untner}%% latex-leseansicht-abspann.tex
%% Abspann für die Leseansicht.
%% Der Schalter \ifkorrekturansicht ist bereits durch den Vorspann gesetzt.

%% latex-abspann.tex
%% Gemeinsamer Abspann für Korrekturansicht und Leseansicht.
%% Setzt den Schalter \ifkorrekturansicht voraus (gesetzt in den
%% einbindenden Dateien latex-korrekturansicht-abspann.tex bzw.
%% latex-leseansicht-abspann.tex).
%% ---------------------------------------------------------------

\normalsize

% Das esempio-Environment wird nur in der Leseansicht benötigt
\ifkorrekturansicht\else
\newenvironment{esempio}[3]%
{
    \vspace{1.5ex}
    \rlap{\underline{#1}}
    \par
    \setlength{\parindent}{0cm}
    \nopagebreak
    \leftskip=#2cm
    \rightskip=#3cm
}
{
    \par
}
\fi

\doendnotes{C}
\bigskip
\vfill

\clearpage

\footnotesize

\ifkorrekturansicht
  \lohead{\textsc{register}}
\fi

% theindex-Environment neu definieren ohne reledmac
\makeatletter
\renewenvironment{theindex}{%
  \ifkorrekturansicht
    \section*{\indexname}%
  \else
    \subsubsection*{Index der erwähnten Entitäten}%
  \fi
  \setlength{\parindent}{0pt}%
  \setlength{\parskip}{0pt plus 0.3pt}%
  \let\item\@idxitem
}{%
  \ifkorrekturansicht\clearpage\fi
}
\makeatother

\IfFileExists{\jobname-pw.ind}{\input{\jobname-pw.ind}}{}

% Quellenangabe nur in der Leseansicht
\ifkorrekturansicht\else
% Fallback-Definitionen, falls die .tex-Datei \titel etc. nicht gesetzt hat
\providecommand{\titel}{}
\providecommand{\editorInnen}{}
\providecommand{\dateiname}{\jobname}

\vspace{3cm}

\vfill

\footnotesize
\textsc{Quelle}: \titel. Herausgegeben von {\editorInnen}. In: \emph{Arthur Schnitzler: Briefwechsel mit Autorinnen und Autoren}.
 Digitale Edition, https://schnitzler-briefe.acdh.oeaw.ac.at/{\dateiname}.html (Stand \today)
\fi

\end{document}


