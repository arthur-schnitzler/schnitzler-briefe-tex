%% latex-korrekturansicht-vorspann.tex
%% Vorspann für die Korrekturansicht.
%% Lädt die gemeinsame Datei latex-vorspann.tex mit gesetztem Schalter.

\newif\ifkorrekturansicht
\korrekturansichttrue

\input{../tex-inputs/latex-vorspann}


\section[ Paul Goldmann an Arthur Schnitzler, 9. {[}4.{]} 1896]{L02771 Paul Goldmann an Arthur Schnitzler, 9. {[}4.{]} 1896}
\nopagebreak\mylabel{L02771v}
\rehead{ }\normalsize\beginnumbering\briefempfaengerindex{Schnitzler, Arthur@\textsc{Schnitzler, Arthur}!zzzGoldmann, Paul@\emph{von Paul Goldmann}!1896-04-091@{9. {[}4.{]} 1896}|(be}
\toendnotes[C]{\smallbreak\pagebreak[2]}\Standort{DLA, A:Schnitzler, HS.NZ85.1.3166.}
\physDesc{Brief, 2 Blätter, 8 Seiten, 2769 Zeichen
\newline{}Handschrift: blaue Tinte, deutsche Kurrent
\newline{}Schnitzler: 1) mit Bleistift Goldmanns\pwindex{Goldmann, Paul 31.01.1865 – 25.09.1935@\textsc{Goldmann, Paul} (31.01.1865 – 25.09.1935), \emph{Schriftsteller/Schriftstellerin, Journalist/Journalistin}|pw}
                                 Datierung »März« durchgestrichen und darunter »April« vermerkt  2) mit rotem Buntstift zwei Unterstreichungen}\toendnotes[C]{\smallbreak}
\pstart
           {\pb}\textcolor{gray}{\textbf{\textbf{Frankfurter Zeitung\orgindex{Frankfurter Zeitung@Frankfurter Zeitung|pw}}}}\hfill \textcolor{gray}{\textbf{Frankfurt a. M.\oindex{Frankfurt am Main@\textbf{Frankfurt am Main}, \emph{P.PPLA3}|pw}, }}9. März \textcolor{gray}{\textbf{189}}6.\pend
           
\pstart
           \textcolor{gray}{\textbf{und}}\pend
           
\pstart
           \textcolor{gray}{\textbf{Handelsblatt\orgindex{Frankfurter Zeitung@Frankfurter Zeitung|pwv}.}}\pend
           
\pstart
           \textcolor{gray}{\textbf{\textbf{Redaktion\orgindex{Frankfurter Zeitung@Frankfurter Zeitung|pwv}.\noindent{}\textcolor{gray}{\textbf{Für die Redaktion\orgindex{Frankfurter Zeitung@Frankfurter Zeitung|pwv} beſtimmte Briefe und Sendungen
                                 wolle man \so{nicht} an die Perſon eines
                                 Redakteurs, ſondern ſtets \textbf{an die Redaktion der Frankfurter Zeitung\orgindex{Frankfurter Zeitung@Frankfurter Zeitung|pw}} adreſſiren}}.}}}\pend
           
\pstart
           \textcolor{gray}{\textbf{Telegramm-Adreſſe:}}\pend
           
\pstart
           \textcolor{gray}{\textbf{\textbf{Zeitung\orgindex{Frankfurter Zeitung@Frankfurter Zeitung|pwv}{ }Frankfurt Main\oindex{Frankfurt am Main@\textbf{Frankfurt am Main}, \emph{P.PPLA3}|pw}.}}}\pend
           
\pstart\center{}Mein lieber Freund,\pend\vspace{0.5em}
\pstart
           Ich bekam Deinen lieben Brief hierher nachgeſandt, kann Dir alſo den Brief, von dem
               Du ſprichſt, erſt nächſte Woche nach meiner Rückkehr zurückfenden.\pend
           
\pstart
           Du ſollſt nur einen kurzen Gruß von unterwegs erhalten. Ich bin hier, müde und
               ruhebedürftig. Mein \strikeout{Au} Auge iſt krank, und \strikeout{d} auch die Ruhe will nicht mehr viel nutzen. Hieſigen
               Eindrücke wenig erfreulich. Meine Familie, die {\pb}friedliche, in \strikeout{z\textcolor{gray}{×}} Parteien geſpalten, – aufgelöſt durch das neu hinzugekommene \label{K_L02771-1v}\edtext{\textsc{dissolvant}}{\lemma{\textnormal{\emph{dissolvant}}}\Cendnote{\textnormal{französisch: Lösungsmittel. Womöglich
                  ist Johanna Schwabacher\pwindex{Mamroth, Johanna 1872-05-19 – 1910-09-12@\textsc{Mamroth, Johanna} (1872-05-19 – 1910-09-12)|pwk} gemeint, deren
                  Heirat mit Fedor Mamroth\pwindex{Mamroth, Fedor 21.02.1851 – 25.06.1907@\textsc{Mamroth, Fedor} (21.02.1851 – 25.06.1907), \emph{Journalist/Journalistin, Kritiker/Kritikerin}|pwk} bevorstand.}}}\label{K_L02771-1}.
               Schlimme Dinge, ſchlimmme Dinge! \pend
           
\pstart
           Von Dir ſpricht alle Welt mit wärmſter Sympathie, und während Deines Aufenhalts in
                  Frankfurt\oindex{Frankfurt am Main@\textbf{Frankfurt am Main}, \emph{P.PPLA3}|pw} haſt Du bei uns alle Herzen
               gewonnen. Freundlich grüßt mich Dein Name aus den Schaufenſtern der
               Buchhandlungen.\pend
           
\pstart
           Was Du mir über Deine Stimmungen ſchreibſt, iſt gar ſeltſam. Daß auch Du dieſe Idee
               haſt, Dein Leben zu verlieren{[},{]}{ }{\pb}Du, deſſen Leben reich iſt, wie kein zweites, das
               ich kenne. So ſcheint es, daß \strikeout{\textcolor{gray}{×}} wir auf allen Stufen, bei allen Geſchicken, im Glück und Unglück das Gefühl
               haben, das Leben zu verlieren; und vielleicht verlieren wirs auch \substVorne{}\textsuperscript{a}\substDazwischen{}A\substHinten{}lle wirklich.\pend
           
\pstart
           Gern möchte ich Dich im Sommer wiederſehen, vorausgeſetzt, daß ich bis dahin noch in
               keinem Spital liege: Holland\oindex{Niederlande@\textbf{Niederlande}, \emph{A.PCLI}|pw}, Dänemark\oindex{Daenemark@\textbf{Dänemark}, \emph{A.PCLI}|pw}, wo Du willſt. Freilich wirſt Du bei unſerem
               Wiederſehen {\pb}merken, daß ſich Manches verändert
               hat.\pend
           
\pstart
           Und warum kommſt Du nicht nach \textsc{Paris\oindex{Paris@\textbf{Paris}, \emph{P.PPLC}|pw}}? \pend
           
\pstart
           Dem \textsc{Hugo\pwindex{Hofmannsthal, Hugo von 1874-02-01 – 1929-07-15@\textsc{Hofmannsthal, Hugo von} (1874-02-01 – 1929-07-15), \emph{Schriftsteller/Schriftstellerin}|pw}} thue ich \uline{nicht} Unrecht. Ich ſoll den \label{K_L02771-2v}\edtext{Artikel\pwindex{Gedichte von Stefan George@\emph{Gedichte von Stefan George}|pwv}}{\lemma{\textnormal{\emph{Artikel}}}\Cendnote{\textnormal{Hugo von Hofmannsthal\pwindex{Hofmannsthal, Hugo von 1874-02-01 – 1929-07-15@\textsc{Hofmannsthal, Hugo von} (1874-02-01 – 1929-07-15), \emph{Schriftsteller/Schriftstellerin}|pwk}: \emph{Gedichte von Stefan George}\pwindex{Gedichte von Stefan George@\emph{Gedichte von Stefan George}|pwk}. In: \emph{Die Zeit}\pwindex{Zeit. Wiener Wochenschrift@\emph{Die Zeit. Wiener Wochenschrift}|pwk}, Bd. 6, Nr. 77, 21. 3. 1896, S. 189–191.}}}\label{K_L02771-2} leſen, als handle er nicht
               von \textsc{St. Georges\pwindex{George, Stefan 17.07.1868 – 04.12.1933@\textsc{George, Stefan} (17.07.1868 – 04.12.1933), \emph{Schriftsteller/Schriftstellerin, Übersetzer/Übersetzerin}|pw}}. Ja, er handelt aber davon. Ich kann Form und Inhalt nicht ſcheiden, beſonders
               nicht bei einer Kritik. Und wenn die Form gut iſt, das Urtheil aber falſch, ſo iſts
               eine ſchlechte Kritik. Auch iſt die Form nicht gut, – verfluchte Manier! {\pb}Hoffentlich nimmſt Du das \label{K_L02771-3v}\edtext{Burgtheater\orgindex{Burgtheater@Burgtheater|pw}-Referat in der »Zeit\pwindex{Zeit. Wiener Wochenschrift@\emph{Die Zeit. Wiener Wochenschrift}|pw}«}{\lemma{\textnormal{\emph{Burgtheater-Referat … »Zeit«}}}\Cendnote{\textnormal{Das hätte 
                  bedeutet, dass Schnitzler alle Rezensionen der \emph{Zeit}\pwindex{Zeit. Wiener Wochenschrift@\emph{Die Zeit. Wiener Wochenschrift}|pwk} von Aufführungen im 
                  \emph{Burgtheater}\orgindex{Burgtheater@Burgtheater|pwk} verantwortet hätte. Dazu kam es nicht.}}}\label{K_L02771-3} an. Du biſt der
               geborene Kritiker – wahrhaftig und unbeſtechlich, ich meine ſeeliſch unbeſtechlich,
               nicht einmal ein \label{K_L02771-4v}\edtext{\begin{otherlanguage}{french}\textsc{emballé}\end{otherlanguage}}{\lemma{\textnormal{\emph{emballé}}}\Cendnote{\textnormal{französisch: Mitgerissener}}}\label{K_L02771-4}, wie
               ich. Und dann Du mit Deinem \strikeout{klug} klugen Urtheil und
               feinen Kunſtſinn! Nimms \strikeout{a} an! \strikeout{Da} Daß Du nicht journaliſtiſch thätig ſein kannſt, {\pb}iſt eine Deiner Wahnideen, die am Beſten durch die
               Praxis widerlegt werden. Auch ſchafft Dir eine regelmäßige kritiſche Thätigkeit
               gewiſſe Lebensgrenzen, – \begin{otherlanguage}{french}Barri\textsc{è}ren\end{otherlanguage}, welche Deine Gedanken verhindern, im Unendlichen Unfug zu
               treiben. Wenn Du genöthigt biſt, \textsc{Rudolf Lothar\pwindex{Lothar, Rudolf 23.2.1865 – 2.10.1943@\textsc{Lothar, Rudolf} (23.2.1865 – 2.10.1943), \emph{Schriftsteller/Schriftstellerin, Journalist/Journalistin, Theaterdirektor/Theaterdirektorin}|pw}} und \textsc{Davis\pwindex{Davis, Gustav 03.03.1856 – 21.08.1951@\textsc{Davis, Gustav} (03.03.1856 – 21.08.1951), \emph{Journalist/Journalistin, Herausgeber/Herausgeberin}|pw}} kritiſch zu behandeln, wirſt Du weniger an den Tod denken.\pend
           
\pstart
           Wie wenn Du mir ein Wort hierher ſchriebeſt? (\textsc{Niddastraſse 37}\oindex{Niddastrasse@\textbf{Niddastraße}, \emph{Straße (K.STR)}|pw}.) Das wäre ſchön\textcolor{gray}{.}\pend
           
\pstart
           {\pb}Iſt Dein \label{K_L02771-5v}\edtext{Stück\pwindex{Freiwild. Schauspiel in 3 Akten@\emph{Freiwild. Schauspiel in 3 Akten}|pwv} fertig}{\lemma{\textnormal{\emph{Stück fertig}}}\Cendnote{\textnormal{Es ging dem Ende zu. Schnitzler begann eine neue Niederschrift von \emph{Freiwild}\pwindex{Freiwild. Schauspiel in 3 Akten@\emph{Freiwild. Schauspiel in 3 Akten}|pwk} am 27. 4. 1896. Am 3. 5. 1896 las er das Stück\pwindex{Freiwild. Schauspiel in 3 Akten@\emph{Freiwild. Schauspiel in 3 Akten}|pwkv}{ }Felix Salten\pwindex{Salten, Felix 06.09.1869 – 08.10.1945@\textsc{Salten, Felix} (06.09.1869 – 08.10.1945), \emph{Schriftsteller/Schriftstellerin, Journalist/Journalistin, Chefredakteur/Chefredakteurin}|pwk} vor, dessen positive
                  Rückmeldung ihn bestärkte. Am 5. 6. 1896 hatte Schnitzler
                  das Stück\pwindex{Freiwild. Schauspiel in 3 Akten@\emph{Freiwild. Schauspiel in 3 Akten}|pwkv} »sozusagen
                     beendet«.}}}\label{K_L02771-5}? Kann man das Manuſkript\pwindex{Freiwild. Schauspiel in 3 Akten@\emph{Freiwild. Schauspiel in 3 Akten}|pwv} ſehen?\pend
           
\pstart
           Bitte, ſchick’ mir nach \textsc{Paris\oindex{Paris@\textbf{Paris}, \emph{P.PPLC}|pw}} die im Buchhandel erſchienenen \textsc{Anatol}-Sachen\pwindex{Anatol@\emph{Anatol}|pw}.\pend
           
\pstart
           Grüß’ Dich Gott, mein lieber Freund!\pend
           
\pstart
           Dein {\\[\baselineskip]}\spacefill\mbox{Paul Goldmann.}\pend
           \leftskip=0em{}
\pstart
           \noindent{}Gruß an \textsc{Richard\pwindex{Beer-Hofmann, Richard 1866-07-11 – 1945-09-26@\textsc{Beer-Hofmann, Richard} (1866-07-11 – 1945-09-26), \emph{Schriftsteller/Schriftstellerin}|pw}}.\pend
           
\pstart
           {\pb}Gefunden in einem alten deutſchen Myſtiker\pwindex{Angelus Silesius vor dem 25.12.1624 – 1677-07-09@\textsc{Angelus Silesius} (vor dem 25.12.1624 – 1677-07-09), \emph{Schriftsteller/Schriftstellerin, Mediziner/Medizinerin, Theologe/Theologin}|pwv}\pwindex{Angelus Silesius vor dem 25.12.1624 – 1677-07-09@\textsc{Angelus Silesius} (vor dem 25.12.1624 – 1677-07-09), \emph{Schriftsteller/Schriftstellerin, Mediziner/Medizinerin, Theologe/Theologin}|pwv}:\pend
           \stanza{} »\label{K_L02771-6v}\edtext{Der Zufall muß hinweg\pwindex{Cherubinischer Wandersmann@\emph{Cherubinischer Wandersmann}|pwv}}{\lemma{\textnormal{\emph{Der Zufall muß hinweg}}}\Cendnote{\textnormal{Epigramm 274 aus \emph{Geistreiche Sinn- und Schlussreime}\pwindex{Geistreiche Sinn- und Schlussreime@\emph{Geistreiche Sinn- und Schlussreime}|pwk} (1657) von Angelus
                           Silesius\pwindex{Angelus Silesius vor dem 25.12.1624 – 1677-07-09@\textsc{Angelus Silesius} (vor dem 25.12.1624 – 1677-07-09), \emph{Schriftsteller/Schriftstellerin, Mediziner/Medizinerin, Theologe/Theologin}|pwk}}}}\label{K_L02771-6}und aller falſcher
                        Schein,\pwindex{Cherubinischer Wandersmann@\emph{Cherubinischer Wandersmann}|pwv}Du mußt ganz weſentlich\pwindex{Cherubinischer Wandersmann@\emph{Cherubinischer Wandersmann}|pwv}und ungefärbet
                     ſein.\pwindex{Cherubinischer Wandersmann@\emph{Cherubinischer Wandersmann}|pwv}«\stanzaend{}
\pstart
           Und was ſagſt Du zu Frau \label{K_L02771-7v}\edtext{\textsc{Lou Andreas\pwindex{Andreas-Salome, Lou 12.02.1861 – 05.02.1937@\textsc{Andreas-Salomé, Lou} (12.02.1861 – 05.02.1937), \emph{Schriftsteller/Schriftstellerin}|pw}}}{\lemma{\textnormal{\emph{Lou Andreas}}}\Cendnote{\textnormal{\emph{Ruth}\pwindex{Ruth. Erzaehlung@\emph{Ruth. Erzählung}|pwk} hatte Schnitzler bereits am 10. 1. 1896 gelesen. Zu Lou Andreas-Salomé\pwindex{Andreas-Salome, Lou 12.02.1861 – 05.02.1937@\textsc{Andreas-Salomé, Lou} (12.02.1861 – 05.02.1937), \emph{Schriftsteller/Schriftstellerin}|pwk} dürfte zu dieser Zeit kein näherer
                     Kontakt bestanden haben.}}}\label{K_L02771-7}’ Buch »Ruth\pwindex{Ruth. Erzaehlung@\emph{Ruth. Erzählung}|pw}«? Hörſt Du etwas von ihr?\pend
           \selectlanguage{ngerman}\endnumbering\briefempfaengerindex{Schnitzler, Arthur@\textsc{Schnitzler, Arthur}!zzzGoldmann, Paul@\emph{von Paul Goldmann}!1896-04-091@{9. {[}4.{]} 1896}|)be}\mylabel{L02771h}  \normalsize

\doendnotes{C}
\bigskip
\vfill

\clearpage

\footnotesize

\lohead{\textsc{register}}

% Definiere theindex-Environment komplett neu ohne reledmac
\makeatletter
\renewenvironment{theindex}{%
  \section*{\indexname}%
  \setlength{\parindent}{0pt}%
  \setlength{\parskip}{0pt plus 0.3pt}%
  \let\item\@idxitem
}{%
  \clearpage
}
\makeatother

\IfFileExists{\jobname-pw.ind}{\input{\jobname-pw.ind}}{}

\end{document}

      