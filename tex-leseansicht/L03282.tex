%% latex-korrekturansicht-vorspann.tex
%% Vorspann für die Korrekturansicht.
%% Lädt die gemeinsame Datei latex-vorspann.tex mit gesetztem Schalter.

\newif\ifkorrekturansicht
\korrekturansichttrue

\input{../tex-inputs/latex-vorspann}


\section[ Felix Salten an Arthur Schnitzler, {[}8.? 9. 1898{]}]{L03282 Felix Salten an Arthur Schnitzler, {[}8.? 9. 1898{]}}
\nopagebreak\mylabel{L03282v}
\rehead{ }\normalsize\beginnumbering\briefempfaengerindex{Schnitzler, Arthur@\textsc{Schnitzler, Arthur}!zzzSalten, Felix@\emph{von Felix Salten}!1898-09-081@{{[}8.? 9. 1898{]}}|(be}
\toendnotes[C]{\smallbreak\pagebreak[2]}\Standort{CUL, Schnitzler, B 89, A 2.}
\physDesc{Brief, 1 Blatt, 1 Seite, 212 Zeichen
\newline{}Handschrift: Bleistift, lateinische Kurrent
\newline{}Schnitzler: mit Bleistift datiert: »Spt 98« 
\newline{}Ordnung: mit Bleistift von unbekannter Hand nummeriert: »106« }\toendnotes[C]{\smallbreak}
\pstart
           \noindent{}{\pb}Lieber Arthur, bin \label{K_L03282-1v}\edtext{heute}{\lemma{\textnormal{\emph{heute}}}\Cendnote{\textnormal{Schnitzler war seit 3. 9. 1898 in Wien\oindex{Wien@\textbf{Wien}, \emph{A.ADM2}|pwk}, am 5. 9. [1898]
                     nahm Salten\pwindex{Salten, Felix 06.09.1869 – 08.10.1945@\textsc{Salten, Felix} (06.09.1869 – 08.10.1945), \emph{Schriftsteller/Schriftstellerin, Journalist/Journalistin, Chefredakteur/Chefredakteurin}|pwk} Kontakt auf. Der nächste Brief vom 13. 9. 1898 schafft für dieses undatierte Schreiben eine zeitliche Grenze nach hinten. Der
                  5. 9. [1898] dürfte nicht gemeint sein, da 
                     für diesen Tag bereits ein anderes Schreiben existiert, in dem nicht von einem abendlichen Treffen die
                     Rede ist. Zum 9. 9. 1898 hat Schnitzler vermerkt,
                     dass die Theatervorstellungen wegen der Ermordung von Kaiserin Elisabeth\pwindex{Elisabeth von Oesterreich-Ungarn 24.12.1837 – 10.9.1898@\textsc{Elisabeth von Österreich-Ungarn} (24.12.1837 – 10.9.1898), \emph{König/Königin, Kaiser/Kaiserin}|pwk} abgesagt wurden. 
                     Damit verbleibt nur mehr der 8. 9. 1898 als 
                     möglicher Tag für dieses Korrespondenzstück. Schnitzler besuchte
                     \emph{Ein Wintermärchen}\pwindex{Wintermaerchen. Schauspiel in vier Aufzuegen@\emph{Ein Wintermärchen. Schauspiel in vier Aufzügen}|pwk} von Shakespeare\pwindex{Shakespeare, William 23.4.1564? – 03.05.1616@\textsc{Shakespeare, William} (23.4.1564? – 03.05.1616), \emph{Schauspieler/Schauspielerin, Dramatiker/Dramatikerin}|pwk} im
                     Burgtheater\oindex{Burgtheater@\textbf{Burgtheater}, \emph{S.THTR}|pwk}.
                  }}}\label{K_L03282-1} nicht im
               Theater, also leider auch nicht \label{K_L03282-2v}\edtext{in der
                  Stadt}{\lemma{\textnormal{\emph{in der
                  Stadt}}}\Cendnote{\textnormal{Salten\pwindex{Salten, Felix 06.09.1869 – 08.10.1945@\textsc{Salten, Felix} (06.09.1869 – 08.10.1945), \emph{Schriftsteller/Schriftstellerin, Journalist/Journalistin, Chefredakteur/Chefredakteurin}|pwk} wohnte in einem Außenbezirk, in Hietzing\oindex{XIII., Hietzing@\textbf{XIII., Hietzing}, \emph{A.ADM3}|pwk} im Südwesten Wiens\oindex{Wien@\textbf{Wien}, \emph{A.ADM2}|pwk}, nahe des Schönbrunner
                     Schlossparks\oindex{Schlosspark Schoenbrunn@\textbf{Schlosspark Schönbrunn}, \emph{Park (K.PRK)}|pwk}.}}}\label{K_L03282-2}. Vielleicht in den nächsten Tagen? Oder vielleicht
               gehen wir an einem der nächsten Nachmittage in Schönbrunn\oindex{Schlosspark Schoenbrunn@\textbf{Schlosspark Schönbrunn}, \emph{Park (K.PRK)}|pw} spazieren?\pend
           
\pstart
           Herzlichst Ihr {\\[\baselineskip]}\spacefill\mbox{Salten}\pend
           \leftskip=0em{}\selectlanguage{ngerman}\endnumbering\briefempfaengerindex{Schnitzler, Arthur@\textsc{Schnitzler, Arthur}!zzzSalten, Felix@\emph{von Felix Salten}!1898-09-081@{{[}8.? 9. 1898{]}}|)be}\mylabel{L03282h}  \normalsize

\doendnotes{C}
\bigskip
\vfill

\clearpage

\footnotesize

\lohead{\textsc{register}}

% Definiere theindex-Environment komplett neu ohne reledmac
\makeatletter
\renewenvironment{theindex}{%
  \section*{\indexname}%
  \setlength{\parindent}{0pt}%
  \setlength{\parskip}{0pt plus 0.3pt}%
  \let\item\@idxitem
}{%
  \clearpage
}
\makeatother

\IfFileExists{\jobname-pw.ind}{\input{\jobname-pw.ind}}{}

\end{document}

      