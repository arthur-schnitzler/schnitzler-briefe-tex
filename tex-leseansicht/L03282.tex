%% latex-leseansicht-vorspann.tex
%% Vorspann für die Leseansicht.
%% Lädt die gemeinsame Datei latex-vorspann.tex mit nicht gesetztem Schalter.

\newif\ifkorrekturansicht
\korrekturansichtfalse

\input{../tex-inputs/latex-vorspann}

\begin{center}
            \textcolor{red}{ENTWURF, NICHT FERTIG KORRIGIERT}
                      \end{center}
            
         
         \renewcommand{\erwaehntePersonen}{Personen:  Elisabeth von Österreich-Ungarn, Felix Salten, William Shakespeare}
         \renewcommand{\erwaehnteOrte}{Orte: Burgtheater, Schlosspark Schönbrunn, Wien, XIII., Hietzing}
         \renewcommand{\erwaehnteWerke}{Werke: Ein Wintermärchen. Schauspiel in vier Aufzügen}
               \section[ Felix Salten an Arthur Schnitzler, {[}8.? 9. 1898{]}]{ Felix Salten an Arthur Schnitzler, {[}8.? 9. 1898{]}}\nopagebreak\mylabel{v}\rehead{ }\begin{ledgroupsized}[t]{13cm}\normalsize\beginnumbering \toendnotes[C]{\smallbreak\pagebreak[2]} \Standort{CUL, Schnitzler, B 89, A 2.}
\physDesc{Brief, 1 Blatt, 1 Seite, 212 Zeichen
\newline{}Handschrift: Bleistift, lateinische Kurrent
\newline{}Schnitzler: mit Bleistift datiert: »Spt 98« 
\newline{}Ordnung: mit Bleistift von unbekannter Hand nummeriert: »106« }\toendnotes[C]{\smallbreak}\pstart
           \noindent{}{\pb}Lieber Arthur, bin \label{K_L03282-1v}\edtext{heute}{\lemma{\textnormal{\emph{heute}}}\Cendnote{\textnormal{Schnitzler\pwindex{Schnitzler, Arthur 15.05.1862 – 21.10.1931@\textsc{Schnitzler, Arthur} (15.05.1862 – 21.10.1931), \emph{Schriftsteller, Mediziner}|pwk} war seit 3. 9. 1898 in Wien\oindex{Wien@\textbf{Wien}|pwk}, am 5. 9. [1898]
                     nahm Salten\pwindex{Salten, Felix 06.09.1869 – 08.10.1945@\textsc{Salten, Felix} (06.09.1869 – 08.10.1945), \emph{Schriftsteller, Journalist}|pwk} Kontakt auf. Der nächste Brief vom 13. 9. 1898 schafft für dieses undatierte Schreiben eine zeitliche Grenze nach hinten. Der
                     5. 9. 1898 dürfte nicht gemeint sein, da 
                     für diesen Tag bereits ein anderes Schreiben existiert, in dem nicht von einem abendlichen Treffen die
                     Rede ist. Am 9. 9. 1898 vermerkt Schnitzler\pwindex{Schnitzler, Arthur 15.05.1862 – 21.10.1931@\textsc{Schnitzler, Arthur} (15.05.1862 – 21.10.1931), \emph{Schriftsteller, Mediziner}|pwk},
                     dass die Theatervorstellungen wegen der Ermordung von Kaiserin Elisabeth\pwindex{Elisabeth von Oesterreich-Ungarn 24.12.1837 – 10.9.1898@\textsc{Elisabeth von Österreich-Ungarn} (24.12.1837 – 10.9.1898), \emph{Königin, Kaiserin}|pwk} abgesagt wurden. 
                     Damit verbleibt nur mehr der 8. 9. 1898 als 
                     möglicher Tag für dieses Korrespondenzstück. Schnitzler\pwindex{Schnitzler, Arthur 15.05.1862 – 21.10.1931@\textsc{Schnitzler, Arthur} (15.05.1862 – 21.10.1931), \emph{Schriftsteller, Mediziner}|pwk} besuchte
                     \emph{Ein Wintermärchen}\pwindex{Shakespeare, William 23.4.1564? – 03.05.1616@\textsc{Shakespeare, William} (23.4.1564? – 03.05.1616), \emph{Schauspieler, Dramatiker}!Wintermaerchen. Schauspiel in vier Aufzuegen@\strich\emph{Ein Wintermärchen. Schauspiel in vier Aufzügen}|pwk} von Shakespeare\pwindex{Shakespeare, William 23.4.1564? – 03.05.1616@\textsc{Shakespeare, William} (23.4.1564? – 03.05.1616), \emph{Schauspieler, Dramatiker}|pwk} im
                     Burgtheater\oindex{Burgtheater@\textbf{Burgtheater}|pwk}.
                  }}}\label{K_L03282-1h} nicht im
               Theater, also leider auch nicht \label{K_L03282-2v}\edtext{in der
                  Stadt}{\lemma{\textnormal{\emph{in der
                  Stadt}}}\Cendnote{\textnormal{Salten\pwindex{Salten, Felix 06.09.1869 – 08.10.1945@\textsc{Salten, Felix} (06.09.1869 – 08.10.1945), \emph{Schriftsteller, Journalist}|pwk} wohnte in einem Außenbezirk, in Hietzing\oindex{XIII., Hietzing@\textbf{XIII., Hietzing}|pwk} im Südwesten Wien\oindex{Wien@\textbf{Wien}|pwk}s, nahe des Schönbrunner
                     Schlosspark\oindex{Schlosspark Schoenbrunn@\textbf{Schlosspark Schönbrunn}|pwk}s.}}}\label{K_L03282-2h}. Vielleicht in den nächsten Tagen? Oder vielleicht
               gehen wir an einem der nächsten Nachmittage in Schönbrunn\oindex{Schlosspark Schoenbrunn@\textbf{Schlosspark Schönbrunn}|pw} spazieren?\pend
           \pstart
           Herzlichst Ihr {\\[\baselineskip]}\spacefill\mbox{Salten}\pend
           \leftskip=0em{}
         
         \endnumbering\mylabel{h}\end{ledgroupsized}  \newcommand{\dateiname}{L03282}\newcommand{\titel}{Felix Salten an Arthur Schnitzler, [8.? 9. 1898]}\newcommand{\editorInnen}{Martin Anton Müller und Laura Untner}%% latex-leseansicht-abspann.tex
%% Abspann für die Leseansicht.
%% Der Schalter \ifkorrekturansicht ist bereits durch den Vorspann gesetzt.

%% latex-abspann.tex
%% Gemeinsamer Abspann für Korrekturansicht und Leseansicht.
%% Setzt den Schalter \ifkorrekturansicht voraus (gesetzt in den
%% einbindenden Dateien latex-korrekturansicht-abspann.tex bzw.
%% latex-leseansicht-abspann.tex).
%% ---------------------------------------------------------------

\normalsize

% Das esempio-Environment wird nur in der Leseansicht benötigt
\ifkorrekturansicht\else
\newenvironment{esempio}[3]%
{
    \vspace{1.5ex}
    \rlap{\underline{#1}}
    \par
    \setlength{\parindent}{0cm}
    \nopagebreak
    \leftskip=#2cm
    \rightskip=#3cm
}
{
    \par
}
\fi

\doendnotes{C}
\bigskip
\vfill

\clearpage

\footnotesize

\ifkorrekturansicht
  \lohead{\textsc{register}}
\fi

% theindex-Environment neu definieren ohne reledmac
\makeatletter
\renewenvironment{theindex}{%
  \ifkorrekturansicht
    \section*{\indexname}%
  \else
    \subsubsection*{Index der erwähnten Entitäten}%
  \fi
  \setlength{\parindent}{0pt}%
  \setlength{\parskip}{0pt plus 0.3pt}%
  \let\item\@idxitem
}{%
  \ifkorrekturansicht\clearpage\fi
}
\makeatother

\IfFileExists{\jobname-pw.ind}{\input{\jobname-pw.ind}}{}

% Quellenangabe nur in der Leseansicht
\ifkorrekturansicht\else
% Fallback-Definitionen, falls die .tex-Datei \titel etc. nicht gesetzt hat
\providecommand{\titel}{}
\providecommand{\editorInnen}{}
\providecommand{\dateiname}{\jobname}

\vspace{3cm}

\vfill

\footnotesize
\textsc{Quelle}: \titel. Herausgegeben von {\editorInnen}. In: \emph{Arthur Schnitzler: Briefwechsel mit Autorinnen und Autoren}.
 Digitale Edition, https://schnitzler-briefe.acdh.oeaw.ac.at/{\dateiname}.html (Stand \today)
\fi

\end{document}


      