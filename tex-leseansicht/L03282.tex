%% latex-leseansicht-vorspann.tex
%% Vorspann für die Leseansicht.
%% Lädt die gemeinsame Datei latex-vorspann.tex mit nicht gesetztem Schalter.

\newif\ifkorrekturansicht
\korrekturansichtfalse

\input{../tex-inputs/latex-vorspann}


\section[ Felix Salten an Arthur Schnitzler, [8.? 9. 1898]]{L03282 Felix Salten an Arthur Schnitzler,  [8.? 9. 1898]}
\nopagebreak\mylabel{L03282v}
\rehead{ }\normalsize\beginnumbering\briefempfaengerindex{Schnitzler, Arthur@\textsc{Schnitzler, Arthur}!zzzSalten, Felix@\emph{von Felix Salten}!1898-09-081@{{[}8.? 9. 1898{]}}|(be}
\toendnotes[C]{\smallbreak\pagebreak[2]}
\correspDesc{Versand  durch Felix Salten am [8.? 9. 1898] in Wien
\newline{}Erhalt  durch Arthur Schnitzler am [8.? 9. 1898] in Wien}\toendnotes[C]{\smallbreak}
\Standort{CUL, Schnitzler, B 89, A 2.}
\physDesc{Brief, 1 Blatt, 1 Seite, 212 Zeichen
\newline{}Handschrift: Bleistift, lateinische Kurrent
\newline{}Schnitzler: mit Bleistift datiert: »Spt 98« 
\newline{}Ordnung: mit Bleistift von unbekannter Hand nummeriert: »106« }\toendnotes[C]{\smallbreak}
\pstart
           \noindent{}{\pb}Lieber Arthur, bin \label{K_L03282-1v}\edtext{heute}{\lemma{\textnormal{\emph{heute}}}\Cendnote{\textnormal{Schnitzler war seit 3. 9. 1898 in Wien\oindex{Wien@\textbf{Wien}, \emph{Verwaltungsgebiet}|pwk}, am XXXX Auszeichnungsfehler: Dokument L03281 nicht gefunden
                     nahm Salten\pwindex{Salten, Felix 6.\,9.\,1869 Budapest – 8.\,10.\,1945 Zürich@\textsc{Salten, Felix} (6.\,9.\,1869 Budapest – 8.\,10.\,1945 Zürich), \emph{Schriftsteller, Journalist, Chefredakteur}|pwk} Kontakt auf. Der nächste Brief vom XXXX Auszeichnungsfehler: Dokument L03283 nicht gefunden schafft für dieses undatierte Schreiben eine zeitliche Grenze nach hinten. Der
                  XXXX Auszeichnungsfehler: Dokument L03281 nicht gefunden dürfte nicht gemeint sein, da 
                     für diesen Tag bereits ein anderes Schreiben existiert, in dem nicht von einem abendlichen Treffen die
                     Rede ist. Zum 9. 9. 1898 hat Schnitzler vermerkt,
                     dass die Theatervorstellungen wegen der Ermordung von Kaiserin Elisabeth\pwindex{Elisabeth von Österreich-Ungarn 24.\,12.\,1837 München – 10.\,9.\,1898 Genf@\textsc{Elisabeth von Österreich-Ungarn} (24.\,12.\,1837 München – 10.\,9.\,1898 Genf), \emph{Königin, Kaiserin}|pwk} abgesagt wurden. 
                     Damit verbleibt nur mehr der 8. 9. 1898 als 
                     möglicher Tag für dieses Korrespondenzstück. Schnitzler besuchte
                     \emph{Ein Wintermärchen}\pwindex{Shakespeare, William 23.\,4.\,1564? Stratford-upon-Avon – 3.\,5.\,1616 ebd.@\textsc{Shakespeare, William} (23.\,4.\,1564? Stratford-upon-Avon – 3.\,5.\,1616 ebd.), \emph{Schauspieler, Dramatiker}!Wintermärchen. Schauspiel in vier Aufzügen@\strich\emph{Ein Wintermärchen. Schauspiel in vier Aufzügen}|pwk} von Shakespeare\pwindex{Shakespeare, William 23.\,4.\,1564? Stratford-upon-Avon – 3.\,5.\,1616 ebd.@\textsc{Shakespeare, William} (23.\,4.\,1564? Stratford-upon-Avon – 3.\,5.\,1616 ebd.), \emph{Schauspieler, Dramatiker}|pwk} im
                     Burgtheater\oindex{Wien@\textbf{Wien}!I., Innere Stadt@\textbf{I., Innere Stadt}!Burgtheater@\textbf{Burgtheater}, \emph{Theater}|pwk}.
                  }}}\label{K_L03282-1} nicht im
               Theater, also leider auch nicht \label{K_L03282-2v}\edtext{in der
                  Stadt}{\lemma{\textnormal{\emph{in der
                  Stadt}}}\Cendnote{\textnormal{Salten\pwindex{Salten, Felix 6.\,9.\,1869 Budapest – 8.\,10.\,1945 Zürich@\textsc{Salten, Felix} (6.\,9.\,1869 Budapest – 8.\,10.\,1945 Zürich), \emph{Schriftsteller, Journalist, Chefredakteur}|pwk} wohnte in einem Außenbezirk, in Hietzing\oindex{XIII., Hietzing@\textbf{XIII., Hietzing}, \emph{Verwaltungsgebiet}|pwk} im Südwesten Wiens\oindex{Wien@\textbf{Wien}, \emph{Verwaltungsgebiet}|pwk}, nahe des Schönbrunner
                     Schlossparks\oindex{Wien@\textbf{Wien}!XIII., Hietzing@\textbf{XIII., Hietzing}!Schlosspark Schönbrunn@\textbf{Schlosspark Schönbrunn}, \emph{Park}|pwk}.}}}\label{K_L03282-2}. Vielleicht in den nächsten Tagen? Oder vielleicht
               gehen wir an einem der nächsten Nachmittage in Schönbrunn\oindex{Wien@\textbf{Wien}!XIII., Hietzing@\textbf{XIII., Hietzing}!Schlosspark Schönbrunn@\textbf{Schlosspark Schönbrunn}, \emph{Park}|pw} spazieren?\pend
           
\pstart
           Herzlichst Ihr {\\[\baselineskip]}\spacefill\mbox{Salten}\pend
           \leftskip=0em{}\selectlanguage{ngerman}\endnumbering\briefempfaengerindex{Schnitzler, Arthur@\textsc{Schnitzler, Arthur}!zzzSalten, Felix@\emph{von Felix Salten}!1898-09-081@{{[}8.? 9. 1898{]}}|)be}\mylabel{L03282h}  \newcommand{\dateiname}{L03282}\newcommand{\titel}{Felix Salten an Arthur Schnitzler, [8.? 9. 1898]}\newcommand{\editorInnen}{Martin Anton Müller und Laura Untner}%% latex-leseansicht-abspann.tex
%% Abspann für die Leseansicht.
%% Der Schalter \ifkorrekturansicht ist bereits durch den Vorspann gesetzt.

%% latex-abspann.tex
%% Gemeinsamer Abspann für Korrekturansicht und Leseansicht.
%% Setzt den Schalter \ifkorrekturansicht voraus (gesetzt in den
%% einbindenden Dateien latex-korrekturansicht-abspann.tex bzw.
%% latex-leseansicht-abspann.tex).
%% ---------------------------------------------------------------

\normalsize

% Das esempio-Environment wird nur in der Leseansicht benötigt
\ifkorrekturansicht\else
\newenvironment{esempio}[3]%
{
    \vspace{1.5ex}
    \rlap{\underline{#1}}
    \par
    \setlength{\parindent}{0cm}
    \nopagebreak
    \leftskip=#2cm
    \rightskip=#3cm
}
{
    \par
}
\fi

\doendnotes{C}
\bigskip
\vfill

\clearpage

\footnotesize

\ifkorrekturansicht
  \lohead{\textsc{register}}
\fi

% theindex-Environment neu definieren ohne reledmac
\makeatletter
\renewenvironment{theindex}{%
  \ifkorrekturansicht
    \section*{\indexname}%
  \else
    \subsubsection*{Index der erwähnten Entitäten}%
  \fi
  \setlength{\parindent}{0pt}%
  \setlength{\parskip}{0pt plus 0.3pt}%
  \let\item\@idxitem
}{%
  \ifkorrekturansicht\clearpage\fi
}
\makeatother

\IfFileExists{\jobname-pw.ind}{\input{\jobname-pw.ind}}{}

% Quellenangabe nur in der Leseansicht
\ifkorrekturansicht\else
% Fallback-Definitionen, falls die .tex-Datei \titel etc. nicht gesetzt hat
\providecommand{\titel}{}
\providecommand{\editorInnen}{}
\providecommand{\dateiname}{\jobname}

\vspace{3cm}

\vfill

\footnotesize
\textsc{Quelle}: \titel. Herausgegeben von {\editorInnen}. In: \emph{Arthur Schnitzler: Briefwechsel mit Autorinnen und Autoren}.
 Digitale Edition, https://schnitzler-briefe.acdh.oeaw.ac.at/{\dateiname}.html (Stand \today)
\fi

\end{document}


