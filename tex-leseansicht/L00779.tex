%% latex-leseansicht-vorspann.tex
%% Vorspann für die Leseansicht.
%% Lädt die gemeinsame Datei latex-vorspann.tex mit nicht gesetztem Schalter.

\newif\ifkorrekturansicht
\korrekturansichtfalse

\input{../tex-inputs/latex-vorspann}


         
         \newcommand{\erwaehntePersonen}{Personen: }
         \newcommand{\erwaehnteInstitutionen}{}
         \newcommand{\erwaehnteOrte}{}
         \newcommand{\erwaehnteWerke}{
               \section[Richard Beer-Hofmann an Arthur Schnitzler, Februar 1898]{ Richard Beer-Hofmann an Arthur Schnitzler, Februar 1898}\nopagebreak\mylabel{v}\rehead{ }\begin{ledgroupsized}[t]{13cm}\normalsize\beginnumbering \toendnotes[C]{\smallbreak\pagebreak[2]} \Standort{CUL, Schnitzler, B 8.}
\physDesc{Notiz auf Konzeptpapier2 Blätter, 3 Seiten
\newline{}Handschrift: Bleistift, deutsche Kurrent
\newline{}Schnitzler: mit Bleistift datiert: »Feber 98« }\buchAbdrucke{\weitereDrucke{Arthur Schnitzler, Richard Beer-Hofmann: \emph{Briefwechsel 1891–1931}. Hg. Konstanze Fliedl. Wien, Zürich: \emph{Europaverlag} 1992, S. 115–116.} }\toendnotes[C]{\smallbreak}\pstart
           \noindent{}{\pb}\uline{\label{K_L00779-1v}\edtext{Der Andere\textcolor{red}{\textsuperscript{XXXX indx}}}{\lemma{\textnormal{\emph{Der Andere}}}\Cendnote{\textnormal{hier und in Folge Überlegungen,
                     welche Texte Schnitzler\pwindex{\textcolor{red}{\textsuperscript{XXXX1 indx}}|pwk} für seine erste
                     Sammlung von Novellen verwenden solle, die wenige Wochen später als \emph{Die Frau des Weisen}\textcolor{red}{\textsuperscript{XXXX indx}} erschien}}}\label{K_L00779-1h}.}\pend
           \pstart
           Wenn möglich anstatt »Gattin\textcolor{red}{\textsuperscript{XXXX indx}}«
               etwas anderes.\pend
           \pstart
           »\uline{blonde}{ }\uline{junge}{ }\uline{schöne} Mann\textcolor{red}{\textsuperscript{XXXX indx}}«\pend
           \pstart
           Jedenfalls aufnehmen in die Sa{\geminationm}lung\pend
           \pstart
           \numberlinefalse{}\centering{}–\numberlinetrue{}\pend
           \pstart
           \noindent{}\uline{Amerika\textcolor{red}{\textsuperscript{XXXX indx}}}\pend
           \pstart
           »Sie liegt mir zu Füßen den Lockenkopf
                  an mein Knie gelehnt\textcolor{red}{\textsuperscript{XXXX indx}}«. Das ist aber schrecklich.\pend
           \pstart
           »die süße weiße Hautstelle hinter dem
                  Ohr\textcolor{red}{\textsuperscript{XXXX indx}}«\pend
           \pstart
           »Eine Fülle von Erinnerungen steigt in
                  mir auf\textcolor{red}{\textsuperscript{XXXX indx}}«\pend
           \pstart
           vorher noch »und stille ist’s im
                  Gemach\textcolor{red}{\textsuperscript{XXXX indx}}«\pend
           \pstart
           {\pb}\strikeout{Die kleine}\textcolor{red}{\textsuperscript{XXXX indx}}\pend
           \pstart
           {\pb}Könnte aufgenommen werden wenn
                  \introOben{}stark\introOben{}{ }\uline{überarbeitet}. Aber es sind so viele Sachen drinn die
               wegmüssten.\pend
           \pstart
           »wie von ihren rothen Lippen der Ruf
                  erschallte.\textcolor{red}{\textsuperscript{XXXX indx}}« u. s. w.\pend
           \noindent\rule{\textwidth}{0.5pt}\pstart
           Bei »mein \uline{Freund Ypsilon}\textcolor{red}{\textsuperscript{XXXX indx}}« ist sehr schad um die Idee. \uline{Aber gewiss} nicht
               aufnehmen\pend
           \noindent\rule{\textwidth}{0.5pt}\pstart
           »Die kleine Komödie\textcolor{red}{\textsuperscript{XXXX indx}}«\pend
           \pstart
           etwas kürzen – nicht viel und aufnehmen. Sie ist anspruchslos und hat keinen
               pretentiösen Ton\pend
           
         
         \endnumbering\mylabel{h}\end{ledgroupsized}  \newcommand{\dateiname}{L00779}\newcommand{\titel}{Richard Beer-Hofmann an Arthur Schnitzler, Februar 1898}\newcommand{\editorInnen}{Martin Anton Müller und Gerd-Hermann Susen}%% latex-leseansicht-abspann.tex
%% Abspann für die Leseansicht.
%% Der Schalter \ifkorrekturansicht ist bereits durch den Vorspann gesetzt.

%% latex-abspann.tex
%% Gemeinsamer Abspann für Korrekturansicht und Leseansicht.
%% Setzt den Schalter \ifkorrekturansicht voraus (gesetzt in den
%% einbindenden Dateien latex-korrekturansicht-abspann.tex bzw.
%% latex-leseansicht-abspann.tex).
%% ---------------------------------------------------------------

\normalsize

% Das esempio-Environment wird nur in der Leseansicht benötigt
\ifkorrekturansicht\else
\newenvironment{esempio}[3]%
{
    \vspace{1.5ex}
    \rlap{\underline{#1}}
    \par
    \setlength{\parindent}{0cm}
    \nopagebreak
    \leftskip=#2cm
    \rightskip=#3cm
}
{
    \par
}
\fi

\doendnotes{C}
\bigskip
\vfill

\clearpage

\footnotesize

\ifkorrekturansicht
  \lohead{\textsc{register}}
\fi

% theindex-Environment neu definieren ohne reledmac
\makeatletter
\renewenvironment{theindex}{%
  \ifkorrekturansicht
    \section*{\indexname}%
  \else
    \subsubsection*{Index der erwähnten Entitäten}%
  \fi
  \setlength{\parindent}{0pt}%
  \setlength{\parskip}{0pt plus 0.3pt}%
  \let\item\@idxitem
}{%
  \ifkorrekturansicht\clearpage\fi
}
\makeatother

\IfFileExists{\jobname-pw.ind}{\input{\jobname-pw.ind}}{}

% Quellenangabe nur in der Leseansicht
\ifkorrekturansicht\else
% Fallback-Definitionen, falls die .tex-Datei \titel etc. nicht gesetzt hat
\providecommand{\titel}{}
\providecommand{\editorInnen}{}
\providecommand{\dateiname}{\jobname}

\vspace{3cm}

\vfill

\footnotesize
\textsc{Quelle}: \titel. Herausgegeben von {\editorInnen}. In: \emph{Arthur Schnitzler: Briefwechsel mit Autorinnen und Autoren}.
 Digitale Edition, https://schnitzler-briefe.acdh.oeaw.ac.at/{\dateiname}.html (Stand \today)
\fi

\end{document}


      