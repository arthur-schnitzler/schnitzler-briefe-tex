%% latex-korrekturansicht-vorspann.tex
%% Vorspann für die Korrekturansicht.
%% Lädt die gemeinsame Datei latex-vorspann.tex mit gesetztem Schalter.

\newif\ifkorrekturansicht
\korrekturansichttrue

\input{../tex-inputs/latex-vorspann}


\section[Richard Beer-Hofmann an Arthur Schnitzler, {[}Februar 1898{]}]{L00779 Richard Beer-Hofmann an Arthur Schnitzler, {[}Februar 1898{]}}
\nopagebreak\mylabel{L00779v}
\rehead{ }\normalsize\beginnumbering\briefempfaengerindex{Schnitzler, Arthur@\textsc{Schnitzler, Arthur}!zzzBeer-Hofmann, Richard@\emph{von Richard Beer-Hofmann}!1898-02-281@{{[}Februar 1898{]}}|(be}
\toendnotes[C]{\smallbreak\pagebreak[2]}\Standort{CUL, Schnitzler, B 8.}
\physDesc{Sonderfall, 2 Blätter, 3 Seiten, 698 Zeichen (Notiz auf Konzeptpapier )
\newline{}Handschrift: Bleistift, deutsche Kurrent
\newline{}Schnitzler: mit Bleistift datiert: »Feber 98« }
\buchAbdrucke{\weitereDrucke{Arthur Schnitzler, Richard Beer-Hofmann: \emph{Briefwechsel 1891–1931}. Wien, Zürich: \emph{Europaverlag} 1992, S. 115–116.} }\toendnotes[C]{\smallbreak}
\pstart
           \noindent{}{\pb}\uline{\label{K_L00779-1v}\edtext{Der Andere\pwindex{Andere. Aus dem Tagebuch eines Hinterbliebenen@\emph{Der Andere. Aus dem Tagebuch eines Hinterbliebenen}|pw}}{\lemma{\textnormal{\emph{Der Andere}}}\Cendnote{\textnormal{hier und in Folge Überlegungen,
                     welche Texte Schnitzler für seine erste
                     Sammlung von Novellen verwenden solle, die wenige Wochen später als \emph{Die Frau des Weisen}\pwindex{Frau des Weisen. Novelletten@\emph{Die Frau des Weisen. Novelletten}|pwk} erschien}}}\label{K_L00779-1}.}\pend
           
\pstart
           Wenn möglich anstatt »Gattin\pwindex{Andere. Aus dem Tagebuch eines Hinterbliebenen@\emph{Der Andere. Aus dem Tagebuch eines Hinterbliebenen}|pwv}«
               etwas anderes.\pend
           
\pstart
           »\uline{blonde}{ }\uline{junge}{ }\uline{schöne} Mann\pwindex{Andere. Aus dem Tagebuch eines Hinterbliebenen@\emph{Der Andere. Aus dem Tagebuch eines Hinterbliebenen}|pwv}«\pend
           
\pstart
           Jedenfalls aufnehmen in die Sa{\geminationm}lung\pend
           
\pstart
           \numberlinefalse{}\centering{}–\numberlinetrue{}\pend
           
\pstart
           \uline{Amerika\pwindex{Amerika@\emph{Amerika}|pw}}\pend
           
\pstart
           »Sie liegt mir zu Füßen den
                  Lockenkopf an mein Knie gelehnt\pwindex{Amerika@\emph{Amerika}|pwv}«. Das ist aber schrecklich.\pend
           
\pstart
           »die süße weiße Hautstelle hinter
                  dem Ohr\pwindex{Amerika@\emph{Amerika}|pwv}«\pend
           
\pstart
           »Eine Fülle von Erinnerungen steigt
                  in mir auf\pwindex{Amerika@\emph{Amerika}|pwv}«\pend
           
\pstart
           vorher noch »und stille ist’s im
                  Gemach\pwindex{Amerika@\emph{Amerika}|pwv}«\pend
           
\pstart
           {\pb}\strikeout{Die kleine}\pwindex{kleine Komoedie@\emph{Die kleine Komödie}|pw}\pend
           
\pstart
           {\pb}Könnte aufgenommen werden wenn
                  \introOben{}stark\introOben{}{ }\uline{überarbeitet}. Aber es sind so viele Sachen drinn die
               wegmüssten.\pend
           
\pstart
           »wie von ihren rothen Lippen der
                  Ruf erschallte.\pwindex{Amerika@\emph{Amerika}|pwv}« u. s. w.\pend
           \noindent\rule{\textwidth}{0.5pt}
\pstart
           Bei »mein \uline{Freund
                     Ypsilon}\pwindex{Mein Freund Ypsilon. Aus den Papieren eines Arztes@\emph{Mein Freund Ypsilon. Aus den Papieren eines Arztes}|pw}« ist sehr schad um die Idee. \uline{Aber gewiss} nicht
               aufnehmen\pend
           \noindent\rule{\textwidth}{0.5pt}
\pstart
           »Die kleine Komödie\pwindex{kleine Komoedie@\emph{Die kleine Komödie}|pw}«\pend
           
\pstart
           etwas kürzen – nicht viel und aufnehmen. Sie ist anspruchslos und hat keinen
               pretentiösen Ton\pend
           \selectlanguage{ngerman}\endnumbering\briefempfaengerindex{Schnitzler, Arthur@\textsc{Schnitzler, Arthur}!zzzBeer-Hofmann, Richard@\emph{von Richard Beer-Hofmann}!1898-02-011@{{[}Februar 1898{]}}|)be}\mylabel{L00779h}  \normalsize

\doendnotes{C}
\bigskip
\vfill

\clearpage

\footnotesize

\lohead{\textsc{register}}

% Definiere theindex-Environment komplett neu ohne reledmac
\makeatletter
\renewenvironment{theindex}{%
  \section*{\indexname}%
  \setlength{\parindent}{0pt}%
  \setlength{\parskip}{0pt plus 0.3pt}%
  \let\item\@idxitem
}{%
  \clearpage
}
\makeatother

\IfFileExists{\jobname-pw.ind}{\input{\jobname-pw.ind}}{}

\end{document}

      