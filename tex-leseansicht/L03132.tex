%% latex-leseansicht-vorspann.tex
%% Vorspann für die Leseansicht.
%% Lädt die gemeinsame Datei latex-vorspann.tex mit nicht gesetztem Schalter.

\newif\ifkorrekturansicht
\korrekturansichtfalse

\input{../tex-inputs/latex-vorspann}


         
         \renewcommand{\erwaehntePersonen}{Personen: Felix Salten, Philipp Salzmann}
         \renewcommand{\erwaehnteOrte}{Orte: Wien}
         \renewcommand{\erwaehnteWerke}{}
               \section[ Felix Salten an Arthur Schnitzler, {[}24. 1. 1894{]}]{ Felix Salten an Arthur Schnitzler, {[}24. 1. 1894{]}}\nopagebreak\mylabel{v}\rehead{ }\begin{ledgroupsized}[t]{13cm}\normalsize\beginnumbering\briefempfaengerindex{Schnitzler, Arthur@\textsc{Schnitzler, Arthur}!zzzSalten, Felix@\emph{von Felix Salten}!1894-01-242@{{[}24. 1. 1894{]}}|(be} \toendnotes[C]{\smallbreak\pagebreak[2]} \Standort{CUL, Schnitzler, B 89, A 1.}
\physDesc{Brief, 1 Blatt, 4 Seiten, 2240 Zeichen
\newline{}Handschrift: Bleistift, lateinische Kurrent
\newline{}Schnitzler: mit Bleistift datiert: »24. 1. 94. « 
\newline{}Ordnung: mit Bleistift von unbekannter Hand nummeriert: »34« }\toendnotes[C]{\smallbreak}\pstart
           \raggedleft{}{\pb}\uline{Mittwoch}.\pend
           \pstart
           Lieber Doctor Schnitzler! Ich sage Ihnen vielen Dank
               für Ihre freundlichen \label{K_L03132-1v}\edtext{Grüße}{\lemma{\textnormal{\emph{Grüße}}}\Cendnote{\textnormal{Evtl. Arthur Schnitzler an Felix Salten, [23. 1. 1894?].
               }}}\label{K_L03132-1h}. Meinem Papa\pwindex{Salzmann, Philipp 1831-12-24 – 1905-04-02@\textsc{Salzmann, Philipp} (1831-12-24 – 1905-04-02), \emph{Bergbauunternehmer, Projektemacher}|pwv} habe ich, – wie schon oft vorher – auch gestern wieder Vorstellungen Ihretwegen gemacht. Er
               beruft sich darauf, dass er jetzt gerade sehr viel Pech in seinem 
               Geschäft und mit allen möglichen
               Widerwärtigkeiten zu kämpfen habe, die er nicht hat voraussehen können; und bittet
               Sie um \label{K_L03132-2v}\edtext{Entschuldigung und um ein wenig Geduld}{\lemma{\textnormal{\emph{Entschuldigung … Geduld}}}\Cendnote{\textnormal{Dies
               deutet auf eine finanzielle Schuld Saltens\pwindex{Salten, Felix 06.09.1869 – 08.10.1945@\textsc{Salten, Felix} (06.09.1869 – 08.10.1945), \emph{Schriftsteller, Journalist, Chefredakteur}|pwk} gegenüber Schnitzler\pwindex{Schnitzler, Arthur 15.05.1862 – 21.10.1931@\textsc{Schnitzler, Arthur} (15.05.1862 – 21.10.1931), \emph{Schriftsteller, Mediziner}|pwk} hin, 
               die dieser nicht rechtzeitig beglichen hat.}}}\label{K_L03132-2h}. Ich selbst {\pb}empfinde diese Affaire am
               schmerzlichsten, warum machen Sie eine Schwenkung weg von mir? Ich
               weiss recht gut, dass diese Sache nicht der Hauptgrund ist, obwol sie dazu beitragen
               mag, eine vorhandene Verstimmung zu vermehren. Ich weiss dass Sie in \label{K_L03132-4v}\edtext{künstlerischer Beziehung}{\lemma{\textnormal{\emph{künstlerischer Beziehung}}}\Cendnote{\textnormal{Vgl. A. S.: \emph{Tagebuch}, 20. 1. 1894.
               }}}\label{K_L03132-4h} in mich Erwartungen setzen, die ich noch nicht eingelöst habe. Aber glauben
               Sie, der Sie mich kennen, dass ich dadurch nicht noch viel mehr herabgedrückt werde,
               und noch mehr leide? Sie kennen meine Situation, Sie sehen es jetzt selbst mit an,
               wie ich für jeden ange{\pb}nehmen Tag durch nachträgliche Plackereien zu leiden habe, wie ich durch eine
               mühsame Reconstruction unseres Hauswesens in allen Studien, u. Lebensbedingungen auf
               Schritt und Tritt gehemmt, zurückbleiben musste, dazu kommt noch das langsame Tempo,
               in dem mein Talent arbeitet, ein Tempo, das sehr
               vornehm sein mag, wenn ich \introOben{}\textcolor{gray}{auch}\introOben{} überhaupt von Talent reden kann. –\pend
           \pstart
           Dass \uline{ich} Ihnen ferne geblieben
               lag wol mehr an den Umständen der letzten Wochen, als an mir. Dass ich Ihnen von
               meiner \label{K_L03132-5v}\edtext{Krankheit}{\lemma{\textnormal{\emph{Krankheit}}}\Cendnote{\textnormal{nicht ermittelt}}}\label{K_L03132-5h} keine Mittheilung
               machte, geschah, weil ich in solcher vermehrter {\pb}Verstimmung nicht für Sie zu
               taugen schien, dann weil ich weiss, dass Ihnen die Behandlung solcher Sachen nicht
               gerade angenehm ist, und endlich, weil ich doch hoffte bis Sonntag wieder soweit zu sein, um Sie zu treffen.\pend
           \pstart
           Jedenfalls danke ich ihnen herzlich für Ihre Grüße von gestern. Ich wäre froh, wenn es zwischen uns nicht mehr der Worte
               bedürfte, um uns\strikeout{er} unserer Gesinnung zu versichern.
               Vielleicht bin \uline{ich} übrigens diesmal Schuld, und war
               der Ton in Ihrem Brief nur eine eingebildete
               und keine thatsächliche Veranlaßung.\pend
           \pstart
           Ich hoffe diesen Winter doch noch mit einem Positivum zu schließen, und bleibe
               bis auf Wiedersehen Ihr {\\[\baselineskip]}unveränderlicher {\\[\baselineskip]}\spacefill\mbox{Salten}\pend
           \leftskip=0em{}\pstart
           \noindent{}Ich kann seit gestern schon auf eine Stunde
                  ausgehen, und \label{K_L03132-6v}\edtext{besuche Sie}{\lemma{\textnormal{\emph{besuche Sie}}}\Cendnote{\textnormal{Nachweislich sahen sie sich erst am 28. 1. 1894
                     wieder.}}}\label{K_L03132-6h} vielleicht morgen.\pend
           
         
         \endnumbering\mylabel{h}\end{ledgroupsized}  \newcommand{\dateiname}{L03132}\newcommand{\titel}{Felix Salten an Arthur Schnitzler, [24. 1. 1894]}\newcommand{\editorInnen}{Martin Anton Müller und Laura Untner}%% latex-leseansicht-abspann.tex
%% Abspann für die Leseansicht.
%% Der Schalter \ifkorrekturansicht ist bereits durch den Vorspann gesetzt.

%% latex-abspann.tex
%% Gemeinsamer Abspann für Korrekturansicht und Leseansicht.
%% Setzt den Schalter \ifkorrekturansicht voraus (gesetzt in den
%% einbindenden Dateien latex-korrekturansicht-abspann.tex bzw.
%% latex-leseansicht-abspann.tex).
%% ---------------------------------------------------------------

\normalsize

% Das esempio-Environment wird nur in der Leseansicht benötigt
\ifkorrekturansicht\else
\newenvironment{esempio}[3]%
{
    \vspace{1.5ex}
    \rlap{\underline{#1}}
    \par
    \setlength{\parindent}{0cm}
    \nopagebreak
    \leftskip=#2cm
    \rightskip=#3cm
}
{
    \par
}
\fi

\doendnotes{C}
\bigskip
\vfill

\clearpage

\footnotesize

\ifkorrekturansicht
  \lohead{\textsc{register}}
\fi

% theindex-Environment neu definieren ohne reledmac
\makeatletter
\renewenvironment{theindex}{%
  \ifkorrekturansicht
    \section*{\indexname}%
  \else
    \subsubsection*{Index der erwähnten Entitäten}%
  \fi
  \setlength{\parindent}{0pt}%
  \setlength{\parskip}{0pt plus 0.3pt}%
  \let\item\@idxitem
}{%
  \ifkorrekturansicht\clearpage\fi
}
\makeatother

\IfFileExists{\jobname-pw.ind}{\input{\jobname-pw.ind}}{}

% Quellenangabe nur in der Leseansicht
\ifkorrekturansicht\else
% Fallback-Definitionen, falls die .tex-Datei \titel etc. nicht gesetzt hat
\providecommand{\titel}{}
\providecommand{\editorInnen}{}
\providecommand{\dateiname}{\jobname}

\vspace{3cm}

\vfill

\footnotesize
\textsc{Quelle}: \titel. Herausgegeben von {\editorInnen}. In: \emph{Arthur Schnitzler: Briefwechsel mit Autorinnen und Autoren}.
 Digitale Edition, https://schnitzler-briefe.acdh.oeaw.ac.at/{\dateiname}.html (Stand \today)
\fi

\end{document}


      