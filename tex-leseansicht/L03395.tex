%% latex-korrekturansicht-vorspann.tex
%% Vorspann für die Korrekturansicht.
%% Lädt die gemeinsame Datei latex-vorspann.tex mit gesetztem Schalter.

\newif\ifkorrekturansicht
\korrekturansichttrue

\input{../tex-inputs/latex-vorspann}


\section[ Felix Salten an Arthur Schnitzler, {[}14. 4. 1904{]}]{L03395 Felix Salten an Arthur Schnitzler, {[}14. 4. 1904{]}}
\nopagebreak\mylabel{L03395v}
\rehead{ }\normalsize\beginnumbering\briefempfaengerindex{Schnitzler, Arthur@\textsc{Schnitzler, Arthur}!zzzSalten, Felix@\emph{von Felix Salten}!1904-04-141@{{[}14. 4. 1904{]}}|(be}
\toendnotes[C]{\smallbreak\pagebreak[2]}\Standort{CUL, Schnitzler, B 89, B 1.}
\physDesc{Brief, 1 Blatt, 2 Seiten, 668 Zeichen
\newline{}Handschrift: Bleistift, lateinische Kurrent
\newline{}Schnitzler: mit Bleistift datiert: »1\substVorne{}\textsuperscript{5}\substDazwischen{}4\substHinten{}/4 904« 
\newline{}Ordnung: mit Bleistift von unbekannter Hand nummeriert: »187« }\toendnotes[C]{\smallbreak}
\pstart
           \raggedleft{}{\pb}Donnerstag\pend
           \vspace{0.5em}
\pstart
           Lieber Arthur,{ }gestern hörte ich durch einen Zufall, dass Ihr Bub\pwindex{Schnitzler, Heinrich 09.08.1902 – 12.07.1982@\textsc{Schnitzler, Heinrich} (09.08.1902 – 12.07.1982), \emph{Regisseur/Regisseurin, Schauspieler/Schauspielerin}|pwv} Masern hat. Ihr \label{K_L03395-1v}\edtext{Brief heute}{\lemma{\textnormal{\emph{Brief heute}}}\Cendnote{\textnormal{Arthur Schnitzler an Felix Salten, 13. 4. 1904.
               }}}\label{K_L03395-1} läßt erfreulicherweise die Vermuthung zu, dass die Sache garnicht arg ist.
               Wollen es hoffen und herzlichst wünschen. Wird Ihre \label{K_L03395-2v}\edtext{Reise}{\lemma{\textnormal{\emph{Reise}}}\Cendnote{\textnormal{Zwischen
                     1. 5. 1904 und
                     29. 5. 1904
                  reisten Arthur und Olga Schnitzler\pwindex{Schnitzler, Olga 17.01.1882 – 13.01.1970@\textsc{Schnitzler, Olga} (17.01.1882 – 13.01.1970), \emph{Schauspieler/Schauspielerin, Sänger/Sängerin}|pwk} nach Italien\oindex{Italien@\textbf{Italien}, \emph{A.PCLI}|pwk}. Die Hauptstationen bildeten Rom\oindex{Rom@\textbf{Rom}, \emph{P.PPLC}|pwk}, Neapel\oindex{Neapel@\textbf{Neapel}, \emph{P.PPLA}|pwk}, Pompeji\oindex{Pompeji@\textbf{Pompeji}, \emph{S.ANS}|pwk}, Palermo\oindex{Palermo@\textbf{Palermo}, \emph{P.PPLA}|pwk} und Taormina\oindex{Taormina@\textbf{Taormina}, \emph{P.PPLA3}|pwk}.}}}\label{K_L03395-2} dadurch wesentlich verschoben?
               Wenn es mit Heini\pwindex{Schnitzler, Heinrich 09.08.1902 – 12.07.1982@\textsc{Schnitzler, Heinrich} (09.08.1902 – 12.07.1982), \emph{Regisseur/Regisseurin, Schauspieler/Schauspielerin}|pw} soweit besser geworden,
               möchten wir Sie gerne noch \label{K_L03395-3v}\edtext{einen Abend
               bei uns sehen}{\lemma{\textnormal{\emph{einen … sehen}}}\Cendnote{\textnormal{Vor der Abreise sahen sich
                     Schnitzler und Salten\pwindex{Salten, Felix 06.09.1869 – 08.10.1945@\textsc{Salten, Felix} (06.09.1869 – 08.10.1945), \emph{Schriftsteller/Schriftstellerin, Journalist/Journalistin, Chefredakteur/Chefredakteurin}|pwk} nachweislich am 27. 4. 1904 im Kaffeehaus.}}}\label{K_L03395-3}, ehe Sie
               abreisen.\pend
           
\pstart
           Über Klein\pwindex{Klein, Richard *~07.08.1873@\textsc{Klein, Richard} (*~07.08.1873), \emph{Maler/Malerin}|pw} würde ich gerne schreiben. Leider
               gehts nicht. Und ich steh’ mit D\textsuperscript{r}{ }H.\pwindex{Haberfeld, Hugo 1875-11-24 – 1946@\textsc{Haberfeld, Hugo} (1875-11-24 – 1946), \emph{Galerist/Galeristin, Kunstkritiker/Kunstkritikerin}|pw} nicht so, dass ich ihm was sagen \substVorne{}\textsuperscript{\textcolor{gray}{un}}\substDazwischen{}kö\substHinten{}nnte.
               Deshalb werde ich also versuchen, Ihre Bitte dem Professor \label{K_L03395-4v}\edtext{Singer\pwindex{Singer, Isidor 16.01.1857 – 08.12.1927@\textsc{Singer, Isidor} (16.01.1857 – 08.12.1927), \emph{Journalist/Journalistin, Herausgeber/Herausgeberin, Soziologe/Soziologin}|pw}}{\lemma{\textnormal{\emph{Singer}}}\Cendnote{\textnormal{Eine Ausstellungsbesprechung konnte nicht nachgewiesen werden.}}}\label{K_L03395-4} zu
               comuniziren.\pend
           
\pstart
           Bitte geben Sie bald Nachricht, {\pb}wie es bei Ihnen geht.\pend
           
\pstart
           Herzl. Grüße von Otti\pwindex{Salten, Ottilie 07.03.1868 – 22.06.1942@\textsc{Salten, Ottilie} (07.03.1868 – 22.06.1942), \emph{Schauspieler/Schauspielerin}|pw} und mir an Sie Beide\pwindex{Schnitzler, Olga 17.01.1882 – 13.01.1970@\textsc{Schnitzler, Olga} (17.01.1882 – 13.01.1970), \emph{Schauspieler/Schauspielerin, Sänger/Sängerin}|pwv}.\pend
           \pstart Ihr 
               \spacefill\mbox{S.}\pend{}\selectlanguage{ngerman}\endnumbering\briefempfaengerindex{Schnitzler, Arthur@\textsc{Schnitzler, Arthur}!zzzSalten, Felix@\emph{von Felix Salten}!1904-04-141@{{[}14. 4. 1904{]}}|)be}\mylabel{L03395h}  \normalsize

\doendnotes{C}
\bigskip
\vfill

\clearpage

\footnotesize

\lohead{\textsc{register}}

% Definiere theindex-Environment komplett neu ohne reledmac
\makeatletter
\renewenvironment{theindex}{%
  \section*{\indexname}%
  \setlength{\parindent}{0pt}%
  \setlength{\parskip}{0pt plus 0.3pt}%
  \let\item\@idxitem
}{%
  \clearpage
}
\makeatother

\IfFileExists{\jobname-pw.ind}{\input{\jobname-pw.ind}}{}

\end{document}

      