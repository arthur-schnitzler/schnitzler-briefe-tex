%% latex-leseansicht-vorspann.tex
%% Vorspann für die Leseansicht.
%% Lädt die gemeinsame Datei latex-vorspann.tex mit nicht gesetztem Schalter.

\newif\ifkorrekturansicht
\korrekturansichtfalse

\input{../tex-inputs/latex-vorspann}


         
         \renewcommand{\erwaehntePersonen}{Personen: Hugo Haberfeld, Richard Klein, Felix Salten, Ottilie Salten, Heinrich Schnitzler, Olga Schnitzler, Isidor Singer}
         \renewcommand{\erwaehnteOrte}{Orte: Italien, Neapel, Palermo, Pompei, Rom, Taormina, Wien}
         \renewcommand{\erwaehnteWerke}{}
               \section[ Felix Salten an Arthur Schnitzler, {[}14. 4. 1904{]}]{ Felix Salten an Arthur Schnitzler, {[}14. 4. 1904{]}}\nopagebreak\mylabel{v}\rehead{ }\begin{ledgroupsized}[t]{13cm}\normalsize\beginnumbering \toendnotes[C]{\smallbreak\pagebreak[2]} \Standort{CUL, Schnitzler, B 89, B 1.}
\physDesc{Brief, 1 Blatt, 2 Seiten, 668 Zeichen
\newline{}Handschrift: Bleistift, lateinische Kurrent
\newline{}Schnitzler: mit Bleistift datiert: »1\substVorne{}\textsuperscript{5}\substDazwischen{}4\substHinten{}/4 904« 
\newline{}Ordnung: mit Bleistift von unbekannter Hand nummeriert: »187« }\toendnotes[C]{\smallbreak}\pstart
           \raggedleft{}{\pb}Donnerstag\pend
           \pstart
           Lieber Arthur,{ }gestern hörte ich durch einen Zufall, dass Ihr Bub\pwindex{Schnitzler, Heinrich 09.08.1902 – 12.07.1982@\textsc{Schnitzler, Heinrich} (09.08.1902 – 12.07.1982), \emph{Regisseur, Schauspieler}|pwv} Masern hat. Ihr \label{K_L03395-1v}\edtext{Brief heute}{\lemma{\textnormal{\emph{Brief heute}}}\Cendnote{\textnormal{Arthur Schnitzler an Felix Salten, 13. 4. 1904}}}\label{K_L03395-1h} läßt erfreulicherweise die Vermuthung zu, dass die Sache garnicht arg ist.
               Wollen es hoffen und herzlichst wünschen. Wird Ihre \label{K_L03395-2v}\edtext{Reise}{\lemma{\textnormal{\emph{Reise}}}\Cendnote{\textnormal{Zwischen
                     1. 5. 1904 und
                     29. 5. 1904
                  reisten Arthur\pwindex{Schnitzler, Arthur 15.05.1862 – 21.10.1931@\textsc{Schnitzler, Arthur} (15.05.1862 – 21.10.1931), \emph{Schriftsteller, Mediziner}|pwk} und Olga Schnitzler\pwindex{Schnitzler, Olga 17.01.1882 – 13.01.1970@\textsc{Schnitzler, Olga} (17.01.1882 – 13.01.1970), \emph{Schauspielerin, Sängerin}|pwk} nach Italien\oindex{Italien@\textbf{Italien}|pwk}. Die Hauptstationen bildeten Rom\oindex{Rom@\textbf{Rom}|pwk}, Neapel\oindex{Neapel@\textbf{Neapel}|pwk}, Pompei\oindex{Pompei@\textbf{Pompei}|pwk}, Palermo\oindex{Palermo@\textbf{Palermo}|pwk} und Taormina\oindex{Taormina@\textbf{Taormina}|pwk}.}}}\label{K_L03395-2h} dadurch wesentlich verschoben?
               Wenn es mit Heini\pwindex{Schnitzler, Heinrich 09.08.1902 – 12.07.1982@\textsc{Schnitzler, Heinrich} (09.08.1902 – 12.07.1982), \emph{Regisseur, Schauspieler}|pw} soweit besser geworden,
               möchten wir Sie gerne noch \label{K_L03395-3v}\edtext{einen Abend
               bei uns sehen}{\lemma{\textnormal{\emph{einen … sehen}}}\Cendnote{\textnormal{Vor der Abreise sahen sich
                     Schnitzler\pwindex{Schnitzler, Arthur 15.05.1862 – 21.10.1931@\textsc{Schnitzler, Arthur} (15.05.1862 – 21.10.1931), \emph{Schriftsteller, Mediziner}|pwk} und Salten\pwindex{Salten, Felix 06.09.1869 – 08.10.1945@\textsc{Salten, Felix} (06.09.1869 – 08.10.1945), \emph{Schriftsteller, Journalist}|pwk} nachweislich am 27. 4. 1904 im Kaffeehaus.}}}\label{K_L03395-3h}, ehe Sie
               abreisen.\pend
           \pstart
           Über Klein\pwindex{Klein, Richard *~07.08.1873@\textsc{Klein, Richard} (*~07.08.1873), \emph{Maler}|pw} würde ich gerne schreiben. Leider
               gehts nicht. Und ich steh’ mit D\textsuperscript{r}{ }H.\pwindex{Haberfeld, Hugo 1875-11-24 – 1946@\textsc{Haberfeld, Hugo} (1875-11-24 – 1946), \emph{Galerist, Kunstkritiker}|pw} nicht so, dass ich ihm was sagen \substVorne{}\textsuperscript{\textcolor{gray}{un}}\substDazwischen{}kö\substHinten{}nnte.
               Deshalb werde ich also versuchen, Ihre Bitte dem Professor \label{K_L03395-4v}\edtext{Singer\pwindex{Singer, Isidor 16.01.1857 – 08.12.1927@\textsc{Singer, Isidor} (16.01.1857 – 08.12.1927), \emph{Journalist, Herausgeber, Soziologe}|pw}}{\lemma{\textnormal{\emph{Singer}}}\Cendnote{\textnormal{Eine Ausstellungsbesprechung konnte nicht nachgewiesen werden.}}}\label{K_L03395-4h} zu
               comuniziren.\pend
           \pstart
           Bitte geben Sie bald Nachricht, {\pb}wie es bei Ihnen geht.\pend
           \pstart
           Herzl. Grüße von Otti\pwindex{Salten, Ottilie 07.03.1868 – 22.06.1942@\textsc{Salten, Ottilie} (07.03.1868 – 22.06.1942), \emph{Schauspielerin}|pw} und mir an Sie Beide\pwindex{Schnitzler, Olga 17.01.1882 – 13.01.1970@\textsc{Schnitzler, Olga} (17.01.1882 – 13.01.1970), \emph{Schauspielerin, Sängerin}|pwv}.\pend
           \pstart Ihr 
               \spacefill\mbox{S.}\pend{}
         
         \endnumbering\mylabel{h}\end{ledgroupsized}  \newcommand{\dateiname}{L03395}\newcommand{\titel}{Felix Salten an Arthur Schnitzler, [14. 4. 1904]}\newcommand{\editorInnen}{Martin Anton Müller und Laura Untner}%% latex-leseansicht-abspann.tex
%% Abspann für die Leseansicht.
%% Der Schalter \ifkorrekturansicht ist bereits durch den Vorspann gesetzt.

%% latex-abspann.tex
%% Gemeinsamer Abspann für Korrekturansicht und Leseansicht.
%% Setzt den Schalter \ifkorrekturansicht voraus (gesetzt in den
%% einbindenden Dateien latex-korrekturansicht-abspann.tex bzw.
%% latex-leseansicht-abspann.tex).
%% ---------------------------------------------------------------

\normalsize

% Das esempio-Environment wird nur in der Leseansicht benötigt
\ifkorrekturansicht\else
\newenvironment{esempio}[3]%
{
    \vspace{1.5ex}
    \rlap{\underline{#1}}
    \par
    \setlength{\parindent}{0cm}
    \nopagebreak
    \leftskip=#2cm
    \rightskip=#3cm
}
{
    \par
}
\fi

\doendnotes{C}
\bigskip
\vfill

\clearpage

\footnotesize

\ifkorrekturansicht
  \lohead{\textsc{register}}
\fi

% theindex-Environment neu definieren ohne reledmac
\makeatletter
\renewenvironment{theindex}{%
  \ifkorrekturansicht
    \section*{\indexname}%
  \else
    \subsubsection*{Index der erwähnten Entitäten}%
  \fi
  \setlength{\parindent}{0pt}%
  \setlength{\parskip}{0pt plus 0.3pt}%
  \let\item\@idxitem
}{%
  \ifkorrekturansicht\clearpage\fi
}
\makeatother

\IfFileExists{\jobname-pw.ind}{\input{\jobname-pw.ind}}{}

% Quellenangabe nur in der Leseansicht
\ifkorrekturansicht\else
% Fallback-Definitionen, falls die .tex-Datei \titel etc. nicht gesetzt hat
\providecommand{\titel}{}
\providecommand{\editorInnen}{}
\providecommand{\dateiname}{\jobname}

\vspace{3cm}

\vfill

\footnotesize
\textsc{Quelle}: \titel. Herausgegeben von {\editorInnen}. In: \emph{Arthur Schnitzler: Briefwechsel mit Autorinnen und Autoren}.
 Digitale Edition, https://schnitzler-briefe.acdh.oeaw.ac.at/{\dateiname}.html (Stand \today)
\fi

\end{document}


      