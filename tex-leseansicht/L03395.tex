%% latex-leseansicht-vorspann.tex
%% Vorspann für die Leseansicht.
%% Lädt die gemeinsame Datei latex-vorspann.tex mit nicht gesetztem Schalter.

\newif\ifkorrekturansicht
\korrekturansichtfalse

\input{../tex-inputs/latex-vorspann}

\begin{center}
            \textcolor{red}{ENTWURF, NICHT FERTIG KORRIGIERT}
                      \end{center}
            
         
         \renewcommand{\erwaehntePersonen}{Personen: Ottilie Salten, Heinrich Schnitzler, Olga Schnitzler}
         \renewcommand{\erwaehnteOrte}{Orte: Wien}
         \renewcommand{\erwaehnteWerke}{}
               \section[Felix Salten an Arthur Schnitzler, {[}14. 4. 1904{]}]{ Felix Salten an Arthur Schnitzler, {[}14. 4. 1904{]}}\nopagebreak\mylabel{v}\rehead{ }\begin{ledgroupsized}[t]{13cm}\normalsize\beginnumbering \toendnotes[C]{\smallbreak\pagebreak[2]} \Standort{CUL, Schnitzler, B 89, B 1.}
\physDesc{Brief, 2 Blätter, 1 Seite
\newline{}Handschrift: Bleistift, lateinische Kurrent
\newline{}Schnitzler: mit Bleistift datiert: »1\substVorne{}\textsuperscript{5}\substDazwischen{}4\substHinten{}/4 {[}1{]}904« \newline{}Ordnung: mit Bleistift von unbekannter Hand nummeriert:
                                    »187« }\toendnotes[C]{\smallbreak}\pstart
           \raggedleft{}{\pb}Donnerstag\pend
           \pstart
           Lieber Arthur, gestern hörte ich durch einen Zufall, dass Ihr Bub\textcolor{red}{\textsuperscript{\textbf{KEY}}} Masern hat. Ihr Brief heute läßt
               erfreulicherweise die Vermuthung zu, dass die Sache garnicht arg ist. Wollen es
               hoffen und herzlichst wünschen. Wird Ihre Reise dadurch wesentlich verschoben? Wenn
               es mit Heini\pwindex{Schnitzler, Heinrich 09.08.1902 – 12.07.1982@\textsc{Schnitzler, Heinrich} (09.08.1902 – 12.07.1982), \emph{Regisseur, Schauspieler}|pw} soweit besser geworden, möchten wir Sie
               gerne noch einen Abend bei uns sehen, ehe Sie abreisen. \pend
           \pstart
           Über Klein\textcolor{red}{\textsuperscript{\textbf{KEY}}} würde ich gerne schreiben. Leider gehts
               nicht. Und ich steh’ mit D\textsuperscript{r} H.\textcolor{red}{\textsuperscript{\textbf{KEY}}} nicht so, dass ich ihm was sagen könnte. Deshalb werde ich also
               versuchen, Ihre Bitte dem Professor Singer\textcolor{red}{\textsuperscript{\textbf{KEY}}} zu
               comuniziren. \pend
           \pstart
           Bitte geben Sie bald Nachricht, {\pb}wie es bei Ihnen geht. \pend
           \pstart
           Herzl. Grüße von Otti\pwindex{Salten, Ottilie 07.03.1868 – 22.06.1942@\textsc{Salten, Ottilie} (07.03.1868 – 22.06.1942), \emph{Schauspielerin}|pw} und mir an Sie Beide\pwindex{Schnitzler, Olga 17.01.1882 – 13.01.1970@\textsc{Schnitzler, Olga} (17.01.1882 – 13.01.1970), \emph{Schauspielerin, Sängerin}|pwv}.\pend
           \pstart
           Ihr {\\[\baselineskip]}\spacefill\mbox{S. }\pend
           \leftskip=0em{}
         
         \endnumbering\mylabel{h}\end{ledgroupsized}\begin{anhang}\end{anhang}\newcommand{\dateiname}{L03395}\newcommand{\titel}{Felix Salten an Arthur Schnitzler, [14. 4. 1904]}\newcommand{\editorInnen}{Martin Anton Müller und Laura Untner}%% latex-leseansicht-abspann.tex
%% Abspann für die Leseansicht.
%% Der Schalter \ifkorrekturansicht ist bereits durch den Vorspann gesetzt.

%% latex-abspann.tex
%% Gemeinsamer Abspann für Korrekturansicht und Leseansicht.
%% Setzt den Schalter \ifkorrekturansicht voraus (gesetzt in den
%% einbindenden Dateien latex-korrekturansicht-abspann.tex bzw.
%% latex-leseansicht-abspann.tex).
%% ---------------------------------------------------------------

\normalsize

% Das esempio-Environment wird nur in der Leseansicht benötigt
\ifkorrekturansicht\else
\newenvironment{esempio}[3]%
{
    \vspace{1.5ex}
    \rlap{\underline{#1}}
    \par
    \setlength{\parindent}{0cm}
    \nopagebreak
    \leftskip=#2cm
    \rightskip=#3cm
}
{
    \par
}
\fi

\doendnotes{C}
\bigskip
\vfill

\clearpage

\footnotesize

\ifkorrekturansicht
  \lohead{\textsc{register}}
\fi

% theindex-Environment neu definieren ohne reledmac
\makeatletter
\renewenvironment{theindex}{%
  \ifkorrekturansicht
    \section*{\indexname}%
  \else
    \subsubsection*{Index der erwähnten Entitäten}%
  \fi
  \setlength{\parindent}{0pt}%
  \setlength{\parskip}{0pt plus 0.3pt}%
  \let\item\@idxitem
}{%
  \ifkorrekturansicht\clearpage\fi
}
\makeatother

\IfFileExists{\jobname-pw.ind}{\input{\jobname-pw.ind}}{}

% Quellenangabe nur in der Leseansicht
\ifkorrekturansicht\else
% Fallback-Definitionen, falls die .tex-Datei \titel etc. nicht gesetzt hat
\providecommand{\titel}{}
\providecommand{\editorInnen}{}
\providecommand{\dateiname}{\jobname}

\vspace{3cm}

\vfill

\footnotesize
\textsc{Quelle}: \titel. Herausgegeben von {\editorInnen}. In: \emph{Arthur Schnitzler: Briefwechsel mit Autorinnen und Autoren}.
 Digitale Edition, https://schnitzler-briefe.acdh.oeaw.ac.at/{\dateiname}.html (Stand \today)
\fi

\end{document}


      