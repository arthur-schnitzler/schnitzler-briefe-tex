%% latex-korrekturansicht-vorspann.tex
%% Vorspann für die Korrekturansicht.
%% Lädt die gemeinsame Datei latex-vorspann.tex mit gesetztem Schalter.

\newif\ifkorrekturansicht
\korrekturansichttrue

\input{../tex-inputs/latex-vorspann}


\section[Hugo von Hofmannsthal an Arthur Schnitzler, 14. 9. {[}1908{]}]{L01789 Hugo von Hofmannsthal an Arthur Schnitzler, 14. 9. {[}1908{]}}
\nopagebreak\mylabel{L01789v}
\rehead{ }\normalsize\beginnumbering\briefempfaengerindex{Schnitzler, Arthur@\textsc{Schnitzler, Arthur}!zzzHofmannsthal, Hugo von@\emph{von Hugo von Hofmannsthal}!1908-09-141@{14. 9. 1908}|(be}
\toendnotes[C]{\smallbreak\pagebreak[2]}\Standort{CUL, Schnitzler, B 43.}
\physDesc{Brief, 1 Blatt, 4 Seiten, 1588 Zeichen
\newline{}Handschrift: schwarze Tinte, deutsche Kurrent
\newline{}Schnitzler: mit Bleistift die Jahreszahl ergänzt: »08« 
\newline{}Ordnung: 1) mit Bleistift von unbekannter Hand nummeriert: »\strikeout{293}«  2) mit Bleistift von unbekannter Hand nummeriert:
                                    »300«}
\buchAbdrucke{\weitereDrucke{Hugo von Hofmannsthal, Arthur Schnitzler: \emph{Briefwechsel}. Frankfurt am Main: \emph{S. Fischer} 1964, S. 240.} }\toendnotes[C]{\smallbreak}
\pstart
           
\pstart
           {\pb}\textcolor{gray}{\textbf{Ramgut\oindex{Ramgut@\textbf{Ramgut}, \emph{Schloss (K.SLS)}|pw}}}\pend
           
\pstart
           \raggedleft{}14 IX.\pend
           \pend
           
\pstart
           \textcolor{gray}{\textbf{Aussee Steyermark}}\pend
           
\pstart{}mein lieber Arthur\pend\vspace{0.5em}
\pstart
           ich war ſehr froh, aus Ihrem Brief und noch ausführlicher durch Waſſermanns\pwindex{Wassermann, Jakob 10.03.1873 – 01.01.1934@\textsc{Wassermann, Jakob} (10.03.1873 – 01.01.1934), \emph{Schriftsteller/Schriftstellerin}|pw} Berichte zu erfahren, einen wie guten friedlichen
               und erfüllten Sommer Sie gehabt haben. Der meinige war vom Auguſt ab nicht ganz ſo
               gut. Ich habe von der Luft im Engadin\oindex{Engadin@\textbf{Engadin}, \emph{T.VAL}|pw} die mir
               nicht zuträglich war, eine Nervendepreſſion {\pb}mitgetragen, oder Nervenirritation
               die beſonders peinlich war, ſolange ſie ſich ſozuſagen latent mit dem Normalen der
               Exiſtenz mitſchleppte – und die ſchließlich zu einer ziemlich peinlichen Art von
               Kriſe führte, damit aber auch abzuklingen anfing, so daſs ich nun hoffen kann den
               letzten Act der Comödie\pwindex{Rosenkavalier@\emph{Der Rosenkavalier}|pwv}
               entweder hier oder auf dem Se{\geminationm}ering\oindex{Semmering@\textbf{Semmering}, \emph{A.ADM3}|pw} oder in \textsc{Rodaun}\oindex{Rodaun@\textbf{Rodaun}, \emph{A.ADM4}|pw} mit ſo viel Freiheit und Munterkeit zu Ende zu {\pb}bringen, als er ſeiner Natur nach
               braucht.\pend
           
\pstart
           \numberlinefalse{}–\numberlinetrue{}\pend
           
\pstart
           Ich habe \label{K_L01789-1v}\edtext{damals}{\lemma{\textnormal{\emph{damals}}}\Cendnote{\textnormal{Siehe Hugo von Hofmannsthal an Arthur Schnitzler, 24. 7. [1908].
               }}}\label{K_L01789-1}, als es mir unanſtändig erſchien, ein negatives Verhältnis zu einer Ihrer Arbeiten\pwindex{Weg ins Freie. Roman@\emph{Der Weg ins Freie. Roman}|pwv} zu verſchleiern, den
               Ausdruck »verſtören« gewählt, weil er mir keine Kritik zu enthalten, ſondern nur eine
               ſubiective Verfaſſung des Leſers auszumalen ſchien. Aus Ihrem Brief ſah ich dann,
               daſs das Wort leider Gottes für Sie doch einen offenſiven {\pb}Beiklang gehabt hatte.\pend
           
\pstart
           Wenn je ein Menſch in den andern hineinſchauen könnte, wenn Sie in mich hineinſchauen
               könnten im Augenblick wo ich etwa allein auf einem Spaziergang oder in meinem Zi{\geminationm}er an Sie denke, an Sie, worunter ich hier ein
               Geſamtweſen aus dem lieben guten Menſchen und dem geiſtigen Phantom, das hinter den
               Arbeiten ſteht, begreife – ſo wäre die Möglichkeit daſs ein Wort von mir Ihnen auch
               nur ein bischen wehthut, überhaupt ausgeſchloſſen.\pend
           
\pstart
           Ich freue mich \uline{sehr} auf Sie.\pend
           
\pstart
           Ihr{\\[\baselineskip]}\spacefill\mbox{Hugo.}\pend
           \leftskip=0em{}\selectlanguage{ngerman}\endnumbering\briefempfaengerindex{Schnitzler, Arthur@\textsc{Schnitzler, Arthur}!zzzHofmannsthal, Hugo von@\emph{von Hugo von Hofmannsthal}!1908-09-141@{14. 9. 1908}|)be}\mylabel{L01789h}  \normalsize

\doendnotes{C}
\bigskip
\vfill

\clearpage

\footnotesize

\lohead{\textsc{register}}

% Definiere theindex-Environment komplett neu ohne reledmac
\makeatletter
\renewenvironment{theindex}{%
  \section*{\indexname}%
  \setlength{\parindent}{0pt}%
  \setlength{\parskip}{0pt plus 0.3pt}%
  \let\item\@idxitem
}{%
  \clearpage
}
\makeatother

\IfFileExists{\jobname-pw.ind}{\input{\jobname-pw.ind}}{}

\end{document}

      