%% latex-leseansicht-vorspann.tex
%% Vorspann für die Leseansicht.
%% Lädt die gemeinsame Datei latex-vorspann.tex mit nicht gesetztem Schalter.

\newif\ifkorrekturansicht
\korrekturansichtfalse

\input{../tex-inputs/latex-vorspann}


         
         \newcommand{\erwaehntePersonen}{Personen: Jakob Wassermann}
         \newcommand{\erwaehnteOrte}{Orte: Bad Aussee, Engadin, Ramgut, Rodaun, Semmering, Wien}
         \newcommand{\erwaehnteWerke}{Werke: Der Rosenkavalier, Der Weg ins Freie. Roman}
               \section[Hugo von Hofmannsthal an Arthur Schnitzler, 14. 9. {[}1908{]}]{ Hugo von Hofmannsthal an Arthur Schnitzler, 14. 9. {[}1908{]}}\nopagebreak\mylabel{v}\rehead{ }\begin{ledgroupsized}[t]{13cm}\normalsize\beginnumbering \toendnotes[C]{\smallbreak\pagebreak[2]} \Standort{CUL, Schnitzler, B 43.}
\physDesc{Brief, 1 Blatt, 4 Seiten
\newline{}Handschrift: schwarze Tinte, deutsche Kurrent
\newline{}Schnitzler: mit Bleistift die Jahreszahl ergänzt: »08« \newline{}Ordnung: 1) mit Bleistift von unbekannter Hand nummeriert: »\strikeout{293}«  2) mit Bleistift von unbekannter Hand nummeriert: »300«}\buchAbdrucke{\weitereDrucke{Hugo von Hofmannsthal, Arthur Schnitzler: \emph{Briefwechsel}. Hg. Therese Nickl und Heinrich Schnitzler. Frankfurt am Main: \emph{S. Fischer} 1964, S. 240.} }\toendnotes[C]{\smallbreak}\pstart
           {\pb}\textcolor{gray}{\textbf{Ramgut\oindex{Ramgut@\textbf{Ramgut}|pw}}}\hfill 14 IX.\pend
           \pstart
           \textcolor{gray}{\textbf{Aussee Steyermark}}\pend
           \pstart{}mein lieber Arthur\pend\pstart
           ich war ſehr froh, aus Ihrem Brief und noch ausführlicher durch Waſſermann\pwindex{Wassermann, Jakob 10.03.1873 – 01.01.1934@\textsc{Wassermann, Jakob} (10.03.1873 – 01.01.1934), \emph{Schriftsteller}|pw}s Berichte zu erfahren, einen wie guten friedlichen
               und erfüllten Sommer Sie gehabt haben. Der meinige war vom Auguſt ab nicht ganz ſo
               gut. Ich habe von der Luft im Engadin\oindex{Engadin@\textbf{Engadin}|pw} die mir nicht
               zuträglich war, eine Nervendepreſſion {\pb}mitgetragen, oder Nervenirritation
               die beſonders peinlich war, ſolange ſie ſich ſozuſagen latent mit dem Normalen der
               Exiſtenz mitſchleppte – und die ſchließlich zu einer ziemlich peinlichen Art von
               Kriſe führte, damit aber auch abzuklingen anfing, so daſs ich nun hoffen kann den
               letzten Act der Comödie\pwindex{Hofmannsthal, Hugo von 1874-02-01 – 1929-07-15@\textsc{Hofmannsthal, Hugo von} (1874-02-01 – 1929-07-15), \emph{Schriftsteller}!Rosenkavalier1911@\strich\emph{Der Rosenkavalier} {[}1911{]}|pwv} entweder
               hier oder auf dem Se{\geminationm}ering\oindex{Semmering@\textbf{Semmering}|pw} oder in \textsc{Rodaun}\oindex{Rodaun@\textbf{Rodaun}|pw} mit ſo viel Freiheit und Munterkeit zu Ende zu {\pb}bringen, als er ſeiner Natur nach
               braucht.\pend
           \pstart
           \numberlinefalse{}–\numberlinetrue{}\pend
           \pstart
           Ich habe \label{K_L01789_1v}\edtext{damals}{\lemma{\textnormal{\emph{damals}}}\Cendnote{\textnormal{siehe Hugo von Hofmannsthal an Arthur Schnitzler, 24. 7. [1908]}}}\label{K_L01789_1h}, als es mir unanſtändig erſchien, ein negatives Verhältnis zu einer
               Ihrer Arbeiten\pwindex{Schnitzler, Arthur 15.05.1862 – 21.10.1931@\textsc{Schnitzler, Arthur} (15.05.1862 – 21.10.1931), \emph{Schriftsteller, Mediziner}!Weg ins Freie. Roman1.1.1908 – 1.6.1908@\strich\emph{Der Weg ins Freie. Roman} {[}1.1.1908 – 1.6.1908{]}|pwv} zu verſchleiern,
               den Ausdruck »verſtören« gewählt, weil er mir keine Kritik zu enthalten, ſondern nur
               eine ſubiective Verfaſſung des Leſers auszumalen ſchien. Aus Ihrem Brief ſah ich
               dann, daſs das Wort leider Gottes für Sie doch einen offenſiven {\pb}Beiklang gehabt hatte.\pend
           \pstart
           Wenn je ein Menſch in den andern hineinſchauen könnte, wenn Sie in mich hineinſchauen
               könnten im Augenblick wo ich etwa allein auf einem Spaziergang oder in meinem Zi{\geminationm}er an Sie denke, an Sie, worunter ich hier ein
               Geſamtweſen aus dem lieben guten Menſchen und dem geiſtigen Phantom, das hinter den
               Arbeiten ſteht, begreife – ſo wäre die Möglichkeit daſs ein Wort von mir Ihnen auch
               nur ein bischen wehthut, überhaupt ausgeſchloſſen.\pend
           \pstart
           Ich freue mich \uline{sehr} auf Sie.\pend
           \pstart
           Ihr{\\[\baselineskip]}\spacefill\mbox{Hugo.}\pend
           \leftskip=0em{}
         
         \endnumbering\mylabel{h}\end{ledgroupsized}  \newcommand{\dateiname}{L01789}\newcommand{\titel}{Hugo von Hofmannsthal an Arthur Schnitzler, 14. 9. [1908]}\newcommand{\editorInnen}{Martin Anton Müller und Gerd-Hermann Susen}%% latex-leseansicht-abspann.tex
%% Abspann für die Leseansicht.
%% Der Schalter \ifkorrekturansicht ist bereits durch den Vorspann gesetzt.

%% latex-abspann.tex
%% Gemeinsamer Abspann für Korrekturansicht und Leseansicht.
%% Setzt den Schalter \ifkorrekturansicht voraus (gesetzt in den
%% einbindenden Dateien latex-korrekturansicht-abspann.tex bzw.
%% latex-leseansicht-abspann.tex).
%% ---------------------------------------------------------------

\normalsize

% Das esempio-Environment wird nur in der Leseansicht benötigt
\ifkorrekturansicht\else
\newenvironment{esempio}[3]%
{
    \vspace{1.5ex}
    \rlap{\underline{#1}}
    \par
    \setlength{\parindent}{0cm}
    \nopagebreak
    \leftskip=#2cm
    \rightskip=#3cm
}
{
    \par
}
\fi

\doendnotes{C}
\bigskip
\vfill

\clearpage

\footnotesize

\ifkorrekturansicht
  \lohead{\textsc{register}}
\fi

% theindex-Environment neu definieren ohne reledmac
\makeatletter
\renewenvironment{theindex}{%
  \ifkorrekturansicht
    \section*{\indexname}%
  \else
    \subsubsection*{Index der erwähnten Entitäten}%
  \fi
  \setlength{\parindent}{0pt}%
  \setlength{\parskip}{0pt plus 0.3pt}%
  \let\item\@idxitem
}{%
  \ifkorrekturansicht\clearpage\fi
}
\makeatother

\IfFileExists{\jobname-pw.ind}{\input{\jobname-pw.ind}}{}

% Quellenangabe nur in der Leseansicht
\ifkorrekturansicht\else
% Fallback-Definitionen, falls die .tex-Datei \titel etc. nicht gesetzt hat
\providecommand{\titel}{}
\providecommand{\editorInnen}{}
\providecommand{\dateiname}{\jobname}

\vspace{3cm}

\vfill

\footnotesize
\textsc{Quelle}: \titel. Herausgegeben von {\editorInnen}. In: \emph{Arthur Schnitzler: Briefwechsel mit Autorinnen und Autoren}.
 Digitale Edition, https://schnitzler-briefe.acdh.oeaw.ac.at/{\dateiname}.html (Stand \today)
\fi

\end{document}


      