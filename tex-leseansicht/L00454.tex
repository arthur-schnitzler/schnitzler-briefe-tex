%% latex-leseansicht-vorspann.tex
%% Vorspann für die Leseansicht.
%% Lädt die gemeinsame Datei latex-vorspann.tex mit nicht gesetztem Schalter.

\newif\ifkorrekturansicht
\korrekturansichtfalse

\input{../tex-inputs/latex-vorspann}


         
         \newcommand{\erwaehntePersonen}{Personen: Richard Beer-Hofmann, Max Eugen Burckhard, Felix Dörmann, Samuel Fischer, Paul Goldmann, Fanny Gröger, Hugo von Hofmannsthal, Karl Kraus, Rudolf Lothar, Felix Salten, Theodor Zasche}
         \newcommand{\erwaehnteInstitutionen}{Institutionen: Neue Freie Presse}
         \newcommand{\erwaehnteOrte}{Orte: Bad Ischl, Berlin, Burgtheater, Böhmen, Caslau, I., Innere Stadt, Kopenhagen, München, Wien}
         \newcommand{\erwaehnteWerke}{Werke: Die kleine Komödie, Fanny Gröger, »Adhimukti«, Frauenlob. Ein Lustspiel in drei Aufzügen, Freiwild. Schauspiel in 3 Akten, Liebelei. Schauspiel in drei Akten, Neue Deutsche Rundschau}
               \section[Arthur Schnitzler an Richard Beer-Hofmann, 15. 6. 1895]{ Arthur Schnitzler an Richard Beer-Hofmann, 15. 6. 1895}\nopagebreak\mylabel{v}\rehead{ }\begin{ledgroupsized}[t]{13cm}\normalsize\beginnumbering \toendnotes[C]{\smallbreak\pagebreak[2]} \Standort{YCGL, MSS 31.}
\physDesc{Brief, 1 Blatt, 4 Seiten, Umschlag
\newline{}Handschrift: 1) Bleistift, deutsche Kurrent\hspace{1em}2) schwarze Tinte, deutsche Kurrent (\noindent{}Umschlag)\hspace{1em}\newline{}Versand: 1) Stempel: »\nobreak{}\oindex{I., Innere Stadt@\textbf{I., Innere Stadt}|pwk}Wien 1/1, 15. 6. 95, 7–8 N\nobreak{}«.   2) Stempel: »\nobreak{}\oindex{Caslau@\textbf{Caslau}|pwk}Časlau, 16 6 95\nobreak{}«. }\buchAbdrucke{\weitereDrucke{1) Arthur Schnitzler: \emph{Briefe 1875–1912}. Hg. Therese Nickl und Heinrich Schnitzler. Frankfurt am Main: \emph{S. Fischer} 1981, S. 260–261.} \weitereDrucke{2) Arthur Schnitzler, Richard Beer-Hofmann: \emph{Briefwechsel 1891–1931}. Hg. Konstanze Fliedl. Wien, Zürich: \emph{Europaverlag} 1992, S. 74–75.} }\toendnotes[C]{\smallbreak}\pstart{}{\pb}Herrn KuK u. u. \textsc{Lieutenant}\pend{}\pstart{}\textsc{Dr. Richard Beer-Hofmann}\pend{}\pstart{}im \textsc{Kh. Landw.-Inf}-Regmt\pend{}\pstart{}\textsc{»Caslau\oindex{Caslau@\textbf{Caslau}|pw}«
                  Nr 12}.\pend{}{\bigskip}\pstart
           \raggedleft{}{\pb}15. Juni 95\pend
           \pstart
           Lieber Richard, heut bin ich ſo ſchlecht aufgelegt, als wär ich in
                  \textsc{Caslau}\oindex{Caslau@\textbf{Caslau}|pw}. – Einer der Gründe: ſchiefe Stellung in der Familie; Bemerkungen, daſs
               ich »ohne einen Kreuzer Geld zu haben« im So{\geminationm}er nach
                  \textsc{Kopenhagen}\oindex{Kopenhagen@\textbf{Kopenhagen}|pw} fahren will – Bemerkungen, die mir von dritter, nein vierter Seite
               zurückkommen. –\pend
           \pstart
           \textsc{Dörmann\pwindex{Doermann, Felix 29.05.1870 – 26.10.1928@\textsc{Dörmann, Felix} (29.05.1870 – 26.10.1928), \emph{Schriftsteller}|pw}} iſt da und erzählt viele Dinge von ſich – er hat 3 Stücke geſchrieben und hat
                  \introOben{}in Berlin\oindex{Berlin@\textbf{Berlin}|pw}\introOben{} 65 Verhältniſſe gehabt. Ich übertreibe nicht. Er aber ja {\dots} a {\dots} a –\pend
           \pstart
           – Die Kritik\pwindex{Fanny Groeger, »Adhimukti«12. 6. 1895@\emph{Fanny Gröger, »Adhimukti«} {[}12. 6. 1895{]}|pwv} vom kleinen Kraus\pwindex{Kraus, Karl 28.04.1874 – 12.06.1936@\textsc{Kraus, Karl} (28.04.1874 – 12.06.1936), \emph{Schriftsteller, Publizist}|pw} in dem {\pb}Abendblatt der
                  N. Fr. Pr.\orgindex{Neue Freie Presse@Neue Freie Presse|pw} über die Gröger\pwindex{Groeger, Fanny 12.01.1869 – 07.04.1936@\textsc{Gröger, Fanny} (12.01.1869 – 07.04.1936), \emph{Schriftstellerin}|pw} haben Sie geleſen? Er benützt die Gelegenheit,
               uns (Sie, \textsc{Loris}\pwindex{Hofmannsthal, Hugo von 1874-02-01 – 1929-07-15@\textsc{Hofmannsthal, Hugo von} (1874-02-01 – 1929-07-15), \emph{Schriftsteller}|pw}{ }\introOben{}\textsc{Salten}\pwindex{Salten, Felix 06.09.1869 – 08.10.1945@\textsc{Salten, Felix} (06.09.1869 – 08.10.1945), \emph{Schriftsteller, Journalist}|pw}\introOben{} mich) in die Waden zu beißen.\strikeout{)} Wir werden
               noch ſchmerzlicheres zu überleben haben. –\pend
           \pstart
           \textsc{Frauenlob}\pwindex{Lothar, Rudolf 23.2.1865 – 2.10.1943@\textsc{Lothar, Rudolf} (23.2.1865 – 2.10.1943), \emph{Schriftsteller, Journalist, Theaterdirektor}!Frauenlob. Ein Lustspiel in drei Aufzuegen1895@\strich\emph{Frauenlob. Ein Lustspiel in drei Aufzügen} {[}1895{]}|pw} von Hrn. \textsc{Lothar\pwindex{Lothar, Rudolf 23.2.1865 – 2.10.1943@\textsc{Lothar, Rudolf} (23.2.1865 – 2.10.1943), \emph{Schriftsteller, Journalist, Theaterdirektor}|pw}} an der Burg\oindex{Burgtheater@\textbf{Burgtheater}|pw}{ }\label{K_L00454_1v}\edtext{angenommen}{\lemma{\textnormal{\emph{angenommen}}}\Cendnote{\textnormal{Zu einer Aufführung kam es aber nicht.}}}\label{K_L00454_1h}. – Gerücht über
                  »Liebelei\pwindex{Schnitzler, Arthur 15.05.1862 – 21.10.1931@\textsc{Schnitzler, Arthur} (15.05.1862 – 21.10.1931), \emph{Schriftsteller, Mediziner}!Liebelei. Schauspiel in drei Akten1895-10-09@\strich\emph{Liebelei. Schauspiel in drei Akten} {[}1895-10-09{]}|pw}«: es werde überhaupt nicht an der
                  Burg\oindex{Burgtheater@\textbf{Burgtheater}|pw} zur Aufführung kommen. Entſtehung liegt
               nahe; werde Burckh.\pwindex{Burckhard, Max Eugen 14.07.1854 – 16.03.1912@\textsc{Burckhard, Max Eugen} (14.07.1854 – 16.03.1912), \emph{Schriftsteller, Rechtswissenschaftler, Theaterleiter}|pw} aufſuchen.\pend
           \pstart
           – Für den Abdruck der \textsc{Kl. Komödie}\pwindex{Schnitzler, Arthur 15.05.1862 – 21.10.1931@\textsc{Schnitzler, Arthur} (15.05.1862 – 21.10.1931), \emph{Schriftsteller, Mediziner}!kleine Komoedie1895-08-01@\strich\emph{Die kleine Komödie} {[}1895-08-01{]}|pw}{ }{\pb}in
               der \textsc{Freien Bühne}\pwindex{Neue Deutsche Rundschau1894-01-01 – 1903-12-31@\emph{Neue Deutsche Rundschau} {[}1894-01-01 – 1903-12-31{]}|pw} will \textsc{Fischer}\pwindex{Fischer, Samuel 24.12.1859 – 15.10.1934@\textsc{Fischer, Samuel} (24.12.1859 – 15.10.1934), \emph{Verleger}|pw} mir 25, bitte, 25 Mark bezahlen. Ich hab ihm einen groben Brief
               geſchrieben – da mir ja nichts dran liegt. Was haben Sie gegen \textsc{Zasche\pwindex{Zasche, Theodor 18.10.1862 – 15.11.1922@\textsc{Zasche, Theodor} (18.10.1862 – 15.11.1922), \emph{Zeichner, Karikaturist}|pw}}? Er wird das ganz hübſch machen. – Die Novelle\pwindex{Schnitzler, Arthur 15.05.1862 – 21.10.1931@\textsc{Schnitzler, Arthur} (15.05.1862 – 21.10.1931), \emph{Schriftsteller, Mediziner}!kleine Komoedie1895-08-01@\strich\emph{Die kleine Komödie} {[}1895-08-01{]}|pwv} zu datiren hat keinen Sinn; es kü{\geminationm}ert
               ſich doch keiner drum und ſieht aus wie eine Entſchuldigung. –\pend
           \pstart
           Ich ſchreibe an meinem Stück\pwindex{Schnitzler, Arthur 15.05.1862 – 21.10.1931@\textsc{Schnitzler, Arthur} (15.05.1862 – 21.10.1931), \emph{Schriftsteller, Mediziner}!Freiwild. Schauspiel in 3 Akten1896@\strich\emph{Freiwild. Schauspiel in 3 Akten} {[}1896{]}|pwv} – vorläufig ohne an eine
                  Aufführungs{\pb}möglichkeit zu denken. –\pend
           \pstart
           Meine Abſicht iſt, Anfang Juli in die böhm.\oindex{Boehmen@\textbf{Böhmen}|pw} Bäder zu reiſen und vor Mitte Juli in Iſchl\oindex{Bad Ischl@\textbf{Bad Ischl}|pw} zu ſein. – Wann wollen Sie nach München\oindex{Muenchen@\textbf{München}|pw} gehn? – Wie ſtehn Sie zu Kopenhagen\oindex{Kopenhagen@\textbf{Kopenhagen}|pw}? Beantworten Sie gütigſt. – Goldmann\pwindex{Goldmann, Paul 31.01.1865 – 25.09.1935@\textsc{Goldmann, Paul} (31.01.1865 – 25.09.1935), \emph{Schriftsteller, Journalist}|pw} wird im Auguſt Urlaub nehmen,
               genaueres unbekannt.\pend
           \pstart
           – Mein rechtes Ohr laß ich behandeln, das macht mich auch recht nervös. –\pend
           \pstart
           Leben Sie wohl, ſeien Sie herzlich gegrüßt.\pend
           \pstart Ihr \spacefill\mbox{Arthur.}\pend{}
         
         \endnumbering\mylabel{h}\end{ledgroupsized}  \newcommand{\dateiname}{L00454}\newcommand{\titel}{Arthur Schnitzler an Richard Beer-Hofmann, 15. 6. 1895}\newcommand{\editorInnen}{Martin Anton Müller und Gerd-Hermann Susen}%% latex-leseansicht-abspann.tex
%% Abspann für die Leseansicht.
%% Der Schalter \ifkorrekturansicht ist bereits durch den Vorspann gesetzt.

%% latex-abspann.tex
%% Gemeinsamer Abspann für Korrekturansicht und Leseansicht.
%% Setzt den Schalter \ifkorrekturansicht voraus (gesetzt in den
%% einbindenden Dateien latex-korrekturansicht-abspann.tex bzw.
%% latex-leseansicht-abspann.tex).
%% ---------------------------------------------------------------

\normalsize

% Das esempio-Environment wird nur in der Leseansicht benötigt
\ifkorrekturansicht\else
\newenvironment{esempio}[3]%
{
    \vspace{1.5ex}
    \rlap{\underline{#1}}
    \par
    \setlength{\parindent}{0cm}
    \nopagebreak
    \leftskip=#2cm
    \rightskip=#3cm
}
{
    \par
}
\fi

\doendnotes{C}
\bigskip
\vfill

\clearpage

\footnotesize

\ifkorrekturansicht
  \lohead{\textsc{register}}
\fi

% theindex-Environment neu definieren ohne reledmac
\makeatletter
\renewenvironment{theindex}{%
  \ifkorrekturansicht
    \section*{\indexname}%
  \else
    \subsubsection*{Index der erwähnten Entitäten}%
  \fi
  \setlength{\parindent}{0pt}%
  \setlength{\parskip}{0pt plus 0.3pt}%
  \let\item\@idxitem
}{%
  \ifkorrekturansicht\clearpage\fi
}
\makeatother

\IfFileExists{\jobname-pw.ind}{\input{\jobname-pw.ind}}{}

% Quellenangabe nur in der Leseansicht
\ifkorrekturansicht\else
% Fallback-Definitionen, falls die .tex-Datei \titel etc. nicht gesetzt hat
\providecommand{\titel}{}
\providecommand{\editorInnen}{}
\providecommand{\dateiname}{\jobname}

\vspace{3cm}

\vfill

\footnotesize
\textsc{Quelle}: \titel. Herausgegeben von {\editorInnen}. In: \emph{Arthur Schnitzler: Briefwechsel mit Autorinnen und Autoren}.
 Digitale Edition, https://schnitzler-briefe.acdh.oeaw.ac.at/{\dateiname}.html (Stand \today)
\fi

\end{document}


      