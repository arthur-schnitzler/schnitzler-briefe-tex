%% latex-leseansicht-vorspann.tex
%% Vorspann für die Leseansicht.
%% Lädt die gemeinsame Datei latex-vorspann.tex mit nicht gesetztem Schalter.

\newif\ifkorrekturansicht
\korrekturansichtfalse

\input{../tex-inputs/latex-vorspann}


\section[Arthur Schnitzler an Richard Beer-Hofmann, 15. 6. 1895]{L00454 Arthur Schnitzler an Richard Beer-Hofmann, 15. 6. 1895}
\nopagebreak\mylabel{L00454v}
\rehead{ }\normalsize\beginnumbering\briefempfaengerindex{Beer-Hofmann, Richard@\textsc{Beer-Hofmann, Richard}!zzzSchnitzler, Arthur@\emph{von Arthur Schnitzler}!1895-06-151@{15. 6. 1895}|(be}
\toendnotes[C]{\smallbreak\pagebreak[2]}
\correspDesc{Versand  durch Arthur Schnitzler am 15. 6. 1895 in Wien
\newline{}Erhalt  durch Richard Beer-Hofmann am 16. 6. 1895 in Caslau}\toendnotes[C]{\smallbreak}
\Standort{YCGL, MSS 31.}
\physDesc{Brief, 1 Blatt, 4 Seiten, Kuvert, 1730 Zeichen
\newline{}Handschrift: 1) Bleistift, deutsche Kurrent\hspace{1em}2) schwarze Tinte, deutsche Kurrent (\noindent{}Umschlag)\hspace{1em}
\newline{}Versand: 1) Stempel: »\nobreak{}\oindex{I., Innere Stadt@\textbf{I., Innere Stadt}, \emph{Verwaltungsgebiet}|pwk}Wien 1/1, 15. 6. 95, 7–8 N\nobreak{}«.   2) Stempel: »\nobreak{}\oindex{Čáslav@\textbf{Čáslav}|pwk}Časlau, 16 6 95\nobreak{}«. }
\buchAbdrucke{\weitereDrucke{1) Arthur Schnitzler: \emph{Briefe 1875–1912}. Herausgegeben von Therese Nickl und Heinrich Schnitzler. Frankfurt am Main: \emph{S. Fischer} 1981, S. 260–261.} \weitereDrucke{2) Arthur Schnitzler, Richard Beer-Hofmann: \emph{Briefwechsel 1891–1931}. Herausgegeben von Konstanze Fliedl. Wien, Zürich: \emph{Europaverlag} 1992, S. 74–75.} }\toendnotes[C]{\smallbreak}\pstart{}{\pb}Herrn KuK u. u. \textsc{Lieutenant}\pend{}\pstart{}\textsc{Dr. Richard Beer-Hofmann}\pend{}\pstart{}im \textsc{Kh. Landw.-Inf}-Regmt\pend{}\pstart{}\textsc{»Caslau\oindex{Čáslav@\textbf{Čáslav}|pw}«
                  Nr 12}.\pend{}{\bigskip}\vspace{1em}
\pstart
           \raggedleft{}{\pb}15. Juni 95\pend
           \vspace{0.5em}
\pstart
           Lieber Richard, heut bin ich{ }ſo{ }ſchlecht aufgelegt, als wär ich in
                  \textsc{Caslau}\oindex{Čáslav@\textbf{Čáslav}|pw}. – Einer der Gründe:{ }ſchiefe Stellung in der Familie; Bemerkungen, daſs ich
               »ohne einen Kreuzer Geld zu haben« im So{\geminationm}er nach \textsc{Kopenhagen}\oindex{Kopenhagen@\textbf{Kopenhagen}, \emph{Hauptstadt}|pw} fahren will – Bemerkungen, die mir von dritter, nein vierter Seite
               zurückkommen. –\pend
           
\pstart
           \textsc{Dörmann\pwindex{Dörmann, Felix 29.\,5.\,1870 Wien – 26.\,10.\,1928 ebd.@\textsc{Dörmann, Felix} (29.\,5.\,1870 Wien – 26.\,10.\,1928 ebd.), \emph{Schriftsteller}|pw}} iſt da und erzählt viele Dinge von{ }ſich – er hat 3 Stücke geſchrieben und hat
                  \introOben{}in Berlin\oindex{Berlin@\textbf{Berlin}, \emph{Hauptstadt}|pw}\introOben{} 65 Verhältniſſe gehabt. Ich übertreibe nicht. Er aber ja {\dots} a {\dots} a –\pend
           
\pstart
           – Die Kritik\pwindex{Kraus, Karl 28.\,4.\,1874 Jičín – 12.\,6.\,1936 Wien@\textsc{Kraus, Karl} (28.\,4.\,1874 Jičín – 12.\,6.\,1936 Wien), \emph{Schriftsteller, Publizist, Schriftsteller}!Fanny Gröger, »Adhimukti«@\strich\emph{Fanny Gröger, »Adhimukti«}|pwv} vom kleinen Kraus\pwindex{Kraus, Karl 28.\,4.\,1874 Jičín – 12.\,6.\,1936 Wien@\textsc{Kraus, Karl} (28.\,4.\,1874 Jičín – 12.\,6.\,1936 Wien), \emph{Schriftsteller, Publizist, Schriftsteller}|pw} in dem {\pb}Abendblatt der N. Fr. Pr.\orgindex{Neue Freie Presse@Neue Freie Presse|pw} über die Gröger\pwindex{Gröger, Fanny 12.\,1.\,1869 Wien – 7.\,4.\,1936 ebd.@\textsc{Gröger, Fanny} (12.\,1.\,1869 Wien – 7.\,4.\,1936 ebd.), \emph{Schriftstellerin}|pw} haben Sie geleſen? Er benützt die
               Gelegenheit, uns (Sie, \textsc{Loris}\pwindex{Hofmannsthal, Hugo von 1.\,2.\,1874 Wien – 15.\,7.\,1929 Rodaun@\textsc{Hofmannsthal, Hugo von} (1.\,2.\,1874 Wien – 15.\,7.\,1929 Rodaun), \emph{Schriftsteller}|pw}{ }\introOben{}\textsc{Salten}\pwindex{Salten, Felix 6.\,9.\,1869 Budapest – 8.\,10.\,1945 Zürich@\textsc{Salten, Felix} (6.\,9.\,1869 Budapest – 8.\,10.\,1945 Zürich), \emph{Schriftsteller, Journalist, Chefredakteur}|pw}\introOben{} mich) in die Waden zu beißen.\strikeout{)} Wir werden noch{ }ſchmerzlicheres zu überleben haben. –\pend
           
\pstart
           \textsc{Frauenlob}\pwindex{Lothar, Rudolf 23.\,2.\,1865 Budapest – 2.\,10.\,1943 ebd.@\textsc{Lothar, Rudolf} (23.\,2.\,1865 Budapest – 2.\,10.\,1943 ebd.), \emph{Schriftsteller, Journalist, Theaterdirektor}!Frauenlob. Ein Lustspiel in drei Aufzügen@\strich\emph{Frauenlob. Ein Lustspiel in drei Aufzügen}|pw} von Hrn. \textsc{Lothar\pwindex{Lothar, Rudolf 23.\,2.\,1865 Budapest – 2.\,10.\,1943 ebd.@\textsc{Lothar, Rudolf} (23.\,2.\,1865 Budapest – 2.\,10.\,1943 ebd.), \emph{Schriftsteller, Journalist, Theaterdirektor}|pw}} an der Burg\oindex{Wien@\textbf{Wien}!I., Innere Stadt@\textbf{I., Innere Stadt}!Burgtheater@\textbf{Burgtheater}, \emph{Theater}|pw}{ }\label{K_L00454-1v}\edtext{angenommen}{\lemma{\textnormal{\emph{angenommen}}}\Cendnote{\textnormal{Zu einer Aufführung kam es aber nicht.}}}\label{K_L00454-1}. – Gerücht über
                  »Liebelei\pwindex{Schnitzler, Arthur 15.\,5.\,1862 Wien – 21.\,10.\,1931 ebd.@\textsc{Schnitzler, Arthur} (15.\,5.\,1862 Wien – 21.\,10.\,1931 ebd.), \emph{Schriftsteller, Mediziner}!Liebelei. Schauspiel in drei Akten@\strich\emph{Liebelei. Schauspiel in drei Akten}|pw}«: es werde überhaupt nicht an der
                  Burg\oindex{Wien@\textbf{Wien}!I., Innere Stadt@\textbf{I., Innere Stadt}!Burgtheater@\textbf{Burgtheater}, \emph{Theater}|pw} zur Aufführung kommen. Entſtehung liegt
               nahe; werde Burckh.\pwindex{Burckhard, Max Eugen 14.\,7.\,1854 Korneuburg – 16.\,3.\,1912 Wien@\textsc{Burckhard, Max Eugen} (14.\,7.\,1854 Korneuburg – 16.\,3.\,1912 Wien), \emph{Schriftsteller, Rechtswissenschaftler, Theaterleiter}|pw} aufſuchen.\pend
           
\pstart
           – Für den Abdruck der \textsc{Kl. Komödie}\pwindex{Schnitzler, Arthur 15.\,5.\,1862 Wien – 21.\,10.\,1931 ebd.@\textsc{Schnitzler, Arthur} (15.\,5.\,1862 Wien – 21.\,10.\,1931 ebd.), \emph{Schriftsteller, Mediziner}!kleine Komödie@\strich\emph{Die kleine Komödie}|pw}{ }{\pb}in der \textsc{Freien Bühne}\pwindex{Neue Deutsche Rundschau@\emph{Neue Deutsche Rundschau}|pw} will \textsc{Fischer}\pwindex{Fischer, Samuel 24.\,12.\,1859 Liptovský Mikuláš – 15.\,10.\,1934 Berlin@\textsc{Fischer, Samuel} (24.\,12.\,1859 Liptovský Mikuláš – 15.\,10.\,1934 Berlin), \emph{Verleger}|pw} mir 25, bitte, 25 Mark bezahlen. Ich hab ihm einen groben Brief geſchrieben –
               da mir ja nichts dran liegt. Was haben Sie gegen \textsc{Zasche\pwindex{Zasche, Theodor 18.\,10.\,1862 Wien – 15.\,11.\,1922 ebd.@\textsc{Zasche, Theodor} (18.\,10.\,1862 Wien – 15.\,11.\,1922 ebd.), \emph{Zeichner, Karikaturist}|pw}}? Er wird das ganz hübſch machen. – Die Novelle\pwindex{Schnitzler, Arthur 15.\,5.\,1862 Wien – 21.\,10.\,1931 ebd.@\textsc{Schnitzler, Arthur} (15.\,5.\,1862 Wien – 21.\,10.\,1931 ebd.), \emph{Schriftsteller, Mediziner}!kleine Komödie@\strich\emph{Die kleine Komödie}|pwv} zu datiren hat keinen Sinn; es kü{\geminationm}ert{ }ſich doch keiner drum und{ }ſieht aus wie eine
               Entſchuldigung. –\pend
           
\pstart
           Ich{ }ſchreibe an meinem Stück\pwindex{Schnitzler, Arthur 15.\,5.\,1862 Wien – 21.\,10.\,1931 ebd.@\textsc{Schnitzler, Arthur} (15.\,5.\,1862 Wien – 21.\,10.\,1931 ebd.), \emph{Schriftsteller, Mediziner}!Freiwild. Schauspiel in 3 Akten@\strich\emph{Freiwild. Schauspiel in 3 Akten}|pwv} –
               vorläufig ohne an eine Aufführungs{\pb}möglichkeit zu
               denken. –\pend
           
\pstart
           Meine Abſicht iſt, Anfang Juli in die böhm.\oindex{Böhmen@\textbf{Böhmen}, \emph{Region}|pw} Bäder zu reiſen und vor Mitte Juli in Iſchl\oindex{Bad Ischl@\textbf{Bad Ischl}|pw} zu{ }ſein. – Wann wollen Sie nach München\oindex{München@\textbf{München}|pw} gehn? – Wie{ }ſtehn Sie zu Kopenhagen\oindex{Kopenhagen@\textbf{Kopenhagen}, \emph{Hauptstadt}|pw}? Beantworten Sie gütigſt. – Goldmann\pwindex{Goldmann, Paul 31.\,1.\,1865 Breslau – 25.\,9.\,1935 Wien@\textsc{Goldmann, Paul} (31.\,1.\,1865 Breslau – 25.\,9.\,1935 Wien), \emph{Schriftsteller, Journalist}|pw} wird im Auguſt Urlaub nehmen, genaueres
               unbekannt.\pend
           
\pstart
           – Mein rechtes Ohr laß ich behandeln, das macht mich auch recht nervös. –\pend
           
\pstart
           Leben Sie wohl,{ }ſeien Sie herzlich gegrüßt.\pend
           \pstart Ihr \spacefill\mbox{Arthur.}\pend{}\selectlanguage{ngerman}\endnumbering\briefempfaengerindex{Beer-Hofmann, Richard@\textsc{Beer-Hofmann, Richard}!zzzSchnitzler, Arthur@\emph{von Arthur Schnitzler}!1895-06-151@{15. 6. 1895}|)be}\mylabel{L00454h}  \newcommand{\dateiname}{L00454}\newcommand{\titel}{Arthur Schnitzler an Richard Beer-Hofmann, 15. 6. 1895}\newcommand{\editorInnen}{Martin Anton Müller und Gerd-Hermann Susen}%% latex-leseansicht-abspann.tex
%% Abspann für die Leseansicht.
%% Der Schalter \ifkorrekturansicht ist bereits durch den Vorspann gesetzt.

%% latex-abspann.tex
%% Gemeinsamer Abspann für Korrekturansicht und Leseansicht.
%% Setzt den Schalter \ifkorrekturansicht voraus (gesetzt in den
%% einbindenden Dateien latex-korrekturansicht-abspann.tex bzw.
%% latex-leseansicht-abspann.tex).
%% ---------------------------------------------------------------

\normalsize

% Das esempio-Environment wird nur in der Leseansicht benötigt
\ifkorrekturansicht\else
\newenvironment{esempio}[3]%
{
    \vspace{1.5ex}
    \rlap{\underline{#1}}
    \par
    \setlength{\parindent}{0cm}
    \nopagebreak
    \leftskip=#2cm
    \rightskip=#3cm
}
{
    \par
}
\fi

\doendnotes{C}
\bigskip
\vfill

\clearpage

\footnotesize

\ifkorrekturansicht
  \lohead{\textsc{register}}
\fi

% theindex-Environment neu definieren ohne reledmac
\makeatletter
\renewenvironment{theindex}{%
  \ifkorrekturansicht
    \section*{\indexname}%
  \else
    \subsubsection*{Index der erwähnten Entitäten}%
  \fi
  \setlength{\parindent}{0pt}%
  \setlength{\parskip}{0pt plus 0.3pt}%
  \let\item\@idxitem
}{%
  \ifkorrekturansicht\clearpage\fi
}
\makeatother

\IfFileExists{\jobname-pw.ind}{\input{\jobname-pw.ind}}{}

% Quellenangabe nur in der Leseansicht
\ifkorrekturansicht\else
% Fallback-Definitionen, falls die .tex-Datei \titel etc. nicht gesetzt hat
\providecommand{\titel}{}
\providecommand{\editorInnen}{}
\providecommand{\dateiname}{\jobname}

\vspace{3cm}

\vfill

\footnotesize
\textsc{Quelle}: \titel. Herausgegeben von {\editorInnen}. In: \emph{Arthur Schnitzler: Briefwechsel mit Autorinnen und Autoren}.
 Digitale Edition, https://schnitzler-briefe.acdh.oeaw.ac.at/{\dateiname}.html (Stand \today)
\fi

\end{document}


