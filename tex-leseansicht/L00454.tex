%% latex-korrekturansicht-vorspann.tex
%% Vorspann für die Korrekturansicht.
%% Lädt die gemeinsame Datei latex-vorspann.tex mit gesetztem Schalter.

\newif\ifkorrekturansicht
\korrekturansichttrue

\input{../tex-inputs/latex-vorspann}


\section[Arthur Schnitzler an Richard Beer-Hofmann, 15. 6. 1895]{L00454 Arthur Schnitzler an Richard Beer-Hofmann, 15. 6. 1895}
\nopagebreak\mylabel{L00454v}
\rehead{ }\normalsize\beginnumbering\briefempfaengerindex{Beer-Hofmann, Richard@\textsc{Beer-Hofmann, Richard}!zzzSchnitzler, Arthur@\emph{von Arthur Schnitzler}!1895-06-151@{15. 6. 1895}|(be}
\toendnotes[C]{\smallbreak\pagebreak[2]}\Standort{YCGL, MSS 31.}
\physDesc{Brief, 1 Blatt, 4 Seiten, Umschlag, 1730 Zeichen
\newline{}Handschrift: 1) Bleistift, deutsche Kurrent\hspace{1em}2) schwarze Tinte, deutsche Kurrent (\noindent{}Umschlag)\hspace{1em}
\newline{}Versand: 1) Stempel: »\nobreak{}\oindex{I., Innere Stadt@\textbf{I., Innere Stadt}, \emph{A.ADM3}|pwk}Wien 1/1, 15. 6. 95, 7–8 N\nobreak{}«.   2) Stempel: »\nobreak{}\oindex{Cáslav@\textbf{Čáslav}, \emph{P.PPL}|pwk}Časlau, 16 6 95\nobreak{}«. }
\buchAbdrucke{\weitereDrucke{1) Arthur Schnitzler: \emph{Briefe 1875–1912}. Frankfurt am Main: \emph{S. Fischer} 1981, S. 260–261.} \weitereDrucke{2) Arthur Schnitzler, Richard Beer-Hofmann: \emph{Briefwechsel 1891–1931}. Wien, Zürich: \emph{Europaverlag} 1992, S. 74–75.} }\toendnotes[C]{\smallbreak}\pstart{}{\pb}Herrn KuK u. u. \textsc{Lieutenant}\pend{}\pstart{}\textsc{Dr. Richard Beer-Hofmann}\pend{}\pstart{}im \textsc{Kh. Landw.-Inf}-Regmt\pend{}\pstart{}\textsc{»Caslau\oindex{Cáslav@\textbf{Čáslav}, \emph{P.PPL}|pw}«
                  Nr 12}.\pend{}{\bigskip}\vspace{1em}
\pstart
           \raggedleft{}{\pb}15. Juni 95\pend
           \vspace{0.5em}
\pstart
           Lieber Richard, heut bin ich ſo ſchlecht aufgelegt, als wär ich in
                  \textsc{Caslau}\oindex{Cáslav@\textbf{Čáslav}, \emph{P.PPL}|pw}. – Einer der Gründe: ſchiefe Stellung in der Familie; Bemerkungen, daſs ich
               »ohne einen Kreuzer Geld zu haben« im So{\geminationm}er nach \textsc{Kopenhagen}\oindex{Kopenhagen@\textbf{Kopenhagen}, \emph{P.PPLC}|pw} fahren will – Bemerkungen, die mir von dritter, nein vierter Seite
               zurückkommen. –\pend
           
\pstart
           \textsc{Dörmann\pwindex{Doermann, Felix 29.05.1870 – 26.10.1928@\textsc{Dörmann, Felix} (29.05.1870 – 26.10.1928), \emph{Schriftsteller/Schriftstellerin}|pw}} iſt da und erzählt viele Dinge von ſich – er hat 3 Stücke geſchrieben und hat
                  \introOben{}in Berlin\oindex{Berlin@\textbf{Berlin}, \emph{P.PPLC}|pw}\introOben{} 65 Verhältniſſe gehabt. Ich übertreibe nicht. Er aber ja {\dots} a {\dots} a –\pend
           
\pstart
           – Die Kritik\pwindex{Fanny Groeger, »Adhimukti«@\emph{Fanny Gröger, »Adhimukti«}|pwv} vom kleinen Kraus\pwindex{Kraus, Karl 28.04.1874 – 12.06.1936@\textsc{Kraus, Karl} (28.04.1874 – 12.06.1936), \emph{Schriftsteller/Schriftstellerin, Publizist/Publizistin, Schriftsteller/Schriftstellerin}|pw} in dem {\pb}Abendblatt der N. Fr. Pr.\orgindex{Neue Freie Presse@Neue Freie Presse|pw} über die Gröger\pwindex{Groeger, Fanny 12.01.1869 – 07.04.1936@\textsc{Gröger, Fanny} (12.01.1869 – 07.04.1936), \emph{Schriftsteller/Schriftstellerin}|pw} haben Sie geleſen? Er benützt die
               Gelegenheit, uns (Sie, \textsc{Loris}\pwindex{Hofmannsthal, Hugo von 1874-02-01 – 1929-07-15@\textsc{Hofmannsthal, Hugo von} (1874-02-01 – 1929-07-15), \emph{Schriftsteller/Schriftstellerin}|pw}{ }\introOben{}\textsc{Salten}\pwindex{Salten, Felix 06.09.1869 – 08.10.1945@\textsc{Salten, Felix} (06.09.1869 – 08.10.1945), \emph{Schriftsteller/Schriftstellerin, Journalist/Journalistin, Chefredakteur/Chefredakteurin}|pw}\introOben{} mich) in die Waden zu beißen.\strikeout{)} Wir werden noch
               ſchmerzlicheres zu überleben haben. –\pend
           
\pstart
           \textsc{Frauenlob}\pwindex{Frauenlob. Ein Lustspiel in drei Aufzuegen@\emph{Frauenlob. Ein Lustspiel in drei Aufzügen}|pw} von Hrn. \textsc{Lothar\pwindex{Lothar, Rudolf 23.2.1865 – 2.10.1943@\textsc{Lothar, Rudolf} (23.2.1865 – 2.10.1943), \emph{Schriftsteller/Schriftstellerin, Journalist/Journalistin, Theaterdirektor/Theaterdirektorin}|pw}} an der Burg\oindex{Burgtheater@\textbf{Burgtheater}, \emph{S.THTR}|pw}{ }\label{K_L00454-1v}\edtext{angenommen}{\lemma{\textnormal{\emph{angenommen}}}\Cendnote{\textnormal{Zu einer Aufführung kam es aber nicht.}}}\label{K_L00454-1}. – Gerücht über
                  »Liebelei\pwindex{Liebelei. Schauspiel in drei Akten@\emph{Liebelei. Schauspiel in drei Akten}|pw}«: es werde überhaupt nicht an der
                  Burg\oindex{Burgtheater@\textbf{Burgtheater}, \emph{S.THTR}|pw} zur Aufführung kommen. Entſtehung liegt
               nahe; werde Burckh.\pwindex{Burckhard, Max Eugen 14.07.1854 – 16.03.1912@\textsc{Burckhard, Max Eugen} (14.07.1854 – 16.03.1912), \emph{Schriftsteller/Schriftstellerin, Rechtswissenschaftler/Rechtswissenschaftlerin, Theaterleiter/Theaterleiterin}|pw} aufſuchen.\pend
           
\pstart
           – Für den Abdruck der \textsc{Kl. Komödie}\pwindex{kleine Komoedie@\emph{Die kleine Komödie}|pw}{ }{\pb}in der \textsc{Freien Bühne}\pwindex{Neue Deutsche Rundschau@\emph{Neue Deutsche Rundschau}|pw} will \textsc{Fischer}\pwindex{Fischer, Samuel 24.12.1859 – 15.10.1934@\textsc{Fischer, Samuel} (24.12.1859 – 15.10.1934), \emph{Verleger/Verlegerin}|pw} mir 25, bitte, 25 Mark bezahlen. Ich hab ihm einen groben Brief geſchrieben –
               da mir ja nichts dran liegt. Was haben Sie gegen \textsc{Zasche\pwindex{Zasche, Theodor 18.10.1862 – 15.11.1922@\textsc{Zasche, Theodor} (18.10.1862 – 15.11.1922), \emph{Zeichner/Zeichnerin, Karikaturist/Karikaturistin}|pw}}? Er wird das ganz hübſch machen. – Die Novelle\pwindex{kleine Komoedie@\emph{Die kleine Komödie}|pwv} zu datiren hat keinen Sinn; es kü{\geminationm}ert ſich doch keiner drum und ſieht aus wie eine
               Entſchuldigung. –\pend
           
\pstart
           Ich ſchreibe an meinem Stück\pwindex{Freiwild. Schauspiel in 3 Akten@\emph{Freiwild. Schauspiel in 3 Akten}|pwv} –
               vorläufig ohne an eine Aufführungs{\pb}möglichkeit zu
               denken. –\pend
           
\pstart
           Meine Abſicht iſt, Anfang Juli in die böhm.\oindex{Boehmen@\textbf{Böhmen}, \emph{L.RGN}|pw} Bäder zu reiſen und vor Mitte Juli in Iſchl\oindex{Bad Ischl@\textbf{Bad Ischl}, \emph{P.PPL}|pw} zu ſein. – Wann wollen Sie nach München\oindex{Muenchen@\textbf{München}, \emph{P.PPLA}|pw} gehn? – Wie ſtehn Sie zu Kopenhagen\oindex{Kopenhagen@\textbf{Kopenhagen}, \emph{P.PPLC}|pw}? Beantworten Sie gütigſt. – Goldmann\pwindex{Goldmann, Paul 31.01.1865 – 25.09.1935@\textsc{Goldmann, Paul} (31.01.1865 – 25.09.1935), \emph{Schriftsteller/Schriftstellerin, Journalist/Journalistin}|pw} wird im Auguſt Urlaub nehmen, genaueres
               unbekannt.\pend
           
\pstart
           – Mein rechtes Ohr laß ich behandeln, das macht mich auch recht nervös. –\pend
           
\pstart
           Leben Sie wohl, ſeien Sie herzlich gegrüßt.\pend
           \pstart Ihr \spacefill\mbox{Arthur.}\pend{}\selectlanguage{ngerman}\endnumbering\briefempfaengerindex{Beer-Hofmann, Richard@\textsc{Beer-Hofmann, Richard}!zzzSchnitzler, Arthur@\emph{von Arthur Schnitzler}!1895-06-151@{15. 6. 1895}|)be}\mylabel{L00454h}  \normalsize

\doendnotes{C}
\bigskip
\vfill

\clearpage

\footnotesize

\lohead{\textsc{register}}

% Definiere theindex-Environment komplett neu ohne reledmac
\makeatletter
\renewenvironment{theindex}{%
  \section*{\indexname}%
  \setlength{\parindent}{0pt}%
  \setlength{\parskip}{0pt plus 0.3pt}%
  \let\item\@idxitem
}{%
  \clearpage
}
\makeatother

\IfFileExists{\jobname-pw.ind}{\input{\jobname-pw.ind}}{}

\end{document}

      