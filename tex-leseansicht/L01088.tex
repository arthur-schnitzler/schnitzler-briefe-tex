%% latex-leseansicht-vorspann.tex
%% Vorspann für die Leseansicht.
%% Lädt die gemeinsame Datei latex-vorspann.tex mit nicht gesetztem Schalter.

\newif\ifkorrekturansicht
\korrekturansichtfalse

\input{../tex-inputs/latex-vorspann}


\section[Hugo von Hofmannsthal an Arthur Schnitzler, 22. 12. 1900]{L01088 Hugo von Hofmannsthal an Arthur Schnitzler, 22. 12. 1900}
\nopagebreak\mylabel{L01088v}
\rehead{ }\normalsize\beginnumbering\briefempfaengerindex{Schnitzler, Arthur@\textsc{Schnitzler, Arthur}!zzzHofmannsthal, Hugo von@\emph{von Hugo von Hofmannsthal}!1900-12-222@{22. 12. 1900}|(be}
\toendnotes[C]{\smallbreak\pagebreak[2]}
\correspDesc{Versand  durch Hugo von Hofmannsthal am 22. 12. 1900 in Wien
\newline{}Erhalt  durch Arthur Schnitzler am 22. 12. 1900 in Wien}\toendnotes[C]{\smallbreak}
\Standort{CUL, Schnitzler, B 43.}
\physDesc{Postkarte, 281 Zeichen
\newline{}Handschrift: schwarze Tinte, deutsche Kurrent
\newline{}Versand: 1) Rohrpost  2) Stempel: »\nobreak{}\oindex{III., Landstraße@\textbf{III., Landstraße}, \emph{Verwaltungsgebiet}|pwk}Wien 3/3, 22 XII 00, 5 30N\nobreak{}«.  3) Stempel: »\nobreak{}\oindex{IX., Alsergrund@\textbf{IX., Alsergrund}, \emph{Verwaltungsgebiet}|pwk}Wien 9/2, \textcolor{gray}{22} XII 00, 5 {[}40N{]}\nobreak{}«. 
\newline{}Schnitzler: mit Bleistift datiert: »25/12 900« 
\newline{}Ordnung: mit Bleistift von unbekannter Hand mehrfach nummeriert, diese
                                 gestrichen und zuletzt geändert zu: »170« }
\buchAbdrucke{\weitereDrucke{Hugo von Hofmannsthal, Arthur Schnitzler: \emph{Briefwechsel}. Herausgegeben von Therese Nickl und Heinrich Schnitzler. Frankfurt am Main: \emph{S. Fischer} 1964, S. 145.} }\toendnotes[C]{\smallbreak}\pstart{}\textsc{{\pb}Herrn D\textsuperscript{r} Arthur Schnitzler}\pend{}\pstart{}\textsc{IX. Franckgasse 1.\oindex{Wien@\textbf{Wien}!IX., Alsergrund@\textbf{IX., Alsergrund}!Frankgasse 1@\textbf{Frankgasse 1}, \emph{Wohngebäude}|pw}}\pend{}\pstart{}\textsc{Wien\oindex{Wien@\textbf{Wien}, \emph{Verwaltungsgebiet}|pw}}\pend{}{\bigskip}\vspace{1em}
\pstart
           \noindent{}{\pb}lieber Arthur, ich bin auch morgen Sonntag wieder bei
                  Richard\pwindex{Beer-Hofmann, Richard 11.\,7.\,1866 Wien – 26.\,9.\,1945 New York City@\textsc{Beer-Hofmann, Richard} (11.\,7.\,1866 Wien – 26.\,9.\,1945 New York City), \emph{Schriftsteller}|pw}, vielleicht daſs Sie gegen
                  ¾ 8 hinko{\geminationm}en, mich abzuholen oder
               gemeinſam dortzubleiben, das wäre{ }ſehr{ }ſchön.\pend
           
\pstart
           Herzlich{\\[\baselineskip]}\spacefill\mbox{Hugo}\pend
           \leftskip=0em{}
\pstart
           Samstag.\pend
           
\pstart
           Man kann Sie nun ruhig den \label{K_L01088-1v}\edtext{\textsc{Kotzebue}\pwindex{Kotzebue, August von 3.\,5.\,1761 Weimar – 23.\,3.\,1819 Mannheim@\textsc{Kotzebue, August von} (3.\,5.\,1761 Weimar – 23.\,3.\,1819 Mannheim), \emph{Schriftsteller}|pwv} der Novelle}{\lemma{\textnormal{\emph{Kotzebue der Novelle}}}\Cendnote{\textnormal{Die Bemerkung erfolgt anlässlich der 
                     bevorstehenden und bereits beworbenen Veröffentlichung von \emph{Lieutenant
                        Gustl}\pwindex{Schnitzler, Arthur 15.\,5.\,1862 Wien – 21.\,10.\,1931 ebd.@\textsc{Schnitzler, Arthur} (15.\,5.\,1862 Wien – 21.\,10.\,1931 ebd.), \emph{Schriftsteller, Mediziner}!Lieutenant Gustl. Novelle@\strich\emph{Lieutenant Gustl. Novelle}|pwk} am 25. 12. 1900 in der \emph{Neuen Freie Presse}\pwindex{Neue Freie Presse@\emph{Neue Freie Presse}|pwk}. Es handelt sich um einen foppenden Vergleich mit August von Kotzebue\pwindex{Kotzebue, August von 3.\,5.\,1761 Weimar – 23.\,3.\,1819 Mannheim@\textsc{Kotzebue, August von} (3.\,5.\,1761 Weimar – 23.\,3.\,1819 Mannheim), \emph{Schriftsteller}|pwk},
                     der ein sehr
                     umfangreiches Theaterwerk von über 200 Stücken hinterlassen hat.}}}\label{K_L01088-1} nennen.\pend
           \selectlanguage{ngerman}\endnumbering\briefempfaengerindex{Schnitzler, Arthur@\textsc{Schnitzler, Arthur}!zzzHofmannsthal, Hugo von@\emph{von Hugo von Hofmannsthal}!1900-12-222@{22. 12. 1900}|)be}\mylabel{L01088h}  \newcommand{\dateiname}{L01088}\newcommand{\titel}{Hugo von Hofmannsthal an Arthur Schnitzler, 22. 12. 1900}\newcommand{\editorInnen}{Martin Anton Müller und Gerd-Hermann Susen}%% latex-leseansicht-abspann.tex
%% Abspann für die Leseansicht.
%% Der Schalter \ifkorrekturansicht ist bereits durch den Vorspann gesetzt.

%% latex-abspann.tex
%% Gemeinsamer Abspann für Korrekturansicht und Leseansicht.
%% Setzt den Schalter \ifkorrekturansicht voraus (gesetzt in den
%% einbindenden Dateien latex-korrekturansicht-abspann.tex bzw.
%% latex-leseansicht-abspann.tex).
%% ---------------------------------------------------------------

\normalsize

% Das esempio-Environment wird nur in der Leseansicht benötigt
\ifkorrekturansicht\else
\newenvironment{esempio}[3]%
{
    \vspace{1.5ex}
    \rlap{\underline{#1}}
    \par
    \setlength{\parindent}{0cm}
    \nopagebreak
    \leftskip=#2cm
    \rightskip=#3cm
}
{
    \par
}
\fi

\doendnotes{C}
\bigskip
\vfill

\clearpage

\footnotesize

\ifkorrekturansicht
  \lohead{\textsc{register}}
\fi

% theindex-Environment neu definieren ohne reledmac
\makeatletter
\renewenvironment{theindex}{%
  \ifkorrekturansicht
    \section*{\indexname}%
  \else
    \subsubsection*{Index der erwähnten Entitäten}%
  \fi
  \setlength{\parindent}{0pt}%
  \setlength{\parskip}{0pt plus 0.3pt}%
  \let\item\@idxitem
}{%
  \ifkorrekturansicht\clearpage\fi
}
\makeatother

\IfFileExists{\jobname-pw.ind}{\input{\jobname-pw.ind}}{}

% Quellenangabe nur in der Leseansicht
\ifkorrekturansicht\else
% Fallback-Definitionen, falls die .tex-Datei \titel etc. nicht gesetzt hat
\providecommand{\titel}{}
\providecommand{\editorInnen}{}
\providecommand{\dateiname}{\jobname}

\vspace{3cm}

\vfill

\footnotesize
\textsc{Quelle}: \titel. Herausgegeben von {\editorInnen}. In: \emph{Arthur Schnitzler: Briefwechsel mit Autorinnen und Autoren}.
 Digitale Edition, https://schnitzler-briefe.acdh.oeaw.ac.at/{\dateiname}.html (Stand \today)
\fi

\end{document}


