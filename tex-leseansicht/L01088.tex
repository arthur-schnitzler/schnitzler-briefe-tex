%% latex-korrekturansicht-vorspann.tex
%% Vorspann für die Korrekturansicht.
%% Lädt die gemeinsame Datei latex-vorspann.tex mit gesetztem Schalter.

\newif\ifkorrekturansicht
\korrekturansichttrue

\input{../tex-inputs/latex-vorspann}


\section[Hugo von Hofmannsthal an Arthur Schnitzler, 22. 12. 1900]{L01088 Hugo von Hofmannsthal an Arthur Schnitzler, 22. 12. 1900}
\nopagebreak\mylabel{L01088v}
\rehead{ }\normalsize\beginnumbering\briefempfaengerindex{Schnitzler, Arthur@\textsc{Schnitzler, Arthur}!zzzHofmannsthal, Hugo von@\emph{von Hugo von Hofmannsthal}!1900-12-221@{22. 12. 1900}|(be}
\toendnotes[C]{\smallbreak\pagebreak[2]}\Standort{CUL, Schnitzler, B 43.}
\physDesc{Postkarte, 281 Zeichen
\newline{}Handschrift: 1) schwarze Tinte, deutsche Kurrent\hspace{1em}2) schwarze Tinte, lateinische Kurrent (\noindent{}Adresse)\hspace{1em}
\newline{}Versand: 1) Rohrpost  2) Stempel: »\nobreak{}\oindex{III., Landstrasse@\textbf{III., Landstraße}, \emph{A.ADM3}|pwk}Wien 3/3, 22 XII 00, 5 30N\nobreak{}«.  3) Stempel: »\nobreak{}\oindex{IX., Alsergrund@\textbf{IX., Alsergrund}, \emph{A.ADM3}|pwk}Wien 9/2, \textcolor{gray}{22} XII 00, 5 {[}40N{]}\nobreak{}«. 
\newline{}Schnitzler: mit Bleistift datiert: »25/12 900« 
\newline{}Ordnung: mit Bleistift von unbekannter Hand mehrfach nummeriert, diese
                                 gestrichen und zuletzt geändert zu: »170« }
\buchAbdrucke{\weitereDrucke{Hugo von Hofmannsthal, Arthur Schnitzler: \emph{Briefwechsel}. Frankfurt am Main: \emph{S. Fischer} 1964, S. 145.} }\toendnotes[C]{\smallbreak}\pstart{}{\pb}Herrn D\textsuperscript{r} Arthur Schnitzler\pend{}\pstart{}IX. Franckgasse 1.\oindex{Frankgasse 1@\textbf{Frankgasse 1}, \emph{Wohngebäude (K.WHS)}|pw}\pend{}\pstart{}Wien\oindex{Wien@\textbf{Wien}, \emph{A.ADM2}|pw}\pend{}{\bigskip}\vspace{1em}
\pstart
           \noindent{}{\pb}lieber Arthur, ich bin auch morgen Sonntag wieder bei
                  Richard\pwindex{Beer-Hofmann, Richard 1866-07-11 – 1945-09-26@\textsc{Beer-Hofmann, Richard} (1866-07-11 – 1945-09-26), \emph{Schriftsteller/Schriftstellerin}|pw}, vielleicht daſs Sie gegen
                  ¾ 8 hinko{\geminationm}en, mich abzuholen oder
               gemeinſam dortzubleiben, das wäre ſehr ſchön.\pend
           
\pstart
           Herzlich{\\[\baselineskip]}\spacefill\mbox{Hugo}\pend
           \leftskip=0em{}
\pstart
           Samstag.\pend
           
\pstart
           Man kann Sie nun ruhig den \label{K_L01088-1v}\edtext{\textsc{Kotzebue}\pwindex{Kotzebue, August von 03.05.1761 – 23.03.1819@\textsc{Kotzebue, August von} (03.05.1761 – 23.03.1819), \emph{Schriftsteller/Schriftstellerin}|pwv} der Novelle}{\lemma{\textnormal{\emph{Kotzebue der Novelle}}}\Cendnote{\textnormal{Die Bemerkung erfolgt anlässlich der 
                     bevorstehenden und bereits beworbenen Veröffentlichung von \emph{Lieutenant
                        Gustl}\pwindex{Lieutenant Gustl. Novelle@\emph{Lieutenant Gustl. Novelle}|pwk} am 25. 12. 1900 in der \emph{Neuen Freie Presse}\pwindex{Neue Freie Presse@\emph{Neue Freie Presse}|pwk}. Es handelt sich um einen foppenden Vergleich mit August von Kotzebue\pwindex{Kotzebue, August von 03.05.1761 – 23.03.1819@\textsc{Kotzebue, August von} (03.05.1761 – 23.03.1819), \emph{Schriftsteller/Schriftstellerin}|pwk},
                     der ein sehr
                     umfangreiches Theaterwerk von über 200 Stücken hinterlassen hat.}}}\label{K_L01088-1} nennen.\pend
           \selectlanguage{ngerman}\endnumbering\briefempfaengerindex{Schnitzler, Arthur@\textsc{Schnitzler, Arthur}!zzzHofmannsthal, Hugo von@\emph{von Hugo von Hofmannsthal}!1900-12-221@{22. 12. 1900}|)be}\mylabel{L01088h}  \normalsize

\doendnotes{C}
\bigskip
\vfill

\clearpage

\footnotesize

\lohead{\textsc{register}}

% Definiere theindex-Environment komplett neu ohne reledmac
\makeatletter
\renewenvironment{theindex}{%
  \section*{\indexname}%
  \setlength{\parindent}{0pt}%
  \setlength{\parskip}{0pt plus 0.3pt}%
  \let\item\@idxitem
}{%
  \clearpage
}
\makeatother

\IfFileExists{\jobname-pw.ind}{\input{\jobname-pw.ind}}{}

\end{document}

      