\input{../tex-inputs/latex-pdf-vorspann}
\begin{center}
            \textcolor{red}{ENTWURF. ENTZIFFERUNG NOCH NICHT KORREKTURGELESEN}
                      \end{center}
            
               \section[Hugo von Hofmannsthal an Arthur Schnitzler, 19. {[}7. 1898{]}]{ Hugo von Hofmannsthal an Arthur Schnitzler, 19. {[}7. 1898{]}}\nopagebreak\mylabel{v}\rehead{ }\begin{ledgroupsized}[t]{13cm}\normalsize\beginnumbering\briefempfaengerindex{Schnitzler, Arthur@\textsc{Schnitzler, Arthur}!zzzHofmannsthal, Hugo von@\emph{von Hugo von Hofmannsthal}!1898-07-192@{19. {[}7. 1898{]}}|(be} \toendnotes[C]{\smallbreak\pagebreak[2]} \Standort{CUL, Schnitzler, B 43.}
\physDesc{Brief, 1 Blatt, 3 Seiten
\newline{}Handschrift: schwarze Tinte, deutsche Kurrent
\newline{}Schnitzler: mit Bleistift Monat und Jahreszahl ergänzt: »7 98« \newline{}Ordnung: 1) mit Bleistift von unbekannter Hand nummeriert: »\strikeout{120}« 2) mit Bleistift von unbekannter Hand nummeriert: »118«}\buchAbdrucke{\weitereDrucke{Hugo von Hofmannsthal, Arthur Schnitzler: \emph{Briefwechsel}. Hg. Therese Nickl und Heinrich Schnitzler. Frankfurt am Main: \emph{S. Fischer} 1964, S. 106.} }\toendnotes[C]{\smallbreak}\pstart
           \raggedleft{}{\pb}\textsc{Czortków\oindex{Tschortkiw@\textbf{Tschortkiw}|pw}}{ }19\textsuperscript{\textsc{ten}}\pend
           \pstart{}mein lieber Arthur\pend\pstart
           es wäre mir eine \uline{ſehr} große Freude, wenn Sie
                    meine Eltern\pwindex{Hofmannsthal, Hugo August von 21.12.1841 – 08.12.1915@\textsc{Hofmannsthal, Hugo August von} (21.12.1841 – 08.12.1915), \emph{Bankdirektor}|pwv}\pwindex{Hofmannsthal, Anna von 27.01.1849 – 22.03.1904@\textsc{Hofmannsthal, Anna von} (27.01.1849 – 22.03.1904)|pwv} beſuchen würden. Sie
                    ſind ſehr allein, und Sie könnten Ihnen auch von unſrem Plan ſprechen: ich hab
                    bis jetzt nichts von unsrem Plan geſchrieben aus einer merkwürdigen
                    abergläubiſchen Feigheit. Ich will nicht viel erwähnen, {\pb}wie es mir geht; es wird mir
                    ja gewiſs ſehr bald viel beſſer gehen.\pend
           \pstart
           In wunderſchöner lebhafter Erinnerung hab ich vom \textsc{Paracelsus}\pwindex{Schnitzler, Arthur 15.05.1862 – 21.10.1931@\textsc{Schnitzler, Arthur} (15.05.1862 – 21.10.1931), \emph{Schriftsteller, Mediziner}!Paracelsus. Versspiel in einem Akt01. 11. 1898@\strich\emph{Paracelsus. Versspiel in einem Akt} {[}01. 11. 1898{]}|pw} die Führung des Ganzen und wie die Figuren gegeneinander ſtehen –
                    vom Witwer\pwindex{Schnitzler, Arthur 15.05.1862 – 21.10.1931@\textsc{Schnitzler, Arthur} (15.05.1862 – 21.10.1931), \emph{Schriftsteller, Mediziner}!Witwer25. 12. 1894@\strich\emph{Der Witwer} {[}25. 12. 1894{]}|pw}\pwindex{Schnitzler, Arthur 15.05.1862 – 21.10.1931@\textsc{Schnitzler, Arthur} (15.05.1862 – 21.10.1931), \emph{Schriftsteller, Mediziner}!Witwer25. 12. 1894@\strich\emph{Der Witwer} {[}25. 12. 1894{]}|pw} die eine reiche bedeutende
                    Geſtalt. {\pb}Leben Sie wohl und
                    ſchreiben mir, ja!, bald wieder.\pend
           \pstart
           Briefe die Sie nach dem 24\textsuperscript{\textsc{ten}} aufgeben, treffen mich am ſicherſten: Hinterbrühl, Gießhüblerſtraße 2\oindex{Giesshueblerstrasse@\textbf{Gießhüblerstraße}|pw}.\pend
           \pstart
           Von Herzen{\\[\baselineskip]}Ihr{\\[\baselineskip]}\spacefill\mbox{Hugo.}\pend
           \leftskip=0em{}\endnumbering\briefempfaengerindex{Schnitzler, Arthur@\textsc{Schnitzler, Arthur}!zzzHofmannsthal, Hugo von@\emph{von Hugo von Hofmannsthal}!1898-07-192@{19. {[}7. 1898{]}}|)be}\mylabel{h}\end{ledgroupsized}  \newcommand{\dateiname}{L00825}\newcommand{\titel}{Hugo von Hofmannsthal an Arthur Schnitzler, 19. [7. 1898]}\newcommand{\editorInnen}{Martin Anton Müller und Gerd-Hermann Susen}\input{../tex-inputs/latex-pdf-abspann}
      