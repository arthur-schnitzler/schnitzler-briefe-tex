%% latex-korrekturansicht-vorspann.tex
%% Vorspann für die Korrekturansicht.
%% Lädt die gemeinsame Datei latex-vorspann.tex mit gesetztem Schalter.

\newif\ifkorrekturansicht
\korrekturansichttrue

\input{../tex-inputs/latex-vorspann}


\section[Hugo von Hofmannsthal an Arthur Schnitzler, 19. {[}7. 1898{]}]{L00825 Hugo von Hofmannsthal an Arthur Schnitzler, 19. {[}7. 1898{]}}
\nopagebreak\mylabel{L00825v}
\rehead{ }\normalsize\beginnumbering\briefempfaengerindex{Schnitzler, Arthur@\textsc{Schnitzler, Arthur}!zzzHofmannsthal, Hugo von@\emph{von Hugo von Hofmannsthal}!1898-07-192@{19. {[}7. 1898{]}}|(be}
\toendnotes[C]{\smallbreak\pagebreak[2]}\Standort{CUL, Schnitzler, B 43.}
\physDesc{Brief, 1 Blatt, 3 Seiten, 711 Zeichen
\newline{}Handschrift: schwarze Tinte, deutsche Kurrent
\newline{}Schnitzler: mit Bleistift Monat und Jahreszahl ergänzt: »7 98« 
\newline{}Ordnung: 1) mit Bleistift von unbekannter Hand nummeriert: »\strikeout{120}«  2) mit Bleistift von unbekannter Hand nummeriert:
                                    »118«}
\buchAbdrucke{\weitereDrucke{Hugo von Hofmannsthal, Arthur Schnitzler: \emph{Briefwechsel}. Frankfurt am Main: \emph{S. Fischer} 1964, S. 106.} }\toendnotes[C]{\smallbreak}
\pstart
           \raggedleft{}{\pb}\textsc{Czortków\oindex{Tschortkiw@\textbf{Tschortkiw}, \emph{P.PPLA2}|pw}}{ }19\textsuperscript{\textsc{ten}}\pend
           
\pstart{}mein lieber Arthur\pend\vspace{0.5em}
\pstart
           es wäre mir eine \uline{ſehr} große Freude, wenn Sie meine
                  Eltern\pwindex{Hofmannsthal, Hugo August von 21.12.1841 – 08.12.1915@\textsc{Hofmannsthal, Hugo August von} (21.12.1841 – 08.12.1915), \emph{Bankdirektor/Bankdirektorin}|pwv}\pwindex{Hofmannsthal, Anna von 27.01.1849 – 22.03.1904@\textsc{Hofmannsthal, Anna von} (27.01.1849 – 22.03.1904)|pwv} beſuchen
               würden. Sie ſind ſehr allein, und Sie könnten Ihnen auch von unſrem Plan ſprechen:
               ich hab bis jetzt nichts von unsrem Plan geſchrieben aus einer merkwürdigen
               abergläubiſchen Feigheit. Ich will nicht viel erwähnen, {\pb}wie es mir geht; es wird mir ja
               gewiſs ſehr bald viel beſſer gehen.\pend
           
\pstart
           In wunderſchöner lebhafter Erinnerung hab ich vom \textsc{Paracelsus}\pwindex{Paracelsus. Versspiel in einem Akt@\emph{Paracelsus. Versspiel in einem Akt}|pw} die Führung des Ganzen und wie die Figuren gegeneinander ſtehen – vom Witwer\pwindex{Witwer@\emph{Der Witwer}|pw} die eine reiche bedeutende Geſtalt. {\pb}Leben Sie wohl und ſchreiben mir,
               ja!, bald wieder.\pend
           
\pstart
           Briefe die Sie nach dem 24\textsuperscript{\textsc{ten}} aufgeben, treffen mich am ſicherſten: Hinterbrühl, Gießhüblerſtraße 2\oindex{Giesshueblerstrasse@\textbf{Gießhüblerstraße}, \emph{Straße (K.STR)}|pw}.\pend
           
\pstart
           Von Herzen{\\[\baselineskip]}Ihr{\\[\baselineskip]}\spacefill\mbox{Hugo.}\pend
           \leftskip=0em{}\selectlanguage{ngerman}\endnumbering\briefempfaengerindex{Schnitzler, Arthur@\textsc{Schnitzler, Arthur}!zzzHofmannsthal, Hugo von@\emph{von Hugo von Hofmannsthal}!1898-07-192@{19. {[}7. 1898{]}}|)be}\mylabel{L00825h}  \normalsize

\doendnotes{C}
\bigskip
\vfill

\clearpage

\footnotesize

\lohead{\textsc{register}}

% Definiere theindex-Environment komplett neu ohne reledmac
\makeatletter
\renewenvironment{theindex}{%
  \section*{\indexname}%
  \setlength{\parindent}{0pt}%
  \setlength{\parskip}{0pt plus 0.3pt}%
  \let\item\@idxitem
}{%
  \clearpage
}
\makeatother

\IfFileExists{\jobname-pw.ind}{\input{\jobname-pw.ind}}{}

\end{document}

      