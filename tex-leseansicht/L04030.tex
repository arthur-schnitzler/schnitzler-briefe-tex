%% latex-leseansicht-vorspann.tex
%% Vorspann für die Leseansicht.
%% Lädt die gemeinsame Datei latex-vorspann.tex mit nicht gesetztem Schalter.

\newif\ifkorrekturansicht
\korrekturansichtfalse

\input{../tex-inputs/latex-vorspann}


\section[Arthur Schnitzler an Gustav Schwarzkopf, 17. 4. 1908]{L04030 Arthur Schnitzler an Gustav Schwarzkopf, 17. 4. 1908}
\nopagebreak\mylabel{L04030v}
\rehead{ }\normalsize\beginnumbering\briefempfaengerindex{Schwarzkopf, Gustav@\textsc{Schwarzkopf, Gustav}!zzzSchnitzler, Arthur@\emph{von Arthur Schnitzler}!1908-04-172@{17. 4. 1908}|(be}
\toendnotes[C]{\smallbreak\pagebreak[2]}
\correspDesc{Versand  durch Arthur Schnitzler am 17. 4. 1908 in Wien
\newline{}Erhalt  durch Gustav Schwarzkopf am 17. 4. 1908 in Wien}\toendnotes[C]{\smallbreak}
\Standort{CUL, Schnitzler, B 96.}
\physDesc{Postkarte, 181 Zeichen
\newline{}Handschrift: Bleistift, deutsche Kurrent
\newline{}Versand: 1) Stempel: »\nobreak{}\oindex{XVIII., Währing@\textbf{XVIII., Währing}, \emph{Verwaltungsgebiet}|pwk}1\textcolor{gray}{8/1} Wien 111, 17. IV {[}8{]}\textcolor{gray}{8}\nobreak{}«.   2) Stempel: »\nobreak{}\oindex{I., Innere Stadt@\textbf{I., Innere Stadt}, \emph{Verwaltungsgebiet}|pwk}Wien 1/1, 17 IV 08, 12 \textcolor{gray}{5}0N\nobreak{}«. }\toendnotes[C]{\smallbreak}\pstart{}{\pb}\textcolor{gray}{\textbf{Dr. Arthur Schnitzler}}\pend{}\pstart{}\textcolor{gray}{\textbf{Wien XVIII.
                        Spoettelgasse 7\oindex{Wien@\textbf{Wien}!XVIII., Währing@\textbf{XVIII., Währing}!Edmund-Weiß-Gasse@\textbf{Edmund-Weiß-Gasse}, \emph{Straße}|pw}.}}\pend{}{\bigskip}\pstart{}\textsc{Hrn Gustav Schwarzkopf}\pend{}\pstart{}\textsc{Wien I\oindex{I., Innere Stadt@\textbf{I., Innere Stadt}, \emph{Verwaltungsgebiet}|pw}}\pend{}\pstart{}\textsc{Tiefer Graben 23\oindex{Wien@\textbf{Wien}!I., Innere Stadt@\textbf{I., Innere Stadt}!Tiefer Graben 23@\textbf{Tiefer Graben 23}, \emph{Wohngebäude}|pw}}\pend{}{\bigskip}\vspace{1em}
\pstart
           \noindent{}{\pb}lieber Guſtav, möchten Sie am
                 Oſterſonntag bei uns \label{K_L04030-1v}\edtext{nachtmahlen}{\lemma{\textnormal{\emph{nachtmahlen}}}\Cendnote{\textnormal{Vgl. A. S.: \emph{Tagebuch}, 19. 4. 1908.}}}\label{K_L04030-1}? Würde uns{ }ſehr freuen. Halb acht.
      Bitte jedenfalls um Antwort\pend
           
\pstart
           Herzlichſt{\\[\baselineskip]} Ihr{\\[\baselineskip]}\spacefill\mbox{A.}\pend
           \leftskip=0em{}\selectlanguage{ngerman}\endnumbering\briefempfaengerindex{Schwarzkopf, Gustav@\textsc{Schwarzkopf, Gustav}!zzzSchnitzler, Arthur@\emph{von Arthur Schnitzler}!1908-04-172@{17. 4. 1908}|)be}\mylabel{L04030h}
\begin{anhang}
\end{anhang}\newcommand{\dateiname}{L04030}\newcommand{\titel}{Arthur Schnitzler an Gustav Schwarzkopf, 17. 4. 1908}\newcommand{\editorInnen}{Herausgegeben von Jahnke, SelmaMüller, Martin Anton}%% latex-leseansicht-abspann.tex
%% Abspann für die Leseansicht.
%% Der Schalter \ifkorrekturansicht ist bereits durch den Vorspann gesetzt.

%% latex-abspann.tex
%% Gemeinsamer Abspann für Korrekturansicht und Leseansicht.
%% Setzt den Schalter \ifkorrekturansicht voraus (gesetzt in den
%% einbindenden Dateien latex-korrekturansicht-abspann.tex bzw.
%% latex-leseansicht-abspann.tex).
%% ---------------------------------------------------------------

\normalsize

% Das esempio-Environment wird nur in der Leseansicht benötigt
\ifkorrekturansicht\else
\newenvironment{esempio}[3]%
{
    \vspace{1.5ex}
    \rlap{\underline{#1}}
    \par
    \setlength{\parindent}{0cm}
    \nopagebreak
    \leftskip=#2cm
    \rightskip=#3cm
}
{
    \par
}
\fi

\doendnotes{C}
\bigskip
\vfill

\clearpage

\footnotesize

\ifkorrekturansicht
  \lohead{\textsc{register}}
\fi

% theindex-Environment neu definieren ohne reledmac
\makeatletter
\renewenvironment{theindex}{%
  \ifkorrekturansicht
    \section*{\indexname}%
  \else
    \subsubsection*{Index der erwähnten Entitäten}%
  \fi
  \setlength{\parindent}{0pt}%
  \setlength{\parskip}{0pt plus 0.3pt}%
  \let\item\@idxitem
}{%
  \ifkorrekturansicht\clearpage\fi
}
\makeatother

\IfFileExists{\jobname-pw.ind}{\input{\jobname-pw.ind}}{}

% Quellenangabe nur in der Leseansicht
\ifkorrekturansicht\else
% Fallback-Definitionen, falls die .tex-Datei \titel etc. nicht gesetzt hat
\providecommand{\titel}{}
\providecommand{\editorInnen}{}
\providecommand{\dateiname}{\jobname}

\vspace{3cm}

\vfill

\footnotesize
\textsc{Quelle}: \titel. Herausgegeben von {\editorInnen}. In: \emph{Arthur Schnitzler: Briefwechsel mit Autorinnen und Autoren}.
 Digitale Edition, https://schnitzler-briefe.acdh.oeaw.ac.at/{\dateiname}.html (Stand \today)
\fi

\end{document}


