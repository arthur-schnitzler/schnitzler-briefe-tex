%% latex-leseansicht-vorspann.tex
%% Vorspann für die Leseansicht.
%% Lädt die gemeinsame Datei latex-vorspann.tex mit nicht gesetztem Schalter.

\newif\ifkorrekturansicht
\korrekturansichtfalse

\input{../tex-inputs/latex-vorspann}

\begin{center}
            \textcolor{red}{ENTWURF, NICHT FERTIG KORRIGIERT}
                      \end{center}
            
         
         \renewcommand{\erwaehntePersonen}{Personen:  ?? [Anwalt von Julius Gans-Ludassy],  ?? [Kinderarzt von Paul Salten], Julius von Gans-Ludassy, Gustav Harpner, Anna Katharina Rehmann, Paul Salten}
         \renewcommand{\erwaehnteOrte}{Orte: Wien}
         \renewcommand{\erwaehnteWerke}{}
               \section[Felix Salten an Arthur Schnitzler, {[}20.? 10. 1906{]}]{ Felix Salten an Arthur Schnitzler, {[}20.? 10. 1906{]}}\nopagebreak\mylabel{v}\rehead{ }\begin{ledgroupsized}[t]{13cm}\normalsize\beginnumbering \toendnotes[C]{\smallbreak\pagebreak[2]} \Standort{CUL, Schnitzler, B 89, B 1.}
\physDesc{Brief, 1 Blatt, 1 Seite, 546 Zeichen
\newline{}Handschrift: schwarze Tinte, lateinische Kurrent
\newline{}Schnitzler: mit Bleistift datiert: »Nov {[}1{]}906« 
\newline{}Ordnung: mit Bleistift von unbekannter Hand nummeriert:
                                    »226« }\toendnotes[C]{\smallbreak}\pstart
           \raggedleft{}{\pb}\label{K_L03435-1v}\edtext{Samstag}{\lemma{\textnormal{\emph{Samstag}}}\Cendnote{\textnormal{Die Datierung dieses Korrespondenstücks ist im Abgleich
                        mit dem vorangehenden (Felix Salten an Arthur Schnitzler, [18.? 10. 1906])
                        möglich, doch widerspricht das der Einordnung Schnitzler\pwindex{Schnitzler, Arthur 15.05.1862 – 21.10.1931@\textsc{Schnitzler, Arthur} (15.05.1862 – 21.10.1931), \emph{Schriftsteller, Mediziner}|pwk}s in den November.}}}\label{K_L03435-1h}. \pend
           \pstart{}Lieber,\pend\pstart
           die Verhandlung Ludaſsy\pwindex{Gans-Ludassy, Julius von 13.04.1858 – 30.09.1922@\textsc{Gans-Ludassy, Julius von} (13.04.1858 – 30.09.1922), \emph{Schriftsteller, Journalist, Herausgeber}|pw} am Montag entfällt, da
               der Advokat\pwindex{?? [Anwalt von Julius Gans-Ludassy] @\textsc{?? [Anwalt von Julius Gans-Ludassy]}|pwv} des \uline{Klägers\pwindex{Gans-Ludassy, Julius von 13.04.1858 – 30.09.1922@\textsc{Gans-Ludassy, Julius von} (13.04.1858 – 30.09.1922), \emph{Schriftsteller, Journalist, Herausgeber}|pwv}} meinen Vertreter bat, es möchte die Sache aussergerichtlich beigelegt werden,
               und D\textsuperscript{r} Harpner\pwindex{Harpner, Gustav 25.03.1864 – 10.07.1924@\textsc{Harpner, Gustav} (25.03.1864 – 10.07.1924), \emph{Rechtsanwalt}|pw}
               leider, ohne mich zu fragen, in eine einstweilige Vertagung gewilligt hat. Ich danke
               Ihnen jedenfalls herzlich, für Ihre Bereitwilligkeit, auszusagen. \pend
           \pstart
           Die Kinder\pwindex{Rehmann, Anna Katharina 18.08.1904 – 27.03.1977@\textsc{Rehmann, Anna Katharina} (18.08.1904 – 27.03.1977), \emph{Schauspielerin}|pwv} sind krank. Paul\pwindex{Salten, Paul 11.08.1903 – 08.05.1937@\textsc{Salten, Paul} (11.08.1903 – 08.05.1937), \emph{Filmcutter}|pw} hat eine starke Angina. Der Arzt\pwindex{?? [Kinderarzt von Paul Salten] @\textsc{?? [Kinderarzt von Paul Salten]}|pwv} fürchtete zuerst
               Scharlach. Vorsichtigerweise kann ich mich jetzt weder auf dem Tennisplatz noch sonst
               wo in die Nähe eines Kindesvaters wagen. \pend
           \pstart
           Aufrichtig Ihr {\\[\baselineskip]}\spacefill\mbox{Felix Salten}\pend
           \leftskip=0em{}
         
         \endnumbering\mylabel{h}\end{ledgroupsized}\begin{anhang}\end{anhang}\newcommand{\dateiname}{L03435}\newcommand{\titel}{Felix Salten an Arthur Schnitzler, [20.? 10. 1906]}\newcommand{\editorInnen}{Martin Anton Müller und Laura Untner}%% latex-leseansicht-abspann.tex
%% Abspann für die Leseansicht.
%% Der Schalter \ifkorrekturansicht ist bereits durch den Vorspann gesetzt.

%% latex-abspann.tex
%% Gemeinsamer Abspann für Korrekturansicht und Leseansicht.
%% Setzt den Schalter \ifkorrekturansicht voraus (gesetzt in den
%% einbindenden Dateien latex-korrekturansicht-abspann.tex bzw.
%% latex-leseansicht-abspann.tex).
%% ---------------------------------------------------------------

\normalsize

% Das esempio-Environment wird nur in der Leseansicht benötigt
\ifkorrekturansicht\else
\newenvironment{esempio}[3]%
{
    \vspace{1.5ex}
    \rlap{\underline{#1}}
    \par
    \setlength{\parindent}{0cm}
    \nopagebreak
    \leftskip=#2cm
    \rightskip=#3cm
}
{
    \par
}
\fi

\doendnotes{C}
\bigskip
\vfill

\clearpage

\footnotesize

\ifkorrekturansicht
  \lohead{\textsc{register}}
\fi

% theindex-Environment neu definieren ohne reledmac
\makeatletter
\renewenvironment{theindex}{%
  \ifkorrekturansicht
    \section*{\indexname}%
  \else
    \subsubsection*{Index der erwähnten Entitäten}%
  \fi
  \setlength{\parindent}{0pt}%
  \setlength{\parskip}{0pt plus 0.3pt}%
  \let\item\@idxitem
}{%
  \ifkorrekturansicht\clearpage\fi
}
\makeatother

\IfFileExists{\jobname-pw.ind}{\input{\jobname-pw.ind}}{}

% Quellenangabe nur in der Leseansicht
\ifkorrekturansicht\else
% Fallback-Definitionen, falls die .tex-Datei \titel etc. nicht gesetzt hat
\providecommand{\titel}{}
\providecommand{\editorInnen}{}
\providecommand{\dateiname}{\jobname}

\vspace{3cm}

\vfill

\footnotesize
\textsc{Quelle}: \titel. Herausgegeben von {\editorInnen}. In: \emph{Arthur Schnitzler: Briefwechsel mit Autorinnen und Autoren}.
 Digitale Edition, https://schnitzler-briefe.acdh.oeaw.ac.at/{\dateiname}.html (Stand \today)
\fi

\end{document}


      