%% latex-korrekturansicht-vorspann.tex
%% Vorspann für die Korrekturansicht.
%% Lädt die gemeinsame Datei latex-vorspann.tex mit gesetztem Schalter.

\newif\ifkorrekturansicht
\korrekturansichttrue

\input{../tex-inputs/latex-vorspann}


\section[ Felix Salten an Arthur Schnitzler, {[}20.? 10. 1906{]}]{L03435 Felix Salten an Arthur Schnitzler, {[}20.? 10. 1906{]}}
\nopagebreak\mylabel{L03435v}
\rehead{ }\normalsize\beginnumbering\briefempfaengerindex{Schnitzler, Arthur@\textsc{Schnitzler, Arthur}!zzzSalten, Felix@\emph{von Felix Salten}!1906-10-201@{{[}20.? 10. 1906{]}}|(be}
\toendnotes[C]{\smallbreak\pagebreak[2]}\Standort{CUL, Schnitzler, B 89, B 1.}
\physDesc{Brief, 1 Blatt, 1 Seite, 541 Zeichen
\newline{}Handschrift: schwarze Tinte, lateinische Kurrent
\newline{}Schnitzler: mit Bleistift datiert: »Nov 90\textcolor{gray}{6}« 
\newline{}Ordnung: mit Bleistift von unbekannter Hand nummeriert: »226« }\toendnotes[C]{\smallbreak}
\pstart
           \raggedleft{}{\pb}\label{K_L03435-1v}\edtext{Samstag}{\lemma{\textnormal{\emph{Samstag}}}\Cendnote{\textnormal{Die Datierung dieses Korrespondenzstücks ist im Abgleich
                        mit dem vorangehenden (Felix Salten an Arthur Schnitzler, [18.? 10. 1906])
                        möglich. Bei der Einordnung durch Schnitzler in den November dürfte es sich um einen Fehler handeln.}}}\label{K_L03435-1}.\pend
           
\pstart{}Lieber,\pend\vspace{0.5em}
\pstart
           die Verhandlung Ludaßy\pwindex{Gans-Ludassy, Julius von 13.04.1858 – 30.09.1922@\textsc{Gans-Ludassy, Julius von} (13.04.1858 – 30.09.1922), \emph{Schriftsteller/Schriftstellerin, Journalist/Journalistin, Herausgeber/Herausgeberin}|pw} am Montag entfällt, da der \label{K_L03435-2v}\edtext{Advokat\pwindex{?? [Anwalt von Julius Gans-Ludassy] @\textsc{?? [Anwalt von Julius Gans-Ludassy]}|pwv} des \uline{Klägers\pwindex{Gans-Ludassy, Julius von 13.04.1858 – 30.09.1922@\textsc{Gans-Ludassy, Julius von} (13.04.1858 – 30.09.1922), \emph{Schriftsteller/Schriftstellerin, Journalist/Journalistin, Herausgeber/Herausgeberin}|pwv}}}{\lemma{\textnormal{\emph{Advokat des Klägers}}}\Cendnote{\textnormal{Julius von Gans-Ludassy\pwindex{Gans-Ludassy, Julius von 13.04.1858 – 30.09.1922@\textsc{Gans-Ludassy, Julius von} (13.04.1858 – 30.09.1922), \emph{Schriftsteller/Schriftstellerin, Journalist/Journalistin, Herausgeber/Herausgeberin}|pwk} wurde von Josef Svatopluk Machar\pwindex{Machar, Josef Svatopluk 29.02.1864 – 17.03.1942@\textsc{Machar, Josef Svatopluk} (29.02.1864 – 17.03.1942), \emph{Schriftsteller/Schriftstellerin}|pwk} vertreten.}}}\label{K_L03435-2}
               meinen Vertreter\pwindex{Harpner, Gustav 25.03.1864 – 10.07.1924@\textsc{Harpner, Gustav} (25.03.1864 – 10.07.1924), \emph{Rechtsanwalt/Rechtsanwältin}|pwv} bat, es
               möchte die Sache aussergerichtlich beigelegt werden, und D\textsuperscript{r}{ }Harpner\pwindex{Harpner, Gustav 25.03.1864 – 10.07.1924@\textsc{Harpner, Gustav} (25.03.1864 – 10.07.1924), \emph{Rechtsanwalt/Rechtsanwältin}|pw} leider, ohne mich zu fragen, in eine
               einstweilige Vertagung gewilligt hat. Ich danke Ihnen jedenfalls herzlich, für Ihre
               Bereitwilligkeit, auszusagen.\pend
           
\pstart
           Die Kinder\pwindex{Rehmann, Anna Katharina 18.08.1904 – 27.03.1977@\textsc{Rehmann, Anna Katharina} (18.08.1904 – 27.03.1977), \emph{Schauspieler/Schauspielerin, Übersetzer/Übersetzerin}|pwv}\pwindex{Salten, Paul 11.08.1903 – 08.05.1937@\textsc{Salten, Paul} (11.08.1903 – 08.05.1937), \emph{Filmcutter/Filmcutterin}|pwv} sind krank. Paul\pwindex{Salten, Paul 11.08.1903 – 08.05.1937@\textsc{Salten, Paul} (11.08.1903 – 08.05.1937), \emph{Filmcutter/Filmcutterin}|pw} hat eine starke Angina. Der \label{K_L03435-3v}\edtext{Arzt\pwindex{?? [Kinderarzt von Paul Salten] @\textsc{?? [Kinderarzt von Paul Salten]}|pwv}}{\lemma{\textnormal{\emph{Arzt}}}\Cendnote{\textnormal{nicht ermittelt}}}\label{K_L03435-3} fürchtete zuerst
               Scharlach. Vorsichtigerweise kann ich mich jetzt weder auf dem Tennisplatz noch sonst
               wo in die Nähe eines Kindesvaters wagen.\pend
           
\pstart
           Aufrichtig Ihr {\\[\baselineskip]}\spacefill\mbox{Felix Salten}\pend
           \leftskip=0em{}\selectlanguage{ngerman}\endnumbering\briefempfaengerindex{Schnitzler, Arthur@\textsc{Schnitzler, Arthur}!zzzSalten, Felix@\emph{von Felix Salten}!1906-10-201@{{[}20.? 10. 1906{]}}|)be}\mylabel{L03435h}  \normalsize

\doendnotes{C}
\bigskip
\vfill

\clearpage

\footnotesize

\lohead{\textsc{register}}

% Definiere theindex-Environment komplett neu ohne reledmac
\makeatletter
\renewenvironment{theindex}{%
  \section*{\indexname}%
  \setlength{\parindent}{0pt}%
  \setlength{\parskip}{0pt plus 0.3pt}%
  \let\item\@idxitem
}{%
  \clearpage
}
\makeatother

\IfFileExists{\jobname-pw.ind}{\input{\jobname-pw.ind}}{}

\end{document}

      