%% latex-korrekturansicht-vorspann.tex
%% Vorspann für die Korrekturansicht.
%% Lädt die gemeinsame Datei latex-vorspann.tex mit gesetztem Schalter.

\newif\ifkorrekturansicht
\korrekturansichttrue

\input{../tex-inputs/latex-vorspann}


\section[Adolf Treibl an Arthur Schnitzler, 18. 1. 1906]{L01574 Adolf Treibl an Arthur Schnitzler, 18. 1. 1906}
\nopagebreak\mylabel{L01574v}
\rehead{ }\normalsize\beginnumbering\briefempfaengerindex{Schnitzler, Arthur@\textsc{Schnitzler, Arthur}!zzzTreibl, Adolf@\emph{von Adolf Treibl}!1906-01-181@{18. 1. 1906}|(be}
\toendnotes[C]{\smallbreak\pagebreak[2]}\Standort{DLA, A:Schnitzler, HS.NZ85.1.4815,1.}
\physDesc{Brief, 1 Blatt, 4 Seiten, 1605 Zeichen
\newline{}Handschrift: schwarze Tinte, deutsche Kurrent
\newline{}Schnitzler: mit Bleistift beschriftet: »\textsc{Ehrenstein (Treibl}« }\toendnotes[C]{\smallbreak}
\pstart
           {\pb}\textsc{Euer Hochwohlgeboren}\pend
           
\pstart{}Hochverehrter Herr \textsc{Doctor}.\pend\vspace{0.5em}
\pstart
           Es iſt halt ein großes \textsc{Kreuz}! Noch einmal appellieren die
                  Eltern\pwindex{Ehrenstein, Charlotte 21.04.1867 – 02.02.1941@\textsc{Ehrenstein, Charlotte} (21.04.1867 – 02.02.1941)|pwv}\pwindex{Ehrenstein, Alexander 29.03.1857 – 29.05.1925@\textsc{Ehrenstein, Alexander} (29.03.1857 – 29.05.1925), \emph{Kassier/Kassierin}|pwv} des
               erkrankten \textsc{Albert Ehrenstein\pwindex{Ehrenstein, Albert 23.12.1886 – 08.04.1950@\textsc{Ehrenstein, Albert} (23.12.1886 – 08.04.1950), \emph{Schriftsteller/Schriftstellerin}|pw}} an die Opferwilligkeit von \textsc{Euer Hochwohlgeboren}.
               Bisher haben drei Ärzte: \textsc{D\textsuperscript{r}{ }Adler\pwindex{Adler, Alfred 07.02.1870 – 28.05.1937@\textsc{Adler, Alfred} (07.02.1870 – 28.05.1937), \emph{Psychiater/Psychiaterin, Neurologe/Neurologin}|pw}}, \textsc{der Hausarzt D\textsuperscript{r}}{ }\label{K_L01574-1v}\edtext{\textsc{Jellenik\pwindex{Jelinek, Edmund 14.05.1852 – 19.04.1928@\textsc{Jelinek, Edmund} (14.05.1852 – 19.04.1928), \emph{Mediziner/Medizinerin, Laryngologe/Laryngologin, Arzt/Ärztin}|pw}}}{\lemma{\textnormal{\emph{Jellenik}}}\Cendnote{\textnormal{Ein Arzt mit Namen »Jellenik« ist in Wien\oindex{Wien@\textbf{Wien}, \emph{A.ADM2}|pwk} nicht nachweisbar. Es dürfte sich um Edmund Jelinek\pwindex{Jelinek, Edmund 14.05.1852 – 19.04.1928@\textsc{Jelinek, Edmund} (14.05.1852 – 19.04.1928), \emph{Mediziner/Medizinerin, Laryngologe/Laryngologin, Arzt/Ärztin}|pwk} handeln (vgl. A. S.: \emph{Tagebuch}, 18. 1. 1906).}}}\label{K_L01574-1} u ein
               von Brünn\oindex{Bruenn@\textbf{Brünn}, \emph{P.PPLA}|pw} berufener Onkel des Patienten \textsc{D\textsuperscript{r}{ }Jakob Ehrenstein\pwindex{Ehrenstein, Jakob 25.9.1844 – 4.11.1917@\textsc{Ehrenstein, Jakob} (25.9.1844 – 4.11.1917), \emph{Mediziner/Medizinerin}|pw}}{ }ſich ziemlich einhellig \strikeout{über} für ein Sanatorium aus{\pb}geſprochen.
               Allerdings \strikeout{über} der Grad der Notwendigkeit dieſer
               Verfügung wurde nicht gleichmäßig betont. Der Kranke ſelbſt hält aber an einer Reiſe
               nach \textsc{Meran\oindex{Meran@\textbf{Meran}, \emph{P.PPLA3}|pw}} feſt, weil Euer Hochwohlgeboren eine ſolche ſeinerzeit empfohlen haben.\pend
           
\pstart
           Heute{ }nachmittags (18/I) treten um ¼ 5\textsuperscript{h} noch einmal der Hausarzt\pwindex{Jelinek, Edmund 14.05.1852 – 19.04.1928@\textsc{Jelinek, Edmund} (14.05.1852 – 19.04.1928), \emph{Mediziner/Medizinerin, Laryngologe/Laryngologin, Arzt/Ärztin}|pwv} und ein Spezialiſt: \textsc{D\textsuperscript{r}{ }Kornfeld\pwindex{Kornfeld, Sigmund 21.04.1859 – 15.04.1927@\textsc{Kornfeld, Sigmund} (21.04.1859 – 15.04.1927), \emph{Psychiater/Psychiaterin}|pw}} zu einem Konzilium zuſammen. Namens und im Auftrag der Eltern\pwindex{Ehrenstein, Charlotte 21.04.1867 – 02.02.1941@\textsc{Ehrenstein, Charlotte} (21.04.1867 – 02.02.1941)|pwv}\pwindex{Ehrenstein, Alexander 29.03.1857 – 29.05.1925@\textsc{Ehrenstein, Alexander} (29.03.1857 – 29.05.1925), \emph{Kassier/Kassierin}|pwv} erlaube ich
               mir nun die Bitte, Euer Hochwohlgeboren mögen die ganz beſondere Güte haben, {\pb}dieſem Konzilium beizuwohnen und den Patienten\pwindex{Ehrenstein, Albert 23.12.1886 – 08.04.1950@\textsc{Ehrenstein, Albert} (23.12.1886 – 08.04.1950), \emph{Schriftsteller/Schriftstellerin}|pwv} im Sinne der zu treffenden
               Maßnahmen beeinflußen.\pend
           
\pstart
           Euer Hochwohlgeboren können verſichert ſein wir wiſſen die Schwere der Opfer, die in
               dieser \textsc{Affaire} Euer Hochwohlgeboren bringen, wohl zu
               würdigen und es iſt nicht Selbſtſucht oder Rückſichtsloſigkeit, die uns neuerlich an
               Herrn \textsc{Doktor} mit dieſer geradezu anmaßlichen Bitte
               herantreten läßt. Wenn der Patient\pwindex{Ehrenstein, Albert 23.12.1886 – 08.04.1950@\textsc{Ehrenstein, Albert} (23.12.1886 – 08.04.1950), \emph{Schriftsteller/Schriftstellerin}|pwv} irgend welchen anderen Einflüſſen, als denen die von Euer
               Hochwohlgeboren ausgehen, zugängig wäre, hätten wir es gewiß nicht {\pb}gewagt, neuerlich zu beläſtigen.\pend
           
\pstart
           Mit der Bitte, um des leidenden Menſchen\pwindex{Ehrenstein, Albert 23.12.1886 – 08.04.1950@\textsc{Ehrenstein, Albert} (23.12.1886 – 08.04.1950), \emph{Schriftsteller/Schriftstellerin}|pwv} willen, dem ausgeſprochenen Wunſche zu willfahren verharret in
               vollkommener Hochachtung\pend
           
\pstart
           Euer Hochwohlgeboren ganz ergebſter{\\[\baselineskip]}\spacefill\mbox{Ad. Treibl}\pend
           \leftskip=0em{}
\pstart
           \noindent{}Adreſſe: \textsc{Alex Ehrenstein}\pwindex{Ehrenstein, Alexander 29.03.1857 – 29.05.1925@\textsc{Ehrenstein, Alexander} (29.03.1857 – 29.05.1925), \emph{Kassier/Kassierin}|pw}\pend
           
\pstart
           Wien XVI\oindex{XVI., Ottakring@\textbf{XVI., Ottakring}, \emph{A.ADM3}|pw}\pend
           
\pstart
           \textsc{Ottakringerstr} 114\oindex{Ottakringer Strasse@\textbf{Ottakringer Straße}, \emph{Straße (K.STR)}|pw}\pend
           
\pstart
           Wien\oindex{Wien@\textbf{Wien}, \emph{A.ADM2}|pw}, 18/I 1906\pend
           \selectlanguage{ngerman}\endnumbering\briefempfaengerindex{Schnitzler, Arthur@\textsc{Schnitzler, Arthur}!zzzTreibl, Adolf@\emph{von Adolf Treibl}!1906-01-181@{18. 1. 1906}|)be}\mylabel{L01574h}  \normalsize

\doendnotes{C}
\bigskip
\vfill

\clearpage

\footnotesize

\lohead{\textsc{register}}

% Definiere theindex-Environment komplett neu ohne reledmac
\makeatletter
\renewenvironment{theindex}{%
  \section*{\indexname}%
  \setlength{\parindent}{0pt}%
  \setlength{\parskip}{0pt plus 0.3pt}%
  \let\item\@idxitem
}{%
  \clearpage
}
\makeatother

\IfFileExists{\jobname-pw.ind}{\input{\jobname-pw.ind}}{}

\end{document}

      