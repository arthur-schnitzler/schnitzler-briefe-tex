%% latex-leseansicht-vorspann.tex
%% Vorspann für die Leseansicht.
%% Lädt die gemeinsame Datei latex-vorspann.tex mit nicht gesetztem Schalter.

\newif\ifkorrekturansicht
\korrekturansichtfalse

\input{../tex-inputs/latex-vorspann}


\section[Adolf Treibl an Arthur Schnitzler, 18. 1. 1906]{L01574 Adolf Treibl an Arthur Schnitzler, 18. 1. 1906}
\nopagebreak\mylabel{L01574v}
\rehead{ }\normalsize\beginnumbering\briefempfaengerindex{Schnitzler, Arthur@\textsc{Schnitzler, Arthur}!zzzTreibl, Adolf@\emph{von Adolf Treibl}!1906-01-181@{18. 1. 1906}|(be}
\toendnotes[C]{\smallbreak\pagebreak[2]}
\correspDesc{Versand  durch Adolf Treibl am 18. 1. 1906 in Wien
\newline{}Erhalt  durch Arthur Schnitzler im Zeitraum [18. 1. 1906
                  – 22. 1. 1906?] in Wien}\toendnotes[C]{\smallbreak}
\Standort{DLA, A:Schnitzler, HS.NZ85.1.4815,1.}
\physDesc{Brief, 1 Blatt, 4 Seiten, 1605 Zeichen
\newline{}Handschrift: schwarze Tinte, deutsche Kurrent
\newline{}Schnitzler: mit Bleistift beschriftet: »\textsc{Ehrenstein (Treibl}« }\toendnotes[C]{\smallbreak}
\pstart
           {\pb}\textsc{Euer Hochwohlgeboren}\pend
           
\pstart{}Hochverehrter Herr \textsc{Doctor}.\pend\vspace{0.5em}
\pstart
           Es iſt halt ein großes \textsc{Kreuz}! Noch einmal appellieren die
                  Eltern\pwindex{Ehrenstein, Charlotte 21.\,4.\,1867 Vrádište – 2.\,2.\,1941 New York City@\textsc{Ehrenstein, Charlotte} (21.\,4.\,1867 Vrádište – 2.\,2.\,1941 New York City)|pwv}\pwindex{Ehrenstein, Alexander 29.\,3.\,1857 Skalice – 29.\,5.\,1925 Wien@\textsc{Ehrenstein, Alexander} (29.\,3.\,1857 Skalice – 29.\,5.\,1925 Wien), \emph{Kassier}|pwv} des
               erkrankten \textsc{Albert Ehrenstein\pwindex{Ehrenstein, Albert 23.\,12.\,1886 Wien – 8.\,4.\,1950 New York City@\textsc{Ehrenstein, Albert} (23.\,12.\,1886 Wien – 8.\,4.\,1950 New York City), \emph{Schriftsteller}|pw}} an die Opferwilligkeit von \textsc{Euer Hochwohlgeboren}.
               Bisher haben drei Ärzte: \textsc{D\textsuperscript{r}{ }Adler\pwindex{Adler, Alfred 7.\,2.\,1870 Wien – 28.\,5.\,1937 Aberdeen@\textsc{Adler, Alfred} (7.\,2.\,1870 Wien – 28.\,5.\,1937 Aberdeen), \emph{Psychiater, Neurologe}|pw}}, \textsc{der Hausarzt D\textsuperscript{r}}{ }\label{K_L01574-1v}\edtext{\textsc{Jellenik\pwindex{Jelinek, Edmund 14.\,5.\,1852 Bzenec – 19.\,4.\,1928 Wien@\textsc{Jelinek, Edmund} (14.\,5.\,1852 Bzenec – 19.\,4.\,1928 Wien), \emph{Mediziner, Laryngologe, Arzt}|pw}}}{\lemma{\textnormal{\emph{Jellenik}}}\Cendnote{\textnormal{Ein Arzt mit Namen »Jellenik« ist in Wien\oindex{Wien@\textbf{Wien}, \emph{Verwaltungsgebiet}|pwk} nicht nachweisbar. Es dürfte sich um Edmund Jelinek\pwindex{Jelinek, Edmund 14.\,5.\,1852 Bzenec – 19.\,4.\,1928 Wien@\textsc{Jelinek, Edmund} (14.\,5.\,1852 Bzenec – 19.\,4.\,1928 Wien), \emph{Mediziner, Laryngologe, Arzt}|pwk} handeln (vgl. A. S.: \emph{Tagebuch}, 18. 1. 1906).}}}\label{K_L01574-1} u ein
               von Brünn\oindex{Brünn@\textbf{Brünn}|pw} berufener Onkel des Patienten \textsc{D\textsuperscript{r}{ }Jakob Ehrenstein\pwindex{Ehrenstein, Jakob 25.\,9.\,1844 Skalice – 4.\,11.\,1917 Brünn@\textsc{Ehrenstein, Jakob} (25.\,9.\,1844 Skalice – 4.\,11.\,1917 Brünn), \emph{Mediziner}|pw}}{ }ſich ziemlich einhellig \strikeout{über} für ein Sanatorium aus{\pb}geſprochen.
               Allerdings \strikeout{über} der Grad der Notwendigkeit dieſer
               Verfügung wurde nicht gleichmäßig betont. Der Kranke{ }ſelbſt hält aber an einer Reiſe
               nach \textsc{Meran\oindex{Meran@\textbf{Meran}, \emph{Hauptstadt}|pw}} feſt, weil Euer Hochwohlgeboren eine{ }ſolche{ }ſeinerzeit empfohlen haben.\pend
           
\pstart
           Heute{ }nachmittags (18/I) treten um ¼ 5\textsuperscript{h} noch einmal der Hausarzt\pwindex{Jelinek, Edmund 14.\,5.\,1852 Bzenec – 19.\,4.\,1928 Wien@\textsc{Jelinek, Edmund} (14.\,5.\,1852 Bzenec – 19.\,4.\,1928 Wien), \emph{Mediziner, Laryngologe, Arzt}|pwv} und ein Spezialiſt: \textsc{D\textsuperscript{r}{ }Kornfeld\pwindex{Kornfeld, Sigmund 21.\,4.\,1859 Golčův Jeníkov – 15.\,4.\,1927 Wien@\textsc{Kornfeld, Sigmund} (21.\,4.\,1859 Golčův Jeníkov – 15.\,4.\,1927 Wien), \emph{Psychiater}|pw}} zu einem Konzilium zuſammen. Namens und im Auftrag der Eltern\pwindex{Ehrenstein, Charlotte 21.\,4.\,1867 Vrádište – 2.\,2.\,1941 New York City@\textsc{Ehrenstein, Charlotte} (21.\,4.\,1867 Vrádište – 2.\,2.\,1941 New York City)|pwv}\pwindex{Ehrenstein, Alexander 29.\,3.\,1857 Skalice – 29.\,5.\,1925 Wien@\textsc{Ehrenstein, Alexander} (29.\,3.\,1857 Skalice – 29.\,5.\,1925 Wien), \emph{Kassier}|pwv} erlaube ich
               mir nun die Bitte, Euer Hochwohlgeboren mögen die ganz beſondere Güte haben, {\pb}dieſem Konzilium beizuwohnen und den Patienten\pwindex{Ehrenstein, Albert 23.\,12.\,1886 Wien – 8.\,4.\,1950 New York City@\textsc{Ehrenstein, Albert} (23.\,12.\,1886 Wien – 8.\,4.\,1950 New York City), \emph{Schriftsteller}|pwv} im Sinne der zu treffenden
               Maßnahmen beeinflußen.\pend
           
\pstart
           Euer Hochwohlgeboren können verſichert{ }ſein wir wiſſen die Schwere der Opfer, die in
               dieser \textsc{Affaire} Euer Hochwohlgeboren bringen, wohl zu
               würdigen und es iſt nicht Selbſtſucht oder Rückſichtsloſigkeit, die uns neuerlich an
               Herrn \textsc{Doktor} mit dieſer geradezu anmaßlichen Bitte
               herantreten läßt. Wenn der Patient\pwindex{Ehrenstein, Albert 23.\,12.\,1886 Wien – 8.\,4.\,1950 New York City@\textsc{Ehrenstein, Albert} (23.\,12.\,1886 Wien – 8.\,4.\,1950 New York City), \emph{Schriftsteller}|pwv} irgend welchen anderen Einflüſſen, als denen die von Euer
               Hochwohlgeboren ausgehen, zugängig wäre, hätten wir es gewiß nicht {\pb}gewagt, neuerlich zu beläſtigen.\pend
           
\pstart
           Mit der Bitte, um des leidenden Menſchen\pwindex{Ehrenstein, Albert 23.\,12.\,1886 Wien – 8.\,4.\,1950 New York City@\textsc{Ehrenstein, Albert} (23.\,12.\,1886 Wien – 8.\,4.\,1950 New York City), \emph{Schriftsteller}|pwv} willen, dem ausgeſprochenen Wunſche zu willfahren verharret in
               vollkommener Hochachtung\pend
           
\pstart
           Euer Hochwohlgeboren ganz ergebſter{\\[\baselineskip]}\spacefill\mbox{Ad. Treibl}\pend
           \leftskip=0em{}
\pstart
           \noindent{}Adreſſe: \textsc{Alex Ehrenstein}\pwindex{Ehrenstein, Alexander 29.\,3.\,1857 Skalice – 29.\,5.\,1925 Wien@\textsc{Ehrenstein, Alexander} (29.\,3.\,1857 Skalice – 29.\,5.\,1925 Wien), \emph{Kassier}|pw}\pend
           
\pstart
           Wien XVI\oindex{XVI., Ottakring@\textbf{XVI., Ottakring}, \emph{Verwaltungsgebiet}|pw}\pend
           
\pstart
           \textsc{Ottakringerstr} 114\oindex{Wien@\textbf{Wien}!XVI., Ottakring@\textbf{XVI., Ottakring}!Ottakringer Straße@\textbf{Ottakringer Straße}, \emph{Straße}|pw}\oindex{Wien@\textbf{Wien}!XVII., Hernals@\textbf{XVII., Hernals}!Ottakringer Straße@\textbf{Ottakringer Straße}, \emph{Straße}|pw}\pend
           
\pstart
           Wien\oindex{Wien@\textbf{Wien}, \emph{Verwaltungsgebiet}|pw}, 18/I 1906\pend
           \selectlanguage{ngerman}\endnumbering\briefempfaengerindex{Schnitzler, Arthur@\textsc{Schnitzler, Arthur}!zzzTreibl, Adolf@\emph{von Adolf Treibl}!1906-01-181@{18. 1. 1906}|)be}\mylabel{L01574h}  \newcommand{\dateiname}{L01574}\newcommand{\titel}{Adolf Treibl an Arthur Schnitzler, 18. 1. 1906}\newcommand{\editorInnen}{Martin Anton Müller und Gerd-Hermann Susen}%% latex-leseansicht-abspann.tex
%% Abspann für die Leseansicht.
%% Der Schalter \ifkorrekturansicht ist bereits durch den Vorspann gesetzt.

%% latex-abspann.tex
%% Gemeinsamer Abspann für Korrekturansicht und Leseansicht.
%% Setzt den Schalter \ifkorrekturansicht voraus (gesetzt in den
%% einbindenden Dateien latex-korrekturansicht-abspann.tex bzw.
%% latex-leseansicht-abspann.tex).
%% ---------------------------------------------------------------

\normalsize

% Das esempio-Environment wird nur in der Leseansicht benötigt
\ifkorrekturansicht\else
\newenvironment{esempio}[3]%
{
    \vspace{1.5ex}
    \rlap{\underline{#1}}
    \par
    \setlength{\parindent}{0cm}
    \nopagebreak
    \leftskip=#2cm
    \rightskip=#3cm
}
{
    \par
}
\fi

\doendnotes{C}
\bigskip
\vfill

\clearpage

\footnotesize

\ifkorrekturansicht
  \lohead{\textsc{register}}
\fi

% theindex-Environment neu definieren ohne reledmac
\makeatletter
\renewenvironment{theindex}{%
  \ifkorrekturansicht
    \section*{\indexname}%
  \else
    \subsubsection*{Index der erwähnten Entitäten}%
  \fi
  \setlength{\parindent}{0pt}%
  \setlength{\parskip}{0pt plus 0.3pt}%
  \let\item\@idxitem
}{%
  \ifkorrekturansicht\clearpage\fi
}
\makeatother

\IfFileExists{\jobname-pw.ind}{\input{\jobname-pw.ind}}{}

% Quellenangabe nur in der Leseansicht
\ifkorrekturansicht\else
% Fallback-Definitionen, falls die .tex-Datei \titel etc. nicht gesetzt hat
\providecommand{\titel}{}
\providecommand{\editorInnen}{}
\providecommand{\dateiname}{\jobname}

\vspace{3cm}

\vfill

\footnotesize
\textsc{Quelle}: \titel. Herausgegeben von {\editorInnen}. In: \emph{Arthur Schnitzler: Briefwechsel mit Autorinnen und Autoren}.
 Digitale Edition, https://schnitzler-briefe.acdh.oeaw.ac.at/{\dateiname}.html (Stand \today)
\fi

\end{document}


