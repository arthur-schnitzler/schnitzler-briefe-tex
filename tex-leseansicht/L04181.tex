%% latex-leseansicht-vorspann.tex
%% Vorspann für die Leseansicht.
%% Lädt die gemeinsame Datei latex-vorspann.tex mit nicht gesetztem Schalter.

\newif\ifkorrekturansicht
\korrekturansichtfalse

\input{../tex-inputs/latex-vorspann}


\section[Arthur Schnitzler an Gustav Schwarzkopf, {{[}}15. 1. 1902?{{]}}]{L04181 Arthur Schnitzler an Gustav Schwarzkopf, {[}15. 1. 1902?{]}}
\nopagebreak\mylabel{L04181v}
\rehead{ }\normalsize\beginnumbering\briefempfaengerindex{Schwarzkopf, Gustav@\textsc{Schwarzkopf, Gustav}!zzzSchnitzler, Arthur@\emph{von Arthur Schnitzler}!1902-01-151@{{[}15. 1. 1902?{]}}|(be}
\toendnotes[C]{\smallbreak\pagebreak[2]}
\correspDesc{Versand  durch Arthur Schnitzler am [15. 1. 1902?] in Wien
\newline{}Erhalt  durch Gustav Schwarzkopf am [15. 1. 1902?] in Wien}\toendnotes[C]{\smallbreak}
\Standort{CUL, Schnitzler, B 96.}
\physDesc{Postkarte, 153 Zeichen
\newline{}Handschrift: Bleistift, deutsche Kurrent
\newline{}Versand: 1) Rohrpost  2) Stempel: »\nobreak{}\oindex{IX., Alsergrund@\textbf{IX., Alsergrund}, \emph{Verwaltungsgebiet}|pwk}Wien 9/\textcolor{gray}{\textsubscript{1}} 66, 8\textsuperscript{30}N\nobreak{}«.  3) Stempel: »\nobreak{}15 \textcolor{gray}{I 02}, 8\textsuperscript{30}N\nobreak{}«. }\toendnotes[C]{\smallbreak}\pstart{}{\pb}Herrn Guſtav
                  Schwarzkopf\pend{}\pstart{}Wien\oindex{Wien@\textbf{Wien}, \emph{Verwaltungsgebiet}|pw}\pend{}\pstart{}I. Tiefer Graben 23\oindex{Wien@\textbf{Wien}!I., Innere Stadt@\textbf{I., Innere Stadt}!Tiefer Graben 23@\textbf{Tiefer Graben 23}, \emph{Wohngebäude}|pw}.\pend{}{\bigskip}\vspace{1em}
\pstart
           \noindent{}{\pb}Ich \label{K_L04181-1v}\edtext{fahre 10 Uhr{ }Südbahn\oindex{Wien@\textbf{Wien}!X., Favoriten@\textbf{X., Favoriten}!Südbahnhof@\textbf{Südbahnhof}, \emph{Bahnhofsgebäude}|pw} nach Mödling\oindex{Mödling@\textbf{Mödling}, \emph{Hauptstadt}|pw}}{\lemma{\textnormal{\emph{fahre … Mödling}}}\Cendnote{\textnormal{Es
                  handelte sich um eine Verabredung für den Folgetag. Schwarzkopf\pwindex{Schwarzkopf, Gustav 7.\,11.\,1853 Wien – 13.\,11.\,1939 ebd.@\textsc{Schwarzkopf, Gustav} (7.\,11.\,1853 Wien – 13.\,11.\,1939 ebd.), \emph{Schriftsteller}|pwk} fuhr nicht mit, siehe XXXX Auszeichnungsfehler: Dokument L04179 nicht gefunden.}}}\label{K_L04181-1} ab, erwarte Sie bis 5 Minuten
               vorher unten in der Halle\pend
           
\pstart
           Herzlichſt Ihr {\\[\baselineskip]}\spacefill\mbox{A.}\pend
           \leftskip=0em{}\selectlanguage{ngerman}\endnumbering\briefempfaengerindex{Schwarzkopf, Gustav@\textsc{Schwarzkopf, Gustav}!zzzSchnitzler, Arthur@\emph{von Arthur Schnitzler}!1902-01-151@{{[}15. 1. 1902?{]}}|)be}\mylabel{L04181h}
\begin{anhang}
\end{anhang}\newcommand{\dateiname}{L04181}\newcommand{\titel}{Arthur Schnitzler an Gustav Schwarzkopf, [15. 1. 1902?]}\newcommand{\editorInnen}{Herausgegeben von Jahnke, SelmaMüller, Martin Anton}%% latex-leseansicht-abspann.tex
%% Abspann für die Leseansicht.
%% Der Schalter \ifkorrekturansicht ist bereits durch den Vorspann gesetzt.

%% latex-abspann.tex
%% Gemeinsamer Abspann für Korrekturansicht und Leseansicht.
%% Setzt den Schalter \ifkorrekturansicht voraus (gesetzt in den
%% einbindenden Dateien latex-korrekturansicht-abspann.tex bzw.
%% latex-leseansicht-abspann.tex).
%% ---------------------------------------------------------------

\normalsize

% Das esempio-Environment wird nur in der Leseansicht benötigt
\ifkorrekturansicht\else
\newenvironment{esempio}[3]%
{
    \vspace{1.5ex}
    \rlap{\underline{#1}}
    \par
    \setlength{\parindent}{0cm}
    \nopagebreak
    \leftskip=#2cm
    \rightskip=#3cm
}
{
    \par
}
\fi

\doendnotes{C}
\bigskip
\vfill

\clearpage

\footnotesize

\ifkorrekturansicht
  \lohead{\textsc{register}}
\fi

% theindex-Environment neu definieren ohne reledmac
\makeatletter
\renewenvironment{theindex}{%
  \ifkorrekturansicht
    \section*{\indexname}%
  \else
    \subsubsection*{Index der erwähnten Entitäten}%
  \fi
  \setlength{\parindent}{0pt}%
  \setlength{\parskip}{0pt plus 0.3pt}%
  \let\item\@idxitem
}{%
  \ifkorrekturansicht\clearpage\fi
}
\makeatother

\IfFileExists{\jobname-pw.ind}{\input{\jobname-pw.ind}}{}

% Quellenangabe nur in der Leseansicht
\ifkorrekturansicht\else
% Fallback-Definitionen, falls die .tex-Datei \titel etc. nicht gesetzt hat
\providecommand{\titel}{}
\providecommand{\editorInnen}{}
\providecommand{\dateiname}{\jobname}

\vspace{3cm}

\vfill

\footnotesize
\textsc{Quelle}: \titel. Herausgegeben von {\editorInnen}. In: \emph{Arthur Schnitzler: Briefwechsel mit Autorinnen und Autoren}.
 Digitale Edition, https://schnitzler-briefe.acdh.oeaw.ac.at/{\dateiname}.html (Stand \today)
\fi

\end{document}


