%% latex-leseansicht-vorspann.tex
%% Vorspann für die Leseansicht.
%% Lädt die gemeinsame Datei latex-vorspann.tex mit nicht gesetztem Schalter.

\newif\ifkorrekturansicht
\korrekturansichtfalse

\input{../tex-inputs/latex-vorspann}


\section[Theodor Herzl an Arthur Schnitzler, 9. 12. 1898]{L03867 Theodor Herzl an Arthur Schnitzler, 9. 12. 1898}
\nopagebreak\mylabel{L03867v}
\rehead{ }\normalsize\beginnumbering\briefempfaengerindex{Schnitzler, Arthur@\textsc{Schnitzler, Arthur}!zzzHerzl, Theodor@\emph{von Theodor Herzl}!1898-12-091@{9. 12. 1898}|(be}
\toendnotes[C]{\smallbreak\pagebreak[2]}
\correspDesc{Versand  durch Theodor Herzl am 9. 12. 1898 in Wien
\newline{}Erhalt  durch Arthur Schnitzler im Zeitraum [9. 12. 1898
                  – 12. 12. 1898?] in Wien}\toendnotes[C]{\smallbreak}
\Standort{CUL, Schnitzler, B 39.}
\physDesc{Brief, 1 Blatt, 1 Seite, 577 Zeichen
\newline{}Handschrift: schwarze Tinte, lateinische Kurrent
\newline{}Ordnung: mit Bleistift von unbekannter Hand nummeriert: »46« }
\buchAbdrucke{\weitereDrucke{Theodor Herzl: \emph{Briefe Anfang Dezember 1898 – Mitte August 1900}. Bearbeitet von Barbara Schäfer in Zusammenarbeit mit Sofia Gelmann, Chaya Harel und Ines Rubin. Berlin, Frankfurt am Main, Wien: \emph{Propyläen} 1991, S. 31 (Briefe und Tagebücher. Herausgegeben von Alex Bein, Hermann Greive, Moshe Schaerf, Julius H. Schoeps und Johannes Wachten, 5).} }\toendnotes[C]{\smallbreak}
\pstart
           {\pb}\textcolor{gray}{\textbf{NEUE FREIE PRESSE\orgindex{Neue Freie Presse@Neue Freie Presse|pw}. }}\hfill 9 XII 98\pend
           
\pstart
           \textcolor{gray}{\textbf{\textsc{Redaction}:}}\pend
           
\pstart
           \textcolor{gray}{\textbf{WIEN\oindex{Wien@\textbf{Wien}, \emph{Verwaltungsgebiet}|pw}}}\pend
           
\pstart
           \textcolor{gray}{\textbf{Kolowratring, Fichtegasse Nr. 11\oindex{Wien@\textbf{Wien}!I., Innere Stadt@\textbf{I., Innere Stadt}!Fichtegasse 11@\textbf{Fichtegasse 11}, \emph{Gebäude}|pw}.}}\pend
           
\pstart{}Mein lieber Schnitzler,\pend\vspace{0.5em}
\pstart
           wir fehlen Ihrem Ruhme nicht mehr, Sie fehlen noch dem unseren. Wollen Sie eine
               kleine Geschichte oder ein Feuilleton – etwa 300 Zeilen – für die N. Fr. Presse\pwindex{Neue Freie Presse@\emph{Neue Freie Presse}|pw} schreiben? Wenn ich das Mscpt rechtzeitig
               bekäme, würde ich es noch \label{K_L03867-1v}\edtext{in der Weihnachtsnummer\pwindex{Neue Freie Presse@\emph{Neue Freie Presse}|pwv}}{\lemma{\textnormal{\emph{in der Weihnachtsnummer}}}\Cendnote{\textnormal{Im Jahr 1898 erschien kein
                  Text von Schnitzler in der Weihnachtsausgabe\pwindex{Neue Freie Presse@\emph{Neue Freie Presse}|pwkv}.}}}\label{K_L03867-1} unterbringen.
               Mit den schönsten Grüssen\pend
           
\pstart
           Ihr nicht so \label{K_L03867-2v}\edtext{grimmiger Feind}{\lemma{\textnormal{\emph{grimmiger Feind}}}\Cendnote{\textnormal{Das sich abkühlenden Verhältnis zwischen
                     Schnitzler und Herzl\pwindex{Herzl, Theodor 2.\,5.\,1860 Budapest – 3.\,7.\,1904 Edlach@\textsc{Herzl, Theodor} (2.\,5.\,1860 Budapest – 3.\,7.\,1904 Edlach), \emph{Schriftsteller, Journalist}|pwk} und die zunehmenden Spannungen lassen sich in Schnitzlers{ }\emph{Tagebuch}\pwindex{Schnitzler, Arthur 15.\,5.\,1862 Wien – 21.\,10.\,1931 ebd.@\textsc{Schnitzler, Arthur} (15.\,5.\,1862 Wien – 21.\,10.\,1931 ebd.), \emph{Schriftsteller, Mediziner}!Tagebuch@\strich\emph{Tagebuch}|pwk} nachverfolgen. Am 11. 3. 1896 heißt es
                  dort: »Trotz aller Mühe die wir uns geben, sind wir einander persönlich
                     nicht sympathisch«, am 7. 3. 1897: »Ein Feuilleton von Herzl\pwindex{Herzl, Theodor 2.\,5.\,1860 Budapest – 3.\,7.\,1904 Edlach@\textsc{Herzl, Theodor} (2.\,5.\,1860 Budapest – 3.\,7.\,1904 Edlach), \emph{Schriftsteller, Journalist}|pw} über Tschaperl\pwindex{Bahr, Hermann 19.\,7.\,1863 Linz – 15.\,1.\,1934 München@\textsc{Bahr, Hermann} (19.\,7.\,1863 Linz – 15.\,1.\,1934 München), \emph{Schriftsteller, Kritiker}!Tschaperl. Ein Wiener Stück in vier Aufzügen@\strich\emph{Das Tschaperl. Ein Wiener Stück in vier Aufzügen}|pw}, ›Jung
                        Oesterreich‹\pwindex{Herzl, Theodor 2.\,5.\,1860 Budapest – 3.\,7.\,1904 Edlach@\textsc{Herzl, Theodor} (2.\,5.\,1860 Budapest – 3.\,7.\,1904 Edlach), \emph{Schriftsteller, Journalist}!Jung-Oesterreich«@\strich\emph{»Jung-Oesterreich«}|pw} genannt, ärgerte mich wegen des frechen und beinah
                     perfiden Tones; dabei fest überzeugt, er hätte anders geschrieben, wenn ich
                     seine Eitelkeit nicht verletzt« und am 13. 2. 1898: »Recht perfides und
                     hochmütiges Feuilleton\pwindex{Herzl, Theodor 2.\,5.\,1860 Budapest – 3.\,7.\,1904 Edlach@\textsc{Herzl, Theodor} (2.\,5.\,1860 Budapest – 3.\,7.\,1904 Edlach), \emph{Schriftsteller, Journalist}!Feuilleton. Carl-Theater. (»Freiwild«, Schauspiel von Arthur Schnitzler.)@\strich\emph{Feuilleton. Carl-Theater. (»Freiwild«, Schauspiel von Arthur Schnitzler.)}|pwv}
                     von Herzl\pwindex{Herzl, Theodor 2.\,5.\,1860 Budapest – 3.\,7.\,1904 Edlach@\textsc{Herzl, Theodor} (2.\,5.\,1860 Budapest – 3.\,7.\,1904 Edlach), \emph{Schriftsteller, Journalist}|pw} in der N. Fr. Pr.\pwindex{Neue Freie Presse@\emph{Neue Freie Presse}|pw} über Freiwild\pwindex{Schnitzler, Arthur 15.\,5.\,1862 Wien – 21.\,10.\,1931 ebd.@\textsc{Schnitzler, Arthur} (15.\,5.\,1862 Wien – 21.\,10.\,1931 ebd.), \emph{Schriftsteller, Mediziner}!Freiwild. Schauspiel in 3 Akten@\strich\emph{Freiwild. Schauspiel in 3 Akten}|pw}«. Zu der ambivalenten Beziehung der beiden siehe auch Schnitzlers{ }Aufzeichnungen\pwindex{Schnitzler, Arthur 15.\,5.\,1862 Wien – 21.\,10.\,1931 ebd.@\textsc{Schnitzler, Arthur} (15.\,5.\,1862 Wien – 21.\,10.\,1931 ebd.), \emph{Schriftsteller, Mediziner}!Theodor Herzl@\strich\emph{Theodor Herzl}|pwkv} über Herzl\pwindex{Herzl, Theodor 2.\,5.\,1860 Budapest – 3.\,7.\,1904 Edlach@\textsc{Herzl, Theodor} (2.\,5.\,1860 Budapest – 3.\,7.\,1904 Edlach), \emph{Schriftsteller, Journalist}|pwk} im Deutschen Literaturarchiv Marbach (Arthur Schnitzler: \emph{Unveröffentliche autobiografische Aufzeichnungen}\pwindex{Schnitzler, Arthur 15.\,5.\,1862 Wien – 21.\,10.\,1931 ebd.@\textsc{Schnitzler, Arthur} (15.\,5.\,1862 Wien – 21.\,10.\,1931 ebd.), \emph{Schriftsteller, Mediziner}!Theodor Herzl@\strich\emph{Theodor Herzl}|pwk}, \emph{Deutsches Literaturarchiv Marbach},  HS.1985.1.198 ).}}}\label{K_L03867-2}, wie Sie glauben{\\[\baselineskip]}\spacefill\mbox{Th Herzl}\pend
           \leftskip=0em{}
\pstart
           \noindent{}Ueber das »Vermächtniss\pwindex{Schnitzler, Arthur 15.\,5.\,1862 Wien – 21.\,10.\,1931 ebd.@\textsc{Schnitzler, Arthur} (15.\,5.\,1862 Wien – 21.\,10.\,1931 ebd.), \emph{Schriftsteller, Mediziner}!Vermächtnis. Schauspiel in drei Akten@\strich\emph{Das Vermächtnis. Schauspiel in drei Akten}|pw}« hätte ich Ihnen
                  Manches zu sagen. Wenn Sie dagegen keine unüberwindliche Abneigung haben, \label{K_L03867-3v}\edtext{kommen Sie doch einmal}{\lemma{\textnormal{\emph{kommen Sie doch einmal}}}\Cendnote{\textnormal{Schnitzler folgte der Einladung gleich am 11. 12. 1898, traf
                     aber nur Herzls\pwindex{Herzl, Theodor 2.\,5.\,1860 Budapest – 3.\,7.\,1904 Edlach@\textsc{Herzl, Theodor} (2.\,5.\,1860 Budapest – 3.\,7.\,1904 Edlach), \emph{Schriftsteller, Journalist}|pwk}{ }Frau Julie\pwindex{Herzl, Julie 1.\,2.\,1868 Budapest – 10.\,11.\,1907 Wien@\textsc{Herzl, Julie} (1.\,2.\,1868 Budapest – 10.\,11.\,1907 Wien)|pwkv} an: »Vorm., nach ›liebenswürdigem‹ Brief Herzls\pwindex{Herzl, Theodor 2.\,5.\,1860 Budapest – 3.\,7.\,1904 Edlach@\textsc{Herzl, Theodor} (2.\,5.\,1860 Budapest – 3.\,7.\,1904 Edlach), \emph{Schriftsteller, Journalist}|pw} bei Frau H.\pwindex{Herzl, Julie 1.\,2.\,1868 Budapest – 10.\,11.\,1907 Wien@\textsc{Herzl, Julie} (1.\,2.\,1868 Budapest – 10.\,11.\,1907 Wien)|pwv}« Am 12. 12. 1898 suchte
                        Herzl\pwindex{Herzl, Theodor 2.\,5.\,1860 Budapest – 3.\,7.\,1904 Edlach@\textsc{Herzl, Theodor} (2.\,5.\,1860 Budapest – 3.\,7.\,1904 Edlach), \emph{Schriftsteller, Journalist}|pwk}{ }Schnitzler auf und es kam zur
                     Aussprache und zulgleich weiterem Befremden: »Herzl\pwindex{Herzl, Theodor 2.\,5.\,1860 Budapest – 3.\,7.\,1904 Edlach@\textsc{Herzl, Theodor} (2.\,5.\,1860 Budapest – 3.\,7.\,1904 Edlach), \emph{Schriftsteller, Journalist}|pw} bei mir, mit ihm bei seinem Schwager, dem
                        Zionsgigerl Paul
                        N\pwindex{Naschauer, Paul 6.\,9.\,1866 Baden bei Wien – 20.\,5.\,1900 Wien@\textsc{Naschauer, Paul} (6.\,9.\,1866 Baden bei Wien – 20.\,5.\,1900 Wien)|pwv}.― H.\pwindex{Herzl, Theodor 2.\,5.\,1860 Budapest – 3.\,7.\,1904 Edlach@\textsc{Herzl, Theodor} (2.\,5.\,1860 Budapest – 3.\,7.\,1904 Edlach), \emph{Schriftsteller, Journalist}|pwv} gab zu,
                        über mich ›aegrirt‹ gewesen zu sein – u. zw. angeblich, weil er mich für
                        einen gehalten, der sich da und dort, z. B. ― Brahm\pwindex{Brahm, Otto 5.\,2.\,1856 Hamburg – 28.\,11.\,1912 Berlin@\textsc{Brahm, Otto} (5.\,2.\,1856 Hamburg – 28.\,11.\,1912 Berlin), \emph{Theaterleiter, Regisseur}|pw} – und Bahr\pwindex{Bahr, Hermann 19.\,7.\,1863 Linz – 15.\,1.\,1934 München@\textsc{Bahr, Hermann} (19.\,7.\,1863 Linz – 15.\,1.\,1934 München), \emph{Schriftsteller, Kritiker}|pw} – verhielte. – Merkwürdige
                        Menschenunkenntnis.«}}}\label{K_L03867-3} in meine Cottagehöhle Carl Ludwigstr 50\oindex{Wien@\textbf{Wien}!XVIII., Währing@\textbf{XVIII., Währing}!Weimarer Straße 50@\textbf{Weimarer Straße 50}, \emph{Wohngebäude}|pw} u. hören sich meine Dummheiten an.\pend
           \selectlanguage{ngerman}\endnumbering\briefempfaengerindex{Schnitzler, Arthur@\textsc{Schnitzler, Arthur}!zzzHerzl, Theodor@\emph{von Theodor Herzl}!1898-12-091@{9. 12. 1898}|)be}\mylabel{L03867h}
\begin{anhang}
\end{anhang}\newcommand{\dateiname}{L03867}\newcommand{\titel}{Theodor Herzl an Arthur Schnitzler, 9. 12. 1898}\newcommand{\editorInnen}{Selma Jahnke und Martin Anton Müller}%% latex-leseansicht-abspann.tex
%% Abspann für die Leseansicht.
%% Der Schalter \ifkorrekturansicht ist bereits durch den Vorspann gesetzt.

%% latex-abspann.tex
%% Gemeinsamer Abspann für Korrekturansicht und Leseansicht.
%% Setzt den Schalter \ifkorrekturansicht voraus (gesetzt in den
%% einbindenden Dateien latex-korrekturansicht-abspann.tex bzw.
%% latex-leseansicht-abspann.tex).
%% ---------------------------------------------------------------

\normalsize

% Das esempio-Environment wird nur in der Leseansicht benötigt
\ifkorrekturansicht\else
\newenvironment{esempio}[3]%
{
    \vspace{1.5ex}
    \rlap{\underline{#1}}
    \par
    \setlength{\parindent}{0cm}
    \nopagebreak
    \leftskip=#2cm
    \rightskip=#3cm
}
{
    \par
}
\fi

\doendnotes{C}
\bigskip
\vfill

\clearpage

\footnotesize

\ifkorrekturansicht
  \lohead{\textsc{register}}
\fi

% theindex-Environment neu definieren ohne reledmac
\makeatletter
\renewenvironment{theindex}{%
  \ifkorrekturansicht
    \section*{\indexname}%
  \else
    \subsubsection*{Index der erwähnten Entitäten}%
  \fi
  \setlength{\parindent}{0pt}%
  \setlength{\parskip}{0pt plus 0.3pt}%
  \let\item\@idxitem
}{%
  \ifkorrekturansicht\clearpage\fi
}
\makeatother

\IfFileExists{\jobname-pw.ind}{\input{\jobname-pw.ind}}{}

% Quellenangabe nur in der Leseansicht
\ifkorrekturansicht\else
% Fallback-Definitionen, falls die .tex-Datei \titel etc. nicht gesetzt hat
\providecommand{\titel}{}
\providecommand{\editorInnen}{}
\providecommand{\dateiname}{\jobname}

\vspace{3cm}

\vfill

\footnotesize
\textsc{Quelle}: \titel. Herausgegeben von {\editorInnen}. In: \emph{Arthur Schnitzler: Briefwechsel mit Autorinnen und Autoren}.
 Digitale Edition, https://schnitzler-briefe.acdh.oeaw.ac.at/{\dateiname}.html (Stand \today)
\fi

\end{document}


