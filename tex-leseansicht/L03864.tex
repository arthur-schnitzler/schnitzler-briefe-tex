%% latex-leseansicht-vorspann.tex
%% Vorspann für die Leseansicht.
%% Lädt die gemeinsame Datei latex-vorspann.tex mit nicht gesetztem Schalter.

\newif\ifkorrekturansicht
\korrekturansichtfalse

\input{../tex-inputs/latex-vorspann}


\section[Theodor Herzl an Arthur Schnitzler, 25. 8. 1895]{L03864 Theodor Herzl an Arthur Schnitzler, 25. 8. 1895}
\nopagebreak\mylabel{L03864v}
\rehead{ }\normalsize\beginnumbering\briefempfaengerindex{Schnitzler, Arthur@\textsc{Schnitzler, Arthur}!zzzHerzl, Theodor@\emph{von Theodor Herzl}!1895-08-251@{25. 8. 1895}|(be}
\toendnotes[C]{\smallbreak\pagebreak[2]}
\correspDesc{Versand  durch Theodor Herzl am 25. 8. 1895 in Bad Aussee
\newline{}Umleitung  am 1895-08-26 in Bad Ischl
\newline{}Erhalt  durch Arthur Schnitzler im Zeitraum [27. 8. 1895
                  – 7. 9. 1895?] \textbf{Ort fehlend} }\toendnotes[C]{\smallbreak}
\Standort{CUL, Schnitzler, B 39.}
\physDesc{Kartenbrief, 458 Zeichen
\newline{}Handschrift: blaue Tinte, lateinische Kurrent
\newline{}Versand: 1) Stempel: »\nobreak{}\oindex{Bad Aussee@\textbf{Bad Aussee}, \emph{Hauptstadt}|pwk}Aussee in Steiermark, 26/8 95\nobreak{}«.   2) Stempel: »\nobreak{}26/8 95, 1 – N\nobreak{}«. 
\newline{}Ordnung: mit Bleistift von unbekannter Hand nummeriert: »44« }
\buchAbdrucke{\weitereDrucke{Theodor Herzl: \emph{Briefe Anfang Mai 1895 – Anfang Dezember 1898}. Bearbeitet von Barbara Schäfer in Zusammenarbeit mit Sofia Gelmann, Chaya Harel, Ines Rubin und Daisy Ticho. Berlin, Frankfurt am Main, Wien: \emph{Propyläen} 1990, S. 63 (Briefe und Tagebücher. Herausgegeben von Alex Bein, Hermann Greive, Moshe Schaerf, Julius H. Schoeps und Johannes Wachten, 4).} }\toendnotes[C]{\smallbreak}\pstart{}{\pb}\textcolor{gray}{\textbf{An}}\pend{}\pstart{}Herrn D\textsuperscript{r} Arthur Schnitzler\pend{}\pstart{}\textcolor{gray}{\textbf{in}}{ }Ischl\oindex{Bad Ischl@\textbf{Bad Ischl}|pw}\pend{}\pstart{}Rudolfshöhe\oindex{Hotel und Pension Rudolfshöhe (Leopold Petter)@\textbf{Hotel und Pension Rudolfshöhe (Leopold Petter)}, \emph{Hotel}|pw}\pend{}{\bigskip}\vspace{1em}
\pstart
           \raggedleft{}{\pb}Aussee\oindex{Bad Aussee@\textbf{Bad Aussee}, \emph{Hauptstadt}|pw}{ }25. VIII. 95\pend
           
\pstart{}Lieber Freund,\pend\vspace{0.5em}
\pstart
           am 20 reiste ich von Salzburg\oindex{Salzburg@\textbf{Salzburg}, \emph{Verwaltungsgebiet}|pw} kommend
               durch Ischl\oindex{Bad Ischl@\textbf{Bad Ischl}|pw}.\pend
           
\pstart
           Die \label{K_L03864-1v}\edtext{Hebe\pwindex{?? [Dienstbotin Pension Rudolfshöhe, 1895] @\textsc{?? [Dienstbotin Pension Rudolfshöhe, 1895]}|pwv}}{\lemma{\textnormal{\emph{Hebe}}}\Cendnote{\textnormal{antike Göttin der Jugend}}}\label{K_L03864-1} von der
                  Rudolfshöhe\oindex{Hotel und Pension Rudolfshöhe (Leopold Petter)@\textbf{Hotel und Pension Rudolfshöhe (Leopold Petter)}, \emph{Hotel}|pw} sagte mir, \label{K_L03864-2v}\edtext{Sie seien verreist}{\lemma{\textnormal{\emph{Sie seien verreist}}}\Cendnote{\textnormal{Schnitzler hatte am 19. 8. 1895{ }Ischl\oindex{Bad Ischl@\textbf{Bad Ischl}|pwk} Richtung Salzburg\oindex{Salzburg@\textbf{Salzburg}, \emph{Verwaltungsgebiet}|pwk} und Umgebung und später München\oindex{München@\textbf{München}|pwk} verlassen und kehrte erst am 7. 9. 1895 nach Wien\oindex{Wien@\textbf{Wien}, \emph{Verwaltungsgebiet}|pwk} zurück. Wohin ihm der Kartenbrief nachgesandt wurde und wann er ihn
                 erreichte, ist nicht bekannt.}}}\label{K_L03864-2}, und als meine \introOben{}dadurch\introOben{} erschütterte Laune noch den Stoss erlitt, dass im Hotel Elisabeth\oindex{Hôtel Kaiserin Elisabeth@\textbf{Hôtel Kaiserin Elisabeth}, \emph{Hotel}|pw} das bestellte Zimmer nicht bereit war, fuhr
               ich mit dem nächsten Zug nach Aussee\oindex{Bad Aussee@\textbf{Bad Aussee}, \emph{Hauptstadt}|pw}.\pend
           
\pstart
           Sehe ich Sie noch einmal hier? Es wäre mir eine Freude.\pend
           
\pstart
           Am 1 Sept. reise ich nach Wien\oindex{Wien@\textbf{Wien}, \emph{Verwaltungsgebiet}|pw} um
               dort zu bleiben.\pend
           
\pstart
           Herzlich Ihr{\\[\baselineskip]}\spacefill\mbox{Th. H.}\pend
           \leftskip=0em{}\selectlanguage{ngerman}\endnumbering\briefempfaengerindex{Schnitzler, Arthur@\textsc{Schnitzler, Arthur}!zzzHerzl, Theodor@\emph{von Theodor Herzl}!1895-08-251@{25. 8. 1895}|)be}\mylabel{L03864h}
\begin{anhang}
\end{anhang}\newcommand{\dateiname}{L03864}\newcommand{\titel}{Theodor Herzl an Arthur Schnitzler, 25. 8. 1895}\newcommand{\editorInnen}{Selma Jahnke und Martin Anton Müller}%% latex-leseansicht-abspann.tex
%% Abspann für die Leseansicht.
%% Der Schalter \ifkorrekturansicht ist bereits durch den Vorspann gesetzt.

%% latex-abspann.tex
%% Gemeinsamer Abspann für Korrekturansicht und Leseansicht.
%% Setzt den Schalter \ifkorrekturansicht voraus (gesetzt in den
%% einbindenden Dateien latex-korrekturansicht-abspann.tex bzw.
%% latex-leseansicht-abspann.tex).
%% ---------------------------------------------------------------

\normalsize

% Das esempio-Environment wird nur in der Leseansicht benötigt
\ifkorrekturansicht\else
\newenvironment{esempio}[3]%
{
    \vspace{1.5ex}
    \rlap{\underline{#1}}
    \par
    \setlength{\parindent}{0cm}
    \nopagebreak
    \leftskip=#2cm
    \rightskip=#3cm
}
{
    \par
}
\fi

\doendnotes{C}
\bigskip
\vfill

\clearpage

\footnotesize

\ifkorrekturansicht
  \lohead{\textsc{register}}
\fi

% theindex-Environment neu definieren ohne reledmac
\makeatletter
\renewenvironment{theindex}{%
  \ifkorrekturansicht
    \section*{\indexname}%
  \else
    \subsubsection*{Index der erwähnten Entitäten}%
  \fi
  \setlength{\parindent}{0pt}%
  \setlength{\parskip}{0pt plus 0.3pt}%
  \let\item\@idxitem
}{%
  \ifkorrekturansicht\clearpage\fi
}
\makeatother

\IfFileExists{\jobname-pw.ind}{\input{\jobname-pw.ind}}{}

% Quellenangabe nur in der Leseansicht
\ifkorrekturansicht\else
% Fallback-Definitionen, falls die .tex-Datei \titel etc. nicht gesetzt hat
\providecommand{\titel}{}
\providecommand{\editorInnen}{}
\providecommand{\dateiname}{\jobname}

\vspace{3cm}

\vfill

\footnotesize
\textsc{Quelle}: \titel. Herausgegeben von {\editorInnen}. In: \emph{Arthur Schnitzler: Briefwechsel mit Autorinnen und Autoren}.
 Digitale Edition, https://schnitzler-briefe.acdh.oeaw.ac.at/{\dateiname}.html (Stand \today)
\fi

\end{document}


