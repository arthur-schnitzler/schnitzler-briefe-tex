%% latex-korrekturansicht-vorspann.tex
%% Vorspann für die Korrekturansicht.
%% Lädt die gemeinsame Datei latex-vorspann.tex mit gesetztem Schalter.

\newif\ifkorrekturansicht
\korrekturansichttrue

\input{../tex-inputs/latex-vorspann}


\section[Hermann Bahr an Arthur Schnitzler, 15. 5. 1902]{L01219 Hermann Bahr an Arthur Schnitzler, 15. 5. 1902}
\nopagebreak\mylabel{L01219v}
\rehead{ }\normalsize\beginnumbering\briefempfaengerindex{Schnitzler, Arthur@\textsc{Schnitzler, Arthur}!zzzBahr, Hermann@\emph{von Hermann Bahr}!1902-05-151@{15. 5. 1902}|(be}
\toendnotes[C]{\smallbreak\pagebreak[2]}\Standort{CUL, Schnitzler, B 5b.}
\physDesc{Brief, 1 Blatt, 4 Seiten, 1503 Zeichen
\newline{}Handschrift: schwarze Tinte, deutsche Kurrent
\newline{}Ordnung: mit Bleistift von unbekannter Hand nummeriert:
                                    »88« }
\buchAbdrucke{\weitereDrucke{Hermann Bahr, Arthur Schnitzler: \emph{Briefwechsel, Aufzeichnungen, Dokumente (1891–1931)}. Göttingen: \emph{Wallstein} 2018, S. 237.} }\toendnotes[C]{\smallbreak}
\pstart
           \raggedleft{}{\pb}15. Mai 1902\pend
           \vspace{0.5em}
\pstart
           Du biſt enttäuſcht, lieber Arthur, da Du geöffnet haſt und ſiehſt,
               daß dieſe Blumen, ſtatt von einem Weibchen, nur von mir ſind. Aber ſie sollen Dir
               halt heute, wo Du ankommst\pend
           
\pstart
           \centering{}\label{K_L01219-1v}\edtext{\textsc{nel mezzo del cammin di nostra vita}\pwindex{goettliche Komoedie@\emph{Die göttliche Komödie}|pwv}}{\lemma{\textnormal{\emph{nel … vita}}}\Cendnote{\textnormal{Dante\pwindex{Dante Alighieri um 1265 – 22.09.1321@\textsc{Dante Alighieri} (um 1265 – 22.09.1321), \emph{Schriftsteller/Schriftstellerin}|pwk}: \emph{Die göttliche Komödie}\pwindex{goettliche Komoedie@\emph{Die göttliche Komödie}|pwk}, 1. Vers des \emph{Inferno}.}}}\label{K_L01219-1},\pend
           
\pstart
           einmal ſagen, daß ich Dich ſehr gern habe und über unſer gut und feſt gewordenes
               Verhältnis froh bin und meine, es könne, was immer etwa noch {[}das{]}{ }Schickſal zwiſchen uns werfen mag, doch eigentlich
               im Grunde {\pb}niemals mehr wankend werden. Und mir
               iſt, frühere Dinge jetzt erſt zu verſtehen, und ich rede mir ein zu meinen, daß, was
               ich einſt gegen Dich empfunden habe, vielleicht auch nur eine freundſchaftliche
               Ungeduld geweſen ſein mag, den zu lange bei ſeiner Jugend Verweilenden ſchneller
               männlich werden zu ſehen. \label{LL248-2v}In meinem Verhältnis
                  zur Duſe\pwindex{Duse, Eleonora 03.10.1858 – 21.04.1924@\textsc{Duse, Eleonora} (03.10.1858 – 21.04.1924), \emph{Schauspieler/Schauspielerin}|pw}{ }{\pb}weiß ich \introOben{}jetzt\introOben{} ganz
                  gewiß, daß die unbegreifliche Wuth, die ich nach meiner erſten Begeiſterung
                  plötzlich auf ſie hatte,\label{LL248-2h} genau mit ihrer inneren Krise zuſammenfiel, aus
               welcher ſie verwandelt emporſtieg. Wäre ich d’Annunzio\pwindex{DAnnunzio, Gabriele 12.03.1863 – 01.03.1938@\textsc{D’Annunzio, Gabriele} (12.03.1863 – 01.03.1938), \emph{Schriftsteller/Schriftstellerin}|pw} und würde auch ſtyliſieren, ſo würde ich ſagen: \label{LL248-1v}Ich bin der Ehrgeiz meiner Freunde\label{LL248-1h} –
                  \label{K_L01219-2v}\edtext{\textsc{io sono l’orgoglio della mia razza}}{\lemma{\textnormal{\emph{io … razza}}}\Cendnote{\textnormal{Kein Zitat, sondern Prägung Bahrs\pwindex{Bahr, Hermann 19.07.1863 – 15.01.1934@\textsc{Bahr, Hermann} (19.07.1863 – 15.01.1934), \emph{Schriftsteller/Schriftstellerin, Kritiker/Kritikerin}|pwk}; die italienische\oindex{Italien@\textbf{Italien}, \emph{A.PCLI}|pwk} Übersetzung ist fehlerhaft, statt »l’orgoglio« (der Stolz)
                  müsste es ›l’ambizione‹ und statt »razza« ›amici‹ heißen. Rückübersetzen lässt
                  sich das Zitat als: »Ich bin der Stolz meiner Art (Rasse)«.}}}\label{K_L01219-2} (was übrigens
               ganz {\pb}gut klingt).\pend
           
\pstart
           Reden wir übrigens nicht vom Vergangenen, blicken wir vor {\dots} und da kann ich Dir nur wünſchen: Die nächſten zehn Jahre mögen Dir ſo reich ſein,
               als es Dir die letzten geweſen \introOben{}ſind\introOben{}! Dann werde ich Dich zu
               Deinem 50. öffentlich anſtrudeln müſſen, was weitaus nicht ſo gemütlich ſein
               wird.\pend
           
\pstart
           Des Herrn \label{K_L01219-3v}\edtext{Je\textcolor{gray}{tt}el\pwindex{Jettel-Ettenach, Emil von 08.04.1846 – 25.04.1925@\textsc{Jettel-Ettenach, Emil von} (08.04.1846 – 25.04.1925), \emph{Rechtswissenschaftler/Rechtswissenschaftlerin, Ministerialbeamter/Ministerialbeamte, Zensor/Zensorin}|pw}}{\lemma{\textnormal{\emph{Jettel}}}\Cendnote{\textnormal{Dieser verantwortete die Zensur des Burgtheaters\oindex{Burgtheater@\textbf{Burgtheater}, \emph{S.THTR}|pwk}. Zur Einordnung der kryptisch
                  bleibenden Stelle lässt sich ein Zusammenhang mit den Überlegungen für ein neues
                  Theatergesetz vermuten, an denen sich Bahr\pwindex{Bahr, Hermann 19.07.1863 – 15.01.1934@\textsc{Bahr, Hermann} (19.07.1863 – 15.01.1934), \emph{Schriftsteller/Schriftstellerin, Kritiker/Kritikerin}|pwk}
                  in dieser Zeit beteiligte.}}}\label{K_L01219-3} will ich gedenken. Wenn Du kommſt, telephonier
               vorher. Herzlichſt\pend
           
\pstart
           Dein{\\[\baselineskip]}\spacefill\mbox{Hermann}\pend
           \leftskip=0em{}\selectlanguage{ngerman}\endnumbering\briefempfaengerindex{Schnitzler, Arthur@\textsc{Schnitzler, Arthur}!zzzBahr, Hermann@\emph{von Hermann Bahr}!1902-05-151@{15. 5. 1902}|)be}\mylabel{L01219h}  \normalsize

\doendnotes{C}
\bigskip
\vfill

\clearpage

\footnotesize

\lohead{\textsc{register}}

% Definiere theindex-Environment komplett neu ohne reledmac
\makeatletter
\renewenvironment{theindex}{%
  \section*{\indexname}%
  \setlength{\parindent}{0pt}%
  \setlength{\parskip}{0pt plus 0.3pt}%
  \let\item\@idxitem
}{%
  \clearpage
}
\makeatother

\IfFileExists{\jobname-pw.ind}{\input{\jobname-pw.ind}}{}

\end{document}

      