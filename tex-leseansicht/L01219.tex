%% latex-leseansicht-vorspann.tex
%% Vorspann für die Leseansicht.
%% Lädt die gemeinsame Datei latex-vorspann.tex mit nicht gesetztem Schalter.

\newif\ifkorrekturansicht
\korrekturansichtfalse

\input{../tex-inputs/latex-vorspann}


\section[Hermann Bahr an Arthur Schnitzler, 15. 5. 1902]{L01219 Hermann Bahr an Arthur Schnitzler, 15. 5. 1902}
\nopagebreak\mylabel{L01219v}
\rehead{ }\normalsize\beginnumbering\briefempfaengerindex{Schnitzler, Arthur@\textsc{Schnitzler, Arthur}!zzzBahr, Hermann@\emph{von Hermann Bahr}!1902-05-151@{15. 5. 1902}|(be}
\toendnotes[C]{\smallbreak\pagebreak[2]}
\correspDesc{Versand  durch Hermann Bahr am 15. 5. 1902 in Wien
\newline{}Erhalt  durch Arthur Schnitzler im Zeitraum [15. 5. 1902
                  – 19. 5. 1902?] in Wien}\toendnotes[C]{\smallbreak}
\Standort{CUL, Schnitzler, B 5b.}
\physDesc{Brief, 1 Blatt, 4 Seiten, 1503 Zeichen
\newline{}Handschrift: schwarze Tinte, deutsche Kurrent
\newline{}Ordnung: mit Bleistift von unbekannter Hand nummeriert:
                                    »88« }
\buchAbdrucke{\weitereDrucke{Hermann Bahr, Arthur Schnitzler: \emph{Briefwechsel, Aufzeichnungen, Dokumente (1891–1931)}. Herausgegeben von Kurt Ifkovits und Martin Anton Müller. Göttingen: \emph{Wallstein} 2018, S. 237.} }\toendnotes[C]{\smallbreak}
\pstart
           \raggedleft{}{\pb}15. Mai 1902\pend
           \vspace{0.5em}
\pstart
           Du biſt enttäuſcht, lieber Arthur, da Du geöffnet haſt und{ }ſiehſt,
               daß dieſe Blumen,{ }ſtatt von einem Weibchen, nur von mir{ }ſind. Aber{ }ſie sollen Dir
               halt heute, wo Du ankommst\pend
           
\pstart
           \centering{}\label{K_L01219-1v}\edtext{\textsc{nel mezzo del cammin di nostra vita}\pwindex{Dante Alighieri um 1265 Florenz – 22.\,9.\,1321 Ravenna@\textsc{Dante Alighieri} (um 1265 Florenz – 22.\,9.\,1321 Ravenna), \emph{Schriftsteller}!göttliche Komödie@\strich\emph{Die göttliche Komödie}|pwv}}{\lemma{\textnormal{\emph{nel … vita}}}\Cendnote{\textnormal{Dante\pwindex{Dante Alighieri um 1265 Florenz – 22.\,9.\,1321 Ravenna@\textsc{Dante Alighieri} (um 1265 Florenz – 22.\,9.\,1321 Ravenna), \emph{Schriftsteller}|pwk}: \emph{Die göttliche Komödie}\pwindex{Dante Alighieri um 1265 Florenz – 22.\,9.\,1321 Ravenna@\textsc{Dante Alighieri} (um 1265 Florenz – 22.\,9.\,1321 Ravenna), \emph{Schriftsteller}!göttliche Komödie@\strich\emph{Die göttliche Komödie}|pwk}, 1. Vers des \emph{Inferno}.}}}\label{K_L01219-1},\pend
           
\pstart
           einmal{ }ſagen, daß ich Dich{ }ſehr gern habe und über unſer gut und feſt gewordenes
               Verhältnis froh bin und meine, es könne, was immer etwa noch {[}das{]}{ }Schickſal zwiſchen uns werfen mag, doch eigentlich
               im Grunde {\pb}niemals mehr wankend werden. Und mir
               iſt, frühere Dinge jetzt erſt zu verſtehen, und ich rede mir ein zu meinen, daß, was
               ich einſt gegen Dich empfunden habe, vielleicht auch nur eine freundſchaftliche
               Ungeduld geweſen{ }ſein mag, den zu lange bei{ }ſeiner Jugend Verweilenden{ }ſchneller
               männlich werden zu{ }ſehen. \label{LL248-2v}In meinem Verhältnis
                  zur Duſe\pwindex{Duse, Eleonora 3.\,10.\,1858 Vigevano – 21.\,4.\,1924 Pittsburgh@\textsc{Duse, Eleonora} (3.\,10.\,1858 Vigevano – 21.\,4.\,1924 Pittsburgh), \emph{Schauspielerin}|pw}{ }{\pb}weiß ich \introOben{}jetzt\introOben{} ganz
                  gewiß, daß die unbegreifliche Wuth, die ich nach meiner erſten Begeiſterung
                  plötzlich auf{ }ſie hatte,\label{LL248-2h} genau mit ihrer inneren Krise zuſammenfiel, aus
               welcher{ }ſie verwandelt emporſtieg. Wäre ich d’Annunzio\pwindex{D’Annunzio, Gabriele 12.\,3.\,1863 Pescara – 1.\,3.\,1938 Cargnacco@\textsc{D’Annunzio, Gabriele} (12.\,3.\,1863 Pescara – 1.\,3.\,1938 Cargnacco), \emph{Schriftsteller}|pw} und würde auch{ }ſtyliſieren,{ }ſo würde ich{ }ſagen: \label{LL248-1v}Ich bin der Ehrgeiz meiner Freunde\label{LL248-1h} –
                  \label{K_L01219-2v}\edtext{\textsc{io sono l’orgoglio della mia razza}}{\lemma{\textnormal{\emph{io … razza}}}\Cendnote{\textnormal{Kein Zitat, sondern Prägung Bahrs\pwindex{Bahr, Hermann 19.\,7.\,1863 Linz – 15.\,1.\,1934 München@\textsc{Bahr, Hermann} (19.\,7.\,1863 Linz – 15.\,1.\,1934 München), \emph{Schriftsteller, Kritiker}|pwk}; die italienische\oindex{Italien@\textbf{Italien}|pwk} Übersetzung ist fehlerhaft, statt »l’orgoglio« (der Stolz)
                  müsste es ›l’ambizione‹ und statt »razza« ›amici‹ heißen. Rückübersetzen lässt
                  sich das Zitat als: »Ich bin der Stolz meiner Art (Rasse)«.}}}\label{K_L01219-2} (was übrigens
               ganz {\pb}gut klingt).\pend
           
\pstart
           Reden wir übrigens nicht vom Vergangenen, blicken wir vor {\dots} und da kann ich Dir nur wünſchen: Die nächſten zehn Jahre mögen Dir{ }ſo reich{ }ſein,
               als es Dir die letzten geweſen \introOben{}ſind\introOben{}! Dann werde ich Dich zu
               Deinem 50. öffentlich anſtrudeln müſſen, was weitaus nicht{ }ſo gemütlich{ }ſein
               wird.\pend
           
\pstart
           Des Herrn \label{K_L01219-3v}\edtext{Je\textcolor{gray}{tt}el\pwindex{Jettel-Ettenach, Emil von 8.\,4.\,1846 Wien – 25.\,4.\,1925 ebd.@\textsc{Jettel-Ettenach, Emil von} (8.\,4.\,1846 Wien – 25.\,4.\,1925 ebd.), \emph{Rechtswissenschaftler, Ministerialbeamter, Zensor}|pw}}{\lemma{\textnormal{\emph{Jettel}}}\Cendnote{\textnormal{Dieser verantwortete die Zensur des Burgtheaters\oindex{Wien@\textbf{Wien}!I., Innere Stadt@\textbf{I., Innere Stadt}!Burgtheater@\textbf{Burgtheater}, \emph{Theater}|pwk}. Zur Einordnung der kryptisch
                  bleibenden Stelle lässt sich ein Zusammenhang mit den Überlegungen für ein neues
                  Theatergesetz vermuten, an denen sich Bahr\pwindex{Bahr, Hermann 19.\,7.\,1863 Linz – 15.\,1.\,1934 München@\textsc{Bahr, Hermann} (19.\,7.\,1863 Linz – 15.\,1.\,1934 München), \emph{Schriftsteller, Kritiker}|pwk}
                  in dieser Zeit beteiligte.}}}\label{K_L01219-3} will ich gedenken. Wenn Du kommſt, telephonier
               vorher. Herzlichſt\pend
           
\pstart
           Dein{\\[\baselineskip]}\spacefill\mbox{Hermann}\pend
           \leftskip=0em{}\selectlanguage{ngerman}\endnumbering\briefempfaengerindex{Schnitzler, Arthur@\textsc{Schnitzler, Arthur}!zzzBahr, Hermann@\emph{von Hermann Bahr}!1902-05-151@{15. 5. 1902}|)be}\mylabel{L01219h}  \newcommand{\dateiname}{L01219}\newcommand{\titel}{Hermann Bahr an Arthur Schnitzler, 15. 5. 1902}\newcommand{\editorInnen}{Herausgegeben von Martin Anton Müller}%% latex-leseansicht-abspann.tex
%% Abspann für die Leseansicht.
%% Der Schalter \ifkorrekturansicht ist bereits durch den Vorspann gesetzt.

%% latex-abspann.tex
%% Gemeinsamer Abspann für Korrekturansicht und Leseansicht.
%% Setzt den Schalter \ifkorrekturansicht voraus (gesetzt in den
%% einbindenden Dateien latex-korrekturansicht-abspann.tex bzw.
%% latex-leseansicht-abspann.tex).
%% ---------------------------------------------------------------

\normalsize

% Das esempio-Environment wird nur in der Leseansicht benötigt
\ifkorrekturansicht\else
\newenvironment{esempio}[3]%
{
    \vspace{1.5ex}
    \rlap{\underline{#1}}
    \par
    \setlength{\parindent}{0cm}
    \nopagebreak
    \leftskip=#2cm
    \rightskip=#3cm
}
{
    \par
}
\fi

\doendnotes{C}
\bigskip
\vfill

\clearpage

\footnotesize

\ifkorrekturansicht
  \lohead{\textsc{register}}
\fi

% theindex-Environment neu definieren ohne reledmac
\makeatletter
\renewenvironment{theindex}{%
  \ifkorrekturansicht
    \section*{\indexname}%
  \else
    \subsubsection*{Index der erwähnten Entitäten}%
  \fi
  \setlength{\parindent}{0pt}%
  \setlength{\parskip}{0pt plus 0.3pt}%
  \let\item\@idxitem
}{%
  \ifkorrekturansicht\clearpage\fi
}
\makeatother

\IfFileExists{\jobname-pw.ind}{\input{\jobname-pw.ind}}{}

% Quellenangabe nur in der Leseansicht
\ifkorrekturansicht\else
% Fallback-Definitionen, falls die .tex-Datei \titel etc. nicht gesetzt hat
\providecommand{\titel}{}
\providecommand{\editorInnen}{}
\providecommand{\dateiname}{\jobname}

\vspace{3cm}

\vfill

\footnotesize
\textsc{Quelle}: \titel. Herausgegeben von {\editorInnen}. In: \emph{Arthur Schnitzler: Briefwechsel mit Autorinnen und Autoren}.
 Digitale Edition, https://schnitzler-briefe.acdh.oeaw.ac.at/{\dateiname}.html (Stand \today)
\fi

\end{document}


