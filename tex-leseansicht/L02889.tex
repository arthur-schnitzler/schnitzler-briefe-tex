%% latex-korrekturansicht-vorspann.tex
%% Vorspann für die Korrekturansicht.
%% Lädt die gemeinsame Datei latex-vorspann.tex mit gesetztem Schalter.

\newif\ifkorrekturansicht
\korrekturansichttrue

\input{../tex-inputs/latex-vorspann}


\section[ Paul Goldmann an Arthur Schnitzler, 8. 10. {[}1899{]}]{L02889 Paul Goldmann an Arthur Schnitzler, 8. 10. {[}1899{]}}
\nopagebreak\mylabel{L02889v}
\rehead{ }\normalsize\beginnumbering\briefempfaengerindex{Schnitzler, Arthur@\textsc{Schnitzler, Arthur}!zzzGoldmann, Paul@\emph{von Paul Goldmann}!1899-10-083@{8. 10. {[}1899{]}}|(be}
\toendnotes[C]{\smallbreak\pagebreak[2]}\Standort{DLA, A:Schnitzler, HS.NZ85.1.3169.}
\physDesc{Brief, 1 Blatt, 3 Seiten, 1455 Zeichen
\newline{}Handschrift: blaue Tinte, deutsche Kurrent
\newline{}Schnitzler: 1) mit Bleistift das Jahr »99.« vermerkt  2) mit rotem Buntstift eine Unterstreichung}\toendnotes[C]{\smallbreak}
\pstart
           {\pb}\textcolor{gray}{\textbf{\begin{otherlanguage}{french}Florence\oindex{Florenz@\textbf{Florenz}, \emph{P.PPLA}|pw} – Hôtel Pension Barbensi\oindex{Hôtel Pension Barbensi@\textbf{Hôtel Pension Barbensi}, \emph{Hotel (K.HTL)}|pw}\end{otherlanguage}}}\hfill 8. Oktober.\pend
           
\pstart
           Lung’Arno Guicciardini\oindex{Lungarno Guicciardini@\textbf{Lungarno Guicciardini}, \emph{Straße (K.STR)}|pw}\pend
           
\pstart
           \textcolor{gray}{\textbf{G. ZANETTA\pwindex{Zanetta, G. @\textsc{Zanetta, G.}, \emph{Hotelbesitzer/Hotelbesitzerin}|pw}{ }{\kaufmannsund} C.\textsuperscript{i}}}\pend
           \vspace{0.5em}
\pstart
           Mein lieber Freund, Ich habe lange unentſchloſſen hin
               und her geſchwankt, ob ich \label{K_L02889-1v}\edtext{nach Wien\oindex{Wien@\textbf{Wien}, \emph{A.ADM2}|pw} kommen}{\lemma{\textnormal{\emph{nach Wien kommen}}}\Cendnote{\textnormal{Goldmann\pwindex{Goldmann, Paul 31.01.1865 – 25.09.1935@\textsc{Goldmann, Paul} (31.01.1865 – 25.09.1935), \emph{Schriftsteller/Schriftstellerin, Journalist/Journalistin}|pwk} kam am 13. 10. 1899 (dem
                  besagten Freitag) nach Wien\oindex{Wien@\textbf{Wien}, \emph{A.ADM2}|pwk} und blieb bis zum 21. 10. 1899. Er wohnte bei Schnitzler.}}}\label{K_L02889-1} ſoll. Der Abſchied von Florenz\oindex{Florenz@\textbf{Florenz}, \emph{P.PPLA}|pw} fällt mir unſagbar ſchwer, und ich wäre gern noch acht
               Tage geblieben. Der Wunſch, Dich noch einmal wiederzuſehen, ehe ich wieder in der
               großen Arbeit untertauche, hat \strikeout{\textcolor{gray}{f}} den Ausſchlag gegeben. Allerdings hätte ich heut beinahe noch mein Reiſe-Projekt rückgängig gemacht, da ich die geſtern von Dir erbetene telegraphiſche Antwort nicht
               erhielt. Aber ich dachte mir am Ende, daß vielleicht nur ein Communications-Hinderniß
               vorliegt, und werde morgen alſo doch nach Venedig\oindex{Venedig@\textbf{Venedig}, \emph{P.PPLA}|pw} reiſen. {\pb}Dort bleibe ich zwei oder drei Tage und komme dann etwa Freitag nach Wien\oindex{Wien@\textbf{Wien}, \emph{A.ADM2}|pw}, um dort mit Dir die
               letzten acht Tage meines Urlaubs zu verbringen. Immerhin bitte ich Dich, mir ſofort
               nach Empfang dieſes Briefes nach Venedig\oindex{Venedig@\textbf{Venedig}, \emph{P.PPLA}|pw}{ }\begin{otherlanguage}{french}\textsc{Poste restante}\end{otherlanguage}{ }zu telegraphiren, ob Dir meine Ankunft am Freitag recht iſt.\pend
           
\pstart
           Ich kann alſo bei Dir wohnen? Denn mein Reiſegeld langt nicht mehr viel weiter als
               zur Beſtreitung der Reiſe nach Wien\oindex{Wien@\textbf{Wien}, \emph{A.ADM2}|pw} und von da
               nach Frankfurt\oindex{Frankfurt am Main@\textbf{Frankfurt am Main}, \emph{P.PPLA3}|pw}. Werde ich aber Dich und die
               Deinigen nicht ſtören?\pend
           
\pstart
           Bitte, ſchreibe an \textsc{Richard\pwindex{Beer-Hofmann, Richard 1866-07-11 – 1945-09-26@\textsc{Beer-Hofmann, Richard} (1866-07-11 – 1945-09-26), \emph{Schriftsteller/Schriftstellerin}|pw}}, daß auch er \label{K_L02889-2v}\edtext{nach Wien\oindex{Wien@\textbf{Wien}, \emph{A.ADM2}|pw} kommt}{\lemma{\textnormal{\emph{nach Wien kommt}}}\Cendnote{\textnormal{Beer-Hofmann\pwindex{Beer-Hofmann, Richard 1866-07-11 – 1945-09-26@\textsc{Beer-Hofmann, Richard} (1866-07-11 – 1945-09-26), \emph{Schriftsteller/Schriftstellerin}|pwk} hielt sich ab dem 16. 10. 1899 wieder in Wien\oindex{Wien@\textbf{Wien}, \emph{A.ADM2}|pwk} auf (vgl. Richard Beer-Hofmann an Arthur Schnitzler, 15. 10. 1899).
                     Schnitzler traf ihn während Goldmanns\pwindex{Goldmann, Paul 31.01.1865 – 25.09.1935@\textsc{Goldmann, Paul} (31.01.1865 – 25.09.1935), \emph{Schriftsteller/Schriftstellerin, Journalist/Journalistin}|pwk} Anwesenheit am 17. 10. 1899 und 19. 10. 1899. Am 21. 10. 1899 besuchte Goldmann\pwindex{Goldmann, Paul 31.01.1865 – 25.09.1935@\textsc{Goldmann, Paul} (31.01.1865 – 25.09.1935), \emph{Schriftsteller/Schriftstellerin, Journalist/Journalistin}|pwk}{ }Beer-Hofmann\pwindex{Beer-Hofmann, Richard 1866-07-11 – 1945-09-26@\textsc{Beer-Hofmann, Richard} (1866-07-11 – 1945-09-26), \emph{Schriftsteller/Schriftstellerin}|pwk}.}}}\label{K_L02889-2}, falls er nicht ſchon
               zurück ſein ſollte.\pend
           
\pstart
           Mir droht ein ſchweres Unheil: Wie ich aus Frankfurt\oindex{Frankfurt am Main@\textbf{Frankfurt am Main}, \emph{P.PPLA3}|pw} höre, wird \textsc{Rottenberg\pwindex{Rottenberg, Ludwig 11.10.1864 – 6.5.1932@\textsc{Rottenberg, Ludwig} (11.10.1864 – 6.5.1932), \emph{Kapellmeister/Kapellmeisterin}|pw}} wahrſcheinlich an \label{K_L02889-3v}\edtext{Stelle von {\pb}\textsc{Fuchs\pwindex{Fuchs, Johann Nepomuk 1842-05-05 – 1899-10-15@\textsc{Fuchs, Johann Nepomuk} (1842-05-05 – 1899-10-15), \emph{Komponist/Komponistin, Hofkapellmeister/Hofkapellmeisterin}|pw}}}{\lemma{\textnormal{\emph{Stelle von Fuchs}}}\Cendnote{\textnormal{Goldmann\pwindex{Goldmann, Paul 31.01.1865 – 25.09.1935@\textsc{Goldmann, Paul} (31.01.1865 – 25.09.1935), \emph{Schriftsteller/Schriftstellerin, Journalist/Journalistin}|pwk} bezog sich höchstwahrscheinlich
                  auf Johann Nepomuk Fuchs\pwindex{Fuchs, Johann Nepomuk 1842-05-05 – 1899-10-15@\textsc{Fuchs, Johann Nepomuk} (1842-05-05 – 1899-10-15), \emph{Komponist/Komponistin, Hofkapellmeister/Hofkapellmeisterin}|pwk}, seit 1894 Vizehofkapellmeister an der \emph{Wiener Hofoper}\orgindex{K.K. Hof-Oper@K.K. Hof-Oper|pwk}, der zu dieser Zeit bereits erkrankt war.
                  Am 15. 10. 1899 verstarb er. Ludwig Rottenberg\pwindex{Rottenberg, Ludwig 11.10.1864 – 6.5.1932@\textsc{Rottenberg, Ludwig} (11.10.1864 – 6.5.1932), \emph{Kapellmeister/Kapellmeisterin}|pwk} war seit 1892
                  Erster Kapellmeister an der \emph{Frankfurter Oper}\orgindex{Frankfurter Opernhaus@Frankfurter Opernhaus|pwk}
                  und gastierte zwischen 15. 10. 1899 und 21. 11. 1899, als die \emph{Hofoper}\orgindex{K.K. Hof-Oper@K.K. Hof-Oper|pwk} Personalmangel verzeichnete, in Wien\oindex{Wien@\textbf{Wien}, \emph{A.ADM2}|pwk}.}}}\label{K_L02889-3} nach Wien\oindex{Wien@\textbf{Wien}, \emph{A.ADM2}|pw}\orgindex{K.K. Hof-Oper@K.K. Hof-Oper|pwv} berufen. \label{K_L02889-4v}\edtext{Das wäre das Ende}{\lemma{\textnormal{\emph{Das wäre das Ende}}}\Cendnote{\textnormal{Bezug auf die Beziehung Goldmanns\pwindex{Goldmann, Paul 31.01.1865 – 25.09.1935@\textsc{Goldmann, Paul} (31.01.1865 – 25.09.1935), \emph{Schriftsteller/Schriftstellerin, Journalist/Journalistin}|pwk} mit Rottenbergs\pwindex{Rottenberg, Ludwig 11.10.1864 – 6.5.1932@\textsc{Rottenberg, Ludwig} (11.10.1864 – 6.5.1932), \emph{Kapellmeister/Kapellmeisterin}|pwk} Ehefrau Theodore\pwindex{Rottenberg, Theodore 1875-09-07 – 1945-04-05@\textsc{Rottenberg, Theodore} (1875-09-07 – 1945-04-05)|pwk}.
                  Diese war, mit Unterbrechungen, von Herbst 1899 bis
                  mindestens Ende Juli 1905{ }Goldmanns\pwindex{Goldmann, Paul 31.01.1865 – 25.09.1935@\textsc{Goldmann, Paul} (31.01.1865 – 25.09.1935), \emph{Schriftsteller/Schriftstellerin, Journalist/Journalistin}|pwk} Geliebte. Aller
                  Wahrscheinlichkeit nach entsprang dieser außerehelichen Beziehung Theodore Rottenbergs\pwindex{Rottenberg, Theodore 1875-09-07 – 1945-04-05@\textsc{Rottenberg, Theodore} (1875-09-07 – 1945-04-05)|pwk} zweite Tochter, Gertrud Rottenberg\pwindex{Rottenberg, Gertrud 1900-08-02 – 1967-03-13@\textsc{Rottenberg, Gertrud} (1900-08-02 – 1967-03-13)|pwk}, verheiratete Hindemith\pwindex{Rottenberg, Gertrud 1900-08-02 – 1967-03-13@\textsc{Rottenberg, Gertrud} (1900-08-02 – 1967-03-13)|pwkv} (siehe Paul Goldmann an Arthur Schnitzler, 27. 11. [1899]).}}}\label{K_L02889-4}.\pend
           
\pstart
           Viele treue Grüße! Und auf baldiges Wiederſehen!\pend
           
\pstart
           Dein {\\[\baselineskip]}\spacefill\mbox{Paul Goldmann.}\pend
           \leftskip=0em{}
\pstart
           \noindent{}Meine Ankunft zeige ich Dir nach Wien\oindex{Wien@\textbf{Wien}, \emph{A.ADM2}|pw}{ }\label{K_L02889-5v}\edtext{telegraphiſch}{\lemma{\textnormal{\emph{telegraphiſch}}}\Cendnote{\textnormal{Siehe Paul Goldmann an Arthur Schnitzler, 13. 10. [1899?].
                  }}}\label{K_L02889-5} an.\pend
           \selectlanguage{ngerman}\endnumbering\briefempfaengerindex{Schnitzler, Arthur@\textsc{Schnitzler, Arthur}!zzzGoldmann, Paul@\emph{von Paul Goldmann}!1899-10-083@{8. 10. {[}1899{]}}|)be}\mylabel{L02889h}  \normalsize

\doendnotes{C}
\bigskip
\vfill

\clearpage

\footnotesize

\lohead{\textsc{register}}

% Definiere theindex-Environment komplett neu ohne reledmac
\makeatletter
\renewenvironment{theindex}{%
  \section*{\indexname}%
  \setlength{\parindent}{0pt}%
  \setlength{\parskip}{0pt plus 0.3pt}%
  \let\item\@idxitem
}{%
  \clearpage
}
\makeatother

\IfFileExists{\jobname-pw.ind}{\input{\jobname-pw.ind}}{}

\end{document}

      