%% latex-leseansicht-vorspann.tex
%% Vorspann für die Leseansicht.
%% Lädt die gemeinsame Datei latex-vorspann.tex mit nicht gesetztem Schalter.

\newif\ifkorrekturansicht
\korrekturansichtfalse

\input{../tex-inputs/latex-vorspann}

\begin{center}
            \textcolor{red}{ENTWURF, NICHT FERTIG KORRIGIERT}
                      \end{center}
            
         
         \renewcommand{\erwaehntePersonen}{Personen: Richard Beer-Hofmann, Johann Nepomuk Fuchs, Ludwig Rottenberg, Theodore Rottenberg, Gertrud Rottenberg, G. Zanetta}
         \renewcommand{\erwaehnteInstitutionen}{Institutionen: Frankfurter Opernhaus, K.K. Hof-Oper}
         \renewcommand{\erwaehnteOrte}{Orte: Berlin, Florenz, Frankfurt am Main, Hôtel Pension Barbensi, Lungarno Guicciardini, Venedig, Wien}
         \renewcommand{\erwaehnteWerke}{}
               \section[ Paul Goldmann an Arthur Schnitzler, 8. 10. {[}1899{]}]{ Paul Goldmann an Arthur Schnitzler, 8. 10. {[}1899{]}}\nopagebreak\mylabel{v}\rehead{ }\begin{ledgroupsized}[t]{13cm}\normalsize\beginnumbering \toendnotes[C]{\smallbreak\pagebreak[2]} \Standort{DLA, A:Schnitzler, HS.NZ85.1.3169.}
\physDesc{Brief, 1 Blatt, 3 Seiten
\newline{}Handschrift: blaue Tinte, deutsche Kurrent
\newline{}Schnitzler: 1) mit Bleistift das Jahr »99.« vermerkt  2) mit rotem Buntstift eine Unterstreichung}\toendnotes[C]{\smallbreak}\pstart
           \noindent{}{\pb}\textcolor{gray}{\textbf{\begin{otherlanguage}{french}Florence\oindex{Florenz@\textbf{Florenz}|pw} – Hôtel Pension Barbensi\oindex{Hôtel Pension Barbensi@\textbf{Hôtel Pension Barbensi}|pw}\end{otherlanguage}}}\hfill 8. Oktober.\pend
           \pstart
           Lung’Arno Guicciardini\oindex{Lungarno Guicciardini@\textbf{Lungarno Guicciardini}|pw}\pend
           \pstart
           \textcolor{gray}{\textbf{G. ZANETTA\pwindex{Zanetta, G. @\textsc{Zanetta, G.}, \emph{Hotelbesitzer/Hotelbesitzerin}|pw}{ }{\kaufmannsund} C.\textsuperscript{i}}}\pend
           \pstart
           Mein lieber Freund, Ich habe lange unentſchloſſen hin
               und her geſchwankt, ob ich \label{K_L02889-1v}\edtext{nach Wien\oindex{Wien@\textbf{Wien}|pw} kommen}{\lemma{\textnormal{\emph{nach Wien kommen}}}\Cendnote{\textnormal{Goldmann\pwindex{Goldmann, Paul 31.01.1865 – 25.09.1935@\textsc{Goldmann, Paul} (31.01.1865 – 25.09.1935), \emph{Schriftsteller, Journalist}|pwk} kam am 13. 10. 1899 (dem
                  besagten Freitag) nach Wien\oindex{Wien@\textbf{Wien}|pwk} und blieb bis zum 21. 10. 1899. Er wohnte bei Schnitzler\pwindex{Schnitzler, Arthur 15.05.1862 – 21.10.1931@\textsc{Schnitzler, Arthur} (15.05.1862 – 21.10.1931), \emph{Schriftsteller, Mediziner}|pwk}.}}}\label{K_L02889-1h} ſoll. Der Abſchied von Florenz\oindex{Florenz@\textbf{Florenz}|pw} fällt mir unſagbar ſchwer, und ich wäre gern noch acht
               Tage geblieben. Der Wunſch, Dich noch einmal wiederzuſehen, ehe ich wieder in der
               großen Arbeit untertauche, hat \strikeout{\textcolor{gray}{f}} den Ausſchlag gegeben. Allerdings hätte ich heut beinahe noch mein Reiſe-Projekt rückgängig gemacht, da ich die geſtern von Dir erbetene telegraphiſche Antwort nicht
               erhielt. Aber ich dachte mir am Ende, daß vielleicht nur ein Comunications-Hinderniß
               vorliegt, und werde morgen alſo doch nach Venedig\oindex{Venedig@\textbf{Venedig}|pw} reiſen. {\pb}Dort bleibe ich zwei oder drei Tage und komme dann etwa Freitag nach Wien\oindex{Wien@\textbf{Wien}|pw}, um dort mit Dir die
               letzten acht Tage meines Urlaubs zu verbringen. Immerhin bitte ich Dich, mir ſofort
               nach Empfang dieſes Briefes nach Venedig\oindex{Venedig@\textbf{Venedig}|pw}{ }\begin{otherlanguage}{french}\textsc{Poste restante}\end{otherlanguage}{ }zu telegraphiren, ob Dir meine Ankunft am Freitag recht iſt.\pend
           \pstart
           Ich kann alſo bei Dir wohnen? Denn mein Reiſegeld langt nicht mehr viel weiter als
               zur Beſtreitung der Reiſe nach Wien\oindex{Wien@\textbf{Wien}|pw} und von da
               nach Frankfurt\oindex{Frankfurt am Main@\textbf{Frankfurt am Main}|pw}. Werde ich aber Dich und die
               Deinigen nicht ſtören?\pend
           \pstart
           Bitte, ſchreibe an \textsc{Richard\pwindex{Beer-Hofmann, Richard 1866-07-11 – 1945-09-26@\textsc{Beer-Hofmann, Richard} (1866-07-11 – 1945-09-26), \emph{Schriftsteller}|pw}}, daß auch er \label{K_L02889-2v}\edtext{nach Wien\oindex{Wien@\textbf{Wien}|pw} kommt}{\lemma{\textnormal{\emph{nach Wien kommt}}}\Cendnote{\textnormal{Beer-Hofmann\pwindex{Beer-Hofmann, Richard 1866-07-11 – 1945-09-26@\textsc{Beer-Hofmann, Richard} (1866-07-11 – 1945-09-26), \emph{Schriftsteller}|pwk} hielt sich ab dem 16. 10. 1899 wieder in Wien\oindex{Wien@\textbf{Wien}|pwk} auf (vgl. Richard Beer-Hofmann an Arthur Schnitzler, 15. 10. 1899).
                     Schnitzler\pwindex{Schnitzler, Arthur 15.05.1862 – 21.10.1931@\textsc{Schnitzler, Arthur} (15.05.1862 – 21.10.1931), \emph{Schriftsteller, Mediziner}|pwk} traf ihn während Goldmann\pwindex{Goldmann, Paul 31.01.1865 – 25.09.1935@\textsc{Goldmann, Paul} (31.01.1865 – 25.09.1935), \emph{Schriftsteller, Journalist}|pwk}s Anwesenheit am 17. 10. 1899 und 19. 10. 1899. Am 21. 10. 1899 besuchte Goldmann\pwindex{Goldmann, Paul 31.01.1865 – 25.09.1935@\textsc{Goldmann, Paul} (31.01.1865 – 25.09.1935), \emph{Schriftsteller, Journalist}|pwk}{ }Beer-Hofmann\pwindex{Beer-Hofmann, Richard 1866-07-11 – 1945-09-26@\textsc{Beer-Hofmann, Richard} (1866-07-11 – 1945-09-26), \emph{Schriftsteller}|pwk}.}}}\label{K_L02889-2h}, falls er nicht ſchon
               zurück ſein ſollte.\pend
           \pstart
           Mir droht ein ſchweres Unheil: Wie ich aus Frankfurt\oindex{Frankfurt am Main@\textbf{Frankfurt am Main}|pw} höre, wird \textsc{Rottenberg\pwindex{Rottenberg, Ludwig 11.10.1864 – 6.5.1932@\textsc{Rottenberg, Ludwig} (11.10.1864 – 6.5.1932), \emph{Kapellmeister}|pw}} wahrſcheinlich an \label{K_L02889-4v}\edtext{Stelle von {\pb}\textsc{Fuchs\pwindex{Fuchs, Johann Nepomuk 1842-05-05 – 1899-10-15@\textsc{Fuchs, Johann Nepomuk} (1842-05-05 – 1899-10-15), \emph{Komponist, Hofkapellmeister}|pw}}}{\lemma{\textnormal{\emph{Stelle von Fuchs}}}\Cendnote{\textnormal{Goldmann\pwindex{Goldmann, Paul 31.01.1865 – 25.09.1935@\textsc{Goldmann, Paul} (31.01.1865 – 25.09.1935), \emph{Schriftsteller, Journalist}|pwk} bezog sich höchstwahrscheinlich
                  auf Johann Nepomuk Fuchs\pwindex{Fuchs, Johann Nepomuk 1842-05-05 – 1899-10-15@\textsc{Fuchs, Johann Nepomuk} (1842-05-05 – 1899-10-15), \emph{Komponist, Hofkapellmeister}|pwk}, seit 1894 Vizehofkapellmeister an der \emph{Wiener Hofoper}\orgindex{K.K. Hof-Oper@K.K. Hof-Oper|pwk}, der zu dieser Zeit bereits erkrankt war.
                  Am 15. 10. 1899 verstarb dieser. Ludwig Rottenberg\pwindex{Rottenberg, Ludwig 11.10.1864 – 6.5.1932@\textsc{Rottenberg, Ludwig} (11.10.1864 – 6.5.1932), \emph{Kapellmeister}|pwk} war seit 1892
                  Erster Kapellmeister an der \emph{Frankfurter Oper}\orgindex{Frankfurter Opernhaus@Frankfurter Opernhaus|pwk}
                  und gastierte zwischen 15. 10. 1899 und 21. 11. 1899, als die \emph{Hofoper}\orgindex{K.K. Hof-Oper@K.K. Hof-Oper|pwk} Personalmangel verzeichnete, in Wien\oindex{Wien@\textbf{Wien}|pwk}.}}}\label{K_L02889-4h} nach Wien\oindex{Wien@\textbf{Wien}|pw}\orgindex{K.K. Hof-Oper@K.K. Hof-Oper|pwv} berufen. \label{K_L02889-33v}\edtext{Das wäre das Ende}{\lemma{\textnormal{\emph{Das wäre das Ende}}}\Cendnote{\textnormal{… der Beziehung Goldmann\pwindex{Goldmann, Paul 31.01.1865 – 25.09.1935@\textsc{Goldmann, Paul} (31.01.1865 – 25.09.1935), \emph{Schriftsteller, Journalist}|pwk}s mit Rottenberg\pwindex{Rottenberg, Ludwig 11.10.1864 – 6.5.1932@\textsc{Rottenberg, Ludwig} (11.10.1864 – 6.5.1932), \emph{Kapellmeister}|pwk}s Ehefrau Theodore\pwindex{Rottenberg, Theodore 1875-09-07 – 1945-04-05@\textsc{Rottenberg, Theodore} (1875-09-07 – 1945-04-05)|pwk}.
                  Diese war jedenfalls im Herbst 1899 und vermutlich auch
                  im Sommer 1903{ }Goldmann\pwindex{Goldmann, Paul 31.01.1865 – 25.09.1935@\textsc{Goldmann, Paul} (31.01.1865 – 25.09.1935), \emph{Schriftsteller, Journalist}|pwk}s Geliebte (vgl. A. S.: \emph{Tagebuch}, 21. 9. 1899, 23. 9. 1899, Paul Goldmann an Arthur Schnitzler, 12. 11. [1899], A. S.: \emph{Tagebuch}, 18. 8. 1903, 19. 8. 1903). Aller Wahrscheinlichkeit nach entsprang dieser
                  außerehelichen Beziehung Theodore
                     Rottenberg\pwindex{Rottenberg, Theodore 1875-09-07 – 1945-04-05@\textsc{Rottenberg, Theodore} (1875-09-07 – 1945-04-05)|pwk}s zweite Tochter, Gertrud
                     Rottenberg\pwindex{Rottenberg, Gertrud 1900-08-02 – 1967-03-13@\textsc{Rottenberg, Gertrud} (1900-08-02 – 1967-03-13)|pwk}, verheiratete Hindemith\pwindex{Rottenberg, Gertrud 1900-08-02 – 1967-03-13@\textsc{Rottenberg, Gertrud} (1900-08-02 – 1967-03-13)|pwkv} (siehe Paul Goldmann an Arthur Schnitzler, 27. 11. [1899]).}}}\label{K_L02889-33h}.\pend
           \pstart
           Viele treue Grüße! Und auf baldiges Wiederſehen!\pend
           \pstart
           Dein {\\[\baselineskip]}\spacefill\mbox{Paul Goldmann.}\pend
           \leftskip=0em{}\pstart
           \noindent{}Meine Ankunft zeige ich Dir nach Wien\oindex{Wien@\textbf{Wien}|pw}{ }\label{K_L02889-3v}\edtext{telegraphiſch}{\lemma{\textnormal{\emph{telegraphiſch}}}\Cendnote{\textnormal{siehe Paul Goldmann an Arthur Schnitzler, 13. 10. [1899?]}}}\label{K_L02889-3h} an.\pend
           
         
         \endnumbering\mylabel{h}\end{ledgroupsized}  \newcommand{\dateiname}{L02889}\newcommand{\titel}{Paul Goldmann an Arthur Schnitzler, 8. 10. [1899]}\newcommand{\editorInnen}{Martin Anton Müller und Laura Untner}%% latex-leseansicht-abspann.tex
%% Abspann für die Leseansicht.
%% Der Schalter \ifkorrekturansicht ist bereits durch den Vorspann gesetzt.

%% latex-abspann.tex
%% Gemeinsamer Abspann für Korrekturansicht und Leseansicht.
%% Setzt den Schalter \ifkorrekturansicht voraus (gesetzt in den
%% einbindenden Dateien latex-korrekturansicht-abspann.tex bzw.
%% latex-leseansicht-abspann.tex).
%% ---------------------------------------------------------------

\normalsize

% Das esempio-Environment wird nur in der Leseansicht benötigt
\ifkorrekturansicht\else
\newenvironment{esempio}[3]%
{
    \vspace{1.5ex}
    \rlap{\underline{#1}}
    \par
    \setlength{\parindent}{0cm}
    \nopagebreak
    \leftskip=#2cm
    \rightskip=#3cm
}
{
    \par
}
\fi

\doendnotes{C}
\bigskip
\vfill

\clearpage

\footnotesize

\ifkorrekturansicht
  \lohead{\textsc{register}}
\fi

% theindex-Environment neu definieren ohne reledmac
\makeatletter
\renewenvironment{theindex}{%
  \ifkorrekturansicht
    \section*{\indexname}%
  \else
    \subsubsection*{Index der erwähnten Entitäten}%
  \fi
  \setlength{\parindent}{0pt}%
  \setlength{\parskip}{0pt plus 0.3pt}%
  \let\item\@idxitem
}{%
  \ifkorrekturansicht\clearpage\fi
}
\makeatother

\IfFileExists{\jobname-pw.ind}{\input{\jobname-pw.ind}}{}

% Quellenangabe nur in der Leseansicht
\ifkorrekturansicht\else
% Fallback-Definitionen, falls die .tex-Datei \titel etc. nicht gesetzt hat
\providecommand{\titel}{}
\providecommand{\editorInnen}{}
\providecommand{\dateiname}{\jobname}

\vspace{3cm}

\vfill

\footnotesize
\textsc{Quelle}: \titel. Herausgegeben von {\editorInnen}. In: \emph{Arthur Schnitzler: Briefwechsel mit Autorinnen und Autoren}.
 Digitale Edition, https://schnitzler-briefe.acdh.oeaw.ac.at/{\dateiname}.html (Stand \today)
\fi

\end{document}


      