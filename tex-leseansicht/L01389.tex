%% latex-leseansicht-vorspann.tex
%% Vorspann für die Leseansicht.
%% Lädt die gemeinsame Datei latex-vorspann.tex mit nicht gesetztem Schalter.

\newif\ifkorrekturansicht
\korrekturansichtfalse

\input{../tex-inputs/latex-vorspann}


\section[Hugo von Hofmannsthal an Arthur Schnitzler, {{[}}1. 4.? 1904{{]}}]{L01389 Hugo von Hofmannsthal an Arthur Schnitzler, {[}1. 4.? 1904{]}}
\nopagebreak\mylabel{L01389v}
\rehead{ }\normalsize\beginnumbering\briefempfaengerindex{Schnitzler, Arthur@\textsc{Schnitzler, Arthur}!zzzHofmannsthal, Hugo von@\emph{von Hugo von Hofmannsthal}!1904-04-011@{{[}1. 4.? 1904{]}}|(be}
\toendnotes[C]{\smallbreak\pagebreak[2]}
\correspDesc{Versand  durch Hugo von Hofmannsthal am [1. 4.? 1904] \textbf{Ort fehlend} 
\newline{}Erhalt  durch Arthur Schnitzler im Zeitraum [1. 4. 1904
                  – 5. 4. 1904?] in Wien}\toendnotes[C]{\smallbreak}
\Standort{CUL, Schnitzler, B 43.}
\physDesc{Brief, 1 Blatt, 1 Seite, 170 Zeichen
\newline{}Handschrift: schwarze Tinte, deutsche Kurrent
\newline{}Schnitzler: mit Bleistift datiert: »1/4? 904.« 
\newline{}Ordnung: 1) mit Bleistift von unbekannter Hand nummeriert: »\strikeout{213}«  2) mit Bleistift von unbekannter Hand nummeriert:
                                    »219«}
\buchAbdrucke{\weitereDrucke{Hugo von Hofmannsthal, Arthur Schnitzler: \emph{Briefwechsel}. Herausgegeben von Therese Nickl und Heinrich Schnitzler. Frankfurt am Main: \emph{S. Fischer} 1964, S. 185.} }
\pstart{}{\pb}mein lieber Arthur\pend\vspace{0.5em}
\pstart
           darf ich Ihnen Herrn Vollmöller\pwindex{Vollmoeller, Karl Gustav 7.\,5.\,1878 Stuttgart – 18.\,10.\,1948 Hollywood@\textsc{Vollmoeller, Karl Gustav} (7.\,5.\,1878 Stuttgart – 18.\,10.\,1948 Hollywood), \emph{Schriftsteller}|pw} den Dichter
               der \textsc{Catharina von Armagnac}\pwindex{Vollmoeller, Karl Gustav 7.\,5.\,1878 Stuttgart – 18.\,10.\,1948 Hollywood@\textsc{Vollmoeller, Karl Gustav} (7.\,5.\,1878 Stuttgart – 18.\,10.\,1948 Hollywood), \emph{Schriftsteller}!Catherina – Gräfin von Armagnac und ihre beiden Liebhaber@\strich\emph{Catherina – Gräfin von Armagnac und ihre beiden Liebhaber}|pw} vorſtellen.\pend
           
\pstart
           Er denkt{ }ſehr vernünftig und mit Liebe von Ihren{ }ſchönen Arbeiten.\pend
           
\pstart
           Ihr{\\[\baselineskip]}\spacefill\mbox{Hugo.}\pend
           \leftskip=0em{}\selectlanguage{ngerman}\endnumbering\briefempfaengerindex{Schnitzler, Arthur@\textsc{Schnitzler, Arthur}!zzzHofmannsthal, Hugo von@\emph{von Hugo von Hofmannsthal}!1904-04-011@{{[}1. 4.? 1904{]}}|)be}\mylabel{L01389h}  \newcommand{\dateiname}{L01389}\newcommand{\titel}{Hugo von Hofmannsthal an Arthur Schnitzler, [1. 4.? 1904]}\newcommand{\editorInnen}{Martin Anton Müller und Gerd-Hermann Susen}%% latex-leseansicht-abspann.tex
%% Abspann für die Leseansicht.
%% Der Schalter \ifkorrekturansicht ist bereits durch den Vorspann gesetzt.

%% latex-abspann.tex
%% Gemeinsamer Abspann für Korrekturansicht und Leseansicht.
%% Setzt den Schalter \ifkorrekturansicht voraus (gesetzt in den
%% einbindenden Dateien latex-korrekturansicht-abspann.tex bzw.
%% latex-leseansicht-abspann.tex).
%% ---------------------------------------------------------------

\normalsize

% Das esempio-Environment wird nur in der Leseansicht benötigt
\ifkorrekturansicht\else
\newenvironment{esempio}[3]%
{
    \vspace{1.5ex}
    \rlap{\underline{#1}}
    \par
    \setlength{\parindent}{0cm}
    \nopagebreak
    \leftskip=#2cm
    \rightskip=#3cm
}
{
    \par
}
\fi

\doendnotes{C}
\bigskip
\vfill

\clearpage

\footnotesize

\ifkorrekturansicht
  \lohead{\textsc{register}}
\fi

% theindex-Environment neu definieren ohne reledmac
\makeatletter
\renewenvironment{theindex}{%
  \ifkorrekturansicht
    \section*{\indexname}%
  \else
    \subsubsection*{Index der erwähnten Entitäten}%
  \fi
  \setlength{\parindent}{0pt}%
  \setlength{\parskip}{0pt plus 0.3pt}%
  \let\item\@idxitem
}{%
  \ifkorrekturansicht\clearpage\fi
}
\makeatother

\IfFileExists{\jobname-pw.ind}{\input{\jobname-pw.ind}}{}

% Quellenangabe nur in der Leseansicht
\ifkorrekturansicht\else
% Fallback-Definitionen, falls die .tex-Datei \titel etc. nicht gesetzt hat
\providecommand{\titel}{}
\providecommand{\editorInnen}{}
\providecommand{\dateiname}{\jobname}

\vspace{3cm}

\vfill

\footnotesize
\textsc{Quelle}: \titel. Herausgegeben von {\editorInnen}. In: \emph{Arthur Schnitzler: Briefwechsel mit Autorinnen und Autoren}.
 Digitale Edition, https://schnitzler-briefe.acdh.oeaw.ac.at/{\dateiname}.html (Stand \today)
\fi

\end{document}


