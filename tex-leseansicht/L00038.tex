%% latex-korrekturansicht-vorspann.tex
%% Vorspann für die Korrekturansicht.
%% Lädt die gemeinsame Datei latex-vorspann.tex mit gesetztem Schalter.

\newif\ifkorrekturansicht
\korrekturansichttrue

\input{../tex-inputs/latex-vorspann}


\section[Hugo von Hofmannsthal an Arthur Schnitzler, 9. 9. {[}1891{]}]{L00038 Hugo von Hofmannsthal an Arthur Schnitzler, 9. 9. {[}1891{]}}
\nopagebreak\mylabel{L00038v}
\rehead{ }\normalsize\beginnumbering\briefempfaengerindex{Schnitzler, Arthur@\textsc{Schnitzler, Arthur}!zzzHofmannsthal, Hugo von@\emph{von Hugo von Hofmannsthal}!1891-09-091@{9. 9. {[}1891{]}}|(be}
\toendnotes[C]{\smallbreak\pagebreak[2]}\Standort{CUL, Schnitzler, B 43.}
\physDesc{Brief, 1 Blatt, 2 Seiten, 542 Zeichen
\newline{}Handschrift: schwarze Tinte, deutsche Kurrent
\newline{}Schnitzler: mit Bleistift die Jahreszahl hinzugefügt:
                                 »91« 
\newline{}Ordnung: mit Bleistift von unbekannter Hand nummeriert:
                                 »7« }
\buchAbdrucke{\weitereDrucke{Hugo von Hofmannsthal, Arthur Schnitzler: \emph{Briefwechsel}. Frankfurt am Main: \emph{S. Fischer} 1964, S. 13.} }\toendnotes[C]{\smallbreak}
\pstart
           \noindent{}{\pb}Daſs Sie mich überhaupt noch
               grüßen laſſen, iſt wirklich hübſch von Ihnen. Der \label{K_L00038-1v}\edtext{Anfang}{\lemma{\textnormal{\emph{Anfang}}}\Cendnote{\textnormal{Arthur Schnitzler: \emph{Reichtum}\pwindex{Reichtum. Erzaehlung@\emph{Reichtum. Erzählung}|pwk}. In: \emph{Moderne
                        Rundschau}\pwindex{Moderne Rundschau@\emph{Moderne Rundschau}|pwk}, Bd. 3, H. 11, 1. 9. 1891, S. 385–391 (1. von 4
                     Teilen).}}}\label{K_L00038-1} von »Reichthum\pwindex{Reichtum. Erzaehlung@\emph{Reichtum. Erzählung}|pw}«
               ſcheint mir mit ſeiner Märchenſtimmung und ſeinen unwahrſcheinlichen
               Ariſtokratennamen etwas phantaſtiſches, arnimeskes\pwindex{Arnim, Achim von 26.01.1781 – 21.01.1831@\textsc{Arnim, Achim von} (26.01.1781 – 21.01.1831), \emph{Schriftsteller/Schriftstellerin}|pwv} zu verſprechen. Dann wäre es mir doppelt
               ſympathiſch.\pend
           
\pstart
           Aber – es wird doch nicht vielleicht eine ſociale Novelle werden wollen? Ich hoffe,
               Sie und Hoffmann\pwindex{Beer-Hofmann, Richard 1866-07-11 – 1945-09-26@\textsc{Beer-Hofmann, Richard} (1866-07-11 – 1945-09-26), \emph{Schriftsteller/Schriftstellerin}|pw} werden mir über die erſten
               8 Tage in {\pb}Wien\oindex{Wien@\textbf{Wien}, \emph{A.ADM2}|pw} hinweghelfen; vorläufig kann ich mir das
                  \label{K_L00038-2v}\edtext{Aufhören}{\lemma{\textnormal{\emph{Aufhören}}}\Cendnote{\textnormal{Mitte September 1891 war Schulbeginn, Hofmannsthals\pwindex{Hofmannsthal, Hugo von 1874-02-01 – 1929-07-15@\textsc{Hofmannsthal, Hugo von} (1874-02-01 – 1929-07-15), \emph{Schriftsteller/Schriftstellerin}|pwk} abschließendes Schuljahr begann.}}}\label{K_L00038-2} oder
               das Ertragen des Aufhörens nicht vorſtellen.\pend
           
\pstart
           Herzlichſt{\\[\baselineskip]}\spacefill\mbox{Loris.}\pend
           \leftskip=0em{}
\pstart
           \textsc{9. IX. im Segelboot.}\pend
           \selectlanguage{ngerman}\endnumbering\briefempfaengerindex{Schnitzler, Arthur@\textsc{Schnitzler, Arthur}!zzzHofmannsthal, Hugo von@\emph{von Hugo von Hofmannsthal}!1891-09-091@{9. 9. {[}1891{]}}|)be}\mylabel{L00038h}  \normalsize

\doendnotes{C}
\bigskip
\vfill

\clearpage

\footnotesize

\lohead{\textsc{register}}

% Definiere theindex-Environment komplett neu ohne reledmac
\makeatletter
\renewenvironment{theindex}{%
  \section*{\indexname}%
  \setlength{\parindent}{0pt}%
  \setlength{\parskip}{0pt plus 0.3pt}%
  \let\item\@idxitem
}{%
  \clearpage
}
\makeatother

\IfFileExists{\jobname-pw.ind}{\input{\jobname-pw.ind}}{}

\end{document}

      