%% latex-leseansicht-vorspann.tex
%% Vorspann für die Leseansicht.
%% Lädt die gemeinsame Datei latex-vorspann.tex mit nicht gesetztem Schalter.

\newif\ifkorrekturansicht
\korrekturansichtfalse

\input{../tex-inputs/latex-vorspann}


               \section[Hugo von Hofmannsthal an Arthur Schnitzler, 9. 9. {[}1891{]}]{ Hugo von Hofmannsthal an Arthur Schnitzler,
                    9. 9. {[}1891{]}}\nopagebreak\mylabel{v}\rehead{ }\begin{ledgroupsized}[t]{13cm}\normalsize\beginnumbering\briefempfaengerindex{Schnitzler, Arthur@\textsc{Schnitzler, Arthur}!zzzHofmannsthal, Hugo von@\emph{von Hugo von Hofmannsthal}!1891-09-091@{9. 9. {[}1891{]}}|(be} \toendnotes[C]{\smallbreak\pagebreak[2]} \Standort{CUL, Schnitzler, B 43.}
\physDesc{Brief, 1 Blatt, 2 Seiten
\newline{}Handschrift: schwarze Tinte, deutsche Kurrent
\newline{}Schnitzler: mit Bleistift die Jahreszahl hinzugefügt: »91« \newline{}Ordnung: mit Bleistift von unbekannter Hand nummeriert:
                                    »7« }\buchAbdrucke{\weitereDrucke{Hugo von Hofmannsthal, Arthur Schnitzler: \emph{Briefwechsel}. Hg. Therese Nickl und Heinrich Schnitzler. Frankfurt am Main: \emph{S. Fischer} 1964, S. 13.} }\toendnotes[C]{\smallbreak}\pstart
           \noindent{}{\pb}Daſs Sie mich überhaupt noch
                    grüßen laſſen, iſt wirklich hübſch von Ihnen. Der \label{K_L00038_1v}\edtext{Anfang}{\lemma{\textnormal{\emph{Anfang}}}\Cendnote{\textnormal{Arthur Schnitzler\pwindex{Schnitzler, Arthur 15.05.1862 – 21.10.1931@\textsc{Schnitzler, Arthur} (15.05.1862 – 21.10.1931), \emph{Schriftsteller, Mediziner}|pwk}: \emph{Reichtum}\pwindex{Schnitzler, Arthur 15.05.1862 – 21.10.1931@\textsc{Schnitzler, Arthur} (15.05.1862 – 21.10.1931), \emph{Schriftsteller, Mediziner}!Reichtum. Erzaehlung1.9.1891 – 15.10.1891@\strich\emph{Reichtum. Erzählung} {[}1.9.1891 – 15.10.1891{]}|pwk}. In: \emph{Moderne
                                Rundschau}\pwindex{Moderne Rundschau1.4.1891 – 31.12.1891@\emph{Moderne Rundschau}|pwk}, Bd. 3, H. 11, 1. 9. 1891, S. 385–391
                            (1. von 4 Teilen).}}}\label{K_L00038_1h} von »Reichthum\pwindex{Schnitzler, Arthur 15.05.1862 – 21.10.1931@\textsc{Schnitzler, Arthur} (15.05.1862 – 21.10.1931), \emph{Schriftsteller, Mediziner}!Reichtum. Erzaehlung1.9.1891 – 15.10.1891@\strich\emph{Reichtum. Erzählung} {[}1.9.1891 – 15.10.1891{]}|pw}« ſcheint mir mit ſeiner Märchenſtimmung und ſeinen
                    unwahrſcheinlichen Ariſtokratennamen etwas phantaſtiſches, arnimeskes\pwindex{Arnim, Achim von 26.01.1781 – 21.01.1831@\textsc{Arnim, Achim von} (26.01.1781 – 21.01.1831), \emph{Schriftsteller}|pwv} zu verſprechen. Dann wäre es
                    mir doppelt ſympathiſch.\pend
           \pstart
           Aber – es wird doch nicht vielleicht eine ſociale Novelle werden wollen? Ich
                    hoffe, Sie und Hoffmann\pwindex{Beer-Hofmann, Richard 11.07.1866 – 26.09.1945@\textsc{Beer-Hofmann, Richard} (11.07.1866 – 26.09.1945), \emph{Schriftsteller}|pw} werden mir über die
                    erſten 8 Tage in {\pb}Wien\oindex{Wien@\textbf{Wien}|pw} hinweghelfen; vorläufig kann ich mir das
                        \label{K_L00038_2v}\edtext{Aufhören}{\lemma{\textnormal{\emph{Aufhören}}}\Cendnote{\textnormal{Mitte September 1891 war Schulbeginn, Hofmannsthal\pwindex{Hofmannsthal, Hugo von 01.02.1874 – 15.07.1929@\textsc{Hofmannsthal, Hugo von} (01.02.1874 – 15.07.1929), \emph{Schriftsteller}|pwk}s abschließendes Schuljahr begann.}}}\label{K_L00038_2h}
                    oder das Ertragen des Aufhörens nicht vorſtellen.\pend
           \pstart
           Herzlichſt{\\[\baselineskip]}\spacefill\mbox{Loris.}\pend
           \leftskip=0em{}\pstart
           \textsc{9. IX. im Segelboot.}\pend
           \endnumbering\briefempfaengerindex{Schnitzler, Arthur@\textsc{Schnitzler, Arthur}!zzzHofmannsthal, Hugo von@\emph{von Hugo von Hofmannsthal}!1891-09-091@{9. 9. {[}1891{]}}|)be}\mylabel{h}\end{ledgroupsized}  \newcommand{\dateiname}{L00038}\newcommand{\titel}{Hugo von Hofmannsthal an Arthur Schnitzler, 9. 9. [1891]}\newcommand{\editorInnen}{Martin Anton Müller und Gerd-Hermann Susen}%% latex-leseansicht-abspann.tex
%% Abspann für die Leseansicht.
%% Der Schalter \ifkorrekturansicht ist bereits durch den Vorspann gesetzt.

%% latex-abspann.tex
%% Gemeinsamer Abspann für Korrekturansicht und Leseansicht.
%% Setzt den Schalter \ifkorrekturansicht voraus (gesetzt in den
%% einbindenden Dateien latex-korrekturansicht-abspann.tex bzw.
%% latex-leseansicht-abspann.tex).
%% ---------------------------------------------------------------

\normalsize

% Das esempio-Environment wird nur in der Leseansicht benötigt
\ifkorrekturansicht\else
\newenvironment{esempio}[3]%
{
    \vspace{1.5ex}
    \rlap{\underline{#1}}
    \par
    \setlength{\parindent}{0cm}
    \nopagebreak
    \leftskip=#2cm
    \rightskip=#3cm
}
{
    \par
}
\fi

\doendnotes{C}
\bigskip
\vfill

\clearpage

\footnotesize

\ifkorrekturansicht
  \lohead{\textsc{register}}
\fi

% theindex-Environment neu definieren ohne reledmac
\makeatletter
\renewenvironment{theindex}{%
  \ifkorrekturansicht
    \section*{\indexname}%
  \else
    \subsubsection*{Index der erwähnten Entitäten}%
  \fi
  \setlength{\parindent}{0pt}%
  \setlength{\parskip}{0pt plus 0.3pt}%
  \let\item\@idxitem
}{%
  \ifkorrekturansicht\clearpage\fi
}
\makeatother

\IfFileExists{\jobname-pw.ind}{\input{\jobname-pw.ind}}{}

% Quellenangabe nur in der Leseansicht
\ifkorrekturansicht\else
% Fallback-Definitionen, falls die .tex-Datei \titel etc. nicht gesetzt hat
\providecommand{\titel}{}
\providecommand{\editorInnen}{}
\providecommand{\dateiname}{\jobname}

\vspace{3cm}

\vfill

\footnotesize
\textsc{Quelle}: \titel. Herausgegeben von {\editorInnen}. In: \emph{Arthur Schnitzler: Briefwechsel mit Autorinnen und Autoren}.
 Digitale Edition, https://schnitzler-briefe.acdh.oeaw.ac.at/{\dateiname}.html (Stand \today)
\fi

\end{document}


      