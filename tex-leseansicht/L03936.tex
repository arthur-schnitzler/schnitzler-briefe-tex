%% latex-leseansicht-vorspann.tex
%% Vorspann für die Leseansicht.
%% Lädt die gemeinsame Datei latex-vorspann.tex mit nicht gesetztem Schalter.

\newif\ifkorrekturansicht
\korrekturansichtfalse

\input{../tex-inputs/latex-vorspann}


\section[Arthur Schnitzler an Theodor Herzl, 17. 11. 1900]{L03936 Arthur Schnitzler an Theodor Herzl, 17. 11. 1900}
\nopagebreak\mylabel{L03936v}
\rehead{ }\normalsize\beginnumbering\briefempfaengerindex{Herzl, Theodor@\textsc{Herzl, Theodor}!zzzSchnitzler, Arthur@\emph{von Arthur Schnitzler}!1900-11-171@{17. 11. 1900}|(be}
\toendnotes[C]{\smallbreak\pagebreak[2]}
\correspDesc{Versand  durch Arthur Schnitzler am 17. 11. 1900 in Wien
\newline{}Erhalt  durch Theodor Herzl in Wien}\toendnotes[C]{\smallbreak}
\Standort{Jerusalem, Central Zionist Archives, H1:1926-2.}
\physDesc{,  Blätter,  Seiten
\newline{}Handschrift: , deutsche Kurrent}
\buchAbdrucke{\weitereDrucke{Arthur Schnitzler: \emph{Briefe 1875–1912}. Herausgegeben von Therese Nickl und Heinrich Schnitzler. Frankfurt am Main: \emph{S. Fischer} 1981, S. 397–398.} }\toendnotes[C]{\smallbreak}
\pstart{}{\pb}Lieber Doctor Herzl,\pend\vspace{0.5em}
\pstart
           hier iſt die Geſchichte, »Lieutenant Guſtl\pwindex{Schnitzler, Arthur 15.\,5.\,1862 Wien – 21.\,10.\,1931 ebd.@\textsc{Schnitzler, Arthur} (15.\,5.\,1862 Wien – 21.\,10.\,1931 ebd.), \emph{Schriftsteller, Mediziner}!Lieutenant Gustl. Novelle@\strich\emph{Lieutenant Gustl. Novelle}|pw}«. Sie
               werde beim Leſen bemerken, daſs einiges geſtrichen werden \uline{muſs}, anderes weggelaſſen werden kann. Ich bitte Sie also, wenn Sie größere
               Striche für gerathen halten, mir das Ding wieder zurückzuſenden! Handelt es{ }ſich nur
               um {[}ein{]} paar Worte{ }ſo kann ich das wohl in der Fahnen{\pb}correctur beſorgen.\pend
           
\pstart
           Ich reiſe Montag, ſpäteſtens Dinſtag auf etwa 10 Tage fort,
               u. zw. nach Breslau\oindex{Breslau@\textbf{Breslau}|pw}, wo ich die Erzählung\pwindex{Schnitzler, Arthur 15.\,5.\,1862 Wien – 21.\,10.\,1931 ebd.@\textsc{Schnitzler, Arthur} (15.\,5.\,1862 Wien – 21.\,10.\,1931 ebd.), \emph{Schriftsteller, Mediziner}!Lieutenant Gustl. Novelle@\strich\emph{Lieutenant Gustl. Novelle}|pwv}{ }vorleſe\eventindex{Breslau@\textbf{Breslau}!Lesung aus Also sprach eine Frau (Elsbeth Meyer-Förster) und von Lieutenant Gustl (Arthur Schnitzler), 23.11.1900@Lesung aus Also sprach eine Frau{\rufezeichen} (Elsbeth Meyer-Förster) und von Lieutenant Gustl (Arthur Schnitzler), 23.11.1900|pwv}; aber \uline{Briefe},
               größere Sendungen nicht, werden mir nachgeſchickt.\pend
           
\pstart
           Ich nehme ferner an, dſs die N. Fr. Pr.\pwindex{Neue Freie Presse@\emph{Neue Freie Presse}|pw} bei der
               Zuerke{\geminationn}ung des Honorars auch die Ausdehnung eines Beitrags in Betracht zieht, ſo
               dſs hierüber nichts weiter zu bemerken iſt.\pend
           
\pstart
           {\pb}– Ich grüße Sie herzlich.\pend
           \pstart Ganz Ihr \spacefill\mbox{ArthurSchnitzler}\pend{}
\pstart
           Wien\oindex{Wien@\textbf{Wien}, \emph{Verwaltungsgebiet}|pw}{ }17. 11. 900.\pend
           \selectlanguage{ngerman}\endnumbering\briefempfaengerindex{Herzl, Theodor@\textsc{Herzl, Theodor}!zzzSchnitzler, Arthur@\emph{von Arthur Schnitzler}!1900-11-171@{17. 11. 1900}|)be}\mylabel{L03936h}
\begin{anhang}
\end{anhang}\newcommand{\dateiname}{L03936}\newcommand{\titel}{Arthur Schnitzler an Theodor Herzl, 17. 11. 1900}\newcommand{\editorInnen}{Herausgegeben von Jahnke, SelmaMüller, Martin Anton}%% latex-leseansicht-abspann.tex
%% Abspann für die Leseansicht.
%% Der Schalter \ifkorrekturansicht ist bereits durch den Vorspann gesetzt.

%% latex-abspann.tex
%% Gemeinsamer Abspann für Korrekturansicht und Leseansicht.
%% Setzt den Schalter \ifkorrekturansicht voraus (gesetzt in den
%% einbindenden Dateien latex-korrekturansicht-abspann.tex bzw.
%% latex-leseansicht-abspann.tex).
%% ---------------------------------------------------------------

\normalsize

% Das esempio-Environment wird nur in der Leseansicht benötigt
\ifkorrekturansicht\else
\newenvironment{esempio}[3]%
{
    \vspace{1.5ex}
    \rlap{\underline{#1}}
    \par
    \setlength{\parindent}{0cm}
    \nopagebreak
    \leftskip=#2cm
    \rightskip=#3cm
}
{
    \par
}
\fi

\doendnotes{C}
\bigskip
\vfill

\clearpage

\footnotesize

\ifkorrekturansicht
  \lohead{\textsc{register}}
\fi

% theindex-Environment neu definieren ohne reledmac
\makeatletter
\renewenvironment{theindex}{%
  \ifkorrekturansicht
    \section*{\indexname}%
  \else
    \subsubsection*{Index der erwähnten Entitäten}%
  \fi
  \setlength{\parindent}{0pt}%
  \setlength{\parskip}{0pt plus 0.3pt}%
  \let\item\@idxitem
}{%
  \ifkorrekturansicht\clearpage\fi
}
\makeatother

\IfFileExists{\jobname-pw.ind}{\input{\jobname-pw.ind}}{}

% Quellenangabe nur in der Leseansicht
\ifkorrekturansicht\else
% Fallback-Definitionen, falls die .tex-Datei \titel etc. nicht gesetzt hat
\providecommand{\titel}{}
\providecommand{\editorInnen}{}
\providecommand{\dateiname}{\jobname}

\vspace{3cm}

\vfill

\footnotesize
\textsc{Quelle}: \titel. Herausgegeben von {\editorInnen}. In: \emph{Arthur Schnitzler: Briefwechsel mit Autorinnen und Autoren}.
 Digitale Edition, https://schnitzler-briefe.acdh.oeaw.ac.at/{\dateiname}.html (Stand \today)
\fi

\end{document}


