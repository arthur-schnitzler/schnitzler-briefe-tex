%% latex-korrekturansicht-vorspann.tex
%% Vorspann für die Korrekturansicht.
%% Lädt die gemeinsame Datei latex-vorspann.tex mit gesetztem Schalter.

\newif\ifkorrekturansicht
\korrekturansichttrue

\input{../tex-inputs/latex-vorspann}


\section[Albert Ehrenstein an Arthur Schnitzler, 3. 12. 1905]{L01569 Albert Ehrenstein an Arthur Schnitzler, 3. 12. 1905}
\nopagebreak\mylabel{L01569v}
\rehead{ }\normalsize\beginnumbering\briefempfaengerindex{Schnitzler, Arthur@\textsc{Schnitzler, Arthur}!zzzEhrenstein, Albert@\emph{von Albert Ehrenstein}!1905-12-031@{3. 12. 1905}|(be}
\toendnotes[C]{\smallbreak\pagebreak[2]}\Standort{CUL, Schnitzler, B 30.}
\physDesc{Brief, 1 Blatt, 1 Seite, 315 Zeichen
\newline{}Handschrift: schwarze Tinte, deutsche Kurrent
\newline{}Schnitzler: mit Bleistift beschriftet: »\textsc{Ehrenst}« und die Adresse ergänzt: »\textsc{Ottakr.str. 114\oindex{Ottakringer Strasse@\textbf{Ottakringer Straße}, \emph{Straße (K.STR)}|pw}}« }
\buchAbdrucke{\weitereDrucke{Albert Ehrenstein: \emph{Briefe}. München: \emph{Boer} 1989, S. 18.} }
\pstart
           \raggedleft{}{\pb}XVI. Wien\oindex{XVI., Ottakring@\textbf{XVI., Ottakring}, \emph{A.ADM3}|pw}, 3. XII. 1905\pend
           
\pstart{}\textsc{Sehr geehrter Herr Doktor!}\pend\vspace{0.5em}
\pstart
           Ermuntert durch Herrn Doktors liebenswürdiges Entgegenkommen erlaube ich mir anbei
               meinen Dialog »Amok\pwindex{Amok@\emph{Amok}|pw}« zu unterbreiten und hoffe
               ich in einiger Zeit das für mich maßgebende Urteil über dieſen Trauerſchwank von
               Herrn Doktor hören zu können.\pend
           
\pstart
           Ergebenſt{\\[\baselineskip]}\spacefill\mbox{Albert Ehrenstein.}\pend
           \leftskip=0em{}\selectlanguage{ngerman}\endnumbering\briefempfaengerindex{Schnitzler, Arthur@\textsc{Schnitzler, Arthur}!zzzEhrenstein, Albert@\emph{von Albert Ehrenstein}!1905-12-031@{3. 12. 1905}|)be}\mylabel{L01569h}  \normalsize

\doendnotes{C}
\bigskip
\vfill

\clearpage

\footnotesize

\lohead{\textsc{register}}

% Definiere theindex-Environment komplett neu ohne reledmac
\makeatletter
\renewenvironment{theindex}{%
  \section*{\indexname}%
  \setlength{\parindent}{0pt}%
  \setlength{\parskip}{0pt plus 0.3pt}%
  \let\item\@idxitem
}{%
  \clearpage
}
\makeatother

\IfFileExists{\jobname-pw.ind}{\input{\jobname-pw.ind}}{}

\end{document}

      