%% latex-leseansicht-vorspann.tex
%% Vorspann für die Leseansicht.
%% Lädt die gemeinsame Datei latex-vorspann.tex mit nicht gesetztem Schalter.

\newif\ifkorrekturansicht
\korrekturansichtfalse

\input{../tex-inputs/latex-vorspann}


         
         \renewcommand{\erwaehntePersonen}{Personen: Richard Beer-Hofmann, Erich Freund, Paul Goldmann, Eduard Grisebach, Ernst Theodor Amadeus Hoffmann, Alfred Kerr, Olga Schnitzler, Elisabeth Steinrück, Jakob Wassermann, Julie Wassermann}
         \renewcommand{\erwaehnteInstitutionen}{Institutionen: Max Hesses Verlag, Neue Freie Presse}
         \renewcommand{\erwaehnteOrte}{Orte: Berlin, Breslau, Dessauer Straße, Leipzig, Reichstag, Rotensterngasse, Wien}
         \renewcommand{\erwaehnteWerke}{Werke: Der Schleier der Beatrice. Schauspiel in fünf Akten, E. T. A. Hoffmanns sämtliche Werke in fünfzehn Bänden, Lieutenant Gustl. Novelle, Neue Freie Presse}
               \section[ Paul Goldmann an Arthur Schnitzler, 30. 10. {[}1900{]}]{ Paul Goldmann an Arthur Schnitzler, 30. 10. {[}1900{]}}\nopagebreak\mylabel{v}\rehead{ }\begin{ledgroupsized}[t]{13cm}\normalsize\beginnumbering\briefempfaengerindex{Schnitzler, Arthur@\textsc{Schnitzler, Arthur}!zzzGoldmann, Paul@\emph{von Paul Goldmann}!1900-10-301@{30. 10. {[}1900{]}}|(be} \toendnotes[C]{\smallbreak\pagebreak[2]} \Standort{DLA, A:Schnitzler, HS.NZ85.1.3170.}
\physDesc{Brief, 1 Blatt, 4 Seiten, 1252 Zeichen
\newline{}Handschrift: blaue Tinte, deutsche Kurrent
\newline{}Schnitzler: 1) mit Bleistift das Jahr »900« vermerkt  2) mit rotem Buntstift vier Unterstreichungen}\toendnotes[C]{\smallbreak}\pstart
           \noindent{}\raggedleft{}{\pb}\textcolor{gray}{\textbf{DESSAUERSTRASSE 19}}\oindex{Dessauer Strasse@\textbf{Dessauer Straße}|pw}\pend
           \pstart
           Berlin\oindex{Berlin@\textbf{Berlin}|pw}, 30. Oktober.\pend
           \pstart\center{}Mein lieber Freund,\pend\pstart
           Als »Menſch« werde ich leider auch nicht nach Breslau\oindex{Breslau@\textbf{Breslau}|pw} kommen. Die \label{K_L02937-1v}\edtext{Aufführung\pwindex{Schnitzler, Arthur 15.05.1862 – 21.10.1931@\textsc{Schnitzler, Arthur} (15.05.1862 – 21.10.1931), \emph{Schriftsteller, Mediziner}!Schleier der Beatrice. Schauspiel in fuenf Akten1900-12-01@\strich\emph{Der Schleier der Beatrice. Schauspiel in fünf Akten} {[}1900-12-01{]}|pwv}}{\lemma{\textnormal{\emph{Aufführung}}}\Cendnote{\textnormal{siehe Paul Goldmann an Arthur Schnitzler, 30. 10. [1900]}}}\label{K_L02937-1h} iſt am 17., und am 14. wird hier\oindex{Berlin@\textbf{Berlin}|pwv} der
                  Reichstag\oindex{Reichstag@\textbf{Reichstag}|pw} eröffnet. Da darf ich mich nicht
               wegrühren. Aber ich rechne beſtimmt darauf, daß Du von Breslau\oindex{Breslau@\textbf{Breslau}|pw} nach Berlin\oindex{Berlin@\textbf{Berlin}|pw} kommſt, damit ich
               wenigſtens die Freude habe, Dich zu ſehen. Auch habe ich die Abſicht, der N. Fr. Pr.\orgindex{Neue Freie Presse@Neue Freie Presse|pw} den \textsc{Dr. Erich Freund\pwindex{Freund, Erich 1866-08-13 – 1940@\textsc{Freund, Erich} (1866-08-13 – 1940), \emph{Kritiker, Journalist}|pw}} in \textsc{Breslau}\oindex{Breslau@\textbf{Breslau}|pw}, den Du ja auch kennſt, {\pb}als Referenten
               vorzuſchlagen, damit wenigſtens ein anſtändiger und ehrlicher Kritiker über Dich
                  \label{K_L02937-2v}\edtext{berichtet}{\lemma{\textnormal{\emph{berichtet}}}\Cendnote{\textnormal{siehe Paul Goldmann an Arthur Schnitzler, 28. 2. [1898] und 3. 12. [1900]}}}\label{K_L02937-2h}.\pend
           \pstart
           Wann gedenkſt Du \label{K_L02937-3v}\edtext{nach Breslau\oindex{Breslau@\textbf{Breslau}|pw}}{\lemma{\textnormal{\emph{nach Breslau}}}\Cendnote{\textnormal{Schnitzler\pwindex{Schnitzler, Arthur 15.05.1862 – 21.10.1931@\textsc{Schnitzler, Arthur} (15.05.1862 – 21.10.1931), \emph{Schriftsteller, Mediziner}|pwk} hielt sich von 22. 11. 1900 bis 24. 11. 1900 und von
                     29. 11. 1900 bis
                     2. 12. 1900 in
                     Breslau\oindex{Breslau@\textbf{Breslau}|pwk} auf. Dazwischen war er in Berlin\oindex{Berlin@\textbf{Berlin}|pwk}.}}}\label{K_L02937-3h} zu reiſen?\pend
           \pstart
           Iſt es \strikeout{\textcolor{gray}{×}} wahr, daß \textsc{Wassermann\pwindex{Wassermann, Jakob 10.03.1873 – 01.01.1934@\textsc{Wassermann, Jakob} (10.03.1873 – 01.01.1934), \emph{Schriftsteller}|pw}} ſich mit einem Frl. \textsc{Speier}\pwindex{Wassermann, Julie 05.12.1876 – April 1963@\textsc{Wassermann, Julie} (05.12.1876 – April 1963), \emph{Schriftstellerin}|pw}{ }\label{K_L02937-4v}\edtext{verlobt}{\lemma{\textnormal{\emph{verlobt}}}\Cendnote{\textnormal{siehe A. S.: \emph{Tagebuch}, 11. 10. 1900}}}\label{K_L02937-4h} hat? Schön und reich?\pend
           \pstart
           Welches iſt die Adreſſe der \label{K_L02937-5v}\edtext{Fräulein\pwindex{Schnitzler, Olga 17.01.1882 – 13.01.1970@\textsc{Schnitzler, Olga} (17.01.1882 – 13.01.1970), \emph{Schauspielerin, Sängerin}|pwv}\pwindex{Steinrueck, Elisabeth 19.11.1885 – 07.04.1920@\textsc{Steinrück, Elisabeth} (19.11.1885 – 07.04.1920)|pwv} aus der Rothen-Stern-Gaſſe\oindex{Rotensterngasse@\textbf{Rotensterngasse}|pw}}{\lemma{\textnormal{\emph{Fräulein … Rothen-Stern-Gaſſe}}}\Cendnote{\textnormal{siehe Paul Goldmann an Arthur Schnitzler, 19. 9. [1900]}}}\label{K_L02937-5h}?\pend
           \pstart
           {\pb}Wann erſcheint der \label{K_L02937-6v}\edtext{»Lieutenant Guſtl\pwindex{Schnitzler, Arthur 15.05.1862 – 21.10.1931@\textsc{Schnitzler, Arthur} (15.05.1862 – 21.10.1931), \emph{Schriftsteller, Mediziner}!Lieutenant Gustl. Novelle1900-12-25@\strich\emph{Lieutenant Gustl. Novelle} {[}1900-12-25{]}|pw}«}{\lemma{\textnormal{\emph{»Lieutenant Guſtl«}}}\Cendnote{\textnormal{Arthur Schnitzler\pwindex{Schnitzler, Arthur 15.05.1862 – 21.10.1931@\textsc{Schnitzler, Arthur} (15.05.1862 – 21.10.1931), \emph{Schriftsteller, Mediziner}|pwk}: \emph{Lieutnant Gustl}\pwindex{Schnitzler, Arthur 15.05.1862 – 21.10.1931@\textsc{Schnitzler, Arthur} (15.05.1862 – 21.10.1931), \emph{Schriftsteller, Mediziner}!Lieutenant Gustl. Novelle1900-12-25@\strich\emph{Lieutenant Gustl. Novelle} {[}1900-12-25{]}|pwk}. In: \emph{Neue Freie Presse}\pwindex{Neue Freie Presse1864 – 1939@\emph{Neue Freie Presse} {[}1864 – 1939{]}|pwk}, Nr. 13053, 25. 12. 1900, Morgenblatt, S. 34–41. Siehe auch A. S.: \emph{Tagebuch}, 25. 12. 1900.}}}\label{K_L02937-6h}?\pend
           \pstart
           Wie geht’s Dir ſonſt? Frauen, Stimmung, Arbeit?\pend
           \pstart
           Mein Leben iſt troſtlos öde, ohne auch nur einen Schimmer von Freude. Aber ich leſe
                  \textsc{E. T. A. Hoffmann\pwindex{Hoffmann, Ernst Theodor Amadeus 1776-01-24 – 1822-06-25@\textsc{Hoffmann, Ernst Theodor Amadeus} (1776-01-24 – 1822-06-25), \emph{Schriftsteller, Komponist, Bildender Künstler}!E. T. A. Hoffmanns saemtliche Werke in fuenfzehn Baenden1900@\strich\emph{E. T. A. Hoffmanns sämtliche Werke in fünfzehn Bänden} {[}1900{]}|pwv}\pwindex{Hoffmann, Ernst Theodor Amadeus 1776-01-24 – 1822-06-25@\textsc{Hoffmann, Ernst Theodor Amadeus} (1776-01-24 – 1822-06-25), \emph{Schriftsteller, Komponist, Bildender Künstler}|pw}}. Bitte, \label{K_L02937-7v}\edtext{thue das auch}{\lemma{\textnormal{\emph{thue das auch}}}\Cendnote{\textnormal{\emph{E. T. A. Hoffmanns sämtliche Werke in fünfzehn
                        Bänden}\pwindex{Hoffmann, Ernst Theodor Amadeus 1776-01-24 – 1822-06-25@\textsc{Hoffmann, Ernst Theodor Amadeus} (1776-01-24 – 1822-06-25), \emph{Schriftsteller, Komponist, Bildender Künstler}!E. T. A. Hoffmanns saemtliche Werke in fuenfzehn Baenden1900@\strich\emph{E. T. A. Hoffmanns sämtliche Werke in fünfzehn Bänden} {[}1900{]}|pwk}. Herausgegeben von Eduard
                     Grisebach\pwindex{Grisebach, Eduard 1845-10-09 – 1906-03-22@\textsc{Grisebach, Eduard} (1845-10-09 – 1906-03-22), \emph{Diplomat, Jurist, Schriftsteller}|pwk}. Leipzig\oindex{Leipzig@\textbf{Leipzig}|pwk}: \emph{Max Hesse}\orgindex{Max Hesses Verlag@Max Hesses Verlag|pwk}{ }1900. Eine neuerliche Lektüre des Werks von Hoffmann\pwindex{Hoffmann, Ernst Theodor Amadeus 1776-01-24 – 1822-06-25@\textsc{Hoffmann, Ernst Theodor Amadeus} (1776-01-24 – 1822-06-25), \emph{Schriftsteller, Komponist, Bildender Künstler}|pwk} mit dieser Ausgabe\pwindex{Hoffmann, Ernst Theodor Amadeus 1776-01-24 – 1822-06-25@\textsc{Hoffmann, Ernst Theodor Amadeus} (1776-01-24 – 1822-06-25), \emph{Schriftsteller, Komponist, Bildender Künstler}!E. T. A. Hoffmanns saemtliche Werke in fuenfzehn Baenden1900@\strich\emph{E. T. A. Hoffmanns sämtliche Werke in fünfzehn Bänden} {[}1900{]}|pwkv} durch Schnitzler\pwindex{Schnitzler, Arthur 15.05.1862 – 21.10.1931@\textsc{Schnitzler, Arthur} (15.05.1862 – 21.10.1931), \emph{Schriftsteller, Mediziner}|pwk} ist nicht
                  bekannt.}}}\label{K_L02937-7h}! (Ausgabe\pwindex{Hoffmann, Ernst Theodor Amadeus 1776-01-24 – 1822-06-25@\textsc{Hoffmann, Ernst Theodor Amadeus} (1776-01-24 – 1822-06-25), \emph{Schriftsteller, Komponist, Bildender Künstler}!E. T. A. Hoffmanns saemtliche Werke in fuenfzehn Baenden1900@\strich\emph{E. T. A. Hoffmanns sämtliche Werke in fünfzehn Bänden} {[}1900{]}|pwv}
               von \textsc{Grisebach\pwindex{Grisebach, Eduard 1845-10-09 – 1906-03-22@\textsc{Grisebach, Eduard} (1845-10-09 – 1906-03-22), \emph{Diplomat, Jurist, Schriftsteller}|pw}}).\pend
           \pstart
           \textsc{Richard\pwindex{Beer-Hofmann, Richard 1866-07-11 – 1945-09-26@\textsc{Beer-Hofmann, Richard} (1866-07-11 – 1945-09-26), \emph{Schriftsteller}|pw}} benimmt ſich wieder einmal abſcheulich. {\pb}Antwortet mir nicht, \label{K_L02937-8v}\edtext{ſchickt mir
               nicht, worum ich ihn gebeten}{\lemma{\textnormal{\emph{ſchickt … gebeten}}}\Cendnote{\textnormal{siehe Paul Goldmann an Arthur Schnitzler, 19. 9. [1900]}}}\label{K_L02937-8h}. Rüttle ihn doch in meinem Namen etwas auf!\pend
           \pstart
           \textsc{Kerr\pwindex{Kerr, Alfred 25.12.1867 – 12.10.1948@\textsc{Kerr, Alfred} (25.12.1867 – 12.10.1948), \emph{Schriftsteller, Kritiker}|pw}} ſehe ich einmal im Monat auf fünf Minuten, die er jedesmal dazu benutzt, um mir
               zu erzählen, wie herrlich das Leben iſt.\pend
           \pstart
           Grüß’ Dich Gott, liebſter Freund! In Treue {\\[\baselineskip]}Dein {\\[\baselineskip]}\spacefill\mbox{Paul Goldmann.}\pend
           \leftskip=0em{}
         
         \endnumbering\mylabel{h}\end{ledgroupsized}  \newcommand{\dateiname}{L02937}\newcommand{\titel}{Paul Goldmann an Arthur Schnitzler, 30. 10. [1900]}\newcommand{\editorInnen}{Martin Anton Müller und Laura Untner}%% latex-leseansicht-abspann.tex
%% Abspann für die Leseansicht.
%% Der Schalter \ifkorrekturansicht ist bereits durch den Vorspann gesetzt.

%% latex-abspann.tex
%% Gemeinsamer Abspann für Korrekturansicht und Leseansicht.
%% Setzt den Schalter \ifkorrekturansicht voraus (gesetzt in den
%% einbindenden Dateien latex-korrekturansicht-abspann.tex bzw.
%% latex-leseansicht-abspann.tex).
%% ---------------------------------------------------------------

\normalsize

% Das esempio-Environment wird nur in der Leseansicht benötigt
\ifkorrekturansicht\else
\newenvironment{esempio}[3]%
{
    \vspace{1.5ex}
    \rlap{\underline{#1}}
    \par
    \setlength{\parindent}{0cm}
    \nopagebreak
    \leftskip=#2cm
    \rightskip=#3cm
}
{
    \par
}
\fi

\doendnotes{C}
\bigskip
\vfill

\clearpage

\footnotesize

\ifkorrekturansicht
  \lohead{\textsc{register}}
\fi

% theindex-Environment neu definieren ohne reledmac
\makeatletter
\renewenvironment{theindex}{%
  \ifkorrekturansicht
    \section*{\indexname}%
  \else
    \subsubsection*{Index der erwähnten Entitäten}%
  \fi
  \setlength{\parindent}{0pt}%
  \setlength{\parskip}{0pt plus 0.3pt}%
  \let\item\@idxitem
}{%
  \ifkorrekturansicht\clearpage\fi
}
\makeatother

\IfFileExists{\jobname-pw.ind}{\input{\jobname-pw.ind}}{}

% Quellenangabe nur in der Leseansicht
\ifkorrekturansicht\else
% Fallback-Definitionen, falls die .tex-Datei \titel etc. nicht gesetzt hat
\providecommand{\titel}{}
\providecommand{\editorInnen}{}
\providecommand{\dateiname}{\jobname}

\vspace{3cm}

\vfill

\footnotesize
\textsc{Quelle}: \titel. Herausgegeben von {\editorInnen}. In: \emph{Arthur Schnitzler: Briefwechsel mit Autorinnen und Autoren}.
 Digitale Edition, https://schnitzler-briefe.acdh.oeaw.ac.at/{\dateiname}.html (Stand \today)
\fi

\end{document}


      