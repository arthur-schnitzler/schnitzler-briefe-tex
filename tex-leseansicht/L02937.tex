%% latex-korrekturansicht-vorspann.tex
%% Vorspann für die Korrekturansicht.
%% Lädt die gemeinsame Datei latex-vorspann.tex mit gesetztem Schalter.

\newif\ifkorrekturansicht
\korrekturansichttrue

\input{../tex-inputs/latex-vorspann}


\section[ Paul Goldmann an Arthur Schnitzler, 30. 10. {[}1900{]}]{L02937 Paul Goldmann an Arthur Schnitzler, 30. 10. {[}1900{]}}
\nopagebreak\mylabel{L02937v}
\rehead{ }\normalsize\beginnumbering\briefempfaengerindex{Schnitzler, Arthur@\textsc{Schnitzler, Arthur}!zzzGoldmann, Paul@\emph{von Paul Goldmann}!1900-10-301@{30. 10. {[}1900{]}}|(be}
\toendnotes[C]{\smallbreak\pagebreak[2]}\Standort{DLA, A:Schnitzler, HS.NZ85.1.3170.}
\physDesc{Brief, 1 Blatt, 4 Seiten, 1252 Zeichen
\newline{}Handschrift: blaue Tinte, deutsche Kurrent
\newline{}Schnitzler: 1) mit Bleistift das Jahr »900« vermerkt  2) mit rotem Buntstift vier Unterstreichungen}\toendnotes[C]{\smallbreak}
\pstart
           \raggedleft{}{\pb}\textcolor{gray}{\textbf{DESSAUERSTRASSE 19}}\oindex{Dessauer Strasse@\textbf{Dessauer Straße}, \emph{Straße (K.STR)}|pw}\pend
           
\pstart
           Berlin\oindex{Berlin@\textbf{Berlin}, \emph{P.PPLC}|pw}, 30. Oktober.\pend
           
\pstart\center{}Mein lieber Freund,\pend\vspace{0.5em}
\pstart
           Als »Menſch« werde ich leider auch nicht nach Breslau\oindex{Breslau@\textbf{Breslau}, \emph{P.PPLA}|pw} kommen. Die \label{K_L02937-1v}\edtext{Aufführung\pwindex{Schleier der Beatrice. Schauspiel in fuenf Akten@\emph{Der Schleier der Beatrice. Schauspiel in fünf Akten}|pwv}}{\lemma{\textnormal{\emph{Aufführung}}}\Cendnote{\textnormal{Siehe Paul Goldmann an Arthur Schnitzler, 30. 10. [1900].
               }}}\label{K_L02937-1} iſt am 17., und am 14. wird hier\oindex{Berlin@\textbf{Berlin}, \emph{P.PPLC}|pwv} der
                  Reichstag\oindex{Reichstag@\textbf{Reichstag}, \emph{Regierungsgebäude (K.RGB)}|pw} eröffnet. Da darf ich mich nicht
               wegrühren. Aber ich rechne beſtimmt darauf, daß Du von Breslau\oindex{Breslau@\textbf{Breslau}, \emph{P.PPLA}|pw} nach Berlin\oindex{Berlin@\textbf{Berlin}, \emph{P.PPLC}|pw} kommſt, damit ich
               wenigſtens die Freude habe, Dich zu ſehen. Auch habe ich die Abſicht, der N. Fr. Pr.\orgindex{Neue Freie Presse@Neue Freie Presse|pw} den \textsc{Dr. Erich Freund\pwindex{Freund, Erich 1866-08-13 – 1940@\textsc{Freund, Erich} (1866-08-13 – 1940), \emph{Kritiker/Kritikerin, Musikjournalist/Musikjournalistin}|pw}} in \textsc{Breslau}\oindex{Breslau@\textbf{Breslau}, \emph{P.PPLA}|pw}, den Du ja auch kennſt, {\pb}als Referenten
               vorzuſchlagen, damit wenigſtens ein anſtändiger und ehrlicher Kritiker über Dich
                  \label{K_L02937-2v}\edtext{berichtet}{\lemma{\textnormal{\emph{berichtet}}}\Cendnote{\textnormal{Siehe Paul Goldmann an Arthur Schnitzler, 28. 2. [1898] und 3. 12. [1900].
               }}}\label{K_L02937-2}.\pend
           
\pstart
           Wann gedenkſt Du \label{K_L02937-3v}\edtext{nach Breslau\oindex{Breslau@\textbf{Breslau}, \emph{P.PPLA}|pw}}{\lemma{\textnormal{\emph{nach Breslau}}}\Cendnote{\textnormal{Schnitzler hielt sich vom 22. 11. 1900 bis zum 24. 11. 1900 und vom
                     29. 11. 1900 bis zum
                     2. 12. 1900 in
                     Breslau\oindex{Breslau@\textbf{Breslau}, \emph{P.PPLA}|pwk} auf. Dazwischen war er in Berlin\oindex{Berlin@\textbf{Berlin}, \emph{P.PPLC}|pwk}.}}}\label{K_L02937-3} zu reiſen?\pend
           
\pstart
           Iſt es \strikeout{\textcolor{gray}{×}} wahr, daß \textsc{Wassermann\pwindex{Wassermann, Jakob 10.03.1873 – 01.01.1934@\textsc{Wassermann, Jakob} (10.03.1873 – 01.01.1934), \emph{Schriftsteller/Schriftstellerin}|pw}} ſich mit einem Frl. \textsc{Speier}\pwindex{Wassermann, Julie 05.12.1876 – April 1963@\textsc{Wassermann, Julie} (05.12.1876 – April 1963), \emph{Schriftsteller/Schriftstellerin}|pw}{ }\label{K_L02937-4v}\edtext{verlobt}{\lemma{\textnormal{\emph{verlobt}}}\Cendnote{\textnormal{Siehe A. S.: \emph{Tagebuch}, 11. 10. 1900.
               }}}\label{K_L02937-4} hat? Schön und reich?\pend
           
\pstart
           Welches iſt die Adreſſe der \label{K_L02937-5v}\edtext{Fräulein\pwindex{Schnitzler, Olga 17.01.1882 – 13.01.1970@\textsc{Schnitzler, Olga} (17.01.1882 – 13.01.1970), \emph{Schauspieler/Schauspielerin, Sänger/Sängerin}|pwv}\pwindex{Steinrueck, Elisabeth 19.11.1885 – 07.04.1920@\textsc{Steinrück, Elisabeth} (19.11.1885 – 07.04.1920)|pwv} aus der Rothen-Stern-Gaſſe\oindex{Rotensterngasse@\textbf{Rotensterngasse}, \emph{Straße (K.STR)}|pw}}{\lemma{\textnormal{\emph{Fräulein … Rothen-Stern-Gaſſe}}}\Cendnote{\textnormal{Siehe Paul Goldmann an Arthur Schnitzler, 19. 9. [1900].
               }}}\label{K_L02937-5}?\pend
           
\pstart
           {\pb}Wann erſcheint der \label{K_L02937-6v}\edtext{»Lieutenant Guſtl\pwindex{Lieutenant Gustl. Novelle@\emph{Lieutenant Gustl. Novelle}|pw}«}{\lemma{\textnormal{\emph{»Lieutenant Guſtl«}}}\Cendnote{\textnormal{Arthur Schnitzler: \emph{Lieutnant Gustl}\pwindex{Lieutenant Gustl. Novelle@\emph{Lieutenant Gustl. Novelle}|pwk}. In: \emph{Neue Freie Presse}\pwindex{Neue Freie Presse@\emph{Neue Freie Presse}|pwk}, Nr. 13.053, 25. 12. 1900, Morgenblatt, S. 34–41. Siehe auch A. S.: \emph{Tagebuch}, 25. 12. 1900.}}}\label{K_L02937-6}?\pend
           
\pstart
           Wie geht’s Dir ſonſt? Frauen, Stimmung, Arbeit?\pend
           
\pstart
           Mein Leben iſt troſtlos öde, ohne auch nur einen Schimmer von Freude. Aber ich leſe
                  \textsc{E. T. A. Hoffmann\pwindex{E. T. A. Hoffmanns saemtliche Werke in fuenfzehn Baenden@\emph{E. T. A. Hoffmanns sämtliche Werke in fünfzehn Bänden}|pwv}\pwindex{Hoffmann, Ernst Theodor Amadeus 1776-01-24 – 1822-06-25@\textsc{Hoffmann, Ernst Theodor Amadeus} (1776-01-24 – 1822-06-25), \emph{Schriftsteller/Schriftstellerin, Komponist/Komponistin, Zeichner/Zeichnerin}|pw}}. Bitte, \label{K_L02937-7v}\edtext{thue das auch}{\lemma{\textnormal{\emph{thue das auch}}}\Cendnote{\textnormal{\emph{E. T. A. Hoffmanns sämtliche Werke in fünfzehn
                        Bänden}\pwindex{E. T. A. Hoffmanns saemtliche Werke in fuenfzehn Baenden@\emph{E. T. A. Hoffmanns sämtliche Werke in fünfzehn Bänden}|pwk}. Herausgegeben von Eduard
                     Grisebach\pwindex{Grisebach, Eduard 1845-10-09 – 1906-03-22@\textsc{Grisebach, Eduard} (1845-10-09 – 1906-03-22), \emph{Schriftsteller/Schriftstellerin, Diplomat/Diplomatin, Jurist/Juristin}|pwk}. Leipzig\oindex{Leipzig@\textbf{Leipzig}, \emph{P.PPLA3}|pwk}: \emph{Max Hesse}\orgindex{Max Hesses Verlag@Max Hesses Verlag|pwk}{ }1900. Eine neuerliche Lektüre des Werks von Hoffmann\pwindex{Hoffmann, Ernst Theodor Amadeus 1776-01-24 – 1822-06-25@\textsc{Hoffmann, Ernst Theodor Amadeus} (1776-01-24 – 1822-06-25), \emph{Schriftsteller/Schriftstellerin, Komponist/Komponistin, Zeichner/Zeichnerin}|pwk} mit dieser Ausgabe\pwindex{E. T. A. Hoffmanns saemtliche Werke in fuenfzehn Baenden@\emph{E. T. A. Hoffmanns sämtliche Werke in fünfzehn Bänden}|pwkv} durch Schnitzler ist nicht
                  bekannt.}}}\label{K_L02937-7}! (Ausgabe\pwindex{E. T. A. Hoffmanns saemtliche Werke in fuenfzehn Baenden@\emph{E. T. A. Hoffmanns sämtliche Werke in fünfzehn Bänden}|pwv}
               von \textsc{Grisebach\pwindex{Grisebach, Eduard 1845-10-09 – 1906-03-22@\textsc{Grisebach, Eduard} (1845-10-09 – 1906-03-22), \emph{Schriftsteller/Schriftstellerin, Diplomat/Diplomatin, Jurist/Juristin}|pw}}).\pend
           
\pstart
           \textsc{Richard\pwindex{Beer-Hofmann, Richard 1866-07-11 – 1945-09-26@\textsc{Beer-Hofmann, Richard} (1866-07-11 – 1945-09-26), \emph{Schriftsteller/Schriftstellerin}|pw}} benimmt ſich wieder einmal abſcheulich. {\pb}Antwortet mir nicht, \label{K_L02937-8v}\edtext{ſchickt mir
               nicht, worum ich ihn gebeten}{\lemma{\textnormal{\emph{ſchickt … gebeten}}}\Cendnote{\textnormal{Siehe Paul Goldmann an Arthur Schnitzler, 19. 9. [1900].
               }}}\label{K_L02937-8}. Rüttle ihn doch in meinem Namen etwas auf!\pend
           
\pstart
           \textsc{Kerr\pwindex{Kerr, Alfred 25.12.1867 – 12.10.1948@\textsc{Kerr, Alfred} (25.12.1867 – 12.10.1948), \emph{Schriftsteller/Schriftstellerin, Kritiker/Kritikerin}|pw}} ſehe ich einmal im Monat auf fünf Minuten, die er jedesmal dazu benutzt, um mir
               zu erzählen, wie herrlich das Leben iſt.\pend
           
\pstart
           Grüß’ Dich Gott, liebſter Freund! In Treue {\\[\baselineskip]}Dein {\\[\baselineskip]}\spacefill\mbox{Paul Goldmann.}\pend
           \leftskip=0em{}\selectlanguage{ngerman}\endnumbering\briefempfaengerindex{Schnitzler, Arthur@\textsc{Schnitzler, Arthur}!zzzGoldmann, Paul@\emph{von Paul Goldmann}!1900-10-301@{30. 10. {[}1900{]}}|)be}\mylabel{L02937h}  \normalsize

\doendnotes{C}
\bigskip
\vfill

\clearpage

\footnotesize

\lohead{\textsc{register}}

% Definiere theindex-Environment komplett neu ohne reledmac
\makeatletter
\renewenvironment{theindex}{%
  \section*{\indexname}%
  \setlength{\parindent}{0pt}%
  \setlength{\parskip}{0pt plus 0.3pt}%
  \let\item\@idxitem
}{%
  \clearpage
}
\makeatother

\IfFileExists{\jobname-pw.ind}{\input{\jobname-pw.ind}}{}

\end{document}

      