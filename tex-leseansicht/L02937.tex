%% latex-leseansicht-vorspann.tex
%% Vorspann für die Leseansicht.
%% Lädt die gemeinsame Datei latex-vorspann.tex mit nicht gesetztem Schalter.

\newif\ifkorrekturansicht
\korrekturansichtfalse

\input{../tex-inputs/latex-vorspann}


\section[ Paul Goldmann an Arthur Schnitzler, 30. 10. [1900]]{L02937 Paul Goldmann an Arthur Schnitzler,  30. 10. [1900]}
\nopagebreak\mylabel{L02937v}
\rehead{ }\normalsize\beginnumbering\briefempfaengerindex{Schnitzler, Arthur@\textsc{Schnitzler, Arthur}!zzzGoldmann, Paul@\emph{von Paul Goldmann}!1900-10-301@{30. 10. [1900]}|(be}
\toendnotes[C]{\smallbreak\pagebreak[2]}
\correspDesc{Versand  durch Paul Goldmann am 30. 10. [1900] in Berlin
\newline{}Erhalt  durch Arthur Schnitzler im Zeitraum [31. 10. 1900 – 4. 11. 1900?] in Wien}\toendnotes[C]{\smallbreak}
\Standort{DLA, A:Schnitzler, HS.NZ85.1.3170.}
\physDesc{Brief, 1 Blatt, 4 Seiten, 1252 Zeichen
\newline{}Handschrift: blaue Tinte, deutsche Kurrent
\newline{}Schnitzler: 1) mit Bleistift das Jahr »900« vermerkt  2) mit rotem Buntstift vier Unterstreichungen}\toendnotes[C]{\smallbreak}
\pstart
           \raggedleft{}{\pb}\textcolor{gray}{\textbf{DESSAUERSTRASSE 19}}\oindex{Dessauer Straße@\textbf{Dessauer Straße}, \emph{Straße}|pw}\pend
           
\pstart
           Berlin\oindex{Berlin@\textbf{Berlin}, \emph{Hauptstadt}|pw}, 30. Oktober.\pend
           
\pstart\center{}Mein lieber Freund,\pend\vspace{0.5em}
\pstart
           Als »Menſch« werde ich leider auch nicht nach Breslau\oindex{Breslau@\textbf{Breslau}|pw} kommen. Die \label{K_L02937-1v}\edtext{Aufführung\pwindex{Schnitzler, Arthur 15.\,5.\,1862 Wien – 21.\,10.\,1931 ebd.@\textsc{Schnitzler, Arthur} (15.\,5.\,1862 Wien – 21.\,10.\,1931 ebd.), \emph{Schriftsteller, Mediziner}!Schleier der Beatrice. Schauspiel in fünf Akten@\strich\emph{Der Schleier der Beatrice. Schauspiel in fünf Akten}|pwv}}{\lemma{\textnormal{\emph{Aufführung}}}\Cendnote{\textnormal{Siehe XXXX Auszeichnungsfehler: Dokument L02937 nicht gefunden.
               }}}\label{K_L02937-1} iſt am 17., und am 14. wird hier\oindex{Berlin@\textbf{Berlin}, \emph{Hauptstadt}|pwv} der
                  Reichstag\oindex{Reichstag@\textbf{Reichstag}, \emph{Regierungsgebäude}|pw} eröffnet. Da darf ich mich nicht
               wegrühren. Aber ich rechne beſtimmt darauf, daß Du von Breslau\oindex{Breslau@\textbf{Breslau}|pw} nach Berlin\oindex{Berlin@\textbf{Berlin}, \emph{Hauptstadt}|pw} kommſt, damit ich
               wenigſtens die Freude habe, Dich zu{ }ſehen. Auch habe ich die Abſicht, der N. Fr. Pr.\orgindex{Neue Freie Presse@Neue Freie Presse|pw} den \textsc{Dr. Erich Freund\pwindex{Freund, Erich 13.\,8.\,1866 Breslau – 1940 Berlin@\textsc{Freund, Erich} (13.\,8.\,1866 Breslau – 1940 Berlin), \emph{Kritiker, Musikjournalist}|pw}} in \textsc{Breslau}\oindex{Breslau@\textbf{Breslau}|pw}, den Du ja auch kennſt, {\pb}als Referenten
               vorzuſchlagen, damit wenigſtens ein anſtändiger und ehrlicher Kritiker über Dich
                  \label{K_L02937-2v}\edtext{berichtet}{\lemma{\textnormal{\emph{berichtet}}}\Cendnote{\textnormal{Siehe XXXX Auszeichnungsfehler: Dokument L02839 nicht gefunden und XXXX Auszeichnungsfehler: Dokument L02943 nicht gefunden.
               }}}\label{K_L02937-2}.\pend
           
\pstart
           Wann gedenkſt Du \label{K_L02937-3v}\edtext{nach Breslau\oindex{Breslau@\textbf{Breslau}|pw}}{\lemma{\textnormal{\emph{nach Breslau}}}\Cendnote{\textnormal{Schnitzler hielt sich vom 22. 11. 1900 bis zum 24. 11. 1900 und vom
                     29. 11. 1900 bis zum
                     2. 12. 1900 in
                     Breslau\oindex{Breslau@\textbf{Breslau}|pwk} auf. Dazwischen war er in Berlin\oindex{Berlin@\textbf{Berlin}, \emph{Hauptstadt}|pwk}.}}}\label{K_L02937-3} zu reiſen?\pend
           
\pstart
           Iſt es \strikeout{\textcolor{gray}{×}} wahr, daß \textsc{Wassermann\pwindex{Wassermann, Jakob 10.\,3.\,1873 Fürth – 1.\,1.\,1934 Altaussee@\textsc{Wassermann, Jakob} (10.\,3.\,1873 Fürth – 1.\,1.\,1934 Altaussee), \emph{Schriftsteller}|pw}}{ }ſich mit einem Frl. \textsc{Speier}\pwindex{Wassermann, Julie 5.\,12.\,1876 Wien – April 1963 Zürich@\textsc{Wassermann, Julie} (5.\,12.\,1876 Wien – April 1963 Zürich), \emph{Schriftstellerin}|pw}{ }\label{K_L02937-4v}\edtext{verlobt}{\lemma{\textnormal{\emph{verlobt}}}\Cendnote{\textnormal{Siehe A. S.: \emph{Tagebuch}, 11. 10. 1900.
               }}}\label{K_L02937-4} hat? Schön und reich?\pend
           
\pstart
           Welches iſt die Adreſſe der \label{K_L02937-5v}\edtext{Fräulein\pwindex{Schnitzler, Olga 17.\,1.\,1882 Wien – 13.\,1.\,1970 Lugano@\textsc{Schnitzler, Olga} (17.\,1.\,1882 Wien – 13.\,1.\,1970 Lugano), \emph{Schauspielerin, Sängerin}|pwv}\pwindex{Steinrück, Elisabeth 19.\,11.\,1885 – 7.\,4.\,1920 Partenkirchen@\textsc{Steinrück, Elisabeth} (19.\,11.\,1885 – 7.\,4.\,1920 Partenkirchen)|pwv} aus der Rothen-Stern-Gaſſe\oindex{Wien@\textbf{Wien}!II., Leopoldstadt@\textbf{II., Leopoldstadt}!Rotensterngasse@\textbf{Rotensterngasse}, \emph{Straße}|pw}}{\lemma{\textnormal{\emph{Fräulein … Rothen-Stern-Gasse}}}\Cendnote{\textnormal{Siehe XXXX Auszeichnungsfehler: Dokument L02931 nicht gefunden.
               }}}\label{K_L02937-5}?\pend
           
\pstart
           {\pb}Wann erſcheint der \label{K_L02937-6v}\edtext{»Lieutenant Guſtl\pwindex{Schnitzler, Arthur 15.\,5.\,1862 Wien – 21.\,10.\,1931 ebd.@\textsc{Schnitzler, Arthur} (15.\,5.\,1862 Wien – 21.\,10.\,1931 ebd.), \emph{Schriftsteller, Mediziner}!Lieutenant Gustl. Novelle@\strich\emph{Lieutenant Gustl. Novelle}|pw}«}{\lemma{\textnormal{\emph{»Lieutenant Gustl«}}}\Cendnote{\textnormal{Arthur Schnitzler: \emph{Lieutnant Gustl}\pwindex{Schnitzler, Arthur 15.\,5.\,1862 Wien – 21.\,10.\,1931 ebd.@\textsc{Schnitzler, Arthur} (15.\,5.\,1862 Wien – 21.\,10.\,1931 ebd.), \emph{Schriftsteller, Mediziner}!Lieutenant Gustl. Novelle@\strich\emph{Lieutenant Gustl. Novelle}|pwk}. In: \emph{Neue Freie Presse}\pwindex{Neue Freie Presse@\emph{Neue Freie Presse}|pwk}, Nr. 13.053, 25. 12. 1900, Morgenblatt, S. 34–41. Siehe auch A. S.: \emph{Tagebuch}, 25. 12. 1900.}}}\label{K_L02937-6}?\pend
           
\pstart
           Wie geht’s Dir{ }ſonſt? Frauen, Stimmung, Arbeit?\pend
           
\pstart
           Mein Leben iſt troſtlos öde, ohne auch nur einen Schimmer von Freude. Aber ich leſe
                  \textsc{E. T. A. Hoffmann\pwindex{Hoffmann, Ernst Theodor Amadeus 24.\,1.\,1776 Kaliningrad – 25.\,6.\,1822 Berlin@\textsc{Hoffmann, Ernst Theodor Amadeus} (24.\,1.\,1776 Kaliningrad – 25.\,6.\,1822 Berlin), \emph{Schriftsteller, Komponist, Zeichner}!E. T. A. Hoffmanns sämtliche Werke in fünfzehn Bänden@\strich\emph{E. T. A. Hoffmanns sämtliche Werke in fünfzehn Bänden}|pwv}\pwindex{Hoffmann, Ernst Theodor Amadeus 24.\,1.\,1776 Kaliningrad – 25.\,6.\,1822 Berlin@\textsc{Hoffmann, Ernst Theodor Amadeus} (24.\,1.\,1776 Kaliningrad – 25.\,6.\,1822 Berlin), \emph{Schriftsteller, Komponist, Zeichner}|pw}}. Bitte, \label{K_L02937-7v}\edtext{thue das auch}{\lemma{\textnormal{\emph{thue das auch}}}\Cendnote{\textnormal{\emph{E. T. A. Hoffmanns sämtliche Werke in fünfzehn
                        Bänden}\pwindex{Hoffmann, Ernst Theodor Amadeus 24.\,1.\,1776 Kaliningrad – 25.\,6.\,1822 Berlin@\textsc{Hoffmann, Ernst Theodor Amadeus} (24.\,1.\,1776 Kaliningrad – 25.\,6.\,1822 Berlin), \emph{Schriftsteller, Komponist, Zeichner}!E. T. A. Hoffmanns sämtliche Werke in fünfzehn Bänden@\strich\emph{E. T. A. Hoffmanns sämtliche Werke in fünfzehn Bänden}|pwk}. Herausgegeben von Eduard
                     Grisebach\pwindex{Grisebach, Eduard 9.\,10.\,1845 Göttingen – 22.\,3.\,1906 Charlottenburg@\textsc{Grisebach, Eduard} (9.\,10.\,1845 Göttingen – 22.\,3.\,1906 Charlottenburg), \emph{Schriftsteller, Diplomat, Jurist}|pwk}. Leipzig\oindex{Leipzig@\textbf{Leipzig}, \emph{Hauptstadt}|pwk}: \emph{Max Hesse}\orgindex{Max Hesses Verlag@Max Hesses Verlag|pwk}{ }1900. Eine neuerliche Lektüre des Werks von Hoffmann\pwindex{Hoffmann, Ernst Theodor Amadeus 24.\,1.\,1776 Kaliningrad – 25.\,6.\,1822 Berlin@\textsc{Hoffmann, Ernst Theodor Amadeus} (24.\,1.\,1776 Kaliningrad – 25.\,6.\,1822 Berlin), \emph{Schriftsteller, Komponist, Zeichner}|pwk} mit dieser Ausgabe\pwindex{Hoffmann, Ernst Theodor Amadeus 24.\,1.\,1776 Kaliningrad – 25.\,6.\,1822 Berlin@\textsc{Hoffmann, Ernst Theodor Amadeus} (24.\,1.\,1776 Kaliningrad – 25.\,6.\,1822 Berlin), \emph{Schriftsteller, Komponist, Zeichner}!E. T. A. Hoffmanns sämtliche Werke in fünfzehn Bänden@\strich\emph{E. T. A. Hoffmanns sämtliche Werke in fünfzehn Bänden}|pwkv} durch Schnitzler ist nicht
                  bekannt.}}}\label{K_L02937-7}! (Ausgabe\pwindex{Hoffmann, Ernst Theodor Amadeus 24.\,1.\,1776 Kaliningrad – 25.\,6.\,1822 Berlin@\textsc{Hoffmann, Ernst Theodor Amadeus} (24.\,1.\,1776 Kaliningrad – 25.\,6.\,1822 Berlin), \emph{Schriftsteller, Komponist, Zeichner}!E. T. A. Hoffmanns sämtliche Werke in fünfzehn Bänden@\strich\emph{E. T. A. Hoffmanns sämtliche Werke in fünfzehn Bänden}|pwv}
               von \textsc{Grisebach\pwindex{Grisebach, Eduard 9.\,10.\,1845 Göttingen – 22.\,3.\,1906 Charlottenburg@\textsc{Grisebach, Eduard} (9.\,10.\,1845 Göttingen – 22.\,3.\,1906 Charlottenburg), \emph{Schriftsteller, Diplomat, Jurist}|pw}}).\pend
           
\pstart
           \textsc{Richard\pwindex{Beer-Hofmann, Richard 11.\,7.\,1866 Wien – 26.\,9.\,1945 New York City@\textsc{Beer-Hofmann, Richard} (11.\,7.\,1866 Wien – 26.\,9.\,1945 New York City), \emph{Schriftsteller}|pw}} benimmt{ }ſich wieder einmal abſcheulich. {\pb}Antwortet mir nicht, \label{K_L02937-8v}\edtext{ſchickt mir
               nicht, worum ich ihn gebeten}{\lemma{\textnormal{\emph{schickt … gebeten}}}\Cendnote{\textnormal{Siehe XXXX Auszeichnungsfehler: Dokument L02931 nicht gefunden.
               }}}\label{K_L02937-8}. Rüttle ihn doch in meinem Namen etwas auf!\pend
           
\pstart
           \textsc{Kerr\pwindex{Kerr, Alfred 25.\,12.\,1867 Breslau – 12.\,10.\,1948 Hamburg@\textsc{Kerr, Alfred} (25.\,12.\,1867 Breslau – 12.\,10.\,1948 Hamburg), \emph{Schriftsteller, Kritiker}|pw}}{ }ſehe ich einmal im Monat auf fünf Minuten, die er jedesmal dazu benutzt, um mir
               zu erzählen, wie herrlich das Leben iſt.\pend
           
\pstart
           Grüß’ Dich Gott, liebſter Freund! In Treue {\\[\baselineskip]}Dein {\\[\baselineskip]}\spacefill\mbox{Paul Goldmann.}\pend
           \leftskip=0em{}\selectlanguage{ngerman}\endnumbering\briefempfaengerindex{Schnitzler, Arthur@\textsc{Schnitzler, Arthur}!zzzGoldmann, Paul@\emph{von Paul Goldmann}!1900-10-301@{30. 10. [1900]}|)be}\mylabel{L02937h}  \newcommand{\dateiname}{L02937}\newcommand{\titel}{Paul Goldmann an Arthur Schnitzler, 30. 10. [1900]}\newcommand{\editorInnen}{Martin Anton Müller und Laura Untner}%% latex-leseansicht-abspann.tex
%% Abspann für die Leseansicht.
%% Der Schalter \ifkorrekturansicht ist bereits durch den Vorspann gesetzt.

%% latex-abspann.tex
%% Gemeinsamer Abspann für Korrekturansicht und Leseansicht.
%% Setzt den Schalter \ifkorrekturansicht voraus (gesetzt in den
%% einbindenden Dateien latex-korrekturansicht-abspann.tex bzw.
%% latex-leseansicht-abspann.tex).
%% ---------------------------------------------------------------

\normalsize

% Das esempio-Environment wird nur in der Leseansicht benötigt
\ifkorrekturansicht\else
\newenvironment{esempio}[3]%
{
    \vspace{1.5ex}
    \rlap{\underline{#1}}
    \par
    \setlength{\parindent}{0cm}
    \nopagebreak
    \leftskip=#2cm
    \rightskip=#3cm
}
{
    \par
}
\fi

\doendnotes{C}
\bigskip
\vfill

\clearpage

\footnotesize

\ifkorrekturansicht
  \lohead{\textsc{register}}
\fi

% theindex-Environment neu definieren ohne reledmac
\makeatletter
\renewenvironment{theindex}{%
  \ifkorrekturansicht
    \section*{\indexname}%
  \else
    \subsubsection*{Index der erwähnten Entitäten}%
  \fi
  \setlength{\parindent}{0pt}%
  \setlength{\parskip}{0pt plus 0.3pt}%
  \let\item\@idxitem
}{%
  \ifkorrekturansicht\clearpage\fi
}
\makeatother

\IfFileExists{\jobname-pw.ind}{\input{\jobname-pw.ind}}{}

% Quellenangabe nur in der Leseansicht
\ifkorrekturansicht\else
% Fallback-Definitionen, falls die .tex-Datei \titel etc. nicht gesetzt hat
\providecommand{\titel}{}
\providecommand{\editorInnen}{}
\providecommand{\dateiname}{\jobname}

\vspace{3cm}

\vfill

\footnotesize
\textsc{Quelle}: \titel. Herausgegeben von {\editorInnen}. In: \emph{Arthur Schnitzler: Briefwechsel mit Autorinnen und Autoren}.
 Digitale Edition, https://schnitzler-briefe.acdh.oeaw.ac.at/{\dateiname}.html (Stand \today)
\fi

\end{document}


