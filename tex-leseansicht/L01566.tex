%% latex-korrekturansicht-vorspann.tex
%% Vorspann für die Korrekturansicht.
%% Lädt die gemeinsame Datei latex-vorspann.tex mit gesetztem Schalter.

\newif\ifkorrekturansicht
\korrekturansichttrue

\input{../tex-inputs/latex-vorspann}


\section[Max Burckhard: Widmungsexemplar Franz Stelzhamer und die oberösterreichische Dialektdichtung für Arthur Schnitzler, {[}27. 10. 1905?{]}]{L01566 Max Burckhard: Widmungsexemplar Franz Stelzhamer und die
               oberösterreichische Dialektdichtung für Arthur Schnitzler, {[}27. 10. 1905?{]}}
\nopagebreak\mylabel{L01566v}
\rehead{ }\normalsize\beginnumbering\briefempfaengerindex{Schnitzler, Arthur@\textsc{Schnitzler, Arthur}!zzzBurckhard, Max Eugen@\emph{von Max Eugen Burckhard}!1905-10-272@{{[}27. 10. 1905?{]}}|(be}
\toendnotes[C]{\smallbreak\pagebreak[2]}\Standort{DLA, G:Schnitzler, Arthur (Sammlung Heinrich Schnitzler).}
\physDesc{Widmung am Vorsatzblatt, 47 Zeichen
\newline{}Handschrift: schwarze Tinte, deutsche Kurrent}\toendnotes[C]{\smallbreak}
\pstart
           \noindent{}{\pb}S. l. Arthur Schnitzler\pend
           \pstart herzlichſt\spacefill\mbox{MaxBurckhard}\pend{}\selectlanguage{ngerman}\vspace{1em}{\vspace{1\baselineskip}}
\pstart
           \centering{}{\pb}\textcolor{gray}{\textbf{Franz Stelzhamer und die oberöſterreichiſche
                     Dialektdichtung\pwindex{Franz Stelzhamer und die oberoesterreichische Dialektdichtung@\emph{Franz Stelzhamer und die oberösterreichische Dialektdichtung}|pw}.}}\pend
           
\pstart
           \centering{}\textcolor{gray}{\textbf{\textsuperscript{von} Max Burckhart.}}\pend
           
\pstart
           \centering{}\textcolor{gray}{\textbf{Zeichnungen von Leop.
                     Forſtner\pwindex{Forstner, Leopold 1878-11-02 – 1936-11-05@\textsc{Forstner, Leopold} (1878-11-02 – 1936-11-05), \emph{Bildender Künstler/Bildende Künstlerin, Mosaizist/Mosaizistin}|pw}.}}\pend
           {\vspace{1\baselineskip}}
\pstart
           \centering{}\textcolor{gray}{\textbf{\label{K_L01566-1v}\edtext{Wiener Verlag\orgindex{Wiener Verlag@Wiener Verlag|pw}}{\lemma{\textnormal{\emph{Wiener Verlag}}}\Cendnote{\textnormal{Am 13. 11. 1905 wurde das Buch vom \emph{Börsenblatt für den deutschen Buchhandel}\pwindex{Boersenblatt fuer den Deutschen Buchhandel@\emph{Börsenblatt für den Deutschen Buchhandel}|pwk}
                     als Neuerscheinung gemeldet. Da zugleich \emph{Charakterbilder aus Oberoesterreich}\pwindex{Charakterbilder aus Oberoesterreich@\emph{Charakterbilder aus Oberösterreich}|pwk} erschienen war, ist es
                     naheliegend, dass Burckhard\pwindex{Burckhard, Max Eugen 14.07.1854 – 16.03.1912@\textsc{Burckhard, Max Eugen} (14.07.1854 – 16.03.1912), \emph{Schriftsteller/Schriftstellerin, Rechtswissenschaftler/Rechtswissenschaftlerin, Theaterleiter/Theaterleiterin}|pwk}{ }Schnitzler beide Bücher zusammen zukommen
                     ließ.}}}\label{K_L01566-1}{ }Wien\oindex{Wien@\textbf{Wien}, \emph{A.ADM2}|pw}{ }u.{ }Leipzig\oindex{Leipzig@\textbf{Leipzig}, \emph{P.PPLA3}|pw}.}}\pend
           \selectlanguage{ngerman}\endnumbering\briefempfaengerindex{Schnitzler, Arthur@\textsc{Schnitzler, Arthur}!zzzBurckhard, Max Eugen@\emph{von Max Eugen Burckhard}!1905-10-272@{{[}27. 10. 1905?{]}}|)be}\mylabel{L01566h}  \normalsize

\doendnotes{C}
\bigskip
\vfill

\clearpage

\footnotesize

\lohead{\textsc{register}}

% Definiere theindex-Environment komplett neu ohne reledmac
\makeatletter
\renewenvironment{theindex}{%
  \section*{\indexname}%
  \setlength{\parindent}{0pt}%
  \setlength{\parskip}{0pt plus 0.3pt}%
  \let\item\@idxitem
}{%
  \clearpage
}
\makeatother

\IfFileExists{\jobname-pw.ind}{\input{\jobname-pw.ind}}{}

\end{document}

      