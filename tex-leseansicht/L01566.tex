%% latex-leseansicht-vorspann.tex
%% Vorspann für die Leseansicht.
%% Lädt die gemeinsame Datei latex-vorspann.tex mit nicht gesetztem Schalter.

\newif\ifkorrekturansicht
\korrekturansichtfalse

\input{../tex-inputs/latex-vorspann}


               \section[Max Burckhard: Widmungsexemplar Franz Stelzhamer und die oberösterreichische Dialektdichtung für Arthur Schnitzler, {[}27. 10. 1905?{]}]{ Max Burckhard: Widmungsexemplar Franz Stelzhamer und die
                    oberösterreichische Dialektdichtung für Arthur Schnitzler, {[}27. 10. 1905?{]}}\nopagebreak\mylabel{v}\rehead{ }\begin{ledgroupsized}[t]{13cm}\normalsize\beginnumbering\briefempfaengerindex{Schnitzler, Arthur@\textsc{Schnitzler, Arthur}!zzzBurckhard, Max Eugen@\emph{von Max Eugen Burckhard}!1905-10-272@{{[}27. 10. 1905?{]}}|(be} \toendnotes[C]{\smallbreak\pagebreak[2]} \Standort{DLA, G:Schnitzler, Arthur (Sammlung Heinrich Schnitzler).}
\physDesc{Widmung am Vorsatzblatt
\newline{}Handschrift: schwarze Tinte, deutsche Kurrent}\toendnotes[C]{\smallbreak}\pstart
           \noindent{}{\pb}S. l. Arthur Schnitzler\pend
           \pstart herzlichſt\spacefill\mbox{MaxBurckhard}\pend{}{\bigskip}\pstart
           \noindent{}\centering{}{\pb}\textcolor{gray}{\textbf{Franz Stelzhamer und die oberöſterreichiſche
                            Dialektdichtung\pwindex{Burckhard, Max Eugen 14.07.1854 – 16.03.1912@\textsc{Burckhard, Max Eugen} (14.07.1854 – 16.03.1912), \emph{Schriftsteller, Rechtswissenschaftler, Theaterleiter}!Franz Stelzhamer und die oberoesterreichische Dialektdichtung1906 – 1906@\strich\emph{Franz Stelzhamer und die oberösterreichische Dialektdichtung} {[}1906 – 1906{]}|pw}.}}\pend
           \pstart
           \noindent{}\centering{}\textcolor{gray}{\textbf{\textsuperscript{von} Max Burckhart.}}\pend
           \pstart
           \noindent{}\centering{}\textcolor{gray}{\textbf{Zeichnungen von Leop.
                            Forſtner\pwindex{Forstner, Leopold 1878-11-02 – 1936-11-05@\textsc{Forstner, Leopold} (1878-11-02 – 1936-11-05), \emph{Bildender Künstler, Mosaizist}|pw}.}}\pend
           {\bigskip}\pstart
           \noindent{}\centering{}\textcolor{gray}{\textbf{\label{K_L01566_1v}\edtext{Wiener Verlag\orgindex{Wiener Verlag@Wiener Verlag|pw}}{\lemma{\textnormal{\emph{Wiener Verlag}}}\Cendnote{\textnormal{am 13. 11. 1905 vom \emph{Börsenblatt für den deutschen
                     Buchhandel}\pwindex{Boersenblatt fuer den deutschen Buchhandel1843-01-03@\emph{Börsenblatt für den deutschen Buchhandel}|pwk} als Neuerscheinung gemeldet. Da
                            zugleich auch das Buch \emph{Charakterbilder aus
                                Oberoesterreich}\pwindex{\textcolor{red}{\textsuperscript{XXXX1 indx}}!Charakterbilder aus Oberoesterreich1905-11-13 – 1905-11-13@\strich\emph{Charakterbilder aus Oberösterreich} {[}1905-11-13 – 1905-11-13{]}|pwk} erschien, ist es naheliegend, dass Burckhard\pwindex{Burckhard, Max Eugen 14.07.1854 – 16.03.1912@\textsc{Burckhard, Max Eugen} (14.07.1854 – 16.03.1912), \emph{Schriftsteller, Rechtswissenschaftler, Theaterleiter}|pwk}{ }Schnitzler\pwindex{Schnitzler, Arthur 15.05.1862 – 21.10.1931@\textsc{Schnitzler, Arthur} (15.05.1862 – 21.10.1931), \emph{Schriftsteller, Mediziner}|pwk} beide Bücher zugleich
                            zukommen ließ.}}}\label{K_L01566_1h}{ }Wien\oindex{Wien@\textbf{Wien}|pw}{ }u.{ }Leipzig\oindex{Leipzig@\textbf{Leipzig}|pw}.}}\pend
           \endnumbering\briefempfaengerindex{Schnitzler, Arthur@\textsc{Schnitzler, Arthur}!zzzBurckhard, Max Eugen@\emph{von Max Eugen Burckhard}!1905-10-272@{{[}27. 10. 1905?{]}}|)be}\mylabel{h}\end{ledgroupsized}  \newcommand{\dateiname}{L01566}\newcommand{\titel}{Max Burckhard: Widmungsexemplar Franz Stelzhamer und die oberösterreichische Dialektdichtung für Arthur Schnitzler, [27. 10. 1905?]}\newcommand{\editorInnen}{Martin Anton Müller und Gerd-Hermann Susen}%% latex-leseansicht-abspann.tex
%% Abspann für die Leseansicht.
%% Der Schalter \ifkorrekturansicht ist bereits durch den Vorspann gesetzt.

%% latex-abspann.tex
%% Gemeinsamer Abspann für Korrekturansicht und Leseansicht.
%% Setzt den Schalter \ifkorrekturansicht voraus (gesetzt in den
%% einbindenden Dateien latex-korrekturansicht-abspann.tex bzw.
%% latex-leseansicht-abspann.tex).
%% ---------------------------------------------------------------

\normalsize

% Das esempio-Environment wird nur in der Leseansicht benötigt
\ifkorrekturansicht\else
\newenvironment{esempio}[3]%
{
    \vspace{1.5ex}
    \rlap{\underline{#1}}
    \par
    \setlength{\parindent}{0cm}
    \nopagebreak
    \leftskip=#2cm
    \rightskip=#3cm
}
{
    \par
}
\fi

\doendnotes{C}
\bigskip
\vfill

\clearpage

\footnotesize

\ifkorrekturansicht
  \lohead{\textsc{register}}
\fi

% theindex-Environment neu definieren ohne reledmac
\makeatletter
\renewenvironment{theindex}{%
  \ifkorrekturansicht
    \section*{\indexname}%
  \else
    \subsubsection*{Index der erwähnten Entitäten}%
  \fi
  \setlength{\parindent}{0pt}%
  \setlength{\parskip}{0pt plus 0.3pt}%
  \let\item\@idxitem
}{%
  \ifkorrekturansicht\clearpage\fi
}
\makeatother

\IfFileExists{\jobname-pw.ind}{\input{\jobname-pw.ind}}{}

% Quellenangabe nur in der Leseansicht
\ifkorrekturansicht\else
% Fallback-Definitionen, falls die .tex-Datei \titel etc. nicht gesetzt hat
\providecommand{\titel}{}
\providecommand{\editorInnen}{}
\providecommand{\dateiname}{\jobname}

\vspace{3cm}

\vfill

\footnotesize
\textsc{Quelle}: \titel. Herausgegeben von {\editorInnen}. In: \emph{Arthur Schnitzler: Briefwechsel mit Autorinnen und Autoren}.
 Digitale Edition, https://schnitzler-briefe.acdh.oeaw.ac.at/{\dateiname}.html (Stand \today)
\fi

\end{document}


      