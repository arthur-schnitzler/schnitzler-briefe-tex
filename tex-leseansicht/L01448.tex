%% latex-korrekturansicht-vorspann.tex
%% Vorspann für die Korrekturansicht.
%% Lädt die gemeinsame Datei latex-vorspann.tex mit gesetztem Schalter.

\newif\ifkorrekturansicht
\korrekturansichttrue

\input{../tex-inputs/latex-vorspann}


\section[Hugo von Hofmannsthal an Arthur Schnitzler, 22. 9. 1904]{L01448 Hugo von Hofmannsthal an Arthur Schnitzler, 22. 9. 1904}
\nopagebreak\mylabel{L01448v}
\rehead{ }\normalsize\beginnumbering\briefempfaengerindex{Schnitzler, Arthur@\textsc{Schnitzler, Arthur}!zzzHofmannsthal, Hugo von@\emph{von Hugo von Hofmannsthal}!1904-09-221@{22. 9. 1904}|(be}
\toendnotes[C]{\smallbreak\pagebreak[2]}\Standort{CUL, Schnitzler, B 43.}
\physDesc{Postkarte, 471 Zeichen
\newline{}Handschrift: 1) schwarze Tinte, deutsche Kurrent\hspace{1em}2) schwarze Tinte, lateinische Kurrent (\noindent{}Adresse)\hspace{1em}
\newline{}Versand: 1) Stempel: »\nobreak{}\oindex{Stazione di Venezia Santa Lucia@\textbf{Stazione di Venezia Santa Lucia}, \emph{Bahnhofsgebäude (K.BHF)}|pwk}Venezia {[}Ferrovia{]}, 22 9 \textcolor{gray}{0}4 , 10S\nobreak{}«.   2) Stempel: »\nobreak{}\oindex{XVIII., Waehring@\textbf{XVIII., Währing}, \emph{A.ADM3}|pwk}18/1 Wien 110, 24. 9. 04, 2.V, Bestellt\nobreak{}«. 
\newline{}Schnitzler: mit Bleistift Monat und Jahreszahl ergänzt: »/9 904« 
\newline{}Ordnung: 1) mit Bleistift von unbekannter Hand nummeriert:
                                    »224«  2) mit Bleistift von unbekannter Hand nummeriert:
                                    »255«}
\buchAbdrucke{\weitereDrucke{Hugo von Hofmannsthal, Arthur Schnitzler: \emph{Briefwechsel}. Frankfurt am Main: \emph{S. Fischer} 1964, S. 202.} }\toendnotes[C]{\smallbreak}\pstart{}{\pb}D\textsuperscript{r}
                  Arthur Schnitzler\pend{}\pstart{}Wien\oindex{Wien@\textbf{Wien}, \emph{A.ADM2}|pw}\pend{}\pstart{}XVIII Spöttelgasse 7\oindex{Edmund-Weiss-Gasse 7@\textbf{Edmund-Weiß-Gasse 7}, \emph{Wohngebäude (K.WHS)}|pw}\pend{}\pstart{}Austria\oindex{Oesterreich@\textbf{Österreich}, \emph{A.PCLI}|pw}\pend{}{\bigskip}\vspace{1em}
\pstart
           \raggedleft{}{\pb}22.\pend
           \vspace{0.5em}
\pstart
           lieber, bin wohl und recht fleißig, bei hellem aber ſehr kühlem
               Wetter. Bitte vielmals ſchicken Sie mir recht bald hieher – ich habe in den
               Abendſtunden gar nichts zu leſen – womöglich: \textsc{H. Mann\pwindex{Mann, Heinrich 27.03.1871 – 11.03.1950@\textsc{Mann, Heinrich} (27.03.1871 – 11.03.1950), \emph{Schriftsteller/Schriftstellerin}|pw}, Herzogin\pwindex{Goettinnen oder Die drei Romane der Herzogin von Assy@\emph{Die Göttinnen oder Die drei Romane der Herzogin von Assy}|pw}}, I u. II (\textsc{Bd III Venus\pwindex{Goettinnen oder Die drei Romane der Herzogin von Assy@\emph{Die Göttinnen oder Die drei Romane der Herzogin von Assy}|pw}} habe ich) und das Heft der Zukunft\pwindex{Zukunft@\emph{Die Zukunft}|pw}, worin
                  \textsc{H}.\pwindex{Mann, Heinrich 27.03.1871 – 11.03.1950@\textsc{Mann, Heinrich} (27.03.1871 – 11.03.1950), \emph{Schriftsteller/Schriftstellerin}|pw} über \textsc{Elektra}\pwindex{Elektra. Tragoedie in einem Aufzug@\emph{Elektra. Tragödie in einem Aufzug}|pw}{ }ſchrieb\pwindex{Elektra@\emph{Elektra}|pwv}. Wenn das nicht
               möglich, ſo vielleicht »\textsc{Jagd nach Liebe}\pwindex{Jagd nach Liebe@\emph{Die Jagd nach Liebe}|pw}«. Voraus dankend, von Herzen\pend
           \pstart \spacefill\mbox{Hugo.}\pend{}
\pstart
           \noindent{}\label{T_L01448-1v}\edtext{\textsc{P. S.} Eben ko{\geminationm}t die »Zukunft\pwindex{Zukunft@\emph{Die Zukunft}|pw}«, alſo die nicht.}{\lemma{\textnormal{\emph{P. S. … nicht.}}}\Cendnote{\textnormal{quer am rechten Rand}}}\label{T_L01448-1}\pend
           \selectlanguage{ngerman}\endnumbering\briefempfaengerindex{Schnitzler, Arthur@\textsc{Schnitzler, Arthur}!zzzHofmannsthal, Hugo von@\emph{von Hugo von Hofmannsthal}!1904-09-221@{22. 9. 1904}|)be}\mylabel{L01448h}  \normalsize

\doendnotes{C}
\bigskip
\vfill

\clearpage

\footnotesize

\lohead{\textsc{register}}

% Definiere theindex-Environment komplett neu ohne reledmac
\makeatletter
\renewenvironment{theindex}{%
  \section*{\indexname}%
  \setlength{\parindent}{0pt}%
  \setlength{\parskip}{0pt plus 0.3pt}%
  \let\item\@idxitem
}{%
  \clearpage
}
\makeatother

\IfFileExists{\jobname-pw.ind}{\input{\jobname-pw.ind}}{}

\end{document}

      