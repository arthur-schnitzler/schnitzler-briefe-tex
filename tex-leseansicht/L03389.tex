%% latex-leseansicht-vorspann.tex
%% Vorspann für die Leseansicht.
%% Lädt die gemeinsame Datei latex-vorspann.tex mit nicht gesetztem Schalter.

\newif\ifkorrekturansicht
\korrekturansichtfalse

\input{../tex-inputs/latex-vorspann}

\begin{center}
            \textcolor{red}{ENTWURF, NICHT FERTIG KORRIGIERT}
                      \end{center}
            
         
         \renewcommand{\erwaehntePersonen}{Personen: Hermann Bahr, Ludwig Bauer, Richard Beer-Hofmann, Emilie Dorothea Popper, Theodore Rottenberg, Olga Schnitzler, Heinrich Schnitzler, Louise Schnitzler, Karl Eduard Vehse, ?? [Berliner Musikkorrespondent der National-Zeitung]}
         \renewcommand{\erwaehnteInstitutionen}{Institutionen: National-Zeitung, Reichstag, Verein zur Förderung der Künste}
         \renewcommand{\erwaehnteOrte}{Orte: Berlin, Dessauer Straße, Frankfurt am Main, Semmering, Wien, Ägypten}
         \renewcommand{\erwaehnteWerke}{Werke: Briefe, die ihn nicht erreichten, Der einsame Weg. Schauspiel in fünf Akten, Die Gouvernante, Neue Freie Presse, Theater- und Kunstnachrichten. [Konzerte.]. [Man schreibt uns aus Berlin], [Aus Berlin wird uns gemeldet: »Der einsame Weg«]}
               \section[ Paul Goldmann an Arthur Schnitzler, 13. 12. {[}1903{]}]{ Paul Goldmann an Arthur Schnitzler, 13. 12. {[}1903{]}}\nopagebreak\mylabel{v}\rehead{ }\begin{ledgroupsized}[t]{13cm}\normalsize\beginnumbering \toendnotes[C]{\smallbreak\pagebreak[2]} \Standort{DLA, A:Schnitzler, HS.NZ85.1.3173.}
\physDesc{Brief, 2 Blätter, 7 Seiten
\newline{}Handschrift: blaue Tinte, deutsche Kurrent
\newline{}Schnitzler: 1) mit Bleistift das Jahr »{[}1{]}903« vermerkt  2) mit rotem Buntstift neun Unterstreichungen}\toendnotes[C]{\smallbreak}\pstart
           \noindent{}\raggedleft{}{\pb}\textcolor{gray}{\textbf{DESSAUERSTRASSE 19\oindex{Dessauer Strasse@\textbf{Dessauer Straße}|pw}}}\pend
           \pstart
           Berlin\oindex{Berlin@\textbf{Berlin}|pw}, 13. Dezember.\pend
           \pstart\center{}Mein lieber Freund,\pend\pstart
           Ich habe mich ſehr gefreut, wieder einmal einen Brief von Dir zu erhalten. Auch die
               guten Nachrichten über Deine »engſte Familie\pwindex{Schnitzler, Olga 17.01.1882 – 13.01.1970@\textsc{Schnitzler, Olga} (17.01.1882 – 13.01.1970), \emph{Schauspielerin, Sängerin}|pwv}\pwindex{Schnitzler, Heinrich 09.08.1902 – 12.07.1982@\textsc{Schnitzler, Heinrich} (09.08.1902 – 12.07.1982), \emph{Regisseur, Schauspieler}|pwv}« haben mir viel Freude bereitet.\pend
           \pstart
           Daß ich \introOben{}für\introOben{}{ }\label{K_L03389-1v}\edtext{Fräulein \textsc{Popper\pwindex{Popper, Emilie Dorothea 1893-10-08 – 1933-11-24@\textsc{Popper, Emilie Dorothea} (1893-10-08 – 1933-11-24), \emph{Pianistin, Pädagogin}|pw}}}{\lemma{\textnormal{\emph{Fräulein Popper}}}\Cendnote{\textnormal{siehe Paul Goldmann an Arthur Schnitzler, 14. 11. [1903]}}}\label{K_L03389-1h}, nachdem ſie mir von Dir und Deiner Mutter\pwindex{Schnitzler, Louise 1840-07-08 – 1911-09-09@\textsc{Schnitzler, Louise} (1840-07-08 – 1911-09-09)|pwv} empfohlen worden, Alles that, was in meiner Macht
               ſtand, iſt ſelbſtverſtändlich. Wenn Du ſie ſiehſt, ſo ſage ihr, daß der Referent\pwindex{[Berliner Musikkorrespondent der National-Zeitung], ?? @\textsc{[Berliner Musikkorrespondent der National-Zeitung], ??}, \emph{Journalist}|pwv} der »Nationalzeitung\orgindex{National-Zeitung@National-Zeitung|pw}«, an den ich ſie empfohlen, ſehr freundlich
               über ſie \label{K_L03389-2v}\edtext{geſchrieben\pwindex{Theater- und Kunstnachrichten. [Konzerte.]. [Man schreibt uns aus Berlin]1903-11-20@\emph{Theater- und Kunstnachrichten. [Konzerte.]. [Man schreibt uns aus Berlin]} {[}1903-11-20{]}|pwv}}{\lemma{\textnormal{\emph{geſchrieben}}}\Cendnote{\textnormal{Höchstwahrscheinlich Bezug auf folgende
                     Meldung\pwindex{Theater- und Kunstnachrichten. [Konzerte.]. [Man schreibt uns aus Berlin]1903-11-20@\emph{Theater- und Kunstnachrichten. [Konzerte.]. [Man schreibt uns aus Berlin]} {[}1903-11-20{]}|pwkv} über ein
                  Konzert von Dora Popper\pwindex{Popper, Emilie Dorothea 1893-10-08 – 1933-11-24@\textsc{Popper, Emilie Dorothea} (1893-10-08 – 1933-11-24), \emph{Pianistin, Pädagogin}|pwk}: [Berliner Musikkorrespondent der
                        National-Zeitung\pwindex{[Berliner Musikkorrespondent der National-Zeitung], ?? @\textsc{[Berliner Musikkorrespondent der National-Zeitung], ??}, \emph{Journalist}|pwk}:] \emph{Theater- und
                        Kunstnachrichten. [Konzerte.]. [Man schreibt uns aus Berlin]}\pwindex{Theater- und Kunstnachrichten. [Konzerte.]. [Man schreibt uns aus Berlin]1903-11-20@\emph{Theater- und Kunstnachrichten. [Konzerte.]. [Man schreibt uns aus Berlin]} {[}1903-11-20{]}|pwk}. In: \emph{Neue Freie Presse}\pwindex{Neue Freie Presse1864 – 1939@\emph{Neue Freie Presse} {[}1864 – 1939{]}|pwk}, Nr. 14093, 20. 11. 1903, S. 9.}}}\label{K_L03389-2h} hat.\pend
           \pstart
           {\pb}Am \label{K_L03389-3v}\edtext{Semmering\oindex{Semmering@\textbf{Semmering}|pw}}{\lemma{\textnormal{\emph{Semmering}}}\Cendnote{\textnormal{Arthur\pwindex{Schnitzler, Arthur 15.05.1862 – 21.10.1931@\textsc{Schnitzler, Arthur} (15.05.1862 – 21.10.1931), \emph{Schriftsteller, Mediziner}|pwk} und Olga Schnitzler\pwindex{Schnitzler, Olga 17.01.1882 – 13.01.1970@\textsc{Schnitzler, Olga} (17.01.1882 – 13.01.1970), \emph{Schauspielerin, Sängerin}|pwk} waren zwischen 6. 11. 1903 und 9. 11. 1903 am Semmering\oindex{Semmering@\textbf{Semmering}|pwk} gewesen.}}}\label{K_L03389-3h} muß es im Spätherbſt ſehr ſchön geweſen ſein.
               Haſt Du weitere Winter-Reiſepläne? Über die Vorleſung Deines Stück\pwindex{Schnitzler, Arthur 15.05.1862 – 21.10.1931@\textsc{Schnitzler, Arthur} (15.05.1862 – 21.10.1931), \emph{Schriftsteller, Mediziner}!Gouvernante2. 12. 1903@\strich\emph{Die Gouvernante} {[}2. 12. 1903{]}|pwv}es durch \textsc{Ludwig Bauer\pwindex{Bauer, Ludwig 05.09.1876 – 01.02.1935@\textsc{Bauer, Ludwig} (05.09.1876 – 01.02.1935), \emph{Schriftsteller, Journalist}|pw}} habe ich ſelbſtverſtändlich ein \label{K_L03389-4v}\edtext{Telegramm}{\lemma{\textnormal{\emph{Telegramm}}}\Cendnote{\textnormal{Ludwig Bauer\pwindex{Bauer, Ludwig 05.09.1876 – 01.02.1935@\textsc{Bauer, Ludwig} (05.09.1876 – 01.02.1935), \emph{Schriftsteller, Journalist}|pwk}s Vorlesung von \emph{Die Gouvernante}\pwindex{Schnitzler, Arthur 15.05.1862 – 21.10.1931@\textsc{Schnitzler, Arthur} (15.05.1862 – 21.10.1931), \emph{Schriftsteller, Mediziner}!Gouvernante2. 12. 1903@\strich\emph{Die Gouvernante} {[}2. 12. 1903{]}|pwk} fand am 2. 12. 1903 in Berlin\oindex{Berlin@\textbf{Berlin}|pwk} statt und
                  wurde vom \emph{Verein zur Förderung der Künste}\orgindex{Verein zur Foerderung der Kuenste@Verein zur Förderung der Künste|pwk}
                  veranstaltet. Siehe auch A. S.: \emph{Tagebuch}, 4. 12. 1903.
                     Goldmann\pwindex{Goldmann, Paul 31.01.1865 – 25.09.1935@\textsc{Goldmann, Paul} (31.01.1865 – 25.09.1935), \emph{Schriftsteller, Journalist}|pwk}s Telegramm dürfte tatsächlich
                  nicht veröffentlicht worden sein.}}}\label{K_L03389-4h} geſandt. Es iſt nicht erſchienen (oder
               ſollte es mir entgangen ſein?) Dieſes Nichterſcheinen richtet ſich aber ſicherlich
               gegen \textsc{Bauer\pwindex{Bauer, Ludwig 05.09.1876 – 01.02.1935@\textsc{Bauer, Ludwig} (05.09.1876 – 01.02.1935), \emph{Schriftsteller, Journalist}|pw}} und nicht gegen Dich. Mein \label{K_L03389-5v}\edtext{Telegramm\pwindex{Aus Berlin wird uns gemeldet: »Der einsame Weg«]1903-12-12@\emph{[Aus Berlin wird uns gemeldet: »Der einsame Weg«]} {[}1903-12-12{]}|pwv}}{\lemma{\textnormal{\emph{Telegramm}}}\Cendnote{\textnormal{[Paul Goldmann\pwindex{Goldmann, Paul 31.01.1865 – 25.09.1935@\textsc{Goldmann, Paul} (31.01.1865 – 25.09.1935), \emph{Schriftsteller, Journalist}|pwk}:] \emph{[Aus Berlin wird uns gemeldet: »Der einsame Weg«]}\pwindex{Aus Berlin wird uns gemeldet: »Der einsame Weg«]1903-12-12@\emph{[Aus Berlin wird uns gemeldet: »Der einsame Weg«]} {[}1903-12-12{]}|pwk}. In:
                        \emph{Neue Freie Presse}\pwindex{Neue Freie Presse1864 – 1939@\emph{Neue Freie Presse} {[}1864 – 1939{]}|pwk}, Nr. 14115, 12. 12. 1903, Morgenblatt, S. 10.}}}\label{K_L03389-5h}
               über das Bevorſtehen Deiner \textsc{\begin{otherlanguage}{french}Première\pwindex{Schnitzler, Arthur 15.05.1862 – 21.10.1931@\textsc{Schnitzler, Arthur} (15.05.1862 – 21.10.1931), \emph{Schriftsteller, Mediziner}!einsame Weg. Schauspiel in fuenf Akten1904@\strich\emph{Der einsame Weg. Schauspiel in fünf Akten} {[}1904{]}|pwv}\end{otherlanguage}} iſt ja erſchienen.\pend
           \pstart
           Zum Leſen komme ich gar nicht mehr, ſeit die furchtbare Reichstag\orgindex{Reichstag@Reichstag|pw}sarbeit begonnen hat.\label{K_L03389-6v}\edtext{\textsc{Vehse\pwindex{Vehse, Karl Eduard 1802-12-18 – 1870-06-18@\textsc{Vehse, Karl Eduard} (1802-12-18 – 1870-06-18), \emph{Historiker}|pw}}}{\lemma{\textnormal{\emph{Vehse}}}\Cendnote{\textnormal{Werk nicht ermittelt}}}\label{K_L03389-6h} habe ich
               habe ich mir gekauft (für 67 \textsc{MK}; was haſt Du gezahlt?).
               Haſt Du das gegenwärtige {\pb}deutſche Modebuch \label{K_L03389-7v}\edtext{»Brieſe\textcolor{gray}{,} die ihn nicht erreichten\pwindex{Briefe, die ihn nicht erreichten1903@\emph{Briefe, die ihn nicht erreichten} {[}1903{]}|pw}«}{\lemma{\textnormal{\emph{»Brieſe, … erreichten«}}}\Cendnote{\textnormal{Schnitzler\pwindex{Schnitzler, Arthur 15.05.1862 – 21.10.1931@\textsc{Schnitzler, Arthur} (15.05.1862 – 21.10.1931), \emph{Schriftsteller, Mediziner}|pwk} hatte den Briefroman\pwindex{Briefe, die ihn nicht erreichten1903@\emph{Briefe, die ihn nicht erreichten} {[}1903{]}|pwkv} nicht gelesen, siehe Paul Goldmann an Arthur Schnitzler, 27. 6. [1903].}}}\label{K_L03389-7h} ſchon geleſen?
               Es iſt zu empfehlen.\pend
           \pstart
           Meine Freundin\pwindex{Rottenberg, Theodore 1875-09-07 – 1945-04-05@\textsc{Rottenberg, Theodore} (1875-09-07 – 1945-04-05)|pwv} in Frankfurt\oindex{Frankfurt am Main@\textbf{Frankfurt am Main}|pw} war krank. Lungenentzündung oder ſo
               etwas. Ich bin ſehr beſorgt. Aus ihren Briefen werde ich nicht recht klug inbezug auf
               ihre Krankheit. Die Ärzte ſagen ihr auch offenbar nicht die Wahrheit; aber aus dem
               Umſtande, daß die Ärzte eine ſofortige Reiſe nach dem Süden, womöglichEgypten\oindex{Aegypten@\textbf{Ägypten}|pw}, empfehlen, folgere ich allerlei
               Schlimmes.\pend
           \pstart
           Als ich \label{K_L03389-8v}\edtext{das letzte Mal in Wien\oindex{Wien@\textbf{Wien}|pw}}{\lemma{\textnormal{\emph{das letzte Mal in Wien}}}\Cendnote{\textnormal{vermutlich Ende September/Anfang Oktober 1903, siehe Paul Goldmann an Arthur Schnitzler, 7. 9. 1903}}}\label{K_L03389-8h} mit Dir und Deiner Frau\pwindex{Schnitzler, Olga 17.01.1882 – 13.01.1970@\textsc{Schnitzler, Olga} (17.01.1882 – 13.01.1970), \emph{Schauspielerin, Sängerin}|pwv} über dieſe Angelegenheit ſprach, ſagteſt Du, daß ich eigentlich nunmehr
               gegen \strikeout{die} meine Freundin\pwindex{Rottenberg, Theodore 1875-09-07 – 1945-04-05@\textsc{Rottenberg, Theodore} (1875-09-07 – 1945-04-05)|pwv} ſei, indem ich ſie in {\pb}der \label{K_L03389-9v}\edtext{Illuſion}{\lemma{\textnormal{\emph{Illuſion}}}\Cendnote{\textnormal{siehe Paul Goldmann an Arthur Schnitzler, 14. 11. [1903]}}}\label{K_L03389-9h} ließe, ich würde ſie\pwindex{Rottenberg, Theodore 1875-09-07 – 1945-04-05@\textsc{Rottenberg, Theodore} (1875-09-07 – 1945-04-05)|pwv}
               heirathen. Ich habe über dieſe Deine Worte oft nachgedacht. \strikeout{\textcolor{gray}{D}} Du haſt im Weſentlichen Recht; und da mich der Vorwurf der Unwahrheit ſehr
               bedrückt, bin ich ſeit Wochen bemüht, in meinen Briefen allmälig zur Wahrheit
               einzulenken. Sie weiß heut, daß ich ſie, fürs Erſte wenigſtens, nicht heirathen kann;
               aber ſie klammert ſich trotzdem an mich als \strikeout{ih\textcolor{gray}{ren}} denjenigen, der ſie, wie ſie ſchreibt, »vom Abgrund zurückgeriſſen hat« und
               als ihren einzigen Halt.\pend
           \pstart
           Was aus Alledem werden ſoll, weiß der liebe Gott allein.\pend
           \pstart
           Das Unglück wollte es, daß {\pb}\strikeout{daß} ich \textsc{Bahr\pwindex{Bahr, Hermann 19.07.1863 – 15.01.1934@\textsc{Bahr, Hermann} (19.07.1863 – 15.01.1934), \emph{Schriftsteller, Kritiker}|pw}}, nachdem ich das Glück gehabt hatte, \strikeout{wahrſ\textcolor{gray}{e}} während ſeines Berlin\oindex{Berlin@\textbf{Berlin}|pw}er Aufenthalts
               nigends mit ihm zuſammen zukommen, \introOben{}geſtern\introOben{} auf der Straße traf. Ich blieb ſtehen, und wir geriethen in ein längeres
               Geſpräch. Dieſer alberne, dünkelhafte und verlogene Menſch hat \strikeout{mich} mich immer heftig gereizt. Diesmal war dies ganz
               beſonders der Fall, und er ſchien es auch darauf angelegt zu haben, mich zu
               provoziren. So theilte er mir Äußerungen mit, die Du und \textsc{Beer-Hofmann\pwindex{Beer-Hofmann, Richard 1866-07-11 – 1945-09-26@\textsc{Beer-Hofmann, Richard} (1866-07-11 – 1945-09-26), \emph{Schriftsteller}|pw}} gethan haben ſollen. Ich gerieth in Hitze und antwortete {\pb}demgemäß. Hinterher wurde es mir klar, daß Deine und
                  \textsc{Richard\pwindex{Beer-Hofmann, Richard 1866-07-11 – 1945-09-26@\textsc{Beer-Hofmann, Richard} (1866-07-11 – 1945-09-26), \emph{Schriftsteller}|pw}s} Äußerungen offenbar entſtellt
               wiedergegeben waren. Ich vermuthe, daß er\pwindex{Bahr, Hermann 19.07.1863 – 15.01.1934@\textsc{Bahr, Hermann} (19.07.1863 – 15.01.1934), \emph{Schriftsteller, Kritiker}|pwv} Dir jetzt auch meine Äußerungen entſtellt \label{K_L03389-11v}\edtext{berichten}{\lemma{\textnormal{\emph{berichten}}}\Cendnote{\textnormal{Schnitzler\pwindex{Schnitzler, Arthur 15.05.1862 – 21.10.1931@\textsc{Schnitzler, Arthur} (15.05.1862 – 21.10.1931), \emph{Schriftsteller, Mediziner}|pwk} und Bahr\pwindex{Bahr, Hermann 19.07.1863 – 15.01.1934@\textsc{Bahr, Hermann} (19.07.1863 – 15.01.1934), \emph{Schriftsteller, Kritiker}|pwk} sprachen jedenfalls kurz darauf über Goldmann\pwindex{Goldmann, Paul 31.01.1865 – 25.09.1935@\textsc{Goldmann, Paul} (31.01.1865 – 25.09.1935), \emph{Schriftsteller, Journalist}|pwk}, vgl. A. S.: \emph{Tagebuch}, 18. 12. 1903 und Bahr/Schnitzler, D041448.}}}\label{K_L03389-11h} wird, und bitte Dich, falls dies geſchehen ſollte,
               nicht darauf zu achten.\pend
           \pstart
           Wenn Du nächſtens einmal wieder Zeit findeſt, mir zu ſchreiben, wirſt Du mir eine
               große Freude machen. Weihnachten gehe ich
               wahrſcheinlich nach {\pb}Frankfurt\oindex{Frankfurt am Main@\textbf{Frankfurt am Main}|pw}.\pend
           \pstart
           Viele herzliche Grüße an Dich und Deine Frau\pwindex{Schnitzler, Olga 17.01.1882 – 13.01.1970@\textsc{Schnitzler, Olga} (17.01.1882 – 13.01.1970), \emph{Schauspielerin, Sängerin}|pwv} von Deinem getreuen {\\[\baselineskip]}\spacefill\mbox{Paul Goldmann.}\pend
           \leftskip=0em{}
         
         \endnumbering\mylabel{h}\end{ledgroupsized}\begin{anhang}\end{anhang}\newcommand{\dateiname}{L03389}\newcommand{\titel}{Paul Goldmann an Arthur Schnitzler, 13. 12. [1903]}\newcommand{\editorInnen}{Martin Anton Müller und Laura Untner}%% latex-leseansicht-abspann.tex
%% Abspann für die Leseansicht.
%% Der Schalter \ifkorrekturansicht ist bereits durch den Vorspann gesetzt.

%% latex-abspann.tex
%% Gemeinsamer Abspann für Korrekturansicht und Leseansicht.
%% Setzt den Schalter \ifkorrekturansicht voraus (gesetzt in den
%% einbindenden Dateien latex-korrekturansicht-abspann.tex bzw.
%% latex-leseansicht-abspann.tex).
%% ---------------------------------------------------------------

\normalsize

% Das esempio-Environment wird nur in der Leseansicht benötigt
\ifkorrekturansicht\else
\newenvironment{esempio}[3]%
{
    \vspace{1.5ex}
    \rlap{\underline{#1}}
    \par
    \setlength{\parindent}{0cm}
    \nopagebreak
    \leftskip=#2cm
    \rightskip=#3cm
}
{
    \par
}
\fi

\doendnotes{C}
\bigskip
\vfill

\clearpage

\footnotesize

\ifkorrekturansicht
  \lohead{\textsc{register}}
\fi

% theindex-Environment neu definieren ohne reledmac
\makeatletter
\renewenvironment{theindex}{%
  \ifkorrekturansicht
    \section*{\indexname}%
  \else
    \subsubsection*{Index der erwähnten Entitäten}%
  \fi
  \setlength{\parindent}{0pt}%
  \setlength{\parskip}{0pt plus 0.3pt}%
  \let\item\@idxitem
}{%
  \ifkorrekturansicht\clearpage\fi
}
\makeatother

\IfFileExists{\jobname-pw.ind}{\input{\jobname-pw.ind}}{}

% Quellenangabe nur in der Leseansicht
\ifkorrekturansicht\else
% Fallback-Definitionen, falls die .tex-Datei \titel etc. nicht gesetzt hat
\providecommand{\titel}{}
\providecommand{\editorInnen}{}
\providecommand{\dateiname}{\jobname}

\vspace{3cm}

\vfill

\footnotesize
\textsc{Quelle}: \titel. Herausgegeben von {\editorInnen}. In: \emph{Arthur Schnitzler: Briefwechsel mit Autorinnen und Autoren}.
 Digitale Edition, https://schnitzler-briefe.acdh.oeaw.ac.at/{\dateiname}.html (Stand \today)
\fi

\end{document}


      