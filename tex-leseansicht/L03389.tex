%% latex-leseansicht-vorspann.tex
%% Vorspann für die Leseansicht.
%% Lädt die gemeinsame Datei latex-vorspann.tex mit nicht gesetztem Schalter.

\newif\ifkorrekturansicht
\korrekturansichtfalse

\input{../tex-inputs/latex-vorspann}


\section[ Paul Goldmann an Arthur Schnitzler, 13. 12. [1903]]{L03389 Paul Goldmann an Arthur Schnitzler,  13. 12. [1903]}
\nopagebreak\mylabel{L03389v}
\rehead{ }\normalsize\beginnumbering\briefempfaengerindex{Schnitzler, Arthur@\textsc{Schnitzler, Arthur}!zzzGoldmann, Paul@\emph{von Paul Goldmann}!1903-12-132@{13. 12. [1903]}|(be}
\toendnotes[C]{\smallbreak\pagebreak[2]}
\correspDesc{Versand  durch Paul Goldmann am 13. 12. [1903] in Berlin
\newline{}Erhalt  durch Arthur Schnitzler im Zeitraum [14. 12. 1903 – 18. 12. 1903?] in Wien}\toendnotes[C]{\smallbreak}
\Standort{DLA, A:Schnitzler, HS.NZ85.1.3173.}
\physDesc{Brief, 2 Blätter, 7 Seiten, 3219 Zeichen
\newline{}Handschrift: blaue Tinte, deutsche Kurrent
\newline{}Schnitzler: 1) mit Bleistift das Jahr »903« vermerkt  2) mit rotem Buntstift neun Unterstreichungen}\toendnotes[C]{\smallbreak}
\pstart
           \raggedleft{}{\pb}\textcolor{gray}{\textbf{DESSAUERSTRASSE 19\oindex{Dessauer Straße@\textbf{Dessauer Straße}, \emph{Straße}|pw}}}\pend
           
\pstart
           Berlin\oindex{Berlin@\textbf{Berlin}, \emph{Hauptstadt}|pw}, 13. Dezember.\pend
           
\pstart\center{}Mein lieber Freund,\pend\vspace{0.5em}
\pstart
           Ich habe mich{ }ſehr gefreut, wieder einmal einen Brief von Dir zu erhalten. Auch die
               guten Nachrichten über Deine »engſte Familie\pwindex{Schnitzler, Olga 17.\,1.\,1882 Wien – 13.\,1.\,1970 Lugano@\textsc{Schnitzler, Olga} (17.\,1.\,1882 Wien – 13.\,1.\,1970 Lugano), \emph{Schauspielerin, Sängerin}|pwv}\pwindex{Schnitzler, Heinrich 9.\,8.\,1902 Hinterbrühl – 12.\,7.\,1982 Wien@\textsc{Schnitzler, Heinrich} (9.\,8.\,1902 Hinterbrühl – 12.\,7.\,1982 Wien), \emph{Regisseur, Schauspieler}|pwv}« haben mir viel Freude bereitet.\pend
           
\pstart
           Daß ich \introOben{}für\introOben{}{ }\label{K_L03389-1v}\edtext{Fräulein \textsc{Popper\pwindex{Popper, Emilie Dorothea 8.\,10.\,1893 Wien – 24.\,11.\,1933 ebd.@\textsc{Popper, Emilie Dorothea} (8.\,10.\,1893 Wien – 24.\,11.\,1933 ebd.), \emph{Pianistin, Pädagogin}|pw}}}{\lemma{\textnormal{\emph{Fräulein Popper}}}\Cendnote{\textnormal{Siehe XXXX Auszeichnungsfehler: Dokument L03388 nicht gefunden.
               }}}\label{K_L03389-1}, nachdem{ }ſie mir von Dir und Deiner Mutter\pwindex{Schnitzler, Louise 8.\,7.\,1840 Kőszeg – 9.\,9.\,1911 Wien@\textsc{Schnitzler, Louise} (8.\,7.\,1840 Kőszeg – 9.\,9.\,1911 Wien)|pwv} empfohlen worden, Alles that, was in meiner Macht{ }ſtand, iſt{ }ſelbſtverſtändlich. Wenn Du{ }ſie{ }ſiehſt,{ }ſo{ }ſage ihr, daß der Referent\pwindex{?? [Berliner Musikkorrespondent der National-Zeitung] @\textsc{?? [Berliner Musikkorrespondent der National-Zeitung]}|pwv} der »Nationalzeitung\orgindex{National-Zeitung@National-Zeitung|pw}«, an den ich{ }ſie empfohlen,{ }ſehr freundlich
               über{ }ſie \label{K_L03389-2v}\edtext{geſchrieben\pwindex{Theater- und Kunstnachrichten. [Man schreibt uns aus Berlin]@\emph{Theater- und Kunstnachrichten. [Man schreibt uns aus Berlin]}|pwv}}{\lemma{\textnormal{\emph{geschrieben}}}\Cendnote{\textnormal{Höchstwahrscheinlich Bezug auf folgende
                     Meldung\pwindex{Theater- und Kunstnachrichten. [Man schreibt uns aus Berlin]@\emph{Theater- und Kunstnachrichten. [Man schreibt uns aus Berlin]}|pwkv} über ein
                  Konzert von Dora Popper\pwindex{Popper, Emilie Dorothea 8.\,10.\,1893 Wien – 24.\,11.\,1933 ebd.@\textsc{Popper, Emilie Dorothea} (8.\,10.\,1893 Wien – 24.\,11.\,1933 ebd.), \emph{Pianistin, Pädagogin}|pwk}: [O. V.\pwindex{?? [Berliner Musikkorrespondent der National-Zeitung] @\textsc{?? [Berliner Musikkorrespondent der National-Zeitung]}|pwk}]: \emph{Theater- und Kunstnachrichten. [Man schreibt uns aus Berlin]}\pwindex{Theater- und Kunstnachrichten. [Man schreibt uns aus Berlin]@\emph{Theater- und Kunstnachrichten. [Man schreibt uns aus Berlin]}|pwk}. In: \emph{Neue Freie Presse}\pwindex{Neue Freie Presse@\emph{Neue Freie Presse}|pwk}, Nr. 14.093, 20. 11. 1903, Morgenblatt, S. 9.}}}\label{K_L03389-2}
               hat.\pend
           
\pstart
           {\pb}Am \label{K_L03389-3v}\edtext{Semmering\oindex{Semmering@\textbf{Semmering}, \emph{Verwaltungsgebiet}|pw}}{\lemma{\textnormal{\emph{Semmering}}}\Cendnote{\textnormal{Arthur und Olga Schnitzler\pwindex{Schnitzler, Olga 17.\,1.\,1882 Wien – 13.\,1.\,1970 Lugano@\textsc{Schnitzler, Olga} (17.\,1.\,1882 Wien – 13.\,1.\,1970 Lugano), \emph{Schauspielerin, Sängerin}|pwk} waren zwischen 6. 11. 1903 und 9. 11. 1903 am Semmering\oindex{Semmering@\textbf{Semmering}, \emph{Verwaltungsgebiet}|pwk} gewesen.}}}\label{K_L03389-3} muß es im Spätherbſt{ }ſehr{ }ſchön geweſen{ }ſein.
               Haſt Du weitere Winter-Reiſepläne? Über die Vorleſung Deines Stück\pwindex{Schnitzler, Arthur 15.\,5.\,1862 Wien – 21.\,10.\,1931 ebd.@\textsc{Schnitzler, Arthur} (15.\,5.\,1862 Wien – 21.\,10.\,1931 ebd.), \emph{Schriftsteller, Mediziner}!Gouvernante@\strich\emph{Die Gouvernante}|pwv}es durch \textsc{Ludwig Bauer\pwindex{Bauer, Ludwig 5.\,9.\,1876 Wien – 1.\,2.\,1935 Lugano@\textsc{Bauer, Ludwig} (5.\,9.\,1876 Wien – 1.\,2.\,1935 Lugano), \emph{Schriftsteller, Journalist}|pw}} habe ich{ }ſelbſtverſtändlich ein \label{K_L03389-4v}\edtext{Telegramm}{\lemma{\textnormal{\emph{Telegramm}}}\Cendnote{\textnormal{Ludwig Bauers\pwindex{Bauer, Ludwig 5.\,9.\,1876 Wien – 1.\,2.\,1935 Lugano@\textsc{Bauer, Ludwig} (5.\,9.\,1876 Wien – 1.\,2.\,1935 Lugano), \emph{Schriftsteller, Journalist}|pwk} Vorlesung von \emph{Die Gouvernante}\pwindex{Schnitzler, Arthur 15.\,5.\,1862 Wien – 21.\,10.\,1931 ebd.@\textsc{Schnitzler, Arthur} (15.\,5.\,1862 Wien – 21.\,10.\,1931 ebd.), \emph{Schriftsteller, Mediziner}!Gouvernante@\strich\emph{Die Gouvernante}|pwk} hatte am 2. 12. 1903 in Berlin\oindex{Berlin@\textbf{Berlin}, \emph{Hauptstadt}|pwk} stattgefunden
                  und war vom \emph{Verein zur Förderung der Künste}\orgindex{Verein zur Förderung der Künste@Verein zur Förderung der Künste|pwk}
                  veranstaltet worden. Siehe auch A. S.: \emph{Tagebuch}, 4. 12. 1903. Goldmanns\pwindex{Goldmann, Paul 31.\,1.\,1865 Breslau – 25.\,9.\,1935 Wien@\textsc{Goldmann, Paul} (31.\,1.\,1865 Breslau – 25.\,9.\,1935 Wien), \emph{Schriftsteller, Journalist}|pwk}
                  Telegramm dürfte tatsächlich nicht veröffentlicht worden sein.}}}\label{K_L03389-4} geſandt. Es
               iſt nicht erſchienen (oder{ }ſollte es mir entgangen{ }ſein?). Dieſes Nichterſcheinen
               richtet{ }ſich aber{ }ſicherlich gegen \textsc{Bauer\pwindex{Bauer, Ludwig 5.\,9.\,1876 Wien – 1.\,2.\,1935 Lugano@\textsc{Bauer, Ludwig} (5.\,9.\,1876 Wien – 1.\,2.\,1935 Lugano), \emph{Schriftsteller, Journalist}|pw}} und nicht gegen Dich. Mein \label{K_L03389-5v}\edtext{Telegramm\pwindex{Aus Berlin wird uns gemeldet: »Der einsame Weg«]@\emph{[Aus Berlin wird uns gemeldet: »Der einsame Weg«]}|pwv}}{\lemma{\textnormal{\emph{Telegramm}}}\Cendnote{\textnormal{[Paul Goldmann\pwindex{Goldmann, Paul 31.\,1.\,1865 Breslau – 25.\,9.\,1935 Wien@\textsc{Goldmann, Paul} (31.\,1.\,1865 Breslau – 25.\,9.\,1935 Wien), \emph{Schriftsteller, Journalist}|pwk}]: \emph{[Aus Berlin wird uns gemeldet: »Der einsame Weg«]}\pwindex{Aus Berlin wird uns gemeldet: »Der einsame Weg«]@\emph{[Aus Berlin wird uns gemeldet: »Der einsame Weg«]}|pwk}. In:
                        \emph{Neue Freie Presse}\pwindex{Neue Freie Presse@\emph{Neue Freie Presse}|pwk}, Nr. 14.115, 12. 12. 1903, Morgenblatt, S. 10.}}}\label{K_L03389-5}
               über das Bevorſtehen Deiner \textsc{\begin{otherlanguage}{french}Première\pwindex{Schnitzler, Arthur 15.\,5.\,1862 Wien – 21.\,10.\,1931 ebd.@\textsc{Schnitzler, Arthur} (15.\,5.\,1862 Wien – 21.\,10.\,1931 ebd.), \emph{Schriftsteller, Mediziner}!einsame Weg. Schauspiel in fünf Akten@\strich\emph{Der einsame Weg. Schauspiel in fünf Akten}|pwv}\end{otherlanguage}} iſt ja erſchienen.\pend
           
\pstart
           Zum Leſen komme ich gar nicht mehr,{ }ſeit die furchtbare Reichstag\orgindex{Reichstag@Reichstag|pw}sarbeit begonnen hat. \label{K_L03389-6v}\edtext{\textsc{Vehse\pwindex{Vehse, Karl Eduard 18.\,12.\,1802 Freiberg – 18.\,6.\,1870 Striesen@\textsc{Vehse, Karl Eduard} (18.\,12.\,1802 Freiberg – 18.\,6.\,1870 Striesen), \emph{Historiker}|pw}}}{\lemma{\textnormal{\emph{Vehse}}}\Cendnote{\textnormal{Werk nicht ermittelt}}}\label{K_L03389-6} habe ich
               habe ich mir gekauft (für 67 \textsc{MK}; was haſt Du gezahlt?).
               Haſt Du das gegenwärtige {\pb}deutſche Modebuch \label{K_L03389-7v}\edtext{»Briefe\textcolor{gray}{,} die ihn nicht erreichten\pwindex{Briefe, die ihn nicht erreichten@\emph{Briefe, die ihn nicht erreichten}|pw}«}{\lemma{\textnormal{\emph{»Briefe, … erreichten«}}}\Cendnote{\textnormal{Schnitzler hatte den Briefroman\pwindex{Briefe, die ihn nicht erreichten@\emph{Briefe, die ihn nicht erreichten}|pwkv} nicht gelesen. Siehe auch
                     XXXX Auszeichnungsfehler: Dokument L03375 nicht gefunden.}}}\label{K_L03389-7}{ }ſchon geleſen? Es iſt
               zu empfehlen.\pend
           
\pstart
           Meine Freundin\pwindex{Rottenberg, Theodore 7.\,9.\,1875 – 5.\,4.\,1945 Limburg an der Lahn@\textsc{Rottenberg, Theodore} (7.\,9.\,1875 – 5.\,4.\,1945 Limburg an der Lahn)|pwv} in Frankfurt\oindex{Frankfurt am Main@\textbf{Frankfurt am Main}, \emph{Hauptstadt}|pw} war krank. Lungenentzündung oder{ }ſo
               etwas. Ich bin{ }ſehr beſorgt. Aus ihren Briefen werde ich nicht recht klug inbezug auf
               ihre Krankheit. Die Ärzte{ }ſagen ihr auch offenbar nicht die Wahrheit; aber aus dem
               Umſtande, daß die Ärzte eine{ }ſofortige Reiſe nach dem Süden, womöglich Egypten\oindex{Ägypten@\textbf{Ägypten}|pw}, empfehlen, folgere ich allerlei
               Schlimmes.\pend
           
\pstart
           Als ich \label{K_L03389-8v}\edtext{das letzte Mal in Wien\oindex{Wien@\textbf{Wien}, \emph{Verwaltungsgebiet}|pw}}{\lemma{\textnormal{\emph{das letzte Mal in Wien}}}\Cendnote{\textnormal{im September 1903, vgl. XXXX Auszeichnungsfehler: Dokument L03386 nicht gefunden.}}}\label{K_L03389-8} mit Dir und Deiner Frau\pwindex{Schnitzler, Olga 17.\,1.\,1882 Wien – 13.\,1.\,1970 Lugano@\textsc{Schnitzler, Olga} (17.\,1.\,1882 Wien – 13.\,1.\,1970 Lugano), \emph{Schauspielerin, Sängerin}|pwv} über dieſe Angelegenheit{ }ſprach,{ }ſagteſt Du, daß ich eigentlich nunmehr
               gegen \strikeout{die} meine Freundin\pwindex{Rottenberg, Theodore 7.\,9.\,1875 – 5.\,4.\,1945 Limburg an der Lahn@\textsc{Rottenberg, Theodore} (7.\,9.\,1875 – 5.\,4.\,1945 Limburg an der Lahn)|pwv}{ }ſei, indem ich{ }ſie in {\pb}der \label{K_L03389-9v}\edtext{Illuſion}{\lemma{\textnormal{\emph{Illusion}}}\Cendnote{\textnormal{Siehe XXXX Auszeichnungsfehler: Dokument L03388 nicht gefunden.
               }}}\label{K_L03389-9} ließe, ich würde ſie\pwindex{Rottenberg, Theodore 7.\,9.\,1875 – 5.\,4.\,1945 Limburg an der Lahn@\textsc{Rottenberg, Theodore} (7.\,9.\,1875 – 5.\,4.\,1945 Limburg an der Lahn)|pwv}
               heirathen. Ich habe über dieſe Deine Worte oft nachgedacht. \strikeout{\textcolor{gray}{D}} Du haſt im Weſentlichen Recht; und da mich der Vorwurf der Unwahrheit{ }ſehr
               bedrückt, bin ich{ }ſeit Wochen bemüht, in meinen Briefen allmälig zur Wahrheit
               einzulenken. Sie weiß heut, daß ich{ }ſie, fürs Erſte wenigſtens, nicht heirathen kann;
               aber{ }ſie klammert{ }ſich trotzdem an mich, als \strikeout{ih\textcolor{gray}{ren}} denjenigen, der{ }ſie, wie{ }ſie{ }ſchreibt, »vom Abgrund zurückgeriſſen hat« und
               als ihren einzigen Halt.\pend
           
\pstart
           Was aus Alledem werden{ }ſoll, weiß der liebe Gott allein.\pend
           
\pstart
           Das Unglück wollte es, daß {\pb}\strikeout{daß} ich \label{K_L03389-10v}\edtext{\textsc{Bahr\pwindex{Bahr, Hermann 19.\,7.\,1863 Linz – 15.\,1.\,1934 München@\textsc{Bahr, Hermann} (19.\,7.\,1863 Linz – 15.\,1.\,1934 München), \emph{Schriftsteller, Kritiker}|pw}}, nachdem ich das Glück gehabt hatte, \strikeout{wahrſ\textcolor{gray}{e}} während{ }ſeines Berlin\oindex{Berlin@\textbf{Berlin}, \emph{Hauptstadt}|pw}er Aufenthalts}{\lemma{\textnormal{\emph{Bahr, … Aufenthalts}}}\Cendnote{\textnormal{Bahr\pwindex{Bahr, Hermann 19.\,7.\,1863 Linz – 15.\,1.\,1934 München@\textsc{Bahr, Hermann} (19.\,7.\,1863 Linz – 15.\,1.\,1934 München), \emph{Schriftsteller, Kritiker}|pwk} war vom 3. 12. 1903 bis zum 14. 12. 1903 in Berlin\oindex{Berlin@\textbf{Berlin}, \emph{Hauptstadt}|pwk} gewesen, um der Uraufführung seiner
                  Komödie \emph{Der Meister}\pwindex{Bahr, Hermann 19.\,7.\,1863 Linz – 15.\,1.\,1934 München@\textsc{Bahr, Hermann} (19.\,7.\,1863 Linz – 15.\,1.\,1934 München), \emph{Schriftsteller, Kritiker}!Meister. Komödie in drei Akten@\strich\emph{Der Meister. Komödie in drei Akten}|pwk} am \emph{Deutschen Theater}\orgindex{Deutsches Theater Berlin@Deutsches Theater Berlin|pwk} beizuwohnen.}}}\label{K_L03389-10} nigends mit ihm
               zuſammenzukommen, \introOben{}geſtern\introOben{} auf der Straße traf. Ich blieb{ }ſtehen, und wir geriethen in ein längeres
               Geſpräch. Dieſer alberne, dünkelhafte und verlogene Menſch\pwindex{Bahr, Hermann 19.\,7.\,1863 Linz – 15.\,1.\,1934 München@\textsc{Bahr, Hermann} (19.\,7.\,1863 Linz – 15.\,1.\,1934 München), \emph{Schriftsteller, Kritiker}|pwv} hat \strikeout{mich} mich immer heftig gereizt. Diesmal war dies ganz beſonders der Fall,
               und er{ }ſchien es auch darauf angelegt zu haben, mich zu provoziren. So theilte er mir
               Äußerungen mit, die Du und \textsc{Beer-Hofmann\pwindex{Beer-Hofmann, Richard 11.\,7.\,1866 Wien – 26.\,9.\,1945 New York City@\textsc{Beer-Hofmann, Richard} (11.\,7.\,1866 Wien – 26.\,9.\,1945 New York City), \emph{Schriftsteller}|pw}} gethan haben{ }ſollen. Ich gerieth in Hitze und antwortete {\pb}demgemäß. Hinterher wurde es mir klar, daß Deine und
                  \textsc{Richards\pwindex{Beer-Hofmann, Richard 11.\,7.\,1866 Wien – 26.\,9.\,1945 New York City@\textsc{Beer-Hofmann, Richard} (11.\,7.\,1866 Wien – 26.\,9.\,1945 New York City), \emph{Schriftsteller}|pw}} Äußerungen offenbar entſtellt
               wiedergegeben waren. Ich vermuthe, daß er\pwindex{Bahr, Hermann 19.\,7.\,1863 Linz – 15.\,1.\,1934 München@\textsc{Bahr, Hermann} (19.\,7.\,1863 Linz – 15.\,1.\,1934 München), \emph{Schriftsteller, Kritiker}|pwv} Dir jetzt auch meine Äußerungen entſtellt \label{K_L03389-11v}\edtext{berichten}{\lemma{\textnormal{\emph{berichten}}}\Cendnote{\textnormal{Schnitzler und Bahr\pwindex{Bahr, Hermann 19.\,7.\,1863 Linz – 15.\,1.\,1934 München@\textsc{Bahr, Hermann} (19.\,7.\,1863 Linz – 15.\,1.\,1934 München), \emph{Schriftsteller, Kritiker}|pwk} sprachen jedenfalls kurz darauf über Goldmann\pwindex{Goldmann, Paul 31.\,1.\,1865 Breslau – 25.\,9.\,1935 Wien@\textsc{Goldmann, Paul} (31.\,1.\,1865 Breslau – 25.\,9.\,1935 Wien), \emph{Schriftsteller, Journalist}|pwk}, vgl. A. S.: \emph{Tagebuch}, 18. 12. 1903 und Hermann Bahr, Arthur Schnitzler: \emph{Briefwechsel, Aufzeichnungen, Dokumente (1891–1931)}, Aufzeichnung von Hermann Bahr, 18. 12. 1903.}}}\label{K_L03389-11} wird, und bitte Dich, falls dies geſchehen{ }ſollte,
               nicht darauf zu achten.\pend
           
\pstart
           Wenn Du nächſtens einmal wieder Zeit findeſt, mir zu{ }ſchreiben, wirſt Du mir eine
               große Freude machen. Weihnachten gehe ich
               wahrſcheinlich nach {\pb}Frankfurt\oindex{Frankfurt am Main@\textbf{Frankfurt am Main}, \emph{Hauptstadt}|pw}.\pend
           
\pstart
           Viele herzliche Grüße an Dich und Deine Frau\pwindex{Schnitzler, Olga 17.\,1.\,1882 Wien – 13.\,1.\,1970 Lugano@\textsc{Schnitzler, Olga} (17.\,1.\,1882 Wien – 13.\,1.\,1970 Lugano), \emph{Schauspielerin, Sängerin}|pwv} von Deinem getreuen {\\[\baselineskip]}\spacefill\mbox{Paul Goldmann.}\pend
           \leftskip=0em{}\selectlanguage{ngerman}\endnumbering\briefempfaengerindex{Schnitzler, Arthur@\textsc{Schnitzler, Arthur}!zzzGoldmann, Paul@\emph{von Paul Goldmann}!1903-12-132@{13. 12. [1903]}|)be}\mylabel{L03389h}  \newcommand{\dateiname}{L03389}\newcommand{\titel}{Paul Goldmann an Arthur Schnitzler, 13. 12. [1903]}\newcommand{\editorInnen}{Martin Anton Müller und Laura Untner}%% latex-leseansicht-abspann.tex
%% Abspann für die Leseansicht.
%% Der Schalter \ifkorrekturansicht ist bereits durch den Vorspann gesetzt.

%% latex-abspann.tex
%% Gemeinsamer Abspann für Korrekturansicht und Leseansicht.
%% Setzt den Schalter \ifkorrekturansicht voraus (gesetzt in den
%% einbindenden Dateien latex-korrekturansicht-abspann.tex bzw.
%% latex-leseansicht-abspann.tex).
%% ---------------------------------------------------------------

\normalsize

% Das esempio-Environment wird nur in der Leseansicht benötigt
\ifkorrekturansicht\else
\newenvironment{esempio}[3]%
{
    \vspace{1.5ex}
    \rlap{\underline{#1}}
    \par
    \setlength{\parindent}{0cm}
    \nopagebreak
    \leftskip=#2cm
    \rightskip=#3cm
}
{
    \par
}
\fi

\doendnotes{C}
\bigskip
\vfill

\clearpage

\footnotesize

\ifkorrekturansicht
  \lohead{\textsc{register}}
\fi

% theindex-Environment neu definieren ohne reledmac
\makeatletter
\renewenvironment{theindex}{%
  \ifkorrekturansicht
    \section*{\indexname}%
  \else
    \subsubsection*{Index der erwähnten Entitäten}%
  \fi
  \setlength{\parindent}{0pt}%
  \setlength{\parskip}{0pt plus 0.3pt}%
  \let\item\@idxitem
}{%
  \ifkorrekturansicht\clearpage\fi
}
\makeatother

\IfFileExists{\jobname-pw.ind}{\input{\jobname-pw.ind}}{}

% Quellenangabe nur in der Leseansicht
\ifkorrekturansicht\else
% Fallback-Definitionen, falls die .tex-Datei \titel etc. nicht gesetzt hat
\providecommand{\titel}{}
\providecommand{\editorInnen}{}
\providecommand{\dateiname}{\jobname}

\vspace{3cm}

\vfill

\footnotesize
\textsc{Quelle}: \titel. Herausgegeben von {\editorInnen}. In: \emph{Arthur Schnitzler: Briefwechsel mit Autorinnen und Autoren}.
 Digitale Edition, https://schnitzler-briefe.acdh.oeaw.ac.at/{\dateiname}.html (Stand \today)
\fi

\end{document}


