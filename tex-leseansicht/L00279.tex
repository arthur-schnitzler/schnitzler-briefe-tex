%% latex-leseansicht-vorspann.tex
%% Vorspann für die Leseansicht.
%% Lädt die gemeinsame Datei latex-vorspann.tex mit nicht gesetztem Schalter.

\newif\ifkorrekturansicht
\korrekturansichtfalse

\input{../tex-inputs/latex-vorspann}


\section[Arthur Schnitzler an Hermann Bahr, 3. 11. 1893]{L00279 Arthur Schnitzler an Hermann Bahr, 3. 11. 1893}
\nopagebreak\mylabel{L00279v}
\rehead{ }\normalsize\beginnumbering\briefempfaengerindex{Bahr, Hermann@\textsc{Bahr, Hermann}!zzzSchnitzler, Arthur@\emph{von Arthur Schnitzler}!1893-11-032@{3. 11. 1893}|(be}
\toendnotes[C]{\smallbreak\pagebreak[2]}
\correspDesc{Versand  durch Arthur Schnitzler am 3. 11. 1893 in Wien
\newline{}Erhalt  durch Hermann Bahr im Zeitraum [3. 11. 1893
                  – 7. 11. 1893?] in Wien}\toendnotes[C]{\smallbreak}
\Standort{TMW, HS AM 23322 Ba.}
\physDesc{Brief, 1 Blatt, 3 Seiten, 1007 Zeichen (Briefpapier mit Trauerrand)
\newline{}Handschrift: schwarze Tinte, deutsche Kurrent
\newline{}Ordnung: Lochung }
\buchAbdrucke{\weitereDrucke{1) \emph{3. 11. 1893.} In: Arthur Schnitzler: \emph{The Letters of Arthur Schnitzler to Hermann Bahr}. Edited, annotated, and with an introduction, by Donald G. Daviau. Chapel Hill: \emph{The University of North Carolina Press} 1978, S. 57 (University of North Carolina studies in the Germanic languages
                        and literatures, 89).} \weitereDrucke{2) Hermann Bahr, Arthur Schnitzler: \emph{Briefwechsel, Aufzeichnungen, Dokumente (1891–1931)}. Herausgegeben von Kurt Ifkovits und Martin Anton Müller. Göttingen: \emph{Wallstein} 2018, S. 46.} }\toendnotes[C]{\smallbreak}
\pstart{}{\pb}Lieber
                  Freund,\pend\vspace{0.5em}
\pstart
           ich beiße bereits{ }ſeit einigen Tagen in den{ }ſauren Apfel, und werde mein Verſprechen
               halten. Es iſt nur wie ein Verhängnis, daſs mir nichts nach Wunsch gelingen will. Es
               iſt, wie we{\geminationn} mich die Empfindung: »man erwartet es von
               Dir« lähmte. –\pend
           
\pstart
           – Seit ich Feuilletons{ }ſchreiben{ }ſoll, hab ich eine ewige unbezwingliche Luſt,
               fünfactige Trauer{\pb}ſpiele zu{ }ſchreiben. Wirken Sie dahin, dſs \textsc{Burkhardt\pwindex{Burckhard, Max Eugen 14.\,7.\,1854 Korneuburg – 16.\,3.\,1912 Wien@\textsc{Burckhard, Max Eugen} (14.\,7.\,1854 Korneuburg – 16.\,3.\,1912 Wien), \emph{Schriftsteller, Rechtswissenschaftler, Theaterleiter}|pw}} eines von mir fordert – ich werde die{ }ſchönste Wien\oindex{Wien@\textbf{Wien}, \emph{Verwaltungsgebiet}|pw}er Geſchichte{ }ſchreiben.\pend
           
\pstart
           Im übrigen haben Sie Dinſtag oder{ }ſpätestens Mittwoch das bewußte Eingangsfeuilleton\pwindex{Schnitzler, Arthur 15.\,5.\,1862 Wien – 21.\,10.\,1931 ebd.@\textsc{Schnitzler, Arthur} (15.\,5.\,1862 Wien – 21.\,10.\,1931 ebd.), \emph{Schriftsteller, Mediziner}!Spaziergang@\strich\emph{Spaziergang}|pwv}. Eventuell werden Sie
               das Bedürfnis haben es zu ändern, wogegen ich principiell nichts einzuwenden habe. –
               (Nur müßt’ ich natürlich wiſſen, wie, wo, \textsc{etc.})\pend
           
\pstart
           {\pb}Vielleicht werd ich
               auch noch im Stande{ }ſein, Ihnen{ }ſtatt des \textsc{\label{K_L00279-1v}\edtext{Artifex\pwindex{Schnitzler, Arthur 15.\,5.\,1862 Wien – 21.\,10.\,1931 ebd.@\textsc{Schnitzler, Arthur} (15.\,5.\,1862 Wien – 21.\,10.\,1931 ebd.), \emph{Schriftsteller, Mediziner}!Artifex@\strich\emph{Artifex}|pw}}{\lemma{\textnormal{\emph{Artifex}}}\Cendnote{\textnormal{\emph{Artifex}\pwindex{Schnitzler, Arthur 15.\,5.\,1862 Wien – 21.\,10.\,1931 ebd.@\textsc{Schnitzler, Arthur} (15.\,5.\,1862 Wien – 21.\,10.\,1931 ebd.), \emph{Schriftsteller, Mediziner}!Artifex@\strich\emph{Artifex}|pwk}, allegorisches Gedicht in Jamben,
                     entstanden im Sommer 1893, unveröffentlicht (\emph{Cambridge University Library}, Schnitzler,
                     A 49). Eine Überarbeitung fand am 19. 11. 1893 statt.}}}\label{K_L00279-1}} was geſcheidteres zu geben. Wollen Sie mir ihn nicht vorläufig zurückleihen,
               damit ich zum mindeſten die böſeſten Verſe in ein behaglicheres Deutſch übertrage? –\pend
           
\pstart
           – Herzlichen Gruſs{\\[\baselineskip]}Ihr sehr ergebner{\\[\baselineskip]}\spacefill\mbox{Arthur Schnitzler.}\pend
           \leftskip=0em{}
\pstart
           Wien\oindex{Wien@\textbf{Wien}, \emph{Verwaltungsgebiet}|pw}{ }3. XI. 93.\pend
           \selectlanguage{ngerman}\endnumbering\briefempfaengerindex{Bahr, Hermann@\textsc{Bahr, Hermann}!zzzSchnitzler, Arthur@\emph{von Arthur Schnitzler}!1893-11-032@{3. 11. 1893}|)be}\mylabel{L00279h}  \newcommand{\dateiname}{L00279}\newcommand{\titel}{Arthur Schnitzler an Hermann Bahr, 3. 11. 1893}\newcommand{\editorInnen}{Herausgegeben von Martin Anton Müller}%% latex-leseansicht-abspann.tex
%% Abspann für die Leseansicht.
%% Der Schalter \ifkorrekturansicht ist bereits durch den Vorspann gesetzt.

%% latex-abspann.tex
%% Gemeinsamer Abspann für Korrekturansicht und Leseansicht.
%% Setzt den Schalter \ifkorrekturansicht voraus (gesetzt in den
%% einbindenden Dateien latex-korrekturansicht-abspann.tex bzw.
%% latex-leseansicht-abspann.tex).
%% ---------------------------------------------------------------

\normalsize

% Das esempio-Environment wird nur in der Leseansicht benötigt
\ifkorrekturansicht\else
\newenvironment{esempio}[3]%
{
    \vspace{1.5ex}
    \rlap{\underline{#1}}
    \par
    \setlength{\parindent}{0cm}
    \nopagebreak
    \leftskip=#2cm
    \rightskip=#3cm
}
{
    \par
}
\fi

\doendnotes{C}
\bigskip
\vfill

\clearpage

\footnotesize

\ifkorrekturansicht
  \lohead{\textsc{register}}
\fi

% theindex-Environment neu definieren ohne reledmac
\makeatletter
\renewenvironment{theindex}{%
  \ifkorrekturansicht
    \section*{\indexname}%
  \else
    \subsubsection*{Index der erwähnten Entitäten}%
  \fi
  \setlength{\parindent}{0pt}%
  \setlength{\parskip}{0pt plus 0.3pt}%
  \let\item\@idxitem
}{%
  \ifkorrekturansicht\clearpage\fi
}
\makeatother

\IfFileExists{\jobname-pw.ind}{\input{\jobname-pw.ind}}{}

% Quellenangabe nur in der Leseansicht
\ifkorrekturansicht\else
% Fallback-Definitionen, falls die .tex-Datei \titel etc. nicht gesetzt hat
\providecommand{\titel}{}
\providecommand{\editorInnen}{}
\providecommand{\dateiname}{\jobname}

\vspace{3cm}

\vfill

\footnotesize
\textsc{Quelle}: \titel. Herausgegeben von {\editorInnen}. In: \emph{Arthur Schnitzler: Briefwechsel mit Autorinnen und Autoren}.
 Digitale Edition, https://schnitzler-briefe.acdh.oeaw.ac.at/{\dateiname}.html (Stand \today)
\fi

\end{document}


