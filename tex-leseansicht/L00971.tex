%% latex-korrekturansicht-vorspann.tex
%% Vorspann für die Korrekturansicht.
%% Lädt die gemeinsame Datei latex-vorspann.tex mit gesetztem Schalter.

\newif\ifkorrekturansicht
\korrekturansichttrue

\input{../tex-inputs/latex-vorspann}


\section[Arthur Schnitzler an Richard Beer-Hofmann, 9. 9. 1899]{L00971 Arthur Schnitzler an Richard Beer-Hofmann, 9. 9. 1899}
\nopagebreak\mylabel{L00971v}
\rehead{ }\normalsize\beginnumbering\briefempfaengerindex{Beer-Hofmann, Richard@\textsc{Beer-Hofmann, Richard}!zzzSchnitzler, Arthur@\emph{von Arthur Schnitzler}!1899-09-091@{9. 9. 1899}|(be}
\toendnotes[C]{\smallbreak\pagebreak[2]}\Standort{YCGL, MSS 31.}
\physDesc{Brief, 1 Blatt, 4 Seiten, Umschlag, 720 Zeichen
\newline{}Handschrift: Bleistift, deutsche Kurrent
\newline{}Versand: 1) Stempel: »\nobreak{}\oindex{Bad Ischl@\textbf{Bad Ischl}, \emph{P.PPL}|pwk}Isc\textcolor{gray}{hl}, 9. {[}9. 1899{]}, 5–6{[}N{]}\nobreak{}«.   2) Stempel: »\nobreak{}\oindex{Sachsenburg@\textbf{Sachsenburg}, \emph{A.ADM3}|pwk}Sachsenburg, 10 9 \textcolor{gray}{99}\nobreak{}«.  3) Stempel: »\nobreak{}\oindex{Vahrn@\textbf{Vahrn}, \emph{P.PPLA3}|pwk}Vahrn, 12 9 99\nobreak{}«.  4) mit schwarzer Tinte von unbekannter Hand nachgesandt nach »\textsc{Vahrn\oindex{Vahrn@\textbf{Vahrn}, \emph{P.PPLA3}|pw} bei Brixen\oindex{Brixen@\textbf{Brixen}, \emph{P.PPLA3}|pw}}«}
\buchAbdrucke{\weitereDrucke{Arthur Schnitzler, Richard Beer-Hofmann: \emph{Briefwechsel 1891–1931}. Wien, Zürich: \emph{Europaverlag} 1992, S. 134.} }\toendnotes[C]{\smallbreak}\pstart{}{\pb}\textsc{Dr Richard Beer-Hofmann}\pend{}\pstart{}\textsc{Sachsenburg}\oindex{Sachsenburg@\textbf{Sachsenburg}, \emph{A.ADM3}|pw}\pend{}\pstart{}Gaſthof Fritz\oindex{Gasthof Fritz@\textbf{Gasthof Fritz}, \emph{Gastgewerbegebäude (K.GGW)}|pw}\pend{}\pstart{}\textsc{Kärnthen}\oindex{Kaernten@\textbf{Kärnten}, \emph{A.ADM1}|pw}\pend{}{\bigskip}\vspace{1em}
\pstart
           
\pstart
           {\pb}\textsc{Ischl}\oindex{Bad Ischl@\textbf{Bad Ischl}, \emph{P.PPL}|pw}.\pend
           
\pstart
           \raggedleft{}9. 9. 99.\pend
           \pend
           
\pstart{}Mein lieber Richard,\pend\vspace{0.5em}
\pstart
           Dinſtag verlaſſe ich Iſchl\oindex{Bad Ischl@\textbf{Bad Ischl}, \emph{P.PPL}|pw} und fahre
               vorerſt nach München\oindex{Muenchen@\textbf{München}, \emph{P.PPLA}|pw}. Ich möchte dort gern \introOben{}Mittwoch o Donnerſtg\introOben{} eine Nachricht von Ihnen \textsc{\uline{post. rest.}} finden.\pend
           
\pstart
           {\pb}Mir iſt’s mit mein\textcolor{gray}{em}{ }Stück\pwindex{Schleier der Beatrice. Schauspiel in fuenf Akten@\emph{Der Schleier der Beatrice. Schauspiel in fünf Akten}|pwv} momentweiſe gut, öfters
               mäßig gegangen, u ich habe es heute mit einem vorläufigen durchaus undefinitiven
               Abschluſs bei Seite gelegt; – auf 1–2\introOben{}–3\introOben{} Tage.\pend
           
\pstart
           {\pb}Ich hoffe, Sie fühlen ſich mit mehr Kraft Ihrem Stoff\pwindex{Graf von Charolais. Ein Trauerspiel@\emph{Der Graf von Charolais. Ein Trauerspiel}|pwv} gegenüber als ich.\pend
           
\pstart
           – Hugo\pwindex{Hofmannsthal, Hugo von 1874-02-01 – 1929-07-15@\textsc{Hofmannsthal, Hugo von} (1874-02-01 – 1929-07-15), \emph{Schriftsteller/Schriftstellerin}|pw} iſt ſchon wieder fort; ich bin ſehr
               froh geweſen, \substVorne{}\textsuperscript{als}\substDazwischen{}dſs\substHinten{} er da war, Sie werden ihn wohl bald ſehen. – Ich bin {\pb}recht ſehr gequält, durch allerlei; – durch das Ohr
               wohl am meiſten u tiefſten augenblicklich.\pend
           
\pstart
           Grüßen Sie Frau\pwindex{Beer-Hofmann, Paula 25.02.1879 – 30.10.1939@\textsc{Beer-Hofmann, Paula} (25.02.1879 – 30.10.1939)|pwv} und Kinder\pwindex{Beer-Hofmann, Naemah 20.12.1898 – 10.11.1971@\textsc{Beer-Hofmann, Naëmah} (20.12.1898 – 10.11.1971)|pwv}\pwindex{Beer-Hofmann, Mirjam 04.09.1897 – 24.12.1984@\textsc{Beer-Hofmann, Mirjam} (04.09.1897 – 24.12.1984)|pwv}\pend
           
\pstart
           Von Herzen Ihr{\\[\baselineskip]}\spacefill\mbox{Arthur}\pend
           \leftskip=0em{}\selectlanguage{ngerman}\endnumbering\briefempfaengerindex{Beer-Hofmann, Richard@\textsc{Beer-Hofmann, Richard}!zzzSchnitzler, Arthur@\emph{von Arthur Schnitzler}!1899-09-091@{9. 9. 1899}|)be}\mylabel{L00971h}  \normalsize

\doendnotes{C}
\bigskip
\vfill

\clearpage

\footnotesize

\lohead{\textsc{register}}

% Definiere theindex-Environment komplett neu ohne reledmac
\makeatletter
\renewenvironment{theindex}{%
  \section*{\indexname}%
  \setlength{\parindent}{0pt}%
  \setlength{\parskip}{0pt plus 0.3pt}%
  \let\item\@idxitem
}{%
  \clearpage
}
\makeatother

\IfFileExists{\jobname-pw.ind}{\input{\jobname-pw.ind}}{}

\end{document}

      