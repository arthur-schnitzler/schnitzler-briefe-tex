%% latex-leseansicht-vorspann.tex
%% Vorspann für die Leseansicht.
%% Lädt die gemeinsame Datei latex-vorspann.tex mit nicht gesetztem Schalter.

\newif\ifkorrekturansicht
\korrekturansichtfalse

\input{../tex-inputs/latex-vorspann}


         
         \renewcommand{\erwaehntePersonen}{Personen: Richard Beer-Hofmann, Paula Beer-Hofmann, Naëmah Beer-Hofmann, Mirjam Beer-Hofmann, Hugo von Hofmannsthal}
         \renewcommand{\erwaehnteOrte}{Orte: Bad Ischl, Brixen, Gasthof Fritz, Kärnten, München, Sachsenburg, Vahrn}
         \renewcommand{\erwaehnteWerke}{Werke: Der Graf von Charolais. Ein Trauerspiel, Der Schleier der Beatrice. Schauspiel in fünf Akten}
               \section[Arthur Schnitzler an Richard Beer-Hofmann, 9. 9. 1899]{ Arthur Schnitzler an Richard Beer-Hofmann, 9. 9. 1899}\nopagebreak\mylabel{v}\rehead{ }\begin{ledgroupsized}[t]{13cm}\normalsize\beginnumbering \toendnotes[C]{\smallbreak\pagebreak[2]} \Standort{YCGL, MSS 31.}
\physDesc{Brief, 1 Blatt, 4 Seiten, Umschlag, 720 Zeichen
\newline{}Handschrift: Bleistift, deutsche Kurrent
\newline{}Versand: 1) Stempel: »\nobreak{}\oindex{Bad Ischl@\textbf{Bad Ischl}|pwk}Isc\textcolor{gray}{hl}, 9. {[}9. 1899{]}, 5–6{[}N{]}\nobreak{}«.   2) Stempel: »\nobreak{}\oindex{Sachsenburg@\textbf{Sachsenburg}|pwk}Sachsenburg, 10 9 \textcolor{gray}{99}\nobreak{}«.  3) Stempel: »\nobreak{}\oindex{Vahrn@\textbf{Vahrn}|pwk}Vahrn, 12 9 99\nobreak{}«.  4) mit schwarzer Tinte von unbekannter Hand nachgesandt nach »\textsc{Vahrn\oindex{Vahrn@\textbf{Vahrn}|pw} bei Brixen\oindex{Brixen@\textbf{Brixen}|pw}}«}\buchAbdrucke{\weitereDrucke{Arthur Schnitzler, Richard Beer-Hofmann: \emph{Briefwechsel 1891–1931}. Hg. Konstanze Fliedl. Wien, Zürich: \emph{Europaverlag} 1992, S. 134.} }\toendnotes[C]{\smallbreak}\pstart{}{\pb}\textsc{Dr Richard Beer-Hofmann}\pend{}\pstart{}\textsc{Sachsenburg}\oindex{Sachsenburg@\textbf{Sachsenburg}|pw}\pend{}\pstart{}Gaſthof Fritz\oindex{Gasthof Fritz@\textbf{Gasthof Fritz}|pw}\pend{}\pstart{}\textsc{Kärnthen}\oindex{Kaernten@\textbf{Kärnten}|pw}\pend{}{\bigskip}\pstart
           {\pb}\textsc{Ischl}\oindex{Bad Ischl@\textbf{Bad Ischl}|pw}.\hfill 9. 9. 99.\pend
           \pstart{}Mein lieber Richard,\pend\pstart
           Dinſtag verlaſſe ich Iſchl\oindex{Bad Ischl@\textbf{Bad Ischl}|pw} und fahre
               vorerſt nach München\oindex{Muenchen@\textbf{München}|pw}. Ich möchte dort gern \introOben{}Mittwoch o Donnerſtg\introOben{} eine Nachricht von Ihnen \textsc{\uline{post. rest.}} finden.\pend
           \pstart
           {\pb}Mir iſt’s mit mein\textcolor{gray}{em}{ }Stück\pwindex{Schnitzler, Arthur 15.05.1862 – 21.10.1931@\textsc{Schnitzler, Arthur} (15.05.1862 – 21.10.1931), \emph{Schriftsteller, Mediziner}!Schleier der Beatrice. Schauspiel in fuenf Akten1900-12-01@\strich\emph{Der Schleier der Beatrice. Schauspiel in fünf Akten} {[}1900-12-01{]}|pwv} momentweiſe gut, öfters
               mäßig gegangen, u ich habe es heute mit einem vorläufigen durchaus undefinitiven
               Abschluſs bei Seite gelegt; – auf 1–2\introOben{}–3\introOben{} Tage.\pend
           \pstart
           {\pb}Ich hoffe, Sie fühlen ſich mit mehr Kraft Ihrem Stoff\pwindex{Beer-Hofmann, Richard 1866-07-11 – 1945-09-26@\textsc{Beer-Hofmann, Richard} (1866-07-11 – 1945-09-26), \emph{Schriftsteller}!Graf von Charolais. Ein Trauerspiel1904-12-23@\strich\emph{Der Graf von Charolais. Ein Trauerspiel} {[}1904-12-23{]}|pwv} gegenüber als ich.\pend
           \pstart
           – Hugo\pwindex{Hofmannsthal, Hugo von 1874-02-01 – 1929-07-15@\textsc{Hofmannsthal, Hugo von} (1874-02-01 – 1929-07-15), \emph{Schriftsteller}|pw} iſt ſchon wieder fort; ich bin ſehr
               froh geweſen, \substVorne{}\textsuperscript{als}\substDazwischen{}dſs\substHinten{} er da war, Sie werden ihn wohl bald ſehen. – Ich bin {\pb}recht ſehr gequält, durch allerlei; – durch das Ohr
               wohl am meiſten u tiefſten augenblicklich.\pend
           \pstart
           Grüßen Sie Frau\pwindex{Beer-Hofmann, Paula 25.02.1879 – 30.10.1939@\textsc{Beer-Hofmann, Paula} (25.02.1879 – 30.10.1939)|pwv} und Kinder\pwindex{Beer-Hofmann, Naemah 20.12.1898 – 10.11.1971@\textsc{Beer-Hofmann, Naëmah} (20.12.1898 – 10.11.1971)|pwv}\pwindex{Beer-Hofmann, Mirjam 04.09.1897 – 24.12.1984@\textsc{Beer-Hofmann, Mirjam} (04.09.1897 – 24.12.1984)|pwv}\pend
           \pstart
           Von Herzen Ihr{\\[\baselineskip]}\spacefill\mbox{Arthur}\pend
           \leftskip=0em{}
         
         \endnumbering\mylabel{h}\end{ledgroupsized}  \newcommand{\dateiname}{L00971}\newcommand{\titel}{Arthur Schnitzler an Richard Beer-Hofmann, 9. 9. 1899}\newcommand{\editorInnen}{Martin Anton Müller und Gerd-Hermann Susen}%% latex-leseansicht-abspann.tex
%% Abspann für die Leseansicht.
%% Der Schalter \ifkorrekturansicht ist bereits durch den Vorspann gesetzt.

%% latex-abspann.tex
%% Gemeinsamer Abspann für Korrekturansicht und Leseansicht.
%% Setzt den Schalter \ifkorrekturansicht voraus (gesetzt in den
%% einbindenden Dateien latex-korrekturansicht-abspann.tex bzw.
%% latex-leseansicht-abspann.tex).
%% ---------------------------------------------------------------

\normalsize

% Das esempio-Environment wird nur in der Leseansicht benötigt
\ifkorrekturansicht\else
\newenvironment{esempio}[3]%
{
    \vspace{1.5ex}
    \rlap{\underline{#1}}
    \par
    \setlength{\parindent}{0cm}
    \nopagebreak
    \leftskip=#2cm
    \rightskip=#3cm
}
{
    \par
}
\fi

\doendnotes{C}
\bigskip
\vfill

\clearpage

\footnotesize

\ifkorrekturansicht
  \lohead{\textsc{register}}
\fi

% theindex-Environment neu definieren ohne reledmac
\makeatletter
\renewenvironment{theindex}{%
  \ifkorrekturansicht
    \section*{\indexname}%
  \else
    \subsubsection*{Index der erwähnten Entitäten}%
  \fi
  \setlength{\parindent}{0pt}%
  \setlength{\parskip}{0pt plus 0.3pt}%
  \let\item\@idxitem
}{%
  \ifkorrekturansicht\clearpage\fi
}
\makeatother

\IfFileExists{\jobname-pw.ind}{\input{\jobname-pw.ind}}{}

% Quellenangabe nur in der Leseansicht
\ifkorrekturansicht\else
% Fallback-Definitionen, falls die .tex-Datei \titel etc. nicht gesetzt hat
\providecommand{\titel}{}
\providecommand{\editorInnen}{}
\providecommand{\dateiname}{\jobname}

\vspace{3cm}

\vfill

\footnotesize
\textsc{Quelle}: \titel. Herausgegeben von {\editorInnen}. In: \emph{Arthur Schnitzler: Briefwechsel mit Autorinnen und Autoren}.
 Digitale Edition, https://schnitzler-briefe.acdh.oeaw.ac.at/{\dateiname}.html (Stand \today)
\fi

\end{document}


      