%% latex-leseansicht-vorspann.tex
%% Vorspann für die Leseansicht.
%% Lädt die gemeinsame Datei latex-vorspann.tex mit nicht gesetztem Schalter.

\newif\ifkorrekturansicht
\korrekturansichtfalse

\input{../tex-inputs/latex-vorspann}


\section[Arthur Schnitzler an Richard Beer-Hofmann, 9. 9. 1899]{L00971 Arthur Schnitzler an Richard Beer-Hofmann, 9. 9. 1899}
\nopagebreak\mylabel{L00971v}
\rehead{ }\normalsize\beginnumbering\briefempfaengerindex{Beer-Hofmann, Richard@\textsc{Beer-Hofmann, Richard}!zzzSchnitzler, Arthur@\emph{von Arthur Schnitzler}!1899-09-091@{9. 9. 1899}|(be}
\toendnotes[C]{\smallbreak\pagebreak[2]}
\correspDesc{Versand  durch Arthur Schnitzler am 9. 9. 1899 in Bad Ischl
\newline{}Umleitung  am 10. 9. 1899 in Sachsenburg
\newline{}Erhalt  durch Richard Beer-Hofmann am 12. 9. 1899 in Vahrn}\toendnotes[C]{\smallbreak}
\Standort{YCGL, MSS 31.}
\physDesc{Brief, 1 Blatt, 4 Seiten, Kuvert, 720 Zeichen
\newline{}Handschrift: Bleistift, deutsche Kurrent
\newline{}Versand: 1) Stempel: »\nobreak{}\oindex{Bad Ischl@\textbf{Bad Ischl}|pwk}Isc\textcolor{gray}{hl}, 9. {[}9. 1899{]}, 5–6{[}N{]}\nobreak{}«.   2) Stempel: »\nobreak{}\oindex{Sachsenburg@\textbf{Sachsenburg}, \emph{Verwaltungsgebiet}|pwk}Sachsenburg, 10 9 \textcolor{gray}{99}\nobreak{}«.  3) Stempel: »\nobreak{}\oindex{Vahrn@\textbf{Vahrn}, \emph{Hauptstadt}|pwk}Vahrn, 12 9 99\nobreak{}«.  4) mit schwarzer Tinte von unbekannter Hand nachgesandt nach »\textsc{Vahrn\oindex{Vahrn@\textbf{Vahrn}, \emph{Hauptstadt}|pw} bei Brixen\oindex{Brixen@\textbf{Brixen}, \emph{Hauptstadt}|pw}}«}
\buchAbdrucke{\weitereDrucke{Arthur Schnitzler, Richard Beer-Hofmann: \emph{Briefwechsel 1891–1931}. Herausgegeben von Konstanze Fliedl. Wien, Zürich: \emph{Europaverlag} 1992, S. 134.} }\toendnotes[C]{\smallbreak}\pstart{}{\pb}\textsc{Dr Richard Beer-Hofmann}\pend{}\pstart{}\textsc{Sachsenburg}\oindex{Sachsenburg@\textbf{Sachsenburg}, \emph{Verwaltungsgebiet}|pw}\pend{}\pstart{}Gaſthof Fritz\oindex{Gasthof Fritz@\textbf{Gasthof Fritz}, \emph{Gastgewerbegebäude}|pw}\pend{}\pstart{}\textsc{Kärnthen}\oindex{Kärnten@\textbf{Kärnten}, \emph{Land}|pw}\pend{}{\bigskip}\vspace{1em}
\pstart
           
\pstart
           {\pb}\textsc{Ischl}\oindex{Bad Ischl@\textbf{Bad Ischl}|pw}.\pend
           
\pstart
           \raggedleft{}9. 9. 99.\pend
           \pend
           
\pstart{}Mein lieber Richard,\pend\vspace{0.5em}
\pstart
           Dinſtag verlaſſe ich Iſchl\oindex{Bad Ischl@\textbf{Bad Ischl}|pw} und fahre
               vorerſt nach München\oindex{München@\textbf{München}|pw}. Ich möchte dort gern \introOben{}Mittwoch o Donnerſtg\introOben{} eine Nachricht von Ihnen \textsc{\uline{post. rest.}} finden.\pend
           
\pstart
           {\pb}Mir iſt’s mit mein\textcolor{gray}{em}{ }Stück\pwindex{Schnitzler, Arthur 15.\,5.\,1862 Wien – 21.\,10.\,1931 ebd.@\textsc{Schnitzler, Arthur} (15.\,5.\,1862 Wien – 21.\,10.\,1931 ebd.), \emph{Schriftsteller, Mediziner}!Schleier der Beatrice. Schauspiel in fünf Akten@\strich\emph{Der Schleier der Beatrice. Schauspiel in fünf Akten}|pwv} momentweiſe gut, öfters
               mäßig gegangen, u ich habe es heute mit einem vorläufigen durchaus undefinitiven
               Abschluſs bei Seite gelegt; – auf 1–2\introOben{}–3\introOben{} Tage.\pend
           
\pstart
           {\pb}Ich hoffe, Sie fühlen{ }ſich mit mehr Kraft Ihrem Stoff\pwindex{Beer-Hofmann, Richard 11.\,7.\,1866 Wien – 26.\,9.\,1945 New York City@\textsc{Beer-Hofmann, Richard} (11.\,7.\,1866 Wien – 26.\,9.\,1945 New York City), \emph{Schriftsteller}!Graf von Charolais. Ein Trauerspiel@\strich\emph{Der Graf von Charolais. Ein Trauerspiel}|pwv} gegenüber als ich.\pend
           
\pstart
           – Hugo\pwindex{Hofmannsthal, Hugo von 1.\,2.\,1874 Wien – 15.\,7.\,1929 Rodaun@\textsc{Hofmannsthal, Hugo von} (1.\,2.\,1874 Wien – 15.\,7.\,1929 Rodaun), \emph{Schriftsteller}|pw} iſt{ }ſchon wieder fort; ich bin{ }ſehr
               froh geweſen, \substVorne{}\textsuperscript{als}\substDazwischen{}dſs\substHinten{} er da war, Sie werden ihn wohl bald{ }ſehen. – Ich bin {\pb}recht{ }ſehr gequält, durch allerlei; – durch das Ohr
               wohl am meiſten u tiefſten augenblicklich.\pend
           
\pstart
           Grüßen Sie Frau\pwindex{Beer-Hofmann, Paula 25.\,2.\,1879 Wien – 30.\,10.\,1939 Zürich@\textsc{Beer-Hofmann, Paula} (25.\,2.\,1879 Wien – 30.\,10.\,1939 Zürich)|pwv} und Kinder\pwindex{Beer-Hofmann, Naëmah 20.\,12.\,1898 Wien – 10.\,11.\,1971 New York City@\textsc{Beer-Hofmann, Naëmah} (20.\,12.\,1898 Wien – 10.\,11.\,1971 New York City)|pwv}\pwindex{Beer-Hofmann, Mirjam 4.\,9.\,1897 Wien – 24.\,12.\,1984 New York City@\textsc{Beer-Hofmann, Mirjam} (4.\,9.\,1897 Wien – 24.\,12.\,1984 New York City)|pwv}\pend
           
\pstart
           Von Herzen Ihr{\\[\baselineskip]}\spacefill\mbox{Arthur}\pend
           \leftskip=0em{}\selectlanguage{ngerman}\endnumbering\briefempfaengerindex{Beer-Hofmann, Richard@\textsc{Beer-Hofmann, Richard}!zzzSchnitzler, Arthur@\emph{von Arthur Schnitzler}!1899-09-091@{9. 9. 1899}|)be}\mylabel{L00971h}  \newcommand{\dateiname}{L00971}\newcommand{\titel}{Arthur Schnitzler an Richard Beer-Hofmann, 9. 9. 1899}\newcommand{\editorInnen}{Martin Anton Müller und Gerd-Hermann Susen}%% latex-leseansicht-abspann.tex
%% Abspann für die Leseansicht.
%% Der Schalter \ifkorrekturansicht ist bereits durch den Vorspann gesetzt.

%% latex-abspann.tex
%% Gemeinsamer Abspann für Korrekturansicht und Leseansicht.
%% Setzt den Schalter \ifkorrekturansicht voraus (gesetzt in den
%% einbindenden Dateien latex-korrekturansicht-abspann.tex bzw.
%% latex-leseansicht-abspann.tex).
%% ---------------------------------------------------------------

\normalsize

% Das esempio-Environment wird nur in der Leseansicht benötigt
\ifkorrekturansicht\else
\newenvironment{esempio}[3]%
{
    \vspace{1.5ex}
    \rlap{\underline{#1}}
    \par
    \setlength{\parindent}{0cm}
    \nopagebreak
    \leftskip=#2cm
    \rightskip=#3cm
}
{
    \par
}
\fi

\doendnotes{C}
\bigskip
\vfill

\clearpage

\footnotesize

\ifkorrekturansicht
  \lohead{\textsc{register}}
\fi

% theindex-Environment neu definieren ohne reledmac
\makeatletter
\renewenvironment{theindex}{%
  \ifkorrekturansicht
    \section*{\indexname}%
  \else
    \subsubsection*{Index der erwähnten Entitäten}%
  \fi
  \setlength{\parindent}{0pt}%
  \setlength{\parskip}{0pt plus 0.3pt}%
  \let\item\@idxitem
}{%
  \ifkorrekturansicht\clearpage\fi
}
\makeatother

\IfFileExists{\jobname-pw.ind}{\input{\jobname-pw.ind}}{}

% Quellenangabe nur in der Leseansicht
\ifkorrekturansicht\else
% Fallback-Definitionen, falls die .tex-Datei \titel etc. nicht gesetzt hat
\providecommand{\titel}{}
\providecommand{\editorInnen}{}
\providecommand{\dateiname}{\jobname}

\vspace{3cm}

\vfill

\footnotesize
\textsc{Quelle}: \titel. Herausgegeben von {\editorInnen}. In: \emph{Arthur Schnitzler: Briefwechsel mit Autorinnen und Autoren}.
 Digitale Edition, https://schnitzler-briefe.acdh.oeaw.ac.at/{\dateiname}.html (Stand \today)
\fi

\end{document}


