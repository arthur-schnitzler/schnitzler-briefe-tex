%% latex-leseansicht-vorspann.tex
%% Vorspann für die Leseansicht.
%% Lädt die gemeinsame Datei latex-vorspann.tex mit nicht gesetztem Schalter.

\newif\ifkorrekturansicht
\korrekturansichtfalse

\input{../tex-inputs/latex-vorspann}


\section[Arthur Schnitzler an Gustav Schwarzkopf, {[}2. 10. 1902?{]}]{L04178 Arthur Schnitzler an Gustav Schwarzkopf, {[}2. 10. 1902?{]}}
\nopagebreak\mylabel{L04178v}
\rehead{ }\normalsize\beginnumbering\briefempfaengerindex{Schwarzkopf, Gustav@\textsc{Schwarzkopf, Gustav}!zzzSchnitzler, Arthur@\emph{von Arthur Schnitzler}!1902-10-021@{{[}2. 10. 1902?{]}}|(be}
\toendnotes[C]{\smallbreak\pagebreak[2]}
\correspDesc{Versand  durch Arthur Schnitzler am [2. 10. 1902?] in Wien
\newline{}Erhalt  durch Gustav Schwarzkopf im Zeitraum [2. 10. 1902 – 3. 10. 1902?] in Wien}\toendnotes[C]{\smallbreak}
\Standort{CUL, Schnitzler, B 96.}
\physDesc{Briefkarte, 151 Zeichen
\newline{}Handschrift: schwarze Tinte, deutsche Kurrent}\toendnotes[C]{\smallbreak}
\pstart
           \noindent{}{\pb}lieber Guſtav, wollen Sie \label{K_L04178-1v}\edtext{morgen
               Freitag{ }Frankgaſſe\oindex{Wien@\textbf{Wien}!IX., Alsergrund@\textbf{IX., Alsergrund}!Frankgasse 1@\textbf{Frankgasse 1}, \emph{Wohngebäude}|pw} nachtmahlen}{\lemma{\textnormal{\emph{morgen … nachtmahlen}}}\Cendnote{\textnormal{Die Karte ist undatiert. Eine 
                  vergleichbare Karte verwandte Schnitzler für sein Schreiben an Beer-Hofmann\pwindex{Beer-Hofmann, Richard 11.\,7.\,1866 Wien – 26.\,9.\,1945 New York City@\textsc{Beer-Hofmann, Richard} (11.\,7.\,1866 Wien – 26.\,9.\,1945 New York City), \emph{Schriftsteller}|pwk} vom
                  XXXX Auszeichnungsfehler: Dokument L01260 nicht gefunden. Unter der Annahme, dass die Korrespondenz
                  weitgehend vollständig überliefert ist, könnte es sich um die Einladung handeln, die Schwarzkopf\pwindex{Schwarzkopf, Gustav 7.\,11.\,1853 Wien – 13.\,11.\,1939 ebd.@\textsc{Schwarzkopf, Gustav} (7.\,11.\,1853 Wien – 13.\,11.\,1939 ebd.), \emph{Schriftsteller}|pwk} 
                  mit seiner undatierten Visitenkarte ({XXXX ref} XXXX) ablehnte. Im Gegenzug erlaubt das
                  auch die mögliche Datierung der Visitenkarte, weil \emph{Die Freundin}\pwindex{Brociner, Marco 20.\,10.\,1852 Iași – 12.\,4.\,1942 Wien@\textsc{Brociner, Marco} (20.\,10.\,1852 Iași – 12.\,4.\,1942 Wien), \emph{Schriftsteller, Journalist, Kritiker}!Freundin. Comödie in vier Aufzügen@\strich\emph{Die Freundin. Comödie in vier Aufzügen}|pwk} nur zweimal an einem
                  Freitag gegeben wurde, am 3. 10. 1902 und am 24. 10. 1902, aber nur beim ersten 
                  Termin scheint Schnitzler den Abend zuhause vebracht zu haben.}}}\label{K_L04178-1}?
      Wein über 45 Kr. der Doppelliter
      garantirt. Papikafiſch wahrſcheinlich.\pend
           
\pstart
           Herzlichst Ihr{\\[\baselineskip]}\spacefill\mbox{A. S.}\pend
           \leftskip=0em{}\selectlanguage{ngerman}\endnumbering\briefempfaengerindex{Schwarzkopf, Gustav@\textsc{Schwarzkopf, Gustav}!zzzSchnitzler, Arthur@\emph{von Arthur Schnitzler}!1902-10-021@{{[}2. 10. 1902?{]}}|)be}\mylabel{L04178h}
\begin{anhang}
\end{anhang}\newcommand{\dateiname}{L04178}\newcommand{\titel}{Arthur Schnitzler an Gustav Schwarzkopf, [2. 10. 1902?]}\newcommand{\editorInnen}{Herausgegeben von Jahnke, SelmaMüller, Martin Anton}%% latex-leseansicht-abspann.tex
%% Abspann für die Leseansicht.
%% Der Schalter \ifkorrekturansicht ist bereits durch den Vorspann gesetzt.

%% latex-abspann.tex
%% Gemeinsamer Abspann für Korrekturansicht und Leseansicht.
%% Setzt den Schalter \ifkorrekturansicht voraus (gesetzt in den
%% einbindenden Dateien latex-korrekturansicht-abspann.tex bzw.
%% latex-leseansicht-abspann.tex).
%% ---------------------------------------------------------------

\normalsize

% Das esempio-Environment wird nur in der Leseansicht benötigt
\ifkorrekturansicht\else
\newenvironment{esempio}[3]%
{
    \vspace{1.5ex}
    \rlap{\underline{#1}}
    \par
    \setlength{\parindent}{0cm}
    \nopagebreak
    \leftskip=#2cm
    \rightskip=#3cm
}
{
    \par
}
\fi

\doendnotes{C}
\bigskip
\vfill

\clearpage

\footnotesize

\ifkorrekturansicht
  \lohead{\textsc{register}}
\fi

% theindex-Environment neu definieren ohne reledmac
\makeatletter
\renewenvironment{theindex}{%
  \ifkorrekturansicht
    \section*{\indexname}%
  \else
    \subsubsection*{Index der erwähnten Entitäten}%
  \fi
  \setlength{\parindent}{0pt}%
  \setlength{\parskip}{0pt plus 0.3pt}%
  \let\item\@idxitem
}{%
  \ifkorrekturansicht\clearpage\fi
}
\makeatother

\IfFileExists{\jobname-pw.ind}{\input{\jobname-pw.ind}}{}

% Quellenangabe nur in der Leseansicht
\ifkorrekturansicht\else
% Fallback-Definitionen, falls die .tex-Datei \titel etc. nicht gesetzt hat
\providecommand{\titel}{}
\providecommand{\editorInnen}{}
\providecommand{\dateiname}{\jobname}

\vspace{3cm}

\vfill

\footnotesize
\textsc{Quelle}: \titel. Herausgegeben von {\editorInnen}. In: \emph{Arthur Schnitzler: Briefwechsel mit Autorinnen und Autoren}.
 Digitale Edition, https://schnitzler-briefe.acdh.oeaw.ac.at/{\dateiname}.html (Stand \today)
\fi

\end{document}


