%% latex-leseansicht-vorspann.tex
%% Vorspann für die Leseansicht.
%% Lädt die gemeinsame Datei latex-vorspann.tex mit nicht gesetztem Schalter.

\newif\ifkorrekturansicht
\korrekturansichtfalse

\input{../tex-inputs/latex-vorspann}


               \section[Richard Beer-Hofmann an Arthur Schnitzler, 18. 7. 1917]{ Richard Beer-Hofmann an Arthur Schnitzler, 18. 7. 1917}\nopagebreak\mylabel{v}\rehead{ }\begin{ledgroupsized}[t]{13cm}\normalsize\beginnumbering\briefempfaengerindex{Schnitzler, Arthur@\textsc{Schnitzler, Arthur}!zzzBeer-Hofmann, Richard@\emph{von Richard Beer-Hofmann}!1917-07-181@{18. 7. 1917}|(be} \toendnotes[C]{\smallbreak\pagebreak[2]} \Standort{CUL, Schnitzler, B 8.}
\physDesc{Brief, 1 Blatt, 2 Seiten
\newline{}Handschrift: Bleistift, lateinische Kurrent
\newline{}Schnitzler: 1) mit Bleistift beschriftet: »\textsc{Richar}« 2) mit rotem Buntstift zwei Unterstreichungen\newline{}Ordnung: mit Bleistift von unbekannter Hand nummeriert:
                                    »263« }\buchAbdrucke{\weitereDrucke{Arthur Schnitzler, Richard Beer-Hofmann: \emph{Briefwechsel 1891–1931}. Hg. Konstanze Fliedl. Wien, Zürich: \emph{Europaverlag} 1992, S. 223–224.} }\toendnotes[C]{\smallbreak}\pstart
           \raggedleft{}{\pb}Ischl\oindex{Bad Ischl@\textbf{Bad Ischl}|pw}{ }18/VII 17\pend
           \pstart
           Lieber Arthur! Ich habe Ihren Brief erwartet. Ich hatte mit Absicht
               Ihnen nicht geschrieben, ich wollte wissen, wie Sie – unbeeinflusst durch meinen
               Bericht – die Sache ansehen. Ich war durch den akuten Anfall, den ich ja durch
               3 Stunden mit ansah (K.\pwindex{Kaufmann, Arthur 04.04.1872 – 25.07.1938@\textsc{Kaufmann, Arthur} (04.04.1872 – 25.07.1938), \emph{Rechtswissenschaftler, Privatgelehrte, Privatier}|pw} hatte nach mir verlangt)
               sehr erschreckt. Sie selbst sahen ja nur einen Zustand, der vom Normalen nicht so
               weit abzuliegen schien. Ich aber verbrachte auch die dem Anfall folgenden Tage, bis
               zu seiner Abreise ins Sanatorium\oindex{Sanatorium Purkersdorf@\textbf{Sanatorium Purkersdorf}|pwv}
               in einer unaufhörlichen Anspannung, da ich mich – es war ja niemand, als seine Schwester\pwindex{Kaufmann, Malvine 1875 – 30.03.1923@\textsc{Kaufmann, Malvine} (1875 – 30.03.1923)|pwv} da – irgendwie
               verantwortlich fühlte. Auch betonte Dozent K.\pwindex{Kaufmann, Rudolf 03.09.1871 – 20.06.1927@\textsc{Kaufmann, Rudolf} (03.09.1871 – 20.06.1927), \emph{Internist}|pw} ja
                  i{\geminationm}er sein Laiesein in derartigen Dingen, sah aber
               recht schwarz {\pb}und ich mit ihm. Was
               mich bestürzte, war, dass es nicht eine Steigerung oder Über-Spannung seiner
               sonstigen Art zu denken war, sondern ein vollständiges Anders-sein, Reden,
               »Philosophiren«, wie es ihm sein Lebtag verhasst und lächerlich erschienen war.
               Niederschreiben mag und kann ich das Alles nicht, und nun – da es ja wieder gutgeht,
               hätte es ja auch nicht viel Sinn, es festzuhalten.\pend
           \pstart
           Ich bin von Herzen froh, dass es so – und nicht anders – ausgieng.\pend
           \pstart
           Von uns ist nichts zu berichten, als dass wir eine schlechte Woche mit \label{KLL02266_AS-1v}\edtext{Schufterl}{\lemma{\textnormal{\emph{Schufterl}}}\Cendnote{\textnormal{ein weißer Spitz, erworben im Dezember 1905}}}\label{KLL02266_AS-1h}
               verbrachten, der fast zwölf Jahre mit uns lebte, und nun im Garten der Villa begraben
               wurde. –\pend
           \pstart
           Werden wir Sie im So{\geminationm}er im Salzka{\geminationm}ergut\oindex{Salzkammergut@\textbf{Salzkammergut}|pw} sehen?\pend
           \pstart
           Alles Herzliche Ihnen, Frau Olga\pwindex{Schnitzler, Olga 17.01.1882 – 13.01.1970@\textsc{Schnitzler, Olga} (17.01.1882 – 13.01.1970), \emph{Schauspielerin, Sängerin}|pw} und den Kindern\pwindex{Schnitzler, Heinrich 09.08.1902 – 12.07.1982@\textsc{Schnitzler, Heinrich} (09.08.1902 – 12.07.1982), \emph{Regisseur, Schauspieler}|pwv}\pwindex{Schnitzler, Lili 13.09.1909 – 26.07.1928@\textsc{Schnitzler, Lili} (13.09.1909 – 26.07.1928)|pwv}! Ihr\pend
           \pstart \spacefill\mbox{Richard}\pend{}\endnumbering\briefempfaengerindex{Schnitzler, Arthur@\textsc{Schnitzler, Arthur}!zzzBeer-Hofmann, Richard@\emph{von Richard Beer-Hofmann}!1917-07-181@{18. 7. 1917}|)be}\mylabel{h}\end{ledgroupsized}  \newcommand{\dateiname}{L02266}\newcommand{\titel}{Richard Beer-Hofmann an Arthur Schnitzler, 18. 7. 1917}\newcommand{\editorInnen}{Martin Anton Müller und Gerd-Hermann Susen}
            \footnotesize
\begin{ledgroupsized}[t]{11.5cm}
\doendnotes{C}
\end{ledgroupsized}
         %% latex-leseansicht-abspann.tex
%% Abspann für die Leseansicht.
%% Der Schalter \ifkorrekturansicht ist bereits durch den Vorspann gesetzt.

%% latex-abspann.tex
%% Gemeinsamer Abspann für Korrekturansicht und Leseansicht.
%% Setzt den Schalter \ifkorrekturansicht voraus (gesetzt in den
%% einbindenden Dateien latex-korrekturansicht-abspann.tex bzw.
%% latex-leseansicht-abspann.tex).
%% ---------------------------------------------------------------

\normalsize

% Das esempio-Environment wird nur in der Leseansicht benötigt
\ifkorrekturansicht\else
\newenvironment{esempio}[3]%
{
    \vspace{1.5ex}
    \rlap{\underline{#1}}
    \par
    \setlength{\parindent}{0cm}
    \nopagebreak
    \leftskip=#2cm
    \rightskip=#3cm
}
{
    \par
}
\fi

\doendnotes{C}
\bigskip
\vfill

\clearpage

\footnotesize

\ifkorrekturansicht
  \lohead{\textsc{register}}
\fi

% theindex-Environment neu definieren ohne reledmac
\makeatletter
\renewenvironment{theindex}{%
  \ifkorrekturansicht
    \section*{\indexname}%
  \else
    \subsubsection*{Index der erwähnten Entitäten}%
  \fi
  \setlength{\parindent}{0pt}%
  \setlength{\parskip}{0pt plus 0.3pt}%
  \let\item\@idxitem
}{%
  \ifkorrekturansicht\clearpage\fi
}
\makeatother

\IfFileExists{\jobname-pw.ind}{\input{\jobname-pw.ind}}{}

% Quellenangabe nur in der Leseansicht
\ifkorrekturansicht\else
% Fallback-Definitionen, falls die .tex-Datei \titel etc. nicht gesetzt hat
\providecommand{\titel}{}
\providecommand{\editorInnen}{}
\providecommand{\dateiname}{\jobname}

\vspace{3cm}

\vfill

\footnotesize
\textsc{Quelle}: \titel. Herausgegeben von {\editorInnen}. In: \emph{Arthur Schnitzler: Briefwechsel mit Autorinnen und Autoren}.
 Digitale Edition, https://schnitzler-briefe.acdh.oeaw.ac.at/{\dateiname}.html (Stand \today)
\fi

\end{document}


      