%% latex-korrekturansicht-vorspann.tex
%% Vorspann für die Korrekturansicht.
%% Lädt die gemeinsame Datei latex-vorspann.tex mit gesetztem Schalter.

\newif\ifkorrekturansicht
\korrekturansichttrue

\input{../tex-inputs/latex-vorspann}


\section[Hugo von Hofmannsthal an Arthur Schnitzler, {[}9. 8. 1897{]}]{L00715 Hugo von Hofmannsthal an Arthur Schnitzler, {[}9. 8. 1897{]}}
\nopagebreak\mylabel{L00715v}
\rehead{ }\normalsize\beginnumbering\briefempfaengerindex{Schnitzler, Arthur@\textsc{Schnitzler, Arthur}!zzzHofmannsthal, Hugo von@\emph{von Hugo von Hofmannsthal}!1897-08-091@{{[}9. 8. 1897{]}}|(be}
\toendnotes[C]{\smallbreak\pagebreak[2]}\Standort{CUL, Schnitzler, B 43.}
\physDesc{Telegramm, 163 Zeichen
\newline{}maschinell
\newline{}Ordnung: von unbekannter Hand nummeriert: »103« }
\buchAbdrucke{\weitereDrucke{Hugo von Hofmannsthal, Arthur Schnitzler: \emph{Briefwechsel}. Frankfurt am Main: \emph{S. Fischer} 1964, S. 95.} }\toendnotes[C]{\smallbreak}
\pstart
           \noindent{}{\pb}win\oindex{Wien@\textbf{Wien}, \emph{A.ADM2}|pw} fr salzburg\oindex{Salzburg@\textbf{Salzburg}, \emph{A.ADM2}|pw}
                  1†1376{ }28{ }11 30m =\pend
           
\pstart
           bitte instaendig \label{T_L00715-1v}\edtext{andrian\pwindex{Andrian-Werburg, Leopold von 09.05.1875 – 19.11.1951@\textsc{Andrian-Werburg, Leopold von} (09.05.1875 – 19.11.1951), \emph{Schriftsteller/Schriftstellerin, Diplomat/Diplomatin}|pw}}{\lemma{\textnormal{\emph{andrian}}}\Cendnote{\textnormal{korrigiert aus:
                  »andrien«}}}\label{T_L00715-1} unbedingt heute 9 uhr abend{ }habsburgergasse 5\oindex{Habsburgergasse@\textbf{Habsburgergasse}, \emph{Straße (K.STR)}|pw} besuchen und ihm zu helfen
               suchen sonst mueste ich \label{K_L00715-1v}\edtext{nach wien\oindex{Wien@\textbf{Wien}, \emph{A.ADM2}|pw}}{\lemma{\textnormal{\emph{nach wien}}}\Cendnote{\textnormal{Ein Brief Hofmannsthals\pwindex{Hofmannsthal, Hugo von 1874-02-01 – 1929-07-15@\textsc{Hofmannsthal, Hugo von} (1874-02-01 – 1929-07-15), \emph{Schriftsteller/Schriftstellerin}|pwk} an seine Eltern\pwindex{Hofmannsthal, Anna von 27.01.1849 – 22.03.1904@\textsc{Hofmannsthal, Anna von} (27.01.1849 – 22.03.1904)|pwk}\pwindex{Hofmannsthal, Hugo August von 21.12.1841 – 08.12.1915@\textsc{Hofmannsthal, Hugo August von} (21.12.1841 – 08.12.1915), \emph{Bankdirektor/Bankdirektorin}|pwk} vom selben Tag erwähnt das Telegramm und
                  erlaubt die Datierung.}}}\label{K_L00715-1}\spacefill\mbox{= hugo =}\pend
           \selectlanguage{ngerman}\endnumbering\briefempfaengerindex{Schnitzler, Arthur@\textsc{Schnitzler, Arthur}!zzzHofmannsthal, Hugo von@\emph{von Hugo von Hofmannsthal}!1897-08-091@{{[}9. 8. 1897{]}}|)be}\mylabel{L00715h}  \normalsize

\doendnotes{C}
\bigskip
\vfill

\clearpage

\footnotesize

\lohead{\textsc{register}}

% Definiere theindex-Environment komplett neu ohne reledmac
\makeatletter
\renewenvironment{theindex}{%
  \section*{\indexname}%
  \setlength{\parindent}{0pt}%
  \setlength{\parskip}{0pt plus 0.3pt}%
  \let\item\@idxitem
}{%
  \clearpage
}
\makeatother

\IfFileExists{\jobname-pw.ind}{\input{\jobname-pw.ind}}{}

\end{document}

      