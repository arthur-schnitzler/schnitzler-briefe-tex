%% latex-leseansicht-vorspann.tex
%% Vorspann für die Leseansicht.
%% Lädt die gemeinsame Datei latex-vorspann.tex mit nicht gesetztem Schalter.

\newif\ifkorrekturansicht
\korrekturansichtfalse

\input{../tex-inputs/latex-vorspann}


\section[Elsa Ginsberg-Plessner an Arthur Schnitzler, 9. 1. 1916]{L03730 Elsa Ginsberg-Plessner an Arthur Schnitzler, 9. 1. 1916}
\nopagebreak\mylabel{L03730v}
\rehead{ }\normalsize\beginnumbering\briefempfaengerindex{Schnitzler, Arthur@\textsc{Schnitzler, Arthur}!zzzPlessner, Elsa@\emph{von Elsa Plessner}!1916-01-091@{9. 1. 1916}|(be}
\toendnotes[C]{\smallbreak\pagebreak[2]}
\correspDesc{Versand  durch Elsa Plessner am 9. 1. 1916 in München
\newline{}Erhalt  durch Arthur Schnitzler im Zeitraum [10. 1. 1916
                  – 12. 1. 1916?] in Wien}\toendnotes[C]{\smallbreak}
\Standort{DLA, A:Schnitzler, HS.1985.1.419.}
\physDesc{Brief, 2 Blätter, 3 Seiten, 4400 Zeichen
\newline{}Handschrift: schwarze Tinte, lateinische Kurrent
\newline{}Schnitzler: 1) mit rotem Buntstift zwei Unterstreichungen  2) mit Bleistift »\textsc{Plessner}«}\toendnotes[C]{\smallbreak}
\pstart
           {\pb}München, Theresienstraße 78,\hspace*{1.5em} Pension Serno\oindex{Pension Serno@\textbf{Pension Serno}, \emph{Beherbergungsgebäude}|pw}\hspace*{1.5em}9. I. 16\pend
           
\pstart\center{}Hochverehrter Herr Doctor!\pend\vspace{0.5em}
\pstart
           Nach vierzehn Jahren geschieht es mir heute zum erstenmale wieder, dass
               ich Ihnen eine \label{K_L03730-1v}\edtext{neue Arbeit\pwindex{Plessner, Elsa 22.\,8.\,1875 Wien – 7.\,5.\,1932 Alicante@\textsc{Plessner, Elsa} (22.\,8.\,1875 Wien – 7.\,5.\,1932 Alicante), \emph{Schriftstellerin}!Musik@\strich\emph{Musik}|pwv}}{\lemma{\textnormal{\emph{neue Arbeit}}}\Cendnote{\textnormal{Das dem Brief beiliegende Schauspiel \emph{Musik}\pwindex{Plessner, Elsa 22.\,8.\,1875 Wien – 7.\,5.\,1932 Alicante@\textsc{Plessner, Elsa} (22.\,8.\,1875 Wien – 7.\,5.\,1932 Alicante), \emph{Schriftstellerin}!Musik@\strich\emph{Musik}|pwk} ist nicht überliefert.}}}\label{K_L03730-1}
               vorzulegen habe, wirklich die erste und einzige seit ganzen vierzehn Jahren, während
               deren ich auch nicht eine Zeile geschrieben habe – – Briefe ausgenommen. – Ich habe
               in dieser Zeit nur ein bisschen gelebt und viel gesungen und nicht einmal mehr den
               Wunsch gefühlt, etwas niederzuschreiben. – Sie werden mich durchaus verändert finden
               und nur Ihr feines Gefühl wird die Linie nachziehen können, die vom »ersten Capitel\pwindex{Plessner, Elsa 22.\,8.\,1875 Wien – 7.\,5.\,1932 Alicante@\textsc{Plessner, Elsa} (22.\,8.\,1875 Wien – 7.\,5.\,1932 Alicante), \emph{Schriftstellerin}!erste Kapitel. Schauspiel in drei Akten@\strich\emph{Das erste Kapitel. Schauspiel in drei Akten}|pw}« zu »Musik\pwindex{Plessner, Elsa 22.\,8.\,1875 Wien – 7.\,5.\,1932 Alicante@\textsc{Plessner, Elsa} (22.\,8.\,1875 Wien – 7.\,5.\,1932 Alicante), \emph{Schriftstellerin}!Musik@\strich\emph{Musik}|pw}« führte. Diese \label{K_L03730-2v}\edtext{Arbeit\pwindex{Plessner, Elsa 22.\,8.\,1875 Wien – 7.\,5.\,1932 Alicante@\textsc{Plessner, Elsa} (22.\,8.\,1875 Wien – 7.\,5.\,1932 Alicante), \emph{Schriftstellerin}!Musik@\strich\emph{Musik}|pwv} ist die Summe}{\lemma{\textnormal{\emph{Arbeit ist die Summe}}}\Cendnote{\textnormal{Schnitzler teilte die positive Einschätzung
                  nicht. Er kommentierte Plessners\pwindex{Plessner, Elsa 22.\,8.\,1875 Wien – 7.\,5.\,1932 Alicante@\textsc{Plessner, Elsa} (22.\,8.\,1875 Wien – 7.\,5.\,1932 Alicante), \emph{Schriftstellerin}|pwk}{ }Stück\pwindex{Plessner, Elsa 22.\,8.\,1875 Wien – 7.\,5.\,1932 Alicante@\textsc{Plessner, Elsa} (22.\,8.\,1875 Wien – 7.\,5.\,1932 Alicante), \emph{Schriftstellerin}!Musik@\strich\emph{Musik}|pwkv} und den vorliegenden
                  Brief im \emph{Tagebuch}\pwindex{Schnitzler, Arthur 15. 5. 1862 Wien – 21. 10. 1931 ebd.@\textsc{Schnitzler, Arthur} (15. 5. 1862 Wien – 21. 10. 1931 ebd.), \emph{Schriftsteller, Mediziner}!Tagebuch@\strich\emph{Tagebuch}|pwk}: »Las Nm. ein
                     schlechtes Buch von Fr. Plessner, Mscrpt. aus München geschickt, mit
                     eingebildetem Brief.«, A. S.: \emph{Tagebuch}, 16. 1. 1916.}}}\label{K_L03730-2} dessen, was ich leisten kann und zu sagen habe –
               bis heute, und musste geschrieben werden. Diesmal wirklich das berühmte »Muss«. –
               Daher weiß ich auch mit merkwürdiger Bestimmtheit, dass die Arbeit\pwindex{Plessner, Elsa 22.\,8.\,1875 Wien – 7.\,5.\,1932 Alicante@\textsc{Plessner, Elsa} (22.\,8.\,1875 Wien – 7.\,5.\,1932 Alicante), \emph{Schriftstellerin}!Musik@\strich\emph{Musik}|pwv} nicht vergebens war. Sowas fühlt
               man entweder – oder man fühlt es nicht. Sollte ich mich darüber dennoch täuschen, so
               ist für mich kein Verlass mehr auf irgend etwas in der Welt.  –\pend
           
\pstart
           Zu dem Stück\pwindex{Plessner, Elsa 22.\,8.\,1875 Wien – 7.\,5.\,1932 Alicante@\textsc{Plessner, Elsa} (22.\,8.\,1875 Wien – 7.\,5.\,1932 Alicante), \emph{Schriftstellerin}!Musik@\strich\emph{Musik}|pwv} selbst habe ich
               zu bemerken, dass es mir damit seltsam ergangen ist. Zu Anfang stand ich auf festem
               Boden – beinahe etwas zu viel »\label{K_L03730-3v}\edtext{\begin{otherlanguage}{french}terre a terre\end{otherlanguage}}{\lemma{\textnormal{\emph{terre a terre}}}\Cendnote{\textnormal{französisch: bodenständig}}}\label{K_L03730-3}«. In
               der Hälften des zweiten Actes begann sich aber mit der Situation und Stimmung
               unwillkürlich der Ton des Ganzen zu verändern und zu heben – – und ich konnte mit der
               größten Mühe kaum den Vers zurückdrängen, der sich mir aufzwingen wollte. Ich sah
               mich plötzlich mitten in der Arbeit ganz unvermuthet vor ein Stilproblem gestellt,
               auf das ich nicht im Geringsten vorbereitet war – was gewiss die Anschauung
               bestätigt, dass {\pb}Frauenarbeit letzten Endes doch immer
               Improvisation bleibt – mag Sie vorher noch so gründlich durchdacht worden sein. – –
               Ich war gezwungen auf einer Linie weiterzugehen, die etwa die Resultirende zwischen
               Conversationsstück und Stildrama sein dürfte, und konnte mich dabei nur auf meinen
               Instinct verlassen. Daraus ist theilweise eine merkwürdige, hauptsächlich auf
               Rhythmus gestellte, Diction entstanden, auf Grund von mir allein fühlbaren \introOben{}musikalischen\introOben{} Gesetzen – und außerdem sehr beschränkt in der
               Wahl der Worte. Denn kein einziges Wort durfte mir unterlaufen, das in unserer
               Umgangssprache nicht gebräuchlich wäre. Sogar Arzt und Diener mussten sich in dieser
               Form ausdrücken können. Ich glaube, diese Diction ist neu – und ich hoffe, dass sie
               auch geglückt ist. – Sie werden ein \label{K_L03730-55v}\edtext{Motiv in der Arbeit\pwindex{Plessner, Elsa 22.\,8.\,1875 Wien – 7.\,5.\,1932 Alicante@\textsc{Plessner, Elsa} (22.\,8.\,1875 Wien – 7.\,5.\,1932 Alicante), \emph{Schriftstellerin}!Musik@\strich\emph{Musik}|pwv} finden,
               das Sie selbst in der Stunde des Erkennens\pwindex{Schnitzler, Arthur 15. 5. 1862 Wien – 21. 10. 1931 ebd.@\textsc{Schnitzler, Arthur} (15. 5. 1862 Wien – 21. 10. 1931 ebd.), \emph{Schriftsteller, Mediziner}!Stunde des Erkennens@\strich\emph{Stunde des Erkennens}|pw}}{\lemma{\textnormal{\emph{Motiv … Erkennens}}}\Cendnote{\textnormal{Schnitzlers Einakter \emph{Stunde des
                     Erkennens}\pwindex{Schnitzler, Arthur 15. 5. 1862 Wien – 21. 10. 1931 ebd.@\textsc{Schnitzler, Arthur} (15. 5. 1862 Wien – 21. 10. 1931 ebd.), \emph{Schriftsteller, Mediziner}!Stunde des Erkennens@\strich\emph{Stunde des Erkennens}|pwk} behandelt ein Ehepaar, das sich zehn Jahre später über eine
                  Affäre der Frau ausspricht. }}}\label{K_L03730-55} gestreift haben. Ich weiß, es ist unnötig,
               Ihnen zu versichern, dass ich mich nicht an Ihrem Eigenthum vergriffen habe, denn die
               Grundlagen meiner Arbeit\pwindex{Plessner, Elsa 22.\,8.\,1875 Wien – 7.\,5.\,1932 Alicante@\textsc{Plessner, Elsa} (22.\,8.\,1875 Wien – 7.\,5.\,1932 Alicante), \emph{Schriftstellerin}!Musik@\strich\emph{Musik}|pwv}
               stehen schon lange fest. Auch glaube ich, dass Sie, verehrter Herr Doctor, der ein
               wenig von meinem Leben weiß, sich selbst sagen können, auf welchem Wege auch ich zu
               diesem Motiv gelangen konnte. – – – –\pend
           
\pstart
           Das ganze Stück\pwindex{Plessner, Elsa 22.\,8.\,1875 Wien – 7.\,5.\,1932 Alicante@\textsc{Plessner, Elsa} (22.\,8.\,1875 Wien – 7.\,5.\,1932 Alicante), \emph{Schriftstellerin}!Musik@\strich\emph{Musik}|pwv} handelt von
               Musik – hörbarer und blos fühlbarer, und ist in Aufbau, Melodik und Klangfarbe
               irgendwie nach den Gesetzen der Musik entstanden. Die harte unbarmherzige Lösung hat
               mich selbst furchtbar erschreckt, als sie mir zuerst aufging. Später wusste ich, dass
               sie die einzig folgerichtige und gerechte sei. Die Sünde gegen den heiligen Geist ist
               die unverzeihliche. –\pend
           
\pstart
           Mehr will ich Ihnen für heute nicht sagen. Sie werden mich ohnedies schon für
               übergeschnappt halten – oder für bodenlos unverschämt. Ich hoffe nicht, dass ich das
               bin. {\pb}Was ich aber bin, verehrter Herr Doctor, brauche
               ich Ihnen nicht erst zu sagen – – – maßlos gespannt, Ihre Meinung über meine Arbeit\pwindex{Plessner, Elsa 22.\,8.\,1875 Wien – 7.\,5.\,1932 Alicante@\textsc{Plessner, Elsa} (22.\,8.\,1875 Wien – 7.\,5.\,1932 Alicante), \emph{Schriftstellerin}!Musik@\strich\emph{Musik}|pwv} zu erfahren, die ich
               von Ihnen mit Rechte einer alten Gewohnheit schlankweg erwarte. Frech, nicht
               wahr? – –\pend
           
\pstart
           Wenn Sie noch so gütig sind, wie vor vierzehn Jahren – und ich habe keinen Grund,
               daran zu zweifeln – – so werden Sie mich nur so lange darauf warten lassen, als
               unbedingt nöthig ist. Schließlich noch die kleine Bemerkung, dass Sie, wie vor langer
               Zeit, auch jetzt wieder der Erste sind, dem meine neue Arbeit\pwindex{Plessner, Elsa 22.\,8.\,1875 Wien – 7.\,5.\,1932 Alicante@\textsc{Plessner, Elsa} (22.\,8.\,1875 Wien – 7.\,5.\,1932 Alicante), \emph{Schriftstellerin}!Musik@\strich\emph{Musik}|pwv} vorliegt. Jung gewohnt – alt
               gethan. Das »alt« bitte \label{K_L03730-4v}\edtext{cum grano salis}{\lemma{\textnormal{\emph{cum grano salis}}}\Cendnote{\textnormal{lateinisch: mit einem Korn Salz; im
                  Sinne von: nicht in jeder Hinsicht so gemeint}}}\label{K_L03730-4}. –\pend
           
\pstart
           Wollen Sie mich Ihrer Frau Gemahlin\pwindex{Schnitzler, Olga 17.\,1.\,1882 Wien – 13.\,1.\,1970 Lugano@\textsc{Schnitzler, Olga} (17.\,1.\,1882 Wien – 13.\,1.\,1970 Lugano), \emph{Schauspielerin, Sängerin}|pwv} bestens empfehlen und selber von mir die Versicherung dankbarster
               Verehrung entgegennehmen.{\\[\baselineskip]}\spacefill\mbox{Else Ginsberg-Plessner.}\pend
           \leftskip=0em{}
\pstart
           \noindent{}P. S. Der Gegenstand meines \label{K_L03730-5v}\edtext{letzten
                     Briefes}{\lemma{\textnormal{\emph{letzten
                     Briefes}}}\Cendnote{\textnormal{XXXX Auszeichnungsfehler: Dokument L03729 nicht gefunden. }}}\label{K_L03730-5} hat sich von selbst
                  erledigt, da Sie nicht nach München\oindex{München@\textbf{München}|pw} kamen.
                  Ich hoffe, dass dies die alleinige Ursache davon war, dass mein Brief – falls Sie ihn
                  überhaupt erhalten haben – unbeantwortet geblieben ist. – Oder waren Sie gar bös
                  auf mich? – – –\pend
           \selectlanguage{ngerman}\endnumbering\briefempfaengerindex{Schnitzler, Arthur@\textsc{Schnitzler, Arthur}!zzzPlessner, Elsa@\emph{von Elsa Plessner}!1916-01-091@{9. 1. 1916}|)be}\mylabel{L03730h}  \newcommand{\dateiname}{L03730}\newcommand{\titel}{Elsa Ginsberg-Plessner an Arthur Schnitzler, 9. 1. 1916}\newcommand{\editorInnen}{Selma Jahnke und Martin Anton Müller}%% latex-leseansicht-abspann.tex
%% Abspann für die Leseansicht.
%% Der Schalter \ifkorrekturansicht ist bereits durch den Vorspann gesetzt.

%% latex-abspann.tex
%% Gemeinsamer Abspann für Korrekturansicht und Leseansicht.
%% Setzt den Schalter \ifkorrekturansicht voraus (gesetzt in den
%% einbindenden Dateien latex-korrekturansicht-abspann.tex bzw.
%% latex-leseansicht-abspann.tex).
%% ---------------------------------------------------------------

\normalsize

% Das esempio-Environment wird nur in der Leseansicht benötigt
\ifkorrekturansicht\else
\newenvironment{esempio}[3]%
{
    \vspace{1.5ex}
    \rlap{\underline{#1}}
    \par
    \setlength{\parindent}{0cm}
    \nopagebreak
    \leftskip=#2cm
    \rightskip=#3cm
}
{
    \par
}
\fi

\doendnotes{C}
\bigskip
\vfill

\clearpage

\footnotesize

\ifkorrekturansicht
  \lohead{\textsc{register}}
\fi

% theindex-Environment neu definieren ohne reledmac
\makeatletter
\renewenvironment{theindex}{%
  \ifkorrekturansicht
    \section*{\indexname}%
  \else
    \subsubsection*{Index der erwähnten Entitäten}%
  \fi
  \setlength{\parindent}{0pt}%
  \setlength{\parskip}{0pt plus 0.3pt}%
  \let\item\@idxitem
}{%
  \ifkorrekturansicht\clearpage\fi
}
\makeatother

\IfFileExists{\jobname-pw.ind}{\input{\jobname-pw.ind}}{}

% Quellenangabe nur in der Leseansicht
\ifkorrekturansicht\else
% Fallback-Definitionen, falls die .tex-Datei \titel etc. nicht gesetzt hat
\providecommand{\titel}{}
\providecommand{\editorInnen}{}
\providecommand{\dateiname}{\jobname}

\vspace{3cm}

\vfill

\footnotesize
\textsc{Quelle}: \titel. Herausgegeben von {\editorInnen}. In: \emph{Arthur Schnitzler: Briefwechsel mit Autorinnen und Autoren}.
 Digitale Edition, https://schnitzler-briefe.acdh.oeaw.ac.at/{\dateiname}.html (Stand \today)
\fi

\end{document}


