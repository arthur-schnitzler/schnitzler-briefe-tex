%% latex-korrekturansicht-vorspann.tex
%% Vorspann für die Korrekturansicht.
%% Lädt die gemeinsame Datei latex-vorspann.tex mit gesetztem Schalter.

\newif\ifkorrekturansicht
\korrekturansichttrue

\input{../tex-inputs/latex-vorspann}


\section[Elsa Plessner an Arthur Schnitzler, 9. 1. 1916]{L03730 Elsa Plessner an Arthur Schnitzler, 9. 1. 1916}
\nopagebreak\mylabel{L03730v}
\rehead{ }\normalsize\beginnumbering\briefempfaengerindex{Schnitzler, Arthur@\textsc{Schnitzler, Arthur}!zzzPlessner, Elsa@\emph{von Elsa Plessner}!1916-01-091@{9. 1. 1916}|(be}
\toendnotes[C]{\smallbreak\pagebreak[2]}\Standort{DLA, A:Schnitzler, HS.1985.1.419.}
\physDesc{Brief,  Blätter, 3 Seiten, 4389 Zeichen
\newline{}Handschrift: , lateinische Kurrent
\newline{}Schnitzler: zwei Unterstreichungen }\toendnotes[C]{\smallbreak}
\pstart
           {\pb}München, Theresienstraße 78, Pension
                     Serno\oindex{Pension Serno@\textbf{Pension Serno}, \emph{Beherbergungsgebäude (K.BHB)}|pw}{ }9. I. 16\pend
           
\pstart{}Hochverehrter Herr Doctor!\pend\vspace{0.5em}
\pstart
           Nach vierzehn Jahren geschieht es mir heute zum erstenmale wieder, dass
               ich Ihnen eine \label{K_L03730-1v}\edtext{neue Arbeit\pwindex{Musik@\emph{Musik}|pwv}}{\lemma{\textnormal{\emph{neue Arbeit}}}\Cendnote{\textnormal{Das dem Brief beiliegende Schauspiel \emph{Musik}\pwindex{Musik@\emph{Musik}|pwk} ist nicht überliefert.}}}\label{K_L03730-1}
               vorzulegen habe, wirklich die erste und einzige seit ganzen vierzehn Jahren, während
               deren ich auch nicht eine Zeile geschrieben habe – – Briefe ausgenommen. – Ich
               habe in dieser Zeit nur ein bißchen gelebt und viel gesungen und nicht einmal mehr
               den Wunsch gefühlt, etwas niederzuschreiben. – Sie werden mich durchaus
               verändert finden und nur Ihr feines Gefühl wird die Linie nachziehen können, die vom
               »ersten Capitel\pwindex{erste Kapitel. Schauspiel in drei Akten@\emph{Das erste Kapitel. Schauspiel in drei Akten}|pw}« zu »Musik\pwindex{Musik@\emph{Musik}|pw}« führte. Diese \label{K_L03730-2v}\edtext{Arbeit\pwindex{Musik@\emph{Musik}|pwv} ist die Summe}{\lemma{\textnormal{\emph{Arbeit ist die Summe}}}\Cendnote{\textnormal{Schnitzler teilte die positive Einschätzung nicht. Er kommentierte Plessners\pwindex{Plessner, Elsa 22.08.1875 – 01.05.1932@\textsc{Plessner, Elsa} (22.08.1875 – 01.05.1932), \emph{Schriftsteller/Schriftstellerin}|pwk}{ }Stück\pwindex{Musik@\emph{Musik}|pwkv} und den vorliegenden Brief im \emph{Tagebuch}\textcolor{red}{\textsuperscript{XXXX indx}}: »Las Nm. ein schlechtes Buch von Fr. Plessner, Mscrpt. aus München geschickt, mit eingebildetem Brief.«, A. S.: \emph{Tagebuch}, 16. 1. 1916.}}}\label{K_L03730-2} dessen, was ich leisten kann und zu
               sagen habe – bis heute, und mußte geschrieben werden. Diesmal wirklich das berühmte
               »Muss« – Daher weiß ich auch mit merkwürdiger Bestimmtheit, dass die Arbeit\pwindex{Musik@\emph{Musik}|pwv} nicht vergebens war. Sowas fühlt
               man entweder – oder man fühlt es nicht. Sollte ich mich darüber dennoch täuschen, so
               ist für mich kein Verlass mehr auf irgend etwas in der Welt.\pend
           
\pstart
           In dem Stück\pwindex{Musik@\emph{Musik}|pwv} selbst habe ich
               zu bemerken, dass es mir damit seltsam ergangen ist. Zu Anfang stand ich auf festem
               Boden, beinahe etwas zu viel »\label{K_L03730-3v}\edtext{\begin{otherlanguage}{french}terre a terre\end{otherlanguage}}{\lemma{\textnormal{\emph{terre a terre}}}\Cendnote{\textnormal{französisch: bodenständig}}}\label{K_L03730-3}«. In der Hälften des zweiten Actes begann sich
               aber mit der Situation und Stimmung unwillkürlich der Ton des Ganzen zu verändern und
               zu heben – – und ich konnte mit der größten Mühe kaum den Vers zurückdrängen, der
               sich mir aufzwingen wollte. Ich sah mich plötzlich mitten in der Arbeit ganz
               unvermuthet vor ein Stilproblem gestellt, auf das ich nicht im Geringsten vorbereitet
               war – was gewiss die Anschauung bestätigt, dass {\pb}Frauenarbeit letzten
               Endes doch immer Improvisation bleibt – mag Sie vorher noch so gründlich durchdacht
               worden sein. Ich war gezwungen auf einer Linie weiterzugehen, die etwa die
               Resultirende zwischen Conversationsstück und Stildrama sein dürfte, und konnte mich
               dabei nur auf meinen Instinct verlassen. Daraus ist theilweise eine merkwürdige,
               hauptsächlich auf Rhythmus gestellte, Diction entstanden, auf Grund von mir allein
               fühlbaren \introOben{}musikalischen\introOben{} Gesetzen – und außerdem sehr beschränkt in der
               Wahl der Worte. Denn kein einziges Wort durfte mir unterlaufen, das in unserer
               Umgangssprache nicht gebräuchlich wäre. Sogar Arzt und Diener mussten sich in dieser
               Form ausdrücken können. Ich glaube, diese Diction ist neu – und ich hoffe, dass sie
               auch geglückt ist. – Sie werden ein Motiv in der Arbeit\pwindex{Musik@\emph{Musik}|pwv} finden, das Sie selbst in der Stunde des Erkennens\pwindex{Stunde des Erkennens@\emph{Stunde des Erkennens}|pw}
               gestreift haben. Ich weiß, es ist unnötig, Ihnen zu versichern, dass ich mich nicht
               an Ihrem Eigenthum vergriffen habe, denn die Grundlagen meiner Arbeit\pwindex{Musik@\emph{Musik}|pwv} stehen schon lange fest. Auch glaube ich, dass Sie,
               verehrter Herr Doctor, der ein wenig von meinem Leben weiß, sich selbst sagen können,
               auf welchem Wege auch ich zu diesem Motiv gelangen konnte. – – – –\pend
           
\pstart
           Das ganze Stück\pwindex{Musik@\emph{Musik}|pwv} handelt von Musik –
               hörbarer und blos fühlbarer, und ist in Aufbau, Melodik und Klangfarbe irgendwie nach
               den Gesetzen der Musik entstanden. Die harte unbarmherzige Lösung hat mich selbst
               furchtbar erschreckt, als sie nur zuerst aufging. Später wusste ich, dass sie die
               einzig folgerichtige und gerechte sei. Die Sünde gegen den heiligen Geist ist die
               unverzeihliche. –\pend
           
\pstart
           Mehr will ich Ihnen für heute nicht sagen. Sie werden mich ohnedies schon für
               übergeschnappt halten – oder für bodenlos unverschämt. Ich hoffe nicht, dass ich das
               bin. {\pb}Was ich aber bin, verehrter Herr Doctor, brauche ich Ihnen nicht
               erst zu sagen – – – maßlos gespannt, Ihre Meinung über meine Arbeit\pwindex{Musik@\emph{Musik}|pwv} zu erfahren, die ich von Ihnen
               mit Rechte einer alten Gewohnheit schlankweg erwarte. Frech, nicht wahr? – – 
               Wenn Sie noch so gütig sind, wie vor vierzehn Jahren – und ich habe keinen Grund,
               daran zu zweifeln – – so werden Sie mich nur so lange darauf warten lassen, als
               unbedingt nöthig ist. Schließlich noch die kleine Bemerkung, dass Sie, wie vor langer
               Zeit, auch jetzt wieder der Erste sind, dem meine neue Arbeit\pwindex{Musik@\emph{Musik}|pwv} vorliegt. Jung gewohnt – alt gethan. Das »alt« bitte
               \label{K_L03730-4v}\edtext{cum grano salis}{\lemma{\textnormal{\emph{cum grano salis}}}\Cendnote{\textnormal{latein: mit einem Korn Salz, (übertr.) mit gewissen Einschränkungen}}}\label{K_L03730-4}. –\pend
           
\pstart
           Wollen Sie mich Ihrer Frau Gemahlin\pwindex{Schnitzler, Olga 17.01.1882 – 13.01.1970@\textsc{Schnitzler, Olga} (17.01.1882 – 13.01.1970), \emph{Schauspieler/Schauspielerin, Sänger/Sängerin}|pwv} bestens empfehlen und selber von mir die Versicherung dankbarster
               Verehrung entgegennehmen.{\\[\baselineskip]}\spacefill\mbox{Else Ginsberg-Plessner}\pend
           \leftskip=0em{}
\pstart
           \noindent{}P. S. Der Gegenstand meines \label{K_L03730-5v}\edtext{letzten Briefes}{\lemma{\textnormal{\emph{letzten Briefes}}}\Cendnote{\textnormal{Elsa Plessner an Arthur Schnitzler, 15. 11. 1915.
                  }}}\label{K_L03730-5} hat sich von selbst erledigt, da
                  Sie nicht nach München\oindex{Muenchen@\textbf{München}, \emph{P.PPLA}|pw} kamen. Ich hoffe,
                  dass dies die alleinige Ursache davon war, dass mein \label{K_L03730-6v}\edtext{Brief}{\lemma{\textnormal{\emph{Brief}}}\Cendnote{\textnormal{Elsa Plessner an Arthur Schnitzler, 15. 11. 1915.
                  }}}\label{K_L03730-6} – falls Sie ihn überhaupt
                  erhalten haben – unbeantwortet geblieben ist. – Oder waren Sie gar bös auf
                  mich? – – –\pend
           \selectlanguage{ngerman}\endnumbering\briefempfaengerindex{Schnitzler, Arthur@\textsc{Schnitzler, Arthur}!zzzPlessner, Elsa@\emph{von Elsa Plessner}!1916-01-091@{9. 1. 1916}|)be}\mylabel{L03730h}
\begin{anhang}
\end{anhang}\normalsize

\doendnotes{C}
\bigskip
\vfill

\clearpage

\footnotesize

\lohead{\textsc{register}}

% Definiere theindex-Environment komplett neu ohne reledmac
\makeatletter
\renewenvironment{theindex}{%
  \section*{\indexname}%
  \setlength{\parindent}{0pt}%
  \setlength{\parskip}{0pt plus 0.3pt}%
  \let\item\@idxitem
}{%
  \clearpage
}
\makeatother

\IfFileExists{\jobname-pw.ind}{\input{\jobname-pw.ind}}{}

\end{document}

      