%% latex-korrekturansicht-vorspann.tex
%% Vorspann für die Korrekturansicht.
%% Lädt die gemeinsame Datei latex-vorspann.tex mit gesetztem Schalter.

\newif\ifkorrekturansicht
\korrekturansichttrue

\input{../tex-inputs/latex-vorspann}


\section[Arthur Schnitzler an Hugo von Hofmannsthal, 15. 7. 1898]{L00820 Arthur Schnitzler an Hugo von Hofmannsthal, 15. 7. 1898}
\nopagebreak\mylabel{L00820v}
\rehead{ }\normalsize\beginnumbering\briefempfaengerindex{Hofmannsthal, Hugo von@\textsc{Hofmannsthal, Hugo von}!zzzSchnitzler, Arthur@\emph{von Arthur Schnitzler}!1898-07-152@{15. 7. 1898}|(be}
\toendnotes[C]{\smallbreak\pagebreak[2]}\Standort{FDH, Hs-30885,70.}
\physDesc{Brief, 1 Blatt, 3 Seiten, 1337 Zeichen
\newline{}Handschrift: Bleistift, deutsche Kurrent}
\buchAbdrucke{\weitereDrucke{Hugo von Hofmannsthal, Arthur Schnitzler: \emph{Briefwechsel}. Frankfurt am Main: \emph{S. Fischer} 1964, S. 105106.} }\toendnotes[C]{\smallbreak}
\pstart
           \raggedleft{}{\pb}Graz\oindex{Graz@\textbf{Graz}, \emph{A.ADM2}|pw}, Freitag{\\}15/7 98\pend
           \vspace{0.5em}
\pstart
           Mein lieber Hugo, meine Abſicht iſt, So{\geminationn}tag von hier fortzureiſen; dann zu Bahn, Rad,
               Wagen weiter, vielleicht ko{\geminationm} ich in die Fuſch\oindex{Bad Fusch@\textbf{Bad Fusch}, \emph{A.ADM3}|pw}, da ſeh ich wohl noch Ihre Eltern\pwindex{Hofmannsthal, Hugo August von 21.12.1841 – 08.12.1915@\textsc{Hofmannsthal, Hugo August von} (21.12.1841 – 08.12.1915), \emph{Bankdirektor/Bankdirektorin}|pwv}\pwindex{Hofmannsthal, Anna von 27.01.1849 – 22.03.1904@\textsc{Hofmannsthal, Anna von} (27.01.1849 – 22.03.1904)|pwv}, Do{\geminationn}erſtag 21.{ }\introOben{}Bad\introOben{} Gaſtein, \textsc{Villa Wassing}\oindex{Villa Dr. Wassing@\textbf{Villa Dr. Wassing}, \emph{Sanatorium (K.SAN)}|pw}, \uline{dort treffen mich Nachrichten bis 23.} (Bei meiner Mama\pwindex{Schnitzler, Louise 1840-07-08 – 1911-09-09@\textsc{Schnitzler, Louise} (1840-07-08 – 1911-09-09)|pwv}).
                  \introOben{}(Alſo nicht offne Karte!)\introOben{} – Da{\geminationn}{ }ſchlängle ich mich allmählich nach Salzburg\oindex{Salzburg@\textbf{Salzburg}, \emph{A.ADM2}|pw} – und weiteres hören Sie noch. – Die Zeit hier vergeht
               leidlich, wenn auch nicht ganz nach meiner Laune; zum Familienleben, {\pb}ſelbſt in mäßigem Umfang bin ich nicht geboren. Auch ſind
               jetzt die Zuſtände durch die merkwürdige Vermengung von illegitimem und anerkanntem,
               Einſicht und Halbheit, ganz unruhig.\pend
           
\pstart
           Zum Arbeiten bin ich gar nicht geko{\geminationm}en; mit einer ſehr
               lebhaften Sehnſucht ruft es mich zu meinem neuen Stück\pwindex{Schleier der Beatrice. Schauspiel in fuenf Akten@\emph{Der Schleier der Beatrice. Schauspiel in fünf Akten}|pwv} – und doch werd ich vorher wahrſcheinlich was anderes
               ſchreiben. Die alte Skizze vom »Sohn\pwindex{Sohn. Aus den Papieren eines Arztes@\emph{Der Sohn. Aus den Papieren eines Arztes}|pw}«
               (Muttermörder) geſtaltet ſich in mir zu irgendwas aus, was beinah {\pb}ein Roman\pwindex{Therese. Chronik eines Frauenlebens@\emph{Therese. Chronik eines Frauenlebens}|pwv}{ }ſein könnte. – Daſs ich von Wien\oindex{Wien@\textbf{Wien}, \emph{A.ADM2}|pw} fort bin, iſt mir recht; daſs es von hier aus bald weiter
               geht, nicht minder. Das Radeln macht mir Freude.\pend
           
\pstart
           Warum ſchreiben Sie mir in Ihrem letzten \introOben{}(vom
                  12.)\introOben{} nicht, wie’s Ihnen geht? Das hoff ich, wenn auch nur mit
               ein paar Zeilen, in Gaſtein\oindex{Bad Gastein@\textbf{Bad Gastein}, \emph{P.PPLA3}|pw} zu erfahren. Richard\pwindex{Beer-Hofmann, Richard 1866-07-11 – 1945-09-26@\textsc{Beer-Hofmann, Richard} (1866-07-11 – 1945-09-26), \emph{Schriftsteller/Schriftstellerin}|pw}{ }ſchrieb mir kurz, ohne beſti{\geminationm}te Zuſage, nicht wohlgelaunt.\pend
           
\pstart
           Laſſen Sie uns auf ein ſchönes Wiederſehen hoffen. Von Herzen Ihr
                  \spacefill\mbox{Arthur}\pend
           \selectlanguage{ngerman}\endnumbering\briefempfaengerindex{Hofmannsthal, Hugo von@\textsc{Hofmannsthal, Hugo von}!zzzSchnitzler, Arthur@\emph{von Arthur Schnitzler}!1898-07-152@{15. 7. 1898}|)be}\mylabel{L00820h}  \normalsize

\doendnotes{C}
\bigskip
\vfill

\clearpage

\footnotesize

\lohead{\textsc{register}}

% Definiere theindex-Environment komplett neu ohne reledmac
\makeatletter
\renewenvironment{theindex}{%
  \section*{\indexname}%
  \setlength{\parindent}{0pt}%
  \setlength{\parskip}{0pt plus 0.3pt}%
  \let\item\@idxitem
}{%
  \clearpage
}
\makeatother

\IfFileExists{\jobname-pw.ind}{\input{\jobname-pw.ind}}{}

\end{document}

      