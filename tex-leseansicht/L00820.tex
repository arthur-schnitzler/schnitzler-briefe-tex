%% latex-leseansicht-vorspann.tex
%% Vorspann für die Leseansicht.
%% Lädt die gemeinsame Datei latex-vorspann.tex mit nicht gesetztem Schalter.

\newif\ifkorrekturansicht
\korrekturansichtfalse

\input{../tex-inputs/latex-vorspann}


\section[Arthur Schnitzler an Hugo von Hofmannsthal, 15. 7. 1898]{L00820 Arthur Schnitzler an Hugo von Hofmannsthal, 15. 7. 1898}
\nopagebreak\mylabel{L00820v}
\rehead{ }\normalsize\beginnumbering\briefempfaengerindex{Hofmannsthal, Hugo von@\textsc{Hofmannsthal, Hugo von}!zzzSchnitzler, Arthur@\emph{von Arthur Schnitzler}!1898-07-152@{15. 7. 1898}|(be}
\toendnotes[C]{\smallbreak\pagebreak[2]}
\correspDesc{Versand  durch Arthur Schnitzler am 15. 7. 1898 in Graz
\newline{}Erhalt  durch Hugo von Hofmannsthal im Zeitraum [16. 7. 1898
                  – 20. 7. 1898?] in Tschortkiw}\toendnotes[C]{\smallbreak}
\Standort{FDH, Hs-30885,70.}
\physDesc{Brief, 1 Blatt, 3 Seiten, 1337 Zeichen
\newline{}Handschrift: Bleistift, deutsche Kurrent}
\buchAbdrucke{\weitereDrucke{Hugo von Hofmannsthal, Arthur Schnitzler: \emph{Briefwechsel}. Herausgegeben von Therese Nickl und Heinrich Schnitzler. Frankfurt am Main: \emph{S. Fischer} 1964, S. 105106.} }\toendnotes[C]{\smallbreak}
\pstart
           \raggedleft{}{\pb}Graz\oindex{Graz@\textbf{Graz}, \emph{Verwaltungsgebiet}|pw}, Freitag{\\}15/7 98\pend
           \vspace{0.5em}
\pstart
           Mein lieber Hugo, meine Abſicht iſt, So{\geminationn}tag von hier fortzureiſen; dann zu Bahn, Rad,
               Wagen weiter, vielleicht ko{\geminationm} ich in die Fuſch\oindex{Bad Fusch@\textbf{Bad Fusch}|pw}, da{ }ſeh ich wohl noch Ihre Eltern\pwindex{Hofmannsthal, Hugo August von 21.\,12.\,1841 Wien – 8.\,12.\,1915 ebd.@\textsc{Hofmannsthal, Hugo August von} (21.\,12.\,1841 Wien – 8.\,12.\,1915 ebd.), \emph{Bankdirektor}|pwv}\pwindex{Hofmannsthal, Anna von 27.\,1.\,1849 Wien – 22.\,3.\,1904 Sanatorium Fürth@\textsc{Hofmannsthal, Anna von} (27.\,1.\,1849 Wien – 22.\,3.\,1904 Sanatorium Fürth)|pwv}, Do{\geminationn}erſtag 21.{ }\introOben{}Bad\introOben{} Gaſtein, \textsc{Villa Wassing}\oindex{Villa Dr. Wassing@\textbf{Villa Dr. Wassing}, \emph{Sanatorium}|pw}, \uline{dort treffen mich Nachrichten bis 23.} (Bei meiner Mama\pwindex{Schnitzler, Louise 8.\,7.\,1840 Kőszeg – 9.\,9.\,1911 Wien@\textsc{Schnitzler, Louise} (8.\,7.\,1840 Kőszeg – 9.\,9.\,1911 Wien)|pwv}).
                  \introOben{}(Alſo nicht offne Karte!)\introOben{} – Da{\geminationn}{ }ſchlängle ich mich allmählich nach Salzburg\oindex{Salzburg@\textbf{Salzburg}, \emph{Verwaltungsgebiet}|pw} – und weiteres hören Sie noch. – Die Zeit hier vergeht
               leidlich, wenn auch nicht ganz nach meiner Laune; zum Familienleben, {\pb}ſelbſt in mäßigem Umfang bin ich nicht geboren. Auch{ }ſind
               jetzt die Zuſtände durch die merkwürdige Vermengung von illegitimem und anerkanntem,
               Einſicht und Halbheit, ganz unruhig.\pend
           
\pstart
           Zum Arbeiten bin ich gar nicht geko{\geminationm}en; mit einer{ }ſehr
               lebhaften Sehnſucht ruft es mich zu meinem neuen Stück\pwindex{Schnitzler, Arthur 15.\,5.\,1862 Wien – 21.\,10.\,1931 ebd.@\textsc{Schnitzler, Arthur} (15.\,5.\,1862 Wien – 21.\,10.\,1931 ebd.), \emph{Schriftsteller, Mediziner}!Schleier der Beatrice. Schauspiel in fünf Akten@\strich\emph{Der Schleier der Beatrice. Schauspiel in fünf Akten}|pwv} – und doch werd ich vorher wahrſcheinlich was anderes{ }ſchreiben. Die alte Skizze vom »Sohn\pwindex{Schnitzler, Arthur 15.\,5.\,1862 Wien – 21.\,10.\,1931 ebd.@\textsc{Schnitzler, Arthur} (15.\,5.\,1862 Wien – 21.\,10.\,1931 ebd.), \emph{Schriftsteller, Mediziner}!Sohn. Aus den Papieren eines Arztes@\strich\emph{Der Sohn. Aus den Papieren eines Arztes}|pw}«
               (Muttermörder) geſtaltet{ }ſich in mir zu irgendwas aus, was beinah {\pb}ein Roman\pwindex{Schnitzler, Arthur 15.\,5.\,1862 Wien – 21.\,10.\,1931 ebd.@\textsc{Schnitzler, Arthur} (15.\,5.\,1862 Wien – 21.\,10.\,1931 ebd.), \emph{Schriftsteller, Mediziner}!Therese. Chronik eines Frauenlebens@\strich\emph{Therese. Chronik eines Frauenlebens}|pwv}{ }ſein könnte. – Daſs ich von Wien\oindex{Wien@\textbf{Wien}, \emph{Verwaltungsgebiet}|pw} fort bin, iſt mir recht; daſs es von hier aus bald weiter
               geht, nicht minder. Das Radeln macht mir Freude.\pend
           
\pstart
           Warum{ }ſchreiben Sie mir in Ihrem letzten \introOben{}(vom
                  12.)\introOben{} nicht, wie’s Ihnen geht? Das hoff ich, wenn auch nur mit
               ein paar Zeilen, in Gaſtein\oindex{Bad Gastein@\textbf{Bad Gastein}, \emph{Hauptstadt}|pw} zu erfahren. Richard\pwindex{Beer-Hofmann, Richard 11.\,7.\,1866 Wien – 26.\,9.\,1945 New York City@\textsc{Beer-Hofmann, Richard} (11.\,7.\,1866 Wien – 26.\,9.\,1945 New York City), \emph{Schriftsteller}|pw}{ }ſchrieb mir kurz, ohne beſti{\geminationm}te Zuſage, nicht wohlgelaunt.\pend
           
\pstart
           Laſſen Sie uns auf ein{ }ſchönes Wiederſehen hoffen. Von Herzen Ihr
                  \spacefill\mbox{Arthur}\pend
           \selectlanguage{ngerman}\endnumbering\briefempfaengerindex{Hofmannsthal, Hugo von@\textsc{Hofmannsthal, Hugo von}!zzzSchnitzler, Arthur@\emph{von Arthur Schnitzler}!1898-07-152@{15. 7. 1898}|)be}\mylabel{L00820h}  \newcommand{\dateiname}{L00820}\newcommand{\titel}{Arthur Schnitzler an Hugo von Hofmannsthal, 15. 7. 1898}\newcommand{\editorInnen}{Martin Anton Müller und Gerd-Hermann Susen}%% latex-leseansicht-abspann.tex
%% Abspann für die Leseansicht.
%% Der Schalter \ifkorrekturansicht ist bereits durch den Vorspann gesetzt.

%% latex-abspann.tex
%% Gemeinsamer Abspann für Korrekturansicht und Leseansicht.
%% Setzt den Schalter \ifkorrekturansicht voraus (gesetzt in den
%% einbindenden Dateien latex-korrekturansicht-abspann.tex bzw.
%% latex-leseansicht-abspann.tex).
%% ---------------------------------------------------------------

\normalsize

% Das esempio-Environment wird nur in der Leseansicht benötigt
\ifkorrekturansicht\else
\newenvironment{esempio}[3]%
{
    \vspace{1.5ex}
    \rlap{\underline{#1}}
    \par
    \setlength{\parindent}{0cm}
    \nopagebreak
    \leftskip=#2cm
    \rightskip=#3cm
}
{
    \par
}
\fi

\doendnotes{C}
\bigskip
\vfill

\clearpage

\footnotesize

\ifkorrekturansicht
  \lohead{\textsc{register}}
\fi

% theindex-Environment neu definieren ohne reledmac
\makeatletter
\renewenvironment{theindex}{%
  \ifkorrekturansicht
    \section*{\indexname}%
  \else
    \subsubsection*{Index der erwähnten Entitäten}%
  \fi
  \setlength{\parindent}{0pt}%
  \setlength{\parskip}{0pt plus 0.3pt}%
  \let\item\@idxitem
}{%
  \ifkorrekturansicht\clearpage\fi
}
\makeatother

\IfFileExists{\jobname-pw.ind}{\input{\jobname-pw.ind}}{}

% Quellenangabe nur in der Leseansicht
\ifkorrekturansicht\else
% Fallback-Definitionen, falls die .tex-Datei \titel etc. nicht gesetzt hat
\providecommand{\titel}{}
\providecommand{\editorInnen}{}
\providecommand{\dateiname}{\jobname}

\vspace{3cm}

\vfill

\footnotesize
\textsc{Quelle}: \titel. Herausgegeben von {\editorInnen}. In: \emph{Arthur Schnitzler: Briefwechsel mit Autorinnen und Autoren}.
 Digitale Edition, https://schnitzler-briefe.acdh.oeaw.ac.at/{\dateiname}.html (Stand \today)
\fi

\end{document}


