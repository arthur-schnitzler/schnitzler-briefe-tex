%% latex-leseansicht-vorspann.tex
%% Vorspann für die Leseansicht.
%% Lädt die gemeinsame Datei latex-vorspann.tex mit nicht gesetztem Schalter.

\newif\ifkorrekturansicht
\korrekturansichtfalse

\input{../tex-inputs/latex-vorspann}


         
         \newcommand{\erwaehntePersonen}{Personen: }
         \newcommand{\erwaehnteInstitutionen}{}
         \newcommand{\erwaehnteOrte}{}
         \newcommand{\erwaehnteWerke}{
               \section[Arthur Schnitzler an Hugo von Hofmannsthal, 15. 7. 1898]{ Arthur Schnitzler an Hugo von Hofmannsthal, 15. 7. 1898}\nopagebreak\mylabel{v}\rehead{ }\begin{ledgroupsized}[t]{13cm}\normalsize\beginnumbering \toendnotes[C]{\smallbreak\pagebreak[2]} \Standort{FDH, Hs-30885,70.}
\physDesc{Brief, 1 Blatt, 3 Seiten
\newline{}Handschrift: Bleistift, deutsche Kurrent}\buchAbdrucke{\weitereDrucke{Hugo von Hofmannsthal, Arthur Schnitzler: \emph{Briefwechsel}. Hg. Therese Nickl und Heinrich Schnitzler. Frankfurt am Main: \emph{S. Fischer} 1964, S. 105106.} }\toendnotes[C]{\smallbreak}\pstart
           \raggedleft{}{\pb}Graz\oindex{XXXX Ortsangabe fehlt|pw}, Freitag{\\}15/7 98\pend
           \pstart
           Mein lieber Hugo, meine Abſicht iſt, So{\geminationn}tag von hier fortzureiſen; dann zu
                    Bahn, Rad, Wagen weiter, vielleicht ko{\geminationm} ich in die
                        Fuſch\oindex{XXXX Ortsangabe fehlt|pw}, da ſeh ich wohl noch Ihre Eltern\pwindex{\textcolor{red}{\textsuperscript{XXXX1 indx}}|pwv}\pwindex{\textcolor{red}{\textsuperscript{XXXX1 indx}}|pwv}, Do{\geminationn}erſtag 21.{ }\introOben{}Bad\introOben{} Gaſtein, \textsc{Villa
                            Wassing}\oindex{XXXX Ortsangabe fehlt|pw}, \uline{dort treffen mich Nachrichten bis
                            23.} (Bei meiner Mama\pwindex{\textcolor{red}{\textsuperscript{XXXX1 indx}}|pwv}). \introOben{}(Alſo nicht offne Karte!)\introOben{} – Da{\geminationn}{ }ſchlängle ich mich allmählich nach Salzburg\oindex{XXXX Ortsangabe fehlt|pw} – und weiteres hören Sie noch. – Die Zeit
                    hier vergeht leidlich, wenn auch nicht ganz nach meiner Laune; zum
                    Familienleben, {\pb}ſelbſt in mäßigem Umfang bin ich
                    nicht geboren. Auch ſind jetzt die Zuſtände durch die merkwürdige Vermengung von
                    illegitimem und anerkanntem, Einſicht und Halbheit, ganz unruhig.\pend
           \pstart
           Zum Arbeiten bin ich gar nicht geko{\geminationm}en; mit einer
                    ſehr lebhaften Sehnſucht ruft es mich zu meinem neuen Stück\textcolor{red}{\textsuperscript{XXXX indx}} – und doch werd ich vorher wahrſcheinlich was
                    anderes ſchreiben. Die alte Skizze vom »Sohn\textcolor{red}{\textsuperscript{XXXX indx}}« (Muttermörder) geſtaltet ſich in mir zu irgendwas aus, was
                    beinah {\pb}ein Roman\textcolor{red}{\textsuperscript{XXXX indx}}{ }ſein könnte. – Daſs ich von Wien\oindex{XXXX Ortsangabe fehlt|pw} fort bin, iſt mir recht; daſs es von hier aus bald weiter
                    geht, nicht minder. Das Radeln macht mir Freude.\pend
           \pstart
           Warum ſchreiben Sie mir in Ihrem letzten \introOben{}(vom
                        12.)\introOben{} nicht, wie’s Ihnen geht? Das hoff ich, wenn auch nur
                    mit ein paar Zeilen, in Gaſtein\oindex{XXXX Ortsangabe fehlt|pw} zu erfahren.
                        Richard\pwindex{\textcolor{red}{\textsuperscript{XXXX1 indx}}|pw}{ }ſchrieb mir kurz, ohne beſti{\geminationm}te Zuſage, nicht wohlgelaunt.\pend
           \pstart
           Laſſen Sie uns auf ein ſchönes Wiederſehen hoffen. Von Herzen Ihr
                        \spacefill\mbox{Arthur}\pend
           
         
         \endnumbering\mylabel{h}\end{ledgroupsized}  \newcommand{\dateiname}{L00820}\newcommand{\titel}{Arthur Schnitzler an Hugo von Hofmannsthal, 15. 7. 1898}\newcommand{\editorInnen}{Martin Anton Müller und Gerd-Hermann Susen}%% latex-leseansicht-abspann.tex
%% Abspann für die Leseansicht.
%% Der Schalter \ifkorrekturansicht ist bereits durch den Vorspann gesetzt.

%% latex-abspann.tex
%% Gemeinsamer Abspann für Korrekturansicht und Leseansicht.
%% Setzt den Schalter \ifkorrekturansicht voraus (gesetzt in den
%% einbindenden Dateien latex-korrekturansicht-abspann.tex bzw.
%% latex-leseansicht-abspann.tex).
%% ---------------------------------------------------------------

\normalsize

% Das esempio-Environment wird nur in der Leseansicht benötigt
\ifkorrekturansicht\else
\newenvironment{esempio}[3]%
{
    \vspace{1.5ex}
    \rlap{\underline{#1}}
    \par
    \setlength{\parindent}{0cm}
    \nopagebreak
    \leftskip=#2cm
    \rightskip=#3cm
}
{
    \par
}
\fi

\doendnotes{C}
\bigskip
\vfill

\clearpage

\footnotesize

\ifkorrekturansicht
  \lohead{\textsc{register}}
\fi

% theindex-Environment neu definieren ohne reledmac
\makeatletter
\renewenvironment{theindex}{%
  \ifkorrekturansicht
    \section*{\indexname}%
  \else
    \subsubsection*{Index der erwähnten Entitäten}%
  \fi
  \setlength{\parindent}{0pt}%
  \setlength{\parskip}{0pt plus 0.3pt}%
  \let\item\@idxitem
}{%
  \ifkorrekturansicht\clearpage\fi
}
\makeatother

\IfFileExists{\jobname-pw.ind}{\input{\jobname-pw.ind}}{}

% Quellenangabe nur in der Leseansicht
\ifkorrekturansicht\else
% Fallback-Definitionen, falls die .tex-Datei \titel etc. nicht gesetzt hat
\providecommand{\titel}{}
\providecommand{\editorInnen}{}
\providecommand{\dateiname}{\jobname}

\vspace{3cm}

\vfill

\footnotesize
\textsc{Quelle}: \titel. Herausgegeben von {\editorInnen}. In: \emph{Arthur Schnitzler: Briefwechsel mit Autorinnen und Autoren}.
 Digitale Edition, https://schnitzler-briefe.acdh.oeaw.ac.at/{\dateiname}.html (Stand \today)
\fi

\end{document}


      