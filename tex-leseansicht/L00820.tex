%% latex-leseansicht-vorspann.tex
%% Vorspann für die Leseansicht.
%% Lädt die gemeinsame Datei latex-vorspann.tex mit nicht gesetztem Schalter.

\newif\ifkorrekturansicht
\korrekturansichtfalse

\input{../tex-inputs/latex-vorspann}


         
         \newcommand{\erwaehntePersonen}{Personen: Richard Beer-Hofmann, Hugo von Hofmannsthal, Hugo August von Hofmannsthal, Anna von Hofmannsthal, Louise Schnitzler}
         \newcommand{\erwaehnteOrte}{Orte: Bad Fusch, Bad Gastein, Graz, Salzburg, Tschortkiw, Villa Dr. Wassing, Wien}
         \newcommand{\erwaehnteWerke}{Werke: Der Schleier der Beatrice. Schauspiel in fünf Akten, Der Sohn. Aus den Papieren eines Arztes, Therese. Chronik eines Frauenlebens}
               \section[Arthur Schnitzler an Hugo von Hofmannsthal, 15. 7. 1898]{ Arthur Schnitzler an Hugo von Hofmannsthal, 15. 7. 1898}\nopagebreak\mylabel{v}\rehead{ }\begin{ledgroupsized}[t]{13cm}\normalsize\beginnumbering \toendnotes[C]{\smallbreak\pagebreak[2]} \Standort{FDH, Hs-30885,70.}
\physDesc{Brief, 1 Blatt, 3 Seiten
\newline{}Handschrift: Bleistift, deutsche Kurrent}\buchAbdrucke{\weitereDrucke{Hugo von Hofmannsthal, Arthur Schnitzler: \emph{Briefwechsel}. Hg. Therese Nickl und Heinrich Schnitzler. Frankfurt am Main: \emph{S. Fischer} 1964, S. 105106.} }\toendnotes[C]{\smallbreak}\pstart
           \raggedleft{}{\pb}Graz\oindex{Graz@\textbf{Graz}|pw}, Freitag{\\}15/7 98\pend
           \pstart
           Mein lieber Hugo, meine Abſicht iſt, So{\geminationn}tag von hier fortzureiſen; dann zu
                    Bahn, Rad, Wagen weiter, vielleicht ko{\geminationm} ich in die
                        Fuſch\oindex{Bad Fusch@\textbf{Bad Fusch}|pw}, da ſeh ich wohl noch Ihre Eltern\pwindex{Hofmannsthal, Hugo August von 21.12.1841 – 08.12.1915@\textsc{Hofmannsthal, Hugo August von} (21.12.1841 – 08.12.1915), \emph{Bankdirektor}|pwv}\pwindex{Hofmannsthal, Anna von 27.01.1849 – 22.03.1904@\textsc{Hofmannsthal, Anna von} (27.01.1849 – 22.03.1904)|pwv}, Do{\geminationn}erſtag 21.{ }\introOben{}Bad\introOben{} Gaſtein, \textsc{Villa
                            Wassing}\oindex{Villa Dr. Wassing@\textbf{Villa Dr. Wassing}|pw}, \uline{dort treffen mich Nachrichten bis
                            23.} (Bei meiner Mama\pwindex{Schnitzler, Louise 1840-07-08 – 1911-09-09@\textsc{Schnitzler, Louise} (1840-07-08 – 1911-09-09)|pwv}). \introOben{}(Alſo nicht offne Karte!)\introOben{} – Da{\geminationn}{ }ſchlängle ich mich allmählich nach Salzburg\oindex{Salzburg@\textbf{Salzburg}|pw} – und weiteres hören Sie noch. – Die Zeit
                    hier vergeht leidlich, wenn auch nicht ganz nach meiner Laune; zum
                    Familienleben, {\pb}ſelbſt in mäßigem Umfang bin ich
                    nicht geboren. Auch ſind jetzt die Zuſtände durch die merkwürdige Vermengung von
                    illegitimem und anerkanntem, Einſicht und Halbheit, ganz unruhig.\pend
           \pstart
           Zum Arbeiten bin ich gar nicht geko{\geminationm}en; mit einer
                    ſehr lebhaften Sehnſucht ruft es mich zu meinem neuen Stück\pwindex{Schnitzler, Arthur 15.05.1862 – 21.10.1931@\textsc{Schnitzler, Arthur} (15.05.1862 – 21.10.1931), \emph{Schriftsteller, Mediziner}!Schleier der Beatrice. Schauspiel in fuenf Akten1900-12-01@\strich\emph{Der Schleier der Beatrice. Schauspiel in fünf Akten} {[}1900-12-01{]}|pwv} – und doch werd ich vorher wahrſcheinlich was
                    anderes ſchreiben. Die alte Skizze vom »Sohn\pwindex{Schnitzler, Arthur 15.05.1862 – 21.10.1931@\textsc{Schnitzler, Arthur} (15.05.1862 – 21.10.1931), \emph{Schriftsteller, Mediziner}!Sohn. Aus den Papieren eines Arztes1. 1. 1892@\strich\emph{Der Sohn. Aus den Papieren eines Arztes} {[}1. 1. 1892{]}|pw}« (Muttermörder) geſtaltet ſich in mir zu irgendwas aus, was
                    beinah {\pb}ein Roman\pwindex{Schnitzler, Arthur 15.05.1862 – 21.10.1931@\textsc{Schnitzler, Arthur} (15.05.1862 – 21.10.1931), \emph{Schriftsteller, Mediziner}!Therese. Chronik eines Frauenlebens1928@\strich\emph{Therese. Chronik eines Frauenlebens} {[}1928{]}|pwv}{ }ſein könnte. – Daſs ich von Wien\oindex{Wien@\textbf{Wien}|pw} fort bin, iſt mir recht; daſs es von hier aus bald weiter
                    geht, nicht minder. Das Radeln macht mir Freude.\pend
           \pstart
           Warum ſchreiben Sie mir in Ihrem letzten \introOben{}(vom
                        12.)\introOben{} nicht, wie’s Ihnen geht? Das hoff ich, wenn auch nur
                    mit ein paar Zeilen, in Gaſtein\oindex{Bad Gastein@\textbf{Bad Gastein}|pw} zu erfahren.
                        Richard\pwindex{Beer-Hofmann, Richard 1866-07-11 – 1945-09-26@\textsc{Beer-Hofmann, Richard} (1866-07-11 – 1945-09-26), \emph{Schriftsteller}|pw}{ }ſchrieb mir kurz, ohne beſti{\geminationm}te Zuſage, nicht wohlgelaunt.\pend
           \pstart
           Laſſen Sie uns auf ein ſchönes Wiederſehen hoffen. Von Herzen Ihr
                        \spacefill\mbox{Arthur}\pend
           
         
         \endnumbering\mylabel{h}\end{ledgroupsized}  \newcommand{\dateiname}{L00820}\newcommand{\titel}{Arthur Schnitzler an Hugo von Hofmannsthal, 15. 7. 1898}\newcommand{\editorInnen}{Martin Anton Müller und Gerd-Hermann Susen}%% latex-leseansicht-abspann.tex
%% Abspann für die Leseansicht.
%% Der Schalter \ifkorrekturansicht ist bereits durch den Vorspann gesetzt.

%% latex-abspann.tex
%% Gemeinsamer Abspann für Korrekturansicht und Leseansicht.
%% Setzt den Schalter \ifkorrekturansicht voraus (gesetzt in den
%% einbindenden Dateien latex-korrekturansicht-abspann.tex bzw.
%% latex-leseansicht-abspann.tex).
%% ---------------------------------------------------------------

\normalsize

% Das esempio-Environment wird nur in der Leseansicht benötigt
\ifkorrekturansicht\else
\newenvironment{esempio}[3]%
{
    \vspace{1.5ex}
    \rlap{\underline{#1}}
    \par
    \setlength{\parindent}{0cm}
    \nopagebreak
    \leftskip=#2cm
    \rightskip=#3cm
}
{
    \par
}
\fi

\doendnotes{C}
\bigskip
\vfill

\clearpage

\footnotesize

\ifkorrekturansicht
  \lohead{\textsc{register}}
\fi

% theindex-Environment neu definieren ohne reledmac
\makeatletter
\renewenvironment{theindex}{%
  \ifkorrekturansicht
    \section*{\indexname}%
  \else
    \subsubsection*{Index der erwähnten Entitäten}%
  \fi
  \setlength{\parindent}{0pt}%
  \setlength{\parskip}{0pt plus 0.3pt}%
  \let\item\@idxitem
}{%
  \ifkorrekturansicht\clearpage\fi
}
\makeatother

\IfFileExists{\jobname-pw.ind}{\input{\jobname-pw.ind}}{}

% Quellenangabe nur in der Leseansicht
\ifkorrekturansicht\else
% Fallback-Definitionen, falls die .tex-Datei \titel etc. nicht gesetzt hat
\providecommand{\titel}{}
\providecommand{\editorInnen}{}
\providecommand{\dateiname}{\jobname}

\vspace{3cm}

\vfill

\footnotesize
\textsc{Quelle}: \titel. Herausgegeben von {\editorInnen}. In: \emph{Arthur Schnitzler: Briefwechsel mit Autorinnen und Autoren}.
 Digitale Edition, https://schnitzler-briefe.acdh.oeaw.ac.at/{\dateiname}.html (Stand \today)
\fi

\end{document}


      