\input{../tex-inputs/latex-pdf-vorspann}
\begin{center}
            \textcolor{red}{ENTWURF. ENTZIFFERUNG NOCH NICHT KORREKTURGELESEN}
                      \end{center}
            
               \section[Arthur Schnitzler an Georg Brandes, 11. 1. 1897]{ Arthur Schnitzler an Georg Brandes, 11. 1. 1897}\nopagebreak\mylabel{v}\rehead{ }\begin{ledgroupsized}[t]{13cm}\normalsize\beginnumbering\briefempfaengerindex{Brandes, Georg@\textsc{Brandes, Georg}!zzzSchnitzler, Arthur@\emph{von Arthur Schnitzler}!1897-01-112@{11. 1. 1897}|(be} \toendnotes[C]{\smallbreak\pagebreak[2]} \Standort{Kopenhagen, Det Kongelige Bibliotek, Georg Brandes Arkiv, box 125.}
\physDesc{Brief, 2 Blätter, 6 Seiten
\newline{}Handschrift: schwarze Tinte, deutsche Kurrent\newline{}Ordnung: mit Bleistift von unbekannter Hand Vermerk auf der ersten Seite: »Schni\textcolor{gray}{tzler}« und nummeriert: »6«, das zweite
                                    Blatt mit »11/1 97« datiert }\buchAbdrucke{\weitereDrucke{1) Georg Brandes, Arthur Schnitzler: \emph{Ein Briefwechsel}. Hg. Kurt Bergel. Bern: \emph{Francke} 1956, S. 59.} \weitereDrucke{2) Arthur Schnitzler: \emph{Briefe 1875–1912}. Hg. Therese Nickl und Heinrich Schnitzler. Frankfurt am Main: \emph{S. Fischer} 1981, S. 311.} }\toendnotes[C]{\smallbreak}\pstart
           \raggedleft{}{\pb}Wien\oindex{Wien@\textbf{Wien}|pw}, 11. 1. 97.\pend
           \pstart{}Verehrteſter Herr Brandes,\pend\pstart
           in dieſem Briefe finden Sie mein neues Stück »Freiwild\pwindex{Schnitzler, Arthur 15.05.1862 – 21.10.1931@\textsc{Schnitzler, Arthur} (15.05.1862 – 21.10.1931), \emph{Schriftsteller, Mediziner}!Freiwild. Schauspiel in 3 Akten1896@\strich\emph{Freiwild. Schauspiel in 3 Akten} {[}1896{]}|pw}« eingeſchloſſen. Nicht »weil ich Ihrer vergeſſen« –
                    muſs ich das wirklich ſagen – ? ſende ich es erſt heute ab! Wie Sie ſehen, ist
                    das Stück noch \textsc{Manuscript}; {\pb}ich habe mich bisher nicht entſchließen
                    können, es als \label{K_L00636_1v}\edtext{Buch erſcheinen}{\lemma{\textnormal{\emph{Buch erſcheinen}}}\Cendnote{\textnormal{Es erschien erst im Folgejahr,
                        rechtzeitig zur Wien\oindex{Wien@\textbf{Wien}|pwk}er Premiere, im
                            Februar 1898 bei \emph{S. Fischer}\orgindex{S. Fischer Verlag@S. Fischer Verlag|pwk}.}}}\label{K_L00636_1h} zu laſſen. Auf dem Theater macht es ja
                        \label{K_L00636_2v}\edtext{ſeine Wirkung}{\lemma{\textnormal{\emph{ſeine Wirkung}}}\Cendnote{\textnormal{Die Uraufführung hatte am
                            3. 11. 1896 im Deutschen Theater in
                            Berlin\oindex{Deutsches Theater Berlin@\textbf{Deutsches Theater Berlin}|pwk}
                   stattgefunden.}}}\label{K_L00636_2h}; in der Lecture ſcheint es dürr
                    und unangenehm. Ich empfinde das umſo verdrießlicher, als ich glaube, dſs mir
                    die Komödie in glücklicherer Sti{\geminationm}ung hätte gelingen
                    müſſen. {\pb}Der Stoff iſt mir lang
                    nachgegangen, und obwohl man heute den Eindruck gewinnen mag, das ganze ſei
                    einer Theſe zu Liebe geſchrieben, ſo iſt es mir ſeinerzeit doch aus dem Leben
                    empor- und entgegengequollen. Und vielleicht ko{\geminationm}t
                    auch das Misglücken ſelbſt wieder aus etwas ſehr lebendigem {\pb}her. Die weibliche Hauptfigur hat
                    namlich gerade in der Zeit, da der Stoff in mir reif wurde, einen Sprung beko{\geminationm}en, der ſich dann, wie in einem an einer Stelle
                    eingedrückten Spiegel nach allen Seiten fortgeſetzt hat. Ich habe das Stück\pwindex{Schnitzler, Arthur 15.05.1862 – 21.10.1931@\textsc{Schnitzler, Arthur} (15.05.1862 – 21.10.1931), \emph{Schriftsteller, Mediziner}!Freiwild. Schauspiel in 3 Akten1896@\strich\emph{Freiwild. Schauspiel in 3 Akten} {[}1896{]}|pwv} ein paar Mal geſchrieben; es iſt
                    techniſch reinlicher, aber innerlich {\pb}nicht beſſer geworden. Ich habe alſo auf ein Schickſalswort gewartet, um Ihnen
                    das Stück zu ſenden. Vielleicht wäre es auch eine Art von Unaufrichtigkeit
                    geweſen, Ihnen, dem ich bisher ſchon ſo wunderbare Worte freundlicher Theilnahme
                    verdanke, dieſes Stück, das ich ja nun doch einmal {\pb}gemacht habe und ſogar habe aufführen
                    laſſen, zu unterſchlagen.\pend
           \pstart
           Hier iſt es\pwindex{Schnitzler, Arthur 15.05.1862 – 21.10.1931@\textsc{Schnitzler, Arthur} (15.05.1862 – 21.10.1931), \emph{Schriftsteller, Mediziner}!Freiwild. Schauspiel in 3 Akten1896@\strich\emph{Freiwild. Schauspiel in 3 Akten} {[}1896{]}|pwv} alſo, und mit ihm die
                    herzlichſten und verehrungsvollſten Grüße Ihres treu ergebnen{\\[\baselineskip]}\spacefill\mbox{ArthurSchnitzler.}\pend
           \leftskip=0em{}\endnumbering\briefempfaengerindex{Brandes, Georg@\textsc{Brandes, Georg}!zzzSchnitzler, Arthur@\emph{von Arthur Schnitzler}!1897-01-112@{11. 1. 1897}|)be}\mylabel{h}\end{ledgroupsized}  \newcommand{\dateiname}{L00636}\newcommand{\titel}{Arthur Schnitzler an Georg Brandes, 11. 1. 1897}\newcommand{\editorInnen}{Martin Anton Müller und Gerd-Hermann Susen}\input{../tex-inputs/latex-pdf-abspann}
      