%% latex-leseansicht-vorspann.tex
%% Vorspann für die Leseansicht.
%% Lädt die gemeinsame Datei latex-vorspann.tex mit nicht gesetztem Schalter.

\newif\ifkorrekturansicht
\korrekturansichtfalse

\input{../tex-inputs/latex-vorspann}


\section[Arthur Schnitzler an Georg Brandes, 11. 1. 1897]{L00636 Arthur Schnitzler an Georg Brandes, 11. 1. 1897}
\nopagebreak\mylabel{L00636v}
\rehead{ }\normalsize\beginnumbering\briefempfaengerindex{Brandes, Georg@\textsc{Brandes, Georg}!zzzSchnitzler, Arthur@\emph{von Arthur Schnitzler}!1897-01-112@{11. 1. 1897}|(be}
\toendnotes[C]{\smallbreak\pagebreak[2]}
\correspDesc{Versand  durch Arthur Schnitzler am 11. 1. 1897 in Wien
\newline{}Erhalt  durch Georg Brandes im Zeitraum [12. 1. 1897
                  – 16. 1. 1897?] in Kopenhagen}\toendnotes[C]{\smallbreak}
\Standort{Kopenhagen, Det Kongelige Bibliotek, Georg Brandes Arkiv, box 125.}
\physDesc{Brief, 2 Blätter, 6 Seiten, 1589 Zeichen
\newline{}Handschrift: schwarze Tinte, deutsche Kurrent
\newline{}Ordnung: mit Bleistift von unbekannter Hand Vermerk auf der ersten Seite:
                                       »Schni\textcolor{gray}{tzler}« und nummeriert: »6«, das zweite Blatt mit »11/1 97« datiert }
\buchAbdrucke{\weitereDrucke{1) Georg Brandes, Arthur Schnitzler: \emph{Ein Briefwechsel}. Herausgegeben von Kurt Bergel. Bern: \emph{Francke} 1956, S. 59.} \weitereDrucke{2) Arthur Schnitzler: \emph{Briefe 1875–1912}. Herausgegeben von Therese Nickl und Heinrich Schnitzler. Frankfurt am Main: \emph{S. Fischer} 1981, S. 311.} }\toendnotes[C]{\smallbreak}
\pstart
           \raggedleft{}{\pb}Wien\oindex{Wien@\textbf{Wien}, \emph{Verwaltungsgebiet}|pw}, 11. 1. 97.\pend
           
\pstart{}Verehrteſter Herr Brandes,\pend\vspace{0.5em}
\pstart
           in dieſem Briefe finden Sie mein neues Stück »Freiwild\pwindex{Schnitzler, Arthur 15.\,5.\,1862 Wien – 21.\,10.\,1931 ebd.@\textsc{Schnitzler, Arthur} (15.\,5.\,1862 Wien – 21.\,10.\,1931 ebd.), \emph{Schriftsteller, Mediziner}!Freiwild. Schauspiel in 3 Akten@\strich\emph{Freiwild. Schauspiel in 3 Akten}|pw}« eingeſchloſſen. Nicht »weil ich Ihrer vergeſſen« – muſs ich das
               wirklich{ }ſagen – ?{ }ſende ich es erſt heute ab! Wie Sie{ }ſehen, ist das Stück noch \textsc{Manuscript}; {\pb}ich habe
               mich bisher nicht entſchließen können, es als \label{K_L00636-1v}\edtext{Buch erſcheinen}{\lemma{\textnormal{\emph{Buch erscheinen}}}\Cendnote{\textnormal{Es
                  erschien erst im Februar 1898 bei \emph{S. Fischer}\orgindex{S. Fischer Verlag@S. Fischer Verlag|pwk}, rechtzeitig zur Wien\oindex{Wien@\textbf{Wien}, \emph{Verwaltungsgebiet}|pwk}er Premiere.}}}\label{K_L00636-1} zu laſſen. Auf dem Theater macht es ja
              \label{K_L00636-2v}\edtext{ſeine Wirkung}{\lemma{\textnormal{\emph{seine Wirkung}}}\Cendnote{\textnormal{Die Uraufführung\eventindex{Deutsches Theater Berlin@\textbf{Deutsches Theater Berlin}!Uraufführung von Freiwild, 3.11.1896@Uraufführung von Freiwild, 3.11.1896|pwkv} hatte am
                     3. 11. 1896 im \emph{Deutschen Theater in
                     Berlin}\orgindex{Deutsches Theater Berlin@Deutsches Theater Berlin|pwk} stattgefunden.}}}\label{K_L00636-2}; in der Lecture{ }ſcheint es dürr und
               unangenehm. Ich empfinde das umſo verdrießlicher, als ich glaube, dſs mir die Komödie
               in glücklicherer Sti{\geminationm}ung hätte gelingen müſſen. {\pb}Der Stoff iſt mir lang nachgegangen, und obwohl
               man heute den Eindruck gewinnen mag, das ganze{ }ſei einer Theſe zu Liebe geſchrieben,{ }ſo iſt es mir{ }ſeinerzeit doch aus dem Leben empor- und entgegengequollen. Und
               vielleicht ko{\geminationm}t auch das Misglücken{ }ſelbſt wieder aus
               etwas{ }ſehr lebendigem {\pb}her. Die weibliche
               Hauptfigur hat namlich gerade in der Zeit, da der Stoff in mir reif wurde, einen
               Sprung beko{\geminationm}en, der{ }ſich dann, wie in einem an einer
               Stelle eingedrückten Spiegel nach allen Seiten fortgeſetzt hat. Ich habe das Stück\pwindex{Schnitzler, Arthur 15.\,5.\,1862 Wien – 21.\,10.\,1931 ebd.@\textsc{Schnitzler, Arthur} (15.\,5.\,1862 Wien – 21.\,10.\,1931 ebd.), \emph{Schriftsteller, Mediziner}!Freiwild. Schauspiel in 3 Akten@\strich\emph{Freiwild. Schauspiel in 3 Akten}|pwv} ein paar Mal geſchrieben;
               es iſt techniſch reinlicher, aber innerlich {\pb}nicht
               beſſer geworden. Ich habe alſo auf ein Schickſalswort gewartet, um Ihnen das Stück zu{ }ſenden. Vielleicht wäre es auch eine Art von Unaufrichtigkeit geweſen, Ihnen, dem ich
               bisher{ }ſchon{ }ſo wunderbare Worte freundlicher Theilnahme verdanke, dieſes Stück, das
               ich ja nun doch einmal {\pb}gemacht habe und{ }ſogar
               habe aufführen laſſen, zu unterſchlagen.\pend
           
\pstart
           Hier iſt es\pwindex{Schnitzler, Arthur 15.\,5.\,1862 Wien – 21.\,10.\,1931 ebd.@\textsc{Schnitzler, Arthur} (15.\,5.\,1862 Wien – 21.\,10.\,1931 ebd.), \emph{Schriftsteller, Mediziner}!Freiwild. Schauspiel in 3 Akten@\strich\emph{Freiwild. Schauspiel in 3 Akten}|pwv} alſo, und mit
               ihm die herzlichſten und verehrungsvollſten Grüße Ihres treu ergebnen{\\[\baselineskip]}\spacefill\mbox{ArthurSchnitzler.}\pend
           \leftskip=0em{}\selectlanguage{ngerman}\endnumbering\briefempfaengerindex{Brandes, Georg@\textsc{Brandes, Georg}!zzzSchnitzler, Arthur@\emph{von Arthur Schnitzler}!1897-01-112@{11. 1. 1897}|)be}\mylabel{L00636h}  \newcommand{\dateiname}{L00636}\newcommand{\titel}{Arthur Schnitzler an Georg Brandes, 11. 1. 1897}\newcommand{\editorInnen}{Martin Anton Müller und Gerd-Hermann Susen}%% latex-leseansicht-abspann.tex
%% Abspann für die Leseansicht.
%% Der Schalter \ifkorrekturansicht ist bereits durch den Vorspann gesetzt.

%% latex-abspann.tex
%% Gemeinsamer Abspann für Korrekturansicht und Leseansicht.
%% Setzt den Schalter \ifkorrekturansicht voraus (gesetzt in den
%% einbindenden Dateien latex-korrekturansicht-abspann.tex bzw.
%% latex-leseansicht-abspann.tex).
%% ---------------------------------------------------------------

\normalsize

% Das esempio-Environment wird nur in der Leseansicht benötigt
\ifkorrekturansicht\else
\newenvironment{esempio}[3]%
{
    \vspace{1.5ex}
    \rlap{\underline{#1}}
    \par
    \setlength{\parindent}{0cm}
    \nopagebreak
    \leftskip=#2cm
    \rightskip=#3cm
}
{
    \par
}
\fi

\doendnotes{C}
\bigskip
\vfill

\clearpage

\footnotesize

\ifkorrekturansicht
  \lohead{\textsc{register}}
\fi

% theindex-Environment neu definieren ohne reledmac
\makeatletter
\renewenvironment{theindex}{%
  \ifkorrekturansicht
    \section*{\indexname}%
  \else
    \subsubsection*{Index der erwähnten Entitäten}%
  \fi
  \setlength{\parindent}{0pt}%
  \setlength{\parskip}{0pt plus 0.3pt}%
  \let\item\@idxitem
}{%
  \ifkorrekturansicht\clearpage\fi
}
\makeatother

\IfFileExists{\jobname-pw.ind}{\input{\jobname-pw.ind}}{}

% Quellenangabe nur in der Leseansicht
\ifkorrekturansicht\else
% Fallback-Definitionen, falls die .tex-Datei \titel etc. nicht gesetzt hat
\providecommand{\titel}{}
\providecommand{\editorInnen}{}
\providecommand{\dateiname}{\jobname}

\vspace{3cm}

\vfill

\footnotesize
\textsc{Quelle}: \titel. Herausgegeben von {\editorInnen}. In: \emph{Arthur Schnitzler: Briefwechsel mit Autorinnen und Autoren}.
 Digitale Edition, https://schnitzler-briefe.acdh.oeaw.ac.at/{\dateiname}.html (Stand \today)
\fi

\end{document}


