%% latex-leseansicht-vorspann.tex
%% Vorspann für die Leseansicht.
%% Lädt die gemeinsame Datei latex-vorspann.tex mit nicht gesetztem Schalter.

\newif\ifkorrekturansicht
\korrekturansichtfalse

\input{../tex-inputs/latex-vorspann}


\section[Hugo von Hofmannsthal an Arthur Schnitzler, 27. 7. 1906]{L01617 Hugo von Hofmannsthal an Arthur Schnitzler, 27. 7. 1906}
\nopagebreak\mylabel{L01617v}
\rehead{ }\normalsize\beginnumbering\briefempfaengerindex{Schnitzler, Arthur@\textsc{Schnitzler, Arthur}!zzzHofmannsthal, Hugo von@\emph{von Hugo von Hofmannsthal}!1906-07-271@{27. 7. 1906}|(be}
\toendnotes[C]{\smallbreak\pagebreak[2]}
\correspDesc{Versand  durch Hugo von Hofmannsthal am 27. 7. 1906 in St. Gilgen
\newline{}Erhalt  durch Arthur Schnitzler am 29. 7. 1906 in Helsingør}\toendnotes[C]{\smallbreak}
\Standort{CUL, Schnitzler, B 43.}
\physDesc{Bildpostkarte, 225 Zeichen
\newline{}Handschrift: Bleistift, deutsche Kurrent
\newline{}Versand: 1) Stempel: »\nobreak{}\oindex{St. Gilgen@\textbf{St. Gilgen}, \emph{Verwaltungsgebiet}|pwk}St. Gilgen\nobreak{}«.   2) Stempel: »\nobreak{}\oindex{Helsingør@\textbf{Helsingør}, \emph{Hauptstadt}|pwk}Helsingør, 29. 7. 06, 11–12F\nobreak{}«. 
\newline{}Ordnung: 1) mit Bleistift von unbekannter Hand nummeriert: »\strikeout{262}«  2) mit Bleistift von unbekannter Hand nummeriert:
                                    »263«}
\buchAbdrucke{\weitereDrucke{Hugo von Hofmannsthal, Arthur Schnitzler: \emph{Briefwechsel}. Herausgegeben von Therese Nickl und Heinrich Schnitzler. Frankfurt am Main: \emph{S. Fischer} 1964, S. 220.} }\toendnotes[C]{\smallbreak}\pstart{}\textsc{{\pb}Herrn Arthur
                  Schnitzler}\pend{}\pstart{}\textsc{Marienlyst\oindex{Marienlyst@\textbf{Marienlyst}, \emph{Gut}|pw}}\pend{}\pstart{}\textsc{Kurhaus\oindex{Kurhotellet@\textbf{Kurhotellet}, \emph{Hotel}|pw}}\pend{}\pstart{}\textsc{per Kopenhagen\oindex{Kopenhagen@\textbf{Kopenhagen}, \emph{Hauptstadt}|pw}}\pend{}\pstart{}\textsc{Dänemark\oindex{Dänemark@\textbf{Dänemark}|pw}}\pend{}{\bigskip}
\pstart
           \noindent{}\centering{}{\pb}\textcolor{gray}{\textbf{Gasthof Lueg\oindex{Hotel und Pension Lueg@\textbf{Hotel und Pension Lueg}, \emph{Hotel}|pw}. Lueg\oindex{Lueg@\textbf{Lueg}, \emph{Teil eines besiedelten Ortes}|pw} bei St. Gilgen\oindex{St. Gilgen@\textbf{St. Gilgen}, \emph{Verwaltungsgebiet}|pw} am
                  Wolfgang-\oindex{Wolfgangsee@\textbf{Wolfgangsee}, \emph{See}|pw}(Aber-)See.}}\pend
           \vspace{1em}
\pstart
           {\pb}27.\pend
           \vspace{0.5em}
\pstart
           Freuen uns herzlich von Ihnen Gutes zu hören.\pend
           
\pstart
           Gehen nächſte Woche 2 Tage Baireuth\oindex{Bayreuth@\textbf{Bayreuth}, \emph{Hauptstadt}|pw},{ }ſind hier
               bis Ende Auguſt mit allen Kindern\pwindex{Zimmer, Christiane 14.\,5.\,1902 Rodaun – 5.\,1.\,1987 New York City@\textsc{Zimmer, Christiane} (14.\,5.\,1902 Rodaun – 5.\,1.\,1987 New York City)|pwv}\pwindex{Hofmannsthal, Raimund von 26.\,5.\,1906 Rodaun – 20.\,3.\,1974 London@\textsc{Hofmannsthal, Raimund von} (26.\,5.\,1906 Rodaun – 20.\,3.\,1974 London)|pwv}\pwindex{Hofmannsthal, Franz von 20.\,10.\,1903 Wien – 13.\,7.\,1929 ebd.@\textsc{Hofmannsthal, Franz von} (20.\,10.\,1903 Wien – 13.\,7.\,1929 ebd.)|pwv}. Bitte bald wieder
               paar Zeilen.\pend
           
\pstart
           Ihr{\\[\baselineskip]}\spacefill\mbox{Hugo.}\pend
           \leftskip=0em{}\selectlanguage{ngerman}\endnumbering\briefempfaengerindex{Schnitzler, Arthur@\textsc{Schnitzler, Arthur}!zzzHofmannsthal, Hugo von@\emph{von Hugo von Hofmannsthal}!1906-07-271@{27. 7. 1906}|)be}\mylabel{L01617h}  \newcommand{\dateiname}{L01617}\newcommand{\titel}{Hugo von Hofmannsthal an Arthur Schnitzler, 27. 7. 1906}\newcommand{\editorInnen}{Martin Anton Müller und Gerd-Hermann Susen}%% latex-leseansicht-abspann.tex
%% Abspann für die Leseansicht.
%% Der Schalter \ifkorrekturansicht ist bereits durch den Vorspann gesetzt.

%% latex-abspann.tex
%% Gemeinsamer Abspann für Korrekturansicht und Leseansicht.
%% Setzt den Schalter \ifkorrekturansicht voraus (gesetzt in den
%% einbindenden Dateien latex-korrekturansicht-abspann.tex bzw.
%% latex-leseansicht-abspann.tex).
%% ---------------------------------------------------------------

\normalsize

% Das esempio-Environment wird nur in der Leseansicht benötigt
\ifkorrekturansicht\else
\newenvironment{esempio}[3]%
{
    \vspace{1.5ex}
    \rlap{\underline{#1}}
    \par
    \setlength{\parindent}{0cm}
    \nopagebreak
    \leftskip=#2cm
    \rightskip=#3cm
}
{
    \par
}
\fi

\doendnotes{C}
\bigskip
\vfill

\clearpage

\footnotesize

\ifkorrekturansicht
  \lohead{\textsc{register}}
\fi

% theindex-Environment neu definieren ohne reledmac
\makeatletter
\renewenvironment{theindex}{%
  \ifkorrekturansicht
    \section*{\indexname}%
  \else
    \subsubsection*{Index der erwähnten Entitäten}%
  \fi
  \setlength{\parindent}{0pt}%
  \setlength{\parskip}{0pt plus 0.3pt}%
  \let\item\@idxitem
}{%
  \ifkorrekturansicht\clearpage\fi
}
\makeatother

\IfFileExists{\jobname-pw.ind}{\input{\jobname-pw.ind}}{}

% Quellenangabe nur in der Leseansicht
\ifkorrekturansicht\else
% Fallback-Definitionen, falls die .tex-Datei \titel etc. nicht gesetzt hat
\providecommand{\titel}{}
\providecommand{\editorInnen}{}
\providecommand{\dateiname}{\jobname}

\vspace{3cm}

\vfill

\footnotesize
\textsc{Quelle}: \titel. Herausgegeben von {\editorInnen}. In: \emph{Arthur Schnitzler: Briefwechsel mit Autorinnen und Autoren}.
 Digitale Edition, https://schnitzler-briefe.acdh.oeaw.ac.at/{\dateiname}.html (Stand \today)
\fi

\end{document}


