%% latex-leseansicht-vorspann.tex
%% Vorspann für die Leseansicht.
%% Lädt die gemeinsame Datei latex-vorspann.tex mit nicht gesetztem Schalter.

\newif\ifkorrekturansicht
\korrekturansichtfalse

\input{../tex-inputs/latex-vorspann}


         
         \renewcommand{\erwaehntePersonen}{Personen: Christiane von Hofmannsthal, Raimund von Hofmannsthal, Franz von Hofmannsthal}
         \renewcommand{\erwaehnteOrte}{Orte: Bayreuth, Dänemark, Helsingør, Hotel und Pension Lueg, Kopenhagen, Kurhotellet, Lueg am Wolfgangsee, Marienlyst, St. Gilgen, Wolfgangsee}
         \renewcommand{\erwaehnteWerke}{}
               \section[Hugo von Hofmannsthal an Arthur Schnitzler, 27. 7. 1906]{ Hugo von Hofmannsthal an Arthur Schnitzler, 27. 7. 1906}\nopagebreak\mylabel{v}\rehead{ }\begin{ledgroupsized}[t]{13cm}\normalsize\beginnumbering \toendnotes[C]{\smallbreak\pagebreak[2]} \Standort{CUL, Schnitzler, B 43.}
\physDesc{Bildpostkarte, 225 Zeichen
\newline{}Handschrift: 1) Bleistift, deutsche Kurrent\hspace{1em}2) Bleistift, lateinische Kurrent (\noindent{}Adresse)\hspace{1em}
\newline{}Versand: 1) Stempel: »\nobreak{}\oindex{St. Gilgen@\textbf{St. Gilgen}|pwk}St. Gilgen\nobreak{}«.   2) Stempel: »\nobreak{}\oindex{Helsingør@\textbf{Helsingør}|pwk}Helsingør, 29. 7. 06, 11–12F\nobreak{}«. 
\newline{}Ordnung: 1) mit Bleistift von unbekannter Hand nummeriert: »\strikeout{262}«  2) mit Bleistift von unbekannter Hand nummeriert:
                                    »263«}\buchAbdrucke{\weitereDrucke{Hugo von Hofmannsthal, Arthur Schnitzler: \emph{Briefwechsel}. Hg. Therese Nickl und Heinrich Schnitzler. Frankfurt am Main: \emph{S. Fischer} 1964, S. 220.} }\toendnotes[C]{\smallbreak}\pstart{}{\pb}Herrn Arthur
                  Schnitzler\pend{}\pstart{}Marienlyst\oindex{Marienlyst@\textbf{Marienlyst}|pw}\pend{}\pstart{}Kurhaus\oindex{Kurhotellet@\textbf{Kurhotellet}|pw}\pend{}\pstart{}per Kopenhagen\oindex{Kopenhagen@\textbf{Kopenhagen}|pw}\pend{}\pstart{}Dänemark\oindex{Daenemark@\textbf{Dänemark}|pw}\pend{}{\bigskip}\pstart
           \noindent{}\centering{}\textcolor{gray}{\textbf{{\pb}Gasthof Lueg\oindex{Hotel und Pension Lueg@\textbf{Hotel und Pension Lueg}|pw}. Lueg\oindex{Lueg am Wolfgangsee@\textbf{Lueg am Wolfgangsee}|pw} bei St. Gilgen\oindex{St. Gilgen@\textbf{St. Gilgen}|pw} am
                        Wolfgang-\oindex{Wolfgangsee@\textbf{Wolfgangsee}|pw}(Aber-) See.}}\pend
           \pstart
           {\pb}27.\pend
           \pstart
           Freuen uns herzlich von Ihnen Gutes zu hören.\pend
           \pstart
           Gehen nächſte Woche 2 Tage Baireuth\oindex{Bayreuth@\textbf{Bayreuth}|pw}, ſind hier
               bis Ende Auguſt mit allen Kindern\pwindex{Hofmannsthal, Christiane von 14.05.1902 – 05.01.1987@\textsc{Hofmannsthal, Christiane von} (14.05.1902 – 05.01.1987)|pwv}\pwindex{Hofmannsthal, Raimund von 26.5.1906 – 20.03.1974@\textsc{Hofmannsthal, Raimund von} (26.5.1906 – 20.03.1974)|pwv}\pwindex{Hofmannsthal, Franz von 20.10.1903 – 13.07.1929@\textsc{Hofmannsthal, Franz von} (20.10.1903 – 13.07.1929)|pwv}. Bitte bald wieder
               paar Zeilen.\pend
           \pstart
           Ihr{\\[\baselineskip]}\spacefill\mbox{Hugo.}\pend
           \leftskip=0em{}
         
         \endnumbering\mylabel{h}\end{ledgroupsized}  \newcommand{\dateiname}{L01617}\newcommand{\titel}{Hugo von Hofmannsthal an Arthur Schnitzler, 27. 7. 1906}\newcommand{\editorInnen}{Martin Anton Müller und Gerd-Hermann Susen}%% latex-leseansicht-abspann.tex
%% Abspann für die Leseansicht.
%% Der Schalter \ifkorrekturansicht ist bereits durch den Vorspann gesetzt.

%% latex-abspann.tex
%% Gemeinsamer Abspann für Korrekturansicht und Leseansicht.
%% Setzt den Schalter \ifkorrekturansicht voraus (gesetzt in den
%% einbindenden Dateien latex-korrekturansicht-abspann.tex bzw.
%% latex-leseansicht-abspann.tex).
%% ---------------------------------------------------------------

\normalsize

% Das esempio-Environment wird nur in der Leseansicht benötigt
\ifkorrekturansicht\else
\newenvironment{esempio}[3]%
{
    \vspace{1.5ex}
    \rlap{\underline{#1}}
    \par
    \setlength{\parindent}{0cm}
    \nopagebreak
    \leftskip=#2cm
    \rightskip=#3cm
}
{
    \par
}
\fi

\doendnotes{C}
\bigskip
\vfill

\clearpage

\footnotesize

\ifkorrekturansicht
  \lohead{\textsc{register}}
\fi

% theindex-Environment neu definieren ohne reledmac
\makeatletter
\renewenvironment{theindex}{%
  \ifkorrekturansicht
    \section*{\indexname}%
  \else
    \subsubsection*{Index der erwähnten Entitäten}%
  \fi
  \setlength{\parindent}{0pt}%
  \setlength{\parskip}{0pt plus 0.3pt}%
  \let\item\@idxitem
}{%
  \ifkorrekturansicht\clearpage\fi
}
\makeatother

\IfFileExists{\jobname-pw.ind}{\input{\jobname-pw.ind}}{}

% Quellenangabe nur in der Leseansicht
\ifkorrekturansicht\else
% Fallback-Definitionen, falls die .tex-Datei \titel etc. nicht gesetzt hat
\providecommand{\titel}{}
\providecommand{\editorInnen}{}
\providecommand{\dateiname}{\jobname}

\vspace{3cm}

\vfill

\footnotesize
\textsc{Quelle}: \titel. Herausgegeben von {\editorInnen}. In: \emph{Arthur Schnitzler: Briefwechsel mit Autorinnen und Autoren}.
 Digitale Edition, https://schnitzler-briefe.acdh.oeaw.ac.at/{\dateiname}.html (Stand \today)
\fi

\end{document}


      