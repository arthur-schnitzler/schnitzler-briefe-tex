%% latex-leseansicht-vorspann.tex
%% Vorspann für die Leseansicht.
%% Lädt die gemeinsame Datei latex-vorspann.tex mit nicht gesetztem Schalter.

\newif\ifkorrekturansicht
\korrekturansichtfalse

\input{../tex-inputs/latex-vorspann}


\section[Arthur Schnitzler an Gustav Schwarzkopf, 23. 1. 1911]{L04170 Arthur Schnitzler an Gustav Schwarzkopf, 23. 1. 1911}
\nopagebreak\mylabel{L04170v}
\rehead{ }\normalsize\beginnumbering\briefempfaengerindex{Schwarzkopf, Gustav@\textsc{Schwarzkopf, Gustav}!zzzSchnitzler, Arthur@\emph{von Arthur Schnitzler}!1911-01-231@{23. 1. 1911}|(be}
\toendnotes[C]{\smallbreak\pagebreak[2]}
\correspDesc{Versand  durch Arthur Schnitzler am 23. 1. 1911 in Wien
\newline{}Übermittlung  am 24. 1. 1911 in Wien
\newline{}Erhalt  durch Gustav Schwarzkopf im Zeitraum [23. 1. 1911 – 26. 1. 1911?] in Wien}\toendnotes[C]{\smallbreak}
\Standort{CUL, Schnitzler, B 96.}
\physDesc{Postkarte, 202 Zeichen
\newline{}Handschrift: Bleistift, deutsche Kurrent
\newline{}Versand: 1) Stempel: »\nobreak{}\oindex{Wien@\textbf{Wien}, \emph{Verwaltungsgebiet}|pwk}Wien 101, 24 I \textcolor{gray}{11}, VII\nobreak{}«.   2) Stempel: »\nobreak{}\oindex{I., Innere Stadt@\textbf{I., Innere Stadt}, \emph{Verwaltungsgebiet}|pwk}1/\textsubscript{1} Wien 2, 24 \textcolor{gray}{I} 11\nobreak{}«.  3) Stempel: »\nobreak{}\oindex{Gemäldegalerie Alte Meister@\textbf{Gemäldegalerie Alte Meister}, \emph{Galerie}|pwk}1/\textsubscript{1} Wien T{[}ele{]}g\textcolor{gray}{r}a{[}ph{]}enzentralstation\nobreak{}«. }\toendnotes[C]{\smallbreak}\pstart{}{\pb}\textcolor{gray}{\textbf{Dr. Arthur Schnitzler}}\pend{}\pstart{}\textcolor{gray}{\textbf{Wien XVIII. Sternwartestrasse 71\oindex{Wien@\textbf{Wien}!XVIII., Währing@\textbf{XVIII., Währing}!Sternwartestraße 71@\textbf{Sternwartestraße 71}, \emph{Wohngebäude}|pw}}}\pend{}{\bigskip}\pstart{}\textsc{Herrn Gustav Schwarzkopf}\pend{}\pstart{}Wien I\oindex{I., Innere Stadt@\textbf{I., Innere Stadt}, \emph{Verwaltungsgebiet}|pw}\pend{}\pstart{}\textsc{Tiefer Graben 23}\oindex{Wien@\textbf{Wien}!I., Innere Stadt@\textbf{I., Innere Stadt}!Tiefer Graben 23@\textbf{Tiefer Graben 23}, \emph{Wohngebäude}|pw}.\pend{}{\bigskip}\vspace{1em}
\pstart
           \raggedleft{}{\pb}Montag\pend
           
\pstart
           \raggedleft{}abend\pend
           
\pstart{}lieber Guſtav,\pend\vspace{0.5em}
\pstart
           \label{K_L04170-1v}\edtext{morgen}{\lemma{\textnormal{\emph{morgen}}}\Cendnote{\textnormal{Vgl. A. S.: \emph{Wiener Schnitzler}, 24. 1. 1911.}}}\label{K_L04170-1}{ }11.20 fahren wir
               auf den Se{\geminationm}ering\oindex{Semmering@\textbf{Semmering}, \emph{Verwaltungsgebiet}|pw} (\textsc{S. B. Hotel\oindex{Südbahnhotel [Semmering]@\textbf{Südbahnhotel [Semmering]}, \emph{Hotel}|pw}})
      denken \label{K_L04170-2v}\edtext{bis Freitag}{\lemma{\textnormal{\emph{bis Freitag}}}\Cendnote{\textnormal{Tatsächlich fuhren sie
         am Freitag, dem 27. 1. 1911 retour.}}}\label{K_L04170-2} zu bleiben,
      wär ſehr ſchön, \label{K_L04170-3v}\edtext{we{\geminationn} Sie hinauf
               kämen}{\lemma{\textnormal{\emph{wenn Sie hinauf
               kämen}}}\Cendnote{\textnormal{Schwarzkopf\pwindex{Schwarzkopf, Gustav 7.\,11.\,1853 Wien – 13.\,11.\,1939 ebd.@\textsc{Schwarzkopf, Gustav} (7.\,11.\,1853 Wien – 13.\,11.\,1939 ebd.), \emph{Schriftsteller}|pwk} kam nicht.}}}\label{K_L04170-3}.\pend
           
\pstart
           Herzlichſt{\\[\baselineskip]} Ihr \spacefill\mbox{A.}\pend
           \leftskip=0em{}\selectlanguage{ngerman}\endnumbering\briefempfaengerindex{Schwarzkopf, Gustav@\textsc{Schwarzkopf, Gustav}!zzzSchnitzler, Arthur@\emph{von Arthur Schnitzler}!1911-01-231@{23. 1. 1911}|)be}\mylabel{L04170h}
\begin{anhang}
\end{anhang}\newcommand{\dateiname}{L04170}\newcommand{\titel}{Arthur Schnitzler an Gustav Schwarzkopf, 23. 1. 1911}\newcommand{\editorInnen}{Herausgegeben von Jahnke, SelmaMüller, Martin Anton}%% latex-leseansicht-abspann.tex
%% Abspann für die Leseansicht.
%% Der Schalter \ifkorrekturansicht ist bereits durch den Vorspann gesetzt.

%% latex-abspann.tex
%% Gemeinsamer Abspann für Korrekturansicht und Leseansicht.
%% Setzt den Schalter \ifkorrekturansicht voraus (gesetzt in den
%% einbindenden Dateien latex-korrekturansicht-abspann.tex bzw.
%% latex-leseansicht-abspann.tex).
%% ---------------------------------------------------------------

\normalsize

% Das esempio-Environment wird nur in der Leseansicht benötigt
\ifkorrekturansicht\else
\newenvironment{esempio}[3]%
{
    \vspace{1.5ex}
    \rlap{\underline{#1}}
    \par
    \setlength{\parindent}{0cm}
    \nopagebreak
    \leftskip=#2cm
    \rightskip=#3cm
}
{
    \par
}
\fi

\doendnotes{C}
\bigskip
\vfill

\clearpage

\footnotesize

\ifkorrekturansicht
  \lohead{\textsc{register}}
\fi

% theindex-Environment neu definieren ohne reledmac
\makeatletter
\renewenvironment{theindex}{%
  \ifkorrekturansicht
    \section*{\indexname}%
  \else
    \subsubsection*{Index der erwähnten Entitäten}%
  \fi
  \setlength{\parindent}{0pt}%
  \setlength{\parskip}{0pt plus 0.3pt}%
  \let\item\@idxitem
}{%
  \ifkorrekturansicht\clearpage\fi
}
\makeatother

\IfFileExists{\jobname-pw.ind}{\input{\jobname-pw.ind}}{}

% Quellenangabe nur in der Leseansicht
\ifkorrekturansicht\else
% Fallback-Definitionen, falls die .tex-Datei \titel etc. nicht gesetzt hat
\providecommand{\titel}{}
\providecommand{\editorInnen}{}
\providecommand{\dateiname}{\jobname}

\vspace{3cm}

\vfill

\footnotesize
\textsc{Quelle}: \titel. Herausgegeben von {\editorInnen}. In: \emph{Arthur Schnitzler: Briefwechsel mit Autorinnen und Autoren}.
 Digitale Edition, https://schnitzler-briefe.acdh.oeaw.ac.at/{\dateiname}.html (Stand \today)
\fi

\end{document}


