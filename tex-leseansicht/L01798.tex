\input{../tex-inputs/latex-pdf-vorspann}
\begin{center}
            \textcolor{red}{ENTWURF. ENTZIFFERUNG NOCH NICHT KORREKTURGELESEN}
                      \end{center}
            
               \section[Arthur Schnitzler an Albert Ehrenstein, 9. 11. 1908]{ Arthur Schnitzler an Albert Ehrenstein, 9. 11. 1908}\nopagebreak\mylabel{v}\rehead{ }\begin{ledgroupsized}[t]{13cm}\normalsize\beginnumbering\briefempfaengerindex{Ehrenstein, Albert@\textsc{Ehrenstein, Albert}!zzzSchnitzler, Arthur@\emph{von Arthur Schnitzler}!1908-11-091@{9. 11. 1908}|(be} \toendnotes[C]{\smallbreak\pagebreak[2]} \Standort{Jerusalem, The National Library of Israel, ARC. Ms. Var. 306 1 118.}
\physDesc{Briefkarte
\newline{}Handschrift: schwarze Tinte, deutsche Kurrent}\toendnotes[C]{\smallbreak}\pstart
           \noindent{}{\pb}\textcolor{gray}{\textbf{Dr. Arthur Schnitzler}}\hfill 9. 11. 08\pend
           \pstart
           \textcolor{gray}{\textbf{Wien XVIII. Spoettelgasse 7\oindex{Edmund-Weiss-Gasse@\textbf{Edmund-Weiß-Gasse}|pw}.}}\pend
           \pstart{}werther Herr Ehrenſtein,\pend\pstart
           we{\geminationn}{ }Sie Freitag etwa
                        ½ 7 Abend{ }\label{K_L01798_1v}\edtext{Zeit haben}{\lemma{\textnormal{\emph{Zeit haben}}}\Cendnote{\textnormal{Das Treffen fand am 13. 11. 1908
                   statt.}}}\label{K_L01798_1h}, ſo ſind Sie willko{\geminationm}en Ihrem\pend
           \pstart ergeb\textcolor{gray}{enen}{ }\spacefill\mbox{A. S.}\pend{}\endnumbering\briefempfaengerindex{Ehrenstein, Albert@\textsc{Ehrenstein, Albert}!zzzSchnitzler, Arthur@\emph{von Arthur Schnitzler}!1908-11-091@{9. 11. 1908}|)be}\mylabel{h}\end{ledgroupsized}  \newcommand{\dateiname}{L01798}\newcommand{\titel}{Arthur Schnitzler an Albert Ehrenstein, 9. 11. 1908}\newcommand{\editorInnen}{Martin Anton Müller und Gerd-Hermann Susen}\input{../tex-inputs/latex-pdf-abspann}
      