%% latex-korrekturansicht-vorspann.tex
%% Vorspann für die Korrekturansicht.
%% Lädt die gemeinsame Datei latex-vorspann.tex mit gesetztem Schalter.

\newif\ifkorrekturansicht
\korrekturansichttrue

\input{../tex-inputs/latex-vorspann}


\section[Arthur Schnitzler an Hermann Bahr, 28. 3. 1903]{L01281 Arthur Schnitzler an Hermann Bahr, 28. 3. 1903}
\nopagebreak\mylabel{L01281v}
\rehead{ }\normalsize\beginnumbering\briefempfaengerindex{Bahr, Hermann@\textsc{Bahr, Hermann}!zzzSchnitzler, Arthur@\emph{von Arthur Schnitzler}!1903-03-281@{28. 3. 1903}|(be}
\toendnotes[C]{\smallbreak\pagebreak[2]}\Standort{TMW, HS AM 23357 Ba.}
\physDesc{Brief, 1 Blatt, 2 Seiten, 624 Zeichen
\newline{}Handschrift: schwarze Tinte, deutsche Kurrent
\newline{}Ordnung: Lochung }
\buchAbdrucke{\weitereDrucke{1) Arthur Schnitzler: \emph{The Letters of Arthur Schnitzler to Hermann Bahr}. Chapel Hill: \emph{The University of North Carolina Press} 1978, S. 80.} \weitereDrucke{2) Hermann Bahr, Arthur Schnitzler: \emph{Briefwechsel, Aufzeichnungen, Dokumente (1891–1931)}. Göttingen: \emph{Wallstein} 2018, S. 256.} }\toendnotes[C]{\smallbreak}
\pstart
           \noindent{}{\pb}lieber Hermann, in etwa 8 Tagen erſcheint im Wiener Verlag\orgindex{Wiener Verlag@Wiener Verlag|pw} der »Reigen\pwindex{Reigen. Zehn Dialoge@\emph{Reigen. Zehn Dialoge}|pw}«.
               Ich weiſs nicht ob du Luſt haſt drüber zu ſchreiben. Falls du aber daran denken
               ſollteſt, wäre es mir natürlich beſonders lieb, wenn deine Anſicht über das Buch \strikeout{ſchon} mit dem Buch zugleich oder gleich nach ihm in die
               Welt käme, – noch vor dem zu erwartenden Heuchel- und {\pb}Schimpfchor beleidigter
               Sittlinge.\pend
           
\pstart
           Das wollt ich dir ſchon neulich ſagen dich aber auch bitten, dieſe ganze Bemerkung
               als ungeſagt oder ungehört zu betrachten, we{\geminationm} es dich
               nicht \uline{freut}, dich über die zehn Dialoge\pwindex{Reigen. Zehn Dialoge@\emph{Reigen. Zehn Dialoge}|pwv} vernehmen zu laſſen.\pend
           
\pstart
           Ich grüße dich von Herzen als{\\[\baselineskip]}dein getreuer{\\[\baselineskip]}\spacefill\mbox{Arthur}\pend
           \leftskip=0em{}
\pstart
           28. \label{T_L01281-1v}\edtext{\textcolor{gray}{3}.}{\lemma{\textnormal{\emph{3.}}}\Cendnote{\textnormal{unterhalb der
                        schwer lesbaren Ziffer »3« von unbekannter Hand fälschlich
                           »9.« geschrieben}}}\label{T_L01281-1} 903. \pend
           \selectlanguage{ngerman}\endnumbering\briefempfaengerindex{Bahr, Hermann@\textsc{Bahr, Hermann}!zzzSchnitzler, Arthur@\emph{von Arthur Schnitzler}!1903-03-281@{28. 3. 1903}|)be}\mylabel{L01281h}  \normalsize

\doendnotes{C}
\bigskip
\vfill

\clearpage

\footnotesize

\lohead{\textsc{register}}

% Definiere theindex-Environment komplett neu ohne reledmac
\makeatletter
\renewenvironment{theindex}{%
  \section*{\indexname}%
  \setlength{\parindent}{0pt}%
  \setlength{\parskip}{0pt plus 0.3pt}%
  \let\item\@idxitem
}{%
  \clearpage
}
\makeatother

\IfFileExists{\jobname-pw.ind}{\input{\jobname-pw.ind}}{}

\end{document}

      