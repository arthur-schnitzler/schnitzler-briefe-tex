%% latex-korrekturansicht-vorspann.tex
%% Vorspann für die Korrekturansicht.
%% Lädt die gemeinsame Datei latex-vorspann.tex mit gesetztem Schalter.

\newif\ifkorrekturansicht
\korrekturansichttrue

\input{../tex-inputs/latex-vorspann}


\section[Richard Beer-Hofmann an Arthur Schnitzler, 2. 9. 1899]{L00967 Richard Beer-Hofmann an Arthur Schnitzler, 2. 9. 1899}
\nopagebreak\mylabel{L00967v}
\rehead{ }\normalsize\beginnumbering\briefempfaengerindex{Schnitzler, Arthur@\textsc{Schnitzler, Arthur}!zzzBeer-Hofmann, Richard@\emph{von Richard Beer-Hofmann}!1899-09-021@{2. 9. 1899}|(be}
\toendnotes[C]{\smallbreak\pagebreak[2]}\Standort{CUL, Schnitzler, B 8.}
\physDesc{Bildpostkarte, 245 Zeichen
\newline{}Handschrift: schwarze Tinte, lateinische Kurrent
\newline{}Versand: 1) Stempel: »\nobreak{}\oindex{Seeboden@\textbf{Seeboden}, \emph{A.ADM3}|pwk}Seeboden, 9 9 99\nobreak{}«.   2) Stempel: »\nobreak{}\oindex{Bad Ischl@\textbf{Bad Ischl}, \emph{P.PPL}|pwk}Ischl, 4. 9. 99, 10–11V\nobreak{}«. 
\newline{}Ordnung: mit Bleistift von unbekannter Hand nummeriert:
                                    »139« }
\buchAbdrucke{\weitereDrucke{Arthur Schnitzler, Richard Beer-Hofmann: \emph{Briefwechsel 1891–1931}. Wien, Zürich: \emph{Europaverlag} 1992, S. 134.} }\toendnotes[C]{\smallbreak}\pstart{}{\pb}D\textsuperscript{r}
                  Arthur Schnitzler\pend{}\pstart{}Ischl\oindex{Bad Ischl@\textbf{Bad Ischl}, \emph{P.PPL}|pw}\pend{}\pstart{}Pension Petter\oindex{Hotel und Pension Rudolfshoehe (Leopold Petter)@\textbf{Hotel und Pension Rudolfshöhe (Leopold Petter)}, \emph{Hotel (K.HTL)}|pw}. \pend{}{\bigskip}
\pstart
           \noindent{}\centering{}{\pb}\textcolor{gray}{\textbf{Villa Platzer\oindex{Villa Platzer@\textbf{Villa Platzer}, \emph{Gebäude (K.GBD)}|pw}}}\pend
           \vspace{1em}
\pstart
           \raggedleft{}{\pb}\textcolor{gray}{\textbf{Seeboden}}\oindex{Seeboden@\textbf{Seeboden}, \emph{A.ADM3}|pw}, { }2/IX \textcolor{gray}{\textbf{18}}99\pend
           \vspace{0.5em}
\pstart
           Erst hab ich »\label{K_L00967-1v}\edtext{Gigarl}{\lemma{\textnormal{\emph{Gigarl}}}\Cendnote{\textnormal{österreichisch: Geck}}}\label{K_L00967-1}« gelesen, dann
               begriffen daß es »Gigant« heißen \uline{muß}. Beflaggen wäre
               zu wenig gewesen. Von \uline{6. Sept} an ist meine Adresse
                  \uline{Sachsenburg\oindex{Sachsenburg@\textbf{Sachsenburg}, \emph{A.ADM3}|pw}{ }Kärnten\oindex{Kaernten@\textbf{Kärnten}, \emph{A.ADM1}|pw}{ }Gasthof ›Fritz‹\oindex{Gasthof Fritz@\textbf{Gasthof Fritz}, \emph{Gastgewerbegebäude (K.GGW)}|pw}.}{ }Hugo\pwindex{Hofmannsthal, Hugo von 1874-02-01 – 1929-07-15@\textsc{Hofmannsthal, Hugo von} (1874-02-01 – 1929-07-15), \emph{Schriftsteller/Schriftstellerin}|pw} u Sie grüßt herzl.\pend
           \pstart \spacefill\mbox{R.}\pend{}\selectlanguage{ngerman}\endnumbering\briefempfaengerindex{Schnitzler, Arthur@\textsc{Schnitzler, Arthur}!zzzBeer-Hofmann, Richard@\emph{von Richard Beer-Hofmann}!1899-09-021@{2. 9. 1899}|)be}\mylabel{L00967h}  \normalsize

\doendnotes{C}
\bigskip
\vfill

\clearpage

\footnotesize

\lohead{\textsc{register}}

% Definiere theindex-Environment komplett neu ohne reledmac
\makeatletter
\renewenvironment{theindex}{%
  \section*{\indexname}%
  \setlength{\parindent}{0pt}%
  \setlength{\parskip}{0pt plus 0.3pt}%
  \let\item\@idxitem
}{%
  \clearpage
}
\makeatother

\IfFileExists{\jobname-pw.ind}{\input{\jobname-pw.ind}}{}

\end{document}

      