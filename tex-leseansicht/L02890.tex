%% latex-leseansicht-vorspann.tex
%% Vorspann für die Leseansicht.
%% Lädt die gemeinsame Datei latex-vorspann.tex mit nicht gesetztem Schalter.

\newif\ifkorrekturansicht
\korrekturansichtfalse

\input{../tex-inputs/latex-vorspann}


\section[ Paul Goldmann an Arthur Schnitzler, 11. 10. [1899]]{L02890 Paul Goldmann an Arthur Schnitzler,  11. 10. [1899]}
\nopagebreak\mylabel{L02890v}
\rehead{ }\normalsize\beginnumbering\briefempfaengerindex{Schnitzler, Arthur@\textsc{Schnitzler, Arthur}!zzzGoldmann, Paul@\emph{von Paul Goldmann}!1899-10-111@{11. 10. [1899]}|(be}
\toendnotes[C]{\smallbreak\pagebreak[2]}
\correspDesc{Versand  durch Paul Goldmann am 11. 10. [1899] in Venedig
\newline{}Erhalt  durch Arthur Schnitzler im Zeitraum [12. 10. 1899 – 16. 10. 1899?] in Wien}\toendnotes[C]{\smallbreak}
\Standort{DLA, A:Schnitzler, HS.NZ85.1.3169.}
\physDesc{Brief, 1 Blatt, 2 Seiten, 1075 Zeichen
\newline{}Handschrift: schwarze Tinte, deutsche Kurrent
\newline{}Schnitzler: mit Bleistift das Jahr »99« vermerkt }\toendnotes[C]{\smallbreak}
\pstart
           \centering{}{\pb}\textcolor{gray}{\textbf{\textbf{HÔTEL ROME\oindex{Hôtel de Rome et Pension Suisse@\textbf{Hôtel de Rome et Pension Suisse}, \emph{Hotel}|pw}}}}\pend
           
\pstart
           \centering{}\textcolor{gray}{\textbf{\begin{otherlanguage}{french}ET \textbf{PENSION SUISSE\oindex{Hôtel de Rome et Pension Suisse@\textbf{Hôtel de Rome et Pension Suisse}, \emph{Hotel}|pw}}\end{otherlanguage}}}\pend
           
\pstart
           \centering{}\textcolor{gray}{\textbf{\begin{otherlanguage}{french}VENISE\oindex{Venedig@\textbf{Venedig}|pw}\end{otherlanguage}}}\pend
           
\pstart
           Venedig\oindex{Venedig@\textbf{Venedig}|pw}{ }11. Oktober.\pend
           
\pstart{}Mein lieber Freund,\pend\vspace{0.5em}
\pstart
           Herzlichſten Dark für Deine Telegramme. Auch das nach Florenz\oindex{Florenz@\textbf{Florenz}|pw} erhielt ich hier\oindex{Venedig@\textbf{Venedig}|pwv}. Ich will Freitag{ }Mittag um 2 von hier wegfahren und bin dann \label{K_L02890-1v}\edtext{Samſtag}{\lemma{\textnormal{\emph{Samstag}}}\Cendnote{\textnormal{Goldmann\pwindex{Goldmann, Paul 31.\,1.\,1865 Breslau – 25.\,9.\,1935 Wien@\textsc{Goldmann, Paul} (31.\,1.\,1865 Breslau – 25.\,9.\,1935 Wien), \emph{Schriftsteller, Journalist}|pwk} kam bereits am Freitag{ }Abend, 13. 10. 1899, in Wien\oindex{Wien@\textbf{Wien}, \emph{Verwaltungsgebiet}|pwk} an.}}}\label{K_L02890-1}{ }\introOben{}\strikeout{fr\textcolor{gray}{ü}}{ }früh\introOben{} um halb oder dreiviertel acht in Wien\oindex{Wien@\textbf{Wien}, \emph{Verwaltungsgebiet}|pw}. Ich bitte Dich \uline{auf das Dringendſte}
               nicht zur Bahn zu kommen. \strikeout{Du} Mir iſt damit nicht im
               Mindeſten gedient. Du aber müßteſt vor 7 Uhr aufſtehen, wäreſt dann den
               ganzen Tag müde, und ich hätte nichts von Dir. Bitte, laß’ es alſo bleiben! {\pb}Ich finde den Weg{ }ſchon ohne Dich und komme direkt
                  \strikeout{z\textcolor{gray}{u}} von der Bahn zu Dir. Es iſt mir ohnehin{ }ſchon äußerſt peinlich,{ }ſo früh bei
               Euch eintreffen zu müſſen; aber es iſt der einzig mögliche Zug. Immerhin bitte ich
               Dich, mich{ }ſchon im Voraus bei Deiner Frau Mutter\pwindex{Schnitzler, Louise 8.\,7.\,1840 Kőszeg – 9.\,9.\,1911 Wien@\textsc{Schnitzler, Louise} (8.\,7.\,1840 Kőszeg – 9.\,9.\,1911 Wien)|pwv} zu entſchuldigen.\pend
           
\pstart
           Ich muß{ }ſo lange hierbleiben, weil ich Depeſchen aus Frankfurt\oindex{Frankfurt am Main@\textbf{Frankfurt am Main}, \emph{Hauptstadt}|pw} erwarte. Dort gehen fürchterliche Dinge vor. \label{K_L02890-2v}\edtext{Eines\pwindex{?? [Frau, die von Goldmanns und Rottenbergs Beziehung wusste] @\textsc{?? [Frau, die von Goldmanns und Rottenbergs Beziehung wusste]}|pwv} der infamſten und
               gemeinſten Klatſchweiber}{\lemma{\textnormal{\emph{Eines … Klatschweiber}}}\Cendnote{\textnormal{nicht
                  identifiziert}}}\label{K_L02890-2} der Stadt\oindex{Frankfurt am Main@\textbf{Frankfurt am Main}, \emph{Hauptstadt}|pwv} hat dem \label{K_L02890-3v}\edtext{Gemahl\pwindex{Rottenberg, Ludwig 11.\,10.\,1864 Czernowitz – 6.\,5.\,1932 Frankfurt am Main@\textsc{Rottenberg, Ludwig} (11.\,10.\,1864 Czernowitz – 6.\,5.\,1932 Frankfurt am Main), \emph{Kapellmeister}|pwv}}{\lemma{\textnormal{\emph{Gemahl}}}\Cendnote{\textnormal{Ludwig Rottenberg\pwindex{Rottenberg, Ludwig 11.\,10.\,1864 Czernowitz – 6.\,5.\,1932 Frankfurt am Main@\textsc{Rottenberg, Ludwig} (11.\,10.\,1864 Czernowitz – 6.\,5.\,1932 Frankfurt am Main), \emph{Kapellmeister}|pwk}, Ehemann von Goldmanns\pwindex{Goldmann, Paul 31.\,1.\,1865 Breslau – 25.\,9.\,1935 Wien@\textsc{Goldmann, Paul} (31.\,1.\,1865 Breslau – 25.\,9.\,1935 Wien), \emph{Schriftsteller, Journalist}|pwk} Geliebter Theodore Rottenberg\pwindex{Rottenberg, Theodore 7.\,9.\,1875 – 5.\,4.\,1945 Limburg an der Lahn@\textsc{Rottenberg, Theodore} (7.\,9.\,1875 – 5.\,4.\,1945 Limburg an der Lahn)|pwk}}}}\label{K_L02890-3} Alles hinterbracht, und Alles{ }ſcheint zu Ende zu gehen. Ich laufe hier herum
               wie ein Verzweifelter und weiß nicht, was ich anfangen{ }ſoll.\pend
           
\pstart
           Viele treue Grüße! {\\[\baselineskip]}Dein {\\[\baselineskip]}\spacefill\mbox{Paul Goldmann}\pend
           \leftskip=0em{}\selectlanguage{ngerman}\endnumbering\briefempfaengerindex{Schnitzler, Arthur@\textsc{Schnitzler, Arthur}!zzzGoldmann, Paul@\emph{von Paul Goldmann}!1899-10-111@{11. 10. [1899]}|)be}\mylabel{L02890h}  \newcommand{\dateiname}{L02890}\newcommand{\titel}{Paul Goldmann an Arthur Schnitzler, 11. 10. [1899]}\newcommand{\editorInnen}{Martin Anton Müller und Laura Untner}%% latex-leseansicht-abspann.tex
%% Abspann für die Leseansicht.
%% Der Schalter \ifkorrekturansicht ist bereits durch den Vorspann gesetzt.

%% latex-abspann.tex
%% Gemeinsamer Abspann für Korrekturansicht und Leseansicht.
%% Setzt den Schalter \ifkorrekturansicht voraus (gesetzt in den
%% einbindenden Dateien latex-korrekturansicht-abspann.tex bzw.
%% latex-leseansicht-abspann.tex).
%% ---------------------------------------------------------------

\normalsize

% Das esempio-Environment wird nur in der Leseansicht benötigt
\ifkorrekturansicht\else
\newenvironment{esempio}[3]%
{
    \vspace{1.5ex}
    \rlap{\underline{#1}}
    \par
    \setlength{\parindent}{0cm}
    \nopagebreak
    \leftskip=#2cm
    \rightskip=#3cm
}
{
    \par
}
\fi

\doendnotes{C}
\bigskip
\vfill

\clearpage

\footnotesize

\ifkorrekturansicht
  \lohead{\textsc{register}}
\fi

% theindex-Environment neu definieren ohne reledmac
\makeatletter
\renewenvironment{theindex}{%
  \ifkorrekturansicht
    \section*{\indexname}%
  \else
    \subsubsection*{Index der erwähnten Entitäten}%
  \fi
  \setlength{\parindent}{0pt}%
  \setlength{\parskip}{0pt plus 0.3pt}%
  \let\item\@idxitem
}{%
  \ifkorrekturansicht\clearpage\fi
}
\makeatother

\IfFileExists{\jobname-pw.ind}{\input{\jobname-pw.ind}}{}

% Quellenangabe nur in der Leseansicht
\ifkorrekturansicht\else
% Fallback-Definitionen, falls die .tex-Datei \titel etc. nicht gesetzt hat
\providecommand{\titel}{}
\providecommand{\editorInnen}{}
\providecommand{\dateiname}{\jobname}

\vspace{3cm}

\vfill

\footnotesize
\textsc{Quelle}: \titel. Herausgegeben von {\editorInnen}. In: \emph{Arthur Schnitzler: Briefwechsel mit Autorinnen und Autoren}.
 Digitale Edition, https://schnitzler-briefe.acdh.oeaw.ac.at/{\dateiname}.html (Stand \today)
\fi

\end{document}


