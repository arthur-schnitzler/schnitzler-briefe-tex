%% latex-korrekturansicht-vorspann.tex
%% Vorspann für die Korrekturansicht.
%% Lädt die gemeinsame Datei latex-vorspann.tex mit gesetztem Schalter.

\newif\ifkorrekturansicht
\korrekturansichttrue

\input{../tex-inputs/latex-vorspann}


\section[ Paul Goldmann an Arthur Schnitzler, 11. 10. {[}1899{]}]{L02890 Paul Goldmann an Arthur Schnitzler, 11. 10. {[}1899{]}}
\nopagebreak\mylabel{L02890v}
\rehead{ }\normalsize\beginnumbering\briefempfaengerindex{Schnitzler, Arthur@\textsc{Schnitzler, Arthur}!zzzGoldmann, Paul@\emph{von Paul Goldmann}!1899-10-111@{11. 10. {[}1899{]}}|(be}
\toendnotes[C]{\smallbreak\pagebreak[2]}\Standort{DLA, A:Schnitzler, HS.NZ85.1.3169.}
\physDesc{Brief, 1 Blatt, 2 Seiten, 1075 Zeichen
\newline{}Handschrift: schwarze Tinte, deutsche Kurrent
\newline{}Schnitzler: mit Bleistift das Jahr »99« vermerkt }\toendnotes[C]{\smallbreak}
\pstart
           \centering{}{\pb}\textcolor{gray}{\textbf{\textbf{HÔTEL ROME\oindex{Hôtel de Rome et Pension Suisse@\textbf{Hôtel de Rome et Pension Suisse}, \emph{Hotel (K.HTL)}|pw}}}}\pend
           
\pstart
           \centering{}\textcolor{gray}{\textbf{\begin{otherlanguage}{french}ET \textbf{PENSION SUISSE\oindex{Hôtel de Rome et Pension Suisse@\textbf{Hôtel de Rome et Pension Suisse}, \emph{Hotel (K.HTL)}|pw}}\end{otherlanguage}}}\pend
           
\pstart
           \centering{}\textcolor{gray}{\textbf{\begin{otherlanguage}{french}VENISE\oindex{Venedig@\textbf{Venedig}, \emph{P.PPLA}|pw}\end{otherlanguage}}}\pend
           
\pstart
           Venedig\oindex{Venedig@\textbf{Venedig}, \emph{P.PPLA}|pw}{ }11. Oktober.\pend
           
\pstart{}Mein lieber Freund,\pend\vspace{0.5em}
\pstart
           Herzlichſten Dark für Deine Telegramme. Auch das nach Florenz\oindex{Florenz@\textbf{Florenz}, \emph{P.PPLA}|pw} erhielt ich hier\oindex{Venedig@\textbf{Venedig}, \emph{P.PPLA}|pwv}. Ich will Freitag{ }Mittag um 2 von hier wegfahren und bin dann \label{K_L02890-1v}\edtext{Samſtag}{\lemma{\textnormal{\emph{Samſtag}}}\Cendnote{\textnormal{Goldmann\pwindex{Goldmann, Paul 31.01.1865 – 25.09.1935@\textsc{Goldmann, Paul} (31.01.1865 – 25.09.1935), \emph{Schriftsteller/Schriftstellerin, Journalist/Journalistin}|pwk} kam bereits am Freitag{ }Abend, 13. 10. 1899, in Wien\oindex{Wien@\textbf{Wien}, \emph{A.ADM2}|pwk} an.}}}\label{K_L02890-1}{ }\introOben{}\strikeout{fr\textcolor{gray}{ü}}{ }früh\introOben{} um halb oder dreiviertel acht in Wien\oindex{Wien@\textbf{Wien}, \emph{A.ADM2}|pw}. Ich bitte Dich \uline{auf das Dringendſte}
               nicht zur Bahn zu kommen. \strikeout{Du} Mir iſt damit nicht im
               Mindeſten gedient. Du aber müßteſt vor 7 Uhr aufſtehen, wäreſt dann den
               ganzen Tag müde, und ich hätte nichts von Dir. Bitte, laß’ es alſo bleiben! {\pb}Ich finde den Weg ſchon ohne Dich und komme direkt
                  \strikeout{z\textcolor{gray}{u}} von der Bahn zu Dir. Es iſt mir ohnehin ſchon äußerſt peinlich, ſo früh bei
               Euch eintreffen zu müſſen; aber es iſt der einzig mögliche Zug. Immerhin bitte ich
               Dich, mich ſchon im Voraus bei Deiner Frau Mutter\pwindex{Schnitzler, Louise 1840-07-08 – 1911-09-09@\textsc{Schnitzler, Louise} (1840-07-08 – 1911-09-09)|pwv} zu entſchuldigen.\pend
           
\pstart
           Ich muß ſo lange hierbleiben, weil ich Depeſchen aus Frankfurt\oindex{Frankfurt am Main@\textbf{Frankfurt am Main}, \emph{P.PPLA3}|pw} erwarte. Dort gehen fürchterliche Dinge vor. \label{K_L02890-2v}\edtext{Eines\pwindex{?? [Frau, die von Goldmanns und Rottenbergs Beziehung wusste] @\textsc{?? [Frau, die von Goldmanns und Rottenbergs Beziehung wusste]}|pwv} der infamſten und
               gemeinſten Klatſchweiber}{\lemma{\textnormal{\emph{Eines … Klatſchweiber}}}\Cendnote{\textnormal{nicht
                  identifiziert}}}\label{K_L02890-2} der Stadt\oindex{Frankfurt am Main@\textbf{Frankfurt am Main}, \emph{P.PPLA3}|pwv} hat dem \label{K_L02890-3v}\edtext{Gemahl\pwindex{Rottenberg, Ludwig 11.10.1864 – 6.5.1932@\textsc{Rottenberg, Ludwig} (11.10.1864 – 6.5.1932), \emph{Kapellmeister/Kapellmeisterin}|pwv}}{\lemma{\textnormal{\emph{Gemahl}}}\Cendnote{\textnormal{Ludwig Rottenberg\pwindex{Rottenberg, Ludwig 11.10.1864 – 6.5.1932@\textsc{Rottenberg, Ludwig} (11.10.1864 – 6.5.1932), \emph{Kapellmeister/Kapellmeisterin}|pwk}, Ehemann von Goldmanns\pwindex{Goldmann, Paul 31.01.1865 – 25.09.1935@\textsc{Goldmann, Paul} (31.01.1865 – 25.09.1935), \emph{Schriftsteller/Schriftstellerin, Journalist/Journalistin}|pwk} Geliebter Theodore Rottenberg\pwindex{Rottenberg, Theodore 1875-09-07 – 1945-04-05@\textsc{Rottenberg, Theodore} (1875-09-07 – 1945-04-05)|pwk}}}}\label{K_L02890-3} Alles hinterbracht, und Alles ſcheint zu Ende zu gehen. Ich laufe hier herum
               wie ein Verzweifelter und weiß nicht, was ich anfangen ſoll.\pend
           
\pstart
           Viele treue Grüße! {\\[\baselineskip]}Dein {\\[\baselineskip]}\spacefill\mbox{Paul Goldmann}\pend
           \leftskip=0em{}\selectlanguage{ngerman}\endnumbering\briefempfaengerindex{Schnitzler, Arthur@\textsc{Schnitzler, Arthur}!zzzGoldmann, Paul@\emph{von Paul Goldmann}!1899-10-111@{11. 10. {[}1899{]}}|)be}\mylabel{L02890h}  \normalsize

\doendnotes{C}
\bigskip
\vfill

\clearpage

\footnotesize

\lohead{\textsc{register}}

% Definiere theindex-Environment komplett neu ohne reledmac
\makeatletter
\renewenvironment{theindex}{%
  \section*{\indexname}%
  \setlength{\parindent}{0pt}%
  \setlength{\parskip}{0pt plus 0.3pt}%
  \let\item\@idxitem
}{%
  \clearpage
}
\makeatother

\IfFileExists{\jobname-pw.ind}{\input{\jobname-pw.ind}}{}

\end{document}

      