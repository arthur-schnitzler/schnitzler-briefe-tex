%% latex-leseansicht-vorspann.tex
%% Vorspann für die Leseansicht.
%% Lädt die gemeinsame Datei latex-vorspann.tex mit nicht gesetztem Schalter.

\newif\ifkorrekturansicht
\korrekturansichtfalse

\input{../tex-inputs/latex-vorspann}


         
         \renewcommand{\erwaehntePersonen}{Personen: Hanns Heinz Ewers, Paul Goldmann, Hugo von Hofmannsthal, Dora Michaelis, Olga Schnitzler, Elisabeth Steinrück, Frida Strindberg, Irene Triesch, Jakob Wassermann}
         \renewcommand{\erwaehnteOrte}{Orte: Berlin, Dessauer Straße, Grünentorgasse, Italien, Welsberg-Taisten, Wien}
         \renewcommand{\erwaehnteWerke}{Werke: Tagebuch}
               \section[ Paul Goldmann an Arthur Schnitzler, 21. 3. {[}1901{]}]{ Paul Goldmann an Arthur Schnitzler, 21. 3. {[}1901{]}}\nopagebreak\mylabel{v}\rehead{ }\begin{ledgroupsized}[t]{13cm}\normalsize\beginnumbering \toendnotes[C]{\smallbreak\pagebreak[2]} \Standort{DLA, A:Schnitzler, HS.NZ85.1.3171.}
\physDesc{Brief, 1 Blatt, 3 Seiten, 926 Zeichen
\newline{}Handschrift: blaue Tinte, deutsche Kurrent
\newline{}Schnitzler: 1) mit Bleistift das Jahr »901« vermerkt  2) mit rotem Buntstift vier
                                 Unterstreichungen}\toendnotes[C]{\smallbreak}\pstart
           \noindent{}\raggedleft{}{\pb}\textcolor{gray}{\textbf{DESSAUERSTRASSE 19}}\oindex{Dessauer Strasse@\textbf{Dessauer Straße}|pw}\pend
           \pstart
           Berlin\oindex{Berlin@\textbf{Berlin}|pw}, 21. März.\pend
           \pstart\center{}Mein lieber Freund,\pend\pstart
           Reiſe glücklich! Komm geſund wieder! Und grüße mir das \label{K_L03062-1v}\edtext{Land der Sehnſucht\oindex{Italien@\textbf{Italien}|pwv}}{\lemma{\textnormal{\emph{Land der Sehnſucht}}}\Cendnote{\textnormal{Bezug auf Schnitzler\pwindex{Schnitzler, Arthur 15.05.1862 – 21.10.1931@\textsc{Schnitzler, Arthur} (15.05.1862 – 21.10.1931), \emph{Schriftsteller, Mediziner}|pwk}s Italien\oindex{Italien@\textbf{Italien}|pwk}reise zwischen 26. 3. 1901 und 18. 4. 1901}}}\label{K_L03062-1h}! Ich wollte, ich könnte mit.\pend
           \pstart
           Hier nichts Neues. Wenn ich nicht irre, hat Frau \textsc{Frida Strindberg\pwindex{Strindberg, Frida 04.04.1872 – 28.06.1943@\textsc{Strindberg, Frida} (04.04.1872 – 28.06.1943)|pw}}{ }hier\oindex{Berlin@\textbf{Berlin}|pwv} mit dem jungen \textsc{Hans Heinz Evers\pwindex{Ewers, Hanns Heinz 03.11.1871 – 12.06.1943@\textsc{Ewers, Hanns Heinz} (03.11.1871 – 12.06.1943), \emph{Schriftsteller}|pw}} ſchleunigſt ein Verhältniß angefangen.\pend
           \pstart
           Daß die \textsc{Triesch\pwindex{Triesch, Irene 13.04.1877 – 24.11.1964@\textsc{Triesch, Irene} (13.04.1877 – 24.11.1964), \emph{Schauspielerin}|pw}} im Sommer \label{K_L03062-2v}\edtext{mit uns kommen}{\lemma{\textnormal{\emph{mit uns kommen}}}\Cendnote{\textnormal{Zu einer gemeinsamen Reise mit Irene Triesch\pwindex{Triesch, Irene 13.04.1877 – 24.11.1964@\textsc{Triesch, Irene} (13.04.1877 – 24.11.1964), \emph{Schauspielerin}|pwk} kam es nicht. Schnitzler\pwindex{Schnitzler, Arthur 15.05.1862 – 21.10.1931@\textsc{Schnitzler, Arthur} (15.05.1862 – 21.10.1931), \emph{Schriftsteller, Mediziner}|pwk} und Goldmann\pwindex{Goldmann, Paul 31.01.1865 – 25.09.1935@\textsc{Goldmann, Paul} (31.01.1865 – 25.09.1935), \emph{Schriftsteller, Journalist}|pwk} begegneten sich im August in Welsberg\oindex{Welsberg-Taisten@\textbf{Welsberg-Taisten}|pwk}.}}}\label{K_L03062-2h} ſoll, iſt mir gar nicht recht.
               Sie hat einfach dekretirt, daß {\pb}ſie mitkommen wird,
               ohne viel zu fragen. Wenn Du willſt, daß ſie kommt, – meinetwegen! Einſtweilen kann
               man immerhin »Ja« ſagen. Im letzten Moment gibt es Ausreden genug.\pend
           \pstart
           Grüße an die Grünethorgaſſe\oindex{Gruenentorgasse@\textbf{Grünentorgasse}|pw}\pwindex{Steinrueck, Elisabeth 19.11.1885 – 07.04.1920@\textsc{Steinrück, Elisabeth} (19.11.1885 – 07.04.1920)|pwv}\pwindex{Schnitzler, Olga 17.01.1882 – 13.01.1970@\textsc{Schnitzler, Olga} (17.01.1882 – 13.01.1970), \emph{Schauspielerin, Sängerin}|pwv}! Ich ſchreibe nächſtens an dieſe Adreſſe. Habe einſtweilen wenig Zeit.\pend
           \pstart
           Darum auch für Dich nur dieſe eiligen Zeilen. Ich {\pb}drücke Dir herzlichſt die Hand. {\\[\baselineskip]}Dein {\\[\baselineskip]}\spacefill\mbox{Paul Goldmann}\pend
           \leftskip=0em{}\pstart
           \noindent{}\textsc{Dora Speyer\pwindex{Michaelis, Dora 23.05.1881 – 22.01.1946@\textsc{Michaelis, Dora} (23.05.1881 – 22.01.1946)|pw}} kennen gelernt. Iſt \label{K_L03062-3v}\edtext{noch
                  immer ſehr in Dich verliebt}{\lemma{\textnormal{\emph{noch … verliebt}}}\Cendnote{\textnormal{vgl. Schnitzler\pwindex{Schnitzler, Arthur 15.05.1862 – 21.10.1931@\textsc{Schnitzler, Arthur} (15.05.1862 – 21.10.1931), \emph{Schriftsteller, Mediziner}|pwk}s \emph{Tagebuch}\pwindex{\textcolor{red}{\textsuperscript{XXXX1 indx}}!Tagebuch1981 – 2000@\strich\emph{Tagebuch} {[}Hrsg., 1981 – 2000{]}|pwk} ab dem 28. 2. 1900}}}\label{K_L03062-3h}. Mein Herz \strikeout{zu} hat ſie zu gewinnen
                  verſucht, indem ſie von \textsc{Hoffmannsthal\pwindex{Hofmannsthal, Hugo von 1874-02-01 – 1929-07-15@\textsc{Hofmannsthal, Hugo von} (1874-02-01 – 1929-07-15), \emph{Schriftsteller}|pw}} und \textsc{Wassermann\pwindex{Wassermann, Jakob 10.03.1873 – 01.01.1934@\textsc{Wassermann, Jakob} (10.03.1873 – 01.01.1934), \emph{Schriftsteller}|pw}} ſchwärmte. Das iſt nicht ganz der richtige Weg.\pend
           
         
         \endnumbering\mylabel{h}\end{ledgroupsized}  \newcommand{\dateiname}{L03062}\newcommand{\titel}{Paul Goldmann an Arthur Schnitzler, 21. 3. [1901]}\newcommand{\editorInnen}{Martin Anton Müller und Laura Untner}%% latex-leseansicht-abspann.tex
%% Abspann für die Leseansicht.
%% Der Schalter \ifkorrekturansicht ist bereits durch den Vorspann gesetzt.

%% latex-abspann.tex
%% Gemeinsamer Abspann für Korrekturansicht und Leseansicht.
%% Setzt den Schalter \ifkorrekturansicht voraus (gesetzt in den
%% einbindenden Dateien latex-korrekturansicht-abspann.tex bzw.
%% latex-leseansicht-abspann.tex).
%% ---------------------------------------------------------------

\normalsize

% Das esempio-Environment wird nur in der Leseansicht benötigt
\ifkorrekturansicht\else
\newenvironment{esempio}[3]%
{
    \vspace{1.5ex}
    \rlap{\underline{#1}}
    \par
    \setlength{\parindent}{0cm}
    \nopagebreak
    \leftskip=#2cm
    \rightskip=#3cm
}
{
    \par
}
\fi

\doendnotes{C}
\bigskip
\vfill

\clearpage

\footnotesize

\ifkorrekturansicht
  \lohead{\textsc{register}}
\fi

% theindex-Environment neu definieren ohne reledmac
\makeatletter
\renewenvironment{theindex}{%
  \ifkorrekturansicht
    \section*{\indexname}%
  \else
    \subsubsection*{Index der erwähnten Entitäten}%
  \fi
  \setlength{\parindent}{0pt}%
  \setlength{\parskip}{0pt plus 0.3pt}%
  \let\item\@idxitem
}{%
  \ifkorrekturansicht\clearpage\fi
}
\makeatother

\IfFileExists{\jobname-pw.ind}{\input{\jobname-pw.ind}}{}

% Quellenangabe nur in der Leseansicht
\ifkorrekturansicht\else
% Fallback-Definitionen, falls die .tex-Datei \titel etc. nicht gesetzt hat
\providecommand{\titel}{}
\providecommand{\editorInnen}{}
\providecommand{\dateiname}{\jobname}

\vspace{3cm}

\vfill

\footnotesize
\textsc{Quelle}: \titel. Herausgegeben von {\editorInnen}. In: \emph{Arthur Schnitzler: Briefwechsel mit Autorinnen und Autoren}.
 Digitale Edition, https://schnitzler-briefe.acdh.oeaw.ac.at/{\dateiname}.html (Stand \today)
\fi

\end{document}


      