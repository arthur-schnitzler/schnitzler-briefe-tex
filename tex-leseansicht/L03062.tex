%% latex-korrekturansicht-vorspann.tex
%% Vorspann für die Korrekturansicht.
%% Lädt die gemeinsame Datei latex-vorspann.tex mit gesetztem Schalter.

\newif\ifkorrekturansicht
\korrekturansichttrue

\input{../tex-inputs/latex-vorspann}


\section[ Paul Goldmann an Arthur Schnitzler, 21. 3. {[}1901{]}]{L03062 Paul Goldmann an Arthur Schnitzler, 21. 3. {[}1901{]}}
\nopagebreak\mylabel{L03062v}
\rehead{ }\normalsize\beginnumbering\briefempfaengerindex{Schnitzler, Arthur@\textsc{Schnitzler, Arthur}!zzzGoldmann, Paul@\emph{von Paul Goldmann}!1901-03-211@{21. 3. {[}1901{]}}|(be}
\toendnotes[C]{\smallbreak\pagebreak[2]}\Standort{DLA, A:Schnitzler, HS.NZ85.1.3171.}
\physDesc{Brief, 1 Blatt, 3 Seiten, 926 Zeichen
\newline{}Handschrift: blaue Tinte, deutsche Kurrent
\newline{}Schnitzler: 1) mit Bleistift das Jahr »901« vermerkt  2) mit rotem Buntstift vier
                                 Unterstreichungen}\toendnotes[C]{\smallbreak}
\pstart
           \raggedleft{}{\pb}\textcolor{gray}{\textbf{DESSAUERSTRASSE 19}}\oindex{Dessauer Strasse@\textbf{Dessauer Straße}, \emph{Straße (K.STR)}|pw}\pend
           
\pstart
           Berlin\oindex{Berlin@\textbf{Berlin}, \emph{P.PPLC}|pw}, 21. März.\pend
           
\pstart\center{}Mein lieber Freund,\pend\vspace{0.5em}
\pstart
           Reiſe glücklich! Komm geſund wieder! Und grüße mir das \label{K_L03062-1v}\edtext{Land der Sehnſucht\oindex{Italien@\textbf{Italien}, \emph{A.PCLI}|pwv}}{\lemma{\textnormal{\emph{Land der Sehnſucht}}}\Cendnote{\textnormal{Bezug auf Schnitzlers{ }Italien\oindex{Italien@\textbf{Italien}, \emph{A.PCLI}|pwk}reise zwischen 26. 3. 1901 und 18. 4. 1901}}}\label{K_L03062-1}! Ich wollte, ich könnte mit.\pend
           
\pstart
           Hier nichts Neues. Wenn ich nicht irre, hat Frau \textsc{Frida Strindberg\pwindex{Strindberg, Frida 04.04.1872 – 28.06.1943@\textsc{Strindberg, Frida} (04.04.1872 – 28.06.1943)|pw}}{ }hier\oindex{Berlin@\textbf{Berlin}, \emph{P.PPLC}|pwv} mit dem jungen \textsc{Hans Heinz Evers\pwindex{Ewers, Hanns Heinz 03.11.1871 – 12.06.1943@\textsc{Ewers, Hanns Heinz} (03.11.1871 – 12.06.1943), \emph{Schriftsteller/Schriftstellerin}|pw}} ſchleunigſt ein Verhältniß angefangen.\pend
           
\pstart
           Daß die \textsc{Triesch\pwindex{Triesch, Irene 13.04.1877 – 24.11.1964@\textsc{Triesch, Irene} (13.04.1877 – 24.11.1964), \emph{Schauspieler/Schauspielerin}|pw}} im Sommer \label{K_L03062-2v}\edtext{mit uns kommen}{\lemma{\textnormal{\emph{mit uns kommen}}}\Cendnote{\textnormal{Zu einer gemeinsamen Reise mit Irene Triesch\pwindex{Triesch, Irene 13.04.1877 – 24.11.1964@\textsc{Triesch, Irene} (13.04.1877 – 24.11.1964), \emph{Schauspieler/Schauspielerin}|pwk} kam es nicht. Schnitzler und Goldmann\pwindex{Goldmann, Paul 31.01.1865 – 25.09.1935@\textsc{Goldmann, Paul} (31.01.1865 – 25.09.1935), \emph{Schriftsteller/Schriftstellerin, Journalist/Journalistin}|pwk} begegneten sich im August in Welsberg\oindex{Welsberg-Taisten@\textbf{Welsberg-Taisten}, \emph{A.ADM3}|pwk}.}}}\label{K_L03062-2} ſoll, iſt mir gar nicht recht.
               Sie hat einfach dekretirt, daß {\pb}ſie mitkommen wird,
               ohne viel zu fragen. Wenn Du willſt, daß ſie kommt, – meinetwegen! Einſtweilen kann
               man immerhin »Ja« ſagen. Im letzten Moment gibt es Ausreden genug.\pend
           
\pstart
           Grüße an die Grünethorgaſſe\oindex{Gruenentorgasse@\textbf{Grünentorgasse}, \emph{Straße (K.STR)}|pw}\pwindex{Steinrueck, Elisabeth 19.11.1885 – 07.04.1920@\textsc{Steinrück, Elisabeth} (19.11.1885 – 07.04.1920)|pwv}\pwindex{Schnitzler, Olga 17.01.1882 – 13.01.1970@\textsc{Schnitzler, Olga} (17.01.1882 – 13.01.1970), \emph{Schauspieler/Schauspielerin, Sänger/Sängerin}|pwv}! Ich ſchreibe nächſtens an dieſe Adreſſe. Habe einſtweilen wenig Zeit.\pend
           
\pstart
           Darum auch für Dich nur dieſe eiligen Zeilen. Ich {\pb}drücke Dir herzlichſt die Hand. {\\[\baselineskip]}Dein {\\[\baselineskip]}\spacefill\mbox{Paul Goldmann}\pend
           \leftskip=0em{}
\pstart
           \noindent{}\textsc{Dora Speyer\pwindex{Michaelis, Dora 23.05.1881 – 22.01.1946@\textsc{Michaelis, Dora} (23.05.1881 – 22.01.1946)|pw}} kennen gelernt. Iſt \label{K_L03062-3v}\edtext{noch
                  immer ſehr in Dich verliebt}{\lemma{\textnormal{\emph{noch … verliebt}}}\Cendnote{\textnormal{Vgl. Schnitzlers{ }\emph{Tagebuch}\pwindex{Tagebuch@\emph{Tagebuch}|pwk} ab dem 28. 2. 1900.
                  }}}\label{K_L03062-3}. Mein Herz \strikeout{zu} hat ſie zu gewinnen
                  verſucht, indem ſie von \textsc{Hoffmannsthal\pwindex{Hofmannsthal, Hugo von 1874-02-01 – 1929-07-15@\textsc{Hofmannsthal, Hugo von} (1874-02-01 – 1929-07-15), \emph{Schriftsteller/Schriftstellerin}|pw}} und \textsc{Wassermann\pwindex{Wassermann, Jakob 10.03.1873 – 01.01.1934@\textsc{Wassermann, Jakob} (10.03.1873 – 01.01.1934), \emph{Schriftsteller/Schriftstellerin}|pw}} ſchwärmte. Das iſt nicht ganz der richtige Weg.\pend
           \selectlanguage{ngerman}\endnumbering\briefempfaengerindex{Schnitzler, Arthur@\textsc{Schnitzler, Arthur}!zzzGoldmann, Paul@\emph{von Paul Goldmann}!1901-03-211@{21. 3. {[}1901{]}}|)be}\mylabel{L03062h}  \normalsize

\doendnotes{C}
\bigskip
\vfill

\clearpage

\footnotesize

\lohead{\textsc{register}}

% Definiere theindex-Environment komplett neu ohne reledmac
\makeatletter
\renewenvironment{theindex}{%
  \section*{\indexname}%
  \setlength{\parindent}{0pt}%
  \setlength{\parskip}{0pt plus 0.3pt}%
  \let\item\@idxitem
}{%
  \clearpage
}
\makeatother

\IfFileExists{\jobname-pw.ind}{\input{\jobname-pw.ind}}{}

\end{document}

      