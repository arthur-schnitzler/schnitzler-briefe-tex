%% latex-leseansicht-vorspann.tex
%% Vorspann für die Leseansicht.
%% Lädt die gemeinsame Datei latex-vorspann.tex mit nicht gesetztem Schalter.

\newif\ifkorrekturansicht
\korrekturansichtfalse

\input{../tex-inputs/latex-vorspann}


\section[ Paul Goldmann an Arthur Schnitzler, 21. 3. {[}1901{]}]{L03062 Paul Goldmann an Arthur Schnitzler,  21. 3. [1901]}
\nopagebreak\mylabel{L03062v}
\rehead{ }\normalsize\beginnumbering\briefempfaengerindex{Schnitzler, Arthur@\textsc{Schnitzler, Arthur}!zzzGoldmann, Paul@\emph{von Paul Goldmann}!1901-03-211@{21. 3. [1901]}|(be}
\toendnotes[C]{\smallbreak\pagebreak[2]}
\correspDesc{Versand  durch Paul Goldmann am 21. 3. [1901] in Berlin
\newline{}Erhalt  durch Arthur Schnitzler im Zeitraum [22. 3. 1901
                  – 25. 3. 1901?] in Wien}\toendnotes[C]{\smallbreak}
\Standort{DLA, A:Schnitzler, HS.NZ85.1.3171.}
\physDesc{Brief, 1 Blatt, 3 Seiten, 926 Zeichen
\newline{}Handschrift: blaue Tinte, deutsche Kurrent
\newline{}Schnitzler: 1) mit Bleistift das Jahr »901« vermerkt  2) mit rotem Buntstift vier
                                 Unterstreichungen}\toendnotes[C]{\smallbreak}
\pstart
           \raggedleft{}{\pb}\textcolor{gray}{\textbf{DESSAUERSTRASSE 19}}\oindex{Dessauer Straße@\textbf{Dessauer Straße}, \emph{Straße}|pw}\pend
           
\pstart
           Berlin\oindex{Berlin@\textbf{Berlin}, \emph{Hauptstadt}|pw}, 21. März.\pend
           
\pstart\center{}Mein lieber Freund,\pend\vspace{0.5em}
\pstart
           Reiſe glücklich! Komm geſund wieder! Und grüße mir das \label{K_L03062-1v}\edtext{Land der Sehnſucht\oindex{Italien@\textbf{Italien}|pwv}}{\lemma{\textnormal{\emph{Land der Sehnsucht}}}\Cendnote{\textnormal{Bezug auf Schnitzlers{ }Italien\oindex{Italien@\textbf{Italien}|pwk}reise zwischen 26. 3. 1901 und 18. 4. 1901}}}\label{K_L03062-1}! Ich wollte, ich könnte mit.\pend
           
\pstart
           Hier nichts Neues. Wenn ich nicht irre, hat Frau \textsc{Frida Strindberg\pwindex{Strindberg, Frida 4.\,4.\,1872 Mondsee – 28.\,6.\,1943 Salzburg@\textsc{Strindberg, Frida} (4.\,4.\,1872 Mondsee – 28.\,6.\,1943 Salzburg)|pw}}{ }hier\oindex{Berlin@\textbf{Berlin}, \emph{Hauptstadt}|pwv} mit dem jungen \textsc{Hans Heinz Evers\pwindex{Ewers, Hanns Heinz 3.\,11.\,1871 Düsseldorf – 12.\,6.\,1943 Berlin@\textsc{Ewers, Hanns Heinz} (3.\,11.\,1871 Düsseldorf – 12.\,6.\,1943 Berlin), \emph{Schriftsteller}|pw}}{ }ſchleunigſt ein Verhältniß angefangen.\pend
           
\pstart
           Daß die \textsc{Triesch\pwindex{Triesch, Irene 13.\,4.\,1877 Wien – 24.\,11.\,1964 Basel@\textsc{Triesch, Irene} (13.\,4.\,1877 Wien – 24.\,11.\,1964 Basel), \emph{Schauspielerin}|pw}} im Sommer \label{K_L03062-2v}\edtext{mit uns kommen}{\lemma{\textnormal{\emph{mit uns kommen}}}\Cendnote{\textnormal{Zu einer gemeinsamen Reise mit Irene Triesch\pwindex{Triesch, Irene 13.\,4.\,1877 Wien – 24.\,11.\,1964 Basel@\textsc{Triesch, Irene} (13.\,4.\,1877 Wien – 24.\,11.\,1964 Basel), \emph{Schauspielerin}|pwk} kam es nicht. Schnitzler und Goldmann\pwindex{Goldmann, Paul 31.\,1.\,1865 Breslau – 25.\,9.\,1935 Wien@\textsc{Goldmann, Paul} (31.\,1.\,1865 Breslau – 25.\,9.\,1935 Wien), \emph{Schriftsteller, Journalist}|pwk} begegneten sich im August in Welsberg\oindex{Welsberg-Taisten@\textbf{Welsberg-Taisten}, \emph{Verwaltungsgebiet}|pwk}.}}}\label{K_L03062-2}{ }ſoll, iſt mir gar nicht recht.
               Sie hat einfach dekretirt, daß {\pb}ſie mitkommen wird,
               ohne viel zu fragen. Wenn Du willſt, daß{ }ſie kommt, – meinetwegen! Einſtweilen kann
               man immerhin »Ja«{ }ſagen. Im letzten Moment gibt es Ausreden genug.\pend
           
\pstart
           Grüße an die Grünethorgaſſe\oindex{Wien@\textbf{Wien}!IX., Alsergrund@\textbf{IX., Alsergrund}!Grünentorgasse@\textbf{Grünentorgasse}, \emph{Straße}|pw}\pwindex{Steinrück, Elisabeth 19.\,11.\,1885 – 7.\,4.\,1920 Partenkirchen@\textsc{Steinrück, Elisabeth} (19.\,11.\,1885 – 7.\,4.\,1920 Partenkirchen)|pwv}\pwindex{Schnitzler, Olga 17.\,1.\,1882 Wien – 13.\,1.\,1970 Lugano@\textsc{Schnitzler, Olga} (17.\,1.\,1882 Wien – 13.\,1.\,1970 Lugano), \emph{Schauspielerin, Sängerin}|pwv}! Ich{ }ſchreibe nächſtens an dieſe Adreſſe. Habe einſtweilen wenig Zeit.\pend
           
\pstart
           Darum auch für Dich nur dieſe eiligen Zeilen. Ich {\pb}drücke Dir herzlichſt die Hand. {\\[\baselineskip]}Dein {\\[\baselineskip]}\spacefill\mbox{Paul Goldmann}\pend
           \leftskip=0em{}
\pstart
           \noindent{}\textsc{Dora Speyer\pwindex{Michaelis, Dora 23.\,5.\,1881 Wien – 22.\,1.\,1946 New York City@\textsc{Michaelis, Dora} (23.\,5.\,1881 Wien – 22.\,1.\,1946 New York City)|pw}} kennen gelernt. Iſt \label{K_L03062-3v}\edtext{noch
                  immer{ }ſehr in Dich verliebt}{\lemma{\textnormal{\emph{noch … verliebt}}}\Cendnote{\textnormal{Vgl. Schnitzlers{ }\emph{Tagebuch}\pwindex{Schnitzler, Arthur 15.\,5.\,1862 Wien – 21.\,10.\,1931 ebd.@\textsc{Schnitzler, Arthur} (15.\,5.\,1862 Wien – 21.\,10.\,1931 ebd.), \emph{Schriftsteller, Mediziner}!Tagebuch@\strich\emph{Tagebuch}|pwk} ab dem 28. 2. 1900.
                  }}}\label{K_L03062-3}. Mein Herz \strikeout{zu} hat{ }ſie zu gewinnen
                  verſucht, indem{ }ſie von \textsc{Hoffmannsthal\pwindex{Hofmannsthal, Hugo von 1.\,2.\,1874 Wien – 15.\,7.\,1929 Rodaun@\textsc{Hofmannsthal, Hugo von} (1.\,2.\,1874 Wien – 15.\,7.\,1929 Rodaun), \emph{Schriftsteller}|pw}} und \textsc{Wassermann\pwindex{Wassermann, Jakob 10.\,3.\,1873 Fürth – 1.\,1.\,1934 Altaussee@\textsc{Wassermann, Jakob} (10.\,3.\,1873 Fürth – 1.\,1.\,1934 Altaussee), \emph{Schriftsteller}|pw}}{ }ſchwärmte. Das iſt nicht ganz der richtige Weg.\pend
           \selectlanguage{ngerman}\endnumbering\briefempfaengerindex{Schnitzler, Arthur@\textsc{Schnitzler, Arthur}!zzzGoldmann, Paul@\emph{von Paul Goldmann}!1901-03-211@{21. 3. [1901]}|)be}\mylabel{L03062h}  \newcommand{\dateiname}{L03062}\newcommand{\titel}{Paul Goldmann an Arthur Schnitzler, 21. 3. [1901]}\newcommand{\editorInnen}{Martin Anton Müller und Laura Untner}%% latex-leseansicht-abspann.tex
%% Abspann für die Leseansicht.
%% Der Schalter \ifkorrekturansicht ist bereits durch den Vorspann gesetzt.

%% latex-abspann.tex
%% Gemeinsamer Abspann für Korrekturansicht und Leseansicht.
%% Setzt den Schalter \ifkorrekturansicht voraus (gesetzt in den
%% einbindenden Dateien latex-korrekturansicht-abspann.tex bzw.
%% latex-leseansicht-abspann.tex).
%% ---------------------------------------------------------------

\normalsize

% Das esempio-Environment wird nur in der Leseansicht benötigt
\ifkorrekturansicht\else
\newenvironment{esempio}[3]%
{
    \vspace{1.5ex}
    \rlap{\underline{#1}}
    \par
    \setlength{\parindent}{0cm}
    \nopagebreak
    \leftskip=#2cm
    \rightskip=#3cm
}
{
    \par
}
\fi

\doendnotes{C}
\bigskip
\vfill

\clearpage

\footnotesize

\ifkorrekturansicht
  \lohead{\textsc{register}}
\fi

% theindex-Environment neu definieren ohne reledmac
\makeatletter
\renewenvironment{theindex}{%
  \ifkorrekturansicht
    \section*{\indexname}%
  \else
    \subsubsection*{Index der erwähnten Entitäten}%
  \fi
  \setlength{\parindent}{0pt}%
  \setlength{\parskip}{0pt plus 0.3pt}%
  \let\item\@idxitem
}{%
  \ifkorrekturansicht\clearpage\fi
}
\makeatother

\IfFileExists{\jobname-pw.ind}{\input{\jobname-pw.ind}}{}

% Quellenangabe nur in der Leseansicht
\ifkorrekturansicht\else
% Fallback-Definitionen, falls die .tex-Datei \titel etc. nicht gesetzt hat
\providecommand{\titel}{}
\providecommand{\editorInnen}{}
\providecommand{\dateiname}{\jobname}

\vspace{3cm}

\vfill

\footnotesize
\textsc{Quelle}: \titel. Herausgegeben von {\editorInnen}. In: \emph{Arthur Schnitzler: Briefwechsel mit Autorinnen und Autoren}.
 Digitale Edition, https://schnitzler-briefe.acdh.oeaw.ac.at/{\dateiname}.html (Stand \today)
\fi

\end{document}


