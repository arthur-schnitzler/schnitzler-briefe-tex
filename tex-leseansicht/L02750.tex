%% latex-leseansicht-vorspann.tex
%% Vorspann für die Leseansicht.
%% Lädt die gemeinsame Datei latex-vorspann.tex mit nicht gesetztem Schalter.

\newif\ifkorrekturansicht
\korrekturansichtfalse

\input{../tex-inputs/latex-vorspann}


         
         \renewcommand{\erwaehntePersonen}{Personen: [MMe. Georges] Aubry, Georges Aubry, Richard Beer-Hofmann, Max Eugen Burckhard, Camilla Gerzhofer, Paul Goldmann, Wilhelm Gollmann, Mary Hargrave, William Heinemann, Theodor Herzl, Anna Kallina, Victor Kutschera, Mathilde Mann, Friedrich Mitterwurzer, Peter Nansen, M. O. Riese, Adele Sandrock, Gustav Slanar, Leopold Sonnemann, Adolf von Sonnenthal, Fanny Walbeck, Carl von Zeska}
         \renewcommand{\erwaehnteInstitutionen}{Institutionen: Frankfurter Zeitung, Institut Rudy, Neue Freie Presse, S. Fischer Verlag, Éditions Perrin}
         \renewcommand{\erwaehnteOrte}{Orte: Berlin, Boulevard de Capucines, Burgtheater, Paris, Rue Royale, Rue de Caumartin, Volksgarten, Wien, rue Feydeau}
         \renewcommand{\erwaehnteWerke}{Werke: Die kleine Komödie, Julies Tagebuch. Roman, La Liberté, La petite comédie. Mœurs viennois, Liebelei. Schauspiel in drei Akten, Maria. Ein Buch der Liebe, Maria. En Bog om Kjærlighed. Roman, Mourir. Roman, Sterben. Novelle, Thêatres. [Notre correspondant de Vienne]}
               \section[Paul Goldmann an Arthur Schnitzler, 7. 10. {[}1895{]}]{ Paul Goldmann an Arthur Schnitzler, 7. 10. {[}1895{]}}\nopagebreak\mylabel{v}\rehead{ }\begin{ledgroupsized}[t]{13cm}\normalsize\beginnumbering\briefempfaengerindex{Schnitzler, Arthur@\textsc{Schnitzler, Arthur}!zzzGoldmann, Paul@\emph{von Paul Goldmann}!1895-10-071@{7. 10. {[}1895{]}}|(be} \toendnotes[C]{\smallbreak\pagebreak[2]} \Standort{DLA, A:Schnitzler, HS.NZ85.1.3165.}
\physDesc{Brief, 5 Blätter, 18 Seiten, 8124 Zeichen
\newline{}Handschrift: blaue Tinte, deutsche Kurrent
\newline{}Beilage: handschriftlicher Brief, 1 Blatt, 1 Seite, deutsche Kurrent; im
                                 Deutschen Literaturarchiv Marbach unter der Signatur
                                 HS.NZ85.1.3166/9 eingeordnet und damit den Korrespondenzstücken des
                                 Jahres 1896 zugeordnet. Bleistiftvermerk
                                 von Schnitzler: »\textsc{Inst. Rud\textcolor{gray}{y}\orgindex{Institut Rudy@Institut Rudy|pw}}« 
\newline{}Schnitzler: mit rotem Buntstift vier Unterstreichungen und das Jahr »95« vermerkt }\toendnotes[C]{\smallbreak}\pstart
           \noindent{}{\pb}\textcolor{gray}{\textbf{\textbf{Frankfurter Zeitung\orgindex{Frankfurter Zeitung@Frankfurter Zeitung|pw}}}}\pend
           \pstart
           \textcolor{gray}{\textbf{(\begin{otherlanguage}{french}Gazette de Francfort\end{otherlanguage}\orgindex{Frankfurter Zeitung@Frankfurter Zeitung|pw}). }}\pend
           \pstart
           \textcolor{gray}{\textbf{\textbf{\begin{otherlanguage}{french}Fondateur M. L.
                              Sonnemann\pwindex{Sonnemann, Leopold 1831-10-29 – 1909-10-30@\textsc{Sonnemann, Leopold} (1831-10-29 – 1909-10-30), \emph{Journalist, Herausgeber}|pw}\end{otherlanguage}.}}}\pend
           \pstart
           \begin{otherlanguage}{french}\textcolor{gray}{\textbf{Journal politique, financier,}}\end{otherlanguage}\hfill \textsc{Paris\oindex{Paris@\textbf{Paris}|pw}}, 7. Oktober.\pend
           \pstart
           \begin{otherlanguage}{french}\textcolor{gray}{\textbf{commercial et littéraire.}}\end{otherlanguage}\pend
           \pstart
           \begin{otherlanguage}{french}\textcolor{gray}{\textbf{\textbf{Paraissant trois fois par jour.}}}\end{otherlanguage}\pend
           \pstart
           \begin{otherlanguage}{french}\textcolor{gray}{\textbf{\textbf{Bureau à Paris\oindex{Paris@\textbf{Paris}|pw}}}}\end{otherlanguage}\pend
           \pstart
           \begin{otherlanguage}{french}\textcolor{gray}{\textbf{\textbf{24. Rue Feydeau\oindex{rue Feydeau@\textbf{rue Feydeau}|pw}.}}}\end{otherlanguage}\pend
           \pstart\center{}Mein lieber Freund,\pend\pstart
           dieſer Brief trifft Dich alſo am Vorabend{ }\label{K_L02750-1v}\edtext{großer Ereigniſſe}{\lemma{\textnormal{\emph{großer Ereigniſſe}}}\Cendnote{\textnormal{Uraufführung von \emph{Liebelei}\pwindex{Schnitzler, Arthur 15.05.1862 – 21.10.1931@\textsc{Schnitzler, Arthur} (15.05.1862 – 21.10.1931), \emph{Schriftsteller, Mediziner}!Liebelei. Schauspiel in drei Akten1895-10-09@\strich\emph{Liebelei. Schauspiel in drei Akten} {[}1895-10-09{]}|pwk} am 9. 10. 1895 im Burgtheater\oindex{Burgtheater@\textbf{Burgtheater}|pwk}}}}\label{K_L02750-1h}\textcolor{gray}{,} oder hoffentlich ſchon am Ereignißtage ſelbſt. Du kannſt Dir denken, mit wie wachſendem Intereſſe
               ich Deine letzten lieben Briefe geleſen. Gern hätte ich ſie raſch beantwortet; aber
               bei mir iſt wieder der Trübſinn eingekehrt; und ich wollte nicht, daß mir allzuviel
               davon in die Feder flöſſe. Ich danke Dir \strikeout{\textcolor{gray}{v}} von Herzen, daß Du mir ſo treulich berichtet haſt. Gern \strikeout{hatte} hätte ich all’ dieſe Zeit {\pb}mit Dir \substVorne{}\textsuperscript{\textcolor{gray}{e}}\substDazwischen{}v\substHinten{}erlebt; aber durch Deine Briefe habe ich doch wenigſtens einen Wellenſchlag
               davon zu ſpüren bekommen. Am \strikeout{\textcolor{gray}{S}\textcolor{gray}{×}\-\textcolor{gray}{×}\-\textcolor{gray}{×}\-\textcolor{gray}{×}\-\textcolor{gray}{×}\textcolor{gray}{ſten}} Schmerzlichſten iſt es mir, daß ich Mittwoch
               nicht da ſein kann. Erſtens, um raſcher zu wiſſen, wie es ausgegangen, und zweitens,
               um \strikeout{D\textcolor{gray}{ir}} mit Dir ein wenig die Zeit bis zum Abend zu verplaudern. Freilich
               hätteſt Du meiner wohl kaum bedurft. Mit großer Freude ſehe ich aus Deinen Briefen,
               wie ruhig Du biſt. Und wenn doch am Mittwoch{ }Nachmittag das Herzklopfen kommen ſollte – in Jener Stunde beſonders, wo
               der Abend über den {\pb}\strikeout{\textcolor{gray}{×}\-\textcolor{gray}{×}\-\textcolor{gray}{×}\-\textcolor{gray}{×}\-\textcolor{gray}{×}\-\textcolor{gray}{×}\-\textcolor{gray}{×}}{ }Volksgarten\oindex{Volksgarten@\textbf{Volksgarten}|pw} niederſinkt, eigens für Dich
               niederſinkt – ſo wirſt Du ſchon eine liebe Hand in Deiner Nähe haben, die bereit iſt,
               die Deinige zu drücken. Ich ſelbſt bin Deiner Sache ſicher. \strikeout{F\textcolor{gray}{a}} Für mich kann es ſich nur um die Größe des Erfolges handeln; ein Mißerfolg iſt
               ausgeſchloſſen, \strikeout{da} aus dem einfachen Grunde, weil
               nicht das ganze Wien\oindex{Wien@\textbf{Wien}|pw}er Publicum plötzlich
               irrſinnig werden kann. Oh, ich glaube, es wird ſchön ſein. Vielleicht nicht allzu
               ſtürmiſch, aber ſchön. Und wenn ich denke, {\pb}daß Du
               dahin gekommen, ſtill und ehrlich, Dir \strikeout{ſ} ſelbſt
               getreu, und einfach Deines lieben Herzens Sprache redend, – ſo fühle ich, daß es ein
               hoher Ehrentag iſt für Dich, für den Poeten ſo ſehr
               wie für den Menſchen, und ein ſtarkes Beiſpiel für uns Alle. Ich habe das Bedürfniß,
               Jeden dieſer Briefe mit Wünſchen zu füllen. Leider kann ich ja bei der ganzen
               Angelegenheit nichts thun, als Dir fortwährend »Glück!« und »Glück!« zurufen. Aber
               hier will ich es wenigſtens an den Meinigen nicht {\pb}fehlen laſſen. So kommt denn noch ein letzter herzinniger Wunſch, daß es gut werden
               möge. Damit umarme ich Dich und laſſe Dich Deinen Weg gehen........\pend
           \pstart
           Den Mittwoch{ }Abend werde ich mit meinen Gedanken in Wien\oindex{Wien@\textbf{Wien}|pw} ſein und werde verſuchen, die Zeit bis zum nächſten
                  Vormittag nicht lang zu finden. Denn, nicht wahr, Du \label{K_L02750-2v}\edtext{telegraphirſt}{\lemma{\textnormal{\emph{telegraphirſt}}}\Cendnote{\textnormal{Schnitzler\pwindex{Schnitzler, Arthur 15.05.1862 – 21.10.1931@\textsc{Schnitzler, Arthur} (15.05.1862 – 21.10.1931), \emph{Schriftsteller, Mediziner}|pwk} schickte tatsächlich ein
                  Telegramm, Goldmanns\pwindex{Goldmann, Paul 31.01.1865 – 25.09.1935@\textsc{Goldmann, Paul} (31.01.1865 – 25.09.1935), \emph{Schriftsteller, Journalist}|pwk} Telegramm vom [10.? 10. 1895] reagiert
                  darauf.}}}\label{K_L02750-2h} mir ein paar Worte? Und dann ſchickſt Du mir auch wohl die
               Referate, ich ſende {\pb}ſie Dir umgehend zurück. Sehr
               lieb wäre es, wenn auch \textsc{Richard\pwindex{Beer-Hofmann, Richard 1866-07-11 – 1945-09-26@\textsc{Beer-Hofmann, Richard} (1866-07-11 – 1945-09-26), \emph{Schriftsteller}|pw}} mir telegraphiren wollte; der könnte ſchon etwas ausführlicher berichten.\pend
           \pstart
           Dabei fällt mir ein, daß es am Ende vielleicht doch gut iſt, wenn ich nicht dabei
               bin. Ich hätte mich ausgenommen, wie die unverheirathete ältere Schweſter auf der
               Hochzeit der Jüngeren........\pend
           \pstart
           Dein letzter Brief war beſonders ſchön. So voll guter Stimmung, ſo zu Herzen gehend!
               Deinem Stück\pwindex{Schnitzler, Arthur 15.05.1862 – 21.10.1931@\textsc{Schnitzler, Arthur} (15.05.1862 – 21.10.1931), \emph{Schriftsteller, Mediziner}!Liebelei. Schauspiel in drei Akten1895-10-09@\strich\emph{Liebelei. Schauspiel in drei Akten} {[}1895-10-09{]}|pwv} thuſt Du aber
               doch wohl Unrecht. Gar ſo {\pb}\strikeout{dün}{ }\strikeout{dü\textcolor{gray}{m}} dünn iſt es, weiß Gott, nicht. Du ſelbſt weißt, was Du hätteſt \strikeout{dazu} noch dazuthun können, der Zuſchauer aber nicht,
               und dieſem erſcheint es voll genug. Eines iſt \strikeout{r\textcolor{gray}{e}} richtig, daß die Figur des Alten\pwindex{Schnitzler, Arthur 15.05.1862 – 21.10.1931@\textsc{Schnitzler, Arthur} (15.05.1862 – 21.10.1931), \emph{Schriftsteller, Mediziner}!Liebelei. Schauspiel in drei Akten1895-10-09@\strich\emph{Liebelei. Schauspiel in drei Akten} {[}1895-10-09{]}|pwv} hätte erweitert und vertieft werden können. Man hätte gern mit ihm
               nähere Bekanntſchaft gemacht. Aber den gibſt Du uns vielleicht in einem neuen Stücke.
               Und wer könnte auch den Reichthum des Lebens auf der Bühne verlangen, wie Du ſagſt?
                  \strikeout{\textcolor{gray}{×}} Das {\pb}Dramatiſche iſt ja gerade eine Auswahl
               aus der Fülle. Nur das Weſentliche gehört \strikeout{a} auf die
               Bühne; und Du weißt ſelbſt am Beſten, daß die dramatiſche Kunſt in der Aus\strikeout{\textcolor{gray}{×}} Ausſcheidung, Beſchränkung, Vereinfachung liegt. Für des Lebens Reichthum und
               Fülle \strikeout{hat das \textcolor{gray}{×}} iſt das Theater zu klein{\dotssix}\pend
           \pstart
           Es iſt ſchön, daß es mit den Proben ſo gut gegangen und daß die Leute ſo
               liebenswürdig zu Dir waren. \strikeout{Nach Allem} Nach den Namen
               der Schauſpieler\pwindex{Sonnenthal, Adolf von 1834-12-21 – 1909-04-04@\textsc{Sonnenthal, Adolf von} (1834-12-21 – 1909-04-04), \emph{Schauspieler}|pwv}\pwindex{Sandrock, Adele 1863-08-19 – 1937-08-30@\textsc{Sandrock, Adele} (1863-08-19 – 1937-08-30), \emph{Schauspielerin}|pwv}\pwindex{Kallina, Anna 31.03.1874 – 04.01.1948@\textsc{Kallina, Anna} (31.03.1874 – 04.01.1948), \emph{Schauspielerin}|pwv}\pwindex{Walbeck, Fanny 1852-10-11 – 1919-08-15@\textsc{Walbeck, Fanny} (1852-10-11 – 1919-08-15), \emph{Schauspielerin}|pwv}\pwindex{Gerzhofer, Camilla 1888-02-02 – 1961-03-08@\textsc{Gerzhofer, Camilla} (1888-02-02 – 1961-03-08), \emph{Schauspielerin}|pwv}\pwindex{Kutschera, Victor 1863-05-02 – 1933-01-20@\textsc{Kutschera, Victor} (1863-05-02 – 1933-01-20), \emph{Regisseur, Schauspieler}|pwv}\pwindex{Zeska, Carl von 1862-10-31 – 1938-07-18@\textsc{Zeska, Carl von} (1862-10-31 – 1938-07-18), \emph{Schauspieler}|pwv}\pwindex{Mitterwurzer, Friedrich 16.10.1844 – 13.02.1897@\textsc{Mitterwurzer, Friedrich} (16.10.1844 – 13.02.1897), \emph{Schauspieler}|pwv}\pwindex{Slanar, Gustav 1861-06-25 – 1944-09-22@\textsc{Slanar, Gustav} (1861-06-25 – 1944-09-22), \emph{Schauspieler, Inspizient}|pwv}{ }\strikeout{\textcolor{gray}{n}} und nach {\pb}dem, was Du ſchreibſt, zu
               ſchließen, wird die Aufführung\pwindex{Schnitzler, Arthur 15.05.1862 – 21.10.1931@\textsc{Schnitzler, Arthur} (15.05.1862 – 21.10.1931), \emph{Schriftsteller, Mediziner}!Liebelei. Schauspiel in drei Akten1895-10-09@\strich\emph{Liebelei. Schauspiel in drei Akten} {[}1895-10-09{]}|pwv}
               eine vorzügliche ſein. Es iſt doch auch gut, wenn ein Director vor einem Stücke Angſt
               hat. So iſt er gezwungen, es zum Erfolg zu führen\strikeout{\textcolor{gray}{,}} und die beſten Kräſte ſeines Theaters dafür einzuſetzen. \textsc{Burckhardts\pwindex{Burckhard, Max Eugen 14.07.1854 – 16.03.1912@\textsc{Burckhard, Max Eugen} (14.07.1854 – 16.03.1912), \emph{Schriftsteller, Wissenschaftler, Theaterleiter}|pw}}{ }\strikeout{Z\textcolor{gray}{o}\textcolor{gray}{×}}{ }Haſenſüßerei, unter der Du ſoviel gelitten, kommt
               Dir hier doch am Ende zugute. So \substVorne{}\textsuperscript{läuft}\substDazwischen{}ſtellt\substHinten{} doch Alles am Ende wieder \strikeout{auf} Alles in den
               Dienſt {\pb}des Guten, ſelbſt das anfangs Hindernde. Die
               große Tragödin\pwindex{Sandrock, Adele 1863-08-19 – 1937-08-30@\textsc{Sandrock, Adele} (1863-08-19 – 1937-08-30), \emph{Schauspielerin}|pwv} zum Beiſpiel!
               Dieſe verſtehe ich beſonders gut in der Sache. Sie hat geſehen, daß die Rolle\pwindex{Schnitzler, Arthur 15.05.1862 – 21.10.1931@\textsc{Schnitzler, Arthur} (15.05.1862 – 21.10.1931), \emph{Schriftsteller, Mediziner}!Liebelei. Schauspiel in drei Akten1895-10-09@\strich\emph{Liebelei. Schauspiel in drei Akten} {[}1895-10-09{]}|pwv} vorzüglich iſt und daß
               ſie Erfolg haben wird. Das iſt doch \strikeout{\textcolor{gray}{w}} noch ein höherer Genuß, als der, \strikeout{I\textcolor{gray}{n}f} einem ehemaligen Geliebten Infamien anzuthun. So
               wird ſie \strikeout{ſüſ} ſüß und zahm. Das läuft auf das heraus,
               was ich immer ſage: Man gebe ſich mit der Komödianten-Gemeinheit {\pb}nicht ab und ſchaffe ruhig weiter. Das unfehlbar
               beſte Mittel gegen \strikeout{Bühnen-} Theater-Intriguen iſt ein
               gutes Stück. Jawohl, mein Freund, der Sieg des Guten und Schönen. Es iſt gar nicht ſo
               gymnaſiaſtenhaft, daran zu glauben, wie Du ſchreibſt. Ich glaube immer mehr daran.
               Die Gemeinheit und alles Schlechte iſt ſehr ſtark hinieden; aber es gibt doch kaum
               etwas, das ſtärker iſt, als dieſe zwei Herkulaſſe: {\pb}Gut und Schön. Auch ahnſt Du gar nicht, wieviel gerade im Falle \textsc{Arthur Schnitzler} liegt, das Einen wieder mit dem Weltlauf
               auszuſöhnen vermag{\dotssix}\pend
           \pstart
           Reden wir ein wenig von Geſchäften. Anbei findeſt Du einen Brief, den ich nicht
               beantworten wollte, ohne Dich zu fragen. Ich rathe Dir ab, vorläufig das
               Überſetzungsrecht der »Liebelei\pwindex{Schnitzler, Arthur 15.05.1862 – 21.10.1931@\textsc{Schnitzler, Arthur} (15.05.1862 – 21.10.1931), \emph{Schriftsteller, Mediziner}!Liebelei. Schauspiel in drei Akten1895-10-09@\strich\emph{Liebelei. Schauspiel in drei Akten} {[}1895-10-09{]}|pw}« zu vergeben.
               Warten wir erſt ab, wie die Dinge gehen. \textsc{Madame Aubry\pwindex{Aubry, [MMe. Georges] @\textsc{Aubry, [MMe. Georges]}, \emph{Übersetzerin}|pw}} iſt mit der Überſetzung\pwindex{Schnitzler, Arthur 15.05.1862 – 21.10.1931@\textsc{Schnitzler, Arthur} (15.05.1862 – 21.10.1931), \emph{Schriftsteller, Mediziner}!petite comedie. Mœurs viennois1895-11-19 – 1895-11-28@\strich\emph{La petite comédie. Mœurs viennois} {[}1895-11-19 – 1895-11-28{]}|pwv}
               der {\pb}»Kleinen
                  Komödie\pwindex{Schnitzler, Arthur 15.05.1862 – 21.10.1931@\textsc{Schnitzler, Arthur} (15.05.1862 – 21.10.1931), \emph{Schriftsteller, Mediziner}!kleine Komoedie1895-08-01@\strich\emph{Die kleine Komödie} {[}1895-08-01{]}|pw}« fertig. Ertheile ihr die Autoriſation in einem \uline{deutſchen} Briefe, den Du mir ſchicken magſt. \textsc{Aubry\pwindex{Aubry, Georges †~1923@\textsc{Aubry, Georges} (†~1923), \emph{Redakteur}|pw}} hat mir verſprochen, einen kleinen \label{K_L02750-3v}\edtext{Bericht\pwindex{Aubry, Georges †~1923@\textsc{Aubry, Georges} (†~1923), \emph{Redakteur}!Thêatres. [Notre correspondant de Vienne]1895-10-12@\strich\emph{Thêatres. [Notre correspondant de Vienne]} {[}1895-10-12{]}|pwv}}{\lemma{\textnormal{\emph{Bericht}}}\Cendnote{\textnormal{[Georges Aubry\pwindex{Aubry, Georges †~1923@\textsc{Aubry, Georges} (†~1923), \emph{Redakteur}|pwk}]: \emph{Théâtres. [Notre correspondant de Vienne]}\pwindex{Aubry, Georges †~1923@\textsc{Aubry, Georges} (†~1923), \emph{Redakteur}!Thêatres. [Notre correspondant de Vienne]1895-10-12@\strich\emph{Thêatres. [Notre correspondant de Vienne]} {[}1895-10-12{]}|pwk}. In: \emph{La Liberté}\pwindex{?? Werk@Nicht ermittelte Verfasserinnen und Verfasser!Liberte1865-07-16 – 1940-06-11@\emph{La Liberté} {[}1865-07-16 – 1940-06-11{]}|pwk}, Jg. 30, Nr. 11.289, 12. 10. 1895, S. 3. Siehe dazu auch Paul Goldmann an Arthur Schnitzler, 13. 10. [1895].}}}\label{K_L02750-3h} über die Aufführung der »Liebelei\pwindex{Schnitzler, Arthur 15.05.1862 – 21.10.1931@\textsc{Schnitzler, Arthur} (15.05.1862 – 21.10.1931), \emph{Schriftsteller, Mediziner}!Liebelei. Schauspiel in drei Akten1895-10-09@\strich\emph{Liebelei. Schauspiel in drei Akten} {[}1895-10-09{]}|pw}« in die »\textsc{Liberté\pwindex{?? Werk@Nicht ermittelte Verfasserinnen und Verfasser!Liberte1865-07-16 – 1940-06-11@\emph{La Liberté} {[}1865-07-16 – 1940-06-11{]}|pw}}« zu bringen. Schon zu dieſem Zweck brauche ich das oben erbetene Telegramm. Dem
                  \textsc{Herzl\pwindex{Herzl, Theodor 1860-05-02 – 1904-07-03@\textsc{Herzl, Theodor} (1860-05-02 – 1904-07-03), \emph{Schriftsteller, Journalist}|pw}} ſollteſt Du \uline{doch} ein Feuilleton geben. Glaub’
               mir, Du kannſt es ſchreiben, es iſt Dir nur unbequem. {\pb}Du haſt doch auch ſchon kürzere Sachen gemacht, zum
               Teufel! Denk’ Dir halt, daß Du es \uline{nicht} für die »Neue Freie Preſſe\orgindex{Neue Freie Presse@Neue Freie Presse|pw}« ſchreibſt. Aber ich halte es
               für ſehr wichtig, daß Dein Name auch dort erſcheint. Daß »Sterben\pwindex{Schnitzler, Arthur 15.05.1862 – 21.10.1931@\textsc{Schnitzler, Arthur} (15.05.1862 – 21.10.1931), \emph{Schriftsteller, Mediziner}!Mourir. Roman1895-04-27 – 1895-06-01@\strich\emph{Mourir. Roman} {[}1895-04-27 – 1895-06-01{]}|pwv}« bei \textsc{Perrin\orgindex{Editions Perrin@Éditions Perrin|pw}} erſcheint, iſt vortrefflich. Es iſt ein anſtändiger Verlag\orgindex{Editions Perrin@Éditions Perrin|pwv}, der ſreilich wenig Verbindungen mit
               Zeitungen hat. Denn hier ſchreibt das Geſindel nur über {\pb}Bücher, wenn der Verleger dem Blatt ein Pauſchale
               zahlt. Aber laß’ gut ſein, ich \strikeout{ſchaff} ſchaff’ Dir
               ſchon eine oder die andere Beſprechung{\dotssix}\pend
           \pstart
           Was Du über »Juliens Tagebuch\pwindex{Nansen, Peter 20.01.1861 – 31.07.1918@\textsc{Nansen, Peter} (20.01.1861 – 31.07.1918), \emph{Schriftsteller, Journalist, Verleger}!Julies Tagebuch. Roman1895-01-01@\strich\emph{Julies Tagebuch. Roman} {[}1895-01-01{]}|pw}« ſchreibſt,
               überzeugt mich nicht. Inzwiſchen habe ich auch »\label{K_L02750-4v}\edtext{Maria\pwindex{Nansen, Peter 20.01.1861 – 31.07.1918@\textsc{Nansen, Peter} (20.01.1861 – 31.07.1918), \emph{Schriftsteller, Journalist, Verleger}!Maria. Ein Buch der Liebe1895@\strich\emph{Maria. Ein Buch der Liebe} {[}1895{]}|pw}}{\lemma{\textnormal{\emph{Maria}}}\Cendnote{\textnormal{Peter Nansen\pwindex{Nansen, Peter 20.01.1861 – 31.07.1918@\textsc{Nansen, Peter} (20.01.1861 – 31.07.1918), \emph{Schriftsteller, Journalist, Verleger}|pwk}: \emph{Maria. Ein Buch der Liebe}\pwindex{Nansen, Peter 20.01.1861 – 31.07.1918@\textsc{Nansen, Peter} (20.01.1861 – 31.07.1918), \emph{Schriftsteller, Journalist, Verleger}!Maria. Ein Buch der Liebe1895@\strich\emph{Maria. Ein Buch der Liebe} {[}1895{]}|pwk}. Autorisierte Übersetzung aus
                     dem Dänischen von Mathilde Mann\pwindex{Mann, Mathilde 1859-11-24 – 1925-11-14@\textsc{Mann, Mathilde} (1859-11-24 – 1925-11-14), \emph{Übersetzerin}|pwk}. Berlin\oindex{Berlin@\textbf{Berlin}|pwk}: \emph{S.
                        Fischer}\orgindex{S. Fischer Verlag@S. Fischer Verlag|pwk}{ }1895. (Originalausgabe: \emph{Maria. En Bog om
                        Kjærlighed. Roman}\pwindex{Nansen, Peter 20.01.1861 – 31.07.1918@\textsc{Nansen, Peter} (20.01.1861 – 31.07.1918), \emph{Schriftsteller, Journalist, Verleger}!Maria. En Bog om Kjærlighed. Roman1894@\strich\emph{Maria. En Bog om Kjærlighed. Roman} {[}1894{]}|pwk}, 1894.)}}}\label{K_L02750-4h}« geleſen. Das geſällt mir viel beſſer. Ich weiß nicht, ob es \strikeout{w\textcolor{gray}{a}} ein wahres Buch iſt; von dieſen Liebes-Dingen verſtehe ich wenig; aber es iſt
               poetiſch und ſtellenweiſe entzückend poetiſch. {\pb}In
                  »Juliens Tagebuch\pwindex{Nansen, Peter 20.01.1861 – 31.07.1918@\textsc{Nansen, Peter} (20.01.1861 – 31.07.1918), \emph{Schriftsteller, Journalist, Verleger}!Julies Tagebuch. Roman1895-01-01@\strich\emph{Julies Tagebuch. Roman} {[}1895-01-01{]}|pw}« mag ich vor Allem den Mann
               nicht, dieſen Schwerenöther, dem alle Weiber zufliegen, der ſeine Syſteme mit ihnen
               hat, der \strikeout{J\textcolor{gray}{e}} auch in dem heißen Sturm mit Julie\pwindex{Nansen, Peter 20.01.1861 – 31.07.1918@\textsc{Nansen, Peter} (20.01.1861 – 31.07.1918), \emph{Schriftsteller, Journalist, Verleger}!Julies Tagebuch. Roman1895-01-01@\strich\emph{Julies Tagebuch. Roman} {[}1895-01-01{]}|pwv} ſtets den Kopf oben behält und der Juliens\pwindex{Nansen, Peter 20.01.1861 – 31.07.1918@\textsc{Nansen, Peter} (20.01.1861 – 31.07.1918), \emph{Schriftsteller, Journalist, Verleger}!Julies Tagebuch. Roman1895-01-01@\strich\emph{Julies Tagebuch. Roman} {[}1895-01-01{]}|pwv} Liebe in genau abgezählten Tropfen zu ſich nimmt:
               Drei Eßlöffel voll und nicht mehr; das Übrige \strikeout{iſt
                     ſein\textcolor{gray}{er}} wäre ſeiner Geſundheit ſchädlich; und ſo hört er auf{[},{]}
               gerade, wo es nöthig iſt. Iſt das wirklich wahr? Du kennſt dieſe Seite des Lebens
               beſſer, wie ich, {\pb}aber ich kanns nicht glauben, daß
               das wahr iſt. Gerade in dieſem Buche\pwindex{Nansen, Peter 20.01.1861 – 31.07.1918@\textsc{Nansen, Peter} (20.01.1861 – 31.07.1918), \emph{Schriftsteller, Journalist, Verleger}!Julies Tagebuch. Roman1895-01-01@\strich\emph{Julies Tagebuch. Roman} {[}1895-01-01{]}|pwv} fehlt mir \strikeout{des Lebens fülle} des Lebens
               Fülle. Gar ſo einfach liegen doch die Dinge nicht. Mir \strikeout{\textcolor{gray}{wa} ſch} riecht \strikeout{das} das Buch\pwindex{Nansen, Peter 20.01.1861 – 31.07.1918@\textsc{Nansen, Peter} (20.01.1861 – 31.07.1918), \emph{Schriftsteller, Journalist, Verleger}!Julies Tagebuch. Roman1895-01-01@\strich\emph{Julies Tagebuch. Roman} {[}1895-01-01{]}|pwv} zu ſehr nach \strikeout{Schreb} Schreibtiſch. In »Maria\pwindex{Nansen, Peter 20.01.1861 – 31.07.1918@\textsc{Nansen, Peter} (20.01.1861 – 31.07.1918), \emph{Schriftsteller, Journalist, Verleger}!Maria. Ein Buch der Liebe1895@\strich\emph{Maria. Ein Buch der Liebe} {[}1895{]}|pw}« iſt Wärme und Süßigkeit. Ich halte das für das erſte
               der beiden Bücher\pwindex{Nansen, Peter 20.01.1861 – 31.07.1918@\textsc{Nansen, Peter} (20.01.1861 – 31.07.1918), \emph{Schriftsteller, Journalist, Verleger}!Maria. Ein Buch der Liebe1895@\strich\emph{Maria. Ein Buch der Liebe} {[}1895{]}|pwv}\pwindex{Nansen, Peter 20.01.1861 – 31.07.1918@\textsc{Nansen, Peter} (20.01.1861 – 31.07.1918), \emph{Schriftsteller, Journalist, Verleger}!Julies Tagebuch. Roman1895-01-01@\strich\emph{Julies Tagebuch. Roman} {[}1895-01-01{]}|pwv},
               und ich finde es unnöthig, daß \textsc{Nansen\pwindex{Nansen, Peter 20.01.1861 – 31.07.1918@\textsc{Nansen, Peter} (20.01.1861 – 31.07.1918), \emph{Schriftsteller, Journalist, Verleger}|pw}} nach der poetiſchen Liebesgeſchichte uns dieſelbe Geſchichte noch einmal »wahr«
               geſchrieben hat. Gibt es überhaupt {\pb}wahre
               Liebesgeſchichten? {\dotsfive} Das iſt vielleicht Alles ſehr \strikeout{du} dumm, was ich da ſage; aber mir fehlt etwas an dem
                  Buche\pwindex{Nansen, Peter 20.01.1861 – 31.07.1918@\textsc{Nansen, Peter} (20.01.1861 – 31.07.1918), \emph{Schriftsteller, Journalist, Verleger}!Julies Tagebuch. Roman1895-01-01@\strich\emph{Julies Tagebuch. Roman} {[}1895-01-01{]}|pwv}, und ich kann nicht
               recht ausdrücken, was mir fehlt{\dots}\pend
           \pstart
           Das wäre wohl Alles für heut. Bald, allerbaldigſt höre
               ich von Dir, nicht wahr?\pend
           \pstart
           Grüß’ Dich Gott, mein lieber Freund!\pend
           \pstart
           Dein treuer {\\[\baselineskip]}\spacefill\mbox{Paul Goldmann}\pend
           \leftskip=0em{}\pstart
           \noindent{}Viele Grüße an \textsc{Richard\pwindex{Beer-Hofmann, Richard 1866-07-11 – 1945-09-26@\textsc{Beer-Hofmann, Richard} (1866-07-11 – 1945-09-26), \emph{Schriftsteller}|pw}}!\pend
           {\bigskip}\pstart
           \noindent{}{\pb}\textcolor{gray}{\textbf{INSTITUT RUDY\orgindex{Institut Rudy@Institut Rudy|pw}}}\hfill \textcolor{gray}{\textbf{Paris\oindex{Paris@\textbf{Paris}|pw}, le}}{ }{[}hs. Riese:{]} { }3 October \textcolor{gray}{\textbf{189}}5\pend
           \pstart
           \textcolor{gray}{\textbf{\begin{otherlanguage}{french}FONDÉ EN\end{otherlanguage}}}{ }1860\pend
           \pstart
           \textcolor{gray}{\textbf{\begin{otherlanguage}{french}LANGUES, LETTRES, SCIENCES\end{otherlanguage}}}\pend
           \pstart
           \textcolor{gray}{\textbf{\begin{otherlanguage}{french}ARTS D’AGRÉMENT\end{otherlanguage}}}\pend
           \pstart
           \textcolor{gray}{\textbf{4, \begin{otherlanguage}{french}RUE CAUMARTIN\end{otherlanguage}, 4\oindex{Rue de Caumartin@\textbf{Rue de Caumartin}|pw}}}\pend
           \pstart
           \textcolor{gray}{\textbf{\emph{\begin{otherlanguage}{french}(BOULEVARD DES CAPUCINES)\oindex{Boulevard de Capucines@\textbf{Boulevard de Capucines}|pw}\end{otherlanguage}}}}\pend
           \pstart
           \textcolor{gray}{\textbf{\begin{otherlanguage}{french}CI-DEVANT: 7, RUE
                              ROYALE\oindex{Rue Royale@\textbf{Rue Royale}|pw}\end{otherlanguage}}}\pend
           \pstart\center{}Sehr geehrter Herr Doctor!\pend\pstart
           Auf Empfehlung des Herrn D\textsuperscript{r}{ }\label{K_L02750-5v}\edtext{Gollmann\pwindex{Gollmann, Wilhelm 1822-09-06 – 1904-01-28@\textsc{Gollmann, Wilhelm} (1822-09-06 – 1904-01-28), \emph{Mediziner}|pw}}{\lemma{\textnormal{\emph{Gollmann}}}\Cendnote{\textnormal{Wilhelm Gollmann\pwindex{Gollmann, Wilhelm 1822-09-06 – 1904-01-28@\textsc{Gollmann, Wilhelm} (1822-09-06 – 1904-01-28), \emph{Mediziner}|pwk} war ein Wien\oindex{Wien@\textbf{Wien}|pwk}er Mediziner, der von Schnitzler\pwindex{Schnitzler, Arthur 15.05.1862 – 21.10.1931@\textsc{Schnitzler, Arthur} (15.05.1862 – 21.10.1931), \emph{Schriftsteller, Mediziner}|pwk} die Erlaubnis hatte, \emph{Sterben}\pwindex{Schnitzler, Arthur 15.05.1862 – 21.10.1931@\textsc{Schnitzler, Arthur} (15.05.1862 – 21.10.1931), \emph{Schriftsteller, Mediziner}!Sterben. Novelle1894-10-01 – 1894-12-01@\strich\emph{Sterben. Novelle} {[}1894-10-01 – 1894-12-01{]}|pwk} ins Englische zu übersetzen. Er delegierte die Aufgabe an Mary Hargrave\pwindex{Hargrave, Mary @\textsc{Hargrave, Mary}, \emph{Übersetzerin}|pwk}. Der Verleger William Heinemann\pwindex{Heinemann, William 18.05.1863 – 05.10.1920@\textsc{Heinemann, William} (18.05.1863 – 05.10.1920), \emph{Verleger}|pwk} sagte aber ab, weil: »\begin{otherlanguage}{english}there has been so marked a reaction in this country
                        of late against the morbid and the horrible in fiction that I feel almost
                        certain the book in spite of its merits would be a failure here\end{otherlanguage}« (Brief von Wilhelm
                        Gollmann\pwindex{Gollmann, Wilhelm 1822-09-06 – 1904-01-28@\textsc{Gollmann, Wilhelm} (1822-09-06 – 1904-01-28), \emph{Mediziner}|pwk} an Schnitzler\pwindex{Schnitzler, Arthur 15.05.1862 – 21.10.1931@\textsc{Schnitzler, Arthur} (15.05.1862 – 21.10.1931), \emph{Schriftsteller, Mediziner}|pwk},
                        21. 9. 1896, \emph{DLA},
                  85.1.3186).}}}\label{K_L02750-5h} erlaube ich mir Sie um die Adreſſe des Herrn \textsc{Schnitzler}, Schriftſteller in \textsc{Wien}\oindex{Wien@\textbf{Wien}|pw}, zu erſuchen, da ich mich beftreffs \label{K_L02750-6v}\edtext{Ueberſetzung}{\lemma{\textnormal{\emph{Ueberſetzung}}}\Cendnote{\textnormal{Eine Übersetzung von \emph{Liebelei}\pwindex{Schnitzler, Arthur 15.05.1862 – 21.10.1931@\textsc{Schnitzler, Arthur} (15.05.1862 – 21.10.1931), \emph{Schriftsteller, Mediziner}!Liebelei. Schauspiel in drei Akten1895-10-09@\strich\emph{Liebelei. Schauspiel in drei Akten} {[}1895-10-09{]}|pwk} durch Riese\pwindex{Riese, M. O. @\textsc{Riese, M. O.}, \emph{Übersetzer, Lehrer}|pwk} ist nicht bekannt.}}}\label{K_L02750-6h}{ }\introOben{}ins Franzöſische\introOben{} ſeines Stück\pwindex{Schnitzler, Arthur 15.05.1862 – 21.10.1931@\textsc{Schnitzler, Arthur} (15.05.1862 – 21.10.1931), \emph{Schriftsteller, Mediziner}!Liebelei. Schauspiel in drei Akten1895-10-09@\strich\emph{Liebelei. Schauspiel in drei Akten} {[}1895-10-09{]}|pwv}es \textsc{Liebelei\pwindex{Schnitzler, Arthur 15.05.1862 – 21.10.1931@\textsc{Schnitzler, Arthur} (15.05.1862 – 21.10.1931), \emph{Schriftsteller, Mediziner}!Liebelei. Schauspiel in drei Akten1895-10-09@\strich\emph{Liebelei. Schauspiel in drei Akten} {[}1895-10-09{]}|pw}} an ihn wenden möchte.\pend
           \pstart
           Ihnen im Voraus für Ihre freundliche Mühe beſtens dankend zeichne\pend
           \pstart
           Hochachtungsvoll {\\[\baselineskip]}\spacefill\mbox{\label{K_L02750-7v}\edtext{M O Riese}{\lemma{\textnormal{\emph{M O Riese}}}\Cendnote{\textnormal{Sprachlehrer\pwindex{Riese, M. O. @\textsc{Riese, M. O.}, \emph{Übersetzer, Lehrer}|pwkv} für
                        Deutsch und Englisch in Paris\oindex{Paris@\textbf{Paris}|pwk}}}}\label{K_L02750-7h}\pwindex{Riese, M. O. @\textsc{Riese, M. O.}, \emph{Übersetzer, Lehrer}|pw}}\pend
           \leftskip=0em{}
         
         \endnumbering\mylabel{h}\end{ledgroupsized}  \newcommand{\dateiname}{L02750}\newcommand{\titel}{Paul Goldmann an Arthur Schnitzler, 7. 10. [1895]}\newcommand{\editorInnen}{Martin Anton Müller und Laura Untner}%% latex-leseansicht-abspann.tex
%% Abspann für die Leseansicht.
%% Der Schalter \ifkorrekturansicht ist bereits durch den Vorspann gesetzt.

%% latex-abspann.tex
%% Gemeinsamer Abspann für Korrekturansicht und Leseansicht.
%% Setzt den Schalter \ifkorrekturansicht voraus (gesetzt in den
%% einbindenden Dateien latex-korrekturansicht-abspann.tex bzw.
%% latex-leseansicht-abspann.tex).
%% ---------------------------------------------------------------

\normalsize

% Das esempio-Environment wird nur in der Leseansicht benötigt
\ifkorrekturansicht\else
\newenvironment{esempio}[3]%
{
    \vspace{1.5ex}
    \rlap{\underline{#1}}
    \par
    \setlength{\parindent}{0cm}
    \nopagebreak
    \leftskip=#2cm
    \rightskip=#3cm
}
{
    \par
}
\fi

\doendnotes{C}
\bigskip
\vfill

\clearpage

\footnotesize

\ifkorrekturansicht
  \lohead{\textsc{register}}
\fi

% theindex-Environment neu definieren ohne reledmac
\makeatletter
\renewenvironment{theindex}{%
  \ifkorrekturansicht
    \section*{\indexname}%
  \else
    \subsubsection*{Index der erwähnten Entitäten}%
  \fi
  \setlength{\parindent}{0pt}%
  \setlength{\parskip}{0pt plus 0.3pt}%
  \let\item\@idxitem
}{%
  \ifkorrekturansicht\clearpage\fi
}
\makeatother

\IfFileExists{\jobname-pw.ind}{\input{\jobname-pw.ind}}{}

% Quellenangabe nur in der Leseansicht
\ifkorrekturansicht\else
% Fallback-Definitionen, falls die .tex-Datei \titel etc. nicht gesetzt hat
\providecommand{\titel}{}
\providecommand{\editorInnen}{}
\providecommand{\dateiname}{\jobname}

\vspace{3cm}

\vfill

\footnotesize
\textsc{Quelle}: \titel. Herausgegeben von {\editorInnen}. In: \emph{Arthur Schnitzler: Briefwechsel mit Autorinnen und Autoren}.
 Digitale Edition, https://schnitzler-briefe.acdh.oeaw.ac.at/{\dateiname}.html (Stand \today)
\fi

\end{document}


      