%% latex-leseansicht-vorspann.tex
%% Vorspann für die Leseansicht.
%% Lädt die gemeinsame Datei latex-vorspann.tex mit nicht gesetztem Schalter.

\newif\ifkorrekturansicht
\korrekturansichtfalse

\input{../tex-inputs/latex-vorspann}


\section[Paul Goldmann an Arthur Schnitzler, 7. 10. [1895]]{L02750 Paul Goldmann an Arthur Schnitzler, 7. 10. [1895]}
\nopagebreak\mylabel{L02750v}
\rehead{ }\normalsize\beginnumbering\briefempfaengerindex{Schnitzler, Arthur@\textsc{Schnitzler, Arthur}!zzzGoldmann, Paul@\emph{von Paul Goldmann}!1895-10-071@{7. 10. [1895]}|(be}
\toendnotes[C]{\smallbreak\pagebreak[2]}
\correspDesc{Versand  durch Paul Goldmann am 7. 10. [1895] in Paris
\newline{}Erhalt  durch Arthur Schnitzler im Zeitraum [8. 10. 1895
                  – 12. 10. 1895?] in Wien}\toendnotes[C]{\smallbreak}
\Standort{DLA, A:Schnitzler, HS.NZ85.1.3165.}
\physDesc{Brief, 5 Blätter, 18 Seiten, 8118 Zeichen
\newline{}Handschrift: blaue Tinte, deutsche Kurrent
\newline{}Beilage: handschriftlicher Brief, 1 Blatt, 1 Seite, deutsche Kurrent; im
                                 Deutschen Literaturarchiv Marbach unter der Signatur
                                 HS.NZ85.1.3166/9 eingeordnet und damit den Korrespondenzstücken des
                                 Jahres 1896 zugeordnet. Bleistiftvermerk
                                 von Schnitzler: »\textsc{Inst. Rud\textcolor{gray}{y}\orgindex{Institut Rudy@Institut Rudy|pw}}« 
\newline{}Schnitzler: mit rotem Buntstift vier Unterstreichungen und das Jahr »95« vermerkt }\toendnotes[C]{\smallbreak}
\pstart
           {\pb}\textcolor{gray}{\textbf{\textbf{Frankfurter Zeitung\orgindex{Frankfurter Zeitung@Frankfurter Zeitung|pw}}}}\pend
           
\pstart
           \textcolor{gray}{\textbf{(\begin{otherlanguage}{french}Gazette de Francfort\end{otherlanguage}\orgindex{Frankfurter Zeitung@Frankfurter Zeitung|pw}).}}\pend
           
\pstart
           \textcolor{gray}{\textbf{\textbf{\begin{otherlanguage}{french}Fondateur M. L.
                              Sonnemann\pwindex{Sonnemann, Leopold 29.\,10.\,1831 Höchberg – 30.\,10.\,1909 Frankfurt am Main@\textsc{Sonnemann, Leopold} (29.\,10.\,1831 Höchberg – 30.\,10.\,1909 Frankfurt am Main), \emph{Journalist, Herausgeber}|pw}\end{otherlanguage}.}}}\pend
           
\pstart
           \begin{otherlanguage}{french}\textcolor{gray}{\textbf{Journal politique, financier,}}\end{otherlanguage}\hfill \textsc{Paris\oindex{Paris@\textbf{Paris}, \emph{Hauptstadt}|pw}}, 7. Oktober.\pend
           
\pstart
           \begin{otherlanguage}{french}\textcolor{gray}{\textbf{commercial et littéraire.}}\end{otherlanguage}\pend
           
\pstart
           \begin{otherlanguage}{french}\textcolor{gray}{\textbf{\textbf{Paraissant trois fois par jour.}}}\end{otherlanguage}\pend
           
\pstart
           \begin{otherlanguage}{french}\textcolor{gray}{\textbf{\textbf{Bureau à Paris\oindex{Paris@\textbf{Paris}, \emph{Hauptstadt}|pw}}}}\end{otherlanguage}\pend
           
\pstart
           \begin{otherlanguage}{french}\textcolor{gray}{\textbf{\textbf{24. Rue Feydeau\oindex{rue Feydeau@\textbf{rue Feydeau}, \emph{Straße}|pw}.}}}\end{otherlanguage}\pend
           
\pstart\center{}Mein lieber Freund,\pend\vspace{0.5em}
\pstart
           dieſer Brief trifft Dich alſo am Vorabend{ }\label{K_L02750-1v}\edtext{großer Ereigniſſe\eventindex{Burgtheater@\textbf{Burgtheater}!Uraufführung von Liebelei, Premiere von Rechte der Seele, 9.10.1895@Uraufführung von Liebelei, Premiere von Rechte der Seele, 9.10.1895|pwv}}{\lemma{\textnormal{\emph{großer Ereignisse}}}\Cendnote{\textnormal{Uraufführung von \emph{Liebelei}\pwindex{Schnitzler, Arthur 15.\,5.\,1862 Wien – 21.\,10.\,1931 ebd.@\textsc{Schnitzler, Arthur} (15.\,5.\,1862 Wien – 21.\,10.\,1931 ebd.), \emph{Schriftsteller, Mediziner}!Liebelei. Schauspiel in drei Akten@\strich\emph{Liebelei. Schauspiel in drei Akten}|pwk}\eventindex{Burgtheater@\textbf{Burgtheater}!Uraufführung von Liebelei, Premiere von Rechte der Seele, 9.10.1895@Uraufführung von Liebelei, Premiere von Rechte der Seele, 9.10.1895|pwk} am 9. 10. 1895 im Burgtheater\oindex{Wien@\textbf{Wien}!I., Innere Stadt@\textbf{I., Innere Stadt}!Burgtheater@\textbf{Burgtheater}, \emph{Theater}|pwk}}}}\label{K_L02750-1}\textcolor{gray}{,} oder hoffentlich{ }ſchon am Ereignißtage{ }ſelbſt. Du kannſt Dir denken, mit wie wachſendem Intereſſe
               ich Deine letzten lieben Briefe geleſen. Gern hätte ich{ }ſie raſch beantwortet; aber
               bei mir iſt wieder der Trübſinn eingekehrt; und ich wollte nicht, daß mir allzuviel
               davon in die Feder flöſſe. Ich danke Dir \strikeout{\textcolor{gray}{v}} von Herzen, daß Du mir{ }ſo treulich berichtet haſt. Gern \strikeout{hatte} hätte ich all’ dieſe Zeit {\pb}mit Dir \substVorne{}\textsuperscript{\textcolor{gray}{e}}\substDazwischen{}v\substHinten{}erlebt; aber durch Deine Briefe habe ich doch wenigſtens einen Wellenſchlag
               davon zu{ }ſpüren bekommen. Am \strikeout{\textcolor{gray}{S}\textcolor{gray}{×}\-\textcolor{gray}{×}\-\textcolor{gray}{×}\-\textcolor{gray}{×}\-\textcolor{gray}{×}\textcolor{gray}{ſten}} Schmerzlichſten iſt es mir, daß ich Mittwoch
               nicht da{ }ſein kann. Erſtens, um raſcher zu wiſſen, wie es ausgegangen, und zweitens,
               um \strikeout{D\textcolor{gray}{ir}} mit Dir ein wenig die Zeit bis zum Abend zu verplaudern. Freilich
               hätteſt Du meiner wohl kaum bedurft. Mit großer Freude{ }ſehe ich aus Deinen Briefen,
               wie ruhig Du biſt. Und wenn doch am Mittwoch{ }Nachmittag das Herzklopfen kommen{ }ſollte – in Jener Stunde beſonders, wo
               der Abend über den {\pb}\strikeout{\textcolor{gray}{×}\-\textcolor{gray}{×}\-\textcolor{gray}{×}\-\textcolor{gray}{×}\-\textcolor{gray}{×}\-\textcolor{gray}{×}\-\textcolor{gray}{×}}{ }Volksgarten\oindex{Wien@\textbf{Wien}!I., Innere Stadt@\textbf{I., Innere Stadt}!Volksgarten@\textbf{Volksgarten}, \emph{Park}|pw} niederſinkt, eigens für Dich
               niederſinkt –{ }ſo wirſt Du{ }ſchon eine liebe Hand in Deiner Nähe haben, die bereit iſt,
               die Deinige zu drücken. Ich{ }ſelbſt bin Deiner Sache{ }ſicher. \strikeout{F\textcolor{gray}{a}} Für mich kann es{ }ſich nur um die Größe des Erfolges handeln; ein Mißerfolg iſt
               ausgeſchloſſen, \strikeout{da} aus dem einfachen Grunde, weil
               nicht das ganze Wien\oindex{Wien@\textbf{Wien}, \emph{Verwaltungsgebiet}|pw}er Publicum plötzlich
               irrſinnig werden kann. Oh, ich glaube, es wird{ }ſchön{ }ſein. Vielleicht nicht allzu{ }ſtürmiſch, aber{ }ſchön. Und wenn ich denke, {\pb}daß Du
               dahin gekommen,{ }ſtill und ehrlich, Dir \strikeout{ſ}{ }ſelbſt
               getreu, und einfach Deines lieben Herzens Sprache redend, –{ }ſo fühle ich, daß es ein
               hoher Ehrentag iſt für Dich, für den Poeten{ }ſo{ }ſehr
               wie für den Menſchen, und ein{ }ſtarkes Beiſpiel für uns Alle. Ich habe das Bedürfniß,
               Jeden dieſer Briefe mit Wünſchen zu füllen. Leider kann ich ja bei der ganzen
               Angelegenheit nichts thun, als Dir fortwährend »Glück!« und »Glück!« zurufen. Aber
               hier will ich es wenigſtens an den Meinigen nicht {\pb}fehlen laſſen. So kommt denn noch ein letzter herzinniger Wunſch, daß es gut werden
               möge. Damit umarme ich Dich und laſſe Dich Deinen Weg gehen........\pend
           
\pstart
           Den Mittwoch{ }Abend werde ich mit meinen Gedanken in Wien\oindex{Wien@\textbf{Wien}, \emph{Verwaltungsgebiet}|pw}{ }ſein und werde verſuchen, die Zeit bis zum nächſten
                  Vormittag nicht lang zu finden. Denn, nicht wahr, Du \label{K_L02750-2v}\edtext{telegraphirſt}{\lemma{\textnormal{\emph{telegraphirst}}}\Cendnote{\textnormal{Schnitzler schickte tatsächlich ein
                  Telegramm, Goldmanns\pwindex{Goldmann, Paul 31.\,1.\,1865 Breslau – 25.\,9.\,1935 Wien@\textsc{Goldmann, Paul} (31.\,1.\,1865 Breslau – 25.\,9.\,1935 Wien), \emph{Schriftsteller, Journalist}|pwk} Telegramm vom XXXX Auszeichnungsfehler: Dokument L02693 nicht gefunden reagiert
                  darauf.}}}\label{K_L02750-2} mir ein paar Worte? Und dann{ }ſchickſt Du mir auch wohl die
               Referate, ich{ }ſende {\pb}ſie Dir umgehend zurück. Sehr
               lieb wäre es, wenn auch \textsc{Richard\pwindex{Beer-Hofmann, Richard 11.\,7.\,1866 Wien – 26.\,9.\,1945 New York City@\textsc{Beer-Hofmann, Richard} (11.\,7.\,1866 Wien – 26.\,9.\,1945 New York City), \emph{Schriftsteller}|pw}} mir telegraphiren wollte; der könnte{ }ſchon etwas ausführlicher berichten.\pend
           
\pstart
           Dabei fällt mir ein, daß es am Ende vielleicht doch gut iſt, wenn ich nicht dabei
               bin. Ich hätte mich ausgenommen, wie die unverheirathete ältere Schweſter auf der
               Hochzeit der Jüngeren........\pend
           
\pstart
           Dein letzter Brief war beſonders{ }ſchön. So voll guter Stimmung,{ }ſo zu Herzen gehend!
               Deinem Stück\pwindex{Schnitzler, Arthur 15.\,5.\,1862 Wien – 21.\,10.\,1931 ebd.@\textsc{Schnitzler, Arthur} (15.\,5.\,1862 Wien – 21.\,10.\,1931 ebd.), \emph{Schriftsteller, Mediziner}!Liebelei. Schauspiel in drei Akten@\strich\emph{Liebelei. Schauspiel in drei Akten}|pwv} thuſt Du aber
               doch wohl Unrecht. Gar{ }ſo {\pb}\strikeout{dün}{ }\strikeout{dü\textcolor{gray}{m}} dünn iſt es, weiß Gott, nicht. Du{ }ſelbſt weißt, was Du hätteſt \strikeout{dazu} noch dazuthun können, der Zuſchauer aber nicht,
               und dieſem erſcheint es voll genug. Eines iſt \strikeout{r\textcolor{gray}{e}} richtig, daß die Figur des Alten\pwindex{Schnitzler, Arthur 15.\,5.\,1862 Wien – 21.\,10.\,1931 ebd.@\textsc{Schnitzler, Arthur} (15.\,5.\,1862 Wien – 21.\,10.\,1931 ebd.), \emph{Schriftsteller, Mediziner}!Liebelei. Schauspiel in drei Akten@\strich\emph{Liebelei. Schauspiel in drei Akten}|pwv} hätte erweitert und vertieft werden können. Man hätte gern mit ihm
               nähere Bekanntſchaft gemacht. Aber den gibſt Du uns vielleicht in einem neuen Stücke.
               Und wer könnte auch den Reichthum des Lebens auf der Bühne verlangen, wie Du{ }ſagſt?
                  \strikeout{\textcolor{gray}{×}} Das {\pb}Dramatiſche iſt ja gerade eine Auswahl
               aus der Fülle. Nur das Weſentliche gehört \strikeout{a} auf die
               Bühne; und Du weißt{ }ſelbſt am Beſten, daß die dramatiſche Kunſt in der Aus\strikeout{\textcolor{gray}{×}} Ausſcheidung, Beſchränkung, Vereinfachung liegt. Für des Lebens Reichthum und
               Fülle \strikeout{hat das \textcolor{gray}{×}} iſt das Theater zu klein{\dotssix}\pend
           
\pstart
           Es iſt{ }ſchön, daß es mit den Proben{ }ſo gut gegangen und daß die Leute{ }ſo
               liebenswürdig zu Dir waren. \strikeout{Nach Allem} Nach den Namen
               der Schauſpieler\pwindex{Sonnenthal, Adolf von 21.\,12.\,1834 Budapest – 4.\,4.\,1909 Prag@\textsc{Sonnenthal, Adolf von} (21.\,12.\,1834 Budapest – 4.\,4.\,1909 Prag), \emph{Schauspieler}|pwv}\pwindex{Sandrock, Adele 19.\,8.\,1863 Rotterdam – 30.\,8.\,1937 Berlin@\textsc{Sandrock, Adele} (19.\,8.\,1863 Rotterdam – 30.\,8.\,1937 Berlin), \emph{Schauspielerin}|pwv}\pwindex{Kallina, Anna 31.\,3.\,1874 Wien – 4.\,1.\,1948 ebd.@\textsc{Kallina, Anna} (31.\,3.\,1874 Wien – 4.\,1.\,1948 ebd.), \emph{Schauspielerin}|pwv}\pwindex{Walbeck, Fanny 11.\,10.\,1850 Wien – 15.\,8.\,1919 ebd.@\textsc{Walbeck, Fanny} (11.\,10.\,1850 Wien – 15.\,8.\,1919 ebd.), \emph{Schauspielerin}|pwv}\pwindex{Gerzhofer, Camilla 2.\,2.\,1888 Wien – 8.\,3.\,1961 Graz@\textsc{Gerzhofer, Camilla} (2.\,2.\,1888 Wien – 8.\,3.\,1961 Graz), \emph{Schauspielerin}|pwv}\pwindex{Kutschera, Victor 2.\,5.\,1863 Wien – 20.\,1.\,1933 ebd.@\textsc{Kutschera, Victor} (2.\,5.\,1863 Wien – 20.\,1.\,1933 ebd.), \emph{Regisseur, Schauspieler}|pwv}\pwindex{Zeska, Carl von 31.\,10.\,1862 Hamburg – 18.\,7.\,1938 Wien@\textsc{Zeska, Carl von} (31.\,10.\,1862 Hamburg – 18.\,7.\,1938 Wien), \emph{Schauspieler}|pwv}\pwindex{Mitterwurzer, Friedrich 16.\,10.\,1844 Dresden – 13.\,2.\,1897 Wien@\textsc{Mitterwurzer, Friedrich} (16.\,10.\,1844 Dresden – 13.\,2.\,1897 Wien), \emph{Schauspieler}|pwv}\pwindex{Slanar, Gustav 25.\,6.\,1861 Brünn – 22.\,9.\,1944 Wien@\textsc{Slanar, Gustav} (25.\,6.\,1861 Brünn – 22.\,9.\,1944 Wien), \emph{Schauspieler, Inspizient}|pwv}{ }\strikeout{\textcolor{gray}{n}} und nach {\pb}dem, was Du{ }ſchreibſt, zu{ }ſchließen, wird die Aufführung\pwindex{Schnitzler, Arthur 15.\,5.\,1862 Wien – 21.\,10.\,1931 ebd.@\textsc{Schnitzler, Arthur} (15.\,5.\,1862 Wien – 21.\,10.\,1931 ebd.), \emph{Schriftsteller, Mediziner}!Liebelei. Schauspiel in drei Akten@\strich\emph{Liebelei. Schauspiel in drei Akten}|pwv}
               eine vorzügliche{ }ſein. Es iſt doch auch gut, wenn ein Director vor einem Stücke Angſt
               hat. So iſt er gezwungen, es zum Erfolg zu führen\strikeout{\textcolor{gray}{,}} und die beſten Kräſte{ }ſeines Theaters dafür einzuſetzen. \textsc{Burckhardts\pwindex{Burckhard, Max Eugen 14.\,7.\,1854 Korneuburg – 16.\,3.\,1912 Wien@\textsc{Burckhard, Max Eugen} (14.\,7.\,1854 Korneuburg – 16.\,3.\,1912 Wien), \emph{Schriftsteller, Rechtswissenschaftler, Theaterleiter}|pw}}{ }\strikeout{Z\textcolor{gray}{o}\textcolor{gray}{×}}{ }Haſenſüßerei, unter der Du{ }ſoviel gelitten, kommt
               Dir hier doch am Ende zugute. So \substVorne{}\textsuperscript{läuft}\substDazwischen{}ſtellt\substHinten{} doch Alles am Ende wieder \strikeout{auf} Alles in den
               Dienſt {\pb}des Guten,{ }ſelbſt das anfangs Hindernde. Die
               große Tragödin\pwindex{Sandrock, Adele 19.\,8.\,1863 Rotterdam – 30.\,8.\,1937 Berlin@\textsc{Sandrock, Adele} (19.\,8.\,1863 Rotterdam – 30.\,8.\,1937 Berlin), \emph{Schauspielerin}|pwv} zum Beiſpiel!
               Dieſe verſtehe ich beſonders gut in der Sache. Sie hat geſehen, daß die Rolle\pwindex{Schnitzler, Arthur 15.\,5.\,1862 Wien – 21.\,10.\,1931 ebd.@\textsc{Schnitzler, Arthur} (15.\,5.\,1862 Wien – 21.\,10.\,1931 ebd.), \emph{Schriftsteller, Mediziner}!Liebelei. Schauspiel in drei Akten@\strich\emph{Liebelei. Schauspiel in drei Akten}|pwv} vorzüglich iſt und daß{ }ſie Erfolg haben wird. Das iſt doch \strikeout{\textcolor{gray}{w}} noch ein höherer Genuß, als der, \strikeout{I\textcolor{gray}{n}f} einem ehemaligen Geliebten Infamien anzuthun. So
               wird{ }ſie \strikeout{ſüſ}{ }ſüß und zahm. Das läuft auf das heraus,
               was ich immer{ }ſage: Man gebe{ }ſich mit der Komödianten-Gemeinheit {\pb}nicht ab und{ }ſchaffe ruhig weiter. Das unfehlbar
               beſte Mittel gegen \strikeout{Bühnen-} Theater-Intriguen iſt ein
               gutes Stück. Jawohl, mein Freund, der Sieg des Guten und Schönen. Es iſt gar nicht{ }ſo
               gymnaſiaſtenhaft, daran zu glauben, wie Du{ }ſchreibſt. Ich glaube immer mehr daran.
               Die Gemeinheit und alles Schlechte iſt{ }ſehr{ }ſtark hinieden; aber es gibt doch kaum
               etwas, das{ }ſtärker iſt, als dieſe zwei Herkulaſſe: {\pb}Gut und Schön. Auch ahnſt Du gar nicht, wieviel gerade im Falle \textsc{Arthur Schnitzler} liegt, das Einen wieder mit dem Weltlauf
               auszuſöhnen vermag{\dotssix}\pend
           
\pstart
           Reden wir ein wenig von Geſchäften. Anbei findeſt Du einen Brief, den ich nicht
               beantworten wollte, ohne Dich zu fragen. Ich rathe Dir ab, vorläufig das
               Überſetzungsrecht der »Liebelei\pwindex{Schnitzler, Arthur 15.\,5.\,1862 Wien – 21.\,10.\,1931 ebd.@\textsc{Schnitzler, Arthur} (15.\,5.\,1862 Wien – 21.\,10.\,1931 ebd.), \emph{Schriftsteller, Mediziner}!Liebelei. Schauspiel in drei Akten@\strich\emph{Liebelei. Schauspiel in drei Akten}|pw}« zu vergeben.
               Warten wir erſt ab, wie die Dinge gehen. \textsc{Madame Aubry\pwindex{Aubry, [MMe. Georges] @\textsc{Aubry, [MMe. Georges]}, \emph{Übersetzerin}|pw}} iſt mit der Überſetzung\pwindex{Schnitzler, Arthur 15.\,5.\,1862 Wien – 21.\,10.\,1931 ebd.@\textsc{Schnitzler, Arthur} (15.\,5.\,1862 Wien – 21.\,10.\,1931 ebd.), \emph{Schriftsteller, Mediziner}!petite comédie. Mœurs viennois@\strich\emph{La petite comédie. Mœurs viennois}|pwv}
               der {\pb}»Kleinen
                  Komödie\pwindex{Schnitzler, Arthur 15.\,5.\,1862 Wien – 21.\,10.\,1931 ebd.@\textsc{Schnitzler, Arthur} (15.\,5.\,1862 Wien – 21.\,10.\,1931 ebd.), \emph{Schriftsteller, Mediziner}!kleine Komödie@\strich\emph{Die kleine Komödie}|pw}« fertig. Ertheile ihr die Autoriſation in einem \uline{deutſchen} Briefe, den Du mir{ }ſchicken magſt. \textsc{Aubry\pwindex{Aubry, Georges †~1923@\textsc{Aubry, Georges} (†~1923), \emph{Redakteur}|pw}} hat mir verſprochen, einen kleinen \label{K_L02750-3v}\edtext{Bericht\pwindex{Aubry, Georges †~1923@\textsc{Aubry, Georges} (†~1923), \emph{Redakteur}!Théâtres. [Notre correspondant de Vienne]@\strich\emph{Théâtres. [Notre correspondant de Vienne]}|pwv}}{\lemma{\textnormal{\emph{Bericht}}}\Cendnote{\textnormal{[Georges Aubry\pwindex{Aubry, Georges †~1923@\textsc{Aubry, Georges} (†~1923), \emph{Redakteur}|pwk}]: \emph{Théâtres. [Notre correspondant de Vienne]}\pwindex{Aubry, Georges †~1923@\textsc{Aubry, Georges} (†~1923), \emph{Redakteur}!Théâtres. [Notre correspondant de Vienne]@\strich\emph{Théâtres. [Notre correspondant de Vienne]}|pwk}. In: \emph{La Liberté}\pwindex{Liberté@\emph{La Liberté}|pwk}, Jg. 30, Nr. 11.289, 12. 10. 1895, S. 3. Siehe dazu auch XXXX Auszeichnungsfehler: Dokument L02751 nicht gefunden.}}}\label{K_L02750-3} über die Aufführung der »Liebelei\pwindex{Schnitzler, Arthur 15.\,5.\,1862 Wien – 21.\,10.\,1931 ebd.@\textsc{Schnitzler, Arthur} (15.\,5.\,1862 Wien – 21.\,10.\,1931 ebd.), \emph{Schriftsteller, Mediziner}!Liebelei. Schauspiel in drei Akten@\strich\emph{Liebelei. Schauspiel in drei Akten}|pw}« in die »\textsc{Liberté\pwindex{Liberté@\emph{La Liberté}|pw}}« zu bringen. Schon zu dieſem Zweck brauche ich das oben erbetene Telegramm. Dem
                  \textsc{Herzl\pwindex{Herzl, Theodor 2.\,5.\,1860 Budapest – 3.\,7.\,1904 Edlach@\textsc{Herzl, Theodor} (2.\,5.\,1860 Budapest – 3.\,7.\,1904 Edlach), \emph{Schriftsteller, Journalist}|pw}}{ }ſollteſt Du \uline{doch} ein Feuilleton geben. Glaub’
               mir, Du kannſt es{ }ſchreiben, es iſt Dir nur unbequem. {\pb}Du haſt doch auch{ }ſchon kürzere Sachen gemacht, zum
               Teufel! Denk’ Dir halt, daß Du es \uline{nicht} für die »Neue Freie Preſſe\orgindex{Neue Freie Presse@Neue Freie Presse|pw}«{ }ſchreibſt. Aber ich halte es
               für{ }ſehr wichtig, daß Dein Name auch dort erſcheint. Daß »Sterben\pwindex{Schnitzler, Arthur 15.\,5.\,1862 Wien – 21.\,10.\,1931 ebd.@\textsc{Schnitzler, Arthur} (15.\,5.\,1862 Wien – 21.\,10.\,1931 ebd.), \emph{Schriftsteller, Mediziner}!Mourir. Roman@\strich\emph{Mourir. Roman}|pwv}« bei \textsc{Perrin\orgindex{Éditions Perrin@Éditions Perrin|pw}} erſcheint, iſt vortrefflich. Es iſt ein anſtändiger Verlag\orgindex{Éditions Perrin@Éditions Perrin|pwv}, der{ }ſreilich wenig Verbindungen mit
               Zeitungen hat. Denn hier{ }ſchreibt das Geſindel nur über {\pb}Bücher, wenn der Verleger dem Blatt ein Pauſchale
               zahlt. Aber laß’ gut{ }ſein, ich \strikeout{ſchaff}{ }ſchaff’ Dir{ }ſchon eine oder die andere Beſprechung{\dotssix}\pend
           
\pstart
           Was Du über »Juliens Tagebuch\pwindex{Nansen, Peter 20.\,1.\,1861 Kopenhagen – 31.\,7.\,1918 Mariager@\textsc{Nansen, Peter} (20.\,1.\,1861 Kopenhagen – 31.\,7.\,1918 Mariager), \emph{Schriftsteller, Journalist, Verleger}!Julies Tagebuch. Roman@\strich\emph{Julies Tagebuch. Roman}|pw}«{ }ſchreibſt,
               überzeugt mich nicht. Inzwiſchen habe ich auch »\label{K_L02750-4v}\edtext{Maria\pwindex{Nansen, Peter 20.\,1.\,1861 Kopenhagen – 31.\,7.\,1918 Mariager@\textsc{Nansen, Peter} (20.\,1.\,1861 Kopenhagen – 31.\,7.\,1918 Mariager), \emph{Schriftsteller, Journalist, Verleger}!Maria. Ein Buch der Liebe@\strich\emph{Maria. Ein Buch der Liebe}|pw}}{\lemma{\textnormal{\emph{Maria}}}\Cendnote{\textnormal{Peter Nansen\pwindex{Nansen, Peter 20.\,1.\,1861 Kopenhagen – 31.\,7.\,1918 Mariager@\textsc{Nansen, Peter} (20.\,1.\,1861 Kopenhagen – 31.\,7.\,1918 Mariager), \emph{Schriftsteller, Journalist, Verleger}|pwk}: \emph{Maria. Ein Buch der Liebe}\pwindex{Nansen, Peter 20.\,1.\,1861 Kopenhagen – 31.\,7.\,1918 Mariager@\textsc{Nansen, Peter} (20.\,1.\,1861 Kopenhagen – 31.\,7.\,1918 Mariager), \emph{Schriftsteller, Journalist, Verleger}!Maria. Ein Buch der Liebe@\strich\emph{Maria. Ein Buch der Liebe}|pwk}. Autorisierte Übersetzung aus
                     dem Dänischen von Mathilde Mann\pwindex{Mann, Mathilde 24.\,11.\,1859 Rostock – 14.\,11.\,1925 ebd.@\textsc{Mann, Mathilde} (24.\,11.\,1859 Rostock – 14.\,11.\,1925 ebd.), \emph{Übersetzerin}|pwk}. Berlin\oindex{Berlin@\textbf{Berlin}, \emph{Hauptstadt}|pwk}: \emph{S.                         Fischer}\orgindex{S. Fischer Verlag@S. Fischer Verlag|pwk}{ }1895. (Originalausgabe: \emph{Maria. En Bog om
                        Kjærlighed. Roman}\pwindex{Nansen, Peter 20.\,1.\,1861 Kopenhagen – 31.\,7.\,1918 Mariager@\textsc{Nansen, Peter} (20.\,1.\,1861 Kopenhagen – 31.\,7.\,1918 Mariager), \emph{Schriftsteller, Journalist, Verleger}!Maria. En Bog om Kjærlighed. Roman@\strich\emph{Maria. En Bog om Kjærlighed. Roman}|pwk}, 1894.)}}}\label{K_L02750-4}« geleſen. Das geſällt mir viel beſſer. Ich weiß nicht, ob es \strikeout{w\textcolor{gray}{a}} ein wahres Buch iſt; von dieſen Liebes-Dingen verſtehe ich wenig; aber es iſt
               poetiſch und{ }ſtellenweiſe entzückend poetiſch. {\pb}In
                  »Juliens Tagebuch\pwindex{Nansen, Peter 20.\,1.\,1861 Kopenhagen – 31.\,7.\,1918 Mariager@\textsc{Nansen, Peter} (20.\,1.\,1861 Kopenhagen – 31.\,7.\,1918 Mariager), \emph{Schriftsteller, Journalist, Verleger}!Julies Tagebuch. Roman@\strich\emph{Julies Tagebuch. Roman}|pw}« mag ich vor Allem den Mann
               nicht, dieſen Schwerenöther, dem alle Weiber zufliegen, der{ }ſeine Syſteme mit ihnen
               hat, der \strikeout{J\textcolor{gray}{e}} auch in dem heißen Sturm mit Julie\pwindex{Nansen, Peter 20.\,1.\,1861 Kopenhagen – 31.\,7.\,1918 Mariager@\textsc{Nansen, Peter} (20.\,1.\,1861 Kopenhagen – 31.\,7.\,1918 Mariager), \emph{Schriftsteller, Journalist, Verleger}!Julies Tagebuch. Roman@\strich\emph{Julies Tagebuch. Roman}|pwv}{ }ſtets den Kopf oben behält und der Juliens\pwindex{Nansen, Peter 20.\,1.\,1861 Kopenhagen – 31.\,7.\,1918 Mariager@\textsc{Nansen, Peter} (20.\,1.\,1861 Kopenhagen – 31.\,7.\,1918 Mariager), \emph{Schriftsteller, Journalist, Verleger}!Julies Tagebuch. Roman@\strich\emph{Julies Tagebuch. Roman}|pwv} Liebe in genau abgezählten Tropfen zu{ }ſich nimmt:
               Drei Eßlöffel voll und nicht mehr; das Übrige \strikeout{iſt{ }ſein\textcolor{gray}{er}} wäre{ }ſeiner Geſundheit{ }ſchädlich; und{ }ſo hört er auf{[},{]}
               gerade, wo es nöthig iſt. Iſt das wirklich wahr? Du kennſt dieſe Seite des Lebens
               beſſer, wie ich, {\pb}aber ich kanns nicht glauben, daß
               das wahr iſt. Gerade in dieſem Buche\pwindex{Nansen, Peter 20.\,1.\,1861 Kopenhagen – 31.\,7.\,1918 Mariager@\textsc{Nansen, Peter} (20.\,1.\,1861 Kopenhagen – 31.\,7.\,1918 Mariager), \emph{Schriftsteller, Journalist, Verleger}!Julies Tagebuch. Roman@\strich\emph{Julies Tagebuch. Roman}|pwv} fehlt mir \strikeout{des Lebens fülle} des Lebens
               Fülle. Gar{ }ſo einfach liegen doch die Dinge nicht. Mir \strikeout{\textcolor{gray}{wa}{ }ſch} riecht \strikeout{das} das Buch\pwindex{Nansen, Peter 20.\,1.\,1861 Kopenhagen – 31.\,7.\,1918 Mariager@\textsc{Nansen, Peter} (20.\,1.\,1861 Kopenhagen – 31.\,7.\,1918 Mariager), \emph{Schriftsteller, Journalist, Verleger}!Julies Tagebuch. Roman@\strich\emph{Julies Tagebuch. Roman}|pwv} zu{ }ſehr nach \strikeout{Schreb} Schreibtiſch. In »Maria\pwindex{Nansen, Peter 20.\,1.\,1861 Kopenhagen – 31.\,7.\,1918 Mariager@\textsc{Nansen, Peter} (20.\,1.\,1861 Kopenhagen – 31.\,7.\,1918 Mariager), \emph{Schriftsteller, Journalist, Verleger}!Maria. Ein Buch der Liebe@\strich\emph{Maria. Ein Buch der Liebe}|pw}« iſt Wärme und Süßigkeit. Ich halte das für das erſte
               der beiden Bücher\pwindex{Nansen, Peter 20.\,1.\,1861 Kopenhagen – 31.\,7.\,1918 Mariager@\textsc{Nansen, Peter} (20.\,1.\,1861 Kopenhagen – 31.\,7.\,1918 Mariager), \emph{Schriftsteller, Journalist, Verleger}!Maria. Ein Buch der Liebe@\strich\emph{Maria. Ein Buch der Liebe}|pwv}\pwindex{Nansen, Peter 20.\,1.\,1861 Kopenhagen – 31.\,7.\,1918 Mariager@\textsc{Nansen, Peter} (20.\,1.\,1861 Kopenhagen – 31.\,7.\,1918 Mariager), \emph{Schriftsteller, Journalist, Verleger}!Julies Tagebuch. Roman@\strich\emph{Julies Tagebuch. Roman}|pwv},
               und ich finde es unnöthig, daß \textsc{Nansen\pwindex{Nansen, Peter 20.\,1.\,1861 Kopenhagen – 31.\,7.\,1918 Mariager@\textsc{Nansen, Peter} (20.\,1.\,1861 Kopenhagen – 31.\,7.\,1918 Mariager), \emph{Schriftsteller, Journalist, Verleger}|pw}} nach der poetiſchen Liebesgeſchichte uns dieſelbe Geſchichte noch einmal »wahr«
               geſchrieben hat. Gibt es überhaupt {\pb}wahre
               Liebesgeſchichten? {\dotsfive} Das iſt vielleicht Alles{ }ſehr \strikeout{du} dumm, was ich da{ }ſage; aber mir fehlt etwas an dem
                  Buche\pwindex{Nansen, Peter 20.\,1.\,1861 Kopenhagen – 31.\,7.\,1918 Mariager@\textsc{Nansen, Peter} (20.\,1.\,1861 Kopenhagen – 31.\,7.\,1918 Mariager), \emph{Schriftsteller, Journalist, Verleger}!Julies Tagebuch. Roman@\strich\emph{Julies Tagebuch. Roman}|pwv}, und ich kann nicht
               recht ausdrücken, was mir fehlt{\dots}\pend
           
\pstart
           Das wäre wohl Alles für heut. Bald, allerbaldigſt höre
               ich von Dir, nicht wahr?\pend
           
\pstart
           Grüß’ Dich Gott, mein lieber Freund!\pend
           
\pstart
           Dein treuer {\\[\baselineskip]}\spacefill\mbox{Paul Goldmann}\pend
           \leftskip=0em{}
\pstart
           \noindent{}Viele Grüße an \textsc{Richard\pwindex{Beer-Hofmann, Richard 11.\,7.\,1866 Wien – 26.\,9.\,1945 New York City@\textsc{Beer-Hofmann, Richard} (11.\,7.\,1866 Wien – 26.\,9.\,1945 New York City), \emph{Schriftsteller}|pw}}!\pend
           \selectlanguage{ngerman}\vspace{1em}{\vspace{1\baselineskip}}
\pstart
           {\pb}\textcolor{gray}{\textbf{INSTITUT RUDY\orgindex{Institut Rudy@Institut Rudy|pw}}}\hfill \textcolor{gray}{\textbf{Paris\oindex{Paris@\textbf{Paris}, \emph{Hauptstadt}|pw}, le}}{ }{[}hs. Riese:{]} { }3 October \textcolor{gray}{\textbf{189}}5\pend
           
\pstart
           \textcolor{gray}{\textbf{\begin{otherlanguage}{french}FONDÉ EN\end{otherlanguage}{ }1860}}\pend
           
\pstart
           \textcolor{gray}{\textbf{\begin{otherlanguage}{french}LANGUES, LETTRES, SCIENCES\end{otherlanguage}}}\pend
           
\pstart
           \textcolor{gray}{\textbf{\begin{otherlanguage}{french}ARTS D’AGRÉMENT\end{otherlanguage}}}\pend
           
\pstart
           \textcolor{gray}{\textbf{4, \begin{otherlanguage}{french}RUE CAUMARTIN\end{otherlanguage}, 4\oindex{Rue de Caumartin@\textbf{Rue de Caumartin}, \emph{Straße}|pw}}}\pend
           
\pstart
           \textcolor{gray}{\textbf{\emph{\begin{otherlanguage}{french}(BOULEVARD DES CAPUCINES)\oindex{Boulevard de Capucines@\textbf{Boulevard de Capucines}, \emph{Straße}|pw}\end{otherlanguage}}}}\pend
           
\pstart
           \textcolor{gray}{\textbf{\begin{otherlanguage}{french}CI-DEVANT: 7, RUE
                              ROYALE\oindex{Rue Royale@\textbf{Rue Royale}, \emph{Straße}|pw}\end{otherlanguage}}}\pend
           
\pstart\center{}Sehr geehrter Herr Doctor!\pend\vspace{0.5em}
\pstart
           Auf Empfehlung des Herrn D\textsuperscript{r}{ }\label{K_L02750-5v}\edtext{Gollmann\pwindex{Gollmann, Wilhelm 6.\,9.\,1822 Wien – 28.\,1.\,1904 ebd.@\textsc{Gollmann, Wilhelm} (6.\,9.\,1822 Wien – 28.\,1.\,1904 ebd.), \emph{Mediziner}|pw}}{\lemma{\textnormal{\emph{Gollmann}}}\Cendnote{\textnormal{Wilhelm Gollmann\pwindex{Gollmann, Wilhelm 6.\,9.\,1822 Wien – 28.\,1.\,1904 ebd.@\textsc{Gollmann, Wilhelm} (6.\,9.\,1822 Wien – 28.\,1.\,1904 ebd.), \emph{Mediziner}|pwk} war ein Wien\oindex{Wien@\textbf{Wien}, \emph{Verwaltungsgebiet}|pwk}er Mediziner, der von Schnitzler die Erlaubnis hatte, \emph{Sterben}\pwindex{Schnitzler, Arthur 15.\,5.\,1862 Wien – 21.\,10.\,1931 ebd.@\textsc{Schnitzler, Arthur} (15.\,5.\,1862 Wien – 21.\,10.\,1931 ebd.), \emph{Schriftsteller, Mediziner}!Sterben. Novelle@\strich\emph{Sterben. Novelle}|pwk} ins Englische zu übersetzen. Er delegierte die Aufgabe an Mary Hargrave\pwindex{Hargrave, Mary @\textsc{Hargrave, Mary}, \emph{Übersetzerin}|pwk}. Der Verleger William Heinemann\pwindex{Heinemann, William 18.\,5.\,1863 Surbiton – 5.\,10.\,1920 London@\textsc{Heinemann, William} (18.\,5.\,1863 Surbiton – 5.\,10.\,1920 London), \emph{Verleger}|pwk} sagte aber ab, weil: »\begin{otherlanguage}{english}there has been so marked a reaction in this country
                        of late against the morbid and the horrible in fiction that I feel almost
                        certain the book in spite of its merits would be a failure here\end{otherlanguage}« (Brief von Wilhelm
                        Gollmann\pwindex{Gollmann, Wilhelm 6.\,9.\,1822 Wien – 28.\,1.\,1904 ebd.@\textsc{Gollmann, Wilhelm} (6.\,9.\,1822 Wien – 28.\,1.\,1904 ebd.), \emph{Mediziner}|pwk} an Schnitzler,
                        21. 9. 1896, \emph{DLA},
                  85.1.3186).}}}\label{K_L02750-5} erlaube ich mir Sie um die Adreſſe des Herrn \textsc{Schnitzler}, Schriftſteller in \textsc{Wien}\oindex{Wien@\textbf{Wien}, \emph{Verwaltungsgebiet}|pw}, zu erſuchen, da ich mich beftreffs \label{K_L02750-6v}\edtext{Ueberſetzung}{\lemma{\textnormal{\emph{Uebersetzung}}}\Cendnote{\textnormal{Eine Übersetzung von \emph{Liebelei}\pwindex{Schnitzler, Arthur 15.\,5.\,1862 Wien – 21.\,10.\,1931 ebd.@\textsc{Schnitzler, Arthur} (15.\,5.\,1862 Wien – 21.\,10.\,1931 ebd.), \emph{Schriftsteller, Mediziner}!Liebelei. Schauspiel in drei Akten@\strich\emph{Liebelei. Schauspiel in drei Akten}|pwk} durch Riese\pwindex{Riese, M. O. @\textsc{Riese, M. O.}, \emph{Übersetzer, Sprachlehrer}|pwk} ist nicht bekannt.}}}\label{K_L02750-6}{ }\introOben{}ins Franzöſische\introOben{}{ }ſeines Stück\pwindex{Schnitzler, Arthur 15.\,5.\,1862 Wien – 21.\,10.\,1931 ebd.@\textsc{Schnitzler, Arthur} (15.\,5.\,1862 Wien – 21.\,10.\,1931 ebd.), \emph{Schriftsteller, Mediziner}!Liebelei. Schauspiel in drei Akten@\strich\emph{Liebelei. Schauspiel in drei Akten}|pwv}es \textsc{Liebelei\pwindex{Schnitzler, Arthur 15.\,5.\,1862 Wien – 21.\,10.\,1931 ebd.@\textsc{Schnitzler, Arthur} (15.\,5.\,1862 Wien – 21.\,10.\,1931 ebd.), \emph{Schriftsteller, Mediziner}!Liebelei. Schauspiel in drei Akten@\strich\emph{Liebelei. Schauspiel in drei Akten}|pw}} an ihn wenden möchte.\pend
           
\pstart
           Ihnen im Voraus für Ihre freundliche Mühe beſtens dankend zeichne\pend
           
\pstart
           Hochachtungsvoll {\\[\baselineskip]}\spacefill\mbox{\label{K_L02750-7v}\edtext{M O Riese}{\lemma{\textnormal{\emph{M O Riese}}}\Cendnote{\textnormal{Sprachlehrer\pwindex{Riese, M. O. @\textsc{Riese, M. O.}, \emph{Übersetzer, Sprachlehrer}|pwkv} für
                        Deutsch und Englisch in Paris\oindex{Paris@\textbf{Paris}, \emph{Hauptstadt}|pwk}}}}\label{K_L02750-7}\pwindex{Riese, M. O. @\textsc{Riese, M. O.}, \emph{Übersetzer, Sprachlehrer}|pw}}\pend
           \leftskip=0em{}\selectlanguage{ngerman}\endnumbering\briefempfaengerindex{Schnitzler, Arthur@\textsc{Schnitzler, Arthur}!zzzGoldmann, Paul@\emph{von Paul Goldmann}!1895-10-071@{7. 10. [1895]}|)be}\mylabel{L02750h}  \newcommand{\dateiname}{L02750}\newcommand{\titel}{Paul Goldmann an Arthur Schnitzler, 7. 10. [1895]}\newcommand{\editorInnen}{Martin Anton Müller und Laura Untner}%% latex-leseansicht-abspann.tex
%% Abspann für die Leseansicht.
%% Der Schalter \ifkorrekturansicht ist bereits durch den Vorspann gesetzt.

%% latex-abspann.tex
%% Gemeinsamer Abspann für Korrekturansicht und Leseansicht.
%% Setzt den Schalter \ifkorrekturansicht voraus (gesetzt in den
%% einbindenden Dateien latex-korrekturansicht-abspann.tex bzw.
%% latex-leseansicht-abspann.tex).
%% ---------------------------------------------------------------

\normalsize

% Das esempio-Environment wird nur in der Leseansicht benötigt
\ifkorrekturansicht\else
\newenvironment{esempio}[3]%
{
    \vspace{1.5ex}
    \rlap{\underline{#1}}
    \par
    \setlength{\parindent}{0cm}
    \nopagebreak
    \leftskip=#2cm
    \rightskip=#3cm
}
{
    \par
}
\fi

\doendnotes{C}
\bigskip
\vfill

\clearpage

\footnotesize

\ifkorrekturansicht
  \lohead{\textsc{register}}
\fi

% theindex-Environment neu definieren ohne reledmac
\makeatletter
\renewenvironment{theindex}{%
  \ifkorrekturansicht
    \section*{\indexname}%
  \else
    \subsubsection*{Index der erwähnten Entitäten}%
  \fi
  \setlength{\parindent}{0pt}%
  \setlength{\parskip}{0pt plus 0.3pt}%
  \let\item\@idxitem
}{%
  \ifkorrekturansicht\clearpage\fi
}
\makeatother

\IfFileExists{\jobname-pw.ind}{\input{\jobname-pw.ind}}{}

% Quellenangabe nur in der Leseansicht
\ifkorrekturansicht\else
% Fallback-Definitionen, falls die .tex-Datei \titel etc. nicht gesetzt hat
\providecommand{\titel}{}
\providecommand{\editorInnen}{}
\providecommand{\dateiname}{\jobname}

\vspace{3cm}

\vfill

\footnotesize
\textsc{Quelle}: \titel. Herausgegeben von {\editorInnen}. In: \emph{Arthur Schnitzler: Briefwechsel mit Autorinnen und Autoren}.
 Digitale Edition, https://schnitzler-briefe.acdh.oeaw.ac.at/{\dateiname}.html (Stand \today)
\fi

\end{document}


