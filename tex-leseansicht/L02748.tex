%% latex-korrekturansicht-vorspann.tex
%% Vorspann für die Korrekturansicht.
%% Lädt die gemeinsame Datei latex-vorspann.tex mit gesetztem Schalter.

\newif\ifkorrekturansicht
\korrekturansichttrue

\input{../tex-inputs/latex-vorspann}


\section[Paul Goldmann an Arthur Schnitzler, 23. 9. {[}1895{]}]{L02748 Paul Goldmann an Arthur Schnitzler, 23. 9. {[}1895{]}}
\nopagebreak\mylabel{L02748v}
\rehead{ }\normalsize\beginnumbering\briefempfaengerindex{Schnitzler, Arthur@\textsc{Schnitzler, Arthur}!zzzGoldmann, Paul@\emph{von Paul Goldmann}!1895-09-233@{23. 9. {[}1895{]}}|(be}
\toendnotes[C]{\smallbreak\pagebreak[2]}\Standort{DLA, A:Schnitzler, HS.NZ85.1.3165.}
\physDesc{Brief, 3 Blätter, 11 Seiten, 4821 Zeichen
\newline{}Handschrift: blaue Tinte, deutsche Kurrent
\newline{}Schnitzler: 1) mit Bleistift das Jahr »95« vermerkt  2) mit rotem Buntstift zehn Unterstreichungen}\toendnotes[C]{\smallbreak}
\pstart
           {\pb}\textcolor{gray}{\textbf{\textbf{Frankfurter Zeitung\orgindex{Frankfurter Zeitung@Frankfurter Zeitung|pw}}}}\pend
           
\pstart
           \textcolor{gray}{\textbf{(\begin{otherlanguage}{french}Gazette de Francfort\end{otherlanguage}\orgindex{Frankfurter Zeitung@Frankfurter Zeitung|pw}). }}\pend
           
\pstart
           \textcolor{gray}{\textbf{\textbf{\begin{otherlanguage}{french}Fondateur M. L.
                                 Sonnemann\pwindex{Sonnemann, Leopold 1831-10-29 – 1909-10-30@\textsc{Sonnemann, Leopold} (1831-10-29 – 1909-10-30), \emph{Journalist/Journalistin, Herausgeber/Herausgeberin}|pw}\end{otherlanguage}.}}}\hfill \textsc{Paris\oindex{Paris@\textbf{Paris}, \emph{P.PPLC}|pw}}, 23. September.\pend
           
\pstart
           \begin{otherlanguage}{french}\textcolor{gray}{\textbf{Journal politique, financier,}}\end{otherlanguage}\pend
           
\pstart
           \begin{otherlanguage}{french}\textcolor{gray}{\textbf{commercial et littéraire.}}\end{otherlanguage}\pend
           
\pstart
           \begin{otherlanguage}{french}\textcolor{gray}{\textbf{\textbf{Paraissant trois fois par jour.}}}\end{otherlanguage}\pend
           
\pstart
           \begin{otherlanguage}{french}\textcolor{gray}{\textbf{\textbf{Bureau à Paris\oindex{Paris@\textbf{Paris}, \emph{P.PPLC}|pw}}}}\end{otherlanguage}\pend
           
\pstart
           \begin{otherlanguage}{french}\textcolor{gray}{\textbf{\textbf{24. Rue Feydeau\oindex{rue Feydeau@\textbf{rue Feydeau}, \emph{Straße (K.STR)}|pw}.}}}\end{otherlanguage}\pend
           
\pstart\center{}Mein lieber Freund,\pend\vspace{0.5em}
\pstart
           Dein Brief beginnt mit allerlei Mißſtimmungs-Äußerungen, macht ſchlimme Erwartungen
               rege, – und ſchließlich kommt \strikeout{Gutes} Gutes, nichts als
               Gutes (unberufen!){[}.{]} Über das Ergebniß der \label{K_L02748-1v}\edtext{Leſeprobe}{\lemma{\textnormal{\emph{Leſeprobe}}}\Cendnote{\textnormal{für die Uraufführung von \emph{Liebelei}\pwindex{Liebelei. Schauspiel in drei Akten@\emph{Liebelei. Schauspiel in drei Akten}|pwk} am \emph{Burgtheater}\orgindex{Burgtheater@Burgtheater|pwk}, siehe A. S.: \emph{Tagebuch}, 18. 9. 1895.
               }}}\label{K_L02748-1} freue ich mich von Herzen, und ich glaube, es iſt Anlaß, Dich dazu zu
               beglückwünſchen. Die Haltung der großen Tragödin\pwindex{Sandrock, Adele 1863-08-19 – 1937-08-30@\textsc{Sandrock, Adele} (1863-08-19 – 1937-08-30), \emph{Schauspieler/Schauspielerin}|pwv} iſt luſtig zum Sich-Schütteln. Gewiß kann noch
               allerlei Tückiſches von dieſer Seite kommen – {\pb}aber,
               glaub’ mir, ſie\pwindex{Sandrock, Adele 1863-08-19 – 1937-08-30@\textsc{Sandrock, Adele} (1863-08-19 – 1937-08-30), \emph{Schauspieler/Schauspielerin}|pwv} kann nichts
               mehr verderben\substVorne{}\textsuperscript{.}\substDazwischen{},\substHinten{} ſie iſt im Grunde machtlos. \substVorne{}\textsuperscript{d}\substDazwischen{}D\substHinten{}as ſcheint ſie übrigens ſelbſt zu ſpüren, denn ſonſt hätte ſie Dir nicht
                  \label{K_L02748-2v}\edtext{telephoniſch gratulirt}{\lemma{\textnormal{\emph{telephoniſch gratulirt}}}\Cendnote{\textnormal{Siehe A. S.: \emph{Tagebuch}, 18. 9. 1895.
               }}}\label{K_L02748-2}. Ein \label{K_L02748-3v}\edtext{von \textsc{Speidel\pwindex{Speidel, Ludwig 1830-04-11 – 1906-02-03@\textsc{Speidel, Ludwig} (1830-04-11 – 1906-02-03), \emph{Journalist/Journalistin, Kritiker/Kritikerin}|pw}} günſtig beurtheiltes Stück\pwindex{Liebelei. Schauspiel in drei Akten@\emph{Liebelei. Schauspiel in drei Akten}|pwv}}{\lemma{\textnormal{\emph{von … Stück}}}\Cendnote{\textnormal{Siehe A. S.: \emph{Tagebuch}, 9. 9. 1895.
               }}}\label{K_L02748-3} iſt doch eine verdammte Geſchichte. Davor muß ſelbſt \substVorne{}\textsuperscript{\textcolor{gray}{L}\textcolor{gray}{×}}\substDazwischen{}die\substHinten{} Luderhaftigkeit ſich beugen. \textsc{Speidel\pwindex{Speidel, Ludwig 1830-04-11 – 1906-02-03@\textsc{Speidel, Ludwig} (1830-04-11 – 1906-02-03), \emph{Journalist/Journalistin, Kritiker/Kritikerin}|pw}} hält ſich übrigens wacker. Bravo! Auch \textsc{Burckhardts\pwindex{Burckhard, Max Eugen 14.07.1854 – 16.03.1912@\textsc{Burckhard, Max Eugen} (14.07.1854 – 16.03.1912), \emph{Schriftsteller/Schriftstellerin, Rechtswissenschaftler/Rechtswissenschaftlerin, Theaterleiter/Theaterleiterin}|pw}}{ }\label{K_L02748-4v}\edtext{Äußerungen über die Beſetzung von \textsc{Anatol\pwindex{Anatol@\emph{Anatol}|pw}}}{\lemma{\textnormal{\emph{Äußerungen … Anatol}}}\Cendnote{\textnormal{Am 8. 9. 1895 hatte Max Burckhard\pwindex{Burckhard, Max Eugen 14.07.1854 – 16.03.1912@\textsc{Burckhard, Max Eugen} (14.07.1854 – 16.03.1912), \emph{Schriftsteller/Schriftstellerin, Rechtswissenschaftler/Rechtswissenschaftlerin, Theaterleiter/Theaterleiterin}|pwk}{ }Schnitzler vorgeschlagen, er selbst solle den Anatol\pwindex{Anatol@\emph{Anatol}|pwkv} spielen, Hermann Bahr\pwindex{Bahr, Hermann 19.07.1863 – 15.01.1934@\textsc{Bahr, Hermann} (19.07.1863 – 15.01.1934), \emph{Schriftsteller/Schriftstellerin, Kritiker/Kritikerin}|pwk} den Max\pwindex{Anatol@\emph{Anatol}|pwkv} und Adele Sandrock\pwindex{Sandrock, Adele 1863-08-19 – 1937-08-30@\textsc{Sandrock, Adele} (1863-08-19 – 1937-08-30), \emph{Schauspieler/Schauspielerin}|pwk} alle weiblichen Rollen.}}}\label{K_L02748-4} ſind ein
               artiges Stück Comödie. Es iſt erſtaunlich, wie luſtig das Leben ſein kann, wenn {\pb}es will.\pend
           
\pstart
           Wie Du ſchreiben kannſt, daß Du um ſieben Jahre zurück ſeieſt, iſt mir unklar. Gibt
               es etwa in der Literatur eine Studien- und Examen-Laufbahn, wie in der Jurisprudenz
               und Medicin? Je ſpäter man zu ſchreiben anfängt, umſomehr hat man vorher gelebt. Und
               wenn in den Werken mehr durchgelebtes Leben drin iſt, ſo iſt das ein Gewinn. Hier
               könnte man das \textsc{Paradoxon} machen, daß in der Literatur die
               verlorenen Semeſter gerade die gewonnenen ſind. Hätteſt Du vor ſieben Jahren {\pb}die »Liebelei\pwindex{Liebelei. Schauspiel in drei Akten@\emph{Liebelei. Schauspiel in drei Akten}|pw}«
               ſchreiben können oder »Sterben\pwindex{Sterben. Novelle@\emph{Sterben. Novelle}|pw}«? Unmöglich,
               nicht wahr? Nun alſo!\pend
           
\pstart
           In der \label{K_L02748-5v}\edtext{Correſpondenz\pwindex{Wiener Brief [Die neue Saison im Burgtheater]@\emph{Wiener Brief [Die neue Saison im Burgtheater]}|pwv}, die ich
                  meinte}{\lemma{\textnormal{\emph{Correſpondenz, … meinte}}}\Cendnote{\textnormal{Siehe Paul Goldmann an Arthur Schnitzler, 12. 9. [1895].
               }}}\label{K_L02748-5}, ſprach \textsc{Uhl\pwindex{Uhl, Friedrich 14.05.1825 – 20.01.1906@\textsc{Uhl, Friedrich} (14.05.1825 – 20.01.1906), \emph{Journalist/Journalistin}|pw}} nicht von Dir. Er ſagte nur: das Burgtheater\orgindex{Burgtheater@Burgtheater|pw} verſpreche eine Reihe von Novitäten; das ſei ſchön; er wolle
               abwarten und am Ende der Saiſon Abrechnung halten, ob die Direction\orgindex{Burgtheater@Burgtheater|pwv} alle Verſprechungen erfüllt.
               Damit ſpielte er wohl auch auf die bisherige Verzögerung der »Liebelei\pwindex{Liebelei. Schauspiel in drei Akten@\emph{Liebelei. Schauspiel in drei Akten}|pw}« an, und ich meinte, {\pb}die Abrechnungs-Drohung ſei geeignet, weitere
               Verſchiebungs-Gelüſte etwas zu dämpfen.\pend
           
\pstart
           Daß \label{K_L02748-6v}\edtext{\textsc{Herzl\pwindex{Herzl, Theodor 1860-05-02 – 1904-07-03@\textsc{Herzl, Theodor} (1860-05-02 – 1904-07-03), \emph{Schriftsteller/Schriftstellerin, Journalist/Journalistin}|pw}} liebenswürdig}{\lemma{\textnormal{\emph{Herzl liebenswürdig}}}\Cendnote{\textnormal{Siehe A. S.: \emph{Tagebuch}, 18. 9. 1895.
               }}}\label{K_L02748-6} iſt, iſt gut u. erſtaunt mich nicht. Ich rathe Dir dringend, ſeine Einladung
               anzunehmen und für die »Neue Fr. Pr.\orgindex{Neue Freie Presse@Neue Freie Presse|pw}« \label{K_L02748-7v}\edtext{Feuilletons}{\lemma{\textnormal{\emph{Feuilletons}}}\Cendnote{\textnormal{Schnitzler hat zu keinem Zeitpunkt
                  seines Lebens Feuilletons geschrieben, trotz mehrfacher Angebote von verschiedenen
                  Seiten.}}}\label{K_L02748-7} zu ſchreiben. Sehr nützlich – beſonders um \strikeout{\textcolor{gray}{nun} glen} gelegentlich einen beſſeren Verleger zu
               finden.\pend
           
\pstart
           {\pb}Zur \textsc{Mad. Candiani\pwindex{Candiani, Regina @\textsc{Candiani, Regina}, \emph{Schriftsteller/Schriftstellerin, Übersetzer/Übersetzerin}|pw}} gehe ich demnächſt. Inzwiſchen hat mich die deutſch\oindex{Deutschland@\textbf{Deutschland}, \emph{A.PCLI}|pwv}e Frau\pwindex{Aubry, [MMe. Georges] @\textsc{Aubry, [MMe. Georges]}, \emph{Übersetzer/Übersetzerin}|pwv} eines fran\oindex{Frankreich@\textbf{Frankreich}, \emph{A.PCLI}|pwv}zöſiſchen Collegen\pwindex{Aubry, Georges †~1923@\textsc{Aubry, Georges} (†~1923), \emph{Redakteur/Redakteurin}|pwv}
               erſucht, ich möchte ihr etwas zum Überſetzen empfehlen. Ich habe ihr die »Kleine Komödie\pwindex{kleine Komoedie@\emph{Die kleine Komödie}|pw}« gegeben. Denn der betr. College\pwindex{Aubry, Georges †~1923@\textsc{Aubry, Georges} (†~1923), \emph{Redakteur/Redakteurin}|pwv} iſt an der »\textsc{Liberté\orgindex{Liberte@La Liberté|pw}}«, einem ſehr angeſehenen u. anſtändigen Blatte\orgindex{Liberte@La Liberté|pwv}, u. könnte vielleicht die \label{K_L02748-8v}\edtext{Überſetzung\pwindex{petite comedie. Mœurs viennois@\emph{La petite comédie. Mœurs viennois}|pwv}}{\lemma{\textnormal{\emph{Überſetzung}}}\Cendnote{\textnormal{Arthur Schnitzler: \emph{La Petite comédie. Mœurs viennois}\pwindex{petite comedie. Mœurs viennois@\emph{La petite comédie. Mœurs viennois}|pwk}. Übersetzt von Mme. Georges Aubry\pwindex{Aubry, [MMe. Georges] @\textsc{Aubry, [MMe. Georges]}, \emph{Übersetzer/Übersetzerin}|pwk}. In: \emph{La Liberté}\pwindex{Liberte@\emph{La Liberté}|pwk}, Jg. 30, Nr. 11.327, 19. 11. 1895 bis Nr. 11.336, 28. 11. 1895 (acht Teile).}}}\label{K_L02748-8} dort placiren. Als
               Zeitungs-Novelle ginge die Geſchichte\pwindex{kleine Komoedie@\emph{Die kleine Komödie}|pwv} recht gut. Kriegen wirſt {\pb}Du
               natürlich nichts, aber es wäre recht hübſch, wenn etwas von Dir in einem \strikeout{fran}{ }Pariſ\oindex{Paris@\textbf{Paris}, \emph{P.PPLC}|pw}er Tagesblatte\pwindex{Liberte@\emph{La Liberté}|pwv} erſchiene. Biſt Du einverſtanden, ſo ſchreib\substVorne{}\textsuperscript{t}\substDazwischen{}e\substHinten{} mir einen Brief\substVorne{}\textsuperscript{.}\substDazwischen{},\substHinten{} gerichtet an \textsc{Madame Aubry\pwindex{Aubry, [MMe. Georges] @\textsc{Aubry, [MMe. Georges]}, \emph{Übersetzer/Übersetzerin}|pw}} (dies der Name). »\begin{otherlanguage}{french}\textsc{Madame\pwindex{Aubry, Georges †~1923@\textsc{Aubry, Georges} (†~1923), \emph{Redakteur/Redakteurin}|pwv}, Je vous
                     autorise bien volontiers à traduire en francais ma nouvelle } »Kleine Komödie\pwindex{kleine Komoedie@\emph{Die kleine Komödie}|pw}«\end{otherlanguage}, u. ſonſt etwas
               Verbindliches. Ich wü{[}r{]}\substVorne{}\textsuperscript{\textcolor{gray}{e}}\substDazwischen{}d\substHinten{}e mich freuen, wenn der kleine Plan gelänge{\dotssix}\pend
           
\pstart
           Die \textsc{Ida Fanjung\pwindex{Van-Jung, Ida @\textsc{Van-Jung, Ida}, \emph{Schauspieler/Schauspielerin}|pw}} iſt hier und läßt Euch Alle grüßen. Eine große {\pb}Freude für mich. Mit ihrem offenen Character und ihrer Geradheit iſt ſie wie ein
               männlicher Freund. Freilich ganz unkünſtleriſch und ohne Feinheiten. Sie ſpürt, daß
               ſie unkünſtleriſch iſt, und iſt darum innerlich mit ſich zerfallen. Hätte wohl nicht
               zur Bühne gehen ſollen{\dotssix}\pend
           
\pstart
           Lies’ \textsc{Rubinstein\pwindex{Rubinstein, Anton 1829-11-28 – 1894-11-20@\textsc{Rubinstein, Anton} (1829-11-28 – 1894-11-20), \emph{Komponist/Komponistin, Musikpädagoge/Musikpädagogin, Dirigent/Dirigentin}|pw}}: »Die Muſik u. ihre Meiſter\pwindex{Musik und ihre Meister. Eine Unterredung@\emph{Die Musik und ihre Meister. Eine Unterredung}|pw}«. Habe ſelten
               etwas ſo Geiſtreiches über Muſik geleſen, – wenn er auch \textsc{Wagner\pwindex{Wagner, Richard 22.05.1813 – 13.02.1883@\textsc{Wagner, Richard} (22.05.1813 – 13.02.1883), \emph{Komponist/Komponistin}|pw}} nicht mag. Von »\label{K_L02748-9v}\edtext{\textsc{Juliens} Tagebuch\pwindex{Julies Tagebuch. Roman@\emph{Julies Tagebuch. Roman}|pw}}{\lemma{\textnormal{\emph{Juliens Tagebuch}}}\Cendnote{\textnormal{Peter Nansen\pwindex{Nansen, Peter 20.01.1861 – 31.07.1918@\textsc{Nansen, Peter} (20.01.1861 – 31.07.1918), \emph{Schriftsteller/Schriftstellerin, Journalist/Journalistin, Verleger/Verlegerin}|pwk}: \emph{Julies Tagebuch. Roman}\pwindex{Julies Tagebuch. Roman@\emph{Julies Tagebuch. Roman}|pwk}. Autorisierte Übersetzung aus
                     dem Dänischen von Mathilde Mann\pwindex{Mann, Mathilde 1859-11-24 – 1925-11-14@\textsc{Mann, Mathilde} (1859-11-24 – 1925-11-14), \emph{Übersetzer/Übersetzerin}|pwk}. In: \emph{Neue Deutsche Rundschau}\pwindex{Neue Deutsche Rundschau@\emph{Neue Deutsche Rundschau}|pwk}, Jg. 6, Nr. 1,
                        Januar 1895, S. 11–38; Nr. 2, Februar 1895,
                     S. 116–143; Nr. 3, März 1895, S. 225–254. Im selben Jahr
                  erschien die Buchausgabe bei \emph{S. Fischer}\orgindex{S. Fischer Verlag@S. Fischer Verlag|pwk}
                     (Originalausgabe: \emph{Julies Dagbog.
                        Roman}\pwindex{Julies Dagbog. Roman@\emph{Julies Dagbog. Roman}|pwk}, 1893).}}}\label{K_L02748-9}« bin ich nicht gar ſo entzückt. {\pb}Ich mag die Bücher nicht, die thun, als ob es nichts in der Welt gäbe, als Liebe,
               und als ob das gar ſo wichtig ſei! Freilich, ein Mann\pwindex{Nansen, Peter 20.01.1861 – 31.07.1918@\textsc{Nansen, Peter} (20.01.1861 – 31.07.1918), \emph{Schriftsteller/Schriftstellerin, Journalist/Journalistin, Verleger/Verlegerin}|pwv} von großem Talent. Packt Einen aber nicht in den
               Tiefen.\pend
           
\pstart
           Was Dir \textsc{Paul Schultz\pwindex{Schulz, Paul 1860-07-01 – 1919-01-31@\textsc{Schulz, Paul} (1860-07-01 – 1919-01-31), \emph{Ministerialbeamter/Ministerialbeamte, Beamter/Beamte}|pw}} geſagt, iſt die \label{K_L02748-10v}\edtext{officiöſe
                  Verſion}{\lemma{\textnormal{\emph{officiöſe
                  Verſion}}}\Cendnote{\textnormal{Am 17. 9. 1895 hatte sich
                     Schnitzler mit Paul Schulz\pwindex{Schulz, Paul 1860-07-01 – 1919-01-31@\textsc{Schulz, Paul} (1860-07-01 – 1919-01-31), \emph{Ministerialbeamter/Ministerialbeamte, Beamter/Beamte}|pwk} unterhalten und dabei erfahren, warum Berthold Frischauer\pwindex{Frischauer, Berthold 1851-09-09 – 1924-02-04@\textsc{Frischauer, Berthold} (1851-09-09 – 1924-02-04), \emph{Journalist/Journalistin}|pwk} zum Paris\oindex{Paris@\textbf{Paris}, \emph{P.PPLC}|pwk}er Korrespondenten der \emph{Neuen Freien Presse}\orgindex{Neue Freie Presse@Neue Freie Presse|pwk} in Nachfolge von Theodor Herzl\pwindex{Herzl, Theodor 1860-05-02 – 1904-07-03@\textsc{Herzl, Theodor} (1860-05-02 – 1904-07-03), \emph{Schriftsteller/Schriftstellerin, Journalist/Journalistin}|pwk} ernannt worden war.}}}\label{K_L02748-10} u. eine alberne Lüge. Ich habe
               hier die Wahrheit gehört. Man hat mich nicht genommen aus verſchiedenen {\pb}perſönlichen Gründen, deren hauptſächlicher die alte
                  \label{K_L02748-11v}\edtext{Todfeindſchaft}{\lemma{\textnormal{\emph{Todfeindſchaft}}}\Cendnote{\textnormal{Siehe Paul Goldmann an Arthur Schnitzler, 1. 5. [1894].
               }}}\label{K_L02748-11} war zwiſchen meinem Onkel\pwindex{Mamroth, Fedor 21.02.1851 – 25.06.1907@\textsc{Mamroth, Fedor} (21.02.1851 – 25.06.1907), \emph{Journalist/Journalistin, Kritiker/Kritikerin}|pwv} und dem Blatte\orgindex{Neue Freie Presse@Neue Freie Presse|pwv}{\dotsfive}\pend
           
\pstart
           Meine Stimmung? Ich wünſchte, es wäre wieder Urlaub und ich wäre wieder mit Dir
               zuſammen.\pend
           
\pstart
           Grüß’ Dich Gott, mein lieber Freund, und ſchreib’ bald, – beſonders, wie die Dinge im
                  Burgtheater\orgindex{Burgtheater@Burgtheater|pw} weitergehen.\pend
           
\pstart
           In Treue {\\[\baselineskip]}Dein {\\[\baselineskip]}\spacefill\mbox{Paul Goldmann}\pend
           \leftskip=0em{}
\pstart
           \noindent{}Wie gefällt Dir folgender Satz: »Und alle möglichen Unzulänglichkeiten
                  menſchlicher Verhältniſſe wurden eilig wieder deutlich.«? Du meinſt, das ſei von
                     \textsc{Goethe\pwindex{Goethe, Johann Wolfgang von 1749-08-28 – 1832-03-22@\textsc{Goethe, Johann Wolfgang von} (1749-08-28 – 1832-03-22), \emph{Schriftsteller/Schriftstellerin}|pw}}. Aber nein, es iſt von \textsc{Arthur Schnitzler} und ſteht
                  in Deinem letzten Briefe. Wäre ich jetzt bei Dir, ſo würde ich Dir ſchleunigſt den
                     \textsc{Goethe\pwindex{Goethe, Johann Wolfgang von 1749-08-28 – 1832-03-22@\textsc{Goethe, Johann Wolfgang von} (1749-08-28 – 1832-03-22), \emph{Schriftsteller/Schriftstellerin}|pw}} wegnehmen. Du glaubſt, der Mann\pwindex{Goethe, Johann Wolfgang von 1749-08-28 – 1832-03-22@\textsc{Goethe, Johann Wolfgang von} (1749-08-28 – 1832-03-22), \emph{Schriftsteller/Schriftstellerin}|pwv} ſchreibe \strikeout{d\textcolor{gray}{a}} die auf ihre urſprüngliche Bedeutung zurückgeführte Sprache, das »Deutſche
                  an {\pb}und für ſich«. Aber nein, er ſchreibt einen
                  Styl, \uline{ſeinen} Styl, der ein ganz anderer iſt, als
                  der \textsc{Schnitzlersche}. Laß’ ihn wirklich einmal ein paar
                  Wochen liegen, den alten Herrn, wenn er ſich ſo hinterliſtig in Deine
                  Individualität einſchleicht, wie obiges Beiſpiel zeigt, das mich nicht wenig
                  vergnügt hat.\pend
           \selectlanguage{ngerman}\endnumbering\briefempfaengerindex{Schnitzler, Arthur@\textsc{Schnitzler, Arthur}!zzzGoldmann, Paul@\emph{von Paul Goldmann}!1895-09-233@{23. 9. {[}1895{]}}|)be}\mylabel{L02748h}  \normalsize

\doendnotes{C}
\bigskip
\vfill

\clearpage

\footnotesize

\lohead{\textsc{register}}

% Definiere theindex-Environment komplett neu ohne reledmac
\makeatletter
\renewenvironment{theindex}{%
  \section*{\indexname}%
  \setlength{\parindent}{0pt}%
  \setlength{\parskip}{0pt plus 0.3pt}%
  \let\item\@idxitem
}{%
  \clearpage
}
\makeatother

\IfFileExists{\jobname-pw.ind}{\input{\jobname-pw.ind}}{}

\end{document}

      