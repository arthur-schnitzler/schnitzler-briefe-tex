%% latex-leseansicht-vorspann.tex
%% Vorspann für die Leseansicht.
%% Lädt die gemeinsame Datei latex-vorspann.tex mit nicht gesetztem Schalter.

\newif\ifkorrekturansicht
\korrekturansichtfalse

\input{../tex-inputs/latex-vorspann}


\section[Paul Goldmann an Arthur Schnitzler, 23. 9. {[}1895{]}]{L02748 Paul Goldmann an Arthur Schnitzler, 23. 9. [1895]}
\nopagebreak\mylabel{L02748v}
\rehead{ }\normalsize\beginnumbering\briefempfaengerindex{Schnitzler, Arthur@\textsc{Schnitzler, Arthur}!zzzGoldmann, Paul@\emph{von Paul Goldmann}!1895-09-233@{23. 9. [1895]}|(be}
\toendnotes[C]{\smallbreak\pagebreak[2]}
\correspDesc{Versand  durch Paul Goldmann am 23. 9. [1895] in Paris
\newline{}Erhalt  durch Arthur Schnitzler im Zeitraum [24. 9. 1895
                  – 28. 9. 1895?] in Wien}\toendnotes[C]{\smallbreak}
\Standort{DLA, A:Schnitzler, HS.NZ85.1.3165.}
\physDesc{Brief, 3 Blätter, 11 Seiten, 4821 Zeichen
\newline{}Handschrift: blaue Tinte, deutsche Kurrent
\newline{}Schnitzler: 1) mit Bleistift das Jahr »95« vermerkt  2) mit rotem Buntstift zehn Unterstreichungen}\toendnotes[C]{\smallbreak}
\pstart
           {\pb}\textcolor{gray}{\textbf{\textbf{Frankfurter Zeitung\orgindex{Frankfurter Zeitung@Frankfurter Zeitung|pw}}}}\pend
           
\pstart
           \textcolor{gray}{\textbf{(\begin{otherlanguage}{french}Gazette de Francfort\end{otherlanguage}\orgindex{Frankfurter Zeitung@Frankfurter Zeitung|pw}).}}\pend
           
\pstart
           \textcolor{gray}{\textbf{\textbf{\begin{otherlanguage}{french}Fondateur M. L.
                                 Sonnemann\pwindex{Sonnemann, Leopold 29.\,10.\,1831 Höchberg – 30.\,10.\,1909 Frankfurt am Main@\textsc{Sonnemann, Leopold} (29.\,10.\,1831 Höchberg – 30.\,10.\,1909 Frankfurt am Main), \emph{Journalist, Herausgeber}|pw}\end{otherlanguage}.}}}\hfill \textsc{Paris\oindex{Paris@\textbf{Paris}, \emph{Hauptstadt}|pw}}, 23. September.\pend
           
\pstart
           \begin{otherlanguage}{french}\textcolor{gray}{\textbf{Journal politique, financier,}}\end{otherlanguage}\pend
           
\pstart
           \begin{otherlanguage}{french}\textcolor{gray}{\textbf{commercial et littéraire.}}\end{otherlanguage}\pend
           
\pstart
           \begin{otherlanguage}{french}\textcolor{gray}{\textbf{\textbf{Paraissant trois fois par jour.}}}\end{otherlanguage}\pend
           
\pstart
           \begin{otherlanguage}{french}\textcolor{gray}{\textbf{\textbf{Bureau à Paris\oindex{Paris@\textbf{Paris}, \emph{Hauptstadt}|pw}}}}\end{otherlanguage}\pend
           
\pstart
           \begin{otherlanguage}{french}\textcolor{gray}{\textbf{\textbf{24. Rue Feydeau\oindex{rue Feydeau@\textbf{rue Feydeau}, \emph{Straße}|pw}.}}}\end{otherlanguage}\pend
           
\pstart\center{}Mein lieber Freund,\pend\vspace{0.5em}
\pstart
           Dein Brief beginnt mit allerlei Mißſtimmungs-Äußerungen, macht{ }ſchlimme Erwartungen
               rege, – und{ }ſchließlich kommt \strikeout{Gutes} Gutes, nichts als
               Gutes (unberufen!){[}.{]} Über das Ergebniß der \label{K_L02748-1v}\edtext{Leſeprobe\eventindex{Burgtheater@\textbf{Burgtheater}!Leseprobe von Liebelei, 18.9.1895@Leseprobe von Liebelei, 18.9.1895|pwv}}{\lemma{\textnormal{\emph{Leseprobe}}}\Cendnote{\textnormal{für die Uraufführung\eventindex{Burgtheater@\textbf{Burgtheater}!Uraufführung von Liebelei, Premiere von Rechte der Seele, 9.10.1895@Uraufführung von Liebelei, Premiere von Rechte der Seele, 9.10.1895|pwkv} von \emph{Liebelei}\pwindex{Schnitzler, Arthur 15.\,5.\,1862 Wien – 21.\,10.\,1931 ebd.@\textsc{Schnitzler, Arthur} (15.\,5.\,1862 Wien – 21.\,10.\,1931 ebd.), \emph{Schriftsteller, Mediziner}!Liebelei. Schauspiel in drei Akten@\strich\emph{Liebelei. Schauspiel in drei Akten}|pwk} am \emph{Burgtheater}\orgindex{Burgtheater@Burgtheater|pwk}, siehe A. S.: \emph{Tagebuch}, 18. 9. 1895.
               }}}\label{K_L02748-1} freue ich mich von Herzen, und ich glaube, es iſt Anlaß, Dich dazu zu
               beglückwünſchen. Die Haltung der großen Tragödin\pwindex{Sandrock, Adele 19.\,8.\,1863 Rotterdam – 30.\,8.\,1937 Berlin@\textsc{Sandrock, Adele} (19.\,8.\,1863 Rotterdam – 30.\,8.\,1937 Berlin), \emph{Schauspielerin}|pwv} iſt luſtig zum Sich-Schütteln. Gewiß kann noch
               allerlei Tückiſches von dieſer Seite kommen – {\pb}aber,
               glaub’ mir, ſie\pwindex{Sandrock, Adele 19.\,8.\,1863 Rotterdam – 30.\,8.\,1937 Berlin@\textsc{Sandrock, Adele} (19.\,8.\,1863 Rotterdam – 30.\,8.\,1937 Berlin), \emph{Schauspielerin}|pwv} kann nichts
               mehr verderben\substVorne{}\textsuperscript{.}\substDazwischen{},\substHinten{}{ }ſie iſt im Grunde machtlos. \substVorne{}\textsuperscript{d}\substDazwischen{}D\substHinten{}as{ }ſcheint{ }ſie übrigens{ }ſelbſt zu{ }ſpüren, denn{ }ſonſt hätte{ }ſie Dir nicht
                  \label{K_L02748-2v}\edtext{telephoniſch gratulirt}{\lemma{\textnormal{\emph{telephonisch gratulirt}}}\Cendnote{\textnormal{Siehe A. S.: \emph{Tagebuch}, 18. 9. 1895.
               }}}\label{K_L02748-2}. Ein \label{K_L02748-3v}\edtext{von \textsc{Speidel\pwindex{Speidel, Ludwig 11.\,4.\,1830 Ulm – 3.\,2.\,1906 Wien@\textsc{Speidel, Ludwig} (11.\,4.\,1830 Ulm – 3.\,2.\,1906 Wien), \emph{Journalist, Kritiker}|pw}} günſtig beurtheiltes Stück\pwindex{Schnitzler, Arthur 15.\,5.\,1862 Wien – 21.\,10.\,1931 ebd.@\textsc{Schnitzler, Arthur} (15.\,5.\,1862 Wien – 21.\,10.\,1931 ebd.), \emph{Schriftsteller, Mediziner}!Liebelei. Schauspiel in drei Akten@\strich\emph{Liebelei. Schauspiel in drei Akten}|pwv}}{\lemma{\textnormal{\emph{von … Stück}}}\Cendnote{\textnormal{Siehe A. S.: \emph{Tagebuch}, 9. 9. 1895.
               }}}\label{K_L02748-3} iſt doch eine verdammte Geſchichte. Davor muß{ }ſelbſt \substVorne{}\textsuperscript{\textcolor{gray}{L}\textcolor{gray}{×}}\substDazwischen{}die\substHinten{} Luderhaftigkeit{ }ſich beugen. \textsc{Speidel\pwindex{Speidel, Ludwig 11.\,4.\,1830 Ulm – 3.\,2.\,1906 Wien@\textsc{Speidel, Ludwig} (11.\,4.\,1830 Ulm – 3.\,2.\,1906 Wien), \emph{Journalist, Kritiker}|pw}} hält{ }ſich übrigens wacker. Bravo! Auch \textsc{Burckhardts\pwindex{Burckhard, Max Eugen 14.\,7.\,1854 Korneuburg – 16.\,3.\,1912 Wien@\textsc{Burckhard, Max Eugen} (14.\,7.\,1854 Korneuburg – 16.\,3.\,1912 Wien), \emph{Schriftsteller, Rechtswissenschaftler, Theaterleiter}|pw}}{ }\label{K_L02748-4v}\edtext{Äußerungen über die Beſetzung von \textsc{Anatol\pwindex{Schnitzler, Arthur 15.\,5.\,1862 Wien – 21.\,10.\,1931 ebd.@\textsc{Schnitzler, Arthur} (15.\,5.\,1862 Wien – 21.\,10.\,1931 ebd.), \emph{Schriftsteller, Mediziner}!Anatol@\strich\emph{Anatol}|pw}}}{\lemma{\textnormal{\emph{Äußerungen … Anatol}}}\Cendnote{\textnormal{Am 8. 9. 1895 hatte Max Burckhard\pwindex{Burckhard, Max Eugen 14.\,7.\,1854 Korneuburg – 16.\,3.\,1912 Wien@\textsc{Burckhard, Max Eugen} (14.\,7.\,1854 Korneuburg – 16.\,3.\,1912 Wien), \emph{Schriftsteller, Rechtswissenschaftler, Theaterleiter}|pwk}{ }Schnitzler vorgeschlagen, er selbst solle den Anatol\pwindex{Schnitzler, Arthur 15.\,5.\,1862 Wien – 21.\,10.\,1931 ebd.@\textsc{Schnitzler, Arthur} (15.\,5.\,1862 Wien – 21.\,10.\,1931 ebd.), \emph{Schriftsteller, Mediziner}!Anatol@\strich\emph{Anatol}|pwkv} spielen, Hermann Bahr\pwindex{Bahr, Hermann 19.\,7.\,1863 Linz – 15.\,1.\,1934 München@\textsc{Bahr, Hermann} (19.\,7.\,1863 Linz – 15.\,1.\,1934 München), \emph{Schriftsteller, Kritiker}|pwk} den Max\pwindex{Schnitzler, Arthur 15.\,5.\,1862 Wien – 21.\,10.\,1931 ebd.@\textsc{Schnitzler, Arthur} (15.\,5.\,1862 Wien – 21.\,10.\,1931 ebd.), \emph{Schriftsteller, Mediziner}!Anatol@\strich\emph{Anatol}|pwkv} und Adele Sandrock\pwindex{Sandrock, Adele 19.\,8.\,1863 Rotterdam – 30.\,8.\,1937 Berlin@\textsc{Sandrock, Adele} (19.\,8.\,1863 Rotterdam – 30.\,8.\,1937 Berlin), \emph{Schauspielerin}|pwk} alle weiblichen Rollen.}}}\label{K_L02748-4}{ }ſind ein
               artiges Stück Comödie. Es iſt erſtaunlich, wie luſtig das Leben{ }ſein kann, wenn {\pb}es will.\pend
           
\pstart
           Wie Du{ }ſchreiben kannſt, daß Du um{ }ſieben Jahre zurück{ }ſeieſt, iſt mir unklar. Gibt
               es etwa in der Literatur eine Studien- und Examen-Laufbahn, wie in der Jurisprudenz
               und Medicin? Je{ }ſpäter man zu{ }ſchreiben anfängt, umſomehr hat man vorher gelebt. Und
               wenn in den Werken mehr durchgelebtes Leben drin iſt,{ }ſo iſt das ein Gewinn. Hier
               könnte man das \textsc{Paradoxon} machen, daß in der Literatur die
               verlorenen Semeſter gerade die gewonnenen{ }ſind. Hätteſt Du vor{ }ſieben Jahren {\pb}die »Liebelei\pwindex{Schnitzler, Arthur 15.\,5.\,1862 Wien – 21.\,10.\,1931 ebd.@\textsc{Schnitzler, Arthur} (15.\,5.\,1862 Wien – 21.\,10.\,1931 ebd.), \emph{Schriftsteller, Mediziner}!Liebelei. Schauspiel in drei Akten@\strich\emph{Liebelei. Schauspiel in drei Akten}|pw}«{ }ſchreiben können oder »Sterben\pwindex{Schnitzler, Arthur 15.\,5.\,1862 Wien – 21.\,10.\,1931 ebd.@\textsc{Schnitzler, Arthur} (15.\,5.\,1862 Wien – 21.\,10.\,1931 ebd.), \emph{Schriftsteller, Mediziner}!Sterben. Novelle@\strich\emph{Sterben. Novelle}|pw}«? Unmöglich,
               nicht wahr? Nun alſo!\pend
           
\pstart
           In der \label{K_L02748-5v}\edtext{Correſpondenz\pwindex{Wiener Brief [Die neue Saison im Burgtheater]@\emph{Wiener Brief [Die neue Saison im Burgtheater]}|pwv}, die ich
                  meinte}{\lemma{\textnormal{\emph{Correspondenz, … meinte}}}\Cendnote{\textnormal{Siehe XXXX Auszeichnungsfehler: Dokument L02747 nicht gefunden.
               }}}\label{K_L02748-5},{ }ſprach \textsc{Uhl\pwindex{Uhl, Friedrich 14.\,5.\,1825 Cieszyn – 20.\,1.\,1906 Mondsee@\textsc{Uhl, Friedrich} (14.\,5.\,1825 Cieszyn – 20.\,1.\,1906 Mondsee), \emph{Journalist}|pw}} nicht von Dir. Er{ }ſagte nur: das Burgtheater\orgindex{Burgtheater@Burgtheater|pw} verſpreche eine Reihe von Novitäten; das{ }ſei{ }ſchön; er wolle
               abwarten und am Ende der Saiſon Abrechnung halten, ob die Direction\orgindex{Burgtheater@Burgtheater|pwv} alle Verſprechungen erfüllt.
               Damit{ }ſpielte er wohl auch auf die bisherige Verzögerung der »Liebelei\pwindex{Schnitzler, Arthur 15.\,5.\,1862 Wien – 21.\,10.\,1931 ebd.@\textsc{Schnitzler, Arthur} (15.\,5.\,1862 Wien – 21.\,10.\,1931 ebd.), \emph{Schriftsteller, Mediziner}!Liebelei. Schauspiel in drei Akten@\strich\emph{Liebelei. Schauspiel in drei Akten}|pw}« an, und ich meinte, {\pb}die Abrechnungs-Drohung{ }ſei geeignet, weitere
               Verſchiebungs-Gelüſte etwas zu dämpfen.\pend
           
\pstart
           Daß \label{K_L02748-6v}\edtext{\textsc{Herzl\pwindex{Herzl, Theodor 2.\,5.\,1860 Budapest – 3.\,7.\,1904 Edlach@\textsc{Herzl, Theodor} (2.\,5.\,1860 Budapest – 3.\,7.\,1904 Edlach), \emph{Schriftsteller, Journalist}|pw}} liebenswürdig}{\lemma{\textnormal{\emph{Herzl liebenswürdig}}}\Cendnote{\textnormal{Siehe A. S.: \emph{Tagebuch}, 18. 9. 1895.
               }}}\label{K_L02748-6} iſt, iſt gut u. erſtaunt mich nicht. Ich rathe Dir dringend,{ }ſeine Einladung
               anzunehmen und für die »Neue Fr. Pr.\orgindex{Neue Freie Presse@Neue Freie Presse|pw}« \label{K_L02748-7v}\edtext{Feuilletons}{\lemma{\textnormal{\emph{Feuilletons}}}\Cendnote{\textnormal{Schnitzler hat zu keinem Zeitpunkt
                  seines Lebens Feuilletons geschrieben, trotz mehrfacher Angebote von verschiedenen
                  Seiten.}}}\label{K_L02748-7} zu{ }ſchreiben. Sehr nützlich – beſonders um \strikeout{\textcolor{gray}{nun} glen} gelegentlich einen beſſeren Verleger zu
               finden.\pend
           
\pstart
           {\pb}Zur \textsc{Mad. Candiani\pwindex{Candiani, Regina @\textsc{Candiani, Regina}, \emph{Schriftstellerin, Übersetzerin}|pw}} gehe ich demnächſt. Inzwiſchen hat mich die deutſch\oindex{Deutschland@\textbf{Deutschland}|pwv}e Frau\pwindex{Aubry, [MMe. Georges] @\textsc{Aubry, [MMe. Georges]}, \emph{Übersetzerin}|pwv} eines fran\oindex{Frankreich@\textbf{Frankreich}|pwv}zöſiſchen Collegen\pwindex{Aubry, Georges †~1923@\textsc{Aubry, Georges} (†~1923), \emph{Redakteur}|pwv}
               erſucht, ich möchte ihr etwas zum Überſetzen empfehlen. Ich habe ihr die »Kleine Komödie\pwindex{Schnitzler, Arthur 15.\,5.\,1862 Wien – 21.\,10.\,1931 ebd.@\textsc{Schnitzler, Arthur} (15.\,5.\,1862 Wien – 21.\,10.\,1931 ebd.), \emph{Schriftsteller, Mediziner}!kleine Komödie@\strich\emph{Die kleine Komödie}|pw}« gegeben. Denn der betr. College\pwindex{Aubry, Georges †~1923@\textsc{Aubry, Georges} (†~1923), \emph{Redakteur}|pwv} iſt an der »\textsc{Liberté\orgindex{Liberté@La Liberté|pw}}«, einem{ }ſehr angeſehenen u. anſtändigen Blatte\orgindex{Liberté@La Liberté|pwv}, u. könnte vielleicht die \label{K_L02748-8v}\edtext{Überſetzung\pwindex{Schnitzler, Arthur 15.\,5.\,1862 Wien – 21.\,10.\,1931 ebd.@\textsc{Schnitzler, Arthur} (15.\,5.\,1862 Wien – 21.\,10.\,1931 ebd.), \emph{Schriftsteller, Mediziner}!petite comédie. Mœurs viennois@\strich\emph{La petite comédie. Mœurs viennois}|pwv}}{\lemma{\textnormal{\emph{Übersetzung}}}\Cendnote{\textnormal{Arthur Schnitzler: \emph{La Petite comédie. Mœurs viennois}\pwindex{Schnitzler, Arthur 15.\,5.\,1862 Wien – 21.\,10.\,1931 ebd.@\textsc{Schnitzler, Arthur} (15.\,5.\,1862 Wien – 21.\,10.\,1931 ebd.), \emph{Schriftsteller, Mediziner}!petite comédie. Mœurs viennois@\strich\emph{La petite comédie. Mœurs viennois}|pwk}. Übersetzt von Mme. Georges Aubry\pwindex{Aubry, [MMe. Georges] @\textsc{Aubry, [MMe. Georges]}, \emph{Übersetzerin}|pwk}. In: \emph{La Liberté}\pwindex{Liberté@\emph{La Liberté}|pwk}, Jg. 30, Nr. 11.327, 19. 11. 1895 bis Nr. 11.336, 28. 11. 1895 (acht Teile).}}}\label{K_L02748-8} dort placiren. Als
               Zeitungs-Novelle ginge die Geſchichte\pwindex{Schnitzler, Arthur 15.\,5.\,1862 Wien – 21.\,10.\,1931 ebd.@\textsc{Schnitzler, Arthur} (15.\,5.\,1862 Wien – 21.\,10.\,1931 ebd.), \emph{Schriftsteller, Mediziner}!kleine Komödie@\strich\emph{Die kleine Komödie}|pwv} recht gut. Kriegen wirſt {\pb}Du
               natürlich nichts, aber es wäre recht hübſch, wenn etwas von Dir in einem \strikeout{fran}{ }Pariſ\oindex{Paris@\textbf{Paris}, \emph{Hauptstadt}|pw}er Tagesblatte\pwindex{Liberté@\emph{La Liberté}|pwv} erſchiene. Biſt Du einverſtanden,{ }ſo{ }ſchreib\substVorne{}\textsuperscript{t}\substDazwischen{}e\substHinten{} mir einen Brief\substVorne{}\textsuperscript{.}\substDazwischen{},\substHinten{} gerichtet an \textsc{Madame Aubry\pwindex{Aubry, [MMe. Georges] @\textsc{Aubry, [MMe. Georges]}, \emph{Übersetzerin}|pw}} (dies der Name). »\begin{otherlanguage}{french}\textsc{Madame\pwindex{Aubry, Georges †~1923@\textsc{Aubry, Georges} (†~1923), \emph{Redakteur}|pwv}, Je vous
                     autorise bien volontiers à traduire en francais ma nouvelle} »Kleine Komödie\pwindex{Schnitzler, Arthur 15.\,5.\,1862 Wien – 21.\,10.\,1931 ebd.@\textsc{Schnitzler, Arthur} (15.\,5.\,1862 Wien – 21.\,10.\,1931 ebd.), \emph{Schriftsteller, Mediziner}!kleine Komödie@\strich\emph{Die kleine Komödie}|pw}«\end{otherlanguage}, u.{ }ſonſt etwas
               Verbindliches. Ich wü{[}r{]}\substVorne{}\textsuperscript{\textcolor{gray}{e}}\substDazwischen{}d\substHinten{}e mich freuen, wenn der kleine Plan gelänge{\dotssix}\pend
           
\pstart
           Die \textsc{Ida Fanjung\pwindex{Van-Jung, Ida @\textsc{Van-Jung, Ida}, \emph{Schauspielerin}|pw}} iſt hier und läßt Euch Alle grüßen. Eine große {\pb}Freude für mich. Mit ihrem offenen Character und ihrer Geradheit iſt{ }ſie wie ein
               männlicher Freund. Freilich ganz unkünſtleriſch und ohne Feinheiten. Sie{ }ſpürt, daß{ }ſie unkünſtleriſch iſt, und iſt darum innerlich mit{ }ſich zerfallen. Hätte wohl nicht
               zur Bühne gehen{ }ſollen{\dotssix}\pend
           
\pstart
           Lies’ \textsc{Rubinstein\pwindex{Rubinstein, Anton 28.\,11.\,1829 Ofatinţi – 20.\,11.\,1894 Petergof@\textsc{Rubinstein, Anton} (28.\,11.\,1829 Ofatinţi – 20.\,11.\,1894 Petergof), \emph{Komponist, Musikpädagoge, Dirigent}|pw}}: »Die Muſik u. ihre Meiſter\pwindex{Rubinstein, Anton 28.\,11.\,1829 Ofatinţi – 20.\,11.\,1894 Petergof@\textsc{Rubinstein, Anton} (28.\,11.\,1829 Ofatinţi – 20.\,11.\,1894 Petergof), \emph{Komponist, Musikpädagoge, Dirigent}!Musik und ihre Meister. Eine Unterredung@\strich\emph{Die Musik und ihre Meister. Eine Unterredung}|pw}«. Habe{ }ſelten
               etwas{ }ſo Geiſtreiches über Muſik geleſen, – wenn er auch \textsc{Wagner\pwindex{Wagner, Richard 22.\,5.\,1813 Leipzig – 13.\,2.\,1883 Venedig@\textsc{Wagner, Richard} (22.\,5.\,1813 Leipzig – 13.\,2.\,1883 Venedig), \emph{Komponist}|pw}} nicht mag. Von »\label{K_L02748-9v}\edtext{\textsc{Juliens} Tagebuch\pwindex{Nansen, Peter 20.\,1.\,1861 Kopenhagen – 31.\,7.\,1918 Mariager@\textsc{Nansen, Peter} (20.\,1.\,1861 Kopenhagen – 31.\,7.\,1918 Mariager), \emph{Schriftsteller, Journalist, Verleger}!Julies Tagebuch. Roman@\strich\emph{Julies Tagebuch. Roman}|pw}}{\lemma{\textnormal{\emph{Juliens Tagebuch}}}\Cendnote{\textnormal{Peter Nansen\pwindex{Nansen, Peter 20.\,1.\,1861 Kopenhagen – 31.\,7.\,1918 Mariager@\textsc{Nansen, Peter} (20.\,1.\,1861 Kopenhagen – 31.\,7.\,1918 Mariager), \emph{Schriftsteller, Journalist, Verleger}|pwk}: \emph{Julies Tagebuch. Roman}\pwindex{Nansen, Peter 20.\,1.\,1861 Kopenhagen – 31.\,7.\,1918 Mariager@\textsc{Nansen, Peter} (20.\,1.\,1861 Kopenhagen – 31.\,7.\,1918 Mariager), \emph{Schriftsteller, Journalist, Verleger}!Julies Tagebuch. Roman@\strich\emph{Julies Tagebuch. Roman}|pwk}. Autorisierte Übersetzung aus
                     dem Dänischen von Mathilde Mann\pwindex{Mann, Mathilde 24.\,11.\,1859 Rostock – 14.\,11.\,1925 ebd.@\textsc{Mann, Mathilde} (24.\,11.\,1859 Rostock – 14.\,11.\,1925 ebd.), \emph{Übersetzerin}|pwk}. In: \emph{Neue Deutsche Rundschau}\pwindex{Neue Deutsche Rundschau@\emph{Neue Deutsche Rundschau}|pwk}, Jg. 6, Nr. 1,
                        Januar 1895, S. 11–38; Nr. 2, Februar 1895,
                     S. 116–143; Nr. 3, März 1895, S. 225–254. Im selben Jahr
                  erschien die Buchausgabe bei \emph{S. Fischer}\orgindex{S. Fischer Verlag@S. Fischer Verlag|pwk}
                     (Originalausgabe: \emph{Julies Dagbog.
                        Roman}\pwindex{Nansen, Peter 20.\,1.\,1861 Kopenhagen – 31.\,7.\,1918 Mariager@\textsc{Nansen, Peter} (20.\,1.\,1861 Kopenhagen – 31.\,7.\,1918 Mariager), \emph{Schriftsteller, Journalist, Verleger}!Julies Dagbog. Roman@\strich\emph{Julies Dagbog. Roman}|pwk}, 1893).}}}\label{K_L02748-9}« bin ich nicht gar{ }ſo entzückt. {\pb}Ich mag die Bücher nicht, die thun, als ob es nichts in der Welt gäbe, als Liebe,
               und als ob das gar{ }ſo wichtig{ }ſei! Freilich, ein Mann\pwindex{Nansen, Peter 20.\,1.\,1861 Kopenhagen – 31.\,7.\,1918 Mariager@\textsc{Nansen, Peter} (20.\,1.\,1861 Kopenhagen – 31.\,7.\,1918 Mariager), \emph{Schriftsteller, Journalist, Verleger}|pwv} von großem Talent. Packt Einen aber nicht in den
               Tiefen.\pend
           
\pstart
           Was Dir \textsc{Paul Schultz\pwindex{Schulz, Paul 1.\,7.\,1860 Wien – 31.\,1.\,1919 Kreuzlingen@\textsc{Schulz, Paul} (1.\,7.\,1860 Wien – 31.\,1.\,1919 Kreuzlingen), \emph{Ministerialbeamter, Beamter}|pw}} geſagt, iſt die \label{K_L02748-10v}\edtext{officiöſe
                  Verſion}{\lemma{\textnormal{\emph{officiöse
                  Version}}}\Cendnote{\textnormal{Am 17. 9. 1895 hatte sich
                     Schnitzler mit Paul Schulz\pwindex{Schulz, Paul 1.\,7.\,1860 Wien – 31.\,1.\,1919 Kreuzlingen@\textsc{Schulz, Paul} (1.\,7.\,1860 Wien – 31.\,1.\,1919 Kreuzlingen), \emph{Ministerialbeamter, Beamter}|pwk} unterhalten und dabei erfahren, warum Berthold Frischauer\pwindex{Frischauer, Berthold 9.\,9.\,1851 Brünn – 4.\,2.\,1924 Wien@\textsc{Frischauer, Berthold} (9.\,9.\,1851 Brünn – 4.\,2.\,1924 Wien), \emph{Journalist}|pwk} zum Paris\oindex{Paris@\textbf{Paris}, \emph{Hauptstadt}|pwk}er Korrespondenten der \emph{Neuen Freien Presse}\orgindex{Neue Freie Presse@Neue Freie Presse|pwk} in Nachfolge von Theodor Herzl\pwindex{Herzl, Theodor 2.\,5.\,1860 Budapest – 3.\,7.\,1904 Edlach@\textsc{Herzl, Theodor} (2.\,5.\,1860 Budapest – 3.\,7.\,1904 Edlach), \emph{Schriftsteller, Journalist}|pwk} ernannt worden war.}}}\label{K_L02748-10} u. eine alberne Lüge. Ich habe
               hier die Wahrheit gehört. Man hat mich nicht genommen aus verſchiedenen {\pb}perſönlichen Gründen, deren hauptſächlicher die alte
                  \label{K_L02748-11v}\edtext{Todfeindſchaft}{\lemma{\textnormal{\emph{Todfeindschaft}}}\Cendnote{\textnormal{Siehe XXXX Auszeichnungsfehler: Dokument L02619 nicht gefunden.
               }}}\label{K_L02748-11} war zwiſchen meinem Onkel\pwindex{Mamroth, Fedor 21.\,2.\,1851 Breslau – 25.\,6.\,1907 Frankfurt am Main@\textsc{Mamroth, Fedor} (21.\,2.\,1851 Breslau – 25.\,6.\,1907 Frankfurt am Main), \emph{Journalist, Kritiker}|pwv} und dem Blatte\orgindex{Neue Freie Presse@Neue Freie Presse|pwv}{\dotsfive}\pend
           
\pstart
           Meine Stimmung? Ich wünſchte, es wäre wieder Urlaub und ich wäre wieder mit Dir
               zuſammen.\pend
           
\pstart
           Grüß’ Dich Gott, mein lieber Freund, und{ }ſchreib’ bald, – beſonders, wie die Dinge im
                  Burgtheater\orgindex{Burgtheater@Burgtheater|pw} weitergehen.\pend
           
\pstart
           In Treue {\\[\baselineskip]}Dein {\\[\baselineskip]}\spacefill\mbox{Paul Goldmann}\pend
           \leftskip=0em{}
\pstart
           \noindent{}Wie gefällt Dir folgender Satz: »Und alle möglichen Unzulänglichkeiten
                  menſchlicher Verhältniſſe wurden eilig wieder deutlich.«? Du meinſt, das{ }ſei von
                     \textsc{Goethe\pwindex{Goethe, Johann Wolfgang von 28.\,8.\,1749 Frankfurt am Main – 22.\,3.\,1832 Weimar@\textsc{Goethe, Johann Wolfgang von} (28.\,8.\,1749 Frankfurt am Main – 22.\,3.\,1832 Weimar), \emph{Schriftsteller}|pw}}. Aber nein, es iſt von \textsc{Arthur Schnitzler} und{ }ſteht
                  in Deinem letzten Briefe. Wäre ich jetzt bei Dir,{ }ſo würde ich Dir{ }ſchleunigſt den
                     \textsc{Goethe\pwindex{Goethe, Johann Wolfgang von 28.\,8.\,1749 Frankfurt am Main – 22.\,3.\,1832 Weimar@\textsc{Goethe, Johann Wolfgang von} (28.\,8.\,1749 Frankfurt am Main – 22.\,3.\,1832 Weimar), \emph{Schriftsteller}|pw}} wegnehmen. Du glaubſt, der Mann\pwindex{Goethe, Johann Wolfgang von 28.\,8.\,1749 Frankfurt am Main – 22.\,3.\,1832 Weimar@\textsc{Goethe, Johann Wolfgang von} (28.\,8.\,1749 Frankfurt am Main – 22.\,3.\,1832 Weimar), \emph{Schriftsteller}|pwv}{ }ſchreibe \strikeout{d\textcolor{gray}{a}} die auf ihre urſprüngliche Bedeutung zurückgeführte Sprache, das »Deutſche
                  an {\pb}und für{ }ſich«. Aber nein, er{ }ſchreibt einen
                  Styl, \uline{ſeinen} Styl, der ein ganz anderer iſt, als
                  der \textsc{Schnitzlersche}. Laß’ ihn wirklich einmal ein paar
                  Wochen liegen, den alten Herrn, wenn er{ }ſich{ }ſo hinterliſtig in Deine
                  Individualität einſchleicht, wie obiges Beiſpiel zeigt, das mich nicht wenig
                  vergnügt hat.\pend
           \selectlanguage{ngerman}\endnumbering\briefempfaengerindex{Schnitzler, Arthur@\textsc{Schnitzler, Arthur}!zzzGoldmann, Paul@\emph{von Paul Goldmann}!1895-09-233@{23. 9. [1895]}|)be}\mylabel{L02748h}  \newcommand{\dateiname}{L02748}\newcommand{\titel}{Paul Goldmann an Arthur Schnitzler, 23. 9. [1895]}\newcommand{\editorInnen}{Martin Anton Müller und Laura Untner}%% latex-leseansicht-abspann.tex
%% Abspann für die Leseansicht.
%% Der Schalter \ifkorrekturansicht ist bereits durch den Vorspann gesetzt.

%% latex-abspann.tex
%% Gemeinsamer Abspann für Korrekturansicht und Leseansicht.
%% Setzt den Schalter \ifkorrekturansicht voraus (gesetzt in den
%% einbindenden Dateien latex-korrekturansicht-abspann.tex bzw.
%% latex-leseansicht-abspann.tex).
%% ---------------------------------------------------------------

\normalsize

% Das esempio-Environment wird nur in der Leseansicht benötigt
\ifkorrekturansicht\else
\newenvironment{esempio}[3]%
{
    \vspace{1.5ex}
    \rlap{\underline{#1}}
    \par
    \setlength{\parindent}{0cm}
    \nopagebreak
    \leftskip=#2cm
    \rightskip=#3cm
}
{
    \par
}
\fi

\doendnotes{C}
\bigskip
\vfill

\clearpage

\footnotesize

\ifkorrekturansicht
  \lohead{\textsc{register}}
\fi

% theindex-Environment neu definieren ohne reledmac
\makeatletter
\renewenvironment{theindex}{%
  \ifkorrekturansicht
    \section*{\indexname}%
  \else
    \subsubsection*{Index der erwähnten Entitäten}%
  \fi
  \setlength{\parindent}{0pt}%
  \setlength{\parskip}{0pt plus 0.3pt}%
  \let\item\@idxitem
}{%
  \ifkorrekturansicht\clearpage\fi
}
\makeatother

\IfFileExists{\jobname-pw.ind}{\input{\jobname-pw.ind}}{}

% Quellenangabe nur in der Leseansicht
\ifkorrekturansicht\else
% Fallback-Definitionen, falls die .tex-Datei \titel etc. nicht gesetzt hat
\providecommand{\titel}{}
\providecommand{\editorInnen}{}
\providecommand{\dateiname}{\jobname}

\vspace{3cm}

\vfill

\footnotesize
\textsc{Quelle}: \titel. Herausgegeben von {\editorInnen}. In: \emph{Arthur Schnitzler: Briefwechsel mit Autorinnen und Autoren}.
 Digitale Edition, https://schnitzler-briefe.acdh.oeaw.ac.at/{\dateiname}.html (Stand \today)
\fi

\end{document}


