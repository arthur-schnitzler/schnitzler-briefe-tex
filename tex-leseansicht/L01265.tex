%% latex-leseansicht-vorspann.tex
%% Vorspann für die Leseansicht.
%% Lädt die gemeinsame Datei latex-vorspann.tex mit nicht gesetztem Schalter.

\newif\ifkorrekturansicht
\korrekturansichtfalse

\input{../tex-inputs/latex-vorspann}


         
         \renewcommand{\erwaehntePersonen}{Personen: Hermann Bahr, Richard Beer-Hofmann,  Schaller, Olga Schnitzler}
         \renewcommand{\erwaehnteOrte}{Orte: Liesingerstraße, Rodaun, Salzburg, Wien, XXIII., Liesing, Österreichischer Hof}
         \renewcommand{\erwaehnteWerke}{}
               \section[Arthur Schnitzler an Richard Beer-Hofmann, 14. 1. 1903]{ Arthur Schnitzler an Richard Beer-Hofmann, 14. 1. 1903}\nopagebreak\mylabel{v}\rehead{ }\begin{ledgroupsized}[t]{13cm}\normalsize\beginnumbering\briefempfaengerindex{Beer-Hofmann, Richard@\textsc{Beer-Hofmann, Richard}!zzzSchnitzler, Arthur@\emph{von Arthur Schnitzler}!1903-01-141@{14. 1. 1903}|(be} \toendnotes[C]{\smallbreak\pagebreak[2]} \Standort{YCGL, MSS 31.}
\physDesc{Brief, 1 Blatt, 4 Seiten, Umschlag, 1100 Zeichen
\newline{}Handschrift: Bleistift, deutsche Kurrent
\newline{}Versand: 1) Stempel: »\nobreak{}\oindex{Salzburg@\textbf{Salzburg}|pwk}Sal\textcolor{gray}{zb}urg, 14. 1. 03, 9–12V\nobreak{}«.   2) Stempel: »\nobreak{}\oindex{Rodaun@\textbf{Rodaun}|pwk}{\pb}Rodaun, 15. 1. 03, 6–7N\nobreak{}«. 
\newline{}Ordnung: mit Bleistift von unbekannter Hand datiert:
                                    »14. 1.« }\buchAbdrucke{\weitereDrucke{Arthur Schnitzler, Richard Beer-Hofmann: \emph{Briefwechsel 1891–1931}. Hg. Konstanze Fliedl. Wien, Zürich: \emph{Europaverlag} 1992, S. 159–160.} }\toendnotes[C]{\smallbreak}\pstart{}{\pb}Hr \textsc{Dr Richard
                     Beer-Hofmann}\pend{}\pstart{}\textsc{Rodaun}\oindex{Rodaun@\textbf{Rodaun}|pw}\pend{}\pstart{}\textsc{bei Liesing\oindex{XXIII., Liesing@\textbf{XXIII., Liesing}|pw}}\pend{}\pstart{}\textsc{b Wien\oindex{Wien@\textbf{Wien}|pw}}\pend{}\pstart{}\textsc{Liesinger Haupts 2\oindex{Liesingerstrasse@\textbf{Liesingerstraße}|pw}.}\pend{}{\bigskip}\pstart
           \raggedleft{}{\pb}\textsc{Salzburg}\oindex{Salzburg@\textbf{Salzburg}|pw}{ }14. 1. 903.{\\}\textsc{Oesterr. Hof}\oindex{Oesterreichischer Hof@\textbf{Österreichischer Hof}|pw}. –\pend
           \pstart
           lieber Richard, bei dem Badebeſitzer \textsc{Schaller}\pwindex{Schaller @\textsc{Schaller}, \emph{Schwimmbadeigentümer}|pw} in Rodaun\oindex{Rodaun@\textbf{Rodaun}|pw}, \textsc{Liesinger}straſſe\oindex{Liesingerstrasse@\textbf{Liesingerstraße}|pw}, wohnt ſeit einigen Tagen unſer
               Hund, \textsc{\label{K_L01265-1v}\edtext{Bern}{\lemma{\textnormal{\emph{Bern}}}\Cendnote{\textnormal{siehe Paul Goldmann an Arthur Schnitzler, 17. 4. [1902] Nach der Absage
                        Beer-Hofmann\pwindex{Beer-Hofmann, Richard 1866-07-11 – 1945-09-26@\textsc{Beer-Hofmann, Richard} (1866-07-11 – 1945-09-26), \emph{Schriftsteller}|pwk}s sagt im
                        April auch Bahr\pwindex{Bahr, Hermann 19.07.1863 – 15.01.1934@\textsc{Bahr, Hermann} (19.07.1863 – 15.01.1934), \emph{Schriftsteller, Kritiker}|pwk} ab.
                        (Hermann Bahr an Arthur Schnitzler, 4. 4. [1903])}}}\label{K_L01265-1h}} genannt. Sie wiſſen dſs wir in Wien\oindex{Wien@\textbf{Wien}|pw} nichts
               mit ihm anfangen können, und daſs wir deshalb jedenfalls auf ſeinen fernern Beſitz
               verzichten {\pb}müſſen. Wenn Sie ihn daher (ſtatt des
                  \label{K_L01265-2v}\edtext{Flirt}{\lemma{\textnormal{\emph{Flirt}}}\Cendnote{\textnormal{der über zehn Jahre alte Hund Beer-Hofmann\pwindex{Beer-Hofmann, Richard 1866-07-11 – 1945-09-26@\textsc{Beer-Hofmann, Richard} (1866-07-11 – 1945-09-26), \emph{Schriftsteller}|pwk}s}}}\label{K_L01265-2h} zu tragen) von mir annehmen wollen, ſo erweiſen Sie
               mir damit nur einen Gefallen. Überlegen Sie ſichs, denn Eile hat die Sache in keiner
               Weiſe. Das Thier wohnt in Ihrer Nähe, warten Sie, bis ihm wieder {\pb}die Haare gewachſen ſind, und fragen Sie ſich, ob Sie
               ſich mit ihm befreunden können. – Wär ich auf dem Land wie Sie, ich behielte ihn
               gern; unter den gegebenen Umſtänden aber wäre mir der Gedanke, daſs \textsc{Bern} in Ihren Beſitz übergeht, der freundlichſte. –\pend
           \pstart
           {\pb}Ich bin mit Olga\pwindex{Schnitzler, Olga 17.01.1882 – 13.01.1970@\textsc{Schnitzler, Olga} (17.01.1882 – 13.01.1970), \emph{Schauspielerin, Sängerin}|pw} ſeit vorgeſtern hier; – und freue mich, inmitten des beruhigenden
               Schneefalls und der winterlichen Stille, daſs ich mich wenigſtens zu diesem
               Entschluſſe aufraffen konnte. Bis Ende der Woche hoffen wir zu bleiben.\pend
           \pstart
           Seien Sie herzlichſt gegrüßt\pend
           \pstart
           Ihr{\\[\baselineskip]}\spacefill\mbox{A.}\pend
           \leftskip=0em{}
         
         \endnumbering\mylabel{h}\end{ledgroupsized}  \newcommand{\dateiname}{L01265}\newcommand{\titel}{Arthur Schnitzler an Richard Beer-Hofmann, 14. 1. 1903}\newcommand{\editorInnen}{Martin Anton Müller und Gerd-Hermann Susen}%% latex-leseansicht-abspann.tex
%% Abspann für die Leseansicht.
%% Der Schalter \ifkorrekturansicht ist bereits durch den Vorspann gesetzt.

%% latex-abspann.tex
%% Gemeinsamer Abspann für Korrekturansicht und Leseansicht.
%% Setzt den Schalter \ifkorrekturansicht voraus (gesetzt in den
%% einbindenden Dateien latex-korrekturansicht-abspann.tex bzw.
%% latex-leseansicht-abspann.tex).
%% ---------------------------------------------------------------

\normalsize

% Das esempio-Environment wird nur in der Leseansicht benötigt
\ifkorrekturansicht\else
\newenvironment{esempio}[3]%
{
    \vspace{1.5ex}
    \rlap{\underline{#1}}
    \par
    \setlength{\parindent}{0cm}
    \nopagebreak
    \leftskip=#2cm
    \rightskip=#3cm
}
{
    \par
}
\fi

\doendnotes{C}
\bigskip
\vfill

\clearpage

\footnotesize

\ifkorrekturansicht
  \lohead{\textsc{register}}
\fi

% theindex-Environment neu definieren ohne reledmac
\makeatletter
\renewenvironment{theindex}{%
  \ifkorrekturansicht
    \section*{\indexname}%
  \else
    \subsubsection*{Index der erwähnten Entitäten}%
  \fi
  \setlength{\parindent}{0pt}%
  \setlength{\parskip}{0pt plus 0.3pt}%
  \let\item\@idxitem
}{%
  \ifkorrekturansicht\clearpage\fi
}
\makeatother

\IfFileExists{\jobname-pw.ind}{\input{\jobname-pw.ind}}{}

% Quellenangabe nur in der Leseansicht
\ifkorrekturansicht\else
% Fallback-Definitionen, falls die .tex-Datei \titel etc. nicht gesetzt hat
\providecommand{\titel}{}
\providecommand{\editorInnen}{}
\providecommand{\dateiname}{\jobname}

\vspace{3cm}

\vfill

\footnotesize
\textsc{Quelle}: \titel. Herausgegeben von {\editorInnen}. In: \emph{Arthur Schnitzler: Briefwechsel mit Autorinnen und Autoren}.
 Digitale Edition, https://schnitzler-briefe.acdh.oeaw.ac.at/{\dateiname}.html (Stand \today)
\fi

\end{document}


      