%% latex-korrekturansicht-vorspann.tex
%% Vorspann für die Korrekturansicht.
%% Lädt die gemeinsame Datei latex-vorspann.tex mit gesetztem Schalter.

\newif\ifkorrekturansicht
\korrekturansichttrue

\input{../tex-inputs/latex-vorspann}


\section[Hugo und Gerty von Hofmannsthal an Arthur Schnitzler, 31. 10. 1908]{L01796 Hugo und Gerty von Hofmannsthal an Arthur Schnitzler,
               31. 10. 1908}
\nopagebreak\mylabel{L01796v}
\rehead{ }\normalsize\beginnumbering\briefempfaengerindex{Schnitzler, Arthur@\textsc{Schnitzler, Arthur}!zzzHofmannsthal, Hugo von@\emph{von Hugo von Hofmannsthal}!1908-10-311@{31. 10. 1908}|(be}
\toendnotes[C]{\smallbreak\pagebreak[2]}\Standort{CUL, Schnitzler, B 43.}
\physDesc{Brief, 1 Blatt, 1 Seite, 4654 Zeichen
\newline{}Schreibmaschine
\newline{}Beilage: Camillo Müller\pwindex{Mueller, Camillo 8.1.1870 – 28.9.1936@\textsc{Müller, Camillo} (8.1.1870 – 28.9.1936), \emph{Rechtsanwalt/Rechtsanwältin, Fechtkämpfer/Fechtkämpferin}|pw}:
                                 eigenhändiger Brief, 2 Blätter, 7 Seiten, schwarze Tinte 
\newline{}Ordnung: 1) Die Abschrift dürfte nach dem Tod Hofmannsthals von seiner Witwe
                                 oder seiner Tochter\pwindex{Zimmer, Christiane 14.05.1902 – 05.01.1987@\textsc{Zimmer, Christiane} (14.05.1902 – 05.01.1987)|pwv} erstellt worden sein. Warum sie sich in
                                 Schnitzlers Nachlass befindet und wo das Original verblieben ist,
                                 bleibt ungeklärt  2) mit Bleistift von unbekannter Hand nummeriert: »\strikeout{296}« 3) mit Bleistift von unbekannter Hand nummeriert:
                                    »303« 4) mit Bleistift von unbekannter Hand
                                    nummeriert: »302«}
\buchAbdrucke{\weitereDrucke{Hugo von Hofmannsthal, Arthur Schnitzler: \emph{Briefwechsel}. Frankfurt am Main: \emph{S. Fischer} 1964, S. 241–242.} }\toendnotes[C]{\smallbreak}
\pstart
           \raggedleft{}{\pb}Rodaun\oindex{Rodaun@\textbf{Rodaun}, \emph{A.ADM4}|pw} d 31 X 08\pend
           
\pstart{}Mein lieber Arthur,\pend\vspace{0.5em}
\pstart
           wegen des Schreibers danke ich sehr aber ich möchte lieber ein Frauenzimmer von
               weiblichem Geschlecht. Um mir das nachzutragen, dürften Sie nicht der berüchtigte
               Erotiker sein! \pend
           
\pstart
           Was den »Morgen\orgindex{Morgen. Wochenschrift fuer deutsche Kultur@Morgen. Wochenschrift für deutsche Kultur|pw}{[}«{]} betrifft, so hänge ich mit diesem schönen Unternehmen
               ausschliesslich nur mehr durch einen Process zusammen, werde aber gern das nächste
               Mal bei Ihnen die Gedichte\pwindex{Gedichte]@\emph{[Gedichte]}|pwv} von
                  Winterstein\pwindex{Winterstein, Alfred von 25.09.1885 – 28.04.1958@\textsc{Winterstein, Alfred von} (25.09.1885 – 28.04.1958), \emph{Schriftsteller/Schriftstellerin, Psychoanalytiker/Psychoanalytikerin, Beamter/Beamte}|pw} anschauen, vielleicht kann man
               sie an Blei\pwindex{Blei, Franz 18.01.1871 – 10.07.1942@\textsc{Blei, Franz} (18.01.1871 – 10.07.1942), \emph{Schriftsteller/Schriftstellerin}|pw} für seine Zeitschrift\orgindex{Hyperion@Hyperion|pwv} schicken oder sonst wo hin.
               Drittens bitte ich Sie recht herzlich den eingelegten Brief mir zuliebe durchzusehen
               und wenn Sie keinen Grund dagegen haben demgemäss dieses Fräulein Braun\pwindex{Braun, Thekla Maria 30.5.1883 – 16. 2. 1965@\textsc{Braun, Thekla Maria} (30.5.1883 – 16. 2. 1965), \emph{Schauspieler/Schauspielerin}|pw} vom Volkstheater\orgindex{Volkstheater@Volkstheater|pw}, das
               sich auch schon direct an Sie gewandt hat, bei sich zu empfangen. Denn ich sage mir
               dass es einem so anständigen Menschen wie Dr. Camillo Müller\pwindex{Mueller, Camillo 8.1.1870 – 28.9.1936@\textsc{Müller, Camillo} (8.1.1870 – 28.9.1936), \emph{Rechtsanwalt/Rechtsanwältin, Fechtkämpfer/Fechtkämpferin}|pw}, der mich ausserdem nur sehr oberflächlich kennt, gewiss
               schwer gefallen ist so ausführlich deswegen an mich zu schreiben und vielleicht hängt
               für die arme Person wirklich unberechenbar viel daran, dass man ihr hilft. Und es ist
               ja sehr möglich, dass sich Herr Weisse\pwindex{Weisse, Adolf 04.04.1855 – 17.07.1933@\textsc{Weisse, Adolf} (04.04.1855 – 17.07.1933), \emph{Theaterleiter/Theaterleiterin, Schauspieler/Schauspielerin}|pw} hier
               wieder einmal wie ein Schwein gegen jemanden benimmt etc.\pend
           
\pstart
           Ich wurschtle mich weiter gegen das Ende meines vierten Aktes\pwindex{Rosenkavalier@\emph{Der Rosenkavalier}|pwv} und bin \pend
           
\pstart
           von Herzen Ihr{\\[\baselineskip]}\spacefill\mbox{Hugo.}\pend
           \leftskip=0em{}
\pstart
           \noindent{}\label{K_L01796-1v}\edtext{Gruss von der Schreiberin}{\lemma{\textnormal{\emph{Gruss … Schreiberin}}}\Cendnote{\textnormal{Das dürfte so zu lesen sein, dass das
                     nicht überlieferte Original von Gerty von
                        Hofmannsthal\pwindex{Hofmannsthal, Gertrude von 16.03.1880 – 09.11.1959@\textsc{Hofmannsthal, Gertrude von} (16.03.1880 – 09.11.1959)|pwk} geschrieben worden war.}}}\label{K_L01796-1}.\pend
           \selectlanguage{ngerman}\vspace{1em}{\vspace{1\baselineskip}}
\pstart
           \raggedleft{}{\pb}{[}hs. :{]} Wien\oindex{Wien@\textbf{Wien}, \emph{A.ADM2}|pw}, 29. Okt. 1908.\pend
           
\pstart{}\textsc{Sehr geehrter Herr!}\pend\vspace{0.5em}
\pstart
           Nehmen Sie es mir, bitte, nicht übel, wenn ich Sie mit einem Anliegen beläſtige, das
               Ihnen etwas ſonderbar erſcheinen mag.\pend
           
\pstart
           Sie ſind, ſoviel ich weiß, mit Hr. D\textsuperscript{r}{ }\textsc{Schnitzler} befreundet, den ich leider perſönlich nicht
               kenne. Wenigſtens habe ich Sie ſeinerzeit in Geſellſchaft des Hr. \textsc{Schnitzler} in \textsc{St. Gilgen}\oindex{St. Gilgen@\textbf{St. Gilgen}, \emph{A.ADM3}|pw} geſehen.\pend
           
\pstart
           Nun ſoll demnächſt im Deutſchen Volkstheater\orgindex{Volkstheater@Volkstheater|pw}{ }\textsc{Schnitzler}’s »\textsc{Liebelei}\pwindex{Liebelei. Schauspiel in drei Akten@\emph{Liebelei. Schauspiel in drei Akten}|pw}« zur Aufführung ge{\pb}langen,
               ſobald nur erſt die Beſetzung der Rolle der »\textsc{\uline{Mizi Schlager}}\pwindex{Liebelei. Schauspiel in drei Akten@\emph{Liebelei. Schauspiel in drei Akten}|pwv}« feſtgeſetzt. Und hier iſt der Punkt, wo ich Ihre gütige Intervention in
               Anſpruch nehmen will.\pend
           
\pstart
           Für dieſe Rolle war nämlich urſprünglich ein Frl. Thekla \textsc{Braun}\pwindex{Braun, Thekla Maria 30.5.1883 – 16. 2. 1965@\textsc{Braun, Thekla Maria} (30.5.1883 – 16. 2. 1965), \emph{Schauspieler/Schauspielerin}|pw} in Ausſicht geno{\geminationm}en, die erſt ſeit Beginn dieſer
                  \textsc{Saison} dem Volkstheater\orgindex{Volkstheater@Volkstheater|pw} angehört. Frl. \textsc{Braun}\pwindex{Braun, Thekla Maria 30.5.1883 – 16. 2. 1965@\textsc{Braun, Thekla Maria} (30.5.1883 – 16. 2. 1965), \emph{Schauspieler/Schauspielerin}|pw} war früher beim Opernballet\orgindex{Opernballett@Opernballett|pw}, dann zwei
               Jahre in Graz\oindex{Graz@\textbf{Graz}, \emph{A.ADM2}|pw} als Schauſpielerin – und hier eben
               ſah ſie Dir. \textsc{Weisse}\pwindex{Weisse, Adolf 04.04.1855 – 17.07.1933@\textsc{Weisse, Adolf} (04.04.1855 – 17.07.1933), \emph{Theaterleiter/Theaterleiterin, Schauspieler/Schauspielerin}|pw} in der Rolle der »\textsc{Schlager Mizi}\pwindex{Liebelei. Schauspiel in drei Akten@\emph{Liebelei. Schauspiel in drei Akten}|pwv}« u. engagierte ſie vom Fleck weg fürs Deutſche
                  Volkstheater\orgindex{Volkstheater@Volkstheater|pw}. Er verſicherte ſie, daſs er die »\textsc{Liebelei}\pwindex{Liebelei. Schauspiel in drei Akten@\emph{Liebelei. Schauspiel in drei Akten}|pw}« fürs Volkstheater\orgindex{Volkstheater@Volkstheater|pw}{ }{\pb}mit Hilfe des Autors – das Stück
               gehörte dem Burgtheater\orgindex{Burgtheater@Burgtheater|pw} – freimachen werde, denn
               er könne das Stück speziell in der Rolle der »\textsc{Schlager}\pwindex{Liebelei. Schauspiel in drei Akten@\emph{Liebelei. Schauspiel in drei Akten}|pwv}« beſſer beſetzen als Dir. \textsc{Schlenther}\pwindex{Schlenther, Paul 20.08.1854 – 30.04.1916@\textsc{Schlenther, Paul} (20.08.1854 – 30.04.1916), \emph{Schriftsteller/Schriftstellerin, Kritiker/Kritikerin, Theaterleiter/Theaterleiterin}|pw} u. dgl. m. Da Frl. \textsc{Braun}\pwindex{Braun, Thekla Maria 30.5.1883 – 16. 2. 1965@\textsc{Braun, Thekla Maria} (30.5.1883 – 16. 2. 1965), \emph{Schauspieler/Schauspielerin}|pw}, die ich ſeit 10 Jahren kenne – ſie war damals ein 15jähriger Backfiſch u. kam
               in die Tanzſtunden zu \textsc{Hassreiter}\pwindex{Hassreiter, Josef 30.12.1845 – 08.02.1940@\textsc{Hassreiter, Josef} (30.12.1845 – 08.02.1940), \emph{Ballettmeister/Ballettmeisterin}|pw}, die ich alter Esel beſuchte – auf meinen Rat das Engagement am Volkstheater\orgindex{Volkstheater@Volkstheater|pw} angeno{\geminationm}en hat,
               obwohl ſie verlockendere Anträge anderer W\textsuperscript{r}\oindex{Wien@\textbf{Wien}, \emph{A.ADM2}|pw} Bühnen beſaß, ſo bin ein bischen engagiert in dieſer Sache u. möchte \substVorne{}\textsuperscript{ihr}\substDazwischen{}ſie\substHinten{} nun in ihrer Leidenbahn {\pb}– das war nämlich bis nun ihr \textsc{Engagement} – nicht ganz im
               Stiche laſſen. Frl. \textsc{Braun}\pwindex{Braun, Thekla Maria 30.5.1883 – 16. 2. 1965@\textsc{Braun, Thekla Maria} (30.5.1883 – 16. 2. 1965), \emph{Schauspieler/Schauspielerin}|pw}, die für erſte Rollen mit einer \textsc{Anfangsgage} von
               5000 K engagiert worden war, kam vorläufig zu keiner einzigen. Meist ſtand ihr Frau
                  \textsc{Glöckner}\pwindex{Gloeckner, Josefine 17.01.1874 – 21.02.1954@\textsc{Glöckner, Josefine} (17.01.1874 – 21.02.1954), \emph{Schauspieler/Schauspielerin}|pw} im Wege. Nun würde ſie i{\geminationm}er wieder auf die »\textsc{Liebelei}\pwindex{Liebelei. Schauspiel in drei Akten@\emph{Liebelei. Schauspiel in drei Akten}|pw}« vertröſtet, die ja noch in \label{K_L01796-2v}\edtext{dieſem Jahre erſcheinen}{\lemma{\textnormal{\emph{dieſem Jahre erſcheinen}}}\Cendnote{\textnormal{Die Aufführung
                  verzögerte sich bis 5. 1. 1909. Thekla Braun\pwindex{Braun, Thekla Maria 30.5.1883 – 16. 2. 1965@\textsc{Braun, Thekla Maria} (30.5.1883 – 16. 2. 1965), \emph{Schauspieler/Schauspielerin}|pwk}
                  wurde nicht eingesetzt, die zweite weibliche Hauptrolle spielte Charlotte Waldow\pwindex{Waldow, Charlotte 06.02.1888 – 15.12.1945@\textsc{Waldow, Charlotte} (06.02.1888 – 15.12.1945), \emph{Schauspieler/Schauspielerin}|pwk}.}}}\label{K_L01796-2}, und in der ſie »ſich machen
               werde.« Siehe da – die »\textsc{Liebelei}\pwindex{Liebelei. Schauspiel in drei Akten@\emph{Liebelei. Schauspiel in drei Akten}|pw}« kam, aber Frl. Braun\pwindex{Braun, Thekla Maria 30.5.1883 – 16. 2. 1965@\textsc{Braun, Thekla Maria} (30.5.1883 – 16. 2. 1965), \emph{Schauspieler/Schauspielerin}|pw} ſoll die \uline{Rolle nicht ſpielen}. \uline{Wer} ſie ſpielen wird, ſteht allerdings noch nicht feſt, u. es ſcheint die
               Beſetzung einige Schwierigkeiten {\pb}zu machen, ſofern man der nageliegendſten, der mit Frl. \textsc{Braun\pwindex{Braun, Thekla Maria 30.5.1883 – 16. 2. 1965@\textsc{Braun, Thekla Maria} (30.5.1883 – 16. 2. 1965), \emph{Schauspieler/Schauspielerin}|pw}} gefliſſentlich aus dem Wege geht. Frl. \textsc{Braun}\pwindex{Braun, Thekla Maria 30.5.1883 – 16. 2. 1965@\textsc{Braun, Thekla Maria} (30.5.1883 – 16. 2. 1965), \emph{Schauspieler/Schauspielerin}|pw} hat daher an Hr. D\textsuperscript{r}{ }\textsc{Schnitzler} die ſchriftliche Bitte gerichtet, ihr zu
               geſtatten, daſs ſie ihm die Rolle der der »\textsc{Mizi Schlager}\pwindex{Liebelei. Schauspiel in drei Akten@\emph{Liebelei. Schauspiel in drei Akten}|pwv}« vorſpreche, damit ſich der Autor ſelbſt, der gewiſs das eminenteſte Intereſſe
               an einer richtigen Beſetzung hat, ein entſprechendes Urteil über die Fähigkeiten des
               Fräuleins bilden kann.\pend
           
\pstart
           Ich möchte nun meinerſeits an Sie, verehrter Herr, die ergebenſte Bitte richten, das
                  {\pb}Anſuchen des Frl. \textsc{Braun}\pwindex{Braun, Thekla Maria 30.5.1883 – 16. 2. 1965@\textsc{Braun, Thekla Maria} (30.5.1883 – 16. 2. 1965), \emph{Schauspieler/Schauspielerin}|pw} bei Herrn D\textsuperscript{r}{ }\textsc{Schnitzler} auf meine Empfehlung hin zu befürworten. Die
               Direktion hat ja dann noch immer freie Hand, und es iſt wenigſtens alles geſchehen,
               um einem allfälligen Miſsgriff vorzubeugen u. auch ein ſtarkes, ſtrebſsames Talent
               vor unverdienter Kränkung zu ſchützen.\pend
           
\pstart
           Falls Sie dem Fräulein \textsc{Braun}\pwindex{Braun, Thekla Maria 30.5.1883 – 16. 2. 1965@\textsc{Braun, Thekla Maria} (30.5.1883 – 16. 2. 1965), \emph{Schauspieler/Schauspielerin}|pw} geſtatten wollten, Sie zu beſuchen, ſo bitte ich um zeitige Bekanntgabe von Tag
               und Stunde, die Ihnen {\pb}genehm
               wären. Jedesfalls wiederhole ich aber meine Bitte um Befürwortung jenes Erſuchens,
               des Frl. \textsc{Braun}\pwindex{Braun, Thekla Maria 30.5.1883 – 16. 2. 1965@\textsc{Braun, Thekla Maria} (30.5.1883 – 16. 2. 1965), \emph{Schauspieler/Schauspielerin}|pw} an D\textsuperscript{r}\textsc{Schnitzler}
               richtete. –\pend
           
\pstart
           Und zum Schluſſe bitte ich nochmals, mir dieſe langweilige, Sie wohl empflindlich
               ſtörende Epiſtel zu verzeihen – ich komm gewiſs kein zweitesmal!\pend
           
\pstart
           In aufrichtiger Verehrung{\\[\baselineskip]}Ihr{\\[\baselineskip]}\spacefill\mbox{Camillo Müller.}\pend
           \leftskip=0em{}
\pstart
           \noindent{}I. Wipplingerstraſſe 33\oindex{Wipplingerstrasse@\textbf{Wipplingerstraße}, \emph{Straße (K.STR)}|pw}, T. 14048.\pend
           
\pstart
           \label{T_L01796-1v}\edtext{Bitte der gnädigen Frau\pwindex{Hofmannsthal, Gertrude von 16.03.1880 – 09.11.1959@\textsc{Hofmannsthal, Gertrude von} (16.03.1880 – 09.11.1959)|pwv} meine Handküſſe zu übermitteln!
                     W. O.}{\lemma{\textnormal{\emph{Bitte … W. O.}}}\Cendnote{\textnormal{in drei Zeilen seitlich zu
                     Schlussformel, Unterschrift und Adresse}}}\label{T_L01796-1}\pend
           \selectlanguage{ngerman}\endnumbering\briefempfaengerindex{Schnitzler, Arthur@\textsc{Schnitzler, Arthur}!zzzHofmannsthal, Hugo von@\emph{von Hugo von Hofmannsthal}!1908-10-311@{31. 10. 1908}|)be}\mylabel{L01796h}  \normalsize

\doendnotes{C}
\bigskip
\vfill

\clearpage

\footnotesize

\lohead{\textsc{register}}

% Definiere theindex-Environment komplett neu ohne reledmac
\makeatletter
\renewenvironment{theindex}{%
  \section*{\indexname}%
  \setlength{\parindent}{0pt}%
  \setlength{\parskip}{0pt plus 0.3pt}%
  \let\item\@idxitem
}{%
  \clearpage
}
\makeatother

\IfFileExists{\jobname-pw.ind}{\input{\jobname-pw.ind}}{}

\end{document}

      