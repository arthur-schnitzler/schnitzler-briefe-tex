%% latex-leseansicht-vorspann.tex
%% Vorspann für die Leseansicht.
%% Lädt die gemeinsame Datei latex-vorspann.tex mit nicht gesetztem Schalter.

\newif\ifkorrekturansicht
\korrekturansichtfalse

\input{../tex-inputs/latex-vorspann}


\section[Hugo von Hofmannsthal an Arthur Schnitzler, {[}13. 6. 1914{]}]{L02182 Hugo von Hofmannsthal an Arthur Schnitzler, {[}13. 6. 1914{]}}
\nopagebreak\mylabel{L02182v}
\rehead{ }\normalsize\beginnumbering\briefempfaengerindex{Schnitzler, Arthur@\textsc{Schnitzler, Arthur}!zzzHofmannsthal, Hugo von@\emph{von Hugo von Hofmannsthal}!1914-06-131@{{[}13. 6. 1914{]}}|(be}
\toendnotes[C]{\smallbreak\pagebreak[2]}
\correspDesc{Versand  durch Hugo von Hofmannsthal am [13. 6. 1914] in Rodaun
\newline{}Erhalt  durch Arthur Schnitzler im Zeitraum [14. 6. 1914
                  – 18. 6. 1914?] in Wien}\toendnotes[C]{\smallbreak}
\Standort{CUL, Schnitzler, B 43.}
\physDesc{Brief, 1 Blatt, 4 Seiten, 1503 Zeichen
\newline{}Handschrift: schwarze Tinte, deutsche Kurrent
\newline{}Schnitzler: mit Bleistift datiert: »Juni 914« und beschriftet: »\textsc{Hugo}« 
\newline{}Ordnung: 1) mit Bleistift von unbekannter Hand nummeriert: »\strikeout{337}«  2) mit Bleistift von unbekannter Hand nummeriert:
                                    »350«}
\buchAbdrucke{\weitereDrucke{Hugo von Hofmannsthal, Arthur Schnitzler: \emph{Briefwechsel}. Herausgegeben von Therese Nickl und Heinrich Schnitzler. Frankfurt am Main: \emph{S. Fischer} 1964, S. 275.} }\toendnotes[C]{\smallbreak}
\pstart
           \raggedleft{}{\pb}Rodaun\oindex{Wien@\textbf{Wien}!XXIII., Liesing@\textbf{XXIII., Liesing}!Rodaun@\textbf{Rodaun}, \emph{Region}|pw}, Samstag\pend
           
\pstart{}mein lieber Arthur\pend\vspace{0.5em}
\pstart
           ich höre, Ihr{ }ſeid von Eurer großen \label{K_L02182-1v}\edtext{Reiſe}{\lemma{\textnormal{\emph{Reise}}}\Cendnote{\textnormal{Sie waren vom 1. 5. 1914 bis zum
                     7. 6. 1914
                  unterwegs, die meiste Zeit mit dem Schiff von Italien\oindex{Italien@\textbf{Italien}|pwk} in die Niederlande\oindex{Niederlande@\textbf{Niederlande}|pwk}.}}}\label{K_L02182-1}
               wohlbehalten zurück, und wir haben den herzlichen Wunſch Euch zu{ }ſehen!\pend
           
\pstart
           Ich war indeſſen \label{K_L02182-2v}\edtext{in Paris\oindex{Paris@\textbf{Paris}, \emph{Hauptstadt}|pw}}{\lemma{\textnormal{\emph{in Paris}}}\Cendnote{\textnormal{von 9. 5. 1914 bis zum
                     20. 5. 1914, wobei die Heimkehr erst am 30. 5. 1914
                  stattfand}}}\label{K_L02182-2}, hatte dort recht trübe niedergeſchlagene Tage (von innen heraus,
               und in{ }ſolchen Zeiten iſt mir eine große fremde Stadt nicht günſtig), traf dann
               meinen Vater\pwindex{Hofmannsthal, Hugo August von 21.\,12.\,1841 Wien – 8.\,12.\,1915 ebd.@\textsc{Hofmannsthal, Hugo August von} (21.\,12.\,1841 Wien – 8.\,12.\,1915 ebd.), \emph{Bankdirektor}|pwv} in Frankfurt\oindex{Frankfurt am Main@\textbf{Frankfurt am Main}, \emph{Hauptstadt}|pw}, brachte ihn nach Nauheim\oindex{Bad Nauheim@\textbf{Bad Nauheim}, \emph{Region}|pw}, wo die Cur ihm, wie es{ }ſcheint, recht wohl tut. – Wie
                  {\pb}könnten wir uns{ }ſehen, Arthur?
               Wir{ }ſind{ }ſicher noch die ganze Woche da \label{K_L02182-3v}\edtext{bis zum 22\textsuperscript{ten} etwa}{\lemma{\textnormal{\emph{bis … etwa}}}\Cendnote{\textnormal{Erst eine Woche danach
                  übersiedelten sie nach Aussee\oindex{Bad Aussee@\textbf{Bad Aussee}, \emph{Hauptstadt}|pwk}.}}}\label{K_L02182-3}.\hspace*{1.5em}Wir haben aber keine Möglichkeit des Übernachtens mehr
               in der Stadt.\hspace*{1.5em}Wenn Ihr wie neulich die Bären\pwindex{Beer-Hofmann, Richard 11.\,7.\,1866 Wien – 26.\,9.\,1945 New York City@\textsc{Beer-Hofmann, Richard} (11.\,7.\,1866 Wien – 26.\,9.\,1945 New York City), \emph{Schriftsteller}|pw}\pwindex{Beer-Hofmann, Paula 25.\,2.\,1879 Wien – 30.\,10.\,1939 Zürich@\textsc{Beer-Hofmann, Paula} (25.\,2.\,1879 Wien – 30.\,10.\,1939 Zürich)|pw}, zu einem gemeinſamen Nachtmahl
               nach Hietzing\oindex{XIII., Hietzing@\textbf{XIII., Hietzing}, \emph{Verwaltungsgebiet}|pw} kämet – und etwa{ }ſchon um
                  7 oder{ }ſo dort wäret, \textsc{rendezvous}{ }\uline{vor} dem Parkhôtel\oindex{Wien@\textbf{Wien}!XIII., Hietzing@\textbf{XIII., Hietzing}!Parkhotel Schönbrunn@\textbf{Parkhotel Schönbrunn}, \emph{Hotel}|pw}, daſs man {\pb}vorher
               eine Stunde miteinander im Schönbrunner Park\oindex{Wien@\textbf{Wien}!XIII., Hietzing@\textbf{XIII., Hietzing}!Schloss Schönbrunn@\textbf{Schloss Schönbrunn}, \emph{Schloss}|pw}
               herumginge oder{ }ſäße – das wäre{ }ſehr{ }ſchön. Schreiben Sie eine Zeile, jeder Tag wird
               uns recht{ }ſein.\pend
           
\pstart
           Noch eines, da Sie ja \uline{mein} eigentlicher Hausarzt{ }ſind. In der (irrigen) Idee von etwas Gicht ließ ich eine Analyſe machen;{ }ſie ergab
               nichts Pathologiſches, nur: Traubenzucker, \uuline{nur} in
               Spuren, {\pb}\uline{quantitativ nicht nachweisbar}. Mein hieſiger Landarzt\pwindex{Wimmer, Maximilian 19.\,2.\,1867 Wien – 1.\,6.\,1951 ebd.@\textsc{Wimmer, Maximilian} (19.\,2.\,1867 Wien – 1.\,6.\,1951 ebd.), \emph{Mediziner}|pwv}, der recht
               geſcheidt, nur etwas{ }ſummariſch iſt,{ }ſagt, das käme bei vielen Leuten vor, habe gar
               nichts auf{ }ſich, bedeute durchaus nicht einen Anfang oder eine Andeutung dieſer
                  Krankheit.\hspace*{1.5em}Iſt das richtig?\pend
           
\pstart
           Von Herzen Ihr{\\[\baselineskip]}\spacefill\mbox{Hugo.}\pend
           \leftskip=0em{}
\pstart
           \noindent{}\textsc{PS}. Meine oben gemeldete Niedergeſchlagenheit hat nichts
                  mit Hypochondrien zu tun, die mich durchaus nicht beſchäftigen; obige Analyſe kam
                  mir erst geſtern vor Augen.\pend
           \selectlanguage{ngerman}\endnumbering\briefempfaengerindex{Schnitzler, Arthur@\textsc{Schnitzler, Arthur}!zzzHofmannsthal, Hugo von@\emph{von Hugo von Hofmannsthal}!1914-06-131@{{[}13. 6. 1914{]}}|)be}\mylabel{L02182h}  \newcommand{\dateiname}{L02182}\newcommand{\titel}{Hugo von Hofmannsthal an Arthur Schnitzler, [13. 6. 1914]}\newcommand{\editorInnen}{Martin Anton Müller und Gerd-Hermann Susen}%% latex-leseansicht-abspann.tex
%% Abspann für die Leseansicht.
%% Der Schalter \ifkorrekturansicht ist bereits durch den Vorspann gesetzt.

%% latex-abspann.tex
%% Gemeinsamer Abspann für Korrekturansicht und Leseansicht.
%% Setzt den Schalter \ifkorrekturansicht voraus (gesetzt in den
%% einbindenden Dateien latex-korrekturansicht-abspann.tex bzw.
%% latex-leseansicht-abspann.tex).
%% ---------------------------------------------------------------

\normalsize

% Das esempio-Environment wird nur in der Leseansicht benötigt
\ifkorrekturansicht\else
\newenvironment{esempio}[3]%
{
    \vspace{1.5ex}
    \rlap{\underline{#1}}
    \par
    \setlength{\parindent}{0cm}
    \nopagebreak
    \leftskip=#2cm
    \rightskip=#3cm
}
{
    \par
}
\fi

\doendnotes{C}
\bigskip
\vfill

\clearpage

\footnotesize

\ifkorrekturansicht
  \lohead{\textsc{register}}
\fi

% theindex-Environment neu definieren ohne reledmac
\makeatletter
\renewenvironment{theindex}{%
  \ifkorrekturansicht
    \section*{\indexname}%
  \else
    \subsubsection*{Index der erwähnten Entitäten}%
  \fi
  \setlength{\parindent}{0pt}%
  \setlength{\parskip}{0pt plus 0.3pt}%
  \let\item\@idxitem
}{%
  \ifkorrekturansicht\clearpage\fi
}
\makeatother

\IfFileExists{\jobname-pw.ind}{\input{\jobname-pw.ind}}{}

% Quellenangabe nur in der Leseansicht
\ifkorrekturansicht\else
% Fallback-Definitionen, falls die .tex-Datei \titel etc. nicht gesetzt hat
\providecommand{\titel}{}
\providecommand{\editorInnen}{}
\providecommand{\dateiname}{\jobname}

\vspace{3cm}

\vfill

\footnotesize
\textsc{Quelle}: \titel. Herausgegeben von {\editorInnen}. In: \emph{Arthur Schnitzler: Briefwechsel mit Autorinnen und Autoren}.
 Digitale Edition, https://schnitzler-briefe.acdh.oeaw.ac.at/{\dateiname}.html (Stand \today)
\fi

\end{document}


