%% latex-leseansicht-vorspann.tex
%% Vorspann für die Leseansicht.
%% Lädt die gemeinsame Datei latex-vorspann.tex mit nicht gesetztem Schalter.

\newif\ifkorrekturansicht
\korrekturansichtfalse

\input{../tex-inputs/latex-vorspann}

\begin{center}
            \textcolor{red}{ENTWURF. ENTZIFFERUNG NOCH NICHT KORREKTURGELESEN}
                      \end{center}
            
               \section[Hermann Bahr an Arthur Schnitzler, 21. 2. 1892]{ Hermann Bahr an Arthur Schnitzler, 21. 2. 1892}\nopagebreak\mylabel{v}\rehead{ }\begin{ledgroupsized}[t]{13cm}\normalsize\beginnumbering\briefempfaengerindex{Schnitzler, Arthur@\textsc{Schnitzler, Arthur}!zzzBahr, Hermann@\emph{von Hermann Bahr}!1892-02-211@{21. 2. 1892}|(be} \toendnotes[C]{\smallbreak\pagebreak[2]} \Standort{CUL, Schnitzler, B 5b.}
\physDesc{Postkarte
\newline{}Handschrift: schwarze Tinte, deutsche Kurrent\newline{}Versand: Stempel: »\nobreak{}\oindex{I., Innere Stadt@\textbf{I., Innere Stadt}|pwk}Wien 1/1, 22{[}.{]} 2. 92, 8–9 V\nobreak{}«.  
\newline{}Schnitzler: mit Bleistift datiert: »22/2 92« \newline{}Ordnung: 1) mit rotem Buntstift von unbekannter Hand nummeriert:
                                       »\strikeout{6}« 2) mit Bleistift von unbekannter Hand nummeriert:
                                 »5«}\buchAbdrucke{\weitereDrucke{Hermann Bahr, Arthur Schnitzler: \emph{Briefwechsel, Aufzeichnungen, Dokumente (1891–1931)}. Hg. Kurt Ifkovits und Martin Anton Müller. Göttingen: \emph{Wallstein} 2018, S. 22.} }\toendnotes[C]{\smallbreak}\pstart{}{\pb}Herrn D\textsuperscript{r} A.
                  Schnitzler \pend{}\pstart{}Kärntnerring 12\oindex{Kaerntnerring@\textbf{Kärntnerring}|pw}\pend{}\pstart{}Wien I\oindex{I., Innere Stadt@\textbf{I., Innere Stadt}|pw}\pend{}{\bigskip}\pstart
           \raggedleft{}\label{K_L00074_1v}\edtext{{\pb}So{\geminationn}tag}{\lemma{\textnormal{\emph{Sotag}}}\Cendnote{\textnormal{geschrieben am 
                     Sonntag, den 21. 2. 1892, aber erst am Folgetag abgeschickt}}}\label{K_L00074_1h}
                  Mittag.\pend
           \pstart{} Lieber Freund!\pend\pstart
           Das \label{K_L00074_2v}\edtext{Mauſerl}{\lemma{\textnormal{\emph{Mauſerl}}}\Cendnote{\textnormal{Ilka Pálmay\pwindex{Pálmay, Ilka 1859-09-21 – 1944-02-17@\textsc{Pálmay, Ilka} (1859-09-21 – 1944-02-17), \emph{Schriftstellerin, Schauspielerin, Sängerin}|pwk}}}}\label{K_L00074_2h}\pwindex{Pálmay, Ilka 1859-09-21 – 1944-02-17@\textsc{Pálmay, Ilka} (1859-09-21 – 1944-02-17), \emph{Schriftstellerin, Schauspielerin, Sängerin}|pwv} will nicht, abſolut nicht. Alles mögliche Schöne u Gute könnte man von ihr
               haben – nur gerade das eine nicht, was wir brauchen. Sie ſagt übrigens ſehr
               vernünftige Gründe u. i{\geminationn}erlich muß ich ihr Recht
               geben.\pend
           \pstart herzlichſt\spacefill\mbox{Bahr}\pend{}\endnumbering\briefempfaengerindex{Schnitzler, Arthur@\textsc{Schnitzler, Arthur}!zzzBahr, Hermann@\emph{von Hermann Bahr}!1892-02-211@{21. 2. 1892}|)be}\mylabel{h}\end{ledgroupsized}  \newcommand{\dateiname}{L00074}\newcommand{\titel}{Hermann Bahr an Arthur Schnitzler, 21. 2. 1892}\newcommand{\editorInnen}{ Kurt Ifkovits,  Martin Anton Müller}%% latex-leseansicht-abspann.tex
%% Abspann für die Leseansicht.
%% Der Schalter \ifkorrekturansicht ist bereits durch den Vorspann gesetzt.

%% latex-abspann.tex
%% Gemeinsamer Abspann für Korrekturansicht und Leseansicht.
%% Setzt den Schalter \ifkorrekturansicht voraus (gesetzt in den
%% einbindenden Dateien latex-korrekturansicht-abspann.tex bzw.
%% latex-leseansicht-abspann.tex).
%% ---------------------------------------------------------------

\normalsize

% Das esempio-Environment wird nur in der Leseansicht benötigt
\ifkorrekturansicht\else
\newenvironment{esempio}[3]%
{
    \vspace{1.5ex}
    \rlap{\underline{#1}}
    \par
    \setlength{\parindent}{0cm}
    \nopagebreak
    \leftskip=#2cm
    \rightskip=#3cm
}
{
    \par
}
\fi

\doendnotes{C}
\bigskip
\vfill

\clearpage

\footnotesize

\ifkorrekturansicht
  \lohead{\textsc{register}}
\fi

% theindex-Environment neu definieren ohne reledmac
\makeatletter
\renewenvironment{theindex}{%
  \ifkorrekturansicht
    \section*{\indexname}%
  \else
    \subsubsection*{Index der erwähnten Entitäten}%
  \fi
  \setlength{\parindent}{0pt}%
  \setlength{\parskip}{0pt plus 0.3pt}%
  \let\item\@idxitem
}{%
  \ifkorrekturansicht\clearpage\fi
}
\makeatother

\IfFileExists{\jobname-pw.ind}{\input{\jobname-pw.ind}}{}

% Quellenangabe nur in der Leseansicht
\ifkorrekturansicht\else
% Fallback-Definitionen, falls die .tex-Datei \titel etc. nicht gesetzt hat
\providecommand{\titel}{}
\providecommand{\editorInnen}{}
\providecommand{\dateiname}{\jobname}

\vspace{3cm}

\vfill

\footnotesize
\textsc{Quelle}: \titel. Herausgegeben von {\editorInnen}. In: \emph{Arthur Schnitzler: Briefwechsel mit Autorinnen und Autoren}.
 Digitale Edition, https://schnitzler-briefe.acdh.oeaw.ac.at/{\dateiname}.html (Stand \today)
\fi

\end{document}


      