%% latex-korrekturansicht-vorspann.tex
%% Vorspann für die Korrekturansicht.
%% Lädt die gemeinsame Datei latex-vorspann.tex mit gesetztem Schalter.

\newif\ifkorrekturansicht
\korrekturansichttrue

\input{../tex-inputs/latex-vorspann}


\section[Sigmund Freud an Arthur Schnitzler, {[}nach dem 6. 5. 1931{]}]{L03814 Sigmund Freud an Arthur Schnitzler, {[}nach dem 6. 5. 1931{]}}
\nopagebreak\mylabel{L03814v}
\rehead{ }\normalsize\beginnumbering\briefempfaengerindex{Schnitzler, Arthur@\textsc{Schnitzler, Arthur}!zzzFreud, Sigmund@\emph{von Sigmund Freud}!1931-05-311@{{[}nach dem 6. 5. 1931{]}}|(be}
\toendnotes[C]{\smallbreak\pagebreak[2]}\Standort{CUL, Schnitzler, B 31.}
\physDesc{Kartenbrief, 1 Blatt, 1 Seite, 191 Zeichen
\newline{}Handschrift: blaue Tinte, deutsche Kurrent}\toendnotes[C]{\smallbreak}
\pstart
           \noindent{}{\pb}\textcolor{gray}{\textbf{Dank für \label{K_L03820-1v}\edtext{Ihre freundliche Anteilnahme}{\lemma{\textnormal{\emph{Ihre … Anteilnahme}}}\Cendnote{\textnormal{Ein Geburtstagsgruß von Schnitzler ist nicht überliefert.}}}\label{K_L03820-1} an meinem \label{K_L03820-2v}\edtext{75. Geburtstag}{\lemma{\textnormal{\emph{75. Geburtstag}}}\Cendnote{\textnormal{Freud war am 6. 5. 1931 75 Jahre alt geworden.}}}\label{K_L03820-2}.}}\pend
           
\pstart
           \textcolor{gray}{\textbf{Wien\oindex{Wien@\textbf{Wien}, \emph{A.ADM2}|pw}, Mai 1931.}}\pend
           \pstart \spacefill\mbox{Ihr Freud}\pend{}
\pstart
           \noindent{}Verehrteſter. Geſtatten Sie mir,
      es ſchon heute vorwegzunehmen,
      wenn ich nächſtes Jahr nicht in
      der Lage ſein ſollte, Ihnen zum
      Schritt über die Altersgrenze
      Glück zu wünſchen.
      \pend
           
\pstart
           Herzlich \spacefill\mbox{Fr.}\pend
           \selectlanguage{ngerman}\endnumbering\briefempfaengerindex{Schnitzler, Arthur@\textsc{Schnitzler, Arthur}!zzzFreud, Sigmund@\emph{von Sigmund Freud}!1931-05-071@{{[}nach dem 6. 5. 1931{]}}|)be}\mylabel{L03814h}
\begin{anhang}
\end{anhang}\normalsize

\doendnotes{C}
\bigskip
\vfill

\clearpage

\footnotesize

\lohead{\textsc{register}}

% Definiere theindex-Environment komplett neu ohne reledmac
\makeatletter
\renewenvironment{theindex}{%
  \section*{\indexname}%
  \setlength{\parindent}{0pt}%
  \setlength{\parskip}{0pt plus 0.3pt}%
  \let\item\@idxitem
}{%
  \clearpage
}
\makeatother

\IfFileExists{\jobname-pw.ind}{\input{\jobname-pw.ind}}{}

\end{document}

      