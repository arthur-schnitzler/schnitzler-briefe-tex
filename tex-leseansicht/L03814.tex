%% latex-leseansicht-vorspann.tex
%% Vorspann für die Leseansicht.
%% Lädt die gemeinsame Datei latex-vorspann.tex mit nicht gesetztem Schalter.

\newif\ifkorrekturansicht
\korrekturansichtfalse

\input{../tex-inputs/latex-vorspann}


\section[Sigmund Freud an Arthur Schnitzler, {[}zwischen 7. 5. 1931 und 31. 5. 1931{]}]{L03814 Sigmund Freud an Arthur Schnitzler, {[}zwischen 7. 5. 1931 und 31. 5. 1931{]}}
\nopagebreak\mylabel{L03814v}
\rehead{ }\normalsize\beginnumbering\briefempfaengerindex{Schnitzler, Arthur@\textsc{Schnitzler, Arthur}!zzzFreud, Sigmund@\emph{von Sigmund Freud}!1931-05-311@{{[}zwischen 7. 5. 1931 und 31. 5. 1931{]}}|(be}
\toendnotes[C]{\smallbreak\pagebreak[2]}
\correspDesc{Versand  durch Sigmund Freud im Zeitraum [zwischen 7. 5. 1931 und 31. 5. 1931] in Wien
\newline{}Erhalt  durch Arthur Schnitzler im Zeitraum [zwischen 7. 5. 1931 und 4. 6. 1931?] in Wien}\toendnotes[C]{\smallbreak}
\Standort{CUL, Schnitzler, B 31.}
\physDesc{Karte, 190 Zeichen
\newline{}Handschrift: blaue Tinte, deutsche Kurrent}
\buchAbdrucke{\weitereDrucke{Sigmund Freud: \emph{Briefe an Arthur Schnitzler.}Herausgegeben von Henry Schnitzler In: \emph{Neue deutsche Rundschau}, Jg. 66 (Januar 1955) Nr. 1, S. 100.} }\toendnotes[C]{\smallbreak}
\pstart
           \noindent{}{\pb}\textcolor{gray}{\textbf{Dank für \label{K_L03814-1v}\edtext{Ihre
                  freundliche Anteilnahme}{\lemma{\textnormal{\emph{Ihre … Anteilnahme}}}\Cendnote{\textnormal{Ein
                     Geburtstagsgruß von Schnitzler ist nicht
                     überliefert.}}}\label{K_L03814-1} an meinem \label{K_L03814-2v}\edtext{75. Geburtstag}{\lemma{\textnormal{\emph{75. Geburtstag}}}\Cendnote{\textnormal{Freud war am
                        6. 5. 1931 75 Jahre alt geworden.}}}\label{K_L03814-2}.}}\pend
           
\pstart
           \textcolor{gray}{\textbf{Wien\oindex{Wien@\textbf{Wien}, \emph{Verwaltungsgebiet}|pw}, Mai 1931.}}\pend
           \pstart \spacefill\mbox{Ihr Freud}\pend{}
\pstart
           \noindent{}Verehrteſter. Geſtatten Sie mir, es ſchon heute vorwegzunehmen, wenn
                  ich nächſtes Jahr nicht in der Lage ſein ſollte, Ihnen zum Schritt
                  über die Altersgrenze Glück zu wünſchen.\pend
           
\pstart
           Herzlich \spacefill\mbox{Fr.}\pend
           \selectlanguage{ngerman}\endnumbering\briefempfaengerindex{Schnitzler, Arthur@\textsc{Schnitzler, Arthur}!zzzFreud, Sigmund@\emph{von Sigmund Freud}!1931-05-071@{{[}zwischen 7. 5. 1931 und 31. 5. 1931{]}}|)be}\mylabel{L03814h}
\begin{anhang}
\end{anhang}\newcommand{\dateiname}{L03814}\newcommand{\titel}{Sigmund Freud an Arthur Schnitzler, [zwischen 7. 5. 1931 und 31. 5. 1931]}\newcommand{\editorInnen}{Selma Jahnke und Martin Anton Müller}%% latex-leseansicht-abspann.tex
%% Abspann für die Leseansicht.
%% Der Schalter \ifkorrekturansicht ist bereits durch den Vorspann gesetzt.

%% latex-abspann.tex
%% Gemeinsamer Abspann für Korrekturansicht und Leseansicht.
%% Setzt den Schalter \ifkorrekturansicht voraus (gesetzt in den
%% einbindenden Dateien latex-korrekturansicht-abspann.tex bzw.
%% latex-leseansicht-abspann.tex).
%% ---------------------------------------------------------------

\normalsize

% Das esempio-Environment wird nur in der Leseansicht benötigt
\ifkorrekturansicht\else
\newenvironment{esempio}[3]%
{
    \vspace{1.5ex}
    \rlap{\underline{#1}}
    \par
    \setlength{\parindent}{0cm}
    \nopagebreak
    \leftskip=#2cm
    \rightskip=#3cm
}
{
    \par
}
\fi

\doendnotes{C}
\bigskip
\vfill

\clearpage

\footnotesize

\ifkorrekturansicht
  \lohead{\textsc{register}}
\fi

% theindex-Environment neu definieren ohne reledmac
\makeatletter
\renewenvironment{theindex}{%
  \ifkorrekturansicht
    \section*{\indexname}%
  \else
    \subsubsection*{Index der erwähnten Entitäten}%
  \fi
  \setlength{\parindent}{0pt}%
  \setlength{\parskip}{0pt plus 0.3pt}%
  \let\item\@idxitem
}{%
  \ifkorrekturansicht\clearpage\fi
}
\makeatother

\IfFileExists{\jobname-pw.ind}{\input{\jobname-pw.ind}}{}

% Quellenangabe nur in der Leseansicht
\ifkorrekturansicht\else
% Fallback-Definitionen, falls die .tex-Datei \titel etc. nicht gesetzt hat
\providecommand{\titel}{}
\providecommand{\editorInnen}{}
\providecommand{\dateiname}{\jobname}

\vspace{3cm}

\vfill

\footnotesize
\textsc{Quelle}: \titel. Herausgegeben von {\editorInnen}. In: \emph{Arthur Schnitzler: Briefwechsel mit Autorinnen und Autoren}.
 Digitale Edition, https://schnitzler-briefe.acdh.oeaw.ac.at/{\dateiname}.html (Stand \today)
\fi

\end{document}


