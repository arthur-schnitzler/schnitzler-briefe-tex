\input{../tex-inputs/latex-pdf-vorspann}
\begin{center}
            \textcolor{red}{ENTWURF. ENTZIFFERUNG NOCH NICHT KORREKTURGELESEN}
                      \end{center}
            
               \section[Arthur Schnitzler an Richard Beer-Hofmann, 18. 2. 1908]{ Arthur Schnitzler an Richard Beer-Hofmann, 18. 2. 1908}\nopagebreak\mylabel{v}\rehead{ }\begin{ledgroupsized}[t]{13cm}\normalsize\beginnumbering\briefempfaengerindex{Beer-Hofmann, Richard@\textsc{Beer-Hofmann, Richard}!zzzSchnitzler, Arthur@\emph{von Arthur Schnitzler}!1908-02-181@{18. 2. 1908}|(be} \toendnotes[C]{\smallbreak\pagebreak[2]} \Standort{YCGL, MSS 31.}
\physDesc{Bildpostkarte
\newline{}Handschrift: Bleistift, deutsche Kurrent\newline{}Versand: Stempel: »\nobreak{}\oindex{Semmering@\textbf{Semmering}|pwk}Semmering, 18. II. \textcolor{gray}{08}, 3\nobreak{}«.  }\buchAbdrucke{\weitereDrucke{Arthur Schnitzler, Richard Beer-Hofmann: \emph{Briefwechsel 1891–1931}. Hg. Konstanze Fliedl. Wien, Zürich: \emph{Europaverlag} 1992, S. 189.} }\toendnotes[C]{\smallbreak}\pstart{}{\pb}\textsc{Dr Richard Beer-Hofma{\geminationn}}\pend{}\pstart{}\textsc{Wien XVIII}\oindex{XVIII., Waehring@\textbf{XVIII., Währing}|pw}\pend{}\pstart{}\textsc{Hasenauerstr 59}\oindex{Hasenauerstrasse@\textbf{Hasenauerstraße}|pw}.\pend{}{\bigskip}\pstart
           \noindent{}\centering{}\textcolor{gray}{\textbf{{\pb}Semmering\oindex{Semmering@\textbf{Semmering}|pw}. Südbahnhotel\oindex{Suedbahnhotel@\textbf{Südbahnhotel}|pw}.}}\pend
           \pstart
           \raggedleft{}{\pb}18. 2.\pend
           \pstart
           lieber Richard, wir wollen \label{K_L01761_1v}\edtext{Donnerſtag}{\lemma{\textnormal{\emph{Donnerſtag}}}\Cendnote{\textnormal{Es verzögerte sich. Vgl. A. S.: \emph{Tagebuch}, 22. 2. 1908}}}\label{K_L01761_1h} in Wien\oindex{Wien@\textbf{Wien}|pw} ſein. Schade dſs Sie nicht
                  heraufgeko{\geminationm}en ſind. Ich bitte Sie dringend, leſen Sie
               nun nicht am Ende den »Weg\pwindex{Schnitzler, Arthur 15.05.1862 – 21.10.1931@\textsc{Schnitzler, Arthur} (15.05.1862 – 21.10.1931), \emph{Schriftsteller, Mediziner}!Weg ins Freie. Roman1.1.1908 – 1.6.1908@\strich\emph{Der Weg ins Freie. Roman} {[}1.1.1908 – 1.6.1908{]}|pw}« in \label{K_L01761_2v}\edtext{Fortſetzungen\pwindex{neue Rundschau1904@\emph{Die neue Rundschau}|pwv}}{\lemma{\textnormal{\emph{Fortſetzungen}}}\Cendnote{\textnormal{Der Roman erschien in sechs Teilen von
                     Januar bis Juni 1908 in der \emph{Neuen Rundschau}\pwindex{neue Rundschau1904@\emph{Die neue Rundschau}|pwk}.}}}\label{K_L01761_2h} weiter, ſondern warten auf das
               Buch.\pend
           \pstart
           Herzlichſt{\\[\baselineskip]}Ihr{\\[\baselineskip]}\spacefill\mbox{A.}\pend
           \leftskip=0em{}\endnumbering\briefempfaengerindex{Beer-Hofmann, Richard@\textsc{Beer-Hofmann, Richard}!zzzSchnitzler, Arthur@\emph{von Arthur Schnitzler}!1908-02-181@{18. 2. 1908}|)be}\mylabel{h}\end{ledgroupsized}  \newcommand{\dateiname}{L01761}\newcommand{\titel}{Arthur Schnitzler an Richard Beer-Hofmann, 18. 2. 1908}\newcommand{\editorInnen}{Martin Anton Müller und Gerd-Hermann Susen}\input{../tex-inputs/latex-pdf-abspann}
      