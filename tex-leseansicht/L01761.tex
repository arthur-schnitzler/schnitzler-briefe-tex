%% latex-korrekturansicht-vorspann.tex
%% Vorspann für die Korrekturansicht.
%% Lädt die gemeinsame Datei latex-vorspann.tex mit gesetztem Schalter.

\newif\ifkorrekturansicht
\korrekturansichttrue

\input{../tex-inputs/latex-vorspann}


\section[Arthur Schnitzler an Richard Beer-Hofmann, 18. 2. 1908]{L01761 Arthur Schnitzler an Richard Beer-Hofmann, 18. 2. 1908}
\nopagebreak\mylabel{L01761v}
\rehead{ }\normalsize\beginnumbering\briefempfaengerindex{Beer-Hofmann, Richard@\textsc{Beer-Hofmann, Richard}!zzzSchnitzler, Arthur@\emph{von Arthur Schnitzler}!1908-02-181@{18. 2. 1908}|(be}
\toendnotes[C]{\smallbreak\pagebreak[2]}\Standort{YCGL, MSS 31.}
\physDesc{Bildpostkarte, 270 Zeichen
\newline{}Handschrift: Bleistift, deutsche Kurrent
\newline{}Versand: Stempel: »\nobreak{}\oindex{Semmering@\textbf{Semmering}, \emph{A.ADM3}|pwk}Semmering, 18. II. \textcolor{gray}{08}, 3\nobreak{}«.  }
\buchAbdrucke{\weitereDrucke{Arthur Schnitzler, Richard Beer-Hofmann: \emph{Briefwechsel 1891–1931}. Wien, Zürich: \emph{Europaverlag} 1992, S. 189.} }\toendnotes[C]{\smallbreak}\pstart{}{\pb}\textsc{Dr Richard Beer-Hofma{\geminationn}}\pend{}\pstart{}\textsc{Wien XVIII}\oindex{XVIII., Waehring@\textbf{XVIII., Währing}, \emph{A.ADM3}|pw}\pend{}\pstart{}\textsc{Hasenauerstr 59}\oindex{Hasenauerstrasse 59@\textbf{Hasenauerstraße 59}, \emph{Wohngebäude (K.WHS)}|pw}.\pend{}{\bigskip}
\pstart
           \noindent{}\centering{}{\pb}\textcolor{gray}{\textbf{Semmering\oindex{Semmering@\textbf{Semmering}, \emph{A.ADM3}|pw}. Südbahnhotel\oindex{Suedbahnhotel [Semmering]@\textbf{Südbahnhotel [Semmering]}, \emph{Hotel (K.HTL)}|pw}.}}\pend
           \vspace{1em}
\pstart
           \raggedleft{}{\pb}18. 2.\pend
           \vspace{0.5em}
\pstart
           lieber Richard, wir wollen \label{K_L01761-1v}\edtext{Donnerſtag}{\lemma{\textnormal{\emph{Donnerſtag}}}\Cendnote{\textnormal{Es verzögerte sich. Vgl. A. S.: \emph{Tagebuch}, 22. 2. 1908.}}}\label{K_L01761-1} in Wien\oindex{Wien@\textbf{Wien}, \emph{A.ADM2}|pw} ſein. Schade dſs Sie nicht
                  heraufgeko{\geminationm}en ſind. Ich bitte Sie dringend, leſen Sie
               nun nicht am Ende den »Weg\pwindex{Weg ins Freie. Roman@\emph{Der Weg ins Freie. Roman}|pw}« in \label{K_L01761-2v}\edtext{Fortſetzungen\pwindex{neue Rundschau@\emph{Die neue Rundschau}|pwv}}{\lemma{\textnormal{\emph{Fortſetzungen}}}\Cendnote{\textnormal{Der Roman erschien in sechs Teilen von
                     Januar bis Juni 1908 in der \emph{Neuen Rundschau}\pwindex{neue Rundschau@\emph{Die neue Rundschau}|pwk}.}}}\label{K_L01761-2} weiter, ſondern warten auf das
               Buch.\pend
           
\pstart
           Herzlichſt{\\[\baselineskip]}Ihr{\\[\baselineskip]}\spacefill\mbox{A.}\pend
           \leftskip=0em{}\selectlanguage{ngerman}\endnumbering\briefempfaengerindex{Beer-Hofmann, Richard@\textsc{Beer-Hofmann, Richard}!zzzSchnitzler, Arthur@\emph{von Arthur Schnitzler}!1908-02-181@{18. 2. 1908}|)be}\mylabel{L01761h}  \normalsize

\doendnotes{C}
\bigskip
\vfill

\clearpage

\footnotesize

\lohead{\textsc{register}}

% Definiere theindex-Environment komplett neu ohne reledmac
\makeatletter
\renewenvironment{theindex}{%
  \section*{\indexname}%
  \setlength{\parindent}{0pt}%
  \setlength{\parskip}{0pt plus 0.3pt}%
  \let\item\@idxitem
}{%
  \clearpage
}
\makeatother

\IfFileExists{\jobname-pw.ind}{\input{\jobname-pw.ind}}{}

\end{document}

      