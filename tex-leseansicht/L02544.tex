%% latex-leseansicht-vorspann.tex
%% Vorspann für die Leseansicht.
%% Lädt die gemeinsame Datei latex-vorspann.tex mit nicht gesetztem Schalter.

\newif\ifkorrekturansicht
\korrekturansichtfalse

\input{../tex-inputs/latex-vorspann}


         
         \renewcommand{\erwaehntePersonen}{Personen: Gertrude von Hofmannsthal, Hugo von Hofmannsthal, Raimund von Hofmannsthal, Franz Schalk, Richard Strauss}
         \renewcommand{\erwaehnteInstitutionen}{Institutionen: Musikverlag Adolph Fürstner}
         \renewcommand{\erwaehnteOrte}{Orte: Wien}
         \renewcommand{\erwaehnteWerke}{}
               \section[Gerty Hofmannsthal an Arthur Schnitzler, {[}5. 3. 1931{]}]{ Gerty Hofmannsthal an Arthur Schnitzler, {[}5. 3. 1931{]}}\nopagebreak\mylabel{v}\rehead{ }\begin{ledgroupsized}[t]{13cm}\normalsize\beginnumbering\briefempfaengerindex{Schnitzler, Arthur@\textsc{Schnitzler, Arthur}!zzzHofmannsthal, Gertrude von@\emph{von Gertrude von Hofmannsthal}!1931-03-051@{{[}5. 3. 1931{]}}|(be} \toendnotes[C]{\smallbreak\pagebreak[2]} \Standort{CUL, Schnitzler, B 43.}
\physDesc{Brief, 1 Blatt, 1 Seite, 1299 Zeichen
\newline{}Schreibmaschine
\newline{}Handschrift: Bleistift (\noindent{}Unterschrift)
\newline{}Schnitzler: mit rotem Buntstift beschriftet »\textsc{Hugo}«, das Datum ergänzt: »\substVorne{}\textsuperscript{5}\substDazwischen{}4\substHinten{}/3 931« und eine Unterstreichung
                                 vorgenommen 
\newline{}Ordnung: von unbekannter Hand nummeriert: »650« }\toendnotes[C]{\smallbreak}\pstart
           \raggedleft{}{\pb}\label{K_L02544-1v}\edtext{Donnerstag}{\lemma{\textnormal{\emph{Donnerstag}}}\Cendnote{\textnormal{Der 5. 3. 1931 war ein
                        Donnerstag, Schnitzler\pwindex{Schnitzler, Arthur 15.05.1862 – 21.10.1931@\textsc{Schnitzler, Arthur} (15.05.1862 – 21.10.1931), \emph{Schriftsteller, Mediziner}|pwk} machte also eine
                        falsche Korrektur an der Datumsangabe.}}}\label{K_L02544-1h}\pend
           \pstart{}Lieber Arthur,\pend\pstart
           Hier zwei Verträge die insofern differieren, als die Summe von Strauss\pwindex{Strauss, Richard 11.06.1864 – 08.09.1949@\textsc{Strauss, Richard} (11.06.1864 – 08.09.1949), \emph{Theaterleiter, Komponist, Dirigent}|pw} an Hugo\pwindex{Hofmannsthal, Hugo von 1874-02-01 – 1929-07-15@\textsc{Hofmannsthal, Hugo von} (1874-02-01 – 1929-07-15), \emph{Schriftsteller}|pw} bei
               Abschluss des Vertrages (die ich übrigens vergass zu erwähnen) in dem früheren
               Vertrag in Mark ausgedrückt war und dann in Dollar, ferner jetzt nur mehr 20{\%} vom Ladenpreis des Buches statt wie im Anfang 25{\%} gezahlt werden. Sonst sehe ich nichts wesentlich
               verschiedenes. Diese Vorauszahlung d. h. einmalige Zahlung von 3500 Dollar vermindert
               etwas die Ungerechtigkeit dass nur Strauss\pwindex{Strauss, Richard 11.06.1864 – 08.09.1949@\textsc{Strauss, Richard} (11.06.1864 – 08.09.1949), \emph{Theaterleiter, Komponist, Dirigent}|pw} von
                  Fürstner\orgindex{Musikverlag Adolph Fuerstner@Musikverlag Adolph Fürstner|pw}{ }so hohe Bezahlung bei Ablieferung der Oper kriegt.
               Aber dass die Oper blos 7{\%} abrechnet ist schon irgend eine
               komische Sache, denn tatsächlich rechnen sie schon mehr ab nur nimmt sich Fürstner\orgindex{Musikverlag Adolph Fuerstner@Musikverlag Adolph Fürstner|pw} wegen der hohen Zahlung an Strauss\pwindex{Strauss, Richard 11.06.1864 – 08.09.1949@\textsc{Strauss, Richard} (11.06.1864 – 08.09.1949), \emph{Theaterleiter, Komponist, Dirigent}|pw} die Differenz, was bei gut gehenden
               Opern doch zu hohen Gewinn für ihn ist. Verstehen tu ich die Sache nicht recht, weiss
               aber dass Hugo\pwindex{Hofmannsthal, Hugo von 1874-02-01 – 1929-07-15@\textsc{Hofmannsthal, Hugo von} (1874-02-01 – 1929-07-15), \emph{Schriftsteller}|pw} mit Schalk\pwindex{Schalk, Franz 27.05.1863 – 03.09.1931@\textsc{Schalk, Franz} (27.05.1863 – 03.09.1931), \emph{Theaterleiter, Dirigent}|pw} darüber sprach, auch mit Strauss\pwindex{Strauss, Richard 11.06.1864 – 08.09.1949@\textsc{Strauss, Richard} (11.06.1864 – 08.09.1949), \emph{Theaterleiter, Komponist, Dirigent}|pw} darüber Aussprachen hatte, die aber zu nichts
               führten.\pend
           \pstart
           Bitte telephonieren Sie mich einmal an, womöglich doch lieber einen Tag früher und
               kommen einen Sprung heraus, auch vormittag wies Ihnen passt.\pend
           \pstart
           Wenn es vor dem 11ten{ }sein könnte wärs mir sehr lieb weil ich dann immer
               unsicher bin ob nicht der Raimund\pwindex{Hofmannsthal, Raimund von 26.5.1906 – 20.03.1974@\textsc{Hofmannsthal, Raimund von} (26.5.1906 – 20.03.1974)|pw} gerade
               ankommt mit dem ich dann hier in aller Eile vieles Geschäftliche zu tun habe.\pend
           \pstart
           Von Herzen Ihre{\\[\baselineskip]}\spacefill\mbox{{[}hs.:{]} Gerty}\pend
           \leftskip=0em{}
         
         \endnumbering\mylabel{h}\end{ledgroupsized}  \newcommand{\dateiname}{L02544}\newcommand{\titel}{Gerty Hofmannsthal an Arthur Schnitzler, [5. 3. 1931]}\newcommand{\editorInnen}{Martin Anton Müller und Gerd-Hermann Susen}%% latex-leseansicht-abspann.tex
%% Abspann für die Leseansicht.
%% Der Schalter \ifkorrekturansicht ist bereits durch den Vorspann gesetzt.

%% latex-abspann.tex
%% Gemeinsamer Abspann für Korrekturansicht und Leseansicht.
%% Setzt den Schalter \ifkorrekturansicht voraus (gesetzt in den
%% einbindenden Dateien latex-korrekturansicht-abspann.tex bzw.
%% latex-leseansicht-abspann.tex).
%% ---------------------------------------------------------------

\normalsize

% Das esempio-Environment wird nur in der Leseansicht benötigt
\ifkorrekturansicht\else
\newenvironment{esempio}[3]%
{
    \vspace{1.5ex}
    \rlap{\underline{#1}}
    \par
    \setlength{\parindent}{0cm}
    \nopagebreak
    \leftskip=#2cm
    \rightskip=#3cm
}
{
    \par
}
\fi

\doendnotes{C}
\bigskip
\vfill

\clearpage

\footnotesize

\ifkorrekturansicht
  \lohead{\textsc{register}}
\fi

% theindex-Environment neu definieren ohne reledmac
\makeatletter
\renewenvironment{theindex}{%
  \ifkorrekturansicht
    \section*{\indexname}%
  \else
    \subsubsection*{Index der erwähnten Entitäten}%
  \fi
  \setlength{\parindent}{0pt}%
  \setlength{\parskip}{0pt plus 0.3pt}%
  \let\item\@idxitem
}{%
  \ifkorrekturansicht\clearpage\fi
}
\makeatother

\IfFileExists{\jobname-pw.ind}{\input{\jobname-pw.ind}}{}

% Quellenangabe nur in der Leseansicht
\ifkorrekturansicht\else
% Fallback-Definitionen, falls die .tex-Datei \titel etc. nicht gesetzt hat
\providecommand{\titel}{}
\providecommand{\editorInnen}{}
\providecommand{\dateiname}{\jobname}

\vspace{3cm}

\vfill

\footnotesize
\textsc{Quelle}: \titel. Herausgegeben von {\editorInnen}. In: \emph{Arthur Schnitzler: Briefwechsel mit Autorinnen und Autoren}.
 Digitale Edition, https://schnitzler-briefe.acdh.oeaw.ac.at/{\dateiname}.html (Stand \today)
\fi

\end{document}


      