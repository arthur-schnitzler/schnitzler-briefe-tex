%% latex-korrekturansicht-vorspann.tex
%% Vorspann für die Korrekturansicht.
%% Lädt die gemeinsame Datei latex-vorspann.tex mit gesetztem Schalter.

\newif\ifkorrekturansicht
\korrekturansichttrue

\input{../tex-inputs/latex-vorspann}


\section[Arthur Schnitzler an Stefan Zweig, 26. {[}10.{]} 1911]{L03787 Arthur Schnitzler an Stefan Zweig, 26. {[}10.{]} 1911}
\nopagebreak\mylabel{L03787v}
\rehead{ }\normalsize\beginnumbering\briefempfaengerindex{Zweig, Stefan@\textsc{Zweig, Stefan}!zzzSchnitzler, Arthur@\emph{von Arthur Schnitzler}!1911-10-261@{26. {[}10.{]} 1911}|(be}
\toendnotes[C]{\smallbreak\pagebreak[2]}\Standort{Jerusalem, National Library of Israel, ARC. Ms. Var. 305 1 58 Stefan Zweig Collection.}
\physDesc{Briefkarte, 1 Blatt, 2 Seiten, 321 Zeichen
\newline{}Handschrift: schwarze Tinte, deutsche Kurrent}\toendnotes[C]{\smallbreak}
\pstart
           {\pb}\textcolor{gray}{\textbf{A. S.}}\hfill \label{K_L03787-1v}\edtext{26. \substVorne{}\textsuperscript{1\textcolor{gray}{1}}\substDazwischen{}X\substHinten{}. 911}{\lemma{\textnormal{\emph{26. X. 911}}}\Cendnote{\textnormal{Mögliche Zweifel an der Datierung auf Oktober werden durch das Antwortschreiben
                           Zweigs\pwindex{Zweig, Stefan 28.11.1881 – 23.02.1942@\textsc{Zweig, Stefan} (28.11.1881 – 23.02.1942), \emph{Schriftsteller/Schriftstellerin}|pwk} vom 27. 10. 1911
                        ausgeräumt. Auch würde eine Verortung des Korrespondenzstücks in den
                        September (›IX‹) inhaltlich wenig Sinn ergeben, da die von Schnitzler angekündigte Reise noch
                        einen Monat entfernt wäre.}}}\label{K_L03787-1}.\pend
           
\pstart{}lieber Doctor Zweig,\pend\vspace{0.5em}
\pstart
           Sie ſind auch in dieſem \label{K_L03787-2v}\edtext{\textsc{Gautier}\pwindex{Gautier, Theophile 1811-08-30 – 1872-10-23@\textsc{Gautier, Théophile} (1811-08-30 – 1872-10-23), \emph{Schriftsteller/Schriftstellerin, Kritiker/Kritikerin, Maler/Malerin}|pw} Comité}{\lemma{\textnormal{\emph{Gautier Comité}}}\Cendnote{\textnormal{Anlässlich des kürzlich vergangenen
                  100. Geburtstages von Théophile Gautier\pwindex{Gautier, Theophile 1811-08-30 – 1872-10-23@\textsc{Gautier, Théophile} (1811-08-30 – 1872-10-23), \emph{Schriftsteller/Schriftstellerin, Kritiker/Kritikerin, Maler/Malerin}|pwk} am 30. 8. 1911
                  bemühte sich seine Tochter, die Schriftstellerin Judith Gautier\pwindex{Gautier, Judith 1845-08-25 – 1917-12-26@\textsc{Gautier, Judith} (1845-08-25 – 1917-12-26), \emph{Schriftsteller/Schriftstellerin}|pwk} und sein Schwiegersohn Émile Bergerat\pwindex{Bergerat, Emile 1845-04-29 – 1923-10-13@\textsc{Bergerat, Émile} (1845-04-29 – 1923-10-13)|pwk} um die Errichtung eines Denkmals. Das
                  Vorhaben gelang nicht.}}}\label{K_L03787-2}. Darf ich Sie fragen, \uline{ob} Sie, \textsc{resp.}{ }\uline{welchen} Beitrag Sie gezeichnet haben oder zeichnen
               wollen? Ich möchte mich nach Ihnen richten. \pend
           
\pstart
           – Nach meiner \label{K_L03787-3v}\edtext{Rückkehr aus Deutsch{\pb}land\oindex{Deutschland@\textbf{Deutschland}, \emph{A.PCLI}|pw}}{\lemma{\textnormal{\emph{Rückkehr aus Deutschland}}}\Cendnote{\textnormal{Schnitzler reiste am 29. 10. 1911 über Prag\oindex{Prag@\textbf{Prag}, \emph{A.ADM1}|pwk} nach Berlin\oindex{Berlin@\textbf{Berlin}, \emph{P.PPLC}|pwk}, Hamburg\oindex{Hamburg@\textbf{Hamburg}, \emph{P.PPLA}|pwk}, München\oindex{Muenchen@\textbf{München}, \emph{P.PPLA}|pwk} und Garmisch-Partenkirchen\oindex{Garmisch-Partenkirchen@\textbf{Garmisch-Partenkirchen}, \emph{P.PPLA3}|pwk}. Am 17. 11. 1911 war er wieder in Wien\oindex{Wien@\textbf{Wien}, \emph{A.ADM2}|pwk}. Erst am 12. 12. 1911 sah man sich wieder.}}}\label{K_L03787-3} hoff ich
               Sie ſehr bald zu längerem Zusa{\geminationm}enſein bei uns zu
               ſehn.\pend
           
\pstart
           Herzlichſt Ihr{\\[\baselineskip]}\spacefill\mbox{ArthSchnitzl}\pend
           \leftskip=0em{}\selectlanguage{ngerman}\endnumbering\briefempfaengerindex{Zweig, Stefan@\textsc{Zweig, Stefan}!zzzSchnitzler, Arthur@\emph{von Arthur Schnitzler}!1911-10-261@{26. {[}10.{]} 1911}|)be}\mylabel{L03787h}  \normalsize

\doendnotes{C}
\bigskip
\vfill

\clearpage

\footnotesize

\lohead{\textsc{register}}

% Definiere theindex-Environment komplett neu ohne reledmac
\makeatletter
\renewenvironment{theindex}{%
  \section*{\indexname}%
  \setlength{\parindent}{0pt}%
  \setlength{\parskip}{0pt plus 0.3pt}%
  \let\item\@idxitem
}{%
  \clearpage
}
\makeatother

\IfFileExists{\jobname-pw.ind}{\input{\jobname-pw.ind}}{}

\end{document}

      