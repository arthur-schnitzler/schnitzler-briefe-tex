%% latex-leseansicht-vorspann.tex
%% Vorspann für die Leseansicht.
%% Lädt die gemeinsame Datei latex-vorspann.tex mit nicht gesetztem Schalter.

\newif\ifkorrekturansicht
\korrekturansichtfalse

\input{../tex-inputs/latex-vorspann}


\section[Arthur Schnitzler an Stefan Zweig, 26. {[}10.{]} 1911]{L03787 Arthur Schnitzler an Stefan Zweig, 26. [10.] 1911}
\nopagebreak\mylabel{L03787v}
\rehead{ }\normalsize\beginnumbering\briefempfaengerindex{Zweig, Stefan@\textsc{Zweig, Stefan}!zzzSchnitzler, Arthur@\emph{von Arthur Schnitzler}!1911-10-261@{26. [10.] 1911}|(be}
\toendnotes[C]{\smallbreak\pagebreak[2]}
\correspDesc{Versand  durch Arthur Schnitzler am 26. [10.] 1911 in Wien
\newline{}Erhalt  durch Stefan Zweig im Zeitraum [26. 10. 1911 –
                  27. 10. 1911] in Wien}\toendnotes[C]{\smallbreak}
\Standort{Jerusalem, National Library of Israel, ARC. Ms. Var. 305 1 58 Stefan Zweig Collection.}
\physDesc{Briefkarte, 321 Zeichen
\newline{}Handschrift: schwarze Tinte, deutsche Kurrent}\toendnotes[C]{\smallbreak}
\pstart
           {\pb}\textcolor{gray}{\textbf{A. S.}}\hfill \label{K_L03787-1v}\edtext{26. \substVorne{}\textsuperscript{1\textcolor{gray}{1}}\substDazwischen{}X\substHinten{}. 911}{\lemma{\textnormal{\emph{26. X. 911}}}\Cendnote{\textnormal{Mögliche Zweifel an der Datierung auf Oktober werden durch das Antwortschreiben
                           Zweigs\pwindex{Zweig, Stefan 28.\,11.\,1881 Wien – 23.\,2.\,1942 Petrópolis@\textsc{Zweig, Stefan} (28.\,11.\,1881 Wien – 23.\,2.\,1942 Petrópolis), \emph{Schriftsteller}|pwk} vom XXXX Auszeichnungsfehler: Dokument L03637 nicht gefunden
                        ausgeräumt. Auch würde eine Verortung des Korrespondenzstücks in den
                        September (›IX‹) inhaltlich wenig Sinn ergeben, da die von Schnitzler angekündigte Reise noch
                        einen Monat entfernt wäre.}}}\label{K_L03787-1}.\pend
           
\pstart{}lieber Doctor Zweig,\pend\vspace{0.5em}
\pstart
           Sie{ }ſind auch in dieſem \label{K_L03787-2v}\edtext{\textsc{Gautier}\pwindex{Gautier, Théophile 30.\,8.\,1811 Tarbes – 23.\,10.\,1872 Neuilly-sur-Seine@\textsc{Gautier, Théophile} (30.\,8.\,1811 Tarbes – 23.\,10.\,1872 Neuilly-sur-Seine), \emph{Schriftsteller, Kritiker, Maler}|pw} Comité}{\lemma{\textnormal{\emph{Gautier Comité}}}\Cendnote{\textnormal{Anlässlich des kürzlich vergangenen
                  100. Geburtstages von Théophile Gautier\pwindex{Gautier, Théophile 30.\,8.\,1811 Tarbes – 23.\,10.\,1872 Neuilly-sur-Seine@\textsc{Gautier, Théophile} (30.\,8.\,1811 Tarbes – 23.\,10.\,1872 Neuilly-sur-Seine), \emph{Schriftsteller, Kritiker, Maler}|pwk} am 30. 8. 1911
                  bemühten sich seine Tochter, die Schriftstellerin Judith Gautier\pwindex{Gautier, Judith 25.\,8.\,1845 Paris – 26.\,12.\,1917 Dinard@\textsc{Gautier, Judith} (25.\,8.\,1845 Paris – 26.\,12.\,1917 Dinard), \emph{Schriftstellerin}|pwk}, und sein Schwiegersohn Émile Bergerat\pwindex{Bergerat, Émile 29.\,4.\,1845 rue de la Vieille-Monnaie – 13.\,10.\,1923 Neuilly-sur-Seine@\textsc{Bergerat, Émile} (29.\,4.\,1845 rue de la Vieille-Monnaie – 13.\,10.\,1923 Neuilly-sur-Seine)|pwk} um die Errichtung eines Denkmals. Das
                  Vorhaben gelang nicht.}}}\label{K_L03787-2}. Darf ich Sie fragen, \uline{ob} Sie, \textsc{resp.}{ }\uline{welchen} Beitrag Sie gezeichnet haben oder zeichnen
               wollen? Ich möchte mich nach Ihnen richten.\pend
           
\pstart
           – Nach meiner \label{K_L03787-3v}\edtext{Rückkehr aus Deutsch{\pb}land\oindex{Deutschland@\textbf{Deutschland}|pw}}{\lemma{\textnormal{\emph{Rückkehr aus Deutschland}}}\Cendnote{\textnormal{Schnitzler reiste am 29. 10. 1911 über Prag\oindex{Prag@\textbf{Prag}, \emph{Land}|pwk} nach Berlin\oindex{Berlin@\textbf{Berlin}, \emph{Hauptstadt}|pwk}, Hamburg\oindex{Hamburg@\textbf{Hamburg}|pwk}, München\oindex{München@\textbf{München}|pwk} und Garmisch-Partenkirchen\oindex{Garmisch-Partenkirchen@\textbf{Garmisch-Partenkirchen}, \emph{Hauptstadt}|pwk}. Am 17. 11. 1911 war er wieder in Wien\oindex{Wien@\textbf{Wien}, \emph{Verwaltungsgebiet}|pwk}. Erst am 12. 12. 1911 sah man sich wieder.}}}\label{K_L03787-3} hoff ich
               Sie{ }ſehr bald zu längerem Zusa{\geminationm}enſein bei uns zu{ }ſehn.\pend
           
\pstart
           Herzlichſt Ihr{\\[\baselineskip]}\spacefill\mbox{ArthSchnitzl}\pend
           \leftskip=0em{}\selectlanguage{ngerman}\endnumbering\briefempfaengerindex{Zweig, Stefan@\textsc{Zweig, Stefan}!zzzSchnitzler, Arthur@\emph{von Arthur Schnitzler}!1911-10-261@{26. [10.] 1911}|)be}\mylabel{L03787h}  \newcommand{\dateiname}{L03787}\newcommand{\titel}{Arthur Schnitzler an Stefan Zweig, 26. [10.] 1911}\newcommand{\editorInnen}{Selma Jahnke und Martin Anton Müller}%% latex-leseansicht-abspann.tex
%% Abspann für die Leseansicht.
%% Der Schalter \ifkorrekturansicht ist bereits durch den Vorspann gesetzt.

%% latex-abspann.tex
%% Gemeinsamer Abspann für Korrekturansicht und Leseansicht.
%% Setzt den Schalter \ifkorrekturansicht voraus (gesetzt in den
%% einbindenden Dateien latex-korrekturansicht-abspann.tex bzw.
%% latex-leseansicht-abspann.tex).
%% ---------------------------------------------------------------

\normalsize

% Das esempio-Environment wird nur in der Leseansicht benötigt
\ifkorrekturansicht\else
\newenvironment{esempio}[3]%
{
    \vspace{1.5ex}
    \rlap{\underline{#1}}
    \par
    \setlength{\parindent}{0cm}
    \nopagebreak
    \leftskip=#2cm
    \rightskip=#3cm
}
{
    \par
}
\fi

\doendnotes{C}
\bigskip
\vfill

\clearpage

\footnotesize

\ifkorrekturansicht
  \lohead{\textsc{register}}
\fi

% theindex-Environment neu definieren ohne reledmac
\makeatletter
\renewenvironment{theindex}{%
  \ifkorrekturansicht
    \section*{\indexname}%
  \else
    \subsubsection*{Index der erwähnten Entitäten}%
  \fi
  \setlength{\parindent}{0pt}%
  \setlength{\parskip}{0pt plus 0.3pt}%
  \let\item\@idxitem
}{%
  \ifkorrekturansicht\clearpage\fi
}
\makeatother

\IfFileExists{\jobname-pw.ind}{\input{\jobname-pw.ind}}{}

% Quellenangabe nur in der Leseansicht
\ifkorrekturansicht\else
% Fallback-Definitionen, falls die .tex-Datei \titel etc. nicht gesetzt hat
\providecommand{\titel}{}
\providecommand{\editorInnen}{}
\providecommand{\dateiname}{\jobname}

\vspace{3cm}

\vfill

\footnotesize
\textsc{Quelle}: \titel. Herausgegeben von {\editorInnen}. In: \emph{Arthur Schnitzler: Briefwechsel mit Autorinnen und Autoren}.
 Digitale Edition, https://schnitzler-briefe.acdh.oeaw.ac.at/{\dateiname}.html (Stand \today)
\fi

\end{document}


