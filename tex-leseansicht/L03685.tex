%% latex-korrekturansicht-vorspann.tex
%% Vorspann für die Korrekturansicht.
%% Lädt die gemeinsame Datei latex-vorspann.tex mit gesetztem Schalter.

\newif\ifkorrekturansicht
\korrekturansichttrue

\input{../tex-inputs/latex-vorspann}


\section[Stefan Zweig an Arthur Schnitzler, 18. 8. 1920]{L03685 Stefan Zweig an Arthur Schnitzler, 18. 8. 1920}
\nopagebreak\mylabel{L03685v}
\rehead{ }\normalsize\beginnumbering\briefempfaengerindex{Schnitzler, Arthur@\textsc{Schnitzler, Arthur}!zzzZweig, Stefan@\emph{von Stefan Zweig}!1920-08-182@{18. 8. 1920}|(be}
\toendnotes[C]{\smallbreak\pagebreak[2]}\Standort{CUL, Schnitzler, B 118.}
\physDesc{Brief, 1 Blatt, 1 Seite, 763 Zeichen
\newline{}Schreibmaschine
\newline{}Handschrift: blaue Tinte, lateinische Kurrent (\noindent{}eine Ergänzung, Unterschrift)
\newline{}Schnitzler: 1) mit Bleistift beschriftet: »\textsc{Zweig}«  2) mit rotem Buntstift eine Unterstreichung}\toendnotes[C]{\smallbreak}
\pstart
           \raggedleft{}{\pb}Salzburg\oindex{Salzburg@\textbf{Salzburg}, \emph{A.ADM2}|pw}, am 18. August 1920.\pend
           
\pstart{}Lieber verehrter Herr Doktor!\pend\vspace{0.5em}
\pstart
           Ich \label{K_L03685-11v}\edtext{telegrafierte}{\lemma{\textnormal{\emph{telegrafierte}}}\Cendnote{\textnormal{Das Telegramm ist nicht überliefert, wurde
                  Schnitzler aber zugestellt, vgl. Arthur Schnitzler an Stefan Zweig, 20. 8. 1920.}}}\label{K_L03685-11} Ihnen den Vorschlag 10{\%} und 100 Dollar Anzahlung. Ich glaube damit,
               Ihnen das anständig erreichbare geraten zu haben. Natürlich kenne ich Ihren Kontrakt
               mit Fischer\pwindex{Fischer, Samuel 24.12.1859 – 15.10.1934@\textsc{Fischer, Samuel} (24.12.1859 – 15.10.1934), \emph{Verleger/Verlegerin}|pw} nicht und weiss nicht, ob Sie diesem auch
               noch etwas abzugeben haben. Immerhin ist es gerade jetzt für uns wichtig draussen
               Fuss zu fassen, weil ja ein Buch dem andern den Weg bahnt.\pend
           
\pstart
           Verzeihen Sie dass ich diktiere, statt Ihnen persönlich zu schreiben, aber es ist
               jetzt die \label{K_L03685-1v}\edtext{Teufelswoche}{\lemma{\textnormal{\emph{Teufelswoche}}}\Cendnote{\textnormal{1920 fanden zum ersten Mal die \emph{Salzburger Festspiele}\orgindex{Salzburger Festspiele@Salzburger Festspiele|pwk} statt.}}}\label{K_L03685-1}\introOben{}»Jedermann\pwindex{Jedermann. Das Spiel vom Sterben des reichen Mannes@\emph{Jedermann. Das Spiel vom Sterben des reichen Mannes}|pw}«\introOben{} in Salzburg\oindex{Salzburg@\textbf{Salzburg}, \emph{A.ADM2}|pw}, wo einem nicht Zeit zum atmen bleibt.\pend
           
\pstart
           Ich hoffe, dass Sie zu stillerer Jahreszeit bald herkommen und mich des lange
               entbehrten Vergnügens erfreuen kann, Sie zu sehen, Sie zu sprechen.\pend
           
\pstart
           In Verehrung getreu Ihr{\\[\baselineskip]}\spacefill\mbox{{[}hs.:{]} Stefan Zweig}\pend
           \leftskip=0em{}\selectlanguage{ngerman}\endnumbering\briefempfaengerindex{Schnitzler, Arthur@\textsc{Schnitzler, Arthur}!zzzZweig, Stefan@\emph{von Stefan Zweig}!1920-08-182@{18. 8. 1920}|)be}\mylabel{L03685h}
\begin{anhang}
\end{anhang}\normalsize

\doendnotes{C}
\bigskip
\vfill

\clearpage

\footnotesize

\lohead{\textsc{register}}

% Definiere theindex-Environment komplett neu ohne reledmac
\makeatletter
\renewenvironment{theindex}{%
  \section*{\indexname}%
  \setlength{\parindent}{0pt}%
  \setlength{\parskip}{0pt plus 0.3pt}%
  \let\item\@idxitem
}{%
  \clearpage
}
\makeatother

\IfFileExists{\jobname-pw.ind}{\input{\jobname-pw.ind}}{}

\end{document}

      