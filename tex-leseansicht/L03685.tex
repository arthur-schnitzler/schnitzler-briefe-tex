%% latex-leseansicht-vorspann.tex
%% Vorspann für die Leseansicht.
%% Lädt die gemeinsame Datei latex-vorspann.tex mit nicht gesetztem Schalter.

\newif\ifkorrekturansicht
\korrekturansichtfalse

\input{../tex-inputs/latex-vorspann}


\section[Stefan Zweig an Arthur Schnitzler, 18. 8. 1920]{L03685 Stefan Zweig an Arthur Schnitzler, 18. 8. 1920}
\nopagebreak\mylabel{L03685v}
\rehead{ }\normalsize\beginnumbering\briefempfaengerindex{Schnitzler, Arthur@\textsc{Schnitzler, Arthur}!zzzZweig, Stefan@\emph{von Stefan Zweig}!1920-08-182@{18. 8. 1920}|(be}
\toendnotes[C]{\smallbreak\pagebreak[2]}
\correspDesc{Versand  durch Stefan Zweig am 18. 8. 1920 in Salzburg
\newline{}Erhalt  durch Arthur Schnitzler im Zeitraum [19. 8. 1920 – 20. 8. 1920] in Wien}\toendnotes[C]{\smallbreak}
\Standort{CUL, Schnitzler, B 118.}
\physDesc{Brief, 1 Blatt, 1 Seite, 763 Zeichen
\newline{}Schreibmaschine
\newline{}Handschrift: blaue Tinte, lateinische Kurrent (\noindent{}eine Ergänzung, Unterschrift)
\newline{}Schnitzler: 1) mit Bleistift beschriftet: »\textsc{Zweig}«  2) mit rotem Buntstift eine Unterstreichung}\toendnotes[C]{\smallbreak}
\pstart
           \raggedleft{}{\pb}Salzburg\oindex{Salzburg@\textbf{Salzburg}, \emph{Verwaltungsgebiet}|pw}, am 18. August 1920.\pend
           
\pstart\center{}Lieber verehrter Herr Doktor!\pend\vspace{0.5em}
\pstart
           Ich \label{K_L03685-1v}\edtext{telegrafierte}{\lemma{\textnormal{\emph{telegrafierte}}}\Cendnote{\textnormal{Das Telegramm ist nicht überliefert, wurde
                  Schnitzler aber zugestellt, vgl. XXXX Auszeichnungsfehler: Dokument L03763 nicht gefunden.}}}\label{K_L03685-1} Ihnen den Vorschlag 10{\%} und 100 Dollar Anzahlung. Ich glaube damit,
               Ihnen das anständig erreichbare geraten zu haben. Natürlich kenne ich Ihren Kontrakt
               mit Fischer\pwindex{Fischer, Samuel 24.\,12.\,1859 Liptovský Mikuláš – 15.\,10.\,1934 Berlin@\textsc{Fischer, Samuel} (24.\,12.\,1859 Liptovský Mikuláš – 15.\,10.\,1934 Berlin), \emph{Verleger}|pw} nicht und weiss nicht, ob Sie diesem auch
               noch etwas abzugeben haben. Immerhin ist es gerade jetzt für uns wichtig draussen
               Fuss zu fassen, weil ja ein Buch dem andern den Weg bahnt.\pend
           
\pstart
           Verzeihen Sie dass ich diktiere, statt Ihnen persönlich zu schreiben, aber es ist
               jetzt die \label{K_L03685-2v}\edtext{Teufelswoche}{\lemma{\textnormal{\emph{Teufelswoche}}}\Cendnote{\textnormal{1920 fanden zum ersten Mal die \emph{Salzburger Festspiele}\orgindex{Salzburger Festspiele@Salzburger Festspiele|pwk} statt.}}}\label{K_L03685-2}{ }\introOben{}»Jedermann\pwindex{Hofmannsthal, Hugo von 1.\,2.\,1874 Wien – 15.\,7.\,1929 Rodaun@\textsc{Hofmannsthal, Hugo von} (1.\,2.\,1874 Wien – 15.\,7.\,1929 Rodaun), \emph{Schriftsteller}!Jedermann. Das Spiel vom Sterben des reichen Mannes@\strich\emph{Jedermann. Das Spiel vom Sterben des reichen Mannes}|pw}«\introOben{} in Salzburg\oindex{Salzburg@\textbf{Salzburg}, \emph{Verwaltungsgebiet}|pw}, wo einem nicht Zeit zum atmen bleibt.\pend
           
\pstart
           Ich hoffe, dass Sie zu stillerer Jahreszeit bald herkommen und {[}ich{]} mich des lange
               entbehrten Vergnügens erfreuen kann, Sie zu sehen, Sie zu sprechen.\pend
           
\pstart
           In Verehrung getreu Ihr{\\[\baselineskip]}\spacefill\mbox{{[}hs.:{]} Stefan Zweig}\pend
           \leftskip=0em{}\selectlanguage{ngerman}\endnumbering\briefempfaengerindex{Schnitzler, Arthur@\textsc{Schnitzler, Arthur}!zzzZweig, Stefan@\emph{von Stefan Zweig}!1920-08-182@{18. 8. 1920}|)be}\mylabel{L03685h}  \newcommand{\dateiname}{L03685}\newcommand{\titel}{Stefan Zweig an Arthur Schnitzler, 18. 8. 1920}\newcommand{\editorInnen}{Selma Jahnke und Martin Anton Müller}%% latex-leseansicht-abspann.tex
%% Abspann für die Leseansicht.
%% Der Schalter \ifkorrekturansicht ist bereits durch den Vorspann gesetzt.

%% latex-abspann.tex
%% Gemeinsamer Abspann für Korrekturansicht und Leseansicht.
%% Setzt den Schalter \ifkorrekturansicht voraus (gesetzt in den
%% einbindenden Dateien latex-korrekturansicht-abspann.tex bzw.
%% latex-leseansicht-abspann.tex).
%% ---------------------------------------------------------------

\normalsize

% Das esempio-Environment wird nur in der Leseansicht benötigt
\ifkorrekturansicht\else
\newenvironment{esempio}[3]%
{
    \vspace{1.5ex}
    \rlap{\underline{#1}}
    \par
    \setlength{\parindent}{0cm}
    \nopagebreak
    \leftskip=#2cm
    \rightskip=#3cm
}
{
    \par
}
\fi

\doendnotes{C}
\bigskip
\vfill

\clearpage

\footnotesize

\ifkorrekturansicht
  \lohead{\textsc{register}}
\fi

% theindex-Environment neu definieren ohne reledmac
\makeatletter
\renewenvironment{theindex}{%
  \ifkorrekturansicht
    \section*{\indexname}%
  \else
    \subsubsection*{Index der erwähnten Entitäten}%
  \fi
  \setlength{\parindent}{0pt}%
  \setlength{\parskip}{0pt plus 0.3pt}%
  \let\item\@idxitem
}{%
  \ifkorrekturansicht\clearpage\fi
}
\makeatother

\IfFileExists{\jobname-pw.ind}{\input{\jobname-pw.ind}}{}

% Quellenangabe nur in der Leseansicht
\ifkorrekturansicht\else
% Fallback-Definitionen, falls die .tex-Datei \titel etc. nicht gesetzt hat
\providecommand{\titel}{}
\providecommand{\editorInnen}{}
\providecommand{\dateiname}{\jobname}

\vspace{3cm}

\vfill

\footnotesize
\textsc{Quelle}: \titel. Herausgegeben von {\editorInnen}. In: \emph{Arthur Schnitzler: Briefwechsel mit Autorinnen und Autoren}.
 Digitale Edition, https://schnitzler-briefe.acdh.oeaw.ac.at/{\dateiname}.html (Stand \today)
\fi

\end{document}


