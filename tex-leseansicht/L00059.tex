%% latex-leseansicht-vorspann.tex
%% Vorspann für die Leseansicht.
%% Lädt die gemeinsame Datei latex-vorspann.tex mit nicht gesetztem Schalter.

\newif\ifkorrekturansicht
\korrekturansichtfalse

\input{../tex-inputs/latex-vorspann}


         
         \newcommand{\erwaehntePersonen}{Personen: Hermann Bahr, Richard Beer-Hofmann, Felix Dörmann, Bertha Hofmann, Felix Salten}
         \newcommand{\erwaehnteInstitutionen}{Institutionen: Österreichischer Gewerbeverein}
         \newcommand{\erwaehnteOrte}{Orte: Eschenbachgasse, III., Landstraße, Kärntnerring, Wien}
         \newcommand{\erwaehnteWerke}{Werke: Sensationen}
               \section[Hugo von Hofmannsthal an Arthur Schnitzler, 1. 1. 1892]{ Hugo von Hofmannsthal an Arthur Schnitzler, 1. 1. 1892}\nopagebreak\mylabel{v}\rehead{ }\begin{ledgroupsized}[t]{13cm}\normalsize\beginnumbering \toendnotes[C]{\smallbreak\pagebreak[2]} \Standort{CUL, Schnitzler, B 43.}
\physDesc{Kartenbrief
\newline{}Handschrift: Bleistift, deutsche Kurrent\newline{}Versand: Stempel: »\nobreak{}\oindex{III., Landstrasse@\textbf{III., Landstraße}|pwk}Wien 3/3, 1. 1. 92, 5–6 N\nobreak{}«.  
\newline{}Schnitzler: mit Bleistift datiert: »1/1 92« \newline{}Ordnung: mit Bleistift von unbekannter Hand nummeriert:
                                        »12« und auf der Rückseite der Adressseite
                                    zugefügt: »14.05 / 7.02 / 6.96 / 7.00 /
                                    13.60« }\buchAbdrucke{\weitereDrucke{1) Hugo von Hofmannsthal, Arthur Schnitzler: \emph{Briefwechsel}. Hg. Therese Nickl und Heinrich Schnitzler. Frankfurt am Main: \emph{S. Fischer} 1964, S. 14.} \weitereDrucke{2) Hermann Bahr, Arthur Schnitzler: \emph{Briefwechsel, Aufzeichnungen, Dokumente
                                (1891–1931)}. Hg. Kurt Ifkovits und Martin Anton Müller. Göttingen: \emph{Wallstein} 2018, S. 18–19.} }\toendnotes[C]{\smallbreak}\pstart{}{\pb}Herrn \textsc{D\textsuperscript{r} Arthur Schnitzler}\pend{}\pstart{}\textsc{Wien\oindex{Wien@\textbf{Wien}|pw}}\pend{}\pstart{}\textsc{I. Kärnthnerring 12.\oindex{Kaerntnerring@\textbf{Kärntnerring}|pw}}\pend{}{\bigskip}\pstart{}{\pb}Lieber Freund!\pend\pstart
           Dörmann\pwindex{Doermann, Felix 29.05.1870 – 26.10.1928@\textsc{Dörmann, Felix} (29.05.1870 – 26.10.1928), \emph{Schriftsteller}|pw} will uns ſein neues \label{K_L00059_1v}\edtext{Buch\pwindex{Doermann, Felix 29.05.1870 – 26.10.1928@\textsc{Dörmann, Felix} (29.05.1870 – 26.10.1928), \emph{Schriftsteller}!Sensationen1892@\strich\emph{Sensationen} {[}1892{]}|pwv}}{\lemma{\textnormal{\emph{Buch}}}\Cendnote{\textnormal{Felix Dörmann\pwindex{Doermann, Felix 29.05.1870 – 26.10.1928@\textsc{Dörmann, Felix} (29.05.1870 – 26.10.1928), \emph{Schriftsteller}|pwk}: \emph{Sensationen}\pwindex{Doermann, Felix 29.05.1870 – 26.10.1928@\textsc{Dörmann, Felix} (29.05.1870 – 26.10.1928), \emph{Schriftsteller}!Sensationen1892@\strich\emph{Sensationen} {[}1892{]}|pwk}. Wien: \emph{Verlag von
                                Leopold Weiss}{ }1892.}}}\label{K_L00059_1h} vorleſen und hat mich gebeten, Sie einzuladen.\pend
           \pstart
           Wenn Sie alſo \label{K_L00059_2v}\edtext{nichts beſſeres}{\lemma{\textnormal{\emph{nichts beſſeres}}}\Cendnote{\textnormal{Schnitzler war bei der Lesung.}}}\label{K_L00059_2h}
                    vorhaben, kommen Sie morgen Samstag, \uline{½ 8} Uhr (pünktlich) Gewerbeverein\orgindex{Oesterreichischer Gewerbeverein@Österreichischer Gewerbeverein|pw}, Eſchenbachgaſſe\oindex{Eschenbachgasse@\textbf{Eschenbachgasse}|pw}, 3 Stock, im Secretariat. Es kommen Salten\pwindex{Salten, Felix 06.09.1869 – 08.10.1945@\textsc{Salten, Felix} (06.09.1869 – 08.10.1945), \emph{Schriftsteller, Journalist}|pw}, Bahr\pwindex{Bahr, Hermann 19.07.1863 – 15.01.1934@\textsc{Bahr, Hermann} (19.07.1863 – 15.01.1934), \emph{Schriftsteller, Kritiker}|pw}, Sie und
                    ich. Wenn Sie nicht können, ſagen Sie bitte mir pneumatiſch ab. Ich war heute
                    bei dem Leichenbegängnis von Richard\pwindex{Beer-Hofmann, Richard 1866-07-11 – 1945-09-26@\textsc{Beer-Hofmann, Richard} (1866-07-11 – 1945-09-26), \emph{Schriftsteller}|pw}s Mutter\pwindex{Hofmann, Bertha 1840 – 30.12.1891@\textsc{Hofmann, Bertha} (1840 – 30.12.1891)|pwv}. Soll man ihn
                    beſuchen? \pend
           \pstart
           Herzlichſt{\\[\baselineskip]}\spacefill\mbox{Loris}\pend
           \leftskip=0em{}
         
         \endnumbering\mylabel{h}\end{ledgroupsized}  \newcommand{\dateiname}{L00059}\newcommand{\titel}{Hugo von Hofmannsthal an Arthur Schnitzler, 1. 1. 1892}\newcommand{\editorInnen}{ Martin Anton Müller und Gerd-Hermann Susen}%% latex-leseansicht-abspann.tex
%% Abspann für die Leseansicht.
%% Der Schalter \ifkorrekturansicht ist bereits durch den Vorspann gesetzt.

%% latex-abspann.tex
%% Gemeinsamer Abspann für Korrekturansicht und Leseansicht.
%% Setzt den Schalter \ifkorrekturansicht voraus (gesetzt in den
%% einbindenden Dateien latex-korrekturansicht-abspann.tex bzw.
%% latex-leseansicht-abspann.tex).
%% ---------------------------------------------------------------

\normalsize

% Das esempio-Environment wird nur in der Leseansicht benötigt
\ifkorrekturansicht\else
\newenvironment{esempio}[3]%
{
    \vspace{1.5ex}
    \rlap{\underline{#1}}
    \par
    \setlength{\parindent}{0cm}
    \nopagebreak
    \leftskip=#2cm
    \rightskip=#3cm
}
{
    \par
}
\fi

\doendnotes{C}
\bigskip
\vfill

\clearpage

\footnotesize

\ifkorrekturansicht
  \lohead{\textsc{register}}
\fi

% theindex-Environment neu definieren ohne reledmac
\makeatletter
\renewenvironment{theindex}{%
  \ifkorrekturansicht
    \section*{\indexname}%
  \else
    \subsubsection*{Index der erwähnten Entitäten}%
  \fi
  \setlength{\parindent}{0pt}%
  \setlength{\parskip}{0pt plus 0.3pt}%
  \let\item\@idxitem
}{%
  \ifkorrekturansicht\clearpage\fi
}
\makeatother

\IfFileExists{\jobname-pw.ind}{\input{\jobname-pw.ind}}{}

% Quellenangabe nur in der Leseansicht
\ifkorrekturansicht\else
% Fallback-Definitionen, falls die .tex-Datei \titel etc. nicht gesetzt hat
\providecommand{\titel}{}
\providecommand{\editorInnen}{}
\providecommand{\dateiname}{\jobname}

\vspace{3cm}

\vfill

\footnotesize
\textsc{Quelle}: \titel. Herausgegeben von {\editorInnen}. In: \emph{Arthur Schnitzler: Briefwechsel mit Autorinnen und Autoren}.
 Digitale Edition, https://schnitzler-briefe.acdh.oeaw.ac.at/{\dateiname}.html (Stand \today)
\fi

\end{document}


      