%% latex-leseansicht-vorspann.tex
%% Vorspann für die Leseansicht.
%% Lädt die gemeinsame Datei latex-vorspann.tex mit nicht gesetztem Schalter.

\newif\ifkorrekturansicht
\korrekturansichtfalse

\input{../tex-inputs/latex-vorspann}


\section[Hugo von Hofmannsthal an Arthur Schnitzler, 1. 1. 1892]{L00059 Hugo von Hofmannsthal an Arthur Schnitzler, 1. 1. 1892}
\nopagebreak\mylabel{L00059v}
\rehead{ }\normalsize\beginnumbering\briefempfaengerindex{Schnitzler, Arthur@\textsc{Schnitzler, Arthur}!zzzHofmannsthal, Hugo von@\emph{von Hugo von Hofmannsthal}!1892-01-013@{1. 1. 1892}|(be}
\toendnotes[C]{\smallbreak\pagebreak[2]}
\correspDesc{Versand  durch Hugo von Hofmannsthal am 1. 1. 1892 in Wien
\newline{}Erhalt  durch Arthur Schnitzler am 1. 1. 92 in Wien}\toendnotes[C]{\smallbreak}
\Standort{CUL, Schnitzler, B 43.}
\physDesc{Kartenbrief, 463 Zeichen
\newline{}Handschrift: Bleistift, deutsche Kurrent
\newline{}Versand: Stempel: »\nobreak{}\oindex{III., Landstraße@\textbf{III., Landstraße}, \emph{Verwaltungsgebiet}|pwk}Wien 3/3, 1. 1. 92, 5–6 N\nobreak{}«.  
\newline{}Schnitzler: mit Bleistift datiert: »1/1 92« 
\newline{}Ordnung: mit Bleistift von unbekannter Hand nummeriert:
                                    »12« und auf der Rückseite der Adressseite
                                 zugefügt: »14.05 / 7.02 / 6.96 / 7.00 /
                                 13.60« }
\buchAbdrucke{\weitereDrucke{1) Hugo von Hofmannsthal, Arthur Schnitzler: \emph{Briefwechsel}. Herausgegeben von Therese Nickl und Heinrich Schnitzler. Frankfurt am Main: \emph{S. Fischer} 1964, S. 14.} \weitereDrucke{2) Hermann Bahr, Arthur Schnitzler: \emph{Briefwechsel, Aufzeichnungen, Dokumente (1891–1931)}. Herausgegeben von Kurt Ifkovits und Martin Anton Müller. Göttingen: \emph{Wallstein} 2018, S. 18–19.} }\toendnotes[C]{\smallbreak}\pstart{}{\pb}Herrn \textsc{D\textsuperscript{r} Arthur Schnitzler}\pend{}\pstart{}\textsc{Wien\oindex{Wien@\textbf{Wien}, \emph{Verwaltungsgebiet}|pw}}\pend{}\pstart{}\textsc{I. Kärnthnerring 12\oindex{Wien@\textbf{Wien}!I., Innere Stadt@\textbf{I., Innere Stadt}!Kärntnerring 12/Bösendorferstraße 11@\textbf{Kärntnerring 12/Bösendorferstraße 11}, \emph{Wohngebäude}|pw}.}\pend{}{\bigskip}\vspace{1em}
\pstart{}{\pb}Lieber Freund!\pend\vspace{0.5em}
\pstart
           Dörmann\pwindex{Dörmann, Felix 29.\,5.\,1870 Wien – 26.\,10.\,1928 ebd.@\textsc{Dörmann, Felix} (29.\,5.\,1870 Wien – 26.\,10.\,1928 ebd.), \emph{Schriftsteller}|pw} will uns{ }ſein neues \label{K_L00059-1v}\edtext{Buch\pwindex{Dörmann, Felix 29.\,5.\,1870 Wien – 26.\,10.\,1928 ebd.@\textsc{Dörmann, Felix} (29.\,5.\,1870 Wien – 26.\,10.\,1928 ebd.), \emph{Schriftsteller}!Sensationen@\strich\emph{Sensationen}|pwv}}{\lemma{\textnormal{\emph{Buch}}}\Cendnote{\textnormal{Felix Dörmann\pwindex{Dörmann, Felix 29.\,5.\,1870 Wien – 26.\,10.\,1928 ebd.@\textsc{Dörmann, Felix} (29.\,5.\,1870 Wien – 26.\,10.\,1928 ebd.), \emph{Schriftsteller}|pwk}: \emph{Sensationen}\pwindex{Dörmann, Felix 29.\,5.\,1870 Wien – 26.\,10.\,1928 ebd.@\textsc{Dörmann, Felix} (29.\,5.\,1870 Wien – 26.\,10.\,1928 ebd.), \emph{Schriftsteller}!Sensationen@\strich\emph{Sensationen}|pwk}. Wien: \emph{Verlag von Leopold
                        Weiss}{ }1892.}}}\label{K_L00059-1} vorleſen und hat mich gebeten, Sie einzuladen.\pend
           
\pstart
           Wenn Sie alſo \label{K_L00059-2v}\edtext{nichts beſſeres}{\lemma{\textnormal{\emph{nichts besseres}}}\Cendnote{\textnormal{Schnitzler war bei der Lesung.}}}\label{K_L00059-2}
               vorhaben, kommen Sie morgen Samstag, \uline{½ 8} Uhr (pünktlich) Gewerbeverein\orgindex{Österreichischer Gewerbeverein@Österreichischer Gewerbeverein|pw}, Eſchenbachgaſſe\oindex{Wien@\textbf{Wien}!I., Innere Stadt@\textbf{I., Innere Stadt}!Eschenbachgasse@\textbf{Eschenbachgasse}, \emph{Straße}|pw}, 3 Stock, im Secretariat. Es kommen Salten\pwindex{Salten, Felix 6.\,9.\,1869 Budapest – 8.\,10.\,1945 Zürich@\textsc{Salten, Felix} (6.\,9.\,1869 Budapest – 8.\,10.\,1945 Zürich), \emph{Schriftsteller, Journalist, Chefredakteur}|pw}, Bahr\pwindex{Bahr, Hermann 19.\,7.\,1863 Linz – 15.\,1.\,1934 München@\textsc{Bahr, Hermann} (19.\,7.\,1863 Linz – 15.\,1.\,1934 München), \emph{Schriftsteller, Kritiker}|pw}, Sie und
               ich. Wenn Sie nicht können,{ }ſagen Sie bitte mir pneumatiſch ab. Ich war heute bei dem
               Leichenbegängnis von Richards\pwindex{Beer-Hofmann, Richard 11.\,7.\,1866 Wien – 26.\,9.\,1945 New York City@\textsc{Beer-Hofmann, Richard} (11.\,7.\,1866 Wien – 26.\,9.\,1945 New York City), \emph{Schriftsteller}|pw}{ }Mutter\pwindex{Hofmann, Bertha 1840 – 30.\,12.\,1891 Wien@\textsc{Hofmann, Bertha} (1840 – 30.\,12.\,1891 Wien)|pwv}. Soll man ihn
               beſuchen?\pend
           
\pstart
           Herzlichſt{\\[\baselineskip]}\spacefill\mbox{Loris}\pend
           \leftskip=0em{}\selectlanguage{ngerman}\endnumbering\briefempfaengerindex{Schnitzler, Arthur@\textsc{Schnitzler, Arthur}!zzzHofmannsthal, Hugo von@\emph{von Hugo von Hofmannsthal}!1892-01-013@{1. 1. 1892}|)be}\mylabel{L00059h}  \newcommand{\dateiname}{L00059}\newcommand{\titel}{Hugo von Hofmannsthal an Arthur Schnitzler, 1. 1. 1892}\newcommand{\editorInnen}{Herausgegeben von Martin Anton Müller}%% latex-leseansicht-abspann.tex
%% Abspann für die Leseansicht.
%% Der Schalter \ifkorrekturansicht ist bereits durch den Vorspann gesetzt.

%% latex-abspann.tex
%% Gemeinsamer Abspann für Korrekturansicht und Leseansicht.
%% Setzt den Schalter \ifkorrekturansicht voraus (gesetzt in den
%% einbindenden Dateien latex-korrekturansicht-abspann.tex bzw.
%% latex-leseansicht-abspann.tex).
%% ---------------------------------------------------------------

\normalsize

% Das esempio-Environment wird nur in der Leseansicht benötigt
\ifkorrekturansicht\else
\newenvironment{esempio}[3]%
{
    \vspace{1.5ex}
    \rlap{\underline{#1}}
    \par
    \setlength{\parindent}{0cm}
    \nopagebreak
    \leftskip=#2cm
    \rightskip=#3cm
}
{
    \par
}
\fi

\doendnotes{C}
\bigskip
\vfill

\clearpage

\footnotesize

\ifkorrekturansicht
  \lohead{\textsc{register}}
\fi

% theindex-Environment neu definieren ohne reledmac
\makeatletter
\renewenvironment{theindex}{%
  \ifkorrekturansicht
    \section*{\indexname}%
  \else
    \subsubsection*{Index der erwähnten Entitäten}%
  \fi
  \setlength{\parindent}{0pt}%
  \setlength{\parskip}{0pt plus 0.3pt}%
  \let\item\@idxitem
}{%
  \ifkorrekturansicht\clearpage\fi
}
\makeatother

\IfFileExists{\jobname-pw.ind}{\input{\jobname-pw.ind}}{}

% Quellenangabe nur in der Leseansicht
\ifkorrekturansicht\else
% Fallback-Definitionen, falls die .tex-Datei \titel etc. nicht gesetzt hat
\providecommand{\titel}{}
\providecommand{\editorInnen}{}
\providecommand{\dateiname}{\jobname}

\vspace{3cm}

\vfill

\footnotesize
\textsc{Quelle}: \titel. Herausgegeben von {\editorInnen}. In: \emph{Arthur Schnitzler: Briefwechsel mit Autorinnen und Autoren}.
 Digitale Edition, https://schnitzler-briefe.acdh.oeaw.ac.at/{\dateiname}.html (Stand \today)
\fi

\end{document}


