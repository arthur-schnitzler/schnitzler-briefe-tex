%% latex-korrekturansicht-vorspann.tex
%% Vorspann für die Korrekturansicht.
%% Lädt die gemeinsame Datei latex-vorspann.tex mit gesetztem Schalter.

\newif\ifkorrekturansicht
\korrekturansichttrue

\input{../tex-inputs/latex-vorspann}


\section[Hugo von Hofmannsthal an Arthur Schnitzler, 1. 1. 1892]{L00059 Hugo von Hofmannsthal an Arthur Schnitzler, 1. 1. 1892}
\nopagebreak\mylabel{L00059v}
\rehead{ }\normalsize\beginnumbering\briefempfaengerindex{Schnitzler, Arthur@\textsc{Schnitzler, Arthur}!zzzHofmannsthal, Hugo von@\emph{von Hugo von Hofmannsthal}!1892-01-013@{1. 1. 1892}|(be}
\toendnotes[C]{\smallbreak\pagebreak[2]}\Standort{CUL, Schnitzler, B 43.}
\physDesc{Kartenbrief, 463 Zeichen
\newline{}Handschrift: Bleistift, deutsche Kurrent
\newline{}Versand: Stempel: »\nobreak{}\oindex{III., Landstrasse@\textbf{III., Landstraße}, \emph{A.ADM3}|pwk}Wien 3/3, 1. 1. 92, 5–6 N\nobreak{}«.  
\newline{}Schnitzler: mit Bleistift datiert: »1/1 92« 
\newline{}Ordnung: mit Bleistift von unbekannter Hand nummeriert:
                                    »12« und auf der Rückseite der Adressseite
                                 zugefügt: »14.05 / 7.02 / 6.96 / 7.00 /
                                 13.60« }
\buchAbdrucke{\weitereDrucke{1) Hugo von Hofmannsthal, Arthur Schnitzler: \emph{Briefwechsel}. Frankfurt am Main: \emph{S. Fischer} 1964, S. 14.} \weitereDrucke{2) Hermann Bahr, Arthur Schnitzler: \emph{Briefwechsel, Aufzeichnungen, Dokumente (1891–1931)}. Göttingen: \emph{Wallstein} 2018, S. 18–19.} }\toendnotes[C]{\smallbreak}\pstart{}{\pb}Herrn \textsc{D\textsuperscript{r} Arthur Schnitzler}\pend{}\pstart{}\textsc{Wien\oindex{Wien@\textbf{Wien}, \emph{A.ADM2}|pw}}\pend{}\pstart{}\textsc{I. Kärnthnerring 12\oindex{Kaerntnerring 12/Boesendorferstrasse 11@\textbf{Kärntnerring 12/Bösendorferstraße 11}, \emph{Wohngebäude (K.WHS)}|pw}.
                  }\pend{}{\bigskip}\vspace{1em}
\pstart{}{\pb}Lieber Freund!\pend\vspace{0.5em}
\pstart
           Dörmann\pwindex{Doermann, Felix 29.05.1870 – 26.10.1928@\textsc{Dörmann, Felix} (29.05.1870 – 26.10.1928), \emph{Schriftsteller/Schriftstellerin}|pw} will uns ſein neues \label{K_L00059-1v}\edtext{Buch\pwindex{Sensationen@\emph{Sensationen}|pwv}}{\lemma{\textnormal{\emph{Buch}}}\Cendnote{\textnormal{Felix Dörmann\pwindex{Doermann, Felix 29.05.1870 – 26.10.1928@\textsc{Dörmann, Felix} (29.05.1870 – 26.10.1928), \emph{Schriftsteller/Schriftstellerin}|pwk}: \emph{Sensationen}\pwindex{Sensationen@\emph{Sensationen}|pwk}. Wien: \emph{Verlag von Leopold
                        Weiss}{ }1892.}}}\label{K_L00059-1} vorleſen und hat mich gebeten, Sie einzuladen.\pend
           
\pstart
           Wenn Sie alſo \label{K_L00059-2v}\edtext{nichts beſſeres}{\lemma{\textnormal{\emph{nichts beſſeres}}}\Cendnote{\textnormal{Schnitzler war bei der Lesung.}}}\label{K_L00059-2}
               vorhaben, kommen Sie morgen Samstag, \uline{½ 8} Uhr (pünktlich) Gewerbeverein\orgindex{Oesterreichischer Gewerbeverein@Österreichischer Gewerbeverein|pw}, Eſchenbachgaſſe\oindex{Eschenbachgasse@\textbf{Eschenbachgasse}, \emph{Straße (K.STR)}|pw}, 3 Stock, im Secretariat. Es kommen Salten\pwindex{Salten, Felix 06.09.1869 – 08.10.1945@\textsc{Salten, Felix} (06.09.1869 – 08.10.1945), \emph{Schriftsteller/Schriftstellerin, Journalist/Journalistin, Chefredakteur/Chefredakteurin}|pw}, Bahr\pwindex{Bahr, Hermann 19.07.1863 – 15.01.1934@\textsc{Bahr, Hermann} (19.07.1863 – 15.01.1934), \emph{Schriftsteller/Schriftstellerin, Kritiker/Kritikerin}|pw}, Sie und
               ich. Wenn Sie nicht können, ſagen Sie bitte mir pneumatiſch ab. Ich war heute bei dem
               Leichenbegängnis von Richards\pwindex{Beer-Hofmann, Richard 1866-07-11 – 1945-09-26@\textsc{Beer-Hofmann, Richard} (1866-07-11 – 1945-09-26), \emph{Schriftsteller/Schriftstellerin}|pw}{ }Mutter\pwindex{Hofmann, Bertha 1840 – 30.12.1891@\textsc{Hofmann, Bertha} (1840 – 30.12.1891)|pwv}. Soll man ihn
               beſuchen? \pend
           
\pstart
           Herzlichſt{\\[\baselineskip]}\spacefill\mbox{Loris}\pend
           \leftskip=0em{}\selectlanguage{ngerman}\endnumbering\briefempfaengerindex{Schnitzler, Arthur@\textsc{Schnitzler, Arthur}!zzzHofmannsthal, Hugo von@\emph{von Hugo von Hofmannsthal}!1892-01-013@{1. 1. 1892}|)be}\mylabel{L00059h}  \normalsize

\doendnotes{C}
\bigskip
\vfill

\clearpage

\footnotesize

\lohead{\textsc{register}}

% Definiere theindex-Environment komplett neu ohne reledmac
\makeatletter
\renewenvironment{theindex}{%
  \section*{\indexname}%
  \setlength{\parindent}{0pt}%
  \setlength{\parskip}{0pt plus 0.3pt}%
  \let\item\@idxitem
}{%
  \clearpage
}
\makeatother

\IfFileExists{\jobname-pw.ind}{\input{\jobname-pw.ind}}{}

\end{document}

      