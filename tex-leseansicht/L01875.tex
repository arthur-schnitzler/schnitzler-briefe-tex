\input{../tex-inputs/latex-pdf-vorspann}
\begin{center}
            \textcolor{red}{ENTWURF. ENTZIFFERUNG NOCH NICHT KORREKTURGELESEN}
                      \end{center}
            
               \section[Max Burckhard an Arthur Schnitzler, 15. 9. 1909]{ Max Burckhard an Arthur Schnitzler, 15. 9. 1909}\nopagebreak\mylabel{v}\rehead{ }\begin{ledgroupsized}[t]{13cm}\normalsize\beginnumbering\briefempfaengerindex{Schnitzler, Arthur@\textsc{Schnitzler, Arthur}!zzzBurckhard, Max Eugen@\emph{von Max Eugen Burckhard}!1909-09-152@{15. 9. 1909}|(be} \toendnotes[C]{\smallbreak\pagebreak[2]} \Standort{CUL, Schnitzler, B 20.}
\physDesc{Telegramm
\newline{}maschinell
\newline{}Schnitzler: mit Bleistift datiert: »15/9 09« }\toendnotes[C]{\smallbreak}\pstart
           {\pb}x st gilgen\oindex{St. Gilgen@\textbf{St. Gilgen}|pw} 329 21/20 15{ }3 10=\pend
           \pstart
           die allerherzlichsten glueckwuensche sendet dem elternpaar\pwindex{Schnitzler, Olga 17.01.1882 – 13.01.1970@\textsc{Schnitzler, Olga} (17.01.1882 – 13.01.1970), \emph{Schauspielerin, Sängerin}|pwv} und geschwisterpaar\pwindex{Schnitzler, Heinrich 09.08.1902 – 12.07.1982@\textsc{Schnitzler, Heinrich} (09.08.1902 – 12.07.1982), \emph{Regisseur, Schauspieler}|pwv}\pwindex{Schnitzler, Lili 13.09.1909 – 26.07.1928@\textsc{Schnitzler, Lili} (13.09.1909 – 26.07.1928)|pwv} vom breitenberg\oindex{Breitenberg@\textbf{Breitenberg}|pw} ihr getreuer \spacefill\mbox{doktor burckhard +}\pend
           \endnumbering\briefempfaengerindex{Schnitzler, Arthur@\textsc{Schnitzler, Arthur}!zzzBurckhard, Max Eugen@\emph{von Max Eugen Burckhard}!1909-09-152@{15. 9. 1909}|)be}\mylabel{h}\end{ledgroupsized}  \newcommand{\dateiname}{L01875}\newcommand{\titel}{Max Burckhard an Arthur Schnitzler, 15. 9. 1909}\newcommand{\editorInnen}{Martin Anton Müller und Gerd-Hermann Susen}\input{../tex-inputs/latex-pdf-abspann}
      