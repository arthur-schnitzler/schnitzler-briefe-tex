%% latex-korrekturansicht-vorspann.tex
%% Vorspann für die Korrekturansicht.
%% Lädt die gemeinsame Datei latex-vorspann.tex mit gesetztem Schalter.

\newif\ifkorrekturansicht
\korrekturansichttrue

\input{../tex-inputs/latex-vorspann}


\section[Hugo Hofmannsthal an Arthur Schnitzler, 2. 7. 1920]{L02345 Hugo Hofmannsthal an Arthur Schnitzler, 2. 7. 1920}
\nopagebreak\mylabel{L02345v}
\rehead{ }\normalsize\beginnumbering\briefempfaengerindex{Schnitzler, Arthur@\textsc{Schnitzler, Arthur}!zzzHofmannsthal, Hugo von@\emph{von Hugo von Hofmannsthal}!1920-07-021@{2. 7. 1920}|(be}
\toendnotes[C]{\smallbreak\pagebreak[2]}\Standort{CUL, Schnitzler, B 43.}
\physDesc{Postkarte, 683 Zeichen
\newline{}Handschrift: 1) schwarze Tinte, deutsche Kurrent\hspace{1em}2) schwarze Tinte, lateinische Kurrent (\noindent{}Adresse)\hspace{1em}
\newline{}Versand: Stempel: »\nobreak{}\oindex{Rodaun@\textbf{Rodaun}, \emph{A.ADM4}|pwk}Rodaun, 2 VII 20, 2\textcolor{gray}{–7}N\nobreak{}«.  
\newline{}Ordnung: 1) mit Bleistift von Frieda
                                    Pollak\pwindex{Pollak, Frieda 08.12.1881 – 13.07.1937@\textsc{Pollak, Frieda} (08.12.1881 – 13.07.1937), \emph{Sekretär/Sekretärin}|pw} (?) mit dem Buchstaben »A«
                                 (Abgeschrieben/Abschrift) gekennzeichnet  2) mit Bleistift von unbekannter Hand nummeriert: »\strikeout{259}« 3) mit Bleistift von unbekannter Hand nummeriert:
                                    »366«}
\buchAbdrucke{\weitereDrucke{Hugo von Hofmannsthal, Arthur Schnitzler: \emph{Briefwechsel}. Frankfurt am Main: \emph{S. Fischer} 1964, S. 293.} }\toendnotes[C]{\smallbreak}\pstart{}{\pb}Herrn D\textsuperscript{r} Arthur Schnitzler\pend{}\pstart{}Wien\oindex{Wien@\textbf{Wien}, \emph{A.ADM2}|pw}\pend{}\pstart{}XVIII. Sternwartestrasse 71\oindex{Sternwartestrasse 71@\textbf{Sternwartestraße 71}, \emph{Wohngebäude (K.WHS)}|pw}\pend{}{\bigskip}\vspace{1em}
\pstart
           \raggedleft{}{\pb}Rodaun\oindex{Rodaun@\textbf{Rodaun}, \emph{A.ADM4}|pw}{ }2 VII 20.\pend
           
\pstart{}mein lieber Arthur, \pend\vspace{0.5em}
\pstart
           ich hörte daſs Sie fort waren, höre nun, daſs Sie wieder da ſind.\pend
           
\pstart
           Gerty\pwindex{Hofmannsthal, Gertrude von 16.03.1880 – 09.11.1959@\textsc{Hofmannsthal, Gertrude von} (16.03.1880 – 09.11.1959)|pw} geht am 7\textsuperscript{ten} mit den Kindern\pwindex{Zimmer, Christiane 14.05.1902 – 05.01.1987@\textsc{Zimmer, Christiane} (14.05.1902 – 05.01.1987)|pwv}\pwindex{Hofmannsthal, Raimund von 26.5.1906 – 20.03.1974@\textsc{Hofmannsthal, Raimund von} (26.5.1906 – 20.03.1974)|pwv}\pwindex{Hofmannsthal, Franz von 20.10.1903 – 13.07.1929@\textsc{Hofmannsthal, Franz von} (20.10.1903 – 13.07.1929)|pwv} nach Auſſee\oindex{Bad Aussee@\textbf{Bad Aussee}, \emph{P.PPLA3}|pw}, ich bleibe noch den
               ganzen Juli da mit {\pb}meiner Arbeit, bringe aber nichts vor mich (vorläufig) ſondern leide bei Tag u.
               Nacht unter dieſem abſurden Wetter, das es ſeit 3 Wochen verübt.\pend
           
\pstart
           Ich möchte vom 8\textsuperscript{ten} ab jeden beliebigen Tag (außer Sonntag) vormittags zeitlich zu Ihnen ko{\geminationm}en (wäre etwa 10\textsuperscript{h} dort) Sie zu einem Spaziergang abholen, etwa dann mit Euch eſſen, wenn das
               geht, oder auch nach dem Spaziergang in die Stadt fahren. Bitte telegrafiren Sie mir
                  \label{T_L02345-1v}\edtext{welchen Tag, ab 8\textsuperscript{ten}, Sie wählen.}{\lemma{\textnormal{\emph{welchen … wählen.}}}\Cendnote{\textnormal{weiter quer am linken
                  Rand}}}\label{T_L02345-1}\pend
           \pstart Ihr\spacefill\mbox{Hugo.}\pend{}\selectlanguage{ngerman}\endnumbering\briefempfaengerindex{Schnitzler, Arthur@\textsc{Schnitzler, Arthur}!zzzHofmannsthal, Hugo von@\emph{von Hugo von Hofmannsthal}!1920-07-021@{2. 7. 1920}|)be}\mylabel{L02345h}  \normalsize

\doendnotes{C}
\bigskip
\vfill

\clearpage

\footnotesize

\lohead{\textsc{register}}

% Definiere theindex-Environment komplett neu ohne reledmac
\makeatletter
\renewenvironment{theindex}{%
  \section*{\indexname}%
  \setlength{\parindent}{0pt}%
  \setlength{\parskip}{0pt plus 0.3pt}%
  \let\item\@idxitem
}{%
  \clearpage
}
\makeatother

\IfFileExists{\jobname-pw.ind}{\input{\jobname-pw.ind}}{}

\end{document}

      