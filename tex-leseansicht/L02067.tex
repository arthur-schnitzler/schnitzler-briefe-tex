%% latex-korrekturansicht-vorspann.tex
%% Vorspann für die Korrekturansicht.
%% Lädt die gemeinsame Datei latex-vorspann.tex mit gesetztem Schalter.

\newif\ifkorrekturansicht
\korrekturansichttrue

\input{../tex-inputs/latex-vorspann}


\section[Hermann Bahr an Arthur Schnitzler, 15. 5. 1912]{L02067 Hermann Bahr an Arthur Schnitzler, 15. 5. 1912}
\nopagebreak\mylabel{L02067v}
\rehead{ }\normalsize\beginnumbering\briefempfaengerindex{Schnitzler, Arthur@\textsc{Schnitzler, Arthur}!zzzBahr, Hermann@\emph{von Hermann Bahr}!1912-05-152@{15. 5. 1912}|(be}
\toendnotes[C]{\smallbreak\pagebreak[2]}\Standort{DLA, A:Schnitzler, 85.1.148.}
\physDesc{Brief, 1 Blatt, 1 Seite, 101 Zeichen
\newline{}Handschrift: schwarze Tinte, deutsche Kurrent
\newline{}Schnitzler: mit rotem Buntstift Vermerk »Bild« }
\buchAbdrucke{\weitereDrucke{1) \emph{Arthur Schnitzler. Sein Leben · Sein Werk · Seine Zeit}. Frankfurt am Main: \emph{S. Fischer} 1981, S. 103.} \weitereDrucke{2) Hans-Ulrich Lindken: \emph{Arthur Schnitzler. Aspekte und Akzente. Materialien zu Leben
                        und Werk}. Frankfurt am Main, Bern, Göttingen: \emph{Peter Lang} 1984, S. 404.} \weitereDrucke{3) Hermann Bahr, Arthur Schnitzler: \emph{Briefwechsel, Aufzeichnungen, Dokumente (1891–1931)}. Göttingen: \emph{Wallstein} 2018, S. 473.} }
\pstart
           \noindent{}{\pb}Alles Schönste, alles Beste wünscht Dir von Herzen zum heutigen Tage\pend
           
\pstart
           Dein mitalternder{\\[\baselineskip]}\spacefill\mbox{Hermann}\pend
           \leftskip=0em{}
\pstart
           15. 5. 12\pend
           \selectlanguage{ngerman}\endnumbering\briefempfaengerindex{Schnitzler, Arthur@\textsc{Schnitzler, Arthur}!zzzBahr, Hermann@\emph{von Hermann Bahr}!1912-05-152@{15. 5. 1912}|)be}\mylabel{L02067h}  \normalsize

\doendnotes{C}
\bigskip
\vfill

\clearpage

\footnotesize

\lohead{\textsc{register}}

% Definiere theindex-Environment komplett neu ohne reledmac
\makeatletter
\renewenvironment{theindex}{%
  \section*{\indexname}%
  \setlength{\parindent}{0pt}%
  \setlength{\parskip}{0pt plus 0.3pt}%
  \let\item\@idxitem
}{%
  \clearpage
}
\makeatother

\IfFileExists{\jobname-pw.ind}{\input{\jobname-pw.ind}}{}

\end{document}

      