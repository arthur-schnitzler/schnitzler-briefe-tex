%% latex-leseansicht-vorspann.tex
%% Vorspann für die Leseansicht.
%% Lädt die gemeinsame Datei latex-vorspann.tex mit nicht gesetztem Schalter.

\newif\ifkorrekturansicht
\korrekturansichtfalse

\input{../tex-inputs/latex-vorspann}


         
         \renewcommand{\erwaehntePersonen}{Personen: Robert Adam, Alexandre père Dumas}
         \renewcommand{\erwaehnteOrte}{Orte: Meidlinger Hauptstraße, Sternwartestraße, Wien, XII., Meidling}
         \renewcommand{\erwaehnteWerke}{Werke: Jean Christophe, Meine Memoiren, Wundervogel}
               \section[Arthur Schnitzler an Robert Adam, 23. 11. 1916]{ Arthur Schnitzler an Robert Adam, 23. 11. 1916}\nopagebreak\mylabel{v}\rehead{ }\begin{ledgroupsized}[t]{13cm}\normalsize\beginnumbering \toendnotes[C]{\smallbreak\pagebreak[2]} \Standort{DLA, 96.34.1/23.}
\physDesc{Briefkarte, Umschlag
\newline{}Schreibmaschine
\newline{}Handschrift: schwarze Tinte (\noindent{}Unterschrift)}\toendnotes[C]{\smallbreak}\pstart{}{\pb}\textcolor{gray}{\textbf{Dr. Arthur Schnitzler}}\pend{}\pstart{}\textcolor{gray}{\textbf{Wien XVIII. Sternwartestrasse 71\oindex{Sternwartestrasse@\textbf{Sternwartestraße}|pw}}}\pend{}{\bigskip}\pstart{}{\pb}Herrn Dr. Robert Adam Pollak\pend{}\pstart{}Wien XII\oindex{XII., Meidling@\textbf{XII., Meidling}|pw}.\pend{}\pstart{}Meidlinger Hauptstrasse 58\oindex{Meidlinger Hauptstrasse@\textbf{Meidlinger Hauptstraße}|pw}.\pend{}{\bigskip}\pstart
           \noindent{}{\pb}\textcolor{gray}{\textbf{Dr. Arthur Schnitzler}}\hfill 23. 11. 1916.\pend
           \pstart
           \textcolor{gray}{\textbf{Wien XVIII. Sternwartestrasse 71\oindex{Sternwartestrasse@\textbf{Sternwartestraße}|pw}}}\pend
           \pstart\center{}Verehrter Herr Doktor.\pend\pstart
           Ohne mich gerade in prosodischen Fragen sonderlich kompetent zu fühlen, bin ich
                    selbstverständlich mit Vergnügen bereit Ihnen unverbindlich den gewünschten
                    Ratschlag zu erteilen. Vielleicht haben Sie die Freundlichkeit mir das Manuscript\pwindex{Adam, Robert 20.04.1877 – 16.10.1961@\textsc{Adam, Robert} (20.04.1877 – 16.10.1961), \emph{Schriftsteller, Richter}!WundervogelNone@\strich\emph{Wundervogel} {[}None{]}|pwv} sowohl die
                    Alexandriner wie die Knittelverse zu übersenden und bald nachdem ich sie gelesen
                    wollen wir uns persönlich darüber weiter unterhalten. Ueber alles andere, Jean Christophe\pwindex{\textcolor{red}{\textsuperscript{XXXX1 indx}}!Jean Christophe1904 – 1912@\strich\emph{Jean Christophe} {[}1904 – 1912{]}|pw} und Dumas\pwindex{Dumas, Alexandre pere 24.07.1802 – 05.12.1870@\textsc{Dumas, Alexandre père} (24.07.1802 – 05.12.1870), \emph{Schriftsteller}!Meine Memoiren1852 – 1856@\strich\emph{Meine Memoiren} {[}1852 – 1856{]}|pwv}\pwindex{Dumas, Alexandre pere 24.07.1802 – 05.12.1870@\textsc{Dumas, Alexandre père} (24.07.1802 – 05.12.1870), \emph{Schriftsteller}|pw} mündlich.\pend
           \pstart
           Herzlichst grüs{[}s{]}end{\\[\baselineskip]}Ihr{\\[\baselineskip]}\spacefill\mbox{{[}hs.:{]} Arthur Schnitzler}\pend
           \leftskip=0em{}
         
         \endnumbering\mylabel{h}\end{ledgroupsized}  \newcommand{\dateiname}{L02247}\newcommand{\titel}{Arthur Schnitzler an Robert Adam, 23. 11. 1916}\newcommand{\editorInnen}{Martin Anton Müller und Gerd-Hermann Susen}%% latex-leseansicht-abspann.tex
%% Abspann für die Leseansicht.
%% Der Schalter \ifkorrekturansicht ist bereits durch den Vorspann gesetzt.

%% latex-abspann.tex
%% Gemeinsamer Abspann für Korrekturansicht und Leseansicht.
%% Setzt den Schalter \ifkorrekturansicht voraus (gesetzt in den
%% einbindenden Dateien latex-korrekturansicht-abspann.tex bzw.
%% latex-leseansicht-abspann.tex).
%% ---------------------------------------------------------------

\normalsize

% Das esempio-Environment wird nur in der Leseansicht benötigt
\ifkorrekturansicht\else
\newenvironment{esempio}[3]%
{
    \vspace{1.5ex}
    \rlap{\underline{#1}}
    \par
    \setlength{\parindent}{0cm}
    \nopagebreak
    \leftskip=#2cm
    \rightskip=#3cm
}
{
    \par
}
\fi

\doendnotes{C}
\bigskip
\vfill

\clearpage

\footnotesize

\ifkorrekturansicht
  \lohead{\textsc{register}}
\fi

% theindex-Environment neu definieren ohne reledmac
\makeatletter
\renewenvironment{theindex}{%
  \ifkorrekturansicht
    \section*{\indexname}%
  \else
    \subsubsection*{Index der erwähnten Entitäten}%
  \fi
  \setlength{\parindent}{0pt}%
  \setlength{\parskip}{0pt plus 0.3pt}%
  \let\item\@idxitem
}{%
  \ifkorrekturansicht\clearpage\fi
}
\makeatother

\IfFileExists{\jobname-pw.ind}{\input{\jobname-pw.ind}}{}

% Quellenangabe nur in der Leseansicht
\ifkorrekturansicht\else
% Fallback-Definitionen, falls die .tex-Datei \titel etc. nicht gesetzt hat
\providecommand{\titel}{}
\providecommand{\editorInnen}{}
\providecommand{\dateiname}{\jobname}

\vspace{3cm}

\vfill

\footnotesize
\textsc{Quelle}: \titel. Herausgegeben von {\editorInnen}. In: \emph{Arthur Schnitzler: Briefwechsel mit Autorinnen und Autoren}.
 Digitale Edition, https://schnitzler-briefe.acdh.oeaw.ac.at/{\dateiname}.html (Stand \today)
\fi

\end{document}


      