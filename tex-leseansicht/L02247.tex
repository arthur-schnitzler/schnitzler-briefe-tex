%% latex-leseansicht-vorspann.tex
%% Vorspann für die Leseansicht.
%% Lädt die gemeinsame Datei latex-vorspann.tex mit nicht gesetztem Schalter.

\newif\ifkorrekturansicht
\korrekturansichtfalse

\input{../tex-inputs/latex-vorspann}


\section[Arthur Schnitzler an Robert Adam, 23. 11. 1916]{L02247 Arthur Schnitzler an Robert Adam, 23. 11. 1916}
\nopagebreak\mylabel{L02247v}
\rehead{ }\normalsize\beginnumbering\briefempfaengerindex{Adam, Robert@\textsc{Adam, Robert}!zzzSchnitzler, Arthur@\emph{von Arthur Schnitzler}!1916-11-231@{23. 11. 1916}|(be}
\toendnotes[C]{\smallbreak\pagebreak[2]}
\correspDesc{Versand  durch Arthur Schnitzler am 23. 11. 1916 in Wien
\newline{}Erhalt  durch Robert Adam im Zeitraum [23. 11. 1916 – 27. 11. 1916?] in Wien}\toendnotes[C]{\smallbreak}
\Standort{DLA, 96.34.1/23.}
\physDesc{Briefkarte, , Kuvert, 571 Zeichen
\newline{}Schreibmaschine
\newline{}Handschrift: schwarze Tinte (\noindent{}Unterschrift)}\toendnotes[C]{\smallbreak}\pstart{}{\pb}\textcolor{gray}{\textbf{Dr. Arthur Schnitzler}}\pend{}\pstart{}\textcolor{gray}{\textbf{Wien XVIII. Sternwartestrasse 71\oindex{Wien@\textbf{Wien}!XVIII., Währing@\textbf{XVIII., Währing}!Sternwartestraße 71@\textbf{Sternwartestraße 71}, \emph{Wohngebäude}|pw}}}\pend{}{\bigskip}\pstart{}{\pb}Herrn Dr. Robert Adam Pollak\pend{}\pstart{}Wien XII\oindex{XII., Meidling@\textbf{XII., Meidling}, \emph{Verwaltungsgebiet}|pw}.\pend{}\pstart{}Meidlinger Hauptstrasse 58\oindex{Wien@\textbf{Wien}!XII., Meidling@\textbf{XII., Meidling}!Meidlinger Hauptstraße@\textbf{Meidlinger Hauptstraße}, \emph{Straße}|pw}.\pend{}{\bigskip}\vspace{1em}
\pstart
           {\pb}\textcolor{gray}{\textbf{Dr. Arthur Schnitzler}}\hfill 23. 11. 1916.\pend
           
\pstart
           \textcolor{gray}{\textbf{Wien XVIII. Sternwartestrasse 71\oindex{Wien@\textbf{Wien}!XVIII., Währing@\textbf{XVIII., Währing}!Sternwartestraße 71@\textbf{Sternwartestraße 71}, \emph{Wohngebäude}|pw}}}\pend
           
\pstart\center{}Verehrter Herr Doktor.\pend\vspace{0.5em}
\pstart
           Ohne mich gerade in prosodischen Fragen sonderlich kompetent zu fühlen, bin ich
               selbstverständlich mit Vergnügen bereit Ihnen unverbindlich den gewünschten Ratschlag
               zu erteilen. Vielleicht haben Sie die Freundlichkeit mir das Manuscript\pwindex{Adam, Robert 20.\,4.\,1877 Wien – 16.\,10.\,1961 Baden bei Wien@\textsc{Adam, Robert} (20.\,4.\,1877 Wien – 16.\,10.\,1961 Baden bei Wien), \emph{Schriftsteller, Richter}!Wundervogel@\strich\emph{Wundervogel}|pwv} sowohl die Alexandriner wie die
               Knittelverse zu übersenden und bald nachdem ich sie gelesen wollen wir uns persönlich
               darüber weiter unterhalten. Ueber alles andere, Jean
                  Christophe\pwindex{\textcolor{red}{\textsuperscript{XXXX indx1}}!Jean-Christophe@\strich\emph{Jean-Christophe}|pw} und Dumas\pwindex{Dumas, Alexandre père 24.\,7.\,1802 Villers-Cotterêts – 5.\,12.\,1870 Puys@\textsc{Dumas, Alexandre père} (24.\,7.\,1802 Villers-Cotterêts – 5.\,12.\,1870 Puys), \emph{Schriftsteller}!Meine Memoiren@\strich\emph{Meine Memoiren}|pwv}\pwindex{Dumas, Alexandre père 24.\,7.\,1802 Villers-Cotterêts – 5.\,12.\,1870 Puys@\textsc{Dumas, Alexandre père} (24.\,7.\,1802 Villers-Cotterêts – 5.\,12.\,1870 Puys), \emph{Schriftsteller}|pw} mündlich.\pend
           
\pstart
           Herzlichst grüs{[}s{]}end{\\[\baselineskip]}Ihr{\\[\baselineskip]}\spacefill\mbox{{[}hs.:{]} Arthur Schnitzler}\pend
           \leftskip=0em{}\selectlanguage{ngerman}\endnumbering\briefempfaengerindex{Adam, Robert@\textsc{Adam, Robert}!zzzSchnitzler, Arthur@\emph{von Arthur Schnitzler}!1916-11-231@{23. 11. 1916}|)be}\mylabel{L02247h}  \newcommand{\dateiname}{L02247}\newcommand{\titel}{Arthur Schnitzler an Robert Adam, 23. 11. 1916}\newcommand{\editorInnen}{Martin Anton Müller und Gerd-Hermann Susen}%% latex-leseansicht-abspann.tex
%% Abspann für die Leseansicht.
%% Der Schalter \ifkorrekturansicht ist bereits durch den Vorspann gesetzt.

%% latex-abspann.tex
%% Gemeinsamer Abspann für Korrekturansicht und Leseansicht.
%% Setzt den Schalter \ifkorrekturansicht voraus (gesetzt in den
%% einbindenden Dateien latex-korrekturansicht-abspann.tex bzw.
%% latex-leseansicht-abspann.tex).
%% ---------------------------------------------------------------

\normalsize

% Das esempio-Environment wird nur in der Leseansicht benötigt
\ifkorrekturansicht\else
\newenvironment{esempio}[3]%
{
    \vspace{1.5ex}
    \rlap{\underline{#1}}
    \par
    \setlength{\parindent}{0cm}
    \nopagebreak
    \leftskip=#2cm
    \rightskip=#3cm
}
{
    \par
}
\fi

\doendnotes{C}
\bigskip
\vfill

\clearpage

\footnotesize

\ifkorrekturansicht
  \lohead{\textsc{register}}
\fi

% theindex-Environment neu definieren ohne reledmac
\makeatletter
\renewenvironment{theindex}{%
  \ifkorrekturansicht
    \section*{\indexname}%
  \else
    \subsubsection*{Index der erwähnten Entitäten}%
  \fi
  \setlength{\parindent}{0pt}%
  \setlength{\parskip}{0pt plus 0.3pt}%
  \let\item\@idxitem
}{%
  \ifkorrekturansicht\clearpage\fi
}
\makeatother

\IfFileExists{\jobname-pw.ind}{\input{\jobname-pw.ind}}{}

% Quellenangabe nur in der Leseansicht
\ifkorrekturansicht\else
% Fallback-Definitionen, falls die .tex-Datei \titel etc. nicht gesetzt hat
\providecommand{\titel}{}
\providecommand{\editorInnen}{}
\providecommand{\dateiname}{\jobname}

\vspace{3cm}

\vfill

\footnotesize
\textsc{Quelle}: \titel. Herausgegeben von {\editorInnen}. In: \emph{Arthur Schnitzler: Briefwechsel mit Autorinnen und Autoren}.
 Digitale Edition, https://schnitzler-briefe.acdh.oeaw.ac.at/{\dateiname}.html (Stand \today)
\fi

\end{document}


