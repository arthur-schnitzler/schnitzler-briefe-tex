%% latex-leseansicht-vorspann.tex
%% Vorspann für die Leseansicht.
%% Lädt die gemeinsame Datei latex-vorspann.tex mit nicht gesetztem Schalter.

\newif\ifkorrekturansicht
\korrekturansichtfalse

\input{../tex-inputs/latex-vorspann}


         
         \newcommand{\erwaehntePersonen}{Personen: Richard Beer-Hofmann, Felix Salten, Gustav Schwarzkopf, Henriette Schönwald}
         \newcommand{\erwaehnteInstitutionen}{}
         \newcommand{\erwaehnteOrte}{Orte: Altaussee, Bad Aussee, Bad Ischl, Puchen, Salzburg, Wien}
         \newcommand{\erwaehnteWerke}{Werke: Ausseer Cur- und Fremden-Liste}
               \section[Arthur Schnitzler an Richard Beer-Hofmann, 5. 8. 1900]{ Arthur Schnitzler an Richard Beer-Hofmann, 5. 8. 1900}\nopagebreak\mylabel{v}\rehead{ }\begin{ledgroupsized}[t]{13cm}\normalsize\beginnumbering \toendnotes[C]{\smallbreak\pagebreak[2]} \Standort{YCGL, MSS 31.}
\physDesc{Postkarte
\newline{}Handschrift: Bleistift, deutsche Kurrent\newline{}Versand: 1) Stempel: »\nobreak{}\oindex{Bad Ischl@\textbf{Bad Ischl}|pwk}Ischl, 6. 8. 00, 6–7V\nobreak{}«.   2) Stempel: »\nobreak{}\oindex{Altaussee@\textbf{Altaussee}|pwk}\textcolor{gray}{Alt-Aussee}, 6. {[}8.{]} \textcolor{gray}{00}\nobreak{}«. }\toendnotes[C]{\smallbreak}\pstart{}{\pb}Hn Dr. \textsc{Richard}\pend{}\pstart{}\textsc{Beer-Hofmann}\pend{}\pstart{}\textsc{Altaussee}\oindex{Altaussee@\textbf{Altaussee}|pw}\pend{}{\bigskip}\pstart
           \raggedleft{}{\pb}5. 8. 900\pend
           \pstart
           lieber Richard, Guſtav\pwindex{Schwarzkopf, Gustav 07.11.1853 – 13.11.1939@\textsc{Schwarzkopf, Gustav} (07.11.1853 – 13.11.1939), \emph{Schriftsteller}|pw} ko{\geminationm}t
                    nicht nach Salzburg\oindex{Salzburg@\textbf{Salzburg}|pw}. Von Salten\pwindex{Salten, Felix 06.09.1869 – 08.10.1945@\textsc{Salten, Felix} (06.09.1869 – 08.10.1945), \emph{Schriftsteller, Journalist}|pw} keine Nachricht. Wir können nun wohl Montag
                        13.{ }Nachmittag 3 Uhr endgiltg abreiſen und S.\oindex{Salzburg@\textbf{Salzburg}|pw} feſtſetzen. Vielleicht ko{\geminationm} ich \label{K_L1900-08-05_01_AS_Beer-Hofmann_1v}\edtext{Mittwoch}{\lemma{\textnormal{\emph{Mittwoch}}}\Cendnote{\textnormal{siehe A. S.: \emph{Tagebuch}, 8. 8. 1900}}}\label{K_L1900-08-05_01_AS_Beer-Hofmann_1h} nach Auſſee\oindex{Bad Aussee@\textbf{Bad Aussee}|pw}, würde Ihnen noch früher
                    ſchreiben \textsc{resp.} telegr.\pend
           \pstart
           Herzlichſt{\\[\baselineskip]}\spacefill\mbox{Arth}\pend
           \leftskip=0em{}\pstart
           \noindent{}\label{T_L1900-08-05_01_AS_Beer-Hofmann_1v}\edtext{\label{K_L1900-08-05_01_AS_Beer-Hofmann_2v}\edtext{Wer wohnt
                            Puchen\oindex{Puchen@\textbf{Puchen}|pw} Nr. 57?}{\lemma{\textnormal{\emph{Wer wohnt
                            Puchen Nr. 57?}}}\Cendnote{\textnormal{Laut \emph{Ausseer Cur- und Fremden-Liste}\pwindex{?? Werk@Nicht ermittelte Verfasserinnen und Verfasser!Ausseer Cur- und Fremden-Liste1894-05-26@\emph{Ausseer Cur- und Fremden-Liste} {[}1894-05-26{]}|pwk} (Nr. 15,
                                    18. 7. 1900, S. 2) »Frau Henriette Schönwald\pwindex{Schoenwald, Henriette †~1906-03-23@\textsc{Schönwald, Henriette} (†~1906-03-23)|pw},
                                Private, mit Familie und Dienerschaft, aus Wien\oindex{Wien@\textbf{Wien}|pw}« (6 Personen)}}}\label{K_L1900-08-05_01_AS_Beer-Hofmann_2h}}{\lemma{\textnormal{\emph{Wer wohnt
                            Puchen Nr. 57?}}}\Cendnote{\textnormal{am oberen
                            Rand auf dem Kopf}}}\label{T_L1900-08-05_01_AS_Beer-Hofmann_1h}\pend
           
         
         \endnumbering\mylabel{h}\end{ledgroupsized}  \newcommand{\dateiname}{L01066}\newcommand{\titel}{Arthur Schnitzler an Richard Beer-Hofmann, 5. 8. 1900}\newcommand{\editorInnen}{Martin Anton Müller und Gerd-Hermann Susen}%% latex-leseansicht-abspann.tex
%% Abspann für die Leseansicht.
%% Der Schalter \ifkorrekturansicht ist bereits durch den Vorspann gesetzt.

%% latex-abspann.tex
%% Gemeinsamer Abspann für Korrekturansicht und Leseansicht.
%% Setzt den Schalter \ifkorrekturansicht voraus (gesetzt in den
%% einbindenden Dateien latex-korrekturansicht-abspann.tex bzw.
%% latex-leseansicht-abspann.tex).
%% ---------------------------------------------------------------

\normalsize

% Das esempio-Environment wird nur in der Leseansicht benötigt
\ifkorrekturansicht\else
\newenvironment{esempio}[3]%
{
    \vspace{1.5ex}
    \rlap{\underline{#1}}
    \par
    \setlength{\parindent}{0cm}
    \nopagebreak
    \leftskip=#2cm
    \rightskip=#3cm
}
{
    \par
}
\fi

\doendnotes{C}
\bigskip
\vfill

\clearpage

\footnotesize

\ifkorrekturansicht
  \lohead{\textsc{register}}
\fi

% theindex-Environment neu definieren ohne reledmac
\makeatletter
\renewenvironment{theindex}{%
  \ifkorrekturansicht
    \section*{\indexname}%
  \else
    \subsubsection*{Index der erwähnten Entitäten}%
  \fi
  \setlength{\parindent}{0pt}%
  \setlength{\parskip}{0pt plus 0.3pt}%
  \let\item\@idxitem
}{%
  \ifkorrekturansicht\clearpage\fi
}
\makeatother

\IfFileExists{\jobname-pw.ind}{\input{\jobname-pw.ind}}{}

% Quellenangabe nur in der Leseansicht
\ifkorrekturansicht\else
% Fallback-Definitionen, falls die .tex-Datei \titel etc. nicht gesetzt hat
\providecommand{\titel}{}
\providecommand{\editorInnen}{}
\providecommand{\dateiname}{\jobname}

\vspace{3cm}

\vfill

\footnotesize
\textsc{Quelle}: \titel. Herausgegeben von {\editorInnen}. In: \emph{Arthur Schnitzler: Briefwechsel mit Autorinnen und Autoren}.
 Digitale Edition, https://schnitzler-briefe.acdh.oeaw.ac.at/{\dateiname}.html (Stand \today)
\fi

\end{document}


      