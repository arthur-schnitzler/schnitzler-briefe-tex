%% latex-korrekturansicht-vorspann.tex
%% Vorspann für die Korrekturansicht.
%% Lädt die gemeinsame Datei latex-vorspann.tex mit gesetztem Schalter.

\newif\ifkorrekturansicht
\korrekturansichttrue

\input{../tex-inputs/latex-vorspann}


\section[Arthur Schnitzler an Richard Beer-Hofmann, 5. 8. 1900]{L01066 Arthur Schnitzler an Richard Beer-Hofmann, 5. 8. 1900}
\nopagebreak\mylabel{L01066v}
\rehead{ }\normalsize\beginnumbering\briefempfaengerindex{Beer-Hofmann, Richard@\textsc{Beer-Hofmann, Richard}!zzzSchnitzler, Arthur@\emph{von Arthur Schnitzler}!1900-08-051@{5. 8. 1900}|(be}
\toendnotes[C]{\smallbreak\pagebreak[2]}\Standort{YCGL, MSS 31.}
\physDesc{Postkarte, 315 Zeichen
\newline{}Handschrift: Bleistift, deutsche Kurrent
\newline{}Versand: 1) Stempel: »\nobreak{}\oindex{Bad Ischl@\textbf{Bad Ischl}, \emph{P.PPL}|pwk}Ischl, 6. 8. 00, 6–7V\nobreak{}«.   2) Stempel: »\nobreak{}\oindex{Altaussee@\textbf{Altaussee}, \emph{A.ADM3}|pwk}\textcolor{gray}{Alt-Aussee}, 6. {[}8.{]}\textcolor{gray}{00}\nobreak{}«. }\toendnotes[C]{\smallbreak}\pstart{}{\pb}Hn Dr. \textsc{Richard}\pend{}\pstart{}\textsc{Beer-Hofmann}\pend{}\pstart{}\textsc{Altaussee}\oindex{Altaussee@\textbf{Altaussee}, \emph{A.ADM3}|pw}\pend{}{\bigskip}\vspace{1em}
\pstart
           \raggedleft{}{\pb}5. 8. 900\pend
           \vspace{0.5em}
\pstart
           lieber Richard,{ }Guſtav\pwindex{Schwarzkopf, Gustav 07.11.1853 – 13.11.1939@\textsc{Schwarzkopf, Gustav} (07.11.1853 – 13.11.1939), \emph{Schriftsteller/Schriftstellerin}|pw} ko{\geminationm}t nicht
               nach Salzburg\oindex{Salzburg@\textbf{Salzburg}, \emph{A.ADM2}|pw}. Von Salten\pwindex{Salten, Felix 06.09.1869 – 08.10.1945@\textsc{Salten, Felix} (06.09.1869 – 08.10.1945), \emph{Schriftsteller/Schriftstellerin, Journalist/Journalistin, Chefredakteur/Chefredakteurin}|pw} keine Nachricht. Wir können nun wohl Montag
                  13.{ }Nachmittag 3 Uhr endgiltg abreiſen und S.\oindex{Salzburg@\textbf{Salzburg}, \emph{A.ADM2}|pw} feſtſetzen. Vielleicht ko{\geminationm} ich \label{K_L01066-1v}\edtext{Mittwoch}{\lemma{\textnormal{\emph{Mittwoch}}}\Cendnote{\textnormal{Siehe A. S.: \emph{Tagebuch}, 8. 8. 1900.
               }}}\label{K_L01066-1} nach Auſſee\oindex{Bad Aussee@\textbf{Bad Aussee}, \emph{P.PPLA3}|pw}, würde Ihnen noch früher
               ſchreiben \textsc{resp.} telegr.\pend
           
\pstart
           Herzlichſt{\\[\baselineskip]}\spacefill\mbox{Arth}\pend
           \leftskip=0em{}
\pstart
           \noindent{}\label{T_L01066-1v}\edtext{\label{K_L01066-2v}\edtext{Wer wohnt
                     Puchen\oindex{Puchen@\textbf{Puchen}, \emph{P.PPL}|pw} Nr. 57?}{\lemma{\textnormal{\emph{Wer wohnt
                     Puchen Nr. 57?}}}\Cendnote{\textnormal{Laut \emph{Ausseer Cur- und Fremden-Liste}\pwindex{Ausseer Cur- und Fremden-Liste@\emph{Ausseer Cur- und Fremden-Liste}|pwk} (Nr. 15,
                           18. 7. 1900, S. 2) »Frau Henriette Schönwald\pwindex{Schoenwald, Henriette †~1906-03-23@\textsc{Schönwald, Henriette} (†~1906-03-23)|pw}, Private,
                        mit Familie und Dienerschaft, aus Wien\oindex{Wien@\textbf{Wien}, \emph{A.ADM2}|pw}« (6 Personen)}}}\label{K_L01066-2}}{\lemma{\textnormal{\emph{Wer wohnt
                     Puchen Nr. 57?}}}\Cendnote{\textnormal{am oberen Rand
                     auf dem Kopf}}}\label{T_L01066-1}\pend
           \selectlanguage{ngerman}\endnumbering\briefempfaengerindex{Beer-Hofmann, Richard@\textsc{Beer-Hofmann, Richard}!zzzSchnitzler, Arthur@\emph{von Arthur Schnitzler}!1900-08-051@{5. 8. 1900}|)be}\mylabel{L01066h}  \normalsize

\doendnotes{C}
\bigskip
\vfill

\clearpage

\footnotesize

\lohead{\textsc{register}}

% Definiere theindex-Environment komplett neu ohne reledmac
\makeatletter
\renewenvironment{theindex}{%
  \section*{\indexname}%
  \setlength{\parindent}{0pt}%
  \setlength{\parskip}{0pt plus 0.3pt}%
  \let\item\@idxitem
}{%
  \clearpage
}
\makeatother

\IfFileExists{\jobname-pw.ind}{\input{\jobname-pw.ind}}{}

\end{document}

      