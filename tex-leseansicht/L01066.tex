%% latex-leseansicht-vorspann.tex
%% Vorspann für die Leseansicht.
%% Lädt die gemeinsame Datei latex-vorspann.tex mit nicht gesetztem Schalter.

\newif\ifkorrekturansicht
\korrekturansichtfalse

\input{../tex-inputs/latex-vorspann}


\section[Arthur Schnitzler an Richard Beer-Hofmann, 5. 8. 1900]{L01066 Arthur Schnitzler an Richard Beer-Hofmann, 5. 8. 1900}
\nopagebreak\mylabel{L01066v}
\rehead{ }\normalsize\beginnumbering\briefempfaengerindex{Beer-Hofmann, Richard@\textsc{Beer-Hofmann, Richard}!zzzSchnitzler, Arthur@\emph{von Arthur Schnitzler}!1900-08-051@{5. 8. 1900}|(be}
\toendnotes[C]{\smallbreak\pagebreak[2]}
\correspDesc{Versand  durch Arthur Schnitzler am 5. 8. 1900 in Bad Ischl
\newline{}Übermittlung  am 6. 8. 1900 in Bad Ischl
\newline{}Erhalt  durch Richard Beer-Hofmann am 6. 8. 1900 in Altaussee}\toendnotes[C]{\smallbreak}
\Standort{YCGL, MSS 31.}
\physDesc{Postkarte, 315 Zeichen
\newline{}Handschrift: Bleistift, deutsche Kurrent
\newline{}Versand: 1) Stempel: »\nobreak{}\oindex{Bad Ischl@\textbf{Bad Ischl}|pwk}Ischl, 6. 8. 00, 6–7V\nobreak{}«.   2) Stempel: »\nobreak{}\oindex{Altaussee@\textbf{Altaussee}, \emph{Verwaltungsgebiet}|pwk}\textcolor{gray}{Alt-Aussee}, 6. {[}8.{]}\textcolor{gray}{00}\nobreak{}«. }\toendnotes[C]{\smallbreak}\pstart{}{\pb}Hn Dr. \textsc{Richard}\pend{}\pstart{}\textsc{Beer-Hofmann}\pend{}\pstart{}\textsc{Altaussee}\oindex{Altaussee@\textbf{Altaussee}, \emph{Verwaltungsgebiet}|pw}\pend{}{\bigskip}\vspace{1em}
\pstart
           \raggedleft{}{\pb}5. 8. 900\pend
           \vspace{0.5em}
\pstart
           lieber Richard,{ }Guſtav\pwindex{Schwarzkopf, Gustav 7.\,11.\,1853 Wien – 13.\,11.\,1939 ebd.@\textsc{Schwarzkopf, Gustav} (7.\,11.\,1853 Wien – 13.\,11.\,1939 ebd.), \emph{Schriftsteller}|pw} ko{\geminationm}t nicht
               nach Salzburg\oindex{Salzburg@\textbf{Salzburg}, \emph{Verwaltungsgebiet}|pw}. Von Salten\pwindex{Salten, Felix 6.\,9.\,1869 Budapest – 8.\,10.\,1945 Zürich@\textsc{Salten, Felix} (6.\,9.\,1869 Budapest – 8.\,10.\,1945 Zürich), \emph{Schriftsteller, Journalist, Chefredakteur}|pw} keine Nachricht. Wir können nun wohl Montag 13.{ }Nachmittag 3 Uhr endgiltg abreiſen und S.\oindex{Salzburg@\textbf{Salzburg}, \emph{Verwaltungsgebiet}|pw} feſtſetzen. Vielleicht ko{\geminationm} ich \label{K_L01066-1v}\edtext{Mittwoch}{\lemma{\textnormal{\emph{Mittwoch}}}\Cendnote{\textnormal{Siehe A. S.: \emph{Tagebuch}, 8. 8. 1900.
               }}}\label{K_L01066-1} nach Auſſee\oindex{Bad Aussee@\textbf{Bad Aussee}, \emph{Hauptstadt}|pw}, würde Ihnen noch früher{ }ſchreiben \textsc{resp.} telegr.\pend
           
\pstart
           Herzlichſt{\\[\baselineskip]}\spacefill\mbox{Arth}\pend
           \leftskip=0em{}
\pstart
           \noindent{}\label{T_L01066-1v}\edtext{\label{K_L01066-2v}\edtext{Wer wohnt
                     Puchen\oindex{Puchen@\textbf{Puchen}|pw} Nr. 57?}{\lemma{\textnormal{\emph{Wer wohnt
                     Puchen Nr. 57?}}}\Cendnote{\textnormal{Laut \emph{Ausseer Cur- und Fremden-Liste}\pwindex{Ausseer Cur- und Fremden-Liste@\emph{Ausseer Cur- und Fremden-Liste}|pwk} (Nr. 15,
                           18. 7. 1900, S. 2) »Frau Henriette Schönwald\pwindex{Schönwald, Henriette †~23.\,3.\,1906 Wien@\textsc{Schönwald, Henriette} (†~23.\,3.\,1906 Wien)|pw}, Private,
                        mit Familie und Dienerschaft, aus Wien\oindex{Wien@\textbf{Wien}, \emph{Verwaltungsgebiet}|pw}« (6 Personen)}}}\label{K_L01066-2}}{\lemma{\textnormal{\emph{Wer wohnt
                     Puchen Nr. 57?}}}\Cendnote{\textnormal{am oberen Rand
                     auf dem Kopf}}}\label{T_L01066-1}\pend
           \selectlanguage{ngerman}\endnumbering\briefempfaengerindex{Beer-Hofmann, Richard@\textsc{Beer-Hofmann, Richard}!zzzSchnitzler, Arthur@\emph{von Arthur Schnitzler}!1900-08-051@{5. 8. 1900}|)be}\mylabel{L01066h}  \newcommand{\dateiname}{L01066}\newcommand{\titel}{Arthur Schnitzler an Richard Beer-Hofmann, 5. 8. 1900}\newcommand{\editorInnen}{Martin Anton Müller und Gerd-Hermann Susen}%% latex-leseansicht-abspann.tex
%% Abspann für die Leseansicht.
%% Der Schalter \ifkorrekturansicht ist bereits durch den Vorspann gesetzt.

%% latex-abspann.tex
%% Gemeinsamer Abspann für Korrekturansicht und Leseansicht.
%% Setzt den Schalter \ifkorrekturansicht voraus (gesetzt in den
%% einbindenden Dateien latex-korrekturansicht-abspann.tex bzw.
%% latex-leseansicht-abspann.tex).
%% ---------------------------------------------------------------

\normalsize

% Das esempio-Environment wird nur in der Leseansicht benötigt
\ifkorrekturansicht\else
\newenvironment{esempio}[3]%
{
    \vspace{1.5ex}
    \rlap{\underline{#1}}
    \par
    \setlength{\parindent}{0cm}
    \nopagebreak
    \leftskip=#2cm
    \rightskip=#3cm
}
{
    \par
}
\fi

\doendnotes{C}
\bigskip
\vfill

\clearpage

\footnotesize

\ifkorrekturansicht
  \lohead{\textsc{register}}
\fi

% theindex-Environment neu definieren ohne reledmac
\makeatletter
\renewenvironment{theindex}{%
  \ifkorrekturansicht
    \section*{\indexname}%
  \else
    \subsubsection*{Index der erwähnten Entitäten}%
  \fi
  \setlength{\parindent}{0pt}%
  \setlength{\parskip}{0pt plus 0.3pt}%
  \let\item\@idxitem
}{%
  \ifkorrekturansicht\clearpage\fi
}
\makeatother

\IfFileExists{\jobname-pw.ind}{\input{\jobname-pw.ind}}{}

% Quellenangabe nur in der Leseansicht
\ifkorrekturansicht\else
% Fallback-Definitionen, falls die .tex-Datei \titel etc. nicht gesetzt hat
\providecommand{\titel}{}
\providecommand{\editorInnen}{}
\providecommand{\dateiname}{\jobname}

\vspace{3cm}

\vfill

\footnotesize
\textsc{Quelle}: \titel. Herausgegeben von {\editorInnen}. In: \emph{Arthur Schnitzler: Briefwechsel mit Autorinnen und Autoren}.
 Digitale Edition, https://schnitzler-briefe.acdh.oeaw.ac.at/{\dateiname}.html (Stand \today)
\fi

\end{document}


