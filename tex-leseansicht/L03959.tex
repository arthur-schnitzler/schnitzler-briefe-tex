%% latex-leseansicht-vorspann.tex
%% Vorspann für die Leseansicht.
%% Lädt die gemeinsame Datei latex-vorspann.tex mit nicht gesetztem Schalter.

\newif\ifkorrekturansicht
\korrekturansichtfalse

\input{../tex-inputs/latex-vorspann}


\section[Arthur Schnitzler an Berta Zuckerkandl, 15. 6. 1925]{L03959 Arthur Schnitzler an Berta Zuckerkandl, 15. 6. 1925}
\nopagebreak\mylabel{L03959v}
\rehead{ }\normalsize\beginnumbering\briefempfaengerindex{Zuckerkandl, Berta@\textsc{Zuckerkandl, Berta}!zzzSchnitzler, Arthur@\emph{von Arthur Schnitzler}!1925-06-151@{15. 6. 1925}|(be}
\toendnotes[C]{\smallbreak\pagebreak[2]}
\correspDesc{Versand  durch Arthur Schnitzler am 15. 6. 1925 in Wien
\newline{}Erhalt  durch Berta Zuckerkandl im Zeitraum [15. 6. 1925
                  – 18. 6. 1925?] in Wien}\toendnotes[C]{\smallbreak}
\Standort{DLA, HS.1985.1.2282.}
\physDesc{Brief, Durchschlag, 1 Blatt, 1 Seite, 986 Zeichen
\newline{}Schreibmaschine
\newline{}Handschrift: roter Buntstift, lateinische Kurrent (\noindent{}beschriftet: »\uline{Zuckerkandl}« und »Frkr«, fünf Unterstreichungen)}\toendnotes[C]{\smallbreak}
\pstart
           \raggedleft{}{\pb}15. 6. 1925.\pend
           
\pstart{}Liebe und verehrte Frau Hofrätin.\pend\vspace{0.5em}
\pstart
           Von Lenormand\pwindex{Lenormand, Henri-René 3.\,5.\,1882 Paris – 16.\,2.\,1951 ebd.@\textsc{Lenormand, Henri-René} (3.\,5.\,1882 Paris – 16.\,2.\,1951 ebd.), \emph{Schriftsteller}|pw} habe ich einen \label{K_L03959-1v}\edtext{Brief}{\lemma{\textnormal{\emph{Brief}}}\Cendnote{\textnormal{Der Brief von Henri-René
                     Lenormand\pwindex{Lenormand, Henri-René 3.\,5.\,1882 Paris – 16.\,2.\,1951 ebd.@\textsc{Lenormand, Henri-René} (3.\,5.\,1882 Paris – 16.\,2.\,1951 ebd.), \emph{Schriftsteller}|pwk} ist nicht überliefert, aber die Antwort darauf: Arthur Schnitzler an Henri-René Lenormand\pwindex{Lenormand, Henri-René 3.\,5.\,1882 Paris – 16.\,2.\,1951 ebd.@\textsc{Lenormand, Henri-René} (3.\,5.\,1882 Paris – 16.\,2.\,1951 ebd.), \emph{Schriftsteller}|pwk}, 15. 6. 1925, \emph{Deutsches Literaturarchiv Marbach},
                  HS.1985.1.1280.}}}\label{K_L03959-1} bekommen, in dem er mich ersucht ihm recht bald die
                  »Liebelei«-Uebersetzung\pwindex{Schnitzler, Arthur 15. 5. 1862 Wien – 21. 10. 1931 ebd.@\textsc{Schnitzler, Arthur} (15. 5. 1862 Wien – 21. 10. 1931 ebd.), \emph{Schriftsteller, Mediziner}!Amourette. Pièce en trois actes. Adaptée de Arthur Schnitzler@\strich\emph{Amourette. Pièce en trois actes. Adaptée de Arthur Schnitzler}|pw}\pwindex{Schnitzler, Arthur 15. 5. 1862 Wien – 21. 10. 1931 ebd.@\textsc{Schnitzler, Arthur} (15. 5. 1862 Wien – 21. 10. 1931 ebd.), \emph{Schriftsteller, Mediziner}!Liebelei. Schauspiel in drei Akten@\strich\emph{Liebelei. Schauspiel in drei Akten}|pw} zu
               schicken. Ich lasse also Abschriften\pwindex{Schnitzler, Arthur 15. 5. 1862 Wien – 21. 10. 1931 ebd.@\textsc{Schnitzler, Arthur} (15. 5. 1862 Wien – 21. 10. 1931 ebd.), \emph{Schriftsteller, Mediziner}!Amourette. Pièce en trois actes. Adaptée de Arthur Schnitzler@\strich\emph{Amourette. Pièce en trois actes. Adaptée de Arthur Schnitzler}|pwv} anfertigen so wie wir besprochen haben. Er schreibt mir auch,
               dass er die Uebersetzung\pwindex{Schnitzler, Arthur 15. 5. 1862 Wien – 21. 10. 1931 ebd.@\textsc{Schnitzler, Arthur} (15. 5. 1862 Wien – 21. 10. 1931 ebd.), \emph{Schriftsteller, Mediziner}!Le Pays de l’âme. Drame en 5 actes@\strich\emph{Le Pays de l’âme. Drame en 5 actes}|pwv}\pwindex{Schnitzler, Arthur 15. 5. 1862 Wien – 21. 10. 1931 ebd.@\textsc{Schnitzler, Arthur} (15. 5. 1862 Wien – 21. 10. 1931 ebd.), \emph{Schriftsteller, Mediziner}!weite Land. Tragikomödie in fünf Akten@\strich\emph{Das weite Land. Tragikomödie in fünf Akten}|pwv} der Mme. Cabire\pwindex{Cabire, Emma @\textsc{Cabire, Emma}, \emph{Übersetzerin, Redakteurin, Literaturagentin}|pw} nicht so
               übel findet: »\label{K_L03959-2v}\edtext{\begin{otherlanguage}{french}Quelques retouches suffireraient\end{otherlanguage}}{\lemma{\textnormal{\emph{Quelques … suffireraient}}}\Cendnote{\textnormal{französisch: Einige Änderungen würden
                  ausreichen.}}}\label{K_L03959-2}«. Er, Fleg\pwindex{Fleg, Edmond 26.\,11.\,1874 Genf – 15.\,10.\,1963 Paris@\textsc{Fleg, Edmond} (26.\,11.\,1874 Genf – 15.\,10.\,1963 Paris), \emph{Schriftsteller, Theaterkritiker}|pw} und Besnard\pwindex{Besnard, Lucien 19.\,1.\,1872 Nonancourt – 1955 Paris@\textsc{Besnard, Lucien} (19.\,1.\,1872 Nonancourt – 1955 Paris), \emph{Schriftsteller}|pw} werden, wie ich ja schon von Ihnen
               weiss, liebe Frau Hofrätin, den Text\pwindex{Schnitzler, Arthur 15. 5. 1862 Wien – 21. 10. 1931 ebd.@\textsc{Schnitzler, Arthur} (15. 5. 1862 Wien – 21. 10. 1931 ebd.), \emph{Schriftsteller, Mediziner}!Le Pays de l’âme. Drame en 5 actes@\strich\emph{Le Pays de l’âme. Drame en 5 actes}|pwv} revidieren. Auch Nathan\pwindex{Nathan, Nicolas @\textsc{Nathan, Nicolas}, \emph{Übersetzer}|pw} hat
               etwas \label{K_L03959-3v}\edtext{von sich hören lassen}{\lemma{\textnormal{\emph{von sich hören lassen}}}\Cendnote{\textnormal{Nicolas Nathans\pwindex{Nathan, Nicolas @\textsc{Nathan, Nicolas}, \emph{Übersetzer}|pwk} Brief ist nicht
                  überliefert, aber die Antwort: Arthur
                     Schnitzler an Nicolas Nathan\pwindex{Nathan, Nicolas @\textsc{Nathan, Nicolas}, \emph{Übersetzer}|pwk},
                     15. 6. 1925, \emph{Deutsches Literaturarchiv Marbach}, HS.1985.1.1485.}}}\label{K_L03959-3}. Jean Nicollier\pwindex{Nicollier, Jean 15.\,11.\,1894 – 1968@\textsc{Nicollier, Jean} (15.\,11.\,1894 – 1968), \emph{Schriftsteller, Übersetzer}|pw} wird »Casanovas Heimfahrt\pwindex{Schnitzler, Arthur 15. 5. 1862 Wien – 21. 10. 1931 ebd.@\textsc{Schnitzler, Arthur} (15. 5. 1862 Wien – 21. 10. 1931 ebd.), \emph{Schriftsteller, Mediziner}!Casanovas Heimfahrt@\strich\emph{Casanovas Heimfahrt}|pw}« übersetzen, Nathan\pwindex{Nathan, Nicolas @\textsc{Nathan, Nicolas}, \emph{Übersetzer}|pw} glaubt, dass das 200jährige \label{K_L03959-4v}\edtext{Casanova\pwindex{Casanova, Giacomo Girolamo 2.\,4.\,1725 Venedig – 4.\,6.\,1798 Duchcov@\textsc{Casanova, Giacomo Girolamo} (2.\,4.\,1725 Venedig – 4.\,6.\,1798 Duchcov), \emph{Schriftsteller, Abenteurer}|pw}-Jubiläum}{\lemma{\textnormal{\emph{Casanova-Jubiläum}}}\Cendnote{\textnormal{Der Schriftsteller Casanova\pwindex{Casanova, Giacomo Girolamo 2.\,4.\,1725 Venedig – 4.\,6.\,1798 Duchcov@\textsc{Casanova, Giacomo Girolamo} (2.\,4.\,1725 Venedig – 4.\,6.\,1798 Duchcov), \emph{Schriftsteller, Abenteurer}|pwk} wurde 1725 geboren.}}}\label{K_L03959-4}, wenn man so sagen darf,
               dem Erfolg des Buches\pwindex{Schnitzler, Arthur 15. 5. 1862 Wien – 21. 10. 1931 ebd.@\textsc{Schnitzler, Arthur} (15. 5. 1862 Wien – 21. 10. 1931 ebd.), \emph{Schriftsteller, Mediziner}!Casanovas Heimfahrt@\strich\emph{Casanovas Heimfahrt}|pwv}{ }\strikeout{er} förderlich sein könnte.\pend
           
\pstart
           \label{K_L03959-5v}\edtext{Ich fahre morgen}{\lemma{\textnormal{\emph{Ich fahre morgen}}}\Cendnote{\textnormal{Schnitzler reiste am 17. 6. 1925 mit seiner
                  Tochter Lili\pwindex{Cappellini, Lili 13.\,9.\,1909 Wien – 26.\,7.\,1928 Venedig@\textsc{Cappellini, Lili} (13.\,9.\,1909 Wien – 26.\,7.\,1928 Venedig)|pwk} zu seiner geschiedenen Frau\pwindex{Schnitzler, Olga 17.\,1.\,1882 Wien – 13.\,1.\,1970 Lugano@\textsc{Schnitzler, Olga} (17.\,1.\,1882 Wien – 13.\,1.\,1970 Lugano), \emph{Schauspielerin, Sängerin}|pwkv} nach Baden-Baden\oindex{Baden-Baden@\textbf{Baden-Baden}|pwk} und mit beiden\pwindex{Schnitzler, Olga 17.\,1.\,1882 Wien – 13.\,1.\,1970 Lugano@\textsc{Schnitzler, Olga} (17.\,1.\,1882 Wien – 13.\,1.\,1970 Lugano), \emph{Schauspielerin, Sängerin}|pwkv}\pwindex{Cappellini, Lili 13.\,9.\,1909 Wien – 26.\,7.\,1928 Venedig@\textsc{Cappellini, Lili} (13.\,9.\,1909 Wien – 26.\,7.\,1928 Venedig)|pwkv} am 22. 7. 1925 nach München\oindex{München@\textbf{München}|pwk}. Am 23. 6. 1925 traf er seine Freundin Clara Katharina Pollaczek\pwindex{Pollaczek, Clara Katharina 15.\,1.\,1875 Wien – 22.\,7.\,1951 ebd.@\textsc{Pollaczek, Clara Katharina} (15.\,1.\,1875 Wien – 22.\,7.\,1951 ebd.), \emph{Schriftstellerin}|pwk}, mit der er sich in Meran\oindex{Meran@\textbf{Meran}, \emph{Hauptstadt}|pwk} und Bozen\oindex{Bozen@\textbf{Bozen}, \emph{Hauptstadt}|pwk}
                  aufhielt, bevor er am 4. 7. 1925 nach Wien\oindex{Wien@\textbf{Wien}, \emph{Verwaltungsgebiet}|pwk}
                  zurückkehrte.}}}\label{K_L03959-5} mit Lili\pwindex{Cappellini, Lili 13.\,9.\,1909 Wien – 26.\,7.\,1928 Venedig@\textsc{Cappellini, Lili} (13.\,9.\,1909 Wien – 26.\,7.\,1928 Venedig)|pw} vorerst nach
                  Baden-Baden\oindex{Baden-Baden@\textbf{Baden-Baden}|pw}, von dort nach Südtirol\oindex{Südtirol@\textbf{Südtirol}, \emph{Verwaltungsgebiet}|pw}. Anfang Juli denke ich wieder hier zu
               sein.\pend
           
\pstart
           Ich hoffe, Sie fühlen sich auf dem Cobenzl\oindex{Wien@\textbf{Wien}!XIX., Döbling@\textbf{XIX., Döbling}!Am Kobenzl@\textbf{Am Kobenzl}, \emph{Berg}|pw}
               wohl. Bald nachdem ich wieder zurück bin werde ich wohl das Vergnügen haben \label{K_L03959-6v}\edtext{Sie oben zu besuchen}{\lemma{\textnormal{\emph{Sie oben zu besuchen}}}\Cendnote{\textnormal{Vgl. A. S.: \emph{Tagebuch}, 17. 7. 1925.}}}\label{K_L03959-6}.\pend
           
\pstart
           Mit den herzlichsten grüssen{\\[\baselineskip]} Ihr\pend
           \leftskip=0em{}{\vspace{1\baselineskip}}
\pstart
           \noindent{}Frau Hofrätin Bertha Zuckerkandl,{\\}Hotel Cobenzl\oindex{Wien@\textbf{Wien}!XIX., Döbling@\textbf{XIX., Döbling}!Schlosshotel Cobenzl@\textbf{Schlosshotel Cobenzl}, \emph{Hotel}|pw}.\pend
           \selectlanguage{ngerman}\endnumbering\briefempfaengerindex{Zuckerkandl, Berta@\textsc{Zuckerkandl, Berta}!zzzSchnitzler, Arthur@\emph{von Arthur Schnitzler}!1925-06-151@{15. 6. 1925}|)be}\mylabel{L03959h}
\begin{anhang}
\end{anhang}\newcommand{\dateiname}{L03959}\newcommand{\titel}{Arthur Schnitzler an Berta Zuckerkandl, 15. 6. 1925}\newcommand{\editorInnen}{Herausgegeben von Jahnke, SelmaMüller, Martin Anton}%% latex-leseansicht-abspann.tex
%% Abspann für die Leseansicht.
%% Der Schalter \ifkorrekturansicht ist bereits durch den Vorspann gesetzt.

%% latex-abspann.tex
%% Gemeinsamer Abspann für Korrekturansicht und Leseansicht.
%% Setzt den Schalter \ifkorrekturansicht voraus (gesetzt in den
%% einbindenden Dateien latex-korrekturansicht-abspann.tex bzw.
%% latex-leseansicht-abspann.tex).
%% ---------------------------------------------------------------

\normalsize

% Das esempio-Environment wird nur in der Leseansicht benötigt
\ifkorrekturansicht\else
\newenvironment{esempio}[3]%
{
    \vspace{1.5ex}
    \rlap{\underline{#1}}
    \par
    \setlength{\parindent}{0cm}
    \nopagebreak
    \leftskip=#2cm
    \rightskip=#3cm
}
{
    \par
}
\fi

\doendnotes{C}
\bigskip
\vfill

\clearpage

\footnotesize

\ifkorrekturansicht
  \lohead{\textsc{register}}
\fi

% theindex-Environment neu definieren ohne reledmac
\makeatletter
\renewenvironment{theindex}{%
  \ifkorrekturansicht
    \section*{\indexname}%
  \else
    \subsubsection*{Index der erwähnten Entitäten}%
  \fi
  \setlength{\parindent}{0pt}%
  \setlength{\parskip}{0pt plus 0.3pt}%
  \let\item\@idxitem
}{%
  \ifkorrekturansicht\clearpage\fi
}
\makeatother

\IfFileExists{\jobname-pw.ind}{\input{\jobname-pw.ind}}{}

% Quellenangabe nur in der Leseansicht
\ifkorrekturansicht\else
% Fallback-Definitionen, falls die .tex-Datei \titel etc. nicht gesetzt hat
\providecommand{\titel}{}
\providecommand{\editorInnen}{}
\providecommand{\dateiname}{\jobname}

\vspace{3cm}

\vfill

\footnotesize
\textsc{Quelle}: \titel. Herausgegeben von {\editorInnen}. In: \emph{Arthur Schnitzler: Briefwechsel mit Autorinnen und Autoren}.
 Digitale Edition, https://schnitzler-briefe.acdh.oeaw.ac.at/{\dateiname}.html (Stand \today)
\fi

\end{document}


