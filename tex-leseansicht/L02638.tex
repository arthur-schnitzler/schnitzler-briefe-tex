%% latex-korrekturansicht-vorspann.tex
%% Vorspann für die Korrekturansicht.
%% Lädt die gemeinsame Datei latex-vorspann.tex mit gesetztem Schalter.

\newif\ifkorrekturansicht
\korrekturansichttrue

\input{../tex-inputs/latex-vorspann}


\section[Paul Goldmann an Arthur Schnitzler, {[}25. – 28.? 2. 1897{]}]{L02638 Paul Goldmann an Arthur Schnitzler, {[}25. – 28.? 2. 1897{]}}
\nopagebreak\mylabel{L02638v}
\rehead{ }\normalsize\beginnumbering\briefempfaengerindex{Schnitzler, Arthur@\textsc{Schnitzler, Arthur}!zzzGoldmann, Paul@\emph{von Paul Goldmann}!1897-02-281@{{[}25. – 28.? 2. 1897{]}}|(be}
\toendnotes[C]{\smallbreak\pagebreak[2]}\Standort{DLA, A:Schnitzler, HS.NZ85.1.3167.}
\physDesc{Telegramm, 445 Zeichen
\newline{}maschinell
\newline{}Schnitzler: mit Bleistift datiert auf »Febr 97« 
\newline{}Ordnung: beschnitten }\toendnotes[C]{\smallbreak}
\pstart
           \noindent{}{\pb}\label{K_L02638-1v}\edtext{hier befindlicher bruder\pwindex{Klein, Richard *~07.08.1873@\textsc{Klein, Richard} (*~07.08.1873), \emph{Maler/Malerin}|pwv}{ }arthur kleins\pwindex{Klein, Arthur 27.11.1868 – 28.07.1943@\textsc{Klein, Arthur} (27.11.1868 – 28.07.1943)|pw}}{\lemma{\textnormal{\emph{hier … kleins}}}\Cendnote{\textnormal{Mehrere Hinweise erlauben eine näherungsweise Datierung des Telegramms. 
                  Schnitzler hat es handschriftlich auf Februar 1897 datuert. Im Brief Goldmanns\pwindex{Goldmann, Paul 31.01.1865 – 25.09.1935@\textsc{Goldmann, Paul} (31.01.1865 – 25.09.1935), \emph{Schriftsteller/Schriftstellerin, Journalist/Journalistin}|pwk} vom 24. 2. [1897] ist noch nicht, im Schreiben vom 11. 3. [1897] hingegen bereits von der erledigten Sache die
                  Rede.}}}\label{K_L02638-1} wird mir unerhoert laestig nachdem ich auf seine aufforderung
               mein \label{T_L02638-1v}\edtext{urtheil}{\lemma{\textnormal{\emph{urtheil}}}\Cendnote{\textnormal{In der Vorlage steht: »urtheit«.}}}\label{T_L02638-1} ueber sein
               bild abgegeben schrieb er mir unverschaemten brief, ich antwortete dass ich mit
               unreifen burschen nicht \label{T_L02638-2v}\edtext{discutire}{\lemma{\textnormal{\emph{discutire}}}\Cendnote{\textnormal{In der Vorlage steht:
                  »discutive«.}}}\label{T_L02638-2} und sandte brief an arthur klein\pwindex{Klein, Arthur 27.11.1868 – 28.07.1943@\textsc{Klein, Arthur} (27.11.1868 – 28.07.1943)|pw}{ }heut{ }\label{T_L02638-3v}\edtext{erhielt}{\lemma{\textnormal{\emph{erhielt}}}\Cendnote{\textnormal{In der Vorlage steht: »orhielt«.}}}\label{T_L02638-3} ich
               herausforderung deren annahme ich natuerlich ablehnte \label{T_L02638-4v}\edtext{der}{\lemma{\textnormal{\emph{der}}}\Cendnote{\textnormal{In der Vorlage steht:
                     »ber«.}}}\label{T_L02638-4} bursch stoesst jetzt drohungen gegen mich aus
               kannst du ihn mir nicht vom halse schaffen\pend
           \pstart gruss \spacefill\mbox{paul goldmann =}\pend{}\selectlanguage{ngerman}\endnumbering\briefempfaengerindex{Schnitzler, Arthur@\textsc{Schnitzler, Arthur}!zzzGoldmann, Paul@\emph{von Paul Goldmann}!1897-02-251@{{[}25. – 28.? 2. 1897{]}}|)be}\mylabel{L02638h}  \normalsize

\doendnotes{C}
\bigskip
\vfill

\clearpage

\footnotesize

\lohead{\textsc{register}}

% Definiere theindex-Environment komplett neu ohne reledmac
\makeatletter
\renewenvironment{theindex}{%
  \section*{\indexname}%
  \setlength{\parindent}{0pt}%
  \setlength{\parskip}{0pt plus 0.3pt}%
  \let\item\@idxitem
}{%
  \clearpage
}
\makeatother

\IfFileExists{\jobname-pw.ind}{\input{\jobname-pw.ind}}{}

\end{document}

      