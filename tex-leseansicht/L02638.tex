%% latex-leseansicht-vorspann.tex
%% Vorspann für die Leseansicht.
%% Lädt die gemeinsame Datei latex-vorspann.tex mit nicht gesetztem Schalter.

\newif\ifkorrekturansicht
\korrekturansichtfalse

\input{../tex-inputs/latex-vorspann}

\begin{center}
            \textcolor{red}{ENTWURF, NICHT FERTIG KORRIGIERT}
                      \end{center}
            
         
         \renewcommand{\erwaehntePersonen}{Personen: Richard Klein, Arthur Klein}
         \renewcommand{\erwaehnteOrte}{Orte: Paris, Wien}
         \renewcommand{\erwaehnteWerke}{}
               \section[Paul Goldmann an Arthur Schnitzler, {[}25.–28.? 2. 1897{]}]{ Paul Goldmann an Arthur Schnitzler, {[}25.–28.? 2. 1897{]}}\nopagebreak\mylabel{v}\rehead{ }\begin{ledgroupsized}[t]{13cm}\normalsize\beginnumbering \toendnotes[C]{\smallbreak\pagebreak[2]} \Standort{DLA, A:Schnitzler, HS.NZ85.1.3167.}
\physDesc{Telegramm
\newline{}maschinell
\newline{}Schnitzler: mit Bleistift datiert auf
                                 »Febr 97« \newline{}Ordnung: beschnitten }\toendnotes[C]{\smallbreak}\pstart
           \noindent{}{\pb}\label{K_L02638-5v}\edtext{hier befindlicher bruder\pwindex{Klein, Richard *~07.08.1873@\textsc{Klein, Richard} (*~07.08.1873), \emph{Maler}|pwv}{ }arthur klein\pwindex{Klein, Arthur 27.11.1868 – 28.07.1943@\textsc{Klein, Arthur} (27.11.1868 – 28.07.1943)|pw}s}{\lemma{\textnormal{\emph{hier … kleins}}}\Cendnote{\textnormal{Eine näherungsweise Datierung des Telegramms geht durch
               Berücksichtigung einerseits der handschriftlichen Datierung Schnitzler\pwindex{Schnitzler, Arthur 15.05.1862 – 21.10.1931@\textsc{Schnitzler, Arthur} (15.05.1862 – 21.10.1931), \emph{Schriftsteller, Mediziner}|pwk}s auf den Februar 1897, andererseits dass
                  im Brief Paul Goldmann an Arthur Schnitzler, 24. 2. [1897] noch nicht, im Paul Goldmann an Arthur Schnitzler, 11. 3. [1897] bereits
               von der erledigten Sache die Rede ist.}}}\label{K_L02638-5h} wird mir unerhoert laestig nachdem ich auf seine
               aufforderung mein \label{T_L02638-3v}\edtext{urtheil}{\lemma{\textnormal{\emph{urtheil}}}\Cendnote{\textnormal{im Text steht:
                  »urtheit«}}}\label{T_L02638-3h} ueber sein bild
               abgegeben schrieb er mir unverschaemten brief, ich antwortete dass ich mit unreifen
               burschen nicht \label{T_L02638-2v}\edtext{discutire}{\lemma{\textnormal{\emph{discutire}}}\Cendnote{\textnormal{im Text steht:
                  »discutive«}}}\label{T_L02638-2h} und sandte brief an arthur klein\pwindex{Klein, Arthur 27.11.1868 – 28.07.1943@\textsc{Klein, Arthur} (27.11.1868 – 28.07.1943)|pw}{ }heut{ }\label{T_L02638-1v}\edtext{erhielt}{\lemma{\textnormal{\emph{erhielt}}}\Cendnote{\textnormal{im Text steht: »orhielt«}}}\label{T_L02638-1h} ich
               herausforderung deren annahme ich natuerlich ablehnte \label{T_L02638-4v}\edtext{der}{\lemma{\textnormal{\emph{der}}}\Cendnote{\textnormal{im Text steht:
                     »ber«}}}\label{T_L02638-4h} bursch stoesst jetzt drohungen gegen mich aus
               kannst du ihn mir nicht vom halse schaffen\pend
           \pstart gruss \spacefill\mbox{paul goldmann =}\pend{}
         
         \endnumbering\mylabel{h}\end{ledgroupsized}  \newcommand{\dateiname}{L02638}\newcommand{\titel}{Paul Goldmann an Arthur Schnitzler, [25.–28.? 2. 1897]}\newcommand{\editorInnen}{Martin Anton Müller und Laura Untner}%% latex-leseansicht-abspann.tex
%% Abspann für die Leseansicht.
%% Der Schalter \ifkorrekturansicht ist bereits durch den Vorspann gesetzt.

%% latex-abspann.tex
%% Gemeinsamer Abspann für Korrekturansicht und Leseansicht.
%% Setzt den Schalter \ifkorrekturansicht voraus (gesetzt in den
%% einbindenden Dateien latex-korrekturansicht-abspann.tex bzw.
%% latex-leseansicht-abspann.tex).
%% ---------------------------------------------------------------

\normalsize

% Das esempio-Environment wird nur in der Leseansicht benötigt
\ifkorrekturansicht\else
\newenvironment{esempio}[3]%
{
    \vspace{1.5ex}
    \rlap{\underline{#1}}
    \par
    \setlength{\parindent}{0cm}
    \nopagebreak
    \leftskip=#2cm
    \rightskip=#3cm
}
{
    \par
}
\fi

\doendnotes{C}
\bigskip
\vfill

\clearpage

\footnotesize

\ifkorrekturansicht
  \lohead{\textsc{register}}
\fi

% theindex-Environment neu definieren ohne reledmac
\makeatletter
\renewenvironment{theindex}{%
  \ifkorrekturansicht
    \section*{\indexname}%
  \else
    \subsubsection*{Index der erwähnten Entitäten}%
  \fi
  \setlength{\parindent}{0pt}%
  \setlength{\parskip}{0pt plus 0.3pt}%
  \let\item\@idxitem
}{%
  \ifkorrekturansicht\clearpage\fi
}
\makeatother

\IfFileExists{\jobname-pw.ind}{\input{\jobname-pw.ind}}{}

% Quellenangabe nur in der Leseansicht
\ifkorrekturansicht\else
% Fallback-Definitionen, falls die .tex-Datei \titel etc. nicht gesetzt hat
\providecommand{\titel}{}
\providecommand{\editorInnen}{}
\providecommand{\dateiname}{\jobname}

\vspace{3cm}

\vfill

\footnotesize
\textsc{Quelle}: \titel. Herausgegeben von {\editorInnen}. In: \emph{Arthur Schnitzler: Briefwechsel mit Autorinnen und Autoren}.
 Digitale Edition, https://schnitzler-briefe.acdh.oeaw.ac.at/{\dateiname}.html (Stand \today)
\fi

\end{document}


      