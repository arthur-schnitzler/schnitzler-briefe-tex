%% latex-leseansicht-vorspann.tex
%% Vorspann für die Leseansicht.
%% Lädt die gemeinsame Datei latex-vorspann.tex mit nicht gesetztem Schalter.

\newif\ifkorrekturansicht
\korrekturansichtfalse

\input{../tex-inputs/latex-vorspann}


\section[Paul Goldmann an Arthur Schnitzler, {[}25. – 28.? 2. 1897{]}]{L02638 Paul Goldmann an Arthur Schnitzler, {[}25. – 28.? 2. 1897{]}}
\nopagebreak\mylabel{L02638v}
\rehead{ }\normalsize\beginnumbering\briefempfaengerindex{Schnitzler, Arthur@\textsc{Schnitzler, Arthur}!zzzGoldmann, Paul@\emph{von Paul Goldmann}!1897-02-281@{{[}25. – 28.? 2. 1897{]}}|(be}
\toendnotes[C]{\smallbreak\pagebreak[2]}
\correspDesc{Versand  durch Paul Goldmann im Zeitraum [25. – 28.? 2. 1897] in Paris
\newline{}Erhalt  durch Arthur Schnitzler im Zeitraum [26. 2. 1897 – 1.? 3. 1897] in Wien}\toendnotes[C]{\smallbreak}
\Standort{DLA, A:Schnitzler, HS.NZ85.1.3167.}
\physDesc{Telegramm, 445 Zeichen
\newline{}maschinell
\newline{}Schnitzler: mit Bleistift datiert auf »Febr 97« 
\newline{}Ordnung: beschnitten }\toendnotes[C]{\smallbreak}
\pstart
           \noindent{}{\pb}\label{K_L02638-1v}\edtext{hier befindlicher bruder\pwindex{Klein, Richard *~7.\,8.\,1873 Baden bei Wien@\textsc{Klein, Richard} (*~7.\,8.\,1873 Baden bei Wien), \emph{Maler}|pwv}{ }arthur kleins\pwindex{Klein, Arthur 27.\,11.\,1868 Wien – 28.\,7.\,1943@\textsc{Klein, Arthur} (27.\,11.\,1868 Wien – 28.\,7.\,1943)|pw}}{\lemma{\textnormal{\emph{hier … kleins}}}\Cendnote{\textnormal{Mehrere Hinweise erlauben eine näherungsweise Datierung des Telegramms. 
                  Schnitzler hat es handschriftlich auf Februar 1897 datuert. Im Brief Goldmanns\pwindex{Goldmann, Paul 31.\,1.\,1865 Breslau – 25.\,9.\,1935 Wien@\textsc{Goldmann, Paul} (31.\,1.\,1865 Breslau – 25.\,9.\,1935 Wien), \emph{Schriftsteller, Journalist}|pwk} vom XXXX Auszeichnungsfehler: Dokument L02804 nicht gefunden ist noch nicht, im Schreiben vom XXXX Auszeichnungsfehler: Dokument L02805 nicht gefunden hingegen bereits von der erledigten Sache die
                  Rede.}}}\label{K_L02638-1} wird mir unerhoert laestig nachdem ich auf seine aufforderung
               mein \label{T_L02638-1v}\edtext{urtheil}{\lemma{\textnormal{\emph{urtheil}}}\Cendnote{\textnormal{In der Vorlage steht: »urtheit«.}}}\label{T_L02638-1} ueber sein
               bild abgegeben schrieb er mir unverschaemten brief, ich antwortete dass ich mit
               unreifen burschen nicht \label{T_L02638-2v}\edtext{discutire}{\lemma{\textnormal{\emph{discutire}}}\Cendnote{\textnormal{In der Vorlage steht:
                  »discutive«.}}}\label{T_L02638-2} und sandte brief an arthur klein\pwindex{Klein, Arthur 27.\,11.\,1868 Wien – 28.\,7.\,1943@\textsc{Klein, Arthur} (27.\,11.\,1868 Wien – 28.\,7.\,1943)|pw}{ }heut{ }\label{T_L02638-3v}\edtext{erhielt}{\lemma{\textnormal{\emph{erhielt}}}\Cendnote{\textnormal{In der Vorlage steht: »orhielt«.}}}\label{T_L02638-3} ich
               herausforderung deren annahme ich natuerlich ablehnte \label{T_L02638-4v}\edtext{der}{\lemma{\textnormal{\emph{der}}}\Cendnote{\textnormal{In der Vorlage steht:
                     »ber«.}}}\label{T_L02638-4} bursch stoesst jetzt drohungen gegen mich aus
               kannst du ihn mir nicht vom halse schaffen\pend
           \pstart gruss \spacefill\mbox{paul goldmann =}\pend{}\selectlanguage{ngerman}\endnumbering\briefempfaengerindex{Schnitzler, Arthur@\textsc{Schnitzler, Arthur}!zzzGoldmann, Paul@\emph{von Paul Goldmann}!1897-02-251@{{[}25. – 28.? 2. 1897{]}}|)be}\mylabel{L02638h}  \newcommand{\dateiname}{L02638}\newcommand{\titel}{Paul Goldmann an Arthur Schnitzler, [25. – 28.? 2. 1897]}\newcommand{\editorInnen}{Martin Anton Müller und Laura Untner}%% latex-leseansicht-abspann.tex
%% Abspann für die Leseansicht.
%% Der Schalter \ifkorrekturansicht ist bereits durch den Vorspann gesetzt.

%% latex-abspann.tex
%% Gemeinsamer Abspann für Korrekturansicht und Leseansicht.
%% Setzt den Schalter \ifkorrekturansicht voraus (gesetzt in den
%% einbindenden Dateien latex-korrekturansicht-abspann.tex bzw.
%% latex-leseansicht-abspann.tex).
%% ---------------------------------------------------------------

\normalsize

% Das esempio-Environment wird nur in der Leseansicht benötigt
\ifkorrekturansicht\else
\newenvironment{esempio}[3]%
{
    \vspace{1.5ex}
    \rlap{\underline{#1}}
    \par
    \setlength{\parindent}{0cm}
    \nopagebreak
    \leftskip=#2cm
    \rightskip=#3cm
}
{
    \par
}
\fi

\doendnotes{C}
\bigskip
\vfill

\clearpage

\footnotesize

\ifkorrekturansicht
  \lohead{\textsc{register}}
\fi

% theindex-Environment neu definieren ohne reledmac
\makeatletter
\renewenvironment{theindex}{%
  \ifkorrekturansicht
    \section*{\indexname}%
  \else
    \subsubsection*{Index der erwähnten Entitäten}%
  \fi
  \setlength{\parindent}{0pt}%
  \setlength{\parskip}{0pt plus 0.3pt}%
  \let\item\@idxitem
}{%
  \ifkorrekturansicht\clearpage\fi
}
\makeatother

\IfFileExists{\jobname-pw.ind}{\input{\jobname-pw.ind}}{}

% Quellenangabe nur in der Leseansicht
\ifkorrekturansicht\else
% Fallback-Definitionen, falls die .tex-Datei \titel etc. nicht gesetzt hat
\providecommand{\titel}{}
\providecommand{\editorInnen}{}
\providecommand{\dateiname}{\jobname}

\vspace{3cm}

\vfill

\footnotesize
\textsc{Quelle}: \titel. Herausgegeben von {\editorInnen}. In: \emph{Arthur Schnitzler: Briefwechsel mit Autorinnen und Autoren}.
 Digitale Edition, https://schnitzler-briefe.acdh.oeaw.ac.at/{\dateiname}.html (Stand \today)
\fi

\end{document}


