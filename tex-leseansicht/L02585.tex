%% latex-leseansicht-vorspann.tex
%% Vorspann für die Leseansicht.
%% Lädt die gemeinsame Datei latex-vorspann.tex mit nicht gesetztem Schalter.

\newif\ifkorrekturansicht
\korrekturansichtfalse

\input{../tex-inputs/latex-vorspann}


         
         \renewcommand{\erwaehntePersonen}{Personen: Auguste Hauschner}
         \renewcommand{\erwaehnteInstitutionen}{Institutionen: Der Tag, Deutsche Revue. Eine Monatsschrift, Neue Revue. Wochenschrift für das öffentliche Leben, Nord und Süd, Westermanns Monatshefte}
         \renewcommand{\erwaehnteOrte}{Orte: Berlin, Wien}
         \renewcommand{\erwaehnteWerke}{Werke: Der Weg ins Freie, Der Weg ins Freie. Roman, Die Familie Lowositz. Roman}
               \section[Arthur Schnitzler an Auguste Hauschner, 12. 10. 1908]{ Arthur Schnitzler an Auguste Hauschner, 12. 10. 1908}\nopagebreak\mylabel{v}\rehead{ }\begin{ledgroupsized}[t]{13cm}\normalsize\beginnumbering \toendnotes[C]{\smallbreak\pagebreak[2]} \Standort{DLA, A:Schnitzler, HS.1985.1.955.}
\physDesc{Brief, Durchschlag, 2 Blätter, 2 Seiten, 1179 Zeichen
\newline{}Schreibmaschine
\newline{}Handschrift: 1) Bleistift, lateinische Kurrent (\noindent{}»Hauschner\pwindex{Hauschner, Auguste 12.02.1850 – 10.04.1924@\textsc{Hauschner, Auguste} (12.02.1850 – 10.04.1924), \emph{Schriftstellerin}|pw}«, dasselbe neuerlich am 2. Blatt und dort auch Datierung: »12/10 08«)\hspace{1em}2) roter Buntstift (\noindent{}vier Unterstreichungen)\hspace{1em}}\toendnotes[C]{\smallbreak}\pstart
           \raggedleft{}{\pb}12. Okt. 08.\pend
           \pstart{}Verehrte Frau,\pend\pstart
           Ich weiss natürlich nicht mit Bestimmtheit zu sagen, in welchen Zeitungen
               Besprechungen meines Roman\pwindex{Schnitzler, Arthur 15.05.1862 – 21.10.1931@\textsc{Schnitzler, Arthur} (15.05.1862 – 21.10.1931), \emph{Schriftsteller, Mediziner}!Weg ins Freie. Roman1.1.1908 – 1.6.1908@\strich\emph{Der Weg ins Freie. Roman} {[}1.1.1908 – 1.6.1908{]}|pwv}s
               noch nicht erschienen sind, da ich ja wahrscheinlich nicht alle Blätter zu Gesicht
               bekommen habe, in denen Kritiken veröffentlicht waren. Nur aufs gerate Wohl kann ich
               einige Zeitungen\orgindex{Tag@Der Tag|pwv}\orgindex{Nord und Sued@Nord und Süd|pwv}\orgindex{Westermanns Monatshefte@Westermanns Monatshefte|pwv}\orgindex{Deutsche Revue. Eine Monatsschrift@Deutsche Revue. Eine Monatsschrift|pwv}\orgindex{Neue Revue. Wochenschrift fuer das oeffentliche Leben@Neue Revue. Wochenschrift für das öffentliche Leben|pwv} nennen, von
               denen ich nicht weiss, ob sie schon etwas gebracht haben, zum Beispiel: »Tag\orgindex{Tag@Der Tag|pw}«, »Nord und
                  Süd\orgindex{Nord und Sued@Nord und Süd|pw}«, »Westermann\orgindex{Westermanns Monatshefte@Westermanns Monatshefte|pw}«, »deutsche Revue\orgindex{Deutsche Revue. Eine Monatsschrift@Deutsche Revue. Eine Monatsschrift|pw}«, »Neue
                  Revue\orgindex{Neue Revue. Wochenschrift fuer das oeffentliche Leben@Neue Revue. Wochenschrift für das öffentliche Leben|pw}« u. s. w. Gewiss haben die meisten dieser Blätter\orgindex{Tag@Der Tag|pwv}\orgindex{Nord und Sued@Nord und Süd|pwv}\orgindex{Westermanns Monatshefte@Westermanns Monatshefte|pwv}\orgindex{Deutsche Revue. Eine Monatsschrift@Deutsche Revue. Eine Monatsschrift|pwv}\orgindex{Neue Revue. Wochenschrift fuer das oeffentliche Leben@Neue Revue. Wochenschrift für das öffentliche Leben|pwv} ständige Berichterstatter und so kann ich Ihnen beim besten Willen
               keinen Rat erteilen. Dass Sie aber irgendwo vergeblich anklopfen könnten, wo die
               Besprechung über meinen Roman\pwindex{Schnitzler, Arthur 15.05.1862 – 21.10.1931@\textsc{Schnitzler, Arthur} (15.05.1862 – 21.10.1931), \emph{Schriftsteller, Mediziner}!Weg ins Freie. Roman1.1.1908 – 1.6.1908@\strich\emph{Der Weg ins Freie. Roman} {[}1.1.1908 – 1.6.1908{]}|pwv}
               noch nicht vergeben wäre, kann ich mir kaum denken und ich möchte gewiss nicht gern
               darauf verzichten Sie irgendwo gedruckt zu lesen, umsoweniger als mir ebenso wie
               Ihnen nicht wenige vollkommen verständnislose zu Gesicht gekommen sind. Ich darf Sie
               wohl darum bitten, mir Ihre Kritik\pwindex{Hauschner, Auguste 12.02.1850 – 10.04.1924@\textsc{Hauschner, Auguste} (12.02.1850 – 10.04.1924), \emph{Schriftstellerin}!Weg ins Freie17. 01. 1909@\strich\emph{Der Weg ins Freie} {[}17. 01. 1909{]}|pwv} nach Erscheinen zuzusenden, danke Ihnen sehr für Ihr Interesse und
               jetzt da ich ihn gelesen habe {\pb}nochmals und herzlich für
               Ihren Roman\pwindex{Hauschner, Auguste 12.02.1850 – 10.04.1924@\textsc{Hauschner, Auguste} (12.02.1850 – 10.04.1924), \emph{Schriftstellerin}!Familie Lowositz. Roman1908-06-02@\strich\emph{Die Familie Lowositz. Roman} {[}1908-06-02{]}|pwv}.\pend
           \pstart
           In aufrichtiger Hochschätzung{\\[\baselineskip]}Ihr sehr ergebener\pend
           \leftskip=0em{}{\bigskip}\pstart
           \noindent{}Frau Auguste Hauschner, Berlin\oindex{Berlin@\textbf{Berlin}|pw}.\pend
           
         
         \endnumbering\mylabel{h}\end{ledgroupsized}  \newcommand{\dateiname}{L02585}\newcommand{\titel}{Arthur Schnitzler an Auguste Hauschner, 12. 10. 1908}\newcommand{\editorInnen}{Martin Anton Müller und Laura Untner}%% latex-leseansicht-abspann.tex
%% Abspann für die Leseansicht.
%% Der Schalter \ifkorrekturansicht ist bereits durch den Vorspann gesetzt.

%% latex-abspann.tex
%% Gemeinsamer Abspann für Korrekturansicht und Leseansicht.
%% Setzt den Schalter \ifkorrekturansicht voraus (gesetzt in den
%% einbindenden Dateien latex-korrekturansicht-abspann.tex bzw.
%% latex-leseansicht-abspann.tex).
%% ---------------------------------------------------------------

\normalsize

% Das esempio-Environment wird nur in der Leseansicht benötigt
\ifkorrekturansicht\else
\newenvironment{esempio}[3]%
{
    \vspace{1.5ex}
    \rlap{\underline{#1}}
    \par
    \setlength{\parindent}{0cm}
    \nopagebreak
    \leftskip=#2cm
    \rightskip=#3cm
}
{
    \par
}
\fi

\doendnotes{C}
\bigskip
\vfill

\clearpage

\footnotesize

\ifkorrekturansicht
  \lohead{\textsc{register}}
\fi

% theindex-Environment neu definieren ohne reledmac
\makeatletter
\renewenvironment{theindex}{%
  \ifkorrekturansicht
    \section*{\indexname}%
  \else
    \subsubsection*{Index der erwähnten Entitäten}%
  \fi
  \setlength{\parindent}{0pt}%
  \setlength{\parskip}{0pt plus 0.3pt}%
  \let\item\@idxitem
}{%
  \ifkorrekturansicht\clearpage\fi
}
\makeatother

\IfFileExists{\jobname-pw.ind}{\input{\jobname-pw.ind}}{}

% Quellenangabe nur in der Leseansicht
\ifkorrekturansicht\else
% Fallback-Definitionen, falls die .tex-Datei \titel etc. nicht gesetzt hat
\providecommand{\titel}{}
\providecommand{\editorInnen}{}
\providecommand{\dateiname}{\jobname}

\vspace{3cm}

\vfill

\footnotesize
\textsc{Quelle}: \titel. Herausgegeben von {\editorInnen}. In: \emph{Arthur Schnitzler: Briefwechsel mit Autorinnen und Autoren}.
 Digitale Edition, https://schnitzler-briefe.acdh.oeaw.ac.at/{\dateiname}.html (Stand \today)
\fi

\end{document}


      