%% latex-leseansicht-vorspann.tex
%% Vorspann für die Leseansicht.
%% Lädt die gemeinsame Datei latex-vorspann.tex mit nicht gesetztem Schalter.

\newif\ifkorrekturansicht
\korrekturansichtfalse

\input{../tex-inputs/latex-vorspann}


\section[Arthur Schnitzler an Gustav Schwarzkopf, 11. 2. 1900]{L04085 Arthur Schnitzler an Gustav Schwarzkopf, 11. 2. 1900}
\nopagebreak\mylabel{L04085v}
\rehead{ }\normalsize\beginnumbering\briefempfaengerindex{Schwarzkopf, Gustav@\textsc{Schwarzkopf, Gustav}!zzzSchnitzler, Arthur@\emph{von Arthur Schnitzler}!1900-02-112@{11. 2. 1900}|(be}
\toendnotes[C]{\smallbreak\pagebreak[2]}
\correspDesc{Versand  durch Arthur Schnitzler am 11. 2. 1900 in Wien
\newline{}Erhalt  durch Gustav Schwarzkopf am 11. 2. 1900 in Wien}\toendnotes[C]{\smallbreak}
\Standort{CUL, Schnitzler, B 96.}
\physDesc{Postkarte, 264 Zeichen
\newline{}Handschrift: Bleistift, deutsche Kurrent
\newline{}Versand: 1) Stempel: »\nobreak{}\oindex{IX., Alsergrund@\textbf{IX., Alsergrund}, \emph{Verwaltungsgebiet}|pwk}Wien 9/1 66, 11. II. 00, 2\textsuperscript{00}N\nobreak{}«.   2) Stempel: »\nobreak{}\oindex{I., Innere Stadt@\textbf{I., Innere Stadt}, \emph{Verwaltungsgebiet}|pwk}Wien 1/1, 11. II. 00, 3\textsuperscript{10}N\nobreak{}«. }\toendnotes[C]{\smallbreak}\pstart{}{\pb}\textsc{pneumatisch}\pend{}\pstart{}\textsc{Herrn Gustav Schwarzkopf}\pend{}\pstart{}Wien\oindex{Wien@\textbf{Wien}, \emph{Verwaltungsgebiet}|pw}\pend{}\pstart{}\textsc{Tiefer
                        Graben 23}\oindex{Wien@\textbf{Wien}!I., Innere Stadt@\textbf{I., Innere Stadt}!Tiefer Graben 23@\textbf{Tiefer Graben 23}, \emph{Wohngebäude}|pw}.\pend{}{\bigskip}\vspace{1em}
\pstart
           \noindent{}{\pb}lieber Guſtav, ich habe Kappers\pwindex{Kapper, Friedrich 21.\,4.\,1861 Wien – 22.\,7.\,1939 ebd.@\textsc{Kapper, Friedrich} (21.\,4.\,1861 Wien – 22.\,7.\,1939 ebd.), \emph{Mediziner}|pw}\pwindex{Kapper, Adele 25.\,1.\,1870 Wien – 1941 Vernichtungslager Maly Trostinez@\textsc{Kapper, Adele} (25.\,1.\,1870 Wien – 1941 Vernichtungslager Maly Trostinez)|pw} in der Stadt\oindex{I., Innere Stadt@\textbf{I., Innere Stadt}, \emph{Verwaltungsgebiet}|pwv}
               getroffen, mit ihnen \label{K_L04085-1v}\edtext{Nachtmahl}{\lemma{\textnormal{\emph{Nachtmahl}}}\Cendnote{\textnormal{Ob es zum Treffen im Riedhof\oindex{Wien@\textbf{Wien}!VIII., Josefstadt@\textbf{VIII., Josefstadt}!Riedhof@\textbf{Riedhof}, \emph{Lokal}|pwk} kam,
                  ist ungewiss; jedenfalls kamen alle im Café Kaiserhof\oindex{Wien@\textbf{Wien}!I., Innere Stadt@\textbf{I., Innere Stadt}!Café Kaiserhof (Inh. Johann Wortner) [Wien]@\textbf{Café Kaiserhof (Inh. Johann Wortner) [Wien]}, \emph{Kaffeehaus}|pwk} zusammen, 
                  vgl. A. S.: \emph{Tagebuch}, 11. 2. 1900. }}}\label{K_L04085-1} in \substVorne{}\textsuperscript{Rdhf\oindex{Wien@\textbf{Wien}!VIII., Josefstadt@\textbf{VIII., Josefstadt}!Riedhof@\textbf{Riedhof}, \emph{Lokal}|pw}}\substDazwischen{}Riedhof\oindex{Wien@\textbf{Wien}!VIII., Josefstadt@\textbf{VIII., Josefstadt}!Riedhof@\textbf{Riedhof}, \emph{Lokal}|pw}\substHinten{} (3) verabredet; die Freud wäre allſeits groſs, we{\geminationn} Sie auch kämen (½  9) – Nachher bin ich im \textsc{Kaiserhof}\oindex{Wien@\textbf{Wien}!I., Innere Stadt@\textbf{I., Innere Stadt}!Café Kaiserhof (Inh. Johann Wortner) [Wien]@\textbf{Café Kaiserhof (Inh. Johann Wortner) [Wien]}, \emph{Kaffeehaus}|pw}.\pend
           
\pstart
           Herzlichſt{\\[\baselineskip]} Ihr \spacefill\mbox{Arthur Sch.}\pend
           \leftskip=0em{}\selectlanguage{ngerman}\endnumbering\briefempfaengerindex{Schwarzkopf, Gustav@\textsc{Schwarzkopf, Gustav}!zzzSchnitzler, Arthur@\emph{von Arthur Schnitzler}!1900-02-112@{11. 2. 1900}|)be}\mylabel{L04085h}
\begin{anhang}
\end{anhang}\newcommand{\dateiname}{L04085}\newcommand{\titel}{Arthur Schnitzler an Gustav Schwarzkopf, 11. 2. 1900}\newcommand{\editorInnen}{Herausgegeben von Jahnke, SelmaMüller, Martin Anton}%% latex-leseansicht-abspann.tex
%% Abspann für die Leseansicht.
%% Der Schalter \ifkorrekturansicht ist bereits durch den Vorspann gesetzt.

%% latex-abspann.tex
%% Gemeinsamer Abspann für Korrekturansicht und Leseansicht.
%% Setzt den Schalter \ifkorrekturansicht voraus (gesetzt in den
%% einbindenden Dateien latex-korrekturansicht-abspann.tex bzw.
%% latex-leseansicht-abspann.tex).
%% ---------------------------------------------------------------

\normalsize

% Das esempio-Environment wird nur in der Leseansicht benötigt
\ifkorrekturansicht\else
\newenvironment{esempio}[3]%
{
    \vspace{1.5ex}
    \rlap{\underline{#1}}
    \par
    \setlength{\parindent}{0cm}
    \nopagebreak
    \leftskip=#2cm
    \rightskip=#3cm
}
{
    \par
}
\fi

\doendnotes{C}
\bigskip
\vfill

\clearpage

\footnotesize

\ifkorrekturansicht
  \lohead{\textsc{register}}
\fi

% theindex-Environment neu definieren ohne reledmac
\makeatletter
\renewenvironment{theindex}{%
  \ifkorrekturansicht
    \section*{\indexname}%
  \else
    \subsubsection*{Index der erwähnten Entitäten}%
  \fi
  \setlength{\parindent}{0pt}%
  \setlength{\parskip}{0pt plus 0.3pt}%
  \let\item\@idxitem
}{%
  \ifkorrekturansicht\clearpage\fi
}
\makeatother

\IfFileExists{\jobname-pw.ind}{\input{\jobname-pw.ind}}{}

% Quellenangabe nur in der Leseansicht
\ifkorrekturansicht\else
% Fallback-Definitionen, falls die .tex-Datei \titel etc. nicht gesetzt hat
\providecommand{\titel}{}
\providecommand{\editorInnen}{}
\providecommand{\dateiname}{\jobname}

\vspace{3cm}

\vfill

\footnotesize
\textsc{Quelle}: \titel. Herausgegeben von {\editorInnen}. In: \emph{Arthur Schnitzler: Briefwechsel mit Autorinnen und Autoren}.
 Digitale Edition, https://schnitzler-briefe.acdh.oeaw.ac.at/{\dateiname}.html (Stand \today)
\fi

\end{document}


