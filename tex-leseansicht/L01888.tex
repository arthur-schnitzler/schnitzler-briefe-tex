%% latex-korrekturansicht-vorspann.tex
%% Vorspann für die Korrekturansicht.
%% Lädt die gemeinsame Datei latex-vorspann.tex mit gesetztem Schalter.

\newif\ifkorrekturansicht
\korrekturansichttrue

\input{../tex-inputs/latex-vorspann}


\section[Arthur Schnitzler an Albert Ehrenstein, 23. 11. 1909]{L01888 Arthur Schnitzler an Albert Ehrenstein, 23. 11. 1909}
\nopagebreak\mylabel{L01888v}
\rehead{ }\normalsize\beginnumbering\briefempfaengerindex{Ehrenstein, Albert@\textsc{Ehrenstein, Albert}!zzzSchnitzler, Arthur@\emph{von Arthur Schnitzler}!1909-11-231@{23. 11. 1909}|(be}
\toendnotes[C]{\smallbreak\pagebreak[2]}\Standort{Jerusalem, The National Library of Israel, ARC. Ms. Var. 306 1 118.}
\physDesc{Brief, 1 Blatt, 1 Seite, 734 Zeichen
\newline{}Schreibmaschine
\newline{}Handschrift: schwarze Tinte, lateinische Kurrent (\noindent{}Schlussformel, Unterschrift, eine Korrektur)}\Standort{DLA, A:Schnitzler, 85.1.642,2.}
\physDesc{Brief, Durchschlag1 Blatt, 1 Seite, 734 Zeichen
\newline{}Schreibmaschine
\newline{}Handschrift: roter Buntstift, lateinische Kurrent (\noindent{}Beschriftung: »Ehrenstein«)}\toendnotes[C]{\smallbreak}
\pstart
           {\pb}\textcolor{gray}{\textbf{Dr. Arthur Schnitzler}}\pend
           
\pstart
           \textcolor{gray}{\textbf{Wien XVIII. Spoettelgasse 7\oindex{Edmund-Weiss-Gasse 7@\textbf{Edmund-Weiß-Gasse 7}, \emph{Wohngebäude (K.WHS)}|pw}.}}\hfill 23. 11. 1909.\pend
           
\pstart{}Lieber Herr Ehrenstein! \pend\vspace{0.5em}
\pstart
           Meine Berlin\oindex{Berlin@\textbf{Berlin}, \emph{P.PPLC}|pw}er Reise dürfte erst im
                  Jänner oder Februar stattfinden. Ich bin noch nicht
               dazugekommen Ihre neuen Manuskripte zu lesen, will es aber in den allernächsten Tagen
                  tun{[}.{]} Hoffentlich  wird die Polgar\pwindex{Polgar, Alfred 17.10.1873 – 24.04.1955@\textsc{Polgar, Alfred} (17.10.1873 – 24.04.1955), \emph{Schriftsteller/Schriftstellerin, Journalist/Journalistin, Kritiker/Kritikerin}|pw}’sche Empfehlung an Bie\pwindex{Bie, Oskar 09.02.1864 – 21.04.1938@\textsc{Bie, Oskar} (09.02.1864 – 21.04.1938), \emph{Schriftsteller/Schriftstellerin, Journalist/Journalistin, Redakteur/Redakteurin}|pw} von Nutzen sein. Vielleicht wäre es nun das Beste, wenn ich an Fischer\pwindex{Fischer, Samuel 24.12.1859 – 15.10.1934@\textsc{Fischer, Samuel} (24.12.1859 – 15.10.1934), \emph{Verleger/Verlegerin}|pw} oder Bie\pwindex{Bie, Oskar 09.02.1864 – 21.04.1938@\textsc{Bie, Oskar} (09.02.1864 – 21.04.1938), \emph{Schriftsteller/Schriftstellerin, Journalist/Journalistin, Redakteur/Redakteurin}|pw} schriebe, dass ich die Absicht hatte persönlich mit dem Verlag\orgindex{S. Fischer Verlag@S. Fischer Verlag|pwv} oder der Redaktion\orgindex{Neue Rundschau, Neue Deutsche Rundschau, Freie Buehne@Neue Rundschau, Neue Deutsche Rundschau, Freie Bühne|pwv} über Ihre Sachen zu sprechen und
               dass ich nur wegen Verzögerung meiner Reise auf schriftlichem Wege die Aufmerksamkeit
               darauf zu lenken genötigt sei. Mehr Erfolg scheint mir ja allerdings der persönliche
               Weg zu versprechen. Hat es bis nächste Woche Zeit, so können wir mündlich darüber
               reden.\pend
           
\pstart
           Bestens grüßend{\\[\baselineskip]}{[}hs. :{]} Ihr{\\[\baselineskip]}\spacefill\mbox{ArthSchnitzler}\pend
           \leftskip=0em{}\selectlanguage{ngerman}\endnumbering\briefempfaengerindex{Ehrenstein, Albert@\textsc{Ehrenstein, Albert}!zzzSchnitzler, Arthur@\emph{von Arthur Schnitzler}!1909-11-231@{23. 11. 1909}|)be}\mylabel{L01888h}  \normalsize

\doendnotes{C}
\bigskip
\vfill

\clearpage

\footnotesize

\lohead{\textsc{register}}

% Definiere theindex-Environment komplett neu ohne reledmac
\makeatletter
\renewenvironment{theindex}{%
  \section*{\indexname}%
  \setlength{\parindent}{0pt}%
  \setlength{\parskip}{0pt plus 0.3pt}%
  \let\item\@idxitem
}{%
  \clearpage
}
\makeatother

\IfFileExists{\jobname-pw.ind}{\input{\jobname-pw.ind}}{}

\end{document}

      