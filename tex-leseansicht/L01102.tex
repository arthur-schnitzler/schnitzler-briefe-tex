%% latex-leseansicht-vorspann.tex
%% Vorspann für die Leseansicht.
%% Lädt die gemeinsame Datei latex-vorspann.tex mit nicht gesetztem Schalter.

\newif\ifkorrekturansicht
\korrekturansichtfalse

\input{../tex-inputs/latex-vorspann}


\section[Hermann Bahr an Arthur Schnitzler, {[}13. 3.? 1901{]}]{L01102 Hermann Bahr an Arthur Schnitzler, {[}13. 3.? 1901{]}}
\nopagebreak\mylabel{L01102v}
\rehead{ }\normalsize\beginnumbering\briefempfaengerindex{Schnitzler, Arthur@\textsc{Schnitzler, Arthur}!zzzBahr, Hermann@\emph{von Hermann Bahr}!1901-03-131@{{[}13. 3.? 1901{]}}|(be}
\toendnotes[C]{\smallbreak\pagebreak[2]}
\correspDesc{Versand  durch Hermann Bahr am [13. 3.? 1901] in Wien
\newline{}Erhalt  durch Arthur Schnitzler im Zeitraum [13. 3. 1901
                  – 17. 3. 1901?] in Wien}\toendnotes[C]{\smallbreak}
\Standort{CUL, Schnitzler, B 5b.}
\physDesc{Brief, 1 Blatt, 1 Seite, 275 Zeichen
\newline{}Handschrift: schwarze Tinte, deutsche Kurrent
\newline{}Schnitzler: mit Bleistift ergänztes Datum: »Feber? 901« 
\newline{}Ordnung: mit Bleistift von unbekannter Hand nummeriert:
                                    »74« }
\buchAbdrucke{\weitereDrucke{Hermann Bahr, Arthur Schnitzler: \emph{Briefwechsel, Aufzeichnungen, Dokumente (1891–1931)}. Herausgegeben von Kurt Ifkovits und Martin Anton Müller. Göttingen: \emph{Wallstein} 2018, S. 201.} }\toendnotes[C]{\smallbreak}
\pstart
           \centering{}{\pb}\textcolor{gray}{\textbf{Redaktion des Neuen Wiener Tagblatt\orgindex{Neues Wiener Tagblatt@Neues Wiener Tagblatt|pw}}}\pend
           
\pstart
           \centering{}\textcolor{gray}{\textbf{\textsc{Wien, I., Rothenturmstrasse,
                        Steyrerhof\oindex{Wien@\textbf{Wien}!I., Innere Stadt@\textbf{I., Innere Stadt}!Steyrerhof@\textbf{Steyrerhof}, \emph{Gebäude}|pw}.}}}\pend
           
\pstart
           \centering{}\textcolor{gray}{\textbf{Telegramm-Adresse: Tagblatt\orgindex{Neues Wiener Tagblatt@Neues Wiener Tagblatt|pw}, Steyrerhof, Wien\oindex{Wien@\textbf{Wien}!I., Innere Stadt@\textbf{I., Innere Stadt}!Steyrerhof@\textbf{Steyrerhof}, \emph{Gebäude}|pw}. –
                     Telephon Nr. 384. Staats-Telephon Nr. 36.}}\pend
           
\pstart
           Mittwoch\pend
           
\pstart\center{}Lieber Arthur!\pend\vspace{0.5em}
\pstart
           Ich bin morgen Vormittag heraußen, doch nur bis 12, wo ich ins \label{K_L01102-1v}\edtext{Künſtlerhaus\oindex{Wien@\textbf{Wien}!I., Innere Stadt@\textbf{I., Innere Stadt}!Künstlerhaus@\textbf{Künstlerhaus}, \emph{Museum}|pw}}{\lemma{\textnormal{\emph{Künstlerhaus}}}\Cendnote{\textnormal{Am Samstag, 16. 3. 1901,
                  eröffnete die 23. Jahresausstellung.}}}\label{K_L01102-1} muß. Freitag,
                  Samſtag unbeſtimmt, wegen \label{K_L01102-2v}\edtext{Seceſſion\oindex{Wien@\textbf{Wien}!I., Innere Stadt@\textbf{I., Innere Stadt}!Secession@\textbf{Secession}, \emph{Museum}|pw}}{\lemma{\textnormal{\emph{Secession}}}\Cendnote{\textnormal{Am Freitag, 15. 3. 1901,
                  eröffnete die 10. Jahresausstellung.}}}\label{K_L01102-2}. Ganz sicher Dienſtag,
               den ganzen Vormittag. Haſt Du was Dringendes,{ }ſo morgen oder Samſtag um
               6 in meiner Redaction\oindex{Wien@\textbf{Wien}!I., Innere Stadt@\textbf{I., Innere Stadt}!Steyrerhof@\textbf{Steyrerhof}, \emph{Gebäude}|pw}.\pend
           
\pstart
           Herzlichſt{\\[\baselineskip]}Dein{\\[\baselineskip]}\spacefill\mbox{Hermann}\pend
           \leftskip=0em{}\selectlanguage{ngerman}\endnumbering\briefempfaengerindex{Schnitzler, Arthur@\textsc{Schnitzler, Arthur}!zzzBahr, Hermann@\emph{von Hermann Bahr}!1901-03-131@{{[}13. 3.? 1901{]}}|)be}\mylabel{L01102h}  \newcommand{\dateiname}{L01102}\newcommand{\titel}{Hermann Bahr an Arthur Schnitzler, [13. 3.? 1901]}\newcommand{\editorInnen}{Herausgegeben von Martin Anton Müller}%% latex-leseansicht-abspann.tex
%% Abspann für die Leseansicht.
%% Der Schalter \ifkorrekturansicht ist bereits durch den Vorspann gesetzt.

%% latex-abspann.tex
%% Gemeinsamer Abspann für Korrekturansicht und Leseansicht.
%% Setzt den Schalter \ifkorrekturansicht voraus (gesetzt in den
%% einbindenden Dateien latex-korrekturansicht-abspann.tex bzw.
%% latex-leseansicht-abspann.tex).
%% ---------------------------------------------------------------

\normalsize

% Das esempio-Environment wird nur in der Leseansicht benötigt
\ifkorrekturansicht\else
\newenvironment{esempio}[3]%
{
    \vspace{1.5ex}
    \rlap{\underline{#1}}
    \par
    \setlength{\parindent}{0cm}
    \nopagebreak
    \leftskip=#2cm
    \rightskip=#3cm
}
{
    \par
}
\fi

\doendnotes{C}
\bigskip
\vfill

\clearpage

\footnotesize

\ifkorrekturansicht
  \lohead{\textsc{register}}
\fi

% theindex-Environment neu definieren ohne reledmac
\makeatletter
\renewenvironment{theindex}{%
  \ifkorrekturansicht
    \section*{\indexname}%
  \else
    \subsubsection*{Index der erwähnten Entitäten}%
  \fi
  \setlength{\parindent}{0pt}%
  \setlength{\parskip}{0pt plus 0.3pt}%
  \let\item\@idxitem
}{%
  \ifkorrekturansicht\clearpage\fi
}
\makeatother

\IfFileExists{\jobname-pw.ind}{\input{\jobname-pw.ind}}{}

% Quellenangabe nur in der Leseansicht
\ifkorrekturansicht\else
% Fallback-Definitionen, falls die .tex-Datei \titel etc. nicht gesetzt hat
\providecommand{\titel}{}
\providecommand{\editorInnen}{}
\providecommand{\dateiname}{\jobname}

\vspace{3cm}

\vfill

\footnotesize
\textsc{Quelle}: \titel. Herausgegeben von {\editorInnen}. In: \emph{Arthur Schnitzler: Briefwechsel mit Autorinnen und Autoren}.
 Digitale Edition, https://schnitzler-briefe.acdh.oeaw.ac.at/{\dateiname}.html (Stand \today)
\fi

\end{document}


