%% latex-korrekturansicht-vorspann.tex
%% Vorspann für die Korrekturansicht.
%% Lädt die gemeinsame Datei latex-vorspann.tex mit gesetztem Schalter.

\newif\ifkorrekturansicht
\korrekturansichttrue

\input{../tex-inputs/latex-vorspann}


\section[Hermann Bahr an Arthur Schnitzler, {[}13. 3.? 1901{]}]{L01102 Hermann Bahr an Arthur Schnitzler, {[}13. 3.? 1901{]}}
\nopagebreak\mylabel{L01102v}
\rehead{ }\normalsize\beginnumbering\briefempfaengerindex{Schnitzler, Arthur@\textsc{Schnitzler, Arthur}!zzzBahr, Hermann@\emph{von Hermann Bahr}!1901-03-131@{{[}13. 3.? 1901{]}}|(be}
\toendnotes[C]{\smallbreak\pagebreak[2]}\Standort{CUL, Schnitzler, B 5b.}
\physDesc{Brief, 1 Blatt, 1 Seite, 275 Zeichen
\newline{}Handschrift: schwarze Tinte, deutsche Kurrent
\newline{}Schnitzler: mit Bleistift ergänztes Datum: »Feber? 901« 
\newline{}Ordnung: mit Bleistift von unbekannter Hand nummeriert:
                                    »74« }
\buchAbdrucke{\weitereDrucke{Hermann Bahr, Arthur Schnitzler: \emph{Briefwechsel, Aufzeichnungen, Dokumente (1891–1931)}. Göttingen: \emph{Wallstein} 2018, S. 201.} }\toendnotes[C]{\smallbreak}
\pstart
           \centering{}{\pb}\textcolor{gray}{\textbf{Redaktion des Neuen Wiener Tagblatt\orgindex{Neues Wiener Tagblatt@Neues Wiener Tagblatt|pw}}}\pend
           
\pstart
           \centering{}\textcolor{gray}{\textbf{\textsc{Wien, I., Rothenturmstrasse,
                        Steyrerhof\oindex{Steyrerhof@\textbf{Steyrerhof}, \emph{Gebäude (K.GBD)}|pw}.}}}\pend
           
\pstart
           \centering{}\textcolor{gray}{\textbf{Telegramm-Adresse: Tagblatt\orgindex{Neues Wiener Tagblatt@Neues Wiener Tagblatt|pw}, Steyrerhof, Wien\oindex{Steyrerhof@\textbf{Steyrerhof}, \emph{Gebäude (K.GBD)}|pw}. –
                     Telephon Nr. 384. Staats-Telephon Nr. 36.}}\pend
           
\pstart
           Mittwoch\pend
           
\pstart\center{}Lieber Arthur!\pend\vspace{0.5em}
\pstart
           Ich bin morgen Vormittag heraußen, doch nur bis 12, wo ich ins \label{K_L01102-1v}\edtext{Künſtlerhaus\oindex{Kuenstlerhaus@\textbf{Künstlerhaus}, \emph{Museum (K.MUS)}|pw}}{\lemma{\textnormal{\emph{Künſtlerhaus}}}\Cendnote{\textnormal{Am Samstag, 16. 3. 1901,
                  eröffnete die 23. Jahresausstellung.}}}\label{K_L01102-1} muß. Freitag,
                  Samſtag unbeſtimmt, wegen \label{K_L01102-2v}\edtext{Seceſſion\oindex{Secession@\textbf{Secession}, \emph{Museum (K.MUS)}|pw}}{\lemma{\textnormal{\emph{Seceſſion}}}\Cendnote{\textnormal{Am Freitag, 15. 3. 1901,
                  eröffnete die 10. Jahresausstellung.}}}\label{K_L01102-2}. Ganz sicher Dienſtag,
               den ganzen Vormittag. Haſt Du was Dringendes, ſo morgen oder Samſtag um
               6 in meiner Redaction\oindex{Steyrerhof@\textbf{Steyrerhof}, \emph{Gebäude (K.GBD)}|pw}.\pend
           
\pstart
           Herzlichſt{\\[\baselineskip]}Dein{\\[\baselineskip]}\spacefill\mbox{Hermann}\pend
           \leftskip=0em{}\selectlanguage{ngerman}\endnumbering\briefempfaengerindex{Schnitzler, Arthur@\textsc{Schnitzler, Arthur}!zzzBahr, Hermann@\emph{von Hermann Bahr}!1901-03-131@{{[}13. 3.? 1901{]}}|)be}\mylabel{L01102h}  \normalsize

\doendnotes{C}
\bigskip
\vfill

\clearpage

\footnotesize

\lohead{\textsc{register}}

% Definiere theindex-Environment komplett neu ohne reledmac
\makeatletter
\renewenvironment{theindex}{%
  \section*{\indexname}%
  \setlength{\parindent}{0pt}%
  \setlength{\parskip}{0pt plus 0.3pt}%
  \let\item\@idxitem
}{%
  \clearpage
}
\makeatother

\IfFileExists{\jobname-pw.ind}{\input{\jobname-pw.ind}}{}

\end{document}

      