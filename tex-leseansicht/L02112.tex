%% latex-korrekturansicht-vorspann.tex
%% Vorspann für die Korrekturansicht.
%% Lädt die gemeinsame Datei latex-vorspann.tex mit gesetztem Schalter.

\newif\ifkorrekturansicht
\korrekturansichttrue

\input{../tex-inputs/latex-vorspann}


\section[Hugo von Hofmannsthal an Arthur Schnitzler, 3. 1. 1913]{L02112 Hugo von Hofmannsthal an Arthur Schnitzler, 3. 1. 1913}
\nopagebreak\mylabel{L02112v}
\rehead{ }\normalsize\beginnumbering\briefempfaengerindex{Schnitzler, Arthur@\textsc{Schnitzler, Arthur}!zzzHofmannsthal, Hugo von@\emph{von Hugo von Hofmannsthal}!1913-01-031@{3. 1. 1913}|(be}
\toendnotes[C]{\smallbreak\pagebreak[2]}\Standort{CUL, Schnitzler, B 43.}
\physDesc{Brief, 1 Blatt, 3 Seiten, 1541 Zeichen
\newline{}Handschrift: schwarze Tinte, deutsche Kurrent
\newline{}Schnitzler: mit Bleistift beschriftet: »\textsc{Hofmannsthal}« 
\newline{}Ordnung: 1) mit Bleistift von unbekannter Hand nummeriert: »\strikeout{333}«  2) mit Bleistift von unbekannter Hand nummeriert:
                                    »346«}
\buchAbdrucke{\weitereDrucke{Hugo von Hofmannsthal, Arthur Schnitzler: \emph{Briefwechsel}. Frankfurt am Main: \emph{S. Fischer} 1964, S. 271–272.} }\toendnotes[C]{\smallbreak}
\pstart
           {\pb}\textcolor{gray}{\textbf{Schloss Neubeuern\oindex{Neubeuern@\textbf{Neubeuern}, \emph{P.PPL}|pw}{ }\textsuperscript{a}/Inn}}\pend
           
\pstart
           \textcolor{gray}{\textbf{Oberbayern}}\pend
           
\pstart
           \raggedleft{}3 I 13.\pend
           
\pstart{}mein lieber Arthur \pend\vspace{0.5em}
\pstart
           Dr. Eger\pwindex{Eger, Paul 23.01.1881 – 09.04.1947@\textsc{Eger, Paul} (23.01.1881 – 09.04.1947), \emph{Schriftsteller/Schriftstellerin, Theaterleiter/Theaterleiterin, Regisseur/Regisseurin}|pw} hat am 28. XII. die
               Sache durch ein directes Geſpräch mit Thimig\pwindex{Thimig, Hugo 16.06.1854 – 24.09.1944@\textsc{Thimig, Hugo} (16.06.1854 – 24.09.1944), \emph{Theaterleiter/Theaterleiterin, Schauspieler/Schauspielerin}|pw}
               recht gut eingeleitet so daſs ich nun ganz ausnahmsweiſe die \strikeout{directe} Bitte an Sie ſtellen möchte, eine Begegnung mit
               dem gleichen Mann mir zu Liebe und mit directem Hinweis auf meine Perſon und meine an
               Sie gerichtete Bitte in der allernächſten Zeit zu ſuchen, nicht mehr ihre
               Herbeiführung dem Zufall zu überlaſſen. Denn es liegt mir doch recht viel an der
               Sache und ſie hat einigermaßen Eile, weil der einzig mögliche Termin vor
                  Oſtern iſt, und zwar 8–10 Tage \uline{vor}{ }Oſtern mindeſtens, und Oſtern fällt ſchon auf den \label{K_L02112-1v}\edtext{22\textsuperscript{ten} März}{\lemma{\textnormal{\emph{22\textsuperscript{ten} März}}}\Cendnote{\textnormal{Ostersonntag war der
                     23. 3. 1913.}}}\label{K_L02112-1}.\pend
           
\pstart
           Thimigs\pwindex{Thimig, Hugo 16.06.1854 – 24.09.1944@\textsc{Thimig, Hugo} (16.06.1854 – 24.09.1944), \emph{Theaterleiter/Theaterleiterin, Schauspieler/Schauspielerin}|pw} einziges Bedenken war, die Kritik
               könne die Reinhardt\pwindex{Reinhardt, Max 09.09.1873 – 30.10.1943@\textsc{Reinhardt, Max} (09.09.1873 – 30.10.1943), \emph{Theaterleiter/Theaterleiterin, Regisseur/Regisseurin, Schauspieler/Schauspielerin}|pw}ſche Aufführung gegen ihn
               ausſpielen, worauf schon Eger\pwindex{Eger, Paul 23.01.1881 – 09.04.1947@\textsc{Eger, Paul} (23.01.1881 – 09.04.1947), \emph{Schriftsteller/Schriftstellerin, Theaterleiter/Theaterleiterin, Regisseur/Regisseurin}|pw} erwiderte:
               1.) ſchreibe gerade in den großen Blättern ein anderer Referent als {\pb}der über R.\pwindex{Reinhardt, Max 09.09.1873 – 30.10.1943@\textsc{Reinhardt, Max} (09.09.1873 – 30.10.1943), \emph{Theaterleiter/Theaterleiterin, Regisseur/Regisseurin, Schauspieler/Schauspielerin}|pw} geſchrieben habe, 2\textsuperscript{t\textcolor{gray}{ens}}: ſei, mit geringen Ausnahmen, immer noch eine reſpectvolle Prädispoſition für
               das Burgtheater\orgindex{Burgtheater@Burgtheater|pw} vorhanden und 3\textsuperscript{\textcolor{gray}{tens}} könne die Vorſtellung gerade dieſes Stückes\pwindex{Jedermann. Das Spiel vom Sterben des reichen Mannes@\emph{Jedermann. Das Spiel vom Sterben des reichen Mannes}|pwv} ganz vortrefflich \label{T_L02112-1v}\edtext{werden und werde (wenn man von dem einzigen \textsc{Moissi}\pwindex{Moissi, Alexander 02.04.1879 – 22.03.1935@\textsc{Moissi, Alexander} (02.04.1879 – 22.03.1935), \emph{Schauspieler/Schauspielerin}|pw} absehe) den Vergleich}{\lemma{\textnormal{\emph{werden … Vergleich}}}\Cendnote{\textnormal{durch
                  Umstellung korrigiert aus: »werden (wenn man von dem einzigen \textsc{Moissi}\pwindex{Moissi, Alexander 02.04.1879 – 22.03.1935@\textsc{Moissi, Alexander} (02.04.1879 – 22.03.1935), \emph{Schauspieler/Schauspielerin}|pw} absehe) und werde den Vergleich«.}}}\label{T_L02112-1} nicht zu ſcheuen
               haben.\pend
           
\pstart
           Ich bin in \uline{dieſem} Falle auch ſicher, dem Regiſſeur
               ſehr erfolgreich zur Seite ſein zu können, da mir nach Reinhardt\pwindex{Reinhardt, Max 09.09.1873 – 30.10.1943@\textsc{Reinhardt, Max} (09.09.1873 – 30.10.1943), \emph{Theaterleiter/Theaterleiterin, Regisseur/Regisseurin, Schauspieler/Schauspielerin}|pw} und nach Dresden\oindex{Dresden@\textbf{Dresden}, \emph{P.PPLA}|pw},
               jedes Detail des Sceniſchen und Schauſpieleriſchen mit ungewöhnlicher Präciſion
               innerlich zur Verfügung iſt.\hspace*{1.5em}Ich würde als Regiſſeur
                  Thimig\pwindex{Thimig, Hugo 16.06.1854 – 24.09.1944@\textsc{Thimig, Hugo} (16.06.1854 – 24.09.1944), \emph{Theaterleiter/Theaterleiterin, Schauspieler/Schauspielerin}|pw}{ }ſelbst oder Heine\pwindex{Heine, Albert 16.11.1867 – 13.4.1949@\textsc{Heine, Albert} (16.11.1867 – 13.4.1949), \emph{Theaterleiter/Theaterleiterin, Schauspieler/Schauspielerin}|pw} zur Bedingung machen.\pend
           
\pstart
           Ich wäre Ihnen herzlich dankbar, lieber Arthur. Ich bin etwa den 8\textsuperscript{ten} wieder in Rodaun\oindex{Rodaun@\textbf{Rodaun}, \emph{A.ADM4}|pw}, vielleicht finde ich da
               ein Wort von Ihnen.\pend
           
\pstart
           Ihr{\\[\baselineskip]}\spacefill\mbox{Hugo.}\pend
           \leftskip=0em{}\selectlanguage{ngerman}\endnumbering\briefempfaengerindex{Schnitzler, Arthur@\textsc{Schnitzler, Arthur}!zzzHofmannsthal, Hugo von@\emph{von Hugo von Hofmannsthal}!1913-01-031@{3. 1. 1913}|)be}\mylabel{L02112h}  \normalsize

\doendnotes{C}
\bigskip
\vfill

\clearpage

\footnotesize

\lohead{\textsc{register}}

% Definiere theindex-Environment komplett neu ohne reledmac
\makeatletter
\renewenvironment{theindex}{%
  \section*{\indexname}%
  \setlength{\parindent}{0pt}%
  \setlength{\parskip}{0pt plus 0.3pt}%
  \let\item\@idxitem
}{%
  \clearpage
}
\makeatother

\IfFileExists{\jobname-pw.ind}{\input{\jobname-pw.ind}}{}

\end{document}

      