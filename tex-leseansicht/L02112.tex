%% latex-leseansicht-vorspann.tex
%% Vorspann für die Leseansicht.
%% Lädt die gemeinsame Datei latex-vorspann.tex mit nicht gesetztem Schalter.

\newif\ifkorrekturansicht
\korrekturansichtfalse

\input{../tex-inputs/latex-vorspann}


         
         \newcommand{\erwaehntePersonen}{Personen: Paul Eger, Albert Heine, Alexander Moissi, Max Reinhardt, Hugo Thimig}
         \newcommand{\erwaehnteInstitutionen}{Institutionen: Burgtheater}
         \newcommand{\erwaehnteOrte}{Orte: Dresden, Neubeuern, Rodaun, Wien}
         \newcommand{\erwaehnteWerke}{Werke: Jedermann. Das Spiel vom Sterben des reichen Mannes}
               \section[Hugo von Hofmannsthal an Arthur Schnitzler, 3. 1. 1913]{ Hugo von Hofmannsthal an Arthur Schnitzler, 3. 1. 1913}\nopagebreak\mylabel{v}\rehead{ }\begin{ledgroupsized}[t]{13cm}\normalsize\beginnumbering \toendnotes[C]{\smallbreak\pagebreak[2]} \Standort{CUL, Schnitzler, B 43.}
\physDesc{Brief, 1 Blatt, 3 Seiten
\newline{}Handschrift: schwarze Tinte, deutsche Kurrent
\newline{}Schnitzler: mit Bleistift beschriftet: »\textsc{Hofmannsthal}« \newline{}Ordnung: 1) mit Bleistift von unbekannter Hand nummeriert: »\strikeout{333}«  2) mit Bleistift von unbekannter Hand nummeriert: »346«}\buchAbdrucke{\weitereDrucke{Hugo von Hofmannsthal, Arthur Schnitzler: \emph{Briefwechsel}. Hg. Therese Nickl und Heinrich Schnitzler. Frankfurt am Main: \emph{S. Fischer} 1964, S. 271–272.} }\toendnotes[C]{\smallbreak}\pstart
           \noindent{}{\pb}\textcolor{gray}{\textbf{Schloss Neubeuern\oindex{Neubeuern@\textbf{Neubeuern}|pw}{ }\textsuperscript{a}/Inn}}\pend
           \pstart
           \textcolor{gray}{\textbf{Oberbayern}}\pend
           \pstart
           \raggedleft{}3 I 13.\pend
           \pstart{}mein lieber Arthur \pend\pstart
           Dr. Eger\pwindex{Eger, Paul 23.01.1881 – 09.04.1947@\textsc{Eger, Paul} (23.01.1881 – 09.04.1947), \emph{Schriftsteller, Theaterleiter, Regisseur}|pw} hat am 28. XII. die Sache
               durch ein directes Geſpräch mit Thimig\pwindex{Thimig, Hugo 16.06.1854 – 24.09.1944@\textsc{Thimig, Hugo} (16.06.1854 – 24.09.1944), \emph{Theaterleiter, Schauspieler}|pw} recht gut
               eingeleitet so daſs ich nun ganz ausnahmsweiſe die \strikeout{directe} Bitte an Sie ſtellen möchte, eine Begegnung mit dem gleichen Mann
               mir zu Liebe und mit directem Hinweis auf meine Perſon und meine an Sie gerichtete
               Bitte in der allernächſten Zeit zu ſuchen, nicht mehr ihre Herbeiführung dem Zufall
               zu überlaſſen. Denn es liegt mir doch recht viel an der Sache und ſie hat
               einigermaßen Eile, weil der einzig mögliche Termin vor Oſtern iſt, und
               zwar 8–10 Tage \uline{vor}{ }Oſtern mindeſtens, und Oſtern fällt ſchon auf den \label{K_L02112_1v}\edtext{22\textsuperscript{ten} März}{\lemma{\textnormal{\emph{22ten März}}}\Cendnote{\textnormal{Ostersonntag war der
                     23. 3. 1913.}}}\label{K_L02112_1h}.\pend
           \pstart
           Thimig\pwindex{Thimig, Hugo 16.06.1854 – 24.09.1944@\textsc{Thimig, Hugo} (16.06.1854 – 24.09.1944), \emph{Theaterleiter, Schauspieler}|pw}s einziges Bedenken war, die Kritik könne
               die Reinhardt\pwindex{Reinhardt, Max 09.09.1873 – 30.10.1943@\textsc{Reinhardt, Max} (09.09.1873 – 30.10.1943), \emph{Theaterleiter, Regisseur, Schauspieler}|pw}ſche Aufführung gegen ihn
               ausſpielen, worauf schon Eger\pwindex{Eger, Paul 23.01.1881 – 09.04.1947@\textsc{Eger, Paul} (23.01.1881 – 09.04.1947), \emph{Schriftsteller, Theaterleiter, Regisseur}|pw} erwiderte:
               1.) ſchreibe gerade in den großen Blättern ein anderer Referent als {\pb}der über R.\pwindex{Reinhardt, Max 09.09.1873 – 30.10.1943@\textsc{Reinhardt, Max} (09.09.1873 – 30.10.1943), \emph{Theaterleiter, Regisseur, Schauspieler}|pw} geſchrieben habe, 2\textsuperscript{t\textcolor{gray}{ens}}: ſei, mit
               geringen Ausnahmen, immer noch eine reſpectvolle Prädispoſition für das Burgtheater\orgindex{Burgtheater@Burgtheater|pw} vorhanden und 3\textsuperscript{\textcolor{gray}{tens}} könne die Vorſtellung gerade dieſes Stückes\pwindex{Hofmannsthal, Hugo von 1874-02-01 – 1929-07-15@\textsc{Hofmannsthal, Hugo von} (1874-02-01 – 1929-07-15), \emph{Schriftsteller}!Jedermann. Das Spiel vom Sterben des reichen Mannes1911@\strich\emph{Jedermann. Das Spiel vom Sterben des reichen Mannes} {[}1911{]}|pwv} ganz vortrefflich \label{T_L02112_1v}\edtext{werden und werde (wenn man von dem
               einzigen \textsc{Moissi}\pwindex{Moissi, Alexander 02.04.1879 – 22.03.1935@\textsc{Moissi, Alexander} (02.04.1879 – 22.03.1935), \emph{Schauspieler}|pw} absehe) den Vergleich}{\lemma{\textnormal{\emph{werden … Vergleich}}}\Cendnote{\textnormal{durch Umstellung korrigiert aus:
                     »werden (wenn man von dem einzigen \textsc{Moissi}\pwindex{Moissi, Alexander 02.04.1879 – 22.03.1935@\textsc{Moissi, Alexander} (02.04.1879 – 22.03.1935), \emph{Schauspieler}|pw} absehe) und werde den Vergleich«.}}}\label{T_L02112_1h} nicht zu ſcheuen
               haben.\pend
           \pstart
           Ich bin in \uline{dieſem} Falle auch ſicher, dem Regiſſeur
               ſehr erfolgreich zur Seite ſein zu können, da mir nach Reinhardt\pwindex{Reinhardt, Max 09.09.1873 – 30.10.1943@\textsc{Reinhardt, Max} (09.09.1873 – 30.10.1943), \emph{Theaterleiter, Regisseur, Schauspieler}|pw} und nach Dresden\oindex{Dresden@\textbf{Dresden}|pw}, jedes Detail
               des Sceniſchen und Schauſpieleriſchen mit ungewöhnlicher Präciſion innerlich zur
               Verfügung iſt.\hspace*{1.5em}Ich würde als Regiſſeur Thimig\pwindex{Thimig, Hugo 16.06.1854 – 24.09.1944@\textsc{Thimig, Hugo} (16.06.1854 – 24.09.1944), \emph{Theaterleiter, Schauspieler}|pw}{ }ſelbst oder Heine\pwindex{Heine, Albert 16.11.1867 – 13.4.1949@\textsc{Heine, Albert} (16.11.1867 – 13.4.1949), \emph{Theaterleiter, Schauspieler}|pw} zur Bedingung machen.\pend
           \pstart
           Ich wäre Ihnen herzlich dankbar, lieber Arthur. Ich bin etwa den 8\textsuperscript{ten} wieder in Rodaun\oindex{Rodaun@\textbf{Rodaun}|pw}, vielleicht finde ich da
               ein Wort von Ihnen.\pend
           \pstart
           Ihr{\\[\baselineskip]}\spacefill\mbox{Hugo.}\pend
           \leftskip=0em{}
         
         \endnumbering\mylabel{h}\end{ledgroupsized}  \newcommand{\dateiname}{L02112}\newcommand{\titel}{Hugo von Hofmannsthal an Arthur Schnitzler, 3. 1. 1913}\newcommand{\editorInnen}{Martin Anton Müller und Gerd-Hermann Susen}%% latex-leseansicht-abspann.tex
%% Abspann für die Leseansicht.
%% Der Schalter \ifkorrekturansicht ist bereits durch den Vorspann gesetzt.

%% latex-abspann.tex
%% Gemeinsamer Abspann für Korrekturansicht und Leseansicht.
%% Setzt den Schalter \ifkorrekturansicht voraus (gesetzt in den
%% einbindenden Dateien latex-korrekturansicht-abspann.tex bzw.
%% latex-leseansicht-abspann.tex).
%% ---------------------------------------------------------------

\normalsize

% Das esempio-Environment wird nur in der Leseansicht benötigt
\ifkorrekturansicht\else
\newenvironment{esempio}[3]%
{
    \vspace{1.5ex}
    \rlap{\underline{#1}}
    \par
    \setlength{\parindent}{0cm}
    \nopagebreak
    \leftskip=#2cm
    \rightskip=#3cm
}
{
    \par
}
\fi

\doendnotes{C}
\bigskip
\vfill

\clearpage

\footnotesize

\ifkorrekturansicht
  \lohead{\textsc{register}}
\fi

% theindex-Environment neu definieren ohne reledmac
\makeatletter
\renewenvironment{theindex}{%
  \ifkorrekturansicht
    \section*{\indexname}%
  \else
    \subsubsection*{Index der erwähnten Entitäten}%
  \fi
  \setlength{\parindent}{0pt}%
  \setlength{\parskip}{0pt plus 0.3pt}%
  \let\item\@idxitem
}{%
  \ifkorrekturansicht\clearpage\fi
}
\makeatother

\IfFileExists{\jobname-pw.ind}{\input{\jobname-pw.ind}}{}

% Quellenangabe nur in der Leseansicht
\ifkorrekturansicht\else
% Fallback-Definitionen, falls die .tex-Datei \titel etc. nicht gesetzt hat
\providecommand{\titel}{}
\providecommand{\editorInnen}{}
\providecommand{\dateiname}{\jobname}

\vspace{3cm}

\vfill

\footnotesize
\textsc{Quelle}: \titel. Herausgegeben von {\editorInnen}. In: \emph{Arthur Schnitzler: Briefwechsel mit Autorinnen und Autoren}.
 Digitale Edition, https://schnitzler-briefe.acdh.oeaw.ac.at/{\dateiname}.html (Stand \today)
\fi

\end{document}


      