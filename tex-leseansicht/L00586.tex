%% latex-leseansicht-vorspann.tex
%% Vorspann für die Leseansicht.
%% Lädt die gemeinsame Datei latex-vorspann.tex mit nicht gesetztem Schalter.

\newif\ifkorrekturansicht
\korrekturansichtfalse

\input{../tex-inputs/latex-vorspann}


               \section[Arthur Schnitzler an Richard Beer-Hofmann, 7. 9. 1896]{ Arthur Schnitzler an Richard Beer-Hofmann, 7. 9. 1896}\nopagebreak\mylabel{v}\rehead{ }\begin{ledgroupsized}[t]{13cm}\normalsize\beginnumbering\briefempfaengerindex{Beer-Hofmann, Richard@\textsc{Beer-Hofmann, Richard}!zzzSchnitzler, Arthur@\emph{von Arthur Schnitzler}!1896-09-071@{7. 9. 1896}|(be} \toendnotes[C]{\smallbreak\pagebreak[2]} \Standort{YCGL, MSS 31.}
\physDesc{Kartenbrief
\newline{}Handschrift: Bleistift, deutsche Kurrent\newline{}Versand: 1) Stempel: »\nobreak{}\oindex{I., Innere Stadt@\textbf{I., Innere Stadt}|pwk}Wien 1/{[}1{]}, 8. 9. {[}96{]}, 8–9 {[}V{]}\nobreak{}«.  2) Stempel: »\nobreak{}\oindex{Baden bei Wien@\textbf{Baden bei Wien}|pwk}Baden 1, 8. 9. 96, 11–2N, Bestellt\nobreak{}«. }\buchAbdrucke{\weitereDrucke{Arthur Schnitzler, Richard Beer-Hofmann: \emph{Briefwechsel 1891–1931}. Hg. Konstanze Fliedl. Wien, Zürich: \emph{Europaverlag} 1992, S. 95–96.} }\toendnotes[C]{\smallbreak}\pstart{}{\pb}Herrn \textsc{Dr. Rich.
                     Beer-Hofmann}\pend{}\pstart{}\textsc{Baden bei Wien\oindex{Baden bei Wien@\textbf{Baden bei Wien}|pw}}\pend{}\pstart{}\textsc{Franzensgassse 54\oindex{Kaiser-Franz-Ring@\textbf{Kaiser-Franz-Ring}|pw}}, Thür 8\pend{}{\bigskip}\pstart
           \raggedleft{}{\pb}Montag\pend
           \pstart
           Lieber Richard, Ihre Karte hab ich bekommen. Morgen wollte ich zu
               Ihnen; aber plötzlich iſt \textsc{Sorma}\pwindex{Sorma, Agnes 17.05.1862 – 10.02.1927@\textsc{Sorma, Agnes} (17.05.1862 – 10.02.1927), \emph{Schauspielerin}|pw} u Gemahl\pwindex{Minotto, Demetrius Mito von 29.07.1856 – 11.05.1920@\textsc{Minotto, Demetrius Mito von} (29.07.1856 – 11.05.1920)|pwv} in Wien\oindex{Wien@\textbf{Wien}|pw} und ich ſpeiſe morgen mit ihnen. Ich ka{\geminationn} Ihnen alſo noch nicht genau ſagen, wann ich nach Baden\oindex{Baden bei Wien@\textbf{Baden bei Wien}|pw} fahre. Wie lange bleiben Sie noch draußen?
               Arbeiten Sie? Haben Sie mit Fiſcher\pwindex{Fischer, Samuel 24.12.1859 – 15.10.1934@\textsc{Fischer, Samuel} (24.12.1859 – 15.10.1934), \emph{Verleger}|pw}, mit Brahm\pwindex{Brahm, Otto 05.02.1856 – 28.11.1912@\textsc{Brahm, Otto} (05.02.1856 – 28.11.1912), \emph{Theaterleiter, Regisseur}|pw} geſprochen? – Von Hugo\pwindex{Hofmannsthal, Hugo von 01.02.1874 – 15.07.1929@\textsc{Hofmannsthal, Hugo von} (01.02.1874 – 15.07.1929), \emph{Schriftsteller}|pw} weiſs ich auch nichts, vor 8 Tagen hab ich ihm nach Alt-Auſſee\oindex{Altaussee@\textbf{Altaussee}|pw} geſchrieben. – Burckhard\pwindex{Burckhard, Max Eugen 14.07.1854 – 16.03.1912@\textsc{Burckhard, Max Eugen} (14.07.1854 – 16.03.1912), \emph{Schriftsteller, Rechtswissenschaftler, Theaterleiter}|pw} hat Freiwild\pwindex{Schnitzler, Arthur 15.05.1862 – 21.10.1931@\textsc{Schnitzler, Arthur} (15.05.1862 – 21.10.1931), \emph{Schriftsteller, Mediziner}!Freiwild. Schauspiel in 3 Akten1896@\strich\emph{Freiwild. Schauspiel in 3 Akten} {[}1896{]}|pw}
               geleſen u gratulirt Brahm\pwindex{Brahm, Otto 05.02.1856 – 28.11.1912@\textsc{Brahm, Otto} (05.02.1856 – 28.11.1912), \emph{Theaterleiter, Regisseur}|pw}, ders aufführen darf;
               hälts für den »pupillarſichern Senſationserfolg{[}«{]}, fährt nach Berlin\oindex{Berlin@\textbf{Berlin}|pw} zur \textsc{Première}. –\pend
           \pstart
           Herzlich Ihr{\\[\baselineskip]}\spacefill\mbox{Arthur}\pend
           \leftskip=0em{}\endnumbering\briefempfaengerindex{Beer-Hofmann, Richard@\textsc{Beer-Hofmann, Richard}!zzzSchnitzler, Arthur@\emph{von Arthur Schnitzler}!1896-09-071@{7. 9. 1896}|)be}\mylabel{h}\end{ledgroupsized}  \newcommand{\dateiname}{L00586}\newcommand{\titel}{Arthur Schnitzler an Richard Beer-Hofmann, 7. 9. 1896}\newcommand{\editorInnen}{Martin Anton Müller und Gerd-Hermann Susen}%% latex-leseansicht-abspann.tex
%% Abspann für die Leseansicht.
%% Der Schalter \ifkorrekturansicht ist bereits durch den Vorspann gesetzt.

%% latex-abspann.tex
%% Gemeinsamer Abspann für Korrekturansicht und Leseansicht.
%% Setzt den Schalter \ifkorrekturansicht voraus (gesetzt in den
%% einbindenden Dateien latex-korrekturansicht-abspann.tex bzw.
%% latex-leseansicht-abspann.tex).
%% ---------------------------------------------------------------

\normalsize

% Das esempio-Environment wird nur in der Leseansicht benötigt
\ifkorrekturansicht\else
\newenvironment{esempio}[3]%
{
    \vspace{1.5ex}
    \rlap{\underline{#1}}
    \par
    \setlength{\parindent}{0cm}
    \nopagebreak
    \leftskip=#2cm
    \rightskip=#3cm
}
{
    \par
}
\fi

\doendnotes{C}
\bigskip
\vfill

\clearpage

\footnotesize

\ifkorrekturansicht
  \lohead{\textsc{register}}
\fi

% theindex-Environment neu definieren ohne reledmac
\makeatletter
\renewenvironment{theindex}{%
  \ifkorrekturansicht
    \section*{\indexname}%
  \else
    \subsubsection*{Index der erwähnten Entitäten}%
  \fi
  \setlength{\parindent}{0pt}%
  \setlength{\parskip}{0pt plus 0.3pt}%
  \let\item\@idxitem
}{%
  \ifkorrekturansicht\clearpage\fi
}
\makeatother

\IfFileExists{\jobname-pw.ind}{\input{\jobname-pw.ind}}{}

% Quellenangabe nur in der Leseansicht
\ifkorrekturansicht\else
% Fallback-Definitionen, falls die .tex-Datei \titel etc. nicht gesetzt hat
\providecommand{\titel}{}
\providecommand{\editorInnen}{}
\providecommand{\dateiname}{\jobname}

\vspace{3cm}

\vfill

\footnotesize
\textsc{Quelle}: \titel. Herausgegeben von {\editorInnen}. In: \emph{Arthur Schnitzler: Briefwechsel mit Autorinnen und Autoren}.
 Digitale Edition, https://schnitzler-briefe.acdh.oeaw.ac.at/{\dateiname}.html (Stand \today)
\fi

\end{document}


      