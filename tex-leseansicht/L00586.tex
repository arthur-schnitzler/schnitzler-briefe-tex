%% latex-korrekturansicht-vorspann.tex
%% Vorspann für die Korrekturansicht.
%% Lädt die gemeinsame Datei latex-vorspann.tex mit gesetztem Schalter.

\newif\ifkorrekturansicht
\korrekturansichttrue

\input{../tex-inputs/latex-vorspann}


\section[Arthur Schnitzler an Richard Beer-Hofmann, 7. 9. 1896]{L00586 Arthur Schnitzler an Richard Beer-Hofmann, 7. 9. 1896}
\nopagebreak\mylabel{L00586v}
\rehead{ }\normalsize\beginnumbering\briefempfaengerindex{Beer-Hofmann, Richard@\textsc{Beer-Hofmann, Richard}!zzzSchnitzler, Arthur@\emph{von Arthur Schnitzler}!1896-09-071@{7. 9. 1896}|(be}
\toendnotes[C]{\smallbreak\pagebreak[2]}\Standort{YCGL, MSS 31.}
\physDesc{Kartenbrief, 621 Zeichen
\newline{}Handschrift: Bleistift, deutsche Kurrent
\newline{}Versand: 1) Stempel: »\nobreak{}\oindex{I., Innere Stadt@\textbf{I., Innere Stadt}, \emph{A.ADM3}|pwk}Wien 1/{[}1{]}, 8. 9. {[}96{]}, 8–9 {[}V{]}\nobreak{}«.   2) Stempel: »\nobreak{}\oindex{Baden bei Wien@\textbf{Baden bei Wien}, \emph{P.PPLA3}|pwk}Baden 1, 8. 9. 96, 11–2N, Bestellt\nobreak{}«. }
\buchAbdrucke{\weitereDrucke{Arthur Schnitzler, Richard Beer-Hofmann: \emph{Briefwechsel 1891–1931}. Wien, Zürich: \emph{Europaverlag} 1992, S. 95–96.} }\toendnotes[C]{\smallbreak}\pstart{}{\pb}Herrn \textsc{Dr. Rich.
                     Beer-Hofmann}\pend{}\pstart{}\textsc{Baden bei Wien\oindex{Baden bei Wien@\textbf{Baden bei Wien}, \emph{P.PPLA3}|pw}}\pend{}\pstart{}\textsc{Franzensgassse 54\oindex{Kaiser-Franz-Ring@\textbf{Kaiser-Franz-Ring}, \emph{Straße (K.STR)}|pw}}, Thür 8\pend{}{\bigskip}\vspace{1em}
\pstart
           \raggedleft{}{\pb}Montag\pend
           \vspace{0.5em}
\pstart
           Lieber Richard, Ihre Karte hab ich bekommen. Morgen wollte ich zu
               Ihnen; aber plötzlich iſt \textsc{Sorma}\pwindex{Sorma, Agnes 17.05.1862 – 10.02.1927@\textsc{Sorma, Agnes} (17.05.1862 – 10.02.1927), \emph{Schauspieler/Schauspielerin}|pw} u Gemahl\pwindex{Minotto, Demetrius Mito von 29.07.1856 – 11.05.1920@\textsc{Minotto, Demetrius Mito von} (29.07.1856 – 11.05.1920)|pwv} in Wien\oindex{Wien@\textbf{Wien}, \emph{A.ADM2}|pw} und ich ſpeiſe morgen mit ihnen. Ich ka{\geminationn} Ihnen alſo noch nicht genau ſagen, wann ich nach Baden\oindex{Baden bei Wien@\textbf{Baden bei Wien}, \emph{P.PPLA3}|pw} fahre. Wie lange bleiben Sie noch draußen?
               Arbeiten Sie? Haben Sie mit Fiſcher\pwindex{Fischer, Samuel 24.12.1859 – 15.10.1934@\textsc{Fischer, Samuel} (24.12.1859 – 15.10.1934), \emph{Verleger/Verlegerin}|pw}, mit Brahm\pwindex{Brahm, Otto 05.02.1856 – 28.11.1912@\textsc{Brahm, Otto} (05.02.1856 – 28.11.1912), \emph{Theaterleiter/Theaterleiterin, Regisseur/Regisseurin}|pw} geſprochen? – Von Hugo\pwindex{Hofmannsthal, Hugo von 1874-02-01 – 1929-07-15@\textsc{Hofmannsthal, Hugo von} (1874-02-01 – 1929-07-15), \emph{Schriftsteller/Schriftstellerin}|pw} weiſs ich auch nichts, vor 8 Tagen hab ich ihm nach Alt-Auſſee\oindex{Altaussee@\textbf{Altaussee}, \emph{A.ADM3}|pw} geſchrieben. – Burckhard\pwindex{Burckhard, Max Eugen 14.07.1854 – 16.03.1912@\textsc{Burckhard, Max Eugen} (14.07.1854 – 16.03.1912), \emph{Schriftsteller/Schriftstellerin, Rechtswissenschaftler/Rechtswissenschaftlerin, Theaterleiter/Theaterleiterin}|pw} hat Freiwild\pwindex{Freiwild. Schauspiel in 3 Akten@\emph{Freiwild. Schauspiel in 3 Akten}|pw}
               geleſen u gratulirt Brahm\pwindex{Brahm, Otto 05.02.1856 – 28.11.1912@\textsc{Brahm, Otto} (05.02.1856 – 28.11.1912), \emph{Theaterleiter/Theaterleiterin, Regisseur/Regisseurin}|pw}, ders aufführen
               darf; hälts für den »pupillarſichern Senſationserfolg{[}«{]}, fährt
               nach Berlin\oindex{Berlin@\textbf{Berlin}, \emph{P.PPLC}|pw} zur \textsc{Première}. –\pend
           
\pstart
           Herzlich Ihr{\\[\baselineskip]}\spacefill\mbox{Arthur}\pend
           \leftskip=0em{}\selectlanguage{ngerman}\endnumbering\briefempfaengerindex{Beer-Hofmann, Richard@\textsc{Beer-Hofmann, Richard}!zzzSchnitzler, Arthur@\emph{von Arthur Schnitzler}!1896-09-071@{7. 9. 1896}|)be}\mylabel{L00586h}  \normalsize

\doendnotes{C}
\bigskip
\vfill

\clearpage

\footnotesize

\lohead{\textsc{register}}

% Definiere theindex-Environment komplett neu ohne reledmac
\makeatletter
\renewenvironment{theindex}{%
  \section*{\indexname}%
  \setlength{\parindent}{0pt}%
  \setlength{\parskip}{0pt plus 0.3pt}%
  \let\item\@idxitem
}{%
  \clearpage
}
\makeatother

\IfFileExists{\jobname-pw.ind}{\input{\jobname-pw.ind}}{}

\end{document}

      