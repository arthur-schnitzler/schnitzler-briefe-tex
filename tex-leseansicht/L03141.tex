%% latex-leseansicht-vorspann.tex
%% Vorspann für die Leseansicht.
%% Lädt die gemeinsame Datei latex-vorspann.tex mit nicht gesetztem Schalter.

\newif\ifkorrekturansicht
\korrekturansichtfalse

\input{../tex-inputs/latex-vorspann}


         
         \renewcommand{\erwaehntePersonen}{Personen: A. Courth, Guglielmo Ferrero, Hans Kurella, Cesare Lombroso, Felix Salten, August Strindberg}
         \renewcommand{\erwaehnteInstitutionen}{Institutionen: Philipp Reclam jun.}
         \renewcommand{\erwaehnteOrte}{Orte: Leipzig, Wien}
         \renewcommand{\erwaehnteWerke}{Werke: Das Weib als Verbrecherin und Prostituierte. Anthropologische Studien, gegründet auf eine Darstellung der Biologie und Psychologie des normalen Weibes., Genie und Irrsinn in ihren Beziehungen zum Gesetz, zur Kritik und zur Geschichte, La donna delinquente: La prostituta e la donna normale}
               \section[Felix Salten an Arthur Schnitzler, {[}28. 7. 1894{]}]{ Felix Salten an Arthur Schnitzler, {[}28. 7. 1894{]}}\nopagebreak\mylabel{v}\rehead{ }\begin{ledgroupsized}[t]{13cm}\normalsize\beginnumbering\briefempfaengerindex{Schnitzler, Arthur@\textsc{Schnitzler, Arthur}!zzzSalten, Felix@\emph{von Felix Salten}!1894-07-281@{{[}28. 7. 1894{]}}|(be} \toendnotes[C]{\smallbreak\pagebreak[2]} \Standort{CUL, Schnitzler, B 89, A 1.}
\physDesc{Brief, 1 Blatt, 1 Seite, 236 Zeichen
\newline{}Handschrift: Bleistift, lateinische Kurrent
\newline{}Schnitzler: mit Bleistift datiert: »\uline{28/7 94}.« 
\newline{}Ordnung: mit Bleistift von unbekannter Hand nummeriert: »42« }\toendnotes[C]{\smallbreak}\pstart
           \noindent{}{\pb}Lieber Freund, bitte können Sie mir jenes \label{K_L03141-1v}\edtext{Buch\pwindex{Courth, A. @\textsc{Courth, A.}, \emph{Übersetzer/Übersetzerin}!Genie und Irrsinn in ihren Beziehungen zum Gesetz, zur Kritik und zur
                  Geschichte1887@\strich\emph{Genie und Irrsinn in ihren Beziehungen zum Gesetz, zur Kritik und zur Geschichte} {[}Übersetzung, 1887{]}|pwv} von Lombroso\pwindex{Lombroso, Cesare 18.11.1836 – 19.10.1909@\textsc{Lombroso, Cesare} (18.11.1836 – 19.10.1909), \emph{Mediziner, Psychologe, Anthropologe}|pw}, das von Verbrecher {\kaufmannsund}
               Irrsinn handelt}{\lemma{\textnormal{\emph{Buch … handelt}}}\Cendnote{\textnormal{Cesare Lombroso\pwindex{Lombroso, Cesare 18.11.1836 – 19.10.1909@\textsc{Lombroso, Cesare} (18.11.1836 – 19.10.1909), \emph{Mediziner, Psychologe, Anthropologe}|pwk}: \emph{Genie und Irrsinn in ihren Beziehungen zum Gesetz, zur
                        Kritik und zur Geschichte}\pwindex{Courth, A. @\textsc{Courth, A.}, \emph{Übersetzer/Übersetzerin}!Genie und Irrsinn in ihren Beziehungen zum Gesetz, zur Kritik und zur
                  Geschichte1887@\strich\emph{Genie und Irrsinn in ihren Beziehungen zum Gesetz, zur Kritik und zur Geschichte} {[}Übersetzung, 1887{]}|pwk}. Übersetzt von A. Courth\pwindex{Courth, A. @\textsc{Courth, A.}, \emph{Übersetzer/Übersetzerin}|pwk}. Leipzig\oindex{Leipzig@\textbf{Leipzig}|pwk}: \emph{Reclam}\orgindex{Philipp Reclam jun.@Philipp Reclam jun.|pwk}{ }1887.}}}\label{K_L03141-1h}, nebst \uline{dem \label{K_L03141-2v}\edtext{neuen Werk\pwindex{Lombroso, Cesare 18.11.1836 – 19.10.1909@\textsc{Lombroso, Cesare} (18.11.1836 – 19.10.1909), \emph{Mediziner, Psychologe, Anthropologe}!Weib als Verbrecherin und Prostituierte. Anthropologische Studien,
                  gegruendet auf eine Darstellung der Biologie und Psychologie des normalen
                  Weibes.1893@\strich\emph{Das Weib als Verbrecherin und Prostituierte. Anthropologische Studien, gegründet auf eine Darstellung der Biologie und Psychologie des normalen Weibes.} {[}1893{]}|pwv}\pwindex{Ferrero, Guglielmo 1871-07-21 – 1942-08-03@\textsc{Ferrero, Guglielmo} (1871-07-21 – 1942-08-03), \emph{Historiker, Soziologe}!Weib als Verbrecherin und Prostituierte. Anthropologische Studien,
                  gegruendet auf eine Darstellung der Biologie und Psychologie des normalen
                  Weibes.1893@\strich\emph{Das Weib als Verbrecherin und Prostituierte. Anthropologische Studien, gegründet auf eine Darstellung der Biologie und Psychologie des normalen Weibes.} {[}1893{]}|pwv}}{\lemma{\textnormal{\emph{neuen Werk}}}\Cendnote{\textnormal{Eben war die Übersetzung des
                     gemeinsam mit Guglielmo Ferrero\pwindex{Ferrero, Guglielmo 1871-07-21 – 1942-08-03@\textsc{Ferrero, Guglielmo} (1871-07-21 – 1942-08-03), \emph{Historiker, Soziologe}|pwk}
                     verfassten Werkes \emph{La donna delinquente: La
                        prostituta e la donna normale}\pwindex{Lombroso, Cesare 18.11.1836 – 19.10.1909@\textsc{Lombroso, Cesare} (18.11.1836 – 19.10.1909), \emph{Mediziner, Psychologe, Anthropologe}!donna delinquente: La prostituta e la donna normale1893@\strich\emph{La donna delinquente: La prostituta e la donna normale} {[}1893{]}|pwk}\pwindex{Ferrero, Guglielmo 1871-07-21 – 1942-08-03@\textsc{Ferrero, Guglielmo} (1871-07-21 – 1942-08-03), \emph{Historiker, Soziologe}!donna delinquente: La prostituta e la donna normale1893@\strich\emph{La donna delinquente: La prostituta e la donna normale} {[}1893{]}|pwk} (1893)
                     erschienen: \emph{Das Weib als Verbrecherin und Prostituierte.
                           Anthropologische Studien, gegründet auf eine Darstellung der Biologie und
                           Psychologie des normalen Weibes}\pwindex{Lombroso, Cesare 18.11.1836 – 19.10.1909@\textsc{Lombroso, Cesare} (18.11.1836 – 19.10.1909), \emph{Mediziner, Psychologe, Anthropologe}!Weib als Verbrecherin und Prostituierte. Anthropologische Studien,
                  gegruendet auf eine Darstellung der Biologie und Psychologie des normalen
                  Weibes.1893@\strich\emph{Das Weib als Verbrecherin und Prostituierte. Anthropologische Studien, gegründet auf eine Darstellung der Biologie und Psychologie des normalen Weibes.} {[}1893{]}|pwk}\pwindex{Ferrero, Guglielmo 1871-07-21 – 1942-08-03@\textsc{Ferrero, Guglielmo} (1871-07-21 – 1942-08-03), \emph{Historiker, Soziologe}!Weib als Verbrecherin und Prostituierte. Anthropologische Studien,
                  gegruendet auf eine Darstellung der Biologie und Psychologie des normalen
                  Weibes.1893@\strich\emph{Das Weib als Verbrecherin und Prostituierte. Anthropologische Studien, gegründet auf eine Darstellung der Biologie und Psychologie des normalen Weibes.} {[}1893{]}|pwk}. Autorisierte Übersetzung von H. Kurella\pwindex{Kurella, Hans 1858-02-20 – 1916-10-25@\textsc{Kurella, Hans} (1858-02-20 – 1916-10-25), \emph{Psychiater, Neurologe}|pwk}.
                        Hamburg: \emph{Verlagsanstalt und Druckerei A.G.
                           (Vorm. J. F. Richter)}{ }1894. Strindberg\pwindex{Strindberg, August 22.01.1849 – 14.05.1912@\textsc{Strindberg, August} (22.01.1849 – 14.05.1912), \emph{Schriftsteller}|pwk} wird darin nicht
                     erwähnt. Im Hinblick auf die Vorgeschichte (vgl. Felix Salten an Arthur Schnitzler, 10. 8. 1892) überrascht Saltens\pwindex{Salten, Felix 06.09.1869 – 08.10.1945@\textsc{Salten, Felix} (06.09.1869 – 08.10.1945), \emph{Schriftsteller, Journalist, Chefredakteur}|pwk} Unbekümmertheit, Schnitzler\pwindex{Schnitzler, Arthur 15.05.1862 – 21.10.1931@\textsc{Schnitzler, Arthur} (15.05.1862 – 21.10.1931), \emph{Schriftsteller, Mediziner}|pwk} um Bücher von Lombroso\pwindex{Lombroso, Cesare 18.11.1836 – 19.10.1909@\textsc{Lombroso, Cesare} (18.11.1836 – 19.10.1909), \emph{Mediziner, Psychologe, Anthropologe}|pwk} zu bitten.}}}\label{K_L03141-2h}, in welchem Strindberg\pwindex{Strindberg, August 22.01.1849 – 14.05.1912@\textsc{Strindberg, August} (22.01.1849 – 14.05.1912), \emph{Schriftsteller}|pw} als wahnsinnig} bezeichnet wird, auf ein
               paar Tage leihen? Ich habe mich auf beides zu beziehen.\pend
           \pstart Ihr \spacefill\mbox{Salten}\pend{}
         
         \endnumbering\mylabel{h}\end{ledgroupsized}  \newcommand{\dateiname}{L03141}\newcommand{\titel}{Felix Salten an Arthur Schnitzler, [28. 7. 1894]}\newcommand{\editorInnen}{Martin Anton Müller und Laura Untner}%% latex-leseansicht-abspann.tex
%% Abspann für die Leseansicht.
%% Der Schalter \ifkorrekturansicht ist bereits durch den Vorspann gesetzt.

%% latex-abspann.tex
%% Gemeinsamer Abspann für Korrekturansicht und Leseansicht.
%% Setzt den Schalter \ifkorrekturansicht voraus (gesetzt in den
%% einbindenden Dateien latex-korrekturansicht-abspann.tex bzw.
%% latex-leseansicht-abspann.tex).
%% ---------------------------------------------------------------

\normalsize

% Das esempio-Environment wird nur in der Leseansicht benötigt
\ifkorrekturansicht\else
\newenvironment{esempio}[3]%
{
    \vspace{1.5ex}
    \rlap{\underline{#1}}
    \par
    \setlength{\parindent}{0cm}
    \nopagebreak
    \leftskip=#2cm
    \rightskip=#3cm
}
{
    \par
}
\fi

\doendnotes{C}
\bigskip
\vfill

\clearpage

\footnotesize

\ifkorrekturansicht
  \lohead{\textsc{register}}
\fi

% theindex-Environment neu definieren ohne reledmac
\makeatletter
\renewenvironment{theindex}{%
  \ifkorrekturansicht
    \section*{\indexname}%
  \else
    \subsubsection*{Index der erwähnten Entitäten}%
  \fi
  \setlength{\parindent}{0pt}%
  \setlength{\parskip}{0pt plus 0.3pt}%
  \let\item\@idxitem
}{%
  \ifkorrekturansicht\clearpage\fi
}
\makeatother

\IfFileExists{\jobname-pw.ind}{\input{\jobname-pw.ind}}{}

% Quellenangabe nur in der Leseansicht
\ifkorrekturansicht\else
% Fallback-Definitionen, falls die .tex-Datei \titel etc. nicht gesetzt hat
\providecommand{\titel}{}
\providecommand{\editorInnen}{}
\providecommand{\dateiname}{\jobname}

\vspace{3cm}

\vfill

\footnotesize
\textsc{Quelle}: \titel. Herausgegeben von {\editorInnen}. In: \emph{Arthur Schnitzler: Briefwechsel mit Autorinnen und Autoren}.
 Digitale Edition, https://schnitzler-briefe.acdh.oeaw.ac.at/{\dateiname}.html (Stand \today)
\fi

\end{document}


      