%% latex-korrekturansicht-vorspann.tex
%% Vorspann für die Korrekturansicht.
%% Lädt die gemeinsame Datei latex-vorspann.tex mit gesetztem Schalter.

\newif\ifkorrekturansicht
\korrekturansichttrue

\input{../tex-inputs/latex-vorspann}


\section[Felix Salten an Arthur Schnitzler, {[}28. 7. 1894{]}]{L03141 Felix Salten an Arthur Schnitzler, {[}28. 7. 1894{]}}
\nopagebreak\mylabel{L03141v}
\rehead{ }\normalsize\beginnumbering\briefempfaengerindex{Schnitzler, Arthur@\textsc{Schnitzler, Arthur}!zzzSalten, Felix@\emph{von Felix Salten}!1894-07-281@{{[}28. 7. 1894{]}}|(be}
\toendnotes[C]{\smallbreak\pagebreak[2]}\Standort{CUL, Schnitzler, B 89, A 1.}
\physDesc{Brief, 1 Blatt, 1 Seite, 236 Zeichen
\newline{}Handschrift: Bleistift, lateinische Kurrent
\newline{}Schnitzler: mit Bleistift datiert: »\uline{28/7 94}.« 
\newline{}Ordnung: mit Bleistift von unbekannter Hand nummeriert: »42« }\toendnotes[C]{\smallbreak}
\pstart
           \noindent{}{\pb}Lieber Freund, bitte können Sie mir jenes \label{K_L03141-1v}\edtext{Buch\pwindex{Genie und Irrsinn in ihren Beziehungen zum Gesetz, zur Kritik und zur Geschichte@\emph{Genie und Irrsinn in ihren Beziehungen zum Gesetz, zur Kritik und zur Geschichte}|pwv} von Lombroso\pwindex{Lombroso, Cesare 18.11.1836 – 19.10.1909@\textsc{Lombroso, Cesare} (18.11.1836 – 19.10.1909), \emph{Mediziner/Medizinerin, Psychologe/Psychologin, Anthropologe/Anthropologin}|pw}, das von Verbrecher {\kaufmannsund}
               Irrsinn handelt}{\lemma{\textnormal{\emph{Buch … handelt}}}\Cendnote{\textnormal{Cesare Lombroso\pwindex{Lombroso, Cesare 18.11.1836 – 19.10.1909@\textsc{Lombroso, Cesare} (18.11.1836 – 19.10.1909), \emph{Mediziner/Medizinerin, Psychologe/Psychologin, Anthropologe/Anthropologin}|pwk}: \emph{Genie und Irrsinn in ihren Beziehungen zum Gesetz, zur
                        Kritik und zur Geschichte}\pwindex{Genie und Irrsinn in ihren Beziehungen zum Gesetz, zur Kritik und zur Geschichte@\emph{Genie und Irrsinn in ihren Beziehungen zum Gesetz, zur Kritik und zur Geschichte}|pwk}. Übersetzt von A. Courth\pwindex{Courth, A. @\textsc{Courth, A.}, \emph{Übersetzer/Übersetzerin}|pwk}. Leipzig\oindex{Leipzig@\textbf{Leipzig}, \emph{P.PPLA3}|pwk}: \emph{Reclam}\orgindex{Philipp Reclam jun.@Philipp Reclam jun.|pwk}{ }1887.}}}\label{K_L03141-1}, nebst \uline{dem \label{K_L03141-2v}\edtext{neuen Werk\pwindex{Weib als Verbrecherin und Prostituierte. Anthropologische Studien, gegruendet auf eine Darstellung der Biologie und Psychologie des normalen Weibes.@\emph{Das Weib als Verbrecherin und Prostituierte. Anthropologische Studien, gegründet auf eine Darstellung der Biologie und Psychologie des normalen Weibes.}|pwv}}{\lemma{\textnormal{\emph{neuen Werk}}}\Cendnote{\textnormal{Eben war die Übersetzung des
                     gemeinsam mit Guglielmo Ferrero\pwindex{Ferrero, Guglielmo 1871-07-21 – 1942-08-03@\textsc{Ferrero, Guglielmo} (1871-07-21 – 1942-08-03), \emph{Historiker/Historikerin, Soziologe/Soziologin}|pwk}
                     verfassten Werkes \emph{La donna delinquente: La
                        prostituta e la donna normale}\pwindex{donna delinquente: La prostituta e la donna normale@\emph{La donna delinquente: La prostituta e la donna normale}|pwk} (1893)
                     erschienen: \emph{Das Weib als Verbrecherin und Prostituierte.
                           Anthropologische Studien, gegründet auf eine Darstellung der Biologie und
                           Psychologie des normalen Weibes}\pwindex{Weib als Verbrecherin und Prostituierte. Anthropologische Studien, gegruendet auf eine Darstellung der Biologie und Psychologie des normalen Weibes.@\emph{Das Weib als Verbrecherin und Prostituierte. Anthropologische Studien, gegründet auf eine Darstellung der Biologie und Psychologie des normalen Weibes.}|pwk}. Autorisierte Übersetzung von H. Kurella\pwindex{Kurella, Hans 1858-02-20 – 1916-10-25@\textsc{Kurella, Hans} (1858-02-20 – 1916-10-25), \emph{Psychiater/Psychiaterin, Neurologe/Neurologin}|pwk}.
                        Hamburg: \emph{Verlagsanstalt und Druckerei A.G.
                           (Vorm. J. F. Richter)}{ }1894. Strindberg\pwindex{Strindberg, August 22.01.1849 – 14.05.1912@\textsc{Strindberg, August} (22.01.1849 – 14.05.1912), \emph{Schriftsteller/Schriftstellerin}|pwk} wird darin nicht
                     erwähnt. Im Hinblick auf die Vorgeschichte (vgl. Felix Salten an Arthur Schnitzler, 10. 8. 1892) überrascht Saltens\pwindex{Salten, Felix 06.09.1869 – 08.10.1945@\textsc{Salten, Felix} (06.09.1869 – 08.10.1945), \emph{Schriftsteller/Schriftstellerin, Journalist/Journalistin, Chefredakteur/Chefredakteurin}|pwk} Unbekümmertheit, Schnitzler um Bücher von Lombroso\pwindex{Lombroso, Cesare 18.11.1836 – 19.10.1909@\textsc{Lombroso, Cesare} (18.11.1836 – 19.10.1909), \emph{Mediziner/Medizinerin, Psychologe/Psychologin, Anthropologe/Anthropologin}|pwk} zu bitten.}}}\label{K_L03141-2}, in welchem Strindberg\pwindex{Strindberg, August 22.01.1849 – 14.05.1912@\textsc{Strindberg, August} (22.01.1849 – 14.05.1912), \emph{Schriftsteller/Schriftstellerin}|pw} als wahnsinnig} bezeichnet wird, auf ein
               paar Tage leihen? Ich habe mich auf beides zu beziehen.\pend
           \pstart Ihr \spacefill\mbox{Salten}\pend{}\selectlanguage{ngerman}\endnumbering\briefempfaengerindex{Schnitzler, Arthur@\textsc{Schnitzler, Arthur}!zzzSalten, Felix@\emph{von Felix Salten}!1894-07-281@{{[}28. 7. 1894{]}}|)be}\mylabel{L03141h}  \normalsize

\doendnotes{C}
\bigskip
\vfill

\clearpage

\footnotesize

\lohead{\textsc{register}}

% Definiere theindex-Environment komplett neu ohne reledmac
\makeatletter
\renewenvironment{theindex}{%
  \section*{\indexname}%
  \setlength{\parindent}{0pt}%
  \setlength{\parskip}{0pt plus 0.3pt}%
  \let\item\@idxitem
}{%
  \clearpage
}
\makeatother

\IfFileExists{\jobname-pw.ind}{\input{\jobname-pw.ind}}{}

\end{document}

      