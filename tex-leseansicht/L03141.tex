%% latex-leseansicht-vorspann.tex
%% Vorspann für die Leseansicht.
%% Lädt die gemeinsame Datei latex-vorspann.tex mit nicht gesetztem Schalter.

\newif\ifkorrekturansicht
\korrekturansichtfalse

\input{../tex-inputs/latex-vorspann}


\section[Felix Salten an Arthur Schnitzler, {{[}}28. 7. 1894{{]}}]{L03141 Felix Salten an Arthur Schnitzler, {[}28. 7. 1894{]}}
\nopagebreak\mylabel{L03141v}
\rehead{ }\normalsize\beginnumbering\briefempfaengerindex{Schnitzler, Arthur@\textsc{Schnitzler, Arthur}!zzzSalten, Felix@\emph{von Felix Salten}!1894-07-281@{{[}28. 7. 1894{]}}|(be}
\toendnotes[C]{\smallbreak\pagebreak[2]}
\correspDesc{Versand  durch Felix Salten am [28. 7. 1894] in Wien
\newline{}Erhalt  durch Arthur Schnitzler im Zeitraum [28. 7. 1894
                  – 31. 7. 1894?] in Wien}\toendnotes[C]{\smallbreak}
\Standort{CUL, Schnitzler, B 89, A 1.}
\physDesc{Brief, 1 Blatt, 1 Seite, 236 Zeichen
\newline{}Handschrift: Bleistift, lateinische Kurrent
\newline{}Schnitzler: mit Bleistift datiert: »\uline{28/7 94}.« 
\newline{}Ordnung: mit Bleistift von unbekannter Hand nummeriert: »42« }\toendnotes[C]{\smallbreak}
\pstart
           \noindent{}{\pb}Lieber Freund, bitte können Sie mir jenes \label{K_L03141-1v}\edtext{Buch\pwindex{Lombroso, Cesare 18.\,11.\,1836 Verona – 19.\,10.\,1909 Turin@\textsc{Lombroso, Cesare} (18.\,11.\,1836 Verona – 19.\,10.\,1909 Turin), \emph{Mediziner, Psychologe, Anthropologe}!Genie und Irrsinn in ihren Beziehungen zum Gesetz, zur Kritik und zur Geschichte@\strich\emph{Genie und Irrsinn in ihren Beziehungen zum Gesetz, zur Kritik und zur Geschichte}|pwv} von Lombroso\pwindex{Lombroso, Cesare 18.\,11.\,1836 Verona – 19.\,10.\,1909 Turin@\textsc{Lombroso, Cesare} (18.\,11.\,1836 Verona – 19.\,10.\,1909 Turin), \emph{Mediziner, Psychologe, Anthropologe}|pw}, das von Verbrecher {\kaufmannsund}
               Irrsinn handelt}{\lemma{\textnormal{\emph{Buch … handelt}}}\Cendnote{\textnormal{Cesare Lombroso\pwindex{Lombroso, Cesare 18.\,11.\,1836 Verona – 19.\,10.\,1909 Turin@\textsc{Lombroso, Cesare} (18.\,11.\,1836 Verona – 19.\,10.\,1909 Turin), \emph{Mediziner, Psychologe, Anthropologe}|pwk}: \emph{Genie und Irrsinn in ihren Beziehungen zum Gesetz, zur
                        Kritik und zur Geschichte}\pwindex{Lombroso, Cesare 18.\,11.\,1836 Verona – 19.\,10.\,1909 Turin@\textsc{Lombroso, Cesare} (18.\,11.\,1836 Verona – 19.\,10.\,1909 Turin), \emph{Mediziner, Psychologe, Anthropologe}!Genie und Irrsinn in ihren Beziehungen zum Gesetz, zur Kritik und zur Geschichte@\strich\emph{Genie und Irrsinn in ihren Beziehungen zum Gesetz, zur Kritik und zur Geschichte}|pwk}. Übersetzt von A. Courth\pwindex{Courth, A. @\textsc{Courth, A.}, \emph{Übersetzer/Übersetzerin}|pwk}. Leipzig\oindex{Leipzig@\textbf{Leipzig}, \emph{Hauptstadt}|pwk}: \emph{Reclam}\orgindex{Philipp Reclam jun.@Philipp Reclam jun.|pwk}{ }1887.}}}\label{K_L03141-1}, nebst \uline{dem \label{K_L03141-2v}\edtext{neuen Werk\pwindex{Lombroso, Cesare 18.\,11.\,1836 Verona – 19.\,10.\,1909 Turin@\textsc{Lombroso, Cesare} (18.\,11.\,1836 Verona – 19.\,10.\,1909 Turin), \emph{Mediziner, Psychologe, Anthropologe}!Weib als Verbrecherin und Prostituierte. Anthropologische Studien, gegründet auf eine Darstellung der Biologie und Psychologie des normalen Weibes.@\strich\emph{Das Weib als Verbrecherin und Prostituierte. Anthropologische Studien, gegründet auf eine Darstellung der Biologie und Psychologie des normalen Weibes.}|pwv}\pwindex{Ferrero, Guglielmo 21.\,7.\,1871 Portici – 3.\,8.\,1942 Mont-Pelerin@\textsc{Ferrero, Guglielmo} (21.\,7.\,1871 Portici – 3.\,8.\,1942 Mont-Pelerin), \emph{Historiker, Soziologe}!Weib als Verbrecherin und Prostituierte. Anthropologische Studien, gegründet auf eine Darstellung der Biologie und Psychologie des normalen Weibes.@\strich\emph{Das Weib als Verbrecherin und Prostituierte. Anthropologische Studien, gegründet auf eine Darstellung der Biologie und Psychologie des normalen Weibes.}|pwv}}{\lemma{\textnormal{\emph{neuen Werk}}}\Cendnote{\textnormal{Eben war die Übersetzung des
                     gemeinsam mit Guglielmo Ferrero\pwindex{Ferrero, Guglielmo 21.\,7.\,1871 Portici – 3.\,8.\,1942 Mont-Pelerin@\textsc{Ferrero, Guglielmo} (21.\,7.\,1871 Portici – 3.\,8.\,1942 Mont-Pelerin), \emph{Historiker, Soziologe}|pwk}
                     verfassten Werkes \emph{La donna delinquente: La
                        prostituta e la donna normale}\pwindex{Lombroso, Cesare 18.\,11.\,1836 Verona – 19.\,10.\,1909 Turin@\textsc{Lombroso, Cesare} (18.\,11.\,1836 Verona – 19.\,10.\,1909 Turin), \emph{Mediziner, Psychologe, Anthropologe}!donna delinquente: La prostituta e la donna normale@\strich\emph{La donna delinquente: La prostituta e la donna normale}|pwk}\pwindex{Ferrero, Guglielmo 21.\,7.\,1871 Portici – 3.\,8.\,1942 Mont-Pelerin@\textsc{Ferrero, Guglielmo} (21.\,7.\,1871 Portici – 3.\,8.\,1942 Mont-Pelerin), \emph{Historiker, Soziologe}!donna delinquente: La prostituta e la donna normale@\strich\emph{La donna delinquente: La prostituta e la donna normale}|pwk} (1893)
                     erschienen: \emph{Das Weib als Verbrecherin und Prostituierte.
                           Anthropologische Studien, gegründet auf eine Darstellung der Biologie und
                           Psychologie des normalen Weibes}\pwindex{Lombroso, Cesare 18.\,11.\,1836 Verona – 19.\,10.\,1909 Turin@\textsc{Lombroso, Cesare} (18.\,11.\,1836 Verona – 19.\,10.\,1909 Turin), \emph{Mediziner, Psychologe, Anthropologe}!Weib als Verbrecherin und Prostituierte. Anthropologische Studien, gegründet auf eine Darstellung der Biologie und Psychologie des normalen Weibes.@\strich\emph{Das Weib als Verbrecherin und Prostituierte. Anthropologische Studien, gegründet auf eine Darstellung der Biologie und Psychologie des normalen Weibes.}|pwk}\pwindex{Ferrero, Guglielmo 21.\,7.\,1871 Portici – 3.\,8.\,1942 Mont-Pelerin@\textsc{Ferrero, Guglielmo} (21.\,7.\,1871 Portici – 3.\,8.\,1942 Mont-Pelerin), \emph{Historiker, Soziologe}!Weib als Verbrecherin und Prostituierte. Anthropologische Studien, gegründet auf eine Darstellung der Biologie und Psychologie des normalen Weibes.@\strich\emph{Das Weib als Verbrecherin und Prostituierte. Anthropologische Studien, gegründet auf eine Darstellung der Biologie und Psychologie des normalen Weibes.}|pwk}. Autorisierte Übersetzung von H. Kurella\pwindex{Kurella, Hans 20.\,2.\,1858 Mainz – 25.\,10.\,1916 Dresden@\textsc{Kurella, Hans} (20.\,2.\,1858 Mainz – 25.\,10.\,1916 Dresden), \emph{Psychiater, Neurologe}|pwk}.
                        Hamburg: \emph{Verlagsanstalt und Druckerei A.G.
                           (Vorm. J. F. Richter)}{ }1894. Strindberg\pwindex{Strindberg, August 22.\,1.\,1849 Stockholm – 14.\,5.\,1912 ebd.@\textsc{Strindberg, August} (22.\,1.\,1849 Stockholm – 14.\,5.\,1912 ebd.), \emph{Schriftsteller}|pwk} wird darin nicht
                     erwähnt. Im Hinblick auf die Vorgeschichte (vgl. XXXX Auszeichnungsfehler: Dokument L03186 nicht gefunden) überrascht Saltens\pwindex{Salten, Felix 6.\,9.\,1869 Budapest – 8.\,10.\,1945 Zürich@\textsc{Salten, Felix} (6.\,9.\,1869 Budapest – 8.\,10.\,1945 Zürich), \emph{Schriftsteller, Journalist, Chefredakteur}|pwk} Unbekümmertheit, Schnitzler um Bücher von Lombroso\pwindex{Lombroso, Cesare 18.\,11.\,1836 Verona – 19.\,10.\,1909 Turin@\textsc{Lombroso, Cesare} (18.\,11.\,1836 Verona – 19.\,10.\,1909 Turin), \emph{Mediziner, Psychologe, Anthropologe}|pwk} zu bitten.}}}\label{K_L03141-2}, in welchem Strindberg\pwindex{Strindberg, August 22.\,1.\,1849 Stockholm – 14.\,5.\,1912 ebd.@\textsc{Strindberg, August} (22.\,1.\,1849 Stockholm – 14.\,5.\,1912 ebd.), \emph{Schriftsteller}|pw} als wahnsinnig} bezeichnet wird, auf ein
               paar Tage leihen? Ich habe mich auf beides zu beziehen.\pend
           \pstart Ihr \spacefill\mbox{Salten}\pend{}\selectlanguage{ngerman}\endnumbering\briefempfaengerindex{Schnitzler, Arthur@\textsc{Schnitzler, Arthur}!zzzSalten, Felix@\emph{von Felix Salten}!1894-07-281@{{[}28. 7. 1894{]}}|)be}\mylabel{L03141h}  \newcommand{\dateiname}{L03141}\newcommand{\titel}{Felix Salten an Arthur Schnitzler, [28. 7. 1894]}\newcommand{\editorInnen}{Martin Anton Müller und Laura Untner}%% latex-leseansicht-abspann.tex
%% Abspann für die Leseansicht.
%% Der Schalter \ifkorrekturansicht ist bereits durch den Vorspann gesetzt.

%% latex-abspann.tex
%% Gemeinsamer Abspann für Korrekturansicht und Leseansicht.
%% Setzt den Schalter \ifkorrekturansicht voraus (gesetzt in den
%% einbindenden Dateien latex-korrekturansicht-abspann.tex bzw.
%% latex-leseansicht-abspann.tex).
%% ---------------------------------------------------------------

\normalsize

% Das esempio-Environment wird nur in der Leseansicht benötigt
\ifkorrekturansicht\else
\newenvironment{esempio}[3]%
{
    \vspace{1.5ex}
    \rlap{\underline{#1}}
    \par
    \setlength{\parindent}{0cm}
    \nopagebreak
    \leftskip=#2cm
    \rightskip=#3cm
}
{
    \par
}
\fi

\doendnotes{C}
\bigskip
\vfill

\clearpage

\footnotesize

\ifkorrekturansicht
  \lohead{\textsc{register}}
\fi

% theindex-Environment neu definieren ohne reledmac
\makeatletter
\renewenvironment{theindex}{%
  \ifkorrekturansicht
    \section*{\indexname}%
  \else
    \subsubsection*{Index der erwähnten Entitäten}%
  \fi
  \setlength{\parindent}{0pt}%
  \setlength{\parskip}{0pt plus 0.3pt}%
  \let\item\@idxitem
}{%
  \ifkorrekturansicht\clearpage\fi
}
\makeatother

\IfFileExists{\jobname-pw.ind}{\input{\jobname-pw.ind}}{}

% Quellenangabe nur in der Leseansicht
\ifkorrekturansicht\else
% Fallback-Definitionen, falls die .tex-Datei \titel etc. nicht gesetzt hat
\providecommand{\titel}{}
\providecommand{\editorInnen}{}
\providecommand{\dateiname}{\jobname}

\vspace{3cm}

\vfill

\footnotesize
\textsc{Quelle}: \titel. Herausgegeben von {\editorInnen}. In: \emph{Arthur Schnitzler: Briefwechsel mit Autorinnen und Autoren}.
 Digitale Edition, https://schnitzler-briefe.acdh.oeaw.ac.at/{\dateiname}.html (Stand \today)
\fi

\end{document}


