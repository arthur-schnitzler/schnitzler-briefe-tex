%% latex-leseansicht-vorspann.tex
%% Vorspann für die Leseansicht.
%% Lädt die gemeinsame Datei latex-vorspann.tex mit nicht gesetztem Schalter.

\newif\ifkorrekturansicht
\korrekturansichtfalse

\input{../tex-inputs/latex-vorspann}

\begin{center}
            \textcolor{red}{ENTWURF, NICHT FERTIG KORRIGIERT}
                      \end{center}
            
         
         \renewcommand{\erwaehntePersonen}{Personen: Cesare Lombroso, August Strindberg}
         \renewcommand{\erwaehnteOrte}{Orte: Wien}
         \renewcommand{\erwaehnteWerke}{Werke: Das Weib als Verbrecherin und Prostituierte, Genie und Irrsinn in ihren Beziehungen zum Gesetz, zur Kritik und zur Geschichte, La donna delinquente: La prostituta e la donna normale}
               \section[Felix Salten an Arthur Schnitzler, {[}28. 7. 1894{]}]{ Felix Salten an Arthur Schnitzler, {[}28. 7. 1894{]}}\nopagebreak\mylabel{v}\rehead{ }\begin{ledgroupsized}[t]{13cm}\normalsize\beginnumbering \toendnotes[C]{\smallbreak\pagebreak[2]} \Standort{CUL, Schnitzler, B 89, A 1.}
\physDesc{Brief, 1 Blatt, 1 Seite, 236 Zeichen
\newline{}Handschrift: Bleistift, lateinische Kurrent
\newline{}Schnitzler: mit Bleistift datiert: »\uline{28/7 94.}« 
\newline{}Ordnung: mit Bleistift von unbekannter Hand nummeriert:
                                    »42« }\toendnotes[C]{\smallbreak}\pstart
           \noindent{}{\pb}Lieber Freund, bitte können Sie mir jenes Buch von Lombroso\pwindex{Lombroso, Cesare 18.11.1836 – 19.10.1909@\textsc{Lombroso, Cesare} (18.11.1836 – 19.10.1909), \emph{Mediziner, Psychologe, Anthropologe}|pw}, das von Verbrecher
                     {\kaufmannsund} Irrsinn\pwindex{Lombroso, Cesare 18.11.1836 – 19.10.1909@\textsc{Lombroso, Cesare} (18.11.1836 – 19.10.1909), \emph{Mediziner, Psychologe, Anthropologe}!Genie und Irrsinn in ihren Beziehungen zum Gesetz, zur Kritik und zur
                  Geschichte1887@\strich\emph{Genie und Irrsinn in ihren Beziehungen zum Gesetz, zur Kritik und zur Geschichte} {[}1887{]}|pw} handelt, nebst \uline{dem \label{K_L03141-1v}\edtext{neuen Werk\pwindex{Lombroso, Cesare 18.11.1836 – 19.10.1909@\textsc{Lombroso, Cesare} (18.11.1836 – 19.10.1909), \emph{Mediziner, Psychologe, Anthropologe}!Weib als Verbrecherin und Prostituierte1894@\strich\emph{Das Weib als Verbrecherin und Prostituierte} {[}1894{]}|pwv}\pwindex{\textcolor{red}{\textsuperscript{XXXX1 indx}}!Weib als Verbrecherin und Prostituierte1894@\strich\emph{Das Weib als Verbrecherin und Prostituierte} {[}1894{]}|pwv}}{\lemma{\textnormal{\emph{neuen Werk}}}\Cendnote{\textnormal{Eben war die Übersetzung von \emph{La donna delinquente: La prostituta e la donna
                        normale}\pwindex{Lombroso, Cesare 18.11.1836 – 19.10.1909@\textsc{Lombroso, Cesare} (18.11.1836 – 19.10.1909), \emph{Mediziner, Psychologe, Anthropologe}!donna delinquente: La prostituta e la donna normale1893@\strich\emph{La donna delinquente: La prostituta e la donna normale} {[}1893{]}|pwk}\pwindex{\textcolor{red}{\textsuperscript{XXXX1 indx}}!donna delinquente: La prostituta e la donna normale1893@\strich\emph{La donna delinquente: La prostituta e la donna normale} {[}1893{]}|pwk} – \emph{Das Weib als Verbrecherin
                        und Prostituierte}\pwindex{Lombroso, Cesare 18.11.1836 – 19.10.1909@\textsc{Lombroso, Cesare} (18.11.1836 – 19.10.1909), \emph{Mediziner, Psychologe, Anthropologe}!Weib als Verbrecherin und Prostituierte1894@\strich\emph{Das Weib als Verbrecherin und Prostituierte} {[}1894{]}|pwk}\pwindex{\textcolor{red}{\textsuperscript{XXXX1 indx}}!Weib als Verbrecherin und Prostituierte1894@\strich\emph{Das Weib als Verbrecherin und Prostituierte} {[}1894{]}|pwk} – erschienen, doch wird darin Strindberg\pwindex{Strindberg, August 22.01.1849 – 14.05.1912@\textsc{Strindberg, August} (22.01.1849 – 14.05.1912), \emph{Schriftsteller}|pwk} nicht erwähnt.}}}\label{K_L03141-1h}, in welchem Strindberg\pwindex{Strindberg, August 22.01.1849 – 14.05.1912@\textsc{Strindberg, August} (22.01.1849 – 14.05.1912), \emph{Schriftsteller}|pw} als wahnsinnig} bezeichnet
               wird, auf ein paar Tage leihen? Ich habe mich auf beides zu beziehen. \pend
           \pstart Ihr \spacefill\mbox{Salten}\pend{}
         
         \endnumbering\mylabel{h}\end{ledgroupsized}\begin{anhang}\end{anhang}\newcommand{\dateiname}{L03141}\newcommand{\titel}{Felix Salten an Arthur Schnitzler, [28. 7. 1894]}\newcommand{\editorInnen}{Martin Anton Müller und Laura Untner}%% latex-leseansicht-abspann.tex
%% Abspann für die Leseansicht.
%% Der Schalter \ifkorrekturansicht ist bereits durch den Vorspann gesetzt.

%% latex-abspann.tex
%% Gemeinsamer Abspann für Korrekturansicht und Leseansicht.
%% Setzt den Schalter \ifkorrekturansicht voraus (gesetzt in den
%% einbindenden Dateien latex-korrekturansicht-abspann.tex bzw.
%% latex-leseansicht-abspann.tex).
%% ---------------------------------------------------------------

\normalsize

% Das esempio-Environment wird nur in der Leseansicht benötigt
\ifkorrekturansicht\else
\newenvironment{esempio}[3]%
{
    \vspace{1.5ex}
    \rlap{\underline{#1}}
    \par
    \setlength{\parindent}{0cm}
    \nopagebreak
    \leftskip=#2cm
    \rightskip=#3cm
}
{
    \par
}
\fi

\doendnotes{C}
\bigskip
\vfill

\clearpage

\footnotesize

\ifkorrekturansicht
  \lohead{\textsc{register}}
\fi

% theindex-Environment neu definieren ohne reledmac
\makeatletter
\renewenvironment{theindex}{%
  \ifkorrekturansicht
    \section*{\indexname}%
  \else
    \subsubsection*{Index der erwähnten Entitäten}%
  \fi
  \setlength{\parindent}{0pt}%
  \setlength{\parskip}{0pt plus 0.3pt}%
  \let\item\@idxitem
}{%
  \ifkorrekturansicht\clearpage\fi
}
\makeatother

\IfFileExists{\jobname-pw.ind}{\input{\jobname-pw.ind}}{}

% Quellenangabe nur in der Leseansicht
\ifkorrekturansicht\else
% Fallback-Definitionen, falls die .tex-Datei \titel etc. nicht gesetzt hat
\providecommand{\titel}{}
\providecommand{\editorInnen}{}
\providecommand{\dateiname}{\jobname}

\vspace{3cm}

\vfill

\footnotesize
\textsc{Quelle}: \titel. Herausgegeben von {\editorInnen}. In: \emph{Arthur Schnitzler: Briefwechsel mit Autorinnen und Autoren}.
 Digitale Edition, https://schnitzler-briefe.acdh.oeaw.ac.at/{\dateiname}.html (Stand \today)
\fi

\end{document}


      