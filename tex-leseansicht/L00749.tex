%% latex-leseansicht-vorspann.tex
%% Vorspann für die Leseansicht.
%% Lädt die gemeinsame Datei latex-vorspann.tex mit nicht gesetztem Schalter.

\newif\ifkorrekturansicht
\korrekturansichtfalse

\input{../tex-inputs/latex-vorspann}


         
         \renewcommand{\erwaehntePersonen}{Personen: Elwin Paetel, Julius Rodenberg, Eduard Liberius Schiff}
         \renewcommand{\erwaehnteInstitutionen}{Institutionen: Deutsche Rundschau, Gebrüder Paetel Verlag}
         \renewcommand{\erwaehnteOrte}{Orte: Berlin, Lützowstraße, Margaretenstraße, Wien}
         \renewcommand{\erwaehnteWerke}{}
               \section[Julius Rodenberg an Arthur Schnitzler, 13. 12. 1897]{ Julius Rodenberg an Arthur Schnitzler, 13. 12. 1897}\nopagebreak\mylabel{v}\rehead{ }\begin{ledgroupsized}[t]{13cm}\normalsize\beginnumbering\briefempfaengerindex{Schnitzler, Arthur@\textsc{Schnitzler, Arthur}!zzzRodenberg, Julius@\emph{von Julius Rodenberg}!1897-12-131@{13. 12. 1897}|(be} \toendnotes[C]{\smallbreak\pagebreak[2]} \Standort{CUL, Schnitzler, B 85.}
\physDesc{Brief, 1 Blatt, 2 Seiten, 994 Zeichen
\newline{}Handschrift: schwarze Tinte, deutsche Kurrent
\newline{}Schnitzler: 1) mit rotem Buntstift vereinzelte Unterstreichungen  2) mit Bleistift beschriftet: »\textsc{Rodenberg}«}\pstart
           \noindent{}\centering{}{\pb}\textcolor{gray}{\textbf{Deutsche Rundschau\orgindex{Deutsche Rundschau@Deutsche Rundschau|pw}}}\pend
           \pstart
           \noindent{}\textcolor{gray}{\textbf{Expedition u. Redaction:}}\hfill \textcolor{gray}{\textbf{Herausgeber:}}\pend
           \pstart
           \textcolor{gray}{\textbf{Gebrüder Paetel\orgindex{Gebrueder Paetel Verlag@Gebrüder Paetel Verlag|pw} in Berlin\oindex{Berlin@\textbf{Berlin}|pw}}}\hfill \textcolor{gray}{\textbf{Julius Rodenberg in Berlin\oindex{Berlin@\textbf{Berlin}|pw}}}\pend
           \pstart
           \textcolor{gray}{\textbf{(Elwin Paetel\pwindex{Paetel, Elwin 13.11.1847 – 04.10.1907@\textsc{Paetel, Elwin} (13.11.1847 – 04.10.1907), \emph{Verleger}|pw})}}\hfill \textcolor{gray}{\textbf{W., Margarethenstr. 1\oindex{Margaretenstrasse@\textbf{Margaretenstraße}|pw}.}}\pend
           \pstart
           \textcolor{gray}{\textbf{W., Lützowstr. 7\oindex{Luetzowstrasse@\textbf{Lützowstraße}|pw}.}}\pend
           \pstart
           \raggedleft{}\textbf{\textcolor{gray}{\textbf{Berlin W.\oindex{Berlin@\textbf{Berlin}|pw},}} den}{ }13. Dec. \textcolor{gray}{\textbf{189}}7.\pend
           \pstart{}Hochgeehrter Herr Doctor!\pend\pstart
           Durch meinen Schwager Dr. Ed. Schiff\pwindex{Schiff, Eduard Liberius 04.03.1849 – 05.03.1913@\textsc{Schiff, Eduard Liberius} (04.03.1849 – 05.03.1913), \emph{Dermatologe}|pw} iſt mir
               die höchſt erfreuliche Kunde geworden, daß die »\textsc{Rundschau}\orgindex{Deutsche Rundschau@Deutsche Rundschau|pw}« ſich Hoffnung machen darf, in nicht allzuferner Zeit einen novelliſtiſchen
               Beitrag von Ihnen zu erhalten. Längſt ſchon iſt dieß mein Wunſch geweſen u. wenn ich
               ihn nicht eher ausſprach, ſo werden Sie ſich das daraus erklären können, daß ich mich
               nicht gern einem Refus ausgeſetzt haben würde. Nun iſt aber bei Ihnen freundliches
                  Entgegenko{\geminationm}en gefunden, will ich nicht zögern, Ihnen
               dafür zu danken u. meine Bitte direct zu wiederholen. Daß Sie dieſer im Augenblick
               nicht zu willfahren vermöchten, hab’ ich vorausgeſetzt, u. darauf ko{\geminationm}t es mir auch nicht an; es genügt mir, zu wißen, daß
               Sie bei nächſter Gelegenheit unſerer Zeitſchrift gedenken wollen, u. {\pb}ich bitte nur, mich eintretenden Falls zu
               benachrichtigen, um Sie nicht unnöthig lang mit dem Abdruck warten laßen zu
               müßen.\pend
           \pstart
           Mit dem Ausdruck beſonderer Hochachtung{\\[\baselineskip]}Ihr ergebener{\\[\baselineskip]}\spacefill\mbox{Dr Julius Rodenberg.}\pend
           \leftskip=0em{}
         
         \endnumbering\mylabel{h}\end{ledgroupsized}  \newcommand{\dateiname}{L00749}\newcommand{\titel}{Julius Rodenberg an Arthur Schnitzler, 13. 12. 1897}\newcommand{\editorInnen}{Martin Anton Müller und Gerd-Hermann Susen}%% latex-leseansicht-abspann.tex
%% Abspann für die Leseansicht.
%% Der Schalter \ifkorrekturansicht ist bereits durch den Vorspann gesetzt.

%% latex-abspann.tex
%% Gemeinsamer Abspann für Korrekturansicht und Leseansicht.
%% Setzt den Schalter \ifkorrekturansicht voraus (gesetzt in den
%% einbindenden Dateien latex-korrekturansicht-abspann.tex bzw.
%% latex-leseansicht-abspann.tex).
%% ---------------------------------------------------------------

\normalsize

% Das esempio-Environment wird nur in der Leseansicht benötigt
\ifkorrekturansicht\else
\newenvironment{esempio}[3]%
{
    \vspace{1.5ex}
    \rlap{\underline{#1}}
    \par
    \setlength{\parindent}{0cm}
    \nopagebreak
    \leftskip=#2cm
    \rightskip=#3cm
}
{
    \par
}
\fi

\doendnotes{C}
\bigskip
\vfill

\clearpage

\footnotesize

\ifkorrekturansicht
  \lohead{\textsc{register}}
\fi

% theindex-Environment neu definieren ohne reledmac
\makeatletter
\renewenvironment{theindex}{%
  \ifkorrekturansicht
    \section*{\indexname}%
  \else
    \subsubsection*{Index der erwähnten Entitäten}%
  \fi
  \setlength{\parindent}{0pt}%
  \setlength{\parskip}{0pt plus 0.3pt}%
  \let\item\@idxitem
}{%
  \ifkorrekturansicht\clearpage\fi
}
\makeatother

\IfFileExists{\jobname-pw.ind}{\input{\jobname-pw.ind}}{}

% Quellenangabe nur in der Leseansicht
\ifkorrekturansicht\else
% Fallback-Definitionen, falls die .tex-Datei \titel etc. nicht gesetzt hat
\providecommand{\titel}{}
\providecommand{\editorInnen}{}
\providecommand{\dateiname}{\jobname}

\vspace{3cm}

\vfill

\footnotesize
\textsc{Quelle}: \titel. Herausgegeben von {\editorInnen}. In: \emph{Arthur Schnitzler: Briefwechsel mit Autorinnen und Autoren}.
 Digitale Edition, https://schnitzler-briefe.acdh.oeaw.ac.at/{\dateiname}.html (Stand \today)
\fi

\end{document}


      