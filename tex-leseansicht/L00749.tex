%% latex-korrekturansicht-vorspann.tex
%% Vorspann für die Korrekturansicht.
%% Lädt die gemeinsame Datei latex-vorspann.tex mit gesetztem Schalter.

\newif\ifkorrekturansicht
\korrekturansichttrue

\input{../tex-inputs/latex-vorspann}


\section[Julius Rodenberg an Arthur Schnitzler, 13. 12. 1897]{L00749 Julius Rodenberg an Arthur Schnitzler, 13. 12. 1897}
\nopagebreak\mylabel{L00749v}
\rehead{ }\normalsize\beginnumbering\briefempfaengerindex{Schnitzler, Arthur@\textsc{Schnitzler, Arthur}!zzzRodenberg, Julius@\emph{von Julius Rodenberg}!1897-12-131@{13. 12. 1897}|(be}
\toendnotes[C]{\smallbreak\pagebreak[2]}\Standort{CUL, Schnitzler, B 85.}
\physDesc{Brief, 1 Blatt, 2 Seiten, 994 Zeichen
\newline{}Handschrift: schwarze Tinte, deutsche Kurrent
\newline{}Schnitzler: 1) mit rotem Buntstift vereinzelte Unterstreichungen  2) mit Bleistift beschriftet: »\textsc{Rodenberg}«}
\pstart
           \centering{}{\pb}\textcolor{gray}{\textbf{Deutsche Rundschau\orgindex{Deutsche Rundschau@Deutsche Rundschau|pw}}}\pend
           
\pstart
           \textcolor{gray}{\textbf{Expedition u. Redaction:}}\hfill \textcolor{gray}{\textbf{Herausgeber:}}\pend
           
\pstart
           \textcolor{gray}{\textbf{Gebrüder Paetel\orgindex{Gebrueder Paetel Verlag@Gebrüder Paetel Verlag|pw} in Berlin\oindex{Berlin@\textbf{Berlin}, \emph{P.PPLC}|pw}}}\hfill \textcolor{gray}{\textbf{Julius Rodenberg in Berlin\oindex{Berlin@\textbf{Berlin}, \emph{P.PPLC}|pw}}}\pend
           
\pstart
           \textcolor{gray}{\textbf{(Elwin Paetel\pwindex{Paetel, Elwin 13.11.1847 – 04.10.1907@\textsc{Paetel, Elwin} (13.11.1847 – 04.10.1907), \emph{Verleger/Verlegerin}|pw})}}\hfill \textcolor{gray}{\textbf{W., Margarethenstr. 1\oindex{Margaretenstrasse [Berlin]@\textbf{Margaretenstraße [Berlin]}, \emph{Straße (K.STR)}|pw}.}}\pend
           
\pstart
           \textcolor{gray}{\textbf{W., Lützowstr. 7\oindex{Luetzowstrasse@\textbf{Lützowstraße}, \emph{Straße (K.STR)}|pw}.}}\pend
           
\pstart
           \raggedleft{}\textbf{\textcolor{gray}{\textbf{Berlin W.\oindex{Berlin@\textbf{Berlin}, \emph{P.PPLC}|pw},}} den}{ }13. Dec. \textcolor{gray}{\textbf{189}}7.\pend
           
\pstart{}Hochgeehrter Herr Doctor!\pend\vspace{0.5em}
\pstart
           Durch meinen Schwager Dr. Ed. Schiff\pwindex{Schiff, Eduard Liberius 04.03.1849 – 05.03.1913@\textsc{Schiff, Eduard Liberius} (04.03.1849 – 05.03.1913), \emph{Dermatologe/Dermatologin}|pw} iſt mir
               die höchſt erfreuliche Kunde geworden, daß die »\textsc{Rundschau}\orgindex{Deutsche Rundschau@Deutsche Rundschau|pw}« ſich Hoffnung machen darf, in nicht allzuferner Zeit einen novelliſtiſchen
               Beitrag von Ihnen zu erhalten. Längſt ſchon iſt dieß mein Wunſch geweſen u. wenn ich
               ihn nicht eher ausſprach, ſo werden Sie ſich das daraus erklären können, daß ich mich
               nicht gern einem Refus ausgeſetzt haben würde. Nun iſt aber bei Ihnen freundliches
                  Entgegenko{\geminationm}en gefunden, will ich nicht zögern, Ihnen
               dafür zu danken u. meine Bitte direct zu wiederholen. Daß Sie dieſer im Augenblick
               nicht zu willfahren vermöchten, hab’ ich vorausgeſetzt, u. darauf ko{\geminationm}t es mir auch nicht an; es genügt mir, zu wißen, daß
               Sie bei nächſter Gelegenheit unſerer Zeitſchrift gedenken wollen, u. {\pb}ich bitte nur, mich eintretenden Falls zu
               benachrichtigen, um Sie nicht unnöthig lang mit dem Abdruck warten laßen zu
               müßen.\pend
           
\pstart
           Mit dem Ausdruck beſonderer Hochachtung{\\[\baselineskip]}Ihr ergebener{\\[\baselineskip]}\spacefill\mbox{Dr Julius Rodenberg.}\pend
           \leftskip=0em{}\selectlanguage{ngerman}\endnumbering\briefempfaengerindex{Schnitzler, Arthur@\textsc{Schnitzler, Arthur}!zzzRodenberg, Julius@\emph{von Julius Rodenberg}!1897-12-131@{13. 12. 1897}|)be}\mylabel{L00749h}  \normalsize

\doendnotes{C}
\bigskip
\vfill

\clearpage

\footnotesize

\lohead{\textsc{register}}

% Definiere theindex-Environment komplett neu ohne reledmac
\makeatletter
\renewenvironment{theindex}{%
  \section*{\indexname}%
  \setlength{\parindent}{0pt}%
  \setlength{\parskip}{0pt plus 0.3pt}%
  \let\item\@idxitem
}{%
  \clearpage
}
\makeatother

\IfFileExists{\jobname-pw.ind}{\input{\jobname-pw.ind}}{}

\end{document}

      