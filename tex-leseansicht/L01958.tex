%% latex-leseansicht-vorspann.tex
%% Vorspann für die Leseansicht.
%% Lädt die gemeinsame Datei latex-vorspann.tex mit nicht gesetztem Schalter.

\newif\ifkorrekturansicht
\korrekturansichtfalse

\input{../tex-inputs/latex-vorspann}


         
         \newcommand{\erwaehntePersonen}{Personen: Anna Bahr-Mildenburg, Lisa Clarus, Olga Schnitzler}
         \newcommand{\erwaehnteInstitutionen}{Institutionen: S. Fischer Verlag}
         \newcommand{\erwaehnteOrte}{Orte: De Keysers Royal Hotel, Deutschland, London, Ober Sankt Veit, Wien}
         \newcommand{\erwaehnteWerke}{Werke: Börsenblatt für den deutschen Buchhandel, Der junge Medardus. Dramatische Historie in einem Vorspiel und fünf Aufzügen, Neues Wiener Journal}
               \section[Hermann Bahr an Arthur Schnitzler, 26. 9. 1910]{ Hermann Bahr an Arthur Schnitzler, 26. 9. 1910}\nopagebreak\mylabel{v}\rehead{ }\begin{ledgroupsized}[t]{13cm}\normalsize\beginnumbering \toendnotes[C]{\smallbreak\pagebreak[2]} \Standort{CUL, Schnitzler, B 5b.}
\physDesc{Brief, 1 Blatt, 3 Seiten
\newline{}Handschrift Lisa Clarus: schwarze Tinte, lateinische Kurrent\newline{}Handschrift Hermann Bahr: schwarze Tinte, deutsche Kurrent (\noindent{}Unterschrift, Nachschrift)
\newline{}Schnitzler: mit Bleistift ergänzt »Bahr« \newline{}Ordnung: mit Bleistift von unbekannter Hand nummeriert:
                                    »166« }\buchAbdrucke{\weitereDrucke{Hermann Bahr, Arthur Schnitzler: \emph{Briefwechsel, Aufzeichnungen, Dokumente (1891–1931)}. Hg. Kurt Ifkovits und Martin Anton Müller. Göttingen: \emph{Wallstein} 2018, S. 437–438.} }\toendnotes[C]{\smallbreak}\pstart
           \raggedleft{}{\pb}Wien XIII/\textsubscript{7}\oindex{Ober Sankt Veit@\textbf{Ober Sankt Veit}|pw}\hspace*{1.5em}26. 9. 10.\pend
           \pstart\center{}Lieber Arthur!\pend\pstart
           Ich fahre \label{K_L01958_1v}\edtext{Samstag für vier Wochen nach
                  London\oindex{London@\textbf{London}|pw}}{\lemma{\textnormal{\emph{Samstag … London}}}\Cendnote{\textnormal{Am 1. 10. 1910 begleitete
                     Bahr\pwindex{Bahr, Hermann 19.07.1863 – 15.01.1934@\textsc{Bahr, Hermann} (19.07.1863 – 15.01.1934), \emph{Schriftsteller, Kritiker}|pwk} seine Frau\pwindex{Bahr-Mildenburg, Anna 29.11.1872 – 27.01.1947@\textsc{Bahr-Mildenburg, Anna} (29.11.1872 – 27.01.1947), \emph{Sängerin}|pwkv} nach London\oindex{London@\textbf{London}|pwk}, am
                     28. 10. 1910 waren sie wieder in Wien\oindex{Wien@\textbf{Wien}|pwk}.}}}\label{K_L01958_1h} und so werde ich leider \label{LL327-1v}bei Deiner Première\pwindex{Schnitzler, Arthur 15.05.1862 – 21.10.1931@\textsc{Schnitzler, Arthur} (15.05.1862 – 21.10.1931), \emph{Schriftsteller, Mediziner}!junge Medardus. Dramatische Historie in einem Vorspiel und fuenf Aufzuegen1910-10-26@\strich\emph{Der junge Medardus. Dramatische Historie in einem Vorspiel und fünf Aufzügen} {[}1910-10-26{]}|pwv} nicht in
                     Wien\oindex{Wien@\textbf{Wien}|pw}\label{LL327-1h}{ }ſein. Es wäre mir aber lieb, wenn Du mir (entweder
               noch vor Samstag hieher oder dann nach London E. C.
                  Victoria Embankment, De Keysers Hotel\oindex{De Keysers Royal Hotel@\textbf{De Keysers Royal Hotel}|pw}) ein \label{K_L01958_2v}\edtext{Buch Deines neuen Stückes\pwindex{Schnitzler, Arthur 15.05.1862 – 21.10.1931@\textsc{Schnitzler, Arthur} (15.05.1862 – 21.10.1931), \emph{Schriftsteller, Mediziner}!junge Medardus. Dramatische Historie in einem Vorspiel und fuenf Aufzuegen1910-10-26@\strich\emph{Der junge Medardus. Dramatische Historie in einem Vorspiel und fünf Aufzügen} {[}1910-10-26{]}|pwv}}{\lemma{\textnormal{\emph{Buch … Stückes}}}\Cendnote{\textnormal{Arthur Schnitzler\pwindex{Schnitzler, Arthur 15.05.1862 – 21.10.1931@\textsc{Schnitzler, Arthur} (15.05.1862 – 21.10.1931), \emph{Schriftsteller, Mediziner}|pwk}: \emph{Der junge Medardus. Dramatische Historie in einem Vorspiel und
                        fünf Aufzügen}\pwindex{Schnitzler, Arthur 15.05.1862 – 21.10.1931@\textsc{Schnitzler, Arthur} (15.05.1862 – 21.10.1931), \emph{Schriftsteller, Mediziner}!junge Medardus. Dramatische Historie in einem Vorspiel und fuenf Aufzuegen1910-10-26@\strich\emph{Der junge Medardus. Dramatische Historie in einem Vorspiel und fünf Aufzügen} {[}1910-10-26{]}|pwk}. Berlin: \emph{S. Fischer}\orgindex{S. Fischer Verlag@S. Fischer Verlag|pwk}{ }1910 (am 26. 10. 1910 vom \emph{Börsenblatt für den deutschen Buchhandel}\pwindex{?? Werk@Nicht ermittelte Verfasserinnen und Verfasser!Boersenblatt fuer den deutschen Buchhandel1843-01-03@\emph{Börsenblatt für den deutschen Buchhandel} {[}1843-01-03{]}|pwk} als
                     Neuerscheinung gemeldet).}}}\label{K_L01958_2h}{ }ſchicken könntest. Macht es mir einen starken
               Eindruck und habe ich darüber wirklich etwas zu sagen, so würde ich das für das Wiener Journal\pwindex{Neues Wiener Journal1893 – 1939@\emph{Neues Wiener Journal} {[}1893 – 1939{]}|pw} von London\oindex{London@\textbf{London}|pw} aus tun und veranlassen, dass von irgend einem der Herren {\pb}der Redaktion eine kurze Notiz über die Aufführung
               und Aufnahme angehängt werde. Kann ich aber in kein rechtes inneres Verhältnis dazu
               kommen, woran ja ebenso der Autor wie der Kritiker schuld sein kann, so ist es uns
               beiden besser, wenn ich die Gelegenheit zu schweigen ausnütze, statt mich um das
               Stück herum zu reden, was mir, je älter ich werde, immer unleidlicher wird.\pend
           \pstart
           Im November komme ich nur auf ein paar Tage zurück, weil ich gleich
               wieder an den Rhein, auf eine \label{K_L01958_3v}\edtext{Vorlesungstournée}{\lemma{\textnormal{\emph{Vorlesungstournée}}}\Cendnote{\textnormal{Vom 17. 11. bis 3. 12. 1910 war Bahr\pwindex{Bahr, Hermann 19.07.1863 – 15.01.1934@\textsc{Bahr, Hermann} (19.07.1863 – 15.01.1934), \emph{Schriftsteller, Kritiker}|pwk} auf einer umfangreichen Tournee durch Deutschland\oindex{Deutschland@\textbf{Deutschland}|pwk}.}}}\label{K_L01958_3h} muss. Aber im
                  Dezember wird es uns dann doch einmal vergönnt sein, in Ruhe zu Euch\pwindex{Schnitzler, Olga 17.01.1882 – 13.01.1970@\textsc{Schnitzler, Olga} (17.01.1882 – 13.01.1970), \emph{Schauspielerin, Sängerin}|pwv} zu kommen.\pend
           \pstart
           {\pb}Mit herzlichen Grüssen von uns\pwindex{Bahr-Mildenburg, Anna 29.11.1872 – 27.01.1947@\textsc{Bahr-Mildenburg, Anna} (29.11.1872 – 27.01.1947), \emph{Sängerin}|pwv} beiden, auch an Deine liebe Frau\pwindex{Schnitzler, Olga 17.01.1882 – 13.01.1970@\textsc{Schnitzler, Olga} (17.01.1882 – 13.01.1970), \emph{Schauspielerin, Sängerin}|pwv},{\\[\baselineskip]}Dein alter{\\[\baselineskip]}\spacefill\mbox{{[}hs. Bahr:{]} Hermann}\pend
           \leftskip=0em{}\pstart
           \noindent{}viel zu nervös, um ſelbſt ſchreiben zu können.\pend
           
         
         \endnumbering\mylabel{h}\end{ledgroupsized}  \newcommand{\dateiname}{L01958}\newcommand{\titel}{Hermann Bahr an Arthur Schnitzler, 26. 9. 1910}\newcommand{\editorInnen}{ Kurt Ifkovits,  Martin Anton Müller}%% latex-leseansicht-abspann.tex
%% Abspann für die Leseansicht.
%% Der Schalter \ifkorrekturansicht ist bereits durch den Vorspann gesetzt.

%% latex-abspann.tex
%% Gemeinsamer Abspann für Korrekturansicht und Leseansicht.
%% Setzt den Schalter \ifkorrekturansicht voraus (gesetzt in den
%% einbindenden Dateien latex-korrekturansicht-abspann.tex bzw.
%% latex-leseansicht-abspann.tex).
%% ---------------------------------------------------------------

\normalsize

% Das esempio-Environment wird nur in der Leseansicht benötigt
\ifkorrekturansicht\else
\newenvironment{esempio}[3]%
{
    \vspace{1.5ex}
    \rlap{\underline{#1}}
    \par
    \setlength{\parindent}{0cm}
    \nopagebreak
    \leftskip=#2cm
    \rightskip=#3cm
}
{
    \par
}
\fi

\doendnotes{C}
\bigskip
\vfill

\clearpage

\footnotesize

\ifkorrekturansicht
  \lohead{\textsc{register}}
\fi

% theindex-Environment neu definieren ohne reledmac
\makeatletter
\renewenvironment{theindex}{%
  \ifkorrekturansicht
    \section*{\indexname}%
  \else
    \subsubsection*{Index der erwähnten Entitäten}%
  \fi
  \setlength{\parindent}{0pt}%
  \setlength{\parskip}{0pt plus 0.3pt}%
  \let\item\@idxitem
}{%
  \ifkorrekturansicht\clearpage\fi
}
\makeatother

\IfFileExists{\jobname-pw.ind}{\input{\jobname-pw.ind}}{}

% Quellenangabe nur in der Leseansicht
\ifkorrekturansicht\else
% Fallback-Definitionen, falls die .tex-Datei \titel etc. nicht gesetzt hat
\providecommand{\titel}{}
\providecommand{\editorInnen}{}
\providecommand{\dateiname}{\jobname}

\vspace{3cm}

\vfill

\footnotesize
\textsc{Quelle}: \titel. Herausgegeben von {\editorInnen}. In: \emph{Arthur Schnitzler: Briefwechsel mit Autorinnen und Autoren}.
 Digitale Edition, https://schnitzler-briefe.acdh.oeaw.ac.at/{\dateiname}.html (Stand \today)
\fi

\end{document}


      