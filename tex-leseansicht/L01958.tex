%% latex-leseansicht-vorspann.tex
%% Vorspann für die Leseansicht.
%% Lädt die gemeinsame Datei latex-vorspann.tex mit nicht gesetztem Schalter.

\newif\ifkorrekturansicht
\korrekturansichtfalse

\input{../tex-inputs/latex-vorspann}


\section[Hermann Bahr an Arthur Schnitzler, 26. 9. 1910]{L01958 Hermann Bahr an Arthur Schnitzler, 26. 9. 1910}
\nopagebreak\mylabel{L01958v}
\rehead{ }\normalsize\beginnumbering\briefempfaengerindex{Schnitzler, Arthur@\textsc{Schnitzler, Arthur}!zzzBahr, Hermann@\emph{von Hermann Bahr}!1910-09-261@{26. 9. 1910}|(be}
\toendnotes[C]{\smallbreak\pagebreak[2]}
\correspDesc{Versand  durch Hermann Bahr am 26. 9. 1910 in Wien
\newline{}Erhalt  durch Arthur Schnitzler im Zeitraum [26. 9. 1910
                  – 30. 9. 1910?] in Wien}\toendnotes[C]{\smallbreak}
\Standort{CUL, Schnitzler, B 5b.}
\physDesc{Brief, 1 Blatt, 3 Seiten, 1205 Zeichen
\newline{}Handschrift Lisa Clarus: schwarze Tinte, lateinische Kurrent
\newline{}Handschrift Hermann Bahr: schwarze Tinte, deutsche Kurrent (\noindent{}Unterschrift, Nachschrift)
\newline{}Schnitzler: mit Bleistift ergänzt »Bahr« 
\newline{}Ordnung: mit Bleistift von unbekannter Hand nummeriert:
                                    »166« }
\buchAbdrucke{\weitereDrucke{Hermann Bahr, Arthur Schnitzler: \emph{Briefwechsel, Aufzeichnungen, Dokumente (1891–1931)}. Herausgegeben von Kurt Ifkovits und Martin Anton Müller. Göttingen: \emph{Wallstein} 2018, S. 437–438.} }\toendnotes[C]{\smallbreak}
\pstart
           \raggedleft{}{\pb}Wien XIII/\textsubscript{7}\oindex{Wien@\textbf{Wien}!XIII., Hietzing@\textbf{XIII., Hietzing}!Ober Sankt Veit@\textbf{Ober Sankt Veit}, \emph{Ehemaliger Ort}|pw}\hspace*{1.5em}26. 9. 10.\pend
           
\pstart\center{}Lieber Arthur!\pend\vspace{0.5em}
\pstart
           Ich fahre \label{K_L01958-1v}\edtext{Samstag für vier Wochen nach
                  London\oindex{London@\textbf{London}, \emph{Hauptstadt}|pw}}{\lemma{\textnormal{\emph{Samstag … London}}}\Cendnote{\textnormal{Am 1. 10. 1910 begleitete
                     Bahr\pwindex{Bahr, Hermann 19.\,7.\,1863 Linz – 15.\,1.\,1934 München@\textsc{Bahr, Hermann} (19.\,7.\,1863 Linz – 15.\,1.\,1934 München), \emph{Schriftsteller, Kritiker}|pwk} seine Frau\pwindex{Bahr-Mildenburg, Anna 29.\,11.\,1872 Wien – 27.\,1.\,1947 ebd.@\textsc{Bahr-Mildenburg, Anna} (29.\,11.\,1872 Wien – 27.\,1.\,1947 ebd.), \emph{Sängerin}|pwkv} nach London\oindex{London@\textbf{London}, \emph{Hauptstadt}|pwk}, am 28. 10. 1910 waren sie wieder in Wien\oindex{Wien@\textbf{Wien}, \emph{Verwaltungsgebiet}|pwk}.}}}\label{K_L01958-1} und so werde ich leider \label{LL327-1v}bei Deiner Première\pwindex{Schnitzler, Arthur 15.\,5.\,1862 Wien – 21.\,10.\,1931 ebd.@\textsc{Schnitzler, Arthur} (15.\,5.\,1862 Wien – 21.\,10.\,1931 ebd.), \emph{Schriftsteller, Mediziner}!junge Medardus. Dramatische Historie in einem Vorspiel und fünf Aufzügen@\strich\emph{Der junge Medardus. Dramatische Historie in einem Vorspiel und fünf Aufzügen}|pwv} nicht in Wien\oindex{Wien@\textbf{Wien}, \emph{Verwaltungsgebiet}|pw}\label{LL327-1h}{ }ſein. Es wäre mir aber lieb, wenn Du mir (entweder
               noch vor Samstag hieher oder dann nach London E. C.
                  Victoria Embankment, De Keysers Hotel\oindex{De Keysers Royal Hotel@\textbf{De Keysers Royal Hotel}, \emph{Hotel}|pw}) ein \label{K_L01958-2v}\edtext{Buch Deines neuen Stückes\pwindex{Schnitzler, Arthur 15.\,5.\,1862 Wien – 21.\,10.\,1931 ebd.@\textsc{Schnitzler, Arthur} (15.\,5.\,1862 Wien – 21.\,10.\,1931 ebd.), \emph{Schriftsteller, Mediziner}!junge Medardus. Dramatische Historie in einem Vorspiel und fünf Aufzügen@\strich\emph{Der junge Medardus. Dramatische Historie in einem Vorspiel und fünf Aufzügen}|pwv}}{\lemma{\textnormal{\emph{Buch … Stückes}}}\Cendnote{\textnormal{Arthur Schnitzler: \emph{Der junge Medardus. Dramatische Historie in einem Vorspiel und
                        fünf Aufzügen}\pwindex{Schnitzler, Arthur 15.\,5.\,1862 Wien – 21.\,10.\,1931 ebd.@\textsc{Schnitzler, Arthur} (15.\,5.\,1862 Wien – 21.\,10.\,1931 ebd.), \emph{Schriftsteller, Mediziner}!junge Medardus. Dramatische Historie in einem Vorspiel und fünf Aufzügen@\strich\emph{Der junge Medardus. Dramatische Historie in einem Vorspiel und fünf Aufzügen}|pwk}. Berlin: \emph{S. Fischer}\orgindex{S. Fischer Verlag@S. Fischer Verlag|pwk}{ }1910. Am 26. 10. 1910 wurde das Buch vom \emph{Börsenblatt für den deutschen Buchhandel}\pwindex{Börsenblatt für den Deutschen Buchhandel@\emph{Börsenblatt für den Deutschen Buchhandel}|pwk} als
                     Neuerscheinung gemeldet.}}}\label{K_L01958-2}{ }ſchicken könntest. Macht es mir einen starken
               Eindruck und habe ich darüber wirklich etwas zu sagen, so würde ich das für das Wiener Journal\pwindex{Neues Wiener Journal@\emph{Neues Wiener Journal}|pw} von London\oindex{London@\textbf{London}, \emph{Hauptstadt}|pw} aus tun und veranlassen, dass von irgend einem der Herren {\pb}der Redaktion eine kurze Notiz über die Aufführung
               und Aufnahme angehängt werde. Kann ich aber in kein rechtes inneres Verhältnis dazu
               kommen, woran ja ebenso der Autor wie der Kritiker schuld sein kann, so ist es uns
               beiden besser, wenn ich die Gelegenheit zu schweigen ausnütze, statt mich um das
               Stück herum zu reden, was mir, je älter ich werde, immer unleidlicher wird.\pend
           
\pstart
           Im November komme ich nur auf ein paar Tage zurück, weil ich gleich
               wieder an den Rhein\oindex{Rhein@\textbf{Rhein}, \emph{Fluss}|pw}, auf eine \label{K_L01958-3v}\edtext{Vorlesungstournée}{\lemma{\textnormal{\emph{Vorlesungstournée}}}\Cendnote{\textnormal{Vom 17. 11. bis 3. 12. 1910 war Bahr\pwindex{Bahr, Hermann 19.\,7.\,1863 Linz – 15.\,1.\,1934 München@\textsc{Bahr, Hermann} (19.\,7.\,1863 Linz – 15.\,1.\,1934 München), \emph{Schriftsteller, Kritiker}|pwk} auf einer umfangreichen Tournee durch
                     Deutschland\oindex{Deutschland@\textbf{Deutschland}|pwk}.}}}\label{K_L01958-3} muss. Aber im
                  Dezember wird es uns dann doch einmal vergönnt sein, in Ruhe zu Euch\pwindex{Schnitzler, Olga 17.\,1.\,1882 Wien – 13.\,1.\,1970 Lugano@\textsc{Schnitzler, Olga} (17.\,1.\,1882 Wien – 13.\,1.\,1970 Lugano), \emph{Schauspielerin, Sängerin}|pwv} zu kommen.\pend
           
\pstart
           {\pb}Mit herzlichen Grüssen von uns\pwindex{Bahr-Mildenburg, Anna 29.\,11.\,1872 Wien – 27.\,1.\,1947 ebd.@\textsc{Bahr-Mildenburg, Anna} (29.\,11.\,1872 Wien – 27.\,1.\,1947 ebd.), \emph{Sängerin}|pwv} beiden, auch an Deine liebe Frau\pwindex{Schnitzler, Olga 17.\,1.\,1882 Wien – 13.\,1.\,1970 Lugano@\textsc{Schnitzler, Olga} (17.\,1.\,1882 Wien – 13.\,1.\,1970 Lugano), \emph{Schauspielerin, Sängerin}|pwv},{\\[\baselineskip]}Dein alter{\\[\baselineskip]}\spacefill\mbox{{[}hs. Bahr:{]} Hermann}\pend
           \leftskip=0em{}
\pstart
           \noindent{}viel zu nervös, um{ }ſelbſt{ }ſchreiben zu können.\pend
           \selectlanguage{ngerman}\endnumbering\briefempfaengerindex{Schnitzler, Arthur@\textsc{Schnitzler, Arthur}!zzzBahr, Hermann@\emph{von Hermann Bahr}!1910-09-261@{26. 9. 1910}|)be}\mylabel{L01958h}  \newcommand{\dateiname}{L01958}\newcommand{\titel}{Hermann Bahr an Arthur Schnitzler, 26. 9. 1910}\newcommand{\editorInnen}{Herausgegeben von Martin Anton Müller}%% latex-leseansicht-abspann.tex
%% Abspann für die Leseansicht.
%% Der Schalter \ifkorrekturansicht ist bereits durch den Vorspann gesetzt.

%% latex-abspann.tex
%% Gemeinsamer Abspann für Korrekturansicht und Leseansicht.
%% Setzt den Schalter \ifkorrekturansicht voraus (gesetzt in den
%% einbindenden Dateien latex-korrekturansicht-abspann.tex bzw.
%% latex-leseansicht-abspann.tex).
%% ---------------------------------------------------------------

\normalsize

% Das esempio-Environment wird nur in der Leseansicht benötigt
\ifkorrekturansicht\else
\newenvironment{esempio}[3]%
{
    \vspace{1.5ex}
    \rlap{\underline{#1}}
    \par
    \setlength{\parindent}{0cm}
    \nopagebreak
    \leftskip=#2cm
    \rightskip=#3cm
}
{
    \par
}
\fi

\doendnotes{C}
\bigskip
\vfill

\clearpage

\footnotesize

\ifkorrekturansicht
  \lohead{\textsc{register}}
\fi

% theindex-Environment neu definieren ohne reledmac
\makeatletter
\renewenvironment{theindex}{%
  \ifkorrekturansicht
    \section*{\indexname}%
  \else
    \subsubsection*{Index der erwähnten Entitäten}%
  \fi
  \setlength{\parindent}{0pt}%
  \setlength{\parskip}{0pt plus 0.3pt}%
  \let\item\@idxitem
}{%
  \ifkorrekturansicht\clearpage\fi
}
\makeatother

\IfFileExists{\jobname-pw.ind}{\input{\jobname-pw.ind}}{}

% Quellenangabe nur in der Leseansicht
\ifkorrekturansicht\else
% Fallback-Definitionen, falls die .tex-Datei \titel etc. nicht gesetzt hat
\providecommand{\titel}{}
\providecommand{\editorInnen}{}
\providecommand{\dateiname}{\jobname}

\vspace{3cm}

\vfill

\footnotesize
\textsc{Quelle}: \titel. Herausgegeben von {\editorInnen}. In: \emph{Arthur Schnitzler: Briefwechsel mit Autorinnen und Autoren}.
 Digitale Edition, https://schnitzler-briefe.acdh.oeaw.ac.at/{\dateiname}.html (Stand \today)
\fi

\end{document}


