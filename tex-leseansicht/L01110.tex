%% latex-leseansicht-vorspann.tex
%% Vorspann für die Leseansicht.
%% Lädt die gemeinsame Datei latex-vorspann.tex mit nicht gesetztem Schalter.

\newif\ifkorrekturansicht
\korrekturansichtfalse

\input{../tex-inputs/latex-vorspann}


\section[Arthur Schnitzler an Hermann Bahr, 19. 4. 1901]{L01110 Arthur Schnitzler an Hermann Bahr, 19. 4. 1901}
\nopagebreak\mylabel{L01110v}
\rehead{ }\normalsize\beginnumbering\briefempfaengerindex{Bahr, Hermann@\textsc{Bahr, Hermann}!zzzSchnitzler, Arthur@\emph{von Arthur Schnitzler}!1901-04-191@{19. 4. 1901}|(be}
\toendnotes[C]{\smallbreak\pagebreak[2]}
\correspDesc{Versand  durch Arthur Schnitzler am 19. 4. 1901 in Wien
\newline{}Erhalt  durch Hermann Bahr im Zeitraum [19. 4. 1901
                  – 23. 4. 1901?] in Wien}\toendnotes[C]{\smallbreak}
\Standort{TMW, HS AM 23342 Ba.}
\physDesc{Brief, 1 Blatt, 2 Seiten, 406 Zeichen
\newline{}Handschrift: schwarze Tinte, deutsche Kurrent
\newline{}Ordnung: 1) Lochung  2) mit Bleistift von unbekannter Hand datiert: »19. 4. 01«}
\buchAbdrucke{\weitereDrucke{1) \emph{19. 4. 1901.} In: Arthur Schnitzler: \emph{The Letters of Arthur Schnitzler to Hermann Bahr}. Edited, annotated, and with an introduction, by Donald G. Daviau. Chapel Hill: \emph{The University of North Carolina Press} 1978, S. 68 (University of North Carolina studies in the Germanic languages
                        and literatures, 89).} \weitereDrucke{2) Hermann Bahr, Arthur Schnitzler: \emph{Briefwechsel, Aufzeichnungen, Dokumente (1891–1931)}. Herausgegeben von Kurt Ifkovits und Martin Anton Müller. Göttingen: \emph{Wallstein} 2018, S. 202.} }\toendnotes[C]{\smallbreak}
\pstart{}{\pb}lieber
                  Hermann,\pend\vspace{0.5em}
\pstart
           die Vorſtellung der Schauſpielſchule\oindex{Wien@\textbf{Wien}!I., Innere Stadt@\textbf{I., Innere Stadt}!Konservatorium der Gesellschaft der Musikfreunde@\textbf{Konservatorium der Gesellschaft der Musikfreunde}, \emph{Konservatorium}|pw} von der ich
               dir neulich geſprochen findet So{\geminationn}tag den 28. April{ }ſtatt; u. das Fräulein Guſsmann\pwindex{Schnitzler, Olga 17.\,1.\,1882 Wien – 13.\,1.\,1970 Lugano@\textsc{Schnitzler, Olga} (17.\,1.\,1882 Wien – 13.\,1.\,1970 Lugano), \emph{Schauspielerin, Sängerin}|pw} wird nicht die \label{K_L01110-1v}\edtext{Rebecca\pwindex{Ibsen, Henrik 20.\,3.\,1828 Skien – 23.\,5.\,1906 Oslo@\textsc{Ibsen, Henrik} (20.\,3.\,1828 Skien – 23.\,5.\,1906 Oslo), \emph{Schriftsteller}!Rosmersholm. Schauspiel in vier Akten@\strich\emph{Rosmersholm. Schauspiel in vier Akten}|pwv}}{\lemma{\textnormal{\emph{Rebecca}}}\Cendnote{\textnormal{Figur aus \emph{Rosmersholm}\pwindex{Ibsen, Henrik 20.\,3.\,1828 Skien – 23.\,5.\,1906 Oslo@\textsc{Ibsen, Henrik} (20.\,3.\,1828 Skien – 23.\,5.\,1906 Oslo), \emph{Schriftsteller}!Rosmersholm. Schauspiel in vier Akten@\strich\emph{Rosmersholm. Schauspiel in vier Akten}|pwk} von Ibsen\pwindex{Ibsen, Henrik 20.\,3.\,1828 Skien – 23.\,5.\,1906 Oslo@\textsc{Ibsen, Henrik} (20.\,3.\,1828 Skien – 23.\,5.\,1906 Oslo), \emph{Schriftsteller}|pwk}}}}\label{K_L01110-1}{ }ſondern die \label{K_L01110-2v}\edtext{Maria Magdalena\pwindex{Hebbel, Friedrich 18.\,3.\,1813 Wesselburen – 13.\,12.\,1863 Wien@\textsc{Hebbel, Friedrich} (18.\,3.\,1813 Wesselburen – 13.\,12.\,1863 Wien), \emph{Schriftsteller}!Maria Magdalena. Ein bürgerliches Trauerspiel in drei Akten@\strich\emph{Maria Magdalena. Ein bürgerliches Trauerspiel in drei Akten}|pwv}}{\lemma{\textnormal{\emph{Maria Magdalena}}}\Cendnote{\textnormal{Olga Gussmann\pwindex{Schnitzler, Olga 17.\,1.\,1882 Wien – 13.\,1.\,1970 Lugano@\textsc{Schnitzler, Olga} (17.\,1.\,1882 Wien – 13.\,1.\,1970 Lugano), \emph{Schauspielerin, Sängerin}|pwk} hatte ursprünglich die Rolle
                  der Protagonistin aus Hebbels\pwindex{Hebbel, Friedrich 18.\,3.\,1813 Wesselburen – 13.\,12.\,1863 Wien@\textsc{Hebbel, Friedrich} (18.\,3.\,1813 Wesselburen – 13.\,12.\,1863 Wien), \emph{Schriftsteller}|pwk}{ }\emph{Maria Magdalena}\pwindex{Hebbel, Friedrich 18.\,3.\,1813 Wesselburen – 13.\,12.\,1863 Wien@\textsc{Hebbel, Friedrich} (18.\,3.\,1813 Wesselburen – 13.\,12.\,1863 Wien), \emph{Schriftsteller}!Maria Magdalena. Ein bürgerliches Trauerspiel in drei Akten@\strich\emph{Maria Magdalena. Ein bürgerliches Trauerspiel in drei Akten}|pwk} ausgesucht; zwischenzeitlich
                  wurde ihr dies aber untersagt (vgl. A. S. \emph{Briefe 1875–1912}, S. 402).}}}\label{K_L01110-2}{ }ſpielen, was vielleicht noch intereſſanter{ }ſein
               dürfte. We{\geminationn} du also Zeit {\pb}und Laune haſt, möcht
               ich dich bitten zu ko{\geminationm}en. Den Sitz erhältſt du
               jedenfalls zugeſandt.\pend
           
\pstart
           Herzlich grüßend dein{\\[\baselineskip]}\spacefill\mbox{Arthur Schnitzler}\pend
           \leftskip=0em{}
\pstart
           Wien\oindex{Wien@\textbf{Wien}, \emph{Verwaltungsgebiet}|pw}, 19. 4. 901.\pend
           \selectlanguage{ngerman}\endnumbering\briefempfaengerindex{Bahr, Hermann@\textsc{Bahr, Hermann}!zzzSchnitzler, Arthur@\emph{von Arthur Schnitzler}!1901-04-191@{19. 4. 1901}|)be}\mylabel{L01110h}  \newcommand{\dateiname}{L01110}\newcommand{\titel}{Arthur Schnitzler an Hermann Bahr, 19. 4. 1901}\newcommand{\editorInnen}{Herausgegeben von Martin Anton Müller}%% latex-leseansicht-abspann.tex
%% Abspann für die Leseansicht.
%% Der Schalter \ifkorrekturansicht ist bereits durch den Vorspann gesetzt.

%% latex-abspann.tex
%% Gemeinsamer Abspann für Korrekturansicht und Leseansicht.
%% Setzt den Schalter \ifkorrekturansicht voraus (gesetzt in den
%% einbindenden Dateien latex-korrekturansicht-abspann.tex bzw.
%% latex-leseansicht-abspann.tex).
%% ---------------------------------------------------------------

\normalsize

% Das esempio-Environment wird nur in der Leseansicht benötigt
\ifkorrekturansicht\else
\newenvironment{esempio}[3]%
{
    \vspace{1.5ex}
    \rlap{\underline{#1}}
    \par
    \setlength{\parindent}{0cm}
    \nopagebreak
    \leftskip=#2cm
    \rightskip=#3cm
}
{
    \par
}
\fi

\doendnotes{C}
\bigskip
\vfill

\clearpage

\footnotesize

\ifkorrekturansicht
  \lohead{\textsc{register}}
\fi

% theindex-Environment neu definieren ohne reledmac
\makeatletter
\renewenvironment{theindex}{%
  \ifkorrekturansicht
    \section*{\indexname}%
  \else
    \subsubsection*{Index der erwähnten Entitäten}%
  \fi
  \setlength{\parindent}{0pt}%
  \setlength{\parskip}{0pt plus 0.3pt}%
  \let\item\@idxitem
}{%
  \ifkorrekturansicht\clearpage\fi
}
\makeatother

\IfFileExists{\jobname-pw.ind}{\input{\jobname-pw.ind}}{}

% Quellenangabe nur in der Leseansicht
\ifkorrekturansicht\else
% Fallback-Definitionen, falls die .tex-Datei \titel etc. nicht gesetzt hat
\providecommand{\titel}{}
\providecommand{\editorInnen}{}
\providecommand{\dateiname}{\jobname}

\vspace{3cm}

\vfill

\footnotesize
\textsc{Quelle}: \titel. Herausgegeben von {\editorInnen}. In: \emph{Arthur Schnitzler: Briefwechsel mit Autorinnen und Autoren}.
 Digitale Edition, https://schnitzler-briefe.acdh.oeaw.ac.at/{\dateiname}.html (Stand \today)
\fi

\end{document}


