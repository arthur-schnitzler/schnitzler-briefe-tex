%% latex-korrekturansicht-vorspann.tex
%% Vorspann für die Korrekturansicht.
%% Lädt die gemeinsame Datei latex-vorspann.tex mit gesetztem Schalter.

\newif\ifkorrekturansicht
\korrekturansichttrue

\input{../tex-inputs/latex-vorspann}


\section[Arthur Schnitzler an Hermann Bahr, 19. 4. 1901]{L01110 Arthur Schnitzler an Hermann Bahr, 19. 4. 1901}
\nopagebreak\mylabel{L01110v}
\rehead{ }\normalsize\beginnumbering\briefempfaengerindex{Bahr, Hermann@\textsc{Bahr, Hermann}!zzzSchnitzler, Arthur@\emph{von Arthur Schnitzler}!1901-04-191@{19. 4. 1901}|(be}
\toendnotes[C]{\smallbreak\pagebreak[2]}\Standort{TMW, HS AM 23342 Ba.}
\physDesc{Brief, 1 Blatt, 2 Seiten, 406 Zeichen
\newline{}Handschrift: schwarze Tinte, deutsche Kurrent
\newline{}Ordnung: 1) Lochung  2) mit Bleistift von unbekannter Hand datiert: »19. 4. 01«}
\buchAbdrucke{\weitereDrucke{1) Arthur Schnitzler: \emph{The Letters of Arthur Schnitzler to Hermann Bahr}. Chapel Hill: \emph{The University of North Carolina Press} 1978, S. 68.} \weitereDrucke{2) Hermann Bahr, Arthur Schnitzler: \emph{Briefwechsel, Aufzeichnungen, Dokumente (1891–1931)}. Göttingen: \emph{Wallstein} 2018, S. 202.} }\toendnotes[C]{\smallbreak}
\pstart{}{\pb}lieber
                  Hermann,\pend\vspace{0.5em}
\pstart
           die Vorſtellung der Schauſpielſchule\oindex{Konservatorium der Gesellschaft der Musikfreunde@\textbf{Konservatorium der Gesellschaft der Musikfreunde}, \emph{Konservatorium (K.KON)}|pw} von der ich
               dir neulich geſprochen findet So{\geminationn}tag den 28.
                  April{ }ſtatt; u. das Fräulein Guſsmann\pwindex{Schnitzler, Olga 17.01.1882 – 13.01.1970@\textsc{Schnitzler, Olga} (17.01.1882 – 13.01.1970), \emph{Schauspieler/Schauspielerin, Sänger/Sängerin}|pw} wird nicht die \label{K_L01110-1v}\edtext{Rebecca\pwindex{Rosmersholm. Schauspiel in vier Akten@\emph{Rosmersholm. Schauspiel in vier Akten}|pwv}}{\lemma{\textnormal{\emph{Rebecca}}}\Cendnote{\textnormal{Figur aus \emph{Rosmersholm}\pwindex{Rosmersholm. Schauspiel in vier Akten@\emph{Rosmersholm. Schauspiel in vier Akten}|pwk} von Ibsen\pwindex{Ibsen, Henrik 20.03.1828 – 23.05.1906@\textsc{Ibsen, Henrik} (20.03.1828 – 23.05.1906), \emph{Schriftsteller/Schriftstellerin}|pwk}}}}\label{K_L01110-1}{ }ſondern die \label{K_L01110-2v}\edtext{Maria Magdalena\pwindex{Maria Magdalena. Ein buergerliches Trauerspiel in drei Akten@\emph{Maria Magdalena. Ein bürgerliches Trauerspiel in drei Akten}|pwv}}{\lemma{\textnormal{\emph{Maria Magdalena}}}\Cendnote{\textnormal{Olga Gussmann\pwindex{Schnitzler, Olga 17.01.1882 – 13.01.1970@\textsc{Schnitzler, Olga} (17.01.1882 – 13.01.1970), \emph{Schauspieler/Schauspielerin, Sänger/Sängerin}|pwk} hatte ursprünglich die Rolle
                  der Protagonistin aus Hebbels\pwindex{Hebbel, Friedrich 18.03.1813 – 13.12.1863@\textsc{Hebbel, Friedrich} (18.03.1813 – 13.12.1863), \emph{Schriftsteller/Schriftstellerin}|pwk}{ }\emph{Maria Magdalena}\pwindex{Maria Magdalena. Ein buergerliches Trauerspiel in drei Akten@\emph{Maria Magdalena. Ein bürgerliches Trauerspiel in drei Akten}|pwk} ausgesucht; zwischenzeitlich
                  wurde ihr dies aber untersagt (vgl. A. S. \emph{Briefe 1875–1912}, S. 402).}}}\label{K_L01110-2}{ }ſpielen, was vielleicht noch intereſſanter ſein
               dürfte. We{\geminationn} du also Zeit {\pb}und Laune haſt, möcht
               ich dich bitten zu ko{\geminationm}en. Den Sitz erhältſt du
               jedenfalls zugeſandt.\pend
           
\pstart
           Herzlich grüßend dein{\\[\baselineskip]}\spacefill\mbox{Arthur Schnitzler}\pend
           \leftskip=0em{}
\pstart
           Wien\oindex{Wien@\textbf{Wien}, \emph{A.ADM2}|pw}, 19. 4. 901.\pend
           \selectlanguage{ngerman}\endnumbering\briefempfaengerindex{Bahr, Hermann@\textsc{Bahr, Hermann}!zzzSchnitzler, Arthur@\emph{von Arthur Schnitzler}!1901-04-191@{19. 4. 1901}|)be}\mylabel{L01110h}  \normalsize

\doendnotes{C}
\bigskip
\vfill

\clearpage

\footnotesize

\lohead{\textsc{register}}

% Definiere theindex-Environment komplett neu ohne reledmac
\makeatletter
\renewenvironment{theindex}{%
  \section*{\indexname}%
  \setlength{\parindent}{0pt}%
  \setlength{\parskip}{0pt plus 0.3pt}%
  \let\item\@idxitem
}{%
  \clearpage
}
\makeatother

\IfFileExists{\jobname-pw.ind}{\input{\jobname-pw.ind}}{}

\end{document}

      