%% latex-leseansicht-vorspann.tex
%% Vorspann für die Leseansicht.
%% Lädt die gemeinsame Datei latex-vorspann.tex mit nicht gesetztem Schalter.

\newif\ifkorrekturansicht
\korrekturansichtfalse

\input{../tex-inputs/latex-vorspann}


         
         \renewcommand{\erwaehntePersonen}{Personen: Georg Brandes, Georges Clemenceau, Édouard Dujardin, Gustave Flaubert, Arno Holz, Jakob Wassermann}
         \renewcommand{\erwaehnteInstitutionen}{Institutionen: Wilhelm Issleib Verlagsbuchhandlung und Buchdruckerei}
         \renewcommand{\erwaehnteOrte}{Orte: Karlsbad, Kopenhagen, Salzburg, Wien}
         \renewcommand{\erwaehnteWerke}{Werke: Die Geschichte der jungen Renate Fuchs, Die Kunst. Ihr Wesen und ihre Gesetze, Die Sanfte, Die Versuchung des heiligen Antonius, La Revue indépendante, Les lauriers sont coupés}
               \section[Georg Brandes an Arthur Schnitzler, 16. 6. 1901]{ Georg Brandes an Arthur Schnitzler, 16. 6. 1901}\nopagebreak\mylabel{v}\rehead{ }\begin{ledgroupsized}[t]{13cm}\normalsize\beginnumbering\briefempfaengerindex{Schnitzler, Arthur@\textsc{Schnitzler, Arthur}!zzzBrandes, Georg@\emph{von Georg Brandes}!1901-06-161@{16. 6. 1901}|(be} \toendnotes[C]{\smallbreak\pagebreak[2]} \Standort{CUL, Schnitzler, B 17.}
\physDesc{Brief, 1 Blatt, 3 Seiten, 1111 Zeichen
\newline{}Handschrift: blaue Tinte, lateinische Kurrent
\newline{}Ordnung: mit Bleistift von unbekannter Hand nummeriert: »25« }\buchAbdrucke{\weitereDrucke{Georg Brandes, Arthur Schnitzler: \emph{Ein Briefwechsel}. Hg. Kurt Bergel. Bern: \emph{Francke} 1956, S. 88–89.} }\toendnotes[C]{\smallbreak}\pstart
           \raggedleft{}{\pb}Kopenhagen\oindex{Kopenhagen@\textbf{Kopenhagen}|pw}\hspace*{1.5em}16 Juni 1901\pend
           \pstart{}Verehrter Freund\pend\pstart
           Zwar ist Krotkaja\pwindex{\textcolor{red}{\textsuperscript{XXXX1 indx}}!Sanfte1876@\strich\emph{Die Sanfte} {[}1876{]}|pw} ein Monolog – es gibt so viele
               Monologe, Flauberts\pwindex{Flaubert, Gustave 12.12.1821 – 08.05.1880@\textsc{Flaubert, Gustave} (12.12.1821 – 08.05.1880), \emph{Schriftsteller}|pw}{ }St. Antoine\pwindex{Flaubert, Gustave 12.12.1821 – 08.05.1880@\textsc{Flaubert, Gustave} (12.12.1821 – 08.05.1880), \emph{Schriftsteller}!Versuchung des heiligen Antonius1874@\strich\emph{Die Versuchung des heiligen Antonius} {[}1874{]}|pw} ist auch ein
               Monolog – aber das kleine Buch hat gar keine Form-Aehnlichkeit mit der Ihrigen. Les lauriers sont coupés\pwindex{Dujardin, Edouard 10.10.1861 – 31.10.1949@\textsc{Dujardin, Édouard} (10.10.1861 – 31.10.1949), \emph{Schriftsteller}!lauriers sont coupes1887@\strich\emph{Les lauriers sont coupés} {[}1887{]}|pw} las ich vor – 16 Jahren
               glaub ich, als die \label{K_L01129-1v}\edtext{Erzählung in la Révue Indépendante\pwindex{?? Werk@Nicht ermittelte Verfasserinnen und Verfasser!Revue independante1884 – 1895@\emph{La Revue indépendante} {[}1884 – 1895{]}|pw}}{\lemma{\textnormal{\emph{Erzählung … Indépendante}}}\Cendnote{\textnormal{Édouard Dujardin\pwindex{Dujardin, Edouard 10.10.1861 – 31.10.1949@\textsc{Dujardin, Édouard} (10.10.1861 – 31.10.1949), \emph{Schriftsteller}|pwk}: \emph{Les lauriers sont coupés}\pwindex{Dujardin, Edouard 10.10.1861 – 31.10.1949@\textsc{Dujardin, Édouard} (10.10.1861 – 31.10.1949), \emph{Schriftsteller}!lauriers sont coupes1887@\strich\emph{Les lauriers sont coupés} {[}1887{]}|pwk}. In: \emph{La Revue indépendante}\pwindex{?? Werk@Nicht ermittelte Verfasserinnen und Verfasser!Revue independante1884 – 1895@\emph{La Revue indépendante} {[}1884 – 1895{]}|pwk}, Bd. 3, H. 7, Mai 1887,
                     S. 289–316; H. 8, Juni 1887, S. 472–494; H. 9, Juli
                        1887, S. 122–137; H. 10, August 1887,
                  S. 221–244.}}}\label{K_L01129-1h} stand, und es machte mir einen starken und originellen
               Eindruck, aber das Einzelne hab ich vergessen.\pend
           \pstart
           Ich kam zwar durch Wien\oindex{Wien@\textbf{Wien}|pw}, blieb aber {\pb}dort nur zwei Stunden. Ich hatte
               eine Scheu, Sie wieder aufzusuchen. Ich finde mich selbst sehr oft für Fremde
               ermüdend, fuhr deshalb nur durch; ich war bewegt, unaufgelegt zum Sprechen.\pend
           \pstart
           Durch Ihre Güte erhielt ich Renate Fuchs\pwindex{Wassermann, Jakob 10.03.1873 – 01.01.1934@\textsc{Wassermann, Jakob} (10.03.1873 – 01.01.1934), \emph{Schriftsteller}!Geschichte der jungen Renate Fuchs1900-02-01@\strich\emph{Die Geschichte der jungen Renate Fuchs} {[}1900-02-01{]}|pw}; es ist
               ein starkes Buch, aber die Grundidee so willkürlich, das Nachtwandern der Heldin. Das
               Beste sind die Details, scheint mir, die vielen tiefen Reflexionen. Im Ganzen jedoch
                  \label{K_L01129-2v}\edtext{Kunst = Kunst, nicht Kunst =
                  Natur}{\lemma{\textnormal{\emph{Kunst = … Natur}}}\Cendnote{\textnormal{Anspielung auf Arno Holz\pwindex{Holz, Arno 26.04.1863 – 26.10.1929@\textsc{Holz, Arno} (26.04.1863 – 26.10.1929), \emph{Schriftsteller}|pwk}’ Formel: »Kunst = Natur – x« aus
                     \emph{Die Kunst. Ihr Wesen und ihre Gesetze}\pwindex{Holz, Arno 26.04.1863 – 26.10.1929@\textsc{Holz, Arno} (26.04.1863 – 26.10.1929), \emph{Schriftsteller}!Kunst. Ihr Wesen und ihre Gesetze1891@\strich\emph{Die Kunst. Ihr Wesen und ihre Gesetze} {[}1891{]}|pwk}.
                  Berlin: \emph{Issleib}\orgindex{Wilhelm Issleib Verlagsbuchhandlung und Buchdruckerei@Wilhelm Issleib Verlagsbuchhandlung und Buchdruckerei|pwk}{ }1891.}}}\label{K_L01129-2h}. Ist es nicht wahr? Aber der Mann\pwindex{Wassermann, Jakob 10.03.1873 – 01.01.1934@\textsc{Wassermann, Jakob} (10.03.1873 – 01.01.1934), \emph{Schriftsteller}|pwv} hat sehr viel Talent.\pend
           \pstart
           {\pb}Hier haben wir scheussliches
               Wetter, fast Winter. Mitte Juli gehe ich nach Karlsbad\oindex{Karlsbad@\textbf{Karlsbad}|pw}, ich habe mit Georges
                  Clemenceau\pwindex{Clemenceau, Georges 1841-09-28 – 1929-11-24@\textsc{Clemenceau, Georges} (1841-09-28 – 1929-11-24), \emph{Politiker}|pw} verabredet, ihn dort zu treffen.\pend
           \pstart
           Von ganzem Herzen\pend
           \pstart
           Ihr{\\[\baselineskip]}\spacefill\mbox{Georg Brandes}\pend
           \leftskip=0em{}
         
         \endnumbering\mylabel{h}\end{ledgroupsized}  \newcommand{\dateiname}{L01129}\newcommand{\titel}{Georg Brandes an Arthur Schnitzler, 16. 6. 1901}\newcommand{\editorInnen}{Martin Anton Müller und Gerd-Hermann Susen}%% latex-leseansicht-abspann.tex
%% Abspann für die Leseansicht.
%% Der Schalter \ifkorrekturansicht ist bereits durch den Vorspann gesetzt.

%% latex-abspann.tex
%% Gemeinsamer Abspann für Korrekturansicht und Leseansicht.
%% Setzt den Schalter \ifkorrekturansicht voraus (gesetzt in den
%% einbindenden Dateien latex-korrekturansicht-abspann.tex bzw.
%% latex-leseansicht-abspann.tex).
%% ---------------------------------------------------------------

\normalsize

% Das esempio-Environment wird nur in der Leseansicht benötigt
\ifkorrekturansicht\else
\newenvironment{esempio}[3]%
{
    \vspace{1.5ex}
    \rlap{\underline{#1}}
    \par
    \setlength{\parindent}{0cm}
    \nopagebreak
    \leftskip=#2cm
    \rightskip=#3cm
}
{
    \par
}
\fi

\doendnotes{C}
\bigskip
\vfill

\clearpage

\footnotesize

\ifkorrekturansicht
  \lohead{\textsc{register}}
\fi

% theindex-Environment neu definieren ohne reledmac
\makeatletter
\renewenvironment{theindex}{%
  \ifkorrekturansicht
    \section*{\indexname}%
  \else
    \subsubsection*{Index der erwähnten Entitäten}%
  \fi
  \setlength{\parindent}{0pt}%
  \setlength{\parskip}{0pt plus 0.3pt}%
  \let\item\@idxitem
}{%
  \ifkorrekturansicht\clearpage\fi
}
\makeatother

\IfFileExists{\jobname-pw.ind}{\input{\jobname-pw.ind}}{}

% Quellenangabe nur in der Leseansicht
\ifkorrekturansicht\else
% Fallback-Definitionen, falls die .tex-Datei \titel etc. nicht gesetzt hat
\providecommand{\titel}{}
\providecommand{\editorInnen}{}
\providecommand{\dateiname}{\jobname}

\vspace{3cm}

\vfill

\footnotesize
\textsc{Quelle}: \titel. Herausgegeben von {\editorInnen}. In: \emph{Arthur Schnitzler: Briefwechsel mit Autorinnen und Autoren}.
 Digitale Edition, https://schnitzler-briefe.acdh.oeaw.ac.at/{\dateiname}.html (Stand \today)
\fi

\end{document}


      