%% latex-korrekturansicht-vorspann.tex
%% Vorspann für die Korrekturansicht.
%% Lädt die gemeinsame Datei latex-vorspann.tex mit gesetztem Schalter.

\newif\ifkorrekturansicht
\korrekturansichttrue

\input{../tex-inputs/latex-vorspann}


\section[Georg Brandes an Arthur Schnitzler, 16. 6. 1901]{L01129 Georg Brandes an Arthur Schnitzler, 16. 6. 1901}
\nopagebreak\mylabel{L01129v}
\rehead{ }\normalsize\beginnumbering\briefempfaengerindex{Schnitzler, Arthur@\textsc{Schnitzler, Arthur}!zzzBrandes, Georg@\emph{von Georg Brandes}!1901-06-161@{16. 6. 1901}|(be}
\toendnotes[C]{\smallbreak\pagebreak[2]}\Standort{CUL, Schnitzler, B 17.}
\physDesc{Brief, 1 Blatt, 3 Seiten, 1111 Zeichen
\newline{}Handschrift: blaue Tinte, lateinische Kurrent
\newline{}Ordnung: mit Bleistift von unbekannter Hand nummeriert: »25« }
\buchAbdrucke{\weitereDrucke{Georg Brandes, Arthur Schnitzler: \emph{Ein Briefwechsel}. Bern: \emph{Francke} 1956, S. 88–89.} }\toendnotes[C]{\smallbreak}
\pstart
           \raggedleft{}{\pb}Kopenhagen\oindex{Kopenhagen@\textbf{Kopenhagen}, \emph{P.PPLC}|pw}\hspace*{1.5em}16 Juni 1901\pend
           
\pstart{}Verehrter Freund\pend\vspace{0.5em}
\pstart
           Zwar ist Krotkaja\pwindex{Sanfte@\emph{Die Sanfte}|pw} ein Monolog – es gibt so viele
               Monologe, Flauberts\pwindex{Flaubert, Gustave 12.12.1821 – 08.05.1880@\textsc{Flaubert, Gustave} (12.12.1821 – 08.05.1880), \emph{Schriftsteller/Schriftstellerin}|pw}{ }St. Antoine\pwindex{Versuchung des heiligen Antonius@\emph{Die Versuchung des heiligen Antonius}|pw} ist auch ein
               Monolog – aber das kleine Buch hat gar keine Form-Aehnlichkeit mit der Ihrigen. Les lauriers sont coupés\pwindex{lauriers sont coupes@\emph{Les lauriers sont coupés}|pw} las ich vor – 16 Jahren
               glaub ich, als die \label{K_L01129-1v}\edtext{Erzählung in la Révue Indépendante\pwindex{Revue independante@\emph{La Revue indépendante}|pw}}{\lemma{\textnormal{\emph{Erzählung … Indépendante}}}\Cendnote{\textnormal{Édouard Dujardin\pwindex{Dujardin, Edouard 10.10.1861 – 31.10.1949@\textsc{Dujardin, Édouard} (10.10.1861 – 31.10.1949), \emph{Schriftsteller/Schriftstellerin}|pwk}: \emph{Les lauriers sont coupés}\pwindex{lauriers sont coupes@\emph{Les lauriers sont coupés}|pwk}. In: \emph{La Revue indépendante}\pwindex{Revue independante@\emph{La Revue indépendante}|pwk}, Bd. 3, H. 7, Mai 1887,
                     S. 289–316; H. 8, Juni 1887, S. 472–494; H. 9, Juli
                        1887, S. 122–137; H. 10, August 1887,
                  S. 221–244.}}}\label{K_L01129-1} stand, und es machte mir einen starken und originellen
               Eindruck, aber das Einzelne hab ich vergessen.\pend
           
\pstart
           Ich kam zwar durch Wien\oindex{Wien@\textbf{Wien}, \emph{A.ADM2}|pw}, blieb aber {\pb}dort nur zwei Stunden. Ich hatte
               eine Scheu, Sie wieder aufzusuchen. Ich finde mich selbst sehr oft für Fremde
               ermüdend, fuhr deshalb nur durch; ich war bewegt, unaufgelegt zum Sprechen.\pend
           
\pstart
           Durch Ihre Güte erhielt ich Renate Fuchs\pwindex{Geschichte der jungen Renate Fuchs@\emph{Die Geschichte der jungen Renate Fuchs}|pw}; es ist
               ein starkes Buch, aber die Grundidee so willkürlich, das Nachtwandern der Heldin. Das
               Beste sind die Details, scheint mir, die vielen tiefen Reflexionen. Im Ganzen jedoch
                  \label{K_L01129-2v}\edtext{Kunst = Kunst, nicht Kunst =
                  Natur}{\lemma{\textnormal{\emph{Kunst = … Natur}}}\Cendnote{\textnormal{Anspielung auf Arno Holz\pwindex{Holz, Arno 26.04.1863 – 26.10.1929@\textsc{Holz, Arno} (26.04.1863 – 26.10.1929), \emph{Schriftsteller/Schriftstellerin}|pwk}’ Formel: »Kunst = Natur – x« aus
                     \emph{Die Kunst. Ihr Wesen und ihre Gesetze}\pwindex{Kunst. Ihr Wesen und ihre Gesetze@\emph{Die Kunst. Ihr Wesen und ihre Gesetze}|pwk}.
                  Berlin: \emph{Issleib}\orgindex{Wilhelm Issleib Verlagsbuchhandlung und Buchdruckerei@Wilhelm Issleib Verlagsbuchhandlung und Buchdruckerei|pwk}{ }1891.}}}\label{K_L01129-2}. Ist es nicht wahr? Aber der Mann\pwindex{Wassermann, Jakob 10.03.1873 – 01.01.1934@\textsc{Wassermann, Jakob} (10.03.1873 – 01.01.1934), \emph{Schriftsteller/Schriftstellerin}|pwv} hat sehr viel Talent.\pend
           
\pstart
           {\pb}Hier haben wir scheussliches
               Wetter, fast Winter. Mitte Juli gehe ich nach Karlsbad\oindex{Karlsbad@\textbf{Karlsbad}, \emph{P.PPLA}|pw}, ich habe mit Georges
                  Clemenceau\pwindex{Clemenceau, Georges 1841-09-28 – 1929-11-24@\textsc{Clemenceau, Georges} (1841-09-28 – 1929-11-24), \emph{Politiker/Politikerin}|pw} verabredet, ihn dort zu treffen.\pend
           
\pstart
           Von ganzem Herzen\pend
           
\pstart
           Ihr{\\[\baselineskip]}\spacefill\mbox{Georg Brandes}\pend
           \leftskip=0em{}\selectlanguage{ngerman}\endnumbering\briefempfaengerindex{Schnitzler, Arthur@\textsc{Schnitzler, Arthur}!zzzBrandes, Georg@\emph{von Georg Brandes}!1901-06-161@{16. 6. 1901}|)be}\mylabel{L01129h}  \normalsize

\doendnotes{C}
\bigskip
\vfill

\clearpage

\footnotesize

\lohead{\textsc{register}}

% Definiere theindex-Environment komplett neu ohne reledmac
\makeatletter
\renewenvironment{theindex}{%
  \section*{\indexname}%
  \setlength{\parindent}{0pt}%
  \setlength{\parskip}{0pt plus 0.3pt}%
  \let\item\@idxitem
}{%
  \clearpage
}
\makeatother

\IfFileExists{\jobname-pw.ind}{\input{\jobname-pw.ind}}{}

\end{document}

      