%% latex-leseansicht-vorspann.tex
%% Vorspann für die Leseansicht.
%% Lädt die gemeinsame Datei latex-vorspann.tex mit nicht gesetztem Schalter.

\newif\ifkorrekturansicht
\korrekturansichtfalse

\input{../tex-inputs/latex-vorspann}


         \renewcommand{\erwaehnteInstitutionen}{Institutionen: Ferienheim}
         \renewcommand{\erwaehnteOrte}{Orte: Berlin, Sofiensäle, Wien}
         \renewcommand{\erwaehnteWerke}{Werke: Abschiedssouper, Weihnachts-Einkäufe}
               \section[Max Burckhard an Arthur Schnitzler, {[}16. 1. 1898{]}]{ Max Burckhard an Arthur Schnitzler, {[}16. 1. 1898{]}}\nopagebreak\mylabel{v}\rehead{ }\begin{ledgroupsized}[t]{13cm}\normalsize\beginnumbering \toendnotes[C]{\smallbreak\pagebreak[2]} \Standort{CUL, Schnitzler, B 20.}
\physDesc{Brief, 1 Blatt, 1 Seite, 262 Zeichen
\newline{}Handschrift: schwarze Tinte, deutsche Kurrent
\newline{}Schnitzler: mit Bleistift datiert: »16/1 98« 
\newline{}Ordnung: mit Bleistift von unbekannter Hand nummeriert:
                                    »11« }\toendnotes[C]{\smallbreak}\pstart
           \raggedleft{}{\pb}\pend
           \pstart{}Sehr verehrter Herr Doctor!\pend\pstart
           Leider iſt mir ein Hindernis für heute unterlaufen, da der Beſitzer der Jagdhütte, wo
               ich den So{\geminationm}er bin, heute Abend anko{\geminationm}t u ich ihn erwarten muß. Ich retourniere alſo mit
               herzlichem Dank die \label{K_L00763-1v}\edtext{Karten}{\lemma{\textnormal{\emph{Karten}}}\Cendnote{\textnormal{Am 16. 1. 1898 fand in den Sofiensälen\oindex{Sofiensaele@\textbf{Sofiensäle}|pwk} in Wien\oindex{Wien@\textbf{Wien}|pwk} eine Wohltätigkeitsveranstaltung zugunsten des Vereines \emph{Ferienheim}\orgindex{Ferienheim@Ferienheim|pwk} statt, der Landaufenthalte von
                  Kindern förderte und organisierte. Von Schnitzler\pwindex{Schnitzler, Arthur 15.05.1862 – 21.10.1931@\textsc{Schnitzler, Arthur} (15.05.1862 – 21.10.1931), \emph{Schriftsteller, Mediziner}|pwk} wurden \emph{Weihnachts-Einkäufe}\pwindex{Schnitzler, Arthur 15.05.1862 – 21.10.1931@\textsc{Schnitzler, Arthur} (15.05.1862 – 21.10.1931), \emph{Schriftsteller, Mediziner}!Weihnachts-Einkaeufe24. 12. 1891@\strich\emph{Weihnachts-Einkäufe} {[}24. 12. 1891{]}|pwk} und \emph{Abschiedssouper}\pwindex{Schnitzler, Arthur 15.05.1862 – 21.10.1931@\textsc{Schnitzler, Arthur} (15.05.1862 – 21.10.1931), \emph{Schriftsteller, Mediziner}!Abschiedssouper1892@\strich\emph{Abschiedssouper} {[}1892{]}|pwk} gegeben.}}}\label{K_L00763-1h}.\pend
           \pstart
           Mit herzlicher Empfehlung{\\[\baselineskip]}\spacefill\mbox{D\textsuperscript{r}Burckhard}\pend
           \leftskip=0em{}
         
         \endnumbering\mylabel{h}\end{ledgroupsized}  \newcommand{\dateiname}{L00763}\newcommand{\titel}{Max Burckhard an Arthur Schnitzler, [16. 1. 1898]}\newcommand{\editorInnen}{Martin Anton Müller und Gerd-Hermann Susen}%% latex-leseansicht-abspann.tex
%% Abspann für die Leseansicht.
%% Der Schalter \ifkorrekturansicht ist bereits durch den Vorspann gesetzt.

%% latex-abspann.tex
%% Gemeinsamer Abspann für Korrekturansicht und Leseansicht.
%% Setzt den Schalter \ifkorrekturansicht voraus (gesetzt in den
%% einbindenden Dateien latex-korrekturansicht-abspann.tex bzw.
%% latex-leseansicht-abspann.tex).
%% ---------------------------------------------------------------

\normalsize

% Das esempio-Environment wird nur in der Leseansicht benötigt
\ifkorrekturansicht\else
\newenvironment{esempio}[3]%
{
    \vspace{1.5ex}
    \rlap{\underline{#1}}
    \par
    \setlength{\parindent}{0cm}
    \nopagebreak
    \leftskip=#2cm
    \rightskip=#3cm
}
{
    \par
}
\fi

\doendnotes{C}
\bigskip
\vfill

\clearpage

\footnotesize

\ifkorrekturansicht
  \lohead{\textsc{register}}
\fi

% theindex-Environment neu definieren ohne reledmac
\makeatletter
\renewenvironment{theindex}{%
  \ifkorrekturansicht
    \section*{\indexname}%
  \else
    \subsubsection*{Index der erwähnten Entitäten}%
  \fi
  \setlength{\parindent}{0pt}%
  \setlength{\parskip}{0pt plus 0.3pt}%
  \let\item\@idxitem
}{%
  \ifkorrekturansicht\clearpage\fi
}
\makeatother

\IfFileExists{\jobname-pw.ind}{\input{\jobname-pw.ind}}{}

% Quellenangabe nur in der Leseansicht
\ifkorrekturansicht\else
% Fallback-Definitionen, falls die .tex-Datei \titel etc. nicht gesetzt hat
\providecommand{\titel}{}
\providecommand{\editorInnen}{}
\providecommand{\dateiname}{\jobname}

\vspace{3cm}

\vfill

\footnotesize
\textsc{Quelle}: \titel. Herausgegeben von {\editorInnen}. In: \emph{Arthur Schnitzler: Briefwechsel mit Autorinnen und Autoren}.
 Digitale Edition, https://schnitzler-briefe.acdh.oeaw.ac.at/{\dateiname}.html (Stand \today)
\fi

\end{document}


      