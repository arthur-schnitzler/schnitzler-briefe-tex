%% latex-leseansicht-vorspann.tex
%% Vorspann für die Leseansicht.
%% Lädt die gemeinsame Datei latex-vorspann.tex mit nicht gesetztem Schalter.

\newif\ifkorrekturansicht
\korrekturansichtfalse

\input{../tex-inputs/latex-vorspann}


\section[Arthur Schnitzler an Stefan Zweig, 11. 2. 1915]{L03772 Arthur Schnitzler an Stefan Zweig, 11. 2. 1915}
\nopagebreak\mylabel{L03772v}
\rehead{ }\normalsize\beginnumbering\briefempfaengerindex{Zweig, Stefan@\textsc{Zweig, Stefan}!zzzSchnitzler, Arthur@\emph{von Arthur Schnitzler}!1915-02-112@{11. 2. 1915}|(be}
\toendnotes[C]{\smallbreak\pagebreak[2]}
\correspDesc{Versand  durch Arthur Schnitzler am 11. 2. 1915 in Wien
\newline{}Erhalt  durch Stefan Zweig im Zeitraum [11. 2. 1915 – 14. 2. 1915?] in Wien}\toendnotes[C]{\smallbreak}
\Standort{Jerusalem, National Library of Israel, ARC. Ms. Var. 305 1 58 Stefan Zweig Collection.}
\physDesc{Briefkarte, 815 Zeichen
\newline{}Handschrift: schwarze Tinte, deutsche Kurrent}\toendnotes[C]{\smallbreak}
\pstart
           {\pb}\textcolor{gray}{\textbf{Dr. Arthur Schnitzler}}\hfill 11. 2. 915\pend
           
\pstart
           \textcolor{gray}{\textbf{Wien XVIII. Sternwartestrasse 71\oindex{Wien@\textbf{Wien}!XVIII., Währing@\textbf{XVIII., Währing}!Sternwartestraße 71@\textbf{Sternwartestraße 71}, \emph{Wohngebäude}|pw}}}\pend
           \vspace{0.5em}
\pstart
           lieber Herr Doktor Zweig, vielen Dank für Ihre \label{K_L03772-1v}\edtext{Karte}{\lemma{\textnormal{\emph{Karte}}}\Cendnote{\textnormal{XXXX Auszeichnungsfehler: Dokument L03651 nicht gefunden.
               }}}\label{K_L03772-1}, die mich veranlaßt hat, auch an Rom.
                  Rolland\pwindex{Rolland, Romain 29.\,1.\,1866 Clamecy – 30.\,12.\,1944 Vézelay@\textsc{Rolland, Romain} (29.\,1.\,1866 Clamecy – 30.\,12.\,1944 Vézelay), \emph{Schriftsteller}|pw} gleich ein \label{K_L03772-2v}\edtext{paar Worte}{\lemma{\textnormal{\emph{paar Worte}}}\Cendnote{\textnormal{XXXX Auszeichnungsfehler: Dokument L04215 nicht gefunden.}}}\label{K_L03772-2} zu{ }ſchreiben. Bisher haben{ }ſich die Angriffe\pwindex{Schnitzler erhebt Einspruch@\emph{Schnitzler erhebt Einspruch}|pwv},
               von denen Sie reden, nur in ein paar antiſemitiſchen Blättern gefunden – und ich habe
               nie davon geträumt, daſs gerade dieſes Ja{\geminationm}ervölkchen in Kriegszeiten Gerechtigkeit u
               Anstand {\pb}kennen würde – da ja auch{ }ſonſt von der
               reinigenden Kraft des Kriegs (hinter den Schützengräben) nicht viel zu verſpüren iſt.
               – Im übrigen hab ich, wie Sie mit{ }ſo freundſchaftlichen Worten wünſchen, thatſächlich
               zu \label{K_L03772-11v}\edtext{arbeiten\pwindex{Schnitzler, Arthur 15.\,5.\,1862 Wien – 21.\,10.\,1931 ebd.@\textsc{Schnitzler, Arthur} (15.\,5.\,1862 Wien – 21.\,10.\,1931 ebd.), \emph{Schriftsteller, Mediziner}!Flucht in die Finsternis@\strich\emph{Flucht in die Finsternis}|pwv}}{\lemma{\textnormal{\emph{arbeiten}}}\Cendnote{\textnormal{Er arbeitete  unter dem Arbeitstitel \emph{Wahnsinn}\pwindex{Schnitzler, Arthur 15.\,5.\,1862 Wien – 21.\,10.\,1931 ebd.@\textsc{Schnitzler, Arthur} (15.\,5.\,1862 Wien – 21.\,10.\,1931 ebd.), \emph{Schriftsteller, Mediziner}!Flucht in die Finsternis@\strich\emph{Flucht in die Finsternis}|pwk}
                  an \emph{Flucht in die Finsternis}\pwindex{Schnitzler, Arthur 15.\,5.\,1862 Wien – 21.\,10.\,1931 ebd.@\textsc{Schnitzler, Arthur} (15.\,5.\,1862 Wien – 21.\,10.\,1931 ebd.), \emph{Schriftsteller, Mediziner}!Flucht in die Finsternis@\strich\emph{Flucht in die Finsternis}|pwk}.}}}\label{K_L03772-11} angefangen – es iſt Pflicht, Rettung, Notwendigkeit, – auch we{\geminationn} für{ }ſpäter nicht gar zu viel herausko{\geminationm}en{ }ſollte. Und Sie, lieber Herr Doctor,{ }ſind ganz
               in Ihr Archiv\orgindex{Kriegsarchiv@Kriegsarchiv|pwv} vergraben?\pend
           
\pstart
           Wir grüßen Sie herzlichſt, auf baldgs Wiederſehn!{\\[\baselineskip]}Ihr \spacefill\mbox{Arthur
                  Schnitzler}\pend
           \leftskip=0em{}\selectlanguage{ngerman}\endnumbering\briefempfaengerindex{Zweig, Stefan@\textsc{Zweig, Stefan}!zzzSchnitzler, Arthur@\emph{von Arthur Schnitzler}!1915-02-112@{11. 2. 1915}|)be}\mylabel{L03772h}  \newcommand{\dateiname}{L03772}\newcommand{\titel}{Arthur Schnitzler an Stefan Zweig, 11. 2. 1915}\newcommand{\editorInnen}{Selma Jahnke und Martin Anton Müller}%% latex-leseansicht-abspann.tex
%% Abspann für die Leseansicht.
%% Der Schalter \ifkorrekturansicht ist bereits durch den Vorspann gesetzt.

%% latex-abspann.tex
%% Gemeinsamer Abspann für Korrekturansicht und Leseansicht.
%% Setzt den Schalter \ifkorrekturansicht voraus (gesetzt in den
%% einbindenden Dateien latex-korrekturansicht-abspann.tex bzw.
%% latex-leseansicht-abspann.tex).
%% ---------------------------------------------------------------

\normalsize

% Das esempio-Environment wird nur in der Leseansicht benötigt
\ifkorrekturansicht\else
\newenvironment{esempio}[3]%
{
    \vspace{1.5ex}
    \rlap{\underline{#1}}
    \par
    \setlength{\parindent}{0cm}
    \nopagebreak
    \leftskip=#2cm
    \rightskip=#3cm
}
{
    \par
}
\fi

\doendnotes{C}
\bigskip
\vfill

\clearpage

\footnotesize

\ifkorrekturansicht
  \lohead{\textsc{register}}
\fi

% theindex-Environment neu definieren ohne reledmac
\makeatletter
\renewenvironment{theindex}{%
  \ifkorrekturansicht
    \section*{\indexname}%
  \else
    \subsubsection*{Index der erwähnten Entitäten}%
  \fi
  \setlength{\parindent}{0pt}%
  \setlength{\parskip}{0pt plus 0.3pt}%
  \let\item\@idxitem
}{%
  \ifkorrekturansicht\clearpage\fi
}
\makeatother

\IfFileExists{\jobname-pw.ind}{\input{\jobname-pw.ind}}{}

% Quellenangabe nur in der Leseansicht
\ifkorrekturansicht\else
% Fallback-Definitionen, falls die .tex-Datei \titel etc. nicht gesetzt hat
\providecommand{\titel}{}
\providecommand{\editorInnen}{}
\providecommand{\dateiname}{\jobname}

\vspace{3cm}

\vfill

\footnotesize
\textsc{Quelle}: \titel. Herausgegeben von {\editorInnen}. In: \emph{Arthur Schnitzler: Briefwechsel mit Autorinnen und Autoren}.
 Digitale Edition, https://schnitzler-briefe.acdh.oeaw.ac.at/{\dateiname}.html (Stand \today)
\fi

\end{document}


