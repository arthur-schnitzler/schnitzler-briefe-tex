%% latex-leseansicht-vorspann.tex
%% Vorspann für die Leseansicht.
%% Lädt die gemeinsame Datei latex-vorspann.tex mit nicht gesetztem Schalter.

\newif\ifkorrekturansicht
\korrekturansichtfalse

\input{../tex-inputs/latex-vorspann}


\section[Robert Adam an Arthur Schnitzler, 21. 5. 1916]{L02229 Robert Adam an Arthur Schnitzler, 21. 5. 1916}
\nopagebreak\mylabel{L02229v}
\rehead{ }\normalsize\beginnumbering\briefempfaengerindex{Schnitzler, Arthur@\textsc{Schnitzler, Arthur}!zzzAdam, Robert@\emph{von Robert Adam}!1916-05-211@{21. 5. 1916}|(be}
\toendnotes[C]{\smallbreak\pagebreak[2]}
\correspDesc{Versand  durch Robert Adam am 21. 5. 1916 in Wien
\newline{}Erhalt  durch Arthur Schnitzler im Zeitraum [21. 5. 1916
                  – 25. 5. 1916?] in Wien}\toendnotes[C]{\smallbreak}
\Standort{Wien, Österreichische Nationalbibliothek, Cod. ser. 52.263.}
\physDesc{Brief, maschinenschriftliche Abschrift, 1 Blatt, 1 Seite, 348 Zeichen
\newline{}Schreibmaschine
\newline{}Handschrift: Bleistift (\noindent{}geringfügige Korrekturen)
\newline{}Ordnung: von unbekannter Hand nummeriert: »176« }
\pstart
           \raggedleft{}{\pb}21. Mai 1916\pend
           
\pstart{}Hochverehrter Herr Doktor!\pend\vspace{0.5em}
\pstart
           Ich sende Ihnen, wie Sie wünschten, ein Exemplar des »Ali ibn Bekkâr\pwindex{Adam, Robert 20.\,4.\,1877 Wien – 16.\,10.\,1961 Baden bei Wien@\textsc{Adam, Robert} (20.\,4.\,1877 Wien – 16.\,10.\,1961 Baden bei Wien), \emph{Schriftsteller, Richter}!Geschichte des Alî ibn Bekkâr mit Schams an-Nahâr@\strich\emph{Die Geschichte des Alî ibn Bekkâr mit Schams an-Nahâr}|pw}« und eines von »In
                  aeternum\pwindex{Adam, Robert 20.\,4.\,1877 Wien – 16.\,10.\,1961 Baden bei Wien@\textsc{Adam, Robert} (20.\,4.\,1877 Wien – 16.\,10.\,1961 Baden bei Wien), \emph{Schriftsteller, Richter}!In aeternum. Eine Phantasie@\strich\emph{In aeternum. Eine Phantasie}|pw}«.\pend
           
\pstart
           Dies mein Erstlingsprodukt hiess ursprünglich »Götterdämmerung\pwindex{Adam, Robert 20.\,4.\,1877 Wien – 16.\,10.\,1961 Baden bei Wien@\textsc{Adam, Robert} (20.\,4.\,1877 Wien – 16.\,10.\,1961 Baden bei Wien), \emph{Schriftsteller, Richter}!In aeternum. Eine Phantasie@\strich\emph{In aeternum. Eine Phantasie}|pw}« und wurde um das Jahr 1900 geschrieben. Soviel
               zur Entschuldigung. Der Schluss dürfte Ihnen übrigens bekannt vorkommen! –\pend
           \pstart Mit ergebensten Grüssen Ihr \spacefill\mbox{RA}\pend{}\selectlanguage{ngerman}\endnumbering\briefempfaengerindex{Schnitzler, Arthur@\textsc{Schnitzler, Arthur}!zzzAdam, Robert@\emph{von Robert Adam}!1916-05-211@{21. 5. 1916}|)be}\mylabel{L02229h}  \newcommand{\dateiname}{L02229}\newcommand{\titel}{Robert Adam an Arthur Schnitzler, 21. 5. 1916}\newcommand{\editorInnen}{Martin Anton Müller und Gerd-Hermann Susen}%% latex-leseansicht-abspann.tex
%% Abspann für die Leseansicht.
%% Der Schalter \ifkorrekturansicht ist bereits durch den Vorspann gesetzt.

%% latex-abspann.tex
%% Gemeinsamer Abspann für Korrekturansicht und Leseansicht.
%% Setzt den Schalter \ifkorrekturansicht voraus (gesetzt in den
%% einbindenden Dateien latex-korrekturansicht-abspann.tex bzw.
%% latex-leseansicht-abspann.tex).
%% ---------------------------------------------------------------

\normalsize

% Das esempio-Environment wird nur in der Leseansicht benötigt
\ifkorrekturansicht\else
\newenvironment{esempio}[3]%
{
    \vspace{1.5ex}
    \rlap{\underline{#1}}
    \par
    \setlength{\parindent}{0cm}
    \nopagebreak
    \leftskip=#2cm
    \rightskip=#3cm
}
{
    \par
}
\fi

\doendnotes{C}
\bigskip
\vfill

\clearpage

\footnotesize

\ifkorrekturansicht
  \lohead{\textsc{register}}
\fi

% theindex-Environment neu definieren ohne reledmac
\makeatletter
\renewenvironment{theindex}{%
  \ifkorrekturansicht
    \section*{\indexname}%
  \else
    \subsubsection*{Index der erwähnten Entitäten}%
  \fi
  \setlength{\parindent}{0pt}%
  \setlength{\parskip}{0pt plus 0.3pt}%
  \let\item\@idxitem
}{%
  \ifkorrekturansicht\clearpage\fi
}
\makeatother

\IfFileExists{\jobname-pw.ind}{\input{\jobname-pw.ind}}{}

% Quellenangabe nur in der Leseansicht
\ifkorrekturansicht\else
% Fallback-Definitionen, falls die .tex-Datei \titel etc. nicht gesetzt hat
\providecommand{\titel}{}
\providecommand{\editorInnen}{}
\providecommand{\dateiname}{\jobname}

\vspace{3cm}

\vfill

\footnotesize
\textsc{Quelle}: \titel. Herausgegeben von {\editorInnen}. In: \emph{Arthur Schnitzler: Briefwechsel mit Autorinnen und Autoren}.
 Digitale Edition, https://schnitzler-briefe.acdh.oeaw.ac.at/{\dateiname}.html (Stand \today)
\fi

\end{document}


