%% latex-leseansicht-vorspann.tex
%% Vorspann für die Leseansicht.
%% Lädt die gemeinsame Datei latex-vorspann.tex mit nicht gesetztem Schalter.

\newif\ifkorrekturansicht
\korrekturansichtfalse

\input{../tex-inputs/latex-vorspann}


         \newcommand{\erwaehnteOrte}{Orte: Wien}
         \newcommand{\erwaehnteWerke}{Werke: Die Geschichte des Alî ibn Bekkâr mit Schams an-Nahâr, In aeternum. Eine Phantasie}
               \section[Robert Adam an Arthur Schnitzler, 21. 5. 1916]{ Robert Adam an Arthur Schnitzler, 21. 5. 1916}\nopagebreak\mylabel{v}\rehead{ }\begin{ledgroupsized}[t]{13cm}\normalsize\beginnumbering \toendnotes[C]{\smallbreak\pagebreak[2]} \Standort{Wien, Österreichische Nationalbibliothek, Cod. ser. 52.263.}
\physDesc{maschinelle Abschrift
\newline{}Handschrift: Bleistift (\noindent{}geringfügige Korrekturen)\newline{}Ordnung: von unbekannter Hand nummeriert:
                                            »176« }\pstart
           \raggedleft{}{\pb}21. Mai 1916\pend
           \pstart{}Hochverehrter Herr Doktor!\pend\pstart
           Ich sende Ihnen, wie Sie wünschten, ein Exemplar des »Ali ibn Bekkâr\pwindex{Adam, Robert 20.04.1877 – 16.10.1961@\textsc{Adam, Robert} (20.04.1877 – 16.10.1961), \emph{Schriftsteller, Richter}!Geschichte des Alî ibn Bekkâr mit Schams an-Nahâr1909@\strich\emph{Die Geschichte des Alî ibn Bekkâr mit Schams an-Nahâr} {[}1909{]}|pw}« und eines von »In aeternum\pwindex{Adam, Robert 20.04.1877 – 16.10.1961@\textsc{Adam, Robert} (20.04.1877 – 16.10.1961), \emph{Schriftsteller, Richter}!In aeternum. Eine Phantasie1905@\strich\emph{In aeternum. Eine Phantasie} {[}1905{]}|pw}«.\pend
           \pstart
           Dies mein Erstlingsprodukt hiess ursprünglich »Götterdämmerung\pwindex{Adam, Robert 20.04.1877 – 16.10.1961@\textsc{Adam, Robert} (20.04.1877 – 16.10.1961), \emph{Schriftsteller, Richter}!In aeternum. Eine Phantasie1905@\strich\emph{In aeternum. Eine Phantasie} {[}1905{]}|pw}« und wurde um das Jahr 1900 geschrieben.
                    Soviel zur Entschuldigung. Der Schluss dürfte Ihnen übrigens bekannt
                    vorkommen! –\pend
           \pstart Mit ergebensten Grüssen Ihr \spacefill\mbox{RA}\pend{}
         
         \endnumbering\mylabel{h}\end{ledgroupsized}  \newcommand{\dateiname}{L02229}\newcommand{\titel}{Robert Adam an Arthur Schnitzler, 21. 5. 1916}\newcommand{\editorInnen}{Martin Anton Müller und Gerd-Hermann Susen}%% latex-leseansicht-abspann.tex
%% Abspann für die Leseansicht.
%% Der Schalter \ifkorrekturansicht ist bereits durch den Vorspann gesetzt.

%% latex-abspann.tex
%% Gemeinsamer Abspann für Korrekturansicht und Leseansicht.
%% Setzt den Schalter \ifkorrekturansicht voraus (gesetzt in den
%% einbindenden Dateien latex-korrekturansicht-abspann.tex bzw.
%% latex-leseansicht-abspann.tex).
%% ---------------------------------------------------------------

\normalsize

% Das esempio-Environment wird nur in der Leseansicht benötigt
\ifkorrekturansicht\else
\newenvironment{esempio}[3]%
{
    \vspace{1.5ex}
    \rlap{\underline{#1}}
    \par
    \setlength{\parindent}{0cm}
    \nopagebreak
    \leftskip=#2cm
    \rightskip=#3cm
}
{
    \par
}
\fi

\doendnotes{C}
\bigskip
\vfill

\clearpage

\footnotesize

\ifkorrekturansicht
  \lohead{\textsc{register}}
\fi

% theindex-Environment neu definieren ohne reledmac
\makeatletter
\renewenvironment{theindex}{%
  \ifkorrekturansicht
    \section*{\indexname}%
  \else
    \subsubsection*{Index der erwähnten Entitäten}%
  \fi
  \setlength{\parindent}{0pt}%
  \setlength{\parskip}{0pt plus 0.3pt}%
  \let\item\@idxitem
}{%
  \ifkorrekturansicht\clearpage\fi
}
\makeatother

\IfFileExists{\jobname-pw.ind}{\input{\jobname-pw.ind}}{}

% Quellenangabe nur in der Leseansicht
\ifkorrekturansicht\else
% Fallback-Definitionen, falls die .tex-Datei \titel etc. nicht gesetzt hat
\providecommand{\titel}{}
\providecommand{\editorInnen}{}
\providecommand{\dateiname}{\jobname}

\vspace{3cm}

\vfill

\footnotesize
\textsc{Quelle}: \titel. Herausgegeben von {\editorInnen}. In: \emph{Arthur Schnitzler: Briefwechsel mit Autorinnen und Autoren}.
 Digitale Edition, https://schnitzler-briefe.acdh.oeaw.ac.at/{\dateiname}.html (Stand \today)
\fi

\end{document}


      