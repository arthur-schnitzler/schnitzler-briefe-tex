%% latex-korrekturansicht-vorspann.tex
%% Vorspann für die Korrekturansicht.
%% Lädt die gemeinsame Datei latex-vorspann.tex mit gesetztem Schalter.

\newif\ifkorrekturansicht
\korrekturansichttrue

\input{../tex-inputs/latex-vorspann}


\section[Robert Adam an Arthur Schnitzler, 21. 5. 1916]{L02229 Robert Adam an Arthur Schnitzler, 21. 5. 1916}
\nopagebreak\mylabel{L02229v}
\rehead{ }\normalsize\beginnumbering\briefempfaengerindex{Schnitzler, Arthur@\textsc{Schnitzler, Arthur}!zzzAdam, Robert@\emph{von Robert Adam}!1916-05-211@{21. 5. 1916}|(be}
\toendnotes[C]{\smallbreak\pagebreak[2]}\Standort{Wien, Österreichische Nationalbibliothek, Cod. ser. 52.263.}
\physDesc{Brief, maschinenschriftliche Abschrift1 Blatt, 1 Seite, 348 Zeichen
\newline{}Schreibmaschine
\newline{}Handschrift: Bleistift (\noindent{}geringfügige Korrekturen)
\newline{}Ordnung: von unbekannter Hand nummeriert: »176« }
\pstart
           \raggedleft{}{\pb}21. Mai 1916\pend
           
\pstart{}Hochverehrter Herr Doktor!\pend\vspace{0.5em}
\pstart
           Ich sende Ihnen, wie Sie wünschten, ein Exemplar des »Ali ibn Bekkâr\pwindex{Geschichte des Alî ibn Bekkâr mit Schams an-Nahâr@\emph{Die Geschichte des Alî ibn Bekkâr mit Schams an-Nahâr}|pw}« und eines von »In
                  aeternum\pwindex{In aeternum. Eine Phantasie@\emph{In aeternum. Eine Phantasie}|pw}«.\pend
           
\pstart
           Dies mein Erstlingsprodukt hiess ursprünglich »Götterdämmerung\pwindex{In aeternum. Eine Phantasie@\emph{In aeternum. Eine Phantasie}|pw}« und wurde um das Jahr 1900 geschrieben. Soviel
               zur Entschuldigung. Der Schluss dürfte Ihnen übrigens bekannt vorkommen! –\pend
           \pstart Mit ergebensten Grüssen Ihr \spacefill\mbox{RA}\pend{}\selectlanguage{ngerman}\endnumbering\briefempfaengerindex{Schnitzler, Arthur@\textsc{Schnitzler, Arthur}!zzzAdam, Robert@\emph{von Robert Adam}!1916-05-211@{21. 5. 1916}|)be}\mylabel{L02229h}  \normalsize

\doendnotes{C}
\bigskip
\vfill

\clearpage

\footnotesize

\lohead{\textsc{register}}

% Definiere theindex-Environment komplett neu ohne reledmac
\makeatletter
\renewenvironment{theindex}{%
  \section*{\indexname}%
  \setlength{\parindent}{0pt}%
  \setlength{\parskip}{0pt plus 0.3pt}%
  \let\item\@idxitem
}{%
  \clearpage
}
\makeatother

\IfFileExists{\jobname-pw.ind}{\input{\jobname-pw.ind}}{}

\end{document}

      