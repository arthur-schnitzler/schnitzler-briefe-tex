%% latex-leseansicht-vorspann.tex
%% Vorspann für die Leseansicht.
%% Lädt die gemeinsame Datei latex-vorspann.tex mit nicht gesetztem Schalter.

\newif\ifkorrekturansicht
\korrekturansichtfalse

\input{../tex-inputs/latex-vorspann}

\begin{center}
            \textcolor{red}{ENTWURF, NICHT FERTIG KORRIGIERT}
                      \end{center}
            
         
         \renewcommand{\erwaehntePersonen}{Personen: Samuel Fischer, Hugo von Hofmannsthal,  Luise von Österreich-Toskana, Leopold Ferdinand Salvator Wölfling}
         \renewcommand{\erwaehnteOrte}{Orte: Genf, Hotel du Cygne Montreux, Montreux, Wien}
         \renewcommand{\erwaehnteWerke}{}
               \section[Felix Salten an Arthur Schnitzler, 28. 12. 1902]{ Felix Salten an Arthur Schnitzler, 28. 12. 1902}\nopagebreak\mylabel{v}\rehead{ }\begin{ledgroupsized}[t]{13cm}\normalsize\beginnumbering \toendnotes[C]{\smallbreak\pagebreak[2]} \Standort{CUL, Schnitzler, B 89, A 2.}
\physDesc{Brief, 1 Blatt, 4 Seiten (Schwan im Prägedruck)
\newline{}Handschrift: Bleistift, lateinische Kurrent\newline{}Ordnung: mit Bleistift von unbekannter Hand nummeriert:
                                    »163« }\toendnotes[C]{\smallbreak}\pstart
           \noindent{}\raggedleft{}{\pb}\textcolor{gray}{\textbf{HOTEL DU CYGNE\oindex{Hotel du Cygne Montreux@\textbf{Hotel du Cygne Montreux}|pw}}}\pend
           \pstart
           \noindent{}\raggedleft{}\textcolor{gray}{\textbf{MONTREUX\oindex{Montreux@\textbf{Montreux}|pw}}}\pend
           \pstart
           \raggedleft{}28. XII. 02\pend
           \pstart
           Liebster, leider konnte ich Sie vor meiner Abreise nicht mehr
               sprechen. Nun hätte ich Ihnen inzwischen noch mehr zu sagen als früher. \pend
           \pstart
           Bin hier bei Erzh. Leopold\pwindex{Woelfling, Leopold Ferdinand Salvator 1868-12-02 – 1935-07-04@\textsc{Wölfling, Leopold Ferdinand Salvator} (1868-12-02 – 1935-07-04), \emph{Erzherzog}|pw} und fahre jetzt
               nach Genf\oindex{Genf@\textbf{Genf}|pw}, um den Nachmittag mit seiner Schwester\pwindex{Luise von Oesterreich-Toskana 1870-09-02 – 1947-03-23@\textsc{Luise von Österreich-Toskana} (1870-09-02 – 1947-03-23), \emph{Prinzessin >> Kronprinz}|pwv} zu verbringen.
               Reise \textcolor{gray}{Montag} nach Wien\oindex{Wien@\textbf{Wien}|pw} zurück,
               wo ich \substVorne{}\textsuperscript{Montag}{\allowbreak}\substDazwischen{}Dienstag\substHinten{}{ }{\pb}früh eintreffe. Vielleicht
               rufen Sie mich Mittag an, oder ich komme so zwischen 4 u 5 zu Ihnen, da es ja aus dem
               Caféhaus doch nichts wird. »Das Leben ist eine Rutschbahn« Könnte der Leop.\pwindex{Woelfling, Leopold Ferdinand Salvator 1868-12-02 – 1935-07-04@\textsc{Wölfling, Leopold Ferdinand Salvator} (1868-12-02 – 1935-07-04), \emph{Erzherzog}|pw} jetzt auch sagen. Er thut mir furchtbar
               leid. Hier ist’s übrigens bald Frühling. \pend
           \pstart
           Herzlichst Ihr {\\[\baselineskip]}\spacefill\mbox{Salten}\pend
           \leftskip=0em{}\pstart
           \noindent{}\label{T_L03338-1v}\edtext{Wenn \label{K_L03338-22v}\edtext{Hofmannsthal\pwindex{Hofmannsthal, Hugo von 1874-02-01 – 1929-07-15@\textsc{Hofmannsthal, Hugo von} (1874-02-01 – 1929-07-15), \emph{Schriftsteller}|pw} noch nicht gelesen}{\lemma{\textnormal{\emph{Hofmannsthal … gelesen}}}\Cendnote{\textnormal{siehe A. S.: \emph{Tagebuch}, 6. 1. 1903}}}\label{K_L03338-22h}\textcolor{gray}{hat}, bitte ich ihn auf mich zu warten. Schreibe ihm das
                     aber.}{\lemma{\textnormal{\emph{Wenn … aber.}}}\Cendnote{\textnormal{am oberen Seitenrand, quer
                     über die ersten beiden Seiten}}}\label{T_L03338-1h}\pend
           \pstart
           {\pb}Sollte S. Fischer\pwindex{Fischer, Samuel 24.12.1859 – 15.10.1934@\textsc{Fischer, Samuel} (24.12.1859 – 15.10.1934), \emph{Verleger}|pw} in Wien\oindex{Wien@\textbf{Wien}|pw}
                  sein, bitte ihm meine Abwesenheit entschuldigen. Habe ihn eingeladen und mußte
                  abreisen. Mittheilen konnte ich ihm nichts davon, weil ich ihn auf dem Weg nach
                     Wien\oindex{Wien@\textbf{Wien}|pw} wusste und eine {\pb}Wien\oindex{Wien@\textbf{Wien}|pw}er Adreße von ihm nicht hatte. \spacefill\mbox{S.
                     F}\pend
           
         
         \endnumbering\mylabel{h}\end{ledgroupsized}\begin{anhang}\end{anhang}\newcommand{\dateiname}{L03338}\newcommand{\titel}{Felix Salten an Arthur Schnitzler, 28. 12. 1902}\newcommand{\editorInnen}{Martin Anton Müller und Laura Untner}%% latex-leseansicht-abspann.tex
%% Abspann für die Leseansicht.
%% Der Schalter \ifkorrekturansicht ist bereits durch den Vorspann gesetzt.

%% latex-abspann.tex
%% Gemeinsamer Abspann für Korrekturansicht und Leseansicht.
%% Setzt den Schalter \ifkorrekturansicht voraus (gesetzt in den
%% einbindenden Dateien latex-korrekturansicht-abspann.tex bzw.
%% latex-leseansicht-abspann.tex).
%% ---------------------------------------------------------------

\normalsize

% Das esempio-Environment wird nur in der Leseansicht benötigt
\ifkorrekturansicht\else
\newenvironment{esempio}[3]%
{
    \vspace{1.5ex}
    \rlap{\underline{#1}}
    \par
    \setlength{\parindent}{0cm}
    \nopagebreak
    \leftskip=#2cm
    \rightskip=#3cm
}
{
    \par
}
\fi

\doendnotes{C}
\bigskip
\vfill

\clearpage

\footnotesize

\ifkorrekturansicht
  \lohead{\textsc{register}}
\fi

% theindex-Environment neu definieren ohne reledmac
\makeatletter
\renewenvironment{theindex}{%
  \ifkorrekturansicht
    \section*{\indexname}%
  \else
    \subsubsection*{Index der erwähnten Entitäten}%
  \fi
  \setlength{\parindent}{0pt}%
  \setlength{\parskip}{0pt plus 0.3pt}%
  \let\item\@idxitem
}{%
  \ifkorrekturansicht\clearpage\fi
}
\makeatother

\IfFileExists{\jobname-pw.ind}{\input{\jobname-pw.ind}}{}

% Quellenangabe nur in der Leseansicht
\ifkorrekturansicht\else
% Fallback-Definitionen, falls die .tex-Datei \titel etc. nicht gesetzt hat
\providecommand{\titel}{}
\providecommand{\editorInnen}{}
\providecommand{\dateiname}{\jobname}

\vspace{3cm}

\vfill

\footnotesize
\textsc{Quelle}: \titel. Herausgegeben von {\editorInnen}. In: \emph{Arthur Schnitzler: Briefwechsel mit Autorinnen und Autoren}.
 Digitale Edition, https://schnitzler-briefe.acdh.oeaw.ac.at/{\dateiname}.html (Stand \today)
\fi

\end{document}


      