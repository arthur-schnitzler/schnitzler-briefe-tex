%% latex-leseansicht-vorspann.tex
%% Vorspann für die Leseansicht.
%% Lädt die gemeinsame Datei latex-vorspann.tex mit nicht gesetztem Schalter.

\newif\ifkorrekturansicht
\korrekturansichtfalse

\input{../tex-inputs/latex-vorspann}


         
         \renewcommand{\erwaehntePersonen}{Personen: Wilhelmine Adamović, Samuel Fischer, André Giron, Hugo von Hofmannsthal,  Luise von Sachsen, Felix Salten, Leopold Ferdinand Salvator Wölfling}
         \renewcommand{\erwaehnteOrte}{Orte: Bern, Europa, Genf, Hotel du Cygne Montreux, Montreux, Schweiz, Wien}
         \renewcommand{\erwaehnteWerke}{Werke: Das gerettete Venedig. Trauerspiel in fünf Aufzügen, Erinnerungen, Tagebuch}
               \section[ Felix Salten an Arthur Schnitzler, 28. 12. 1902]{ Felix Salten an Arthur Schnitzler, 28. 12. 1902}\nopagebreak\mylabel{v}\rehead{ }\begin{ledgroupsized}[t]{13cm}\normalsize\beginnumbering\briefempfaengerindex{Schnitzler, Arthur@\textsc{Schnitzler, Arthur}!zzzSalten, Felix@\emph{von Felix Salten}!1902-12-282@{28. 12. 1902}|(be} \toendnotes[C]{\smallbreak\pagebreak[2]} \Standort{CUL, Schnitzler, B 89, A 2.}
\physDesc{Brief, 1 Blatt, 4 Seiten, 903 Zeichen (Schwan im Prägedruck)
\newline{}Handschrift: Bleistift, lateinische Kurrent
\newline{}Ordnung: mit Bleistift von unbekannter Hand nummeriert: »163« }\toendnotes[C]{\smallbreak}\pstart
           \noindent{}\raggedleft{}{\pb}\textcolor{gray}{\textbf{HOTEL DU CYGNE\oindex{Hotel du Cygne Montreux@\textbf{Hotel du Cygne Montreux}|pw}}}\pend
           \pstart
           \noindent{}\raggedleft{}\textcolor{gray}{\textbf{MONTREUX\oindex{Montreux@\textbf{Montreux}|pw}}}\pend
           \pstart
           \raggedleft{}28. XII. 02\pend
           \pstart
           Liebster leider konnte ich Sie vor meiner Abreise nicht mehr
               sprechen. Nun habe ich Ihnen inzwischen noch mehr zu sagen als früher.\pend
           \pstart
           Bin hier bei \label{K_L03338-1v}\edtext{Erzh. Leopold\pwindex{Woelfling, Leopold Ferdinand Salvator 1868-12-02 – 1935-07-04@\textsc{Wölfling, Leopold Ferdinand Salvator} (1868-12-02 – 1935-07-04), \emph{Erzherzog}|pw}}{\lemma{\textnormal{\emph{Erzh. Leopold}}}\Cendnote{\textnormal{Leopold Ferdinand von Österreich-Toskana\pwindex{Woelfling, Leopold Ferdinand Salvator 1868-12-02 – 1935-07-04@\textsc{Wölfling, Leopold Ferdinand Salvator} (1868-12-02 – 1935-07-04), \emph{Erzherzog}|pwk} war mit seiner
                  Haushälterin und zukünftigen Ehefrau, der ehemaligen Sexarbeiterin Wilhelmine Adamović\pwindex{Adamović, Wilhelmine 1877-05-01 – 1910-05-17@\textsc{Adamović, Wilhelmine} (1877-05-01 – 1910-05-17), \emph{Sexarbeiterin, Haushälterin}|pwk}, in die Schweiz\oindex{Schweiz@\textbf{Schweiz}|pwk} geflohen und aus dem Kaiserhaus
                  ausgetreten, um fortan als einfacher Bürger unter dem Namen Leopold Wölfling\pwindex{Woelfling, Leopold Ferdinand Salvator 1868-12-02 – 1935-07-04@\textsc{Wölfling, Leopold Ferdinand Salvator} (1868-12-02 – 1935-07-04), \emph{Erzherzog}|pwk} leben zu können. 
                  Zugleich hatte seine Schwester, die verheiratete Luise von Sachsen\pwindex{Luise von Sachsen 02.03.1870 – 23.03.1947@\textsc{Luise von Sachsen} (02.03.1870 – 23.03.1947), \emph{Erzherzogin, Kronprinzessin, Prinzessin}|pwk} ihren
                  Ehepartner verlassen, um mit dem Sprachlehrer André Giron\pwindex{Giron, Andre @\textsc{Giron, André}, \emph{Sprachlehrer}|pwk} ebenfalls in die 
                  Schweiz\oindex{Schweiz@\textbf{Schweiz}|pwk} zu fliehen.
                  In seinen \emph{Erinnerungen}\pwindex{Salten, Felix 06.09.1869 – 08.10.1945@\textsc{Salten, Felix} (06.09.1869 – 08.10.1945), \emph{Schriftsteller, Journalist, Chefredakteur}!Erinnerungen@\strich\emph{Erinnerungen}|pwk} schrieb Salten\pwindex{Salten, Felix 06.09.1869 – 08.10.1945@\textsc{Salten, Felix} (06.09.1869 – 08.10.1945), \emph{Schriftsteller, Journalist, Chefredakteur}|pwk}:
                  »Zu Weihnachten 1902 ging durch all europäischen\oindex{Europa@\textbf{Europa}|pw} Zeitungen die
                     Sensationsnachricht, Erzherzog Leopold Ferdinand\pwindex{Woelfling, Leopold Ferdinand Salvator 1868-12-02 – 1935-07-04@\textsc{Wölfling, Leopold Ferdinand Salvator} (1868-12-02 – 1935-07-04), \emph{Erzherzog}|pw} und seine Schwester
                     Kronprinzessin Luisa von Sachsen\pwindex{Luise von Sachsen 02.03.1870 – 23.03.1947@\textsc{Luise von Sachsen} (02.03.1870 – 23.03.1947), \emph{Erzherzogin, Kronprinzessin, Prinzessin}|pw} seien nach Genf\oindex{Genf@\textbf{Genf}|pw} durchgebrannt.
                     Leopold\pwindex{Woelfling, Leopold Ferdinand Salvator 1868-12-02 – 1935-07-04@\textsc{Wölfling, Leopold Ferdinand Salvator} (1868-12-02 – 1935-07-04), \emph{Erzherzog}|pw} telegraphierte mir, wenn ich Abschied nehmen wolle, wäre
                     ich ihm willkommen. Ich fuhr sofort nach Genf\oindex{Genf@\textbf{Genf}|pw}. In Bern\oindex{Bern@\textbf{Bern}|pw} während der
                        Zug hielt, wurde mir ein Telegramm gereicht, Leopold\pwindex{Woelfling, Leopold Ferdinand Salvator 1868-12-02 – 1935-07-04@\textsc{Wölfling, Leopold Ferdinand Salvator} (1868-12-02 – 1935-07-04), \emph{Erzherzog}|pw} bat mich nach
                        Montreux\oindex{Montreux@\textbf{Montreux}|pw} zu kommen. Mit meiner Redaktion vereinbarte ich einen
                     Code und in Montreux\oindex{Montreux@\textbf{Montreux}|pw} erwartete mich Leopold\pwindex{Woelfling, Leopold Ferdinand Salvator 1868-12-02 – 1935-07-04@\textsc{Wölfling, Leopold Ferdinand Salvator} (1868-12-02 – 1935-07-04), \emph{Erzherzog}|pw} auf dem Bahnhof.
                     ›Was sagen sie zu dem Wirbel, den wir gemacht haben?‹ Die Zeitungen hatten gemeldet, Leopold\pwindex{Woelfling, Leopold Ferdinand Salvator 1868-12-02 – 1935-07-04@\textsc{Wölfling, Leopold Ferdinand Salvator} (1868-12-02 – 1935-07-04), \emph{Erzherzog}|pw} sei mit seiner Geliebten Adamowitsch\pwindex{Adamović, Wilhelmine 1877-05-01 – 1910-05-17@\textsc{Adamović, Wilhelmine} (1877-05-01 – 1910-05-17), \emph{Sexarbeiterin, Haushälterin}|pw}
                     durchgebrannt, die Kronprinzessin\pwindex{Luise von Sachsen 02.03.1870 – 23.03.1947@\textsc{Luise von Sachsen} (02.03.1870 – 23.03.1947), \emph{Erzherzogin, Kronprinzessin, Prinzessin}|pwv}, Leopolds\pwindex{Woelfling, Leopold Ferdinand Salvator 1868-12-02 – 1935-07-04@\textsc{Wölfling, Leopold Ferdinand Salvator} (1868-12-02 – 1935-07-04), \emph{Erzherzog}|pw} Schwester mit ihrem
                        Schatz Giron\pwindex{Giron, Andre @\textsc{Giron, André}, \emph{Sprachlehrer}|pw}.« (\emph{Wienbibliothek im Rathaus}, Nachlass
                           Salten\pwindex{Salten, Felix 06.09.1869 – 08.10.1945@\textsc{Salten, Felix} (06.09.1869 – 08.10.1945), \emph{Schriftsteller, Journalist, Chefredakteur}|pwk}, ZPH 1681/1 1.1.1.9.1, [S. 11]) 
                  Vgl. Paul Goldmann an Arthur Schnitzler, 22. 3. [1900].
               }}}\label{K_L03338-1h} und fahre jetzt nach Genf\oindex{Genf@\textbf{Genf}|pw} um den
                  \label{K_L03338-2v}\edtext{Nachmittag mit seiner Schwester\pwindex{Luise von Sachsen 02.03.1870 – 23.03.1947@\textsc{Luise von Sachsen} (02.03.1870 – 23.03.1947), \emph{Erzherzogin, Kronprinzessin, Prinzessin}|pwv}}{\lemma{\textnormal{\emph{Nachmittag … Schwester}}}\Cendnote{\textnormal{Diesen Besuch schildert Salten\pwindex{Salten, Felix 06.09.1869 – 08.10.1945@\textsc{Salten, Felix} (06.09.1869 – 08.10.1945), \emph{Schriftsteller, Journalist, Chefredakteur}|pwk}
                         in seinen \emph{Erinnerungen}\pwindex{Salten, Felix 06.09.1869 – 08.10.1945@\textsc{Salten, Felix} (06.09.1869 – 08.10.1945), \emph{Schriftsteller, Journalist, Chefredakteur}!Erinnerungen@\strich\emph{Erinnerungen}|pwk}.}}}\label{K_L03338-2h} zu verbringen\pend
           \pstart
           Reise \textcolor{gray}{morgen} nach Wien\oindex{Wien@\textbf{Wien}|pw} zurück, wo ich \substVorne{}\textsuperscript{Montag}{\allowbreak}\substDazwischen{}Dienstag\substHinten{}{ }{\pb}früh eintreffe. Vielleicht rufen Sie mich \textcolor{gray}{V.}
                  Mittg an, oder ich \label{K_L03338-3v}\edtext{\textcolor{gray}{komme} so zwischen 4 {\kaufmannsund} 5
               zu Ihnen}{\lemma{\textnormal{\emph{komme … Ihnen}}}\Cendnote{\textnormal{Zu einem Treffen kam es erst am
                     2. 1. 1903.}}}\label{K_L03338-3h}, da es ja aus dem Cafébesuch von mir nichts wird. »Das Leben ist eine
               Rutschbahn« könnte der Leop.\pwindex{Woelfling, Leopold Ferdinand Salvator 1868-12-02 – 1935-07-04@\textsc{Wölfling, Leopold Ferdinand Salvator} (1868-12-02 – 1935-07-04), \emph{Erzherzog}|pw} jetzt auch sagen.
               Er thut mir furchtbar leid. Hier ist’s übrigens bald
                  Frühling\textcolor{gray}{.}\pend
           \pstart
           Herzlichst Ihr {\\[\baselineskip]}\spacefill\mbox{Salten}\pend
           \leftskip=0em{}\pstart
           \noindent{}\label{T_L03338-1v}\edtext{Wenn \label{K_L03338-4v}\edtext{Hofmannsthal\pwindex{Hofmannsthal, Hugo von 1874-02-01 – 1929-07-15@\textsc{Hofmannsthal, Hugo von} (1874-02-01 – 1929-07-15), \emph{Schriftsteller}|pw} noch nicht gelesen\pwindex{Hofmannsthal, Hugo von 1874-02-01 – 1929-07-15@\textsc{Hofmannsthal, Hugo von} (1874-02-01 – 1929-07-15), \emph{Schriftsteller}!gerettete Venedig. Trauerspiel in fuenf Aufzuegen1905@\strich\emph{Das gerettete Venedig. Trauerspiel in fünf Aufzügen} {[}1905{]}|pwv}}{\lemma{\textnormal{\emph{Hofmannsthal … gelesen}}}\Cendnote{\textnormal{Siehe A. S.: \emph{Tagebuch}, 6. 1. 1903.
                  }}}\label{K_L03338-4h}{ }hat, bitte ich ihn auf mich zu warten.
                     \textcolor{gray}{S}chreibe ihm das aber.}{\lemma{\textnormal{\emph{Wenn … aber.}}}\Cendnote{\textnormal{am oberen Seitenrand, quer zum Text über die ersten beiden
                     Seiten}}}\label{T_L03338-1h}\pend
           \pstart
           {\pb}Sollte \label{K_L03338-5v}\edtext{S.
                     Fischer\pwindex{Fischer, Samuel 24.12.1859 – 15.10.1934@\textsc{Fischer, Samuel} (24.12.1859 – 15.10.1934), \emph{Verleger}|pw} in Wien\oindex{Wien@\textbf{Wien}|pw}}{\lemma{\textnormal{\emph{S.
                     Fischer in Wien}}}\Cendnote{\textnormal{Zumindest im \emph{Tagebuch}\pwindex{\textcolor{red}{\textsuperscript{XXXX1 indx}}!Tagebuch1981 – 2000@\strich\emph{Tagebuch} {[}Hrsg., 1981 – 2000{]}|pwk}{ }Schnitzlers\pwindex{Schnitzler, Arthur 15.05.1862 – 21.10.1931@\textsc{Schnitzler, Arthur} (15.05.1862 – 21.10.1931), \emph{Schriftsteller, Mediziner}|pwk} ist in diesen Tagen keine
                     Anwesenheit Fischers\pwindex{Fischer, Samuel 24.12.1859 – 15.10.1934@\textsc{Fischer, Samuel} (24.12.1859 – 15.10.1934), \emph{Verleger}|pwk} in Wien\oindex{Wien@\textbf{Wien}|pwk} erwähnt.}}}\label{K_L03338-5h} sein, bitte ihm meine Abwesenheit
                  entschuldigen.\pend
           \pstart
           habe ihn eingeladen und mußte abreisen. Mittheilen konnte ich ihm nichts davon,
                  weil ich ihn auf dem Weg nach Wien\oindex{Wien@\textbf{Wien}|pw} glaubte und
                  eine {\pb}Wien\oindex{Wien@\textbf{Wien}|pw}er Adreße von ihm nicht hatte. {\\}\spacefill\mbox{F. S}\pend
           
         
         \endnumbering\mylabel{h}\end{ledgroupsized}  \newcommand{\dateiname}{L03338}\newcommand{\titel}{Felix Salten an Arthur Schnitzler, 28. 12. 1902}\newcommand{\editorInnen}{Martin Anton Müller und Laura Untner}%% latex-leseansicht-abspann.tex
%% Abspann für die Leseansicht.
%% Der Schalter \ifkorrekturansicht ist bereits durch den Vorspann gesetzt.

%% latex-abspann.tex
%% Gemeinsamer Abspann für Korrekturansicht und Leseansicht.
%% Setzt den Schalter \ifkorrekturansicht voraus (gesetzt in den
%% einbindenden Dateien latex-korrekturansicht-abspann.tex bzw.
%% latex-leseansicht-abspann.tex).
%% ---------------------------------------------------------------

\normalsize

% Das esempio-Environment wird nur in der Leseansicht benötigt
\ifkorrekturansicht\else
\newenvironment{esempio}[3]%
{
    \vspace{1.5ex}
    \rlap{\underline{#1}}
    \par
    \setlength{\parindent}{0cm}
    \nopagebreak
    \leftskip=#2cm
    \rightskip=#3cm
}
{
    \par
}
\fi

\doendnotes{C}
\bigskip
\vfill

\clearpage

\footnotesize

\ifkorrekturansicht
  \lohead{\textsc{register}}
\fi

% theindex-Environment neu definieren ohne reledmac
\makeatletter
\renewenvironment{theindex}{%
  \ifkorrekturansicht
    \section*{\indexname}%
  \else
    \subsubsection*{Index der erwähnten Entitäten}%
  \fi
  \setlength{\parindent}{0pt}%
  \setlength{\parskip}{0pt plus 0.3pt}%
  \let\item\@idxitem
}{%
  \ifkorrekturansicht\clearpage\fi
}
\makeatother

\IfFileExists{\jobname-pw.ind}{\input{\jobname-pw.ind}}{}

% Quellenangabe nur in der Leseansicht
\ifkorrekturansicht\else
% Fallback-Definitionen, falls die .tex-Datei \titel etc. nicht gesetzt hat
\providecommand{\titel}{}
\providecommand{\editorInnen}{}
\providecommand{\dateiname}{\jobname}

\vspace{3cm}

\vfill

\footnotesize
\textsc{Quelle}: \titel. Herausgegeben von {\editorInnen}. In: \emph{Arthur Schnitzler: Briefwechsel mit Autorinnen und Autoren}.
 Digitale Edition, https://schnitzler-briefe.acdh.oeaw.ac.at/{\dateiname}.html (Stand \today)
\fi

\end{document}


      