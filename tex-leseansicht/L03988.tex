%% latex-leseansicht-vorspann.tex
%% Vorspann für die Leseansicht.
%% Lädt die gemeinsame Datei latex-vorspann.tex mit nicht gesetztem Schalter.

\newif\ifkorrekturansicht
\korrekturansichtfalse

\input{../tex-inputs/latex-vorspann}


\section[Felix Salten: Widmungsexemplar von Gestalten und Erscheinungen an Olga Schnitzler, Oktober 1913]{L03988 Felix Salten: Widmungsexemplar von Gestalten und Erscheinungen an Olga
               Schnitzler, Oktober 1913}
\nopagebreak\mylabel{L03988v}
\rehead{ }\normalsize\beginnumbering\briefempfaengerindex{Schnitzler, Olga@\textsc{Schnitzler, Olga}!zzzSalten, Felix@\emph{von Felix Salten}!1913-01-301@{Oktober 1913}|(be}
\toendnotes[C]{\smallbreak\pagebreak[2]}
\correspDesc{Versand  durch Felix Salten im Zeitraum Oktober 1913 in Wien
\newline{}Erhalt  durch Olga Schnitzler im Zeitraum Oktober 1913 in Wien}\toendnotes[C]{\smallbreak}
\Standort{Wien, Österreichische Nationalbibliothek, Lit-1,918.010-B.}
\physDesc{Widmung am Titelblatt, 100 Zeichen
\newline{}Handschrift: schwarze Tinte, lateinische Kurrent}\toendnotes[C]{\smallbreak}
\pstart
           \noindent{}\centering{}{\pb}\textcolor{gray}{\textbf{GESTALTEN\pwindex{Salten, Felix 6.\,9.\,1869 Budapest – 8.\,10.\,1945 Zürich@\textsc{Salten, Felix} (6.\,9.\,1869 Budapest – 8.\,10.\,1945 Zürich), \emph{Schriftsteller, Journalist, Chefredakteur}!Gestalten und Erscheinungen@\strich\emph{Gestalten und Erscheinungen}|pw}}}\pend
           
\pstart
           \centering{}\textcolor{gray}{\textbf{UND\pwindex{Salten, Felix 6.\,9.\,1869 Budapest – 8.\,10.\,1945 Zürich@\textsc{Salten, Felix} (6.\,9.\,1869 Budapest – 8.\,10.\,1945 Zürich), \emph{Schriftsteller, Journalist, Chefredakteur}!Gestalten und Erscheinungen@\strich\emph{Gestalten und Erscheinungen}|pw}}}\pend
           
\pstart
           \centering{}\textcolor{gray}{\textbf{ERSCHEINUNGEN\pwindex{Salten, Felix 6.\,9.\,1869 Budapest – 8.\,10.\,1945 Zürich@\textsc{Salten, Felix} (6.\,9.\,1869 Budapest – 8.\,10.\,1945 Zürich), \emph{Schriftsteller, Journalist, Chefredakteur}!Gestalten und Erscheinungen@\strich\emph{Gestalten und Erscheinungen}|pw}}}\pend
           {\vspace{1\baselineskip}}
\pstart
           \raggedleft{}\label{K_L03988-1v}\edtext{Olga Schnitzler}{\lemma{\textnormal{\emph{Olga Schnitzler}}}\Cendnote{\textnormal{Ein Grund dafür, dass Salten\pwindex{Salten, Felix 6.\,9.\,1869 Budapest – 8.\,10.\,1945 Zürich@\textsc{Salten, Felix} (6.\,9.\,1869 Budapest – 8.\,10.\,1945 Zürich), \emph{Schriftsteller, Journalist, Chefredakteur}|pwk} das Buch nicht Schnitzler selbst widmete, könnte daran liegen, dass es einen Aufsatz\pwindex{Salten, Felix 6.\,9.\,1869 Budapest – 8.\,10.\,1945 Zürich@\textsc{Salten, Felix} (6.\,9.\,1869 Budapest – 8.\,10.\,1945 Zürich), \emph{Schriftsteller, Journalist, Chefredakteur}!Artur Schnitzler. (Zum 50. Geburtstag)@\strich\emph{Artur Schnitzler. (Zum 50. Geburtstag)}|pwkv} über diesen enthält. Der Aufsatz
                  war zuerst anlässlich von Schnitzlers
                  50. Geburtstag im \emph{Fremden-Blatt}\pwindex{Fremden-Blatt@\emph{Fremden-Blatt}|pwk}
                  erschienen.}}}\label{K_L03988-1}\pend
           
\pstart
           \raggedleft{}in aufrichtiger Verehrung und herzlicher Freundschaft\pend
           
\pstart
           \raggedleft{}Felix Salten\pend
           
\pstart
           \raggedleft{}Wien\oindex{Wien@\textbf{Wien}, \emph{Verwaltungsgebiet}|pw}, Oktober 1913\pend
           {\vspace{1\baselineskip}}
\pstart
           \centering{}\textcolor{gray}{\textbf{VON}}\pend
           
\pstart
           \centering{}\textcolor{gray}{\textbf{FELIX SALTEN}}\pend
           {\vspace{1\baselineskip}}
\pstart
           \centering{}\textcolor{gray}{\textbf{1913}}\pend
           
\pstart
           \centering{}\textcolor{gray}{\textbf{S. FISCHER ⋅ VERLAG\orgindex{S. Fischer Verlag@S. Fischer Verlag|pw} ⋅ BERLIN\oindex{Berlin@\textbf{Berlin}, \emph{Hauptstadt}|pw}}}\pend
           \selectlanguage{ngerman}\endnumbering\briefempfaengerindex{Schnitzler, Olga@\textsc{Schnitzler, Olga}!zzzSalten, Felix@\emph{von Felix Salten}!1913-01-011@{Oktober 1913}|)be}\mylabel{L03988h}
\begin{anhang}
\end{anhang}\newcommand{\dateiname}{L03988}\newcommand{\titel}{Felix Salten: Widmungsexemplar von Gestalten und Erscheinungen an Olga Schnitzler, Oktober 1913}\newcommand{\editorInnen}{Selma Jahnke und Martin Anton Müller}%% latex-leseansicht-abspann.tex
%% Abspann für die Leseansicht.
%% Der Schalter \ifkorrekturansicht ist bereits durch den Vorspann gesetzt.

%% latex-abspann.tex
%% Gemeinsamer Abspann für Korrekturansicht und Leseansicht.
%% Setzt den Schalter \ifkorrekturansicht voraus (gesetzt in den
%% einbindenden Dateien latex-korrekturansicht-abspann.tex bzw.
%% latex-leseansicht-abspann.tex).
%% ---------------------------------------------------------------

\normalsize

% Das esempio-Environment wird nur in der Leseansicht benötigt
\ifkorrekturansicht\else
\newenvironment{esempio}[3]%
{
    \vspace{1.5ex}
    \rlap{\underline{#1}}
    \par
    \setlength{\parindent}{0cm}
    \nopagebreak
    \leftskip=#2cm
    \rightskip=#3cm
}
{
    \par
}
\fi

\doendnotes{C}
\bigskip
\vfill

\clearpage

\footnotesize

\ifkorrekturansicht
  \lohead{\textsc{register}}
\fi

% theindex-Environment neu definieren ohne reledmac
\makeatletter
\renewenvironment{theindex}{%
  \ifkorrekturansicht
    \section*{\indexname}%
  \else
    \subsubsection*{Index der erwähnten Entitäten}%
  \fi
  \setlength{\parindent}{0pt}%
  \setlength{\parskip}{0pt plus 0.3pt}%
  \let\item\@idxitem
}{%
  \ifkorrekturansicht\clearpage\fi
}
\makeatother

\IfFileExists{\jobname-pw.ind}{\input{\jobname-pw.ind}}{}

% Quellenangabe nur in der Leseansicht
\ifkorrekturansicht\else
% Fallback-Definitionen, falls die .tex-Datei \titel etc. nicht gesetzt hat
\providecommand{\titel}{}
\providecommand{\editorInnen}{}
\providecommand{\dateiname}{\jobname}

\vspace{3cm}

\vfill

\footnotesize
\textsc{Quelle}: \titel. Herausgegeben von {\editorInnen}. In: \emph{Arthur Schnitzler: Briefwechsel mit Autorinnen und Autoren}.
 Digitale Edition, https://schnitzler-briefe.acdh.oeaw.ac.at/{\dateiname}.html (Stand \today)
\fi

\end{document}


