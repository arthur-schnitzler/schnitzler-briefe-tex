%% latex-korrekturansicht-vorspann.tex
%% Vorspann für die Korrekturansicht.
%% Lädt die gemeinsame Datei latex-vorspann.tex mit gesetztem Schalter.

\newif\ifkorrekturansicht
\korrekturansichttrue

\input{../tex-inputs/latex-vorspann}


\section[Arthur Schnitzler an Richard Beer-Hofmann, {[}26. 3. 1895{]}]{L00425 Arthur Schnitzler an Richard Beer-Hofmann, {[}26. 3. 1895{]}}
\nopagebreak\mylabel{L00425v}
\rehead{ }\normalsize\beginnumbering\briefempfaengerindex{Beer-Hofmann, Richard@\textsc{Beer-Hofmann, Richard}!zzzSchnitzler, Arthur@\emph{von Arthur Schnitzler}!1895-03-261@{{[}26. 3. 1895{]}}|(be}
\toendnotes[C]{\smallbreak\pagebreak[2]}\Standort{YCGL, MSS 31.}
\physDesc{Brief, 1 Blatt, 4 Seiten, Umschlag, 895 Zeichen
\newline{}Handschrift: Bleistift, deutsche Kurrent
\newline{}Versand: ohne postalischen Übermittlungsvermerk }
\buchAbdrucke{\weitereDrucke{Arthur Schnitzler, Richard Beer-Hofmann: \emph{Briefwechsel 1891–1931}. Wien, Zürich: \emph{Europaverlag} 1992, S. 71–72.} }\toendnotes[C]{\smallbreak}\pstart{}{\pb}Herrn Dr. \textsc{Richard
                     Beer-Hofmann}\pend{}\pstart{}Wien\oindex{Wien@\textbf{Wien}, \emph{A.ADM2}|pw}\pend{}\pstart{}\textsc{I. Wollzeile 15\oindex{Wollzeile@\textbf{Wollzeile}, \emph{Straße (K.STR)}|pw}}, 4. Stock.\pend{}{\bigskip}\vspace{1em}
\pstart\center{}{\pb}Lieber Richard.\pend\vspace{0.5em}
\pstart
           1) Ich habe noch nichts zu \textsc{Faust}\pwindex{Faust. Eine Tragoedie@\emph{Faust. Eine Tragödie}|pw}, da ich den beſtechlichen nicht fand; ich zweifle aber nicht, daſs ich morgen
               Vormittag welche beko{\geminationm}en werde, reflectiren Sie denn
               drauf? Und,\pend
           
\pstart
           2.) we{\geminationn} ich keine bekomm, wollen Sie mit mir morgen in
               ein andres Theater (»Karlsſchülerin\pwindex{Karlsschuelerin@\emph{Die Karlsschülerin}|pw}« oder »Touriſten\pwindex{Wiener Touristen@\emph{Wiener Touristen}|pw}«) gehn?\pend
           
\pstart
           3.) \textsc{Herzl}\pwindex{Herzl, Theodor 1860-05-02 – 1904-07-03@\textsc{Herzl, Theodor} (1860-05-02 – 1904-07-03), \emph{Schriftsteller/Schriftstellerin, Journalist/Journalistin}|pw} iſt da, möchte mit uns, {\pb}dh. Ihnen, \textsc{Hugo}\pwindex{Hofmannsthal, Hugo von 1874-02-01 – 1929-07-15@\textsc{Hofmannsthal, Hugo von} (1874-02-01 – 1929-07-15), \emph{Schriftsteller/Schriftstellerin}|pw}, mir, eventuell Bahr\pwindex{Bahr, Hermann 19.07.1863 – 15.01.1934@\textsc{Bahr, Hermann} (19.07.1863 – 15.01.1934), \emph{Schriftsteller/Schriftstellerin, Kritiker/Kritikerin}|pw}{ }ſoupiren. Ich ſagte ihm, Freitag nach dem \textsc{Hubermann}\pwindex{Huberman, Bronisław 19.12.1882 – 16.6.1947@\textsc{Huberman, Bronisław} (19.12.1882 – 16.6.1947), \emph{Schriftsteller/Schriftstellerin, Musiker/Musikerin, Violinist/Violinistin}|pw}\textsc{concert} – Sie ſind doch einverſtanden? Zu \textsc{Bahr}\pwindex{Bahr, Hermann 19.07.1863 – 15.01.1934@\textsc{Bahr, Hermann} (19.07.1863 – 15.01.1934), \emph{Schriftsteller/Schriftstellerin, Kritiker/Kritikerin}|pw}{ }ſagen Sie vorläufig nichts, weil ich noch ein
               definitives Wort von \textsc{Herzl}\pwindex{Herzl, Theodor 1860-05-02 – 1904-07-03@\textsc{Herzl, Theodor} (1860-05-02 – 1904-07-03), \emph{Schriftsteller/Schriftstellerin, Journalist/Journalistin}|pw} erwarte. \textsc{Hugo}\pwindex{Hofmannsthal, Hugo von 1874-02-01 – 1929-07-15@\textsc{Hofmannsthal, Hugo von} (1874-02-01 – 1929-07-15), \emph{Schriftsteller/Schriftstellerin}|pw} theilen Sie’s vielleicht mit?\pend
           
\pstart
           4.) bitte kaufen Sie \textsc{vis à vis}{ }{\pb}bei \textsc{Goldschmidt}\orgindex{Hermann Goldschmiedt und Co.@Hermann Goldschmiedt {\kaufmannsund}  Co.|pw} die Münchner Allgemeine\pwindex{Allgemeine Zeitung@\emph{Allgemeine Zeitung}|pw} von Samstag
                  den 23. d. mit \label{K_L00425-1v}\edtext{Beilage}{\lemma{\textnormal{\emph{Beilage}}}\Cendnote{\textnormal{Wohl wegen: b. m.\pwindex{b.m. 1895 – 1895@\textsc{b.m.} (1895 – 1895), \emph{Journalist/Journalistin}|pwk}: \emph{Arthur Schnitzler: Sterben}\pwindex{Arthur Schnitzler: Sterben@\emph{Arthur Schnitzler: Sterben}|pwk}. In: \emph{Beilage zur Allgemeinen Zeitung}\pwindex{Allgemeine Zeitung@\emph{Allgemeine Zeitung}|pwk}, Beilage-Nr. 69,
                        23. 3. 1895, S. 5.
               }}}\label{K_L00425-1} für mich.\pend
           
\pstart
           5.) hier iſt \label{K_L00425-2v}\edtext{\textsc{Carlos\pwindex{Don Karlos, Infant von Spanien@\emph{Don Karlos, Infant von Spanien}|pw}{ }Schnabl\pwindex{Schnabel, C. @\textsc{Schnabel, C.}, \emph{Herausgeber/Herausgeberin, Lehrer/Lehrerin}|pw}}}{\lemma{\textnormal{\emph{Carlos Schnabl}}}\Cendnote{\textnormal{vermutlich die Edition: \emph{Don Carlos, Infant von Spanien. Ein
                        dramatisches Gedicht. Zum Uebersetzen aus dem Deutschen in das Französische
                        für bereits vorgerückte Schüler, die in den Geist der beiden Idiome tiefer
                        eindringen und die Conversationssprache sich aneignen wollen. Mit
                        Anmerkungen der nöthigen Phraseologie und einem Wörterbuche. Zum Schul- und
                        Privatgebrauch}\pwindex{Don Carlos, Infant von Spanien. Ein dramatisches Gedicht. Zum Uebersetzen aus dem Deutschen in das Franzoesische fuer bereits vorgerueckte Schueler@\emph{Don Carlos, Infant von Spanien. Ein dramatisches Gedicht. Zum Uebersetzen aus dem Deutschen in das Französische für bereits vorgerückte Schüler}|pwk}. Herausgegeben von C.
                        Schnabel\pwindex{Schnabel, C. @\textsc{Schnabel, C.}, \emph{Herausgeber/Herausgeberin, Lehrer/Lehrerin}|pwk}, öffentlicher Lehrer. Leipzig: \emph{Baumgärtner’sche Buchhandlung}\orgindex{Baumgaertnersche Buchhandlung@Baumgärtnersche Buchhandlung|pwk}{ }1846.}}}\label{K_L00425-2}.\pend
           
\pstart
           6.) vielleicht – ſo jetzt haben Sie mir telephonirt, alſo es bleibt dabei, {\pb}wir treffen uns im \textsc{Griensteidl}\oindex{Cafe Griensteidl@\textbf{Café Griensteidl}, \emph{Kaffeehaus (K.KAF)}|pw} gegen 8. Herzlich\pend
           \pstart Ihr \spacefill\mbox{Arth}\pend{}\selectlanguage{ngerman}\endnumbering\briefempfaengerindex{Beer-Hofmann, Richard@\textsc{Beer-Hofmann, Richard}!zzzSchnitzler, Arthur@\emph{von Arthur Schnitzler}!1895-03-261@{{[}26. 3. 1895{]}}|)be}\mylabel{L00425h}  \normalsize

\doendnotes{C}
\bigskip
\vfill

\clearpage

\footnotesize

\lohead{\textsc{register}}

% Definiere theindex-Environment komplett neu ohne reledmac
\makeatletter
\renewenvironment{theindex}{%
  \section*{\indexname}%
  \setlength{\parindent}{0pt}%
  \setlength{\parskip}{0pt plus 0.3pt}%
  \let\item\@idxitem
}{%
  \clearpage
}
\makeatother

\IfFileExists{\jobname-pw.ind}{\input{\jobname-pw.ind}}{}

\end{document}

      