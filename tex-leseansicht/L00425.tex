%% latex-leseansicht-vorspann.tex
%% Vorspann für die Leseansicht.
%% Lädt die gemeinsame Datei latex-vorspann.tex mit nicht gesetztem Schalter.

\newif\ifkorrekturansicht
\korrekturansichtfalse

\input{../tex-inputs/latex-vorspann}


\section[Arthur Schnitzler an Richard Beer-Hofmann, {[}26. 3. 1895{]}]{L00425 Arthur Schnitzler an Richard Beer-Hofmann, {[}26. 3. 1895{]}}
\nopagebreak\mylabel{L00425v}
\rehead{ }\normalsize\beginnumbering\briefempfaengerindex{Beer-Hofmann, Richard@\textsc{Beer-Hofmann, Richard}!zzzSchnitzler, Arthur@\emph{von Arthur Schnitzler}!1895-03-261@{{[}26. 3. 1895{]}}|(be}
\toendnotes[C]{\smallbreak\pagebreak[2]}
\correspDesc{Versand  durch Arthur Schnitzler am [26. 3. 1895] in Wien
\newline{}Erhalt  durch Richard Beer-Hofmann im Zeitraum [26. 3. 1895
                  – 30. 3. 1895?] in Wien}\toendnotes[C]{\smallbreak}
\Standort{YCGL, MSS 31.}
\physDesc{Brief, 1 Blatt, 4 Seiten, Kuvert, 895 Zeichen
\newline{}Handschrift: Bleistift, deutsche Kurrent
\newline{}Versand: ohne postalischen Übermittlungsvermerk }
\buchAbdrucke{\weitereDrucke{Arthur Schnitzler, Richard Beer-Hofmann: \emph{Briefwechsel 1891–1931}. Herausgegeben von Konstanze Fliedl. Wien, Zürich: \emph{Europaverlag} 1992, S. 71–72.} }\toendnotes[C]{\smallbreak}\pstart{}{\pb}Herrn Dr. \textsc{Richard
                     Beer-Hofmann}\pend{}\pstart{}Wien\oindex{Wien@\textbf{Wien}, \emph{Verwaltungsgebiet}|pw}\pend{}\pstart{}\textsc{I. Wollzeile 15\oindex{Wien@\textbf{Wien}!I., Innere Stadt@\textbf{I., Innere Stadt}!Wollzeile 15 (»Berthahof«)@\textbf{Wollzeile 15 (»Berthahof«)}, \emph{Wohngebäude}|pw}}, 4. Stock.\pend{}{\bigskip}\vspace{1em}
\pstart\center{}{\pb}Lieber Richard.\pend\vspace{0.5em}
\pstart
           1) Ich habe noch nichts zu \textsc{Faust}\pwindex{\textcolor{red}{\textsuperscript{XXXX indx1}}!Faust. Eine Tragödie@\strich\emph{Faust. Eine Tragödie}|pw}, da ich den beſtechlichen nicht fand; ich zweifle aber nicht, daſs ich morgen
               Vormittag welche beko{\geminationm}en werde, reflectiren Sie denn
               drauf? Und,\pend
           
\pstart
           2.) we{\geminationn} ich keine bekomm, wollen Sie mit mir morgen in
               ein andres Theater (»Karlsſchülerin\pwindex{\textcolor{red}{\textsuperscript{XXXX indx1}}!Karlsschülerin@\strich\emph{Die Karlsschülerin}|pw}« oder »Touriſten\pwindex{\textcolor{red}{\textsuperscript{XXXX indx1}}!Wiener Touristen@\strich\emph{Wiener Touristen}|pw}\pwindex{\textcolor{red}{\textsuperscript{XXXX indx1}}!Wiener Touristen@\strich\emph{Wiener Touristen}|pw}«) gehn?\pend
           
\pstart
           3.) \textsc{Herzl}\pwindex{Herzl, Theodor 2.\,5.\,1860 Budapest – 3.\,7.\,1904 Edlach@\textsc{Herzl, Theodor} (2.\,5.\,1860 Budapest – 3.\,7.\,1904 Edlach), \emph{Schriftsteller, Journalist}|pw} iſt da, möchte mit uns, {\pb}dh. Ihnen, \textsc{Hugo}\pwindex{Hofmannsthal, Hugo von 1.\,2.\,1874 Wien – 15.\,7.\,1929 Rodaun@\textsc{Hofmannsthal, Hugo von} (1.\,2.\,1874 Wien – 15.\,7.\,1929 Rodaun), \emph{Schriftsteller}|pw}, mir, eventuell Bahr\pwindex{Bahr, Hermann 19.\,7.\,1863 Linz – 15.\,1.\,1934 München@\textsc{Bahr, Hermann} (19.\,7.\,1863 Linz – 15.\,1.\,1934 München), \emph{Schriftsteller, Kritiker}|pw}{ }ſoupiren. Ich{ }ſagte ihm, Freitag nach dem \textsc{Hubermann}\pwindex{Huberman, Bronisław 19.\,12.\,1882 Czenstochau – 16.\,6.\,1947 Corsier-sur-Vevey@\textsc{Huberman, Bronisław} (19.\,12.\,1882 Czenstochau – 16.\,6.\,1947 Corsier-sur-Vevey), \emph{Schriftsteller, Musiker, Violinist}|pw}\textsc{concert} – Sie{ }ſind doch einverſtanden? Zu \textsc{Bahr}\pwindex{Bahr, Hermann 19.\,7.\,1863 Linz – 15.\,1.\,1934 München@\textsc{Bahr, Hermann} (19.\,7.\,1863 Linz – 15.\,1.\,1934 München), \emph{Schriftsteller, Kritiker}|pw}{ }ſagen Sie vorläufig nichts, weil ich noch ein
               definitives Wort von \textsc{Herzl}\pwindex{Herzl, Theodor 2.\,5.\,1860 Budapest – 3.\,7.\,1904 Edlach@\textsc{Herzl, Theodor} (2.\,5.\,1860 Budapest – 3.\,7.\,1904 Edlach), \emph{Schriftsteller, Journalist}|pw} erwarte. \textsc{Hugo}\pwindex{Hofmannsthal, Hugo von 1.\,2.\,1874 Wien – 15.\,7.\,1929 Rodaun@\textsc{Hofmannsthal, Hugo von} (1.\,2.\,1874 Wien – 15.\,7.\,1929 Rodaun), \emph{Schriftsteller}|pw} theilen Sie’s vielleicht mit?\pend
           
\pstart
           4.) bitte kaufen Sie \textsc{vis à vis}{ }{\pb}bei \textsc{Goldschmidt}\orgindex{Hermann Goldschmiedt und Co.@Hermann Goldschmiedt {\kaufmannsund}  Co.|pw} die Münchner Allgemeine\pwindex{Allgemeine Zeitung@\emph{Allgemeine Zeitung}|pw} von Samstag den 23. d. mit \label{K_L00425-1v}\edtext{Beilage}{\lemma{\textnormal{\emph{Beilage}}}\Cendnote{\textnormal{Wohl wegen: b. m.\pwindex{b.m. 1895 – 1895@\textsc{b.m.} (1895 – 1895), \emph{Journalist/Journalistin}|pwk}: \emph{Arthur Schnitzler: Sterben}\pwindex{b.m. 1895 – 1895@\textsc{b.m.} (1895 – 1895), \emph{Journalist/Journalistin}!Arthur Schnitzler: Sterben@\strich\emph{Arthur Schnitzler: Sterben}|pwk}. In: \emph{Beilage zur Allgemeinen Zeitung}\pwindex{Allgemeine Zeitung@\emph{Allgemeine Zeitung}|pwk}, Beilage-Nr. 69,
                        23. 3. 1895, S. 5.
               }}}\label{K_L00425-1} für mich.\pend
           
\pstart
           5.) hier iſt \label{K_L00425-2v}\edtext{\textsc{Carlos\pwindex{\textcolor{red}{\textsuperscript{XXXX indx1}}!Dom Karlos, Infant von Spanien@\strich\emph{Dom Karlos, Infant von Spanien}|pw}{ }Schnabl\pwindex{Schnabel, C. @\textsc{Schnabel, C.}, \emph{Herausgeber, Lehrer}|pw}}}{\lemma{\textnormal{\emph{Carlos Schnabl}}}\Cendnote{\textnormal{vermutlich die Edition: \emph{Don Carlos, Infant von Spanien. Ein
                        dramatisches Gedicht. Zum Uebersetzen aus dem Deutschen in das Französische
                        für bereits vorgerückte Schüler, die in den Geist der beiden Idiome tiefer
                        eindringen und die Conversationssprache sich aneignen wollen. Mit
                        Anmerkungen der nöthigen Phraseologie und einem Wörterbuche. Zum Schul- und
                        Privatgebrauch}\pwindex{Don Carlos, Infant von Spanien. Ein dramatisches Gedicht. Zum Uebersetzen aus dem Deutschen in das Französische für bereits vorgerückte Schüler@\emph{Don Carlos, Infant von Spanien. Ein dramatisches Gedicht. Zum Uebersetzen aus dem Deutschen in das Französische für bereits vorgerückte Schüler}|pwk}. Herausgegeben von C.
                        Schnabel\pwindex{Schnabel, C. @\textsc{Schnabel, C.}, \emph{Herausgeber, Lehrer}|pwk}, öffentlicher Lehrer. Leipzig: \emph{Baumgärtner’sche Buchhandlung}\orgindex{Baumgärtnersche Buchhandlung@Baumgärtnersche Buchhandlung|pwk}{ }1846.}}}\label{K_L00425-2}.\pend
           
\pstart
           6.) vielleicht –{ }ſo jetzt haben Sie mir telephonirt, alſo es bleibt dabei, {\pb}wir treffen uns im \textsc{Griensteidl}\oindex{Wien@\textbf{Wien}!I., Innere Stadt@\textbf{I., Innere Stadt}!Café Griensteidl@\textbf{Café Griensteidl}, \emph{Kaffeehaus}|pw} gegen 8. Herzlich\pend
           \pstart Ihr \spacefill\mbox{Arth}\pend{}\selectlanguage{ngerman}\endnumbering\briefempfaengerindex{Beer-Hofmann, Richard@\textsc{Beer-Hofmann, Richard}!zzzSchnitzler, Arthur@\emph{von Arthur Schnitzler}!1895-03-261@{{[}26. 3. 1895{]}}|)be}\mylabel{L00425h}  \newcommand{\dateiname}{L00425}\newcommand{\titel}{Arthur Schnitzler an Richard Beer-Hofmann, [26. 3. 1895]}\newcommand{\editorInnen}{Herausgegeben von Martin Anton Müller}%% latex-leseansicht-abspann.tex
%% Abspann für die Leseansicht.
%% Der Schalter \ifkorrekturansicht ist bereits durch den Vorspann gesetzt.

%% latex-abspann.tex
%% Gemeinsamer Abspann für Korrekturansicht und Leseansicht.
%% Setzt den Schalter \ifkorrekturansicht voraus (gesetzt in den
%% einbindenden Dateien latex-korrekturansicht-abspann.tex bzw.
%% latex-leseansicht-abspann.tex).
%% ---------------------------------------------------------------

\normalsize

% Das esempio-Environment wird nur in der Leseansicht benötigt
\ifkorrekturansicht\else
\newenvironment{esempio}[3]%
{
    \vspace{1.5ex}
    \rlap{\underline{#1}}
    \par
    \setlength{\parindent}{0cm}
    \nopagebreak
    \leftskip=#2cm
    \rightskip=#3cm
}
{
    \par
}
\fi

\doendnotes{C}
\bigskip
\vfill

\clearpage

\footnotesize

\ifkorrekturansicht
  \lohead{\textsc{register}}
\fi

% theindex-Environment neu definieren ohne reledmac
\makeatletter
\renewenvironment{theindex}{%
  \ifkorrekturansicht
    \section*{\indexname}%
  \else
    \subsubsection*{Index der erwähnten Entitäten}%
  \fi
  \setlength{\parindent}{0pt}%
  \setlength{\parskip}{0pt plus 0.3pt}%
  \let\item\@idxitem
}{%
  \ifkorrekturansicht\clearpage\fi
}
\makeatother

\IfFileExists{\jobname-pw.ind}{\input{\jobname-pw.ind}}{}

% Quellenangabe nur in der Leseansicht
\ifkorrekturansicht\else
% Fallback-Definitionen, falls die .tex-Datei \titel etc. nicht gesetzt hat
\providecommand{\titel}{}
\providecommand{\editorInnen}{}
\providecommand{\dateiname}{\jobname}

\vspace{3cm}

\vfill

\footnotesize
\textsc{Quelle}: \titel. Herausgegeben von {\editorInnen}. In: \emph{Arthur Schnitzler: Briefwechsel mit Autorinnen und Autoren}.
 Digitale Edition, https://schnitzler-briefe.acdh.oeaw.ac.at/{\dateiname}.html (Stand \today)
\fi

\end{document}


