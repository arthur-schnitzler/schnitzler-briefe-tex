%% latex-leseansicht-vorspann.tex
%% Vorspann für die Leseansicht.
%% Lädt die gemeinsame Datei latex-vorspann.tex mit nicht gesetztem Schalter.

\newif\ifkorrekturansicht
\korrekturansichtfalse

\input{../tex-inputs/latex-vorspann}


         
         \renewcommand{\erwaehntePersonen}{Personen: Hermann Bahr, Richard Beer-Hofmann, Theodor Herzl, Hugo von Hofmannsthal, Bronisław Huberman, C. Schnabel,  b.m.}
         \renewcommand{\erwaehnteInstitutionen}{Institutionen: Baumgärtnersche Buchhandlung, Hermann Goldschmiedt {\kaufmannsund} Co.}
         \renewcommand{\erwaehnteOrte}{Orte: Café Griensteidl, Wien, Wollzeile}
         \renewcommand{\erwaehnteWerke}{Werke: Allgemeine Zeitung, Arthur Schnitzler: Sterben, Die Karlsschülerin, Don Carlos, Infant von Spanien. Ein dramatisches Gedicht. Zum Uebersetzen aus dem Deutschen in das Französische für bereits vorgerückte Schüler, Don Karlos, Infant von Spanien, Faust. Eine Tragödie, Wiener Touristen}
               \section[Arthur Schnitzler an Richard Beer-Hofmann, {[}26. 3. 1895{]}]{ Arthur Schnitzler an Richard Beer-Hofmann, {[}26. 3. 1895{]}}\nopagebreak\mylabel{v}\rehead{ }\begin{ledgroupsized}[t]{13cm}\normalsize\beginnumbering \toendnotes[C]{\smallbreak\pagebreak[2]} \Standort{YCGL, MSS 31.}
\physDesc{Brief, 1 Blatt, 4 Seiten, Umschlag
\newline{}Handschrift: Bleistift, deutsche Kurrent\newline{}Versand: ohne postalischen Übermittlungsvermerk }\buchAbdrucke{\weitereDrucke{Arthur Schnitzler, Richard Beer-Hofmann: \emph{Briefwechsel 1891–1931}. Hg. Konstanze Fliedl. Wien, Zürich: \emph{Europaverlag} 1992, S. 71–72.} }\toendnotes[C]{\smallbreak}\pstart{}{\pb}Herrn Dr. \textsc{Richard
                     Beer-Hofmann}\pend{}\pstart{}Wien\oindex{Wien@\textbf{Wien}|pw}\pend{}\pstart{}\textsc{I. Wollzeile 15\oindex{Wollzeile@\textbf{Wollzeile}|pw}}, 4. Stock.\pend{}{\bigskip}\pstart\center{}{\pb}Lieber Richard.\pend\pstart
           1) Ich habe noch nichts zu \textsc{Faust}\pwindex{\textcolor{red}{\textsuperscript{XXXX1 indx}}!Faust. Eine Tragoedie1808@\strich\emph{Faust. Eine Tragödie} {[}1808{]}|pw}, da ich den beſtechlichen nicht fand; ich zweifle aber nicht, daſs ich morgen
               Vormittag welche beko{\geminationm}en werde, reflectiren Sie denn
               drauf? Und,\pend
           \pstart
           2.) we{\geminationn} ich keine bekomm, wollen Sie mit mir morgen in
               ein andres Theater (»Karlsſchülerin\pwindex{\textcolor{red}{\textsuperscript{XXXX1 indx}}!Karlsschuelerin1895@\strich\emph{Die Karlsschülerin} {[}1895{]}|pw}« oder »Touriſten\pwindex{\textcolor{red}{\textsuperscript{XXXX1 indx}}!Wiener Touristen1895@\strich\emph{Wiener Touristen} {[}1895{]}|pw}\pwindex{\textcolor{red}{\textsuperscript{XXXX1 indx}}!Wiener Touristen1895@\strich\emph{Wiener Touristen} {[}1895{]}|pw}«) gehn?\pend
           \pstart
           3.) \textsc{Herzl}\pwindex{Herzl, Theodor 1860-05-02 – 1904-07-03@\textsc{Herzl, Theodor} (1860-05-02 – 1904-07-03), \emph{Schriftsteller, Journalist}|pw} iſt da, möchte mit uns, {\pb}dh. Ihnen, \textsc{Hugo}\pwindex{Hofmannsthal, Hugo von 1874-02-01 – 1929-07-15@\textsc{Hofmannsthal, Hugo von} (1874-02-01 – 1929-07-15), \emph{Schriftsteller}|pw}, mir, eventuell Bahr\pwindex{Bahr, Hermann 19.07.1863 – 15.01.1934@\textsc{Bahr, Hermann} (19.07.1863 – 15.01.1934), \emph{Schriftsteller, Kritiker}|pw}{ }ſoupiren. Ich ſagte ihm, Freitag nach dem \textsc{Hubermann}\pwindex{Huberman, Bronisław 19.12.1882 – 16.6.1947@\textsc{Huberman, Bronisław} (19.12.1882 – 16.6.1947), \emph{Musiker, Violinist, Schriftsteller}|pw}\textsc{concert} – Sie ſind doch einverſtanden? Zu \textsc{Bahr}\pwindex{Bahr, Hermann 19.07.1863 – 15.01.1934@\textsc{Bahr, Hermann} (19.07.1863 – 15.01.1934), \emph{Schriftsteller, Kritiker}|pw}{ }ſagen Sie vorläufig nichts, weil ich noch ein
               definitives Wort von \textsc{Herzl}\pwindex{Herzl, Theodor 1860-05-02 – 1904-07-03@\textsc{Herzl, Theodor} (1860-05-02 – 1904-07-03), \emph{Schriftsteller, Journalist}|pw} erwarte. \textsc{Hugo}\pwindex{Hofmannsthal, Hugo von 1874-02-01 – 1929-07-15@\textsc{Hofmannsthal, Hugo von} (1874-02-01 – 1929-07-15), \emph{Schriftsteller}|pw} theilen Sie’s vielleicht mit?\pend
           \pstart
           4.) bitte kaufen Sie \textsc{vis à vis}{ }{\pb}bei \textsc{Goldschmidt}\orgindex{Hermann Goldschmiedt und Co.@Hermann Goldschmiedt {\kaufmannsund}  Co.|pw} die Münchner Allgemeine\pwindex{?? Werk@Nicht ermittelte Verfasserinnen und Verfasser!Allgemeine Zeitung1798 – 30.6.1929@\emph{Allgemeine Zeitung} {[}1798 – 30.6.1929{]}|pw} von Samstag den
                  23. d. mit \label{K_L00425_1v}\edtext{Beilage}{\lemma{\textnormal{\emph{Beilage}}}\Cendnote{\textnormal{wohl wegen: b. m.\pwindex{b.m. 1895 – 1895@\textsc{b.m.} (1895 – 1895), \emph{Journalist/Journalistin}|pwk}: \emph{Arthur
                        Schnitzler: Sterben}\pwindex{Arthur Schnitzler: Sterben23. 3. 1895@\emph{Arthur Schnitzler: Sterben} {[}23. 3. 1895{]}|pwk}. In: \emph{Beilage zur
                        Allgemeinen Zeitung}\pwindex{?? Werk@Nicht ermittelte Verfasserinnen und Verfasser!Allgemeine Zeitung1798 – 30.6.1929@\emph{Allgemeine Zeitung} {[}1798 – 30.6.1929{]}|pwk}, Beilage-Nr. 69, 23. 3. 1895,
                     S. 5}}}\label{K_L00425_1h} für mich.\pend
           \pstart
           5.) hier iſt \label{K_L00425_2v}\edtext{\textsc{Carlos\pwindex{\textcolor{red}{\textsuperscript{XXXX1 indx}}!Don Karlos, Infant von Spanien1787@\strich\emph{Don Karlos, Infant von Spanien} {[}1787{]}|pw}{ }Schnabl\pwindex{Schnabel, C. @\textsc{Schnabel, C.}, \emph{Herausgeber, Lehrer}|pw}}}{\lemma{\textnormal{\emph{Carlos Schnabl}}}\Cendnote{\textnormal{vermutlich die Edition: \emph{Don Carlos, Infant von Spanien. Ein
                     dramatisches Gedicht. Zum Uebersetzen aus dem Deutschen in das Französische für
                     bereits vorgerückte Schüler, die in den Geist der beiden Idiome tiefer
                     eindringen und die Conversationssprache sich aneignen wollen. Mit Anmerkungen
                     der nöthigen Phraseologie und einem Wörterbuche. Zum Schul- und Privatgebrauch}\pwindex{Don Carlos, Infant von Spanien. Ein dramatisches Gedicht. Zum Uebersetzen aus dem Deutschen in das Franzoesische fuer bereits vorgerueckte Schueler1846@\emph{Don Carlos, Infant von Spanien. Ein dramatisches Gedicht. Zum Uebersetzen aus dem Deutschen in das Französische für bereits vorgerückte Schüler} {[}1846{]}|pwk}.
                     Herausgegeben von C. Schnabel\pwindex{Schnabel, C. @\textsc{Schnabel, C.}, \emph{Herausgeber, Lehrer}|pwk}, öffentlicher
                     Lehrer. Leipzig: \emph{Baumgärtner’sche Buchhandlung}\orgindex{Baumgaertnersche Buchhandlung@Baumgärtnersche Buchhandlung|pwk}{ }1846.}}}\label{K_L00425_2h}.\pend
           \pstart
           6.) vielleicht – ſo jetzt haben Sie mir telephonirt, alſo es bleibt dabei, {\pb}wir treffen uns im \textsc{Griensteidl}\oindex{Cafe Griensteidl@\textbf{Café Griensteidl}|pw} gegen 8. Herzlich\pend
           \pstart Ihr \spacefill\mbox{Arth}\pend{}
         
         \endnumbering\mylabel{h}\end{ledgroupsized}  \newcommand{\dateiname}{L00425}\newcommand{\titel}{Arthur Schnitzler an Richard Beer-Hofmann, [26. 3. 1895]}\newcommand{\editorInnen}{ Martin Anton Müller und Gerd-Hermann Susen}%% latex-leseansicht-abspann.tex
%% Abspann für die Leseansicht.
%% Der Schalter \ifkorrekturansicht ist bereits durch den Vorspann gesetzt.

%% latex-abspann.tex
%% Gemeinsamer Abspann für Korrekturansicht und Leseansicht.
%% Setzt den Schalter \ifkorrekturansicht voraus (gesetzt in den
%% einbindenden Dateien latex-korrekturansicht-abspann.tex bzw.
%% latex-leseansicht-abspann.tex).
%% ---------------------------------------------------------------

\normalsize

% Das esempio-Environment wird nur in der Leseansicht benötigt
\ifkorrekturansicht\else
\newenvironment{esempio}[3]%
{
    \vspace{1.5ex}
    \rlap{\underline{#1}}
    \par
    \setlength{\parindent}{0cm}
    \nopagebreak
    \leftskip=#2cm
    \rightskip=#3cm
}
{
    \par
}
\fi

\doendnotes{C}
\bigskip
\vfill

\clearpage

\footnotesize

\ifkorrekturansicht
  \lohead{\textsc{register}}
\fi

% theindex-Environment neu definieren ohne reledmac
\makeatletter
\renewenvironment{theindex}{%
  \ifkorrekturansicht
    \section*{\indexname}%
  \else
    \subsubsection*{Index der erwähnten Entitäten}%
  \fi
  \setlength{\parindent}{0pt}%
  \setlength{\parskip}{0pt plus 0.3pt}%
  \let\item\@idxitem
}{%
  \ifkorrekturansicht\clearpage\fi
}
\makeatother

\IfFileExists{\jobname-pw.ind}{\input{\jobname-pw.ind}}{}

% Quellenangabe nur in der Leseansicht
\ifkorrekturansicht\else
% Fallback-Definitionen, falls die .tex-Datei \titel etc. nicht gesetzt hat
\providecommand{\titel}{}
\providecommand{\editorInnen}{}
\providecommand{\dateiname}{\jobname}

\vspace{3cm}

\vfill

\footnotesize
\textsc{Quelle}: \titel. Herausgegeben von {\editorInnen}. In: \emph{Arthur Schnitzler: Briefwechsel mit Autorinnen und Autoren}.
 Digitale Edition, https://schnitzler-briefe.acdh.oeaw.ac.at/{\dateiname}.html (Stand \today)
\fi

\end{document}


      