%% latex-leseansicht-vorspann.tex
%% Vorspann für die Leseansicht.
%% Lädt die gemeinsame Datei latex-vorspann.tex mit nicht gesetztem Schalter.

\newif\ifkorrekturansicht
\korrekturansichtfalse

\input{../tex-inputs/latex-vorspann}


\section[ Paul Goldmann an Arthur Schnitzler, 4. 3. 1900]{L02906 Paul Goldmann an Arthur Schnitzler,  4. 3. 1900}
\nopagebreak\mylabel{L02906v}
\rehead{ }\normalsize\beginnumbering\briefempfaengerindex{Schnitzler, Arthur@\textsc{Schnitzler, Arthur}!zzzGoldmann, Paul@\emph{von Paul Goldmann}!1900-03-041@{4. 3. 1900}|(be}
\toendnotes[C]{\smallbreak\pagebreak[2]}
\correspDesc{Versand  durch Paul Goldmann am 4. 3. 1900 in Berlin
\newline{}Erhalt  durch Arthur Schnitzler im Zeitraum [6. 3. 1900
                  – 9. 3. 1900?] in Wien}\toendnotes[C]{\smallbreak}
\Standort{DLA, A:Schnitzler, HS.NZ85.1.3170.}
\physDesc{Brief, 1 Blatt, 1 Seite, 214 Zeichen
\newline{}Handschrift: schwarze Tinte, deutsche Kurrent}
\pstart
           \centering{}{\pb}\textcolor{gray}{\textbf{\textbf{HOTEL SAXONIA\oindex{Hotel Saxonia@\textbf{Hotel Saxonia}, \emph{Hotel}|pw}}}}\pend
           
\pstart
           \raggedleft{}\textcolor{gray}{\textbf{am Potsdamer Platz\oindex{Potsdamer Platz@\textbf{Potsdamer Platz}, \emph{Platz}|pw} und
                        Thiergarten\oindex{Tiergarten@\textbf{Tiergarten}, \emph{Ehemaliger Ort}|pw}}}\pend
           
\pstart
           \centering{}\textcolor{gray}{\textbf{D. W. SCHRÖDER\pwindex{Schröder, D. W. @\textsc{Schröder, D. W.}, \emph{Hotelbesitzer/Hotelbesitzerin}|pw}.}}\pend
           
\pstart
           \textcolor{gray}{\textbf{Fernsprecher:}}\pend
           
\pstart
           \textcolor{gray}{\textbf{\textbf{Amt VI. No. 2838.}}}\pend
           
\pstart
           \raggedleft{}\textcolor{gray}{\textbf{\emph{BERLIN W.}\oindex{Berlin@\textbf{Berlin}, \emph{Hauptstadt}|pw}, den}}{ }4. März \textcolor{gray}{\textbf{1}}900.\pend
           
\pstart
           \raggedleft{}\textcolor{gray}{\textbf{Königgrätzerstrasse 10\oindex{Stresemannstraße@\textbf{Stresemannstraße}, \emph{Straße}|pw}.}}\pend
           
\pstart\center{}Mein lieber Freund,\pend\vspace{0.5em}
\pstart
           Ich beziehe heut mein hieſige Wohnung, und meine
               Adreſſe lautet jetzt: Deſſauer Straße 19\oindex{Dessauer Straße@\textbf{Dessauer Straße}, \emph{Straße}|pw}.\pend
           
\pstart
           Es iſt{ }ſehr bedauerlich, daß Du{ }ſo{ }ſchreibfaul geworden biſt.\pend
           
\pstart
           Viele treue Grüße! {\\[\baselineskip]}Dein {\\[\baselineskip]}\spacefill\mbox{Paul Goldmann}\pend
           \leftskip=0em{}\selectlanguage{ngerman}\endnumbering\briefempfaengerindex{Schnitzler, Arthur@\textsc{Schnitzler, Arthur}!zzzGoldmann, Paul@\emph{von Paul Goldmann}!1900-03-041@{4. 3. 1900}|)be}\mylabel{L02906h}  \newcommand{\dateiname}{L02906}\newcommand{\titel}{Paul Goldmann an Arthur Schnitzler, 4. 3. 1900}\newcommand{\editorInnen}{Martin Anton Müller und Laura Untner}%% latex-leseansicht-abspann.tex
%% Abspann für die Leseansicht.
%% Der Schalter \ifkorrekturansicht ist bereits durch den Vorspann gesetzt.

%% latex-abspann.tex
%% Gemeinsamer Abspann für Korrekturansicht und Leseansicht.
%% Setzt den Schalter \ifkorrekturansicht voraus (gesetzt in den
%% einbindenden Dateien latex-korrekturansicht-abspann.tex bzw.
%% latex-leseansicht-abspann.tex).
%% ---------------------------------------------------------------

\normalsize

% Das esempio-Environment wird nur in der Leseansicht benötigt
\ifkorrekturansicht\else
\newenvironment{esempio}[3]%
{
    \vspace{1.5ex}
    \rlap{\underline{#1}}
    \par
    \setlength{\parindent}{0cm}
    \nopagebreak
    \leftskip=#2cm
    \rightskip=#3cm
}
{
    \par
}
\fi

\doendnotes{C}
\bigskip
\vfill

\clearpage

\footnotesize

\ifkorrekturansicht
  \lohead{\textsc{register}}
\fi

% theindex-Environment neu definieren ohne reledmac
\makeatletter
\renewenvironment{theindex}{%
  \ifkorrekturansicht
    \section*{\indexname}%
  \else
    \subsubsection*{Index der erwähnten Entitäten}%
  \fi
  \setlength{\parindent}{0pt}%
  \setlength{\parskip}{0pt plus 0.3pt}%
  \let\item\@idxitem
}{%
  \ifkorrekturansicht\clearpage\fi
}
\makeatother

\IfFileExists{\jobname-pw.ind}{\input{\jobname-pw.ind}}{}

% Quellenangabe nur in der Leseansicht
\ifkorrekturansicht\else
% Fallback-Definitionen, falls die .tex-Datei \titel etc. nicht gesetzt hat
\providecommand{\titel}{}
\providecommand{\editorInnen}{}
\providecommand{\dateiname}{\jobname}

\vspace{3cm}

\vfill

\footnotesize
\textsc{Quelle}: \titel. Herausgegeben von {\editorInnen}. In: \emph{Arthur Schnitzler: Briefwechsel mit Autorinnen und Autoren}.
 Digitale Edition, https://schnitzler-briefe.acdh.oeaw.ac.at/{\dateiname}.html (Stand \today)
\fi

\end{document}


