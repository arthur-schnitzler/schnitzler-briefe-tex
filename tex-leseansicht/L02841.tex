%% latex-leseansicht-vorspann.tex
%% Vorspann für die Leseansicht.
%% Lädt die gemeinsame Datei latex-vorspann.tex mit nicht gesetztem Schalter.

\newif\ifkorrekturansicht
\korrekturansichtfalse

\input{../tex-inputs/latex-vorspann}


         
         \newcommand{\erwaehntePersonen}{Personen: Theodor Herzl, Marie Reinhard, Leopold Sonnemann}
         \newcommand{\erwaehnteInstitutionen}{Institutionen: Frankfurter Zeitung, Preussen}
         \newcommand{\erwaehnteOrte}{Orte: Frankfurt am Main, Genua, Italien, Paris, Wien, rue de la Bourse}
         \newcommand{\erwaehnteWerke}{Werke: Das neue Ghetto, Feuilleton. Carl-Theater. (»Freiwild«, Schauspiel von Arthur Schnitzler.)}
               \section[ Paul Goldmann an Arthur Schnitzler, 7. 3. {[}1898{]}]{ Paul Goldmann an Arthur Schnitzler, 7. 3. {[}1898{]}}\nopagebreak\mylabel{v}\rehead{ }\begin{ledgroupsized}[t]{13cm}\normalsize\beginnumbering \toendnotes[C]{\smallbreak\pagebreak[2]} \Standort{DLA, A:Schnitzler, HS.NZ85.1.3168.}
\physDesc{Brief, 1 Blatt, 4 Seiten
\newline{}Handschrift: blaue Tinte, deutsche Kurrent
\newline{}Schnitzler: 1) mit Bleistift das Jahr »98« vermerkt  2) mit rotem Buntstift zwei Unterstreichungen}\toendnotes[C]{\smallbreak}\pstart
           \noindent{}{\pb}\textcolor{gray}{\textbf{\textbf{Frankfurter Zeitung\orgindex{Frankfurter Zeitung@Frankfurter Zeitung|pw}}}}\pend
           \pstart
           \textcolor{gray}{\textbf{(\begin{otherlanguage}{french}Gazette de Francfort\end{otherlanguage}\orgindex{Frankfurter Zeitung@Frankfurter Zeitung|pw}).}}\pend
           \pstart
           \textcolor{gray}{\textbf{\textbf{\begin{otherlanguage}{french}Fondateur M.\end{otherlanguage}{ }L. Sonnemann\pwindex{Sonnemann, Leopold 1831-10-29 – 1909-10-30@\textsc{Sonnemann, Leopold} (1831-10-29 – 1909-10-30), \emph{Journalist, Herausgeber}|pw}.}}}\pend
           \pstart
           \begin{otherlanguage}{french}\textcolor{gray}{\textbf{Journal politique, financier,}}\end{otherlanguage}\pend
           \pstart
           \begin{otherlanguage}{french}\textcolor{gray}{\textbf{commercial et littéraire.}}\end{otherlanguage}\pend
           \pstart
           \begin{otherlanguage}{french}\textcolor{gray}{\textbf{\textbf{Paraissant trois fois par jour.}}}\end{otherlanguage}\hfill \textsc{Paris\oindex{Paris@\textbf{Paris}|pw}}, 7. März.\pend
           \pstart
           \begin{otherlanguage}{french}\textcolor{gray}{\textbf{\textbf{Bureau à Paris\oindex{Paris@\textbf{Paris}|pw}}}}\end{otherlanguage}\pend
           \pstart
           \begin{otherlanguage}{french}\textcolor{gray}{\textbf{\textbf{10 \so{Rue de la Bourse}\oindex{rue de la Bourse@\textbf{rue de la Bourse}|pw}.}}}\end{otherlanguage}\pend
           \pstart\center{}Mein lieber Freund,\pend\pstart
           Ich ſchicke Dir \label{K_L02841-1v}\edtext{\textsc{Herzl\pwindex{Herzl, Theodor 1860-05-02 – 1904-07-03@\textsc{Herzl, Theodor} (1860-05-02 – 1904-07-03), \emph{Schriftsteller, Journalist}|pw}s}{ }Feuilleton\pwindex{Feuilleton. Carl-Theater. (»Freiwild«, Schauspiel von Arthur Schnitzler.)1898-02-13@\emph{Feuilleton. Carl-Theater. (»Freiwild«, Schauspiel von Arthur Schnitzler.)} {[}1898-02-13{]}|pwv}}{\lemma{\textnormal{\emph{Herzls Feuilleton}}}\Cendnote{\textnormal{siehe Paul Goldmann an Arthur Schnitzler, 28. 2. [1898]}}}\label{K_L02841-1h} zurück. Es hat mich \strikeout{recht} recht ſehr amüſirt:
               Mißgunſt, welche von Unverſtändniß ſo glücklich unterſtützt wird, daß ſie beinahe zum
               guten Glauben wird! Die »größeren
                  Fragen\pwindex{Feuilleton. Carl-Theater. (»Freiwild«, Schauspiel von Arthur Schnitzler.)1898-02-13@\emph{Feuilleton. Carl-Theater. (»Freiwild«, Schauspiel von Arthur Schnitzler.)} {[}1898-02-13{]}|pwv}« ſind Dir nicht zugänglich, mein armer Freund! Du lebſt und producirſt
               im Kleinen und ahnſt nicht, daß es hoch über dem Allen den \textsc{Zionismus} gibt. Wenn Du aber wiſſen willſt, wie man auf dem Theater etwas
               beweiſt mit »geſchloſſenen und
                  wetterfeſten Gründen\pwindex{Feuilleton. Carl-Theater. (»Freiwild«, Schauspiel von Arthur Schnitzler.)1898-02-13@\emph{Feuilleton. Carl-Theater. (»Freiwild«, Schauspiel von Arthur Schnitzler.)} {[}1898-02-13{]}|pwv}«, ſo kannſt Du {\pb}das aus
               dem »neuen \textsc{Ghetto}\pwindex{Herzl, Theodor 1860-05-02 – 1904-07-03@\textsc{Herzl, Theodor} (1860-05-02 – 1904-07-03), \emph{Schriftsteller, Journalist}!neue Ghetto1894@\strich\emph{Das neue Ghetto} {[}1894{]}|pw}« lernen.\pend
           \pstart
           Geh’, kümmere Dich nicht um das, was ſo ein Schafskopf\pwindex{Herzl, Theodor 1860-05-02 – 1904-07-03@\textsc{Herzl, Theodor} (1860-05-02 – 1904-07-03), \emph{Schriftsteller, Journalist}|pwv} ſchreibt, und geh’ Du nur ruhig weiter Deinen Weg.
               Ich ſehe aus Deinem lieben Briefe, daß Du wieder arbeitsluſtig biſt und \textsc{voll} von Plänen ſteckſt. Sehr ſchön! Du kannſt Herrn \textsc{Herzl\pwindex{Herzl, Theodor 1860-05-02 – 1904-07-03@\textsc{Herzl, Theodor} (1860-05-02 – 1904-07-03), \emph{Schriftsteller, Journalist}|pw}} durch nichts einen größeren Schmerz zufügen, als dadurch, daß Du ein neues
               gutes Stück ſchreibſt. Ich fürchte, wir werden ihm dieſen Schmerz nicht erſparen
               können.\pend
           \pstart
           Mein Schiff\orgindex{Preussen@Preussen|pwv}s-Platz iſt
               genommen. Ab \textsc{Genua\oindex{Genua@\textbf{Genua}|pw}}, 5. \strikeout{An} April.
               Aber die Vertretung\orgindex{Frankfurter Zeitung@Frankfurter Zeitung|pwv}s-Frage iſt
               nicht geregelt, und die {\pb}Sache kann ſich immer noch
               in letzter Stunde zerſchlagen.\pend
           \pstart
           Mir iſt recht unheimlich. Ich glaube, ich komme nicht lebendig zurück. Das wäre aber
               noch nicht ſo ſchlimm, wie die Furcht vor der neuen journaliſtiſchen Aufgabe, der ich
                  \strikeout{\textcolor{gray}{w}} wohl kaum gewachſen ſein
               werde: In der Haſt einer Reiſe, in einem feindlichen Klima, unter ganz veränderten
               Lebens-Verhältniſſen Eindrücke von Ländern zu geben, \strikeout{\textcolor{gray}{×}}{ }\strikeout{f\textcolor{gray}{ü}r} von denen man auch nicht die
               leiſeſte Ahnung hat! Mir grauſt, und ich fürchte, ich werde ſehr enttäuſchen. Im
               Übrigen bin ich ſicher caput zu gehen. Ich komme durch tropiſche {\pb}Gegenden, und dicke Leute ſterben immer am
               Fieber.\pend
           \pstart
           Weißt Du, was ſchön wäre? Wenn Du ſo \strikeout{A\textcolor{gray}{n}} Ende März nach Italien\oindex{Italien@\textbf{Italien}|pw} gingeſt und ſo um den 5. April herum
               auch \label{K_L02841-4v}\edtext{in \textsc{Genua\oindex{Genua@\textbf{Genua}|pw}} wäreſt}{\lemma{\textnormal{\emph{in Genua wäreſt}}}\Cendnote{\textnormal{Dazu kam es nicht.}}}\label{K_L02841-4h}!
               Ich möchte Dich gern noch einmal zum Abſchied umarmen!\pend
           \pstart
           Schreib’ mir bald noch einmal hierher; denn ich fahre vielleicht ſchon nächſte Woche
               nach Frankfurt\oindex{Frankfurt am Main@\textbf{Frankfurt am Main}|pw}.\pend
           \pstart
           Viele treue Grüße!\pend
           \pstart
           Dein {\\[\baselineskip]}\spacefill\mbox{Paul Goldmn}\pend
           \leftskip=0em{}\pstart
           \noindent{}Schönen Grüß an Deine Freundin\pwindex{Reinhard, Marie 1871-03-13 – 1899-03-18@\textsc{Reinhard, Marie} (1871-03-13 – 1899-03-18), \emph{Gesangspädagogin}|pwv}!\pend
           
         
         \endnumbering\mylabel{h}\end{ledgroupsized}  \newcommand{\dateiname}{L02841}\newcommand{\titel}{Paul Goldmann an Arthur Schnitzler, 7. 3. [1898]}\newcommand{\editorInnen}{Martin Anton Müller und Laura Untner}%% latex-leseansicht-abspann.tex
%% Abspann für die Leseansicht.
%% Der Schalter \ifkorrekturansicht ist bereits durch den Vorspann gesetzt.

%% latex-abspann.tex
%% Gemeinsamer Abspann für Korrekturansicht und Leseansicht.
%% Setzt den Schalter \ifkorrekturansicht voraus (gesetzt in den
%% einbindenden Dateien latex-korrekturansicht-abspann.tex bzw.
%% latex-leseansicht-abspann.tex).
%% ---------------------------------------------------------------

\normalsize

% Das esempio-Environment wird nur in der Leseansicht benötigt
\ifkorrekturansicht\else
\newenvironment{esempio}[3]%
{
    \vspace{1.5ex}
    \rlap{\underline{#1}}
    \par
    \setlength{\parindent}{0cm}
    \nopagebreak
    \leftskip=#2cm
    \rightskip=#3cm
}
{
    \par
}
\fi

\doendnotes{C}
\bigskip
\vfill

\clearpage

\footnotesize

\ifkorrekturansicht
  \lohead{\textsc{register}}
\fi

% theindex-Environment neu definieren ohne reledmac
\makeatletter
\renewenvironment{theindex}{%
  \ifkorrekturansicht
    \section*{\indexname}%
  \else
    \subsubsection*{Index der erwähnten Entitäten}%
  \fi
  \setlength{\parindent}{0pt}%
  \setlength{\parskip}{0pt plus 0.3pt}%
  \let\item\@idxitem
}{%
  \ifkorrekturansicht\clearpage\fi
}
\makeatother

\IfFileExists{\jobname-pw.ind}{\input{\jobname-pw.ind}}{}

% Quellenangabe nur in der Leseansicht
\ifkorrekturansicht\else
% Fallback-Definitionen, falls die .tex-Datei \titel etc. nicht gesetzt hat
\providecommand{\titel}{}
\providecommand{\editorInnen}{}
\providecommand{\dateiname}{\jobname}

\vspace{3cm}

\vfill

\footnotesize
\textsc{Quelle}: \titel. Herausgegeben von {\editorInnen}. In: \emph{Arthur Schnitzler: Briefwechsel mit Autorinnen und Autoren}.
 Digitale Edition, https://schnitzler-briefe.acdh.oeaw.ac.at/{\dateiname}.html (Stand \today)
\fi

\end{document}


      