%% latex-korrekturansicht-vorspann.tex
%% Vorspann für die Korrekturansicht.
%% Lädt die gemeinsame Datei latex-vorspann.tex mit gesetztem Schalter.

\newif\ifkorrekturansicht
\korrekturansichttrue

\input{../tex-inputs/latex-vorspann}


\section[Hugo von Hofmannsthal an Arthur Schnitzler, 4. 1. 1904]{L01357 Hugo von Hofmannsthal an Arthur Schnitzler, 4. 1. 1904}
\nopagebreak\mylabel{L01357v}
\rehead{ }\normalsize\beginnumbering\briefempfaengerindex{Schnitzler, Arthur@\textsc{Schnitzler, Arthur}!zzzHofmannsthal, Hugo von@\emph{von Hugo von Hofmannsthal}!1904-01-042@{4. 1. 1904}|(be}
\toendnotes[C]{\smallbreak\pagebreak[2]}\Standort{CUL, Schnitzler, B 43.}
\physDesc{Postkarte, 151 Zeichen
\newline{}Handschrift: 1) schwarze Tinte, deutsche Kurrent\hspace{1em}2) schwarze Tinte, lateinische Kurrent (\noindent{}Adresse)\hspace{1em}
\newline{}Versand: 1) Rohrpost  2) Stempel: »\nobreak{}\oindex{I., Innere Stadt@\textbf{I., Innere Stadt}, \emph{A.ADM3}|pwk}Wien 1/1, 4 1 04, 8 40N\nobreak{}«.  3) Stempel: »\nobreak{}\oindex{XVIII., Waehring@\textbf{XVIII., Währing}, \emph{A.ADM3}|pwk}18/1 Wien, 4 1 04, 9 30N\nobreak{}«. 
\newline{}Schnitzler: mit Bleistift beschrift: »Hugo« 
\newline{}Ordnung: 1) mit Bleistift von unbekannter Hand nummeriert: »\strikeout{\textcolor{gray}{249}}«  2) mit Bleistift von unbekannter Hand nummeriert:
                                    »211«}\toendnotes[C]{\smallbreak}\pstart{}{\pb}Herrn D\textsuperscript{r} Arthur Schnitzler\pend{}\pstart{}Wien\oindex{Wien@\textbf{Wien}, \emph{A.ADM2}|pw}\pend{}\pstart{}XVIII Spöttelgasse 7\oindex{Edmund-Weiss-Gasse 7@\textbf{Edmund-Weiß-Gasse 7}, \emph{Wohngebäude (K.WHS)}|pw}\pend{}{\bigskip}\vspace{1em}
\pstart
           \noindent{}{\pb}\textsc{Miss Isadora Duncan\pwindex{Duncan, Isadora 1877-05-27 – 1927-09-14@\textsc{Duncan, Isadora} (1877-05-27 – 1927-09-14), \emph{Tänzer/Tänzerin, Choreograf/Choreografin}|pw}}\pend
           
\pstart
           bittet \textsc{Herrn u. Frau D\textsuperscript{r} Schnitzler\pwindex{Schnitzler, Olga 17.01.1882 – 13.01.1970@\textsc{Schnitzler, Olga} (17.01.1882 – 13.01.1970), \emph{Schauspieler/Schauspielerin, Sänger/Sängerin}|pw}} zu Ihrer \label{K_L01357-1v}\edtext{Generalprobe}{\lemma{\textnormal{\emph{Generalprobe}}}\Cendnote{\textnormal{Hofmannsthal\pwindex{Hofmannsthal, Hugo von 1874-02-01 – 1929-07-15@\textsc{Hofmannsthal, Hugo von} (1874-02-01 – 1929-07-15), \emph{Schriftsteller/Schriftstellerin}|pwk} sollte einleitende Worte
                  sprechen; siehe A. S.: \emph{Tagebuch}, 5. 1. 1904.
               }}}\label{K_L01357-1}\pend
           
\pstart
           morgen Dienstag{ }\uline{3½} Uhr\pend
           
\pstart
           Carltheater\oindex{Carl-Theater@\textbf{Carl-Theater}, \emph{Theater (K.THE)}|pw}.\pend
           \selectlanguage{ngerman}\endnumbering\briefempfaengerindex{Schnitzler, Arthur@\textsc{Schnitzler, Arthur}!zzzHofmannsthal, Hugo von@\emph{von Hugo von Hofmannsthal}!1904-01-042@{4. 1. 1904}|)be}\mylabel{L01357h}  \normalsize

\doendnotes{C}
\bigskip
\vfill

\clearpage

\footnotesize

\lohead{\textsc{register}}

% Definiere theindex-Environment komplett neu ohne reledmac
\makeatletter
\renewenvironment{theindex}{%
  \section*{\indexname}%
  \setlength{\parindent}{0pt}%
  \setlength{\parskip}{0pt plus 0.3pt}%
  \let\item\@idxitem
}{%
  \clearpage
}
\makeatother

\IfFileExists{\jobname-pw.ind}{\input{\jobname-pw.ind}}{}

\end{document}

      