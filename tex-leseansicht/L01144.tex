%% latex-leseansicht-vorspann.tex
%% Vorspann für die Leseansicht.
%% Lädt die gemeinsame Datei latex-vorspann.tex mit nicht gesetztem Schalter.

\newif\ifkorrekturansicht
\korrekturansichtfalse

\input{../tex-inputs/latex-vorspann}


\section[Richard Beer-Hofmann an Arthur Schnitzler, 10. 7. 1901]{L01144 Richard Beer-Hofmann an Arthur Schnitzler, 10. 7. 1901}
\nopagebreak\mylabel{L01144v}
\rehead{ }\normalsize\beginnumbering\briefempfaengerindex{Schnitzler, Arthur@\textsc{Schnitzler, Arthur}!zzzBeer-Hofmann, Richard@\emph{von Richard Beer-Hofmann}!1901-07-101@{10. 7. 1901}|(be}
\toendnotes[C]{\smallbreak\pagebreak[2]}
\correspDesc{Versand  durch Richard Beer-Hofmann am 10. 7. 1901 in Pörtschach
\newline{}Erhalt  durch Arthur Schnitzler im Zeitraum [10. 7. 1901
                  – 14. 7. 1901?] \textbf{Ort fehlend} }\toendnotes[C]{\smallbreak}
\Standort{CUL, Schnitzler, B 8.}
\physDesc{Brief, 1 Blatt, 2 Seiten, 610 Zeichen
\newline{}Handschrift: blauer Buntstift, lateinische Kurrent
\newline{}Ordnung: mit Bleistift von unbekannter Hand nummeriert:
                                    »164« }
\buchAbdrucke{\weitereDrucke{Arthur Schnitzler, Richard Beer-Hofmann: \emph{Briefwechsel 1891–1931}. Herausgegeben von Konstanze Fliedl. Wien, Zürich: \emph{Europaverlag} 1992, S. 153.} }\toendnotes[C]{\smallbreak}
\pstart
           \raggedleft{}{\pb}Pörtschach\oindex{Pörtschach am Wörthersee@\textbf{Pörtschach am Wörthersee}|pw}{ }10/VII. 1901\pend
           \vspace{0.5em}
\pstart
           Lieber Arthur! Wir waren am 1, 2,
                  3, in Wien\oindex{Wien@\textbf{Wien}, \emph{Verwaltungsgebiet}|pw}; seit 4.
               sind wir wieder hier mit Papa Hermann\pwindex{Beer, Hermann 10.\,8.\,1835 Radiměř – 3.\,10.\,1902 Wien@\textsc{Beer, Hermann} (10.\,8.\,1835 Radiměř – 3.\,10.\,1902 Wien), \emph{Rechtsanwalt}|pw}, dessen
                  Frau\pwindex{Beer, Rosa 20.\,7.\,1847 Borek Fałęcki – 1.\,7.\,1901 Wien@\textsc{Beer, Rosa} (20.\,7.\,1847 Borek Fałęcki – 1.\,7.\,1901 Wien)|pwv} am 1.
               gestorben ist. Da der Papa\pwindex{Beer, Hermann 10.\,8.\,1835 Radiměř – 3.\,10.\,1902 Wien@\textsc{Beer, Hermann} (10.\,8.\,1835 Radiměř – 3.\,10.\,1902 Wien), \emph{Rechtsanwalt}|pwv}
               auch physisch sehr hergeno{\geminationm}en ist, haben wir vorläufig
               mit ihm zu tun. Obgleich er nicht lange hierbleiben will, weiß ich doch nicht ob ich
               gegen Mitte oder Ende August Sie irgendwo werde treffen
               können.\pend
           
\pstart
           {\pb}Auch nicht ob ich Lust haben werde
               irgendwohin zu reisen, da ich endlich arbeiten möchte. Ich freue mich sehr daß Sie
               sich wol fühlen. Hoffentlich nimmt »man« Ihnen Ihre Grillparzer\pwindex{Grillparzer, Franz 15.\,1.\,1791 Wien – 21.\,1.\,1872 ebd.@\textsc{Grillparzer, Franz} (15.\,1.\,1791 Wien – 21.\,1.\,1872 ebd.), \emph{Schriftsteller, Beamter}|pw}grämlichkeit der letzten Zeit. Schreiben Sie mir bald und
               viel.\pend
           
\pstart
           Von Herzen Ihr{\\[\baselineskip]}\spacefill\mbox{Richard}\pend
           \leftskip=0em{}\selectlanguage{ngerman}\endnumbering\briefempfaengerindex{Schnitzler, Arthur@\textsc{Schnitzler, Arthur}!zzzBeer-Hofmann, Richard@\emph{von Richard Beer-Hofmann}!1901-07-101@{10. 7. 1901}|)be}\mylabel{L01144h}  \newcommand{\dateiname}{L01144}\newcommand{\titel}{Richard Beer-Hofmann an Arthur Schnitzler, 10. 7. 1901}\newcommand{\editorInnen}{Martin Anton Müller und Gerd-Hermann Susen}%% latex-leseansicht-abspann.tex
%% Abspann für die Leseansicht.
%% Der Schalter \ifkorrekturansicht ist bereits durch den Vorspann gesetzt.

%% latex-abspann.tex
%% Gemeinsamer Abspann für Korrekturansicht und Leseansicht.
%% Setzt den Schalter \ifkorrekturansicht voraus (gesetzt in den
%% einbindenden Dateien latex-korrekturansicht-abspann.tex bzw.
%% latex-leseansicht-abspann.tex).
%% ---------------------------------------------------------------

\normalsize

% Das esempio-Environment wird nur in der Leseansicht benötigt
\ifkorrekturansicht\else
\newenvironment{esempio}[3]%
{
    \vspace{1.5ex}
    \rlap{\underline{#1}}
    \par
    \setlength{\parindent}{0cm}
    \nopagebreak
    \leftskip=#2cm
    \rightskip=#3cm
}
{
    \par
}
\fi

\doendnotes{C}
\bigskip
\vfill

\clearpage

\footnotesize

\ifkorrekturansicht
  \lohead{\textsc{register}}
\fi

% theindex-Environment neu definieren ohne reledmac
\makeatletter
\renewenvironment{theindex}{%
  \ifkorrekturansicht
    \section*{\indexname}%
  \else
    \subsubsection*{Index der erwähnten Entitäten}%
  \fi
  \setlength{\parindent}{0pt}%
  \setlength{\parskip}{0pt plus 0.3pt}%
  \let\item\@idxitem
}{%
  \ifkorrekturansicht\clearpage\fi
}
\makeatother

\IfFileExists{\jobname-pw.ind}{\input{\jobname-pw.ind}}{}

% Quellenangabe nur in der Leseansicht
\ifkorrekturansicht\else
% Fallback-Definitionen, falls die .tex-Datei \titel etc. nicht gesetzt hat
\providecommand{\titel}{}
\providecommand{\editorInnen}{}
\providecommand{\dateiname}{\jobname}

\vspace{3cm}

\vfill

\footnotesize
\textsc{Quelle}: \titel. Herausgegeben von {\editorInnen}. In: \emph{Arthur Schnitzler: Briefwechsel mit Autorinnen und Autoren}.
 Digitale Edition, https://schnitzler-briefe.acdh.oeaw.ac.at/{\dateiname}.html (Stand \today)
\fi

\end{document}


