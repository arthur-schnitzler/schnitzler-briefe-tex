%% latex-leseansicht-vorspann.tex
%% Vorspann für die Leseansicht.
%% Lädt die gemeinsame Datei latex-vorspann.tex mit nicht gesetztem Schalter.

\newif\ifkorrekturansicht
\korrekturansichtfalse

\input{../tex-inputs/latex-vorspann}


         
         \renewcommand{\erwaehntePersonen}{Personen: Vincenz Czerny, Fedor Mamroth, Josef Rosengart, Julius Schnitzler}
         \renewcommand{\erwaehnteOrte}{Orte: Frankfurt am Main, Heidelberg, Wien}
         \renewcommand{\erwaehnteWerke}{}
               \section[ Paul Goldmann an Arthur Schnitzler, 20. 4. {[}1906{]}]{ Paul Goldmann an Arthur Schnitzler, 20. 4. {[}1906{]}}\nopagebreak\mylabel{v}\rehead{ }\begin{ledgroupsized}[t]{13cm}\normalsize\beginnumbering \toendnotes[C]{\smallbreak\pagebreak[2]} \Standort{DLA, A:Schnitzler, HS.NZ85.1.3175.}
\physDesc{Brief, 1 Blatt, 2 Seiten
\newline{}Handschrift: schwarze Tinte, deutsche Kurrent
\newline{}Schnitzler: mit Bleistift das Jahr »{[}1{]}906« vermerkt }\toendnotes[C]{\smallbreak}\pstart
           \centering{}{\pb}Frankfurt\oindex{Frankfurt am Main@\textbf{Frankfurt am Main}|pw}{ }20. April.\pend
           \pstart
           Lieber Freund, Ich danke Dir \textcolor{gray}{und}
               Deinem Bruder\pwindex{Schnitzler, Julius 13.07.1865 – 29.06.1939@\textsc{Schnitzler, Julius} (13.07.1865 – 29.06.1939), \emph{Chirurg}|pwv} auf das
               Herzlichſte für die raſche Antwort. Daß eine Autorität \strikeout{\textcolor{gray}{×}} wie Dein Bruder\pwindex{Schnitzler, Julius 13.07.1865 – 29.06.1939@\textsc{Schnitzler, Julius} (13.07.1865 – 29.06.1939), \emph{Chirurg}|pwv} zur
                  \label{K_L03244-11v}\edtext{Operation}{\lemma{\textnormal{\emph{Operation}}}\Cendnote{\textnormal{siehe Paul Goldmann an Arthur Schnitzler, 9. 4. [1906] und Paul Goldmann an Arthur Schnitzler, 16. 4. [1906]}}}\label{K_L03244-11h}{ }\strikeout{r} rät, iſt für uns wichtig zu
               wiſſen, und ich habe von meinem Schwager\pwindex{Rosengart, Josef 1860-02-08 – 1927-08-04@\textsc{Rosengart, Josef} (1860-02-08 – 1927-08-04), \emph{Arzt}|pwv}, der ſich ſchon entſchloſſen hatte, nichts weiter zu tun,
               wenigſtens erreicht, daß er nach Heidelberg\oindex{Heidelberg@\textbf{Heidelberg}|pw}
               fahren wird, um ſich mit \textsc{Czerny\pwindex{Czerny, Vincenz 1842-11-19 – 1916-10-03@\textsc{Czerny, Vincenz} (1842-11-19 – 1916-10-03), \emph{Chirurg, Arzt, Geheimer Rat}|pw}} zu beſprechen. Der Sitz des \textsc{tumors} iſt allerdings {\pb}ein derartiger, daß eine Operation faſt unmöglich
               erſcheint. Auch ſprechen ſtarke pſychiſche Gründe dagegen, indem man den Kranken\pwindex{Mamroth, Fedor 21.02.1851 – 25.06.1907@\textsc{Mamroth, Fedor} (21.02.1851 – 25.06.1907), \emph{Journalist, Kritiker}|pwv} nicht noch einmal zur
               Operation veranlaſſen kann, ohne ihm die volle Wahrheit zu ſagen. Immerhin, \textsc{Czerny\pwindex{Czerny, Vincenz 1842-11-19 – 1916-10-03@\textsc{Czerny, Vincenz} (1842-11-19 – 1916-10-03), \emph{Chirurg, Arzt, Geheimer Rat}|pw}} ſoll entſcheiden.\pend
           \pstart
           Dir und Deinem Bruder\pwindex{Schnitzler, Julius 13.07.1865 – 29.06.1939@\textsc{Schnitzler, Julius} (13.07.1865 – 29.06.1939), \emph{Chirurg}|pw} tauſend Dank für
               den Freundſchaftsdienſt, den Ihr mir geleiſtet habt, und viele treue Grüße! {\\[\baselineskip]}Dein
                  \spacefill\mbox{Paul Goldmnn}\pend
           \leftskip=0em{}
         
         \endnumbering\mylabel{h}\end{ledgroupsized}  \newcommand{\dateiname}{L03244}\newcommand{\titel}{Paul Goldmann an Arthur Schnitzler, 20. 4. [1906]}\newcommand{\editorInnen}{Martin Anton Müller und Laura Untner}%% latex-leseansicht-abspann.tex
%% Abspann für die Leseansicht.
%% Der Schalter \ifkorrekturansicht ist bereits durch den Vorspann gesetzt.

%% latex-abspann.tex
%% Gemeinsamer Abspann für Korrekturansicht und Leseansicht.
%% Setzt den Schalter \ifkorrekturansicht voraus (gesetzt in den
%% einbindenden Dateien latex-korrekturansicht-abspann.tex bzw.
%% latex-leseansicht-abspann.tex).
%% ---------------------------------------------------------------

\normalsize

% Das esempio-Environment wird nur in der Leseansicht benötigt
\ifkorrekturansicht\else
\newenvironment{esempio}[3]%
{
    \vspace{1.5ex}
    \rlap{\underline{#1}}
    \par
    \setlength{\parindent}{0cm}
    \nopagebreak
    \leftskip=#2cm
    \rightskip=#3cm
}
{
    \par
}
\fi

\doendnotes{C}
\bigskip
\vfill

\clearpage

\footnotesize

\ifkorrekturansicht
  \lohead{\textsc{register}}
\fi

% theindex-Environment neu definieren ohne reledmac
\makeatletter
\renewenvironment{theindex}{%
  \ifkorrekturansicht
    \section*{\indexname}%
  \else
    \subsubsection*{Index der erwähnten Entitäten}%
  \fi
  \setlength{\parindent}{0pt}%
  \setlength{\parskip}{0pt plus 0.3pt}%
  \let\item\@idxitem
}{%
  \ifkorrekturansicht\clearpage\fi
}
\makeatother

\IfFileExists{\jobname-pw.ind}{\input{\jobname-pw.ind}}{}

% Quellenangabe nur in der Leseansicht
\ifkorrekturansicht\else
% Fallback-Definitionen, falls die .tex-Datei \titel etc. nicht gesetzt hat
\providecommand{\titel}{}
\providecommand{\editorInnen}{}
\providecommand{\dateiname}{\jobname}

\vspace{3cm}

\vfill

\footnotesize
\textsc{Quelle}: \titel. Herausgegeben von {\editorInnen}. In: \emph{Arthur Schnitzler: Briefwechsel mit Autorinnen und Autoren}.
 Digitale Edition, https://schnitzler-briefe.acdh.oeaw.ac.at/{\dateiname}.html (Stand \today)
\fi

\end{document}


      