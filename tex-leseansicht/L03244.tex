%% latex-leseansicht-vorspann.tex
%% Vorspann für die Leseansicht.
%% Lädt die gemeinsame Datei latex-vorspann.tex mit nicht gesetztem Schalter.

\newif\ifkorrekturansicht
\korrekturansichtfalse

\input{../tex-inputs/latex-vorspann}


\section[ Paul Goldmann an Arthur Schnitzler, 20. 4. [1906]]{L03244 Paul Goldmann an Arthur Schnitzler,  20. 4. [1906]}
\nopagebreak\mylabel{L03244v}
\rehead{ }\normalsize\beginnumbering\briefempfaengerindex{Schnitzler, Arthur@\textsc{Schnitzler, Arthur}!zzzGoldmann, Paul@\emph{von Paul Goldmann}!1906-04-201@{20. 4. [1906]}|(be}
\toendnotes[C]{\smallbreak\pagebreak[2]}
\correspDesc{Versand  durch Paul Goldmann am 20. 4. [1906] in Frankfurt am Main
\newline{}Erhalt  durch Arthur Schnitzler im Zeitraum [21. 4. 1906
                  – 25. 4. 1906?] in Wien}\toendnotes[C]{\smallbreak}
\Standort{DLA, A:Schnitzler, HS.NZ85.1.3175.}
\physDesc{Brief, 1 Blatt, 2 Seiten, 774 Zeichen
\newline{}Handschrift: schwarze Tinte, deutsche Kurrent
\newline{}Schnitzler: mit Bleistift das Jahr »906« vermerkt }\toendnotes[C]{\smallbreak}
\pstart
           \centering{}{\pb}Frankfurt\oindex{Frankfurt am Main@\textbf{Frankfurt am Main}, \emph{Hauptstadt}|pw}{ }20. April.\pend
           \vspace{0.5em}
\pstart
           Lieber Freund, Ich danke Dir \textcolor{gray}{und}
               Deinem Bruder\pwindex{Schnitzler, Julius 13.\,7.\,1865 Wien – 29.\,6.\,1939 ebd.@\textsc{Schnitzler, Julius} (13.\,7.\,1865 Wien – 29.\,6.\,1939 ebd.), \emph{Chirurg}|pwv} auf das
               Herzlichſte für die raſche Antwort. Daß eine Autorität \strikeout{\textcolor{gray}{×}} wie Dein Bruder\pwindex{Schnitzler, Julius 13.\,7.\,1865 Wien – 29.\,6.\,1939 ebd.@\textsc{Schnitzler, Julius} (13.\,7.\,1865 Wien – 29.\,6.\,1939 ebd.), \emph{Chirurg}|pwv} zur
                  \label{K_L03244-1v}\edtext{Operation}{\lemma{\textnormal{\emph{Operation}}}\Cendnote{\textnormal{Siehe XXXX Auszeichnungsfehler: Dokument L03242 nicht gefunden und XXXX Auszeichnungsfehler: Dokument L03243 nicht gefunden.
               }}}\label{K_L03244-1}{ }\strikeout{r} rät, iſt für uns wichtig zu wiſſen, und ich habe
               von meinem Schwager\pwindex{Rosengart, Josef 8.\,2.\,1860 Laupheim – 4.\,8.\,1927 Frankfurt am Main@\textsc{Rosengart, Josef} (8.\,2.\,1860 Laupheim – 4.\,8.\,1927 Frankfurt am Main), \emph{Arzt}|pwv}, der{ }ſich{ }ſchon entſchloſſen hatte, nichts weiter zu tun, wenigſtens erreicht, daß er nach
                  Heidelberg\oindex{Heidelberg@\textbf{Heidelberg}, \emph{Hauptstadt}|pw} fahren wird, um{ }ſich mit \textsc{Czerny\pwindex{Czerny, Vincenz 19.\,11.\,1842 Trautenau – 3.\,10.\,1916 Heidelberg@\textsc{Czerny, Vincenz} (19.\,11.\,1842 Trautenau – 3.\,10.\,1916 Heidelberg), \emph{Chirurg, Arzt, Geheimer Rat}|pw}} zu beſprechen. Der Sitz des \textsc{tumors} iſt allerdings {\pb}ein derartiger, daß eine Operation faſt unmöglich
               erſcheint. Auch{ }ſprechen{ }ſtarke pſychiſche Gründe dagegen, indem man den Kranken\pwindex{Mamroth, Fedor 21.\,2.\,1851 Breslau – 25.\,6.\,1907 Frankfurt am Main@\textsc{Mamroth, Fedor} (21.\,2.\,1851 Breslau – 25.\,6.\,1907 Frankfurt am Main), \emph{Journalist, Kritiker}|pwv} nicht noch einmal zur
               Operation veranlaſſen kann, ohne ihm die volle Wahrheit zu{ }ſagen. Immerhin, \textsc{Czerny\pwindex{Czerny, Vincenz 19.\,11.\,1842 Trautenau – 3.\,10.\,1916 Heidelberg@\textsc{Czerny, Vincenz} (19.\,11.\,1842 Trautenau – 3.\,10.\,1916 Heidelberg), \emph{Chirurg, Arzt, Geheimer Rat}|pw}}{ }ſoll entſcheiden.\pend
           
\pstart
           Dir und Deinem Bruder\pwindex{Schnitzler, Julius 13.\,7.\,1865 Wien – 29.\,6.\,1939 ebd.@\textsc{Schnitzler, Julius} (13.\,7.\,1865 Wien – 29.\,6.\,1939 ebd.), \emph{Chirurg}|pw} tauſend Dank für
               den Freundſchaftsdienſt, den Ihr mir geleiſtet habt, und viele treue Grüße! {\\[\baselineskip]}Dein
                  \spacefill\mbox{Paul Goldmnn}\pend
           \leftskip=0em{}\selectlanguage{ngerman}\endnumbering\briefempfaengerindex{Schnitzler, Arthur@\textsc{Schnitzler, Arthur}!zzzGoldmann, Paul@\emph{von Paul Goldmann}!1906-04-201@{20. 4. [1906]}|)be}\mylabel{L03244h}  \newcommand{\dateiname}{L03244}\newcommand{\titel}{Paul Goldmann an Arthur Schnitzler, 20. 4. [1906]}\newcommand{\editorInnen}{Martin Anton Müller und Laura Untner}%% latex-leseansicht-abspann.tex
%% Abspann für die Leseansicht.
%% Der Schalter \ifkorrekturansicht ist bereits durch den Vorspann gesetzt.

%% latex-abspann.tex
%% Gemeinsamer Abspann für Korrekturansicht und Leseansicht.
%% Setzt den Schalter \ifkorrekturansicht voraus (gesetzt in den
%% einbindenden Dateien latex-korrekturansicht-abspann.tex bzw.
%% latex-leseansicht-abspann.tex).
%% ---------------------------------------------------------------

\normalsize

% Das esempio-Environment wird nur in der Leseansicht benötigt
\ifkorrekturansicht\else
\newenvironment{esempio}[3]%
{
    \vspace{1.5ex}
    \rlap{\underline{#1}}
    \par
    \setlength{\parindent}{0cm}
    \nopagebreak
    \leftskip=#2cm
    \rightskip=#3cm
}
{
    \par
}
\fi

\doendnotes{C}
\bigskip
\vfill

\clearpage

\footnotesize

\ifkorrekturansicht
  \lohead{\textsc{register}}
\fi

% theindex-Environment neu definieren ohne reledmac
\makeatletter
\renewenvironment{theindex}{%
  \ifkorrekturansicht
    \section*{\indexname}%
  \else
    \subsubsection*{Index der erwähnten Entitäten}%
  \fi
  \setlength{\parindent}{0pt}%
  \setlength{\parskip}{0pt plus 0.3pt}%
  \let\item\@idxitem
}{%
  \ifkorrekturansicht\clearpage\fi
}
\makeatother

\IfFileExists{\jobname-pw.ind}{\input{\jobname-pw.ind}}{}

% Quellenangabe nur in der Leseansicht
\ifkorrekturansicht\else
% Fallback-Definitionen, falls die .tex-Datei \titel etc. nicht gesetzt hat
\providecommand{\titel}{}
\providecommand{\editorInnen}{}
\providecommand{\dateiname}{\jobname}

\vspace{3cm}

\vfill

\footnotesize
\textsc{Quelle}: \titel. Herausgegeben von {\editorInnen}. In: \emph{Arthur Schnitzler: Briefwechsel mit Autorinnen und Autoren}.
 Digitale Edition, https://schnitzler-briefe.acdh.oeaw.ac.at/{\dateiname}.html (Stand \today)
\fi

\end{document}


