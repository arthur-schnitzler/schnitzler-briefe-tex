%% latex-korrekturansicht-vorspann.tex
%% Vorspann für die Korrekturansicht.
%% Lädt die gemeinsame Datei latex-vorspann.tex mit gesetztem Schalter.

\newif\ifkorrekturansicht
\korrekturansichttrue

\input{../tex-inputs/latex-vorspann}


\section[ Paul Goldmann an Arthur Schnitzler, 20. 4. {[}1906{]}]{L03244 Paul Goldmann an Arthur Schnitzler, 20. 4. {[}1906{]}}
\nopagebreak\mylabel{L03244v}
\rehead{ }\normalsize\beginnumbering\briefempfaengerindex{Schnitzler, Arthur@\textsc{Schnitzler, Arthur}!zzzGoldmann, Paul@\emph{von Paul Goldmann}!1906-04-201@{20. 4. {[}1906{]}}|(be}
\toendnotes[C]{\smallbreak\pagebreak[2]}\Standort{DLA, A:Schnitzler, HS.NZ85.1.3175.}
\physDesc{Brief, 1 Blatt, 2 Seiten, 774 Zeichen
\newline{}Handschrift: schwarze Tinte, deutsche Kurrent
\newline{}Schnitzler: mit Bleistift das Jahr »906« vermerkt }\toendnotes[C]{\smallbreak}
\pstart
           \centering{}{\pb}Frankfurt\oindex{Frankfurt am Main@\textbf{Frankfurt am Main}, \emph{P.PPLA3}|pw}{ }20. April.\pend
           \vspace{0.5em}
\pstart
           Lieber Freund, Ich danke Dir \textcolor{gray}{und}
               Deinem Bruder\pwindex{Schnitzler, Julius 13.07.1865 – 29.06.1939@\textsc{Schnitzler, Julius} (13.07.1865 – 29.06.1939), \emph{Chirurg/Chirurgin}|pwv} auf das
               Herzlichſte für die raſche Antwort. Daß eine Autorität \strikeout{\textcolor{gray}{×}} wie Dein Bruder\pwindex{Schnitzler, Julius 13.07.1865 – 29.06.1939@\textsc{Schnitzler, Julius} (13.07.1865 – 29.06.1939), \emph{Chirurg/Chirurgin}|pwv} zur
                  \label{K_L03244-1v}\edtext{Operation}{\lemma{\textnormal{\emph{Operation}}}\Cendnote{\textnormal{Siehe Paul Goldmann an Arthur Schnitzler, 9. 4. [1906] und Paul Goldmann an Arthur Schnitzler, 16. 4. [1906].
               }}}\label{K_L03244-1}{ }\strikeout{r} rät, iſt für uns wichtig zu wiſſen, und ich habe
               von meinem Schwager\pwindex{Rosengart, Josef 1860-02-08 – 1927-08-04@\textsc{Rosengart, Josef} (1860-02-08 – 1927-08-04), \emph{Arzt/Ärztin}|pwv}, der
               ſich ſchon entſchloſſen hatte, nichts weiter zu tun, wenigſtens erreicht, daß er nach
                  Heidelberg\oindex{Heidelberg@\textbf{Heidelberg}, \emph{P.PPLA3}|pw} fahren wird, um ſich mit \textsc{Czerny\pwindex{Czerny, Vincenz 1842-11-19 – 1916-10-03@\textsc{Czerny, Vincenz} (1842-11-19 – 1916-10-03), \emph{Chirurg/Chirurgin, Arzt/Ärztin, Geheimer Rat/Geheime Rätin}|pw}} zu beſprechen. Der Sitz des \textsc{tumors} iſt allerdings {\pb}ein derartiger, daß eine Operation faſt unmöglich
               erſcheint. Auch ſprechen ſtarke pſychiſche Gründe dagegen, indem man den Kranken\pwindex{Mamroth, Fedor 21.02.1851 – 25.06.1907@\textsc{Mamroth, Fedor} (21.02.1851 – 25.06.1907), \emph{Journalist/Journalistin, Kritiker/Kritikerin}|pwv} nicht noch einmal zur
               Operation veranlaſſen kann, ohne ihm die volle Wahrheit zu ſagen. Immerhin, \textsc{Czerny\pwindex{Czerny, Vincenz 1842-11-19 – 1916-10-03@\textsc{Czerny, Vincenz} (1842-11-19 – 1916-10-03), \emph{Chirurg/Chirurgin, Arzt/Ärztin, Geheimer Rat/Geheime Rätin}|pw}} ſoll entſcheiden.\pend
           
\pstart
           Dir und Deinem Bruder\pwindex{Schnitzler, Julius 13.07.1865 – 29.06.1939@\textsc{Schnitzler, Julius} (13.07.1865 – 29.06.1939), \emph{Chirurg/Chirurgin}|pw} tauſend Dank für
               den Freundſchaftsdienſt, den Ihr mir geleiſtet habt, und viele treue Grüße! {\\[\baselineskip]}Dein
                  \spacefill\mbox{Paul Goldmnn}\pend
           \leftskip=0em{}\selectlanguage{ngerman}\endnumbering\briefempfaengerindex{Schnitzler, Arthur@\textsc{Schnitzler, Arthur}!zzzGoldmann, Paul@\emph{von Paul Goldmann}!1906-04-201@{20. 4. {[}1906{]}}|)be}\mylabel{L03244h}  \normalsize

\doendnotes{C}
\bigskip
\vfill

\clearpage

\footnotesize

\lohead{\textsc{register}}

% Definiere theindex-Environment komplett neu ohne reledmac
\makeatletter
\renewenvironment{theindex}{%
  \section*{\indexname}%
  \setlength{\parindent}{0pt}%
  \setlength{\parskip}{0pt plus 0.3pt}%
  \let\item\@idxitem
}{%
  \clearpage
}
\makeatother

\IfFileExists{\jobname-pw.ind}{\input{\jobname-pw.ind}}{}

\end{document}

      