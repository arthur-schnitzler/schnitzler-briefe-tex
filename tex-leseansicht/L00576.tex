%% latex-leseansicht-vorspann.tex
%% Vorspann für die Leseansicht.
%% Lädt die gemeinsame Datei latex-vorspann.tex mit nicht gesetztem Schalter.

\newif\ifkorrekturansicht
\korrekturansichtfalse

\input{../tex-inputs/latex-vorspann}


\section[Felix Salten und Hugo von Hofmannsthal an Arthur Schnitzler und Richard Beer-Hofmann, 1. 8. 1896]{L00576 Felix Salten und Hugo von Hofmannsthal an Arthur Schnitzler und Richard
               Beer-Hofmann, 1. 8. 1896}
\nopagebreak\mylabel{L00576v}
\rehead{ }\normalsize\beginnumbering\briefempfaengerindex{Beer-Hofmann, Richard@\textsc{Beer-Hofmann, Richard}!zzzSalten, Felix@\emph{von Felix Salten}!1896-08-011@{1. 8. 1896}|(be}\briefempfaengerindex{Beer-Hofmann, Richard@\textsc{Beer-Hofmann, Richard}!zzzHofmannsthal, Hugo von@\emph{von Hugo von Hofmannsthal}!1896-08-011@{1. 8. 1896}|(be}\briefempfaengerindex{Schnitzler, Arthur@\textsc{Schnitzler, Arthur}!zzzSalten, Felix@\emph{von Felix Salten}!1896-08-011@{1. 8. 1896}|(be}\briefempfaengerindex{Schnitzler, Arthur@\textsc{Schnitzler, Arthur}!zzzHofmannsthal, Hugo von@\emph{von Hugo von Hofmannsthal}!1896-08-011@{1. 8. 1896}|(be}
\toendnotes[C]{\smallbreak\pagebreak[2]}
\correspDesc{Versand  durch Hugo von Hofmannsthal, Felix Salten am 1. 8. 1896 in Bad Ischl
\newline{}Erhalt  durch Arthur Schnitzler, Richard Beer-Hofmann am 3. 8. 1896 in Kopenhagen}\toendnotes[C]{\smallbreak}
\Standort{CUL, Schnitzler, B 89.}
\physDesc{Postkarte, 411 Zeichen
\newline{}Handschrift Felix Salten: Bleistift, lateinische Kurrent
\newline{}Handschrift Hugo von Hofmannsthal: Bleistift, lateinische Kurrent
\newline{}Versand: 1) Stempel: »\nobreak{}\oindex{Bad Ischl@\textbf{Bad Ischl}|pwk}Ischl, 1 8 {[}96{]}, A\nobreak{}«.   2) Stempel: »\nobreak{}\oindex{Kopenhagen@\textbf{Kopenhagen}, \emph{Hauptstadt}|pwk}Kjøbenhavn, 3–8 96, 20MB\nobreak{}«. 
\newline{}Ordnung: 1) mit Bleistift von unbekannter Hand die Jahreszahl »1896« bei
                                 der geschriebenen Datumsangabe ergänzt  2) mit Bleistift von unbekannter Hand nummeriert: »75«
\newline{}Editorischer Hinweis: Hofmannsthals Ergänzungen in den Textklammern sind gut
                                 erkennbar, weswegen hier auf die Auszeichung der vielen
                                 Schriftwechsel verzichtet wird }\toendnotes[C]{\smallbreak}\pstart{}{\pb}Herrn D\textsuperscript{r} Arthur Schnitzler\pend{}\pstart{}Kopenhagen\oindex{Kopenhagen@\textbf{Kopenhagen}, \emph{Hauptstadt}|pw}\pend{}\pstart{}Dänemark\oindex{Dänemark@\textbf{Dänemark}|pw}\pend{}\pstart{}poste restante\pend{}{\bigskip}\vspace{1em}
\pstart
           {\pb}{[}hs. Hofmannsthal:{]} Für \uline{Arthur {\kaufmannsund} Richard}\pend
           
\pstart
           \raggedleft{}{[}hs. Salten:{]} Ischl\oindex{Bad Ischl@\textbf{Bad Ischl}|pw}, 1. August\pend
           \vspace{0.5em}
\pstart
           Wir haben uns zufällig getroffen, und da hat er mir {[}hs. Hofmannsthal:{]} (ich ihm) {[}hs. Salten:{]} natürlich gleich eine \label{K_L00576-1v}\edtext{Novelle\pwindex{Hofmannsthal, Hugo von 1.\,2.\,1874 Wien – 15.\,7.\,1929 Rodaun@\textsc{Hofmannsthal, Hugo von} (1.\,2.\,1874 Wien – 15.\,7.\,1929 Rodaun), \emph{Schriftsteller}!Geschichte der beiden Liebespaare@\strich\emph{Geschichte der beiden Liebespaare}|pwv}}{\lemma{\textnormal{\emph{Novelle}}}\Cendnote{\textnormal{\emph{Geschichte der beiden Liebespaare}\pwindex{Hofmannsthal, Hugo von 1.\,2.\,1874 Wien – 15.\,7.\,1929 Rodaun@\textsc{Hofmannsthal, Hugo von} (1.\,2.\,1874 Wien – 15.\,7.\,1929 Rodaun), \emph{Schriftsteller}!Geschichte der beiden Liebespaare@\strich\emph{Geschichte der beiden Liebespaare}|pwk}}}}\label{K_L00576-1} vorgelesen. {[}hs. Hofmannsthal:{]} Sie hat ihm {[}hs. Salten:{]} (mir) {[}hs. Hofmannsthal:{]} recht gut {[}hs. Salten:{]} (sehr gut! das »recht gut« ist nur {[}hs. Hofmannsthal:{]} meine {[}hs. Salten:{]} ((seine)) Bescheidenheit)
               gefallen. {[}hs. Hofmannsthal:{]} Natürlich ist er {[}hs. Salten:{]} (ich) {[}hs. Hofmannsthal:{]} \uline{sofort} wieder \label{K_L00576-2v}\edtext{abgereist}{\lemma{\textnormal{\emph{abgereist}}}\Cendnote{\textnormal{Hofmannsthal\pwindex{Hofmannsthal, Hugo von 1.\,2.\,1874 Wien – 15.\,7.\,1929 Rodaun@\textsc{Hofmannsthal, Hugo von} (1.\,2.\,1874 Wien – 15.\,7.\,1929 Rodaun), \emph{Schriftsteller}|pwk} urlaubte im gut 25 km
                     entfernten Aussee\oindex{Bad Aussee@\textbf{Bad Aussee}, \emph{Hauptstadt}|pwk}.}}}\label{K_L00576-2}. {[}hs. Salten:{]} Das hat er {[}hs. Hofmannsthal:{]} (habe
               ich) {[}hs. Salten:{]} seit sechs Wochen vorher gewusst. Dies wünscht Euch\pend
           
\pstart
           \spacefill\mbox{Salten}{\\[\baselineskip]}\spacefill\mbox{{[}hs. Hofmannsthal:{]} Hugo}\pend
           \leftskip=0em{}\selectlanguage{ngerman}\endnumbering\briefempfaengerindex{Beer-Hofmann, Richard@\textsc{Beer-Hofmann, Richard}!zzzSalten, Felix@\emph{von Felix Salten}!1896-08-011@{1. 8. 1896}|)be}\briefempfaengerindex{Beer-Hofmann, Richard@\textsc{Beer-Hofmann, Richard}!zzzHofmannsthal, Hugo von@\emph{von Hugo von Hofmannsthal}!1896-08-011@{1. 8. 1896}|)be}\briefempfaengerindex{Schnitzler, Arthur@\textsc{Schnitzler, Arthur}!zzzSalten, Felix@\emph{von Felix Salten}!1896-08-011@{1. 8. 1896}|)be}\briefempfaengerindex{Schnitzler, Arthur@\textsc{Schnitzler, Arthur}!zzzHofmannsthal, Hugo von@\emph{von Hugo von Hofmannsthal}!1896-08-011@{1. 8. 1896}|)be}\mylabel{L00576h}  \newcommand{\dateiname}{L00576}\newcommand{\titel}{Felix Salten und Hugo von Hofmannsthal an Arthur Schnitzler und Richard Beer-Hofmann, 1. 8. 1896}\newcommand{\editorInnen}{Martin Anton Müller und Gerd-Hermann Susen}%% latex-leseansicht-abspann.tex
%% Abspann für die Leseansicht.
%% Der Schalter \ifkorrekturansicht ist bereits durch den Vorspann gesetzt.

%% latex-abspann.tex
%% Gemeinsamer Abspann für Korrekturansicht und Leseansicht.
%% Setzt den Schalter \ifkorrekturansicht voraus (gesetzt in den
%% einbindenden Dateien latex-korrekturansicht-abspann.tex bzw.
%% latex-leseansicht-abspann.tex).
%% ---------------------------------------------------------------

\normalsize

% Das esempio-Environment wird nur in der Leseansicht benötigt
\ifkorrekturansicht\else
\newenvironment{esempio}[3]%
{
    \vspace{1.5ex}
    \rlap{\underline{#1}}
    \par
    \setlength{\parindent}{0cm}
    \nopagebreak
    \leftskip=#2cm
    \rightskip=#3cm
}
{
    \par
}
\fi

\doendnotes{C}
\bigskip
\vfill

\clearpage

\footnotesize

\ifkorrekturansicht
  \lohead{\textsc{register}}
\fi

% theindex-Environment neu definieren ohne reledmac
\makeatletter
\renewenvironment{theindex}{%
  \ifkorrekturansicht
    \section*{\indexname}%
  \else
    \subsubsection*{Index der erwähnten Entitäten}%
  \fi
  \setlength{\parindent}{0pt}%
  \setlength{\parskip}{0pt plus 0.3pt}%
  \let\item\@idxitem
}{%
  \ifkorrekturansicht\clearpage\fi
}
\makeatother

\IfFileExists{\jobname-pw.ind}{\input{\jobname-pw.ind}}{}

% Quellenangabe nur in der Leseansicht
\ifkorrekturansicht\else
% Fallback-Definitionen, falls die .tex-Datei \titel etc. nicht gesetzt hat
\providecommand{\titel}{}
\providecommand{\editorInnen}{}
\providecommand{\dateiname}{\jobname}

\vspace{3cm}

\vfill

\footnotesize
\textsc{Quelle}: \titel. Herausgegeben von {\editorInnen}. In: \emph{Arthur Schnitzler: Briefwechsel mit Autorinnen und Autoren}.
 Digitale Edition, https://schnitzler-briefe.acdh.oeaw.ac.at/{\dateiname}.html (Stand \today)
\fi

\end{document}


