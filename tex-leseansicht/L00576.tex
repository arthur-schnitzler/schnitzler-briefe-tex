%% latex-korrekturansicht-vorspann.tex
%% Vorspann für die Korrekturansicht.
%% Lädt die gemeinsame Datei latex-vorspann.tex mit gesetztem Schalter.

\newif\ifkorrekturansicht
\korrekturansichttrue

\input{../tex-inputs/latex-vorspann}


\section[Felix Salten und Hugo von Hofmannsthal an Arthur Schnitzler und Richard Beer-Hofmann, 1. 8. 1896]{L00576 Felix Salten und Hugo von Hofmannsthal an Arthur Schnitzler und Richard
               Beer-Hofmann, 1. 8. 1896}
\nopagebreak\mylabel{L00576v}
\rehead{ }\normalsize\beginnumbering\briefempfaengerindex{Beer-Hofmann, Richard@\textsc{Beer-Hofmann, Richard}!zzzSalten, Felix@\emph{von Felix Salten}!1896-08-011@{1. 8. 1896}|(be}\briefempfaengerindex{Beer-Hofmann, Richard@\textsc{Beer-Hofmann, Richard}!zzzHofmannsthal, Hugo von@\emph{von Hugo von Hofmannsthal}!1896-08-011@{1. 8. 1896}|(be}\briefempfaengerindex{Schnitzler, Arthur@\textsc{Schnitzler, Arthur}!zzzSalten, Felix@\emph{von Felix Salten}!1896-08-011@{1. 8. 1896}|(be}\briefempfaengerindex{Schnitzler, Arthur@\textsc{Schnitzler, Arthur}!zzzHofmannsthal, Hugo von@\emph{von Hugo von Hofmannsthal}!1896-08-011@{1. 8. 1896}|(be}
\toendnotes[C]{\smallbreak\pagebreak[2]}\Standort{CUL, Schnitzler, B 89.}
\physDesc{Postkarte, 411 Zeichen
\newline{}Handschrift Felix Salten: Bleistift, lateinische Kurrent
\newline{}Handschrift Hugo von Hofmannsthal: Bleistift, lateinische Kurrent
\newline{}Versand: 1) Stempel: »\nobreak{}\oindex{Bad Ischl@\textbf{Bad Ischl}, \emph{P.PPL}|pwk}Ischl, 1 8 {[}96{]}, A\nobreak{}«.   2) Stempel: »\nobreak{}\oindex{Kopenhagen@\textbf{Kopenhagen}, \emph{P.PPLC}|pwk}Kjøbenhavn, 3–8 96, 20MB\nobreak{}«. 
\newline{}Ordnung: 1) mit Bleistift von unbekannter Hand die Jahreszahl »1896« bei
                                 der geschriebenen Datumsangabe ergänzt  2) mit Bleistift von unbekannter Hand nummeriert: »75«
\newline{}Editorischer Hinweis: Hofmannsthals Ergänzungen in den Textklammern sind gut
                                 erkennbar, weswegen hier auf die Auszeichung der vielen
                                 Schriftwechsel verzichtet wird }\toendnotes[C]{\smallbreak}\pstart{}{\pb}Herrn D\textsuperscript{r} Arthur Schnitzler\pend{}\pstart{}Kopenhagen\oindex{Kopenhagen@\textbf{Kopenhagen}, \emph{P.PPLC}|pw}\pend{}\pstart{}Dänemark\oindex{Daenemark@\textbf{Dänemark}, \emph{A.PCLI}|pw}\pend{}\pstart{}poste restante\pend{}{\bigskip}\vspace{1em}
\pstart
           {\pb}{[}hs. :{]} Für \uline{Arthur {\kaufmannsund} Richard}\pend
           
\pstart
           \raggedleft{}{[}hs. :{]} Ischl\oindex{Bad Ischl@\textbf{Bad Ischl}, \emph{P.PPL}|pw}, 1. August\pend
           \vspace{0.5em}
\pstart
           Wir haben uns zufällig getroffen, und da hat er mir {[}hs. :{]} (ich ihm) {[}hs. :{]} natürlich gleich eine \label{K_L00576-1v}\edtext{Novelle\pwindex{Geschichte der beiden Liebespaare@\emph{Geschichte der beiden Liebespaare}|pwv}}{\lemma{\textnormal{\emph{Novelle}}}\Cendnote{\textnormal{\emph{Geschichte der beiden Liebespaare}\pwindex{Geschichte der beiden Liebespaare@\emph{Geschichte der beiden Liebespaare}|pwk}}}}\label{K_L00576-1} vorgelesen. {[}hs. :{]} Sie hat ihm {[}hs. :{]} (mir) {[}hs. :{]} recht gut {[}hs. :{]} (sehr gut! das »recht gut« ist nur {[}hs. :{]} meine {[}hs. :{]} ((seine)) Bescheidenheit)
               gefallen. {[}hs. :{]} Natürlich ist er {[}hs. :{]} (ich) {[}hs. :{]} \uline{sofort} wieder \label{K_L00576-2v}\edtext{abgereist}{\lemma{\textnormal{\emph{abgereist}}}\Cendnote{\textnormal{Hofmannsthal\pwindex{Hofmannsthal, Hugo von 1874-02-01 – 1929-07-15@\textsc{Hofmannsthal, Hugo von} (1874-02-01 – 1929-07-15), \emph{Schriftsteller/Schriftstellerin}|pwk} urlaubte im gut 25 km
                     entfernten Aussee\oindex{Bad Aussee@\textbf{Bad Aussee}, \emph{P.PPLA3}|pwk}.}}}\label{K_L00576-2}. {[}hs. :{]} Das hat er {[}hs. :{]} (habe
               ich) {[}hs. :{]} seit sechs Wochen vorher gewusst. Dies wünscht Euch\pend
           
\pstart
           \spacefill\mbox{Salten}{\\[\baselineskip]}\spacefill\mbox{{[}hs. :{]} Hugo}\pend
           \leftskip=0em{}\selectlanguage{ngerman}\endnumbering\briefempfaengerindex{Beer-Hofmann, Richard@\textsc{Beer-Hofmann, Richard}!zzzSalten, Felix@\emph{von Felix Salten}!1896-08-011@{1. 8. 1896}|)be}\briefempfaengerindex{Beer-Hofmann, Richard@\textsc{Beer-Hofmann, Richard}!zzzHofmannsthal, Hugo von@\emph{von Hugo von Hofmannsthal}!1896-08-011@{1. 8. 1896}|)be}\briefempfaengerindex{Schnitzler, Arthur@\textsc{Schnitzler, Arthur}!zzzSalten, Felix@\emph{von Felix Salten}!1896-08-011@{1. 8. 1896}|)be}\briefempfaengerindex{Schnitzler, Arthur@\textsc{Schnitzler, Arthur}!zzzHofmannsthal, Hugo von@\emph{von Hugo von Hofmannsthal}!1896-08-011@{1. 8. 1896}|)be}\mylabel{L00576h}  \normalsize

\doendnotes{C}
\bigskip
\vfill

\clearpage

\footnotesize

\lohead{\textsc{register}}

% Definiere theindex-Environment komplett neu ohne reledmac
\makeatletter
\renewenvironment{theindex}{%
  \section*{\indexname}%
  \setlength{\parindent}{0pt}%
  \setlength{\parskip}{0pt plus 0.3pt}%
  \let\item\@idxitem
}{%
  \clearpage
}
\makeatother

\IfFileExists{\jobname-pw.ind}{\input{\jobname-pw.ind}}{}

\end{document}

      