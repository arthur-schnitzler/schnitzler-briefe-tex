%% latex-leseansicht-vorspann.tex
%% Vorspann für die Leseansicht.
%% Lädt die gemeinsame Datei latex-vorspann.tex mit nicht gesetztem Schalter.

\newif\ifkorrekturansicht
\korrekturansichtfalse

\input{../tex-inputs/latex-vorspann}


\section[Paul Goldmann an Arthur Schnitzler, 16. 5. 1891]{L02663 Paul Goldmann an Arthur Schnitzler, 16. 5. 1891}
\nopagebreak\mylabel{L02663v}
\rehead{ }\normalsize\beginnumbering\briefempfaengerindex{Schnitzler, Arthur@\textsc{Schnitzler, Arthur}!zzzGoldmann, Paul@\emph{von Paul Goldmann}!1891-05-161@{16. 5. 1891}|(be}
\toendnotes[C]{\smallbreak\pagebreak[2]}
\correspDesc{Versand  durch Paul Goldmann am 16. 5. 1891 in Brüssel
\newline{}Erhalt  durch Arthur Schnitzler im Zeitraum [17. 5. 1891
                  – 21. 5. 1891?] in Wien}\toendnotes[C]{\smallbreak}
\Standort{DLA, A:Schnitzler, HS.NZ85.1.3162.}
\physDesc{Brief, 2 Blätter, 5 Seiten, 5480 Zeichen
\newline{}Handschrift: schwarze Tinte, deutsche Kurrent}\toendnotes[C]{\smallbreak}
\pstart
           \centering{}{\pb}\textsc{Brüssel\oindex{Brüssel@\textbf{Brüssel}, \emph{Hauptstadt}|pw}}{ }16. \textsc{mai} 1891.\pend
           
\pstart\center{}Mein lieber Arthur!\pend\vspace{0.5em}
\pstart
           Dein Brief als erſten Freundesgruß i\substVorne{}\textsuperscript{m}\substDazwischen{}n\substHinten{} fremder Stadt\oindex{Brüssel@\textbf{Brüssel}, \emph{Hauptstadt}|pwv} – das
               hat mir aufrichtig wohlgethan. Sei von Herzen bedankt für Deine Treue! {\dots} Wenn ich Dir von unterwegs eine Karte{ }ſchickte,{ }ſo
               geschah das nicht, um Dir zu{ }ſchreiben,{ }ſondern um Dir einen Beweis zu geben, daß ich
               mitten im \textcolor{gray}{Wirrwar} der neuen Eindrücke und im Fieber der Arbeit
               Deiner denke. Das war eine harte Zeit – dieſe{ }ſechs Tage\textcolor{gray}{.} Morgens
               in der Regel um fünf Uhr aufſtehen, um die Bergleute noch vor der Einfahrt in den
               Schacht zu{ }ſehen,{ }ſtundenlang im glühenden Sonnenbrand über{ }ſtaubige \label{K_L02663-1v}\edtext{\begin{otherlanguage}{french}Chauſſéen\end{otherlanguage}}{\lemma{\textnormal{\emph{Chausséen}}}\Cendnote{\textnormal{französisch: Landstraßen}}}\label{K_L02663-1} wandern,{ }ſich täglich von \textcolor{gray}{vertrackten} Localbahnen das Herz aus dem Leibe{ }ſchütteln laſſen, und Abends, todtmüde, den Bericht{ }ſchreiben (um ihn dann\strikeout{)}, einige Tage{ }ſpäter, elend zuſammengeſtrichen oder
               gar nicht im Blatte\pwindex{Frankfurter Zeitung@\emph{Frankfurter Zeitung}|pwv} zu
               finden). Endlich \strikeout{i\textcolor{gray}{n}} bin ich heut nach Brüſſel\oindex{Brüssel@\textbf{Brüssel}, \emph{Hauptstadt}|pw} gekommen; aber{ }ſei es nun die Nervenreaction gegen die
               Überanſtrengung der letzten Tage,{ }ſei es das Erwachen des Bewußtſeins aus dem Rauſche
               der Arbeit – ich fühle mich todtenbang und pſychiſch elend. Und als ich Deinen Brief
               las, war es ein veritables tiefes, tiefes Heimweh nach Wien\oindex{Wien@\textbf{Wien}, \emph{Verwaltungsgebiet}|pw}, das mir durch das Herz{ }ſchnitt, {\pb}wie
               nur ein Heimweh{ }ſchneiden kann. Und es war nicht blos ein Heimweh nach Wien\oindex{Wien@\textbf{Wien}, \emph{Verwaltungsgebiet}|pw},{ }ſondern eine Sehnſucht nach der beſſeren Welt
               dort, die ich auf immer verloren. Du kennſt ja meinen Neid mit der umgekehrten
               Spitze, der{ }ſich nicht gegen den Andern{ }ſondern gegen mich{ }ſelbſt kehrt. Und{ }ſo war
               es mir ein gar bitteres Gefühl, als ich von Deinen Erfolgen las, daß ich{ }ſo ganz aus
               der Reihe Jener geriſſen bin, die nach dem hohen Ziele{ }ſtreben, das nicht mehr das
                  \strikeout{d\textcolor{gray}{e}} meine{ }ſein darf. Wir{ }ſind eine Zeitlang Seite an Seite gewandert; jetzt bin
               ich an einem Stein am Wege unterwegs{ }ſtehen geblieben und{ }ſehe Dir wehmüthig nach,
               wie Du emporſteigſt. Das iſt die Schlacke, die meine Empfindung der Freude an deinen
               Erfolgen aufſetzt; wir{ }ſind eben Alle keine Menſchen der reinen Empfindungen; vom
               Herzen, dem d\substVorne{}\textsuperscript{a}\substDazwischen{}ie\substHinten{}s Gefühl entſtrömt, tropft immer ein wenig Ich mit hinein. Ich{ }ſage Dir das
               eigentlich nur, um auf der andern Seite das Recht zu haben, von der warmen
               Aufrichtigkeit meiner Mitfreude zu{ }ſprechen. Nur{ }ſo weiter! Stark und tapfer! Und ich
               habe nur einen Wunſch für Dich: daß \strikeout{\textcolor{gray}{al}} Dir \strikeout{gelingen} die Kraft werde, all’ das Schöne
               aus Dir herauszuarbeiten, was – meiner feſten Überzeugung nach – in Dir{ }ſteckt. Die
               Kritiken{ }ſchickſt Du mir wohl alle; Du bekommſt{ }ſie pünktlich zurück; ebenſo werde
               ich Dich, wenn ich mich erſt ein wenig eingearbeitet und mir Zeit genommen habe, um
               alle \uline{drei} Acte des Stück\pwindex{Schnitzler, Arthur 15.\,5.\,1862 Wien – 21.\,10.\,1931 ebd.@\textsc{Schnitzler, Arthur} (15.\,5.\,1862 Wien – 21.\,10.\,1931 ebd.), \emph{Schriftsteller, Mediziner}!Märchen. Schauspiel in drei Aufzügen@\strich\emph{Das Märchen. Schauspiel in drei Aufzügen}|pwv}es
                {\pb}bitten. Desgleichen{ }ſollſt Du mir \uline{bald}{ }Folgendes schreiben: 1.) wie Du Deinen Tag
               verbringſt, mit genauer trockener Aufzählung der regelmäßigen Beſchäftigung von Früh
               bis Abend 2.) ob \label{K_L02663-2v}\edtext{\textsc{Schwarzkopf\pwindex{Schwarzkopf, Gustav 7.\,11.\,1853 Wien – 13.\,11.\,1939 ebd.@\textsc{Schwarzkopf, Gustav} (7.\,11.\,1853 Wien – 13.\,11.\,1939 ebd.), \emph{Schriftsteller}|pw}} dein Stück\pwindex{Schnitzler, Arthur 15.\,5.\,1862 Wien – 21.\,10.\,1931 ebd.@\textsc{Schnitzler, Arthur} (15.\,5.\,1862 Wien – 21.\,10.\,1931 ebd.), \emph{Schriftsteller, Mediziner}!Märchen. Schauspiel in drei Aufzügen@\strich\emph{Das Märchen. Schauspiel in drei Aufzügen}|pwv}
               bereits geleſen}{\lemma{\textnormal{\emph{Schwarzkopf … gelesen}}}\Cendnote{\textnormal{Gustav Schwarzkopf\pwindex{Schwarzkopf, Gustav 7.\,11.\,1853 Wien – 13.\,11.\,1939 ebd.@\textsc{Schwarzkopf, Gustav} (7.\,11.\,1853 Wien – 13.\,11.\,1939 ebd.), \emph{Schriftsteller}|pwk} dürfte \emph{Das Märchen}\pwindex{Schnitzler, Arthur 15.\,5.\,1862 Wien – 21.\,10.\,1931 ebd.@\textsc{Schnitzler, Arthur} (15.\,5.\,1862 Wien – 21.\,10.\,1931 ebd.), \emph{Schriftsteller, Mediziner}!Märchen. Schauspiel in drei Aufzügen@\strich\emph{Das Märchen. Schauspiel in drei Aufzügen}|pwk} erst am 25. 6. 1891 kennengelernt haben, als Schnitzler es ihm und anderen Freunden\pwindex{Schwarzkopf, Gustav 7.\,11.\,1853 Wien – 13.\,11.\,1939 ebd.@\textsc{Schwarzkopf, Gustav} (7.\,11.\,1853 Wien – 13.\,11.\,1939 ebd.), \emph{Schriftsteller}|pwkv}\pwindex{Hofmannsthal, Hugo von 1.\,2.\,1874 Wien – 15.\,7.\,1929 Rodaun@\textsc{Hofmannsthal, Hugo von} (1.\,2.\,1874 Wien – 15.\,7.\,1929 Rodaun), \emph{Schriftsteller}|pwkv}\pwindex{Salten, Felix 6.\,9.\,1869 Budapest – 8.\,10.\,1945 Zürich@\textsc{Salten, Felix} (6.\,9.\,1869 Budapest – 8.\,10.\,1945 Zürich), \emph{Schriftsteller, Journalist, Chefredakteur}|pwkv}\pwindex{Beer-Hofmann, Richard 11.\,7.\,1866 Wien – 26.\,9.\,1945 New York City@\textsc{Beer-Hofmann, Richard} (11.\,7.\,1866 Wien – 26.\,9.\,1945 New York City), \emph{Schriftsteller}|pwkv}\pwindex{Kulka, Julius 25.\,9.\,1865 Lipník nad Bečvou – 22.\,9.\,1893 Wien@\textsc{Kulka, Julius} (25.\,9.\,1865 Lipník nad Bečvou – 22.\,9.\,1893 Wien), \emph{Rechtsanwalt}|pwkv}\pwindex{Schupp, Falk 21.\,9.\,1870 Darmstadt – 6.\,2.\,1922 München@\textsc{Schupp, Falk} (21.\,9.\,1870 Darmstadt – 6.\,2.\,1922 München), \emph{Historiker, Zahnarzt}|pwkv}\pwindex{Joachim, Jaques 24.\,11.\,1866 Wien – 7.\,11.\,1925 ebd.@\textsc{Joachim, Jaques} (24.\,11.\,1866 Wien – 7.\,11.\,1925 ebd.), \emph{Rechtswissenschaftler, Rechtsanwalt, Herausgeber}|pwkv} vorlas.}}}\label{K_L02663-2} hat? 3.) ob Du noch mit \label{K_L02663-3v}\edtext{Jung-Wien\orgindex{Jung Wien@Jung Wien|pw}}{\lemma{\textnormal{\emph{Jung-Wien}}}\Cendnote{\textnormal{Gemeint ist ein loser Verein\orgindex{Jung Wien@Jung Wien|pwkv}, bei dem immer Dienstags neue
                  Texte vorgelesen wurden. Das erste Treffen (mit Beteiligung Goldmanns\pwindex{Goldmann, Paul 31.\,1.\,1865 Breslau – 25.\,9.\,1935 Wien@\textsc{Goldmann, Paul} (31.\,1.\,1865 Breslau – 25.\,9.\,1935 Wien), \emph{Schriftsteller, Journalist}|pwk}) fand am 17. 3. 1891 in der Weinhandlung Wieninger\oindex{Wien@\textbf{Wien}!I., Innere Stadt@\textbf{I., Innere Stadt}!Joseph G. Wieninger, Weinhandlung@\textbf{Joseph G. Wieninger, Weinhandlung}, \emph{Gastgewerbegebäude}|pwk} statt, das letzte, das Schnitzler erwähnte, am 5. 5. 1891, eventuell
                  auch am darauffolgenden Dienstag.}}}\label{K_L02663-3} verkehrſt? 4.) ob Du noch zu \label{K_L02663-4v}\edtext{\textsc{Fanjung\pwindex{Van-Jung, Boris 15.\,10.\,1872 Odessa – 3.\,10.\,1899 Wien@\textsc{Van-Jung, Boris} (15.\,10.\,1872 Odessa – 3.\,10.\,1899 Wien), \emph{Mediziner}|pwv}\pwindex{Van-Jung, Leo 15.\,10.\,1866 Odessa – 2.\,7.\,1939 Riga@\textsc{Van-Jung, Leo} (15.\,10.\,1866 Odessa – 2.\,7.\,1939 Riga), \emph{Gesangspädagoge, Mathematiker}|pwv}}’s}{\lemma{\textnormal{\emph{Fanjung’s}}}\Cendnote{\textnormal{Das Brüderpaar Leo\pwindex{Van-Jung, Leo 15.\,10.\,1866 Odessa – 2.\,7.\,1939 Riga@\textsc{Van-Jung, Leo} (15.\,10.\,1866 Odessa – 2.\,7.\,1939 Riga), \emph{Gesangspädagoge, Mathematiker}|pwk} und Boris
                     Van-Jung\pwindex{Van-Jung, Boris 15.\,10.\,1872 Odessa – 3.\,10.\,1899 Wien@\textsc{Van-Jung, Boris} (15.\,10.\,1872 Odessa – 3.\,10.\,1899 Wien), \emph{Mediziner}|pwk} erwähnte Schnitzler im \emph{Tagebuch}\pwindex{Schnitzler, Arthur 15.\,5.\,1862 Wien – 21.\,10.\,1931 ebd.@\textsc{Schnitzler, Arthur} (15.\,5.\,1862 Wien – 21.\,10.\,1931 ebd.), \emph{Schriftsteller, Mediziner}!Tagebuch@\strich\emph{Tagebuch}|pwk} im Jahr 1891 nur am 5. 2. 1891, in den
                  Folgejahren jedoch öfter.}}}\label{K_L02663-4} kommſt? 5.) wer jetzt Deinen hauptſächlichen
               Verkehr bildet? 6.) was \textsc{Olga\pwindex{Waissnix, Olga 3.\,11.\,1862 Wien – 4.\,11.\,1897 ebd.@\textsc{Waissnix, Olga} (3.\,11.\,1862 Wien – 4.\,11.\,1897 ebd.), \emph{Hotelière}|pwu}} macht? 7.) was Du lieſt? und 8.) was Du zu{ }ſchreiben gedenkſt? –
                  j\textcolor{gray}{a} richtig und 9.) noch was Du für den \label{K_L02663-5v}\edtext{Sommer vorhaſt}{\lemma{\textnormal{\emph{Sommer vorhast}}}\Cendnote{\textnormal{Schnitzler verbrachte den Sommer 1891 unter anderem in Baden\oindex{Baden bei Wien@\textbf{Baden bei Wien}, \emph{Hauptstadt}|pwk}, Ischl\oindex{Bad Ischl@\textbf{Bad Ischl}|pwk} und Halle
                     an der Saale\oindex{Halle (Saale)@\textbf{Halle (Saale)}|pwk}.}}}\label{K_L02663-5}? Du wirſt zwar nach Beantwortung all’ dieſer Fragen{ }ſo erſchöpft von der Anſtrengung{ }ſein, daß Du wirſt eine einwöchentliche
               Kaltwaſſerkur gebrauchen müſſen (\label{K_L02663-6v}\edtext{Briefkaſtenwitz}{\lemma{\textnormal{\emph{Briefkastenwitz}}}\Cendnote{\textnormal{Unklare Anspielung;
                  eventuell bezieht sich der Ausdruck »Briefkasten« auf einen in vielen
                  Zeitschriften enthaltenen Abschnitt, in dem unter diesem Titel 
                  Antworten der Herausgeberinnen und Herausgeber auf Zuschriften des Publikums in
                  knapper, oft auch satirischer Form gegeben wurden.}}}\label{K_L02663-6}) – aber Du thuſt mir’s
               wohl aus alter Freundſchaft.\pend
           
\pstart
           Meinen gegenwärtigen Lebensinhalt wirſt du wohl aus dem, was am Eingang dieses
               Briefes{ }ſteht, zur Genüge erkennen. Brüſſel\oindex{Brüssel@\textbf{Brüssel}, \emph{Hauptstadt}|pw}{ }ſagt
               mir vorläufig gar nichts – es{ }ſei denn, daß es eine unſäglich theure Stadt\oindex{Brüssel@\textbf{Brüssel}, \emph{Hauptstadt}|pwv} iſt und daß ich keine Ahnung habe, wie
               ich hier mit meinem kleinen Gehalt und meinen großen Schulden leben{ }ſoll. Große
               Sorgen machen mir ferner die äußerſt verzwickten politiſchen Verhältniſſe, in die
               mich einzuarbeiten ich Monate Zeit haben müßte, während man {\pb}mein{ }ſofortiges Treten in Action verlangt{ }ſowie
               meine Unkenntniß im Franzöſiſchen. Meine Fähigkeit zu verſtehen iſt gleich Null; und
               wenn es noch vier Grad weniger gibt als Null,{ }ſo bezeichnet dieſes meine Fähigkeit
               mich verſtändlich zu machen. Von{ }ſelbſt wird das nicht kommen; Alle lügen, die{ }ſagen,
               man lerne die Sprache durch einen Aufenthalt im fremden Lande von{ }ſelbſt; und Zeit
               zum Studiren habe ich abſolut nicht. Zwei Eigenthümlichkeiten von Belgien\oindex{Belgien@\textbf{Belgien}|pw}{ }ſind mir beſonders ins Auge gefallen: es iſt ein Land\oindex{Belgien@\textbf{Belgien}|pwv}, in dem es keine
               Zahnſtocher gibt, und in dem man die Thürklinken durch einen Druck von unten nach
               oben öffnet. Außerdem{ }ſind die Kellner hier von einer unerhörten Unhöflichkeit und
               Schlamperei, und ich muß oft an Dich denken, der Du – nachdem Du mit Kellnern keinen
               Spaß verſtehſt – längſt einem dieſer Kerle ein Meſſer in den Leib geſtoßen haben
               würdeſt, hoffentlich gewinnen die Dinge ein freundlicheres Ausſehen für mich. Heut komme ich mir – wie nie vorher – vor wie in der
               Verbannung, und alle meine Wünſche regen{ }ſich, um dieſen Brief zu begleiten in das
                  \label{K_L02663-7v}\edtext{trauliche, von Cigarettendampf
               erfüllte Zimmer}{\lemma{\textnormal{\emph{trauliche, … Zimmer}}}\Cendnote{\textnormal{Hierbei dürfte es sich um
                  eine Beschreibung von Schnitzlers Zimmer
                  handeln.}}}\label{K_L02663-7} mit dem Divan, in deſſen reichen und coquett geordneten Kiſſen es{ }ſich{ }ſo weich ruht und von dem man einen Ausblick hat auf das »\label{K_L02663-8v}\edtext{Pfühl}{\lemma{\textnormal{\emph{Pfühl}}}\Cendnote{\textnormal{österreichisch: Polster}}}\label{K_L02663-8}« im \label{K_L02663-9v}\edtext{Alkoven}{\lemma{\textnormal{\emph{Alkoven}}}\Cendnote{\textnormal{Bettnische}}}\label{K_L02663-9} und die \label{K_L02663-10v}\edtext{Landſchaft mit dem unglaublichen Mond}{\lemma{\textnormal{\emph{Landschaft … Mond}}}\Cendnote{\textnormal{Sofern hier ein Bildobjekt (Gemälde, Stich, …) alludiert wird, so ist nicht klar,
                  welches gemeint ist.}}}\label{K_L02663-10} darüber{\dots} Gott grüße Dich,
               mein lieber kleiner Arthur! Ich umarme Dich in alter Freundſchaft und drücke Dir
               beide Hände dazu.\pend
           \pstart Dein treuer \spacefill\mbox{Paul Goldmann.}\pend{}
\pstart
           \noindent{}Sobald ich eine Adreſſe habe, theile ich{ }ſie Dir mit{\dots}\pend
           
\pstart
           Empfiehl’ mich den Deinen! Die Meinigen haben Dich {\pb}mehreremale grüßen laſſen, aber ich habe immer
                  vergeſſen, Dir’s zu{ }ſchreiben{\dots}{ }\textsc{\begin{otherlanguage}{french}À propos\end{otherlanguage}}: wenn Du Herauskriegen könnteſt, warum mir der Schurke, der \label{K_L02663-11v}\edtext{\textsc{Beer-Hoffmann\pwindex{Beer-Hofmann, Richard 11.\,7.\,1866 Wien – 26.\,9.\,1945 New York City@\textsc{Beer-Hofmann, Richard} (11.\,7.\,1866 Wien – 26.\,9.\,1945 New York City), \emph{Schriftsteller}|pw}}, nicht{ }ſchreibt}{\lemma{\textnormal{\emph{Beer-Hoffmann, nicht schreibt}}}\Cendnote{\textnormal{Der erste
                     überlieferte Brief Goldmanns\pwindex{Goldmann, Paul 31.\,1.\,1865 Breslau – 25.\,9.\,1935 Wien@\textsc{Goldmann, Paul} (31.\,1.\,1865 Breslau – 25.\,9.\,1935 Wien), \emph{Schriftsteller, Journalist}|pwk} an Beer-Hofmann\pwindex{Beer-Hofmann, Richard 11.\,7.\,1866 Wien – 26.\,9.\,1945 New York City@\textsc{Beer-Hofmann, Richard} (11.\,7.\,1866 Wien – 26.\,9.\,1945 New York City), \emph{Schriftsteller}|pwk} ist vom
                        10. 4. 1891, danach folgt eine Lücke bis zum November des Jahres. (\emph{Houghton Library}\orgindex{Houghton Library@Houghton Library|pwk}, MS Ger 183,
                        Box 4.)}}}\label{K_L02663-11} wäre ich Dir{ }ſehr dankbar.\pend
           \selectlanguage{ngerman}\endnumbering\briefempfaengerindex{Schnitzler, Arthur@\textsc{Schnitzler, Arthur}!zzzGoldmann, Paul@\emph{von Paul Goldmann}!1891-05-161@{16. 5. 1891}|)be}\mylabel{L02663h}  \newcommand{\dateiname}{L02663}\newcommand{\titel}{Paul Goldmann an Arthur Schnitzler, 16. 5. 1891}\newcommand{\editorInnen}{Martin Anton Müller und Laura Untner}%% latex-leseansicht-abspann.tex
%% Abspann für die Leseansicht.
%% Der Schalter \ifkorrekturansicht ist bereits durch den Vorspann gesetzt.

%% latex-abspann.tex
%% Gemeinsamer Abspann für Korrekturansicht und Leseansicht.
%% Setzt den Schalter \ifkorrekturansicht voraus (gesetzt in den
%% einbindenden Dateien latex-korrekturansicht-abspann.tex bzw.
%% latex-leseansicht-abspann.tex).
%% ---------------------------------------------------------------

\normalsize

% Das esempio-Environment wird nur in der Leseansicht benötigt
\ifkorrekturansicht\else
\newenvironment{esempio}[3]%
{
    \vspace{1.5ex}
    \rlap{\underline{#1}}
    \par
    \setlength{\parindent}{0cm}
    \nopagebreak
    \leftskip=#2cm
    \rightskip=#3cm
}
{
    \par
}
\fi

\doendnotes{C}
\bigskip
\vfill

\clearpage

\footnotesize

\ifkorrekturansicht
  \lohead{\textsc{register}}
\fi

% theindex-Environment neu definieren ohne reledmac
\makeatletter
\renewenvironment{theindex}{%
  \ifkorrekturansicht
    \section*{\indexname}%
  \else
    \subsubsection*{Index der erwähnten Entitäten}%
  \fi
  \setlength{\parindent}{0pt}%
  \setlength{\parskip}{0pt plus 0.3pt}%
  \let\item\@idxitem
}{%
  \ifkorrekturansicht\clearpage\fi
}
\makeatother

\IfFileExists{\jobname-pw.ind}{\input{\jobname-pw.ind}}{}

% Quellenangabe nur in der Leseansicht
\ifkorrekturansicht\else
% Fallback-Definitionen, falls die .tex-Datei \titel etc. nicht gesetzt hat
\providecommand{\titel}{}
\providecommand{\editorInnen}{}
\providecommand{\dateiname}{\jobname}

\vspace{3cm}

\vfill

\footnotesize
\textsc{Quelle}: \titel. Herausgegeben von {\editorInnen}. In: \emph{Arthur Schnitzler: Briefwechsel mit Autorinnen und Autoren}.
 Digitale Edition, https://schnitzler-briefe.acdh.oeaw.ac.at/{\dateiname}.html (Stand \today)
\fi

\end{document}


