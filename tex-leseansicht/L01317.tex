%% latex-korrekturansicht-vorspann.tex
%% Vorspann für die Korrekturansicht.
%% Lädt die gemeinsame Datei latex-vorspann.tex mit gesetztem Schalter.

\newif\ifkorrekturansicht
\korrekturansichttrue

\input{../tex-inputs/latex-vorspann}


\section[Hugo von Hofmannsthal an Arthur Schnitzler, 28. 8. 1903]{L01317 Hugo von Hofmannsthal an Arthur Schnitzler, 28. 8. 1903}
\nopagebreak\mylabel{L01317v}
\rehead{ }\normalsize\beginnumbering\briefempfaengerindex{Schnitzler, Arthur@\textsc{Schnitzler, Arthur}!zzzHofmannsthal, Hugo von@\emph{von Hugo von Hofmannsthal}!1903-08-281@{28. 8. 1903}|(be}
\toendnotes[C]{\smallbreak\pagebreak[2]}\Standort{CUL, Schnitzler, B 43.}
\physDesc{Bildpostkarte, 299 Zeichen
\newline{}Handschrift: schwarze Tinte, deutsche Kurrent
\newline{}Versand: Stempel: »\nobreak{}\oindex{Weimar@\textbf{Weimar}, \emph{A.ADM3}|pwk}Weimar, \textcolor{gray}{28. 08. 03}, 12–1\nobreak{}«.  
\newline{}Ordnung: 1) mit Bleistift von unbekannter Hand nummeriert: »\strikeout{228}«  2) mit Bleistift von unbekannter Hand nummeriert:
                                    »199«}
\buchAbdrucke{\weitereDrucke{Hugo von Hofmannsthal, Arthur Schnitzler: \emph{Briefwechsel}. Frankfurt am Main: \emph{S. Fischer} 1964, S. 173.} }\toendnotes[C]{\smallbreak}\pstart{}{\pb}\textsc{Herrn D\textsuperscript{r} Arthur Schnitzler}\pend{}\pstart{}\textsc{Wien}\oindex{Wien@\textbf{Wien}, \emph{A.ADM2}|pw}\pend{}\pstart{}\textsc{IX. Franckgasse 1}.\oindex{Frankgasse 1@\textbf{Frankgasse 1}, \emph{Wohngebäude (K.WHS)}|pw}\pend{}{\bigskip}
\pstart
           \noindent{}\centering{}{\pb}\textcolor{gray}{\textbf{Weimar\oindex{Weimar@\textbf{Weimar}, \emph{A.ADM3}|pw}, Gœthe\pwindex{Goethe, Johann Wolfgang von 1749-08-28 – 1832-03-22@\textsc{Goethe, Johann Wolfgang von} (1749-08-28 – 1832-03-22), \emph{Schriftsteller/Schriftstellerin}|pw}’s Gartenhaus\oindex{Gartenhaus [Goethe]@\textbf{Gartenhaus [Goethe]}, \emph{Gebäude (K.GBD)}|pw}.}}\pend
           \vspace{1em}
\pstart
           \raggedleft{}{\pb}28 VIII.\pend
           \vspace{0.5em}
\pstart
           Eine weimar\oindex{Weimar@\textbf{Weimar}, \emph{A.ADM3}|pw}iſche Hofdame\pwindex{?? [Hofdame] 28.8.1903 – 28.8.1903@\textsc{?? [Hofdame]} (28.8.1903 – 28.8.1903)|pwv}: »Ich intereſſiere mich ſehr für
               einen Landsmann von Ihnen, der ſo früh geſtorben iſt und deſſen Werke alle erſt nach
               ſeinem Tod erſchienen ſind: Arthur Schnitzler.« Ich: Ich glaube, Sie irren ſich. Sie:
               O, besti{\geminationm}t nicht. \pend
           \selectlanguage{ngerman}\endnumbering\briefempfaengerindex{Schnitzler, Arthur@\textsc{Schnitzler, Arthur}!zzzHofmannsthal, Hugo von@\emph{von Hugo von Hofmannsthal}!1903-08-281@{28. 8. 1903}|)be}\mylabel{L01317h}  \normalsize

\doendnotes{C}
\bigskip
\vfill

\clearpage

\footnotesize

\lohead{\textsc{register}}

% Definiere theindex-Environment komplett neu ohne reledmac
\makeatletter
\renewenvironment{theindex}{%
  \section*{\indexname}%
  \setlength{\parindent}{0pt}%
  \setlength{\parskip}{0pt plus 0.3pt}%
  \let\item\@idxitem
}{%
  \clearpage
}
\makeatother

\IfFileExists{\jobname-pw.ind}{\input{\jobname-pw.ind}}{}

\end{document}

      