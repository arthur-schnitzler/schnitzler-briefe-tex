%% latex-korrekturansicht-vorspann.tex
%% Vorspann für die Korrekturansicht.
%% Lädt die gemeinsame Datei latex-vorspann.tex mit gesetztem Schalter.

\newif\ifkorrekturansicht
\korrekturansichttrue

\input{../tex-inputs/latex-vorspann}


\section[Georg Brandes an Arthur Schnitzler, 19. 12. 1917]{L02282 Georg Brandes an Arthur Schnitzler, 19. 12. 1917}
\nopagebreak\mylabel{L02282v}
\rehead{ }\normalsize\beginnumbering\briefempfaengerindex{Schnitzler, Arthur@\textsc{Schnitzler, Arthur}!zzzBrandes, Georg@\emph{von Georg Brandes}!1917-12-191@{19. 12. 1917}|(be}
\toendnotes[C]{\smallbreak\pagebreak[2]}\Standort{CUL, Schnitzler, B 17.}
\physDesc{Postkarte, 1305 Zeichen
\newline{}Handschrift: schwarze Tinte, lateinische Kurrent
\newline{}Versand: 1) Stempel: »\nobreak{}\oindex{Kopenhagen@\textbf{Kopenhagen}, \emph{P.PPLC}|pwk}Kjøbenhavn, 20. 12. 17, 5–6F\nobreak{}«.   2) Stempel: »\nobreak{}Zensuriert {[}k.{]} u. k. Zensurstelle Wien\orgindex{K. u. k. Zensurstelle@K. u. k. Zensurstelle|pw}\nobreak{}«. 
\newline{}Ordnung: mit Bleistift von unbekannter Hand nummeriert:
                                    »48« }
\buchAbdrucke{\weitereDrucke{Georg Brandes, Arthur Schnitzler: \emph{Ein Briefwechsel}. Bern: \emph{Francke} 1956, S. 122.} }\toendnotes[C]{\smallbreak}\pstart{}{\pb}Herrn Dr. Arthur
                  Schnitzler\pend{}\pstart{}Sternwartestrasse 71\oindex{Sternwartestrasse 71@\textbf{Sternwartestraße 71}, \emph{Wohngebäude (K.WHS)}|pw}\pend{}\pstart{}Wien \textsubscript{XVIII}\oindex{XVIII., Waehring@\textbf{XVIII., Währing}, \emph{A.ADM3}|pw}\pend{}{\bigskip}\vspace{1em}
\pstart
           \raggedleft{}{\pb}Kopenhagen\oindex{Kopenhagen@\textbf{Kopenhagen}, \emph{P.PPLC}|pw}{ }19 Dec. 17\pend
           \vspace{0.5em}
\pstart
           Verehrter, lieber Freund\hspace*{3.5em}Niemand ist treu und liebenswürdig wie Sie. Obwohl
               ich nie in der Lage bin, Vergelt zu üben, senden Sie mir fortwährend Ihre Erzählungen
               und Schauspiele, die mir so viel Freude bereiten. Nun das letzte Mal \uline{Fink und Fliederbusch}\pwindex{Fink und Fliederbusch. Komoedie in drei Akten@\emph{Fink und Fliederbusch. Komödie in drei Akten}|pw}, ein heiteres Stück in trauriger Zeit, nicht ohne satirischen Stachel, aber
               dennoch human. Ein Franzose\pwindex{Geruzez, Nicolas Eugene 06.01.1799 – 29.05.1865@\textsc{Géruzez, Nicolas Eugène} (06.01.1799 – 29.05.1865), \emph{Kritiker/Kritikerin}|pwv}
               sagte: \label{K_L02282-1v}\edtext{L’âge mûr méprise avec
                  tolérance}{\lemma{\textnormal{\emph{L’âge … tolérance}}}\Cendnote{\textnormal{französisch: Das reife Alter
                  verachtet durch Toleranz}}}\label{K_L02282-1}.\pend
           
\pstart
           Wäre ich so glücklich all das was ich geschrieben habe, seit wir uns sahen, würde es
               eine stattliche Reihe Bücher ausmachen, nicht weniger als 7 schwere Bände. Mein Buch über den Weltkrieg\pwindex{Verdenskrigen [The World at War]@\emph{Verdenskrigen [The World at War]}|pwv}
               erreicht in diesen Tagen hier die vierte Ausgabe, hat in Nordamerika\oindex{Amerika@\textbf{Amerika}, \emph{kein passender Code gefunden}|pw} zwei. Die Bücher über Goethe\pwindex{Goethe, Johann Wolfgang von 1749-08-28 – 1832-03-22@\textsc{Goethe, Johann Wolfgang von} (1749-08-28 – 1832-03-22), \emph{Schriftsteller/Schriftstellerin}|pw}\pwindex{Wolfgang Goethe@\emph{Wolfgang Goethe}|pwv}, über Voltaire\pwindex{Voltaire 21.11.1694 – 30.05.1778@\textsc{Voltaire} (21.11.1694 – 30.05.1778), \emph{Schriftsteller/Schriftstellerin, Philosoph/Philosophin}|pw}\pwindex{Voltaire und sein Jahrhundert@\emph{Voltaire und sein Jahrhundert}|pwv} usw. sind gut gegangen. Ein Buch\pwindex{Udvalgte Stykker@\emph{Udvalgte Stykker}|pwv}, worin ich meine letzten Essays und Reden gesammelt habe, wurde {\pb}in nur 14 Tagen ausverkauft. Seit
                  April bin ich damit beschäftigt eine grosse Maschine\pwindex{Gaius Julius Cæsar@\emph{Gaius Julius Cæsar}|pwv} über meinen vergötterten Cajus Julius Cäsar\pwindex{Caesar, Gaius Iulius 13.7.100? v. Chr. – 15.3.44 v. Chr.@\textsc{Caesar, Gaius Iulius} (13.7.100? v. Chr. – 15.3.44 v. Chr.), \emph{Politiker/Politikerin, Kaiser/Kaiserin, Heerführer/Heerführerin}|pw} zu fabricieren, wird wol
               auch ein paar Bände werden. Der Stoff ist sehr umfangreich, röm\oindex{Rom@\textbf{Rom}, \emph{P.PPLC}|pw}isches Leben von c. 120 bis c. 40,
               aber er fesselt mich sehr. Bin ich doch kein Erfinder, nur ein enthusiastischer
               Forscher. Ich hoffe, dass es Ihnen und den Ihrigen, auch unseren wenigen gemeinsamen
               Freunden wohl geht. Ihr \spacefill\mbox{Georg Brandes}\pend
           \selectlanguage{ngerman}\endnumbering\briefempfaengerindex{Schnitzler, Arthur@\textsc{Schnitzler, Arthur}!zzzBrandes, Georg@\emph{von Georg Brandes}!1917-12-191@{19. 12. 1917}|)be}\mylabel{L02282h}  \normalsize

\doendnotes{C}
\bigskip
\vfill

\clearpage

\footnotesize

\lohead{\textsc{register}}

% Definiere theindex-Environment komplett neu ohne reledmac
\makeatletter
\renewenvironment{theindex}{%
  \section*{\indexname}%
  \setlength{\parindent}{0pt}%
  \setlength{\parskip}{0pt plus 0.3pt}%
  \let\item\@idxitem
}{%
  \clearpage
}
\makeatother

\IfFileExists{\jobname-pw.ind}{\input{\jobname-pw.ind}}{}

\end{document}

      