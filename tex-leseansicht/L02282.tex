\input{../tex-inputs/latex-pdf-vorspann}
\begin{center}
            \textcolor{red}{ENTWURF. ENTZIFFERUNG NOCH NICHT KORREKTURGELESEN}
                      \end{center}
            
               \section[Georg Brandes an Arthur Schnitzler, 19. 12. 1917]{ Georg Brandes an Arthur Schnitzler, 19. 12. 1917}\nopagebreak\mylabel{v}\rehead{ }\begin{ledgroupsized}[t]{13cm}\normalsize\beginnumbering\briefempfaengerindex{Schnitzler, Arthur@\textsc{Schnitzler, Arthur}!zzzBrandes, Georg@\emph{von Georg Brandes}!1917-12-191@{19. 12. 1917}|(be} \toendnotes[C]{\smallbreak\pagebreak[2]} \Standort{CUL, Schnitzler, B 17.}
\physDesc{Postkarte
\newline{}Handschrift: schwarze Tinte, lateinische Kurrent\newline{}Versand: 1) Stempel: »\nobreak{}\oindex{Kopenhagen@\textbf{Kopenhagen}|pwk}Kjøbenhavn, 20. 12. 17, 5–6F\nobreak{}«.  2) Stempel: »\nobreak{}Zensuriert {[}k.{]} u. k. Zensurstelle Wien\orgindex{K. u. k. Zensurstelle@K. u. k. Zensurstelle|pw}\nobreak{}«. \newline{}Ordnung: mit Bleistift von unbekannter Hand nummeriert: »48« }\buchAbdrucke{\weitereDrucke{Georg Brandes, Arthur Schnitzler: \emph{Ein Briefwechsel}. Hg. Kurt Bergel. Bern: \emph{Francke} 1956, S. 122.} }\toendnotes[C]{\smallbreak}\pstart{}{\pb}Herrn Dr. Arthur
                        Schnitzler\pend{}\pstart{}Sternwartestrasse 71\oindex{Sternwartestrasse@\textbf{Sternwartestraße}|pw}\pend{}\pstart{}Wien \textsubscript{XVIII}\oindex{XVIII., Waehring@\textbf{XVIII., Währing}|pw}\pend{}{\bigskip}\pstart
           \raggedleft{}{\pb}Kopenhagen\oindex{Kopenhagen@\textbf{Kopenhagen}|pw}{ }19 Dec. 17\pend
           \pstart
           Verehrter, lieber Freund \hspace*{3.5em}Niemand ist treu und liebenswürdig wie Sie.
                    Obwohl ich nie in der Lage bin, Vergelt zu üben, senden Sie mir fortwährend Ihre
                    Erzählungen und Schauspiele, die mir so viel Freude bereiten. Nun das letzte Mal
                        \uline{Fink und Fliederbusch}\pwindex{Schnitzler, Arthur 15.05.1862 – 21.10.1931@\textsc{Schnitzler, Arthur} (15.05.1862 – 21.10.1931), \emph{Schriftsteller, Mediziner}!Fink und Fliederbusch. Komoedie in drei Akten1917@\strich\emph{Fink und Fliederbusch. Komödie in drei Akten} {[}1917{]}|pw}, ein heiteres Stück in trauriger Zeit, nicht ohne satirischen Stachel,
                    aber dennoch human. Ein Franzose\pwindex{Geruzez, Nicolas Eugene 06.01.1799 – 29.05.1865@\textsc{Géruzez, Nicolas Eugène} (06.01.1799 – 29.05.1865), \emph{Kritiker}|pwv}
              sagte: \label{K_L02282_1v}\edtext{L’âge mûr
               méprise avec tolérance}{\lemma{\textnormal{\emph{L’âge … tolérance}}}\Cendnote{\textnormal{französisch: Das reife
                        Alter verachtet durch Toleranz}}}\label{K_L02282_1h}.\pend
           \pstart
           Wäre ich so glücklich all das was ich geschrieben habe, seit wir uns sahen, würde
                    es eine stattliche Reihe Bücher ausmachen, nicht weniger als 7 schwere Bände.
                    Mein Buch über den Weltkrieg\pwindex{Brandes, Georg 04.02.1842 – 19.02.1927@\textsc{Brandes, Georg} (04.02.1842 – 19.02.1927)!Verdenskrigen [The World at War]1915@\strich\emph{Verdenskrigen [The World at War]} {[}1915{]}|pwv}
                    erreicht in diesen Tagen hier die vierte Ausgabe, hat in Nordamerika\oindex{Amerika@\textbf{Amerika}|pw} zwei. Die Bücher über Goethe\pwindex{Goethe, Johann Wolfgang von 28.08.1749 – 22.03.1832@\textsc{Goethe, Johann Wolfgang von} (28.08.1749 – 22.03.1832), \emph{Schriftsteller}|pw}\pwindex{Brandes, Georg 04.02.1842 – 19.02.1927@\textsc{Brandes, Georg} (04.02.1842 – 19.02.1927)!Wolfgang Goethe1915@\strich\emph{Wolfgang Goethe} {[}1915{]}|pwv}, über Voltaire\pwindex{Voltaire 21.11.1694 – 30.05.1778@\textsc{Voltaire} (21.11.1694 – 30.05.1778), \emph{Schriftsteller, Philosoph}|pw}\pwindex{Brandes, Georg 04.02.1842 – 19.02.1927@\textsc{Brandes, Georg} (04.02.1842 – 19.02.1927)!Voltaire und sein Jahrhundert1916 – 1917@\strich\emph{Voltaire und sein Jahrhundert} {[}1916 – 1917{]}|pwv} usw. sind gut gegangen. Ein Buch\pwindex{Brandes, Georg 04.02.1842 – 19.02.1927@\textsc{Brandes, Georg} (04.02.1842 – 19.02.1927)!Udvalgte Stykker1917@\strich\emph{Udvalgte Stykker} {[}1917{]}|pwv}, worin ich meine letzten Essays und Reden gesammelt habe, wurde
                        {\pb}in nur 14 Tagen
                    ausverkauft. Seit April bin ich damit beschäftigt eine grosse Maschine\pwindex{Brandes, Georg 04.02.1842 – 19.02.1927@\textsc{Brandes, Georg} (04.02.1842 – 19.02.1927)!Gaius Julius Cæsar1918@\strich\emph{Gaius Julius Cæsar} {[}1918{]}|pwv} über meinen
                    vergötterten Cajus Julius Cäsar\pwindex{Caesar, Gaius Iulius 13.7.100? v. Chr. – 15.3.44 v. Chr.@\textsc{Caesar, Gaius Iulius} (13.7.100? v. Chr. – 15.3.44 v. Chr.), \emph{Politiker, Kaiser, Heerführer}|pw} zu
                    fabricieren, wird wol auch ein paar Bände werden. Der Stoff ist sehr
                    umfangreich, röm\oindex{Rom@\textbf{Rom}|pw}isches Leben von
                        c. 120 bis c. 40, aber er fesselt mich sehr. Bin
                    ich doch kein Erfinder, nur ein enthusiastischer Forscher. Ich hoffe, dass es
                    Ihnen und den Ihrigen, auch unseren wenigen gemeinsamen Freunden wohl geht. Ihr
                        \spacefill\mbox{Georg Brandes}\pend
           \endnumbering\briefempfaengerindex{Schnitzler, Arthur@\textsc{Schnitzler, Arthur}!zzzBrandes, Georg@\emph{von Georg Brandes}!1917-12-191@{19. 12. 1917}|)be}\mylabel{h}\end{ledgroupsized}  \newcommand{\dateiname}{L02282}\newcommand{\titel}{Georg Brandes an Arthur Schnitzler, 19. 12. 1917}\newcommand{\editorInnen}{Martin Anton Müller und Gerd-Hermann Susen}\input{../tex-inputs/latex-pdf-abspann}
      