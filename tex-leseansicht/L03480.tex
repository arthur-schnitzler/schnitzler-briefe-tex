%% latex-korrekturansicht-vorspann.tex
%% Vorspann für die Korrekturansicht.
%% Lädt die gemeinsame Datei latex-vorspann.tex mit gesetztem Schalter.

\newif\ifkorrekturansicht
\korrekturansichttrue

\input{../tex-inputs/latex-vorspann}


\section[ Paul Goldmann an Arthur Schnitzler, 20. 11. 1925]{L03480 Paul Goldmann an Arthur Schnitzler, 20. 11. 1925}
\nopagebreak\mylabel{L03480v}
\rehead{ }\normalsize\beginnumbering\briefempfaengerindex{Schnitzler, Arthur@\textsc{Schnitzler, Arthur}!zzzGoldmann, Paul@\emph{von Paul Goldmann}!1925-11-201@{20. 11. 1925}|(be}
\toendnotes[C]{\smallbreak\pagebreak[2]}\Standort{DLA, A:Schnitzler, HS.NZ85.1.3176.}
\physDesc{Brief, 1 Blatt, 2 Seiten, 796 Zeichen
\newline{}Handschrift: lila Tinte, deutsche Kurrent
\newline{}Schnitzler: 1) mit Bleistift Vermerk »Goldm{[}ann{]}\pwindex{Goldmann, Paul 31.01.1865 – 25.09.1935@\textsc{Goldmann, Paul} (31.01.1865 – 25.09.1935), \emph{Schriftsteller/Schriftstellerin, Journalist/Journalistin}|pw}«  2) mit rotem Buntstift zwei Unterstreichungen}\toendnotes[C]{\smallbreak}
\pstart
           {\pb}Berlin\oindex{Berlin@\textbf{Berlin}, \emph{P.PPLC}|pw}, 20. 11. 25.\pend
           
\pstart{}Lieber Freund,\pend\vspace{0.5em}
\pstart
           Mit großer Bewegung leſe ich ſoeben ein \label{K_L03480-1v}\edtext{München\oindex{Muenchen@\textbf{München}, \emph{P.PPLA}|pw}er Telegramm}{\lemma{\textnormal{\emph{Münchener Telegramm}}}\Cendnote{\textnormal{Gemeint ist eine in einer Zeitung abgedruckte Kurzmeldung,
                  die telegrafisch übermittelt wurde.}}}\label{K_L03480-1}, das den \label{K_L03480-2v}\edtext{Tod von \textsc{Marie Glümer\pwindex{Gluemer, Marie 03.07.1867 – 16.11.1925@\textsc{Glümer, Marie} (03.07.1867 – 16.11.1925), \emph{Schauspieler/Schauspielerin}|pw}}}{\lemma{\textnormal{\emph{Tod von Marie Glümer}}}\Cendnote{\textnormal{Marie Glümer\pwindex{Gluemer, Marie 03.07.1867 – 16.11.1925@\textsc{Glümer, Marie} (03.07.1867 – 16.11.1925), \emph{Schauspieler/Schauspielerin}|pwk}, Schnitzlers wichtigste Partnerin in der Zeit der ersten
                  Bekanntschaft mit Goldmann\pwindex{Goldmann, Paul 31.01.1865 – 25.09.1935@\textsc{Goldmann, Paul} (31.01.1865 – 25.09.1935), \emph{Schriftsteller/Schriftstellerin, Journalist/Journalistin}|pwk}, war am 16. 11. 1925 verstorben. Siehe A. S.: \emph{Tagebuch}, 17. 11. 1925.}}}\label{K_L03480-2} meldet.
               Alte Zeiten werden wieder lebendig, Bilder aus ferner Vergangenheit ſteigen auf. Ich
               ſehe das junge Mädchen, das die Verſtorbene\pwindex{Gluemer, Marie 03.07.1867 – 16.11.1925@\textsc{Glümer, Marie} (03.07.1867 – 16.11.1925), \emph{Schauspieler/Schauspielerin}|pwv} einſt war, ſehe Dich, ihren Freund, den jungen Arzt u. Dichter,
               ſehe mich im Beiſammenſein mit euch Beiden\pwindex{Gluemer, Marie 03.07.1867 – 16.11.1925@\textsc{Glümer, Marie} (03.07.1867 – 16.11.1925), \emph{Schauspieler/Schauspielerin}|pwv}. Geſpräche, die ich damals mit Dir geführt, wachen
               wieder auf, – ich erinnere mich an Wien\oindex{Wien@\textbf{Wien}, \emph{A.ADM2}|pw}, an
                  \label{K_L03480-3v}\edtext{Salzburg\oindex{Salzburg@\textbf{Salzburg}, \emph{A.ADM2}|pw}}{\lemma{\textnormal{\emph{Salzburg}}}\Cendnote{\textnormal{Vgl. Paul Goldmann an Arthur Schnitzler, 1. 10. 1890.
               }}}\label{K_L03480-3}.\pend
           
\pstart
           Die Frau\pwindex{Gluemer, Marie 03.07.1867 – 16.11.1925@\textsc{Glümer, Marie} (03.07.1867 – 16.11.1925), \emph{Schauspieler/Schauspielerin}|pwv}, die dahingegangen
               iſt, war längſt aus Deinem Leben ausgeſchieden. Aber ſie hat Dir einſt viel bedeutet.
                  \strikeout{Ich habe an jenem Teile Deines Lebens
                     \textcolor{gray}{te}} Ich habe an alledem {\pb}teilgenommen u. will
               Dir nur ſagen, daß ich deſſen eingedenk bin u. daß mich der Tod Deiner einſtigen Freundin\pwindex{Gluemer, Marie 03.07.1867 – 16.11.1925@\textsc{Glümer, Marie} (03.07.1867 – 16.11.1925), \emph{Schauspieler/Schauspielerin}|pwv} ſehr ergriffen
               hat.\pend
           
\pstart
           Mit herzlichem Gruß {\\[\baselineskip]}Dein {\\[\baselineskip]}\spacefill\mbox{Paul Goldmann.}\pend
           \leftskip=0em{}\selectlanguage{ngerman}\endnumbering\briefempfaengerindex{Schnitzler, Arthur@\textsc{Schnitzler, Arthur}!zzzGoldmann, Paul@\emph{von Paul Goldmann}!1925-11-201@{20. 11. 1925}|)be}\mylabel{L03480h}  \normalsize

\doendnotes{C}
\bigskip
\vfill

\clearpage

\footnotesize

\lohead{\textsc{register}}

% Definiere theindex-Environment komplett neu ohne reledmac
\makeatletter
\renewenvironment{theindex}{%
  \section*{\indexname}%
  \setlength{\parindent}{0pt}%
  \setlength{\parskip}{0pt plus 0.3pt}%
  \let\item\@idxitem
}{%
  \clearpage
}
\makeatother

\IfFileExists{\jobname-pw.ind}{\input{\jobname-pw.ind}}{}

\end{document}

      