%% latex-leseansicht-vorspann.tex
%% Vorspann für die Leseansicht.
%% Lädt die gemeinsame Datei latex-vorspann.tex mit nicht gesetztem Schalter.

\newif\ifkorrekturansicht
\korrekturansichtfalse

\input{../tex-inputs/latex-vorspann}


\section[ Paul Goldmann an Arthur Schnitzler, 20. 11. 1925]{L03480 Paul Goldmann an Arthur Schnitzler,  20. 11. 1925}
\nopagebreak\mylabel{L03480v}
\rehead{ }\normalsize\beginnumbering\briefempfaengerindex{Schnitzler, Arthur@\textsc{Schnitzler, Arthur}!zzzGoldmann, Paul@\emph{von Paul Goldmann}!1925-11-201@{20. 11. 1925}|(be}
\toendnotes[C]{\smallbreak\pagebreak[2]}
\correspDesc{Versand  durch Paul Goldmann am 20. 11. 1925 in Berlin
\newline{}Erhalt  durch Arthur Schnitzler im Zeitraum [21. 11. 1925 – 25. 11. 1925?] in Wien}\toendnotes[C]{\smallbreak}
\Standort{DLA, A:Schnitzler, HS.NZ85.1.3176.}
\physDesc{Brief, 1 Blatt, 2 Seiten, 796 Zeichen
\newline{}Handschrift: lila Tinte, deutsche Kurrent
\newline{}Schnitzler: 1) mit Bleistift Vermerk »Goldm{[}ann{]}\pwindex{Goldmann, Paul 31.\,1.\,1865 Breslau – 25.\,9.\,1935 Wien@\textsc{Goldmann, Paul} (31.\,1.\,1865 Breslau – 25.\,9.\,1935 Wien), \emph{Schriftsteller, Journalist}|pw}«  2) mit rotem Buntstift zwei Unterstreichungen}\toendnotes[C]{\smallbreak}
\pstart
           {\pb}Berlin\oindex{Berlin@\textbf{Berlin}, \emph{Hauptstadt}|pw}, 20. 11. 25.\pend
           
\pstart{}Lieber Freund,\pend\vspace{0.5em}
\pstart
           Mit großer Bewegung leſe ich{ }ſoeben ein \label{K_L03480-1v}\edtext{München\oindex{München@\textbf{München}|pw}er Telegramm}{\lemma{\textnormal{\emph{Münchener Telegramm}}}\Cendnote{\textnormal{Gemeint ist eine in einer Zeitung abgedruckte Kurzmeldung,
                  die telegrafisch übermittelt wurde.}}}\label{K_L03480-1}, das den \label{K_L03480-2v}\edtext{Tod von \textsc{Marie Glümer\pwindex{Glümer, Marie 3.\,7.\,1867 Wien – 16.\,11.\,1925 München@\textsc{Glümer, Marie} (3.\,7.\,1867 Wien – 16.\,11.\,1925 München), \emph{Schauspielerin}|pw}}}{\lemma{\textnormal{\emph{Tod von Marie Glümer}}}\Cendnote{\textnormal{Marie Glümer\pwindex{Glümer, Marie 3.\,7.\,1867 Wien – 16.\,11.\,1925 München@\textsc{Glümer, Marie} (3.\,7.\,1867 Wien – 16.\,11.\,1925 München), \emph{Schauspielerin}|pwk}, Schnitzlers wichtigste Partnerin in der Zeit der ersten
                  Bekanntschaft mit Goldmann\pwindex{Goldmann, Paul 31.\,1.\,1865 Breslau – 25.\,9.\,1935 Wien@\textsc{Goldmann, Paul} (31.\,1.\,1865 Breslau – 25.\,9.\,1935 Wien), \emph{Schriftsteller, Journalist}|pwk}, war am 16. 11. 1925 verstorben. Siehe A. S.: \emph{Tagebuch}, 17. 11. 1925.}}}\label{K_L03480-2} meldet.
               Alte Zeiten werden wieder lebendig, Bilder aus ferner Vergangenheit{ }ſteigen auf. Ich{ }ſehe das junge Mädchen, das die Verſtorbene\pwindex{Glümer, Marie 3.\,7.\,1867 Wien – 16.\,11.\,1925 München@\textsc{Glümer, Marie} (3.\,7.\,1867 Wien – 16.\,11.\,1925 München), \emph{Schauspielerin}|pwv} einſt war,{ }ſehe Dich, ihren Freund, den jungen Arzt u. Dichter,{ }ſehe mich im Beiſammenſein mit euch Beiden\pwindex{Glümer, Marie 3.\,7.\,1867 Wien – 16.\,11.\,1925 München@\textsc{Glümer, Marie} (3.\,7.\,1867 Wien – 16.\,11.\,1925 München), \emph{Schauspielerin}|pwv}. Geſpräche, die ich damals mit Dir geführt, wachen
               wieder auf, – ich erinnere mich an Wien\oindex{Wien@\textbf{Wien}, \emph{Verwaltungsgebiet}|pw}, an
                  \label{K_L03480-3v}\edtext{Salzburg\oindex{Salzburg@\textbf{Salzburg}, \emph{Verwaltungsgebiet}|pw}}{\lemma{\textnormal{\emph{Salzburg}}}\Cendnote{\textnormal{Vgl. XXXX Auszeichnungsfehler: Dokument L02651 nicht gefunden.
               }}}\label{K_L03480-3}.\pend
           
\pstart
           Die Frau\pwindex{Glümer, Marie 3.\,7.\,1867 Wien – 16.\,11.\,1925 München@\textsc{Glümer, Marie} (3.\,7.\,1867 Wien – 16.\,11.\,1925 München), \emph{Schauspielerin}|pwv}, die dahingegangen
               iſt, war längſt aus Deinem Leben ausgeſchieden. Aber{ }ſie hat Dir einſt viel bedeutet.
                  \strikeout{Ich habe an jenem Teile Deines Lebens
                     \textcolor{gray}{te}} Ich habe an alledem {\pb}teilgenommen u. will
               Dir nur{ }ſagen, daß ich deſſen eingedenk bin u. daß mich der Tod Deiner einſtigen Freundin\pwindex{Glümer, Marie 3.\,7.\,1867 Wien – 16.\,11.\,1925 München@\textsc{Glümer, Marie} (3.\,7.\,1867 Wien – 16.\,11.\,1925 München), \emph{Schauspielerin}|pwv}{ }ſehr ergriffen
               hat.\pend
           
\pstart
           Mit herzlichem Gruß {\\[\baselineskip]}Dein {\\[\baselineskip]}\spacefill\mbox{Paul Goldmann.}\pend
           \leftskip=0em{}\selectlanguage{ngerman}\endnumbering\briefempfaengerindex{Schnitzler, Arthur@\textsc{Schnitzler, Arthur}!zzzGoldmann, Paul@\emph{von Paul Goldmann}!1925-11-201@{20. 11. 1925}|)be}\mylabel{L03480h}  \newcommand{\dateiname}{L03480}\newcommand{\titel}{Paul Goldmann an Arthur Schnitzler, 20. 11. 1925}\newcommand{\editorInnen}{Martin Anton Müller und Laura Untner}%% latex-leseansicht-abspann.tex
%% Abspann für die Leseansicht.
%% Der Schalter \ifkorrekturansicht ist bereits durch den Vorspann gesetzt.

%% latex-abspann.tex
%% Gemeinsamer Abspann für Korrekturansicht und Leseansicht.
%% Setzt den Schalter \ifkorrekturansicht voraus (gesetzt in den
%% einbindenden Dateien latex-korrekturansicht-abspann.tex bzw.
%% latex-leseansicht-abspann.tex).
%% ---------------------------------------------------------------

\normalsize

% Das esempio-Environment wird nur in der Leseansicht benötigt
\ifkorrekturansicht\else
\newenvironment{esempio}[3]%
{
    \vspace{1.5ex}
    \rlap{\underline{#1}}
    \par
    \setlength{\parindent}{0cm}
    \nopagebreak
    \leftskip=#2cm
    \rightskip=#3cm
}
{
    \par
}
\fi

\doendnotes{C}
\bigskip
\vfill

\clearpage

\footnotesize

\ifkorrekturansicht
  \lohead{\textsc{register}}
\fi

% theindex-Environment neu definieren ohne reledmac
\makeatletter
\renewenvironment{theindex}{%
  \ifkorrekturansicht
    \section*{\indexname}%
  \else
    \subsubsection*{Index der erwähnten Entitäten}%
  \fi
  \setlength{\parindent}{0pt}%
  \setlength{\parskip}{0pt plus 0.3pt}%
  \let\item\@idxitem
}{%
  \ifkorrekturansicht\clearpage\fi
}
\makeatother

\IfFileExists{\jobname-pw.ind}{\input{\jobname-pw.ind}}{}

% Quellenangabe nur in der Leseansicht
\ifkorrekturansicht\else
% Fallback-Definitionen, falls die .tex-Datei \titel etc. nicht gesetzt hat
\providecommand{\titel}{}
\providecommand{\editorInnen}{}
\providecommand{\dateiname}{\jobname}

\vspace{3cm}

\vfill

\footnotesize
\textsc{Quelle}: \titel. Herausgegeben von {\editorInnen}. In: \emph{Arthur Schnitzler: Briefwechsel mit Autorinnen und Autoren}.
 Digitale Edition, https://schnitzler-briefe.acdh.oeaw.ac.at/{\dateiname}.html (Stand \today)
\fi

\end{document}


