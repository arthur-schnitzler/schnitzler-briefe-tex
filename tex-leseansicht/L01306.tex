%% latex-korrekturansicht-vorspann.tex
%% Vorspann für die Korrekturansicht.
%% Lädt die gemeinsame Datei latex-vorspann.tex mit gesetztem Schalter.

\newif\ifkorrekturansicht
\korrekturansichttrue

\input{../tex-inputs/latex-vorspann}


\section[Arthur Schnitzler an Richard Beer-Hofmann, 6. 8. 1903]{L01306 Arthur Schnitzler an Richard Beer-Hofmann, 6. 8. 1903}
\nopagebreak\mylabel{L01306v}
\rehead{ }\normalsize\beginnumbering\briefempfaengerindex{Beer-Hofmann, Richard@\textsc{Beer-Hofmann, Richard}!zzzSchnitzler, Arthur@\emph{von Arthur Schnitzler}!1903-08-061@{6. 8. 1903}|(be}
\toendnotes[C]{\smallbreak\pagebreak[2]}\Standort{YCGL, MSS 31.}
\physDesc{Brief, 1 Blatt, 3 Seiten, Umschlag, 412 Zeichen
\newline{}Handschrift: Bleistift, deutsche Kurrent
\newline{}Versand: 1) Stempel: »\nobreak{}\oindex{XIII., Hietzing@\textbf{XIII., Hietzing}, \emph{A.ADM3}|pwk}Wien 13/2 89, 1\textcolor{gray}{1}–12V\nobreak{}«.   2) Stempel: »\nobreak{}\oindex{Rodaun@\textbf{Rodaun}, \emph{A.ADM4}|pwk}{\pb}Rodaun, 6. 8. 03, 2{[}–{]}4N\nobreak{}«. 
\newline{}Ordnung: mit Bleistift von unbekannter Hand datiert: »6. 8.« }\toendnotes[C]{\smallbreak}\pstart{}{\pb}Herrn \textsc{Dr Richard
                     Beer-Hofmann}\pend{}\pstart{}\textsc{Rodaun bei Liesing}\oindex{Rodaun@\textbf{Rodaun}, \emph{A.ADM4}|pw}\pend{}\pstart{}\textsc{Liesinger Straße 2}\oindex{Liesingerstrasse@\textbf{Liesingerstraße}, \emph{Straße (K.STR)}|pw}.\pend{}{\bigskip}\vspace{1em}
\pstart
           \raggedleft{}{\pb}\uline{Donnerſtag}. \pend
           
\pstart{}lieber Richard,\pend\vspace{0.5em}
\pstart
           könnte man ſich vielleicht \label{K_L01306-1v}\edtext{Samſtag}{\lemma{\textnormal{\emph{Samſtag}}}\Cendnote{\textnormal{Siehe A. S.: \emph{Tagebuch}, 8. 8. 1903.
               }}}\label{K_L01306-1} zum Nachtmahl in dem Hietzinger \textsc{Restaurant}\oindex{Ottakringer Braeu@\textbf{Ottakringer Bräu}, \emph{Bierhaus (K.BIR)}|pw} treffen? Vorher Schönbrunn\oindex{Schloss Schoenbrunn@\textbf{Schloss Schönbrunn}, \emph{Schloss (K.SLS)}|pw}? \textsc{Hugo\pwindex{Hofmannsthal, Hugo von 1874-02-01 – 1929-07-15@\textsc{Hofmannsthal, Hugo von} (1874-02-01 – 1929-07-15), \emph{Schriftsteller/Schriftstellerin}|pw}} desgleichen\textcolor{gray}{?}\pend
           
\pstart
           Schreiben Sie mir bitte ein Wort {\pb}Stund und Ort zu
               fixiren (Glashaus\oindex{Palmenhaus Schoenbrunn@\textbf{Palmenhaus Schönbrunn}, \emph{Gebäude (K.GBD)}|pw}? Sieben?)
               Schlechtes Wetter ſollte (in Hinſicht aufs \textsc{Rest.}) kein
               Hindernis ſein.\pend
           
\pstart
           Herzlichſt{\\[\baselineskip]}Ihr{\\[\baselineskip]}\spacefill\mbox{Arthur}\pend
           \leftskip=0em{}
\pstart
           \noindent{}{\pb}Wenn Sie telephoniren bitte am beſten 2–3.\pend
           \selectlanguage{ngerman}\endnumbering\briefempfaengerindex{Beer-Hofmann, Richard@\textsc{Beer-Hofmann, Richard}!zzzSchnitzler, Arthur@\emph{von Arthur Schnitzler}!1903-08-061@{6. 8. 1903}|)be}\mylabel{L01306h}  \normalsize

\doendnotes{C}
\bigskip
\vfill

\clearpage

\footnotesize

\lohead{\textsc{register}}

% Definiere theindex-Environment komplett neu ohne reledmac
\makeatletter
\renewenvironment{theindex}{%
  \section*{\indexname}%
  \setlength{\parindent}{0pt}%
  \setlength{\parskip}{0pt plus 0.3pt}%
  \let\item\@idxitem
}{%
  \clearpage
}
\makeatother

\IfFileExists{\jobname-pw.ind}{\input{\jobname-pw.ind}}{}

\end{document}

      