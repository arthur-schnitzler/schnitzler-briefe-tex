\input{../tex-inputs/latex-pdf-vorspann}
\begin{center}
            \textcolor{red}{ENTWURF. ENTZIFFERUNG NOCH NICHT KORREKTURGELESEN}
                      \end{center}
            
               \section[Arthur Schnitzler an Richard Beer-Hofmann, 6. 8. 1903]{ Arthur Schnitzler an Richard Beer-Hofmann, 6. 8. 1903}\nopagebreak\mylabel{v}\rehead{ }\begin{ledgroupsized}[t]{13cm}\normalsize\beginnumbering\briefempfaengerindex{Beer-Hofmann, Richard@\textsc{Beer-Hofmann, Richard}!zzzSchnitzler, Arthur@\emph{von Arthur Schnitzler}!1903-08-061@{6. 8. 1903}|(be} \toendnotes[C]{\smallbreak\pagebreak[2]} \Standort{YCGL, MSS 31.}
\physDesc{Brief, 1 Blatt, 3 Seiten, Umschlag
\newline{}Handschrift: Bleistift, deutsche Kurrent\newline{}Versand: 1) Stempel: »\nobreak{}\oindex{XIII., Hietzing@\textbf{XIII., Hietzing}|pwk}Wien 13/2 89, 1\textcolor{gray}{1}–12V\nobreak{}«.  2) Stempel: »\nobreak{}\oindex{Rodaun@\textbf{Rodaun}|pwk}{\pb}Rodaun, 6. 8. 03, 2{[}–{]}4N\nobreak{}«. \newline{}Ordnung: mit Bleistift von unbekannter Hand datiert: »6. 8.« }\toendnotes[C]{\smallbreak}\pstart{}{\pb}Herrn \textsc{Dr Richard
                            Beer-Hofmann}\pend{}\pstart{}\textsc{Rodaun bei Liesing}\oindex{Rodaun@\textbf{Rodaun}|pw}\pend{}\pstart{}\textsc{Liesinger Straße 2}\oindex{Liesingerstrasse@\textbf{Liesingerstraße}|pw}.\pend{}{\bigskip}\pstart
           \raggedleft{}{\pb}\uline{Donnerſtag}.
                    \pend
           \pstart{}lieber Richard,\pend\pstart
           könnte man ſich vielleicht \label{K_L01306_1v}\edtext{Samſtag}{\lemma{\textnormal{\emph{Samſtag}}}\Cendnote{\textnormal{siehe A. S.: \emph{Tagebuch}, 8. 8. 1903}}}\label{K_L01306_1h} zum Nachtmahl in dem Hietzinger \textsc{Restaurant}\oindex{Ottakringer Braeu@\textbf{Ottakringer Bräu}|pw} treffen? Vorher Schönbrunn\oindex{Schloss Schoenbrunn@\textbf{Schloß Schönbrunn}|pw}? \textsc{Hugo\pwindex{Hofmannsthal, Hugo von 01.02.1874 – 15.07.1929@\textsc{Hofmannsthal, Hugo von} (01.02.1874 – 15.07.1929), \emph{Schriftsteller}|pw}} desgleichen\textcolor{gray}{?}\pend
           \pstart
           Schreiben Sie mir bitte ein Wort {\pb}Stund und Ort zu
                    fixiren (Glashaus\oindex{Palmenhaus Schoenbrunn@\textbf{Palmenhaus Schönbrunn}|pw}? Sieben?)
                    Schlechtes Wetter ſollte (in Hinſicht aufs \textsc{Rest.}) kein
                    Hindernis ſein.\pend
           \pstart
           Herzlichſt{\\[\baselineskip]}Ihr{\\[\baselineskip]}\spacefill\mbox{Arthur}\pend
           \leftskip=0em{}\pstart
           \noindent{}{\pb}Wenn Sie telephoniren bitte am beſten
                        2–3.\pend
           \endnumbering\briefempfaengerindex{Beer-Hofmann, Richard@\textsc{Beer-Hofmann, Richard}!zzzSchnitzler, Arthur@\emph{von Arthur Schnitzler}!1903-08-061@{6. 8. 1903}|)be}\mylabel{h}\end{ledgroupsized}  \newcommand{\dateiname}{L01306}\newcommand{\titel}{Arthur Schnitzler an Richard Beer-Hofmann, 6. 8. 1903}\newcommand{\editorInnen}{Martin Anton Müller und Gerd-Hermann Susen}\input{../tex-inputs/latex-pdf-abspann}
      