%% latex-korrekturansicht-vorspann.tex
%% Vorspann für die Korrekturansicht.
%% Lädt die gemeinsame Datei latex-vorspann.tex mit gesetztem Schalter.

\newif\ifkorrekturansicht
\korrekturansichttrue

\input{../tex-inputs/latex-vorspann}


\section[Arthur Schnitzler an Richard Beer-Hofmann, 16. 11. 1902]{L01248 Arthur Schnitzler an Richard Beer-Hofmann, 16. 11. 1902}
\nopagebreak\mylabel{L01248v}
\rehead{ }\normalsize\beginnumbering\briefempfaengerindex{Beer-Hofmann, Richard@\textsc{Beer-Hofmann, Richard}!zzzSchnitzler, Arthur@\emph{von Arthur Schnitzler}!1902-11-161@{16. 11. 1902}|(be}
\toendnotes[C]{\smallbreak\pagebreak[2]}\Standort{YCGL, MSS 31.}
\physDesc{Brief, 1 Blatt, 1 Seite, Umschlag, 289 Zeichen
\newline{}Handschrift: 1) Bleistift, deutsche Kurrent\hspace{1em}2) Bleistift, lateinische Kurrent (\noindent{}Adresse)\hspace{1em}
\newline{}Versand: 1) Stempel: »\nobreak{}\oindex{IX., Alsergrund@\textbf{IX., Alsergrund}, \emph{A.ADM3}|pwk}Wien 9/1, 17. 11. 02, 11–12V\nobreak{}«.   2) Stempel: »\nobreak{}\oindex{Rodaun@\textbf{Rodaun}, \emph{A.ADM4}|pwk}{\pb}Rodaun, 17. 11. 02, 2–4N\nobreak{}«. 
\newline{}Ordnung: mit Bleistift von unbekannter Hand datiert: »16. 11.« }
\buchAbdrucke{\weitereDrucke{Arthur Schnitzler, Richard Beer-Hofmann: \emph{Briefwechsel 1891–1931}. Wien, Zürich: \emph{Europaverlag} 1992, S. 159.} }\pstart{}{\pb}Herrn Dr Rich Beer-Hofmann\pend{}\pstart{}Rodaun\oindex{Rodaun@\textbf{Rodaun}, \emph{A.ADM4}|pw}\pend{}\pstart{}Liesinger Straße 2\oindex{Liesingerstrasse@\textbf{Liesingerstraße}, \emph{Straße (K.STR)}|pw}\pend{}{\bigskip}\vspace{1em}
\pstart
           \raggedleft{}{\pb}16. 11. 902\pend
           \vspace{0.5em}
\pstart
           lieber Richard, die nächſte Zeit ko{\geminationm}
               ich kaum nach Rodaun\oindex{Rodaun@\textbf{Rodaun}, \emph{A.ADM4}|pw}; die Vormittage ſind zu
               kurz, Nachmittg arbeite ich. Könnte man ſich de{\geminationn} nicht
               in Wien\oindex{Wien@\textbf{Wien}, \emph{A.ADM2}|pw} ſehn? Sie ko{\geminationm}en ja ſo oft herein. Das wär doch fürs erſte viel einfacher. Herzlichſt\pend
           \pstart Ihr \spacefill\mbox{A.}\pend{}\selectlanguage{ngerman}\endnumbering\briefempfaengerindex{Beer-Hofmann, Richard@\textsc{Beer-Hofmann, Richard}!zzzSchnitzler, Arthur@\emph{von Arthur Schnitzler}!1902-11-161@{16. 11. 1902}|)be}\mylabel{L01248h}  \normalsize

\doendnotes{C}
\bigskip
\vfill

\clearpage

\footnotesize

\lohead{\textsc{register}}

% Definiere theindex-Environment komplett neu ohne reledmac
\makeatletter
\renewenvironment{theindex}{%
  \section*{\indexname}%
  \setlength{\parindent}{0pt}%
  \setlength{\parskip}{0pt plus 0.3pt}%
  \let\item\@idxitem
}{%
  \clearpage
}
\makeatother

\IfFileExists{\jobname-pw.ind}{\input{\jobname-pw.ind}}{}

\end{document}

      