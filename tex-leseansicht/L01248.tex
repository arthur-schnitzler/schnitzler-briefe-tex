\input{../tex-inputs/latex-pdf-vorspann}
\begin{center}
            \textcolor{red}{ENTWURF. ENTZIFFERUNG NOCH NICHT KORREKTURGELESEN}
                      \end{center}
            
               \section[Arthur Schnitzler an Richard Beer-Hofmann, 16. 11. 1902]{ Arthur Schnitzler an Richard Beer-Hofmann, 16. 11. 1902}\nopagebreak\mylabel{v}\rehead{ }\begin{ledgroupsized}[t]{13cm}\normalsize\beginnumbering\briefempfaengerindex{Beer-Hofmann, Richard@\textsc{Beer-Hofmann, Richard}!zzzSchnitzler, Arthur@\emph{von Arthur Schnitzler}!1902-11-161@{16. 11. 1902}|(be} \toendnotes[C]{\smallbreak\pagebreak[2]} \Standort{YCGL, MSS 31.}
\physDesc{Brief, 1 Blatt, 1 Seite, Umschlag
\newline{}Handschrift: Bleistift, deutsche Kurrent\newline{}Versand: 1) Stempel: »\nobreak{}\oindex{IX., Alsergrund@\textbf{IX., Alsergrund}|pwk}Wien 9/1, 17. 11. 02, 11–12V\nobreak{}«.  2) Stempel: »\nobreak{}\oindex{Rodaun@\textbf{Rodaun}|pwk}{\pb}Rodaun, 17. 11. 02, 2–4N\nobreak{}«. \newline{}Ordnung: mit Bleistift von unbekannter Hand datiert: »16. 11.« }\buchAbdrucke{\weitereDrucke{Arthur Schnitzler, Richard Beer-Hofmann: \emph{Briefwechsel 1891–1931}. Hg. Konstanze Fliedl. Wien, Zürich: \emph{Europaverlag} 1992, S. 159.} }\pstart{}{\pb}\textsc{Herrn Dr Rich
                     Beer-Hofmann}\pend{}\pstart{}\textsc{Rodaun}\oindex{Rodaun@\textbf{Rodaun}|pw}\pend{}\pstart{}\textsc{Liesinger Straße 2\oindex{Liesingerstrasse@\textbf{Liesingerstraße}|pw}}\pend{}{\bigskip}\pstart
           \raggedleft{}{\pb}16. 11. 902\pend
           \pstart
           lieber Richard, die nächſte Zeit ko{\geminationm}
               ich kaum nach Rodaun\oindex{Rodaun@\textbf{Rodaun}|pw}; die Vormittage ſind zu kurz,
               Nachmittg arbeite ich. Könnte man ſich de{\geminationn} nicht in Wien\oindex{Wien@\textbf{Wien}|pw} ſehn? Sie ko{\geminationm}en ja ſo oft herein. Das wär
               doch fürs erſte viel einfacher. Herzlichſt\pend
           \pstart Ihr \spacefill\mbox{A.}\pend{}\endnumbering\briefempfaengerindex{Beer-Hofmann, Richard@\textsc{Beer-Hofmann, Richard}!zzzSchnitzler, Arthur@\emph{von Arthur Schnitzler}!1902-11-161@{16. 11. 1902}|)be}\mylabel{h}\end{ledgroupsized}  \newcommand{\dateiname}{L01248}\newcommand{\titel}{Arthur Schnitzler an Richard Beer-Hofmann, 16. 11. 1902}\newcommand{\editorInnen}{Martin Anton Müller und Gerd-Hermann Susen}\input{../tex-inputs/latex-pdf-abspann}
      