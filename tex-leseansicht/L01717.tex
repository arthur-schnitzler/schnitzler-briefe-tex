%% latex-leseansicht-vorspann.tex
%% Vorspann für die Leseansicht.
%% Lädt die gemeinsame Datei latex-vorspann.tex mit nicht gesetztem Schalter.

\newif\ifkorrekturansicht
\korrekturansichtfalse

\input{../tex-inputs/latex-vorspann}


               \section[Stefan Großmann an Arthur Schnitzler, 7. 10. 1907]{ Stefan Großmann an Arthur Schnitzler, 7. 10. 1907}\nopagebreak\mylabel{v}\rehead{ }\begin{ledgroupsized}[t]{13cm}\normalsize\beginnumbering\briefempfaengerindex{Schnitzler, Arthur@\textsc{Schnitzler, Arthur}!zzzGrossmann, Stefan@\emph{von Stefan Großmann}!1907-10-073@{7. 10. 1907}|(be} \toendnotes[C]{\smallbreak\pagebreak[2]} \Standort{CUL, Schnitzler, B 34.}
\physDesc{Brief, 1 Blatt (Briefpapier mit Trauerrand), 3 Seiten
\newline{}Handschrift: schwarze Tinte, deutsche Kurrent
\newline{}Schnitzler: 1) mit Bleistift beschriftet: »Großma{\geminationn}« 2) auf der dritten Seite eine Antwortskizze mit Bleistift, die nur
                                 unsicher zu entziffern ist: »\noindent{}{\pb}Unter d Dichg – find
                                       ich nicht\textcolor{gray}{s} heiter –{ / }glaube, daſ\textcolor{gray}{s} 1 Nur \textsc{Excentric} für \strikeout{das}{ }N V\orgindex{Wiener Freie Volksbuehne@Wiener Freie Volksbühne|pw} Publ
                                          paſſ\textcolor{gray}{e} (L Pb
                                          amuſi\textcolor{gray}{erte} ſehr.)–{ / }\uline{Nummer des Hauses}? –{ / }\strikeout{Bin froh}{ }\strikeout{Wo ist genau \textcolor{gray}{×}}.{ / }Bei\textcolor{gray}{de} Titel, d i. nicht ofter{ / }K\textcolor{gray}{ö}nnte: N. L.\textcolor{red}{\textsuperscript{XXXX indx}} – D. l.
                                             \textcolor{gray}{M}.\textcolor{red}{\textsuperscript{XXXX indx}}«\newline{}Ordnung: mit Bleistift von unbekannter Hand nummeriert:
                                 »6« }\toendnotes[C]{\smallbreak}\pstart
           \noindent{}{\pb}\textcolor{gray}{\textbf{Freie Volksbühne\orgindex{Wiener Freie Volksbuehne@Wiener Freie Volksbühne|pw}}}\pend
           \pstart
           \textcolor{gray}{\textbf{Wien\oindex{Wien@\textbf{Wien}|pw} VI/\textsubscript{1}.}}\pend
           \pstart
           \textcolor{gray}{\textbf{Mariahilferſtraße Nr. 89\oindex{Mariahilferstrasse@\textbf{Mariahilferstraße}|pw}.}}\hfill \textcolor{gray}{\textbf{Wien\oindex{Wien@\textbf{Wien}|pw}, am}} 7. Okt. \textcolor{gray}{\textbf{190}}7\pend
           \pstart
           \textcolor{gray}{\textbf{Poſtſparkaſſen-Konto Nr. 87.544.}}\pend
           \pstart\center{}Sehr geehrter Herr.\pend\pstart
           Ich bitte um Entſchuldigung, daſs ich Ihr freundliches Schreiben 2 Tage unerledigt
               ließ.\pend
           \pstart
           Diese 2 Tage wurden jedoch zur Aufnehmung des Vortraglokales benöthigt. Wenn es Ihnen
               alſo recht iſt, findet die Vorleſung\pend
           \pstart
           \centering{}\uline{Mit{[}t{]}woch}, den \uline{16.} Oktober\pend
           \pstart
           \noindent{}\centering{}\uline{acht Uhr abends}\pend
           \pstart
           \noindent{}im Saale des \uline{Verbandsheim\oindex{Verbandsheim@\textbf{Verbandsheim}|pw}}{ }Wien VI\oindex{VI., Mariahilf@\textbf{VI., Mariahilf}|pw}. \uline{Königsegggaſſe\oindex{Koenigseggasse@\textbf{Königseggasse}|pw}} (neben der Gumpendorferſtraße\oindex{Gumpendorfer Strasse@\textbf{Gumpendorfer Straße}|pw}) statt. Der
               Saal faſst 500 Personen.\pend
           \pstart
           Auch ich würde es für ſehr gut halten, wenn außer {\pb}dem »\textsc{\uline{\label{T_L01717_1v}\edtext{Lieutenant}{\lemma{\textnormal{\emph{Lieutenant}}}\Cendnote{\textnormal{Er schreibt:
                              »Leuitenant«.}}}\label{T_L01717_1h}}}\textsc{\uline{{ }Gustl}}\pwindex{Schnitzler, Arthur 15.05.1862 – 21.10.1931@\textsc{Schnitzler, Arthur} (15.05.1862 – 21.10.1931), \emph{Schriftsteller, Mediziner}!Lieutenant Gustl. Novelle25. 12. 1900@\strich\emph{Lieutenant Gustl. Novelle} {[}25. 12. 1900{]}|pw}« eine \uline{dialogiſche} Arbeit vorgeleſen würde, weil
               dies als Contraſt zu jenem großen \strikeout{Monl} Monolog
               belebend wirken würde. Leider kann ich beim beſten Willen die \strikeout{Werk} Titel nicht entziffern, die Sie angeben.\pend
           \pstart
           Es verſteht ſich von ſelbſt, daſs jene Arbeiten die paſſendsten ſind, die mit dem
               Ideenkreis der Zuhörer \introOben{}durch\introOben{} die ſtärkſten \strikeout{Be} Berührungspunkte verbunden ſind.\pend
           \pstart
           Und im Übrigen würde ich den Leuten nach der ſcharfen Eindringlichkeit des »\label{T_L01717_2v}\edtext{Leuitenant}{\lemma{\textnormal{\emph{Leuitenant}}}\Cendnote{\textnormal{Er schreibt: »Leuitenant«.}}}\label{T_L01717_2h} Guſtl\pwindex{Schnitzler, Arthur 15.05.1862 – 21.10.1931@\textsc{Schnitzler, Arthur} (15.05.1862 – 21.10.1931), \emph{Schriftsteller, Mediziner}!Lieutenant Gustl. Novelle25. 12. 1900@\strich\emph{Lieutenant Gustl. Novelle} {[}25. 12. 1900{]}|pw}«
               eine \strikeout{Erl} Weile Lächeln u Lachen gönnen.\pend
           \pstart
           Ihre gütige Entſcheidungen erhoffend{\\[\baselineskip]}ſehr ergeben:\spacefill\mbox{Stefan
                  Großmann}\pend
           \leftskip=0em{}\endnumbering\briefempfaengerindex{Schnitzler, Arthur@\textsc{Schnitzler, Arthur}!zzzGrossmann, Stefan@\emph{von Stefan Großmann}!1907-10-073@{7. 10. 1907}|)be}\mylabel{h}\end{ledgroupsized}  \newcommand{\dateiname}{L01717}\newcommand{\titel}{Stefan Großmann an Arthur Schnitzler, 7. 10. 1907}\newcommand{\editorInnen}{ Martin Anton Müller und Gerd-Hermann Susen}%% latex-leseansicht-abspann.tex
%% Abspann für die Leseansicht.
%% Der Schalter \ifkorrekturansicht ist bereits durch den Vorspann gesetzt.

%% latex-abspann.tex
%% Gemeinsamer Abspann für Korrekturansicht und Leseansicht.
%% Setzt den Schalter \ifkorrekturansicht voraus (gesetzt in den
%% einbindenden Dateien latex-korrekturansicht-abspann.tex bzw.
%% latex-leseansicht-abspann.tex).
%% ---------------------------------------------------------------

\normalsize

% Das esempio-Environment wird nur in der Leseansicht benötigt
\ifkorrekturansicht\else
\newenvironment{esempio}[3]%
{
    \vspace{1.5ex}
    \rlap{\underline{#1}}
    \par
    \setlength{\parindent}{0cm}
    \nopagebreak
    \leftskip=#2cm
    \rightskip=#3cm
}
{
    \par
}
\fi

\doendnotes{C}
\bigskip
\vfill

\clearpage

\footnotesize

\ifkorrekturansicht
  \lohead{\textsc{register}}
\fi

% theindex-Environment neu definieren ohne reledmac
\makeatletter
\renewenvironment{theindex}{%
  \ifkorrekturansicht
    \section*{\indexname}%
  \else
    \subsubsection*{Index der erwähnten Entitäten}%
  \fi
  \setlength{\parindent}{0pt}%
  \setlength{\parskip}{0pt plus 0.3pt}%
  \let\item\@idxitem
}{%
  \ifkorrekturansicht\clearpage\fi
}
\makeatother

\IfFileExists{\jobname-pw.ind}{\input{\jobname-pw.ind}}{}

% Quellenangabe nur in der Leseansicht
\ifkorrekturansicht\else
% Fallback-Definitionen, falls die .tex-Datei \titel etc. nicht gesetzt hat
\providecommand{\titel}{}
\providecommand{\editorInnen}{}
\providecommand{\dateiname}{\jobname}

\vspace{3cm}

\vfill

\footnotesize
\textsc{Quelle}: \titel. Herausgegeben von {\editorInnen}. In: \emph{Arthur Schnitzler: Briefwechsel mit Autorinnen und Autoren}.
 Digitale Edition, https://schnitzler-briefe.acdh.oeaw.ac.at/{\dateiname}.html (Stand \today)
\fi

\end{document}


      