%% latex-korrekturansicht-vorspann.tex
%% Vorspann für die Korrekturansicht.
%% Lädt die gemeinsame Datei latex-vorspann.tex mit gesetztem Schalter.

\newif\ifkorrekturansicht
\korrekturansichttrue

\input{../tex-inputs/latex-vorspann}


\section[Hermann Bahr: Widmungsexemplar Erinnerung an Burckhard für Arthur Schnitzler, 7. 10. 1913]{L02151 Hermann Bahr: Widmungsexemplar Erinnerung an Burckhard für Arthur
               Schnitzler, 7. 10. 1913}
\nopagebreak\mylabel{L02151v}
\rehead{ }\normalsize\beginnumbering\briefempfaengerindex{Schnitzler, Arthur@\textsc{Schnitzler, Arthur}!zzzBahr, Hermann@\emph{von Hermann Bahr}!1913-10-071@{7. 10. 1913}|(be}
\toendnotes[C]{\smallbreak\pagebreak[2]}\Standort{DLA, G:Schnitzler, Arthur (Sammlung Heinrich Schnitzler).}
\physDesc{Widmung am Vorsatzblatt, 51 Zeichen
\newline{}Handschrift: schwarze Tinte, deutsche Kurrent
\newline{}Ordnung: bei der Enteignung des Exemplars 1938 von
                                 unbekannter Hand mit Bleistift ergänzte Informationen: »\noindent{}\strikeout{\textcolor{gray}{×}\-\textcolor{gray}{×}}{ / }\strikeout{Dubl. z.}« }
\buchAbdrucke{\weitereDrucke{Hermann Bahr, Arthur Schnitzler: \emph{Briefwechsel, Aufzeichnungen, Dokumente (1891–1931)}. Göttingen: \emph{Wallstein} 2018, S. 490.} }
\pstart
           \noindent{}{\pb}Seinem lieben Arthur\pend
           
\pstart
           herzlichſt{\\[\baselineskip]}\spacefill\mbox{Hermann}\pend
           \leftskip=0em{}
\pstart
           \noindent{}7. 10. 1913\pend
           \selectlanguage{ngerman}\vspace{1em}{\vspace{1\baselineskip}}
\pstart
           \centering{}{\pb}\textcolor{gray}{\textbf{Hermann Bahr}}\pend
           
\pstart
           \centering{}\textcolor{gray}{\textbf{Erinnerung an Burckhard\pwindex{Erinnerung an Burckhard@\emph{Erinnerung an Burckhard}|pw}}}\pend
           {\vspace{1\baselineskip}}
\pstart
           \centering{}\textcolor{gray}{\textbf{Übrigens, ich sehe nicht ein, warum man sich schämen muß, ein
                  prachtvoller Mensch zu sein.}}\pend
           
\pstart
           \centering{}\textcolor{gray}{\textbf{Dostojewski\pwindex{Dostojevskij, Fjodor Mihajlovic 11.11.1821 – 09.02.1881@\textsc{Dostojevskij, Fjodor Mihajlovič} (11.11.1821 – 09.02.1881), \emph{Schriftsteller/Schriftstellerin}|pw} (Dämonen\pwindex{Daemonen@\emph{Die Dämonen}|pw})}}\pend
           {\vspace{1\baselineskip}}
\pstart
           \centering{}\textcolor{gray}{\textbf{1913}}\pend
           
\pstart
           \centering{}\textcolor{gray}{\textbf{S. FISCHER, VERLAG\orgindex{S. Fischer Verlag@S. Fischer Verlag|pw}, BERLIN\oindex{Berlin@\textbf{Berlin}, \emph{P.PPLC}|pw}}}\pend
           \selectlanguage{ngerman}\endnumbering\briefempfaengerindex{Schnitzler, Arthur@\textsc{Schnitzler, Arthur}!zzzBahr, Hermann@\emph{von Hermann Bahr}!1913-10-071@{7. 10. 1913}|)be}\mylabel{L02151h}  \normalsize

\doendnotes{C}
\bigskip
\vfill

\clearpage

\footnotesize

\lohead{\textsc{register}}

% Definiere theindex-Environment komplett neu ohne reledmac
\makeatletter
\renewenvironment{theindex}{%
  \section*{\indexname}%
  \setlength{\parindent}{0pt}%
  \setlength{\parskip}{0pt plus 0.3pt}%
  \let\item\@idxitem
}{%
  \clearpage
}
\makeatother

\IfFileExists{\jobname-pw.ind}{\input{\jobname-pw.ind}}{}

\end{document}

      