%% latex-korrekturansicht-vorspann.tex
%% Vorspann für die Korrekturansicht.
%% Lädt die gemeinsame Datei latex-vorspann.tex mit gesetztem Schalter.

\newif\ifkorrekturansicht
\korrekturansichttrue

\input{../tex-inputs/latex-vorspann}


\section[Theodor Herzl an Arthur Schnitzler, 6. 8. 1893]{L03832 Theodor Herzl an Arthur Schnitzler, 6. 8. 1893}
\nopagebreak\mylabel{L03832v}
\rehead{ }\normalsize\beginnumbering\briefempfaengerindex{Schnitzler, Arthur@\textsc{Schnitzler, Arthur}!zzzHerzl, Theodor@\emph{von Theodor Herzl}!1893-08-061@{6. 8. 1893}|(be}
\toendnotes[C]{\smallbreak\pagebreak[2]}\Standort{CUL, Schnitzler, B 39.}
\physDesc{Brief, 1 Blatt, 1 Seite, 316 Zeichen
\newline{}Handschrift: schwarze Tinte, lateinische Kurrent}\toendnotes[C]{\smallbreak}
\pstart
           {\pb}\textcolor{gray}{\textbf{NOUVELLE PRESSE LIBRE }}\orgindex{Neue Freie Presse@Neue Freie Presse|pw}\hfill \textcolor{gray}{\textbf{8, Rue de Monceau }}\oindex{8, Rue de Monceau@\textbf{8, Rue de Monceau}, \emph{Wohngebäude (K.WHS)}|pw}\pend
           
\pstart
           \textcolor{gray}{\textbf{D\textsuperscript{R} TH. HERZL}}\pend
           
\pstart{}Lieber Freund!\pend\vspace{0.5em}
\pstart
           Die Wahrheit ist, dass ich unterwegs nach Oestreich\oindex{Oesterreich-Ungarn@\textbf{Österreich-Ungarn}, \emph{Land (A.LND)}|pw} war und \label{K_L03832-1v}\edtext{von den
                  Ereignissen}{\lemma{\textnormal{\emph{von den
                  Ereignissen}}}\Cendnote{\textnormal{Am
                     2. 7. 1893 waren in Paris\oindex{Paris@\textbf{Paris}, \emph{P.PPLC}|pwk} Unruhen ausgebrochen, sodass Herzl\pwindex{Herzl, Theodor 1860-05-02 – 1904-07-03@\textsc{Herzl, Theodor} (1860-05-02 – 1904-07-03), \emph{Schriftsteller/Schriftstellerin, Journalist/Journalistin}|pwk} den Urlaub abbrechen und seinen Korrespondentenposten wieder
                  einnehmen musste.}}}\label{K_L03832-1} inmitten, in der schönen Mitten meines Urlaubs beim
               Kragen genommen u. nach Paris\oindex{Paris@\textbf{Paris}, \emph{P.PPLC}|pw} zurückgeschleppt
               wurde. \pend
           
\pstart
           Wenn möglich komme ich im September nach Wien\oindex{Wien@\textbf{Wien}, \emph{A.ADM2}|pw}. Dann sehen wir uns. \pend
           
\pstart
           In Eile aber immer herzlich ergeben{\\[\baselineskip]} Ihr{\\[\baselineskip]}\spacefill\mbox{Th Herzl}\pend
           \leftskip=0em{}
\pstart
           \noindent{}Paris \oindex{Paris@\textbf{Paris}, \emph{P.PPLC}|pw}\pend
           
\pstart
           6 Aug. 93\pend
           \selectlanguage{ngerman}\endnumbering\briefempfaengerindex{Schnitzler, Arthur@\textsc{Schnitzler, Arthur}!zzzHerzl, Theodor@\emph{von Theodor Herzl}!1893-08-061@{6. 8. 1893}|)be}\mylabel{L03832h}
\begin{anhang}
\end{anhang}\normalsize

\doendnotes{C}
\bigskip
\vfill

\clearpage

\footnotesize

\lohead{\textsc{register}}

% Definiere theindex-Environment komplett neu ohne reledmac
\makeatletter
\renewenvironment{theindex}{%
  \section*{\indexname}%
  \setlength{\parindent}{0pt}%
  \setlength{\parskip}{0pt plus 0.3pt}%
  \let\item\@idxitem
}{%
  \clearpage
}
\makeatother

\IfFileExists{\jobname-pw.ind}{\input{\jobname-pw.ind}}{}

\end{document}

      