%% latex-leseansicht-vorspann.tex
%% Vorspann für die Leseansicht.
%% Lädt die gemeinsame Datei latex-vorspann.tex mit nicht gesetztem Schalter.

\newif\ifkorrekturansicht
\korrekturansichtfalse

\input{../tex-inputs/latex-vorspann}


\section[Arthur Schnitzler an Richard Beer-Hofmann, 2. 3. 1900]{L01017 Arthur Schnitzler an Richard Beer-Hofmann, 2. 3. 1900}
\nopagebreak\mylabel{L01017v}
\rehead{ }\normalsize\beginnumbering\briefempfaengerindex{Beer-Hofmann, Richard@\textsc{Beer-Hofmann, Richard}!zzzSchnitzler, Arthur@\emph{von Arthur Schnitzler}!1900-03-021@{2. 3. 1900}|(be}
\toendnotes[C]{\smallbreak\pagebreak[2]}
\correspDesc{Versand  durch Arthur Schnitzler am 2. 3. 1900 in Reichenau an der Rax
\newline{}Erhalt  durch Richard Beer-Hofmann im Zeitraum [3. 3. 1900
                  – 7. 3. 1900?] in Florenz}\toendnotes[C]{\smallbreak}
\Standort{YCGL, MSS 31.}
\physDesc{Brief, 3 Blätter, 10 Seiten, 2970 Zeichen
\newline{}Handschrift: Bleistift, deutsche Kurrent}
\buchAbdrucke{\weitereDrucke{1) Arthur Schnitzler: \emph{Briefe 1875–1912}. Herausgegeben von Therese Nickl und Heinrich Schnitzler. Frankfurt am Main: \emph{S. Fischer} 1981, S. 380–382.} \weitereDrucke{2) Arthur Schnitzler, Richard Beer-Hofmann: \emph{Briefwechsel 1891–1931}. Herausgegeben von Konstanze Fliedl. Wien, Zürich: \emph{Europaverlag} 1992, S. 144–145.} }\toendnotes[C]{\smallbreak}
\pstart
           \raggedleft{}{\pb}\textsc{Edlacher Hof}\oindex{Hotel Edlacherhof@\textbf{Hotel Edlacherhof}, \emph{Hotel}|pw},\pend
           
\pstart
           \raggedleft{}2. März 1900.\pend
           \vspace{0.5em}
\pstart
           mein lieber Richard, vorgeſtern Abend bin ich hier angeko{\geminationm}en, ich wollte dem Frühling entgegenfahren – und{ }ſeit
               geſtern{ }ſchneit und friert es. I{\geminationm}erhin iſt es in den
               Mittagſtunden{ }ſchön. Heut{ }ſowohl als geſtern bin ich nahezu 6 Stunden{ }ſpazieren
               gegangen. Weniger lang {\pb}ſchrieb ich an der Novelle\pwindex{Schnitzler, Arthur 15.\,5.\,1862 Wien – 21.\,10.\,1931 ebd.@\textsc{Schnitzler, Arthur} (15.\,5.\,1862 Wien – 21.\,10.\,1931 ebd.), \emph{Schriftsteller, Mediziner}!Frau Bertha Garlan. Roman@\strich\emph{Frau Bertha Garlan. Roman}|pwv}, für die ich keinen
               Namen habe.\pend
           
\pstart
           Ihre\pwindex{Beer-Hofmann, Richard 11.\,7.\,1866 Wien – 26.\,9.\,1945 New York City@\textsc{Beer-Hofmann, Richard} (11.\,7.\,1866 Wien – 26.\,9.\,1945 New York City), \emph{Schriftsteller}!Tod Georgs@\strich\emph{Der Tod Georgs}|pwv} hab’ ich in 2 Etappen
               geleſen, die erſten 2 Capitel in der Eiſenbahn, die letzten 2 geſtern Abend auf
               meinem Zimmer (3. außer 4. im Bett) Also glauben Sie mir: es iſt ein wundervolles Buch\pwindex{Beer-Hofmann, Richard 11.\,7.\,1866 Wien – 26.\,9.\,1945 New York City@\textsc{Beer-Hofmann, Richard} (11.\,7.\,1866 Wien – 26.\,9.\,1945 New York City), \emph{Schriftsteller}!Tod Georgs@\strich\emph{Der Tod Georgs}|pwv}. Man hat allerdings das
               Gefühl, als wenn die aneinandergereihten Edelſteine nicht auf einer Schnur, {\pb}ſondern auf einem Zwirnsfaden – oder gar nur in der
               Luft aneinandergereiht wären – aber man muſs nicht alles als Kette um den Hals tragen
               können. Im vierten Kapitel\pwindex{Beer-Hofmann, Richard 11.\,7.\,1866 Wien – 26.\,9.\,1945 New York City@\textsc{Beer-Hofmann, Richard} (11.\,7.\,1866 Wien – 26.\,9.\,1945 New York City), \emph{Schriftsteller}!Tod Georgs@\strich\emph{Der Tod Georgs}|pwv}{ }ſteckt übrigens irgend wo ein frecher Schwindel –
               das dürfte Ihnen nicht unbekannt{ }ſein. Sie{ }ſetzen{ }ſich{ }ſozuſagen plötzlich an eine
               andre Orgel, die auch herrlich klingt – {\pb}aber das
               beweiſt nichts. – Nicht überall{ }ſcheint es mir geglückt, daſs gegenwärtiges und
               erinnertes{ }ſich gegeneinander abhebt, wie es{ }ſoll; daſs man das Bedürfnis hat, das
                  Buch\pwindex{Beer-Hofmann, Richard 11.\,7.\,1866 Wien – 26.\,9.\,1945 New York City@\textsc{Beer-Hofmann, Richard} (11.\,7.\,1866 Wien – 26.\,9.\,1945 New York City), \emph{Schriftsteller}!Tod Georgs@\strich\emph{Der Tod Georgs}|pwv} wieder zu leſen \strikeout{dagegen} iſt ja{ }ſehr{ }ſchön; aber dſs man es entschieden
               2–3 Mal lesen \uline{muſs}, iſt vielleicht ein Fehler. Ihre
               Bilderpracht{ }ſchreit nach Jamben {\pb}und nach Drama. Ja
               es verlangt mich geradezu, einige von Ihren Vergleichen in Ihren Stücken
               wiederzufinden und{ }ſie auf der Bühne{ }ſprechen zu hören. – Wunderbar iſt, wie{ }ſcheinbar belangloſe Details zu ihrer Zeit ausgenützt und nachträglich voll Belang
               erſcheinen. Das gibt den gewiſſen Schauer. Überhaupt: meiner {\pb}Empfindg nach{ }ſteckt viel mehr Dichteriſches in dem
                  Buch\pwindex{Beer-Hofmann, Richard 11.\,7.\,1866 Wien – 26.\,9.\,1945 New York City@\textsc{Beer-Hofmann, Richard} (11.\,7.\,1866 Wien – 26.\,9.\,1945 New York City), \emph{Schriftsteller}!Tod Georgs@\strich\emph{Der Tod Georgs}|pwv} als, wie gewiſs
               vielfach behauptet werden wird, Verſtand. Sie wiſſen wie ich das meine. So geſcheidt
               iſt bald einer – aber die Dinge \uline{ſo} sagen – ! Um Goethe\pwindex{Goethe, Johann Wolfgang von 28.\,8.\,1749 Frankfurt am Main – 22.\,3.\,1832 Weimar@\textsc{Goethe, Johann Wolfgang von} (28.\,8.\,1749 Frankfurt am Main – 22.\,3.\,1832 Weimar), \emph{Schriftsteller}|pw} zu variiren: Alles gescheidte iſt{ }ſchon einmal geſagt worden:
                  man muſs nur verſuchen, es \label{K_L01017-1v}\edtext{– ganz
                  anders zu sagen.}{\lemma{\textnormal{\emph{– ganz
                  anders zu sagen.}}}\Cendnote{\textnormal{Bei Goethe\pwindex{Goethe, Johann Wolfgang von 28.\,8.\,1749 Frankfurt am Main – 22.\,3.\,1832 Weimar@\textsc{Goethe, Johann Wolfgang von} (28.\,8.\,1749 Frankfurt am Main – 22.\,3.\,1832 Weimar), \emph{Schriftsteller}|pwk} endet es: »es noch einmal
                        zu denken.«}}}\label{K_L01017-1}\pwindex{Goethe, Johann Wolfgang von 28.\,8.\,1749 Frankfurt am Main – 22.\,3.\,1832 Weimar@\textsc{Goethe, Johann Wolfgang von} (28.\,8.\,1749 Frankfurt am Main – 22.\,3.\,1832 Weimar), \emph{Schriftsteller}!Wilhelm Meisters Wanderjahre@\strich\emph{Wilhelm Meisters Wanderjahre}|pwv}{ }{\pb}Und »\label{K_L01017-2v}\edtext{\textsc{ma foi}}{\lemma{\textnormal{\emph{ma foi}}}\Cendnote{\textnormal{französisch: meiner Treu}}}\label{K_L01017-2}« das
               haben Sie gethan. –\pend
           
\pstart
           Während ich dieſes{ }ſchreibe{ }ſitze ich allein im Speiſeſaal, abends
               9 Uhr. Außer mir lebt hier nemlich nur ein (noch) älterer Herr. Montag fahr
               ich wohl wieder nach Wien\oindex{Wien@\textbf{Wien}, \emph{Verwaltungsgebiet}|pw}. Ich{ }ſehn mich nach
               niemandem – niemand{ }ſehnt{ }ſich nach mir. Das iſt nicht{ }ſenti{\pb}mental –{ }ſondern das iſt eben{ }ſo. Heut vor einem Jahr
               war alles noch{ }ſo anders – und doch \label{K_L01017-3v}\edtext{ſchwebte es{ }ſchon über uns}{\lemma{\textnormal{\emph{schwebte … uns}}}\Cendnote{\textnormal{der Tod Marie Reinhards\pwindex{Reinhard, Marie 13.\,3.\,1871 Wien – 18.\,3.\,1899 ebd.@\textsc{Reinhard, Marie} (13.\,3.\,1871 Wien – 18.\,3.\,1899 ebd.), \emph{Gesangspädagogin}|pwk}}}}\label{K_L01017-3}{\dotstwo} Ja ja, es{ }ſchwebt immer{\dots} »Zeit iſt nur ein Wort –\pwindex{Schnitzler, Arthur 15.\,5.\,1862 Wien – 21.\,10.\,1931 ebd.@\textsc{Schnitzler, Arthur} (15.\,5.\,1862 Wien – 21.\,10.\,1931 ebd.), \emph{Schriftsteller, Mediziner}!Schleier der Beatrice. Schauspiel in fünf Akten@\strich\emph{Der Schleier der Beatrice. Schauspiel in fünf Akten}|pwv}« Könnte
               von Ihnen, von Hugo\pwindex{Hofmannsthal, Hugo von 1.\,2.\,1874 Wien – 15.\,7.\,1929 Rodaun@\textsc{Hofmannsthal, Hugo von} (1.\,2.\,1874 Wien – 15.\,7.\,1929 Rodaun), \emph{Schriftsteller}|pw} und von mir \introOben{}(und etlichen andern)\introOben{}{ }ſein. Zufällig{ }ſagt es \textsc{Beatrice}\pwindex{Schnitzler, Arthur 15.\,5.\,1862 Wien – 21.\,10.\,1931 ebd.@\textsc{Schnitzler, Arthur} (15.\,5.\,1862 Wien – 21.\,10.\,1931 ebd.), \emph{Schriftsteller, Mediziner}!Schleier der Beatrice. Schauspiel in fünf Akten@\strich\emph{Der Schleier der Beatrice. Schauspiel in fünf Akten}|pwv}. –\pend
           
\pstart
           Wie lang denken Sie noch auf Reiſen zu{ }ſein? Ich{ }ſchicke {\pb}dieſen Brief nach Florenz\oindex{Florenz@\textbf{Florenz}|pw}, wo ich Sie, glücklicher und wenn Sie wünſchen weniger witzig als in
                  \textsc{Sanremo}\oindex{Sanremo@\textbf{Sanremo}, \emph{Hauptstadt}|pw} vermuthe. – Mirjam\pwindex{Beer-Hofmann, Mirjam 4.\,9.\,1897 Wien – 24.\,12.\,1984 New York City@\textsc{Beer-Hofmann, Mirjam} (4.\,9.\,1897 Wien – 24.\,12.\,1984 New York City)|pw} hoff ich{ }ſo luſtig als{ }ſie war und Ihre Frau\pwindex{Beer-Hofmann, Paula 25.\,2.\,1879 Wien – 30.\,10.\,1939 Zürich@\textsc{Beer-Hofmann, Paula} (25.\,2.\,1879 Wien – 30.\,10.\,1939 Zürich)|pwv}{ }ſo
               erholt, als man es von italieniſcher\oindex{Italien@\textbf{Italien}|pw} Luft
               erwarten sollte. –\pend
           
\pstart
           Von Hugo\pwindex{Hofmannsthal, Hugo von 1.\,2.\,1874 Wien – 15.\,7.\,1929 Rodaun@\textsc{Hofmannsthal, Hugo von} (1.\,2.\,1874 Wien – 15.\,7.\,1929 Rodaun), \emph{Schriftsteller}|pw} weiſs ich noch immer nichts, und Gustav\pwindex{Schwarzkopf, Gustav 7.\,11.\,1853 Wien – 13.\,11.\,1939 ebd.@\textsc{Schwarzkopf, Gustav} (7.\,11.\,1853 Wien – 13.\,11.\,1939 ebd.), \emph{Schriftsteller}|pw}{ }{\pb}hab ich von Ihnen gegrüßt. Thun Sie das gleiche von
               mir an \textsc{Mayer}\pwindex{Mayer, Oskar 1876 – 15.\,5.\,1915 München@\textsc{Mayer, Oskar} (1876 – 15.\,5.\,1915 München), \emph{Schriftsteller, Beamter}|pw}, we{\geminationn} er{ }ſchon mit Ihnen zuſa{\geminationm}engeſtoßen iſt (– was hoffentlich nicht weh gethan
               hat.)\pend
           
\pstart
           Leben Sie wohl!\pend
           \pstart Ihr \spacefill\mbox{Arthur}\pend{}\selectlanguage{ngerman}\endnumbering\briefempfaengerindex{Beer-Hofmann, Richard@\textsc{Beer-Hofmann, Richard}!zzzSchnitzler, Arthur@\emph{von Arthur Schnitzler}!1900-03-021@{2. 3. 1900}|)be}\mylabel{L01017h}  \newcommand{\dateiname}{L01017}\newcommand{\titel}{Arthur Schnitzler an Richard Beer-Hofmann, 2. 3. 1900}\newcommand{\editorInnen}{Martin Anton Müller und Gerd-Hermann Susen}%% latex-leseansicht-abspann.tex
%% Abspann für die Leseansicht.
%% Der Schalter \ifkorrekturansicht ist bereits durch den Vorspann gesetzt.

%% latex-abspann.tex
%% Gemeinsamer Abspann für Korrekturansicht und Leseansicht.
%% Setzt den Schalter \ifkorrekturansicht voraus (gesetzt in den
%% einbindenden Dateien latex-korrekturansicht-abspann.tex bzw.
%% latex-leseansicht-abspann.tex).
%% ---------------------------------------------------------------

\normalsize

% Das esempio-Environment wird nur in der Leseansicht benötigt
\ifkorrekturansicht\else
\newenvironment{esempio}[3]%
{
    \vspace{1.5ex}
    \rlap{\underline{#1}}
    \par
    \setlength{\parindent}{0cm}
    \nopagebreak
    \leftskip=#2cm
    \rightskip=#3cm
}
{
    \par
}
\fi

\doendnotes{C}
\bigskip
\vfill

\clearpage

\footnotesize

\ifkorrekturansicht
  \lohead{\textsc{register}}
\fi

% theindex-Environment neu definieren ohne reledmac
\makeatletter
\renewenvironment{theindex}{%
  \ifkorrekturansicht
    \section*{\indexname}%
  \else
    \subsubsection*{Index der erwähnten Entitäten}%
  \fi
  \setlength{\parindent}{0pt}%
  \setlength{\parskip}{0pt plus 0.3pt}%
  \let\item\@idxitem
}{%
  \ifkorrekturansicht\clearpage\fi
}
\makeatother

\IfFileExists{\jobname-pw.ind}{\input{\jobname-pw.ind}}{}

% Quellenangabe nur in der Leseansicht
\ifkorrekturansicht\else
% Fallback-Definitionen, falls die .tex-Datei \titel etc. nicht gesetzt hat
\providecommand{\titel}{}
\providecommand{\editorInnen}{}
\providecommand{\dateiname}{\jobname}

\vspace{3cm}

\vfill

\footnotesize
\textsc{Quelle}: \titel. Herausgegeben von {\editorInnen}. In: \emph{Arthur Schnitzler: Briefwechsel mit Autorinnen und Autoren}.
 Digitale Edition, https://schnitzler-briefe.acdh.oeaw.ac.at/{\dateiname}.html (Stand \today)
\fi

\end{document}


