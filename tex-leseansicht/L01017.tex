%% latex-leseansicht-vorspann.tex
%% Vorspann für die Leseansicht.
%% Lädt die gemeinsame Datei latex-vorspann.tex mit nicht gesetztem Schalter.

\newif\ifkorrekturansicht
\korrekturansichtfalse

\input{../tex-inputs/latex-vorspann}


               \section[Arthur Schnitzler an Richard Beer-Hofmann, 2. 3. 1900]{ Arthur Schnitzler an Richard Beer-Hofmann, 2. 3. 1900}\nopagebreak\mylabel{v}\rehead{ }\begin{ledgroupsized}[t]{13cm}\normalsize\beginnumbering\briefempfaengerindex{Beer-Hofmann, Richard@\textsc{Beer-Hofmann, Richard}!zzzSchnitzler, Arthur@\emph{von Arthur Schnitzler}!1900-03-021@{2. 3. 1900}|(be} \toendnotes[C]{\smallbreak\pagebreak[2]} \Standort{YCGL, MSS 31.}
\physDesc{Brief, 3 Blätter, 10 Seiten
\newline{}Handschrift: Bleistift, deutsche Kurrent}\buchAbdrucke{\weitereDrucke{1) Arthur Schnitzler: \emph{Briefe 1875–1912}. Hg. Therese Nickl und Heinrich Schnitzler. Frankfurt am Main: \emph{S. Fischer} 1981, S. 380–382.} \weitereDrucke{2) Arthur Schnitzler, Richard Beer-Hofmann: \emph{Briefwechsel 1891–1931}. Hg. Konstanze Fliedl. Wien, Zürich: \emph{Europaverlag} 1992, S. 144–145.} }\toendnotes[C]{\smallbreak}\pstart
           \raggedleft{}{\pb}\textsc{Edlacher Hof}\oindex{Hotel Edlacherhof@\textbf{Hotel Edlacherhof}|pw},\pend
           \pstart
           \raggedleft{}2. März 1900.\pend
           \pstart
           mein lieber Richard, vorgeſtern Abend bin ich hier angeko{\geminationm}en, ich wollte dem Frühling entgegenfahren – und ſeit
               geſtern ſchneit und friert es. I{\geminationm}erhin iſt es in den
               Mittagſtunden ſchön. Heut ſowohl als geſtern bin ich nahezu 6 Stunden ſpazieren
               gegangen. Weniger lang {\pb}ſchrieb ich an der Novelle\pwindex{Schnitzler, Arthur 15.05.1862 – 21.10.1931@\textsc{Schnitzler, Arthur} (15.05.1862 – 21.10.1931), \emph{Schriftsteller, Mediziner}!Frau Bertha Garlan. Roman15.1.1901 – 15.3.1901@\strich\emph{Frau Bertha Garlan. Roman} {[}15.1.1901 – 15.3.1901{]}|pwv}, für die ich keinen Namen
               habe.\pend
           \pstart
           Ihre\pwindex{Beer-Hofmann, Richard 11.07.1866 – 26.09.1945@\textsc{Beer-Hofmann, Richard} (11.07.1866 – 26.09.1945), \emph{Schriftsteller}!Tod Georgs1900@\strich\emph{Der Tod Georgs} {[}1900{]}|pwv} hab’ ich in 2 Etappen
               geleſen, die erſten 2 Capitel in der Eiſenbahn, die letzten 2 geſtern Abend auf
               meinem Zimmer (3. außer 4. im Bett) Also glauben Sie mir: es iſt ein
               wundervolles Buch\pwindex{Beer-Hofmann, Richard 11.07.1866 – 26.09.1945@\textsc{Beer-Hofmann, Richard} (11.07.1866 – 26.09.1945), \emph{Schriftsteller}!Tod Georgs1900@\strich\emph{Der Tod Georgs} {[}1900{]}|pwv}. Man hat
               allerdings das Gefühl, als wenn die aneinandergereihten Edelſteine nicht auf einer
               Schnur, {\pb}ſondern auf einem Zwirnsfaden – oder gar nur
               in der Luft aneinandergereiht wären – aber man muſs nicht alles als Kette um den Hals
               tragen können. Im vierten Kapitel\pwindex{Beer-Hofmann, Richard 11.07.1866 – 26.09.1945@\textsc{Beer-Hofmann, Richard} (11.07.1866 – 26.09.1945), \emph{Schriftsteller}!Tod Georgs1900@\strich\emph{Der Tod Georgs} {[}1900{]}|pwv}{ }ſteckt übrigens irgend wo ein frecher Schwindel –
               das dürfte Ihnen nicht unbekannt ſein. Sie ſetzen ſich ſozuſagen plötzlich an eine
               andre Orgel, die auch herrlich klingt – {\pb}aber das
               beweiſt nichts. – Nicht überall ſcheint es mir geglückt, daſs gegenwärtiges und
               erinnertes ſich gegeneinander abhebt, wie es ſoll; daſs man das Bedürfnis hat, das
                  Buch\pwindex{Beer-Hofmann, Richard 11.07.1866 – 26.09.1945@\textsc{Beer-Hofmann, Richard} (11.07.1866 – 26.09.1945), \emph{Schriftsteller}!Tod Georgs1900@\strich\emph{Der Tod Georgs} {[}1900{]}|pwv} wieder zu leſen \strikeout{dagegen} iſt ja ſehr ſchön; aber dſs man es entschieden
               2–3 Mal lesen \uline{muſs}, iſt vielleicht ein Fehler. Ihre
               Bilderpracht ſchreit nach Jamben {\pb}und nach Drama. Ja
               es verlangt mich geradezu, einige von Ihren Vergleichen in Ihren Stücken
               wiederzufinden und ſie auf der Bühne ſprechen zu hören. – Wunderbar iſt, wie
               ſcheinbar belangloſe Details zu ihrer Zeit ausgenützt und nachträglich voll Belang
               erſcheinen. Das gibt den gewiſſen Schauer. Überhaupt: meiner {\pb}Empfindg nach ſteckt viel mehr Dichteriſches in dem
                  Buch\pwindex{Beer-Hofmann, Richard 11.07.1866 – 26.09.1945@\textsc{Beer-Hofmann, Richard} (11.07.1866 – 26.09.1945), \emph{Schriftsteller}!Tod Georgs1900@\strich\emph{Der Tod Georgs} {[}1900{]}|pwv} als, wie gewiſs vielfach
               behauptet werden wird, Verſtand. Sie wiſſen wie ich das meine. So geſcheidt iſt bald
               einer – aber die Dinge \uline{ſo} sagen – ! Um Goethe\pwindex{Goethe, Johann Wolfgang von 28.08.1749 – 22.03.1832@\textsc{Goethe, Johann Wolfgang von} (28.08.1749 – 22.03.1832), \emph{Schriftsteller}|pw} zu variiren: Alles gescheidte iſt ſchon einmal geſagt worden: man muſs nur
                  verſuchen, es \label{K_L01017_1v}\edtext{– ganz anders zu
                     sagen.}{\lemma{\textnormal{\emph{– ganz anders zu
                     sagen.}}}\Cendnote{\textnormal{Bei Goethe\pwindex{Goethe, Johann Wolfgang von 28.08.1749 – 22.03.1832@\textsc{Goethe, Johann Wolfgang von} (28.08.1749 – 22.03.1832), \emph{Schriftsteller}|pwk} endet es: »es noch einmal zu
                        denken.«}}}\label{K_L01017_1h}\pwindex{Goethe, Johann Wolfgang von 28.08.1749 – 22.03.1832@\textsc{Goethe, Johann Wolfgang von} (28.08.1749 – 22.03.1832), \emph{Schriftsteller}!Wilhelm Meisters Wanderjahre1821@\strich\emph{Wilhelm Meisters Wanderjahre} {[}1821{]}|pwv}{ }{\pb}Und »\label{K_L01017_2v}\edtext{\textsc{ma foi}}{\lemma{\textnormal{\emph{ma foi}}}\Cendnote{\textnormal{französisch: meiner Treu}}}\label{K_L01017_2h}« das haben Sie
               gethan. –\pend
           \pstart
           Während ich dieſes ſchreibe ſitze ich allein im Speiſeſaal, abends
               9 Uhr. Außer mir lebt hier nemlich nur ein (noch) älterer Herr. Montag fahr
               ich wohl wieder nach Wien\oindex{Wien@\textbf{Wien}|pw}. Ich ſehn mich nach
               niemandem – niemand ſehnt ſich nach mir. Das iſt nicht ſenti{\pb}mental – ſondern das iſt eben ſo. Heut vor einem Jahr
               war alles noch ſo anders – und doch \label{K_L01017_3v}\edtext{ſchwebte es ſchon über uns}{\lemma{\textnormal{\emph{ſchwebte … uns}}}\Cendnote{\textnormal{der Tod Marie Reinhard\pwindex{Reinhard, Marie 13.03.1871 – 18.03.1899@\textsc{Reinhard, Marie} (13.03.1871 – 18.03.1899), \emph{Gesangspädagogin}|pwk}s}}}\label{K_L01017_3h}{\dotstwo} Ja ja, es ſchwebt immer{\dots} »Zeit iſt nur ein Wort —\pwindex{Schnitzler, Arthur 15.05.1862 – 21.10.1931@\textsc{Schnitzler, Arthur} (15.05.1862 – 21.10.1931), \emph{Schriftsteller, Mediziner}!Schleier der Beatrice. Schauspiel in fuenf Akten1900-12-01 – 1900-12-01@\strich\emph{Der Schleier der Beatrice. Schauspiel in fünf Akten} {[}1900-12-01 – 1900-12-01{]}|pwv}« Könnte
               von Ihnen, von Hugo\pwindex{Hofmannsthal, Hugo von 01.02.1874 – 15.07.1929@\textsc{Hofmannsthal, Hugo von} (01.02.1874 – 15.07.1929), \emph{Schriftsteller}|pw} und von mir \introOben{}(und etlichen andern)\introOben{} ſein. Zufällig ſagt es \textsc{Beatrice}\pwindex{Schnitzler, Arthur 15.05.1862 – 21.10.1931@\textsc{Schnitzler, Arthur} (15.05.1862 – 21.10.1931), \emph{Schriftsteller, Mediziner}!Schleier der Beatrice. Schauspiel in fuenf Akten1900-12-01 – 1900-12-01@\strich\emph{Der Schleier der Beatrice. Schauspiel in fünf Akten} {[}1900-12-01 – 1900-12-01{]}|pwv}. –\pend
           \pstart
           Wie lang denken Sie noch auf Reiſen zu ſein? Ich ſchicke {\pb}dieſen Brief nach Florenz\oindex{Florenz@\textbf{Florenz}|pw}, wo ich Sie, glücklicher und wenn Sie wünſchen weniger witzig als in
                  \textsc{Sanremo}\oindex{Sanremo@\textbf{Sanremo}|pw} vermuthe. – Mirjam\pwindex{Beer-Hofmann, Mirjam 04.09.1897 – 24.12.1984@\textsc{Beer-Hofmann, Mirjam} (04.09.1897 – 24.12.1984)|pw} hoff ich ſo luſtig als
               ſie war und Ihre Frau\pwindex{Beer-Hofmann, Paula 25.02.1879 – 30.10.1939@\textsc{Beer-Hofmann, Paula} (25.02.1879 – 30.10.1939)|pwv} ſo
               erholt, als man es von italieniſcher\oindex{Italien@\textbf{Italien}|pw} Luft erwarten
               sollte. –\pend
           \pstart
           Von Hugo\pwindex{Hofmannsthal, Hugo von 01.02.1874 – 15.07.1929@\textsc{Hofmannsthal, Hugo von} (01.02.1874 – 15.07.1929), \emph{Schriftsteller}|pw} weiſs ich noch immer nichts, und Gustav\pwindex{Schwarzkopf, Gustav 07.11.1853 – 13.11.1939@\textsc{Schwarzkopf, Gustav} (07.11.1853 – 13.11.1939), \emph{Schriftsteller}|pw}{ }{\pb}hab ich von Ihnen gegrüßt. Thun Sie das gleiche von
               mir an \textsc{Mayer}\pwindex{Mayer, Oskar 1876 – 15.05.1915@\textsc{Mayer, Oskar} (1876 – 15.05.1915), \emph{Schriftsteller, Beamter}|pw}, we{\geminationn} er ſchon mit Ihnen zuſa{\geminationm}engeſtoßen iſt (– was hoffentlich nicht weh gethan
               hat.)\pend
           \pstart
           Leben Sie wohl!\pend
           \pstart Ihr \spacefill\mbox{Arthur}\pend{}\endnumbering\briefempfaengerindex{Beer-Hofmann, Richard@\textsc{Beer-Hofmann, Richard}!zzzSchnitzler, Arthur@\emph{von Arthur Schnitzler}!1900-03-021@{2. 3. 1900}|)be}\mylabel{h}\end{ledgroupsized}  \newcommand{\dateiname}{L01017}\newcommand{\titel}{Arthur Schnitzler an Richard Beer-Hofmann, 2. 3. 1900}\newcommand{\editorInnen}{Martin Anton Müller und Gerd-Hermann Susen}%% latex-leseansicht-abspann.tex
%% Abspann für die Leseansicht.
%% Der Schalter \ifkorrekturansicht ist bereits durch den Vorspann gesetzt.

%% latex-abspann.tex
%% Gemeinsamer Abspann für Korrekturansicht und Leseansicht.
%% Setzt den Schalter \ifkorrekturansicht voraus (gesetzt in den
%% einbindenden Dateien latex-korrekturansicht-abspann.tex bzw.
%% latex-leseansicht-abspann.tex).
%% ---------------------------------------------------------------

\normalsize

% Das esempio-Environment wird nur in der Leseansicht benötigt
\ifkorrekturansicht\else
\newenvironment{esempio}[3]%
{
    \vspace{1.5ex}
    \rlap{\underline{#1}}
    \par
    \setlength{\parindent}{0cm}
    \nopagebreak
    \leftskip=#2cm
    \rightskip=#3cm
}
{
    \par
}
\fi

\doendnotes{C}
\bigskip
\vfill

\clearpage

\footnotesize

\ifkorrekturansicht
  \lohead{\textsc{register}}
\fi

% theindex-Environment neu definieren ohne reledmac
\makeatletter
\renewenvironment{theindex}{%
  \ifkorrekturansicht
    \section*{\indexname}%
  \else
    \subsubsection*{Index der erwähnten Entitäten}%
  \fi
  \setlength{\parindent}{0pt}%
  \setlength{\parskip}{0pt plus 0.3pt}%
  \let\item\@idxitem
}{%
  \ifkorrekturansicht\clearpage\fi
}
\makeatother

\IfFileExists{\jobname-pw.ind}{\input{\jobname-pw.ind}}{}

% Quellenangabe nur in der Leseansicht
\ifkorrekturansicht\else
% Fallback-Definitionen, falls die .tex-Datei \titel etc. nicht gesetzt hat
\providecommand{\titel}{}
\providecommand{\editorInnen}{}
\providecommand{\dateiname}{\jobname}

\vspace{3cm}

\vfill

\footnotesize
\textsc{Quelle}: \titel. Herausgegeben von {\editorInnen}. In: \emph{Arthur Schnitzler: Briefwechsel mit Autorinnen und Autoren}.
 Digitale Edition, https://schnitzler-briefe.acdh.oeaw.ac.at/{\dateiname}.html (Stand \today)
\fi

\end{document}


      