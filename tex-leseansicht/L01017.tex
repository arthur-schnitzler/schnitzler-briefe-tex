%% latex-korrekturansicht-vorspann.tex
%% Vorspann für die Korrekturansicht.
%% Lädt die gemeinsame Datei latex-vorspann.tex mit gesetztem Schalter.

\newif\ifkorrekturansicht
\korrekturansichttrue

\input{../tex-inputs/latex-vorspann}


\section[Arthur Schnitzler an Richard Beer-Hofmann, 2. 3. 1900]{L01017 Arthur Schnitzler an Richard Beer-Hofmann, 2. 3. 1900}
\nopagebreak\mylabel{L01017v}
\rehead{ }\normalsize\beginnumbering\briefempfaengerindex{Beer-Hofmann, Richard@\textsc{Beer-Hofmann, Richard}!zzzSchnitzler, Arthur@\emph{von Arthur Schnitzler}!1900-03-021@{2. 3. 1900}|(be}
\toendnotes[C]{\smallbreak\pagebreak[2]}\Standort{YCGL, MSS 31.}
\physDesc{Brief, 3 Blätter, 10 Seiten, 2970 Zeichen
\newline{}Handschrift: Bleistift, deutsche Kurrent}
\buchAbdrucke{\weitereDrucke{1) Arthur Schnitzler: \emph{Briefe 1875–1912}. Frankfurt am Main: \emph{S. Fischer} 1981, S. 380–382.} \weitereDrucke{2) Arthur Schnitzler, Richard Beer-Hofmann: \emph{Briefwechsel 1891–1931}. Wien, Zürich: \emph{Europaverlag} 1992, S. 144–145.} }\toendnotes[C]{\smallbreak}
\pstart
           \raggedleft{}{\pb}\textsc{Edlacher Hof}\oindex{Hotel Edlacherhof@\textbf{Hotel Edlacherhof}, \emph{Hotel (K.HTL)}|pw},\pend
           
\pstart
           \raggedleft{}2. März 1900.\pend
           \vspace{0.5em}
\pstart
           mein lieber Richard, vorgeſtern Abend bin ich hier angeko{\geminationm}en, ich wollte dem Frühling entgegenfahren – und ſeit
               geſtern ſchneit und friert es. I{\geminationm}erhin iſt es in den
               Mittagſtunden ſchön. Heut ſowohl als geſtern bin ich nahezu 6 Stunden ſpazieren
               gegangen. Weniger lang {\pb}ſchrieb ich an der Novelle\pwindex{Frau Bertha Garlan. Roman@\emph{Frau Bertha Garlan. Roman}|pwv}, für die ich keinen
               Namen habe.\pend
           
\pstart
           Ihre\pwindex{Tod Georgs@\emph{Der Tod Georgs}|pwv} hab’ ich in 2 Etappen
               geleſen, die erſten 2 Capitel in der Eiſenbahn, die letzten 2 geſtern Abend auf
               meinem Zimmer (3. außer 4. im Bett) Also glauben Sie mir: es iſt ein wundervolles Buch\pwindex{Tod Georgs@\emph{Der Tod Georgs}|pwv}. Man hat allerdings das
               Gefühl, als wenn die aneinandergereihten Edelſteine nicht auf einer Schnur, {\pb}ſondern auf einem Zwirnsfaden – oder gar nur in der
               Luft aneinandergereiht wären – aber man muſs nicht alles als Kette um den Hals tragen
               können. Im vierten Kapitel\pwindex{Tod Georgs@\emph{Der Tod Georgs}|pwv}{ }ſteckt übrigens irgend wo ein frecher Schwindel –
               das dürfte Ihnen nicht unbekannt ſein. Sie ſetzen ſich ſozuſagen plötzlich an eine
               andre Orgel, die auch herrlich klingt – {\pb}aber das
               beweiſt nichts. – Nicht überall ſcheint es mir geglückt, daſs gegenwärtiges und
               erinnertes ſich gegeneinander abhebt, wie es ſoll; daſs man das Bedürfnis hat, das
                  Buch\pwindex{Tod Georgs@\emph{Der Tod Georgs}|pwv} wieder zu leſen \strikeout{dagegen} iſt ja ſehr ſchön; aber dſs man es entschieden
               2–3 Mal lesen \uline{muſs}, iſt vielleicht ein Fehler. Ihre
               Bilderpracht ſchreit nach Jamben {\pb}und nach Drama. Ja
               es verlangt mich geradezu, einige von Ihren Vergleichen in Ihren Stücken
               wiederzufinden und ſie auf der Bühne ſprechen zu hören. – Wunderbar iſt, wie
               ſcheinbar belangloſe Details zu ihrer Zeit ausgenützt und nachträglich voll Belang
               erſcheinen. Das gibt den gewiſſen Schauer. Überhaupt: meiner {\pb}Empfindg nach ſteckt viel mehr Dichteriſches in dem
                  Buch\pwindex{Tod Georgs@\emph{Der Tod Georgs}|pwv} als, wie gewiſs
               vielfach behauptet werden wird, Verſtand. Sie wiſſen wie ich das meine. So geſcheidt
               iſt bald einer – aber die Dinge \uline{ſo} sagen – ! Um Goethe\pwindex{Goethe, Johann Wolfgang von 1749-08-28 – 1832-03-22@\textsc{Goethe, Johann Wolfgang von} (1749-08-28 – 1832-03-22), \emph{Schriftsteller/Schriftstellerin}|pw} zu variiren: Alles gescheidte iſt ſchon einmal geſagt worden:
                  man muſs nur verſuchen, es \label{K_L01017-1v}\edtext{– ganz
                  anders zu sagen.}{\lemma{\textnormal{\emph{– ganz
                  anders zu sagen.}}}\Cendnote{\textnormal{Bei Goethe\pwindex{Goethe, Johann Wolfgang von 1749-08-28 – 1832-03-22@\textsc{Goethe, Johann Wolfgang von} (1749-08-28 – 1832-03-22), \emph{Schriftsteller/Schriftstellerin}|pwk} endet es: »es noch einmal
                        zu denken.«}}}\label{K_L01017-1}\pwindex{Wilhelm Meisters Wanderjahre@\emph{Wilhelm Meisters Wanderjahre}|pwv}{ }{\pb}Und »\label{K_L01017-2v}\edtext{\textsc{ma foi}}{\lemma{\textnormal{\emph{ma foi}}}\Cendnote{\textnormal{französisch: meiner Treu}}}\label{K_L01017-2}« das
               haben Sie gethan. –\pend
           
\pstart
           Während ich dieſes ſchreibe ſitze ich allein im Speiſeſaal, abends
               9 Uhr. Außer mir lebt hier nemlich nur ein (noch) älterer Herr. Montag fahr
               ich wohl wieder nach Wien\oindex{Wien@\textbf{Wien}, \emph{A.ADM2}|pw}. Ich ſehn mich nach
               niemandem – niemand ſehnt ſich nach mir. Das iſt nicht ſenti{\pb}mental – ſondern das iſt eben ſo. Heut vor einem Jahr
               war alles noch ſo anders – und doch \label{K_L01017-3v}\edtext{ſchwebte es ſchon über uns}{\lemma{\textnormal{\emph{ſchwebte … uns}}}\Cendnote{\textnormal{der Tod Marie Reinhards\pwindex{Reinhard, Marie 1871-03-13 – 1899-03-18@\textsc{Reinhard, Marie} (1871-03-13 – 1899-03-18), \emph{Gesangspädagoge/Gesangspädagogin}|pwk}}}}\label{K_L01017-3}{\dotstwo} Ja ja, es ſchwebt immer{\dots} »Zeit iſt nur ein Wort –\pwindex{Schleier der Beatrice. Schauspiel in fuenf Akten@\emph{Der Schleier der Beatrice. Schauspiel in fünf Akten}|pwv}« Könnte
               von Ihnen, von Hugo\pwindex{Hofmannsthal, Hugo von 1874-02-01 – 1929-07-15@\textsc{Hofmannsthal, Hugo von} (1874-02-01 – 1929-07-15), \emph{Schriftsteller/Schriftstellerin}|pw} und von mir \introOben{}(und etlichen andern)\introOben{} ſein. Zufällig ſagt es \textsc{Beatrice}\pwindex{Schleier der Beatrice. Schauspiel in fuenf Akten@\emph{Der Schleier der Beatrice. Schauspiel in fünf Akten}|pwv}. –\pend
           
\pstart
           Wie lang denken Sie noch auf Reiſen zu ſein? Ich ſchicke {\pb}dieſen Brief nach Florenz\oindex{Florenz@\textbf{Florenz}, \emph{P.PPLA}|pw}, wo ich Sie, glücklicher und wenn Sie wünſchen weniger witzig als in
                  \textsc{Sanremo}\oindex{Sanremo@\textbf{Sanremo}, \emph{P.PPLA3}|pw} vermuthe. – Mirjam\pwindex{Beer-Hofmann, Mirjam 04.09.1897 – 24.12.1984@\textsc{Beer-Hofmann, Mirjam} (04.09.1897 – 24.12.1984)|pw} hoff ich ſo luſtig als
               ſie war und Ihre Frau\pwindex{Beer-Hofmann, Paula 25.02.1879 – 30.10.1939@\textsc{Beer-Hofmann, Paula} (25.02.1879 – 30.10.1939)|pwv} ſo
               erholt, als man es von italieniſcher\oindex{Italien@\textbf{Italien}, \emph{A.PCLI}|pw} Luft
               erwarten sollte. –\pend
           
\pstart
           Von Hugo\pwindex{Hofmannsthal, Hugo von 1874-02-01 – 1929-07-15@\textsc{Hofmannsthal, Hugo von} (1874-02-01 – 1929-07-15), \emph{Schriftsteller/Schriftstellerin}|pw} weiſs ich noch immer nichts, und Gustav\pwindex{Schwarzkopf, Gustav 07.11.1853 – 13.11.1939@\textsc{Schwarzkopf, Gustav} (07.11.1853 – 13.11.1939), \emph{Schriftsteller/Schriftstellerin}|pw}{ }{\pb}hab ich von Ihnen gegrüßt. Thun Sie das gleiche von
               mir an \textsc{Mayer}\pwindex{Mayer, Oskar 1876 – 15.05.1915@\textsc{Mayer, Oskar} (1876 – 15.05.1915), \emph{Schriftsteller/Schriftstellerin, Beamter/Beamte}|pw}, we{\geminationn} er ſchon mit Ihnen zuſa{\geminationm}engeſtoßen iſt (– was hoffentlich nicht weh gethan
               hat.)\pend
           
\pstart
           Leben Sie wohl!\pend
           \pstart Ihr \spacefill\mbox{Arthur}\pend{}\selectlanguage{ngerman}\endnumbering\briefempfaengerindex{Beer-Hofmann, Richard@\textsc{Beer-Hofmann, Richard}!zzzSchnitzler, Arthur@\emph{von Arthur Schnitzler}!1900-03-021@{2. 3. 1900}|)be}\mylabel{L01017h}  \normalsize

\doendnotes{C}
\bigskip
\vfill

\clearpage

\footnotesize

\lohead{\textsc{register}}

% Definiere theindex-Environment komplett neu ohne reledmac
\makeatletter
\renewenvironment{theindex}{%
  \section*{\indexname}%
  \setlength{\parindent}{0pt}%
  \setlength{\parskip}{0pt plus 0.3pt}%
  \let\item\@idxitem
}{%
  \clearpage
}
\makeatother

\IfFileExists{\jobname-pw.ind}{\input{\jobname-pw.ind}}{}

\end{document}

      