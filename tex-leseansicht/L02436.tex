%% latex-leseansicht-vorspann.tex
%% Vorspann für die Leseansicht.
%% Lädt die gemeinsame Datei latex-vorspann.tex mit nicht gesetztem Schalter.

\newif\ifkorrekturansicht
\korrekturansichtfalse

\input{../tex-inputs/latex-vorspann}


         
         \renewcommand{\erwaehntePersonen}{Personen:  Benedikt XII., Hugo von Hofmannsthal,  Johannes XXII.}
         \renewcommand{\erwaehnteOrte}{Orte: Avignon, Bahnhof Marseille, Kathedrale von Avignon, Sternwartestraße 71, Wien, Österreich}
         \renewcommand{\erwaehnteWerke}{}
               \section[Hugo Hofmannsthal an Arthur Schnitzler, 3. 3. 1925]{ Hugo Hofmannsthal an Arthur Schnitzler, 3. 3. 1925}\nopagebreak\mylabel{v}\rehead{ }\begin{ledgroupsized}[t]{13cm}\normalsize\beginnumbering\briefempfaengerindex{Schnitzler, Arthur@\textsc{Schnitzler, Arthur}!zzzHofmannsthal, Hugo von@\emph{von Hugo von Hofmannsthal}!1925-03-031@{3. 3. 1925}|(be} \toendnotes[C]{\smallbreak\pagebreak[2]} \Standort{CUL, Schnitzler, B 43.}
\physDesc{Bildpostkarte, 138 Zeichen
\newline{}Handschrift: Bleistift, lateinische Kurrent
\newline{}Versand: Stempel: »\nobreak{}\oindex{Bahnhof Marseille@\textbf{Bahnhof Marseille}|pwk}Marseille – Gare B\textsuperscript{ches} du Rhône, 4-III 1925, 17\textsuperscript{30}\nobreak{}«.  
\newline{}Ordnung: 1) mit Bleistift von unbekannter Hand nummeriert: »\strikeout{388}«  2) mit Bleistift von unbekannter Hand nummeriert:
                                    »388«}\buchAbdrucke{\weitereDrucke{Hugo von Hofmannsthal, Arthur Schnitzler: \emph{Briefwechsel}. Hg. Therese Nickl und Heinrich Schnitzler. Frankfurt am Main: \emph{S. Fischer} 1964, S. 301.} }\pstart{}{\pb}Arthur Schnitzler\pend{}\pstart{}XVIII Sternwartestrasse 71\oindex{Sternwartestrasse 71@\textbf{Sternwartestraße 71}|pw}\pend{}\pstart{}Wien\oindex{Wien@\textbf{Wien}|pw}\pend{}\pstart{}Autriche\oindex{Oesterreich@\textbf{Österreich}|pw}\pend{}{\bigskip}\pstart
           \noindent{}{\pb}\textcolor{gray}{\textbf{\begin{otherlanguage}{french}AVIGNON\oindex{Avignon@\textbf{Avignon}|pw} – N.-D des Doms (Cathédrale\oindex{Kathedrale von Avignon@\textbf{Kathedrale von Avignon}|pw}) Tour
                     Campane et partie du Palais construite à l’avènement du Pape Jean XXII\pwindex{Johannes XXII. 1244 – 1334-12-03@\textsc{Johannes XXII.} (1244 – 1334-12-03), \emph{Papst}|pw} (1316), sur l’emplacement de
                     l’Eglise St-Etienne. Benoit XII\pwindex{Benedikt XII. 1285 – 1342-04-25@\textsc{Benedikt XII.} (1285 – 1342-04-25), \emph{Papst}|pw},
                        (1335–1342) sur les plans de Pierre Obrerie
                     architecte français compléta l’œuvre de son prédécesseur. Cette partie du
                     monument renferme aujourd’hui les Archives départementales.\end{otherlanguage}}}\pend
           \pstart
           {\pb}Da ist schon die erste Karte mit
               vielen Grüßen u. Gedanken!\pend
           \pstart \spacefill\mbox{Hugo.}\pend{}\pstart
           Avignon\oindex{Avignon@\textbf{Avignon}|pw}{ }3 III 1925.\pend
           
         
         \endnumbering\mylabel{h}\end{ledgroupsized}  \newcommand{\dateiname}{L02436}\newcommand{\titel}{Hugo Hofmannsthal an Arthur Schnitzler, 3. 3. 1925}\newcommand{\editorInnen}{Martin Anton Müller und Gerd-Hermann Susen}%% latex-leseansicht-abspann.tex
%% Abspann für die Leseansicht.
%% Der Schalter \ifkorrekturansicht ist bereits durch den Vorspann gesetzt.

%% latex-abspann.tex
%% Gemeinsamer Abspann für Korrekturansicht und Leseansicht.
%% Setzt den Schalter \ifkorrekturansicht voraus (gesetzt in den
%% einbindenden Dateien latex-korrekturansicht-abspann.tex bzw.
%% latex-leseansicht-abspann.tex).
%% ---------------------------------------------------------------

\normalsize

% Das esempio-Environment wird nur in der Leseansicht benötigt
\ifkorrekturansicht\else
\newenvironment{esempio}[3]%
{
    \vspace{1.5ex}
    \rlap{\underline{#1}}
    \par
    \setlength{\parindent}{0cm}
    \nopagebreak
    \leftskip=#2cm
    \rightskip=#3cm
}
{
    \par
}
\fi

\doendnotes{C}
\bigskip
\vfill

\clearpage

\footnotesize

\ifkorrekturansicht
  \lohead{\textsc{register}}
\fi

% theindex-Environment neu definieren ohne reledmac
\makeatletter
\renewenvironment{theindex}{%
  \ifkorrekturansicht
    \section*{\indexname}%
  \else
    \subsubsection*{Index der erwähnten Entitäten}%
  \fi
  \setlength{\parindent}{0pt}%
  \setlength{\parskip}{0pt plus 0.3pt}%
  \let\item\@idxitem
}{%
  \ifkorrekturansicht\clearpage\fi
}
\makeatother

\IfFileExists{\jobname-pw.ind}{\input{\jobname-pw.ind}}{}

% Quellenangabe nur in der Leseansicht
\ifkorrekturansicht\else
% Fallback-Definitionen, falls die .tex-Datei \titel etc. nicht gesetzt hat
\providecommand{\titel}{}
\providecommand{\editorInnen}{}
\providecommand{\dateiname}{\jobname}

\vspace{3cm}

\vfill

\footnotesize
\textsc{Quelle}: \titel. Herausgegeben von {\editorInnen}. In: \emph{Arthur Schnitzler: Briefwechsel mit Autorinnen und Autoren}.
 Digitale Edition, https://schnitzler-briefe.acdh.oeaw.ac.at/{\dateiname}.html (Stand \today)
\fi

\end{document}


      