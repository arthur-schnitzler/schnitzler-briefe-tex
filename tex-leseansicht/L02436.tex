%% latex-korrekturansicht-vorspann.tex
%% Vorspann für die Korrekturansicht.
%% Lädt die gemeinsame Datei latex-vorspann.tex mit gesetztem Schalter.

\newif\ifkorrekturansicht
\korrekturansichttrue

\input{../tex-inputs/latex-vorspann}


\section[Hugo Hofmannsthal an Arthur Schnitzler, 3. 3. 1925]{L02436 Hugo Hofmannsthal an Arthur Schnitzler, 3. 3. 1925}
\nopagebreak\mylabel{L02436v}
\rehead{ }\normalsize\beginnumbering\briefempfaengerindex{Schnitzler, Arthur@\textsc{Schnitzler, Arthur}!zzzHofmannsthal, Hugo von@\emph{von Hugo von Hofmannsthal}!1925-03-031@{3. 3. 1925}|(be}
\toendnotes[C]{\smallbreak\pagebreak[2]}\Standort{CUL, Schnitzler, B 43.}
\physDesc{Bildpostkarte, 138 Zeichen
\newline{}Handschrift: Bleistift, lateinische Kurrent
\newline{}Versand: Stempel: »\nobreak{}\oindex{Bahnhof Marseille@\textbf{Bahnhof Marseille}, \emph{Bahnhofsgebäude (K.BHF)}|pwk}Marseille – Gare B\textsuperscript{ches} du Rhône, 4-III 1925, 17\textsuperscript{30}\nobreak{}«.  
\newline{}Ordnung: 1) mit Bleistift von unbekannter Hand nummeriert: »\strikeout{388}«  2) mit Bleistift von unbekannter Hand nummeriert:
                                    »388«}
\buchAbdrucke{\weitereDrucke{Hugo von Hofmannsthal, Arthur Schnitzler: \emph{Briefwechsel}. Frankfurt am Main: \emph{S. Fischer} 1964, S. 301.} }\pstart{}{\pb}Arthur Schnitzler\pend{}\pstart{}XVIII Sternwartestrasse 71\oindex{Sternwartestrasse 71@\textbf{Sternwartestraße 71}, \emph{Wohngebäude (K.WHS)}|pw}\pend{}\pstart{}Wien\oindex{Wien@\textbf{Wien}, \emph{A.ADM2}|pw}\pend{}\pstart{}Autriche\oindex{Oesterreich@\textbf{Österreich}, \emph{A.PCLI}|pw}\pend{}{\bigskip}
\pstart
           \noindent{}{\pb}\textcolor{gray}{\textbf{\begin{otherlanguage}{french}AVIGNON\oindex{Avignon@\textbf{Avignon}, \emph{P.PPLA2}|pw} – N.-D des Doms (Cathédrale\oindex{Kathedrale von Avignon@\textbf{Kathedrale von Avignon}, \emph{Kirche (K.KRC)}|pw}) Tour
                     Campane et partie du Palais construite à l’avènement du Pape Jean XXII\pwindex{Johannes XXII. 1244 – 1334-12-03@\textsc{Johannes XXII.} (1244 – 1334-12-03), \emph{Papst/Päpstin}|pw} (1316), sur l’emplacement de
                     l’Eglise St-Etienne. Benoit XII\pwindex{Benedikt XII. 1285 – 1342-04-25@\textsc{Benedikt XII.} (1285 – 1342-04-25), \emph{Papst/Päpstin}|pw},
                     (1335–1342) sur les plans de Pierre Obrerie
                     architecte français compléta l’œuvre de son prédécesseur. Cette partie du
                     monument renferme aujourd’hui les Archives départementales.\end{otherlanguage}}}\pend
           \vspace{1em}
\pstart
           \noindent{}{\pb}Da ist schon die erste Karte mit
               vielen Grüßen u. Gedanken!\pend
           \pstart \spacefill\mbox{Hugo.}\pend{}
\pstart
           Avignon\oindex{Avignon@\textbf{Avignon}, \emph{P.PPLA2}|pw}{ }3 III 1925.\pend
           \selectlanguage{ngerman}\endnumbering\briefempfaengerindex{Schnitzler, Arthur@\textsc{Schnitzler, Arthur}!zzzHofmannsthal, Hugo von@\emph{von Hugo von Hofmannsthal}!1925-03-031@{3. 3. 1925}|)be}\mylabel{L02436h}  \normalsize

\doendnotes{C}
\bigskip
\vfill

\clearpage

\footnotesize

\lohead{\textsc{register}}

% Definiere theindex-Environment komplett neu ohne reledmac
\makeatletter
\renewenvironment{theindex}{%
  \section*{\indexname}%
  \setlength{\parindent}{0pt}%
  \setlength{\parskip}{0pt plus 0.3pt}%
  \let\item\@idxitem
}{%
  \clearpage
}
\makeatother

\IfFileExists{\jobname-pw.ind}{\input{\jobname-pw.ind}}{}

\end{document}

      