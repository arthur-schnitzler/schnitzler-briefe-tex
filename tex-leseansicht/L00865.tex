%% latex-leseansicht-vorspann.tex
%% Vorspann für die Leseansicht.
%% Lädt die gemeinsame Datei latex-vorspann.tex mit nicht gesetztem Schalter.

\newif\ifkorrekturansicht
\korrekturansichtfalse

\input{../tex-inputs/latex-vorspann}


         
         \newcommand{\erwaehntePersonen}{Personen: Otto Eisenschitz, Amilcare Lauria}
         \newcommand{\erwaehnteInstitutionen}{Institutionen: Fanfulla della domenica}
         \newcommand{\erwaehnteOrte}{Orte: Rom, Wien}
         \newcommand{\erwaehnteWerke}{Werke: Ein Epilog}
               \section[Hugo von Hofmannsthal an Arthur Schnitzler, 3. 12. 1898]{ Hugo von Hofmannsthal an Arthur Schnitzler, 3. 12. 1898}\nopagebreak\mylabel{v}\rehead{ }\begin{ledgroupsized}[t]{13cm}\normalsize\beginnumbering \toendnotes[C]{\smallbreak\pagebreak[2]} \Standort{CUL, Schnitzler, B 43.}
\physDesc{Brief, 1 Blatt, 4 Seiten
\newline{}Handschrift: schwarze Tinte, deutsche Kurrent\newline{}Ordnung: mit Bleistift von unbekannter Hand nummeriert: »\strikeout{131}
                                    128« }\buchAbdrucke{\weitereDrucke{Hugo von Hofmannsthal, Arthur Schnitzler: \emph{Briefwechsel}. Hg. Therese Nickl und Heinrich Schnitzler. Frankfurt am Main: \emph{S. Fischer} 1964, S. 115.} }\pstart
           \raggedleft{}{\pb}3. XII. 98.\pend
           \pstart{}mein lieber Arthur\pend\pstart
           ich bitte Sie vielmals um eine Gefälligkeit, nämlich daſs Sie Herrn Otto Eiſenſchitz\pwindex{Eisenschitz, Otto 22.02.1863 – 11.09.1942@\textsc{Eisenschitz, Otto} (22.02.1863 – 11.09.1942), \emph{Schriftsteller, Journalist, Dramaturg}|pw}, den Sie ja perſönlich kennen,
                    einen Brief ſchreiben, oder daſs Sie ihm dieſen Brief hier ſchicken.\pend
           \pstart
           Herr \textsc{Lauria}\pwindex{Lauria, Amilcare 3.4.1854 – 1932@\textsc{Lauria, Amilcare} (3.4.1854 – 1932), \emph{Journalist, Kritiker}|pw} in \textsc{Rom}\oindex{Rom@\textbf{Rom}|pw}, Redacteur der \textsc{Fanfulla}\orgindex{Fanfulla della domenica@Fanfulla della domenica|pw}, hat ſich an mich um \textsc{Intervention}{ }{\pb}gewandt, weil Herr Eiſenſchitz\pwindex{Eisenschitz, Otto 22.02.1863 – 11.09.1942@\textsc{Eisenschitz, Otto} (22.02.1863 – 11.09.1942), \emph{Schriftsteller, Journalist, Dramaturg}|pw} ein einactiges Manuſcript von ihm
                        »\textsc{ein Epilog}\pwindex{Lauria, Amilcare 3.4.1854 – 1932@\textsc{Lauria, Amilcare} (3.4.1854 – 1932), \emph{Journalist, Kritiker}!EpilogNone@\strich\emph{Ein Epilog} {[}None{]}|pw}« zum Überſetzen und zum Vertrieb bei den Bühnen übernommen hat und
                    Herr \textsc{Lauria}\pwindex{Lauria, Amilcare 3.4.1854 – 1932@\textsc{Lauria, Amilcare} (3.4.1854 – 1932), \emph{Journalist, Kritiker}|pw} nun trotz mehrfacher Briefe keine Auskunft über den Verlauf dieſer
                    Sache bekommen kann, ja nicht einmal {\pb}weiß, ob das Stück bis jetzt
                        \introOben{}von Herrn Eiſenſchitz\pwindex{Eisenschitz, Otto 22.02.1863 – 11.09.1942@\textsc{Eisenschitz, Otto} (22.02.1863 – 11.09.1942), \emph{Schriftsteller, Journalist, Dramaturg}|pw}\introOben{} ins Deutſche überſetzt wurde.\pend
           \pstart
           Wahrſcheinlich liegt hier ein Miſsverſtändnis vor und Herr Eiſenſchitz\pwindex{Eisenschitz, Otto 22.02.1863 – 11.09.1942@\textsc{Eisenschitz, Otto} (22.02.1863 – 11.09.1942), \emph{Schriftsteller, Journalist, Dramaturg}|pw} wird wohl ſo freundlich ſein, an Sie
                    eine aufklärende Zeile zu richten. Übrigens iſt Herr \textsc{Lauria}\pwindex{Lauria, Amilcare 3.4.1854 – 1932@\textsc{Lauria, Amilcare} (3.4.1854 – 1932), \emph{Journalist, Kritiker}|pw} ein {\pb}Autor, von
                    dem ich viel Gutes gehört habe.\pend
           \pstart
           Herzlich Ihr{\\[\baselineskip]}\spacefill\mbox{Hofmannsthal}\pend
           \leftskip=0em{}
         
         \endnumbering\mylabel{h}\end{ledgroupsized}  \newcommand{\dateiname}{L00865}\newcommand{\titel}{Hugo von Hofmannsthal an Arthur Schnitzler, 3. 12. 1898}\newcommand{\editorInnen}{Martin Anton Müller und Gerd-Hermann Susen}%% latex-leseansicht-abspann.tex
%% Abspann für die Leseansicht.
%% Der Schalter \ifkorrekturansicht ist bereits durch den Vorspann gesetzt.

%% latex-abspann.tex
%% Gemeinsamer Abspann für Korrekturansicht und Leseansicht.
%% Setzt den Schalter \ifkorrekturansicht voraus (gesetzt in den
%% einbindenden Dateien latex-korrekturansicht-abspann.tex bzw.
%% latex-leseansicht-abspann.tex).
%% ---------------------------------------------------------------

\normalsize

% Das esempio-Environment wird nur in der Leseansicht benötigt
\ifkorrekturansicht\else
\newenvironment{esempio}[3]%
{
    \vspace{1.5ex}
    \rlap{\underline{#1}}
    \par
    \setlength{\parindent}{0cm}
    \nopagebreak
    \leftskip=#2cm
    \rightskip=#3cm
}
{
    \par
}
\fi

\doendnotes{C}
\bigskip
\vfill

\clearpage

\footnotesize

\ifkorrekturansicht
  \lohead{\textsc{register}}
\fi

% theindex-Environment neu definieren ohne reledmac
\makeatletter
\renewenvironment{theindex}{%
  \ifkorrekturansicht
    \section*{\indexname}%
  \else
    \subsubsection*{Index der erwähnten Entitäten}%
  \fi
  \setlength{\parindent}{0pt}%
  \setlength{\parskip}{0pt plus 0.3pt}%
  \let\item\@idxitem
}{%
  \ifkorrekturansicht\clearpage\fi
}
\makeatother

\IfFileExists{\jobname-pw.ind}{\input{\jobname-pw.ind}}{}

% Quellenangabe nur in der Leseansicht
\ifkorrekturansicht\else
% Fallback-Definitionen, falls die .tex-Datei \titel etc. nicht gesetzt hat
\providecommand{\titel}{}
\providecommand{\editorInnen}{}
\providecommand{\dateiname}{\jobname}

\vspace{3cm}

\vfill

\footnotesize
\textsc{Quelle}: \titel. Herausgegeben von {\editorInnen}. In: \emph{Arthur Schnitzler: Briefwechsel mit Autorinnen und Autoren}.
 Digitale Edition, https://schnitzler-briefe.acdh.oeaw.ac.at/{\dateiname}.html (Stand \today)
\fi

\end{document}


      