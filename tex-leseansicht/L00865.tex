%% latex-korrekturansicht-vorspann.tex
%% Vorspann für die Korrekturansicht.
%% Lädt die gemeinsame Datei latex-vorspann.tex mit gesetztem Schalter.

\newif\ifkorrekturansicht
\korrekturansichttrue

\input{../tex-inputs/latex-vorspann}


\section[Hugo von Hofmannsthal an Arthur Schnitzler, 3. 12. 1898]{L00865 Hugo von Hofmannsthal an Arthur Schnitzler, 3. 12. 1898}
\nopagebreak\mylabel{L00865v}
\rehead{ }\normalsize\beginnumbering\briefempfaengerindex{Schnitzler, Arthur@\textsc{Schnitzler, Arthur}!zzzHofmannsthal, Hugo von@\emph{von Hugo von Hofmannsthal}!1898-12-031@{3. 12. 1898}|(be}
\toendnotes[C]{\smallbreak\pagebreak[2]}\Standort{CUL, Schnitzler, B 43.}
\physDesc{Brief, 1 Blatt, 4 Seiten, 849 Zeichen
\newline{}Handschrift: schwarze Tinte, deutsche Kurrent
\newline{}Ordnung: mit Bleistift von unbekannter Hand nummeriert: »\strikeout{131} 128« }
\buchAbdrucke{\weitereDrucke{Hugo von Hofmannsthal, Arthur Schnitzler: \emph{Briefwechsel}. Frankfurt am Main: \emph{S. Fischer} 1964, S. 115.} }
\pstart
           \raggedleft{}{\pb}3. XII. 98.\pend
           
\pstart{}mein lieber Arthur\pend\vspace{0.5em}
\pstart
           ich bitte Sie vielmals um eine Gefälligkeit, nämlich daſs Sie Herrn Otto Eiſenſchitz\pwindex{Eisenschitz, Otto 22.02.1863 – 11.09.1942@\textsc{Eisenschitz, Otto} (22.02.1863 – 11.09.1942), \emph{Schriftsteller/Schriftstellerin, Journalist/Journalistin, Dramaturg/Dramaturgin}|pw}, den Sie ja perſönlich kennen,
               einen Brief ſchreiben, oder daſs Sie ihm dieſen Brief hier ſchicken.\pend
           
\pstart
           Herr \textsc{Lauria}\pwindex{Lauria, Amilcare 3.4.1854 – 1932@\textsc{Lauria, Amilcare} (3.4.1854 – 1932), \emph{Journalist/Journalistin, Kritiker/Kritikerin}|pw} in \textsc{Rom}\oindex{Rom@\textbf{Rom}, \emph{P.PPLC}|pw}, Redacteur der \textsc{Fanfulla}\orgindex{Fanfulla della domenica@Fanfulla della domenica|pw}, hat ſich an mich um \textsc{Intervention}{ }{\pb}gewandt, weil Herr Eiſenſchitz\pwindex{Eisenschitz, Otto 22.02.1863 – 11.09.1942@\textsc{Eisenschitz, Otto} (22.02.1863 – 11.09.1942), \emph{Schriftsteller/Schriftstellerin, Journalist/Journalistin, Dramaturg/Dramaturgin}|pw} ein einactiges Manuſcript von ihm
                  »\textsc{ein Epilog}\pwindex{Epilog@\emph{Ein Epilog}|pw}« zum Überſetzen und zum Vertrieb bei den Bühnen übernommen hat und Herr \textsc{Lauria}\pwindex{Lauria, Amilcare 3.4.1854 – 1932@\textsc{Lauria, Amilcare} (3.4.1854 – 1932), \emph{Journalist/Journalistin, Kritiker/Kritikerin}|pw} nun trotz mehrfacher Briefe keine Auskunft über den Verlauf dieſer Sache
               bekommen kann, ja nicht einmal {\pb}weiß, ob das Stück bis jetzt \introOben{}von Herrn Eiſenſchitz\pwindex{Eisenschitz, Otto 22.02.1863 – 11.09.1942@\textsc{Eisenschitz, Otto} (22.02.1863 – 11.09.1942), \emph{Schriftsteller/Schriftstellerin, Journalist/Journalistin, Dramaturg/Dramaturgin}|pw}\introOben{} ins Deutſche überſetzt wurde.\pend
           
\pstart
           Wahrſcheinlich liegt hier ein Miſsverſtändnis vor und Herr Eiſenſchitz\pwindex{Eisenschitz, Otto 22.02.1863 – 11.09.1942@\textsc{Eisenschitz, Otto} (22.02.1863 – 11.09.1942), \emph{Schriftsteller/Schriftstellerin, Journalist/Journalistin, Dramaturg/Dramaturgin}|pw} wird wohl ſo freundlich ſein, an Sie eine
               aufklärende Zeile zu richten. Übrigens iſt Herr \textsc{Lauria}\pwindex{Lauria, Amilcare 3.4.1854 – 1932@\textsc{Lauria, Amilcare} (3.4.1854 – 1932), \emph{Journalist/Journalistin, Kritiker/Kritikerin}|pw} ein {\pb}Autor, von dem ich viel
               Gutes gehört habe.\pend
           
\pstart
           Herzlich Ihr{\\[\baselineskip]}\spacefill\mbox{Hofmannsthal}\pend
           \leftskip=0em{}\selectlanguage{ngerman}\endnumbering\briefempfaengerindex{Schnitzler, Arthur@\textsc{Schnitzler, Arthur}!zzzHofmannsthal, Hugo von@\emph{von Hugo von Hofmannsthal}!1898-12-031@{3. 12. 1898}|)be}\mylabel{L00865h}  \normalsize

\doendnotes{C}
\bigskip
\vfill

\clearpage

\footnotesize

\lohead{\textsc{register}}

% Definiere theindex-Environment komplett neu ohne reledmac
\makeatletter
\renewenvironment{theindex}{%
  \section*{\indexname}%
  \setlength{\parindent}{0pt}%
  \setlength{\parskip}{0pt plus 0.3pt}%
  \let\item\@idxitem
}{%
  \clearpage
}
\makeatother

\IfFileExists{\jobname-pw.ind}{\input{\jobname-pw.ind}}{}

\end{document}

      