\input{../tex-inputs/latex-pdf-vorspann}
\begin{center}
            \textcolor{red}{ENTWURF. ENTZIFFERUNG NOCH NICHT KORREKTURGELESEN}
                      \end{center}
            
               \section[Hugo von Hofmannsthal an Arthur Schnitzler, {[}24?. 11. 1902{]}]{ Hugo von Hofmannsthal an Arthur Schnitzler, {[}24?. 11. 1902{]}}\nopagebreak\mylabel{v}\rehead{ }\begin{ledgroupsized}[t]{13cm}\normalsize\beginnumbering\briefempfaengerindex{Schnitzler, Arthur@\textsc{Schnitzler, Arthur}!zzzHofmannsthal, Hugo von@\emph{von Hugo von Hofmannsthal}!1902-11-241@{{[}24?. 11. 1902{]}}|(be} \toendnotes[C]{\smallbreak\pagebreak[2]} \Standort{CUL, Schnitzler, B 43.}
\physDesc{Brief, 1 Blatt, 2 Seiten
\newline{}Handschrift: schwarze Tinte, deutsche Kurrent
\newline{}Schnitzler: mit Bleistift datiert: »23/11 902« \newline{}Ordnung: mit Bleistift von unbekannter Hand nummeriert:
                                    »188« }\buchAbdrucke{\weitereDrucke{Hugo von Hofmannsthal, Arthur Schnitzler: \emph{Briefwechsel}. Hg. Therese Nickl und Heinrich Schnitzler. Frankfurt am Main: \emph{S. Fischer} 1964, S. 163.} }\toendnotes[C]{\smallbreak}\pstart
           \raggedleft{}{\pb}Rodaun\oindex{Rodaun@\textbf{Rodaun}|pw}{ }\label{K_L01251_1v}\edtext{Montg}{\lemma{\textnormal{\emph{Montg}}}\Cendnote{\textnormal{Schnitzler\pwindex{Schnitzler, Arthur 15.05.1862 – 21.10.1931@\textsc{Schnitzler, Arthur} (15.05.1862 – 21.10.1931), \emph{Schriftsteller, Mediziner}|pwk}s Datierung – 23. –
                     weist auf einen Sonntag. Hier unter Annahme, dass er sich um einen Tag vertut,
                     auf 24. datiert.}}}\label{K_L01251_1h}\pend
           \pstart
           lieber, ich muſste Venedig\oindex{Venedig@\textbf{Venedig}|pw} wegen
               unerträglicher Kälte in unheizbarem Zimmer \label{K_L01251_2v}\edtext{aufgeben}{\lemma{\textnormal{\emph{aufgeben}}}\Cendnote{\textnormal{Er kam am
                     19. 11. 1902 retour.}}}\label{K_L01251_2h}. Ich bringe mit I und II Act\pwindex{Hofmannsthal, Hugo von 01.02.1874 – 15.07.1929@\textsc{Hofmannsthal, Hugo von} (01.02.1874 – 15.07.1929), \emph{Schriftsteller}!gerettete Venedig. Trauerspiel in fuenf Aufzuegen1905@\strich\emph{Das gerettete Venedig. Trauerspiel in fünf Aufzügen} {[}1905{]}|pwv} definitiv fertig (beide ſehr
               lang) vom III IV und V welche jeder ſehr kurz werden müſſen, bedeutende Theile ſchon
               ausgeführt; den Reſt könnte ich hoffen, hier in 3 Wochen zu machen. –\pend
           \pstart
           \label{K_L01251_3v}\edtext{Beiliegendes}{\lemma{\textnormal{\emph{Beiliegendes}}}\Cendnote{\textnormal{Das betreffende Korrespondenzstück, worin Hofmannsthal\pwindex{Hofmannsthal, Hugo von 01.02.1874 – 15.07.1929@\textsc{Hofmannsthal, Hugo von} (01.02.1874 – 15.07.1929), \emph{Schriftsteller}|pwk} zu einem Besuch in Rodaun\oindex{Rodaun@\textbf{Rodaun}|pwk} lädt, ist abgedruckt in: \emph{Hofmannsthal-Blätter}, H. 37/38 (1988).}}}\label{K_L01251_3h} an Hauptmann\pwindex{Hauptmann, Gerhart 15.11.1862 – 06.06.1946@\textsc{Hauptmann, Gerhart} (15.11.1862 – 06.06.1946), \emph{Schriftsteller}|pw} bitte ich Sie
               neu zu adreſſieren falls Sie eine beſſere Adreſſe wiſſen. Ich bitte ihn darin, mir zu
               ſagen {\pb}welchen Tag vor oder \uuline{nach} der \textsc{première} er hier
               heraußen mit Ihnen, Hans\pwindex{Schlesinger, Hans Bernhard 20.07.1875 – 13.3.1932@\textsc{Schlesinger, Hans Bernhard} (20.07.1875 – 13.3.1932), \emph{Maler}|pw} und uns beiden (ſonſt
               niemand, allenfalls die Gräfin Thun\pwindex{Thun-Hohenstein-Salm-Reifferscheidt, Christiane von 12.06.1859 – 06.08.1935@\textsc{Thun-Hohenstein-Salm-Reifferscheidt, Christiane von} (12.06.1859 – 06.08.1935), \emph{Schriftstellerin}|pw}, wenn ſie da
               iſt) eſſen will. Es möchte mir eine \uline{große} Freude
               machen, ich hoffe es geht zuſa{\geminationm}en. Vielleicht verabreden
               Sie ſich mit ihm zum Herausfahren, das wird es erleichtern.\pend
           \pstart
           Ein baldiges anderes Mal dann hoffe ich ſehr, Sie kommen nachmittag oder abends
               gemütlich mit Olga\pwindex{Schnitzler, Olga 17.01.1882 – 13.01.1970@\textsc{Schnitzler, Olga} (17.01.1882 – 13.01.1970), \emph{Schauspielerin, Sängerin}|pw}. Ich dürfte wegen Arbeit nicht
                  \uline{vor}{ }\label{K_L01251_4v}\edtext{\textsc{première}}{\lemma{\textnormal{\emph{première}}}\Cendnote{\textnormal{am 29. 11. 1902
                  Uraufführung von \emph{Der arme Heinrich – Eine deutsche
                     Sage}\pwindex{Hauptmann, Gerhart 15.11.1862 – 06.06.1946@\textsc{Hauptmann, Gerhart} (15.11.1862 – 06.06.1946), \emph{Schriftsteller}!arme Heinrich – Eine deutsche Sage29.11.1902 – 29.11.1902@\strich\emph{Der arme Heinrich – Eine deutsche Sage} {[}29.11.1902 – 29.11.1902{]}|pwk} am Burgtheater\oindex{Burgtheater@\textbf{Burgtheater}|pwk}}}}\label{K_L01251_4h} (armer Heinrich\pwindex{Hauptmann, Gerhart 15.11.1862 – 06.06.1946@\textsc{Hauptmann, Gerhart} (15.11.1862 – 06.06.1946), \emph{Schriftsteller}!arme Heinrich – Eine deutsche Sage29.11.1902 – 29.11.1902@\strich\emph{Der arme Heinrich – Eine deutsche Sage} {[}29.11.1902 – 29.11.1902{]}|pw}) nach Wien\oindex{Wien@\textbf{Wien}|pw} ko{\geminationm}en. Von Herzen \spacefill\mbox{Hugo.}\pend
           \endnumbering\briefempfaengerindex{Schnitzler, Arthur@\textsc{Schnitzler, Arthur}!zzzHofmannsthal, Hugo von@\emph{von Hugo von Hofmannsthal}!1902-11-241@{{[}24?. 11. 1902{]}}|)be}\mylabel{h}\end{ledgroupsized}  \newcommand{\dateiname}{L01251}\newcommand{\titel}{Hugo von Hofmannsthal an Arthur Schnitzler, [24?. 11. 1902]}\newcommand{\editorInnen}{Martin Anton Müller und Gerd-Hermann Susen}\input{../tex-inputs/latex-pdf-abspann}
      