%% latex-leseansicht-vorspann.tex
%% Vorspann für die Leseansicht.
%% Lädt die gemeinsame Datei latex-vorspann.tex mit nicht gesetztem Schalter.

\newif\ifkorrekturansicht
\korrekturansichtfalse

\input{../tex-inputs/latex-vorspann}


\section[Hugo von Hofmannsthal an Arthur Schnitzler, {[}24?. 11. 1902{]}]{L01251 Hugo von Hofmannsthal an Arthur Schnitzler, {[}24?. 11. 1902{]}}
\nopagebreak\mylabel{L01251v}
\rehead{ }\normalsize\beginnumbering\briefempfaengerindex{Schnitzler, Arthur@\textsc{Schnitzler, Arthur}!zzzHofmannsthal, Hugo von@\emph{von Hugo von Hofmannsthal}!1902-11-241@{{[}24?. 11. 1902{]}}|(be}
\toendnotes[C]{\smallbreak\pagebreak[2]}
\correspDesc{Versand  durch Hugo von Hofmannsthal am [24?. 11. 1902] in Rodaun
\newline{}Erhalt  durch Arthur Schnitzler im Zeitraum [25. 11. 1902 – 29. 11. 1902?] in Wien}\toendnotes[C]{\smallbreak}
\Standort{CUL, Schnitzler, B 43.}
\physDesc{Brief, 1 Blatt, 2 Seiten, 917 Zeichen
\newline{}Handschrift: schwarze Tinte, deutsche Kurrent
\newline{}Schnitzler: mit Bleistift datiert: »23/11 902« 
\newline{}Ordnung: mit Bleistift von unbekannter Hand nummeriert:
                                    »188« }
\buchAbdrucke{\weitereDrucke{Hugo von Hofmannsthal, Arthur Schnitzler: \emph{Briefwechsel}. Herausgegeben von Therese Nickl und Heinrich Schnitzler. Frankfurt am Main: \emph{S. Fischer} 1964, S. 163.} }\toendnotes[C]{\smallbreak}
\pstart
           \raggedleft{}{\pb}Rodaun\oindex{Wien@\textbf{Wien}!XXIII., Liesing@\textbf{XXIII., Liesing}!Rodaun@\textbf{Rodaun}, \emph{Region}|pw}{ }\label{K_L01251-1v}\edtext{Montg}{\lemma{\textnormal{\emph{Montg}}}\Cendnote{\textnormal{Schnitzlers Datierung – 23.
                     – weist auf einen Sonntag. Hier unter der Annahme, dass er sich um einen Tag
                     vertan hat, auf 24. datiert.}}}\label{K_L01251-1}\pend
           \vspace{0.5em}
\pstart
           lieber, ich muſste Venedig\oindex{Venedig@\textbf{Venedig}|pw} wegen
               unerträglicher Kälte in unheizbarem Zimmer \label{K_L01251-2v}\edtext{aufgeben}{\lemma{\textnormal{\emph{aufgeben}}}\Cendnote{\textnormal{Er kam am
                     19. 11. 1902 retour.}}}\label{K_L01251-2}. Ich bringe mit I und II Act\pwindex{Hofmannsthal, Hugo von 1.\,2.\,1874 Wien – 15.\,7.\,1929 Rodaun@\textsc{Hofmannsthal, Hugo von} (1.\,2.\,1874 Wien – 15.\,7.\,1929 Rodaun), \emph{Schriftsteller}!gerettete Venedig. Trauerspiel in fünf Aufzügen@\strich\emph{Das gerettete Venedig. Trauerspiel in fünf Aufzügen}|pwv} definitiv fertig (beide{ }ſehr lang) vom III IV und V welche jeder{ }ſehr kurz werden müſſen, bedeutende Theile{ }ſchon ausgeführt; den Reſt könnte ich hoffen, hier in 3 Wochen zu machen. –\pend
           
\pstart
           \label{K_L01251-3v}\edtext{Beiliegendes}{\lemma{\textnormal{\emph{Beiliegendes}}}\Cendnote{\textnormal{Das betreffende Korrespondenzstück, worin Hofmannsthal\pwindex{Hofmannsthal, Hugo von 1.\,2.\,1874 Wien – 15.\,7.\,1929 Rodaun@\textsc{Hofmannsthal, Hugo von} (1.\,2.\,1874 Wien – 15.\,7.\,1929 Rodaun), \emph{Schriftsteller}|pwk} zu einem Besuch in Rodaun\oindex{Wien@\textbf{Wien}!XXIII., Liesing@\textbf{XXIII., Liesing}!Rodaun@\textbf{Rodaun}, \emph{Region}|pwk} lädt, ist abgedruckt in: \emph{Hofmannsthal-Blätter}, H. 37/38, 1988.}}}\label{K_L01251-3} an Hauptmann\pwindex{Hauptmann, Gerhart 15.\,11.\,1862 Szczawno-Zdrój – 6.\,6.\,1946 Jagniątków@\textsc{Hauptmann, Gerhart} (15.\,11.\,1862 Szczawno-Zdrój – 6.\,6.\,1946 Jagniątków), \emph{Schriftsteller}|pw} bitte ich Sie
               neu zu adreſſieren falls Sie eine beſſere Adreſſe wiſſen. Ich bitte ihn darin, mir zu{ }ſagen {\pb}welchen Tag vor oder \uuline{nach} der \textsc{première} er hier
               heraußen mit Ihnen, Hans\pwindex{Schlesinger, Hans Bernhard 20.\,7.\,1875 Wien – 13.\,3.\,1932 Salzburg@\textsc{Schlesinger, Hans Bernhard} (20.\,7.\,1875 Wien – 13.\,3.\,1932 Salzburg), \emph{Maler}|pw} und uns beiden (ſonſt
               niemand, allenfalls die Gräfin Thun\pwindex{Thun-Hohenstein-Salm-Reifferscheidt, Christiane von 12.\,6.\,1859 Doksy – 6.\,8.\,1935 Prag@\textsc{Thun-Hohenstein-Salm-Reifferscheidt, Christiane von} (12.\,6.\,1859 Doksy – 6.\,8.\,1935 Prag), \emph{Schriftstellerin}|pw}, wenn{ }ſie
               da iſt) eſſen will. Es möchte mir eine \uline{große} Freude
               machen, ich hoffe es geht zuſa{\geminationm}en. Vielleicht verabreden
               Sie{ }ſich mit ihm zum Herausfahren, das wird es erleichtern.\pend
           
\pstart
           Ein baldiges anderes Mal dann hoffe ich{ }ſehr, Sie kommen nachmittag oder abends
               gemütlich mit Olga\pwindex{Schnitzler, Olga 17.\,1.\,1882 Wien – 13.\,1.\,1970 Lugano@\textsc{Schnitzler, Olga} (17.\,1.\,1882 Wien – 13.\,1.\,1970 Lugano), \emph{Schauspielerin, Sängerin}|pw}. Ich dürfte wegen Arbeit
               nicht \uline{vor}{ }\label{K_L01251-4v}\edtext{\textsc{première\eventindex{Burgtheater@\textbf{Burgtheater}!Uraufführung Der arme Heinrich, 29.11.1902@Uraufführung Der arme Heinrich, 29.11.1902|pw}}}{\lemma{\textnormal{\emph{première}}}\Cendnote{\textnormal{am 29. 11. 1902{ }Uraufführung\eventindex{Burgtheater@\textbf{Burgtheater}!Uraufführung Der arme Heinrich, 29.11.1902@Uraufführung Der arme Heinrich, 29.11.1902|pwkv} von \emph{Der arme Heinrich – Eine deutsche
                     Sage}\pwindex{Hauptmann, Gerhart 15.\,11.\,1862 Szczawno-Zdrój – 6.\,6.\,1946 Jagniątków@\textsc{Hauptmann, Gerhart} (15.\,11.\,1862 Szczawno-Zdrój – 6.\,6.\,1946 Jagniątków), \emph{Schriftsteller}!arme Heinrich – Eine deutsche Sage@\strich\emph{Der arme Heinrich – Eine deutsche Sage}|pwk} am \emph{Burgtheater}\orgindex{Burgtheater@Burgtheater|pwk}}}}\label{K_L01251-4} (armer Heinrich\pwindex{Hauptmann, Gerhart 15.\,11.\,1862 Szczawno-Zdrój – 6.\,6.\,1946 Jagniątków@\textsc{Hauptmann, Gerhart} (15.\,11.\,1862 Szczawno-Zdrój – 6.\,6.\,1946 Jagniątków), \emph{Schriftsteller}!arme Heinrich – Eine deutsche Sage@\strich\emph{Der arme Heinrich – Eine deutsche Sage}|pw}) nach Wien\oindex{Wien@\textbf{Wien}, \emph{Verwaltungsgebiet}|pw} ko{\geminationm}en. Von Herzen
                  \spacefill\mbox{Hugo.}\pend
           \selectlanguage{ngerman}\endnumbering\briefempfaengerindex{Schnitzler, Arthur@\textsc{Schnitzler, Arthur}!zzzHofmannsthal, Hugo von@\emph{von Hugo von Hofmannsthal}!1902-11-241@{{[}24?. 11. 1902{]}}|)be}\mylabel{L01251h}  \newcommand{\dateiname}{L01251}\newcommand{\titel}{Hugo von Hofmannsthal an Arthur Schnitzler, [24?. 11. 1902]}\newcommand{\editorInnen}{Martin Anton Müller und Gerd-Hermann Susen}%% latex-leseansicht-abspann.tex
%% Abspann für die Leseansicht.
%% Der Schalter \ifkorrekturansicht ist bereits durch den Vorspann gesetzt.

%% latex-abspann.tex
%% Gemeinsamer Abspann für Korrekturansicht und Leseansicht.
%% Setzt den Schalter \ifkorrekturansicht voraus (gesetzt in den
%% einbindenden Dateien latex-korrekturansicht-abspann.tex bzw.
%% latex-leseansicht-abspann.tex).
%% ---------------------------------------------------------------

\normalsize

% Das esempio-Environment wird nur in der Leseansicht benötigt
\ifkorrekturansicht\else
\newenvironment{esempio}[3]%
{
    \vspace{1.5ex}
    \rlap{\underline{#1}}
    \par
    \setlength{\parindent}{0cm}
    \nopagebreak
    \leftskip=#2cm
    \rightskip=#3cm
}
{
    \par
}
\fi

\doendnotes{C}
\bigskip
\vfill

\clearpage

\footnotesize

\ifkorrekturansicht
  \lohead{\textsc{register}}
\fi

% theindex-Environment neu definieren ohne reledmac
\makeatletter
\renewenvironment{theindex}{%
  \ifkorrekturansicht
    \section*{\indexname}%
  \else
    \subsubsection*{Index der erwähnten Entitäten}%
  \fi
  \setlength{\parindent}{0pt}%
  \setlength{\parskip}{0pt plus 0.3pt}%
  \let\item\@idxitem
}{%
  \ifkorrekturansicht\clearpage\fi
}
\makeatother

\IfFileExists{\jobname-pw.ind}{\input{\jobname-pw.ind}}{}

% Quellenangabe nur in der Leseansicht
\ifkorrekturansicht\else
% Fallback-Definitionen, falls die .tex-Datei \titel etc. nicht gesetzt hat
\providecommand{\titel}{}
\providecommand{\editorInnen}{}
\providecommand{\dateiname}{\jobname}

\vspace{3cm}

\vfill

\footnotesize
\textsc{Quelle}: \titel. Herausgegeben von {\editorInnen}. In: \emph{Arthur Schnitzler: Briefwechsel mit Autorinnen und Autoren}.
 Digitale Edition, https://schnitzler-briefe.acdh.oeaw.ac.at/{\dateiname}.html (Stand \today)
\fi

\end{document}


