%% latex-korrekturansicht-vorspann.tex
%% Vorspann für die Korrekturansicht.
%% Lädt die gemeinsame Datei latex-vorspann.tex mit gesetztem Schalter.

\newif\ifkorrekturansicht
\korrekturansichttrue

\input{../tex-inputs/latex-vorspann}


\section[Hugo von Hofmannsthal an Arthur Schnitzler, {[}24?. 11. 1902{]}]{L01251 Hugo von Hofmannsthal an Arthur Schnitzler, {[}24?. 11. 1902{]}}
\nopagebreak\mylabel{L01251v}
\rehead{ }\normalsize\beginnumbering\briefempfaengerindex{Schnitzler, Arthur@\textsc{Schnitzler, Arthur}!zzzHofmannsthal, Hugo von@\emph{von Hugo von Hofmannsthal}!1902-11-241@{{[}24?. 11. 1902{]}}|(be}
\toendnotes[C]{\smallbreak\pagebreak[2]}\Standort{CUL, Schnitzler, B 43.}
\physDesc{Brief, 1 Blatt, 2 Seiten, 917 Zeichen
\newline{}Handschrift: schwarze Tinte, deutsche Kurrent
\newline{}Schnitzler: mit Bleistift datiert: »23/11 902« 
\newline{}Ordnung: mit Bleistift von unbekannter Hand nummeriert:
                                    »188« }
\buchAbdrucke{\weitereDrucke{Hugo von Hofmannsthal, Arthur Schnitzler: \emph{Briefwechsel}. Frankfurt am Main: \emph{S. Fischer} 1964, S. 163.} }\toendnotes[C]{\smallbreak}
\pstart
           \raggedleft{}{\pb}Rodaun\oindex{Rodaun@\textbf{Rodaun}, \emph{A.ADM4}|pw}{ }\label{K_L01251-1v}\edtext{Montg}{\lemma{\textnormal{\emph{Montg}}}\Cendnote{\textnormal{Schnitzlers Datierung – 23.
                     – weist auf einen Sonntag. Hier unter der Annahme, dass er sich um einen Tag
                     vertan hat, auf 24. datiert.}}}\label{K_L01251-1}\pend
           \vspace{0.5em}
\pstart
           lieber, ich muſste Venedig\oindex{Venedig@\textbf{Venedig}, \emph{P.PPLA}|pw} wegen
               unerträglicher Kälte in unheizbarem Zimmer \label{K_L01251-2v}\edtext{aufgeben}{\lemma{\textnormal{\emph{aufgeben}}}\Cendnote{\textnormal{Er kam am
                     19. 11. 1902 retour.}}}\label{K_L01251-2}. Ich bringe mit I und II Act\pwindex{gerettete Venedig. Trauerspiel in fuenf Aufzuegen@\emph{Das gerettete Venedig. Trauerspiel in fünf Aufzügen}|pwv} definitiv fertig (beide
               ſehr lang) vom III IV und V welche jeder ſehr kurz werden müſſen, bedeutende Theile
               ſchon ausgeführt; den Reſt könnte ich hoffen, hier in 3 Wochen zu machen. –\pend
           
\pstart
           \label{K_L01251-3v}\edtext{Beiliegendes}{\lemma{\textnormal{\emph{Beiliegendes}}}\Cendnote{\textnormal{Das betreffende Korrespondenzstück, worin Hofmannsthal\pwindex{Hofmannsthal, Hugo von 1874-02-01 – 1929-07-15@\textsc{Hofmannsthal, Hugo von} (1874-02-01 – 1929-07-15), \emph{Schriftsteller/Schriftstellerin}|pwk} zu einem Besuch in Rodaun\oindex{Rodaun@\textbf{Rodaun}, \emph{A.ADM4}|pwk} lädt, ist abgedruckt in: \emph{Hofmannsthal-Blätter}, H. 37/38, 1988.}}}\label{K_L01251-3} an Hauptmann\pwindex{Hauptmann, Gerhart 15.11.1862 – 06.06.1946@\textsc{Hauptmann, Gerhart} (15.11.1862 – 06.06.1946), \emph{Schriftsteller/Schriftstellerin}|pw} bitte ich Sie
               neu zu adreſſieren falls Sie eine beſſere Adreſſe wiſſen. Ich bitte ihn darin, mir zu
               ſagen {\pb}welchen Tag vor oder \uuline{nach} der \textsc{première} er hier
               heraußen mit Ihnen, Hans\pwindex{Schlesinger, Hans Bernhard 20.07.1875 – 13.3.1932@\textsc{Schlesinger, Hans Bernhard} (20.07.1875 – 13.3.1932), \emph{Maler/Malerin}|pw} und uns beiden (ſonſt
               niemand, allenfalls die Gräfin Thun\pwindex{Thun-Hohenstein-Salm-Reifferscheidt, Christiane von 12.06.1859 – 06.08.1935@\textsc{Thun-Hohenstein-Salm-Reifferscheidt, Christiane von} (12.06.1859 – 06.08.1935), \emph{Schriftsteller/Schriftstellerin}|pw}, wenn ſie
               da iſt) eſſen will. Es möchte mir eine \uline{große} Freude
               machen, ich hoffe es geht zuſa{\geminationm}en. Vielleicht verabreden
               Sie ſich mit ihm zum Herausfahren, das wird es erleichtern.\pend
           
\pstart
           Ein baldiges anderes Mal dann hoffe ich ſehr, Sie kommen nachmittag oder abends
               gemütlich mit Olga\pwindex{Schnitzler, Olga 17.01.1882 – 13.01.1970@\textsc{Schnitzler, Olga} (17.01.1882 – 13.01.1970), \emph{Schauspieler/Schauspielerin, Sänger/Sängerin}|pw}. Ich dürfte wegen Arbeit
               nicht \uline{vor}{ }\label{K_L01251-4v}\edtext{\textsc{première\eventindex{Burgtheater@\textbf{Burgtheater}!Urauffuehrung Der arme Heinrich, 29.11.1902@Uraufführung Der arme Heinrich, 29.11.1902|pw}}}{\lemma{\textnormal{\emph{première}}}\Cendnote{\textnormal{am 29. 11. 1902{ }Uraufführung\eventindex{Burgtheater@\textbf{Burgtheater}!Urauffuehrung Der arme Heinrich, 29.11.1902@Uraufführung Der arme Heinrich, 29.11.1902|pwkv} von \emph{Der arme Heinrich – Eine deutsche
                     Sage}\pwindex{arme Heinrich – Eine deutsche Sage@\emph{Der arme Heinrich – Eine deutsche Sage}|pwk} am \emph{Burgtheater}\orgindex{Burgtheater@Burgtheater|pwk}}}}\label{K_L01251-4} (armer Heinrich\pwindex{arme Heinrich – Eine deutsche Sage@\emph{Der arme Heinrich – Eine deutsche Sage}|pw}) nach Wien\oindex{Wien@\textbf{Wien}, \emph{A.ADM2}|pw} ko{\geminationm}en. Von Herzen
                  \spacefill\mbox{Hugo.}\pend
           \selectlanguage{ngerman}\endnumbering\briefempfaengerindex{Schnitzler, Arthur@\textsc{Schnitzler, Arthur}!zzzHofmannsthal, Hugo von@\emph{von Hugo von Hofmannsthal}!1902-11-241@{{[}24?. 11. 1902{]}}|)be}\mylabel{L01251h}  \normalsize

\doendnotes{C}
\bigskip
\vfill

\clearpage

\footnotesize

\lohead{\textsc{register}}

% Definiere theindex-Environment komplett neu ohne reledmac
\makeatletter
\renewenvironment{theindex}{%
  \section*{\indexname}%
  \setlength{\parindent}{0pt}%
  \setlength{\parskip}{0pt plus 0.3pt}%
  \let\item\@idxitem
}{%
  \clearpage
}
\makeatother

\IfFileExists{\jobname-pw.ind}{\input{\jobname-pw.ind}}{}

\end{document}

      