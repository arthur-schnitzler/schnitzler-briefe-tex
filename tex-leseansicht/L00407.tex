%% latex-korrekturansicht-vorspann.tex
%% Vorspann für die Korrekturansicht.
%% Lädt die gemeinsame Datei latex-vorspann.tex mit gesetztem Schalter.

\newif\ifkorrekturansicht
\korrekturansichttrue

\input{../tex-inputs/latex-vorspann}


\section[Friedrich M. Fels an Arthur Schnitzler, {[}26. 11. 1894{]}]{L00407 Friedrich M. Fels an Arthur Schnitzler, {[}26. 11. 1894{]}}
\nopagebreak\mylabel{L00407v}
\rehead{ }\normalsize\beginnumbering\briefempfaengerindex{Schnitzler, Arthur@\textsc{Schnitzler, Arthur}!zzzFels, Friedrich Michael@\emph{von Friedrich Michael Fels}!1894-11-262@{{[}26. 11. 1894{]}}|(be}
\toendnotes[C]{\smallbreak\pagebreak[2]}\Standort{DLA, A:Schnitzler, HS.NZ85.1.2956.}
\physDesc{Brief, 1 Blatt, 1 Seite, 679 Zeichen
\newline{}Handschrift: schwarze Tinte, lateinische Kurrent
\newline{}Schnitzler: 1) mit Bleistift nummeriert: »20«  2) mit schwarzer Tinte datiert: »26. 11. 94« 3) mit rotem Buntstift eine Unterstreichung}\toendnotes[C]{\smallbreak}
\pstart{}{\pb}Lieber Dr. Schnitzler!\pend\vspace{0.5em}
\pstart
           Vielleicht hätten Sie die Freundlichkeit, möglichst bald \uline{Hugo Gerlach\pwindex{Gerlach, Hugo 26.10.1870 – 1930@\textsc{Gerlach, Hugo} (26.10.1870 – 1930), \emph{Schriftsteller/Schriftstellerin, Redakteur/Redakteurin}|pw}} zu besuchen. Er hat vielleicht die Diphteritis. Wohnung: XVIII (Währing), Sechsschi{\geminationm}elgaſse 4\oindex{Sechsschimmelgasse@\textbf{Sechsschimmelgasse}, \emph{Straße (K.STR)}|pw} II. Stock Thür 12. –\pend
           
\pstart
           Vielleicht sind \introOben{}Sie\introOben{} auch so gütig, mir \uline{1 fl} zu geben, den Sie bei Gerlach\pwindex{Gerlach, Hugo 26.10.1870 – 1930@\textsc{Gerlach, Hugo} (26.10.1870 – 1930), \emph{Schriftsteller/Schriftstellerin, Redakteur/Redakteurin}|pw} zurücklassen. Herzl. Dank. – Vom alten Mayer\pwindex{Mayer, Edmund 23.7.1842 – 15.4.1921@\textsc{Mayer, Edmund} (23.7.1842 – 15.4.1921), \emph{Chefredakteur/Chefredakteurin}|pw} hab ich keine Antwort. Die Kölnische Zeitung\orgindex{Koelnische Zeitung@Kölnische Zeitung|pw} hat meinen Artikel »\label{K_L00407-1v}\edtext{Skandinavien in Deutschland\pwindex{Skandinavien in Deutschland@\emph{Skandinavien in Deutschland}|pw}}{\lemma{\textnormal{\emph{Skandinavien in Deutschland}}}\Cendnote{\textnormal{Friedr. M. Fels\pwindex{Fels, Friedrich Michael *~1864@\textsc{Fels, Friedrich Michael} (*~1864), \emph{Journalist/Journalistin}|pwk}: \emph{Skandinavien in Deutschland}\pwindex{Skandinavien in Deutschland@\emph{Skandinavien in Deutschland}|pwk}. In: \emph{Kölnische Zeitung}\pwindex{Koelnische Zeitung@\emph{Kölnische Zeitung}|pwk}, 1. 1. 1895, Nr. 2,
                     Beilage zur Morgen-Ausgabe, S. [1–2].
               }}}\label{K_L00407-1}« acceptiert unter der Bedingung, daſs ich ihn um ⅓ kürze. Mein Roman wächst,
               blüht und gedeiht – ich habe früher nur den Ton nicht getroffen; jetzt nachdem ich
               der Kälte und Ironie den Abschied gegeben und \introOben{}auf\introOben{} harmlos
               humoristische Wirkung denke, gehts famos.\pend
           
\pstart
           Gruſs und Dank{\\[\baselineskip]}\spacefill\mbox{Fels}\pend
           \leftskip=0em{}\selectlanguage{ngerman}\endnumbering\briefempfaengerindex{Schnitzler, Arthur@\textsc{Schnitzler, Arthur}!zzzFels, Friedrich Michael@\emph{von Friedrich Michael Fels}!1894-11-262@{{[}26. 11. 1894{]}}|)be}\mylabel{L00407h}  \normalsize

\doendnotes{C}
\bigskip
\vfill

\clearpage

\footnotesize

\lohead{\textsc{register}}

% Definiere theindex-Environment komplett neu ohne reledmac
\makeatletter
\renewenvironment{theindex}{%
  \section*{\indexname}%
  \setlength{\parindent}{0pt}%
  \setlength{\parskip}{0pt plus 0.3pt}%
  \let\item\@idxitem
}{%
  \clearpage
}
\makeatother

\IfFileExists{\jobname-pw.ind}{\input{\jobname-pw.ind}}{}

\end{document}

      