%% latex-leseansicht-vorspann.tex
%% Vorspann für die Leseansicht.
%% Lädt die gemeinsame Datei latex-vorspann.tex mit nicht gesetztem Schalter.

\newif\ifkorrekturansicht
\korrekturansichtfalse

\input{../tex-inputs/latex-vorspann}


\section[Friedrich M. Fels an Arthur Schnitzler, {{[}}26. 11. 1894{{]}}]{L00407 Friedrich M. Fels an Arthur Schnitzler, {[}26. 11. 1894{]}}
\nopagebreak\mylabel{L00407v}
\rehead{ }\normalsize\beginnumbering\briefempfaengerindex{Schnitzler, Arthur@\textsc{Schnitzler, Arthur}!zzzFels, Friedrich Michael@\emph{von Friedrich Michael Fels}!1894-11-262@{{[}26. 11. 1894{]}}|(be}
\toendnotes[C]{\smallbreak\pagebreak[2]}
\correspDesc{Versand  durch Friedrich M. Fels am [26. 11. 1894] in Wien
\newline{}Erhalt  durch Arthur Schnitzler im Zeitraum [26. 11. 1894 – 30. 11. 1894?] in Wien}\toendnotes[C]{\smallbreak}
\Standort{DLA, A:Schnitzler, HS.NZ85.1.2956.}
\physDesc{Brief, 1 Blatt, 1 Seite, 679 Zeichen
\newline{}Handschrift: schwarze Tinte, lateinische Kurrent
\newline{}Schnitzler: 1) mit Bleistift nummeriert: »20«  2) mit schwarzer Tinte datiert: »26. 11. 94« 3) mit rotem Buntstift eine Unterstreichung}\toendnotes[C]{\smallbreak}
\pstart{}{\pb}Lieber Dr. Schnitzler!\pend\vspace{0.5em}
\pstart
           Vielleicht hätten Sie die Freundlichkeit, möglichst bald \uline{Hugo Gerlach\pwindex{Gerlach, Hugo 26.\,10.\,1870 Krosno Odrzańskie – 1930@\textsc{Gerlach, Hugo} (26.\,10.\,1870 Krosno Odrzańskie – 1930), \emph{Schriftsteller, Redakteur}|pw}} zu besuchen. Er hat vielleicht die Diphteritis. Wohnung: XVIII (Währing), Sechsschi{\geminationm}elgaſse 4\oindex{Wien@\textbf{Wien}!IX., Alsergrund@\textbf{IX., Alsergrund}!Sechsschimmelgasse@\textbf{Sechsschimmelgasse}, \emph{Straße}|pw} II. Stock Thür 12. –\pend
           
\pstart
           Vielleicht sind \introOben{}Sie\introOben{} auch so gütig, mir \uline{1 fl} zu geben, den Sie bei Gerlach\pwindex{Gerlach, Hugo 26.\,10.\,1870 Krosno Odrzańskie – 1930@\textsc{Gerlach, Hugo} (26.\,10.\,1870 Krosno Odrzańskie – 1930), \emph{Schriftsteller, Redakteur}|pw} zurücklassen. Herzl. Dank. – Vom alten Mayer\pwindex{Mayer, Edmund 23.\,7.\,1842 Wien – 15.\,4.\,1921 ebd.@\textsc{Mayer, Edmund} (23.\,7.\,1842 Wien – 15.\,4.\,1921 ebd.), \emph{Chefredakteur}|pw} hab ich keine Antwort. Die Kölnische Zeitung\orgindex{Kölnische Zeitung@Kölnische Zeitung|pw} hat meinen Artikel »\label{K_L00407-1v}\edtext{Skandinavien in Deutschland\pwindex{Fels, Friedrich Michael *~1864 Bad Dürkheim@\textsc{Fels, Friedrich Michael} (*~1864 Bad Dürkheim), \emph{Journalist}!Skandinavien in Deutschland@\strich\emph{Skandinavien in Deutschland}|pw}}{\lemma{\textnormal{\emph{Skandinavien in Deutschland}}}\Cendnote{\textnormal{Friedr. M. Fels\pwindex{Fels, Friedrich Michael *~1864 Bad Dürkheim@\textsc{Fels, Friedrich Michael} (*~1864 Bad Dürkheim), \emph{Journalist}|pwk}: \emph{Skandinavien in Deutschland}\pwindex{Fels, Friedrich Michael *~1864 Bad Dürkheim@\textsc{Fels, Friedrich Michael} (*~1864 Bad Dürkheim), \emph{Journalist}!Skandinavien in Deutschland@\strich\emph{Skandinavien in Deutschland}|pwk}. In: \emph{Kölnische Zeitung}\pwindex{Kölnische Zeitung@\emph{Kölnische Zeitung}|pwk}, 1. 1. 1895, Nr. 2,
                     Beilage zur Morgen-Ausgabe, S. [1–2].
               }}}\label{K_L00407-1}« acceptiert unter der Bedingung, daſs ich ihn um ⅓ kürze. Mein Roman wächst,
               blüht und gedeiht – ich habe früher nur den Ton nicht getroffen; jetzt nachdem ich
               der Kälte und Ironie den Abschied gegeben und \introOben{}auf\introOben{} harmlos
               humoristische Wirkung denke, gehts famos.\pend
           
\pstart
           Gruſs und Dank{\\[\baselineskip]}\spacefill\mbox{Fels}\pend
           \leftskip=0em{}\selectlanguage{ngerman}\endnumbering\briefempfaengerindex{Schnitzler, Arthur@\textsc{Schnitzler, Arthur}!zzzFels, Friedrich Michael@\emph{von Friedrich Michael Fels}!1894-11-262@{{[}26. 11. 1894{]}}|)be}\mylabel{L00407h}  \newcommand{\dateiname}{L00407}\newcommand{\titel}{Friedrich M. Fels an Arthur Schnitzler, [26. 11. 1894]}\newcommand{\editorInnen}{Martin Anton Müller und Gerd-Hermann Susen}%% latex-leseansicht-abspann.tex
%% Abspann für die Leseansicht.
%% Der Schalter \ifkorrekturansicht ist bereits durch den Vorspann gesetzt.

%% latex-abspann.tex
%% Gemeinsamer Abspann für Korrekturansicht und Leseansicht.
%% Setzt den Schalter \ifkorrekturansicht voraus (gesetzt in den
%% einbindenden Dateien latex-korrekturansicht-abspann.tex bzw.
%% latex-leseansicht-abspann.tex).
%% ---------------------------------------------------------------

\normalsize

% Das esempio-Environment wird nur in der Leseansicht benötigt
\ifkorrekturansicht\else
\newenvironment{esempio}[3]%
{
    \vspace{1.5ex}
    \rlap{\underline{#1}}
    \par
    \setlength{\parindent}{0cm}
    \nopagebreak
    \leftskip=#2cm
    \rightskip=#3cm
}
{
    \par
}
\fi

\doendnotes{C}
\bigskip
\vfill

\clearpage

\footnotesize

\ifkorrekturansicht
  \lohead{\textsc{register}}
\fi

% theindex-Environment neu definieren ohne reledmac
\makeatletter
\renewenvironment{theindex}{%
  \ifkorrekturansicht
    \section*{\indexname}%
  \else
    \subsubsection*{Index der erwähnten Entitäten}%
  \fi
  \setlength{\parindent}{0pt}%
  \setlength{\parskip}{0pt plus 0.3pt}%
  \let\item\@idxitem
}{%
  \ifkorrekturansicht\clearpage\fi
}
\makeatother

\IfFileExists{\jobname-pw.ind}{\input{\jobname-pw.ind}}{}

% Quellenangabe nur in der Leseansicht
\ifkorrekturansicht\else
% Fallback-Definitionen, falls die .tex-Datei \titel etc. nicht gesetzt hat
\providecommand{\titel}{}
\providecommand{\editorInnen}{}
\providecommand{\dateiname}{\jobname}

\vspace{3cm}

\vfill

\footnotesize
\textsc{Quelle}: \titel. Herausgegeben von {\editorInnen}. In: \emph{Arthur Schnitzler: Briefwechsel mit Autorinnen und Autoren}.
 Digitale Edition, https://schnitzler-briefe.acdh.oeaw.ac.at/{\dateiname}.html (Stand \today)
\fi

\end{document}


