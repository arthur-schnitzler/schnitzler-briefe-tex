%% latex-leseansicht-vorspann.tex
%% Vorspann für die Leseansicht.
%% Lädt die gemeinsame Datei latex-vorspann.tex mit nicht gesetztem Schalter.

\newif\ifkorrekturansicht
\korrekturansichtfalse

\input{../tex-inputs/latex-vorspann}


\section[Arthur Schnitzler an Richard Beer-Hofmann, 31. 1. 1896]{L00531 Arthur Schnitzler an Richard Beer-Hofmann, 31. 1. 1896}
\nopagebreak\mylabel{L00531v}
\rehead{ }\normalsize\beginnumbering\briefempfaengerindex{Beer-Hofmann, Richard@\textsc{Beer-Hofmann, Richard}!zzzSchnitzler, Arthur@\emph{von Arthur Schnitzler}!1896-01-311@{31. 1. 1896}|(be}
\toendnotes[C]{\smallbreak\pagebreak[2]}
\correspDesc{Versand  durch Arthur Schnitzler am 31. 1. 1896 in Berlin
\newline{}Erhalt  durch Richard Beer-Hofmann im Zeitraum [1. 2. 1896
                  – 5. 2. 1896?] in Wien}\toendnotes[C]{\smallbreak}
\Standort{YCGL, MSS 31.}
\physDesc{Brief, 2 Blätter, 7 Seiten, Kuvert, 2200 Zeichen
\newline{}Handschrift: 1) Bleistift, deutsche Kurrent\hspace{1em}2) schwarze Tinte, deutsche Kurrent (\noindent{}Umschlag)\hspace{1em}
\newline{}Versand: Stempel: »\nobreak{}\oindex{Berlin@\textbf{Berlin}, \emph{Hauptstadt}|pwk}Berlin W., 31 1 96, 9–10N\nobreak{}«.  }
\buchAbdrucke{\weitereDrucke{Arthur Schnitzler, Richard Beer-Hofmann: \emph{Briefwechsel 1891–1931}. Herausgegeben von Konstanze Fliedl. Wien, Zürich: \emph{Europaverlag} 1992, S. 89–90.} }\toendnotes[C]{\smallbreak}\pstart{}{\pb}\textsc{Dr. Arthur Schnitzler, Berlin\oindex{Berlin@\textbf{Berlin}, \emph{Hauptstadt}|pw}, Westminster Hotel\oindex{Hotel Westminster@\textbf{Hotel Westminster}, \emph{Hotel}|pw}}.\pend{}{\bigskip}\pstart{}{\pb}Herrn \textsc{Dr. Richard
                     Beer-Hofmann}\pend{}\pstart{}Wien\oindex{Wien@\textbf{Wien}, \emph{Verwaltungsgebiet}|pw}\pend{}\pstart{}\textsc{I. Wollzeile 15\oindex{Wien@\textbf{Wien}!I., Innere Stadt@\textbf{I., Innere Stadt}!Wollzeile 15 (»Berthahof«)@\textbf{Wollzeile 15 (»Berthahof«)}, \emph{Wohngebäude}|pw}}.\pend{}{\bigskip}\vspace{1em}
\pstart{}{\pb}Lieber Richard,\pend\vspace{0.5em}
\pstart
           Erſtens ist \textsc{Westminster Hotel}\oindex{Hotel Westminster@\textbf{Hotel Westminster}, \emph{Hotel}|pw} ein Protzenhotel, wie mir von den verſchiedenſten Seiten verſichert wird. Aber
               ich wohne doch dort. –\pend
           
\pstart
           Zweitens war{ }ſelbſtverſtändlich der erſte Menſch, dem ich begegnete, »College« Stümke\pwindex{Stümcke, Heinrich 7.\,5.\,1872 Jekaterinburg – 19.\,1.\,1923 Berlin@\textsc{Stümcke, Heinrich} (7.\,5.\,1872 Jekaterinburg – 19.\,1.\,1923 Berlin), \emph{Schriftsteller, Journalist}|pw}, der zur Zeit Berlin\oindex{Berlin@\textbf{Berlin}, \emph{Hauptstadt}|pw} vielfach anſpuckt und mehr Unſinn redet, als (über den
                  {\pb}Vergleich denk ich nächſtens nach). Er fragte
               gleich nach der \textsc{Brion\pwindex{Brion, Lou 17.\,12.\,1864 Besançon – 16.\,5.\,1942 Wien@\textsc{Brion, Lou} (17.\,12.\,1864 Besançon – 16.\,5.\,1942 Wien), \emph{Schauspielerin}|pw}}. Ein Herr \textsc{Ehrenzweig}\pwindex{Ehrenzweig @\textsc{Ehrenzweig}|pw}, den ich vorher ke{\geminationn}en gelernt hatte (folglich war
                  Stümke\pwindex{Stümcke, Heinrich 7.\,5.\,1872 Jekaterinburg – 19.\,1.\,1923 Berlin@\textsc{Stümcke, Heinrich} (7.\,5.\,1872 Jekaterinburg – 19.\,1.\,1923 Berlin), \emph{Schriftsteller, Journalist}|pw} nicht der erſte Menſch \introOben{}\textsc{etc}\introOben{}) und{ }ſich an meiner Seite befand, kannte die \textsc{Brion\pwindex{Brion, Lou 17.\,12.\,1864 Besançon – 16.\,5.\,1942 Wien@\textsc{Brion, Lou} (17.\,12.\,1864 Besançon – 16.\,5.\,1942 Wien), \emph{Schauspielerin}|pw}} natürlich auch. Ich ahnte fürchterliches. Aber wir{ }ſchweiften ab (Ich meine es
               nicht{ }ſo.)\pend
           
\pstart
           Geſtern war ich bei der Jüdin von Toledo\pwindex{\textcolor{red}{\textsuperscript{XXXX indx1}}!Jüdin von Toledo@\strich\emph{Die Jüdin von Toledo}|pw} und
               verliebte mich in {\pb}die Sorma\pwindex{Sorma, Agnes 17.\,5.\,1862 Breslau – 10.\,2.\,1927 Crown King@\textsc{Sorma, Agnes} (17.\,5.\,1862 Breslau – 10.\,2.\,1927 Crown King), \emph{Schauspielerin}|pw}; aber Kainz\pwindex{Kainz, Josef 2.\,1.\,1858 Mosonmagyaróvár – 20.\,9.\,1910 Wien@\textsc{Kainz, Josef} (2.\,1.\,1858 Mosonmagyaróvár – 20.\,9.\,1910 Wien), \emph{Schauspieler}|pw} war
               ebenſo herrlich.\pend
           
\pstart
           Mit Brahm\pwindex{Brahm, Otto 5.\,2.\,1856 Hamburg – 28.\,11.\,1912 Berlin@\textsc{Brahm, Otto} (5.\,2.\,1856 Hamburg – 28.\,11.\,1912 Berlin), \emph{Theaterleiter, Regisseur}|pw} hab ich mich{ }ſofort gezankt, er hat
               das Kind der \textsc{Katharina Binder}\pwindex{Schnitzler, Arthur 15.\,5.\,1862 Wien – 21.\,10.\,1931 ebd.@\textsc{Schnitzler, Arthur} (15.\,5.\,1862 Wien – 21.\,10.\,1931 ebd.), \emph{Schriftsteller, Mediziner}!Liebelei. Schauspiel in drei Akten@\strich\emph{Liebelei. Schauspiel in drei Akten}|pwv} gemordet – angeblich aus künſtleriſchen Gründen. Als ich dieſelben wi\strikeout{e}derlegte,{ }ſtellte{ }ſich heraus, daſs er überhaupt kein
               Kind zur Verfügung hatte. Ein paar Striche, die ganz überflüſſiger Weiſe geſchehn
               waren, machte ich wieder auf.\pend
           
\pstart
           {\pb}Heute war Probe. Ich unterhielt mich{ }ſehr gut. Sie
               wollen mehr wiſſen? Gelegentlich.\pend
           
\pstart
           Stümke\pwindex{Stümcke, Heinrich 7.\,5.\,1872 Jekaterinburg – 19.\,1.\,1923 Berlin@\textsc{Stümcke, Heinrich} (7.\,5.\,1872 Jekaterinburg – 19.\,1.\,1923 Berlin), \emph{Schriftsteller, Journalist}|pw} möchte nicht in meiner Haut{ }ſtecken
               (Gegenſeitig!) Nemlich weil die Sti{\geminationm}ung gegen Brahm\pwindex{Brahm, Otto 5.\,2.\,1856 Hamburg – 28.\,11.\,1912 Berlin@\textsc{Brahm, Otto} (5.\,2.\,1856 Hamburg – 28.\,11.\,1912 Berlin), \emph{Theaterleiter, Regisseur}|pw}{ }ſehr heftig iſt und bei den \textsc{Premièren} »jedenfalls« auf Hausſchlüſſeln gepfiffen wird. Ich ka{\geminationn} natürlich kein Auge zuthun. »Gehn S’,{ }ſein S’ feſch,
                  {\pb}und ko{\geminationm}en S’ her!«
               Glauben Sie, daſs Librettiſten auf Nachſchlüſſeln pfeifen? (Herrn \textsc{Julius Bauer}\pwindex{Bauer, Julius 15.\,10.\,1853 Szigetvár – 11.\,6.\,1941 Wien@\textsc{Bauer, Julius} (15.\,10.\,1853 Szigetvár – 11.\,6.\,1941 Wien), \emph{Schriftsteller, Journalist, Kritiker}|pw} wohlgeboren)\pend
           
\pstart
           – Wohin war mein erſter Gang? Zu dem Hauſe, das \textsc{ICH} vor
               8 Jahren bewohnt hatte. Jedes Poëtchen hat sein Pietätchen.\pend
           
\pstart
           Schneit es in Wien\oindex{Wien@\textbf{Wien}, \emph{Verwaltungsgebiet}|pw} noch{ }ſo vehement, und wie geht
               es Paula\pwindex{Beer-Hofmann, Paula 25.\,2.\,1879 Wien – 30.\,10.\,1939 Zürich@\textsc{Beer-Hofmann, Paula} (25.\,2.\,1879 Wien – 30.\,10.\,1939 Zürich)|pw}? (\introOben{}Ja we{\geminationn} Sie wüßten was ich urſprünglich in diese Kla{\geminationm}er{ }ſchreiben wollte!\introOben{})\pend
           
\pstart
           {\pb}\textsc{Jarno}\pwindex{Jarno, Josef 24.\,8.\,1865 Budapest – 11.\,1.\,1932 Wien@\textsc{Jarno, Josef} (24.\,8.\,1865 Budapest – 11.\,1.\,1932 Wien), \emph{Theaterleiter, Schauspieler}|pw} läßt Sie grüßen; Sie waren{ }ſeine erſte Frage. Die Staglé\pwindex{Staglé, Helene @\textsc{Staglé, Helene}, \emph{Schauspielerin}|pw} ist engagirt,{ }ſpielt im »zerbrochnen Krug\pwindex{\textcolor{red}{\textsuperscript{XXXX indx1}}!zerbrochene Krug. Ein Lustspiel in drei Aufzügen@\strich\emph{Der zerbrochene Krug. Ein Lustspiel in drei Aufzügen}|pw}« mit, der zur Liebelei\pwindex{Schnitzler, Arthur 15.\,5.\,1862 Wien – 21.\,10.\,1931 ebd.@\textsc{Schnitzler, Arthur} (15.\,5.\,1862 Wien – 21.\,10.\,1931 ebd.), \emph{Schriftsteller, Mediziner}!Liebelei. Schauspiel in drei Akten@\strich\emph{Liebelei. Schauspiel in drei Akten}|pw}
               dazu gegeben wird.\pend
           
\pstart
           – Jetzt kleid ich mich um, gehe zum \textsc{König Chilperich}\pwindex{\textcolor{red}{\textsuperscript{XXXX indx1}}!König Chilperich@\strich\emph{König Chilperich}|pw}. Da{\geminationn} bin ich eingeladen\pwindex{Burchardt, Augusta *~4.\,7.\,1855 Triest@\textsc{Burchardt, Augusta} (*~4.\,7.\,1855 Triest)|pwv}. \label{K_L00531-1v}\edtext{\textsc{Si vous croyez, que c’est rigolo}!}{\lemma{\textnormal{\emph{Si … rigolo!}}}\Cendnote{\textnormal{französisch: Glauben Sie ja nicht, dass das unterhaltsam
                  ist!}}}\label{K_L00531-1} – Womöglich als Zitat entnommen aus: Gyp\pwindex{Mirabeau, Sibylle de 15.\,8.\,1850 Plumergat – 30.\,6.\,1932 Neuilly-sur-Seine@\textsc{Mirabeau, Sibylle de} (15.\,8.\,1850 Plumergat – 30.\,6.\,1932 Neuilly-sur-Seine), \emph{Schriftstellerin}|pwk}: \emph{Le
                     Mariage de Chiffon}\pwindex{Mirabeau, Sibylle de 15.\,8.\,1850 Plumergat – 30.\,6.\,1932 Neuilly-sur-Seine@\textsc{Mirabeau, Sibylle de} (15.\,8.\,1850 Plumergat – 30.\,6.\,1932 Neuilly-sur-Seine), \emph{Schriftstellerin}!Ehe von Chiffon@\strich\emph{Die Ehe von Chiffon}|pwk}. Paris: \emph{Calmann-Lévy}\orgindex{Calmann-Lévy@Calmann-Lévy|pwk}{ }1894, S. 47.\pend
           
\pstart
           Grüßen Sie Salten\pwindex{Salten, Felix 6.\,9.\,1869 Budapest – 8.\,10.\,1945 Zürich@\textsc{Salten, Felix} (6.\,9.\,1869 Budapest – 8.\,10.\,1945 Zürich), \emph{Schriftsteller, Journalist, Chefredakteur}|pw}, Hugo\pwindex{Hofmannsthal, Hugo von 1.\,2.\,1874 Wien – 15.\,7.\,1929 Rodaun@\textsc{Hofmannsthal, Hugo von} (1.\,2.\,1874 Wien – 15.\,7.\,1929 Rodaun), \emph{Schriftsteller}|pw} un\textcolor{gray}{d} manche andre. Schreiben {\pb}Sie mir.\pend
           
\pstart
           Herzlich der Ihre{\\[\baselineskip]}\spacefill\mbox{Arth}\pend
           \leftskip=0em{}\selectlanguage{ngerman}\endnumbering\briefempfaengerindex{Beer-Hofmann, Richard@\textsc{Beer-Hofmann, Richard}!zzzSchnitzler, Arthur@\emph{von Arthur Schnitzler}!1896-01-311@{31. 1. 1896}|)be}\mylabel{L00531h}  \newcommand{\dateiname}{L00531}\newcommand{\titel}{Arthur Schnitzler an Richard Beer-Hofmann, 31. 1. 1896}\newcommand{\editorInnen}{Martin Anton Müller und Gerd-Hermann Susen}%% latex-leseansicht-abspann.tex
%% Abspann für die Leseansicht.
%% Der Schalter \ifkorrekturansicht ist bereits durch den Vorspann gesetzt.

%% latex-abspann.tex
%% Gemeinsamer Abspann für Korrekturansicht und Leseansicht.
%% Setzt den Schalter \ifkorrekturansicht voraus (gesetzt in den
%% einbindenden Dateien latex-korrekturansicht-abspann.tex bzw.
%% latex-leseansicht-abspann.tex).
%% ---------------------------------------------------------------

\normalsize

% Das esempio-Environment wird nur in der Leseansicht benötigt
\ifkorrekturansicht\else
\newenvironment{esempio}[3]%
{
    \vspace{1.5ex}
    \rlap{\underline{#1}}
    \par
    \setlength{\parindent}{0cm}
    \nopagebreak
    \leftskip=#2cm
    \rightskip=#3cm
}
{
    \par
}
\fi

\doendnotes{C}
\bigskip
\vfill

\clearpage

\footnotesize

\ifkorrekturansicht
  \lohead{\textsc{register}}
\fi

% theindex-Environment neu definieren ohne reledmac
\makeatletter
\renewenvironment{theindex}{%
  \ifkorrekturansicht
    \section*{\indexname}%
  \else
    \subsubsection*{Index der erwähnten Entitäten}%
  \fi
  \setlength{\parindent}{0pt}%
  \setlength{\parskip}{0pt plus 0.3pt}%
  \let\item\@idxitem
}{%
  \ifkorrekturansicht\clearpage\fi
}
\makeatother

\IfFileExists{\jobname-pw.ind}{\input{\jobname-pw.ind}}{}

% Quellenangabe nur in der Leseansicht
\ifkorrekturansicht\else
% Fallback-Definitionen, falls die .tex-Datei \titel etc. nicht gesetzt hat
\providecommand{\titel}{}
\providecommand{\editorInnen}{}
\providecommand{\dateiname}{\jobname}

\vspace{3cm}

\vfill

\footnotesize
\textsc{Quelle}: \titel. Herausgegeben von {\editorInnen}. In: \emph{Arthur Schnitzler: Briefwechsel mit Autorinnen und Autoren}.
 Digitale Edition, https://schnitzler-briefe.acdh.oeaw.ac.at/{\dateiname}.html (Stand \today)
\fi

\end{document}


