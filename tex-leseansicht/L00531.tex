%% latex-korrekturansicht-vorspann.tex
%% Vorspann für die Korrekturansicht.
%% Lädt die gemeinsame Datei latex-vorspann.tex mit gesetztem Schalter.

\newif\ifkorrekturansicht
\korrekturansichttrue

\input{../tex-inputs/latex-vorspann}


\section[Arthur Schnitzler an Richard Beer-Hofmann, 31. 1. 1896]{L00531 Arthur Schnitzler an Richard Beer-Hofmann, 31. 1. 1896}
\nopagebreak\mylabel{L00531v}
\rehead{ }\normalsize\beginnumbering\briefempfaengerindex{Beer-Hofmann, Richard@\textsc{Beer-Hofmann, Richard}!zzzSchnitzler, Arthur@\emph{von Arthur Schnitzler}!1896-01-311@{31. 1. 1896}|(be}
\toendnotes[C]{\smallbreak\pagebreak[2]}\Standort{YCGL, MSS 31.}
\physDesc{Brief, 2 Blätter, 7 Seiten, Umschlag, 2200 Zeichen
\newline{}Handschrift: 1) Bleistift, deutsche Kurrent\hspace{1em}2) schwarze Tinte, deutsche Kurrent (\noindent{}Umschlag)\hspace{1em}
\newline{}Versand: Stempel: »\nobreak{}\oindex{Berlin@\textbf{Berlin}, \emph{P.PPLC}|pwk}Berlin W., 31 1 96, 9–10N\nobreak{}«.  }
\buchAbdrucke{\weitereDrucke{Arthur Schnitzler, Richard Beer-Hofmann: \emph{Briefwechsel 1891–1931}. Wien, Zürich: \emph{Europaverlag} 1992, S. 89–90.} }\toendnotes[C]{\smallbreak}\pstart{}{\pb}\textsc{Dr. Arthur Schnitzler, Berlin\oindex{Berlin@\textbf{Berlin}, \emph{P.PPLC}|pw}, Westminster Hotel\oindex{Hotel Westminster@\textbf{Hotel Westminster}, \emph{Hotel (K.HTL)}|pw}}.\pend{}{\bigskip}\pstart{}{\pb}Herrn \textsc{Dr. Richard
                     Beer-Hofmann}\pend{}\pstart{}Wien\oindex{Wien@\textbf{Wien}, \emph{A.ADM2}|pw}\pend{}\pstart{}\textsc{I. Wollzeile 15\oindex{Wollzeile@\textbf{Wollzeile}, \emph{Straße (K.STR)}|pw}}.\pend{}{\bigskip}\vspace{1em}
\pstart{}{\pb}Lieber Richard,\pend\vspace{0.5em}
\pstart
           Erſtens ist \textsc{Westminster Hotel}\oindex{Hotel Westminster@\textbf{Hotel Westminster}, \emph{Hotel (K.HTL)}|pw} ein Protzenhotel, wie mir von den verſchiedenſten Seiten verſichert wird. Aber
               ich wohne doch dort. –\pend
           
\pstart
           Zweitens war ſelbſtverſtändlich der erſte Menſch, dem ich begegnete, »College« Stümke\pwindex{Stuemcke, Heinrich 07.05.1872 – 19.01.1923@\textsc{Stümcke, Heinrich} (07.05.1872 – 19.01.1923), \emph{Schriftsteller/Schriftstellerin, Journalist/Journalistin}|pw}, der zur Zeit Berlin\oindex{Berlin@\textbf{Berlin}, \emph{P.PPLC}|pw} vielfach anſpuckt und mehr Unſinn redet, als (über den
                  {\pb}Vergleich denk ich nächſtens nach). Er fragte
               gleich nach der \textsc{Brion\pwindex{Brion, Lou 17.12.1864 – 16.05.1942@\textsc{Brion, Lou} (17.12.1864 – 16.05.1942), \emph{Schauspieler/Schauspielerin}|pw}}. Ein Herr \textsc{Ehrenzweig}\pwindex{Ehrenzweig @\textsc{Ehrenzweig}|pw}, den ich vorher ke{\geminationn}en gelernt hatte (folglich war
                  Stümke\pwindex{Stuemcke, Heinrich 07.05.1872 – 19.01.1923@\textsc{Stümcke, Heinrich} (07.05.1872 – 19.01.1923), \emph{Schriftsteller/Schriftstellerin, Journalist/Journalistin}|pw} nicht der erſte Menſch \introOben{}\textsc{etc}\introOben{}) und ſich an meiner Seite befand, kannte die \textsc{Brion\pwindex{Brion, Lou 17.12.1864 – 16.05.1942@\textsc{Brion, Lou} (17.12.1864 – 16.05.1942), \emph{Schauspieler/Schauspielerin}|pw}} natürlich auch. Ich ahnte fürchterliches. Aber wir ſchweiften ab (Ich meine es
               nicht ſo.)\pend
           
\pstart
           Geſtern war ich bei der Jüdin von Toledo\pwindex{Juedin von Toledo@\emph{Die Jüdin von Toledo}|pw} und
               verliebte mich in {\pb}die Sorma\pwindex{Sorma, Agnes 17.05.1862 – 10.02.1927@\textsc{Sorma, Agnes} (17.05.1862 – 10.02.1927), \emph{Schauspieler/Schauspielerin}|pw}; aber Kainz\pwindex{Kainz, Josef 02.01.1858 – 20.09.1910@\textsc{Kainz, Josef} (02.01.1858 – 20.09.1910), \emph{Schauspieler/Schauspielerin}|pw} war
               ebenſo herrlich.\pend
           
\pstart
           Mit Brahm\pwindex{Brahm, Otto 05.02.1856 – 28.11.1912@\textsc{Brahm, Otto} (05.02.1856 – 28.11.1912), \emph{Theaterleiter/Theaterleiterin, Regisseur/Regisseurin}|pw} hab ich mich ſofort gezankt, er hat
               das Kind der \textsc{Katharina Binder}\pwindex{Liebelei. Schauspiel in drei Akten@\emph{Liebelei. Schauspiel in drei Akten}|pwv} gemordet – angeblich aus künſtleriſchen Gründen. Als ich dieſelben wi\strikeout{e}derlegte, ſtellte ſich heraus, daſs er überhaupt kein
               Kind zur Verfügung hatte. Ein paar Striche, die ganz überflüſſiger Weiſe geſchehn
               waren, machte ich wieder auf.\pend
           
\pstart
           {\pb}Heute war Probe. Ich unterhielt mich ſehr gut. Sie
               wollen mehr wiſſen? Gelegentlich.\pend
           
\pstart
           Stümke\pwindex{Stuemcke, Heinrich 07.05.1872 – 19.01.1923@\textsc{Stümcke, Heinrich} (07.05.1872 – 19.01.1923), \emph{Schriftsteller/Schriftstellerin, Journalist/Journalistin}|pw} möchte nicht in meiner Haut ſtecken
               (Gegenſeitig!) Nemlich weil die Sti{\geminationm}ung gegen Brahm\pwindex{Brahm, Otto 05.02.1856 – 28.11.1912@\textsc{Brahm, Otto} (05.02.1856 – 28.11.1912), \emph{Theaterleiter/Theaterleiterin, Regisseur/Regisseurin}|pw}{ }ſehr heftig iſt und bei den \textsc{Premièren} »jedenfalls« auf Hausſchlüſſeln gepfiffen wird. Ich ka{\geminationn} natürlich kein Auge zuthun. »Gehn S’, ſein S’ feſch,
                  {\pb}und ko{\geminationm}en S’ her!«
               Glauben Sie, daſs Librettiſten auf Nachſchlüſſeln pfeifen? (Herrn \textsc{Julius Bauer}\pwindex{Bauer, Julius 15.10.1853 – 11.06.1941@\textsc{Bauer, Julius} (15.10.1853 – 11.06.1941), \emph{Schriftsteller/Schriftstellerin, Journalist/Journalistin, Kritiker/Kritikerin}|pw} wohlgeboren)\pend
           
\pstart
           – Wohin war mein erſter Gang? Zu dem Hauſe, das \textsc{ICH} vor
               8 Jahren bewohnt hatte. Jedes Poëtchen hat sein Pietätchen.\pend
           
\pstart
           Schneit es in Wien\oindex{Wien@\textbf{Wien}, \emph{A.ADM2}|pw} noch ſo vehement, und wie geht
               es Paula\pwindex{Beer-Hofmann, Paula 25.02.1879 – 30.10.1939@\textsc{Beer-Hofmann, Paula} (25.02.1879 – 30.10.1939)|pw}? (\introOben{}Ja we{\geminationn} Sie wüßten was ich urſprünglich in diese Kla{\geminationm}er ſchreiben wollte!\introOben{})\pend
           
\pstart
           {\pb}\textsc{Jarno}\pwindex{Jarno, Josef 24.08.1865 – 11.01.1932@\textsc{Jarno, Josef} (24.08.1865 – 11.01.1932), \emph{Theaterleiter/Theaterleiterin, Schauspieler/Schauspielerin}|pw} läßt Sie grüßen; Sie waren ſeine erſte Frage. Die Staglé\pwindex{Stagle, Helene @\textsc{Staglé, Helene}, \emph{Schauspieler/Schauspielerin}|pw} ist engagirt, ſpielt im »zerbrochnen Krug\pwindex{zerbrochene Krug. Ein Lustspiel in drei Aufzuegen@\emph{Der zerbrochene Krug. Ein Lustspiel in drei Aufzügen}|pw}« mit, der zur Liebelei\pwindex{Liebelei. Schauspiel in drei Akten@\emph{Liebelei. Schauspiel in drei Akten}|pw}
               dazu gegeben wird.\pend
           
\pstart
           – Jetzt kleid ich mich um, gehe zum \textsc{König Chilperich}\pwindex{Koenig Chilperich@\emph{König Chilperich}|pw}. Da{\geminationn} bin ich eingeladen\pwindex{Burchardt, Augusta *~4.7.1855@\textsc{Burchardt, Augusta} (*~4.7.1855)|pwv}. \label{K_L00531-1v}\edtext{\textsc{Si vous croyez, que c’est rigolo}!}{\lemma{\textnormal{\emph{Si … rigolo!}}}\Cendnote{\textnormal{französisch: Glauben Sie ja nicht, dass das unterhaltsam
                  ist!}}}\label{K_L00531-1} – Womöglich als Zitat entnommen aus: Gyp\pwindex{Mirabeau, Sibylle de 15.08.1850 – 30.06.1932@\textsc{Mirabeau, Sibylle de} (15.08.1850 – 30.06.1932), \emph{Schriftsteller/Schriftstellerin}|pwk}: \emph{Le
                     Mariage de Chiffon}\pwindex{Ehe von Chiffon@\emph{Die Ehe von Chiffon}|pwk}. Paris: \emph{Calmann-Lévy}\orgindex{Calmann-Levy@Calmann-Lévy|pwk}{ }1894, S. 47.\pend
           
\pstart
           Grüßen Sie Salten\pwindex{Salten, Felix 06.09.1869 – 08.10.1945@\textsc{Salten, Felix} (06.09.1869 – 08.10.1945), \emph{Schriftsteller/Schriftstellerin, Journalist/Journalistin, Chefredakteur/Chefredakteurin}|pw}, Hugo\pwindex{Hofmannsthal, Hugo von 1874-02-01 – 1929-07-15@\textsc{Hofmannsthal, Hugo von} (1874-02-01 – 1929-07-15), \emph{Schriftsteller/Schriftstellerin}|pw} un\textcolor{gray}{d} manche andre. Schreiben {\pb}Sie mir.\pend
           
\pstart
           Herzlich der Ihre{\\[\baselineskip]}\spacefill\mbox{Arth}\pend
           \leftskip=0em{}\selectlanguage{ngerman}\endnumbering\briefempfaengerindex{Beer-Hofmann, Richard@\textsc{Beer-Hofmann, Richard}!zzzSchnitzler, Arthur@\emph{von Arthur Schnitzler}!1896-01-311@{31. 1. 1896}|)be}\mylabel{L00531h}  \normalsize

\doendnotes{C}
\bigskip
\vfill

\clearpage

\footnotesize

\lohead{\textsc{register}}

% Definiere theindex-Environment komplett neu ohne reledmac
\makeatletter
\renewenvironment{theindex}{%
  \section*{\indexname}%
  \setlength{\parindent}{0pt}%
  \setlength{\parskip}{0pt plus 0.3pt}%
  \let\item\@idxitem
}{%
  \clearpage
}
\makeatother

\IfFileExists{\jobname-pw.ind}{\input{\jobname-pw.ind}}{}

\end{document}

      