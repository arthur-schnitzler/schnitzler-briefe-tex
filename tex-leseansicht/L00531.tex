%% latex-leseansicht-vorspann.tex
%% Vorspann für die Leseansicht.
%% Lädt die gemeinsame Datei latex-vorspann.tex mit nicht gesetztem Schalter.

\newif\ifkorrekturansicht
\korrekturansichtfalse

\input{../tex-inputs/latex-vorspann}


         
         \renewcommand{\erwaehntePersonen}{Personen: Julius Bauer, Richard Beer-Hofmann, Paula Beer-Hofmann, Otto Brahm, Lou Brion, Augusta Burchardt,  Ehrenzweig, Hugo von Hofmannsthal, Josef Jarno, Josef Kainz, Sibylle de Mirabeau, Felix Salten, Agnes Sorma, Helene Staglé, Heinrich Stümcke}
         \renewcommand{\erwaehnteInstitutionen}{Institutionen: Calmann-Lévy}
         \renewcommand{\erwaehnteOrte}{Orte: Berlin, Hotel Westminster, Wien, Wollzeile}
         \renewcommand{\erwaehnteWerke}{Werke: Der zerbrochene Krug, Die Ehe von Chiffon, Die Jüdin von Toledo, König Chilperich, Liebelei. Schauspiel in drei Akten}
               \section[Arthur Schnitzler an Richard Beer-Hofmann, 31. 1. 1896]{ Arthur Schnitzler an Richard Beer-Hofmann,
               31. 1. 1896}\nopagebreak\mylabel{v}\rehead{ }\begin{ledgroupsized}[t]{13cm}\normalsize\beginnumbering \toendnotes[C]{\smallbreak\pagebreak[2]} \Standort{YCGL, MSS 31.}
\physDesc{Brief, 2 Blätter, 7 Seiten, Umschlag
\newline{}Handschrift: 1) Bleistift, deutsche Kurrent\hspace{1em}2) schwarze Tinte, deutsche Kurrent (\noindent{}Umschlag)\hspace{1em}\newline{}Versand: Stempel: »\nobreak{}\oindex{Berlin@\textbf{Berlin}|pwk}Berlin W., 31 1 96, 9–10N\nobreak{}«.  }\buchAbdrucke{\weitereDrucke{Arthur Schnitzler, Richard Beer-Hofmann: \emph{Briefwechsel 1891–1931}. Hg. Konstanze Fliedl. Wien, Zürich: \emph{Europaverlag} 1992, S. 89–90.} }\toendnotes[C]{\smallbreak}\pstart{}{\pb}\textsc{Dr. Arthur Schnitzler, Berlin\oindex{Berlin@\textbf{Berlin}|pw}, Westminster Hotel\oindex{Hotel Westminster@\textbf{Hotel Westminster}|pw}}.\pend{}{\bigskip}\pstart{}{\pb}Herrn \textsc{Dr. Richard
                     Beer-Hofmann}\pend{}\pstart{}Wien\oindex{Wien@\textbf{Wien}|pw}\pend{}\pstart{}\textsc{I. Wollzeile 15\oindex{Wollzeile@\textbf{Wollzeile}|pw}}.\pend{}{\bigskip}\pstart{}{\pb}Lieber Richard,\pend\pstart
           Erſtens ist \textsc{Westminster Hotel}\oindex{Hotel Westminster@\textbf{Hotel Westminster}|pw} ein Protzenhotel, wie mir von den verſchiedenſten Seiten verſichert wird. Aber
               ich wohne doch dort. –\pend
           \pstart
           Zweitens war ſelbſtverſtändlich der erſte Menſch, dem ich begegnete, »College« Stümke\pwindex{Stuemcke, Heinrich 07.05.1872 – 19.01.1923@\textsc{Stümcke, Heinrich} (07.05.1872 – 19.01.1923), \emph{Schriftsteller, Journalist}|pw}, der zur Zeit Berlin\oindex{Berlin@\textbf{Berlin}|pw} vielfach anſpuckt und mehr Unſinn redet, als (über den {\pb}Vergleich denk ich nächſtens nach). Er fragte
               gleich nach der \textsc{Brion\pwindex{Brion, Lou 17.12.1864 – 16.05.1942@\textsc{Brion, Lou} (17.12.1864 – 16.05.1942), \emph{Schauspielerin}|pw}}.
               Ein Herr \textsc{Ehrenzweig}\pwindex{Ehrenzweig @\textsc{Ehrenzweig}|pw}, den ich vorher ke{\geminationn}en gelernt hatte (folglich war
                  Stümke\pwindex{Stuemcke, Heinrich 07.05.1872 – 19.01.1923@\textsc{Stümcke, Heinrich} (07.05.1872 – 19.01.1923), \emph{Schriftsteller, Journalist}|pw} nicht der erſte Menſch \introOben{}\textsc{etc}\introOben{}) und ſich an meiner Seite
               befand, kannte die \textsc{Brion\pwindex{Brion, Lou 17.12.1864 – 16.05.1942@\textsc{Brion, Lou} (17.12.1864 – 16.05.1942), \emph{Schauspielerin}|pw}}
               natürlich auch. Ich ahnte fürchterliches. Aber wir ſchweiften ab (Ich meine es nicht
               ſo.)\pend
           \pstart
           Geſtern war ich bei der Jüdin von Toledo\pwindex{\textcolor{red}{\textsuperscript{XXXX1 indx}}!Juedin von Toledo22. 11. 1872@\strich\emph{Die Jüdin von Toledo} {[}22. 11. 1872{]}|pw} und
               verliebte mich in {\pb}die Sorma\pwindex{Sorma, Agnes 17.05.1862 – 10.02.1927@\textsc{Sorma, Agnes} (17.05.1862 – 10.02.1927), \emph{Schauspielerin}|pw}; aber Kainz\pwindex{Kainz, Josef 02.01.1858 – 20.09.1910@\textsc{Kainz, Josef} (02.01.1858 – 20.09.1910), \emph{Schauspieler}|pw} war
               ebenſo herrlich.\pend
           \pstart
           Mit Brahm\pwindex{Brahm, Otto 05.02.1856 – 28.11.1912@\textsc{Brahm, Otto} (05.02.1856 – 28.11.1912), \emph{Theaterleiter, Regisseur}|pw} hab ich mich ſofort gezankt, er hat das
               Kind der \textsc{Katharina Binder}\pwindex{Schnitzler, Arthur 15.05.1862 – 21.10.1931@\textsc{Schnitzler, Arthur} (15.05.1862 – 21.10.1931), \emph{Schriftsteller, Mediziner}!Liebelei. Schauspiel in drei Akten1895-10-09@\strich\emph{Liebelei. Schauspiel in drei Akten} {[}1895-10-09{]}|pwv} gemordet – angeblich aus künſtleriſchen Gründen. Als ich dieſelben wi\strikeout{e}derlegte, ſtellte ſich heraus, daſs er überhaupt kein
               Kind zur Verfügung hatte. Ein paar Striche, die ganz überflüſſiger Weiſe geſchehn
               waren, machte ich wieder auf.\pend
           \pstart
           {\pb}Heute war Probe. Ich unterhielt mich ſehr gut.
               Sie wollen mehr wiſſen? Gelegentlich.\pend
           \pstart
           Stümke\pwindex{Stuemcke, Heinrich 07.05.1872 – 19.01.1923@\textsc{Stümcke, Heinrich} (07.05.1872 – 19.01.1923), \emph{Schriftsteller, Journalist}|pw} möchte nicht in meiner Haut ſtecken
               (Gegenſeitig!) Nemlich weil die Sti{\geminationm}ung gegen Brahm\pwindex{Brahm, Otto 05.02.1856 – 28.11.1912@\textsc{Brahm, Otto} (05.02.1856 – 28.11.1912), \emph{Theaterleiter, Regisseur}|pw}{ }ſehr heftig iſt und bei den \textsc{Premièren} »jedenfalls« auf Hausſchlüſſeln gepfiffen wird.
               Ich ka{\geminationn} natürlich kein Auge zuthun. »Gehn S’, ſein S’
               feſch, {\pb}und ko{\geminationm}en S’
               her!« Glauben Sie, daſs Librettiſten auf Nachſchlüſſeln pfeifen? (Herrn \textsc{Julius Bauer}\pwindex{Bauer, Julius 15.10.1853 – 11.06.1941@\textsc{Bauer, Julius} (15.10.1853 – 11.06.1941), \emph{Schriftsteller, Journalist, Kritiker}|pw} wohlgeboren)\pend
           \pstart
           – Wohin war mein erſter Gang? Zu dem Hauſe, das \textsc{ICH} vor
               8 Jahren bewohnt hatte. Jedes Poëtchen hat sein Pietätchen.\pend
           \pstart
           Schneit es in Wien\oindex{Wien@\textbf{Wien}|pw} noch ſo vehement, und wie geht es
                  Paula\pwindex{Beer-Hofmann, Paula 25.02.1879 – 30.10.1939@\textsc{Beer-Hofmann, Paula} (25.02.1879 – 30.10.1939)|pw}? (\introOben{}Ja we{\geminationn}
                  Sie wüßten was ich urſprünglich in diese Kla{\geminationm}er ſchreiben wollte!\introOben{})\pend
           \pstart
           {\pb}\textsc{Jarno}\pwindex{Jarno, Josef 24.08.1865 – 11.01.1932@\textsc{Jarno, Josef} (24.08.1865 – 11.01.1932), \emph{Theaterleiter, Schauspieler}|pw} läßt Sie grüßen; Sie waren ſeine erſte Frage. Die Staglé\pwindex{Stagle, Helene @\textsc{Staglé, Helene}, \emph{Schauspielerin}|pw} ist engagirt, ſpielt im »zerbrochnen Krug\pwindex{\textcolor{red}{\textsuperscript{XXXX1 indx}}!zerbrochene Krug1808@\strich\emph{Der zerbrochene Krug} {[}1808{]}|pw}« mit, der zur Liebelei\pwindex{Schnitzler, Arthur 15.05.1862 – 21.10.1931@\textsc{Schnitzler, Arthur} (15.05.1862 – 21.10.1931), \emph{Schriftsteller, Mediziner}!Liebelei. Schauspiel in drei Akten1895-10-09@\strich\emph{Liebelei. Schauspiel in drei Akten} {[}1895-10-09{]}|pw}
               dazu gegeben wird.\pend
           \pstart
           – Jetzt kleid ich mich um, gehe zum \textsc{König Chilperich}\pwindex{\textcolor{red}{\textsuperscript{XXXX1 indx}}!Koenig Chilperich24. 10. 1868@\strich\emph{König Chilperich} {[}24. 10. 1868{]}|pw}. Da{\geminationn} bin ich eingeladen\pwindex{Burchardt, Augusta *~4.7.1855@\textsc{Burchardt, Augusta} (*~4.7.1855)|pwv}. \label{K_L00531_1v}\edtext{\textsc{Si vous croyez, que c’est rigolo}!}{\lemma{\textnormal{\emph{Si … rigolo!}}}\Cendnote{\textnormal{französisch: Glauben Sie ja nicht, dass das unterhaltsam
                  ist!}}}\label{K_L00531_1h} – Womöglich als Zitat entnommen aus: Gyp\pwindex{Mirabeau, Sibylle de 15.08.1850 – 30.06.1932@\textsc{Mirabeau, Sibylle de} (15.08.1850 – 30.06.1932), \emph{Schriftstellerin}|pwk}: \emph{Le Mariage de Chiffon}\pwindex{Mirabeau, Sibylle de 15.08.1850 – 30.06.1932@\textsc{Mirabeau, Sibylle de} (15.08.1850 – 30.06.1932), \emph{Schriftstellerin}!Ehe von Chiffon1894@\strich\emph{Die Ehe von Chiffon} {[}1894{]}|pwk}.
                  Paris: \emph{Calmann-Lévy}\orgindex{Calmann-Levy@Calmann-Lévy|pwk}{ }1894, S. 47.\pend
           \pstart
           Grüßen Sie Salten\pwindex{Salten, Felix 06.09.1869 – 08.10.1945@\textsc{Salten, Felix} (06.09.1869 – 08.10.1945), \emph{Schriftsteller, Journalist}|pw}, Hugo\pwindex{Hofmannsthal, Hugo von 1874-02-01 – 1929-07-15@\textsc{Hofmannsthal, Hugo von} (1874-02-01 – 1929-07-15), \emph{Schriftsteller}|pw} un\textcolor{gray}{d} manche andre. Schreiben {\pb}Sie
               mir.\pend
           \pstart
           Herzlich der Ihre{\\[\baselineskip]}\spacefill\mbox{Arth}\pend
           \leftskip=0em{}
         
         \endnumbering\mylabel{h}\end{ledgroupsized}  \newcommand{\dateiname}{L00531}\newcommand{\titel}{Arthur Schnitzler an Richard Beer-Hofmann, 31. 1. 1896}\newcommand{\editorInnen}{Martin Anton Müller und Gerd-Hermann Susen}%% latex-leseansicht-abspann.tex
%% Abspann für die Leseansicht.
%% Der Schalter \ifkorrekturansicht ist bereits durch den Vorspann gesetzt.

%% latex-abspann.tex
%% Gemeinsamer Abspann für Korrekturansicht und Leseansicht.
%% Setzt den Schalter \ifkorrekturansicht voraus (gesetzt in den
%% einbindenden Dateien latex-korrekturansicht-abspann.tex bzw.
%% latex-leseansicht-abspann.tex).
%% ---------------------------------------------------------------

\normalsize

% Das esempio-Environment wird nur in der Leseansicht benötigt
\ifkorrekturansicht\else
\newenvironment{esempio}[3]%
{
    \vspace{1.5ex}
    \rlap{\underline{#1}}
    \par
    \setlength{\parindent}{0cm}
    \nopagebreak
    \leftskip=#2cm
    \rightskip=#3cm
}
{
    \par
}
\fi

\doendnotes{C}
\bigskip
\vfill

\clearpage

\footnotesize

\ifkorrekturansicht
  \lohead{\textsc{register}}
\fi

% theindex-Environment neu definieren ohne reledmac
\makeatletter
\renewenvironment{theindex}{%
  \ifkorrekturansicht
    \section*{\indexname}%
  \else
    \subsubsection*{Index der erwähnten Entitäten}%
  \fi
  \setlength{\parindent}{0pt}%
  \setlength{\parskip}{0pt plus 0.3pt}%
  \let\item\@idxitem
}{%
  \ifkorrekturansicht\clearpage\fi
}
\makeatother

\IfFileExists{\jobname-pw.ind}{\input{\jobname-pw.ind}}{}

% Quellenangabe nur in der Leseansicht
\ifkorrekturansicht\else
% Fallback-Definitionen, falls die .tex-Datei \titel etc. nicht gesetzt hat
\providecommand{\titel}{}
\providecommand{\editorInnen}{}
\providecommand{\dateiname}{\jobname}

\vspace{3cm}

\vfill

\footnotesize
\textsc{Quelle}: \titel. Herausgegeben von {\editorInnen}. In: \emph{Arthur Schnitzler: Briefwechsel mit Autorinnen und Autoren}.
 Digitale Edition, https://schnitzler-briefe.acdh.oeaw.ac.at/{\dateiname}.html (Stand \today)
\fi

\end{document}


      