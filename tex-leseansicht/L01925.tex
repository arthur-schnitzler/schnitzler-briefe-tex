%% latex-leseansicht-vorspann.tex
%% Vorspann für die Leseansicht.
%% Lädt die gemeinsame Datei latex-vorspann.tex mit nicht gesetztem Schalter.

\newif\ifkorrekturansicht
\korrekturansichtfalse

\input{../tex-inputs/latex-vorspann}


         
         \renewcommand{\erwaehntePersonen}{Personen: Anton Bettelheim, Marie von Ebner-Eschenbach, Hugo von Hofmannsthal, Ludwig Lobmeyr, Erich Schmidt, Karl Schönherr}
         \renewcommand{\erwaehnteInstitutionen}{Institutionen: Ebner-Eschenbach-Stiftung}
         \renewcommand{\erwaehnteOrte}{Orte: Wien}
         \renewcommand{\erwaehnteWerke}{}
               \section[Arthur Schnitzler an Hugo von Hofmannsthal, 21. 4. 1910]{ Arthur Schnitzler an Hugo von Hofmannsthal, 21. 4. 1910}\nopagebreak\mylabel{v}\rehead{ }\begin{ledgroupsized}[t]{13cm}\normalsize\beginnumbering\briefempfaengerindex{Hofmannsthal, Hugo von@\textsc{Hofmannsthal, Hugo von}!zzzSchnitzler, Arthur@\emph{von Arthur Schnitzler}!1910-04-211@{21. 4. 1910}|(be} \toendnotes[C]{\smallbreak\pagebreak[2]} \Standort{FDH, Hs-30885,136.}
\physDesc{Brief, Durchschlag, 1 Blatt, 1 Seite, 426 Zeichen
\newline{}Schreibmaschine
\newline{}Handschrift: roter Buntstift, deutsche Kurrent (\noindent{}Beschriftung: »\textsc{Hofmannsthal}« und eine Unterstreichung)
\newline{}Zusatz: Die Überlieferung im Nachlass Hofmannsthals deutet darauf hin,
                                 dass Schnitzler mit den eigenen Durchschlägen bei der Durchsicht
                                 seiner Briefe an Hofmannsthal 1929, Lücken
                                 ergänzte. }\buchAbdrucke{\weitereDrucke{Hugo von Hofmannsthal, Arthur Schnitzler: \emph{Briefwechsel}. Hg. Therese Nickl und Heinrich Schnitzler. Frankfurt am Main: \emph{S. Fischer} 1964, S. 249.} }\toendnotes[C]{\smallbreak}\pstart
           \raggedleft{}{\pb}21. 4. 1910.\pend
           \pstart\center{}Lieber Hugo. \pend\pstart
           Traf eben Dr. Anton
                  Bet{[}t{]}elheim\pwindex{Bettelheim, Anton 18.11.1851 – 29.03.1930@\textsc{Bettelheim, Anton} (18.11.1851 – 29.03.1930), \emph{Kritiker, Lexikograf}|pw}; er wollte Ihnen schreiben. Es handelt sich
               um eine \label{K_L01925-1v}\edtext{Ebner-Eschenbach-Stiftung\orgindex{Ebner-Eschenbach-Stiftung@Ebner-Eschenbach-Stiftung|pw}}{\lemma{\textnormal{\emph{Ebner-Eschenbach-Stiftung}}}\Cendnote{\textnormal{Obzwar im April 1910 ins
                  Leben gerufen, versandete das Unternehmen schnell. Ob tatsächlich Schulkindern zum
                  Schulabschluss Werke Ebner-Eschenbach\pwindex{Ebner-Eschenbach, Marie von 13.09.1830 – 12.03.1916@\textsc{Ebner-Eschenbach, Marie von} (13.09.1830 – 12.03.1916), \emph{Schriftstellerin}|pwk}s
                  geschenkt wurden, ist nicht nachgewiesen.}}}\label{K_L01925-1h} zum 80. Geburtstag. Aufruf: Erich Schmiedt\pwindex{Schmidt, Erich 20.06.1853 – 29.04.1913@\textsc{Schmidt, Erich} (20.06.1853 – 29.04.1913)|pw}{ }Lobmeyer\pwindex{Lobmeyr, Ludwig 02.08.1829 – 25.03.1917@\textsc{Lobmeyr, Ludwig} (02.08.1829 – 25.03.1917), \emph{Sammler, Fabrikant, Unternehmer}|pw}, Schönherr\pwindex{Schoenherr, Karl 24.02.1867 – 15.03.1943@\textsc{Schönherr, Karl} (24.02.1867 – 15.03.1943), \emph{Schriftsteller, Mediziner}|pw}, ich etc. Sie werden gebeten auch zu unterschreiben.
               Verpflichtungen sind damit keine übernommen, man zeichnet dann einen kleinen Betrag
               (ich zum Exempel etwa 20 K.). Bitte um eine Zeile, ob ich Bettelheim\pwindex{Bettelheim, Anton 18.11.1851 – 29.03.1930@\textsc{Bettelheim, Anton} (18.11.1851 – 29.03.1930), \emph{Kritiker, Lexikograf}|pw} Ihre Zustimmung vermelden darf.\pend
           
         
         \endnumbering\mylabel{h}\end{ledgroupsized}  \newcommand{\dateiname}{L01925}\newcommand{\titel}{Arthur Schnitzler an Hugo von Hofmannsthal, 21. 4. 1910}\newcommand{\editorInnen}{Martin Anton Müller und Gerd-Hermann Susen}%% latex-leseansicht-abspann.tex
%% Abspann für die Leseansicht.
%% Der Schalter \ifkorrekturansicht ist bereits durch den Vorspann gesetzt.

%% latex-abspann.tex
%% Gemeinsamer Abspann für Korrekturansicht und Leseansicht.
%% Setzt den Schalter \ifkorrekturansicht voraus (gesetzt in den
%% einbindenden Dateien latex-korrekturansicht-abspann.tex bzw.
%% latex-leseansicht-abspann.tex).
%% ---------------------------------------------------------------

\normalsize

% Das esempio-Environment wird nur in der Leseansicht benötigt
\ifkorrekturansicht\else
\newenvironment{esempio}[3]%
{
    \vspace{1.5ex}
    \rlap{\underline{#1}}
    \par
    \setlength{\parindent}{0cm}
    \nopagebreak
    \leftskip=#2cm
    \rightskip=#3cm
}
{
    \par
}
\fi

\doendnotes{C}
\bigskip
\vfill

\clearpage

\footnotesize

\ifkorrekturansicht
  \lohead{\textsc{register}}
\fi

% theindex-Environment neu definieren ohne reledmac
\makeatletter
\renewenvironment{theindex}{%
  \ifkorrekturansicht
    \section*{\indexname}%
  \else
    \subsubsection*{Index der erwähnten Entitäten}%
  \fi
  \setlength{\parindent}{0pt}%
  \setlength{\parskip}{0pt plus 0.3pt}%
  \let\item\@idxitem
}{%
  \ifkorrekturansicht\clearpage\fi
}
\makeatother

\IfFileExists{\jobname-pw.ind}{\input{\jobname-pw.ind}}{}

% Quellenangabe nur in der Leseansicht
\ifkorrekturansicht\else
% Fallback-Definitionen, falls die .tex-Datei \titel etc. nicht gesetzt hat
\providecommand{\titel}{}
\providecommand{\editorInnen}{}
\providecommand{\dateiname}{\jobname}

\vspace{3cm}

\vfill

\footnotesize
\textsc{Quelle}: \titel. Herausgegeben von {\editorInnen}. In: \emph{Arthur Schnitzler: Briefwechsel mit Autorinnen und Autoren}.
 Digitale Edition, https://schnitzler-briefe.acdh.oeaw.ac.at/{\dateiname}.html (Stand \today)
\fi

\end{document}


      