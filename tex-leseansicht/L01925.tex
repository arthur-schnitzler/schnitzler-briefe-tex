%% latex-korrekturansicht-vorspann.tex
%% Vorspann für die Korrekturansicht.
%% Lädt die gemeinsame Datei latex-vorspann.tex mit gesetztem Schalter.

\newif\ifkorrekturansicht
\korrekturansichttrue

\input{../tex-inputs/latex-vorspann}


\section[Arthur Schnitzler an Hugo von Hofmannsthal, 21. 4. 1910]{L01925 Arthur Schnitzler an Hugo von Hofmannsthal, 21. 4. 1910}
\nopagebreak\mylabel{L01925v}
\rehead{ }\normalsize\beginnumbering\briefempfaengerindex{Hofmannsthal, Hugo von@\textsc{Hofmannsthal, Hugo von}!zzzSchnitzler, Arthur@\emph{von Arthur Schnitzler}!1910-04-211@{21. 4. 1910}|(be}
\toendnotes[C]{\smallbreak\pagebreak[2]}\Standort{FDH, Hs-30885,136.}
\physDesc{Brief, Durchschlag1 Blatt, 1 Seite, 426 Zeichen
\newline{}Schreibmaschine
\newline{}Handschrift: roter Buntstift, deutsche Kurrent (\noindent{}Beschriftung: »\textsc{Hofmannsthal}« und eine Unterstreichung)
\newline{}Zusatz: Die Überlieferung im Nachlass Hofmannsthals deutet darauf hin,
                                 dass Schnitzler mit den eigenen Durchschlägen bei der Durchsicht
                                 seiner Briefe an Hofmannsthal 1929, Lücken
                                 ergänzte. }
\buchAbdrucke{\weitereDrucke{Hugo von Hofmannsthal, Arthur Schnitzler: \emph{Briefwechsel}. Frankfurt am Main: \emph{S. Fischer} 1964, S. 249.} }\toendnotes[C]{\smallbreak}
\pstart
           \raggedleft{}{\pb}21. 4. 1910.\pend
           
\pstart\center{}Lieber Hugo. \pend\vspace{0.5em}
\pstart
           Traf eben Dr. Anton
                  Bet{[}t{]}elheim\pwindex{Bettelheim, Anton 18.11.1851 – 29.03.1930@\textsc{Bettelheim, Anton} (18.11.1851 – 29.03.1930), \emph{Kritiker/Kritikerin, Lexikograf/Lexikografin}|pw}; er wollte Ihnen schreiben. Es handelt sich
               um eine \label{K_L01925-1v}\edtext{Ebner-Eschenbach-Stiftung\orgindex{Ebner-Eschenbach-Stiftung@Ebner-Eschenbach-Stiftung|pw}}{\lemma{\textnormal{\emph{Ebner-Eschenbach-Stiftung}}}\Cendnote{\textnormal{Obzwar im April 1910 ins
                  Leben gerufen, versandete das Unternehmen schnell. Ob tatsächlich Schulkindern zum
                  Schulabschluss Werke Ebner-Eschenbachs\pwindex{Ebner-Eschenbach, Marie von 13.09.1830 – 12.03.1916@\textsc{Ebner-Eschenbach, Marie von} (13.09.1830 – 12.03.1916), \emph{Schriftsteller/Schriftstellerin, Schriftsteller/Schriftstellerin}|pwk}
                  geschenkt wurden, ist nicht belegt.}}}\label{K_L01925-1} zum 80. Geburtstag. Aufruf: Erich Schmiedt\pwindex{Schmidt, Erich 20.06.1853 – 29.04.1913@\textsc{Schmidt, Erich} (20.06.1853 – 29.04.1913)|pw}{ }Lobmeyer\pwindex{Lobmeyr, Ludwig 02.08.1829 – 25.03.1917@\textsc{Lobmeyr, Ludwig} (02.08.1829 – 25.03.1917), \emph{Kunstsammler/Kunstsammlerin, Glasfabrikant/Glasfabrikantin, Unternehmer/Unternehmerin}|pw}, Schönherr\pwindex{Schoenherr, Karl 24.02.1867 – 15.03.1943@\textsc{Schönherr, Karl} (24.02.1867 – 15.03.1943), \emph{Schriftsteller/Schriftstellerin, Mediziner/Medizinerin}|pw}, ich etc. Sie werden gebeten auch zu unterschreiben.
               Verpflichtungen sind damit keine übernommen, man zeichnet dann einen kleinen Betrag
               (ich zum Exempel etwa 20 K.). Bitte um eine Zeile, ob ich Bettelheim\pwindex{Bettelheim, Anton 18.11.1851 – 29.03.1930@\textsc{Bettelheim, Anton} (18.11.1851 – 29.03.1930), \emph{Kritiker/Kritikerin, Lexikograf/Lexikografin}|pw} Ihre Zustimmung vermelden darf.\pend
           \selectlanguage{ngerman}\endnumbering\briefempfaengerindex{Hofmannsthal, Hugo von@\textsc{Hofmannsthal, Hugo von}!zzzSchnitzler, Arthur@\emph{von Arthur Schnitzler}!1910-04-211@{21. 4. 1910}|)be}\mylabel{L01925h}  \normalsize

\doendnotes{C}
\bigskip
\vfill

\clearpage

\footnotesize

\lohead{\textsc{register}}

% Definiere theindex-Environment komplett neu ohne reledmac
\makeatletter
\renewenvironment{theindex}{%
  \section*{\indexname}%
  \setlength{\parindent}{0pt}%
  \setlength{\parskip}{0pt plus 0.3pt}%
  \let\item\@idxitem
}{%
  \clearpage
}
\makeatother

\IfFileExists{\jobname-pw.ind}{\input{\jobname-pw.ind}}{}

\end{document}

      