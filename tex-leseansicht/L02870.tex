%% latex-leseansicht-vorspann.tex
%% Vorspann für die Leseansicht.
%% Lädt die gemeinsame Datei latex-vorspann.tex mit nicht gesetztem Schalter.

\newif\ifkorrekturansicht
\korrekturansichtfalse

\input{../tex-inputs/latex-vorspann}


\section[ Paul Goldmann an Arthur Schnitzler, 20. 3. [1899]]{L02870 Paul Goldmann an Arthur Schnitzler,  20. 3. [1899]}
\nopagebreak\mylabel{L02870v}
\rehead{ }\normalsize\beginnumbering\briefempfaengerindex{Schnitzler, Arthur@\textsc{Schnitzler, Arthur}!zzzGoldmann, Paul@\emph{von Paul Goldmann}!1899-03-202@{20. 3. [1899]}|(be}
\toendnotes[C]{\smallbreak\pagebreak[2]}
\correspDesc{Versand  durch Paul Goldmann am 20. 3. [1899] in Frankfurt am Main
\newline{}Erhalt  durch Arthur Schnitzler im Zeitraum [21. 3. 1899
                  – 25. 3. 1899?] in Wien}\toendnotes[C]{\smallbreak}
\Standort{DLA, A:Schnitzler, HS.NZ85.1.3169.}
\physDesc{Brief, 1 Blatt, 2 Seiten, 1022 Zeichen
\newline{}Handschrift: schwarze Tinte, deutsche Kurrent
\newline{}Schnitzler: mit Bleistift das Jahr »99« vermerkt }\toendnotes[C]{\smallbreak}
\pstart
           \raggedleft{}{\pb}Frankfurt\oindex{Frankfurt am Main@\textbf{Frankfurt am Main}, \emph{Hauptstadt}|pw}, 20. März.\pend
           
\pstart{}Mein armer lieber Freund,\pend\vspace{0.5em}
\pstart
           \label{K_L02870-1v}\edtext{Es iſt entſetzlich,}{\lemma{\textnormal{\emph{Es ist entsetzlich,}}}\Cendnote{\textnormal{Bezug auf Marie Reinhards\pwindex{Reinhard, Marie 13.\,3.\,1871 Wien – 18.\,3.\,1899 ebd.@\textsc{Reinhard, Marie} (13.\,3.\,1871 Wien – 18.\,3.\,1899 ebd.), \emph{Gesangspädagogin}|pwk} Tod am 18. 3. 1899}}}\label{K_L02870-1} und ich kann es kaum faſſen. \substVorne{}\textsuperscript{D\textcolor{gray}{u}\textcolor{gray}{×}}\substDazwischen{}Die\substHinten{} arme junge Frau\pwindex{Reinhard, Marie 13.\,3.\,1871 Wien – 18.\,3.\,1899 ebd.@\textsc{Reinhard, Marie} (13.\,3.\,1871 Wien – 18.\,3.\,1899 ebd.), \emph{Gesangspädagogin}|pwv}! Hat{ }ſie wenigſtens nicht allzuviel gelitten? Warum gerade{ }ſie es{ }ſein mußte? Und warum
               Dir das Unheil mit{ }ſo wahnſinniger \label{K_L02870-2v}\edtext{Hartnäckigkeit}{\lemma{\textnormal{\emph{Hartnäckigkeit}}}\Cendnote{\textnormal{womöglich Bezug auf
                     Olga Waissnix\pwindex{Waissnix, Olga 3.\,11.\,1862 Wien – 4.\,11.\,1897 ebd.@\textsc{Waissnix, Olga} (3.\,11.\,1862 Wien – 4.\,11.\,1897 ebd.), \emph{Hotelière}|pwk}’ Tod nur eineinhalb Jahre
                  zuvor, am 4. 11. 1897}}}\label{K_L02870-2} zuſetzt? {\dots} Ich fand heut{ }früh Deine Depeſche, die mich wie ein Donnerſchlag traf. Wie hätte man
               darauf gefaßt{ }ſein{ }ſollen! Sagen kann man nichts dazu. Nur bei Dir{ }ſein möchte ich.
               Nimm’ Dich zuſammen, Lieber, Guter! Trage auch das! Suche ein wenig Ruhe zu finden
               bei dem Gedanken an das, was geweſen iſt und was Dir kein Tod rauben kann. {\pb}Du mußt ſie\pwindex{Reinhard, Marie 13.\,3.\,1871 Wien – 18.\,3.\,1899 ebd.@\textsc{Reinhard, Marie} (13.\,3.\,1871 Wien – 18.\,3.\,1899 ebd.), \emph{Gesangspädagogin}|pwv}{ }ſanft betrauern. Das iſt die Trauer, die im Sinne ihres
               armen Herzens iſt. Und Du mußt, Du \uline{mußt} Dich zu der
               Erkenntniß durchringen, daß{ }ſelbſt jetzt nicht Alles zu Ende iſt und daß{ }ſelbſt nach
               dieſem Schlage das Leben weitergeht.\pend
           
\pstart
           Ich umarme Dich von Herzen und in Treue {\\[\baselineskip]}Dein {\\[\baselineskip]}\spacefill\mbox{Paul Goldmann.}\pend
           \leftskip=0em{}
\pstart
           \noindent{}Jetzt{ }ſollſt Du mir nichts{ }ſchreiben. Aber bitte,{ }ſobald Du kannſt, theile mir
                  etwas Näheres mit!\pend
           
\pstart
           Könnteſt Du nicht auf ein paar Wochen \label{K_L02870-3v}\edtext{von Wien\oindex{Wien@\textbf{Wien}, \emph{Verwaltungsgebiet}|pw} fort}{\lemma{\textnormal{\emph{von Wien fort}}}\Cendnote{\textnormal{Schnitzler blieb – abgesehen von einem
                     kurzen Aufenthalt in der Steiermark\oindex{Steiermark@\textbf{Steiermark}, \emph{Land}|pwk} (1. 4. 1899 – 3. 4. 1899) – bis
                     zu seiner Reise nach Berlin\oindex{Berlin@\textbf{Berlin}, \emph{Hauptstadt}|pwk} Ende April 1899 (siehe XXXX Auszeichnungsfehler: Dokument L02869 nicht gefunden) in Wien\oindex{Wien@\textbf{Wien}, \emph{Verwaltungsgebiet}|pwk}.}}}\label{K_L02870-3}?\pend
           \selectlanguage{ngerman}\endnumbering\briefempfaengerindex{Schnitzler, Arthur@\textsc{Schnitzler, Arthur}!zzzGoldmann, Paul@\emph{von Paul Goldmann}!1899-03-202@{20. 3. [1899]}|)be}\mylabel{L02870h}  \newcommand{\dateiname}{L02870}\newcommand{\titel}{Paul Goldmann an Arthur Schnitzler, 20. 3. [1899]}\newcommand{\editorInnen}{Martin Anton Müller und Laura Untner}%% latex-leseansicht-abspann.tex
%% Abspann für die Leseansicht.
%% Der Schalter \ifkorrekturansicht ist bereits durch den Vorspann gesetzt.

%% latex-abspann.tex
%% Gemeinsamer Abspann für Korrekturansicht und Leseansicht.
%% Setzt den Schalter \ifkorrekturansicht voraus (gesetzt in den
%% einbindenden Dateien latex-korrekturansicht-abspann.tex bzw.
%% latex-leseansicht-abspann.tex).
%% ---------------------------------------------------------------

\normalsize

% Das esempio-Environment wird nur in der Leseansicht benötigt
\ifkorrekturansicht\else
\newenvironment{esempio}[3]%
{
    \vspace{1.5ex}
    \rlap{\underline{#1}}
    \par
    \setlength{\parindent}{0cm}
    \nopagebreak
    \leftskip=#2cm
    \rightskip=#3cm
}
{
    \par
}
\fi

\doendnotes{C}
\bigskip
\vfill

\clearpage

\footnotesize

\ifkorrekturansicht
  \lohead{\textsc{register}}
\fi

% theindex-Environment neu definieren ohne reledmac
\makeatletter
\renewenvironment{theindex}{%
  \ifkorrekturansicht
    \section*{\indexname}%
  \else
    \subsubsection*{Index der erwähnten Entitäten}%
  \fi
  \setlength{\parindent}{0pt}%
  \setlength{\parskip}{0pt plus 0.3pt}%
  \let\item\@idxitem
}{%
  \ifkorrekturansicht\clearpage\fi
}
\makeatother

\IfFileExists{\jobname-pw.ind}{\input{\jobname-pw.ind}}{}

% Quellenangabe nur in der Leseansicht
\ifkorrekturansicht\else
% Fallback-Definitionen, falls die .tex-Datei \titel etc. nicht gesetzt hat
\providecommand{\titel}{}
\providecommand{\editorInnen}{}
\providecommand{\dateiname}{\jobname}

\vspace{3cm}

\vfill

\footnotesize
\textsc{Quelle}: \titel. Herausgegeben von {\editorInnen}. In: \emph{Arthur Schnitzler: Briefwechsel mit Autorinnen und Autoren}.
 Digitale Edition, https://schnitzler-briefe.acdh.oeaw.ac.at/{\dateiname}.html (Stand \today)
\fi

\end{document}


