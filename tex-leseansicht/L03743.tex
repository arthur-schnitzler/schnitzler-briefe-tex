%% latex-korrekturansicht-vorspann.tex
%% Vorspann für die Korrekturansicht.
%% Lädt die gemeinsame Datei latex-vorspann.tex mit gesetztem Schalter.

\newif\ifkorrekturansicht
\korrekturansichttrue

\input{../tex-inputs/latex-vorspann}


\section[Arthur Schnitzler an Stefan Zweig, 4. 7. 1929]{L03743 Arthur Schnitzler an Stefan Zweig, 4. 7. 1929}
\nopagebreak\mylabel{L03743v}
\rehead{ }\normalsize\beginnumbering\briefempfaengerindex{Zweig, Stefan@\textsc{Zweig, Stefan}!zzzSchnitzler, Arthur@\emph{von Arthur Schnitzler}!1929-07-041@{4. 7. 1929}|(be}
\toendnotes[C]{\smallbreak\pagebreak[2]}\Standort{Jerusalem, National Library of Israel, ARC. Ms. Var. 305 1 58 Stefan Zweig Collection.}
\physDesc{Brief, 1 Blatt, 2 Seiten, 1468 Zeichen
\newline{}Handschrift: Bleistift, lateinische Kurrent (\noindent{}zwei Unterstreichungen, eine Streichung, eine Ergänzung,
                                 Schlussformel und Unterschrift)}\toendnotes[C]{\smallbreak}
\pstart
           {\pb}\textcolor{gray}{\textbf{D\textsuperscript{R} ARTHUR SCHNITZLER}}\hfill 4. 7. 1929.\pend
           
\pstart
           \textcolor{gray}{\textbf{WIEN, XVIII.
                        STERNWARTESTRASSE 71\oindex{Sternwartestrasse 71@\textbf{Sternwartestraße 71}, \emph{Wohngebäude (K.WHS)}|pw}.}}\pend
           
\pstart{}Lieber Doktor Zweig.\pend\vspace{0.5em}
\pstart
           Sie wissen jedesfalls von der Absicht des Schutzverbandes\orgindex{Schutzverband deutscher Schriftsteller@Schutzverband deutscher Schriftsteller|pw} zu wohltätigem Zwecke in einer der grossen Berlin\oindex{Berlin@\textbf{Berlin}, \emph{P.PPLC}|pw}er Kunsthandlungen eine Ausstellung von Manuscripten und
               im Anschluss daran eine \label{K_L03743-11v}\edtext{Versteigerung}{\lemma{\textnormal{\emph{Versteigerung}}}\Cendnote{\textnormal{Die Veranstaltung kam nicht zustande.}}}\label{K_L03743-11} vornehmen zu lassen. Roda-Roda\pwindex{Roda Roda, Alexander 13.04.1872 – 20.08.1945@\textsc{Roda Roda, Alexander} (13.04.1872 – 20.08.1945), \emph{Schriftsteller/Schriftstellerin}|pw}, der mir über die Sache geschrieben hat, ist dafür,
               dass die Sammlung entweder im Ganzen oder aber in »Loten« von etwa 10 Manuscripten
               aufgeboten werden müsse. Er schliesst sich übrigens meiner Meinung an, dass man \uline{Ausrufpreise} ansetze, \uline{unterhalb} deren ein Verkauf nicht stattfinden dürfe. Sie, lieber Doktor
               Zweig, sind ja in solchen Handschriften-Angelegenheiten besonders sachverständig. Und
               ich frage daher bei Ihnen ganz unverbindlich an, welche Ausrufpreise Sie im
               allgemeinen und im besonderen für richtig fänden, wenn ich \label{K_L03743-1v}\edtext{z. E.}{\lemma{\textnormal{\emph{z. E.}}}\Cendnote{\textnormal{zum
                  Einen}}}\label{K_L03743-1} für eine solche Versteigerung alte Manuscripte von Gedichten oder
                  \strikeout{beispielsweise} das erste bleistiftgeschriebene
               Manuscript des »Grünen Kakadu\pwindex{gruene Kakadu. Groteske in einem Akt@\emph{Der grüne Kakadu. Groteske in einem Akt}|pw}« oder eines \introOben{}andern\introOben{} Einakters zur Verfügung stellte. \pend
           
\pstart
           Dieser Brief trifft Sie wohl noch in Salzburg\oindex{Salzburg@\textbf{Salzburg}, \emph{A.ADM2}|pw} an.
               Teilen Sie mir bitte mit, wohin Ihre Sommerpläne gehen. Es wäre eine angenehme
               Aussicht Ihnen wieder einmal in der Schweiz\oindex{Schweiz@\textbf{Schweiz}, \emph{A.PCLI}|pw}
               oder sonstwo zu begegnen. Salzburg\oindex{Salzburg@\textbf{Salzburg}, \emph{A.ADM2}|pw} und Wien\oindex{Wien@\textbf{Wien}, \emph{A.ADM2}|pw} liegen offenbar zu nah von einander. Zu
               wievielen Erfolgen habe ich Ihnen eigentlich zu gratulieren, seit wir uns zuletzt
               gesehen haben? Nehmen Sie eine Gesamtgratulation zugleich mit meinen herzlichsten
               Grüssen freundlichst entgegen. \pend
           
\pstart
           {[}hs.:{]} Ihr{\\[\baselineskip]}\spacefill\mbox{ArtSchnitzl}\pend
           \leftskip=0em{}
\pstart
           \noindent{}{[}ms.:{]} Herrn Dr. Stefan Zweig\pend
           
\pstart
           \noindent{}Salzburg\oindex{Salzburg@\textbf{Salzburg}, \emph{A.ADM2}|pw}.\pend
           \selectlanguage{ngerman}\endnumbering\briefempfaengerindex{Zweig, Stefan@\textsc{Zweig, Stefan}!zzzSchnitzler, Arthur@\emph{von Arthur Schnitzler}!1929-07-041@{4. 7. 1929}|)be}\mylabel{L03743h}
\begin{anhang}
\end{anhang}\normalsize

\doendnotes{C}
\bigskip
\vfill

\clearpage

\footnotesize

\lohead{\textsc{register}}

% Definiere theindex-Environment komplett neu ohne reledmac
\makeatletter
\renewenvironment{theindex}{%
  \section*{\indexname}%
  \setlength{\parindent}{0pt}%
  \setlength{\parskip}{0pt plus 0.3pt}%
  \let\item\@idxitem
}{%
  \clearpage
}
\makeatother

\IfFileExists{\jobname-pw.ind}{\input{\jobname-pw.ind}}{}

\end{document}

      