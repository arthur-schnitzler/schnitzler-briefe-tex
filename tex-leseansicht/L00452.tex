%% latex-leseansicht-vorspann.tex
%% Vorspann für die Leseansicht.
%% Lädt die gemeinsame Datei latex-vorspann.tex mit nicht gesetztem Schalter.

\newif\ifkorrekturansicht
\korrekturansichtfalse

\input{../tex-inputs/latex-vorspann}


\section[Hermann Bahr an Arthur Schnitzler, {{[}}13. 6. 1895{{]}}]{L00452 Hermann Bahr an Arthur Schnitzler, {[}13. 6. 1895{]}}
\nopagebreak\mylabel{L00452v}
\rehead{ }\normalsize\beginnumbering\briefempfaengerindex{Schnitzler, Arthur@\textsc{Schnitzler, Arthur}!zzzBahr, Hermann@\emph{von Hermann Bahr}!1895-06-131@{{[}13. 6. 1895{]}}|(be}
\toendnotes[C]{\smallbreak\pagebreak[2]}
\correspDesc{Versand  durch Hermann Bahr am [13. 6. 1895] in Wien
\newline{}Erhalt  durch Arthur Schnitzler im Zeitraum [13. 6. 1895
                  – 17. 6. 1895?] in Wien}\toendnotes[C]{\smallbreak}
\Standort{CUL, Schnitzler, B 5b.}
\physDesc{Brief, 1 Blatt, 1 Seite, 326 Zeichen
\newline{}Handschrift: schwarze Tinte, deutsche Kurrent
\newline{}Schnitzler: mit Bleistift datiert: »13/6 95« 
\newline{}Ordnung: 1) mit rotem Buntstift von unbekannter Hand nummeriert:
                                    »28«  2) mit Bleistift von unbekannter Hand nummeriert:
                                    »28«}
\buchAbdrucke{\weitereDrucke{Hermann Bahr, Arthur Schnitzler: \emph{Briefwechsel, Aufzeichnungen, Dokumente (1891–1931)}. Herausgegeben von Kurt Ifkovits und Martin Anton Müller. Göttingen: \emph{Wallstein} 2018, S. 102.} }\toendnotes[C]{\smallbreak}
\pstart
           {\pb}\textcolor{gray}{\textbf{»Die Zeit\orgindex{Zeit. Wiener Wochenschrift@Die Zeit. Wiener Wochenschrift|pw}«}}\hfill \textcolor{gray}{\textbf{\textbf{Wien\oindex{Wien@\textbf{Wien}, \emph{Verwaltungsgebiet}|pw}}, den ..........189{\dotstwo}}}\pend
           
\pstart
           \textcolor{gray}{\textbf{Wiener Wochenſchrift}}\hfill \textcolor{gray}{\textbf{IX/3, Günthergaſſe 1\oindex{Wien@\textbf{Wien}!IX., Alsergrund@\textbf{IX., Alsergrund}!Günthergasse@\textbf{Günthergasse}, \emph{Straße}|pw}.}}\pend
           
\pstart
           \textcolor{gray}{\textbf{\textbf{Herausgeber}:}}{\\}\textcolor{gray}{\textbf{Profeſſor Dr. I. Singer\pwindex{Singer, Isidor 16.\,1.\,1857 Budapest – 8.\,12.\,1927 Wien@\textsc{Singer, Isidor} (16.\,1.\,1857 Budapest – 8.\,12.\,1927 Wien), \emph{Journalist, Herausgeber, Soziologe}|pw}, Hermann Bahr\pwindex{Bahr, Hermann 19.\,7.\,1863 Linz – 15.\,1.\,1934 München@\textsc{Bahr, Hermann} (19.\,7.\,1863 Linz – 15.\,1.\,1934 München), \emph{Schriftsteller, Kritiker}|pw},
                        Dr. Heinrich Kanner\pwindex{Kanner, Heinrich 9.\,11.\,1864 Galați – 15.\,2.\,1930 Wien@\textsc{Kanner, Heinrich} (9.\,11.\,1864 Galați – 15.\,2.\,1930 Wien), \emph{Herausgeber, Publizist}|pw}.}}\pend
           
\pstart
           \textcolor{gray}{\textbf{Telephon Nr. 6415.}}\pend
           
\pstart\center{}Lieber Arthur!\pend\vspace{0.5em}
\pstart
           Ich bin von \textsc{Fischer}\pwindex{Fischer, Samuel 24.\,12.\,1859 Liptovský Mikuláš – 15.\,10.\,1934 Berlin@\textsc{Fischer, Samuel} (24.\,12.\,1859 Liptovský Mikuláš – 15.\,10.\,1934 Berlin), \emph{Verleger}|pw} nie pro Seite, nie pro Werk \label{K_L00452-1v}\edtext{bezahlt}{\lemma{\textnormal{\emph{bezahlt}}}\Cendnote{\textnormal{Vgl. XXXX Auszeichnungsfehler: Dokument L00454 nicht gefunden; siehe auch die
                  Antwort Fischers\pwindex{Fischer, Samuel 24.\,12.\,1859 Liptovský Mikuláš – 15.\,10.\,1934 Berlin@\textsc{Fischer, Samuel} (24.\,12.\,1859 Liptovský Mikuláš – 15.\,10.\,1934 Berlin), \emph{Verleger}|pwk} vom gleichen Tag (Samuel Fischer\pwindex{Fischer, Samuel 24.\,12.\,1859 Liptovský Mikuláš – 15.\,10.\,1934 Berlin@\textsc{Fischer, Samuel} (24.\,12.\,1859 Liptovský Mikuláš – 15.\,10.\,1934 Berlin), \emph{Verleger}|pwk}, Hedwig Fischer\pwindex{Fischer, Hedwig 8.\,9.\,1871 Szczecin – 11.\,4.\,1952 Königstein im Taunus@\textsc{Fischer, Hedwig} (8.\,9.\,1871 Szczecin – 11.\,4.\,1952 Königstein im Taunus)|pwk}: \emph{Briefwechsel mit
                        Autoren}. Herausgegeben von Dierk Rodewald und Corinna Fiedler. Mit
                     einer Einführung von Bernhard Zeller. Frankfurt am Main:
                     \emph{S. Fischer}{ }1989, S. 55).}}}\label{K_L00452-1} worden,{ }ſondern
               er hat mir tauſend Mark geliehen, dann habe ich einiges geſchrieben, dann hat er mir
               wieder geliehen und wir waren beide immer überzeugt, daß der andere ein großer Schuft
               ist. Deshalb kann ich Deine Frage nicht beantworten.\pend
           
\pstart
           Herzlichſt{\\[\baselineskip]}Dein{\\[\baselineskip]}\spacefill\mbox{Hermann}\pend
           \leftskip=0em{}
\pstart
           \textcolor{gray}{\textbf{\label{T_L00452-1v}\edtext{Alle für »Die Zeit\orgindex{Zeit. Wiener Wochenschrift@Die Zeit. Wiener Wochenschrift|pw}« beſtimmten Zuſchriften und Sendungen{ }ſind an die
                  Redaction der »Zeit\orgindex{Zeit. Wiener Wochenschrift@Die Zeit. Wiener Wochenschrift|pw}« und \textbf{nicht} an die Perſon eines der Herausgeber zu richten.}{\lemma{\textnormal{\emph{Alle … richten.}}}\Cendnote{\textnormal{am unteren Rand der Seite}}}\label{T_L00452-1}}}\pend
           \selectlanguage{ngerman}\endnumbering\briefempfaengerindex{Schnitzler, Arthur@\textsc{Schnitzler, Arthur}!zzzBahr, Hermann@\emph{von Hermann Bahr}!1895-06-131@{{[}13. 6. 1895{]}}|)be}\mylabel{L00452h}  \newcommand{\dateiname}{L00452}\newcommand{\titel}{Hermann Bahr an Arthur Schnitzler, [13. 6. 1895]}\newcommand{\editorInnen}{Herausgegeben von Martin Anton Müller}%% latex-leseansicht-abspann.tex
%% Abspann für die Leseansicht.
%% Der Schalter \ifkorrekturansicht ist bereits durch den Vorspann gesetzt.

%% latex-abspann.tex
%% Gemeinsamer Abspann für Korrekturansicht und Leseansicht.
%% Setzt den Schalter \ifkorrekturansicht voraus (gesetzt in den
%% einbindenden Dateien latex-korrekturansicht-abspann.tex bzw.
%% latex-leseansicht-abspann.tex).
%% ---------------------------------------------------------------

\normalsize

% Das esempio-Environment wird nur in der Leseansicht benötigt
\ifkorrekturansicht\else
\newenvironment{esempio}[3]%
{
    \vspace{1.5ex}
    \rlap{\underline{#1}}
    \par
    \setlength{\parindent}{0cm}
    \nopagebreak
    \leftskip=#2cm
    \rightskip=#3cm
}
{
    \par
}
\fi

\doendnotes{C}
\bigskip
\vfill

\clearpage

\footnotesize

\ifkorrekturansicht
  \lohead{\textsc{register}}
\fi

% theindex-Environment neu definieren ohne reledmac
\makeatletter
\renewenvironment{theindex}{%
  \ifkorrekturansicht
    \section*{\indexname}%
  \else
    \subsubsection*{Index der erwähnten Entitäten}%
  \fi
  \setlength{\parindent}{0pt}%
  \setlength{\parskip}{0pt plus 0.3pt}%
  \let\item\@idxitem
}{%
  \ifkorrekturansicht\clearpage\fi
}
\makeatother

\IfFileExists{\jobname-pw.ind}{\input{\jobname-pw.ind}}{}

% Quellenangabe nur in der Leseansicht
\ifkorrekturansicht\else
% Fallback-Definitionen, falls die .tex-Datei \titel etc. nicht gesetzt hat
\providecommand{\titel}{}
\providecommand{\editorInnen}{}
\providecommand{\dateiname}{\jobname}

\vspace{3cm}

\vfill

\footnotesize
\textsc{Quelle}: \titel. Herausgegeben von {\editorInnen}. In: \emph{Arthur Schnitzler: Briefwechsel mit Autorinnen und Autoren}.
 Digitale Edition, https://schnitzler-briefe.acdh.oeaw.ac.at/{\dateiname}.html (Stand \today)
\fi

\end{document}


