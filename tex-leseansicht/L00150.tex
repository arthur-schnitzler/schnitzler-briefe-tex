%% latex-korrekturansicht-vorspann.tex
%% Vorspann für die Korrekturansicht.
%% Lädt die gemeinsame Datei latex-vorspann.tex mit gesetztem Schalter.

\newif\ifkorrekturansicht
\korrekturansichttrue

\input{../tex-inputs/latex-vorspann}


\section[Karl Kraus an Arthur Schnitzler, 31. 12. 1892]{L00150 Karl Kraus an Arthur Schnitzler, 31. 12. 1892}
\nopagebreak\mylabel{L00150v}
\rehead{ }\normalsize\beginnumbering\briefempfaengerindex{Schnitzler, Arthur@\textsc{Schnitzler, Arthur}!zzzKraus, Karl@\emph{von Karl Kraus}!1892-12-311@{31. 12. 1892}|(be}
\toendnotes[C]{\smallbreak\pagebreak[2]}\Standort{CUL, Schnitzler, B 55.}
\physDesc{Postkarte, 367 Zeichen
\newline{}Handschrift: schwarze Tinte, deutsche Kurrent
\newline{}Versand: Stempel: »\nobreak{}\oindex{I., Innere Stadt@\textbf{I., Innere Stadt}, \emph{A.ADM3}|pwk}Wien 1/1, 31. 12. 92, 7–8 N\nobreak{}«.  }
\buchAbdrucke{\weitereDrucke{\emph{Literatur und Kritik}, Bd. 49, Oktober 1970, S. 514.} }\toendnotes[C]{\smallbreak}\pstart{}{\pb}Herrn Schriftsteller\pend{}\pstart{}D\textsuperscript{r.} Arthur Schnitzler,\pend{}\pstart{}Wien I\oindex{I., Innere Stadt@\textbf{I., Innere Stadt}, \emph{A.ADM3}|pw}\pend{}\pstart{}Grillparzerstr. 7\oindex{Grillparzerstrasse@\textbf{Grillparzerstraße}, \emph{R.ST}|pw}.\pend{}{\bigskip}\vspace{1em}
\pstart{}{\pb}Mein lieber Herr Doctor!\pend\vspace{0.5em}
\pstart
           Die \label{K_L00150-1v}\edtext{Kritik\pwindex{Arthur Schnitzler, Anatol@\emph{Arthur Schnitzler, Anatol}|pwv}}{\lemma{\textnormal{\emph{Kritik}}}\Cendnote{\textnormal{Karl Kraus\pwindex{Kraus, Karl 28.04.1874 – 12.06.1936@\textsc{Kraus, Karl} (28.04.1874 – 12.06.1936), \emph{Schriftsteller/Schriftstellerin, Publizist/Publizistin, Schriftsteller/Schriftstellerin}|pwk}: \emph{Arthur Schnitzler, Anatol}\pwindex{Arthur Schnitzler, Anatol@\emph{Arthur Schnitzler, Anatol}|pwk}. In: \emph{Die Gesellschaft}\pwindex{Gesellschaft. Monatsschrift fuer Litteratur, Kunst und Sozialpolitik@\emph{Die Gesellschaft. Monatsschrift für Litteratur, Kunst und Sozialpolitik}|pwk}, Jg. 9, Nr. 1, 1. 1. 1893,
                     S. 109–110.}}}\label{K_L00150-1} über »Anatol\pwindex{Anatol@\emph{Anatol}|pw}«
               (2 Spalten) iſt im Jännerheft der »Geſellſch.\orgindex{Gesellschaft@Die Gesellschaft|pw}«
               erſchienen. Beleg wird die Schriftleitung an den Verlag\orgindex{Bibliographisches Bureau@Bibliographisches Bureau|pwv} nach Berlin\oindex{Berlin@\textbf{Berlin}, \emph{P.PPLC}|pw}{ }ſchicken. Warum kommen Sie nicht mehr ins Grienſteidl\oindex{Cafe Griensteidl@\textbf{Café Griensteidl}, \emph{Kaffeehaus (K.KAF)}|pw}? Wie geht’s?\pend
           
\pstart
           Herzlichſte Grüße!{\\[\baselineskip]}Prost Neujahr!{\\[\baselineskip]}Ihr sehr ergeb.{\\[\baselineskip]}\spacefill\mbox{Karl Kraus,}\pend
           \leftskip=0em{}
\pstart
           \noindent{}I Maximilianstr. 13\oindex{Mahlerstrasse@\textbf{Mahlerstraße}, \emph{Straße (K.STR)}|pw}.\pend
           \selectlanguage{ngerman}\endnumbering\briefempfaengerindex{Schnitzler, Arthur@\textsc{Schnitzler, Arthur}!zzzKraus, Karl@\emph{von Karl Kraus}!1892-12-311@{31. 12. 1892}|)be}\mylabel{L00150h}  \normalsize

\doendnotes{C}
\bigskip
\vfill

\clearpage

\footnotesize

\lohead{\textsc{register}}

% Definiere theindex-Environment komplett neu ohne reledmac
\makeatletter
\renewenvironment{theindex}{%
  \section*{\indexname}%
  \setlength{\parindent}{0pt}%
  \setlength{\parskip}{0pt plus 0.3pt}%
  \let\item\@idxitem
}{%
  \clearpage
}
\makeatother

\IfFileExists{\jobname-pw.ind}{\input{\jobname-pw.ind}}{}

\end{document}

      