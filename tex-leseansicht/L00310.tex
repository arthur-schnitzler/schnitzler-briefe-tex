%% latex-leseansicht-vorspann.tex
%% Vorspann für die Leseansicht.
%% Lädt die gemeinsame Datei latex-vorspann.tex mit nicht gesetztem Schalter.

\newif\ifkorrekturansicht
\korrekturansichtfalse

\input{../tex-inputs/latex-vorspann}


               \section[Hermann Bahr an Arthur Schnitzler, 2. 4. 1894]{ Hermann Bahr an Arthur Schnitzler, 2. 4. 1894}\nopagebreak\mylabel{v}\rehead{ }\begin{ledgroupsized}[t]{13cm}\normalsize\beginnumbering\briefempfaengerindex{Schnitzler, Arthur@\textsc{Schnitzler, Arthur}!zzzBahr, Hermann@\emph{von Hermann Bahr}!1894-04-022@{2. 4. 1894}|(be} \toendnotes[C]{\smallbreak\pagebreak[2]} \Standort{TMW, HS AM 39930 Ba.}
\physDesc{maschinelle Abschrift
\newline{}Schreibmaschine\newline{}Zusatz: Original nicht nachweisbar; es wurde von Heinrich Schnitzler\pwindex{Schnitzler, Heinrich 09.08.1902 – 12.07.1982@\textsc{Schnitzler, Heinrich} (09.08.1902 – 12.07.1982), \emph{Regisseur, Schauspieler}|pw} am
                                    22. 8. 1937 dem Wolf-Museum\oindex{Wolf-Museum@\textbf{Wolf-Museum}|pw} in Eisenstadt\oindex{Eisenstadt@\textbf{Eisenstadt}|pw} geschenkt. Sándor
                                    Wolf\pwindex{Wolf, Sándor 21.12.1871 – 02.01.1946@\textsc{Wolf, Sándor} (21.12.1871 – 02.01.1946), \emph{Mäzen, Kaufmann, Sammler}|pw} emigrierte 1938 nach Israel\oindex{Israel@\textbf{Israel}|pw}, wohin seine Bibliothek nachzuholen ihm
                                 möglicherweise gelang. Nach seinem Tod im Jahr 1946
                                 ließ seine Schwester Frieda
                                    Löwy\pwindex{Loewy, Frieda 1877-01-21 – 1963-07-11@\textsc{Löwy, Frieda} (1877-01-21 – 1963-07-11)|pw} einen Teil der Sammlung 1958 in Luzern\oindex{Luzern@\textbf{Luzern}|pw} versteigern, der Brief
                                 dürfte sich nicht darunter befunden haben. }\buchAbdrucke{\weitereDrucke{Hermann Bahr, Arthur Schnitzler: \emph{Briefwechsel, Aufzeichnungen, Dokumente (1891–1931)}. Hg. Kurt Ifkovits und Martin Anton Müller. Göttingen: \emph{Wallstein} 2018, S. 68.} }\pstart
           \raggedleft{}{\pb}2. 4. 1894\pend
           \pstart{}Lieber Schnitzler,\pend\pstart
           ich habe mir die Geschichte mit dem Bicycle doch anders überlegt – lieber nicht. Der
               Gedanke, da umständlich zu lernen und mich mit einem fremden Instrument zu peinigen,
               macht mich nur nervöse. Sei deswegen nicht böse \pend
           \pstart
           Deinem treuen{\\[\baselineskip]}\spacefill\mbox{Bahr}\pend
           \leftskip=0em{}\endnumbering\briefempfaengerindex{Schnitzler, Arthur@\textsc{Schnitzler, Arthur}!zzzBahr, Hermann@\emph{von Hermann Bahr}!1894-04-022@{2. 4. 1894}|)be}\mylabel{h}\end{ledgroupsized}  \newcommand{\dateiname}{L00310}\newcommand{\titel}{Hermann Bahr an Arthur Schnitzler, 2. 4. 1894}\newcommand{\editorInnen}{ Kurt Ifkovits,  Martin Anton Müller}%% latex-leseansicht-abspann.tex
%% Abspann für die Leseansicht.
%% Der Schalter \ifkorrekturansicht ist bereits durch den Vorspann gesetzt.

%% latex-abspann.tex
%% Gemeinsamer Abspann für Korrekturansicht und Leseansicht.
%% Setzt den Schalter \ifkorrekturansicht voraus (gesetzt in den
%% einbindenden Dateien latex-korrekturansicht-abspann.tex bzw.
%% latex-leseansicht-abspann.tex).
%% ---------------------------------------------------------------

\normalsize

% Das esempio-Environment wird nur in der Leseansicht benötigt
\ifkorrekturansicht\else
\newenvironment{esempio}[3]%
{
    \vspace{1.5ex}
    \rlap{\underline{#1}}
    \par
    \setlength{\parindent}{0cm}
    \nopagebreak
    \leftskip=#2cm
    \rightskip=#3cm
}
{
    \par
}
\fi

\doendnotes{C}
\bigskip
\vfill

\clearpage

\footnotesize

\ifkorrekturansicht
  \lohead{\textsc{register}}
\fi

% theindex-Environment neu definieren ohne reledmac
\makeatletter
\renewenvironment{theindex}{%
  \ifkorrekturansicht
    \section*{\indexname}%
  \else
    \subsubsection*{Index der erwähnten Entitäten}%
  \fi
  \setlength{\parindent}{0pt}%
  \setlength{\parskip}{0pt plus 0.3pt}%
  \let\item\@idxitem
}{%
  \ifkorrekturansicht\clearpage\fi
}
\makeatother

\IfFileExists{\jobname-pw.ind}{\input{\jobname-pw.ind}}{}

% Quellenangabe nur in der Leseansicht
\ifkorrekturansicht\else
% Fallback-Definitionen, falls die .tex-Datei \titel etc. nicht gesetzt hat
\providecommand{\titel}{}
\providecommand{\editorInnen}{}
\providecommand{\dateiname}{\jobname}

\vspace{3cm}

\vfill

\footnotesize
\textsc{Quelle}: \titel. Herausgegeben von {\editorInnen}. In: \emph{Arthur Schnitzler: Briefwechsel mit Autorinnen und Autoren}.
 Digitale Edition, https://schnitzler-briefe.acdh.oeaw.ac.at/{\dateiname}.html (Stand \today)
\fi

\end{document}


      