%% latex-korrekturansicht-vorspann.tex
%% Vorspann für die Korrekturansicht.
%% Lädt die gemeinsame Datei latex-vorspann.tex mit gesetztem Schalter.

\newif\ifkorrekturansicht
\korrekturansichttrue

\input{../tex-inputs/latex-vorspann}


\section[Hermann Bahr an Arthur Schnitzler, 2. 4. 1894]{L00310 Hermann Bahr an Arthur Schnitzler, 2. 4. 1894}
\nopagebreak\mylabel{L00310v}
\rehead{ }\normalsize\beginnumbering\briefempfaengerindex{Schnitzler, Arthur@\textsc{Schnitzler, Arthur}!zzzBahr, Hermann@\emph{von Hermann Bahr}!1894-04-023@{2. 4. 1894}|(be}
\toendnotes[C]{\smallbreak\pagebreak[2]}\Standort{TMW, HS AM 39930 Ba.}
\physDesc{Brief, maschinenschriftliche Abschrift1 Blatt, 1 Seite, 262 Zeichen
\newline{}Schreibmaschine
\newline{}Zusatz: Original nicht nachweisbar; es wurde von Heinrich Schnitzler\pwindex{Schnitzler, Heinrich 09.08.1902 – 12.07.1982@\textsc{Schnitzler, Heinrich} (09.08.1902 – 12.07.1982), \emph{Regisseur/Regisseurin, Schauspieler/Schauspielerin}|pw} am
                                    22. 8. 1937 dem Wolf-Museum\oindex{Wolf-Museum@\textbf{Wolf-Museum}, \emph{Museum (K.MUS)}|pw} in Eisenstadt\oindex{Eisenstadt@\textbf{Eisenstadt}, \emph{P.PPLA}|pw} geschenkt. Sándor Wolf\pwindex{Wolf, Sándor 21.12.1871 – 02.01.1946@\textsc{Wolf, Sándor} (21.12.1871 – 02.01.1946), \emph{Mäzen/Mäzenin, Kaufmann/Kauffrau, Sammler/Sammlerin}|pw} emigrierte 1938 nach Israel\oindex{Israel@\textbf{Israel}, \emph{A.PCLI}|pw}, wohin seine Bibliothek
                                 nachzuholen ihm möglicherweise gelang. Nach seinem Tod im Jahr
                                    1946 ließ seine Schwester Frieda Löwy\pwindex{Loewy, Frieda 1877-01-21 – 1963-07-11@\textsc{Löwy, Frieda} (1877-01-21 – 1963-07-11)|pw} einen Teil der Sammlung
                                    1958 in Luzern\oindex{Luzern@\textbf{Luzern}, \emph{P.PPLA}|pw}
                                 versteigern, der Brief dürfte sich nicht darunter befunden
                                 haben. }
\buchAbdrucke{\weitereDrucke{Hermann Bahr, Arthur Schnitzler: \emph{Briefwechsel, Aufzeichnungen, Dokumente (1891–1931)}. Göttingen: \emph{Wallstein} 2018, S. 68.} }
\pstart
           \raggedleft{}{\pb}2. 4. 1894\pend
           
\pstart{}Lieber Schnitzler,\pend\vspace{0.5em}
\pstart
           ich habe mir die Geschichte mit dem Bicycle doch anders überlegt – lieber nicht. Der
               Gedanke, da umständlich zu lernen und mich mit einem fremden Instrument zu peinigen,
               macht mich nur nervöse. Sei deswegen nicht böse \pend
           
\pstart
           Deinem treuen{\\[\baselineskip]}\spacefill\mbox{Bahr}\pend
           \leftskip=0em{}\selectlanguage{ngerman}\endnumbering\briefempfaengerindex{Schnitzler, Arthur@\textsc{Schnitzler, Arthur}!zzzBahr, Hermann@\emph{von Hermann Bahr}!1894-04-023@{2. 4. 1894}|)be}\mylabel{L00310h}  \normalsize

\doendnotes{C}
\bigskip
\vfill

\clearpage

\footnotesize

\lohead{\textsc{register}}

% Definiere theindex-Environment komplett neu ohne reledmac
\makeatletter
\renewenvironment{theindex}{%
  \section*{\indexname}%
  \setlength{\parindent}{0pt}%
  \setlength{\parskip}{0pt plus 0.3pt}%
  \let\item\@idxitem
}{%
  \clearpage
}
\makeatother

\IfFileExists{\jobname-pw.ind}{\input{\jobname-pw.ind}}{}

\end{document}

      