%% latex-leseansicht-vorspann.tex
%% Vorspann für die Leseansicht.
%% Lädt die gemeinsame Datei latex-vorspann.tex mit nicht gesetztem Schalter.

\newif\ifkorrekturansicht
\korrekturansichtfalse

\input{../tex-inputs/latex-vorspann}


\section[Hermann Bahr an Arthur Schnitzler, 2. 4. 1894]{L00310 Hermann Bahr an Arthur Schnitzler, 2. 4. 1894}
\nopagebreak\mylabel{L00310v}
\rehead{ }\normalsize\beginnumbering\briefempfaengerindex{Schnitzler, Arthur@\textsc{Schnitzler, Arthur}!zzzBahr, Hermann@\emph{von Hermann Bahr}!1894-04-023@{2. 4. 1894}|(be}
\toendnotes[C]{\smallbreak\pagebreak[2]}
\correspDesc{Versand  durch Hermann Bahr am 2. 4. 1894 in Wien
\newline{}Erhalt  durch Arthur Schnitzler im Zeitraum [2. 4. 1894
                  – 6. 4. 1894?] in Wien}\toendnotes[C]{\smallbreak}
\Standort{TMW, HS AM 39930 Ba.}
\physDesc{Brief, maschinenschriftliche Abschrift, 1 Blatt, 1 Seite, 262 Zeichen
\newline{}Schreibmaschine
\newline{}Zusatz: Original nicht nachweisbar; es wurde von Heinrich Schnitzler\pwindex{Schnitzler, Heinrich 9.\,8.\,1902 Hinterbrühl – 12.\,7.\,1982 Wien@\textsc{Schnitzler, Heinrich} (9.\,8.\,1902 Hinterbrühl – 12.\,7.\,1982 Wien), \emph{Regisseur, Schauspieler}|pw} am
                                 22. 8. 1937 dem Wolf-Museum\orgindex{Landesmuseum Burgenland@Landesmuseum Burgenland|pw} in Eisenstadt\oindex{Eisenstadt@\textbf{Eisenstadt}|pw} geschenkt. Sándor Wolf\pwindex{Wolf, Sándor 21.\,12.\,1871 Eisenstadt – 2.\,1.\,1946 Haifa@\textsc{Wolf, Sándor} (21.\,12.\,1871 Eisenstadt – 2.\,1.\,1946 Haifa), \emph{Mäzen, Kaufmann, Sammler}|pw} emigrierte 1938 nach Israel\oindex{Israel@\textbf{Israel}|pw}, wohin seine Bibliothek
                                 nachzuholen ihm möglicherweise gelang. Nach seinem Tod im Jahr
                                    1946 ließ seine Schwester Frieda Löwy\pwindex{Löwy, Frieda 21.\,1.\,1877 Eisenstadt – 11.\,7.\,1963 Haifa@\textsc{Löwy, Frieda} (21.\,1.\,1877 Eisenstadt – 11.\,7.\,1963 Haifa)|pw} einen Teil der Sammlung
                                    1958 in Luzern\oindex{Luzern@\textbf{Luzern}|pw}
                                 versteigern, der Brief dürfte sich nicht darunter befunden
                                 haben. }
\buchAbdrucke{\weitereDrucke{Hermann Bahr, Arthur Schnitzler: \emph{Briefwechsel, Aufzeichnungen, Dokumente (1891–1931)}. Herausgegeben von Kurt Ifkovits und Martin Anton Müller. Göttingen: \emph{Wallstein} 2018, S. 68.} }
\pstart
           \raggedleft{}{\pb}2. 4. 1894\pend
           
\pstart{}Lieber Schnitzler,\pend\vspace{0.5em}
\pstart
           ich habe mir die Geschichte mit dem Bicycle doch anders überlegt – lieber nicht. Der
               Gedanke, da umständlich zu lernen und mich mit einem fremden Instrument zu peinigen,
               macht mich nur nervöse. Sei deswegen nicht böse\pend
           
\pstart
           Deinem treuen{\\[\baselineskip]}\spacefill\mbox{Bahr}\pend
           \leftskip=0em{}\selectlanguage{ngerman}\endnumbering\briefempfaengerindex{Schnitzler, Arthur@\textsc{Schnitzler, Arthur}!zzzBahr, Hermann@\emph{von Hermann Bahr}!1894-04-023@{2. 4. 1894}|)be}\mylabel{L00310h}  \newcommand{\dateiname}{L00310}\newcommand{\titel}{Hermann Bahr an Arthur Schnitzler, 2. 4. 1894}\newcommand{\editorInnen}{Herausgegeben von Martin Anton Müller}%% latex-leseansicht-abspann.tex
%% Abspann für die Leseansicht.
%% Der Schalter \ifkorrekturansicht ist bereits durch den Vorspann gesetzt.

%% latex-abspann.tex
%% Gemeinsamer Abspann für Korrekturansicht und Leseansicht.
%% Setzt den Schalter \ifkorrekturansicht voraus (gesetzt in den
%% einbindenden Dateien latex-korrekturansicht-abspann.tex bzw.
%% latex-leseansicht-abspann.tex).
%% ---------------------------------------------------------------

\normalsize

% Das esempio-Environment wird nur in der Leseansicht benötigt
\ifkorrekturansicht\else
\newenvironment{esempio}[3]%
{
    \vspace{1.5ex}
    \rlap{\underline{#1}}
    \par
    \setlength{\parindent}{0cm}
    \nopagebreak
    \leftskip=#2cm
    \rightskip=#3cm
}
{
    \par
}
\fi

\doendnotes{C}
\bigskip
\vfill

\clearpage

\footnotesize

\ifkorrekturansicht
  \lohead{\textsc{register}}
\fi

% theindex-Environment neu definieren ohne reledmac
\makeatletter
\renewenvironment{theindex}{%
  \ifkorrekturansicht
    \section*{\indexname}%
  \else
    \subsubsection*{Index der erwähnten Entitäten}%
  \fi
  \setlength{\parindent}{0pt}%
  \setlength{\parskip}{0pt plus 0.3pt}%
  \let\item\@idxitem
}{%
  \ifkorrekturansicht\clearpage\fi
}
\makeatother

\IfFileExists{\jobname-pw.ind}{\input{\jobname-pw.ind}}{}

% Quellenangabe nur in der Leseansicht
\ifkorrekturansicht\else
% Fallback-Definitionen, falls die .tex-Datei \titel etc. nicht gesetzt hat
\providecommand{\titel}{}
\providecommand{\editorInnen}{}
\providecommand{\dateiname}{\jobname}

\vspace{3cm}

\vfill

\footnotesize
\textsc{Quelle}: \titel. Herausgegeben von {\editorInnen}. In: \emph{Arthur Schnitzler: Briefwechsel mit Autorinnen und Autoren}.
 Digitale Edition, https://schnitzler-briefe.acdh.oeaw.ac.at/{\dateiname}.html (Stand \today)
\fi

\end{document}


