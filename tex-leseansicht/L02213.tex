%% latex-korrekturansicht-vorspann.tex
%% Vorspann für die Korrekturansicht.
%% Lädt die gemeinsame Datei latex-vorspann.tex mit gesetztem Schalter.

\newif\ifkorrekturansicht
\korrekturansichttrue

\input{../tex-inputs/latex-vorspann}


\section[Arthur Schnitzler an Robert Adam, 11. 7. 1915]{L02213 Arthur Schnitzler an Robert Adam, 11. 7. 1915}
\nopagebreak\mylabel{L02213v}
\rehead{ }\normalsize\beginnumbering\briefempfaengerindex{Adam, Robert@\textsc{Adam, Robert}!zzzSchnitzler, Arthur@\emph{von Arthur Schnitzler}!1915-07-111@{11. 7. 1915}|(be}
\toendnotes[C]{\smallbreak\pagebreak[2]}\Standort{DLA, 96.34.1/14.}
\physDesc{Briefkarte, , Umschlag, 980 Zeichen
\newline{}Handschrift: schwarze Tinte, lateinische Kurrent
\newline{}Versand: Stempel: »\nobreak{}Wien, 1\textcolor{gray}{2}. VII. 15\nobreak{}«.  }\toendnotes[C]{\smallbreak}\pstart{}{\pb}\textcolor{gray}{\textbf{Dr. Arthur Schnitzler}}\pend{}\pstart{}\textcolor{gray}{\textbf{Wien XVIII. Sternwartestrasse 71\oindex{Sternwartestrasse 71@\textbf{Sternwartestraße 71}, \emph{Wohngebäude (K.WHS)}|pw}}}\pend{}{\bigskip}\pstart{}{\pb}Herrn Dr. Robert Adam Pollak,\pend{}\pstart{}Bezirksrichter in Zistersdorf\oindex{Zistersdorf@\textbf{Zistersdorf}, \emph{A.ADM3}|pw}\pend{}\pstart{}N. Oe.\oindex{Niederoesterreich@\textbf{Niederösterreich}, \emph{A.ADM1}|pw} – \pend{}{\bigskip}\vspace{1em}
\pstart
           {\pb}\textcolor{gray}{\textbf{Dr. Arthur Schnitzler}}\hfill 11/7 1915\pend
           
\pstart
           \textcolor{gray}{\textbf{Wien XVIII. Sternwartestrasse 71\oindex{Sternwartestrasse 71@\textbf{Sternwartestraße 71}, \emph{Wohngebäude (K.WHS)}|pw}}}\pend
           \vspace{0.5em}
\pstart
           Verehrter Herr Doctor, erst gestern Abend bin ich dazu geko{\geminationm}en Ihre Komoedie\pwindex{Gesellschaft [Eine Gaunerkomoedie]@\emph{Gesellschaft [Eine Gaunerkomödie]}|pwv} zu lesen – in einem Zug, da sie mich amusiert hat;
               technisch ist sie auch nicht übel – aber im ganzen ist es dann eine etwas grobe und
               in ihrer \textcolor{gray}{Accentu}iertheit unwahrscheinliche und recht willkürlich
               wirkende Sache, mit der nicht übermäßig \introOben{}viel\introOben{} dichterische
               Ehren aufzuheben sind. I{\geminationm}erhin ist sie spielbar und ich
               denke, Residenzbühne\oindex{Kammerspiele Wien@\textbf{Kammerspiele Wien}, \emph{Theater (K.THE)}|pw} oder Neue Bühne\oindex{Neue Wiener Buehne@\textbf{Neue Wiener Bühne}, \emph{Theater (K.THE)}|pw} würden sich gegen den Versuch nicht wehren. Daß Sie
               jede einzelne Figur persönlich kennen, {\pb}will ich gerne
               glauben – und jede einzelne wirkte am Ende, in irgend ein andres Stück gestellt,
               lebendig wirken; – so auf einen Fleck zusa{\geminationm}engebracht,
               in theatralische Beziehun\textcolor{gray}{ge}n \substVorne{}\textsuperscript{auf}\substDazwischen{}zu\substHinten{}einander, zweifelt man gelegentlich auch an ihrer Lebenswahrheit. De{\geminationn} nichts ist rachsüchtiger als die Kunst – bis zur
               Ungerechtigkeit! –\pend
           
\pstart
           Seien Sie herzlich gegrüßt von Ihrem Sie sehr hochschätzenden{\\[\baselineskip]}\spacefill\mbox{Arthur Schnitzler}\pend
           \leftskip=0em{}\selectlanguage{ngerman}\endnumbering\briefempfaengerindex{Adam, Robert@\textsc{Adam, Robert}!zzzSchnitzler, Arthur@\emph{von Arthur Schnitzler}!1915-07-111@{11. 7. 1915}|)be}\mylabel{L02213h}  \normalsize

\doendnotes{C}
\bigskip
\vfill

\clearpage

\footnotesize

\lohead{\textsc{register}}

% Definiere theindex-Environment komplett neu ohne reledmac
\makeatletter
\renewenvironment{theindex}{%
  \section*{\indexname}%
  \setlength{\parindent}{0pt}%
  \setlength{\parskip}{0pt plus 0.3pt}%
  \let\item\@idxitem
}{%
  \clearpage
}
\makeatother

\IfFileExists{\jobname-pw.ind}{\input{\jobname-pw.ind}}{}

\end{document}

      