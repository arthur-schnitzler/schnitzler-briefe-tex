%% latex-leseansicht-vorspann.tex
%% Vorspann für die Leseansicht.
%% Lädt die gemeinsame Datei latex-vorspann.tex mit nicht gesetztem Schalter.

\newif\ifkorrekturansicht
\korrekturansichtfalse

\input{../tex-inputs/latex-vorspann}


         
         \renewcommand{\erwaehntePersonen}{Personen: Robert Adam}
         \renewcommand{\erwaehnteOrte}{Orte: Kammerspiele Wien, Neue Wiener Bühne, Niederösterreich, Sternwartestraße, Wien, Zistersdorf}
         \renewcommand{\erwaehnteWerke}{Werke: Gesellschaft [Eine Gaunerkomödie]}
               \section[Arthur Schnitzler an Robert Adam, 11. 7. 1915]{ Arthur Schnitzler an Robert Adam, 11. 7. 1915}\nopagebreak\mylabel{v}\rehead{ }\begin{ledgroupsized}[t]{13cm}\normalsize\beginnumbering\briefempfaengerindex{Adam, Robert@\textsc{Adam, Robert}!zzzSchnitzler, Arthur@\emph{von Arthur Schnitzler}!1915-07-111@{11. 7. 1915}|(be} \toendnotes[C]{\smallbreak\pagebreak[2]} \Standort{DLA, 96.34.1/14.}
\physDesc{Briefkarte, , Umschlag, 980 Zeichen
\newline{}Handschrift: schwarze Tinte, lateinische Kurrent
\newline{}Versand: Stempel: »\nobreak{}Wien, 1\textcolor{gray}{2}. VII. 15\nobreak{}«.  }\toendnotes[C]{\smallbreak}\pstart{}{\pb}\textcolor{gray}{\textbf{Dr. Arthur Schnitzler}}\pend{}\pstart{}\textcolor{gray}{\textbf{Wien XVIII. Sternwartestrasse 71\oindex{XXXX Ortsangabe fehlt|pw}}}\pend{}{\bigskip}\pstart{}{\pb}Herrn Dr. Robert Adam Pollak,\pend{}\pstart{}Bezirksrichter in Zistersdorf\oindex{Zistersdorf@\textbf{Zistersdorf}|pw}\pend{}\pstart{}N. Oe.\oindex{Niederoesterreich@\textbf{Niederösterreich}|pw} – \pend{}{\bigskip}\pstart
           \noindent{}{\pb}\textcolor{gray}{\textbf{Dr. Arthur Schnitzler}}\hfill 11/7 1915\pend
           \pstart
           \textcolor{gray}{\textbf{Wien XVIII. Sternwartestrasse 71\oindex{XXXX Ortsangabe fehlt|pw}}}\pend
           \pstart
           Verehrter Herr Doctor, erst gestern Abend bin ich dazu geko{\geminationm}en Ihre Komoedie\pwindex{Adam, Robert 20.04.1877 – 16.10.1961@\textsc{Adam, Robert} (20.04.1877 – 16.10.1961), \emph{Schriftsteller, Richter}!Gesellschaft [Eine Gaunerkomoedie]None@\strich\emph{Gesellschaft [Eine Gaunerkomödie]} {[}None{]}|pwv} zu lesen – in einem Zug, da sie mich amusiert hat;
               technisch ist sie auch nicht übel – aber im ganzen ist es dann eine etwas grobe und
               in ihrer \textcolor{gray}{Accentu}iertheit unwahrscheinliche und recht willkürlich
               wirkende Sache, mit der nicht übermäßig \introOben{}viel\introOben{} dichterische
               Ehren aufzuheben sind. I{\geminationm}erhin ist sie spielbar und ich
               denke, Residenzbühne\oindex{Kammerspiele Wien@\textbf{Kammerspiele Wien}|pw} oder Neue Bühne\oindex{Neue Wiener Buehne@\textbf{Neue Wiener Bühne}|pw} würden sich gegen den Versuch nicht wehren. Daß Sie
               jede einzelne Figur persönlich kennen, {\pb}will ich gerne
               glauben – und jede einzelne wirkte am Ende, in irgend ein andres Stück gestellt,
               lebendig wirken; – so auf einen Fleck zusa{\geminationm}engebracht,
               in theatralische Beziehun\textcolor{gray}{ge}n \substVorne{}\textsuperscript{auf}\substDazwischen{}zu\substHinten{}einander, zweifelt man gelegentlich auch an ihrer Lebenswahrheit. De{\geminationn} nichts ist rachsüchtiger als die Kunst – bis zur
               Ungerechtigkeit! –\pend
           \pstart
           Seien Sie herzlich gegrüßt von Ihrem Sie sehr hochschätzenden{\\[\baselineskip]}\spacefill\mbox{Arthur Schnitzler}\pend
           \leftskip=0em{}
         
         \endnumbering\mylabel{h}\end{ledgroupsized}  \newcommand{\dateiname}{L02213}\newcommand{\titel}{Arthur Schnitzler an Robert Adam, 11. 7. 1915}\newcommand{\editorInnen}{Martin Anton Müller und Gerd-Hermann Susen}%% latex-leseansicht-abspann.tex
%% Abspann für die Leseansicht.
%% Der Schalter \ifkorrekturansicht ist bereits durch den Vorspann gesetzt.

%% latex-abspann.tex
%% Gemeinsamer Abspann für Korrekturansicht und Leseansicht.
%% Setzt den Schalter \ifkorrekturansicht voraus (gesetzt in den
%% einbindenden Dateien latex-korrekturansicht-abspann.tex bzw.
%% latex-leseansicht-abspann.tex).
%% ---------------------------------------------------------------

\normalsize

% Das esempio-Environment wird nur in der Leseansicht benötigt
\ifkorrekturansicht\else
\newenvironment{esempio}[3]%
{
    \vspace{1.5ex}
    \rlap{\underline{#1}}
    \par
    \setlength{\parindent}{0cm}
    \nopagebreak
    \leftskip=#2cm
    \rightskip=#3cm
}
{
    \par
}
\fi

\doendnotes{C}
\bigskip
\vfill

\clearpage

\footnotesize

\ifkorrekturansicht
  \lohead{\textsc{register}}
\fi

% theindex-Environment neu definieren ohne reledmac
\makeatletter
\renewenvironment{theindex}{%
  \ifkorrekturansicht
    \section*{\indexname}%
  \else
    \subsubsection*{Index der erwähnten Entitäten}%
  \fi
  \setlength{\parindent}{0pt}%
  \setlength{\parskip}{0pt plus 0.3pt}%
  \let\item\@idxitem
}{%
  \ifkorrekturansicht\clearpage\fi
}
\makeatother

\IfFileExists{\jobname-pw.ind}{\input{\jobname-pw.ind}}{}

% Quellenangabe nur in der Leseansicht
\ifkorrekturansicht\else
% Fallback-Definitionen, falls die .tex-Datei \titel etc. nicht gesetzt hat
\providecommand{\titel}{}
\providecommand{\editorInnen}{}
\providecommand{\dateiname}{\jobname}

\vspace{3cm}

\vfill

\footnotesize
\textsc{Quelle}: \titel. Herausgegeben von {\editorInnen}. In: \emph{Arthur Schnitzler: Briefwechsel mit Autorinnen und Autoren}.
 Digitale Edition, https://schnitzler-briefe.acdh.oeaw.ac.at/{\dateiname}.html (Stand \today)
\fi

\end{document}


      