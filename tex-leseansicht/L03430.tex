%% latex-leseansicht-vorspann.tex
%% Vorspann für die Leseansicht.
%% Lädt die gemeinsame Datei latex-vorspann.tex mit nicht gesetztem Schalter.

\newif\ifkorrekturansicht
\korrekturansichtfalse

\input{../tex-inputs/latex-vorspann}

\begin{center}
            \textcolor{red}{ENTWURF, NICHT FERTIG KORRIGIERT}
                      \end{center}
            
         
         \renewcommand{\erwaehntePersonen}{Personen: Ludwig Bauer, Ernst Egon Friedegg, Julius von Gans-Ludassy, Max Liebermann, Anna Katharina Rehmann, Ottilie Salten, Paul Salten, Ignaz Salzmann, Agathe Salzmann, Marie Salzmann, Katharina Salzmann, Philipp Salzmann, Olga Schnitzler, Heinrich Schnitzler, Richard Strauss}
         \renewcommand{\erwaehnteInstitutionen}{Institutionen: Die Zeit, Morgen. Wochenschrift für deutsche Kultur}
         \renewcommand{\erwaehnteOrte}{Orte: Bansin, Berlin, Cambridge, Cherbourg-Octeville, Deutschland, Dornbach, England, Kopenhagen, London, Marienlyst, Nordsee, Plymouth, Preußen, Seestraße, Southampton, Stratford-upon-Avon, Wien, Österreich}
         \renewcommand{\erwaehnteWerke}{Werke: Morgen. Wochenschrift für deutsche Kultur}
               \section[ Felix Salten an Arthur Schnitzler, 6. 7. 1906]{ Felix Salten an Arthur Schnitzler, 6. 7. 1906}\nopagebreak\mylabel{v}\rehead{ }\begin{ledgroupsized}[t]{13cm}\normalsize\beginnumbering \toendnotes[C]{\smallbreak\pagebreak[2]} \Standort{CUL, Schnitzler, B 89, B 1.}
\physDesc{Brief, 1 Blatt, 4 Seiten, 4140 Zeichen
\newline{}Handschrift: schwarze Tinte, lateinische Kurrent
\newline{}Schnitzler: mit rotem Buntstift fünf Unterstreichungen 
\newline{}Ordnung: mit Bleistift von unbekannter Hand nummeriert: »221« }\toendnotes[C]{\smallbreak}\pstart
           \raggedleft{}{\pb}Berlin\oindex{Berlin@\textbf{Berlin}|pw}, 6. 7. 06.\pend
           \pstart
           Lieber, wie Schade, dass Sie gerade jetzt \label{K_L03430-1v}\edtext{durch Berlin\oindex{Berlin@\textbf{Berlin}|pw} kamen}{\lemma{\textnormal{\emph{durch Berlin kamen}}}\Cendnote{\textnormal{auf dem Weg nach Marienlyst\oindex{Marienlyst@\textbf{Marienlyst}|pwk}, siehe A. S.: \emph{Tagebuch}, 26. 6. 1906}}}\label{K_L03430-1h}, während meiner Abwesenheit. Man hätte vielleicht doch eine Stunde gehabt, um
               sich auszusprechen. Schreiben ist in manchen Fällen so schwer. Was ich jetzt, in der
               nächsten Zeit, beginne, liegt noch im Halbdunkel; und was ich Ihnen davon mitteile,
               ist – einstweilen – nur für Sie. In Berlin\oindex{Berlin@\textbf{Berlin}|pw} will
               ich nicht bleiben; kann es ehrlicherweise garnicht tun und spüre, dass ein Bruch in
               mein Leben käme, wollte ich versuchen\textcolor{gray}{,} mich zu zwingen. \label{K_L03430-2v}\edtext{»Die
                  Zeit\orgindex{Zeit@Die Zeit|pw}« will mich wieder haben}{\lemma{\textnormal{\emph{»Die … haben}}}\Cendnote{\textnormal{Salten\pwindex{Salten, Felix 06.09.1869 – 08.10.1945@\textsc{Salten, Felix} (06.09.1869 – 08.10.1945), \emph{Schriftsteller, Journalist}|pwk} arbeitete ab Oktober 1906 wieder für \emph{Die
                  Zeit}\orgindex{Zeit@Die Zeit|pwk}.}}}\label{K_L03430-2h}, und ich bin gerne geneigt, abzuschließen. Dabei bietet sich
               hier der Plan zu einer \label{K_L03430-3v}\edtext{Wochenschrift\pwindex{Morgen. Wochenschrift fuer deutsche Kultur1907 – 1908@\emph{Morgen. Wochenschrift für deutsche Kultur} {[}1907 – 1908{]}|pwuv}\orgindex{Morgen. Wochenschrift fuer deutsche Kultur@Morgen. Wochenschrift für deutsche Kultur|pwu}}{\lemma{\textnormal{\emph{Wochenschrift}}}\Cendnote{\textnormal{Der \emph{\emph{Morgen}\pwindex{Morgen. Wochenschrift fuer deutsche Kultur1907 – 1908@\emph{Morgen. Wochenschrift für deutsche Kultur} {[}1907 – 1908{]}|pwuk}}\orgindex{Morgen. Wochenschrift fuer deutsche Kultur@Morgen. Wochenschrift für deutsche Kultur|pwk}?}}}\label{K_L03430-3h}, die ich mit Max Liebermann\pwindex{Liebermann, Max 20.07.1847 – 08.02.1935@\textsc{Liebermann, Max} (20.07.1847 – 08.02.1935), \emph{Maler}|pw}
               und Rich. Strauß\pwindex{Strauss, Richard 11.06.1864 – 08.09.1949@\textsc{Strauss, Richard} (11.06.1864 – 08.09.1949), \emph{Theaterleiter, Komponist, Dirigent}|pw} zusammen herausgeben, und
               allein leiten soll. Ihr Bestand ist für drei Jahre garantirt. Honorarbudget, ohne
               meine Gage, nur für Mitarbeiter 1000 Mark pro Nummer. Sie soll das Blatt der
               »anständigen Leute« werden, der Besten, ganz einfach. Ein kleiner, exclusiver,
               ständiger Mitarbeiterkreis. Ich hätte ausser der Gage noch einen Besitzanteil. Jetzt
               überleg ich mir’s, ob ich die Sache nicht von Wien\oindex{Wien@\textbf{Wien}|pw}
               aus machen kann. Technisch gehts ganz gut. Die Schwierigkeiten, die sich freilich
               ergeben, würden reichlich durch manche Vorteile, {\pb}die sich dran knüpfen,
               aufgewogen. Ich könnte z. B. die Berlin\oindex{Berlin@\textbf{Berlin}|pw}er u. Wien\oindex{Wien@\textbf{Wien}|pw}er Theater zusammen überschauen und besprechen.
               Würde bei allen wichtigen Aufführungen (an die Premiere bin ich ja nicht gebunden) in
                  Berlin\oindex{Berlin@\textbf{Berlin}|pw} sein. Könnte deutsch\oindex{Deutschland@\textbf{Deutschland}|pwv}e und österreich\oindex{Oesterreich@\textbf{Österreich}|pwv}ische Kultur- und
               Gesellschaftskritik zusammen treiben, was dem Blatte\pwindex{Morgen. Wochenschrift fuer deutsche Kultur1907 – 1908@\emph{Morgen. Wochenschrift für deutsche Kultur} {[}1907 – 1908{]}|pwuv}\orgindex{Morgen. Wochenschrift fuer deutsche Kultur@Morgen. Wochenschrift für deutsche Kultur|pwu} ebenso wie meiner Stellung etwas ganz Besonderes gäbe. Und wenn – binnen Kurzem
               – ein Thronwechsel in Österreich\oindex{Oesterreich@\textbf{Österreich}|pw} alles
               Interesse erregt, wär’s für eine solche Wochenschrift eine ganz einzige Conjunctur.
               Ganz abgesehen davon, dass ich, als in Wien\oindex{Wien@\textbf{Wien}|pw}
               lebend, nicht mehr unter der Fuchtel der politischen Polizei in Preussen\oindex{Preussen@\textbf{Preußen}|pw}, die ärger ist als man glaubt, und nicht mehr unter
               der Ausweisungsgefahr leben müßte.\pend
           \pstart
           Glauben Sie, dass mein Wiedereintritt in die die »Zeit\orgindex{Zeit@Die Zeit|pw}« für mich gut wäre? Dass man mich dort braucht, sehe ich, und dass die
                  »Zeit\orgindex{Zeit@Die Zeit|pw}« jetzt ihre literarische Stimme
               eingebüßt hat, kann ich wol, ohne Ihrem Freund \label{K_L03430-4v}\edtext{Bauer\pwindex{Bauer, Ludwig 05.09.1876 – 01.02.1935@\textsc{Bauer, Ludwig} (05.09.1876 – 01.02.1935), \emph{Schriftsteller, Journalist}|pw}}{\lemma{\textnormal{\emph{Bauer}}}\Cendnote{\textnormal{Salten\pwindex{Salten, Felix 06.09.1869 – 08.10.1945@\textsc{Salten, Felix} (06.09.1869 – 08.10.1945), \emph{Schriftsteller, Journalist}|pwk}s Nachfolger, vgl. A. S.: \emph{Tagebuch}, 15. 2. 1906}}}\label{K_L03430-4h} allzu unrecht zu thun, sagen.\pend
           \pstart
           Von sonstigen Dingen: dass Herr Friedegg\pwindex{Friedegg, Ernst Egon 1883-04-29 – Juli 1935@\textsc{Friedegg, Ernst Egon} (1883-04-29 – Juli 1935), \emph{Schriftsteller, Journalist}|pw}
               knapp vor der Verhandlung eine umfassende Ehrenerklärung abgegeben hat. Dass der
                  \label{K_L03430-5v}\edtext{Ludassy\pwindex{Gans-Ludassy, Julius von 13.04.1858 – 30.09.1922@\textsc{Gans-Ludassy, Julius von} (13.04.1858 – 30.09.1922), \emph{Schriftsteller, Journalist, Herausgeber}|pw}-Prozess}{\lemma{\textnormal{\emph{Ludassy-Prozess}}}\Cendnote{\textnormal{siehe Felix Salten an Arthur Schnitzler, 9. 3. 1906}}}\label{K_L03430-5h} vertagt ist. Dass mein Bruder\pwindex{Salzmann, Ignaz 1858-12-30 – 1932-08-08@\textsc{Salzmann, Ignaz} (1858-12-30 – 1932-08-08), \emph{Kaufmann}|pwv} leider weit davon entfernt ist, ein Millionär {\pb}zu sein, dass er aber freilich,
               gottseidank, ein so ahnsehnliches Geld verdient hat, dass ich – hoffentlich – für
               alle Zukunft der Sorge um ihn und um meine Familie enthoben bin. \uline{Wie} viel er besitzt, weiß ich nicht, weiß nur, dass er mit seiner Frau\pwindex{Salzmann, Agathe 26.12.1857 – 1938-10-08@\textsc{Salzmann, Agathe} (26.12.1857 – 1938-10-08)|pwv} sechs Wochen in England\oindex{England@\textbf{England}|pw} war, ihr um 20.000 Kronen Schmuck gekauft
               hat, für meine Mama\pwindex{Salzmann, Marie 1833-10-27 – 1909-12-01@\textsc{Salzmann, Marie} (1833-10-27 – 1909-12-01)|pwv} alles
               Erdenkliche tut, und meiner sel. Schwester\pwindex{Salzmann, Katharina 1865/1866 – 1883-08-07@\textsc{Salzmann, Katharina} (1865/1866 – 1883-08-07), \emph{Pflegerin}|pwv} wie meinem Papa\pwindex{Salzmann, Philipp 1831-12-24 – 1905-04-02@\textsc{Salzmann, Philipp} (1831-12-24 – 1905-04-02), \emph{Bergbauunternehmer, Projektemacher}|pwv} ein kostbares Grabmonument hat errichten laßen, dass er bei alledem
               doch weit von einer Million entfernt, und bei alledem von seinem Glück geradezu
               melancholisch geworden ist, weil der Papa\pwindex{Salzmann, Philipp 1831-12-24 – 1905-04-02@\textsc{Salzmann, Philipp} (1831-12-24 – 1905-04-02), \emph{Bergbauunternehmer, Projektemacher}|pwv} jahrelang darauf gewartet hat, und – genau \label{K_L03430-6v}\edtext{zwei Wochen zu früh starb}{\lemma{\textnormal{\emph{zwei … starb}}}\Cendnote{\textnormal{Philip Salzmann\pwindex{Salzmann, Philipp 1831-12-24 – 1905-04-02@\textsc{Salzmann, Philipp} (1831-12-24 – 1905-04-02), \emph{Bergbauunternehmer, Projektemacher}|pwk} war am 2. 4. 1905 verstorben.}}}\label{K_L03430-6h}.\pend
           \pstart
           Ich hatte im \label{K_L03430-7v}\edtext{Mai eine heftige Nierenkolik}{\lemma{\textnormal{\emph{Mai … Nierenkolik}}}\Cendnote{\textnormal{siehe Felix Salten u. a. an Arthur Schnitzler, 4. 6. 1906}}}\label{K_L03430-7h}. Zweimal an zwei aufeinanderfolgenden Tagen. Bekam zweimal Morphium,
               beidemale mit einer unreinen Spritze oder mit einer mangelhaft gekochten Lösung.
               Musste dann fünf Tage lang rasende Schmerzen leiden, und am Ende froh sein, dass
               nicht Schlimmeres geschah. Dabei weiß ich trotz zweier Ärzte nicht, ob ich den
               Nierenstein habe, oder ob es nur eine akute Sache gewesen ist.\pend
           \pstart
           Otti\pwindex{Salten, Ottilie 07.03.1868 – 22.06.1942@\textsc{Salten, Ottilie} (07.03.1868 – 22.06.1942), \emph{Schauspielerin}|pw} und die Kinder\pwindex{Rehmann, Anna Katharina 18.08.1904 – 27.03.1977@\textsc{Rehmann, Anna Katharina} (18.08.1904 – 27.03.1977), \emph{Schauspielerin, Übersetzerin}|pwv}\pwindex{Salten, Paul 11.08.1903 – 08.05.1937@\textsc{Salten, Paul} (11.08.1903 – 08.05.1937), \emph{Filmcutter}|pwv} sind wol und frisch in Bansin\oindex{Bansin@\textbf{Bansin}|pw}, dessen sonstige Gesellschaft mir als
               der Ausbund alles Grausenhaften geschildert wird. Ich gehe am 15. Juli zu ihnen. Dann wollen wir einmal, vielleicht sogar mit den Kindern\pwindex{Rehmann, Anna Katharina 18.08.1904 – 27.03.1977@\textsc{Rehmann, Anna Katharina} (18.08.1904 – 27.03.1977), \emph{Schauspielerin, Übersetzerin}|pwv}\pwindex{Salten, Paul 11.08.1903 – 08.05.1937@\textsc{Salten, Paul} (11.08.1903 – 08.05.1937), \emph{Filmcutter}|pwv}, per Schiff
                  \label{K_L03430-8v}\edtext{nach Kopenhagen\oindex{Kopenhagen@\textbf{Kopenhagen}|pw}, wo wir uns sehen {\pb}könnten}{\lemma{\textnormal{\emph{nach … könnten}}}\Cendnote{\textnormal{Sie sahen sich nicht in Kopenhagen\oindex{Kopenhagen@\textbf{Kopenhagen}|pwk}, aber am 2. 8. 1906 in Marienlyst\oindex{Marienlyst@\textbf{Marienlyst}|pwk}.}}}\label{K_L03430-8h}. An dem Ausflug an die Nordsee\oindex{Nordsee@\textbf{Nordsee}|pw} werd ich wol nicht teil nehmen. Ich will, wenn’s geht, in Bansin\oindex{Bansin@\textbf{Bansin}|pw} noch arbeiten. Die vierzehn Tage London\oindex{London@\textbf{London}|pw} – Stratford\oindex{Stratford-upon-Avon@\textbf{Stratford-upon-Avon}|pw} – Cambridge\oindex{Cambridge@\textbf{Cambridge}|pw} waren sehr schön.
               Die Seefahrt – hin nach Southampton\oindex{Southampton@\textbf{Southampton}|pw}, zurück von
                  Plymouth\oindex{Plymouth@\textbf{Plymouth}|pw} über Cherbourg\oindex{Cherbourg-Octeville@\textbf{Cherbourg-Octeville}|pw} – wundervoll. Die engl\oindex{England@\textbf{England}|pwv}ische Landschaft ist beinahe überall so
               schön wie Dornbach\oindex{Dornbach@\textbf{Dornbach}|pw}.\pend
           \pstart
           Schreiben Sie mir bis zum 14. nach Berlin\oindex{Berlin@\textbf{Berlin}|pw}. Von da ab Seebad Bansin\oindex{Bansin@\textbf{Bansin}|pw}, Seestraße 5\oindex{Seestrasse@\textbf{Seestraße}|pw}.\pend
           \pstart
           Viele herzliche Grüße Ihnen, Frau Olga\pwindex{Schnitzler, Olga 17.01.1882 – 13.01.1970@\textsc{Schnitzler, Olga} (17.01.1882 – 13.01.1970), \emph{Schauspielerin, Sängerin}|pw} und Heini\pwindex{Schnitzler, Heinrich 09.08.1902 – 12.07.1982@\textsc{Schnitzler, Heinrich} (09.08.1902 – 12.07.1982), \emph{Regisseur, Schauspieler}|pw}.\pend
           \pstart Ihr \spacefill\mbox{Salten}\pend{}
         
         \endnumbering\mylabel{h}\end{ledgroupsized}  \newcommand{\dateiname}{L03430}\newcommand{\titel}{Felix Salten an Arthur Schnitzler, 6. 7. 1906}\newcommand{\editorInnen}{Martin Anton Müller und Laura Untner}%% latex-leseansicht-abspann.tex
%% Abspann für die Leseansicht.
%% Der Schalter \ifkorrekturansicht ist bereits durch den Vorspann gesetzt.

%% latex-abspann.tex
%% Gemeinsamer Abspann für Korrekturansicht und Leseansicht.
%% Setzt den Schalter \ifkorrekturansicht voraus (gesetzt in den
%% einbindenden Dateien latex-korrekturansicht-abspann.tex bzw.
%% latex-leseansicht-abspann.tex).
%% ---------------------------------------------------------------

\normalsize

% Das esempio-Environment wird nur in der Leseansicht benötigt
\ifkorrekturansicht\else
\newenvironment{esempio}[3]%
{
    \vspace{1.5ex}
    \rlap{\underline{#1}}
    \par
    \setlength{\parindent}{0cm}
    \nopagebreak
    \leftskip=#2cm
    \rightskip=#3cm
}
{
    \par
}
\fi

\doendnotes{C}
\bigskip
\vfill

\clearpage

\footnotesize

\ifkorrekturansicht
  \lohead{\textsc{register}}
\fi

% theindex-Environment neu definieren ohne reledmac
\makeatletter
\renewenvironment{theindex}{%
  \ifkorrekturansicht
    \section*{\indexname}%
  \else
    \subsubsection*{Index der erwähnten Entitäten}%
  \fi
  \setlength{\parindent}{0pt}%
  \setlength{\parskip}{0pt plus 0.3pt}%
  \let\item\@idxitem
}{%
  \ifkorrekturansicht\clearpage\fi
}
\makeatother

\IfFileExists{\jobname-pw.ind}{\input{\jobname-pw.ind}}{}

% Quellenangabe nur in der Leseansicht
\ifkorrekturansicht\else
% Fallback-Definitionen, falls die .tex-Datei \titel etc. nicht gesetzt hat
\providecommand{\titel}{}
\providecommand{\editorInnen}{}
\providecommand{\dateiname}{\jobname}

\vspace{3cm}

\vfill

\footnotesize
\textsc{Quelle}: \titel. Herausgegeben von {\editorInnen}. In: \emph{Arthur Schnitzler: Briefwechsel mit Autorinnen und Autoren}.
 Digitale Edition, https://schnitzler-briefe.acdh.oeaw.ac.at/{\dateiname}.html (Stand \today)
\fi

\end{document}


      