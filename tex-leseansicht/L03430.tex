%% latex-leseansicht-vorspann.tex
%% Vorspann für die Leseansicht.
%% Lädt die gemeinsame Datei latex-vorspann.tex mit nicht gesetztem Schalter.

\newif\ifkorrekturansicht
\korrekturansichtfalse

\input{../tex-inputs/latex-vorspann}


\section[ Felix Salten an Arthur Schnitzler, 6. 7. 1906]{L03430 Felix Salten an Arthur Schnitzler,  6. 7. 1906}
\nopagebreak\mylabel{L03430v}
\rehead{ }\normalsize\beginnumbering\briefempfaengerindex{Schnitzler, Arthur@\textsc{Schnitzler, Arthur}!zzzSalten, Felix@\emph{von Felix Salten}!1906-07-061@{6. 7. 1906}|(be}
\toendnotes[C]{\smallbreak\pagebreak[2]}
\correspDesc{Versand  durch Felix Salten am 6. 7. 1906 in Berlin
\newline{}Erhalt  durch Arthur Schnitzler im Zeitraum [7. 7. 1906
                  – 11. 7. 1906?] in Wien}\toendnotes[C]{\smallbreak}
\Standort{CUL, Schnitzler, B 89, B 1.}
\physDesc{Brief, 1 Blatt, 4 Seiten, 4137 Zeichen
\newline{}Handschrift: schwarze Tinte, lateinische Kurrent
\newline{}Schnitzler: mit rotem Buntstift fünf Unterstreichungen 
\newline{}Ordnung: mit Bleistift von unbekannter Hand nummeriert: »221« }\toendnotes[C]{\smallbreak}
\pstart
           \raggedleft{}{\pb}Berlin\oindex{Berlin@\textbf{Berlin}, \emph{Hauptstadt}|pw}, 6. 7. 06.\pend
           \vspace{0.5em}
\pstart
           Lieber, wie Schade, dass Sie gerade jetzt \label{K_L03430-1v}\edtext{durch Berlin\oindex{Berlin@\textbf{Berlin}, \emph{Hauptstadt}|pw} kamen}{\lemma{\textnormal{\emph{durch Berlin kamen}}}\Cendnote{\textnormal{Schnitzler reiste über Berlin\oindex{Berlin@\textbf{Berlin}, \emph{Hauptstadt}|pwk} nach Marienlyst\oindex{Marienlyst@\textbf{Marienlyst}, \emph{Gut}|pwk}, siehe A. S.: \emph{Tagebuch}, 26. 6. 1906.}}}\label{K_L03430-1}, während meiner Abwesenheit. Man hätte vielleicht doch eine Stunde gehabt, um
               sich auszusprechen. Schreiben ist in manchen Fällen so schwer. Was ich jetzt, \label{K_L03430-2v}\edtext{in der
               nächsten Zeit, beginne}{\lemma{\textnormal{\emph{in … beginne}}}\Cendnote{\textnormal{Mit 12. 7. 1906 endete
                  Saltens\pwindex{Salten, Felix 6.\,9.\,1869 Budapest – 8.\,10.\,1945 Zürich@\textsc{Salten, Felix} (6.\,9.\,1869 Budapest – 8.\,10.\,1945 Zürich), \emph{Schriftsteller, Journalist, Chefredakteur}|pwk} berufliches Engagement bei der \emph{Berliner Morgenpost}\orgindex{Berliner Morgenpost@Berliner Morgenpost|pwk} und der
                  \emph{B. Z. am Mittag}\orgindex{B.Z. am Mittag@B.Z. am Mittag|pwk}. Siehe Marcel Atze: \emph{»Unser aller Feldmarschall mit der Feder«. Felix Saltens halbes Jahrhundert
                     als Journalist.} In: Marcel Atze, unter Mitarbeit von Tanja Gausterer
                  (Herausgeber): \emph{Im Schatten von Bambi. Felix Salten entdeckt die Wiener
                     Moderne. Leben und Werk}.
                  Salzburg/Wien:
                  \emph{Residenz}{ }2020, S. 260–289, hier 285.
               }}}\label{K_L03430-2}, liegt noch im Halbdunkel; und was ich Ihnen davon mitteile,
               ist – einstweilen – nur für Sie. In Berlin\oindex{Berlin@\textbf{Berlin}, \emph{Hauptstadt}|pw} will
               ich nicht bleiben; kann es ehrlicherweise garnicht tun und spüre, dass ein Bruch in
               mein Leben käme, wollte ich versuchen\textcolor{gray}{,} mich zu zwingen. \label{K_L03430-3v}\edtext{»Die
                  Zeit\orgindex{Zeit@Die Zeit|pw}« will mich wieder haben}{\lemma{\textnormal{\emph{»Die … haben}}}\Cendnote{\textnormal{Salten\pwindex{Salten, Felix 6.\,9.\,1869 Budapest – 8.\,10.\,1945 Zürich@\textsc{Salten, Felix} (6.\,9.\,1869 Budapest – 8.\,10.\,1945 Zürich), \emph{Schriftsteller, Journalist, Chefredakteur}|pwk} arbeitete ab Oktober 1906 wieder für \emph{Die
                  Zeit}\orgindex{Zeit@Die Zeit|pwk}.}}}\label{K_L03430-3}, und ich bin gerne geneigt, abzuschließen. Dabei bietet sich
               hier der Plan zu einer \label{K_L03430-4v}\edtext{Wochenschrift\pwindex{Morgen. Wochenschrift für deutsche Kultur@\emph{Morgen. Wochenschrift für deutsche Kultur}|pwuv}\orgindex{Morgen. Wochenschrift für deutsche Kultur@Morgen. Wochenschrift für deutsche Kultur|pwu}}{\lemma{\textnormal{\emph{Wochenschrift}}}\Cendnote{\textnormal{Es dürfte vom \emph{Morgen}\orgindex{Morgen. Wochenschrift für deutsche Kultur@Morgen. Wochenschrift für deutsche Kultur|pwk} die Rede sein, den, unter anderen, Richard Strauß\pwindex{Strauss, Richard 11.\,6.\,1864 München – 8.\,9.\,1949 Garmisch-Partenkirchen@\textsc{Strauss, Richard} (11.\,6.\,1864 München – 8.\,9.\,1949 Garmisch-Partenkirchen), \emph{Theaterleiter, Komponist, Dirigent}|pwk} (ohne Salten\pwindex{Salten, Felix 6.\,9.\,1869 Budapest – 8.\,10.\,1945 Zürich@\textsc{Salten, Felix} (6.\,9.\,1869 Budapest – 8.\,10.\,1945 Zürich), \emph{Schriftsteller, Journalist, Chefredakteur}|pwk}) ab 1907 herausgab. }}}\label{K_L03430-4}, die ich mit Max Liebermann\pwindex{Liebermann, Max 20.\,7.\,1847 Berlin – 8.\,2.\,1935 ebd.@\textsc{Liebermann, Max} (20.\,7.\,1847 Berlin – 8.\,2.\,1935 ebd.), \emph{Maler, Maler, Maler}|pw} und Rich. Strauß\pwindex{Strauss, Richard 11.\,6.\,1864 München – 8.\,9.\,1949 Garmisch-Partenkirchen@\textsc{Strauss, Richard} (11.\,6.\,1864 München – 8.\,9.\,1949 Garmisch-Partenkirchen), \emph{Theaterleiter, Komponist, Dirigent}|pw} zusammen herausgeben, und allein leiten soll.
               Ihr Bestand ist für drei Jahre garantirt. Honorarbudget, ohne meine Gage, nur für
               Mitarbeiter 1000 Mark pro Nummer. Sie soll das Blatt der »anständigen Leute« werden,
               der Besten, ganz einfach. Ein kleiner, exclusiver, ständiger Mitarbeiterkreis. Ich
               hätte ausser der Gage noch einen Besitzanteil. Jetzt überleg ich mir’s, ob ich die
               Sache nicht von Wien\oindex{Wien@\textbf{Wien}, \emph{Verwaltungsgebiet}|pw} aus machen kann. Technisch
               gehts ganz gut. Die Schwierigkeiten, die sich freilich ergeben, würden reichlich
               durch manche Vorteile, {\pb}die
               sich dran knüpfen, aufgewogen. Ich könnte z. B. die Berlin\oindex{Berlin@\textbf{Berlin}, \emph{Hauptstadt}|pw}er u. Wien\oindex{Wien@\textbf{Wien}, \emph{Verwaltungsgebiet}|pw}er Theater zusammen
               überschauen und besprechen. Würde bei allen wichtigen Aufführungen (an die Premiere
               bin ich ja nicht gebunden) in Berlin\oindex{Berlin@\textbf{Berlin}, \emph{Hauptstadt}|pw} sein. Könnte
                  deutsch\oindex{Deutschland@\textbf{Deutschland}|pwv}e und österreich\oindex{Österreich@\textbf{Österreich}|pwv}ische Kultur- und
               Gesellschaftskritik zusammen treiben, was dem Blatte\pwindex{Morgen. Wochenschrift für deutsche Kultur@\emph{Morgen. Wochenschrift für deutsche Kultur}|pwuv}\orgindex{Morgen. Wochenschrift für deutsche Kultur@Morgen. Wochenschrift für deutsche Kultur|pwu} ebenso wie meiner Stellung etwas ganz Besonderes gäbe. Und wenn – binnen Kurzem
               – ein Thronwechsel in Österreich\oindex{Österreich@\textbf{Österreich}|pw} alles
               Interesse erregt, wär’s für eine solche Wochenschrift eine ganz einzige Conjunctur.
               Ganz abgesehen davon, dass ich, als in Wien\oindex{Wien@\textbf{Wien}, \emph{Verwaltungsgebiet}|pw}
               lebend, nicht mehr unter der Fuchtel der politischen Polizei in Preussen\oindex{Preußen@\textbf{Preußen}|pw}, die ärger ist als man glaubt, und nicht mehr unter
               der Ausweisungsgefahr leben müßte.\pend
           
\pstart
           Glauben Sie, dass mein Wiedereintritt in die die »Zeit\orgindex{Zeit@Die Zeit|pw}« für mich gut wäre? Dass man mich dort braucht, sehe ich, und dass die
                  »Zeit\orgindex{Zeit@Die Zeit|pw}« jetzt ihre literarische Stimme
               eingebüßt hat, kann ich wol, ohne Ihrem Freund \label{K_L03430-5v}\edtext{Bauer\pwindex{Bauer, Ludwig 5.\,9.\,1876 Wien – 1.\,2.\,1935 Lugano@\textsc{Bauer, Ludwig} (5.\,9.\,1876 Wien – 1.\,2.\,1935 Lugano), \emph{Schriftsteller, Journalist}|pw}}{\lemma{\textnormal{\emph{Bauer}}}\Cendnote{\textnormal{Saltens\pwindex{Salten, Felix 6.\,9.\,1869 Budapest – 8.\,10.\,1945 Zürich@\textsc{Salten, Felix} (6.\,9.\,1869 Budapest – 8.\,10.\,1945 Zürich), \emph{Schriftsteller, Journalist, Chefredakteur}|pwk} Nachfolger, vgl. A. S.: \emph{Tagebuch}, 15. 2. 1906.}}}\label{K_L03430-5} allzu unrecht zu thun, sagen.\pend
           
\pstart
           Von sonstigen Dingen: dass Herr Friedegg\pwindex{Friedegg, Ernst Egon 29.\,4.\,1883 Wien – Juli 1935 Prag@\textsc{Friedegg, Ernst Egon} (29.\,4.\,1883 Wien – Juli 1935 Prag), \emph{Schriftsteller, Journalist}|pw}
               knapp vor der Verhandlung eine umfassende Ehrenerklärung abgegeben hat. Dass der
                  \label{K_L03430-6v}\edtext{Ludassy\pwindex{Gans-Ludassy, Julius von 13.\,4.\,1858 Wien – 30.\,9.\,1922 ebd.@\textsc{Gans-Ludassy, Julius von} (13.\,4.\,1858 Wien – 30.\,9.\,1922 ebd.), \emph{Schriftsteller, Journalist, Herausgeber}|pw}-Prozess}{\lemma{\textnormal{\emph{Ludassy-Prozess}}}\Cendnote{\textnormal{Siehe XXXX Auszeichnungsfehler: Dokument L03415 nicht gefunden.
               }}}\label{K_L03430-6} vertagt ist. Dass mein Bruder\pwindex{Salzmann, Ignaz 30.\,12.\,1858 Budapest – 8.\,8.\,1932 Wien@\textsc{Salzmann, Ignaz} (30.\,12.\,1858 Budapest – 8.\,8.\,1932 Wien), \emph{Kaufmann}|pwv} leider weit davon entfernt ist, ein Millionär {\pb}zu sein, dass er aber freilich,
               gottseidank, ein so ahnsehnliches Geld verdient hat, dass ich – hoffentlich – für
               alle Zukunft der Sorge um ihn und um meine Familie enthoben bin. \uline{Wie} viel er besitzt, weiß ich nicht, weiß nur, dass er mit seiner Frau\pwindex{Salzmann, Agathe 26.\,12.\,1857 Niedernberg – 8.\,10.\,1938 Wien@\textsc{Salzmann, Agathe} (26.\,12.\,1857 Niedernberg – 8.\,10.\,1938 Wien)|pwv} sechs Wochen in England\oindex{England@\textbf{England}, \emph{Land}|pw} war, ihr um 20.000 Kronen Schmuck gekauft
               hat, für meine Mama\pwindex{Salzmann, Marie 27.\,10.\,1833 Budapest – 1.\,12.\,1909 Wien@\textsc{Salzmann, Marie} (27.\,10.\,1833 Budapest – 1.\,12.\,1909 Wien)|pwv} alles
               Erdenkliche tut, und meiner sel. Schwester\pwindex{Salzmann, Katharina 1865/1866 Budapest – 7.\,8.\,1883 Wien@\textsc{Salzmann, Katharina} (1865/1866 Budapest – 7.\,8.\,1883 Wien), \emph{Pflegerin}|pwv} wie meinem Papa\pwindex{Salzmann, Philipp 24.\,12.\,1831 Miskolc – 2.\,4.\,1905 Wien@\textsc{Salzmann, Philipp} (24.\,12.\,1831 Miskolc – 2.\,4.\,1905 Wien), \emph{Bergbauunternehmer, Projektemacher}|pwv} ein kostbares Grabmonument hat errichten laßen, dass er bei alledem
               doch weit von einer Million entfernt, und bei alledem von seinem Glück geradezu
               melancholisch geworden ist, weil der Papa\pwindex{Salzmann, Philipp 24.\,12.\,1831 Miskolc – 2.\,4.\,1905 Wien@\textsc{Salzmann, Philipp} (24.\,12.\,1831 Miskolc – 2.\,4.\,1905 Wien), \emph{Bergbauunternehmer, Projektemacher}|pwv} jahrelang darauf gewartet hat, und – genau \label{K_L03430-7v}\edtext{zwei Wochen zu früh starb}{\lemma{\textnormal{\emph{zwei … starb}}}\Cendnote{\textnormal{Philip Salzmann\pwindex{Salzmann, Philipp 24.\,12.\,1831 Miskolc – 2.\,4.\,1905 Wien@\textsc{Salzmann, Philipp} (24.\,12.\,1831 Miskolc – 2.\,4.\,1905 Wien), \emph{Bergbauunternehmer, Projektemacher}|pwk} war am 2. 4. 1905 verstorben.}}}\label{K_L03430-7}.\pend
           
\pstart
           Ich hatte im \label{K_L03430-8v}\edtext{Mai eine heftige Nierenkolik}{\lemma{\textnormal{\emph{Mai … Nierenkolik}}}\Cendnote{\textnormal{Siehe XXXX Auszeichnungsfehler: Dokument L03426 nicht gefunden.
               }}}\label{K_L03430-8}. Zweimal an zwei aufeinanderfolgenden Tagen. Bekam zweimal Morphium,
               beidemale mit einer unreinen Spritze oder mit einer mangelhaft gekochten Lösung.
               Musste dann fünf Tage lang rasende Schmerzen leiden, und am Ende froh sein, dass
               nicht Schlimmeres geschah. Dabei weiß ich trotz zweier Ärzte nicht, ob ich den
               Nierenstein habe, oder ob es nur eine akute Sache gewesen ist.\pend
           
\pstart
           Otti\pwindex{Salten, Ottilie 7.\,3.\,1868 Prag – 22.\,6.\,1942 Zürich@\textsc{Salten, Ottilie} (7.\,3.\,1868 Prag – 22.\,6.\,1942 Zürich), \emph{Schauspielerin}|pw} und die Kinder\pwindex{Rehmann, Anna Katharina 18.\,8.\,1904 Wien – 27.\,3.\,1977 Zürich@\textsc{Rehmann, Anna Katharina} (18.\,8.\,1904 Wien – 27.\,3.\,1977 Zürich), \emph{Schauspielerin, Übersetzerin}|pwv}\pwindex{Salten, Paul 11.\,8.\,1903 Wien – 8.\,5.\,1937 ebd.@\textsc{Salten, Paul} (11.\,8.\,1903 Wien – 8.\,5.\,1937 ebd.), \emph{Filmcutter}|pwv} sind wol und frisch in Bansin\oindex{Bansin@\textbf{Bansin}|pw}, dessen sonstige Gesellschaft mir als
               der Ausbund alles Grausenhaften geschildert wird. Ich gehe am 15. Juli zu ihnen. Dann wollen wir einmal, vielleicht sogar mit den Kindern\pwindex{Rehmann, Anna Katharina 18.\,8.\,1904 Wien – 27.\,3.\,1977 Zürich@\textsc{Rehmann, Anna Katharina} (18.\,8.\,1904 Wien – 27.\,3.\,1977 Zürich), \emph{Schauspielerin, Übersetzerin}|pwv}\pwindex{Salten, Paul 11.\,8.\,1903 Wien – 8.\,5.\,1937 ebd.@\textsc{Salten, Paul} (11.\,8.\,1903 Wien – 8.\,5.\,1937 ebd.), \emph{Filmcutter}|pwv}, per Schiff
                  \label{K_L03430-9v}\edtext{nach Kopenhagen\oindex{Kopenhagen@\textbf{Kopenhagen}, \emph{Hauptstadt}|pw}, wo wir uns sehen {\pb}könnten}{\lemma{\textnormal{\emph{nach … könnten}}}\Cendnote{\textnormal{Sie sahen sich nicht in Kopenhagen\oindex{Kopenhagen@\textbf{Kopenhagen}, \emph{Hauptstadt}|pwk}, aber am 2. 8. 1906 in Marienlyst\oindex{Marienlyst@\textbf{Marienlyst}, \emph{Gut}|pwk}.}}}\label{K_L03430-9}. An dem \label{K_L03430-10v}\edtext{Ausflug an die Nordsee\oindex{Nordsee@\textbf{Nordsee}, \emph{Meer}|pw}}{\lemma{\textnormal{\emph{Ausflug an die Nordsee}}}\Cendnote{\textnormal{Nordwijk\oindex{Noordwijk@\textbf{Noordwijk}, \emph{Verwaltungsgebiet}|pwk}, vgl. XXXX Auszeichnungsfehler: Dokument L03419 nicht gefunden.}}}\label{K_L03430-10} werd ich wol nicht teil nehmen. Ich will, wenn’s geht, in Bansin\oindex{Bansin@\textbf{Bansin}|pw} noch arbeiten. Die \label{K_L03430-11v}\edtext{vierzehn Tage London\oindex{London@\textbf{London}, \emph{Hauptstadt}|pw} – Stratford\oindex{Stratford-upon-Avon@\textbf{Stratford-upon-Avon}|pw} – Cambridge\oindex{Cambridge@\textbf{Cambridge}, \emph{Hauptstadt}|pw}}{\lemma{\textnormal{\emph{vierzehn … Cambridge}}}\Cendnote{\textnormal{Vgl. XXXX Auszeichnungsfehler: Dokument L03427 nicht gefunden.
               }}}\label{K_L03430-11} waren sehr schön. Die Seefahrt – hin nach Southampton\oindex{Southampton@\textbf{Southampton}, \emph{Hauptstadt}|pw}, zurück von Plymouth\oindex{Plymouth@\textbf{Plymouth}, \emph{Hauptstadt}|pw} über Cherbourg\oindex{Cherbourg-Octeville@\textbf{Cherbourg-Octeville}|pw} – wundervoll.
               Die engl\oindex{England@\textbf{England}, \emph{Land}|pwv}ische Landschaft ist
               beinahe überall so schön wie Dornbach\oindex{Wien@\textbf{Wien}!XVII., Hernals@\textbf{XVII., Hernals}!Dornbach@\textbf{Dornbach}|pw}.\pend
           
\pstart
           Schreiben Sie mir bis zum 14. nach Berlin\oindex{Berlin@\textbf{Berlin}, \emph{Hauptstadt}|pw}. Von da ab Seebad Bansin\oindex{Bansin@\textbf{Bansin}|pw}, Seestraße 5\oindex{Seestraße@\textbf{Seestraße}, \emph{Straße}|pw}.\pend
           
\pstart
           Viele herzliche Grüße Ihnen, Frau Olga\pwindex{Schnitzler, Olga 17.\,1.\,1882 Wien – 13.\,1.\,1970 Lugano@\textsc{Schnitzler, Olga} (17.\,1.\,1882 Wien – 13.\,1.\,1970 Lugano), \emph{Schauspielerin, Sängerin}|pw} und Heini\pwindex{Schnitzler, Heinrich 9.\,8.\,1902 Hinterbrühl – 12.\,7.\,1982 Wien@\textsc{Schnitzler, Heinrich} (9.\,8.\,1902 Hinterbrühl – 12.\,7.\,1982 Wien), \emph{Regisseur, Schauspieler}|pw}.\pend
           \pstart Ihr \spacefill\mbox{Salten}\pend{}\selectlanguage{ngerman}\endnumbering\briefempfaengerindex{Schnitzler, Arthur@\textsc{Schnitzler, Arthur}!zzzSalten, Felix@\emph{von Felix Salten}!1906-07-061@{6. 7. 1906}|)be}\mylabel{L03430h}  \newcommand{\dateiname}{L03430}\newcommand{\titel}{Felix Salten an Arthur Schnitzler, 6. 7. 1906}\newcommand{\editorInnen}{Martin Anton Müller und Laura Untner}%% latex-leseansicht-abspann.tex
%% Abspann für die Leseansicht.
%% Der Schalter \ifkorrekturansicht ist bereits durch den Vorspann gesetzt.

%% latex-abspann.tex
%% Gemeinsamer Abspann für Korrekturansicht und Leseansicht.
%% Setzt den Schalter \ifkorrekturansicht voraus (gesetzt in den
%% einbindenden Dateien latex-korrekturansicht-abspann.tex bzw.
%% latex-leseansicht-abspann.tex).
%% ---------------------------------------------------------------

\normalsize

% Das esempio-Environment wird nur in der Leseansicht benötigt
\ifkorrekturansicht\else
\newenvironment{esempio}[3]%
{
    \vspace{1.5ex}
    \rlap{\underline{#1}}
    \par
    \setlength{\parindent}{0cm}
    \nopagebreak
    \leftskip=#2cm
    \rightskip=#3cm
}
{
    \par
}
\fi

\doendnotes{C}
\bigskip
\vfill

\clearpage

\footnotesize

\ifkorrekturansicht
  \lohead{\textsc{register}}
\fi

% theindex-Environment neu definieren ohne reledmac
\makeatletter
\renewenvironment{theindex}{%
  \ifkorrekturansicht
    \section*{\indexname}%
  \else
    \subsubsection*{Index der erwähnten Entitäten}%
  \fi
  \setlength{\parindent}{0pt}%
  \setlength{\parskip}{0pt plus 0.3pt}%
  \let\item\@idxitem
}{%
  \ifkorrekturansicht\clearpage\fi
}
\makeatother

\IfFileExists{\jobname-pw.ind}{\input{\jobname-pw.ind}}{}

% Quellenangabe nur in der Leseansicht
\ifkorrekturansicht\else
% Fallback-Definitionen, falls die .tex-Datei \titel etc. nicht gesetzt hat
\providecommand{\titel}{}
\providecommand{\editorInnen}{}
\providecommand{\dateiname}{\jobname}

\vspace{3cm}

\vfill

\footnotesize
\textsc{Quelle}: \titel. Herausgegeben von {\editorInnen}. In: \emph{Arthur Schnitzler: Briefwechsel mit Autorinnen und Autoren}.
 Digitale Edition, https://schnitzler-briefe.acdh.oeaw.ac.at/{\dateiname}.html (Stand \today)
\fi

\end{document}


