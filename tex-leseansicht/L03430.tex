%% latex-korrekturansicht-vorspann.tex
%% Vorspann für die Korrekturansicht.
%% Lädt die gemeinsame Datei latex-vorspann.tex mit gesetztem Schalter.

\newif\ifkorrekturansicht
\korrekturansichttrue

\input{../tex-inputs/latex-vorspann}


\section[ Felix Salten an Arthur Schnitzler, 6. 7. 1906]{L03430 Felix Salten an Arthur Schnitzler, 6. 7. 1906}
\nopagebreak\mylabel{L03430v}
\rehead{ }\normalsize\beginnumbering\briefempfaengerindex{Schnitzler, Arthur@\textsc{Schnitzler, Arthur}!zzzSalten, Felix@\emph{von Felix Salten}!1906-07-061@{6. 7. 1906}|(be}
\toendnotes[C]{\smallbreak\pagebreak[2]}\Standort{CUL, Schnitzler, B 89, B 1.}
\physDesc{Brief, 1 Blatt, 4 Seiten, 4137 Zeichen
\newline{}Handschrift: schwarze Tinte, lateinische Kurrent
\newline{}Schnitzler: mit rotem Buntstift fünf Unterstreichungen 
\newline{}Ordnung: mit Bleistift von unbekannter Hand nummeriert: »221« }\toendnotes[C]{\smallbreak}
\pstart
           \raggedleft{}{\pb}Berlin\oindex{Berlin@\textbf{Berlin}, \emph{P.PPLC}|pw}, 6. 7. 06.\pend
           \vspace{0.5em}
\pstart
           Lieber, wie Schade, dass Sie gerade jetzt \label{K_L03430-1v}\edtext{durch Berlin\oindex{Berlin@\textbf{Berlin}, \emph{P.PPLC}|pw} kamen}{\lemma{\textnormal{\emph{durch Berlin kamen}}}\Cendnote{\textnormal{Schnitzler reiste über Berlin\oindex{Berlin@\textbf{Berlin}, \emph{P.PPLC}|pwk} nach Marienlyst\oindex{Marienlyst@\textbf{Marienlyst}, \emph{S.EST}|pwk}, siehe A. S.: \emph{Tagebuch}, 26. 6. 1906.}}}\label{K_L03430-1}, während meiner Abwesenheit. Man hätte vielleicht doch eine Stunde gehabt, um
               sich auszusprechen. Schreiben ist in manchen Fällen so schwer. Was ich jetzt, \label{K_L03430-2v}\edtext{in der
               nächsten Zeit, beginne}{\lemma{\textnormal{\emph{in … beginne}}}\Cendnote{\textnormal{Mit 12. 7. 1906 endete
                  Saltens\pwindex{Salten, Felix 06.09.1869 – 08.10.1945@\textsc{Salten, Felix} (06.09.1869 – 08.10.1945), \emph{Schriftsteller/Schriftstellerin, Journalist/Journalistin, Chefredakteur/Chefredakteurin}|pwk} berufliches Engagement bei der \emph{Berliner Morgenpost}\orgindex{Berliner Morgenpost@Berliner Morgenpost|pwk} und der
                  \emph{B. Z. am Mittag}\orgindex{B.Z. am Mittag@B.Z. am Mittag|pwk}. Siehe Marcel Atze: \emph{»Unser aller Feldmarschall mit der Feder«. Felix Saltens halbes Jahrhundert
                     als Journalist.} In: Marcel Atze, unter Mitarbeit von Tanja Gausterer
                  (Herausgeber): \emph{Im Schatten von Bambi. Felix Salten entdeckt die Wiener
                     Moderne. Leben und Werk}.
                  Salzburg/Wien:
                  \emph{Residenz}{ }2020, S. 260–289, hier 285.
               }}}\label{K_L03430-2}, liegt noch im Halbdunkel; und was ich Ihnen davon mitteile,
               ist – einstweilen – nur für Sie. In Berlin\oindex{Berlin@\textbf{Berlin}, \emph{P.PPLC}|pw} will
               ich nicht bleiben; kann es ehrlicherweise garnicht tun und spüre, dass ein Bruch in
               mein Leben käme, wollte ich versuchen\textcolor{gray}{,} mich zu zwingen. \label{K_L03430-3v}\edtext{»Die
                  Zeit\orgindex{Zeit@Die Zeit|pw}« will mich wieder haben}{\lemma{\textnormal{\emph{»Die … haben}}}\Cendnote{\textnormal{Salten\pwindex{Salten, Felix 06.09.1869 – 08.10.1945@\textsc{Salten, Felix} (06.09.1869 – 08.10.1945), \emph{Schriftsteller/Schriftstellerin, Journalist/Journalistin, Chefredakteur/Chefredakteurin}|pwk} arbeitete ab Oktober 1906 wieder für \emph{Die
                  Zeit}\orgindex{Zeit@Die Zeit|pwk}.}}}\label{K_L03430-3}, und ich bin gerne geneigt, abzuschließen. Dabei bietet sich
               hier der Plan zu einer \label{K_L03430-4v}\edtext{Wochenschrift\pwindex{Morgen. Wochenschrift fuer deutsche Kultur@\emph{Morgen. Wochenschrift für deutsche Kultur}|pwuv}\orgindex{Morgen. Wochenschrift fuer deutsche Kultur@Morgen. Wochenschrift für deutsche Kultur|pwu}}{\lemma{\textnormal{\emph{Wochenschrift}}}\Cendnote{\textnormal{Es dürfte vom \emph{Morgen}\orgindex{Morgen. Wochenschrift fuer deutsche Kultur@Morgen. Wochenschrift für deutsche Kultur|pwk} die Rede sein, den, unter anderen, Richard Strauß\pwindex{Strauss, Richard 11.06.1864 – 08.09.1949@\textsc{Strauss, Richard} (11.06.1864 – 08.09.1949), \emph{Theaterleiter/Theaterleiterin, Komponist/Komponistin, Dirigent/Dirigentin}|pwk} (ohne Salten\pwindex{Salten, Felix 06.09.1869 – 08.10.1945@\textsc{Salten, Felix} (06.09.1869 – 08.10.1945), \emph{Schriftsteller/Schriftstellerin, Journalist/Journalistin, Chefredakteur/Chefredakteurin}|pwk}) ab 1907 herausgab. }}}\label{K_L03430-4}, die ich mit Max Liebermann\pwindex{Liebermann, Max 20.07.1847 – 08.02.1935@\textsc{Liebermann, Max} (20.07.1847 – 08.02.1935), \emph{Maler/Malerin, Maler/Malerin, Maler/Malerin}|pw} und Rich. Strauß\pwindex{Strauss, Richard 11.06.1864 – 08.09.1949@\textsc{Strauss, Richard} (11.06.1864 – 08.09.1949), \emph{Theaterleiter/Theaterleiterin, Komponist/Komponistin, Dirigent/Dirigentin}|pw} zusammen herausgeben, und allein leiten soll.
               Ihr Bestand ist für drei Jahre garantirt. Honorarbudget, ohne meine Gage, nur für
               Mitarbeiter 1000 Mark pro Nummer. Sie soll das Blatt der »anständigen Leute« werden,
               der Besten, ganz einfach. Ein kleiner, exclusiver, ständiger Mitarbeiterkreis. Ich
               hätte ausser der Gage noch einen Besitzanteil. Jetzt überleg ich mir’s, ob ich die
               Sache nicht von Wien\oindex{Wien@\textbf{Wien}, \emph{A.ADM2}|pw} aus machen kann. Technisch
               gehts ganz gut. Die Schwierigkeiten, die sich freilich ergeben, würden reichlich
               durch manche Vorteile, {\pb}die
               sich dran knüpfen, aufgewogen. Ich könnte z. B. die Berlin\oindex{Berlin@\textbf{Berlin}, \emph{P.PPLC}|pw}er u. Wien\oindex{Wien@\textbf{Wien}, \emph{A.ADM2}|pw}er Theater zusammen
               überschauen und besprechen. Würde bei allen wichtigen Aufführungen (an die Premiere
               bin ich ja nicht gebunden) in Berlin\oindex{Berlin@\textbf{Berlin}, \emph{P.PPLC}|pw} sein. Könnte
                  deutsch\oindex{Deutschland@\textbf{Deutschland}, \emph{A.PCLI}|pwv}e und österreich\oindex{Oesterreich@\textbf{Österreich}, \emph{A.PCLI}|pwv}ische Kultur- und
               Gesellschaftskritik zusammen treiben, was dem Blatte\pwindex{Morgen. Wochenschrift fuer deutsche Kultur@\emph{Morgen. Wochenschrift für deutsche Kultur}|pwuv}\orgindex{Morgen. Wochenschrift fuer deutsche Kultur@Morgen. Wochenschrift für deutsche Kultur|pwu} ebenso wie meiner Stellung etwas ganz Besonderes gäbe. Und wenn – binnen Kurzem
               – ein Thronwechsel in Österreich\oindex{Oesterreich@\textbf{Österreich}, \emph{A.PCLI}|pw} alles
               Interesse erregt, wär’s für eine solche Wochenschrift eine ganz einzige Conjunctur.
               Ganz abgesehen davon, dass ich, als in Wien\oindex{Wien@\textbf{Wien}, \emph{A.ADM2}|pw}
               lebend, nicht mehr unter der Fuchtel der politischen Polizei in Preussen\oindex{Preussen@\textbf{Preußen}, \emph{Land (A.LND)}|pw}, die ärger ist als man glaubt, und nicht mehr unter
               der Ausweisungsgefahr leben müßte.\pend
           
\pstart
           Glauben Sie, dass mein Wiedereintritt in die die »Zeit\orgindex{Zeit@Die Zeit|pw}« für mich gut wäre? Dass man mich dort braucht, sehe ich, und dass die
                  »Zeit\orgindex{Zeit@Die Zeit|pw}« jetzt ihre literarische Stimme
               eingebüßt hat, kann ich wol, ohne Ihrem Freund \label{K_L03430-5v}\edtext{Bauer\pwindex{Bauer, Ludwig 05.09.1876 – 01.02.1935@\textsc{Bauer, Ludwig} (05.09.1876 – 01.02.1935), \emph{Schriftsteller/Schriftstellerin, Journalist/Journalistin}|pw}}{\lemma{\textnormal{\emph{Bauer}}}\Cendnote{\textnormal{Saltens\pwindex{Salten, Felix 06.09.1869 – 08.10.1945@\textsc{Salten, Felix} (06.09.1869 – 08.10.1945), \emph{Schriftsteller/Schriftstellerin, Journalist/Journalistin, Chefredakteur/Chefredakteurin}|pwk} Nachfolger, vgl. A. S.: \emph{Tagebuch}, 15. 2. 1906.}}}\label{K_L03430-5} allzu unrecht zu thun, sagen.\pend
           
\pstart
           Von sonstigen Dingen: dass Herr Friedegg\pwindex{Friedegg, Ernst Egon 1883-04-29 – Juli 1935@\textsc{Friedegg, Ernst Egon} (1883-04-29 – Juli 1935), \emph{Schriftsteller/Schriftstellerin, Journalist/Journalistin}|pw}
               knapp vor der Verhandlung eine umfassende Ehrenerklärung abgegeben hat. Dass der
                  \label{K_L03430-6v}\edtext{Ludassy\pwindex{Gans-Ludassy, Julius von 13.04.1858 – 30.09.1922@\textsc{Gans-Ludassy, Julius von} (13.04.1858 – 30.09.1922), \emph{Schriftsteller/Schriftstellerin, Journalist/Journalistin, Herausgeber/Herausgeberin}|pw}-Prozess}{\lemma{\textnormal{\emph{Ludassy-Prozess}}}\Cendnote{\textnormal{Siehe Felix Salten an Arthur Schnitzler, 9. 3. 1906.
               }}}\label{K_L03430-6} vertagt ist. Dass mein Bruder\pwindex{Salzmann, Ignaz 1858-12-30 – 1932-08-08@\textsc{Salzmann, Ignaz} (1858-12-30 – 1932-08-08), \emph{Kaufmann/Kauffrau}|pwv} leider weit davon entfernt ist, ein Millionär {\pb}zu sein, dass er aber freilich,
               gottseidank, ein so ahnsehnliches Geld verdient hat, dass ich – hoffentlich – für
               alle Zukunft der Sorge um ihn und um meine Familie enthoben bin. \uline{Wie} viel er besitzt, weiß ich nicht, weiß nur, dass er mit seiner Frau\pwindex{Salzmann, Agathe 26.12.1857 – 1938-10-08@\textsc{Salzmann, Agathe} (26.12.1857 – 1938-10-08)|pwv} sechs Wochen in England\oindex{England@\textbf{England}, \emph{A.ADM1}|pw} war, ihr um 20.000 Kronen Schmuck gekauft
               hat, für meine Mama\pwindex{Salzmann, Marie 1833-10-27 – 1909-12-01@\textsc{Salzmann, Marie} (1833-10-27 – 1909-12-01)|pwv} alles
               Erdenkliche tut, und meiner sel. Schwester\pwindex{Salzmann, Katharina 1865/1866 – 1883-08-07@\textsc{Salzmann, Katharina} (1865/1866 – 1883-08-07), \emph{Pfleger/Pflegerin}|pwv} wie meinem Papa\pwindex{Salzmann, Philipp 1831-12-24 – 1905-04-02@\textsc{Salzmann, Philipp} (1831-12-24 – 1905-04-02), \emph{Bergbauunternehmer/Bergbauunternehmerin, Projektemacher/Projektemacherin}|pwv} ein kostbares Grabmonument hat errichten laßen, dass er bei alledem
               doch weit von einer Million entfernt, und bei alledem von seinem Glück geradezu
               melancholisch geworden ist, weil der Papa\pwindex{Salzmann, Philipp 1831-12-24 – 1905-04-02@\textsc{Salzmann, Philipp} (1831-12-24 – 1905-04-02), \emph{Bergbauunternehmer/Bergbauunternehmerin, Projektemacher/Projektemacherin}|pwv} jahrelang darauf gewartet hat, und – genau \label{K_L03430-7v}\edtext{zwei Wochen zu früh starb}{\lemma{\textnormal{\emph{zwei … starb}}}\Cendnote{\textnormal{Philip Salzmann\pwindex{Salzmann, Philipp 1831-12-24 – 1905-04-02@\textsc{Salzmann, Philipp} (1831-12-24 – 1905-04-02), \emph{Bergbauunternehmer/Bergbauunternehmerin, Projektemacher/Projektemacherin}|pwk} war am 2. 4. 1905 verstorben.}}}\label{K_L03430-7}.\pend
           
\pstart
           Ich hatte im \label{K_L03430-8v}\edtext{Mai eine heftige Nierenkolik}{\lemma{\textnormal{\emph{Mai … Nierenkolik}}}\Cendnote{\textnormal{Siehe Felix Salten u. a. an Arthur Schnitzler, 4. 6. 1906.
               }}}\label{K_L03430-8}. Zweimal an zwei aufeinanderfolgenden Tagen. Bekam zweimal Morphium,
               beidemale mit einer unreinen Spritze oder mit einer mangelhaft gekochten Lösung.
               Musste dann fünf Tage lang rasende Schmerzen leiden, und am Ende froh sein, dass
               nicht Schlimmeres geschah. Dabei weiß ich trotz zweier Ärzte nicht, ob ich den
               Nierenstein habe, oder ob es nur eine akute Sache gewesen ist.\pend
           
\pstart
           Otti\pwindex{Salten, Ottilie 07.03.1868 – 22.06.1942@\textsc{Salten, Ottilie} (07.03.1868 – 22.06.1942), \emph{Schauspieler/Schauspielerin}|pw} und die Kinder\pwindex{Rehmann, Anna Katharina 18.08.1904 – 27.03.1977@\textsc{Rehmann, Anna Katharina} (18.08.1904 – 27.03.1977), \emph{Schauspieler/Schauspielerin, Übersetzer/Übersetzerin}|pwv}\pwindex{Salten, Paul 11.08.1903 – 08.05.1937@\textsc{Salten, Paul} (11.08.1903 – 08.05.1937), \emph{Filmcutter/Filmcutterin}|pwv} sind wol und frisch in Bansin\oindex{Bansin@\textbf{Bansin}, \emph{P.PPL}|pw}, dessen sonstige Gesellschaft mir als
               der Ausbund alles Grausenhaften geschildert wird. Ich gehe am 15. Juli zu ihnen. Dann wollen wir einmal, vielleicht sogar mit den Kindern\pwindex{Rehmann, Anna Katharina 18.08.1904 – 27.03.1977@\textsc{Rehmann, Anna Katharina} (18.08.1904 – 27.03.1977), \emph{Schauspieler/Schauspielerin, Übersetzer/Übersetzerin}|pwv}\pwindex{Salten, Paul 11.08.1903 – 08.05.1937@\textsc{Salten, Paul} (11.08.1903 – 08.05.1937), \emph{Filmcutter/Filmcutterin}|pwv}, per Schiff
                  \label{K_L03430-9v}\edtext{nach Kopenhagen\oindex{Kopenhagen@\textbf{Kopenhagen}, \emph{P.PPLC}|pw}, wo wir uns sehen {\pb}könnten}{\lemma{\textnormal{\emph{nach … könnten}}}\Cendnote{\textnormal{Sie sahen sich nicht in Kopenhagen\oindex{Kopenhagen@\textbf{Kopenhagen}, \emph{P.PPLC}|pwk}, aber am 2. 8. 1906 in Marienlyst\oindex{Marienlyst@\textbf{Marienlyst}, \emph{S.EST}|pwk}.}}}\label{K_L03430-9}. An dem \label{K_L03430-10v}\edtext{Ausflug an die Nordsee\oindex{Nordsee@\textbf{Nordsee}, \emph{H.SEA}|pw}}{\lemma{\textnormal{\emph{Ausflug an die Nordsee}}}\Cendnote{\textnormal{Nordwijk\oindex{Noordwijk@\textbf{Noordwijk}, \emph{A.ADM2}|pwk}, vgl. Felix Salten u. a. an Arthur Schnitzler, 19. 4. 1906.}}}\label{K_L03430-10} werd ich wol nicht teil nehmen. Ich will, wenn’s geht, in Bansin\oindex{Bansin@\textbf{Bansin}, \emph{P.PPL}|pw} noch arbeiten. Die \label{K_L03430-11v}\edtext{vierzehn Tage London\oindex{London@\textbf{London}, \emph{P.PPLC}|pw} – Stratford\oindex{Stratford-upon-Avon@\textbf{Stratford-upon-Avon}, \emph{P.PPL}|pw} – Cambridge\oindex{Cambridge@\textbf{Cambridge}, \emph{P.PPLA2}|pw}}{\lemma{\textnormal{\emph{vierzehn … Cambridge}}}\Cendnote{\textnormal{Vgl. Felix Salten an Arthur Schnitzler, 19. 6. 1906.
               }}}\label{K_L03430-11} waren sehr schön. Die Seefahrt – hin nach Southampton\oindex{Southampton@\textbf{Southampton}, \emph{P.PPLA2}|pw}, zurück von Plymouth\oindex{Plymouth@\textbf{Plymouth}, \emph{P.PPLA2}|pw} über Cherbourg\oindex{Cherbourg-Octeville@\textbf{Cherbourg-Octeville}, \emph{kein passender Code gefunden}|pw} – wundervoll.
               Die engl\oindex{England@\textbf{England}, \emph{A.ADM1}|pwv}ische Landschaft ist
               beinahe überall so schön wie Dornbach\oindex{Dornbach@\textbf{Dornbach}, \emph{eingemeindeter Ort (A.VOO)}|pw}.\pend
           
\pstart
           Schreiben Sie mir bis zum 14. nach Berlin\oindex{Berlin@\textbf{Berlin}, \emph{P.PPLC}|pw}. Von da ab Seebad Bansin\oindex{Bansin@\textbf{Bansin}, \emph{P.PPL}|pw}, Seestraße 5\oindex{Seestrasse@\textbf{Seestraße}, \emph{Straße (K.STR)}|pw}.\pend
           
\pstart
           Viele herzliche Grüße Ihnen, Frau Olga\pwindex{Schnitzler, Olga 17.01.1882 – 13.01.1970@\textsc{Schnitzler, Olga} (17.01.1882 – 13.01.1970), \emph{Schauspieler/Schauspielerin, Sänger/Sängerin}|pw} und Heini\pwindex{Schnitzler, Heinrich 09.08.1902 – 12.07.1982@\textsc{Schnitzler, Heinrich} (09.08.1902 – 12.07.1982), \emph{Regisseur/Regisseurin, Schauspieler/Schauspielerin}|pw}.\pend
           \pstart Ihr \spacefill\mbox{Salten}\pend{}\selectlanguage{ngerman}\endnumbering\briefempfaengerindex{Schnitzler, Arthur@\textsc{Schnitzler, Arthur}!zzzSalten, Felix@\emph{von Felix Salten}!1906-07-061@{6. 7. 1906}|)be}\mylabel{L03430h}  \normalsize

\doendnotes{C}
\bigskip
\vfill

\clearpage

\footnotesize

\lohead{\textsc{register}}

% Definiere theindex-Environment komplett neu ohne reledmac
\makeatletter
\renewenvironment{theindex}{%
  \section*{\indexname}%
  \setlength{\parindent}{0pt}%
  \setlength{\parskip}{0pt plus 0.3pt}%
  \let\item\@idxitem
}{%
  \clearpage
}
\makeatother

\IfFileExists{\jobname-pw.ind}{\input{\jobname-pw.ind}}{}

\end{document}

      