%% latex-leseansicht-vorspann.tex
%% Vorspann für die Leseansicht.
%% Lädt die gemeinsame Datei latex-vorspann.tex mit nicht gesetztem Schalter.

\newif\ifkorrekturansicht
\korrekturansichtfalse

\input{../tex-inputs/latex-vorspann}


\section[Arthur Schnitzler an Gustav Schwarzkopf, 23. 7. 1905]{L04066 Arthur Schnitzler an Gustav Schwarzkopf, 23. 7. 1905}
\nopagebreak\mylabel{L04066v}
\rehead{ }\normalsize\beginnumbering\briefempfaengerindex{Schwarzkopf, Gustav@\textsc{Schwarzkopf, Gustav}!zzzSchnitzler, Arthur@\emph{von Arthur Schnitzler}!1905-07-231@{23. 7. 1905}|(be}
\toendnotes[C]{\smallbreak\pagebreak[2]}
\correspDesc{Versand  durch Arthur Schnitzler am 23. 7. 1905 in Reichenau an der Rax
\newline{}Erhalt  durch Gustav Schwarzkopf im Zeitraum [24. 7. 1905 – 28. 7. 1905?] in Wien}\toendnotes[C]{\smallbreak}
\Standort{CUL, Schnitzler, B 96.}
\physDesc{Kartenbrief, 444 Zeichen
\newline{}Handschrift: schwarze Tinte, deutsche Kurrent
\newline{}Versand: Stempel: »\nobreak{}\oindex{Reichenau an der Rax@\textbf{Reichenau an der Rax}, \emph{Verwaltungsgebiet}|pwk}Reichenau bei Payerbach, 23 7 0{[}5{]}, 7–\textcolor{gray}{8}V\nobreak{}«.  }\toendnotes[C]{\smallbreak}\pstart{}{\pb}Herrn Guſtav Schwarzkopf\pend{}\pstart{}\textsc{Wien}\oindex{I., Innere Stadt@\textbf{I., Innere Stadt}, \emph{Verwaltungsgebiet}|pw}\pend{}\pstart{}I. Tiefer Graben 23\oindex{Wien@\textbf{Wien}!I., Innere Stadt@\textbf{I., Innere Stadt}!Tiefer Graben 23@\textbf{Tiefer Graben 23}, \emph{Wohngebäude}|pw}.\pend{}{\bigskip}\vspace{1em}
\pstart
           \raggedleft{}{\pb}23. 7. 905.\pend
           \vspace{0.5em}
\pstart
           lieber Guſtav, von der Hochschwabtour\oindex{Hochschwab@\textbf{Hochschwab}, \emph{Gebirge}|pw}
      zurück find ich Ihre \label{K_L04066-1v}\edtext{Karte}{\lemma{\textnormal{\emph{Karte}}}\Cendnote{\textnormal{{XXXX ref} XXXX vom 20.7.1905
      }}}\label{K_L04066-1}. Wir
      bleiben kaum mehr länger hier
      als 8–10 Tage u gehen da{\geminationn} (zu
      ziemlich bleibendem Aufenthalt) nach
      Wien\oindex{Wien@\textbf{Wien}, \emph{Verwaltungsgebiet}|pw}. Da werden wir uns wohl,
               da Sie doch nicht ſo bald aus Sekirn\oindex{Sekirn@\textbf{Sekirn}|pw}
      weg fahren werden, erſt in Wien\oindex{Wien@\textbf{Wien}, \emph{Verwaltungsgebiet}|pw} wieder{ }ſehen. Schade. Grüßen Sie was zu
      grüßen iſt u. laſſen Sie sichs wohl
      ergehen,\pend
           
\pstart
           Herzlichſt, mit allſeitigen Grüßen,{\\[\baselineskip]} Ihr{\\[\baselineskip]}\spacefill\mbox{Arthur.}\pend
           \leftskip=0em{}\selectlanguage{ngerman}\endnumbering\briefempfaengerindex{Schwarzkopf, Gustav@\textsc{Schwarzkopf, Gustav}!zzzSchnitzler, Arthur@\emph{von Arthur Schnitzler}!1905-07-231@{23. 7. 1905}|)be}\mylabel{L04066h}
\begin{anhang}
\end{anhang}\newcommand{\dateiname}{L04066}\newcommand{\titel}{Arthur Schnitzler an Gustav Schwarzkopf, 23. 7. 1905}\newcommand{\editorInnen}{Herausgegeben von Jahnke, SelmaMüller, Martin Anton}%% latex-leseansicht-abspann.tex
%% Abspann für die Leseansicht.
%% Der Schalter \ifkorrekturansicht ist bereits durch den Vorspann gesetzt.

%% latex-abspann.tex
%% Gemeinsamer Abspann für Korrekturansicht und Leseansicht.
%% Setzt den Schalter \ifkorrekturansicht voraus (gesetzt in den
%% einbindenden Dateien latex-korrekturansicht-abspann.tex bzw.
%% latex-leseansicht-abspann.tex).
%% ---------------------------------------------------------------

\normalsize

% Das esempio-Environment wird nur in der Leseansicht benötigt
\ifkorrekturansicht\else
\newenvironment{esempio}[3]%
{
    \vspace{1.5ex}
    \rlap{\underline{#1}}
    \par
    \setlength{\parindent}{0cm}
    \nopagebreak
    \leftskip=#2cm
    \rightskip=#3cm
}
{
    \par
}
\fi

\doendnotes{C}
\bigskip
\vfill

\clearpage

\footnotesize

\ifkorrekturansicht
  \lohead{\textsc{register}}
\fi

% theindex-Environment neu definieren ohne reledmac
\makeatletter
\renewenvironment{theindex}{%
  \ifkorrekturansicht
    \section*{\indexname}%
  \else
    \subsubsection*{Index der erwähnten Entitäten}%
  \fi
  \setlength{\parindent}{0pt}%
  \setlength{\parskip}{0pt plus 0.3pt}%
  \let\item\@idxitem
}{%
  \ifkorrekturansicht\clearpage\fi
}
\makeatother

\IfFileExists{\jobname-pw.ind}{\input{\jobname-pw.ind}}{}

% Quellenangabe nur in der Leseansicht
\ifkorrekturansicht\else
% Fallback-Definitionen, falls die .tex-Datei \titel etc. nicht gesetzt hat
\providecommand{\titel}{}
\providecommand{\editorInnen}{}
\providecommand{\dateiname}{\jobname}

\vspace{3cm}

\vfill

\footnotesize
\textsc{Quelle}: \titel. Herausgegeben von {\editorInnen}. In: \emph{Arthur Schnitzler: Briefwechsel mit Autorinnen und Autoren}.
 Digitale Edition, https://schnitzler-briefe.acdh.oeaw.ac.at/{\dateiname}.html (Stand \today)
\fi

\end{document}


