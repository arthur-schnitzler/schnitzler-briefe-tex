%% latex-korrekturansicht-vorspann.tex
%% Vorspann für die Korrekturansicht.
%% Lädt die gemeinsame Datei latex-vorspann.tex mit gesetztem Schalter.

\newif\ifkorrekturansicht
\korrekturansichttrue

\input{../tex-inputs/latex-vorspann}


\section[Arthur Schnitzler an Hugo von Hofmannsthal, 21. 2. 1910]{L01914 Arthur Schnitzler an Hugo von Hofmannsthal, 21. 2. 1910}
\nopagebreak\mylabel{L01914v}
\rehead{ }\normalsize\beginnumbering\briefempfaengerindex{Hofmannsthal, Hugo von@\textsc{Hofmannsthal, Hugo von}!zzzSchnitzler, Arthur@\emph{von Arthur Schnitzler}!1910-02-211@{21. 2. 1910}|(be}
\toendnotes[C]{\smallbreak\pagebreak[2]}\Standort{FDH, Hs-30885,135.}
\physDesc{Briefkarte, 409 Zeichen
\newline{}Handschrift: schwarze Tinte, deutsche Kurrent}
\buchAbdrucke{\weitereDrucke{Hugo von Hofmannsthal, Arthur Schnitzler: \emph{Briefwechsel}. Frankfurt am Main: \emph{S. Fischer} 1964, S. 248.} }
\pstart
           {\pb}\textcolor{gray}{\textbf{Dr. Arthur Schnitzler}}\hfill 21. 2. 10\pend
           
\pstart
           \textcolor{gray}{\textbf{Wien XVIII. Spoettelgasse 7\oindex{Edmund-Weiss-Gasse 7@\textbf{Edmund-Weiß-Gasse 7}, \emph{Wohngebäude (K.WHS)}|pw}.}}\pend
           \vspace{0.5em}
\pstart
           lieber Hugo, ich danke Ihnen herzlich für die Komoedie von \textsc{Cristinas} Heimreiſe\pwindex{Cristinas Heimreise. Komoedie@\emph{Cristinas Heimreise. Komödie}|pw}; mit Vergnügen, bei mancherlei
               Bedenken mehr dramaturgiſcher Natur, hab ich ſie geleſen, und erwarte mir {\pb}ihre baldige \introOben{}Bühnen-\introOben{}Auferſtehung
               in concentrirterer Form. Worüber ich mich, auf Wunſch, gern und bald eingehender und
               mündlicher, vernehmen laſſe.\pend
           
\pstart
           Morgen fahren wir auf ein paar Tage ſe{\geminationm}ering\oindex{Semmering@\textbf{Semmering}, \emph{A.ADM3}|pw}wärts. Herzlichſt, auf bald\pend
           
\pstart
           Ihr{\\[\baselineskip]}\spacefill\mbox{A.}\pend
           \leftskip=0em{}\selectlanguage{ngerman}\endnumbering\briefempfaengerindex{Hofmannsthal, Hugo von@\textsc{Hofmannsthal, Hugo von}!zzzSchnitzler, Arthur@\emph{von Arthur Schnitzler}!1910-02-211@{21. 2. 1910}|)be}\mylabel{L01914h}  \normalsize

\doendnotes{C}
\bigskip
\vfill

\clearpage

\footnotesize

\lohead{\textsc{register}}

% Definiere theindex-Environment komplett neu ohne reledmac
\makeatletter
\renewenvironment{theindex}{%
  \section*{\indexname}%
  \setlength{\parindent}{0pt}%
  \setlength{\parskip}{0pt plus 0.3pt}%
  \let\item\@idxitem
}{%
  \clearpage
}
\makeatother

\IfFileExists{\jobname-pw.ind}{\input{\jobname-pw.ind}}{}

\end{document}

      