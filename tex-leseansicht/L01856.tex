\input{../tex-inputs/latex-pdf-vorspann}
\begin{center}
            \textcolor{red}{ENTWURF. ENTZIFFERUNG NOCH NICHT KORREKTURGELESEN}
                      \end{center}
            
               \section[Max Burckhard an Arthur Schnitzler, 12. 7. 1909]{ Max Burckhard an Arthur Schnitzler, 12. 7. 1909}\nopagebreak\mylabel{v}\rehead{ }\begin{ledgroupsized}[t]{13cm}\normalsize\beginnumbering\briefempfaengerindex{Schnitzler, Arthur@\textsc{Schnitzler, Arthur}!zzzBurckhard, Max Eugen@\emph{von Max Eugen Burckhard}!1909-07-122@{12. 7. 1909}|(be} \toendnotes[C]{\smallbreak\pagebreak[2]} \Standort{CUL, Schnitzler, B 20.}
\physDesc{Bildpostkarte
\newline{}Handschrift: schwarze Tinte, deutsche Kurrent\newline{}Versand: 1) Stempel: »\nobreak{}\oindex{Lueg am Wolfgangsee@\textbf{Lueg am Wolfgangsee}|pwk}Lueg (St. Gilgen)\nobreak{}«.  2) Stempel: »\nobreak{}\oindex{Salzburg@\textbf{Salzburg}|pwk}Salzburg, 12. 7. 09\nobreak{}«. 
\newline{}Schnitzler: mit Bleistift beschriftet: »B \textsc{Burckhard}« }\toendnotes[C]{\smallbreak}\pstart{}{\pb}\textsc{H. D\textsuperscript{r} Artur Schnitzler}\pend{}\pstart{}\textsc{Wien\oindex{Wien@\textbf{Wien}|pw}}\pend{}\pstart{}\textsc{XVIIII Spöttelgaße 7\oindex{Edmund-Weiss-Gasse@\textbf{Edmund-Weiß-Gasse}|pw}}\pend{}{\bigskip}\pstart
           \noindent{}\centering{}{\pb}{[}Burckhards Haus auf der Franzosenschanze\oindex{Franzosenschanze@\textbf{Franzosenschanze}|pw} in St. Gilgen\oindex{St. Gilgen@\textbf{St. Gilgen}|pw}{]}\pend
           \pstart{}{\pb}Lieber verehrter Herr
                        Doctor!\pend\pstart
           Leider muß ich ſagen: ſeien Sie froh, daſs Sie fort ſind, denn es gießt hier
                    ununterbrochen\pend
           \pstart
           Ich hoffe, daſs es Ihrem Kleinen\pwindex{Schnitzler, Heinrich 09.08.1902 – 12.07.1982@\textsc{Schnitzler, Heinrich} (09.08.1902 – 12.07.1982), \emph{Regisseur, Schauspieler}|pwv}{ }ſo gut geht als es eben bei Huſten ſein kann,
                    Ihnen beiden\pwindex{Schnitzler, Olga 17.01.1882 – 13.01.1970@\textsc{Schnitzler, Olga} (17.01.1882 – 13.01.1970), \emph{Schauspielerin, Sängerin}|pwv} aber in jeder
                    Hinſicht glänzend.\pend
           \pstart Herzlich\spacefill\mbox{DrBurc\textcolor{gray}{khard}}\pend{}\endnumbering\briefempfaengerindex{Schnitzler, Arthur@\textsc{Schnitzler, Arthur}!zzzBurckhard, Max Eugen@\emph{von Max Eugen Burckhard}!1909-07-122@{12. 7. 1909}|)be}\mylabel{h}\end{ledgroupsized}  \newcommand{\dateiname}{L01856}\newcommand{\titel}{Max Burckhard an Arthur Schnitzler, 12. 7. 1909}\newcommand{\editorInnen}{Martin Anton Müller und Gerd-Hermann Susen}\input{../tex-inputs/latex-pdf-abspann}
      