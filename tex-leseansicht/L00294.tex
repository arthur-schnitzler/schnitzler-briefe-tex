%% latex-leseansicht-vorspann.tex
%% Vorspann für die Leseansicht.
%% Lädt die gemeinsame Datei latex-vorspann.tex mit nicht gesetztem Schalter.

\newif\ifkorrekturansicht
\korrekturansichtfalse

\input{../tex-inputs/latex-vorspann}


\section[Arthur Schnitzler an Hugo von Hofmannsthal, {{[}}18. 1. 1894{{]}}]{L00294 Arthur Schnitzler an Hugo von Hofmannsthal, {[}18. 1. 1894{]}}
\nopagebreak\mylabel{L00294v}
\rehead{ }\normalsize\beginnumbering\briefempfaengerindex{Hofmannsthal, Hugo von@\textsc{Hofmannsthal, Hugo von}!zzzSchnitzler, Arthur@\emph{von Arthur Schnitzler}!1894-01-181@{{[}18. 1. 1894{]}}|(be}
\toendnotes[C]{\smallbreak\pagebreak[2]}
\correspDesc{Versand  durch Arthur Schnitzler am [18. 1. 1894] in Wien
\newline{}Erhalt  durch Hugo von Hofmannsthal im Zeitraum [18. 1. 1894
                  – 22. 1. 1894?] in Wien}\toendnotes[C]{\smallbreak}
\Standort{FDH, Hs-30885,41.}
\physDesc{Brief, 1 Blatt, 3 Seiten, 735 Zeichen (Briefpapier mit Trauerrand)
\newline{}Handschrift: Bleistift, deutsche Kurrent
\newline{}Ordnung: mit Bleistift von Schnitzler mutmaßlich bei der Durchsicht der Korrespondenz
                                    1929  datiert: »18/1 94« }
\buchAbdrucke{\weitereDrucke{Hugo von Hofmannsthal, Arthur Schnitzler: \emph{Briefwechsel}. Herausgegeben von Therese Nickl und Heinrich Schnitzler. Frankfurt am Main: \emph{S. Fischer} 1964, S. 49.} }
\pstart
           \raggedleft{}{\pb}\uline{Donnerſtag.}\pend
           
\pstart{}Lieber Hugo,\pend\vspace{0.5em}
\pstart
           vielleicht ko{\geminationm}en die beiliegenden 3 Ka{\geminationm}ermuſikabende Ihrem Muſikbedürfnis entgegen. Iſt’s
               Ihnen alſo recht,{ }ſo möchte ich Ihnen einen Sitz neben mir, womöglich Gallerie
               nehmen. – Hier iſt der Sitz für {\pb}\textsc{Mounet Sully}\pwindex{Mounet-Sully, Jean 27.\,2.\,1841 Bergerac – 1.\,3.\,1916 Paris@\textsc{Mounet-Sully, Jean} (27.\,2.\,1841 Bergerac – 1.\,3.\,1916 Paris), \emph{Schauspieler, Rechtsanwalt}|pw}; 4 fl. 20; was freilich für einen armen Dichter viel iſt. –\pend
           
\pstart
           So{\geminationn}tag werd ich vor dem Theater kaum zu Richard\pwindex{Beer-Hofmann, Richard 11.\,7.\,1866 Wien – 26.\,9.\,1945 New York City@\textsc{Beer-Hofmann, Richard} (11.\,7.\,1866 Wien – 26.\,9.\,1945 New York City), \emph{Schriftsteller}|pw} kö{\geminationn}en;
               (höchſtens Sie \introOben{}von dort\introOben{} abholen), weil ich vorher irgendwo
               (bei Wetzler\pwindex{Wetzler, Bernhard 24.\,6.\,1839 Metzling – 10.\,5.\,1922 Wien@\textsc{Wetzler, Bernhard} (24.\,6.\,1839 Metzling – 10.\,5.\,1922 Wien), \emph{Industrieller, Unternehmer, Bankier}|pw}’s) einen Thee trinken muſs. –\pend
           
\pstart
           Herentgegen müßte es mit dem Teufel zugehen {\pb}we{\geminationn} ich nicht heute Abends um 10 ins Café Central\oindex{Wien@\textbf{Wien}!I., Innere Stadt@\textbf{I., Innere Stadt}!Café Central@\textbf{Café Central}, \emph{Kaffeehaus}|pw} käme, wo wir dann immer ein Stündchen
               plaudern könnten – freilich nur wenn Sie dort{ }ſind. Für alle Fälle pneumatiſiren Sie
               mir wegen der Ka{\geminationm}ermuſik und behalten mich in
               freundlicher Erinnerung.\pend
           \pstart Ihr \spacefill\mbox{Arthur}\pend{}\selectlanguage{ngerman}\endnumbering\briefempfaengerindex{Hofmannsthal, Hugo von@\textsc{Hofmannsthal, Hugo von}!zzzSchnitzler, Arthur@\emph{von Arthur Schnitzler}!1894-01-181@{{[}18. 1. 1894{]}}|)be}\mylabel{L00294h}  \newcommand{\dateiname}{L00294}\newcommand{\titel}{Arthur Schnitzler an Hugo von Hofmannsthal, [18. 1. 1894]}\newcommand{\editorInnen}{Martin Anton Müller und Gerd-Hermann Susen}%% latex-leseansicht-abspann.tex
%% Abspann für die Leseansicht.
%% Der Schalter \ifkorrekturansicht ist bereits durch den Vorspann gesetzt.

%% latex-abspann.tex
%% Gemeinsamer Abspann für Korrekturansicht und Leseansicht.
%% Setzt den Schalter \ifkorrekturansicht voraus (gesetzt in den
%% einbindenden Dateien latex-korrekturansicht-abspann.tex bzw.
%% latex-leseansicht-abspann.tex).
%% ---------------------------------------------------------------

\normalsize

% Das esempio-Environment wird nur in der Leseansicht benötigt
\ifkorrekturansicht\else
\newenvironment{esempio}[3]%
{
    \vspace{1.5ex}
    \rlap{\underline{#1}}
    \par
    \setlength{\parindent}{0cm}
    \nopagebreak
    \leftskip=#2cm
    \rightskip=#3cm
}
{
    \par
}
\fi

\doendnotes{C}
\bigskip
\vfill

\clearpage

\footnotesize

\ifkorrekturansicht
  \lohead{\textsc{register}}
\fi

% theindex-Environment neu definieren ohne reledmac
\makeatletter
\renewenvironment{theindex}{%
  \ifkorrekturansicht
    \section*{\indexname}%
  \else
    \subsubsection*{Index der erwähnten Entitäten}%
  \fi
  \setlength{\parindent}{0pt}%
  \setlength{\parskip}{0pt plus 0.3pt}%
  \let\item\@idxitem
}{%
  \ifkorrekturansicht\clearpage\fi
}
\makeatother

\IfFileExists{\jobname-pw.ind}{\input{\jobname-pw.ind}}{}

% Quellenangabe nur in der Leseansicht
\ifkorrekturansicht\else
% Fallback-Definitionen, falls die .tex-Datei \titel etc. nicht gesetzt hat
\providecommand{\titel}{}
\providecommand{\editorInnen}{}
\providecommand{\dateiname}{\jobname}

\vspace{3cm}

\vfill

\footnotesize
\textsc{Quelle}: \titel. Herausgegeben von {\editorInnen}. In: \emph{Arthur Schnitzler: Briefwechsel mit Autorinnen und Autoren}.
 Digitale Edition, https://schnitzler-briefe.acdh.oeaw.ac.at/{\dateiname}.html (Stand \today)
\fi

\end{document}


