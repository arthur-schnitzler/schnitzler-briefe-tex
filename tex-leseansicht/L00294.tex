%% latex-korrekturansicht-vorspann.tex
%% Vorspann für die Korrekturansicht.
%% Lädt die gemeinsame Datei latex-vorspann.tex mit gesetztem Schalter.

\newif\ifkorrekturansicht
\korrekturansichttrue

\input{../tex-inputs/latex-vorspann}


\section[Arthur Schnitzler an Hugo von Hofmannsthal, {[}18. 1. 1894{]}]{L00294 Arthur Schnitzler an Hugo von Hofmannsthal, {[}18. 1. 1894{]}}
\nopagebreak\mylabel{L00294v}
\rehead{ }\normalsize\beginnumbering\briefempfaengerindex{Hofmannsthal, Hugo von@\textsc{Hofmannsthal, Hugo von}!zzzSchnitzler, Arthur@\emph{von Arthur Schnitzler}!1894-01-181@{{[}18. 1. 1894{]}}|(be}
\toendnotes[C]{\smallbreak\pagebreak[2]}\Standort{FDH, Hs-30885,41.}
\physDesc{Brief, 1 Blatt, 3 Seiten, 735 Zeichen (Briefpapier mit Trauerrand)
\newline{}Handschrift: Bleistift, deutsche Kurrent
\newline{}Ordnung: mit Bleistift von Schnitzler mutmaßlich bei der Durchsicht der Korrespondenz
                                    1929  datiert: »18/1 94« }
\buchAbdrucke{\weitereDrucke{Hugo von Hofmannsthal, Arthur Schnitzler: \emph{Briefwechsel}. Frankfurt am Main: \emph{S. Fischer} 1964, S. 49.} }
\pstart
           \raggedleft{}{\pb}\uline{Donnerſtag.}\pend
           
\pstart{}Lieber Hugo,\pend\vspace{0.5em}
\pstart
           vielleicht ko{\geminationm}en die beiliegenden 3 Ka{\geminationm}ermuſikabende Ihrem Muſikbedürfnis entgegen. Iſt’s
               Ihnen alſo recht, ſo möchte ich Ihnen einen Sitz neben mir, womöglich Gallerie
               nehmen. – Hier iſt der Sitz für {\pb}\textsc{Mounet Sully}\pwindex{Mounet-Sully, Jean 27.02.1841 – 01.03.1916@\textsc{Mounet-Sully, Jean} (27.02.1841 – 01.03.1916), \emph{Schauspieler/Schauspielerin, Rechtsanwalt/Rechtsanwältin}|pw}; 4 fl. 20; was freilich für einen armen Dichter viel iſt. –\pend
           
\pstart
           So{\geminationn}tag werd ich vor dem Theater kaum zu Richard\pwindex{Beer-Hofmann, Richard 1866-07-11 – 1945-09-26@\textsc{Beer-Hofmann, Richard} (1866-07-11 – 1945-09-26), \emph{Schriftsteller/Schriftstellerin}|pw} kö{\geminationn}en;
               (höchſtens Sie \introOben{}von dort\introOben{} abholen), weil ich vorher irgendwo
               (bei Wetzler\pwindex{Wetzler, Bernhard 24.06.1839 – 10.05.1922@\textsc{Wetzler, Bernhard} (24.06.1839 – 10.05.1922), \emph{Industrieller/Industrielle, Unternehmer/Unternehmerin, Bankier/Bankierin}|pw}’s) einen Thee trinken muſs. –\pend
           
\pstart
           Herentgegen müßte es mit dem Teufel zugehen {\pb}we{\geminationn} ich nicht heute Abends um 10 ins Café Central\oindex{Cafe Central@\textbf{Café Central}, \emph{Kaffeehaus (K.KAF)}|pw} käme, wo wir dann immer ein Stündchen
               plaudern könnten – freilich nur wenn Sie dort ſind. Für alle Fälle pneumatiſiren Sie
               mir wegen der Ka{\geminationm}ermuſik und behalten mich in
               freundlicher Erinnerung.\pend
           \pstart Ihr \spacefill\mbox{Arthur}\pend{}\selectlanguage{ngerman}\endnumbering\briefempfaengerindex{Hofmannsthal, Hugo von@\textsc{Hofmannsthal, Hugo von}!zzzSchnitzler, Arthur@\emph{von Arthur Schnitzler}!1894-01-181@{{[}18. 1. 1894{]}}|)be}\mylabel{L00294h}  \normalsize

\doendnotes{C}
\bigskip
\vfill

\clearpage

\footnotesize

\lohead{\textsc{register}}

% Definiere theindex-Environment komplett neu ohne reledmac
\makeatletter
\renewenvironment{theindex}{%
  \section*{\indexname}%
  \setlength{\parindent}{0pt}%
  \setlength{\parskip}{0pt plus 0.3pt}%
  \let\item\@idxitem
}{%
  \clearpage
}
\makeatother

\IfFileExists{\jobname-pw.ind}{\input{\jobname-pw.ind}}{}

\end{document}

      