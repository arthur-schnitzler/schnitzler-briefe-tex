%% latex-leseansicht-vorspann.tex
%% Vorspann für die Leseansicht.
%% Lädt die gemeinsame Datei latex-vorspann.tex mit nicht gesetztem Schalter.

\newif\ifkorrekturansicht
\korrekturansichtfalse

\input{../tex-inputs/latex-vorspann}

\begin{center}
            \textcolor{red}{ENTWURF, NICHT FERTIG KORRIGIERT}
                      \end{center}
            
         
         \renewcommand{\erwaehntePersonen}{Personen: Lili Cappellini, Paul Goldmann, Eva Marie Goldmann, Rudolf Lothar, Olga Schnitzler, Heinrich Schnitzler}
         \renewcommand{\erwaehnteInstitutionen}{Institutionen: k. k. Post- und Telegraphenverwaltung}
         \renewcommand{\erwaehnteOrte}{Orte: Bahnhof Payerbach-Reichenau, Berlin, Deutsches Theater Berlin, Edlach, Hotel Edlacherhof, Niederösterreich, Semmering, Wien}
         \renewcommand{\erwaehnteWerke}{Werke: Faust bei Reinhardt, Faust. Eine Tragödie, Pester Lloyd}
               \section[ Paul Goldmann an Arthur Schnitzler, 26. 4. 1909]{ Paul Goldmann an Arthur Schnitzler, 26. 4. 1909}\nopagebreak\mylabel{v}\rehead{ }\begin{ledgroupsized}[t]{13cm}\normalsize\beginnumbering\briefempfaengerindex{Schnitzler, Arthur@\textsc{Schnitzler, Arthur}!zzzGoldmann, Paul@\emph{von Paul Goldmann}!1909-04-261@{26. 4. 1909}|(be} \toendnotes[C]{\smallbreak\pagebreak[2]} \Standort{DLA, A:Schnitzler, HS.NZ85.1.3175.}
\physDesc{Brief, 1 Blatt, 3 Seiten, 1022 Zeichen
\newline{}Handschrift: schwarze Tinte, deutsche Kurrent
\newline{}Schnitzler: mit Bleistift »Goldm{[}ann{]}\pwindex{Goldmann, Paul 31.01.1865 – 25.09.1935@\textsc{Goldmann, Paul} (31.01.1865 – 25.09.1935), \emph{Schriftsteller, Journalist}|pw}« vermerkt }\toendnotes[C]{\smallbreak}\pstart
           \noindent{}{\pb}\textcolor{gray}{\textbf{HÔTEL EDLACHERHOF\oindex{Hotel Edlacherhof@\textbf{Hotel Edlacherhof}|pw}}}\pend
           \pstart
           \raggedleft{}\textcolor{gray}{\textbf{IN EDLACH\oindex{Edlach@\textbf{Edlach}|pw}, N.-Ö.\oindex{Niederoesterreich@\textbf{Niederösterreich}|pw}}}\pend
           \pstart
           \noindent{}\raggedleft{}\textcolor{gray}{\textbf{\textbf{Südbahnstation Payerbach-Reichenau\oindex{Bahnhof Payerbach-Reichenau@\textbf{Bahnhof Payerbach-Reichenau}|pw}}}}\pend
           \pstart
           \noindent{}\textcolor{gray}{\textbf{Telegramm-Adresse:}}{\\}\textcolor{gray}{\textbf{\textbf{EDLACHERHOF\oindex{Hotel Edlacherhof@\textbf{Hotel Edlacherhof}|pw}, EDLACH\oindex{Edlach@\textbf{Edlach}|pw}.}}}{\\}\textcolor{gray}{\textbf{INTERURBAN TELEPHON}}{\\}\textcolor{gray}{\textbf{\textbf{EDLACH\oindex{Edlach@\textbf{Edlach}|pw}
                        Nr. 1.}}}\pend
           \pstart
           \textcolor{gray}{\textbf{K. k. Post- und Telegraphen-Amt\orgindex{k. k. Post- und Telegraphenverwaltung@k. k. Post- und Telegraphenverwaltung|pw}}}\hfill \textcolor{gray}{\textbf{Edlacherhof\oindex{Hotel Edlacherhof@\textbf{Hotel Edlacherhof}|pw},}}{ }26. 4. 09.\pend
           \pstart
           \textcolor{gray}{\textbf{Edlach\oindex{Edlach@\textbf{Edlach}|pw}.}}\pend
           \pstart\center{}Lieber Freund,\pend\pstart
           Beifolgendes \label{K_L03468-1v}\edtext{Feuilleton\pwindex{Lothar, Rudolf 23.2.1865 – 2.10.1943@\textsc{Lothar, Rudolf} (23.2.1865 – 2.10.1943), \emph{Schriftsteller, Journalist, Theaterdirektor}!Faust bei Reinhardt1909-04-22@\strich\emph{Faust bei Reinhardt} {[}1909-04-22{]}|pwv}}{\lemma{\textnormal{\emph{Feuilleton}}}\Cendnote{\textnormal{Beilage nicht erhalten. Die Bezugnahme
                  auf Rudolf Lothar\pwindex{Lothar, Rudolf 23.2.1865 – 2.10.1943@\textsc{Lothar, Rudolf} (23.2.1865 – 2.10.1943), \emph{Schriftsteller, Journalist, Theaterdirektor}|pwk}: \emph{Faust bei Reinhardt}\pwindex{Lothar, Rudolf 23.2.1865 – 2.10.1943@\textsc{Lothar, Rudolf} (23.2.1865 – 2.10.1943), \emph{Schriftsteller, Journalist, Theaterdirektor}!Faust bei Reinhardt1909-04-22@\strich\emph{Faust bei Reinhardt} {[}1909-04-22{]}|pwk}. In: \emph{Pester Lloyd}\pwindex{?? Werk@Nicht ermittelte Verfasserinnen und Verfasser!Pester Lloyd@\emph{Pester Lloyd}|pwk}, Jg. 46, Nr. 95, 22. 4. 1909, Morgenblatt, S. 1–2 scheint zweifellos. Das Feuilleton\pwindex{Lothar, Rudolf 23.2.1865 – 2.10.1943@\textsc{Lothar, Rudolf} (23.2.1865 – 2.10.1943), \emph{Schriftsteller, Journalist, Theaterdirektor}!Faust bei Reinhardt1909-04-22@\strich\emph{Faust bei Reinhardt} {[}1909-04-22{]}|pwkv} beginnt wie
                  folgt: »Fünfundzwanzig Jahre sind es
                        her, da nahmen zwei junge Leute, die Poeten werden wollten, Abschied von Wien\oindex{Wien@\textbf{Wien}|pw}. Sie hatten die Absicht, die Welt zu
                        sehen und ihr erstes Ziel war Berlin\oindex{Berlin@\textbf{Berlin}|pw}.
                        Der eine dieser beiden Wanderer war Arthur
                           Schnitzler\pwindex{Schnitzler, Arthur 15.05.1862 – 21.10.1931@\textsc{Schnitzler, Arthur} (15.05.1862 – 21.10.1931), \emph{Schriftsteller, Mediziner}|pw}, der andere war ich\pwindex{Lothar, Rudolf 23.2.1865 – 2.10.1943@\textsc{Lothar, Rudolf} (23.2.1865 – 2.10.1943), \emph{Schriftsteller, Journalist, Theaterdirektor}|pwv}. Wir kamen mittags in Berlin\oindex{Berlin@\textbf{Berlin}|pw} an und saßen abends schon im Theater. Im Deutschen Theater\oindex{Deutsches Theater Berlin@\textbf{Deutsches Theater Berlin}|pw}.\pwindex{Lothar, Rudolf 23.2.1865 – 2.10.1943@\textsc{Lothar, Rudolf} (23.2.1865 – 2.10.1943), \emph{Schriftsteller, Journalist, Theaterdirektor}!Faust bei Reinhardt1909-04-22@\strich\emph{Faust bei Reinhardt} {[}1909-04-22{]}|pwv}« (S. 1) Lothar\pwindex{Lothar, Rudolf 23.2.1865 – 2.10.1943@\textsc{Lothar, Rudolf} (23.2.1865 – 2.10.1943), \emph{Schriftsteller, Journalist, Theaterdirektor}|pwk}
                  erinnerte sich an die gemeinsame Berlin\oindex{Berlin@\textbf{Berlin}|pwk}-Reise
                  im Frühjahr 1888. Schnitzler\pwindex{Schnitzler, Arthur 15.05.1862 – 21.10.1931@\textsc{Schnitzler, Arthur} (15.05.1862 – 21.10.1931), \emph{Schriftsteller, Mediziner}|pwk} war bereits ein paar Tage früher in Berlin\oindex{Berlin@\textbf{Berlin}|pwk} angekommen, Lothar\pwindex{Lothar, Rudolf 23.2.1865 – 2.10.1943@\textsc{Lothar, Rudolf} (23.2.1865 – 2.10.1943), \emph{Schriftsteller, Journalist, Theaterdirektor}|pwk} kam am 12. 4. 1888 an und an diesem Tag besuchten beide \emph{Faust}\pwindex{\textcolor{red}{\textsuperscript{XXXX1 indx}}!Faust. Eine Tragoedie1808@\strich\emph{Faust. Eine Tragödie} {[}1808{]}|pwk} im Deutschen
                     Theater\oindex{Deutsches Theater Berlin@\textbf{Deutsches Theater Berlin}|pwk}.}}}\label{K_L03468-1h} von \textsc{Rudolf Lothar\pwindex{Lothar, Rudolf 23.2.1865 – 2.10.1943@\textsc{Lothar, Rudolf} (23.2.1865 – 2.10.1943), \emph{Schriftsteller, Journalist, Theaterdirektor}|pw}} wird Dich vielleicht ebenſo amüſiren, wie es mich amüſirt hat. \pend
           \pstart
           Wir\pwindex{Goldmann, Eva Marie 27.10.1877 – 02.11.1937@\textsc{Goldmann, Eva Marie} (27.10.1877 – 02.11.1937)|pwv} haben acht Tage der Ruhe
               in dem reizenden Edlach\oindex{Edlach@\textbf{Edlach}|pw} verbracht, das ich Dir
               nicht dringend genug \label{K_L03468-2v}\edtext{empfehlen}{\lemma{\textnormal{\emph{empfehlen}}}\Cendnote{\textnormal{Schnitzler\pwindex{Schnitzler, Arthur 15.05.1862 – 21.10.1931@\textsc{Schnitzler, Arthur} (15.05.1862 – 21.10.1931), \emph{Schriftsteller, Mediziner}|pwk} kannte Edlach\oindex{Edlach@\textbf{Edlach}|pwk} und war hier mehrfach abgestiegen.}}}\label{K_L03468-2h} kann, wenn
               Du fern von allem mondainen Getriebe (wie es in den \textsc{Hotels}
               auf dem Gipfel des Semmering\oindex{Semmering@\textbf{Semmering}|pw} herrſcht) in
               erfriſchender Luft {\pb}Dich eine Zeit lang erholen
               willſt. Heut kehren wir nach Wien\oindex{Wien@\textbf{Wien}|pw} zurück, von wo aus wir in einigen Tagen die Rückreiſe nach
                  Berlin\oindex{Berlin@\textbf{Berlin}|pw} antreten.\pend
           \pstart
           Aufſuchen konnte ich Dich vor meiner Abreiſe nach Edlach\oindex{Edlach@\textbf{Edlach}|pw} nicht mehr, weil ich buchſtäblich keine Stunde frei hatte.\pend
           \pstart
           Die \label{K_L03468-3v}\edtext{Spannung}{\lemma{\textnormal{\emph{Spannung}}}\Cendnote{\textnormal{Im Detail ist das unklar, doch scheint es naheliegend, dass
                     Eva Goldmann\pwindex{Goldmann, Eva Marie 27.10.1877 – 02.11.1937@\textsc{Goldmann, Eva Marie} (27.10.1877 – 02.11.1937)|pwk} dabei war, als ihr Mann am
                     12. 4. 1909Schnitzler\pwindex{Schnitzler, Arthur 15.05.1862 – 21.10.1931@\textsc{Schnitzler, Arthur} (15.05.1862 – 21.10.1931), \emph{Schriftsteller, Mediziner}|pwk} zuhause besuchte.}}}\label{K_L03468-3h} zwiſchen
               unſeren beiderſeitigen Frauen\pwindex{Goldmann, Eva Marie 27.10.1877 – 02.11.1937@\textsc{Goldmann, Eva Marie} (27.10.1877 – 02.11.1937)|pwv}\pwindex{Schnitzler, Olga 17.01.1882 – 13.01.1970@\textsc{Schnitzler, Olga} (17.01.1882 – 13.01.1970), \emph{Schauspielerin, Sängerin}|pwv} wird ſich hoffentlich beilegen laſſen. Jedenfalls aber wird zwiſchen
               uns Beiden hoffentlich Alles ſo bleiben, wie bisher.\pend
           \pstart
           Ich wünſche Dir einen \label{K_L03468-4v}\edtext{zweiten
                  Sohn}{\lemma{\textnormal{\emph{zweiten
                  Sohn}}}\Cendnote{\textnormal{Olga Schnitzler\pwindex{Schnitzler, Olga 17.01.1882 – 13.01.1970@\textsc{Schnitzler, Olga} (17.01.1882 – 13.01.1970), \emph{Schauspielerin, Sängerin}|pwk} war mit Lili Schnitzler\pwindex{Cappellini, Lili 13.09.1909 – 26.07.1928@\textsc{Cappellini, Lili} (13.09.1909 – 26.07.1928)|pwk} schwanger. Sie wurde am 13. 9. 1909
                  geboren.}}}\label{K_L03468-4h}, der \strikeout{ſ\textcolor{gray}{o}} ein {\pb}ebenſo prächtiger Burſch ſein möge, wie
               der erſte\pwindex{Schnitzler, Heinrich 09.08.1902 – 12.07.1982@\textsc{Schnitzler, Heinrich} (09.08.1902 – 12.07.1982), \emph{Regisseur, Schauspieler}|pwv}, – oder, wenn Du
               Dir eine Tochter wünſcheſt, ſo bin ich auch mit einer Tochter einverſtanden, – u. bin
               mit herzlichen Grüßen (auch von meiner Frau\pwindex{Goldmann, Eva Marie 27.10.1877 – 02.11.1937@\textsc{Goldmann, Eva Marie} (27.10.1877 – 02.11.1937)|pwv})\pend
           \pstart
           Dein {\\[\baselineskip]}\spacefill\mbox{Paul Goldmann.}\pend
           \leftskip=0em{}
         
         \endnumbering\mylabel{h}\end{ledgroupsized}  \newcommand{\dateiname}{L03468}\newcommand{\titel}{Paul Goldmann an Arthur Schnitzler, 26. 4. 1909}\newcommand{\editorInnen}{Martin Anton Müller und Laura Untner}%% latex-leseansicht-abspann.tex
%% Abspann für die Leseansicht.
%% Der Schalter \ifkorrekturansicht ist bereits durch den Vorspann gesetzt.

%% latex-abspann.tex
%% Gemeinsamer Abspann für Korrekturansicht und Leseansicht.
%% Setzt den Schalter \ifkorrekturansicht voraus (gesetzt in den
%% einbindenden Dateien latex-korrekturansicht-abspann.tex bzw.
%% latex-leseansicht-abspann.tex).
%% ---------------------------------------------------------------

\normalsize

% Das esempio-Environment wird nur in der Leseansicht benötigt
\ifkorrekturansicht\else
\newenvironment{esempio}[3]%
{
    \vspace{1.5ex}
    \rlap{\underline{#1}}
    \par
    \setlength{\parindent}{0cm}
    \nopagebreak
    \leftskip=#2cm
    \rightskip=#3cm
}
{
    \par
}
\fi

\doendnotes{C}
\bigskip
\vfill

\clearpage

\footnotesize

\ifkorrekturansicht
  \lohead{\textsc{register}}
\fi

% theindex-Environment neu definieren ohne reledmac
\makeatletter
\renewenvironment{theindex}{%
  \ifkorrekturansicht
    \section*{\indexname}%
  \else
    \subsubsection*{Index der erwähnten Entitäten}%
  \fi
  \setlength{\parindent}{0pt}%
  \setlength{\parskip}{0pt plus 0.3pt}%
  \let\item\@idxitem
}{%
  \ifkorrekturansicht\clearpage\fi
}
\makeatother

\IfFileExists{\jobname-pw.ind}{\input{\jobname-pw.ind}}{}

% Quellenangabe nur in der Leseansicht
\ifkorrekturansicht\else
% Fallback-Definitionen, falls die .tex-Datei \titel etc. nicht gesetzt hat
\providecommand{\titel}{}
\providecommand{\editorInnen}{}
\providecommand{\dateiname}{\jobname}

\vspace{3cm}

\vfill

\footnotesize
\textsc{Quelle}: \titel. Herausgegeben von {\editorInnen}. In: \emph{Arthur Schnitzler: Briefwechsel mit Autorinnen und Autoren}.
 Digitale Edition, https://schnitzler-briefe.acdh.oeaw.ac.at/{\dateiname}.html (Stand \today)
\fi

\end{document}


      