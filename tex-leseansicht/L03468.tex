%% latex-leseansicht-vorspann.tex
%% Vorspann für die Leseansicht.
%% Lädt die gemeinsame Datei latex-vorspann.tex mit nicht gesetztem Schalter.

\newif\ifkorrekturansicht
\korrekturansichtfalse

\input{../tex-inputs/latex-vorspann}


\section[ Paul Goldmann an Arthur Schnitzler, 26. 4. 1909]{L03468 Paul Goldmann an Arthur Schnitzler,  26. 4. 1909}
\nopagebreak\mylabel{L03468v}
\rehead{ }\normalsize\beginnumbering\briefempfaengerindex{Schnitzler, Arthur@\textsc{Schnitzler, Arthur}!zzzGoldmann, Paul@\emph{von Paul Goldmann}!1909-04-261@{26. 4. 1909}|(be}
\toendnotes[C]{\smallbreak\pagebreak[2]}
\correspDesc{Versand  durch Paul Goldmann am 26. 4. 1909 in Edlach
\newline{}Erhalt  durch Arthur Schnitzler im Zeitraum [27. 4. 1909
                  – 1. 5. 1909?] in Wien}\toendnotes[C]{\smallbreak}
\Standort{DLA, A:Schnitzler, HS.NZ85.1.3175.}
\physDesc{Brief, 1 Blatt, 3 Seiten, 1022 Zeichen
\newline{}Handschrift: schwarze Tinte, deutsche Kurrent
\newline{}Schnitzler: mit Bleistift »Goldm{[}ann{]}\pwindex{Goldmann, Paul 31.\,1.\,1865 Breslau – 25.\,9.\,1935 Wien@\textsc{Goldmann, Paul} (31.\,1.\,1865 Breslau – 25.\,9.\,1935 Wien), \emph{Schriftsteller, Journalist}|pw}« vermerkt }\toendnotes[C]{\smallbreak}
\pstart
           {\pb}\textcolor{gray}{\textbf{HÔTEL EDLACHERHOF\oindex{Hotel Edlacherhof@\textbf{Hotel Edlacherhof}, \emph{Hotel}|pw}}}\pend
           
\pstart
           \raggedleft{}\textcolor{gray}{\textbf{IN EDLACH\oindex{Edlach@\textbf{Edlach}|pw}, N.-Ö.\oindex{Niederösterreich@\textbf{Niederösterreich}, \emph{Land}|pw}}}\pend
           
\pstart
           \raggedleft{}\textcolor{gray}{\textbf{\textbf{Südbahnstation Payerbach-Reichenau\oindex{Bahnhof Payerbach-Reichenau@\textbf{Bahnhof Payerbach-Reichenau}, \emph{Bahnhofsgebäude}|pw}}}}\pend
           
\pstart
           \textcolor{gray}{\textbf{Telegramm-Adresse:}}{\\}\textcolor{gray}{\textbf{\textbf{EDLACHERHOF\oindex{Hotel Edlacherhof@\textbf{Hotel Edlacherhof}, \emph{Hotel}|pw}, EDLACH\oindex{Edlach@\textbf{Edlach}|pw}.}}}{\\}\textcolor{gray}{\textbf{INTERURBAN. TELEPHON}}{\\}\textcolor{gray}{\textbf{\textbf{EDLACH\oindex{Edlach@\textbf{Edlach}|pw} Nr. 1.}}}\pend
           
\pstart
           \textcolor{gray}{\textbf{K. k. Post- und Telegraphen-Amt\orgindex{k. k. Post- und Telegraphenverwaltung@k. k. Post- und Telegraphenverwaltung|pw}}}\hfill \textcolor{gray}{\textbf{Edlacherhof\oindex{Hotel Edlacherhof@\textbf{Hotel Edlacherhof}, \emph{Hotel}|pw},}}{ }26. 4. 09.\pend
           
\pstart
           \textcolor{gray}{\textbf{Edlach\oindex{Edlach@\textbf{Edlach}|pw}.}}\pend
           
\pstart\center{}Lieber Freund,\pend\vspace{0.5em}
\pstart
           Beifolgendes \label{K_L03468-1v}\edtext{Feuilleton\pwindex{Lothar, Rudolf 23.\,2.\,1865 Budapest – 2.\,10.\,1943 ebd.@\textsc{Lothar, Rudolf} (23.\,2.\,1865 Budapest – 2.\,10.\,1943 ebd.), \emph{Schriftsteller, Journalist, Theaterdirektor}!Faust bei Reinhardt@\strich\emph{Faust bei Reinhardt}|pwv}}{\lemma{\textnormal{\emph{Feuilleton}}}\Cendnote{\textnormal{Beilage nicht erhalten. Die Bezugnahme
                  auf Rudolf Lothar\pwindex{Lothar, Rudolf 23.\,2.\,1865 Budapest – 2.\,10.\,1943 ebd.@\textsc{Lothar, Rudolf} (23.\,2.\,1865 Budapest – 2.\,10.\,1943 ebd.), \emph{Schriftsteller, Journalist, Theaterdirektor}|pwk}: \emph{Faust bei Reinhardt}\pwindex{Lothar, Rudolf 23.\,2.\,1865 Budapest – 2.\,10.\,1943 ebd.@\textsc{Lothar, Rudolf} (23.\,2.\,1865 Budapest – 2.\,10.\,1943 ebd.), \emph{Schriftsteller, Journalist, Theaterdirektor}!Faust bei Reinhardt@\strich\emph{Faust bei Reinhardt}|pwk}. In: \emph{Pester Lloyd}\pwindex{Pester Lloyd@\emph{Pester Lloyd}|pwk}, Jg. 46, Nr. 95, 22. 4. 1909, Morgenblatt, S. 1–2 scheint zweifellos. Das Feuilleton\pwindex{Lothar, Rudolf 23.\,2.\,1865 Budapest – 2.\,10.\,1943 ebd.@\textsc{Lothar, Rudolf} (23.\,2.\,1865 Budapest – 2.\,10.\,1943 ebd.), \emph{Schriftsteller, Journalist, Theaterdirektor}!Faust bei Reinhardt@\strich\emph{Faust bei Reinhardt}|pwkv} beginnt wie
                  folgt: »Fünfundzwanzig Jahre sind es
                        her, da nahmen zwei junge Leute, die Poeten werden wollten, Abschied von Wien\oindex{Wien@\textbf{Wien}, \emph{Verwaltungsgebiet}|pw}. Sie hatten die Absicht, die Welt zu
                        sehen und ihr erstes Ziel war Berlin\oindex{Berlin@\textbf{Berlin}, \emph{Hauptstadt}|pw}.
                        Der eine dieser beiden Wanderer war Arthur
                           Schnitzler, der andere war ich\pwindex{Lothar, Rudolf 23.\,2.\,1865 Budapest – 2.\,10.\,1943 ebd.@\textsc{Lothar, Rudolf} (23.\,2.\,1865 Budapest – 2.\,10.\,1943 ebd.), \emph{Schriftsteller, Journalist, Theaterdirektor}|pwv}. Wir kamen mittags in Berlin\oindex{Berlin@\textbf{Berlin}, \emph{Hauptstadt}|pw} an und saßen abends schon im Theater. Im Deutschen Theater\oindex{Deutsches Theater Berlin@\textbf{Deutsches Theater Berlin}, \emph{Theater}|pw}.\pwindex{Lothar, Rudolf 23.\,2.\,1865 Budapest – 2.\,10.\,1943 ebd.@\textsc{Lothar, Rudolf} (23.\,2.\,1865 Budapest – 2.\,10.\,1943 ebd.), \emph{Schriftsteller, Journalist, Theaterdirektor}!Faust bei Reinhardt@\strich\emph{Faust bei Reinhardt}|pwv}« (S. 1) Lothar\pwindex{Lothar, Rudolf 23.\,2.\,1865 Budapest – 2.\,10.\,1943 ebd.@\textsc{Lothar, Rudolf} (23.\,2.\,1865 Budapest – 2.\,10.\,1943 ebd.), \emph{Schriftsteller, Journalist, Theaterdirektor}|pwk}
                  erinnert sich darin an die gemeinsame Berlin\oindex{Berlin@\textbf{Berlin}, \emph{Hauptstadt}|pwk}-Reise
                  im Frühjahr 1888. Schnitzler war bereits ein paar Tage vor ihm in Berlin\oindex{Berlin@\textbf{Berlin}, \emph{Hauptstadt}|pwk} angekommen, Lothar\pwindex{Lothar, Rudolf 23.\,2.\,1865 Budapest – 2.\,10.\,1943 ebd.@\textsc{Lothar, Rudolf} (23.\,2.\,1865 Budapest – 2.\,10.\,1943 ebd.), \emph{Schriftsteller, Journalist, Theaterdirektor}|pwk} kam am 12. 4. 1888 an und an diesem Tag besuchten beide \emph{Faust}\pwindex{\textcolor{red}{\textsuperscript{XXXX indx1}}!Faust. Eine Tragödie@\strich\emph{Faust. Eine Tragödie}|pwk} im Deutschen
                     Theater\oindex{Deutsches Theater Berlin@\textbf{Deutsches Theater Berlin}, \emph{Theater}|pwk}.}}}\label{K_L03468-1} von \textsc{Rudolf Lothar\pwindex{Lothar, Rudolf 23.\,2.\,1865 Budapest – 2.\,10.\,1943 ebd.@\textsc{Lothar, Rudolf} (23.\,2.\,1865 Budapest – 2.\,10.\,1943 ebd.), \emph{Schriftsteller, Journalist, Theaterdirektor}|pw}} wird Dich vielleicht ebenſo amüſiren, wie es mich amüſirt hat.\pend
           
\pstart
           Wir\pwindex{Goldmann, Eva Marie 27.\,10.\,1877 Wien – 2.\,11.\,1937 ebd.@\textsc{Goldmann, Eva Marie} (27.\,10.\,1877 Wien – 2.\,11.\,1937 ebd.)|pwv} haben acht Tage der Ruhe
               in dem reizenden Edlach\oindex{Edlach@\textbf{Edlach}|pw} verbracht, das ich Dir
               nicht dringend genug \label{K_L03468-2v}\edtext{empfehlen}{\lemma{\textnormal{\emph{empfehlen}}}\Cendnote{\textnormal{Schnitzler kannte Edlach\oindex{Edlach@\textbf{Edlach}|pwk} und war hier mehrfach abgestiegen.}}}\label{K_L03468-2} kann, wenn
               Du fern von allem mondainen Getriebe (wie es in den \textsc{Hotels}
               auf dem Gipfel des Semmering\oindex{Semmering@\textbf{Semmering}, \emph{Verwaltungsgebiet}|pw} herrſcht) in
               erfriſchender Luft {\pb}Dich eine Zeit lang erholen
               willſt. Heut kehren wir nach Wien\oindex{Wien@\textbf{Wien}, \emph{Verwaltungsgebiet}|pw} zurück, von wo aus wir in wenigen Tagen die Rückreiſe nach
                  Berlin\oindex{Berlin@\textbf{Berlin}, \emph{Hauptstadt}|pw} antreten.\pend
           
\pstart
           Aufſuchen konnte ich Dich vor meiner Abreiſe nach Edlach\oindex{Edlach@\textbf{Edlach}|pw} nicht mehr, weil ich buchſtäblich keine Stunde frei hatte.\pend
           
\pstart
           Die \label{K_L03468-3v}\edtext{Spannung}{\lemma{\textnormal{\emph{Spannung}}}\Cendnote{\textnormal{Im Detail ist das unklar, doch scheint es naheliegend, dass
                     Eva Goldmann\pwindex{Goldmann, Eva Marie 27.\,10.\,1877 Wien – 2.\,11.\,1937 ebd.@\textsc{Goldmann, Eva Marie} (27.\,10.\,1877 Wien – 2.\,11.\,1937 ebd.)|pwk} dabei war, als ihr Mann am
                     12. 4. 1909{ }Schnitzler zu Hause besucht hatte.}}}\label{K_L03468-3}
               zwiſchen unſeren beiderſeitigen Frauen\pwindex{Goldmann, Eva Marie 27.\,10.\,1877 Wien – 2.\,11.\,1937 ebd.@\textsc{Goldmann, Eva Marie} (27.\,10.\,1877 Wien – 2.\,11.\,1937 ebd.)|pwv}\pwindex{Schnitzler, Olga 17.\,1.\,1882 Wien – 13.\,1.\,1970 Lugano@\textsc{Schnitzler, Olga} (17.\,1.\,1882 Wien – 13.\,1.\,1970 Lugano), \emph{Schauspielerin, Sängerin}|pwv} wird{ }ſich hoffentlich beilegen laſſen.
               Jedenfalls aber wird zwiſchen uns Beiden hoffentlich Alles{ }ſo bleiben, wie
               bisher.\pend
           
\pstart
           Ich wünſche Dir einen \label{K_L03468-4v}\edtext{zweiten
                  Sohn}{\lemma{\textnormal{\emph{zweiten
                  Sohn}}}\Cendnote{\textnormal{Olga Schnitzler\pwindex{Schnitzler, Olga 17.\,1.\,1882 Wien – 13.\,1.\,1970 Lugano@\textsc{Schnitzler, Olga} (17.\,1.\,1882 Wien – 13.\,1.\,1970 Lugano), \emph{Schauspielerin, Sängerin}|pwk} war mit Lili\pwindex{Cappellini, Lili 13.\,9.\,1909 Wien – 26.\,7.\,1928 Venedig@\textsc{Cappellini, Lili} (13.\,9.\,1909 Wien – 26.\,7.\,1928 Venedig)|pwk} schwanger, die am 13. 9. 1909
                  geboren wurde.}}}\label{K_L03468-4}, der \strikeout{ſ\textcolor{gray}{o}} ein {\pb}ebenſo prächtiger Burſch{ }ſein möge, wie
               der erſte\pwindex{Schnitzler, Heinrich 9.\,8.\,1902 Hinterbrühl – 12.\,7.\,1982 Wien@\textsc{Schnitzler, Heinrich} (9.\,8.\,1902 Hinterbrühl – 12.\,7.\,1982 Wien), \emph{Regisseur, Schauspieler}|pwv}, – oder, wenn Du
               Dir eine Tochter wünſcheſt,{ }ſo bin ich auch mit einer Tochter einverſtanden, – u. bin
               mit herzlichen Grüßen (auch von meiner Frau\pwindex{Goldmann, Eva Marie 27.\,10.\,1877 Wien – 2.\,11.\,1937 ebd.@\textsc{Goldmann, Eva Marie} (27.\,10.\,1877 Wien – 2.\,11.\,1937 ebd.)|pwv})\pend
           
\pstart
           Dein {\\[\baselineskip]}\spacefill\mbox{Paul Goldmann.}\pend
           \leftskip=0em{}\selectlanguage{ngerman}\endnumbering\briefempfaengerindex{Schnitzler, Arthur@\textsc{Schnitzler, Arthur}!zzzGoldmann, Paul@\emph{von Paul Goldmann}!1909-04-261@{26. 4. 1909}|)be}\mylabel{L03468h}  \newcommand{\dateiname}{L03468}\newcommand{\titel}{Paul Goldmann an Arthur Schnitzler, 26. 4. 1909}\newcommand{\editorInnen}{Martin Anton Müller und Laura Untner}%% latex-leseansicht-abspann.tex
%% Abspann für die Leseansicht.
%% Der Schalter \ifkorrekturansicht ist bereits durch den Vorspann gesetzt.

%% latex-abspann.tex
%% Gemeinsamer Abspann für Korrekturansicht und Leseansicht.
%% Setzt den Schalter \ifkorrekturansicht voraus (gesetzt in den
%% einbindenden Dateien latex-korrekturansicht-abspann.tex bzw.
%% latex-leseansicht-abspann.tex).
%% ---------------------------------------------------------------

\normalsize

% Das esempio-Environment wird nur in der Leseansicht benötigt
\ifkorrekturansicht\else
\newenvironment{esempio}[3]%
{
    \vspace{1.5ex}
    \rlap{\underline{#1}}
    \par
    \setlength{\parindent}{0cm}
    \nopagebreak
    \leftskip=#2cm
    \rightskip=#3cm
}
{
    \par
}
\fi

\doendnotes{C}
\bigskip
\vfill

\clearpage

\footnotesize

\ifkorrekturansicht
  \lohead{\textsc{register}}
\fi

% theindex-Environment neu definieren ohne reledmac
\makeatletter
\renewenvironment{theindex}{%
  \ifkorrekturansicht
    \section*{\indexname}%
  \else
    \subsubsection*{Index der erwähnten Entitäten}%
  \fi
  \setlength{\parindent}{0pt}%
  \setlength{\parskip}{0pt plus 0.3pt}%
  \let\item\@idxitem
}{%
  \ifkorrekturansicht\clearpage\fi
}
\makeatother

\IfFileExists{\jobname-pw.ind}{\input{\jobname-pw.ind}}{}

% Quellenangabe nur in der Leseansicht
\ifkorrekturansicht\else
% Fallback-Definitionen, falls die .tex-Datei \titel etc. nicht gesetzt hat
\providecommand{\titel}{}
\providecommand{\editorInnen}{}
\providecommand{\dateiname}{\jobname}

\vspace{3cm}

\vfill

\footnotesize
\textsc{Quelle}: \titel. Herausgegeben von {\editorInnen}. In: \emph{Arthur Schnitzler: Briefwechsel mit Autorinnen und Autoren}.
 Digitale Edition, https://schnitzler-briefe.acdh.oeaw.ac.at/{\dateiname}.html (Stand \today)
\fi

\end{document}


