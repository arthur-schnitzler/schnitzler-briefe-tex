%% latex-korrekturansicht-vorspann.tex
%% Vorspann für die Korrekturansicht.
%% Lädt die gemeinsame Datei latex-vorspann.tex mit gesetztem Schalter.

\newif\ifkorrekturansicht
\korrekturansichttrue

\input{../tex-inputs/latex-vorspann}


\section[Georg Brandes an Arthur Schnitzler, 16. 3. 1898]{L00784 Georg Brandes an Arthur Schnitzler, 16. 3. 1898}
\nopagebreak\mylabel{L00784v}
\rehead{ }\normalsize\beginnumbering\briefempfaengerindex{Schnitzler, Arthur@\textsc{Schnitzler, Arthur}!zzzBrandes, Georg@\emph{von Georg Brandes}!1898-03-161@{16. 3. 1898}|(be}
\toendnotes[C]{\smallbreak\pagebreak[2]}\Standort{CUL, Schnitzler, B 17.}
\physDesc{Brief, 1 Blatt, 4 Seiten, 2271 Zeichen
\newline{}Handschrift: schwarze Tinte, lateinische Kurrent
\newline{}Ordnung: von unbekannter Hand nummeriert: »9« }
\buchAbdrucke{\weitereDrucke{Georg Brandes, Arthur Schnitzler: \emph{Ein Briefwechsel}. Bern: \emph{Francke} 1956, S. 66–67.} }\toendnotes[C]{\smallbreak}
\pstart
           \raggedleft{}{\pb}Taormina (Sicilia) Hotel Timeo\oindex{Hotel Timeo@\textbf{Hotel Timeo}, \emph{Hotel (K.HTL)}|pw}{\\}16 März 98\pend
           
\pstart\center{}Liebster Dr. Schnitzler\pend\vspace{0.5em}
\pstart
           Ich fühle mich Ihnen und Herrn Dr. Beer-Hofmann\pwindex{Beer-Hofmann, Richard 1866-07-11 – 1945-09-26@\textsc{Beer-Hofmann, Richard} (1866-07-11 – 1945-09-26), \emph{Schriftsteller/Schriftstellerin}|pw} gegenüber wirklich wie ein Schweinehund. Ich nehme in Wien\oindex{Wien@\textbf{Wien}, \emph{A.ADM2}|pw} Ihre Zeit in Anspruch; Sie sehen täglich, wie
               es mir geht, Sie beide sind die letzten, die mich in Wien\oindex{Wien@\textbf{Wien}, \emph{A.ADM2}|pw} besuchen; ich reise fort und lasse nicht von mir hören, danke Ihnen
               nicht einmal. Nur hoffe ich dass Sie einen stummen Gruss von mir bekommen haben, da
               ich meine Tochter\pwindex{Philipp, Edith 17.01.1879 – 1968-02-16@\textsc{Philipp, Edith} (17.01.1879 – 1968-02-16)|pwv} bat, Ihnen
               einige alten Drucksachen zu senden.\pend
           
\pstart
           Der Anfang meiner Reise in Italien\oindex{Italien@\textbf{Italien}, \emph{A.PCLI}|pw} war durch
               die Krankheit meiner Mutter\pwindex{Brandes, Emilie 22.03.1818 – 27.12.1898@\textsc{Brandes, Emilie} (22.03.1818 – 27.12.1898)|pwv}
               sehr {\pb}verdüstert. Indes, sie lebt.
               Sie liegt zwar noch zu Bette aber es geht ihr besser; sie kann täglich eine Stunde
               aus dem Bette sein.\pend
           
\pstart
           Ich las irgendwo, in Florenz\oindex{Florenz@\textbf{Florenz}, \emph{P.PPLA}|pw} glaub’ ich, etwas
               über die Aufführung Ihres Stückes\pwindex{Freiwild. Schauspiel in 3 Akten@\emph{Freiwild. Schauspiel in 3 Akten}|pwv} in einem deutschen Blatt, konnte aber nicht daraus klug werden. Sind
               Sie mit dem Resultat zufrieden gewesen?\pend
           
\pstart
           Ich ging von Florenz\oindex{Florenz@\textbf{Florenz}, \emph{P.PPLA}|pw} nach Rom\oindex{Rom@\textbf{Rom}, \emph{P.PPLC}|pw}, wo die Studenten der philosophischen Fakultät\oindex{Universitaet La Sapienza@\textbf{Universität La Sapienza}, \emph{Universität (K.UNI)}|pw} artig genug waren mich mit einer sehr netten
               Adresse zu begrüssen. Es war dort bald kalt, bald warm, doch trocken, aber in Neapel\oindex{Neapel@\textbf{Neapel}, \emph{P.PPLA}|pw} wurde ich von argem Regenwetter verfolgt.
               Dort sah ich curios genug die ganze Aristokratie, da man mich {\pb}viel in diesen Kreisen einlud,
               obwohl ich nicht einmal Empfehlungsschreiben hatte.\pend
           
\pstart
           Hier in diesem gesegneten und verhungernden Land hatte ich wieder fast immer Regen.
               Ich bin schon mehr als 14 Tage hier. Aber wenn es bisweilen schön ist, dann ist es
               hier am Fusse des Etna\oindex{Aetna@\textbf{Ätna}, \emph{T.VLC}|pw} in der starken herrlichen
               Wärme am Ufer des Meeres wahrlich sehr schön. Hier hat jeder Fleck ihre Geschichte,
               hier haben Araber und Normannen usw. Spuren hinterlassen, hier hat Heine\pwindex{Heine, Heinrich 13.12.1797 – 17.02.1856@\textsc{Heine, Heinrich} (13.12.1797 – 17.02.1856), \emph{Schriftsteller/Schriftstellerin}|pw}’s Platen\pwindex{Platen, August von 24.10.1796 – 05.12.1835@\textsc{Platen, August von} (24.10.1796 – 05.12.1835), \emph{Schriftsteller/Schriftstellerin}|pw} gelebt, und noch giebt es hier in Taormina\oindex{Taormina@\textbf{Taormina}, \emph{P.PPLA3}|pw} nicht wenige deutsche Herren mit seinen Leidenschaften.\pend
           
\pstart
           Ich lebe hier gesellig am Tage, einsam {\pb}von 5 Uhr ab, lese und schreibe
               viel, oder so viel ich vermag, denn alt und dumm bin ich.\pend
           
\pstart
           Ich danke Herrn Beer Hofmann\pwindex{Beer-Hofmann, Richard 1866-07-11 – 1945-09-26@\textsc{Beer-Hofmann, Richard} (1866-07-11 – 1945-09-26), \emph{Schriftsteller/Schriftstellerin}|pw} viel für das Buch\pwindex{Lust@\emph{Lust}|pwv} von d’Annunzio\pwindex{DAnnunzio, Gabriele 12.03.1863 – 01.03.1938@\textsc{D’Annunzio, Gabriele} (12.03.1863 – 01.03.1938), \emph{Schriftsteller/Schriftstellerin}|pw}, das ich zwischen Wien\oindex{Wien@\textbf{Wien}, \emph{A.ADM2}|pw} und Florenz\oindex{Florenz@\textbf{Florenz}, \emph{P.PPLA}|pw} las; es war mir
               eigentlich zuwider, und ich mag auch das Uebrige von d’Annunzio\pwindex{DAnnunzio, Gabriele 12.03.1863 – 01.03.1938@\textsc{D’Annunzio, Gabriele} (12.03.1863 – 01.03.1938), \emph{Schriftsteller/Schriftstellerin}|pw} nur wenig. Uebrigens war die Uebersetzung sehr stark gekürzt, als
               ich sie mit dem Original verglich. Grüssen Sie mir sehr herzlich den weisen Mann\pwindex{Beer-Hofmann, Richard 1866-07-11 – 1945-09-26@\textsc{Beer-Hofmann, Richard} (1866-07-11 – 1945-09-26), \emph{Schriftsteller/Schriftstellerin}|pwv}, Wollzeile 15\oindex{Wollzeile@\textbf{Wollzeile}, \emph{Straße (K.STR)}|pw}, I\pend
           
\pstart
           Ich bitte Sie mich auch Ihrer Frau Mutter\pwindex{Schnitzler, Louise 1840-07-08 – 1911-09-09@\textsc{Schnitzler, Louise} (1840-07-08 – 1911-09-09)|pwv} bestens zu empfehlen.\pend
           
\pstart
           Ihr ergebener{\\[\baselineskip]}\spacefill\mbox{Georg Brandes}\pend
           \leftskip=0em{}\selectlanguage{ngerman}\endnumbering\briefempfaengerindex{Schnitzler, Arthur@\textsc{Schnitzler, Arthur}!zzzBrandes, Georg@\emph{von Georg Brandes}!1898-03-161@{16. 3. 1898}|)be}\mylabel{L00784h}  \normalsize

\doendnotes{C}
\bigskip
\vfill

\clearpage

\footnotesize

\lohead{\textsc{register}}

% Definiere theindex-Environment komplett neu ohne reledmac
\makeatletter
\renewenvironment{theindex}{%
  \section*{\indexname}%
  \setlength{\parindent}{0pt}%
  \setlength{\parskip}{0pt plus 0.3pt}%
  \let\item\@idxitem
}{%
  \clearpage
}
\makeatother

\IfFileExists{\jobname-pw.ind}{\input{\jobname-pw.ind}}{}

\end{document}

      