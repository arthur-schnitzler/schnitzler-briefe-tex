%% latex-korrekturansicht-vorspann.tex
%% Vorspann für die Korrekturansicht.
%% Lädt die gemeinsame Datei latex-vorspann.tex mit gesetztem Schalter.

\newif\ifkorrekturansicht
\korrekturansichttrue

\input{../tex-inputs/latex-vorspann}


\section[Arthur Schnitzler an Richard Beer-Hofmann, 14. 12. 1892]{L00143 Arthur Schnitzler an Richard Beer-Hofmann, 14. 12. 1892}
\nopagebreak\mylabel{L00143v}
\rehead{ }\normalsize\beginnumbering\briefempfaengerindex{Beer-Hofmann, Richard@\textsc{Beer-Hofmann, Richard}!zzzSchnitzler, Arthur@\emph{von Arthur Schnitzler}!1892-12-141@{14. 12. 1892}|(be}
\toendnotes[C]{\smallbreak\pagebreak[2]}\Standort{YCGL, MSS 31.}
\physDesc{Kartenbrief, 393 Zeichen
\newline{}Handschrift: 1) Bleistift, deutsche Kurrent\hspace{1em}2) Bleistift, lateinische Kurrent (\noindent{}Adresse)\hspace{1em}
\newline{}Versand: 1) Stempel: »\nobreak{}\oindex{IX., Alsergrund@\textbf{IX., Alsergrund}, \emph{A.ADM3}|pwk}Wien 9/3, 14 12 92, 2–3\nobreak{}«.   2) Stempel: »\nobreak{}Wien 1/1, 14/12. 92, 5–6½ N, Bestellt\nobreak{}«. }
\buchAbdrucke{\weitereDrucke{Arthur Schnitzler, Richard Beer-Hofmann: \emph{Briefwechsel 1891–1931}. Wien, Zürich: \emph{Europaverlag} 1992, S. 40.} }\toendnotes[C]{\smallbreak}\pstart{}{\pb}Hrn Dr. Rich Beer Hofmann\pend{}\pstart{}Wien\oindex{Wien@\textbf{Wien}, \emph{A.ADM2}|pw}\pend{}\pstart{}I Wollzeile 15\oindex{Wollzeile@\textbf{Wollzeile}, \emph{Straße (K.STR)}|pw}. \pend{}{\bigskip}\vspace{1em}
\pstart
           \noindent{}{\pb}Lieber Richard! War geſtern bei Singers\pwindex{Singer, Marie 03.04.1850 – 30.04.1918@\textsc{Singer, Marie} (03.04.1850 – 30.04.1918)|pw}\pwindex{Singer, Alexander 16.11.1841 – 30.11.1906@\textsc{Singer, Alexander} (16.11.1841 – 30.11.1906), \emph{Herausgeber/Herausgeberin, Administrator/Administratorin}|pw}, dort \substVorne{}\textsuperscript{\textcolor{gray}{bed}}\substDazwischen{}Frau\substHinten{}{ }\textsc{Flegm}.\pwindex{Flegmann, Bertha 27.05.1852 – 24.6.1933@\textsc{Flegmann, Bertha} (27.05.1852 – 24.6.1933), \emph{männliche Salonnière/Salonnière}|pw} – Bitte ſehr, ko{\geminationm}en Sie Freitag mit mir zu ihr? Ja?\pend
           
\pstart
           Die Anatols\pwindex{Anatol@\emph{Anatol}|pw}{ }ſollen nicht in \textsc{Rdlfsh}\oindex{Volkstheater in Rudolfsheim@\textbf{Volkstheater in Rudolfsheim}, \emph{Theater (K.THE)}|pw}, ſondern event. privat aufgeführt werden.\pend
           
\pstart
           Wollen Sie mich Freitag um 6, ½ 7 abholen? Es
               wäre mir angenehm, wenn wir beide hingingen. –\pend
           
\pstart
           Geſtern 2. Akt\pwindex{Familie@\emph{Familie}|pwv} vollendet.
               –\pend
           \pstart Herzlich Ihr \spacefill\mbox{Arthur}\pend{}
\pstart
           \noindent{}Heute will ich zur Jüdin von Toledo\pwindex{Juedin von Toledo@\emph{Die Jüdin von Toledo}|pw} gehn.\pend
           \selectlanguage{ngerman}\endnumbering\briefempfaengerindex{Beer-Hofmann, Richard@\textsc{Beer-Hofmann, Richard}!zzzSchnitzler, Arthur@\emph{von Arthur Schnitzler}!1892-12-141@{14. 12. 1892}|)be}\mylabel{L00143h}  \normalsize

\doendnotes{C}
\bigskip
\vfill

\clearpage

\footnotesize

\lohead{\textsc{register}}

% Definiere theindex-Environment komplett neu ohne reledmac
\makeatletter
\renewenvironment{theindex}{%
  \section*{\indexname}%
  \setlength{\parindent}{0pt}%
  \setlength{\parskip}{0pt plus 0.3pt}%
  \let\item\@idxitem
}{%
  \clearpage
}
\makeatother

\IfFileExists{\jobname-pw.ind}{\input{\jobname-pw.ind}}{}

\end{document}

      