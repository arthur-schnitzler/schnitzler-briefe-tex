%% latex-leseansicht-vorspann.tex
%% Vorspann für die Leseansicht.
%% Lädt die gemeinsame Datei latex-vorspann.tex mit nicht gesetztem Schalter.

\newif\ifkorrekturansicht
\korrekturansichtfalse

\input{../tex-inputs/latex-vorspann}


         
         \renewcommand{\erwaehntePersonen}{Personen: Richard Beer-Hofmann, Bertha Flegmann, Marie Singer, Alexander Singer}
         \renewcommand{\erwaehnteOrte}{Orte: Volkstheater in Rudolphsheim, Wien, Wollzeile}
         \renewcommand{\erwaehnteWerke}{Werke: Anatol, Die Jüdin von Toledo, Familie}
               \section[Arthur Schnitzler an Richard Beer-Hofmann, 14. 12. 1892]{ Arthur Schnitzler an Richard Beer-Hofmann, 14. 12. 1892}\nopagebreak\mylabel{v}\rehead{ }\begin{ledgroupsized}[t]{13cm}\normalsize\beginnumbering \toendnotes[C]{\smallbreak\pagebreak[2]} \Standort{YCGL, MSS 31.}
\physDesc{Kartenbrief, 393 Zeichen
\newline{}Handschrift: 1) Bleistift, deutsche Kurrent\hspace{1em}2) Bleistift, lateinische Kurrent (\noindent{}Adresse)\hspace{1em}
\newline{}Versand: 1) Stempel: »\nobreak{}Wien 9/3, 14 12 92, 2–3\nobreak{}«.   2) Stempel: »\nobreak{}Wien 1/1, 14/12. 92, 5–6½ N, Bestellt\nobreak{}«. }\buchAbdrucke{\weitereDrucke{Arthur Schnitzler, Richard Beer-Hofmann: \emph{Briefwechsel 1891–1931}. Hg. Konstanze Fliedl. Wien, Zürich: \emph{Europaverlag} 1992, S. 40.} }\toendnotes[C]{\smallbreak}\pstart{}{\pb}Hrn Dr. Rich Beer Hofmann\pend{}\pstart{}Wien\oindex{Wien@\textbf{Wien}|pw}\pend{}\pstart{}I Wollzeile 15\oindex{Wollzeile@\textbf{Wollzeile}|pw}. \pend{}{\bigskip}\pstart
           \noindent{}{\pb}Lieber Richard! War geſtern bei Singers\pwindex{Singer, Marie 03.04.1850 – 30.04.1918@\textsc{Singer, Marie} (03.04.1850 – 30.04.1918)|pw}\pwindex{Singer, Alexander 16.11.1841 – 30.11.1906@\textsc{Singer, Alexander} (16.11.1841 – 30.11.1906), \emph{Herausgeber, Administrator}|pw}, dort \substVorne{}\textsuperscript{\textcolor{gray}{bed}}\substDazwischen{}Frau\substHinten{}{ }\textsc{Flegm}.\pwindex{Flegmann, Bertha 27.05.1852 – 24.6.1933@\textsc{Flegmann, Bertha} (27.05.1852 – 24.6.1933), \emph{Salonnière}|pw} – Bitte ſehr, ko{\geminationm}en Sie Freitag mit mir zu ihr? Ja?\pend
           \pstart
           Die Anatols\pwindex{Schnitzler, Arthur 15.05.1862 – 21.10.1931@\textsc{Schnitzler, Arthur} (15.05.1862 – 21.10.1931), \emph{Schriftsteller, Mediziner}!Anatol1892-10-29@\strich\emph{Anatol} {[}1892-10-29{]}|pw}{ }ſollen nicht in \textsc{Rdlfsh}\oindex{Volkstheater in Rudolphsheim@\textbf{Volkstheater in Rudolphsheim}|pw}, ſondern event. privat aufgeführt werden.\pend
           \pstart
           Wollen Sie mich Freitag um 6, ½ 7 abholen? Es
               wäre mir angenehm, wenn wir beide hingingen. –\pend
           \pstart
           Geſtern 2. Akt\pwindex{Schnitzler, Arthur 15.05.1862 – 21.10.1931@\textsc{Schnitzler, Arthur} (15.05.1862 – 21.10.1931), \emph{Schriftsteller, Mediziner}!Familie1977@\strich\emph{Familie} {[}1977{]}|pwv} vollendet.
               –\pend
           \pstart Herzlich Ihr \spacefill\mbox{Arthur}\pend{}\pstart
           \noindent{}Heute will ich zur Jüdin von Toledo\pwindex{\textcolor{red}{\textsuperscript{XXXX1 indx}}!Juedin von Toledo22. 11. 1872@\strich\emph{Die Jüdin von Toledo} {[}22. 11. 1872{]}|pw} gehn.\pend
           
         
         \endnumbering\mylabel{h}\end{ledgroupsized}  \newcommand{\dateiname}{L00143}\newcommand{\titel}{Arthur Schnitzler an Richard Beer-Hofmann, 14. 12. 1892}\newcommand{\editorInnen}{Martin Anton Müller und Gerd-Hermann Susen}%% latex-leseansicht-abspann.tex
%% Abspann für die Leseansicht.
%% Der Schalter \ifkorrekturansicht ist bereits durch den Vorspann gesetzt.

%% latex-abspann.tex
%% Gemeinsamer Abspann für Korrekturansicht und Leseansicht.
%% Setzt den Schalter \ifkorrekturansicht voraus (gesetzt in den
%% einbindenden Dateien latex-korrekturansicht-abspann.tex bzw.
%% latex-leseansicht-abspann.tex).
%% ---------------------------------------------------------------

\normalsize

% Das esempio-Environment wird nur in der Leseansicht benötigt
\ifkorrekturansicht\else
\newenvironment{esempio}[3]%
{
    \vspace{1.5ex}
    \rlap{\underline{#1}}
    \par
    \setlength{\parindent}{0cm}
    \nopagebreak
    \leftskip=#2cm
    \rightskip=#3cm
}
{
    \par
}
\fi

\doendnotes{C}
\bigskip
\vfill

\clearpage

\footnotesize

\ifkorrekturansicht
  \lohead{\textsc{register}}
\fi

% theindex-Environment neu definieren ohne reledmac
\makeatletter
\renewenvironment{theindex}{%
  \ifkorrekturansicht
    \section*{\indexname}%
  \else
    \subsubsection*{Index der erwähnten Entitäten}%
  \fi
  \setlength{\parindent}{0pt}%
  \setlength{\parskip}{0pt plus 0.3pt}%
  \let\item\@idxitem
}{%
  \ifkorrekturansicht\clearpage\fi
}
\makeatother

\IfFileExists{\jobname-pw.ind}{\input{\jobname-pw.ind}}{}

% Quellenangabe nur in der Leseansicht
\ifkorrekturansicht\else
% Fallback-Definitionen, falls die .tex-Datei \titel etc. nicht gesetzt hat
\providecommand{\titel}{}
\providecommand{\editorInnen}{}
\providecommand{\dateiname}{\jobname}

\vspace{3cm}

\vfill

\footnotesize
\textsc{Quelle}: \titel. Herausgegeben von {\editorInnen}. In: \emph{Arthur Schnitzler: Briefwechsel mit Autorinnen und Autoren}.
 Digitale Edition, https://schnitzler-briefe.acdh.oeaw.ac.at/{\dateiname}.html (Stand \today)
\fi

\end{document}


      