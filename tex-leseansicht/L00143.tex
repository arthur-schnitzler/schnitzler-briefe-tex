\input{../tex-inputs/latex-pdf-vorspann}
\begin{center}
            \textcolor{red}{ENTWURF. ENTZIFFERUNG NOCH NICHT KORREKTURGELESEN}
                      \end{center}
            
               \section[Arthur Schnitzler an Richard Beer-Hofmann, 14. 12. 1892]{ Arthur Schnitzler an Richard Beer-Hofmann, 14. 12. 1892}\nopagebreak\mylabel{v}\rehead{ }\begin{ledgroupsized}[t]{13cm}\normalsize\beginnumbering\briefempfaengerindex{Beer-Hofmann, Richard@\textsc{Beer-Hofmann, Richard}!zzzSchnitzler, Arthur@\emph{von Arthur Schnitzler}!1892-12-141@{14. 12. 1892}|(be} \toendnotes[C]{\smallbreak\pagebreak[2]} \Standort{YCGL, MSS 31.}
\physDesc{Kartenbrief
\newline{}Handschrift: Bleistift, deutsche Kurrent\newline{}Versand: 1) Stempel: »\nobreak{}Wien 9/3, 14 12 92, 2–3\nobreak{}«.  2) Stempel: »\nobreak{}Wien 1/1, 14/12. 92, 5–6½ N, Bestellt\nobreak{}«. }\buchAbdrucke{\weitereDrucke{Arthur Schnitzler, Richard Beer-Hofmann: \emph{Briefwechsel 1891–1931}. Hg. Konstanze Fliedl. Wien, Zürich: \emph{Europaverlag} 1992, S. 40.} }\toendnotes[C]{\smallbreak}\pstart{}{\pb}\textsc{Hrn Dr. Rich Beer Hofmann}\pend{}\pstart{}\textsc{Wien\oindex{Wien@\textbf{Wien}|pw}}\pend{}\pstart{}\textsc{I
                  Wollzeile 15\oindex{Wollzeile@\textbf{Wollzeile}|pw}.}\pend{}{\bigskip}\pstart
           \noindent{}{\pb}Lieber Richard! War geſtern bei Singers\pwindex{Singer, Marie 03.04.1850 – 30.04.1918@\textsc{Singer, Marie} (03.04.1850 – 30.04.1918)|pw}\pwindex{Singer, Alexander 16.11.1841 – 30.11.1906@\textsc{Singer, Alexander} (16.11.1841 – 30.11.1906), \emph{Herausgeber, Administrator}|pw}, dort \substVorne{}\textsuperscript{\textcolor{gray}{bed}}\substDazwischen{}Frau\substHinten{}{ }\textsc{Flegm}.\pwindex{Flegmann, Bertha 27.05.1852 – 24.6.1933@\textsc{Flegmann, Bertha} (27.05.1852 – 24.6.1933), \emph{Salonnière}|pw} – Bitte ſehr, ko{\geminationm}en Sie Freitag mit mir zu ihr? Ja?\pend
           \pstart
           Die Anatols\pwindex{Schnitzler, Arthur 15.05.1862 – 21.10.1931@\textsc{Schnitzler, Arthur} (15.05.1862 – 21.10.1931), \emph{Schriftsteller, Mediziner}!Anatol1892-10-29 – 1892-10-29@\strich\emph{Anatol} {[}1892-10-29 – 1892-10-29{]}|pw}{ }ſollen nicht in \textsc{Rdlfsh}\oindex{Volkstheater in Rudolphsheim@\textbf{Volkstheater in Rudolphsheim}|pw}, ſondern event. privat aufgeführt werden.\pend
           \pstart
           Wollen Sie mich Freitag um 6, ½ 7 abholen? Es
               wäre mir angenehm, wenn wir beide hingingen. –\pend
           \pstart
           Geſtern 2. Akt\pwindex{Schnitzler, Arthur 15.05.1862 – 21.10.1931@\textsc{Schnitzler, Arthur} (15.05.1862 – 21.10.1931), \emph{Schriftsteller, Mediziner}!Familie1977@\strich\emph{Familie} {[}1977{]}|pwv} vollendet. –\pend
           \pstart Herzlich Ihr \spacefill\mbox{Arthur}\pend{}\pstart
           \noindent{}Heute will ich zur Jüdin von Toledo\pwindex{\textcolor{red}{\textsuperscript{XXXX1 indx}}!Juedin von Toledo22. 11. 1872@\strich\emph{Die Jüdin von Toledo} {[}22. 11. 1872{]}|pw} gehn.\pend
           \endnumbering\briefempfaengerindex{Beer-Hofmann, Richard@\textsc{Beer-Hofmann, Richard}!zzzSchnitzler, Arthur@\emph{von Arthur Schnitzler}!1892-12-141@{14. 12. 1892}|)be}\mylabel{h}\end{ledgroupsized}  \newcommand{\dateiname}{L00143}\newcommand{\titel}{Arthur Schnitzler an Richard Beer-Hofmann, 14. 12. 1892}\newcommand{\editorInnen}{Martin Anton Müller und Gerd-Hermann Susen}\input{../tex-inputs/latex-pdf-abspann}
      