%% latex-leseansicht-vorspann.tex
%% Vorspann für die Leseansicht.
%% Lädt die gemeinsame Datei latex-vorspann.tex mit nicht gesetztem Schalter.

\newif\ifkorrekturansicht
\korrekturansichtfalse

\input{../tex-inputs/latex-vorspann}


\section[Carl Sternheim u.a. an Arthur Schnitzler, 22. 12. 1911]{L04013 Carl Sternheim u.a. an Arthur Schnitzler, 22. 12. 1911}
\nopagebreak\mylabel{L04013v}
\rehead{ }\normalsize\beginnumbering\briefempfaengerindex{Schnitzler, Arthur@\textsc{Schnitzler, Arthur}!zzzBorngräber, Otto@\emph{von Otto Borngräber}!1911-12-221@{22. 12. 1911}|(be}\briefempfaengerindex{Schnitzler, Arthur@\textsc{Schnitzler, Arthur}!zzzEulenberg, Herbert@\emph{von Herbert Eulenberg}!1911-12-221@{22. 12. 1911}|(be}\briefempfaengerindex{Schnitzler, Arthur@\textsc{Schnitzler, Arthur}!zzzWedekind, Frank@\emph{von Frank Wedekind}!1911-12-221@{22. 12. 1911}|(be}\briefempfaengerindex{Schnitzler, Arthur@\textsc{Schnitzler, Arthur}!zzzSternheim, Carl@\emph{von Carl Sternheim}!1911-12-221@{22. 12. 1911}|(be}
\toendnotes[C]{\smallbreak\pagebreak[2]}
\correspDesc{Versand  durch Carl Sternheim, Frank Wedekind, Herbert Eulenberg, Otto Borngräber am 22. 12. 1911 in Pullach im Isartal
\newline{}Erhalt  durch Arthur Schnitzler im Zeitraum [23. 12. 1911 – 27. 12. 1911?] in Wien}\toendnotes[C]{\smallbreak}
\Standort{DLA, A:Schnitzler, HS.1985.1.5742.}
\physDesc{Brief, Durchschlag, 2 Blätter, 2 Seiten, 2435 Zeichen
\newline{}Schreibmaschine
\newline{}Handschrift Carl Sternheim: 1) schwarze Tinte (\noindent{}Originalunterschrift)\hspace{1em}2) Bleistift, lateinische Kurrent (\noindent{}Ergänzung)\hspace{1em}
\newline{}Schnitzler: 1) mit rotem Buntstift zwei Unterstreichungen und auf dem zweiten Blatt Vermerk:
                                       »\textsc{Sternheim}« und
                                 Datierung: »22/12 11«  2) mit Bleistift Vermerk: »\textsc{Sternheim}«}
\buchAbdrucke{\weitereDrucke{\emph{Frank Wedekinds Korrespondenz digital}. (21. 1. 2025) \url{https://briefedition.wedekind.h-da.de/view/document/single.xhtml?contentType=1&documentId=5771}.} }\toendnotes[C]{\smallbreak}
\pstart
           \centering{}{\pb}\textcolor{gray}{\textbf{Bellemaison\oindex{Villa Bellemaison@\textbf{Villa Bellemaison}, \emph{Wohngebäude}|pw}}}\pend
           
\pstart
           \centering{}\textcolor{gray}{\textbf{Höllriegelskreuth bei München\oindex{Höllriegelskreuth@\textbf{Höllriegelskreuth}, \emph{Teil eines besiedelten Ortes}|pw}.}}\pend
           
\pstart
           \centering{}den 22. Dezember 1911.\pend
           
\pstart{}Sehr geehrter Herr.\pend\vspace{0.5em}
\pstart
           Im Namen der Herren Frank Wedekind, Herbert Eulenberg, Otto Borngräber und im eigenen
               habe ich die Ehre, Ihnen folgendes mitzuteilen:\pend
           
\pstart
           Wir haben die Absicht, das große Publikum durch beifolgenden Aufruf\pwindex{Sternheim, Carl 1.\,4.\,1878 Leipzig – 3.\,11.\,1942 Brüssel@\textsc{Sternheim, Carl} (1.\,4.\,1878 Leipzig – 3.\,11.\,1942 Brüssel), \emph{Schriftsteller}!Aufruf [Aus der Mitte des Publikums…]@\strich\emph{Aufruf [Aus der Mitte des Publikums…]}|pwv}\pwindex{Wedekind, Frank 24.\,7.\,1864 Hannover – 9.\,3.\,1918 München@\textsc{Wedekind, Frank} (24.\,7.\,1864 Hannover – 9.\,3.\,1918 München), \emph{Schriftsteller, Schauspieler, Schriftsteller}!Aufruf [Aus der Mitte des Publikums…]@\strich\emph{Aufruf [Aus der Mitte des Publikums…]}|pwv}\pwindex{Borngräber, Otto 19.\,11.\,1874 Stendal – 19.\,10.\,1916 Lugano@\textsc{Borngräber, Otto} (19.\,11.\,1874 Stendal – 19.\,10.\,1916 Lugano), \emph{Schriftsteller}!Aufruf [Aus der Mitte des Publikums…]@\strich\emph{Aufruf [Aus der Mitte des Publikums…]}|pwv}\pwindex{Eulenberg, Herbert 25.\,1.\,1876 Mülheim [Köln] – 4.\,9.\,1949 Düsseldorf@\textsc{Eulenberg, Herbert} (25.\,1.\,1876 Mülheim [Köln] – 4.\,9.\,1949 Düsseldorf), \emph{Schriftsteller}!Aufruf [Aus der Mitte des Publikums…]@\strich\emph{Aufruf [Aus der Mitte des Publikums…]}|pwv}, der in Zukunft
               jedem Buch der unten angegebenen Autoren beiliegen soll und auch den Tageszeitungen
               mitgeteilt wird, zum Beitritt zu einem allgemeinen Protest aufzufordern.\pend
           
\pstart
           In Anbetracht \label{T_L04013-1v}\edtext{der}{\lemma{\textnormal{\emph{der}}}\Cendnote{\textnormal{Er schreibt: »dern«.}}}\label{T_L04013-1} ungeheuren Wichtigkeit und Dringlichkeit der
                  Sa{[}c{]}he hoffen wir, Sie werden möglichst umgehend dem
               Unterzeichneten Ihre Zustimmung mitteilen und die Erlaubnis erteilen, Ihren Namen
               unter den Aufruf setzen zu dürfen.\pend
           
\pstart
           Ihr sehr ergebener{\\[\baselineskip]}\spacefill\mbox{{[}hs.:{]} Carl Sternheim}\pend
           \leftskip=0em{}
\pstart
           \noindent{}{[}ms.:{]} Aufgefordert wurden folgende Autoren:\pend
           \settowidth{\longeste}{Otto Borngräber}\settowidth{\longestz}{Hugo von Hofmannsthal}\settowidth{\longestd}{Wilhelm Schmidtbonn}\settowidth{\longestv}{}\settowidth{\longestf}{}\addtolength\longeste{1em}
        \addtolength\longestz{1em}
        \addtolength\longestd{1em}
      \pstart\noindent\makebox[\the\longeste][l]{Hermann Bahr\pwindex{Bahr, Hermann 19.\,7.\,1863 Linz – 15.\,1.\,1934 München@\textsc{Bahr, Hermann} (19.\,7.\,1863 Linz – 15.\,1.\,1934 München), \emph{Schriftsteller, Kritiker}|pw}}\makebox[\the\longestz][l]{Herbert \label{T_L04013-2v}\edtext{Eulenberg}{\lemma{\textnormal{\emph{Eulenberg}}}\Cendnote{\textnormal{Er schreibt: »Eulenburg«.}}}\label{T_L04013-2}}
                  \makebox[\the\longestd][l]{Wilhelm Schmidtbonn\pwindex{Schmidtbonn, Wilhelm 6.\,2.\,1876 Bonn – 3.\,7.\,1952 Bad Godesberg@\textsc{Schmidtbonn, Wilhelm} (6.\,2.\,1876 Bonn – 3.\,7.\,1952 Bad Godesberg), \emph{Schriftsteller}|pw}}\pend\pstart\noindent\makebox[\the\longeste][l]{Franz Blei\pwindex{Blei, Franz 18.\,1.\,1871 Wien – 10.\,7.\,1942 Westbury@\textsc{Blei, Franz} (18.\,1.\,1871 Wien – 10.\,7.\,1942 Westbury), \emph{Schriftsteller}|pw}}\makebox[\the\longestz][l]{Gerhart Hauptmann\pwindex{Hauptmann, Gerhart 15.\,11.\,1862 Szczawno-Zdrój – 6.\,6.\,1946 Jagniątków@\textsc{Hauptmann, Gerhart} (15.\,11.\,1862 Szczawno-Zdrój – 6.\,6.\,1946 Jagniątków), \emph{Schriftsteller}|pw}}
                  \makebox[\the\longestd][l]{Arthur Schnitzler}\pend\pstart\noindent\makebox[\the\longeste][l]{Otto Borngräber}\makebox[\the\longestz][l]{Hugo von Hofmannsthal\pwindex{Hofmannsthal, Hugo von 1.\,2.\,1874 Wien – 15.\,7.\,1929 Rodaun@\textsc{Hofmannsthal, Hugo von} (1.\,2.\,1874 Wien – 15.\,7.\,1929 Rodaun), \emph{Schriftsteller}|pw}}
                  \makebox[\the\longestd][l]{Carl Sternheim}\pend\pstart\noindent\makebox[\the\longeste][l]{Max Dauthendey\pwindex{Dauthendey, Max 25.\,7.\,1867 Würzburg – 29.\,8.\,1918 Malang@\textsc{Dauthendey, Max} (25.\,7.\,1867 Würzburg – 29.\,8.\,1918 Malang), \emph{Schriftsteller}|pw}}\makebox[\the\longestz][l]{Heinrich Mann\pwindex{Mann, Heinrich 27.\,3.\,1871 Lübeck – 11.\,3.\,1950 Santa Monica@\textsc{Mann, Heinrich} (27.\,3.\,1871 Lübeck – 11.\,3.\,1950 Santa Monica), \emph{Schriftsteller}|pw}}
                  \makebox[\the\longestd][l]{Karl Vollmöller\pwindex{Vollmoeller, Karl Gustav 7.\,5.\,1878 Stuttgart – 18.\,10.\,1948 Hollywood@\textsc{Vollmoeller, Karl Gustav} (7.\,5.\,1878 Stuttgart – 18.\,10.\,1948 Hollywood), \emph{Schriftsteller}|pw}}\pend\pstart\noindent\makebox[\the\longeste][l]{Richard Dehmel\pwindex{Dehmel, Richard 18.\,11.\,1863 Hermsdorf – 8.\,2.\,1920 Blankenese@\textsc{Dehmel, Richard} (18.\,11.\,1863 Hermsdorf – 8.\,2.\,1920 Blankenese), \emph{Schriftsteller, Schriftsteller, Krimiautor}|pw}}\makebox[\the\longestz][l]{Thomas Mann\pwindex{Mann, Thomas 6.\,6.\,1875 Lübeck – 12.\,8.\,1955 Zürich@\textsc{Mann, Thomas} (6.\,6.\,1875 Lübeck – 12.\,8.\,1955 Zürich), \emph{Schriftsteller}|pw}}
                  \makebox[\the\longestd][l]{Frank Wedekind}\pend
\pstart
           Sie sind gebeten, über Vorstehendendes vorläufig Verschwiegenheit zu wahren.\pend
           \selectlanguage{ngerman}\vspace{1em}
\pstart
           \noindent{}{\pb}Aus der Mitte des Publikums kommt Anfrage auf Anfrage an
               uns: Es fühle sich durch die fortwährenden Polizeiverbote in seinem Empfinden, seinem
               Urteil auf’s höchste verwirrt und beunruhigt. Der Gatte wisse nicht mehr, was er
               seiner Frau, der Erzieher nicht, was er den Zöglingen von unseren Büchern anbieten
               dürfe.\pend
           
\pstart
           Sei denn wirklich aus dem Geist unserer Schriften die Polizei zu ihrem Vorgehen gegen
               uns nicht befugt? Könnten wir Autoren auf unsere Ehre versichern, wir stellten in
               unseren Dichtungen dem sittlichen Gefühl einer großen Nation, deren geistiges Wohl
               uns anvertraut ist, keine Falle?\pend
           
\pstart
           Nun denn im Bewußtsein unserer unendlichen Verantwortung auf Manneswort für jetzt und
               alle Zukunft: Mit Andacht und Demut hören wir auf die Empfindungen der Kinder unserer
               Zeit. Keine irdische Macht, aber die Stimme Gottes aus der Brust des Menschen
               diktiert uns die Forderungen der Vernunft, Schönheit und Sittlichkeit. Ihr Leser,
               nicht wir Schreibenden (die wir ja nur Euer geheimstes Sprachrohr in die Welt sind)
               prägt diese neuen Wahrheiten, die den Hütern der alten bequemen Formeln ein Entsetzen
               sind, und die sie im Keim zu vernichten trachten.\pend
           
\pstart
           Ihr Leser selbst seid in Eurem Willen zu heutiger Wahrheit aufs heftigste
               angegriffen. Helft Euch, indem Ihr zum Kampf entschlossen, einem geharnischten
               Proteste beitretet. Helft endlich uns mit der Tat gegen Willkür und Ver\introOben{}ge\introOben{}waltigung. wir gehen gemeinsam in neue Zeiten hinein!\pend
           \selectlanguage{ngerman}\endnumbering\briefempfaengerindex{Schnitzler, Arthur@\textsc{Schnitzler, Arthur}!zzzBorngräber, Otto@\emph{von Otto Borngräber}!1911-12-221@{22. 12. 1911}|)be}\briefempfaengerindex{Schnitzler, Arthur@\textsc{Schnitzler, Arthur}!zzzEulenberg, Herbert@\emph{von Herbert Eulenberg}!1911-12-221@{22. 12. 1911}|)be}\briefempfaengerindex{Schnitzler, Arthur@\textsc{Schnitzler, Arthur}!zzzWedekind, Frank@\emph{von Frank Wedekind}!1911-12-221@{22. 12. 1911}|)be}\briefempfaengerindex{Schnitzler, Arthur@\textsc{Schnitzler, Arthur}!zzzSternheim, Carl@\emph{von Carl Sternheim}!1911-12-221@{22. 12. 1911}|)be}\mylabel{L04013h}  \newcommand{\dateiname}{L04013}\newcommand{\titel}{Carl Sternheim u.a. an Arthur Schnitzler, 22. 12. 1911}\newcommand{\editorInnen}{Selma Jahnke und Martin Anton Müller}%% latex-leseansicht-abspann.tex
%% Abspann für die Leseansicht.
%% Der Schalter \ifkorrekturansicht ist bereits durch den Vorspann gesetzt.

%% latex-abspann.tex
%% Gemeinsamer Abspann für Korrekturansicht und Leseansicht.
%% Setzt den Schalter \ifkorrekturansicht voraus (gesetzt in den
%% einbindenden Dateien latex-korrekturansicht-abspann.tex bzw.
%% latex-leseansicht-abspann.tex).
%% ---------------------------------------------------------------

\normalsize

% Das esempio-Environment wird nur in der Leseansicht benötigt
\ifkorrekturansicht\else
\newenvironment{esempio}[3]%
{
    \vspace{1.5ex}
    \rlap{\underline{#1}}
    \par
    \setlength{\parindent}{0cm}
    \nopagebreak
    \leftskip=#2cm
    \rightskip=#3cm
}
{
    \par
}
\fi

\doendnotes{C}
\bigskip
\vfill

\clearpage

\footnotesize

\ifkorrekturansicht
  \lohead{\textsc{register}}
\fi

% theindex-Environment neu definieren ohne reledmac
\makeatletter
\renewenvironment{theindex}{%
  \ifkorrekturansicht
    \section*{\indexname}%
  \else
    \subsubsection*{Index der erwähnten Entitäten}%
  \fi
  \setlength{\parindent}{0pt}%
  \setlength{\parskip}{0pt plus 0.3pt}%
  \let\item\@idxitem
}{%
  \ifkorrekturansicht\clearpage\fi
}
\makeatother

\IfFileExists{\jobname-pw.ind}{\input{\jobname-pw.ind}}{}

% Quellenangabe nur in der Leseansicht
\ifkorrekturansicht\else
% Fallback-Definitionen, falls die .tex-Datei \titel etc. nicht gesetzt hat
\providecommand{\titel}{}
\providecommand{\editorInnen}{}
\providecommand{\dateiname}{\jobname}

\vspace{3cm}

\vfill

\footnotesize
\textsc{Quelle}: \titel. Herausgegeben von {\editorInnen}. In: \emph{Arthur Schnitzler: Briefwechsel mit Autorinnen und Autoren}.
 Digitale Edition, https://schnitzler-briefe.acdh.oeaw.ac.at/{\dateiname}.html (Stand \today)
\fi

\end{document}


