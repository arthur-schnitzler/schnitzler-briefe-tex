%% latex-leseansicht-vorspann.tex
%% Vorspann für die Leseansicht.
%% Lädt die gemeinsame Datei latex-vorspann.tex mit nicht gesetztem Schalter.

\newif\ifkorrekturansicht
\korrekturansichtfalse

\input{../tex-inputs/latex-vorspann}


\section[Arthur Schnitzler an Gustav Schwarzkopf, 15. 7. 1897]{L04121 Arthur Schnitzler an Gustav Schwarzkopf, 15. 7. 1897}
\nopagebreak\mylabel{L04121v}
\rehead{ }\normalsize\beginnumbering\briefempfaengerindex{Schwarzkopf, Gustav@\textsc{Schwarzkopf, Gustav}!zzzSchnitzler, Arthur@\emph{von Arthur Schnitzler}!1897-07-152@{15. 7. 1897}|(be}
\toendnotes[C]{\smallbreak\pagebreak[2]}
\correspDesc{Versand  durch Arthur Schnitzler am 15. 7. 1897 in Bad Ischl
\newline{}Erhalt  durch Gustav Schwarzkopf im Zeitraum [16. 7. 1897
                  – 20. 7. 1897?] in Wien}\toendnotes[C]{\smallbreak}
\Standort{CUL, Schnitzler, B 96.}
\physDesc{Brief, 1 Blatt, 4 Seiten, 1294 Zeichen
\newline{}Handschrift: Bleistift, deutsche Kurrent}\toendnotes[C]{\smallbreak}
\pstart
           \noindent{}{\pb}Lieber Guſtav, Ihre Profezeihungen ſind zum Theil eingetroffen; am
               wenigſten die (allerdings unvorſichtigſte) mit dem Arbeiten, am ſtärksten die
               (cyniſche) mit dem Zeitvertreib. Ich habe ein Stück\pwindex{Schnitzler, Arthur 15. 5. 1862 Wien – 21. 10. 1931 ebd.@\textsc{Schnitzler, Arthur} (15. 5. 1862 Wien – 21. 10. 1931 ebd.), \emph{Schriftsteller, Mediziner}!Vermächtnis. Schauspiel in drei Akten@\strich\emph{Das Vermächtnis. Schauspiel in drei Akten}|pwv} zu ſchreiben begonnen; ein ganz andres, als ich
               urſprünglich wollte; war im Anfang ſehr begeiſtert, bin aber nun merklich abgekühlt.
               Radeln thu ich viel – noch i{\geminationm}er mit dem alten Rad – das
                  Londoner\oindex{London@\textbf{London}, \emph{Hauptstadt}|pw} iſt noch nicht einmal fertig. {\pb}Aber die Axen beginnen zu knacken, und
               ich würde es nicht einmal mehr wagen, Ihnen dieſen alten Kaſten anzutragen. Mit der
               Zeit werden Räder nemlich lebensgefährliche Geſchenke (auch für die, die{ }ſich
               draufſetzen.) Sehr bald muſs ich fort von hier und glaube mit Sicherheit drauf
               rechnen zu können, daß ich Sie in Wien\oindex{Wien@\textbf{Wien}, \emph{Verwaltungsgebiet}|pw} noch
               antreffe. Ich muſs nemlich in der Ihnen bekannten \label{K_L04121-1v}\edtext{Affaire}{\lemma{\textnormal{\emph{Affaire}}}\Cendnote{\textnormal{Seine
                  Lebensgefährtin Marie Reinhard\pwindex{Reinhard, Marie 13.\,3.\,1871 Wien – 18.\,3.\,1899 ebd.@\textsc{Reinhard, Marie} (13.\,3.\,1871 Wien – 18.\,3.\,1899 ebd.), \emph{Gesangspädagogin}|pwk} war
                  schwanger. (Das Kind\pwindex{?? [Totgeborener Sohn von Arthur Schnitzler und Marie Reinhard] 24.\,9.\,1897 Endresstraße 68 – 24.\,9.\,1897 ebd.@\textsc{?? [Totgeborener Sohn von Arthur Schnitzler und Marie Reinhard]} (24.\,9.\,1897 Endresstraße 68 – 24.\,9.\,1897 ebd.)|pwkv}
                  überlebte die Geburt nicht.)}}}\label{K_L04121-1} »\textsc{per} ſofort« wie
               der ſchöne Ausdruck lautet, \label{K_L04121-2v}\edtext{Woh{\pb}nung ſuchen, da jener biedere Forſtmann\pwindex{\textcolor{red}{\textsuperscript{XXXX indx1}}|pwv} über die von mir
               (allerdgs ohne Angabe) gemiethete bereits verfügt hat}{\lemma{\textnormal{\emph{Wohnung suchen, … hat}}}\Cendnote{\textnormal{Am 8. 3. 1897 hatte er mit Marie
                     Reinhard\pwindex{Reinhard, Marie 13.\,3.\,1871 Wien – 18.\,3.\,1899 ebd.@\textsc{Reinhard, Marie} (13.\,3.\,1871 Wien – 18.\,3.\,1899 ebd.), \emph{Gesangspädagogin}|pwk} in Türnitz\oindex{XXXX Ortsangabe fehlt|pwk} im Pamsenhof\oindex{XXXX Ortsangabe fehlt|pwk} eine Bleibe angesehen, in der
                     Marie Reinhard\pwindex{Reinhard, Marie 13.\,3.\,1871 Wien – 18.\,3.\,1899 ebd.@\textsc{Reinhard, Marie} (13.\,3.\,1871 Wien – 18.\,3.\,1899 ebd.), \emph{Gesangspädagogin}|pwk} das Kind Ende August 1897 auf die Welt
                  bringen sollte. Vermutlich am 26. 3. 1897 kam der Förster
                     Loidl\pwindex{\textcolor{red}{\textsuperscript{XXXX indx1}}|pwk} nach Wien\oindex{Wien@\textbf{Wien}, \emph{Verwaltungsgebiet}|pwk} und die Abmachung wurde getroffen (vgl. Leopoldine Kirchrath\pwindex{\textcolor{red}{\textsuperscript{XXXX indx1}}|pwk} an Schnitzler, 24. 3. 1897,
                        \emph{Deutsches Literaturarchiv Marbach},
                        HS.1985.1.03675). Am 29. 6. 1897
                  erfuhr Schnitzler durch einen nicht überlieferten Brief von Kirchrath\pwindex{\textcolor{red}{\textsuperscript{XXXX indx1}}|pwk}, dass der »Pamsenhof\oindex{XXXX Ortsangabe fehlt|pw} fort«, also vergeben war. Zu diesem Zeitpunkt 
                  hielt sich Marie Reinhard\pwindex{Reinhard, Marie 13.\,3.\,1871 Wien – 18.\,3.\,1899 ebd.@\textsc{Reinhard, Marie} (13.\,3.\,1871 Wien – 18.\,3.\,1899 ebd.), \emph{Gesangspädagogin}|pwk} noch in der Schweiz\oindex{XXXX Ortsangabe fehlt|pwk}
                  auf.}}}\label{K_L04121-2}. – Ahnen Sie, wie unbequem{ }ſolche
               Sachen ſind?– Aber ſpäter veranlangt man da{\geminationn}
               Dankbarkeit, Liebe, ja ſogar Reſpekt! –\pend
           
\pstart
           Sind Sie ſchon über Ihren So{\geminationm}er{ }ſchlüſſig? Am \label{K_L04121-3v}\edtext{26.{ }\textsc{circa}}{\lemma{\textnormal{\emph{26. circa}}}\Cendnote{\textnormal{Es war bereits am 25. 7. 1897 wieder in Wien\oindex{Wien@\textbf{Wien}, \emph{Verwaltungsgebiet}|pwk}.}}}\label{K_L04121-3} bin ich in Wien\oindex{Wien@\textbf{Wien}, \emph{Verwaltungsgebiet}|pw}. Ich ſuch Sie bald auf. Iſt Ihr Bruder\pwindex{Schwarzkopf, Max 12.\,6.\,1857 Wien – 14.\,4.\,1928 ebd.@\textsc{Schwarzkopf, Max} (12.\,6.\,1857 Wien – 14.\,4.\,1928 ebd.), \emph{Rechtsanwalt}|pwv} noch in Wien\oindex{Wien@\textbf{Wien}, \emph{Verwaltungsgebiet}|pw}? Grüßen Sie Ihn für dieſen Fall.\pend
           
\pstart
           {\pb}Auf baldges Wiederſehn{\\[\baselineskip]} alſo und
               herzliche Grüße{\\[\baselineskip]} von Ihrem \spacefill\mbox{ArthSch}\pend
           \leftskip=0em{}
\pstart
           \textsc{Ischl\oindex{Bad Ischl@\textbf{Bad Ischl}|pw}}{ }15/7 97\pend
           \selectlanguage{ngerman}\endnumbering\briefempfaengerindex{Schwarzkopf, Gustav@\textsc{Schwarzkopf, Gustav}!zzzSchnitzler, Arthur@\emph{von Arthur Schnitzler}!1897-07-152@{15. 7. 1897}|)be}\mylabel{L04121h}
\begin{anhang}
\end{anhang}\newcommand{\dateiname}{L04121}\newcommand{\titel}{Arthur Schnitzler an Gustav Schwarzkopf, 15. 7. 1897}\newcommand{\editorInnen}{Herausgegeben von Jahnke, SelmaMüller, Martin Anton}%% latex-leseansicht-abspann.tex
%% Abspann für die Leseansicht.
%% Der Schalter \ifkorrekturansicht ist bereits durch den Vorspann gesetzt.

%% latex-abspann.tex
%% Gemeinsamer Abspann für Korrekturansicht und Leseansicht.
%% Setzt den Schalter \ifkorrekturansicht voraus (gesetzt in den
%% einbindenden Dateien latex-korrekturansicht-abspann.tex bzw.
%% latex-leseansicht-abspann.tex).
%% ---------------------------------------------------------------

\normalsize

% Das esempio-Environment wird nur in der Leseansicht benötigt
\ifkorrekturansicht\else
\newenvironment{esempio}[3]%
{
    \vspace{1.5ex}
    \rlap{\underline{#1}}
    \par
    \setlength{\parindent}{0cm}
    \nopagebreak
    \leftskip=#2cm
    \rightskip=#3cm
}
{
    \par
}
\fi

\doendnotes{C}
\bigskip
\vfill

\clearpage

\footnotesize

\ifkorrekturansicht
  \lohead{\textsc{register}}
\fi

% theindex-Environment neu definieren ohne reledmac
\makeatletter
\renewenvironment{theindex}{%
  \ifkorrekturansicht
    \section*{\indexname}%
  \else
    \subsubsection*{Index der erwähnten Entitäten}%
  \fi
  \setlength{\parindent}{0pt}%
  \setlength{\parskip}{0pt plus 0.3pt}%
  \let\item\@idxitem
}{%
  \ifkorrekturansicht\clearpage\fi
}
\makeatother

\IfFileExists{\jobname-pw.ind}{\input{\jobname-pw.ind}}{}

% Quellenangabe nur in der Leseansicht
\ifkorrekturansicht\else
% Fallback-Definitionen, falls die .tex-Datei \titel etc. nicht gesetzt hat
\providecommand{\titel}{}
\providecommand{\editorInnen}{}
\providecommand{\dateiname}{\jobname}

\vspace{3cm}

\vfill

\footnotesize
\textsc{Quelle}: \titel. Herausgegeben von {\editorInnen}. In: \emph{Arthur Schnitzler: Briefwechsel mit Autorinnen und Autoren}.
 Digitale Edition, https://schnitzler-briefe.acdh.oeaw.ac.at/{\dateiname}.html (Stand \today)
\fi

\end{document}


