%% latex-korrekturansicht-vorspann.tex
%% Vorspann für die Korrekturansicht.
%% Lädt die gemeinsame Datei latex-vorspann.tex mit gesetztem Schalter.

\newif\ifkorrekturansicht
\korrekturansichttrue

\input{../tex-inputs/latex-vorspann}


\section[Arthur Schnitzler an Hermann Bahr, {[}6.{]} 12. 1900]{L01085 Arthur Schnitzler an Hermann Bahr, {[}6.{]} 12. 1900}
\nopagebreak\mylabel{L01085v}
\rehead{ }\normalsize\beginnumbering\briefempfaengerindex{Bahr, Hermann@\textsc{Bahr, Hermann}!zzzSchnitzler, Arthur@\emph{von Arthur Schnitzler}!1900-12-061@{{[}6.{]} 12. 1900}|(be}
\toendnotes[C]{\smallbreak\pagebreak[2]}\Standort{TMW, HS AM 60151 Ba.}
\physDesc{Briefkarte, 345 Zeichen
\newline{}Handschrift: schwarze Tinte, deutsche Kurrent
\newline{}Ordnung: Lochung }
\buchAbdrucke{\weitereDrucke{1) Arthur Schnitzler: \emph{The Letters of Arthur Schnitzler to Hermann Bahr}. Chapel Hill: \emph{The University of North Carolina Press} 1978, S. 67.} \weitereDrucke{2) Hermann Bahr, Arthur Schnitzler: \emph{Briefwechsel, Aufzeichnungen, Dokumente (1891–1931)}. Göttingen: \emph{Wallstein} 2018, S. 191.} }\toendnotes[C]{\smallbreak}
\pstart
           \noindent{}{\pb}lieber Hermann, ich muſs \damage{dir}{ }ſagen, wie sehr mich dein Feuilleton\pwindex{Schleier der Beatrice. (Schauspiel in fuenf Akten von Arthur Schnitzler. Zum ersten Male aufgefuehrt am Breslauer Lobe-Theater am 1. Dez. 1900)@\emph{Der Schleier der Beatrice. (Schauspiel in fünf Akten von Arthur Schnitzler. Zum ersten Male aufgeführt am Breslauer Lobe-Theater am 1. Dez. 1900)}|pwv} über die \textsc{Beatrice\pwindex{Schleier der Beatrice. Schauspiel in fuenf Akten@\emph{Der Schleier der Beatrice. Schauspiel in fünf Akten}|pw}} gefreut hat. Und zugleich noch einmal danken, dſs du nach Breslau\oindex{Breslau@\textbf{Breslau}, \emph{P.PPLA}|pw} gefahren biſt. Du erlaubſt mir gewiſs, darin {\pb}\introOben{}noch\introOben{} et\damage{wa}s andres zu ſehen als die Erfüllg einer »journaliſtiſchen Pflicht«\substVorne{}\textsuperscript{.}\substDazwischen{},\substHinten{} wie du neulich geſagt haſt.\pend
           
\pstart
           Auf baldiges Wiederſehen.{\\[\baselineskip]}Herzlichſt dein{\\[\baselineskip]}\spacefill\mbox{Arthur}\pend
           \leftskip=0em{}
\pstart
           \damage{6.} 12. 900.\pend
           \selectlanguage{ngerman}\endnumbering\briefempfaengerindex{Bahr, Hermann@\textsc{Bahr, Hermann}!zzzSchnitzler, Arthur@\emph{von Arthur Schnitzler}!1900-12-061@{{[}6.{]} 12. 1900}|)be}\mylabel{L01085h}  \normalsize

\doendnotes{C}
\bigskip
\vfill

\clearpage

\footnotesize

\lohead{\textsc{register}}

% Definiere theindex-Environment komplett neu ohne reledmac
\makeatletter
\renewenvironment{theindex}{%
  \section*{\indexname}%
  \setlength{\parindent}{0pt}%
  \setlength{\parskip}{0pt plus 0.3pt}%
  \let\item\@idxitem
}{%
  \clearpage
}
\makeatother

\IfFileExists{\jobname-pw.ind}{\input{\jobname-pw.ind}}{}

\end{document}

      