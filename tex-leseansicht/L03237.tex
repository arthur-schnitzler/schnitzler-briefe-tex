%% latex-leseansicht-vorspann.tex
%% Vorspann für die Leseansicht.
%% Lädt die gemeinsame Datei latex-vorspann.tex mit nicht gesetztem Schalter.

\newif\ifkorrekturansicht
\korrekturansichtfalse

\input{../tex-inputs/latex-vorspann}


         
         \renewcommand{\erwaehntePersonen}{Personen: Siegfried Jacobsohn}
         \renewcommand{\erwaehnteOrte}{Orte: Berlin, Hotel Bristol Berlin, Hotel Continental (Berlin), Staatsoper Berlin}
         \renewcommand{\erwaehnteWerke}{Werke: Fidelio}
               \section[ Paul Goldmann an Arthur Schnitzler, 20. 11. 1905]{ Paul Goldmann an Arthur Schnitzler, 20. 11. 1905}\nopagebreak\mylabel{v}\rehead{ }\begin{ledgroupsized}[t]{13cm}\normalsize\beginnumbering \toendnotes[C]{\smallbreak\pagebreak[2]} \Standort{DLA, A:Schnitzler, HS.NZ85.1.3175.}
\physDesc{Postkarte
\newline{}Handschrift: 1) blaue Tinte, deutsche Kurrent\hspace{1em}2) blaue Tinte, lateinische Kurrent (\noindent{}Adresse)\hspace{1em}\newline{}Versand: 1) Stempel: »\nobreak{}\oindex{Berlin@\textbf{Berlin}|pwk}Berlin S. W. 11, 20. 11. 05, 11\textsuperscript{20} V.\nobreak{}«.   2) Stempel: »\nobreak{}\oindex{Berlin@\textbf{Berlin}|pwk}Berlin N. W. 7, 20. 11. 05, 11\textsuperscript{40} V.\nobreak{}«.  3) Stempel: »\nobreak{}\oindex{Hotel Continental (Berlin)@\textbf{Hotel Continental (Berlin)}|pwk}Continental
                                          Hotel, Nov 19, 11\textsubscript{58}PM\nobreak{}«. 
\newline{}Schnitzler: mit Bleistift die Jahreszahl »{[}19{]}05« und das Datum »20/11« vermerkt }\toendnotes[C]{\smallbreak}\pstart{}{\pb}Rohr-\textcolor{gray}{\textbf{Poſtkarte}}\pend{}\pstart{}Herrn\pend{}\pstart{}Dr. Arthur Schnitzler\pend{}\pstart{}Berlin\oindex{Berlin@\textbf{Berlin}|pw}\pend{}\pstart{}Hotel \substVorne{}\textsuperscript{B\textcolor{gray}{ri}st\textcolor{gray}{o}l\oindex{Hotel Bristol Berlin@\textbf{Hotel Bristol Berlin}|pw}}{\allowbreak}\substDazwischen{}Continental\oindex{Hotel Continental (Berlin)@\textbf{Hotel Continental (Berlin)}|pw}\substHinten{}\pend{}{\bigskip}\pstart
           \noindent{}{\pb}Montag. Lieber Freund,
               Es hat mir ſehr leid gethan, Deinen lieben Beſuch geſtern verſäumt zu haben. Ich muß wenige Minuten vorher weggegangen ſein.
               Hätteſt Du mir telephonirt, ſo hätte ich Dich gern erwartet.\pend
           \pstart
           Willſt Du heut{ }Abend mit mir in die \label{K-L03237-1v}\edtext{Oper\oindex{Staatsoper Berlin@\textbf{Staatsoper Berlin}|pwv}}{\lemma{\textnormal{\emph{Oper}}}\Cendnote{\textnormal{Schnitzler\pwindex{Schnitzler, Arthur 15.05.1862 – 21.10.1931@\textsc{Schnitzler, Arthur} (15.05.1862 – 21.10.1931), \emph{Schriftsteller, Mediziner}|pwk} verbrachte den Abend
                  nicht mit Goldmann\pwindex{Goldmann, Paul 31.01.1865 – 25.09.1935@\textsc{Goldmann, Paul} (31.01.1865 – 25.09.1935), \emph{Schriftsteller, Journalist}|pwk}, sondern mit Siegfried Jacobsohn\pwindex{Jacobsohn, Siegfried 28.01.1881 – 03.12.1926@\textsc{Jacobsohn, Siegfried} (28.01.1881 – 03.12.1926), \emph{Journalist, Kritiker, Publizist}|pwk}. Siehe A. S.: \emph{Tagebuch}, 20. 11. 1905.}}}\label{K-L03237-1h} gehen
                  (\textsc{Fidelio\pwindex{\textcolor{red}{\textsuperscript{XXXX1 indx}}!Fidelio1805/1806/1814@\strich\emph{Fidelio} {[}1805/1806/1814{]}|pw}}, Urfaſſung)? Bis 4 Uhr halte ich das Billet zu Deiner Verfügung.
               Erbitte telephoniſche Antwort.\pend
           \pstart
           Herzlichſt {\\[\baselineskip]}Dein
               \spacefill\mbox{Paul Goldmann}\pend
           \leftskip=0em{}
         
         \endnumbering\mylabel{h}\end{ledgroupsized}  \newcommand{\dateiname}{L03237}\newcommand{\titel}{Paul Goldmann an Arthur Schnitzler, 20. 11. 1905}\newcommand{\editorInnen}{Martin Anton Müller und Laura Untner}%% latex-leseansicht-abspann.tex
%% Abspann für die Leseansicht.
%% Der Schalter \ifkorrekturansicht ist bereits durch den Vorspann gesetzt.

%% latex-abspann.tex
%% Gemeinsamer Abspann für Korrekturansicht und Leseansicht.
%% Setzt den Schalter \ifkorrekturansicht voraus (gesetzt in den
%% einbindenden Dateien latex-korrekturansicht-abspann.tex bzw.
%% latex-leseansicht-abspann.tex).
%% ---------------------------------------------------------------

\normalsize

% Das esempio-Environment wird nur in der Leseansicht benötigt
\ifkorrekturansicht\else
\newenvironment{esempio}[3]%
{
    \vspace{1.5ex}
    \rlap{\underline{#1}}
    \par
    \setlength{\parindent}{0cm}
    \nopagebreak
    \leftskip=#2cm
    \rightskip=#3cm
}
{
    \par
}
\fi

\doendnotes{C}
\bigskip
\vfill

\clearpage

\footnotesize

\ifkorrekturansicht
  \lohead{\textsc{register}}
\fi

% theindex-Environment neu definieren ohne reledmac
\makeatletter
\renewenvironment{theindex}{%
  \ifkorrekturansicht
    \section*{\indexname}%
  \else
    \subsubsection*{Index der erwähnten Entitäten}%
  \fi
  \setlength{\parindent}{0pt}%
  \setlength{\parskip}{0pt plus 0.3pt}%
  \let\item\@idxitem
}{%
  \ifkorrekturansicht\clearpage\fi
}
\makeatother

\IfFileExists{\jobname-pw.ind}{\input{\jobname-pw.ind}}{}

% Quellenangabe nur in der Leseansicht
\ifkorrekturansicht\else
% Fallback-Definitionen, falls die .tex-Datei \titel etc. nicht gesetzt hat
\providecommand{\titel}{}
\providecommand{\editorInnen}{}
\providecommand{\dateiname}{\jobname}

\vspace{3cm}

\vfill

\footnotesize
\textsc{Quelle}: \titel. Herausgegeben von {\editorInnen}. In: \emph{Arthur Schnitzler: Briefwechsel mit Autorinnen und Autoren}.
 Digitale Edition, https://schnitzler-briefe.acdh.oeaw.ac.at/{\dateiname}.html (Stand \today)
\fi

\end{document}


      