%% latex-korrekturansicht-vorspann.tex
%% Vorspann für die Korrekturansicht.
%% Lädt die gemeinsame Datei latex-vorspann.tex mit gesetztem Schalter.

\newif\ifkorrekturansicht
\korrekturansichttrue

\input{../tex-inputs/latex-vorspann}


\section[ Paul Goldmann an Arthur Schnitzler, 20. 11. 1905]{L03237 Paul Goldmann an Arthur Schnitzler, 20. 11. 1905}
\nopagebreak\mylabel{L03237v}
\rehead{ }\normalsize\beginnumbering\briefempfaengerindex{Schnitzler, Arthur@\textsc{Schnitzler, Arthur}!zzzGoldmann, Paul@\emph{von Paul Goldmann}!1905-11-201@{20. 11. 1905}|(be}
\toendnotes[C]{\smallbreak\pagebreak[2]}\Standort{DLA, A:Schnitzler, HS.NZ85.1.3175.}
\physDesc{Postkarte, 436 Zeichen
\newline{}Handschrift: 1) blaue Tinte, deutsche Kurrent\hspace{1em}2) blaue Tinte, lateinische Kurrent (\noindent{}Adresse)\hspace{1em}
\newline{}Versand: 1) Stempel: »\nobreak{}\oindex{Berlin@\textbf{Berlin}, \emph{P.PPLC}|pwk}Berlin S. W. 11, 20. 11. 05, 11\textsuperscript{20} V.\nobreak{}«.   2) Stempel: »\nobreak{}\oindex{Berlin@\textbf{Berlin}, \emph{P.PPLC}|pwk}Berlin N. W. 7, 20. 11. 05, 11\textsuperscript{40} V.\nobreak{}«.  3) Stempel: »\nobreak{}\oindex{Hotel Continental [Berlin]@\textbf{Hotel Continental [Berlin]}, \emph{Hotel (K.HTL)}|pwk}Continental Hotel, Nov 19, 11\textsubscript{58}PM\nobreak{}«. 
\newline{}Schnitzler: mit Bleistift die Jahreszahl »{[}19{]}05« und das Datum »20/11« vermerkt }\toendnotes[C]{\smallbreak}\pstart{}{\pb}Rohr-\textcolor{gray}{\textbf{Poſtkarte}}\pend{}\pstart{}Herrn\pend{}\pstart{}Dr. Arthur Schnitzler\pend{}\pstart{}Berlin\oindex{Berlin@\textbf{Berlin}, \emph{P.PPLC}|pw}\pend{}\pstart{}Hotel \substVorne{}\textsuperscript{B\textcolor{gray}{ri}st\textcolor{gray}{o}l\oindex{Hotel Bristol Berlin@\textbf{Hotel Bristol Berlin}, \emph{Hotel (K.HTL)}|pw}}\substDazwischen{}Continental\oindex{Hotel Continental [Berlin]@\textbf{Hotel Continental [Berlin]}, \emph{Hotel (K.HTL)}|pw}\substHinten{}\pend{}{\bigskip}\vspace{1em}
\pstart
           \noindent{}{\pb}Montag. Lieber Freund,
               Es hat mir ſehr leid gethan, Deinen lieben Beſuch geſtern verſäumt zu haben. Ich muß wenige Minuten vorher weggegangen ſein.
               Hätteſt Du mir telephonirt, ſo hätte ich Dich gern erwartet.\pend
           
\pstart
           Willſt Du heut{ }Abend mit mir in die \label{K_L03237-1v}\edtext{Oper\oindex{Staatsoper Berlin@\textbf{Staatsoper Berlin}, \emph{S.OPRA}|pwv}}{\lemma{\textnormal{\emph{Oper}}}\Cendnote{\textnormal{Schnitzler verbrachte den Abend
                  nicht mit Goldmann\pwindex{Goldmann, Paul 31.01.1865 – 25.09.1935@\textsc{Goldmann, Paul} (31.01.1865 – 25.09.1935), \emph{Schriftsteller/Schriftstellerin, Journalist/Journalistin}|pwk}, sondern mit Siegfried Jacobsohn\pwindex{Jacobsohn, Siegfried 28.01.1881 – 03.12.1926@\textsc{Jacobsohn, Siegfried} (28.01.1881 – 03.12.1926), \emph{Journalist/Journalistin, Kritiker/Kritikerin, Publizist/Publizistin}|pwk}. Siehe A. S.: \emph{Tagebuch}, 20. 11. 1905.}}}\label{K_L03237-1} gehen
                  (\textsc{Fidelio\pwindex{Fidelio@\emph{Fidelio}|pw}}, Urfaſſung)? Bis 4 Uhr halte ich das Billet zu Deiner Verfügung.
               Erbitte telephoniſche Antwort.\pend
           
\pstart
           Herzlichſt {\\[\baselineskip]}Dein \spacefill\mbox{Paul Goldmann}\pend
           \leftskip=0em{}\selectlanguage{ngerman}\endnumbering\briefempfaengerindex{Schnitzler, Arthur@\textsc{Schnitzler, Arthur}!zzzGoldmann, Paul@\emph{von Paul Goldmann}!1905-11-201@{20. 11. 1905}|)be}\mylabel{L03237h}  \normalsize

\doendnotes{C}
\bigskip
\vfill

\clearpage

\footnotesize

\lohead{\textsc{register}}

% Definiere theindex-Environment komplett neu ohne reledmac
\makeatletter
\renewenvironment{theindex}{%
  \section*{\indexname}%
  \setlength{\parindent}{0pt}%
  \setlength{\parskip}{0pt plus 0.3pt}%
  \let\item\@idxitem
}{%
  \clearpage
}
\makeatother

\IfFileExists{\jobname-pw.ind}{\input{\jobname-pw.ind}}{}

\end{document}

      