%% latex-leseansicht-vorspann.tex
%% Vorspann für die Leseansicht.
%% Lädt die gemeinsame Datei latex-vorspann.tex mit nicht gesetztem Schalter.

\newif\ifkorrekturansicht
\korrekturansichtfalse

\input{../tex-inputs/latex-vorspann}


\section[Arthur Schnitzler an Richard Beer-Hofmann, 30. 7. 1916]{L02235 Arthur Schnitzler an Richard Beer-Hofmann, 30. 7. 1916}
\nopagebreak\mylabel{L02235v}
\rehead{ }\normalsize\beginnumbering\briefempfaengerindex{Beer-Hofmann, Richard@\textsc{Beer-Hofmann, Richard}!zzzSchnitzler, Arthur@\emph{von Arthur Schnitzler}!1916-07-301@{30. 7. 1916}|(be}
\toendnotes[C]{\smallbreak\pagebreak[2]}
\correspDesc{Versand  durch Arthur Schnitzler am 30. 7. 1916 in Altaussee
\newline{}Erhalt  durch Richard Beer-Hofmann im Zeitraum [31. 7. 1916
                  – 4. 8. 1916?] in Bad Ischl}\toendnotes[C]{\smallbreak}
\Standort{YCGL, MSS 31.}
\physDesc{Bildpostkarte, 496 Zeichen
\newline{}Handschrift: Bleistift, lateinische Kurrent
\newline{}Versand: Stempel: »\nobreak{}30. VII. 16\nobreak{}«.  
\newline{}Beer-Hofmann: mit blauem Buntstift Erhalt
                                 festgehalten: »E.« }\toendnotes[C]{\smallbreak}\pstart{}{\pb}Fischerndorf 79\oindex{Fischerndorf@\textbf{Fischerndorf}|pw}\pend{}\pstart{}Schnitzler\pend{}{\bigskip}\pstart{}Hrn Dr. Richard Beer-Hofmann\pend{}\pstart{}Bad Ischl\oindex{Bad Ischl@\textbf{Bad Ischl}|pw}.\pend{}\pstart{}Grazerstr 52\oindex{Grazer Straße [Bad Ischl]@\textbf{Grazer Straße [Bad Ischl]}, \emph{Straße}|pw}\pend{}{\bigskip}
\pstart
           \noindent{}\centering{}{\pb}\textcolor{gray}{\textbf{ALT-AUSSEE\oindex{Altaussee@\textbf{Altaussee}, \emph{Verwaltungsgebiet}|pw} 717 m Seehöhe mit dem Dachstein\oindex{Dachstein@\textbf{Dachstein}, \emph{Berg}|pw}
                  2996 m. Steier. Salzkammergut\oindex{Salzkammergut@\textbf{Salzkammergut}, \emph{Region}|pw}.}}\pend
           \vspace{1em}
\pstart
           \raggedleft{}{\pb}30. 7.\pend
           \vspace{0.5em}
\pstart
           lieber Richard,{ }\label{K_L02235-1v}\edtext{Mittwoch}{\lemma{\textnormal{\emph{Mittwoch}}}\Cendnote{\textnormal{Es wurde Donnerstag, vgl. A. S.: \emph{Tagebuch}, 3. 8. 1916.}}}\label{K_L02235-1} dürften wir bei schönem Wetter in Ischl\oindex{Bad Ischl@\textbf{Bad Ischl}|pw}
               sein u dort (wahrscheinlich Kaiserkrone\oindex{Hotel Kaiserkrone@\textbf{Hotel Kaiserkrone}, \emph{Hotel}|pw})
               übernachten. Ich will zu Fuß hinüber und um 1, ½ 2 in der
                  Kaiserkr.\oindex{Hotel Kaiserkrone@\textbf{Hotel Kaiserkrone}, \emph{Hotel}|pw} speisen. Olga\pwindex{Schnitzler, Olga 17.\,1.\,1882 Wien – 13.\,1.\,1970 Lugano@\textsc{Schnitzler, Olga} (17.\,1.\,1882 Wien – 13.\,1.\,1970 Lugano), \emph{Schauspielerin, Sängerin}|pw} ko{\geminationm}t erst
                  Nachm. (\textcolor{gray}{Kopfwaschen}, Aschau\oindex{Aschau@\textbf{Aschau}|pw} etc.) Am Abend hoffen wir bei Sonnenschein\oindex{Restaurant Sonnenschein@\textbf{Restaurant Sonnenschein}, \emph{Restaurant}|pw} mit Ihnen Beiden\pwindex{Beer-Hofmann, Paula 25.\,2.\,1879 Wien – 30.\,10.\,1939 Zürich@\textsc{Beer-Hofmann, Paula} (25.\,2.\,1879 Wien – 30.\,10.\,1939 Zürich)|pw} zu nachtmahlen – aber ich suche Sie (binden Sie sich nicht!) {\pb}we{\geminationn}s irgend geht schon
               vorher auf. Herzlichst und viele Grüße von uns zu Ihnen.\pend
           
\pstart
           Ihr{\\[\baselineskip]}\spacefill\mbox{Arthur}\pend
           \leftskip=0em{}\selectlanguage{ngerman}\endnumbering\briefempfaengerindex{Beer-Hofmann, Richard@\textsc{Beer-Hofmann, Richard}!zzzSchnitzler, Arthur@\emph{von Arthur Schnitzler}!1916-07-301@{30. 7. 1916}|)be}\mylabel{L02235h}  \newcommand{\dateiname}{L02235}\newcommand{\titel}{Arthur Schnitzler an Richard Beer-Hofmann, 30. 7. 1916}\newcommand{\editorInnen}{Martin Anton Müller und Gerd-Hermann Susen}%% latex-leseansicht-abspann.tex
%% Abspann für die Leseansicht.
%% Der Schalter \ifkorrekturansicht ist bereits durch den Vorspann gesetzt.

%% latex-abspann.tex
%% Gemeinsamer Abspann für Korrekturansicht und Leseansicht.
%% Setzt den Schalter \ifkorrekturansicht voraus (gesetzt in den
%% einbindenden Dateien latex-korrekturansicht-abspann.tex bzw.
%% latex-leseansicht-abspann.tex).
%% ---------------------------------------------------------------

\normalsize

% Das esempio-Environment wird nur in der Leseansicht benötigt
\ifkorrekturansicht\else
\newenvironment{esempio}[3]%
{
    \vspace{1.5ex}
    \rlap{\underline{#1}}
    \par
    \setlength{\parindent}{0cm}
    \nopagebreak
    \leftskip=#2cm
    \rightskip=#3cm
}
{
    \par
}
\fi

\doendnotes{C}
\bigskip
\vfill

\clearpage

\footnotesize

\ifkorrekturansicht
  \lohead{\textsc{register}}
\fi

% theindex-Environment neu definieren ohne reledmac
\makeatletter
\renewenvironment{theindex}{%
  \ifkorrekturansicht
    \section*{\indexname}%
  \else
    \subsubsection*{Index der erwähnten Entitäten}%
  \fi
  \setlength{\parindent}{0pt}%
  \setlength{\parskip}{0pt plus 0.3pt}%
  \let\item\@idxitem
}{%
  \ifkorrekturansicht\clearpage\fi
}
\makeatother

\IfFileExists{\jobname-pw.ind}{\input{\jobname-pw.ind}}{}

% Quellenangabe nur in der Leseansicht
\ifkorrekturansicht\else
% Fallback-Definitionen, falls die .tex-Datei \titel etc. nicht gesetzt hat
\providecommand{\titel}{}
\providecommand{\editorInnen}{}
\providecommand{\dateiname}{\jobname}

\vspace{3cm}

\vfill

\footnotesize
\textsc{Quelle}: \titel. Herausgegeben von {\editorInnen}. In: \emph{Arthur Schnitzler: Briefwechsel mit Autorinnen und Autoren}.
 Digitale Edition, https://schnitzler-briefe.acdh.oeaw.ac.at/{\dateiname}.html (Stand \today)
\fi

\end{document}


