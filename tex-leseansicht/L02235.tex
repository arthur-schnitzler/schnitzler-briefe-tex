%% latex-korrekturansicht-vorspann.tex
%% Vorspann für die Korrekturansicht.
%% Lädt die gemeinsame Datei latex-vorspann.tex mit gesetztem Schalter.

\newif\ifkorrekturansicht
\korrekturansichttrue

\input{../tex-inputs/latex-vorspann}


\section[Arthur Schnitzler an Richard Beer-Hofmann, 30. 7. 1916]{L02235 Arthur Schnitzler an Richard Beer-Hofmann, 30. 7. 1916}
\nopagebreak\mylabel{L02235v}
\rehead{ }\normalsize\beginnumbering\briefempfaengerindex{Beer-Hofmann, Richard@\textsc{Beer-Hofmann, Richard}!zzzSchnitzler, Arthur@\emph{von Arthur Schnitzler}!1916-07-301@{30. 7. 1916}|(be}
\toendnotes[C]{\smallbreak\pagebreak[2]}\Standort{YCGL, MSS 31.}
\physDesc{Bildpostkarte, 496 Zeichen
\newline{}Handschrift: Bleistift, lateinische Kurrent
\newline{}Versand: Stempel: »\nobreak{}30. VII. 16\nobreak{}«.  
\newline{}Beer-Hofmann: mit blauem Buntstift Erhalt
                                 festgehalten: »E.« }\toendnotes[C]{\smallbreak}\pstart{}{\pb}Fischerndorf 79\oindex{Fischerndorf@\textbf{Fischerndorf}, \emph{P.PPL}|pw}\pend{}\pstart{}Schnitzler\pend{}{\bigskip}\pstart{}Hrn Dr. Richard Beer-Hofmann\pend{}\pstart{}Bad Ischl\oindex{Bad Ischl@\textbf{Bad Ischl}, \emph{P.PPL}|pw}.\pend{}\pstart{}Grazerstr 52\oindex{Grazer Strasse [Bad Ischl]@\textbf{Grazer Straße [Bad Ischl]}, \emph{Straße (K.STR)}|pw}\pend{}{\bigskip}
\pstart
           \noindent{}\centering{}{\pb}\textcolor{gray}{\textbf{ALT-AUSSEE\oindex{Altaussee@\textbf{Altaussee}, \emph{A.ADM3}|pw} 717 m Seehöhe mit dem Dachstein\oindex{Dachstein@\textbf{Dachstein}, \emph{Berg (N.BRG)}|pw}
                  2996 m. Steier. Salzkammergut\oindex{Salzkammergut@\textbf{Salzkammergut}, \emph{L.RGN}|pw}.}}\pend
           \vspace{1em}
\pstart
           \raggedleft{}{\pb}30. 7.\pend
           \vspace{0.5em}
\pstart
           lieber Richard,{ }\label{K_L02235-1v}\edtext{Mittwoch}{\lemma{\textnormal{\emph{Mittwoch}}}\Cendnote{\textnormal{Es wurde Donnerstag, vgl. A. S.: \emph{Tagebuch}, 3. 8. 1916.}}}\label{K_L02235-1} dürften wir bei schönem Wetter in Ischl\oindex{Bad Ischl@\textbf{Bad Ischl}, \emph{P.PPL}|pw}
               sein u dort (wahrscheinlich Kaiserkrone\oindex{Hotel Kaiserkrone@\textbf{Hotel Kaiserkrone}, \emph{Hotel (K.HTL)}|pw})
               übernachten. Ich will zu Fuß hinüber und um 1, ½ 2 in der
                  Kaiserkr.\oindex{Hotel Kaiserkrone@\textbf{Hotel Kaiserkrone}, \emph{Hotel (K.HTL)}|pw} speisen. Olga\pwindex{Schnitzler, Olga 17.01.1882 – 13.01.1970@\textsc{Schnitzler, Olga} (17.01.1882 – 13.01.1970), \emph{Schauspieler/Schauspielerin, Sänger/Sängerin}|pw} ko{\geminationm}t erst
                  Nachm. (\textcolor{gray}{Kopfwaschen}, Aschau\oindex{Aschau@\textbf{Aschau}, \emph{eingemeindeter Ort (A.VOO)}|pw} etc.) Am Abend hoffen wir bei Sonnenschein\oindex{Restaurant Sonnenschein@\textbf{Restaurant Sonnenschein}, \emph{S.REST}|pw} mit Ihnen Beiden\pwindex{Beer-Hofmann, Paula 25.02.1879 – 30.10.1939@\textsc{Beer-Hofmann, Paula} (25.02.1879 – 30.10.1939)|pw} zu nachtmahlen – aber ich suche Sie (binden Sie sich nicht!) {\pb}we{\geminationn}s irgend geht schon
               vorher auf. Herzlichst und viele Grüße von uns zu Ihnen.\pend
           
\pstart
           Ihr{\\[\baselineskip]}\spacefill\mbox{Arthur}\pend
           \leftskip=0em{}\selectlanguage{ngerman}\endnumbering\briefempfaengerindex{Beer-Hofmann, Richard@\textsc{Beer-Hofmann, Richard}!zzzSchnitzler, Arthur@\emph{von Arthur Schnitzler}!1916-07-301@{30. 7. 1916}|)be}\mylabel{L02235h}  \normalsize

\doendnotes{C}
\bigskip
\vfill

\clearpage

\footnotesize

\lohead{\textsc{register}}

% Definiere theindex-Environment komplett neu ohne reledmac
\makeatletter
\renewenvironment{theindex}{%
  \section*{\indexname}%
  \setlength{\parindent}{0pt}%
  \setlength{\parskip}{0pt plus 0.3pt}%
  \let\item\@idxitem
}{%
  \clearpage
}
\makeatother

\IfFileExists{\jobname-pw.ind}{\input{\jobname-pw.ind}}{}

\end{document}

      