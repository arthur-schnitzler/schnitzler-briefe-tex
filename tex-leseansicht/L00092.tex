%% latex-leseansicht-vorspann.tex
%% Vorspann für die Leseansicht.
%% Lädt die gemeinsame Datei latex-vorspann.tex mit nicht gesetztem Schalter.

\newif\ifkorrekturansicht
\korrekturansichtfalse

\input{../tex-inputs/latex-vorspann}


\section[Hugo von Hofmannsthal an Arthur Schnitzler, {[}4. 4. 1892?{]}]{L00092 Hugo von Hofmannsthal an Arthur Schnitzler, {[}4. 4. 1892?{]}}
\nopagebreak\mylabel{L00092v}
\rehead{ }\normalsize\beginnumbering\briefempfaengerindex{Schnitzler, Arthur@\textsc{Schnitzler, Arthur}!zzzHofmannsthal, Hugo von@\emph{von Hugo von Hofmannsthal}!1892-04-041@{{[}4. 4. 1892?{]}}|(be}
\toendnotes[C]{\smallbreak\pagebreak[2]}
\correspDesc{Versand  durch Hugo von Hofmannsthal am [4. 4. 1892?] in Wien
\newline{}Erhalt  durch Arthur Schnitzler im Zeitraum [4. 4. 1892
                  – 8. 4. 1892?] in Wien}\toendnotes[C]{\smallbreak}
\Standort{CUL, Schnitzler, B 43.}
\physDesc{Brief, 1 Blatt, 4 Seiten, 1362 Zeichen (aufgeprägtes Wappen)
\newline{}Handschrift: schwarze Tinte, deutsche Kurrent
\newline{}Schnitzler: mit Bleistift datiert: »Anf April 92« und
                                 nummeriert: »24« 
\newline{}Editorischer Hinweis: eine Doppelseite fehlt; diese wird nach der Abschrift
                                 zitiert }
\buchAbdrucke{\weitereDrucke{Hugo von Hofmannsthal, Arthur Schnitzler: \emph{Briefwechsel}. Herausgegeben von Therese Nickl und Heinrich Schnitzler. Frankfurt am Main: \emph{S. Fischer} 1964, S. 19–20.} }\toendnotes[C]{\smallbreak}
\pstart{}{\pb}Lieber Freund.\pend\vspace{0.5em}
\pstart
           Ich habe ausdrücklich und wiederholt gebeten, meinen Namen als Überſetzer auf den
                  \label{K_L00092-1v}\edtext{Einladungen}{\lemma{\textnormal{\emph{Einladungen}}}\Cendnote{\textnormal{Es handelt sich um die Einladung für die
                  Veranstaltung am 13. 4. 1892 (Maeterlincks\pwindex{Maeterlinck, Maurice 29.\,8.\,1862 Gent – 6.\,5.\,1949 Nizza@\textsc{Maeterlinck, Maurice} (29.\,8.\,1862 Gent – 6.\,5.\,1949 Nizza), \emph{Schriftsteller}|pwk}{ }\emph{L’Intruse}\pwindex{Maeterlinck, Maurice 29.\,8.\,1862 Gent – 6.\,5.\,1949 Nizza@\textsc{Maeterlinck, Maurice} (29.\,8.\,1862 Gent – 6.\,5.\,1949 Nizza), \emph{Schriftsteller}!Intruse@\strich\emph{L’Intruse}|pwk}, in der
                  Übersetzung von Ferry Beraton\pwindex{Bératon, Ferry 6.\,12.\,1859 Wien – 11.\,2.\,1900 Venedig@\textsc{Bératon, Ferry} (6.\,12.\,1859 Wien – 11.\,2.\,1900 Venedig), \emph{Schriftsteller, Journalist, Maler}|pwk} sowie eine
                  einleitende »Conferènce« von Hermann Bahr\pwindex{Bahr, Hermann 19.\,7.\,1863 Linz – 15.\,1.\,1934 München@\textsc{Bahr, Hermann} (19.\,7.\,1863 Linz – 15.\,1.\,1934 München), \emph{Schriftsteller, Kritiker}|pwk}),
                  die, da vergessen worden war, eine polizeiliche Genehmigung einzuholen,
                  kurzfristig abgesagt wurde. Sie wurde dann – durch das Verbot mit gestiegenem
                  Publikumsinteresse – am 2. 5. 1892 abgehalten. Die Einladungskarte an
                     Marie Herzfeld\pwindex{Herzfeld, Marie 20.\,3.\,1855 Kőszeg – 22.\,9.\,1940 Mining@\textsc{Herzfeld, Marie} (20.\,3.\,1855 Kőszeg – 22.\,9.\,1940 Mining), \emph{Schriftstellerin, Übersetzerin}|pwk} wurde am
                     4. 4. 1892 aufgegeben (Hugo von Hofmannsthal\pwindex{Hofmannsthal, Hugo von 1.\,2.\,1874 Wien – 15.\,7.\,1929 Rodaun@\textsc{Hofmannsthal, Hugo von} (1.\,2.\,1874 Wien – 15.\,7.\,1929 Rodaun), \emph{Schriftsteller}|pwk}: \emph{Briefe an Marie Herzfeld}. Herausgegeben von Horst Weber. Heidelberg:
                        \emph{Lothar Stiehm}{ }1967, S. 24), am Vorabend fand eine Besprechung statt –
                  was die zeitliche Einordnung ermöglichen dürfte.}}}\label{K_L00092-1} nicht zu nennen. Man hat
               zwar mit Herrn von Goldſchmid\pwindex{Goldschmidt, Adalbert von 5.\,5.\,1848 Wien – 21.\,12.\,1906 ebd.@\textsc{Goldschmidt, Adalbert von} (5.\,5.\,1848 Wien – 21.\,12.\,1906 ebd.), \emph{Schriftsteller, Komponist}|pw} dieſe Rückſicht
               gehabt, mit mir aber nicht. Ich{ }ſtreiche auf meinen Einladungen, um weiter keine
               Geſchichten zu machen, das Loris einfach durch. Ich habe {\pb}weder Lust für Beratons\pwindex{Bératon, Ferry 6.\,12.\,1859 Wien – 11.\,2.\,1900 Venedig@\textsc{Bératon, Ferry} (6.\,12.\,1859 Wien – 11.\,2.\,1900 Venedig), \emph{Schriftsteller, Journalist, Maler}|pw}{ }Ueberſetzung\pwindex{Maeterlinck, Maurice 29.\,8.\,1862 Gent – 6.\,5.\,1949 Nizza@\textsc{Maeterlinck, Maurice} (29.\,8.\,1862 Gent – 6.\,5.\,1949 Nizza), \emph{Schriftsteller}!Intruse@\strich\emph{L’Intruse}|pw}, die ich nicht
               kenne, einzustehen noch hätte ich eine von mir unterzeichnete Ueberſetzung jemals von
                  Beraton\pwindex{Bératon, Ferry 6.\,12.\,1859 Wien – 11.\,2.\,1900 Venedig@\textsc{Bératon, Ferry} (6.\,12.\,1859 Wien – 11.\,2.\,1900 Venedig), \emph{Schriftsteller, Journalist, Maler}|pw} korrigieren lassen. Diesen groben
               Brief bekommen Sie, weil mir die andere{[}n{]} wurst sind, und Sie
               verdienen ihn auch, weil Sie bei der Besprechung (½ 11) wahrscheinlich{ }ſchläfrig waren und nicht aufgelegt, Tactlosigkeiten zu verhindern.\pend
           
\pstart
           Ich bitte Sie, zu veranlassen, dass mein Name auf den übrigen Einladungen
               ausgestrichen wird. Uebrigens ist der Stil der Einladungen ebenso hübsch als ihr
               Inhalt unzureichend – »werden zur Aufführung gelangen« iſt gerade lächerlich
                  »werden{[}«{]} – wieso? von wem? wodurch?\pend
           
\pstart
           Das ganze sieht aus als ob schon eine (gescheidte) Erklärung vorangegangen wäre. l’Intrus\pwindex{Maeterlinck, Maurice 29.\,8.\,1862 Gent – 6.\,5.\,1949 Nizza@\textsc{Maeterlinck, Maurice} (29.\,8.\,1862 Gent – 6.\,5.\,1949 Nizza), \emph{Schriftsteller}!Intruse@\strich\emph{L’Intruse}|pw} ist eine directe Verfälschung, das Stück
               heisst l’Intruse\pwindex{Maeterlinck, Maurice 29.\,8.\,1862 Gent – 6.\,5.\,1949 Nizza@\textsc{Maeterlinck, Maurice} (29.\,8.\,1862 Gent – 6.\,5.\,1949 Nizza), \emph{Schriftsteller}!Intruse@\strich\emph{L’Intruse}|pw}. {\pb}Seit wann ändert man Titel?\pend
           
\pstart
           Ich weiß noch nicht, ob ich mich entſchließen werde, dieſe Wiſche auszuſchicken. Wozu
               haben Sie dann geſtern die Geſchichte vor mir feſtgeſetzt? Wozu{ }ſind überhaupt
               Beſprechungen, wenn hinterdrein immer alles geändert wird?\pend
           
\pstart
           Ekelhaft!{\\[\baselineskip]}\spacefill\mbox{Loris.}\pend
           \leftskip=0em{}\selectlanguage{ngerman}\endnumbering\briefempfaengerindex{Schnitzler, Arthur@\textsc{Schnitzler, Arthur}!zzzHofmannsthal, Hugo von@\emph{von Hugo von Hofmannsthal}!1892-04-041@{{[}4. 4. 1892?{]}}|)be}\mylabel{L00092h}  \newcommand{\dateiname}{L00092}\newcommand{\titel}{Hugo von Hofmannsthal an Arthur Schnitzler, [4. 4. 1892?]}\newcommand{\editorInnen}{Martin Anton Müller und Gerd-Hermann Susen}%% latex-leseansicht-abspann.tex
%% Abspann für die Leseansicht.
%% Der Schalter \ifkorrekturansicht ist bereits durch den Vorspann gesetzt.

%% latex-abspann.tex
%% Gemeinsamer Abspann für Korrekturansicht und Leseansicht.
%% Setzt den Schalter \ifkorrekturansicht voraus (gesetzt in den
%% einbindenden Dateien latex-korrekturansicht-abspann.tex bzw.
%% latex-leseansicht-abspann.tex).
%% ---------------------------------------------------------------

\normalsize

% Das esempio-Environment wird nur in der Leseansicht benötigt
\ifkorrekturansicht\else
\newenvironment{esempio}[3]%
{
    \vspace{1.5ex}
    \rlap{\underline{#1}}
    \par
    \setlength{\parindent}{0cm}
    \nopagebreak
    \leftskip=#2cm
    \rightskip=#3cm
}
{
    \par
}
\fi

\doendnotes{C}
\bigskip
\vfill

\clearpage

\footnotesize

\ifkorrekturansicht
  \lohead{\textsc{register}}
\fi

% theindex-Environment neu definieren ohne reledmac
\makeatletter
\renewenvironment{theindex}{%
  \ifkorrekturansicht
    \section*{\indexname}%
  \else
    \subsubsection*{Index der erwähnten Entitäten}%
  \fi
  \setlength{\parindent}{0pt}%
  \setlength{\parskip}{0pt plus 0.3pt}%
  \let\item\@idxitem
}{%
  \ifkorrekturansicht\clearpage\fi
}
\makeatother

\IfFileExists{\jobname-pw.ind}{\input{\jobname-pw.ind}}{}

% Quellenangabe nur in der Leseansicht
\ifkorrekturansicht\else
% Fallback-Definitionen, falls die .tex-Datei \titel etc. nicht gesetzt hat
\providecommand{\titel}{}
\providecommand{\editorInnen}{}
\providecommand{\dateiname}{\jobname}

\vspace{3cm}

\vfill

\footnotesize
\textsc{Quelle}: \titel. Herausgegeben von {\editorInnen}. In: \emph{Arthur Schnitzler: Briefwechsel mit Autorinnen und Autoren}.
 Digitale Edition, https://schnitzler-briefe.acdh.oeaw.ac.at/{\dateiname}.html (Stand \today)
\fi

\end{document}


