%% latex-leseansicht-vorspann.tex
%% Vorspann für die Leseansicht.
%% Lädt die gemeinsame Datei latex-vorspann.tex mit nicht gesetztem Schalter.

\newif\ifkorrekturansicht
\korrekturansichtfalse

\input{../tex-inputs/latex-vorspann}


\section[Arthur Schnitzler an Hugo von Hofmannsthal, 4. 10. 1898]{L00850 Arthur Schnitzler an Hugo von Hofmannsthal, 4. 10. 1898}
\nopagebreak\mylabel{L00850v}
\rehead{ }\normalsize\beginnumbering\briefempfaengerindex{Hofmannsthal, Hugo von@\textsc{Hofmannsthal, Hugo von}!zzzSchnitzler, Arthur@\emph{von Arthur Schnitzler}!1898-10-041@{4. 10. 1898}|(be}
\toendnotes[C]{\smallbreak\pagebreak[2]}
\correspDesc{Versand  durch Arthur Schnitzler am 4. 10. 1898 in Berlin
\newline{}Erhalt  durch Hugo von Hofmannsthal im Zeitraum [5. 10. 1898
                  – 9. 10. 1898?] in Wien}\toendnotes[C]{\smallbreak}
\Standort{FDH, Hs-30885,77.}
\physDesc{Brief, 1 Blatt, 4 Seiten, 1187 Zeichen
\newline{}Handschrift: Bleistift, deutsche Kurrent}
\buchAbdrucke{\weitereDrucke{Hugo von Hofmannsthal, Arthur Schnitzler: \emph{Briefwechsel}. Herausgegeben von Therese Nickl und Heinrich Schnitzler. Frankfurt am Main: \emph{S. Fischer} 1964, S. 112–113.} }\toendnotes[C]{\smallbreak}
\pstart
           {\pb}Dinſtag 4. X. 98.\pend
           \vspace{0.5em}
\pstart
           Mein lieber Hugo, heut vor der Probe hat mir Brahm\pwindex{Brahm, Otto 5.\,2.\,1856 Hamburg – 28.\,11.\,1912 Berlin@\textsc{Brahm, Otto} (5.\,2.\,1856 Hamburg – 28.\,11.\,1912 Berlin), \emph{Theaterleiter, Regisseur}|pw} Ihren Brief gegeben; er hat mir große Freude gemacht. Von
               dem Vermächtnis\pwindex{Schnitzler, Arthur 15.\,5.\,1862 Wien – 21.\,10.\,1931 ebd.@\textsc{Schnitzler, Arthur} (15.\,5.\,1862 Wien – 21.\,10.\,1931 ebd.), \emph{Schriftsteller, Mediziner}!Vermächtnis. Schauspiel in drei Akten@\strich\emph{Das Vermächtnis. Schauspiel in drei Akten}|pw} hab ich nicht viel Spaſs; die
               Sache iſt die: Das Stück iſt nur solang gut, als die »Heldin« nicht auf der Bühne
               iſt. Erster Akt – und der dritte wieder,{ }ſobald{ }ſich das Frauenzi{\geminationm}er ins {\pb}Waſſer{ }ſtürzt. Da{ }ſind alle übrigen Figuren wie von einem Bann befreit, nachdem dieſes Geſpenſt
               angebracht iſt, und reden vernünftige, lebendige, menſchliche, nahezu{ }ſchöne
               Sachen. – Dabei iſt mir heute paſſirt, während d\textcolor{gray}{er} Probe, dſs mir
               das Stück\pwindex{Schnitzler, Arthur 15.\,5.\,1862 Wien – 21.\,10.\,1931 ebd.@\textsc{Schnitzler, Arthur} (15.\,5.\,1862 Wien – 21.\,10.\,1931 ebd.), \emph{Schriftsteller, Mediziner}!Vermächtnis. Schauspiel in drei Akten@\strich\emph{Das Vermächtnis. Schauspiel in drei Akten}|pwv} ganz neu, in 5 {\pb}Akten, dramatiſch eingefallen iſt. Wär ich anſtändg,{ }ſo
               zög ichs zurück, wie es jetzt iſt.\pend
           
\pstart
           Ich freu mich auf Ihre venez.
                  Comödie\pwindex{Hofmannsthal, Hugo von 1.\,2.\,1874 Wien – 15.\,7.\,1929 Rodaun@\textsc{Hofmannsthal, Hugo von} (1.\,2.\,1874 Wien – 15.\,7.\,1929 Rodaun), \emph{Schriftsteller}!Abenteurer und die Sängerin oder Die Geschenke des Lebens@\strich\emph{Der Abenteurer und die Sängerin oder Die Geschenke des Lebens}|pwv};{ }ſo wäre ja der Theaterabend fertig. In Wien\oindex{Wien@\textbf{Wien}, \emph{Verwaltungsgebiet}|pw} find ich Sie{ }ſchon; ich ko{\geminationm}e wohl Mitte
               nächſter Woche.\pend
           
\pstart
           – Mein Ohr{ }ſtört mich wieder mehr als je. Solch{ }ſchleichende, {\pb}i{\geminationm}er gegenwärtige u unaufhaltſame Dinge in uns{ }ſind doch
               die perfideſte Art, wie Alter und Vernichtung{ }ſich ankündigen.\pend
           
\pstart
           Leben Sie wohl. Das mit dem Thurm\oindex{Le due Torri: Garisenda e degli Asinelli@\textbf{Le due Torri: Garisenda e degli Asinelli}, \emph{Monument}|pw} war ja nur
               ein Spaſs. Ich hab ja gar kein Recht, Ihnen einen Thurm\oindex{Bologna@\textbf{Bologna}|pwv} zu{ }ſchenken, der in Bologna\oindex{Bologna@\textbf{Bologna}|pw}{ }ſteht. Und was für Scherereien hätten Sie an der
               Grenze!\pend
           \pstart Von Herzen Ihr \spacefill\mbox{Arthur}\pend{}\selectlanguage{ngerman}\endnumbering\briefempfaengerindex{Hofmannsthal, Hugo von@\textsc{Hofmannsthal, Hugo von}!zzzSchnitzler, Arthur@\emph{von Arthur Schnitzler}!1898-10-041@{4. 10. 1898}|)be}\mylabel{L00850h}  \newcommand{\dateiname}{L00850}\newcommand{\titel}{Arthur Schnitzler an Hugo von Hofmannsthal, 4. 10. 1898}\newcommand{\editorInnen}{Martin Anton Müller und Gerd-Hermann Susen}%% latex-leseansicht-abspann.tex
%% Abspann für die Leseansicht.
%% Der Schalter \ifkorrekturansicht ist bereits durch den Vorspann gesetzt.

%% latex-abspann.tex
%% Gemeinsamer Abspann für Korrekturansicht und Leseansicht.
%% Setzt den Schalter \ifkorrekturansicht voraus (gesetzt in den
%% einbindenden Dateien latex-korrekturansicht-abspann.tex bzw.
%% latex-leseansicht-abspann.tex).
%% ---------------------------------------------------------------

\normalsize

% Das esempio-Environment wird nur in der Leseansicht benötigt
\ifkorrekturansicht\else
\newenvironment{esempio}[3]%
{
    \vspace{1.5ex}
    \rlap{\underline{#1}}
    \par
    \setlength{\parindent}{0cm}
    \nopagebreak
    \leftskip=#2cm
    \rightskip=#3cm
}
{
    \par
}
\fi

\doendnotes{C}
\bigskip
\vfill

\clearpage

\footnotesize

\ifkorrekturansicht
  \lohead{\textsc{register}}
\fi

% theindex-Environment neu definieren ohne reledmac
\makeatletter
\renewenvironment{theindex}{%
  \ifkorrekturansicht
    \section*{\indexname}%
  \else
    \subsubsection*{Index der erwähnten Entitäten}%
  \fi
  \setlength{\parindent}{0pt}%
  \setlength{\parskip}{0pt plus 0.3pt}%
  \let\item\@idxitem
}{%
  \ifkorrekturansicht\clearpage\fi
}
\makeatother

\IfFileExists{\jobname-pw.ind}{\input{\jobname-pw.ind}}{}

% Quellenangabe nur in der Leseansicht
\ifkorrekturansicht\else
% Fallback-Definitionen, falls die .tex-Datei \titel etc. nicht gesetzt hat
\providecommand{\titel}{}
\providecommand{\editorInnen}{}
\providecommand{\dateiname}{\jobname}

\vspace{3cm}

\vfill

\footnotesize
\textsc{Quelle}: \titel. Herausgegeben von {\editorInnen}. In: \emph{Arthur Schnitzler: Briefwechsel mit Autorinnen und Autoren}.
 Digitale Edition, https://schnitzler-briefe.acdh.oeaw.ac.at/{\dateiname}.html (Stand \today)
\fi

\end{document}


