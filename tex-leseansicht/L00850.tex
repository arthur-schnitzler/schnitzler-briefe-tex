%% latex-korrekturansicht-vorspann.tex
%% Vorspann für die Korrekturansicht.
%% Lädt die gemeinsame Datei latex-vorspann.tex mit gesetztem Schalter.

\newif\ifkorrekturansicht
\korrekturansichttrue

\input{../tex-inputs/latex-vorspann}


\section[Arthur Schnitzler an Hugo von Hofmannsthal, 4. 10. 1898]{L00850 Arthur Schnitzler an Hugo von Hofmannsthal, 4. 10. 1898}
\nopagebreak\mylabel{L00850v}
\rehead{ }\normalsize\beginnumbering\briefempfaengerindex{Hofmannsthal, Hugo von@\textsc{Hofmannsthal, Hugo von}!zzzSchnitzler, Arthur@\emph{von Arthur Schnitzler}!1898-10-041@{4. 10. 1898}|(be}
\toendnotes[C]{\smallbreak\pagebreak[2]}\Standort{FDH, Hs-30885,77.}
\physDesc{Brief, 1 Blatt, 4 Seiten, 1187 Zeichen
\newline{}Handschrift: Bleistift, deutsche Kurrent}
\buchAbdrucke{\weitereDrucke{Hugo von Hofmannsthal, Arthur Schnitzler: \emph{Briefwechsel}. Frankfurt am Main: \emph{S. Fischer} 1964, S. 112–113.} }\toendnotes[C]{\smallbreak}
\pstart
           {\pb}Dinſtag 4. X. 98.\pend
           \vspace{0.5em}
\pstart
           Mein lieber Hugo, heut vor der Probe hat mir Brahm\pwindex{Brahm, Otto 05.02.1856 – 28.11.1912@\textsc{Brahm, Otto} (05.02.1856 – 28.11.1912), \emph{Theaterleiter/Theaterleiterin, Regisseur/Regisseurin}|pw} Ihren Brief gegeben; er hat mir große Freude gemacht. Von
               dem Vermächtnis\pwindex{Vermaechtnis. Schauspiel in drei Akten@\emph{Das Vermächtnis. Schauspiel in drei Akten}|pw} hab ich nicht viel Spaſs; die
               Sache iſt die: Das Stück iſt nur solang gut, als die »Heldin« nicht auf der Bühne
               iſt. Erster Akt – und der dritte wieder, ſobald ſich das Frauenzi{\geminationm}er ins {\pb}Waſſer ſtürzt. Da
               ſind alle übrigen Figuren wie von einem Bann befreit, nachdem dieſes Geſpenſt
               angebracht iſt, und reden vernünftige, lebendige, menſchliche, nahezu ſchöne
               Sachen. – Dabei iſt mir heute paſſirt, während d\textcolor{gray}{er} Probe, dſs mir
               das Stück\pwindex{Vermaechtnis. Schauspiel in drei Akten@\emph{Das Vermächtnis. Schauspiel in drei Akten}|pwv} ganz neu, in 5 {\pb}Akten, dramatiſch eingefallen iſt. Wär ich anſtändg, ſo
               zög ichs zurück, wie es jetzt iſt.\pend
           
\pstart
           Ich freu mich auf Ihre venez.
                  Comödie\pwindex{Abenteurer und die Saengerin oder Die Geschenke des Lebens@\emph{Der Abenteurer und die Sängerin oder Die Geschenke des Lebens}|pwv}; ſo wäre ja der Theaterabend fertig. In Wien\oindex{Wien@\textbf{Wien}, \emph{A.ADM2}|pw} find ich Sie ſchon; ich ko{\geminationm}e wohl Mitte
               nächſter Woche.\pend
           
\pstart
           – Mein Ohr ſtört mich wieder mehr als je. Solch ſchleichende, {\pb}i{\geminationm}er gegenwärtige u unaufhaltſame Dinge in uns ſind doch
               die perfideſte Art, wie Alter und Vernichtung ſich ankündigen.\pend
           
\pstart
           Leben Sie wohl. Das mit dem Thurm\oindex{Le due Torri: Garisenda e degli Asinelli@\textbf{Le due Torri: Garisenda e degli Asinelli}, \emph{Monument (K.MON)}|pw} war ja nur
               ein Spaſs. Ich hab ja gar kein Recht, Ihnen einen Thurm\oindex{Bologna@\textbf{Bologna}, \emph{P.PPLA}|pwv} zu ſchenken, der in Bologna\oindex{Bologna@\textbf{Bologna}, \emph{P.PPLA}|pw}{ }ſteht. Und was für Scherereien hätten Sie an der
               Grenze!\pend
           \pstart Von Herzen Ihr \spacefill\mbox{Arthur}\pend{}\selectlanguage{ngerman}\endnumbering\briefempfaengerindex{Hofmannsthal, Hugo von@\textsc{Hofmannsthal, Hugo von}!zzzSchnitzler, Arthur@\emph{von Arthur Schnitzler}!1898-10-041@{4. 10. 1898}|)be}\mylabel{L00850h}  \normalsize

\doendnotes{C}
\bigskip
\vfill

\clearpage

\footnotesize

\lohead{\textsc{register}}

% Definiere theindex-Environment komplett neu ohne reledmac
\makeatletter
\renewenvironment{theindex}{%
  \section*{\indexname}%
  \setlength{\parindent}{0pt}%
  \setlength{\parskip}{0pt plus 0.3pt}%
  \let\item\@idxitem
}{%
  \clearpage
}
\makeatother

\IfFileExists{\jobname-pw.ind}{\input{\jobname-pw.ind}}{}

\end{document}

      