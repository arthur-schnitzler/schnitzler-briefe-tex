%% latex-korrekturansicht-vorspann.tex
%% Vorspann für die Korrekturansicht.
%% Lädt die gemeinsame Datei latex-vorspann.tex mit gesetztem Schalter.

\newif\ifkorrekturansicht
\korrekturansichttrue

\input{../tex-inputs/latex-vorspann}


\section[Arthur Schnitzler an Hermann Bahr, {[}1{]}3. 9. 1901]{L01173 Arthur Schnitzler an Hermann Bahr, {[}1{]}3. 9. 1901}
\nopagebreak\mylabel{L01173v}
\rehead{ }\normalsize\beginnumbering\briefempfaengerindex{Bahr, Hermann@\textsc{Bahr, Hermann}!zzzSchnitzler, Arthur@\emph{von Arthur Schnitzler}!1901-09-131@{{[}1{]}3. 9. 1901}|(be}
\toendnotes[C]{\smallbreak\pagebreak[2]}\Standort{TMW, HS AM 23340 Ba.}
\physDesc{Brief, 1 Blatt, 2 Seiten, 806 Zeichen
\newline{}Handschrift: schwarze Tinte, deutsche Kurrent}
\buchAbdrucke{\weitereDrucke{1) Arthur Schnitzler: \emph{The Letters of Arthur Schnitzler to Hermann Bahr}. Chapel Hill: \emph{The University of North Carolina Press} 1978, S. 71.} \weitereDrucke{2) Hermann Bahr, Arthur Schnitzler: \emph{Briefwechsel, Aufzeichnungen, Dokumente (1891–1931)}. Göttingen: \emph{Wallstein} 2018, S. 215.} }\toendnotes[C]{\smallbreak}
\pstart
           \noindent{}{\pb}lieber Hermann, es iſt ſehr freundlich von dir, daſs du die kleinen
                  Sachen\pwindex{Frau mit dem Dolche@\emph{Die Frau mit dem Dolche}|pwv}\pwindex{Literatur@\emph{Literatur}|pwv}\pwindex{Lebendige Stunden@\emph{Lebendige Stunden}|pwv}{ }ſo ſchnell geleſen haſt. Die
               Verwandlungsſchwierigkeit in der Frau mit dem
                  Dolch\pwindex{Frau mit dem Dolche@\emph{Die Frau mit dem Dolche}|pw} wird hoffentlich zu beheben ſein, – der Idiotismus des Publikums wohl
               ſchwerer. Mehr Sorgen aber macht mir die Beſetzung. Ich bin nun mit einem 4. Einakter\pwindex{Puppenspieler. Studie in einem Aufzuge@\emph{Der Puppenspieler. Studie in einem Aufzuge}|pwv} beſchäftigt, für
               den ich mir gern den Mitterwurzer\pwindex{Mitterwurzer, Friedrich 16.10.1844 – 13.02.1897@\textsc{Mitterwurzer, Friedrich} (16.10.1844 – 13.02.1897), \emph{Schauspieler/Schauspielerin}|pw} aus der Erde
               kratzen möchte, u daſs ich auch noch \strikeout{den} einen
                  \label{K_L01173-1v}\edtext{fünften\pwindex{letzten Masken@\emph{Die letzten Masken}|pwv}}{\lemma{\textnormal{\emph{fünften}}}\Cendnote{\textnormal{\emph{Die letzten Masken}\pwindex{letzten Masken@\emph{Die letzten Masken}|pwk}}}}\label{K_L01173-1}{ }ſchreibe, iſt ziemlich {\pb}ſicher. In dieſen
               beiden Stücken wird nun allerdings der »Literaten«typus beträchtlich erweitert,
               dadurch aber für die »Uneingeweihten« klarer ſein. Schön wärs halt, wenn einem ein
               ſehr ſcharfes Wort als \label{K_L01173-2v}\edtext{Gesamttitel}{\lemma{\textnormal{\emph{Gesamttitel}}}\Cendnote{\textnormal{Nur \emph{Die letzten Masken}\pwindex{letzten Masken@\emph{Die letzten Masken}|pwk} wurde letztlich zu den bestehenden drei
                  Einaktern hinzugefügt, und diese wurden unter dem Titel \emph{Lebendige Stunden. Vier Einakter}\pwindex{Lebendige Stunden. Vier Einakter@\emph{Lebendige Stunden. Vier Einakter}|pwk} (Berlin: \emph{S. Fischer}\orgindex{S. Fischer Verlag@S. Fischer Verlag|pwk}{ }1902) zusammengefasst.}}}\label{K_L01173-2} einfiele, das für die anderen ſo deutlich wäre,
               wie für unſereinen das Wort »Literat«; aber doch noch mehr ſagt.\pend
           
\pstart
           Herzlichen Gruß. Dein{\\[\baselineskip]}\spacefill\mbox{Arthur}\pend
           \leftskip=0em{}
\pstart
           \textcolor{gray}{1}3. 9. 901.\pend
           \selectlanguage{ngerman}\endnumbering\briefempfaengerindex{Bahr, Hermann@\textsc{Bahr, Hermann}!zzzSchnitzler, Arthur@\emph{von Arthur Schnitzler}!1901-09-131@{{[}1{]}3. 9. 1901}|)be}\mylabel{L01173h}  \normalsize

\doendnotes{C}
\bigskip
\vfill

\clearpage

\footnotesize

\lohead{\textsc{register}}

% Definiere theindex-Environment komplett neu ohne reledmac
\makeatletter
\renewenvironment{theindex}{%
  \section*{\indexname}%
  \setlength{\parindent}{0pt}%
  \setlength{\parskip}{0pt plus 0.3pt}%
  \let\item\@idxitem
}{%
  \clearpage
}
\makeatother

\IfFileExists{\jobname-pw.ind}{\input{\jobname-pw.ind}}{}

\end{document}

      