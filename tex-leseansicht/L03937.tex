%% latex-leseansicht-vorspann.tex
%% Vorspann für die Leseansicht.
%% Lädt die gemeinsame Datei latex-vorspann.tex mit nicht gesetztem Schalter.

\newif\ifkorrekturansicht
\korrekturansichtfalse

\input{../tex-inputs/latex-vorspann}


\section[Arthur Schnitzler an Theodor Herzl, 15. 11. 1900]{L03937 Arthur Schnitzler an Theodor Herzl, 15. 11. 1900}
\nopagebreak\mylabel{L03937v}
\rehead{ }\normalsize\beginnumbering\briefempfaengerindex{Herzl, Theodor@\textsc{Herzl, Theodor}!zzzSchnitzler, Arthur@\emph{von Arthur Schnitzler}!1900-11-151@{15. 11. 1900}|(be}
\toendnotes[C]{\smallbreak\pagebreak[2]}
\correspDesc{Versand  durch Arthur Schnitzler am 15. 11. 1900 in Wien
\newline{}Erhalt  durch Theodor Herzl in Wien}\toendnotes[C]{\smallbreak}
\Standort{Jerusalem, Central Zionist Archives, H1:1926-3.}
\physDesc{,  Blätter,  Seiten
\newline{}Handschrift: , deutsche Kurrent}
\buchAbdrucke{\weitereDrucke{Arthur Schnitzler: \emph{Briefe 1875–1912}. Herausgegeben von Therese Nickl und Heinrich Schnitzler. Frankfurt am Main: \emph{S. Fischer} 1981, S. 397.} }\toendnotes[C]{\smallbreak}
\pstart{}{\pb}lieber Doctor Herzl,\pend\vspace{0.5em}
\pstart
           ich habe eine Erzählung\pwindex{Schnitzler, Arthur 15.\,5.\,1862 Wien – 21.\,10.\,1931 ebd.@\textsc{Schnitzler, Arthur} (15.\,5.\,1862 Wien – 21.\,10.\,1931 ebd.), \emph{Schriftsteller, Mediziner}!Lieutenant Gustl. Novelle@\strich\emph{Lieutenant Gustl. Novelle}|pwv}
               geſchrieben, die ich Ihrem freundlichen Wunſch entſprechend, gern für die Weihnachtsnu{\geminationm}er\pwindex{Neue Freie Presse@\emph{Neue Freie Presse}|pwv} hergäbe. Nur
               iſt ſie etwas lang gerathen, etwa 9 Längſpalten (1 ½ Bogen).
               Andrerſeits läßt{ }ſie ſich aber gar nicht theilen, und ſo wäre vielleicht gerade die
               Weihnachtsbeilage der N. Fr. Pr.\pwindex{Neue Freie Presse@\emph{Neue Freie Presse}|pw}{\pb}die richtige Stelle für sie. Bitte ſagen Sie mir, ob ich
               Ihnen die Geſchichte\pwindex{Schnitzler, Arthur 15.\,5.\,1862 Wien – 21.\,10.\,1931 ebd.@\textsc{Schnitzler, Arthur} (15.\,5.\,1862 Wien – 21.\,10.\,1931 ebd.), \emph{Schriftsteller, Mediziner}!Lieutenant Gustl. Novelle@\strich\emph{Lieutenant Gustl. Novelle}|pwv}{ }ſenden
               darf.\pend
           
\pstart
           Noch eins bei dieſer Gelegenheit. Durch \textsc{Beer Hofmann\pwindex{Beer-Hofmann, Richard 11.\,7.\,1866 Wien – 26.\,9.\,1945 New York City@\textsc{Beer-Hofmann, Richard} (11.\,7.\,1866 Wien – 26.\,9.\,1945 New York City), \emph{Schriftsteller}|pw}} habe ich erfahren, daſs Sie ſich \label{K_L03937-2v}\edtext{heuer in Aussee\oindex{Bad Aussee@\textbf{Bad Aussee}, \emph{Hauptstadt}|pw}}{\lemma{\textnormal{\emph{heuer in Aussee}}}\Cendnote{\textnormal{XXXX}}}\label{K_L03937-2} durch eine gewiſs nicht ſehr geiſtreiche aber
               eben ſo gewiſs nicht bös gemeinte Bemerkung von mir verſti{\geminationm}t oder gar verletzt
               gefühlt haben. Das thut mir ſehr leid. Alle äußeren Entfremdungen {\pb}und Misvertändniſſe, die im Lauf der Jahre zwiſchen uns
               vorgeko{\geminationm}en ſind und nach der Natur der Dinge und unseren Naturen wahrſcheinlich
               vorko{\geminationm}en mußten, haben, meiner aufrichtigen und in vieler Beziehung ſehr herzlichen,
               Verehrung für Sie nichts angehabt. Ich kann nicht denken, daſs Sie einen Scherz übel
               nehmen wollen, dem auch die leiſeſte Spur einer kränkenden Abſicht fehlte. Da Sie {\pb}das wunderlicher Weiſe nicht ſelbſt fühlten, muſs ich es
               heute ſagen; de{\geminationn} er wäre beinah leichtfertig, eine Unklarheit, die ſo leicht aus dem
               Wege zu räumen iſt, zwiſchen uns zu belaſſen.\pend
           
\pstart
           Ich drücke Ihnen die Hand und bin Ihr{\\[\baselineskip]}herzlich ergebner{\\[\baselineskip]}\spacefill\mbox{Arthur Schnitzler}\pend
           \leftskip=0em{}
\pstart
           Wien\oindex{Wien@\textbf{Wien}, \emph{Verwaltungsgebiet}|pw}{ }15. 11. 900.\pend
           \selectlanguage{ngerman}\endnumbering\briefempfaengerindex{Herzl, Theodor@\textsc{Herzl, Theodor}!zzzSchnitzler, Arthur@\emph{von Arthur Schnitzler}!1900-11-151@{15. 11. 1900}|)be}\mylabel{L03937h}
\begin{anhang}
\end{anhang}\newcommand{\dateiname}{L03937}\newcommand{\titel}{Arthur Schnitzler an Theodor Herzl, 15. 11. 1900}\newcommand{\editorInnen}{Herausgegeben von Jahnke, SelmaMüller, Martin Anton}%% latex-leseansicht-abspann.tex
%% Abspann für die Leseansicht.
%% Der Schalter \ifkorrekturansicht ist bereits durch den Vorspann gesetzt.

%% latex-abspann.tex
%% Gemeinsamer Abspann für Korrekturansicht und Leseansicht.
%% Setzt den Schalter \ifkorrekturansicht voraus (gesetzt in den
%% einbindenden Dateien latex-korrekturansicht-abspann.tex bzw.
%% latex-leseansicht-abspann.tex).
%% ---------------------------------------------------------------

\normalsize

% Das esempio-Environment wird nur in der Leseansicht benötigt
\ifkorrekturansicht\else
\newenvironment{esempio}[3]%
{
    \vspace{1.5ex}
    \rlap{\underline{#1}}
    \par
    \setlength{\parindent}{0cm}
    \nopagebreak
    \leftskip=#2cm
    \rightskip=#3cm
}
{
    \par
}
\fi

\doendnotes{C}
\bigskip
\vfill

\clearpage

\footnotesize

\ifkorrekturansicht
  \lohead{\textsc{register}}
\fi

% theindex-Environment neu definieren ohne reledmac
\makeatletter
\renewenvironment{theindex}{%
  \ifkorrekturansicht
    \section*{\indexname}%
  \else
    \subsubsection*{Index der erwähnten Entitäten}%
  \fi
  \setlength{\parindent}{0pt}%
  \setlength{\parskip}{0pt plus 0.3pt}%
  \let\item\@idxitem
}{%
  \ifkorrekturansicht\clearpage\fi
}
\makeatother

\IfFileExists{\jobname-pw.ind}{\input{\jobname-pw.ind}}{}

% Quellenangabe nur in der Leseansicht
\ifkorrekturansicht\else
% Fallback-Definitionen, falls die .tex-Datei \titel etc. nicht gesetzt hat
\providecommand{\titel}{}
\providecommand{\editorInnen}{}
\providecommand{\dateiname}{\jobname}

\vspace{3cm}

\vfill

\footnotesize
\textsc{Quelle}: \titel. Herausgegeben von {\editorInnen}. In: \emph{Arthur Schnitzler: Briefwechsel mit Autorinnen und Autoren}.
 Digitale Edition, https://schnitzler-briefe.acdh.oeaw.ac.at/{\dateiname}.html (Stand \today)
\fi

\end{document}


