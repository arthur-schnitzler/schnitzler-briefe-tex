%% latex-korrekturansicht-vorspann.tex
%% Vorspann für die Korrekturansicht.
%% Lädt die gemeinsame Datei latex-vorspann.tex mit gesetztem Schalter.

\newif\ifkorrekturansicht
\korrekturansichttrue

\input{../tex-inputs/latex-vorspann}


\section[Richard Dehmel an Arthur Schnitzler, {[}18. 11. 1913?{]}]{L02157 Richard Dehmel an Arthur Schnitzler, {[}18. 11. 1913?{]}}
\nopagebreak\mylabel{L02157v}
\rehead{ }\normalsize\beginnumbering\briefempfaengerindex{Schnitzler, Arthur@\textsc{Schnitzler, Arthur}!zzzDehmel, Richard@\emph{von Richard Dehmel}!1913-11-181@{{[}18. 11. 1913?{]}}|(be}
\toendnotes[C]{\smallbreak\pagebreak[2]}\Standort{CUL, Schnitzler, B 26.}
\physDesc{Brief, 1 Blatt, 3 Seiten, 2514 Zeichen
\newline{}Druck
\newline{}Schnitzler: 1) mit Bleistift beschrieben: »\textsc{Dehmel}«  2) mit rotem Buntstift Vermerk »\textsc{\uline{(nicht abschr!)}}«
\newline{}Ordnung: mit Bleistift von unbekannter Hand datiert: »1913« 
\newline{}Zusatz: Im Nachlass von Martin
                                    Sturm\pwindex{Sturm, Martin 31.08.1894 – 1957?@\textsc{Sturm, Martin} (31.08.1894 – 1957?)|pw} (Heinrich-Heine-Institut, Düsseldorf,
                                    HHI.94.5036.281) findet sich der gleiche Druck
                                 einschließlich des Briefumschlags, der genau am Tag des
                                 50. Geburtstages, am 18. 11. 1913 in Blankenese\oindex{Blankenese@\textbf{Blankenese}, \emph{P.PPLX}|pw} gestempelt
                                 ist. }\toendnotes[C]{\smallbreak}
\pstart
           \noindent{}\centering{}{\pb}\textbf{Das Haus des Dichters}\pwindex{Haus des Dichters@\emph{Das Haus des Dichters}|pw}\pend
           
\pstart
           \centering{}*\pend
           
\pstart
           \centering{}Allen Freunden zur Erinnerung{\\}an meinen 50. Geburtstag\pend
           
\pstart
           \centering{}⋅ Richard Dehmel ⋅\pend
           
\pstart
           \centering{}*\pend
           \stanza{}O bleib, Phönix, verlaß mich nicht,\pwindex{Haus des Dichters@\emph{Das Haus des Dichters}|pw}Traumfeuervogel, mein göttlicher,\pwindex{Haus des Dichters@\emph{Das Haus des Dichters}|pw}wie ſchweiften wir frei von Herd zu Herd!\pwindex{Haus des Dichters@\emph{Das Haus des Dichters}|pw}Wenn ich ſcheu, ich ſtaubgeborener Wicht,\pwindex{Haus des Dichters@\emph{Das Haus des Dichters}|pw}in die Aſche blies mit finſteren Geſicht,\pwindex{Haus des Dichters@\emph{Das Haus des Dichters}|pw}flogſt du goldrot auf, immer neu hellauf,\pwindex{Haus des Dichters@\emph{Das Haus des Dichters}|pw}unbeſchwert,\pwindex{Haus des Dichters@\emph{Das Haus des Dichters}|pw}und Sternbilder ſprühten von deinen
                     Schwingen.\pwindex{Haus des Dichters@\emph{Das Haus des Dichters}|pw}Bis ein Abend kam, wo ich müd dir grollte,\pwindex{Haus des Dichters@\emph{Das Haus des Dichters}|pw}unter fremden Fichten, in
                     Menſchenſehnſuchtsqual,\pwindex{Haus des Dichters@\emph{Das Haus des Dichters}|pw}nicht mehr von dir träumen wollte,\pwindex{Haus des Dichters@\emph{Das Haus des Dichters}|pw}von deinem ewigen Zauberſtrahl\pwindex{Haus des Dichters@\emph{Das Haus des Dichters}|pw}und nie erlebten Wunderdingen,\pwindex{Haus des Dichters@\emph{Das Haus des Dichters}|pw}nur von Heimat, Heimat endlich einmal –\pwindex{Haus des Dichters@\emph{Das Haus des Dichters}|pw}da huben die Sterne an zu klingen:\pwindex{Haus des Dichters@\emph{Das Haus des Dichters}|pw}Ja, die ganze Welt kannſt du wild
                     durchſchweifen\pwindex{Haus des Dichters@\emph{Das Haus des Dichters}|pw}in deinem freiheitstrunknen Flug,\pwindex{Haus des Dichters@\emph{Das Haus des Dichters}|pw}kannſt Kometen begleiten durch
                     Urnebelſtreifen,\pwindex{Haus des Dichters@\emph{Das Haus des Dichters}|pw}Stürme, Wolken, Blitz dir zum Spielzeug
                     greifen,\pwindex{Haus des Dichters@\emph{Das Haus des Dichters}|pw}{\pb}ach, und haſt nicht Kraft
                     genug,\pwindex{Haus des Dichters@\emph{Das Haus des Dichters}|pw}ein Haus auf der feſten Erde zu bauen,\pwindex{Haus des Dichters@\emph{Das Haus des Dichters}|pw}für dich und die Deinen ein ſichres Bett,\pwindex{Haus des Dichters@\emph{Das Haus des Dichters}|pw}kannſt dir nicht einen Balken ſelber hauen,\pwindex{Haus des Dichters@\emph{Das Haus des Dichters}|pw}nicht ein Tiſchlein zu zimmern dich getrauen,\pwindex{Haus des Dichters@\emph{Das Haus des Dichters}|pw}nicht ein Brett,\pwindex{Haus des Dichters@\emph{Das Haus des Dichters}|pw}hockſt wie ein unbeholfnes Tier\pwindex{Haus des Dichters@\emph{Das Haus des Dichters}|pw}unter den fremden Fichten hier –\pwindex{Haus des Dichters@\emph{Das Haus des Dichters}|pw}ſo erklangen die Sterne – da flucht’ ich dir.\pwindex{Haus des Dichters@\emph{Das Haus des Dichters}|pw}Bis der Morgen graute, bis Menſchen kamen,\pwindex{Haus des Dichters@\emph{Das Haus des Dichters}|pw}hilfreich kamen, Mann für Mann,\pwindex{Haus des Dichters@\emph{Das Haus des Dichters}|pw}mich herzlich bei den Händen nahmen,\pwindex{Haus des Dichters@\emph{Das Haus des Dichters}|pw}und holde Frauen lachten mich an:\pwindex{Haus des Dichters@\emph{Das Haus des Dichters}|pw}Sieh doch, da ſteht das Haus ſchon errichtet;\pwindex{Haus des Dichters@\emph{Das Haus des Dichters}|pw}während du ſchweifteſt von Traum zu Traum,\pwindex{Haus des Dichters@\emph{Das Haus des Dichters}|pw}ward Stein auf Stein zur Mauer geſchichtet,\pwindex{Haus des Dichters@\emph{Das Haus des Dichters}|pw}der dunkle Hain zum Garten gelichtet,\pwindex{Haus des Dichters@\emph{Das Haus des Dichters}|pw}dir zum heimatlichen Raum.\pwindex{Haus des Dichters@\emph{Das Haus des Dichters}|pw}Nach freudiger Menſchheit ging dein Trachten;\pwindex{Haus des Dichters@\emph{Das Haus des Dichters}|pw}weil du ſie träumteſt, lebt ſie nun;\pwindex{Haus des Dichters@\emph{Das Haus des Dichters}|pw}du halfeſt ihr ſich göttlich achten,\pwindex{Haus des Dichters@\emph{Das Haus des Dichters}|pw}empfang als Schöpferlohn ihr Tun;\pwindex{Haus des Dichters@\emph{Das Haus des Dichters}|pw}laß dir aus unſern ſchwachen Händen\pwindex{Haus des Dichters@\emph{Das Haus des Dichters}|pw}den Segen vieler ſtarken ſpenden!\pwindex{Haus des Dichters@\emph{Das Haus des Dichters}|pw}{\pb}So ſprachen ſtrahlend zwei der
                     Frauen,\pwindex{Haus des Dichters@\emph{Das Haus des Dichters}|pw}mich aber wehte ein Bangen an:\pwindex{Haus des Dichters@\emph{Das Haus des Dichters}|pw}verflogen war das Morgengrauen,\pwindex{Haus des Dichters@\emph{Das Haus des Dichters}|pw}und über dem ſonneblanken Tann\pwindex{Haus des Dichters@\emph{Das Haus des Dichters}|pw}fern im Blauen\pwindex{Haus des Dichters@\emph{Das Haus des Dichters}|pw}ſah ich ſtarr dich mit zitternden Klauen\pwindex{Haus des Dichters@\emph{Das Haus des Dichters}|pw}ſchreckbeſchwert\pwindex{Haus des Dichters@\emph{Das Haus des Dichters}|pw}– Phönix – ſprühend niederſchauen\pwindex{Haus des Dichters@\emph{Das Haus des Dichters}|pw}auf meinen Herd.\pwindex{Haus des Dichters@\emph{Das Haus des Dichters}|pw}Wie Sankt Johannes zwiſchen den ſieben
                     Leuchtern\pwindex{Haus des Dichters@\emph{Das Haus des Dichters}|pw}mit gen Boden gebeugtem Geſicht\pwindex{Haus des Dichters@\emph{Das Haus des Dichters}|pw}barg ich unter den hohen Bäumen\pwindex{Haus des Dichters@\emph{Das Haus des Dichters}|pw}meinen Blick vor all dem Gnadenlicht;\pwindex{Haus des Dichters@\emph{Das Haus des Dichters}|pw}in meinen Tränen ſtoſſen zu taumelnden
                     Flammen\pwindex{Haus des Dichters@\emph{Das Haus des Dichters}|pw}die Menſchen rings mit euch zuſammen,\pwindex{Haus des Dichters@\emph{Das Haus des Dichters}|pw}ihr alten Fichten um dies neue Dach –\pwindex{Haus des Dichters@\emph{Das Haus des Dichters}|pw}was rauſcht ihr mir Erinnrung, ach!\pwindex{Haus des Dichters@\emph{Das Haus des Dichters}|pw}Ich fühl’s noch heute beim Schwanken eurer
                     Zweige,\pwindex{Haus des Dichters@\emph{Das Haus des Dichters}|pw}wie ich erſchüttert den Nacken neige,\pwindex{Haus des Dichters@\emph{Das Haus des Dichters}|pw}weil mir zum Dank die Kraft gebricht.\pwindex{Haus des Dichters@\emph{Das Haus des Dichters}|pw}Ich kann ja nichts als immer wieder träumen\pwindex{Haus des Dichters@\emph{Das Haus des Dichters}|pw}von ſeligem Aufflug zu den freien Räumen –\pwindex{Haus des Dichters@\emph{Das Haus des Dichters}|pw}O Phönix, Phönix, verlaß mich nicht! –\pwindex{Haus des Dichters@\emph{Das Haus des Dichters}|pw}\stanzaend{}
\pstart
           \centering{}* * *\pend
           
\pstart
           \centering{}{\pb}\label{T_L02157-1v}\edtext{WD}{\lemma{\textnormal{\emph{WD}}}\Cendnote{\textnormal{in Form eines Adlers, die nächste Zeile als
                  Wappenspruch}}}\label{T_L02157-1}\pend
           
\pstart
           \centering{}Force m’est trop\pend
           \selectlanguage{ngerman}\endnumbering\briefempfaengerindex{Schnitzler, Arthur@\textsc{Schnitzler, Arthur}!zzzDehmel, Richard@\emph{von Richard Dehmel}!1913-11-181@{{[}18. 11. 1913?{]}}|)be}\mylabel{L02157h}  \normalsize

\doendnotes{C}
\bigskip
\vfill

\clearpage

\footnotesize

\lohead{\textsc{register}}

% Definiere theindex-Environment komplett neu ohne reledmac
\makeatletter
\renewenvironment{theindex}{%
  \section*{\indexname}%
  \setlength{\parindent}{0pt}%
  \setlength{\parskip}{0pt plus 0.3pt}%
  \let\item\@idxitem
}{%
  \clearpage
}
\makeatother

\IfFileExists{\jobname-pw.ind}{\input{\jobname-pw.ind}}{}

\end{document}

      