%% latex-leseansicht-vorspann.tex
%% Vorspann für die Leseansicht.
%% Lädt die gemeinsame Datei latex-vorspann.tex mit nicht gesetztem Schalter.

\newif\ifkorrekturansicht
\korrekturansichtfalse

\input{../tex-inputs/latex-vorspann}


\section[Richard Dehmel an Arthur Schnitzler, {{[}}18. 11. 1913?{{]}}]{L02157 Richard Dehmel an Arthur Schnitzler, {[}18. 11. 1913?{]}}
\nopagebreak\mylabel{L02157v}
\rehead{ }\normalsize\beginnumbering\briefempfaengerindex{Schnitzler, Arthur@\textsc{Schnitzler, Arthur}!zzzDehmel, Richard@\emph{von Richard Dehmel}!1913-11-181@{{[}18. 11. 1913?{]}}|(be}
\toendnotes[C]{\smallbreak\pagebreak[2]}
\correspDesc{Versand  durch Richard Dehmel am [18. 11. 1913?] in Hamburg
\newline{}Erhalt  durch Arthur Schnitzler im Zeitraum [19. 11. 1913 – 23. 11. 1913?] in Wien}\toendnotes[C]{\smallbreak}
\Standort{CUL, Schnitzler, B 26.}
\physDesc{Brief, 1 Blatt, 3 Seiten, 2514 Zeichen
\newline{}Druck
\newline{}Schnitzler: 1) mit Bleistift beschrieben: »\textsc{Dehmel}«  2) mit rotem Buntstift Vermerk »\textsc{\uline{(nicht abschr!)}}«
\newline{}Ordnung: mit Bleistift von unbekannter Hand datiert: »1913« 
\newline{}Zusatz: Im Nachlass von Martin
                                    Sturm\pwindex{Sturm, Martin 31.\,8.\,1894 – 1957?@\textsc{Sturm, Martin} (31.\,8.\,1894 – 1957?)|pw} (Heinrich-Heine-Institut, Düsseldorf,
                                    HHI.94.5036.281) findet sich der gleiche Druck
                                 einschließlich des Briefumschlags, der genau am Tag des
                                 50. Geburtstages, am 18. 11. 1913 in Blankenese\oindex{Blankenese@\textbf{Blankenese}, \emph{Ehemaliger Ort}|pw} gestempelt
                                 ist. }\toendnotes[C]{\smallbreak}
\pstart
           \noindent{}\centering{}{\pb}\textbf{Das Haus des Dichters}\pwindex{Dehmel, Richard 18.\,11.\,1863 Hermsdorf – 8.\,2.\,1920 Blankenese@\textsc{Dehmel, Richard} (18.\,11.\,1863 Hermsdorf – 8.\,2.\,1920 Blankenese), \emph{Schriftsteller, Schriftsteller, Krimiautor}!Haus des Dichters@\strich\emph{Das Haus des Dichters}|pw}\pend
           
\pstart
           \centering{}*\pend
           
\pstart
           \centering{}Allen Freunden zur Erinnerung{\\}an meinen 50. Geburtstag\pend
           
\pstart
           \centering{}⋅ Richard Dehmel ⋅\pend
           
\pstart
           \centering{}*\pend
           \stanza{}O bleib, Phönix, verlaß mich nicht,\pwindex{Dehmel, Richard 18.\,11.\,1863 Hermsdorf – 8.\,2.\,1920 Blankenese@\textsc{Dehmel, Richard} (18.\,11.\,1863 Hermsdorf – 8.\,2.\,1920 Blankenese), \emph{Schriftsteller, Schriftsteller, Krimiautor}!Haus des Dichters@\strich\emph{Das Haus des Dichters}|pw}\newverse{}Traumfeuervogel, mein göttlicher,\pwindex{Dehmel, Richard 18.\,11.\,1863 Hermsdorf – 8.\,2.\,1920 Blankenese@\textsc{Dehmel, Richard} (18.\,11.\,1863 Hermsdorf – 8.\,2.\,1920 Blankenese), \emph{Schriftsteller, Schriftsteller, Krimiautor}!Haus des Dichters@\strich\emph{Das Haus des Dichters}|pw}\newverse{}wie{ }ſchweiften wir frei von Herd zu Herd!\pwindex{Dehmel, Richard 18.\,11.\,1863 Hermsdorf – 8.\,2.\,1920 Blankenese@\textsc{Dehmel, Richard} (18.\,11.\,1863 Hermsdorf – 8.\,2.\,1920 Blankenese), \emph{Schriftsteller, Schriftsteller, Krimiautor}!Haus des Dichters@\strich\emph{Das Haus des Dichters}|pw}\newverse{}Wenn ich{ }ſcheu, ich{ }ſtaubgeborener Wicht,\pwindex{Dehmel, Richard 18.\,11.\,1863 Hermsdorf – 8.\,2.\,1920 Blankenese@\textsc{Dehmel, Richard} (18.\,11.\,1863 Hermsdorf – 8.\,2.\,1920 Blankenese), \emph{Schriftsteller, Schriftsteller, Krimiautor}!Haus des Dichters@\strich\emph{Das Haus des Dichters}|pw}\newverse{}in die Aſche blies mit finſteren Geſicht,\pwindex{Dehmel, Richard 18.\,11.\,1863 Hermsdorf – 8.\,2.\,1920 Blankenese@\textsc{Dehmel, Richard} (18.\,11.\,1863 Hermsdorf – 8.\,2.\,1920 Blankenese), \emph{Schriftsteller, Schriftsteller, Krimiautor}!Haus des Dichters@\strich\emph{Das Haus des Dichters}|pw}\newverse{}flogſt du goldrot auf, immer neu hellauf,\pwindex{Dehmel, Richard 18.\,11.\,1863 Hermsdorf – 8.\,2.\,1920 Blankenese@\textsc{Dehmel, Richard} (18.\,11.\,1863 Hermsdorf – 8.\,2.\,1920 Blankenese), \emph{Schriftsteller, Schriftsteller, Krimiautor}!Haus des Dichters@\strich\emph{Das Haus des Dichters}|pw}\newverse{}unbeſchwert,\pwindex{Dehmel, Richard 18.\,11.\,1863 Hermsdorf – 8.\,2.\,1920 Blankenese@\textsc{Dehmel, Richard} (18.\,11.\,1863 Hermsdorf – 8.\,2.\,1920 Blankenese), \emph{Schriftsteller, Schriftsteller, Krimiautor}!Haus des Dichters@\strich\emph{Das Haus des Dichters}|pw}\newverse{}und Sternbilder{ }ſprühten von deinen
                     Schwingen.\pwindex{Dehmel, Richard 18.\,11.\,1863 Hermsdorf – 8.\,2.\,1920 Blankenese@\textsc{Dehmel, Richard} (18.\,11.\,1863 Hermsdorf – 8.\,2.\,1920 Blankenese), \emph{Schriftsteller, Schriftsteller, Krimiautor}!Haus des Dichters@\strich\emph{Das Haus des Dichters}|pw}\newverse{}Bis ein Abend kam, wo ich müd dir grollte,\pwindex{Dehmel, Richard 18.\,11.\,1863 Hermsdorf – 8.\,2.\,1920 Blankenese@\textsc{Dehmel, Richard} (18.\,11.\,1863 Hermsdorf – 8.\,2.\,1920 Blankenese), \emph{Schriftsteller, Schriftsteller, Krimiautor}!Haus des Dichters@\strich\emph{Das Haus des Dichters}|pw}\newverse{}unter fremden Fichten, in
                     Menſchenſehnſuchtsqual,\pwindex{Dehmel, Richard 18.\,11.\,1863 Hermsdorf – 8.\,2.\,1920 Blankenese@\textsc{Dehmel, Richard} (18.\,11.\,1863 Hermsdorf – 8.\,2.\,1920 Blankenese), \emph{Schriftsteller, Schriftsteller, Krimiautor}!Haus des Dichters@\strich\emph{Das Haus des Dichters}|pw}\newverse{}nicht mehr von dir träumen wollte,\pwindex{Dehmel, Richard 18.\,11.\,1863 Hermsdorf – 8.\,2.\,1920 Blankenese@\textsc{Dehmel, Richard} (18.\,11.\,1863 Hermsdorf – 8.\,2.\,1920 Blankenese), \emph{Schriftsteller, Schriftsteller, Krimiautor}!Haus des Dichters@\strich\emph{Das Haus des Dichters}|pw}\newverse{}von deinem ewigen Zauberſtrahl\pwindex{Dehmel, Richard 18.\,11.\,1863 Hermsdorf – 8.\,2.\,1920 Blankenese@\textsc{Dehmel, Richard} (18.\,11.\,1863 Hermsdorf – 8.\,2.\,1920 Blankenese), \emph{Schriftsteller, Schriftsteller, Krimiautor}!Haus des Dichters@\strich\emph{Das Haus des Dichters}|pw}\newverse{}und nie erlebten Wunderdingen,\pwindex{Dehmel, Richard 18.\,11.\,1863 Hermsdorf – 8.\,2.\,1920 Blankenese@\textsc{Dehmel, Richard} (18.\,11.\,1863 Hermsdorf – 8.\,2.\,1920 Blankenese), \emph{Schriftsteller, Schriftsteller, Krimiautor}!Haus des Dichters@\strich\emph{Das Haus des Dichters}|pw}\newverse{}nur von Heimat, Heimat endlich einmal –\pwindex{Dehmel, Richard 18.\,11.\,1863 Hermsdorf – 8.\,2.\,1920 Blankenese@\textsc{Dehmel, Richard} (18.\,11.\,1863 Hermsdorf – 8.\,2.\,1920 Blankenese), \emph{Schriftsteller, Schriftsteller, Krimiautor}!Haus des Dichters@\strich\emph{Das Haus des Dichters}|pw}\newverse{}da huben die Sterne an zu klingen:\pwindex{Dehmel, Richard 18.\,11.\,1863 Hermsdorf – 8.\,2.\,1920 Blankenese@\textsc{Dehmel, Richard} (18.\,11.\,1863 Hermsdorf – 8.\,2.\,1920 Blankenese), \emph{Schriftsteller, Schriftsteller, Krimiautor}!Haus des Dichters@\strich\emph{Das Haus des Dichters}|pw}\newverse{}Ja, die ganze Welt kannſt du wild
                     durchſchweifen\pwindex{Dehmel, Richard 18.\,11.\,1863 Hermsdorf – 8.\,2.\,1920 Blankenese@\textsc{Dehmel, Richard} (18.\,11.\,1863 Hermsdorf – 8.\,2.\,1920 Blankenese), \emph{Schriftsteller, Schriftsteller, Krimiautor}!Haus des Dichters@\strich\emph{Das Haus des Dichters}|pw}\newverse{}in deinem freiheitstrunknen Flug,\pwindex{Dehmel, Richard 18.\,11.\,1863 Hermsdorf – 8.\,2.\,1920 Blankenese@\textsc{Dehmel, Richard} (18.\,11.\,1863 Hermsdorf – 8.\,2.\,1920 Blankenese), \emph{Schriftsteller, Schriftsteller, Krimiautor}!Haus des Dichters@\strich\emph{Das Haus des Dichters}|pw}\newverse{}kannſt Kometen begleiten durch
                     Urnebelſtreifen,\pwindex{Dehmel, Richard 18.\,11.\,1863 Hermsdorf – 8.\,2.\,1920 Blankenese@\textsc{Dehmel, Richard} (18.\,11.\,1863 Hermsdorf – 8.\,2.\,1920 Blankenese), \emph{Schriftsteller, Schriftsteller, Krimiautor}!Haus des Dichters@\strich\emph{Das Haus des Dichters}|pw}\newverse{}Stürme, Wolken, Blitz dir zum Spielzeug
                     greifen,\pwindex{Dehmel, Richard 18.\,11.\,1863 Hermsdorf – 8.\,2.\,1920 Blankenese@\textsc{Dehmel, Richard} (18.\,11.\,1863 Hermsdorf – 8.\,2.\,1920 Blankenese), \emph{Schriftsteller, Schriftsteller, Krimiautor}!Haus des Dichters@\strich\emph{Das Haus des Dichters}|pw}\newverse{}{\pb}ach, und haſt nicht Kraft
                     genug,\pwindex{Dehmel, Richard 18.\,11.\,1863 Hermsdorf – 8.\,2.\,1920 Blankenese@\textsc{Dehmel, Richard} (18.\,11.\,1863 Hermsdorf – 8.\,2.\,1920 Blankenese), \emph{Schriftsteller, Schriftsteller, Krimiautor}!Haus des Dichters@\strich\emph{Das Haus des Dichters}|pw}\newverse{}ein Haus auf der feſten Erde zu bauen,\pwindex{Dehmel, Richard 18.\,11.\,1863 Hermsdorf – 8.\,2.\,1920 Blankenese@\textsc{Dehmel, Richard} (18.\,11.\,1863 Hermsdorf – 8.\,2.\,1920 Blankenese), \emph{Schriftsteller, Schriftsteller, Krimiautor}!Haus des Dichters@\strich\emph{Das Haus des Dichters}|pw}\newverse{}für dich und die Deinen ein{ }ſichres Bett,\pwindex{Dehmel, Richard 18.\,11.\,1863 Hermsdorf – 8.\,2.\,1920 Blankenese@\textsc{Dehmel, Richard} (18.\,11.\,1863 Hermsdorf – 8.\,2.\,1920 Blankenese), \emph{Schriftsteller, Schriftsteller, Krimiautor}!Haus des Dichters@\strich\emph{Das Haus des Dichters}|pw}\newverse{}kannſt dir nicht einen Balken{ }ſelber hauen,\pwindex{Dehmel, Richard 18.\,11.\,1863 Hermsdorf – 8.\,2.\,1920 Blankenese@\textsc{Dehmel, Richard} (18.\,11.\,1863 Hermsdorf – 8.\,2.\,1920 Blankenese), \emph{Schriftsteller, Schriftsteller, Krimiautor}!Haus des Dichters@\strich\emph{Das Haus des Dichters}|pw}\newverse{}nicht ein Tiſchlein zu zimmern dich getrauen,\pwindex{Dehmel, Richard 18.\,11.\,1863 Hermsdorf – 8.\,2.\,1920 Blankenese@\textsc{Dehmel, Richard} (18.\,11.\,1863 Hermsdorf – 8.\,2.\,1920 Blankenese), \emph{Schriftsteller, Schriftsteller, Krimiautor}!Haus des Dichters@\strich\emph{Das Haus des Dichters}|pw}\newverse{}nicht ein Brett,\pwindex{Dehmel, Richard 18.\,11.\,1863 Hermsdorf – 8.\,2.\,1920 Blankenese@\textsc{Dehmel, Richard} (18.\,11.\,1863 Hermsdorf – 8.\,2.\,1920 Blankenese), \emph{Schriftsteller, Schriftsteller, Krimiautor}!Haus des Dichters@\strich\emph{Das Haus des Dichters}|pw}\newverse{}hockſt wie ein unbeholfnes Tier\pwindex{Dehmel, Richard 18.\,11.\,1863 Hermsdorf – 8.\,2.\,1920 Blankenese@\textsc{Dehmel, Richard} (18.\,11.\,1863 Hermsdorf – 8.\,2.\,1920 Blankenese), \emph{Schriftsteller, Schriftsteller, Krimiautor}!Haus des Dichters@\strich\emph{Das Haus des Dichters}|pw}\newverse{}unter den fremden Fichten hier –\pwindex{Dehmel, Richard 18.\,11.\,1863 Hermsdorf – 8.\,2.\,1920 Blankenese@\textsc{Dehmel, Richard} (18.\,11.\,1863 Hermsdorf – 8.\,2.\,1920 Blankenese), \emph{Schriftsteller, Schriftsteller, Krimiautor}!Haus des Dichters@\strich\emph{Das Haus des Dichters}|pw}\newverse{}ſo erklangen die Sterne – da flucht’ ich dir.\pwindex{Dehmel, Richard 18.\,11.\,1863 Hermsdorf – 8.\,2.\,1920 Blankenese@\textsc{Dehmel, Richard} (18.\,11.\,1863 Hermsdorf – 8.\,2.\,1920 Blankenese), \emph{Schriftsteller, Schriftsteller, Krimiautor}!Haus des Dichters@\strich\emph{Das Haus des Dichters}|pw}\newverse{}Bis der Morgen graute, bis Menſchen kamen,\pwindex{Dehmel, Richard 18.\,11.\,1863 Hermsdorf – 8.\,2.\,1920 Blankenese@\textsc{Dehmel, Richard} (18.\,11.\,1863 Hermsdorf – 8.\,2.\,1920 Blankenese), \emph{Schriftsteller, Schriftsteller, Krimiautor}!Haus des Dichters@\strich\emph{Das Haus des Dichters}|pw}\newverse{}hilfreich kamen, Mann für Mann,\pwindex{Dehmel, Richard 18.\,11.\,1863 Hermsdorf – 8.\,2.\,1920 Blankenese@\textsc{Dehmel, Richard} (18.\,11.\,1863 Hermsdorf – 8.\,2.\,1920 Blankenese), \emph{Schriftsteller, Schriftsteller, Krimiautor}!Haus des Dichters@\strich\emph{Das Haus des Dichters}|pw}\newverse{}mich herzlich bei den Händen nahmen,\pwindex{Dehmel, Richard 18.\,11.\,1863 Hermsdorf – 8.\,2.\,1920 Blankenese@\textsc{Dehmel, Richard} (18.\,11.\,1863 Hermsdorf – 8.\,2.\,1920 Blankenese), \emph{Schriftsteller, Schriftsteller, Krimiautor}!Haus des Dichters@\strich\emph{Das Haus des Dichters}|pw}\newverse{}und holde Frauen lachten mich an:\pwindex{Dehmel, Richard 18.\,11.\,1863 Hermsdorf – 8.\,2.\,1920 Blankenese@\textsc{Dehmel, Richard} (18.\,11.\,1863 Hermsdorf – 8.\,2.\,1920 Blankenese), \emph{Schriftsteller, Schriftsteller, Krimiautor}!Haus des Dichters@\strich\emph{Das Haus des Dichters}|pw}\newverse{}Sieh doch, da{ }ſteht das Haus{ }ſchon errichtet;\pwindex{Dehmel, Richard 18.\,11.\,1863 Hermsdorf – 8.\,2.\,1920 Blankenese@\textsc{Dehmel, Richard} (18.\,11.\,1863 Hermsdorf – 8.\,2.\,1920 Blankenese), \emph{Schriftsteller, Schriftsteller, Krimiautor}!Haus des Dichters@\strich\emph{Das Haus des Dichters}|pw}\newverse{}während du{ }ſchweifteſt von Traum zu Traum,\pwindex{Dehmel, Richard 18.\,11.\,1863 Hermsdorf – 8.\,2.\,1920 Blankenese@\textsc{Dehmel, Richard} (18.\,11.\,1863 Hermsdorf – 8.\,2.\,1920 Blankenese), \emph{Schriftsteller, Schriftsteller, Krimiautor}!Haus des Dichters@\strich\emph{Das Haus des Dichters}|pw}\newverse{}ward Stein auf Stein zur Mauer geſchichtet,\pwindex{Dehmel, Richard 18.\,11.\,1863 Hermsdorf – 8.\,2.\,1920 Blankenese@\textsc{Dehmel, Richard} (18.\,11.\,1863 Hermsdorf – 8.\,2.\,1920 Blankenese), \emph{Schriftsteller, Schriftsteller, Krimiautor}!Haus des Dichters@\strich\emph{Das Haus des Dichters}|pw}\newverse{}der dunkle Hain zum Garten gelichtet,\pwindex{Dehmel, Richard 18.\,11.\,1863 Hermsdorf – 8.\,2.\,1920 Blankenese@\textsc{Dehmel, Richard} (18.\,11.\,1863 Hermsdorf – 8.\,2.\,1920 Blankenese), \emph{Schriftsteller, Schriftsteller, Krimiautor}!Haus des Dichters@\strich\emph{Das Haus des Dichters}|pw}\newverse{}dir zum heimatlichen Raum.\pwindex{Dehmel, Richard 18.\,11.\,1863 Hermsdorf – 8.\,2.\,1920 Blankenese@\textsc{Dehmel, Richard} (18.\,11.\,1863 Hermsdorf – 8.\,2.\,1920 Blankenese), \emph{Schriftsteller, Schriftsteller, Krimiautor}!Haus des Dichters@\strich\emph{Das Haus des Dichters}|pw}\newverse{}Nach freudiger Menſchheit ging dein Trachten;\pwindex{Dehmel, Richard 18.\,11.\,1863 Hermsdorf – 8.\,2.\,1920 Blankenese@\textsc{Dehmel, Richard} (18.\,11.\,1863 Hermsdorf – 8.\,2.\,1920 Blankenese), \emph{Schriftsteller, Schriftsteller, Krimiautor}!Haus des Dichters@\strich\emph{Das Haus des Dichters}|pw}\newverse{}weil du{ }ſie träumteſt, lebt{ }ſie nun;\pwindex{Dehmel, Richard 18.\,11.\,1863 Hermsdorf – 8.\,2.\,1920 Blankenese@\textsc{Dehmel, Richard} (18.\,11.\,1863 Hermsdorf – 8.\,2.\,1920 Blankenese), \emph{Schriftsteller, Schriftsteller, Krimiautor}!Haus des Dichters@\strich\emph{Das Haus des Dichters}|pw}\newverse{}du halfeſt ihr{ }ſich göttlich achten,\pwindex{Dehmel, Richard 18.\,11.\,1863 Hermsdorf – 8.\,2.\,1920 Blankenese@\textsc{Dehmel, Richard} (18.\,11.\,1863 Hermsdorf – 8.\,2.\,1920 Blankenese), \emph{Schriftsteller, Schriftsteller, Krimiautor}!Haus des Dichters@\strich\emph{Das Haus des Dichters}|pw}\newverse{}empfang als Schöpferlohn ihr Tun;\pwindex{Dehmel, Richard 18.\,11.\,1863 Hermsdorf – 8.\,2.\,1920 Blankenese@\textsc{Dehmel, Richard} (18.\,11.\,1863 Hermsdorf – 8.\,2.\,1920 Blankenese), \emph{Schriftsteller, Schriftsteller, Krimiautor}!Haus des Dichters@\strich\emph{Das Haus des Dichters}|pw}\newverse{}laß dir aus unſern{ }ſchwachen Händen\pwindex{Dehmel, Richard 18.\,11.\,1863 Hermsdorf – 8.\,2.\,1920 Blankenese@\textsc{Dehmel, Richard} (18.\,11.\,1863 Hermsdorf – 8.\,2.\,1920 Blankenese), \emph{Schriftsteller, Schriftsteller, Krimiautor}!Haus des Dichters@\strich\emph{Das Haus des Dichters}|pw}\newverse{}den Segen vieler{ }ſtarken{ }ſpenden!\pwindex{Dehmel, Richard 18.\,11.\,1863 Hermsdorf – 8.\,2.\,1920 Blankenese@\textsc{Dehmel, Richard} (18.\,11.\,1863 Hermsdorf – 8.\,2.\,1920 Blankenese), \emph{Schriftsteller, Schriftsteller, Krimiautor}!Haus des Dichters@\strich\emph{Das Haus des Dichters}|pw}\newverse{}{\pb}So{ }ſprachen{ }ſtrahlend zwei der
                     Frauen,\pwindex{Dehmel, Richard 18.\,11.\,1863 Hermsdorf – 8.\,2.\,1920 Blankenese@\textsc{Dehmel, Richard} (18.\,11.\,1863 Hermsdorf – 8.\,2.\,1920 Blankenese), \emph{Schriftsteller, Schriftsteller, Krimiautor}!Haus des Dichters@\strich\emph{Das Haus des Dichters}|pw}\newverse{}mich aber wehte ein Bangen an:\pwindex{Dehmel, Richard 18.\,11.\,1863 Hermsdorf – 8.\,2.\,1920 Blankenese@\textsc{Dehmel, Richard} (18.\,11.\,1863 Hermsdorf – 8.\,2.\,1920 Blankenese), \emph{Schriftsteller, Schriftsteller, Krimiautor}!Haus des Dichters@\strich\emph{Das Haus des Dichters}|pw}\newverse{}verflogen war das Morgengrauen,\pwindex{Dehmel, Richard 18.\,11.\,1863 Hermsdorf – 8.\,2.\,1920 Blankenese@\textsc{Dehmel, Richard} (18.\,11.\,1863 Hermsdorf – 8.\,2.\,1920 Blankenese), \emph{Schriftsteller, Schriftsteller, Krimiautor}!Haus des Dichters@\strich\emph{Das Haus des Dichters}|pw}\newverse{}und über dem{ }ſonneblanken Tann\pwindex{Dehmel, Richard 18.\,11.\,1863 Hermsdorf – 8.\,2.\,1920 Blankenese@\textsc{Dehmel, Richard} (18.\,11.\,1863 Hermsdorf – 8.\,2.\,1920 Blankenese), \emph{Schriftsteller, Schriftsteller, Krimiautor}!Haus des Dichters@\strich\emph{Das Haus des Dichters}|pw}\newverse{}fern im Blauen\pwindex{Dehmel, Richard 18.\,11.\,1863 Hermsdorf – 8.\,2.\,1920 Blankenese@\textsc{Dehmel, Richard} (18.\,11.\,1863 Hermsdorf – 8.\,2.\,1920 Blankenese), \emph{Schriftsteller, Schriftsteller, Krimiautor}!Haus des Dichters@\strich\emph{Das Haus des Dichters}|pw}\newverse{}ſah ich{ }ſtarr dich mit zitternden Klauen\pwindex{Dehmel, Richard 18.\,11.\,1863 Hermsdorf – 8.\,2.\,1920 Blankenese@\textsc{Dehmel, Richard} (18.\,11.\,1863 Hermsdorf – 8.\,2.\,1920 Blankenese), \emph{Schriftsteller, Schriftsteller, Krimiautor}!Haus des Dichters@\strich\emph{Das Haus des Dichters}|pw}\newverse{}ſchreckbeſchwert\pwindex{Dehmel, Richard 18.\,11.\,1863 Hermsdorf – 8.\,2.\,1920 Blankenese@\textsc{Dehmel, Richard} (18.\,11.\,1863 Hermsdorf – 8.\,2.\,1920 Blankenese), \emph{Schriftsteller, Schriftsteller, Krimiautor}!Haus des Dichters@\strich\emph{Das Haus des Dichters}|pw}\newverse{}– Phönix –{ }ſprühend niederſchauen\pwindex{Dehmel, Richard 18.\,11.\,1863 Hermsdorf – 8.\,2.\,1920 Blankenese@\textsc{Dehmel, Richard} (18.\,11.\,1863 Hermsdorf – 8.\,2.\,1920 Blankenese), \emph{Schriftsteller, Schriftsteller, Krimiautor}!Haus des Dichters@\strich\emph{Das Haus des Dichters}|pw}\newverse{}auf meinen Herd.\pwindex{Dehmel, Richard 18.\,11.\,1863 Hermsdorf – 8.\,2.\,1920 Blankenese@\textsc{Dehmel, Richard} (18.\,11.\,1863 Hermsdorf – 8.\,2.\,1920 Blankenese), \emph{Schriftsteller, Schriftsteller, Krimiautor}!Haus des Dichters@\strich\emph{Das Haus des Dichters}|pw}\newverse{}Wie Sankt Johannes zwiſchen den{ }ſieben
                     Leuchtern\pwindex{Dehmel, Richard 18.\,11.\,1863 Hermsdorf – 8.\,2.\,1920 Blankenese@\textsc{Dehmel, Richard} (18.\,11.\,1863 Hermsdorf – 8.\,2.\,1920 Blankenese), \emph{Schriftsteller, Schriftsteller, Krimiautor}!Haus des Dichters@\strich\emph{Das Haus des Dichters}|pw}\newverse{}mit gen Boden gebeugtem Geſicht\pwindex{Dehmel, Richard 18.\,11.\,1863 Hermsdorf – 8.\,2.\,1920 Blankenese@\textsc{Dehmel, Richard} (18.\,11.\,1863 Hermsdorf – 8.\,2.\,1920 Blankenese), \emph{Schriftsteller, Schriftsteller, Krimiautor}!Haus des Dichters@\strich\emph{Das Haus des Dichters}|pw}\newverse{}barg ich unter den hohen Bäumen\pwindex{Dehmel, Richard 18.\,11.\,1863 Hermsdorf – 8.\,2.\,1920 Blankenese@\textsc{Dehmel, Richard} (18.\,11.\,1863 Hermsdorf – 8.\,2.\,1920 Blankenese), \emph{Schriftsteller, Schriftsteller, Krimiautor}!Haus des Dichters@\strich\emph{Das Haus des Dichters}|pw}\newverse{}meinen Blick vor all dem Gnadenlicht;\pwindex{Dehmel, Richard 18.\,11.\,1863 Hermsdorf – 8.\,2.\,1920 Blankenese@\textsc{Dehmel, Richard} (18.\,11.\,1863 Hermsdorf – 8.\,2.\,1920 Blankenese), \emph{Schriftsteller, Schriftsteller, Krimiautor}!Haus des Dichters@\strich\emph{Das Haus des Dichters}|pw}\newverse{}in meinen Tränen{ }ſtoſſen zu taumelnden
                     Flammen\pwindex{Dehmel, Richard 18.\,11.\,1863 Hermsdorf – 8.\,2.\,1920 Blankenese@\textsc{Dehmel, Richard} (18.\,11.\,1863 Hermsdorf – 8.\,2.\,1920 Blankenese), \emph{Schriftsteller, Schriftsteller, Krimiautor}!Haus des Dichters@\strich\emph{Das Haus des Dichters}|pw}\newverse{}die Menſchen rings mit euch zuſammen,\pwindex{Dehmel, Richard 18.\,11.\,1863 Hermsdorf – 8.\,2.\,1920 Blankenese@\textsc{Dehmel, Richard} (18.\,11.\,1863 Hermsdorf – 8.\,2.\,1920 Blankenese), \emph{Schriftsteller, Schriftsteller, Krimiautor}!Haus des Dichters@\strich\emph{Das Haus des Dichters}|pw}\newverse{}ihr alten Fichten um dies neue Dach –\pwindex{Dehmel, Richard 18.\,11.\,1863 Hermsdorf – 8.\,2.\,1920 Blankenese@\textsc{Dehmel, Richard} (18.\,11.\,1863 Hermsdorf – 8.\,2.\,1920 Blankenese), \emph{Schriftsteller, Schriftsteller, Krimiautor}!Haus des Dichters@\strich\emph{Das Haus des Dichters}|pw}\newverse{}was rauſcht ihr mir Erinnrung, ach!\pwindex{Dehmel, Richard 18.\,11.\,1863 Hermsdorf – 8.\,2.\,1920 Blankenese@\textsc{Dehmel, Richard} (18.\,11.\,1863 Hermsdorf – 8.\,2.\,1920 Blankenese), \emph{Schriftsteller, Schriftsteller, Krimiautor}!Haus des Dichters@\strich\emph{Das Haus des Dichters}|pw}\newverse{}Ich fühl’s noch heute beim Schwanken eurer
                     Zweige,\pwindex{Dehmel, Richard 18.\,11.\,1863 Hermsdorf – 8.\,2.\,1920 Blankenese@\textsc{Dehmel, Richard} (18.\,11.\,1863 Hermsdorf – 8.\,2.\,1920 Blankenese), \emph{Schriftsteller, Schriftsteller, Krimiautor}!Haus des Dichters@\strich\emph{Das Haus des Dichters}|pw}\newverse{}wie ich erſchüttert den Nacken neige,\pwindex{Dehmel, Richard 18.\,11.\,1863 Hermsdorf – 8.\,2.\,1920 Blankenese@\textsc{Dehmel, Richard} (18.\,11.\,1863 Hermsdorf – 8.\,2.\,1920 Blankenese), \emph{Schriftsteller, Schriftsteller, Krimiautor}!Haus des Dichters@\strich\emph{Das Haus des Dichters}|pw}\newverse{}weil mir zum Dank die Kraft gebricht.\pwindex{Dehmel, Richard 18.\,11.\,1863 Hermsdorf – 8.\,2.\,1920 Blankenese@\textsc{Dehmel, Richard} (18.\,11.\,1863 Hermsdorf – 8.\,2.\,1920 Blankenese), \emph{Schriftsteller, Schriftsteller, Krimiautor}!Haus des Dichters@\strich\emph{Das Haus des Dichters}|pw}\newverse{}Ich kann ja nichts als immer wieder träumen\pwindex{Dehmel, Richard 18.\,11.\,1863 Hermsdorf – 8.\,2.\,1920 Blankenese@\textsc{Dehmel, Richard} (18.\,11.\,1863 Hermsdorf – 8.\,2.\,1920 Blankenese), \emph{Schriftsteller, Schriftsteller, Krimiautor}!Haus des Dichters@\strich\emph{Das Haus des Dichters}|pw}\newverse{}von{ }ſeligem Aufflug zu den freien Räumen –\pwindex{Dehmel, Richard 18.\,11.\,1863 Hermsdorf – 8.\,2.\,1920 Blankenese@\textsc{Dehmel, Richard} (18.\,11.\,1863 Hermsdorf – 8.\,2.\,1920 Blankenese), \emph{Schriftsteller, Schriftsteller, Krimiautor}!Haus des Dichters@\strich\emph{Das Haus des Dichters}|pw}\newverse{}O Phönix, Phönix, verlaß mich nicht! –\pwindex{Dehmel, Richard 18.\,11.\,1863 Hermsdorf – 8.\,2.\,1920 Blankenese@\textsc{Dehmel, Richard} (18.\,11.\,1863 Hermsdorf – 8.\,2.\,1920 Blankenese), \emph{Schriftsteller, Schriftsteller, Krimiautor}!Haus des Dichters@\strich\emph{Das Haus des Dichters}|pw}\stanzaend{}
\pstart
           \centering{}* * *\pend
           
\pstart
           \centering{}{\pb}\label{T_L02157-1v}\edtext{WD}{\lemma{\textnormal{\emph{WD}}}\Cendnote{\textnormal{in Form eines Adlers, die nächste Zeile als
                  Wappenspruch}}}\label{T_L02157-1}\pend
           
\pstart
           \centering{}Force m’est trop\pend
           \selectlanguage{ngerman}\endnumbering\briefempfaengerindex{Schnitzler, Arthur@\textsc{Schnitzler, Arthur}!zzzDehmel, Richard@\emph{von Richard Dehmel}!1913-11-181@{{[}18. 11. 1913?{]}}|)be}\mylabel{L02157h}  \newcommand{\dateiname}{L02157}\newcommand{\titel}{Richard Dehmel an Arthur Schnitzler, [18. 11. 1913?]}\newcommand{\editorInnen}{Martin Anton Müller und Gerd-Hermann Susen}%% latex-leseansicht-abspann.tex
%% Abspann für die Leseansicht.
%% Der Schalter \ifkorrekturansicht ist bereits durch den Vorspann gesetzt.

%% latex-abspann.tex
%% Gemeinsamer Abspann für Korrekturansicht und Leseansicht.
%% Setzt den Schalter \ifkorrekturansicht voraus (gesetzt in den
%% einbindenden Dateien latex-korrekturansicht-abspann.tex bzw.
%% latex-leseansicht-abspann.tex).
%% ---------------------------------------------------------------

\normalsize

% Das esempio-Environment wird nur in der Leseansicht benötigt
\ifkorrekturansicht\else
\newenvironment{esempio}[3]%
{
    \vspace{1.5ex}
    \rlap{\underline{#1}}
    \par
    \setlength{\parindent}{0cm}
    \nopagebreak
    \leftskip=#2cm
    \rightskip=#3cm
}
{
    \par
}
\fi

\doendnotes{C}
\bigskip
\vfill

\clearpage

\footnotesize

\ifkorrekturansicht
  \lohead{\textsc{register}}
\fi

% theindex-Environment neu definieren ohne reledmac
\makeatletter
\renewenvironment{theindex}{%
  \ifkorrekturansicht
    \section*{\indexname}%
  \else
    \subsubsection*{Index der erwähnten Entitäten}%
  \fi
  \setlength{\parindent}{0pt}%
  \setlength{\parskip}{0pt plus 0.3pt}%
  \let\item\@idxitem
}{%
  \ifkorrekturansicht\clearpage\fi
}
\makeatother

\IfFileExists{\jobname-pw.ind}{\input{\jobname-pw.ind}}{}

% Quellenangabe nur in der Leseansicht
\ifkorrekturansicht\else
% Fallback-Definitionen, falls die .tex-Datei \titel etc. nicht gesetzt hat
\providecommand{\titel}{}
\providecommand{\editorInnen}{}
\providecommand{\dateiname}{\jobname}

\vspace{3cm}

\vfill

\footnotesize
\textsc{Quelle}: \titel. Herausgegeben von {\editorInnen}. In: \emph{Arthur Schnitzler: Briefwechsel mit Autorinnen und Autoren}.
 Digitale Edition, https://schnitzler-briefe.acdh.oeaw.ac.at/{\dateiname}.html (Stand \today)
\fi

\end{document}


