%% latex-leseansicht-vorspann.tex
%% Vorspann für die Leseansicht.
%% Lädt die gemeinsame Datei latex-vorspann.tex mit nicht gesetztem Schalter.

\newif\ifkorrekturansicht
\korrekturansichtfalse

\input{../tex-inputs/latex-vorspann}

\begin{center}
            \textcolor{red}{ENTWURF, NICHT FERTIG KORRIGIERT}
                      \end{center}
            
               \section[Arthur Schnitzler an Lotte Bloch-Zavřel, 11. 4. 1927]{ Arthur Schnitzler an Lotte Bloch-Zavřel, 11. 4. 1927}\nopagebreak\mylabel{v}\rehead{ }\begin{ledgroupsized}[t]{13cm}\normalsize\beginnumbering\briefempfaengerindex{Bloch-Zavřel, Lotte@\textsc{Bloch-Zavřel, Lotte}!zzzSchnitzler, Arthur@\emph{von Arthur Schnitzler}!1927-04-111@{11. 4. 1927}|(be} \toendnotes[C]{\smallbreak\pagebreak[2]} \Standort{DLA, A:Schnitzler, 85.1.405.}
\physDesc{Brief, 1 Blatt, 1 Seite, maschineller Durchschlag
\newline{}Handschrift: roter Buntstift, lateinische Kurrent (\noindent{}Beschriftungen »Bloch«, »Brief « und
                                       »Berlin\oindex{Berlin@\textbf{Berlin}|pw}«)
\newline{}Bloch-Zavřel: mit rotem Buntstift drei Unterstreichungen }\pstart
           \raggedleft{}{\pb}11. 4. 1927.\pend
           \pstart{}Verehrte gnädige Frau.\pend\pstart
           Auf Ihre freundliche Anfrage erlaube ich mir zu erwiedern, dass ich gegen eine
               Veröffentlichung des unbeträchtlichen Briefes in Ihrem Sammelbändchen »Briefe an Auguste Hauschner\pwindex{Briefe an Auguste Hauschner1929@\emph{Briefe an Auguste Hauschner} {[}1929{]}|pw}« nichts einzuwenden habe
               und bin mit vorzüglicher Hochachtung\pend
           \pstart Ihr ergebener\pend{}{\bigskip}\pstart
           \noindent{}Frau Lotte Bloch-Zavřel\pwindex{Bloch-Zavřel, Lotte 1886-11-05 – 1979-03-13@\textsc{Bloch-Zavřel, Lotte} (1886-11-05 – 1979-03-13), \emph{Journalistin, Herausgeberin}|pw},\pend
           \pstart
           Charlottenburg\oindex{Charlottenburg@\textbf{Charlottenburg}|pw}.\pend
                     \endnumbering\briefempfaengerindex{Bloch-Zavřel, Lotte@\textsc{Bloch-Zavřel, Lotte}!zzzSchnitzler, Arthur@\emph{von Arthur Schnitzler}!1927-04-111@{11. 4. 1927}|)be}\mylabel{h}\end{ledgroupsized}  \newcommand{\dateiname}{L02584}\newcommand{\titel}{Arthur Schnitzler an Lotte Bloch-Zavřel, 11. 4. 1927}\newcommand{\editorInnen}{Martin Anton Müller und Laura Untner}%% latex-leseansicht-abspann.tex
%% Abspann für die Leseansicht.
%% Der Schalter \ifkorrekturansicht ist bereits durch den Vorspann gesetzt.

%% latex-abspann.tex
%% Gemeinsamer Abspann für Korrekturansicht und Leseansicht.
%% Setzt den Schalter \ifkorrekturansicht voraus (gesetzt in den
%% einbindenden Dateien latex-korrekturansicht-abspann.tex bzw.
%% latex-leseansicht-abspann.tex).
%% ---------------------------------------------------------------

\normalsize

% Das esempio-Environment wird nur in der Leseansicht benötigt
\ifkorrekturansicht\else
\newenvironment{esempio}[3]%
{
    \vspace{1.5ex}
    \rlap{\underline{#1}}
    \par
    \setlength{\parindent}{0cm}
    \nopagebreak
    \leftskip=#2cm
    \rightskip=#3cm
}
{
    \par
}
\fi

\doendnotes{C}
\bigskip
\vfill

\clearpage

\footnotesize

\ifkorrekturansicht
  \lohead{\textsc{register}}
\fi

% theindex-Environment neu definieren ohne reledmac
\makeatletter
\renewenvironment{theindex}{%
  \ifkorrekturansicht
    \section*{\indexname}%
  \else
    \subsubsection*{Index der erwähnten Entitäten}%
  \fi
  \setlength{\parindent}{0pt}%
  \setlength{\parskip}{0pt plus 0.3pt}%
  \let\item\@idxitem
}{%
  \ifkorrekturansicht\clearpage\fi
}
\makeatother

\IfFileExists{\jobname-pw.ind}{\input{\jobname-pw.ind}}{}

% Quellenangabe nur in der Leseansicht
\ifkorrekturansicht\else
% Fallback-Definitionen, falls die .tex-Datei \titel etc. nicht gesetzt hat
\providecommand{\titel}{}
\providecommand{\editorInnen}{}
\providecommand{\dateiname}{\jobname}

\vspace{3cm}

\vfill

\footnotesize
\textsc{Quelle}: \titel. Herausgegeben von {\editorInnen}. In: \emph{Arthur Schnitzler: Briefwechsel mit Autorinnen und Autoren}.
 Digitale Edition, https://schnitzler-briefe.acdh.oeaw.ac.at/{\dateiname}.html (Stand \today)
\fi

\end{document}


      