%% latex-korrekturansicht-vorspann.tex
%% Vorspann für die Korrekturansicht.
%% Lädt die gemeinsame Datei latex-vorspann.tex mit gesetztem Schalter.

\newif\ifkorrekturansicht
\korrekturansichttrue

\input{../tex-inputs/latex-vorspann}


\section[Arthur Schnitzler an Lotte Bloch-Zavřel, 11. 4. 1927]{L02584 Arthur Schnitzler an Lotte Bloch-Zavřel, 11. 4. 1927}
\nopagebreak\mylabel{L02584v}
\rehead{ }\normalsize\beginnumbering\briefempfaengerindex{Bloch-Zavřel, Lotte@\textsc{Bloch-Zavřel, Lotte}!zzzSchnitzler, Arthur@\emph{von Arthur Schnitzler}!1927-04-111@{11. 4. 1927}|(be}
\toendnotes[C]{\smallbreak\pagebreak[2]}\Standort{DLA, A:Schnitzler, 85.1.405.}
\physDesc{Brief, Durchschlag1 Blatt, 1 Seite, 324 Zeichen
\newline{}Handschrift: roter Buntstift, lateinische Kurrent (\noindent{}Beschriftungen »Bloch\pwindex{Bloch-Zavřel, Lotte 1886-11-05 – 1979-03-13@\textsc{Bloch-Zavřel, Lotte} (1886-11-05 – 1979-03-13), \emph{Journalist/Journalistin, Herausgeber/Herausgeberin}|pw}«, »Brief « und »Berlin\oindex{Berlin@\textbf{Berlin}, \emph{P.PPLC}|pw}«)
\newline{}Bloch-Zavřel: mit rotem Buntstift drei Unterstreichungen }
\pstart
           \raggedleft{}{\pb}11. 4. 1927.\pend
           
\pstart{}Verehrte gnädige Frau.\pend\vspace{0.5em}
\pstart
           Auf Ihre freundliche Anfrage erlaube ich mir zu erwiedern, dass ich gegen eine
               Veröffentlichung des unbeträchtlichen Briefs in Ihrem Sammelbändchen »Briefe an Auguste Hauschner\pwindex{Briefe an Auguste Hauschner@\emph{Briefe an Auguste Hauschner}|pw}« nichts einzuwenden habe und bin
               mit vorzüglicher Hochachtung\pend
           \pstart Ihr ergebener\pend{}{\vspace{1\baselineskip}}
\pstart
           \noindent{}Frau Lotte Bloch-Zavrel\pwindex{Bloch-Zavřel, Lotte 1886-11-05 – 1979-03-13@\textsc{Bloch-Zavřel, Lotte} (1886-11-05 – 1979-03-13), \emph{Journalist/Journalistin, Herausgeber/Herausgeberin}|pw},\pend
           
\pstart
           Charlottenburg\oindex{Charlottenburg@\textbf{Charlottenburg}, \emph{P.PPLX}|pw}.\pend
           \selectlanguage{ngerman}\endnumbering\briefempfaengerindex{Bloch-Zavřel, Lotte@\textsc{Bloch-Zavřel, Lotte}!zzzSchnitzler, Arthur@\emph{von Arthur Schnitzler}!1927-04-111@{11. 4. 1927}|)be}\mylabel{L02584h}  \normalsize

\doendnotes{C}
\bigskip
\vfill

\clearpage

\footnotesize

\lohead{\textsc{register}}

% Definiere theindex-Environment komplett neu ohne reledmac
\makeatletter
\renewenvironment{theindex}{%
  \section*{\indexname}%
  \setlength{\parindent}{0pt}%
  \setlength{\parskip}{0pt plus 0.3pt}%
  \let\item\@idxitem
}{%
  \clearpage
}
\makeatother

\IfFileExists{\jobname-pw.ind}{\input{\jobname-pw.ind}}{}

\end{document}

      