%% latex-leseansicht-vorspann.tex
%% Vorspann für die Leseansicht.
%% Lädt die gemeinsame Datei latex-vorspann.tex mit nicht gesetztem Schalter.

\newif\ifkorrekturansicht
\korrekturansichtfalse

\input{../tex-inputs/latex-vorspann}


         
         \renewcommand{\erwaehntePersonen}{Personen: Hermann Bahr, Max Eugen Burckhard, Rémy de Gourmont, Gisela Hajek, Heinrich Kanner, Louise Schnitzler}
         \renewcommand{\erwaehnteInstitutionen}{Institutionen: Burgtheater, Die Zeit. Wiener Wochenschrift, Frankfurter Zeitung, Neues Wiener Tagblatt, Wiener Rundschau, Österreichische Volks-Zeitung}
         \renewcommand{\erwaehnteOrte}{Orte: Frankfurt am Main, Paris, Velden am Wörthersee, Wien}
         \renewcommand{\erwaehnteWerke}{}
               \section[ Paul Goldmann an Arthur Schnitzler, 17. 7. 1899]{ Paul Goldmann an Arthur Schnitzler, 17. 7. 1899}\nopagebreak\mylabel{v}\rehead{ }\begin{ledgroupsized}[t]{13cm}\normalsize\beginnumbering \toendnotes[C]{\smallbreak\pagebreak[2]} \Standort{DLA, A:Schnitzler, HS.NZ85.1.3169.}
\physDesc{Brief, 1 Blatt, 2 Seiten, 915 Zeichen
\newline{}Handschrift: schwarze Tinte, deutsche Kurrent
\newline{}Schnitzler: mit rotem Buntstift drei Unterstreichungen }\toendnotes[C]{\smallbreak}\pstart
           \noindent{}{\pb}\textcolor{gray}{\textbf{\textbf{Frankfurter Zeitung}}}\orgindex{Frankfurter Zeitung@Frankfurter Zeitung|pw}\hfill \textcolor{gray}{\textbf{\textbf{Frankfurt a. M.\oindex{Frankfurt am Main@\textbf{Frankfurt am Main}|pw},}}}{ }17. Juli \textcolor{gray}{\textbf{189}}9.\pend
           \pstart
           \textcolor{gray}{\textbf{und}}\pend
           \pstart
           \textcolor{gray}{\textbf{Handelsblatt.}}\pend
           \pstart
           \textcolor{gray}{\textbf{\textbf{Redaktion\orgindex{Frankfurter Zeitung@Frankfurter Zeitung|pwv}.}\footnote{\noindent{}\textcolor{gray}{\textbf{Für die Redaktion\orgindex{Frankfurter Zeitung@Frankfurter Zeitung|pwv} beſtimmte Briefe und Sendungen wolle man
                                 \so{nicht} an die Perſon eines Redakteurs,
                              ſondern ſtets \textbf{an die Redaktion der Frankfurter Zeitung\orgindex{Frankfurter Zeitung@Frankfurter Zeitung|pw}} adreſſiren. }}}}}\pend
           \pstart
           \textcolor{gray}{\textbf{Telegramm-Adreſſe:}}\pend
           \pstart
           \textcolor{gray}{\textbf{\textbf{Zeitung\orgindex{Frankfurter Zeitung@Frankfurter Zeitung|pwv}{ }Frankfurt Main\oindex{Frankfurt am Main@\textbf{Frankfurt am Main}|pw}.}}}\pend
           \pstart\center{}Mein lieber Freund,\pend\pstart
           Unſere Briefe haben ſich gekreuzt. Ich \label{K_L02879-1v}\edtext{ſchrieb Dir geſtern}{\lemma{\textnormal{\emph{ſchrieb Dir geſtern}}}\Cendnote{\textnormal{siehe Paul Goldmann an Arthur Schnitzler, 16. 7. 1899}}}\label{K_L02879-1h} nach Wien\oindex{Wien@\textbf{Wien}|pw} und theilte Dir meine
               veränderten Sommer-Dispoſitionen mit. Der Brief wird Dir hoffentlich
               nachgeſchickt.\pend
           \pstart
           Daß \label{K_L02879-2v}\edtext{\textsc{Bahr\pwindex{Bahr, Hermann 19.07.1863 – 15.01.1934@\textsc{Bahr, Hermann} (19.07.1863 – 15.01.1934), \emph{Schriftsteller, Kritiker}|pw}} von der »Zeit\orgindex{Zeit. Wiener Wochenschrift@Die Zeit. Wiener Wochenschrift|pw}« weggeht}{\lemma{\textnormal{\emph{Bahr … weggeht}}}\Cendnote{\textnormal{Im Herbst 1899
                  folgte der ehemalige \emph{Burgtheater}\orgindex{Burgtheater@Burgtheater|pwk}direktor Max Burckhard\pwindex{Burckhard, Max Eugen 14.07.1854 – 16.03.1912@\textsc{Burckhard, Max Eugen} (14.07.1854 – 16.03.1912), \emph{Schriftsteller, Wissenschaftler, Theaterleiter}|pwk} als Leiter des Kulturteils der
                     \emph{Zeit}\orgindex{Zeit. Wiener Wochenschrift@Die Zeit. Wiener Wochenschrift|pwk} nach. Bahr\pwindex{Bahr, Hermann 19.07.1863 – 15.01.1934@\textsc{Bahr, Hermann} (19.07.1863 – 15.01.1934), \emph{Schriftsteller, Kritiker}|pwk} schrieb fortan Feuilletons und Theaterkritiken für
                  die \emph{Österreichische Volks-Zeitung}\orgindex{Oesterreichische Volks-Zeitung@Österreichische Volks-Zeitung|pwk} und das \emph{Neue Wiener Tagblatt}\orgindex{Neues Wiener Tagblatt@Neues Wiener Tagblatt|pwk}.}}}\label{K_L02879-2h}, iſt ein Glück für
               das Blatt\orgindex{Zeit. Wiener Wochenschrift@Die Zeit. Wiener Wochenschrift|pwv}. Wer wird an ſeine
               Stelle kommen? Wenn Du \textsc{Kanner\pwindex{Kanner, Heinrich 09.11.1864 – 15.02.1930@\textsc{Kanner, Heinrich} (09.11.1864 – 15.02.1930), \emph{Herausgeber, Publizist}|pw}} ſiehſt, ſo ſag’ ihm, ich laſſe ihn bitten, es ſich ſo einzurichten, daß er
               nicht vor Ende Auguſt{ }\strikeout{hie\textcolor{gray}{r}}{ }hierher\oindex{Frankfurt am Main@\textbf{Frankfurt am Main}|pwv}kommt. Sonſt trifft er
               mich nicht, und ich möchte ihn doch gar zu gern ſehen. Von \label{K_L02879-3v}\edtext{\textsc{Remy de Gourmont\pwindex{Gourmont, Remy de 1858-04-04 – 1915-09-27@\textsc{Gourmont, Rémy de} (1858-04-04 – 1915-09-27), \emph{Schriftsteller, Kritiker}|pw}}}{\lemma{\textnormal{\emph{Remy de Gourmont}}}\Cendnote{\textnormal{Die Erwähnung Kanner\pwindex{Kanner, Heinrich 09.11.1864 – 15.02.1930@\textsc{Kanner, Heinrich} (09.11.1864 – 15.02.1930), \emph{Herausgeber, Publizist}|pwk}s könnte als Hinweis genommen werden, dass Gourmont\pwindex{Gourmont, Remy de 1858-04-04 – 1915-09-27@\textsc{Gourmont, Rémy de} (1858-04-04 – 1915-09-27), \emph{Schriftsteller, Kritiker}|pwk} in irgendeiner Funktion für die \emph{Zeit}\orgindex{Zeit. Wiener Wochenschrift@Die Zeit. Wiener Wochenschrift|pwk} angedacht war. Er begann aber
                     1899 für die \emph{Wiener Rundschau}\orgindex{Wiener Rundschau@Wiener Rundschau|pwk}
                  aus Paris\oindex{Paris@\textbf{Paris}|pwk} zu berichten.}}}\label{K_L02879-3h} weiß ich {\pb}wenig. Ich muß mich infolgedeſſen des Urtheils
               einſtweilen enthalten und will über dieſen oder einen anderen Pariſ\oindex{Paris@\textbf{Paris}|pw}er Correſpondenten nachdenken.\pend
           \pstart
           Ich freue mich, daß Du Dich zerſtreuſt. Könnte ich Dich nur endlich einmal wieder
               ſehen!\pend
           \pstart
           Erhole Dich nach Möglichkeit, ſchreib’ mir bald und ſei von Herzen gegrüßt!\pend
           \pstart
           Dein treuer {\\[\baselineskip]}\spacefill\mbox{Paul Goldmann}\pend
           \leftskip=0em{}\pstart
           \noindent{}Bitte, viele Grüße an Deine Frau Mutter\pwindex{Schnitzler, Louise 1840-07-08 – 1911-09-09@\textsc{Schnitzler, Louise} (1840-07-08 – 1911-09-09)|pwv} und Frau Schweſter\pwindex{Hajek, Gisela 20.12.1867 – 03.02.1953@\textsc{Hajek, Gisela} (20.12.1867 – 03.02.1953)|pwv} zu beſtellen!\pend
           
         
         \endnumbering\mylabel{h}\end{ledgroupsized}  \newcommand{\dateiname}{L02879}\newcommand{\titel}{Paul Goldmann an Arthur Schnitzler, 17. 7. 1899}\newcommand{\editorInnen}{Martin Anton Müller und Laura Untner}%% latex-leseansicht-abspann.tex
%% Abspann für die Leseansicht.
%% Der Schalter \ifkorrekturansicht ist bereits durch den Vorspann gesetzt.

%% latex-abspann.tex
%% Gemeinsamer Abspann für Korrekturansicht und Leseansicht.
%% Setzt den Schalter \ifkorrekturansicht voraus (gesetzt in den
%% einbindenden Dateien latex-korrekturansicht-abspann.tex bzw.
%% latex-leseansicht-abspann.tex).
%% ---------------------------------------------------------------

\normalsize

% Das esempio-Environment wird nur in der Leseansicht benötigt
\ifkorrekturansicht\else
\newenvironment{esempio}[3]%
{
    \vspace{1.5ex}
    \rlap{\underline{#1}}
    \par
    \setlength{\parindent}{0cm}
    \nopagebreak
    \leftskip=#2cm
    \rightskip=#3cm
}
{
    \par
}
\fi

\doendnotes{C}
\bigskip
\vfill

\clearpage

\footnotesize

\ifkorrekturansicht
  \lohead{\textsc{register}}
\fi

% theindex-Environment neu definieren ohne reledmac
\makeatletter
\renewenvironment{theindex}{%
  \ifkorrekturansicht
    \section*{\indexname}%
  \else
    \subsubsection*{Index der erwähnten Entitäten}%
  \fi
  \setlength{\parindent}{0pt}%
  \setlength{\parskip}{0pt plus 0.3pt}%
  \let\item\@idxitem
}{%
  \ifkorrekturansicht\clearpage\fi
}
\makeatother

\IfFileExists{\jobname-pw.ind}{\input{\jobname-pw.ind}}{}

% Quellenangabe nur in der Leseansicht
\ifkorrekturansicht\else
% Fallback-Definitionen, falls die .tex-Datei \titel etc. nicht gesetzt hat
\providecommand{\titel}{}
\providecommand{\editorInnen}{}
\providecommand{\dateiname}{\jobname}

\vspace{3cm}

\vfill

\footnotesize
\textsc{Quelle}: \titel. Herausgegeben von {\editorInnen}. In: \emph{Arthur Schnitzler: Briefwechsel mit Autorinnen und Autoren}.
 Digitale Edition, https://schnitzler-briefe.acdh.oeaw.ac.at/{\dateiname}.html (Stand \today)
\fi

\end{document}


      