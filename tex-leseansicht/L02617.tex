%% latex-leseansicht-vorspann.tex
%% Vorspann für die Leseansicht.
%% Lädt die gemeinsame Datei latex-vorspann.tex mit nicht gesetztem Schalter.

\newif\ifkorrekturansicht
\korrekturansichtfalse

\input{../tex-inputs/latex-vorspann}


\section[ Paul Goldmann an Arthur Schnitzler, 21. 4. {[}1894{]}]{L02617 Paul Goldmann an Arthur Schnitzler,  21. 4. [1894]}
\nopagebreak\mylabel{L02617v}
\rehead{ }\normalsize\beginnumbering\briefempfaengerindex{Schnitzler, Arthur@\textsc{Schnitzler, Arthur}!zzzGoldmann, Paul@\emph{von Paul Goldmann}!1894-04-212@{21. 4. [1894]}|(be}
\toendnotes[C]{\smallbreak\pagebreak[2]}
\correspDesc{Versand  durch Paul Goldmann am 21. 4. [1894] in Paris
\newline{}Erhalt  durch Arthur Schnitzler im Zeitraum [22. 4. 1894
                  – 26. 4. 1894?] in Wien}\toendnotes[C]{\smallbreak}
\Standort{DLA, A:Schnitzler, HS.NZ85.1.3164.}
\physDesc{Brief, 2 Blätter, 7 Seiten, 3150 Zeichen
\newline{}Handschrift: schwarze Tinte, deutsche Kurrent
\newline{}Schnitzler: 1) mit Bleistift auf dem ersten Blatt die Jahreszahl »94« vermerkt  2) mit rotem Buntstift zwei Unterstreichungen}\toendnotes[C]{\smallbreak}
\pstart
           {\pb}\textcolor{gray}{\textbf{Frankfurter Zeitung\orgindex{Frankfurter Zeitung@Frankfurter Zeitung|pw}.}}\pend
           
\pstart
           \textcolor{gray}{\textbf{(Gazette de
                     Francfort\orgindex{Frankfurter Zeitung@Frankfurter Zeitung|pw}.)}}\pend
           
\pstart
           \textcolor{gray}{\textbf{Directeur \textbf{M. L. Sonnemann\pwindex{Sonnemann, Leopold 29.\,10.\,1831 Höchberg – 30.\,10.\,1909 Frankfurt am Main@\textsc{Sonnemann, Leopold} (29.\,10.\,1831 Höchberg – 30.\,10.\,1909 Frankfurt am Main), \emph{Journalist, Herausgeber}|pw}}.}}\hfill \textsc{Paris\oindex{Paris@\textbf{Paris}, \emph{Hauptstadt}|pw}}, \textcolor{gray}{2}1. April.\pend
           
\pstart
           \textcolor{gray}{\textbf{\begin{otherlanguage}{french}Journal politique, financier,\end{otherlanguage}}}\pend
           
\pstart
           \textcolor{gray}{\textbf{\begin{otherlanguage}{french}commercial et litteraire.\end{otherlanguage}}}\pend
           
\pstart
           \textcolor{gray}{\textbf{\begin{otherlanguage}{french}\textbf{Paraissant trois fois par jour}\end{otherlanguage}}}\pend
           
\pstart
           \textcolor{gray}{\textbf{\begin{otherlanguage}{french}\textbf{Bureaux à Paris\oindex{Paris@\textbf{Paris}, \emph{Hauptstadt}|pw}:}\end{otherlanguage}}}\pend
           
\pstart
           \textcolor{gray}{\textbf{\begin{otherlanguage}{french}rue Richelieu 75\oindex{rue Richelieu@\textbf{rue Richelieu}, \emph{Straße}|pw}.\end{otherlanguage}}}\pend
           
\pstart\center{}Mein lieber Arthur,\pend\vspace{0.5em}
\pstart
           Von morgen ab wechſele ich meine Adreſſe, die fortan
               lautet: \textsc{\uline{24. Rue Feydeau}\oindex{rue Feydeau@\textbf{rue Feydeau}, \emph{Straße}|pw}}.\pend
           
\pstart
           Ich verzichte darauf, Dir \strikeout{zu{ }ſ\textcolor{gray}{a}} jedes mal zu{ }ſagen, eine wie große Freude Du mir{ }ſtets mit Deinen lieben
               Briefen machſt. Du ahnſt nicht, wie wohl mir Deine treue Freundſchaft thut. Ein
               Feſttag in meinem armen Leben. Und ich bin Dir{ }ſo von Herzen dankbar.\pend
           
\pstart
           Ich habe mich{ }ſehr {\pb}gefreut, daß Du mir die
               Bekanntſchaft mit Fräulein \textsc{Sandrock\pwindex{Sandrock, Adele 19.\,8.\,1863 Rotterdam – 30.\,8.\,1937 Berlin@\textsc{Sandrock, Adele} (19.\,8.\,1863 Rotterdam – 30.\,8.\,1937 Berlin), \emph{Schauspielerin}|pw}} vermittelt, und ich danke Dir{ }ſehr für dieſe neue intereſſante Beziehung.\pend
           
\pstart
           \textsc{Albert\pwindex{Albert, Henri 16.\,11.\,1869 Niederbronn-les-Bains – 3.\,8.\,1921 Straßburg@\textsc{Albert, Henri} (16.\,11.\,1869 Niederbronn-les-Bains – 3.\,8.\,1921 Straßburg), \emph{Journalist, Kritiker, Übersetzer}|pw}} habe ich einige Tage lang nicht geſehen. Ich glaube, er wird{ }ſich nun bald an
               Deine Überſetzung\pwindex{Schnitzler, Arthur 15.\,5.\,1862 Wien – 21.\,10.\,1931 ebd.@\textsc{Schnitzler, Arthur} (15.\,5.\,1862 Wien – 21.\,10.\,1931 ebd.), \emph{Schriftsteller, Mediziner}!Emplettes de Noël@\strich\emph{Les Emplettes de Noël}|pwv} machen. Auch
               die Frage der Aufführung an einem hieſigen Theater haben wir oft erörtert. Wir
               verkennen aber Beide\pwindex{Albert, Henri 16.\,11.\,1869 Niederbronn-les-Bains – 3.\,8.\,1921 Straßburg@\textsc{Albert, Henri} (16.\,11.\,1869 Niederbronn-les-Bains – 3.\,8.\,1921 Straßburg), \emph{Journalist, Kritiker, Übersetzer}|pwv} nicht
               die Schwierigkeiten. Fremde Stücke führen hier überhaupt nur die freien Bühnen\orgindex{Théâtre Libre@Théâtre Libre|pwv}\orgindex{Théâtre de l’Œuvre@Théâtre de l’Œuvre|pwv} auf, alſo »\textsc{Théâtre Libre\orgindex{Théâtre Libre@Théâtre Libre|pw}}« und »\textsc{Oeuvre\orgindex{Théâtre de l’Œuvre@Théâtre de l’Œuvre|pw}}«. Während Du alſo bei den übrigen Theatern kaum {\pb}ankommen könnteſt, weil Du ein deutſcher Dichter
               biſt,{ }ſo{ }ſteht Dir bei den beiden letz{[}t{]}genannten der Umſtand
               entgegen, daß Du in Geiſt und Sprache zu fein und zu franz\oindex{Frankreich@\textbf{Frankreich}|pwv}öſiſch biſt. Die Freien Bühnen\orgindex{Théâtre Libre@Théâtre Libre|pwv}\orgindex{Théâtre de l’Œuvre@Théâtre de l’Œuvre|pwv}{ }ſuchen in den
               deutſchen Stücken das für \textsc{Paris\oindex{Paris@\textbf{Paris}, \emph{Hauptstadt}|pw}} Fremdartige: Myſticismus, Romantik, überhaupt die germaniſche Note. Der Director\pwindex{Lugné-Poe, Aurélien-Marie 27.\,12.\,1869 Paris – 19.\,6.\,1940 Villeneuve-les-Avignon@\textsc{Lugné-Poe, Aurélien-Marie} (27.\,12.\,1869 Paris – 19.\,6.\,1940 Villeneuve-les-Avignon), \emph{Theaterleiter, Regisseur, Schauspieler}|pwv} des »\textsc{Oeuvre\orgindex{Théâtre de l’Œuvre@Théâtre de l’Œuvre|pw}}« bereitet für die nächſte \textsc{Saison} zum Beiſpiel als
               beſondere Delikateſſe \textsc{Schillers\pwindex{Schiller, Friedrich von 10.\,11.\,1759 Marbach am Neckar – 9.\,5.\,1805 Weimar@\textsc{Schiller, Friedrich von} (10.\,11.\,1759 Marbach am Neckar – 9.\,5.\,1805 Weimar), \emph{Schriftsteller, Historiker, Philosoph}|pw}} »Räuber\pwindex{Schiller, Friedrich von 10.\,11.\,1759 Marbach am Neckar – 9.\,5.\,1805 Weimar@\textsc{Schiller, Friedrich von} (10.\,11.\,1759 Marbach am Neckar – 9.\,5.\,1805 Weimar), \emph{Schriftsteller, Historiker, Philosoph}!Räuber. Ein Schauspiel@\strich\emph{Die Räuber. Ein Schauspiel}|pw}« vor. Kurzum, die
               Aufführungs-Chancen{ }ſtehen nicht gut für Dich. Ich habe mir bereits ebenſo redlich
               als vergeblich Mühe gegeben. Trotzdem gebe ichs nicht auf; eine {\pb}Möglichkeit kann{ }ſich immer noch bieten. Vielleicht
               gelingt es, für die \label{K_L02617-1v}\edtext{»Wien\oindex{Wien@\textbf{Wien}, \emph{Verwaltungsgebiet}|pw}er Schule«}{\lemma{\textnormal{\emph{»Wiener Schule«}}}\Cendnote{\textnormal{Die Verwendung
                  des Ausdrucks ›Wien\oindex{Wien@\textbf{Wien}, \emph{Verwaltungsgebiet}|pwk}er Schule‹ kann als Hinweis gelesen werden, dass es noch keinen etablierten Begriff für die
                  neuere Literaturströmung gab, die dann später, mit propagandistischem Zutun von
                     Hermann Bahr\pwindex{Bahr, Hermann 19.\,7.\,1863 Linz – 15.\,1.\,1934 München@\textsc{Bahr, Hermann} (19.\,7.\,1863 Linz – 15.\,1.\,1934 München), \emph{Schriftsteller, Kritiker}|pwk}, als »Jung-Wien\oindex{Wien@\textbf{Wien}, \emph{Verwaltungsgebiet}|pwk}« in die Literaturgeschichte einging. (Der Begriff
                     »Jung-Wien\oindex{Wien@\textbf{Wien}, \emph{Verwaltungsgebiet}|pwk}« war zu dem Zeitpunkt bereits in
                  Verwendung, vgl. XXXX Auszeichnungsfehler: Dokument L02663 nicht gefunden, vgl. A. S.: \emph{Tagebuch}, 17. 3. 1890 und den
                  gleichnamigen Verein\orgindex{Jung Wien@Jung Wien|pwkv}, der
                  sich zumindest zwischen 17. 3. 1891 und 5. 5. 1891 wöchentlich traf.)}}}\label{K_L02617-1} in den \textsc{Revuen} Skandal zu machen,{ }ſo daß man dann auch nach ihrem
               Theater verlangt. Auch ein in Deutschland\oindex{Deutschland@\textbf{Deutschland}|pw}
               davongetragener großer Erfolg würde Dir{ }ſehr für \textsc{Paris\oindex{Paris@\textbf{Paris}, \emph{Hauptstadt}|pw}} zu Statten kommen \textsc{etc}. Alles Dich betreffende
               Literariſche will Dir übrigens \label{K_L02617-2v}\edtext{\textsc{Albert\pwindex{Albert, Henri 16.\,11.\,1869 Niederbronn-les-Bains – 3.\,8.\,1921 Straßburg@\textsc{Albert, Henri} (16.\,11.\,1869 Niederbronn-les-Bains – 3.\,8.\,1921 Straßburg), \emph{Journalist, Kritiker, Übersetzer}|pw}} direct{ }ſchreiben}{\lemma{\textnormal{\emph{Albert direct schreiben}}}\Cendnote{\textnormal{Das verzögerte
                  sich, Alberts\pwindex{Albert, Henri 16.\,11.\,1869 Niederbronn-les-Bains – 3.\,8.\,1921 Straßburg@\textsc{Albert, Henri} (16.\,11.\,1869 Niederbronn-les-Bains – 3.\,8.\,1921 Straßburg), \emph{Journalist, Kritiker, Übersetzer}|pwk} Brief ist mit
                     23. 5. 1894 datiert. Das Projekt einer Aufführung wird in einem
                  Satz abgehandelt: »Für das ›Abschiedsouper\pwindex{Schnitzler, Arthur 15.\,5.\,1862 Wien – 21.\,10.\,1931 ebd.@\textsc{Schnitzler, Arthur} (15.\,5.\,1862 Wien – 21.\,10.\,1931 ebd.), \emph{Schriftsteller, Mediziner}!Abschiedssouper@\strich\emph{Abschiedssouper}|pw}‹ denke ich einen Versuch an einer hiesigen Freien Bühne
                     zu machen« (\emph{DLA}, HS.1985.1.2331,2).}}}\label{K_L02617-2}.\pend
           
\pstart
           Deine große Productivität, über die \substVorne{}\textsuperscript{\textcolor{gray}{D}i\textcolor{gray}{r}}\substDazwischen{}mir\substHinten{} Deine Briefe berichten, freut mich von Herzen. Ich möchte gern bei
               Gelegenheit etwas von Deinen \label{K_L02617-3v}\edtext{neuen
                  Stücken}{\lemma{\textnormal{\emph{neuen
                  Stücken}}}\Cendnote{\textnormal{Am 29. 3. 1894 hatte Schnitzler eine zweite Fassung des später \emph{Liebelei}\pwindex{Schnitzler, Arthur 15.\,5.\,1862 Wien – 21.\,10.\,1931 ebd.@\textsc{Schnitzler, Arthur} (15.\,5.\,1862 Wien – 21.\,10.\,1931 ebd.), \emph{Schriftsteller, Mediziner}!Liebelei. Schauspiel in drei Akten@\strich\emph{Liebelei. Schauspiel in drei Akten}|pwk} genannten Stücks\pwindex{Schnitzler, Arthur 15.\,5.\,1862 Wien – 21.\,10.\,1931 ebd.@\textsc{Schnitzler, Arthur} (15.\,5.\,1862 Wien – 21.\,10.\,1931 ebd.), \emph{Schriftsteller, Mediziner}!Liebelei. Schauspiel in drei Akten@\strich\emph{Liebelei. Schauspiel in drei Akten}|pwkv} beendet. Am 14. 6. 1894 begann er eine dritte Fassung.
                  Ein nur als späteres Typoskript überlieferter Text ist zeitlich dazwischen
                  angesiedelt, was belegt, dass Schnitzler
                     weiter daran arbeitete. (A. S.: \emph{Liebelei}\pwindex{Schnitzler, Arthur 15.\,5.\,1862 Wien – 21.\,10.\,1931 ebd.@\textsc{Schnitzler, Arthur} (15.\,5.\,1862 Wien – 21.\,10.\,1931 ebd.), \emph{Schriftsteller, Mediziner}!Liebelei. Schauspiel in drei Akten@\strich\emph{Liebelei. Schauspiel in drei Akten}|pwk}. Historisch-kritische Ausgabe. Herausgegeben von  Peter Michael Braunwarth,
                     Gerhard Hubmann und Isabella Schwentner. Berlin, Boston: \emph{de
                        Gruyter}{ }2014 (Werke in historisch-kritischen Ausgaben, herausgegeben von  Konstanze
                     Fliedl), S. 5.) Ansonsten beschäftigte sich Schnitzler in diesen Tagen laut seinem \emph{Tagebuch}\pwindex{Schnitzler, Arthur 15.\,5.\,1862 Wien – 21.\,10.\,1931 ebd.@\textsc{Schnitzler, Arthur} (15.\,5.\,1862 Wien – 21.\,10.\,1931 ebd.), \emph{Schriftsteller, Mediziner}!Tagebuch@\strich\emph{Tagebuch}|pwk} vor allem mit Prosawerken: \emph{Sterben}\pwindex{Schnitzler, Arthur 15.\,5.\,1862 Wien – 21.\,10.\,1931 ebd.@\textsc{Schnitzler, Arthur} (15.\,5.\,1862 Wien – 21.\,10.\,1931 ebd.), \emph{Schriftsteller, Mediziner}!Sterben. Novelle@\strich\emph{Sterben. Novelle}|pwk}, \emph{Geschichte vom
                     greisen Dichter}\pwindex{Schnitzler, Arthur 15.\,5.\,1862 Wien – 21.\,10.\,1931 ebd.@\textsc{Schnitzler, Arthur} (15.\,5.\,1862 Wien – 21.\,10.\,1931 ebd.), \emph{Schriftsteller, Mediziner}!Später Ruhm@\strich\emph{Später Ruhm}|pwk} (\emph{Später Ruhm}\pwindex{Schnitzler, Arthur 15.\,5.\,1862 Wien – 21.\,10.\,1931 ebd.@\textsc{Schnitzler, Arthur} (15.\,5.\,1862 Wien – 21.\,10.\,1931 ebd.), \emph{Schriftsteller, Mediziner}!Später Ruhm@\strich\emph{Später Ruhm}|pwk}) und \emph{Die kleine Komödie}\pwindex{Schnitzler, Arthur 15.\,5.\,1862 Wien – 21.\,10.\,1931 ebd.@\textsc{Schnitzler, Arthur} (15.\,5.\,1862 Wien – 21.\,10.\,1931 ebd.), \emph{Schriftsteller, Mediziner}!kleine Komödie@\strich\emph{Die kleine Komödie}|pwk}.}}}\label{K_L02617-3} hören. Daß Du \strikeout{V\textcolor{gray}{e}} »verdichteſt«, iſt gewiß recht. Ich werde ein {\pb}immer überzeugterer Anhänger von Kürze und Einfachheit.\pend
           
\pstart
           Was Du mir über \substVorne{}\textsuperscript{\textcolor{gray}{Deine}}\substDazwischen{}meine\substHinten{} letzte \label{K_L02617-4v}\edtext{Arbeit\pwindex{Goldmann, Paul 31.\,1.\,1865 Breslau – 25.\,9.\,1935 Wien@\textsc{Goldmann, Paul} (31.\,1.\,1865 Breslau – 25.\,9.\,1935 Wien), \emph{Schriftsteller, Journalist}!Charles Meunier@\strich\emph{Charles Meunier}|pwuv}}{\lemma{\textnormal{\emph{Arbeit}}}\Cendnote{\textnormal{Wohl Paul Goldmann\pwindex{Goldmann, Paul 31.\,1.\,1865 Breslau – 25.\,9.\,1935 Wien@\textsc{Goldmann, Paul} (31.\,1.\,1865 Breslau – 25.\,9.\,1935 Wien), \emph{Schriftsteller, Journalist}|pwk}: \emph{Charles Meunier. Ein Jugendleben}\pwindex{Goldmann, Paul 31.\,1.\,1865 Breslau – 25.\,9.\,1935 Wien@\textsc{Goldmann, Paul} (31.\,1.\,1865 Breslau – 25.\,9.\,1935 Wien), \emph{Schriftsteller, Journalist}!Charles Meunier@\strich\emph{Charles Meunier}|pwk}. In: \emph{Frankfurter Zeitung}\pwindex{Frankfurter Zeitung@\emph{Frankfurter Zeitung}|pwk}, Jg. 38, Nr. 90, 1. 4. 1894, Erstes Morgenblatt, S. 1–2. Siehe XXXX Auszeichnungsfehler: Dokument L02615 nicht gefunden. }}}\label{K_L02617-4}{ }ſchreibſt,
               iſt eitel Güte und Freundſchaft. Aber außer Dir und{ }ſonſt noch ein paar lieben Leuten
               habe ich kein Publikum. Meine Erfolge{ }ſind rein moraliſcher Natur, – kein materielles
               Vorwärtskommen. Meine Laufbahn iſt auf ihrem Gipfel angelangt – der niedrig genug iſt
               – und jetzt gibt es nur ein Hinunterſteigen.\pend
           
\pstart
           {\pb}Mein Schwager\pwindex{Rosengart, Josef 8.\,2.\,1860 Laupheim – 4.\,8.\,1927 Frankfurt am Main@\textsc{Rosengart, Josef} (8.\,2.\,1860 Laupheim – 4.\,8.\,1927 Frankfurt am Main), \emph{Arzt}|pwv} meint, einer der Hauptgründe des mangelnden
               Heilerfolges{ }ſei der Umſtand, daß mir die geiſtige Ruhe während der Kur gefehlt hat.
               Es iſt etwas Richtiges daran. Wenn ich nicht geſund werde und nimmer geſund werden
               kann,{ }ſo liegt das auch an dem anſtregenden Berufe. Darum{ }ſoll ich wenigſtens auf 4
               Wochen nach Frankfurt\oindex{Frankfurt am Main@\textbf{Frankfurt am Main}, \emph{Hauptstadt}|pw}\substVorne{}\textsuperscript{.}\substDazwischen{},\substHinten{} um in Ruhe behandelt werden zu können. Freilich war es den ganzen Winter
               lang mein Traum, im Herbſt mit Dir zu reiſen. Nun muß ich darauf verzichten. Das thut
               mir in der Seele {\pb}weh. Aber es war{ }ſo{ }ſelbſtverſtändlich, daß ich auf dieſen Wunſch, weil er mir gar{ }ſo lieb war, würde
               verzichten müſſen.\pend
           
\pstart
           Grüß’ Dich Gott, mein lieber Freund! Sei recht froh! Und{ }ſchreib’ mir bald!\pend
           
\pstart
           In Treue {\\[\baselineskip]}Dein {\\[\baselineskip]}\spacefill\mbox{Paul Goldmann.}\pend
           \leftskip=0em{}\selectlanguage{ngerman}\endnumbering\briefempfaengerindex{Schnitzler, Arthur@\textsc{Schnitzler, Arthur}!zzzGoldmann, Paul@\emph{von Paul Goldmann}!1894-04-212@{21. 4. [1894]}|)be}\mylabel{L02617h}  \newcommand{\dateiname}{L02617}\newcommand{\titel}{Paul Goldmann an Arthur Schnitzler, 21. 4. [1894]}\newcommand{\editorInnen}{Martin Anton Müller und Laura Untner}%% latex-leseansicht-abspann.tex
%% Abspann für die Leseansicht.
%% Der Schalter \ifkorrekturansicht ist bereits durch den Vorspann gesetzt.

%% latex-abspann.tex
%% Gemeinsamer Abspann für Korrekturansicht und Leseansicht.
%% Setzt den Schalter \ifkorrekturansicht voraus (gesetzt in den
%% einbindenden Dateien latex-korrekturansicht-abspann.tex bzw.
%% latex-leseansicht-abspann.tex).
%% ---------------------------------------------------------------

\normalsize

% Das esempio-Environment wird nur in der Leseansicht benötigt
\ifkorrekturansicht\else
\newenvironment{esempio}[3]%
{
    \vspace{1.5ex}
    \rlap{\underline{#1}}
    \par
    \setlength{\parindent}{0cm}
    \nopagebreak
    \leftskip=#2cm
    \rightskip=#3cm
}
{
    \par
}
\fi

\doendnotes{C}
\bigskip
\vfill

\clearpage

\footnotesize

\ifkorrekturansicht
  \lohead{\textsc{register}}
\fi

% theindex-Environment neu definieren ohne reledmac
\makeatletter
\renewenvironment{theindex}{%
  \ifkorrekturansicht
    \section*{\indexname}%
  \else
    \subsubsection*{Index der erwähnten Entitäten}%
  \fi
  \setlength{\parindent}{0pt}%
  \setlength{\parskip}{0pt plus 0.3pt}%
  \let\item\@idxitem
}{%
  \ifkorrekturansicht\clearpage\fi
}
\makeatother

\IfFileExists{\jobname-pw.ind}{\input{\jobname-pw.ind}}{}

% Quellenangabe nur in der Leseansicht
\ifkorrekturansicht\else
% Fallback-Definitionen, falls die .tex-Datei \titel etc. nicht gesetzt hat
\providecommand{\titel}{}
\providecommand{\editorInnen}{}
\providecommand{\dateiname}{\jobname}

\vspace{3cm}

\vfill

\footnotesize
\textsc{Quelle}: \titel. Herausgegeben von {\editorInnen}. In: \emph{Arthur Schnitzler: Briefwechsel mit Autorinnen und Autoren}.
 Digitale Edition, https://schnitzler-briefe.acdh.oeaw.ac.at/{\dateiname}.html (Stand \today)
\fi

\end{document}


