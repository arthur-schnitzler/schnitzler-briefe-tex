%% latex-korrekturansicht-vorspann.tex
%% Vorspann für die Korrekturansicht.
%% Lädt die gemeinsame Datei latex-vorspann.tex mit gesetztem Schalter.

\newif\ifkorrekturansicht
\korrekturansichttrue

\input{../tex-inputs/latex-vorspann}


\section[ Paul Goldmann an Arthur Schnitzler, 21. 4. {[}1894{]}]{L02617 Paul Goldmann an Arthur Schnitzler, 21. 4. {[}1894{]}}
\nopagebreak\mylabel{L02617v}
\rehead{ }\normalsize\beginnumbering\briefempfaengerindex{Schnitzler, Arthur@\textsc{Schnitzler, Arthur}!zzzGoldmann, Paul@\emph{von Paul Goldmann}!1894-04-212@{21. 4. {[}1894{]}}|(be}
\toendnotes[C]{\smallbreak\pagebreak[2]}\Standort{DLA, A:Schnitzler, HS.NZ85.1.3164.}
\physDesc{Brief, 2 Blätter, 7 Seiten, 3150 Zeichen
\newline{}Handschrift: schwarze Tinte, deutsche Kurrent
\newline{}Schnitzler: 1) mit Bleistift auf dem ersten Blatt die Jahreszahl »94« vermerkt  2) mit rotem Buntstift zwei Unterstreichungen}\toendnotes[C]{\smallbreak}
\pstart
           {\pb}\textcolor{gray}{\textbf{Frankfurter Zeitung\orgindex{Frankfurter Zeitung@Frankfurter Zeitung|pw}.}}\pend
           
\pstart
           \textcolor{gray}{\textbf{(Gazette de
                     Francfort\orgindex{Frankfurter Zeitung@Frankfurter Zeitung|pw}.)}}\pend
           
\pstart
           \textcolor{gray}{\textbf{Directeur \textbf{M. L. Sonnemann\pwindex{Sonnemann, Leopold 1831-10-29 – 1909-10-30@\textsc{Sonnemann, Leopold} (1831-10-29 – 1909-10-30), \emph{Journalist/Journalistin, Herausgeber/Herausgeberin}|pw}}.}}\hfill \textsc{Paris\oindex{Paris@\textbf{Paris}, \emph{P.PPLC}|pw}}, \textcolor{gray}{2}1. April.\pend
           
\pstart
           \textcolor{gray}{\textbf{\begin{otherlanguage}{french}Journal politique, financier,\end{otherlanguage}}}\pend
           
\pstart
           \textcolor{gray}{\textbf{\begin{otherlanguage}{french}commercial et litteraire.\end{otherlanguage}}}\pend
           
\pstart
           \textcolor{gray}{\textbf{\begin{otherlanguage}{french}\textbf{Paraissant trois fois par jour}\end{otherlanguage}}}\pend
           
\pstart
           \textcolor{gray}{\textbf{\begin{otherlanguage}{french}\textbf{Bureaux à Paris\oindex{Paris@\textbf{Paris}, \emph{P.PPLC}|pw}:}\end{otherlanguage}}}\pend
           
\pstart
           \textcolor{gray}{\textbf{\begin{otherlanguage}{french}rue Richelieu 75\oindex{rue Richelieu@\textbf{rue Richelieu}, \emph{Straße (K.STR)}|pw}.\end{otherlanguage}}}\pend
           
\pstart\center{}Mein lieber Arthur,\pend\vspace{0.5em}
\pstart
           Von morgen ab wechſele ich meine Adreſſe, die fortan
               lautet: \textsc{\uline{24. Rue Feydeau}\oindex{rue Feydeau@\textbf{rue Feydeau}, \emph{Straße (K.STR)}|pw}}.\pend
           
\pstart
           Ich verzichte darauf, Dir \strikeout{zu ſ\textcolor{gray}{a}} jedes mal zu ſagen, eine wie große Freude Du mir ſtets mit Deinen lieben
               Briefen machſt. Du ahnſt nicht, wie wohl mir Deine treue Freundſchaft thut. Ein
               Feſttag in meinem armen Leben. Und ich bin Dir ſo von Herzen dankbar.\pend
           
\pstart
           Ich habe mich ſehr {\pb}gefreut, daß Du mir die
               Bekanntſchaft mit Fräulein \textsc{Sandrock\pwindex{Sandrock, Adele 1863-08-19 – 1937-08-30@\textsc{Sandrock, Adele} (1863-08-19 – 1937-08-30), \emph{Schauspieler/Schauspielerin}|pw}} vermittelt, und ich danke Dir ſehr für dieſe neue intereſſante Beziehung.\pend
           
\pstart
           \textsc{Albert\pwindex{Albert, Henri 1869-11-16 – 1921-08-03@\textsc{Albert, Henri} (1869-11-16 – 1921-08-03), \emph{Journalist/Journalistin, Kritiker/Kritikerin, Übersetzer/Übersetzerin}|pw}} habe ich einige Tage lang nicht geſehen. Ich glaube, er wird ſich nun bald an
               Deine Überſetzung\pwindex{Emplettes de Noel@\emph{Les Emplettes de Noël}|pwv} machen. Auch
               die Frage der Aufführung an einem hieſigen Theater haben wir oft erörtert. Wir
               verkennen aber Beide\pwindex{Albert, Henri 1869-11-16 – 1921-08-03@\textsc{Albert, Henri} (1869-11-16 – 1921-08-03), \emph{Journalist/Journalistin, Kritiker/Kritikerin, Übersetzer/Übersetzerin}|pwv} nicht
               die Schwierigkeiten. Fremde Stücke führen hier überhaupt nur die freien Bühnen\orgindex{Theâtre Libre@Théâtre Libre|pwv}\orgindex{Theâtre de l Œuvre@Théâtre de l’Œuvre|pwv} auf, alſo »\textsc{Théâtre Libre\orgindex{Theâtre Libre@Théâtre Libre|pw}}« und »\textsc{Oeuvre\orgindex{Theâtre de l Œuvre@Théâtre de l’Œuvre|pw}}«. Während Du alſo bei den übrigen Theatern kaum {\pb}ankommen könnteſt, weil Du ein deutſcher Dichter
               biſt, ſo ſteht Dir bei den beiden letz{[}t{]}genannten der Umſtand
               entgegen, daß Du in Geiſt und Sprache zu fein und zu franz\oindex{Frankreich@\textbf{Frankreich}, \emph{A.PCLI}|pwv}öſiſch biſt. Die Freien Bühnen\orgindex{Theâtre Libre@Théâtre Libre|pwv}\orgindex{Theâtre de l Œuvre@Théâtre de l’Œuvre|pwv} ſuchen in den
               deutſchen Stücken das für \textsc{Paris\oindex{Paris@\textbf{Paris}, \emph{P.PPLC}|pw}} Fremdartige: Myſticismus, Romantik, überhaupt die germaniſche Note. Der Director\pwindex{Lugne-Poe, Aurelien-Marie 1869-12-27 – 1940-06-19@\textsc{Lugné-Poe, Aurélien-Marie} (1869-12-27 – 1940-06-19), \emph{Theaterleiter/Theaterleiterin, Regisseur/Regisseurin, Schauspieler/Schauspielerin}|pwv} des »\textsc{Oeuvre\orgindex{Theâtre de l Œuvre@Théâtre de l’Œuvre|pw}}« bereitet für die nächſte \textsc{Saison} zum Beiſpiel als
               beſondere Delikateſſe \textsc{Schillers\pwindex{Schiller, Friedrich von 10.11.1759 – 09.05.1805@\textsc{Schiller, Friedrich von} (10.11.1759 – 09.05.1805), \emph{Schriftsteller/Schriftstellerin, Historiker/Historikerin, Philosoph/Philosophin}|pw}} »Räuber\pwindex{Raeuber. Ein Schauspiel@\emph{Die Räuber. Ein Schauspiel}|pw}« vor. Kurzum, die
               Aufführungs-Chancen ſtehen nicht gut für Dich. Ich habe mir bereits ebenſo redlich
               als vergeblich Mühe gegeben. Trotzdem gebe ichs nicht auf; eine {\pb}Möglichkeit kann ſich immer noch bieten. Vielleicht
               gelingt es, für die \label{K_L02617-1v}\edtext{»Wien\oindex{Wien@\textbf{Wien}, \emph{A.ADM2}|pw}er Schule«}{\lemma{\textnormal{\emph{»Wiener Schule«}}}\Cendnote{\textnormal{Die Verwendung
                  des Ausdrucks ›Wien\oindex{Wien@\textbf{Wien}, \emph{A.ADM2}|pwk}er Schule‹ kann als Hinweis gelesen werden, dass es noch keinen etablierten Begriff für die
                  neuere Literaturströmung gab, die dann später, mit propagandistischem Zutun von
                     Hermann Bahr\pwindex{Bahr, Hermann 19.07.1863 – 15.01.1934@\textsc{Bahr, Hermann} (19.07.1863 – 15.01.1934), \emph{Schriftsteller/Schriftstellerin, Kritiker/Kritikerin}|pwk}, als »Jung-Wien\oindex{Wien@\textbf{Wien}, \emph{A.ADM2}|pwk}« in die Literaturgeschichte einging. (Der Begriff
                     »Jung-Wien\oindex{Wien@\textbf{Wien}, \emph{A.ADM2}|pwk}« war zu dem Zeitpunkt bereits in
                  Verwendung, vgl. Paul Goldmann an Arthur Schnitzler, 16. 5. 1891, vgl. A. S.: \emph{Tagebuch}, 17. 3. 1890 und den
                  gleichnamigen Verein\orgindex{Jung Wien@Jung Wien|pwkv}, der
                  sich zumindest zwischen 17. 3. 1891 und 5. 5. 1891 wöchentlich traf.)}}}\label{K_L02617-1} in den \textsc{Revuen} Skandal zu machen, ſo daß man dann auch nach ihrem
               Theater verlangt. Auch ein in Deutschland\oindex{Deutschland@\textbf{Deutschland}, \emph{A.PCLI}|pw}
               davongetragener großer Erfolg würde Dir ſehr für \textsc{Paris\oindex{Paris@\textbf{Paris}, \emph{P.PPLC}|pw}} zu Statten kommen \textsc{etc}. Alles Dich betreffende
               Literariſche will Dir übrigens \label{K_L02617-2v}\edtext{\textsc{Albert\pwindex{Albert, Henri 1869-11-16 – 1921-08-03@\textsc{Albert, Henri} (1869-11-16 – 1921-08-03), \emph{Journalist/Journalistin, Kritiker/Kritikerin, Übersetzer/Übersetzerin}|pw}} direct ſchreiben}{\lemma{\textnormal{\emph{Albert direct ſchreiben}}}\Cendnote{\textnormal{Das verzögerte
                  sich, Alberts\pwindex{Albert, Henri 1869-11-16 – 1921-08-03@\textsc{Albert, Henri} (1869-11-16 – 1921-08-03), \emph{Journalist/Journalistin, Kritiker/Kritikerin, Übersetzer/Übersetzerin}|pwk} Brief ist mit
                     23. 5. 1894 datiert. Das Projekt einer Aufführung wird in einem
                  Satz abgehandelt: »Für das ›Abschiedsouper\pwindex{Abschiedssouper@\emph{Abschiedssouper}|pw}‹ denke ich einen Versuch an einer hiesigen Freien Bühne
                     zu machen« (\emph{DLA}, HS.1985.1.2331,2).}}}\label{K_L02617-2}.\pend
           
\pstart
           Deine große Productivität, über die \substVorne{}\textsuperscript{\textcolor{gray}{D}i\textcolor{gray}{r}}\substDazwischen{}mir\substHinten{} Deine Briefe berichten, freut mich von Herzen. Ich möchte gern bei
               Gelegenheit etwas von Deinen \label{K_L02617-3v}\edtext{neuen
                  Stücken}{\lemma{\textnormal{\emph{neuen
                  Stücken}}}\Cendnote{\textnormal{Am 29. 3. 1894 hatte Schnitzler eine zweite Fassung des später \emph{Liebelei}\pwindex{Liebelei. Schauspiel in drei Akten@\emph{Liebelei. Schauspiel in drei Akten}|pwk} genannten Stücks\pwindex{Liebelei. Schauspiel in drei Akten@\emph{Liebelei. Schauspiel in drei Akten}|pwkv} beendet. Am 14. 6. 1894 begann er eine dritte Fassung.
                  Ein nur als späteres Typoskript überlieferter Text ist zeitlich dazwischen
                  angesiedelt, was belegt, dass Schnitzler
                     weiter daran arbeitete. (A. S.: \emph{Liebelei}\pwindex{Liebelei. Schauspiel in drei Akten@\emph{Liebelei. Schauspiel in drei Akten}|pwk}. Historisch-kritische Ausgabe. Herausgegeben von  Peter Michael Braunwarth,
                     Gerhard Hubmann und Isabella Schwentner. Berlin, Boston: \emph{de
                        Gruyter}{ }2014 (Werke in historisch-kritischen Ausgaben, herausgegeben von  Konstanze
                     Fliedl), S. 5.) Ansonsten beschäftigte sich Schnitzler in diesen Tagen laut seinem \emph{Tagebuch}\pwindex{Tagebuch@\emph{Tagebuch}|pwk} vor allem mit Prosawerken: \emph{Sterben}\pwindex{Sterben. Novelle@\emph{Sterben. Novelle}|pwk}, \emph{Geschichte vom
                     greisen Dichter}\pwindex{Spaeter Ruhm@\emph{Später Ruhm}|pwk} (\emph{Später Ruhm}\pwindex{Spaeter Ruhm@\emph{Später Ruhm}|pwk}) und \emph{Die kleine Komödie}\pwindex{kleine Komoedie@\emph{Die kleine Komödie}|pwk}.}}}\label{K_L02617-3} hören. Daß Du \strikeout{V\textcolor{gray}{e}} »verdichteſt«, iſt gewiß recht. Ich werde ein {\pb}immer überzeugterer Anhänger von Kürze und Einfachheit.\pend
           
\pstart
           Was Du mir über \substVorne{}\textsuperscript{\textcolor{gray}{Deine}}\substDazwischen{}meine\substHinten{} letzte \label{K_L02617-4v}\edtext{Arbeit\pwindex{Charles Meunier@\emph{Charles Meunier}|pwuv}}{\lemma{\textnormal{\emph{Arbeit}}}\Cendnote{\textnormal{Wohl Paul Goldmann\pwindex{Goldmann, Paul 31.01.1865 – 25.09.1935@\textsc{Goldmann, Paul} (31.01.1865 – 25.09.1935), \emph{Schriftsteller/Schriftstellerin, Journalist/Journalistin}|pwk}: \emph{Charles Meunier. Ein Jugendleben}\pwindex{Charles Meunier@\emph{Charles Meunier}|pwk}. In: \emph{Frankfurter Zeitung}\pwindex{Frankfurter Zeitung@\emph{Frankfurter Zeitung}|pwk}, Jg. 38, Nr. 90, 1. 4. 1894, Erstes Morgenblatt, S. 1–2. Siehe Paul Goldmann an Arthur Schnitzler, 3. 4. [1894]. }}}\label{K_L02617-4} ſchreibſt,
               iſt eitel Güte und Freundſchaft. Aber außer Dir und ſonſt noch ein paar lieben Leuten
               habe ich kein Publikum. Meine Erfolge ſind rein moraliſcher Natur, – kein materielles
               Vorwärtskommen. Meine Laufbahn iſt auf ihrem Gipfel angelangt – der niedrig genug iſt
               – und jetzt gibt es nur ein Hinunterſteigen.\pend
           
\pstart
           {\pb}Mein Schwager\pwindex{Rosengart, Josef 1860-02-08 – 1927-08-04@\textsc{Rosengart, Josef} (1860-02-08 – 1927-08-04), \emph{Arzt/Ärztin}|pwv} meint, einer der Hauptgründe des mangelnden
               Heilerfolges ſei der Umſtand, daß mir die geiſtige Ruhe während der Kur gefehlt hat.
               Es iſt etwas Richtiges daran. Wenn ich nicht geſund werde und nimmer geſund werden
               kann, ſo liegt das auch an dem anſtregenden Berufe. Darum ſoll ich wenigſtens auf 4
               Wochen nach Frankfurt\oindex{Frankfurt am Main@\textbf{Frankfurt am Main}, \emph{P.PPLA3}|pw}\substVorne{}\textsuperscript{.}\substDazwischen{},\substHinten{} um in Ruhe behandelt werden zu können. Freilich war es den ganzen Winter
               lang mein Traum, im Herbſt mit Dir zu reiſen. Nun muß ich darauf verzichten. Das thut
               mir in der Seele {\pb}weh. Aber es war ſo
               ſelbſtverſtändlich, daß ich auf dieſen Wunſch, weil er mir gar ſo lieb war, würde
               verzichten müſſen.\pend
           
\pstart
           Grüß’ Dich Gott, mein lieber Freund! Sei recht froh! Und ſchreib’ mir bald!\pend
           
\pstart
           In Treue {\\[\baselineskip]}Dein {\\[\baselineskip]}\spacefill\mbox{Paul Goldmann.}\pend
           \leftskip=0em{}\selectlanguage{ngerman}\endnumbering\briefempfaengerindex{Schnitzler, Arthur@\textsc{Schnitzler, Arthur}!zzzGoldmann, Paul@\emph{von Paul Goldmann}!1894-04-212@{21. 4. {[}1894{]}}|)be}\mylabel{L02617h}  \normalsize

\doendnotes{C}
\bigskip
\vfill

\clearpage

\footnotesize

\lohead{\textsc{register}}

% Definiere theindex-Environment komplett neu ohne reledmac
\makeatletter
\renewenvironment{theindex}{%
  \section*{\indexname}%
  \setlength{\parindent}{0pt}%
  \setlength{\parskip}{0pt plus 0.3pt}%
  \let\item\@idxitem
}{%
  \clearpage
}
\makeatother

\IfFileExists{\jobname-pw.ind}{\input{\jobname-pw.ind}}{}

\end{document}

      