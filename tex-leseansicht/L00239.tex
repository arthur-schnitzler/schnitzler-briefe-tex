%% latex-leseansicht-vorspann.tex
%% Vorspann für die Leseansicht.
%% Lädt die gemeinsame Datei latex-vorspann.tex mit nicht gesetztem Schalter.

\newif\ifkorrekturansicht
\korrekturansichtfalse

\input{../tex-inputs/latex-vorspann}


\section[Karl Kraus an Arthur Schnitzler, 21. 7. 1893]{L00239 Karl Kraus an Arthur Schnitzler, 21. 7. 1893}
\nopagebreak\mylabel{L00239v}
\rehead{ }\normalsize\beginnumbering\briefempfaengerindex{Schnitzler, Arthur@\textsc{Schnitzler, Arthur}!zzzKraus, Karl@\emph{von Karl Kraus}!1893-07-211@{21. 7. 1893}|(be}
\toendnotes[C]{\smallbreak\pagebreak[2]}
\correspDesc{Versand  durch Karl Kraus am 21. 7. 1893 in Bad Ischl
\newline{}Erhalt  durch Arthur Schnitzler im Zeitraum [22. 7. 1893 – 26. 7. 1893?] in Wien}\toendnotes[C]{\smallbreak}
\Standort{CUL, Schnitzler, B 55.}
\physDesc{Brief, 1 Blatt, 4 Seiten, 3571 Zeichen
\newline{}Handschrift: schwarze Tinte, deutsche Kurrent
\newline{}Beilage: Manuskript auf dem gleichen Briefpapier, 1 Blatt,
                                            1 Seite, schwarze Tinte }
\buchAbdrucke{\weitereDrucke{\emph{Karl Kraus und Arthur Schnitzler. Eine Dokumentation.}Herausgegeben von Reinhard Urbach In: \emph{Literatur und Kritik}, Bd. 49, Oktober 1970, S. 518–519.} }\toendnotes[C]{\smallbreak}\pstart{}{\pb}Schnitzler\pend{}{\bigskip}\vspace{1em}
\pstart
           {\pb}\textcolor{gray}{\textbf{KARL KRAUS}}\hfill \substVorne{}\textsuperscript{\textcolor{gray}{\textbf{Wien I., Maximilianstrasse
                                            13\oindex{Wien@\textbf{Wien}!I., Innere Stadt@\textbf{I., Innere Stadt}!Mahlerstraße@\textbf{Mahlerstraße}, \emph{Straße}|pw}.}}}\substDazwischen{}Ischl\oindex{Bad Ischl@\textbf{Bad Ischl}|pw}\substHinten{}{ }21. Juli, \textcolor{gray}{\textbf{189}}3\pend
           
\pstart{}Mein liebſter, verehrter Herr Doctor!\pend\vspace{0.5em}
\pstart
           Daſs Sie{ }ſo »ſpurlos«{ }ſich auch dem Staube gemacht haben, thut mir{ }ſehr leid.
                    Seit Ihrer Vorſtellung haben wir uns ja gar nicht gesprochen.\pend
           
\pstart
           »Sieh’ſt du, \uline{das}{ }\uuline{\edtext{hätt}{\Cendnote{siebenfach unterstrichen}}}’ (!!!!) ich dir \introOben{}doch\introOben{} nicht geſagt!« – ich werde dieſen genialen Zug in Frl. Falkner\pwindex{Falkner, Julie 26.\,7.\,1867 Telč – 13.\,4.\,1937 Gleisdorf@\textsc{Falkner, Julie} (26.\,7.\,1867 Telč – 13.\,4.\,1937 Gleisdorf), \emph{Schauspielerin}|pw}’s Darſtellung nie vergeſſen. Und
                    darauf noch dröhnender Abgangsapplaus, der \strikeout{d}
                    auch die \uline{zweite}{ }Schluſspointe (»Es iſt ja leicht gegangen etc\pwindex{Schnitzler, Arthur 15.\,5.\,1862 Wien – 21.\,10.\,1931 ebd.@\textsc{Schnitzler, Arthur} (15.\,5.\,1862 Wien – 21.\,10.\,1931 ebd.), \emph{Schriftsteller, Mediziner}!Abschiedssouper@\strich\emph{Abschiedssouper}|pwv}«)
                    unmöglich machte! Von dem »Bordellſtück« »Abſchiedsouper\pwindex{Schnitzler, Arthur 15.\,5.\,1862 Wien – 21.\,10.\,1931 ebd.@\textsc{Schnitzler, Arthur} (15.\,5.\,1862 Wien – 21.\,10.\,1931 ebd.), \emph{Schriftsteller, Mediziner}!Abschiedssouper@\strich\emph{Abschiedssouper}|pw}« wird hier viel geſprochen.\pend
           
\pstart
           Meine herzlichſte Gratulation zur \label{K_L00239-1v}\edtext{Kritik\pwindex{Aus Ischl, 14. Juli, schreibt man uns: …@\emph{Aus Ischl, 14. Juli, schreibt man uns: …}|pwv}}{\lemma{\textnormal{\emph{Kritik}}}\Cendnote{\textnormal{[O. V.]: \emph{[Aus Ischl, 14. Juli,
                                schreibt man uns]}\pwindex{Aus Ischl, 14. Juli, schreibt man uns: …@\emph{Aus Ischl, 14. Juli, schreibt man uns: …}|pwk}. In: \emph{Neue
                                Freie Presse}\pwindex{Neue Freie Presse@\emph{Neue Freie Presse}|pwk}, Nr. 10.381, 18. 7. 1893,
                        S. 6.}}}\label{K_L00239-1} in N. Fr. Preſſe\pwindex{Neue Freie Presse@\emph{Neue Freie Presse}|pw}
                    (und \label{K_L00239-2v}\edtext{Bauer\pwindex{Bauer, Julius 15.\,10.\,1853 Szigetvár – 11.\,6.\,1941 Wien@\textsc{Bauer, Julius} (15.\,10.\,1853 Szigetvár – 11.\,6.\,1941 Wien), \emph{Schriftsteller, Journalist, Kritiker}|pw}\pwindex{Abschiedsouper in Ischl]@\emph{[Abschiedsouper in Ischl]}|pwv}}{\lemma{\textnormal{\emph{Bauer}}}\Cendnote{\textnormal{[Julius Bauer\pwindex{Bauer, Julius 15.\,10.\,1853 Szigetvár – 11.\,6.\,1941 Wien@\textsc{Bauer, Julius} (15.\,10.\,1853 Szigetvár – 11.\,6.\,1941 Wien), \emph{Schriftsteller, Journalist, Kritiker}|pwk}]: \emph{[Abschiedssouper in Ischl]}\pwindex{Abschiedsouper in Ischl]@\emph{[Abschiedsouper in Ischl]}|pwk}. In: \emph{Illustrirtes Wiener Extrablatt}\pwindex{Illustrirtes Wiener Extrablatt@\emph{Illustrirtes Wiener Extrablatt}|pwk},
                            Jg. 22, Nr. 196, 18. 7. 1893, S. 5.}}}\label{K_L00239-2} im Extrablatt\orgindex{Illustrirtes Wiener Extrablatt@Illustrirtes Wiener Extrablatt|pw})! Sehr dämlich hat{ }ſich Herr
                        \label{K_L00239-3v}\edtext{Skrein\pwindex{Skrein, Stefan 14.\,9.\,1856 Holešov – 22.\,9.\,1931 Wien@\textsc{Skrein, Stefan} (14.\,9.\,1856 Holešov – 22.\,9.\,1931 Wien), \emph{Schriftsteller, Journalist}|pw}}{\lemma{\textnormal{\emph{Skrein}}}\Cendnote{\textnormal{Stefan\pwindex{Skrein, Stefan 14.\,9.\,1856 Holešov – 22.\,9.\,1931 Wien@\textsc{Skrein, Stefan} (14.\,9.\,1856 Holešov – 22.\,9.\,1931 Wien), \emph{Schriftsteller, Journalist}|pwk}: \emph{Ischler Brief}\pwindex{Skrein, Stefan 14.\,9.\,1856 Holešov – 22.\,9.\,1931 Wien@\textsc{Skrein, Stefan} (14.\,9.\,1856 Holešov – 22.\,9.\,1931 Wien), \emph{Schriftsteller, Journalist}!Ischler Brief@\strich\emph{Ischler Brief}|pwk}. In: \emph{Wiener Allgemeine Zeitung}\pwindex{Wiener Allgemeine Zeitung@\emph{Wiener Allgemeine Zeitung}|pwk}, Jg. 14, Nr. 4593,
                                18. 7. 1893, S. 2.}}}\label{K_L00239-3} in der »Allgemeinen\orgindex{Wiener Allgemeine Zeitung@Wiener Allgemeine Zeitung|pw}« geäußert\pwindex{Skrein, Stefan 14.\,9.\,1856 Holešov – 22.\,9.\,1931 Wien@\textsc{Skrein, Stefan} (14.\,9.\,1856 Holešov – 22.\,9.\,1931 Wien), \emph{Schriftsteller, Journalist}!Ischler Brief@\strich\emph{Ischler Brief}|pwv}.\pend
           
\pstart
           Dies mal haben N. Fr. Pr.\orgindex{Neue Freie Presse@Neue Freie Presse|pw} u. Allgemeine\orgindex{Wiener Allgemeine Zeitung@Wiener Allgemeine Zeitung|pw} die Rollen getauſcht.\pend
           
\pstart
           {\pb}Ich habe eine Notiz\pwindex{Ischler Sommertheater@\emph{Ischler Sommertheater}|pwv} an das Wiener Tagblatt\orgindex{Wiener Tagblatt@Wiener Tagblatt|pw}{ }\label{K_L00239-4v}\edtext{geſchickt}{\lemma{\textnormal{\emph{geschickt}}}\Cendnote{\textnormal{[O. V.]: \emph{Ischler Sommertheater}\pwindex{Ischler Sommertheater@\emph{Ischler Sommertheater}|pwk}.
                            In: \emph{Wiener Abendblatt}\pwindex{Neues Wiener Tagblatt@\emph{Neues Wiener Tagblatt}|pwk}, Jg. 29,
                            Nr. 199, 21. 7. 1893, S. 4.}}}\label{K_L00239-4}; \uline{hoffentlich}{ }\uline{wird} (oder, wenn Sie dieſen Brief erhalten) \uline{wurde} es gedruckt. Im \uline{Magazin}\orgindex{Magazin für die Literatur des Auslandes@Magazin für die Literatur des Auslandes|pw} wird nichts erſcheinen. Allerdings bin ich nicht{ }ſchuld. Damit Sie meinen
                    guten Willen{ }ſehen,{ }ſende ich Ihnen beiliegend meine \substVorne{}\textsuperscript{Kritik}\substDazwischen{}Notiz\substHinten{}, die mir heute Neumann-Hofer\pwindex{Neumann-Hofer, Gilbert Otto 4.\,2.\,1857 Bol’shiye Berezhki – 14.\,4.\,1941 Detmold@\textsc{Neumann-Hofer, Gilbert Otto} (4.\,2.\,1857 Bol’shiye Berezhki – 14.\,4.\,1941 Detmold), \emph{Kritiker, Theaterleiter}|pw}
                    zurückſandte – mit der Bemerkung:\pend
           
\pstart
           »Eine Vorſtellung in Ischl\oindex{Bad Ischl@\textbf{Bad Ischl}|pw} kann in einem
                    Wochenblatte nicht beſprochen werden. Solche gelegentlichen Ereigniſſe{ }ſind auf
                    die Notiznahme{ }ſeitens der Tagesblätter beſchränkt.« Na, alſo! –\pend
           
\pstart
           Devrient\pwindex{Devrient, Max 12.\,12.\,1857 Hannover – 13.\,6.\,1929 Chur@\textsc{Devrient, Max} (12.\,12.\,1857 Hannover – 13.\,6.\,1929 Chur), \emph{Regisseur, Schauspieler}|pw}’s Vorleſung war famos: namentlich
                        Fontane\pwindex{Fontane, Theodor 30.\,12.\,1819 Neuruppin – 20.\,9.\,1898 Berlin@\textsc{Fontane, Theodor} (30.\,12.\,1819 Neuruppin – 20.\,9.\,1898 Berlin), \emph{Schriftsteller, Kritiker, Apotheker}|pw}.\pend
           
\pstart
           Ich habe ihm gleich nach unſerer{ }ſeinerzeit. Unterredung nach Wien\oindex{Wien@\textbf{Wien}, \emph{Verwaltungsgebiet}|pw} geſchrieben, er{ }ſolle {\pb}Liliencron\pwindex{Liliencron, Detlev von 3.\,6.\,1844 Kiel – 22.\,7.\,1909 Rahlstedt@\textsc{Liliencron, Detlev von} (3.\,6.\,1844 Kiel – 22.\,7.\,1909 Rahlstedt), \emph{Schriftsteller, Dichter, Dramatiker}|pw} leſen. Nun hat er mich –{ }ſelbſt aufgeſucht. Liebenswürdig, was? Wie gedruckt; Liliencron\pwindex{Liliencron, Detlev von 3.\,6.\,1844 Kiel – 22.\,7.\,1909 Rahlstedt@\textsc{Liliencron, Detlev von} (3.\,6.\,1844 Kiel – 22.\,7.\,1909 Rahlstedt), \emph{Schriftsteller, Dichter, Dramatiker}|pw}, den er{ }ſich gleich kaufte, hat ihn \uline{entzückt} u. er wird ihn beſtimmt in \uline{Wien}\oindex{Wien@\textbf{Wien}, \emph{Verwaltungsgebiet}|pw} vorleſen. Er fragte mich auch, ob ich Gedichte von \uuline{Ihnen} hätte; er wollte{ }ſie nämlich in Marienbad\oindex{Marienbad@\textbf{Marienbad}|pw}, wohin er{ }ſich noch am Tage des Beſuches begab,
                    vorleſen. Da nun aber die Vorleſung gleich auf den nächſten Tag angeſetzt war,
                    lehnte er auch eine eventuelles Telegramm an Sie (zu dem ich mich bereit
                    erklärte) ab. Aber im Winter will er’s nachholen.\pend
           
\pstart
           Leben Sie wohl, bitte beſte Grüße an Loris\pwindex{Hofmannsthal, Hugo von 1.\,2.\,1874 Wien – 15.\,7.\,1929 Rodaun@\textsc{Hofmannsthal, Hugo von} (1.\,2.\,1874 Wien – 15.\,7.\,1929 Rodaun), \emph{Schriftsteller}|pw}
                    u Salten\pwindex{Salten, Felix 6.\,9.\,1869 Budapest – 8.\,10.\,1945 Zürich@\textsc{Salten, Felix} (6.\,9.\,1869 Budapest – 8.\,10.\,1945 Zürich), \emph{Schriftsteller, Journalist, Chefredakteur}|pw} auszurichten!\pend
           
\pstart
           Herzlichſt Ihr{ }ſehr ergebener{\\[\baselineskip]}\spacefill\mbox{KarlKraus}\pend
           \leftskip=0em{}
\pstart
           \noindent{}N.B. Was{ }ſagen Sie zur »Freien Bühne\orgindex{»Freie Bühne« Verein für moderne Literatur@»Freie Bühne« Verein für moderne Literatur|pw}« in
                            Wien\oindex{Wien@\textbf{Wien}, \emph{Verwaltungsgebiet}|pw}, die – Elbogen\pwindex{Elbogen, Friedrich 20.\,5.\,1854 Prag – 15.\,4.\,1909 Wien@\textsc{Elbogen, Friedrich} (20.\,5.\,1854 Prag – 15.\,4.\,1909 Wien), \emph{Schriftsteller, Kritiker, Rechtsanwalt}|pw} aufführt. Ist das nicht zum Todtlachen? Die
                        Veranstalter{ }ſind Revolverjournalisten.\pend
           \selectlanguage{ngerman}\vspace{1em}
\pstart
           \noindent{}
\pstart
           {\pb}\textcolor{gray}{\textbf{KARL KRAUS}}\pend
           
\pstart
           \raggedleft{}\substVorne{}\textsuperscript{\textcolor{gray}{\textbf{Wien I., Maximilianstrasse
                                        13\oindex{Wien@\textbf{Wien}!I., Innere Stadt@\textbf{I., Innere Stadt}!Mahlerstraße@\textbf{Mahlerstraße}, \emph{Straße}|pw}.}}}\substDazwischen{}Ischl\oindex{Bad Ischl@\textbf{Bad Ischl}|pw}\substHinten{}{ }15. VII \textcolor{gray}{\textbf{189}}3\pend
           \pend
           
\pstart
           \uline{Arthur Schnitzlers} einaktige Komödie »Abſchiedssouper\pwindex{Schnitzler, Arthur 15.\,5.\,1862 Wien – 21.\,10.\,1931 ebd.@\textsc{Schnitzler, Arthur} (15.\,5.\,1862 Wien – 21.\,10.\,1931 ebd.), \emph{Schriftsteller, Mediziner}!Abschiedssouper@\strich\emph{Abschiedssouper}|pw}« fand im Ischler Stadttheater\oindex{Lehártheater@\textbf{Lehártheater}, \emph{Theater}|pw} ihre Probeaufführung. Das kleine oberöſterreichiſche\oindex{Oberösterreich@\textbf{Oberösterreich}, \emph{Land}|pw} Curorttheater iſt die
                    erſte Bühne, die{ }ſich des prächtigen Stückleins angenommen hat.\pend
           
\pstart
           Der überaus lebendige, geiſtreiche Einakter, der eine geradezu bravouröſe Technik
                    aufweist, iſt die wirkſamſte der{ }ſieben »Anatol\pwindex{Schnitzler, Arthur 15.\,5.\,1862 Wien – 21.\,10.\,1931 ebd.@\textsc{Schnitzler, Arthur} (15.\,5.\,1862 Wien – 21.\,10.\,1931 ebd.), \emph{Schriftsteller, Mediziner}!Anatol@\strich\emph{Anatol}|pw}«studien (siehe \label{K_L00239-5v}\edtext{Besprechung\pwindex{Kraus, Karl 28.\,4.\,1874 Jičín – 12.\,6.\,1936 Wien@\textsc{Kraus, Karl} (28.\,4.\,1874 Jičín – 12.\,6.\,1936 Wien), \emph{Schriftsteller, Publizist, Schriftsteller}!Wiener Dichter@\strich\emph{Wiener Dichter}|pwv}}{\lemma{\textnormal{\emph{Besprechung}}}\Cendnote{\textnormal{[O. V.]: \emph{Arthur Schnitzler}\pwindex{Kraus, Karl 28.\,4.\,1874 Jičín – 12.\,6.\,1936 Wien@\textsc{Kraus, Karl} (28.\,4.\,1874 Jičín – 12.\,6.\,1936 Wien), \emph{Schriftsteller, Publizist, Schriftsteller}!Wiener Dichter@\strich\emph{Wiener Dichter}|pwk}. In:
                                \emph{Das Magazin für Litteratur}\pwindex{Magazin für die Literatur des Auslandes@\emph{Magazin für die Literatur des Auslandes}|pwk},
                            Jg. 62, Nr. 18, 6. 5. 1893, S. 294.}}}\label{K_L00239-5} in N\textsuperscript{r.} 18) und fand den lebhafteſten Beifall, den nur
                    einige »verſchämte«, in ihren heiligſten Gefühlen verletzte Curgäſte im
                    Intereſſe der \substVorne{}\textsuperscript{publiken und privaten}\substDazwischen{}privaten und publiken\substHinten{}{ }Sicherheit abwehren zu müſſen glaubten.
                    Geſpielt wurde recht brav; namentlich zeichnete{ }ſich der treffliche \uline{Jarno}\pwindex{Jarno, Josef 24.\,8.\,1865 Budapest – 11.\,1.\,1932 Wien@\textsc{Jarno, Josef} (24.\,8.\,1865 Budapest – 11.\,1.\,1932 Wien), \emph{Theaterleiter, Schauspieler}|pw} vom berliner Reſidenztheater\oindex{Wallnertheater@\textbf{Wallnertheater}, \emph{Theater}|pw} als Max\pwindex{Schnitzler, Arthur 15.\,5.\,1862 Wien – 21.\,10.\,1931 ebd.@\textsc{Schnitzler, Arthur} (15.\,5.\,1862 Wien – 21.\,10.\,1931 ebd.), \emph{Schriftsteller, Mediziner}!Anatol@\strich\emph{Anatol}|pwv} aus. Die famoſe
                    Schluſspointe gieng leider wirkungslos, weil unverſtanden, vorüber. –\pend
           
\pstart
           Arthur Schnitzler, neben Loris\pwindex{Beer-Hofmann, Richard 11.\,7.\,1866 Wien – 26.\,9.\,1945 New York City@\textsc{Beer-Hofmann, Richard} (11.\,7.\,1866 Wien – 26.\,9.\,1945 New York City), \emph{Schriftsteller}|pw} der
                    talentvollſte unter den wenigen talentierten Wien\oindex{Wien@\textbf{Wien}, \emph{Verwaltungsgebiet}|pw}ern, \strikeout{muſste} hat an dieſem Abend
                    die Concurrenz – der Herren Moſer\pwindex{Moser, Gustav von 11.\,5.\,1825 Spandau – 23.\,10.\,1903 Görlitz@\textsc{Moser, Gustav von} (11.\,5.\,1825 Spandau – 23.\,10.\,1903 Görlitz), \emph{Schriftsteller}|pw}{ }{\kaufmannsund}{ }Miſch\pwindex{Misch, Robert 6.\,2.\,1860 Żarczyn – 27.\,11.\,1929 Berlin@\textsc{Misch, Robert} (6.\,2.\,1860 Żarczyn – 27.\,11.\,1929 Berlin), \emph{Schriftsteller}|pw} aushalten müſſen, deren \introOben{}dreiaktiger\introOben{}{ }Schwank »Fräulein Frau\pwindex{Misch, Robert 6.\,2.\,1860 Żarczyn – 27.\,11.\,1929 Berlin@\textsc{Misch, Robert} (6.\,2.\,1860 Żarczyn – 27.\,11.\,1929 Berlin), \emph{Schriftsteller}!Fräulein Frau@\strich\emph{Fräulein Frau}|pw}\pwindex{Moser, Gustav von 11.\,5.\,1825 Spandau – 23.\,10.\,1903 Görlitz@\textsc{Moser, Gustav von} (11.\,5.\,1825 Spandau – 23.\,10.\,1903 Görlitz), \emph{Schriftsteller}!Fräulein Frau@\strich\emph{Fräulein Frau}|pw}« gegeben wurde. Nach dem grobkörnigen Schablonenmachwerk
                    das graziöſe Kunſtwerkchen! Das war denn nun ein beſchämend leichter Sieg für
                    Arthur Schnitzler. Daſs{ }ſich gleichwohl die beiden Schwankherren\pwindex{Moser, Gustav von 11.\,5.\,1825 Spandau – 23.\,10.\,1903 Görlitz@\textsc{Moser, Gustav von} (11.\,5.\,1825 Spandau – 23.\,10.\,1903 Görlitz), \emph{Schriftsteller}|pwv}\pwindex{Misch, Robert 6.\,2.\,1860 Żarczyn – 27.\,11.\,1929 Berlin@\textsc{Misch, Robert} (6.\,2.\,1860 Żarczyn – 27.\,11.\,1929 Berlin), \emph{Schriftsteller}|pwv} mit ihrem
                        »Fräulein Frau\pwindex{Misch, Robert 6.\,2.\,1860 Żarczyn – 27.\,11.\,1929 Berlin@\textsc{Misch, Robert} (6.\,2.\,1860 Żarczyn – 27.\,11.\,1929 Berlin), \emph{Schriftsteller}!Fräulein Frau@\strich\emph{Fräulein Frau}|pw}\pwindex{Moser, Gustav von 11.\,5.\,1825 Spandau – 23.\,10.\,1903 Görlitz@\textsc{Moser, Gustav von} (11.\,5.\,1825 Spandau – 23.\,10.\,1903 Görlitz), \emph{Schriftsteller}!Fräulein Frau@\strich\emph{Fräulein Frau}|pw}« die Bühnen früher
                    erobert haben als Schnitzler, der ja doch zu den böſen Modernen i. e.
                    »Unſittlichen« gehört, mit irgend einem{ }ſeiner Werke, iſt bei der
                    Einſichtsloſigkeit unſerer Bühnenleiter begreiflich. \spacefill\mbox{(K.K.)}\pend
           \selectlanguage{ngerman}\endnumbering\briefempfaengerindex{Schnitzler, Arthur@\textsc{Schnitzler, Arthur}!zzzKraus, Karl@\emph{von Karl Kraus}!1893-07-211@{21. 7. 1893}|)be}\mylabel{L00239h}  \newcommand{\dateiname}{L00239}\newcommand{\titel}{Karl Kraus an Arthur Schnitzler, 21. 7. 1893}\newcommand{\editorInnen}{Martin Anton Müller und Gerd-Hermann Susen}%% latex-leseansicht-abspann.tex
%% Abspann für die Leseansicht.
%% Der Schalter \ifkorrekturansicht ist bereits durch den Vorspann gesetzt.

%% latex-abspann.tex
%% Gemeinsamer Abspann für Korrekturansicht und Leseansicht.
%% Setzt den Schalter \ifkorrekturansicht voraus (gesetzt in den
%% einbindenden Dateien latex-korrekturansicht-abspann.tex bzw.
%% latex-leseansicht-abspann.tex).
%% ---------------------------------------------------------------

\normalsize

% Das esempio-Environment wird nur in der Leseansicht benötigt
\ifkorrekturansicht\else
\newenvironment{esempio}[3]%
{
    \vspace{1.5ex}
    \rlap{\underline{#1}}
    \par
    \setlength{\parindent}{0cm}
    \nopagebreak
    \leftskip=#2cm
    \rightskip=#3cm
}
{
    \par
}
\fi

\doendnotes{C}
\bigskip
\vfill

\clearpage

\footnotesize

\ifkorrekturansicht
  \lohead{\textsc{register}}
\fi

% theindex-Environment neu definieren ohne reledmac
\makeatletter
\renewenvironment{theindex}{%
  \ifkorrekturansicht
    \section*{\indexname}%
  \else
    \subsubsection*{Index der erwähnten Entitäten}%
  \fi
  \setlength{\parindent}{0pt}%
  \setlength{\parskip}{0pt plus 0.3pt}%
  \let\item\@idxitem
}{%
  \ifkorrekturansicht\clearpage\fi
}
\makeatother

\IfFileExists{\jobname-pw.ind}{\input{\jobname-pw.ind}}{}

% Quellenangabe nur in der Leseansicht
\ifkorrekturansicht\else
% Fallback-Definitionen, falls die .tex-Datei \titel etc. nicht gesetzt hat
\providecommand{\titel}{}
\providecommand{\editorInnen}{}
\providecommand{\dateiname}{\jobname}

\vspace{3cm}

\vfill

\footnotesize
\textsc{Quelle}: \titel. Herausgegeben von {\editorInnen}. In: \emph{Arthur Schnitzler: Briefwechsel mit Autorinnen und Autoren}.
 Digitale Edition, https://schnitzler-briefe.acdh.oeaw.ac.at/{\dateiname}.html (Stand \today)
\fi

\end{document}


