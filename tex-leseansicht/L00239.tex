%% latex-korrekturansicht-vorspann.tex
%% Vorspann für die Korrekturansicht.
%% Lädt die gemeinsame Datei latex-vorspann.tex mit gesetztem Schalter.

\newif\ifkorrekturansicht
\korrekturansichttrue

\input{../tex-inputs/latex-vorspann}


\section[Karl Kraus an Arthur Schnitzler, 21. 7. 1893]{L00239 Karl Kraus an Arthur Schnitzler, 21. 7. 1893}
\nopagebreak\mylabel{L00239v}
\rehead{ }\normalsize\beginnumbering\briefempfaengerindex{Schnitzler, Arthur@\textsc{Schnitzler, Arthur}!zzzKraus, Karl@\emph{von Karl Kraus}!1893-07-211@{21. 7. 1893}|(be}
\toendnotes[C]{\smallbreak\pagebreak[2]}\Standort{CUL, Schnitzler, B 55.}
\physDesc{Brief, 1 Blatt, 4 Seiten, 3571 Zeichen
\newline{}Handschrift: schwarze Tinte, deutsche Kurrent
\newline{}Beilage: Manuskript auf dem gleichen Briefpapier, 1 Blatt, 1 Seite,
                                 schwarze Tinte }
\buchAbdrucke{\weitereDrucke{\emph{Literatur und Kritik}, Bd. 49, Oktober 1970, S. 518–519.} }\toendnotes[C]{\smallbreak}\pstart{}{\pb}Schnitzler\pend{}{\bigskip}\vspace{1em}
\pstart
           {\pb}\textcolor{gray}{\textbf{KARL KRAUS}}\hfill \substVorne{}\textsuperscript{\textcolor{gray}{\textbf{Wien I., Maximilianstrasse
                              13\oindex{Mahlerstrasse@\textbf{Mahlerstraße}, \emph{Straße (K.STR)}|pw}.}}}\substDazwischen{}Ischl\oindex{Bad Ischl@\textbf{Bad Ischl}, \emph{P.PPL}|pw}\substHinten{}{ }21. Juli, \textcolor{gray}{\textbf{189}}3\pend
           
\pstart{}Mein liebſter, verehrter Herr Doctor!\pend\vspace{0.5em}
\pstart
           Daſs Sie ſo »ſpurlos« ſich auch dem Staube gemacht haben, thut mir ſehr leid. Seit
               Ihrer Vorſtellung haben wir uns ja gar nicht gesprochen.\pend
           
\pstart
           »Sieh’ſt du, \uline{das}{ }\uuline{\edtext{hätt}{\Cendnote{siebenfach unterstrichen}}}’ (!!!!) ich dir \introOben{}doch\introOben{}
               nicht geſagt!« – ich werde dieſen genialen Zug in Frl. Falkner\pwindex{Falkner, Julie 26.07.1867 – 13.04.1937@\textsc{Falkner, Julie} (26.07.1867 – 13.04.1937), \emph{Schauspieler/Schauspielerin}|pw}’s Darſtellung nie vergeſſen. Und darauf noch dröhnender
               Abgangsapplaus, der \strikeout{d} auch die \uline{zweite}{ }Schluſspointe (»Es iſt ja leicht gegangen etc\pwindex{Abschiedssouper@\emph{Abschiedssouper}|pwv}«) unmöglich machte! Von dem
               »Bordellſtück« »Abſchiedsouper\pwindex{Abschiedssouper@\emph{Abschiedssouper}|pw}« wird hier viel
               geſprochen.\pend
           
\pstart
           Meine herzlichſte Gratulation zur \label{K_L00239-1v}\edtext{Kritik\pwindex{Aus Ischl, 14. Juli, schreibt man uns: …@\emph{Aus Ischl, 14. Juli, schreibt man uns: …}|pwv}}{\lemma{\textnormal{\emph{Kritik}}}\Cendnote{\textnormal{[O. V.]: \emph{[Aus Ischl, 14. Juli, schreibt man
                        uns]}\pwindex{Aus Ischl, 14. Juli, schreibt man uns: …@\emph{Aus Ischl, 14. Juli, schreibt man uns: …}|pwk}. In: \emph{Neue Freie Presse}\pwindex{Neue Freie Presse@\emph{Neue Freie Presse}|pwk},
                     Nr. 10.381, 18. 7. 1893, S. 6.}}}\label{K_L00239-1} in N. Fr. Preſſe\pwindex{Neue Freie Presse@\emph{Neue Freie Presse}|pw} (und \label{K_L00239-2v}\edtext{Bauer\pwindex{Bauer, Julius 15.10.1853 – 11.06.1941@\textsc{Bauer, Julius} (15.10.1853 – 11.06.1941), \emph{Schriftsteller/Schriftstellerin, Journalist/Journalistin, Kritiker/Kritikerin}|pw}\pwindex{Abschiedsouper in Ischl]@\emph{[Abschiedsouper in Ischl]}|pwv}}{\lemma{\textnormal{\emph{Bauer}}}\Cendnote{\textnormal{[Julius Bauer\pwindex{Bauer, Julius 15.10.1853 – 11.06.1941@\textsc{Bauer, Julius} (15.10.1853 – 11.06.1941), \emph{Schriftsteller/Schriftstellerin, Journalist/Journalistin, Kritiker/Kritikerin}|pwk}]: \emph{[Abschiedssouper in Ischl]}\pwindex{Abschiedsouper in Ischl]@\emph{[Abschiedsouper in Ischl]}|pwk}. In: \emph{Illustrirtes Wiener Extrablatt}\pwindex{Illustrirtes Wiener Extrablatt@\emph{Illustrirtes Wiener Extrablatt}|pwk}, Jg. 22,
                     Nr. 196, 18. 7. 1893, S. 5.}}}\label{K_L00239-2} im Extrablatt\orgindex{Illustrirtes Wiener Extrablatt@Illustrirtes Wiener Extrablatt|pw})! Sehr dämlich hat ſich Herr \label{K_L00239-3v}\edtext{Skrein\pwindex{Skrein, Stefan 14.09.1856 – 22.09.1931@\textsc{Skrein, Stefan} (14.09.1856 – 22.09.1931), \emph{Schriftsteller/Schriftstellerin, Journalist/Journalistin}|pw}}{\lemma{\textnormal{\emph{Skrein}}}\Cendnote{\textnormal{Stefan\pwindex{Skrein, Stefan 14.09.1856 – 22.09.1931@\textsc{Skrein, Stefan} (14.09.1856 – 22.09.1931), \emph{Schriftsteller/Schriftstellerin, Journalist/Journalistin}|pwk}: \emph{Ischler Brief}\pwindex{Ischler Brief@\emph{Ischler Brief}|pwk}. In: \emph{Wiener Allgemeine
                        Zeitung}\pwindex{Wiener Allgemeine Zeitung@\emph{Wiener Allgemeine Zeitung}|pwk}, Jg. 14, Nr. 4593, 18. 7. 1893,
                  S. 2.}}}\label{K_L00239-3} in der »Allgemeinen\orgindex{Wiener Allgemeine Zeitung@Wiener Allgemeine Zeitung|pw}« geäußert\pwindex{Ischler Brief@\emph{Ischler Brief}|pwv}.\pend
           
\pstart
           Dies mal haben N. Fr. Pr.\orgindex{Neue Freie Presse@Neue Freie Presse|pw} u. Allgemeine\orgindex{Wiener Allgemeine Zeitung@Wiener Allgemeine Zeitung|pw} die Rollen getauſcht.\pend
           
\pstart
           {\pb}Ich habe eine Notiz\pwindex{Ischler Sommertheater@\emph{Ischler Sommertheater}|pwv} an das Wiener Tagblatt\orgindex{Wiener Tagblatt@Wiener Tagblatt|pw}{ }\label{K_L00239-4v}\edtext{geſchickt}{\lemma{\textnormal{\emph{geſchickt}}}\Cendnote{\textnormal{[O. V.]: \emph{Ischler Sommertheater}\pwindex{Ischler Sommertheater@\emph{Ischler Sommertheater}|pwk}. In: \emph{Wiener Abendblatt}\pwindex{Neues Wiener Tagblatt@\emph{Neues Wiener Tagblatt}|pwk}, Jg. 29, Nr. 199,
                        21. 7. 1893, S. 4.}}}\label{K_L00239-4}; \uline{hoffentlich}{ }\uline{wird} (oder, wenn Sie dieſen Brief erhalten) \uline{wurde} es gedruckt. Im \uline{Magazin}\orgindex{Magazin fuer die Literatur des Auslandes@Magazin für die Literatur des Auslandes|pw} wird nichts erſcheinen. Allerdings bin ich nicht ſchuld. Damit Sie meinen guten
               Willen ſehen, ſende ich Ihnen beiliegend meine \substVorne{}\textsuperscript{Kritik}\substDazwischen{}Notiz\substHinten{}, die mir heute Neumann-Hofer\pwindex{Neumann-Hofer, Gilbert Otto 04.02.1857 – 14.04.1941@\textsc{Neumann-Hofer, Gilbert Otto} (04.02.1857 – 14.04.1941), \emph{Kritiker/Kritikerin, Theaterleiter/Theaterleiterin}|pw}
               zurückſandte – mit der Bemerkung:\pend
           
\pstart
           »Eine Vorſtellung in Ischl\oindex{Bad Ischl@\textbf{Bad Ischl}, \emph{P.PPL}|pw} kann in einem
               Wochenblatte nicht beſprochen werden. Solche gelegentlichen Ereigniſſe ſind auf die
               Notiznahme ſeitens der Tagesblätter beſchränkt.« Na, alſo! –\pend
           
\pstart
           Devrient\pwindex{Devrient, Max 12.12.1857 – 13.06.1929@\textsc{Devrient, Max} (12.12.1857 – 13.06.1929), \emph{Regisseur/Regisseurin, Schauspieler/Schauspielerin}|pw}’s Vorleſung war famos: namentlich Fontane\pwindex{Fontane, Theodor 30.12.1819 – 20.09.1898@\textsc{Fontane, Theodor} (30.12.1819 – 20.09.1898), \emph{Schriftsteller/Schriftstellerin, Kritiker/Kritikerin, Apotheker/Apothekerin}|pw}.\pend
           
\pstart
            Ich habe ihm gleich nach unſerer ſeinerzeit. Unterredung nach Wien\oindex{Wien@\textbf{Wien}, \emph{A.ADM2}|pw} geſchrieben, er ſolle {\pb}Liliencron\pwindex{Liliencron, Detlev von 03.06.1844 – 22.07.1909@\textsc{Liliencron, Detlev von} (03.06.1844 – 22.07.1909), \emph{Schriftsteller/Schriftstellerin, Dichter/Dichterin, Dramatiker/Dramatikerin}|pw} leſen. Nun hat er mich – ſelbſt
               aufgeſucht. Liebenswürdig, was? Wie gedruckt; Liliencron\pwindex{Liliencron, Detlev von 03.06.1844 – 22.07.1909@\textsc{Liliencron, Detlev von} (03.06.1844 – 22.07.1909), \emph{Schriftsteller/Schriftstellerin, Dichter/Dichterin, Dramatiker/Dramatikerin}|pw}, den er ſich gleich kaufte, hat ihn \uline{entzückt} u. er wird ihn beſtimmt in \uline{Wien}\oindex{Wien@\textbf{Wien}, \emph{A.ADM2}|pw} vorleſen. Er fragte mich auch, ob ich Gedichte von \uuline{Ihnen} hätte; er wollte ſie nämlich in Marienbad\oindex{Marienbad@\textbf{Marienbad}, \emph{P.PPL}|pw}, wohin er ſich noch am Tage des Beſuches begab, vorleſen. Da nun
               aber die Vorleſung gleich auf den nächſten Tag angeſetzt war, lehnte er auch eine
               eventuelles Telegramm an Sie (zu dem ich mich bereit erklärte) ab. Aber im
                  Winter will er’s nachholen.\pend
           
\pstart
           Leben Sie wohl, bitte beſte Grüße an Loris\pwindex{Hofmannsthal, Hugo von 1874-02-01 – 1929-07-15@\textsc{Hofmannsthal, Hugo von} (1874-02-01 – 1929-07-15), \emph{Schriftsteller/Schriftstellerin}|pw} u
                  Salten\pwindex{Salten, Felix 06.09.1869 – 08.10.1945@\textsc{Salten, Felix} (06.09.1869 – 08.10.1945), \emph{Schriftsteller/Schriftstellerin, Journalist/Journalistin, Chefredakteur/Chefredakteurin}|pw} auszurichten!\pend
           
\pstart
           Herzlichſt Ihr ſehr ergebener{\\[\baselineskip]}\spacefill\mbox{KarlKraus}\pend
           \leftskip=0em{}
\pstart
           \noindent{}N.B. Was ſagen Sie zur »Freien Bühne\orgindex{»Freie Buehne« Verein fuer moderne Literatur@»Freie Bühne« Verein für moderne Literatur|pw}« in Wien\oindex{Wien@\textbf{Wien}, \emph{A.ADM2}|pw}, die – Elbogen\pwindex{Elbogen, Friedrich 20.05.1854 – 15.04.1909@\textsc{Elbogen, Friedrich} (20.05.1854 – 15.04.1909), \emph{Schriftsteller/Schriftstellerin, Kritiker/Kritikerin, Rechtsanwalt/Rechtsanwältin}|pw} aufführt. Ist das nicht zum Todtlachen? Die Veranstalter ſind
                  Revolverjournalisten.\pend
           \selectlanguage{ngerman}\vspace{1em}
\pstart
           \noindent{}
\pstart
           {\pb}\textcolor{gray}{\textbf{KARL KRAUS}}\pend
           
\pstart
           \raggedleft{}\substVorne{}\textsuperscript{\textcolor{gray}{\textbf{Wien I., Maximilianstrasse 13\oindex{Mahlerstrasse@\textbf{Mahlerstraße}, \emph{Straße (K.STR)}|pw}.}}}\substDazwischen{}Ischl\oindex{Bad Ischl@\textbf{Bad Ischl}, \emph{P.PPL}|pw}\substHinten{}{ }15. VII \textcolor{gray}{\textbf{189}}3\pend
           \pend
           
\pstart
           \uline{Arthur Schnitzlers} einaktige Komödie »Abſchiedssouper\pwindex{Abschiedssouper@\emph{Abschiedssouper}|pw}« fand im Ischler Stadttheater\oindex{Lehártheater@\textbf{Lehártheater}, \emph{Theater (K.THE)}|pw} ihre Probeaufführung. Das kleine oberöſterreichiſche\oindex{Oberoesterreich@\textbf{Oberösterreich}, \emph{A.ADM1}|pw} Curorttheater iſt die erſte
               Bühne, die ſich des prächtigen Stückleins angenommen hat.\pend
           
\pstart
           Der überaus lebendige, geiſtreiche Einakter, der eine geradezu bravouröſe Technik
               aufweist, iſt die wirkſamſte der ſieben »Anatol\pwindex{Anatol@\emph{Anatol}|pw}«studien (siehe \label{K_L00239-5v}\edtext{Besprechung\pwindex{Wiener Dichter@\emph{Wiener Dichter}|pwv}}{\lemma{\textnormal{\emph{Besprechung}}}\Cendnote{\textnormal{[O. V.]: \emph{Arthur Schnitzler}\pwindex{Wiener Dichter@\emph{Wiener Dichter}|pwk}. In: \emph{Das Magazin für Litteratur}\pwindex{Magazin fuer die Literatur des Auslandes@\emph{Magazin für die Literatur des Auslandes}|pwk}, Jg. 62, Nr. 18,
                        6. 5. 1893, S. 294.}}}\label{K_L00239-5} in N\textsuperscript{r.} 18) und fand den lebhafteſten Beifall, den nur einige »verſchämte«, in
               ihren heiligſten Gefühlen verletzte Curgäſte im Intereſſe der \substVorne{}\textsuperscript{publiken und privaten}\substDazwischen{}privaten und publiken\substHinten{}{ }Sicherheit abwehren zu müſſen glaubten. Geſpielt
               wurde recht brav; namentlich zeichnete ſich der treffliche \uline{Jarno}\pwindex{Jarno, Josef 24.08.1865 – 11.01.1932@\textsc{Jarno, Josef} (24.08.1865 – 11.01.1932), \emph{Theaterleiter/Theaterleiterin, Schauspieler/Schauspielerin}|pw} vom berliner Reſidenztheater\oindex{Wallnertheater@\textbf{Wallnertheater}, \emph{Theater (K.THE)}|pw} als Max\pwindex{Anatol@\emph{Anatol}|pwv} aus. Die famoſe
               Schluſspointe gieng leider wirkungslos, weil unverſtanden, vorüber. –\pend
           
\pstart
           Arthur Schnitzler, neben Loris\pwindex{Beer-Hofmann, Richard 1866-07-11 – 1945-09-26@\textsc{Beer-Hofmann, Richard} (1866-07-11 – 1945-09-26), \emph{Schriftsteller/Schriftstellerin}|pw} der
               talentvollſte unter den wenigen talentierten Wien\oindex{Wien@\textbf{Wien}, \emph{A.ADM2}|pw}ern, \strikeout{muſste} hat an dieſem Abend die
               Concurrenz – der Herren Moſer\pwindex{Moser, Gustav von 11.05.1825 – 23.10.1903@\textsc{Moser, Gustav von} (11.05.1825 – 23.10.1903), \emph{Schriftsteller/Schriftstellerin}|pw}{ }{\kaufmannsund}{ }Miſch\pwindex{Misch, Robert 06.02.1860 – 27.11.1929@\textsc{Misch, Robert} (06.02.1860 – 27.11.1929), \emph{Schriftsteller/Schriftstellerin}|pw} aushalten müſſen, deren \introOben{}dreiaktiger\introOben{}{ }Schwank »Fräulein
                  Frau\pwindex{Fraeulein Frau@\emph{Fräulein Frau}|pw}« gegeben wurde. Nach dem grobkörnigen Schablonenmachwerk das graziöſe
               Kunſtwerkchen! Das war denn nun ein beſchämend leichter Sieg für Arthur Schnitzler.
               Daſs ſich gleichwohl die beiden Schwankherren\pwindex{Moser, Gustav von 11.05.1825 – 23.10.1903@\textsc{Moser, Gustav von} (11.05.1825 – 23.10.1903), \emph{Schriftsteller/Schriftstellerin}|pwv}\pwindex{Misch, Robert 06.02.1860 – 27.11.1929@\textsc{Misch, Robert} (06.02.1860 – 27.11.1929), \emph{Schriftsteller/Schriftstellerin}|pwv} mit ihrem »Fräulein Frau\pwindex{Fraeulein Frau@\emph{Fräulein Frau}|pw}« die Bühnen früher erobert haben als Schnitzler, der ja doch
               zu den böſen Modernen i. e. »Unſittlichen« gehört, mit irgend einem ſeiner Werke, iſt
               bei der Einſichtsloſigkeit unſerer Bühnenleiter begreiflich. \spacefill\mbox{(K.K.)}\pend
           \selectlanguage{ngerman}\endnumbering\briefempfaengerindex{Schnitzler, Arthur@\textsc{Schnitzler, Arthur}!zzzKraus, Karl@\emph{von Karl Kraus}!1893-07-211@{21. 7. 1893}|)be}\mylabel{L00239h}  \normalsize

\doendnotes{C}
\bigskip
\vfill

\clearpage

\footnotesize

\lohead{\textsc{register}}

% Definiere theindex-Environment komplett neu ohne reledmac
\makeatletter
\renewenvironment{theindex}{%
  \section*{\indexname}%
  \setlength{\parindent}{0pt}%
  \setlength{\parskip}{0pt plus 0.3pt}%
  \let\item\@idxitem
}{%
  \clearpage
}
\makeatother

\IfFileExists{\jobname-pw.ind}{\input{\jobname-pw.ind}}{}

\end{document}

      