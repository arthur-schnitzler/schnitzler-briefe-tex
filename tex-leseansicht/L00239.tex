%% latex-leseansicht-vorspann.tex
%% Vorspann für die Leseansicht.
%% Lädt die gemeinsame Datei latex-vorspann.tex mit nicht gesetztem Schalter.

\newif\ifkorrekturansicht
\korrekturansichtfalse

\input{../tex-inputs/latex-vorspann}


               \section[Karl Kraus an Arthur Schnitzler, 21. 7. 1893]{ Karl Kraus an Arthur Schnitzler, 21. 7. 1893}\nopagebreak\mylabel{v}\rehead{ }\begin{ledgroupsized}[t]{13cm}\normalsize\beginnumbering\briefempfaengerindex{Schnitzler, Arthur@\textsc{Schnitzler, Arthur}!zzzKraus, Karl@\emph{von Karl Kraus}!1893-07-211@{21. 7. 1893}|(be} \toendnotes[C]{\smallbreak\pagebreak[2]} \Standort{CUL, Schnitzler, B 55.}
\physDesc{Brief, 1 Blatt, 4 Seiten
\newline{}Handschrift: schwarze Tinte, deutsche Kurrent\newline{}Beilage: Manuskript auf dem gleichen Briefpapier, 1 Blatt, 1 Seite,
                                 schwarze Tinte }\buchAbdrucke{\weitereDrucke{\emph{Karl Kraus und Arthur Schnitzler. Eine Dokumentation.} Hg. Reinhard Urbach. In: \emph{Literatur und Kritik}, Bd. 49, Oktober 1970, S. 518–519.} }\toendnotes[C]{\smallbreak}\pstart{}{\pb}Schnitzler\pend{}{\bigskip}\pstart
           \noindent{}{\pb}\textcolor{gray}{\textbf{KARL KRAUS}}\hfill \substVorne{}\textsuperscript{\textcolor{gray}{\textbf{Wien I., Maximilianstrasse 13\oindex{Mahlerstrasse@\textbf{Mahlerstraße}|pw}.}}}{\allowbreak}\substDazwischen{}Ischl\oindex{Bad Ischl@\textbf{Bad Ischl}|pw}\substHinten{}{ }21. Juli, \textcolor{gray}{\textbf{189}}3\pend
           \pstart{}Mein liebſter, verehrter Herr Doctor!\pend\pstart
           Daſs Sie ſo »ſpurlos« ſich auch dem Staube gemacht haben, thut mir ſehr leid. Seit
               Ihrer Vorſtellung haben wir uns ja gar nicht gesprochen.\pend
           \pstart
           »Sieh’ſt du, \uline{das}{ }\uuline{\edtext{hätt}{\Cendnote{siebenfach unterstrichen}}}’ (!!!!) ich dir \introOben{}doch\introOben{}
               nicht geſagt!« – ich werde dieſen genialen Zug in Frl. Falkner\pwindex{Falkner, Julie 26.07.1867 – 13.04.1937@\textsc{Falkner, Julie} (26.07.1867 – 13.04.1937), \emph{Schauspielerin}|pw}’s Darſtellung nie vergeſſen. Und darauf noch dröhnender
               Abgangsapplaus, der \strikeout{d} auch die \uline{zweite}{ }Schluſspointe (»Es iſt ja leicht gegangen etc\pwindex{Schnitzler, Arthur 15.05.1862 – 21.10.1931@\textsc{Schnitzler, Arthur} (15.05.1862 – 21.10.1931), \emph{Schriftsteller, Mediziner}!Abschiedssouper1892@\strich\emph{Abschiedssouper} {[}1892{]}|pwv}«) unmöglich machte! Von dem
               »Bordellſtück« »Abſchiedsouper\pwindex{Schnitzler, Arthur 15.05.1862 – 21.10.1931@\textsc{Schnitzler, Arthur} (15.05.1862 – 21.10.1931), \emph{Schriftsteller, Mediziner}!Abschiedssouper1892@\strich\emph{Abschiedssouper} {[}1892{]}|pw}« wird hier viel
               geſprochen.\pend
           \pstart
           Meine herzlichſte Gratulation zur \label{K_L00239_1v}\edtext{Kritik\pwindex{?? Werk@Nicht ermittelte Verfasserinnen und Verfasser!Aus Ischl, 14. Juli, schreibt man uns: …18.07.1893 – 18.07.1893@\emph{Aus Ischl, 14. Juli, schreibt man uns: …} {[}18.07.1893 – 18.07.1893{]}|pwv}}{\lemma{\textnormal{\emph{Kritik}}}\Cendnote{\textnormal{[O. V.:] \emph{[Aus Ischl, 14. Juli, schreibt man
                        uns]}\pwindex{?? Werk@Nicht ermittelte Verfasserinnen und Verfasser!Aus Ischl, 14. Juli, schreibt man uns: …18.07.1893 – 18.07.1893@\emph{Aus Ischl, 14. Juli, schreibt man uns: …} {[}18.07.1893 – 18.07.1893{]}|pwk}. In: \emph{Neue Freie Presse}\pwindex{Neue Freie Presse1864 – 1939@\emph{Neue Freie Presse}|pwk},
                     Nr. 10.381, 18. 7. 1893, S. 6.}}}\label{K_L00239_1h} in N. Fr. Preſſe\pwindex{Neue Freie Presse1864 – 1939@\emph{Neue Freie Presse}|pw} (und \label{K_L00239_2v}\edtext{Bauer\pwindex{Bauer, Julius 15.10.1853 – 11.06.1941@\textsc{Bauer, Julius} (15.10.1853 – 11.06.1941), \emph{Schriftsteller, Journalist, Kritiker}|pw}\pwindex{Abschiedsouper in Ischl]18. 07. 1893@\emph{[Abschiedsouper in Ischl]} {[}18. 07. 1893{]}|pwv}}{\lemma{\textnormal{\emph{Bauer}}}\Cendnote{\textnormal{[O. V. = Julius Bauer\pwindex{Bauer, Julius 15.10.1853 – 11.06.1941@\textsc{Bauer, Julius} (15.10.1853 – 11.06.1941), \emph{Schriftsteller, Journalist, Kritiker}|pwk}:] \emph{[Abschiedssouper in Ischl]}\pwindex{Abschiedsouper in Ischl]18. 07. 1893@\emph{[Abschiedsouper in Ischl]} {[}18. 07. 1893{]}|pwk}. In: \emph{Illustrirtes Wiener Extrablatt}\pwindex{Illustriertes Wiener Extrablatt1872 – 1928@\emph{Illustriertes Wiener Extrablatt}|pwk}, Jg. 22, Nr. 196,
                        18. 7. 1893, S. 5.}}}\label{K_L00239_2h} im Extrablatt\orgindex{Illustrirtes Wiener Extrablatt@Illustrirtes Wiener Extrablatt|pw})! Sehr dämlich hat ſich Herr \label{K_L00239_3v}\edtext{Skrein\pwindex{Skrein, Stefan 14.09.1856 – 22.09.1931@\textsc{Skrein, Stefan} (14.09.1856 – 22.09.1931), \emph{Schriftsteller, Journalist}|pw}}{\lemma{\textnormal{\emph{Skrein}}}\Cendnote{\textnormal{Stefan\pwindex{Skrein, Stefan 14.09.1856 – 22.09.1931@\textsc{Skrein, Stefan} (14.09.1856 – 22.09.1931), \emph{Schriftsteller, Journalist}|pwk}: \emph{Ischler Brief}\pwindex{Ischler Brief18.07.1893 – 18.07.1893@\emph{Ischler Brief} {[}18.07.1893 – 18.07.1893{]}|pwk}. In: \emph{Wiener Allgemeine
                        Zeitung}\pwindex{Wiener Allgemeine Zeitung1.3.1880 – 11.2.1934@\emph{Wiener Allgemeine Zeitung}|pwk}, Jg. 14, Nr. 4593, 18. 7. 1893,
                  S. 2.}}}\label{K_L00239_3h} in der »Allgemeinen\orgindex{Wiener Allgemeine Zeitung@Wiener Allgemeine Zeitung|pw}«
                  geäußert\pwindex{Ischler Brief18.07.1893 – 18.07.1893@\emph{Ischler Brief} {[}18.07.1893 – 18.07.1893{]}|pwv}.\pend
           \pstart
           Dies mal haben N. Fr. Pr.\orgindex{Neue Freie Presse@Neue Freie Presse|pw} u. Allgemeine\orgindex{Wiener Allgemeine Zeitung@Wiener Allgemeine Zeitung|pw} die Rollen getauſcht.\pend
           \pstart
           {\pb}Ich habe eine Notiz\pwindex{Ischler Sommertheater21. 07. 1893@\emph{Ischler Sommertheater} {[}21. 07. 1893{]}|pwv} an das Wiener Tagblatt\orgindex{Wiener Tagblatt@Wiener Tagblatt|pw}{ }\label{K_L00239_4v}\edtext{geſchickt}{\lemma{\textnormal{\emph{geſchickt}}}\Cendnote{\textnormal{[O. V.:] \emph{Ischler Sommertheater}\pwindex{Ischler Sommertheater21. 07. 1893@\emph{Ischler Sommertheater} {[}21. 07. 1893{]}|pwk}. In: \emph{Wiener Abendblatt}\pwindex{Neues Wiener Tagblatt1867 – 1945@\emph{Neues Wiener Tagblatt}|pwk}, Jg. 29, Nr. 199,
                        21. 7. 1893, S. 4.}}}\label{K_L00239_4h}; \uline{hoffentlich}{ }\uline{wird} (oder, wenn Sie dieſen Brief erhalten) \uline{wurde} es gedruckt. Im \uline{Magazin}\orgindex{Magazin fuer die Literatur des Auslandes@Magazin für die Literatur des Auslandes|pw} wird nichts erſcheinen. Allerdings bin ich nicht ſchuld. Damit Sie meinen guten
               Willen ſehen, ſende ich Ihnen beiliegend meine \substVorne{}\textsuperscript{Kritik}{\allowbreak}\substDazwischen{}Notiz\substHinten{}, die mir heute Neumann-Hofer\pwindex{Neumann-Hofer, Gilbert Otto 04.02.1857 – 14.04.1941@\textsc{Neumann-Hofer, Gilbert Otto} (04.02.1857 – 14.04.1941), \emph{Kritiker, Theaterleiter}|pw}
               zurückſandte – mit der Bemerkung:\pend
           \pstart
           »Eine Vorſtellung in Ischl\oindex{Bad Ischl@\textbf{Bad Ischl}|pw} kann in einem
               Wochenblatte nicht beſprochen werden. Solche gelegentlichen Ereigniſſe ſind auf die
               Notiznahme ſeitens der Tagesblätter beſchränkt.« Na, alſo! –\pend
           \pstart
           Devrient\pwindex{Devrient, Max 12.12.1857 – 13.06.1929@\textsc{Devrient, Max} (12.12.1857 – 13.06.1929), \emph{Regisseur, Schauspieler}|pw}’s Vorleſung war famos: namentlich Fontane\pwindex{Fontane, Theodor 30.12.1819 – 20.09.1898@\textsc{Fontane, Theodor} (30.12.1819 – 20.09.1898), \emph{Schriftsteller, Kritiker, Apotheker}|pw}.\pend
           \pstart
            Ich habe ihm gleich nach unſerer ſeinerzeit. Unterredung nach Wien\oindex{Wien@\textbf{Wien}|pw} geſchrieben, er ſolle {\pb}Liliencron\pwindex{Liliencron, Detlev von 03.06.1844 – 22.07.1909@\textsc{Liliencron, Detlev von} (03.06.1844 – 22.07.1909)|pw} leſen. Nun hat er mich – ſelbſt
               aufgeſucht. Liebenswürdig, was? Wie gedruckt; Liliencron\pwindex{Liliencron, Detlev von 03.06.1844 – 22.07.1909@\textsc{Liliencron, Detlev von} (03.06.1844 – 22.07.1909)|pw}, den er ſich gleich kaufte, hat ihn \uline{entzückt} u. er wird ihn beſtimmt in \uline{Wien}\oindex{Wien@\textbf{Wien}|pw} vorleſen. Er fragte mich auch, ob ich Gedichte von \uuline{Ihnen} hätte; er wollte ſie nämlich in Marienbad\oindex{Marienbad@\textbf{Marienbad}|pw}, wohin er ſich noch am Tage des Beſuches begab, vorleſen. Da nun
               aber die Vorleſung gleich auf den nächſten Tag angeſetzt war, lehnte er auch eine
               eventuelles Telegramm an Sie (zu dem ich mich bereit erklärte) ab. Aber im
                  Winter will er’s nachholen.\pend
           \pstart
           Leben Sie wohl, bitte beſte Grüße an Loris\pwindex{Hofmannsthal, Hugo von 01.02.1874 – 15.07.1929@\textsc{Hofmannsthal, Hugo von} (01.02.1874 – 15.07.1929), \emph{Schriftsteller}|pw} u Salten\pwindex{Salten, Felix 06.09.1869 – 08.10.1945@\textsc{Salten, Felix} (06.09.1869 – 08.10.1945), \emph{Schriftsteller, Journalist}|pw} auszurichten!\pend
           \pstart
           Herzlichſst Ihr ſehr ergebener{\\[\baselineskip]}\spacefill\mbox{KarlKraus}\pend
           \leftskip=0em{}\pstart
           \noindent{}N.B. Was ſagen Sie zur »Freien Bühne\orgindex{»Freie Buehne« Verein fuer moderne Literatur@»Freie Bühne« Verein für moderne Literatur|pw}« in Wien\oindex{Wien@\textbf{Wien}|pw}, die – Elbogen\pwindex{Elbogen, Friedrich 20.05.1854 – 15.04.1909@\textsc{Elbogen, Friedrich} (20.05.1854 – 15.04.1909), \emph{Schriftsteller, Kritiker, Rechtsanwalt}|pw} aufführt. Ist das nicht zum Todtlachen? Die Veranstalter ſind
                  Revolverjournalisten.\pend
           \pstart
           \noindent{}{\pb}\textcolor{gray}{\textbf{KARL KRAUS}}\hfill \substVorne{}\textsuperscript{\textcolor{gray}{\textbf{Wien I., Maximilianstrasse 13\oindex{Mahlerstrasse@\textbf{Mahlerstraße}|pw}.}}}{\allowbreak}\substDazwischen{}Ischl\oindex{Bad Ischl@\textbf{Bad Ischl}|pw}\substHinten{}{ }15. VII \textcolor{gray}{\textbf{189}}3\pend
           \pstart
           \uline{Arthur Schnitzlers} einaktige Komödie »Abſchiedssouper\pwindex{Schnitzler, Arthur 15.05.1862 – 21.10.1931@\textsc{Schnitzler, Arthur} (15.05.1862 – 21.10.1931), \emph{Schriftsteller, Mediziner}!Abschiedssouper1892@\strich\emph{Abschiedssouper} {[}1892{]}|pw}« fand im Ischler Stadttheater\oindex{Stadttheater (Bad Ischl)@\textbf{Stadttheater (Bad Ischl)}|pw} ihre Probeaufführung. Das kleine oberöſterreichiſche\oindex{Oberoesterreich@\textbf{Oberösterreich}|pw} Curorttheater iſt die erſte
               Bühne, die ſich des prächtigen Stückleins angenommen hat.\pend
           \pstart
           Der überaus lebendige, geiſtreiche Einakter, der eine geradezu bravouröſe Technik
               aufweist, iſt die wirkſamſte der ſieben »Anatol\pwindex{Schnitzler, Arthur 15.05.1862 – 21.10.1931@\textsc{Schnitzler, Arthur} (15.05.1862 – 21.10.1931), \emph{Schriftsteller, Mediziner}!Anatol1892-10-29 – 1892-10-29@\strich\emph{Anatol} {[}1892-10-29 – 1892-10-29{]}|pw}«studien (siehe \label{K_L00239_5v}\edtext{Besprechung\pwindex{Kraus, Karl 28.04.1874 – 12.06.1936@\textsc{Kraus, Karl} (28.04.1874 – 12.06.1936), \emph{Schriftsteller, Publizist}!Wiener Dichter06. 05. 1893@\strich\emph{Wiener Dichter} {[}06. 05. 1893{]}|pwv}}{\lemma{\textnormal{\emph{Besprechung}}}\Cendnote{\textnormal{[O. V.:] \emph{Arthur Schnitzler}\pwindex{Kraus, Karl 28.04.1874 – 12.06.1936@\textsc{Kraus, Karl} (28.04.1874 – 12.06.1936), \emph{Schriftsteller, Publizist}!Wiener Dichter06. 05. 1893@\strich\emph{Wiener Dichter} {[}06. 05. 1893{]}|pwk}. In: \emph{Das Magazin für Litteratur}\pwindex{Magazin fuer die Literatur des Auslandes1832 – 1915@\emph{Magazin für die Literatur des Auslandes}|pwk}, Jg. 62, Nr. 18,
                        6. 5. 1893, S. 294.}}}\label{K_L00239_5h} in N\textsuperscript{r.} 18) und fand den lebhafteſten Beifall, den nur einige »verſchämte«, in
               ihren heiligſten Gefühlen verletzte Curgäſte im Intereſſe der \substVorne{}\textsuperscript{publiken und privaten}{\allowbreak}\substDazwischen{}privaten und publiken\substHinten{}{ }Sicherheit abwehren zu müſſen glaubten. Geſpielt
               wurde recht brav; namentlich zeichnete ſich der treffliche \uline{Jarno}\pwindex{Jarno, Josef 24.08.1865 – 11.01.1932@\textsc{Jarno, Josef} (24.08.1865 – 11.01.1932), \emph{Theaterleiter, Schauspieler}|pw} vom berliner Reſidenztheater\oindex{Wallnertheater@\textbf{Wallnertheater}|pw} als Max\pwindex{Schnitzler, Arthur 15.05.1862 – 21.10.1931@\textsc{Schnitzler, Arthur} (15.05.1862 – 21.10.1931), \emph{Schriftsteller, Mediziner}!Anatol1892-10-29 – 1892-10-29@\strich\emph{Anatol} {[}1892-10-29 – 1892-10-29{]}|pwv} aus. Die famoſe Schluſspointe
               gieng leider wirkungslos, weil unverſtanden, vorüber. –\pend
           \pstart
           Arthur Schnitzler, neben Loris\pwindex{Beer-Hofmann, Richard 11.07.1866 – 26.09.1945@\textsc{Beer-Hofmann, Richard} (11.07.1866 – 26.09.1945), \emph{Schriftsteller}|pw} der talentvollſte
               unter den wenigen talentierten Wien\oindex{Wien@\textbf{Wien}|pw}ern, \strikeout{muſste} hat an dieſem Abend die Concurrenz – der Herren
                  Moſer\pwindex{Moser, Gustav von 11.05.1825 – 23.10.1903@\textsc{Moser, Gustav von} (11.05.1825 – 23.10.1903), \emph{Schriftsteller}|pw}{ }{\kaufmannsund}{ }Miſch\pwindex{Misch, Robert 06.02.1860 – 27.11.1929@\textsc{Misch, Robert} (06.02.1860 – 27.11.1929), \emph{Schriftsteller}|pw} aushalten müſſen, deren \introOben{}dreiaktiger\introOben{}{ }Schwank »Fräulein
                  Frau\pwindex{Misch, Robert 06.02.1860 – 27.11.1929@\textsc{Misch, Robert} (06.02.1860 – 27.11.1929), \emph{Schriftsteller}!Fraeulein Frau1891@\strich\emph{Fräulein Frau} {[}1891{]}|pw}\pwindex{Moser, Gustav von 11.05.1825 – 23.10.1903@\textsc{Moser, Gustav von} (11.05.1825 – 23.10.1903), \emph{Schriftsteller}!Fraeulein Frau1891@\strich\emph{Fräulein Frau} {[}1891{]}|pw}« gegeben wurde. Nach dem grobkörnigen Schablonenmachwerk das graziöſe
               Kunſtwerkchen! Das war denn nun ein beſchämend leichter Sieg für Arthur Schnitzler.
               Daſs ſich gleichwohl die beiden Schwankherren\pwindex{Moser, Gustav von 11.05.1825 – 23.10.1903@\textsc{Moser, Gustav von} (11.05.1825 – 23.10.1903), \emph{Schriftsteller}|pwv}\pwindex{Misch, Robert 06.02.1860 – 27.11.1929@\textsc{Misch, Robert} (06.02.1860 – 27.11.1929), \emph{Schriftsteller}|pwv} mit ihrem »Fräulein Frau\pwindex{Misch, Robert 06.02.1860 – 27.11.1929@\textsc{Misch, Robert} (06.02.1860 – 27.11.1929), \emph{Schriftsteller}!Fraeulein Frau1891@\strich\emph{Fräulein Frau} {[}1891{]}|pw}\pwindex{Moser, Gustav von 11.05.1825 – 23.10.1903@\textsc{Moser, Gustav von} (11.05.1825 – 23.10.1903), \emph{Schriftsteller}!Fraeulein Frau1891@\strich\emph{Fräulein Frau} {[}1891{]}|pw}« die Bühnen früher erobert haben als Schnitzler, der ja doch
               zu den böſen Modernen i. e. »Unſittlichen« gehört, mit irgend einem ſeiner Werke, iſt
               bei der Einſichtsloſigkeit unſerer Bühnenleiter begreiflich. \spacefill\mbox{(K.K.)}\pend
           \endnumbering\briefempfaengerindex{Schnitzler, Arthur@\textsc{Schnitzler, Arthur}!zzzKraus, Karl@\emph{von Karl Kraus}!1893-07-211@{21. 7. 1893}|)be}\mylabel{h}\end{ledgroupsized}  \newcommand{\dateiname}{L00239}\newcommand{\titel}{Karl Kraus an Arthur Schnitzler, 21. 7. 1893}\newcommand{\editorInnen}{Martin Anton Müller und Gerd-Hermann Susen}%% latex-leseansicht-abspann.tex
%% Abspann für die Leseansicht.
%% Der Schalter \ifkorrekturansicht ist bereits durch den Vorspann gesetzt.

%% latex-abspann.tex
%% Gemeinsamer Abspann für Korrekturansicht und Leseansicht.
%% Setzt den Schalter \ifkorrekturansicht voraus (gesetzt in den
%% einbindenden Dateien latex-korrekturansicht-abspann.tex bzw.
%% latex-leseansicht-abspann.tex).
%% ---------------------------------------------------------------

\normalsize

% Das esempio-Environment wird nur in der Leseansicht benötigt
\ifkorrekturansicht\else
\newenvironment{esempio}[3]%
{
    \vspace{1.5ex}
    \rlap{\underline{#1}}
    \par
    \setlength{\parindent}{0cm}
    \nopagebreak
    \leftskip=#2cm
    \rightskip=#3cm
}
{
    \par
}
\fi

\doendnotes{C}
\bigskip
\vfill

\clearpage

\footnotesize

\ifkorrekturansicht
  \lohead{\textsc{register}}
\fi

% theindex-Environment neu definieren ohne reledmac
\makeatletter
\renewenvironment{theindex}{%
  \ifkorrekturansicht
    \section*{\indexname}%
  \else
    \subsubsection*{Index der erwähnten Entitäten}%
  \fi
  \setlength{\parindent}{0pt}%
  \setlength{\parskip}{0pt plus 0.3pt}%
  \let\item\@idxitem
}{%
  \ifkorrekturansicht\clearpage\fi
}
\makeatother

\IfFileExists{\jobname-pw.ind}{\input{\jobname-pw.ind}}{}

% Quellenangabe nur in der Leseansicht
\ifkorrekturansicht\else
% Fallback-Definitionen, falls die .tex-Datei \titel etc. nicht gesetzt hat
\providecommand{\titel}{}
\providecommand{\editorInnen}{}
\providecommand{\dateiname}{\jobname}

\vspace{3cm}

\vfill

\footnotesize
\textsc{Quelle}: \titel. Herausgegeben von {\editorInnen}. In: \emph{Arthur Schnitzler: Briefwechsel mit Autorinnen und Autoren}.
 Digitale Edition, https://schnitzler-briefe.acdh.oeaw.ac.at/{\dateiname}.html (Stand \today)
\fi

\end{document}


      