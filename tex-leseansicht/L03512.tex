%% latex-leseansicht-vorspann.tex
%% Vorspann für die Leseansicht.
%% Lädt die gemeinsame Datei latex-vorspann.tex mit nicht gesetztem Schalter.

\newif\ifkorrekturansicht
\korrekturansichtfalse

\input{../tex-inputs/latex-vorspann}


\section[ Felix Salten an Arthur Schnitzler, 1. 10. 1907]{L03512 Felix Salten an Arthur Schnitzler,  1. 10. 1907}
\nopagebreak\mylabel{L03512v}
\rehead{ }\normalsize\beginnumbering\briefempfaengerindex{Schnitzler, Arthur@\textsc{Schnitzler, Arthur}!zzzSalten, Felix@\emph{von Felix Salten}!1907-10-011@{1. 10. 1907}|(be}
\toendnotes[C]{\smallbreak\pagebreak[2]}
\correspDesc{Versand  durch Felix Salten am 1. 10. 1907 in Wien
\newline{}Übermittlung  am 2. 10. 1907 in Wien
\newline{}Erhalt  durch Arthur Schnitzler am 2. 10. 1907 in Wien}\toendnotes[C]{\smallbreak}
\Standort{CUL, Schnitzler, B 89, B 1.}
\physDesc{Postkarte, 559 Zeichen
\newline{}Handschrift: schwarze Tinte, lateinische Kurrent
\newline{}Versand: Stempel: »\nobreak{}\oindex{I., Innere Stadt@\textbf{I., Innere Stadt}, \emph{Verwaltungsgebiet}|pwk}1/\textsubscript{1} Wien \textcolor{gray}{5}, 2. X. 07, VII\nobreak{}«. Stempel: »\nobreak{}\oindex{XVIII., Währing@\textbf{XVIII., Währing}, \emph{Verwaltungsgebiet}|pwk}18/\textsubscript{1} Wien 110, 2. X. 07, XII\nobreak{}«.  
\newline{}Ordnung: mit Bleistift von unbekannter Hand nummeriert: »235« }\toendnotes[C]{\smallbreak}\pstart{}{\pb}Salten\pend{}\pstart{}Wien XIX.\oindex{XIX., Döbling@\textbf{XIX., Döbling}, \emph{Verwaltungsgebiet}|pw}\pend{}\pstart{}Armbrustergasse 6\oindex{Wien@\textbf{Wien}!XIX., Döbling@\textbf{XIX., Döbling}!Armbrustergasse@\textbf{Armbrustergasse}, \emph{Straße}|pw}\pend{}{\bigskip}\pstart{}Herrn D\textsuperscript{r} Arthur Schnitzler\pend{}\pstart{}Wien XVIII.\oindex{XVIII., Währing@\textbf{XVIII., Währing}, \emph{Verwaltungsgebiet}|pw}\pend{}\pstart{}Spöttelgaße 7\oindex{Wien@\textbf{Wien}!XVIII., Währing@\textbf{XVIII., Währing}!Edmund-Weiß-Gasse 7@\textbf{Edmund-Weiß-Gasse 7}, \emph{Wohngebäude}|pw}.\pend{}{\bigskip}\vspace{1em}
\pstart
           \raggedleft{}{\pb}Heiligenstadt\oindex{Wien@\textbf{Wien}!XIX., Döbling@\textbf{XIX., Döbling}!Heiligenstadt@\textbf{Heiligenstadt}|pw}, 1. X. 07\pend
           \vspace{0.5em}
\pstart
           Lieber, es geht leider am Freitag
               nicht. Die \label{K_L03512-1v}\edtext{Première\pwindex{Fellner-Feldegg, Ferdinand von 10.\,3.\,1855 Piacenza – 8.\,12.\,1936 Wien@\textsc{Fellner-Feldegg, Ferdinand von} (10.\,3.\,1855 Piacenza – 8.\,12.\,1936 Wien), \emph{Schriftsteller, Architekt}!Mit seinem Gotte allein. Volksschauspiel in 4 Aufzügen@\strich\emph{Mit seinem Gotte allein. Volksschauspiel in 4 Aufzügen}|pwv}}{\lemma{\textnormal{\emph{Première}}}\Cendnote{\textnormal{Uraufführung von \emph{Mit seinem Gotte allein. Volksschauspiel in 4 Aufzügen}\pwindex{Fellner-Feldegg, Ferdinand von 10.\,3.\,1855 Piacenza – 8.\,12.\,1936 Wien@\textsc{Fellner-Feldegg, Ferdinand von} (10.\,3.\,1855 Piacenza – 8.\,12.\,1936 Wien), \emph{Schriftsteller, Architekt}!Mit seinem Gotte allein. Volksschauspiel in 4 Aufzügen@\strich\emph{Mit seinem Gotte allein. Volksschauspiel in 4 Aufzügen}|pwk}
                  von Ferdinand von Fellner-Feldegg\pwindex{Fellner-Feldegg, Ferdinand von 10.\,3.\,1855 Piacenza – 8.\,12.\,1936 Wien@\textsc{Fellner-Feldegg, Ferdinand von} (10.\,3.\,1855 Piacenza – 8.\,12.\,1936 Wien), \emph{Schriftsteller, Architekt}|pwk}}}}\label{K_L03512-1} im Raimund-Theater\oindex{Wien@\textbf{Wien}!VI., Mariahilf@\textbf{VI., Mariahilf}!Raimund-Theater@\textbf{Raimund-Theater}, \emph{Theater}|pw} ist vom Samstag auf den Freitag
               rückverlegt worden, und da muß ich eben hinein. Ich bin aber sehr wahrscheinlich noch
               in der nächsten Woche hier, denn ich höre – indirekt – dass ich in Berlin\oindex{Berlin@\textbf{Berlin}, \emph{Hauptstadt}|pw} erst am \label{K_L03512-2v}\edtext{19. Okt.{ }drankomme\pwindex{Salten, Felix 6.\,9.\,1869 Budapest – 8.\,10.\,1945 Zürich@\textsc{Salten, Felix} (6.\,9.\,1869 Budapest – 8.\,10.\,1945 Zürich), \emph{Schriftsteller, Journalist, Chefredakteur}!Vom andern Ufer. Einakter@\strich\emph{Vom andern Ufer. Einakter}|pwv}}{\lemma{\textnormal{\emph{19. Okt. drankomme}}}\Cendnote{\textnormal{Die Uraufführung von Saltens\pwindex{Salten, Felix 6.\,9.\,1869 Budapest – 8.\,10.\,1945 Zürich@\textsc{Salten, Felix} (6.\,9.\,1869 Budapest – 8.\,10.\,1945 Zürich), \emph{Schriftsteller, Journalist, Chefredakteur}|pwk} Einakterreihe \emph{Vom
                     andern Ufer}\pwindex{Salten, Felix 6.\,9.\,1869 Budapest – 8.\,10.\,1945 Zürich@\textsc{Salten, Felix} (6.\,9.\,1869 Budapest – 8.\,10.\,1945 Zürich), \emph{Schriftsteller, Journalist, Chefredakteur}!Vom andern Ufer. Einakter@\strich\emph{Vom andern Ufer. Einakter}|pwk} fand vier Tage früher, am 15. 10. 1907, am \emph{Lessing-Theater}\orgindex{Lessing-Theater@Lessing-Theater|pwk}
                  statt.}}}\label{K_L03512-2}, und erhalte wol morgen od. übermorgen eine direkte Verständigung. Wenn Ihnen der
                  \label{K_L03512-3v}\edtext{Sonntag}{\lemma{\textnormal{\emph{Sonntag}}}\Cendnote{\textnormal{Vgl. A. S.: \emph{Tagebuch}, 6. 10. 1907.
               }}}\label{K_L03512-3} nicht passt, machen wir vielleicht Freitag
               beim Tennis einen andern Tag aus.\pend
           
\pstart
           Herzlichst{\\[\baselineskip]} Ihr \spacefill\mbox{Salten}\pend
           \leftskip=0em{}\selectlanguage{ngerman}\endnumbering\briefempfaengerindex{Schnitzler, Arthur@\textsc{Schnitzler, Arthur}!zzzSalten, Felix@\emph{von Felix Salten}!1907-10-011@{1. 10. 1907}|)be}\mylabel{L03512h}  \newcommand{\dateiname}{L03512}\newcommand{\titel}{Felix Salten an Arthur Schnitzler, 1. 10. 1907}\newcommand{\editorInnen}{Martin Anton Müller und Laura Untner}%% latex-leseansicht-abspann.tex
%% Abspann für die Leseansicht.
%% Der Schalter \ifkorrekturansicht ist bereits durch den Vorspann gesetzt.

%% latex-abspann.tex
%% Gemeinsamer Abspann für Korrekturansicht und Leseansicht.
%% Setzt den Schalter \ifkorrekturansicht voraus (gesetzt in den
%% einbindenden Dateien latex-korrekturansicht-abspann.tex bzw.
%% latex-leseansicht-abspann.tex).
%% ---------------------------------------------------------------

\normalsize

% Das esempio-Environment wird nur in der Leseansicht benötigt
\ifkorrekturansicht\else
\newenvironment{esempio}[3]%
{
    \vspace{1.5ex}
    \rlap{\underline{#1}}
    \par
    \setlength{\parindent}{0cm}
    \nopagebreak
    \leftskip=#2cm
    \rightskip=#3cm
}
{
    \par
}
\fi

\doendnotes{C}
\bigskip
\vfill

\clearpage

\footnotesize

\ifkorrekturansicht
  \lohead{\textsc{register}}
\fi

% theindex-Environment neu definieren ohne reledmac
\makeatletter
\renewenvironment{theindex}{%
  \ifkorrekturansicht
    \section*{\indexname}%
  \else
    \subsubsection*{Index der erwähnten Entitäten}%
  \fi
  \setlength{\parindent}{0pt}%
  \setlength{\parskip}{0pt plus 0.3pt}%
  \let\item\@idxitem
}{%
  \ifkorrekturansicht\clearpage\fi
}
\makeatother

\IfFileExists{\jobname-pw.ind}{\input{\jobname-pw.ind}}{}

% Quellenangabe nur in der Leseansicht
\ifkorrekturansicht\else
% Fallback-Definitionen, falls die .tex-Datei \titel etc. nicht gesetzt hat
\providecommand{\titel}{}
\providecommand{\editorInnen}{}
\providecommand{\dateiname}{\jobname}

\vspace{3cm}

\vfill

\footnotesize
\textsc{Quelle}: \titel. Herausgegeben von {\editorInnen}. In: \emph{Arthur Schnitzler: Briefwechsel mit Autorinnen und Autoren}.
 Digitale Edition, https://schnitzler-briefe.acdh.oeaw.ac.at/{\dateiname}.html (Stand \today)
\fi

\end{document}


