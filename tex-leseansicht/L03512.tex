%% latex-korrekturansicht-vorspann.tex
%% Vorspann für die Korrekturansicht.
%% Lädt die gemeinsame Datei latex-vorspann.tex mit gesetztem Schalter.

\newif\ifkorrekturansicht
\korrekturansichttrue

\input{../tex-inputs/latex-vorspann}


\section[ Felix Salten an Arthur Schnitzler, 1. 10. 1907]{L03512 Felix Salten an Arthur Schnitzler, 1. 10. 1907}
\nopagebreak\mylabel{L03512v}
\rehead{ }\normalsize\beginnumbering\briefempfaengerindex{Schnitzler, Arthur@\textsc{Schnitzler, Arthur}!zzzSalten, Felix@\emph{von Felix Salten}!1907-10-011@{1. 10. 1907}|(be}
\toendnotes[C]{\smallbreak\pagebreak[2]}\Standort{CUL, Schnitzler, B 89, B 1.}
\physDesc{Postkarte, 559 Zeichen
\newline{}Handschrift: schwarze Tinte, lateinische Kurrent
\newline{}Versand: Stempel: »\nobreak{}\oindex{I., Innere Stadt@\textbf{I., Innere Stadt}, \emph{A.ADM3}|pwk}1/\textsubscript{1} Wien \textcolor{gray}{5}, 2. X. 07, VII\nobreak{}«. Stempel: »\nobreak{}\oindex{XVIII., Waehring@\textbf{XVIII., Währing}, \emph{A.ADM3}|pwk}18/\textsubscript{1} Wien 110, 2. X. 07, XII\nobreak{}«.  
\newline{}Ordnung: mit Bleistift von unbekannter Hand nummeriert: »235« }\toendnotes[C]{\smallbreak}\pstart{}{\pb}Salten\pend{}\pstart{}Wien XIX.\oindex{XIX., Doebling@\textbf{XIX., Döbling}, \emph{A.ADM3}|pw}\pend{}\pstart{}Armbrustergasse 6\oindex{Armbrustergasse@\textbf{Armbrustergasse}, \emph{R.ST}|pw}\pend{}{\bigskip}\pstart{}Herrn D\textsuperscript{r} Arthur Schnitzler\pend{}\pstart{}Wien XVIII.\oindex{XVIII., Waehring@\textbf{XVIII., Währing}, \emph{A.ADM3}|pw}\pend{}\pstart{}Spöttelgaße 7\oindex{Edmund-Weiss-Gasse 7@\textbf{Edmund-Weiß-Gasse 7}, \emph{Wohngebäude (K.WHS)}|pw}.\pend{}{\bigskip}\vspace{1em}
\pstart
           \raggedleft{}{\pb}Heiligenstadt\oindex{Heiligenstadt@\textbf{Heiligenstadt}, \emph{P.PPL}|pw}, 1. X. 07\pend
           \vspace{0.5em}
\pstart
           Lieber, es geht leider am Freitag
               nicht. Die \label{K_L03512-1v}\edtext{Première\pwindex{Mit seinem Gotte allein. Volksschauspiel in 4 Aufzuegen@\emph{Mit seinem Gotte allein. Volksschauspiel in 4 Aufzügen}|pwv}}{\lemma{\textnormal{\emph{Première}}}\Cendnote{\textnormal{Uraufführung von \emph{Mit seinem Gotte allein. Volksschauspiel in 4 Aufzügen}\pwindex{Mit seinem Gotte allein. Volksschauspiel in 4 Aufzuegen@\emph{Mit seinem Gotte allein. Volksschauspiel in 4 Aufzügen}|pwk}
                  von Ferdinand von Fellner-Feldegg\pwindex{Fellner-Feldegg, Ferdinand von 1855-03-10 – 1936-12-08@\textsc{Fellner-Feldegg, Ferdinand von} (1855-03-10 – 1936-12-08), \emph{Schriftsteller/Schriftstellerin, Architekt/Architektin}|pwk}}}}\label{K_L03512-1} im Raimund-Theater\oindex{Raimund-Theater@\textbf{Raimund-Theater}, \emph{Theater (K.THE)}|pw} ist vom Samstag auf den Freitag
               rückverlegt worden, und da muß ich eben hinein. Ich bin aber sehr wahrscheinlich noch
               in der nächsten Woche hier, denn ich höre – indirekt – dass ich in Berlin\oindex{Berlin@\textbf{Berlin}, \emph{P.PPLC}|pw} erst am \label{K_L03512-2v}\edtext{19. Okt.{ }drankomme\pwindex{Vom andern Ufer. Einakter@\emph{Vom andern Ufer. Einakter}|pwv}}{\lemma{\textnormal{\emph{19. Okt. drankomme}}}\Cendnote{\textnormal{Die Uraufführung von Saltens\pwindex{Salten, Felix 06.09.1869 – 08.10.1945@\textsc{Salten, Felix} (06.09.1869 – 08.10.1945), \emph{Schriftsteller/Schriftstellerin, Journalist/Journalistin, Chefredakteur/Chefredakteurin}|pwk} Einakterreihe \emph{Vom
                     andern Ufer}\pwindex{Vom andern Ufer. Einakter@\emph{Vom andern Ufer. Einakter}|pwk} fand vier Tage früher, am 15. 10. 1907, am \emph{Lessing-Theater}\orgindex{Lessing-Theater@Lessing-Theater|pwk}
                  statt.}}}\label{K_L03512-2}, und erhalte wol morgen od. übermorgen eine direkte Verständigung. Wenn Ihnen der
                  \label{K_L03512-3v}\edtext{Sonntag}{\lemma{\textnormal{\emph{Sonntag}}}\Cendnote{\textnormal{Vgl. A. S.: \emph{Tagebuch}, 6. 10. 1907.
               }}}\label{K_L03512-3} nicht passt, machen wir vielleicht Freitag
               beim Tennis einen andern Tag aus.\pend
           
\pstart
           Herzlichst{\\[\baselineskip]} Ihr \spacefill\mbox{Salten}\pend
           \leftskip=0em{}\selectlanguage{ngerman}\endnumbering\briefempfaengerindex{Schnitzler, Arthur@\textsc{Schnitzler, Arthur}!zzzSalten, Felix@\emph{von Felix Salten}!1907-10-011@{1. 10. 1907}|)be}\mylabel{L03512h}  \normalsize

\doendnotes{C}
\bigskip
\vfill

\clearpage

\footnotesize

\lohead{\textsc{register}}

% Definiere theindex-Environment komplett neu ohne reledmac
\makeatletter
\renewenvironment{theindex}{%
  \section*{\indexname}%
  \setlength{\parindent}{0pt}%
  \setlength{\parskip}{0pt plus 0.3pt}%
  \let\item\@idxitem
}{%
  \clearpage
}
\makeatother

\IfFileExists{\jobname-pw.ind}{\input{\jobname-pw.ind}}{}

\end{document}

      