%% latex-leseansicht-vorspann.tex
%% Vorspann für die Leseansicht.
%% Lädt die gemeinsame Datei latex-vorspann.tex mit nicht gesetztem Schalter.

\newif\ifkorrekturansicht
\korrekturansichtfalse

\input{../tex-inputs/latex-vorspann}

\begin{center}
            \textcolor{red}{ENTWURF, NICHT FERTIG KORRIGIERT}
                      \end{center}
            
         
         \renewcommand{\erwaehntePersonen}{Personen: Ferdinand von Fellner-Feldegg}
         \renewcommand{\erwaehnteInstitutionen}{Institutionen: Lessing-Theater}
         \renewcommand{\erwaehnteOrte}{Orte: Armbrustergasse, Berlin, Edmund-Weiß-Gasse, Heiligenstadt, I., Innere Stadt, Raimund-Theater, Wien, XIX., Döbling, XVIII., Währing}
         \renewcommand{\erwaehnteWerke}{Werke: Mit seinem Gotte allein. Volksschauspiel in 4 Aufzügen, Vom andern Ufer. Einakter}
               \section[Felix Salten an Arthur Schnitzler, 1. 10. 1907]{ Felix Salten an Arthur Schnitzler, 1. 10. 1907}\nopagebreak\mylabel{v}\rehead{ }\begin{ledgroupsized}[t]{13cm}\normalsize\beginnumbering \toendnotes[C]{\smallbreak\pagebreak[2]} \Standort{CUL, Schnitzler, B 89, B 1.}
\physDesc{Postkarte, 573 Zeichen
\newline{}Handschrift: schwarze Tinte, lateinische Kurrent
\newline{}Versand: 1) Stempel: »\nobreak{}\oindex{I., Innere Stadt@\textbf{I., Innere Stadt}|pwk}1/\textcolor{gray}{\textsubscript{1}} Wien \textcolor{gray}{5}, 2. X. 07, VII\nobreak{}«.   2) Stempel: »\nobreak{}\oindex{XVIII., Waehring@\textbf{XVIII., Währing}|pwk}18/\textsubscript{1} Wien
                                       110, 2. X. 07, XII\nobreak{}«. 
\newline{}Ordnung: mit Bleistift von unbekannter Hand nummeriert:
                                    »235« }\toendnotes[C]{\smallbreak}\pstart{}{\pb}Salten\pend{}\pstart{}Wien XIX.\oindex{XIX., Doebling@\textbf{XIX., Döbling}|pw}\pend{}\pstart{}Armbrustergasse 6\oindex{Armbrustergasse@\textbf{Armbrustergasse}|pw}\pend{}{\bigskip}\pstart{}Herrn D\textsuperscript{r} Arthur Schnitzler\pend{}\pstart{}Wien XVIII.\oindex{XVIII., Waehring@\textbf{XVIII., Währing}|pw}\pend{}\pstart{}Spöttelgaſse 7\oindex{Edmund-Weiss-Gasse@\textbf{Edmund-Weiß-Gasse}|pw}.\pend{}{\bigskip}\pstart
           \raggedleft{}{\pb}Heiligenstadt\oindex{Heiligenstadt@\textbf{Heiligenstadt}|pw}, 1. X. 07\pend
           \pstart
           Lieber, es geht leider am Freitag nicht. Die \label{K_L03512-1v}\edtext{Première\pwindex{Fellner-Feldegg, Ferdinand von 1855-03-10 – 1936-12-08@\textsc{Fellner-Feldegg, Ferdinand von} (1855-03-10 – 1936-12-08), \emph{Schriftsteller, Architekt, Schriftsteller}!Mit seinem Gotte allein. Volksschauspiel in 4 Aufzuegen1907-10-04@\strich\emph{Mit seinem Gotte allein. Volksschauspiel in 4 Aufzügen} {[}1907-10-04{]}|pw}}{\lemma{\textnormal{\emph{Première}}}\Cendnote{\textnormal{Uraufführung von \emph{Mit seinem Gotte allein. Volksschauspiel in 4 Aufzügen}\pwindex{Fellner-Feldegg, Ferdinand von 1855-03-10 – 1936-12-08@\textsc{Fellner-Feldegg, Ferdinand von} (1855-03-10 – 1936-12-08), \emph{Schriftsteller, Architekt, Schriftsteller}!Mit seinem Gotte allein. Volksschauspiel in 4 Aufzuegen1907-10-04@\strich\emph{Mit seinem Gotte allein. Volksschauspiel in 4 Aufzügen} {[}1907-10-04{]}|pwk}
                  von Ferdinand von Fellner-Feldegg\pwindex{Fellner-Feldegg, Ferdinand von 1855-03-10 – 1936-12-08@\textsc{Fellner-Feldegg, Ferdinand von} (1855-03-10 – 1936-12-08), \emph{Schriftsteller, Architekt, Schriftsteller}|pwk}}}}\label{K_L03512-1h} im Raimund-Theater\oindex{Raimund-Theater@\textbf{Raimund-Theater}|pw} ist vom Samstag auf den
               Freitag rückverlegt worden, und da muß ich eben hinein. Ich bin aber sehr
               wahrscheinlich noch in der nächsten Woche hier, denn ich höre – indirekt – dass ich
               in Berlin\oindex{Berlin@\textbf{Berlin}|pw} erst am \label{K_L03512-2v}\edtext{19. Okt. drankomme\pwindex{Salten, Felix 06.09.1869 – 08.10.1945@\textsc{Salten, Felix} (06.09.1869 – 08.10.1945), \emph{Schriftsteller, Journalist}!Vom andern Ufer. Einakter1907-10-15@\strich\emph{Vom andern Ufer. Einakter} {[}1907-10-15{]}|pwv}}{\lemma{\textnormal{\emph{19. Okt. drankomme}}}\Cendnote{\textnormal{Die Uraufführung der Einakterreihe \emph{Vom andern Ufer}\pwindex{Salten, Felix 06.09.1869 – 08.10.1945@\textsc{Salten, Felix} (06.09.1869 – 08.10.1945), \emph{Schriftsteller, Journalist}!Vom andern Ufer. Einakter1907-10-15@\strich\emph{Vom andern Ufer. Einakter} {[}1907-10-15{]}|pwk} fand sogar vier Tage früher,
                  am 15. 10. 1907, am \emph{Lessing-Theater}\orgindex{Lessing-Theater@Lessing-Theater|pwk} statt.}}}\label{K_L03512-2h}, und erhalte wol morgen od. übermorgen eine
               direkte Verständigung. Wenn Ihnen der Sonntag nicht passt, machen wir vielleicht
               Freitag beim Tennis einen andern Tag aus. \pend
           \pstart
           Herzlichst{\\[\baselineskip]} Ihr \spacefill\mbox{Salten}\pend
           \leftskip=0em{}
         
         \endnumbering\mylabel{h}\end{ledgroupsized}\begin{anhang}\end{anhang}\newcommand{\dateiname}{L03512}\newcommand{\titel}{Felix Salten an Arthur Schnitzler, 1. 10. 1907}\newcommand{\editorInnen}{Martin Anton Müller und Laura Untner}%% latex-leseansicht-abspann.tex
%% Abspann für die Leseansicht.
%% Der Schalter \ifkorrekturansicht ist bereits durch den Vorspann gesetzt.

%% latex-abspann.tex
%% Gemeinsamer Abspann für Korrekturansicht und Leseansicht.
%% Setzt den Schalter \ifkorrekturansicht voraus (gesetzt in den
%% einbindenden Dateien latex-korrekturansicht-abspann.tex bzw.
%% latex-leseansicht-abspann.tex).
%% ---------------------------------------------------------------

\normalsize

% Das esempio-Environment wird nur in der Leseansicht benötigt
\ifkorrekturansicht\else
\newenvironment{esempio}[3]%
{
    \vspace{1.5ex}
    \rlap{\underline{#1}}
    \par
    \setlength{\parindent}{0cm}
    \nopagebreak
    \leftskip=#2cm
    \rightskip=#3cm
}
{
    \par
}
\fi

\doendnotes{C}
\bigskip
\vfill

\clearpage

\footnotesize

\ifkorrekturansicht
  \lohead{\textsc{register}}
\fi

% theindex-Environment neu definieren ohne reledmac
\makeatletter
\renewenvironment{theindex}{%
  \ifkorrekturansicht
    \section*{\indexname}%
  \else
    \subsubsection*{Index der erwähnten Entitäten}%
  \fi
  \setlength{\parindent}{0pt}%
  \setlength{\parskip}{0pt plus 0.3pt}%
  \let\item\@idxitem
}{%
  \ifkorrekturansicht\clearpage\fi
}
\makeatother

\IfFileExists{\jobname-pw.ind}{\input{\jobname-pw.ind}}{}

% Quellenangabe nur in der Leseansicht
\ifkorrekturansicht\else
% Fallback-Definitionen, falls die .tex-Datei \titel etc. nicht gesetzt hat
\providecommand{\titel}{}
\providecommand{\editorInnen}{}
\providecommand{\dateiname}{\jobname}

\vspace{3cm}

\vfill

\footnotesize
\textsc{Quelle}: \titel. Herausgegeben von {\editorInnen}. In: \emph{Arthur Schnitzler: Briefwechsel mit Autorinnen und Autoren}.
 Digitale Edition, https://schnitzler-briefe.acdh.oeaw.ac.at/{\dateiname}.html (Stand \today)
\fi

\end{document}


      