%% latex-leseansicht-vorspann.tex
%% Vorspann für die Leseansicht.
%% Lädt die gemeinsame Datei latex-vorspann.tex mit nicht gesetztem Schalter.

\newif\ifkorrekturansicht
\korrekturansichtfalse

\input{../tex-inputs/latex-vorspann}


\section[ Paul Goldmann an Arthur Schnitzler, 27. 11. [1899]]{L02895 Paul Goldmann an Arthur Schnitzler,  27. 11. [1899]}
\nopagebreak\mylabel{L02895v}
\rehead{ }\normalsize\beginnumbering\briefempfaengerindex{Schnitzler, Arthur@\textsc{Schnitzler, Arthur}!zzzGoldmann, Paul@\emph{von Paul Goldmann}!1899-11-271@{27. 11. [1899]}|(be}
\toendnotes[C]{\smallbreak\pagebreak[2]}
\correspDesc{Versand  durch Paul Goldmann am 27. 11. [1899] in Frankfurt am Main
\newline{}Erhalt  durch Arthur Schnitzler im Zeitraum [28. 11. 1899 – 2. 12. 1899?] in Wien}\toendnotes[C]{\smallbreak}
\Standort{DLA, A:Schnitzler, HS.NZ85.1.3169.}
\physDesc{Brief, 1 Blatt, 2 Seiten, 697 Zeichen
\newline{}Handschrift: schwarze Tinte, deutsche Kurrent
\newline{}Schnitzler: 1) mit Bleistift das Jahr »99« vermerkt  2) mit rotem Buntstift eine Unterstreichung}\toendnotes[C]{\smallbreak}
\pstart
           {\pb}\textcolor{gray}{\textbf{\textbf{Frankfurter Zeitung}}}\orgindex{Frankfurter Zeitung@Frankfurter Zeitung|pw}\hfill \textcolor{gray}{\textbf{\textbf{Frankfurt a. M.\oindex{Frankfurt am Main@\textbf{Frankfurt am Main}, \emph{Hauptstadt}|pw},}}}{ }27. November.\pend
           
\pstart
           \textcolor{gray}{\textbf{und}}\pend
           
\pstart
           \textcolor{gray}{\textbf{Handelsblatt.}}\pend
           
\pstart
           \textcolor{gray}{\textbf{\textbf{Redaktion\orgindex{Frankfurter Zeitung@Frankfurter Zeitung|pwv}.}\footnote{\noindent{}\textcolor{gray}{\textbf{Für die Redaktion\orgindex{Frankfurter Zeitung@Frankfurter Zeitung|pwv} beſtimmte Briefe und Sendungen wolle man
                                 \so{nicht} an die Perſon eines Redakteurs,{ }ſondern{ }ſtets \textbf{an die Redaktion der Frankfurter Zeitung\orgindex{Frankfurter Zeitung@Frankfurter Zeitung|pw}} adreſſiren.}}}}}\pend
           
\pstart
           \textcolor{gray}{\textbf{Telegramm-Adreſſe:}}\pend
           
\pstart
           \textcolor{gray}{\textbf{\textbf{Zeitung\orgindex{Frankfurter Zeitung@Frankfurter Zeitung|pwv}{ }Frankfurt Main\oindex{Frankfurt am Main@\textbf{Frankfurt am Main}, \emph{Hauptstadt}|pw}}}}\pend
           
\pstart{}Mein lieber Freund,\pend\vspace{0.5em}
\pstart
           Du{ }ſchreibſt mir nicht. Aber jetzt \uline{mußt} Du mir
               antworten. Ich brauche dringen{[}d{]} Deinen Rath. Meine Freundin\pwindex{Rottenberg, Theodore 7.\,9.\,1875 – 5.\,4.\,1945 Limburg an der Lahn@\textsc{Rottenberg, Theodore} (7.\,9.\,1875 – 5.\,4.\,1945 Limburg an der Lahn)|pwv} bekommt ein \label{K_L02895-1v}\edtext{Kind\pwindex{Rottenberg, Gertrud 2.\,8.\,1900 Frankfurt am Main – 13.\,3.\,1967@\textsc{Rottenberg, Gertrud} (2.\,8.\,1900 Frankfurt am Main – 13.\,3.\,1967)|pwuv}}{\lemma{\textnormal{\emph{Kind}}}\Cendnote{\textnormal{Theodore Rottenberg\pwindex{Rottenberg, Theodore 7.\,9.\,1875 – 5.\,4.\,1945 Limburg an der Lahn@\textsc{Rottenberg, Theodore} (7.\,9.\,1875 – 5.\,4.\,1945 Limburg an der Lahn)|pwk} gebar am 2. 8. 1900 eine Tochter, Gertrud\pwindex{Rottenberg, Gertrud 2.\,8.\,1900 Frankfurt am Main – 13.\,3.\,1967@\textsc{Rottenberg, Gertrud} (2.\,8.\,1900 Frankfurt am Main – 13.\,3.\,1967)|pwk}. Seit 11. 6. 1895
                  war Theodore\pwindex{Rottenberg, Theodore 7.\,9.\,1875 – 5.\,4.\,1945 Limburg an der Lahn@\textsc{Rottenberg, Theodore} (7.\,9.\,1875 – 5.\,4.\,1945 Limburg an der Lahn)|pwk} mit dem Komponisten und
                  Dirigenten Ludwig Rottenberg\pwindex{Rottenberg, Ludwig 11.\,10.\,1864 Czernowitz – 6.\,5.\,1932 Frankfurt am Main@\textsc{Rottenberg, Ludwig} (11.\,10.\,1864 Czernowitz – 6.\,5.\,1932 Frankfurt am Main), \emph{Kapellmeister}|pwk} verheiratet,
                  dies war das zweite Kind, das der Ehe entsprang. Da sich der Ehemann\pwindex{Rottenberg, Ludwig 11.\,10.\,1864 Czernowitz – 6.\,5.\,1932 Frankfurt am Main@\textsc{Rottenberg, Ludwig} (11.\,10.\,1864 Czernowitz – 6.\,5.\,1932 Frankfurt am Main), \emph{Kapellmeister}|pwkv} während des mutmaßlichen
                  Zeitraums der Zeugung in Wien\oindex{Wien@\textbf{Wien}, \emph{Verwaltungsgebiet}|pwk} aufhielt (vgl. XXXX Auszeichnungsfehler: Dokument L02893 nicht gefunden), dürfte Goldmann\pwindex{Goldmann, Paul 31.\,1.\,1865 Breslau – 25.\,9.\,1935 Wien@\textsc{Goldmann, Paul} (31.\,1.\,1865 Breslau – 25.\,9.\,1935 Wien), \emph{Schriftsteller, Journalist}|pwk} aller Wahrscheinlichkeit nach der
                  biologische Vater sein.}}}\label{K_L02895-1} von mir und \uline{darf}
               keines bekommen. Gibt es ein{ }ſicheres Mittel, das zu verhindern? Die Sache iſt{ }ſechs
               Wochen alt. Aus Gründen, die hier nicht näher erörtert werden können, wäre es
               unmöglich, daß der Mann\pwindex{Rottenberg, Ludwig 11.\,10.\,1864 Czernowitz – 6.\,5.\,1932 Frankfurt am Main@\textsc{Rottenberg, Ludwig} (11.\,10.\,1864 Czernowitz – 6.\,5.\,1932 Frankfurt am Main), \emph{Kapellmeister}|pwv} der
               Vater des Kind\pwindex{Rottenberg, Gertrud 2.\,8.\,1900 Frankfurt am Main – 13.\,3.\,1967@\textsc{Rottenberg, Gertrud} (2.\,8.\,1900 Frankfurt am Main – 13.\,3.\,1967)|pwuv}es
               wäre. Die Kataſtrophe, die wir eben erſt mit Mühe verhindert haben, würde alſo umſo{ }ſicherer \strikeout{\textcolor{gray}{×}\-\textcolor{gray}{×}} und umſo{ }ſchmachvoller für die arme Frau\pwindex{Rottenberg, Theodore 7.\,9.\,1875 – 5.\,4.\,1945 Limburg an der Lahn@\textsc{Rottenberg, Theodore} (7.\,9.\,1875 – 5.\,4.\,1945 Limburg an der Lahn)|pwv} hereinbrechen, und ich{ }ſtünde plötzlich da mit Weib und
               Kind und wahrſcheinlich ohne Stellung. Hier habe ich Niemanden, der mir rathen kann.
                  {\pb}Bitte, rathe Du mir!\pend
           
\pstart
           Viele treue Grüße! {\\[\baselineskip]}Dein {\\[\baselineskip]}\spacefill\mbox{Paul Goldmann.}\pend
           \leftskip=0em{}\selectlanguage{ngerman}\endnumbering\briefempfaengerindex{Schnitzler, Arthur@\textsc{Schnitzler, Arthur}!zzzGoldmann, Paul@\emph{von Paul Goldmann}!1899-11-271@{27. 11. [1899]}|)be}\mylabel{L02895h}  \newcommand{\dateiname}{L02895}\newcommand{\titel}{Paul Goldmann an Arthur Schnitzler, 27. 11. [1899]}\newcommand{\editorInnen}{Martin Anton Müller und Laura Untner}%% latex-leseansicht-abspann.tex
%% Abspann für die Leseansicht.
%% Der Schalter \ifkorrekturansicht ist bereits durch den Vorspann gesetzt.

%% latex-abspann.tex
%% Gemeinsamer Abspann für Korrekturansicht und Leseansicht.
%% Setzt den Schalter \ifkorrekturansicht voraus (gesetzt in den
%% einbindenden Dateien latex-korrekturansicht-abspann.tex bzw.
%% latex-leseansicht-abspann.tex).
%% ---------------------------------------------------------------

\normalsize

% Das esempio-Environment wird nur in der Leseansicht benötigt
\ifkorrekturansicht\else
\newenvironment{esempio}[3]%
{
    \vspace{1.5ex}
    \rlap{\underline{#1}}
    \par
    \setlength{\parindent}{0cm}
    \nopagebreak
    \leftskip=#2cm
    \rightskip=#3cm
}
{
    \par
}
\fi

\doendnotes{C}
\bigskip
\vfill

\clearpage

\footnotesize

\ifkorrekturansicht
  \lohead{\textsc{register}}
\fi

% theindex-Environment neu definieren ohne reledmac
\makeatletter
\renewenvironment{theindex}{%
  \ifkorrekturansicht
    \section*{\indexname}%
  \else
    \subsubsection*{Index der erwähnten Entitäten}%
  \fi
  \setlength{\parindent}{0pt}%
  \setlength{\parskip}{0pt plus 0.3pt}%
  \let\item\@idxitem
}{%
  \ifkorrekturansicht\clearpage\fi
}
\makeatother

\IfFileExists{\jobname-pw.ind}{\input{\jobname-pw.ind}}{}

% Quellenangabe nur in der Leseansicht
\ifkorrekturansicht\else
% Fallback-Definitionen, falls die .tex-Datei \titel etc. nicht gesetzt hat
\providecommand{\titel}{}
\providecommand{\editorInnen}{}
\providecommand{\dateiname}{\jobname}

\vspace{3cm}

\vfill

\footnotesize
\textsc{Quelle}: \titel. Herausgegeben von {\editorInnen}. In: \emph{Arthur Schnitzler: Briefwechsel mit Autorinnen und Autoren}.
 Digitale Edition, https://schnitzler-briefe.acdh.oeaw.ac.at/{\dateiname}.html (Stand \today)
\fi

\end{document}


