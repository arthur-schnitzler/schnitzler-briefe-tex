%% latex-leseansicht-vorspann.tex
%% Vorspann für die Leseansicht.
%% Lädt die gemeinsame Datei latex-vorspann.tex mit nicht gesetztem Schalter.

\newif\ifkorrekturansicht
\korrekturansichtfalse

\input{../tex-inputs/latex-vorspann}


\section[Hermann Bahr an Arthur Schnitzler, 20. 3. 1930]{L02534 Hermann Bahr an Arthur Schnitzler, 20. 3. 1930}
\nopagebreak\mylabel{L02534v}
\rehead{ }\normalsize\beginnumbering\briefempfaengerindex{Schnitzler, Arthur@\textsc{Schnitzler, Arthur}!zzzBahr, Hermann@\emph{von Hermann Bahr}!1930-03-201@{20. 3. 1930}|(be}
\toendnotes[C]{\smallbreak\pagebreak[2]}
\correspDesc{Versand  durch Hermann Bahr am 20. 3. 1930 in München
\newline{}Erhalt  durch Arthur Schnitzler im Zeitraum [21. 3. 1930
                  – 25. 3. 1930?] in Wien}\toendnotes[C]{\smallbreak}
\Standort{CUL, Schnitzler, B 5b.}
\physDesc{Brief, 1 Blatt, 2 Seiten, 1807 Zeichen
\newline{}Handschrift: schwarze Tinte, deutsche Kurrent
\newline{}Schnitzler: mit rotem Buntstift mehrere Unterstreichungen 
\newline{}Ordnung: mit Bleistift von unbekannter Hand nummeriert:
                                    »187« }
\buchAbdrucke{\weitereDrucke{Hermann Bahr, Arthur Schnitzler: \emph{Briefwechsel, Aufzeichnungen, Dokumente (1891–1931)}. Herausgegeben von Kurt Ifkovits und Martin Anton Müller. Göttingen: \emph{Wallstein} 2018, S. 596–597.} }\toendnotes[C]{\smallbreak}
\pstart
           \raggedleft{}{\pb}München Barerſtr. 50\oindex{Barerstraße@\textbf{Barerstraße}, \emph{Straße}|pw}{\\}20. 3. 30\pend
           
\pstart{}Mein lieber Arthur!\pend\vspace{0.5em}
\pstart
           Woltun bringt Zinſen, aber ich bin undankbar genug, Dir die Wohltat, die mir Dein
               lieber Brief erweiſt, übel zu vergelten: durch Jammern über mein Münchener\oindex{München@\textbf{München}|pw} Ungemach. Du fragſt, warum wir\pwindex{Bahr-Mildenburg, Anna 29.\,11.\,1872 Wien – 27.\,1.\,1947 ebd.@\textsc{Bahr-Mildenburg, Anna} (29.\,11.\,1872 Wien – 27.\,1.\,1947 ebd.), \emph{Sängerin}|pwv} nach München\oindex{München@\textbf{München}|pw} überſiedelten. Wir waren Beide »ſtellungslos«, als ich zur Leitung
               des Burgtheaters\oindex{Wien@\textbf{Wien}!I., Innere Stadt@\textbf{I., Innere Stadt}!Burgtheater@\textbf{Burgtheater}, \emph{Theater}|pw} berufen wurde – viel zu{ }ſpät, um
               noch etwas künſtleriſch leiſten oder doch retten zu können. Um dieſe Zeit begann auch
               die öſterreichiſche\oindex{Österreich@\textbf{Österreich}|pw} Währung{ }ſchon zu wanken.
               Das bischen »Vermögen«, das mir mein Vater\pwindex{Bahr, Alois 11.\,4.\,1834 Brünn – 5.\,9.\,1898 Salzburg@\textsc{Bahr, Alois} (11.\,4.\,1834 Brünn – 5.\,9.\,1898 Salzburg), \emph{Notar, Politiker}|pwv} hinterlaſſen hatte, begann zu{ }ſchmelzen; der Reſt
               ging dann bei der deutſchen\oindex{Deutschland@\textbf{Deutschland}|pw} Inflation vollends
               auf. Ganz unverhofft ging da an meine Frau\pwindex{Bahr-Mildenburg, Anna 29.\,11.\,1872 Wien – 27.\,1.\,1947 ebd.@\textsc{Bahr-Mildenburg, Anna} (29.\,11.\,1872 Wien – 27.\,1.\,1947 ebd.), \emph{Sängerin}|pwv} der Ruf, an der Münchener Akademie\orgindex{Akademie der Tonkunst@Akademie der Tonkunst|pw} eine Profeſſur anzunehmen,{ }ſie griff mit beiden Händen
               zu, wir waren die Sorge los, wovon wir morgen unſer Mittagmal {\pb}beſtreiten{ }ſollten; nach einer Reihe von Jahren
               erhält meine Frau\pwindex{Bahr-Mildenburg, Anna 29.\,11.\,1872 Wien – 27.\,1.\,1947 ebd.@\textsc{Bahr-Mildenburg, Anna} (29.\,11.\,1872 Wien – 27.\,1.\,1947 ebd.), \emph{Sängerin}|pwv} als
               Penſion ihren vollen Gehalt. An{ }ſie kam übrigens auch ein \label{K_L02534-1v}\edtext{Ruf}{\lemma{\textnormal{\emph{Ruf}}}\Cendnote{\textnormal{Anfang Januar 1927 ging eine solche Übersiedlung durch die
                  Zeitungen.}}}\label{K_L02534-1} an die Berliner
                  Muſikhochſchule\orgindex{Akademische Hochschule für Musik@Akademische Hochschule für Musik|pw}, den sie natürlich ausſchlug, weil Berlin\oindex{Berlin@\textbf{Berlin}, \emph{Hauptstadt}|pw} noch weiter von ihrem unvergeßlichen Wien\oindex{Wien@\textbf{Wien}, \emph{Verwaltungsgebiet}|pw} iſt als München\oindex{München@\textbf{München}|pw}. Mir
               perſönlich iſt es im Grunde wurſcht, in welcher Stadt ich lebe, ich würde{ }ſchließlich
               auch auf dem Monde ganz gemütlich leben können. Es fällt mir nur{ }ſchwer meine Frau\pwindex{Bahr-Mildenburg, Anna 29.\,11.\,1872 Wien – 27.\,1.\,1947 ebd.@\textsc{Bahr-Mildenburg, Anna} (29.\,11.\,1872 Wien – 27.\,1.\,1947 ebd.), \emph{Sängerin}|pwv}{ }ſich{ }ſo von Sehnſucht nach Wien\oindex{Wien@\textbf{Wien}, \emph{Verwaltungsgebiet}|pw} verzehren zu{ }ſehen. Ich \label{K_L02534-2v}\edtext{ſprach vor einigen Jahren}{\lemma{\textnormal{\emph{sprach … Jahren}}}\Cendnote{\textnormal{Das dürfte sich auf ein Gespräch beziehen, das zwischen dem
                     26. und 29. 9. 1923 in Wien\oindex{Wien@\textbf{Wien}, \emph{Verwaltungsgebiet}|pwk} stattgefunden hat (\emph{Schicksalsjahre Österreichs. Die Erinnerungen und Tagebücher
                        Josef Redlichs 1869–1936.} Herausgegeben von Fritz Fellner und Doris A. Corradini.
                     Wien: \emph{Böhlau} 2011, II, S. 624).}}}\label{K_L02534-2} mit dem
               Prälaten Seipel\pwindex{Seipel, Ignaz 19.\,7.\,1876 Wien – 2.\,8.\,1932 Pernitz@\textsc{Seipel, Ignaz} (19.\,7.\,1876 Wien – 2.\,8.\,1932 Pernitz), \emph{Politiker, Prälat, Bundeskanzler}|pw}, den ich{ }ſehr \substVorne{}\textsuperscript{ſ}\substDazwischen{}l\substHinten{}ange kenne, über die Möglichkeit einer Berufung meiner Frau\pwindex{Bahr-Mildenburg, Anna 29.\,11.\,1872 Wien – 27.\,1.\,1947 ebd.@\textsc{Bahr-Mildenburg, Anna} (29.\,11.\,1872 Wien – 27.\,1.\,1947 ebd.), \emph{Sängerin}|pwv} nach Wien\oindex{Wien@\textbf{Wien}, \emph{Verwaltungsgebiet}|pw},{ }ſei’s auch nur in der Form, daß sie zwei Mal im Jahre, jedes Mal drei
               Wochen, Lehrkurſe an der Wiener »Hochſchule und
                  Akademie für Muſik und darſtellende Kunſt«\oindex{Wien@\textbf{Wien}!I., Innere Stadt@\textbf{I., Innere Stadt}!Hochschule und Akademie für Musik und Darstellende Kunst@\textbf{Hochschule und Akademie für Musik und Darstellende Kunst}, \emph{Universität}|pw} halten{ }ſollte. Seipel\pwindex{Seipel, Ignaz 19.\,7.\,1876 Wien – 2.\,8.\,1932 Pernitz@\textsc{Seipel, Ignaz} (19.\,7.\,1876 Wien – 2.\,8.\,1932 Pernitz), \emph{Politiker, Prälat, Bundeskanzler}|pw} ließ mir dann{ }ſagen, der betreffende »Akt« liege{ }ſchon
               im Unterrichtsminiſterium\oindex{Wien@\textbf{Wien}!I., Innere Stadt@\textbf{I., Innere Stadt}!Ministerium für Unterricht@\textbf{Ministerium für Unterricht}|pw}. Dort liegt er offenbar
               noch heute. »\label{K_L02534-3v}\edtext{Segens{ }ſo heiter iſt das
               Leben in Wien\oindex{Wien@\textbf{Wien}, \emph{Verwaltungsgebiet}|pw}!}{\lemma{\textnormal{\emph{Segens … Wien!}}}\Cendnote{\textnormal{Titel eines Couplets aus \emph{Die Wienerstadt in Wort und Bild}\pwindex{Bauer, Julius 15.\,10.\,1853 Szigetvár – 11.\,6.\,1941 Wien@\textsc{Bauer, Julius} (15.\,10.\,1853 Szigetvár – 11.\,6.\,1941 Wien), \emph{Schriftsteller, Journalist, Kritiker}!Wienerstadt in Wort und Bild@\strich\emph{Die Wienerstadt in Wort und Bild}|pwk}\pwindex{Fuchs, Isidor 25.\,9.\,1849 Lipnik Górny – um den 20.8.1920 Schruns@\textsc{Fuchs, Isidor} (25.\,9.\,1849 Lipnik Górny – um den 20.8.1920 Schruns), \emph{Schriftsteller, Journalist}!Wienerstadt in Wort und Bild@\strich\emph{Die Wienerstadt in Wort und Bild}|pwk}\pwindex{Walzel, Camillo 11.\,2.\,1829 Magdeburg – 17.\,3.\,1895 Wien@\textsc{Walzel, Camillo} (11.\,2.\,1829 Magdeburg – 17.\,3.\,1895 Wien), \emph{Schriftsteller, Theaterleiter}!Wienerstadt in Wort und Bild@\strich\emph{Die Wienerstadt in Wort und Bild}|pwk} von Julius Bauer\pwindex{Bauer, Julius 15.\,10.\,1853 Szigetvár – 11.\,6.\,1941 Wien@\textsc{Bauer, Julius} (15.\,10.\,1853 Szigetvár – 11.\,6.\,1941 Wien), \emph{Schriftsteller, Journalist, Kritiker}|pwk}, Isidor
                     Fuchs\pwindex{Fuchs, Isidor 25.\,9.\,1849 Lipnik Górny – um den 20.8.1920 Schruns@\textsc{Fuchs, Isidor} (25.\,9.\,1849 Lipnik Górny – um den 20.8.1920 Schruns), \emph{Schriftsteller, Journalist}|pwk} und Camillo Walzel\pwindex{Walzel, Camillo 11.\,2.\,1829 Magdeburg – 17.\,3.\,1895 Wien@\textsc{Walzel, Camillo} (11.\,2.\,1829 Magdeburg – 17.\,3.\,1895 Wien), \emph{Schriftsteller, Theaterleiter}|pwk}
                  (1887).}}}\label{K_L02534-3}«\pend
           
\pstart
           Verzeih die lange Epiſtel\hspace*{1.5em}Deinem getreuen{\\[\baselineskip]}\spacefill\mbox{Hermann}\pend
           \leftskip=0em{}\selectlanguage{ngerman}\endnumbering\briefempfaengerindex{Schnitzler, Arthur@\textsc{Schnitzler, Arthur}!zzzBahr, Hermann@\emph{von Hermann Bahr}!1930-03-201@{20. 3. 1930}|)be}\mylabel{L02534h}  \newcommand{\dateiname}{L02534}\newcommand{\titel}{Hermann Bahr an Arthur Schnitzler, 20. 3. 1930}\newcommand{\editorInnen}{Herausgegeben von Martin Anton Müller}%% latex-leseansicht-abspann.tex
%% Abspann für die Leseansicht.
%% Der Schalter \ifkorrekturansicht ist bereits durch den Vorspann gesetzt.

%% latex-abspann.tex
%% Gemeinsamer Abspann für Korrekturansicht und Leseansicht.
%% Setzt den Schalter \ifkorrekturansicht voraus (gesetzt in den
%% einbindenden Dateien latex-korrekturansicht-abspann.tex bzw.
%% latex-leseansicht-abspann.tex).
%% ---------------------------------------------------------------

\normalsize

% Das esempio-Environment wird nur in der Leseansicht benötigt
\ifkorrekturansicht\else
\newenvironment{esempio}[3]%
{
    \vspace{1.5ex}
    \rlap{\underline{#1}}
    \par
    \setlength{\parindent}{0cm}
    \nopagebreak
    \leftskip=#2cm
    \rightskip=#3cm
}
{
    \par
}
\fi

\doendnotes{C}
\bigskip
\vfill

\clearpage

\footnotesize

\ifkorrekturansicht
  \lohead{\textsc{register}}
\fi

% theindex-Environment neu definieren ohne reledmac
\makeatletter
\renewenvironment{theindex}{%
  \ifkorrekturansicht
    \section*{\indexname}%
  \else
    \subsubsection*{Index der erwähnten Entitäten}%
  \fi
  \setlength{\parindent}{0pt}%
  \setlength{\parskip}{0pt plus 0.3pt}%
  \let\item\@idxitem
}{%
  \ifkorrekturansicht\clearpage\fi
}
\makeatother

\IfFileExists{\jobname-pw.ind}{\input{\jobname-pw.ind}}{}

% Quellenangabe nur in der Leseansicht
\ifkorrekturansicht\else
% Fallback-Definitionen, falls die .tex-Datei \titel etc. nicht gesetzt hat
\providecommand{\titel}{}
\providecommand{\editorInnen}{}
\providecommand{\dateiname}{\jobname}

\vspace{3cm}

\vfill

\footnotesize
\textsc{Quelle}: \titel. Herausgegeben von {\editorInnen}. In: \emph{Arthur Schnitzler: Briefwechsel mit Autorinnen und Autoren}.
 Digitale Edition, https://schnitzler-briefe.acdh.oeaw.ac.at/{\dateiname}.html (Stand \today)
\fi

\end{document}


