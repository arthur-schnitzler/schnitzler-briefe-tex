%% latex-korrekturansicht-vorspann.tex
%% Vorspann für die Korrekturansicht.
%% Lädt die gemeinsame Datei latex-vorspann.tex mit gesetztem Schalter.

\newif\ifkorrekturansicht
\korrekturansichttrue

\input{../tex-inputs/latex-vorspann}


\section[ Felix Salten an Arthur Schnitzler, 24. 3. 1908]{L03493 Felix Salten an Arthur Schnitzler, 24. 3. 1908}
\nopagebreak\mylabel{L03493v}
\rehead{ }\normalsize\beginnumbering\briefempfaengerindex{Schnitzler, Arthur@\textsc{Schnitzler, Arthur}!zzzSalten, Felix@\emph{von Felix Salten}!1908-03-241@{24. 3. 1908}|(be}
\toendnotes[C]{\smallbreak\pagebreak[2]}\Standort{CUL, Schnitzler, B 89, B 1.}
\physDesc{Postkarte, 360 Zeichen
\newline{}Handschrift: schwarze Tinte, lateinische Kurrent
\newline{}Versand: Stempel: »\nobreak{}\oindex{I., Innere Stadt@\textbf{I., Innere Stadt}, \emph{A.ADM3}|pwk}1/\textsubscript{1} Wien 6, 24. III. {[}0{]}8, 6\nobreak{}«.  
\newline{}Schnitzler: mit Bleistift datiert: »26/3 0\textcolor{gray}{8}« und Vermerk: »\textsc{S}{[}alten{]}. « 
\newline{}Ordnung: mit Bleistift von unbekannter Hand nummeriert: »243« }\toendnotes[C]{\smallbreak}\pstart{}{\pb}Salten, Wien XIX.\oindex{XIX., Doebling@\textbf{XIX., Döbling}, \emph{A.ADM3}|pw}\pend{}\pstart{}Armbrustergaße 6\oindex{Armbrustergasse@\textbf{Armbrustergasse}, \emph{R.ST}|pw}\pend{}{\bigskip}\pstart{}Herrn D\textsuperscript{r} Arthur Schnitzler\pend{}\pstart{}Wien XVIII. Währing\oindex{XVIII., Waehring@\textbf{XVIII., Währing}, \emph{A.ADM3}|pw}\pend{}\pstart{}Spöttelgaße 7\oindex{Edmund-Weiss-Gasse 7@\textbf{Edmund-Weiß-Gasse 7}, \emph{Wohngebäude (K.WHS)}|pw}\pend{}{\bigskip}\vspace{1em}
\pstart
           \raggedleft{}{\pb}Dienstag.\pend
           
\pstart{}Lieber,\pend\vspace{0.5em}
\pstart
           wollen wir nicht \label{K_L03493-1v}\edtext{dieser Tage einmal
                  beisa{\geminationm}en sein}{\lemma{\textnormal{\emph{dieser … sein}}}\Cendnote{\textnormal{Sie sahen sich das nächste Mal am 30. 3. 1908, womöglich
                  im Theater. Sicher zusammen im Theater\oindex{Ronacher@\textbf{Ronacher}, \emph{Theater (K.THE)}|pwkv} waren sie am 2. 4. 1908.}}}\label{K_L03493-1}? Vielleicht benachrichtigen Sie mich, wenn Sie
               mit Ihrer Frau\pwindex{Schnitzler, Olga 17.01.1882 – 13.01.1970@\textsc{Schnitzler, Olga} (17.01.1882 – 13.01.1970), \emph{Schauspieler/Schauspielerin, Sänger/Sängerin}|pwv} einmal im
               Konzert oder im Theater sind, und wir essen dann zusammen. Oder wir gehen einmal alle
               in’s Apollo\oindex{Apollo-Theater [Wien]@\textbf{Apollo-Theater [Wien]}, \emph{Theater (K.THE)}|pw}, Kolosseum\oindex{Colosseum@\textbf{Colosseum}, \emph{Gastgewerbegebäude (K.GGW)}|pw} od. dergl.?\pend
           
\pstart
           Herzlichst{\\[\baselineskip]}Ihr \spacefill\mbox{Salten}\pend
           \leftskip=0em{}\selectlanguage{ngerman}\endnumbering\briefempfaengerindex{Schnitzler, Arthur@\textsc{Schnitzler, Arthur}!zzzSalten, Felix@\emph{von Felix Salten}!1908-03-241@{24. 3. 1908}|)be}\mylabel{L03493h}  \normalsize

\doendnotes{C}
\bigskip
\vfill

\clearpage

\footnotesize

\lohead{\textsc{register}}

% Definiere theindex-Environment komplett neu ohne reledmac
\makeatletter
\renewenvironment{theindex}{%
  \section*{\indexname}%
  \setlength{\parindent}{0pt}%
  \setlength{\parskip}{0pt plus 0.3pt}%
  \let\item\@idxitem
}{%
  \clearpage
}
\makeatother

\IfFileExists{\jobname-pw.ind}{\input{\jobname-pw.ind}}{}

\end{document}

      