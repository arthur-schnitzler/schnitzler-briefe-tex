%% latex-leseansicht-vorspann.tex
%% Vorspann für die Leseansicht.
%% Lädt die gemeinsame Datei latex-vorspann.tex mit nicht gesetztem Schalter.

\newif\ifkorrekturansicht
\korrekturansichtfalse

\input{../tex-inputs/latex-vorspann}

\begin{center}
            \textcolor{red}{ENTWURF, NICHT FERTIG KORRIGIERT}
                      \end{center}
            
         
         \renewcommand{\erwaehntePersonen}{Personen: Olga Schnitzler}
         \renewcommand{\erwaehnteOrte}{Orte: Apollo-Theater, Armbrustergasse, Colosseum, Edmund-Weiß-Gasse 7, I., Innere Stadt, Ronacher, Wien, XIX., Döbling, XVIII., Währing}
         \renewcommand{\erwaehnteWerke}{}
               \section[ Felix Salten an Arthur Schnitzler, 24. 3. 1908]{ Felix Salten an Arthur Schnitzler, 24. 3. 1908}\nopagebreak\mylabel{v}\rehead{ }\begin{ledgroupsized}[t]{13cm}\normalsize\beginnumbering \toendnotes[C]{\smallbreak\pagebreak[2]} \Standort{CUL, Schnitzler, B 89, B 1.}
\physDesc{Postkarte, 360 Zeichen
\newline{}Handschrift: schwarze Tinte, lateinische Kurrent
\newline{}Versand: Stempel: »\nobreak{}\oindex{I., Innere Stadt@\textbf{I., Innere Stadt}|pwk}1/1 Wien 6 a, 24. III. {[}0{]}8, 6\nobreak{}«.  
\newline{}Schnitzler: mit Bleistift datiert: »26/3 0\textcolor{gray}{8}« und Vermerk: »\textsc{S{[}alten{]}.}« 
\newline{}Ordnung: mit Bleistift von unbekannter Hand nummeriert: »243« }\toendnotes[C]{\smallbreak}\pstart{}{\pb}Salten, Wien XIX.\oindex{XIX., Doebling@\textbf{XIX., Döbling}|pw}\pend{}\pstart{}Armbrustergaße 6\oindex{Armbrustergasse@\textbf{Armbrustergasse}|pw}\pend{}{\bigskip}\pstart{}Herrn D\textsuperscript{r} Arthur Schnitzler\pend{}\pstart{}Wien XVIII. Währing\oindex{XVIII., Waehring@\textbf{XVIII., Währing}|pw}\pend{}\pstart{}Spöttelgaße 7\oindex{Edmund-Weiss-Gasse 7@\textbf{Edmund-Weiß-Gasse 7}|pw}\pend{}{\bigskip}\pstart
           \raggedleft{}{\pb}Dienstag.\pend
           \pstart{}Lieber,\pend\pstart
           wollen wir nicht \label{K_L03493-1v}\edtext{dieser Tage einmal
                  beisa{\geminationm}en sein}{\lemma{\textnormal{\emph{dieser … sein}}}\Cendnote{\textnormal{Siehe A. S.: \emph{Tagebuch}, 30. 3. 1908. Im Theater\oindex{Ronacher@\textbf{Ronacher}|pwkv} waren sie außerdem am
                     2. 4. 1908.}}}\label{K_L03493-1h}? Vielleicht benachrichtigen Sie mich, wenn Sie mit Ihrer Frau\pwindex{Schnitzler, Olga 17.01.1882 – 13.01.1970@\textsc{Schnitzler, Olga} (17.01.1882 – 13.01.1970), \emph{Schauspielerin, Sängerin}|pwv} einmal im Konzert oder im
               Theater sind, und wir essen dann zusammen. Oder wir gehen einmal alle in’s Apollo\oindex{Apollo-Theater@\textbf{Apollo-Theater}|pw}, Kolosseum\oindex{Colosseum@\textbf{Colosseum}|pw} od. dergl.? {\\}Herzlichst\pend
           \pstart Ihr \spacefill\mbox{Salten}\pend{}
         
         \endnumbering\mylabel{h}\end{ledgroupsized}  \newcommand{\dateiname}{L03493}\newcommand{\titel}{Felix Salten an Arthur Schnitzler, 24. 3. 1908}\newcommand{\editorInnen}{Martin Anton Müller und Laura Untner}%% latex-leseansicht-abspann.tex
%% Abspann für die Leseansicht.
%% Der Schalter \ifkorrekturansicht ist bereits durch den Vorspann gesetzt.

%% latex-abspann.tex
%% Gemeinsamer Abspann für Korrekturansicht und Leseansicht.
%% Setzt den Schalter \ifkorrekturansicht voraus (gesetzt in den
%% einbindenden Dateien latex-korrekturansicht-abspann.tex bzw.
%% latex-leseansicht-abspann.tex).
%% ---------------------------------------------------------------

\normalsize

% Das esempio-Environment wird nur in der Leseansicht benötigt
\ifkorrekturansicht\else
\newenvironment{esempio}[3]%
{
    \vspace{1.5ex}
    \rlap{\underline{#1}}
    \par
    \setlength{\parindent}{0cm}
    \nopagebreak
    \leftskip=#2cm
    \rightskip=#3cm
}
{
    \par
}
\fi

\doendnotes{C}
\bigskip
\vfill

\clearpage

\footnotesize

\ifkorrekturansicht
  \lohead{\textsc{register}}
\fi

% theindex-Environment neu definieren ohne reledmac
\makeatletter
\renewenvironment{theindex}{%
  \ifkorrekturansicht
    \section*{\indexname}%
  \else
    \subsubsection*{Index der erwähnten Entitäten}%
  \fi
  \setlength{\parindent}{0pt}%
  \setlength{\parskip}{0pt plus 0.3pt}%
  \let\item\@idxitem
}{%
  \ifkorrekturansicht\clearpage\fi
}
\makeatother

\IfFileExists{\jobname-pw.ind}{\input{\jobname-pw.ind}}{}

% Quellenangabe nur in der Leseansicht
\ifkorrekturansicht\else
% Fallback-Definitionen, falls die .tex-Datei \titel etc. nicht gesetzt hat
\providecommand{\titel}{}
\providecommand{\editorInnen}{}
\providecommand{\dateiname}{\jobname}

\vspace{3cm}

\vfill

\footnotesize
\textsc{Quelle}: \titel. Herausgegeben von {\editorInnen}. In: \emph{Arthur Schnitzler: Briefwechsel mit Autorinnen und Autoren}.
 Digitale Edition, https://schnitzler-briefe.acdh.oeaw.ac.at/{\dateiname}.html (Stand \today)
\fi

\end{document}


      