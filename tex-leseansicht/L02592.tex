%% latex-korrekturansicht-vorspann.tex
%% Vorspann für die Korrekturansicht.
%% Lädt die gemeinsame Datei latex-vorspann.tex mit gesetztem Schalter.

\newif\ifkorrekturansicht
\korrekturansichttrue

\input{../tex-inputs/latex-vorspann}


\section[Marie Herzfeld an Arthur Schnitzler, 16. 1. 1908]{L02592 Marie Herzfeld an Arthur Schnitzler, 16. 1. 1908}
\nopagebreak\mylabel{L02592v}
\rehead{ }\normalsize\beginnumbering\briefempfaengerindex{Schnitzler, Arthur@\textsc{Schnitzler, Arthur}!zzzHerzfeld, Marie@\emph{von Marie Herzfeld}!1908-01-163@{16. 1. 1908}|(be}
\toendnotes[C]{\smallbreak\pagebreak[2]}\Standort{DLA, A:Schnitzler, HS.1985.1.03436,3.}
\physDesc{Brief, 1 Blatt, 4 Seiten, 1380 Zeichen
\newline{}Handschrift: schwarze Tinte, lateinische Kurrent
\newline{}Schnitzler: 1) mit Bleistift Vermerk »\textsc{Marie Herzfeld}«  2) mit rotem Buntstift Vermerk »\textsc{Grillp{[}.{]}
                                             Prei{[}s{]}\orgindex{Franz-Grillparzer-Preis@Franz-Grillparzer-Preis|pwv}}« und fünf Unterstreichungen}\toendnotes[C]{\smallbreak}
\pstart
           \raggedleft{}{\pb}Wien\oindex{Wien@\textbf{Wien}, \emph{A.ADM2}|pw}, den 16. Jan. 1908\pend
           
\pstart\center{}Lieber und sehr geehrter Dr Schnitzler!\pend\vspace{0.5em}
\pstart
           Gestatten Sie mir Ihnen zu sagen, wie sehr ich mich freue, dass Ihnen der \label{K_L02592-1v}\edtext{Grillparzer-Preis\orgindex{Franz-Grillparzer-Preis@Franz-Grillparzer-Preis|pw} verliehen worden}{\lemma{\textnormal{\emph{Grillparzer-Preis … worden}}}\Cendnote{\textnormal{Das Auswahlkomitee hatte am 15. 1. 1908
                  entschieden, Schnitzler für seine
                  Komödie \emph{Zwischenspiel}\pwindex{Zwischenspiel. Komoedie in drei Akten@\emph{Zwischenspiel. Komödie in drei Akten}|pwk} den mit 5000 Kronen
                  dotierten \emph{Grillparzer-Preis}\orgindex{Franz-Grillparzer-Preis@Franz-Grillparzer-Preis|pwk} zu verleihen. In
                  den Jahren zuvor war er zwar immer wieder als Favorit gehandelt worden, doch
                  stellte das Zerwürfnis mit dem \emph{Burgtheater}\orgindex{Burgtheater@Burgtheater|pwk} in
                  Folge der Rückgabe von \emph{Der Schleier der
                     Beatrice}\pwindex{Schleier der Beatrice. Schauspiel in fuenf Akten@\emph{Der Schleier der Beatrice. Schauspiel in fünf Akten}|pwk} (1901) ein Hindernis dar. Seit Sommer 1905 war der Konflikt behoben und Schnitzler konnte wieder bei der Preisvergabe\orgindex{Franz-Grillparzer-Preis@Franz-Grillparzer-Preis|pwkv} berücksichtigt
                  werden.}}}\label{K_L02592-1} und damit öffentlich ausgesprochen ist, wie ungerecht Sie – und
               übrigens nicht bloß Sie – in Oestreich\oindex{Oesterreich@\textbf{Österreich}, \emph{A.PCLI}|pw} gerade
               verkannt werden. Es drängt mich umso mehr, Ihnen das zu {\pb}sagen, weil ich einmal vor Jahren, wenngleich privat, in den gleichen Fehler
               verfiel. Als mich damals – es war in den \label{K_L02592-2v}\edtext{Anatol\pwindex{Anatol@\emph{Anatol}|pw}zeiten}{\lemma{\textnormal{\emph{Anatolzeiten}}}\Cendnote{\textnormal{\emph{Anatol}\pwindex{Anatol@\emph{Anatol}|pwk} erschien gesammelt in Buchform im
                     Oktober 1892. In den Jahren zuvor waren einzelne Szenen in Zeitschriften abgedruckt
                  worden. Dieser
                  Zeitraum dürfte gemeint sein.}}}\label{K_L02592-2} – Ihr Herr Vater\pwindex{Schnitzler, Johann 10.04.1835 – 02.05.1893@\textsc{Schnitzler, Johann} (10.04.1835 – 02.05.1893), \emph{Laryngologe/Laryngologin}|pwv} einmal traf und mit mir über Sie
               sprach und mir die Ehre erwies, mich um meine Meinung über die Tragkraft und
               Spannweite Ihres Talentes zu fragen, da konnte ich nicht anders als meinem {\pb}Eindruck gemäß sagen, es schiene mir mehr wie ein sehr
               empfindlicher Resonnanzboden als wie ein selbstständiges Instrument. Als ich nicht
               lang darauf \uline{Gedichte} von Ihnen hörte, wurde ich zum
               erstenmale stutzig und seit dem Band \label{K_L02592-3v}\edtext{»Der Stein des Weisen\pwindex{Frau des Weisen. Novelletten@\emph{Die Frau des Weisen. Novelletten}|pw}«}{\lemma{\textnormal{\emph{»Der Stein des Weisen«}}}\Cendnote{\textnormal{Herzfeld\pwindex{Herzfeld, Marie 20.03.1855 – 22.09.1940@\textsc{Herzfeld, Marie} (20.03.1855 – 22.09.1940), \emph{Schriftsteller/Schriftstellerin, Übersetzer/Übersetzerin}|pwk} meint den 1898
                  erschienenen Erzählband \emph{Die Frau des Weisen}\pwindex{Frau des Weisen. Novelletten@\emph{Die Frau des Weisen. Novelletten}|pwk}.
               }}}\label{K_L02592-3} weiß ich, dass ich mich sehr geirrt habe, fühle es mit Vergnügen immer wieder
               – (»Dämmerseelen\pwindex{Daemmerseelen. Novellen@\emph{Dämmerseelen. Novellen}|pw}« sind ein Meisterwerk und nicht
                  \uline{einzig} in Ihrem Repertoire –); es hat mich {\pb}dieser Irrtum viel gelehrt und vorsichtig und nachdenklich
               gemacht: übrigens ist mein Instinkt sonst ziemlich sicher.\pend
           
\pstart
           Also nochmals meinen wärmsten Glückwunsch! Und dabei ist \label{K_L02592-4v}\edtext{Schönherr\pwindex{Schoenherr, Karl 24.02.1867 – 15.03.1943@\textsc{Schönherr, Karl} (24.02.1867 – 15.03.1943), \emph{Schriftsteller/Schriftstellerin, Mediziner/Medizinerin}|pw}}{\lemma{\textnormal{\emph{Schönherr}}}\Cendnote{\textnormal{In den Zeitungen wurde \emph{Familie}\pwindex{Familie@\emph{Familie}|pwk} von Karl
                     Schönherr\pwindex{Schoenherr, Karl 24.02.1867 – 15.03.1943@\textsc{Schönherr, Karl} (24.02.1867 – 15.03.1943), \emph{Schriftsteller/Schriftstellerin, Mediziner/Medizinerin}|pwk} als möglicher Preisträger genannt ([O. V.]: \emph{Theater und Kunst. Verleihung des
                        Grillparzer-Preises an Arthur Schnitzler}\pwindex{Theater und Kunst. Verleihung des Grillparzer-Preises an Arthur Schnitzler@\emph{Theater und Kunst. Verleihung des Grillparzer-Preises an Arthur Schnitzler}|pwk}. In: \emph{Die Zeit}\pwindex{Zeit@\emph{Die Zeit}|pwk}, Nr. 1907, 15. 1. 1908,
                     S. 18).}}}\label{K_L02592-4} keiner, den man misachten darf. Ich glaube, man w\uline{ollte} im »Zwischenspiel\pwindex{Zwischenspiel. Komoedie in drei Akten@\emph{Zwischenspiel. Komödie in drei Akten}|pw}« Arthur \uline{Schnitzler} ehren.\pend
           
\pstart
           Mit vielen Grüßen, {\\[\baselineskip]}\spacefill\mbox{Marie Herzfeld}\pend
           \leftskip=0em{}\selectlanguage{ngerman}\endnumbering\briefempfaengerindex{Schnitzler, Arthur@\textsc{Schnitzler, Arthur}!zzzHerzfeld, Marie@\emph{von Marie Herzfeld}!1908-01-163@{16. 1. 1908}|)be}\mylabel{L02592h}  \normalsize

\doendnotes{C}
\bigskip
\vfill

\clearpage

\footnotesize

\lohead{\textsc{register}}

% Definiere theindex-Environment komplett neu ohne reledmac
\makeatletter
\renewenvironment{theindex}{%
  \section*{\indexname}%
  \setlength{\parindent}{0pt}%
  \setlength{\parskip}{0pt plus 0.3pt}%
  \let\item\@idxitem
}{%
  \clearpage
}
\makeatother

\IfFileExists{\jobname-pw.ind}{\input{\jobname-pw.ind}}{}

\end{document}

      