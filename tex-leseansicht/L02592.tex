\input{../tex-inputs/latex-pdf-vorspann}
\begin{center}
            \textcolor{red}{ENTWURF. ENTZIFFERUNG NOCH NICHT KORREKTURGELESEN}
                      \end{center}
            
               \section[Marie Herzfeld an Arthur Schnitzler, 16. 1. 1908]{ Marie Herzfeld an Arthur Schnitzler, 16. 1. 1908}\nopagebreak\mylabel{v}\rehead{ }\begin{ledgroupsized}[t]{13cm}\normalsize\beginnumbering\briefempfaengerindex{Schnitzler, Arthur@\textsc{Schnitzler, Arthur}!zzzHerzfeld, Marie@\emph{von Marie Herzfeld}!1908-01-162@{16. 1. 1908}|(be} \toendnotes[C]{\smallbreak\pagebreak[2]} \Standort{DLA, A:Schnitzler, HS.1985.1.03436,3.}
\physDesc{Brief, 1 Blatt, 4 Seiten
\newline{}Handschrift: schwarze Tinte, lateinische Kurrent
\newline{}Schnitzler: 1) mit Bleistift Vermerk »\textsc{Marie
                                       Herzfeld}« 2) mit rotem Buntstift Vermerk »\textsc{Grillp{[}.{]}
                                       Prei{[}s{]}\orgindex{Franz-Grillparzer-Preis@Franz-Grillparzer-Preis|pwv}}« und fünf Unterstreichungen}\toendnotes[C]{\smallbreak}\pstart
           \raggedleft{}{\pb}Wien\oindex{Wien@\textbf{Wien}|pw}, den 16. Jan. 1908\pend
           \pstart\center{}Lieber und sehr geehrter Dr Schnitzler!\pend\pstart
           Gestatten Sie mir Ihnen zu sagen, wie sehr ich mich freue, dass Ihnen der \label{K_L02592-1v}\edtext{Grillparzer-Preis\orgindex{Franz-Grillparzer-Preis@Franz-Grillparzer-Preis|pw} verliehen worden}{\lemma{\textnormal{\emph{Grillparzer-Preis … worden}}}\Cendnote{\textnormal{Die Zuerkennung des \emph{Grillparzer-Preises}\orgindex{Franz-Grillparzer-Preis@Franz-Grillparzer-Preis|pwk} für das \emph{Zwischenspiel}\pwindex{Schnitzler, Arthur 15.05.1862 – 21.10.1931@\textsc{Schnitzler, Arthur} (15.05.1862 – 21.10.1931), \emph{Schriftsteller, Mediziner}!Zwischenspiel. Komoedie in drei Akten1905-10-12 – 1905-10-12@\strich\emph{Zwischenspiel. Komödie in drei Akten} {[}1905-10-12 – 1905-10-12{]}|pwk} wurde am 15. 1. 1908 verlautbart.}}}\label{K_L02592-1h} und damit öffentlich ausgesprochen
               ist, wie ungerecht Sie – und übrigens nicht bloß Sie – in Oestreich\oindex{Oesterreich@\textbf{Österreich}|pw} gerade verkannt werden. Es drängt mich umso mehr,
               Ihnen das zu {\pb}sagen, weil ich einmal vor Jahren,
               wenngleich privat, in den gleichen Fehler verfiel. Als mich damals – es war in den
                  \label{K_L02592-2v}\edtext{Anatol\pwindex{Schnitzler, Arthur 15.05.1862 – 21.10.1931@\textsc{Schnitzler, Arthur} (15.05.1862 – 21.10.1931), \emph{Schriftsteller, Mediziner}!Anatol1892-10-29 – 1892-10-29@\strich\emph{Anatol} {[}1892-10-29 – 1892-10-29{]}|pw}zeiten}{\lemma{\textnormal{\emph{Anatolzeiten}}}\Cendnote{\textnormal{\emph{Anatol}\pwindex{Schnitzler, Arthur 15.05.1862 – 21.10.1931@\textsc{Schnitzler, Arthur} (15.05.1862 – 21.10.1931), \emph{Schriftsteller, Mediziner}!Anatol1892-10-29 – 1892-10-29@\strich\emph{Anatol} {[}1892-10-29 – 1892-10-29{]}|pwk} erschien gesammelt in Buchform im Oktober 1892, die einzelnen Szenen in den Jahren zuvor; dieser Zeitraum dürfte
                  gemeint sein.}}}\label{K_L02592-2h} – Ihr Herr Vater\pwindex{Schnitzler, Johann 10.04.1835 – 02.05.1893@\textsc{Schnitzler, Johann} (10.04.1835 – 02.05.1893), \emph{Laryngologe}|pwv} einmal traf und mit mir über Sie sprach und mir die Ehre erwies, mich
               um meine Meinung über die Tragkraft und Spannweite Ihres Talentes zu fragen, da
               konnte ich nicht anders als meinem {\pb}Eindruck gemäß sagen,
               es schiene mir mehr wie ein sehr empfindlicher Resonnanzboden als wie ein
               selbstständiges Instrument. Als ich nicht lang darauf \uline{Gedichte} von Ihnen hörte, wurde ich zum erstenmale stutzig und seit dem Band
                  \label{K_L02592-3v}\edtext{»Der
                  Stein des Weisen\pwindex{Schnitzler, Arthur 15.05.1862 – 21.10.1931@\textsc{Schnitzler, Arthur} (15.05.1862 – 21.10.1931), \emph{Schriftsteller, Mediziner}!Frau des Weisen. Novelletten1898@\strich\emph{Die Frau des Weisen. Novelletten} {[}1898{]}|pw}«}{\lemma{\textnormal{\emph{»Der
                  Stein des Weisen«}}}\Cendnote{\textnormal{gemeint ist der
                     1898 erschienene Band \emph{Die Frau des
                     Weisen}\pwindex{Schnitzler, Arthur 15.05.1862 – 21.10.1931@\textsc{Schnitzler, Arthur} (15.05.1862 – 21.10.1931), \emph{Schriftsteller, Mediziner}!Frau des Weisen. Novelletten1898@\strich\emph{Die Frau des Weisen. Novelletten} {[}1898{]}|pwk}}}}\label{K_L02592-3h} weiß ich, dass ich mich sehr geirrt habe, fühle es mit
               Vergnügen immer wieder – (»Dämmerseelen\pwindex{Schnitzler, Arthur 15.05.1862 – 21.10.1931@\textsc{Schnitzler, Arthur} (15.05.1862 – 21.10.1931), \emph{Schriftsteller, Mediziner}!Daemmerseelen. Novellen1907@\strich\emph{Dämmerseelen. Novellen} {[}1907{]}|pw}« sind ein
               Meisterwerk und nicht \uline{einzig} in Ihrem Repertoire –);
               es hat mich {\pb}dieser Irrtum viel gelehrt und vorsichtig und
               nachdenklich gemacht: übrigens ist mein Instinkt sonst ziemlich sicher.\pend
           \pstart
           Also nochmals meinen wärmsten Glückwunsch! Und dabei ist \label{K_L02592-4v}\edtext{Schönherr\pwindex{Schoenherr, Karl 24.02.1867 – 15.03.1943@\textsc{Schönherr, Karl} (24.02.1867 – 15.03.1943), \emph{Schriftsteller, Mediziner}|pw}}{\lemma{\textnormal{\emph{Schönherr}}}\Cendnote{\textnormal{In den Zeitungen wurde \emph{Familie}\pwindex{Schoenherr, Karl 24.02.1867 – 15.03.1943@\textsc{Schönherr, Karl} (24.02.1867 – 15.03.1943), \emph{Schriftsteller, Mediziner}!Familie1905@\strich\emph{Familie} {[}1905{]}|pwk} von Karl
                     Schönherr\pwindex{Schoenherr, Karl 24.02.1867 – 15.03.1943@\textsc{Schönherr, Karl} (24.02.1867 – 15.03.1943), \emph{Schriftsteller, Mediziner}|pwk} als möglicher Preisträger genannt ([O. V.:] \emph{Theater und Kunst. Verleihung des
                        Grillparzer-Preises an Arthur Schnitzler}\pwindex{?? Werk@Nicht ermittelte Verfasserinnen und Verfasser!Theater und Kunst. Verleihung des Grillparzer-Preises an Arthur Schnitzler1908-01-15 – 1908-01-15@\emph{Theater und Kunst. Verleihung des Grillparzer-Preises an Arthur Schnitzler} {[}1908-01-15 – 1908-01-15{]}|pwk}. In: \emph{Die Zeit}\pwindex{Zeit1902 – 1919@\emph{Die Zeit}|pwk}, Nr. 1907,
                        15. 1. 1908, S. 18).}}}\label{K_L02592-4h} keiner, den
               man misachten darf. Ich glaube, man w\uline{ollte} im »Zwischenspiel\pwindex{Schnitzler, Arthur 15.05.1862 – 21.10.1931@\textsc{Schnitzler, Arthur} (15.05.1862 – 21.10.1931), \emph{Schriftsteller, Mediziner}!Zwischenspiel. Komoedie in drei Akten1905-10-12 – 1905-10-12@\strich\emph{Zwischenspiel. Komödie in drei Akten} {[}1905-10-12 – 1905-10-12{]}|pw}« Arthur \uline{Schnitzler} ehren.\pend
           \pstart
           Mit vielen Grüßen, {\\[\baselineskip]}\spacefill\mbox{Marie Herzfeld}\pend
           \leftskip=0em{}\endnumbering\briefempfaengerindex{Schnitzler, Arthur@\textsc{Schnitzler, Arthur}!zzzHerzfeld, Marie@\emph{von Marie Herzfeld}!1908-01-162@{16. 1. 1908}|)be}\mylabel{h}\end{ledgroupsized}  \newcommand{\dateiname}{L02592}\newcommand{\titel}{Marie Herzfeld an Arthur Schnitzler, 16. 1. 1908}\newcommand{\editorInnen}{Martin Anton Müller und Laura Untner}\input{../tex-inputs/latex-pdf-abspann}
      