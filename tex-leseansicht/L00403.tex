%% latex-korrekturansicht-vorspann.tex
%% Vorspann für die Korrekturansicht.
%% Lädt die gemeinsame Datei latex-vorspann.tex mit gesetztem Schalter.

\newif\ifkorrekturansicht
\korrekturansichttrue

\input{../tex-inputs/latex-vorspann}


\section[Arthur Schnitzler an Max Burckhard, 21. 11. 1894]{L00403 Arthur Schnitzler an Max Burckhard, 21. 11. 1894}
\nopagebreak\mylabel{L00403v}
\rehead{ }\normalsize\beginnumbering\briefempfaengerindex{Burckhard, Max Eugen@\textsc{Burckhard, Max Eugen}!zzzSchnitzler, Arthur@\emph{von Arthur Schnitzler}!1894-11-211@{21. 11. 1894}|(be}
\toendnotes[C]{\smallbreak\pagebreak[2]}\buchAlsQuelle{\pwindex{Schnitzlers Einzug ins Burgtheater@\emph{Schnitzlers Einzug ins Burgtheater}|pwk}\pwindex{Neue Freie Presse@\emph{Neue Freie Presse}|pwk}\emph{Neue Freie Presse}, Nr. 24162, 19. 12. 1931, S. 14.}
\buchAbdrucke{\weitereDrucke{1) \pwindex{Schnitzlers Einzug ins Burgtheater@\emph{Schnitzlers Einzug ins Burgtheater}|pwk}\emph{Wiener Studien und Dokumente}. Wien: \emph{Steyrermühl} 1933, S. 166–168.} \weitereDrucke{2) Hans-Ulrich Lindken: \emph{Arthur Schnitzler. Aspekte und Akzente. Materialien zu Leben
                        und Werk}. Frankfurt am Main, Bern, Göttingen: \emph{Peter Lang} 1984, S. 243–246.} }\toendnotes[C]{\smallbreak}
\pstart
           \noindent{}{\pb}\so{Schnitzler an Burckhard}, 21. November 1894:
               »Sehr geehrter Herr Direktor, ſollte ſich mein Stück\pwindex{Liebelei. Schauspiel in drei Akten@\emph{Liebelei. Schauspiel in drei Akten}|pwv} jetzt in Ihren Händen befinden, ſo würde ich bitten,
               es mir recht bald für einige Zeit – hoffentlich nicht für immer – ſenden zu wollen.
               Ich möchte es ſehr gern \label{K_L00403-1v}\edtext{jemandem}{\lemma{\textnormal{\emph{jemandem}}}\Cendnote{\textnormal{Eventuell bezieht sich das auf Adele Sandrock\pwindex{Sandrock, Adele 1863-08-19 – 1937-08-30@\textsc{Sandrock, Adele} (1863-08-19 – 1937-08-30), \emph{Schauspieler/Schauspielerin}|pwk}, der er am 1. 12. 1894{ }\emph{Liebelei}\pwindex{Liebelei. Schauspiel in drei Akten@\emph{Liebelei. Schauspiel in drei Akten}|pwk} vorgelesen hat.}}}\label{K_L00403-1} zeigen und kann die
               neue Abſchrift, die ich mir wieder nach meinem ſehr ſchlecht leſerlichen Manuſkript
               anfertigen laſſe, erſt im Laufe der nächſten Woche erhalten. Sollte ſich Frau Hohenfels\pwindex{Hohenfels, Stella 16.04.1857 – 21.02.1920@\textsc{Hohenfels, Stella} (16.04.1857 – 21.02.1920), \emph{Schauspieler/Schauspielerin}|pw} intereſſieren, in günſtigem Sinne
               entſcheiden – um ſo beſſer. Wenn nicht, ſo werde ich mir jedenfalls erlauben, auf
               Ihren liebenswürdigen Vorſchlag in Betreff Frau Sorma\pwindex{Sorma, Agnes 17.05.1862 – 10.02.1927@\textsc{Sorma, Agnes} (17.05.1862 – 10.02.1927), \emph{Schauspieler/Schauspielerin}|pw} zurückzukommen. Ich kann dieſe Gelegenheit nicht vorübergehen laſſen,
               ohne Ihnen wieder, mein ſehr verehrter Herr Direktor, für Ihre Freundlichkeit und
               Ihre Bemühungen aufs allerwärmſte zu danken. Ihr Entgegenkommen läßt mich noch immer
               an einen ſchließlichen Erfolg glauben. Ihr Sie aufrichtig hochſchätzender Arthur
               Schnitzler.«\pend
           \selectlanguage{ngerman}\endnumbering\briefempfaengerindex{Burckhard, Max Eugen@\textsc{Burckhard, Max Eugen}!zzzSchnitzler, Arthur@\emph{von Arthur Schnitzler}!1894-11-211@{21. 11. 1894}|)be}\mylabel{L00403h}  \normalsize

\doendnotes{C}
\bigskip
\vfill

\clearpage

\footnotesize

\lohead{\textsc{register}}

% Definiere theindex-Environment komplett neu ohne reledmac
\makeatletter
\renewenvironment{theindex}{%
  \section*{\indexname}%
  \setlength{\parindent}{0pt}%
  \setlength{\parskip}{0pt plus 0.3pt}%
  \let\item\@idxitem
}{%
  \clearpage
}
\makeatother

\IfFileExists{\jobname-pw.ind}{\input{\jobname-pw.ind}}{}

\end{document}

      