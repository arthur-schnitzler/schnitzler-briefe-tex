%% latex-leseansicht-vorspann.tex
%% Vorspann für die Leseansicht.
%% Lädt die gemeinsame Datei latex-vorspann.tex mit nicht gesetztem Schalter.

\newif\ifkorrekturansicht
\korrekturansichtfalse

\input{../tex-inputs/latex-vorspann}


\section[Arthur Schnitzler an Max Burckhard, 21. 11. 1894]{L00403 Arthur Schnitzler an Max Burckhard, 21. 11. 1894}
\nopagebreak\mylabel{L00403v}
\rehead{ }\normalsize\beginnumbering\briefempfaengerindex{Burckhard, Max Eugen@\textsc{Burckhard, Max Eugen}!zzzSchnitzler, Arthur@\emph{von Arthur Schnitzler}!1894-11-211@{21. 11. 1894}|(be}
\toendnotes[C]{\smallbreak\pagebreak[2]}
\correspDesc{Versand  durch Arthur Schnitzler am 21. 11. 1894 in Wien
\newline{}Erhalt  durch Max Burckhard im Zeitraum [21. 11. 1894 – 25. 11. 1894?] in Wien}\toendnotes[C]{\smallbreak}
\buchAlsQuelle{\pwindex{Glossy, Karl 7.\,3.\,1848 Wien – 9.\,9.\,1937 ebd.@\textsc{Glossy, Karl} (7.\,3.\,1848 Wien – 9.\,9.\,1937 ebd.), \emph{Schriftsteller, Museumsleiter, Zensurbeirat}!Schnitzlers Einzug ins Burgtheater@\strich\emph{Schnitzlers Einzug ins Burgtheater}|pwk}\pwindex{Neue Freie Presse@\emph{Neue Freie Presse}|pwk}Karl Glossy: \emph{Schnitzlers Einzug ins Burgtheater. Unbekannte Briefe des Dichters.} In: \emph{Neue Freie Presse}, Nr. 24162, 19. 12. 1931, S. 14.}
\buchAbdrucke{\weitereDrucke{1) \pwindex{Glossy, Karl 7.\,3.\,1848 Wien – 9.\,9.\,1937 ebd.@\textsc{Glossy, Karl} (7.\,3.\,1848 Wien – 9.\,9.\,1937 ebd.), \emph{Schriftsteller, Museumsleiter, Zensurbeirat}!Schnitzlers Einzug ins Burgtheater@\strich\emph{Schnitzlers Einzug ins Burgtheater}|pwk}Karl Glossy: \emph{Schnitzlers Einzug ins Burgtheater. Unbekannte Briefe des Dichters.} In: \emph{Wiener Studien und Dokumente}. Zum 85. Geburtstag des Verfassers herausgegeben von seinen Freunden. Wien: \emph{Steyrermühl} 1933, S. 166–168.} \weitereDrucke{2) Hans-Ulrich Lindken: \emph{Arthur Schnitzler. Aspekte und Akzente. Materialien zu Leben
                        und Werk}. Frankfurt am Main, Bern, Göttingen: \emph{Peter Lang} 1984, S. 243–246 (Europäische Hochschulschriften, Reihe 1, Deutsche Sprache und
                        Literatur, 754).} }\toendnotes[C]{\smallbreak}
\pstart
           \noindent{}{\pb}\so{Schnitzler an Burckhard}, 21. November 1894:
               »Sehr geehrter Herr Direktor,{ }ſollte{ }ſich mein Stück\pwindex{Schnitzler, Arthur 15.\,5.\,1862 Wien – 21.\,10.\,1931 ebd.@\textsc{Schnitzler, Arthur} (15.\,5.\,1862 Wien – 21.\,10.\,1931 ebd.), \emph{Schriftsteller, Mediziner}!Liebelei. Schauspiel in drei Akten@\strich\emph{Liebelei. Schauspiel in drei Akten}|pwv} jetzt in Ihren Händen befinden,{ }ſo würde ich bitten,
               es mir recht bald für einige Zeit – hoffentlich nicht für immer –{ }ſenden zu wollen.
               Ich möchte es{ }ſehr gern \label{K_L00403-1v}\edtext{jemandem}{\lemma{\textnormal{\emph{jemandem}}}\Cendnote{\textnormal{Eventuell bezieht sich das auf Adele Sandrock\pwindex{Sandrock, Adele 19.\,8.\,1863 Rotterdam – 30.\,8.\,1937 Berlin@\textsc{Sandrock, Adele} (19.\,8.\,1863 Rotterdam – 30.\,8.\,1937 Berlin), \emph{Schauspielerin}|pwk}, der er am 1. 12. 1894{ }\emph{Liebelei}\pwindex{Schnitzler, Arthur 15.\,5.\,1862 Wien – 21.\,10.\,1931 ebd.@\textsc{Schnitzler, Arthur} (15.\,5.\,1862 Wien – 21.\,10.\,1931 ebd.), \emph{Schriftsteller, Mediziner}!Liebelei. Schauspiel in drei Akten@\strich\emph{Liebelei. Schauspiel in drei Akten}|pwk} vorgelesen hat.}}}\label{K_L00403-1} zeigen und kann die
               neue Abſchrift, die ich mir wieder nach meinem{ }ſehr{ }ſchlecht leſerlichen Manuſkript
               anfertigen laſſe, erſt im Laufe der nächſten Woche erhalten. Sollte{ }ſich Frau Hohenfels\pwindex{Hohenfels, Stella 16.\,4.\,1857 Florenz – 21.\,2.\,1920 Wien@\textsc{Hohenfels, Stella} (16.\,4.\,1857 Florenz – 21.\,2.\,1920 Wien), \emph{Schauspielerin}|pw} intereſſieren, in günſtigem Sinne
               entſcheiden – um{ }ſo beſſer. Wenn nicht,{ }ſo werde ich mir jedenfalls erlauben, auf
               Ihren liebenswürdigen Vorſchlag in Betreff Frau Sorma\pwindex{Sorma, Agnes 17.\,5.\,1862 Breslau – 10.\,2.\,1927 Crown King@\textsc{Sorma, Agnes} (17.\,5.\,1862 Breslau – 10.\,2.\,1927 Crown King), \emph{Schauspielerin}|pw} zurückzukommen. Ich kann dieſe Gelegenheit nicht vorübergehen laſſen,
               ohne Ihnen wieder, mein{ }ſehr verehrter Herr Direktor, für Ihre Freundlichkeit und
               Ihre Bemühungen aufs allerwärmſte zu danken. Ihr Entgegenkommen läßt mich noch immer
               an einen{ }ſchließlichen Erfolg glauben. Ihr Sie aufrichtig hochſchätzender Arthur
               Schnitzler.«\pend
           \selectlanguage{ngerman}\endnumbering\briefempfaengerindex{Burckhard, Max Eugen@\textsc{Burckhard, Max Eugen}!zzzSchnitzler, Arthur@\emph{von Arthur Schnitzler}!1894-11-211@{21. 11. 1894}|)be}\mylabel{L00403h}  \newcommand{\dateiname}{L00403}\newcommand{\titel}{Arthur Schnitzler an Max Burckhard, 21. 11. 1894}\newcommand{\editorInnen}{Martin Anton Müller und Gerd-Hermann Susen}%% latex-leseansicht-abspann.tex
%% Abspann für die Leseansicht.
%% Der Schalter \ifkorrekturansicht ist bereits durch den Vorspann gesetzt.

%% latex-abspann.tex
%% Gemeinsamer Abspann für Korrekturansicht und Leseansicht.
%% Setzt den Schalter \ifkorrekturansicht voraus (gesetzt in den
%% einbindenden Dateien latex-korrekturansicht-abspann.tex bzw.
%% latex-leseansicht-abspann.tex).
%% ---------------------------------------------------------------

\normalsize

% Das esempio-Environment wird nur in der Leseansicht benötigt
\ifkorrekturansicht\else
\newenvironment{esempio}[3]%
{
    \vspace{1.5ex}
    \rlap{\underline{#1}}
    \par
    \setlength{\parindent}{0cm}
    \nopagebreak
    \leftskip=#2cm
    \rightskip=#3cm
}
{
    \par
}
\fi

\doendnotes{C}
\bigskip
\vfill

\clearpage

\footnotesize

\ifkorrekturansicht
  \lohead{\textsc{register}}
\fi

% theindex-Environment neu definieren ohne reledmac
\makeatletter
\renewenvironment{theindex}{%
  \ifkorrekturansicht
    \section*{\indexname}%
  \else
    \subsubsection*{Index der erwähnten Entitäten}%
  \fi
  \setlength{\parindent}{0pt}%
  \setlength{\parskip}{0pt plus 0.3pt}%
  \let\item\@idxitem
}{%
  \ifkorrekturansicht\clearpage\fi
}
\makeatother

\IfFileExists{\jobname-pw.ind}{\input{\jobname-pw.ind}}{}

% Quellenangabe nur in der Leseansicht
\ifkorrekturansicht\else
% Fallback-Definitionen, falls die .tex-Datei \titel etc. nicht gesetzt hat
\providecommand{\titel}{}
\providecommand{\editorInnen}{}
\providecommand{\dateiname}{\jobname}

\vspace{3cm}

\vfill

\footnotesize
\textsc{Quelle}: \titel. Herausgegeben von {\editorInnen}. In: \emph{Arthur Schnitzler: Briefwechsel mit Autorinnen und Autoren}.
 Digitale Edition, https://schnitzler-briefe.acdh.oeaw.ac.at/{\dateiname}.html (Stand \today)
\fi

\end{document}


