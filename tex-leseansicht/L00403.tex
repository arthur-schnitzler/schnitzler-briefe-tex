\input{../tex-inputs/latex-pdf-vorspann}
\begin{center}
            \textcolor{red}{ENTWURF. ENTZIFFERUNG NOCH NICHT KORREKTURGELESEN}
                      \end{center}
            
               \section[Arthur Schnitzler an Max Burckhard, 21. 11. 1894]{ Arthur Schnitzler an Max Burckhard, 21. 11. 1894}\nopagebreak\mylabel{v}\rehead{ }\begin{ledgroupsized}[t]{13cm}\normalsize\beginnumbering\briefempfaengerindex{Burckhard, Max Eugen@\textsc{Burckhard, Max Eugen}!zzzSchnitzler, Arthur@\emph{von Arthur Schnitzler}!1894-11-211@{21. 11. 1894}|(be} \toendnotes[C]{\smallbreak\pagebreak[2]} \buchAlsQuelle{\pwindex{Glossy, Karl 07.03.1848 – 09.09.1937@\textsc{Glossy, Karl} (07.03.1848 – 09.09.1937), \emph{Schriftsteller, Museumsleiter, Zensurbeirat}!Schnitzlers Einzug ins Burgtheater19. 12. 1931@\strich\emph{Schnitzlers Einzug ins Burgtheater} {[}19. 12. 1931{]}|pwk}\pwindex{Neue Freie Presse1864 – 1939@\emph{Neue Freie Presse}|pwk}Karl Glossy: \emph{Schnitzlers Einzug ins Burgtheater. Unbekannte Briefe des Dichters.} In: \emph{Neue Freie Presse}, Nr. 24162, 19. 12. 1931, S. 14.}\buchAbdrucke{\weitereDrucke{1) \pwindex{Glossy, Karl 07.03.1848 – 09.09.1937@\textsc{Glossy, Karl} (07.03.1848 – 09.09.1937), \emph{Schriftsteller, Museumsleiter, Zensurbeirat}!Schnitzlers Einzug ins Burgtheater19. 12. 1931@\strich\emph{Schnitzlers Einzug ins Burgtheater} {[}19. 12. 1931{]}|pwk}Karl Glossy: \emph{Schnitzlers Einzug ins Burgtheater. Unbekannte Briefe des Dichters.} In: \emph{Wiener Studien und Dokumente}. Zum 85. Geburtstag des Verfassers hg. von seinen Freunden. Wien: \emph{Steyrermühl} 1933, S. 166–168.} \weitereDrucke{2) Hans-Ulrich Lindken: \emph{Arthur Schnitzler. Aspekte und Akzente. Materialien zu Leben
                        und Werk}. Frankfurt am Main, Bern, Göttingen: \emph{Peter Lang} 1984, S. 243–246 (Europäische Hochschulschriften, Reihe 1, Deutsche Sprache und
                        Literatur, 754).} }\toendnotes[C]{\smallbreak}\pstart
           \noindent{}{\pb}\so{Schnitzler an Burckhard}, 21. November 1894:
               »Sehr geehrter Herr Direktor, ſollte ſich mein Stück\pwindex{Schnitzler, Arthur 15.05.1862 – 21.10.1931@\textsc{Schnitzler, Arthur} (15.05.1862 – 21.10.1931), \emph{Schriftsteller, Mediziner}!Liebelei. Schauspiel in drei Akten9. 10. 1895@\strich\emph{Liebelei. Schauspiel in drei Akten} {[}9. 10. 1895{]}|pwv} jetzt in Ihren Händen befinden, ſo würde ich bitten,
               es mir recht bald für einige Zeit – hoffentlich nicht für immer – ſenden zu wollen.
               Ich möchte es ſehr gern \label{K_L00403_1v}\edtext{jemandem}{\lemma{\textnormal{\emph{jemandem}}}\Cendnote{\textnormal{eventuell Adele Sandrock\pwindex{Sandrock, Adele 19.08.1863 – 30.08.1937@\textsc{Sandrock, Adele} (19.08.1863 – 30.08.1937), \emph{Schauspielerin}|pwk}, der er am
                  1. 12. 1894{ }\emph{Liebelei}\pwindex{Schnitzler, Arthur 15.05.1862 – 21.10.1931@\textsc{Schnitzler, Arthur} (15.05.1862 – 21.10.1931), \emph{Schriftsteller, Mediziner}!Liebelei. Schauspiel in drei Akten9. 10. 1895@\strich\emph{Liebelei. Schauspiel in drei Akten} {[}9. 10. 1895{]}|pwk} vorliest.}}}\label{K_L00403_1h} zeigen und kann die neue Abſchrift,
               die ich mir wieder nach meinem ſehr ſchlecht leſerlichen Manuſkript anfertigen laſſe,
               erſt im Laufe der nächſten Woche erhalten. Sollte ſich Frau Hohenfels\pwindex{Hohenfels, Stella 16.04.1854 – 21.02.1920@\textsc{Hohenfels, Stella} (16.04.1854 – 21.02.1920), \emph{Schauspielerin}|pw} intereſſieren, in günſtigem Sinne entſcheiden – um ſo
               beſſer. Wenn nicht, ſo werde ich mir jedenfalls erlauben, auf Ihren liebenswürdigen
               Vorſchlag in Betreff Frau Sorma\pwindex{Sorma, Agnes 17.05.1862 – 10.02.1927@\textsc{Sorma, Agnes} (17.05.1862 – 10.02.1927), \emph{Schauspielerin}|pw} zurückzukommen.
               Ich kann dieſe Gelegenheit nicht vorübergehen laſſen, ohne Ihnen wieder, mein ſehr
               verehrter Herr Direktor, für Ihre Freundlichkeit und Ihre Bemühungen aufs
               allerwärmſte zu danken. Ihr Entgegenkommen läßt mich noch immer an einen
               ſchließlichen Erfolg glauben. Ihr Sie aufrichtig hochſchätzender Arthur
               Schnitzler.«\pend
           \endnumbering\briefempfaengerindex{Burckhard, Max Eugen@\textsc{Burckhard, Max Eugen}!zzzSchnitzler, Arthur@\emph{von Arthur Schnitzler}!1894-11-211@{21. 11. 1894}|)be}\mylabel{h}\end{ledgroupsized}  \newcommand{\dateiname}{L00403}\newcommand{\titel}{Arthur Schnitzler an Max Burckhard, 21. 11. 1894}\newcommand{\editorInnen}{Martin Anton Müller und Gerd-Hermann Susen}\input{../tex-inputs/latex-pdf-abspann}
      