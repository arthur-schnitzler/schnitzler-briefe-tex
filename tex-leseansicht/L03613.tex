%% latex-leseansicht-vorspann.tex
%% Vorspann für die Leseansicht.
%% Lädt die gemeinsame Datei latex-vorspann.tex mit nicht gesetztem Schalter.

\newif\ifkorrekturansicht
\korrekturansichtfalse

\input{../tex-inputs/latex-vorspann}

\begin{center}
            \textcolor{red}{ENTWURF, NICHT FERTIG KORRIGIERT}
                      \end{center}
            
         
         \renewcommand{\erwaehntePersonen}{Personen: Robert Adam, Felix Salten}
         \renewcommand{\erwaehnteInstitutionen}{Institutionen: S. Fischer Verlag}
         \renewcommand{\erwaehnteOrte}{Orte: Berlin, Wien}
         \renewcommand{\erwaehnteWerke}{Werke: Komödie der Worte. Drei Einakter}
               \section[Arthur Schnitzler: Widmungsexemplar Komödie der Worte für Felix Salten, {[}zwischen 12. und 20.?{]} 10. 1915]{ Arthur Schnitzler: Widmungsexemplar Komödie der Worte für Felix Salten,
               {[}zwischen 12. und 20.?{]} 10. 1915}\nopagebreak\mylabel{v}\rehead{ }\begin{ledgroupsized}[t]{13cm}\normalsize\beginnumbering \toendnotes[C]{\smallbreak\pagebreak[2]} \Standort{Wienbibliothek im Rathaus, A-61459/4.Ex., DS-2019-4166.}
\physDesc{Widmung am Vorsatzblatt, 61 Zeichen
\newline{}Handschrift: schwarze Tinte, deutsche Kurrent}\toendnotes[C]{\smallbreak}\pstart
           \noindent{}{\pb}Meinem lieben Felix Salten\pend
           \pstart
           herzlichſt {\\[\baselineskip]}\spacefill\mbox{Arthur Sch}\pend
           \leftskip=0em{}\pstart
           Wien\oindex{Wien@\textbf{Wien}|pw}{ }\label{K_L03613-1v}\edtext{Oct. 915}{\lemma{\textnormal{\emph{Oct. 915}}}\Cendnote{\textnormal{Das Stück\pwindex{Schnitzler, Arthur 15.05.1862 – 21.10.1931@\textsc{Schnitzler, Arthur} (15.05.1862 – 21.10.1931), \emph{Schriftsteller, Mediziner}!Komoedie der Worte. Drei Einakter1915-10-12@\strich\emph{Komödie der Worte. Drei Einakter} {[}1915-10-12{]}|pwkv} hatte am 12. 10. 1915 die Uraufführung. Dieses Datum dürfte also den
                        frühesten Erscheinungstag und damit eine Begrenzung der Datierung nach
                        hinten darstellen. Robert Adam\pwindex{Adam, Robert 20.04.1877 – 16.10.1961@\textsc{Adam, Robert} (20.04.1877 – 16.10.1961), \emph{Schriftsteller, Richter}|pwk}
                        bestätigte den Erhalt seines Exemplars am 21. 10. 1915, womit anzunehmen ist, dass Schnitzler\pwindex{Schnitzler, Arthur 15.05.1862 – 21.10.1931@\textsc{Schnitzler, Arthur} (15.05.1862 – 21.10.1931), \emph{Schriftsteller, Mediziner}|pwk} zu diesem Zeitpunkt die
                        Widmungsexemplare fertiggestellt hatte.}}}\label{K_L03613-1h}\pend
           {\bigskip}\pstart
           \noindent{}\centering{}{\pb}\textcolor{gray}{\textbf{\so{Komödie der Worte}\pwindex{Schnitzler, Arthur 15.05.1862 – 21.10.1931@\textsc{Schnitzler, Arthur} (15.05.1862 – 21.10.1931), \emph{Schriftsteller, Mediziner}!Komoedie der Worte. Drei Einakter1915-10-12@\strich\emph{Komödie der Worte. Drei Einakter} {[}1915-10-12{]}|pw}}}\pend
           \pstart
           \noindent{}\centering{}\textcolor{gray}{\textbf{Drei Einakter}}{\\}\textcolor{gray}{\textbf{von}}{\\}\textcolor{gray}{\textbf{\so{Arthur Schnitzler}}}\pend
           {\bigskip}\pstart
           \noindent{}\centering{}\textcolor{gray}{\textbf{S. Fiſcher, Verlag\orgindex{S. Fischer Verlag@S. Fischer Verlag|pw}, Berlin\oindex{Berlin@\textbf{Berlin}|pw}}}\pend
           \pstart
           \noindent{}\centering{}\textcolor{gray}{\textbf{1915}}\pend
           
         
         \endnumbering\mylabel{h}\end{ledgroupsized}  \newcommand{\dateiname}{L03613}\newcommand{\titel}{Arthur Schnitzler: Widmungsexemplar Komödie der Worte für Felix Salten, [zwischen 12. und 20.?] 10. 1915}\newcommand{\editorInnen}{Martin Anton Müller und Laura Untner}%% latex-leseansicht-abspann.tex
%% Abspann für die Leseansicht.
%% Der Schalter \ifkorrekturansicht ist bereits durch den Vorspann gesetzt.

%% latex-abspann.tex
%% Gemeinsamer Abspann für Korrekturansicht und Leseansicht.
%% Setzt den Schalter \ifkorrekturansicht voraus (gesetzt in den
%% einbindenden Dateien latex-korrekturansicht-abspann.tex bzw.
%% latex-leseansicht-abspann.tex).
%% ---------------------------------------------------------------

\normalsize

% Das esempio-Environment wird nur in der Leseansicht benötigt
\ifkorrekturansicht\else
\newenvironment{esempio}[3]%
{
    \vspace{1.5ex}
    \rlap{\underline{#1}}
    \par
    \setlength{\parindent}{0cm}
    \nopagebreak
    \leftskip=#2cm
    \rightskip=#3cm
}
{
    \par
}
\fi

\doendnotes{C}
\bigskip
\vfill

\clearpage

\footnotesize

\ifkorrekturansicht
  \lohead{\textsc{register}}
\fi

% theindex-Environment neu definieren ohne reledmac
\makeatletter
\renewenvironment{theindex}{%
  \ifkorrekturansicht
    \section*{\indexname}%
  \else
    \subsubsection*{Index der erwähnten Entitäten}%
  \fi
  \setlength{\parindent}{0pt}%
  \setlength{\parskip}{0pt plus 0.3pt}%
  \let\item\@idxitem
}{%
  \ifkorrekturansicht\clearpage\fi
}
\makeatother

\IfFileExists{\jobname-pw.ind}{\input{\jobname-pw.ind}}{}

% Quellenangabe nur in der Leseansicht
\ifkorrekturansicht\else
% Fallback-Definitionen, falls die .tex-Datei \titel etc. nicht gesetzt hat
\providecommand{\titel}{}
\providecommand{\editorInnen}{}
\providecommand{\dateiname}{\jobname}

\vspace{3cm}

\vfill

\footnotesize
\textsc{Quelle}: \titel. Herausgegeben von {\editorInnen}. In: \emph{Arthur Schnitzler: Briefwechsel mit Autorinnen und Autoren}.
 Digitale Edition, https://schnitzler-briefe.acdh.oeaw.ac.at/{\dateiname}.html (Stand \today)
\fi

\end{document}


      