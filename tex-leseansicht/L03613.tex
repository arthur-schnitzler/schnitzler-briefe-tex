%% latex-korrekturansicht-vorspann.tex
%% Vorspann für die Korrekturansicht.
%% Lädt die gemeinsame Datei latex-vorspann.tex mit gesetztem Schalter.

\newif\ifkorrekturansicht
\korrekturansichttrue

\input{../tex-inputs/latex-vorspann}


\section[Arthur Schnitzler: Widmungsexemplar Komödie der Worte für Felix Salten, {[}zwischen 18. und 20.?{]} 10. 1915]{L03613 Arthur Schnitzler: Widmungsexemplar Komödie der Worte für Felix Salten,
               {[}zwischen 18. und 20.?{]} 10. 1915}
\nopagebreak\mylabel{L03613v}
\rehead{ }\normalsize\beginnumbering\briefempfaengerindex{Salten, Felix@\textsc{Salten, Felix}!zzzSchnitzler, Arthur@\emph{von Arthur Schnitzler}!1915-10-201@{{[}zwischen 18. und 20.?{]} 10. 1915}|(be}
\toendnotes[C]{\smallbreak\pagebreak[2]}\Standort{Wienbibliothek im Rathaus, A-61459/4.Ex., DS-2019-4166.}
\physDesc{Widmung am Vorsatzblatt, 59 Zeichen
\newline{}Handschrift: schwarze Tinte, deutsche Kurrent}\toendnotes[C]{\smallbreak}
\pstart
           \noindent{}{\pb}Meinem lieben Felix Salten\pend
           
\pstart
           herzlichſt {\\[\baselineskip]}\spacefill\mbox{Arthur S}\pend
           \leftskip=0em{}
\pstart
           Wien\oindex{Wien@\textbf{Wien}, \emph{A.ADM2}|pw}{ }\label{K_L03613-1v}\edtext{Oct. 915}{\lemma{\textnormal{\emph{Oct. 915}}}\Cendnote{\textnormal{Am 12. 10. 1915 fand die Uraufführung des Stücks\pwindex{Komoedie der Worte. Drei Einakter@\emph{Komödie der Worte. Drei Einakter}|pwkv} statt. Dieses Datum stellt eine Begrenzung der Datierung
                        nach vorne dar. Robert Adam\pwindex{Adam, Robert 20.04.1877 – 16.10.1961@\textsc{Adam, Robert} (20.04.1877 – 16.10.1961), \emph{Schriftsteller/Schriftstellerin, Richter/Richterin}|pwk}
                        bestätigte den Erhalt seines Exemplars am 21. 10. 1915 und auch das Exemplar, das Georg Brandes\pwindex{Brandes, Georg 04.02.1842 – 19.02.1927@\textsc{Brandes, Georg} (04.02.1842 – 19.02.1927)|pwk}
                        erhielt, dürfte am 20. 10. 1915 abgeschickt
                        worden sein (vgl. Georg Brandes an Arthur Schnitzler, 4. 12. 1915). Damit ist anzunehmen, dass Schnitzler um den 20. 10. 1915 die
                        Widmungsexemplare versandte.}}}\label{K_L03613-1}\pend
           \selectlanguage{ngerman}\vspace{1em}{\vspace{1\baselineskip}}
\pstart
           \centering{}{\pb}\textcolor{gray}{\textbf{\so{Komödie der Worte}\pwindex{Komoedie der Worte. Drei Einakter@\emph{Komödie der Worte. Drei Einakter}|pw}}}\pend
           
\pstart
           \centering{}\textcolor{gray}{\textbf{\so{Drei Einakter}}}{\\}\textcolor{gray}{\textbf{\so{von}}}{\\}\textcolor{gray}{\textbf{\so{Arthur Schnitzler}}}\pend
           {\vspace{1\baselineskip}}
\pstart
           \centering{}\textcolor{gray}{\textbf{\so{S. Fiſcher, Verlag}\orgindex{S. Fischer Verlag@S. Fischer Verlag|pw}\so{,}\hspace*{1em}\so{Berlin}\oindex{Berlin@\textbf{Berlin}, \emph{P.PPLC}|pw}}}\pend
           
\pstart
           \centering{}\textcolor{gray}{\textbf{\so{1915}}}\pend
           \selectlanguage{ngerman}\endnumbering\briefempfaengerindex{Salten, Felix@\textsc{Salten, Felix}!zzzSchnitzler, Arthur@\emph{von Arthur Schnitzler}!1915-10-181@{{[}zwischen 18. und 20.?{]} 10. 1915}|)be}\mylabel{L03613h}  \normalsize

\doendnotes{C}
\bigskip
\vfill

\clearpage

\footnotesize

\lohead{\textsc{register}}

% Definiere theindex-Environment komplett neu ohne reledmac
\makeatletter
\renewenvironment{theindex}{%
  \section*{\indexname}%
  \setlength{\parindent}{0pt}%
  \setlength{\parskip}{0pt plus 0.3pt}%
  \let\item\@idxitem
}{%
  \clearpage
}
\makeatother

\IfFileExists{\jobname-pw.ind}{\input{\jobname-pw.ind}}{}

\end{document}

      