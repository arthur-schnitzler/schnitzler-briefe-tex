%% latex-leseansicht-vorspann.tex
%% Vorspann für die Leseansicht.
%% Lädt die gemeinsame Datei latex-vorspann.tex mit nicht gesetztem Schalter.

\newif\ifkorrekturansicht
\korrekturansichtfalse

\input{../tex-inputs/latex-vorspann}


\section[Hermann Bahr an Arthur Schnitzler, 19. {[}10. 1904{]}]{L01458 Hermann Bahr an Arthur Schnitzler, 19. [10. 1904]}
\nopagebreak\mylabel{L01458v}
\rehead{ }\normalsize\beginnumbering\briefempfaengerindex{Schnitzler, Arthur@\textsc{Schnitzler, Arthur}!zzzBahr, Hermann@\emph{von Hermann Bahr}!1904-10-191@{19. [10. 1904]}|(be}
\toendnotes[C]{\smallbreak\pagebreak[2]}
\correspDesc{Versand  durch Hermann Bahr am 19. [10. 1904] in Wien
\newline{}Erhalt  durch Arthur Schnitzler im Zeitraum [19. 10. 1904 – 23. 10. 1904?] in Wien}\toendnotes[C]{\smallbreak}
\Standort{CUL, Schnitzler, B 5b.}
\physDesc{Brief, 1 Blatt, 1 Seite, 252 Zeichen
\newline{}Handschrift: schwarze Tinte, deutsche Kurrent
\newline{}Schnitzler: mit Bleistift Monats- und Jahresangabe ergänzt: »10. 904« ergänzt 
\newline{}Ordnung: mit Bleistift von unbekannter Hand nummeriert:
                                    »121« }
\buchAbdrucke{\weitereDrucke{Hermann Bahr, Arthur Schnitzler: \emph{Briefwechsel, Aufzeichnungen, Dokumente (1891–1931)}. Herausgegeben von Kurt Ifkovits und Martin Anton Müller. Göttingen: \emph{Wallstein} 2018, S. 325.} }\toendnotes[C]{\smallbreak}
\pstart
           \raggedleft{}{\pb}19. früh\pend
           
\pstart\center{}Lieber Arthur!\pend\vspace{0.5em}
\pstart
           Leider reiſe ich eben nach \label{K_L01458-1v}\edtext{\textsc{Ragusa}\oindex{Dubrovnik@\textbf{Dubrovnik}|pw}}{\lemma{\textnormal{\emph{Ragusa}}}\Cendnote{\textnormal{Bahr\pwindex{Bahr, Hermann 19.\,7.\,1863 Linz – 15.\,1.\,1934 München@\textsc{Bahr, Hermann} (19.\,7.\,1863 Linz – 15.\,1.\,1934 München), \emph{Schriftsteller, Kritiker}|pwk} war vom 19. bis
                     23. 10. 1904 in Dalmatien\oindex{Dalmatien@\textbf{Dalmatien}, \emph{Ehemalige Region}|pwk}.}}}\label{K_L01458-1}. Hoffentlich nächſtens einmal.\pend
           
\pstart
           Deine Frau\pwindex{Schnitzler, Olga 17.\,1.\,1882 Wien – 13.\,1.\,1970 Lugano@\textsc{Schnitzler, Olga} (17.\,1.\,1882 Wien – 13.\,1.\,1970 Lugano), \emph{Schauspielerin, Sängerin}|pwv}, die ich
               herzlichſt grüße,{ }ſoll jedenfalls zu den Schweſtern Flöge\pwindex{Flöge, Pauline 21.\,12.\,1866 – 3.\,7.\,1917 Wien@\textsc{Flöge, Pauline} (21.\,12.\,1866 – 3.\,7.\,1917 Wien), \emph{Modistin}|pw}\pwindex{Flöge, Helene 20.\,5.\,1871 Wien – 26.\,1.\,1926 ebd.@\textsc{Flöge, Helene} (20.\,5.\,1871 Wien – 26.\,1.\,1926 ebd.), \emph{Modistin}|pw}\pwindex{Flöge, Emilie 30.\,8.\,1874 Wien – 26.\,5.\,1952 ebd.@\textsc{Flöge, Emilie} (30.\,8.\,1874 Wien – 26.\,5.\,1952 ebd.), \emph{Modistin, Modeschöpferin}|pw} gehen Mariahilferſtr. 1\oindex{Wien@\textbf{Wien}!VI., Mariahilf@\textbf{VI., Mariahilf}!Mariahilfer Straße@\textbf{Mariahilfer Straße}, \emph{Straße}|pw} (\textsc{Casa \textcolor{gray}{p}iccola\oindex{Wien@\textbf{Wien}!VI., Mariahilf@\textbf{VI., Mariahilf}!Casa Piccola@\textbf{Casa Piccola}, \emph{Gebäude}|pw}}), die für die meinige\pwindex{Bahr, Rosa 26.\,10.\,1871 Prag – 17.\,2.\,1940 Berlin@\textsc{Bahr, Rosa} (26.\,10.\,1871 Prag – 17.\,2.\,1940 Berlin), \emph{Schauspielerin}|pwv}
               herrlich gearbeitet haben.\pend
           
\pstart
           Herzlichſt{\\[\baselineskip]}\spacefill\mbox{H.}\pend
           \leftskip=0em{}\selectlanguage{ngerman}\endnumbering\briefempfaengerindex{Schnitzler, Arthur@\textsc{Schnitzler, Arthur}!zzzBahr, Hermann@\emph{von Hermann Bahr}!1904-10-191@{19. [10. 1904]}|)be}\mylabel{L01458h}  \newcommand{\dateiname}{L01458}\newcommand{\titel}{Hermann Bahr an Arthur Schnitzler, 19. [10. 1904]}\newcommand{\editorInnen}{Herausgegeben von Martin Anton Müller}%% latex-leseansicht-abspann.tex
%% Abspann für die Leseansicht.
%% Der Schalter \ifkorrekturansicht ist bereits durch den Vorspann gesetzt.

%% latex-abspann.tex
%% Gemeinsamer Abspann für Korrekturansicht und Leseansicht.
%% Setzt den Schalter \ifkorrekturansicht voraus (gesetzt in den
%% einbindenden Dateien latex-korrekturansicht-abspann.tex bzw.
%% latex-leseansicht-abspann.tex).
%% ---------------------------------------------------------------

\normalsize

% Das esempio-Environment wird nur in der Leseansicht benötigt
\ifkorrekturansicht\else
\newenvironment{esempio}[3]%
{
    \vspace{1.5ex}
    \rlap{\underline{#1}}
    \par
    \setlength{\parindent}{0cm}
    \nopagebreak
    \leftskip=#2cm
    \rightskip=#3cm
}
{
    \par
}
\fi

\doendnotes{C}
\bigskip
\vfill

\clearpage

\footnotesize

\ifkorrekturansicht
  \lohead{\textsc{register}}
\fi

% theindex-Environment neu definieren ohne reledmac
\makeatletter
\renewenvironment{theindex}{%
  \ifkorrekturansicht
    \section*{\indexname}%
  \else
    \subsubsection*{Index der erwähnten Entitäten}%
  \fi
  \setlength{\parindent}{0pt}%
  \setlength{\parskip}{0pt plus 0.3pt}%
  \let\item\@idxitem
}{%
  \ifkorrekturansicht\clearpage\fi
}
\makeatother

\IfFileExists{\jobname-pw.ind}{\input{\jobname-pw.ind}}{}

% Quellenangabe nur in der Leseansicht
\ifkorrekturansicht\else
% Fallback-Definitionen, falls die .tex-Datei \titel etc. nicht gesetzt hat
\providecommand{\titel}{}
\providecommand{\editorInnen}{}
\providecommand{\dateiname}{\jobname}

\vspace{3cm}

\vfill

\footnotesize
\textsc{Quelle}: \titel. Herausgegeben von {\editorInnen}. In: \emph{Arthur Schnitzler: Briefwechsel mit Autorinnen und Autoren}.
 Digitale Edition, https://schnitzler-briefe.acdh.oeaw.ac.at/{\dateiname}.html (Stand \today)
\fi

\end{document}


