%% latex-korrekturansicht-vorspann.tex
%% Vorspann für die Korrekturansicht.
%% Lädt die gemeinsame Datei latex-vorspann.tex mit gesetztem Schalter.

\newif\ifkorrekturansicht
\korrekturansichttrue

\input{../tex-inputs/latex-vorspann}


\section[Hermann Bahr an Arthur Schnitzler, 19. {[}10. 1904{]}]{L01458 Hermann Bahr an Arthur Schnitzler, 19. {[}10. 1904{]}}
\nopagebreak\mylabel{L01458v}
\rehead{ }\normalsize\beginnumbering\briefempfaengerindex{Schnitzler, Arthur@\textsc{Schnitzler, Arthur}!zzzBahr, Hermann@\emph{von Hermann Bahr}!1904-10-191@{19. {[}10. 1904{]}}|(be}
\toendnotes[C]{\smallbreak\pagebreak[2]}\Standort{CUL, Schnitzler, B 5b.}
\physDesc{Brief, 1 Blatt, 1 Seite, 252 Zeichen
\newline{}Handschrift: schwarze Tinte, deutsche Kurrent
\newline{}Schnitzler: mit Bleistift Monats- und Jahresangabe ergänzt: »10. 904« ergänzt 
\newline{}Ordnung: mit Bleistift von unbekannter Hand nummeriert:
                                    »121« }
\buchAbdrucke{\weitereDrucke{Hermann Bahr, Arthur Schnitzler: \emph{Briefwechsel, Aufzeichnungen, Dokumente (1891–1931)}. Göttingen: \emph{Wallstein} 2018, S. 325.} }\toendnotes[C]{\smallbreak}
\pstart
           \raggedleft{}{\pb}19. früh\pend
           
\pstart\center{}Lieber Arthur!\pend\vspace{0.5em}
\pstart
           Leider reiſe ich eben nach \label{K_L01458-1v}\edtext{\textsc{Ragusa}\oindex{Dubrovnik@\textbf{Dubrovnik}, \emph{P.PPLA}|pw}}{\lemma{\textnormal{\emph{Ragusa}}}\Cendnote{\textnormal{Bahr\pwindex{Bahr, Hermann 19.07.1863 – 15.01.1934@\textsc{Bahr, Hermann} (19.07.1863 – 15.01.1934), \emph{Schriftsteller/Schriftstellerin, Kritiker/Kritikerin}|pwk} war vom 19. bis
                     23. 10. 1904 in Dalmatien\oindex{Dalmatien@\textbf{Dalmatien}, \emph{L.RGNH}|pwk}.}}}\label{K_L01458-1}. Hoffentlich nächſtens einmal.\pend
           
\pstart
           Deine Frau\pwindex{Schnitzler, Olga 17.01.1882 – 13.01.1970@\textsc{Schnitzler, Olga} (17.01.1882 – 13.01.1970), \emph{Schauspieler/Schauspielerin, Sänger/Sängerin}|pwv}, die ich
               herzlichſt grüße, ſoll jedenfalls zu den Schweſtern Flöge\pwindex{Floege, Pauline 1866-12-21 – 1917-07-03@\textsc{Flöge, Pauline} (1866-12-21 – 1917-07-03), \emph{Modist/Modistin}|pw}\pwindex{Floege, Helene 1871-05-20 – 1926-01-26@\textsc{Flöge, Helene} (1871-05-20 – 1926-01-26), \emph{Modist/Modistin}|pw}\pwindex{Floege, Emilie 1874-08-30 – 1952-5-26@\textsc{Flöge, Emilie} (1874-08-30 – 1952-5-26), \emph{Modist/Modistin, Modeschöpfer/Modeschöpferin}|pw} gehen Mariahilferſtr. 1\oindex{Mariahilfer Strasse@\textbf{Mariahilfer Straße}, \emph{Straße (K.STR)}|pw} (\textsc{Casa \textcolor{gray}{p}iccola\oindex{Casa Piccola@\textbf{Casa Piccola}, \emph{Gebäude (K.GBD)}|pw}}), die für die meinige\pwindex{Bahr, Rosa 26.10.1871 – 17.02.1940@\textsc{Bahr, Rosa} (26.10.1871 – 17.02.1940), \emph{Schauspieler/Schauspielerin}|pwv}
               herrlich gearbeitet haben.\pend
           
\pstart
           Herzlichſt{\\[\baselineskip]}\spacefill\mbox{H.}\pend
           \leftskip=0em{}\selectlanguage{ngerman}\endnumbering\briefempfaengerindex{Schnitzler, Arthur@\textsc{Schnitzler, Arthur}!zzzBahr, Hermann@\emph{von Hermann Bahr}!1904-10-191@{19. {[}10. 1904{]}}|)be}\mylabel{L01458h}  \normalsize

\doendnotes{C}
\bigskip
\vfill

\clearpage

\footnotesize

\lohead{\textsc{register}}

% Definiere theindex-Environment komplett neu ohne reledmac
\makeatletter
\renewenvironment{theindex}{%
  \section*{\indexname}%
  \setlength{\parindent}{0pt}%
  \setlength{\parskip}{0pt plus 0.3pt}%
  \let\item\@idxitem
}{%
  \clearpage
}
\makeatother

\IfFileExists{\jobname-pw.ind}{\input{\jobname-pw.ind}}{}

\end{document}

      