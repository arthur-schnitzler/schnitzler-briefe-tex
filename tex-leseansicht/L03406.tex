%% latex-leseansicht-vorspann.tex
%% Vorspann für die Leseansicht.
%% Lädt die gemeinsame Datei latex-vorspann.tex mit nicht gesetztem Schalter.

\newif\ifkorrekturansicht
\korrekturansichtfalse

\input{../tex-inputs/latex-vorspann}


         
         \renewcommand{\erwaehntePersonen}{Personen: Camill Hoffmann, Felix Salten}
         \renewcommand{\erwaehnteOrte}{Orte: Edmund-Weiß-Gasse 7, IX., Alsergrund, Wien, XVIII., Währing}
         \renewcommand{\erwaehnteWerke}{}
               \section[ Felix Salten an Arthur Schnitzler, 20. 1. 1905]{ Felix Salten an Arthur Schnitzler, 20. 1. 1905}\nopagebreak\mylabel{v}\rehead{ }\begin{ledgroupsized}[t]{13cm}\normalsize\beginnumbering \toendnotes[C]{\smallbreak\pagebreak[2]} \Standort{CUL, Schnitzler, B 89, B 1.}
\physDesc{Kartenbrief, 577 Zeichen
\newline{}Handschrift: schwarze Tinte, lateinische Kurrent
\newline{}Versand: 1) Stempel: »\nobreak{}\oindex{IX., Alsergrund@\textbf{IX., Alsergrund}|pwk}Wien 9/1 66, 20 I 05, 4 40 V\nobreak{}«.   2) Stempel: »\nobreak{}\oindex{XVIII., Waehring@\textbf{XVIII., Währing}|pwk}\textcolor{gray}{18/1 Wi}en 111, \textcolor{gray}{21.} 1. \textcolor{gray}{0}5, 5\textsuperscript{20}\nobreak{}«. 
\newline{}Schnitzler: mit Bleistift datiert: »20/1 905« 
\newline{}Ordnung: mit Bleistift von unbekannter Hand nummeriert: »198a« }\toendnotes[C]{\smallbreak}\pstart{}{\pb}Herrn D\textsuperscript{r} Arthur Schnitzler\pend{}\pstart{}Wien XVIII.\oindex{XVIII., Waehring@\textbf{XVIII., Währing}|pw}\pend{}\pstart{}Spöttelgaße 7\oindex{Edmund-Weiss-Gasse 7@\textbf{Edmund-Weiß-Gasse 7}|pw}.\pend{}{\bigskip}\pstart
           \noindent{}{\pb}Lieber Freund, selbstverständlich werde ich die Publication des
                  \label{K_L03406-1v}\edtext{Interviews}{\lemma{\textnormal{\emph{Interviews}}}\Cendnote{\textnormal{siehe A. S.: \emph{Tagebuch}, 19. 1. 1905 und 21. 1. 1905 sowie A. S.: \emph{»Das Zeitlose ist von kürzester Dauer«}, [Camill Hoffmann]: Wien – Berlin. Theaterfragen, 22. 1. 1905}}}\label{K_L03406-1h} verhindern. Herr Hoffmann\pwindex{Hoffmann, Camill 31.10.1878 – 01.10.1944@\textsc{Hoffmann, Camill} (31.10.1878 – 01.10.1944), \emph{Schriftsteller, Journalist}|pw} ist freilich
               sehr betrübt darüber und wird versuchen Ihnen das, was er geschrieben hat,
               vorzulegen. Wenn Sie mir aber nicht direct, oder durch K. Hoffmann\pwindex{Hoffmann, Camill 31.10.1878 – 01.10.1944@\textsc{Hoffmann, Camill} (31.10.1878 – 01.10.1944), \emph{Schriftsteller, Journalist}|pw} mittheilen, dass Sie Ihren Entschluß geändert haben,
               dann bleibt’s bei Ihrem heutigen Brief.\pend
           \pstart
           Es ist wol überflüßig, zu betonen, dass ich persönlich dabei garnicht in Frage komme,
               und dass Sie sich \uline{nicht etwa durch eine Rücksicht auf
                  mich} sollen bestimmen laßen!\pend
           \pstart
           Herzlichst {\\[\baselineskip]}Ihr \spacefill\mbox{Salten}\pend
           \leftskip=0em{}
         
         \endnumbering\mylabel{h}\end{ledgroupsized}  \newcommand{\dateiname}{L03406}\newcommand{\titel}{Felix Salten an Arthur Schnitzler, 20. 1. 1905}\newcommand{\editorInnen}{Martin Anton Müller und Laura Untner}%% latex-leseansicht-abspann.tex
%% Abspann für die Leseansicht.
%% Der Schalter \ifkorrekturansicht ist bereits durch den Vorspann gesetzt.

%% latex-abspann.tex
%% Gemeinsamer Abspann für Korrekturansicht und Leseansicht.
%% Setzt den Schalter \ifkorrekturansicht voraus (gesetzt in den
%% einbindenden Dateien latex-korrekturansicht-abspann.tex bzw.
%% latex-leseansicht-abspann.tex).
%% ---------------------------------------------------------------

\normalsize

% Das esempio-Environment wird nur in der Leseansicht benötigt
\ifkorrekturansicht\else
\newenvironment{esempio}[3]%
{
    \vspace{1.5ex}
    \rlap{\underline{#1}}
    \par
    \setlength{\parindent}{0cm}
    \nopagebreak
    \leftskip=#2cm
    \rightskip=#3cm
}
{
    \par
}
\fi

\doendnotes{C}
\bigskip
\vfill

\clearpage

\footnotesize

\ifkorrekturansicht
  \lohead{\textsc{register}}
\fi

% theindex-Environment neu definieren ohne reledmac
\makeatletter
\renewenvironment{theindex}{%
  \ifkorrekturansicht
    \section*{\indexname}%
  \else
    \subsubsection*{Index der erwähnten Entitäten}%
  \fi
  \setlength{\parindent}{0pt}%
  \setlength{\parskip}{0pt plus 0.3pt}%
  \let\item\@idxitem
}{%
  \ifkorrekturansicht\clearpage\fi
}
\makeatother

\IfFileExists{\jobname-pw.ind}{\input{\jobname-pw.ind}}{}

% Quellenangabe nur in der Leseansicht
\ifkorrekturansicht\else
% Fallback-Definitionen, falls die .tex-Datei \titel etc. nicht gesetzt hat
\providecommand{\titel}{}
\providecommand{\editorInnen}{}
\providecommand{\dateiname}{\jobname}

\vspace{3cm}

\vfill

\footnotesize
\textsc{Quelle}: \titel. Herausgegeben von {\editorInnen}. In: \emph{Arthur Schnitzler: Briefwechsel mit Autorinnen und Autoren}.
 Digitale Edition, https://schnitzler-briefe.acdh.oeaw.ac.at/{\dateiname}.html (Stand \today)
\fi

\end{document}


      