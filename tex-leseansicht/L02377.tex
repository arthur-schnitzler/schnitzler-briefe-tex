%% latex-leseansicht-vorspann.tex
%% Vorspann für die Leseansicht.
%% Lädt die gemeinsame Datei latex-vorspann.tex mit nicht gesetztem Schalter.

\newif\ifkorrekturansicht
\korrekturansichtfalse

\input{../tex-inputs/latex-vorspann}


\section[Arthur Schnitzler an Gerhart Hauptmann, 17. 3. 1922]{L02377 Arthur Schnitzler an Gerhart Hauptmann, 17. 3. 1922}
\nopagebreak\mylabel{L02377v}
\rehead{ }\normalsize\beginnumbering\briefempfaengerindex{Hauptmann, Gerhart@\textsc{Hauptmann, Gerhart}!zzzSchnitzler, Arthur@\emph{von Arthur Schnitzler}!1922-03-171@{17. 3. 1922}|(be}
\toendnotes[C]{\smallbreak\pagebreak[2]}
\correspDesc{Versand  durch Arthur Schnitzler am 17. 3. 1922 in Wien
\newline{}Erhalt  durch Gerhart Hauptmann im Zeitraum [17. 3. 1922
                  – 21. 3. 1922?] \textbf{Ort fehlend} }\toendnotes[C]{\smallbreak}
\buchAlsQuelle{\emph{Festschrift zum 60. Geburtstag Gerhart Hauptmanns}. Im Auftrage der Genossenschaft Deutscher Bühnenangehöriger herausgegeben von Felix Hollaender. Berlin: \emph{Rudolf Mosse} 1922, S. 4.}
\buchAbdrucke{\weitereDrucke{1) \emph{Programmheft des deutschen Volkstheaters}, Nr. 7, 1962/1963, S. 2 und 5.} \weitereDrucke{2) Arthur Schnitzler: \emph{Kritisches. Gerhart Hauptmann. Zum 60. Geburtstag.} In: \emph{Neue Zürcher Zeitung}, 28. 10. 1962.} \weitereDrucke{3) Arthur Schnitzler: \emph{Briefe 1913–1931}. Herausgegeben von Peter Michael Braunwarth, Richard Miklin, Susanne Pertlik und Heinrich Schnitzler. Frankfurt am Main: \emph{S. Fischer} 1984, S. 271–273.} \weitereDrucke{4) Hans-Ulrich Lindken: \emph{Arthur Schnitzler. Aspekte und Akzente. Materialien zu Leben
                        und Werk}. Frankfurt am Main, Bern, Göttingen: \emph{Peter Lang} 1984, S. 178–180 (Europäische Hochschulschriften, Reihe 1, Deutsche Sprache und
                        Literatur, 754).} }\toendnotes[C]{\smallbreak}
\pstart
           \noindent{}{\pb}\label{K_L02377-1v}\edtext{Verſuche ich}{\lemma{\textnormal{\emph{Versuche ich}}}\Cendnote{\textnormal{Ein Typoskript mit Textabweichungen (6 Bl., 6 S., mit
                  handschriftlichen Korrekturen Schnitzlers mit Bleistift) findet sich im Nachlass in der \emph{Cambridge University Library}\orgindex{Cambridge University Library@Cambridge University Library|pwk} (A 17,4) und ist abgedruckt in: \emph{Briefe 1913–1931}, S. 271–273.}}}\label{K_L02377-1}, mein
               lieber und verehrter Gerhart Hauptmann, während ich mich in Gedanken mit Ihrem
               bevorſtehenden Geburtstage beſchäftige, mir die einzelnen Momente unſerer
               Bekanntſchaft oder, wenn ich mich kühner ausdrücken{ }ſoll, die Geſchichte unſerer
               Beziehungen zu vergegenwärtigen,{ }ſo wundere ich mich{ }ſelbſt, wie{ }ſpärlich an Zahl und
               wie kurz gemeſſen die perſönlichen Begegnungen{ }ſind, die ich in meinem Gedächtnis
               verzeichnet finde. Ich denke des \label{K_L02377-2v}\edtext{Abends bei Brahm\pwindex{Brahm, Otto 5.\,2.\,1856 Hamburg – 28.\,11.\,1912 Berlin@\textsc{Brahm, Otto} (5.\,2.\,1856 Hamburg – 28.\,11.\,1912 Berlin), \emph{Theaterleiter, Regisseur}|pw}}{\lemma{\textnormal{\emph{Abends bei Brahm}}}\Cendnote{\textnormal{Vgl. A. S.: \emph{Tagebuch}, 28. 10. 1896.
               }}}\label{K_L02377-2} im Jahre 1896, an dem ich Sie kennenlernte –, eines \label{K_L02377-3v}\edtext{Spazierganges}{\lemma{\textnormal{\emph{Spazierganges}}}\Cendnote{\textnormal{Vgl. A. S.: \emph{Tagebuch}, 22. 1. 1899.
               }}}\label{K_L02377-3} in der Semmering\oindex{Semmering@\textbf{Semmering}, \emph{Verwaltungsgebiet}|pw}er Landſchaft im
                  Winter 1899, der grauverhängten, doch warmdurchleuchteten \label{K_L02377-4v}\edtext{Spätoktobertage 1902}{\lemma{\textnormal{\emph{Spätoktobertage 1902}}}\Cendnote{\textnormal{Vgl. A. S.: \emph{Tagebuch}, 19. 10. 1902.
               }}}\label{K_L02377-4} in Ihrem Agnetendorf\oindex{Jagniątków@\textbf{Jagniątków}|pw}, des traurigen
                  \label{K_L02377-5v}\edtext{Novembertages 1912}{\lemma{\textnormal{\emph{Novembertages 1912}}}\Cendnote{\textnormal{Vgl. A. S.: \emph{Tagebuch}, 1. 12. 1912.
               }}}\label{K_L02377-5}, an dem wir unſerem dahingeſchiedenen wunderbaren Freunde\pwindex{Brahm, Otto 5.\,2.\,1856 Hamburg – 28.\,11.\,1912 Berlin@\textsc{Brahm, Otto} (5.\,2.\,1856 Hamburg – 28.\,11.\,1912 Berlin), \emph{Theaterleiter, Regisseur}|pwv} in einer dämmerigen Halle
               Abſchiedsworte in den Sarg nachriefen –, und endlich einer letzten, vorläufig letzten
               harmloſen, doch nicht unbeſchwingten Unterhaltung in Wien\oindex{Wien@\textbf{Wien}, \emph{Verwaltungsgebiet}|pw}. Wenn ich{ }ſo, mit anderen mehr oder minder flüchtigen Begegnungen alle
               Stunden zuſammenrechne, in denen wir uns von Angeſicht zu Angeſicht gegenüberſaßen,{ }ſo kommt gewiß keine ganze Woche heraus. Wie erkläre ich’s mir nur, daß mir heute
               trotzdem zumute iſt, als richtete ich dieſe Worte nicht nur an den weltberühmten
               Dichter,{ }ſondern als dürfte ich zugleich zu einem Freunde{ }ſprechen, zu einem lieben
               alten Freunde, der mir das von Jahr zu Jahr in höherem Maße wurde – ohne{ }ſein oder
               mein Dazutun, einfach durch die Tatſache{ }ſeines Daſeins und Wirkens? Da ich mich von
               aufdringlichen Neigungen ziemlich frei weiß,{ }ſo iſt dieſes Gefühl zum Teil gewiß
               darin begründet, daß Künſtler Ihrer hohen und reinen Art, je entſchiedener{ }ſie der
               Welt gehören, eine immer wärmere Atmoſphäre der Menſchlichkeit und Beglückung um{ }ſich
               verbreiten, an der jeder Empfängliche, jeder Dankbare teilnehmen darf. Da aber nicht
               alle dieſe Dankbar-Empfänglichen{ }ſchon darum allein das Recht für{ }ſich in Anſpruch
               nehmen dürften, einen Mann wie Sie mit dem{ }ſtolzen Worte Freund zu grüßen,{ }ſo wage
               ich es, meine wirkliche oder eingebildete Berechtigung dazu aus der Empfindung
               herzuleiten, daß mir aus Ihrem Weſen, abgeſehen von jenem allgemein-zugänglichen
               Glanze, etwas entgegenſtrahlt, das in irgendeiner Weise mir ganz perſönlich gilt –
               vielleicht als einem, der ungefähr gleichaltrig mit Ihnen, dem gleichen Berufe
               hingegeben, nun{ }ſeit{ }ſo langer Zeit in beſcheidener Nachbarſchaft{ }ſeine Straße zieht
               und deſſen innige Bewunderung für Sie und Ihr Werk im Laufe dieſer Jahre nicht nur
               ihm{ }ſelbſt,{ }ſondern auch Ihnen immer{ }ſtärker bewußt wurde. Wenn der geheimnisvolle
               Satz von den Parallelen, die{ }ſich erſt in der Unendlichkeit begegnen, auch für
               Menſchenwege zutrifft, die in der gleichen Ebene laufen,{ }ſo mag er für Dichterwege
               ganz beſonders gelten, – und je mehr wir abendwärts wandeln, jener Unendlichkeit zu,
               die uns einmal alle umfangen wird, um{ }ſo mehr{ }ſcheinen für unſer{ }ſterbliches Auge{ }ſich dieſe Wege einander zu nähern und um{ }ſo vertrauter klingen Rufe aller Art
               zwiſchen den Wanderern hin und her. Wenn Sie heute, Gerhart Hauptmann, aus den meinen
               herausgehört haben, was Sie und Ihre Kunft mir bedeuten,{ }ſo will ich zufrieden{ }ſein
               und Ihnen nicht erſt ausdrücklich und ausführlich{ }ſagen, welche Wünſche ich Ihnen,
               mir und uns allen aus erfüllter Seele darbringe.\pend
           \selectlanguage{ngerman}\endnumbering\briefempfaengerindex{Hauptmann, Gerhart@\textsc{Hauptmann, Gerhart}!zzzSchnitzler, Arthur@\emph{von Arthur Schnitzler}!1922-03-171@{17. 3. 1922}|)be}\mylabel{L02377h}  \newcommand{\dateiname}{L02377}\newcommand{\titel}{Arthur Schnitzler an Gerhart Hauptmann, 17. 3. 1922}\newcommand{\editorInnen}{Herausgegeben von Martin Anton Müller}%% latex-leseansicht-abspann.tex
%% Abspann für die Leseansicht.
%% Der Schalter \ifkorrekturansicht ist bereits durch den Vorspann gesetzt.

%% latex-abspann.tex
%% Gemeinsamer Abspann für Korrekturansicht und Leseansicht.
%% Setzt den Schalter \ifkorrekturansicht voraus (gesetzt in den
%% einbindenden Dateien latex-korrekturansicht-abspann.tex bzw.
%% latex-leseansicht-abspann.tex).
%% ---------------------------------------------------------------

\normalsize

% Das esempio-Environment wird nur in der Leseansicht benötigt
\ifkorrekturansicht\else
\newenvironment{esempio}[3]%
{
    \vspace{1.5ex}
    \rlap{\underline{#1}}
    \par
    \setlength{\parindent}{0cm}
    \nopagebreak
    \leftskip=#2cm
    \rightskip=#3cm
}
{
    \par
}
\fi

\doendnotes{C}
\bigskip
\vfill

\clearpage

\footnotesize

\ifkorrekturansicht
  \lohead{\textsc{register}}
\fi

% theindex-Environment neu definieren ohne reledmac
\makeatletter
\renewenvironment{theindex}{%
  \ifkorrekturansicht
    \section*{\indexname}%
  \else
    \subsubsection*{Index der erwähnten Entitäten}%
  \fi
  \setlength{\parindent}{0pt}%
  \setlength{\parskip}{0pt plus 0.3pt}%
  \let\item\@idxitem
}{%
  \ifkorrekturansicht\clearpage\fi
}
\makeatother

\IfFileExists{\jobname-pw.ind}{\input{\jobname-pw.ind}}{}

% Quellenangabe nur in der Leseansicht
\ifkorrekturansicht\else
% Fallback-Definitionen, falls die .tex-Datei \titel etc. nicht gesetzt hat
\providecommand{\titel}{}
\providecommand{\editorInnen}{}
\providecommand{\dateiname}{\jobname}

\vspace{3cm}

\vfill

\footnotesize
\textsc{Quelle}: \titel. Herausgegeben von {\editorInnen}. In: \emph{Arthur Schnitzler: Briefwechsel mit Autorinnen und Autoren}.
 Digitale Edition, https://schnitzler-briefe.acdh.oeaw.ac.at/{\dateiname}.html (Stand \today)
\fi

\end{document}


