%% latex-leseansicht-vorspann.tex
%% Vorspann für die Leseansicht.
%% Lädt die gemeinsame Datei latex-vorspann.tex mit nicht gesetztem Schalter.

\newif\ifkorrekturansicht
\korrekturansichtfalse

\input{../tex-inputs/latex-vorspann}


               \section[Robert Adam an Arthur Schnitzler, 23. 11. 1917]{ Robert Adam an Arthur Schnitzler, 23. 11. 1917}\nopagebreak\mylabel{v}\rehead{ }\begin{ledgroupsized}[t]{13cm}\normalsize\beginnumbering\briefempfaengerindex{Schnitzler, Arthur@\textsc{Schnitzler, Arthur}!zzzAdam, Robert@\emph{von Robert Adam}!1917-11-231@{23. 11. 1917}|(be} \toendnotes[C]{\smallbreak\pagebreak[2]} \Standort{CUL, Schnitzler, B 1.}
\physDesc{Brief, 1 Blatt, 3 Seiten
\newline{}Handschrift: schwarze Tinte, deutsche Kurrent
\newline{}Schnitzler: 1) mit Bleistift beschriftet: »\textsc{Adam}« 2) mit rotem Buntstift zwei Unterstreichungen\newline{}Ordnung: von unbekannter Hand nummeriert: »2« }\Standort{Wien, Österreichische Nationalbibliothek, Cod.ser. 52.263, 205 recto.}
\physDesc{Brief, maschinelle Abschrift
\newline{}Schreibmaschine}\toendnotes[C]{\smallbreak}\pstart
           \raggedleft{}{\pb}Wien\oindex{Wien@\textbf{Wien}|pw}, am 23. November 1917\pend
           \pstart\center{}Hochverehrter Herr Doktor!\pend\pstart
           Empfangen Sie meinen herzlichſten Dank für Ihre neue Komödie\pwindex{Schnitzler, Arthur 15.05.1862 – 21.10.1931@\textsc{Schnitzler, Arthur} (15.05.1862 – 21.10.1931), \emph{Schriftsteller, Mediziner}!Fink und Fliederbusch. Komoedie in drei Akten1917@\strich\emph{Fink und Fliederbusch. Komödie in drei Akten} {[}1917{]}|pwv}, die mich, wie alles, was Ihrem Geiſte entſpringt,
               auf’s Höchſte gefeſſelt und befriedigt hat!\pend
           \pstart
           Nun, da ich ſie kenne, iſt mir das Geſchrei, das in den Theaterurteilen der
               Tagespreſſe erſcholl, vollkommen erklärlich. Die Herren zeichnen ſich vor allem durch
               große Wehleidigkeit aus und ſchrecken vor nichts ſo ſehr zurück als vor dem, was
               ihnen die Gefahr der Selbſterkenntnis droht. Sie wollen nur angreifen, nicht
               angegriffen werden, und wenn ſie ſchon einen Angriff hinnehmen müſſen, ſo ſoll doch
               nicht etwas wie Mitleid mit ihnen {\pb}darin
               vernehmbar ſein. Journaliſten und Weiber wollen voll genommen werden, in Liebe und
               Haß, in Krieg und Frieden. Sie aber haben ſie nicht voll genommen, und Sie haben ein
               weiteres Verbrechen begangen: Sie haben hinter das Dogma ein Fragezeichen geſetzt,
               auf dem der Weſensſtolz des Journaliſten ruht: daß »Geſinnung« den Mann mache (\textsc{my platform is my castle}). Nimmt man hinzu, daß in einigen
               Sätzen Ihres Leuchter\pwindex{Schnitzler, Arthur 15.05.1862 – 21.10.1931@\textsc{Schnitzler, Arthur} (15.05.1862 – 21.10.1931), \emph{Schriftsteller, Mediziner}!Fink und Fliederbusch. Komoedie in drei Akten1917@\strich\emph{Fink und Fliederbusch. Komödie in drei Akten} {[}1917{]}|pwv}
               Anſpielungen auf die Totſchweigepolitik des »\label{K_L02281_1v}\edtext{Trompeters von Jericho}{\lemma{\textnormal{\emph{Trompeters von Jericho}}}\Cendnote{\textnormal{unklare Anspielung}}}\label{K_L02281_1h}« erblickt werden konnten, ſo iſt der
               Zorn derer von der »\label{K_L02281_2v}\edtext{Gegenwart}{\lemma{\textnormal{\emph{Gegenwart}}}\Cendnote{\textnormal{das und das folgende fiktive Blätter aus
                     \emph{Fink und Fliederbusch}\pwindex{Schnitzler, Arthur 15.05.1862 – 21.10.1931@\textsc{Schnitzler, Arthur} (15.05.1862 – 21.10.1931), \emph{Schriftsteller, Mediziner}!Fink und Fliederbusch. Komoedie in drei Akten1917@\strich\emph{Fink und Fliederbusch. Komödie in drei Akten} {[}1917{]}|pwk}.}}}\label{K_L02281_2h}« noch
               erklärlicher; und die »Elegante Welt«, die Ihnen vieles noch nicht verziehen hat,
               geht eben mit. Sie haben ſich alle, alle doch ſolidariſch erklärt: ſie bleiben – im
               Grunde, was ſie ſind. –\pend
           \pstart
           Mit den herzlichſten Grüßen {\pb}und
               Empfehlungen Ihr ergebener{\\[\baselineskip]}\spacefill\mbox{Robert Adam}\pend
           \leftskip=0em{}\endnumbering\briefempfaengerindex{Schnitzler, Arthur@\textsc{Schnitzler, Arthur}!zzzAdam, Robert@\emph{von Robert Adam}!1917-11-231@{23. 11. 1917}|)be}\mylabel{h}\end{ledgroupsized}  \newcommand{\dateiname}{L02281}\newcommand{\titel}{Robert Adam an Arthur Schnitzler, 23. 11. 1917}\newcommand{\editorInnen}{Martin Anton Müller und Gerd-Hermann Susen}
            \footnotesize
\begin{ledgroupsized}[t]{11.5cm}
\doendnotes{C}
\end{ledgroupsized}
         %% latex-leseansicht-abspann.tex
%% Abspann für die Leseansicht.
%% Der Schalter \ifkorrekturansicht ist bereits durch den Vorspann gesetzt.

%% latex-abspann.tex
%% Gemeinsamer Abspann für Korrekturansicht und Leseansicht.
%% Setzt den Schalter \ifkorrekturansicht voraus (gesetzt in den
%% einbindenden Dateien latex-korrekturansicht-abspann.tex bzw.
%% latex-leseansicht-abspann.tex).
%% ---------------------------------------------------------------

\normalsize

% Das esempio-Environment wird nur in der Leseansicht benötigt
\ifkorrekturansicht\else
\newenvironment{esempio}[3]%
{
    \vspace{1.5ex}
    \rlap{\underline{#1}}
    \par
    \setlength{\parindent}{0cm}
    \nopagebreak
    \leftskip=#2cm
    \rightskip=#3cm
}
{
    \par
}
\fi

\doendnotes{C}
\bigskip
\vfill

\clearpage

\footnotesize

\ifkorrekturansicht
  \lohead{\textsc{register}}
\fi

% theindex-Environment neu definieren ohne reledmac
\makeatletter
\renewenvironment{theindex}{%
  \ifkorrekturansicht
    \section*{\indexname}%
  \else
    \subsubsection*{Index der erwähnten Entitäten}%
  \fi
  \setlength{\parindent}{0pt}%
  \setlength{\parskip}{0pt plus 0.3pt}%
  \let\item\@idxitem
}{%
  \ifkorrekturansicht\clearpage\fi
}
\makeatother

\IfFileExists{\jobname-pw.ind}{\input{\jobname-pw.ind}}{}

% Quellenangabe nur in der Leseansicht
\ifkorrekturansicht\else
% Fallback-Definitionen, falls die .tex-Datei \titel etc. nicht gesetzt hat
\providecommand{\titel}{}
\providecommand{\editorInnen}{}
\providecommand{\dateiname}{\jobname}

\vspace{3cm}

\vfill

\footnotesize
\textsc{Quelle}: \titel. Herausgegeben von {\editorInnen}. In: \emph{Arthur Schnitzler: Briefwechsel mit Autorinnen und Autoren}.
 Digitale Edition, https://schnitzler-briefe.acdh.oeaw.ac.at/{\dateiname}.html (Stand \today)
\fi

\end{document}


      