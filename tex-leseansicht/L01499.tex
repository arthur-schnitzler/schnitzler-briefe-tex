%% latex-korrekturansicht-vorspann.tex
%% Vorspann für die Korrekturansicht.
%% Lädt die gemeinsame Datei latex-vorspann.tex mit gesetztem Schalter.

\newif\ifkorrekturansicht
\korrekturansichttrue

\input{../tex-inputs/latex-vorspann}


\section[Hermann Bahr an Arthur Schnitzler, 9. 2. {[}1905{]}]{L01499 Hermann Bahr an Arthur Schnitzler, 9. 2. {[}1905{]}}
\nopagebreak\mylabel{L01499v}
\rehead{ }\normalsize\beginnumbering\briefempfaengerindex{Schnitzler, Arthur@\textsc{Schnitzler, Arthur}!zzzBahr, Hermann@\emph{von Hermann Bahr}!1905-02-091@{9. 2. 1905}|(be}
\toendnotes[C]{\smallbreak\pagebreak[2]}\Standort{CUL, Schnitzler, B 5b.}
\physDesc{Brief, 1 Blatt, 1 Seite, 389 Zeichen
\newline{}Handschrift: blaue Tinte, deutsche Kurrent
\newline{}Ordnung: mit Bleistift von unbekannter Hand nummeriert:
                                    »127« }
\buchAbdrucke{\weitereDrucke{Hermann Bahr, Arthur Schnitzler: \emph{Briefwechsel, Aufzeichnungen, Dokumente (1891–1931)}. Göttingen: \emph{Wallstein} 2018, S. 343.} }\toendnotes[C]{\smallbreak}
\pstart
           \raggedleft{}{\pb}9. 2.\pend
           
\pstart\center{}Lieber Arthur!\pend\vspace{0.5em}
\pstart
           Von den Leuten, bei welchen ich herumgefragt habe, weiß Niemand ein
                  \textcolor{gray}{deutſ}ches Wort für \textcolor{gray}{\label{K_L01499-1v}\edtext{\textsc{massier}}{\lemma{\textnormal{\emph{massier}}}\Cendnote{\textnormal{unsichere Lesart; ein ›massier‹ war
                     eine Art Waffenträger, teilweise auch als Leibwächter dienend, der bei
                     Zeremonien zum Einsatz kommt. In einer weiteren Bedeutung handelte es sich in einem
                     Künstleratelier um einen Schüler, der für Assistenzzwecke eingesetzt wurde und
                     der dafür zuständig war, das Lehrgeld einzuheben.}}}\label{K_L01499-1}}, ſchon deshalb nicht, weil wir die Inſtitution gar nicht haben.\pend
           
\pstart
           Ich habe mehrere anonyme Briefe bekommen, in welchen ich beſchimpft wurde, weil ich
                  »Freiwild\pwindex{Freiwild. Schauspiel in 3 Akten@\emph{Freiwild. Schauspiel in 3 Akten}|pw}«, Dein »beſtes Stück«, nicht genug
               gelobt hätte, denk Dir!\pend
           
\pstart
           Deine liebe Frau\pwindex{Schnitzler, Olga 17.01.1882 – 13.01.1970@\textsc{Schnitzler, Olga} (17.01.1882 – 13.01.1970), \emph{Schauspieler/Schauspielerin, Sänger/Sängerin}|pwv} und
               Dich herzlichſt grüßend bin ich{\\[\baselineskip]}Dein alter{\\[\baselineskip]}\spacefill\mbox{H.}\pend
           \leftskip=0em{}\selectlanguage{ngerman}\endnumbering\briefempfaengerindex{Schnitzler, Arthur@\textsc{Schnitzler, Arthur}!zzzBahr, Hermann@\emph{von Hermann Bahr}!1905-02-091@{9. 2. 1905}|)be}\mylabel{L01499h}  \normalsize

\doendnotes{C}
\bigskip
\vfill

\clearpage

\footnotesize

\lohead{\textsc{register}}

% Definiere theindex-Environment komplett neu ohne reledmac
\makeatletter
\renewenvironment{theindex}{%
  \section*{\indexname}%
  \setlength{\parindent}{0pt}%
  \setlength{\parskip}{0pt plus 0.3pt}%
  \let\item\@idxitem
}{%
  \clearpage
}
\makeatother

\IfFileExists{\jobname-pw.ind}{\input{\jobname-pw.ind}}{}

\end{document}

      