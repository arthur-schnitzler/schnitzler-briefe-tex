%% latex-leseansicht-vorspann.tex
%% Vorspann für die Leseansicht.
%% Lädt die gemeinsame Datei latex-vorspann.tex mit nicht gesetztem Schalter.

\newif\ifkorrekturansicht
\korrekturansichtfalse

\input{../tex-inputs/latex-vorspann}


         
         \renewcommand{\erwaehntePersonen}{Personen: Veronica ,  ?? [Schweinehirt in Bad Fusch], Hermann Bahr, Maurice Barrès, Georges Boulanger, Joseph von Eichendorff, Paul Goldmann, Hugo von Hofmannsthal,  Homer, Henry Irving, Guy de Maupassant, Friedrich Nietzsche, Laurence Oliphant, Jane Emily Łaszowska}
         \renewcommand{\erwaehnteInstitutionen}{Institutionen: E. Pierson’s Verlag, Linzer Volksblatt, S. Fischer Verlag, Schmeitzner Verlagsbuchhandlung}
         \renewcommand{\erwaehnteOrte}{Orte: Bad Fusch, Dresden, Dublin, Irland, Salzburg, Sri Lanka, Wien}
         \renewcommand{\erwaehnteWerke}{Werke: Der Garten der Bérenice, Gestern. Dramatische Studie in einem Akt in Versen, Menschliches, Allzumenschliches. Ein Buch für freie Geister, Moderne Rundschau, Russische Reise, Vorsatz [Einleitung Russische Reise]}
               \section[Hugo von Hofmannsthal an Arthur Schnitzler, 13. 7. {[}1891{]}]{ Hugo von Hofmannsthal an Arthur Schnitzler, 13. 7. {[}1891{]}}\nopagebreak\mylabel{v}\rehead{ }\begin{ledgroupsized}[t]{13cm}\normalsize\beginnumbering \toendnotes[C]{\smallbreak\pagebreak[2]} \Standort{CUL, Schnitzler, B 43.}
\physDesc{Brief, 1 Blatt, 3 Seiten, 2773 Zeichen (aufgeprägtes Wappen)
\newline{}Handschrift: schwarze Tinte, deutsche Kurrent
\newline{}Schnitzler: mit Bleistift die Jahreszahl hinzugefügt:
                                 »1891« 
\newline{}Ordnung: mit Bleistift von unbekannter Hand nummeriert:
                                 »4« }\Standort{FDH, Hs-29002.}
\physDesc{Briefentwurf, 1 Blatt, 1 Seite, 2773 Zeichen (Briefpapier mit aufgeprägtem Wappen)
\newline{}Handschrift: schwarze Tinte, deutsche Kurrent}\buchAbdrucke{\weitereDrucke{1) Hugo von Hofmannsthal: \emph{Briefe. 1890–1901}. Berlin: \emph{S. Fischer} 1935, S. 21–23.} \weitereDrucke{2) Hugo von Hofmannsthal, Arthur Schnitzler: \emph{Briefwechsel}. Hg. Therese Nickl und Heinrich Schnitzler. Frankfurt am Main: \emph{S. Fischer} 1964, S. 7–8.} }\toendnotes[C]{\smallbreak}\pstart
           \raggedleft{}{\pb}Bad Fuſch\oindex{Bad Fusch@\textbf{Bad Fusch}|pw}, 13 Juli.\pend
           \pstart
           Mir fehlt hier irgend etwas; was, weiß ich ſelbſt nicht. Vielleicht Sonne. Vielleicht
               Lärm. Dann wird wohl Salzburg\oindex{Salzburg@\textbf{Salzburg}|pw} helfen. Ich habe
               einen dicken \label{K_L00023-1v}\edtext{Paletot}{\lemma{\textnormal{\emph{Paletot}}}\Cendnote{\textnormal{Herrenmantel}}}\label{K_L00023-1h} an, auf dem Papier
               tanzen grelle kalte Lichter, der Brunnen plätſchert und es riecht nach reinlichen
               kleinen Kindern. Wenn das eine Stimmung iſt, ſo iſts zumindeſten nicht die, die ich
               brauchen kann. \label{K_L00023-2v}\edtext{\textsc{En attendant}}{\lemma{\textnormal{\emph{En attendant}}}\Cendnote{\textnormal{französisch: in Erwartung}}}\label{K_L00023-2h}{ }\label{K_L00023-3v}\edtext{les’ ich Nietzſche\pwindex{Nietzsche, Friedrich 15.10.1844 – 25.08.1900@\textsc{Nietzsche, Friedrich} (15.10.1844 – 25.08.1900), \emph{Schriftsteller, Philosoph}|pw}}{\lemma{\textnormal{\emph{les’ ich Nietzſche}}}\Cendnote{\textnormal{Friedrich Nietzsche: \emph{Menschliches,
                        Allzumenschliches. Ein Buch für freie Geister}\pwindex{Nietzsche, Friedrich 15.10.1844 – 25.08.1900@\textsc{Nietzsche, Friedrich} (15.10.1844 – 25.08.1900), \emph{Schriftsteller, Philosoph}!Menschliches, Allzumenschliches. Ein Buch fuer freie Geister1878@\strich\emph{Menschliches, Allzumenschliches. Ein Buch für freie Geister} {[}1878{]}|pwk}. Chemnitz: \emph{Schmeitzner}\orgindex{Schmeitzner Verlagsbuchhandlung@Schmeitzner Verlagsbuchhandlung|pwk}{ }1878.}}}\label{K_L00023-3h} und freue mich wie in ſeiner kalten Klarheit, der »\label{K_L00023-4v}\edtext{hellen Luft der Cordilleren}{\lemma{\textnormal{\emph{hellen … Cordilleren}}}\Cendnote{\textnormal{Hofmannsthal\pwindex{Hofmannsthal, Hugo von 1874-02-01 – 1929-07-15@\textsc{Hofmannsthal, Hugo von} (1874-02-01 – 1929-07-15), \emph{Schriftsteller}|pwk} markiert die Stelle eindeutig
                  als Zitat. Dabei variiert er zumindest seine eigenen Aufzeichnungen vom
                     21. 5. 1891: »In Nietzsche\pwindex{Nietzsche, Friedrich 15.10.1844 – 25.08.1900@\textsc{Nietzsche, Friedrich} (15.10.1844 – 25.08.1900), \emph{Schriftsteller, Philosoph}|pw} ist die freudige Klarheit der Zerstörung wie in einem einem
                     hellen Sturm der Cordilleren oder in dem reinen Lodern grosser
                  Flammen«. (Hugo von Hofmannsthal\pwindex{Hofmannsthal, Hugo von 1874-02-01 – 1929-07-15@\textsc{Hofmannsthal, Hugo von} (1874-02-01 – 1929-07-15), \emph{Schriftsteller}|pwk}: \emph{Aufzeichnungen}. Hg. Rudolf Hirsch † und Ellen Ritter † in
                     Zusammenarbeit mit Konrad Heumann und Peter Michael Braunwarth. Frankfurt am
                     Main: \emph{S. Fischer}\orgindex{S. Fischer Verlag@S. Fischer Verlag|pwk}{ }2013, S. 108 (\emph{Sämtliche Werke},
                     XXXIX).) Vgl. auch den Briefentwurf in der gedruckten Ausgabe,
                  S. 323.}}}\label{K_L00023-4h}«, meine eigenen Gedanken ſchön cryſtalliſieren. Ich denke ſehr
               viel, wie immer wenn mir nichts einfällt, und ſchlecke künftige Geburtstagstorten ab:
               das heißt, ich genieße in zahlloſen Plänen das Beſte von künftigen Arbeiten: das
               Grauen vor der tragiſchen Situation und die Freude am Combinieren. Wozu verdirbt man
               ſich das eigentlich alles, indem man die ſchlechteſte Momentphotographie davon
               feſthält und aufhebt? Dumme Frage {\pb}übrigens, Kunſt kommt von Können und Können heißt ſchreibenkönnen. (\label{K_L00023-5v}\edtext{\textsc{Mod. Rundschau}\pwindex{Moderne Rundschau1.4.1891 – 31.12.1891@\emph{Moderne Rundschau} {[}1.4.1891 – 31.12.1891{]}|pw} 5 u. 6 Heft, Seite 17{\dots}ff.}{\lemma{\textnormal{\emph{Mod. … Seite 17ff.}}}\Cendnote{\textnormal{Hermann Bahr\pwindex{Bahr, Hermann 19.07.1863 – 15.01.1934@\textsc{Bahr, Hermann} (19.07.1863 – 15.01.1934), \emph{Schriftsteller, Kritiker}|pwk}: \emph{Vorsatz.}\pwindex{Bahr, Hermann 19.07.1863 – 15.01.1934@\textsc{Bahr, Hermann} (19.07.1863 – 15.01.1934), \emph{Schriftsteller, Kritiker}!Vorsatz [Einleitung Russische Reise]15. 06. 1891@\strich\emph{Vorsatz [Einleitung Russische Reise]} {[}15. 06. 1891{]}|pwk} In: \emph{Moderne
                        Rundschau}\pwindex{Moderne Rundschau1.4.1891 – 31.12.1891@\emph{Moderne Rundschau} {[}1.4.1891 – 31.12.1891{]}|pwk}, Bd. 3, H. 5/6, 15. 6. 1891, S. 178–180.
                  Es handelt sich um die »Einleitung zu Bahr\pwindex{Bahr, Hermann 19.07.1863 – 15.01.1934@\textsc{Bahr, Hermann} (19.07.1863 – 15.01.1934), \emph{Schriftsteller, Kritiker}|pw}’s demnächst (bei E. Pierson\orgindex{E. Pierson s Verlag@E. Pierson’s Verlag|pw}
                     in Dresden\oindex{Dresden@\textbf{Dresden}|pw}) erscheinendem neuesten Buche:
                        ›Russische Reise, ein lyrischer
                        Zwischenakt\pwindex{Bahr, Hermann 19.07.1863 – 15.01.1934@\textsc{Bahr, Hermann} (19.07.1863 – 15.01.1934), \emph{Schriftsteller, Kritiker}!Russische Reise1891@\strich\emph{Russische Reise} {[}1891{]}|pw}‹.«}}}\label{K_L00023-5h})\pend
           \pstart
           So dumme Fragen frage ich nur wenn ich Gedanken denke ſtatt mein Leben zu leben. Ich
               möchte mich alſo verlieben, oder täglich \textsc{lawn-tennis}{ }ſpielen, oder meinetwegen \label{K_L00023-6v}\edtext{\textsc{Macao}}{\lemma{\textnormal{\emph{Macao}}}\Cendnote{\textnormal{Glücksspiel mit Karten}}}\label{K_L00023-6h}, oder
               ſonſt eine Beſchäftigung erleben.\pend
           \pstart
           Sonſt werd ich noch ein »ganzer Politiker«, wie der Sauhirt\pwindex{?? [Schweinehirt in Bad Fusch] @\textsc{?? [Schweinehirt in Bad Fusch]}|pwv} von ſeinem alten Vorſtehhund neulich ſagte, der aus
               Altersſchwäche dumm geworden iſt. Der Sauhirt\pwindex{?? [Schweinehirt in Bad Fusch] @\textsc{?? [Schweinehirt in Bad Fusch]}|pwv} iſt keine \textsc{Fiction},
               ſondern mein liebſter Umgang, ſeine Tochter\pwindex{Veronica @\textsc{Veronica}, \emph{Kellnerin}|pwv} aber, das liebliche Saumenſch, heißt \textsc{Berenike}\pwindex{Barres, Maurice 1862-08-19 – 1923-12-04@\textsc{Barrès, Maurice} (1862-08-19 – 1923-12-04), \emph{Schriftsteller}!Garten der Berenice1891@\strich\emph{Der Garten der Bérenice} {[}1891{]}|pwv} (abgek. \textsc{Vroni}\pwindex{Veronica @\textsc{Veronica}, \emph{Kellnerin}|pw}) und war zu ihrer Blütezeit Kellnerin. Außerdem laſſe ich mir von einer alten
                  Engländerin\pwindex{Laszowska, Jane Emily 07.05.1849 – 11.01.1905@\textsc{Łaszowska, Jane Emily} (07.05.1849 – 11.01.1905), \emph{Schriftstellerin}|pwv} auf naſskalten
               Spaziergängen viel erzählen: von der \textsc{Mozambiquebai}, wo die Leute meiſtens Würmer unter der Haut haben (ſie war dort als
               junge Frau) oder von dem häſslichen \textsc{boycott} in \textsc{Irland}\oindex{Irland@\textbf{Irland}|pw} und den ſchönen rothhaarigen \textsc{Cocotten} von Dublin\oindex{Dublin@\textbf{Dublin}|pw} (von denen ſpricht ſie ſo giftig gut, wie
               aus einem \textsc{ressentiment} heraus, ſie muſs dort etwas
               unangenehmes erlebt haben) oder von \textsc{Henry Irving}\pwindex{Irving, Henry 06.02.1838 – 13.10.1905@\textsc{Irving, Henry} (06.02.1838 – 13.10.1905), \emph{Schauspieler}|pw} oder von \textsc{Sir Laurence Oliphant}\pwindex{Oliphant, Laurence 03.08.1829 – 23.12.1888@\textsc{Oliphant, Laurence} (03.08.1829 – 23.12.1888), \emph{Forscher}|pw}, dem großen Medium.\pend
           \pstart
           {\pb}Ihre \label{K_L00023-7v}\edtext{Tochter}{\lemma{\textnormal{\emph{Tochter}}}\Cendnote{\textnormal{keine
                  weiblichen Nachkommen nachweisbar}}}\label{K_L00023-7h} wäre mir natürlich lieber, aber die iſt
               in Ceylon\oindex{Sri Lanka@\textbf{Sri Lanka}|pw}. Ich leſe \textsc{Homer}\pwindex{Homer @\textsc{Homer}, \emph{Schriftsteller}|pw}, \textsc{Maupassant}\pwindex{Maupassant, Guy de 05.08.1850 – 07.07.1893@\textsc{Maupassant, Guy de} (05.08.1850 – 07.07.1893), \emph{Schriftsteller}|pw}, das Linzer Volksblatt\orgindex{Linzer Volksblatt@Linzer Volksblatt|pw}, Eichendorff\pwindex{Eichendorff, Joseph von 10.03.1788 – 26.11.1857@\textsc{Eichendorff, Joseph von} (10.03.1788 – 26.11.1857), \emph{Schriftsteller}|pw} und \label{K_L00023-8v}\edtext{\textsc{cette touchante histoire de petite Secousse}}{\lemma{\textnormal{\emph{cette … Secousse}}}\Cendnote{\textnormal{französisch: die rührende Geschichte
                     von der kleinen Schüttlerin (Barrès\pwindex{Barres, Maurice 1862-08-19 – 1923-12-04@\textsc{Barrès, Maurice} (1862-08-19 – 1923-12-04), \emph{Schriftsteller}|pwk}
                     bezeichnet so die Hauptfigur Berénice\pwindex{Barres, Maurice 1862-08-19 – 1923-12-04@\textsc{Barrès, Maurice} (1862-08-19 – 1923-12-04), \emph{Schriftsteller}!Garten der Berenice1891@\strich\emph{Der Garten der Bérenice} {[}1891{]}|pwkv}.)}}}\label{K_L00023-8h}\pwindex{Barres, Maurice 1862-08-19 – 1923-12-04@\textsc{Barrès, Maurice} (1862-08-19 – 1923-12-04), \emph{Schriftsteller}!Garten der Berenice1891@\strich\emph{Der Garten der Bérenice} {[}1891{]}|pwv}, die manchmal ſo ſchön iſt, \label{K_L00023-9v}\edtext{\textsc{qu’elle donne presque envie de pleurer}}{\lemma{\textnormal{\emph{qu’elle … pleurer}}}\Cendnote{\textnormal{französisch: dass sie nahezu Lust zu
                  weinen macht}}}\label{K_L00023-9h}, trotz \textsc{Boulange}\pwindex{Boulanger, Georges 29.04.1837 – 30.09.1891@\textsc{Boulanger, Georges} (29.04.1837 – 30.09.1891), \emph{Politiker, Militär}|pwv}, Mysti-, \strikeout{Ch\textcolor{gray}{×}\-\textcolor{gray}{×}\-\textcolor{gray}{×}- }, Stoi- und Katholi-cismus. Ich habe gar keine eigenen
               Empfindungen, citiere fortwährend in Gedanken mich ſelbſt oder andere, habe auch die
               dumme letzte Scene von »Geſtern\pwindex{Hofmannsthal, Hugo von 1874-02-01 – 1929-07-15@\textsc{Hofmannsthal, Hugo von} (1874-02-01 – 1929-07-15), \emph{Schriftsteller}!Gestern. Dramatische Studie in einem Akt in Versen15. 10. 1891@\strich\emph{Gestern. Dramatische Studie in einem Akt in Versen} {[}15. 10. 1891{]}|pw}« noch immer
               nicht fertig gebracht, dafür aber von Goldmann\pwindex{Goldmann, Paul 31.01.1865 – 25.09.1935@\textsc{Goldmann, Paul} (31.01.1865 – 25.09.1935), \emph{Schriftsteller, Journalist}|pw}, der immer auf der Eiſenbahn zu ſein ſcheint eine, ſoweit man ſie
               leſen kann, ſehr herzliche Karte bekommen. Jetzt muſs ich packen (ganz origineller
               Abgang!) ſchreiben Sie mir, mein verehrter Freund, bitte, bald und geben Sie Ihr
               Project mich irgendwo zu beſuchen, nicht auf.\pend
           \pstart
           Herzlichſt{\\[\baselineskip]}\spacefill\mbox{Loris}\pend
           \leftskip=0em{}
         
         \endnumbering\mylabel{h}\end{ledgroupsized}  \newcommand{\dateiname}{L00023}\newcommand{\titel}{Hugo von Hofmannsthal an Arthur Schnitzler, 13. 7. [1891]}\newcommand{\editorInnen}{Martin Anton Müller und Gerd-Hermann Susen}%% latex-leseansicht-abspann.tex
%% Abspann für die Leseansicht.
%% Der Schalter \ifkorrekturansicht ist bereits durch den Vorspann gesetzt.

%% latex-abspann.tex
%% Gemeinsamer Abspann für Korrekturansicht und Leseansicht.
%% Setzt den Schalter \ifkorrekturansicht voraus (gesetzt in den
%% einbindenden Dateien latex-korrekturansicht-abspann.tex bzw.
%% latex-leseansicht-abspann.tex).
%% ---------------------------------------------------------------

\normalsize

% Das esempio-Environment wird nur in der Leseansicht benötigt
\ifkorrekturansicht\else
\newenvironment{esempio}[3]%
{
    \vspace{1.5ex}
    \rlap{\underline{#1}}
    \par
    \setlength{\parindent}{0cm}
    \nopagebreak
    \leftskip=#2cm
    \rightskip=#3cm
}
{
    \par
}
\fi

\doendnotes{C}
\bigskip
\vfill

\clearpage

\footnotesize

\ifkorrekturansicht
  \lohead{\textsc{register}}
\fi

% theindex-Environment neu definieren ohne reledmac
\makeatletter
\renewenvironment{theindex}{%
  \ifkorrekturansicht
    \section*{\indexname}%
  \else
    \subsubsection*{Index der erwähnten Entitäten}%
  \fi
  \setlength{\parindent}{0pt}%
  \setlength{\parskip}{0pt plus 0.3pt}%
  \let\item\@idxitem
}{%
  \ifkorrekturansicht\clearpage\fi
}
\makeatother

\IfFileExists{\jobname-pw.ind}{\input{\jobname-pw.ind}}{}

% Quellenangabe nur in der Leseansicht
\ifkorrekturansicht\else
% Fallback-Definitionen, falls die .tex-Datei \titel etc. nicht gesetzt hat
\providecommand{\titel}{}
\providecommand{\editorInnen}{}
\providecommand{\dateiname}{\jobname}

\vspace{3cm}

\vfill

\footnotesize
\textsc{Quelle}: \titel. Herausgegeben von {\editorInnen}. In: \emph{Arthur Schnitzler: Briefwechsel mit Autorinnen und Autoren}.
 Digitale Edition, https://schnitzler-briefe.acdh.oeaw.ac.at/{\dateiname}.html (Stand \today)
\fi

\end{document}


      