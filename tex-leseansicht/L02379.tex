%% latex-leseansicht-vorspann.tex
%% Vorspann für die Leseansicht.
%% Lädt die gemeinsame Datei latex-vorspann.tex mit nicht gesetztem Schalter.

\newif\ifkorrekturansicht
\korrekturansichtfalse

\input{../tex-inputs/latex-vorspann}


\section[Arthur Schnitzler an Richard Beer-Hofmann, 4. 5. 1922]{L02379 Arthur Schnitzler an Richard Beer-Hofmann, 4. 5. 1922}
\nopagebreak\mylabel{L02379v}
\rehead{ }\normalsize\beginnumbering\briefempfaengerindex{Beer-Hofmann, Richard@\textsc{Beer-Hofmann, Richard}!zzzSchnitzler, Arthur@\emph{von Arthur Schnitzler}!1922-05-041@{4. 5. 1922}|(be}
\toendnotes[C]{\smallbreak\pagebreak[2]}
\correspDesc{Versand  durch Arthur Schnitzler am 4. 5. 1922 in Den Haag
\newline{}Erhalt  durch Richard Beer-Hofmann im Zeitraum [5. 5. 1922
                  – 9. 5. 1922?] in Wien}\toendnotes[C]{\smallbreak}
\Standort{CUL, Schnitzler, B 8.1, S. 156.}
\physDesc{Brief, maschinenschriftliche Abschrift, 1 Blatt, 1 Seite, 306 Zeichen
\newline{}Schreibmaschine
\newline{}Ordnung: von unbekannter Hand als Briefnummer »352«
                                 gekennzeichnet }
\buchAbdrucke{\weitereDrucke{Arthur Schnitzler, Richard Beer-Hofmann: \emph{Briefwechsel 1891–1931}. Herausgegeben von Konstanze Fliedl. Wien, Zürich: \emph{Europaverlag} 1992, S. 228.} }\toendnotes[C]{\smallbreak}
\pstart
           \raggedleft{}{\pb}Haag\oindex{Den Haag@\textbf{Den Haag}, \emph{Hauptstadt}|pw}, 4. 5. 1922.\pend
           \vspace{0.5em}
\pstart
           Herzliche Grüsse! Ich habe hier angenehme bewegte Tage verbracht, in Rotterdam\oindex{Rotterdam@\textbf{Rotterdam}|pw}, Haag\oindex{Den Haag@\textbf{Den Haag}, \emph{Hauptstadt}|pw}, Amsterdam\oindex{Amsterdam@\textbf{Amsterdam}, \emph{Hauptstadt}|pw} mit Erfolg (in jeder
               Hinsicht) gelesen, fahre Samstag 6. aufs Land nach Osterbeck\oindex{Oosterbeek@\textbf{Oosterbeek}|pw} zu Brevées\pwindex{Brevée, Bertha 14.\,10.\,1883 Sint-Maartensdijk – 6.\,5.\,1953 Taormina@\textsc{Brevée, Bertha} (14.\,10.\,1883 Sint-Maartensdijk – 6.\,5.\,1953 Taormina), \emph{Schriftstellerin, Schauspielerin}|pw}\pwindex{Brevée, Isaäc 14.\,5.\,1879 Sluis – 15.\,4.\,1952 Oosterbeek@\textsc{Brevée, Isaäc} (14.\,5.\,1879 Sluis – 15.\,4.\,1952 Oosterbeek), \emph{Mediziner}|pw}, circa am 8. nach Berlin\oindex{Berlin@\textbf{Berlin}, \emph{Hauptstadt}|pw} weiter; dürfte zwischen 18. u. 20. in Wien\oindex{Wien@\textbf{Wien}, \emph{Verwaltungsgebiet}|pw} sein. Grüssen Sie Paula\pwindex{Beer-Hofmann, Paula 25.\,2.\,1879 Wien – 30.\,10.\,1939 Zürich@\textsc{Beer-Hofmann, Paula} (25.\,2.\,1879 Wien – 30.\,10.\,1939 Zürich)|pw} und die Kinder\pwindex{Beer-Hofmann, Naëmah 20.\,12.\,1898 Wien – 10.\,11.\,1971 New York City@\textsc{Beer-Hofmann, Naëmah} (20.\,12.\,1898 Wien – 10.\,11.\,1971 New York City)|pwv}\pwindex{Beer-Hofmann, Mirjam 4.\,9.\,1897 Wien – 24.\,12.\,1984 New York City@\textsc{Beer-Hofmann, Mirjam} (4.\,9.\,1897 Wien – 24.\,12.\,1984 New York City)|pwv}\pwindex{Beer-Hofmann, Gabriel 9.\,1.\,1901 Wien – 24.\,3.\,1971 St Albans@\textsc{Beer-Hofmann, Gabriel} (9.\,1.\,1901 Wien – 24.\,3.\,1971 St Albans), \emph{Schriftsteller, Filmagent}|pwv}! Ihr
                  \spacefill\mbox{A.}\pend
           
\pstart
           \noindent{}(nach Wien\oindex{Wien@\textbf{Wien}, \emph{Verwaltungsgebiet}|pw})\pend
           \selectlanguage{ngerman}\endnumbering\briefempfaengerindex{Beer-Hofmann, Richard@\textsc{Beer-Hofmann, Richard}!zzzSchnitzler, Arthur@\emph{von Arthur Schnitzler}!1922-05-041@{4. 5. 1922}|)be}\mylabel{L02379h}  \newcommand{\dateiname}{L02379}\newcommand{\titel}{Arthur Schnitzler an Richard Beer-Hofmann, 4. 5. 1922}\newcommand{\editorInnen}{Martin Anton Müller und Gerd-Hermann Susen}%% latex-leseansicht-abspann.tex
%% Abspann für die Leseansicht.
%% Der Schalter \ifkorrekturansicht ist bereits durch den Vorspann gesetzt.

%% latex-abspann.tex
%% Gemeinsamer Abspann für Korrekturansicht und Leseansicht.
%% Setzt den Schalter \ifkorrekturansicht voraus (gesetzt in den
%% einbindenden Dateien latex-korrekturansicht-abspann.tex bzw.
%% latex-leseansicht-abspann.tex).
%% ---------------------------------------------------------------

\normalsize

% Das esempio-Environment wird nur in der Leseansicht benötigt
\ifkorrekturansicht\else
\newenvironment{esempio}[3]%
{
    \vspace{1.5ex}
    \rlap{\underline{#1}}
    \par
    \setlength{\parindent}{0cm}
    \nopagebreak
    \leftskip=#2cm
    \rightskip=#3cm
}
{
    \par
}
\fi

\doendnotes{C}
\bigskip
\vfill

\clearpage

\footnotesize

\ifkorrekturansicht
  \lohead{\textsc{register}}
\fi

% theindex-Environment neu definieren ohne reledmac
\makeatletter
\renewenvironment{theindex}{%
  \ifkorrekturansicht
    \section*{\indexname}%
  \else
    \subsubsection*{Index der erwähnten Entitäten}%
  \fi
  \setlength{\parindent}{0pt}%
  \setlength{\parskip}{0pt plus 0.3pt}%
  \let\item\@idxitem
}{%
  \ifkorrekturansicht\clearpage\fi
}
\makeatother

\IfFileExists{\jobname-pw.ind}{\input{\jobname-pw.ind}}{}

% Quellenangabe nur in der Leseansicht
\ifkorrekturansicht\else
% Fallback-Definitionen, falls die .tex-Datei \titel etc. nicht gesetzt hat
\providecommand{\titel}{}
\providecommand{\editorInnen}{}
\providecommand{\dateiname}{\jobname}

\vspace{3cm}

\vfill

\footnotesize
\textsc{Quelle}: \titel. Herausgegeben von {\editorInnen}. In: \emph{Arthur Schnitzler: Briefwechsel mit Autorinnen und Autoren}.
 Digitale Edition, https://schnitzler-briefe.acdh.oeaw.ac.at/{\dateiname}.html (Stand \today)
\fi

\end{document}


