%% latex-korrekturansicht-vorspann.tex
%% Vorspann für die Korrekturansicht.
%% Lädt die gemeinsame Datei latex-vorspann.tex mit gesetztem Schalter.

\newif\ifkorrekturansicht
\korrekturansichttrue

\input{../tex-inputs/latex-vorspann}


\section[Richard Beer-Hofmann an Arthur Schnitzler, {[}17. 2. 1895?{]}]{L00421 Richard Beer-Hofmann an Arthur Schnitzler, {[}17. 2. 1895?{]}}
\nopagebreak\mylabel{L00421v}
\rehead{ }\normalsize\beginnumbering\briefempfaengerindex{Schnitzler, Arthur@\textsc{Schnitzler, Arthur}!zzzBeer-Hofmann, Richard@\emph{von Richard Beer-Hofmann}!1895-02-171@{{[}17. 2. 1895?{]}}|(be}
\toendnotes[C]{\smallbreak\pagebreak[2]}\Standort{CUL, Schnitzler, B 8.}
\physDesc{Visitenkarte, 286 Zeichen
\newline{}Handschrift: Bleistift, lateinische Kurrent
\newline{}Schnitzler: mit Bleistift datiert: »17/2 95.« und nummeriert:
                                    »556« }
\buchAbdrucke{\weitereDrucke{Arthur Schnitzler, Richard Beer-Hofmann: \emph{Briefwechsel 1891–1931}. Wien, Zürich: \emph{Europaverlag} 1992, S. 71.} }\toendnotes[C]{\smallbreak}
\pstart
           \noindent{}{\pb}Lieber Arthur! Ich bin \label{K_L00421-1v}\edtext{heute}{\lemma{\textnormal{\emph{heute}}}\Cendnote{\textnormal{Obzwar von Schnitzler datiert, sind Zweifel anzumelden, da Beer-Hofmann\pwindex{Beer-Hofmann, Richard 1866-07-11 – 1945-09-26@\textsc{Beer-Hofmann, Richard} (1866-07-11 – 1945-09-26), \emph{Schriftsteller/Schriftstellerin}|pwk} den Abend erst recht in der
                  Gesellschaft Schnitzlers verbrachte, eine
                  Teilnahme Hofmannsthals\pwindex{Hofmannsthal, Hugo von 1874-02-01 – 1929-07-15@\textsc{Hofmannsthal, Hugo von} (1874-02-01 – 1929-07-15), \emph{Schriftsteller/Schriftstellerin}|pwk} wiederum nicht
                  nachgewiesen werden kann.}}}\label{K_L00421-1} Nachmittag zu Hause und, arbeite. Wegen des Herrn
               Hund’s werde ich kaum \strikeout{Nachmittag} Abends ins Gasthaus
               gehen können, weil das Stubenmädchen\pwindex{?? [Stubenfrau bei Richard Beer-Hofmann] @\textsc{?? [Stubenfrau bei Richard Beer-Hofmann]}|pwv} weggeht. Wenn Sie und Hugo\pwindex{Hofmannsthal, Hugo von 1874-02-01 – 1929-07-15@\textsc{Hofmannsthal, Hugo von} (1874-02-01 – 1929-07-15), \emph{Schriftsteller/Schriftstellerin}|pw} am Abend {\pb}vielleicht
               vorüber kommen schauen oder läuten Sie vielleicht zu mir herauf\pend
           
\pstart
           herzlichst{\\[\baselineskip]}\spacefill\mbox{Richard}\pend
           \leftskip=0em{}
\pstart
           \centering{}\label{T_L00421-1v}\edtext{\textcolor{gray}{\textbf{D\textsuperscript{r} Richard Beer-Hofmann}}}{\lemma{\textnormal{\emph{D\textsuperscript{r} … Beer-Hofmann}}}\Cendnote{\textnormal{Die Visitenkarte wurde so beschrieben,
                  dass der Aufdruck auf dem Kopf steht.}}}\label{T_L00421-1}\pend
           \selectlanguage{ngerman}\endnumbering\briefempfaengerindex{Schnitzler, Arthur@\textsc{Schnitzler, Arthur}!zzzBeer-Hofmann, Richard@\emph{von Richard Beer-Hofmann}!1895-02-171@{{[}17. 2. 1895?{]}}|)be}\mylabel{L00421h}  \normalsize

\doendnotes{C}
\bigskip
\vfill

\clearpage

\footnotesize

\lohead{\textsc{register}}

% Definiere theindex-Environment komplett neu ohne reledmac
\makeatletter
\renewenvironment{theindex}{%
  \section*{\indexname}%
  \setlength{\parindent}{0pt}%
  \setlength{\parskip}{0pt plus 0.3pt}%
  \let\item\@idxitem
}{%
  \clearpage
}
\makeatother

\IfFileExists{\jobname-pw.ind}{\input{\jobname-pw.ind}}{}

\end{document}

      