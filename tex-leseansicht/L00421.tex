\input{../tex-inputs/latex-pdf-vorspann}
\begin{center}
            \textcolor{red}{ENTWURF. ENTZIFFERUNG NOCH NICHT KORREKTURGELESEN}
                      \end{center}
            
               \section[Richard Beer-Hofmann an Arthur Schnitzler, {[}17. 2. 1895?{]}]{ Richard Beer-Hofmann an Arthur Schnitzler, {[}17. 2. 1895?{]}}\nopagebreak\mylabel{v}\rehead{ }\begin{ledgroupsized}[t]{13cm}\normalsize\beginnumbering\briefempfaengerindex{Schnitzler, Arthur@\textsc{Schnitzler, Arthur}!zzzBeer-Hofmann, Richard@\emph{von Richard Beer-Hofmann}!1895-02-171@{{[}17. 2. 1895?{]}}|(be} \toendnotes[C]{\smallbreak\pagebreak[2]} \Standort{CUL, Schnitzler, B 8.}
\physDesc{Visitenkarte
\newline{}Handschrift: Bleistift, lateinische Kurrent
\newline{}Schnitzler: mit Bleistift datiert: »17/2 95.« und nummeriert: »556« }\buchAbdrucke{\weitereDrucke{Arthur Schnitzler, Richard Beer-Hofmann: \emph{Briefwechsel 1891–1931}. Hg. Konstanze Fliedl. Wien, Zürich: \emph{Europaverlag} 1992, S. 71.} }\toendnotes[C]{\smallbreak}\pstart
           \noindent{}{\pb}Lieber Arthur! Ich bin \label{K_L00421_1v}\edtext{heute}{\lemma{\textnormal{\emph{heute}}}\Cendnote{\textnormal{Obzwar von Schnitzler\pwindex{Schnitzler, Arthur 15.05.1862 – 21.10.1931@\textsc{Schnitzler, Arthur} (15.05.1862 – 21.10.1931), \emph{Schriftsteller, Mediziner}|pwk} datiert, sind Zweifel
                  anzumelden, da Beer-Hofmann\pwindex{Beer-Hofmann, Richard 11.07.1866 – 26.09.1945@\textsc{Beer-Hofmann, Richard} (11.07.1866 – 26.09.1945), \emph{Schriftsteller}|pwk} den Abend erst
                  recht in der Gesellschaft Schnitzler\pwindex{Schnitzler, Arthur 15.05.1862 – 21.10.1931@\textsc{Schnitzler, Arthur} (15.05.1862 – 21.10.1931), \emph{Schriftsteller, Mediziner}|pwk}s
                  verbrachte, eine Teilnahme Hofmannsthals\pwindex{Hofmannsthal, Hugo von 01.02.1874 – 15.07.1929@\textsc{Hofmannsthal, Hugo von} (01.02.1874 – 15.07.1929), \emph{Schriftsteller}|pwk}
                  wiederum nicht nachgewiesen werden kann.}}}\label{K_L00421_1h} Nachmittag zu Hause und, arbeite.
               Wegen des Herrn Hund’s werde ich kaum \strikeout{Nachmittag}
               Abends ins Gasthaus gehen können, weil das Stubenmädchen\pwindex{?? [Stubenfrau bei Richard Beer-Hofmann] 17.2.1895 – 17.2.1895@\textsc{?? [Stubenfrau bei Richard Beer-Hofmann]} (17.2.1895 – 17.2.1895)|pwv} weggeht. Wenn Sie und Hugo\pwindex{Hofmannsthal, Hugo von 01.02.1874 – 15.07.1929@\textsc{Hofmannsthal, Hugo von} (01.02.1874 – 15.07.1929), \emph{Schriftsteller}|pw}
               am Abend {\pb}vielleicht vorüber kommen
               schauen oder läuten Sie vielleicht zu mir herauf\pend
           \pstart
           herzlichst{\\[\baselineskip]}\spacefill\mbox{Richard}\pend
           \leftskip=0em{}\pstart
           \centering{}\label{T_L00421_1v}\edtext{\textcolor{gray}{\textbf{D\textsuperscript{r} Richard Beer-Hofmann}}}{\lemma{\textnormal{\emph{Dr Richard Beer-Hofmann}}}\Cendnote{\textnormal{Die Visitenkarte wurde so beschrieben, dass der
                  Aufdruck auf dem Kopf steht.}}}\label{T_L00421_1h}\pend
           \endnumbering\briefempfaengerindex{Schnitzler, Arthur@\textsc{Schnitzler, Arthur}!zzzBeer-Hofmann, Richard@\emph{von Richard Beer-Hofmann}!1895-02-171@{{[}17. 2. 1895?{]}}|)be}\mylabel{h}\end{ledgroupsized}  \newcommand{\dateiname}{L00421}\newcommand{\titel}{Richard Beer-Hofmann an Arthur Schnitzler, [17. 2. 1895?]}\newcommand{\editorInnen}{Martin Anton Müller und Gerd-Hermann Susen}\input{../tex-inputs/latex-pdf-abspann}
      