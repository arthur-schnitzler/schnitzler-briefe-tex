%% latex-leseansicht-vorspann.tex
%% Vorspann für die Leseansicht.
%% Lädt die gemeinsame Datei latex-vorspann.tex mit nicht gesetztem Schalter.

\newif\ifkorrekturansicht
\korrekturansichtfalse

\input{../tex-inputs/latex-vorspann}


\section[Richard Beer-Hofmann und Arthur Kaufmann an Arthur Schnitzler, {[}23.? 7. 1904{]}]{L01417 Richard Beer-Hofmann und Arthur Kaufmann an Arthur Schnitzler, {[}23.? 7. 1904{]}}
\nopagebreak\mylabel{L01417v}
\rehead{ }\normalsize\beginnumbering\briefempfaengerindex{Schnitzler, Arthur@\textsc{Schnitzler, Arthur}!zzzKaufmann, Arthur@\emph{von Arthur Kaufmann}!1904-07-231@{{[}23.? 7. 1904{]}}|(be}\briefempfaengerindex{Schnitzler, Arthur@\textsc{Schnitzler, Arthur}!zzzBeer-Hofmann, Richard@\emph{von Richard Beer-Hofmann}!1904-07-231@{{[}23.? 7. 1904{]}}|(be}
\toendnotes[C]{\smallbreak\pagebreak[2]}
\correspDesc{Versand  durch Richard Beer-Hofmann, Arthur Kaufmann am [23.? 7. 1904] in Bad Aussee
\newline{}Erhalt  durch Arthur Schnitzler im Zeitraum [24. 7. 1904
                  – 28. 7. 1904?] in Wien}\toendnotes[C]{\smallbreak}
\Standort{CUL, Schnitzler, B 8.}
\physDesc{Brief, 1 Blatt, 1 Seite, 132 Zeichen
\newline{}\noindent{}Gedruckter Theaterzettel (\noindent{}Gedruckter Theaterzettel)
\newline{}Handschrift Richard Beer-Hofmann: Bleistift, lateinische Kurrent
\newline{}Handschrift Arthur Kaufmann: Bleistift, lateinische Kurrent
\newline{}Ordnung: mit Bleistift von unbekannter Hand nummeriert:
                                    »185a« }\toendnotes[C]{\smallbreak}
\pstart
           \noindent{}\centering{}{\pb}\textcolor{gray}{\textbf{Kurtheater\orgindex{Kurtheater in Aussee@Kurtheater in Aussee|pw} in Aussee\oindex{Bad Aussee@\textbf{Bad Aussee}, \emph{Hauptstadt}|pw}.}}\pend
           
\pstart
           \centering{}\textcolor{gray}{\textbf{Direktion: Gustav Charlé\pwindex{Charlé, Gustav 28.\,2.\,1871 Wien – 1940?@\textsc{Charlé, Gustav} (28.\,2.\,1871 Wien – 1940?), \emph{Theaterleiter, Schauspieler}|pw}
                  und Gustav Müller\pwindex{Müller, Gustav September 1873 – 27.\,8.\,1936 Wien@\textsc{Müller, Gustav} (September 1873 – 27.\,8.\,1936 Wien), \emph{Regisseur, Schauspieler, Sänger}|pw}.}}\pend
           
\pstart
           \centering{}\textcolor{gray}{\textbf{Samstag, den 23. Juli 1904}}\pend
           {\vspace{1\baselineskip}}
\pstart
           \centering{}\textcolor{gray}{\textbf{Bunter Abend}}\pend
           
\pstart
           \centering{}\textcolor{gray}{\textbf{Gastspiel der Frau Emmy
                     Förster\pwindex{Förster, Emmy 28.\,11.\,1865 Žďár nad Sázavou – 5.\,12.\,1942 Konzentrationslager Theresienstadt@\textsc{Förster, Emmy} (28.\,11.\,1865 Žďár nad Sázavou – 5.\,12.\,1942 Konzentrationslager Theresienstadt), \emph{Schauspielerin}|pw}}}\pend
           {\vspace{1\baselineskip}}
\pstart
           \centering{}\textcolor{gray}{\textbf{Den Anfang macht}}\pend
           
\pstart
           \centering{}\textcolor{gray}{\textbf{Kollegen\pwindex{Neumann-Hofer, Annie 20.\,3.\,1868 New York City – nach 1944@\textsc{Neumann-Hofer, Annie} (20.\,3.\,1868 New York City – nach 1944), \emph{Schriftstellerin}!Kollegen@\strich\emph{Kollegen{\rufezeichen}}|pw}}}\pend
           
\pstart
           \centering{}\textcolor{gray}{\textbf{Komödie in 1 Akt von Annie
                     Neumann\pwindex{Neumann-Hofer, Annie 20.\,3.\,1868 New York City – nach 1944@\textsc{Neumann-Hofer, Annie} (20.\,3.\,1868 New York City – nach 1944), \emph{Schriftstellerin}|pw}.}}\pend
           
\pstart
           \centering{}\textcolor{gray}{\textbf{(Regisseur Direktor Müller\pwindex{Müller, Gustav September 1873 – 27.\,8.\,1936 Wien@\textsc{Müller, Gustav} (September 1873 – 27.\,8.\,1936 Wien), \emph{Regisseur, Schauspieler, Sänger}|pw}).}}\pend
           
\pstart
           \centering{}\textcolor{gray}{\textbf{PERSONEN:}}\pend
           
\pstart
           
\pstart
           \textcolor{gray}{\textbf{Stella v. Balakow-Hartmann, Geigen-Virtuosin}}\pend
           
\pstart
           \raggedleft{}\textcolor{gray}{\textbf{\textsuperscript{*}\textsubscript{*}\textsuperscript{*}}}\pend
           \pend
           
\pstart
           
\pstart
           \textcolor{gray}{\textbf{Werner Hartmann, ihr Gatte, Klavier-Virtuose}}\pend
           
\pstart
           \raggedleft{}\textcolor{gray}{\textbf{Oskar Beraun\pwindex{Beraun, Oskar @\textsc{Beraun, Oskar}, \emph{Schauspieler}|pw}}}\pend
           \pend
           
\pstart
           
\pstart
           \textcolor{gray}{\textbf{Arthur v. Bront, Klavier-Virtuose}}\pend
           
\pstart
           \raggedleft{}\textcolor{gray}{\textbf{Theodor Robert\pwindex{Robert, Theodor @\textsc{Robert, Theodor}, \emph{Schauspieler}|pw}}}\pend
           \pend
           
\pstart
           
\pstart
           \textcolor{gray}{\textbf{Schwarz, Impresario}}\pend
           
\pstart
           \raggedleft{}\textcolor{gray}{\textbf{Dir. Gustav Müller\pwindex{Müller, Gustav September 1873 – 27.\,8.\,1936 Wien@\textsc{Müller, Gustav} (September 1873 – 27.\,8.\,1936 Wien), \emph{Regisseur, Schauspieler, Sänger}|pw}}}\pend
           \pend
           
\pstart
           
\pstart
           \textcolor{gray}{\textbf{Minna, Kammermädchen bei Hartmann}}\pend
           
\pstart
           \raggedleft{}\textcolor{gray}{\textbf{Rosa Vennyer\pwindex{Vennyer, Rosa †~nach 1922@\textsc{Vennyer, Rosa} (†~nach 1922), \emph{Schauspielerin, Sängerin}|pw}}}.\pend
           \pend
           
\pstart
           
\pstart
           \textcolor{gray}{\textbf{Franz{[},{]} Diener {[}bei
                        Hartmann{]}}}\pend
           
\pstart
           \raggedleft{}\textcolor{gray}{\textbf{Fritz Schönhof\pwindex{Schönhof, Friedrich †~nach 1930@\textsc{Schönhof, Friedrich} (†~nach 1930), \emph{Schauspieler}|pw}}}\pend
           \pend
           
\pstart
           \centering{}\textcolor{gray}{\textbf{Zeit: die Gegenwart. Ein Winter{[}-{]}Nachmittag
                  von 4 bis halb 8 Uhr. Ort: Berlin\oindex{Berlin@\textbf{Berlin}, \emph{Hauptstadt}|pw}.}}\pend
           
\pstart
           
\pstart
           \textcolor{gray}{\textbf{\textsuperscript{*}\textsubscript{*}\textsuperscript{*} Stella}}\pend
           
\pstart
           \raggedleft{}\textcolor{gray}{\textbf{Frau Emmy Förster\pwindex{Förster, Emmy 28.\,11.\,1865 Žďár nad Sázavou – 5.\,12.\,1942 Konzentrationslager Theresienstadt@\textsc{Förster, Emmy} (28.\,11.\,1865 Žďár nad Sázavou – 5.\,12.\,1942 Konzentrationslager Theresienstadt), \emph{Schauspielerin}|pw} als
                     Gast.}}\pend
           \pend
           {\vspace{1\baselineskip}}
\pstart
           \centering{}\textcolor{gray}{\textbf{Hierauf:}}\pend
           
\pstart
           
\pstart
           \textcolor{gray}{\textbf{Vorträge}}\pend
           
\pstart
           \raggedleft{}\textcolor{gray}{\textbf{Frl. Hel. Robert\pwindex{Robert, Helene 28.\,7.\,1880 Wien – 13.\,9.\,1963 München@\textsc{Robert, Helene} (28.\,7.\,1880 Wien – 13.\,9.\,1963 München), \emph{Schauspielerin}|pw}}}\pend
           \pend
           
\pstart
           \centering{}\textcolor{gray}{\textbf{»Frauentypen\pwindex{Pserhofer, Arthur 28.\,10.\,1873 Wien – 13.\,1.\,1907 Berlin@\textsc{Pserhofer, Arthur} (28.\,10.\,1873 Wien – 13.\,1.\,1907 Berlin), \emph{Schriftsteller, Theaterleiter}!Frauentypen@\strich\emph{Frauentypen}|pw}« von Arthur Pserhofer\pwindex{Pserhofer, Arthur 28.\,10.\,1873 Wien – 13.\,1.\,1907 Berlin@\textsc{Pserhofer, Arthur} (28.\,10.\,1873 Wien – 13.\,1.\,1907 Berlin), \emph{Schriftsteller, Theaterleiter}|pw}}}\pend
           
\pstart
           \centering{}\textcolor{gray}{\textbf{»Capricio\pwindex{Madeleine, Marie 4.\,4.\,1881 Tschernyschewskoje – 27.\,9.\,1944 Katzenelnbogen@\textsc{Madeleine, Marie} (4.\,4.\,1881 Tschernyschewskoje – 27.\,9.\,1944 Katzenelnbogen), \emph{Schriftstellerin, Lyrikerin}!Auf Kypros@\strich\emph{Auf Kypros}|pw}« aus dem
                  Tagebuche einer Demi-Vierge von Marie
                     Madeleine\pwindex{Madeleine, Marie 4.\,4.\,1881 Tschernyschewskoje – 27.\,9.\,1944 Katzenelnbogen@\textsc{Madeleine, Marie} (4.\,4.\,1881 Tschernyschewskoje – 27.\,9.\,1944 Katzenelnbogen), \emph{Schriftstellerin, Lyrikerin}|pw}}}\pend
           
\pstart
           \centering{}\textcolor{gray}{\textbf{»Das Mädel ohne Bräutigam\pwindex{Bierbaum, Otto Julius 28.\,6.\,1865 Zielona Góra – 1.\,2.\,1910 Dresden@\textsc{Bierbaum, Otto Julius} (28.\,6.\,1865 Zielona Góra – 1.\,2.\,1910 Dresden)!Mädchen ohne Bräutigam@\strich\emph{Das Mädchen ohne Bräutigam}|pw}«
                  v. Otto Jul. Bierbaum\pwindex{Bierbaum, Otto Julius 28.\,6.\,1865 Zielona Góra – 1.\,2.\,1910 Dresden@\textsc{Bierbaum, Otto Julius} (28.\,6.\,1865 Zielona Góra – 1.\,2.\,1910 Dresden)|pw}}}\pend
           {\vspace{1\baselineskip}}
\pstart
           \centering{}\textcolor{gray}{\textbf{Am Theater und im Leben\pwindex{Müller, Gustav September 1873 – 27.\,8.\,1936 Wien@\textsc{Müller, Gustav} (September 1873 – 27.\,8.\,1936 Wien), \emph{Regisseur, Schauspieler, Sänger}!Am Theater und im Leben@\strich\emph{Am Theater und im Leben}|pw}\pwindex{Jenik, Hilda @\textsc{Jenik, Hilda}, \emph{Schauspielerin, Sängerin}!Am Theater und im Leben@\strich\emph{Am Theater und im Leben}|pw}.}}\pend
           
\pstart
           \centering{}\textcolor{gray}{\textbf{Tanz-Duett, vorgetragen von Frln \textbf{Jenik}\pwindex{Jenik, Hilda @\textsc{Jenik, Hilda}, \emph{Schauspielerin, Sängerin}|pw} und Direktor \textbf{Müller}\pwindex{Müller, Gustav September 1873 – 27.\,8.\,1936 Wien@\textsc{Müller, Gustav} (September 1873 – 27.\,8.\,1936 Wien), \emph{Regisseur, Schauspieler, Sänger}|pw}.}}\pend
           {\vspace{1\baselineskip}}
\pstart
           \centering{}\textcolor{gray}{\textbf{Zum Schlusse:}}\pend
           
\pstart
           \centering{}\textcolor{gray}{\textbf{Literatur\pwindex{Schnitzler, Arthur 15.\,5.\,1862 Wien – 21.\,10.\,1931 ebd.@\textsc{Schnitzler, Arthur} (15.\,5.\,1862 Wien – 21.\,10.\,1931 ebd.), \emph{Schriftsteller, Mediziner}!Literatur@\strich\emph{Literatur}|pw}}}\pend
           
\pstart
           \centering{}\textcolor{gray}{\textbf{Lustspiel in 1 Akt von Arthur Schnitzler.}}\pend
           
\pstart
           \centering{}\textcolor{gray}{\textbf{(Regisseur Direktor Müller\pwindex{Müller, Gustav September 1873 – 27.\,8.\,1936 Wien@\textsc{Müller, Gustav} (September 1873 – 27.\,8.\,1936 Wien), \emph{Regisseur, Schauspieler, Sänger}|pw}).}}\pend
           
\pstart
           \centering{}\textcolor{gray}{\textbf{Personen:}}\pend
           
\pstart
           
\pstart
           \textcolor{gray}{\textbf{Margarethe}}\pend
           
\pstart
           \raggedleft{}\textcolor{gray}{\textbf{\textsuperscript{*}\textsubscript{*}\textsuperscript{*}}}\pend
           \pend
           
\pstart
           
\pstart
           \textcolor{gray}{\textbf{Clemens}}\pend
           
\pstart
           \raggedleft{}\textcolor{gray}{\textbf{Dir. Gustav Müller\pwindex{Müller, Gustav September 1873 – 27.\,8.\,1936 Wien@\textsc{Müller, Gustav} (September 1873 – 27.\,8.\,1936 Wien), \emph{Regisseur, Schauspieler, Sänger}|pw}}}\pend
           \pend
           
\pstart
           
\pstart
           \textcolor{gray}{\textbf{Gilbert}}\pend
           
\pstart
           \raggedleft{}\textcolor{gray}{\textbf{Fritz Digruber\pwindex{Digruber, Friedrich @\textsc{Digruber, Friedrich}|pw}}}\pend
           \pend
           
\pstart
           
\pstart
           \textcolor{gray}{\textbf{\textsuperscript{*}\textsubscript{*}\textsuperscript{*} Margarethe}}\pend
           
\pstart
           \raggedleft{}\textcolor{gray}{\textbf{Frau Emmy Förster\pwindex{Förster, Emmy 28.\,11.\,1865 Žďár nad Sázavou – 5.\,12.\,1942 Konzentrationslager Theresienstadt@\textsc{Förster, Emmy} (28.\,11.\,1865 Žďár nad Sázavou – 5.\,12.\,1942 Konzentrationslager Theresienstadt), \emph{Schauspielerin}|pw} als
                     Gast.}}\pend
           \pend
           {\vspace{1\baselineskip}}
\pstart
           \textcolor{gray}{\textbf{\label{T_L01417-1v}\edtext{Preise}{\lemma{\textnormal{\emph{Preise}}}\Cendnote{\textnormal{Druckfehler, korrigiert aus »Priese«}}}\label{T_L01417-1} der
                  Plätze:}}\pend
           \settowidth{\longeste}{Ein Parterresitz, 8.–15. Reihe}\settowidth{\longestz}{„}\settowidth{\longestd}{–.80}\settowidth{\longestv}{}\settowidth{\longestf}{}\addtolength\longeste{1em}
        \addtolength\longestz{1em}
        \addtolength\longestd{1em}
      \pstart\noindent\makebox[\the\longeste][l]{\textcolor{gray}{\textbf{Eine Loge für 4 Personen}}}\makebox[\the\longestz][l]{\textcolor{gray}{\textbf{K}}}
                  \makebox[\the\longestd][l]{\textcolor{gray}{\textbf{16.–}}}\pend\pstart\noindent\makebox[\the\longeste][l]{\textcolor{gray}{\textbf{Eine Logensitz}}}\makebox[\the\longestz][l]{\textcolor{gray}{\textbf{„}}}
                  \makebox[\the\longestd][l]{\textcolor{gray}{\textbf{5.–}}}\pend\pstart\noindent\makebox[\the\longeste][l]{\textcolor{gray}{\textbf{Ein Orchestersitz, 1.–3. Reihe}}}\makebox[\the\longestz][l]{\textcolor{gray}{\textbf{„}}}
                  \makebox[\the\longestd][l]{\textcolor{gray}{\textbf{4.–}}}\pend\pstart\noindent\makebox[\the\longeste][l]{\textcolor{gray}{\textbf{Ein Parkettsitz, 4.–7. Reihe}}}\makebox[\the\longestz][l]{\textcolor{gray}{\textbf{„}}}
                  \makebox[\the\longestd][l]{\textcolor{gray}{\textbf{3.–}}}\pend\pstart\noindent\makebox[\the\longeste][l]{\textcolor{gray}{\textbf{Ein Parterresitz, 8.–15. Reihe}}}\makebox[\the\longestz][l]{\textcolor{gray}{\textbf{„}}}
                  \makebox[\the\longestd][l]{\textcolor{gray}{\textbf{2.–}}}\pend\pstart\noindent\makebox[\the\longeste][l]{\textcolor{gray}{\textbf{Ein Parterrestehplatz}}}\makebox[\the\longestz][l]{\textcolor{gray}{\textbf{„}}}
                  \makebox[\the\longestd][l]{\textcolor{gray}{\textbf{–.80}}}\pend
\pstart
           \centering{}\textcolor{gray}{\textbf{Die Tageskasse befindet sich Ischlerstrasse 72\oindex{Ischler Straße@\textbf{Ischler Straße}, \emph{Straße}|pw} und ist geöffnet von 9 Uhr vormittags bis 1 Uhr mittags
                  und von 3–5 nachmittags.}}\pend
           
\pstart
           Nicht umzubringen! »\label{K_L01417-1v}\edtext{In Zeit und
                  Ewigkeit}{\lemma{\textnormal{\emph{In Zeit und
                  Ewigkeit}}}\Cendnote{\textnormal{kein Zitat, sondern stehende
                  Wendung in der katholischen Fachsprache}}}\label{K_L01417-1}«!\pend
           
\pstart
           Seien Sie Beide\pwindex{Schnitzler, Olga 17.\,1.\,1882 Wien – 13.\,1.\,1970 Lugano@\textsc{Schnitzler, Olga} (17.\,1.\,1882 Wien – 13.\,1.\,1970 Lugano), \emph{Schauspielerin, Sängerin}|pwv} herzlich
               gegrüsst von \spacefill\mbox{Richard und Paula\pwindex{Beer-Hofmann, Paula 25.\,2.\,1879 Wien – 30.\,10.\,1939 Zürich@\textsc{Beer-Hofmann, Paula} (25.\,2.\,1879 Wien – 30.\,10.\,1939 Zürich)|pw}.}\pend
           
\pstart
           {[}hs. Kaufmann:{]} Es grüsst Sie herzlich{\\}\spacefill\mbox{AKaufmann}\pend
           \selectlanguage{ngerman}\endnumbering\briefempfaengerindex{Schnitzler, Arthur@\textsc{Schnitzler, Arthur}!zzzKaufmann, Arthur@\emph{von Arthur Kaufmann}!1904-07-231@{{[}23.? 7. 1904{]}}|)be}\briefempfaengerindex{Schnitzler, Arthur@\textsc{Schnitzler, Arthur}!zzzBeer-Hofmann, Richard@\emph{von Richard Beer-Hofmann}!1904-07-231@{{[}23.? 7. 1904{]}}|)be}\mylabel{L01417h}  \newcommand{\dateiname}{L01417}\newcommand{\titel}{Richard Beer-Hofmann und Arthur Kaufmann an Arthur Schnitzler, [23.? 7. 1904]}\newcommand{\editorInnen}{Martin Anton Müller und Gerd-Hermann Susen}%% latex-leseansicht-abspann.tex
%% Abspann für die Leseansicht.
%% Der Schalter \ifkorrekturansicht ist bereits durch den Vorspann gesetzt.

%% latex-abspann.tex
%% Gemeinsamer Abspann für Korrekturansicht und Leseansicht.
%% Setzt den Schalter \ifkorrekturansicht voraus (gesetzt in den
%% einbindenden Dateien latex-korrekturansicht-abspann.tex bzw.
%% latex-leseansicht-abspann.tex).
%% ---------------------------------------------------------------

\normalsize

% Das esempio-Environment wird nur in der Leseansicht benötigt
\ifkorrekturansicht\else
\newenvironment{esempio}[3]%
{
    \vspace{1.5ex}
    \rlap{\underline{#1}}
    \par
    \setlength{\parindent}{0cm}
    \nopagebreak
    \leftskip=#2cm
    \rightskip=#3cm
}
{
    \par
}
\fi

\doendnotes{C}
\bigskip
\vfill

\clearpage

\footnotesize

\ifkorrekturansicht
  \lohead{\textsc{register}}
\fi

% theindex-Environment neu definieren ohne reledmac
\makeatletter
\renewenvironment{theindex}{%
  \ifkorrekturansicht
    \section*{\indexname}%
  \else
    \subsubsection*{Index der erwähnten Entitäten}%
  \fi
  \setlength{\parindent}{0pt}%
  \setlength{\parskip}{0pt plus 0.3pt}%
  \let\item\@idxitem
}{%
  \ifkorrekturansicht\clearpage\fi
}
\makeatother

\IfFileExists{\jobname-pw.ind}{\input{\jobname-pw.ind}}{}

% Quellenangabe nur in der Leseansicht
\ifkorrekturansicht\else
% Fallback-Definitionen, falls die .tex-Datei \titel etc. nicht gesetzt hat
\providecommand{\titel}{}
\providecommand{\editorInnen}{}
\providecommand{\dateiname}{\jobname}

\vspace{3cm}

\vfill

\footnotesize
\textsc{Quelle}: \titel. Herausgegeben von {\editorInnen}. In: \emph{Arthur Schnitzler: Briefwechsel mit Autorinnen und Autoren}.
 Digitale Edition, https://schnitzler-briefe.acdh.oeaw.ac.at/{\dateiname}.html (Stand \today)
\fi

\end{document}


