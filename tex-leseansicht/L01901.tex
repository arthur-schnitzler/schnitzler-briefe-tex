%% latex-korrekturansicht-vorspann.tex
%% Vorspann für die Korrekturansicht.
%% Lädt die gemeinsame Datei latex-vorspann.tex mit gesetztem Schalter.

\newif\ifkorrekturansicht
\korrekturansichttrue

\input{../tex-inputs/latex-vorspann}


\section[Arthur Schnitzler an Hermann Bahr, 14. 12. 1909]{L01901 Arthur Schnitzler an Hermann Bahr, 14. 12. 1909}
\nopagebreak\mylabel{L01901v}
\rehead{ }\normalsize\beginnumbering\briefempfaengerindex{Bahr, Hermann@\textsc{Bahr, Hermann}!zzzSchnitzler, Arthur@\emph{von Arthur Schnitzler}!1909-12-141@{14. 12. 1909}|(be}
\toendnotes[C]{\smallbreak\pagebreak[2]}\Standort{TMW, HS AM 60150 Ba.}
\physDesc{Briefkarte, 2 Blätter, 4 Seiten, , 1554 Zeichen (die zweite Karte von Schnitzler mit »II.«
                                 beschriftet)
\newline{}Handschrift: schwarze Tinte, deutsche Kurrent}
\buchAbdrucke{\weitereDrucke{1) Arthur Schnitzler: \emph{The Letters of Arthur Schnitzler to Hermann Bahr}. Chapel Hill: \emph{The University of North Carolina Press} 1978, S. 104–105.} \weitereDrucke{2) Hermann Bahr, Arthur Schnitzler: \emph{Briefwechsel, Aufzeichnungen, Dokumente (1891–1931)}. Göttingen: \emph{Wallstein} 2018, S. 428–429.} }\toendnotes[C]{\smallbreak}
\pstart
           {\pb}\textcolor{gray}{\textbf{Dr. Arthur Schnitzler}}\hfill 14/12 09\pend
           
\pstart
           \textcolor{gray}{\textbf{Wien XVIII. Spoettelgasse 7\oindex{Edmund-Weiss-Gasse 7@\textbf{Edmund-Weiß-Gasse 7}, \emph{Wohngebäude (K.WHS)}|pw}.}}\pend
           \vspace{0.5em}
\pstart
           mein lieber Hermann, bei Berlin\oindex{Berlin@\textbf{Berlin}, \emph{P.PPLC}|pw}er Gelegenheit einmal Halle\oindex{Halle (Saale)@\textbf{Halle (Saale)}, \emph{P.PPL}|pw}
               mitzunehmen hab ich mir längſt vorgeno{\geminationm}en – nur fügt es
               ſich immer ſo ſchwer, weil man ja viel früher einen besti{\geminationm}ten Vorleſe-Tag fixiren muſs als man den Berliner\oindex{Berlin@\textbf{Berlin}, \emph{P.PPLC}|pw} Premièrentag weiſs. Und mir perſönlich macht weder das Zweck-Reiſen
               noch das Vorleſen (in großen Räumen) ſonderlich {\pb}viel Spaſs. Aber wir
               wollen ſehen. Deine Gicht aber laß dir lieber von einem Dichter als von einem Oberingenieur\pwindex{Bacher, Oskar @\textsc{Bacher, Oskar}, \emph{Oberingenieur/Oberingenieurin}|pwv} behandeln – (nur
               nicht von einem Arzt natürlich) Ich ſtehe dir ſtets zur Verfügung – und hoffe
               mediziniſch ſchon genug vergeſſen zu haben, um dir nicht empfindlich zu ſchaden.\pend
           
\pstart
           Ja, wenn ich eine luſtige Novelle hätte! Und \textcolor{gray}{nun} gar eine kurze!
               Mit dem Gegentheil ka{\geminationn} ich dienen: Tragoedie in 5 Akten und einem Vorſpiel\pwindex{junge Medardus. Dramatische Historie in einem Vorspiel und fuenf Aufzuegen@\emph{Der junge Medardus. Dramatische Historie in einem Vorspiel und fünf Aufzügen}|pwv}
               aber die eignet ſich eher zum Aufgeführtwerden {\pb}(Wie du ſchon daraus
               erſehen kannſt, daſs es mir nicht möglich iſt, von \textsc{Schlenther}\pwindex{Schlenther, Paul 20.08.1854 – 30.04.1916@\textsc{Schlenther, Paul} (20.08.1854 – 30.04.1916), \emph{Schriftsteller/Schriftstellerin, Kritiker/Kritikerin, Theaterleiter/Theaterleiterin}|pw}{ }ſowohl als von \textsc{Reinhardt}\pwindex{Reinhardt, Max 09.09.1873 – 30.10.1943@\textsc{Reinhardt, Max} (09.09.1873 – 30.10.1943), \emph{Theaterleiter/Theaterleiterin, Regisseur/Regisseurin, Schauspieler/Schauspielerin}|pw} eine endgiltige Entſcheidung zu kriegen.) – Die \textsc{Comtesse Mizzi}\pwindex{Komtesse Mizzi oder: Der Familientag@\emph{Komtesse Mizzi oder: Der Familientag}|pw} wird nun doch nicht zu deinem »Concert\pwindex{Konzert. Lustspiel in drei Akten@\emph{Das Konzert. Lustspiel in drei Akten}|pw}«
               gegeben, der Abend würde zu lang, ſchreibt Brahm\pwindex{Brahm, Otto 05.02.1856 – 28.11.1912@\textsc{Brahm, Otto} (05.02.1856 – 28.11.1912), \emph{Theaterleiter/Theaterleiterin, Regisseur/Regisseurin}|pw}. Dabei hatt ich ſchon an den Münchner\oindex{Muenchen@\textbf{München}, \emph{P.PPLA}|pw}{ }Speidel\pwindex{Speidel, Albert von 26.01.1858 – 01.09.1912@\textsc{Speidel, Albert von} (26.01.1858 – 01.09.1912), \emph{Theaterleiter/Theaterleiterin}|pw}{ }{\pb}ſchreiben laſſen, er
               möchte auch womöglich die
                  zwei Stücke\pwindex{Komtesse Mizzi oder: Der Familientag@\emph{Komtesse Mizzi oder: Der Familientag}|pwv}\pwindex{Konzert. Lustspiel in drei Akten@\emph{Das Konzert. Lustspiel in drei Akten}|pwv} zusa{\geminationm}enſpielen. Nun hat \textsc{Speidel}\pwindex{Speidel, Albert von 26.01.1858 – 01.09.1912@\textsc{Speidel, Albert von} (26.01.1858 – 01.09.1912), \emph{Theaterleiter/Theaterleiterin}|pw} aber die \textsc{Comtesse}\pwindex{Komtesse Mizzi oder: Der Familientag@\emph{Komtesse Mizzi oder: Der Familientag}|pw} wegen Frivolität, Kinderkriegen und Liebhaber-haben refuſirt.\pend
           
\pstart
           Die Hoffnung dich wieder einmal zu ſprechen, geb ich noch immer nicht auf. Vielleicht
               auf dem Se{\geminationm}ering\oindex{Semmering@\textbf{Semmering}, \emph{A.ADM3}|pw}.
               Und daſs du den Leuten allerorten ſo viel von mir erzählſt, dank ich dir von Herzen.
                  Wir\pwindex{Schnitzler, Olga 17.01.1882 – 13.01.1970@\textsc{Schnitzler, Olga} (17.01.1882 – 13.01.1970), \emph{Schauspieler/Schauspielerin, Sänger/Sängerin}|pwv} grüßen alle aufs beſte
               und wollen auch Deiner verehrten Frau\pwindex{Bahr-Mildenburg, Anna 29.11.1872 – 27.01.1947@\textsc{Bahr-Mildenburg, Anna} (29.11.1872 – 27.01.1947), \emph{Sänger/Sängerin}|pwv} empfohlen ſein.\pend
           \pstart Dein getreuer \spacefill\mbox{Arthur.}\pend{}\selectlanguage{ngerman}\endnumbering\briefempfaengerindex{Bahr, Hermann@\textsc{Bahr, Hermann}!zzzSchnitzler, Arthur@\emph{von Arthur Schnitzler}!1909-12-141@{14. 12. 1909}|)be}\mylabel{L01901h}  \normalsize

\doendnotes{C}
\bigskip
\vfill

\clearpage

\footnotesize

\lohead{\textsc{register}}

% Definiere theindex-Environment komplett neu ohne reledmac
\makeatletter
\renewenvironment{theindex}{%
  \section*{\indexname}%
  \setlength{\parindent}{0pt}%
  \setlength{\parskip}{0pt plus 0.3pt}%
  \let\item\@idxitem
}{%
  \clearpage
}
\makeatother

\IfFileExists{\jobname-pw.ind}{\input{\jobname-pw.ind}}{}

\end{document}

      