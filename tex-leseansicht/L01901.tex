%% latex-leseansicht-vorspann.tex
%% Vorspann für die Leseansicht.
%% Lädt die gemeinsame Datei latex-vorspann.tex mit nicht gesetztem Schalter.

\newif\ifkorrekturansicht
\korrekturansichtfalse

\input{../tex-inputs/latex-vorspann}


         
         \newcommand{\erwaehntePersonen}{Personen: Oskar Bacher, Hermann Bahr, Anna Bahr-Mildenburg, Otto Brahm, Max Reinhardt, Paul Schlenther, Olga Schnitzler, Albert von Speidel}
         \newcommand{\erwaehnteInstitutionen}{}
         \newcommand{\erwaehnteOrte}{Orte: Berlin, Edmund-Weiß-Gasse, Halle an der Saale, München, Semmering, Wien}
         \newcommand{\erwaehnteWerke}{Werke: Das Konzert. Lustspiel in drei Akten, Der junge Medardus. Dramatische Historie in einem Vorspiel und fünf Aufzügen, Komtesse Mizzi oder Der Familientag}
               \section[Arthur Schnitzler an Hermann Bahr, 14. 12. 1909]{ Arthur Schnitzler an Hermann Bahr, 14. 12. 1909}\nopagebreak\mylabel{v}\rehead{ }\begin{ledgroupsized}[t]{13cm}\normalsize\beginnumbering \toendnotes[C]{\smallbreak\pagebreak[2]} \Standort{TMW, HS AM 60150 Ba.}
\physDesc{Briefkarte, 2 Karten (die zweite Karte von Schnitzler mit »II.« beschriftet), 4 Seiten
\newline{}Handschrift: schwarze Tinte, deutsche Kurrent}\buchAbdrucke{\weitereDrucke{1) \emph{14. 12. 1909, Abschrift.} In: Arthur Schnitzler: \emph{The Letters of Arthur Schnitzler to Hermann Bahr}. Edited, annotated, and with an introduction, by Donald G.
                        Daviau. Chapel Hill: \emph{The University of North Carolina Press} 1978, S. 104–105 (University of North Carolina studies in the Germanic languages
                        and literatures, 89).} \weitereDrucke{2) Hermann Bahr, Arthur Schnitzler: \emph{Briefwechsel, Aufzeichnungen, Dokumente (1891–1931)}. Hg. Kurt Ifkovits und Martin Anton Müller. Göttingen: \emph{Wallstein} 2018, S. 428–429.} }\toendnotes[C]{\smallbreak}\pstart
           \noindent{}{\pb}\textcolor{gray}{\textbf{Dr. Arthur Schnitzler}}\hfill 14/12 09\pend
           \pstart
           \textcolor{gray}{\textbf{Wien XVIII. Spoettelgasse 7\oindex{Edmund-Weiss-Gasse@\textbf{Edmund-Weiß-Gasse}|pw}.}}\pend
           \pstart
           mein lieber Hermann, bei Berlin\oindex{Berlin@\textbf{Berlin}|pw}er
               Gelegenheit einmal Halle\oindex{Halle an der Saale@\textbf{Halle an der Saale}|pw} mitzunehmen hab ich mir
               längſt vorgeno{\geminationm}en – nur fügt es ſich immer ſo ſchwer,
               weil man ja viel früher einen besti{\geminationm}ten Vorleſe-Tag
               fixiren muſs als man den Berliner\oindex{Berlin@\textbf{Berlin}|pw} Premièrentag
               weiſs. Und mir perſönlich macht weder das Zweck-Reiſen noch das Vorleſen (in großen
               Räumen) ſonderlich {\pb}viel Spaſs. Aber wir wollen ſehen. Deine Gicht aber laß dir lieber von einem
               Dichter als von einem Oberingenieur\pwindex{Bacher, Oskar @\textsc{Bacher, Oskar}, \emph{Oberingenieur}|pwv} behandeln – (nur nicht von einem Arzt natürlich) Ich ſtehe dir
               ſtets zur Verfügung – und hoffe mediziniſch ſchon genug vergeſſen zu haben, um dir
               nicht empfindlich zu ſchaden.\pend
           \pstart
           Ja, wenn ich eine luſtige Novelle hätte! Und \textcolor{gray}{nun} gar eine kurze!
               Mit dem Gegentheil ka{\geminationn} ich dienen: Tragoedie in 5 Akten und einem Vorſpiel\pwindex{Schnitzler, Arthur 15.05.1862 – 21.10.1931@\textsc{Schnitzler, Arthur} (15.05.1862 – 21.10.1931), \emph{Schriftsteller, Mediziner}!junge Medardus. Dramatische Historie in einem Vorspiel und fuenf Aufzuegen1910-10-26@\strich\emph{Der junge Medardus. Dramatische Historie in einem Vorspiel und fünf Aufzügen} {[}1910-10-26{]}|pwv} aber
               die eignet ſich eher zum Aufgeführtwerden {\pb}(Wie du ſchon daraus
               erſehen kannſt, daſs es mir nicht möglich iſt, von \textsc{Schlenther}\pwindex{Schlenther, Paul 20.08.1854 – 30.04.1916@\textsc{Schlenther, Paul} (20.08.1854 – 30.04.1916), \emph{Schriftsteller, Kritiker, Theaterleiter}|pw}{ }ſowohl als von \textsc{Reinhardt}\pwindex{Reinhardt, Max 09.09.1873 – 30.10.1943@\textsc{Reinhardt, Max} (09.09.1873 – 30.10.1943), \emph{Theaterleiter, Regisseur, Schauspieler}|pw} eine endgiltige Entſcheidung zu kriegen.) – Die \textsc{Comtesse Mizzi}\pwindex{Schnitzler, Arthur 15.05.1862 – 21.10.1931@\textsc{Schnitzler, Arthur} (15.05.1862 – 21.10.1931), \emph{Schriftsteller, Mediziner}!Komtesse Mizzi oder Der Familientag1908-04-19@\strich\emph{Komtesse Mizzi oder Der Familientag} {[}1908-04-19{]}|pw} wird nun doch nicht zu deinem »Concert\pwindex{Bahr, Hermann 19.07.1863 – 15.01.1934@\textsc{Bahr, Hermann} (19.07.1863 – 15.01.1934), \emph{Schriftsteller, Kritiker}!Konzert. Lustspiel in drei Akten1909@\strich\emph{Das Konzert. Lustspiel in drei Akten} {[}1909{]}|pw}«
               gegeben, der Abend würde zu lang, ſchreibt Brahm\pwindex{Brahm, Otto 05.02.1856 – 28.11.1912@\textsc{Brahm, Otto} (05.02.1856 – 28.11.1912), \emph{Theaterleiter, Regisseur}|pw}.
               Dabei hatt ich ſchon an den Münchner\oindex{Muenchen@\textbf{München}|pw}{ }Speidel\pwindex{Speidel, Albert von 26.01.1858 – 01.09.1912@\textsc{Speidel, Albert von} (26.01.1858 – 01.09.1912), \emph{Theaterleiter}|pw}{ }{\pb}ſchreiben laſſen, er
               möchte auch womöglich die zwei
                  Stücke\pwindex{Schnitzler, Arthur 15.05.1862 – 21.10.1931@\textsc{Schnitzler, Arthur} (15.05.1862 – 21.10.1931), \emph{Schriftsteller, Mediziner}!Komtesse Mizzi oder Der Familientag1908-04-19@\strich\emph{Komtesse Mizzi oder Der Familientag} {[}1908-04-19{]}|pwv}\pwindex{Bahr, Hermann 19.07.1863 – 15.01.1934@\textsc{Bahr, Hermann} (19.07.1863 – 15.01.1934), \emph{Schriftsteller, Kritiker}!Konzert. Lustspiel in drei Akten1909@\strich\emph{Das Konzert. Lustspiel in drei Akten} {[}1909{]}|pwv} zusa{\geminationm}enſpielen. Nun hat \textsc{Speidel}\pwindex{Speidel, Albert von 26.01.1858 – 01.09.1912@\textsc{Speidel, Albert von} (26.01.1858 – 01.09.1912), \emph{Theaterleiter}|pw} aber die \textsc{Comtesse}\pwindex{Schnitzler, Arthur 15.05.1862 – 21.10.1931@\textsc{Schnitzler, Arthur} (15.05.1862 – 21.10.1931), \emph{Schriftsteller, Mediziner}!Komtesse Mizzi oder Der Familientag1908-04-19@\strich\emph{Komtesse Mizzi oder Der Familientag} {[}1908-04-19{]}|pw} wegen Frivolität, Kinderkriegen und Liebhaber-haben refuſirt.\pend
           \pstart
           Die Hoffnung dich wieder einmal zu ſprechen, geb ich noch immer nicht auf. Vielleicht
               auf dem Se{\geminationm}ering\oindex{Semmering@\textbf{Semmering}|pw}. Und daſs du den Leuten allerorten
               ſo viel von mir erzählſt, dank ich dir von Herzen. Wir\pwindex{Schnitzler, Olga 17.01.1882 – 13.01.1970@\textsc{Schnitzler, Olga} (17.01.1882 – 13.01.1970), \emph{Schauspielerin, Sängerin}|pwv} grüßen alle aufs beſte und wollen auch Deiner verehrten
                  Frau\pwindex{Bahr-Mildenburg, Anna 29.11.1872 – 27.01.1947@\textsc{Bahr-Mildenburg, Anna} (29.11.1872 – 27.01.1947), \emph{Sängerin}|pwv} empfohlen ſein.\pend
           \pstart Dein getreuer \spacefill\mbox{Arthur.}\pend{}
         
         \endnumbering\mylabel{h}\end{ledgroupsized}  \newcommand{\dateiname}{L01901}\newcommand{\titel}{Arthur Schnitzler an Hermann Bahr, 14. 12. 1909}\newcommand{\editorInnen}{ Kurt Ifkovits,  Martin Anton Müller}%% latex-leseansicht-abspann.tex
%% Abspann für die Leseansicht.
%% Der Schalter \ifkorrekturansicht ist bereits durch den Vorspann gesetzt.

%% latex-abspann.tex
%% Gemeinsamer Abspann für Korrekturansicht und Leseansicht.
%% Setzt den Schalter \ifkorrekturansicht voraus (gesetzt in den
%% einbindenden Dateien latex-korrekturansicht-abspann.tex bzw.
%% latex-leseansicht-abspann.tex).
%% ---------------------------------------------------------------

\normalsize

% Das esempio-Environment wird nur in der Leseansicht benötigt
\ifkorrekturansicht\else
\newenvironment{esempio}[3]%
{
    \vspace{1.5ex}
    \rlap{\underline{#1}}
    \par
    \setlength{\parindent}{0cm}
    \nopagebreak
    \leftskip=#2cm
    \rightskip=#3cm
}
{
    \par
}
\fi

\doendnotes{C}
\bigskip
\vfill

\clearpage

\footnotesize

\ifkorrekturansicht
  \lohead{\textsc{register}}
\fi

% theindex-Environment neu definieren ohne reledmac
\makeatletter
\renewenvironment{theindex}{%
  \ifkorrekturansicht
    \section*{\indexname}%
  \else
    \subsubsection*{Index der erwähnten Entitäten}%
  \fi
  \setlength{\parindent}{0pt}%
  \setlength{\parskip}{0pt plus 0.3pt}%
  \let\item\@idxitem
}{%
  \ifkorrekturansicht\clearpage\fi
}
\makeatother

\IfFileExists{\jobname-pw.ind}{\input{\jobname-pw.ind}}{}

% Quellenangabe nur in der Leseansicht
\ifkorrekturansicht\else
% Fallback-Definitionen, falls die .tex-Datei \titel etc. nicht gesetzt hat
\providecommand{\titel}{}
\providecommand{\editorInnen}{}
\providecommand{\dateiname}{\jobname}

\vspace{3cm}

\vfill

\footnotesize
\textsc{Quelle}: \titel. Herausgegeben von {\editorInnen}. In: \emph{Arthur Schnitzler: Briefwechsel mit Autorinnen und Autoren}.
 Digitale Edition, https://schnitzler-briefe.acdh.oeaw.ac.at/{\dateiname}.html (Stand \today)
\fi

\end{document}


      