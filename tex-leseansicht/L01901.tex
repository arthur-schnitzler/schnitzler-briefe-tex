%% latex-leseansicht-vorspann.tex
%% Vorspann für die Leseansicht.
%% Lädt die gemeinsame Datei latex-vorspann.tex mit nicht gesetztem Schalter.

\newif\ifkorrekturansicht
\korrekturansichtfalse

\input{../tex-inputs/latex-vorspann}


\section[Arthur Schnitzler an Hermann Bahr, 14. 12. 1909]{L01901 Arthur Schnitzler an Hermann Bahr, 14. 12. 1909}
\nopagebreak\mylabel{L01901v}
\rehead{ }\normalsize\beginnumbering\briefempfaengerindex{Bahr, Hermann@\textsc{Bahr, Hermann}!zzzSchnitzler, Arthur@\emph{von Arthur Schnitzler}!1909-12-141@{14. 12. 1909}|(be}
\toendnotes[C]{\smallbreak\pagebreak[2]}
\correspDesc{Versand  durch Arthur Schnitzler am 14. 12. 1909 in Wien
\newline{}Erhalt  durch Hermann Bahr im Zeitraum [14. 12. 1909 – 18. 12. 1909?] \textbf{Ort fehlend} }\toendnotes[C]{\smallbreak}
\Standort{TMW, HS AM 60150 Ba.}
\physDesc{Briefkarte, 2 Blätter, 4 Seiten, , 1554 Zeichen (die zweite Karte von Schnitzler mit »II.«
                                 beschriftet)
\newline{}Handschrift: schwarze Tinte, deutsche Kurrent}
\buchAbdrucke{\weitereDrucke{1) \emph{14. 12. 1909, Abschrift.} In: Arthur Schnitzler: \emph{The Letters of Arthur Schnitzler to Hermann Bahr}. Edited, annotated, and with an introduction, by Donald G. Daviau. Chapel Hill: \emph{The University of North Carolina Press} 1978, S. 104–105 (University of North Carolina studies in the Germanic languages
                        and literatures, 89).} \weitereDrucke{2) Hermann Bahr, Arthur Schnitzler: \emph{Briefwechsel, Aufzeichnungen, Dokumente (1891–1931)}. Herausgegeben von Kurt Ifkovits und Martin Anton Müller. Göttingen: \emph{Wallstein} 2018, S. 428–429.} }\toendnotes[C]{\smallbreak}
\pstart
           {\pb}\textcolor{gray}{\textbf{Dr. Arthur Schnitzler}}\hfill 14/12 09\pend
           
\pstart
           \textcolor{gray}{\textbf{Wien XVIII. Spoettelgasse 7\oindex{Wien@\textbf{Wien}!XVIII., Währing@\textbf{XVIII., Währing}!Edmund-Weiß-Gasse 7@\textbf{Edmund-Weiß-Gasse 7}, \emph{Wohngebäude}|pw}.}}\pend
           \vspace{0.5em}
\pstart
           mein lieber Hermann, bei Berlin\oindex{Berlin@\textbf{Berlin}, \emph{Hauptstadt}|pw}er Gelegenheit einmal Halle\oindex{Halle (Saale)@\textbf{Halle (Saale)}|pw}
               mitzunehmen hab ich mir längſt vorgeno{\geminationm}en – nur fügt es{ }ſich immer{ }ſo{ }ſchwer, weil man ja viel früher einen besti{\geminationm}ten Vorleſe-Tag fixiren muſs als man den Berliner\oindex{Berlin@\textbf{Berlin}, \emph{Hauptstadt}|pw} Premièrentag weiſs. Und mir perſönlich macht weder das Zweck-Reiſen
               noch das Vorleſen (in großen Räumen){ }ſonderlich {\pb}viel Spaſs. Aber wir
               wollen{ }ſehen. Deine Gicht aber laß dir lieber von einem Dichter als von einem Oberingenieur\pwindex{Bacher, Oskar @\textsc{Bacher, Oskar}, \emph{Oberingenieur}|pwv} behandeln – (nur
               nicht von einem Arzt natürlich) Ich{ }ſtehe dir{ }ſtets zur Verfügung – und hoffe
               mediziniſch{ }ſchon genug vergeſſen zu haben, um dir nicht empfindlich zu{ }ſchaden.\pend
           
\pstart
           Ja, wenn ich eine luſtige Novelle hätte! Und \textcolor{gray}{nun} gar eine kurze!
               Mit dem Gegentheil ka{\geminationn} ich dienen: Tragoedie in 5 Akten und einem Vorſpiel\pwindex{Schnitzler, Arthur 15.\,5.\,1862 Wien – 21.\,10.\,1931 ebd.@\textsc{Schnitzler, Arthur} (15.\,5.\,1862 Wien – 21.\,10.\,1931 ebd.), \emph{Schriftsteller, Mediziner}!junge Medardus. Dramatische Historie in einem Vorspiel und fünf Aufzügen@\strich\emph{Der junge Medardus. Dramatische Historie in einem Vorspiel und fünf Aufzügen}|pwv}
               aber die eignet{ }ſich eher zum Aufgeführtwerden {\pb}(Wie du{ }ſchon daraus
               erſehen kannſt, daſs es mir nicht möglich iſt, von \textsc{Schlenther}\pwindex{Schlenther, Paul 20.\,8.\,1854 Chernyakhovsk – 30.\,4.\,1916 Berlin@\textsc{Schlenther, Paul} (20.\,8.\,1854 Chernyakhovsk – 30.\,4.\,1916 Berlin), \emph{Schriftsteller, Kritiker, Theaterleiter}|pw}{ }ſowohl als von \textsc{Reinhardt}\pwindex{Reinhardt, Max 9.\,9.\,1873 Baden bei Wien – 30.\,10.\,1943 New York City@\textsc{Reinhardt, Max} (9.\,9.\,1873 Baden bei Wien – 30.\,10.\,1943 New York City), \emph{Theaterleiter, Regisseur, Schauspieler}|pw} eine endgiltige Entſcheidung zu kriegen.) – Die \textsc{Comtesse Mizzi}\pwindex{Schnitzler, Arthur 15.\,5.\,1862 Wien – 21.\,10.\,1931 ebd.@\textsc{Schnitzler, Arthur} (15.\,5.\,1862 Wien – 21.\,10.\,1931 ebd.), \emph{Schriftsteller, Mediziner}!Komtesse Mizzi oder: Der Familientag@\strich\emph{Komtesse Mizzi oder: Der Familientag}|pw} wird nun doch nicht zu deinem »Concert\pwindex{Bahr, Hermann 19.\,7.\,1863 Linz – 15.\,1.\,1934 München@\textsc{Bahr, Hermann} (19.\,7.\,1863 Linz – 15.\,1.\,1934 München), \emph{Schriftsteller, Kritiker}!Konzert. Lustspiel in drei Akten@\strich\emph{Das Konzert. Lustspiel in drei Akten}|pw}«
               gegeben, der Abend würde zu lang,{ }ſchreibt Brahm\pwindex{Brahm, Otto 5.\,2.\,1856 Hamburg – 28.\,11.\,1912 Berlin@\textsc{Brahm, Otto} (5.\,2.\,1856 Hamburg – 28.\,11.\,1912 Berlin), \emph{Theaterleiter, Regisseur}|pw}. Dabei hatt ich{ }ſchon an den Münchner\oindex{München@\textbf{München}|pw}{ }Speidel\pwindex{Speidel, Albert von 26.\,1.\,1858 München – 1.\,9.\,1912 ebd.@\textsc{Speidel, Albert von} (26.\,1.\,1858 München – 1.\,9.\,1912 ebd.), \emph{Theaterleiter}|pw}{ }{\pb}ſchreiben laſſen, er
               möchte auch womöglich die
                  zwei Stücke\pwindex{Schnitzler, Arthur 15.\,5.\,1862 Wien – 21.\,10.\,1931 ebd.@\textsc{Schnitzler, Arthur} (15.\,5.\,1862 Wien – 21.\,10.\,1931 ebd.), \emph{Schriftsteller, Mediziner}!Komtesse Mizzi oder: Der Familientag@\strich\emph{Komtesse Mizzi oder: Der Familientag}|pwv}\pwindex{Bahr, Hermann 19.\,7.\,1863 Linz – 15.\,1.\,1934 München@\textsc{Bahr, Hermann} (19.\,7.\,1863 Linz – 15.\,1.\,1934 München), \emph{Schriftsteller, Kritiker}!Konzert. Lustspiel in drei Akten@\strich\emph{Das Konzert. Lustspiel in drei Akten}|pwv} zusa{\geminationm}enſpielen. Nun hat \textsc{Speidel}\pwindex{Speidel, Albert von 26.\,1.\,1858 München – 1.\,9.\,1912 ebd.@\textsc{Speidel, Albert von} (26.\,1.\,1858 München – 1.\,9.\,1912 ebd.), \emph{Theaterleiter}|pw} aber die \textsc{Comtesse}\pwindex{Schnitzler, Arthur 15.\,5.\,1862 Wien – 21.\,10.\,1931 ebd.@\textsc{Schnitzler, Arthur} (15.\,5.\,1862 Wien – 21.\,10.\,1931 ebd.), \emph{Schriftsteller, Mediziner}!Komtesse Mizzi oder: Der Familientag@\strich\emph{Komtesse Mizzi oder: Der Familientag}|pw} wegen Frivolität, Kinderkriegen und Liebhaber-haben refuſirt.\pend
           
\pstart
           Die Hoffnung dich wieder einmal zu{ }ſprechen, geb ich noch immer nicht auf. Vielleicht
               auf dem Se{\geminationm}ering\oindex{Semmering@\textbf{Semmering}, \emph{Verwaltungsgebiet}|pw}.
               Und daſs du den Leuten allerorten{ }ſo viel von mir erzählſt, dank ich dir von Herzen.
                  Wir\pwindex{Schnitzler, Olga 17.\,1.\,1882 Wien – 13.\,1.\,1970 Lugano@\textsc{Schnitzler, Olga} (17.\,1.\,1882 Wien – 13.\,1.\,1970 Lugano), \emph{Schauspielerin, Sängerin}|pwv} grüßen alle aufs beſte
               und wollen auch Deiner verehrten Frau\pwindex{Bahr-Mildenburg, Anna 29.\,11.\,1872 Wien – 27.\,1.\,1947 ebd.@\textsc{Bahr-Mildenburg, Anna} (29.\,11.\,1872 Wien – 27.\,1.\,1947 ebd.), \emph{Sängerin}|pwv} empfohlen{ }ſein.\pend
           \pstart Dein getreuer \spacefill\mbox{Arthur.}\pend{}\selectlanguage{ngerman}\endnumbering\briefempfaengerindex{Bahr, Hermann@\textsc{Bahr, Hermann}!zzzSchnitzler, Arthur@\emph{von Arthur Schnitzler}!1909-12-141@{14. 12. 1909}|)be}\mylabel{L01901h}  \newcommand{\dateiname}{L01901}\newcommand{\titel}{Arthur Schnitzler an Hermann Bahr, 14. 12. 1909}\newcommand{\editorInnen}{Herausgegeben von Martin Anton Müller}%% latex-leseansicht-abspann.tex
%% Abspann für die Leseansicht.
%% Der Schalter \ifkorrekturansicht ist bereits durch den Vorspann gesetzt.

%% latex-abspann.tex
%% Gemeinsamer Abspann für Korrekturansicht und Leseansicht.
%% Setzt den Schalter \ifkorrekturansicht voraus (gesetzt in den
%% einbindenden Dateien latex-korrekturansicht-abspann.tex bzw.
%% latex-leseansicht-abspann.tex).
%% ---------------------------------------------------------------

\normalsize

% Das esempio-Environment wird nur in der Leseansicht benötigt
\ifkorrekturansicht\else
\newenvironment{esempio}[3]%
{
    \vspace{1.5ex}
    \rlap{\underline{#1}}
    \par
    \setlength{\parindent}{0cm}
    \nopagebreak
    \leftskip=#2cm
    \rightskip=#3cm
}
{
    \par
}
\fi

\doendnotes{C}
\bigskip
\vfill

\clearpage

\footnotesize

\ifkorrekturansicht
  \lohead{\textsc{register}}
\fi

% theindex-Environment neu definieren ohne reledmac
\makeatletter
\renewenvironment{theindex}{%
  \ifkorrekturansicht
    \section*{\indexname}%
  \else
    \subsubsection*{Index der erwähnten Entitäten}%
  \fi
  \setlength{\parindent}{0pt}%
  \setlength{\parskip}{0pt plus 0.3pt}%
  \let\item\@idxitem
}{%
  \ifkorrekturansicht\clearpage\fi
}
\makeatother

\IfFileExists{\jobname-pw.ind}{\input{\jobname-pw.ind}}{}

% Quellenangabe nur in der Leseansicht
\ifkorrekturansicht\else
% Fallback-Definitionen, falls die .tex-Datei \titel etc. nicht gesetzt hat
\providecommand{\titel}{}
\providecommand{\editorInnen}{}
\providecommand{\dateiname}{\jobname}

\vspace{3cm}

\vfill

\footnotesize
\textsc{Quelle}: \titel. Herausgegeben von {\editorInnen}. In: \emph{Arthur Schnitzler: Briefwechsel mit Autorinnen und Autoren}.
 Digitale Edition, https://schnitzler-briefe.acdh.oeaw.ac.at/{\dateiname}.html (Stand \today)
\fi

\end{document}


