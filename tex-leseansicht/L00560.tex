%% latex-leseansicht-vorspann.tex
%% Vorspann für die Leseansicht.
%% Lädt die gemeinsame Datei latex-vorspann.tex mit nicht gesetztem Schalter.

\newif\ifkorrekturansicht
\korrekturansichtfalse

\input{../tex-inputs/latex-vorspann}


               \section[Arthur Schnitzler an Richard Beer-Hofmann, 7. 7. 1896]{ Arthur Schnitzler an Richard Beer-Hofmann, 7. 7. 1896}\nopagebreak\mylabel{v}\rehead{ }\begin{ledgroupsized}[t]{13cm}\normalsize\beginnumbering\briefempfaengerindex{Beer-Hofmann, Richard@\textsc{Beer-Hofmann, Richard}!zzzSchnitzler, Arthur@\emph{von Arthur Schnitzler}!1896-07-071@{7. 7. 1896}|(be} \toendnotes[C]{\smallbreak\pagebreak[2]} \Standort{YCGL, MSS 31.}
\physDesc{Postkarte
\newline{}Handschrift: Bleistift, deutsche Kurrent\newline{}Versand: 1) Stempel: »\nobreak{}\oindex{Hotel zum Kronprinzen@\textbf{Hotel zum Kronprinzen}|pwk}Hotel zum Kronprinzen Hamburg.\nobreak{}«.  2) Stempel: »\nobreak{}\oindex{Luebeck@\textbf{Lübeck}|pwk}Lübeck, 7. 7. 96, 6–7N\nobreak{}«. 3) Stempel: »\nobreak{}\oindex{St. Gilgen@\textbf{St. Gilgen}|pwk}St. Gilgen, 9. 7. 96\nobreak{}«. }\buchAbdrucke{\weitereDrucke{Arthur Schnitzler, Richard Beer-Hofmann: \emph{Briefwechsel 1891–1931}. Hg. Konstanze Fliedl. Wien, Zürich: \emph{Europaverlag} 1992, S. 92.} }\toendnotes[C]{\smallbreak}\pstart{}{\pb}Herrn \textsc{Dr. Richard
                     Beer-Hofmann}\pend{}\pstart{}\textsc{Fürberg am St. Wolfgangsee}\oindex{Fuerberg@\textbf{Fürberg}|pw}\pend{}\pstart{}\textsc{in Oberoesterreich\oindex{Oberoesterreich@\textbf{Oberösterreich}|pw}}\pend{}{\bigskip}\pstart
           \noindent{}{\pb}Lieber Richard, leider muſs ich Mitteleuropa\oindex{Europa@\textbf{Europa}|pw} verlaſſen, ohne weitere Nachricht von Ihnen gefunden zu haben.
               Ich war 3 Tage in \textsc{Hamburg}\oindex{Hamburg@\textbf{Hamburg}|pw} u ſchreibe dieſe Karte in \textsc{Lübeck}\oindex{Luebeck@\textbf{Lübeck}|pw}, wohin ich mich ein Ausflug führte. Ich bin guter, aber nicht hoher Sti{\geminationm}ung. Heute Abend geh ich »an Bord« der \textsc{Sverre Sigurdson}. Iſt’s nicht ein trauriges Leben, darin man
               nicht einmal mehr »an Bord« ohne Anführungszeichen ſchreiben kann? – Ich hoffe
               in \textsc{Trondjhem}\oindex{Trondheim@\textbf{Trondheim}|pw} Briefe von Ihnen zu finden. Grüße Sie herzlich; grüßen Sie auch Paula\pwindex{Beer-Hofmann, Paula 25.02.1879 – 30.10.1939@\textsc{Beer-Hofmann, Paula} (25.02.1879 – 30.10.1939)|pw}\pend
           \pstart \label{T_L00560_1v}\edtext{Ihr}{\lemma{\textnormal{\emph{Ihr}}}\Cendnote{\textnormal{am
                  oberen Rand auf dem Kopf}}}\label{T_L00560_1h}\spacefill\mbox{Arthur}\pend{}\endnumbering\briefempfaengerindex{Beer-Hofmann, Richard@\textsc{Beer-Hofmann, Richard}!zzzSchnitzler, Arthur@\emph{von Arthur Schnitzler}!1896-07-071@{7. 7. 1896}|)be}\mylabel{h}\end{ledgroupsized}  \newcommand{\dateiname}{L00560}\newcommand{\titel}{Arthur Schnitzler an Richard Beer-Hofmann, 7. 7. 1896}\newcommand{\editorInnen}{Martin Anton Müller und Gerd-Hermann Susen}%% latex-leseansicht-abspann.tex
%% Abspann für die Leseansicht.
%% Der Schalter \ifkorrekturansicht ist bereits durch den Vorspann gesetzt.

%% latex-abspann.tex
%% Gemeinsamer Abspann für Korrekturansicht und Leseansicht.
%% Setzt den Schalter \ifkorrekturansicht voraus (gesetzt in den
%% einbindenden Dateien latex-korrekturansicht-abspann.tex bzw.
%% latex-leseansicht-abspann.tex).
%% ---------------------------------------------------------------

\normalsize

% Das esempio-Environment wird nur in der Leseansicht benötigt
\ifkorrekturansicht\else
\newenvironment{esempio}[3]%
{
    \vspace{1.5ex}
    \rlap{\underline{#1}}
    \par
    \setlength{\parindent}{0cm}
    \nopagebreak
    \leftskip=#2cm
    \rightskip=#3cm
}
{
    \par
}
\fi

\doendnotes{C}
\bigskip
\vfill

\clearpage

\footnotesize

\ifkorrekturansicht
  \lohead{\textsc{register}}
\fi

% theindex-Environment neu definieren ohne reledmac
\makeatletter
\renewenvironment{theindex}{%
  \ifkorrekturansicht
    \section*{\indexname}%
  \else
    \subsubsection*{Index der erwähnten Entitäten}%
  \fi
  \setlength{\parindent}{0pt}%
  \setlength{\parskip}{0pt plus 0.3pt}%
  \let\item\@idxitem
}{%
  \ifkorrekturansicht\clearpage\fi
}
\makeatother

\IfFileExists{\jobname-pw.ind}{\input{\jobname-pw.ind}}{}

% Quellenangabe nur in der Leseansicht
\ifkorrekturansicht\else
% Fallback-Definitionen, falls die .tex-Datei \titel etc. nicht gesetzt hat
\providecommand{\titel}{}
\providecommand{\editorInnen}{}
\providecommand{\dateiname}{\jobname}

\vspace{3cm}

\vfill

\footnotesize
\textsc{Quelle}: \titel. Herausgegeben von {\editorInnen}. In: \emph{Arthur Schnitzler: Briefwechsel mit Autorinnen und Autoren}.
 Digitale Edition, https://schnitzler-briefe.acdh.oeaw.ac.at/{\dateiname}.html (Stand \today)
\fi

\end{document}


      