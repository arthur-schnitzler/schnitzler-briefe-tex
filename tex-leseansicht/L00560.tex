%% latex-korrekturansicht-vorspann.tex
%% Vorspann für die Korrekturansicht.
%% Lädt die gemeinsame Datei latex-vorspann.tex mit gesetztem Schalter.

\newif\ifkorrekturansicht
\korrekturansichttrue

\input{../tex-inputs/latex-vorspann}


\section[Arthur Schnitzler an Richard Beer-Hofmann, 7. 7. 1896]{L00560 Arthur Schnitzler an Richard Beer-Hofmann, 7. 7. 1896}
\nopagebreak\mylabel{L00560v}
\rehead{ }\normalsize\beginnumbering\briefempfaengerindex{Beer-Hofmann, Richard@\textsc{Beer-Hofmann, Richard}!zzzSchnitzler, Arthur@\emph{von Arthur Schnitzler}!1896-07-071@{7. 7. 1896}|(be}
\toendnotes[C]{\smallbreak\pagebreak[2]}\Standort{YCGL, MSS 31.}
\physDesc{Postkarte, 571 Zeichen
\newline{}Handschrift: Bleistift, deutsche Kurrent
\newline{}Versand: 1) Stempel: »\nobreak{}\oindex{Hotel zum Kronprinzen [Hamburg, –1911]@\textbf{Hotel zum Kronprinzen [Hamburg, –1911]}, \emph{Hotel (K.HTL)}|pwk}Hotel zum Kronprinzen
                                       Hamburg.\nobreak{}«.   2) Stempel: »\nobreak{}\oindex{Luebeck@\textbf{Lübeck}, \emph{P.PPLA3}|pwk}Lübeck, 7. 7. 96, 6–7N\nobreak{}«.  3) Stempel: »\nobreak{}\oindex{St. Gilgen@\textbf{St. Gilgen}, \emph{A.ADM3}|pwk}St. Gilgen, 9. 7. 96\nobreak{}«. }
\buchAbdrucke{\weitereDrucke{Arthur Schnitzler, Richard Beer-Hofmann: \emph{Briefwechsel 1891–1931}. Wien, Zürich: \emph{Europaverlag} 1992, S. 92.} }\toendnotes[C]{\smallbreak}\pstart{}{\pb}Herrn \textsc{Dr. Richard
                     Beer-Hofmann}\pend{}\pstart{}\textsc{Fürberg am St. Wolfgangsee}\oindex{Fuerberg@\textbf{Fürberg}, \emph{P.PPL}|pw}\pend{}\pstart{}\textsc{in Oberoesterreich\oindex{Oberoesterreich@\textbf{Oberösterreich}, \emph{A.ADM1}|pw}}\pend{}{\bigskip}\vspace{1em}
\pstart
           \noindent{}{\pb}Lieber Richard, leider muſs ich Mitteleuropa\oindex{Europa@\textbf{Europa}, \emph{Kontinent (A.KNT)}|pw} verlaſſen, ohne weitere Nachricht von Ihnen gefunden zu haben.
               Ich war 3 Tage in \textsc{Hamburg}\oindex{Hamburg@\textbf{Hamburg}, \emph{P.PPLA}|pw} u ſchreibe dieſe Karte in \textsc{Lübeck}\oindex{Luebeck@\textbf{Lübeck}, \emph{P.PPLA3}|pw}, wohin ich mich ein Ausflug führte. Ich bin guter, aber nicht hoher Sti{\geminationm}ung. Heute Abend geh ich »an Bord« der \textsc{Sverre Sigurdson}. Iſt’s nicht ein trauriges Leben, darin man
               nicht einmal mehr »an Bord« ohne Anführungszeichen ſchreiben kann? – Ich hoffe in \textsc{Trondjhem}\oindex{Trondheim@\textbf{Trondheim}, \emph{P.PPLA2}|pw} Briefe von Ihnen zu finden. Grüße Sie herzlich; grüßen Sie auch Paula\pwindex{Beer-Hofmann, Paula 25.02.1879 – 30.10.1939@\textsc{Beer-Hofmann, Paula} (25.02.1879 – 30.10.1939)|pw}\pend
           \pstart \label{T_L00560-1v}\edtext{Ihr}{\lemma{\textnormal{\emph{Ihr}}}\Cendnote{\textnormal{am oberen Rand auf dem Kopf}}}\label{T_L00560-1}\spacefill\mbox{Arthur}\pend{}\selectlanguage{ngerman}\endnumbering\briefempfaengerindex{Beer-Hofmann, Richard@\textsc{Beer-Hofmann, Richard}!zzzSchnitzler, Arthur@\emph{von Arthur Schnitzler}!1896-07-071@{7. 7. 1896}|)be}\mylabel{L00560h}  \normalsize

\doendnotes{C}
\bigskip
\vfill

\clearpage

\footnotesize

\lohead{\textsc{register}}

% Definiere theindex-Environment komplett neu ohne reledmac
\makeatletter
\renewenvironment{theindex}{%
  \section*{\indexname}%
  \setlength{\parindent}{0pt}%
  \setlength{\parskip}{0pt plus 0.3pt}%
  \let\item\@idxitem
}{%
  \clearpage
}
\makeatother

\IfFileExists{\jobname-pw.ind}{\input{\jobname-pw.ind}}{}

\end{document}

      