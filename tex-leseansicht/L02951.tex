%% latex-korrekturansicht-vorspann.tex
%% Vorspann für die Korrekturansicht.
%% Lädt die gemeinsame Datei latex-vorspann.tex mit gesetztem Schalter.

\newif\ifkorrekturansicht
\korrekturansichttrue

\input{../tex-inputs/latex-vorspann}


\section[Arthur Schnitzler an Felix Salten, {[}10.? 9. 1891{]}]{L02951 Arthur Schnitzler an Felix Salten, {[}10.? 9. 1891{]}}
\nopagebreak\mylabel{L02951v}
\rehead{ }\normalsize\beginnumbering\briefempfaengerindex{Salten, Felix@\textsc{Salten, Felix}!zzzSchnitzler, Arthur@\emph{von Arthur Schnitzler}!1891-09-102@{{[}10.? 9. 1891{]}}|(be}
\toendnotes[C]{\smallbreak\pagebreak[2]}\Standort{Wienbibliothek im Rathaus, ZPH 1681, 2.1.516.}
\physDesc{Brief, 2 Blätter, 8 Seiten, 1636 Zeichen
\newline{}Handschrift: Bleistift, deutsche Kurrent
\newline{}Ordnung: mit Bleistift von unbekannter Hand Nummerierung der Doppelseiten:
                                    »25«–»28« }\toendnotes[C]{\smallbreak}
\pstart
           \raggedleft{}{\pb}Do{\geminationn}erſtg{ }Abend. \pend
           \vspace{0.5em}
\pstart
           Lieber Freund, ko{\geminationm} nach
               Hauſe, ſpät Abends, finde Ihren \label{K_L02951-1v}\edtext{Brief}{\lemma{\textnormal{\emph{Brief}}}\Cendnote{\textnormal{Felix Salten an Arthur Schnitzler, [10. 9. 1891].
               }}}\label{K_L02951-1}. Wie Sie in dieſem Augenblick jedenfalls ſchon wissen, hab ich Ihnen bereits
               2mal geſchrieben. Der erſte {\pb}Brief, den ich
               einfach an \textsc{F. S.} aus Wien\oindex{Wien@\textbf{Wien}, \emph{A.ADM2}|pw} in \textsc{Miskolez\oindex{Miskolc@\textbf{Miskolc}, \emph{P.PPLA}|pw}} adreſſirte, iſt offenbar nicht angeko{\geminationm}en, den
               zweiten mit der Hoteladreſſe, die ich im \textsc{Café Kugel\oindex{Cafe Kugel@\textbf{Café Kugel}, \emph{Kaffeehaus (K.KAF)}|pw}} erfuhr und den ich heute{ }Vormittag abſandte\textcolor{gray}{,} haben Sie wohl ſchon. Ihre
               Aufregung ist vollko{\geminationm}en {\pb}überflüſſig – ich habe nichts erfahren,
               nichts, nichts, und was ich geſehn habe, iſt, wie mein letzter Brief Ihnen wohl klar
               macht, harmlos genug. Und warum haben Sie denn plötzlich einen Rückfall\pwindex{Karlsburg, Bertha @\textsc{Karlsburg, Bertha}, \emph{Schauspieler/Schauspielerin}|pwuv}? Beko{\geminationm}en­ Sie nicht regelmäßig Nachricht? {\pb}Sind die Briefe nicht ſo wie Sie ſie
               wünſchen? – Bitte, reclamiren Sie meinen ersten Brief bei der Poſt. Von mir ſelbſt
               iſt nichts neues zu melden. Und fern am Horizont – Sie wissen schon, da leuchtet {\pb}\label{K_L02951-2v}\edtext{ſie\pwindex{Gluemer, Marie 03.07.1867 – 16.11.1925@\textsc{Glümer, Marie} (03.07.1867 – 16.11.1925), \emph{Schauspieler/Schauspielerin}|pwv}}{\lemma{\textnormal{\emph{ſie}}}\Cendnote{\textnormal{Marie Glümer\pwindex{Gluemer, Marie 03.07.1867 – 16.11.1925@\textsc{Glümer, Marie} (03.07.1867 – 16.11.1925), \emph{Schauspieler/Schauspielerin}|pwk}, mit der Schnitzler eine Liebesbeziehung führte}}}\label{K_L02951-2} manchmal
                  auf{\dotstwo} – Zuweilen waren es wohl auch Blitze. Aber es ist
               wunderſchön, wie ſie »an meinen Schmerz heranzureichen« ſucht, und die alte süße
               Lüge, daß es ja diesmal etwas andres, ach etwas ganz andres ist, beko{\geminationm}t {\pb}einen
               betäubenden Duft nach Wahrheit. – Schreiben Sie mir gleich wieder, wie es Ihnen geht,
               wie Sie Ihre Zeit verbringen. Wa{\geminationn} ko{\geminationm}en Sie \label{K_L02951-3v}\edtext{zurück}{\lemma{\textnormal{\emph{zurück}}}\Cendnote{\textnormal{Nachweislich sahen sich die
                  beiden erst am [28. 9. 1891?]
                  wieder.}}}\label{K_L02951-3}? Je eher, je lieber. Nicht wahr, wir \label{K_L02951-4v}\edtext{reisen miteinander}{\lemma{\textnormal{\emph{reisen miteinander}}}\Cendnote{\textnormal{Sie hatten eine gemeinsame Reise nach Italien\oindex{Italien@\textbf{Italien}, \emph{A.PCLI}|pwk} abgemacht; dazu kam es nicht.}}}\label{K_L02951-4}? Haben {\pb}Sie etwas gearbeitet? Waren Sie in Sti{\geminationm}ung? Ja richtig, Ihr \label{K_L02951-5v}\edtext{Stück\pwindex{?? [Drama ueber Offizier, der Partnerin eines Untergebenen verfuehrt]@\emph{?? [Drama über Offizier, der Partnerin eines Untergebenen verführt]}|pwv}}{\lemma{\textnormal{\emph{Stück}}}\Cendnote{\textnormal{nicht ermittelt}}}\label{K_L02951-5} hat sich neulich
               irgendwo \label{K_L02951-6v}\edtext{ereignet}{\lemma{\textnormal{\emph{ereignet}}}\Cendnote{\textnormal{nicht rekonstruierbar}}}\label{K_L02951-6} – ein
               Offizier, der die Geliebte ſeines Untergebnen verführte – die nähern Umſtände hab ich
               vergeſſen – auch {\pb}in welcher Zeitung ichs
               las, obwohl ich mir die Sache genau notiren wollte\textcolor{gray}{.}– \pend
           
\pstart
           Also geben Sie mir bald\textcolor{gray}{,} dh gleich Nachrichten über Ihr
               Befinden. {\\[\baselineskip]}Herzlich Ihr {\\[\baselineskip]}\spacefill\mbox{ArthSch}\pend
           \leftskip=0em{}\selectlanguage{ngerman}\endnumbering\briefempfaengerindex{Salten, Felix@\textsc{Salten, Felix}!zzzSchnitzler, Arthur@\emph{von Arthur Schnitzler}!1891-09-102@{{[}10.? 9. 1891{]}}|)be}\mylabel{L02951h}  \normalsize

\doendnotes{C}
\bigskip
\vfill

\clearpage

\footnotesize

\lohead{\textsc{register}}

% Definiere theindex-Environment komplett neu ohne reledmac
\makeatletter
\renewenvironment{theindex}{%
  \section*{\indexname}%
  \setlength{\parindent}{0pt}%
  \setlength{\parskip}{0pt plus 0.3pt}%
  \let\item\@idxitem
}{%
  \clearpage
}
\makeatother

\IfFileExists{\jobname-pw.ind}{\input{\jobname-pw.ind}}{}

\end{document}

      