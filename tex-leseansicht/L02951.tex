%% latex-leseansicht-vorspann.tex
%% Vorspann für die Leseansicht.
%% Lädt die gemeinsame Datei latex-vorspann.tex mit nicht gesetztem Schalter.

\newif\ifkorrekturansicht
\korrekturansichtfalse

\input{../tex-inputs/latex-vorspann}


\section[Arthur Schnitzler an Felix Salten, {{[}}10.? 9. 1891{{]}}]{L02951 Arthur Schnitzler an Felix Salten, {[}10.? 9. 1891{]}}
\nopagebreak\mylabel{L02951v}
\rehead{ }\normalsize\beginnumbering\briefempfaengerindex{Salten, Felix@\textsc{Salten, Felix}!zzzSchnitzler, Arthur@\emph{von Arthur Schnitzler}!1891-09-102@{{[}10.? 9. 1891{]}}|(be}
\toendnotes[C]{\smallbreak\pagebreak[2]}
\correspDesc{Versand  durch Arthur Schnitzler am [10.? 9. 1891] in Wien
\newline{}Erhalt  durch Felix Salten im Zeitraum [11. 9. 1891–12.? 9. 1891] in Miskolc}\toendnotes[C]{\smallbreak}
\Standort{Wienbibliothek im Rathaus, ZPH 1681, 2.1.516.}
\physDesc{Brief, 2 Blätter, 8 Seiten, 1636 Zeichen
\newline{}Handschrift: Bleistift, deutsche Kurrent
\newline{}Ordnung: mit Bleistift von unbekannter Hand Nummerierung der Doppelseiten:
                                    »25«–»28« }\toendnotes[C]{\smallbreak}
\pstart
           \raggedleft{}{\pb}Do{\geminationn}erſtg{ }Abend.\pend
           \vspace{0.5em}
\pstart
           Lieber Freund, ko{\geminationm} nach
               Hauſe,{ }ſpät Abends, finde Ihren \label{K_L02951-1v}\edtext{Brief}{\lemma{\textnormal{\emph{Brief}}}\Cendnote{\textnormal{XXXX Auszeichnungsfehler: Dokument L03104 nicht gefunden.
               }}}\label{K_L02951-1}. Wie Sie in dieſem Augenblick jedenfalls{ }ſchon wissen, hab ich Ihnen bereits
               2mal geſchrieben. Der erſte {\pb}Brief, den ich
               einfach an \textsc{F. S.} aus Wien\oindex{Wien@\textbf{Wien}, \emph{Verwaltungsgebiet}|pw} in \textsc{Miskolez\oindex{Miskolc@\textbf{Miskolc}|pw}} adreſſirte, iſt offenbar nicht angeko{\geminationm}en, den
               zweiten mit der Hoteladreſſe, die ich im \textsc{Café Kugel\oindex{Wien@\textbf{Wien}!I., Innere Stadt@\textbf{I., Innere Stadt}!Café Kugel@\textbf{Café Kugel}, \emph{Kaffeehaus}|pw}} erfuhr und den ich heute{ }Vormittag abſandte\textcolor{gray}{,} haben Sie wohl{ }ſchon. Ihre
               Aufregung ist vollko{\geminationm}en {\pb}überflüſſig – ich habe nichts erfahren,
               nichts, nichts, und was ich geſehn habe, iſt, wie mein letzter Brief Ihnen wohl klar
               macht, harmlos genug. Und warum haben Sie denn plötzlich einen Rückfall\pwindex{Karlsburg, Bertha @\textsc{Karlsburg, Bertha}, \emph{Schauspielerin}|pwuv}? Beko{\geminationm}en­ Sie nicht regelmäßig Nachricht? {\pb}Sind die Briefe nicht{ }ſo wie Sie{ }ſie
               wünſchen? – Bitte, reclamiren Sie meinen ersten Brief bei der Poſt. Von mir{ }ſelbſt
               iſt nichts neues zu melden. Und fern am Horizont – Sie wissen schon, da leuchtet {\pb}\label{K_L02951-2v}\edtext{ſie\pwindex{Glümer, Marie 3.\,7.\,1867 Wien – 16.\,11.\,1925 München@\textsc{Glümer, Marie} (3.\,7.\,1867 Wien – 16.\,11.\,1925 München), \emph{Schauspielerin}|pwv}}{\lemma{\textnormal{\emph{sie}}}\Cendnote{\textnormal{Marie Glümer\pwindex{Glümer, Marie 3.\,7.\,1867 Wien – 16.\,11.\,1925 München@\textsc{Glümer, Marie} (3.\,7.\,1867 Wien – 16.\,11.\,1925 München), \emph{Schauspielerin}|pwk}, mit der Schnitzler eine Liebesbeziehung führte}}}\label{K_L02951-2} manchmal
                  auf{\dotstwo} – Zuweilen waren es wohl auch Blitze. Aber es ist
               wunderſchön, wie{ }ſie »an meinen Schmerz heranzureichen«{ }ſucht, und die alte süße
               Lüge, daß es ja diesmal etwas andres, ach etwas ganz andres ist, beko{\geminationm}t {\pb}einen
               betäubenden Duft nach Wahrheit. – Schreiben Sie mir gleich wieder, wie es Ihnen geht,
               wie Sie Ihre Zeit verbringen. Wa{\geminationn} ko{\geminationm}en Sie \label{K_L02951-3v}\edtext{zurück}{\lemma{\textnormal{\emph{zurück}}}\Cendnote{\textnormal{Nachweislich sahen sich die
                  beiden erst am XXXX Auszeichnungsfehler: Dokument L03106 nicht gefunden
                  wieder.}}}\label{K_L02951-3}? Je eher, je lieber. Nicht wahr, wir \label{K_L02951-4v}\edtext{reisen miteinander}{\lemma{\textnormal{\emph{reisen miteinander}}}\Cendnote{\textnormal{Sie hatten eine gemeinsame Reise nach Italien\oindex{Italien@\textbf{Italien}|pwk} abgemacht; dazu kam es nicht.}}}\label{K_L02951-4}? Haben {\pb}Sie etwas gearbeitet? Waren Sie in Sti{\geminationm}ung? Ja richtig, Ihr \label{K_L02951-5v}\edtext{Stück\pwindex{Salten, Felix 6.\,9.\,1869 Budapest – 8.\,10.\,1945 Zürich@\textsc{Salten, Felix} (6.\,9.\,1869 Budapest – 8.\,10.\,1945 Zürich), \emph{Schriftsteller, Journalist, Chefredakteur}!?? [Drama über Offizier, der Partnerin eines Untergebenen verführt]@\strich\emph{?? [Drama über Offizier, der Partnerin eines Untergebenen verführt]}|pwv}}{\lemma{\textnormal{\emph{Stück}}}\Cendnote{\textnormal{nicht ermittelt}}}\label{K_L02951-5} hat sich neulich
               irgendwo \label{K_L02951-6v}\edtext{ereignet}{\lemma{\textnormal{\emph{ereignet}}}\Cendnote{\textnormal{nicht rekonstruierbar}}}\label{K_L02951-6} – ein
               Offizier, der die Geliebte{ }ſeines Untergebnen verführte – die nähern Umſtände hab ich
               vergeſſen – auch {\pb}in welcher Zeitung ichs
               las, obwohl ich mir die Sache genau notiren wollte\textcolor{gray}{.}–\pend
           
\pstart
           Also geben Sie mir bald\textcolor{gray}{,} dh gleich Nachrichten über Ihr
               Befinden. {\\[\baselineskip]}Herzlich Ihr {\\[\baselineskip]}\spacefill\mbox{ArthSch}\pend
           \leftskip=0em{}\selectlanguage{ngerman}\endnumbering\briefempfaengerindex{Salten, Felix@\textsc{Salten, Felix}!zzzSchnitzler, Arthur@\emph{von Arthur Schnitzler}!1891-09-102@{{[}10.? 9. 1891{]}}|)be}\mylabel{L02951h}  \newcommand{\dateiname}{L02951}\newcommand{\titel}{Arthur Schnitzler an Felix Salten, [10.? 9. 1891]}\newcommand{\editorInnen}{Martin Anton Müller und Laura Untner}%% latex-leseansicht-abspann.tex
%% Abspann für die Leseansicht.
%% Der Schalter \ifkorrekturansicht ist bereits durch den Vorspann gesetzt.

%% latex-abspann.tex
%% Gemeinsamer Abspann für Korrekturansicht und Leseansicht.
%% Setzt den Schalter \ifkorrekturansicht voraus (gesetzt in den
%% einbindenden Dateien latex-korrekturansicht-abspann.tex bzw.
%% latex-leseansicht-abspann.tex).
%% ---------------------------------------------------------------

\normalsize

% Das esempio-Environment wird nur in der Leseansicht benötigt
\ifkorrekturansicht\else
\newenvironment{esempio}[3]%
{
    \vspace{1.5ex}
    \rlap{\underline{#1}}
    \par
    \setlength{\parindent}{0cm}
    \nopagebreak
    \leftskip=#2cm
    \rightskip=#3cm
}
{
    \par
}
\fi

\doendnotes{C}
\bigskip
\vfill

\clearpage

\footnotesize

\ifkorrekturansicht
  \lohead{\textsc{register}}
\fi

% theindex-Environment neu definieren ohne reledmac
\makeatletter
\renewenvironment{theindex}{%
  \ifkorrekturansicht
    \section*{\indexname}%
  \else
    \subsubsection*{Index der erwähnten Entitäten}%
  \fi
  \setlength{\parindent}{0pt}%
  \setlength{\parskip}{0pt plus 0.3pt}%
  \let\item\@idxitem
}{%
  \ifkorrekturansicht\clearpage\fi
}
\makeatother

\IfFileExists{\jobname-pw.ind}{\input{\jobname-pw.ind}}{}

% Quellenangabe nur in der Leseansicht
\ifkorrekturansicht\else
% Fallback-Definitionen, falls die .tex-Datei \titel etc. nicht gesetzt hat
\providecommand{\titel}{}
\providecommand{\editorInnen}{}
\providecommand{\dateiname}{\jobname}

\vspace{3cm}

\vfill

\footnotesize
\textsc{Quelle}: \titel. Herausgegeben von {\editorInnen}. In: \emph{Arthur Schnitzler: Briefwechsel mit Autorinnen und Autoren}.
 Digitale Edition, https://schnitzler-briefe.acdh.oeaw.ac.at/{\dateiname}.html (Stand \today)
\fi

\end{document}


