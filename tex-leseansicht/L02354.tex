%% latex-leseansicht-vorspann.tex
%% Vorspann für die Leseansicht.
%% Lädt die gemeinsame Datei latex-vorspann.tex mit nicht gesetztem Schalter.

\newif\ifkorrekturansicht
\korrekturansichtfalse

\input{../tex-inputs/latex-vorspann}


\section[Georg Brandes an Arthur Schnitzler, 17. 8. 1920]{L02354 Georg Brandes an Arthur Schnitzler, 17. 8. 1920}
\nopagebreak\mylabel{L02354v}
\rehead{ }\normalsize\beginnumbering\briefempfaengerindex{Schnitzler, Arthur@\textsc{Schnitzler, Arthur}!zzzBrandes, Georg@\emph{von Georg Brandes}!1920-08-171@{17. 8. 1920}|(be}
\toendnotes[C]{\smallbreak\pagebreak[2]}
\correspDesc{Versand  durch Georg Brandes am 17. 8. 1920 in Kopenhagen
\newline{}Erhalt  durch Arthur Schnitzler im Zeitraum [18. 8. 1920
                  – 22. 8. 1920?] in Wien}\toendnotes[C]{\smallbreak}
\Standort{CUL, Schnitzler, B 17.}
\physDesc{Postkarte, 838 Zeichen
\newline{}Handschrift: schwarze Tinte, lateinische Kurrent
\newline{}Versand: Stempel: »\nobreak{}\oindex{Kopenhagen@\textbf{Kopenhagen}, \emph{Hauptstadt}|pwk}Kjøbenhavn, 17. 8. 20, 6–7 E\nobreak{}«.  
\newline{}Schnitzler: mit Bleistift beschriftet: »\textsc{Brandes}« 
\newline{}Ordnung: mit Bleistift von unbekannter Hand nummeriert:
                                    »51« }
\buchAbdrucke{\weitereDrucke{Georg Brandes, Arthur Schnitzler: \emph{Ein Briefwechsel}. Herausgegeben von Kurt Bergel. Bern: \emph{Francke} 1956, S. 130–131.} }\toendnotes[C]{\smallbreak}\pstart{}{\pb}Herrn Dr. Arthur
                  Schnitzler\pend{}\pstart{}Sternwartestraße 71\oindex{Wien@\textbf{Wien}!XVIII., Währing@\textbf{XVIII., Währing}!Sternwartestraße 71@\textbf{Sternwartestraße 71}, \emph{Wohngebäude}|pw}\pend{}\pstart{}Wien \textsubscript{XVIII}\oindex{XVIII., Währing@\textbf{XVIII., Währing}, \emph{Verwaltungsgebiet}|pw}\pend{}{\bigskip}\vspace{1em}
\pstart
           \raggedleft{}{\pb}Kopenhagen\oindex{Kopenhagen@\textbf{Kopenhagen}, \emph{Hauptstadt}|pw}{ }17 August 20\pend
           
\pstart{}Verehrtester Freund\pend\vspace{0.5em}
\pstart
           Am 13 Juni schrieb ich Ihnen nach langem Schweigen einen sehr langen und
               ausführlichen Brief in der Hoffnung ein wenig über Sie, die Ihrigen und gemeinsame
               Freunde zu hören.\pend
           
\pstart
           Ich erhielt nie eine Zeile Antwort, und da es immerhin möglich ist, dass mein Brief
               Sie nicht erreicht hat, erlaube ich mir die Anfrage, ob Sie ihn erhalten haben, ob
               Sie zum Antworten – was ich höchst natürlich finde – nicht aufgelegt waren. Ein
               Vorwurf würde Sie wahrlich nicht treffen. Aber in früherer Zeit antworteten Sie
               willig, obwol {\pb}die Correspondenz
               uns Allen ein \label{K_L02354-1v}\edtext{corvée}{\lemma{\textnormal{\emph{corvée}}}\Cendnote{\textnormal{französisch: Mühsal}}}\label{K_L02354-1} geworden
               ist.\pend
           
\pstart
           Die Verhältnisse sind ja in Wien\oindex{Wien@\textbf{Wien}, \emph{Verwaltungsgebiet}|pw} besonders
               schwierig und traurig. Ich denke mir, dass Sie überhaupt nicht den Sommer in Wien\oindex{Wien@\textbf{Wien}, \emph{Verwaltungsgebiet}|pw} verbringen.\pend
           \pstart Ihr in alter Freundschaft ergebener \spacefill\mbox{Georg Brandes}\pend{}\selectlanguage{ngerman}\endnumbering\briefempfaengerindex{Schnitzler, Arthur@\textsc{Schnitzler, Arthur}!zzzBrandes, Georg@\emph{von Georg Brandes}!1920-08-171@{17. 8. 1920}|)be}\mylabel{L02354h}  \newcommand{\dateiname}{L02354}\newcommand{\titel}{Georg Brandes an Arthur Schnitzler, 17. 8. 1920}\newcommand{\editorInnen}{Martin Anton Müller und Gerd-Hermann Susen}%% latex-leseansicht-abspann.tex
%% Abspann für die Leseansicht.
%% Der Schalter \ifkorrekturansicht ist bereits durch den Vorspann gesetzt.

%% latex-abspann.tex
%% Gemeinsamer Abspann für Korrekturansicht und Leseansicht.
%% Setzt den Schalter \ifkorrekturansicht voraus (gesetzt in den
%% einbindenden Dateien latex-korrekturansicht-abspann.tex bzw.
%% latex-leseansicht-abspann.tex).
%% ---------------------------------------------------------------

\normalsize

% Das esempio-Environment wird nur in der Leseansicht benötigt
\ifkorrekturansicht\else
\newenvironment{esempio}[3]%
{
    \vspace{1.5ex}
    \rlap{\underline{#1}}
    \par
    \setlength{\parindent}{0cm}
    \nopagebreak
    \leftskip=#2cm
    \rightskip=#3cm
}
{
    \par
}
\fi

\doendnotes{C}
\bigskip
\vfill

\clearpage

\footnotesize

\ifkorrekturansicht
  \lohead{\textsc{register}}
\fi

% theindex-Environment neu definieren ohne reledmac
\makeatletter
\renewenvironment{theindex}{%
  \ifkorrekturansicht
    \section*{\indexname}%
  \else
    \subsubsection*{Index der erwähnten Entitäten}%
  \fi
  \setlength{\parindent}{0pt}%
  \setlength{\parskip}{0pt plus 0.3pt}%
  \let\item\@idxitem
}{%
  \ifkorrekturansicht\clearpage\fi
}
\makeatother

\IfFileExists{\jobname-pw.ind}{\input{\jobname-pw.ind}}{}

% Quellenangabe nur in der Leseansicht
\ifkorrekturansicht\else
% Fallback-Definitionen, falls die .tex-Datei \titel etc. nicht gesetzt hat
\providecommand{\titel}{}
\providecommand{\editorInnen}{}
\providecommand{\dateiname}{\jobname}

\vspace{3cm}

\vfill

\footnotesize
\textsc{Quelle}: \titel. Herausgegeben von {\editorInnen}. In: \emph{Arthur Schnitzler: Briefwechsel mit Autorinnen und Autoren}.
 Digitale Edition, https://schnitzler-briefe.acdh.oeaw.ac.at/{\dateiname}.html (Stand \today)
\fi

\end{document}


