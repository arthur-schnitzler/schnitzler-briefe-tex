%% latex-korrekturansicht-vorspann.tex
%% Vorspann für die Korrekturansicht.
%% Lädt die gemeinsame Datei latex-vorspann.tex mit gesetztem Schalter.

\newif\ifkorrekturansicht
\korrekturansichttrue

\input{../tex-inputs/latex-vorspann}


\section[Arthur Schnitzler: Widmungsexemplar Anatol für Hugo von Hofmannsthal, 4. 12. 1900]{L01084 Arthur Schnitzler: Widmungsexemplar Anatol für Hugo von Hofmannsthal,
               4. 12. 1900}
\nopagebreak\mylabel{L01084v}
\rehead{ }\normalsize\beginnumbering\briefempfaengerindex{Hofmannsthal, Hugo von@\textsc{Hofmannsthal, Hugo von}!zzzSchnitzler, Arthur@\emph{von Arthur Schnitzler}!1900-12-041@{4. 12. 1900}|(be}
\toendnotes[C]{\smallbreak\pagebreak[2]}\Standort{FDH, FDH 258.}
\physDesc{Widmung am Schmutztitel, 48 Zeichen
\newline{}Handschrift: schwarze Tinte, deutsche Kurrent
\newline{}Ordnung: mit Bleistift von unbekannter Hand beschriftet: »HvH-S.
                                    CI, 48« }
\buchAbdrucke{\weitereDrucke{Hugo von Hofmannsthal: \emph{Bibliothek}. Frankfurt am Main: \emph{S. Fischer} 2011, S. 603.} }
\pstart
           \noindent{}{\pb}Meinem lieben Hugo\pend
           \pstart \spacefill\mbox{Arthur}\pend{}
\pstart
           Wien\oindex{Wien@\textbf{Wien}, \emph{A.ADM2}|pw}{ }4. Dezember 900.\pend
           {\vspace{1\baselineskip}}
\pstart
           \centering{}\textcolor{gray}{\textbf{ANATOL\pwindex{Anatol@\emph{Anatol}|pw}}}\pend
           \selectlanguage{ngerman}\vspace{1em}{\vspace{1\baselineskip}}
\pstart
           \centering{}{\pb}\textcolor{gray}{\textbf{ARTHUR SCHNITZLER}}\pend
           
\pstart
           \centering{}\textcolor{gray}{\textbf{ANATOL\pwindex{Anatol. Illustriert@\emph{Anatol. Illustriert}|pw}}}\pend
           
\pstart
           \centering{}\textcolor{gray}{\textbf{illustriert von}}\pend
           
\pstart
           \centering{}\textcolor{gray}{\textbf{M. COSCHELL.\pwindex{Coschell, Moritz 1872-09-18 – 1943-07-11@\textsc{Coschell, Moritz} (1872-09-18 – 1943-07-11), \emph{Maler/Malerin}|pw}}}\pend
           {\vspace{1\baselineskip}}
\pstart
           \centering{}\textcolor{gray}{\textbf{Erste bis dritte Auflage}}\pend
           
\pstart
           \centering{}\textcolor{gray}{\textbf{Berlin\oindex{Berlin@\textbf{Berlin}, \emph{P.PPLC}|pw}}}\pend
           
\pstart
           \centering{}\textcolor{gray}{\textbf{\so{S. Fischer, Verlag}\orgindex{S. Fischer Verlag@S. Fischer Verlag|pw}}}\pend
           
\pstart
           \centering{}\textcolor{gray}{\textbf{1901.}}\pend
           \selectlanguage{ngerman}\endnumbering\briefempfaengerindex{Hofmannsthal, Hugo von@\textsc{Hofmannsthal, Hugo von}!zzzSchnitzler, Arthur@\emph{von Arthur Schnitzler}!1900-12-041@{4. 12. 1900}|)be}\mylabel{L01084h}  \normalsize

\doendnotes{C}
\bigskip
\vfill

\clearpage

\footnotesize

\lohead{\textsc{register}}

% Definiere theindex-Environment komplett neu ohne reledmac
\makeatletter
\renewenvironment{theindex}{%
  \section*{\indexname}%
  \setlength{\parindent}{0pt}%
  \setlength{\parskip}{0pt plus 0.3pt}%
  \let\item\@idxitem
}{%
  \clearpage
}
\makeatother

\IfFileExists{\jobname-pw.ind}{\input{\jobname-pw.ind}}{}

\end{document}

      