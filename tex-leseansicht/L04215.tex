%% latex-leseansicht-vorspann.tex
%% Vorspann für die Leseansicht.
%% Lädt die gemeinsame Datei latex-vorspann.tex mit nicht gesetztem Schalter.

\newif\ifkorrekturansicht
\korrekturansichtfalse

\input{../tex-inputs/latex-vorspann}


\section[Arthur Schnitzler und Olga Gussmann an Gustav Schwarzkopf, 22. 7. 1901]{L04215 Arthur Schnitzler und Olga Gussmann an Gustav Schwarzkopf, 22. 7. 1901}
\nopagebreak\mylabel{L04215v}
\rehead{ }\normalsize\beginnumbering\briefempfaengerindex{Schwarzkopf, Gustav@\textsc{Schwarzkopf, Gustav}!zzzSchnitzler, Arthur@\emph{von Arthur Schnitzler}!1901-07-223@{22. 7. 1901}|(be}
\toendnotes[C]{\smallbreak\pagebreak[2]}
\correspDesc{Versand  durch Arthur Schnitzler am 22. 7. 1901 in Vahrn
\newline{}Erhalt  durch Gustav Schwarzkopf im Zeitraum [23. 7. 1901 – 27. 7. 1901?] in Wien}\toendnotes[C]{\smallbreak}
\Standort{CUL, Schnitzler, B 96.}
\physDesc{Brief, 1 Blatt, 4 Seiten, 1867 Zeichen
\newline{}Handschrift Arthur Schnitzler: schwarze Tinte, deutsche Kurrent
\newline{}Handschrift Olga Schnitzler: schwarze Tinte, lateinische Kurrent}\toendnotes[C]{\smallbreak}
\pstart
           \noindent{}{\pb}lieber Guſtav, Sie waren ſehr freundlich,{ }ſo raſch die etwas
               unbeſcheidene Bitte zu \strikeout{über-} erfüllen, die wir an Sie
               gerichtet haben, nehmen Sie meinen herzlichſten Dank. Wahrſcheinlich werden die \label{K_L04215-1v}\edtext{Mädchen\pwindex{Steinrück, Elisabeth 19.\,11.\,1885 Wien – 7.\,4.\,1920 Partenkirchen@\textsc{Steinrück, Elisabeth} (19.\,11.\,1885 Wien – 7.\,4.\,1920 Partenkirchen)|pwv} vorläufig die Wohnung\oindex{Wien@\textbf{Wien}!IX., Alsergrund@\textbf{IX., Alsergrund}!Pension Powolny@\textbf{Pension Powolny}, \emph{Beherbergungsgebäude}|pwuv}}{\lemma{\textnormal{\emph{Mädchen … Wohnung}}}\Cendnote{\textnormal{Nach der Rückkehr
                        von ihrer Reise, Ende August 1901, übersiedelten Olga\pwindex{Schnitzler, Olga 17.\,1.\,1882 Wien – 13.\,1.\,1970 Lugano@\textsc{Schnitzler, Olga} (17.\,1.\,1882 Wien – 13.\,1.\,1970 Lugano), \emph{Schauspielerin, Sängerin}|pwk} und ihre Schwester Elisabeth Gussmann\pwindex{Steinrück, Elisabeth 19.\,11.\,1885 Wien – 7.\,4.\,1920 Partenkirchen@\textsc{Steinrück, Elisabeth} (19.\,11.\,1885 Wien – 7.\,4.\,1920 Partenkirchen)|pwk}
                        für drei Monate in die Pension Powolny\oindex{Wien@\textbf{Wien}!IX., Alsergrund@\textbf{IX., Alsergrund}!Pension Powolny@\textbf{Pension Powolny}, \emph{Beherbergungsgebäude}|pwk} im 9. Wiener Gemeindebezirk\oindex{IX., Alsergrund@\textbf{IX., Alsergrund}, \emph{Verwaltungsgebiet}|pwk}. 
                     }}}\label{K_L04215-1} nehmen. Über verſchiedene häuſliche
               Widerwärtigkeiten erfahren Sie nächſtens wahrscheinlich mehreres, umſomehr da es
               unausweichlich{ }ſein dürfte, \label{K_L04215-2v}\edtext{juridiſchen Rath}{\lemma{\textnormal{\emph{juridischen Rath}}}\Cendnote{\textnormal{Offiziell standen Olga\pwindex{Schnitzler, Olga 17.\,1.\,1882 Wien – 13.\,1.\,1970 Lugano@\textsc{Schnitzler, Olga} (17.\,1.\,1882 Wien – 13.\,1.\,1970 Lugano), \emph{Schauspielerin, Sängerin}|pwk} und Elisabeth Gussmann\pwindex{Steinrück, Elisabeth 19.\,11.\,1885 Wien – 7.\,4.\,1920 Partenkirchen@\textsc{Steinrück, Elisabeth} (19.\,11.\,1885 Wien – 7.\,4.\,1920 Partenkirchen)|pwk} unter
                  Vormundschaft ihres Vaters\pwindex{Gussmann, Rudolf 5.\,3.\,1842 Veprovac – 24.\,1.\,1921 Wien@\textsc{Gussmann, Rudolf} (5.\,3.\,1842 Veprovac – 24.\,1.\,1921 Wien), \emph{Handelsagent}|pwkv}.}}}\label{K_L04215-2} einzuholen und ich dabei lebhaft an
               Ihren Bruder\pwindex{Schwarzkopf, Max 12.\,6.\,1857 Wien – 14.\,4.\,1928 ebd.@\textsc{Schwarzkopf, Max} (12.\,6.\,1857 Wien – 14.\,4.\,1928 ebd.), \emph{Rechtsanwalt}|pwv} denke. Seien
               Sie nicht bös, we{\geminationn} ich bei dieser Gelegenheit{ }ſchon
               heute die dringende Bitte an Sie und ihn richte, in ſolchen Fällen nach jeder {\pb}Richtung davon abzuſehen, daſs Sie
               und ich {\dots} wie ſoll man das nur Ihnen gegenüber ausdrücken
               – – ziemlich gute Beka{\geminationn}te sind –\pend
           
\pstart
           Wir befinden uns indeſs alle hier recht wohl. Der Ort\oindex{Vahrn@\textbf{Vahrn}, \emph{Hauptstadt}|pwv} ist hübſch, Wald hinterm Hotel\oindex{Gasthof und Pension Waldsacker@\textbf{Gasthof und Pension Waldsacker}, \emph{Beherbergungsgebäude}|pwv}, das Eſſen gut und reichlich; von Hitze keine Spur,
               die Abende und Frühe eher kühl; Publicum irrelevant; die Zimmer reizend, alles
               billig. Trotzdem iſt nicht unſres Bleibens, in etwa 10 Tagen geht es ſüdlicher und
               höher, einem drängenden WunſcheGoldmanns\pwindex{Goldmann, Paul 31.\,1.\,1865 Breslau – 25.\,9.\,1935 Wien@\textsc{Goldmann, Paul} (31.\,1.\,1865 Breslau – 25.\,9.\,1935 Wien), \emph{Schriftsteller, Journalist}|pw} zu
               Folge. G.\pwindex{Goldmann, Paul 31.\,1.\,1865 Breslau – 25.\,9.\,1935 Wien@\textsc{Goldmann, Paul} (31.\,1.\,1865 Breslau – 25.\,9.\,1935 Wien), \emph{Schriftsteller, Journalist}|pw} dürfte dieſer Tage am Wörtherſee\oindex{Wörthersee@\textbf{Wörthersee}, \emph{See}|pw} ein{\pb}treffen; we{\geminationn} Sie auch hinko{\geminationm}en,{ }ſollten Sie{ }ſich ihm anſchließen und mit ihm zu uns stoßen. Südtirol\oindex{Südtirol@\textbf{Südtirol}, \emph{Verwaltungsgebiet}|pw} iſt wahrhaft nicht mehr weit vom Wörtherſee\oindex{Wörthersee@\textbf{Wörthersee}, \emph{See}|pw}; überdies werden Sie ja wohl (wie auch G.\pwindex{Goldmann, Paul 31.\,1.\,1865 Breslau – 25.\,9.\,1935 Wien@\textsc{Goldmann, Paul} (31.\,1.\,1865 Breslau – 25.\,9.\,1935 Wien), \emph{Schriftsteller, Journalist}|pw} auf der Südbahn\orgindex{Südbahnstrecke@Südbahnstrecke|pw}) ein Freibillet bekommen können, und man lebt hier überall billig.
               Wohin wir gehn, ſteht noch nicht feſt; ich ſuche auf Radausflügen nach dem idealen
               Ort. \label{K_L04215-3v}\edtext{Heut}{\lemma{\textnormal{\emph{Heut}}}\Cendnote{\textnormal{Vgl. A. S.: \emph{Tagebuch}, 22. 7. 1901.}}}\label{K_L04215-3} fahre ich nach Bozen\oindex{Bozen@\textbf{Bozen}, \emph{Hauptstadt}|pw} u beſuche \label{K_L04215-4v}\edtext{morgen am
                  Karerſee\oindex{Karersee@\textbf{Karersee}, \emph{See}|pw}{ } Bruder\pwindex{Schnitzler, Julius 13.\,7.\,1865 Wien – 29.\,6.\,1939 ebd.@\textsc{Schnitzler, Julius} (13.\,7.\,1865 Wien – 29.\,6.\,1939 ebd.), \emph{Chirurg}|pwv} u Schwägerin\pwindex{Schnitzler, Helene 16.\,7.\,1871 Budapest – September 1941 Atlantischer Ozean@\textsc{Schnitzler, Helene} (16.\,7.\,1871 Budapest – September 1941 Atlantischer Ozean)|pwv}}{\lemma{\textnormal{\emph{morgen … Schwägerin}}}\Cendnote{\textnormal{Vgl. A. S.: \emph{Tagebuch}, 23. 7. 1901. }}}\label{K_L04215-4}, da
               nehm ich gleich das Tierſer Thal\oindex{Tierser Tal@\textbf{Tierser Tal}, \emph{Tal}|pw} mit. {\pb}Ich habe auch einigermaßen gearbeitet
               und hoffe ein zahlloſe Abende füllendes Stück\pwindex{Schnitzler, Arthur 15. 5. 1862 Wien – 21. 10. 1931 ebd.@\textsc{Schnitzler, Arthur} (15. 5. 1862 Wien – 21. 10. 1931 ebd.), \emph{Schriftsteller, Mediziner}!einsame Weg. Schauspiel in fünf Akten@\strich\emph{Der einsame Weg. Schauspiel in fünf Akten}|pwv} mitzubringen. Und nun nochmals vielen Dank! Schreiben
               Sie bald wieder ein Wort!\pend
           
\pstart
           Herzlichſt Ihr{\\[\baselineskip]}\spacefill\mbox{ArtSch}\pend
           \leftskip=0em{}
\pstart
           \textsc{Vahrn\oindex{Vahrn@\textbf{Vahrn}, \emph{Hauptstadt}|pw}}, 22/7  901.\pend
           
\pstart
           \noindent{}Die Kinder\pwindex{Steinrück, Elisabeth 19.\,11.\,1885 Wien – 7.\,4.\,1920 Partenkirchen@\textsc{Steinrück, Elisabeth} (19.\,11.\,1885 Wien – 7.\,4.\,1920 Partenkirchen)|pwv} grüßen u danken
                  vielmals\pend
           \selectlanguage{ngerman}\vspace{1em}
\pstart
           \noindent{}{[}hs. Schnitzler:{]} Ja, das thun sie\pwindex{Steinrück, Elisabeth 19.\,11.\,1885 Wien – 7.\,4.\,1920 Partenkirchen@\textsc{Steinrück, Elisabeth} (19.\,11.\,1885 Wien – 7.\,4.\,1920 Partenkirchen)|pwv} auch wirklich von Herzen und bitten Sie, Arthurs Rat zu
               befolgen und zu uns zu kommen.\pend
           \pstart \spacefill\mbox{OlgaS.}\pend{}\selectlanguage{ngerman}\endnumbering\briefempfaengerindex{Schwarzkopf, Gustav@\textsc{Schwarzkopf, Gustav}!zzzSchnitzler, Arthur@\emph{von Arthur Schnitzler}!1901-07-223@{22. 7. 1901}|)be}\mylabel{L04215h}
\begin{anhang}
\end{anhang}\newcommand{\dateiname}{L04215}\newcommand{\titel}{Arthur Schnitzler und Olga Gussmann an Gustav Schwarzkopf, 22. 7. 1901}\newcommand{\editorInnen}{Herausgegeben von Jahnke, SelmaMüller, Martin Anton}%% latex-leseansicht-abspann.tex
%% Abspann für die Leseansicht.
%% Der Schalter \ifkorrekturansicht ist bereits durch den Vorspann gesetzt.

%% latex-abspann.tex
%% Gemeinsamer Abspann für Korrekturansicht und Leseansicht.
%% Setzt den Schalter \ifkorrekturansicht voraus (gesetzt in den
%% einbindenden Dateien latex-korrekturansicht-abspann.tex bzw.
%% latex-leseansicht-abspann.tex).
%% ---------------------------------------------------------------

\normalsize

% Das esempio-Environment wird nur in der Leseansicht benötigt
\ifkorrekturansicht\else
\newenvironment{esempio}[3]%
{
    \vspace{1.5ex}
    \rlap{\underline{#1}}
    \par
    \setlength{\parindent}{0cm}
    \nopagebreak
    \leftskip=#2cm
    \rightskip=#3cm
}
{
    \par
}
\fi

\doendnotes{C}
\bigskip
\vfill

\clearpage

\footnotesize

\ifkorrekturansicht
  \lohead{\textsc{register}}
\fi

% theindex-Environment neu definieren ohne reledmac
\makeatletter
\renewenvironment{theindex}{%
  \ifkorrekturansicht
    \section*{\indexname}%
  \else
    \subsubsection*{Index der erwähnten Entitäten}%
  \fi
  \setlength{\parindent}{0pt}%
  \setlength{\parskip}{0pt plus 0.3pt}%
  \let\item\@idxitem
}{%
  \ifkorrekturansicht\clearpage\fi
}
\makeatother

\IfFileExists{\jobname-pw.ind}{\input{\jobname-pw.ind}}{}

% Quellenangabe nur in der Leseansicht
\ifkorrekturansicht\else
% Fallback-Definitionen, falls die .tex-Datei \titel etc. nicht gesetzt hat
\providecommand{\titel}{}
\providecommand{\editorInnen}{}
\providecommand{\dateiname}{\jobname}

\vspace{3cm}

\vfill

\footnotesize
\textsc{Quelle}: \titel. Herausgegeben von {\editorInnen}. In: \emph{Arthur Schnitzler: Briefwechsel mit Autorinnen und Autoren}.
 Digitale Edition, https://schnitzler-briefe.acdh.oeaw.ac.at/{\dateiname}.html (Stand \today)
\fi

\end{document}


