%% latex-korrekturansicht-vorspann.tex
%% Vorspann für die Korrekturansicht.
%% Lädt die gemeinsame Datei latex-vorspann.tex mit gesetztem Schalter.

\newif\ifkorrekturansicht
\korrekturansichttrue

\input{../tex-inputs/latex-vorspann}


\section[Stefan Zweig an Arthur Schnitzler, 27. 11. 1910]{L03628 Stefan Zweig an Arthur Schnitzler, 27. 11. 1910}
\nopagebreak\mylabel{L03628v}
\rehead{ }\normalsize\beginnumbering\briefempfaengerindex{Schnitzler, Arthur@\textsc{Schnitzler, Arthur}!zzzZweig, Stefan@\emph{von Stefan Zweig}!1910-11-271@{27. 11. 1910}|(be}
\toendnotes[C]{\smallbreak\pagebreak[2]}\Standort{CUL, Schnitzler, B 118}
\physDesc{1 Blatt, 1 Seite, 403 Zeichen
\newline{}Handschrift: lila Tinte, lateinische Kurrent}
\buchAbdrucke{\weitereDrucke{Stefan Zweig: \emph{Briefwechsel mit Hermann Bahr, Sigmund Freud, Rainer Maria
                        Rilke und Arthur Schnitzler}. Frankfurt am Main: \emph{S. Fischer} 1987, S. 361.} }\toendnotes[C]{\smallbreak}
\pstart
           {\pb}\textcolor{gray}{\textbf{SZ}}\hfill \textcolor{gray}{\textbf{VIII. KOCHGASSE 8\oindex{Kochgasse 8@\textbf{Kochgasse 8}|pw}}}\pend
           
\pstart
           \raggedleft{}\textcolor{gray}{\textbf{WIEN\oindex{Wien@\textbf{Wien}|pw},}}{ }27. Nov 10\pend
           {\vspace{1\baselineskip}}
\pstart{}Verehrter Herr Doktor,\pend\vspace{0.5em}
\pstart
           von einem Winkel der Galerie herab sah ich \label{K_L03628-1v}\edtext{Medardi\pwindex{junge Medardus. Dramatische Historie in einem Vorspiel und fuenf
                  Aufzuegen@\emph{Der junge Medardus. Dramatische Historie in einem Vorspiel und fünf Aufzügen}|pw} Schicksal\eventindex{Burgtheater@\textbf{Burgtheater}!Urauffuehrung von Der junge Medardus, 24.11.1910@Uraufführung von Der junge Medardus, 24.11.1910|pwv}}{\lemma{\textnormal{\emph{Medardi Schicksal}}}\Cendnote{\textnormal{Schnitzlers dramatische Historie \emph{Der junge Medardus}\pwindex{junge Medardus. Dramatische Historie in einem Vorspiel und fuenf
                  Aufzuegen@\emph{Der junge Medardus. Dramatische Historie in einem Vorspiel und fünf Aufzügen}|pwk} wurde am
                     24. 11. 1910 am Burgtheater\oindex{Burgtheater@\textbf{Burgtheater}|pwk} in
                  seiner Anwesenheit uraufgeführt\eventindex{Burgtheater@\textbf{Burgtheater}!Urauffuehrung von Der junge Medardus, 24.11.1910@Uraufführung von Der junge Medardus, 24.11.1910|pwkv}. Der Vorstellung am 27. 11. 1910 wohnte Schnitzler ebenfalls bei, vgl. A. S.: \emph{Tagebuch}, 27. 11. 1910.}}}\label{K_L03628-1} und war beglückt, immer wieder Ihr
               Antlitz vor dem Jubel erscheinen zu sehn. Ich freue mich, dass nun alle Ihre Dramen,
               eines nach dem andern (und hoffentlich auch bald \label{K_L03628-2v}\edtext{die »Beatrice\pwindex{Schleier der Beatrice. Schauspiel in fuenf Akten@\emph{Der Schleier der Beatrice. Schauspiel in fünf Akten}|pw}«}{\lemma{\textnormal{\emph{die »Beatrice«}}}\Cendnote{\textnormal{Das Versdrama \emph{Der Schleier der Beatrice}\pwindex{Schleier der Beatrice. Schauspiel in fuenf Akten@\emph{Der Schleier der Beatrice. Schauspiel in fünf Akten}|pwk} gehört zu den Stücken Schnitzlers, von dessen Qualität er selbst
                  überzeugt war. Entsprechend schwer traf ihn die magere Bühnenkarriere. Schnitzler hatte es 1899 am Burgtheater\oindex{Burgtheater@\textbf{Burgtheater}|pwk} eingereicht, war dort abgelehnt
                  worden, was zur Protestschreiben und einem Skandal geführt hatte, in deren Folge
                     Schnitzlers Dramen für fünf Jahre nicht
                  am Burgtheater\oindex{Burgtheater@\textbf{Burgtheater}|pwk} aufgeführt worden waren. \emph{Der Schleier der Beatrice}\pwindex{Schleier der Beatrice. Schauspiel in fuenf Akten@\emph{Der Schleier der Beatrice. Schauspiel in fünf Akten}|pwk} wurde am 1. 12. 1900 in Breslau\oindex{Breslau@\textbf{Breslau}|pwk} uraufgeführt und erst 25 Jahre später,
                  am 23. 5. 1925, erstmals
                  am Burgtheater\oindex{Burgtheater@\textbf{Burgtheater}|pwk} inszeniert.}}}\label{K_L03628-2}) sich die
               Bühne erobern und damit uns, die wir schon vom Buch gefangen waren, zum zweitenmal.
               In Treuen \pend
           
\pstart
           Ihr ergebener{\\[\baselineskip]}\spacefill\mbox{Stefan Zweig}\pend
           \leftskip=0em{}\selectlanguage{ngerman}\endnumbering\mylabel{L03628h}  \normalsize

\doendnotes{C}
\bigskip
\vfill

\clearpage

\footnotesize

\lohead{\textsc{register}}

% Definiere theindex-Environment komplett neu ohne reledmac
\makeatletter
\renewenvironment{theindex}{%
  \section*{\indexname}%
  \setlength{\parindent}{0pt}%
  \setlength{\parskip}{0pt plus 0.3pt}%
  \let\item\@idxitem
}{%
  \clearpage
}
\makeatother

\IfFileExists{\jobname-pw.ind}{\input{\jobname-pw.ind}}{}

\end{document}

      