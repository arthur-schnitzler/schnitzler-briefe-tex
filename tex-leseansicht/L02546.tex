%% latex-leseansicht-vorspann.tex
%% Vorspann für die Leseansicht.
%% Lädt die gemeinsame Datei latex-vorspann.tex mit nicht gesetztem Schalter.

\newif\ifkorrekturansicht
\korrekturansichtfalse

\input{../tex-inputs/latex-vorspann}


\section[Arthur Schnitzler an Hermann Bahr, 5. 9. 1931]{L02546 Arthur Schnitzler an Hermann Bahr, 5. 9. 1931}
\nopagebreak\mylabel{L02546v}
\rehead{ }\normalsize\beginnumbering\briefempfaengerindex{Bahr, Hermann@\textsc{Bahr, Hermann}!zzzSchnitzler, Arthur@\emph{von Arthur Schnitzler}!1931-09-051@{5. 9. 1931}|(be}
\toendnotes[C]{\smallbreak\pagebreak[2]}
\correspDesc{Versand  durch Arthur Schnitzler am 5. 9. 1931 in Wien
\newline{}Erhalt  durch Hermann Bahr im Zeitraum [6. 9. 1931
                  – 10. 9. 1931?] in München}\toendnotes[C]{\smallbreak}
\Standort{TMW, HS AM 23400 Ba.}
\physDesc{Brief, 1 Blatt, 1 Seite, 929 Zeichen
\newline{}Schreibmaschine
\newline{}Handschrift: Bleistift, lateinische Kurrent (\noindent{}Unterschrift und Grußformel)
\newline{}Bahr: mit rotem Buntstift ergänzt: »Unmittelbar vor Fahrt nach
                                       Garmisch\oindex{Hotel Riessersee@\textbf{Hotel Riessersee}, \emph{Hotel}|pw}« 
\newline{}Ordnung: mit Bleistift von unbekannter Hand beschriftet:
                                    »erledigt« }\Standort{DLA, A:Schnitzler, 85.1.294/8.}
\physDesc{Brief, Durchschlag, 1 Blatt, 1 Seite, 929 Zeichen
\newline{}Schreibmaschine}
\buchAbdrucke{\weitereDrucke{1) \emph{5. 9. 1931.} In: Arthur Schnitzler: \emph{The Letters of Arthur Schnitzler to Hermann Bahr}. Edited, annotated, and with an introduction, by Donald G. Daviau. Chapel Hill: \emph{The University of North Carolina Press} 1978, S. 118 (University of North Carolina studies in the Germanic languages
                        and literatures, 89).} \weitereDrucke{2) Hermann Bahr, Arthur Schnitzler: \emph{Briefwechsel, Aufzeichnungen, Dokumente (1891–1931)}. Herausgegeben von Kurt Ifkovits und Martin Anton Müller. Göttingen: \emph{Wallstein} 2018, S. 598–599.} }\toendnotes[C]{\smallbreak}
\pstart
           {\pb}\textcolor{gray}{\textbf{D\textsuperscript{r} Arthur Schnitzler}}\hfill 5. 9. 1931.\pend
           
\pstart
           \textcolor{gray}{\textbf{Wien. XVIII. Sternwartestrasse 71\oindex{Wien@\textbf{Wien}!XVIII., Währing@\textbf{XVIII., Währing}!Sternwartestraße 71@\textbf{Sternwartestraße 71}, \emph{Wohngebäude}|pw}.}}\pend
           
\pstart{}Lieber Hermann.\pend\vspace{0.5em}
\pstart
           Ich lese, dass Dein »Konzert\pwindex{Bahr, Hermann 19.\,7.\,1863 Linz – 15.\,1.\,1934 München@\textsc{Bahr, Hermann} (19.\,7.\,1863 Linz – 15.\,1.\,1934 München), \emph{Schriftsteller, Kritiker}!Konzert. Lustspiel in drei Akten@\strich\emph{Das Konzert. Lustspiel in drei Akten}|pw}« jetzt als \label{K_L02546-1v}\edtext{Tonfilm\pwindex{Mittler, Leo 18.\,12.\,1893 Wien – 16.\,5.\,1958 Berlin@\textsc{Mittler, Leo} (18.\,12.\,1893 Wien – 16.\,5.\,1958 Berlin), \emph{Schauspieler, Theater- und Filmregisseur}!Konzert@\strich\emph{Das Konzert}|pwv} erscheint}{\lemma{\textnormal{\emph{Tonfilm erscheint}}}\Cendnote{\textnormal{Die Verfilmung durch Leo Mittler\pwindex{Mittler, Leo 18.\,12.\,1893 Wien – 16.\,5.\,1958 Berlin@\textsc{Mittler, Leo} (18.\,12.\,1893 Wien – 16.\,5.\,1958 Berlin), \emph{Schauspieler, Theater- und Filmregisseur}|pwk} lief bereits seit 28. 8. 1931 in den Wien\oindex{Wien@\textbf{Wien}, \emph{Verwaltungsgebiet}|pwk}er
                  Kinos.}}}\label{K_L02546-1}, nachdem es vorher, so weit ich mich erinnere, auch schon als \label{K_L02546-2v}\edtext{stummer Film}{\lemma{\textnormal{\emph{stummer Film}}}\Cendnote{\textnormal{\emph{The Concert}\pwindex{Schertzinger, Victor 8.\,4.\,1888 Mahanoy City – 1941 Hollywood@\textsc{Schertzinger, Victor} (8.\,4.\,1888 Mahanoy City – 1941 Hollywood)!Concert@\strich\emph{The Concert}|pwk} (1921), Regie Victor Schertzinger\pwindex{Schertzinger, Victor 8.\,4.\,1888 Mahanoy City – 1941 Hollywood@\textsc{Schertzinger, Victor} (8.\,4.\,1888 Mahanoy City – 1941 Hollywood)|pwk},
                     von demselben neuerlich unter dem Titel \emph{Fashions in Love}\pwindex{Schertzinger, Victor 8.\,4.\,1888 Mahanoy City – 1941 Hollywood@\textsc{Schertzinger, Victor} (8.\,4.\,1888 Mahanoy City – 1941 Hollywood)!Fashions in Love@\strich\emph{Fashions in Love}|pwk} (1929) als Tonfilm realisiert.}}}\label{K_L02546-2}\pwindex{Schertzinger, Victor 8.\,4.\,1888 Mahanoy City – 1941 Hollywood@\textsc{Schertzinger, Victor} (8.\,4.\,1888 Mahanoy City – 1941 Hollywood)!Concert@\strich\emph{The Concert}|pwv} zu sehen war. Ich möchte nun gern wissen – falls es Dir nicht unbequem ist mir
               darauf zu antworten – ob, resp. welche Ansprüche die seinerzeitigen Verfertiger des
               stummen Films an Dich gestellt haben. Ich erlebe es in jedem einzelnen Fall, so mit
                  »\label{K_L02546-3v}\edtext{Liebelei\pwindex{Schnitzler, Arthur 15.\,5.\,1862 Wien – 21.\,10.\,1931 ebd.@\textsc{Schnitzler, Arthur} (15.\,5.\,1862 Wien – 21.\,10.\,1931 ebd.), \emph{Schriftsteller, Mediziner}!Liebelei. Schauspiel in drei Akten@\strich\emph{Liebelei. Schauspiel in drei Akten}|pw}}{\lemma{\textnormal{\emph{Liebelei}}}\Cendnote{\textnormal{Erstmals wurde Schnitzlers Stück \emph{Liebelei}\pwindex{Schnitzler, Arthur 15.\,5.\,1862 Wien – 21.\,10.\,1931 ebd.@\textsc{Schnitzler, Arthur} (15.\,5.\,1862 Wien – 21.\,10.\,1931 ebd.), \emph{Schriftsteller, Mediziner}!Liebelei. Schauspiel in drei Akten@\strich\emph{Liebelei. Schauspiel in drei Akten}|pwk}{ }1914 verfilmt (\emph{Elskovsleg}\pwindex{Holger-Madsen 11.\,4.\,1878 Kopenhagen – 30.\,11.\,1943 ebd.@\textsc{Holger-Madsen} (11.\,4.\,1878 Kopenhagen – 30.\,11.\,1943 ebd.), \emph{Regisseur, Schauspieler, Drehbuchautor}!Elskovsleg@\strich\emph{Elskovsleg}|pwk}\pwindex{Blom, August 26.\,12.\,1869 Kopenhagen – 10.\,1.\,1947 ebd.@\textsc{Blom, August} (26.\,12.\,1869 Kopenhagen – 10.\,1.\,1947 ebd.)!Elskovsleg@\strich\emph{Elskovsleg}|pwk}, Regie Holger-Madsen\pwindex{Holger-Madsen 11.\,4.\,1878 Kopenhagen – 30.\,11.\,1943 ebd.@\textsc{Holger-Madsen} (11.\,4.\,1878 Kopenhagen – 30.\,11.\,1943 ebd.), \emph{Regisseur, Schauspieler, Drehbuchautor}|pwk} und August
                     Blom\pwindex{Blom, August 26.\,12.\,1869 Kopenhagen – 10.\,1.\,1947 ebd.@\textsc{Blom, August} (26.\,12.\,1869 Kopenhagen – 10.\,1.\,1947 ebd.)|pwk}). Ab 1921 gab es Verhandlungen über eine Neuverfilmung,
                  vgl. Arthur Schnitzler: \emph{Filmarbeiten. Drehbücher, Entwürfe, Skizzen}. Herausgegeben von  Achim
                     Aurnhammer, Hans Peter Buohler, Philipp Gresser, Julia Ilgner, Carolin Maikler,
                     Lea Marquart. Würzburg: \emph{Ergon}{ }2015, S. 101–103. Eine neuerliche Verfilmung\pwindex{Fleck, Luise 1.\,8.\,1873 Wien – 15.\,3.\,1950 ebd.@\textsc{Fleck, Luise} (1.\,8.\,1873 Wien – 15.\,3.\,1950 ebd.), \emph{Schauspielerin, Filmproduzentin}!Liebelei [Film, 1927]@\strich\emph{Liebelei [Film, 1927]}|pwkv}\pwindex{Fleck, Jacob Julius 18.\,11.\,1881 Wien – 19.\,9.\,1953 ebd.@\textsc{Fleck, Jacob Julius} (18.\,11.\,1881 Wien – 19.\,9.\,1953 ebd.), \emph{Regisseur, Filmproduzent, Drehbuchautor}!Liebelei [Film, 1927]@\strich\emph{Liebelei [Film, 1927]}|pwkv}\pwindex{Fleck, Luise 1.\,8.\,1873 Wien – 15.\,3.\,1950 ebd.@\textsc{Fleck, Luise} (1.\,8.\,1873 Wien – 15.\,3.\,1950 ebd.), \emph{Schauspielerin, Filmproduzentin}!Liebelei [Film, 1927]@\strich\emph{Liebelei [Film, 1927]}|pwkv} kam
                  1927 in die Kinos (Regie Jakob\pwindex{Fleck, Jacob Julius 18.\,11.\,1881 Wien – 19.\,9.\,1953 ebd.@\textsc{Fleck, Jacob Julius} (18.\,11.\,1881 Wien – 19.\,9.\,1953 ebd.), \emph{Regisseur, Filmproduzent, Drehbuchautor}|pwk} und Luise Fleck\pwindex{Fleck, Luise 1.\,8.\,1873 Wien – 15.\,3.\,1950 ebd.@\textsc{Fleck, Luise} (1.\,8.\,1873 Wien – 15.\,3.\,1950 ebd.), \emph{Schauspielerin, Filmproduzentin}|pwk}).}}}\label{K_L02546-3}«, »\label{K_L02546-4v}\edtext{Anatol\pwindex{Schnitzler, Arthur 15.\,5.\,1862 Wien – 21.\,10.\,1931 ebd.@\textsc{Schnitzler, Arthur} (15.\,5.\,1862 Wien – 21.\,10.\,1931 ebd.), \emph{Schriftsteller, Mediziner}!Anatol@\strich\emph{Anatol}|pw}}{\lemma{\textnormal{\emph{Anatol}}}\Cendnote{\textnormal{\emph{The Affairs of Anatol}\pwindex{DeMille, Cecil B. 12.\,8.\,1881 Ashfield – 21.\,1.\,1959 Hollywood@\textsc{DeMille, Cecil B.} (12.\,8.\,1881 Ashfield – 21.\,1.\,1959 Hollywood), \emph{Filmproduzent}!Affairs of Anatol@\strich\emph{The Affairs of Anatol}|pwk} (1921),
                  Regie Cecil B. DeMille\pwindex{DeMille, Cecil B. 12.\,8.\,1881 Ashfield – 21.\,1.\,1959 Hollywood@\textsc{DeMille, Cecil B.} (12.\,8.\,1881 Ashfield – 21.\,1.\,1959 Hollywood), \emph{Filmproduzent}|pwk}. Zu dem Plan einer
                  neuerlichen Verfilmung, die nicht realisiert wurde, gibt es Hinweise in Schnitzlers{ }\emph{Tagebuch}\pwindex{Schnitzler, Arthur 15.\,5.\,1862 Wien – 21.\,10.\,1931 ebd.@\textsc{Schnitzler, Arthur} (15.\,5.\,1862 Wien – 21.\,10.\,1931 ebd.), \emph{Schriftsteller, Mediziner}!Tagebuch@\strich\emph{Tagebuch}|pwk} zwischen 3. 11. 1930 und 4. 5. 1931.}}}\label{K_L02546-4}«, »\label{K_L02546-5v}\edtext{Fräulein Else\pwindex{Schnitzler, Arthur 15.\,5.\,1862 Wien – 21.\,10.\,1931 ebd.@\textsc{Schnitzler, Arthur} (15.\,5.\,1862 Wien – 21.\,10.\,1931 ebd.), \emph{Schriftsteller, Mediziner}!Fräulein Else@\strich\emph{Fräulein Else}|pw}}{\lemma{\textnormal{\emph{Fräulein Else}}}\Cendnote{\textnormal{\emph{Fräulein Else}\pwindex{Czinner, Paul 30.\,5.\,1890 Budapest – 22.\,6.\,1972 London@\textsc{Czinner, Paul} (30.\,5.\,1890 Budapest – 22.\,6.\,1972 London), \emph{Schriftsteller, Filmregisseur}!Fräulein Else@\strich\emph{Fräulein Else}|pwk} wurde 1929 unter der Regie von
                        Paul Czinner\pwindex{Czinner, Paul 30.\,5.\,1890 Budapest – 22.\,6.\,1972 London@\textsc{Czinner, Paul} (30.\,5.\,1890 Budapest – 22.\,6.\,1972 London), \emph{Schriftsteller, Filmregisseur}|pwk} verfilmt.
               }}}\label{K_L02546-5}«, dass sich die seinerzeitigen Verfertiger der stummen Fassung
               freundlich-erpresserisch gebärden, in welcher Haltung die Leute durch allerlei
               Gesetze, Auffassungen, Bestimmungen – auch insoweit sie nicht vorhanden sind – mehr
               oder weniger unterstützt werden.\pend
           
\pstart
           Wolltest Du mir bei dieser Gelegenheit auch sonst ein Wort über Dich und Dein
               Befinden sagen, so wird es mich herzlich freuen.\pend
           
\pstart
           {[}hs.:{]} Mit vielen Grüßen und der Bitte mich deiner verehrten Gattin\pwindex{Bahr-Mildenburg, Anna 29.\,11.\,1872 Wien – 27.\,1.\,1947 ebd.@\textsc{Bahr-Mildenburg, Anna} (29.\,11.\,1872 Wien – 27.\,1.\,1947 ebd.), \emph{Sängerin}|pwv} zu empfehlen{\\[\baselineskip]}Dein{\\[\baselineskip]}\spacefill\mbox{Arth}\pend
           \leftskip=0em{}
\pstart
           \noindent{}{[}ms.:{]}  Herrn Hermann Bahr,{\\}München\oindex{München@\textbf{München}|pw}.\pend
           \selectlanguage{ngerman}\endnumbering\briefempfaengerindex{Bahr, Hermann@\textsc{Bahr, Hermann}!zzzSchnitzler, Arthur@\emph{von Arthur Schnitzler}!1931-09-051@{5. 9. 1931}|)be}\mylabel{L02546h}  \newcommand{\dateiname}{L02546}\newcommand{\titel}{Arthur Schnitzler an Hermann Bahr, 5. 9. 1931}\newcommand{\editorInnen}{Herausgegeben von Martin Anton Müller}%% latex-leseansicht-abspann.tex
%% Abspann für die Leseansicht.
%% Der Schalter \ifkorrekturansicht ist bereits durch den Vorspann gesetzt.

%% latex-abspann.tex
%% Gemeinsamer Abspann für Korrekturansicht und Leseansicht.
%% Setzt den Schalter \ifkorrekturansicht voraus (gesetzt in den
%% einbindenden Dateien latex-korrekturansicht-abspann.tex bzw.
%% latex-leseansicht-abspann.tex).
%% ---------------------------------------------------------------

\normalsize

% Das esempio-Environment wird nur in der Leseansicht benötigt
\ifkorrekturansicht\else
\newenvironment{esempio}[3]%
{
    \vspace{1.5ex}
    \rlap{\underline{#1}}
    \par
    \setlength{\parindent}{0cm}
    \nopagebreak
    \leftskip=#2cm
    \rightskip=#3cm
}
{
    \par
}
\fi

\doendnotes{C}
\bigskip
\vfill

\clearpage

\footnotesize

\ifkorrekturansicht
  \lohead{\textsc{register}}
\fi

% theindex-Environment neu definieren ohne reledmac
\makeatletter
\renewenvironment{theindex}{%
  \ifkorrekturansicht
    \section*{\indexname}%
  \else
    \subsubsection*{Index der erwähnten Entitäten}%
  \fi
  \setlength{\parindent}{0pt}%
  \setlength{\parskip}{0pt plus 0.3pt}%
  \let\item\@idxitem
}{%
  \ifkorrekturansicht\clearpage\fi
}
\makeatother

\IfFileExists{\jobname-pw.ind}{\input{\jobname-pw.ind}}{}

% Quellenangabe nur in der Leseansicht
\ifkorrekturansicht\else
% Fallback-Definitionen, falls die .tex-Datei \titel etc. nicht gesetzt hat
\providecommand{\titel}{}
\providecommand{\editorInnen}{}
\providecommand{\dateiname}{\jobname}

\vspace{3cm}

\vfill

\footnotesize
\textsc{Quelle}: \titel. Herausgegeben von {\editorInnen}. In: \emph{Arthur Schnitzler: Briefwechsel mit Autorinnen und Autoren}.
 Digitale Edition, https://schnitzler-briefe.acdh.oeaw.ac.at/{\dateiname}.html (Stand \today)
\fi

\end{document}


