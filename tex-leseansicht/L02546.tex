%% latex-leseansicht-vorspann.tex
%% Vorspann für die Leseansicht.
%% Lädt die gemeinsame Datei latex-vorspann.tex mit nicht gesetztem Schalter.

\newif\ifkorrekturansicht
\korrekturansichtfalse

\input{../tex-inputs/latex-vorspann}


         
         \newcommand{\erwaehntePersonen}{Personen: Hermann Bahr, Anna Bahr-Mildenburg, August Blom, Paul Czinner, Cecil B. DeMille, Jacob Julius Fleck, Luise Fleck,  Holger-Madsen, Leo Mittler, Victor Schertzinger}
         \newcommand{\erwaehnteOrte}{Orte: Hotel Riessersee, München, Sternwartestraße, Wien}
         \newcommand{\erwaehnteWerke}{Werke: Anatol, Das Konzert, Das Konzert. Lustspiel in drei Akten, Elskovsleg, Fashions in Love, Fräulein Else, Fräulein Else (Film), Liebelei (Film, 1927), Liebelei. Schauspiel in drei Akten, Tagebuch, The Affairs of Anatol, The Concert}
               \section[Arthur Schnitzler an Hermann Bahr, 5. 9. 1931]{ Arthur Schnitzler an Hermann Bahr, 5. 9. 1931}\nopagebreak\mylabel{v}\rehead{ }\begin{ledgroupsized}[t]{13cm}\normalsize\beginnumbering \toendnotes[C]{\smallbreak\pagebreak[2]} \Standort{TMW, HS AM 23400 Ba.}
\physDesc{Brief, 1 Blatt, 1 Seite
\newline{}Schreibmaschine
\newline{}Handschrift: Bleistift, lateinische Kurrent (\noindent{}Unterschrift und Grußformel)
\newline{}Bahr: mit rotem Buntstift ergänzt: »Unmittelbar vor Fahrt nach
                                       Garmisch\oindex{Hotel Riessersee@\textbf{Hotel Riessersee}|pw}« \newline{}Ordnung: mit Bleistift von unbekannter Hand beschriftet:
                                    »erledigt« }\Standort{DLA, A:Schnitzler, 85.1.294/8.}
\physDesc{Brief, 1 Blatt, 1 Seite, maschineller Durchschlag
\newline{}Schreibmaschine}\buchAbdrucke{\weitereDrucke{1) \emph{5. 9. 1931.} In: Arthur Schnitzler: \emph{The Letters of Arthur Schnitzler to Hermann Bahr}. Edited, annotated, and with an introduction, by Donald G.
                        Daviau. Chapel Hill: \emph{The University of North Carolina Press} 1978, S. 118 (University of North Carolina studies in the Germanic languages
                        and literatures, 89).} \weitereDrucke{2) Hermann Bahr, Arthur Schnitzler: \emph{Briefwechsel, Aufzeichnungen, Dokumente (1891–1931)}. Hg. Kurt Ifkovits und Martin Anton Müller. Göttingen: \emph{Wallstein} 2018, S. 598–599.} }\toendnotes[C]{\smallbreak}\pstart
           \noindent{}{\pb}\textcolor{gray}{\textbf{D\textsuperscript{r} Arthur Schnitzler}}\hfill 5. 9. 1931.\pend
           \pstart
           \textcolor{gray}{\textbf{Wien. XVIII. Sternwartestrasse 71\oindex{Sternwartestrasse@\textbf{Sternwartestraße}|pw}.}}\pend
           \pstart{}Lieber Hermann.\pend\pstart
           Ich lese, dass Dein »Konzert\pwindex{Bahr, Hermann 19.07.1863 – 15.01.1934@\textsc{Bahr, Hermann} (19.07.1863 – 15.01.1934), \emph{Schriftsteller, Kritiker}!Konzert. Lustspiel in drei Akten1909@\strich\emph{Das Konzert. Lustspiel in drei Akten} {[}1909{]}|pw}« jetzt als \label{K_L02546_1v}\edtext{Tonfilm\pwindex{Mittler, Leo 18.12.1893 – 16.05.1958@\textsc{Mittler, Leo} (18.12.1893 – 16.05.1958), \emph{Schauspieler, Theater- und Filmregisseur}!Konzert1931@\strich\emph{Das Konzert} {[}1931{]}|pwv} erscheint}{\lemma{\textnormal{\emph{Tonfilm erscheint}}}\Cendnote{\textnormal{Die Verfilmung durch Leo Mittler\pwindex{Mittler, Leo 18.12.1893 – 16.05.1958@\textsc{Mittler, Leo} (18.12.1893 – 16.05.1958), \emph{Schauspieler, Theater- und Filmregisseur}|pwk} lief bereits seit 28. 8. 1931 in den Wien\oindex{Wien@\textbf{Wien}|pwk}er
               Kinos.}}}\label{K_L02546_1h}, nachdem es vorher, so weit ich mich erinnere, auch schon als \label{K_L02546_2v}\edtext{stummer Film}{\lemma{\textnormal{\emph{stummer Film}}}\Cendnote{\textnormal{\emph{The Concert}\pwindex{Schertzinger, Victor 1888-04-08 – 1941@\textsc{Schertzinger, Victor} (1888-04-08 – 1941)!Concert1921@\strich\emph{The Concert} {[}1921{]}|pwk} (1921), Regie Victor Schertzinger\pwindex{Schertzinger, Victor 1888-04-08 – 1941@\textsc{Schertzinger, Victor} (1888-04-08 – 1941)|pwk}, von
                     demselben neuerlich unter dem Titel \emph{Fashions in
                        Love}\pwindex{Schertzinger, Victor 1888-04-08 – 1941@\textsc{Schertzinger, Victor} (1888-04-08 – 1941)!Fashions in Love1929@\strich\emph{Fashions in Love} {[}1929{]}|pwk} (1929) als Tonfilm realisiert.}}}\label{K_L02546_2h}\pwindex{Schertzinger, Victor 1888-04-08 – 1941@\textsc{Schertzinger, Victor} (1888-04-08 – 1941)!Concert1921@\strich\emph{The Concert} {[}1921{]}|pwv} zu sehen war. Ich möchte nun gern wissen – falls es Dir nicht unbequem ist mir
               darauf zu antworten – ob, resp. welche Ansprüche die seinerzeitigen Verfertiger des
               stummen Films an Dich gestellt haben. Ich erlebe es in jedem einzelnen Fall, so mit
                  »\label{K_L02546_3v}\edtext{Liebelei\pwindex{Schnitzler, Arthur 15.05.1862 – 21.10.1931@\textsc{Schnitzler, Arthur} (15.05.1862 – 21.10.1931), \emph{Schriftsteller, Mediziner}!Liebelei. Schauspiel in drei Akten1895-10-09@\strich\emph{Liebelei. Schauspiel in drei Akten} {[}1895-10-09{]}|pw}}{\lemma{\textnormal{\emph{Liebelei}}}\Cendnote{\textnormal{Erstmals verfilmt 1914 (\emph{Elskovsleg}\pwindex{Holger-Madsen 1878-04-11 – 1943-11-30@\textsc{Holger-Madsen} (1878-04-11 – 1943-11-30), \emph{Filmschauspieler}!Elskovsleg1914@\strich\emph{Elskovsleg} {[}1914{]}|pwk}\pwindex{Blom, August 1869-12-26 – 1947-01-10@\textsc{Blom, August} (1869-12-26 – 1947-01-10)!Elskovsleg1914@\strich\emph{Elskovsleg} {[}1914{]}|pwk}, Regie Holger-Madsen\pwindex{Holger-Madsen 1878-04-11 – 1943-11-30@\textsc{Holger-Madsen} (1878-04-11 – 1943-11-30), \emph{Filmschauspieler}|pwk} und August Blom\pwindex{Blom, August 1869-12-26 – 1947-01-10@\textsc{Blom, August} (1869-12-26 – 1947-01-10)|pwk}). Ab
                     1921 Verhandlungen über eine Neuverfilmung, vgl. Arthur Schnitzler\pwindex{Schnitzler, Arthur 15.05.1862 – 21.10.1931@\textsc{Schnitzler, Arthur} (15.05.1862 – 21.10.1931), \emph{Schriftsteller, Mediziner}|pwk}: \emph{Filmarbeiten. Drehbücher, Entwürfe, Skizzen}. Hg. Achim Aurnhammer,
                     Hans Peter Buohler, Philipp Gresser, Julia Ilgner, Carolin Maikler, Lea
                     Marquart. Würzburg: \emph{Ergon}{ }2015, S. 101–103. Neuerliche Verfilmung\pwindex{Fleck, Jacob Julius 1881-11-18 – 1953-09-19@\textsc{Fleck, Jacob Julius} (1881-11-18 – 1953-09-19)!Liebelei (Film, 1927)1927@\strich\emph{Liebelei (Film, 1927)} {[}1927{]}|pwkv}\pwindex{Fleck, Luise 01.08.1873 – 15.03.1950@\textsc{Fleck, Luise} (01.08.1873 – 15.03.1950), \emph{Filmproduzentin}!Liebelei (Film, 1927)1927@\strich\emph{Liebelei (Film, 1927)} {[}1927{]}|pwkv}{ }1927 (Regie Jakob\pwindex{Fleck, Jacob Julius 1881-11-18 – 1953-09-19@\textsc{Fleck, Jacob Julius} (1881-11-18 – 1953-09-19)|pwk} und Luise Fleck\pwindex{Fleck, Luise 01.08.1873 – 15.03.1950@\textsc{Fleck, Luise} (01.08.1873 – 15.03.1950), \emph{Filmproduzentin}|pwk}).}}}\label{K_L02546_3h}«, »\label{K_L02546_4v}\edtext{Anatol\pwindex{Schnitzler, Arthur 15.05.1862 – 21.10.1931@\textsc{Schnitzler, Arthur} (15.05.1862 – 21.10.1931), \emph{Schriftsteller, Mediziner}!Anatol1892-10-29@\strich\emph{Anatol} {[}1892-10-29{]}|pw}}{\lemma{\textnormal{\emph{Anatol}}}\Cendnote{\textnormal{\emph{The Affairs of Anatol}\pwindex{DeMille, Cecil B. 1881-08-12 – 1959-01-21@\textsc{DeMille, Cecil B.} (1881-08-12 – 1959-01-21), \emph{Filmproduzent}!Affairs of Anatol1921@\strich\emph{The Affairs of Anatol} {[}1921{]}|pwk} (1921), Regie
                     Cecil B. DeMille\pwindex{DeMille, Cecil B. 1881-08-12 – 1959-01-21@\textsc{DeMille, Cecil B.} (1881-08-12 – 1959-01-21), \emph{Filmproduzent}|pwk}. Zu dem Plan einer
                  neuerlichen Verfilmung, die nicht realisiert wurde, gibt es Hinweise in Schnitzler\pwindex{Schnitzler, Arthur 15.05.1862 – 21.10.1931@\textsc{Schnitzler, Arthur} (15.05.1862 – 21.10.1931), \emph{Schriftsteller, Mediziner}|pwk}s \emph{Tagebuch}\pwindex{Schnitzler, Arthur 15.05.1862 – 21.10.1931@\textsc{Schnitzler, Arthur} (15.05.1862 – 21.10.1931), \emph{Schriftsteller, Mediziner}!Tagebuch1981 – 2000@\strich\emph{Tagebuch} {[}1981 – 2000{]}|pwk} zwischen 3. 11. 1930 und 4. 5. 1931.}}}\label{K_L02546_4h}«, »\label{K_L02546_5v}\edtext{Fräulein Else\pwindex{Schnitzler, Arthur 15.05.1862 – 21.10.1931@\textsc{Schnitzler, Arthur} (15.05.1862 – 21.10.1931), \emph{Schriftsteller, Mediziner}!Fraeulein Else01. 10. 1924@\strich\emph{Fräulein Else} {[}01. 10. 1924{]}|pw}}{\lemma{\textnormal{\emph{Fräulein Else}}}\Cendnote{\textnormal{\emph{Fräulein Else}\pwindex{Czinner, Paul 30.05.1890 – 22.06.1972@\textsc{Czinner, Paul} (30.05.1890 – 22.06.1972), \emph{Schriftsteller, Filmregisseur}!Fraeulein Else (Film)1929@\strich\emph{Fräulein Else (Film)} {[}1929{]}|pwk} (1929), Regie Paul Czinner\pwindex{Czinner, Paul 30.05.1890 – 22.06.1972@\textsc{Czinner, Paul} (30.05.1890 – 22.06.1972), \emph{Schriftsteller, Filmregisseur}|pwk}}}}\label{K_L02546_5h}«, dass sich die seinerzeitigen Verfertiger der stummen Fassung
               freundlich-erpresserisch gebärden, in welcher Haltung die Leute durch allerlei
               Gesetze, Auffassungen, Bestimmungen – auch insoweit sie nicht vorhanden sind – mehr
               oder weniger unterstützt werden.\pend
           \pstart
           Wolltest Du mir bei dieser Gelegenheit auch sonst ein Wort über Dich und Dein
               Befinden sagen, so wird es mich herzlich freuen.\pend
           \pstart
           {[}hs.:{]} Mit vielen Grüßen und der Bitte mich deiner verehrten Gattin\pwindex{Bahr-Mildenburg, Anna 29.11.1872 – 27.01.1947@\textsc{Bahr-Mildenburg, Anna} (29.11.1872 – 27.01.1947), \emph{Sängerin}|pwv} zu empfehlen{\\[\baselineskip]}Dein{\\[\baselineskip]}\spacefill\mbox{Arth}\pend
           \leftskip=0em{}\pstart
           \noindent{}{[}ms.:{]}  Herrn Hermann Bahr,{\\}München\oindex{Muenchen@\textbf{München}|pw}.\pend
           
         
         \endnumbering\mylabel{h}\end{ledgroupsized}  \newcommand{\dateiname}{L02546}\newcommand{\titel}{Arthur Schnitzler an Hermann Bahr, 5. 9. 1931}\newcommand{\editorInnen}{ Kurt Ifkovits,  Martin Anton Müller}%% latex-leseansicht-abspann.tex
%% Abspann für die Leseansicht.
%% Der Schalter \ifkorrekturansicht ist bereits durch den Vorspann gesetzt.

%% latex-abspann.tex
%% Gemeinsamer Abspann für Korrekturansicht und Leseansicht.
%% Setzt den Schalter \ifkorrekturansicht voraus (gesetzt in den
%% einbindenden Dateien latex-korrekturansicht-abspann.tex bzw.
%% latex-leseansicht-abspann.tex).
%% ---------------------------------------------------------------

\normalsize

% Das esempio-Environment wird nur in der Leseansicht benötigt
\ifkorrekturansicht\else
\newenvironment{esempio}[3]%
{
    \vspace{1.5ex}
    \rlap{\underline{#1}}
    \par
    \setlength{\parindent}{0cm}
    \nopagebreak
    \leftskip=#2cm
    \rightskip=#3cm
}
{
    \par
}
\fi

\doendnotes{C}
\bigskip
\vfill

\clearpage

\footnotesize

\ifkorrekturansicht
  \lohead{\textsc{register}}
\fi

% theindex-Environment neu definieren ohne reledmac
\makeatletter
\renewenvironment{theindex}{%
  \ifkorrekturansicht
    \section*{\indexname}%
  \else
    \subsubsection*{Index der erwähnten Entitäten}%
  \fi
  \setlength{\parindent}{0pt}%
  \setlength{\parskip}{0pt plus 0.3pt}%
  \let\item\@idxitem
}{%
  \ifkorrekturansicht\clearpage\fi
}
\makeatother

\IfFileExists{\jobname-pw.ind}{\input{\jobname-pw.ind}}{}

% Quellenangabe nur in der Leseansicht
\ifkorrekturansicht\else
% Fallback-Definitionen, falls die .tex-Datei \titel etc. nicht gesetzt hat
\providecommand{\titel}{}
\providecommand{\editorInnen}{}
\providecommand{\dateiname}{\jobname}

\vspace{3cm}

\vfill

\footnotesize
\textsc{Quelle}: \titel. Herausgegeben von {\editorInnen}. In: \emph{Arthur Schnitzler: Briefwechsel mit Autorinnen und Autoren}.
 Digitale Edition, https://schnitzler-briefe.acdh.oeaw.ac.at/{\dateiname}.html (Stand \today)
\fi

\end{document}


      