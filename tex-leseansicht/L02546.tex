%% latex-korrekturansicht-vorspann.tex
%% Vorspann für die Korrekturansicht.
%% Lädt die gemeinsame Datei latex-vorspann.tex mit gesetztem Schalter.

\newif\ifkorrekturansicht
\korrekturansichttrue

\input{../tex-inputs/latex-vorspann}


\section[Arthur Schnitzler an Hermann Bahr, 5. 9. 1931]{L02546 Arthur Schnitzler an Hermann Bahr, 5. 9. 1931}
\nopagebreak\mylabel{L02546v}
\rehead{ }\normalsize\beginnumbering\briefempfaengerindex{Bahr, Hermann@\textsc{Bahr, Hermann}!zzzSchnitzler, Arthur@\emph{von Arthur Schnitzler}!1931-09-051@{5. 9. 1931}|(be}
\toendnotes[C]{\smallbreak\pagebreak[2]}\Standort{TMW, HS AM 23400 Ba.}
\physDesc{Brief, 1 Blatt, 1 Seite, 929 Zeichen
\newline{}Schreibmaschine
\newline{}Handschrift: Bleistift, lateinische Kurrent (\noindent{}Unterschrift und Grußformel)
\newline{}Bahr: mit rotem Buntstift ergänzt: »Unmittelbar vor Fahrt nach
                                       Garmisch\oindex{Hotel Riessersee@\textbf{Hotel Riessersee}, \emph{Hotel (K.HTL)}|pw}« 
\newline{}Ordnung: mit Bleistift von unbekannter Hand beschriftet:
                                    »erledigt« }\Standort{DLA, A:Schnitzler, 85.1.294/8.}
\physDesc{Brief, Durchschlag1 Blatt, 1 Seite, 929 Zeichen
\newline{}Schreibmaschine}
\buchAbdrucke{\weitereDrucke{1) Arthur Schnitzler: \emph{The Letters of Arthur Schnitzler to Hermann Bahr}. Chapel Hill: \emph{The University of North Carolina Press} 1978, S. 118.} \weitereDrucke{2) Hermann Bahr, Arthur Schnitzler: \emph{Briefwechsel, Aufzeichnungen, Dokumente (1891–1931)}. Göttingen: \emph{Wallstein} 2018, S. 598–599.} }\toendnotes[C]{\smallbreak}
\pstart
           {\pb}\textcolor{gray}{\textbf{D\textsuperscript{r} Arthur Schnitzler}}\hfill 5. 9. 1931.\pend
           
\pstart
           \textcolor{gray}{\textbf{Wien. XVIII. Sternwartestrasse 71\oindex{Sternwartestrasse 71@\textbf{Sternwartestraße 71}, \emph{Wohngebäude (K.WHS)}|pw}.}}\pend
           
\pstart{}Lieber Hermann.\pend\vspace{0.5em}
\pstart
           Ich lese, dass Dein »Konzert\pwindex{Konzert. Lustspiel in drei Akten@\emph{Das Konzert. Lustspiel in drei Akten}|pw}« jetzt als \label{K_L02546-1v}\edtext{Tonfilm\pwindex{Konzert@\emph{Das Konzert}|pwv} erscheint}{\lemma{\textnormal{\emph{Tonfilm erscheint}}}\Cendnote{\textnormal{Die Verfilmung durch Leo Mittler\pwindex{Mittler, Leo 18.12.1893 – 16.05.1958@\textsc{Mittler, Leo} (18.12.1893 – 16.05.1958), \emph{Schauspieler/Schauspielerin, Theater- und Filmregisseur/Theater- und Filmregisseurin}|pwk} lief bereits seit 28. 8. 1931 in den Wien\oindex{Wien@\textbf{Wien}, \emph{A.ADM2}|pwk}er
                  Kinos.}}}\label{K_L02546-1}, nachdem es vorher, so weit ich mich erinnere, auch schon als \label{K_L02546-2v}\edtext{stummer Film}{\lemma{\textnormal{\emph{stummer Film}}}\Cendnote{\textnormal{\emph{The Concert}\pwindex{Concert@\emph{The Concert}|pwk} (1921), Regie Victor Schertzinger\pwindex{Schertzinger, Victor 1888-04-08 – 1941@\textsc{Schertzinger, Victor} (1888-04-08 – 1941)|pwk},
                     von demselben neuerlich unter dem Titel \emph{Fashions in Love}\pwindex{Fashions in Love@\emph{Fashions in Love}|pwk} (1929) als Tonfilm realisiert.}}}\label{K_L02546-2}\pwindex{Concert@\emph{The Concert}|pwv} zu sehen war. Ich möchte nun gern wissen – falls es Dir nicht unbequem ist mir
               darauf zu antworten – ob, resp. welche Ansprüche die seinerzeitigen Verfertiger des
               stummen Films an Dich gestellt haben. Ich erlebe es in jedem einzelnen Fall, so mit
                  »\label{K_L02546-3v}\edtext{Liebelei\pwindex{Liebelei. Schauspiel in drei Akten@\emph{Liebelei. Schauspiel in drei Akten}|pw}}{\lemma{\textnormal{\emph{Liebelei}}}\Cendnote{\textnormal{Erstmals wurde Schnitzlers Stück \emph{Liebelei}\pwindex{Liebelei. Schauspiel in drei Akten@\emph{Liebelei. Schauspiel in drei Akten}|pwk}{ }1914 verfilmt (\emph{Elskovsleg}\pwindex{Elskovsleg@\emph{Elskovsleg}|pwk}, Regie Holger-Madsen\pwindex{Holger-Madsen 1878-04-11 – 1943-11-30@\textsc{Holger-Madsen} (1878-04-11 – 1943-11-30), \emph{Regisseur/Regisseurin, Schauspieler/Schauspielerin, Drehbuchautor/Drehbuchautorin}|pwk} und August
                     Blom\pwindex{Blom, August 1869-12-26 – 1947-01-10@\textsc{Blom, August} (1869-12-26 – 1947-01-10)|pwk}). Ab 1921 gab es Verhandlungen über eine Neuverfilmung,
                  vgl. Arthur Schnitzler: \emph{Filmarbeiten. Drehbücher, Entwürfe, Skizzen}. Herausgegeben von  Achim
                     Aurnhammer, Hans Peter Buohler, Philipp Gresser, Julia Ilgner, Carolin Maikler,
                     Lea Marquart. Würzburg: \emph{Ergon}{ }2015, S. 101–103. Eine neuerliche Verfilmung\pwindex{Liebelei [Film, 1927]@\emph{Liebelei [Film, 1927]}|pwkv} kam
                  1927 in die Kinos (Regie Jakob\pwindex{Fleck, Jacob Julius 1881-11-18 – 1953-09-19@\textsc{Fleck, Jacob Julius} (1881-11-18 – 1953-09-19), \emph{Regisseur/Regisseurin, Filmproduzent/Filmproduzentin, Drehbuchautor/Drehbuchautorin}|pwk} und Luise Fleck\pwindex{Fleck, Luise 01.08.1873 – 15.03.1950@\textsc{Fleck, Luise} (01.08.1873 – 15.03.1950), \emph{Schauspieler/Schauspielerin, Filmproduzent/Filmproduzentin}|pwk}).}}}\label{K_L02546-3}«, »\label{K_L02546-4v}\edtext{Anatol\pwindex{Anatol@\emph{Anatol}|pw}}{\lemma{\textnormal{\emph{Anatol}}}\Cendnote{\textnormal{\emph{The Affairs of Anatol}\pwindex{Affairs of Anatol@\emph{The Affairs of Anatol}|pwk} (1921),
                  Regie Cecil B. DeMille\pwindex{DeMille, Cecil B. 1881-08-12 – 1959-01-21@\textsc{DeMille, Cecil B.} (1881-08-12 – 1959-01-21), \emph{Filmproduzent/Filmproduzentin}|pwk}. Zu dem Plan einer
                  neuerlichen Verfilmung, die nicht realisiert wurde, gibt es Hinweise in Schnitzlers{ }\emph{Tagebuch}\pwindex{Tagebuch@\emph{Tagebuch}|pwk} zwischen 3. 11. 1930 und 4. 5. 1931.}}}\label{K_L02546-4}«, »\label{K_L02546-5v}\edtext{Fräulein Else\pwindex{Fraeulein Else@\emph{Fräulein Else}|pw}}{\lemma{\textnormal{\emph{Fräulein Else}}}\Cendnote{\textnormal{\emph{Fräulein Else}\pwindex{Fraeulein Else@\emph{Fräulein Else}|pwk} wurde 1929 unter der Regie von
                        Paul Czinner\pwindex{Czinner, Paul 30.05.1890 – 22.06.1972@\textsc{Czinner, Paul} (30.05.1890 – 22.06.1972), \emph{Schriftsteller/Schriftstellerin, Filmregisseur/Filmregisseurin}|pwk} verfilmt.
               }}}\label{K_L02546-5}«, dass sich die seinerzeitigen Verfertiger der stummen Fassung
               freundlich-erpresserisch gebärden, in welcher Haltung die Leute durch allerlei
               Gesetze, Auffassungen, Bestimmungen – auch insoweit sie nicht vorhanden sind – mehr
               oder weniger unterstützt werden.\pend
           
\pstart
           Wolltest Du mir bei dieser Gelegenheit auch sonst ein Wort über Dich und Dein
               Befinden sagen, so wird es mich herzlich freuen.\pend
           
\pstart
           {[}hs.:{]} Mit vielen Grüßen und der Bitte mich deiner verehrten Gattin\pwindex{Bahr-Mildenburg, Anna 29.11.1872 – 27.01.1947@\textsc{Bahr-Mildenburg, Anna} (29.11.1872 – 27.01.1947), \emph{Sänger/Sängerin}|pwv} zu empfehlen{\\[\baselineskip]}Dein{\\[\baselineskip]}\spacefill\mbox{Arth}\pend
           \leftskip=0em{}
\pstart
           \noindent{}{[}ms.:{]}  Herrn Hermann Bahr,{\\}München\oindex{Muenchen@\textbf{München}, \emph{P.PPLA}|pw}.\pend
           \selectlanguage{ngerman}\endnumbering\briefempfaengerindex{Bahr, Hermann@\textsc{Bahr, Hermann}!zzzSchnitzler, Arthur@\emph{von Arthur Schnitzler}!1931-09-051@{5. 9. 1931}|)be}\mylabel{L02546h}  \normalsize

\doendnotes{C}
\bigskip
\vfill

\clearpage

\footnotesize

\lohead{\textsc{register}}

% Definiere theindex-Environment komplett neu ohne reledmac
\makeatletter
\renewenvironment{theindex}{%
  \section*{\indexname}%
  \setlength{\parindent}{0pt}%
  \setlength{\parskip}{0pt plus 0.3pt}%
  \let\item\@idxitem
}{%
  \clearpage
}
\makeatother

\IfFileExists{\jobname-pw.ind}{\input{\jobname-pw.ind}}{}

\end{document}

      