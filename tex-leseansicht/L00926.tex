%% latex-leseansicht-vorspann.tex
%% Vorspann für die Leseansicht.
%% Lädt die gemeinsame Datei latex-vorspann.tex mit nicht gesetztem Schalter.

\newif\ifkorrekturansicht
\korrekturansichtfalse

\input{../tex-inputs/latex-vorspann}


\section[Richard Beer-Hofmann an Arthur Schnitzler, 16. [6.] 1899]{L00926 Richard Beer-Hofmann an Arthur Schnitzler, 16. [6.] 1899}
\nopagebreak\mylabel{L00926v}
\rehead{ }\normalsize\beginnumbering\briefempfaengerindex{Schnitzler, Arthur@\textsc{Schnitzler, Arthur}!zzzBeer-Hofmann, Richard@\emph{von Richard Beer-Hofmann}!1899-06-161@{16. [6.] 1899}|(be}
\toendnotes[C]{\smallbreak\pagebreak[2]}
\correspDesc{Versand  durch Richard Beer-Hofmann am 16. [6.] 1899 in Seeboden
\newline{}Erhalt  durch Arthur Schnitzler im Zeitraum [17. 6. 1899
                  – 21. 6. 1899?] in Wien}\toendnotes[C]{\smallbreak}
\Standort{CUL, Schnitzler, B 8.}
\physDesc{Brief, 2 Blätter, 8 Seiten, 1647 Zeichen
\newline{}Handschrift: Bleistift, lateinische Kurrent
\newline{}Schnitzler: mit rotem Buntstift die Monatszahl »VII« zu
                                    »6« korrigiert 
\newline{}Ordnung: mit Bleistift von unbekannter Hand nummeriert:
                                    »129« }
\buchAbdrucke{\weitereDrucke{Arthur Schnitzler, Richard Beer-Hofmann: \emph{Briefwechsel 1891–1931}. Herausgegeben von Konstanze Fliedl. Wien, Zürich: \emph{Europaverlag} 1992, S. 129–130.} }\toendnotes[C]{\smallbreak}
\pstart
           \raggedleft{}{\pb}Seeboden\oindex{Seeboden am Millstättersee@\textbf{Seeboden am Millstättersee}|pw}{ }16/VII 1899.\pend
           \vspace{0.5em}
\pstart
           Lieber Arthur! ich schreibe Ihnen an einem jener »Abende am Wasser«
               die Sie so fürchten, und die ich nicht sehr liebe. Auf den Bergen liegt neuer Schnee,
               tagsüber hat’s geregnet und in der Villa nebenan spielen 4 junge Mäd{\pb}chen bei offenem Fenster Clavier,
               singen »ich bin eine Wittwe\pwindex{kleine Witwe@\emph{Die kleine Witwe}|pw}« und tollen mit
               einer empörenden Lustigkeit umher die alles nur nicht jung und unbefangen ist.\pend
           
\pstart
           Ich wollte mit meiner Antwort warten bis ich in besserer Sti{\geminationm}ung wäre; aber wann {\pb}wird das sein? Ich bin recht
                  versti{\geminationm}t und traurig; aus vielen Gründen; aus solchen
                  \strikeout{ke} die ich kenne und aus vielen anderen die ich
               nicht kenne, die aber sicher vorhanden sind und gegen die man noch machtloser {\pb}ist als gegen die anderen. Von Mayer\pwindex{Mayer, Oskar 1876 – 15.\,5.\,1915 München@\textsc{Mayer, Oskar} (1876 – 15.\,5.\,1915 München), \emph{Schriftsteller, Beamter}|pw} hatte ich dieser Tage Brief; er wollte
               näheres von mir hören wann wir unsere Fußpartie machen würden.\pend
           
\pstart
           Am selben Tag habe ich einen Brief aus Wien\oindex{Wien@\textbf{Wien}, \emph{Verwaltungsgebiet}|pw}
               erhalten daß Professor Fuchs\pwindex{Fuchs, Ernst 14.\,6.\,1851 Kritzendorf – 21.\,11.\,1930 Wien@\textsc{Fuchs, Ernst} (14.\,6.\,1851 Kritzendorf – 21.\,11.\,1930 Wien), \emph{Augenarzt}|pw}{ }{\pb}bei meinem Vater\pwindex{Beer, Hermann 10.\,8.\,1835 Radiměř – 3.\,10.\,1902 Wien@\textsc{Beer, Hermann} (10.\,8.\,1835 Radiměř – 3.\,10.\,1902 Wien), \emph{Rechtsanwalt}|pwv} (– D\textsuperscript{r}{ }Beer\pwindex{Beer, Hermann 10.\,8.\,1835 Radiměř – 3.\,10.\,1902 Wien@\textsc{Beer, Hermann} (10.\,8.\,1835 Radiměř – 3.\,10.\,1902 Wien), \emph{Rechtsanwalt}|pw} –) grauen Staar diagnosticirte. Ich
               erhielt die Nachricht indirekt und wußte daher absolut nicht wie oder wo ich meinen
                  So{\geminationm}er verbringen würde. Habe daher an Mayer\pwindex{Mayer, Oskar 1876 – 15.\,5.\,1915 München@\textsc{Mayer, Oskar} (1876 – 15.\,5.\,1915 München), \emph{Schriftsteller, Beamter}|pw} nur kurz geschrieben {\pb}daß ich momentan nicht über meine
               Zeit disponiren könne.\pend
           
\pstart
           Inzwischen habe ich bessere Nachrichten von meinem Vater\pwindex{Beer, Hermann 10.\,8.\,1835 Radiměř – 3.\,10.\,1902 Wien@\textsc{Beer, Hermann} (10.\,8.\,1835 Radiměř – 3.\,10.\,1902 Wien), \emph{Rechtsanwalt}|pwv}; es hat noch 1–2 Jahre eventuell Zeit mit einer
               Operation u sein moralischer Zustand ist kein schlechter. {\pb}Sollten Sie Mayer\pwindex{Mayer, Oskar 1876 – 15.\,5.\,1915 München@\textsc{Mayer, Oskar} (1876 – 15.\,5.\,1915 München), \emph{Schriftsteller, Beamter}|pw} sehen so besprechen Sie mit ihm das Nötige wegen einer
               Fußtour; ich schließe mich an.\pend
           
\pstart
           Wann wollen Sie hieher ko{\geminationm}en? Schreiben Sie mir früher
               damit ich Zi{\geminationm}er etc. versorge. Vielleicht hole ich Sie
               an irgend einer Bahnstation ab.\pend
           
\pstart
           {\pb}Bitte wie ist Pauls\pwindex{Goldmann, Paul 31.\,1.\,1865 Breslau – 25.\,9.\,1935 Wien@\textsc{Goldmann, Paul} (31.\,1.\,1865 Breslau – 25.\,9.\,1935 Wien), \emph{Schriftsteller, Journalist}|pw} Adresse in \uline{Frankfurt}\oindex{Frankfurt am Main@\textbf{Frankfurt am Main}, \emph{Hauptstadt}|pw}? Grüßen Sie Schwarzkopf\pwindex{Schwarzkopf, Gustav 7.\,11.\,1853 Wien – 13.\,11.\,1939 ebd.@\textsc{Schwarzkopf, Gustav} (7.\,11.\,1853 Wien – 13.\,11.\,1939 ebd.), \emph{Schriftsteller}|pw} und Hugo\pwindex{Hofmannsthal, Hugo von 1.\,2.\,1874 Wien – 15.\,7.\,1929 Rodaun@\textsc{Hofmannsthal, Hugo von} (1.\,2.\,1874 Wien – 15.\,7.\,1929 Rodaun), \emph{Schriftsteller}|pw}. Von Herzen\pend
           \pstart Ihr \spacefill\mbox{Richard}\pend{}\selectlanguage{ngerman}\endnumbering\briefempfaengerindex{Schnitzler, Arthur@\textsc{Schnitzler, Arthur}!zzzBeer-Hofmann, Richard@\emph{von Richard Beer-Hofmann}!1899-06-161@{16. [6.] 1899}|)be}\mylabel{L00926h}  \newcommand{\dateiname}{L00926}\newcommand{\titel}{Richard Beer-Hofmann an Arthur Schnitzler, 16. [6.] 1899}\newcommand{\editorInnen}{Martin Anton Müller und Gerd-Hermann Susen}%% latex-leseansicht-abspann.tex
%% Abspann für die Leseansicht.
%% Der Schalter \ifkorrekturansicht ist bereits durch den Vorspann gesetzt.

%% latex-abspann.tex
%% Gemeinsamer Abspann für Korrekturansicht und Leseansicht.
%% Setzt den Schalter \ifkorrekturansicht voraus (gesetzt in den
%% einbindenden Dateien latex-korrekturansicht-abspann.tex bzw.
%% latex-leseansicht-abspann.tex).
%% ---------------------------------------------------------------

\normalsize

% Das esempio-Environment wird nur in der Leseansicht benötigt
\ifkorrekturansicht\else
\newenvironment{esempio}[3]%
{
    \vspace{1.5ex}
    \rlap{\underline{#1}}
    \par
    \setlength{\parindent}{0cm}
    \nopagebreak
    \leftskip=#2cm
    \rightskip=#3cm
}
{
    \par
}
\fi

\doendnotes{C}
\bigskip
\vfill

\clearpage

\footnotesize

\ifkorrekturansicht
  \lohead{\textsc{register}}
\fi

% theindex-Environment neu definieren ohne reledmac
\makeatletter
\renewenvironment{theindex}{%
  \ifkorrekturansicht
    \section*{\indexname}%
  \else
    \subsubsection*{Index der erwähnten Entitäten}%
  \fi
  \setlength{\parindent}{0pt}%
  \setlength{\parskip}{0pt plus 0.3pt}%
  \let\item\@idxitem
}{%
  \ifkorrekturansicht\clearpage\fi
}
\makeatother

\IfFileExists{\jobname-pw.ind}{\input{\jobname-pw.ind}}{}

% Quellenangabe nur in der Leseansicht
\ifkorrekturansicht\else
% Fallback-Definitionen, falls die .tex-Datei \titel etc. nicht gesetzt hat
\providecommand{\titel}{}
\providecommand{\editorInnen}{}
\providecommand{\dateiname}{\jobname}

\vspace{3cm}

\vfill

\footnotesize
\textsc{Quelle}: \titel. Herausgegeben von {\editorInnen}. In: \emph{Arthur Schnitzler: Briefwechsel mit Autorinnen und Autoren}.
 Digitale Edition, https://schnitzler-briefe.acdh.oeaw.ac.at/{\dateiname}.html (Stand \today)
\fi

\end{document}


