%% latex-leseansicht-vorspann.tex
%% Vorspann für die Leseansicht.
%% Lädt die gemeinsame Datei latex-vorspann.tex mit nicht gesetztem Schalter.

\newif\ifkorrekturansicht
\korrekturansichtfalse

\input{../tex-inputs/latex-vorspann}

\begin{center}
            \textcolor{red}{ENTWURF, NICHT FERTIG KORRIGIERT}
                      \end{center}
            
         
         \renewcommand{\erwaehntePersonen}{Personen: Peter Dorner, Paul Goldmann, Clementine Goldmann, Hugo von Hofmannsthal, Olga Schnitzler, Heinrich Schnitzler, Elisabeth Steinrück}
         \renewcommand{\erwaehnteInstitutionen}{Institutionen: Schiller-Theater}
         \renewcommand{\erwaehnteOrte}{Orte: Berlin, Dessauer Straße, Hinterbrühl, Welsberg-Taisten, Wien}
         \renewcommand{\erwaehnteWerke}{}
               \section[ Paul Goldmann an Olga Gussmann, 9. 7. {[}1902{]}]{ Paul Goldmann an Olga Gussmann, 9. 7. {[}1902{]}}\nopagebreak\mylabel{v}\rehead{ }\begin{ledgroupsized}[t]{13cm}\normalsize\beginnumbering \toendnotes[C]{\smallbreak\pagebreak[2]} \Standort{DLA, A:Schnitzler, HS.NZ85.1.5247.}
\physDesc{Brief, 1 Blatt, 4 Seiten, 1818 Zeichen
\newline{}Handschrift: blaue Tinte, deutsche Kurrent}\toendnotes[C]{\smallbreak}\pstart
           \noindent{}\raggedleft{}{\pb}\textcolor{gray}{\textbf{DESSAUERSTRASSE 19\oindex{Dessauer Strasse@\textbf{Dessauer Straße}|pw}}}\pend
           \pstart
           Berlin\oindex{Berlin@\textbf{Berlin}|pw}, 9. Juli.\pend
           \pstart\center{}Liebe Freundin,\pend\pstart
           Bitte, laſſen Sie das Danken ſein. Das war doch Alles ſelbſtverſtändlich. Es iſt
                  \textcolor{gray}{d}och die erſte und einfachſte Pflicht der Freundſchaft, in
               wichtigen Lebensangelegenheiten \label{K_L03532-1v}\edtext{Beiſtand}{\lemma{\textnormal{\emph{Beiſtand}}}\Cendnote{\textnormal{siehe Paul Goldmann an Arthur Schnitzler, 16. 6. [1902]}}}\label{K_L03532-1h} zu leiſten.\pend
           \pstart
           Ihre lieben Mittheilungen über \label{K_L03532-2v}\edtext{\textsc{Peter Dorner\pwindex{Dorner, Peter 17.02.1857 – 01.04.1931@\textsc{Dorner, Peter} (17.02.1857 – 01.04.1931), \emph{Schmied, Kunsthandwerker, Kunstschmied}|pw} etc}}{\lemma{\textnormal{\emph{Peter Dorner etc}}}\Cendnote{\textnormal{vermutlich Bezug auf Peter Dorner\pwindex{Dorner, Peter 17.02.1857 – 01.04.1931@\textsc{Dorner, Peter} (17.02.1857 – 01.04.1931), \emph{Schmied, Kunsthandwerker, Kunstschmied}|pwk}s kürzlich vollzogene Verlobung, vgl. Paul Goldmann an Arthur Schnitzler, 4. 8. [1904]}}}\label{K_L03532-2h}. haben mich ſehr intereſſirt. Nur hätte ich gern auch etwas Näheres über Ihr
               Ergehen gehört.\pend
           \pstart
           Daß unſer lieber \label{K_L03532-3v}\edtext{\textsc{Welsberg\oindex{Welsberg-Taisten@\textbf{Welsberg-Taisten}|pw}} von \textsc{Hoffmannsthal\pwindex{Hofmannsthal, Hugo von 1874-02-01 – 1929-07-15@\textsc{Hofmannsthal, Hugo von} (1874-02-01 – 1929-07-15), \emph{Schriftsteller}|pw}} »entdeckt«}{\lemma{\textnormal{\emph{Welsberg … »entdeckt«}}}\Cendnote{\textnormal{Hugo von Hofmannsthal\pwindex{Hofmannsthal, Hugo von 1874-02-01 – 1929-07-15@\textsc{Hofmannsthal, Hugo von} (1874-02-01 – 1929-07-15), \emph{Schriftsteller}|pwk} hielt sich im Sommer
                     1902 ebenso in Welsberg\oindex{Welsberg-Taisten@\textbf{Welsberg-Taisten}|pwk} auf (vgl. A. S.: \emph{Tagebuch}, 5. 7. 1902 und 5. 7. 1902).}}}\label{K_L03532-3h} worden iſt, thut mir leid. Es wird jetzt ein
               literariſcher Ort werden – obwohl \strikeout{es}{ }{\pb}es doch ein beſſeres Schickſal verdient hätte.\pend
           \pstart
           Meine Mutter\pwindex{Goldmann, Clementine 1842-05-15 – 1924-02-24@\textsc{Goldmann, Clementine} (1842-05-15 – 1924-02-24)|pwv} hat ſich ſehr
               über Ihre und \textsc{Liesl\pwindex{Steinrueck, Elisabeth 19.11.1885 – 07.04.1920@\textsc{Steinrück, Elisabeth} (19.11.1885 – 07.04.1920)|pw}s} Grüße gefreut und erwidert ſie
               auf das Herzlichſte.\pend
           \pstart
           Bitte, grüßen Sie meinen lieben \textsc{Arthur\pwindex{Schnitzler, Arthur 15.05.1862 – 21.10.1931@\textsc{Schnitzler, Arthur} (15.05.1862 – 21.10.1931), \emph{Schriftsteller, Mediziner}|pw}}, wenn er \label{K_L03532-4v}\edtext{morgen}{\lemma{\textnormal{\emph{morgen}}}\Cendnote{\textnormal{Auch das ist ein Hinweis darauf, dass
                  der Brief aus dem Jahr 1902 stammt. Schnitzler\pwindex{Schnitzler, Arthur 15.05.1862 – 21.10.1931@\textsc{Schnitzler, Arthur} (15.05.1862 – 21.10.1931), \emph{Schriftsteller, Mediziner}|pwk} war am 8. 7. 1902 nach Wien\oindex{Wien@\textbf{Wien}|pwk} zurückgekehrt, Goldmann\pwindex{Goldmann, Paul 31.01.1865 – 25.09.1935@\textsc{Goldmann, Paul} (31.01.1865 – 25.09.1935), \emph{Schriftsteller, Journalist}|pwk} irrte
                  sich also nur um zwei Tage.}}}\label{K_L03532-4h} zurückkommt, vielmals von mir. Ich \strikeout{leſ} danke ihm für ſeine Karten von unterwegs und hoffe,
               bald Ausführlicheres von ihm zu hören.\pend
           \pstart
           Wenn Ihnen der blöde Fratz\pwindex{Steinrueck, Elisabeth 19.11.1885 – 07.04.1920@\textsc{Steinrück, Elisabeth} (19.11.1885 – 07.04.1920)|pwv}
               (ich meine natürlich \textsc{Liesl\pwindex{Steinrueck, Elisabeth 19.11.1885 – 07.04.1920@\textsc{Steinrück, Elisabeth} (19.11.1885 – 07.04.1920)|pw}}), erzählt hat, daß ich über Sie »geſchimpft« habe, ſo hat ſie wieder einmal {\pb}geſprochen, was ſie\pwindex{Steinrueck, Elisabeth 19.11.1885 – 07.04.1920@\textsc{Steinrück, Elisabeth} (19.11.1885 – 07.04.1920)|pwv} nicht verantworten kann. Ich habe ihr nur geſagt (weil ſie
               mir durch Äußerungen und Verhalten dazu Anlaß gegeben hatte), was ich auch Ihnen
               ſchon geſagt habe: wie wenig Sie Beide\pwindex{Steinrueck, Elisabeth 19.11.1885 – 07.04.1920@\textsc{Steinrück, Elisabeth} (19.11.1885 – 07.04.1920)|pwv} mich \label{K_L03532-5v}\edtext{verſtehen}{\lemma{\textnormal{\emph{verſtehen}}}\Cendnote{\textnormal{Bezug unklar}}}\label{K_L03532-5h} und wie ſehr es \strikeout{mich} mir leid thut, daß ich gerade i\substVorne{}\textsuperscript{m}\substDazwischen{}n\substHinten{} einem Kreiſe, dem ich ſo nahe ſtehe, ſo wenig Verſtändniß finde. An Ihrer
               freundſchaftlichen Geſinnung für mich zweifle ich keinen Augenblick, ebenſo wie Sie
               hoffentlich nicht an der meinigen zweifeln. Das Wort »Haß« ſollte in einem Briefe,
               den Sie mir ſchreiben, wirklich nicht ſtehen.\pend
           \pstart
           {\pb}Es thut mir leid, daß ich nicht \label{K_L03532-6v}\edtext{auch Ihnen zu einem Engagement in einem
                  Berlin\oindex{Berlin@\textbf{Berlin}|pw}er Theater verhelfen}{\lemma{\textnormal{\emph{auch … verhelfen}}}\Cendnote{\textnormal{Bezug auf Elisabeth Gussmann\pwindex{Steinrueck, Elisabeth 19.11.1885 – 07.04.1920@\textsc{Steinrück, Elisabeth} (19.11.1885 – 07.04.1920)|pwk}s Engagement am \emph{Schiller-Theater}\orgindex{Schiller-Theater@Schiller-Theater|pwk} ab dem 1. 9. 1902, siehe Paul Goldmann an Arthur Schnitzler, 16. 6. [1902].}}}\label{K_L03532-6h} kann; aber
                  ich\strikeout{\textcolor{gray}{d}} denke mir, daß Sie \label{K_L03532-7v}\edtext{Beſſeres
                  gefunden}{\lemma{\textnormal{\emph{Beſſeres
                  gefunden}}}\Cendnote{\textnormal{vermutlich eine Anspielung
                  auf Schnitzler\pwindex{Schnitzler, Arthur 15.05.1862 – 21.10.1931@\textsc{Schnitzler, Arthur} (15.05.1862 – 21.10.1931), \emph{Schriftsteller, Mediziner}|pwk} bzw. die bevorstehende Geburt
                  des gemeinsamen Sohns Heinrich\pwindex{Schnitzler, Heinrich 09.08.1902 – 12.07.1982@\textsc{Schnitzler, Heinrich} (09.08.1902 – 12.07.1982), \emph{Regisseur, Schauspieler}|pwk}}}}\label{K_L03532-7h} haben, als Ihnen die größte Stellung an der größten Bühne jemals hätte bieten
               können.\pend
           \pstart
           Mit herzlichen Grüßen an Sie und \textsc{Liesl\pwindex{Steinrueck, Elisabeth 19.11.1885 – 07.04.1920@\textsc{Steinrück, Elisabeth} (19.11.1885 – 07.04.1920)|pw}} (der ich für ihren Brief danke) bin ich {\\[\baselineskip]}Ihr ergebener {\\[\baselineskip]}\spacefill\mbox{Dr. Paul Goldmann.}\pend
           \leftskip=0em{}
         
         \endnumbering\mylabel{h}\end{ledgroupsized}\begin{anhang}\end{anhang}\newcommand{\dateiname}{L03532}\newcommand{\titel}{Paul Goldmann an Olga Gussmann, 9. 7. [1902]}\newcommand{\editorInnen}{Martin Anton Müller und Laura Untner}%% latex-leseansicht-abspann.tex
%% Abspann für die Leseansicht.
%% Der Schalter \ifkorrekturansicht ist bereits durch den Vorspann gesetzt.

%% latex-abspann.tex
%% Gemeinsamer Abspann für Korrekturansicht und Leseansicht.
%% Setzt den Schalter \ifkorrekturansicht voraus (gesetzt in den
%% einbindenden Dateien latex-korrekturansicht-abspann.tex bzw.
%% latex-leseansicht-abspann.tex).
%% ---------------------------------------------------------------

\normalsize

% Das esempio-Environment wird nur in der Leseansicht benötigt
\ifkorrekturansicht\else
\newenvironment{esempio}[3]%
{
    \vspace{1.5ex}
    \rlap{\underline{#1}}
    \par
    \setlength{\parindent}{0cm}
    \nopagebreak
    \leftskip=#2cm
    \rightskip=#3cm
}
{
    \par
}
\fi

\doendnotes{C}
\bigskip
\vfill

\clearpage

\footnotesize

\ifkorrekturansicht
  \lohead{\textsc{register}}
\fi

% theindex-Environment neu definieren ohne reledmac
\makeatletter
\renewenvironment{theindex}{%
  \ifkorrekturansicht
    \section*{\indexname}%
  \else
    \subsubsection*{Index der erwähnten Entitäten}%
  \fi
  \setlength{\parindent}{0pt}%
  \setlength{\parskip}{0pt plus 0.3pt}%
  \let\item\@idxitem
}{%
  \ifkorrekturansicht\clearpage\fi
}
\makeatother

\IfFileExists{\jobname-pw.ind}{\input{\jobname-pw.ind}}{}

% Quellenangabe nur in der Leseansicht
\ifkorrekturansicht\else
% Fallback-Definitionen, falls die .tex-Datei \titel etc. nicht gesetzt hat
\providecommand{\titel}{}
\providecommand{\editorInnen}{}
\providecommand{\dateiname}{\jobname}

\vspace{3cm}

\vfill

\footnotesize
\textsc{Quelle}: \titel. Herausgegeben von {\editorInnen}. In: \emph{Arthur Schnitzler: Briefwechsel mit Autorinnen und Autoren}.
 Digitale Edition, https://schnitzler-briefe.acdh.oeaw.ac.at/{\dateiname}.html (Stand \today)
\fi

\end{document}


      