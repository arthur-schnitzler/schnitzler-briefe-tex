%% latex-leseansicht-vorspann.tex
%% Vorspann für die Leseansicht.
%% Lädt die gemeinsame Datei latex-vorspann.tex mit nicht gesetztem Schalter.

\newif\ifkorrekturansicht
\korrekturansichtfalse

\input{../tex-inputs/latex-vorspann}

\begin{center}
            \textcolor{red}{ENTWURF, NICHT FERTIG KORRIGIERT}
                      \end{center}
            
         
         \renewcommand{\erwaehntePersonen}{Personen: Hugo von Hofmannsthal, Olga Schnitzler, Elisabeth Steinrück}
         \renewcommand{\erwaehnteOrte}{Orte: Berlin, Welsberg-Taisten, Wien}
         \renewcommand{\erwaehnteWerke}{}
               \section[ Paul Goldmann an Olga XXXX Gussmann/Schnitzler, 9. 7. {[}XXXX{]}]{ Paul Goldmann an Olga XXXX Gussmann/Schnitzler, 9. 7. {[}XXXX{]}}\nopagebreak\mylabel{v}\rehead{ }\begin{ledgroupsized}[t]{13cm}\normalsize\beginnumbering \toendnotes[C]{\smallbreak\pagebreak[2]} \Standort{DLA, A:Schnitzler, HS.1985.1.5247.}
\physDesc{,  Blätter,  Seiten
\newline{}Handschrift: , deutsche Kurrent}\toendnotes[C]{\smallbreak}{\pb}\textcolor{gray}{\textbf{DESSAUERSTRASSE 19\oindex{XXXX Ortsangabe fehlt|pw}}}\textcolor{red}{\textsuperscript{\textbf{KEY}}}\pstart
           Berlin\oindex{Berlin@\textbf{Berlin}|pw}, 9. Juli.\pend
           \pstart{}Liebe Freundin,\pend\pstart
           Bitte, laſſen Sie das Danken ſein. Das war doch Alles ſelbſtverſtändlich. Es iſt
                  \textcolor{gray}{d}och die erſte und einfachſte Pflicht der Freundſchaft, in
               wichtigen Lebensangelegenheiten Beiſtand zu leiſten.\pend
           \pstart
           Ihre lieben Mittheilungen über \textsc{Peter Dorner\textcolor{red}{\textsuperscript{\textbf{KEY}}} etc}. haben mich ſehr intereſſirt.
               Nur hätte ich gern auch etwas Näheres über Ihr Ergehen gehört.\pend
           \pstart
           Daß unſer lieber \textsc{Welsberg\oindex{Welsberg-Taisten@\textbf{Welsberg-Taisten}|pw}} von \textsc{Hoffmannsthal\pwindex{Hofmannsthal, Hugo von 1874-02-01 – 1929-07-15@\textsc{Hofmannsthal, Hugo von} (1874-02-01 – 1929-07-15), \emph{Schriftsteller}|pw}} »entdeckt« worden iſt, thut mir leid. Es wird jetzt ein literariſcher Ort
               werden – obwohl \strikeout{es}{\pb} es doch ein beſſeres
               Schickſal verdient hätte.\pend
           \pstart
           Meine Mutter\textcolor{red}{\textsuperscript{\textbf{KEY}}} hat ſich ſehr über Ihre und \textsc{Liesl\pwindex{Steinrueck, Elisabeth 19.11.1885 – 07.04.1920@\textsc{Steinrück, Elisabeth} (19.11.1885 – 07.04.1920)|pw}s} Grüße gefreut und erwidert ſie
               auf das Herzlichſte.\pend
           \pstart
           Bitte, grüßen Sie meinen lieben \textsc{Arthur\pwindex{Schnitzler, Arthur 15.05.1862 – 21.10.1931@\textsc{Schnitzler, Arthur} (15.05.1862 – 21.10.1931), \emph{Schriftsteller, Mediziner}|pw}}, wenn er morgen zurückkommt, vielmals von mir. Ich \strikeout{leſ} danke ihm für ſeine Karten von unterwegs und hoffe,
               bald Ausführlicheres von ihm zu hören.\pend
           \pstart
           Wenn Ihnen der blöde Fratz (ich meine natürlich \textsc{Liesl\pwindex{Steinrueck, Elisabeth 19.11.1885 – 07.04.1920@\textsc{Steinrück, Elisabeth} (19.11.1885 – 07.04.1920)|pw}}), erzählt hat, daß ich über Sie »geſchimpft« habe, ſo hat ſie wieder einmal {\pb} geſprochen, was ſie\textcolor{red}{\textsuperscript{\textbf{KEY}}} nicht verantworten kann. Ich habe ihr nur geſagt
               (weil ſie mir durch Äußerungen und Verhalten dazu Anlaß gegeben hatte), was ich auch
               Ihnen ſchon geſagt habe: wie wenig Sie Beide\textcolor{red}{\textsuperscript{\textbf{KEY}}} mich verſtehen und wie ſehr es \strikeout{mich}
               mir leid thut, daß ich gerade i\strikeout{m
               }n einem Kreiſe, dem ich ſo nahe ſtehe, ſo wenig Verſtändniß finde. An
               Ihrer freundſchaftlichen Geſinnung für mich zweifle ich keinen Augenblick, ebenſo wie
               Sie hoffentlich nicht an der meinigen zweifeln. Das Wort »Haß« ſollte in einem
               Briefe, den Sie mir ſchreiben, wirklich nicht ſtehen. {\pb}\pend
           \pstart
           Es thut mir leid, daß ich nicht auch Ihnen zu einem Engagement. in einem Berlin\oindex{Berlin@\textbf{Berlin}|pw}er Theater verhelfen kann; aber ich\strikeout{\textcolor{gray}{d}} denke mir, daß Sie Beſſeres gefunden haben, als Ihnen die größte Stellung an
               der größten Bühne jemals hätte bieten können. {\\[\baselineskip]}Mit herzlichen Grüßen an
               Sie\pend
           \leftskip=0em{}\pstart
           {\\[\baselineskip]}und \textsc{Liesl\pwindex{Steinrueck, Elisabeth 19.11.1885 – 07.04.1920@\textsc{Steinrück, Elisabeth} (19.11.1885 – 07.04.1920)|pw}} (der ich für ihren\pend
           \leftskip=0em{}\pstart
           {\\[\baselineskip]}Brief danke) bin ich\pend
           \leftskip=0em{}\pstart
           {\\[\baselineskip]}Ihr ergebener\pend
           \leftskip=0em{}\pstart
           {\\[\baselineskip]}\spacefill\mbox{Dr. Paul Goldmann.}\pend
           \leftskip=0em{}
         
         \endnumbering\mylabel{h}\end{ledgroupsized}\begin{anhang}\end{anhang}\newcommand{\dateiname}{L03532}\newcommand{\titel}{Paul Goldmann an Olga XXXX Gussmann/Schnitzler, 9. 7. [XXXX]}\newcommand{\editorInnen}{Martin Anton Müller und Laura Untner}%% latex-leseansicht-abspann.tex
%% Abspann für die Leseansicht.
%% Der Schalter \ifkorrekturansicht ist bereits durch den Vorspann gesetzt.

%% latex-abspann.tex
%% Gemeinsamer Abspann für Korrekturansicht und Leseansicht.
%% Setzt den Schalter \ifkorrekturansicht voraus (gesetzt in den
%% einbindenden Dateien latex-korrekturansicht-abspann.tex bzw.
%% latex-leseansicht-abspann.tex).
%% ---------------------------------------------------------------

\normalsize

% Das esempio-Environment wird nur in der Leseansicht benötigt
\ifkorrekturansicht\else
\newenvironment{esempio}[3]%
{
    \vspace{1.5ex}
    \rlap{\underline{#1}}
    \par
    \setlength{\parindent}{0cm}
    \nopagebreak
    \leftskip=#2cm
    \rightskip=#3cm
}
{
    \par
}
\fi

\doendnotes{C}
\bigskip
\vfill

\clearpage

\footnotesize

\ifkorrekturansicht
  \lohead{\textsc{register}}
\fi

% theindex-Environment neu definieren ohne reledmac
\makeatletter
\renewenvironment{theindex}{%
  \ifkorrekturansicht
    \section*{\indexname}%
  \else
    \subsubsection*{Index der erwähnten Entitäten}%
  \fi
  \setlength{\parindent}{0pt}%
  \setlength{\parskip}{0pt plus 0.3pt}%
  \let\item\@idxitem
}{%
  \ifkorrekturansicht\clearpage\fi
}
\makeatother

\IfFileExists{\jobname-pw.ind}{\input{\jobname-pw.ind}}{}

% Quellenangabe nur in der Leseansicht
\ifkorrekturansicht\else
% Fallback-Definitionen, falls die .tex-Datei \titel etc. nicht gesetzt hat
\providecommand{\titel}{}
\providecommand{\editorInnen}{}
\providecommand{\dateiname}{\jobname}

\vspace{3cm}

\vfill

\footnotesize
\textsc{Quelle}: \titel. Herausgegeben von {\editorInnen}. In: \emph{Arthur Schnitzler: Briefwechsel mit Autorinnen und Autoren}.
 Digitale Edition, https://schnitzler-briefe.acdh.oeaw.ac.at/{\dateiname}.html (Stand \today)
\fi

\end{document}


      