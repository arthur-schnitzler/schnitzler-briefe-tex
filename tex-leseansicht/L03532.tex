%% latex-leseansicht-vorspann.tex
%% Vorspann für die Leseansicht.
%% Lädt die gemeinsame Datei latex-vorspann.tex mit nicht gesetztem Schalter.

\newif\ifkorrekturansicht
\korrekturansichtfalse

\input{../tex-inputs/latex-vorspann}


\section[ Paul Goldmann an Olga Gussmann, 9. 7. {[}1902?{]}]{L03532 Paul Goldmann an Olga Gussmann,  9. 7. [1902?]}
\nopagebreak\mylabel{L03532v}
\rehead{ }\normalsize\beginnumbering\briefempfaengerindex{Schnitzler, Olga@\textsc{Schnitzler, Olga}!zzzGoldmann, Paul@\emph{von Paul Goldmann}!1902-07-092@{9. 7. [1902?]}|(be}
\toendnotes[C]{\smallbreak\pagebreak[2]}
\correspDesc{Versand  durch Paul Goldmann am 9. 7. [1902?] in Berlin
\newline{}Erhalt  durch Olga Gussmann im Zeitraum [10. 7. 1902
                  – 14. 7. 1902?] in Hinterbrühl}\toendnotes[C]{\smallbreak}
\Standort{DLA, A:Schnitzler, HS.NZ85.1.5247.}
\physDesc{Brief, 1 Blatt, 4 Seiten, 1819 Zeichen
\newline{}Handschrift: blaue Tinte, deutsche Kurrent}\toendnotes[C]{\smallbreak}
\pstart
           \raggedleft{}{\pb}\textcolor{gray}{\textbf{DESSAUERSTRASSE 19\oindex{Dessauer Straße@\textbf{Dessauer Straße}, \emph{Straße}|pw}}}\pend
           
\pstart
           Berlin\oindex{Berlin@\textbf{Berlin}, \emph{Hauptstadt}|pw}, 9. Juli.\pend
           
\pstart\center{}Liebe Freundin,\pend\vspace{0.5em}
\pstart
           Bitte, laſſen Sie das Danken{ }ſein. Das war doch Alles{ }ſelbſtverſtändlich. Es iſt noch
               die erſte und einfachſte Pflicht der Freundſchaft, in wichtigen Lebensangelegenheiten
                  \label{K_L03532-1v}\edtext{Beiſtand}{\lemma{\textnormal{\emph{Beistand}}}\Cendnote{\textnormal{Siehe XXXX Auszeichnungsfehler: Dokument L03211 nicht gefunden.
               }}}\label{K_L03532-1} zu leiſten.\pend
           
\pstart
           Ihre lieben \label{K_L03532-2v}\edtext{Mittheilungen über 
               \textsc{Peter Dorner}\pwindex{Dorner, Peter 17.\,2.\,1857 Welsberg-Taisten – 1.\,4.\,1931 ebd.@\textsc{Dorner, Peter} (17.\,2.\,1857 Welsberg-Taisten – 1.\,4.\,1931 ebd.), \emph{Schmied, Kunsthandwerker, Kunstschmied}|pw}}{\lemma{\textnormal{\emph{Mittheilungen … Dorner}}}\Cendnote{\textnormal{Arthur Schnitzler hatte
                  den Kunstschmied am 4. 7. 1902 in dessen Atelier aufgesucht.}}}\label{K_L03532-2}{ }\textsc{etc.} haben
               mich{ }ſehr intereſſirt. Nur hätte ich gern auch etwas Näheres über Ihr Ergehen
               gehört.\pend
           
\pstart
           Daß unſer liebes \label{K_L03532-3v}\edtext{\textsc{Welsberg\oindex{Welsberg-Taisten@\textbf{Welsberg-Taisten}, \emph{Verwaltungsgebiet}|pw}} von \textsc{Hoffmannsthal\pwindex{Hofmannsthal, Hugo von 1.\,2.\,1874 Wien – 15.\,7.\,1929 Rodaun@\textsc{Hofmannsthal, Hugo von} (1.\,2.\,1874 Wien – 15.\,7.\,1929 Rodaun), \emph{Schriftsteller}|pw}} »entdeckt«}{\lemma{\textnormal{\emph{Welsberg … »entdeckt«}}}\Cendnote{\textnormal{Hugo von Hofmannsthal\pwindex{Hofmannsthal, Hugo von 1.\,2.\,1874 Wien – 15.\,7.\,1929 Rodaun@\textsc{Hofmannsthal, Hugo von} (1.\,2.\,1874 Wien – 15.\,7.\,1929 Rodaun), \emph{Schriftsteller}|pwk} reiste am 4. 7. 1902 gemeinsam
                  mit Schnitzler nach Welsberg\oindex{Welsberg-Taisten@\textbf{Welsberg-Taisten}, \emph{Verwaltungsgebiet}|pwk} und blieb nach Schnitzlers Abreise ein paar Tage länger (siehe XXXX Auszeichnungsfehler: Dokument L01229 nicht gefunden).}}}\label{K_L03532-3} worden iſt,
               thut mir leid. Es wird jetzt ein literariſcher Ort werden – obwohl \strikeout{es}{ }{\pb}es doch ein beſſeres Schickſal verdient hätte.\pend
           
\pstart
           Meine Mutter\pwindex{Goldmann, Clementine 15.\,5.\,1842 Breslau – 24.\,2.\,1924 Frankfurt am Main@\textsc{Goldmann, Clementine} (15.\,5.\,1842 Breslau – 24.\,2.\,1924 Frankfurt am Main)|pwv} hat{ }ſich{ }ſehr
               über Ihre und \textsc{Liesls\pwindex{Steinrück, Elisabeth 19.\,11.\,1885 – 7.\,4.\,1920 Partenkirchen@\textsc{Steinrück, Elisabeth} (19.\,11.\,1885 – 7.\,4.\,1920 Partenkirchen)|pw}} Grüße gefreut und erwidert{ }ſie
               auf das Herzlichſte.\pend
           
\pstart
           Bitte, grüßen Sie meinen lieben \textsc{Arthur}, wenn er \label{K_L03532-4v}\edtext{morgen}{\lemma{\textnormal{\emph{morgen}}}\Cendnote{\textnormal{Goldmann\pwindex{Goldmann, Paul 31.\,1.\,1865 Breslau – 25.\,9.\,1935 Wien@\textsc{Goldmann, Paul} (31.\,1.\,1865 Breslau – 25.\,9.\,1935 Wien), \emph{Schriftsteller, Journalist}|pwk} war nicht auf dem aktuellen Stand, Schnitzler war bereits seit 8. 7. 1902 wieder in
                     Wien\oindex{Wien@\textbf{Wien}, \emph{Verwaltungsgebiet}|pwk}.}}}\label{K_L03532-4} zurückkommt, vielmals von mir.
               Ich \strikeout{\textcolor{gray}{bed}} danke ihm für{ }ſeine Karten von unterwegs und hoffe, bald Ausführlicheres von
               ihm zu hören.\pend
           
\pstart
           Wenn Ihnen der blöde Fratz\pwindex{Steinrück, Elisabeth 19.\,11.\,1885 – 7.\,4.\,1920 Partenkirchen@\textsc{Steinrück, Elisabeth} (19.\,11.\,1885 – 7.\,4.\,1920 Partenkirchen)|pwv}
               (ich meine natürlich \textsc{Liesl\pwindex{Steinrück, Elisabeth 19.\,11.\,1885 – 7.\,4.\,1920 Partenkirchen@\textsc{Steinrück, Elisabeth} (19.\,11.\,1885 – 7.\,4.\,1920 Partenkirchen)|pw}}) erzählt hat, daß ich über Sie »geſchimpft« habe,{ }ſo hat{ }ſie wieder einmal {\pb}geſprochen, was ſie\pwindex{Steinrück, Elisabeth 19.\,11.\,1885 – 7.\,4.\,1920 Partenkirchen@\textsc{Steinrück, Elisabeth} (19.\,11.\,1885 – 7.\,4.\,1920 Partenkirchen)|pwv} nicht verantworten kann. Ich habe ihr nur geſagt (weil{ }ſie
               mir durch Äußerungen und Verhalten dazu Anlaß gegeben hatte), was ich auch Ihnen{ }ſchon geſagt habe: wie wenig Sie Beide\pwindex{Steinrück, Elisabeth 19.\,11.\,1885 – 7.\,4.\,1920 Partenkirchen@\textsc{Steinrück, Elisabeth} (19.\,11.\,1885 – 7.\,4.\,1920 Partenkirchen)|pwv} mich verſtehen und wie{ }ſehr es \strikeout{mich}
               mir leid thut, daß ich gerade i\substVorne{}\textsuperscript{m}\substDazwischen{}n\substHinten{} einem Kreiſe, dem ich{ }ſo nahe{ }ſtehe,{ }ſo wenig Verſtändniß finde. An Ihrer
               freundſchaftlichen Geſinnung für mich zweifle ich keinen Augenblick, ebenſo wie Sie
               hoffentlich nicht an der meinigen zweifeln. Das Wort »Haß«{ }ſollte in einem Briefe,
               den Sie mir{ }ſchreiben, wirklich nicht{ }ſtehen.\pend
           
\pstart
           {\pb}Es thut mir leid, daß ich nicht \label{K_L03532-5v}\edtext{auch Ihnen zu einem Engagement an einem Berlin\oindex{Berlin@\textbf{Berlin}, \emph{Hauptstadt}|pw}er Theater verhelfen}{\lemma{\textnormal{\emph{auch … verhelfen}}}\Cendnote{\textnormal{Bezug auf Elisabeth
                     Gussmanns\pwindex{Steinrück, Elisabeth 19.\,11.\,1885 – 7.\,4.\,1920 Partenkirchen@\textsc{Steinrück, Elisabeth} (19.\,11.\,1885 – 7.\,4.\,1920 Partenkirchen)|pwk} Engagement am \emph{Schiller-Theater}\orgindex{Schiller-Theater@Schiller-Theater|pwk} ab dem 1. 9. 1902, siehe XXXX Auszeichnungsfehler: Dokument L03211 nicht gefunden.}}}\label{K_L03532-5} kann; aber ich \strikeout{\textcolor{gray}{d}} denke mir, daß Sie \label{K_L03532-6v}\edtext{Beſſeres
                  gefunden}{\lemma{\textnormal{\emph{Besseres
                  gefunden}}}\Cendnote{\textnormal{Er meint, die Rolle als Schnitzlers Partnerin und Mutter des
                  gemeinsamen Sohnes Heinrich\pwindex{Schnitzler, Heinrich 9.\,8.\,1902 Hinterbrühl – 12.\,7.\,1982 Wien@\textsc{Schnitzler, Heinrich} (9.\,8.\,1902 Hinterbrühl – 12.\,7.\,1982 Wien), \emph{Regisseur, Schauspieler}|pwk}, dessen Geburt
                  bevorstand, sei wichtiger als ihre Karriere.}}}\label{K_L03532-6} haben, als Ihnen die größte
               Stellung an der größten Bühne jemals hätte bieten können.\pend
           
\pstart
           Mit herzlichen Grüßen an Sie und \textsc{Liesl\pwindex{Steinrück, Elisabeth 19.\,11.\,1885 – 7.\,4.\,1920 Partenkirchen@\textsc{Steinrück, Elisabeth} (19.\,11.\,1885 – 7.\,4.\,1920 Partenkirchen)|pw}} (der ich für ihren Brief danke) bin ich {\\[\baselineskip]}Ihr ergebener {\\[\baselineskip]}\spacefill\mbox{Dr. Paul Goldmann.}\pend
           \leftskip=0em{}\selectlanguage{ngerman}\endnumbering\briefempfaengerindex{Schnitzler, Olga@\textsc{Schnitzler, Olga}!zzzGoldmann, Paul@\emph{von Paul Goldmann}!1902-07-092@{9. 7. [1902?]}|)be}\mylabel{L03532h}  \newcommand{\dateiname}{L03532}\newcommand{\titel}{Paul Goldmann an Olga Gussmann, 9. 7. [1902?]}\newcommand{\editorInnen}{Martin Anton Müller und Laura Untner}%% latex-leseansicht-abspann.tex
%% Abspann für die Leseansicht.
%% Der Schalter \ifkorrekturansicht ist bereits durch den Vorspann gesetzt.

%% latex-abspann.tex
%% Gemeinsamer Abspann für Korrekturansicht und Leseansicht.
%% Setzt den Schalter \ifkorrekturansicht voraus (gesetzt in den
%% einbindenden Dateien latex-korrekturansicht-abspann.tex bzw.
%% latex-leseansicht-abspann.tex).
%% ---------------------------------------------------------------

\normalsize

% Das esempio-Environment wird nur in der Leseansicht benötigt
\ifkorrekturansicht\else
\newenvironment{esempio}[3]%
{
    \vspace{1.5ex}
    \rlap{\underline{#1}}
    \par
    \setlength{\parindent}{0cm}
    \nopagebreak
    \leftskip=#2cm
    \rightskip=#3cm
}
{
    \par
}
\fi

\doendnotes{C}
\bigskip
\vfill

\clearpage

\footnotesize

\ifkorrekturansicht
  \lohead{\textsc{register}}
\fi

% theindex-Environment neu definieren ohne reledmac
\makeatletter
\renewenvironment{theindex}{%
  \ifkorrekturansicht
    \section*{\indexname}%
  \else
    \subsubsection*{Index der erwähnten Entitäten}%
  \fi
  \setlength{\parindent}{0pt}%
  \setlength{\parskip}{0pt plus 0.3pt}%
  \let\item\@idxitem
}{%
  \ifkorrekturansicht\clearpage\fi
}
\makeatother

\IfFileExists{\jobname-pw.ind}{\input{\jobname-pw.ind}}{}

% Quellenangabe nur in der Leseansicht
\ifkorrekturansicht\else
% Fallback-Definitionen, falls die .tex-Datei \titel etc. nicht gesetzt hat
\providecommand{\titel}{}
\providecommand{\editorInnen}{}
\providecommand{\dateiname}{\jobname}

\vspace{3cm}

\vfill

\footnotesize
\textsc{Quelle}: \titel. Herausgegeben von {\editorInnen}. In: \emph{Arthur Schnitzler: Briefwechsel mit Autorinnen und Autoren}.
 Digitale Edition, https://schnitzler-briefe.acdh.oeaw.ac.at/{\dateiname}.html (Stand \today)
\fi

\end{document}


