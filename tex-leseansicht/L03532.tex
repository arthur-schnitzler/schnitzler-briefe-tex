%% latex-korrekturansicht-vorspann.tex
%% Vorspann für die Korrekturansicht.
%% Lädt die gemeinsame Datei latex-vorspann.tex mit gesetztem Schalter.

\newif\ifkorrekturansicht
\korrekturansichttrue

\input{../tex-inputs/latex-vorspann}


\section[ Paul Goldmann an Olga Gussmann, 9. 7. {[}1902?{]}]{L03532 Paul Goldmann an Olga Gussmann, 9. 7. {[}1902?{]}}
\nopagebreak\mylabel{L03532v}
\rehead{ }\normalsize\beginnumbering\briefempfaengerindex{Schnitzler, Olga@\textsc{Schnitzler, Olga}!zzzGoldmann, Paul@\emph{von Paul Goldmann}!1902-07-092@{9. 7. {[}1902?{]}}|(be}
\toendnotes[C]{\smallbreak\pagebreak[2]}\Standort{DLA, A:Schnitzler, HS.NZ85.1.5247.}
\physDesc{Brief, 1 Blatt, 4 Seiten, 1819 Zeichen
\newline{}Handschrift: blaue Tinte, deutsche Kurrent}\toendnotes[C]{\smallbreak}
\pstart
           \raggedleft{}{\pb}\textcolor{gray}{\textbf{DESSAUERSTRASSE 19\oindex{Dessauer Strasse@\textbf{Dessauer Straße}, \emph{Straße (K.STR)}|pw}}}\pend
           
\pstart
           Berlin\oindex{Berlin@\textbf{Berlin}, \emph{P.PPLC}|pw}, 9. Juli.\pend
           
\pstart\center{}Liebe Freundin,\pend\vspace{0.5em}
\pstart
           Bitte, laſſen Sie das Danken ſein. Das war doch Alles ſelbſtverſtändlich. Es iſt noch
               die erſte und einfachſte Pflicht der Freundſchaft, in wichtigen Lebensangelegenheiten
                  \label{K_L03532-1v}\edtext{Beiſtand}{\lemma{\textnormal{\emph{Beiſtand}}}\Cendnote{\textnormal{Siehe Paul Goldmann an Arthur Schnitzler, 16. 6. [1902].
               }}}\label{K_L03532-1} zu leiſten.\pend
           
\pstart
           Ihre lieben \label{K_L03532-2v}\edtext{Mittheilungen über 
               \textsc{Peter Dorner}\pwindex{Dorner, Peter 17.02.1857 – 01.04.1931@\textsc{Dorner, Peter} (17.02.1857 – 01.04.1931), \emph{Schmied/Schmiedin, Kunsthandwerker/Kunsthandwerkerin, Kunstschmied/Kunstschmiedin}|pw}}{\lemma{\textnormal{\emph{Mittheilungen … Dorner}}}\Cendnote{\textnormal{Arthur Schnitzler hatte
                  den Kunstschmied am 4. 7. 1902 in dessen Atelier aufgesucht.}}}\label{K_L03532-2}{ }\textsc{etc.} haben
               mich ſehr intereſſirt. Nur hätte ich gern auch etwas Näheres über Ihr Ergehen
               gehört.\pend
           
\pstart
           Daß unſer liebes \label{K_L03532-3v}\edtext{\textsc{Welsberg\oindex{Welsberg-Taisten@\textbf{Welsberg-Taisten}, \emph{A.ADM3}|pw}} von \textsc{Hoffmannsthal\pwindex{Hofmannsthal, Hugo von 1874-02-01 – 1929-07-15@\textsc{Hofmannsthal, Hugo von} (1874-02-01 – 1929-07-15), \emph{Schriftsteller/Schriftstellerin}|pw}} »entdeckt«}{\lemma{\textnormal{\emph{Welsberg … »entdeckt«}}}\Cendnote{\textnormal{Hugo von Hofmannsthal\pwindex{Hofmannsthal, Hugo von 1874-02-01 – 1929-07-15@\textsc{Hofmannsthal, Hugo von} (1874-02-01 – 1929-07-15), \emph{Schriftsteller/Schriftstellerin}|pwk} reiste am 4. 7. 1902 gemeinsam
                  mit Schnitzler nach Welsberg\oindex{Welsberg-Taisten@\textbf{Welsberg-Taisten}, \emph{A.ADM3}|pwk} und blieb nach Schnitzlers Abreise ein paar Tage länger (siehe Arthur Schnitzler an Hermann Bahr, [9. 7. 1902]).}}}\label{K_L03532-3} worden iſt,
               thut mir leid. Es wird jetzt ein literariſcher Ort werden – obwohl \strikeout{es}{ }{\pb}es doch ein beſſeres Schickſal verdient hätte.\pend
           
\pstart
           Meine Mutter\pwindex{Goldmann, Clementine 1842-05-15 – 1924-02-24@\textsc{Goldmann, Clementine} (1842-05-15 – 1924-02-24)|pwv} hat ſich ſehr
               über Ihre und \textsc{Liesls\pwindex{Steinrueck, Elisabeth 19.11.1885 – 07.04.1920@\textsc{Steinrück, Elisabeth} (19.11.1885 – 07.04.1920)|pw}} Grüße gefreut und erwidert ſie
               auf das Herzlichſte.\pend
           
\pstart
           Bitte, grüßen Sie meinen lieben \textsc{Arthur}, wenn er \label{K_L03532-4v}\edtext{morgen}{\lemma{\textnormal{\emph{morgen}}}\Cendnote{\textnormal{Goldmann\pwindex{Goldmann, Paul 31.01.1865 – 25.09.1935@\textsc{Goldmann, Paul} (31.01.1865 – 25.09.1935), \emph{Schriftsteller/Schriftstellerin, Journalist/Journalistin}|pwk} war nicht auf dem aktuellen Stand, Schnitzler war bereits seit 8. 7. 1902 wieder in
                     Wien\oindex{Wien@\textbf{Wien}, \emph{A.ADM2}|pwk}.}}}\label{K_L03532-4} zurückkommt, vielmals von mir.
               Ich \strikeout{\textcolor{gray}{bed}} danke ihm für ſeine Karten von unterwegs und hoffe, bald Ausführlicheres von
               ihm zu hören.\pend
           
\pstart
           Wenn Ihnen der blöde Fratz\pwindex{Steinrueck, Elisabeth 19.11.1885 – 07.04.1920@\textsc{Steinrück, Elisabeth} (19.11.1885 – 07.04.1920)|pwv}
               (ich meine natürlich \textsc{Liesl\pwindex{Steinrueck, Elisabeth 19.11.1885 – 07.04.1920@\textsc{Steinrück, Elisabeth} (19.11.1885 – 07.04.1920)|pw}}) erzählt hat, daß ich über Sie »geſchimpft« habe, ſo hat ſie wieder einmal {\pb}geſprochen, was ſie\pwindex{Steinrueck, Elisabeth 19.11.1885 – 07.04.1920@\textsc{Steinrück, Elisabeth} (19.11.1885 – 07.04.1920)|pwv} nicht verantworten kann. Ich habe ihr nur geſagt (weil ſie
               mir durch Äußerungen und Verhalten dazu Anlaß gegeben hatte), was ich auch Ihnen
               ſchon geſagt habe: wie wenig Sie Beide\pwindex{Steinrueck, Elisabeth 19.11.1885 – 07.04.1920@\textsc{Steinrück, Elisabeth} (19.11.1885 – 07.04.1920)|pwv} mich verſtehen und wie ſehr es \strikeout{mich}
               mir leid thut, daß ich gerade i\substVorne{}\textsuperscript{m}\substDazwischen{}n\substHinten{} einem Kreiſe, dem ich ſo nahe ſtehe, ſo wenig Verſtändniß finde. An Ihrer
               freundſchaftlichen Geſinnung für mich zweifle ich keinen Augenblick, ebenſo wie Sie
               hoffentlich nicht an der meinigen zweifeln. Das Wort »Haß« ſollte in einem Briefe,
               den Sie mir ſchreiben, wirklich nicht ſtehen.\pend
           
\pstart
           {\pb}Es thut mir leid, daß ich nicht \label{K_L03532-5v}\edtext{auch Ihnen zu einem Engagement an einem Berlin\oindex{Berlin@\textbf{Berlin}, \emph{P.PPLC}|pw}er Theater verhelfen}{\lemma{\textnormal{\emph{auch … verhelfen}}}\Cendnote{\textnormal{Bezug auf Elisabeth
                     Gussmanns\pwindex{Steinrueck, Elisabeth 19.11.1885 – 07.04.1920@\textsc{Steinrück, Elisabeth} (19.11.1885 – 07.04.1920)|pwk} Engagement am \emph{Schiller-Theater}\orgindex{Schiller-Theater@Schiller-Theater|pwk} ab dem 1. 9. 1902, siehe Paul Goldmann an Arthur Schnitzler, 16. 6. [1902].}}}\label{K_L03532-5} kann; aber ich \strikeout{\textcolor{gray}{d}} denke mir, daß Sie \label{K_L03532-6v}\edtext{Beſſeres
                  gefunden}{\lemma{\textnormal{\emph{Beſſeres
                  gefunden}}}\Cendnote{\textnormal{Er meint, die Rolle als Schnitzlers Partnerin und Mutter des
                  gemeinsamen Sohnes Heinrich\pwindex{Schnitzler, Heinrich 09.08.1902 – 12.07.1982@\textsc{Schnitzler, Heinrich} (09.08.1902 – 12.07.1982), \emph{Regisseur/Regisseurin, Schauspieler/Schauspielerin}|pwk}, dessen Geburt
                  bevorstand, sei wichtiger als ihre Karriere.}}}\label{K_L03532-6} haben, als Ihnen die größte
               Stellung an der größten Bühne jemals hätte bieten können.\pend
           
\pstart
           Mit herzlichen Grüßen an Sie und \textsc{Liesl\pwindex{Steinrueck, Elisabeth 19.11.1885 – 07.04.1920@\textsc{Steinrück, Elisabeth} (19.11.1885 – 07.04.1920)|pw}} (der ich für ihren Brief danke) bin ich {\\[\baselineskip]}Ihr ergebener {\\[\baselineskip]}\spacefill\mbox{Dr. Paul Goldmann.}\pend
           \leftskip=0em{}\selectlanguage{ngerman}\endnumbering\briefempfaengerindex{Schnitzler, Olga@\textsc{Schnitzler, Olga}!zzzGoldmann, Paul@\emph{von Paul Goldmann}!1902-07-092@{9. 7. {[}1902?{]}}|)be}\mylabel{L03532h}  \normalsize

\doendnotes{C}
\bigskip
\vfill

\clearpage

\footnotesize

\lohead{\textsc{register}}

% Definiere theindex-Environment komplett neu ohne reledmac
\makeatletter
\renewenvironment{theindex}{%
  \section*{\indexname}%
  \setlength{\parindent}{0pt}%
  \setlength{\parskip}{0pt plus 0.3pt}%
  \let\item\@idxitem
}{%
  \clearpage
}
\makeatother

\IfFileExists{\jobname-pw.ind}{\input{\jobname-pw.ind}}{}

\end{document}

      