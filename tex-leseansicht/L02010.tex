%% latex-korrekturansicht-vorspann.tex
%% Vorspann für die Korrekturansicht.
%% Lädt die gemeinsame Datei latex-vorspann.tex mit gesetztem Schalter.

\newif\ifkorrekturansicht
\korrekturansichttrue

\input{../tex-inputs/latex-vorspann}


\section[Arthur Schnitzler an Robert Adam, 11. 2. 1911]{L02010 Arthur Schnitzler an Robert Adam, 11. 2. 1911}
\nopagebreak\mylabel{L02010v}
\rehead{ }\normalsize\beginnumbering\briefempfaengerindex{Adam, Robert@\textsc{Adam, Robert}!zzzSchnitzler, Arthur@\emph{von Arthur Schnitzler}!1911-02-111@{11. 2. 1911}|(be}
\toendnotes[C]{\smallbreak\pagebreak[2]}\Standort{DLA, 96.34.1/5.}
\physDesc{Brief, 1 Blatt, 1 Seite, Umschlag, 585 Zeichen
\newline{}Schreibmaschine
\newline{}Handschrift: schwarze Tinte (\noindent{}Unterschrift)
\newline{}Versand: Stempel: »\nobreak{}Wien\nobreak{}«.  }\Standort{DLA, A:Schnitzler, 85.1.1621.}
\physDesc{Brief, Durchschlag1 Blatt, 1 Seite, Umschlag, 585 Zeichen
\newline{}Schreibmaschine
\newline{}Handschrift: roter Buntstift, lateinische Kurrent (\noindent{}Beschriftung »Adam«)}\toendnotes[C]{\smallbreak}\pstart{}{\pb}\textcolor{gray}{\textbf{Dr. Arthur Schnitzler}}\pend{}\pstart{}\textcolor{gray}{\textbf{Wien, XVIII. Sternwartestrasse 71}}\oindex{Sternwartestrasse 71@\textbf{Sternwartestraße 71}, \emph{Wohngebäude (K.WHS)}|pw}\pend{}{\bigskip}\pstart{}{\pb}Herrn Robert Adam\pend{}\pstart{}\so{Wien XII}\oindex{XII., Meidling@\textbf{XII., Meidling}, \emph{A.ADM3}|pw}.\pend{}\pstart{}Meidlinger Hauptstraße 56\oindex{Meidlinger Hauptstrasse@\textbf{Meidlinger Hauptstraße}, \emph{Straße (K.STR)}|pw}.\pend{}{\bigskip}\vspace{1em}
\pstart
           
\pstart
           {\pb}\textcolor{gray}{\textbf{Dr. Arthur Schnitzler}}\pend
           
\pstart
           \raggedleft{}11. 2. 1911.\pend
           \pend
           
\pstart
           \textcolor{gray}{\textbf{Wien XVIII. Sternwartestrasse 71\oindex{Sternwartestrasse 71@\textbf{Sternwartestraße 71}, \emph{Wohngebäude (K.WHS)}|pw}}}\pend
           
\pstart\center{}Sehr geehrter Herr Adam.\pend\vspace{0.5em}
\pstart
           Es tut mir leid, dass Ihnen bei S. Fischer\orgindex{S. Fischer Verlag@S. Fischer Verlag|pw} kein
               Erfolg beschieden war. Ob ein weiteres Herumschicken des Manuscriptes\pwindex{Neidhard@\emph{Neidhard}|pwv} an Verleger Ihre Sache fördern
               könnte, ist schwer zu entscheiden. Von der Wertlosigkeit meiner Empfehlung haben Sie
               sich wohl überzeugt. Versuche einzelne Szenen bei Zeitschriften unterzubringen,
               sollten Sie keineswegs unterlassen. Hier kämen meines Erachtens »Merker\orgindex{Merker@Der Merker|pw}« und »Schaubühne\orgindex{Schaubuehne / Die Weltbuehne@Die Schaubühne / Die Weltbühne|pw}« vor
               allem in Betracht.\pend
           
\pstart
           Mit verbindlichen Grüssen{\\[\baselineskip]}Ihr ergebener{\\[\baselineskip]}\spacefill\mbox{{[}hs.:{]} ArthSchnitzler}\pend
           \leftskip=0em{}
\pstart
           \noindent{}{[}ms.:{]} Herrn Robert Adam, Wien\oindex{Wien@\textbf{Wien}, \emph{A.ADM2}|pw}.\pend
           \selectlanguage{ngerman}\endnumbering\briefempfaengerindex{Adam, Robert@\textsc{Adam, Robert}!zzzSchnitzler, Arthur@\emph{von Arthur Schnitzler}!1911-02-111@{11. 2. 1911}|)be}\mylabel{L02010h}  \normalsize

\doendnotes{C}
\bigskip
\vfill

\clearpage

\footnotesize

\lohead{\textsc{register}}

% Definiere theindex-Environment komplett neu ohne reledmac
\makeatletter
\renewenvironment{theindex}{%
  \section*{\indexname}%
  \setlength{\parindent}{0pt}%
  \setlength{\parskip}{0pt plus 0.3pt}%
  \let\item\@idxitem
}{%
  \clearpage
}
\makeatother

\IfFileExists{\jobname-pw.ind}{\input{\jobname-pw.ind}}{}

\end{document}

      