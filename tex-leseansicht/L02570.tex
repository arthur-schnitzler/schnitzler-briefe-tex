%% latex-leseansicht-vorspann.tex
%% Vorspann für die Leseansicht.
%% Lädt die gemeinsame Datei latex-vorspann.tex mit nicht gesetztem Schalter.

\newif\ifkorrekturansicht
\korrekturansichtfalse

\input{../tex-inputs/latex-vorspann}


         
         \renewcommand{\erwaehntePersonen}{Personen: Johann Wolfgang von Goethe, Maximilian Herz, Julius von Hochenegg, Ferdinand Onno, Johann Schnitzler}
         \renewcommand{\erwaehnteInstitutionen}{Institutionen: Allgemeine Poliklinik}
         \renewcommand{\erwaehnteOrte}{Orte: Berlin, Wien}
         \renewcommand{\erwaehnteWerke}{Werke: An den Mond, Professor Bernhardi. Komödie in fünf Akten}
               \section[Therese Rie-Andro an Arthur Schnitzler, 29. 11. 1912]{ Therese Rie-Andro an Arthur Schnitzler, 29. 11. 1912}\nopagebreak\mylabel{v}\rehead{ }\begin{ledgroupsized}[t]{13cm}\normalsize\beginnumbering \toendnotes[C]{\smallbreak\pagebreak[2]} \Standort{DLA, A:Schnitzler, 85.1.4310.}
\physDesc{Brief, 1 Blatt, 3 Seiten, 2362 Zeichen
\newline{}Handschrift: blaue Tinte, lateinische Kurrent
\newline{}Schnitzler: 1) mit Bleistift beschriftet: »\textsc{Andro}«  2) mit rotem Buntstift mehrere Unterstreichungen}\toendnotes[C]{\smallbreak}\pstart
           \raggedleft{}{\pb}Wien\oindex{Wien@\textbf{Wien}|pw}, d. 29. Nov. 12.\pend
           \pstart{}Sehr geehrter Herr,\pend\pstart
           Sie haben mir vor einigen Monaten einen Brief geschrieben, der mich sehr sehr erfreut
               hat; dennoch würde ich Ihnen gewiß nicht schreiben, wenn ich nicht unter einem
               ungeheuer starken künstlerischen Eindruck stünde: es ist der »Professor Bernhardi\pwindex{Schnitzler, Arthur 15.05.1862 – 21.10.1931@\textsc{Schnitzler, Arthur} (15.05.1862 – 21.10.1931), \emph{Schriftsteller, Mediziner}!Professor Bernhardi. Komoedie in fuenf Akten1912@\strich\emph{Professor Bernhardi. Komödie in fünf Akten} {[}1912{]}|pw}«, den ich (durch dessen \label{K_L02570-1v}\edtext{Vorlesung}{\lemma{\textnormal{\emph{Vorlesung}}}\Cendnote{\textnormal{In Wien\oindex{Wien@\textbf{Wien}|pwk} wurde am
                     28. 11. 1912 – dem Tag der Berlin\oindex{Berlin@\textbf{Berlin}|pwk}er Uraufführung – eine Lesung durch Ferdinand Onno\pwindex{Onno, Ferdinand 19.10.1881 – 18.08.1969@\textsc{Onno, Ferdinand} (19.10.1881 – 18.08.1969), \emph{Schauspieler}|pwk} veranstaltet. Gelesen wurden der 1. Akt,
                  das Gespräch von Bernhardi und Flint im 2. Akt, der 3. Akt, das Gespräch von
                  Bernhardi und dem Pfarrer im 4. Akt und der 5. Akt. Um die Lücken zu überbrücken,
                  schrieb Schnitzler\pwindex{Schnitzler, Arthur 15.05.1862 – 21.10.1931@\textsc{Schnitzler, Arthur} (15.05.1862 – 21.10.1931), \emph{Schriftsteller, Mediziner}|pwk} kurze Verbindungstexte,
                  die im Nachlass als Durchschläge erhalten sind (\emph{Cambridge}, A 117,2, freundliche Auskunft von
                  Judith Beniston).}}}\label{K_L02570-1h}) kennen lernte. Sie werden ja jetzt soviel Schönes drüber
               hören und lesen, daß ich es wol kaum wagen kann, Ihnen etwas zu sagen; ich versuch’s
               auch gar nicht erſt. Aber diese in milder Heiterkeit sich lösende Tragödie des
               aufrechten Menschen, dieser wunderbar in Goethe\pwindex{Goethe, Johann Wolfgang von 1749-08-28 – 1832-03-22@\textsc{Goethe, Johann Wolfgang von} (1749-08-28 – 1832-03-22), \emph{Schriftsteller}|pw}’sche Sti{\geminationm}ung ausklingende Schluß: »Selig wer sich vor der Welt \uline{ohne Haß verschließt}\pwindex{Goethe, Johann Wolfgang von 1749-08-28 – 1832-03-22@\textsc{Goethe, Johann Wolfgang von} (1749-08-28 – 1832-03-22), \emph{Schriftsteller}!An den Mond1778@\strich\emph{An den Mond} {[}1778{]}|pwv}« – {\pb}die gehen mir selbst in diesen trüben
               ahnungsschweren Kriegszeiten i{\geminationm}er noch nach.\pend
           \pstart
           Aber noch anderes war es mir und mehr: die Erläuterung längst entschwundener
               Kindheitserlebnisse, das Emportauchen von damals kaum begriffenen und doch erfaßten
               Dingen. Mein Vater\pwindex{Herz, Maximilian 1837-07-05 – 1890-07-13@\textsc{Herz, Maximilian} (1837-07-05 – 1890-07-13), \emph{Mediziner}|pwv} war
               Abteilungsvorstand an der Poliklinik\orgindex{Allgemeine Poliklinik@Allgemeine Poliklinik|pw}, als Ihr Vater\pwindex{Schnitzler, Johann 10.04.1835 – 02.05.1893@\textsc{Schnitzler, Johann} (10.04.1835 – 02.05.1893), \emph{Mediziner}|pwv} (den ich gekannt und
               geliebt habe) Direktor war. Oft iſt er heiß und erregt nach Hause geko{\geminationm}en, hat vor mir, dem kleinen Kinde, auf das niemand
               achtete, gesprochen. Es war ein Kampf, den die rechtlichen Leute alle dort führten,
               vornehmlich gegen \label{K_L02570-2v}\edtext{Einen\pwindex{Hochenegg, Julius von 02.08.1859 – 11.05.1940@\textsc{Hochenegg, Julius von} (02.08.1859 – 11.05.1940), \emph{Mediziner}|pwuv}}{\lemma{\textnormal{\emph{Einen}}}\Cendnote{\textnormal{Sie könnte Julius Hochenegg\pwindex{Hochenegg, Julius von 02.08.1859 – 11.05.1940@\textsc{Hochenegg, Julius von} (02.08.1859 – 11.05.1940), \emph{Mediziner}|pwk} meinen, den Schnitzler\pwindex{Schnitzler, Arthur 15.05.1862 – 21.10.1931@\textsc{Schnitzler, Arthur} (15.05.1862 – 21.10.1931), \emph{Schriftsteller, Mediziner}|pwk} als Vorlage für die Figur des Professor Ebenwald\pwindex{Schnitzler, Arthur 15.05.1862 – 21.10.1931@\textsc{Schnitzler, Arthur} (15.05.1862 – 21.10.1931), \emph{Schriftsteller, Mediziner}!Professor Bernhardi. Komoedie in fuenf Akten1912@\strich\emph{Professor Bernhardi. Komödie in fünf Akten} {[}1912{]}|pwkv} verwendet hat (vgl. A. S.: \emph{Tagebuch}, 23. 10. 1922).}}}\label{K_L02570-2h} führten,
               der, glaube ich, \introOben{}leider\introOben{} Vize-Direktor war. Ich weiß, daß Ihr
                  Stück\pwindex{Schnitzler, Arthur 15.05.1862 – 21.10.1931@\textsc{Schnitzler, Arthur} (15.05.1862 – 21.10.1931), \emph{Schriftsteller, Mediziner}!Professor Bernhardi. Komoedie in fuenf Akten1912@\strich\emph{Professor Bernhardi. Komödie in fünf Akten} {[}1912{]}|pwv} nicht an Geschehnisse
               anknüpft, aber an innere Erlebnisse, an Sti{\geminationm}ungen, die
               damals in der Luft gelegen haben müßen und ich kann Ihnen nicht beschreiben, wie es
               mich durchschauert hat, als ich diese Atmosphäre emportauchen fühlte, in der mein Vater\pwindex{Herz, Maximilian 1837-07-05 – 1890-07-13@\textsc{Herz, Maximilian} (1837-07-05 – 1890-07-13), \emph{Mediziner}|pwv} (er starb
                  1890, als ich noch ein Kind war) gelebt hat, mitgekämpft und
               mitgelitten hat. Obgleich er in Ihrem Stück\pwindex{Schnitzler, Arthur 15.05.1862 – 21.10.1931@\textsc{Schnitzler, Arthur} (15.05.1862 – 21.10.1931), \emph{Schriftsteller, Mediziner}!Professor Bernhardi. Komoedie in fuenf Akten1912@\strich\emph{Professor Bernhardi. Komödie in fünf Akten} {[}1912{]}|pwv}{ }\strikeout{sicherlich} nicht »vorko{\geminationm}t« (um den banalen Ausdruck der Leute zu gebrauchen, die dem dichterischen Schaffen
               ganz ferne stehen) war es mir einen Augenblick, als wäre mir etwas von ihm, an dem
               ich mit meiner ganzen Kinderleidenschaft hing, zurückgekehrt: so sehr hat Ihr Stück\pwindex{Schnitzler, Arthur 15.05.1862 – 21.10.1931@\textsc{Schnitzler, Arthur} (15.05.1862 – 21.10.1931), \emph{Schriftsteller, Mediziner}!Professor Bernhardi. Komoedie in fuenf Akten1912@\strich\emph{Professor Bernhardi. Komödie in fünf Akten} {[}1912{]}|pwv} das Schicksal des Arztes
               ins Typische erhöht, stilisiert. Und darum müßen Sie begreifen, wie sehr ergriffen
               ich von Ihrem Stück\pwindex{Schnitzler, Arthur 15.05.1862 – 21.10.1931@\textsc{Schnitzler, Arthur} (15.05.1862 – 21.10.1931), \emph{Schriftsteller, Mediziner}!Professor Bernhardi. Komoedie in fuenf Akten1912@\strich\emph{Professor Bernhardi. Komödie in fünf Akten} {[}1912{]}|pwv} war, wie
               ich {\pb}es mit der ganz tiefen Dankbarkeit in mich aufgeno{\geminationm}en habe, als sei mir ein unbekanntes Stück meines
               eigenen Lebens gedeutet worden. Und deshalb sind Sie mir, verehrter Herr Doctor, auch
               nicht böse, wenn ich – ungeru\textcolor{gray}{f}en, und still wieder gehend – ko{\geminationm}e, um Ihnen das zu sagen!\pend
           \pstart \spacefill\mbox{L. Andro.}\pend{}
         
         \endnumbering\mylabel{h}\end{ledgroupsized}  \newcommand{\dateiname}{L02570}\newcommand{\titel}{Therese Rie-Andro an Arthur Schnitzler, 29. 11. 1912}\newcommand{\editorInnen}{Martin Anton Müller und Gerd-Hermann Susen}%% latex-leseansicht-abspann.tex
%% Abspann für die Leseansicht.
%% Der Schalter \ifkorrekturansicht ist bereits durch den Vorspann gesetzt.

%% latex-abspann.tex
%% Gemeinsamer Abspann für Korrekturansicht und Leseansicht.
%% Setzt den Schalter \ifkorrekturansicht voraus (gesetzt in den
%% einbindenden Dateien latex-korrekturansicht-abspann.tex bzw.
%% latex-leseansicht-abspann.tex).
%% ---------------------------------------------------------------

\normalsize

% Das esempio-Environment wird nur in der Leseansicht benötigt
\ifkorrekturansicht\else
\newenvironment{esempio}[3]%
{
    \vspace{1.5ex}
    \rlap{\underline{#1}}
    \par
    \setlength{\parindent}{0cm}
    \nopagebreak
    \leftskip=#2cm
    \rightskip=#3cm
}
{
    \par
}
\fi

\doendnotes{C}
\bigskip
\vfill

\clearpage

\footnotesize

\ifkorrekturansicht
  \lohead{\textsc{register}}
\fi

% theindex-Environment neu definieren ohne reledmac
\makeatletter
\renewenvironment{theindex}{%
  \ifkorrekturansicht
    \section*{\indexname}%
  \else
    \subsubsection*{Index der erwähnten Entitäten}%
  \fi
  \setlength{\parindent}{0pt}%
  \setlength{\parskip}{0pt plus 0.3pt}%
  \let\item\@idxitem
}{%
  \ifkorrekturansicht\clearpage\fi
}
\makeatother

\IfFileExists{\jobname-pw.ind}{\input{\jobname-pw.ind}}{}

% Quellenangabe nur in der Leseansicht
\ifkorrekturansicht\else
% Fallback-Definitionen, falls die .tex-Datei \titel etc. nicht gesetzt hat
\providecommand{\titel}{}
\providecommand{\editorInnen}{}
\providecommand{\dateiname}{\jobname}

\vspace{3cm}

\vfill

\footnotesize
\textsc{Quelle}: \titel. Herausgegeben von {\editorInnen}. In: \emph{Arthur Schnitzler: Briefwechsel mit Autorinnen und Autoren}.
 Digitale Edition, https://schnitzler-briefe.acdh.oeaw.ac.at/{\dateiname}.html (Stand \today)
\fi

\end{document}


      