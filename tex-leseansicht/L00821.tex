%% latex-korrekturansicht-vorspann.tex
%% Vorspann für die Korrekturansicht.
%% Lädt die gemeinsame Datei latex-vorspann.tex mit gesetztem Schalter.

\newif\ifkorrekturansicht
\korrekturansichttrue

\input{../tex-inputs/latex-vorspann}


\section[Richard Beer-Hofmann an Arthur Schnitzler, 17. 7. 1898]{L00821 Richard Beer-Hofmann an Arthur Schnitzler, 17. 7. 1898}
\nopagebreak\mylabel{L00821v}
\rehead{ }\normalsize\beginnumbering\briefempfaengerindex{Schnitzler, Arthur@\textsc{Schnitzler, Arthur}!zzzBeer-Hofmann, Richard@\emph{von Richard Beer-Hofmann}!1898-07-171@{17. 7. 1898}|(be}
\toendnotes[C]{\smallbreak\pagebreak[2]}\Standort{CUL, Schnitzler, B 8.}
\physDesc{Brief, 1 Blatt, 3 Seiten, 588 Zeichen
\newline{}Handschrift: Bleistift, lateinische Kurrent
\newline{}Ordnung: mit Bleistift von unbekannter Hand nummeriert:
                                    »121« }
\buchAbdrucke{\weitereDrucke{Arthur Schnitzler, Richard Beer-Hofmann: \emph{Briefwechsel 1891–1931}. Wien, Zürich: \emph{Europaverlag} 1992, S. 123.} }
\pstart
           \raggedleft{}{\pb}Steindorf\oindex{Steindorf am Ossiacher See@\textbf{Steindorf am Ossiacher See}, \emph{A.ADM3}|pw}{ }Sonntag 17/VII 98\pend
           \vspace{0.5em}
\pstart
           Lieber Arthur! Brief aus Graz\oindex{Graz@\textbf{Graz}, \emph{A.ADM2}|pw}
               erhalten. Weiß noch gar nichts Besti{\geminationm}tes\pend
           
\pstart
           Hugo\pwindex{Hofmannsthal, Hugo von 1874-02-01 – 1929-07-15@\textsc{Hofmannsthal, Hugo von} (1874-02-01 – 1929-07-15), \emph{Schriftsteller/Schriftstellerin}|pw} will daß ich die 10 Tage mitmache, und
               dann mit ihm in Ober-Italien\oindex{Italien@\textbf{Italien}, \emph{A.PCLI}|pw}{ }{\pb}dh. an einem der Seen bleibe. Die
               10 Tage unwahrscheinlich. Eher das letztere nur wäre mir Venedig\oindex{Venedig@\textbf{Venedig}, \emph{P.PPLA}|pw} – Seebad lieber, da Venedig\oindex{Venedig@\textbf{Venedig}, \emph{P.PPLA}|pw}{ }\uline{6} Stunden die Seen mindestens 15–16 {\pb}Stunden weit sind\pend
           
\pstart
           Bitte geben Sie mir bis zum letzten Salzburg\oindex{Salzburg@\textbf{Salzburg}, \emph{A.ADM2}|pw}er Tag
                  i{\geminationm}er Nachricht wo Sie Brief oder Telegr.
               erreicht.\pend
           
\pstart
           Paula\pwindex{Beer-Hofmann, Paula 25.02.1879 – 30.10.1939@\textsc{Beer-Hofmann, Paula} (25.02.1879 – 30.10.1939)|pw} u. Mirjam\pwindex{Beer-Hofmann, Mirjam 04.09.1897 – 24.12.1984@\textsc{Beer-Hofmann, Mirjam} (04.09.1897 – 24.12.1984)|pw} dan{\pb}ken für d. Gruß
               u. erwiedern ihn. Mirjam\pwindex{Beer-Hofmann, Mirjam 04.09.1897 – 24.12.1984@\textsc{Beer-Hofmann, Mirjam} (04.09.1897 – 24.12.1984)|pw} freut sich riesig wenn
               ich ihr Ihre Briefe vorlese. Schreiben Sie also oft.\pend
           
\pstart
           Von Herzen{\\[\baselineskip]}\spacefill\mbox{Richard}\pend
           \leftskip=0em{}\selectlanguage{ngerman}\endnumbering\briefempfaengerindex{Schnitzler, Arthur@\textsc{Schnitzler, Arthur}!zzzBeer-Hofmann, Richard@\emph{von Richard Beer-Hofmann}!1898-07-171@{17. 7. 1898}|)be}\mylabel{L00821h}  \normalsize

\doendnotes{C}
\bigskip
\vfill

\clearpage

\footnotesize

\lohead{\textsc{register}}

% Definiere theindex-Environment komplett neu ohne reledmac
\makeatletter
\renewenvironment{theindex}{%
  \section*{\indexname}%
  \setlength{\parindent}{0pt}%
  \setlength{\parskip}{0pt plus 0.3pt}%
  \let\item\@idxitem
}{%
  \clearpage
}
\makeatother

\IfFileExists{\jobname-pw.ind}{\input{\jobname-pw.ind}}{}

\end{document}

      