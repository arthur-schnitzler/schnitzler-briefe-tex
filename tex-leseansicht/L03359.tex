%% latex-korrekturansicht-vorspann.tex
%% Vorspann für die Korrekturansicht.
%% Lädt die gemeinsame Datei latex-vorspann.tex mit gesetztem Schalter.

\newif\ifkorrekturansicht
\korrekturansichttrue

\input{../tex-inputs/latex-vorspann}


\section[ Arthur Schnitzler an Felix Salten, 12. 11. 1903]{L03359 Arthur Schnitzler an Felix Salten, 12. 11. 1903}
\nopagebreak\mylabel{L03359v}
\rehead{ }\normalsize\beginnumbering\briefempfaengerindex{Salten, Felix@\textsc{Salten, Felix}!zzzSchnitzler, Arthur@\emph{von Arthur Schnitzler}!1903-11-122@{12. 11. 1903}|(be}
\toendnotes[C]{\smallbreak\pagebreak[2]}\Standort{Wienbibliothek im Rathaus, Nachlass Salten, ZPH 1681, 17.3.11.11.40.2.}
\physDesc{Fotografie, 51 Zeichen
\newline{}Handschrift: schwarze Tinte, deutsche Kurrent
\newline{}Editorischer Hinweis: mit Bleistift auf der Fotografie die handschriftliche Signatur
                                 »\textsc{A Hertwig}\pwindex{Hertwig, Aura 06.06.1861 – 28.09.1944@\textsc{Hertwig, Aura} (06.06.1861 – 28.09.1944), \emph{Fotograf/Fotografin}|pw}« und »1903« }\toendnotes[C]{\smallbreak}\begin{figure}[H]\centering\includegraphics[width=10cm]{../tex-inputs/img/ZPH1681_Box_17_3_11_11_40_2_0001_1.jpg}\end{figure}\vspace{1em}
\pstart
           \noindent{}{\pb}Meinem lieben Felix Salten\pwindex{Arthur Schnitzler [Halbprofil 1903]@\emph{Arthur Schnitzler [Halbprofil 1903]}|pwv}\pend
           \pstart \spacefill\mbox{Arth Sch}\pend{}
\pstart
           Wien\oindex{Wien@\textbf{Wien}, \emph{A.ADM2}|pw}{ }12. 11. 903.\pend
           \selectlanguage{ngerman}\endnumbering\briefempfaengerindex{Salten, Felix@\textsc{Salten, Felix}!zzzSchnitzler, Arthur@\emph{von Arthur Schnitzler}!1903-11-122@{12. 11. 1903}|)be}\mylabel{L03359h}  \normalsize

\doendnotes{C}
\bigskip
\vfill

\clearpage

\footnotesize

\lohead{\textsc{register}}

% Definiere theindex-Environment komplett neu ohne reledmac
\makeatletter
\renewenvironment{theindex}{%
  \section*{\indexname}%
  \setlength{\parindent}{0pt}%
  \setlength{\parskip}{0pt plus 0.3pt}%
  \let\item\@idxitem
}{%
  \clearpage
}
\makeatother

\IfFileExists{\jobname-pw.ind}{\input{\jobname-pw.ind}}{}

\end{document}

      