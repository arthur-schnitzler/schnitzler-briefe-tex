%% latex-leseansicht-vorspann.tex
%% Vorspann für die Leseansicht.
%% Lädt die gemeinsame Datei latex-vorspann.tex mit nicht gesetztem Schalter.

\newif\ifkorrekturansicht
\korrekturansichtfalse

\input{../tex-inputs/latex-vorspann}


\section[Lou Andreas-Salomé an Arthur Schnitzler, 19. 2. 1906]{L01585 Lou Andreas-Salomé an Arthur Schnitzler, 19. 2. 1906}
\nopagebreak\mylabel{L01585v}
\rehead{ }\normalsize\beginnumbering\briefempfaengerindex{Schnitzler, Arthur@\textsc{Schnitzler, Arthur}!zzzAndreas-Salomé, Lou@\emph{von Lou Andreas-Salomé}!1906-02-191@{19. 2. 1906}|(be}
\toendnotes[C]{\smallbreak\pagebreak[2]}
\correspDesc{Versand  durch Lou Andreas-Salomé am 19. 2. 1906 in Berlin
\newline{}Erhalt  durch Arthur Schnitzler am 20. 2. 1906 in Berlin}\toendnotes[C]{\smallbreak}
\Standort{CUL, Schnitzler, B 3.}
\physDesc{Postkarte, 449 Zeichen
\newline{}Handschrift: schwarze Tinte, deutsche Kurrent
\newline{}Versand: 1) Stempel: »\nobreak{}\oindex{Berlin@\textbf{Berlin}, \emph{Hauptstadt}|pwk}Berlin W 50, 19.\,2.\,6, 9 10 N\nobreak{}«.   2) Stempel: »\nobreak{}20.\,2.\,6\nobreak{}«. 
\newline{}Schnitzler: 1) mit Bleistift datiert: »19/2 06«  2) mit rotem Buntstift eine Unterstreichung
\newline{}Ordnung: mit Bleistift von unbekannter Hand nummeriert:
                                            »20« }\toendnotes[C]{\smallbreak}\pstart{}{\pb}Herrn \textsc{D\textsuperscript{r} Arthur Schnitzler}\pend{}\pstart{}\textsc{Berlin C.}\pend{}\pstart{}\textsc{Hôtel Continental\oindex{Hotel Continental [Berlin]@\textbf{Hotel Continental [Berlin]}, \emph{Hotel}|pw}.}\pend{}{\bigskip}\vspace{1em}
\pstart
           \noindent{}{\pb}Lieber Doktor \textsc{Schnitzler}, darf ich
                    Sie um die Erlaubniß bitten, am \label{K_L01585-1v}\edtext{Freitag}{\lemma{\textnormal{\emph{Freitag}}}\Cendnote{\textnormal{Die Generalprobe\eventindex{Lessing-Theater@\textbf{Lessing-Theater}!Generalprobe von Der Ruf des Lebens, 23.2.1906@Generalprobe von Der Ruf des Lebens, 23.2.1906|pwk} von
                            \emph{Der Ruf des Lebens}\pwindex{Schnitzler, Arthur 15.\,5.\,1862 Wien – 21.\,10.\,1931 ebd.@\textsc{Schnitzler, Arthur} (15.\,5.\,1862 Wien – 21.\,10.\,1931 ebd.), \emph{Schriftsteller, Mediziner}!Ruf des Lebens. Schauspiel in drei Akten@\strich\emph{Der Ruf des Lebens. Schauspiel in drei Akten}|pwk} am \emph{Lessing-Theater in Berlin}\orgindex{Lessing-Theater@Lessing-Theater|pwk} fand am
                        Freitag, den 23. 2. 1906 statt. Ob Andreas-Salomé\pwindex{Andreas-Salomé, Lou 12.\,2.\,1861 Sankt Petersburg – 5.\,2.\,1937 Göttingen@\textsc{Andreas-Salomé, Lou} (12.\,2.\,1861 Sankt Petersburg – 5.\,2.\,1937 Göttingen), \emph{Schriftstellerin}|pwk} teilnahm, ist nicht nachgewiesen. Die Uraufführung\eventindex{Lessing-Theater@\textbf{Lessing-Theater}!Premiere von Der Ruf des Lebens, 24.2.1906@Premiere von Der Ruf des Lebens, 24.2.1906|pwkv}
                        fand am 24. 2. 1906, dem Folgetag, statt. Hier erwähnt Schnitzler ihre Anwesenheit nach der
                        Veranstaltung im \emph{Tagebuch}\pwindex{Schnitzler, Arthur 15.\,5.\,1862 Wien – 21.\,10.\,1931 ebd.@\textsc{Schnitzler, Arthur} (15.\,5.\,1862 Wien – 21.\,10.\,1931 ebd.), \emph{Schriftsteller, Mediziner}!Tagebuch@\strich\emph{Tagebuch}|pwk}.}}}\label{K_L01585-1} der
                        Generalprobe\pwindex{Schnitzler, Arthur 15.\,5.\,1862 Wien – 21.\,10.\,1931 ebd.@\textsc{Schnitzler, Arthur} (15.\,5.\,1862 Wien – 21.\,10.\,1931 ebd.), \emph{Schriftsteller, Mediziner}!Ruf des Lebens. Schauspiel in drei Akten@\strich\emph{Der Ruf des Lebens. Schauspiel in drei Akten}|pwv}
                    beiwohnen zu dürfen? Wenn Sie »Ja« dazu{ }ſagen, machen Sie mir eine große Freude!
                    Ich glaube, \textsc{Brahm}\pwindex{Brahm, Otto 5.\,2.\,1856 Hamburg – 28.\,11.\,1912 Berlin@\textsc{Brahm, Otto} (5.\,2.\,1856 Hamburg – 28.\,11.\,1912 Berlin), \emph{Theaterleiter, Regisseur}|pw} würde nichts dagegen haben weil ich ja auch bei Hauptmann\pwindex{Hauptmann, Gerhart 15.\,11.\,1862 Szczawno-Zdrój – 6.\,6.\,1946 Jagniątków@\textsc{Hauptmann, Gerhart} (15.\,11.\,1862 Szczawno-Zdrój – 6.\,6.\,1946 Jagniątków), \emph{Schriftsteller}|pw}’ſchen Generalproben öfters (auch letztes
                        \label{K_L01585-2v}\edtext{Mal\pwindex{Hauptmann, Gerhart 15.\,11.\,1862 Szczawno-Zdrój – 6.\,6.\,1946 Jagniątków@\textsc{Hauptmann, Gerhart} (15.\,11.\,1862 Szczawno-Zdrój – 6.\,6.\,1946 Jagniątków), \emph{Schriftsteller}!Und Pippa tanzt@\strich\emph{Und Pippa tanzt{\rufezeichen}}|pwv}}{\lemma{\textnormal{\emph{Mal}}}\Cendnote{\textnormal{Die Uraufführung\eventindex{Lessing-Theater@\textbf{Lessing-Theater}!Uraufführung von Und Pippa tanzt@Uraufführung von Und Pippa tanzt{\rufezeichen}|pwkv} von \emph{Und Pippa tanzt. Ein Glashüttenmärchen}\pwindex{Hauptmann, Gerhart 15.\,11.\,1862 Szczawno-Zdrój – 6.\,6.\,1946 Jagniątków@\textsc{Hauptmann, Gerhart} (15.\,11.\,1862 Szczawno-Zdrój – 6.\,6.\,1946 Jagniątków), \emph{Schriftsteller}!Und Pippa tanzt@\strich\emph{Und Pippa tanzt{\rufezeichen}}|pwk}
                        fand am 19. 1. 1906 am \emph{Lessing-Theater}\orgindex{Lessing-Theater@Lessing-Theater|pwk} statt.}}}\label{K_L01585-2}) zugegen war. Wollen Sie mir’s{ }ſchreiben in
                    die \textsc{Marburger}ſtr. 4\oindex{Marburger Straße@\textbf{Marburger Straße}, \emph{Straße}|pw}, Hospiz des Weſtens\oindex{Hospiz des Westens@\textbf{Hospiz des Westens}, \emph{Hotel}|pw}?\pend
           
\pstart
           In alter Verehrung Ihre{\\[\baselineskip]}\spacefill\mbox{Frau Lou.}\pend
           \leftskip=0em{}\selectlanguage{ngerman}\endnumbering\briefempfaengerindex{Schnitzler, Arthur@\textsc{Schnitzler, Arthur}!zzzAndreas-Salomé, Lou@\emph{von Lou Andreas-Salomé}!1906-02-191@{19. 2. 1906}|)be}\mylabel{L01585h}  \newcommand{\dateiname}{L01585}\newcommand{\titel}{Lou Andreas-Salomé an Arthur Schnitzler, 19. 2. 1906}\newcommand{\editorInnen}{Martin Anton Müller und Gerd-Hermann Susen}%% latex-leseansicht-abspann.tex
%% Abspann für die Leseansicht.
%% Der Schalter \ifkorrekturansicht ist bereits durch den Vorspann gesetzt.

%% latex-abspann.tex
%% Gemeinsamer Abspann für Korrekturansicht und Leseansicht.
%% Setzt den Schalter \ifkorrekturansicht voraus (gesetzt in den
%% einbindenden Dateien latex-korrekturansicht-abspann.tex bzw.
%% latex-leseansicht-abspann.tex).
%% ---------------------------------------------------------------

\normalsize

% Das esempio-Environment wird nur in der Leseansicht benötigt
\ifkorrekturansicht\else
\newenvironment{esempio}[3]%
{
    \vspace{1.5ex}
    \rlap{\underline{#1}}
    \par
    \setlength{\parindent}{0cm}
    \nopagebreak
    \leftskip=#2cm
    \rightskip=#3cm
}
{
    \par
}
\fi

\doendnotes{C}
\bigskip
\vfill

\clearpage

\footnotesize

\ifkorrekturansicht
  \lohead{\textsc{register}}
\fi

% theindex-Environment neu definieren ohne reledmac
\makeatletter
\renewenvironment{theindex}{%
  \ifkorrekturansicht
    \section*{\indexname}%
  \else
    \subsubsection*{Index der erwähnten Entitäten}%
  \fi
  \setlength{\parindent}{0pt}%
  \setlength{\parskip}{0pt plus 0.3pt}%
  \let\item\@idxitem
}{%
  \ifkorrekturansicht\clearpage\fi
}
\makeatother

\IfFileExists{\jobname-pw.ind}{\input{\jobname-pw.ind}}{}

% Quellenangabe nur in der Leseansicht
\ifkorrekturansicht\else
% Fallback-Definitionen, falls die .tex-Datei \titel etc. nicht gesetzt hat
\providecommand{\titel}{}
\providecommand{\editorInnen}{}
\providecommand{\dateiname}{\jobname}

\vspace{3cm}

\vfill

\footnotesize
\textsc{Quelle}: \titel. Herausgegeben von {\editorInnen}. In: \emph{Arthur Schnitzler: Briefwechsel mit Autorinnen und Autoren}.
 Digitale Edition, https://schnitzler-briefe.acdh.oeaw.ac.at/{\dateiname}.html (Stand \today)
\fi

\end{document}


