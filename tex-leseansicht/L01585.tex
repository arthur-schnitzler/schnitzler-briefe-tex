%% latex-leseansicht-vorspann.tex
%% Vorspann für die Leseansicht.
%% Lädt die gemeinsame Datei latex-vorspann.tex mit nicht gesetztem Schalter.

\newif\ifkorrekturansicht
\korrekturansichtfalse

\input{../tex-inputs/latex-vorspann}


         
         \renewcommand{\erwaehntePersonen}{Personen: Lou Andreas-Salomé, Otto Brahm, Gerhart Hauptmann}
         \renewcommand{\erwaehnteOrte}{Orte: Berlin, Deutsches Theater Berlin, Hospiz des Westens, Hotel Continental (Berlin), Marburger Straße}
         \renewcommand{\erwaehnteWerke}{Werke: Der Ruf des Lebens. Schauspiel in drei Akten, Und Pippa tanzt{\rufezeichen}}
               \section[Lou Andreas-Salomé an Arthur Schnitzler, 19. 2. 1906]{ Lou Andreas-Salomé an Arthur Schnitzler, 19. 2. 1906}\nopagebreak\mylabel{v}\rehead{ }\begin{ledgroupsized}[t]{13cm}\normalsize\beginnumbering \toendnotes[C]{\smallbreak\pagebreak[2]} \Standort{CUL, Schnitzler, B 3.}
\physDesc{Postkarte, 449 Zeichen
\newline{}Handschrift: schwarze Tinte, deutsche Kurrent
\newline{}Versand: 1) Stempel: »\nobreak{}\oindex{Berlin@\textbf{Berlin}|pwk}Berlin W 50, 19. 2. 06, 9 10 N\nobreak{}«.   2) Stempel: »\nobreak{}20. 2. 06\nobreak{}«. 
\newline{}Schnitzler: 1) mit Bleistift datiert: »19/2 06«  2) mit rotem Buntstift eine Unterstreichung
\newline{}Ordnung: mit Bleistift von unbekannter Hand nummeriert: »20« }\toendnotes[C]{\smallbreak}\pstart{}{\pb}Herrn \textsc{D\textsuperscript{r} Arthur Schnitzler}\pend{}\pstart{}\textsc{Berlin C.}\pend{}\pstart{}\textsc{Hôtel Continental\oindex{Hotel Continental (Berlin)@\textbf{Hotel Continental (Berlin)}|pw}.}\pend{}{\bigskip}\pstart
           \noindent{}{\pb}Lieber Doktor \textsc{Schnitzler}, darf ich Sie um
               die Erlaubniß bitten, am \label{K_L01585-1v}\edtext{Freitag}{\lemma{\textnormal{\emph{Freitag}}}\Cendnote{\textnormal{Die Generalprobe von \emph{Der Ruf des Lebens}\pwindex{Schnitzler, Arthur 15.05.1862 – 21.10.1931@\textsc{Schnitzler, Arthur} (15.05.1862 – 21.10.1931), \emph{Schriftsteller, Mediziner}!Ruf des Lebens. Schauspiel in drei Akten1906-02-20@\strich\emph{Der Ruf des Lebens. Schauspiel in drei Akten} {[}1906-02-20{]}|pwk} am Deutschen
                     Theater in Berlin\oindex{Deutsches Theater Berlin@\textbf{Deutsches Theater Berlin}|pwk} fand am Freitag, den 23. 3. 1906, statt. 
                  Ob Andreas-Salomé\pwindex{Andreas-Salome, Lou 12.02.1861 – 05.02.1937@\textsc{Andreas-Salomé, Lou} (12.02.1861 – 05.02.1937), \emph{Schriftstellerin}|pwk} teilnahm, ist nicht nachgewiesen. Die 
                  Uraufführung fand am 23. 3. 1906, dem Folgetag, statt. Hier
                  erwähnt Schnitzler\pwindex{Schnitzler, Arthur 15.05.1862 – 21.10.1931@\textsc{Schnitzler, Arthur} (15.05.1862 – 21.10.1931), \emph{Schriftsteller, Mediziner}|pwk} ihre Anwesenheit nach der Veranstaltung im \emph{Tagebuch}\textcolor{red}{\textsuperscript{XXXX indx}}.}}}\label{K_L01585-1h} der Generalprobe\pwindex{Schnitzler, Arthur 15.05.1862 – 21.10.1931@\textsc{Schnitzler, Arthur} (15.05.1862 – 21.10.1931), \emph{Schriftsteller, Mediziner}!Ruf des Lebens. Schauspiel in drei Akten1906-02-20@\strich\emph{Der Ruf des Lebens. Schauspiel in drei Akten} {[}1906-02-20{]}|pwv} beiwohnen zu dürfen? Wenn Sie »Ja« dazu ſagen,
               machen Sie mir eine große Freude! Ich glaube, \textsc{Brahm}\pwindex{Brahm, Otto 05.02.1856 – 28.11.1912@\textsc{Brahm, Otto} (05.02.1856 – 28.11.1912), \emph{Theaterleiter, Regisseur}|pw} würde nichts dagegen haben weil ich ja auch bei Hauptmann\pwindex{Hauptmann, Gerhart 15.11.1862 – 06.06.1946@\textsc{Hauptmann, Gerhart} (15.11.1862 – 06.06.1946), \emph{Schriftsteller}|pw}’ſchen Generalproben öfters (auch letztes \label{K_L01585-2v}\edtext{Mal\pwindex{Hauptmann, Gerhart 15.11.1862 – 06.06.1946@\textsc{Hauptmann, Gerhart} (15.11.1862 – 06.06.1946), \emph{Schriftsteller}!Und Pippa tanzt1906@\strich\emph{Und Pippa tanzt{\rufezeichen}} {[}1906{]}|pwv}}{\lemma{\textnormal{\emph{Mal}}}\Cendnote{\textnormal{\emph{Und Pippa tanzt. Ein Glashüttenmärchen}\pwindex{Hauptmann, Gerhart 15.11.1862 – 06.06.1946@\textsc{Hauptmann, Gerhart} (15.11.1862 – 06.06.1946), \emph{Schriftsteller}!Und Pippa tanzt1906@\strich\emph{Und Pippa tanzt{\rufezeichen}} {[}1906{]}|pwk}, hatte
                  am 19. 1. 1906 Uraufführung am Deutschen
                     Theater\oindex{Deutsches Theater Berlin@\textbf{Deutsches Theater Berlin}|pwk}.}}}\label{K_L01585-2h}) zugegen war. Wollen Sie mir's ſchreiben in die \textsc{Marburger}ſtr. 4\oindex{Marburger Strasse@\textbf{Marburger Straße}|pw}, Hospiz des Weſtens\oindex{Hospiz des Westens@\textbf{Hospiz des Westens}|pw}? \pend
           \pstart
           In alter Verehrung Ihre{\\[\baselineskip]}\spacefill\mbox{Frau Lou.}\pend
           \leftskip=0em{}
         
         \endnumbering\mylabel{h}\end{ledgroupsized}  \newcommand{\dateiname}{L01585}\newcommand{\titel}{Lou Andreas-Salomé an Arthur Schnitzler, 19. 2. 1906}\newcommand{\editorInnen}{Martin Anton Müller und Gerd-Hermann Susen}%% latex-leseansicht-abspann.tex
%% Abspann für die Leseansicht.
%% Der Schalter \ifkorrekturansicht ist bereits durch den Vorspann gesetzt.

%% latex-abspann.tex
%% Gemeinsamer Abspann für Korrekturansicht und Leseansicht.
%% Setzt den Schalter \ifkorrekturansicht voraus (gesetzt in den
%% einbindenden Dateien latex-korrekturansicht-abspann.tex bzw.
%% latex-leseansicht-abspann.tex).
%% ---------------------------------------------------------------

\normalsize

% Das esempio-Environment wird nur in der Leseansicht benötigt
\ifkorrekturansicht\else
\newenvironment{esempio}[3]%
{
    \vspace{1.5ex}
    \rlap{\underline{#1}}
    \par
    \setlength{\parindent}{0cm}
    \nopagebreak
    \leftskip=#2cm
    \rightskip=#3cm
}
{
    \par
}
\fi

\doendnotes{C}
\bigskip
\vfill

\clearpage

\footnotesize

\ifkorrekturansicht
  \lohead{\textsc{register}}
\fi

% theindex-Environment neu definieren ohne reledmac
\makeatletter
\renewenvironment{theindex}{%
  \ifkorrekturansicht
    \section*{\indexname}%
  \else
    \subsubsection*{Index der erwähnten Entitäten}%
  \fi
  \setlength{\parindent}{0pt}%
  \setlength{\parskip}{0pt plus 0.3pt}%
  \let\item\@idxitem
}{%
  \ifkorrekturansicht\clearpage\fi
}
\makeatother

\IfFileExists{\jobname-pw.ind}{\input{\jobname-pw.ind}}{}

% Quellenangabe nur in der Leseansicht
\ifkorrekturansicht\else
% Fallback-Definitionen, falls die .tex-Datei \titel etc. nicht gesetzt hat
\providecommand{\titel}{}
\providecommand{\editorInnen}{}
\providecommand{\dateiname}{\jobname}

\vspace{3cm}

\vfill

\footnotesize
\textsc{Quelle}: \titel. Herausgegeben von {\editorInnen}. In: \emph{Arthur Schnitzler: Briefwechsel mit Autorinnen und Autoren}.
 Digitale Edition, https://schnitzler-briefe.acdh.oeaw.ac.at/{\dateiname}.html (Stand \today)
\fi

\end{document}


      