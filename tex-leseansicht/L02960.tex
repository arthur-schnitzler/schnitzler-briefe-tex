%% latex-leseansicht-vorspann.tex
%% Vorspann für die Leseansicht.
%% Lädt die gemeinsame Datei latex-vorspann.tex mit nicht gesetztem Schalter.

\newif\ifkorrekturansicht
\korrekturansichtfalse

\input{../tex-inputs/latex-vorspann}

\begin{center}
            \textcolor{red}{ENTWURF, NICHT FERTIG KORRIGIERT}
                      \end{center}
            
         
         \renewcommand{\erwaehntePersonen}{Personen: Else Berger, Oskar Blumenthal, Antonie Cuny-Pierron, Rudolf Eduard von Cuny-Pierron, Gisela Fischer, Marie Glümer, Paul Goldmann, Felix Salten}
         \renewcommand{\erwaehnteOrte}{Orte: Baden bei Wien, Brühl, Diglas’ Restaurant »Zur schönen Aussicht«, Dölsach, Heiligenstadt, Klosterneuburg, Linz, Salzburg, Wien, XIX., Döbling}
         \renewcommand{\erwaehnteWerke}{}
               \section[Arthur Schnitzler an Felix Salten, {[}14. 8. 1893{]}]{ Arthur Schnitzler an Felix Salten, {[}14. 8. 1893{]}}\nopagebreak\mylabel{v}\rehead{ }\begin{ledgroupsized}[t]{13cm}\normalsize\beginnumbering \toendnotes[C]{\smallbreak\pagebreak[2]} \Standort{Wienbibliothek im Rathaus, ZPH 1681, 2.1.516.}
\physDesc{Brief, 2 Blätter, 8 Seiten, 2093 Zeichen
\newline{}Handschrift: Bleistift, deutsche Kurrent
\newline{}Ordnung: mit Bleistift von unbekannter Hand Nummerierung der Blätter des
                                 Konvoluts: »7«–»10« }\toendnotes[C]{\smallbreak}\pstart
           \noindent{}{\pb}Bei der »ſchönen Ausſicht\oindex{Diglas Restaurant »Zur schoenen Aussicht«@\textbf{Diglas’ Restaurant »Zur schönen Aussicht«}|pw}« – in Döbling\oindex{XIX., Doebling@\textbf{XIX., Döbling}|pw} – dort,
               bei der Buche, lehnt mein Rad. – Sehr, ſehr, ſehr allein. – Unten die dunkle Stadt
               und die Lichter von den fernen Landſtraßen. Um mich nachtmahlende recht vergnügte
               Bürger, \textcolor{gray}{ſpärlich} eigentlich.– Es iſt gegen neun, u ich halte bei
               der Virginier, da ich beim Schein der Gartenlaterne {\pb}einen Brief ſchreibe, dürfte ich für einen
               begabten Selbſtmörder gehalten werden.– Hergeko{\geminationm}en über
               einige unwahrſcheinliche Ortſchaften – mit einem Wort: Heiligenſtadt\oindex{Heiligenstadt@\textbf{Heiligenstadt}|pw}. War in Kloſterneuburg\oindex{Klosterneuburg@\textbf{Klosterneuburg}|pw};
               Bei Gelegenheit meines verbogenen Pedals eine herrliche jüdiſche Schloſſerfamilie {\pb}ſtudirt. »Wunderſchön«\textsuperscript{⋅)}, wie plötzlich zwei ältere jüdiſche Kloſterneuburg\oindex{Klosterneuburg@\textbf{Klosterneuburg}|pw}. »Gigerl« bei d\textcolor{gray}{er} Thür erſcheinen {\kaufmannsund} den \textcolor{gray}{beſuchten} Schloſſer ſagten,
               »Nu, \textcolor{gray}{V}ägel ä Tarotpartie?« und die 16jährige Tochter, die auch
               offenbar ſofort richtig taquirte, bemerkte »Klab raus jedu!!!«\footnote{\noindent{}Salten.–}\pend
           \pstart
           {\pb}– Eben \textcolor{gray}{trink} ich wieder
               einen Schluck Bier {\kaufmannsund} bemerke meine Einſamkeit. Ich lüge
               mir ſoeben vor, daſs ich begi{\geminationn}e, philoſophiſch und
               gleichgiltg zu werden – gegen »all d\textcolor{gray}{em} Tand, der uns von draußen
                  ko{\geminationm}t –« Frl. G.\pwindex{Gluemer, Marie 03.07.1867 – 16.11.1925@\textsc{Glümer, Marie} (03.07.1867 – 16.11.1925), \emph{Schauspielerin}|pw} war 2 oder 3 mal da; und es war wie i{\geminationm}er;–
               ich hab nie geahnt, daſs Weiber wegen ein u derſelben Sache \uline{ſo}{\pb}viel Thränen haben! – Von \textsc{Blumenthal\pwindex{Blumenthal, Oskar 13.03.1852 – 24.04.1917@\textsc{Blumenthal, Oskar} (13.03.1852 – 24.04.1917), \emph{Schriftsteller, Journalist, Theaterleiter}|pw}} kam geſtern ein Brief mit entrüſtenden Phraſen. – Merken Sie, Goldchnittpapier?
               Ich glaube, Frl. \textsc{Diglas\pwindex{Cuny-Pierron, Antonie 19.11.1871 – 1962-12-15@\textsc{Cuny-Pierron, Antonie} (19.11.1871 – 1962-12-15), \emph{Sängerin}|pw}} hat es dem Kellner zur Verfügung geſtellt.–\pend
           \pstart
           – Goldma{\geminationn}\pwindex{Goldmann, Paul 31.01.1865 – 25.09.1935@\textsc{Goldmann, Paul} (31.01.1865 – 25.09.1935), \emph{Schriftsteller, Journalist}|pw} ko{\geminationm}t wahrſcheinlich Anfang September nach \textsc{Salzburg\oindex{Salzburg@\textbf{Salzburg}|pw}}, ich ſchreib ihm – Ende {\pb}Auguſt. Bitte ſa{\geminationm}eln Sie unſere Daten über
               unſere Partie u. entſchließen Sie ſich zu einem ausführlichen Schreiben.– \pend
           \pstart
           – Nun fahr ich hinein, morgen in die Brühl\oindex{Bruehl@\textbf{Brühl}|pw},
               übermorgen zur Liebſten, hihihihihihihihihihi! {\pb}Geſtern war ich \textsc{per}
               Bic (Reichſtraße) Baden\oindex{Baden bei Wien@\textbf{Baden bei Wien}|pw}; wurde ſehr ſehnſüchtig
               und jung \label{K_L02960-11v}\edtext{geliebt\pwindex{Berger, Else 20.10.1874 – 24.11.1956@\textsc{Berger, Else} (20.10.1874 – 24.11.1956)|pwv}}{\lemma{\textnormal{\emph{geliebt}}}\Cendnote{\textnormal{vgl. A. S.: \emph{Tagebuch}, 13. 8. 1893}}}\label{K_L02960-11h}. Sonderbar: in demſelben Garten, in dem ich vor etwa 7 Jahren ein junges
                  \label{K_L02960-22v}\edtext{Mädel\pwindex{Fischer, Gisela 14.01.1866 – 25.06.1939@\textsc{Fischer, Gisela} (14.01.1866 – 25.06.1939)|pwv}}{\lemma{\textnormal{\emph{Mädel}}}\Cendnote{\textnormal{siehe A. S.: \emph{Tagebuch}, 12. 8. 1886}}}\label{K_L02960-22h} wahnſi{\geminationn}ig »herzte« u küſſte, das jetzt längſt
               verheiratet iſt – bis hundert Jahr.\pend
           \pstart
           {\pb}Wa{\geminationn} ich
               wegfahre, weiſs ich noch nicht. Wohl So{\geminationn}tag.– \pend
           \pstart
           Leben Sie woh, ſchreiben Sie was ſchönes und grüßen Sie mir die »wackern« Linz\oindex{Linz@\textbf{Linz}|pw}er Radfahrer. \pend
           \pstart
           All heil!–\pend
           \pstart
           \noindent{}\label{T_L02960-1v}\edtext{Nach Schluſs – Eben ging Hr P.\pwindex{Cuny-Pierron, Rudolf Eduard von 01.01.1853 – 15.07.1922@\textsc{Cuny-Pierron, Rudolf Eduard von} (01.01.1853 – 15.07.1922), \emph{Kaufmann}|pw}{ }\textsc{l’amant de M A. D.\pwindex{Cuny-Pierron, Antonie 19.11.1871 – 1962-12-15@\textsc{Cuny-Pierron, Antonie} (19.11.1871 – 1962-12-15), \emph{Sängerin}|pw}} an mir vorbei; Cretin!}{\lemma{\textnormal{\emph{Nach … Cretin!}}}\Cendnote{\textnormal{In einem
                     gezeichneten Kasten quer zum Text eingefügt.}}}\label{T_L02960-1h}\pend
           
         
         \endnumbering\mylabel{h}\end{ledgroupsized}\begin{anhang}\end{anhang}\newcommand{\dateiname}{L02960}\newcommand{\titel}{Arthur Schnitzler an Felix Salten, [14. 8. 1893]}\newcommand{\editorInnen}{Martin Anton Müller und Laura Untner}%% latex-leseansicht-abspann.tex
%% Abspann für die Leseansicht.
%% Der Schalter \ifkorrekturansicht ist bereits durch den Vorspann gesetzt.

%% latex-abspann.tex
%% Gemeinsamer Abspann für Korrekturansicht und Leseansicht.
%% Setzt den Schalter \ifkorrekturansicht voraus (gesetzt in den
%% einbindenden Dateien latex-korrekturansicht-abspann.tex bzw.
%% latex-leseansicht-abspann.tex).
%% ---------------------------------------------------------------

\normalsize

% Das esempio-Environment wird nur in der Leseansicht benötigt
\ifkorrekturansicht\else
\newenvironment{esempio}[3]%
{
    \vspace{1.5ex}
    \rlap{\underline{#1}}
    \par
    \setlength{\parindent}{0cm}
    \nopagebreak
    \leftskip=#2cm
    \rightskip=#3cm
}
{
    \par
}
\fi

\doendnotes{C}
\bigskip
\vfill

\clearpage

\footnotesize

\ifkorrekturansicht
  \lohead{\textsc{register}}
\fi

% theindex-Environment neu definieren ohne reledmac
\makeatletter
\renewenvironment{theindex}{%
  \ifkorrekturansicht
    \section*{\indexname}%
  \else
    \subsubsection*{Index der erwähnten Entitäten}%
  \fi
  \setlength{\parindent}{0pt}%
  \setlength{\parskip}{0pt plus 0.3pt}%
  \let\item\@idxitem
}{%
  \ifkorrekturansicht\clearpage\fi
}
\makeatother

\IfFileExists{\jobname-pw.ind}{\input{\jobname-pw.ind}}{}

% Quellenangabe nur in der Leseansicht
\ifkorrekturansicht\else
% Fallback-Definitionen, falls die .tex-Datei \titel etc. nicht gesetzt hat
\providecommand{\titel}{}
\providecommand{\editorInnen}{}
\providecommand{\dateiname}{\jobname}

\vspace{3cm}

\vfill

\footnotesize
\textsc{Quelle}: \titel. Herausgegeben von {\editorInnen}. In: \emph{Arthur Schnitzler: Briefwechsel mit Autorinnen und Autoren}.
 Digitale Edition, https://schnitzler-briefe.acdh.oeaw.ac.at/{\dateiname}.html (Stand \today)
\fi

\end{document}


      