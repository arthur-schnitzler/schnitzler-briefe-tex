%% latex-leseansicht-vorspann.tex
%% Vorspann für die Leseansicht.
%% Lädt die gemeinsame Datei latex-vorspann.tex mit nicht gesetztem Schalter.

\newif\ifkorrekturansicht
\korrekturansichtfalse

\input{../tex-inputs/latex-vorspann}


         
         \renewcommand{\erwaehntePersonen}{Personen: Else Berger, Oskar Blumenthal, Antonie Cuny-Pierron, Rudolf Eduard von Cuny-Pierron, Gisela Fischer, Marie Glümer, Paul Goldmann, Felix Salten, Heinrich Schnitzler, Josefine Lydia von Weisswasser}
         \renewcommand{\erwaehnteOrte}{Orte: Baden bei Wien, Brühl, Diglas’ Restaurant »Zur schönen Aussicht«, Dölsach, Heiligenstadt, Klosterneuburg, Lienz, Linz, Salzburg, Wien, XIX., Döbling}
         \renewcommand{\erwaehnteWerke}{Werke: Abschiedssouper, Eine Partie Klabrias im Café Spitzer, Faust. Eine Tragödie}
               \section[Arthur Schnitzler an Felix Salten, {[}14. 8. 1893{]}]{ Arthur Schnitzler an Felix Salten, {[}14. 8. 1893{]}}\nopagebreak\mylabel{v}\rehead{ }\begin{ledgroupsized}[t]{13cm}\normalsize\beginnumbering\briefempfaengerindex{Salten, Felix@\textsc{Salten, Felix}!zzzSchnitzler, Arthur@\emph{von Arthur Schnitzler}!1893-08-141@{{[}14. 8. 1893{]}}|(be} \toendnotes[C]{\smallbreak\pagebreak[2]} \Standort{Wienbibliothek im Rathaus, ZPH 1681, 2.1.516.}
\physDesc{Brief, 2 Blätter, 8 Seiten, 2099 Zeichen (Briefpapier mit Trauerrand)
\newline{}Handschrift: Bleistift, deutsche Kurrent
\newline{}Ordnung: mit Bleistift von unbekannter Hand Nummerierung der Doppelseiten des Konvoluts:
                                    »7«–»10« }\buchAbdrucke{\weitereDrucke{Arthur Schnitzler: \emph{Briefe 1875–1912}. Hg. Therese Nickl und Heinrich Schnitzler. Frankfurt am Main: \emph{S. Fischer} 1981, S. 211–212.} }\toendnotes[C]{\smallbreak}\pstart
           \noindent{}{\pb}\label{K_L02960-1v}\edtext{Bei der »ſchönen Ausſicht\oindex{Diglas Restaurant »Zur schoenen Aussicht«@\textbf{Diglas’ Restaurant »Zur schönen Aussicht«}|pw}«}{\lemma{\textnormal{\emph{Bei … Ausſicht«}}}\Cendnote{\textnormal{Der Brief ist ungewöhnlich, da er weder eine Andrede noch
                  eine Unterschrift aufweist. Das ließe sich damit erklären, dass Schnitzler\pwindex{Schnitzler, Arthur 15.05.1862 – 21.10.1931@\textsc{Schnitzler, Arthur} (15.05.1862 – 21.10.1931), \emph{Schriftsteller, Mediziner}|pwk} das Schreiben nicht auf dem üblichen Postweg
                  versandte, sondern als offenes Schreiben jemandem mitgab. Ob das der Fall war,
                  lässt sich wegen des fehlenden Umschlags nicht bestimmen.}}}\label{K_L02960-1h} – in Döbling\oindex{XIX., Doebling@\textbf{XIX., Döbling}|pw} – dort, bei der Buche, lehnt mein Rad. –
               Sehr, ſehr, ſehr allein. – Unten die dunkle Stadt\oindex{Wien@\textbf{Wien}|pwv} und die Lichter von den fernen Landſtraßen. Um mich
               nachtmahlende recht vergnügte Bürger, ſpärlich eigentlich. – Es iſt gegen
                  neun, u ich \label{K_L02960-2v}\edtext{halte bei
               der Virginier}{\lemma{\textnormal{\emph{halte bei
               der Virginier}}}\Cendnote{\textnormal{Er drückt aus, dass der das Rauchen
                  seiner Zigarre unterbricht.}}}\label{K_L02960-2h}. Da ich beim Schein der Gartenlaterne {\pb}einen Brief ſchreibe, dürfte ich für einen
               begabten Selbſtmörder gehalten werden. – Hergeko{\geminationm}en über
               einige unwahrſcheinliche Ortſchaften – mit einem Wort: Heiligenſtadt\oindex{Heiligenstadt@\textbf{Heiligenstadt}|pw}. War in Kloſterneuburg\oindex{Klosterneuburg@\textbf{Klosterneuburg}|pw};
               Bei Gelegenheit meines verbogenen Pedales eine herrliche jüdiſche Schloſſerfamilie
                  {\pb}ſtudirt. »Wunderſchön«\footnote{\noindent{}Salten. –}, wie plötzlich zwei ältere jüdiſche Kloſterneuburg\oindex{Klosterneuburg@\textbf{Klosterneuburg}|pw}. »\label{K_L02960-3v}\edtext{Gig\textcolor{gray}{o}hl}{\lemma{\textnormal{\emph{Gigohl}}}\Cendnote{\textnormal{womöglich
                  eine Dialektvariation für ›Gigerl‹ (Modenarr, Dandy)}}}\label{K_L02960-3h}« bei
                  d\textcolor{gray}{er} Thür erſcheinen {\kaufmannsund}
                  de\textcolor{gray}{m}
               barfußen Schloſſer ſagten, »Nü,
                  \textcolor{gray}{M}äxel, was is mit ä Tarotpartie?« und die 16jährige \label{K_L02960-4v}\edtext{Tochter, die mich offenbar ſofort richtig
                  taxirte, bemerkte
                  »Klabriaspartie\pwindex{\textcolor{red}{\textsuperscript{XXXX1 indx}}!Eine Partie Klabrias im Cafe Spitzer1890@\strich\emph{Eine Partie Klabrias im Café Spitzer} {[}1890{]}|pw}}{\lemma{\textnormal{\emph{Tochter, … »Klabriaspartie}}}\Cendnote{\textnormal{Die Tochter dürfte Einvernehmen
                     herstellen, dass es sich hier um eine Anspielung auf die (jüdische) Erfolgsposse
                     \emph{Eine Partie Klabrias}\pwindex{\textcolor{red}{\textsuperscript{XXXX1 indx}}!Eine Partie Klabrias im Cafe Spitzer1890@\strich\emph{Eine Partie Klabrias im Café Spitzer} {[}1890{]}|pwk} handelte. Heinrich Schnitzler\pwindex{Schnitzler, Heinrich 09.08.1902 – 12.07.1982@\textsc{Schnitzler, Heinrich} (09.08.1902 – 12.07.1982), \emph{Regisseur, Schauspieler}|pwk} kommentierte im Erstdruck diese Stelle
                     mit einem beliebten Ausspruch seines Vaters: »Zitate sind entweder aus Faust\pwindex{\textcolor{red}{\textsuperscript{XXXX1 indx}}!Faust. Eine Tragoedie1808@\strich\emph{Faust. Eine Tragödie} {[}1808{]}|pw} oder
                        aus der Klabriaspartie\pwindex{\textcolor{red}{\textsuperscript{XXXX1 indx}}!Eine Partie Klabrias im Cafe Spitzer1890@\strich\emph{Eine Partie Klabrias im Café Spitzer} {[}1890{]}|pw}.«}}}\label{K_L02960-4h}!«\pend
           \pstart
           {\pb}– Eben \substVorne{}\textsuperscript{machte}{\allowbreak}\substDazwischen{}trank\substHinten{} ich wieder einen Schluck Bier {\kaufmannsund} bemerke meine
               Einſamkeit. Ich lüge mir ſoeben vor, daſs ich begi{\geminationn}e,
               philoſophiſch und gleichgilt\textcolor{gray}{i}g zu werden – gegen »\label{K_L02960-5v}\edtext{all d\textcolor{gray}{en}{ }Tand, der uns von draußen ko{\geminationm}t\pwindex{Schnitzler, Arthur 15.05.1862 – 21.10.1931@\textsc{Schnitzler, Arthur} (15.05.1862 – 21.10.1931), \emph{Schriftsteller, Mediziner}!Abschiedssouper1892@\strich\emph{Abschiedssouper} {[}1892{]}|pwv}}{\lemma{\textnormal{\emph{all … kommt}}}\Cendnote{\textnormal{Selbstzitat aus \emph{Abschiedssouper}\pwindex{Schnitzler, Arthur 15.05.1862 – 21.10.1931@\textsc{Schnitzler, Arthur} (15.05.1862 – 21.10.1931), \emph{Schriftsteller, Mediziner}!Abschiedssouper1892@\strich\emph{Abschiedssouper} {[}1892{]}|pwk}: »Als
                     wenn es keine Feierlichkeiten der Seele gäbe, die mit all’ dieſem Tand, der uns
                     von dem Draußen kommt, gar nichts zu thun haben –«}}}\label{K_L02960-5h} –« Frl. G.\pwindex{Gluemer, Marie 03.07.1867 – 16.11.1925@\textsc{Glümer, Marie} (03.07.1867 – 16.11.1925), \emph{Schauspielerin}|pw} war 2 oder 3 mal da; und es war wie i{\geminationm}er; – ich hab nie geahnt, daſs Weiber wegen ein u
               derſelben Sache \uline{ſo}{ }{\pb}viel Thränen haben! – Von \textsc{Blumenthal\pwindex{Blumenthal, Oskar 13.03.1852 – 24.04.1917@\textsc{Blumenthal, Oskar} (13.03.1852 – 24.04.1917), \emph{Schriftsteller, Journalist, Theaterleiter}|pw}} kam geſtern ein \label{K_L02960-6v}\edtext{Brief}{\lemma{\textnormal{\emph{Brief}}}\Cendnote{\textnormal{Oscar Blumenthal an Arthur Schnitzler, 12. 8. 1893.
               }}}\label{K_L02960-6h} mit vertröſtenden Phraſen. – Merken Sie, Goldchnittpapier? Ich glaube, Frl.
                  \textsc{Diglas\pwindex{Cuny-Pierron, Antonie 19.11.1871 – 1962-12-15@\textsc{Cuny-Pierron, Antonie} (19.11.1871 – 1962-12-15), \emph{Sängerin}|pw}} hat es dem Kellner zur Verfügung geſtellt.– \pend
           \pstart
           – Goldma{\geminationn}\pwindex{Goldmann, Paul 31.01.1865 – 25.09.1935@\textsc{Goldmann, Paul} (31.01.1865 – 25.09.1935), \emph{Schriftsteller, Journalist}|pw} ko{\geminationm}t wahrſcheinlich \label{K_L02960-7v}\edtext{Anfang September nach \textsc{Salzburg\oindex{Salzburg@\textbf{Salzburg}|pw}}}{\lemma{\textnormal{\emph{Anfang … Salzburg}}}\Cendnote{\textnormal{Siehe Paul Goldmann an Arthur Schnitzler, 18. 8. [1893].
               }}}\label{K_L02960-7h}, ich ſchreib ihm – Ende {\pb}Auguſt. Bitte ſa{\geminationm}eln Sie
               nähere Daten über unſre Partie u. entſchließen Sie ſich zu einem ausführlichen
               Schreiben. –\pend
           \pstart
           – Nun fahr ich hinein, \label{K_L02960-8v}\edtext{morgen in die Brühl\oindex{Bruehl@\textbf{Brühl}|pw}, übermorgen zur »Liebſten\pwindex{Weisswasser, Josefine Lydia von *~01.03.1864@\textsc{Weisswasser, Josefine Lydia von} (*~01.03.1864)|pw}«}{\lemma{\textnormal{\emph{morgen … »Liebſten«}}}\Cendnote{\textnormal{Siehe A. S.: \emph{Tagebuch}, 15. 8. 1893 und 16. 8. 1893.
               }}}\label{K_L02960-8h}, hihihihihihihihihihi!\pend
           \pstart
           {\pb}Geſtern war ich \textsc{per} Bic
                  (\label{K_L02960-9v}\edtext{Reichſtraße}{\lemma{\textnormal{\emph{Reichſtraße}}}\Cendnote{\textnormal{Fernstraße}}}\label{K_L02960-9h}) Baden\oindex{Baden bei Wien@\textbf{Baden bei Wien}|pw};
               wurde ſehr ſehnſüchtig u jung \label{K_L02960-10v}\edtext{geliebt\pwindex{Berger, Else 20.10.1874 – 24.11.1956@\textsc{Berger, Else} (20.10.1874 – 24.11.1956)|pwv}}{\lemma{\textnormal{\emph{geliebt}}}\Cendnote{\textnormal{Siehe A. S.: \emph{Tagebuch}, 13. 8. 1893.
               }}}\label{K_L02960-10h}. Sonderbar! in demſelben Garten, in dem ich vor etwa 7 Jahren ein junges
                  \label{K_L02960-11v}\edtext{Mädel\pwindex{Fischer, Gisela 14.01.1866 – 25.06.1939@\textsc{Fischer, Gisela} (14.01.1866 – 25.06.1939)|pwv}}{\lemma{\textnormal{\emph{Mädel}}}\Cendnote{\textnormal{Siehe A. S.: \emph{Tagebuch}, 12. 8. 1886.
               }}}\label{K_L02960-11h} wahnſi{\geminationn}ig »herzte« u küſſte, das jetzt längſt
               verheiratet iſt – bis hundert Jahr.\pend
           \pstart
           {\pb}Wa{\geminationn} ich
               wegfahre, weiſs ich noch nicht. Wohl \label{K_L02960-12v}\edtext{So{\geminationn}tag}{\lemma{\textnormal{\emph{Sonntag}}}\Cendnote{\textnormal{Schnitzler\pwindex{Schnitzler, Arthur 15.05.1862 – 21.10.1931@\textsc{Schnitzler, Arthur} (15.05.1862 – 21.10.1931), \emph{Schriftsteller, Mediziner}|pwk} reiste am Dienstag, 22. 8. 1893, aus Wien\oindex{Wien@\textbf{Wien}|pwkv} ab.}}}\label{K_L02960-12h}. –\pend
           \pstart
           Leben Sie wohl, ſchreiben Sie was ſchönes und grüßen Sie mir die »wackern« \label{K_L02960-13v}\edtext{Linz\oindex{Linz@\textbf{Linz}|pw}er Radfahrer}{\lemma{\textnormal{\emph{Linzer Radfahrer}}}\Cendnote{\textnormal{Er dürfte wohl eher die Lienzer\oindex{Lienz@\textbf{Lienz}|pwk} Radfahrer meinen, vgl. Felix Salten an Arthur Schnitzler, 12. 8. 1893.}}}\label{K_L02960-13h}.\pend
           \pstart
           All heil! –\pend
           \pstart
           \noindent{}\label{T_L02960-1v}\edtext{Nach Schluſs – Eben ging Hr P.\pwindex{Cuny-Pierron, Rudolf Eduard von 01.01.1853 – 15.07.1922@\textsc{Cuny-Pierron, Rudolf Eduard von} (01.01.1853 – 15.07.1922), \emph{Kaufmann}|pw}{ }\textsc{\label{K_L02960-14v}\edtext{\begin{otherlanguage}{french}l’amant de\end{otherlanguage}}{\lemma{\textnormal{\emph{l’amant de}}}\Cendnote{\textnormal{französisch: Liebhaber von}}}\label{K_L02960-14h}
                     M A. D.\pwindex{Cuny-Pierron, Antonie 19.11.1871 – 1962-12-15@\textsc{Cuny-Pierron, Antonie} (19.11.1871 – 1962-12-15), \emph{Sängerin}|pw}} an mir vorbei, \label{K_L02960-15v}\edtext{\begin{otherlanguage}{french}Cretin\end{otherlanguage}}{\lemma{\textnormal{\emph{Cretin}}}\Cendnote{\textnormal{französisch: Dummkopf,
                     Idiot}}}\label{K_L02960-15h}!}{\lemma{\textnormal{\emph{Nach … Cretin!}}}\Cendnote{\textnormal{in einem gezeichneten
                     Kasten quer zum Text}}}\label{T_L02960-1h}\pend
           
         
         \endnumbering\mylabel{h}\end{ledgroupsized}  \newcommand{\dateiname}{L02960}\newcommand{\titel}{Arthur Schnitzler an Felix Salten, [14. 8. 1893]}\newcommand{\editorInnen}{Martin Anton Müller und Laura Untner}%% latex-leseansicht-abspann.tex
%% Abspann für die Leseansicht.
%% Der Schalter \ifkorrekturansicht ist bereits durch den Vorspann gesetzt.

%% latex-abspann.tex
%% Gemeinsamer Abspann für Korrekturansicht und Leseansicht.
%% Setzt den Schalter \ifkorrekturansicht voraus (gesetzt in den
%% einbindenden Dateien latex-korrekturansicht-abspann.tex bzw.
%% latex-leseansicht-abspann.tex).
%% ---------------------------------------------------------------

\normalsize

% Das esempio-Environment wird nur in der Leseansicht benötigt
\ifkorrekturansicht\else
\newenvironment{esempio}[3]%
{
    \vspace{1.5ex}
    \rlap{\underline{#1}}
    \par
    \setlength{\parindent}{0cm}
    \nopagebreak
    \leftskip=#2cm
    \rightskip=#3cm
}
{
    \par
}
\fi

\doendnotes{C}
\bigskip
\vfill

\clearpage

\footnotesize

\ifkorrekturansicht
  \lohead{\textsc{register}}
\fi

% theindex-Environment neu definieren ohne reledmac
\makeatletter
\renewenvironment{theindex}{%
  \ifkorrekturansicht
    \section*{\indexname}%
  \else
    \subsubsection*{Index der erwähnten Entitäten}%
  \fi
  \setlength{\parindent}{0pt}%
  \setlength{\parskip}{0pt plus 0.3pt}%
  \let\item\@idxitem
}{%
  \ifkorrekturansicht\clearpage\fi
}
\makeatother

\IfFileExists{\jobname-pw.ind}{\input{\jobname-pw.ind}}{}

% Quellenangabe nur in der Leseansicht
\ifkorrekturansicht\else
% Fallback-Definitionen, falls die .tex-Datei \titel etc. nicht gesetzt hat
\providecommand{\titel}{}
\providecommand{\editorInnen}{}
\providecommand{\dateiname}{\jobname}

\vspace{3cm}

\vfill

\footnotesize
\textsc{Quelle}: \titel. Herausgegeben von {\editorInnen}. In: \emph{Arthur Schnitzler: Briefwechsel mit Autorinnen und Autoren}.
 Digitale Edition, https://schnitzler-briefe.acdh.oeaw.ac.at/{\dateiname}.html (Stand \today)
\fi

\end{document}


      