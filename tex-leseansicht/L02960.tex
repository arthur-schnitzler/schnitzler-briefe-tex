%% latex-leseansicht-vorspann.tex
%% Vorspann für die Leseansicht.
%% Lädt die gemeinsame Datei latex-vorspann.tex mit nicht gesetztem Schalter.

\newif\ifkorrekturansicht
\korrekturansichtfalse

\input{../tex-inputs/latex-vorspann}


\section[Arthur Schnitzler an Felix Salten, {{[}}14. 8. 1893{{]}}]{L02960 Arthur Schnitzler an Felix Salten, {[}14. 8. 1893{]}}
\nopagebreak\mylabel{L02960v}
\rehead{ }\normalsize\beginnumbering\briefempfaengerindex{Salten, Felix@\textsc{Salten, Felix}!zzzSchnitzler, Arthur@\emph{von Arthur Schnitzler}!1893-08-141@{{[}14. 8. 1893{]}}|(be}
\toendnotes[C]{\smallbreak\pagebreak[2]}
\correspDesc{Versand  durch Arthur Schnitzler am [14. 8. 1893] in Wien
\newline{}Erhalt  durch Felix Salten im Zeitraum [15. 8. 1893
                  – 17. 8. 1893?] in Dölsach}\toendnotes[C]{\smallbreak}
\Standort{Wienbibliothek im Rathaus, ZPH 1681, 2.1.516.}
\physDesc{Brief, 2 Blätter, 8 Seiten, 2099 Zeichen (Briefpapier mit Trauerrand)
\newline{}Handschrift: Bleistift, deutsche Kurrent
\newline{}Ordnung: mit Bleistift von unbekannter Hand Nummerierung der Doppelseiten des Konvoluts:
                                    »7«–»10« }
\buchAbdrucke{\weitereDrucke{Arthur Schnitzler: \emph{Briefe 1875–1912}. Herausgegeben von Therese Nickl und Heinrich Schnitzler. Frankfurt am Main: \emph{S. Fischer} 1981, S. 211–212.} }\toendnotes[C]{\smallbreak}
\pstart
           \noindent{}{\pb}\label{K_L02960-1v}\edtext{Bei der »ſchönen Ausſicht\oindex{Wien@\textbf{Wien}!XIX., Döbling@\textbf{XIX., Döbling}!Diglas’ Restaurant »Zur schönen Aussicht«@\textbf{Diglas’ Restaurant »Zur schönen Aussicht«}, \emph{Gastgewerbegebäude}|pw}«}{\lemma{\textnormal{\emph{Bei … Aussicht«}}}\Cendnote{\textnormal{Der Brief ist ungewöhnlich, da er weder eine Andrede noch
                  eine Unterschrift aufweist. Das ließe sich damit erklären, dass Schnitzler das Schreiben nicht auf dem üblichen Postweg
                  versandte, sondern als offenes Schreiben jemandem mitgab. Ob das der Fall war,
                  lässt sich wegen des fehlenden Umschlags nicht bestimmen.}}}\label{K_L02960-1} – in Döbling\oindex{XIX., Döbling@\textbf{XIX., Döbling}, \emph{Verwaltungsgebiet}|pw} – dort, bei der Buche, lehnt mein Rad. –
               Sehr,{ }ſehr,{ }ſehr allein. – Unten die dunkle Stadt\oindex{Wien@\textbf{Wien}, \emph{Verwaltungsgebiet}|pwv} und die Lichter von den fernen Landſtraßen. Um mich
               nachtmahlende recht vergnügte Bürger,{ }ſpärlich eigentlich. – Es iſt gegen
                  neun, u ich \label{K_L02960-2v}\edtext{halte bei
               der Virginier}{\lemma{\textnormal{\emph{halte bei
               der Virginier}}}\Cendnote{\textnormal{Er drückt aus, dass der das Rauchen
                  seiner Zigarre unterbricht.}}}\label{K_L02960-2}. Da ich beim Schein der Gartenlaterne {\pb}einen Brief{ }ſchreibe, dürfte ich für einen
               begabten Selbſtmörder gehalten werden. – Hergeko{\geminationm}en über
               einige unwahrſcheinliche Ortſchaften – mit einem Wort: Heiligenſtadt\oindex{Wien@\textbf{Wien}!XIX., Döbling@\textbf{XIX., Döbling}!Heiligenstadt@\textbf{Heiligenstadt}|pw}. War in Kloſterneuburg\oindex{Klosterneuburg@\textbf{Klosterneuburg}, \emph{Hauptstadt}|pw};
               Bei Gelegenheit meines verbogenen Pedales eine herrliche jüdiſche Schloſſerfamilie
                  {\pb}ſtudirt. »Wunderſchön«\footnote{\noindent{}Salten. –}, wie plötzlich zwei ältere jüdiſche Kloſterneuburg\oindex{Klosterneuburg@\textbf{Klosterneuburg}, \emph{Hauptstadt}|pw}. »\label{K_L02960-3v}\edtext{Gig\textcolor{gray}{o}hl}{\lemma{\textnormal{\emph{Gigohl}}}\Cendnote{\textnormal{womöglich
                  eine Dialektvariation für ›Gigerl‹ (Modenarr, Dandy)}}}\label{K_L02960-3}« bei
                  d\textcolor{gray}{er} Thür erſcheinen {\kaufmannsund}
                  de\textcolor{gray}{m}
               barfußen Schloſſer{ }ſagten, »Nü,
                  \textcolor{gray}{M}äxel, was is mit ä Tarotpartie?« und die 16jährige \label{K_L02960-4v}\edtext{Tochter, die mich offenbar{ }ſofort richtig
                  taxirte, bemerkte
                  »Klabriaspartie\pwindex{\textcolor{red}{\textsuperscript{XXXX indx1}}!Eine Partie Klabrias im Café Spitzer@\strich\emph{Eine Partie Klabrias im Café Spitzer}|pw}}{\lemma{\textnormal{\emph{Tochter, … »Klabriaspartie}}}\Cendnote{\textnormal{Die Tochter dürfte Einvernehmen
                     herstellen, dass es sich hier um eine Anspielung auf die (jüdische) Erfolgsposse
                     \emph{Eine Partie Klabrias}\pwindex{\textcolor{red}{\textsuperscript{XXXX indx1}}!Eine Partie Klabrias im Café Spitzer@\strich\emph{Eine Partie Klabrias im Café Spitzer}|pwk} handelte. Heinrich Schnitzler\pwindex{Schnitzler, Heinrich 9.\,8.\,1902 Hinterbrühl – 12.\,7.\,1982 Wien@\textsc{Schnitzler, Heinrich} (9.\,8.\,1902 Hinterbrühl – 12.\,7.\,1982 Wien), \emph{Regisseur, Schauspieler}|pwk} kommentierte im Erstdruck diese Stelle
                     mit einem beliebten Ausspruch seines Vaters: »Zitate sind entweder aus Faust\pwindex{\textcolor{red}{\textsuperscript{XXXX indx1}}!Faust. Eine Tragödie@\strich\emph{Faust. Eine Tragödie}|pw} oder
                        aus der Klabriaspartie\pwindex{\textcolor{red}{\textsuperscript{XXXX indx1}}!Eine Partie Klabrias im Café Spitzer@\strich\emph{Eine Partie Klabrias im Café Spitzer}|pw}.«}}}\label{K_L02960-4}!«\pend
           
\pstart
           {\pb}– Eben \substVorne{}\textsuperscript{machte}\substDazwischen{}trank\substHinten{} ich wieder einen Schluck Bier {\kaufmannsund} bemerke meine
               Einſamkeit. Ich lüge mir{ }ſoeben vor, daſs ich begi{\geminationn}e,
               philoſophiſch und gleichgilt\textcolor{gray}{i}g zu werden – gegen »\label{K_L02960-5v}\edtext{all d\textcolor{gray}{en}{ }Tand, der uns von draußen ko{\geminationm}t\pwindex{Schnitzler, Arthur 15.\,5.\,1862 Wien – 21.\,10.\,1931 ebd.@\textsc{Schnitzler, Arthur} (15.\,5.\,1862 Wien – 21.\,10.\,1931 ebd.), \emph{Schriftsteller, Mediziner}!Abschiedssouper@\strich\emph{Abschiedssouper}|pwv}}{\lemma{\textnormal{\emph{all … kommt}}}\Cendnote{\textnormal{Selbstzitat aus \emph{Abschiedssouper}\pwindex{Schnitzler, Arthur 15.\,5.\,1862 Wien – 21.\,10.\,1931 ebd.@\textsc{Schnitzler, Arthur} (15.\,5.\,1862 Wien – 21.\,10.\,1931 ebd.), \emph{Schriftsteller, Mediziner}!Abschiedssouper@\strich\emph{Abschiedssouper}|pwk}: »Als
                     wenn es keine Feierlichkeiten der Seele gäbe, die mit all’ dieſem Tand, der uns
                     von dem Draußen kommt, gar nichts zu thun haben –«}}}\label{K_L02960-5} –« Frl. G.\pwindex{Glümer, Marie 3.\,7.\,1867 Wien – 16.\,11.\,1925 München@\textsc{Glümer, Marie} (3.\,7.\,1867 Wien – 16.\,11.\,1925 München), \emph{Schauspielerin}|pw} war 2 oder 3 mal da; und es war wie i{\geminationm}er; – ich hab nie geahnt, daſs Weiber wegen ein u
               derſelben Sache \uline{ſo}{ }{\pb}viel Thränen haben! – Von \textsc{Blumenthal\pwindex{Blumenthal, Oskar 13.\,3.\,1852 Berlin – 24.\,4.\,1917 ebd.@\textsc{Blumenthal, Oskar} (13.\,3.\,1852 Berlin – 24.\,4.\,1917 ebd.), \emph{Schriftsteller, Journalist, Theaterleiter}|pw}} kam geſtern ein \label{K_L02960-6v}\edtext{Brief}{\lemma{\textnormal{\emph{Brief}}}\Cendnote{\textnormal{XXXX Auszeichnungsfehler: Dokument L00253 nicht gefunden.
               }}}\label{K_L02960-6} mit vertröſtenden Phraſen. – Merken Sie, Goldchnittpapier? Ich glaube, Frl.
                  \textsc{Diglas\pwindex{Cuny-Pierron, Antonie 19.\,11.\,1871 Wien – 15.\,12.\,1962 ebd.@\textsc{Cuny-Pierron, Antonie} (19.\,11.\,1871 Wien – 15.\,12.\,1962 ebd.), \emph{Sängerin}|pw}} hat es dem Kellner zur Verfügung geſtellt.– \pend
           
\pstart
           – Goldma{\geminationn}\pwindex{Goldmann, Paul 31.\,1.\,1865 Breslau – 25.\,9.\,1935 Wien@\textsc{Goldmann, Paul} (31.\,1.\,1865 Breslau – 25.\,9.\,1935 Wien), \emph{Schriftsteller, Journalist}|pw} ko{\geminationm}t wahrſcheinlich \label{K_L02960-7v}\edtext{Anfang September nach \textsc{Salzburg\oindex{Salzburg@\textbf{Salzburg}, \emph{Verwaltungsgebiet}|pw}}}{\lemma{\textnormal{\emph{Anfang … Salzburg}}}\Cendnote{\textnormal{Siehe XXXX Auszeichnungsfehler: Dokument L02712 nicht gefunden.
               }}}\label{K_L02960-7}, ich{ }ſchreib ihm – Ende {\pb}Auguſt. Bitte{ }ſa{\geminationm}eln Sie
               nähere Daten über unſre Partie u. entſchließen Sie{ }ſich zu einem ausführlichen
               Schreiben. –\pend
           
\pstart
           – Nun fahr ich hinein, \label{K_L02960-8v}\edtext{morgen in die Brühl\oindex{Brühl@\textbf{Brühl}, \emph{Tal}|pw}, übermorgen zur »Liebſten\pwindex{Weisswasser, Josefine Lydia von *~1.\,3.\,1864 Moşna@\textsc{Weisswasser, Josefine Lydia von} (*~1.\,3.\,1864 Moşna)|pw}«}{\lemma{\textnormal{\emph{morgen … »Liebsten«}}}\Cendnote{\textnormal{Siehe A. S.: \emph{Tagebuch}, 15. 8. 1893 und 16. 8. 1893.
               }}}\label{K_L02960-8}, hihihihihihihihihihi!\pend
           
\pstart
           {\pb}Geſtern war ich \textsc{per} Bic
                  (\label{K_L02960-9v}\edtext{Reichſtraße}{\lemma{\textnormal{\emph{Reichstraße}}}\Cendnote{\textnormal{Fernstraße}}}\label{K_L02960-9}) Baden\oindex{Baden bei Wien@\textbf{Baden bei Wien}, \emph{Hauptstadt}|pw};
               wurde{ }ſehr{ }ſehnſüchtig u jung \label{K_L02960-10v}\edtext{geliebt\pwindex{Berger, Else 20.\,10.\,1874 Wien – 24.\,11.\,1956 ebd.@\textsc{Berger, Else} (20.\,10.\,1874 Wien – 24.\,11.\,1956 ebd.)|pwv}}{\lemma{\textnormal{\emph{geliebt}}}\Cendnote{\textnormal{Siehe A. S.: \emph{Tagebuch}, 13. 8. 1893.
               }}}\label{K_L02960-10}. Sonderbar! in demſelben Garten, in dem ich vor etwa 7 Jahren ein junges
                  \label{K_L02960-11v}\edtext{Mädel\pwindex{Fischer, Gisela 14.\,1.\,1866 Wien – 25.\,6.\,1939 ebd.@\textsc{Fischer, Gisela} (14.\,1.\,1866 Wien – 25.\,6.\,1939 ebd.)|pwv}}{\lemma{\textnormal{\emph{Mädel}}}\Cendnote{\textnormal{Siehe A. S.: \emph{Tagebuch}, 12. 8. 1886.
               }}}\label{K_L02960-11} wahnſi{\geminationn}ig »herzte« u küſſte, das jetzt längſt
               verheiratet iſt – bis hundert Jahr.\pend
           
\pstart
           {\pb}Wa{\geminationn} ich
               wegfahre, weiſs ich noch nicht. Wohl \label{K_L02960-12v}\edtext{So{\geminationn}tag}{\lemma{\textnormal{\emph{Sonntag}}}\Cendnote{\textnormal{Schnitzler reiste am Dienstag, 22. 8. 1893, aus Wien\oindex{Wien@\textbf{Wien}, \emph{Verwaltungsgebiet}|pwkv} ab.}}}\label{K_L02960-12}. –\pend
           
\pstart
           Leben Sie wohl,{ }ſchreiben Sie was{ }ſchönes und grüßen Sie mir die »wackern« \label{K_L02960-13v}\edtext{Linz\oindex{Linz@\textbf{Linz}|pw}er Radfahrer}{\lemma{\textnormal{\emph{Linzer Radfahrer}}}\Cendnote{\textnormal{Er dürfte wohl eher die Lienzer\oindex{Lienz@\textbf{Lienz}, \emph{Hauptstadt}|pwk} Radfahrer meinen, vgl. XXXX Auszeichnungsfehler: Dokument L03126 nicht gefunden.}}}\label{K_L02960-13}.\pend
           
\pstart
           All heil! –\pend
           
\pstart
           \noindent{}\label{T_L02960-1v}\edtext{Nach Schluſs – Eben ging Hr P.\pwindex{Cuny-Pierron, Rudolf Eduard von 1.\,1.\,1853 Wien – 15.\,7.\,1922 Gmunden@\textsc{Cuny-Pierron, Rudolf Eduard von} (1.\,1.\,1853 Wien – 15.\,7.\,1922 Gmunden), \emph{Kaufmann}|pw}{ }\textsc{\label{K_L02960-14v}\edtext{\begin{otherlanguage}{french}l’amant de\end{otherlanguage}}{\lemma{\textnormal{\emph{l’amant de}}}\Cendnote{\textnormal{französisch: Liebhaber von}}}\label{K_L02960-14}
                     M A. D.\pwindex{Cuny-Pierron, Antonie 19.\,11.\,1871 Wien – 15.\,12.\,1962 ebd.@\textsc{Cuny-Pierron, Antonie} (19.\,11.\,1871 Wien – 15.\,12.\,1962 ebd.), \emph{Sängerin}|pw}} an mir vorbei, \label{K_L02960-15v}\edtext{\begin{otherlanguage}{french}Cretin\end{otherlanguage}}{\lemma{\textnormal{\emph{Cretin}}}\Cendnote{\textnormal{französisch: Dummkopf,
                     Idiot}}}\label{K_L02960-15}!}{\lemma{\textnormal{\emph{Nach … Cretin!}}}\Cendnote{\textnormal{in einem gezeichneten
                     Kasten quer zum Text}}}\label{T_L02960-1}\pend
           \selectlanguage{ngerman}\endnumbering\briefempfaengerindex{Salten, Felix@\textsc{Salten, Felix}!zzzSchnitzler, Arthur@\emph{von Arthur Schnitzler}!1893-08-141@{{[}14. 8. 1893{]}}|)be}\mylabel{L02960h}  \newcommand{\dateiname}{L02960}\newcommand{\titel}{Arthur Schnitzler an Felix Salten, [14. 8. 1893]}\newcommand{\editorInnen}{Martin Anton Müller und Laura Untner}%% latex-leseansicht-abspann.tex
%% Abspann für die Leseansicht.
%% Der Schalter \ifkorrekturansicht ist bereits durch den Vorspann gesetzt.

%% latex-abspann.tex
%% Gemeinsamer Abspann für Korrekturansicht und Leseansicht.
%% Setzt den Schalter \ifkorrekturansicht voraus (gesetzt in den
%% einbindenden Dateien latex-korrekturansicht-abspann.tex bzw.
%% latex-leseansicht-abspann.tex).
%% ---------------------------------------------------------------

\normalsize

% Das esempio-Environment wird nur in der Leseansicht benötigt
\ifkorrekturansicht\else
\newenvironment{esempio}[3]%
{
    \vspace{1.5ex}
    \rlap{\underline{#1}}
    \par
    \setlength{\parindent}{0cm}
    \nopagebreak
    \leftskip=#2cm
    \rightskip=#3cm
}
{
    \par
}
\fi

\doendnotes{C}
\bigskip
\vfill

\clearpage

\footnotesize

\ifkorrekturansicht
  \lohead{\textsc{register}}
\fi

% theindex-Environment neu definieren ohne reledmac
\makeatletter
\renewenvironment{theindex}{%
  \ifkorrekturansicht
    \section*{\indexname}%
  \else
    \subsubsection*{Index der erwähnten Entitäten}%
  \fi
  \setlength{\parindent}{0pt}%
  \setlength{\parskip}{0pt plus 0.3pt}%
  \let\item\@idxitem
}{%
  \ifkorrekturansicht\clearpage\fi
}
\makeatother

\IfFileExists{\jobname-pw.ind}{\input{\jobname-pw.ind}}{}

% Quellenangabe nur in der Leseansicht
\ifkorrekturansicht\else
% Fallback-Definitionen, falls die .tex-Datei \titel etc. nicht gesetzt hat
\providecommand{\titel}{}
\providecommand{\editorInnen}{}
\providecommand{\dateiname}{\jobname}

\vspace{3cm}

\vfill

\footnotesize
\textsc{Quelle}: \titel. Herausgegeben von {\editorInnen}. In: \emph{Arthur Schnitzler: Briefwechsel mit Autorinnen und Autoren}.
 Digitale Edition, https://schnitzler-briefe.acdh.oeaw.ac.at/{\dateiname}.html (Stand \today)
\fi

\end{document}


