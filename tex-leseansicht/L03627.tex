%% latex-korrekturansicht-vorspann.tex
%% Vorspann für die Korrekturansicht.
%% Lädt die gemeinsame Datei latex-vorspann.tex mit gesetztem Schalter.

\newif\ifkorrekturansicht
\korrekturansichttrue

\input{../tex-inputs/latex-vorspann}


\section[Stefan Zweig an Arthur Schnitzler, 4. 2. 191{[}1{]}]{L03627 Stefan Zweig an Arthur Schnitzler, 4. 2. 191{[}1{]}}
\nopagebreak\mylabel{L03627v}
\rehead{ }\normalsize\beginnumbering\briefempfaengerindex{Schnitzler, Arthur@\textsc{Schnitzler, Arthur}!zzzZweig, Stefan@\emph{von Stefan Zweig}!1911-02-041@{4. 2. 191{[}1{]}}|(be}
\toendnotes[C]{\smallbreak\pagebreak[2]}\Standort{CUL, Schnitzler, B 118.}
\physDesc{Brief, 1 Blatt, 2 Seiten, 1090 Zeichen
\newline{}Handschrift: lila Tinte, lateinische Kurrent
\newline{}Schnitzler: mit Bleistift »\textsc{Zweig}« }
\buchAbdrucke{\weitereDrucke{Stefan Zweig: \emph{Briefwechsel mit Hermann Bahr, Sigmund Freud, Rainer
                                Maria Rilke und Arthur Schnitzler}. Frankfurt am Main: \emph{S. Fischer} 1987, S. 362.} }\toendnotes[C]{\smallbreak}
\pstart
           {\pb}Wien\oindex{Wien@\textbf{Wien}, \emph{A.ADM2}|pw}{ }4. Februar \label{K_L03627-1v}\edtext{1910}{\lemma{\textnormal{\emph{1910}}}\Cendnote{\textnormal{Schreibfehler Zweigs\pwindex{Zweig, Stefan 28.11.1881 – 23.02.1942@\textsc{Zweig, Stefan} (28.11.1881 – 23.02.1942), \emph{Schriftsteller/Schriftstellerin}|pwk}, 
                                der sich durch den Bezug auf den Gesangsauftritt von Olga Schnitzler\pwindex{Schnitzler, Olga 17.01.1882 – 13.01.1970@\textsc{Schnitzler, Olga} (17.01.1882 – 13.01.1970), \emph{Schauspieler/Schauspielerin, Sänger/Sängerin}|pwk}\eventindex{Volkshochschule Ottakring@\textbf{Volkshochschule Ottakring}!Gesangskonzert Olga Schnitzler, 5.2.1911@Gesangskonzert Olga Schnitzler, 5.2.1911|pwkv}
                                im Volksheim\oindex{Volkshochschule Ottakring@\textbf{Volkshochschule Ottakring}, \emph{Gebäude (K.GBD)}|pwk} am 5. 2. 1911
                        richtigstellen lässt.}}}\label{K_L03627-1}\pend
           
\pstart{}Sehr verehrter Herr Doktor,\pend\vspace{0.5em}
\pstart
           ich danke Ihnen viel – vielmals für das »\label{K_L03627-2v}\edtext{Weite
                        Land\pwindex{weite Land. Tragikomoedie in fuenf Akten@\emph{Das weite Land. Tragikomödie in fünf Akten}|pw}}{\lemma{\textnormal{\emph{Weite
                        Land}}}\Cendnote{\textnormal{Es handelte sich um das Bühnenmanuskript, die gedruckte
                        Fassung erschien erst im September 1911.}}}\label{K_L03627-2}«, das ich natürlich sofort mit aller Ungeduld der Erwartung
                    gelesen habe. Und freue mich, dass ich da kein Urteil habe, sondern nur einen
                    restlosen und innigen Glückwunsch. Es ist wie eine Zusammenfassung Ihres ganzen
                    Werkes in einen Rahmen, ungemein reich (erst beim zweiten Lesen tut sich einem
                    die hinter das anscheinend Zufällige versteckte Schönheit voll auf) und von
                    einer s\substVorne{}\textsuperscript{ch}\substDazwischen{}ü\substHinten{}ssen Reife, saftig und bunt – die schönste Frucht vielleicht aus {\pb}Ihrem Garten. Wie wundervoll, dass
                    Sie in einem Alter, wo andere sich schon zu wiederholen beginnen, erst alles
                    Frühere in einer neuen Zusammenfassung übertreffen, wie tröstend für uns
                    Jüngere, wie herrlich ermutigend! Nun haben Sie mir die Ungeduld genommen, es
                    kennen zu lernen und schon habe ich eine neue: es bald \label{K_L03627-3v}\edtext{auf der Bühne}{\lemma{\textnormal{\emph{auf der Bühne}}}\Cendnote{\textnormal{Die zu diesem Zeitpunkt noch für den Frühling geplante deutschsprachige Uraufführung\eventindex{Burgtheater@\textbf{Burgtheater}!Premiere von Das weite Land, 14.10.1911 [I.]@Premiere von Das weite Land, 14.10.1911 [I.]|pwkv} verzögerte sich für mehrere Monate und fand am 14. 10. 1911 in
                        mehreren Städten parallel statt.}}}\label{K_L03627-3} zu \label{T_L03627-1v}\edtext{sehn}{\lemma{\textnormal{\emph{sehn}}}\Cendnote{\textnormal{Er schreibt: »sein«.}}}\label{T_L03627-1}. Ich weiss es wird ein
                    Triumph werden!\pend
           
\pstart
           Morgen\eventindex{Volkshochschule Ottakring@\textbf{Volkshochschule Ottakring}!Gesangskonzert Olga Schnitzler, 5.2.1911@Gesangskonzert Olga Schnitzler, 5.2.1911|pwv} komme ich ins Volksheim\oindex{Volkshochschule Ottakring@\textbf{Volkshochschule Ottakring}, \emph{Gebäude (K.GBD)}|pw} und bin schon in freudiger
                    Neugierde erregt, Ihre Frau Gemahlin\pwindex{Schnitzler, Olga 17.01.1882 – 13.01.1970@\textsc{Schnitzler, Olga} (17.01.1882 – 13.01.1970), \emph{Schauspieler/Schauspielerin, Sänger/Sängerin}|pwv}{ }
                    singen zu hören. All meine guten Grüsse voraus. In Treue ergeben\pend
           \pstart Ihr \spacefill\mbox{Stefan Zweig}\pend{}\selectlanguage{ngerman}\endnumbering\briefempfaengerindex{Schnitzler, Arthur@\textsc{Schnitzler, Arthur}!zzzZweig, Stefan@\emph{von Stefan Zweig}!1911-02-041@{4. 2. 191{[}1{]}}|)be}\mylabel{L03627h}  \normalsize

\doendnotes{C}
\bigskip
\vfill

\clearpage

\footnotesize

\lohead{\textsc{register}}

% Definiere theindex-Environment komplett neu ohne reledmac
\makeatletter
\renewenvironment{theindex}{%
  \section*{\indexname}%
  \setlength{\parindent}{0pt}%
  \setlength{\parskip}{0pt plus 0.3pt}%
  \let\item\@idxitem
}{%
  \clearpage
}
\makeatother

\IfFileExists{\jobname-pw.ind}{\input{\jobname-pw.ind}}{}

\end{document}

      