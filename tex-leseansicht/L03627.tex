%% latex-leseansicht-vorspann.tex
%% Vorspann für die Leseansicht.
%% Lädt die gemeinsame Datei latex-vorspann.tex mit nicht gesetztem Schalter.

\newif\ifkorrekturansicht
\korrekturansichtfalse

\input{../tex-inputs/latex-vorspann}


\section[Stefan Zweig an Arthur Schnitzler, 4. 2. 191{[}1{]}]{L03627 Stefan Zweig an Arthur Schnitzler, 4. 2. 191[1]}
\nopagebreak\mylabel{L03627v}
\rehead{ }\normalsize\beginnumbering\briefempfaengerindex{Schnitzler, Arthur@\textsc{Schnitzler, Arthur}!zzzZweig, Stefan@\emph{von Stefan Zweig}!1911-02-041@{4. 2. 191[1]}|(be}
\toendnotes[C]{\smallbreak\pagebreak[2]}
\correspDesc{Versand  durch Stefan Zweig am 4. 2. 191[1] in Wien
\newline{}Erhalt  durch Arthur Schnitzler im Zeitraum [4. 2. 1911 – 7. 2. 1911?] in Wien}\toendnotes[C]{\smallbreak}
\Standort{CUL, Schnitzler, B 118.}
\physDesc{Brief, 1 Blatt, 2 Seiten, 1090 Zeichen
\newline{}Handschrift: blaue Tinte, lateinische Kurrent
\newline{}Schnitzler: mit Bleistift »\textsc{Zweig}« }
\buchAbdrucke{\weitereDrucke{Stefan Zweig: \emph{Briefwechsel mit Hermann Bahr, Sigmund Freud, Rainer
                                Maria Rilke und Arthur Schnitzler}. Herausgegeben von Jeffrey B. Berlin, Hans-Ulrich Lindken und Donald A. Prater. Frankfurt am Main: \emph{S. Fischer} 1987, S. 362.} }\toendnotes[C]{\smallbreak}
\pstart
           {\pb}Wien\oindex{Wien@\textbf{Wien}, \emph{Verwaltungsgebiet}|pw}{ }4. Februar \label{K_L03627-1v}\edtext{1910}{\lemma{\textnormal{\emph{1910}}}\Cendnote{\textnormal{Schreibfehler Zweigs\pwindex{Zweig, Stefan 28.\,11.\,1881 Wien – 23.\,2.\,1942 Petrópolis@\textsc{Zweig, Stefan} (28.\,11.\,1881 Wien – 23.\,2.\,1942 Petrópolis), \emph{Schriftsteller}|pwk}, 
                                der sich durch den Bezug auf den Gesangsauftritt von Olga Schnitzler\pwindex{Schnitzler, Olga 17.\,1.\,1882 Wien – 13.\,1.\,1970 Lugano@\textsc{Schnitzler, Olga} (17.\,1.\,1882 Wien – 13.\,1.\,1970 Lugano), \emph{Schauspielerin, Sängerin}|pwk}\eventindex{Volkshochschule Ottakring@\textbf{Volkshochschule Ottakring}!Gesangskonzert Olga Schnitzler, 5.2.1911@Gesangskonzert Olga Schnitzler, 5.2.1911|pwkv}
                                im Volksheim\oindex{Wien@\textbf{Wien}!XVI., Ottakring@\textbf{XVI., Ottakring}!Volkshochschule Ottakring@\textbf{Volkshochschule Ottakring}, \emph{Gebäude}|pwk} am 5. 2. 1911
                        richtigstellen lässt.}}}\label{K_L03627-1}\pend
           
\pstart{}Sehr verehrter Herr Doktor,\pend\vspace{0.5em}
\pstart
           ich danke Ihnen viel – vielmals für das »\label{K_L03627-2v}\edtext{Weite
                        Land\pwindex{Schnitzler, Arthur 15.\,5.\,1862 Wien – 21.\,10.\,1931 ebd.@\textsc{Schnitzler, Arthur} (15.\,5.\,1862 Wien – 21.\,10.\,1931 ebd.), \emph{Schriftsteller, Mediziner}!weite Land. Tragikomödie in fünf Akten@\strich\emph{Das weite Land. Tragikomödie in fünf Akten}|pw}}{\lemma{\textnormal{\emph{Weite
                        Land}}}\Cendnote{\textnormal{Es handelte sich um das Bühnenmanuskript, die gedruckte
                        Fassung erschien erst im September 1911.}}}\label{K_L03627-2}«, das ich natürlich sofort mit aller Ungeduld der Erwartung
                    gelesen habe. Und freue mich, dass ich da kein Urteil habe, sondern nur einen
                    restlosen und innigen Glückwunsch. Es ist wie eine Zusammenfassung Ihres ganzen
                    Werkes in einen Rahmen, ungemein reich (erst beim zweiten Lesen tut sich einem
                    die hinter das anscheinend Zufällige versteckte Schönheit voll auf) und von
                    einer s\substVorne{}\textsuperscript{ch}\substDazwischen{}ü\substHinten{}ssen Reife, saftig und bunt – die schönste Frucht vielleicht aus {\pb}Ihrem Garten. Wie wundervoll, dass
                    Sie in einem Alter, wo andere sich schon zu wiederholen beginnen, erst alles
                    Frühere in einer neuen Zusammenfassung übertreffen, wie tröstend für uns
                    Jüngere, wie herrlich ermutigend! Nun haben Sie mir die Ungeduld genommen, es
                    kennen zu lernen und schon habe ich eine neue: es bald \label{K_L03627-3v}\edtext{auf der Bühne}{\lemma{\textnormal{\emph{auf der Bühne}}}\Cendnote{\textnormal{Die zu diesem Zeitpunkt noch für den Frühling geplante deutschsprachige Uraufführung\eventindex{Burgtheater@\textbf{Burgtheater}!Premiere von Das weite Land, 14.10.1911 [I.]@Premiere von Das weite Land, 14.10.1911 [I.]|pwkv} verzögerte sich für mehrere Monate und fand am 14. 10. 1911 in
                        mehreren Städten parallel statt.}}}\label{K_L03627-3} zu \label{T_L03627-1v}\edtext{sehn}{\lemma{\textnormal{\emph{sehn}}}\Cendnote{\textnormal{Er schreibt: »sein«.}}}\label{T_L03627-1}. Ich weiss es wird ein
                    Triumph werden!\pend
           
\pstart
           Morgen\eventindex{Volkshochschule Ottakring@\textbf{Volkshochschule Ottakring}!Gesangskonzert Olga Schnitzler, 5.2.1911@Gesangskonzert Olga Schnitzler, 5.2.1911|pwv} komme ich ins Volksheim\oindex{Wien@\textbf{Wien}!XVI., Ottakring@\textbf{XVI., Ottakring}!Volkshochschule Ottakring@\textbf{Volkshochschule Ottakring}, \emph{Gebäude}|pw} und bin schon in freudiger
                    Neugierde erregt, Ihre Frau Gemahlin\pwindex{Schnitzler, Olga 17.\,1.\,1882 Wien – 13.\,1.\,1970 Lugano@\textsc{Schnitzler, Olga} (17.\,1.\,1882 Wien – 13.\,1.\,1970 Lugano), \emph{Schauspielerin, Sängerin}|pwv}{ }
                    singen zu hören. All meine guten Grüsse voraus. In Treue ergeben\pend
           \pstart Ihr \spacefill\mbox{Stefan Zweig}\pend{}\selectlanguage{ngerman}\endnumbering\briefempfaengerindex{Schnitzler, Arthur@\textsc{Schnitzler, Arthur}!zzzZweig, Stefan@\emph{von Stefan Zweig}!1911-02-041@{4. 2. 191[1]}|)be}\mylabel{L03627h}  \newcommand{\dateiname}{L03627}\newcommand{\titel}{Stefan Zweig an Arthur Schnitzler, 4. 2. 191[1]}\newcommand{\editorInnen}{Selma Jahnke und Martin Anton Müller}%% latex-leseansicht-abspann.tex
%% Abspann für die Leseansicht.
%% Der Schalter \ifkorrekturansicht ist bereits durch den Vorspann gesetzt.

%% latex-abspann.tex
%% Gemeinsamer Abspann für Korrekturansicht und Leseansicht.
%% Setzt den Schalter \ifkorrekturansicht voraus (gesetzt in den
%% einbindenden Dateien latex-korrekturansicht-abspann.tex bzw.
%% latex-leseansicht-abspann.tex).
%% ---------------------------------------------------------------

\normalsize

% Das esempio-Environment wird nur in der Leseansicht benötigt
\ifkorrekturansicht\else
\newenvironment{esempio}[3]%
{
    \vspace{1.5ex}
    \rlap{\underline{#1}}
    \par
    \setlength{\parindent}{0cm}
    \nopagebreak
    \leftskip=#2cm
    \rightskip=#3cm
}
{
    \par
}
\fi

\doendnotes{C}
\bigskip
\vfill

\clearpage

\footnotesize

\ifkorrekturansicht
  \lohead{\textsc{register}}
\fi

% theindex-Environment neu definieren ohne reledmac
\makeatletter
\renewenvironment{theindex}{%
  \ifkorrekturansicht
    \section*{\indexname}%
  \else
    \subsubsection*{Index der erwähnten Entitäten}%
  \fi
  \setlength{\parindent}{0pt}%
  \setlength{\parskip}{0pt plus 0.3pt}%
  \let\item\@idxitem
}{%
  \ifkorrekturansicht\clearpage\fi
}
\makeatother

\IfFileExists{\jobname-pw.ind}{\input{\jobname-pw.ind}}{}

% Quellenangabe nur in der Leseansicht
\ifkorrekturansicht\else
% Fallback-Definitionen, falls die .tex-Datei \titel etc. nicht gesetzt hat
\providecommand{\titel}{}
\providecommand{\editorInnen}{}
\providecommand{\dateiname}{\jobname}

\vspace{3cm}

\vfill

\footnotesize
\textsc{Quelle}: \titel. Herausgegeben von {\editorInnen}. In: \emph{Arthur Schnitzler: Briefwechsel mit Autorinnen und Autoren}.
 Digitale Edition, https://schnitzler-briefe.acdh.oeaw.ac.at/{\dateiname}.html (Stand \today)
\fi

\end{document}


