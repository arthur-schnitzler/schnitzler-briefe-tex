%% latex-leseansicht-vorspann.tex
%% Vorspann für die Leseansicht.
%% Lädt die gemeinsame Datei latex-vorspann.tex mit nicht gesetztem Schalter.

\newif\ifkorrekturansicht
\korrekturansichtfalse

\input{../tex-inputs/latex-vorspann}

\begin{center}
            \textcolor{red}{ENTWURF. ENTZIFFERUNG NOCH NICHT KORREKTURGELESEN}
                      \end{center}
            
               \section[Richard Beer-Hofmann an Arthur Schnitzler, 20. 10. 1894]{ Richard Beer-Hofmann an Arthur Schnitzler, 20. 10. 1894}\nopagebreak\mylabel{v}\rehead{ }\begin{ledgroupsized}[t]{13cm}\normalsize\beginnumbering\briefempfaengerindex{Schnitzler, Arthur@\textsc{Schnitzler, Arthur}!zzzBeer-Hofmann, Richard@\emph{von Richard Beer-Hofmann}!1894-10-202@{20. 10. 1894}|(be} \toendnotes[C]{\smallbreak\pagebreak[2]} \Standort{CUL, Schnitzler, B 8.}
\physDesc{Brief, 1 Blatt, 4 Seiten
\newline{}Handschrift: Bleistift, lateinische Kurrent
\newline{}Schnitzler: mit Bleistift beschriftet: »\textsc{Bajae} 20 Oct 94« und nummeriert:
            »50« }\buchAbdrucke{\weitereDrucke{Arthur Schnitzler, Richard Beer-Hofmann: \emph{Briefwechsel 1891–1931}. Hg. Konstanze Fliedl. Wien, Zürich: \emph{Europaverlag} 1992, S. 65–66.} }\toendnotes[C]{\smallbreak}\pstart
           \noindent{}{\pb}Lieber Arthur!
                    Gerade, wie ich in den Wagen steige, bekomme ich Ihre Karte. Meinen
                    Brief \strikeout{ha} und Karte haben Sie wohl?\pend
           \pstart
           \uline{Das}
               schreibe ich beim schwarzen Kaffee auf einer
                    Terrasse am Meer in \uline{Bajae\oindex{Baia@\textbf{Baia}|pw}} – (Bitte lesen Sie zu Hause über Bajae\oindex{Baia@\textbf{Baia}|pw}
                    nach.) Abends bin ich wieder in Neapel\oindex{Neapel@\textbf{Neapel}|pw}, dann
                    morgen und die nächsten Tage Capri\oindex{Capri@\textbf{Capri}|pw},
                        Sorrent\oindex{Sorrent@\textbf{Sorrent}|pw} dann Venedig\oindex{Venedig@\textbf{Venedig}|pw}. Adressiren Sie bitte Briefe und die 4. Nr. der
                        Zeit\orgindex{Zeit. Wiener Wochenschrift@Die Zeit. Wiener Wochenschrift|pw} nach Venedig,
                            \uline{Bauer und Grünwald}\oindex{Grand Hotel Bauer-Gruenwald@\textbf{Grand Hotel Bauer-Grünwald}|pw}. – Die 1te und 2. Nu{\geminationm}er habe ich; 3\textsuperscript{te} erwarte ich. {\pb}À propos (warum à propos,
                    warum fällt mir das jetzt ein?) was stand auf den in Verlust gerathenen
                        Pallanza\oindex{Pallanza@\textbf{Pallanza}|pw}er Karten? Bahr\pwindex{Bahr, Hermann 19.07.1863 – 15.01.1934@\textsc{Bahr, Hermann} (19.07.1863 – 15.01.1934), \emph{Schriftsteller, Kritiker}|pw} bitte grüßen Sie herzlich, und der »\label{K_L00388_1v}\edtext{Abonnent\pwindex{Abonnent06. 10. 1894@\emph{Der Abonnent} {[}06. 10. 1894{]}|pw}}{\lemma{\textnormal{\emph{Abonnent}}}\Cendnote{\textnormal{Caph [= Hermann Bahr]\pwindex{Bahr, Hermann 19.07.1863 – 15.01.1934@\textsc{Bahr, Hermann} (19.07.1863 – 15.01.1934), \emph{Schriftsteller, Kritiker}|pwk}: \emph{Der Abonnent}\pwindex{Abonnent06. 10. 1894@\emph{Der Abonnent} {[}06. 10. 1894{]}|pwk}. In: \emph{Die Zeit}\pwindex{Zeit. Wiener Wochenschrift1894 – 1904@\emph{Die Zeit. Wiener Wochenschrift}|pwk}, Bd. 1, Nr. 1, 6. 10. 1894,
                            S. 6–7.}}}\label{K_L00388_1h}« hat mir »\uline{wol}
                    getan«, und das »\label{K_L00388_2v}\edtext{Burgtheater\pwindex{Abonnent06. 10. 1894@\emph{Der Abonnent} {[}06. 10. 1894{]}|pw}}{\lemma{\textnormal{\emph{Burgtheater}}}\Cendnote{\textnormal{Hermann Bahr\pwindex{Bahr, Hermann 19.07.1863 – 15.01.1934@\textsc{Bahr, Hermann} (19.07.1863 – 15.01.1934), \emph{Schriftsteller, Kritiker}|pwk}: \emph{Burgtheater}\pwindex{Abonnent06. 10. 1894@\emph{Der Abonnent} {[}06. 10. 1894{]}|pwk}. In: \emph{Die Zeit}\pwindex{Zeit. Wiener Wochenschrift1894 – 1904@\emph{Die Zeit. Wiener Wochenschrift}|pwk}, Bd. 1, Nr. 1, 6. 10. 1894,
                            S. 9–10.}}}\label{K_L00388_2h}« (Burkhard\pwindex{Burckhard, Max Eugen 14.07.1854 – 16.03.1912@\textsc{Burckhard, Max Eugen} (14.07.1854 – 16.03.1912), \emph{Schriftsteller, Rechtswissenschaftler, Theaterleiter}|pw})
                    war gescheidt \uline{und} diplomatisch. Und die
                        »Schmetterlingsschlacht\pwindex{\textcolor{red}{\textsuperscript{XXXX1 indx}}!Schmetterlingsschlacht1894@\strich\emph{Die Schmetterlingsschlacht} {[}1894{]}|pw}« hat er sich
                    teilweise eingeredet – ich kenne \strikeout{S}sie nicht, –
                    aber ich mißbillige \strikeout{S}sie. Kleine Probleme von
                    kleinen Warten und anstatt tiefster Auffassung des {\pb}Lebens bürgerlich-ideale
                    Moral auf dem Grunde; und die Belohnung \textcolor{gray}{×}\-\textcolor{gray}{×}\-\textcolor{gray}{×} guter Sitten in reicher Heirath, und die Versorgung, –
                    der Blick in die Zukunft.\pend
           \pstart
           Das Meer ist viel schöner. Und viele andere, viel kleinere Dinge auch. Lieber
                    Arthur, bitte schreiben Sie mir \uline{sehr sicher} nach
                        Venedig\oindex{Venedig@\textbf{Venedig}|pw}, und viel; denn Sie würden
                    unendlich leiden unter dem Gedanken, wie peinlich ich es empfinden müsste in
                        Venedig\oindex{Venedig@\textbf{Venedig}|pw} keinen Brief {\pb}zu finden, nachdem auf der
                    ganzen Fahrt dahin mich drauf gefreut habe.\pend
           \pstart
           Es gibt Studenten des jus in Prag\oindex{Prag@\textbf{Prag}|pw} die sehr
                    gut Lawn-Tennis spielen, nicht antisemitisch, gegen den deutschen Schulverein\orgindex{Deutscher Schulverein@Deutscher Schulverein|pw} und die Politik, und insbesondere den
                    Liberalismus sind; Maupassant\pwindex{Maupassant, Guy de 05.08.1850 – 07.07.1893@\textsc{Maupassant, Guy de} (05.08.1850 – 07.07.1893), \emph{Schriftsteller}|pw} lesen, den
                        Bahr\pwindex{Bahr, Hermann 19.07.1863 – 15.01.1934@\textsc{Bahr, Hermann} (19.07.1863 – 15.01.1934), \emph{Schriftsteller, Kritiker}|pw} teilweise (Dora\pwindex{Bahr, Hermann 19.07.1863 – 15.01.1934@\textsc{Bahr, Hermann} (19.07.1863 – 15.01.1934), \emph{Schriftsteller, Kritiker}!Dora1892@\strich\emph{Dora} {[}1892{]}|pw}) kennen, und freudig erschauern wenn ich sage daß ich
                        Bahr\pwindex{Bahr, Hermann 19.07.1863 – 15.01.1934@\textsc{Bahr, Hermann} (19.07.1863 – 15.01.1934), \emph{Schriftsteller, Kritiker}|pw} kenne (\uline{einen} gibt es \uline{sicher}). Die Leute die
                    heute 17 u. 19 sind, werden die sein die in 10 Jahren sich uns neigen werden –
                    oder früher? Das »uns« nehme ich \uline{principiell}
                    zurück. \spacefill\mbox{Richard.}\pend
           \endnumbering\briefempfaengerindex{Schnitzler, Arthur@\textsc{Schnitzler, Arthur}!zzzBeer-Hofmann, Richard@\emph{von Richard Beer-Hofmann}!1894-10-202@{20. 10. 1894}|)be}\mylabel{h}\end{ledgroupsized}  \newcommand{\dateiname}{L00388}\newcommand{\titel}{Richard Beer-Hofmann an Arthur Schnitzler, 20. 10. 1894}\newcommand{\editorInnen}{ Martin Anton Müller und Gerd-Hermann Susen}%% latex-leseansicht-abspann.tex
%% Abspann für die Leseansicht.
%% Der Schalter \ifkorrekturansicht ist bereits durch den Vorspann gesetzt.

%% latex-abspann.tex
%% Gemeinsamer Abspann für Korrekturansicht und Leseansicht.
%% Setzt den Schalter \ifkorrekturansicht voraus (gesetzt in den
%% einbindenden Dateien latex-korrekturansicht-abspann.tex bzw.
%% latex-leseansicht-abspann.tex).
%% ---------------------------------------------------------------

\normalsize

% Das esempio-Environment wird nur in der Leseansicht benötigt
\ifkorrekturansicht\else
\newenvironment{esempio}[3]%
{
    \vspace{1.5ex}
    \rlap{\underline{#1}}
    \par
    \setlength{\parindent}{0cm}
    \nopagebreak
    \leftskip=#2cm
    \rightskip=#3cm
}
{
    \par
}
\fi

\doendnotes{C}
\bigskip
\vfill

\clearpage

\footnotesize

\ifkorrekturansicht
  \lohead{\textsc{register}}
\fi

% theindex-Environment neu definieren ohne reledmac
\makeatletter
\renewenvironment{theindex}{%
  \ifkorrekturansicht
    \section*{\indexname}%
  \else
    \subsubsection*{Index der erwähnten Entitäten}%
  \fi
  \setlength{\parindent}{0pt}%
  \setlength{\parskip}{0pt plus 0.3pt}%
  \let\item\@idxitem
}{%
  \ifkorrekturansicht\clearpage\fi
}
\makeatother

\IfFileExists{\jobname-pw.ind}{\input{\jobname-pw.ind}}{}

% Quellenangabe nur in der Leseansicht
\ifkorrekturansicht\else
% Fallback-Definitionen, falls die .tex-Datei \titel etc. nicht gesetzt hat
\providecommand{\titel}{}
\providecommand{\editorInnen}{}
\providecommand{\dateiname}{\jobname}

\vspace{3cm}

\vfill

\footnotesize
\textsc{Quelle}: \titel. Herausgegeben von {\editorInnen}. In: \emph{Arthur Schnitzler: Briefwechsel mit Autorinnen und Autoren}.
 Digitale Edition, https://schnitzler-briefe.acdh.oeaw.ac.at/{\dateiname}.html (Stand \today)
\fi

\end{document}


      