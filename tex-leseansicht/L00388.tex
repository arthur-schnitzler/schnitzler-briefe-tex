%% latex-korrekturansicht-vorspann.tex
%% Vorspann für die Korrekturansicht.
%% Lädt die gemeinsame Datei latex-vorspann.tex mit gesetztem Schalter.

\newif\ifkorrekturansicht
\korrekturansichttrue

\input{../tex-inputs/latex-vorspann}


\section[Richard Beer-Hofmann an Arthur Schnitzler, 20. 10. 1894]{L00388 Richard Beer-Hofmann an Arthur Schnitzler, 20. 10. 1894}
\nopagebreak\mylabel{L00388v}
\rehead{ }\normalsize\beginnumbering\briefempfaengerindex{Schnitzler, Arthur@\textsc{Schnitzler, Arthur}!zzzBeer-Hofmann, Richard@\emph{von Richard Beer-Hofmann}!1894-10-202@{20. 10. 1894}|(be}
\toendnotes[C]{\smallbreak\pagebreak[2]}\Standort{CUL, Schnitzler, B 8.}
\physDesc{Brief, 1 Blatt, 4 Seiten, 1802 Zeichen
\newline{}Handschrift: Bleistift, lateinische Kurrent
\newline{}Schnitzler: mit Bleistift beschriftet: »\textsc{Bajae} 20 Oct 94« und
                                 nummeriert: »50« }
\buchAbdrucke{\weitereDrucke{Arthur Schnitzler, Richard Beer-Hofmann: \emph{Briefwechsel 1891–1931}. Wien, Zürich: \emph{Europaverlag} 1992, S. 65–66.} }\toendnotes[C]{\smallbreak}
\pstart
           \noindent{}{\pb}Lieber Arthur! Gerade, wie ich in den Wagen steige, bekomme ich Ihre
               Karte. Meinen Brief \strikeout{ha} und Karte haben Sie wohl?\pend
           
\pstart
           \uline{Das} schreibe ich beim schwarzen Kaffee auf einer
               Terrasse am Meer in \uline{Bajae\oindex{Baia@\textbf{Baia}, \emph{P.PPL}|pw}} – (Bitte lesen Sie zu Hause über Bajae\oindex{Baia@\textbf{Baia}, \emph{P.PPL}|pw}
               nach.) Abends bin ich wieder in Neapel\oindex{Neapel@\textbf{Neapel}, \emph{P.PPLA}|pw}, dann
               morgen und die nächsten Tage Capri\oindex{Capri@\textbf{Capri}, \emph{T.ISL}|pw}, Sorrent\oindex{Sorrent@\textbf{Sorrent}, \emph{P.PPLA3}|pw} dann Venedig\oindex{Venedig@\textbf{Venedig}, \emph{P.PPLA}|pw}. Adressiren Sie bitte Briefe und die 4. Nr. der Zeit\orgindex{Zeit. Wiener Wochenschrift@Die Zeit. Wiener Wochenschrift|pw} nach Venedig, \uline{Bauer und Grünwald}\oindex{Grand Hotel Bauer-Gruenwald@\textbf{Grand Hotel Bauer-Grünwald}, \emph{Hotel (K.HTL)}|pw}. – Die 1te und 2. Nu{\geminationm}er habe ich; 3\textsuperscript{te} erwarte ich. {\pb}À propos (warum à propos, warum
               fällt mir das jetzt ein?) was stand auf den in Verlust gerathenen Pallanza\oindex{Pallanza@\textbf{Pallanza}, \emph{P.PPL}|pw}er Karten? Bahr\pwindex{Bahr, Hermann 19.07.1863 – 15.01.1934@\textsc{Bahr, Hermann} (19.07.1863 – 15.01.1934), \emph{Schriftsteller/Schriftstellerin, Kritiker/Kritikerin}|pw}
               bitte grüßen Sie herzlich, und der »\label{K_L00388-1v}\edtext{Abonnent\pwindex{Abonnent@\emph{Der Abonnent}|pw}}{\lemma{\textnormal{\emph{Abonnent}}}\Cendnote{\textnormal{Caph [ = Hermann Bahr]\pwindex{Bahr, Hermann 19.07.1863 – 15.01.1934@\textsc{Bahr, Hermann} (19.07.1863 – 15.01.1934), \emph{Schriftsteller/Schriftstellerin, Kritiker/Kritikerin}|pwk}: \emph{Der Abonnent}\pwindex{Abonnent@\emph{Der Abonnent}|pwk}. In: \emph{Die Zeit}\pwindex{Zeit. Wiener Wochenschrift@\emph{Die Zeit. Wiener Wochenschrift}|pwk}, Bd. 1, Nr. 1, 6. 10. 1894,
                     S. 6–7.}}}\label{K_L00388-1}« hat mir »\uline{wol} getan«, und
               das »\label{K_L00388-2v}\edtext{Burgtheater\pwindex{Abonnent@\emph{Der Abonnent}|pw}}{\lemma{\textnormal{\emph{Burgtheater}}}\Cendnote{\textnormal{Hermann Bahr\pwindex{Bahr, Hermann 19.07.1863 – 15.01.1934@\textsc{Bahr, Hermann} (19.07.1863 – 15.01.1934), \emph{Schriftsteller/Schriftstellerin, Kritiker/Kritikerin}|pwk}: \emph{Burgtheater}\pwindex{Abonnent@\emph{Der Abonnent}|pwk}. In: \emph{Die
                        Zeit}\pwindex{Zeit. Wiener Wochenschrift@\emph{Die Zeit. Wiener Wochenschrift}|pwk}, Bd. 1, Nr. 1, 6. 10. 1894, S. 9–10.}}}\label{K_L00388-2}«
                  (Burkhard\pwindex{Burckhard, Max Eugen 14.07.1854 – 16.03.1912@\textsc{Burckhard, Max Eugen} (14.07.1854 – 16.03.1912), \emph{Schriftsteller/Schriftstellerin, Rechtswissenschaftler/Rechtswissenschaftlerin, Theaterleiter/Theaterleiterin}|pw}) war gescheidt \uline{und} diplomatisch. Und die »Schmetterlingsschlacht\pwindex{Schmetterlingsschlacht. Komoedie in 4 Akten@\emph{Die Schmetterlingsschlacht. Komödie in 4 Akten}|pw}« hat er sich teilweise eingeredet – ich
               kenne \strikeout{S}sie nicht, – aber ich mißbillige \strikeout{S}sie. Kleine Probleme von kleinen Warten und anstatt
               tiefster Auffassung des {\pb}Lebens
               bürgerlich-ideale Moral auf dem Grunde; und die Belohnung \textcolor{gray}{×}\-\textcolor{gray}{×}\-\textcolor{gray}{×} guter Sitten in reicher Heirath, und die
               Versorgung, – der Blick in die Zukunft.\pend
           
\pstart
           Das Meer ist viel schöner. Und viele andere, viel kleinere Dinge auch. Lieber Arthur,
               bitte schreiben Sie mir \uline{sehr sicher} nach Venedig\oindex{Venedig@\textbf{Venedig}, \emph{P.PPLA}|pw}, und viel; denn Sie würden unendlich
               leiden unter dem Gedanken, wie peinlich ich es empfinden müsste in Venedig\oindex{Venedig@\textbf{Venedig}, \emph{P.PPLA}|pw} keinen Brief {\pb}zu finden, nachdem auf der ganzen
               Fahrt dahin mich drauf gefreut habe.\pend
           
\pstart
           Es gibt Studenten des jus in Prag\oindex{Prag@\textbf{Prag}, \emph{A.ADM1}|pw} die sehr gut
               Lawn-Tennis spielen, nicht antisemitisch, gegen den deutschen Schulverein\orgindex{Deutscher Schulverein@Deutscher Schulverein|pw} und die Politik, und insbesondere den Liberalismus
               sind; Maupassant\pwindex{Maupassant, Guy de 05.08.1850 – 07.07.1893@\textsc{Maupassant, Guy de} (05.08.1850 – 07.07.1893), \emph{Schriftsteller/Schriftstellerin}|pw} lesen, den Bahr\pwindex{Bahr, Hermann 19.07.1863 – 15.01.1934@\textsc{Bahr, Hermann} (19.07.1863 – 15.01.1934), \emph{Schriftsteller/Schriftstellerin, Kritiker/Kritikerin}|pw} teilweise (Dora\pwindex{Dora@\emph{Dora}|pw}) kennen, und freudig erschauern wenn ich sage daß ich Bahr\pwindex{Bahr, Hermann 19.07.1863 – 15.01.1934@\textsc{Bahr, Hermann} (19.07.1863 – 15.01.1934), \emph{Schriftsteller/Schriftstellerin, Kritiker/Kritikerin}|pw} kenne (\uline{einen} gibt es \uline{sicher}). Die Leute die heute
               17 u. 19 sind, werden die sein die in 10 Jahren sich uns neigen werden – oder früher?
               Das »uns« nehme ich \uline{principiell} zurück.
                  \spacefill\mbox{Richard.}\pend
           \selectlanguage{ngerman}\endnumbering\briefempfaengerindex{Schnitzler, Arthur@\textsc{Schnitzler, Arthur}!zzzBeer-Hofmann, Richard@\emph{von Richard Beer-Hofmann}!1894-10-202@{20. 10. 1894}|)be}\mylabel{L00388h}  \normalsize

\doendnotes{C}
\bigskip
\vfill

\clearpage

\footnotesize

\lohead{\textsc{register}}

% Definiere theindex-Environment komplett neu ohne reledmac
\makeatletter
\renewenvironment{theindex}{%
  \section*{\indexname}%
  \setlength{\parindent}{0pt}%
  \setlength{\parskip}{0pt plus 0.3pt}%
  \let\item\@idxitem
}{%
  \clearpage
}
\makeatother

\IfFileExists{\jobname-pw.ind}{\input{\jobname-pw.ind}}{}

\end{document}

      