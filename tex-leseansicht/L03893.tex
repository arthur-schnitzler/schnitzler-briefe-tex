%% latex-leseansicht-vorspann.tex
%% Vorspann für die Leseansicht.
%% Lädt die gemeinsame Datei latex-vorspann.tex mit nicht gesetztem Schalter.

\newif\ifkorrekturansicht
\korrekturansichtfalse

\input{../tex-inputs/latex-vorspann}


\section[Theodor Herzl an Arthur Schnitzler, {[}20. oder 21. 1. 1895?{]}]{L03893 Theodor Herzl an Arthur Schnitzler, {[}20. oder 21. 1. 1895?{]}}
\nopagebreak\mylabel{L03893v}
\rehead{ }\normalsize\beginnumbering\briefempfaengerindex{Schnitzler, Arthur@\textsc{Schnitzler, Arthur}!zzzHerzl, Theodor@\emph{von Theodor Herzl}!1895-01-212@{{[}20. oder 21. 1. 1895?{]}}|(be}
\toendnotes[C]{\smallbreak\pagebreak[2]}
\correspDesc{Versand  durch Theodor Herzl im Zeitraum [20. oder
                  21. 1. 1895?] in Paris
\newline{}Erhalt  durch Arthur Schnitzler im Zeitraum [2. 1. 1895
                  – 6. 1. 1895?] in Wien}\toendnotes[C]{\smallbreak}
\Standort{Wien, Österreichische Gesellschaft für Literatur, Abschrift Herzl.}
\physDesc{Telegramm, maschinenschriftliche Abschrift, 63 Zeichen
\newline{}maschinell
\newline{}Zusatz: In der Nachlassmappe B 39 hat Heinrich Schnitzler\pwindex{Schnitzler, Heinrich 9.\,8.\,1902 Hinterbrühl – 12.\,7.\,1982 Wien@\textsc{Schnitzler, Heinrich} (9.\,8.\,1902 Hinterbrühl – 12.\,7.\,1982 Wien), \emph{Regisseur, Schauspieler}|pw} vermerkt: »\noindent{}2 Briefe
                                       geschenkt ans Wolf-Museum Eisenstadt\orgindex{Landesmuseum Burgenland@Landesmuseum Burgenland|pw}{ }22. VIII. 1937.{ / }1 Brief entnommen{ / }1 Brief geschenkt an Paul Marx\pwindex{Marx, Paul 21.\,7.\,1879 Wien – 30.\,10.\,1956 ebd.@\textsc{Marx, Paul} (21.\,7.\,1879 Wien – 30.\,10.\,1956 ebd.), \emph{Regisseur, Schauspieler}|pw}{ }15. VIII. 1936.{ / }1 Brief gegeben an Mutter\pwindex{Schnitzler, Olga 17.\,1.\,1882 Wien – 13.\,1.\,1970 Lugano@\textsc{Schnitzler, Olga} (17.\,1.\,1882 Wien – 13.\,1.\,1970 Lugano), \emph{Schauspielerin, Sängerin}|pwv}, 15. VIII. 36.« Das entspricht
                                 der Anzahl von fünf Korrespondenzstücken von Herzl, die nicht im Original überliefert sind. Alle finden sich in einer Abschrift, die nach
                                 Arthur Schnitzlers Tod im Zeitraum 1932 bis 1936 entstanden sein dürfte. }
\buchAbdrucke{\weitereDrucke{Theodor Herzl: \emph{Briefe und
                        autobiographische Notizen 1866–1895}. Bearbeitet von Johannes Wachten in Zusammenarbeit mit Chaya Harel, Daisy Tycho und Manfred Winkler. Berlin, Frankfurt am Main, Wien: \emph{Propyläen} 1983, S. 570 (Briefe und Tagebücher. Herausgegeben von Alex Bein, Hermann Greive, Moshe Schaerf, Julius H. Schoeps und Johannes Wachten, 1).} }\toendnotes[C]{\smallbreak}
\pstart
           {\pb}H 29\pend
           
\pstart
           \centering{}\so{Telegramm}\pend
           
\pstart
           \raggedleft{}Paris\oindex{Paris@\textbf{Paris}, \emph{Hauptstadt}|pw}, Ende Jänner 95.\pend
           \vspace{0.5em}
\pstart
           HOCHERFREUT. \label{K_L03893-1v}\edtext{GRATULIERE}{\lemma{\textnormal{\emph{Gratuliere}}}\Cendnote{\textnormal{Das Original des Telegramms nicht erhalten. Die
                  überliefernde, posthum entstandene Schreibmaschinenabschrift, die sich in der
                  Österreichische Gesellschaft für Literatur (Wien\oindex{Wien@\textbf{Wien}, \emph{Verwaltungsgebiet}|pwk}) befindet, datiert das Telegramm auf Ende Januar 1895, doch es war vermutlich die erste Reaktion auf die
                  Nachricht aus Schnitzlers Brief vom
                  refXXXX19.1.1895, dass die \emph{Liebelei}\pwindex{Schnitzler, Arthur 15.\,5.\,1862 Wien – 21.\,10.\,1931 ebd.@\textsc{Schnitzler, Arthur} (15.\,5.\,1862 Wien – 21.\,10.\,1931 ebd.), \emph{Schriftsteller, Mediziner}!Liebelei. Schauspiel in drei Akten@\strich\emph{Liebelei. Schauspiel in drei Akten}|pwk} Aufnahme
                  am \emph{Burgtheater}\orgindex{Burgtheater@Burgtheater|pwk} gefunden habe, und entstand
                  direkt nach Erhalt des Briefes. Kurz darauf wiederholte Herzl\pwindex{Herzl, Theodor 2.\,5.\,1860 Budapest – 3.\,7.\,1904 Edlach@\textsc{Herzl, Theodor} (2.\,5.\,1860 Budapest – 3.\,7.\,1904 Edlach), \emph{Schriftsteller, Journalist}|pwk} seine Gratulation ein weiteres Mal in einem Brief,
                  der ebenfalls in den Tagen ab dem 20. 1. 1895 verfasst wurde, siehe XXXX Auszeichnungsfehler: Dokument L03846 nicht gefunden.}}}\label{K_L03893-1}. HERZL.\pend
           \selectlanguage{ngerman}\endnumbering\briefempfaengerindex{Schnitzler, Arthur@\textsc{Schnitzler, Arthur}!zzzHerzl, Theodor@\emph{von Theodor Herzl}!1895-01-202@{{[}20. oder 21. 1. 1895?{]}}|)be}\mylabel{L03893h}
\begin{anhang}
\end{anhang}\newcommand{\dateiname}{L03893}\newcommand{\titel}{Theodor Herzl an Arthur Schnitzler, [20. oder 21. 1. 1895?]}\newcommand{\editorInnen}{Selma Jahnke und Martin Anton Müller}%% latex-leseansicht-abspann.tex
%% Abspann für die Leseansicht.
%% Der Schalter \ifkorrekturansicht ist bereits durch den Vorspann gesetzt.

%% latex-abspann.tex
%% Gemeinsamer Abspann für Korrekturansicht und Leseansicht.
%% Setzt den Schalter \ifkorrekturansicht voraus (gesetzt in den
%% einbindenden Dateien latex-korrekturansicht-abspann.tex bzw.
%% latex-leseansicht-abspann.tex).
%% ---------------------------------------------------------------

\normalsize

% Das esempio-Environment wird nur in der Leseansicht benötigt
\ifkorrekturansicht\else
\newenvironment{esempio}[3]%
{
    \vspace{1.5ex}
    \rlap{\underline{#1}}
    \par
    \setlength{\parindent}{0cm}
    \nopagebreak
    \leftskip=#2cm
    \rightskip=#3cm
}
{
    \par
}
\fi

\doendnotes{C}
\bigskip
\vfill

\clearpage

\footnotesize

\ifkorrekturansicht
  \lohead{\textsc{register}}
\fi

% theindex-Environment neu definieren ohne reledmac
\makeatletter
\renewenvironment{theindex}{%
  \ifkorrekturansicht
    \section*{\indexname}%
  \else
    \subsubsection*{Index der erwähnten Entitäten}%
  \fi
  \setlength{\parindent}{0pt}%
  \setlength{\parskip}{0pt plus 0.3pt}%
  \let\item\@idxitem
}{%
  \ifkorrekturansicht\clearpage\fi
}
\makeatother

\IfFileExists{\jobname-pw.ind}{\input{\jobname-pw.ind}}{}

% Quellenangabe nur in der Leseansicht
\ifkorrekturansicht\else
% Fallback-Definitionen, falls die .tex-Datei \titel etc. nicht gesetzt hat
\providecommand{\titel}{}
\providecommand{\editorInnen}{}
\providecommand{\dateiname}{\jobname}

\vspace{3cm}

\vfill

\footnotesize
\textsc{Quelle}: \titel. Herausgegeben von {\editorInnen}. In: \emph{Arthur Schnitzler: Briefwechsel mit Autorinnen und Autoren}.
 Digitale Edition, https://schnitzler-briefe.acdh.oeaw.ac.at/{\dateiname}.html (Stand \today)
\fi

\end{document}


