%% latex-leseansicht-vorspann.tex
%% Vorspann für die Leseansicht.
%% Lädt die gemeinsame Datei latex-vorspann.tex mit nicht gesetztem Schalter.

\newif\ifkorrekturansicht
\korrekturansichtfalse

\input{../tex-inputs/latex-vorspann}

\begin{center}
            \textcolor{red}{ENTWURF, NICHT FERTIG KORRIGIERT}
                      \end{center}
            
         
         \renewcommand{\erwaehntePersonen}{Personen: Paul Goldmann}
         \renewcommand{\erwaehnteInstitutionen}{Institutionen: S. Fischer Verlag}
         \renewcommand{\erwaehnteOrte}{Orte: Berlin, Wien}
         \renewcommand{\erwaehnteWerke}{Werke: Berliner Theater. (»Der Schleier der Beatrice« von Arthur Schnitzler.), Der Schleier der Beatrice. Schauspiel in fünf Akten, Komödie der Worte. Drei Einakter, Lebendige Stunden. Vier Einakter, Neue Freie Presse, Stunde des Erkennens, Tagebuch}
               \section[ Paul Goldmann an Arthur Schnitzler, 16. 2. 1925]{ Paul Goldmann an Arthur Schnitzler, 16. 2. 1925}\nopagebreak\mylabel{v}\rehead{ }\begin{ledgroupsized}[t]{13cm}\normalsize\beginnumbering \toendnotes[C]{\smallbreak\pagebreak[2]} \Standort{DLA, A:Schnitzler, HS.NZ85.1.3176.}
\physDesc{Brief, 3 Blätter, 11 Seiten, 5130 Zeichen
\newline{}Handschrift: lila Tinte, deutsche Kurrent
\newline{}Schnitzler: mit rotem Buntstift siebzehn Unterstreichungen }\toendnotes[C]{\smallbreak}\pstart
           {\pb}Berlin\oindex{Berlin@\textbf{Berlin}|pw}, 16. 2. 25.\pend
           \pstart{}Lieber Arthur,\pend\pstart
           Es hat mich ſehr gerührt, daß Du mir zu meinem \label{K_L03478-1v}\edtext{60. Geburtstage}{\lemma{\textnormal{\emph{60. Geburtstage}}}\Cendnote{\textnormal{am 31. 1. 1925}}}\label{K_L03478-1h} gratulirt haſt, u. ich danke Dir von Herzen für Deinen Brief. Er hat mich
               erfreut – u. ein wenig beſchämt. Denn als Du \label{K_L03478-2v}\edtext{vor wenigen Jahren}{\lemma{\textnormal{\emph{vor wenigen Jahren}}}\Cendnote{\textnormal{am 15. 5. 1922}}}\label{K_L03478-2h} Deinen 60. Geburtstag gefeiert haſt, wollte ich Dir ſchreiben, brachte es
               aber nicht über mich, weil ich den Ton nicht finden konnte. Ohne Dir zu ſchreiben,
               habe ich Dir aber, glaube es mir!, alles Gute gewünſcht, wie ich überhaupt, von fern
                  {\pb}u. in \strikeout{aller}
               der Stille, an allen Deinen Lebensſchickſalen ftets den herzlichſten Anteil genommen
               habe.\pend
           \pstart
           In unſeren Jahren – traurig, nicht wahr?, daß wir bereits »in unſeren Jahren« ſind! –
               vermeidet man gern Ausſprachen u. läßt die Dinge beſtehen, wie das Leben ſie
               geſtaltet hat. Ich habe aber das Gefühl, daß Dein Brief mich zu einer Angabe von
               Gründen für mein Verhalten verpflichtet, u. daß ich Dir für die ſchönen Worte, die Du
               mir geſchrieben haſt, volle Offenheit ſchulde.\pend
           \pstart
           Unſere Wege haben ſich vor Jahren getrennt. {\pb}Es
               gab damals einen \label{K_L03478-3v}\edtext{Streit}{\lemma{\textnormal{\emph{Streit}}}\Cendnote{\textnormal{Persönlich hatten sie am 26. 12. 1910 und vor
                  allem am 28. 12. 1910
                  gestritten. Zum großen Bruch war es dann Anfang 1911
                  gekommen, siehe Paul Goldmann an Arthur Schnitzler, 13. 1. 1911.}}}\label{K_L03478-3h}
               zwiſchen uns. Du hatteſt mir vorgeworfen, daß ich über \label{K_L03478-4v}\edtext{eines\pwindex{Schnitzler, Arthur 15.05.1862 – 21.10.1931@\textsc{Schnitzler, Arthur} (15.05.1862 – 21.10.1931), \emph{Schriftsteller, Mediziner}!Schleier der Beatrice. Schauspiel in fuenf Akten1900-12-01@\strich\emph{Der Schleier der Beatrice. Schauspiel in fünf Akten} {[}1900-12-01{]}|pwv} Deiner Stücke}{\lemma{\textnormal{\emph{eines Deiner Stücke}}}\Cendnote{\textnormal{In dem Streit war es um \emph{Der Schleier der Beatrice}\pwindex{Schnitzler, Arthur 15.05.1862 – 21.10.1931@\textsc{Schnitzler, Arthur} (15.05.1862 – 21.10.1931), \emph{Schriftsteller, Mediziner}!Schleier der Beatrice. Schauspiel in fuenf Akten1900-12-01@\strich\emph{Der Schleier der Beatrice. Schauspiel in fünf Akten} {[}1900-12-01{]}|pwk} und \emph{Lebendige Stunden}\pwindex{Schnitzler, Arthur 15.05.1862 – 21.10.1931@\textsc{Schnitzler, Arthur} (15.05.1862 – 21.10.1931), \emph{Schriftsteller, Mediziner}!Lebendige Stunden. Vier Einakter1901-12-23@\strich\emph{Lebendige Stunden. Vier Einakter} {[}1901-12-23{]}|pwk} gegangen. Hier bezog sich Goldmann\pwindex{Goldmann, Paul 31.01.1865 – 25.09.1935@\textsc{Goldmann, Paul} (31.01.1865 – 25.09.1935), \emph{Schriftsteller, Journalist}|pwk} auf die \emph{Beatrice}\pwindex{Schnitzler, Arthur 15.05.1862 – 21.10.1931@\textsc{Schnitzler, Arthur} (15.05.1862 – 21.10.1931), \emph{Schriftsteller, Mediziner}!Schleier der Beatrice. Schauspiel in fuenf Akten1900-12-01@\strich\emph{Der Schleier der Beatrice. Schauspiel in fünf Akten} {[}1900-12-01{]}|pwk} und seine Kritik\pwindex{Goldmann, Paul 31.01.1865 – 25.09.1935@\textsc{Goldmann, Paul} (31.01.1865 – 25.09.1935), \emph{Schriftsteller, Journalist}!Berliner Theater. (»Der Schleier der Beatrice« von Arthur Schnitzler.)1903-03-19@\strich\emph{Berliner Theater. (»Der Schleier der Beatrice« von Arthur Schnitzler.)} {[}1903-03-19{]}|pwkv} darüber: Paul Goldmann\pwindex{Goldmann, Paul 31.01.1865 – 25.09.1935@\textsc{Goldmann, Paul} (31.01.1865 – 25.09.1935), \emph{Schriftsteller, Journalist}|pwk}: \emph{Berliner Theater. (»Der Schleier der Beatrice« von Arthur
                        Schnitzler.)}\pwindex{Goldmann, Paul 31.01.1865 – 25.09.1935@\textsc{Goldmann, Paul} (31.01.1865 – 25.09.1935), \emph{Schriftsteller, Journalist}!Berliner Theater. (»Der Schleier der Beatrice« von Arthur Schnitzler.)1903-03-19@\strich\emph{Berliner Theater. (»Der Schleier der Beatrice« von Arthur Schnitzler.)} {[}1903-03-19{]}|pwk}. In: \emph{Neue Freie
                        Presse}\pwindex{Neue Freie Presse1864 – 1939@\emph{Neue Freie Presse} {[}1864 – 1939{]}|pwk}, Nr. 13.851, 19. 3. 1903,
                     Morgenblatt, S. 1–5.}}}\label{K_L03478-4h} in der Öffentlichkeit\pwindex{Goldmann, Paul 31.01.1865 – 25.09.1935@\textsc{Goldmann, Paul} (31.01.1865 – 25.09.1935), \emph{Schriftsteller, Journalist}!Berliner Theater. (»Der Schleier der Beatrice« von Arthur Schnitzler.)1903-03-19@\strich\emph{Berliner Theater. (»Der Schleier der Beatrice« von Arthur Schnitzler.)} {[}1903-03-19{]}|pwv} anders geurteilt hätte, als ich dies vorher
               in einem Privatbriefe an Dich getan hatte. Ich empfand dies als eine ſchwere
               Kränkung. Denn wenn ich heut auf mein langes Journaliſten-Leben zurückblicke, darf
               ich von mir ſagen, daß ich (in weſentlichen Fragen) öffentlich niemals anders
               geſprochen habe, als ich wirklich gedacht habe, – daß ich niemals zwei verſchiedene
               Meinungen gehabt habe, eine öffentliche {\pb}u. eine
               private. Als ich dann meinen \label{K_L03478-5v}\edtext{Brief}{\lemma{\textnormal{\emph{Brief}}}\Cendnote{\textnormal{siehe Paul Goldmann an Arthur Schnitzler, 11. 2. 1900. Er dürfte sich auf
                  die teilweise Abschrift seiner Briefe aus dem Jahr 1900 beziehen,
                     vgl. Paul Goldmann an Arthur Schnitzler, 30. 12. 1910}}}\label{K_L03478-5h} an Dich nachlas, fand ich beſtätigt, daß Du mir Unrecht getan hatteſt. Denn
               ſchon in dieſem Briefe waren Einwendungen angedeutet u. Vorbehalte gemacht – nur
               waren dieſe Einwendungen u. Vorbehalte in rückſichtsvolle Form gekleidet. Denn in
               einem Privatbriefe an einen Freund ſind Rückſichten erlaubt, ja geboten, während man
               zu rückhaltsloſer Ausſprache ſeiner Meinung verpflichtet iſt, wenn man als Kritiker
               zum Publikum ſpricht.\pend
           \pstart
           Aber, wäre es nur {\pb}dieſe Kränkung geweſen, – ich
               hätte ſie längſt vergeſſen u. wäre längſt wieder zu Dir gekommen, um Dir die Hand zu
               bieten. Die Erinnerungen an ſchöne gemeinſame Jugendjahre, \substVorne{}\textsuperscript{\textcolor{gray}{ſ}}\substDazwischen{}d\substHinten{}ie auch Du in Deinem Briefe jetzt erwähnſt, leben weiter u. ziehen mich zu
               Dir, der Du ja überhaupt unter all’ den Menſchen, denen ich auf meinem Lebenswege
               begegnet bin, einer der Beſten u. Liebenswerteſten biſt.\pend
           \pstart
           Was mich von Dir ferngehalten hat, war etwas anderes. In \label{K_L03478-6v}\edtext{einem Deiner Briefe}{\lemma{\textnormal{\emph{einem Deiner Briefe}}}\Cendnote{\textnormal{Der Brief ist nicht erhalten. Auffällig ist vielleicht die
                  Verwendung des Wortes »unkünstlerisch«, das in Schnitzler\pwindex{Schnitzler, Arthur 15.05.1862 – 21.10.1931@\textsc{Schnitzler, Arthur} (15.05.1862 – 21.10.1931), \emph{Schriftsteller, Mediziner}|pwk}s \emph{Tagebuch}\pwindex{\textcolor{red}{\textsuperscript{XXXX1 indx}}!Tagebuch1981 – 2000@\strich\emph{Tagebuch} {[}Hrsg., 1981 – 2000{]}|pwk}
                  kein einziges Mal verwendet wird, in Goldmann\pwindex{Goldmann, Paul 31.01.1865 – 25.09.1935@\textsc{Goldmann, Paul} (31.01.1865 – 25.09.1935), \emph{Schriftsteller, Journalist}|pwk}s Briefen aber (einschließlich der vorliegenden Stelle) fünf
                  Mal.}}}\label{K_L03478-6h}, die unſer damaliger Konflikt {\pb}hervorrief, fand ſich folgende Äußerung über mich (ich zitire nur die
               hauptſächlichen Worte, ſoweit ſie mir in der Erinnerung geblieben ſind): »Du biſt ein
               Menſch ohne jede Phantaſie – eine gänzlich unkünſtleriſche Natur.« Das iſt ſchlimmer
               als eine Kränkung – das iſt ein Urteil – ein Urteil, das meine Perſon, meine ganze
               Lebensarbeit tief herabſetzt. Ich fand dasſelbe Urteil noch einmal wieder in einem\pwindex{Schnitzler, Arthur 15.05.1862 – 21.10.1931@\textsc{Schnitzler, Arthur} (15.05.1862 – 21.10.1931), \emph{Schriftsteller, Mediziner}!Stunde des Erkennens1915@\strich\emph{Stunde des Erkennens} {[}1915{]}|pwv} Deiner Stücke, wo, in
               unverkennbarer Anſpielung auf mich, {\pb}von einem
               Journaliſten die Rede iſt, einem »\label{K_L03478-7v}\edtext{\textsc{\begin{otherlanguage}{french}raté\end{otherlanguage}}}{\lemma{\textnormal{\emph{raté}}}\Cendnote{\textnormal{französisch: Versager; vgl. Paul Goldmann an Arthur Schnitzler, 6. 7. [1895]}}}\label{K_L03478-7h}«, der »\label{K_L03478-8v}\edtext{zu den Menſchen gehört, die eine
                  poetiſche Seele, aber kein poetiſches Talent haben.\pwindex{Schnitzler, Arthur 15.05.1862 – 21.10.1931@\textsc{Schnitzler, Arthur} (15.05.1862 – 21.10.1931), \emph{Schriftsteller, Mediziner}!Stunde des Erkennens1915@\strich\emph{Stunde des Erkennens} {[}1915{]}|pwv}}{\lemma{\textnormal{\emph{zu … haben.}}}\Cendnote{\textnormal{Goldmann\pwindex{Goldmann, Paul 31.01.1865 – 25.09.1935@\textsc{Goldmann, Paul} (31.01.1865 – 25.09.1935), \emph{Schriftsteller, Journalist}|pwk} dürfte sich durch diese Stelle im Einakter \emph{Stunde des Erkennens}\pwindex{Schnitzler, Arthur 15.05.1862 – 21.10.1931@\textsc{Schnitzler, Arthur} (15.05.1862 – 21.10.1931), \emph{Schriftsteller, Mediziner}!Stunde des Erkennens1915@\strich\emph{Stunde des Erkennens} {[}1915{]}|pwk} angesprochen gefühlt
                     haben: »Und vergiß nicht, mir Flöding\pwindex{Schnitzler, Arthur 15.05.1862 – 21.10.1931@\textsc{Schnitzler, Arthur} (15.05.1862 – 21.10.1931), \emph{Schriftsteller, Mediziner}!Stunde des Erkennens1915@\strich\emph{Stunde des Erkennens} {[}1915{]}|pwv} zu grüßen. Du kannſt ihm auch ſagen, daß es
                     eine ganz beſondere Gemeinheit iſt, ſo abſolut nichts mehr von ſich hören zu
                     laſſen, wenn man einmal ſo ›befreundet‹ war, wie er behauptet mit mir geweſen
                     zu ſein.« (\emph{Komödie der Worte.
                        Drei Einakter}\pwindex{Schnitzler, Arthur 15.05.1862 – 21.10.1931@\textsc{Schnitzler, Arthur} (15.05.1862 – 21.10.1931), \emph{Schriftsteller, Mediziner}!Komoedie der Worte. Drei Einakter1915-10-12@\strich\emph{Komödie der Worte. Drei Einakter} {[}1915-10-12{]}|pwk}. Berlin: \emph{S. Fischer Verlag}\orgindex{S. Fischer Verlag@S. Fischer Verlag|pwk} 1915, S. 21.) Wenige
                  Zeilen später wird Flöding\pwindex{Schnitzler, Arthur 15.05.1862 – 21.10.1931@\textsc{Schnitzler, Arthur} (15.05.1862 – 21.10.1931), \emph{Schriftsteller, Mediziner}!Stunde des Erkennens1915@\strich\emph{Stunde des Erkennens} {[}1915{]}|pwkv}
                  als »ein wenig hinkend geschildert (Goldmann\pwindex{Goldmann, Paul 31.01.1865 – 25.09.1935@\textsc{Goldmann, Paul} (31.01.1865 – 25.09.1935), \emph{Schriftsteller, Journalist}|pwk} hatte einen Buckel.) Und dann folgt die von Goldmann\pwindex{Goldmann, Paul 31.01.1865 – 25.09.1935@\textsc{Goldmann, Paul} (31.01.1865 – 25.09.1935), \emph{Schriftsteller, Journalist}|pwk} zitierte Stelle: »Schlimmer
                     find’ ich, daß er eine ſo poetiſche Seele beſitzt und kein poetiſches Talent.
                     Das verdirbt den Charakter, wie es ſcheint.«
                     (S. 21–22). Ob Schnitzler\pwindex{Schnitzler, Arthur 15.05.1862 – 21.10.1931@\textsc{Schnitzler, Arthur} (15.05.1862 – 21.10.1931), \emph{Schriftsteller, Mediziner}|pwk} hier tatsächlich an Goldmann\pwindex{Goldmann, Paul 31.01.1865 – 25.09.1935@\textsc{Goldmann, Paul} (31.01.1865 – 25.09.1935), \emph{Schriftsteller, Journalist}|pwk} gedacht hat, ist zweifelhaft.}}}\label{K_L03478-8h}« \pend
           \pstart
           Ich halte Dein Urteil über mich für unrichtig, finde, daß es mich gänzlich verkennt,
               u. habe \strikeout{\textcolor{gray}{d}} damals eine tiefe Bitterkeit darüber gefühlt, daß mich derjenige ſo verkennt,
               der lange Jahre hindurch mein nächſter Freund war. Dieſes Dein Urteil über mich hat
               mich damals von Dir entfernt u. hat mich bis heut von Dir ferngehalten. Ein Urteil
               aber, wie geſagt, {\pb}iſt ſchlimmer als eine
               Kränkung. Denn eine Kränkung löſcht die Zeit aus. Das hätte ſie namentlich in unſerem
               Falle getan. \strikeout{\textcolor{gray}{S}} Denn die Vergangenheit wird ein Ganzes, u. in dieſem Ganzen iſt ſo viel Gutes,
               das ich Dir verdanke, daß der \uline{eine} Grund, Dir böſe zu
               ſein, dagegen nicht in Betracht kommt.\pend
           \pstart
           Ein Urteil jedoch bleibt. Gewiß, es kann revidirt werden. Aber Du haft es ſicherlich
               nicht revidirt. Denn wenn Du ſchon in der Zeit, als wir nahe Freunde waren, Dir eine
               ſo unrichtige Anſchauung über mich {\pb}gebildet
               haft, warum ſollteſt Du ſie geändert haben in den Jahren, ſeit wir fern von einander
               leben? Ich verlange auch keine Reviſion Deines Urteils über mich. Ich laſſe Jedem
               ſeine Überzeugung, auch wenn ich ſie für irrig halte, – ſo wie ich beanſpruche, daß
               man mir meine Überzeugung läßt. Daß Du Dir aber dieſe Überzeugung über mich gebildet
               haſt, das macht es mir ſo ſchwer, den Weg wieder zu Dir zu finden. Gewiß, ich bin es
               gewohnt, verkannt u. unterſchätzt {\pb}zu werden, –
               u. ich habe mich damit abgefunden. Schließlich wird einem das Urteil der meiſten
               Menſchen gleichgiltig, u. man findet \substVorne{}\textsuperscript{ſich \textcolor{gray}{×}\-\textcolor{gray}{×}\-\textcolor{gray}{×}\-\textcolor{gray}{×}\-\textcolor{gray}{×}\-\textcolor{gray}{×}\-\textcolor{gray}{×}\-\textcolor{gray}{×}b}{\allowbreak}\substDazwischen{}ſeine Entſchädigung darin\substHinten{}, daß ein paar Freunde wiſſen, wer man iſt.\pend
           \pstart
           Ein Freund jedoch, \strikeout{\textcolor{gray}{[2 Zeilen unleserlich{]} }}
               der ſich dem herabſetzenden Urteil der anderen Menſchen anſchließt, – gewiß,
               auch der Freund hat das Recht, ſich in voller Freiheit ſein Urteil zu bilden, – ich
               aber kann es nicht über mich {\pb}gewinnen, den
               Freund, der mich kennen müßte u. nicht kennt, noch als Freund zu betrachten{\dots}\pend
           \pstart
           Und nun ſei nochmals herzlichſt bedankt für Deinen lieben Brief! Sei überzeugt,
               daß ich, trotz allem, in meiner Geſinnung Dir gegenüber der Alte geblieben bin! Und
               laß’ Dir von Herzen alles Gute wünſchen!\pend
           \pstart
           Dein {\\[\baselineskip]}\spacefill\mbox{Paul Goldmann.}\pend
           \leftskip=0em{}
         
         \endnumbering\mylabel{h}\end{ledgroupsized}  \newcommand{\dateiname}{L03478}\newcommand{\titel}{Paul Goldmann an Arthur Schnitzler, 16. 2. 1925}\newcommand{\editorInnen}{Martin Anton Müller und Laura Untner}%% latex-leseansicht-abspann.tex
%% Abspann für die Leseansicht.
%% Der Schalter \ifkorrekturansicht ist bereits durch den Vorspann gesetzt.

%% latex-abspann.tex
%% Gemeinsamer Abspann für Korrekturansicht und Leseansicht.
%% Setzt den Schalter \ifkorrekturansicht voraus (gesetzt in den
%% einbindenden Dateien latex-korrekturansicht-abspann.tex bzw.
%% latex-leseansicht-abspann.tex).
%% ---------------------------------------------------------------

\normalsize

% Das esempio-Environment wird nur in der Leseansicht benötigt
\ifkorrekturansicht\else
\newenvironment{esempio}[3]%
{
    \vspace{1.5ex}
    \rlap{\underline{#1}}
    \par
    \setlength{\parindent}{0cm}
    \nopagebreak
    \leftskip=#2cm
    \rightskip=#3cm
}
{
    \par
}
\fi

\doendnotes{C}
\bigskip
\vfill

\clearpage

\footnotesize

\ifkorrekturansicht
  \lohead{\textsc{register}}
\fi

% theindex-Environment neu definieren ohne reledmac
\makeatletter
\renewenvironment{theindex}{%
  \ifkorrekturansicht
    \section*{\indexname}%
  \else
    \subsubsection*{Index der erwähnten Entitäten}%
  \fi
  \setlength{\parindent}{0pt}%
  \setlength{\parskip}{0pt plus 0.3pt}%
  \let\item\@idxitem
}{%
  \ifkorrekturansicht\clearpage\fi
}
\makeatother

\IfFileExists{\jobname-pw.ind}{\input{\jobname-pw.ind}}{}

% Quellenangabe nur in der Leseansicht
\ifkorrekturansicht\else
% Fallback-Definitionen, falls die .tex-Datei \titel etc. nicht gesetzt hat
\providecommand{\titel}{}
\providecommand{\editorInnen}{}
\providecommand{\dateiname}{\jobname}

\vspace{3cm}

\vfill

\footnotesize
\textsc{Quelle}: \titel. Herausgegeben von {\editorInnen}. In: \emph{Arthur Schnitzler: Briefwechsel mit Autorinnen und Autoren}.
 Digitale Edition, https://schnitzler-briefe.acdh.oeaw.ac.at/{\dateiname}.html (Stand \today)
\fi

\end{document}


      