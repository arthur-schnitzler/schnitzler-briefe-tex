%% latex-korrekturansicht-vorspann.tex
%% Vorspann für die Korrekturansicht.
%% Lädt die gemeinsame Datei latex-vorspann.tex mit gesetztem Schalter.

\newif\ifkorrekturansicht
\korrekturansichttrue

\input{../tex-inputs/latex-vorspann}


\section[Arthur Schnitzler an Richard Beer-Hofmann, 16. 9. 1891]{L00040 Arthur Schnitzler an Richard Beer-Hofmann, 16. 9. 1891}
\nopagebreak\mylabel{L00040v}
\rehead{ }\normalsize\beginnumbering\briefempfaengerindex{Beer-Hofmann, Richard@\textsc{Beer-Hofmann, Richard}!zzzSchnitzler, Arthur@\emph{von Arthur Schnitzler}!1891-09-161@{16. 9. 1891}|(be}
\toendnotes[C]{\smallbreak\pagebreak[2]}\Standort{YCGL, MSS 31.}
\physDesc{Briefkarte, , Umschlag, 310 Zeichen
\newline{}Handschrift: 1) schwarze Tinte, deutsche Kurrent\hspace{1em}2) schwarze Tinte, lateinische Kurrent (\noindent{}Adresse)\hspace{1em}
\newline{}Versand: 1) Stempel: »\nobreak{}\oindex{VIII., Josefstadt@\textbf{VIII., Josefstadt}, \emph{A.ADM3}|pwk}Wien 8, 16 9 91, \textcolor{gray}{5} N\nobreak{}«.   2) Stempel: »\nobreak{}\oindex{III., Landstrasse@\textbf{III., Landstraße}, \emph{A.ADM3}|pwk}Wien 3/2, 16-9 91, 5 7. N, Bestellt\nobreak{}«. }\toendnotes[C]{\smallbreak}\pstart{}{\pb}Herrn Dr. Rich. Beer Hofmann\pend{}\pstart{}Wien\oindex{Wien@\textbf{Wien}, \emph{A.ADM2}|pw}\pend{}\pstart{}III. Seidlgasse 30\oindex{Seidlgasse@\textbf{Seidlgasse}, \emph{Straße (K.STR)}|pw}.\pend{}{\bigskip}\vspace{1em}
\pstart
           \noindent{}{\pb}Lieber Freund, man will Sie bereits vor 14 Tagen in Baden\oindex{Baden bei Wien@\textbf{Baden bei Wien}, \emph{P.PPLA3}|pw} geſehen haben. Sind Sie da? Ich verreiſe am
                  Samſtag auf etwa 8 Tage nach \textsc{Halle an der Saale}\oindex{Halle (Saale)@\textbf{Halle (Saale)}, \emph{P.PPL}|pw} zur \label{K_L00040-1v}\edtext{Natur{\pb}forſcherverſa{\geminationm}lung}{\lemma{\textnormal{\emph{Naturforſcherverſammlung}}}\Cendnote{\textnormal{Die 64. Versammlung der \emph{Gesellschaft Deutscher Naturforscher und Ärzte}\orgindex{Gesellschaft Deutscher Naturforscher und Aerzte@Gesellschaft Deutscher Naturforscher und Ärzte|pwk} fand vom
                     21. bis 25. 9. 1891 in Halle an der Saale\oindex{Halle (Saale)@\textbf{Halle (Saale)}, \emph{P.PPL}|pwk}{ }statt.}}}\label{K_L00040-1}. – Wie ſteht’s mit Italien\oindex{Italien@\textbf{Italien}, \emph{A.PCLI}|pw}? Ka{\geminationn} ich
               für den Anfang Oktober auf Sie rechnen?\pend
           
\pstart
           Herzlich Ihr{\\[\baselineskip]}\spacefill\mbox{Arthur}\pend
           \leftskip=0em{}\selectlanguage{ngerman}\endnumbering\briefempfaengerindex{Beer-Hofmann, Richard@\textsc{Beer-Hofmann, Richard}!zzzSchnitzler, Arthur@\emph{von Arthur Schnitzler}!1891-09-161@{16. 9. 1891}|)be}\mylabel{L00040h}  \normalsize

\doendnotes{C}
\bigskip
\vfill

\clearpage

\footnotesize

\lohead{\textsc{register}}

% Definiere theindex-Environment komplett neu ohne reledmac
\makeatletter
\renewenvironment{theindex}{%
  \section*{\indexname}%
  \setlength{\parindent}{0pt}%
  \setlength{\parskip}{0pt plus 0.3pt}%
  \let\item\@idxitem
}{%
  \clearpage
}
\makeatother

\IfFileExists{\jobname-pw.ind}{\input{\jobname-pw.ind}}{}

\end{document}

      