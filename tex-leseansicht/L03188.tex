%% latex-leseansicht-vorspann.tex
%% Vorspann für die Leseansicht.
%% Lädt die gemeinsame Datei latex-vorspann.tex mit nicht gesetztem Schalter.

\newif\ifkorrekturansicht
\korrekturansichtfalse

\input{../tex-inputs/latex-vorspann}


\section[ Arthur Schnitzler an Paul Goldmann, 1. 7. 1900]{L03188 Arthur Schnitzler an Paul Goldmann,  1. 7. 1900}
\nopagebreak\mylabel{L03188v}
\rehead{ }\normalsize\beginnumbering\briefempfaengerindex{Goldmann, Paul@\textsc{Goldmann, Paul}!zzzSchnitzler, Arthur@\emph{von Arthur Schnitzler}!1900-07-011@{1. 7. 1900}|(be}
\toendnotes[C]{\smallbreak\pagebreak[2]}
\correspDesc{Versand  durch Arthur Schnitzler am 1. 7. 1900 in Grundlsee
\newline{}Erhalt  durch Paul Goldmann am 8. 7. 1900 in Berlin}\toendnotes[C]{\smallbreak}
\Standort{Berlin, Akademie der Künste, Alfred Kerr-Archiv, 2487.}
\physDesc{Bildpostkarte, 328 Zeichen
\newline{}Handschrift: Bleistift, deutsche Kurrent
\newline{}Versand: 1) Stempel: »\nobreak{}\oindex{Grundlsee [Gemeinde]@\textbf{Grundlsee [Gemeinde]}|pwk}G\textcolor{gray}{r}{[}undlse{]}e, 1. 7. 00\nobreak{}«.   2) Stempel: »\nobreak{}8/7 00, 7¼–8½ V., Bestellt vom Postamt\nobreak{}«. }\toendnotes[C]{\smallbreak}\pstart{}\textsc{{\pb}Dr. Paul Goldmann}\pend{}\pstart{}\textsc{Berlin W\textcolor{gray}{.}\oindex{Berlin@\textbf{Berlin}, \emph{Hauptstadt}|pw}}\pend{}\pstart{}\textsc{Dessauerstraße 19\oindex{Dessauer Straße@\textbf{Dessauer Straße}, \emph{Straße}|pw}.}\pend{}{\bigskip}
\pstart
           {\pb}\textcolor{gray}{\textbf{\textbf{Gruss} aus Grundlsee\oindex{Grundlsee [Gemeinde]@\textbf{Grundlsee [Gemeinde]}|pw}.}}\hfill \textcolor{gray}{\textbf{Totalansicht.}}\pend
           \vspace{1em}
\pstart
           \noindent{}{\pb}Ich bin hier per Rad, \label{K_L03188-1v}\edtext{aus \textsc{Altaussee}\oindex{Altaussee@\textbf{Altaussee}, \emph{Verwaltungsgebiet}|pw}}{\lemma{\textnormal{\emph{aus Altaussee}}}\Cendnote{\textnormal{Schnitzler hielt sich seit 29. 6. 1900 in Altaussee\oindex{Altaussee@\textbf{Altaussee}, \emph{Verwaltungsgebiet}|pwk} auf. Am 3. 7. 1900 fuhr er
                  weiter nach Admont\oindex{Admont@\textbf{Admont}, \emph{Verwaltungsgebiet}|pwk}.}}}\label{K_L03188-1}, (Richard\pwindex{Beer-Hofmann, Richard 11.\,7.\,1866 Wien – 26.\,9.\,1945 New York City@\textsc{Beer-Hofmann, Richard} (11.\,7.\,1866 Wien – 26.\,9.\,1945 New York City), \emph{Schriftsteller}|pw} nicht.) – Die berühmte \label{K_L03188-2v}\edtext{Fußpartie}{\lemma{\textnormal{\emph{Fußpartie}}}\Cendnote{\textnormal{Siehe XXXX Auszeichnungsfehler: Dokument L02920 nicht gefunden.
               }}}\label{K_L03188-2} wäre am erwünſchteſten we{\geminationn}{ }ſie etwa Mitte Auguſt begänne. Rendezvous Salzburg\oindex{Salzburg@\textbf{Salzburg}, \emph{Verwaltungsgebiet}|pw} oder Innsbruck\oindex{Innsbruck@\textbf{Innsbruck}, \emph{Verwaltungsgebiet}|pw}. – Nach Salzburg\oindex{Salzburg@\textbf{Salzburg}, \emph{Verwaltungsgebiet}|pw} auf ein paar
               Tage könnte Richard\pwindex{Beer-Hofmann, Richard 11.\,7.\,1866 Wien – 26.\,9.\,1945 New York City@\textsc{Beer-Hofmann, Richard} (11.\,7.\,1866 Wien – 26.\,9.\,1945 New York City), \emph{Schriftsteller}|pw} (u ich) (und Leo\pwindex{Van-Jung, Leo 15.\,10.\,1866 Odessa – 2.\,7.\,1939 Riga@\textsc{Van-Jung, Leo} (15.\,10.\,1866 Odessa – 2.\,7.\,1939 Riga), \emph{Gesangspädagoge, Mathematiker}|pw}) auch Anfang
                  Auguſt.\pend
           \pstart Herzlichſt \spacefill\mbox{Arthur}\pend{}
\pstart
           \noindent{}Bitte das auch \label{K_L03188-3v}\edtext{\textsc{Kerr\pwindex{Kerr, Alfred 25.\,12.\,1867 Breslau – 12.\,10.\,1948 Hamburg@\textsc{Kerr, Alfred} (25.\,12.\,1867 Breslau – 12.\,10.\,1948 Hamburg), \emph{Schriftsteller, Kritiker}|pw}} zu{ }ſagen}{\lemma{\textnormal{\emph{Kerr zu sagen}}}\Cendnote{\textnormal{Goldmann\pwindex{Goldmann, Paul 31.\,1.\,1865 Breslau – 25.\,9.\,1935 Wien@\textsc{Goldmann, Paul} (31.\,1.\,1865 Breslau – 25.\,9.\,1935 Wien), \emph{Schriftsteller, Journalist}|pwk} sandte die Karte direkt weiter,
                     sie ist im Nachlass Kerrs\pwindex{Kerr, Alfred 25.\,12.\,1867 Breslau – 12.\,10.\,1948 Hamburg@\textsc{Kerr, Alfred} (25.\,12.\,1867 Breslau – 12.\,10.\,1948 Hamburg), \emph{Schriftsteller, Kritiker}|pwk}
                     überliefert.}}}\label{K_L03188-3}\pend
           \selectlanguage{ngerman}\endnumbering\briefempfaengerindex{Goldmann, Paul@\textsc{Goldmann, Paul}!zzzSchnitzler, Arthur@\emph{von Arthur Schnitzler}!1900-07-011@{1. 7. 1900}|)be}\mylabel{L03188h}  \newcommand{\dateiname}{L03188}\newcommand{\titel}{Arthur Schnitzler an Paul Goldmann, 1. 7. 1900}\newcommand{\editorInnen}{Martin Anton Müller und Laura Untner}%% latex-leseansicht-abspann.tex
%% Abspann für die Leseansicht.
%% Der Schalter \ifkorrekturansicht ist bereits durch den Vorspann gesetzt.

%% latex-abspann.tex
%% Gemeinsamer Abspann für Korrekturansicht und Leseansicht.
%% Setzt den Schalter \ifkorrekturansicht voraus (gesetzt in den
%% einbindenden Dateien latex-korrekturansicht-abspann.tex bzw.
%% latex-leseansicht-abspann.tex).
%% ---------------------------------------------------------------

\normalsize

% Das esempio-Environment wird nur in der Leseansicht benötigt
\ifkorrekturansicht\else
\newenvironment{esempio}[3]%
{
    \vspace{1.5ex}
    \rlap{\underline{#1}}
    \par
    \setlength{\parindent}{0cm}
    \nopagebreak
    \leftskip=#2cm
    \rightskip=#3cm
}
{
    \par
}
\fi

\doendnotes{C}
\bigskip
\vfill

\clearpage

\footnotesize

\ifkorrekturansicht
  \lohead{\textsc{register}}
\fi

% theindex-Environment neu definieren ohne reledmac
\makeatletter
\renewenvironment{theindex}{%
  \ifkorrekturansicht
    \section*{\indexname}%
  \else
    \subsubsection*{Index der erwähnten Entitäten}%
  \fi
  \setlength{\parindent}{0pt}%
  \setlength{\parskip}{0pt plus 0.3pt}%
  \let\item\@idxitem
}{%
  \ifkorrekturansicht\clearpage\fi
}
\makeatother

\IfFileExists{\jobname-pw.ind}{\input{\jobname-pw.ind}}{}

% Quellenangabe nur in der Leseansicht
\ifkorrekturansicht\else
% Fallback-Definitionen, falls die .tex-Datei \titel etc. nicht gesetzt hat
\providecommand{\titel}{}
\providecommand{\editorInnen}{}
\providecommand{\dateiname}{\jobname}

\vspace{3cm}

\vfill

\footnotesize
\textsc{Quelle}: \titel. Herausgegeben von {\editorInnen}. In: \emph{Arthur Schnitzler: Briefwechsel mit Autorinnen und Autoren}.
 Digitale Edition, https://schnitzler-briefe.acdh.oeaw.ac.at/{\dateiname}.html (Stand \today)
\fi

\end{document}


