%% latex-leseansicht-vorspann.tex
%% Vorspann für die Leseansicht.
%% Lädt die gemeinsame Datei latex-vorspann.tex mit nicht gesetztem Schalter.

\newif\ifkorrekturansicht
\korrekturansichtfalse

\input{../tex-inputs/latex-vorspann}

\begin{center}
            \textcolor{red}{ENTWURF, NICHT FERTIG KORRIGIERT}
                      \end{center}
            
         
         \renewcommand{\erwaehntePersonen}{Personen: Richard Beer-Hofmann, Paul Goldmann, Alfred Kerr, Leo Van-Jung}
         \renewcommand{\erwaehnteOrte}{Orte: Altaussee, Berlin, Dessauer Straße, Grundlsee (Gemeinde), Innsbruck, Salzburg}
         \renewcommand{\erwaehnteWerke}{}
               \section[Arthur Schnitzler an Paul Goldmann, 1. 7. 1900]{ Arthur Schnitzler an Paul Goldmann, 1. 7. 1900}\nopagebreak\mylabel{v}\rehead{ }\begin{ledgroupsized}[t]{13cm}\normalsize\beginnumbering\briefempfaengerindex{Goldmann, Paul@\textsc{Goldmann, Paul}!zzzSchnitzler, Arthur@\emph{von Arthur Schnitzler}!1900-07-011@{1. 7. 1900}|(be} \toendnotes[C]{\smallbreak\pagebreak[2]} \Standort{Berlin, Akademie der Künste, Alfred Kerr-Archiv, 2487.}
\physDesc{Bildpostkarte, 334 Zeichen
\newline{}Handschrift: Bleistift, deutsche Kurrent
\newline{}Versand: 1) Stempel: »\nobreak{}\oindex{Grundlsee (Gemeinde)@\textbf{Grundlsee (Gemeinde)}|pwk}Gr{[}undlsee{]}, 1. 7. 00\nobreak{}«.   2) Stempel: »\nobreak{}8/7 00, 7¼–8½V., Bestellt vom Postamt\nobreak{}«. }\pstart{}{\pb}Dr. Paul Goldmann\pend{}\pstart{}Berlin W\oindex{Berlin@\textbf{Berlin}|pw}\pend{}\pstart{}Dessauerstraße 19\oindex{Dessauer Strasse@\textbf{Dessauer Straße}|pw}.\pend{}{\bigskip}\pstart
           \noindent{}{\pb}\textcolor{gray}{\textbf{Gruss aus Grundlsee\oindex{Grundlsee (Gemeinde)@\textbf{Grundlsee (Gemeinde)}|pw}.}}\hfill \textcolor{gray}{\textbf{Totalansicht.}}\pend
           \pstart
           Ich bin hier per Rad, aus \textsc{Altaussee}\oindex{Altaussee@\textbf{Altaussee}|pw}, (Richard\pwindex{Beer-Hofmann, Richard 1866-07-11 – 1945-09-26@\textsc{Beer-Hofmann, Richard} (1866-07-11 – 1945-09-26), \emph{Schriftsteller}|pw} nicht.) – Die berühmte
               Fußpartie wäre am erwünſchtesten, we{\geminationn} ſie etwa Mitte
               Auguſt begänne. Rendezvous Salzburg\oindex{Salzburg@\textbf{Salzburg}|pw} oder Innsbruck\oindex{Innsbruck@\textbf{Innsbruck}|pw}.– Nach Salzburg\oindex{Salzburg@\textbf{Salzburg}|pw} auf ein paar Tage könnte Richard\pwindex{Beer-Hofmann, Richard 1866-07-11 – 1945-09-26@\textsc{Beer-Hofmann, Richard} (1866-07-11 – 1945-09-26), \emph{Schriftsteller}|pw} (u ich) (und Leo\pwindex{Van-Jung, Leo 15.10.1866 – 02.07.1939@\textsc{Van-Jung, Leo} (15.10.1866 – 02.07.1939), \emph{Gesangspädagoge, Mathematiker}|pw}) auch Anfang
               Auguſt.\pend
           \pstart
           Herzlichſt{\\[\baselineskip]}\spacefill\mbox{Arthur}\pend
           \leftskip=0em{}\pstart
           \noindent{}Bitte das auch Kerr\pwindex{Kerr, Alfred 25.12.1867 – 12.10.1948@\textsc{Kerr, Alfred} (25.12.1867 – 12.10.1948), \emph{Schriftsteller, Kritiker}|pw} zu ſagen.\pend
           
         
         \endnumbering\mylabel{h}\end{ledgroupsized}\begin{anhang}\end{anhang}\newcommand{\dateiname}{L03188}\newcommand{\titel}{Arthur Schnitzler an Paul Goldmann, 1. 7. 1900}\newcommand{\editorInnen}{Martin Anton Müller und Laura Untner}%% latex-leseansicht-abspann.tex
%% Abspann für die Leseansicht.
%% Der Schalter \ifkorrekturansicht ist bereits durch den Vorspann gesetzt.

%% latex-abspann.tex
%% Gemeinsamer Abspann für Korrekturansicht und Leseansicht.
%% Setzt den Schalter \ifkorrekturansicht voraus (gesetzt in den
%% einbindenden Dateien latex-korrekturansicht-abspann.tex bzw.
%% latex-leseansicht-abspann.tex).
%% ---------------------------------------------------------------

\normalsize

% Das esempio-Environment wird nur in der Leseansicht benötigt
\ifkorrekturansicht\else
\newenvironment{esempio}[3]%
{
    \vspace{1.5ex}
    \rlap{\underline{#1}}
    \par
    \setlength{\parindent}{0cm}
    \nopagebreak
    \leftskip=#2cm
    \rightskip=#3cm
}
{
    \par
}
\fi

\doendnotes{C}
\bigskip
\vfill

\clearpage

\footnotesize

\ifkorrekturansicht
  \lohead{\textsc{register}}
\fi

% theindex-Environment neu definieren ohne reledmac
\makeatletter
\renewenvironment{theindex}{%
  \ifkorrekturansicht
    \section*{\indexname}%
  \else
    \subsubsection*{Index der erwähnten Entitäten}%
  \fi
  \setlength{\parindent}{0pt}%
  \setlength{\parskip}{0pt plus 0.3pt}%
  \let\item\@idxitem
}{%
  \ifkorrekturansicht\clearpage\fi
}
\makeatother

\IfFileExists{\jobname-pw.ind}{\input{\jobname-pw.ind}}{}

% Quellenangabe nur in der Leseansicht
\ifkorrekturansicht\else
% Fallback-Definitionen, falls die .tex-Datei \titel etc. nicht gesetzt hat
\providecommand{\titel}{}
\providecommand{\editorInnen}{}
\providecommand{\dateiname}{\jobname}

\vspace{3cm}

\vfill

\footnotesize
\textsc{Quelle}: \titel. Herausgegeben von {\editorInnen}. In: \emph{Arthur Schnitzler: Briefwechsel mit Autorinnen und Autoren}.
 Digitale Edition, https://schnitzler-briefe.acdh.oeaw.ac.at/{\dateiname}.html (Stand \today)
\fi

\end{document}


      