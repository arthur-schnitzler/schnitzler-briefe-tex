%% latex-korrekturansicht-vorspann.tex
%% Vorspann für die Korrekturansicht.
%% Lädt die gemeinsame Datei latex-vorspann.tex mit gesetztem Schalter.

\newif\ifkorrekturansicht
\korrekturansichttrue

\input{../tex-inputs/latex-vorspann}


\section[ Arthur Schnitzler an Paul Goldmann, 1. 7. 1900]{L03188 Arthur Schnitzler an Paul Goldmann, 1. 7. 1900}
\nopagebreak\mylabel{L03188v}
\rehead{ }\normalsize\beginnumbering\briefempfaengerindex{Goldmann, Paul@\textsc{Goldmann, Paul}!zzzSchnitzler, Arthur@\emph{von Arthur Schnitzler}!1900-07-011@{1. 7. 1900}|(be}
\toendnotes[C]{\smallbreak\pagebreak[2]}\Standort{Berlin, Akademie der Künste, Alfred Kerr-Archiv, 2487.}
\physDesc{Bildpostkarte, 328 Zeichen
\newline{}Handschrift: 1) Bleistift, deutsche Kurrent\hspace{1em}2) Bleistift, lateinische Kurrent (\noindent{}Adresse)\hspace{1em}
\newline{}Versand: 1) Stempel: »\nobreak{}\oindex{Grundlsee [Gemeinde]@\textbf{Grundlsee [Gemeinde]}|pwk}G\textcolor{gray}{r}{[}undlse{]}e, 1. 7. 00\nobreak{}«.   2) Stempel: »\nobreak{}8/7 00, 7¼–8½ V., Bestellt vom Postamt\nobreak{}«. }\toendnotes[C]{\smallbreak}\pstart{}{\pb}Dr. Paul Goldmann\pend{}\pstart{}Berlin W\textcolor{gray}{.}\oindex{Berlin@\textbf{Berlin}|pw}\pend{}\pstart{}Dessauerstraße 19\oindex{Dessauer Strasse@\textbf{Dessauer Straße}|pw}.\pend{}{\bigskip}
\pstart
           {\pb}\textcolor{gray}{\textbf{\textbf{Gruss} aus Grundlsee\oindex{Grundlsee [Gemeinde]@\textbf{Grundlsee [Gemeinde]}|pw}.}}\hfill \textcolor{gray}{\textbf{Totalansicht.}}\pend
           \vspace{1em}
\pstart
           \noindent{}{\pb}Ich bin hier per Rad, \label{K_L03188-1v}\edtext{aus \textsc{Altaussee}\oindex{Altaussee@\textbf{Altaussee}|pw}}{\lemma{\textnormal{\emph{aus Altaussee}}}\Cendnote{\textnormal{Schnitzler hielt sich seit 29. 6. 1900 in Altaussee\oindex{Altaussee@\textbf{Altaussee}|pwk} auf. Am 3. 7. 1900 fuhr er
                  weiter nach Admont\oindex{Admont@\textbf{Admont}|pwk}.}}}\label{K_L03188-1}, (Richard\pwindex{Beer-Hofmann, Richard 1866-07-11 – 1945-09-26@\textsc{Beer-Hofmann, Richard} (1866-07-11 – 1945-09-26), \emph{Schriftsteller/Schriftstellerin}|pw} nicht.) – Die berühmte \label{K_L03188-2v}\edtext{Fußpartie}{\lemma{\textnormal{\emph{Fußpartie}}}\Cendnote{\textnormal{Siehe XXXX Auszeichnungsfehler: Dokument L02920 nicht gefunden.
               }}}\label{K_L03188-2} wäre am erwünschtesten we{\geminationn}{ }sie etwa Mitte August begänne. Rendezvous Salzburg\oindex{Salzburg@\textbf{Salzburg}|pw} oder Innsbruck\oindex{Innsbruck@\textbf{Innsbruck}|pw}. – Nach Salzburg\oindex{Salzburg@\textbf{Salzburg}|pw} auf ein paar
               Tage könnte Richard\pwindex{Beer-Hofmann, Richard 1866-07-11 – 1945-09-26@\textsc{Beer-Hofmann, Richard} (1866-07-11 – 1945-09-26), \emph{Schriftsteller/Schriftstellerin}|pw} (u ich) (und Leo\pwindex{Van-Jung, Leo 15.10.1866 – 02.07.1939@\textsc{Van-Jung, Leo} (15.10.1866 – 02.07.1939), \emph{Gesangspädagoge/Gesangspädagogin, Mathematiker/Mathematikerin}|pw}) auch Anfang
                  August.\pend
           \pstart Herzlichst \spacefill\mbox{Arthur}\pend{}
\pstart
           \noindent{}Bitte das auch \label{K_L03188-3v}\edtext{\textsc{Kerr\pwindex{Kerr, Alfred 25.12.1867 – 12.10.1948@\textsc{Kerr, Alfred} (25.12.1867 – 12.10.1948), \emph{Schriftsteller/Schriftstellerin, Kritiker/Kritikerin}|pw}} zu{ }sagen}{\lemma{\textnormal{\emph{Kerr zu sagen}}}\Cendnote{\textnormal{Goldmann\pwindex{Goldmann, Paul 31.01.1865 – 25.09.1935@\textsc{Goldmann, Paul} (31.01.1865 – 25.09.1935), \emph{Schriftsteller/Schriftstellerin, Journalist/Journalistin}|pwk} sandte die Karte direkt weiter,
                     sie ist im Nachlass Kerrs\pwindex{Kerr, Alfred 25.12.1867 – 12.10.1948@\textsc{Kerr, Alfred} (25.12.1867 – 12.10.1948), \emph{Schriftsteller/Schriftstellerin, Kritiker/Kritikerin}|pwk}
                     überliefert.}}}\label{K_L03188-3}\pend
           \selectlanguage{ngerman}\endnumbering\briefempfaengerindex{Goldmann, Paul@\textsc{Goldmann, Paul}!zzzSchnitzler, Arthur@\emph{von Arthur Schnitzler}!1900-07-011@{1. 7. 1900}|)be}\mylabel{L03188h}  \normalsize

\doendnotes{C}
\bigskip
\vfill

\clearpage

\footnotesize

\lohead{\textsc{register}}

% Definiere theindex-Environment komplett neu ohne reledmac
\makeatletter
\renewenvironment{theindex}{%
  \section*{\indexname}%
  \setlength{\parindent}{0pt}%
  \setlength{\parskip}{0pt plus 0.3pt}%
  \let\item\@idxitem
}{%
  \clearpage
}
\makeatother

\IfFileExists{\jobname-pw.ind}{\input{\jobname-pw.ind}}{}

\end{document}

      