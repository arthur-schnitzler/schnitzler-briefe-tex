%% latex-leseansicht-vorspann.tex
%% Vorspann für die Leseansicht.
%% Lädt die gemeinsame Datei latex-vorspann.tex mit nicht gesetztem Schalter.

\newif\ifkorrekturansicht
\korrekturansichtfalse

\input{../tex-inputs/latex-vorspann}


         
         \renewcommand{\erwaehntePersonen}{Personen: Richard Beer-Hofmann, Houston Stewart Chamberlain, Maxime Dethomas, Hugo von Hofmannsthal, Jeanne Marni, Marie Reinhard, Therese Reinhard, Louise Schnitzler, Leopold Sonnemann, Leo Van-Jung, Richard Wagner}
         \renewcommand{\erwaehnteInstitutionen}{Institutionen: Bayreuther Festspiele, Frankfurter Zeitung}
         \renewcommand{\erwaehnteOrte}{Orte: Andermatt, Bad Ischl, Bayreuth, Louveciennes, Paris, Schweiz, Wien, rue de la Bourse}
         \renewcommand{\erwaehnteWerke}{Werke: Das Vermächtnis. Schauspiel in drei Akten, Der Tod Georgs, Parsifal, Richard Wagner, Tagebuch}
               \section[ Paul Goldmann an Arthur Schnitzler, 2. 7. {[}1897{]}]{ Paul Goldmann an Arthur Schnitzler, 2. 7. {[}1897{]}}\nopagebreak\mylabel{v}\rehead{ }\begin{ledgroupsized}[t]{13cm}\normalsize\beginnumbering \toendnotes[C]{\smallbreak\pagebreak[2]} \Standort{DLA, A:Schnitzler, HS.NZ85.1.3167.}
\physDesc{Brief, 1 Blatt, 4 Seiten
\newline{}Handschrift: blaue Tinte, deutsche Kurrent
\newline{}Schnitzler: 1) mit Bleistift das Jahr »97« vermerkt  2) mit rotem Buntstift fünf Unterstreichungen}\toendnotes[C]{\smallbreak}\pstart
           \noindent{}{\pb}\textcolor{gray}{\textbf{\textbf{Frankfurter Zeitung\orgindex{Frankfurter Zeitung@Frankfurter Zeitung|pw}}}}\pend
           \pstart
           \textcolor{gray}{\textbf{(\begin{otherlanguage}{french}Gazette de Francfort\end{otherlanguage}\orgindex{Frankfurter Zeitung@Frankfurter Zeitung|pw}).}}\pend
           \pstart
           \textcolor{gray}{\textbf{\textbf{\begin{otherlanguage}{french}Fondateur M.\end{otherlanguage}{ }L. Sonnemann\pwindex{Sonnemann, Leopold 1831-10-29 – 1909-10-30@\textsc{Sonnemann, Leopold} (1831-10-29 – 1909-10-30), \emph{Journalist, Herausgeber}|pw}.}}}\pend
           \pstart
           \begin{otherlanguage}{french}\textcolor{gray}{\textbf{Journal politique, financier,}}\end{otherlanguage}\hfill \textsc{Paris\oindex{Paris@\textbf{Paris}|pw}}, 2. Juli.\pend
           \pstart
           \begin{otherlanguage}{french}\textcolor{gray}{\textbf{commercial et littéraire.}}\end{otherlanguage}\pend
           \pstart
           \begin{otherlanguage}{french}\textcolor{gray}{\textbf{\textbf{Paraissant trois fois par jour.}}}\end{otherlanguage}\pend
           \pstart
           \begin{otherlanguage}{french}\textcolor{gray}{\textbf{\textbf{Bureau à Paris\oindex{Paris@\textbf{Paris}|pw}}}}\end{otherlanguage}\pend
           \pstart
           \begin{otherlanguage}{french}\textcolor{gray}{\textbf{\textbf{10 \so{Rue de la Bourse}\oindex{rue de la Bourse@\textbf{rue de la Bourse}|pw}.}}}\end{otherlanguage}\pend
           \pstart\center{}Mein lieber Freund,\pend\pstart
           Ich danke Dir für Deinen lieben Brief und Deine Correſpondenz-Karte. All’ dieſe Tage
               konnte ich nicht die Zeit zur Antwort finden. Auch bin ich krank und mißmuthig.\pend
           \pstart
           \label{K_L02816-87v}\edtext{Aus der Schweiz\oindex{Schweiz@\textbf{Schweiz}|pw}}{\lemma{\textnormal{\emph{Aus der Schweiz}}}\Cendnote{\textnormal{Marie Reinhard\pwindex{Reinhard, Marie 1871-03-13 – 1899-03-18@\textsc{Reinhard, Marie} (1871-03-13 – 1899-03-18), \emph{Gesangspädagogin}|pwk} war zu dieser Zeit mit ihrer
                     Mutter\pwindex{Reinhard, Therese 13.12.1844 – 25.03.1926@\textsc{Reinhard, Therese} (13.12.1844 – 25.03.1926)|pwkv} in Andermatt\oindex{Andermatt@\textbf{Andermatt}|pwk} – vor allem, um die Schwangerschaft
                  vor der Wien\oindex{Wien@\textbf{Wien}|pwk}er Gesellschaft zu verbergen. Siehe A. S.: \emph{Tagebuch}, 13. 7. 1897.}}}\label{K_L02816-87h} habe ich
               plötzlich die \label{K_L02816-12v}\edtext{\textsc{Wagner\pwindex{Wagner, Richard 22.05.1813 – 13.02.1883@\textsc{Wagner, Richard} (22.05.1813 – 13.02.1883), \emph{Komponist}|pw}}-Biographie\pwindex{Chamberlain, Houston Stewart 09.09.1855 – 09.01.1927@\textsc{Chamberlain, Houston Stewart} (09.09.1855 – 09.01.1927), \emph{Schriftsteller}!Richard WagnerOktober 1895@\strich\emph{Richard Wagner} {[}Oktober 1895{]}|pwv}}{\lemma{\textnormal{\emph{Wagner-Biographie}}}\Cendnote{\textnormal{vermutlich: Houston Stewart Chamberlain\pwindex{Chamberlain, Houston Stewart 09.09.1855 – 09.01.1927@\textsc{Chamberlain, Houston Stewart} (09.09.1855 – 09.01.1927), \emph{Schriftsteller}|pwk}: \emph{Richard Wagner}\pwindex{Chamberlain, Houston Stewart 09.09.1855 – 09.01.1927@\textsc{Chamberlain, Houston Stewart} (09.09.1855 – 09.01.1927), \emph{Schriftsteller}!Richard WagnerOktober 1895@\strich\emph{Richard Wagner} {[}Oktober 1895{]}|pwk}. Mit zahlreichen Porträts,
                     Faksimiles, Illustrationen und Beilagen. München:
                        \emph{Verlagssanstalt für Kunst und Wissenschaft (vormals Friedrich
                        Bruckmann)}{ }1896 [vordatiert von Oktober 1895]}}}\label{K_L02816-12h} erhalten. Ihr\pwindex{Reinhard, Marie 1871-03-13 – 1899-03-18@\textsc{Reinhard, Marie} (1871-03-13 – 1899-03-18), \emph{Gesangspädagogin}|pwv} ſeid
               wirklich zu lieb und gut! Ich hoffte ſchon, Ihr hättet es vergeſſen. Ich freue mich
               ſehr über das ſchöne Buch\pwindex{Chamberlain, Houston Stewart 09.09.1855 – 09.01.1927@\textsc{Chamberlain, Houston Stewart} (09.09.1855 – 09.01.1927), \emph{Schriftsteller}!Richard WagnerOktober 1895@\strich\emph{Richard Wagner} {[}Oktober 1895{]}|pwv}.
               Bitte, theile mir die Schweiz\oindex{Schweiz@\textbf{Schweiz}|pw}er Adreſſe mit,
               damit ich danken kann. Und was wird aus dem \label{K_L02816-88v}\edtext{Opernglas}{\lemma{\textnormal{\emph{Opernglas}}}\Cendnote{\textnormal{Obzwar Goldmann\pwindex{Goldmann, Paul 31.01.1865 – 25.09.1935@\textsc{Goldmann, Paul} (31.01.1865 – 25.09.1935), \emph{Schriftsteller, Journalist}|pwk} bereits früher für Schnitzler\pwindex{Schnitzler, Arthur 15.05.1862 – 21.10.1931@\textsc{Schnitzler, Arthur} (15.05.1862 – 21.10.1931), \emph{Schriftsteller, Mediziner}|pwk} ein Opernglas besorgt hatte (vgl. Paul Goldmann an Arthur Schnitzler, 11. 1. [1896]), dürfte sich diese
                  Stelle auf eine neuerliche Bitte bezogen haben.}}}\label{K_L02816-88h}? Willſt Du mich denn unter
               allen Umſtänden zwingen, die 10 \textsc{Francs}, die Du mir dafür
               gegeben haſt, zu unterſchlagen? Bitte, laß’ mir die Redlichkeit meiner Seele, mein
               einziges Gut.\pend
           \pstart
           {\pb}Wenn ich daran denke, daß Du noch vor Kurzem hier
               geweſen biſt, ſo will ich es gar nicht glauben. Das iſt ſo fern, und ich bin ſo
               einſam!\pend
           \pstart
           Brauche ich Dir zu ſagen, daß es mein Herzenswunſch iſt, Dich in dieſem Sommer
                  \label{K_L02816-2v}\edtext{noch ein paar Tage zu ſehen}{\lemma{\textnormal{\emph{noch … ſehen}}}\Cendnote{\textnormal{Zwischen 19. 8. 1897 und 30. 8. 1897 sahen sich Schnitzler\pwindex{Schnitzler, Arthur 15.05.1862 – 21.10.1931@\textsc{Schnitzler, Arthur} (15.05.1862 – 21.10.1931), \emph{Schriftsteller, Mediziner}|pwk} und Goldmann\pwindex{Goldmann, Paul 31.01.1865 – 25.09.1935@\textsc{Goldmann, Paul} (31.01.1865 – 25.09.1935), \emph{Schriftsteller, Journalist}|pwk} mehrmals in Bad Ischl\oindex{Bad Ischl@\textbf{Bad Ischl}|pwk}
                  wieder.}}}\label{K_L02816-2h}? Aber die Reiſe nach \textsc{Ischl\oindex{Bad Ischl@\textbf{Bad Ischl}|pw}} iſt ſo weit und theuer. Für die Hin- und Rückfahrt geht allein \strikeout{\textcolor{gray}{×}\-\textcolor{gray}{×}\-\textcolor{gray}{×}\-\textcolor{gray}{×}} der
               größere Theil des Geldes drauf, das ich ausgeben \strikeout{k}
               kann. Ich kann noch gar nichts Beſtimmtes ſagen. Was würde mich die Penſion in \textsc{Ischl\oindex{Bad Ischl@\textbf{Bad Ischl}|pw} pro} Tag koſten? Natürlich dürfte
               das Zimmer nicht allzu ſchlecht ſein.\pend
           \pstart
           Fahre ich nach \textsc{Ischl\oindex{Bad Ischl@\textbf{Bad Ischl}|pw}}, ſo gehe ich über \label{K_L02816-3v}\edtext{\textsc{Bayreuth\oindex{Bayreuth@\textbf{Bayreuth}|pw}\orgindex{Bayreuther Festspiele@Bayreuther Festspiele|pwv}} zu einer der \textsc{Parſifal\pwindex{Wagner, Richard 22.05.1813 – 13.02.1883@\textsc{Wagner, Richard} (22.05.1813 – 13.02.1883), \emph{Komponist}!Parsifal1882@\strich\emph{Parsifal} {[}1882{]}|pw}}-Vorſtellungen{ }}{\lemma{\textnormal{\emph{Bayreuth … Parſifal-Vorſtellungen}}}\Cendnote{\textnormal{siehe Paul Goldmann an Arthur Schnitzler, 15. 6. [1897]}}}\label{K_L02816-3h}{\pb}am 8, 9, oder 11 Auguſt. Wenn Du
               ſchon nicht hinkommen kannſt, vielleicht kann \textsc{Richard\pwindex{Beer-Hofmann, Richard 1866-07-11 – 1945-09-26@\textsc{Beer-Hofmann, Richard} (1866-07-11 – 1945-09-26), \emph{Schriftsteller}|pw}} auf ein paar Tage herüberfahren? Es iſt nicht unmöglich, daß von hier aus \textsc{Maxime Dethomas\pwindex{Dethomas, Maxime 13.10.1867 – 1929-01-21@\textsc{Dethomas, Maxime} (13.10.1867 – 1929-01-21), \emph{Maler, Illustrator}|pw}} mitkommt. \textsc{Leo\pwindex{Van-Jung, Leo 15.10.1866 – 02.07.1939@\textsc{Van-Jung, Leo} (15.10.1866 – 02.07.1939), \emph{Gesangspädagoge, Mathematiker}|pw}} wiederzuſehen würde mich unendlich freuen. Von \textsc{Hugo\pwindex{Hofmannsthal, Hugo von 1874-02-01 – 1929-07-15@\textsc{Hofmannsthal, Hugo von} (1874-02-01 – 1929-07-15), \emph{Schriftsteller}|pw}} mag ich nichts wiſſen, ganz und gar nichts. Ich mag mir auch nicht die Mühe
               nehmen, ihn wiederzufinden. Er hätte mich ja blos nicht zu verlieren brauchen.\pend
           \pstart
           Vorgeſtern habe ich bei \textsc{Madame
                     Marni\pwindex{Marni, Jeanne 1854-01-31 – 1910-01-06@\textsc{Marni, Jeanne} (1854-01-31 – 1910-01-06), \emph{Schriftstellerin}|pw}} in \textsc{Louveciennes\oindex{Louveciennes@\textbf{Louveciennes}|pw}} gefrühſtückt. Sie hat ſich ſehr mit Deinen Grüßen gefreut und ſich
               angelegentlich nach Dir erkundigt.\pend
           \pstart
           Ich hoffe, es geht Dir gut in \label{K_L02816-5v}\edtext{\textsc{Ischl\oindex{Bad Ischl@\textbf{Bad Ischl}|pw}}}{\lemma{\textnormal{\emph{Ischl}}}\Cendnote{\textnormal{Schnitzler\pwindex{Schnitzler, Arthur 15.05.1862 – 21.10.1931@\textsc{Schnitzler, Arthur} (15.05.1862 – 21.10.1931), \emph{Schriftsteller, Mediziner}|pwk} hielt sich seit 26. 6. 1897 und noch
                  bis 24. 7. 1897 in
                     Ischl\oindex{Bad Ischl@\textbf{Bad Ischl}|pwk} auf.}}}\label{K_L02816-5h}. Mit beſonderer Freude
               habe ich vernommen, daß das {\pb}neue \label{K_L02816-7v}\edtext{Stück\pwindex{Schnitzler, Arthur 15.05.1862 – 21.10.1931@\textsc{Schnitzler, Arthur} (15.05.1862 – 21.10.1931), \emph{Schriftsteller, Mediziner}!Vermaechtnis. Schauspiel in drei Akten1898@\strich\emph{Das Vermächtnis. Schauspiel in drei Akten} {[}1898{]}|pwv}}{\lemma{\textnormal{\emph{Stück}}}\Cendnote{\textnormal{der Dreiakter \emph{Das Vermächtnis}\pwindex{Schnitzler, Arthur 15.05.1862 – 21.10.1931@\textsc{Schnitzler, Arthur} (15.05.1862 – 21.10.1931), \emph{Schriftsteller, Mediziner}!Vermaechtnis. Schauspiel in drei Akten1898@\strich\emph{Das Vermächtnis. Schauspiel in drei Akten} {[}1898{]}|pwk}, an dem Schnitzler\pwindex{Schnitzler, Arthur 15.05.1862 – 21.10.1931@\textsc{Schnitzler, Arthur} (15.05.1862 – 21.10.1931), \emph{Schriftsteller, Mediziner}|pwk} seit dem 26. 6. 1897 arbeitete}}}\label{K_L02816-7h} zum Leben erwacht. Trag’ es nur mit Dir
               herum, bis die gewünſchte Klarheit da iſt. Und wenn Du Dich jetzt nicht zum Arbeiten
               geſtimmt fühlſt, ſo überſtürze es nicht und laß’ Dir Ruhe. Es iſt durchaus nicht
               nöthig, daß Du für die nächſte Saiſon gleich wieder mit einem neuen Stücke da
               biſt.\pend
           \pstart
           Schreib’ mir, bitte, recht bald und recht ausführlich: 1.) Wie es Dir geht (\label{K_L02816-9v}\edtext{körperlich}{\lemma{\textnormal{\emph{körperlich}}}\Cendnote{\textnormal{Schnitzler\pwindex{Schnitzler, Arthur 15.05.1862 – 21.10.1931@\textsc{Schnitzler, Arthur} (15.05.1862 – 21.10.1931), \emph{Schriftsteller, Mediziner}|pwk} notierte zu dieser Zeit keine
                  akuten Beschwerden im \emph{Tagebuch}\pwindex{Schnitzler, Arthur 15.05.1862 – 21.10.1931@\textsc{Schnitzler, Arthur} (15.05.1862 – 21.10.1931), \emph{Schriftsteller, Mediziner}!Tagebuch1981 – 2000@\strich\emph{Tagebuch} {[}1981 – 2000{]}|pwk}, hatte aber
                  seit Herbst 1896 fortlaufend mit seiner Otosklerose zu
                  kämpfen.}}}\label{K_L02816-9h} auch)? 2.) Wie Du \label{K_L02816-11v}\edtext{\textsc{Richard\pwindex{Beer-Hofmann, Richard 1866-07-11 – 1945-09-26@\textsc{Beer-Hofmann, Richard} (1866-07-11 – 1945-09-26), \emph{Schriftsteller}|pw}\pwindex{Beer-Hofmann, Richard 1866-07-11 – 1945-09-26@\textsc{Beer-Hofmann, Richard} (1866-07-11 – 1945-09-26), \emph{Schriftsteller}!Tod Georgs1900@\strich\emph{Der Tod Georgs} {[}1900{]}|pwv}} gefunden}{\lemma{\textnormal{\emph{Richard gefunden}}}\Cendnote{\textnormal{Richard Beer-Hofmann\pwindex{Beer-Hofmann, Richard 1866-07-11 – 1945-09-26@\textsc{Beer-Hofmann, Richard} (1866-07-11 – 1945-09-26), \emph{Schriftsteller}|pwk} arbeitete an der
                  Erzählung \emph{Der Tod Georgs}\pwindex{Beer-Hofmann, Richard 1866-07-11 – 1945-09-26@\textsc{Beer-Hofmann, Richard} (1866-07-11 – 1945-09-26), \emph{Schriftsteller}!Tod Georgs1900@\strich\emph{Der Tod Georgs} {[}1900{]}|pwk}, damals noch unter
                  dem Titel Der
                     Götterliebling\pwindex{Beer-Hofmann, Richard 1866-07-11 – 1945-09-26@\textsc{Beer-Hofmann, Richard} (1866-07-11 – 1945-09-26), \emph{Schriftsteller}!Tod Georgs1900@\strich\emph{Der Tod Georgs} {[}1900{]}|pwkv}. Er hatte Schnitzler\pwindex{Schnitzler, Arthur 15.05.1862 – 21.10.1931@\textsc{Schnitzler, Arthur} (15.05.1862 – 21.10.1931), \emph{Schriftsteller, Mediziner}|pwk}
                  daraus bereits am 1. 1. 1897 vorgelesen und tat es auch kurz nach diesem Brief, am
                     17. 7. 1897.}}}\label{K_L02816-11h} haſt? 3.) Welche Nachrichten Du aus der Schweiz\oindex{Schweiz@\textbf{Schweiz}|pw} haſt? Und was weiter geſchehen wird?\pend
           \pstart
           Ich begrüße Dich von Herzen und in Treue {\\[\baselineskip]}Dein{\\[\baselineskip]}\spacefill\mbox{Paul Goldmann.}\pend
           \leftskip=0em{}\pstart
           \noindent{}Wenn Deine Frau Mutter\pwindex{Schnitzler, Louise 1840-07-08 – 1911-09-09@\textsc{Schnitzler, Louise} (1840-07-08 – 1911-09-09)|pwv}{ }\label{K_L02816-67v}\edtext{mit Dir iſt}{\lemma{\textnormal{\emph{mit Dir iſt}}}\Cendnote{\textnormal{Louise Schnitzler\pwindex{Schnitzler, Louise 1840-07-08 – 1911-09-09@\textsc{Schnitzler, Louise} (1840-07-08 – 1911-09-09)|pwk} kam am 3. 7. 1897 in Bad Ischl\oindex{Bad Ischl@\textbf{Bad Ischl}|pwk} an.}}}\label{K_L02816-67h}, ſo empfiehl’ mich,
                  bitte.\pend
           
         
         \endnumbering\mylabel{h}\end{ledgroupsized}  \newcommand{\dateiname}{L02816}\newcommand{\titel}{Paul Goldmann an Arthur Schnitzler, 2. 7. [1897]}\newcommand{\editorInnen}{Martin Anton Müller und Laura Untner}%% latex-leseansicht-abspann.tex
%% Abspann für die Leseansicht.
%% Der Schalter \ifkorrekturansicht ist bereits durch den Vorspann gesetzt.

%% latex-abspann.tex
%% Gemeinsamer Abspann für Korrekturansicht und Leseansicht.
%% Setzt den Schalter \ifkorrekturansicht voraus (gesetzt in den
%% einbindenden Dateien latex-korrekturansicht-abspann.tex bzw.
%% latex-leseansicht-abspann.tex).
%% ---------------------------------------------------------------

\normalsize

% Das esempio-Environment wird nur in der Leseansicht benötigt
\ifkorrekturansicht\else
\newenvironment{esempio}[3]%
{
    \vspace{1.5ex}
    \rlap{\underline{#1}}
    \par
    \setlength{\parindent}{0cm}
    \nopagebreak
    \leftskip=#2cm
    \rightskip=#3cm
}
{
    \par
}
\fi

\doendnotes{C}
\bigskip
\vfill

\clearpage

\footnotesize

\ifkorrekturansicht
  \lohead{\textsc{register}}
\fi

% theindex-Environment neu definieren ohne reledmac
\makeatletter
\renewenvironment{theindex}{%
  \ifkorrekturansicht
    \section*{\indexname}%
  \else
    \subsubsection*{Index der erwähnten Entitäten}%
  \fi
  \setlength{\parindent}{0pt}%
  \setlength{\parskip}{0pt plus 0.3pt}%
  \let\item\@idxitem
}{%
  \ifkorrekturansicht\clearpage\fi
}
\makeatother

\IfFileExists{\jobname-pw.ind}{\input{\jobname-pw.ind}}{}

% Quellenangabe nur in der Leseansicht
\ifkorrekturansicht\else
% Fallback-Definitionen, falls die .tex-Datei \titel etc. nicht gesetzt hat
\providecommand{\titel}{}
\providecommand{\editorInnen}{}
\providecommand{\dateiname}{\jobname}

\vspace{3cm}

\vfill

\footnotesize
\textsc{Quelle}: \titel. Herausgegeben von {\editorInnen}. In: \emph{Arthur Schnitzler: Briefwechsel mit Autorinnen und Autoren}.
 Digitale Edition, https://schnitzler-briefe.acdh.oeaw.ac.at/{\dateiname}.html (Stand \today)
\fi

\end{document}


      