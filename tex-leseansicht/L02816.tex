%% latex-leseansicht-vorspann.tex
%% Vorspann für die Leseansicht.
%% Lädt die gemeinsame Datei latex-vorspann.tex mit nicht gesetztem Schalter.

\newif\ifkorrekturansicht
\korrekturansichtfalse

\input{../tex-inputs/latex-vorspann}


\section[ Paul Goldmann an Arthur Schnitzler, 2. 7. {[}1897{]}]{L02816 Paul Goldmann an Arthur Schnitzler,  2. 7. [1897]}
\nopagebreak\mylabel{L02816v}
\rehead{ }\normalsize\beginnumbering\briefempfaengerindex{Schnitzler, Arthur@\textsc{Schnitzler, Arthur}!zzzGoldmann, Paul@\emph{von Paul Goldmann}!1897-07-021@{2. 7. [1897]}|(be}
\toendnotes[C]{\smallbreak\pagebreak[2]}
\correspDesc{Versand  durch Paul Goldmann am 2. 7. [1897] in Paris
\newline{}Erhalt  durch Arthur Schnitzler im Zeitraum [3. 7. 1897
                  – 7. 7. 1897?] in Bad Ischl}\toendnotes[C]{\smallbreak}
\Standort{DLA, A:Schnitzler, HS.NZ85.1.3167.}
\physDesc{Brief, 1 Blatt, 4 Seiten, 2504 Zeichen
\newline{}Handschrift: blaue Tinte, deutsche Kurrent
\newline{}Schnitzler: 1) mit Bleistift das Jahr »97« vermerkt  2) mit rotem Buntstift fünf Unterstreichungen}\toendnotes[C]{\smallbreak}
\pstart
           {\pb}\textcolor{gray}{\textbf{\textbf{Frankfurter Zeitung\orgindex{Frankfurter Zeitung@Frankfurter Zeitung|pw}}}}\pend
           
\pstart
           \textcolor{gray}{\textbf{(\begin{otherlanguage}{french}Gazette de Francfort\end{otherlanguage}\orgindex{Frankfurter Zeitung@Frankfurter Zeitung|pw}).}}\pend
           
\pstart
           \textcolor{gray}{\textbf{\textbf{\begin{otherlanguage}{french}Fondateur M.\end{otherlanguage}{ }L. Sonnemann\pwindex{Sonnemann, Leopold 29.\,10.\,1831 Höchberg – 30.\,10.\,1909 Frankfurt am Main@\textsc{Sonnemann, Leopold} (29.\,10.\,1831 Höchberg – 30.\,10.\,1909 Frankfurt am Main), \emph{Journalist, Herausgeber}|pw}.}}}\pend
           
\pstart
           \begin{otherlanguage}{french}\textcolor{gray}{\textbf{Journal politique, financier,}}\end{otherlanguage}\hfill \textsc{Paris\oindex{Paris@\textbf{Paris}, \emph{Hauptstadt}|pw}}, 2. Juli.\pend
           
\pstart
           \begin{otherlanguage}{french}\textcolor{gray}{\textbf{commercial et littéraire.}}\end{otherlanguage}\pend
           
\pstart
           \begin{otherlanguage}{french}\textcolor{gray}{\textbf{\textbf{Paraissant trois fois par jour.}}}\end{otherlanguage}\pend
           
\pstart
           \begin{otherlanguage}{french}\textcolor{gray}{\textbf{\textbf{Bureau à Paris\oindex{Paris@\textbf{Paris}, \emph{Hauptstadt}|pw}}}}\end{otherlanguage}\pend
           
\pstart
           \begin{otherlanguage}{french}\textcolor{gray}{\textbf{\textbf{10 \so{Rue de la Bourse}\oindex{rue de la Bourse@\textbf{rue de la Bourse}, \emph{Straße}|pw}.}}}\end{otherlanguage}\pend
           
\pstart\center{}Mein lieber Freund,\pend\vspace{0.5em}
\pstart
           Ich danke Dir für Deinen lieben Brief und Deine Correſpondenz-Karte. All’ dieſe Tage
               konnte ich nicht die Zeit zur Antwort finden. Auch bin ich krank und mißmuthig.\pend
           
\pstart
           \label{K_L02816-1v}\edtext{Aus der Schweiz\oindex{Schweiz@\textbf{Schweiz}|pw}}{\lemma{\textnormal{\emph{Aus der Schweiz}}}\Cendnote{\textnormal{Marie Reinhard\pwindex{Reinhard, Marie 13.\,3.\,1871 Wien – 18.\,3.\,1899 ebd.@\textsc{Reinhard, Marie} (13.\,3.\,1871 Wien – 18.\,3.\,1899 ebd.), \emph{Gesangspädagogin}|pwk} war zu dieser Zeit mit ihrer
                     Mutter\pwindex{Reinhard, Therese 13.\,12.\,1844 Wien – 25.\,3.\,1926 ebd.@\textsc{Reinhard, Therese} (13.\,12.\,1844 Wien – 25.\,3.\,1926 ebd.)|pwkv} in Andermatt\oindex{Andermatt@\textbf{Andermatt}|pwk} – vor allem, um die Schwangerschaft
                  vor der Wien\oindex{Wien@\textbf{Wien}, \emph{Verwaltungsgebiet}|pwk}er Gesellschaft zu verbergen. Siehe A. S.: \emph{Tagebuch}, 13. 7. 1897.}}}\label{K_L02816-1} habe ich
               plötzlich die \label{K_L02816-2v}\edtext{\textsc{Wagner\pwindex{Wagner, Richard 22.\,5.\,1813 Leipzig – 13.\,2.\,1883 Venedig@\textsc{Wagner, Richard} (22.\,5.\,1813 Leipzig – 13.\,2.\,1883 Venedig), \emph{Komponist}|pw}}-Biographie\pwindex{Chamberlain, Houston Stewart 9.\,9.\,1855 Portsmouth – 9.\,1.\,1927 Bayreuth@\textsc{Chamberlain, Houston Stewart} (9.\,9.\,1855 Portsmouth – 9.\,1.\,1927 Bayreuth), \emph{Schriftsteller}!Richard Wagner@\strich\emph{Richard Wagner}|pwv}}{\lemma{\textnormal{\emph{Wagner-Biographie}}}\Cendnote{\textnormal{Vermutlich: Houston Stewart Chamberlain\pwindex{Chamberlain, Houston Stewart 9.\,9.\,1855 Portsmouth – 9.\,1.\,1927 Bayreuth@\textsc{Chamberlain, Houston Stewart} (9.\,9.\,1855 Portsmouth – 9.\,1.\,1927 Bayreuth), \emph{Schriftsteller}|pwk}: \emph{Richard Wagner}\pwindex{Chamberlain, Houston Stewart 9.\,9.\,1855 Portsmouth – 9.\,1.\,1927 Bayreuth@\textsc{Chamberlain, Houston Stewart} (9.\,9.\,1855 Portsmouth – 9.\,1.\,1927 Bayreuth), \emph{Schriftsteller}!Richard Wagner@\strich\emph{Richard Wagner}|pwk}. Mit zahlreichen Porträts,
                     Faksimiles, Illustrationen und Beilagen. München:
                        \emph{Verlagsanstalt für Kunst und Wissenschaft (vormals Friedrich
                        Bruckmann)}{ }1896 [vordatiert von Oktober 1895].
               }}}\label{K_L02816-2} erhalten. Ihr\pwindex{Reinhard, Marie 13.\,3.\,1871 Wien – 18.\,3.\,1899 ebd.@\textsc{Reinhard, Marie} (13.\,3.\,1871 Wien – 18.\,3.\,1899 ebd.), \emph{Gesangspädagogin}|pwv}{ }ſeid
               wirklich zu lieb und gut! Ich hoffte{ }ſchon, Ihr hättet es vergeſſen. Ich freue mich{ }ſehr über das{ }ſchöne Buch\pwindex{Chamberlain, Houston Stewart 9.\,9.\,1855 Portsmouth – 9.\,1.\,1927 Bayreuth@\textsc{Chamberlain, Houston Stewart} (9.\,9.\,1855 Portsmouth – 9.\,1.\,1927 Bayreuth), \emph{Schriftsteller}!Richard Wagner@\strich\emph{Richard Wagner}|pwv}.
               Bitte, theile mir die Schweiz\oindex{Schweiz@\textbf{Schweiz}|pw}er Adreſſe mit,
               damit ich danken kann. Und was wird aus dem \label{K_L02816-3v}\edtext{Opernglas}{\lemma{\textnormal{\emph{Opernglas}}}\Cendnote{\textnormal{Obzwar Goldmann\pwindex{Goldmann, Paul 31.\,1.\,1865 Breslau – 25.\,9.\,1935 Wien@\textsc{Goldmann, Paul} (31.\,1.\,1865 Breslau – 25.\,9.\,1935 Wien), \emph{Schriftsteller, Journalist}|pwk} bereits früher für Schnitzler ein Opernglas besorgt hatte (vgl. XXXX Auszeichnungsfehler: Dokument L02762 nicht gefunden), dürfte sich diese
                  Stelle auf eine neuerliche Bitte beziehen.}}}\label{K_L02816-3}? Willſt Du mich denn unter
               allen Umſtänden zwingen, die 10 \textsc{Francs}, die Du mir dafür
               gegeben haſt, zu unterſchlagen? Bitte, laß’ mir die Redlichkeit meiner Seele, mein
               einziges Gut.\pend
           
\pstart
           {\pb}Wenn ich daran denke, daß Du noch vor Kurzem hier
               geweſen biſt,{ }ſo will ich es gar nicht glauben. Das iſt{ }ſo fern, und ich bin{ }ſo
               einſam!\pend
           
\pstart
           Brauche ich Dir zu{ }ſagen, daß es mein Herzenswunſch iſt, Dich in dieſem Sommer
                  \label{K_L02816-4v}\edtext{noch ein paar Tage zu{ }ſehen}{\lemma{\textnormal{\emph{noch … sehen}}}\Cendnote{\textnormal{Zwischen 19. 8. 1897 und 30. 8. 1897 sahen sich Schnitzler und Goldmann\pwindex{Goldmann, Paul 31.\,1.\,1865 Breslau – 25.\,9.\,1935 Wien@\textsc{Goldmann, Paul} (31.\,1.\,1865 Breslau – 25.\,9.\,1935 Wien), \emph{Schriftsteller, Journalist}|pwk} mehrmals in Bad Ischl\oindex{Bad Ischl@\textbf{Bad Ischl}|pwk}
                  wieder.}}}\label{K_L02816-4}? Aber die Reiſe nach \textsc{Ischl\oindex{Bad Ischl@\textbf{Bad Ischl}|pw}} iſt{ }ſo weit und theuer. Für die Hin- und Rückfahrt geht allein \strikeout{\textcolor{gray}{×}\-\textcolor{gray}{×}\-\textcolor{gray}{×}\-\textcolor{gray}{×}} der größere Theil des Geldes drauf, das ich ausgeben \strikeout{k} kann. Ich kann noch gar nichts Beſtimmtes{ }ſagen. Was würde mich die
               Penſion in \textsc{Ischl\oindex{Bad Ischl@\textbf{Bad Ischl}|pw} pro} Tag koſten? Natürlich dürfte
               das Zimmer nicht allzu{ }ſchlecht{ }ſein.\pend
           
\pstart
           Fahre ich nach \textsc{Ischl\oindex{Bad Ischl@\textbf{Bad Ischl}|pw}},{ }ſo gehe ich über \label{K_L02816-5v}\edtext{\textsc{Bayreuth\oindex{Bayreuth@\textbf{Bayreuth}, \emph{Hauptstadt}|pw}\orgindex{Bayreuther Festspiele@Bayreuther Festspiele|pwv}} zu einer der \textsc{Parſifal\pwindex{Wagner, Richard 22.\,5.\,1813 Leipzig – 13.\,2.\,1883 Venedig@\textsc{Wagner, Richard} (22.\,5.\,1813 Leipzig – 13.\,2.\,1883 Venedig), \emph{Komponist}!Parsifal@\strich\emph{Parsifal}|pw}}-Vorſtellungen{ }}{\lemma{\textnormal{\emph{Bayreuth … Parsifal-Vorstellungen}}}\Cendnote{\textnormal{Siehe XXXX Auszeichnungsfehler: Dokument L02814 nicht gefunden.
               }}}\label{K_L02816-5}{\pb}am 8, 9, oder 11 Auguſt. Wenn Du{ }ſchon nicht hinkommen kannſt, vielleicht kann \textsc{Richard\pwindex{Beer-Hofmann, Richard 11.\,7.\,1866 Wien – 26.\,9.\,1945 New York City@\textsc{Beer-Hofmann, Richard} (11.\,7.\,1866 Wien – 26.\,9.\,1945 New York City), \emph{Schriftsteller}|pw}} auf ein paar Tage herüberfahren? Es iſt nicht unmöglich, daß von hier aus \textsc{Maxime Dethomas\pwindex{Dethomas, Maxime 13.\,10.\,1867 Garches – 21.\,1.\,1929 Paris@\textsc{Dethomas, Maxime} (13.\,10.\,1867 Garches – 21.\,1.\,1929 Paris), \emph{Maler, Illustrator}|pw}} mitkommt. \textsc{Leo\pwindex{Van-Jung, Leo 15.\,10.\,1866 Odessa – 2.\,7.\,1939 Riga@\textsc{Van-Jung, Leo} (15.\,10.\,1866 Odessa – 2.\,7.\,1939 Riga), \emph{Gesangspädagoge, Mathematiker}|pw}} wiederzuſehen würde mich unendlich freuen. Von \textsc{Hugo\pwindex{Hofmannsthal, Hugo von 1.\,2.\,1874 Wien – 15.\,7.\,1929 Rodaun@\textsc{Hofmannsthal, Hugo von} (1.\,2.\,1874 Wien – 15.\,7.\,1929 Rodaun), \emph{Schriftsteller}|pw}} mag ich nichts wiſſen, ganz und gar nichts. Ich mag mir auch nicht die Mühe
               nehmen, ihn wiederzufinden. Er hätte mich ja blos nicht zu verlieren brauchen.\pend
           
\pstart
           Vorgeſtern habe ich bei \textsc{Madame
                     Marni\pwindex{Marni, Jeanne 31.\,1.\,1854 Toulouse – 6.\,1.\,1910 Cannes@\textsc{Marni, Jeanne} (31.\,1.\,1854 Toulouse – 6.\,1.\,1910 Cannes), \emph{Schriftstellerin}|pw}} in \textsc{Louveciennes\oindex{Louveciennes@\textbf{Louveciennes}|pw}} gefrühſtückt. Sie hat{ }ſich{ }ſehr mit Deinen Grüßen gefreut und{ }ſich
               angelegentlich nach Dir erkundigt.\pend
           
\pstart
           Ich hoffe, es geht Dir gut in \label{K_L02816-6v}\edtext{\textsc{Ischl\oindex{Bad Ischl@\textbf{Bad Ischl}|pw}}}{\lemma{\textnormal{\emph{Ischl}}}\Cendnote{\textnormal{Schnitzler hielt sich seit 26. 6. 1897 und noch
                  bis 24. 7. 1897 in
                     Ischl\oindex{Bad Ischl@\textbf{Bad Ischl}|pwk} auf.}}}\label{K_L02816-6}. Mit beſonderer Freude
               habe ich vernommen, daß das {\pb}neue \label{K_L02816-7v}\edtext{Stück\pwindex{Schnitzler, Arthur 15.\,5.\,1862 Wien – 21.\,10.\,1931 ebd.@\textsc{Schnitzler, Arthur} (15.\,5.\,1862 Wien – 21.\,10.\,1931 ebd.), \emph{Schriftsteller, Mediziner}!Vermächtnis. Schauspiel in drei Akten@\strich\emph{Das Vermächtnis. Schauspiel in drei Akten}|pwv}}{\lemma{\textnormal{\emph{Stück}}}\Cendnote{\textnormal{der Dreiakter \emph{Das Vermächtnis}\pwindex{Schnitzler, Arthur 15.\,5.\,1862 Wien – 21.\,10.\,1931 ebd.@\textsc{Schnitzler, Arthur} (15.\,5.\,1862 Wien – 21.\,10.\,1931 ebd.), \emph{Schriftsteller, Mediziner}!Vermächtnis. Schauspiel in drei Akten@\strich\emph{Das Vermächtnis. Schauspiel in drei Akten}|pwk}, an dem Schnitzler seit dem 26. 6. 1897 arbeitete}}}\label{K_L02816-7} zum Leben erwacht. Trag’ es nur mit Dir
               herum, bis die gewünſchte Klarheit da iſt. Und wenn Du Dich jetzt nicht zum Arbeiten
               geſtimmt fühlſt,{ }ſo überſtürze es nicht und laß’ Dir Ruhe. Es iſt durchaus nicht
               nöthig, daß Du für die nächſte Saiſon gleich wieder mit einem neuen Stücke da
               biſt.\pend
           
\pstart
           Schreib’ mir, bitte, recht bald und recht ausführlich: 1.) Wie es Dir geht (\label{K_L02816-8v}\edtext{körperlich}{\lemma{\textnormal{\emph{körperlich}}}\Cendnote{\textnormal{Schnitzler notierte zu dieser Zeit keine
                  akuten Beschwerden im \emph{Tagebuch}\pwindex{Schnitzler, Arthur 15.\,5.\,1862 Wien – 21.\,10.\,1931 ebd.@\textsc{Schnitzler, Arthur} (15.\,5.\,1862 Wien – 21.\,10.\,1931 ebd.), \emph{Schriftsteller, Mediziner}!Tagebuch@\strich\emph{Tagebuch}|pwk}, hatte aber
                  seit Herbst 1896 fortlaufend mit seiner Otosklerose zu
                  kämpfen.}}}\label{K_L02816-8} auch)? 2.) Wie Du \label{K_L02816-9v}\edtext{\textsc{Richard\pwindex{Beer-Hofmann, Richard 11.\,7.\,1866 Wien – 26.\,9.\,1945 New York City@\textsc{Beer-Hofmann, Richard} (11.\,7.\,1866 Wien – 26.\,9.\,1945 New York City), \emph{Schriftsteller}|pw}\pwindex{Beer-Hofmann, Richard 11.\,7.\,1866 Wien – 26.\,9.\,1945 New York City@\textsc{Beer-Hofmann, Richard} (11.\,7.\,1866 Wien – 26.\,9.\,1945 New York City), \emph{Schriftsteller}!Tod Georgs@\strich\emph{Der Tod Georgs}|pwv}} gefunden}{\lemma{\textnormal{\emph{Richard gefunden}}}\Cendnote{\textnormal{Richard Beer-Hofmann\pwindex{Beer-Hofmann, Richard 11.\,7.\,1866 Wien – 26.\,9.\,1945 New York City@\textsc{Beer-Hofmann, Richard} (11.\,7.\,1866 Wien – 26.\,9.\,1945 New York City), \emph{Schriftsteller}|pwk} arbeitete an der
                  Erzählung \emph{Der Tod Georgs}\pwindex{Beer-Hofmann, Richard 11.\,7.\,1866 Wien – 26.\,9.\,1945 New York City@\textsc{Beer-Hofmann, Richard} (11.\,7.\,1866 Wien – 26.\,9.\,1945 New York City), \emph{Schriftsteller}!Tod Georgs@\strich\emph{Der Tod Georgs}|pwk}, damals noch unter
                  dem Titel \emph{Der
                     Götterliebling}\pwindex{Beer-Hofmann, Richard 11.\,7.\,1866 Wien – 26.\,9.\,1945 New York City@\textsc{Beer-Hofmann, Richard} (11.\,7.\,1866 Wien – 26.\,9.\,1945 New York City), \emph{Schriftsteller}!Tod Georgs@\strich\emph{Der Tod Georgs}|pwk}. Er hatte Schnitzler
                  daraus bereits am 1. 1. 1897 vorgelesen und tat es auch kurz nach diesem Brief, am
                     17. 7. 1897.}}}\label{K_L02816-9} haſt? 3.) Welche Nachrichten Du aus der Schweiz\oindex{Schweiz@\textbf{Schweiz}|pw} haſt? Und was weiter geſchehen wird?\pend
           
\pstart
           Ich begrüße Dich von Herzen und in Treue {\\[\baselineskip]}Dein{\\[\baselineskip]}\spacefill\mbox{Paul Goldmann.}\pend
           \leftskip=0em{}
\pstart
           \noindent{}Wenn Deine Frau Mutter\pwindex{Schnitzler, Louise 8.\,7.\,1840 Kőszeg – 9.\,9.\,1911 Wien@\textsc{Schnitzler, Louise} (8.\,7.\,1840 Kőszeg – 9.\,9.\,1911 Wien)|pwv}{ }\label{K_L02816-10v}\edtext{mit Dir iſt}{\lemma{\textnormal{\emph{mit Dir ist}}}\Cendnote{\textnormal{Louise Schnitzler\pwindex{Schnitzler, Louise 8.\,7.\,1840 Kőszeg – 9.\,9.\,1911 Wien@\textsc{Schnitzler, Louise} (8.\,7.\,1840 Kőszeg – 9.\,9.\,1911 Wien)|pwk} kam am 3. 7. 1897 in Bad Ischl\oindex{Bad Ischl@\textbf{Bad Ischl}|pwk} an.}}}\label{K_L02816-10},{ }ſo empfiehl’ mich,
                  bitte.\pend
           \selectlanguage{ngerman}\endnumbering\briefempfaengerindex{Schnitzler, Arthur@\textsc{Schnitzler, Arthur}!zzzGoldmann, Paul@\emph{von Paul Goldmann}!1897-07-021@{2. 7. [1897]}|)be}\mylabel{L02816h}  \newcommand{\dateiname}{L02816}\newcommand{\titel}{Paul Goldmann an Arthur Schnitzler, 2. 7. [1897]}\newcommand{\editorInnen}{Martin Anton Müller und Laura Untner}%% latex-leseansicht-abspann.tex
%% Abspann für die Leseansicht.
%% Der Schalter \ifkorrekturansicht ist bereits durch den Vorspann gesetzt.

%% latex-abspann.tex
%% Gemeinsamer Abspann für Korrekturansicht und Leseansicht.
%% Setzt den Schalter \ifkorrekturansicht voraus (gesetzt in den
%% einbindenden Dateien latex-korrekturansicht-abspann.tex bzw.
%% latex-leseansicht-abspann.tex).
%% ---------------------------------------------------------------

\normalsize

% Das esempio-Environment wird nur in der Leseansicht benötigt
\ifkorrekturansicht\else
\newenvironment{esempio}[3]%
{
    \vspace{1.5ex}
    \rlap{\underline{#1}}
    \par
    \setlength{\parindent}{0cm}
    \nopagebreak
    \leftskip=#2cm
    \rightskip=#3cm
}
{
    \par
}
\fi

\doendnotes{C}
\bigskip
\vfill

\clearpage

\footnotesize

\ifkorrekturansicht
  \lohead{\textsc{register}}
\fi

% theindex-Environment neu definieren ohne reledmac
\makeatletter
\renewenvironment{theindex}{%
  \ifkorrekturansicht
    \section*{\indexname}%
  \else
    \subsubsection*{Index der erwähnten Entitäten}%
  \fi
  \setlength{\parindent}{0pt}%
  \setlength{\parskip}{0pt plus 0.3pt}%
  \let\item\@idxitem
}{%
  \ifkorrekturansicht\clearpage\fi
}
\makeatother

\IfFileExists{\jobname-pw.ind}{\input{\jobname-pw.ind}}{}

% Quellenangabe nur in der Leseansicht
\ifkorrekturansicht\else
% Fallback-Definitionen, falls die .tex-Datei \titel etc. nicht gesetzt hat
\providecommand{\titel}{}
\providecommand{\editorInnen}{}
\providecommand{\dateiname}{\jobname}

\vspace{3cm}

\vfill

\footnotesize
\textsc{Quelle}: \titel. Herausgegeben von {\editorInnen}. In: \emph{Arthur Schnitzler: Briefwechsel mit Autorinnen und Autoren}.
 Digitale Edition, https://schnitzler-briefe.acdh.oeaw.ac.at/{\dateiname}.html (Stand \today)
\fi

\end{document}


