%% latex-leseansicht-vorspann.tex
%% Vorspann für die Leseansicht.
%% Lädt die gemeinsame Datei latex-vorspann.tex mit nicht gesetztem Schalter.

\newif\ifkorrekturansicht
\korrekturansichtfalse

\input{../tex-inputs/latex-vorspann}


               \section[Richard Beer-Hofmann an Arthur Schnitzler, 9. 11. 1904]{ Richard Beer-Hofmann an Arthur Schnitzler,
               9. 11. 1904}\nopagebreak\mylabel{v}\rehead{ }\begin{ledgroupsized}[t]{13cm}\normalsize\beginnumbering\briefempfaengerindex{Schnitzler, Arthur@\textsc{Schnitzler, Arthur}!zzzBeer-Hofmann, Richard@\emph{von Richard Beer-Hofmann}!1904-11-091@{9. 11. 1904}|(be} \toendnotes[C]{\smallbreak\pagebreak[2]} \Standort{CUL, Schnitzler, B 8.}
\physDesc{Brief, 1 Blatt, 1 Seite
\newline{}Handschrift: schwarze Tinte, lateinische Kurrent\newline{}Ordnung: mit Bleistift von unbekannter Hand nummeriert: »195« }\buchAbdrucke{\weitereDrucke{Arthur Schnitzler, Richard Beer-Hofmann: \emph{Briefwechsel 1891–1931}. Hg. Konstanze Fliedl. Wien, Zürich: \emph{Europaverlag} 1992, S. 169.} }\toendnotes[C]{\smallbreak}\pstart
           \centering{}\uline{Noch} – Rodaun\oindex{Rodaun@\textbf{Rodaun}|pw}{ }9./XI. 04\pend
           \pstart
           Lieber Arthur! Ich bin selbstverständlich ohne jede Nachricht von
                  Berlin\oindex{Berlin@\textbf{Berlin}|pw}. Werde morgen telegraphiren. Wenn
               erfolglos, werde ich Alles auf Ihre Schultern laden. Jedenfalls:\pend
           \pstart
           1) Wann fahren Sie – \label{K_L01468_1v}\edtext{Samstag}{\lemma{\textnormal{\emph{Samstag}}}\Cendnote{\textnormal{vgl. A. S.: \emph{Tagebuch}, 12. 11. 1904}}}\label{K_L01468_1h}? \introOben{}(Stunde Bahnhof)\introOben{}\pend
           \pstart
           2.) Wo wohnen Sie in Berlin\oindex{Berlin@\textbf{Berlin}|pw}?\pend
           \pstart
           Mein Hausherr\pwindex{Berger, Rudolf *~10.9.1858@\textsc{Berger, Rudolf} (*~10.9.1858), \emph{Vermieter, Metzger}|pwv}? »Arisch«
               »Bodenständig« »Deutsche Biederkeit« »Ehrliches Bürgerthum« »Gerader deutscher Sinn«
               »Abhold jeder Tücke« »Germanische Treue«. Sie – die Selcherin\pwindex{Berger @\textsc{Berger}, \emph{Metzgerin}|pwv} – hat einen Hausaltar – und die Kinder\pwindex{Berger @\textsc{Berger}|pwv} ko{\geminationm}en nach
                  Kalksburg\orgindex{Kollegium Kalksburg@Kollegium Kalksburg|pw}.\pend
           \pstart
           Herzlichst Ihr{\\[\baselineskip]}\spacefill\mbox{Richard}\pend
           \leftskip=0em{}\endnumbering\briefempfaengerindex{Schnitzler, Arthur@\textsc{Schnitzler, Arthur}!zzzBeer-Hofmann, Richard@\emph{von Richard Beer-Hofmann}!1904-11-091@{9. 11. 1904}|)be}\mylabel{h}\end{ledgroupsized}  \newcommand{\dateiname}{L01468}\newcommand{\titel}{Richard Beer-Hofmann an Arthur Schnitzler, 9. 11. 1904}\newcommand{\editorInnen}{Martin Anton Müller und Gerd-Hermann Susen}%% latex-leseansicht-abspann.tex
%% Abspann für die Leseansicht.
%% Der Schalter \ifkorrekturansicht ist bereits durch den Vorspann gesetzt.

%% latex-abspann.tex
%% Gemeinsamer Abspann für Korrekturansicht und Leseansicht.
%% Setzt den Schalter \ifkorrekturansicht voraus (gesetzt in den
%% einbindenden Dateien latex-korrekturansicht-abspann.tex bzw.
%% latex-leseansicht-abspann.tex).
%% ---------------------------------------------------------------

\normalsize

% Das esempio-Environment wird nur in der Leseansicht benötigt
\ifkorrekturansicht\else
\newenvironment{esempio}[3]%
{
    \vspace{1.5ex}
    \rlap{\underline{#1}}
    \par
    \setlength{\parindent}{0cm}
    \nopagebreak
    \leftskip=#2cm
    \rightskip=#3cm
}
{
    \par
}
\fi

\doendnotes{C}
\bigskip
\vfill

\clearpage

\footnotesize

\ifkorrekturansicht
  \lohead{\textsc{register}}
\fi

% theindex-Environment neu definieren ohne reledmac
\makeatletter
\renewenvironment{theindex}{%
  \ifkorrekturansicht
    \section*{\indexname}%
  \else
    \subsubsection*{Index der erwähnten Entitäten}%
  \fi
  \setlength{\parindent}{0pt}%
  \setlength{\parskip}{0pt plus 0.3pt}%
  \let\item\@idxitem
}{%
  \ifkorrekturansicht\clearpage\fi
}
\makeatother

\IfFileExists{\jobname-pw.ind}{\input{\jobname-pw.ind}}{}

% Quellenangabe nur in der Leseansicht
\ifkorrekturansicht\else
% Fallback-Definitionen, falls die .tex-Datei \titel etc. nicht gesetzt hat
\providecommand{\titel}{}
\providecommand{\editorInnen}{}
\providecommand{\dateiname}{\jobname}

\vspace{3cm}

\vfill

\footnotesize
\textsc{Quelle}: \titel. Herausgegeben von {\editorInnen}. In: \emph{Arthur Schnitzler: Briefwechsel mit Autorinnen und Autoren}.
 Digitale Edition, https://schnitzler-briefe.acdh.oeaw.ac.at/{\dateiname}.html (Stand \today)
\fi

\end{document}


      