%% latex-korrekturansicht-vorspann.tex
%% Vorspann für die Korrekturansicht.
%% Lädt die gemeinsame Datei latex-vorspann.tex mit gesetztem Schalter.

\newif\ifkorrekturansicht
\korrekturansichttrue

\input{../tex-inputs/latex-vorspann}


\section[Richard Beer-Hofmann an Arthur Schnitzler, 9. 11. 1904]{L01468 Richard Beer-Hofmann an Arthur Schnitzler, 9. 11. 1904}
\nopagebreak\mylabel{L01468v}
\rehead{ }\normalsize\beginnumbering\briefempfaengerindex{Schnitzler, Arthur@\textsc{Schnitzler, Arthur}!zzzBeer-Hofmann, Richard@\emph{von Richard Beer-Hofmann}!1904-11-091@{9. 11. 1904}|(be}
\toendnotes[C]{\smallbreak\pagebreak[2]}\Standort{CUL, Schnitzler, B 8.}
\physDesc{Brief, 1 Blatt, 1 Seite, 507 Zeichen
\newline{}Handschrift: schwarze Tinte, lateinische Kurrent
\newline{}Ordnung: mit Bleistift von unbekannter Hand nummeriert:
                                    »195« }
\buchAbdrucke{\weitereDrucke{Arthur Schnitzler, Richard Beer-Hofmann: \emph{Briefwechsel 1891–1931}. Wien, Zürich: \emph{Europaverlag} 1992, S. 169.} }\toendnotes[C]{\smallbreak}
\pstart
           \centering{}{\pb}\uline{Noch} – Rodaun\oindex{Rodaun@\textbf{Rodaun}, \emph{A.ADM4}|pw}{ }9./XI. 04\pend
           \vspace{0.5em}
\pstart
           Lieber Arthur! Ich bin selbstverständlich ohne jede Nachricht von
                  Berlin\oindex{Berlin@\textbf{Berlin}, \emph{P.PPLC}|pw}. Werde morgen telegraphiren. Wenn
               erfolglos, werde ich Alles auf Ihre Schultern laden. Jedenfalls:\pend
           
\pstart
           1) Wann fahren Sie – \label{K_L01468-1v}\edtext{Samstag}{\lemma{\textnormal{\emph{Samstag}}}\Cendnote{\textnormal{Vgl. A. S.: \emph{Tagebuch}, 12. 11. 1904.
               }}}\label{K_L01468-1}? \introOben{}(Stunde Bahnhof)\introOben{}\pend
           
\pstart
           2.) Wo wohnen Sie in Berlin\oindex{Berlin@\textbf{Berlin}, \emph{P.PPLC}|pw}?\pend
           
\pstart
           Mein Hausherr\pwindex{Berger, Rudolf *~10.9.1858@\textsc{Berger, Rudolf} (*~10.9.1858), \emph{Vermieter/Vermieterin, Metzger/Metzgerin}|pwv}? »Arisch«
               »Bodenständig« »Deutsche Biederkeit« »Ehrliches Bürgerthum« »Gerader deutscher Sinn«
               »Abhold jeder Tücke« »Germanische Treue«. Sie – die Selcherin\pwindex{Berger @\textsc{Berger}, \emph{Metzger/Metzgerin}|pwv} – hat einen Hausaltar – und die Kinder\pwindex{Berger @\textsc{Berger}|pwv} ko{\geminationm}en
               nach Kalksburg\orgindex{Kollegium Kalksburg@Kollegium Kalksburg|pw}.\pend
           
\pstart
           Herzlichst Ihr{\\[\baselineskip]}\spacefill\mbox{Richard}\pend
           \leftskip=0em{}\selectlanguage{ngerman}\endnumbering\briefempfaengerindex{Schnitzler, Arthur@\textsc{Schnitzler, Arthur}!zzzBeer-Hofmann, Richard@\emph{von Richard Beer-Hofmann}!1904-11-091@{9. 11. 1904}|)be}\mylabel{L01468h}  \normalsize

\doendnotes{C}
\bigskip
\vfill

\clearpage

\footnotesize

\lohead{\textsc{register}}

% Definiere theindex-Environment komplett neu ohne reledmac
\makeatletter
\renewenvironment{theindex}{%
  \section*{\indexname}%
  \setlength{\parindent}{0pt}%
  \setlength{\parskip}{0pt plus 0.3pt}%
  \let\item\@idxitem
}{%
  \clearpage
}
\makeatother

\IfFileExists{\jobname-pw.ind}{\input{\jobname-pw.ind}}{}

\end{document}

      