\input{../tex-inputs/latex-pdf-vorspann}
\begin{center}
            \textcolor{red}{ENTWURF. ENTZIFFERUNG NOCH NICHT KORREKTURGELESEN}
                      \end{center}
            
               \section[Arthur Schnitzler an Richard Beer-Hofmann, 3. 8. 1900]{ Arthur Schnitzler an Richard Beer-Hofmann, 3. 8. 1900}\nopagebreak\mylabel{v}\rehead{ }\begin{ledgroupsized}[t]{13cm}\normalsize\beginnumbering\briefempfaengerindex{Beer-Hofmann, Richard@\textsc{Beer-Hofmann, Richard}!zzzSchnitzler, Arthur@\emph{von Arthur Schnitzler}!1900-08-031@{3. 8. 1900}|(be} \toendnotes[C]{\smallbreak\pagebreak[2]} \Standort{YCGL, MSS 31.}
\physDesc{Brief, 2 Blätter, 7 Seiten, Umschlag
\newline{}Handschrift: Bleistift, deutsche Kurrent\newline{}Versand: 1) Stempel: »\nobreak{}\oindex{Bad Ischl@\textbf{Bad Ischl}|pwk}Ischl, 3. 3. {[}1900{]}, 2–3N\nobreak{}«.  2) Stempel: »\nobreak{}\oindex{Altaussee@\textbf{Altaussee}|pwk}Alt-Aussee, 4/8 \textcolor{gray}{00}\nobreak{}«. 
\newline{}Beer-Hofmann: mit Bleistift am Umschlag eine Notiz in Lateinschrift: »\noindent{}{\pb}\uline{Tuch 20}{ / }\uline{Karten 40}{ / }Rahmen 18{ / }\uline{40}« }\buchAbdrucke{\weitereDrucke{Arthur Schnitzler, Richard Beer-Hofmann: \emph{Briefwechsel 1891–1931}. Hg. Konstanze Fliedl. Wien, Zürich: \emph{Europaverlag} 1992, S. 149–151.} }\toendnotes[C]{\smallbreak}\pstart{}{\pb}Herrn \textsc{Dr. Rich.
                     Beer-Hofmann}\pend{}\pstart{}\textsc{Altaussee\oindex{Altaussee@\textbf{Altaussee}|pw}}\pend{}{\bigskip}\pstart
           \raggedleft{}3. 8. 900.\pend
           \pstart
           {\pb}lieber Richard, ich ka{\geminationn} den Vortheil Ihres neuen
               Vorſchlag\textcolor{gray}{e}s nicht einſehn. Das miſſliche daran iſt: \uline{doch}{ }\textsc{per} Bahn nach Jenbach\oindex{Jenbach@\textbf{Jenbach}|pw}
               fahren müſſsen, dann wieder von Sterzing\oindex{Sterzing@\textbf{Sterzing}|pw} nach Innsbruck\oindex{Innsbruck@\textbf{Innsbruck}|pw} zurück müſſen. Vergeſſen Sie nicht, unſre
               Abſicht iſt: von Zell a/See\oindex{Zell am See@\textbf{Zell am See}|pw} nach Innsbruck\oindex{Innsbruck@\textbf{Innsbruck}|pw}, auf einem neuen Weg, zu kommen. {\pb}Überdies \substVorne{}\textsuperscript{\textcolor{gray}{×}}\substDazwischen{}k\substHinten{}oſtet Ihre Tour 1 Tag mehr, \textcolor{gray}{u}. Kerr\pwindex{Kerr, Alfred 25.12.1867 – 12.10.1948@\textsc{Kerr, Alfred} (25.12.1867 – 12.10.1948), \emph{Schriftsteller, Kritiker}|pw} möchte uns in Innsbruck\oindex{Innsbruck@\textbf{Innsbruck}|pw}
               treffen.\pend
           \pstart
           Nach \uline{meinem} Reiſebuch bietet das Pfitſcher Joch kaum mehr als \textsc{Krimml}\oindex{Krimml@\textbf{Krimml}|pw} und \textsc{Gerlos}\oindex{Gerlos@\textbf{Gerlos}|pw}, und die Sache iſt weit bequemer.\pend
           \pstart
           Ich ſchlage alſo vor:\pend
           \pstart
           Salzburg\oindex{Salzburg@\textbf{Salzburg}|pw} ab Montag (ſpäteſtens
                  Dinſtag) Nachmittag 3.12.\pend
           \leftskip=3em{}\pstart
           \noindent{}{\pb}Ankunft Zell am See{ }5.43.\pend
           \leftskip=0em{}\leftskip=3em{}\pstart
           Poſt Keſſelfall\oindex{Alpenhaus Kesselfall@\textbf{Alpenhaus Kesselfall}|pw}\pend
           \leftskip=0em{}\leftskip=3em{}\pstart
           Übernachten.\pend
           \leftskip=0em{}\pstart
           \noindent{}\uline{Dinſtag.} (\textsc{resp}. Mittwoch)\pend
           \leftskip=3em{}\pstart
           \noindent{}Spazierg Moſerboden\oindex{Mooserboden@\textbf{Mooserboden}|pw}, zurück Keſſelfall\oindex{Alpenhaus Kesselfall@\textbf{Alpenhaus Kesselfall}|pw}, bis Zell
                  am See\oindex{Zell am See@\textbf{Zell am See}|pw}\pend
           \leftskip=0em{}\leftskip=3em{}\pstart
           Bahn (4.50 nach \textsc{Kri{\geminationm}l}\oindex{Krimml@\textbf{Krimml}|pw})\pend
           \leftskip=0em{}\leftskip=3em{}\pstart
           Übernachten.\pend
           \leftskip=0em{}\pstart
           \noindent{}\uline{Mittwoch}{ }\introOben{}(\textsc{resp}{ }Do{\geminationn})\introOben{}{ }\textsc{Kri{\geminationm}l}\oindex{Krimml@\textbf{Krimml}|pw}{ }\textsc{Gerlos}\oindex{Gerlos@\textbf{Gerlos}|pw} (Fußpartie – 4 Stunden)\pend
           \pstart
           \textsc{Gerlos\oindex{Gerlos@\textbf{Gerlos}|pw}} – \textsc{Zell} (Zillerthal)\oindex{Zell am Ziller@\textbf{Zell am Ziller}|pw} 4 Stunden\pend
           \pstart
           \textsc{Zell\oindex{Zell am Ziller@\textbf{Zell am Ziller}|pw} – Jenbach\oindex{Jenbach@\textbf{Jenbach}|pw}} (Wagen\textcolor{gray}{)}\pend
           \leftskip=3em{}\pstart
           \noindent{}abds{ }Innsbruck\oindex{Innsbruck@\textbf{Innsbruck}|pw}, 4 Stunden.\pend
           \leftskip=0em{}\pstart
           \noindent{}{\pb}Das Pfitſcher
                  Joch iſt einfach »lohnend«, hat nicht einmal einen Stern! – und iſt
               viel ſchwerer als \textsc{Gerlos}\oindex{Gerlos@\textbf{Gerlos}|pw}. –\pend
           \pstart
           Was nun die Schweiz\oindex{Schweiz@\textbf{Schweiz}|pw} anbelangt: Übergang direct
               nach \textsc{Klosters}\oindex{Klosters Dorf@\textbf{Klosters Dorf}|pw} dem Überg nach \textsc{Küblis}\oindex{Kueblis@\textbf{Küblis}|pw} vorzuziehn, da wir jedenfalls nach \textsc{Klosters}\oindex{Klosters Dorf@\textbf{Klosters Dorf}|pw}{ }{\pb}und von da nach \textsc{Davos}\oindex{Davos@\textbf{Davos}|pw} müſſen; von da \textsc{\uline{Flüelapass}}\oindex{Flueelapass@\textbf{Flüelapass}|pw} nach \textsc{Samaden}\oindex{Samedan@\textbf{Samedan}|pw} u \textsc{Pontresina}\oindex{Pontresina@\textbf{Pontresina}|pw}. (Fahrſtraſſe)\pend
           \pstart
           – Im übrigen werden wir keinen Richter brauchen, dagegen Träger. –\pend
           \pstart
           Georg H.\pwindex{Hirschfeld, Georg 11.02.1873 – 17.01.1942@\textsc{Hirschfeld, Georg} (11.02.1873 – 17.01.1942), \emph{Schriftsteller}|pw} wird faſt ſicher \uline{nicht} mitko{\geminationm}en, obwohl ich ihn auf den
               Knieen beſchworen habe. Menſch{\pb}licher Vorausſicht nach
               (faſſen Sie dieſes »Menſch-« nicht falſch auf) werd’ ich Sonntag \introOben{}den\introOben{}{ }12. in Salzburg\oindex{Salzburg@\textbf{Salzburg}|pw}{ }ſein. Ich bin ſehr dafür, ſchon Montag
               abzufahren.\pend
           \pstart
           Von Schwarzk.\pwindex{Schwarzkopf, Gustav 07.11.1853 – 13.11.1939@\textsc{Schwarzkopf, Gustav} (07.11.1853 – 13.11.1939), \emph{Schriftsteller}|pw} u Salten\pwindex{Salten, Felix 06.09.1869 – 08.10.1945@\textsc{Salten, Felix} (06.09.1869 – 08.10.1945), \emph{Schriftsteller, Journalist}|pw} noch keine Nachricht. Auch von Paul
                  G.\pwindex{Goldmann, Paul 31.01.1865 – 25.09.1935@\textsc{Goldmann, Paul} (31.01.1865 – 25.09.1935), \emph{Schriftsteller, Journalist}|pw} nichts neues. –\pend
           \pstart
           {\pb}Leben Sie wohl. –\pend
           \pstart
           Herzlichſt Ihr{\\[\baselineskip]}\spacefill\mbox{Arthur}\pend
           \leftskip=0em{}\pstart
           Hugo\pwindex{Hofmannsthal, Hugo von 01.02.1874 – 15.07.1929@\textsc{Hofmannsthal, Hugo von} (01.02.1874 – 15.07.1929), \emph{Schriftsteller}|pw} hat mir \label{K_L01063_1v}\edtext{geſchrieben}{\lemma{\textnormal{\emph{geſchrieben}}}\Cendnote{\textnormal{Hugo von Hofmannsthal an Arthur Schnitzler, 27. 7. 1900}}}\label{K_L01063_1h} iſt wohl ſchon in Salzburg\oindex{Salzburg@\textbf{Salzburg}|pw} bleibt bis
                  15. Er ſchrieb mir auch von ſeiner Verlobung.\pend
           \endnumbering\briefempfaengerindex{Beer-Hofmann, Richard@\textsc{Beer-Hofmann, Richard}!zzzSchnitzler, Arthur@\emph{von Arthur Schnitzler}!1900-08-031@{3. 8. 1900}|)be}\mylabel{h}\end{ledgroupsized}  \newcommand{\dateiname}{L01063}\newcommand{\titel}{Arthur Schnitzler an Richard Beer-Hofmann, 3. 8. 1900}\newcommand{\editorInnen}{Martin Anton Müller und Gerd-Hermann Susen}\input{../tex-inputs/latex-pdf-abspann}
      