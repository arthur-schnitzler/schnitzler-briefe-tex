%% latex-leseansicht-vorspann.tex
%% Vorspann für die Leseansicht.
%% Lädt die gemeinsame Datei latex-vorspann.tex mit nicht gesetztem Schalter.

\newif\ifkorrekturansicht
\korrekturansichtfalse

\input{../tex-inputs/latex-vorspann}


\section[Arthur Schnitzler an Richard Beer-Hofmann, 3. 8. 1900]{L01063 Arthur Schnitzler an Richard Beer-Hofmann, 3. 8. 1900}
\nopagebreak\mylabel{L01063v}
\rehead{ }\normalsize\beginnumbering\briefempfaengerindex{Beer-Hofmann, Richard@\textsc{Beer-Hofmann, Richard}!zzzSchnitzler, Arthur@\emph{von Arthur Schnitzler}!1900-08-031@{3. 8. 1900}|(be}
\toendnotes[C]{\smallbreak\pagebreak[2]}
\correspDesc{Versand  durch Arthur Schnitzler am 3. 8. 1900 in Bad Ischl
\newline{}Erhalt  durch Richard Beer-Hofmann am 4. 8. 1900 in Altaussee}\toendnotes[C]{\smallbreak}
\Standort{YCGL, MSS 31.}
\physDesc{Brief, 2 Blätter, 7 Seiten, Kuvert, 1702 Zeichen
\newline{}Handschrift: Bleistift, deutsche Kurrent
\newline{}Versand: 1) Stempel: »\nobreak{}\oindex{Bad Ischl@\textbf{Bad Ischl}|pwk}Ischl, 3. 3. {[}1900{]}, 2–3N\nobreak{}«.   2) Stempel: »\nobreak{}\oindex{Altaussee@\textbf{Altaussee}, \emph{Verwaltungsgebiet}|pwk}Alt-Aussee, 4/8 \textcolor{gray}{00}\nobreak{}«. 
\newline{}Beer-Hofmann: mit Bleistift am Umschlag eine Notiz in Lateinschrift: »\noindent{}{\pb}\uline{Tuch 20}{ / }\uline{Karten 40}{ / }Rahmen 18{ / }\uline{40}« }
\buchAbdrucke{\weitereDrucke{Arthur Schnitzler, Richard Beer-Hofmann: \emph{Briefwechsel 1891–1931}. Herausgegeben von Konstanze Fliedl. Wien, Zürich: \emph{Europaverlag} 1992, S. 149–151.} }\toendnotes[C]{\smallbreak}\pstart{}{\pb}Herrn \textsc{Dr. Rich.
                     Beer-Hofmann}\pend{}\pstart{}\textsc{Altaussee\oindex{Altaussee@\textbf{Altaussee}, \emph{Verwaltungsgebiet}|pw}}\pend{}{\bigskip}\vspace{1em}
\pstart
           \raggedleft{}3. 8. 900.\pend
           \vspace{0.5em}
\pstart
           {\pb}lieber Richard, ich ka{\geminationn} den Vortheil Ihres neuen
               Vorſchlag\textcolor{gray}{e}s nicht einſehn. Das miſſliche daran iſt: \uline{doch}{ }\textsc{per} Bahn nach Jenbach\oindex{Jenbach@\textbf{Jenbach}, \emph{Verwaltungsgebiet}|pw}
               fahren müſſsen, dann wieder von Sterzing\oindex{Sterzing@\textbf{Sterzing}, \emph{Hauptstadt}|pw} nach
                  Innsbruck\oindex{Innsbruck@\textbf{Innsbruck}, \emph{Verwaltungsgebiet}|pw} zurück müſſen. Vergeſſen Sie nicht,
               unſre Abſicht iſt: von Zell a/See\oindex{Zell am See@\textbf{Zell am See}, \emph{Hauptstadt}|pw} nach Innsbruck\oindex{Innsbruck@\textbf{Innsbruck}, \emph{Verwaltungsgebiet}|pw}, auf einem neuen Weg, zu kommen. {\pb}Überdies \substVorne{}\textsuperscript{\textcolor{gray}{×}}\substDazwischen{}k\substHinten{}oſtet Ihre Tour 1 Tag mehr, \textcolor{gray}{u}. Kerr\pwindex{Kerr, Alfred 25.\,12.\,1867 Breslau – 12.\,10.\,1948 Hamburg@\textsc{Kerr, Alfred} (25.\,12.\,1867 Breslau – 12.\,10.\,1948 Hamburg), \emph{Schriftsteller, Kritiker}|pw} möchte uns in Innsbruck\oindex{Innsbruck@\textbf{Innsbruck}, \emph{Verwaltungsgebiet}|pw} treffen.\pend
           
\pstart
           Nach \uline{meinem} Reiſebuch bietet das Pfitſcher Joch\oindex{Pfitscher Joch@\textbf{Pfitscher Joch}, \emph{Berg}|pw} kaum mehr als \textsc{Krimml}\oindex{Krimml@\textbf{Krimml}, \emph{Verwaltungsgebiet}|pw} und \textsc{Gerlos}\oindex{Gerlos@\textbf{Gerlos}, \emph{Verwaltungsgebiet}|pw}, und die Sache iſt weit bequemer.\pend
           
\pstart
           Ich{ }ſchlage alſo vor:\pend
           
\pstart
           Salzburg\oindex{Salzburg@\textbf{Salzburg}, \emph{Verwaltungsgebiet}|pw} ab Montag (ſpäteſtens
                  Dinſtag) Nachmittag 3.12.\pend
           \leftskip=3em{}
\pstart
           \noindent{}{\pb}Ankunft Zell am See{ }5.43.\pend
           \leftskip=0em{}\leftskip=3em{}
\pstart
           Poſt Keſſelfall\oindex{Alpenhaus Kesselfall@\textbf{Alpenhaus Kesselfall}, \emph{Beherbergungsgebäude}|pw}\pend
           \leftskip=0em{}\leftskip=3em{}
\pstart
           Übernachten.\pend
           \leftskip=0em{}
\pstart
           \uline{Dinſtag.} (\textsc{resp}. Mittwoch)\pend
           \leftskip=3em{}
\pstart
           \noindent{}Spazierg Moſerboden\oindex{Mooserboden@\textbf{Mooserboden}, \emph{Berg}|pw}, zurück Keſſelfall\oindex{Alpenhaus Kesselfall@\textbf{Alpenhaus Kesselfall}, \emph{Beherbergungsgebäude}|pw}, bis Zell am See\oindex{Zell am See@\textbf{Zell am See}, \emph{Hauptstadt}|pw}\pend
           \leftskip=0em{}\leftskip=3em{}
\pstart
           Bahn (4.50 nach \textsc{Kri{\geminationm}l}\oindex{Krimml@\textbf{Krimml}, \emph{Verwaltungsgebiet}|pw})\pend
           \leftskip=0em{}\leftskip=3em{}
\pstart
           Übernachten.\pend
           \leftskip=0em{}
\pstart
           \uline{Mittwoch}{ }\introOben{}(\textsc{resp}{ }Do{\geminationn})\introOben{}{ }\textsc{Kri{\geminationm}l}\oindex{Krimml@\textbf{Krimml}, \emph{Verwaltungsgebiet}|pw}{ }\textsc{Gerlos}\oindex{Gerlos@\textbf{Gerlos}, \emph{Verwaltungsgebiet}|pw} (Fußpartie – 4 Stunden)\pend
           
\pstart
           \textsc{Gerlos\oindex{Gerlos@\textbf{Gerlos}, \emph{Verwaltungsgebiet}|pw}} – \textsc{Zell} (Zillerthal)\oindex{Zell am Ziller@\textbf{Zell am Ziller}, \emph{Verwaltungsgebiet}|pw} 4 Stunden\pend
           
\pstart
           \textsc{Zell\oindex{Zell am Ziller@\textbf{Zell am Ziller}, \emph{Verwaltungsgebiet}|pw} – Jenbach\oindex{Jenbach@\textbf{Jenbach}, \emph{Verwaltungsgebiet}|pw}} (Wagen\textcolor{gray}{)}\pend
           \leftskip=3em{}
\pstart
           \noindent{}abds{ }Innsbruck\oindex{Innsbruck@\textbf{Innsbruck}, \emph{Verwaltungsgebiet}|pw}, 4 Stunden.\pend
           \leftskip=0em{}
\pstart
           {\pb}Das Pfitſcher
                  Joch\oindex{Pfitscher Joch@\textbf{Pfitscher Joch}, \emph{Berg}|pw} iſt einfach »lohnend«, hat nicht einmal einen Stern! – und iſt
               viel{ }ſchwerer als \textsc{Gerlos}\oindex{Gerlos@\textbf{Gerlos}, \emph{Verwaltungsgebiet}|pw}. –\pend
           
\pstart
           Was nun die Schweiz\oindex{Schweiz@\textbf{Schweiz}|pw} anbelangt: Übergang direct
               nach \textsc{Klosters}\oindex{Klosters Dorf@\textbf{Klosters Dorf}|pw} dem Überg nach \textsc{Küblis}\oindex{Küblis@\textbf{Küblis}|pw} vorzuziehn, da wir jedenfalls nach \textsc{Klosters}\oindex{Klosters Dorf@\textbf{Klosters Dorf}|pw}{ }{\pb}und von da nach \textsc{Davos}\oindex{Davos@\textbf{Davos}|pw} müſſen; von da \textsc{\uline{Flüelapass}}\oindex{Flüelapass@\textbf{Flüelapass}, \emph{Pass}|pw} nach \textsc{Samaden}\oindex{Samedan@\textbf{Samedan}, \emph{Verwaltungsgebiet}|pw} u \textsc{Pontresina}\oindex{Pontresina@\textbf{Pontresina}|pw}. (Fahrſtraſſe)\pend
           
\pstart
           – Im übrigen werden wir keinen Richter brauchen, dagegen Träger. –\pend
           
\pstart
           \label{K_L01063-1v}\edtext{Georg H.\pwindex{Hirschfeld, Georg 11.\,2.\,1873 Berlin – 17.\,1.\,1942 München@\textsc{Hirschfeld, Georg} (11.\,2.\,1873 Berlin – 17.\,1.\,1942 München), \emph{Schriftsteller}|pw} wird faſt{ }ſicher \uline{nicht} mitko{\geminationm}en}{\lemma{\textnormal{\emph{Georg … mitkommen}}}\Cendnote{\textnormal{Vgl. XXXX Auszeichnungsfehler: Dokument L02924 nicht gefunden.
               }}}\label{K_L01063-1}, obwohl ich ihn auf den Knieen beſchworen habe. Menſch{\pb}licher Vorausſicht nach (faſſen Sie dieſes »Menſch-«
               nicht falſch auf) werd’ ich Sonntag \introOben{}den\introOben{}{ }12. in Salzburg\oindex{Salzburg@\textbf{Salzburg}, \emph{Verwaltungsgebiet}|pw}{ }ſein. Ich bin{ }ſehr dafür,{ }ſchon Montag
               abzufahren.\pend
           
\pstart
           Von Schwarzk.\pwindex{Schwarzkopf, Gustav 7.\,11.\,1853 Wien – 13.\,11.\,1939 ebd.@\textsc{Schwarzkopf, Gustav} (7.\,11.\,1853 Wien – 13.\,11.\,1939 ebd.), \emph{Schriftsteller}|pw} u Salten\pwindex{Salten, Felix 6.\,9.\,1869 Budapest – 8.\,10.\,1945 Zürich@\textsc{Salten, Felix} (6.\,9.\,1869 Budapest – 8.\,10.\,1945 Zürich), \emph{Schriftsteller, Journalist, Chefredakteur}|pw} noch keine Nachricht. Auch von Paul G.\pwindex{Goldmann, Paul 31.\,1.\,1865 Breslau – 25.\,9.\,1935 Wien@\textsc{Goldmann, Paul} (31.\,1.\,1865 Breslau – 25.\,9.\,1935 Wien), \emph{Schriftsteller, Journalist}|pw} nichts neues. –\pend
           
\pstart
           {\pb}Leben Sie wohl. –\pend
           
\pstart
           Herzlichſt Ihr{\\[\baselineskip]}\spacefill\mbox{Arthur}\pend
           \leftskip=0em{}
\pstart
           Hugo\pwindex{Hofmannsthal, Hugo von 1.\,2.\,1874 Wien – 15.\,7.\,1929 Rodaun@\textsc{Hofmannsthal, Hugo von} (1.\,2.\,1874 Wien – 15.\,7.\,1929 Rodaun), \emph{Schriftsteller}|pw} hat mir \label{K_L01063-2v}\edtext{geſchrieben}{\lemma{\textnormal{\emph{geschrieben}}}\Cendnote{\textnormal{XXXX Auszeichnungsfehler: Dokument L01061 nicht gefunden.
               }}}\label{K_L01063-2} iſt wohl{ }ſchon in Salzburg\oindex{Salzburg@\textbf{Salzburg}, \emph{Verwaltungsgebiet}|pw} bleibt bis
                  15. Er{ }ſchrieb mir auch von{ }ſeiner Verlobung.\pend
           \selectlanguage{ngerman}\endnumbering\briefempfaengerindex{Beer-Hofmann, Richard@\textsc{Beer-Hofmann, Richard}!zzzSchnitzler, Arthur@\emph{von Arthur Schnitzler}!1900-08-031@{3. 8. 1900}|)be}\mylabel{L01063h}  \newcommand{\dateiname}{L01063}\newcommand{\titel}{Arthur Schnitzler an Richard Beer-Hofmann, 3. 8. 1900}\newcommand{\editorInnen}{Martin Anton Müller und Gerd-Hermann Susen}%% latex-leseansicht-abspann.tex
%% Abspann für die Leseansicht.
%% Der Schalter \ifkorrekturansicht ist bereits durch den Vorspann gesetzt.

%% latex-abspann.tex
%% Gemeinsamer Abspann für Korrekturansicht und Leseansicht.
%% Setzt den Schalter \ifkorrekturansicht voraus (gesetzt in den
%% einbindenden Dateien latex-korrekturansicht-abspann.tex bzw.
%% latex-leseansicht-abspann.tex).
%% ---------------------------------------------------------------

\normalsize

% Das esempio-Environment wird nur in der Leseansicht benötigt
\ifkorrekturansicht\else
\newenvironment{esempio}[3]%
{
    \vspace{1.5ex}
    \rlap{\underline{#1}}
    \par
    \setlength{\parindent}{0cm}
    \nopagebreak
    \leftskip=#2cm
    \rightskip=#3cm
}
{
    \par
}
\fi

\doendnotes{C}
\bigskip
\vfill

\clearpage

\footnotesize

\ifkorrekturansicht
  \lohead{\textsc{register}}
\fi

% theindex-Environment neu definieren ohne reledmac
\makeatletter
\renewenvironment{theindex}{%
  \ifkorrekturansicht
    \section*{\indexname}%
  \else
    \subsubsection*{Index der erwähnten Entitäten}%
  \fi
  \setlength{\parindent}{0pt}%
  \setlength{\parskip}{0pt plus 0.3pt}%
  \let\item\@idxitem
}{%
  \ifkorrekturansicht\clearpage\fi
}
\makeatother

\IfFileExists{\jobname-pw.ind}{\input{\jobname-pw.ind}}{}

% Quellenangabe nur in der Leseansicht
\ifkorrekturansicht\else
% Fallback-Definitionen, falls die .tex-Datei \titel etc. nicht gesetzt hat
\providecommand{\titel}{}
\providecommand{\editorInnen}{}
\providecommand{\dateiname}{\jobname}

\vspace{3cm}

\vfill

\footnotesize
\textsc{Quelle}: \titel. Herausgegeben von {\editorInnen}. In: \emph{Arthur Schnitzler: Briefwechsel mit Autorinnen und Autoren}.
 Digitale Edition, https://schnitzler-briefe.acdh.oeaw.ac.at/{\dateiname}.html (Stand \today)
\fi

\end{document}


