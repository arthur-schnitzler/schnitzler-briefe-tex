%% latex-korrekturansicht-vorspann.tex
%% Vorspann für die Korrekturansicht.
%% Lädt die gemeinsame Datei latex-vorspann.tex mit gesetztem Schalter.

\newif\ifkorrekturansicht
\korrekturansichttrue

\input{../tex-inputs/latex-vorspann}


\section[Arthur Schnitzler an Richard Beer-Hofmann, 3. 8. 1900]{L01063 Arthur Schnitzler an Richard Beer-Hofmann, 3. 8. 1900}
\nopagebreak\mylabel{L01063v}
\rehead{ }\normalsize\beginnumbering\briefempfaengerindex{Beer-Hofmann, Richard@\textsc{Beer-Hofmann, Richard}!zzzSchnitzler, Arthur@\emph{von Arthur Schnitzler}!1900-08-031@{3. 8. 1900}|(be}
\toendnotes[C]{\smallbreak\pagebreak[2]}\Standort{YCGL, MSS 31.}
\physDesc{Brief, 2 Blätter, 7 Seiten, Umschlag, 1702 Zeichen
\newline{}Handschrift: Bleistift, deutsche Kurrent
\newline{}Versand: 1) Stempel: »\nobreak{}\oindex{Bad Ischl@\textbf{Bad Ischl}, \emph{P.PPL}|pwk}Ischl, 3. 3. {[}1900{]}, 2–3N\nobreak{}«.   2) Stempel: »\nobreak{}\oindex{Altaussee@\textbf{Altaussee}, \emph{A.ADM3}|pwk}Alt-Aussee, 4/8 \textcolor{gray}{00}\nobreak{}«. 
\newline{}Beer-Hofmann: mit Bleistift am Umschlag eine Notiz in Lateinschrift: »\noindent{}{\pb}\uline{Tuch 20}{ / }\uline{Karten 40}{ / }Rahmen 18{ / }\uline{40}« }
\buchAbdrucke{\weitereDrucke{Arthur Schnitzler, Richard Beer-Hofmann: \emph{Briefwechsel 1891–1931}. Wien, Zürich: \emph{Europaverlag} 1992, S. 149–151.} }\toendnotes[C]{\smallbreak}\pstart{}{\pb}Herrn \textsc{Dr. Rich.
                     Beer-Hofmann}\pend{}\pstart{}\textsc{Altaussee\oindex{Altaussee@\textbf{Altaussee}, \emph{A.ADM3}|pw}}\pend{}{\bigskip}\vspace{1em}
\pstart
           \raggedleft{}3. 8. 900.\pend
           \vspace{0.5em}
\pstart
           {\pb}lieber Richard, ich ka{\geminationn} den Vortheil Ihres neuen
               Vorſchlag\textcolor{gray}{e}s nicht einſehn. Das miſſliche daran iſt: \uline{doch}{ }\textsc{per} Bahn nach Jenbach\oindex{Jenbach@\textbf{Jenbach}, \emph{A.ADM3}|pw}
               fahren müſſsen, dann wieder von Sterzing\oindex{Sterzing@\textbf{Sterzing}, \emph{P.PPLA3}|pw} nach
                  Innsbruck\oindex{Innsbruck@\textbf{Innsbruck}, \emph{A.ADM2}|pw} zurück müſſen. Vergeſſen Sie nicht,
               unſre Abſicht iſt: von Zell a/See\oindex{Zell am See@\textbf{Zell am See}, \emph{P.PPLA3}|pw} nach Innsbruck\oindex{Innsbruck@\textbf{Innsbruck}, \emph{A.ADM2}|pw}, auf einem neuen Weg, zu kommen. {\pb}Überdies \substVorne{}\textsuperscript{\textcolor{gray}{×}}\substDazwischen{}k\substHinten{}oſtet Ihre Tour 1 Tag mehr, \textcolor{gray}{u}. Kerr\pwindex{Kerr, Alfred 25.12.1867 – 12.10.1948@\textsc{Kerr, Alfred} (25.12.1867 – 12.10.1948), \emph{Schriftsteller/Schriftstellerin, Kritiker/Kritikerin}|pw} möchte uns in Innsbruck\oindex{Innsbruck@\textbf{Innsbruck}, \emph{A.ADM2}|pw} treffen.\pend
           
\pstart
           Nach \uline{meinem} Reiſebuch bietet das Pfitſcher Joch\oindex{Pfitscher Joch@\textbf{Pfitscher Joch}, \emph{Berg (N.BRG)}|pw} kaum mehr als \textsc{Krimml}\oindex{Krimml@\textbf{Krimml}, \emph{A.ADM3}|pw} und \textsc{Gerlos}\oindex{Gerlos@\textbf{Gerlos}, \emph{A.ADM3}|pw}, und die Sache iſt weit bequemer.\pend
           
\pstart
           Ich ſchlage alſo vor:\pend
           
\pstart
           Salzburg\oindex{Salzburg@\textbf{Salzburg}, \emph{A.ADM2}|pw} ab Montag (ſpäteſtens
                  Dinſtag) Nachmittag 3.12.\pend
           \leftskip=3em{}
\pstart
           \noindent{}{\pb}Ankunft Zell am See{ }5.43.\pend
           \leftskip=0em{}\leftskip=3em{}
\pstart
           Poſt Keſſelfall\oindex{Alpenhaus Kesselfall@\textbf{Alpenhaus Kesselfall}, \emph{Beherbergungsgebäude (K.BHB)}|pw}\pend
           \leftskip=0em{}\leftskip=3em{}
\pstart
           Übernachten.\pend
           \leftskip=0em{}
\pstart
           \uline{Dinſtag.} (\textsc{resp}. Mittwoch)\pend
           \leftskip=3em{}
\pstart
           \noindent{}Spazierg Moſerboden\oindex{Mooserboden@\textbf{Mooserboden}, \emph{Berg (N.BRG)}|pw}, zurück Keſſelfall\oindex{Alpenhaus Kesselfall@\textbf{Alpenhaus Kesselfall}, \emph{Beherbergungsgebäude (K.BHB)}|pw}, bis Zell am See\oindex{Zell am See@\textbf{Zell am See}, \emph{P.PPLA3}|pw}\pend
           \leftskip=0em{}\leftskip=3em{}
\pstart
           Bahn (4.50 nach \textsc{Kri{\geminationm}l}\oindex{Krimml@\textbf{Krimml}, \emph{A.ADM3}|pw})\pend
           \leftskip=0em{}\leftskip=3em{}
\pstart
           Übernachten.\pend
           \leftskip=0em{}
\pstart
           \uline{Mittwoch}{ }\introOben{}(\textsc{resp}{ }Do{\geminationn})\introOben{}{ }\textsc{Kri{\geminationm}l}\oindex{Krimml@\textbf{Krimml}, \emph{A.ADM3}|pw}{ }\textsc{Gerlos}\oindex{Gerlos@\textbf{Gerlos}, \emph{A.ADM3}|pw} (Fußpartie – 4 Stunden)\pend
           
\pstart
           \textsc{Gerlos\oindex{Gerlos@\textbf{Gerlos}, \emph{A.ADM3}|pw}} – \textsc{Zell} (Zillerthal)\oindex{Zell am Ziller@\textbf{Zell am Ziller}, \emph{A.ADM3}|pw} 4 Stunden\pend
           
\pstart
           \textsc{Zell\oindex{Zell am Ziller@\textbf{Zell am Ziller}, \emph{A.ADM3}|pw} – Jenbach\oindex{Jenbach@\textbf{Jenbach}, \emph{A.ADM3}|pw}} (Wagen\textcolor{gray}{)}\pend
           \leftskip=3em{}
\pstart
           \noindent{}abds{ }Innsbruck\oindex{Innsbruck@\textbf{Innsbruck}, \emph{A.ADM2}|pw}, 4 Stunden.\pend
           \leftskip=0em{}
\pstart
           {\pb}Das Pfitſcher
                  Joch\oindex{Pfitscher Joch@\textbf{Pfitscher Joch}, \emph{Berg (N.BRG)}|pw} iſt einfach »lohnend«, hat nicht einmal einen Stern! – und iſt
               viel ſchwerer als \textsc{Gerlos}\oindex{Gerlos@\textbf{Gerlos}, \emph{A.ADM3}|pw}. –\pend
           
\pstart
           Was nun die Schweiz\oindex{Schweiz@\textbf{Schweiz}, \emph{A.PCLI}|pw} anbelangt: Übergang direct
               nach \textsc{Klosters}\oindex{Klosters Dorf@\textbf{Klosters Dorf}, \emph{P.PPL}|pw} dem Überg nach \textsc{Küblis}\oindex{Kueblis@\textbf{Küblis}, \emph{P.PPL}|pw} vorzuziehn, da wir jedenfalls nach \textsc{Klosters}\oindex{Klosters Dorf@\textbf{Klosters Dorf}, \emph{P.PPL}|pw}{ }{\pb}und von da nach \textsc{Davos}\oindex{Davos@\textbf{Davos}, \emph{P.PPL}|pw} müſſen; von da \textsc{\uline{Flüelapass}}\oindex{Flueelapass@\textbf{Flüelapass}, \emph{Pass (N.PAS)}|pw} nach \textsc{Samaden}\oindex{Samedan@\textbf{Samedan}, \emph{A.ADM3}|pw} u \textsc{Pontresina}\oindex{Pontresina@\textbf{Pontresina}, \emph{P.PPL}|pw}. (Fahrſtraſſe)\pend
           
\pstart
           – Im übrigen werden wir keinen Richter brauchen, dagegen Träger. –\pend
           
\pstart
           \label{K_L01063-1v}\edtext{Georg H.\pwindex{Hirschfeld, Georg 11.02.1873 – 17.01.1942@\textsc{Hirschfeld, Georg} (11.02.1873 – 17.01.1942), \emph{Schriftsteller/Schriftstellerin}|pw} wird faſt ſicher \uline{nicht} mitko{\geminationm}en}{\lemma{\textnormal{\emph{Georg … mitkommen}}}\Cendnote{\textnormal{Vgl. Paul Goldmann an Arthur Schnitzler, 18. 7. [1900].
               }}}\label{K_L01063-1}, obwohl ich ihn auf den Knieen beſchworen habe. Menſch{\pb}licher Vorausſicht nach (faſſen Sie dieſes »Menſch-«
               nicht falſch auf) werd’ ich Sonntag \introOben{}den\introOben{}{ }12. in Salzburg\oindex{Salzburg@\textbf{Salzburg}, \emph{A.ADM2}|pw}{ }ſein. Ich bin ſehr dafür, ſchon Montag
               abzufahren.\pend
           
\pstart
           Von Schwarzk.\pwindex{Schwarzkopf, Gustav 07.11.1853 – 13.11.1939@\textsc{Schwarzkopf, Gustav} (07.11.1853 – 13.11.1939), \emph{Schriftsteller/Schriftstellerin}|pw} u Salten\pwindex{Salten, Felix 06.09.1869 – 08.10.1945@\textsc{Salten, Felix} (06.09.1869 – 08.10.1945), \emph{Schriftsteller/Schriftstellerin, Journalist/Journalistin, Chefredakteur/Chefredakteurin}|pw} noch keine Nachricht. Auch von Paul G.\pwindex{Goldmann, Paul 31.01.1865 – 25.09.1935@\textsc{Goldmann, Paul} (31.01.1865 – 25.09.1935), \emph{Schriftsteller/Schriftstellerin, Journalist/Journalistin}|pw} nichts neues. –\pend
           
\pstart
           {\pb}Leben Sie wohl. –\pend
           
\pstart
           Herzlichſt Ihr{\\[\baselineskip]}\spacefill\mbox{Arthur}\pend
           \leftskip=0em{}
\pstart
           Hugo\pwindex{Hofmannsthal, Hugo von 1874-02-01 – 1929-07-15@\textsc{Hofmannsthal, Hugo von} (1874-02-01 – 1929-07-15), \emph{Schriftsteller/Schriftstellerin}|pw} hat mir \label{K_L01063-2v}\edtext{geſchrieben}{\lemma{\textnormal{\emph{geſchrieben}}}\Cendnote{\textnormal{Hugo von Hofmannsthal an Arthur Schnitzler, 27. 7. 1900.
               }}}\label{K_L01063-2} iſt wohl ſchon in Salzburg\oindex{Salzburg@\textbf{Salzburg}, \emph{A.ADM2}|pw} bleibt bis
                  15. Er ſchrieb mir auch von ſeiner Verlobung.\pend
           \selectlanguage{ngerman}\endnumbering\briefempfaengerindex{Beer-Hofmann, Richard@\textsc{Beer-Hofmann, Richard}!zzzSchnitzler, Arthur@\emph{von Arthur Schnitzler}!1900-08-031@{3. 8. 1900}|)be}\mylabel{L01063h}  \normalsize

\doendnotes{C}
\bigskip
\vfill

\clearpage

\footnotesize

\lohead{\textsc{register}}

% Definiere theindex-Environment komplett neu ohne reledmac
\makeatletter
\renewenvironment{theindex}{%
  \section*{\indexname}%
  \setlength{\parindent}{0pt}%
  \setlength{\parskip}{0pt plus 0.3pt}%
  \let\item\@idxitem
}{%
  \clearpage
}
\makeatother

\IfFileExists{\jobname-pw.ind}{\input{\jobname-pw.ind}}{}

\end{document}

      