%% latex-leseansicht-vorspann.tex
%% Vorspann für die Leseansicht.
%% Lädt die gemeinsame Datei latex-vorspann.tex mit nicht gesetztem Schalter.

\newif\ifkorrekturansicht
\korrekturansichtfalse

\input{../tex-inputs/latex-vorspann}


\section[ Felix Salten an Arthur Schnitzler, 22. 7. 1897]{L03270 Felix Salten an Arthur Schnitzler,  22. 7. 1897}
\nopagebreak\mylabel{L03270v}
\rehead{ }\normalsize\beginnumbering\briefempfaengerindex{Schnitzler, Arthur@\textsc{Schnitzler, Arthur}!zzzSalten, Felix@\emph{von Felix Salten}!1897-07-222@{22. 7. 1897}|(be}
\toendnotes[C]{\smallbreak\pagebreak[2]}
\correspDesc{Versand  durch Felix Salten am 22. 7. 1897 in Wien
\newline{}Übermittlung  am 23. 7. 1897 in Wien
\newline{}Erhalt  durch Arthur Schnitzler am 24. 7. 1897 in Ischl}\toendnotes[C]{\smallbreak}
\Standort{CUL, Schnitzler, B 89, A 2.}
\physDesc{Postkarte, 501 Zeichen
\newline{}Handschrift: Bleistift, lateinische Kurrent
\newline{}Versand: Stempel: »\nobreak{}\oindex{VIII., Josefstadt@\textbf{VIII., Josefstadt}, \emph{Verwaltungsgebiet}|pwk}Wien 8/1 64, 23. 7. 97, 3–4 N\nobreak{}«. Stempel: »\nobreak{}\oindex{Bad Ischl@\textbf{Bad Ischl}|pwk}{[}Ischl{]}, 6–\textcolor{gray}{7 V}\nobreak{}«.  
\newline{}Ordnung: mit Bleistift von unbekannter Hand nummeriert: »93« }
\buchAbdrucke{\weitereDrucke{Hermann Bahr, Arthur Schnitzler: \emph{Briefwechsel, Aufzeichnungen, Dokumente (1891–1931)}. Herausgegeben von Kurt Ifkovits und Martin Anton Müller. Göttingen: \emph{Wallstein} 2018, S. 150.} }\toendnotes[C]{\smallbreak}\pstart{}{\pb}Herrn D\textsuperscript{r} Arthur Schnitzler\pend{}\pstart{}Ischl\oindex{Bad Ischl@\textbf{Bad Ischl}|pw}\pend{}\pstart{}Kaltenbach, Pension Petter\oindex{Hotel und Pension Rudolfshöhe (Leopold Petter)@\textbf{Hotel und Pension Rudolfshöhe (Leopold Petter)}, \emph{Hotel}|pw}\pend{}{\bigskip}\vspace{1em}
\pstart
           \noindent{}{\pb}Lieber Freund, ich lese soeben im 6-Uhr-Blatt\pwindex{Wiener Allgemeine Zeitung@\emph{Wiener Allgemeine Zeitung}|pw} die \label{K_L03270-1v}\edtext{Notiz\pwindex{Theater, Kunst und Literatur [Agnes Jordan nicht am Burgtheater]@\emph{Theater, Kunst und Literatur [Agnes Jordan nicht am Burgtheater]}|pwv}}{\lemma{\textnormal{\emph{Notiz}}}\Cendnote{\textnormal{ »– Wie wir aus verläßlicher
                     Quelle erfahren, ist die Direction des \so{Hofburgtheaters}\orgindex{Burgtheater@Burgtheater|pw} von der Absicht, Georg \so{Hirschfeld}\pwindex{Hirschfeld, Georg 11.\,2.\,1873 Berlin – 17.\,1.\,1942 München@\textsc{Hirschfeld, Georg} (11.\,2.\,1873 Berlin – 17.\,1.\,1942 München), \emph{Schriftsteller}|pw}’s neues Drama ›\so{Agnes Jordan}\pwindex{Hirschfeld, Georg 11.\,2.\,1873 Berlin – 17.\,1.\,1942 München@\textsc{Hirschfeld, Georg} (11.\,2.\,1873 Berlin – 17.\,1.\,1942 München), \emph{Schriftsteller}!Agnes Jordan. Schauspiel in fünf Akten@\strich\emph{Agnes Jordan. Schauspiel in fünf Akten}|pw}‹ nächste Saison zur Aufführung zu bringen, abgekommen.«
                     ([O. V.]: \emph{Theater, Kunst und
                        Literatur}\pwindex{Theater, Kunst und Literatur [Agnes Jordan nicht am Burgtheater]@\emph{Theater, Kunst und Literatur [Agnes Jordan nicht am Burgtheater]}|pwk}. In: \emph{Wiener Allgemeine
                        Zeitung}\pwindex{Wiener Allgemeine Zeitung@\emph{Wiener Allgemeine Zeitung}|pwk}, Nr. 5818, 23. 7. 1897,
                     S. 3.)}}}\label{K_L03270-1} von Agnes Jordan\pwindex{Hirschfeld, Georg 11.\,2.\,1873 Berlin – 17.\,1.\,1942 München@\textsc{Hirschfeld, Georg} (11.\,2.\,1873 Berlin – 17.\,1.\,1942 München), \emph{Schriftsteller}!Agnes Jordan. Schauspiel in fünf Akten@\strich\emph{Agnes Jordan. Schauspiel in fünf Akten}|pw}. Ich
               brauche Ihnen wol nicht erst zu sagen, dass ich derselben vollständig ferne stehe.
               Ich weiß absolut nicht \label{K_L03270-2v}\edtext{durch wen}{\lemma{\textnormal{\emph{durch wen}}}\Cendnote{\textnormal{Siehe XXXX Auszeichnungsfehler: Dokument L03271 nicht gefunden.
               }}}\label{K_L03270-2} man das erfahren hat. Morgen{ }Abend reise ich nach Salzburg\oindex{Salzburg@\textbf{Salzburg}, \emph{Verwaltungsgebiet}|pw}, für
               ein paar Tage – Vielleicht \label{K_L03270-3v}\edtext{kommen Sie
                  hin}{\lemma{\textnormal{\emph{kommen Sie
                  hin}}}\Cendnote{\textnormal{Dazu kam es nicht, siehe XXXX Auszeichnungsfehler: Dokument L03268 nicht gefunden.
               }}}\label{K_L03270-3}, ehe Sie nach Wien\oindex{Wien@\textbf{Wien}, \emph{Verwaltungsgebiet}|pw} fahren. Wir reisen dann
               zusammen nach Wien\oindex{Wien@\textbf{Wien}, \emph{Verwaltungsgebiet}|pw} zurück. Nachricht trifft mich
               in Salzburg\oindex{Salzburg@\textbf{Salzburg}, \emph{Verwaltungsgebiet}|pw} poste restante. Herzlich
                  \spacefill\mbox{Salten}\pend
           
\pstart
           \label{T_L03270-1v}\edtext{22./7. 97\textcolor{gray}{.}{ }½ 12 Nachm im Café.}{\lemma{\textnormal{\emph{22./7. 97. … Café.}}}\Cendnote{\textnormal{am
                     linken Rand, quer zum Text}}}\label{T_L03270-1}\pend
           \selectlanguage{ngerman}\endnumbering\briefempfaengerindex{Schnitzler, Arthur@\textsc{Schnitzler, Arthur}!zzzSalten, Felix@\emph{von Felix Salten}!1897-07-222@{22. 7. 1897}|)be}\mylabel{L03270h}  \newcommand{\dateiname}{L03270}\newcommand{\titel}{Felix Salten an Arthur Schnitzler, 22. 7. 1897}\newcommand{\editorInnen}{Martin Anton Müller und Laura Untner}%% latex-leseansicht-abspann.tex
%% Abspann für die Leseansicht.
%% Der Schalter \ifkorrekturansicht ist bereits durch den Vorspann gesetzt.

%% latex-abspann.tex
%% Gemeinsamer Abspann für Korrekturansicht und Leseansicht.
%% Setzt den Schalter \ifkorrekturansicht voraus (gesetzt in den
%% einbindenden Dateien latex-korrekturansicht-abspann.tex bzw.
%% latex-leseansicht-abspann.tex).
%% ---------------------------------------------------------------

\normalsize

% Das esempio-Environment wird nur in der Leseansicht benötigt
\ifkorrekturansicht\else
\newenvironment{esempio}[3]%
{
    \vspace{1.5ex}
    \rlap{\underline{#1}}
    \par
    \setlength{\parindent}{0cm}
    \nopagebreak
    \leftskip=#2cm
    \rightskip=#3cm
}
{
    \par
}
\fi

\doendnotes{C}
\bigskip
\vfill

\clearpage

\footnotesize

\ifkorrekturansicht
  \lohead{\textsc{register}}
\fi

% theindex-Environment neu definieren ohne reledmac
\makeatletter
\renewenvironment{theindex}{%
  \ifkorrekturansicht
    \section*{\indexname}%
  \else
    \subsubsection*{Index der erwähnten Entitäten}%
  \fi
  \setlength{\parindent}{0pt}%
  \setlength{\parskip}{0pt plus 0.3pt}%
  \let\item\@idxitem
}{%
  \ifkorrekturansicht\clearpage\fi
}
\makeatother

\IfFileExists{\jobname-pw.ind}{\input{\jobname-pw.ind}}{}

% Quellenangabe nur in der Leseansicht
\ifkorrekturansicht\else
% Fallback-Definitionen, falls die .tex-Datei \titel etc. nicht gesetzt hat
\providecommand{\titel}{}
\providecommand{\editorInnen}{}
\providecommand{\dateiname}{\jobname}

\vspace{3cm}

\vfill

\footnotesize
\textsc{Quelle}: \titel. Herausgegeben von {\editorInnen}. In: \emph{Arthur Schnitzler: Briefwechsel mit Autorinnen und Autoren}.
 Digitale Edition, https://schnitzler-briefe.acdh.oeaw.ac.at/{\dateiname}.html (Stand \today)
\fi

\end{document}


