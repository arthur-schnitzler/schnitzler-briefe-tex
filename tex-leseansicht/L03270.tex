%% latex-leseansicht-vorspann.tex
%% Vorspann für die Leseansicht.
%% Lädt die gemeinsame Datei latex-vorspann.tex mit nicht gesetztem Schalter.

\newif\ifkorrekturansicht
\korrekturansichtfalse

\input{../tex-inputs/latex-vorspann}

\begin{center}
            \textcolor{red}{ENTWURF, NICHT FERTIG KORRIGIERT}
                      \end{center}
            
         
         \renewcommand{\erwaehntePersonen}{Personen: Georg Hirschfeld}
         \renewcommand{\erwaehnteOrte}{Orte: Bad Ischl, Burgtheater, Hotel und Pension Rudolfshöhe (Leopold Petter), Salzburg, Wien}
         \renewcommand{\erwaehnteWerke}{Werke: Agnes Jordan. Schauspiel in fünf Akten, Theater, Kunst und Literatur [Agnes Jordan nicht am Burgtheater], Wiener Allgemeine Zeitung}
               \section[Felix Salten an Arthur Schnitzler, 22. 7. 1897]{ Felix Salten an Arthur Schnitzler, 22. 7. 1897}\nopagebreak\mylabel{v}\rehead{ }\begin{ledgroupsized}[t]{13cm}\normalsize\beginnumbering \toendnotes[C]{\smallbreak\pagebreak[2]} \Standort{CUL, Schnitzler, B 89, A 2.}
\physDesc{Postkarte
\newline{}Handschrift: Bleistift, lateinische Kurrent\newline{}Ordnung: mit Bleistift von unbekannter Hand nummeriert:
                                    »93« }\toendnotes[C]{\smallbreak}\pstart{}{\pb}Herrn D\textsuperscript{r} Arthur Schnitzler\pend{}\pstart{}Ischl\oindex{Bad Ischl@\textbf{Bad Ischl}|pw}\pend{}\pstart{}Kaltenbach, Pension Petter\oindex{Hotel und Pension Rudolfshoehe (Leopold Petter)@\textbf{Hotel und Pension Rudolfshöhe (Leopold Petter)}|pw}\pend{}{\bigskip}\pstart
           \noindent{}{\pb}Lieber Freund, ich lese soeben im 6-Uhr Blatt\pwindex{?? Werk@Nicht ermittelte Verfasserinnen und Verfasser!Wiener Allgemeine Zeitung1.3.1880 – 11.2.1934@\emph{Wiener Allgemeine Zeitung} {[}1.3.1880 – 11.2.1934{]}|pw} die \label{K_L03270-1v}\edtext{Notiz\pwindex{?? Werk@Nicht ermittelte Verfasserinnen und Verfasser!Theater, Kunst und Literatur [Agnes Jordan nicht am Burgtheater]23. 07. 1897@\emph{Theater, Kunst und Literatur [Agnes Jordan nicht am Burgtheater]} {[}23. 07. 1897{]}|pwv}}{\lemma{\textnormal{\emph{Notiz}}}\Cendnote{\textnormal{\emph{Theater, Kunst und Literatur}\pwindex{?? Werk@Nicht ermittelte Verfasserinnen und Verfasser!Theater, Kunst und Literatur [Agnes Jordan nicht am Burgtheater]23. 07. 1897@\emph{Theater, Kunst und Literatur [Agnes Jordan nicht am Burgtheater]} {[}23. 07. 1897{]}|pwk}. In: \emph{Wiener Allgemeine Zeitung}\pwindex{?? Werk@Nicht ermittelte Verfasserinnen und Verfasser!Wiener Allgemeine Zeitung1.3.1880 – 11.2.1934@\emph{Wiener Allgemeine Zeitung} {[}1.3.1880 – 11.2.1934{]}|pwk}, Nr. 5.818,
                        23. 7. 1897, S. 3: »– Wie wir aus
                     verläßlicher Quelle erfahren, ist die Direction des \so{Hofburgtheaters}\oindex{Burgtheater@\textbf{Burgtheater}|pw} von der Absicht, Georg \so{Hirschfeld}\pwindex{Hirschfeld, Georg 11.02.1873 – 17.01.1942@\textsc{Hirschfeld, Georg} (11.02.1873 – 17.01.1942), \emph{Schriftsteller}|pw}’s neues Drama ›\so{Agnes Jordan}\pwindex{Hirschfeld, Georg 11.02.1873 – 17.01.1942@\textsc{Hirschfeld, Georg} (11.02.1873 – 17.01.1942), \emph{Schriftsteller}!Agnes Jordan. Schauspiel in fuenf Akten1897@\strich\emph{Agnes Jordan. Schauspiel in fünf Akten} {[}1897{]}|pw}‹ nächste Saison zur Aufführung zu bringen, abgekommen.«}}}\label{K_L03270-1h}
               von Agnes Jordan\pwindex{Hirschfeld, Georg 11.02.1873 – 17.01.1942@\textsc{Hirschfeld, Georg} (11.02.1873 – 17.01.1942), \emph{Schriftsteller}!Agnes Jordan. Schauspiel in fuenf Akten1897@\strich\emph{Agnes Jordan. Schauspiel in fünf Akten} {[}1897{]}|pw}. Ich brauche Ihnen wol nicht
               erst zu sagen, dass ich derselben vollständig ferne stehe. Ich weiß absolut nicht
               durch wen man das erfahren hat. Morgen Abend reise ich nach Salzburg\oindex{Salzburg@\textbf{Salzburg}|pw}, für ein paar Tage – vielleicht kommen Sie
               hin, ehe Sie nach Wien\oindex{Wien@\textbf{Wien}|pw} fahren. Wir reisen dann
               zusammen nach Wien\oindex{Wien@\textbf{Wien}|pw} zurück. Nachricht trifft mich
               in Salzburg\oindex{Salzburg@\textbf{Salzburg}|pw} poste restante. \pend
           \pstart Herzlich \spacefill\mbox{Salten}\pend{}\pstart
           22./7. 97{ }½ 12 Nachm Im Café.\pend
           
         
         \endnumbering\mylabel{h}\end{ledgroupsized}\begin{anhang}\end{anhang}\newcommand{\dateiname}{L03270}\newcommand{\titel}{Felix Salten an Arthur Schnitzler, 22. 7. 1897}\newcommand{\editorInnen}{Martin Anton Müller und Laura Untner}%% latex-leseansicht-abspann.tex
%% Abspann für die Leseansicht.
%% Der Schalter \ifkorrekturansicht ist bereits durch den Vorspann gesetzt.

%% latex-abspann.tex
%% Gemeinsamer Abspann für Korrekturansicht und Leseansicht.
%% Setzt den Schalter \ifkorrekturansicht voraus (gesetzt in den
%% einbindenden Dateien latex-korrekturansicht-abspann.tex bzw.
%% latex-leseansicht-abspann.tex).
%% ---------------------------------------------------------------

\normalsize

% Das esempio-Environment wird nur in der Leseansicht benötigt
\ifkorrekturansicht\else
\newenvironment{esempio}[3]%
{
    \vspace{1.5ex}
    \rlap{\underline{#1}}
    \par
    \setlength{\parindent}{0cm}
    \nopagebreak
    \leftskip=#2cm
    \rightskip=#3cm
}
{
    \par
}
\fi

\doendnotes{C}
\bigskip
\vfill

\clearpage

\footnotesize

\ifkorrekturansicht
  \lohead{\textsc{register}}
\fi

% theindex-Environment neu definieren ohne reledmac
\makeatletter
\renewenvironment{theindex}{%
  \ifkorrekturansicht
    \section*{\indexname}%
  \else
    \subsubsection*{Index der erwähnten Entitäten}%
  \fi
  \setlength{\parindent}{0pt}%
  \setlength{\parskip}{0pt plus 0.3pt}%
  \let\item\@idxitem
}{%
  \ifkorrekturansicht\clearpage\fi
}
\makeatother

\IfFileExists{\jobname-pw.ind}{\input{\jobname-pw.ind}}{}

% Quellenangabe nur in der Leseansicht
\ifkorrekturansicht\else
% Fallback-Definitionen, falls die .tex-Datei \titel etc. nicht gesetzt hat
\providecommand{\titel}{}
\providecommand{\editorInnen}{}
\providecommand{\dateiname}{\jobname}

\vspace{3cm}

\vfill

\footnotesize
\textsc{Quelle}: \titel. Herausgegeben von {\editorInnen}. In: \emph{Arthur Schnitzler: Briefwechsel mit Autorinnen und Autoren}.
 Digitale Edition, https://schnitzler-briefe.acdh.oeaw.ac.at/{\dateiname}.html (Stand \today)
\fi

\end{document}


      