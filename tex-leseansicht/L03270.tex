%% latex-korrekturansicht-vorspann.tex
%% Vorspann für die Korrekturansicht.
%% Lädt die gemeinsame Datei latex-vorspann.tex mit gesetztem Schalter.

\newif\ifkorrekturansicht
\korrekturansichttrue

\input{../tex-inputs/latex-vorspann}


\section[ Felix Salten an Arthur Schnitzler, 22. 7. 1897]{L03270 Felix Salten an Arthur Schnitzler, 22. 7. 1897}
\nopagebreak\mylabel{L03270v}
\rehead{ }\normalsize\beginnumbering\briefempfaengerindex{Schnitzler, Arthur@\textsc{Schnitzler, Arthur}!zzzSalten, Felix@\emph{von Felix Salten}!1897-07-222@{22. 7. 1897}|(be}
\toendnotes[C]{\smallbreak\pagebreak[2]}\Standort{CUL, Schnitzler, B 89, A 2.}
\physDesc{Postkarte, 501 Zeichen
\newline{}Handschrift: Bleistift, lateinische Kurrent
\newline{}Versand: Stempel: »\nobreak{}\oindex{VIII., Josefstadt@\textbf{VIII., Josefstadt}, \emph{A.ADM3}|pwk}Wien 8/1 64, 23. 7. 97, 3–4 N\nobreak{}«. Stempel: »\nobreak{}\oindex{Bad Ischl@\textbf{Bad Ischl}, \emph{P.PPL}|pwk}{[}Ischl{]}, 6–\textcolor{gray}{7 V}\nobreak{}«.  
\newline{}Ordnung: mit Bleistift von unbekannter Hand nummeriert: »93« }
\buchAbdrucke{\weitereDrucke{Hermann Bahr, Arthur Schnitzler: \emph{Briefwechsel, Aufzeichnungen, Dokumente (1891–1931)}. Göttingen: \emph{Wallstein} 2018, S. 150.} }\toendnotes[C]{\smallbreak}\pstart{}{\pb}Herrn D\textsuperscript{r} Arthur Schnitzler\pend{}\pstart{}Ischl\oindex{Bad Ischl@\textbf{Bad Ischl}, \emph{P.PPL}|pw}\pend{}\pstart{}Kaltenbach, Pension Petter\oindex{Hotel und Pension Rudolfshoehe (Leopold Petter)@\textbf{Hotel und Pension Rudolfshöhe (Leopold Petter)}, \emph{Hotel (K.HTL)}|pw}\pend{}{\bigskip}\vspace{1em}
\pstart
           \noindent{}{\pb}Lieber Freund, ich lese soeben im 6-Uhr-Blatt\pwindex{Wiener Allgemeine Zeitung@\emph{Wiener Allgemeine Zeitung}|pw} die \label{K_L03270-1v}\edtext{Notiz\pwindex{Theater, Kunst und Literatur [Agnes Jordan nicht am Burgtheater]@\emph{Theater, Kunst und Literatur [Agnes Jordan nicht am Burgtheater]}|pwv}}{\lemma{\textnormal{\emph{Notiz}}}\Cendnote{\textnormal{ »– Wie wir aus verläßlicher
                     Quelle erfahren, ist die Direction des \so{Hofburgtheaters}\orgindex{Burgtheater@Burgtheater|pw} von der Absicht, Georg \so{Hirschfeld}\pwindex{Hirschfeld, Georg 11.02.1873 – 17.01.1942@\textsc{Hirschfeld, Georg} (11.02.1873 – 17.01.1942), \emph{Schriftsteller/Schriftstellerin}|pw}’s neues Drama ›\so{Agnes Jordan}\pwindex{Agnes Jordan. Schauspiel in fuenf Akten@\emph{Agnes Jordan. Schauspiel in fünf Akten}|pw}‹ nächste Saison zur Aufführung zu bringen, abgekommen.«
                     ([O. V.]: \emph{Theater, Kunst und
                        Literatur}\pwindex{Theater, Kunst und Literatur [Agnes Jordan nicht am Burgtheater]@\emph{Theater, Kunst und Literatur [Agnes Jordan nicht am Burgtheater]}|pwk}. In: \emph{Wiener Allgemeine
                        Zeitung}\pwindex{Wiener Allgemeine Zeitung@\emph{Wiener Allgemeine Zeitung}|pwk}, Nr. 5818, 23. 7. 1897,
                     S. 3.)}}}\label{K_L03270-1} von Agnes Jordan\pwindex{Agnes Jordan. Schauspiel in fuenf Akten@\emph{Agnes Jordan. Schauspiel in fünf Akten}|pw}. Ich
               brauche Ihnen wol nicht erst zu sagen, dass ich derselben vollständig ferne stehe.
               Ich weiß absolut nicht \label{K_L03270-2v}\edtext{durch wen}{\lemma{\textnormal{\emph{durch wen}}}\Cendnote{\textnormal{Siehe Felix Salten an Arthur Schnitzler, 23. 7. 1897.
               }}}\label{K_L03270-2} man das erfahren hat. Morgen{ }Abend reise ich nach Salzburg\oindex{Salzburg@\textbf{Salzburg}, \emph{A.ADM2}|pw}, für
               ein paar Tage – Vielleicht \label{K_L03270-3v}\edtext{kommen Sie
                  hin}{\lemma{\textnormal{\emph{kommen Sie
                  hin}}}\Cendnote{\textnormal{Dazu kam es nicht, siehe Felix Salten an Arthur Schnitzler, 13. 7. 1897.
               }}}\label{K_L03270-3}, ehe Sie nach Wien\oindex{Wien@\textbf{Wien}, \emph{A.ADM2}|pw} fahren. Wir reisen dann
               zusammen nach Wien\oindex{Wien@\textbf{Wien}, \emph{A.ADM2}|pw} zurück. Nachricht trifft mich
               in Salzburg\oindex{Salzburg@\textbf{Salzburg}, \emph{A.ADM2}|pw} poste restante. Herzlich
                  \spacefill\mbox{Salten}\pend
           
\pstart
           \label{T_L03270-1v}\edtext{22./7. 97\textcolor{gray}{.}{ }½ 12 Nachm im Café.}{\lemma{\textnormal{\emph{22./7. 97. … Café.}}}\Cendnote{\textnormal{am
                     linken Rand, quer zum Text}}}\label{T_L03270-1}\pend
           \selectlanguage{ngerman}\endnumbering\briefempfaengerindex{Schnitzler, Arthur@\textsc{Schnitzler, Arthur}!zzzSalten, Felix@\emph{von Felix Salten}!1897-07-222@{22. 7. 1897}|)be}\mylabel{L03270h}  \normalsize

\doendnotes{C}
\bigskip
\vfill

\clearpage

\footnotesize

\lohead{\textsc{register}}

% Definiere theindex-Environment komplett neu ohne reledmac
\makeatletter
\renewenvironment{theindex}{%
  \section*{\indexname}%
  \setlength{\parindent}{0pt}%
  \setlength{\parskip}{0pt plus 0.3pt}%
  \let\item\@idxitem
}{%
  \clearpage
}
\makeatother

\IfFileExists{\jobname-pw.ind}{\input{\jobname-pw.ind}}{}

\end{document}

      