%% latex-leseansicht-vorspann.tex
%% Vorspann für die Leseansicht.
%% Lädt die gemeinsame Datei latex-vorspann.tex mit nicht gesetztem Schalter.

\newif\ifkorrekturansicht
\korrekturansichtfalse

\input{../tex-inputs/latex-vorspann}


         
         \renewcommand{\erwaehntePersonen}{Personen:  ?? [Journalist, der fiktives russisches Interview verantwortet], Georg Brandes, Hans Brecka, Eugen Deimel, Anatole France, Maurice Maeterlinck, Romain Rolland, William Shakespeare, Leo N. von Tolstoi}
         \renewcommand{\erwaehnteInstitutionen}{Institutionen: Burgtheater, New Yorker Staats-Zeitung, [Russische Zeitschrift, in der 1914 gefälschtes Interview von Schnitzler erschien]}
         \renewcommand{\erwaehnteOrte}{Orte: Amerika, Deutschland, Russland, Schweiz, Sternwartestraße 71, Wien, Österreich, Österreich-Ungarn}
         \renewcommand{\erwaehnteWerke}{Werke: ?? [Fiktives Interview aus der Kriegszeit], Deutsche Tageszeitung, Kampf gegen den Theaterschund und Bühnenschmutz, Komödie der Worte, Komödie der Worte. Drei Einakter, Reichspost}
               \section[Arthur Schnitzler an Georg Brandes, 22. 12. 1915]{ Arthur Schnitzler an Georg Brandes, 22. 12. 1915}\nopagebreak\mylabel{v}\rehead{ }\begin{ledgroupsized}[t]{13cm}\normalsize\beginnumbering\briefempfaengerindex{Brandes, Georg@\textsc{Brandes, Georg}!zzzSchnitzler, Arthur@\emph{von Arthur Schnitzler}!1915-12-221@{22. 12. 1915}|(be} \toendnotes[C]{\smallbreak\pagebreak[2]} \Standort{Kopenhagen, Det Kongelige Bibliotek, Georg Brandes Arkiv, box 125.}
\physDesc{Brief, 2 Blätter, 3 Seiten, 2897 Zeichen (Seite 3 mit Schreibmaschine paginiert)
\newline{}Schreibmaschine
\newline{}Handschrift: schwarze Tinte (\noindent{}Überarbeitung, Unterstreichung, Unterschrift)
\newline{}Ordnung: mit Bleistift von unbekannter Hand auf dem ersten Blatt
                                 nummeriert: »39.«, das zweite Blatt datiert mit »22/12 15« }\buchAbdrucke{\weitereDrucke{1) Georg Brandes, Arthur Schnitzler: \emph{Ein Briefwechsel}. Hg. Kurt Bergel. Bern: \emph{Francke} 1956, S. 120–121.} \weitereDrucke{2) Arthur Schnitzler: \emph{Briefe 1913–1931}. Hg. Peter Michael Braunwarth, Richard Miklin, Susanne Pertlik und Heinrich Schnitzler. Frankfurt am Main: \emph{S. Fischer} 1984, S. 109–110.} }\toendnotes[C]{\smallbreak}\pstart
           \noindent{}{\pb}\textcolor{gray}{\textbf{Dr. Arthur Schnitzler}}{\\}\textcolor{gray}{\textbf{Wien XVIII. Sternwartestrasse 71\oindex{Sternwartestrasse 71@\textbf{Sternwartestraße 71}|pw}}}\pend
           \pstart
           \raggedleft{}22. 12. 1915. \pend
           \pstart\center{}Lieber und verehrter Freund.\pend\pstart
           Herzlichsten Dank für Ihre rasche Antwort\introOben{},\introOben{} und zugleich
               eine Aufklärung. Es ist mir gar nicht eingefallen eine \introOben{}»\introOben{}Anspielung\introOben{}«\introOben{} zu machen, d\substVorne{}\textsuperscript{a}\substDazwischen{}e\substHinten{}nn das, worauf ich Ihrer Meinung nach angespielt habe, ist mir bis zum
               Eintreffen Ihres Briefes total unbekannt geblieben. Wenn ich diesen richtig
               verstanden habe, hat man Ihnen offenbar Aeusserungen in den Mund gelegt, die Sie
               niemals getan haben. Mir ist gleich zu Anfang des Krieges ganz Aehnliches passiert.
               Von Freunden in Russland\oindex{Russland@\textbf{Russland}|pw} wurde ich in Kenntnis
               gesetzt, es sei in dortigen Zeitungen ein Interview\pwindex{?? [Journalist, der fiktives russisches Interview verantwortet] @\textsc{?? [Journalist, der fiktives russisches Interview verantwortet]}!?? [Fiktives Interview aus der Kriegszeit]November 1914@\strich\emph{?? [Fiktives Interview aus der Kriegszeit]} {[}November 1914{]}|pwv} erschienenen, in dem ich irgend einem Journalisten\pwindex{?? [Journalist, der fiktives russisches Interview verantwortet] @\textsc{?? [Journalist, der fiktives russisches Interview verantwortet]}|pwv} gegenüber die
               albernsten Dinge über Tolstoi\pwindex{Tolstoi, Leo N. von 9.09.1828 – 20.11.1910@\textsc{Tolstoi, Leo N. von} (9.09.1828 – 20.11.1910), \emph{Schriftsteller}|pw}, Anatole France\pwindex{France, Anatole 16.04.1844 – 12.10.1924@\textsc{France, Anatole} (16.04.1844 – 12.10.1924), \emph{Schriftsteller}|pw}, Shakespeare\pwindex{Shakespeare, William 23.4.1564? – 03.05.1616@\textsc{Shakespeare, William} (23.4.1564? – 03.05.1616), \emph{Schauspieler, Dramatiker}|pw} und Maeterlin\introOben{}c\introOben{}k\pwindex{Maeterlinck, Maurice 29.08.1862 – 06.05.1949@\textsc{Maeterlinck, Maurice} (29.08.1862 – 06.05.1949), \emph{Schriftsteller}|pw} geäussert hätte. Man riet mir dringend etwas
               dagegen zu unternehmen (was ich anfangs nicht wollte), weil man in Russland\oindex{Russland@\textbf{Russland}|pw} all diesen Unsinn glaubte. Durch
               Vermittlung Romain Rollands\pwindex{Rolland, Romain 29.01.1866 – 30.12.1944@\textsc{Rolland, Romain} (29.01.1866 – 30.12.1944), \emph{Schriftsteller}|pw} liess ich nun in
                  Schweiz\oindex{Schweiz@\textbf{Schweiz}|pw}er Blättern eine Entgegnung
                  erschei{\pb}nen, in der ich versicherte, dass ich
               niemals ein Wort von all dem Widersinn geäussert und bald darauf stellte sich das
               Ganze auch als die \label{K_L02224-1v}\edtext{Mystifikation}{\lemma{\textnormal{\emph{Mystifikation}}}\Cendnote{\textnormal{nicht ermittelt}}}\label{K_L02224-1h} irgend eines russischen Winkelblattes\orgindex{[Russische Zeitschrift, in der 1914 gefaelschtes Interview von Schnitzler erschien]@[Russische Zeitschrift, in der 1914 gefälschtes Interview von Schnitzler erschien]|pwv} heraus.
               Hingegen wurde ich von gewissen deutschen\oindex{Deutschland@\textbf{Deutschland}|pw} und österreichischen\oindex{Oesterreich@\textbf{Österreich}|pw}, selbstverständlich
               antisemitischen Blättern in der blödesten Weise angegriffen, weil ich es für
               notwendig gefunden hatte jene erlogenen Aeusserungen über die feindesländischen
               Dichter richtig zu stellen. Und noch bei Gelegenheit meiner letzten \label{K_L02224-2v}\edtext{Premiere\pwindex{Schnitzler, Arthur 15.05.1862 – 21.10.1931@\textsc{Schnitzler, Arthur} (15.05.1862 – 21.10.1931), \emph{Schriftsteller, Mediziner}!Komoedie der Worte. Drei Einakter1915-10-12@\strich\emph{Komödie der Worte. Drei Einakter} {[}1915-10-12{]}|pwv}}{\lemma{\textnormal{\emph{Premiere}}}\Cendnote{\textnormal{
                  Die Uraufführung von 
                  \emph{Komödie der Worte}\pwindex{Schnitzler, Arthur 15.05.1862 – 21.10.1931@\textsc{Schnitzler, Arthur} (15.05.1862 – 21.10.1931), \emph{Schriftsteller, Mediziner}!Komoedie der Worte. Drei Einakter1915-10-12@\strich\emph{Komödie der Worte. Drei Einakter} {[}1915-10-12{]}|pwk} hatte am 12. 10. 1915 
                  am \emph{Burgtheater}\orgindex{Burgtheater@Burgtheater|pwk} stattgefunden.}}}\label{K_L02224-2h} bekam ich es in irgend einem solchen, sich
                  patr\introOben{}i\introOben{}otisch gebärdenden Journal zu lesen, dass mir das
               Organ für diese Zeit fehle, wie ich ja schon zu Beginn des Krieges (wörtlich)
                  »\label{K_L02224-3v}\edtext{Torheiten über unsere Feinde\pwindex{?? Werk@Nicht ermittelte Verfasserinnen und Verfasser!Komoedie der Worte15. 10. 1915@\emph{Komödie der Worte} {[}15. 10. 1915{]}|pwv}}{\lemma{\textnormal{\emph{Torheiten … Feinde}}}\Cendnote{\textnormal{[O. V.]: \emph{Komödie der Worte}\pwindex{?? Werk@Nicht ermittelte Verfasserinnen und Verfasser!Komoedie der Worte15. 10. 1915@\emph{Komödie der Worte} {[}15. 10. 1915{]}|pwk}. In: \emph{Deutsche Tageszeitung}\pwindex{?? Werk@Nicht ermittelte Verfasserinnen und Verfasser!Deutsche Tageszeitung1.9.1894 – 1934@\emph{Deutsche Tageszeitung} {[}1.9.1894 – 1934{]}|pwk}, Jg. 22, Nr. 517,
                        15. 10. 1915, S. 6. Als unmittelbare Quelle bietet sich
                  die – möglicherweise von Hans Brecka\pwindex{Brecka, Hans 02.01.1885 – 07.10.1954@\textsc{Brecka, Hans} (02.01.1885 – 07.10.1954), \emph{Journalist}|pwk}
                  gestaltete – Zusammenstellung \emph{Kampf gegen den
                     Theaterschund und Bühnenschmutz}\pwindex{?? Werk@Nicht ermittelte Verfasserinnen und Verfasser!Kampf gegen den Theaterschund und Buehnenschmutz28. 10. 1915@\emph{Kampf gegen den Theaterschund und Bühnenschmutz} {[}28. 10. 1915{]}|pwk} ([O. V.], in: \emph{Reichspost}\pwindex{Reichspost1893 – 1938@\emph{Reichspost} {[}1893 – 1938{]}|pwk}, Jg. 22, Nr. 508, 28. 10. 1915,
                     S. 9) an.}}}\label{K_L02224-3h}« geäussert. Sie können sich also denken, lieber Freund,
               dass es mir schon a priori näher liegen müsste \strikeout{dergleichen} Zeitungsgeschwätz anzuzweifeln als es auf Treu und Glauben
               hinzunehmen. Meine von Ihnen missverstandene Bemerkung aber bezog sich nur auf den
               Umstand, dass unseres Wissens in den {\pb}ersten
               Monaten des Krieges die Presse aller neutralen Länder ihre Nachrichten – nicht nur
               über den Krieg selbst, sondern auch über die \uline{inneren}
               Zustände Deutschlands\oindex{Deutschland@\textbf{Deutschland}|pw} und Oesterreich-Ungarns\oindex{Oesterreich-Ungarn@\textbf{Österreich-Ungarn}|pw} in reicherem Mass von der Entente als von
               den Zentralmächten bezog, sowie ich mich auch gedrängt fühlte Freunde\pwindex{Deimel, Eugen Maerz 1860 – 10.03.1920@\textsc{Deimel, Eugen} (März 1860 – 10.03.1920), \emph{Journalist}|pwv} in Amerika\oindex{Amerika@\textbf{Amerika}|pw} in diesem Sinne nach Möglichkeit aufzuklären (was übrigens zur Folge
               hatte, dass einer dieser \label{K_L02224-4v}\edtext{Privatbriefe}{\lemma{\textnormal{\emph{Privatbriefe}}}\Cendnote{\textnormal{Die ganze Angelegenheit
                  wird ausführlicher in Schnitzlers\pwindex{Schnitzler, Arthur 15.05.1862 – 21.10.1931@\textsc{Schnitzler, Arthur} (15.05.1862 – 21.10.1931), \emph{Schriftsteller, Mediziner}|pwk} Brief an Eugen Deimel\pwindex{Deimel, Eugen Maerz 1860 – 10.03.1920@\textsc{Deimel, Eugen} (März 1860 – 10.03.1920), \emph{Journalist}|pwk} vom 25. 11. 1914
                  dargestellt
                     (Heinz P. Adamek: \emph{In die Neue Welt… Arthur Schnitzler –
                        Eugen Deimel Briefwechsel}. Wien: \emph{Holzhausen}{ }2003, S. 210–211). Siehe A. S.: \emph{»Das Zeitlose ist von kürzester Dauer«}, Ein Brief von Artur Schnitzler, 20. 11. 1914.}}}\label{K_L02224-4h} ganz entstellt in ein New-Yorker Blatt\orgindex{New Yorker Staats-Zeitung@New Yorker Staats-Zeitung|pwv} und von dort
               wieder \introOben{}–\introOben{} noch entstellter in deutsche Blätter überging.
               Also ich denke wir wissen beide wie viel wir von dem zu halten haben, was in den
               Zeitungen steht\substVorne{}\textsuperscript{.}\substDazwischen{}!\substHinten{}\introOben{})\introOben{}\pend
           \pstart
           Für heute nur so viel; mögen Ihnen die Feiertage lauter Gutes, insbesondere völlige
               Genesung bringen und uns allen eine gegründetere Hoffnung auf die baldige Wiederkehr
               schönerer Zeiten, als wir sie nach dem augenblicklichen Stand der Dinge hegen
               dürfen.\pend
           \pstart
           Mit herzlichen Grüssen{\\[\baselineskip]}Ihr allezeit freundschaftlich ergebener{\\[\baselineskip]}\spacefill\mbox{{[}hs.:{]} Arthur Schnitzler}\pend
           \leftskip=0em{}
         
         \endnumbering\mylabel{h}\end{ledgroupsized}  \newcommand{\dateiname}{L02224}\newcommand{\titel}{Arthur Schnitzler an Georg Brandes, 22. 12. 1915}\newcommand{\editorInnen}{Martin Anton Müller und Gerd-Hermann Susen}%% latex-leseansicht-abspann.tex
%% Abspann für die Leseansicht.
%% Der Schalter \ifkorrekturansicht ist bereits durch den Vorspann gesetzt.

%% latex-abspann.tex
%% Gemeinsamer Abspann für Korrekturansicht und Leseansicht.
%% Setzt den Schalter \ifkorrekturansicht voraus (gesetzt in den
%% einbindenden Dateien latex-korrekturansicht-abspann.tex bzw.
%% latex-leseansicht-abspann.tex).
%% ---------------------------------------------------------------

\normalsize

% Das esempio-Environment wird nur in der Leseansicht benötigt
\ifkorrekturansicht\else
\newenvironment{esempio}[3]%
{
    \vspace{1.5ex}
    \rlap{\underline{#1}}
    \par
    \setlength{\parindent}{0cm}
    \nopagebreak
    \leftskip=#2cm
    \rightskip=#3cm
}
{
    \par
}
\fi

\doendnotes{C}
\bigskip
\vfill

\clearpage

\footnotesize

\ifkorrekturansicht
  \lohead{\textsc{register}}
\fi

% theindex-Environment neu definieren ohne reledmac
\makeatletter
\renewenvironment{theindex}{%
  \ifkorrekturansicht
    \section*{\indexname}%
  \else
    \subsubsection*{Index der erwähnten Entitäten}%
  \fi
  \setlength{\parindent}{0pt}%
  \setlength{\parskip}{0pt plus 0.3pt}%
  \let\item\@idxitem
}{%
  \ifkorrekturansicht\clearpage\fi
}
\makeatother

\IfFileExists{\jobname-pw.ind}{\input{\jobname-pw.ind}}{}

% Quellenangabe nur in der Leseansicht
\ifkorrekturansicht\else
% Fallback-Definitionen, falls die .tex-Datei \titel etc. nicht gesetzt hat
\providecommand{\titel}{}
\providecommand{\editorInnen}{}
\providecommand{\dateiname}{\jobname}

\vspace{3cm}

\vfill

\footnotesize
\textsc{Quelle}: \titel. Herausgegeben von {\editorInnen}. In: \emph{Arthur Schnitzler: Briefwechsel mit Autorinnen und Autoren}.
 Digitale Edition, https://schnitzler-briefe.acdh.oeaw.ac.at/{\dateiname}.html (Stand \today)
\fi

\end{document}


      