%% latex-leseansicht-vorspann.tex
%% Vorspann für die Leseansicht.
%% Lädt die gemeinsame Datei latex-vorspann.tex mit nicht gesetztem Schalter.

\newif\ifkorrekturansicht
\korrekturansichtfalse

\input{../tex-inputs/latex-vorspann}


         
         \renewcommand{\erwaehntePersonen}{Personen: Hermann Bahr, Richard Beer-Hofmann, Oskar Blumenthal, Wilhelm Bölsche, Marie Herzfeld, Theodor Herzl, Hugo von Hofmannsthal, Julius Kulka, Falk Schupp, Gustav Schwarzkopf}
         \renewcommand{\erwaehnteInstitutionen}{Institutionen: Bibliographisches Bureau, Weltausstellung 1893}
         \renewcommand{\erwaehnteOrte}{Orte: Bad Ischl, Berlin, Chicago, Prag, Wien}
         \renewcommand{\erwaehnteWerke}{Werke: Anatol, Der Sohn. Aus den Papieren eines Arztes}
               \section[Arthur Schnitzler an Hugo von Hofmannsthal, {[}6. 8. 1892{]}]{ Arthur Schnitzler an Hugo von Hofmannsthal, {[}6. 8. 1892{]}}\nopagebreak\mylabel{v}\rehead{ }\begin{ledgroupsized}[t]{13cm}\normalsize\beginnumbering \toendnotes[C]{\smallbreak\pagebreak[2]} \Standort{FDH, Hs-30885,24.}
\physDesc{Brief, 2 Blätter, 6 Seiten, 2153 Zeichen
\newline{}Handschrift: schwarze Tinte, deutsche Kurrent
\newline{}Ordnung: mit Bleistift von Schnitzler mutmaßlich bei der Durchsicht der
                                 Briefe 1929 das erste Blatt beschriftet: »Wien\oindex{Wien@\textbf{Wien}|pw}« und datiert: »6. 8. 92«. Das zweite Blatt datiert: »(6. 8. 92{[}){]}« }\buchAbdrucke{\weitereDrucke{1) Hugo von Hofmannsthal, Arthur Schnitzler: \emph{Briefwechsel}. Hg. Therese Nickl und Heinrich Schnitzler. Frankfurt am Main: \emph{S. Fischer} 1964, S. 27–28.} \weitereDrucke{2) Hermann Bahr, Arthur Schnitzler: \emph{Briefwechsel, Aufzeichnungen, Dokumente (1891–1931)}. Hg. Kurt Ifkovits und Martin Anton Müller. Göttingen: \emph{Wallstein} 2018.} }\toendnotes[C]{\smallbreak}\pstart{}{\pb}Mein lieber Loris,\pend\pstart
           vielen Dank für den überſandten Brief. Es ſtehen geſcheidte Sachen drin. Es iſt ſogar
               möglich, daſs die H.\pwindex{Herzfeld, Marie 20.03.1855 – 22.09.1940@\textsc{Herzfeld, Marie} (20.03.1855 – 22.09.1940), \emph{Schriftstellerin, Übersetzerin}|pw} mit all ihrem Tadel Recht
               hat: gewiſs aber hat ſie manches zu loben vergeſſen. Daſs ſie den »Sohn\pwindex{Schnitzler, Arthur 15.05.1862 – 21.10.1931@\textsc{Schnitzler, Arthur} (15.05.1862 – 21.10.1931), \emph{Schriftsteller, Mediziner}!Sohn. Aus den Papieren eines Arztes1. 1. 1892@\strich\emph{Der Sohn. Aus den Papieren eines Arztes} {[}1. 1. 1892{]}|pw}« ſo beſonders gut findet zeigt mir, daſs ſie
               ein wenig vom Berliner\oindex{Berlin@\textbf{Berlin}|pw}-Bölſche\pwindex{Boelsche, Wilhelm 02.01.1861 – 31.08.1939@\textsc{Bölsche, Wilhelm} (02.01.1861 – 31.08.1939), \emph{Schriftsteller, Publizist}|pw}thum beeinflußt iſt. Ich habe den Eindruck, daſs ſie
               alles einzelne an mir verſteht, wie das bei ihrer kritiſchen {\pb}Begabung ſelbſtverſtändlich – nur meine \uline{Atmosphäre} nicht. –\pend
           \pstart
           Das Anatol\pwindex{Schnitzler, Arthur 15.05.1862 – 21.10.1931@\textsc{Schnitzler, Arthur} (15.05.1862 – 21.10.1931), \emph{Schriftsteller, Mediziner}!Anatol1892-10-29@\strich\emph{Anatol} {[}1892-10-29{]}|pw}-Buch erſcheint im \textsc{Bibliogr. Bureau\orgindex{Bibliographisches Bureau@Bibliographisches Bureau|pw}, Berlin\oindex{Berlin@\textbf{Berlin}|pw}}. –\pend
           \pstart
           Von Blumenthal\pwindex{Blumenthal, Oskar 13.03.1852 – 24.04.1917@\textsc{Blumenthal, Oskar} (13.03.1852 – 24.04.1917), \emph{Schriftsteller, Journalist, Theaterleiter}|pw} hab ich Nachricht: 2. Quartal,
               d. h. Jaenner–März 93 Etwas ſpät! Umſomehr als ich heute
               aus Prag\oindex{Prag@\textbf{Prag}|pw} die Mittheilung erhalte, daſs das Stück\pwindex{Schnitzler, Arthur 15.05.1862 – 21.10.1931@\textsc{Schnitzler, Arthur} (15.05.1862 – 21.10.1931), \emph{Schriftsteller, Mediziner}!Anatol1892-10-29@\strich\emph{Anatol} {[}1892-10-29{]}|pwv} im Oktober dranko{\geminationm}en dürfte! Zugleich hat man mir meine Luſtſpiele von
               dort retournirt, da ſie für eine Provinzbühne zu gewagt ſeien.\pend
           \pstart
           {\pb}– \textsc{Schupp}\pwindex{Schupp, Falk 21.09.1870 – 06.02.1922@\textsc{Schupp, Falk} (21.09.1870 – 06.02.1922), \emph{Historiker, Zahnarzt}|pw} iſt Secretär des Preſsausſchuſses für d. \textsc{Chicago\oindex{Chicago@\textbf{Chicago}|pw}. W. A.\orgindex{Weltausstellung 1893@Weltausstellung 1893|pw}} –\pend
           \pstart
           – \textsc{Von Theodor Herzl\pwindex{Herzl, Theodor 1860-05-02 – 1904-07-03@\textsc{Herzl, Theodor} (1860-05-02 – 1904-07-03), \emph{Schriftsteller, Journalist}|pw}} hab ich einen reizenden Brief beko{\geminationm}en. –\pend
           \pstart
           Vielleicht ſehen wir uns doch im Laufe dieſes So{\geminationm}ers.
               Ich habe nämlich keine Einberufung zur Waffenübung beko{\geminationm}en, und fahre vielleicht Ende Auguſt nach Iſchl\oindex{Bad Ischl@\textbf{Bad Ischl}|pw}. – Wohin gehn Sie im September? –\pend
           \pstart
           – Ich kam die letzten Tage nicht zum Schreiben; die äußerliche Thätigkeit ſtört doch.
               Hoffentlich bald! – Sie {\pb}ko{\geminationm}en ja ſicher mit den ganzen 5 Akten zurück! ––\pend
           \pstart
           Haben Sie Recht, von einem »\uline{herrſchenden}
               Novellendrama« zu ſprechen? – Berechtigung hat die Form gewiſs – ſobald nur \uline{ein} bedeutender Menſch da iſt, der daran Freude
               findet. Ueber den gewiſſen Fundamentalſatz: »Das iſt eben kein rechtes Drama, das
               nicht von der Bühne herab wirkt (oder gar ›auf die Menge‹ wirkt\strikeout{«})« hab ich {\pb}mich i{\geminationm}er geärgert. Eventuell will ich mir, mir ganz allein
               was vorſpielen laſſen! – Na, Sie wiſſen ja, Kulka\pwindex{Kulka, Julius 25.09.1865 – 22.09.1893@\textsc{Kulka, Julius} (25.09.1865 – 22.09.1893), \emph{Rechtsanwalt}|pw} hat ja das wichtigſte über dieſes Thema ſchon gesagt. –\pend
           \pstart
           – Wa{\geminationn} wird man ſich Briefe phonographiren können? – Die
               Zeit ſeh ich ko{\geminationm}en, wo die Leute über unſre mühſelige
               Correſpondenzerei lächeln und ſtaunen werden.\pend
           \pstart
           {\pb}Auf dieſer Seite ſteht nur mehr, daſs ich Sie, liebſter
               Freund, aufs Herzlichſte grüße!\pend
           \pstart
           Ganz der Ihre{\\[\baselineskip]}\spacefill\mbox{Arthur.}\pend
           \leftskip=0em{}\pstart
           \noindent{}Was macht \textsc{Richard}\pwindex{Beer-Hofmann, Richard 1866-07-11 – 1945-09-26@\textsc{Beer-Hofmann, Richard} (1866-07-11 – 1945-09-26), \emph{Schriftsteller}|pw}? –\pend
           \pstart
           – Mit \textsc{Schwarzkopf}\pwindex{Schwarzkopf, Gustav 07.11.1853 – 13.11.1939@\textsc{Schwarzkopf, Gustav} (07.11.1853 – 13.11.1939), \emph{Schriftsteller}|pw} war ich einige Male auf dem Land. –\pend
           \pstart
           \textsc{Bahr}\pwindex{Bahr, Hermann 19.07.1863 – 15.01.1934@\textsc{Bahr, Hermann} (19.07.1863 – 15.01.1934), \emph{Schriftsteller, Kritiker}|pw} iſt verzweifelt; – er wurde einberufen und fahndet nun nach einer
                  Befreiung. –\pend
           
         
         \endnumbering\mylabel{h}\end{ledgroupsized}  \newcommand{\dateiname}{L00112}\newcommand{\titel}{Arthur Schnitzler an Hugo von Hofmannsthal, [6. 8. 1892]}\newcommand{\editorInnen}{ Martin Anton Müller und Gerd-Hermann Susen}%% latex-leseansicht-abspann.tex
%% Abspann für die Leseansicht.
%% Der Schalter \ifkorrekturansicht ist bereits durch den Vorspann gesetzt.

%% latex-abspann.tex
%% Gemeinsamer Abspann für Korrekturansicht und Leseansicht.
%% Setzt den Schalter \ifkorrekturansicht voraus (gesetzt in den
%% einbindenden Dateien latex-korrekturansicht-abspann.tex bzw.
%% latex-leseansicht-abspann.tex).
%% ---------------------------------------------------------------

\normalsize

% Das esempio-Environment wird nur in der Leseansicht benötigt
\ifkorrekturansicht\else
\newenvironment{esempio}[3]%
{
    \vspace{1.5ex}
    \rlap{\underline{#1}}
    \par
    \setlength{\parindent}{0cm}
    \nopagebreak
    \leftskip=#2cm
    \rightskip=#3cm
}
{
    \par
}
\fi

\doendnotes{C}
\bigskip
\vfill

\clearpage

\footnotesize

\ifkorrekturansicht
  \lohead{\textsc{register}}
\fi

% theindex-Environment neu definieren ohne reledmac
\makeatletter
\renewenvironment{theindex}{%
  \ifkorrekturansicht
    \section*{\indexname}%
  \else
    \subsubsection*{Index der erwähnten Entitäten}%
  \fi
  \setlength{\parindent}{0pt}%
  \setlength{\parskip}{0pt plus 0.3pt}%
  \let\item\@idxitem
}{%
  \ifkorrekturansicht\clearpage\fi
}
\makeatother

\IfFileExists{\jobname-pw.ind}{\input{\jobname-pw.ind}}{}

% Quellenangabe nur in der Leseansicht
\ifkorrekturansicht\else
% Fallback-Definitionen, falls die .tex-Datei \titel etc. nicht gesetzt hat
\providecommand{\titel}{}
\providecommand{\editorInnen}{}
\providecommand{\dateiname}{\jobname}

\vspace{3cm}

\vfill

\footnotesize
\textsc{Quelle}: \titel. Herausgegeben von {\editorInnen}. In: \emph{Arthur Schnitzler: Briefwechsel mit Autorinnen und Autoren}.
 Digitale Edition, https://schnitzler-briefe.acdh.oeaw.ac.at/{\dateiname}.html (Stand \today)
\fi

\end{document}


      