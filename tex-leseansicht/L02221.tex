%% latex-korrekturansicht-vorspann.tex
%% Vorspann für die Korrekturansicht.
%% Lädt die gemeinsame Datei latex-vorspann.tex mit gesetztem Schalter.

\newif\ifkorrekturansicht
\korrekturansichttrue

\input{../tex-inputs/latex-vorspann}


\section[Georg Brandes an Arthur Schnitzler, 4. 12. 1915]{L02221 Georg Brandes an Arthur Schnitzler, 4. 12. 1915}
\nopagebreak\mylabel{L02221v}
\rehead{ }\normalsize\beginnumbering\briefempfaengerindex{Schnitzler, Arthur@\textsc{Schnitzler, Arthur}!zzzBrandes, Georg@\emph{von Georg Brandes}!1915-12-041@{4. 12. 1915}|(be}
\toendnotes[C]{\smallbreak\pagebreak[2]}\Standort{CUL, Schnitzler, B 17.}
\physDesc{Brief, 1 Blatt, 4 Seiten, 2317 Zeichen
\newline{}Handschrift: schwarze Tinte, lateinische Kurrent
\newline{}Schnitzler: mit rotem Buntstift mehrere Unterstreichungen 
\newline{}Ordnung: mit Bleistift von unbekannter Hand nummeriert:
                                    »45« }
\buchAbdrucke{\weitereDrucke{Georg Brandes, Arthur Schnitzler: \emph{Ein Briefwechsel}. Bern: \emph{Francke} 1956, S. 114–115.} }\toendnotes[C]{\smallbreak}
\pstart
           \raggedleft{}{\pb}Kopenhagen\oindex{Kopenhagen@\textbf{Kopenhagen}, \emph{P.PPLC}|pw} (genügende Adresse){\\}4 December 15\pend
           
\pstart{}Verehrter Freund\pend\vspace{0.5em}
\pstart
           Drei Jahre sind vergangen, seit ich Ihr Gast war und die Freude hatte, in Ihrem Heim
               mit Ihnen, Ihrer Frau Gemahlin\pwindex{Schnitzler, Olga 17.01.1882 – 13.01.1970@\textsc{Schnitzler, Olga} (17.01.1882 – 13.01.1970), \emph{Schauspieler/Schauspielerin, Sänger/Sängerin}|pwv} und Ihren Freunden zu verkehren. Seit dem – wie viel unerhörtes ist
               geschehen! Alles ist anders geworden.\pend
           
\pstart
           Ich wollte Ihnen schon vor einem Monat für Ihre dauerhafte Freundschaft danken, dass
               Sie mir die \label{K_L02221-1v}\edtext{\uline{Komödie der Worte}\pwindex{Komoedie der Worte. Drei Einakter@\emph{Komödie der Worte. Drei Einakter}|pw} sandten}{\lemma{\textnormal{\emph{Komödie … sandten}}}\Cendnote{\textnormal{Vermutlich tat Schnitzler dies am 20. 10. 1914. Am 21. 10. 1915 bedankte sich
                  Robert Adam\pwindex{Adam, Robert 20.04.1877 – 16.10.1961@\textsc{Adam, Robert} (20.04.1877 – 16.10.1961), \emph{Schriftsteller/Schriftstellerin, Richter/Richterin}|pwk} für die Zusendung des Buchs.}}}\label{K_L02221-1}. Sie haben wieder einmal das Labyrinthische dargestellt der erotischen
               Neigungen und wie die Ehen die Herzen hemmen und fesseln. Tragisches und
               Possierliches ist nach Ihrer Gewohnheit gemischt. Mir war Alles lieb.\pend
           
\pstart
           Vor etwa drei Wochen sah ich in {\pb}einem grossen privaten Verein hier Ihren \uline{Dr. Bernhardi}\pwindex{Professor Bernhardi. Komoedie in fuenf Akten@\emph{Professor Bernhardi. Komödie in fünf Akten}|pw} im Wesentlichen ganz vorzüglich aufgeführt. Das Stück ist mir theuer; nur kann
               ich mich nicht mit der Logik recht befreunden, dass weil jemand nicht zum Märtyrer
               geeignet ist, er überhaupt nicht für seine Ueberzeugung eintreten solle. Wir lassen
               ja alle ohne Protest das meiste hingehen, weil das Protestiren doch nichts nützt;
               aber Sie sollten nicht unsere Handlungskraft durch Entmuthigung lähmen. Das ist die
               alte »Ironie« der Romantiker, die dem Pathos die Spitze abbricht.\pend
           
\pstart
           Doch, was liegt heutzutage an all dem! Macduff sagt:\pend
           \stanza{}O horror, horror, horror\pwindex{Macbeth@\emph{Macbeth}|pwv}Tongue nor heart\pwindex{Macbeth@\emph{Macbeth}|pwv}Cannot conceive nor name
                     thee.\pwindex{Macbeth@\emph{Macbeth}|pwv}\stanzaend{}
\pstart
           {\pb}Ich habe leider im Augenblick
               wieder einen Anfall von meiner chronischen Krankheit, der Venenentzündung. Sie kam
               zum ersten mal in 1871 nach einem Typhus, und seit 1897
               wieder nur zu oft. Nach 2 ½ Jahren macht sie mir wieder ihren Besuch.\pend
           
\pstart
           Die grosse Maschine\pwindex{Wolfgang Goethe@\emph{Wolfgang Goethe}|pwv}, die ich
               über Goethe\pwindex{Goethe, Johann Wolfgang von 1749-08-28 – 1832-03-22@\textsc{Goethe, Johann Wolfgang von} (1749-08-28 – 1832-03-22), \emph{Schriftsteller/Schriftstellerin}|pw} machte, wurde schnell (in diesem
               kleinen Land\oindex{Daenemark@\textbf{Dänemark}, \emph{A.PCLI}|pwv}) in 3,500
               Exemplaren verkauft. Eine neue Auflage ein wenig verbessert, ist erschienen. Es sind
               zwei recht dicke Bände. Ausserdem habe ich viele grössere und kleinere Artikel über
               die Zustände – leider in unserer Geheimsprache – geschrieben.\pend
           
\pstart
           Peter Nansen\pwindex{Nansen, Peter 20.01.1861 – 31.07.1918@\textsc{Nansen, Peter} (20.01.1861 – 31.07.1918), \emph{Schriftsteller/Schriftstellerin, Journalist/Journalistin, Verleger/Verlegerin}|pw}, den Sie kennen, hat seine
               Production wieder aufgenommen und u. a. eine nicht unbedeutende grössere Novelle\pwindex{Brueder Menthe@\emph{Die Brüder Menthe}|pwv} erscheinen lassen.
               Selbst liegt er leider krank. Er hat zuviele {\pb}Cigaretten geraucht, zuviel Whisky
               getrunken, sein Herz scheint gelitten zu haben, er hat seit 3–4 Wochen ein\strikeout{en} schwaches Fieber, das nicht weichen will. Ich liebe
               ihn sehr und bin um ihn bekümmert.\pend
           
\pstart
           Liebster Freund\hspace*{1.5em}Empfehlen Sie mich den Ihrigen und
               bleiben Sie mir gut.\pend
           \pstart Ihr \spacefill\mbox{Georg Brandes}\pend{}\selectlanguage{ngerman}\endnumbering\briefempfaengerindex{Schnitzler, Arthur@\textsc{Schnitzler, Arthur}!zzzBrandes, Georg@\emph{von Georg Brandes}!1915-12-041@{4. 12. 1915}|)be}\mylabel{L02221h}  \normalsize

\doendnotes{C}
\bigskip
\vfill

\clearpage

\footnotesize

\lohead{\textsc{register}}

% Definiere theindex-Environment komplett neu ohne reledmac
\makeatletter
\renewenvironment{theindex}{%
  \section*{\indexname}%
  \setlength{\parindent}{0pt}%
  \setlength{\parskip}{0pt plus 0.3pt}%
  \let\item\@idxitem
}{%
  \clearpage
}
\makeatother

\IfFileExists{\jobname-pw.ind}{\input{\jobname-pw.ind}}{}

\end{document}

      