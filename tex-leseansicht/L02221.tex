%% latex-leseansicht-vorspann.tex
%% Vorspann für die Leseansicht.
%% Lädt die gemeinsame Datei latex-vorspann.tex mit nicht gesetztem Schalter.

\newif\ifkorrekturansicht
\korrekturansichtfalse

\input{../tex-inputs/latex-vorspann}


\section[Georg Brandes an Arthur Schnitzler, 4. 12. 1915]{L02221 Georg Brandes an Arthur Schnitzler, 4. 12. 1915}
\nopagebreak\mylabel{L02221v}
\rehead{ }\normalsize\beginnumbering\briefempfaengerindex{Schnitzler, Arthur@\textsc{Schnitzler, Arthur}!zzzBrandes, Georg@\emph{von Georg Brandes}!1915-12-041@{4. 12. 1915}|(be}
\toendnotes[C]{\smallbreak\pagebreak[2]}
\correspDesc{Versand  durch Georg Brandes am 4. 12. 1915 in Kopenhagen
\newline{}Erhalt  durch Arthur Schnitzler im Zeitraum [5. 12. 1915
                  – 9. 12. 1915?] in Wien}\toendnotes[C]{\smallbreak}
\Standort{CUL, Schnitzler, B 17.}
\physDesc{Brief, 1 Blatt, 4 Seiten, 2317 Zeichen
\newline{}Handschrift: schwarze Tinte, lateinische Kurrent
\newline{}Schnitzler: mit rotem Buntstift mehrere Unterstreichungen 
\newline{}Ordnung: mit Bleistift von unbekannter Hand nummeriert:
                                    »45« }
\buchAbdrucke{\weitereDrucke{Georg Brandes, Arthur Schnitzler: \emph{Ein Briefwechsel}. Herausgegeben von Kurt Bergel. Bern: \emph{Francke} 1956, S. 114–115.} }\toendnotes[C]{\smallbreak}
\pstart
           \raggedleft{}{\pb}Kopenhagen\oindex{Kopenhagen@\textbf{Kopenhagen}, \emph{Hauptstadt}|pw} (genügende Adresse){\\}4 December 15\pend
           
\pstart{}Verehrter Freund\pend\vspace{0.5em}
\pstart
           Drei Jahre sind vergangen, seit ich Ihr Gast war und die Freude hatte, in Ihrem Heim
               mit Ihnen, Ihrer Frau Gemahlin\pwindex{Schnitzler, Olga 17.\,1.\,1882 Wien – 13.\,1.\,1970 Lugano@\textsc{Schnitzler, Olga} (17.\,1.\,1882 Wien – 13.\,1.\,1970 Lugano), \emph{Schauspielerin, Sängerin}|pwv} und Ihren Freunden zu verkehren. Seit dem – wie viel unerhörtes ist
               geschehen! Alles ist anders geworden.\pend
           
\pstart
           Ich wollte Ihnen schon vor einem Monat für Ihre dauerhafte Freundschaft danken, dass
               Sie mir die \label{K_L02221-1v}\edtext{\uline{Komödie der Worte}\pwindex{Schnitzler, Arthur 15.\,5.\,1862 Wien – 21.\,10.\,1931 ebd.@\textsc{Schnitzler, Arthur} (15.\,5.\,1862 Wien – 21.\,10.\,1931 ebd.), \emph{Schriftsteller, Mediziner}!Komödie der Worte. Drei Einakter@\strich\emph{Komödie der Worte. Drei Einakter}|pw} sandten}{\lemma{\textnormal{\emph{Komödie … sandten}}}\Cendnote{\textnormal{Vermutlich tat Schnitzler dies am 20. 10. 1914. Am XXXX Auszeichnungsfehler: Dokument L02220 nicht gefunden bedankte sich
                  Robert Adam\pwindex{Adam, Robert 20.\,4.\,1877 Wien – 16.\,10.\,1961 Baden bei Wien@\textsc{Adam, Robert} (20.\,4.\,1877 Wien – 16.\,10.\,1961 Baden bei Wien), \emph{Schriftsteller, Richter}|pwk} für die Zusendung des Buchs.}}}\label{K_L02221-1}. Sie haben wieder einmal das Labyrinthische dargestellt der erotischen
               Neigungen und wie die Ehen die Herzen hemmen und fesseln. Tragisches und
               Possierliches ist nach Ihrer Gewohnheit gemischt. Mir war Alles lieb.\pend
           
\pstart
           Vor etwa drei Wochen sah ich in {\pb}einem grossen privaten Verein hier Ihren \uline{Dr. Bernhardi}\pwindex{Schnitzler, Arthur 15.\,5.\,1862 Wien – 21.\,10.\,1931 ebd.@\textsc{Schnitzler, Arthur} (15.\,5.\,1862 Wien – 21.\,10.\,1931 ebd.), \emph{Schriftsteller, Mediziner}!Professor Bernhardi. Komödie in fünf Akten@\strich\emph{Professor Bernhardi. Komödie in fünf Akten}|pw} im Wesentlichen ganz vorzüglich aufgeführt. Das Stück ist mir theuer; nur kann
               ich mich nicht mit der Logik recht befreunden, dass weil jemand nicht zum Märtyrer
               geeignet ist, er überhaupt nicht für seine Ueberzeugung eintreten solle. Wir lassen
               ja alle ohne Protest das meiste hingehen, weil das Protestiren doch nichts nützt;
               aber Sie sollten nicht unsere Handlungskraft durch Entmuthigung lähmen. Das ist die
               alte »Ironie« der Romantiker, die dem Pathos die Spitze abbricht.\pend
           
\pstart
           Doch, was liegt heutzutage an all dem! Macduff sagt:\pend
           \stanza{}O horror, horror, horror\pwindex{\textcolor{red}{\textsuperscript{XXXX indx1}}!Macbeth@\strich\emph{Macbeth}|pwv}\newverse{}Tongue nor heart\pwindex{\textcolor{red}{\textsuperscript{XXXX indx1}}!Macbeth@\strich\emph{Macbeth}|pwv}\newverse{}Cannot conceive nor name
                     thee.\pwindex{\textcolor{red}{\textsuperscript{XXXX indx1}}!Macbeth@\strich\emph{Macbeth}|pwv}\stanzaend{}
\pstart
           {\pb}Ich habe leider im Augenblick
               wieder einen Anfall von meiner chronischen Krankheit, der Venenentzündung. Sie kam
               zum ersten mal in 1871 nach einem Typhus, und seit 1897
               wieder nur zu oft. Nach 2 ½ Jahren macht sie mir wieder ihren Besuch.\pend
           
\pstart
           Die grosse Maschine\pwindex{Brandes, Georg 4.\,2.\,1842 Kopenhagen – 19.\,2.\,1927 ebd.@\textsc{Brandes, Georg} (4.\,2.\,1842 Kopenhagen – 19.\,2.\,1927 ebd.)!Wolfgang Goethe@\strich\emph{Wolfgang Goethe}|pwv}, die ich
               über Goethe\pwindex{Goethe, Johann Wolfgang von 28.\,8.\,1749 Frankfurt am Main – 22.\,3.\,1832 Weimar@\textsc{Goethe, Johann Wolfgang von} (28.\,8.\,1749 Frankfurt am Main – 22.\,3.\,1832 Weimar), \emph{Schriftsteller}|pw} machte, wurde schnell (in diesem
               kleinen Land\oindex{Dänemark@\textbf{Dänemark}|pwv}) in 3,500
               Exemplaren verkauft. Eine neue Auflage ein wenig verbessert, ist erschienen. Es sind
               zwei recht dicke Bände. Ausserdem habe ich viele grössere und kleinere Artikel über
               die Zustände – leider in unserer Geheimsprache – geschrieben.\pend
           
\pstart
           Peter Nansen\pwindex{Nansen, Peter 20.\,1.\,1861 Kopenhagen – 31.\,7.\,1918 Mariager@\textsc{Nansen, Peter} (20.\,1.\,1861 Kopenhagen – 31.\,7.\,1918 Mariager), \emph{Schriftsteller, Journalist, Verleger}|pw}, den Sie kennen, hat seine
               Production wieder aufgenommen und u. a. eine nicht unbedeutende grössere Novelle\pwindex{Nansen, Peter 20.\,1.\,1861 Kopenhagen – 31.\,7.\,1918 Mariager@\textsc{Nansen, Peter} (20.\,1.\,1861 Kopenhagen – 31.\,7.\,1918 Mariager), \emph{Schriftsteller, Journalist, Verleger}!Brüder Menthe@\strich\emph{Die Brüder Menthe}|pwv} erscheinen lassen.
               Selbst liegt er leider krank. Er hat zuviele {\pb}Cigaretten geraucht, zuviel Whisky
               getrunken, sein Herz scheint gelitten zu haben, er hat seit 3–4 Wochen ein\strikeout{en} schwaches Fieber, das nicht weichen will. Ich liebe
               ihn sehr und bin um ihn bekümmert.\pend
           
\pstart
           Liebster Freund\hspace*{1.5em}Empfehlen Sie mich den Ihrigen und
               bleiben Sie mir gut.\pend
           \pstart Ihr \spacefill\mbox{Georg Brandes}\pend{}\selectlanguage{ngerman}\endnumbering\briefempfaengerindex{Schnitzler, Arthur@\textsc{Schnitzler, Arthur}!zzzBrandes, Georg@\emph{von Georg Brandes}!1915-12-041@{4. 12. 1915}|)be}\mylabel{L02221h}  \newcommand{\dateiname}{L02221}\newcommand{\titel}{Georg Brandes an Arthur Schnitzler, 4. 12. 1915}\newcommand{\editorInnen}{Martin Anton Müller und Gerd-Hermann Susen}%% latex-leseansicht-abspann.tex
%% Abspann für die Leseansicht.
%% Der Schalter \ifkorrekturansicht ist bereits durch den Vorspann gesetzt.

%% latex-abspann.tex
%% Gemeinsamer Abspann für Korrekturansicht und Leseansicht.
%% Setzt den Schalter \ifkorrekturansicht voraus (gesetzt in den
%% einbindenden Dateien latex-korrekturansicht-abspann.tex bzw.
%% latex-leseansicht-abspann.tex).
%% ---------------------------------------------------------------

\normalsize

% Das esempio-Environment wird nur in der Leseansicht benötigt
\ifkorrekturansicht\else
\newenvironment{esempio}[3]%
{
    \vspace{1.5ex}
    \rlap{\underline{#1}}
    \par
    \setlength{\parindent}{0cm}
    \nopagebreak
    \leftskip=#2cm
    \rightskip=#3cm
}
{
    \par
}
\fi

\doendnotes{C}
\bigskip
\vfill

\clearpage

\footnotesize

\ifkorrekturansicht
  \lohead{\textsc{register}}
\fi

% theindex-Environment neu definieren ohne reledmac
\makeatletter
\renewenvironment{theindex}{%
  \ifkorrekturansicht
    \section*{\indexname}%
  \else
    \subsubsection*{Index der erwähnten Entitäten}%
  \fi
  \setlength{\parindent}{0pt}%
  \setlength{\parskip}{0pt plus 0.3pt}%
  \let\item\@idxitem
}{%
  \ifkorrekturansicht\clearpage\fi
}
\makeatother

\IfFileExists{\jobname-pw.ind}{\input{\jobname-pw.ind}}{}

% Quellenangabe nur in der Leseansicht
\ifkorrekturansicht\else
% Fallback-Definitionen, falls die .tex-Datei \titel etc. nicht gesetzt hat
\providecommand{\titel}{}
\providecommand{\editorInnen}{}
\providecommand{\dateiname}{\jobname}

\vspace{3cm}

\vfill

\footnotesize
\textsc{Quelle}: \titel. Herausgegeben von {\editorInnen}. In: \emph{Arthur Schnitzler: Briefwechsel mit Autorinnen und Autoren}.
 Digitale Edition, https://schnitzler-briefe.acdh.oeaw.ac.at/{\dateiname}.html (Stand \today)
\fi

\end{document}


