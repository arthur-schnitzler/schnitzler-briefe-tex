%% latex-leseansicht-vorspann.tex
%% Vorspann für die Leseansicht.
%% Lädt die gemeinsame Datei latex-vorspann.tex mit nicht gesetztem Schalter.

\newif\ifkorrekturansicht
\korrekturansichtfalse

\input{../tex-inputs/latex-vorspann}


               \section[Georg Brandes an Arthur Schnitzler, 4. 12. 1915]{ Georg Brandes an Arthur Schnitzler, 4. 12. 1915}\nopagebreak\mylabel{v}\rehead{ }\begin{ledgroupsized}[t]{13cm}\normalsize\beginnumbering\briefempfaengerindex{Schnitzler, Arthur@\textsc{Schnitzler, Arthur}!zzzBrandes, Georg@\emph{von Georg Brandes}!1915-12-041@{4. 12. 1915}|(be} \toendnotes[C]{\smallbreak\pagebreak[2]} \Standort{CUL, Schnitzler, B 17.}
\physDesc{Brief, 1 Blatt, 4 Seiten
\newline{}Handschrift: schwarze Tinte, lateinische Kurrent
\newline{}Schnitzler: mit rotem Buntstift mehrere Unterstreichungen \newline{}Ordnung: mit Bleistift von unbekannter Hand nummeriert: »45« }\buchAbdrucke{\weitereDrucke{Georg Brandes, Arthur Schnitzler: \emph{Ein Briefwechsel}. Hg. Kurt Bergel. Bern: \emph{Francke} 1956, S. 114–115.} }\toendnotes[C]{\smallbreak}\pstart
           \raggedleft{}{\pb}Kopenhagen\oindex{Kopenhagen@\textbf{Kopenhagen}|pw} (genügende Adresse){\\}4 December 15\pend
           \pstart{}Verehrter Freund\pend\pstart
           Drei Jahre sind vergangen, seit ich Ihr Gast war und die Freude hatte, in Ihrem
                    Heim mit Ihnen, Ihrer Frau Gemahlin\pwindex{Schnitzler, Olga 17.01.1882 – 13.01.1970@\textsc{Schnitzler, Olga} (17.01.1882 – 13.01.1970), \emph{Schauspielerin, Sängerin}|pwv} und Ihren Freunden zu verkehren. Seit dem – wie viel
                    unerhörtes ist geschehen! Alles ist anders geworden.\pend
           \pstart
           Ich wollte Ihnen schon vor einem Monat für Ihre dauerhafte Freundschaft danken,
                    dass Sie mir die \uline{Komödie der Worte}\pwindex{Schnitzler, Arthur 15.05.1862 – 21.10.1931@\textsc{Schnitzler, Arthur} (15.05.1862 – 21.10.1931), \emph{Schriftsteller, Mediziner}!Komoedie der Worte. Drei Einakter1915@\strich\emph{Komödie der Worte. Drei Einakter} {[}1915{]}|pw}
               sandten. Sie haben wieder einmal das Labyrinthische dargestellt der
                    erotischen Neigungen und wie die Ehen die Herzen hemmen und fesseln. Tragisches
                    und Possierliches ist nach Ihrer Gewohnheit gemischt. Mir war Alles lieb.\pend
           \pstart
           Vor etwa drei Wochen sah ich in {\pb}einem grossen privaten
                    Verein hier Ihren \uline{Dr. Bernhardi}\pwindex{Schnitzler, Arthur 15.05.1862 – 21.10.1931@\textsc{Schnitzler, Arthur} (15.05.1862 – 21.10.1931), \emph{Schriftsteller, Mediziner}!Professor Bernhardi. Komoedie in fuenf Akten1912@\strich\emph{Professor Bernhardi. Komödie in fünf Akten} {[}1912{]}|pw} im Wesentlichen ganz vorzüglich aufgeführt. Das Stück ist mir theuer; nur
                    kann ich mich nicht mit der Logik recht befreunden, dass weil jemand nicht zum
                    Märtyrer geeignet ist, er überhaupt nicht für seine Ueberzeugung eintreten
                    solle. Wir lassen ja alle ohne Protest das meiste hingehen, weil das Protestiren
                    doch nichts nützt; aber Sie sollten nicht unsere Handlungskraft durch
                    Entmuthigung lähmen. Das ist die alte »Ironie« der Romantiker, die dem Pathos
                    die Spitze abbricht.\pend
           \pstart
           Doch, was liegt heutzutage an all dem! Macduff sagt:\pend
           \stanza{}O horror, horror, horror\pwindex{\textcolor{red}{\textsuperscript{XXXX1 indx}}!Macbeth1606@\strich\emph{Macbeth} {[}1606{]}|pwv}\newverse{}Tongue nor heart\pwindex{\textcolor{red}{\textsuperscript{XXXX1 indx}}!Macbeth1606@\strich\emph{Macbeth} {[}1606{]}|pwv}\newverse{}Cannot conceive nor name
                            thee.\pwindex{\textcolor{red}{\textsuperscript{XXXX1 indx}}!Macbeth1606@\strich\emph{Macbeth} {[}1606{]}|pwv}\stanzaend{}\pstart
           {\pb}Ich habe leider im
                    Augenblick wieder einen Anfall von meiner chronischen Krankheit, der
                    Venenentzündung. Sie kam zum ersten mal in 1871 nach einem Typhus,
                    und seit 1897 wieder nur zu oft. Nach 2 ½ Jahren macht sie mir
                    wieder ihren Besuch.\pend
           \pstart
           Die grosse Maschine\pwindex{Brandes, Georg 04.02.1842 – 19.02.1927@\textsc{Brandes, Georg} (04.02.1842 – 19.02.1927)!Wolfgang Goethe1915@\strich\emph{Wolfgang Goethe} {[}1915{]}|pwv}, die ich
                    über Goethe\pwindex{Goethe, Johann Wolfgang von 28.08.1749 – 22.03.1832@\textsc{Goethe, Johann Wolfgang von} (28.08.1749 – 22.03.1832), \emph{Schriftsteller}|pw} machte, wurde schnell (in diesem
                    kleinen Land\oindex{Daenemark@\textbf{Dänemark}|pwv}) in 3,500
                    Exemplaren verkauft. Eine neue Auflage ein wenig verbessert, ist erschienen. Es
                    sind zwei recht dicke Bände. Ausserdem habe ich viele grössere und kleinere
                    Artikel über die Zustände – leider in unserer Geheimsprache – geschrieben.\pend
           \pstart
           Peter Nansen\pwindex{Nansen, Peter 20.01.1861 – 31.07.1918@\textsc{Nansen, Peter} (20.01.1861 – 31.07.1918), \emph{Schriftsteller, Journalist, Verleger}|pw}, den Sie kennen, hat seine
                    Production wieder aufgenommen und u. a. eine nicht unbedeutende grössere Novelle\pwindex{Nansen, Peter 20.01.1861 – 31.07.1918@\textsc{Nansen, Peter} (20.01.1861 – 31.07.1918), \emph{Schriftsteller, Journalist, Verleger}!Brueder Menthe1915@\strich\emph{Die Brüder Menthe} {[}1915{]}|pwv} erscheinen lassen.
                    Selbst liegt er leider krank. Er hat zuviele {\pb}Cigaretten geraucht, zuviel
                    Whisky getrunken, sein Herz scheint gelitten zu haben, er hat seit 3–4 Wochen
                        ein\strikeout{en}
               schwaches Fieber, das nicht weichen
                    will. Ich liebe ihn sehr und bin um ihn bekümmert.\pend
           \pstart
           Liebster Freund\hspace*{1.5em}Empfehlen Sie mich den Ihrigen
                    und bleiben Sie mir gut.\pend
           \pstart Ihr \spacefill\mbox{Georg Brandes}\pend{}\endnumbering\briefempfaengerindex{Schnitzler, Arthur@\textsc{Schnitzler, Arthur}!zzzBrandes, Georg@\emph{von Georg Brandes}!1915-12-041@{4. 12. 1915}|)be}\mylabel{h}\end{ledgroupsized}  \newcommand{\dateiname}{L02221}\newcommand{\titel}{Georg Brandes an Arthur Schnitzler, 4. 12. 1915}\newcommand{\editorInnen}{Martin Anton Müller und Gerd-Hermann Susen}%% latex-leseansicht-abspann.tex
%% Abspann für die Leseansicht.
%% Der Schalter \ifkorrekturansicht ist bereits durch den Vorspann gesetzt.

%% latex-abspann.tex
%% Gemeinsamer Abspann für Korrekturansicht und Leseansicht.
%% Setzt den Schalter \ifkorrekturansicht voraus (gesetzt in den
%% einbindenden Dateien latex-korrekturansicht-abspann.tex bzw.
%% latex-leseansicht-abspann.tex).
%% ---------------------------------------------------------------

\normalsize

% Das esempio-Environment wird nur in der Leseansicht benötigt
\ifkorrekturansicht\else
\newenvironment{esempio}[3]%
{
    \vspace{1.5ex}
    \rlap{\underline{#1}}
    \par
    \setlength{\parindent}{0cm}
    \nopagebreak
    \leftskip=#2cm
    \rightskip=#3cm
}
{
    \par
}
\fi

\doendnotes{C}
\bigskip
\vfill

\clearpage

\footnotesize

\ifkorrekturansicht
  \lohead{\textsc{register}}
\fi

% theindex-Environment neu definieren ohne reledmac
\makeatletter
\renewenvironment{theindex}{%
  \ifkorrekturansicht
    \section*{\indexname}%
  \else
    \subsubsection*{Index der erwähnten Entitäten}%
  \fi
  \setlength{\parindent}{0pt}%
  \setlength{\parskip}{0pt plus 0.3pt}%
  \let\item\@idxitem
}{%
  \ifkorrekturansicht\clearpage\fi
}
\makeatother

\IfFileExists{\jobname-pw.ind}{\input{\jobname-pw.ind}}{}

% Quellenangabe nur in der Leseansicht
\ifkorrekturansicht\else
% Fallback-Definitionen, falls die .tex-Datei \titel etc. nicht gesetzt hat
\providecommand{\titel}{}
\providecommand{\editorInnen}{}
\providecommand{\dateiname}{\jobname}

\vspace{3cm}

\vfill

\footnotesize
\textsc{Quelle}: \titel. Herausgegeben von {\editorInnen}. In: \emph{Arthur Schnitzler: Briefwechsel mit Autorinnen und Autoren}.
 Digitale Edition, https://schnitzler-briefe.acdh.oeaw.ac.at/{\dateiname}.html (Stand \today)
\fi

\end{document}


      