%% latex-leseansicht-vorspann.tex
%% Vorspann für die Leseansicht.
%% Lädt die gemeinsame Datei latex-vorspann.tex mit nicht gesetztem Schalter.

\newif\ifkorrekturansicht
\korrekturansichtfalse

\input{../tex-inputs/latex-vorspann}


         
         \renewcommand{\erwaehntePersonen}{Personen: Ludwig Michalek, H. Roth, Adalbert Franz Seligmann}
         \renewcommand{\erwaehnteInstitutionen}{Institutionen: Wiener Frauenakademie}
         \renewcommand{\erwaehnteOrte}{Orte: Tuchlauben, Wien}
         \renewcommand{\erwaehnteWerke}{}
               \section[Adalbert Seligmann an Arthur Schnitzler, 1. 10. 1901]{ Adalbert Seligmann an Arthur Schnitzler, 1. 10. 1901}\nopagebreak\mylabel{v}\rehead{ }\begin{ledgroupsized}[t]{13cm}\normalsize\beginnumbering \toendnotes[C]{\smallbreak\pagebreak[2]} \Standort{CUL, Schnitzler, B 97.}
\physDesc{Brief, 1 Blatt, 2 Seiten, 730 Zeichen
\newline{}Handschrift: schwarze Tinte, deutsche Kurrent
\newline{}Schnitzler: 1) mit Bleistift nummeriert: »3«  2) mit rotem Buntstift eine Unterstreichung}\pstart
           \raggedleft{}{\pb}1/10 1901\pend
           \pstart
           Lieber Herr Doctor! Natürlich gibt es eine gute Schule für Damen in
                  Wien\oindex{Wien@\textbf{Wien}|pw} – die, an der u. a. auch ich Lehrer bin,
               (Sie begreifen doch meine Gründe?!) d. i. der Verein »Kunſtſchule für Frauen und Mädchen\orgindex{Wiener Frauenakademie@Wiener Frauenakademie|pw}« I.
                  Tuchlauben 8\oindex{Tuchlauben@\textbf{Tuchlauben}|pw}. Dortſelbſt wird auch unter Leitung von Prof. Michalek\pwindex{Michalek, Ludwig 1859-04-13 – 1942-09-24@\textsc{Michalek, Ludwig} (1859-04-13 – 1942-09-24), \emph{Maler, Radierer, Bildender Künstler}|pw} ein Radirkurs abgehalten. Mit dem Schaben ſieht es
               bei uns allerdings noch ſchäbig aus, – verzeihen Sie den ſo {\pb}naheliegenden Kalauer – doch wird ſich
               möglicherweiſe auch dafür Rath ſchaffen laſſen. Material, Preſſe u. ſ. w. ſind in
               unſerer Schule vorhanden. Die Bedingungen ſind auf dem Proſpect erſichtlich der
               jederzeit bei unſerer Sekretärin Frl. H. Roth\pwindex{Roth, H. @\textsc{Roth, H.}, \emph{Sekretärin}|pw},
                  Tuchlauben 8\oindex{Tuchlauben@\textbf{Tuchlauben}|pw}. (von 10–12 Vormittags) behoben
               werden kann.\pend
           \pstart
           Mit beſten Grüßen{\\[\baselineskip]}Ihr ergebener{\\[\baselineskip]}\spacefill\mbox{Seligmann}\pend
           \leftskip=0em{}
         
         \endnumbering\mylabel{h}\end{ledgroupsized}  \newcommand{\dateiname}{L01179}\newcommand{\titel}{Adalbert Seligmann an Arthur Schnitzler, 1. 10. 1901}\newcommand{\editorInnen}{Martin Anton Müller und Gerd-Hermann Susen}%% latex-leseansicht-abspann.tex
%% Abspann für die Leseansicht.
%% Der Schalter \ifkorrekturansicht ist bereits durch den Vorspann gesetzt.

%% latex-abspann.tex
%% Gemeinsamer Abspann für Korrekturansicht und Leseansicht.
%% Setzt den Schalter \ifkorrekturansicht voraus (gesetzt in den
%% einbindenden Dateien latex-korrekturansicht-abspann.tex bzw.
%% latex-leseansicht-abspann.tex).
%% ---------------------------------------------------------------

\normalsize

% Das esempio-Environment wird nur in der Leseansicht benötigt
\ifkorrekturansicht\else
\newenvironment{esempio}[3]%
{
    \vspace{1.5ex}
    \rlap{\underline{#1}}
    \par
    \setlength{\parindent}{0cm}
    \nopagebreak
    \leftskip=#2cm
    \rightskip=#3cm
}
{
    \par
}
\fi

\doendnotes{C}
\bigskip
\vfill

\clearpage

\footnotesize

\ifkorrekturansicht
  \lohead{\textsc{register}}
\fi

% theindex-Environment neu definieren ohne reledmac
\makeatletter
\renewenvironment{theindex}{%
  \ifkorrekturansicht
    \section*{\indexname}%
  \else
    \subsubsection*{Index der erwähnten Entitäten}%
  \fi
  \setlength{\parindent}{0pt}%
  \setlength{\parskip}{0pt plus 0.3pt}%
  \let\item\@idxitem
}{%
  \ifkorrekturansicht\clearpage\fi
}
\makeatother

\IfFileExists{\jobname-pw.ind}{\input{\jobname-pw.ind}}{}

% Quellenangabe nur in der Leseansicht
\ifkorrekturansicht\else
% Fallback-Definitionen, falls die .tex-Datei \titel etc. nicht gesetzt hat
\providecommand{\titel}{}
\providecommand{\editorInnen}{}
\providecommand{\dateiname}{\jobname}

\vspace{3cm}

\vfill

\footnotesize
\textsc{Quelle}: \titel. Herausgegeben von {\editorInnen}. In: \emph{Arthur Schnitzler: Briefwechsel mit Autorinnen und Autoren}.
 Digitale Edition, https://schnitzler-briefe.acdh.oeaw.ac.at/{\dateiname}.html (Stand \today)
\fi

\end{document}


      