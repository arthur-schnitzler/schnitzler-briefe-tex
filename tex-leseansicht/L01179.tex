%% latex-leseansicht-vorspann.tex
%% Vorspann für die Leseansicht.
%% Lädt die gemeinsame Datei latex-vorspann.tex mit nicht gesetztem Schalter.

\newif\ifkorrekturansicht
\korrekturansichtfalse

\input{../tex-inputs/latex-vorspann}


\section[Adalbert Seligmann an Arthur Schnitzler, 1. 10. 1901]{L01179 Adalbert Seligmann an Arthur Schnitzler, 1. 10. 1901}
\nopagebreak\mylabel{L01179v}
\rehead{ }\normalsize\beginnumbering\briefempfaengerindex{Schnitzler, Arthur@\textsc{Schnitzler, Arthur}!zzzSeligmann, Adalbert Franz@\emph{von Adalbert Franz Seligmann}!1901-10-012@{1. 10. 1901}|(be}
\toendnotes[C]{\smallbreak\pagebreak[2]}
\correspDesc{Versand  durch Adalbert Seligmann am 1. 10. 1901 in Wien
\newline{}Erhalt  durch Arthur Schnitzler im Zeitraum [1. 10. 1901
                  – 5. 10. 1901?] in Wien}\toendnotes[C]{\smallbreak}
\Standort{CUL, Schnitzler, B 97.}
\physDesc{Brief, 1 Blatt, 2 Seiten, 730 Zeichen
\newline{}Handschrift: schwarze Tinte, deutsche Kurrent
\newline{}Schnitzler: 1) mit Bleistift nummeriert: »3«  2) mit rotem Buntstift eine Unterstreichung}
\pstart
           \raggedleft{}{\pb}1/10 1901\pend
           \vspace{0.5em}
\pstart
           Lieber Herr Doctor! Natürlich gibt es eine gute Schule für Damen in
                  Wien\oindex{Wien@\textbf{Wien}, \emph{Verwaltungsgebiet}|pw} – die, an der u. a. auch ich Lehrer bin,
               (Sie begreifen doch meine Gründe?!) d. i. der Verein »Kunſtſchule für Frauen und Mädchen\orgindex{Wiener Frauenakademie@Wiener Frauenakademie|pw}« I.
                  Tuchlauben 8\oindex{Wien@\textbf{Wien}!I., Innere Stadt@\textbf{I., Innere Stadt}!Tuchlauben@\textbf{Tuchlauben}, \emph{Straße}|pw}. Dortſelbſt wird auch unter Leitung von Prof. Michalek\pwindex{Michalek, Ludwig 13.\,4.\,1859 Timișoara – 24.\,9.\,1942@\textsc{Michalek, Ludwig} (13.\,4.\,1859 Timișoara – 24.\,9.\,1942), \emph{Maler, Radierer, Bildender Künstler}|pw} ein Radirkurs abgehalten. Mit dem Schaben{ }ſieht es
               bei uns allerdings noch{ }ſchäbig aus, – verzeihen Sie den{ }ſo {\pb}naheliegenden Kalauer – doch wird{ }ſich
               möglicherweiſe auch dafür Rath{ }ſchaffen laſſen. Material, Preſſe u. ſ. w.{ }ſind in
               unſerer Schule vorhanden. Die Bedingungen{ }ſind auf dem Proſpect erſichtlich der
               jederzeit bei unſerer Sekretärin Frl. H. Roth\pwindex{Roth, H. @\textsc{Roth, H.}, \emph{Sekretärin}|pw},
                  Tuchlauben 8\oindex{Wien@\textbf{Wien}!I., Innere Stadt@\textbf{I., Innere Stadt}!Tuchlauben@\textbf{Tuchlauben}, \emph{Straße}|pw}. (von 10–12 Vormittags) behoben
               werden kann.\pend
           
\pstart
           Mit beſten Grüßen{\\[\baselineskip]}Ihr ergebener{\\[\baselineskip]}\spacefill\mbox{Seligmann}\pend
           \leftskip=0em{}\selectlanguage{ngerman}\endnumbering\briefempfaengerindex{Schnitzler, Arthur@\textsc{Schnitzler, Arthur}!zzzSeligmann, Adalbert Franz@\emph{von Adalbert Franz Seligmann}!1901-10-012@{1. 10. 1901}|)be}\mylabel{L01179h}  \newcommand{\dateiname}{L01179}\newcommand{\titel}{Adalbert Seligmann an Arthur Schnitzler, 1. 10. 1901}\newcommand{\editorInnen}{Martin Anton Müller und Gerd-Hermann Susen}%% latex-leseansicht-abspann.tex
%% Abspann für die Leseansicht.
%% Der Schalter \ifkorrekturansicht ist bereits durch den Vorspann gesetzt.

%% latex-abspann.tex
%% Gemeinsamer Abspann für Korrekturansicht und Leseansicht.
%% Setzt den Schalter \ifkorrekturansicht voraus (gesetzt in den
%% einbindenden Dateien latex-korrekturansicht-abspann.tex bzw.
%% latex-leseansicht-abspann.tex).
%% ---------------------------------------------------------------

\normalsize

% Das esempio-Environment wird nur in der Leseansicht benötigt
\ifkorrekturansicht\else
\newenvironment{esempio}[3]%
{
    \vspace{1.5ex}
    \rlap{\underline{#1}}
    \par
    \setlength{\parindent}{0cm}
    \nopagebreak
    \leftskip=#2cm
    \rightskip=#3cm
}
{
    \par
}
\fi

\doendnotes{C}
\bigskip
\vfill

\clearpage

\footnotesize

\ifkorrekturansicht
  \lohead{\textsc{register}}
\fi

% theindex-Environment neu definieren ohne reledmac
\makeatletter
\renewenvironment{theindex}{%
  \ifkorrekturansicht
    \section*{\indexname}%
  \else
    \subsubsection*{Index der erwähnten Entitäten}%
  \fi
  \setlength{\parindent}{0pt}%
  \setlength{\parskip}{0pt plus 0.3pt}%
  \let\item\@idxitem
}{%
  \ifkorrekturansicht\clearpage\fi
}
\makeatother

\IfFileExists{\jobname-pw.ind}{\input{\jobname-pw.ind}}{}

% Quellenangabe nur in der Leseansicht
\ifkorrekturansicht\else
% Fallback-Definitionen, falls die .tex-Datei \titel etc. nicht gesetzt hat
\providecommand{\titel}{}
\providecommand{\editorInnen}{}
\providecommand{\dateiname}{\jobname}

\vspace{3cm}

\vfill

\footnotesize
\textsc{Quelle}: \titel. Herausgegeben von {\editorInnen}. In: \emph{Arthur Schnitzler: Briefwechsel mit Autorinnen und Autoren}.
 Digitale Edition, https://schnitzler-briefe.acdh.oeaw.ac.at/{\dateiname}.html (Stand \today)
\fi

\end{document}


