%% latex-korrekturansicht-vorspann.tex
%% Vorspann für die Korrekturansicht.
%% Lädt die gemeinsame Datei latex-vorspann.tex mit gesetztem Schalter.

\newif\ifkorrekturansicht
\korrekturansichttrue

\input{../tex-inputs/latex-vorspann}


\section[Adalbert Seligmann an Arthur Schnitzler, 1. 10. 1901]{L01179 Adalbert Seligmann an Arthur Schnitzler, 1. 10. 1901}
\nopagebreak\mylabel{L01179v}
\rehead{ }\normalsize\beginnumbering\briefempfaengerindex{Schnitzler, Arthur@\textsc{Schnitzler, Arthur}!zzzSeligmann, Adalbert Franz@\emph{von Adalbert Franz Seligmann}!1901-10-012@{1. 10. 1901}|(be}
\toendnotes[C]{\smallbreak\pagebreak[2]}\Standort{CUL, Schnitzler, B 97.}
\physDesc{Brief, 1 Blatt, 2 Seiten, 730 Zeichen
\newline{}Handschrift: schwarze Tinte, deutsche Kurrent
\newline{}Schnitzler: 1) mit Bleistift nummeriert: »3«  2) mit rotem Buntstift eine Unterstreichung}
\pstart
           \raggedleft{}{\pb}1/10 1901\pend
           \vspace{0.5em}
\pstart
           Lieber Herr Doctor! Natürlich gibt es eine gute Schule für Damen in
                  Wien\oindex{Wien@\textbf{Wien}, \emph{A.ADM2}|pw} – die, an der u. a. auch ich Lehrer bin,
               (Sie begreifen doch meine Gründe?!) d. i. der Verein »Kunſtſchule für Frauen und Mädchen\orgindex{Wiener Frauenakademie@Wiener Frauenakademie|pw}« I.
                  Tuchlauben 8\oindex{Tuchlauben@\textbf{Tuchlauben}, \emph{Straße (K.STR)}|pw}. Dortſelbſt wird auch unter Leitung von Prof. Michalek\pwindex{Michalek, Ludwig 1859-04-13 – 1942-09-24@\textsc{Michalek, Ludwig} (1859-04-13 – 1942-09-24), \emph{Maler/Malerin, Radierer/Radiererin, Bildender Künstler/Bildende Künstlerin}|pw} ein Radirkurs abgehalten. Mit dem Schaben ſieht es
               bei uns allerdings noch ſchäbig aus, – verzeihen Sie den ſo {\pb}naheliegenden Kalauer – doch wird ſich
               möglicherweiſe auch dafür Rath ſchaffen laſſen. Material, Preſſe u. ſ. w. ſind in
               unſerer Schule vorhanden. Die Bedingungen ſind auf dem Proſpect erſichtlich der
               jederzeit bei unſerer Sekretärin Frl. H. Roth\pwindex{Roth, H. @\textsc{Roth, H.}, \emph{Sekretär/Sekretärin}|pw},
                  Tuchlauben 8\oindex{Tuchlauben@\textbf{Tuchlauben}, \emph{Straße (K.STR)}|pw}. (von 10–12 Vormittags) behoben
               werden kann.\pend
           
\pstart
           Mit beſten Grüßen{\\[\baselineskip]}Ihr ergebener{\\[\baselineskip]}\spacefill\mbox{Seligmann}\pend
           \leftskip=0em{}\selectlanguage{ngerman}\endnumbering\briefempfaengerindex{Schnitzler, Arthur@\textsc{Schnitzler, Arthur}!zzzSeligmann, Adalbert Franz@\emph{von Adalbert Franz Seligmann}!1901-10-012@{1. 10. 1901}|)be}\mylabel{L01179h}  \normalsize

\doendnotes{C}
\bigskip
\vfill

\clearpage

\footnotesize

\lohead{\textsc{register}}

% Definiere theindex-Environment komplett neu ohne reledmac
\makeatletter
\renewenvironment{theindex}{%
  \section*{\indexname}%
  \setlength{\parindent}{0pt}%
  \setlength{\parskip}{0pt plus 0.3pt}%
  \let\item\@idxitem
}{%
  \clearpage
}
\makeatother

\IfFileExists{\jobname-pw.ind}{\input{\jobname-pw.ind}}{}

\end{document}

      